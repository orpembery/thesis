\section{Introduction and Motivation from UQ}\label{sec:intronbpc}

\subsection{Motivation from uncertainty quantification of the Helmholtz equation} 
Consider the stochastic Helmholtz equation 
\beq\label{eq:nbpchh}
\nabla\cdot\big(A(\omega,\bx) \nabla u(\omega,\bx) \big) + k^2 n(\omega,\bx) u(\omega,\bx) =-f(\bx), \quad \bx\in\Dp,
\eeq
as defined in \cref{chap:stochastic}. To calculate quantities of interest of the solution $u(\omega,\cdot)$, one must solve many deterministic Helmholtz problems, with each one corresponding to different realisations of the coefficients $A(\omega,\cdot)$ and $n(\omega,\cdot)$.
Solving all these deterministic problems is a very computationally-intensive task because linear systems arising from discretisations of the Helmholtz equation are notoriously difficult to solve; see the discussion in \cref{sec:numsolve} above.

Moreover, when using preconditioned iterative methods, as described in \cref{sec:numsolve} above, the cost of \emph{constructing} preconditioners for the linear systems is also very expensive. For example, if our preconditioner is given by an exact $LU$ factorisation, then (in 3-d) the cost of calculating the preconditioner is $\cO\mleft(N^2\mright)$ and the cost of applying the preconditioner is $\cO\mleft(N^{4/3}\mright)$ (where $N$ is the number of unknowns), see, e.g., \cite[Section 4]{GaZh:19}.

Observe that the cost of applying the preconditioner is significantly less than the cost of constructing the preconditioner. Therefore, if one calculates a preconditioner for one realisation of \cref{eq:nbpchh} and then re-uses it for many `nearby' (in some sense) realisations of \cref{eq:nbpchh}, then one could obtain considerable computational savings. This idea of `reusing' preconditioners for `nearby' linear systems is the `nearby preconditioning' strategy proposed in this \lcnamecref{chap:nbpc}.

If one does not use an exact $LU$-factorisation as the preconditioner, one may still obtain considerable savings by reusing preconditioners. Even if one uses a modern Helmholtz preconditioner such as a sweeping preconditioner or a domain-decompostion preconditioner (see the recent review article \cite{GaZh:19} for an overview of many types of Helmholtz preconditioners) one still performs $LU$-factorisations (or direct solves) on subdomains of $\Dp$ (see, e.g., \cite[Section 2]{GaZh:19}). Therefore resuing preconditioners, and thereby calculating fewer $LU$ factorisations may still lead to considerable savings.

In this \lcnamecref{chap:nbpc}, for simplicity, we restrict our attention to the case where the preconditioner is an exact $LU$ decomposition. In this case, the only error that arises in the nearby preconditioning process is due to applying a preconditioner constructed for one realisation of the Helmholtz equation to a different, nearby, realisation. Therefore  we do not need to consider the error arising from the preconditioning process itself.% The answer to \cref{it:nbpcq1} above makes $k$-explicit the extent to which the $LU$ factorisation for one realisation of \cref{eq:nbpchh} can be used as an effective preconditioner for Galerkin matrices arising from different realisations of $A$ and $n$.

\subsection{Statement of the problem}\label{sec:problem}
Let $\Aj, \nj$, $j=1,2$ satisfy the properties of $A$ and $n$ in \cref{prob:vedp}, with $\uj$ the corresponding solution and $\Dm,$ $f$, etc. as in \cref{prob:vedp}. Let $\Amatj$, $j=1,2,$ be the Galerkin matrices corresponding to $h$-finite-element discretisations of \cref{prob:vedp} (see \cref{eq:matrixAjdef} below for a precise definition of $\Amatj$). The results we prove in this \lcnamecref{chap:nbpc} also hold for the TEDP, \cref{prob:vtedp}, see \cref{sec:TEDP} below for the minor changes one must make in this case.
 
This \lcnamecref{chap:nbpc} answers the following question:

% Inspired by the usage of enumitem here: https://tex.stackexchange.com/a/58714
\ben[label=Q1., ref=Q1]
\item\label[itemblank]{it:nbpcq1} How small must $\NLi{\Aso - \Ast}$ and 
$\NLi{\nso - \nst}$ be (in some norm, in terms of $k$-dependence) for GMRES 
applied to $(\Amat^{(1)})^{-1}\Amat^{(2)}$ to converge in a $k$-independent number of iterations
%$ to be a good preconditioner for $\Amat^{(2)}$
 for arbitrarily large $k$? 
\een


\subsection{Statement of the main results}\label{sec:main}
Our main results about \cref{it:nbpcq1} are \cref{cor:1,cor:1a} below. \Cref{cor:1a} gives results in the \emph{Euclidean} norm on matrices, denoted by $\|\cdot\|_2$ (induced by the Euclidean norm on vectors), whereas \Cref{cor:1} gives results in the \emph{weighted} norms $\NDmatk{\cdot}$ and $\NDmatkI{\cdot}$. These weighted norms are induced by the corresponding vector norms
\beq\label{eq:Dk}
\NDmatk{\bv}^2:= \big( \Dmatk \bv, \bv\big)_2 = %\big( (\Smat_I + k^2 \Mmat_1)\bv,\bv\big)_2 
\N{v_h}^2_{\HokDR}
\quad \tand
\quad \NDmatkI{\bv}^2:= \big( \Dmatk^{-1} \bv, \bv\big)_2 %= %\big( (\Smat_I + k^2 \Mmat_1)\bv,\bv\big)_2 
%\N{v_h}^2_{\HokDR}
\eeq
where $\Dmatk$ is given in terms of familiar finite-element stiffness- and mass-matrices by \cref{eq:Dk2} below and $\vh =\sum_i \vi \phii$, where the $\phii$ are the finite-element basis functions.

As described in \cref{sec:wpdisc}, the PDE analysis of the Helmholtz equation naturally takes place in the norm $\NHokDR{\cdot}$, and \cref{eq:Dk} shows that the norm $\NDmatk{\cdot}$ is
the discrete analogue of the norm $\NHokDR{\cdot}$. %See \cref{eq:Dk2} and \cref{eq:Dk3} below for this norm expressed in terms of the Euc
The norms $\NDmatk{\cdot}$ and $\NDmatkI{\cdot}$ recently appeared in results about the convergence of domain-decomposition methods %in this norm are proved 
for the Helmholtz equation \cite{GrSpVa:17}, \cite{GrSpZo:18}, and for the time-harmonic Maxwell equations \cite{BoDoGrSpTo:19}. 

\Cref{cor:1,cor:1a} are proved under the \Cref{cond:1nbpc,cond:2} below. These \lcnamecrefs{cond:1nbpc} can be informally stated as 
\bit
\item the obstacle $\Dm$ and the coefficients $\Aso$ and $\nso$ are such that $u^{(1)}$ exists and the problem is \emph{nontrapping} (in the sense described in  \cref{sec:wpdisc} above), and
\item the meshsize $h$ and polynomial degree $p$ in the finite-element method are chosen to depend on $k$ to ensure that the 
finite-element solution to the problem with coefficients $\Aso$ and $\nso$ exists, is unique, and 
%Galerkin method (with coefficients $\Aso$ and $\no$) 
is uniformly accurate as $k\tendi$. 
\eit 

\begin{condition}[Nontrapping bound on $u^{(1)}$]\label{cond:1nbpc}
$\Aso, \nso,$ and $\Dm$ are such that, given $f\in L^2(\DR)$ with $\supp \, f \subset \BR$, 
the solution of \cref{prob:vedp} %(\cref{prob:edp}) 
$u^{(1)}$ exists, is unique, and, given $k_0>0$, $u^{(1)}$ satisfies the bound 
\beq\label{eq:bound1}
\big\|u^{(1)}\big\|_{\HokDR} \leq C^{(1)}_{\rm bound} \N{f}_{L^2(\Dp)} \quad \tfa k\geq k_0,
\eeq
where $C^{(1)}_{\rm bound}$ is independent of $k$, but dependent on $\Aso, \nso, \Dm, R$, and $k_0$.
\end{condition}

\begin{condition}[$k$-independent accuracy of the FE solution for $a^{(1)}(\cdot,\cdot)$]
\label{cond:2}

\

(i) Given $k_0>0$, $h$ and $p$ are such that, given $f\in \LtDR$ with $\supp\, f \subset \BR$, the solution $u_h$ of the Galerkin method \cref{eq:galerkin} with $a(\cdot,\cdot)=a^{(1)}(\cdot,\cdot)$ (and with $\FE(v)$ defined in \cref{eq:Ledp}) exists and is unique for all $k\geq k_0$. % (so that the matrix $\Amato$ is invertible). 
Furthermore, if $f= n\sum_j \alpha_j\phi_j$ for some $\alpha_j \in \CC$ and  $n\in \LiDRRR$  (i.e.~$f$ is an arbitrary element of $\Vhp$ multiplied by $n$), then
\beq\label{eq:bound3}
\N{u-u_h}_{\HokDR} \leq C^{(1)}_{\rm FEM1} \N{f}_{\LtDR} \quad\tfa k\geq k_0, 
\eeq
where $C^{(1)}_{\rm FEM1}$  is independent of $k$ and $h$, but dependent on $\Aso, \nso, \Dm, R, k_0$, and $p$.

(ii) Given $k_0>0$, $h$ and $p$ are such that, given $F\in (\HozDDR)'$, the solution $u_h$ of the Galerkin method \cref{eq:galerkin} 
with $a(\cdot,\cdot)=a^{(1)}(\cdot,\cdot)$
exists, and is unique for all $k\geq k_0$.
%(so that the matrix $\Amato$ is invertible). 
Furthermore, if $F(v)= (A\nabla \widetilde{f},\nabla v)_{\LtDR}$, where $A\in L^\infty(\DR, \RR^{d\times d})$, $A$ is symmetric, and $\widetilde{f} := \sum_j \alpha_j \phi_j$ with $\alpha_j\in \CC$
 (i.e.~$\widetilde{f}$ is an arbitrary element of $\Vhp$), then
\beq\label{eq:bound4}
\N{u-u_h}_{\HokDR} \leq C^{(1)}_{\rm FEM2}\,k\, \N{F}_{(\HokDR)'} \quad\tfa k\geq k_0, 
\eeq
where $C^{(1)}_{\rm FEM2}$  is independent of $k$ and $h$, but dependent on $\Aso, \nso, \Dm, R, k_0$, and $p$.
\end{condition}

\bre[\Cref{eq:bound4} is $k$-independent]\label{rem:yesitis}
Note that \cref{eq:bound4} is a $k$-independent bound on $\NHokDR{\utilde},$ despite the fact that a factor $k$ appears on the right-hand side. The factor $k$ appears because the weighted norm $\NHokDR{\cdot}$ is used in the definition of $\NHokDRs{\cdot},$ and therefore $\NHokDRs{\Ftilde} \sim \NHoDRs{\Ftilde}/k.$
\ere




\bth[Answer to \cref{it:nbpcq1}: $k$-independent weighted GMRES iterations]\label{cor:1}
Assume that $\Dm$, $\Aso$, and $\nso$ satisfy \cref{cond:1nbpc}, and $h$ and $p$ satisfy \cref{cond:2}. Given $k_0>0$, there exist constants $\Co$ and $\Ct$  independent of $h$ and $k$ (but dependent on $\Dm, \Aso, \nso, p$, and $\kz$) such that if 
% there exists $C_2>0$, independent of $h$ and $k$ (but dependent on $\Dm, \Aso, \nso$, $p$, and $k_0$) and given explicitly in \cref{eq:C2} below,
% such that if 
\beq\label{eq:cond}
C_1 \,k \,\NLiDRRRdtd{\Aso-\Ast} +C_2 \, k\, \NLiDRRR{\nso-\nst}
\leq \frac{1}{2}
\eeq
for all $k\geq k_0$, then \emph{both} weighted GMRES working in $\|\cdot\|_{\Dmat_k}$ (and the associated inner product) applied to 
\beq\label{eq:pcsystem1}
(\Amat^{(1)})^{-1}\Amat^{(2)}\bu = \bff
\eeq
\emph{and} weighted GMRES working in $\|\cdot\|_{(\Dmat_k)^{-1}}$ (and the associated inner product) applied to 
\beq\label{eq:pcsystem2}
\Amat^{(2)}(\Amat^{(1)})^{-1}\bv = \bff
\eeq
 converge in a $k$-independent number of iterations for all $k\geq k_0$.
\enth

\bth[Answer to \cref{it:nbpcq1}: $k$-independent (unweighted) GMRES iterations]\label{cor:1a}
Assume that $\Dm$, $\Aso$, and $\nso$ satisfy \cref{cond:1nbpc}, and $h$ and $p$ satisfy \cref{cond:2}. Given $k_0>0$,
let $C_1$ and $C_2$ be as in \cref{cor:1}, and let $s_{\pm}$ and $m_{\pm}$ be as in \cref{lem:normequiv} below (note that all these constants are independent of $k$, $h$, and $p$). Then if 
% there exists $C_2>0$, independent of $h$ and $k$ (but dependent on $\Dm, \Aso, \nso$, $p$, and $k_0$) and given explicitly in \cref{eq:C2} below,
% such that if 
\beq\label{eq:conda}
 C_1 \,\left(\frac{s_+}{m_-}\right) \,\frac{1}{h} \,
\NLiDRRRdtd{\Aso-\Ast} + C_2 \, \left(\frac{m_+}{m_-} \right)k \, \NLiDRRR{\nso-\nst}
\leq \frac{1}{2}
\eeq
for all $k\geq k_0$, then standard GMRES (working in the Euclidean norm and inner product) applied to either of the equations \cref{eq:pcsystem1} or \cref{eq:pcsystem2}
%\beqs
%(\Amat^{(1)})^{-1}\Amat^{(2)}\bu = \bff\quad\text{ or } \quad\Amat^{(2)}(\Amat^{(1)})^{-1}\bv = \bff
%\eeqs
 converges in a $k$-independent number of iterations for all $k\geq k_0$.
\enth

Two notes regarding \cref{cor:1,cor1a}: (i) the constants $C_1$ and $C_2$ are expressed explicitly in \cref{eq:C1nbpc} and \cref{eq:C2} below in terms of constants appearing in \cref{cond:1nbpc,cond:2}, and (ii) the $L^\infty(\DR)$ norm on a matrix-valued functions appearing on the right-hand sides of \cref{eq:main1} and \cref{eq:main1a} is defined by
\beqs
\N{A}_{L^\infty(\DR)}:= \esssup_{\bx\in\DR}\N{A(\bx)}_2.
\eeqs


The factor $1/2$ on the right-hand sides of \cref{eq:cond} and \cref{eq:conda} can be replaced by any number $<1$ and the result still holds, although the number of GMRES iterations is then larger -- but still independent of $k$.
%In \cref{sec:proofFEM}, the constant $C_2$ is expressed explicitly in terms of $C_1$.

\bre
When $h\sim  k^{-1}$, the bounds \cref{eq:main1} and \cref{eq:main1a} (and hence also \cref{eq:cond} and \cref{eq:conda}) are identical in their $k$-dependence; however, when $h\ll k^{-1}$ (as one needs to take to overcome the pollution effect, as discussed in \cref{sec:helmfedisc}) the bound \cref{eq:main1a} for standard GMRES is more pessimistic than the bound \cref{eq:main1} for weighted GMRES.
\ere


\paragraph{How sharp are \cref{cor:1,cor:1a} in their $k$-dependence?}
Numerical experiments in \cref{sec:num} indicate that the condition \cref{eq:cond} is sharp, i.e., that the $k$ in \cref{eq:cond} cannot be replaced by $k^\alpha$ for $\alpha<1$. This indicated sharpness of \cref{eq:cond} is also supported by the PDE-result \cref{thm:2} below. Indeed, \cref{thm:2} % and \cref{lem:1} 
 shows that the condition
\beq\label{eq:sufficientlysmall}
k\,
\NLiDRRRdtd{\Aso-\Ast} \quad\text{ and } \quad k\,\NLiDRRR{\nso-\nst}
%\Big) 
\quad\text{ both sufficiently small}
\eeq
is not only an answer to \cref{it:nbpcq1} (about finite-element discretisations), but is also the natural answer to the analogue of \cref{it:nbpcq1} at the level of PDEs, namely 
\ben[label=Q2., ref=Q2]
\item\label[itemblank]{it:nbpcq2}
How small must $\NLiDRRRdtd{\Aso - \Ast}$ and 
$\NLiDRRR{\nso - \nst}$ be (in terms of $k$-dependence) for the relative error in approximating 
%$u^{(1)}$ to be a good approximation to 
$u^{(2)}$ by $u^{(1)}$ to be bounded independently of $k$ for arbitrarily-large $k$? 
\een
\cref{lem:sharp} then shows that the condition ``$k\NLiDRRR{\nso - \nst}$ sufficiently small" is the \emph{provably-sharp} answer to \cref{it:nbpcq2} when $\Aso= \Ast= I$.

%Before stating these PDE results, we define the weighted $H^1$ norm
%\beq\label{eq:1knorm}
%\N{v}^2_{\HokDR} := \N{\grad v}^2_{L^2(\DR)} + k^2 \N{v}^2_{L^2(\DR)} \quad \tfor v \in H^1_{0,D}(\DR),
%\eeq
%where the space $H^1_{0,D}(\DR)$, defined by \cref{eq:spaceEDP} below, is the natural space containing the solution of the exterior Dirichlet problem. 
To state these PDE results, we use the notation for $a,b>0$ that $a\lesssim b$ when $a\leq C b$ for some $C>0$, independent of $k$, and $a\sim b$ if $a\lesssim b$ and $b\lesssim a$.


%The sharpness of \cref{eq:cond} and \cref{eq:main1} is also supported by the answer to the analogue of \cref{it:nbpcq1} at the level of PDEs. Indeed, the following \cref{thm:2} is the analogue of \cref{thm:1} 

\begin{theorem}[Answer to \cref{it:nbpcq2} (the PDE analogue of \cref{it:nbpcq1})]\label{thm:2}
%Given $f\in L^2(\DR)$ such that $\supp \, f \subset \BR$, 
Let $\Dm$, $\Aso$, and $\nso$ satisfy \cref{cond:1nbpc}, and let $\Dm$, $\Ast$, and $\nst$ be such that $u^{(2)}$ exists
for any $f\in L^2(\DR)$ such that $\supp \, f \subset \BR$. 
Then, given $k_0>0$, there exists $C_3>0$, independent of $k$ and given explicitly in terms of $\Dm$, $\Aso$, and $\nso$ in \cref{eq:C3} below, such that
\beq\label{eq:PDEbound}
\frac{\big\|u^{(1)}-u^{(2)}\big\|_{\HokDR}
}{
\N{u^{(2)}}_{\HokDR}
}\leq C_3 \,k\, \max\set{\NLiDRRRdtd{\Aso-\Ast}\,,\, \NLiDRRR{\nso-\nst}}
\eeq
for all $k\geq k_0$. 
\end{theorem}

\ble[Sharpness of the bound \cref{eq:PDEbound} when $\Aso = \Ast= I$]\label{lem:sharp}
There exist $f, \,\nso$, and $\nst$ (with $\nso\neq \nst$) such that 
the corresponding solutions $u^{(1)}$ and $u^{(2)}$ of \cref{prob:edp} with $\Aso = \Ast= I$ exist, are unique, and satisfy
\beq\label{eq:sharp1}
\frac{\N{u^{(1)}-u^{(2)}}_{\HokDR}
}{
\N{u^{(2)}}_{\HokDR}
}
\sim 
\frac{\N{u^{(1)}-u^{(2)}}_{L^2(\DR)}
}{
\N{u^{(2)}}_{L^2(\DR)}
}\sim k \NLiDRRR{\nso-\nst}.
\eeq
%\noi (ii) There exist $f, \Aso, \Ast$, (with $\Aso\not\equiv \Ast$), such that 
%the corresponding solutions $u^{(1)}$ and $u^{(2)}$ of the exterior Dirichlet problem with $\nso \equiv \nst\equiv 1$ exist, are unique, and satisfy
%%There exist $f\in L^2(\DR), \Aj \in C^{0,1}(\DR)$, $j=1,2$ (with $\Aso\not\equiv \Ast$), such that the corresponding solutions $u^{(1)}$ and $u^{(2)}$ of the exterior Dirichlet problem with $\nso \equiv \nst\equiv 1$ satisfy
%\beq\label{eq:sharp2}
%\frac{\N{u^{(1)}-u^{(2)}}_{\HokDR}
%}{
%\N{u^{(2)}}_{\HokDR}
%}
%\sim 
%\frac{\N{u^{(1)}-u^{(2)}}_{L^2(\DR)}
%}{
%\N{u^{(2)}}_{L^2(\DR)}
%}\sim k \big\|\Aso-\Ast\big\|_{L^\infty(\DR)}.
%\eeq
\ele

\bre[Physical interpretation for $k$-dependence]\label{rem:physical1k}
It is unsurprising that the condition \cref{eq:sufficientlysmall} is a sufficient condition to answer both \cref{it:nbpcq1} and \cref{it:nbpcq2}. Recall that $1/k$ is proportional to the wavelength $2\pi/k$ of the wave $u$ (at least when $A=I$ and $n=1$). As the wavelength is the natural length scale associated with the wave $u$, it is unsurprising that perturbations of size (up to) $1/k$ give bounded relative difference (in \cref{it:nbpcq2}) and bounded GMRES iterations for the nearby-preconditioned linear system (in \cref{it:nbpcq1}). In some sense, perturbations of size up to $1/k$ are `unseen' by the PDE or numerical method. Perturbations of order $1/k$ being `unseen' by the PDE can also be seen in bounds proved for $u$ where $n = \no + \eta,$ with $\no$ nontrapping and $\NLiDRRR{\eta} \lesssim 1/k,$ see \cref{rem:kdep} above.
\ere


\section{Numerical experiments}\label{sec:num}
\subsection{Investigating the sharpness of \cref{thm:1} and \cref{cor:1}}

\Cref{thm:1,cor:1} were stated for the exterior Dirichlet problem, but, as highlighted in \cref{sec:problem} and shown in \cref{sec:TEDP}, they hold also for the truncated exterior Dirichlet problem (TEDP) \cref{prob:vtedp}. The numerical experiments in this section seek to verify the analogues of \cref{thm:1} and \cref{cor:1} for the TEDP, and investigate their sharpness. More specifically, the experiments seek to verify whether the condition \cref{eq:cond} is:
\ben
\item sufficient, and
\item necessary
  \een
  for \emph{standard} GMRES applied to \cref{eq:pcsystem1} to converge in a number of iterations that is independent of $k.$

Based on the PDE results \cref{thm:2,lem:sharp} above, we expect that the condition \cref{eq:sufficientlysmall} is a necessary and sufficient condition for standard GMRES applied to \cref{eq:pcsystem1} to converge in a $k$-independent number of iterations, even though we can only prove this is a sufficient condition for \emph{weighted} GMRES. We expect this because \cref{eq:sufficientlysmall} is a sufficient condition for \cref{it:nbpcq2}, the PDE analogue of \cref{it:nbpcq1}.

To verify this expected behaviour, we perform numerical experiments for the setup described in \cref{app:compsetup} with $\Aso = I$ and $\nso = 1$. We define $f$ and $\gI$ to correspond to a plane wave incident from the bottom left passing through a homogeneous medium given by coefficients $\Aso$ and $\nso$. We perform experiments for $A$ and $n$ separately, i.e., first we perform experiments with $\Ast=I$ and $\nst$ varying, and then we perform experiments with $\Ast$ varying and $\nst=1.$

We define $\Ast$ and $\nst$ to be piecewise constant on a $10\times10$ square grid, with their values on each square chosen independently at random from a $\Unif\mleft(1-\alpha,1+\alpha\mright)$ distribution, with $\alpha \in (0,1)$ chosen as described below. For $\Ast,$ we also impose the restriction that $\Ast$ is (piecewise) positive-definite almost surely. We solve the linear systems \cref{eq:pcsystem1} for $k = 20,40,60,80,100$ using standard GMRES and record the number of GMRES iterations taken to achieve a (relative) tolerance of $10^{-5}$ (relative to $\Nt{\bfb}$).

We perform experiments for three different functional forms of $\alpha$  taking $\alpha = 0.5,\, 0.5/k^{1/2},$ and $ 0.5/k.$ We expect that when $\alpha = 0.5$ or $\alpha = 0.5/k^{1/2},$ the number of GMRES iterations required for convergence will increase as $k$ increases, whereas we expect that when $\alpha = 0.5/k$ the number of GMRES iterations required for convergence will remain bounded as $k$ increases, even though this behaviour has only been proven for $\NLiDRRRdtd{\Aso-\Ast}$ for weighted GMRES.


  For both $\NLiDRRRdtd{\Aso-\Ast}$ and $\NLiDRRR{\nso-\nst}$ we see the expected behaviour in all three cases. When $\alpha = 0.5$ we see growth in the number of GMRES iterations needed to achieve convergence, and when $\alpha = 0.5/k,$ we see that the number of GMRES iterations is bounded as $k$ increases. When $\alpha = 0.5/k^{1/2}$ we see slower growth as $k$ increases, (at least for $\Aso-\Ast$) but growth nonetheless.

  The behaviour for $\alpha = 0.5$ indicates that the effectiveness of this `nearby preconditioning' strategy deteriorates rapdily as $k$ increases when $\NLiDRRRdtd{\Aso-\Ast}$ and $\NLiDRRR{\nso-\nst}$ are independent of $k$. The behaviour for $\alpha = 0.5/k$ verifies that the condition \cref{eq:sufficientlysmall} is sufficient to achieve bounded GMRES iterations in this case, and the behaviour for $\alpha = 0.5/k^{1/2}$ indicates that this condition is sharp; reducing the magnitude of the difference betwen the Helmholtz problems as $k$ increases, but not at the rate $1/k$ is insufficient for the nearby preconditioning strategy to achieve $k$-independent GMRES iterations.
  \begin{figure}
    \centering
    \begin{subfigure}{\textwidth}
      \centering
%% Creator: Matplotlib, PGF backend
%%
%% To include the figure in your LaTeX document, write
%%   \input{<filename>.pgf}
%%
%% Make sure the required packages are loaded in your preamble
%%   \usepackage{pgf}
%%
%% Figures using additional raster images can only be included by \input if
%% they are in the same directory as the main LaTeX file. For loading figures
%% from other directories you can use the `import` package
%%   \usepackage{import}
%% and then include the figures with
%%   \import{<path to file>}{<filename>.pgf}
%%
%% Matplotlib used the following preamble
%%   \usepackage{fontspec}
%%   \setmainfont{DejaVuSerif.ttf}[Path=/home/owen/progs/firedrake-complex/firedrake/lib/python3.5/site-packages/matplotlib/mpl-data/fonts/ttf/]
%%   \setsansfont{DejaVuSans.ttf}[Path=/home/owen/progs/firedrake-complex/firedrake/lib/python3.5/site-packages/matplotlib/mpl-data/fonts/ttf/]
%%   \setmonofont{DejaVuSansMono.ttf}[Path=/home/owen/progs/firedrake-complex/firedrake/lib/python3.5/site-packages/matplotlib/mpl-data/fonts/ttf/]
%%
\begingroup%
\makeatletter%
\begin{pgfpicture}%
\pgfpathrectangle{\pgfpointorigin}{\pgfqpoint{6.400000in}{4.800000in}}%
\pgfusepath{use as bounding box, clip}%
\begin{pgfscope}%
\pgfsetbuttcap%
\pgfsetmiterjoin%
\definecolor{currentfill}{rgb}{1.000000,1.000000,1.000000}%
\pgfsetfillcolor{currentfill}%
\pgfsetlinewidth{0.000000pt}%
\definecolor{currentstroke}{rgb}{1.000000,1.000000,1.000000}%
\pgfsetstrokecolor{currentstroke}%
\pgfsetdash{}{0pt}%
\pgfpathmoveto{\pgfqpoint{0.000000in}{0.000000in}}%
\pgfpathlineto{\pgfqpoint{6.400000in}{0.000000in}}%
\pgfpathlineto{\pgfqpoint{6.400000in}{4.800000in}}%
\pgfpathlineto{\pgfqpoint{0.000000in}{4.800000in}}%
\pgfpathclose%
\pgfusepath{fill}%
\end{pgfscope}%
\begin{pgfscope}%
\pgfsetbuttcap%
\pgfsetmiterjoin%
\definecolor{currentfill}{rgb}{1.000000,1.000000,1.000000}%
\pgfsetfillcolor{currentfill}%
\pgfsetlinewidth{0.000000pt}%
\definecolor{currentstroke}{rgb}{0.000000,0.000000,0.000000}%
\pgfsetstrokecolor{currentstroke}%
\pgfsetstrokeopacity{0.000000}%
\pgfsetdash{}{0pt}%
\pgfpathmoveto{\pgfqpoint{0.800000in}{0.528000in}}%
\pgfpathlineto{\pgfqpoint{5.760000in}{0.528000in}}%
\pgfpathlineto{\pgfqpoint{5.760000in}{4.224000in}}%
\pgfpathlineto{\pgfqpoint{0.800000in}{4.224000in}}%
\pgfpathclose%
\pgfusepath{fill}%
\end{pgfscope}%
\begin{pgfscope}%
\pgfpathrectangle{\pgfqpoint{0.800000in}{0.528000in}}{\pgfqpoint{4.960000in}{3.696000in}}%
\pgfusepath{clip}%
\pgfsetbuttcap%
\pgfsetroundjoin%
\definecolor{currentfill}{rgb}{0.000000,0.000000,0.000000}%
\pgfsetfillcolor{currentfill}%
\pgfsetlinewidth{1.003750pt}%
\definecolor{currentstroke}{rgb}{0.000000,0.000000,0.000000}%
\pgfsetstrokecolor{currentstroke}%
\pgfsetdash{}{0pt}%
\pgfpathmoveto{\pgfqpoint{1.025906in}{0.698302in}}%
\pgfpathcurveto{\pgfqpoint{1.036956in}{0.698302in}}{\pgfqpoint{1.047555in}{0.702692in}}{\pgfqpoint{1.055369in}{0.710506in}}%
\pgfpathcurveto{\pgfqpoint{1.063182in}{0.718319in}}{\pgfqpoint{1.067573in}{0.728918in}}{\pgfqpoint{1.067573in}{0.739969in}}%
\pgfpathcurveto{\pgfqpoint{1.067573in}{0.751019in}}{\pgfqpoint{1.063182in}{0.761618in}}{\pgfqpoint{1.055369in}{0.769431in}}%
\pgfpathcurveto{\pgfqpoint{1.047555in}{0.777245in}}{\pgfqpoint{1.036956in}{0.781635in}}{\pgfqpoint{1.025906in}{0.781635in}}%
\pgfpathcurveto{\pgfqpoint{1.014856in}{0.781635in}}{\pgfqpoint{1.004257in}{0.777245in}}{\pgfqpoint{0.996443in}{0.769431in}}%
\pgfpathcurveto{\pgfqpoint{0.988630in}{0.761618in}}{\pgfqpoint{0.984239in}{0.751019in}}{\pgfqpoint{0.984239in}{0.739969in}}%
\pgfpathcurveto{\pgfqpoint{0.984239in}{0.728918in}}{\pgfqpoint{0.988630in}{0.718319in}}{\pgfqpoint{0.996443in}{0.710506in}}%
\pgfpathcurveto{\pgfqpoint{1.004257in}{0.702692in}}{\pgfqpoint{1.014856in}{0.698302in}}{\pgfqpoint{1.025906in}{0.698302in}}%
\pgfpathclose%
\pgfusepath{stroke,fill}%
\end{pgfscope}%
\begin{pgfscope}%
\pgfpathrectangle{\pgfqpoint{0.800000in}{0.528000in}}{\pgfqpoint{4.960000in}{3.696000in}}%
\pgfusepath{clip}%
\pgfsetbuttcap%
\pgfsetroundjoin%
\definecolor{currentfill}{rgb}{0.000000,0.000000,0.000000}%
\pgfsetfillcolor{currentfill}%
\pgfsetlinewidth{1.003750pt}%
\definecolor{currentstroke}{rgb}{0.000000,0.000000,0.000000}%
\pgfsetstrokecolor{currentstroke}%
\pgfsetdash{}{0pt}%
\pgfpathmoveto{\pgfqpoint{1.025906in}{0.698302in}}%
\pgfpathcurveto{\pgfqpoint{1.036956in}{0.698302in}}{\pgfqpoint{1.047555in}{0.702692in}}{\pgfqpoint{1.055369in}{0.710506in}}%
\pgfpathcurveto{\pgfqpoint{1.063182in}{0.718319in}}{\pgfqpoint{1.067573in}{0.728918in}}{\pgfqpoint{1.067573in}{0.739969in}}%
\pgfpathcurveto{\pgfqpoint{1.067573in}{0.751019in}}{\pgfqpoint{1.063182in}{0.761618in}}{\pgfqpoint{1.055369in}{0.769431in}}%
\pgfpathcurveto{\pgfqpoint{1.047555in}{0.777245in}}{\pgfqpoint{1.036956in}{0.781635in}}{\pgfqpoint{1.025906in}{0.781635in}}%
\pgfpathcurveto{\pgfqpoint{1.014856in}{0.781635in}}{\pgfqpoint{1.004257in}{0.777245in}}{\pgfqpoint{0.996443in}{0.769431in}}%
\pgfpathcurveto{\pgfqpoint{0.988630in}{0.761618in}}{\pgfqpoint{0.984239in}{0.751019in}}{\pgfqpoint{0.984239in}{0.739969in}}%
\pgfpathcurveto{\pgfqpoint{0.984239in}{0.728918in}}{\pgfqpoint{0.988630in}{0.718319in}}{\pgfqpoint{0.996443in}{0.710506in}}%
\pgfpathcurveto{\pgfqpoint{1.004257in}{0.702692in}}{\pgfqpoint{1.014856in}{0.698302in}}{\pgfqpoint{1.025906in}{0.698302in}}%
\pgfpathclose%
\pgfusepath{stroke,fill}%
\end{pgfscope}%
\begin{pgfscope}%
\pgfpathrectangle{\pgfqpoint{0.800000in}{0.528000in}}{\pgfqpoint{4.960000in}{3.696000in}}%
\pgfusepath{clip}%
\pgfsetbuttcap%
\pgfsetroundjoin%
\definecolor{currentfill}{rgb}{0.000000,0.000000,0.000000}%
\pgfsetfillcolor{currentfill}%
\pgfsetlinewidth{1.003750pt}%
\definecolor{currentstroke}{rgb}{0.000000,0.000000,0.000000}%
\pgfsetstrokecolor{currentstroke}%
\pgfsetdash{}{0pt}%
\pgfpathmoveto{\pgfqpoint{1.025906in}{0.676804in}}%
\pgfpathcurveto{\pgfqpoint{1.036956in}{0.676804in}}{\pgfqpoint{1.047555in}{0.681195in}}{\pgfqpoint{1.055369in}{0.689008in}}%
\pgfpathcurveto{\pgfqpoint{1.063182in}{0.696822in}}{\pgfqpoint{1.067573in}{0.707421in}}{\pgfqpoint{1.067573in}{0.718471in}}%
\pgfpathcurveto{\pgfqpoint{1.067573in}{0.729521in}}{\pgfqpoint{1.063182in}{0.740120in}}{\pgfqpoint{1.055369in}{0.747934in}}%
\pgfpathcurveto{\pgfqpoint{1.047555in}{0.755748in}}{\pgfqpoint{1.036956in}{0.760138in}}{\pgfqpoint{1.025906in}{0.760138in}}%
\pgfpathcurveto{\pgfqpoint{1.014856in}{0.760138in}}{\pgfqpoint{1.004257in}{0.755748in}}{\pgfqpoint{0.996443in}{0.747934in}}%
\pgfpathcurveto{\pgfqpoint{0.988630in}{0.740120in}}{\pgfqpoint{0.984239in}{0.729521in}}{\pgfqpoint{0.984239in}{0.718471in}}%
\pgfpathcurveto{\pgfqpoint{0.984239in}{0.707421in}}{\pgfqpoint{0.988630in}{0.696822in}}{\pgfqpoint{0.996443in}{0.689008in}}%
\pgfpathcurveto{\pgfqpoint{1.004257in}{0.681195in}}{\pgfqpoint{1.014856in}{0.676804in}}{\pgfqpoint{1.025906in}{0.676804in}}%
\pgfpathclose%
\pgfusepath{stroke,fill}%
\end{pgfscope}%
\begin{pgfscope}%
\pgfpathrectangle{\pgfqpoint{0.800000in}{0.528000in}}{\pgfqpoint{4.960000in}{3.696000in}}%
\pgfusepath{clip}%
\pgfsetbuttcap%
\pgfsetroundjoin%
\definecolor{currentfill}{rgb}{0.000000,0.000000,0.000000}%
\pgfsetfillcolor{currentfill}%
\pgfsetlinewidth{1.003750pt}%
\definecolor{currentstroke}{rgb}{0.000000,0.000000,0.000000}%
\pgfsetstrokecolor{currentstroke}%
\pgfsetdash{}{0pt}%
\pgfpathmoveto{\pgfqpoint{1.025906in}{0.655307in}}%
\pgfpathcurveto{\pgfqpoint{1.036956in}{0.655307in}}{\pgfqpoint{1.047555in}{0.659697in}}{\pgfqpoint{1.055369in}{0.667511in}}%
\pgfpathcurveto{\pgfqpoint{1.063182in}{0.675324in}}{\pgfqpoint{1.067573in}{0.685924in}}{\pgfqpoint{1.067573in}{0.696974in}}%
\pgfpathcurveto{\pgfqpoint{1.067573in}{0.708024in}}{\pgfqpoint{1.063182in}{0.718623in}}{\pgfqpoint{1.055369in}{0.726436in}}%
\pgfpathcurveto{\pgfqpoint{1.047555in}{0.734250in}}{\pgfqpoint{1.036956in}{0.738640in}}{\pgfqpoint{1.025906in}{0.738640in}}%
\pgfpathcurveto{\pgfqpoint{1.014856in}{0.738640in}}{\pgfqpoint{1.004257in}{0.734250in}}{\pgfqpoint{0.996443in}{0.726436in}}%
\pgfpathcurveto{\pgfqpoint{0.988630in}{0.718623in}}{\pgfqpoint{0.984239in}{0.708024in}}{\pgfqpoint{0.984239in}{0.696974in}}%
\pgfpathcurveto{\pgfqpoint{0.984239in}{0.685924in}}{\pgfqpoint{0.988630in}{0.675324in}}{\pgfqpoint{0.996443in}{0.667511in}}%
\pgfpathcurveto{\pgfqpoint{1.004257in}{0.659697in}}{\pgfqpoint{1.014856in}{0.655307in}}{\pgfqpoint{1.025906in}{0.655307in}}%
\pgfpathclose%
\pgfusepath{stroke,fill}%
\end{pgfscope}%
\begin{pgfscope}%
\pgfpathrectangle{\pgfqpoint{0.800000in}{0.528000in}}{\pgfqpoint{4.960000in}{3.696000in}}%
\pgfusepath{clip}%
\pgfsetbuttcap%
\pgfsetroundjoin%
\definecolor{currentfill}{rgb}{0.000000,0.000000,0.000000}%
\pgfsetfillcolor{currentfill}%
\pgfsetlinewidth{1.003750pt}%
\definecolor{currentstroke}{rgb}{0.000000,0.000000,0.000000}%
\pgfsetstrokecolor{currentstroke}%
\pgfsetdash{}{0pt}%
\pgfpathmoveto{\pgfqpoint{1.025906in}{0.698302in}}%
\pgfpathcurveto{\pgfqpoint{1.036956in}{0.698302in}}{\pgfqpoint{1.047555in}{0.702692in}}{\pgfqpoint{1.055369in}{0.710506in}}%
\pgfpathcurveto{\pgfqpoint{1.063182in}{0.718319in}}{\pgfqpoint{1.067573in}{0.728918in}}{\pgfqpoint{1.067573in}{0.739969in}}%
\pgfpathcurveto{\pgfqpoint{1.067573in}{0.751019in}}{\pgfqpoint{1.063182in}{0.761618in}}{\pgfqpoint{1.055369in}{0.769431in}}%
\pgfpathcurveto{\pgfqpoint{1.047555in}{0.777245in}}{\pgfqpoint{1.036956in}{0.781635in}}{\pgfqpoint{1.025906in}{0.781635in}}%
\pgfpathcurveto{\pgfqpoint{1.014856in}{0.781635in}}{\pgfqpoint{1.004257in}{0.777245in}}{\pgfqpoint{0.996443in}{0.769431in}}%
\pgfpathcurveto{\pgfqpoint{0.988630in}{0.761618in}}{\pgfqpoint{0.984239in}{0.751019in}}{\pgfqpoint{0.984239in}{0.739969in}}%
\pgfpathcurveto{\pgfqpoint{0.984239in}{0.728918in}}{\pgfqpoint{0.988630in}{0.718319in}}{\pgfqpoint{0.996443in}{0.710506in}}%
\pgfpathcurveto{\pgfqpoint{1.004257in}{0.702692in}}{\pgfqpoint{1.014856in}{0.698302in}}{\pgfqpoint{1.025906in}{0.698302in}}%
\pgfpathclose%
\pgfusepath{stroke,fill}%
\end{pgfscope}%
\begin{pgfscope}%
\pgfpathrectangle{\pgfqpoint{0.800000in}{0.528000in}}{\pgfqpoint{4.960000in}{3.696000in}}%
\pgfusepath{clip}%
\pgfsetbuttcap%
\pgfsetroundjoin%
\definecolor{currentfill}{rgb}{0.000000,0.000000,0.000000}%
\pgfsetfillcolor{currentfill}%
\pgfsetlinewidth{1.003750pt}%
\definecolor{currentstroke}{rgb}{0.000000,0.000000,0.000000}%
\pgfsetstrokecolor{currentstroke}%
\pgfsetdash{}{0pt}%
\pgfpathmoveto{\pgfqpoint{1.025906in}{0.719799in}}%
\pgfpathcurveto{\pgfqpoint{1.036956in}{0.719799in}}{\pgfqpoint{1.047555in}{0.724190in}}{\pgfqpoint{1.055369in}{0.732003in}}%
\pgfpathcurveto{\pgfqpoint{1.063182in}{0.739817in}}{\pgfqpoint{1.067573in}{0.750416in}}{\pgfqpoint{1.067573in}{0.761466in}}%
\pgfpathcurveto{\pgfqpoint{1.067573in}{0.772516in}}{\pgfqpoint{1.063182in}{0.783115in}}{\pgfqpoint{1.055369in}{0.790929in}}%
\pgfpathcurveto{\pgfqpoint{1.047555in}{0.798743in}}{\pgfqpoint{1.036956in}{0.803133in}}{\pgfqpoint{1.025906in}{0.803133in}}%
\pgfpathcurveto{\pgfqpoint{1.014856in}{0.803133in}}{\pgfqpoint{1.004257in}{0.798743in}}{\pgfqpoint{0.996443in}{0.790929in}}%
\pgfpathcurveto{\pgfqpoint{0.988630in}{0.783115in}}{\pgfqpoint{0.984239in}{0.772516in}}{\pgfqpoint{0.984239in}{0.761466in}}%
\pgfpathcurveto{\pgfqpoint{0.984239in}{0.750416in}}{\pgfqpoint{0.988630in}{0.739817in}}{\pgfqpoint{0.996443in}{0.732003in}}%
\pgfpathcurveto{\pgfqpoint{1.004257in}{0.724190in}}{\pgfqpoint{1.014856in}{0.719799in}}{\pgfqpoint{1.025906in}{0.719799in}}%
\pgfpathclose%
\pgfusepath{stroke,fill}%
\end{pgfscope}%
\begin{pgfscope}%
\pgfpathrectangle{\pgfqpoint{0.800000in}{0.528000in}}{\pgfqpoint{4.960000in}{3.696000in}}%
\pgfusepath{clip}%
\pgfsetbuttcap%
\pgfsetroundjoin%
\definecolor{currentfill}{rgb}{0.000000,0.000000,0.000000}%
\pgfsetfillcolor{currentfill}%
\pgfsetlinewidth{1.003750pt}%
\definecolor{currentstroke}{rgb}{0.000000,0.000000,0.000000}%
\pgfsetstrokecolor{currentstroke}%
\pgfsetdash{}{0pt}%
\pgfpathmoveto{\pgfqpoint{1.025906in}{0.676804in}}%
\pgfpathcurveto{\pgfqpoint{1.036956in}{0.676804in}}{\pgfqpoint{1.047555in}{0.681195in}}{\pgfqpoint{1.055369in}{0.689008in}}%
\pgfpathcurveto{\pgfqpoint{1.063182in}{0.696822in}}{\pgfqpoint{1.067573in}{0.707421in}}{\pgfqpoint{1.067573in}{0.718471in}}%
\pgfpathcurveto{\pgfqpoint{1.067573in}{0.729521in}}{\pgfqpoint{1.063182in}{0.740120in}}{\pgfqpoint{1.055369in}{0.747934in}}%
\pgfpathcurveto{\pgfqpoint{1.047555in}{0.755748in}}{\pgfqpoint{1.036956in}{0.760138in}}{\pgfqpoint{1.025906in}{0.760138in}}%
\pgfpathcurveto{\pgfqpoint{1.014856in}{0.760138in}}{\pgfqpoint{1.004257in}{0.755748in}}{\pgfqpoint{0.996443in}{0.747934in}}%
\pgfpathcurveto{\pgfqpoint{0.988630in}{0.740120in}}{\pgfqpoint{0.984239in}{0.729521in}}{\pgfqpoint{0.984239in}{0.718471in}}%
\pgfpathcurveto{\pgfqpoint{0.984239in}{0.707421in}}{\pgfqpoint{0.988630in}{0.696822in}}{\pgfqpoint{0.996443in}{0.689008in}}%
\pgfpathcurveto{\pgfqpoint{1.004257in}{0.681195in}}{\pgfqpoint{1.014856in}{0.676804in}}{\pgfqpoint{1.025906in}{0.676804in}}%
\pgfpathclose%
\pgfusepath{stroke,fill}%
\end{pgfscope}%
\begin{pgfscope}%
\pgfpathrectangle{\pgfqpoint{0.800000in}{0.528000in}}{\pgfqpoint{4.960000in}{3.696000in}}%
\pgfusepath{clip}%
\pgfsetbuttcap%
\pgfsetroundjoin%
\definecolor{currentfill}{rgb}{0.000000,0.000000,0.000000}%
\pgfsetfillcolor{currentfill}%
\pgfsetlinewidth{1.003750pt}%
\definecolor{currentstroke}{rgb}{0.000000,0.000000,0.000000}%
\pgfsetstrokecolor{currentstroke}%
\pgfsetdash{}{0pt}%
\pgfpathmoveto{\pgfqpoint{1.025906in}{0.676804in}}%
\pgfpathcurveto{\pgfqpoint{1.036956in}{0.676804in}}{\pgfqpoint{1.047555in}{0.681195in}}{\pgfqpoint{1.055369in}{0.689008in}}%
\pgfpathcurveto{\pgfqpoint{1.063182in}{0.696822in}}{\pgfqpoint{1.067573in}{0.707421in}}{\pgfqpoint{1.067573in}{0.718471in}}%
\pgfpathcurveto{\pgfqpoint{1.067573in}{0.729521in}}{\pgfqpoint{1.063182in}{0.740120in}}{\pgfqpoint{1.055369in}{0.747934in}}%
\pgfpathcurveto{\pgfqpoint{1.047555in}{0.755748in}}{\pgfqpoint{1.036956in}{0.760138in}}{\pgfqpoint{1.025906in}{0.760138in}}%
\pgfpathcurveto{\pgfqpoint{1.014856in}{0.760138in}}{\pgfqpoint{1.004257in}{0.755748in}}{\pgfqpoint{0.996443in}{0.747934in}}%
\pgfpathcurveto{\pgfqpoint{0.988630in}{0.740120in}}{\pgfqpoint{0.984239in}{0.729521in}}{\pgfqpoint{0.984239in}{0.718471in}}%
\pgfpathcurveto{\pgfqpoint{0.984239in}{0.707421in}}{\pgfqpoint{0.988630in}{0.696822in}}{\pgfqpoint{0.996443in}{0.689008in}}%
\pgfpathcurveto{\pgfqpoint{1.004257in}{0.681195in}}{\pgfqpoint{1.014856in}{0.676804in}}{\pgfqpoint{1.025906in}{0.676804in}}%
\pgfpathclose%
\pgfusepath{stroke,fill}%
\end{pgfscope}%
\begin{pgfscope}%
\pgfpathrectangle{\pgfqpoint{0.800000in}{0.528000in}}{\pgfqpoint{4.960000in}{3.696000in}}%
\pgfusepath{clip}%
\pgfsetbuttcap%
\pgfsetroundjoin%
\definecolor{currentfill}{rgb}{0.000000,0.000000,0.000000}%
\pgfsetfillcolor{currentfill}%
\pgfsetlinewidth{1.003750pt}%
\definecolor{currentstroke}{rgb}{0.000000,0.000000,0.000000}%
\pgfsetstrokecolor{currentstroke}%
\pgfsetdash{}{0pt}%
\pgfpathmoveto{\pgfqpoint{1.025906in}{0.655307in}}%
\pgfpathcurveto{\pgfqpoint{1.036956in}{0.655307in}}{\pgfqpoint{1.047555in}{0.659697in}}{\pgfqpoint{1.055369in}{0.667511in}}%
\pgfpathcurveto{\pgfqpoint{1.063182in}{0.675324in}}{\pgfqpoint{1.067573in}{0.685924in}}{\pgfqpoint{1.067573in}{0.696974in}}%
\pgfpathcurveto{\pgfqpoint{1.067573in}{0.708024in}}{\pgfqpoint{1.063182in}{0.718623in}}{\pgfqpoint{1.055369in}{0.726436in}}%
\pgfpathcurveto{\pgfqpoint{1.047555in}{0.734250in}}{\pgfqpoint{1.036956in}{0.738640in}}{\pgfqpoint{1.025906in}{0.738640in}}%
\pgfpathcurveto{\pgfqpoint{1.014856in}{0.738640in}}{\pgfqpoint{1.004257in}{0.734250in}}{\pgfqpoint{0.996443in}{0.726436in}}%
\pgfpathcurveto{\pgfqpoint{0.988630in}{0.718623in}}{\pgfqpoint{0.984239in}{0.708024in}}{\pgfqpoint{0.984239in}{0.696974in}}%
\pgfpathcurveto{\pgfqpoint{0.984239in}{0.685924in}}{\pgfqpoint{0.988630in}{0.675324in}}{\pgfqpoint{0.996443in}{0.667511in}}%
\pgfpathcurveto{\pgfqpoint{1.004257in}{0.659697in}}{\pgfqpoint{1.014856in}{0.655307in}}{\pgfqpoint{1.025906in}{0.655307in}}%
\pgfpathclose%
\pgfusepath{stroke,fill}%
\end{pgfscope}%
\begin{pgfscope}%
\pgfpathrectangle{\pgfqpoint{0.800000in}{0.528000in}}{\pgfqpoint{4.960000in}{3.696000in}}%
\pgfusepath{clip}%
\pgfsetbuttcap%
\pgfsetroundjoin%
\definecolor{currentfill}{rgb}{0.000000,0.000000,0.000000}%
\pgfsetfillcolor{currentfill}%
\pgfsetlinewidth{1.003750pt}%
\definecolor{currentstroke}{rgb}{0.000000,0.000000,0.000000}%
\pgfsetstrokecolor{currentstroke}%
\pgfsetdash{}{0pt}%
\pgfpathmoveto{\pgfqpoint{1.025906in}{0.698302in}}%
\pgfpathcurveto{\pgfqpoint{1.036956in}{0.698302in}}{\pgfqpoint{1.047555in}{0.702692in}}{\pgfqpoint{1.055369in}{0.710506in}}%
\pgfpathcurveto{\pgfqpoint{1.063182in}{0.718319in}}{\pgfqpoint{1.067573in}{0.728918in}}{\pgfqpoint{1.067573in}{0.739969in}}%
\pgfpathcurveto{\pgfqpoint{1.067573in}{0.751019in}}{\pgfqpoint{1.063182in}{0.761618in}}{\pgfqpoint{1.055369in}{0.769431in}}%
\pgfpathcurveto{\pgfqpoint{1.047555in}{0.777245in}}{\pgfqpoint{1.036956in}{0.781635in}}{\pgfqpoint{1.025906in}{0.781635in}}%
\pgfpathcurveto{\pgfqpoint{1.014856in}{0.781635in}}{\pgfqpoint{1.004257in}{0.777245in}}{\pgfqpoint{0.996443in}{0.769431in}}%
\pgfpathcurveto{\pgfqpoint{0.988630in}{0.761618in}}{\pgfqpoint{0.984239in}{0.751019in}}{\pgfqpoint{0.984239in}{0.739969in}}%
\pgfpathcurveto{\pgfqpoint{0.984239in}{0.728918in}}{\pgfqpoint{0.988630in}{0.718319in}}{\pgfqpoint{0.996443in}{0.710506in}}%
\pgfpathcurveto{\pgfqpoint{1.004257in}{0.702692in}}{\pgfqpoint{1.014856in}{0.698302in}}{\pgfqpoint{1.025906in}{0.698302in}}%
\pgfpathclose%
\pgfusepath{stroke,fill}%
\end{pgfscope}%
\begin{pgfscope}%
\pgfpathrectangle{\pgfqpoint{0.800000in}{0.528000in}}{\pgfqpoint{4.960000in}{3.696000in}}%
\pgfusepath{clip}%
\pgfsetbuttcap%
\pgfsetroundjoin%
\definecolor{currentfill}{rgb}{0.000000,0.000000,0.000000}%
\pgfsetfillcolor{currentfill}%
\pgfsetlinewidth{1.003750pt}%
\definecolor{currentstroke}{rgb}{0.000000,0.000000,0.000000}%
\pgfsetstrokecolor{currentstroke}%
\pgfsetdash{}{0pt}%
\pgfpathmoveto{\pgfqpoint{1.025906in}{0.719799in}}%
\pgfpathcurveto{\pgfqpoint{1.036956in}{0.719799in}}{\pgfqpoint{1.047555in}{0.724190in}}{\pgfqpoint{1.055369in}{0.732003in}}%
\pgfpathcurveto{\pgfqpoint{1.063182in}{0.739817in}}{\pgfqpoint{1.067573in}{0.750416in}}{\pgfqpoint{1.067573in}{0.761466in}}%
\pgfpathcurveto{\pgfqpoint{1.067573in}{0.772516in}}{\pgfqpoint{1.063182in}{0.783115in}}{\pgfqpoint{1.055369in}{0.790929in}}%
\pgfpathcurveto{\pgfqpoint{1.047555in}{0.798743in}}{\pgfqpoint{1.036956in}{0.803133in}}{\pgfqpoint{1.025906in}{0.803133in}}%
\pgfpathcurveto{\pgfqpoint{1.014856in}{0.803133in}}{\pgfqpoint{1.004257in}{0.798743in}}{\pgfqpoint{0.996443in}{0.790929in}}%
\pgfpathcurveto{\pgfqpoint{0.988630in}{0.783115in}}{\pgfqpoint{0.984239in}{0.772516in}}{\pgfqpoint{0.984239in}{0.761466in}}%
\pgfpathcurveto{\pgfqpoint{0.984239in}{0.750416in}}{\pgfqpoint{0.988630in}{0.739817in}}{\pgfqpoint{0.996443in}{0.732003in}}%
\pgfpathcurveto{\pgfqpoint{1.004257in}{0.724190in}}{\pgfqpoint{1.014856in}{0.719799in}}{\pgfqpoint{1.025906in}{0.719799in}}%
\pgfpathclose%
\pgfusepath{stroke,fill}%
\end{pgfscope}%
\begin{pgfscope}%
\pgfpathrectangle{\pgfqpoint{0.800000in}{0.528000in}}{\pgfqpoint{4.960000in}{3.696000in}}%
\pgfusepath{clip}%
\pgfsetbuttcap%
\pgfsetroundjoin%
\definecolor{currentfill}{rgb}{0.000000,0.000000,0.000000}%
\pgfsetfillcolor{currentfill}%
\pgfsetlinewidth{1.003750pt}%
\definecolor{currentstroke}{rgb}{0.000000,0.000000,0.000000}%
\pgfsetstrokecolor{currentstroke}%
\pgfsetdash{}{0pt}%
\pgfpathmoveto{\pgfqpoint{1.025906in}{0.698302in}}%
\pgfpathcurveto{\pgfqpoint{1.036956in}{0.698302in}}{\pgfqpoint{1.047555in}{0.702692in}}{\pgfqpoint{1.055369in}{0.710506in}}%
\pgfpathcurveto{\pgfqpoint{1.063182in}{0.718319in}}{\pgfqpoint{1.067573in}{0.728918in}}{\pgfqpoint{1.067573in}{0.739969in}}%
\pgfpathcurveto{\pgfqpoint{1.067573in}{0.751019in}}{\pgfqpoint{1.063182in}{0.761618in}}{\pgfqpoint{1.055369in}{0.769431in}}%
\pgfpathcurveto{\pgfqpoint{1.047555in}{0.777245in}}{\pgfqpoint{1.036956in}{0.781635in}}{\pgfqpoint{1.025906in}{0.781635in}}%
\pgfpathcurveto{\pgfqpoint{1.014856in}{0.781635in}}{\pgfqpoint{1.004257in}{0.777245in}}{\pgfqpoint{0.996443in}{0.769431in}}%
\pgfpathcurveto{\pgfqpoint{0.988630in}{0.761618in}}{\pgfqpoint{0.984239in}{0.751019in}}{\pgfqpoint{0.984239in}{0.739969in}}%
\pgfpathcurveto{\pgfqpoint{0.984239in}{0.728918in}}{\pgfqpoint{0.988630in}{0.718319in}}{\pgfqpoint{0.996443in}{0.710506in}}%
\pgfpathcurveto{\pgfqpoint{1.004257in}{0.702692in}}{\pgfqpoint{1.014856in}{0.698302in}}{\pgfqpoint{1.025906in}{0.698302in}}%
\pgfpathclose%
\pgfusepath{stroke,fill}%
\end{pgfscope}%
\begin{pgfscope}%
\pgfpathrectangle{\pgfqpoint{0.800000in}{0.528000in}}{\pgfqpoint{4.960000in}{3.696000in}}%
\pgfusepath{clip}%
\pgfsetbuttcap%
\pgfsetroundjoin%
\definecolor{currentfill}{rgb}{0.000000,0.000000,0.000000}%
\pgfsetfillcolor{currentfill}%
\pgfsetlinewidth{1.003750pt}%
\definecolor{currentstroke}{rgb}{0.000000,0.000000,0.000000}%
\pgfsetstrokecolor{currentstroke}%
\pgfsetdash{}{0pt}%
\pgfpathmoveto{\pgfqpoint{1.025906in}{0.719799in}}%
\pgfpathcurveto{\pgfqpoint{1.036956in}{0.719799in}}{\pgfqpoint{1.047555in}{0.724190in}}{\pgfqpoint{1.055369in}{0.732003in}}%
\pgfpathcurveto{\pgfqpoint{1.063182in}{0.739817in}}{\pgfqpoint{1.067573in}{0.750416in}}{\pgfqpoint{1.067573in}{0.761466in}}%
\pgfpathcurveto{\pgfqpoint{1.067573in}{0.772516in}}{\pgfqpoint{1.063182in}{0.783115in}}{\pgfqpoint{1.055369in}{0.790929in}}%
\pgfpathcurveto{\pgfqpoint{1.047555in}{0.798743in}}{\pgfqpoint{1.036956in}{0.803133in}}{\pgfqpoint{1.025906in}{0.803133in}}%
\pgfpathcurveto{\pgfqpoint{1.014856in}{0.803133in}}{\pgfqpoint{1.004257in}{0.798743in}}{\pgfqpoint{0.996443in}{0.790929in}}%
\pgfpathcurveto{\pgfqpoint{0.988630in}{0.783115in}}{\pgfqpoint{0.984239in}{0.772516in}}{\pgfqpoint{0.984239in}{0.761466in}}%
\pgfpathcurveto{\pgfqpoint{0.984239in}{0.750416in}}{\pgfqpoint{0.988630in}{0.739817in}}{\pgfqpoint{0.996443in}{0.732003in}}%
\pgfpathcurveto{\pgfqpoint{1.004257in}{0.724190in}}{\pgfqpoint{1.014856in}{0.719799in}}{\pgfqpoint{1.025906in}{0.719799in}}%
\pgfpathclose%
\pgfusepath{stroke,fill}%
\end{pgfscope}%
\begin{pgfscope}%
\pgfpathrectangle{\pgfqpoint{0.800000in}{0.528000in}}{\pgfqpoint{4.960000in}{3.696000in}}%
\pgfusepath{clip}%
\pgfsetbuttcap%
\pgfsetroundjoin%
\definecolor{currentfill}{rgb}{0.000000,0.000000,0.000000}%
\pgfsetfillcolor{currentfill}%
\pgfsetlinewidth{1.003750pt}%
\definecolor{currentstroke}{rgb}{0.000000,0.000000,0.000000}%
\pgfsetstrokecolor{currentstroke}%
\pgfsetdash{}{0pt}%
\pgfpathmoveto{\pgfqpoint{1.025906in}{0.719799in}}%
\pgfpathcurveto{\pgfqpoint{1.036956in}{0.719799in}}{\pgfqpoint{1.047555in}{0.724190in}}{\pgfqpoint{1.055369in}{0.732003in}}%
\pgfpathcurveto{\pgfqpoint{1.063182in}{0.739817in}}{\pgfqpoint{1.067573in}{0.750416in}}{\pgfqpoint{1.067573in}{0.761466in}}%
\pgfpathcurveto{\pgfqpoint{1.067573in}{0.772516in}}{\pgfqpoint{1.063182in}{0.783115in}}{\pgfqpoint{1.055369in}{0.790929in}}%
\pgfpathcurveto{\pgfqpoint{1.047555in}{0.798743in}}{\pgfqpoint{1.036956in}{0.803133in}}{\pgfqpoint{1.025906in}{0.803133in}}%
\pgfpathcurveto{\pgfqpoint{1.014856in}{0.803133in}}{\pgfqpoint{1.004257in}{0.798743in}}{\pgfqpoint{0.996443in}{0.790929in}}%
\pgfpathcurveto{\pgfqpoint{0.988630in}{0.783115in}}{\pgfqpoint{0.984239in}{0.772516in}}{\pgfqpoint{0.984239in}{0.761466in}}%
\pgfpathcurveto{\pgfqpoint{0.984239in}{0.750416in}}{\pgfqpoint{0.988630in}{0.739817in}}{\pgfqpoint{0.996443in}{0.732003in}}%
\pgfpathcurveto{\pgfqpoint{1.004257in}{0.724190in}}{\pgfqpoint{1.014856in}{0.719799in}}{\pgfqpoint{1.025906in}{0.719799in}}%
\pgfpathclose%
\pgfusepath{stroke,fill}%
\end{pgfscope}%
\begin{pgfscope}%
\pgfpathrectangle{\pgfqpoint{0.800000in}{0.528000in}}{\pgfqpoint{4.960000in}{3.696000in}}%
\pgfusepath{clip}%
\pgfsetbuttcap%
\pgfsetroundjoin%
\definecolor{currentfill}{rgb}{0.000000,0.000000,0.000000}%
\pgfsetfillcolor{currentfill}%
\pgfsetlinewidth{1.003750pt}%
\definecolor{currentstroke}{rgb}{0.000000,0.000000,0.000000}%
\pgfsetstrokecolor{currentstroke}%
\pgfsetdash{}{0pt}%
\pgfpathmoveto{\pgfqpoint{1.025906in}{0.698302in}}%
\pgfpathcurveto{\pgfqpoint{1.036956in}{0.698302in}}{\pgfqpoint{1.047555in}{0.702692in}}{\pgfqpoint{1.055369in}{0.710506in}}%
\pgfpathcurveto{\pgfqpoint{1.063182in}{0.718319in}}{\pgfqpoint{1.067573in}{0.728918in}}{\pgfqpoint{1.067573in}{0.739969in}}%
\pgfpathcurveto{\pgfqpoint{1.067573in}{0.751019in}}{\pgfqpoint{1.063182in}{0.761618in}}{\pgfqpoint{1.055369in}{0.769431in}}%
\pgfpathcurveto{\pgfqpoint{1.047555in}{0.777245in}}{\pgfqpoint{1.036956in}{0.781635in}}{\pgfqpoint{1.025906in}{0.781635in}}%
\pgfpathcurveto{\pgfqpoint{1.014856in}{0.781635in}}{\pgfqpoint{1.004257in}{0.777245in}}{\pgfqpoint{0.996443in}{0.769431in}}%
\pgfpathcurveto{\pgfqpoint{0.988630in}{0.761618in}}{\pgfqpoint{0.984239in}{0.751019in}}{\pgfqpoint{0.984239in}{0.739969in}}%
\pgfpathcurveto{\pgfqpoint{0.984239in}{0.728918in}}{\pgfqpoint{0.988630in}{0.718319in}}{\pgfqpoint{0.996443in}{0.710506in}}%
\pgfpathcurveto{\pgfqpoint{1.004257in}{0.702692in}}{\pgfqpoint{1.014856in}{0.698302in}}{\pgfqpoint{1.025906in}{0.698302in}}%
\pgfpathclose%
\pgfusepath{stroke,fill}%
\end{pgfscope}%
\begin{pgfscope}%
\pgfpathrectangle{\pgfqpoint{0.800000in}{0.528000in}}{\pgfqpoint{4.960000in}{3.696000in}}%
\pgfusepath{clip}%
\pgfsetbuttcap%
\pgfsetroundjoin%
\definecolor{currentfill}{rgb}{0.000000,0.000000,0.000000}%
\pgfsetfillcolor{currentfill}%
\pgfsetlinewidth{1.003750pt}%
\definecolor{currentstroke}{rgb}{0.000000,0.000000,0.000000}%
\pgfsetstrokecolor{currentstroke}%
\pgfsetdash{}{0pt}%
\pgfpathmoveto{\pgfqpoint{1.025906in}{0.698302in}}%
\pgfpathcurveto{\pgfqpoint{1.036956in}{0.698302in}}{\pgfqpoint{1.047555in}{0.702692in}}{\pgfqpoint{1.055369in}{0.710506in}}%
\pgfpathcurveto{\pgfqpoint{1.063182in}{0.718319in}}{\pgfqpoint{1.067573in}{0.728918in}}{\pgfqpoint{1.067573in}{0.739969in}}%
\pgfpathcurveto{\pgfqpoint{1.067573in}{0.751019in}}{\pgfqpoint{1.063182in}{0.761618in}}{\pgfqpoint{1.055369in}{0.769431in}}%
\pgfpathcurveto{\pgfqpoint{1.047555in}{0.777245in}}{\pgfqpoint{1.036956in}{0.781635in}}{\pgfqpoint{1.025906in}{0.781635in}}%
\pgfpathcurveto{\pgfqpoint{1.014856in}{0.781635in}}{\pgfqpoint{1.004257in}{0.777245in}}{\pgfqpoint{0.996443in}{0.769431in}}%
\pgfpathcurveto{\pgfqpoint{0.988630in}{0.761618in}}{\pgfqpoint{0.984239in}{0.751019in}}{\pgfqpoint{0.984239in}{0.739969in}}%
\pgfpathcurveto{\pgfqpoint{0.984239in}{0.728918in}}{\pgfqpoint{0.988630in}{0.718319in}}{\pgfqpoint{0.996443in}{0.710506in}}%
\pgfpathcurveto{\pgfqpoint{1.004257in}{0.702692in}}{\pgfqpoint{1.014856in}{0.698302in}}{\pgfqpoint{1.025906in}{0.698302in}}%
\pgfpathclose%
\pgfusepath{stroke,fill}%
\end{pgfscope}%
\begin{pgfscope}%
\pgfpathrectangle{\pgfqpoint{0.800000in}{0.528000in}}{\pgfqpoint{4.960000in}{3.696000in}}%
\pgfusepath{clip}%
\pgfsetbuttcap%
\pgfsetroundjoin%
\definecolor{currentfill}{rgb}{0.000000,0.000000,0.000000}%
\pgfsetfillcolor{currentfill}%
\pgfsetlinewidth{1.003750pt}%
\definecolor{currentstroke}{rgb}{0.000000,0.000000,0.000000}%
\pgfsetstrokecolor{currentstroke}%
\pgfsetdash{}{0pt}%
\pgfpathmoveto{\pgfqpoint{1.025906in}{0.698302in}}%
\pgfpathcurveto{\pgfqpoint{1.036956in}{0.698302in}}{\pgfqpoint{1.047555in}{0.702692in}}{\pgfqpoint{1.055369in}{0.710506in}}%
\pgfpathcurveto{\pgfqpoint{1.063182in}{0.718319in}}{\pgfqpoint{1.067573in}{0.728918in}}{\pgfqpoint{1.067573in}{0.739969in}}%
\pgfpathcurveto{\pgfqpoint{1.067573in}{0.751019in}}{\pgfqpoint{1.063182in}{0.761618in}}{\pgfqpoint{1.055369in}{0.769431in}}%
\pgfpathcurveto{\pgfqpoint{1.047555in}{0.777245in}}{\pgfqpoint{1.036956in}{0.781635in}}{\pgfqpoint{1.025906in}{0.781635in}}%
\pgfpathcurveto{\pgfqpoint{1.014856in}{0.781635in}}{\pgfqpoint{1.004257in}{0.777245in}}{\pgfqpoint{0.996443in}{0.769431in}}%
\pgfpathcurveto{\pgfqpoint{0.988630in}{0.761618in}}{\pgfqpoint{0.984239in}{0.751019in}}{\pgfqpoint{0.984239in}{0.739969in}}%
\pgfpathcurveto{\pgfqpoint{0.984239in}{0.728918in}}{\pgfqpoint{0.988630in}{0.718319in}}{\pgfqpoint{0.996443in}{0.710506in}}%
\pgfpathcurveto{\pgfqpoint{1.004257in}{0.702692in}}{\pgfqpoint{1.014856in}{0.698302in}}{\pgfqpoint{1.025906in}{0.698302in}}%
\pgfpathclose%
\pgfusepath{stroke,fill}%
\end{pgfscope}%
\begin{pgfscope}%
\pgfpathrectangle{\pgfqpoint{0.800000in}{0.528000in}}{\pgfqpoint{4.960000in}{3.696000in}}%
\pgfusepath{clip}%
\pgfsetbuttcap%
\pgfsetroundjoin%
\definecolor{currentfill}{rgb}{0.000000,0.000000,0.000000}%
\pgfsetfillcolor{currentfill}%
\pgfsetlinewidth{1.003750pt}%
\definecolor{currentstroke}{rgb}{0.000000,0.000000,0.000000}%
\pgfsetstrokecolor{currentstroke}%
\pgfsetdash{}{0pt}%
\pgfpathmoveto{\pgfqpoint{1.025906in}{0.676804in}}%
\pgfpathcurveto{\pgfqpoint{1.036956in}{0.676804in}}{\pgfqpoint{1.047555in}{0.681195in}}{\pgfqpoint{1.055369in}{0.689008in}}%
\pgfpathcurveto{\pgfqpoint{1.063182in}{0.696822in}}{\pgfqpoint{1.067573in}{0.707421in}}{\pgfqpoint{1.067573in}{0.718471in}}%
\pgfpathcurveto{\pgfqpoint{1.067573in}{0.729521in}}{\pgfqpoint{1.063182in}{0.740120in}}{\pgfqpoint{1.055369in}{0.747934in}}%
\pgfpathcurveto{\pgfqpoint{1.047555in}{0.755748in}}{\pgfqpoint{1.036956in}{0.760138in}}{\pgfqpoint{1.025906in}{0.760138in}}%
\pgfpathcurveto{\pgfqpoint{1.014856in}{0.760138in}}{\pgfqpoint{1.004257in}{0.755748in}}{\pgfqpoint{0.996443in}{0.747934in}}%
\pgfpathcurveto{\pgfqpoint{0.988630in}{0.740120in}}{\pgfqpoint{0.984239in}{0.729521in}}{\pgfqpoint{0.984239in}{0.718471in}}%
\pgfpathcurveto{\pgfqpoint{0.984239in}{0.707421in}}{\pgfqpoint{0.988630in}{0.696822in}}{\pgfqpoint{0.996443in}{0.689008in}}%
\pgfpathcurveto{\pgfqpoint{1.004257in}{0.681195in}}{\pgfqpoint{1.014856in}{0.676804in}}{\pgfqpoint{1.025906in}{0.676804in}}%
\pgfpathclose%
\pgfusepath{stroke,fill}%
\end{pgfscope}%
\begin{pgfscope}%
\pgfpathrectangle{\pgfqpoint{0.800000in}{0.528000in}}{\pgfqpoint{4.960000in}{3.696000in}}%
\pgfusepath{clip}%
\pgfsetbuttcap%
\pgfsetroundjoin%
\definecolor{currentfill}{rgb}{0.000000,0.000000,0.000000}%
\pgfsetfillcolor{currentfill}%
\pgfsetlinewidth{1.003750pt}%
\definecolor{currentstroke}{rgb}{0.000000,0.000000,0.000000}%
\pgfsetstrokecolor{currentstroke}%
\pgfsetdash{}{0pt}%
\pgfpathmoveto{\pgfqpoint{1.025906in}{0.655307in}}%
\pgfpathcurveto{\pgfqpoint{1.036956in}{0.655307in}}{\pgfqpoint{1.047555in}{0.659697in}}{\pgfqpoint{1.055369in}{0.667511in}}%
\pgfpathcurveto{\pgfqpoint{1.063182in}{0.675324in}}{\pgfqpoint{1.067573in}{0.685924in}}{\pgfqpoint{1.067573in}{0.696974in}}%
\pgfpathcurveto{\pgfqpoint{1.067573in}{0.708024in}}{\pgfqpoint{1.063182in}{0.718623in}}{\pgfqpoint{1.055369in}{0.726436in}}%
\pgfpathcurveto{\pgfqpoint{1.047555in}{0.734250in}}{\pgfqpoint{1.036956in}{0.738640in}}{\pgfqpoint{1.025906in}{0.738640in}}%
\pgfpathcurveto{\pgfqpoint{1.014856in}{0.738640in}}{\pgfqpoint{1.004257in}{0.734250in}}{\pgfqpoint{0.996443in}{0.726436in}}%
\pgfpathcurveto{\pgfqpoint{0.988630in}{0.718623in}}{\pgfqpoint{0.984239in}{0.708024in}}{\pgfqpoint{0.984239in}{0.696974in}}%
\pgfpathcurveto{\pgfqpoint{0.984239in}{0.685924in}}{\pgfqpoint{0.988630in}{0.675324in}}{\pgfqpoint{0.996443in}{0.667511in}}%
\pgfpathcurveto{\pgfqpoint{1.004257in}{0.659697in}}{\pgfqpoint{1.014856in}{0.655307in}}{\pgfqpoint{1.025906in}{0.655307in}}%
\pgfpathclose%
\pgfusepath{stroke,fill}%
\end{pgfscope}%
\begin{pgfscope}%
\pgfpathrectangle{\pgfqpoint{0.800000in}{0.528000in}}{\pgfqpoint{4.960000in}{3.696000in}}%
\pgfusepath{clip}%
\pgfsetbuttcap%
\pgfsetroundjoin%
\definecolor{currentfill}{rgb}{0.000000,0.000000,0.000000}%
\pgfsetfillcolor{currentfill}%
\pgfsetlinewidth{1.003750pt}%
\definecolor{currentstroke}{rgb}{0.000000,0.000000,0.000000}%
\pgfsetstrokecolor{currentstroke}%
\pgfsetdash{}{0pt}%
\pgfpathmoveto{\pgfqpoint{1.025906in}{0.676804in}}%
\pgfpathcurveto{\pgfqpoint{1.036956in}{0.676804in}}{\pgfqpoint{1.047555in}{0.681195in}}{\pgfqpoint{1.055369in}{0.689008in}}%
\pgfpathcurveto{\pgfqpoint{1.063182in}{0.696822in}}{\pgfqpoint{1.067573in}{0.707421in}}{\pgfqpoint{1.067573in}{0.718471in}}%
\pgfpathcurveto{\pgfqpoint{1.067573in}{0.729521in}}{\pgfqpoint{1.063182in}{0.740120in}}{\pgfqpoint{1.055369in}{0.747934in}}%
\pgfpathcurveto{\pgfqpoint{1.047555in}{0.755748in}}{\pgfqpoint{1.036956in}{0.760138in}}{\pgfqpoint{1.025906in}{0.760138in}}%
\pgfpathcurveto{\pgfqpoint{1.014856in}{0.760138in}}{\pgfqpoint{1.004257in}{0.755748in}}{\pgfqpoint{0.996443in}{0.747934in}}%
\pgfpathcurveto{\pgfqpoint{0.988630in}{0.740120in}}{\pgfqpoint{0.984239in}{0.729521in}}{\pgfqpoint{0.984239in}{0.718471in}}%
\pgfpathcurveto{\pgfqpoint{0.984239in}{0.707421in}}{\pgfqpoint{0.988630in}{0.696822in}}{\pgfqpoint{0.996443in}{0.689008in}}%
\pgfpathcurveto{\pgfqpoint{1.004257in}{0.681195in}}{\pgfqpoint{1.014856in}{0.676804in}}{\pgfqpoint{1.025906in}{0.676804in}}%
\pgfpathclose%
\pgfusepath{stroke,fill}%
\end{pgfscope}%
\begin{pgfscope}%
\pgfpathrectangle{\pgfqpoint{0.800000in}{0.528000in}}{\pgfqpoint{4.960000in}{3.696000in}}%
\pgfusepath{clip}%
\pgfsetbuttcap%
\pgfsetroundjoin%
\definecolor{currentfill}{rgb}{0.000000,0.000000,0.000000}%
\pgfsetfillcolor{currentfill}%
\pgfsetlinewidth{1.003750pt}%
\definecolor{currentstroke}{rgb}{0.000000,0.000000,0.000000}%
\pgfsetstrokecolor{currentstroke}%
\pgfsetdash{}{0pt}%
\pgfpathmoveto{\pgfqpoint{1.025906in}{0.655307in}}%
\pgfpathcurveto{\pgfqpoint{1.036956in}{0.655307in}}{\pgfqpoint{1.047555in}{0.659697in}}{\pgfqpoint{1.055369in}{0.667511in}}%
\pgfpathcurveto{\pgfqpoint{1.063182in}{0.675324in}}{\pgfqpoint{1.067573in}{0.685924in}}{\pgfqpoint{1.067573in}{0.696974in}}%
\pgfpathcurveto{\pgfqpoint{1.067573in}{0.708024in}}{\pgfqpoint{1.063182in}{0.718623in}}{\pgfqpoint{1.055369in}{0.726436in}}%
\pgfpathcurveto{\pgfqpoint{1.047555in}{0.734250in}}{\pgfqpoint{1.036956in}{0.738640in}}{\pgfqpoint{1.025906in}{0.738640in}}%
\pgfpathcurveto{\pgfqpoint{1.014856in}{0.738640in}}{\pgfqpoint{1.004257in}{0.734250in}}{\pgfqpoint{0.996443in}{0.726436in}}%
\pgfpathcurveto{\pgfqpoint{0.988630in}{0.718623in}}{\pgfqpoint{0.984239in}{0.708024in}}{\pgfqpoint{0.984239in}{0.696974in}}%
\pgfpathcurveto{\pgfqpoint{0.984239in}{0.685924in}}{\pgfqpoint{0.988630in}{0.675324in}}{\pgfqpoint{0.996443in}{0.667511in}}%
\pgfpathcurveto{\pgfqpoint{1.004257in}{0.659697in}}{\pgfqpoint{1.014856in}{0.655307in}}{\pgfqpoint{1.025906in}{0.655307in}}%
\pgfpathclose%
\pgfusepath{stroke,fill}%
\end{pgfscope}%
\begin{pgfscope}%
\pgfpathrectangle{\pgfqpoint{0.800000in}{0.528000in}}{\pgfqpoint{4.960000in}{3.696000in}}%
\pgfusepath{clip}%
\pgfsetbuttcap%
\pgfsetroundjoin%
\definecolor{currentfill}{rgb}{0.000000,0.000000,0.000000}%
\pgfsetfillcolor{currentfill}%
\pgfsetlinewidth{1.003750pt}%
\definecolor{currentstroke}{rgb}{0.000000,0.000000,0.000000}%
\pgfsetstrokecolor{currentstroke}%
\pgfsetdash{}{0pt}%
\pgfpathmoveto{\pgfqpoint{1.025906in}{0.676804in}}%
\pgfpathcurveto{\pgfqpoint{1.036956in}{0.676804in}}{\pgfqpoint{1.047555in}{0.681195in}}{\pgfqpoint{1.055369in}{0.689008in}}%
\pgfpathcurveto{\pgfqpoint{1.063182in}{0.696822in}}{\pgfqpoint{1.067573in}{0.707421in}}{\pgfqpoint{1.067573in}{0.718471in}}%
\pgfpathcurveto{\pgfqpoint{1.067573in}{0.729521in}}{\pgfqpoint{1.063182in}{0.740120in}}{\pgfqpoint{1.055369in}{0.747934in}}%
\pgfpathcurveto{\pgfqpoint{1.047555in}{0.755748in}}{\pgfqpoint{1.036956in}{0.760138in}}{\pgfqpoint{1.025906in}{0.760138in}}%
\pgfpathcurveto{\pgfqpoint{1.014856in}{0.760138in}}{\pgfqpoint{1.004257in}{0.755748in}}{\pgfqpoint{0.996443in}{0.747934in}}%
\pgfpathcurveto{\pgfqpoint{0.988630in}{0.740120in}}{\pgfqpoint{0.984239in}{0.729521in}}{\pgfqpoint{0.984239in}{0.718471in}}%
\pgfpathcurveto{\pgfqpoint{0.984239in}{0.707421in}}{\pgfqpoint{0.988630in}{0.696822in}}{\pgfqpoint{0.996443in}{0.689008in}}%
\pgfpathcurveto{\pgfqpoint{1.004257in}{0.681195in}}{\pgfqpoint{1.014856in}{0.676804in}}{\pgfqpoint{1.025906in}{0.676804in}}%
\pgfpathclose%
\pgfusepath{stroke,fill}%
\end{pgfscope}%
\begin{pgfscope}%
\pgfpathrectangle{\pgfqpoint{0.800000in}{0.528000in}}{\pgfqpoint{4.960000in}{3.696000in}}%
\pgfusepath{clip}%
\pgfsetbuttcap%
\pgfsetroundjoin%
\definecolor{currentfill}{rgb}{0.000000,0.000000,0.000000}%
\pgfsetfillcolor{currentfill}%
\pgfsetlinewidth{1.003750pt}%
\definecolor{currentstroke}{rgb}{0.000000,0.000000,0.000000}%
\pgfsetstrokecolor{currentstroke}%
\pgfsetdash{}{0pt}%
\pgfpathmoveto{\pgfqpoint{1.025906in}{0.719799in}}%
\pgfpathcurveto{\pgfqpoint{1.036956in}{0.719799in}}{\pgfqpoint{1.047555in}{0.724190in}}{\pgfqpoint{1.055369in}{0.732003in}}%
\pgfpathcurveto{\pgfqpoint{1.063182in}{0.739817in}}{\pgfqpoint{1.067573in}{0.750416in}}{\pgfqpoint{1.067573in}{0.761466in}}%
\pgfpathcurveto{\pgfqpoint{1.067573in}{0.772516in}}{\pgfqpoint{1.063182in}{0.783115in}}{\pgfqpoint{1.055369in}{0.790929in}}%
\pgfpathcurveto{\pgfqpoint{1.047555in}{0.798743in}}{\pgfqpoint{1.036956in}{0.803133in}}{\pgfqpoint{1.025906in}{0.803133in}}%
\pgfpathcurveto{\pgfqpoint{1.014856in}{0.803133in}}{\pgfqpoint{1.004257in}{0.798743in}}{\pgfqpoint{0.996443in}{0.790929in}}%
\pgfpathcurveto{\pgfqpoint{0.988630in}{0.783115in}}{\pgfqpoint{0.984239in}{0.772516in}}{\pgfqpoint{0.984239in}{0.761466in}}%
\pgfpathcurveto{\pgfqpoint{0.984239in}{0.750416in}}{\pgfqpoint{0.988630in}{0.739817in}}{\pgfqpoint{0.996443in}{0.732003in}}%
\pgfpathcurveto{\pgfqpoint{1.004257in}{0.724190in}}{\pgfqpoint{1.014856in}{0.719799in}}{\pgfqpoint{1.025906in}{0.719799in}}%
\pgfpathclose%
\pgfusepath{stroke,fill}%
\end{pgfscope}%
\begin{pgfscope}%
\pgfpathrectangle{\pgfqpoint{0.800000in}{0.528000in}}{\pgfqpoint{4.960000in}{3.696000in}}%
\pgfusepath{clip}%
\pgfsetbuttcap%
\pgfsetroundjoin%
\definecolor{currentfill}{rgb}{0.000000,0.000000,0.000000}%
\pgfsetfillcolor{currentfill}%
\pgfsetlinewidth{1.003750pt}%
\definecolor{currentstroke}{rgb}{0.000000,0.000000,0.000000}%
\pgfsetstrokecolor{currentstroke}%
\pgfsetdash{}{0pt}%
\pgfpathmoveto{\pgfqpoint{1.025906in}{0.719799in}}%
\pgfpathcurveto{\pgfqpoint{1.036956in}{0.719799in}}{\pgfqpoint{1.047555in}{0.724190in}}{\pgfqpoint{1.055369in}{0.732003in}}%
\pgfpathcurveto{\pgfqpoint{1.063182in}{0.739817in}}{\pgfqpoint{1.067573in}{0.750416in}}{\pgfqpoint{1.067573in}{0.761466in}}%
\pgfpathcurveto{\pgfqpoint{1.067573in}{0.772516in}}{\pgfqpoint{1.063182in}{0.783115in}}{\pgfqpoint{1.055369in}{0.790929in}}%
\pgfpathcurveto{\pgfqpoint{1.047555in}{0.798743in}}{\pgfqpoint{1.036956in}{0.803133in}}{\pgfqpoint{1.025906in}{0.803133in}}%
\pgfpathcurveto{\pgfqpoint{1.014856in}{0.803133in}}{\pgfqpoint{1.004257in}{0.798743in}}{\pgfqpoint{0.996443in}{0.790929in}}%
\pgfpathcurveto{\pgfqpoint{0.988630in}{0.783115in}}{\pgfqpoint{0.984239in}{0.772516in}}{\pgfqpoint{0.984239in}{0.761466in}}%
\pgfpathcurveto{\pgfqpoint{0.984239in}{0.750416in}}{\pgfqpoint{0.988630in}{0.739817in}}{\pgfqpoint{0.996443in}{0.732003in}}%
\pgfpathcurveto{\pgfqpoint{1.004257in}{0.724190in}}{\pgfqpoint{1.014856in}{0.719799in}}{\pgfqpoint{1.025906in}{0.719799in}}%
\pgfpathclose%
\pgfusepath{stroke,fill}%
\end{pgfscope}%
\begin{pgfscope}%
\pgfpathrectangle{\pgfqpoint{0.800000in}{0.528000in}}{\pgfqpoint{4.960000in}{3.696000in}}%
\pgfusepath{clip}%
\pgfsetbuttcap%
\pgfsetroundjoin%
\definecolor{currentfill}{rgb}{0.000000,0.000000,0.000000}%
\pgfsetfillcolor{currentfill}%
\pgfsetlinewidth{1.003750pt}%
\definecolor{currentstroke}{rgb}{0.000000,0.000000,0.000000}%
\pgfsetstrokecolor{currentstroke}%
\pgfsetdash{}{0pt}%
\pgfpathmoveto{\pgfqpoint{1.025906in}{0.676804in}}%
\pgfpathcurveto{\pgfqpoint{1.036956in}{0.676804in}}{\pgfqpoint{1.047555in}{0.681195in}}{\pgfqpoint{1.055369in}{0.689008in}}%
\pgfpathcurveto{\pgfqpoint{1.063182in}{0.696822in}}{\pgfqpoint{1.067573in}{0.707421in}}{\pgfqpoint{1.067573in}{0.718471in}}%
\pgfpathcurveto{\pgfqpoint{1.067573in}{0.729521in}}{\pgfqpoint{1.063182in}{0.740120in}}{\pgfqpoint{1.055369in}{0.747934in}}%
\pgfpathcurveto{\pgfqpoint{1.047555in}{0.755748in}}{\pgfqpoint{1.036956in}{0.760138in}}{\pgfqpoint{1.025906in}{0.760138in}}%
\pgfpathcurveto{\pgfqpoint{1.014856in}{0.760138in}}{\pgfqpoint{1.004257in}{0.755748in}}{\pgfqpoint{0.996443in}{0.747934in}}%
\pgfpathcurveto{\pgfqpoint{0.988630in}{0.740120in}}{\pgfqpoint{0.984239in}{0.729521in}}{\pgfqpoint{0.984239in}{0.718471in}}%
\pgfpathcurveto{\pgfqpoint{0.984239in}{0.707421in}}{\pgfqpoint{0.988630in}{0.696822in}}{\pgfqpoint{0.996443in}{0.689008in}}%
\pgfpathcurveto{\pgfqpoint{1.004257in}{0.681195in}}{\pgfqpoint{1.014856in}{0.676804in}}{\pgfqpoint{1.025906in}{0.676804in}}%
\pgfpathclose%
\pgfusepath{stroke,fill}%
\end{pgfscope}%
\begin{pgfscope}%
\pgfpathrectangle{\pgfqpoint{0.800000in}{0.528000in}}{\pgfqpoint{4.960000in}{3.696000in}}%
\pgfusepath{clip}%
\pgfsetbuttcap%
\pgfsetroundjoin%
\definecolor{currentfill}{rgb}{0.000000,0.000000,0.000000}%
\pgfsetfillcolor{currentfill}%
\pgfsetlinewidth{1.003750pt}%
\definecolor{currentstroke}{rgb}{0.000000,0.000000,0.000000}%
\pgfsetstrokecolor{currentstroke}%
\pgfsetdash{}{0pt}%
\pgfpathmoveto{\pgfqpoint{1.025906in}{0.719799in}}%
\pgfpathcurveto{\pgfqpoint{1.036956in}{0.719799in}}{\pgfqpoint{1.047555in}{0.724190in}}{\pgfqpoint{1.055369in}{0.732003in}}%
\pgfpathcurveto{\pgfqpoint{1.063182in}{0.739817in}}{\pgfqpoint{1.067573in}{0.750416in}}{\pgfqpoint{1.067573in}{0.761466in}}%
\pgfpathcurveto{\pgfqpoint{1.067573in}{0.772516in}}{\pgfqpoint{1.063182in}{0.783115in}}{\pgfqpoint{1.055369in}{0.790929in}}%
\pgfpathcurveto{\pgfqpoint{1.047555in}{0.798743in}}{\pgfqpoint{1.036956in}{0.803133in}}{\pgfqpoint{1.025906in}{0.803133in}}%
\pgfpathcurveto{\pgfqpoint{1.014856in}{0.803133in}}{\pgfqpoint{1.004257in}{0.798743in}}{\pgfqpoint{0.996443in}{0.790929in}}%
\pgfpathcurveto{\pgfqpoint{0.988630in}{0.783115in}}{\pgfqpoint{0.984239in}{0.772516in}}{\pgfqpoint{0.984239in}{0.761466in}}%
\pgfpathcurveto{\pgfqpoint{0.984239in}{0.750416in}}{\pgfqpoint{0.988630in}{0.739817in}}{\pgfqpoint{0.996443in}{0.732003in}}%
\pgfpathcurveto{\pgfqpoint{1.004257in}{0.724190in}}{\pgfqpoint{1.014856in}{0.719799in}}{\pgfqpoint{1.025906in}{0.719799in}}%
\pgfpathclose%
\pgfusepath{stroke,fill}%
\end{pgfscope}%
\begin{pgfscope}%
\pgfpathrectangle{\pgfqpoint{0.800000in}{0.528000in}}{\pgfqpoint{4.960000in}{3.696000in}}%
\pgfusepath{clip}%
\pgfsetbuttcap%
\pgfsetroundjoin%
\definecolor{currentfill}{rgb}{0.000000,0.000000,0.000000}%
\pgfsetfillcolor{currentfill}%
\pgfsetlinewidth{1.003750pt}%
\definecolor{currentstroke}{rgb}{0.000000,0.000000,0.000000}%
\pgfsetstrokecolor{currentstroke}%
\pgfsetdash{}{0pt}%
\pgfpathmoveto{\pgfqpoint{1.025906in}{0.719799in}}%
\pgfpathcurveto{\pgfqpoint{1.036956in}{0.719799in}}{\pgfqpoint{1.047555in}{0.724190in}}{\pgfqpoint{1.055369in}{0.732003in}}%
\pgfpathcurveto{\pgfqpoint{1.063182in}{0.739817in}}{\pgfqpoint{1.067573in}{0.750416in}}{\pgfqpoint{1.067573in}{0.761466in}}%
\pgfpathcurveto{\pgfqpoint{1.067573in}{0.772516in}}{\pgfqpoint{1.063182in}{0.783115in}}{\pgfqpoint{1.055369in}{0.790929in}}%
\pgfpathcurveto{\pgfqpoint{1.047555in}{0.798743in}}{\pgfqpoint{1.036956in}{0.803133in}}{\pgfqpoint{1.025906in}{0.803133in}}%
\pgfpathcurveto{\pgfqpoint{1.014856in}{0.803133in}}{\pgfqpoint{1.004257in}{0.798743in}}{\pgfqpoint{0.996443in}{0.790929in}}%
\pgfpathcurveto{\pgfqpoint{0.988630in}{0.783115in}}{\pgfqpoint{0.984239in}{0.772516in}}{\pgfqpoint{0.984239in}{0.761466in}}%
\pgfpathcurveto{\pgfqpoint{0.984239in}{0.750416in}}{\pgfqpoint{0.988630in}{0.739817in}}{\pgfqpoint{0.996443in}{0.732003in}}%
\pgfpathcurveto{\pgfqpoint{1.004257in}{0.724190in}}{\pgfqpoint{1.014856in}{0.719799in}}{\pgfqpoint{1.025906in}{0.719799in}}%
\pgfpathclose%
\pgfusepath{stroke,fill}%
\end{pgfscope}%
\begin{pgfscope}%
\pgfpathrectangle{\pgfqpoint{0.800000in}{0.528000in}}{\pgfqpoint{4.960000in}{3.696000in}}%
\pgfusepath{clip}%
\pgfsetbuttcap%
\pgfsetroundjoin%
\definecolor{currentfill}{rgb}{0.000000,0.000000,0.000000}%
\pgfsetfillcolor{currentfill}%
\pgfsetlinewidth{1.003750pt}%
\definecolor{currentstroke}{rgb}{0.000000,0.000000,0.000000}%
\pgfsetstrokecolor{currentstroke}%
\pgfsetdash{}{0pt}%
\pgfpathmoveto{\pgfqpoint{1.025906in}{0.698302in}}%
\pgfpathcurveto{\pgfqpoint{1.036956in}{0.698302in}}{\pgfqpoint{1.047555in}{0.702692in}}{\pgfqpoint{1.055369in}{0.710506in}}%
\pgfpathcurveto{\pgfqpoint{1.063182in}{0.718319in}}{\pgfqpoint{1.067573in}{0.728918in}}{\pgfqpoint{1.067573in}{0.739969in}}%
\pgfpathcurveto{\pgfqpoint{1.067573in}{0.751019in}}{\pgfqpoint{1.063182in}{0.761618in}}{\pgfqpoint{1.055369in}{0.769431in}}%
\pgfpathcurveto{\pgfqpoint{1.047555in}{0.777245in}}{\pgfqpoint{1.036956in}{0.781635in}}{\pgfqpoint{1.025906in}{0.781635in}}%
\pgfpathcurveto{\pgfqpoint{1.014856in}{0.781635in}}{\pgfqpoint{1.004257in}{0.777245in}}{\pgfqpoint{0.996443in}{0.769431in}}%
\pgfpathcurveto{\pgfqpoint{0.988630in}{0.761618in}}{\pgfqpoint{0.984239in}{0.751019in}}{\pgfqpoint{0.984239in}{0.739969in}}%
\pgfpathcurveto{\pgfqpoint{0.984239in}{0.728918in}}{\pgfqpoint{0.988630in}{0.718319in}}{\pgfqpoint{0.996443in}{0.710506in}}%
\pgfpathcurveto{\pgfqpoint{1.004257in}{0.702692in}}{\pgfqpoint{1.014856in}{0.698302in}}{\pgfqpoint{1.025906in}{0.698302in}}%
\pgfpathclose%
\pgfusepath{stroke,fill}%
\end{pgfscope}%
\begin{pgfscope}%
\pgfpathrectangle{\pgfqpoint{0.800000in}{0.528000in}}{\pgfqpoint{4.960000in}{3.696000in}}%
\pgfusepath{clip}%
\pgfsetbuttcap%
\pgfsetroundjoin%
\definecolor{currentfill}{rgb}{0.000000,0.000000,0.000000}%
\pgfsetfillcolor{currentfill}%
\pgfsetlinewidth{1.003750pt}%
\definecolor{currentstroke}{rgb}{0.000000,0.000000,0.000000}%
\pgfsetstrokecolor{currentstroke}%
\pgfsetdash{}{0pt}%
\pgfpathmoveto{\pgfqpoint{1.025906in}{0.676804in}}%
\pgfpathcurveto{\pgfqpoint{1.036956in}{0.676804in}}{\pgfqpoint{1.047555in}{0.681195in}}{\pgfqpoint{1.055369in}{0.689008in}}%
\pgfpathcurveto{\pgfqpoint{1.063182in}{0.696822in}}{\pgfqpoint{1.067573in}{0.707421in}}{\pgfqpoint{1.067573in}{0.718471in}}%
\pgfpathcurveto{\pgfqpoint{1.067573in}{0.729521in}}{\pgfqpoint{1.063182in}{0.740120in}}{\pgfqpoint{1.055369in}{0.747934in}}%
\pgfpathcurveto{\pgfqpoint{1.047555in}{0.755748in}}{\pgfqpoint{1.036956in}{0.760138in}}{\pgfqpoint{1.025906in}{0.760138in}}%
\pgfpathcurveto{\pgfqpoint{1.014856in}{0.760138in}}{\pgfqpoint{1.004257in}{0.755748in}}{\pgfqpoint{0.996443in}{0.747934in}}%
\pgfpathcurveto{\pgfqpoint{0.988630in}{0.740120in}}{\pgfqpoint{0.984239in}{0.729521in}}{\pgfqpoint{0.984239in}{0.718471in}}%
\pgfpathcurveto{\pgfqpoint{0.984239in}{0.707421in}}{\pgfqpoint{0.988630in}{0.696822in}}{\pgfqpoint{0.996443in}{0.689008in}}%
\pgfpathcurveto{\pgfqpoint{1.004257in}{0.681195in}}{\pgfqpoint{1.014856in}{0.676804in}}{\pgfqpoint{1.025906in}{0.676804in}}%
\pgfpathclose%
\pgfusepath{stroke,fill}%
\end{pgfscope}%
\begin{pgfscope}%
\pgfpathrectangle{\pgfqpoint{0.800000in}{0.528000in}}{\pgfqpoint{4.960000in}{3.696000in}}%
\pgfusepath{clip}%
\pgfsetbuttcap%
\pgfsetroundjoin%
\definecolor{currentfill}{rgb}{0.000000,0.000000,0.000000}%
\pgfsetfillcolor{currentfill}%
\pgfsetlinewidth{1.003750pt}%
\definecolor{currentstroke}{rgb}{0.000000,0.000000,0.000000}%
\pgfsetstrokecolor{currentstroke}%
\pgfsetdash{}{0pt}%
\pgfpathmoveto{\pgfqpoint{1.025906in}{0.676804in}}%
\pgfpathcurveto{\pgfqpoint{1.036956in}{0.676804in}}{\pgfqpoint{1.047555in}{0.681195in}}{\pgfqpoint{1.055369in}{0.689008in}}%
\pgfpathcurveto{\pgfqpoint{1.063182in}{0.696822in}}{\pgfqpoint{1.067573in}{0.707421in}}{\pgfqpoint{1.067573in}{0.718471in}}%
\pgfpathcurveto{\pgfqpoint{1.067573in}{0.729521in}}{\pgfqpoint{1.063182in}{0.740120in}}{\pgfqpoint{1.055369in}{0.747934in}}%
\pgfpathcurveto{\pgfqpoint{1.047555in}{0.755748in}}{\pgfqpoint{1.036956in}{0.760138in}}{\pgfqpoint{1.025906in}{0.760138in}}%
\pgfpathcurveto{\pgfqpoint{1.014856in}{0.760138in}}{\pgfqpoint{1.004257in}{0.755748in}}{\pgfqpoint{0.996443in}{0.747934in}}%
\pgfpathcurveto{\pgfqpoint{0.988630in}{0.740120in}}{\pgfqpoint{0.984239in}{0.729521in}}{\pgfqpoint{0.984239in}{0.718471in}}%
\pgfpathcurveto{\pgfqpoint{0.984239in}{0.707421in}}{\pgfqpoint{0.988630in}{0.696822in}}{\pgfqpoint{0.996443in}{0.689008in}}%
\pgfpathcurveto{\pgfqpoint{1.004257in}{0.681195in}}{\pgfqpoint{1.014856in}{0.676804in}}{\pgfqpoint{1.025906in}{0.676804in}}%
\pgfpathclose%
\pgfusepath{stroke,fill}%
\end{pgfscope}%
\begin{pgfscope}%
\pgfpathrectangle{\pgfqpoint{0.800000in}{0.528000in}}{\pgfqpoint{4.960000in}{3.696000in}}%
\pgfusepath{clip}%
\pgfsetbuttcap%
\pgfsetroundjoin%
\definecolor{currentfill}{rgb}{0.000000,0.000000,0.000000}%
\pgfsetfillcolor{currentfill}%
\pgfsetlinewidth{1.003750pt}%
\definecolor{currentstroke}{rgb}{0.000000,0.000000,0.000000}%
\pgfsetstrokecolor{currentstroke}%
\pgfsetdash{}{0pt}%
\pgfpathmoveto{\pgfqpoint{1.025906in}{0.676804in}}%
\pgfpathcurveto{\pgfqpoint{1.036956in}{0.676804in}}{\pgfqpoint{1.047555in}{0.681195in}}{\pgfqpoint{1.055369in}{0.689008in}}%
\pgfpathcurveto{\pgfqpoint{1.063182in}{0.696822in}}{\pgfqpoint{1.067573in}{0.707421in}}{\pgfqpoint{1.067573in}{0.718471in}}%
\pgfpathcurveto{\pgfqpoint{1.067573in}{0.729521in}}{\pgfqpoint{1.063182in}{0.740120in}}{\pgfqpoint{1.055369in}{0.747934in}}%
\pgfpathcurveto{\pgfqpoint{1.047555in}{0.755748in}}{\pgfqpoint{1.036956in}{0.760138in}}{\pgfqpoint{1.025906in}{0.760138in}}%
\pgfpathcurveto{\pgfqpoint{1.014856in}{0.760138in}}{\pgfqpoint{1.004257in}{0.755748in}}{\pgfqpoint{0.996443in}{0.747934in}}%
\pgfpathcurveto{\pgfqpoint{0.988630in}{0.740120in}}{\pgfqpoint{0.984239in}{0.729521in}}{\pgfqpoint{0.984239in}{0.718471in}}%
\pgfpathcurveto{\pgfqpoint{0.984239in}{0.707421in}}{\pgfqpoint{0.988630in}{0.696822in}}{\pgfqpoint{0.996443in}{0.689008in}}%
\pgfpathcurveto{\pgfqpoint{1.004257in}{0.681195in}}{\pgfqpoint{1.014856in}{0.676804in}}{\pgfqpoint{1.025906in}{0.676804in}}%
\pgfpathclose%
\pgfusepath{stroke,fill}%
\end{pgfscope}%
\begin{pgfscope}%
\pgfpathrectangle{\pgfqpoint{0.800000in}{0.528000in}}{\pgfqpoint{4.960000in}{3.696000in}}%
\pgfusepath{clip}%
\pgfsetbuttcap%
\pgfsetroundjoin%
\definecolor{currentfill}{rgb}{0.000000,0.000000,0.000000}%
\pgfsetfillcolor{currentfill}%
\pgfsetlinewidth{1.003750pt}%
\definecolor{currentstroke}{rgb}{0.000000,0.000000,0.000000}%
\pgfsetstrokecolor{currentstroke}%
\pgfsetdash{}{0pt}%
\pgfpathmoveto{\pgfqpoint{1.025906in}{0.698302in}}%
\pgfpathcurveto{\pgfqpoint{1.036956in}{0.698302in}}{\pgfqpoint{1.047555in}{0.702692in}}{\pgfqpoint{1.055369in}{0.710506in}}%
\pgfpathcurveto{\pgfqpoint{1.063182in}{0.718319in}}{\pgfqpoint{1.067573in}{0.728918in}}{\pgfqpoint{1.067573in}{0.739969in}}%
\pgfpathcurveto{\pgfqpoint{1.067573in}{0.751019in}}{\pgfqpoint{1.063182in}{0.761618in}}{\pgfqpoint{1.055369in}{0.769431in}}%
\pgfpathcurveto{\pgfqpoint{1.047555in}{0.777245in}}{\pgfqpoint{1.036956in}{0.781635in}}{\pgfqpoint{1.025906in}{0.781635in}}%
\pgfpathcurveto{\pgfqpoint{1.014856in}{0.781635in}}{\pgfqpoint{1.004257in}{0.777245in}}{\pgfqpoint{0.996443in}{0.769431in}}%
\pgfpathcurveto{\pgfqpoint{0.988630in}{0.761618in}}{\pgfqpoint{0.984239in}{0.751019in}}{\pgfqpoint{0.984239in}{0.739969in}}%
\pgfpathcurveto{\pgfqpoint{0.984239in}{0.728918in}}{\pgfqpoint{0.988630in}{0.718319in}}{\pgfqpoint{0.996443in}{0.710506in}}%
\pgfpathcurveto{\pgfqpoint{1.004257in}{0.702692in}}{\pgfqpoint{1.014856in}{0.698302in}}{\pgfqpoint{1.025906in}{0.698302in}}%
\pgfpathclose%
\pgfusepath{stroke,fill}%
\end{pgfscope}%
\begin{pgfscope}%
\pgfpathrectangle{\pgfqpoint{0.800000in}{0.528000in}}{\pgfqpoint{4.960000in}{3.696000in}}%
\pgfusepath{clip}%
\pgfsetbuttcap%
\pgfsetroundjoin%
\definecolor{currentfill}{rgb}{0.000000,0.000000,0.000000}%
\pgfsetfillcolor{currentfill}%
\pgfsetlinewidth{1.003750pt}%
\definecolor{currentstroke}{rgb}{0.000000,0.000000,0.000000}%
\pgfsetstrokecolor{currentstroke}%
\pgfsetdash{}{0pt}%
\pgfpathmoveto{\pgfqpoint{1.025906in}{0.698302in}}%
\pgfpathcurveto{\pgfqpoint{1.036956in}{0.698302in}}{\pgfqpoint{1.047555in}{0.702692in}}{\pgfqpoint{1.055369in}{0.710506in}}%
\pgfpathcurveto{\pgfqpoint{1.063182in}{0.718319in}}{\pgfqpoint{1.067573in}{0.728918in}}{\pgfqpoint{1.067573in}{0.739969in}}%
\pgfpathcurveto{\pgfqpoint{1.067573in}{0.751019in}}{\pgfqpoint{1.063182in}{0.761618in}}{\pgfqpoint{1.055369in}{0.769431in}}%
\pgfpathcurveto{\pgfqpoint{1.047555in}{0.777245in}}{\pgfqpoint{1.036956in}{0.781635in}}{\pgfqpoint{1.025906in}{0.781635in}}%
\pgfpathcurveto{\pgfqpoint{1.014856in}{0.781635in}}{\pgfqpoint{1.004257in}{0.777245in}}{\pgfqpoint{0.996443in}{0.769431in}}%
\pgfpathcurveto{\pgfqpoint{0.988630in}{0.761618in}}{\pgfqpoint{0.984239in}{0.751019in}}{\pgfqpoint{0.984239in}{0.739969in}}%
\pgfpathcurveto{\pgfqpoint{0.984239in}{0.728918in}}{\pgfqpoint{0.988630in}{0.718319in}}{\pgfqpoint{0.996443in}{0.710506in}}%
\pgfpathcurveto{\pgfqpoint{1.004257in}{0.702692in}}{\pgfqpoint{1.014856in}{0.698302in}}{\pgfqpoint{1.025906in}{0.698302in}}%
\pgfpathclose%
\pgfusepath{stroke,fill}%
\end{pgfscope}%
\begin{pgfscope}%
\pgfpathrectangle{\pgfqpoint{0.800000in}{0.528000in}}{\pgfqpoint{4.960000in}{3.696000in}}%
\pgfusepath{clip}%
\pgfsetbuttcap%
\pgfsetroundjoin%
\definecolor{currentfill}{rgb}{0.000000,0.000000,0.000000}%
\pgfsetfillcolor{currentfill}%
\pgfsetlinewidth{1.003750pt}%
\definecolor{currentstroke}{rgb}{0.000000,0.000000,0.000000}%
\pgfsetstrokecolor{currentstroke}%
\pgfsetdash{}{0pt}%
\pgfpathmoveto{\pgfqpoint{1.025906in}{0.719799in}}%
\pgfpathcurveto{\pgfqpoint{1.036956in}{0.719799in}}{\pgfqpoint{1.047555in}{0.724190in}}{\pgfqpoint{1.055369in}{0.732003in}}%
\pgfpathcurveto{\pgfqpoint{1.063182in}{0.739817in}}{\pgfqpoint{1.067573in}{0.750416in}}{\pgfqpoint{1.067573in}{0.761466in}}%
\pgfpathcurveto{\pgfqpoint{1.067573in}{0.772516in}}{\pgfqpoint{1.063182in}{0.783115in}}{\pgfqpoint{1.055369in}{0.790929in}}%
\pgfpathcurveto{\pgfqpoint{1.047555in}{0.798743in}}{\pgfqpoint{1.036956in}{0.803133in}}{\pgfqpoint{1.025906in}{0.803133in}}%
\pgfpathcurveto{\pgfqpoint{1.014856in}{0.803133in}}{\pgfqpoint{1.004257in}{0.798743in}}{\pgfqpoint{0.996443in}{0.790929in}}%
\pgfpathcurveto{\pgfqpoint{0.988630in}{0.783115in}}{\pgfqpoint{0.984239in}{0.772516in}}{\pgfqpoint{0.984239in}{0.761466in}}%
\pgfpathcurveto{\pgfqpoint{0.984239in}{0.750416in}}{\pgfqpoint{0.988630in}{0.739817in}}{\pgfqpoint{0.996443in}{0.732003in}}%
\pgfpathcurveto{\pgfqpoint{1.004257in}{0.724190in}}{\pgfqpoint{1.014856in}{0.719799in}}{\pgfqpoint{1.025906in}{0.719799in}}%
\pgfpathclose%
\pgfusepath{stroke,fill}%
\end{pgfscope}%
\begin{pgfscope}%
\pgfpathrectangle{\pgfqpoint{0.800000in}{0.528000in}}{\pgfqpoint{4.960000in}{3.696000in}}%
\pgfusepath{clip}%
\pgfsetbuttcap%
\pgfsetroundjoin%
\definecolor{currentfill}{rgb}{0.000000,0.000000,0.000000}%
\pgfsetfillcolor{currentfill}%
\pgfsetlinewidth{1.003750pt}%
\definecolor{currentstroke}{rgb}{0.000000,0.000000,0.000000}%
\pgfsetstrokecolor{currentstroke}%
\pgfsetdash{}{0pt}%
\pgfpathmoveto{\pgfqpoint{1.025906in}{0.698302in}}%
\pgfpathcurveto{\pgfqpoint{1.036956in}{0.698302in}}{\pgfqpoint{1.047555in}{0.702692in}}{\pgfqpoint{1.055369in}{0.710506in}}%
\pgfpathcurveto{\pgfqpoint{1.063182in}{0.718319in}}{\pgfqpoint{1.067573in}{0.728918in}}{\pgfqpoint{1.067573in}{0.739969in}}%
\pgfpathcurveto{\pgfqpoint{1.067573in}{0.751019in}}{\pgfqpoint{1.063182in}{0.761618in}}{\pgfqpoint{1.055369in}{0.769431in}}%
\pgfpathcurveto{\pgfqpoint{1.047555in}{0.777245in}}{\pgfqpoint{1.036956in}{0.781635in}}{\pgfqpoint{1.025906in}{0.781635in}}%
\pgfpathcurveto{\pgfqpoint{1.014856in}{0.781635in}}{\pgfqpoint{1.004257in}{0.777245in}}{\pgfqpoint{0.996443in}{0.769431in}}%
\pgfpathcurveto{\pgfqpoint{0.988630in}{0.761618in}}{\pgfqpoint{0.984239in}{0.751019in}}{\pgfqpoint{0.984239in}{0.739969in}}%
\pgfpathcurveto{\pgfqpoint{0.984239in}{0.728918in}}{\pgfqpoint{0.988630in}{0.718319in}}{\pgfqpoint{0.996443in}{0.710506in}}%
\pgfpathcurveto{\pgfqpoint{1.004257in}{0.702692in}}{\pgfqpoint{1.014856in}{0.698302in}}{\pgfqpoint{1.025906in}{0.698302in}}%
\pgfpathclose%
\pgfusepath{stroke,fill}%
\end{pgfscope}%
\begin{pgfscope}%
\pgfpathrectangle{\pgfqpoint{0.800000in}{0.528000in}}{\pgfqpoint{4.960000in}{3.696000in}}%
\pgfusepath{clip}%
\pgfsetbuttcap%
\pgfsetroundjoin%
\definecolor{currentfill}{rgb}{0.000000,0.000000,0.000000}%
\pgfsetfillcolor{currentfill}%
\pgfsetlinewidth{1.003750pt}%
\definecolor{currentstroke}{rgb}{0.000000,0.000000,0.000000}%
\pgfsetstrokecolor{currentstroke}%
\pgfsetdash{}{0pt}%
\pgfpathmoveto{\pgfqpoint{1.025906in}{0.719799in}}%
\pgfpathcurveto{\pgfqpoint{1.036956in}{0.719799in}}{\pgfqpoint{1.047555in}{0.724190in}}{\pgfqpoint{1.055369in}{0.732003in}}%
\pgfpathcurveto{\pgfqpoint{1.063182in}{0.739817in}}{\pgfqpoint{1.067573in}{0.750416in}}{\pgfqpoint{1.067573in}{0.761466in}}%
\pgfpathcurveto{\pgfqpoint{1.067573in}{0.772516in}}{\pgfqpoint{1.063182in}{0.783115in}}{\pgfqpoint{1.055369in}{0.790929in}}%
\pgfpathcurveto{\pgfqpoint{1.047555in}{0.798743in}}{\pgfqpoint{1.036956in}{0.803133in}}{\pgfqpoint{1.025906in}{0.803133in}}%
\pgfpathcurveto{\pgfqpoint{1.014856in}{0.803133in}}{\pgfqpoint{1.004257in}{0.798743in}}{\pgfqpoint{0.996443in}{0.790929in}}%
\pgfpathcurveto{\pgfqpoint{0.988630in}{0.783115in}}{\pgfqpoint{0.984239in}{0.772516in}}{\pgfqpoint{0.984239in}{0.761466in}}%
\pgfpathcurveto{\pgfqpoint{0.984239in}{0.750416in}}{\pgfqpoint{0.988630in}{0.739817in}}{\pgfqpoint{0.996443in}{0.732003in}}%
\pgfpathcurveto{\pgfqpoint{1.004257in}{0.724190in}}{\pgfqpoint{1.014856in}{0.719799in}}{\pgfqpoint{1.025906in}{0.719799in}}%
\pgfpathclose%
\pgfusepath{stroke,fill}%
\end{pgfscope}%
\begin{pgfscope}%
\pgfpathrectangle{\pgfqpoint{0.800000in}{0.528000in}}{\pgfqpoint{4.960000in}{3.696000in}}%
\pgfusepath{clip}%
\pgfsetbuttcap%
\pgfsetroundjoin%
\definecolor{currentfill}{rgb}{0.000000,0.000000,0.000000}%
\pgfsetfillcolor{currentfill}%
\pgfsetlinewidth{1.003750pt}%
\definecolor{currentstroke}{rgb}{0.000000,0.000000,0.000000}%
\pgfsetstrokecolor{currentstroke}%
\pgfsetdash{}{0pt}%
\pgfpathmoveto{\pgfqpoint{1.025906in}{0.676804in}}%
\pgfpathcurveto{\pgfqpoint{1.036956in}{0.676804in}}{\pgfqpoint{1.047555in}{0.681195in}}{\pgfqpoint{1.055369in}{0.689008in}}%
\pgfpathcurveto{\pgfqpoint{1.063182in}{0.696822in}}{\pgfqpoint{1.067573in}{0.707421in}}{\pgfqpoint{1.067573in}{0.718471in}}%
\pgfpathcurveto{\pgfqpoint{1.067573in}{0.729521in}}{\pgfqpoint{1.063182in}{0.740120in}}{\pgfqpoint{1.055369in}{0.747934in}}%
\pgfpathcurveto{\pgfqpoint{1.047555in}{0.755748in}}{\pgfqpoint{1.036956in}{0.760138in}}{\pgfqpoint{1.025906in}{0.760138in}}%
\pgfpathcurveto{\pgfqpoint{1.014856in}{0.760138in}}{\pgfqpoint{1.004257in}{0.755748in}}{\pgfqpoint{0.996443in}{0.747934in}}%
\pgfpathcurveto{\pgfqpoint{0.988630in}{0.740120in}}{\pgfqpoint{0.984239in}{0.729521in}}{\pgfqpoint{0.984239in}{0.718471in}}%
\pgfpathcurveto{\pgfqpoint{0.984239in}{0.707421in}}{\pgfqpoint{0.988630in}{0.696822in}}{\pgfqpoint{0.996443in}{0.689008in}}%
\pgfpathcurveto{\pgfqpoint{1.004257in}{0.681195in}}{\pgfqpoint{1.014856in}{0.676804in}}{\pgfqpoint{1.025906in}{0.676804in}}%
\pgfpathclose%
\pgfusepath{stroke,fill}%
\end{pgfscope}%
\begin{pgfscope}%
\pgfpathrectangle{\pgfqpoint{0.800000in}{0.528000in}}{\pgfqpoint{4.960000in}{3.696000in}}%
\pgfusepath{clip}%
\pgfsetbuttcap%
\pgfsetroundjoin%
\definecolor{currentfill}{rgb}{0.000000,0.000000,0.000000}%
\pgfsetfillcolor{currentfill}%
\pgfsetlinewidth{1.003750pt}%
\definecolor{currentstroke}{rgb}{0.000000,0.000000,0.000000}%
\pgfsetstrokecolor{currentstroke}%
\pgfsetdash{}{0pt}%
\pgfpathmoveto{\pgfqpoint{1.025906in}{0.719799in}}%
\pgfpathcurveto{\pgfqpoint{1.036956in}{0.719799in}}{\pgfqpoint{1.047555in}{0.724190in}}{\pgfqpoint{1.055369in}{0.732003in}}%
\pgfpathcurveto{\pgfqpoint{1.063182in}{0.739817in}}{\pgfqpoint{1.067573in}{0.750416in}}{\pgfqpoint{1.067573in}{0.761466in}}%
\pgfpathcurveto{\pgfqpoint{1.067573in}{0.772516in}}{\pgfqpoint{1.063182in}{0.783115in}}{\pgfqpoint{1.055369in}{0.790929in}}%
\pgfpathcurveto{\pgfqpoint{1.047555in}{0.798743in}}{\pgfqpoint{1.036956in}{0.803133in}}{\pgfqpoint{1.025906in}{0.803133in}}%
\pgfpathcurveto{\pgfqpoint{1.014856in}{0.803133in}}{\pgfqpoint{1.004257in}{0.798743in}}{\pgfqpoint{0.996443in}{0.790929in}}%
\pgfpathcurveto{\pgfqpoint{0.988630in}{0.783115in}}{\pgfqpoint{0.984239in}{0.772516in}}{\pgfqpoint{0.984239in}{0.761466in}}%
\pgfpathcurveto{\pgfqpoint{0.984239in}{0.750416in}}{\pgfqpoint{0.988630in}{0.739817in}}{\pgfqpoint{0.996443in}{0.732003in}}%
\pgfpathcurveto{\pgfqpoint{1.004257in}{0.724190in}}{\pgfqpoint{1.014856in}{0.719799in}}{\pgfqpoint{1.025906in}{0.719799in}}%
\pgfpathclose%
\pgfusepath{stroke,fill}%
\end{pgfscope}%
\begin{pgfscope}%
\pgfpathrectangle{\pgfqpoint{0.800000in}{0.528000in}}{\pgfqpoint{4.960000in}{3.696000in}}%
\pgfusepath{clip}%
\pgfsetbuttcap%
\pgfsetroundjoin%
\definecolor{currentfill}{rgb}{0.000000,0.000000,0.000000}%
\pgfsetfillcolor{currentfill}%
\pgfsetlinewidth{1.003750pt}%
\definecolor{currentstroke}{rgb}{0.000000,0.000000,0.000000}%
\pgfsetstrokecolor{currentstroke}%
\pgfsetdash{}{0pt}%
\pgfpathmoveto{\pgfqpoint{1.025906in}{0.719799in}}%
\pgfpathcurveto{\pgfqpoint{1.036956in}{0.719799in}}{\pgfqpoint{1.047555in}{0.724190in}}{\pgfqpoint{1.055369in}{0.732003in}}%
\pgfpathcurveto{\pgfqpoint{1.063182in}{0.739817in}}{\pgfqpoint{1.067573in}{0.750416in}}{\pgfqpoint{1.067573in}{0.761466in}}%
\pgfpathcurveto{\pgfqpoint{1.067573in}{0.772516in}}{\pgfqpoint{1.063182in}{0.783115in}}{\pgfqpoint{1.055369in}{0.790929in}}%
\pgfpathcurveto{\pgfqpoint{1.047555in}{0.798743in}}{\pgfqpoint{1.036956in}{0.803133in}}{\pgfqpoint{1.025906in}{0.803133in}}%
\pgfpathcurveto{\pgfqpoint{1.014856in}{0.803133in}}{\pgfqpoint{1.004257in}{0.798743in}}{\pgfqpoint{0.996443in}{0.790929in}}%
\pgfpathcurveto{\pgfqpoint{0.988630in}{0.783115in}}{\pgfqpoint{0.984239in}{0.772516in}}{\pgfqpoint{0.984239in}{0.761466in}}%
\pgfpathcurveto{\pgfqpoint{0.984239in}{0.750416in}}{\pgfqpoint{0.988630in}{0.739817in}}{\pgfqpoint{0.996443in}{0.732003in}}%
\pgfpathcurveto{\pgfqpoint{1.004257in}{0.724190in}}{\pgfqpoint{1.014856in}{0.719799in}}{\pgfqpoint{1.025906in}{0.719799in}}%
\pgfpathclose%
\pgfusepath{stroke,fill}%
\end{pgfscope}%
\begin{pgfscope}%
\pgfpathrectangle{\pgfqpoint{0.800000in}{0.528000in}}{\pgfqpoint{4.960000in}{3.696000in}}%
\pgfusepath{clip}%
\pgfsetbuttcap%
\pgfsetroundjoin%
\definecolor{currentfill}{rgb}{0.000000,0.000000,0.000000}%
\pgfsetfillcolor{currentfill}%
\pgfsetlinewidth{1.003750pt}%
\definecolor{currentstroke}{rgb}{0.000000,0.000000,0.000000}%
\pgfsetstrokecolor{currentstroke}%
\pgfsetdash{}{0pt}%
\pgfpathmoveto{\pgfqpoint{1.025906in}{0.719799in}}%
\pgfpathcurveto{\pgfqpoint{1.036956in}{0.719799in}}{\pgfqpoint{1.047555in}{0.724190in}}{\pgfqpoint{1.055369in}{0.732003in}}%
\pgfpathcurveto{\pgfqpoint{1.063182in}{0.739817in}}{\pgfqpoint{1.067573in}{0.750416in}}{\pgfqpoint{1.067573in}{0.761466in}}%
\pgfpathcurveto{\pgfqpoint{1.067573in}{0.772516in}}{\pgfqpoint{1.063182in}{0.783115in}}{\pgfqpoint{1.055369in}{0.790929in}}%
\pgfpathcurveto{\pgfqpoint{1.047555in}{0.798743in}}{\pgfqpoint{1.036956in}{0.803133in}}{\pgfqpoint{1.025906in}{0.803133in}}%
\pgfpathcurveto{\pgfqpoint{1.014856in}{0.803133in}}{\pgfqpoint{1.004257in}{0.798743in}}{\pgfqpoint{0.996443in}{0.790929in}}%
\pgfpathcurveto{\pgfqpoint{0.988630in}{0.783115in}}{\pgfqpoint{0.984239in}{0.772516in}}{\pgfqpoint{0.984239in}{0.761466in}}%
\pgfpathcurveto{\pgfqpoint{0.984239in}{0.750416in}}{\pgfqpoint{0.988630in}{0.739817in}}{\pgfqpoint{0.996443in}{0.732003in}}%
\pgfpathcurveto{\pgfqpoint{1.004257in}{0.724190in}}{\pgfqpoint{1.014856in}{0.719799in}}{\pgfqpoint{1.025906in}{0.719799in}}%
\pgfpathclose%
\pgfusepath{stroke,fill}%
\end{pgfscope}%
\begin{pgfscope}%
\pgfpathrectangle{\pgfqpoint{0.800000in}{0.528000in}}{\pgfqpoint{4.960000in}{3.696000in}}%
\pgfusepath{clip}%
\pgfsetbuttcap%
\pgfsetroundjoin%
\definecolor{currentfill}{rgb}{0.000000,0.000000,0.000000}%
\pgfsetfillcolor{currentfill}%
\pgfsetlinewidth{1.003750pt}%
\definecolor{currentstroke}{rgb}{0.000000,0.000000,0.000000}%
\pgfsetstrokecolor{currentstroke}%
\pgfsetdash{}{0pt}%
\pgfpathmoveto{\pgfqpoint{1.025906in}{0.698302in}}%
\pgfpathcurveto{\pgfqpoint{1.036956in}{0.698302in}}{\pgfqpoint{1.047555in}{0.702692in}}{\pgfqpoint{1.055369in}{0.710506in}}%
\pgfpathcurveto{\pgfqpoint{1.063182in}{0.718319in}}{\pgfqpoint{1.067573in}{0.728918in}}{\pgfqpoint{1.067573in}{0.739969in}}%
\pgfpathcurveto{\pgfqpoint{1.067573in}{0.751019in}}{\pgfqpoint{1.063182in}{0.761618in}}{\pgfqpoint{1.055369in}{0.769431in}}%
\pgfpathcurveto{\pgfqpoint{1.047555in}{0.777245in}}{\pgfqpoint{1.036956in}{0.781635in}}{\pgfqpoint{1.025906in}{0.781635in}}%
\pgfpathcurveto{\pgfqpoint{1.014856in}{0.781635in}}{\pgfqpoint{1.004257in}{0.777245in}}{\pgfqpoint{0.996443in}{0.769431in}}%
\pgfpathcurveto{\pgfqpoint{0.988630in}{0.761618in}}{\pgfqpoint{0.984239in}{0.751019in}}{\pgfqpoint{0.984239in}{0.739969in}}%
\pgfpathcurveto{\pgfqpoint{0.984239in}{0.728918in}}{\pgfqpoint{0.988630in}{0.718319in}}{\pgfqpoint{0.996443in}{0.710506in}}%
\pgfpathcurveto{\pgfqpoint{1.004257in}{0.702692in}}{\pgfqpoint{1.014856in}{0.698302in}}{\pgfqpoint{1.025906in}{0.698302in}}%
\pgfpathclose%
\pgfusepath{stroke,fill}%
\end{pgfscope}%
\begin{pgfscope}%
\pgfpathrectangle{\pgfqpoint{0.800000in}{0.528000in}}{\pgfqpoint{4.960000in}{3.696000in}}%
\pgfusepath{clip}%
\pgfsetbuttcap%
\pgfsetroundjoin%
\definecolor{currentfill}{rgb}{0.000000,0.000000,0.000000}%
\pgfsetfillcolor{currentfill}%
\pgfsetlinewidth{1.003750pt}%
\definecolor{currentstroke}{rgb}{0.000000,0.000000,0.000000}%
\pgfsetstrokecolor{currentstroke}%
\pgfsetdash{}{0pt}%
\pgfpathmoveto{\pgfqpoint{1.025906in}{0.719799in}}%
\pgfpathcurveto{\pgfqpoint{1.036956in}{0.719799in}}{\pgfqpoint{1.047555in}{0.724190in}}{\pgfqpoint{1.055369in}{0.732003in}}%
\pgfpathcurveto{\pgfqpoint{1.063182in}{0.739817in}}{\pgfqpoint{1.067573in}{0.750416in}}{\pgfqpoint{1.067573in}{0.761466in}}%
\pgfpathcurveto{\pgfqpoint{1.067573in}{0.772516in}}{\pgfqpoint{1.063182in}{0.783115in}}{\pgfqpoint{1.055369in}{0.790929in}}%
\pgfpathcurveto{\pgfqpoint{1.047555in}{0.798743in}}{\pgfqpoint{1.036956in}{0.803133in}}{\pgfqpoint{1.025906in}{0.803133in}}%
\pgfpathcurveto{\pgfqpoint{1.014856in}{0.803133in}}{\pgfqpoint{1.004257in}{0.798743in}}{\pgfqpoint{0.996443in}{0.790929in}}%
\pgfpathcurveto{\pgfqpoint{0.988630in}{0.783115in}}{\pgfqpoint{0.984239in}{0.772516in}}{\pgfqpoint{0.984239in}{0.761466in}}%
\pgfpathcurveto{\pgfqpoint{0.984239in}{0.750416in}}{\pgfqpoint{0.988630in}{0.739817in}}{\pgfqpoint{0.996443in}{0.732003in}}%
\pgfpathcurveto{\pgfqpoint{1.004257in}{0.724190in}}{\pgfqpoint{1.014856in}{0.719799in}}{\pgfqpoint{1.025906in}{0.719799in}}%
\pgfpathclose%
\pgfusepath{stroke,fill}%
\end{pgfscope}%
\begin{pgfscope}%
\pgfpathrectangle{\pgfqpoint{0.800000in}{0.528000in}}{\pgfqpoint{4.960000in}{3.696000in}}%
\pgfusepath{clip}%
\pgfsetbuttcap%
\pgfsetroundjoin%
\definecolor{currentfill}{rgb}{0.000000,0.000000,0.000000}%
\pgfsetfillcolor{currentfill}%
\pgfsetlinewidth{1.003750pt}%
\definecolor{currentstroke}{rgb}{0.000000,0.000000,0.000000}%
\pgfsetstrokecolor{currentstroke}%
\pgfsetdash{}{0pt}%
\pgfpathmoveto{\pgfqpoint{1.025906in}{0.676804in}}%
\pgfpathcurveto{\pgfqpoint{1.036956in}{0.676804in}}{\pgfqpoint{1.047555in}{0.681195in}}{\pgfqpoint{1.055369in}{0.689008in}}%
\pgfpathcurveto{\pgfqpoint{1.063182in}{0.696822in}}{\pgfqpoint{1.067573in}{0.707421in}}{\pgfqpoint{1.067573in}{0.718471in}}%
\pgfpathcurveto{\pgfqpoint{1.067573in}{0.729521in}}{\pgfqpoint{1.063182in}{0.740120in}}{\pgfqpoint{1.055369in}{0.747934in}}%
\pgfpathcurveto{\pgfqpoint{1.047555in}{0.755748in}}{\pgfqpoint{1.036956in}{0.760138in}}{\pgfqpoint{1.025906in}{0.760138in}}%
\pgfpathcurveto{\pgfqpoint{1.014856in}{0.760138in}}{\pgfqpoint{1.004257in}{0.755748in}}{\pgfqpoint{0.996443in}{0.747934in}}%
\pgfpathcurveto{\pgfqpoint{0.988630in}{0.740120in}}{\pgfqpoint{0.984239in}{0.729521in}}{\pgfqpoint{0.984239in}{0.718471in}}%
\pgfpathcurveto{\pgfqpoint{0.984239in}{0.707421in}}{\pgfqpoint{0.988630in}{0.696822in}}{\pgfqpoint{0.996443in}{0.689008in}}%
\pgfpathcurveto{\pgfqpoint{1.004257in}{0.681195in}}{\pgfqpoint{1.014856in}{0.676804in}}{\pgfqpoint{1.025906in}{0.676804in}}%
\pgfpathclose%
\pgfusepath{stroke,fill}%
\end{pgfscope}%
\begin{pgfscope}%
\pgfpathrectangle{\pgfqpoint{0.800000in}{0.528000in}}{\pgfqpoint{4.960000in}{3.696000in}}%
\pgfusepath{clip}%
\pgfsetbuttcap%
\pgfsetroundjoin%
\definecolor{currentfill}{rgb}{0.000000,0.000000,0.000000}%
\pgfsetfillcolor{currentfill}%
\pgfsetlinewidth{1.003750pt}%
\definecolor{currentstroke}{rgb}{0.000000,0.000000,0.000000}%
\pgfsetstrokecolor{currentstroke}%
\pgfsetdash{}{0pt}%
\pgfpathmoveto{\pgfqpoint{1.025906in}{0.676804in}}%
\pgfpathcurveto{\pgfqpoint{1.036956in}{0.676804in}}{\pgfqpoint{1.047555in}{0.681195in}}{\pgfqpoint{1.055369in}{0.689008in}}%
\pgfpathcurveto{\pgfqpoint{1.063182in}{0.696822in}}{\pgfqpoint{1.067573in}{0.707421in}}{\pgfqpoint{1.067573in}{0.718471in}}%
\pgfpathcurveto{\pgfqpoint{1.067573in}{0.729521in}}{\pgfqpoint{1.063182in}{0.740120in}}{\pgfqpoint{1.055369in}{0.747934in}}%
\pgfpathcurveto{\pgfqpoint{1.047555in}{0.755748in}}{\pgfqpoint{1.036956in}{0.760138in}}{\pgfqpoint{1.025906in}{0.760138in}}%
\pgfpathcurveto{\pgfqpoint{1.014856in}{0.760138in}}{\pgfqpoint{1.004257in}{0.755748in}}{\pgfqpoint{0.996443in}{0.747934in}}%
\pgfpathcurveto{\pgfqpoint{0.988630in}{0.740120in}}{\pgfqpoint{0.984239in}{0.729521in}}{\pgfqpoint{0.984239in}{0.718471in}}%
\pgfpathcurveto{\pgfqpoint{0.984239in}{0.707421in}}{\pgfqpoint{0.988630in}{0.696822in}}{\pgfqpoint{0.996443in}{0.689008in}}%
\pgfpathcurveto{\pgfqpoint{1.004257in}{0.681195in}}{\pgfqpoint{1.014856in}{0.676804in}}{\pgfqpoint{1.025906in}{0.676804in}}%
\pgfpathclose%
\pgfusepath{stroke,fill}%
\end{pgfscope}%
\begin{pgfscope}%
\pgfpathrectangle{\pgfqpoint{0.800000in}{0.528000in}}{\pgfqpoint{4.960000in}{3.696000in}}%
\pgfusepath{clip}%
\pgfsetbuttcap%
\pgfsetroundjoin%
\definecolor{currentfill}{rgb}{0.000000,0.000000,0.000000}%
\pgfsetfillcolor{currentfill}%
\pgfsetlinewidth{1.003750pt}%
\definecolor{currentstroke}{rgb}{0.000000,0.000000,0.000000}%
\pgfsetstrokecolor{currentstroke}%
\pgfsetdash{}{0pt}%
\pgfpathmoveto{\pgfqpoint{1.025906in}{0.676804in}}%
\pgfpathcurveto{\pgfqpoint{1.036956in}{0.676804in}}{\pgfqpoint{1.047555in}{0.681195in}}{\pgfqpoint{1.055369in}{0.689008in}}%
\pgfpathcurveto{\pgfqpoint{1.063182in}{0.696822in}}{\pgfqpoint{1.067573in}{0.707421in}}{\pgfqpoint{1.067573in}{0.718471in}}%
\pgfpathcurveto{\pgfqpoint{1.067573in}{0.729521in}}{\pgfqpoint{1.063182in}{0.740120in}}{\pgfqpoint{1.055369in}{0.747934in}}%
\pgfpathcurveto{\pgfqpoint{1.047555in}{0.755748in}}{\pgfqpoint{1.036956in}{0.760138in}}{\pgfqpoint{1.025906in}{0.760138in}}%
\pgfpathcurveto{\pgfqpoint{1.014856in}{0.760138in}}{\pgfqpoint{1.004257in}{0.755748in}}{\pgfqpoint{0.996443in}{0.747934in}}%
\pgfpathcurveto{\pgfqpoint{0.988630in}{0.740120in}}{\pgfqpoint{0.984239in}{0.729521in}}{\pgfqpoint{0.984239in}{0.718471in}}%
\pgfpathcurveto{\pgfqpoint{0.984239in}{0.707421in}}{\pgfqpoint{0.988630in}{0.696822in}}{\pgfqpoint{0.996443in}{0.689008in}}%
\pgfpathcurveto{\pgfqpoint{1.004257in}{0.681195in}}{\pgfqpoint{1.014856in}{0.676804in}}{\pgfqpoint{1.025906in}{0.676804in}}%
\pgfpathclose%
\pgfusepath{stroke,fill}%
\end{pgfscope}%
\begin{pgfscope}%
\pgfpathrectangle{\pgfqpoint{0.800000in}{0.528000in}}{\pgfqpoint{4.960000in}{3.696000in}}%
\pgfusepath{clip}%
\pgfsetbuttcap%
\pgfsetroundjoin%
\definecolor{currentfill}{rgb}{0.000000,0.000000,0.000000}%
\pgfsetfillcolor{currentfill}%
\pgfsetlinewidth{1.003750pt}%
\definecolor{currentstroke}{rgb}{0.000000,0.000000,0.000000}%
\pgfsetstrokecolor{currentstroke}%
\pgfsetdash{}{0pt}%
\pgfpathmoveto{\pgfqpoint{1.025906in}{0.676804in}}%
\pgfpathcurveto{\pgfqpoint{1.036956in}{0.676804in}}{\pgfqpoint{1.047555in}{0.681195in}}{\pgfqpoint{1.055369in}{0.689008in}}%
\pgfpathcurveto{\pgfqpoint{1.063182in}{0.696822in}}{\pgfqpoint{1.067573in}{0.707421in}}{\pgfqpoint{1.067573in}{0.718471in}}%
\pgfpathcurveto{\pgfqpoint{1.067573in}{0.729521in}}{\pgfqpoint{1.063182in}{0.740120in}}{\pgfqpoint{1.055369in}{0.747934in}}%
\pgfpathcurveto{\pgfqpoint{1.047555in}{0.755748in}}{\pgfqpoint{1.036956in}{0.760138in}}{\pgfqpoint{1.025906in}{0.760138in}}%
\pgfpathcurveto{\pgfqpoint{1.014856in}{0.760138in}}{\pgfqpoint{1.004257in}{0.755748in}}{\pgfqpoint{0.996443in}{0.747934in}}%
\pgfpathcurveto{\pgfqpoint{0.988630in}{0.740120in}}{\pgfqpoint{0.984239in}{0.729521in}}{\pgfqpoint{0.984239in}{0.718471in}}%
\pgfpathcurveto{\pgfqpoint{0.984239in}{0.707421in}}{\pgfqpoint{0.988630in}{0.696822in}}{\pgfqpoint{0.996443in}{0.689008in}}%
\pgfpathcurveto{\pgfqpoint{1.004257in}{0.681195in}}{\pgfqpoint{1.014856in}{0.676804in}}{\pgfqpoint{1.025906in}{0.676804in}}%
\pgfpathclose%
\pgfusepath{stroke,fill}%
\end{pgfscope}%
\begin{pgfscope}%
\pgfpathrectangle{\pgfqpoint{0.800000in}{0.528000in}}{\pgfqpoint{4.960000in}{3.696000in}}%
\pgfusepath{clip}%
\pgfsetbuttcap%
\pgfsetroundjoin%
\definecolor{currentfill}{rgb}{0.000000,0.000000,0.000000}%
\pgfsetfillcolor{currentfill}%
\pgfsetlinewidth{1.003750pt}%
\definecolor{currentstroke}{rgb}{0.000000,0.000000,0.000000}%
\pgfsetstrokecolor{currentstroke}%
\pgfsetdash{}{0pt}%
\pgfpathmoveto{\pgfqpoint{1.025906in}{0.698302in}}%
\pgfpathcurveto{\pgfqpoint{1.036956in}{0.698302in}}{\pgfqpoint{1.047555in}{0.702692in}}{\pgfqpoint{1.055369in}{0.710506in}}%
\pgfpathcurveto{\pgfqpoint{1.063182in}{0.718319in}}{\pgfqpoint{1.067573in}{0.728918in}}{\pgfqpoint{1.067573in}{0.739969in}}%
\pgfpathcurveto{\pgfqpoint{1.067573in}{0.751019in}}{\pgfqpoint{1.063182in}{0.761618in}}{\pgfqpoint{1.055369in}{0.769431in}}%
\pgfpathcurveto{\pgfqpoint{1.047555in}{0.777245in}}{\pgfqpoint{1.036956in}{0.781635in}}{\pgfqpoint{1.025906in}{0.781635in}}%
\pgfpathcurveto{\pgfqpoint{1.014856in}{0.781635in}}{\pgfqpoint{1.004257in}{0.777245in}}{\pgfqpoint{0.996443in}{0.769431in}}%
\pgfpathcurveto{\pgfqpoint{0.988630in}{0.761618in}}{\pgfqpoint{0.984239in}{0.751019in}}{\pgfqpoint{0.984239in}{0.739969in}}%
\pgfpathcurveto{\pgfqpoint{0.984239in}{0.728918in}}{\pgfqpoint{0.988630in}{0.718319in}}{\pgfqpoint{0.996443in}{0.710506in}}%
\pgfpathcurveto{\pgfqpoint{1.004257in}{0.702692in}}{\pgfqpoint{1.014856in}{0.698302in}}{\pgfqpoint{1.025906in}{0.698302in}}%
\pgfpathclose%
\pgfusepath{stroke,fill}%
\end{pgfscope}%
\begin{pgfscope}%
\pgfpathrectangle{\pgfqpoint{0.800000in}{0.528000in}}{\pgfqpoint{4.960000in}{3.696000in}}%
\pgfusepath{clip}%
\pgfsetbuttcap%
\pgfsetroundjoin%
\definecolor{currentfill}{rgb}{0.000000,0.000000,0.000000}%
\pgfsetfillcolor{currentfill}%
\pgfsetlinewidth{1.003750pt}%
\definecolor{currentstroke}{rgb}{0.000000,0.000000,0.000000}%
\pgfsetstrokecolor{currentstroke}%
\pgfsetdash{}{0pt}%
\pgfpathmoveto{\pgfqpoint{1.025906in}{0.698302in}}%
\pgfpathcurveto{\pgfqpoint{1.036956in}{0.698302in}}{\pgfqpoint{1.047555in}{0.702692in}}{\pgfqpoint{1.055369in}{0.710506in}}%
\pgfpathcurveto{\pgfqpoint{1.063182in}{0.718319in}}{\pgfqpoint{1.067573in}{0.728918in}}{\pgfqpoint{1.067573in}{0.739969in}}%
\pgfpathcurveto{\pgfqpoint{1.067573in}{0.751019in}}{\pgfqpoint{1.063182in}{0.761618in}}{\pgfqpoint{1.055369in}{0.769431in}}%
\pgfpathcurveto{\pgfqpoint{1.047555in}{0.777245in}}{\pgfqpoint{1.036956in}{0.781635in}}{\pgfqpoint{1.025906in}{0.781635in}}%
\pgfpathcurveto{\pgfqpoint{1.014856in}{0.781635in}}{\pgfqpoint{1.004257in}{0.777245in}}{\pgfqpoint{0.996443in}{0.769431in}}%
\pgfpathcurveto{\pgfqpoint{0.988630in}{0.761618in}}{\pgfqpoint{0.984239in}{0.751019in}}{\pgfqpoint{0.984239in}{0.739969in}}%
\pgfpathcurveto{\pgfqpoint{0.984239in}{0.728918in}}{\pgfqpoint{0.988630in}{0.718319in}}{\pgfqpoint{0.996443in}{0.710506in}}%
\pgfpathcurveto{\pgfqpoint{1.004257in}{0.702692in}}{\pgfqpoint{1.014856in}{0.698302in}}{\pgfqpoint{1.025906in}{0.698302in}}%
\pgfpathclose%
\pgfusepath{stroke,fill}%
\end{pgfscope}%
\begin{pgfscope}%
\pgfpathrectangle{\pgfqpoint{0.800000in}{0.528000in}}{\pgfqpoint{4.960000in}{3.696000in}}%
\pgfusepath{clip}%
\pgfsetbuttcap%
\pgfsetroundjoin%
\definecolor{currentfill}{rgb}{0.000000,0.000000,0.000000}%
\pgfsetfillcolor{currentfill}%
\pgfsetlinewidth{1.003750pt}%
\definecolor{currentstroke}{rgb}{0.000000,0.000000,0.000000}%
\pgfsetstrokecolor{currentstroke}%
\pgfsetdash{}{0pt}%
\pgfpathmoveto{\pgfqpoint{1.025906in}{0.719799in}}%
\pgfpathcurveto{\pgfqpoint{1.036956in}{0.719799in}}{\pgfqpoint{1.047555in}{0.724190in}}{\pgfqpoint{1.055369in}{0.732003in}}%
\pgfpathcurveto{\pgfqpoint{1.063182in}{0.739817in}}{\pgfqpoint{1.067573in}{0.750416in}}{\pgfqpoint{1.067573in}{0.761466in}}%
\pgfpathcurveto{\pgfqpoint{1.067573in}{0.772516in}}{\pgfqpoint{1.063182in}{0.783115in}}{\pgfqpoint{1.055369in}{0.790929in}}%
\pgfpathcurveto{\pgfqpoint{1.047555in}{0.798743in}}{\pgfqpoint{1.036956in}{0.803133in}}{\pgfqpoint{1.025906in}{0.803133in}}%
\pgfpathcurveto{\pgfqpoint{1.014856in}{0.803133in}}{\pgfqpoint{1.004257in}{0.798743in}}{\pgfqpoint{0.996443in}{0.790929in}}%
\pgfpathcurveto{\pgfqpoint{0.988630in}{0.783115in}}{\pgfqpoint{0.984239in}{0.772516in}}{\pgfqpoint{0.984239in}{0.761466in}}%
\pgfpathcurveto{\pgfqpoint{0.984239in}{0.750416in}}{\pgfqpoint{0.988630in}{0.739817in}}{\pgfqpoint{0.996443in}{0.732003in}}%
\pgfpathcurveto{\pgfqpoint{1.004257in}{0.724190in}}{\pgfqpoint{1.014856in}{0.719799in}}{\pgfqpoint{1.025906in}{0.719799in}}%
\pgfpathclose%
\pgfusepath{stroke,fill}%
\end{pgfscope}%
\begin{pgfscope}%
\pgfpathrectangle{\pgfqpoint{0.800000in}{0.528000in}}{\pgfqpoint{4.960000in}{3.696000in}}%
\pgfusepath{clip}%
\pgfsetbuttcap%
\pgfsetroundjoin%
\definecolor{currentfill}{rgb}{0.000000,0.000000,0.000000}%
\pgfsetfillcolor{currentfill}%
\pgfsetlinewidth{1.003750pt}%
\definecolor{currentstroke}{rgb}{0.000000,0.000000,0.000000}%
\pgfsetstrokecolor{currentstroke}%
\pgfsetdash{}{0pt}%
\pgfpathmoveto{\pgfqpoint{1.025906in}{0.676804in}}%
\pgfpathcurveto{\pgfqpoint{1.036956in}{0.676804in}}{\pgfqpoint{1.047555in}{0.681195in}}{\pgfqpoint{1.055369in}{0.689008in}}%
\pgfpathcurveto{\pgfqpoint{1.063182in}{0.696822in}}{\pgfqpoint{1.067573in}{0.707421in}}{\pgfqpoint{1.067573in}{0.718471in}}%
\pgfpathcurveto{\pgfqpoint{1.067573in}{0.729521in}}{\pgfqpoint{1.063182in}{0.740120in}}{\pgfqpoint{1.055369in}{0.747934in}}%
\pgfpathcurveto{\pgfqpoint{1.047555in}{0.755748in}}{\pgfqpoint{1.036956in}{0.760138in}}{\pgfqpoint{1.025906in}{0.760138in}}%
\pgfpathcurveto{\pgfqpoint{1.014856in}{0.760138in}}{\pgfqpoint{1.004257in}{0.755748in}}{\pgfqpoint{0.996443in}{0.747934in}}%
\pgfpathcurveto{\pgfqpoint{0.988630in}{0.740120in}}{\pgfqpoint{0.984239in}{0.729521in}}{\pgfqpoint{0.984239in}{0.718471in}}%
\pgfpathcurveto{\pgfqpoint{0.984239in}{0.707421in}}{\pgfqpoint{0.988630in}{0.696822in}}{\pgfqpoint{0.996443in}{0.689008in}}%
\pgfpathcurveto{\pgfqpoint{1.004257in}{0.681195in}}{\pgfqpoint{1.014856in}{0.676804in}}{\pgfqpoint{1.025906in}{0.676804in}}%
\pgfpathclose%
\pgfusepath{stroke,fill}%
\end{pgfscope}%
\begin{pgfscope}%
\pgfpathrectangle{\pgfqpoint{0.800000in}{0.528000in}}{\pgfqpoint{4.960000in}{3.696000in}}%
\pgfusepath{clip}%
\pgfsetbuttcap%
\pgfsetroundjoin%
\definecolor{currentfill}{rgb}{0.000000,0.000000,0.000000}%
\pgfsetfillcolor{currentfill}%
\pgfsetlinewidth{1.003750pt}%
\definecolor{currentstroke}{rgb}{0.000000,0.000000,0.000000}%
\pgfsetstrokecolor{currentstroke}%
\pgfsetdash{}{0pt}%
\pgfpathmoveto{\pgfqpoint{1.025906in}{0.762794in}}%
\pgfpathcurveto{\pgfqpoint{1.036956in}{0.762794in}}{\pgfqpoint{1.047555in}{0.767185in}}{\pgfqpoint{1.055369in}{0.774998in}}%
\pgfpathcurveto{\pgfqpoint{1.063182in}{0.782812in}}{\pgfqpoint{1.067573in}{0.793411in}}{\pgfqpoint{1.067573in}{0.804461in}}%
\pgfpathcurveto{\pgfqpoint{1.067573in}{0.815511in}}{\pgfqpoint{1.063182in}{0.826110in}}{\pgfqpoint{1.055369in}{0.833924in}}%
\pgfpathcurveto{\pgfqpoint{1.047555in}{0.841738in}}{\pgfqpoint{1.036956in}{0.846128in}}{\pgfqpoint{1.025906in}{0.846128in}}%
\pgfpathcurveto{\pgfqpoint{1.014856in}{0.846128in}}{\pgfqpoint{1.004257in}{0.841738in}}{\pgfqpoint{0.996443in}{0.833924in}}%
\pgfpathcurveto{\pgfqpoint{0.988630in}{0.826110in}}{\pgfqpoint{0.984239in}{0.815511in}}{\pgfqpoint{0.984239in}{0.804461in}}%
\pgfpathcurveto{\pgfqpoint{0.984239in}{0.793411in}}{\pgfqpoint{0.988630in}{0.782812in}}{\pgfqpoint{0.996443in}{0.774998in}}%
\pgfpathcurveto{\pgfqpoint{1.004257in}{0.767185in}}{\pgfqpoint{1.014856in}{0.762794in}}{\pgfqpoint{1.025906in}{0.762794in}}%
\pgfpathclose%
\pgfusepath{stroke,fill}%
\end{pgfscope}%
\begin{pgfscope}%
\pgfpathrectangle{\pgfqpoint{0.800000in}{0.528000in}}{\pgfqpoint{4.960000in}{3.696000in}}%
\pgfusepath{clip}%
\pgfsetbuttcap%
\pgfsetroundjoin%
\definecolor{currentfill}{rgb}{0.000000,0.000000,0.000000}%
\pgfsetfillcolor{currentfill}%
\pgfsetlinewidth{1.003750pt}%
\definecolor{currentstroke}{rgb}{0.000000,0.000000,0.000000}%
\pgfsetstrokecolor{currentstroke}%
\pgfsetdash{}{0pt}%
\pgfpathmoveto{\pgfqpoint{1.025906in}{0.698302in}}%
\pgfpathcurveto{\pgfqpoint{1.036956in}{0.698302in}}{\pgfqpoint{1.047555in}{0.702692in}}{\pgfqpoint{1.055369in}{0.710506in}}%
\pgfpathcurveto{\pgfqpoint{1.063182in}{0.718319in}}{\pgfqpoint{1.067573in}{0.728918in}}{\pgfqpoint{1.067573in}{0.739969in}}%
\pgfpathcurveto{\pgfqpoint{1.067573in}{0.751019in}}{\pgfqpoint{1.063182in}{0.761618in}}{\pgfqpoint{1.055369in}{0.769431in}}%
\pgfpathcurveto{\pgfqpoint{1.047555in}{0.777245in}}{\pgfqpoint{1.036956in}{0.781635in}}{\pgfqpoint{1.025906in}{0.781635in}}%
\pgfpathcurveto{\pgfqpoint{1.014856in}{0.781635in}}{\pgfqpoint{1.004257in}{0.777245in}}{\pgfqpoint{0.996443in}{0.769431in}}%
\pgfpathcurveto{\pgfqpoint{0.988630in}{0.761618in}}{\pgfqpoint{0.984239in}{0.751019in}}{\pgfqpoint{0.984239in}{0.739969in}}%
\pgfpathcurveto{\pgfqpoint{0.984239in}{0.728918in}}{\pgfqpoint{0.988630in}{0.718319in}}{\pgfqpoint{0.996443in}{0.710506in}}%
\pgfpathcurveto{\pgfqpoint{1.004257in}{0.702692in}}{\pgfqpoint{1.014856in}{0.698302in}}{\pgfqpoint{1.025906in}{0.698302in}}%
\pgfpathclose%
\pgfusepath{stroke,fill}%
\end{pgfscope}%
\begin{pgfscope}%
\pgfpathrectangle{\pgfqpoint{0.800000in}{0.528000in}}{\pgfqpoint{4.960000in}{3.696000in}}%
\pgfusepath{clip}%
\pgfsetbuttcap%
\pgfsetroundjoin%
\definecolor{currentfill}{rgb}{0.000000,0.000000,0.000000}%
\pgfsetfillcolor{currentfill}%
\pgfsetlinewidth{1.003750pt}%
\definecolor{currentstroke}{rgb}{0.000000,0.000000,0.000000}%
\pgfsetstrokecolor{currentstroke}%
\pgfsetdash{}{0pt}%
\pgfpathmoveto{\pgfqpoint{1.025906in}{0.698302in}}%
\pgfpathcurveto{\pgfqpoint{1.036956in}{0.698302in}}{\pgfqpoint{1.047555in}{0.702692in}}{\pgfqpoint{1.055369in}{0.710506in}}%
\pgfpathcurveto{\pgfqpoint{1.063182in}{0.718319in}}{\pgfqpoint{1.067573in}{0.728918in}}{\pgfqpoint{1.067573in}{0.739969in}}%
\pgfpathcurveto{\pgfqpoint{1.067573in}{0.751019in}}{\pgfqpoint{1.063182in}{0.761618in}}{\pgfqpoint{1.055369in}{0.769431in}}%
\pgfpathcurveto{\pgfqpoint{1.047555in}{0.777245in}}{\pgfqpoint{1.036956in}{0.781635in}}{\pgfqpoint{1.025906in}{0.781635in}}%
\pgfpathcurveto{\pgfqpoint{1.014856in}{0.781635in}}{\pgfqpoint{1.004257in}{0.777245in}}{\pgfqpoint{0.996443in}{0.769431in}}%
\pgfpathcurveto{\pgfqpoint{0.988630in}{0.761618in}}{\pgfqpoint{0.984239in}{0.751019in}}{\pgfqpoint{0.984239in}{0.739969in}}%
\pgfpathcurveto{\pgfqpoint{0.984239in}{0.728918in}}{\pgfqpoint{0.988630in}{0.718319in}}{\pgfqpoint{0.996443in}{0.710506in}}%
\pgfpathcurveto{\pgfqpoint{1.004257in}{0.702692in}}{\pgfqpoint{1.014856in}{0.698302in}}{\pgfqpoint{1.025906in}{0.698302in}}%
\pgfpathclose%
\pgfusepath{stroke,fill}%
\end{pgfscope}%
\begin{pgfscope}%
\pgfpathrectangle{\pgfqpoint{0.800000in}{0.528000in}}{\pgfqpoint{4.960000in}{3.696000in}}%
\pgfusepath{clip}%
\pgfsetbuttcap%
\pgfsetroundjoin%
\definecolor{currentfill}{rgb}{0.000000,0.000000,0.000000}%
\pgfsetfillcolor{currentfill}%
\pgfsetlinewidth{1.003750pt}%
\definecolor{currentstroke}{rgb}{0.000000,0.000000,0.000000}%
\pgfsetstrokecolor{currentstroke}%
\pgfsetdash{}{0pt}%
\pgfpathmoveto{\pgfqpoint{1.025906in}{0.676804in}}%
\pgfpathcurveto{\pgfqpoint{1.036956in}{0.676804in}}{\pgfqpoint{1.047555in}{0.681195in}}{\pgfqpoint{1.055369in}{0.689008in}}%
\pgfpathcurveto{\pgfqpoint{1.063182in}{0.696822in}}{\pgfqpoint{1.067573in}{0.707421in}}{\pgfqpoint{1.067573in}{0.718471in}}%
\pgfpathcurveto{\pgfqpoint{1.067573in}{0.729521in}}{\pgfqpoint{1.063182in}{0.740120in}}{\pgfqpoint{1.055369in}{0.747934in}}%
\pgfpathcurveto{\pgfqpoint{1.047555in}{0.755748in}}{\pgfqpoint{1.036956in}{0.760138in}}{\pgfqpoint{1.025906in}{0.760138in}}%
\pgfpathcurveto{\pgfqpoint{1.014856in}{0.760138in}}{\pgfqpoint{1.004257in}{0.755748in}}{\pgfqpoint{0.996443in}{0.747934in}}%
\pgfpathcurveto{\pgfqpoint{0.988630in}{0.740120in}}{\pgfqpoint{0.984239in}{0.729521in}}{\pgfqpoint{0.984239in}{0.718471in}}%
\pgfpathcurveto{\pgfqpoint{0.984239in}{0.707421in}}{\pgfqpoint{0.988630in}{0.696822in}}{\pgfqpoint{0.996443in}{0.689008in}}%
\pgfpathcurveto{\pgfqpoint{1.004257in}{0.681195in}}{\pgfqpoint{1.014856in}{0.676804in}}{\pgfqpoint{1.025906in}{0.676804in}}%
\pgfpathclose%
\pgfusepath{stroke,fill}%
\end{pgfscope}%
\begin{pgfscope}%
\pgfpathrectangle{\pgfqpoint{0.800000in}{0.528000in}}{\pgfqpoint{4.960000in}{3.696000in}}%
\pgfusepath{clip}%
\pgfsetbuttcap%
\pgfsetroundjoin%
\definecolor{currentfill}{rgb}{0.000000,0.000000,0.000000}%
\pgfsetfillcolor{currentfill}%
\pgfsetlinewidth{1.003750pt}%
\definecolor{currentstroke}{rgb}{0.000000,0.000000,0.000000}%
\pgfsetstrokecolor{currentstroke}%
\pgfsetdash{}{0pt}%
\pgfpathmoveto{\pgfqpoint{1.025906in}{0.719799in}}%
\pgfpathcurveto{\pgfqpoint{1.036956in}{0.719799in}}{\pgfqpoint{1.047555in}{0.724190in}}{\pgfqpoint{1.055369in}{0.732003in}}%
\pgfpathcurveto{\pgfqpoint{1.063182in}{0.739817in}}{\pgfqpoint{1.067573in}{0.750416in}}{\pgfqpoint{1.067573in}{0.761466in}}%
\pgfpathcurveto{\pgfqpoint{1.067573in}{0.772516in}}{\pgfqpoint{1.063182in}{0.783115in}}{\pgfqpoint{1.055369in}{0.790929in}}%
\pgfpathcurveto{\pgfqpoint{1.047555in}{0.798743in}}{\pgfqpoint{1.036956in}{0.803133in}}{\pgfqpoint{1.025906in}{0.803133in}}%
\pgfpathcurveto{\pgfqpoint{1.014856in}{0.803133in}}{\pgfqpoint{1.004257in}{0.798743in}}{\pgfqpoint{0.996443in}{0.790929in}}%
\pgfpathcurveto{\pgfqpoint{0.988630in}{0.783115in}}{\pgfqpoint{0.984239in}{0.772516in}}{\pgfqpoint{0.984239in}{0.761466in}}%
\pgfpathcurveto{\pgfqpoint{0.984239in}{0.750416in}}{\pgfqpoint{0.988630in}{0.739817in}}{\pgfqpoint{0.996443in}{0.732003in}}%
\pgfpathcurveto{\pgfqpoint{1.004257in}{0.724190in}}{\pgfqpoint{1.014856in}{0.719799in}}{\pgfqpoint{1.025906in}{0.719799in}}%
\pgfpathclose%
\pgfusepath{stroke,fill}%
\end{pgfscope}%
\begin{pgfscope}%
\pgfpathrectangle{\pgfqpoint{0.800000in}{0.528000in}}{\pgfqpoint{4.960000in}{3.696000in}}%
\pgfusepath{clip}%
\pgfsetbuttcap%
\pgfsetroundjoin%
\definecolor{currentfill}{rgb}{0.000000,0.000000,0.000000}%
\pgfsetfillcolor{currentfill}%
\pgfsetlinewidth{1.003750pt}%
\definecolor{currentstroke}{rgb}{0.000000,0.000000,0.000000}%
\pgfsetstrokecolor{currentstroke}%
\pgfsetdash{}{0pt}%
\pgfpathmoveto{\pgfqpoint{1.025906in}{0.719799in}}%
\pgfpathcurveto{\pgfqpoint{1.036956in}{0.719799in}}{\pgfqpoint{1.047555in}{0.724190in}}{\pgfqpoint{1.055369in}{0.732003in}}%
\pgfpathcurveto{\pgfqpoint{1.063182in}{0.739817in}}{\pgfqpoint{1.067573in}{0.750416in}}{\pgfqpoint{1.067573in}{0.761466in}}%
\pgfpathcurveto{\pgfqpoint{1.067573in}{0.772516in}}{\pgfqpoint{1.063182in}{0.783115in}}{\pgfqpoint{1.055369in}{0.790929in}}%
\pgfpathcurveto{\pgfqpoint{1.047555in}{0.798743in}}{\pgfqpoint{1.036956in}{0.803133in}}{\pgfqpoint{1.025906in}{0.803133in}}%
\pgfpathcurveto{\pgfqpoint{1.014856in}{0.803133in}}{\pgfqpoint{1.004257in}{0.798743in}}{\pgfqpoint{0.996443in}{0.790929in}}%
\pgfpathcurveto{\pgfqpoint{0.988630in}{0.783115in}}{\pgfqpoint{0.984239in}{0.772516in}}{\pgfqpoint{0.984239in}{0.761466in}}%
\pgfpathcurveto{\pgfqpoint{0.984239in}{0.750416in}}{\pgfqpoint{0.988630in}{0.739817in}}{\pgfqpoint{0.996443in}{0.732003in}}%
\pgfpathcurveto{\pgfqpoint{1.004257in}{0.724190in}}{\pgfqpoint{1.014856in}{0.719799in}}{\pgfqpoint{1.025906in}{0.719799in}}%
\pgfpathclose%
\pgfusepath{stroke,fill}%
\end{pgfscope}%
\begin{pgfscope}%
\pgfpathrectangle{\pgfqpoint{0.800000in}{0.528000in}}{\pgfqpoint{4.960000in}{3.696000in}}%
\pgfusepath{clip}%
\pgfsetbuttcap%
\pgfsetroundjoin%
\definecolor{currentfill}{rgb}{0.000000,0.000000,0.000000}%
\pgfsetfillcolor{currentfill}%
\pgfsetlinewidth{1.003750pt}%
\definecolor{currentstroke}{rgb}{0.000000,0.000000,0.000000}%
\pgfsetstrokecolor{currentstroke}%
\pgfsetdash{}{0pt}%
\pgfpathmoveto{\pgfqpoint{1.025906in}{0.698302in}}%
\pgfpathcurveto{\pgfqpoint{1.036956in}{0.698302in}}{\pgfqpoint{1.047555in}{0.702692in}}{\pgfqpoint{1.055369in}{0.710506in}}%
\pgfpathcurveto{\pgfqpoint{1.063182in}{0.718319in}}{\pgfqpoint{1.067573in}{0.728918in}}{\pgfqpoint{1.067573in}{0.739969in}}%
\pgfpathcurveto{\pgfqpoint{1.067573in}{0.751019in}}{\pgfqpoint{1.063182in}{0.761618in}}{\pgfqpoint{1.055369in}{0.769431in}}%
\pgfpathcurveto{\pgfqpoint{1.047555in}{0.777245in}}{\pgfqpoint{1.036956in}{0.781635in}}{\pgfqpoint{1.025906in}{0.781635in}}%
\pgfpathcurveto{\pgfqpoint{1.014856in}{0.781635in}}{\pgfqpoint{1.004257in}{0.777245in}}{\pgfqpoint{0.996443in}{0.769431in}}%
\pgfpathcurveto{\pgfqpoint{0.988630in}{0.761618in}}{\pgfqpoint{0.984239in}{0.751019in}}{\pgfqpoint{0.984239in}{0.739969in}}%
\pgfpathcurveto{\pgfqpoint{0.984239in}{0.728918in}}{\pgfqpoint{0.988630in}{0.718319in}}{\pgfqpoint{0.996443in}{0.710506in}}%
\pgfpathcurveto{\pgfqpoint{1.004257in}{0.702692in}}{\pgfqpoint{1.014856in}{0.698302in}}{\pgfqpoint{1.025906in}{0.698302in}}%
\pgfpathclose%
\pgfusepath{stroke,fill}%
\end{pgfscope}%
\begin{pgfscope}%
\pgfpathrectangle{\pgfqpoint{0.800000in}{0.528000in}}{\pgfqpoint{4.960000in}{3.696000in}}%
\pgfusepath{clip}%
\pgfsetbuttcap%
\pgfsetroundjoin%
\definecolor{currentfill}{rgb}{0.000000,0.000000,0.000000}%
\pgfsetfillcolor{currentfill}%
\pgfsetlinewidth{1.003750pt}%
\definecolor{currentstroke}{rgb}{0.000000,0.000000,0.000000}%
\pgfsetstrokecolor{currentstroke}%
\pgfsetdash{}{0pt}%
\pgfpathmoveto{\pgfqpoint{1.025906in}{0.719799in}}%
\pgfpathcurveto{\pgfqpoint{1.036956in}{0.719799in}}{\pgfqpoint{1.047555in}{0.724190in}}{\pgfqpoint{1.055369in}{0.732003in}}%
\pgfpathcurveto{\pgfqpoint{1.063182in}{0.739817in}}{\pgfqpoint{1.067573in}{0.750416in}}{\pgfqpoint{1.067573in}{0.761466in}}%
\pgfpathcurveto{\pgfqpoint{1.067573in}{0.772516in}}{\pgfqpoint{1.063182in}{0.783115in}}{\pgfqpoint{1.055369in}{0.790929in}}%
\pgfpathcurveto{\pgfqpoint{1.047555in}{0.798743in}}{\pgfqpoint{1.036956in}{0.803133in}}{\pgfqpoint{1.025906in}{0.803133in}}%
\pgfpathcurveto{\pgfqpoint{1.014856in}{0.803133in}}{\pgfqpoint{1.004257in}{0.798743in}}{\pgfqpoint{0.996443in}{0.790929in}}%
\pgfpathcurveto{\pgfqpoint{0.988630in}{0.783115in}}{\pgfqpoint{0.984239in}{0.772516in}}{\pgfqpoint{0.984239in}{0.761466in}}%
\pgfpathcurveto{\pgfqpoint{0.984239in}{0.750416in}}{\pgfqpoint{0.988630in}{0.739817in}}{\pgfqpoint{0.996443in}{0.732003in}}%
\pgfpathcurveto{\pgfqpoint{1.004257in}{0.724190in}}{\pgfqpoint{1.014856in}{0.719799in}}{\pgfqpoint{1.025906in}{0.719799in}}%
\pgfpathclose%
\pgfusepath{stroke,fill}%
\end{pgfscope}%
\begin{pgfscope}%
\pgfpathrectangle{\pgfqpoint{0.800000in}{0.528000in}}{\pgfqpoint{4.960000in}{3.696000in}}%
\pgfusepath{clip}%
\pgfsetbuttcap%
\pgfsetroundjoin%
\definecolor{currentfill}{rgb}{0.000000,0.000000,0.000000}%
\pgfsetfillcolor{currentfill}%
\pgfsetlinewidth{1.003750pt}%
\definecolor{currentstroke}{rgb}{0.000000,0.000000,0.000000}%
\pgfsetstrokecolor{currentstroke}%
\pgfsetdash{}{0pt}%
\pgfpathmoveto{\pgfqpoint{1.025906in}{0.676804in}}%
\pgfpathcurveto{\pgfqpoint{1.036956in}{0.676804in}}{\pgfqpoint{1.047555in}{0.681195in}}{\pgfqpoint{1.055369in}{0.689008in}}%
\pgfpathcurveto{\pgfqpoint{1.063182in}{0.696822in}}{\pgfqpoint{1.067573in}{0.707421in}}{\pgfqpoint{1.067573in}{0.718471in}}%
\pgfpathcurveto{\pgfqpoint{1.067573in}{0.729521in}}{\pgfqpoint{1.063182in}{0.740120in}}{\pgfqpoint{1.055369in}{0.747934in}}%
\pgfpathcurveto{\pgfqpoint{1.047555in}{0.755748in}}{\pgfqpoint{1.036956in}{0.760138in}}{\pgfqpoint{1.025906in}{0.760138in}}%
\pgfpathcurveto{\pgfqpoint{1.014856in}{0.760138in}}{\pgfqpoint{1.004257in}{0.755748in}}{\pgfqpoint{0.996443in}{0.747934in}}%
\pgfpathcurveto{\pgfqpoint{0.988630in}{0.740120in}}{\pgfqpoint{0.984239in}{0.729521in}}{\pgfqpoint{0.984239in}{0.718471in}}%
\pgfpathcurveto{\pgfqpoint{0.984239in}{0.707421in}}{\pgfqpoint{0.988630in}{0.696822in}}{\pgfqpoint{0.996443in}{0.689008in}}%
\pgfpathcurveto{\pgfqpoint{1.004257in}{0.681195in}}{\pgfqpoint{1.014856in}{0.676804in}}{\pgfqpoint{1.025906in}{0.676804in}}%
\pgfpathclose%
\pgfusepath{stroke,fill}%
\end{pgfscope}%
\begin{pgfscope}%
\pgfpathrectangle{\pgfqpoint{0.800000in}{0.528000in}}{\pgfqpoint{4.960000in}{3.696000in}}%
\pgfusepath{clip}%
\pgfsetbuttcap%
\pgfsetroundjoin%
\definecolor{currentfill}{rgb}{0.000000,0.000000,0.000000}%
\pgfsetfillcolor{currentfill}%
\pgfsetlinewidth{1.003750pt}%
\definecolor{currentstroke}{rgb}{0.000000,0.000000,0.000000}%
\pgfsetstrokecolor{currentstroke}%
\pgfsetdash{}{0pt}%
\pgfpathmoveto{\pgfqpoint{1.025906in}{0.698302in}}%
\pgfpathcurveto{\pgfqpoint{1.036956in}{0.698302in}}{\pgfqpoint{1.047555in}{0.702692in}}{\pgfqpoint{1.055369in}{0.710506in}}%
\pgfpathcurveto{\pgfqpoint{1.063182in}{0.718319in}}{\pgfqpoint{1.067573in}{0.728918in}}{\pgfqpoint{1.067573in}{0.739969in}}%
\pgfpathcurveto{\pgfqpoint{1.067573in}{0.751019in}}{\pgfqpoint{1.063182in}{0.761618in}}{\pgfqpoint{1.055369in}{0.769431in}}%
\pgfpathcurveto{\pgfqpoint{1.047555in}{0.777245in}}{\pgfqpoint{1.036956in}{0.781635in}}{\pgfqpoint{1.025906in}{0.781635in}}%
\pgfpathcurveto{\pgfqpoint{1.014856in}{0.781635in}}{\pgfqpoint{1.004257in}{0.777245in}}{\pgfqpoint{0.996443in}{0.769431in}}%
\pgfpathcurveto{\pgfqpoint{0.988630in}{0.761618in}}{\pgfqpoint{0.984239in}{0.751019in}}{\pgfqpoint{0.984239in}{0.739969in}}%
\pgfpathcurveto{\pgfqpoint{0.984239in}{0.728918in}}{\pgfqpoint{0.988630in}{0.718319in}}{\pgfqpoint{0.996443in}{0.710506in}}%
\pgfpathcurveto{\pgfqpoint{1.004257in}{0.702692in}}{\pgfqpoint{1.014856in}{0.698302in}}{\pgfqpoint{1.025906in}{0.698302in}}%
\pgfpathclose%
\pgfusepath{stroke,fill}%
\end{pgfscope}%
\begin{pgfscope}%
\pgfpathrectangle{\pgfqpoint{0.800000in}{0.528000in}}{\pgfqpoint{4.960000in}{3.696000in}}%
\pgfusepath{clip}%
\pgfsetbuttcap%
\pgfsetroundjoin%
\definecolor{currentfill}{rgb}{0.000000,0.000000,0.000000}%
\pgfsetfillcolor{currentfill}%
\pgfsetlinewidth{1.003750pt}%
\definecolor{currentstroke}{rgb}{0.000000,0.000000,0.000000}%
\pgfsetstrokecolor{currentstroke}%
\pgfsetdash{}{0pt}%
\pgfpathmoveto{\pgfqpoint{1.025906in}{0.698302in}}%
\pgfpathcurveto{\pgfqpoint{1.036956in}{0.698302in}}{\pgfqpoint{1.047555in}{0.702692in}}{\pgfqpoint{1.055369in}{0.710506in}}%
\pgfpathcurveto{\pgfqpoint{1.063182in}{0.718319in}}{\pgfqpoint{1.067573in}{0.728918in}}{\pgfqpoint{1.067573in}{0.739969in}}%
\pgfpathcurveto{\pgfqpoint{1.067573in}{0.751019in}}{\pgfqpoint{1.063182in}{0.761618in}}{\pgfqpoint{1.055369in}{0.769431in}}%
\pgfpathcurveto{\pgfqpoint{1.047555in}{0.777245in}}{\pgfqpoint{1.036956in}{0.781635in}}{\pgfqpoint{1.025906in}{0.781635in}}%
\pgfpathcurveto{\pgfqpoint{1.014856in}{0.781635in}}{\pgfqpoint{1.004257in}{0.777245in}}{\pgfqpoint{0.996443in}{0.769431in}}%
\pgfpathcurveto{\pgfqpoint{0.988630in}{0.761618in}}{\pgfqpoint{0.984239in}{0.751019in}}{\pgfqpoint{0.984239in}{0.739969in}}%
\pgfpathcurveto{\pgfqpoint{0.984239in}{0.728918in}}{\pgfqpoint{0.988630in}{0.718319in}}{\pgfqpoint{0.996443in}{0.710506in}}%
\pgfpathcurveto{\pgfqpoint{1.004257in}{0.702692in}}{\pgfqpoint{1.014856in}{0.698302in}}{\pgfqpoint{1.025906in}{0.698302in}}%
\pgfpathclose%
\pgfusepath{stroke,fill}%
\end{pgfscope}%
\begin{pgfscope}%
\pgfpathrectangle{\pgfqpoint{0.800000in}{0.528000in}}{\pgfqpoint{4.960000in}{3.696000in}}%
\pgfusepath{clip}%
\pgfsetbuttcap%
\pgfsetroundjoin%
\definecolor{currentfill}{rgb}{0.000000,0.000000,0.000000}%
\pgfsetfillcolor{currentfill}%
\pgfsetlinewidth{1.003750pt}%
\definecolor{currentstroke}{rgb}{0.000000,0.000000,0.000000}%
\pgfsetstrokecolor{currentstroke}%
\pgfsetdash{}{0pt}%
\pgfpathmoveto{\pgfqpoint{1.025906in}{0.698302in}}%
\pgfpathcurveto{\pgfqpoint{1.036956in}{0.698302in}}{\pgfqpoint{1.047555in}{0.702692in}}{\pgfqpoint{1.055369in}{0.710506in}}%
\pgfpathcurveto{\pgfqpoint{1.063182in}{0.718319in}}{\pgfqpoint{1.067573in}{0.728918in}}{\pgfqpoint{1.067573in}{0.739969in}}%
\pgfpathcurveto{\pgfqpoint{1.067573in}{0.751019in}}{\pgfqpoint{1.063182in}{0.761618in}}{\pgfqpoint{1.055369in}{0.769431in}}%
\pgfpathcurveto{\pgfqpoint{1.047555in}{0.777245in}}{\pgfqpoint{1.036956in}{0.781635in}}{\pgfqpoint{1.025906in}{0.781635in}}%
\pgfpathcurveto{\pgfqpoint{1.014856in}{0.781635in}}{\pgfqpoint{1.004257in}{0.777245in}}{\pgfqpoint{0.996443in}{0.769431in}}%
\pgfpathcurveto{\pgfqpoint{0.988630in}{0.761618in}}{\pgfqpoint{0.984239in}{0.751019in}}{\pgfqpoint{0.984239in}{0.739969in}}%
\pgfpathcurveto{\pgfqpoint{0.984239in}{0.728918in}}{\pgfqpoint{0.988630in}{0.718319in}}{\pgfqpoint{0.996443in}{0.710506in}}%
\pgfpathcurveto{\pgfqpoint{1.004257in}{0.702692in}}{\pgfqpoint{1.014856in}{0.698302in}}{\pgfqpoint{1.025906in}{0.698302in}}%
\pgfpathclose%
\pgfusepath{stroke,fill}%
\end{pgfscope}%
\begin{pgfscope}%
\pgfpathrectangle{\pgfqpoint{0.800000in}{0.528000in}}{\pgfqpoint{4.960000in}{3.696000in}}%
\pgfusepath{clip}%
\pgfsetbuttcap%
\pgfsetroundjoin%
\definecolor{currentfill}{rgb}{0.000000,0.000000,0.000000}%
\pgfsetfillcolor{currentfill}%
\pgfsetlinewidth{1.003750pt}%
\definecolor{currentstroke}{rgb}{0.000000,0.000000,0.000000}%
\pgfsetstrokecolor{currentstroke}%
\pgfsetdash{}{0pt}%
\pgfpathmoveto{\pgfqpoint{1.025906in}{0.719799in}}%
\pgfpathcurveto{\pgfqpoint{1.036956in}{0.719799in}}{\pgfqpoint{1.047555in}{0.724190in}}{\pgfqpoint{1.055369in}{0.732003in}}%
\pgfpathcurveto{\pgfqpoint{1.063182in}{0.739817in}}{\pgfqpoint{1.067573in}{0.750416in}}{\pgfqpoint{1.067573in}{0.761466in}}%
\pgfpathcurveto{\pgfqpoint{1.067573in}{0.772516in}}{\pgfqpoint{1.063182in}{0.783115in}}{\pgfqpoint{1.055369in}{0.790929in}}%
\pgfpathcurveto{\pgfqpoint{1.047555in}{0.798743in}}{\pgfqpoint{1.036956in}{0.803133in}}{\pgfqpoint{1.025906in}{0.803133in}}%
\pgfpathcurveto{\pgfqpoint{1.014856in}{0.803133in}}{\pgfqpoint{1.004257in}{0.798743in}}{\pgfqpoint{0.996443in}{0.790929in}}%
\pgfpathcurveto{\pgfqpoint{0.988630in}{0.783115in}}{\pgfqpoint{0.984239in}{0.772516in}}{\pgfqpoint{0.984239in}{0.761466in}}%
\pgfpathcurveto{\pgfqpoint{0.984239in}{0.750416in}}{\pgfqpoint{0.988630in}{0.739817in}}{\pgfqpoint{0.996443in}{0.732003in}}%
\pgfpathcurveto{\pgfqpoint{1.004257in}{0.724190in}}{\pgfqpoint{1.014856in}{0.719799in}}{\pgfqpoint{1.025906in}{0.719799in}}%
\pgfpathclose%
\pgfusepath{stroke,fill}%
\end{pgfscope}%
\begin{pgfscope}%
\pgfpathrectangle{\pgfqpoint{0.800000in}{0.528000in}}{\pgfqpoint{4.960000in}{3.696000in}}%
\pgfusepath{clip}%
\pgfsetbuttcap%
\pgfsetroundjoin%
\definecolor{currentfill}{rgb}{0.000000,0.000000,0.000000}%
\pgfsetfillcolor{currentfill}%
\pgfsetlinewidth{1.003750pt}%
\definecolor{currentstroke}{rgb}{0.000000,0.000000,0.000000}%
\pgfsetstrokecolor{currentstroke}%
\pgfsetdash{}{0pt}%
\pgfpathmoveto{\pgfqpoint{1.025906in}{0.698302in}}%
\pgfpathcurveto{\pgfqpoint{1.036956in}{0.698302in}}{\pgfqpoint{1.047555in}{0.702692in}}{\pgfqpoint{1.055369in}{0.710506in}}%
\pgfpathcurveto{\pgfqpoint{1.063182in}{0.718319in}}{\pgfqpoint{1.067573in}{0.728918in}}{\pgfqpoint{1.067573in}{0.739969in}}%
\pgfpathcurveto{\pgfqpoint{1.067573in}{0.751019in}}{\pgfqpoint{1.063182in}{0.761618in}}{\pgfqpoint{1.055369in}{0.769431in}}%
\pgfpathcurveto{\pgfqpoint{1.047555in}{0.777245in}}{\pgfqpoint{1.036956in}{0.781635in}}{\pgfqpoint{1.025906in}{0.781635in}}%
\pgfpathcurveto{\pgfqpoint{1.014856in}{0.781635in}}{\pgfqpoint{1.004257in}{0.777245in}}{\pgfqpoint{0.996443in}{0.769431in}}%
\pgfpathcurveto{\pgfqpoint{0.988630in}{0.761618in}}{\pgfqpoint{0.984239in}{0.751019in}}{\pgfqpoint{0.984239in}{0.739969in}}%
\pgfpathcurveto{\pgfqpoint{0.984239in}{0.728918in}}{\pgfqpoint{0.988630in}{0.718319in}}{\pgfqpoint{0.996443in}{0.710506in}}%
\pgfpathcurveto{\pgfqpoint{1.004257in}{0.702692in}}{\pgfqpoint{1.014856in}{0.698302in}}{\pgfqpoint{1.025906in}{0.698302in}}%
\pgfpathclose%
\pgfusepath{stroke,fill}%
\end{pgfscope}%
\begin{pgfscope}%
\pgfpathrectangle{\pgfqpoint{0.800000in}{0.528000in}}{\pgfqpoint{4.960000in}{3.696000in}}%
\pgfusepath{clip}%
\pgfsetbuttcap%
\pgfsetroundjoin%
\definecolor{currentfill}{rgb}{0.000000,0.000000,0.000000}%
\pgfsetfillcolor{currentfill}%
\pgfsetlinewidth{1.003750pt}%
\definecolor{currentstroke}{rgb}{0.000000,0.000000,0.000000}%
\pgfsetstrokecolor{currentstroke}%
\pgfsetdash{}{0pt}%
\pgfpathmoveto{\pgfqpoint{1.025906in}{0.698302in}}%
\pgfpathcurveto{\pgfqpoint{1.036956in}{0.698302in}}{\pgfqpoint{1.047555in}{0.702692in}}{\pgfqpoint{1.055369in}{0.710506in}}%
\pgfpathcurveto{\pgfqpoint{1.063182in}{0.718319in}}{\pgfqpoint{1.067573in}{0.728918in}}{\pgfqpoint{1.067573in}{0.739969in}}%
\pgfpathcurveto{\pgfqpoint{1.067573in}{0.751019in}}{\pgfqpoint{1.063182in}{0.761618in}}{\pgfqpoint{1.055369in}{0.769431in}}%
\pgfpathcurveto{\pgfqpoint{1.047555in}{0.777245in}}{\pgfqpoint{1.036956in}{0.781635in}}{\pgfqpoint{1.025906in}{0.781635in}}%
\pgfpathcurveto{\pgfqpoint{1.014856in}{0.781635in}}{\pgfqpoint{1.004257in}{0.777245in}}{\pgfqpoint{0.996443in}{0.769431in}}%
\pgfpathcurveto{\pgfqpoint{0.988630in}{0.761618in}}{\pgfqpoint{0.984239in}{0.751019in}}{\pgfqpoint{0.984239in}{0.739969in}}%
\pgfpathcurveto{\pgfqpoint{0.984239in}{0.728918in}}{\pgfqpoint{0.988630in}{0.718319in}}{\pgfqpoint{0.996443in}{0.710506in}}%
\pgfpathcurveto{\pgfqpoint{1.004257in}{0.702692in}}{\pgfqpoint{1.014856in}{0.698302in}}{\pgfqpoint{1.025906in}{0.698302in}}%
\pgfpathclose%
\pgfusepath{stroke,fill}%
\end{pgfscope}%
\begin{pgfscope}%
\pgfpathrectangle{\pgfqpoint{0.800000in}{0.528000in}}{\pgfqpoint{4.960000in}{3.696000in}}%
\pgfusepath{clip}%
\pgfsetbuttcap%
\pgfsetroundjoin%
\definecolor{currentfill}{rgb}{0.000000,0.000000,0.000000}%
\pgfsetfillcolor{currentfill}%
\pgfsetlinewidth{1.003750pt}%
\definecolor{currentstroke}{rgb}{0.000000,0.000000,0.000000}%
\pgfsetstrokecolor{currentstroke}%
\pgfsetdash{}{0pt}%
\pgfpathmoveto{\pgfqpoint{1.025906in}{0.676804in}}%
\pgfpathcurveto{\pgfqpoint{1.036956in}{0.676804in}}{\pgfqpoint{1.047555in}{0.681195in}}{\pgfqpoint{1.055369in}{0.689008in}}%
\pgfpathcurveto{\pgfqpoint{1.063182in}{0.696822in}}{\pgfqpoint{1.067573in}{0.707421in}}{\pgfqpoint{1.067573in}{0.718471in}}%
\pgfpathcurveto{\pgfqpoint{1.067573in}{0.729521in}}{\pgfqpoint{1.063182in}{0.740120in}}{\pgfqpoint{1.055369in}{0.747934in}}%
\pgfpathcurveto{\pgfqpoint{1.047555in}{0.755748in}}{\pgfqpoint{1.036956in}{0.760138in}}{\pgfqpoint{1.025906in}{0.760138in}}%
\pgfpathcurveto{\pgfqpoint{1.014856in}{0.760138in}}{\pgfqpoint{1.004257in}{0.755748in}}{\pgfqpoint{0.996443in}{0.747934in}}%
\pgfpathcurveto{\pgfqpoint{0.988630in}{0.740120in}}{\pgfqpoint{0.984239in}{0.729521in}}{\pgfqpoint{0.984239in}{0.718471in}}%
\pgfpathcurveto{\pgfqpoint{0.984239in}{0.707421in}}{\pgfqpoint{0.988630in}{0.696822in}}{\pgfqpoint{0.996443in}{0.689008in}}%
\pgfpathcurveto{\pgfqpoint{1.004257in}{0.681195in}}{\pgfqpoint{1.014856in}{0.676804in}}{\pgfqpoint{1.025906in}{0.676804in}}%
\pgfpathclose%
\pgfusepath{stroke,fill}%
\end{pgfscope}%
\begin{pgfscope}%
\pgfpathrectangle{\pgfqpoint{0.800000in}{0.528000in}}{\pgfqpoint{4.960000in}{3.696000in}}%
\pgfusepath{clip}%
\pgfsetbuttcap%
\pgfsetroundjoin%
\definecolor{currentfill}{rgb}{0.000000,0.000000,0.000000}%
\pgfsetfillcolor{currentfill}%
\pgfsetlinewidth{1.003750pt}%
\definecolor{currentstroke}{rgb}{0.000000,0.000000,0.000000}%
\pgfsetstrokecolor{currentstroke}%
\pgfsetdash{}{0pt}%
\pgfpathmoveto{\pgfqpoint{1.025906in}{0.719799in}}%
\pgfpathcurveto{\pgfqpoint{1.036956in}{0.719799in}}{\pgfqpoint{1.047555in}{0.724190in}}{\pgfqpoint{1.055369in}{0.732003in}}%
\pgfpathcurveto{\pgfqpoint{1.063182in}{0.739817in}}{\pgfqpoint{1.067573in}{0.750416in}}{\pgfqpoint{1.067573in}{0.761466in}}%
\pgfpathcurveto{\pgfqpoint{1.067573in}{0.772516in}}{\pgfqpoint{1.063182in}{0.783115in}}{\pgfqpoint{1.055369in}{0.790929in}}%
\pgfpathcurveto{\pgfqpoint{1.047555in}{0.798743in}}{\pgfqpoint{1.036956in}{0.803133in}}{\pgfqpoint{1.025906in}{0.803133in}}%
\pgfpathcurveto{\pgfqpoint{1.014856in}{0.803133in}}{\pgfqpoint{1.004257in}{0.798743in}}{\pgfqpoint{0.996443in}{0.790929in}}%
\pgfpathcurveto{\pgfqpoint{0.988630in}{0.783115in}}{\pgfqpoint{0.984239in}{0.772516in}}{\pgfqpoint{0.984239in}{0.761466in}}%
\pgfpathcurveto{\pgfqpoint{0.984239in}{0.750416in}}{\pgfqpoint{0.988630in}{0.739817in}}{\pgfqpoint{0.996443in}{0.732003in}}%
\pgfpathcurveto{\pgfqpoint{1.004257in}{0.724190in}}{\pgfqpoint{1.014856in}{0.719799in}}{\pgfqpoint{1.025906in}{0.719799in}}%
\pgfpathclose%
\pgfusepath{stroke,fill}%
\end{pgfscope}%
\begin{pgfscope}%
\pgfpathrectangle{\pgfqpoint{0.800000in}{0.528000in}}{\pgfqpoint{4.960000in}{3.696000in}}%
\pgfusepath{clip}%
\pgfsetbuttcap%
\pgfsetroundjoin%
\definecolor{currentfill}{rgb}{0.000000,0.000000,0.000000}%
\pgfsetfillcolor{currentfill}%
\pgfsetlinewidth{1.003750pt}%
\definecolor{currentstroke}{rgb}{0.000000,0.000000,0.000000}%
\pgfsetstrokecolor{currentstroke}%
\pgfsetdash{}{0pt}%
\pgfpathmoveto{\pgfqpoint{1.025906in}{0.676804in}}%
\pgfpathcurveto{\pgfqpoint{1.036956in}{0.676804in}}{\pgfqpoint{1.047555in}{0.681195in}}{\pgfqpoint{1.055369in}{0.689008in}}%
\pgfpathcurveto{\pgfqpoint{1.063182in}{0.696822in}}{\pgfqpoint{1.067573in}{0.707421in}}{\pgfqpoint{1.067573in}{0.718471in}}%
\pgfpathcurveto{\pgfqpoint{1.067573in}{0.729521in}}{\pgfqpoint{1.063182in}{0.740120in}}{\pgfqpoint{1.055369in}{0.747934in}}%
\pgfpathcurveto{\pgfqpoint{1.047555in}{0.755748in}}{\pgfqpoint{1.036956in}{0.760138in}}{\pgfqpoint{1.025906in}{0.760138in}}%
\pgfpathcurveto{\pgfqpoint{1.014856in}{0.760138in}}{\pgfqpoint{1.004257in}{0.755748in}}{\pgfqpoint{0.996443in}{0.747934in}}%
\pgfpathcurveto{\pgfqpoint{0.988630in}{0.740120in}}{\pgfqpoint{0.984239in}{0.729521in}}{\pgfqpoint{0.984239in}{0.718471in}}%
\pgfpathcurveto{\pgfqpoint{0.984239in}{0.707421in}}{\pgfqpoint{0.988630in}{0.696822in}}{\pgfqpoint{0.996443in}{0.689008in}}%
\pgfpathcurveto{\pgfqpoint{1.004257in}{0.681195in}}{\pgfqpoint{1.014856in}{0.676804in}}{\pgfqpoint{1.025906in}{0.676804in}}%
\pgfpathclose%
\pgfusepath{stroke,fill}%
\end{pgfscope}%
\begin{pgfscope}%
\pgfpathrectangle{\pgfqpoint{0.800000in}{0.528000in}}{\pgfqpoint{4.960000in}{3.696000in}}%
\pgfusepath{clip}%
\pgfsetbuttcap%
\pgfsetroundjoin%
\definecolor{currentfill}{rgb}{0.000000,0.000000,0.000000}%
\pgfsetfillcolor{currentfill}%
\pgfsetlinewidth{1.003750pt}%
\definecolor{currentstroke}{rgb}{0.000000,0.000000,0.000000}%
\pgfsetstrokecolor{currentstroke}%
\pgfsetdash{}{0pt}%
\pgfpathmoveto{\pgfqpoint{1.025906in}{0.698302in}}%
\pgfpathcurveto{\pgfqpoint{1.036956in}{0.698302in}}{\pgfqpoint{1.047555in}{0.702692in}}{\pgfqpoint{1.055369in}{0.710506in}}%
\pgfpathcurveto{\pgfqpoint{1.063182in}{0.718319in}}{\pgfqpoint{1.067573in}{0.728918in}}{\pgfqpoint{1.067573in}{0.739969in}}%
\pgfpathcurveto{\pgfqpoint{1.067573in}{0.751019in}}{\pgfqpoint{1.063182in}{0.761618in}}{\pgfqpoint{1.055369in}{0.769431in}}%
\pgfpathcurveto{\pgfqpoint{1.047555in}{0.777245in}}{\pgfqpoint{1.036956in}{0.781635in}}{\pgfqpoint{1.025906in}{0.781635in}}%
\pgfpathcurveto{\pgfqpoint{1.014856in}{0.781635in}}{\pgfqpoint{1.004257in}{0.777245in}}{\pgfqpoint{0.996443in}{0.769431in}}%
\pgfpathcurveto{\pgfqpoint{0.988630in}{0.761618in}}{\pgfqpoint{0.984239in}{0.751019in}}{\pgfqpoint{0.984239in}{0.739969in}}%
\pgfpathcurveto{\pgfqpoint{0.984239in}{0.728918in}}{\pgfqpoint{0.988630in}{0.718319in}}{\pgfqpoint{0.996443in}{0.710506in}}%
\pgfpathcurveto{\pgfqpoint{1.004257in}{0.702692in}}{\pgfqpoint{1.014856in}{0.698302in}}{\pgfqpoint{1.025906in}{0.698302in}}%
\pgfpathclose%
\pgfusepath{stroke,fill}%
\end{pgfscope}%
\begin{pgfscope}%
\pgfpathrectangle{\pgfqpoint{0.800000in}{0.528000in}}{\pgfqpoint{4.960000in}{3.696000in}}%
\pgfusepath{clip}%
\pgfsetbuttcap%
\pgfsetroundjoin%
\definecolor{currentfill}{rgb}{0.000000,0.000000,0.000000}%
\pgfsetfillcolor{currentfill}%
\pgfsetlinewidth{1.003750pt}%
\definecolor{currentstroke}{rgb}{0.000000,0.000000,0.000000}%
\pgfsetstrokecolor{currentstroke}%
\pgfsetdash{}{0pt}%
\pgfpathmoveto{\pgfqpoint{1.025906in}{0.698302in}}%
\pgfpathcurveto{\pgfqpoint{1.036956in}{0.698302in}}{\pgfqpoint{1.047555in}{0.702692in}}{\pgfqpoint{1.055369in}{0.710506in}}%
\pgfpathcurveto{\pgfqpoint{1.063182in}{0.718319in}}{\pgfqpoint{1.067573in}{0.728918in}}{\pgfqpoint{1.067573in}{0.739969in}}%
\pgfpathcurveto{\pgfqpoint{1.067573in}{0.751019in}}{\pgfqpoint{1.063182in}{0.761618in}}{\pgfqpoint{1.055369in}{0.769431in}}%
\pgfpathcurveto{\pgfqpoint{1.047555in}{0.777245in}}{\pgfqpoint{1.036956in}{0.781635in}}{\pgfqpoint{1.025906in}{0.781635in}}%
\pgfpathcurveto{\pgfqpoint{1.014856in}{0.781635in}}{\pgfqpoint{1.004257in}{0.777245in}}{\pgfqpoint{0.996443in}{0.769431in}}%
\pgfpathcurveto{\pgfqpoint{0.988630in}{0.761618in}}{\pgfqpoint{0.984239in}{0.751019in}}{\pgfqpoint{0.984239in}{0.739969in}}%
\pgfpathcurveto{\pgfqpoint{0.984239in}{0.728918in}}{\pgfqpoint{0.988630in}{0.718319in}}{\pgfqpoint{0.996443in}{0.710506in}}%
\pgfpathcurveto{\pgfqpoint{1.004257in}{0.702692in}}{\pgfqpoint{1.014856in}{0.698302in}}{\pgfqpoint{1.025906in}{0.698302in}}%
\pgfpathclose%
\pgfusepath{stroke,fill}%
\end{pgfscope}%
\begin{pgfscope}%
\pgfpathrectangle{\pgfqpoint{0.800000in}{0.528000in}}{\pgfqpoint{4.960000in}{3.696000in}}%
\pgfusepath{clip}%
\pgfsetbuttcap%
\pgfsetroundjoin%
\definecolor{currentfill}{rgb}{0.000000,0.000000,0.000000}%
\pgfsetfillcolor{currentfill}%
\pgfsetlinewidth{1.003750pt}%
\definecolor{currentstroke}{rgb}{0.000000,0.000000,0.000000}%
\pgfsetstrokecolor{currentstroke}%
\pgfsetdash{}{0pt}%
\pgfpathmoveto{\pgfqpoint{1.025906in}{0.676804in}}%
\pgfpathcurveto{\pgfqpoint{1.036956in}{0.676804in}}{\pgfqpoint{1.047555in}{0.681195in}}{\pgfqpoint{1.055369in}{0.689008in}}%
\pgfpathcurveto{\pgfqpoint{1.063182in}{0.696822in}}{\pgfqpoint{1.067573in}{0.707421in}}{\pgfqpoint{1.067573in}{0.718471in}}%
\pgfpathcurveto{\pgfqpoint{1.067573in}{0.729521in}}{\pgfqpoint{1.063182in}{0.740120in}}{\pgfqpoint{1.055369in}{0.747934in}}%
\pgfpathcurveto{\pgfqpoint{1.047555in}{0.755748in}}{\pgfqpoint{1.036956in}{0.760138in}}{\pgfqpoint{1.025906in}{0.760138in}}%
\pgfpathcurveto{\pgfqpoint{1.014856in}{0.760138in}}{\pgfqpoint{1.004257in}{0.755748in}}{\pgfqpoint{0.996443in}{0.747934in}}%
\pgfpathcurveto{\pgfqpoint{0.988630in}{0.740120in}}{\pgfqpoint{0.984239in}{0.729521in}}{\pgfqpoint{0.984239in}{0.718471in}}%
\pgfpathcurveto{\pgfqpoint{0.984239in}{0.707421in}}{\pgfqpoint{0.988630in}{0.696822in}}{\pgfqpoint{0.996443in}{0.689008in}}%
\pgfpathcurveto{\pgfqpoint{1.004257in}{0.681195in}}{\pgfqpoint{1.014856in}{0.676804in}}{\pgfqpoint{1.025906in}{0.676804in}}%
\pgfpathclose%
\pgfusepath{stroke,fill}%
\end{pgfscope}%
\begin{pgfscope}%
\pgfpathrectangle{\pgfqpoint{0.800000in}{0.528000in}}{\pgfqpoint{4.960000in}{3.696000in}}%
\pgfusepath{clip}%
\pgfsetbuttcap%
\pgfsetroundjoin%
\definecolor{currentfill}{rgb}{0.000000,0.000000,0.000000}%
\pgfsetfillcolor{currentfill}%
\pgfsetlinewidth{1.003750pt}%
\definecolor{currentstroke}{rgb}{0.000000,0.000000,0.000000}%
\pgfsetstrokecolor{currentstroke}%
\pgfsetdash{}{0pt}%
\pgfpathmoveto{\pgfqpoint{1.025906in}{0.719799in}}%
\pgfpathcurveto{\pgfqpoint{1.036956in}{0.719799in}}{\pgfqpoint{1.047555in}{0.724190in}}{\pgfqpoint{1.055369in}{0.732003in}}%
\pgfpathcurveto{\pgfqpoint{1.063182in}{0.739817in}}{\pgfqpoint{1.067573in}{0.750416in}}{\pgfqpoint{1.067573in}{0.761466in}}%
\pgfpathcurveto{\pgfqpoint{1.067573in}{0.772516in}}{\pgfqpoint{1.063182in}{0.783115in}}{\pgfqpoint{1.055369in}{0.790929in}}%
\pgfpathcurveto{\pgfqpoint{1.047555in}{0.798743in}}{\pgfqpoint{1.036956in}{0.803133in}}{\pgfqpoint{1.025906in}{0.803133in}}%
\pgfpathcurveto{\pgfqpoint{1.014856in}{0.803133in}}{\pgfqpoint{1.004257in}{0.798743in}}{\pgfqpoint{0.996443in}{0.790929in}}%
\pgfpathcurveto{\pgfqpoint{0.988630in}{0.783115in}}{\pgfqpoint{0.984239in}{0.772516in}}{\pgfqpoint{0.984239in}{0.761466in}}%
\pgfpathcurveto{\pgfqpoint{0.984239in}{0.750416in}}{\pgfqpoint{0.988630in}{0.739817in}}{\pgfqpoint{0.996443in}{0.732003in}}%
\pgfpathcurveto{\pgfqpoint{1.004257in}{0.724190in}}{\pgfqpoint{1.014856in}{0.719799in}}{\pgfqpoint{1.025906in}{0.719799in}}%
\pgfpathclose%
\pgfusepath{stroke,fill}%
\end{pgfscope}%
\begin{pgfscope}%
\pgfpathrectangle{\pgfqpoint{0.800000in}{0.528000in}}{\pgfqpoint{4.960000in}{3.696000in}}%
\pgfusepath{clip}%
\pgfsetbuttcap%
\pgfsetroundjoin%
\definecolor{currentfill}{rgb}{0.000000,0.000000,0.000000}%
\pgfsetfillcolor{currentfill}%
\pgfsetlinewidth{1.003750pt}%
\definecolor{currentstroke}{rgb}{0.000000,0.000000,0.000000}%
\pgfsetstrokecolor{currentstroke}%
\pgfsetdash{}{0pt}%
\pgfpathmoveto{\pgfqpoint{1.025906in}{0.719799in}}%
\pgfpathcurveto{\pgfqpoint{1.036956in}{0.719799in}}{\pgfqpoint{1.047555in}{0.724190in}}{\pgfqpoint{1.055369in}{0.732003in}}%
\pgfpathcurveto{\pgfqpoint{1.063182in}{0.739817in}}{\pgfqpoint{1.067573in}{0.750416in}}{\pgfqpoint{1.067573in}{0.761466in}}%
\pgfpathcurveto{\pgfqpoint{1.067573in}{0.772516in}}{\pgfqpoint{1.063182in}{0.783115in}}{\pgfqpoint{1.055369in}{0.790929in}}%
\pgfpathcurveto{\pgfqpoint{1.047555in}{0.798743in}}{\pgfqpoint{1.036956in}{0.803133in}}{\pgfqpoint{1.025906in}{0.803133in}}%
\pgfpathcurveto{\pgfqpoint{1.014856in}{0.803133in}}{\pgfqpoint{1.004257in}{0.798743in}}{\pgfqpoint{0.996443in}{0.790929in}}%
\pgfpathcurveto{\pgfqpoint{0.988630in}{0.783115in}}{\pgfqpoint{0.984239in}{0.772516in}}{\pgfqpoint{0.984239in}{0.761466in}}%
\pgfpathcurveto{\pgfqpoint{0.984239in}{0.750416in}}{\pgfqpoint{0.988630in}{0.739817in}}{\pgfqpoint{0.996443in}{0.732003in}}%
\pgfpathcurveto{\pgfqpoint{1.004257in}{0.724190in}}{\pgfqpoint{1.014856in}{0.719799in}}{\pgfqpoint{1.025906in}{0.719799in}}%
\pgfpathclose%
\pgfusepath{stroke,fill}%
\end{pgfscope}%
\begin{pgfscope}%
\pgfpathrectangle{\pgfqpoint{0.800000in}{0.528000in}}{\pgfqpoint{4.960000in}{3.696000in}}%
\pgfusepath{clip}%
\pgfsetbuttcap%
\pgfsetroundjoin%
\definecolor{currentfill}{rgb}{0.000000,0.000000,0.000000}%
\pgfsetfillcolor{currentfill}%
\pgfsetlinewidth{1.003750pt}%
\definecolor{currentstroke}{rgb}{0.000000,0.000000,0.000000}%
\pgfsetstrokecolor{currentstroke}%
\pgfsetdash{}{0pt}%
\pgfpathmoveto{\pgfqpoint{1.025906in}{0.698302in}}%
\pgfpathcurveto{\pgfqpoint{1.036956in}{0.698302in}}{\pgfqpoint{1.047555in}{0.702692in}}{\pgfqpoint{1.055369in}{0.710506in}}%
\pgfpathcurveto{\pgfqpoint{1.063182in}{0.718319in}}{\pgfqpoint{1.067573in}{0.728918in}}{\pgfqpoint{1.067573in}{0.739969in}}%
\pgfpathcurveto{\pgfqpoint{1.067573in}{0.751019in}}{\pgfqpoint{1.063182in}{0.761618in}}{\pgfqpoint{1.055369in}{0.769431in}}%
\pgfpathcurveto{\pgfqpoint{1.047555in}{0.777245in}}{\pgfqpoint{1.036956in}{0.781635in}}{\pgfqpoint{1.025906in}{0.781635in}}%
\pgfpathcurveto{\pgfqpoint{1.014856in}{0.781635in}}{\pgfqpoint{1.004257in}{0.777245in}}{\pgfqpoint{0.996443in}{0.769431in}}%
\pgfpathcurveto{\pgfqpoint{0.988630in}{0.761618in}}{\pgfqpoint{0.984239in}{0.751019in}}{\pgfqpoint{0.984239in}{0.739969in}}%
\pgfpathcurveto{\pgfqpoint{0.984239in}{0.728918in}}{\pgfqpoint{0.988630in}{0.718319in}}{\pgfqpoint{0.996443in}{0.710506in}}%
\pgfpathcurveto{\pgfqpoint{1.004257in}{0.702692in}}{\pgfqpoint{1.014856in}{0.698302in}}{\pgfqpoint{1.025906in}{0.698302in}}%
\pgfpathclose%
\pgfusepath{stroke,fill}%
\end{pgfscope}%
\begin{pgfscope}%
\pgfpathrectangle{\pgfqpoint{0.800000in}{0.528000in}}{\pgfqpoint{4.960000in}{3.696000in}}%
\pgfusepath{clip}%
\pgfsetbuttcap%
\pgfsetroundjoin%
\definecolor{currentfill}{rgb}{0.000000,0.000000,0.000000}%
\pgfsetfillcolor{currentfill}%
\pgfsetlinewidth{1.003750pt}%
\definecolor{currentstroke}{rgb}{0.000000,0.000000,0.000000}%
\pgfsetstrokecolor{currentstroke}%
\pgfsetdash{}{0pt}%
\pgfpathmoveto{\pgfqpoint{1.025906in}{0.676804in}}%
\pgfpathcurveto{\pgfqpoint{1.036956in}{0.676804in}}{\pgfqpoint{1.047555in}{0.681195in}}{\pgfqpoint{1.055369in}{0.689008in}}%
\pgfpathcurveto{\pgfqpoint{1.063182in}{0.696822in}}{\pgfqpoint{1.067573in}{0.707421in}}{\pgfqpoint{1.067573in}{0.718471in}}%
\pgfpathcurveto{\pgfqpoint{1.067573in}{0.729521in}}{\pgfqpoint{1.063182in}{0.740120in}}{\pgfqpoint{1.055369in}{0.747934in}}%
\pgfpathcurveto{\pgfqpoint{1.047555in}{0.755748in}}{\pgfqpoint{1.036956in}{0.760138in}}{\pgfqpoint{1.025906in}{0.760138in}}%
\pgfpathcurveto{\pgfqpoint{1.014856in}{0.760138in}}{\pgfqpoint{1.004257in}{0.755748in}}{\pgfqpoint{0.996443in}{0.747934in}}%
\pgfpathcurveto{\pgfqpoint{0.988630in}{0.740120in}}{\pgfqpoint{0.984239in}{0.729521in}}{\pgfqpoint{0.984239in}{0.718471in}}%
\pgfpathcurveto{\pgfqpoint{0.984239in}{0.707421in}}{\pgfqpoint{0.988630in}{0.696822in}}{\pgfqpoint{0.996443in}{0.689008in}}%
\pgfpathcurveto{\pgfqpoint{1.004257in}{0.681195in}}{\pgfqpoint{1.014856in}{0.676804in}}{\pgfqpoint{1.025906in}{0.676804in}}%
\pgfpathclose%
\pgfusepath{stroke,fill}%
\end{pgfscope}%
\begin{pgfscope}%
\pgfpathrectangle{\pgfqpoint{0.800000in}{0.528000in}}{\pgfqpoint{4.960000in}{3.696000in}}%
\pgfusepath{clip}%
\pgfsetbuttcap%
\pgfsetroundjoin%
\definecolor{currentfill}{rgb}{0.000000,0.000000,0.000000}%
\pgfsetfillcolor{currentfill}%
\pgfsetlinewidth{1.003750pt}%
\definecolor{currentstroke}{rgb}{0.000000,0.000000,0.000000}%
\pgfsetstrokecolor{currentstroke}%
\pgfsetdash{}{0pt}%
\pgfpathmoveto{\pgfqpoint{1.025906in}{0.698302in}}%
\pgfpathcurveto{\pgfqpoint{1.036956in}{0.698302in}}{\pgfqpoint{1.047555in}{0.702692in}}{\pgfqpoint{1.055369in}{0.710506in}}%
\pgfpathcurveto{\pgfqpoint{1.063182in}{0.718319in}}{\pgfqpoint{1.067573in}{0.728918in}}{\pgfqpoint{1.067573in}{0.739969in}}%
\pgfpathcurveto{\pgfqpoint{1.067573in}{0.751019in}}{\pgfqpoint{1.063182in}{0.761618in}}{\pgfqpoint{1.055369in}{0.769431in}}%
\pgfpathcurveto{\pgfqpoint{1.047555in}{0.777245in}}{\pgfqpoint{1.036956in}{0.781635in}}{\pgfqpoint{1.025906in}{0.781635in}}%
\pgfpathcurveto{\pgfqpoint{1.014856in}{0.781635in}}{\pgfqpoint{1.004257in}{0.777245in}}{\pgfqpoint{0.996443in}{0.769431in}}%
\pgfpathcurveto{\pgfqpoint{0.988630in}{0.761618in}}{\pgfqpoint{0.984239in}{0.751019in}}{\pgfqpoint{0.984239in}{0.739969in}}%
\pgfpathcurveto{\pgfqpoint{0.984239in}{0.728918in}}{\pgfqpoint{0.988630in}{0.718319in}}{\pgfqpoint{0.996443in}{0.710506in}}%
\pgfpathcurveto{\pgfqpoint{1.004257in}{0.702692in}}{\pgfqpoint{1.014856in}{0.698302in}}{\pgfqpoint{1.025906in}{0.698302in}}%
\pgfpathclose%
\pgfusepath{stroke,fill}%
\end{pgfscope}%
\begin{pgfscope}%
\pgfpathrectangle{\pgfqpoint{0.800000in}{0.528000in}}{\pgfqpoint{4.960000in}{3.696000in}}%
\pgfusepath{clip}%
\pgfsetbuttcap%
\pgfsetroundjoin%
\definecolor{currentfill}{rgb}{0.000000,0.000000,0.000000}%
\pgfsetfillcolor{currentfill}%
\pgfsetlinewidth{1.003750pt}%
\definecolor{currentstroke}{rgb}{0.000000,0.000000,0.000000}%
\pgfsetstrokecolor{currentstroke}%
\pgfsetdash{}{0pt}%
\pgfpathmoveto{\pgfqpoint{1.025906in}{0.698302in}}%
\pgfpathcurveto{\pgfqpoint{1.036956in}{0.698302in}}{\pgfqpoint{1.047555in}{0.702692in}}{\pgfqpoint{1.055369in}{0.710506in}}%
\pgfpathcurveto{\pgfqpoint{1.063182in}{0.718319in}}{\pgfqpoint{1.067573in}{0.728918in}}{\pgfqpoint{1.067573in}{0.739969in}}%
\pgfpathcurveto{\pgfqpoint{1.067573in}{0.751019in}}{\pgfqpoint{1.063182in}{0.761618in}}{\pgfqpoint{1.055369in}{0.769431in}}%
\pgfpathcurveto{\pgfqpoint{1.047555in}{0.777245in}}{\pgfqpoint{1.036956in}{0.781635in}}{\pgfqpoint{1.025906in}{0.781635in}}%
\pgfpathcurveto{\pgfqpoint{1.014856in}{0.781635in}}{\pgfqpoint{1.004257in}{0.777245in}}{\pgfqpoint{0.996443in}{0.769431in}}%
\pgfpathcurveto{\pgfqpoint{0.988630in}{0.761618in}}{\pgfqpoint{0.984239in}{0.751019in}}{\pgfqpoint{0.984239in}{0.739969in}}%
\pgfpathcurveto{\pgfqpoint{0.984239in}{0.728918in}}{\pgfqpoint{0.988630in}{0.718319in}}{\pgfqpoint{0.996443in}{0.710506in}}%
\pgfpathcurveto{\pgfqpoint{1.004257in}{0.702692in}}{\pgfqpoint{1.014856in}{0.698302in}}{\pgfqpoint{1.025906in}{0.698302in}}%
\pgfpathclose%
\pgfusepath{stroke,fill}%
\end{pgfscope}%
\begin{pgfscope}%
\pgfpathrectangle{\pgfqpoint{0.800000in}{0.528000in}}{\pgfqpoint{4.960000in}{3.696000in}}%
\pgfusepath{clip}%
\pgfsetbuttcap%
\pgfsetroundjoin%
\definecolor{currentfill}{rgb}{0.000000,0.000000,0.000000}%
\pgfsetfillcolor{currentfill}%
\pgfsetlinewidth{1.003750pt}%
\definecolor{currentstroke}{rgb}{0.000000,0.000000,0.000000}%
\pgfsetstrokecolor{currentstroke}%
\pgfsetdash{}{0pt}%
\pgfpathmoveto{\pgfqpoint{1.025906in}{0.719799in}}%
\pgfpathcurveto{\pgfqpoint{1.036956in}{0.719799in}}{\pgfqpoint{1.047555in}{0.724190in}}{\pgfqpoint{1.055369in}{0.732003in}}%
\pgfpathcurveto{\pgfqpoint{1.063182in}{0.739817in}}{\pgfqpoint{1.067573in}{0.750416in}}{\pgfqpoint{1.067573in}{0.761466in}}%
\pgfpathcurveto{\pgfqpoint{1.067573in}{0.772516in}}{\pgfqpoint{1.063182in}{0.783115in}}{\pgfqpoint{1.055369in}{0.790929in}}%
\pgfpathcurveto{\pgfqpoint{1.047555in}{0.798743in}}{\pgfqpoint{1.036956in}{0.803133in}}{\pgfqpoint{1.025906in}{0.803133in}}%
\pgfpathcurveto{\pgfqpoint{1.014856in}{0.803133in}}{\pgfqpoint{1.004257in}{0.798743in}}{\pgfqpoint{0.996443in}{0.790929in}}%
\pgfpathcurveto{\pgfqpoint{0.988630in}{0.783115in}}{\pgfqpoint{0.984239in}{0.772516in}}{\pgfqpoint{0.984239in}{0.761466in}}%
\pgfpathcurveto{\pgfqpoint{0.984239in}{0.750416in}}{\pgfqpoint{0.988630in}{0.739817in}}{\pgfqpoint{0.996443in}{0.732003in}}%
\pgfpathcurveto{\pgfqpoint{1.004257in}{0.724190in}}{\pgfqpoint{1.014856in}{0.719799in}}{\pgfqpoint{1.025906in}{0.719799in}}%
\pgfpathclose%
\pgfusepath{stroke,fill}%
\end{pgfscope}%
\begin{pgfscope}%
\pgfpathrectangle{\pgfqpoint{0.800000in}{0.528000in}}{\pgfqpoint{4.960000in}{3.696000in}}%
\pgfusepath{clip}%
\pgfsetbuttcap%
\pgfsetroundjoin%
\definecolor{currentfill}{rgb}{0.000000,0.000000,0.000000}%
\pgfsetfillcolor{currentfill}%
\pgfsetlinewidth{1.003750pt}%
\definecolor{currentstroke}{rgb}{0.000000,0.000000,0.000000}%
\pgfsetstrokecolor{currentstroke}%
\pgfsetdash{}{0pt}%
\pgfpathmoveto{\pgfqpoint{1.025906in}{0.698302in}}%
\pgfpathcurveto{\pgfqpoint{1.036956in}{0.698302in}}{\pgfqpoint{1.047555in}{0.702692in}}{\pgfqpoint{1.055369in}{0.710506in}}%
\pgfpathcurveto{\pgfqpoint{1.063182in}{0.718319in}}{\pgfqpoint{1.067573in}{0.728918in}}{\pgfqpoint{1.067573in}{0.739969in}}%
\pgfpathcurveto{\pgfqpoint{1.067573in}{0.751019in}}{\pgfqpoint{1.063182in}{0.761618in}}{\pgfqpoint{1.055369in}{0.769431in}}%
\pgfpathcurveto{\pgfqpoint{1.047555in}{0.777245in}}{\pgfqpoint{1.036956in}{0.781635in}}{\pgfqpoint{1.025906in}{0.781635in}}%
\pgfpathcurveto{\pgfqpoint{1.014856in}{0.781635in}}{\pgfqpoint{1.004257in}{0.777245in}}{\pgfqpoint{0.996443in}{0.769431in}}%
\pgfpathcurveto{\pgfqpoint{0.988630in}{0.761618in}}{\pgfqpoint{0.984239in}{0.751019in}}{\pgfqpoint{0.984239in}{0.739969in}}%
\pgfpathcurveto{\pgfqpoint{0.984239in}{0.728918in}}{\pgfqpoint{0.988630in}{0.718319in}}{\pgfqpoint{0.996443in}{0.710506in}}%
\pgfpathcurveto{\pgfqpoint{1.004257in}{0.702692in}}{\pgfqpoint{1.014856in}{0.698302in}}{\pgfqpoint{1.025906in}{0.698302in}}%
\pgfpathclose%
\pgfusepath{stroke,fill}%
\end{pgfscope}%
\begin{pgfscope}%
\pgfpathrectangle{\pgfqpoint{0.800000in}{0.528000in}}{\pgfqpoint{4.960000in}{3.696000in}}%
\pgfusepath{clip}%
\pgfsetbuttcap%
\pgfsetroundjoin%
\definecolor{currentfill}{rgb}{0.000000,0.000000,0.000000}%
\pgfsetfillcolor{currentfill}%
\pgfsetlinewidth{1.003750pt}%
\definecolor{currentstroke}{rgb}{0.000000,0.000000,0.000000}%
\pgfsetstrokecolor{currentstroke}%
\pgfsetdash{}{0pt}%
\pgfpathmoveto{\pgfqpoint{1.025906in}{0.655307in}}%
\pgfpathcurveto{\pgfqpoint{1.036956in}{0.655307in}}{\pgfqpoint{1.047555in}{0.659697in}}{\pgfqpoint{1.055369in}{0.667511in}}%
\pgfpathcurveto{\pgfqpoint{1.063182in}{0.675324in}}{\pgfqpoint{1.067573in}{0.685924in}}{\pgfqpoint{1.067573in}{0.696974in}}%
\pgfpathcurveto{\pgfqpoint{1.067573in}{0.708024in}}{\pgfqpoint{1.063182in}{0.718623in}}{\pgfqpoint{1.055369in}{0.726436in}}%
\pgfpathcurveto{\pgfqpoint{1.047555in}{0.734250in}}{\pgfqpoint{1.036956in}{0.738640in}}{\pgfqpoint{1.025906in}{0.738640in}}%
\pgfpathcurveto{\pgfqpoint{1.014856in}{0.738640in}}{\pgfqpoint{1.004257in}{0.734250in}}{\pgfqpoint{0.996443in}{0.726436in}}%
\pgfpathcurveto{\pgfqpoint{0.988630in}{0.718623in}}{\pgfqpoint{0.984239in}{0.708024in}}{\pgfqpoint{0.984239in}{0.696974in}}%
\pgfpathcurveto{\pgfqpoint{0.984239in}{0.685924in}}{\pgfqpoint{0.988630in}{0.675324in}}{\pgfqpoint{0.996443in}{0.667511in}}%
\pgfpathcurveto{\pgfqpoint{1.004257in}{0.659697in}}{\pgfqpoint{1.014856in}{0.655307in}}{\pgfqpoint{1.025906in}{0.655307in}}%
\pgfpathclose%
\pgfusepath{stroke,fill}%
\end{pgfscope}%
\begin{pgfscope}%
\pgfpathrectangle{\pgfqpoint{0.800000in}{0.528000in}}{\pgfqpoint{4.960000in}{3.696000in}}%
\pgfusepath{clip}%
\pgfsetbuttcap%
\pgfsetroundjoin%
\definecolor{currentfill}{rgb}{0.000000,0.000000,0.000000}%
\pgfsetfillcolor{currentfill}%
\pgfsetlinewidth{1.003750pt}%
\definecolor{currentstroke}{rgb}{0.000000,0.000000,0.000000}%
\pgfsetstrokecolor{currentstroke}%
\pgfsetdash{}{0pt}%
\pgfpathmoveto{\pgfqpoint{1.025906in}{0.676804in}}%
\pgfpathcurveto{\pgfqpoint{1.036956in}{0.676804in}}{\pgfqpoint{1.047555in}{0.681195in}}{\pgfqpoint{1.055369in}{0.689008in}}%
\pgfpathcurveto{\pgfqpoint{1.063182in}{0.696822in}}{\pgfqpoint{1.067573in}{0.707421in}}{\pgfqpoint{1.067573in}{0.718471in}}%
\pgfpathcurveto{\pgfqpoint{1.067573in}{0.729521in}}{\pgfqpoint{1.063182in}{0.740120in}}{\pgfqpoint{1.055369in}{0.747934in}}%
\pgfpathcurveto{\pgfqpoint{1.047555in}{0.755748in}}{\pgfqpoint{1.036956in}{0.760138in}}{\pgfqpoint{1.025906in}{0.760138in}}%
\pgfpathcurveto{\pgfqpoint{1.014856in}{0.760138in}}{\pgfqpoint{1.004257in}{0.755748in}}{\pgfqpoint{0.996443in}{0.747934in}}%
\pgfpathcurveto{\pgfqpoint{0.988630in}{0.740120in}}{\pgfqpoint{0.984239in}{0.729521in}}{\pgfqpoint{0.984239in}{0.718471in}}%
\pgfpathcurveto{\pgfqpoint{0.984239in}{0.707421in}}{\pgfqpoint{0.988630in}{0.696822in}}{\pgfqpoint{0.996443in}{0.689008in}}%
\pgfpathcurveto{\pgfqpoint{1.004257in}{0.681195in}}{\pgfqpoint{1.014856in}{0.676804in}}{\pgfqpoint{1.025906in}{0.676804in}}%
\pgfpathclose%
\pgfusepath{stroke,fill}%
\end{pgfscope}%
\begin{pgfscope}%
\pgfpathrectangle{\pgfqpoint{0.800000in}{0.528000in}}{\pgfqpoint{4.960000in}{3.696000in}}%
\pgfusepath{clip}%
\pgfsetbuttcap%
\pgfsetroundjoin%
\definecolor{currentfill}{rgb}{0.000000,0.000000,0.000000}%
\pgfsetfillcolor{currentfill}%
\pgfsetlinewidth{1.003750pt}%
\definecolor{currentstroke}{rgb}{0.000000,0.000000,0.000000}%
\pgfsetstrokecolor{currentstroke}%
\pgfsetdash{}{0pt}%
\pgfpathmoveto{\pgfqpoint{1.025906in}{0.698302in}}%
\pgfpathcurveto{\pgfqpoint{1.036956in}{0.698302in}}{\pgfqpoint{1.047555in}{0.702692in}}{\pgfqpoint{1.055369in}{0.710506in}}%
\pgfpathcurveto{\pgfqpoint{1.063182in}{0.718319in}}{\pgfqpoint{1.067573in}{0.728918in}}{\pgfqpoint{1.067573in}{0.739969in}}%
\pgfpathcurveto{\pgfqpoint{1.067573in}{0.751019in}}{\pgfqpoint{1.063182in}{0.761618in}}{\pgfqpoint{1.055369in}{0.769431in}}%
\pgfpathcurveto{\pgfqpoint{1.047555in}{0.777245in}}{\pgfqpoint{1.036956in}{0.781635in}}{\pgfqpoint{1.025906in}{0.781635in}}%
\pgfpathcurveto{\pgfqpoint{1.014856in}{0.781635in}}{\pgfqpoint{1.004257in}{0.777245in}}{\pgfqpoint{0.996443in}{0.769431in}}%
\pgfpathcurveto{\pgfqpoint{0.988630in}{0.761618in}}{\pgfqpoint{0.984239in}{0.751019in}}{\pgfqpoint{0.984239in}{0.739969in}}%
\pgfpathcurveto{\pgfqpoint{0.984239in}{0.728918in}}{\pgfqpoint{0.988630in}{0.718319in}}{\pgfqpoint{0.996443in}{0.710506in}}%
\pgfpathcurveto{\pgfqpoint{1.004257in}{0.702692in}}{\pgfqpoint{1.014856in}{0.698302in}}{\pgfqpoint{1.025906in}{0.698302in}}%
\pgfpathclose%
\pgfusepath{stroke,fill}%
\end{pgfscope}%
\begin{pgfscope}%
\pgfpathrectangle{\pgfqpoint{0.800000in}{0.528000in}}{\pgfqpoint{4.960000in}{3.696000in}}%
\pgfusepath{clip}%
\pgfsetbuttcap%
\pgfsetroundjoin%
\definecolor{currentfill}{rgb}{0.000000,0.000000,0.000000}%
\pgfsetfillcolor{currentfill}%
\pgfsetlinewidth{1.003750pt}%
\definecolor{currentstroke}{rgb}{0.000000,0.000000,0.000000}%
\pgfsetstrokecolor{currentstroke}%
\pgfsetdash{}{0pt}%
\pgfpathmoveto{\pgfqpoint{1.025906in}{0.698302in}}%
\pgfpathcurveto{\pgfqpoint{1.036956in}{0.698302in}}{\pgfqpoint{1.047555in}{0.702692in}}{\pgfqpoint{1.055369in}{0.710506in}}%
\pgfpathcurveto{\pgfqpoint{1.063182in}{0.718319in}}{\pgfqpoint{1.067573in}{0.728918in}}{\pgfqpoint{1.067573in}{0.739969in}}%
\pgfpathcurveto{\pgfqpoint{1.067573in}{0.751019in}}{\pgfqpoint{1.063182in}{0.761618in}}{\pgfqpoint{1.055369in}{0.769431in}}%
\pgfpathcurveto{\pgfqpoint{1.047555in}{0.777245in}}{\pgfqpoint{1.036956in}{0.781635in}}{\pgfqpoint{1.025906in}{0.781635in}}%
\pgfpathcurveto{\pgfqpoint{1.014856in}{0.781635in}}{\pgfqpoint{1.004257in}{0.777245in}}{\pgfqpoint{0.996443in}{0.769431in}}%
\pgfpathcurveto{\pgfqpoint{0.988630in}{0.761618in}}{\pgfqpoint{0.984239in}{0.751019in}}{\pgfqpoint{0.984239in}{0.739969in}}%
\pgfpathcurveto{\pgfqpoint{0.984239in}{0.728918in}}{\pgfqpoint{0.988630in}{0.718319in}}{\pgfqpoint{0.996443in}{0.710506in}}%
\pgfpathcurveto{\pgfqpoint{1.004257in}{0.702692in}}{\pgfqpoint{1.014856in}{0.698302in}}{\pgfqpoint{1.025906in}{0.698302in}}%
\pgfpathclose%
\pgfusepath{stroke,fill}%
\end{pgfscope}%
\begin{pgfscope}%
\pgfpathrectangle{\pgfqpoint{0.800000in}{0.528000in}}{\pgfqpoint{4.960000in}{3.696000in}}%
\pgfusepath{clip}%
\pgfsetbuttcap%
\pgfsetroundjoin%
\definecolor{currentfill}{rgb}{0.000000,0.000000,0.000000}%
\pgfsetfillcolor{currentfill}%
\pgfsetlinewidth{1.003750pt}%
\definecolor{currentstroke}{rgb}{0.000000,0.000000,0.000000}%
\pgfsetstrokecolor{currentstroke}%
\pgfsetdash{}{0pt}%
\pgfpathmoveto{\pgfqpoint{1.025906in}{0.719799in}}%
\pgfpathcurveto{\pgfqpoint{1.036956in}{0.719799in}}{\pgfqpoint{1.047555in}{0.724190in}}{\pgfqpoint{1.055369in}{0.732003in}}%
\pgfpathcurveto{\pgfqpoint{1.063182in}{0.739817in}}{\pgfqpoint{1.067573in}{0.750416in}}{\pgfqpoint{1.067573in}{0.761466in}}%
\pgfpathcurveto{\pgfqpoint{1.067573in}{0.772516in}}{\pgfqpoint{1.063182in}{0.783115in}}{\pgfqpoint{1.055369in}{0.790929in}}%
\pgfpathcurveto{\pgfqpoint{1.047555in}{0.798743in}}{\pgfqpoint{1.036956in}{0.803133in}}{\pgfqpoint{1.025906in}{0.803133in}}%
\pgfpathcurveto{\pgfqpoint{1.014856in}{0.803133in}}{\pgfqpoint{1.004257in}{0.798743in}}{\pgfqpoint{0.996443in}{0.790929in}}%
\pgfpathcurveto{\pgfqpoint{0.988630in}{0.783115in}}{\pgfqpoint{0.984239in}{0.772516in}}{\pgfqpoint{0.984239in}{0.761466in}}%
\pgfpathcurveto{\pgfqpoint{0.984239in}{0.750416in}}{\pgfqpoint{0.988630in}{0.739817in}}{\pgfqpoint{0.996443in}{0.732003in}}%
\pgfpathcurveto{\pgfqpoint{1.004257in}{0.724190in}}{\pgfqpoint{1.014856in}{0.719799in}}{\pgfqpoint{1.025906in}{0.719799in}}%
\pgfpathclose%
\pgfusepath{stroke,fill}%
\end{pgfscope}%
\begin{pgfscope}%
\pgfpathrectangle{\pgfqpoint{0.800000in}{0.528000in}}{\pgfqpoint{4.960000in}{3.696000in}}%
\pgfusepath{clip}%
\pgfsetbuttcap%
\pgfsetroundjoin%
\definecolor{currentfill}{rgb}{0.000000,0.000000,0.000000}%
\pgfsetfillcolor{currentfill}%
\pgfsetlinewidth{1.003750pt}%
\definecolor{currentstroke}{rgb}{0.000000,0.000000,0.000000}%
\pgfsetstrokecolor{currentstroke}%
\pgfsetdash{}{0pt}%
\pgfpathmoveto{\pgfqpoint{1.025906in}{0.698302in}}%
\pgfpathcurveto{\pgfqpoint{1.036956in}{0.698302in}}{\pgfqpoint{1.047555in}{0.702692in}}{\pgfqpoint{1.055369in}{0.710506in}}%
\pgfpathcurveto{\pgfqpoint{1.063182in}{0.718319in}}{\pgfqpoint{1.067573in}{0.728918in}}{\pgfqpoint{1.067573in}{0.739969in}}%
\pgfpathcurveto{\pgfqpoint{1.067573in}{0.751019in}}{\pgfqpoint{1.063182in}{0.761618in}}{\pgfqpoint{1.055369in}{0.769431in}}%
\pgfpathcurveto{\pgfqpoint{1.047555in}{0.777245in}}{\pgfqpoint{1.036956in}{0.781635in}}{\pgfqpoint{1.025906in}{0.781635in}}%
\pgfpathcurveto{\pgfqpoint{1.014856in}{0.781635in}}{\pgfqpoint{1.004257in}{0.777245in}}{\pgfqpoint{0.996443in}{0.769431in}}%
\pgfpathcurveto{\pgfqpoint{0.988630in}{0.761618in}}{\pgfqpoint{0.984239in}{0.751019in}}{\pgfqpoint{0.984239in}{0.739969in}}%
\pgfpathcurveto{\pgfqpoint{0.984239in}{0.728918in}}{\pgfqpoint{0.988630in}{0.718319in}}{\pgfqpoint{0.996443in}{0.710506in}}%
\pgfpathcurveto{\pgfqpoint{1.004257in}{0.702692in}}{\pgfqpoint{1.014856in}{0.698302in}}{\pgfqpoint{1.025906in}{0.698302in}}%
\pgfpathclose%
\pgfusepath{stroke,fill}%
\end{pgfscope}%
\begin{pgfscope}%
\pgfpathrectangle{\pgfqpoint{0.800000in}{0.528000in}}{\pgfqpoint{4.960000in}{3.696000in}}%
\pgfusepath{clip}%
\pgfsetbuttcap%
\pgfsetroundjoin%
\definecolor{currentfill}{rgb}{0.000000,0.000000,0.000000}%
\pgfsetfillcolor{currentfill}%
\pgfsetlinewidth{1.003750pt}%
\definecolor{currentstroke}{rgb}{0.000000,0.000000,0.000000}%
\pgfsetstrokecolor{currentstroke}%
\pgfsetdash{}{0pt}%
\pgfpathmoveto{\pgfqpoint{1.025906in}{0.719799in}}%
\pgfpathcurveto{\pgfqpoint{1.036956in}{0.719799in}}{\pgfqpoint{1.047555in}{0.724190in}}{\pgfqpoint{1.055369in}{0.732003in}}%
\pgfpathcurveto{\pgfqpoint{1.063182in}{0.739817in}}{\pgfqpoint{1.067573in}{0.750416in}}{\pgfqpoint{1.067573in}{0.761466in}}%
\pgfpathcurveto{\pgfqpoint{1.067573in}{0.772516in}}{\pgfqpoint{1.063182in}{0.783115in}}{\pgfqpoint{1.055369in}{0.790929in}}%
\pgfpathcurveto{\pgfqpoint{1.047555in}{0.798743in}}{\pgfqpoint{1.036956in}{0.803133in}}{\pgfqpoint{1.025906in}{0.803133in}}%
\pgfpathcurveto{\pgfqpoint{1.014856in}{0.803133in}}{\pgfqpoint{1.004257in}{0.798743in}}{\pgfqpoint{0.996443in}{0.790929in}}%
\pgfpathcurveto{\pgfqpoint{0.988630in}{0.783115in}}{\pgfqpoint{0.984239in}{0.772516in}}{\pgfqpoint{0.984239in}{0.761466in}}%
\pgfpathcurveto{\pgfqpoint{0.984239in}{0.750416in}}{\pgfqpoint{0.988630in}{0.739817in}}{\pgfqpoint{0.996443in}{0.732003in}}%
\pgfpathcurveto{\pgfqpoint{1.004257in}{0.724190in}}{\pgfqpoint{1.014856in}{0.719799in}}{\pgfqpoint{1.025906in}{0.719799in}}%
\pgfpathclose%
\pgfusepath{stroke,fill}%
\end{pgfscope}%
\begin{pgfscope}%
\pgfpathrectangle{\pgfqpoint{0.800000in}{0.528000in}}{\pgfqpoint{4.960000in}{3.696000in}}%
\pgfusepath{clip}%
\pgfsetbuttcap%
\pgfsetroundjoin%
\definecolor{currentfill}{rgb}{0.000000,0.000000,0.000000}%
\pgfsetfillcolor{currentfill}%
\pgfsetlinewidth{1.003750pt}%
\definecolor{currentstroke}{rgb}{0.000000,0.000000,0.000000}%
\pgfsetstrokecolor{currentstroke}%
\pgfsetdash{}{0pt}%
\pgfpathmoveto{\pgfqpoint{1.025906in}{0.698302in}}%
\pgfpathcurveto{\pgfqpoint{1.036956in}{0.698302in}}{\pgfqpoint{1.047555in}{0.702692in}}{\pgfqpoint{1.055369in}{0.710506in}}%
\pgfpathcurveto{\pgfqpoint{1.063182in}{0.718319in}}{\pgfqpoint{1.067573in}{0.728918in}}{\pgfqpoint{1.067573in}{0.739969in}}%
\pgfpathcurveto{\pgfqpoint{1.067573in}{0.751019in}}{\pgfqpoint{1.063182in}{0.761618in}}{\pgfqpoint{1.055369in}{0.769431in}}%
\pgfpathcurveto{\pgfqpoint{1.047555in}{0.777245in}}{\pgfqpoint{1.036956in}{0.781635in}}{\pgfqpoint{1.025906in}{0.781635in}}%
\pgfpathcurveto{\pgfqpoint{1.014856in}{0.781635in}}{\pgfqpoint{1.004257in}{0.777245in}}{\pgfqpoint{0.996443in}{0.769431in}}%
\pgfpathcurveto{\pgfqpoint{0.988630in}{0.761618in}}{\pgfqpoint{0.984239in}{0.751019in}}{\pgfqpoint{0.984239in}{0.739969in}}%
\pgfpathcurveto{\pgfqpoint{0.984239in}{0.728918in}}{\pgfqpoint{0.988630in}{0.718319in}}{\pgfqpoint{0.996443in}{0.710506in}}%
\pgfpathcurveto{\pgfqpoint{1.004257in}{0.702692in}}{\pgfqpoint{1.014856in}{0.698302in}}{\pgfqpoint{1.025906in}{0.698302in}}%
\pgfpathclose%
\pgfusepath{stroke,fill}%
\end{pgfscope}%
\begin{pgfscope}%
\pgfpathrectangle{\pgfqpoint{0.800000in}{0.528000in}}{\pgfqpoint{4.960000in}{3.696000in}}%
\pgfusepath{clip}%
\pgfsetbuttcap%
\pgfsetroundjoin%
\definecolor{currentfill}{rgb}{0.000000,0.000000,0.000000}%
\pgfsetfillcolor{currentfill}%
\pgfsetlinewidth{1.003750pt}%
\definecolor{currentstroke}{rgb}{0.000000,0.000000,0.000000}%
\pgfsetstrokecolor{currentstroke}%
\pgfsetdash{}{0pt}%
\pgfpathmoveto{\pgfqpoint{1.025906in}{0.698302in}}%
\pgfpathcurveto{\pgfqpoint{1.036956in}{0.698302in}}{\pgfqpoint{1.047555in}{0.702692in}}{\pgfqpoint{1.055369in}{0.710506in}}%
\pgfpathcurveto{\pgfqpoint{1.063182in}{0.718319in}}{\pgfqpoint{1.067573in}{0.728918in}}{\pgfqpoint{1.067573in}{0.739969in}}%
\pgfpathcurveto{\pgfqpoint{1.067573in}{0.751019in}}{\pgfqpoint{1.063182in}{0.761618in}}{\pgfqpoint{1.055369in}{0.769431in}}%
\pgfpathcurveto{\pgfqpoint{1.047555in}{0.777245in}}{\pgfqpoint{1.036956in}{0.781635in}}{\pgfqpoint{1.025906in}{0.781635in}}%
\pgfpathcurveto{\pgfqpoint{1.014856in}{0.781635in}}{\pgfqpoint{1.004257in}{0.777245in}}{\pgfqpoint{0.996443in}{0.769431in}}%
\pgfpathcurveto{\pgfqpoint{0.988630in}{0.761618in}}{\pgfqpoint{0.984239in}{0.751019in}}{\pgfqpoint{0.984239in}{0.739969in}}%
\pgfpathcurveto{\pgfqpoint{0.984239in}{0.728918in}}{\pgfqpoint{0.988630in}{0.718319in}}{\pgfqpoint{0.996443in}{0.710506in}}%
\pgfpathcurveto{\pgfqpoint{1.004257in}{0.702692in}}{\pgfqpoint{1.014856in}{0.698302in}}{\pgfqpoint{1.025906in}{0.698302in}}%
\pgfpathclose%
\pgfusepath{stroke,fill}%
\end{pgfscope}%
\begin{pgfscope}%
\pgfpathrectangle{\pgfqpoint{0.800000in}{0.528000in}}{\pgfqpoint{4.960000in}{3.696000in}}%
\pgfusepath{clip}%
\pgfsetbuttcap%
\pgfsetroundjoin%
\definecolor{currentfill}{rgb}{0.000000,0.000000,0.000000}%
\pgfsetfillcolor{currentfill}%
\pgfsetlinewidth{1.003750pt}%
\definecolor{currentstroke}{rgb}{0.000000,0.000000,0.000000}%
\pgfsetstrokecolor{currentstroke}%
\pgfsetdash{}{0pt}%
\pgfpathmoveto{\pgfqpoint{1.025906in}{0.719799in}}%
\pgfpathcurveto{\pgfqpoint{1.036956in}{0.719799in}}{\pgfqpoint{1.047555in}{0.724190in}}{\pgfqpoint{1.055369in}{0.732003in}}%
\pgfpathcurveto{\pgfqpoint{1.063182in}{0.739817in}}{\pgfqpoint{1.067573in}{0.750416in}}{\pgfqpoint{1.067573in}{0.761466in}}%
\pgfpathcurveto{\pgfqpoint{1.067573in}{0.772516in}}{\pgfqpoint{1.063182in}{0.783115in}}{\pgfqpoint{1.055369in}{0.790929in}}%
\pgfpathcurveto{\pgfqpoint{1.047555in}{0.798743in}}{\pgfqpoint{1.036956in}{0.803133in}}{\pgfqpoint{1.025906in}{0.803133in}}%
\pgfpathcurveto{\pgfqpoint{1.014856in}{0.803133in}}{\pgfqpoint{1.004257in}{0.798743in}}{\pgfqpoint{0.996443in}{0.790929in}}%
\pgfpathcurveto{\pgfqpoint{0.988630in}{0.783115in}}{\pgfqpoint{0.984239in}{0.772516in}}{\pgfqpoint{0.984239in}{0.761466in}}%
\pgfpathcurveto{\pgfqpoint{0.984239in}{0.750416in}}{\pgfqpoint{0.988630in}{0.739817in}}{\pgfqpoint{0.996443in}{0.732003in}}%
\pgfpathcurveto{\pgfqpoint{1.004257in}{0.724190in}}{\pgfqpoint{1.014856in}{0.719799in}}{\pgfqpoint{1.025906in}{0.719799in}}%
\pgfpathclose%
\pgfusepath{stroke,fill}%
\end{pgfscope}%
\begin{pgfscope}%
\pgfpathrectangle{\pgfqpoint{0.800000in}{0.528000in}}{\pgfqpoint{4.960000in}{3.696000in}}%
\pgfusepath{clip}%
\pgfsetbuttcap%
\pgfsetroundjoin%
\definecolor{currentfill}{rgb}{0.000000,0.000000,0.000000}%
\pgfsetfillcolor{currentfill}%
\pgfsetlinewidth{1.003750pt}%
\definecolor{currentstroke}{rgb}{0.000000,0.000000,0.000000}%
\pgfsetstrokecolor{currentstroke}%
\pgfsetdash{}{0pt}%
\pgfpathmoveto{\pgfqpoint{1.025906in}{0.698302in}}%
\pgfpathcurveto{\pgfqpoint{1.036956in}{0.698302in}}{\pgfqpoint{1.047555in}{0.702692in}}{\pgfqpoint{1.055369in}{0.710506in}}%
\pgfpathcurveto{\pgfqpoint{1.063182in}{0.718319in}}{\pgfqpoint{1.067573in}{0.728918in}}{\pgfqpoint{1.067573in}{0.739969in}}%
\pgfpathcurveto{\pgfqpoint{1.067573in}{0.751019in}}{\pgfqpoint{1.063182in}{0.761618in}}{\pgfqpoint{1.055369in}{0.769431in}}%
\pgfpathcurveto{\pgfqpoint{1.047555in}{0.777245in}}{\pgfqpoint{1.036956in}{0.781635in}}{\pgfqpoint{1.025906in}{0.781635in}}%
\pgfpathcurveto{\pgfqpoint{1.014856in}{0.781635in}}{\pgfqpoint{1.004257in}{0.777245in}}{\pgfqpoint{0.996443in}{0.769431in}}%
\pgfpathcurveto{\pgfqpoint{0.988630in}{0.761618in}}{\pgfqpoint{0.984239in}{0.751019in}}{\pgfqpoint{0.984239in}{0.739969in}}%
\pgfpathcurveto{\pgfqpoint{0.984239in}{0.728918in}}{\pgfqpoint{0.988630in}{0.718319in}}{\pgfqpoint{0.996443in}{0.710506in}}%
\pgfpathcurveto{\pgfqpoint{1.004257in}{0.702692in}}{\pgfqpoint{1.014856in}{0.698302in}}{\pgfqpoint{1.025906in}{0.698302in}}%
\pgfpathclose%
\pgfusepath{stroke,fill}%
\end{pgfscope}%
\begin{pgfscope}%
\pgfpathrectangle{\pgfqpoint{0.800000in}{0.528000in}}{\pgfqpoint{4.960000in}{3.696000in}}%
\pgfusepath{clip}%
\pgfsetbuttcap%
\pgfsetroundjoin%
\definecolor{currentfill}{rgb}{0.000000,0.000000,0.000000}%
\pgfsetfillcolor{currentfill}%
\pgfsetlinewidth{1.003750pt}%
\definecolor{currentstroke}{rgb}{0.000000,0.000000,0.000000}%
\pgfsetstrokecolor{currentstroke}%
\pgfsetdash{}{0pt}%
\pgfpathmoveto{\pgfqpoint{1.025906in}{0.676804in}}%
\pgfpathcurveto{\pgfqpoint{1.036956in}{0.676804in}}{\pgfqpoint{1.047555in}{0.681195in}}{\pgfqpoint{1.055369in}{0.689008in}}%
\pgfpathcurveto{\pgfqpoint{1.063182in}{0.696822in}}{\pgfqpoint{1.067573in}{0.707421in}}{\pgfqpoint{1.067573in}{0.718471in}}%
\pgfpathcurveto{\pgfqpoint{1.067573in}{0.729521in}}{\pgfqpoint{1.063182in}{0.740120in}}{\pgfqpoint{1.055369in}{0.747934in}}%
\pgfpathcurveto{\pgfqpoint{1.047555in}{0.755748in}}{\pgfqpoint{1.036956in}{0.760138in}}{\pgfqpoint{1.025906in}{0.760138in}}%
\pgfpathcurveto{\pgfqpoint{1.014856in}{0.760138in}}{\pgfqpoint{1.004257in}{0.755748in}}{\pgfqpoint{0.996443in}{0.747934in}}%
\pgfpathcurveto{\pgfqpoint{0.988630in}{0.740120in}}{\pgfqpoint{0.984239in}{0.729521in}}{\pgfqpoint{0.984239in}{0.718471in}}%
\pgfpathcurveto{\pgfqpoint{0.984239in}{0.707421in}}{\pgfqpoint{0.988630in}{0.696822in}}{\pgfqpoint{0.996443in}{0.689008in}}%
\pgfpathcurveto{\pgfqpoint{1.004257in}{0.681195in}}{\pgfqpoint{1.014856in}{0.676804in}}{\pgfqpoint{1.025906in}{0.676804in}}%
\pgfpathclose%
\pgfusepath{stroke,fill}%
\end{pgfscope}%
\begin{pgfscope}%
\pgfpathrectangle{\pgfqpoint{0.800000in}{0.528000in}}{\pgfqpoint{4.960000in}{3.696000in}}%
\pgfusepath{clip}%
\pgfsetbuttcap%
\pgfsetroundjoin%
\definecolor{currentfill}{rgb}{0.000000,0.000000,0.000000}%
\pgfsetfillcolor{currentfill}%
\pgfsetlinewidth{1.003750pt}%
\definecolor{currentstroke}{rgb}{0.000000,0.000000,0.000000}%
\pgfsetstrokecolor{currentstroke}%
\pgfsetdash{}{0pt}%
\pgfpathmoveto{\pgfqpoint{1.025906in}{0.698302in}}%
\pgfpathcurveto{\pgfqpoint{1.036956in}{0.698302in}}{\pgfqpoint{1.047555in}{0.702692in}}{\pgfqpoint{1.055369in}{0.710506in}}%
\pgfpathcurveto{\pgfqpoint{1.063182in}{0.718319in}}{\pgfqpoint{1.067573in}{0.728918in}}{\pgfqpoint{1.067573in}{0.739969in}}%
\pgfpathcurveto{\pgfqpoint{1.067573in}{0.751019in}}{\pgfqpoint{1.063182in}{0.761618in}}{\pgfqpoint{1.055369in}{0.769431in}}%
\pgfpathcurveto{\pgfqpoint{1.047555in}{0.777245in}}{\pgfqpoint{1.036956in}{0.781635in}}{\pgfqpoint{1.025906in}{0.781635in}}%
\pgfpathcurveto{\pgfqpoint{1.014856in}{0.781635in}}{\pgfqpoint{1.004257in}{0.777245in}}{\pgfqpoint{0.996443in}{0.769431in}}%
\pgfpathcurveto{\pgfqpoint{0.988630in}{0.761618in}}{\pgfqpoint{0.984239in}{0.751019in}}{\pgfqpoint{0.984239in}{0.739969in}}%
\pgfpathcurveto{\pgfqpoint{0.984239in}{0.728918in}}{\pgfqpoint{0.988630in}{0.718319in}}{\pgfqpoint{0.996443in}{0.710506in}}%
\pgfpathcurveto{\pgfqpoint{1.004257in}{0.702692in}}{\pgfqpoint{1.014856in}{0.698302in}}{\pgfqpoint{1.025906in}{0.698302in}}%
\pgfpathclose%
\pgfusepath{stroke,fill}%
\end{pgfscope}%
\begin{pgfscope}%
\pgfpathrectangle{\pgfqpoint{0.800000in}{0.528000in}}{\pgfqpoint{4.960000in}{3.696000in}}%
\pgfusepath{clip}%
\pgfsetbuttcap%
\pgfsetroundjoin%
\definecolor{currentfill}{rgb}{0.000000,0.000000,0.000000}%
\pgfsetfillcolor{currentfill}%
\pgfsetlinewidth{1.003750pt}%
\definecolor{currentstroke}{rgb}{0.000000,0.000000,0.000000}%
\pgfsetstrokecolor{currentstroke}%
\pgfsetdash{}{0pt}%
\pgfpathmoveto{\pgfqpoint{1.025906in}{0.698302in}}%
\pgfpathcurveto{\pgfqpoint{1.036956in}{0.698302in}}{\pgfqpoint{1.047555in}{0.702692in}}{\pgfqpoint{1.055369in}{0.710506in}}%
\pgfpathcurveto{\pgfqpoint{1.063182in}{0.718319in}}{\pgfqpoint{1.067573in}{0.728918in}}{\pgfqpoint{1.067573in}{0.739969in}}%
\pgfpathcurveto{\pgfqpoint{1.067573in}{0.751019in}}{\pgfqpoint{1.063182in}{0.761618in}}{\pgfqpoint{1.055369in}{0.769431in}}%
\pgfpathcurveto{\pgfqpoint{1.047555in}{0.777245in}}{\pgfqpoint{1.036956in}{0.781635in}}{\pgfqpoint{1.025906in}{0.781635in}}%
\pgfpathcurveto{\pgfqpoint{1.014856in}{0.781635in}}{\pgfqpoint{1.004257in}{0.777245in}}{\pgfqpoint{0.996443in}{0.769431in}}%
\pgfpathcurveto{\pgfqpoint{0.988630in}{0.761618in}}{\pgfqpoint{0.984239in}{0.751019in}}{\pgfqpoint{0.984239in}{0.739969in}}%
\pgfpathcurveto{\pgfqpoint{0.984239in}{0.728918in}}{\pgfqpoint{0.988630in}{0.718319in}}{\pgfqpoint{0.996443in}{0.710506in}}%
\pgfpathcurveto{\pgfqpoint{1.004257in}{0.702692in}}{\pgfqpoint{1.014856in}{0.698302in}}{\pgfqpoint{1.025906in}{0.698302in}}%
\pgfpathclose%
\pgfusepath{stroke,fill}%
\end{pgfscope}%
\begin{pgfscope}%
\pgfpathrectangle{\pgfqpoint{0.800000in}{0.528000in}}{\pgfqpoint{4.960000in}{3.696000in}}%
\pgfusepath{clip}%
\pgfsetbuttcap%
\pgfsetroundjoin%
\definecolor{currentfill}{rgb}{0.000000,0.000000,0.000000}%
\pgfsetfillcolor{currentfill}%
\pgfsetlinewidth{1.003750pt}%
\definecolor{currentstroke}{rgb}{0.000000,0.000000,0.000000}%
\pgfsetstrokecolor{currentstroke}%
\pgfsetdash{}{0pt}%
\pgfpathmoveto{\pgfqpoint{1.025906in}{0.698302in}}%
\pgfpathcurveto{\pgfqpoint{1.036956in}{0.698302in}}{\pgfqpoint{1.047555in}{0.702692in}}{\pgfqpoint{1.055369in}{0.710506in}}%
\pgfpathcurveto{\pgfqpoint{1.063182in}{0.718319in}}{\pgfqpoint{1.067573in}{0.728918in}}{\pgfqpoint{1.067573in}{0.739969in}}%
\pgfpathcurveto{\pgfqpoint{1.067573in}{0.751019in}}{\pgfqpoint{1.063182in}{0.761618in}}{\pgfqpoint{1.055369in}{0.769431in}}%
\pgfpathcurveto{\pgfqpoint{1.047555in}{0.777245in}}{\pgfqpoint{1.036956in}{0.781635in}}{\pgfqpoint{1.025906in}{0.781635in}}%
\pgfpathcurveto{\pgfqpoint{1.014856in}{0.781635in}}{\pgfqpoint{1.004257in}{0.777245in}}{\pgfqpoint{0.996443in}{0.769431in}}%
\pgfpathcurveto{\pgfqpoint{0.988630in}{0.761618in}}{\pgfqpoint{0.984239in}{0.751019in}}{\pgfqpoint{0.984239in}{0.739969in}}%
\pgfpathcurveto{\pgfqpoint{0.984239in}{0.728918in}}{\pgfqpoint{0.988630in}{0.718319in}}{\pgfqpoint{0.996443in}{0.710506in}}%
\pgfpathcurveto{\pgfqpoint{1.004257in}{0.702692in}}{\pgfqpoint{1.014856in}{0.698302in}}{\pgfqpoint{1.025906in}{0.698302in}}%
\pgfpathclose%
\pgfusepath{stroke,fill}%
\end{pgfscope}%
\begin{pgfscope}%
\pgfpathrectangle{\pgfqpoint{0.800000in}{0.528000in}}{\pgfqpoint{4.960000in}{3.696000in}}%
\pgfusepath{clip}%
\pgfsetbuttcap%
\pgfsetroundjoin%
\definecolor{currentfill}{rgb}{0.000000,0.000000,0.000000}%
\pgfsetfillcolor{currentfill}%
\pgfsetlinewidth{1.003750pt}%
\definecolor{currentstroke}{rgb}{0.000000,0.000000,0.000000}%
\pgfsetstrokecolor{currentstroke}%
\pgfsetdash{}{0pt}%
\pgfpathmoveto{\pgfqpoint{1.025906in}{0.655307in}}%
\pgfpathcurveto{\pgfqpoint{1.036956in}{0.655307in}}{\pgfqpoint{1.047555in}{0.659697in}}{\pgfqpoint{1.055369in}{0.667511in}}%
\pgfpathcurveto{\pgfqpoint{1.063182in}{0.675324in}}{\pgfqpoint{1.067573in}{0.685924in}}{\pgfqpoint{1.067573in}{0.696974in}}%
\pgfpathcurveto{\pgfqpoint{1.067573in}{0.708024in}}{\pgfqpoint{1.063182in}{0.718623in}}{\pgfqpoint{1.055369in}{0.726436in}}%
\pgfpathcurveto{\pgfqpoint{1.047555in}{0.734250in}}{\pgfqpoint{1.036956in}{0.738640in}}{\pgfqpoint{1.025906in}{0.738640in}}%
\pgfpathcurveto{\pgfqpoint{1.014856in}{0.738640in}}{\pgfqpoint{1.004257in}{0.734250in}}{\pgfqpoint{0.996443in}{0.726436in}}%
\pgfpathcurveto{\pgfqpoint{0.988630in}{0.718623in}}{\pgfqpoint{0.984239in}{0.708024in}}{\pgfqpoint{0.984239in}{0.696974in}}%
\pgfpathcurveto{\pgfqpoint{0.984239in}{0.685924in}}{\pgfqpoint{0.988630in}{0.675324in}}{\pgfqpoint{0.996443in}{0.667511in}}%
\pgfpathcurveto{\pgfqpoint{1.004257in}{0.659697in}}{\pgfqpoint{1.014856in}{0.655307in}}{\pgfqpoint{1.025906in}{0.655307in}}%
\pgfpathclose%
\pgfusepath{stroke,fill}%
\end{pgfscope}%
\begin{pgfscope}%
\pgfpathrectangle{\pgfqpoint{0.800000in}{0.528000in}}{\pgfqpoint{4.960000in}{3.696000in}}%
\pgfusepath{clip}%
\pgfsetbuttcap%
\pgfsetroundjoin%
\definecolor{currentfill}{rgb}{0.000000,0.000000,0.000000}%
\pgfsetfillcolor{currentfill}%
\pgfsetlinewidth{1.003750pt}%
\definecolor{currentstroke}{rgb}{0.000000,0.000000,0.000000}%
\pgfsetstrokecolor{currentstroke}%
\pgfsetdash{}{0pt}%
\pgfpathmoveto{\pgfqpoint{1.025906in}{0.719799in}}%
\pgfpathcurveto{\pgfqpoint{1.036956in}{0.719799in}}{\pgfqpoint{1.047555in}{0.724190in}}{\pgfqpoint{1.055369in}{0.732003in}}%
\pgfpathcurveto{\pgfqpoint{1.063182in}{0.739817in}}{\pgfqpoint{1.067573in}{0.750416in}}{\pgfqpoint{1.067573in}{0.761466in}}%
\pgfpathcurveto{\pgfqpoint{1.067573in}{0.772516in}}{\pgfqpoint{1.063182in}{0.783115in}}{\pgfqpoint{1.055369in}{0.790929in}}%
\pgfpathcurveto{\pgfqpoint{1.047555in}{0.798743in}}{\pgfqpoint{1.036956in}{0.803133in}}{\pgfqpoint{1.025906in}{0.803133in}}%
\pgfpathcurveto{\pgfqpoint{1.014856in}{0.803133in}}{\pgfqpoint{1.004257in}{0.798743in}}{\pgfqpoint{0.996443in}{0.790929in}}%
\pgfpathcurveto{\pgfqpoint{0.988630in}{0.783115in}}{\pgfqpoint{0.984239in}{0.772516in}}{\pgfqpoint{0.984239in}{0.761466in}}%
\pgfpathcurveto{\pgfqpoint{0.984239in}{0.750416in}}{\pgfqpoint{0.988630in}{0.739817in}}{\pgfqpoint{0.996443in}{0.732003in}}%
\pgfpathcurveto{\pgfqpoint{1.004257in}{0.724190in}}{\pgfqpoint{1.014856in}{0.719799in}}{\pgfqpoint{1.025906in}{0.719799in}}%
\pgfpathclose%
\pgfusepath{stroke,fill}%
\end{pgfscope}%
\begin{pgfscope}%
\pgfpathrectangle{\pgfqpoint{0.800000in}{0.528000in}}{\pgfqpoint{4.960000in}{3.696000in}}%
\pgfusepath{clip}%
\pgfsetbuttcap%
\pgfsetroundjoin%
\definecolor{currentfill}{rgb}{0.000000,0.000000,0.000000}%
\pgfsetfillcolor{currentfill}%
\pgfsetlinewidth{1.003750pt}%
\definecolor{currentstroke}{rgb}{0.000000,0.000000,0.000000}%
\pgfsetstrokecolor{currentstroke}%
\pgfsetdash{}{0pt}%
\pgfpathmoveto{\pgfqpoint{1.025906in}{0.655307in}}%
\pgfpathcurveto{\pgfqpoint{1.036956in}{0.655307in}}{\pgfqpoint{1.047555in}{0.659697in}}{\pgfqpoint{1.055369in}{0.667511in}}%
\pgfpathcurveto{\pgfqpoint{1.063182in}{0.675324in}}{\pgfqpoint{1.067573in}{0.685924in}}{\pgfqpoint{1.067573in}{0.696974in}}%
\pgfpathcurveto{\pgfqpoint{1.067573in}{0.708024in}}{\pgfqpoint{1.063182in}{0.718623in}}{\pgfqpoint{1.055369in}{0.726436in}}%
\pgfpathcurveto{\pgfqpoint{1.047555in}{0.734250in}}{\pgfqpoint{1.036956in}{0.738640in}}{\pgfqpoint{1.025906in}{0.738640in}}%
\pgfpathcurveto{\pgfqpoint{1.014856in}{0.738640in}}{\pgfqpoint{1.004257in}{0.734250in}}{\pgfqpoint{0.996443in}{0.726436in}}%
\pgfpathcurveto{\pgfqpoint{0.988630in}{0.718623in}}{\pgfqpoint{0.984239in}{0.708024in}}{\pgfqpoint{0.984239in}{0.696974in}}%
\pgfpathcurveto{\pgfqpoint{0.984239in}{0.685924in}}{\pgfqpoint{0.988630in}{0.675324in}}{\pgfqpoint{0.996443in}{0.667511in}}%
\pgfpathcurveto{\pgfqpoint{1.004257in}{0.659697in}}{\pgfqpoint{1.014856in}{0.655307in}}{\pgfqpoint{1.025906in}{0.655307in}}%
\pgfpathclose%
\pgfusepath{stroke,fill}%
\end{pgfscope}%
\begin{pgfscope}%
\pgfpathrectangle{\pgfqpoint{0.800000in}{0.528000in}}{\pgfqpoint{4.960000in}{3.696000in}}%
\pgfusepath{clip}%
\pgfsetbuttcap%
\pgfsetroundjoin%
\definecolor{currentfill}{rgb}{0.000000,0.000000,0.000000}%
\pgfsetfillcolor{currentfill}%
\pgfsetlinewidth{1.003750pt}%
\definecolor{currentstroke}{rgb}{0.000000,0.000000,0.000000}%
\pgfsetstrokecolor{currentstroke}%
\pgfsetdash{}{0pt}%
\pgfpathmoveto{\pgfqpoint{1.025906in}{0.676804in}}%
\pgfpathcurveto{\pgfqpoint{1.036956in}{0.676804in}}{\pgfqpoint{1.047555in}{0.681195in}}{\pgfqpoint{1.055369in}{0.689008in}}%
\pgfpathcurveto{\pgfqpoint{1.063182in}{0.696822in}}{\pgfqpoint{1.067573in}{0.707421in}}{\pgfqpoint{1.067573in}{0.718471in}}%
\pgfpathcurveto{\pgfqpoint{1.067573in}{0.729521in}}{\pgfqpoint{1.063182in}{0.740120in}}{\pgfqpoint{1.055369in}{0.747934in}}%
\pgfpathcurveto{\pgfqpoint{1.047555in}{0.755748in}}{\pgfqpoint{1.036956in}{0.760138in}}{\pgfqpoint{1.025906in}{0.760138in}}%
\pgfpathcurveto{\pgfqpoint{1.014856in}{0.760138in}}{\pgfqpoint{1.004257in}{0.755748in}}{\pgfqpoint{0.996443in}{0.747934in}}%
\pgfpathcurveto{\pgfqpoint{0.988630in}{0.740120in}}{\pgfqpoint{0.984239in}{0.729521in}}{\pgfqpoint{0.984239in}{0.718471in}}%
\pgfpathcurveto{\pgfqpoint{0.984239in}{0.707421in}}{\pgfqpoint{0.988630in}{0.696822in}}{\pgfqpoint{0.996443in}{0.689008in}}%
\pgfpathcurveto{\pgfqpoint{1.004257in}{0.681195in}}{\pgfqpoint{1.014856in}{0.676804in}}{\pgfqpoint{1.025906in}{0.676804in}}%
\pgfpathclose%
\pgfusepath{stroke,fill}%
\end{pgfscope}%
\begin{pgfscope}%
\pgfpathrectangle{\pgfqpoint{0.800000in}{0.528000in}}{\pgfqpoint{4.960000in}{3.696000in}}%
\pgfusepath{clip}%
\pgfsetbuttcap%
\pgfsetroundjoin%
\definecolor{currentfill}{rgb}{0.000000,0.000000,0.000000}%
\pgfsetfillcolor{currentfill}%
\pgfsetlinewidth{1.003750pt}%
\definecolor{currentstroke}{rgb}{0.000000,0.000000,0.000000}%
\pgfsetstrokecolor{currentstroke}%
\pgfsetdash{}{0pt}%
\pgfpathmoveto{\pgfqpoint{1.025906in}{0.698302in}}%
\pgfpathcurveto{\pgfqpoint{1.036956in}{0.698302in}}{\pgfqpoint{1.047555in}{0.702692in}}{\pgfqpoint{1.055369in}{0.710506in}}%
\pgfpathcurveto{\pgfqpoint{1.063182in}{0.718319in}}{\pgfqpoint{1.067573in}{0.728918in}}{\pgfqpoint{1.067573in}{0.739969in}}%
\pgfpathcurveto{\pgfqpoint{1.067573in}{0.751019in}}{\pgfqpoint{1.063182in}{0.761618in}}{\pgfqpoint{1.055369in}{0.769431in}}%
\pgfpathcurveto{\pgfqpoint{1.047555in}{0.777245in}}{\pgfqpoint{1.036956in}{0.781635in}}{\pgfqpoint{1.025906in}{0.781635in}}%
\pgfpathcurveto{\pgfqpoint{1.014856in}{0.781635in}}{\pgfqpoint{1.004257in}{0.777245in}}{\pgfqpoint{0.996443in}{0.769431in}}%
\pgfpathcurveto{\pgfqpoint{0.988630in}{0.761618in}}{\pgfqpoint{0.984239in}{0.751019in}}{\pgfqpoint{0.984239in}{0.739969in}}%
\pgfpathcurveto{\pgfqpoint{0.984239in}{0.728918in}}{\pgfqpoint{0.988630in}{0.718319in}}{\pgfqpoint{0.996443in}{0.710506in}}%
\pgfpathcurveto{\pgfqpoint{1.004257in}{0.702692in}}{\pgfqpoint{1.014856in}{0.698302in}}{\pgfqpoint{1.025906in}{0.698302in}}%
\pgfpathclose%
\pgfusepath{stroke,fill}%
\end{pgfscope}%
\begin{pgfscope}%
\pgfpathrectangle{\pgfqpoint{0.800000in}{0.528000in}}{\pgfqpoint{4.960000in}{3.696000in}}%
\pgfusepath{clip}%
\pgfsetbuttcap%
\pgfsetroundjoin%
\definecolor{currentfill}{rgb}{0.000000,0.000000,0.000000}%
\pgfsetfillcolor{currentfill}%
\pgfsetlinewidth{1.003750pt}%
\definecolor{currentstroke}{rgb}{0.000000,0.000000,0.000000}%
\pgfsetstrokecolor{currentstroke}%
\pgfsetdash{}{0pt}%
\pgfpathmoveto{\pgfqpoint{1.025906in}{0.676804in}}%
\pgfpathcurveto{\pgfqpoint{1.036956in}{0.676804in}}{\pgfqpoint{1.047555in}{0.681195in}}{\pgfqpoint{1.055369in}{0.689008in}}%
\pgfpathcurveto{\pgfqpoint{1.063182in}{0.696822in}}{\pgfqpoint{1.067573in}{0.707421in}}{\pgfqpoint{1.067573in}{0.718471in}}%
\pgfpathcurveto{\pgfqpoint{1.067573in}{0.729521in}}{\pgfqpoint{1.063182in}{0.740120in}}{\pgfqpoint{1.055369in}{0.747934in}}%
\pgfpathcurveto{\pgfqpoint{1.047555in}{0.755748in}}{\pgfqpoint{1.036956in}{0.760138in}}{\pgfqpoint{1.025906in}{0.760138in}}%
\pgfpathcurveto{\pgfqpoint{1.014856in}{0.760138in}}{\pgfqpoint{1.004257in}{0.755748in}}{\pgfqpoint{0.996443in}{0.747934in}}%
\pgfpathcurveto{\pgfqpoint{0.988630in}{0.740120in}}{\pgfqpoint{0.984239in}{0.729521in}}{\pgfqpoint{0.984239in}{0.718471in}}%
\pgfpathcurveto{\pgfqpoint{0.984239in}{0.707421in}}{\pgfqpoint{0.988630in}{0.696822in}}{\pgfqpoint{0.996443in}{0.689008in}}%
\pgfpathcurveto{\pgfqpoint{1.004257in}{0.681195in}}{\pgfqpoint{1.014856in}{0.676804in}}{\pgfqpoint{1.025906in}{0.676804in}}%
\pgfpathclose%
\pgfusepath{stroke,fill}%
\end{pgfscope}%
\begin{pgfscope}%
\pgfpathrectangle{\pgfqpoint{0.800000in}{0.528000in}}{\pgfqpoint{4.960000in}{3.696000in}}%
\pgfusepath{clip}%
\pgfsetbuttcap%
\pgfsetroundjoin%
\definecolor{currentfill}{rgb}{0.000000,0.000000,0.000000}%
\pgfsetfillcolor{currentfill}%
\pgfsetlinewidth{1.003750pt}%
\definecolor{currentstroke}{rgb}{0.000000,0.000000,0.000000}%
\pgfsetstrokecolor{currentstroke}%
\pgfsetdash{}{0pt}%
\pgfpathmoveto{\pgfqpoint{2.518786in}{1.171247in}}%
\pgfpathcurveto{\pgfqpoint{2.529836in}{1.171247in}}{\pgfqpoint{2.540435in}{1.175637in}}{\pgfqpoint{2.548249in}{1.183451in}}%
\pgfpathcurveto{\pgfqpoint{2.556062in}{1.191264in}}{\pgfqpoint{2.560452in}{1.201863in}}{\pgfqpoint{2.560452in}{1.212913in}}%
\pgfpathcurveto{\pgfqpoint{2.560452in}{1.223964in}}{\pgfqpoint{2.556062in}{1.234563in}}{\pgfqpoint{2.548249in}{1.242376in}}%
\pgfpathcurveto{\pgfqpoint{2.540435in}{1.250190in}}{\pgfqpoint{2.529836in}{1.254580in}}{\pgfqpoint{2.518786in}{1.254580in}}%
\pgfpathcurveto{\pgfqpoint{2.507736in}{1.254580in}}{\pgfqpoint{2.497137in}{1.250190in}}{\pgfqpoint{2.489323in}{1.242376in}}%
\pgfpathcurveto{\pgfqpoint{2.481509in}{1.234563in}}{\pgfqpoint{2.477119in}{1.223964in}}{\pgfqpoint{2.477119in}{1.212913in}}%
\pgfpathcurveto{\pgfqpoint{2.477119in}{1.201863in}}{\pgfqpoint{2.481509in}{1.191264in}}{\pgfqpoint{2.489323in}{1.183451in}}%
\pgfpathcurveto{\pgfqpoint{2.497137in}{1.175637in}}{\pgfqpoint{2.507736in}{1.171247in}}{\pgfqpoint{2.518786in}{1.171247in}}%
\pgfpathclose%
\pgfusepath{stroke,fill}%
\end{pgfscope}%
\begin{pgfscope}%
\pgfpathrectangle{\pgfqpoint{0.800000in}{0.528000in}}{\pgfqpoint{4.960000in}{3.696000in}}%
\pgfusepath{clip}%
\pgfsetbuttcap%
\pgfsetroundjoin%
\definecolor{currentfill}{rgb}{0.000000,0.000000,0.000000}%
\pgfsetfillcolor{currentfill}%
\pgfsetlinewidth{1.003750pt}%
\definecolor{currentstroke}{rgb}{0.000000,0.000000,0.000000}%
\pgfsetstrokecolor{currentstroke}%
\pgfsetdash{}{0pt}%
\pgfpathmoveto{\pgfqpoint{2.518786in}{1.214242in}}%
\pgfpathcurveto{\pgfqpoint{2.529836in}{1.214242in}}{\pgfqpoint{2.540435in}{1.218632in}}{\pgfqpoint{2.548249in}{1.226446in}}%
\pgfpathcurveto{\pgfqpoint{2.556062in}{1.234259in}}{\pgfqpoint{2.560452in}{1.244858in}}{\pgfqpoint{2.560452in}{1.255908in}}%
\pgfpathcurveto{\pgfqpoint{2.560452in}{1.266959in}}{\pgfqpoint{2.556062in}{1.277558in}}{\pgfqpoint{2.548249in}{1.285371in}}%
\pgfpathcurveto{\pgfqpoint{2.540435in}{1.293185in}}{\pgfqpoint{2.529836in}{1.297575in}}{\pgfqpoint{2.518786in}{1.297575in}}%
\pgfpathcurveto{\pgfqpoint{2.507736in}{1.297575in}}{\pgfqpoint{2.497137in}{1.293185in}}{\pgfqpoint{2.489323in}{1.285371in}}%
\pgfpathcurveto{\pgfqpoint{2.481509in}{1.277558in}}{\pgfqpoint{2.477119in}{1.266959in}}{\pgfqpoint{2.477119in}{1.255908in}}%
\pgfpathcurveto{\pgfqpoint{2.477119in}{1.244858in}}{\pgfqpoint{2.481509in}{1.234259in}}{\pgfqpoint{2.489323in}{1.226446in}}%
\pgfpathcurveto{\pgfqpoint{2.497137in}{1.218632in}}{\pgfqpoint{2.507736in}{1.214242in}}{\pgfqpoint{2.518786in}{1.214242in}}%
\pgfpathclose%
\pgfusepath{stroke,fill}%
\end{pgfscope}%
\begin{pgfscope}%
\pgfpathrectangle{\pgfqpoint{0.800000in}{0.528000in}}{\pgfqpoint{4.960000in}{3.696000in}}%
\pgfusepath{clip}%
\pgfsetbuttcap%
\pgfsetroundjoin%
\definecolor{currentfill}{rgb}{0.000000,0.000000,0.000000}%
\pgfsetfillcolor{currentfill}%
\pgfsetlinewidth{1.003750pt}%
\definecolor{currentstroke}{rgb}{0.000000,0.000000,0.000000}%
\pgfsetstrokecolor{currentstroke}%
\pgfsetdash{}{0pt}%
\pgfpathmoveto{\pgfqpoint{2.518786in}{1.106754in}}%
\pgfpathcurveto{\pgfqpoint{2.529836in}{1.106754in}}{\pgfqpoint{2.540435in}{1.111145in}}{\pgfqpoint{2.548249in}{1.118958in}}%
\pgfpathcurveto{\pgfqpoint{2.556062in}{1.126772in}}{\pgfqpoint{2.560452in}{1.137371in}}{\pgfqpoint{2.560452in}{1.148421in}}%
\pgfpathcurveto{\pgfqpoint{2.560452in}{1.159471in}}{\pgfqpoint{2.556062in}{1.170070in}}{\pgfqpoint{2.548249in}{1.177884in}}%
\pgfpathcurveto{\pgfqpoint{2.540435in}{1.185697in}}{\pgfqpoint{2.529836in}{1.190088in}}{\pgfqpoint{2.518786in}{1.190088in}}%
\pgfpathcurveto{\pgfqpoint{2.507736in}{1.190088in}}{\pgfqpoint{2.497137in}{1.185697in}}{\pgfqpoint{2.489323in}{1.177884in}}%
\pgfpathcurveto{\pgfqpoint{2.481509in}{1.170070in}}{\pgfqpoint{2.477119in}{1.159471in}}{\pgfqpoint{2.477119in}{1.148421in}}%
\pgfpathcurveto{\pgfqpoint{2.477119in}{1.137371in}}{\pgfqpoint{2.481509in}{1.126772in}}{\pgfqpoint{2.489323in}{1.118958in}}%
\pgfpathcurveto{\pgfqpoint{2.497137in}{1.111145in}}{\pgfqpoint{2.507736in}{1.106754in}}{\pgfqpoint{2.518786in}{1.106754in}}%
\pgfpathclose%
\pgfusepath{stroke,fill}%
\end{pgfscope}%
\begin{pgfscope}%
\pgfpathrectangle{\pgfqpoint{0.800000in}{0.528000in}}{\pgfqpoint{4.960000in}{3.696000in}}%
\pgfusepath{clip}%
\pgfsetbuttcap%
\pgfsetroundjoin%
\definecolor{currentfill}{rgb}{0.000000,0.000000,0.000000}%
\pgfsetfillcolor{currentfill}%
\pgfsetlinewidth{1.003750pt}%
\definecolor{currentstroke}{rgb}{0.000000,0.000000,0.000000}%
\pgfsetstrokecolor{currentstroke}%
\pgfsetdash{}{0pt}%
\pgfpathmoveto{\pgfqpoint{2.518786in}{1.085257in}}%
\pgfpathcurveto{\pgfqpoint{2.529836in}{1.085257in}}{\pgfqpoint{2.540435in}{1.089647in}}{\pgfqpoint{2.548249in}{1.097461in}}%
\pgfpathcurveto{\pgfqpoint{2.556062in}{1.105274in}}{\pgfqpoint{2.560452in}{1.115873in}}{\pgfqpoint{2.560452in}{1.126923in}}%
\pgfpathcurveto{\pgfqpoint{2.560452in}{1.137974in}}{\pgfqpoint{2.556062in}{1.148573in}}{\pgfqpoint{2.548249in}{1.156386in}}%
\pgfpathcurveto{\pgfqpoint{2.540435in}{1.164200in}}{\pgfqpoint{2.529836in}{1.168590in}}{\pgfqpoint{2.518786in}{1.168590in}}%
\pgfpathcurveto{\pgfqpoint{2.507736in}{1.168590in}}{\pgfqpoint{2.497137in}{1.164200in}}{\pgfqpoint{2.489323in}{1.156386in}}%
\pgfpathcurveto{\pgfqpoint{2.481509in}{1.148573in}}{\pgfqpoint{2.477119in}{1.137974in}}{\pgfqpoint{2.477119in}{1.126923in}}%
\pgfpathcurveto{\pgfqpoint{2.477119in}{1.115873in}}{\pgfqpoint{2.481509in}{1.105274in}}{\pgfqpoint{2.489323in}{1.097461in}}%
\pgfpathcurveto{\pgfqpoint{2.497137in}{1.089647in}}{\pgfqpoint{2.507736in}{1.085257in}}{\pgfqpoint{2.518786in}{1.085257in}}%
\pgfpathclose%
\pgfusepath{stroke,fill}%
\end{pgfscope}%
\begin{pgfscope}%
\pgfpathrectangle{\pgfqpoint{0.800000in}{0.528000in}}{\pgfqpoint{4.960000in}{3.696000in}}%
\pgfusepath{clip}%
\pgfsetbuttcap%
\pgfsetroundjoin%
\definecolor{currentfill}{rgb}{0.000000,0.000000,0.000000}%
\pgfsetfillcolor{currentfill}%
\pgfsetlinewidth{1.003750pt}%
\definecolor{currentstroke}{rgb}{0.000000,0.000000,0.000000}%
\pgfsetstrokecolor{currentstroke}%
\pgfsetdash{}{0pt}%
\pgfpathmoveto{\pgfqpoint{2.518786in}{1.192744in}}%
\pgfpathcurveto{\pgfqpoint{2.529836in}{1.192744in}}{\pgfqpoint{2.540435in}{1.197135in}}{\pgfqpoint{2.548249in}{1.204948in}}%
\pgfpathcurveto{\pgfqpoint{2.556062in}{1.212762in}}{\pgfqpoint{2.560452in}{1.223361in}}{\pgfqpoint{2.560452in}{1.234411in}}%
\pgfpathcurveto{\pgfqpoint{2.560452in}{1.245461in}}{\pgfqpoint{2.556062in}{1.256060in}}{\pgfqpoint{2.548249in}{1.263874in}}%
\pgfpathcurveto{\pgfqpoint{2.540435in}{1.271687in}}{\pgfqpoint{2.529836in}{1.276078in}}{\pgfqpoint{2.518786in}{1.276078in}}%
\pgfpathcurveto{\pgfqpoint{2.507736in}{1.276078in}}{\pgfqpoint{2.497137in}{1.271687in}}{\pgfqpoint{2.489323in}{1.263874in}}%
\pgfpathcurveto{\pgfqpoint{2.481509in}{1.256060in}}{\pgfqpoint{2.477119in}{1.245461in}}{\pgfqpoint{2.477119in}{1.234411in}}%
\pgfpathcurveto{\pgfqpoint{2.477119in}{1.223361in}}{\pgfqpoint{2.481509in}{1.212762in}}{\pgfqpoint{2.489323in}{1.204948in}}%
\pgfpathcurveto{\pgfqpoint{2.497137in}{1.197135in}}{\pgfqpoint{2.507736in}{1.192744in}}{\pgfqpoint{2.518786in}{1.192744in}}%
\pgfpathclose%
\pgfusepath{stroke,fill}%
\end{pgfscope}%
\begin{pgfscope}%
\pgfpathrectangle{\pgfqpoint{0.800000in}{0.528000in}}{\pgfqpoint{4.960000in}{3.696000in}}%
\pgfusepath{clip}%
\pgfsetbuttcap%
\pgfsetroundjoin%
\definecolor{currentfill}{rgb}{0.000000,0.000000,0.000000}%
\pgfsetfillcolor{currentfill}%
\pgfsetlinewidth{1.003750pt}%
\definecolor{currentstroke}{rgb}{0.000000,0.000000,0.000000}%
\pgfsetstrokecolor{currentstroke}%
\pgfsetdash{}{0pt}%
\pgfpathmoveto{\pgfqpoint{2.518786in}{1.235739in}}%
\pgfpathcurveto{\pgfqpoint{2.529836in}{1.235739in}}{\pgfqpoint{2.540435in}{1.240130in}}{\pgfqpoint{2.548249in}{1.247943in}}%
\pgfpathcurveto{\pgfqpoint{2.556062in}{1.255757in}}{\pgfqpoint{2.560452in}{1.266356in}}{\pgfqpoint{2.560452in}{1.277406in}}%
\pgfpathcurveto{\pgfqpoint{2.560452in}{1.288456in}}{\pgfqpoint{2.556062in}{1.299055in}}{\pgfqpoint{2.548249in}{1.306869in}}%
\pgfpathcurveto{\pgfqpoint{2.540435in}{1.314682in}}{\pgfqpoint{2.529836in}{1.319073in}}{\pgfqpoint{2.518786in}{1.319073in}}%
\pgfpathcurveto{\pgfqpoint{2.507736in}{1.319073in}}{\pgfqpoint{2.497137in}{1.314682in}}{\pgfqpoint{2.489323in}{1.306869in}}%
\pgfpathcurveto{\pgfqpoint{2.481509in}{1.299055in}}{\pgfqpoint{2.477119in}{1.288456in}}{\pgfqpoint{2.477119in}{1.277406in}}%
\pgfpathcurveto{\pgfqpoint{2.477119in}{1.266356in}}{\pgfqpoint{2.481509in}{1.255757in}}{\pgfqpoint{2.489323in}{1.247943in}}%
\pgfpathcurveto{\pgfqpoint{2.497137in}{1.240130in}}{\pgfqpoint{2.507736in}{1.235739in}}{\pgfqpoint{2.518786in}{1.235739in}}%
\pgfpathclose%
\pgfusepath{stroke,fill}%
\end{pgfscope}%
\begin{pgfscope}%
\pgfpathrectangle{\pgfqpoint{0.800000in}{0.528000in}}{\pgfqpoint{4.960000in}{3.696000in}}%
\pgfusepath{clip}%
\pgfsetbuttcap%
\pgfsetroundjoin%
\definecolor{currentfill}{rgb}{0.000000,0.000000,0.000000}%
\pgfsetfillcolor{currentfill}%
\pgfsetlinewidth{1.003750pt}%
\definecolor{currentstroke}{rgb}{0.000000,0.000000,0.000000}%
\pgfsetstrokecolor{currentstroke}%
\pgfsetdash{}{0pt}%
\pgfpathmoveto{\pgfqpoint{2.518786in}{1.063759in}}%
\pgfpathcurveto{\pgfqpoint{2.529836in}{1.063759in}}{\pgfqpoint{2.540435in}{1.068150in}}{\pgfqpoint{2.548249in}{1.075963in}}%
\pgfpathcurveto{\pgfqpoint{2.556062in}{1.083777in}}{\pgfqpoint{2.560452in}{1.094376in}}{\pgfqpoint{2.560452in}{1.105426in}}%
\pgfpathcurveto{\pgfqpoint{2.560452in}{1.116476in}}{\pgfqpoint{2.556062in}{1.127075in}}{\pgfqpoint{2.548249in}{1.134889in}}%
\pgfpathcurveto{\pgfqpoint{2.540435in}{1.142702in}}{\pgfqpoint{2.529836in}{1.147093in}}{\pgfqpoint{2.518786in}{1.147093in}}%
\pgfpathcurveto{\pgfqpoint{2.507736in}{1.147093in}}{\pgfqpoint{2.497137in}{1.142702in}}{\pgfqpoint{2.489323in}{1.134889in}}%
\pgfpathcurveto{\pgfqpoint{2.481509in}{1.127075in}}{\pgfqpoint{2.477119in}{1.116476in}}{\pgfqpoint{2.477119in}{1.105426in}}%
\pgfpathcurveto{\pgfqpoint{2.477119in}{1.094376in}}{\pgfqpoint{2.481509in}{1.083777in}}{\pgfqpoint{2.489323in}{1.075963in}}%
\pgfpathcurveto{\pgfqpoint{2.497137in}{1.068150in}}{\pgfqpoint{2.507736in}{1.063759in}}{\pgfqpoint{2.518786in}{1.063759in}}%
\pgfpathclose%
\pgfusepath{stroke,fill}%
\end{pgfscope}%
\begin{pgfscope}%
\pgfpathrectangle{\pgfqpoint{0.800000in}{0.528000in}}{\pgfqpoint{4.960000in}{3.696000in}}%
\pgfusepath{clip}%
\pgfsetbuttcap%
\pgfsetroundjoin%
\definecolor{currentfill}{rgb}{0.000000,0.000000,0.000000}%
\pgfsetfillcolor{currentfill}%
\pgfsetlinewidth{1.003750pt}%
\definecolor{currentstroke}{rgb}{0.000000,0.000000,0.000000}%
\pgfsetstrokecolor{currentstroke}%
\pgfsetdash{}{0pt}%
\pgfpathmoveto{\pgfqpoint{2.518786in}{1.042262in}}%
\pgfpathcurveto{\pgfqpoint{2.529836in}{1.042262in}}{\pgfqpoint{2.540435in}{1.046652in}}{\pgfqpoint{2.548249in}{1.054466in}}%
\pgfpathcurveto{\pgfqpoint{2.556062in}{1.062279in}}{\pgfqpoint{2.560452in}{1.072878in}}{\pgfqpoint{2.560452in}{1.083928in}}%
\pgfpathcurveto{\pgfqpoint{2.560452in}{1.094979in}}{\pgfqpoint{2.556062in}{1.105578in}}{\pgfqpoint{2.548249in}{1.113391in}}%
\pgfpathcurveto{\pgfqpoint{2.540435in}{1.121205in}}{\pgfqpoint{2.529836in}{1.125595in}}{\pgfqpoint{2.518786in}{1.125595in}}%
\pgfpathcurveto{\pgfqpoint{2.507736in}{1.125595in}}{\pgfqpoint{2.497137in}{1.121205in}}{\pgfqpoint{2.489323in}{1.113391in}}%
\pgfpathcurveto{\pgfqpoint{2.481509in}{1.105578in}}{\pgfqpoint{2.477119in}{1.094979in}}{\pgfqpoint{2.477119in}{1.083928in}}%
\pgfpathcurveto{\pgfqpoint{2.477119in}{1.072878in}}{\pgfqpoint{2.481509in}{1.062279in}}{\pgfqpoint{2.489323in}{1.054466in}}%
\pgfpathcurveto{\pgfqpoint{2.497137in}{1.046652in}}{\pgfqpoint{2.507736in}{1.042262in}}{\pgfqpoint{2.518786in}{1.042262in}}%
\pgfpathclose%
\pgfusepath{stroke,fill}%
\end{pgfscope}%
\begin{pgfscope}%
\pgfpathrectangle{\pgfqpoint{0.800000in}{0.528000in}}{\pgfqpoint{4.960000in}{3.696000in}}%
\pgfusepath{clip}%
\pgfsetbuttcap%
\pgfsetroundjoin%
\definecolor{currentfill}{rgb}{0.000000,0.000000,0.000000}%
\pgfsetfillcolor{currentfill}%
\pgfsetlinewidth{1.003750pt}%
\definecolor{currentstroke}{rgb}{0.000000,0.000000,0.000000}%
\pgfsetstrokecolor{currentstroke}%
\pgfsetdash{}{0pt}%
\pgfpathmoveto{\pgfqpoint{2.518786in}{1.106754in}}%
\pgfpathcurveto{\pgfqpoint{2.529836in}{1.106754in}}{\pgfqpoint{2.540435in}{1.111145in}}{\pgfqpoint{2.548249in}{1.118958in}}%
\pgfpathcurveto{\pgfqpoint{2.556062in}{1.126772in}}{\pgfqpoint{2.560452in}{1.137371in}}{\pgfqpoint{2.560452in}{1.148421in}}%
\pgfpathcurveto{\pgfqpoint{2.560452in}{1.159471in}}{\pgfqpoint{2.556062in}{1.170070in}}{\pgfqpoint{2.548249in}{1.177884in}}%
\pgfpathcurveto{\pgfqpoint{2.540435in}{1.185697in}}{\pgfqpoint{2.529836in}{1.190088in}}{\pgfqpoint{2.518786in}{1.190088in}}%
\pgfpathcurveto{\pgfqpoint{2.507736in}{1.190088in}}{\pgfqpoint{2.497137in}{1.185697in}}{\pgfqpoint{2.489323in}{1.177884in}}%
\pgfpathcurveto{\pgfqpoint{2.481509in}{1.170070in}}{\pgfqpoint{2.477119in}{1.159471in}}{\pgfqpoint{2.477119in}{1.148421in}}%
\pgfpathcurveto{\pgfqpoint{2.477119in}{1.137371in}}{\pgfqpoint{2.481509in}{1.126772in}}{\pgfqpoint{2.489323in}{1.118958in}}%
\pgfpathcurveto{\pgfqpoint{2.497137in}{1.111145in}}{\pgfqpoint{2.507736in}{1.106754in}}{\pgfqpoint{2.518786in}{1.106754in}}%
\pgfpathclose%
\pgfusepath{stroke,fill}%
\end{pgfscope}%
\begin{pgfscope}%
\pgfpathrectangle{\pgfqpoint{0.800000in}{0.528000in}}{\pgfqpoint{4.960000in}{3.696000in}}%
\pgfusepath{clip}%
\pgfsetbuttcap%
\pgfsetroundjoin%
\definecolor{currentfill}{rgb}{0.000000,0.000000,0.000000}%
\pgfsetfillcolor{currentfill}%
\pgfsetlinewidth{1.003750pt}%
\definecolor{currentstroke}{rgb}{0.000000,0.000000,0.000000}%
\pgfsetstrokecolor{currentstroke}%
\pgfsetdash{}{0pt}%
\pgfpathmoveto{\pgfqpoint{2.518786in}{1.149749in}}%
\pgfpathcurveto{\pgfqpoint{2.529836in}{1.149749in}}{\pgfqpoint{2.540435in}{1.154140in}}{\pgfqpoint{2.548249in}{1.161953in}}%
\pgfpathcurveto{\pgfqpoint{2.556062in}{1.169767in}}{\pgfqpoint{2.560452in}{1.180366in}}{\pgfqpoint{2.560452in}{1.191416in}}%
\pgfpathcurveto{\pgfqpoint{2.560452in}{1.202466in}}{\pgfqpoint{2.556062in}{1.213065in}}{\pgfqpoint{2.548249in}{1.220879in}}%
\pgfpathcurveto{\pgfqpoint{2.540435in}{1.228692in}}{\pgfqpoint{2.529836in}{1.233083in}}{\pgfqpoint{2.518786in}{1.233083in}}%
\pgfpathcurveto{\pgfqpoint{2.507736in}{1.233083in}}{\pgfqpoint{2.497137in}{1.228692in}}{\pgfqpoint{2.489323in}{1.220879in}}%
\pgfpathcurveto{\pgfqpoint{2.481509in}{1.213065in}}{\pgfqpoint{2.477119in}{1.202466in}}{\pgfqpoint{2.477119in}{1.191416in}}%
\pgfpathcurveto{\pgfqpoint{2.477119in}{1.180366in}}{\pgfqpoint{2.481509in}{1.169767in}}{\pgfqpoint{2.489323in}{1.161953in}}%
\pgfpathcurveto{\pgfqpoint{2.497137in}{1.154140in}}{\pgfqpoint{2.507736in}{1.149749in}}{\pgfqpoint{2.518786in}{1.149749in}}%
\pgfpathclose%
\pgfusepath{stroke,fill}%
\end{pgfscope}%
\begin{pgfscope}%
\pgfpathrectangle{\pgfqpoint{0.800000in}{0.528000in}}{\pgfqpoint{4.960000in}{3.696000in}}%
\pgfusepath{clip}%
\pgfsetbuttcap%
\pgfsetroundjoin%
\definecolor{currentfill}{rgb}{0.000000,0.000000,0.000000}%
\pgfsetfillcolor{currentfill}%
\pgfsetlinewidth{1.003750pt}%
\definecolor{currentstroke}{rgb}{0.000000,0.000000,0.000000}%
\pgfsetstrokecolor{currentstroke}%
\pgfsetdash{}{0pt}%
\pgfpathmoveto{\pgfqpoint{2.518786in}{1.214242in}}%
\pgfpathcurveto{\pgfqpoint{2.529836in}{1.214242in}}{\pgfqpoint{2.540435in}{1.218632in}}{\pgfqpoint{2.548249in}{1.226446in}}%
\pgfpathcurveto{\pgfqpoint{2.556062in}{1.234259in}}{\pgfqpoint{2.560452in}{1.244858in}}{\pgfqpoint{2.560452in}{1.255908in}}%
\pgfpathcurveto{\pgfqpoint{2.560452in}{1.266959in}}{\pgfqpoint{2.556062in}{1.277558in}}{\pgfqpoint{2.548249in}{1.285371in}}%
\pgfpathcurveto{\pgfqpoint{2.540435in}{1.293185in}}{\pgfqpoint{2.529836in}{1.297575in}}{\pgfqpoint{2.518786in}{1.297575in}}%
\pgfpathcurveto{\pgfqpoint{2.507736in}{1.297575in}}{\pgfqpoint{2.497137in}{1.293185in}}{\pgfqpoint{2.489323in}{1.285371in}}%
\pgfpathcurveto{\pgfqpoint{2.481509in}{1.277558in}}{\pgfqpoint{2.477119in}{1.266959in}}{\pgfqpoint{2.477119in}{1.255908in}}%
\pgfpathcurveto{\pgfqpoint{2.477119in}{1.244858in}}{\pgfqpoint{2.481509in}{1.234259in}}{\pgfqpoint{2.489323in}{1.226446in}}%
\pgfpathcurveto{\pgfqpoint{2.497137in}{1.218632in}}{\pgfqpoint{2.507736in}{1.214242in}}{\pgfqpoint{2.518786in}{1.214242in}}%
\pgfpathclose%
\pgfusepath{stroke,fill}%
\end{pgfscope}%
\begin{pgfscope}%
\pgfpathrectangle{\pgfqpoint{0.800000in}{0.528000in}}{\pgfqpoint{4.960000in}{3.696000in}}%
\pgfusepath{clip}%
\pgfsetbuttcap%
\pgfsetroundjoin%
\definecolor{currentfill}{rgb}{0.000000,0.000000,0.000000}%
\pgfsetfillcolor{currentfill}%
\pgfsetlinewidth{1.003750pt}%
\definecolor{currentstroke}{rgb}{0.000000,0.000000,0.000000}%
\pgfsetstrokecolor{currentstroke}%
\pgfsetdash{}{0pt}%
\pgfpathmoveto{\pgfqpoint{2.518786in}{1.192744in}}%
\pgfpathcurveto{\pgfqpoint{2.529836in}{1.192744in}}{\pgfqpoint{2.540435in}{1.197135in}}{\pgfqpoint{2.548249in}{1.204948in}}%
\pgfpathcurveto{\pgfqpoint{2.556062in}{1.212762in}}{\pgfqpoint{2.560452in}{1.223361in}}{\pgfqpoint{2.560452in}{1.234411in}}%
\pgfpathcurveto{\pgfqpoint{2.560452in}{1.245461in}}{\pgfqpoint{2.556062in}{1.256060in}}{\pgfqpoint{2.548249in}{1.263874in}}%
\pgfpathcurveto{\pgfqpoint{2.540435in}{1.271687in}}{\pgfqpoint{2.529836in}{1.276078in}}{\pgfqpoint{2.518786in}{1.276078in}}%
\pgfpathcurveto{\pgfqpoint{2.507736in}{1.276078in}}{\pgfqpoint{2.497137in}{1.271687in}}{\pgfqpoint{2.489323in}{1.263874in}}%
\pgfpathcurveto{\pgfqpoint{2.481509in}{1.256060in}}{\pgfqpoint{2.477119in}{1.245461in}}{\pgfqpoint{2.477119in}{1.234411in}}%
\pgfpathcurveto{\pgfqpoint{2.477119in}{1.223361in}}{\pgfqpoint{2.481509in}{1.212762in}}{\pgfqpoint{2.489323in}{1.204948in}}%
\pgfpathcurveto{\pgfqpoint{2.497137in}{1.197135in}}{\pgfqpoint{2.507736in}{1.192744in}}{\pgfqpoint{2.518786in}{1.192744in}}%
\pgfpathclose%
\pgfusepath{stroke,fill}%
\end{pgfscope}%
\begin{pgfscope}%
\pgfpathrectangle{\pgfqpoint{0.800000in}{0.528000in}}{\pgfqpoint{4.960000in}{3.696000in}}%
\pgfusepath{clip}%
\pgfsetbuttcap%
\pgfsetroundjoin%
\definecolor{currentfill}{rgb}{0.000000,0.000000,0.000000}%
\pgfsetfillcolor{currentfill}%
\pgfsetlinewidth{1.003750pt}%
\definecolor{currentstroke}{rgb}{0.000000,0.000000,0.000000}%
\pgfsetstrokecolor{currentstroke}%
\pgfsetdash{}{0pt}%
\pgfpathmoveto{\pgfqpoint{2.518786in}{1.149749in}}%
\pgfpathcurveto{\pgfqpoint{2.529836in}{1.149749in}}{\pgfqpoint{2.540435in}{1.154140in}}{\pgfqpoint{2.548249in}{1.161953in}}%
\pgfpathcurveto{\pgfqpoint{2.556062in}{1.169767in}}{\pgfqpoint{2.560452in}{1.180366in}}{\pgfqpoint{2.560452in}{1.191416in}}%
\pgfpathcurveto{\pgfqpoint{2.560452in}{1.202466in}}{\pgfqpoint{2.556062in}{1.213065in}}{\pgfqpoint{2.548249in}{1.220879in}}%
\pgfpathcurveto{\pgfqpoint{2.540435in}{1.228692in}}{\pgfqpoint{2.529836in}{1.233083in}}{\pgfqpoint{2.518786in}{1.233083in}}%
\pgfpathcurveto{\pgfqpoint{2.507736in}{1.233083in}}{\pgfqpoint{2.497137in}{1.228692in}}{\pgfqpoint{2.489323in}{1.220879in}}%
\pgfpathcurveto{\pgfqpoint{2.481509in}{1.213065in}}{\pgfqpoint{2.477119in}{1.202466in}}{\pgfqpoint{2.477119in}{1.191416in}}%
\pgfpathcurveto{\pgfqpoint{2.477119in}{1.180366in}}{\pgfqpoint{2.481509in}{1.169767in}}{\pgfqpoint{2.489323in}{1.161953in}}%
\pgfpathcurveto{\pgfqpoint{2.497137in}{1.154140in}}{\pgfqpoint{2.507736in}{1.149749in}}{\pgfqpoint{2.518786in}{1.149749in}}%
\pgfpathclose%
\pgfusepath{stroke,fill}%
\end{pgfscope}%
\begin{pgfscope}%
\pgfpathrectangle{\pgfqpoint{0.800000in}{0.528000in}}{\pgfqpoint{4.960000in}{3.696000in}}%
\pgfusepath{clip}%
\pgfsetbuttcap%
\pgfsetroundjoin%
\definecolor{currentfill}{rgb}{0.000000,0.000000,0.000000}%
\pgfsetfillcolor{currentfill}%
\pgfsetlinewidth{1.003750pt}%
\definecolor{currentstroke}{rgb}{0.000000,0.000000,0.000000}%
\pgfsetstrokecolor{currentstroke}%
\pgfsetdash{}{0pt}%
\pgfpathmoveto{\pgfqpoint{2.518786in}{1.128252in}}%
\pgfpathcurveto{\pgfqpoint{2.529836in}{1.128252in}}{\pgfqpoint{2.540435in}{1.132642in}}{\pgfqpoint{2.548249in}{1.140456in}}%
\pgfpathcurveto{\pgfqpoint{2.556062in}{1.148269in}}{\pgfqpoint{2.560452in}{1.158868in}}{\pgfqpoint{2.560452in}{1.169918in}}%
\pgfpathcurveto{\pgfqpoint{2.560452in}{1.180969in}}{\pgfqpoint{2.556062in}{1.191568in}}{\pgfqpoint{2.548249in}{1.199381in}}%
\pgfpathcurveto{\pgfqpoint{2.540435in}{1.207195in}}{\pgfqpoint{2.529836in}{1.211585in}}{\pgfqpoint{2.518786in}{1.211585in}}%
\pgfpathcurveto{\pgfqpoint{2.507736in}{1.211585in}}{\pgfqpoint{2.497137in}{1.207195in}}{\pgfqpoint{2.489323in}{1.199381in}}%
\pgfpathcurveto{\pgfqpoint{2.481509in}{1.191568in}}{\pgfqpoint{2.477119in}{1.180969in}}{\pgfqpoint{2.477119in}{1.169918in}}%
\pgfpathcurveto{\pgfqpoint{2.477119in}{1.158868in}}{\pgfqpoint{2.481509in}{1.148269in}}{\pgfqpoint{2.489323in}{1.140456in}}%
\pgfpathcurveto{\pgfqpoint{2.497137in}{1.132642in}}{\pgfqpoint{2.507736in}{1.128252in}}{\pgfqpoint{2.518786in}{1.128252in}}%
\pgfpathclose%
\pgfusepath{stroke,fill}%
\end{pgfscope}%
\begin{pgfscope}%
\pgfpathrectangle{\pgfqpoint{0.800000in}{0.528000in}}{\pgfqpoint{4.960000in}{3.696000in}}%
\pgfusepath{clip}%
\pgfsetbuttcap%
\pgfsetroundjoin%
\definecolor{currentfill}{rgb}{0.000000,0.000000,0.000000}%
\pgfsetfillcolor{currentfill}%
\pgfsetlinewidth{1.003750pt}%
\definecolor{currentstroke}{rgb}{0.000000,0.000000,0.000000}%
\pgfsetstrokecolor{currentstroke}%
\pgfsetdash{}{0pt}%
\pgfpathmoveto{\pgfqpoint{2.518786in}{1.106754in}}%
\pgfpathcurveto{\pgfqpoint{2.529836in}{1.106754in}}{\pgfqpoint{2.540435in}{1.111145in}}{\pgfqpoint{2.548249in}{1.118958in}}%
\pgfpathcurveto{\pgfqpoint{2.556062in}{1.126772in}}{\pgfqpoint{2.560452in}{1.137371in}}{\pgfqpoint{2.560452in}{1.148421in}}%
\pgfpathcurveto{\pgfqpoint{2.560452in}{1.159471in}}{\pgfqpoint{2.556062in}{1.170070in}}{\pgfqpoint{2.548249in}{1.177884in}}%
\pgfpathcurveto{\pgfqpoint{2.540435in}{1.185697in}}{\pgfqpoint{2.529836in}{1.190088in}}{\pgfqpoint{2.518786in}{1.190088in}}%
\pgfpathcurveto{\pgfqpoint{2.507736in}{1.190088in}}{\pgfqpoint{2.497137in}{1.185697in}}{\pgfqpoint{2.489323in}{1.177884in}}%
\pgfpathcurveto{\pgfqpoint{2.481509in}{1.170070in}}{\pgfqpoint{2.477119in}{1.159471in}}{\pgfqpoint{2.477119in}{1.148421in}}%
\pgfpathcurveto{\pgfqpoint{2.477119in}{1.137371in}}{\pgfqpoint{2.481509in}{1.126772in}}{\pgfqpoint{2.489323in}{1.118958in}}%
\pgfpathcurveto{\pgfqpoint{2.497137in}{1.111145in}}{\pgfqpoint{2.507736in}{1.106754in}}{\pgfqpoint{2.518786in}{1.106754in}}%
\pgfpathclose%
\pgfusepath{stroke,fill}%
\end{pgfscope}%
\begin{pgfscope}%
\pgfpathrectangle{\pgfqpoint{0.800000in}{0.528000in}}{\pgfqpoint{4.960000in}{3.696000in}}%
\pgfusepath{clip}%
\pgfsetbuttcap%
\pgfsetroundjoin%
\definecolor{currentfill}{rgb}{0.000000,0.000000,0.000000}%
\pgfsetfillcolor{currentfill}%
\pgfsetlinewidth{1.003750pt}%
\definecolor{currentstroke}{rgb}{0.000000,0.000000,0.000000}%
\pgfsetstrokecolor{currentstroke}%
\pgfsetdash{}{0pt}%
\pgfpathmoveto{\pgfqpoint{2.518786in}{1.128252in}}%
\pgfpathcurveto{\pgfqpoint{2.529836in}{1.128252in}}{\pgfqpoint{2.540435in}{1.132642in}}{\pgfqpoint{2.548249in}{1.140456in}}%
\pgfpathcurveto{\pgfqpoint{2.556062in}{1.148269in}}{\pgfqpoint{2.560452in}{1.158868in}}{\pgfqpoint{2.560452in}{1.169918in}}%
\pgfpathcurveto{\pgfqpoint{2.560452in}{1.180969in}}{\pgfqpoint{2.556062in}{1.191568in}}{\pgfqpoint{2.548249in}{1.199381in}}%
\pgfpathcurveto{\pgfqpoint{2.540435in}{1.207195in}}{\pgfqpoint{2.529836in}{1.211585in}}{\pgfqpoint{2.518786in}{1.211585in}}%
\pgfpathcurveto{\pgfqpoint{2.507736in}{1.211585in}}{\pgfqpoint{2.497137in}{1.207195in}}{\pgfqpoint{2.489323in}{1.199381in}}%
\pgfpathcurveto{\pgfqpoint{2.481509in}{1.191568in}}{\pgfqpoint{2.477119in}{1.180969in}}{\pgfqpoint{2.477119in}{1.169918in}}%
\pgfpathcurveto{\pgfqpoint{2.477119in}{1.158868in}}{\pgfqpoint{2.481509in}{1.148269in}}{\pgfqpoint{2.489323in}{1.140456in}}%
\pgfpathcurveto{\pgfqpoint{2.497137in}{1.132642in}}{\pgfqpoint{2.507736in}{1.128252in}}{\pgfqpoint{2.518786in}{1.128252in}}%
\pgfpathclose%
\pgfusepath{stroke,fill}%
\end{pgfscope}%
\begin{pgfscope}%
\pgfpathrectangle{\pgfqpoint{0.800000in}{0.528000in}}{\pgfqpoint{4.960000in}{3.696000in}}%
\pgfusepath{clip}%
\pgfsetbuttcap%
\pgfsetroundjoin%
\definecolor{currentfill}{rgb}{0.000000,0.000000,0.000000}%
\pgfsetfillcolor{currentfill}%
\pgfsetlinewidth{1.003750pt}%
\definecolor{currentstroke}{rgb}{0.000000,0.000000,0.000000}%
\pgfsetstrokecolor{currentstroke}%
\pgfsetdash{}{0pt}%
\pgfpathmoveto{\pgfqpoint{2.518786in}{1.128252in}}%
\pgfpathcurveto{\pgfqpoint{2.529836in}{1.128252in}}{\pgfqpoint{2.540435in}{1.132642in}}{\pgfqpoint{2.548249in}{1.140456in}}%
\pgfpathcurveto{\pgfqpoint{2.556062in}{1.148269in}}{\pgfqpoint{2.560452in}{1.158868in}}{\pgfqpoint{2.560452in}{1.169918in}}%
\pgfpathcurveto{\pgfqpoint{2.560452in}{1.180969in}}{\pgfqpoint{2.556062in}{1.191568in}}{\pgfqpoint{2.548249in}{1.199381in}}%
\pgfpathcurveto{\pgfqpoint{2.540435in}{1.207195in}}{\pgfqpoint{2.529836in}{1.211585in}}{\pgfqpoint{2.518786in}{1.211585in}}%
\pgfpathcurveto{\pgfqpoint{2.507736in}{1.211585in}}{\pgfqpoint{2.497137in}{1.207195in}}{\pgfqpoint{2.489323in}{1.199381in}}%
\pgfpathcurveto{\pgfqpoint{2.481509in}{1.191568in}}{\pgfqpoint{2.477119in}{1.180969in}}{\pgfqpoint{2.477119in}{1.169918in}}%
\pgfpathcurveto{\pgfqpoint{2.477119in}{1.158868in}}{\pgfqpoint{2.481509in}{1.148269in}}{\pgfqpoint{2.489323in}{1.140456in}}%
\pgfpathcurveto{\pgfqpoint{2.497137in}{1.132642in}}{\pgfqpoint{2.507736in}{1.128252in}}{\pgfqpoint{2.518786in}{1.128252in}}%
\pgfpathclose%
\pgfusepath{stroke,fill}%
\end{pgfscope}%
\begin{pgfscope}%
\pgfpathrectangle{\pgfqpoint{0.800000in}{0.528000in}}{\pgfqpoint{4.960000in}{3.696000in}}%
\pgfusepath{clip}%
\pgfsetbuttcap%
\pgfsetroundjoin%
\definecolor{currentfill}{rgb}{0.000000,0.000000,0.000000}%
\pgfsetfillcolor{currentfill}%
\pgfsetlinewidth{1.003750pt}%
\definecolor{currentstroke}{rgb}{0.000000,0.000000,0.000000}%
\pgfsetstrokecolor{currentstroke}%
\pgfsetdash{}{0pt}%
\pgfpathmoveto{\pgfqpoint{2.518786in}{1.085257in}}%
\pgfpathcurveto{\pgfqpoint{2.529836in}{1.085257in}}{\pgfqpoint{2.540435in}{1.089647in}}{\pgfqpoint{2.548249in}{1.097461in}}%
\pgfpathcurveto{\pgfqpoint{2.556062in}{1.105274in}}{\pgfqpoint{2.560452in}{1.115873in}}{\pgfqpoint{2.560452in}{1.126923in}}%
\pgfpathcurveto{\pgfqpoint{2.560452in}{1.137974in}}{\pgfqpoint{2.556062in}{1.148573in}}{\pgfqpoint{2.548249in}{1.156386in}}%
\pgfpathcurveto{\pgfqpoint{2.540435in}{1.164200in}}{\pgfqpoint{2.529836in}{1.168590in}}{\pgfqpoint{2.518786in}{1.168590in}}%
\pgfpathcurveto{\pgfqpoint{2.507736in}{1.168590in}}{\pgfqpoint{2.497137in}{1.164200in}}{\pgfqpoint{2.489323in}{1.156386in}}%
\pgfpathcurveto{\pgfqpoint{2.481509in}{1.148573in}}{\pgfqpoint{2.477119in}{1.137974in}}{\pgfqpoint{2.477119in}{1.126923in}}%
\pgfpathcurveto{\pgfqpoint{2.477119in}{1.115873in}}{\pgfqpoint{2.481509in}{1.105274in}}{\pgfqpoint{2.489323in}{1.097461in}}%
\pgfpathcurveto{\pgfqpoint{2.497137in}{1.089647in}}{\pgfqpoint{2.507736in}{1.085257in}}{\pgfqpoint{2.518786in}{1.085257in}}%
\pgfpathclose%
\pgfusepath{stroke,fill}%
\end{pgfscope}%
\begin{pgfscope}%
\pgfpathrectangle{\pgfqpoint{0.800000in}{0.528000in}}{\pgfqpoint{4.960000in}{3.696000in}}%
\pgfusepath{clip}%
\pgfsetbuttcap%
\pgfsetroundjoin%
\definecolor{currentfill}{rgb}{0.000000,0.000000,0.000000}%
\pgfsetfillcolor{currentfill}%
\pgfsetlinewidth{1.003750pt}%
\definecolor{currentstroke}{rgb}{0.000000,0.000000,0.000000}%
\pgfsetstrokecolor{currentstroke}%
\pgfsetdash{}{0pt}%
\pgfpathmoveto{\pgfqpoint{2.518786in}{1.085257in}}%
\pgfpathcurveto{\pgfqpoint{2.529836in}{1.085257in}}{\pgfqpoint{2.540435in}{1.089647in}}{\pgfqpoint{2.548249in}{1.097461in}}%
\pgfpathcurveto{\pgfqpoint{2.556062in}{1.105274in}}{\pgfqpoint{2.560452in}{1.115873in}}{\pgfqpoint{2.560452in}{1.126923in}}%
\pgfpathcurveto{\pgfqpoint{2.560452in}{1.137974in}}{\pgfqpoint{2.556062in}{1.148573in}}{\pgfqpoint{2.548249in}{1.156386in}}%
\pgfpathcurveto{\pgfqpoint{2.540435in}{1.164200in}}{\pgfqpoint{2.529836in}{1.168590in}}{\pgfqpoint{2.518786in}{1.168590in}}%
\pgfpathcurveto{\pgfqpoint{2.507736in}{1.168590in}}{\pgfqpoint{2.497137in}{1.164200in}}{\pgfqpoint{2.489323in}{1.156386in}}%
\pgfpathcurveto{\pgfqpoint{2.481509in}{1.148573in}}{\pgfqpoint{2.477119in}{1.137974in}}{\pgfqpoint{2.477119in}{1.126923in}}%
\pgfpathcurveto{\pgfqpoint{2.477119in}{1.115873in}}{\pgfqpoint{2.481509in}{1.105274in}}{\pgfqpoint{2.489323in}{1.097461in}}%
\pgfpathcurveto{\pgfqpoint{2.497137in}{1.089647in}}{\pgfqpoint{2.507736in}{1.085257in}}{\pgfqpoint{2.518786in}{1.085257in}}%
\pgfpathclose%
\pgfusepath{stroke,fill}%
\end{pgfscope}%
\begin{pgfscope}%
\pgfpathrectangle{\pgfqpoint{0.800000in}{0.528000in}}{\pgfqpoint{4.960000in}{3.696000in}}%
\pgfusepath{clip}%
\pgfsetbuttcap%
\pgfsetroundjoin%
\definecolor{currentfill}{rgb}{0.000000,0.000000,0.000000}%
\pgfsetfillcolor{currentfill}%
\pgfsetlinewidth{1.003750pt}%
\definecolor{currentstroke}{rgb}{0.000000,0.000000,0.000000}%
\pgfsetstrokecolor{currentstroke}%
\pgfsetdash{}{0pt}%
\pgfpathmoveto{\pgfqpoint{2.518786in}{1.149749in}}%
\pgfpathcurveto{\pgfqpoint{2.529836in}{1.149749in}}{\pgfqpoint{2.540435in}{1.154140in}}{\pgfqpoint{2.548249in}{1.161953in}}%
\pgfpathcurveto{\pgfqpoint{2.556062in}{1.169767in}}{\pgfqpoint{2.560452in}{1.180366in}}{\pgfqpoint{2.560452in}{1.191416in}}%
\pgfpathcurveto{\pgfqpoint{2.560452in}{1.202466in}}{\pgfqpoint{2.556062in}{1.213065in}}{\pgfqpoint{2.548249in}{1.220879in}}%
\pgfpathcurveto{\pgfqpoint{2.540435in}{1.228692in}}{\pgfqpoint{2.529836in}{1.233083in}}{\pgfqpoint{2.518786in}{1.233083in}}%
\pgfpathcurveto{\pgfqpoint{2.507736in}{1.233083in}}{\pgfqpoint{2.497137in}{1.228692in}}{\pgfqpoint{2.489323in}{1.220879in}}%
\pgfpathcurveto{\pgfqpoint{2.481509in}{1.213065in}}{\pgfqpoint{2.477119in}{1.202466in}}{\pgfqpoint{2.477119in}{1.191416in}}%
\pgfpathcurveto{\pgfqpoint{2.477119in}{1.180366in}}{\pgfqpoint{2.481509in}{1.169767in}}{\pgfqpoint{2.489323in}{1.161953in}}%
\pgfpathcurveto{\pgfqpoint{2.497137in}{1.154140in}}{\pgfqpoint{2.507736in}{1.149749in}}{\pgfqpoint{2.518786in}{1.149749in}}%
\pgfpathclose%
\pgfusepath{stroke,fill}%
\end{pgfscope}%
\begin{pgfscope}%
\pgfpathrectangle{\pgfqpoint{0.800000in}{0.528000in}}{\pgfqpoint{4.960000in}{3.696000in}}%
\pgfusepath{clip}%
\pgfsetbuttcap%
\pgfsetroundjoin%
\definecolor{currentfill}{rgb}{0.000000,0.000000,0.000000}%
\pgfsetfillcolor{currentfill}%
\pgfsetlinewidth{1.003750pt}%
\definecolor{currentstroke}{rgb}{0.000000,0.000000,0.000000}%
\pgfsetstrokecolor{currentstroke}%
\pgfsetdash{}{0pt}%
\pgfpathmoveto{\pgfqpoint{2.518786in}{1.106754in}}%
\pgfpathcurveto{\pgfqpoint{2.529836in}{1.106754in}}{\pgfqpoint{2.540435in}{1.111145in}}{\pgfqpoint{2.548249in}{1.118958in}}%
\pgfpathcurveto{\pgfqpoint{2.556062in}{1.126772in}}{\pgfqpoint{2.560452in}{1.137371in}}{\pgfqpoint{2.560452in}{1.148421in}}%
\pgfpathcurveto{\pgfqpoint{2.560452in}{1.159471in}}{\pgfqpoint{2.556062in}{1.170070in}}{\pgfqpoint{2.548249in}{1.177884in}}%
\pgfpathcurveto{\pgfqpoint{2.540435in}{1.185697in}}{\pgfqpoint{2.529836in}{1.190088in}}{\pgfqpoint{2.518786in}{1.190088in}}%
\pgfpathcurveto{\pgfqpoint{2.507736in}{1.190088in}}{\pgfqpoint{2.497137in}{1.185697in}}{\pgfqpoint{2.489323in}{1.177884in}}%
\pgfpathcurveto{\pgfqpoint{2.481509in}{1.170070in}}{\pgfqpoint{2.477119in}{1.159471in}}{\pgfqpoint{2.477119in}{1.148421in}}%
\pgfpathcurveto{\pgfqpoint{2.477119in}{1.137371in}}{\pgfqpoint{2.481509in}{1.126772in}}{\pgfqpoint{2.489323in}{1.118958in}}%
\pgfpathcurveto{\pgfqpoint{2.497137in}{1.111145in}}{\pgfqpoint{2.507736in}{1.106754in}}{\pgfqpoint{2.518786in}{1.106754in}}%
\pgfpathclose%
\pgfusepath{stroke,fill}%
\end{pgfscope}%
\begin{pgfscope}%
\pgfpathrectangle{\pgfqpoint{0.800000in}{0.528000in}}{\pgfqpoint{4.960000in}{3.696000in}}%
\pgfusepath{clip}%
\pgfsetbuttcap%
\pgfsetroundjoin%
\definecolor{currentfill}{rgb}{0.000000,0.000000,0.000000}%
\pgfsetfillcolor{currentfill}%
\pgfsetlinewidth{1.003750pt}%
\definecolor{currentstroke}{rgb}{0.000000,0.000000,0.000000}%
\pgfsetstrokecolor{currentstroke}%
\pgfsetdash{}{0pt}%
\pgfpathmoveto{\pgfqpoint{2.518786in}{1.106754in}}%
\pgfpathcurveto{\pgfqpoint{2.529836in}{1.106754in}}{\pgfqpoint{2.540435in}{1.111145in}}{\pgfqpoint{2.548249in}{1.118958in}}%
\pgfpathcurveto{\pgfqpoint{2.556062in}{1.126772in}}{\pgfqpoint{2.560452in}{1.137371in}}{\pgfqpoint{2.560452in}{1.148421in}}%
\pgfpathcurveto{\pgfqpoint{2.560452in}{1.159471in}}{\pgfqpoint{2.556062in}{1.170070in}}{\pgfqpoint{2.548249in}{1.177884in}}%
\pgfpathcurveto{\pgfqpoint{2.540435in}{1.185697in}}{\pgfqpoint{2.529836in}{1.190088in}}{\pgfqpoint{2.518786in}{1.190088in}}%
\pgfpathcurveto{\pgfqpoint{2.507736in}{1.190088in}}{\pgfqpoint{2.497137in}{1.185697in}}{\pgfqpoint{2.489323in}{1.177884in}}%
\pgfpathcurveto{\pgfqpoint{2.481509in}{1.170070in}}{\pgfqpoint{2.477119in}{1.159471in}}{\pgfqpoint{2.477119in}{1.148421in}}%
\pgfpathcurveto{\pgfqpoint{2.477119in}{1.137371in}}{\pgfqpoint{2.481509in}{1.126772in}}{\pgfqpoint{2.489323in}{1.118958in}}%
\pgfpathcurveto{\pgfqpoint{2.497137in}{1.111145in}}{\pgfqpoint{2.507736in}{1.106754in}}{\pgfqpoint{2.518786in}{1.106754in}}%
\pgfpathclose%
\pgfusepath{stroke,fill}%
\end{pgfscope}%
\begin{pgfscope}%
\pgfpathrectangle{\pgfqpoint{0.800000in}{0.528000in}}{\pgfqpoint{4.960000in}{3.696000in}}%
\pgfusepath{clip}%
\pgfsetbuttcap%
\pgfsetroundjoin%
\definecolor{currentfill}{rgb}{0.000000,0.000000,0.000000}%
\pgfsetfillcolor{currentfill}%
\pgfsetlinewidth{1.003750pt}%
\definecolor{currentstroke}{rgb}{0.000000,0.000000,0.000000}%
\pgfsetstrokecolor{currentstroke}%
\pgfsetdash{}{0pt}%
\pgfpathmoveto{\pgfqpoint{2.518786in}{1.171247in}}%
\pgfpathcurveto{\pgfqpoint{2.529836in}{1.171247in}}{\pgfqpoint{2.540435in}{1.175637in}}{\pgfqpoint{2.548249in}{1.183451in}}%
\pgfpathcurveto{\pgfqpoint{2.556062in}{1.191264in}}{\pgfqpoint{2.560452in}{1.201863in}}{\pgfqpoint{2.560452in}{1.212913in}}%
\pgfpathcurveto{\pgfqpoint{2.560452in}{1.223964in}}{\pgfqpoint{2.556062in}{1.234563in}}{\pgfqpoint{2.548249in}{1.242376in}}%
\pgfpathcurveto{\pgfqpoint{2.540435in}{1.250190in}}{\pgfqpoint{2.529836in}{1.254580in}}{\pgfqpoint{2.518786in}{1.254580in}}%
\pgfpathcurveto{\pgfqpoint{2.507736in}{1.254580in}}{\pgfqpoint{2.497137in}{1.250190in}}{\pgfqpoint{2.489323in}{1.242376in}}%
\pgfpathcurveto{\pgfqpoint{2.481509in}{1.234563in}}{\pgfqpoint{2.477119in}{1.223964in}}{\pgfqpoint{2.477119in}{1.212913in}}%
\pgfpathcurveto{\pgfqpoint{2.477119in}{1.201863in}}{\pgfqpoint{2.481509in}{1.191264in}}{\pgfqpoint{2.489323in}{1.183451in}}%
\pgfpathcurveto{\pgfqpoint{2.497137in}{1.175637in}}{\pgfqpoint{2.507736in}{1.171247in}}{\pgfqpoint{2.518786in}{1.171247in}}%
\pgfpathclose%
\pgfusepath{stroke,fill}%
\end{pgfscope}%
\begin{pgfscope}%
\pgfpathrectangle{\pgfqpoint{0.800000in}{0.528000in}}{\pgfqpoint{4.960000in}{3.696000in}}%
\pgfusepath{clip}%
\pgfsetbuttcap%
\pgfsetroundjoin%
\definecolor{currentfill}{rgb}{0.000000,0.000000,0.000000}%
\pgfsetfillcolor{currentfill}%
\pgfsetlinewidth{1.003750pt}%
\definecolor{currentstroke}{rgb}{0.000000,0.000000,0.000000}%
\pgfsetstrokecolor{currentstroke}%
\pgfsetdash{}{0pt}%
\pgfpathmoveto{\pgfqpoint{2.518786in}{1.171247in}}%
\pgfpathcurveto{\pgfqpoint{2.529836in}{1.171247in}}{\pgfqpoint{2.540435in}{1.175637in}}{\pgfqpoint{2.548249in}{1.183451in}}%
\pgfpathcurveto{\pgfqpoint{2.556062in}{1.191264in}}{\pgfqpoint{2.560452in}{1.201863in}}{\pgfqpoint{2.560452in}{1.212913in}}%
\pgfpathcurveto{\pgfqpoint{2.560452in}{1.223964in}}{\pgfqpoint{2.556062in}{1.234563in}}{\pgfqpoint{2.548249in}{1.242376in}}%
\pgfpathcurveto{\pgfqpoint{2.540435in}{1.250190in}}{\pgfqpoint{2.529836in}{1.254580in}}{\pgfqpoint{2.518786in}{1.254580in}}%
\pgfpathcurveto{\pgfqpoint{2.507736in}{1.254580in}}{\pgfqpoint{2.497137in}{1.250190in}}{\pgfqpoint{2.489323in}{1.242376in}}%
\pgfpathcurveto{\pgfqpoint{2.481509in}{1.234563in}}{\pgfqpoint{2.477119in}{1.223964in}}{\pgfqpoint{2.477119in}{1.212913in}}%
\pgfpathcurveto{\pgfqpoint{2.477119in}{1.201863in}}{\pgfqpoint{2.481509in}{1.191264in}}{\pgfqpoint{2.489323in}{1.183451in}}%
\pgfpathcurveto{\pgfqpoint{2.497137in}{1.175637in}}{\pgfqpoint{2.507736in}{1.171247in}}{\pgfqpoint{2.518786in}{1.171247in}}%
\pgfpathclose%
\pgfusepath{stroke,fill}%
\end{pgfscope}%
\begin{pgfscope}%
\pgfpathrectangle{\pgfqpoint{0.800000in}{0.528000in}}{\pgfqpoint{4.960000in}{3.696000in}}%
\pgfusepath{clip}%
\pgfsetbuttcap%
\pgfsetroundjoin%
\definecolor{currentfill}{rgb}{0.000000,0.000000,0.000000}%
\pgfsetfillcolor{currentfill}%
\pgfsetlinewidth{1.003750pt}%
\definecolor{currentstroke}{rgb}{0.000000,0.000000,0.000000}%
\pgfsetstrokecolor{currentstroke}%
\pgfsetdash{}{0pt}%
\pgfpathmoveto{\pgfqpoint{2.518786in}{1.063759in}}%
\pgfpathcurveto{\pgfqpoint{2.529836in}{1.063759in}}{\pgfqpoint{2.540435in}{1.068150in}}{\pgfqpoint{2.548249in}{1.075963in}}%
\pgfpathcurveto{\pgfqpoint{2.556062in}{1.083777in}}{\pgfqpoint{2.560452in}{1.094376in}}{\pgfqpoint{2.560452in}{1.105426in}}%
\pgfpathcurveto{\pgfqpoint{2.560452in}{1.116476in}}{\pgfqpoint{2.556062in}{1.127075in}}{\pgfqpoint{2.548249in}{1.134889in}}%
\pgfpathcurveto{\pgfqpoint{2.540435in}{1.142702in}}{\pgfqpoint{2.529836in}{1.147093in}}{\pgfqpoint{2.518786in}{1.147093in}}%
\pgfpathcurveto{\pgfqpoint{2.507736in}{1.147093in}}{\pgfqpoint{2.497137in}{1.142702in}}{\pgfqpoint{2.489323in}{1.134889in}}%
\pgfpathcurveto{\pgfqpoint{2.481509in}{1.127075in}}{\pgfqpoint{2.477119in}{1.116476in}}{\pgfqpoint{2.477119in}{1.105426in}}%
\pgfpathcurveto{\pgfqpoint{2.477119in}{1.094376in}}{\pgfqpoint{2.481509in}{1.083777in}}{\pgfqpoint{2.489323in}{1.075963in}}%
\pgfpathcurveto{\pgfqpoint{2.497137in}{1.068150in}}{\pgfqpoint{2.507736in}{1.063759in}}{\pgfqpoint{2.518786in}{1.063759in}}%
\pgfpathclose%
\pgfusepath{stroke,fill}%
\end{pgfscope}%
\begin{pgfscope}%
\pgfpathrectangle{\pgfqpoint{0.800000in}{0.528000in}}{\pgfqpoint{4.960000in}{3.696000in}}%
\pgfusepath{clip}%
\pgfsetbuttcap%
\pgfsetroundjoin%
\definecolor{currentfill}{rgb}{0.000000,0.000000,0.000000}%
\pgfsetfillcolor{currentfill}%
\pgfsetlinewidth{1.003750pt}%
\definecolor{currentstroke}{rgb}{0.000000,0.000000,0.000000}%
\pgfsetstrokecolor{currentstroke}%
\pgfsetdash{}{0pt}%
\pgfpathmoveto{\pgfqpoint{2.518786in}{1.192744in}}%
\pgfpathcurveto{\pgfqpoint{2.529836in}{1.192744in}}{\pgfqpoint{2.540435in}{1.197135in}}{\pgfqpoint{2.548249in}{1.204948in}}%
\pgfpathcurveto{\pgfqpoint{2.556062in}{1.212762in}}{\pgfqpoint{2.560452in}{1.223361in}}{\pgfqpoint{2.560452in}{1.234411in}}%
\pgfpathcurveto{\pgfqpoint{2.560452in}{1.245461in}}{\pgfqpoint{2.556062in}{1.256060in}}{\pgfqpoint{2.548249in}{1.263874in}}%
\pgfpathcurveto{\pgfqpoint{2.540435in}{1.271687in}}{\pgfqpoint{2.529836in}{1.276078in}}{\pgfqpoint{2.518786in}{1.276078in}}%
\pgfpathcurveto{\pgfqpoint{2.507736in}{1.276078in}}{\pgfqpoint{2.497137in}{1.271687in}}{\pgfqpoint{2.489323in}{1.263874in}}%
\pgfpathcurveto{\pgfqpoint{2.481509in}{1.256060in}}{\pgfqpoint{2.477119in}{1.245461in}}{\pgfqpoint{2.477119in}{1.234411in}}%
\pgfpathcurveto{\pgfqpoint{2.477119in}{1.223361in}}{\pgfqpoint{2.481509in}{1.212762in}}{\pgfqpoint{2.489323in}{1.204948in}}%
\pgfpathcurveto{\pgfqpoint{2.497137in}{1.197135in}}{\pgfqpoint{2.507736in}{1.192744in}}{\pgfqpoint{2.518786in}{1.192744in}}%
\pgfpathclose%
\pgfusepath{stroke,fill}%
\end{pgfscope}%
\begin{pgfscope}%
\pgfpathrectangle{\pgfqpoint{0.800000in}{0.528000in}}{\pgfqpoint{4.960000in}{3.696000in}}%
\pgfusepath{clip}%
\pgfsetbuttcap%
\pgfsetroundjoin%
\definecolor{currentfill}{rgb}{0.000000,0.000000,0.000000}%
\pgfsetfillcolor{currentfill}%
\pgfsetlinewidth{1.003750pt}%
\definecolor{currentstroke}{rgb}{0.000000,0.000000,0.000000}%
\pgfsetstrokecolor{currentstroke}%
\pgfsetdash{}{0pt}%
\pgfpathmoveto{\pgfqpoint{2.518786in}{1.278734in}}%
\pgfpathcurveto{\pgfqpoint{2.529836in}{1.278734in}}{\pgfqpoint{2.540435in}{1.283125in}}{\pgfqpoint{2.548249in}{1.290938in}}%
\pgfpathcurveto{\pgfqpoint{2.556062in}{1.298752in}}{\pgfqpoint{2.560452in}{1.309351in}}{\pgfqpoint{2.560452in}{1.320401in}}%
\pgfpathcurveto{\pgfqpoint{2.560452in}{1.331451in}}{\pgfqpoint{2.556062in}{1.342050in}}{\pgfqpoint{2.548249in}{1.349864in}}%
\pgfpathcurveto{\pgfqpoint{2.540435in}{1.357677in}}{\pgfqpoint{2.529836in}{1.362068in}}{\pgfqpoint{2.518786in}{1.362068in}}%
\pgfpathcurveto{\pgfqpoint{2.507736in}{1.362068in}}{\pgfqpoint{2.497137in}{1.357677in}}{\pgfqpoint{2.489323in}{1.349864in}}%
\pgfpathcurveto{\pgfqpoint{2.481509in}{1.342050in}}{\pgfqpoint{2.477119in}{1.331451in}}{\pgfqpoint{2.477119in}{1.320401in}}%
\pgfpathcurveto{\pgfqpoint{2.477119in}{1.309351in}}{\pgfqpoint{2.481509in}{1.298752in}}{\pgfqpoint{2.489323in}{1.290938in}}%
\pgfpathcurveto{\pgfqpoint{2.497137in}{1.283125in}}{\pgfqpoint{2.507736in}{1.278734in}}{\pgfqpoint{2.518786in}{1.278734in}}%
\pgfpathclose%
\pgfusepath{stroke,fill}%
\end{pgfscope}%
\begin{pgfscope}%
\pgfpathrectangle{\pgfqpoint{0.800000in}{0.528000in}}{\pgfqpoint{4.960000in}{3.696000in}}%
\pgfusepath{clip}%
\pgfsetbuttcap%
\pgfsetroundjoin%
\definecolor{currentfill}{rgb}{0.000000,0.000000,0.000000}%
\pgfsetfillcolor{currentfill}%
\pgfsetlinewidth{1.003750pt}%
\definecolor{currentstroke}{rgb}{0.000000,0.000000,0.000000}%
\pgfsetstrokecolor{currentstroke}%
\pgfsetdash{}{0pt}%
\pgfpathmoveto{\pgfqpoint{2.518786in}{1.149749in}}%
\pgfpathcurveto{\pgfqpoint{2.529836in}{1.149749in}}{\pgfqpoint{2.540435in}{1.154140in}}{\pgfqpoint{2.548249in}{1.161953in}}%
\pgfpathcurveto{\pgfqpoint{2.556062in}{1.169767in}}{\pgfqpoint{2.560452in}{1.180366in}}{\pgfqpoint{2.560452in}{1.191416in}}%
\pgfpathcurveto{\pgfqpoint{2.560452in}{1.202466in}}{\pgfqpoint{2.556062in}{1.213065in}}{\pgfqpoint{2.548249in}{1.220879in}}%
\pgfpathcurveto{\pgfqpoint{2.540435in}{1.228692in}}{\pgfqpoint{2.529836in}{1.233083in}}{\pgfqpoint{2.518786in}{1.233083in}}%
\pgfpathcurveto{\pgfqpoint{2.507736in}{1.233083in}}{\pgfqpoint{2.497137in}{1.228692in}}{\pgfqpoint{2.489323in}{1.220879in}}%
\pgfpathcurveto{\pgfqpoint{2.481509in}{1.213065in}}{\pgfqpoint{2.477119in}{1.202466in}}{\pgfqpoint{2.477119in}{1.191416in}}%
\pgfpathcurveto{\pgfqpoint{2.477119in}{1.180366in}}{\pgfqpoint{2.481509in}{1.169767in}}{\pgfqpoint{2.489323in}{1.161953in}}%
\pgfpathcurveto{\pgfqpoint{2.497137in}{1.154140in}}{\pgfqpoint{2.507736in}{1.149749in}}{\pgfqpoint{2.518786in}{1.149749in}}%
\pgfpathclose%
\pgfusepath{stroke,fill}%
\end{pgfscope}%
\begin{pgfscope}%
\pgfpathrectangle{\pgfqpoint{0.800000in}{0.528000in}}{\pgfqpoint{4.960000in}{3.696000in}}%
\pgfusepath{clip}%
\pgfsetbuttcap%
\pgfsetroundjoin%
\definecolor{currentfill}{rgb}{0.000000,0.000000,0.000000}%
\pgfsetfillcolor{currentfill}%
\pgfsetlinewidth{1.003750pt}%
\definecolor{currentstroke}{rgb}{0.000000,0.000000,0.000000}%
\pgfsetstrokecolor{currentstroke}%
\pgfsetdash{}{0pt}%
\pgfpathmoveto{\pgfqpoint{2.518786in}{1.106754in}}%
\pgfpathcurveto{\pgfqpoint{2.529836in}{1.106754in}}{\pgfqpoint{2.540435in}{1.111145in}}{\pgfqpoint{2.548249in}{1.118958in}}%
\pgfpathcurveto{\pgfqpoint{2.556062in}{1.126772in}}{\pgfqpoint{2.560452in}{1.137371in}}{\pgfqpoint{2.560452in}{1.148421in}}%
\pgfpathcurveto{\pgfqpoint{2.560452in}{1.159471in}}{\pgfqpoint{2.556062in}{1.170070in}}{\pgfqpoint{2.548249in}{1.177884in}}%
\pgfpathcurveto{\pgfqpoint{2.540435in}{1.185697in}}{\pgfqpoint{2.529836in}{1.190088in}}{\pgfqpoint{2.518786in}{1.190088in}}%
\pgfpathcurveto{\pgfqpoint{2.507736in}{1.190088in}}{\pgfqpoint{2.497137in}{1.185697in}}{\pgfqpoint{2.489323in}{1.177884in}}%
\pgfpathcurveto{\pgfqpoint{2.481509in}{1.170070in}}{\pgfqpoint{2.477119in}{1.159471in}}{\pgfqpoint{2.477119in}{1.148421in}}%
\pgfpathcurveto{\pgfqpoint{2.477119in}{1.137371in}}{\pgfqpoint{2.481509in}{1.126772in}}{\pgfqpoint{2.489323in}{1.118958in}}%
\pgfpathcurveto{\pgfqpoint{2.497137in}{1.111145in}}{\pgfqpoint{2.507736in}{1.106754in}}{\pgfqpoint{2.518786in}{1.106754in}}%
\pgfpathclose%
\pgfusepath{stroke,fill}%
\end{pgfscope}%
\begin{pgfscope}%
\pgfpathrectangle{\pgfqpoint{0.800000in}{0.528000in}}{\pgfqpoint{4.960000in}{3.696000in}}%
\pgfusepath{clip}%
\pgfsetbuttcap%
\pgfsetroundjoin%
\definecolor{currentfill}{rgb}{0.000000,0.000000,0.000000}%
\pgfsetfillcolor{currentfill}%
\pgfsetlinewidth{1.003750pt}%
\definecolor{currentstroke}{rgb}{0.000000,0.000000,0.000000}%
\pgfsetstrokecolor{currentstroke}%
\pgfsetdash{}{0pt}%
\pgfpathmoveto{\pgfqpoint{2.518786in}{1.171247in}}%
\pgfpathcurveto{\pgfqpoint{2.529836in}{1.171247in}}{\pgfqpoint{2.540435in}{1.175637in}}{\pgfqpoint{2.548249in}{1.183451in}}%
\pgfpathcurveto{\pgfqpoint{2.556062in}{1.191264in}}{\pgfqpoint{2.560452in}{1.201863in}}{\pgfqpoint{2.560452in}{1.212913in}}%
\pgfpathcurveto{\pgfqpoint{2.560452in}{1.223964in}}{\pgfqpoint{2.556062in}{1.234563in}}{\pgfqpoint{2.548249in}{1.242376in}}%
\pgfpathcurveto{\pgfqpoint{2.540435in}{1.250190in}}{\pgfqpoint{2.529836in}{1.254580in}}{\pgfqpoint{2.518786in}{1.254580in}}%
\pgfpathcurveto{\pgfqpoint{2.507736in}{1.254580in}}{\pgfqpoint{2.497137in}{1.250190in}}{\pgfqpoint{2.489323in}{1.242376in}}%
\pgfpathcurveto{\pgfqpoint{2.481509in}{1.234563in}}{\pgfqpoint{2.477119in}{1.223964in}}{\pgfqpoint{2.477119in}{1.212913in}}%
\pgfpathcurveto{\pgfqpoint{2.477119in}{1.201863in}}{\pgfqpoint{2.481509in}{1.191264in}}{\pgfqpoint{2.489323in}{1.183451in}}%
\pgfpathcurveto{\pgfqpoint{2.497137in}{1.175637in}}{\pgfqpoint{2.507736in}{1.171247in}}{\pgfqpoint{2.518786in}{1.171247in}}%
\pgfpathclose%
\pgfusepath{stroke,fill}%
\end{pgfscope}%
\begin{pgfscope}%
\pgfpathrectangle{\pgfqpoint{0.800000in}{0.528000in}}{\pgfqpoint{4.960000in}{3.696000in}}%
\pgfusepath{clip}%
\pgfsetbuttcap%
\pgfsetroundjoin%
\definecolor{currentfill}{rgb}{0.000000,0.000000,0.000000}%
\pgfsetfillcolor{currentfill}%
\pgfsetlinewidth{1.003750pt}%
\definecolor{currentstroke}{rgb}{0.000000,0.000000,0.000000}%
\pgfsetstrokecolor{currentstroke}%
\pgfsetdash{}{0pt}%
\pgfpathmoveto{\pgfqpoint{2.518786in}{1.128252in}}%
\pgfpathcurveto{\pgfqpoint{2.529836in}{1.128252in}}{\pgfqpoint{2.540435in}{1.132642in}}{\pgfqpoint{2.548249in}{1.140456in}}%
\pgfpathcurveto{\pgfqpoint{2.556062in}{1.148269in}}{\pgfqpoint{2.560452in}{1.158868in}}{\pgfqpoint{2.560452in}{1.169918in}}%
\pgfpathcurveto{\pgfqpoint{2.560452in}{1.180969in}}{\pgfqpoint{2.556062in}{1.191568in}}{\pgfqpoint{2.548249in}{1.199381in}}%
\pgfpathcurveto{\pgfqpoint{2.540435in}{1.207195in}}{\pgfqpoint{2.529836in}{1.211585in}}{\pgfqpoint{2.518786in}{1.211585in}}%
\pgfpathcurveto{\pgfqpoint{2.507736in}{1.211585in}}{\pgfqpoint{2.497137in}{1.207195in}}{\pgfqpoint{2.489323in}{1.199381in}}%
\pgfpathcurveto{\pgfqpoint{2.481509in}{1.191568in}}{\pgfqpoint{2.477119in}{1.180969in}}{\pgfqpoint{2.477119in}{1.169918in}}%
\pgfpathcurveto{\pgfqpoint{2.477119in}{1.158868in}}{\pgfqpoint{2.481509in}{1.148269in}}{\pgfqpoint{2.489323in}{1.140456in}}%
\pgfpathcurveto{\pgfqpoint{2.497137in}{1.132642in}}{\pgfqpoint{2.507736in}{1.128252in}}{\pgfqpoint{2.518786in}{1.128252in}}%
\pgfpathclose%
\pgfusepath{stroke,fill}%
\end{pgfscope}%
\begin{pgfscope}%
\pgfpathrectangle{\pgfqpoint{0.800000in}{0.528000in}}{\pgfqpoint{4.960000in}{3.696000in}}%
\pgfusepath{clip}%
\pgfsetbuttcap%
\pgfsetroundjoin%
\definecolor{currentfill}{rgb}{0.000000,0.000000,0.000000}%
\pgfsetfillcolor{currentfill}%
\pgfsetlinewidth{1.003750pt}%
\definecolor{currentstroke}{rgb}{0.000000,0.000000,0.000000}%
\pgfsetstrokecolor{currentstroke}%
\pgfsetdash{}{0pt}%
\pgfpathmoveto{\pgfqpoint{2.518786in}{1.171247in}}%
\pgfpathcurveto{\pgfqpoint{2.529836in}{1.171247in}}{\pgfqpoint{2.540435in}{1.175637in}}{\pgfqpoint{2.548249in}{1.183451in}}%
\pgfpathcurveto{\pgfqpoint{2.556062in}{1.191264in}}{\pgfqpoint{2.560452in}{1.201863in}}{\pgfqpoint{2.560452in}{1.212913in}}%
\pgfpathcurveto{\pgfqpoint{2.560452in}{1.223964in}}{\pgfqpoint{2.556062in}{1.234563in}}{\pgfqpoint{2.548249in}{1.242376in}}%
\pgfpathcurveto{\pgfqpoint{2.540435in}{1.250190in}}{\pgfqpoint{2.529836in}{1.254580in}}{\pgfqpoint{2.518786in}{1.254580in}}%
\pgfpathcurveto{\pgfqpoint{2.507736in}{1.254580in}}{\pgfqpoint{2.497137in}{1.250190in}}{\pgfqpoint{2.489323in}{1.242376in}}%
\pgfpathcurveto{\pgfqpoint{2.481509in}{1.234563in}}{\pgfqpoint{2.477119in}{1.223964in}}{\pgfqpoint{2.477119in}{1.212913in}}%
\pgfpathcurveto{\pgfqpoint{2.477119in}{1.201863in}}{\pgfqpoint{2.481509in}{1.191264in}}{\pgfqpoint{2.489323in}{1.183451in}}%
\pgfpathcurveto{\pgfqpoint{2.497137in}{1.175637in}}{\pgfqpoint{2.507736in}{1.171247in}}{\pgfqpoint{2.518786in}{1.171247in}}%
\pgfpathclose%
\pgfusepath{stroke,fill}%
\end{pgfscope}%
\begin{pgfscope}%
\pgfpathrectangle{\pgfqpoint{0.800000in}{0.528000in}}{\pgfqpoint{4.960000in}{3.696000in}}%
\pgfusepath{clip}%
\pgfsetbuttcap%
\pgfsetroundjoin%
\definecolor{currentfill}{rgb}{0.000000,0.000000,0.000000}%
\pgfsetfillcolor{currentfill}%
\pgfsetlinewidth{1.003750pt}%
\definecolor{currentstroke}{rgb}{0.000000,0.000000,0.000000}%
\pgfsetstrokecolor{currentstroke}%
\pgfsetdash{}{0pt}%
\pgfpathmoveto{\pgfqpoint{2.518786in}{1.171247in}}%
\pgfpathcurveto{\pgfqpoint{2.529836in}{1.171247in}}{\pgfqpoint{2.540435in}{1.175637in}}{\pgfqpoint{2.548249in}{1.183451in}}%
\pgfpathcurveto{\pgfqpoint{2.556062in}{1.191264in}}{\pgfqpoint{2.560452in}{1.201863in}}{\pgfqpoint{2.560452in}{1.212913in}}%
\pgfpathcurveto{\pgfqpoint{2.560452in}{1.223964in}}{\pgfqpoint{2.556062in}{1.234563in}}{\pgfqpoint{2.548249in}{1.242376in}}%
\pgfpathcurveto{\pgfqpoint{2.540435in}{1.250190in}}{\pgfqpoint{2.529836in}{1.254580in}}{\pgfqpoint{2.518786in}{1.254580in}}%
\pgfpathcurveto{\pgfqpoint{2.507736in}{1.254580in}}{\pgfqpoint{2.497137in}{1.250190in}}{\pgfqpoint{2.489323in}{1.242376in}}%
\pgfpathcurveto{\pgfqpoint{2.481509in}{1.234563in}}{\pgfqpoint{2.477119in}{1.223964in}}{\pgfqpoint{2.477119in}{1.212913in}}%
\pgfpathcurveto{\pgfqpoint{2.477119in}{1.201863in}}{\pgfqpoint{2.481509in}{1.191264in}}{\pgfqpoint{2.489323in}{1.183451in}}%
\pgfpathcurveto{\pgfqpoint{2.497137in}{1.175637in}}{\pgfqpoint{2.507736in}{1.171247in}}{\pgfqpoint{2.518786in}{1.171247in}}%
\pgfpathclose%
\pgfusepath{stroke,fill}%
\end{pgfscope}%
\begin{pgfscope}%
\pgfpathrectangle{\pgfqpoint{0.800000in}{0.528000in}}{\pgfqpoint{4.960000in}{3.696000in}}%
\pgfusepath{clip}%
\pgfsetbuttcap%
\pgfsetroundjoin%
\definecolor{currentfill}{rgb}{0.000000,0.000000,0.000000}%
\pgfsetfillcolor{currentfill}%
\pgfsetlinewidth{1.003750pt}%
\definecolor{currentstroke}{rgb}{0.000000,0.000000,0.000000}%
\pgfsetstrokecolor{currentstroke}%
\pgfsetdash{}{0pt}%
\pgfpathmoveto{\pgfqpoint{2.518786in}{1.171247in}}%
\pgfpathcurveto{\pgfqpoint{2.529836in}{1.171247in}}{\pgfqpoint{2.540435in}{1.175637in}}{\pgfqpoint{2.548249in}{1.183451in}}%
\pgfpathcurveto{\pgfqpoint{2.556062in}{1.191264in}}{\pgfqpoint{2.560452in}{1.201863in}}{\pgfqpoint{2.560452in}{1.212913in}}%
\pgfpathcurveto{\pgfqpoint{2.560452in}{1.223964in}}{\pgfqpoint{2.556062in}{1.234563in}}{\pgfqpoint{2.548249in}{1.242376in}}%
\pgfpathcurveto{\pgfqpoint{2.540435in}{1.250190in}}{\pgfqpoint{2.529836in}{1.254580in}}{\pgfqpoint{2.518786in}{1.254580in}}%
\pgfpathcurveto{\pgfqpoint{2.507736in}{1.254580in}}{\pgfqpoint{2.497137in}{1.250190in}}{\pgfqpoint{2.489323in}{1.242376in}}%
\pgfpathcurveto{\pgfqpoint{2.481509in}{1.234563in}}{\pgfqpoint{2.477119in}{1.223964in}}{\pgfqpoint{2.477119in}{1.212913in}}%
\pgfpathcurveto{\pgfqpoint{2.477119in}{1.201863in}}{\pgfqpoint{2.481509in}{1.191264in}}{\pgfqpoint{2.489323in}{1.183451in}}%
\pgfpathcurveto{\pgfqpoint{2.497137in}{1.175637in}}{\pgfqpoint{2.507736in}{1.171247in}}{\pgfqpoint{2.518786in}{1.171247in}}%
\pgfpathclose%
\pgfusepath{stroke,fill}%
\end{pgfscope}%
\begin{pgfscope}%
\pgfpathrectangle{\pgfqpoint{0.800000in}{0.528000in}}{\pgfqpoint{4.960000in}{3.696000in}}%
\pgfusepath{clip}%
\pgfsetbuttcap%
\pgfsetroundjoin%
\definecolor{currentfill}{rgb}{0.000000,0.000000,0.000000}%
\pgfsetfillcolor{currentfill}%
\pgfsetlinewidth{1.003750pt}%
\definecolor{currentstroke}{rgb}{0.000000,0.000000,0.000000}%
\pgfsetstrokecolor{currentstroke}%
\pgfsetdash{}{0pt}%
\pgfpathmoveto{\pgfqpoint{2.518786in}{1.128252in}}%
\pgfpathcurveto{\pgfqpoint{2.529836in}{1.128252in}}{\pgfqpoint{2.540435in}{1.132642in}}{\pgfqpoint{2.548249in}{1.140456in}}%
\pgfpathcurveto{\pgfqpoint{2.556062in}{1.148269in}}{\pgfqpoint{2.560452in}{1.158868in}}{\pgfqpoint{2.560452in}{1.169918in}}%
\pgfpathcurveto{\pgfqpoint{2.560452in}{1.180969in}}{\pgfqpoint{2.556062in}{1.191568in}}{\pgfqpoint{2.548249in}{1.199381in}}%
\pgfpathcurveto{\pgfqpoint{2.540435in}{1.207195in}}{\pgfqpoint{2.529836in}{1.211585in}}{\pgfqpoint{2.518786in}{1.211585in}}%
\pgfpathcurveto{\pgfqpoint{2.507736in}{1.211585in}}{\pgfqpoint{2.497137in}{1.207195in}}{\pgfqpoint{2.489323in}{1.199381in}}%
\pgfpathcurveto{\pgfqpoint{2.481509in}{1.191568in}}{\pgfqpoint{2.477119in}{1.180969in}}{\pgfqpoint{2.477119in}{1.169918in}}%
\pgfpathcurveto{\pgfqpoint{2.477119in}{1.158868in}}{\pgfqpoint{2.481509in}{1.148269in}}{\pgfqpoint{2.489323in}{1.140456in}}%
\pgfpathcurveto{\pgfqpoint{2.497137in}{1.132642in}}{\pgfqpoint{2.507736in}{1.128252in}}{\pgfqpoint{2.518786in}{1.128252in}}%
\pgfpathclose%
\pgfusepath{stroke,fill}%
\end{pgfscope}%
\begin{pgfscope}%
\pgfpathrectangle{\pgfqpoint{0.800000in}{0.528000in}}{\pgfqpoint{4.960000in}{3.696000in}}%
\pgfusepath{clip}%
\pgfsetbuttcap%
\pgfsetroundjoin%
\definecolor{currentfill}{rgb}{0.000000,0.000000,0.000000}%
\pgfsetfillcolor{currentfill}%
\pgfsetlinewidth{1.003750pt}%
\definecolor{currentstroke}{rgb}{0.000000,0.000000,0.000000}%
\pgfsetstrokecolor{currentstroke}%
\pgfsetdash{}{0pt}%
\pgfpathmoveto{\pgfqpoint{2.518786in}{1.128252in}}%
\pgfpathcurveto{\pgfqpoint{2.529836in}{1.128252in}}{\pgfqpoint{2.540435in}{1.132642in}}{\pgfqpoint{2.548249in}{1.140456in}}%
\pgfpathcurveto{\pgfqpoint{2.556062in}{1.148269in}}{\pgfqpoint{2.560452in}{1.158868in}}{\pgfqpoint{2.560452in}{1.169918in}}%
\pgfpathcurveto{\pgfqpoint{2.560452in}{1.180969in}}{\pgfqpoint{2.556062in}{1.191568in}}{\pgfqpoint{2.548249in}{1.199381in}}%
\pgfpathcurveto{\pgfqpoint{2.540435in}{1.207195in}}{\pgfqpoint{2.529836in}{1.211585in}}{\pgfqpoint{2.518786in}{1.211585in}}%
\pgfpathcurveto{\pgfqpoint{2.507736in}{1.211585in}}{\pgfqpoint{2.497137in}{1.207195in}}{\pgfqpoint{2.489323in}{1.199381in}}%
\pgfpathcurveto{\pgfqpoint{2.481509in}{1.191568in}}{\pgfqpoint{2.477119in}{1.180969in}}{\pgfqpoint{2.477119in}{1.169918in}}%
\pgfpathcurveto{\pgfqpoint{2.477119in}{1.158868in}}{\pgfqpoint{2.481509in}{1.148269in}}{\pgfqpoint{2.489323in}{1.140456in}}%
\pgfpathcurveto{\pgfqpoint{2.497137in}{1.132642in}}{\pgfqpoint{2.507736in}{1.128252in}}{\pgfqpoint{2.518786in}{1.128252in}}%
\pgfpathclose%
\pgfusepath{stroke,fill}%
\end{pgfscope}%
\begin{pgfscope}%
\pgfpathrectangle{\pgfqpoint{0.800000in}{0.528000in}}{\pgfqpoint{4.960000in}{3.696000in}}%
\pgfusepath{clip}%
\pgfsetbuttcap%
\pgfsetroundjoin%
\definecolor{currentfill}{rgb}{0.000000,0.000000,0.000000}%
\pgfsetfillcolor{currentfill}%
\pgfsetlinewidth{1.003750pt}%
\definecolor{currentstroke}{rgb}{0.000000,0.000000,0.000000}%
\pgfsetstrokecolor{currentstroke}%
\pgfsetdash{}{0pt}%
\pgfpathmoveto{\pgfqpoint{2.518786in}{1.149749in}}%
\pgfpathcurveto{\pgfqpoint{2.529836in}{1.149749in}}{\pgfqpoint{2.540435in}{1.154140in}}{\pgfqpoint{2.548249in}{1.161953in}}%
\pgfpathcurveto{\pgfqpoint{2.556062in}{1.169767in}}{\pgfqpoint{2.560452in}{1.180366in}}{\pgfqpoint{2.560452in}{1.191416in}}%
\pgfpathcurveto{\pgfqpoint{2.560452in}{1.202466in}}{\pgfqpoint{2.556062in}{1.213065in}}{\pgfqpoint{2.548249in}{1.220879in}}%
\pgfpathcurveto{\pgfqpoint{2.540435in}{1.228692in}}{\pgfqpoint{2.529836in}{1.233083in}}{\pgfqpoint{2.518786in}{1.233083in}}%
\pgfpathcurveto{\pgfqpoint{2.507736in}{1.233083in}}{\pgfqpoint{2.497137in}{1.228692in}}{\pgfqpoint{2.489323in}{1.220879in}}%
\pgfpathcurveto{\pgfqpoint{2.481509in}{1.213065in}}{\pgfqpoint{2.477119in}{1.202466in}}{\pgfqpoint{2.477119in}{1.191416in}}%
\pgfpathcurveto{\pgfqpoint{2.477119in}{1.180366in}}{\pgfqpoint{2.481509in}{1.169767in}}{\pgfqpoint{2.489323in}{1.161953in}}%
\pgfpathcurveto{\pgfqpoint{2.497137in}{1.154140in}}{\pgfqpoint{2.507736in}{1.149749in}}{\pgfqpoint{2.518786in}{1.149749in}}%
\pgfpathclose%
\pgfusepath{stroke,fill}%
\end{pgfscope}%
\begin{pgfscope}%
\pgfpathrectangle{\pgfqpoint{0.800000in}{0.528000in}}{\pgfqpoint{4.960000in}{3.696000in}}%
\pgfusepath{clip}%
\pgfsetbuttcap%
\pgfsetroundjoin%
\definecolor{currentfill}{rgb}{0.000000,0.000000,0.000000}%
\pgfsetfillcolor{currentfill}%
\pgfsetlinewidth{1.003750pt}%
\definecolor{currentstroke}{rgb}{0.000000,0.000000,0.000000}%
\pgfsetstrokecolor{currentstroke}%
\pgfsetdash{}{0pt}%
\pgfpathmoveto{\pgfqpoint{2.518786in}{1.149749in}}%
\pgfpathcurveto{\pgfqpoint{2.529836in}{1.149749in}}{\pgfqpoint{2.540435in}{1.154140in}}{\pgfqpoint{2.548249in}{1.161953in}}%
\pgfpathcurveto{\pgfqpoint{2.556062in}{1.169767in}}{\pgfqpoint{2.560452in}{1.180366in}}{\pgfqpoint{2.560452in}{1.191416in}}%
\pgfpathcurveto{\pgfqpoint{2.560452in}{1.202466in}}{\pgfqpoint{2.556062in}{1.213065in}}{\pgfqpoint{2.548249in}{1.220879in}}%
\pgfpathcurveto{\pgfqpoint{2.540435in}{1.228692in}}{\pgfqpoint{2.529836in}{1.233083in}}{\pgfqpoint{2.518786in}{1.233083in}}%
\pgfpathcurveto{\pgfqpoint{2.507736in}{1.233083in}}{\pgfqpoint{2.497137in}{1.228692in}}{\pgfqpoint{2.489323in}{1.220879in}}%
\pgfpathcurveto{\pgfqpoint{2.481509in}{1.213065in}}{\pgfqpoint{2.477119in}{1.202466in}}{\pgfqpoint{2.477119in}{1.191416in}}%
\pgfpathcurveto{\pgfqpoint{2.477119in}{1.180366in}}{\pgfqpoint{2.481509in}{1.169767in}}{\pgfqpoint{2.489323in}{1.161953in}}%
\pgfpathcurveto{\pgfqpoint{2.497137in}{1.154140in}}{\pgfqpoint{2.507736in}{1.149749in}}{\pgfqpoint{2.518786in}{1.149749in}}%
\pgfpathclose%
\pgfusepath{stroke,fill}%
\end{pgfscope}%
\begin{pgfscope}%
\pgfpathrectangle{\pgfqpoint{0.800000in}{0.528000in}}{\pgfqpoint{4.960000in}{3.696000in}}%
\pgfusepath{clip}%
\pgfsetbuttcap%
\pgfsetroundjoin%
\definecolor{currentfill}{rgb}{0.000000,0.000000,0.000000}%
\pgfsetfillcolor{currentfill}%
\pgfsetlinewidth{1.003750pt}%
\definecolor{currentstroke}{rgb}{0.000000,0.000000,0.000000}%
\pgfsetstrokecolor{currentstroke}%
\pgfsetdash{}{0pt}%
\pgfpathmoveto{\pgfqpoint{2.518786in}{1.257237in}}%
\pgfpathcurveto{\pgfqpoint{2.529836in}{1.257237in}}{\pgfqpoint{2.540435in}{1.261627in}}{\pgfqpoint{2.548249in}{1.269441in}}%
\pgfpathcurveto{\pgfqpoint{2.556062in}{1.277254in}}{\pgfqpoint{2.560452in}{1.287853in}}{\pgfqpoint{2.560452in}{1.298903in}}%
\pgfpathcurveto{\pgfqpoint{2.560452in}{1.309954in}}{\pgfqpoint{2.556062in}{1.320553in}}{\pgfqpoint{2.548249in}{1.328366in}}%
\pgfpathcurveto{\pgfqpoint{2.540435in}{1.336180in}}{\pgfqpoint{2.529836in}{1.340570in}}{\pgfqpoint{2.518786in}{1.340570in}}%
\pgfpathcurveto{\pgfqpoint{2.507736in}{1.340570in}}{\pgfqpoint{2.497137in}{1.336180in}}{\pgfqpoint{2.489323in}{1.328366in}}%
\pgfpathcurveto{\pgfqpoint{2.481509in}{1.320553in}}{\pgfqpoint{2.477119in}{1.309954in}}{\pgfqpoint{2.477119in}{1.298903in}}%
\pgfpathcurveto{\pgfqpoint{2.477119in}{1.287853in}}{\pgfqpoint{2.481509in}{1.277254in}}{\pgfqpoint{2.489323in}{1.269441in}}%
\pgfpathcurveto{\pgfqpoint{2.497137in}{1.261627in}}{\pgfqpoint{2.507736in}{1.257237in}}{\pgfqpoint{2.518786in}{1.257237in}}%
\pgfpathclose%
\pgfusepath{stroke,fill}%
\end{pgfscope}%
\begin{pgfscope}%
\pgfpathrectangle{\pgfqpoint{0.800000in}{0.528000in}}{\pgfqpoint{4.960000in}{3.696000in}}%
\pgfusepath{clip}%
\pgfsetbuttcap%
\pgfsetroundjoin%
\definecolor{currentfill}{rgb}{0.000000,0.000000,0.000000}%
\pgfsetfillcolor{currentfill}%
\pgfsetlinewidth{1.003750pt}%
\definecolor{currentstroke}{rgb}{0.000000,0.000000,0.000000}%
\pgfsetstrokecolor{currentstroke}%
\pgfsetdash{}{0pt}%
\pgfpathmoveto{\pgfqpoint{2.518786in}{1.214242in}}%
\pgfpathcurveto{\pgfqpoint{2.529836in}{1.214242in}}{\pgfqpoint{2.540435in}{1.218632in}}{\pgfqpoint{2.548249in}{1.226446in}}%
\pgfpathcurveto{\pgfqpoint{2.556062in}{1.234259in}}{\pgfqpoint{2.560452in}{1.244858in}}{\pgfqpoint{2.560452in}{1.255908in}}%
\pgfpathcurveto{\pgfqpoint{2.560452in}{1.266959in}}{\pgfqpoint{2.556062in}{1.277558in}}{\pgfqpoint{2.548249in}{1.285371in}}%
\pgfpathcurveto{\pgfqpoint{2.540435in}{1.293185in}}{\pgfqpoint{2.529836in}{1.297575in}}{\pgfqpoint{2.518786in}{1.297575in}}%
\pgfpathcurveto{\pgfqpoint{2.507736in}{1.297575in}}{\pgfqpoint{2.497137in}{1.293185in}}{\pgfqpoint{2.489323in}{1.285371in}}%
\pgfpathcurveto{\pgfqpoint{2.481509in}{1.277558in}}{\pgfqpoint{2.477119in}{1.266959in}}{\pgfqpoint{2.477119in}{1.255908in}}%
\pgfpathcurveto{\pgfqpoint{2.477119in}{1.244858in}}{\pgfqpoint{2.481509in}{1.234259in}}{\pgfqpoint{2.489323in}{1.226446in}}%
\pgfpathcurveto{\pgfqpoint{2.497137in}{1.218632in}}{\pgfqpoint{2.507736in}{1.214242in}}{\pgfqpoint{2.518786in}{1.214242in}}%
\pgfpathclose%
\pgfusepath{stroke,fill}%
\end{pgfscope}%
\begin{pgfscope}%
\pgfpathrectangle{\pgfqpoint{0.800000in}{0.528000in}}{\pgfqpoint{4.960000in}{3.696000in}}%
\pgfusepath{clip}%
\pgfsetbuttcap%
\pgfsetroundjoin%
\definecolor{currentfill}{rgb}{0.000000,0.000000,0.000000}%
\pgfsetfillcolor{currentfill}%
\pgfsetlinewidth{1.003750pt}%
\definecolor{currentstroke}{rgb}{0.000000,0.000000,0.000000}%
\pgfsetstrokecolor{currentstroke}%
\pgfsetdash{}{0pt}%
\pgfpathmoveto{\pgfqpoint{2.518786in}{1.192744in}}%
\pgfpathcurveto{\pgfqpoint{2.529836in}{1.192744in}}{\pgfqpoint{2.540435in}{1.197135in}}{\pgfqpoint{2.548249in}{1.204948in}}%
\pgfpathcurveto{\pgfqpoint{2.556062in}{1.212762in}}{\pgfqpoint{2.560452in}{1.223361in}}{\pgfqpoint{2.560452in}{1.234411in}}%
\pgfpathcurveto{\pgfqpoint{2.560452in}{1.245461in}}{\pgfqpoint{2.556062in}{1.256060in}}{\pgfqpoint{2.548249in}{1.263874in}}%
\pgfpathcurveto{\pgfqpoint{2.540435in}{1.271687in}}{\pgfqpoint{2.529836in}{1.276078in}}{\pgfqpoint{2.518786in}{1.276078in}}%
\pgfpathcurveto{\pgfqpoint{2.507736in}{1.276078in}}{\pgfqpoint{2.497137in}{1.271687in}}{\pgfqpoint{2.489323in}{1.263874in}}%
\pgfpathcurveto{\pgfqpoint{2.481509in}{1.256060in}}{\pgfqpoint{2.477119in}{1.245461in}}{\pgfqpoint{2.477119in}{1.234411in}}%
\pgfpathcurveto{\pgfqpoint{2.477119in}{1.223361in}}{\pgfqpoint{2.481509in}{1.212762in}}{\pgfqpoint{2.489323in}{1.204948in}}%
\pgfpathcurveto{\pgfqpoint{2.497137in}{1.197135in}}{\pgfqpoint{2.507736in}{1.192744in}}{\pgfqpoint{2.518786in}{1.192744in}}%
\pgfpathclose%
\pgfusepath{stroke,fill}%
\end{pgfscope}%
\begin{pgfscope}%
\pgfpathrectangle{\pgfqpoint{0.800000in}{0.528000in}}{\pgfqpoint{4.960000in}{3.696000in}}%
\pgfusepath{clip}%
\pgfsetbuttcap%
\pgfsetroundjoin%
\definecolor{currentfill}{rgb}{0.000000,0.000000,0.000000}%
\pgfsetfillcolor{currentfill}%
\pgfsetlinewidth{1.003750pt}%
\definecolor{currentstroke}{rgb}{0.000000,0.000000,0.000000}%
\pgfsetstrokecolor{currentstroke}%
\pgfsetdash{}{0pt}%
\pgfpathmoveto{\pgfqpoint{2.518786in}{1.192744in}}%
\pgfpathcurveto{\pgfqpoint{2.529836in}{1.192744in}}{\pgfqpoint{2.540435in}{1.197135in}}{\pgfqpoint{2.548249in}{1.204948in}}%
\pgfpathcurveto{\pgfqpoint{2.556062in}{1.212762in}}{\pgfqpoint{2.560452in}{1.223361in}}{\pgfqpoint{2.560452in}{1.234411in}}%
\pgfpathcurveto{\pgfqpoint{2.560452in}{1.245461in}}{\pgfqpoint{2.556062in}{1.256060in}}{\pgfqpoint{2.548249in}{1.263874in}}%
\pgfpathcurveto{\pgfqpoint{2.540435in}{1.271687in}}{\pgfqpoint{2.529836in}{1.276078in}}{\pgfqpoint{2.518786in}{1.276078in}}%
\pgfpathcurveto{\pgfqpoint{2.507736in}{1.276078in}}{\pgfqpoint{2.497137in}{1.271687in}}{\pgfqpoint{2.489323in}{1.263874in}}%
\pgfpathcurveto{\pgfqpoint{2.481509in}{1.256060in}}{\pgfqpoint{2.477119in}{1.245461in}}{\pgfqpoint{2.477119in}{1.234411in}}%
\pgfpathcurveto{\pgfqpoint{2.477119in}{1.223361in}}{\pgfqpoint{2.481509in}{1.212762in}}{\pgfqpoint{2.489323in}{1.204948in}}%
\pgfpathcurveto{\pgfqpoint{2.497137in}{1.197135in}}{\pgfqpoint{2.507736in}{1.192744in}}{\pgfqpoint{2.518786in}{1.192744in}}%
\pgfpathclose%
\pgfusepath{stroke,fill}%
\end{pgfscope}%
\begin{pgfscope}%
\pgfpathrectangle{\pgfqpoint{0.800000in}{0.528000in}}{\pgfqpoint{4.960000in}{3.696000in}}%
\pgfusepath{clip}%
\pgfsetbuttcap%
\pgfsetroundjoin%
\definecolor{currentfill}{rgb}{0.000000,0.000000,0.000000}%
\pgfsetfillcolor{currentfill}%
\pgfsetlinewidth{1.003750pt}%
\definecolor{currentstroke}{rgb}{0.000000,0.000000,0.000000}%
\pgfsetstrokecolor{currentstroke}%
\pgfsetdash{}{0pt}%
\pgfpathmoveto{\pgfqpoint{2.518786in}{1.085257in}}%
\pgfpathcurveto{\pgfqpoint{2.529836in}{1.085257in}}{\pgfqpoint{2.540435in}{1.089647in}}{\pgfqpoint{2.548249in}{1.097461in}}%
\pgfpathcurveto{\pgfqpoint{2.556062in}{1.105274in}}{\pgfqpoint{2.560452in}{1.115873in}}{\pgfqpoint{2.560452in}{1.126923in}}%
\pgfpathcurveto{\pgfqpoint{2.560452in}{1.137974in}}{\pgfqpoint{2.556062in}{1.148573in}}{\pgfqpoint{2.548249in}{1.156386in}}%
\pgfpathcurveto{\pgfqpoint{2.540435in}{1.164200in}}{\pgfqpoint{2.529836in}{1.168590in}}{\pgfqpoint{2.518786in}{1.168590in}}%
\pgfpathcurveto{\pgfqpoint{2.507736in}{1.168590in}}{\pgfqpoint{2.497137in}{1.164200in}}{\pgfqpoint{2.489323in}{1.156386in}}%
\pgfpathcurveto{\pgfqpoint{2.481509in}{1.148573in}}{\pgfqpoint{2.477119in}{1.137974in}}{\pgfqpoint{2.477119in}{1.126923in}}%
\pgfpathcurveto{\pgfqpoint{2.477119in}{1.115873in}}{\pgfqpoint{2.481509in}{1.105274in}}{\pgfqpoint{2.489323in}{1.097461in}}%
\pgfpathcurveto{\pgfqpoint{2.497137in}{1.089647in}}{\pgfqpoint{2.507736in}{1.085257in}}{\pgfqpoint{2.518786in}{1.085257in}}%
\pgfpathclose%
\pgfusepath{stroke,fill}%
\end{pgfscope}%
\begin{pgfscope}%
\pgfpathrectangle{\pgfqpoint{0.800000in}{0.528000in}}{\pgfqpoint{4.960000in}{3.696000in}}%
\pgfusepath{clip}%
\pgfsetbuttcap%
\pgfsetroundjoin%
\definecolor{currentfill}{rgb}{0.000000,0.000000,0.000000}%
\pgfsetfillcolor{currentfill}%
\pgfsetlinewidth{1.003750pt}%
\definecolor{currentstroke}{rgb}{0.000000,0.000000,0.000000}%
\pgfsetstrokecolor{currentstroke}%
\pgfsetdash{}{0pt}%
\pgfpathmoveto{\pgfqpoint{2.518786in}{1.149749in}}%
\pgfpathcurveto{\pgfqpoint{2.529836in}{1.149749in}}{\pgfqpoint{2.540435in}{1.154140in}}{\pgfqpoint{2.548249in}{1.161953in}}%
\pgfpathcurveto{\pgfqpoint{2.556062in}{1.169767in}}{\pgfqpoint{2.560452in}{1.180366in}}{\pgfqpoint{2.560452in}{1.191416in}}%
\pgfpathcurveto{\pgfqpoint{2.560452in}{1.202466in}}{\pgfqpoint{2.556062in}{1.213065in}}{\pgfqpoint{2.548249in}{1.220879in}}%
\pgfpathcurveto{\pgfqpoint{2.540435in}{1.228692in}}{\pgfqpoint{2.529836in}{1.233083in}}{\pgfqpoint{2.518786in}{1.233083in}}%
\pgfpathcurveto{\pgfqpoint{2.507736in}{1.233083in}}{\pgfqpoint{2.497137in}{1.228692in}}{\pgfqpoint{2.489323in}{1.220879in}}%
\pgfpathcurveto{\pgfqpoint{2.481509in}{1.213065in}}{\pgfqpoint{2.477119in}{1.202466in}}{\pgfqpoint{2.477119in}{1.191416in}}%
\pgfpathcurveto{\pgfqpoint{2.477119in}{1.180366in}}{\pgfqpoint{2.481509in}{1.169767in}}{\pgfqpoint{2.489323in}{1.161953in}}%
\pgfpathcurveto{\pgfqpoint{2.497137in}{1.154140in}}{\pgfqpoint{2.507736in}{1.149749in}}{\pgfqpoint{2.518786in}{1.149749in}}%
\pgfpathclose%
\pgfusepath{stroke,fill}%
\end{pgfscope}%
\begin{pgfscope}%
\pgfpathrectangle{\pgfqpoint{0.800000in}{0.528000in}}{\pgfqpoint{4.960000in}{3.696000in}}%
\pgfusepath{clip}%
\pgfsetbuttcap%
\pgfsetroundjoin%
\definecolor{currentfill}{rgb}{0.000000,0.000000,0.000000}%
\pgfsetfillcolor{currentfill}%
\pgfsetlinewidth{1.003750pt}%
\definecolor{currentstroke}{rgb}{0.000000,0.000000,0.000000}%
\pgfsetstrokecolor{currentstroke}%
\pgfsetdash{}{0pt}%
\pgfpathmoveto{\pgfqpoint{2.518786in}{1.106754in}}%
\pgfpathcurveto{\pgfqpoint{2.529836in}{1.106754in}}{\pgfqpoint{2.540435in}{1.111145in}}{\pgfqpoint{2.548249in}{1.118958in}}%
\pgfpathcurveto{\pgfqpoint{2.556062in}{1.126772in}}{\pgfqpoint{2.560452in}{1.137371in}}{\pgfqpoint{2.560452in}{1.148421in}}%
\pgfpathcurveto{\pgfqpoint{2.560452in}{1.159471in}}{\pgfqpoint{2.556062in}{1.170070in}}{\pgfqpoint{2.548249in}{1.177884in}}%
\pgfpathcurveto{\pgfqpoint{2.540435in}{1.185697in}}{\pgfqpoint{2.529836in}{1.190088in}}{\pgfqpoint{2.518786in}{1.190088in}}%
\pgfpathcurveto{\pgfqpoint{2.507736in}{1.190088in}}{\pgfqpoint{2.497137in}{1.185697in}}{\pgfqpoint{2.489323in}{1.177884in}}%
\pgfpathcurveto{\pgfqpoint{2.481509in}{1.170070in}}{\pgfqpoint{2.477119in}{1.159471in}}{\pgfqpoint{2.477119in}{1.148421in}}%
\pgfpathcurveto{\pgfqpoint{2.477119in}{1.137371in}}{\pgfqpoint{2.481509in}{1.126772in}}{\pgfqpoint{2.489323in}{1.118958in}}%
\pgfpathcurveto{\pgfqpoint{2.497137in}{1.111145in}}{\pgfqpoint{2.507736in}{1.106754in}}{\pgfqpoint{2.518786in}{1.106754in}}%
\pgfpathclose%
\pgfusepath{stroke,fill}%
\end{pgfscope}%
\begin{pgfscope}%
\pgfpathrectangle{\pgfqpoint{0.800000in}{0.528000in}}{\pgfqpoint{4.960000in}{3.696000in}}%
\pgfusepath{clip}%
\pgfsetbuttcap%
\pgfsetroundjoin%
\definecolor{currentfill}{rgb}{0.000000,0.000000,0.000000}%
\pgfsetfillcolor{currentfill}%
\pgfsetlinewidth{1.003750pt}%
\definecolor{currentstroke}{rgb}{0.000000,0.000000,0.000000}%
\pgfsetstrokecolor{currentstroke}%
\pgfsetdash{}{0pt}%
\pgfpathmoveto{\pgfqpoint{2.518786in}{1.128252in}}%
\pgfpathcurveto{\pgfqpoint{2.529836in}{1.128252in}}{\pgfqpoint{2.540435in}{1.132642in}}{\pgfqpoint{2.548249in}{1.140456in}}%
\pgfpathcurveto{\pgfqpoint{2.556062in}{1.148269in}}{\pgfqpoint{2.560452in}{1.158868in}}{\pgfqpoint{2.560452in}{1.169918in}}%
\pgfpathcurveto{\pgfqpoint{2.560452in}{1.180969in}}{\pgfqpoint{2.556062in}{1.191568in}}{\pgfqpoint{2.548249in}{1.199381in}}%
\pgfpathcurveto{\pgfqpoint{2.540435in}{1.207195in}}{\pgfqpoint{2.529836in}{1.211585in}}{\pgfqpoint{2.518786in}{1.211585in}}%
\pgfpathcurveto{\pgfqpoint{2.507736in}{1.211585in}}{\pgfqpoint{2.497137in}{1.207195in}}{\pgfqpoint{2.489323in}{1.199381in}}%
\pgfpathcurveto{\pgfqpoint{2.481509in}{1.191568in}}{\pgfqpoint{2.477119in}{1.180969in}}{\pgfqpoint{2.477119in}{1.169918in}}%
\pgfpathcurveto{\pgfqpoint{2.477119in}{1.158868in}}{\pgfqpoint{2.481509in}{1.148269in}}{\pgfqpoint{2.489323in}{1.140456in}}%
\pgfpathcurveto{\pgfqpoint{2.497137in}{1.132642in}}{\pgfqpoint{2.507736in}{1.128252in}}{\pgfqpoint{2.518786in}{1.128252in}}%
\pgfpathclose%
\pgfusepath{stroke,fill}%
\end{pgfscope}%
\begin{pgfscope}%
\pgfpathrectangle{\pgfqpoint{0.800000in}{0.528000in}}{\pgfqpoint{4.960000in}{3.696000in}}%
\pgfusepath{clip}%
\pgfsetbuttcap%
\pgfsetroundjoin%
\definecolor{currentfill}{rgb}{0.000000,0.000000,0.000000}%
\pgfsetfillcolor{currentfill}%
\pgfsetlinewidth{1.003750pt}%
\definecolor{currentstroke}{rgb}{0.000000,0.000000,0.000000}%
\pgfsetstrokecolor{currentstroke}%
\pgfsetdash{}{0pt}%
\pgfpathmoveto{\pgfqpoint{2.518786in}{1.171247in}}%
\pgfpathcurveto{\pgfqpoint{2.529836in}{1.171247in}}{\pgfqpoint{2.540435in}{1.175637in}}{\pgfqpoint{2.548249in}{1.183451in}}%
\pgfpathcurveto{\pgfqpoint{2.556062in}{1.191264in}}{\pgfqpoint{2.560452in}{1.201863in}}{\pgfqpoint{2.560452in}{1.212913in}}%
\pgfpathcurveto{\pgfqpoint{2.560452in}{1.223964in}}{\pgfqpoint{2.556062in}{1.234563in}}{\pgfqpoint{2.548249in}{1.242376in}}%
\pgfpathcurveto{\pgfqpoint{2.540435in}{1.250190in}}{\pgfqpoint{2.529836in}{1.254580in}}{\pgfqpoint{2.518786in}{1.254580in}}%
\pgfpathcurveto{\pgfqpoint{2.507736in}{1.254580in}}{\pgfqpoint{2.497137in}{1.250190in}}{\pgfqpoint{2.489323in}{1.242376in}}%
\pgfpathcurveto{\pgfqpoint{2.481509in}{1.234563in}}{\pgfqpoint{2.477119in}{1.223964in}}{\pgfqpoint{2.477119in}{1.212913in}}%
\pgfpathcurveto{\pgfqpoint{2.477119in}{1.201863in}}{\pgfqpoint{2.481509in}{1.191264in}}{\pgfqpoint{2.489323in}{1.183451in}}%
\pgfpathcurveto{\pgfqpoint{2.497137in}{1.175637in}}{\pgfqpoint{2.507736in}{1.171247in}}{\pgfqpoint{2.518786in}{1.171247in}}%
\pgfpathclose%
\pgfusepath{stroke,fill}%
\end{pgfscope}%
\begin{pgfscope}%
\pgfpathrectangle{\pgfqpoint{0.800000in}{0.528000in}}{\pgfqpoint{4.960000in}{3.696000in}}%
\pgfusepath{clip}%
\pgfsetbuttcap%
\pgfsetroundjoin%
\definecolor{currentfill}{rgb}{0.000000,0.000000,0.000000}%
\pgfsetfillcolor{currentfill}%
\pgfsetlinewidth{1.003750pt}%
\definecolor{currentstroke}{rgb}{0.000000,0.000000,0.000000}%
\pgfsetstrokecolor{currentstroke}%
\pgfsetdash{}{0pt}%
\pgfpathmoveto{\pgfqpoint{2.518786in}{1.128252in}}%
\pgfpathcurveto{\pgfqpoint{2.529836in}{1.128252in}}{\pgfqpoint{2.540435in}{1.132642in}}{\pgfqpoint{2.548249in}{1.140456in}}%
\pgfpathcurveto{\pgfqpoint{2.556062in}{1.148269in}}{\pgfqpoint{2.560452in}{1.158868in}}{\pgfqpoint{2.560452in}{1.169918in}}%
\pgfpathcurveto{\pgfqpoint{2.560452in}{1.180969in}}{\pgfqpoint{2.556062in}{1.191568in}}{\pgfqpoint{2.548249in}{1.199381in}}%
\pgfpathcurveto{\pgfqpoint{2.540435in}{1.207195in}}{\pgfqpoint{2.529836in}{1.211585in}}{\pgfqpoint{2.518786in}{1.211585in}}%
\pgfpathcurveto{\pgfqpoint{2.507736in}{1.211585in}}{\pgfqpoint{2.497137in}{1.207195in}}{\pgfqpoint{2.489323in}{1.199381in}}%
\pgfpathcurveto{\pgfqpoint{2.481509in}{1.191568in}}{\pgfqpoint{2.477119in}{1.180969in}}{\pgfqpoint{2.477119in}{1.169918in}}%
\pgfpathcurveto{\pgfqpoint{2.477119in}{1.158868in}}{\pgfqpoint{2.481509in}{1.148269in}}{\pgfqpoint{2.489323in}{1.140456in}}%
\pgfpathcurveto{\pgfqpoint{2.497137in}{1.132642in}}{\pgfqpoint{2.507736in}{1.128252in}}{\pgfqpoint{2.518786in}{1.128252in}}%
\pgfpathclose%
\pgfusepath{stroke,fill}%
\end{pgfscope}%
\begin{pgfscope}%
\pgfpathrectangle{\pgfqpoint{0.800000in}{0.528000in}}{\pgfqpoint{4.960000in}{3.696000in}}%
\pgfusepath{clip}%
\pgfsetbuttcap%
\pgfsetroundjoin%
\definecolor{currentfill}{rgb}{0.000000,0.000000,0.000000}%
\pgfsetfillcolor{currentfill}%
\pgfsetlinewidth{1.003750pt}%
\definecolor{currentstroke}{rgb}{0.000000,0.000000,0.000000}%
\pgfsetstrokecolor{currentstroke}%
\pgfsetdash{}{0pt}%
\pgfpathmoveto{\pgfqpoint{2.518786in}{1.192744in}}%
\pgfpathcurveto{\pgfqpoint{2.529836in}{1.192744in}}{\pgfqpoint{2.540435in}{1.197135in}}{\pgfqpoint{2.548249in}{1.204948in}}%
\pgfpathcurveto{\pgfqpoint{2.556062in}{1.212762in}}{\pgfqpoint{2.560452in}{1.223361in}}{\pgfqpoint{2.560452in}{1.234411in}}%
\pgfpathcurveto{\pgfqpoint{2.560452in}{1.245461in}}{\pgfqpoint{2.556062in}{1.256060in}}{\pgfqpoint{2.548249in}{1.263874in}}%
\pgfpathcurveto{\pgfqpoint{2.540435in}{1.271687in}}{\pgfqpoint{2.529836in}{1.276078in}}{\pgfqpoint{2.518786in}{1.276078in}}%
\pgfpathcurveto{\pgfqpoint{2.507736in}{1.276078in}}{\pgfqpoint{2.497137in}{1.271687in}}{\pgfqpoint{2.489323in}{1.263874in}}%
\pgfpathcurveto{\pgfqpoint{2.481509in}{1.256060in}}{\pgfqpoint{2.477119in}{1.245461in}}{\pgfqpoint{2.477119in}{1.234411in}}%
\pgfpathcurveto{\pgfqpoint{2.477119in}{1.223361in}}{\pgfqpoint{2.481509in}{1.212762in}}{\pgfqpoint{2.489323in}{1.204948in}}%
\pgfpathcurveto{\pgfqpoint{2.497137in}{1.197135in}}{\pgfqpoint{2.507736in}{1.192744in}}{\pgfqpoint{2.518786in}{1.192744in}}%
\pgfpathclose%
\pgfusepath{stroke,fill}%
\end{pgfscope}%
\begin{pgfscope}%
\pgfpathrectangle{\pgfqpoint{0.800000in}{0.528000in}}{\pgfqpoint{4.960000in}{3.696000in}}%
\pgfusepath{clip}%
\pgfsetbuttcap%
\pgfsetroundjoin%
\definecolor{currentfill}{rgb}{0.000000,0.000000,0.000000}%
\pgfsetfillcolor{currentfill}%
\pgfsetlinewidth{1.003750pt}%
\definecolor{currentstroke}{rgb}{0.000000,0.000000,0.000000}%
\pgfsetstrokecolor{currentstroke}%
\pgfsetdash{}{0pt}%
\pgfpathmoveto{\pgfqpoint{2.518786in}{1.106754in}}%
\pgfpathcurveto{\pgfqpoint{2.529836in}{1.106754in}}{\pgfqpoint{2.540435in}{1.111145in}}{\pgfqpoint{2.548249in}{1.118958in}}%
\pgfpathcurveto{\pgfqpoint{2.556062in}{1.126772in}}{\pgfqpoint{2.560452in}{1.137371in}}{\pgfqpoint{2.560452in}{1.148421in}}%
\pgfpathcurveto{\pgfqpoint{2.560452in}{1.159471in}}{\pgfqpoint{2.556062in}{1.170070in}}{\pgfqpoint{2.548249in}{1.177884in}}%
\pgfpathcurveto{\pgfqpoint{2.540435in}{1.185697in}}{\pgfqpoint{2.529836in}{1.190088in}}{\pgfqpoint{2.518786in}{1.190088in}}%
\pgfpathcurveto{\pgfqpoint{2.507736in}{1.190088in}}{\pgfqpoint{2.497137in}{1.185697in}}{\pgfqpoint{2.489323in}{1.177884in}}%
\pgfpathcurveto{\pgfqpoint{2.481509in}{1.170070in}}{\pgfqpoint{2.477119in}{1.159471in}}{\pgfqpoint{2.477119in}{1.148421in}}%
\pgfpathcurveto{\pgfqpoint{2.477119in}{1.137371in}}{\pgfqpoint{2.481509in}{1.126772in}}{\pgfqpoint{2.489323in}{1.118958in}}%
\pgfpathcurveto{\pgfqpoint{2.497137in}{1.111145in}}{\pgfqpoint{2.507736in}{1.106754in}}{\pgfqpoint{2.518786in}{1.106754in}}%
\pgfpathclose%
\pgfusepath{stroke,fill}%
\end{pgfscope}%
\begin{pgfscope}%
\pgfpathrectangle{\pgfqpoint{0.800000in}{0.528000in}}{\pgfqpoint{4.960000in}{3.696000in}}%
\pgfusepath{clip}%
\pgfsetbuttcap%
\pgfsetroundjoin%
\definecolor{currentfill}{rgb}{0.000000,0.000000,0.000000}%
\pgfsetfillcolor{currentfill}%
\pgfsetlinewidth{1.003750pt}%
\definecolor{currentstroke}{rgb}{0.000000,0.000000,0.000000}%
\pgfsetstrokecolor{currentstroke}%
\pgfsetdash{}{0pt}%
\pgfpathmoveto{\pgfqpoint{2.518786in}{1.278734in}}%
\pgfpathcurveto{\pgfqpoint{2.529836in}{1.278734in}}{\pgfqpoint{2.540435in}{1.283125in}}{\pgfqpoint{2.548249in}{1.290938in}}%
\pgfpathcurveto{\pgfqpoint{2.556062in}{1.298752in}}{\pgfqpoint{2.560452in}{1.309351in}}{\pgfqpoint{2.560452in}{1.320401in}}%
\pgfpathcurveto{\pgfqpoint{2.560452in}{1.331451in}}{\pgfqpoint{2.556062in}{1.342050in}}{\pgfqpoint{2.548249in}{1.349864in}}%
\pgfpathcurveto{\pgfqpoint{2.540435in}{1.357677in}}{\pgfqpoint{2.529836in}{1.362068in}}{\pgfqpoint{2.518786in}{1.362068in}}%
\pgfpathcurveto{\pgfqpoint{2.507736in}{1.362068in}}{\pgfqpoint{2.497137in}{1.357677in}}{\pgfqpoint{2.489323in}{1.349864in}}%
\pgfpathcurveto{\pgfqpoint{2.481509in}{1.342050in}}{\pgfqpoint{2.477119in}{1.331451in}}{\pgfqpoint{2.477119in}{1.320401in}}%
\pgfpathcurveto{\pgfqpoint{2.477119in}{1.309351in}}{\pgfqpoint{2.481509in}{1.298752in}}{\pgfqpoint{2.489323in}{1.290938in}}%
\pgfpathcurveto{\pgfqpoint{2.497137in}{1.283125in}}{\pgfqpoint{2.507736in}{1.278734in}}{\pgfqpoint{2.518786in}{1.278734in}}%
\pgfpathclose%
\pgfusepath{stroke,fill}%
\end{pgfscope}%
\begin{pgfscope}%
\pgfpathrectangle{\pgfqpoint{0.800000in}{0.528000in}}{\pgfqpoint{4.960000in}{3.696000in}}%
\pgfusepath{clip}%
\pgfsetbuttcap%
\pgfsetroundjoin%
\definecolor{currentfill}{rgb}{0.000000,0.000000,0.000000}%
\pgfsetfillcolor{currentfill}%
\pgfsetlinewidth{1.003750pt}%
\definecolor{currentstroke}{rgb}{0.000000,0.000000,0.000000}%
\pgfsetstrokecolor{currentstroke}%
\pgfsetdash{}{0pt}%
\pgfpathmoveto{\pgfqpoint{2.518786in}{1.106754in}}%
\pgfpathcurveto{\pgfqpoint{2.529836in}{1.106754in}}{\pgfqpoint{2.540435in}{1.111145in}}{\pgfqpoint{2.548249in}{1.118958in}}%
\pgfpathcurveto{\pgfqpoint{2.556062in}{1.126772in}}{\pgfqpoint{2.560452in}{1.137371in}}{\pgfqpoint{2.560452in}{1.148421in}}%
\pgfpathcurveto{\pgfqpoint{2.560452in}{1.159471in}}{\pgfqpoint{2.556062in}{1.170070in}}{\pgfqpoint{2.548249in}{1.177884in}}%
\pgfpathcurveto{\pgfqpoint{2.540435in}{1.185697in}}{\pgfqpoint{2.529836in}{1.190088in}}{\pgfqpoint{2.518786in}{1.190088in}}%
\pgfpathcurveto{\pgfqpoint{2.507736in}{1.190088in}}{\pgfqpoint{2.497137in}{1.185697in}}{\pgfqpoint{2.489323in}{1.177884in}}%
\pgfpathcurveto{\pgfqpoint{2.481509in}{1.170070in}}{\pgfqpoint{2.477119in}{1.159471in}}{\pgfqpoint{2.477119in}{1.148421in}}%
\pgfpathcurveto{\pgfqpoint{2.477119in}{1.137371in}}{\pgfqpoint{2.481509in}{1.126772in}}{\pgfqpoint{2.489323in}{1.118958in}}%
\pgfpathcurveto{\pgfqpoint{2.497137in}{1.111145in}}{\pgfqpoint{2.507736in}{1.106754in}}{\pgfqpoint{2.518786in}{1.106754in}}%
\pgfpathclose%
\pgfusepath{stroke,fill}%
\end{pgfscope}%
\begin{pgfscope}%
\pgfpathrectangle{\pgfqpoint{0.800000in}{0.528000in}}{\pgfqpoint{4.960000in}{3.696000in}}%
\pgfusepath{clip}%
\pgfsetbuttcap%
\pgfsetroundjoin%
\definecolor{currentfill}{rgb}{0.000000,0.000000,0.000000}%
\pgfsetfillcolor{currentfill}%
\pgfsetlinewidth{1.003750pt}%
\definecolor{currentstroke}{rgb}{0.000000,0.000000,0.000000}%
\pgfsetstrokecolor{currentstroke}%
\pgfsetdash{}{0pt}%
\pgfpathmoveto{\pgfqpoint{2.518786in}{1.171247in}}%
\pgfpathcurveto{\pgfqpoint{2.529836in}{1.171247in}}{\pgfqpoint{2.540435in}{1.175637in}}{\pgfqpoint{2.548249in}{1.183451in}}%
\pgfpathcurveto{\pgfqpoint{2.556062in}{1.191264in}}{\pgfqpoint{2.560452in}{1.201863in}}{\pgfqpoint{2.560452in}{1.212913in}}%
\pgfpathcurveto{\pgfqpoint{2.560452in}{1.223964in}}{\pgfqpoint{2.556062in}{1.234563in}}{\pgfqpoint{2.548249in}{1.242376in}}%
\pgfpathcurveto{\pgfqpoint{2.540435in}{1.250190in}}{\pgfqpoint{2.529836in}{1.254580in}}{\pgfqpoint{2.518786in}{1.254580in}}%
\pgfpathcurveto{\pgfqpoint{2.507736in}{1.254580in}}{\pgfqpoint{2.497137in}{1.250190in}}{\pgfqpoint{2.489323in}{1.242376in}}%
\pgfpathcurveto{\pgfqpoint{2.481509in}{1.234563in}}{\pgfqpoint{2.477119in}{1.223964in}}{\pgfqpoint{2.477119in}{1.212913in}}%
\pgfpathcurveto{\pgfqpoint{2.477119in}{1.201863in}}{\pgfqpoint{2.481509in}{1.191264in}}{\pgfqpoint{2.489323in}{1.183451in}}%
\pgfpathcurveto{\pgfqpoint{2.497137in}{1.175637in}}{\pgfqpoint{2.507736in}{1.171247in}}{\pgfqpoint{2.518786in}{1.171247in}}%
\pgfpathclose%
\pgfusepath{stroke,fill}%
\end{pgfscope}%
\begin{pgfscope}%
\pgfpathrectangle{\pgfqpoint{0.800000in}{0.528000in}}{\pgfqpoint{4.960000in}{3.696000in}}%
\pgfusepath{clip}%
\pgfsetbuttcap%
\pgfsetroundjoin%
\definecolor{currentfill}{rgb}{0.000000,0.000000,0.000000}%
\pgfsetfillcolor{currentfill}%
\pgfsetlinewidth{1.003750pt}%
\definecolor{currentstroke}{rgb}{0.000000,0.000000,0.000000}%
\pgfsetstrokecolor{currentstroke}%
\pgfsetdash{}{0pt}%
\pgfpathmoveto{\pgfqpoint{2.518786in}{1.149749in}}%
\pgfpathcurveto{\pgfqpoint{2.529836in}{1.149749in}}{\pgfqpoint{2.540435in}{1.154140in}}{\pgfqpoint{2.548249in}{1.161953in}}%
\pgfpathcurveto{\pgfqpoint{2.556062in}{1.169767in}}{\pgfqpoint{2.560452in}{1.180366in}}{\pgfqpoint{2.560452in}{1.191416in}}%
\pgfpathcurveto{\pgfqpoint{2.560452in}{1.202466in}}{\pgfqpoint{2.556062in}{1.213065in}}{\pgfqpoint{2.548249in}{1.220879in}}%
\pgfpathcurveto{\pgfqpoint{2.540435in}{1.228692in}}{\pgfqpoint{2.529836in}{1.233083in}}{\pgfqpoint{2.518786in}{1.233083in}}%
\pgfpathcurveto{\pgfqpoint{2.507736in}{1.233083in}}{\pgfqpoint{2.497137in}{1.228692in}}{\pgfqpoint{2.489323in}{1.220879in}}%
\pgfpathcurveto{\pgfqpoint{2.481509in}{1.213065in}}{\pgfqpoint{2.477119in}{1.202466in}}{\pgfqpoint{2.477119in}{1.191416in}}%
\pgfpathcurveto{\pgfqpoint{2.477119in}{1.180366in}}{\pgfqpoint{2.481509in}{1.169767in}}{\pgfqpoint{2.489323in}{1.161953in}}%
\pgfpathcurveto{\pgfqpoint{2.497137in}{1.154140in}}{\pgfqpoint{2.507736in}{1.149749in}}{\pgfqpoint{2.518786in}{1.149749in}}%
\pgfpathclose%
\pgfusepath{stroke,fill}%
\end{pgfscope}%
\begin{pgfscope}%
\pgfpathrectangle{\pgfqpoint{0.800000in}{0.528000in}}{\pgfqpoint{4.960000in}{3.696000in}}%
\pgfusepath{clip}%
\pgfsetbuttcap%
\pgfsetroundjoin%
\definecolor{currentfill}{rgb}{0.000000,0.000000,0.000000}%
\pgfsetfillcolor{currentfill}%
\pgfsetlinewidth{1.003750pt}%
\definecolor{currentstroke}{rgb}{0.000000,0.000000,0.000000}%
\pgfsetstrokecolor{currentstroke}%
\pgfsetdash{}{0pt}%
\pgfpathmoveto{\pgfqpoint{2.518786in}{1.257237in}}%
\pgfpathcurveto{\pgfqpoint{2.529836in}{1.257237in}}{\pgfqpoint{2.540435in}{1.261627in}}{\pgfqpoint{2.548249in}{1.269441in}}%
\pgfpathcurveto{\pgfqpoint{2.556062in}{1.277254in}}{\pgfqpoint{2.560452in}{1.287853in}}{\pgfqpoint{2.560452in}{1.298903in}}%
\pgfpathcurveto{\pgfqpoint{2.560452in}{1.309954in}}{\pgfqpoint{2.556062in}{1.320553in}}{\pgfqpoint{2.548249in}{1.328366in}}%
\pgfpathcurveto{\pgfqpoint{2.540435in}{1.336180in}}{\pgfqpoint{2.529836in}{1.340570in}}{\pgfqpoint{2.518786in}{1.340570in}}%
\pgfpathcurveto{\pgfqpoint{2.507736in}{1.340570in}}{\pgfqpoint{2.497137in}{1.336180in}}{\pgfqpoint{2.489323in}{1.328366in}}%
\pgfpathcurveto{\pgfqpoint{2.481509in}{1.320553in}}{\pgfqpoint{2.477119in}{1.309954in}}{\pgfqpoint{2.477119in}{1.298903in}}%
\pgfpathcurveto{\pgfqpoint{2.477119in}{1.287853in}}{\pgfqpoint{2.481509in}{1.277254in}}{\pgfqpoint{2.489323in}{1.269441in}}%
\pgfpathcurveto{\pgfqpoint{2.497137in}{1.261627in}}{\pgfqpoint{2.507736in}{1.257237in}}{\pgfqpoint{2.518786in}{1.257237in}}%
\pgfpathclose%
\pgfusepath{stroke,fill}%
\end{pgfscope}%
\begin{pgfscope}%
\pgfpathrectangle{\pgfqpoint{0.800000in}{0.528000in}}{\pgfqpoint{4.960000in}{3.696000in}}%
\pgfusepath{clip}%
\pgfsetbuttcap%
\pgfsetroundjoin%
\definecolor{currentfill}{rgb}{0.000000,0.000000,0.000000}%
\pgfsetfillcolor{currentfill}%
\pgfsetlinewidth{1.003750pt}%
\definecolor{currentstroke}{rgb}{0.000000,0.000000,0.000000}%
\pgfsetstrokecolor{currentstroke}%
\pgfsetdash{}{0pt}%
\pgfpathmoveto{\pgfqpoint{2.518786in}{1.214242in}}%
\pgfpathcurveto{\pgfqpoint{2.529836in}{1.214242in}}{\pgfqpoint{2.540435in}{1.218632in}}{\pgfqpoint{2.548249in}{1.226446in}}%
\pgfpathcurveto{\pgfqpoint{2.556062in}{1.234259in}}{\pgfqpoint{2.560452in}{1.244858in}}{\pgfqpoint{2.560452in}{1.255908in}}%
\pgfpathcurveto{\pgfqpoint{2.560452in}{1.266959in}}{\pgfqpoint{2.556062in}{1.277558in}}{\pgfqpoint{2.548249in}{1.285371in}}%
\pgfpathcurveto{\pgfqpoint{2.540435in}{1.293185in}}{\pgfqpoint{2.529836in}{1.297575in}}{\pgfqpoint{2.518786in}{1.297575in}}%
\pgfpathcurveto{\pgfqpoint{2.507736in}{1.297575in}}{\pgfqpoint{2.497137in}{1.293185in}}{\pgfqpoint{2.489323in}{1.285371in}}%
\pgfpathcurveto{\pgfqpoint{2.481509in}{1.277558in}}{\pgfqpoint{2.477119in}{1.266959in}}{\pgfqpoint{2.477119in}{1.255908in}}%
\pgfpathcurveto{\pgfqpoint{2.477119in}{1.244858in}}{\pgfqpoint{2.481509in}{1.234259in}}{\pgfqpoint{2.489323in}{1.226446in}}%
\pgfpathcurveto{\pgfqpoint{2.497137in}{1.218632in}}{\pgfqpoint{2.507736in}{1.214242in}}{\pgfqpoint{2.518786in}{1.214242in}}%
\pgfpathclose%
\pgfusepath{stroke,fill}%
\end{pgfscope}%
\begin{pgfscope}%
\pgfpathrectangle{\pgfqpoint{0.800000in}{0.528000in}}{\pgfqpoint{4.960000in}{3.696000in}}%
\pgfusepath{clip}%
\pgfsetbuttcap%
\pgfsetroundjoin%
\definecolor{currentfill}{rgb}{0.000000,0.000000,0.000000}%
\pgfsetfillcolor{currentfill}%
\pgfsetlinewidth{1.003750pt}%
\definecolor{currentstroke}{rgb}{0.000000,0.000000,0.000000}%
\pgfsetstrokecolor{currentstroke}%
\pgfsetdash{}{0pt}%
\pgfpathmoveto{\pgfqpoint{2.518786in}{1.149749in}}%
\pgfpathcurveto{\pgfqpoint{2.529836in}{1.149749in}}{\pgfqpoint{2.540435in}{1.154140in}}{\pgfqpoint{2.548249in}{1.161953in}}%
\pgfpathcurveto{\pgfqpoint{2.556062in}{1.169767in}}{\pgfqpoint{2.560452in}{1.180366in}}{\pgfqpoint{2.560452in}{1.191416in}}%
\pgfpathcurveto{\pgfqpoint{2.560452in}{1.202466in}}{\pgfqpoint{2.556062in}{1.213065in}}{\pgfqpoint{2.548249in}{1.220879in}}%
\pgfpathcurveto{\pgfqpoint{2.540435in}{1.228692in}}{\pgfqpoint{2.529836in}{1.233083in}}{\pgfqpoint{2.518786in}{1.233083in}}%
\pgfpathcurveto{\pgfqpoint{2.507736in}{1.233083in}}{\pgfqpoint{2.497137in}{1.228692in}}{\pgfqpoint{2.489323in}{1.220879in}}%
\pgfpathcurveto{\pgfqpoint{2.481509in}{1.213065in}}{\pgfqpoint{2.477119in}{1.202466in}}{\pgfqpoint{2.477119in}{1.191416in}}%
\pgfpathcurveto{\pgfqpoint{2.477119in}{1.180366in}}{\pgfqpoint{2.481509in}{1.169767in}}{\pgfqpoint{2.489323in}{1.161953in}}%
\pgfpathcurveto{\pgfqpoint{2.497137in}{1.154140in}}{\pgfqpoint{2.507736in}{1.149749in}}{\pgfqpoint{2.518786in}{1.149749in}}%
\pgfpathclose%
\pgfusepath{stroke,fill}%
\end{pgfscope}%
\begin{pgfscope}%
\pgfpathrectangle{\pgfqpoint{0.800000in}{0.528000in}}{\pgfqpoint{4.960000in}{3.696000in}}%
\pgfusepath{clip}%
\pgfsetbuttcap%
\pgfsetroundjoin%
\definecolor{currentfill}{rgb}{0.000000,0.000000,0.000000}%
\pgfsetfillcolor{currentfill}%
\pgfsetlinewidth{1.003750pt}%
\definecolor{currentstroke}{rgb}{0.000000,0.000000,0.000000}%
\pgfsetstrokecolor{currentstroke}%
\pgfsetdash{}{0pt}%
\pgfpathmoveto{\pgfqpoint{2.518786in}{1.128252in}}%
\pgfpathcurveto{\pgfqpoint{2.529836in}{1.128252in}}{\pgfqpoint{2.540435in}{1.132642in}}{\pgfqpoint{2.548249in}{1.140456in}}%
\pgfpathcurveto{\pgfqpoint{2.556062in}{1.148269in}}{\pgfqpoint{2.560452in}{1.158868in}}{\pgfqpoint{2.560452in}{1.169918in}}%
\pgfpathcurveto{\pgfqpoint{2.560452in}{1.180969in}}{\pgfqpoint{2.556062in}{1.191568in}}{\pgfqpoint{2.548249in}{1.199381in}}%
\pgfpathcurveto{\pgfqpoint{2.540435in}{1.207195in}}{\pgfqpoint{2.529836in}{1.211585in}}{\pgfqpoint{2.518786in}{1.211585in}}%
\pgfpathcurveto{\pgfqpoint{2.507736in}{1.211585in}}{\pgfqpoint{2.497137in}{1.207195in}}{\pgfqpoint{2.489323in}{1.199381in}}%
\pgfpathcurveto{\pgfqpoint{2.481509in}{1.191568in}}{\pgfqpoint{2.477119in}{1.180969in}}{\pgfqpoint{2.477119in}{1.169918in}}%
\pgfpathcurveto{\pgfqpoint{2.477119in}{1.158868in}}{\pgfqpoint{2.481509in}{1.148269in}}{\pgfqpoint{2.489323in}{1.140456in}}%
\pgfpathcurveto{\pgfqpoint{2.497137in}{1.132642in}}{\pgfqpoint{2.507736in}{1.128252in}}{\pgfqpoint{2.518786in}{1.128252in}}%
\pgfpathclose%
\pgfusepath{stroke,fill}%
\end{pgfscope}%
\begin{pgfscope}%
\pgfpathrectangle{\pgfqpoint{0.800000in}{0.528000in}}{\pgfqpoint{4.960000in}{3.696000in}}%
\pgfusepath{clip}%
\pgfsetbuttcap%
\pgfsetroundjoin%
\definecolor{currentfill}{rgb}{0.000000,0.000000,0.000000}%
\pgfsetfillcolor{currentfill}%
\pgfsetlinewidth{1.003750pt}%
\definecolor{currentstroke}{rgb}{0.000000,0.000000,0.000000}%
\pgfsetstrokecolor{currentstroke}%
\pgfsetdash{}{0pt}%
\pgfpathmoveto{\pgfqpoint{2.518786in}{1.063759in}}%
\pgfpathcurveto{\pgfqpoint{2.529836in}{1.063759in}}{\pgfqpoint{2.540435in}{1.068150in}}{\pgfqpoint{2.548249in}{1.075963in}}%
\pgfpathcurveto{\pgfqpoint{2.556062in}{1.083777in}}{\pgfqpoint{2.560452in}{1.094376in}}{\pgfqpoint{2.560452in}{1.105426in}}%
\pgfpathcurveto{\pgfqpoint{2.560452in}{1.116476in}}{\pgfqpoint{2.556062in}{1.127075in}}{\pgfqpoint{2.548249in}{1.134889in}}%
\pgfpathcurveto{\pgfqpoint{2.540435in}{1.142702in}}{\pgfqpoint{2.529836in}{1.147093in}}{\pgfqpoint{2.518786in}{1.147093in}}%
\pgfpathcurveto{\pgfqpoint{2.507736in}{1.147093in}}{\pgfqpoint{2.497137in}{1.142702in}}{\pgfqpoint{2.489323in}{1.134889in}}%
\pgfpathcurveto{\pgfqpoint{2.481509in}{1.127075in}}{\pgfqpoint{2.477119in}{1.116476in}}{\pgfqpoint{2.477119in}{1.105426in}}%
\pgfpathcurveto{\pgfqpoint{2.477119in}{1.094376in}}{\pgfqpoint{2.481509in}{1.083777in}}{\pgfqpoint{2.489323in}{1.075963in}}%
\pgfpathcurveto{\pgfqpoint{2.497137in}{1.068150in}}{\pgfqpoint{2.507736in}{1.063759in}}{\pgfqpoint{2.518786in}{1.063759in}}%
\pgfpathclose%
\pgfusepath{stroke,fill}%
\end{pgfscope}%
\begin{pgfscope}%
\pgfpathrectangle{\pgfqpoint{0.800000in}{0.528000in}}{\pgfqpoint{4.960000in}{3.696000in}}%
\pgfusepath{clip}%
\pgfsetbuttcap%
\pgfsetroundjoin%
\definecolor{currentfill}{rgb}{0.000000,0.000000,0.000000}%
\pgfsetfillcolor{currentfill}%
\pgfsetlinewidth{1.003750pt}%
\definecolor{currentstroke}{rgb}{0.000000,0.000000,0.000000}%
\pgfsetstrokecolor{currentstroke}%
\pgfsetdash{}{0pt}%
\pgfpathmoveto{\pgfqpoint{2.518786in}{1.128252in}}%
\pgfpathcurveto{\pgfqpoint{2.529836in}{1.128252in}}{\pgfqpoint{2.540435in}{1.132642in}}{\pgfqpoint{2.548249in}{1.140456in}}%
\pgfpathcurveto{\pgfqpoint{2.556062in}{1.148269in}}{\pgfqpoint{2.560452in}{1.158868in}}{\pgfqpoint{2.560452in}{1.169918in}}%
\pgfpathcurveto{\pgfqpoint{2.560452in}{1.180969in}}{\pgfqpoint{2.556062in}{1.191568in}}{\pgfqpoint{2.548249in}{1.199381in}}%
\pgfpathcurveto{\pgfqpoint{2.540435in}{1.207195in}}{\pgfqpoint{2.529836in}{1.211585in}}{\pgfqpoint{2.518786in}{1.211585in}}%
\pgfpathcurveto{\pgfqpoint{2.507736in}{1.211585in}}{\pgfqpoint{2.497137in}{1.207195in}}{\pgfqpoint{2.489323in}{1.199381in}}%
\pgfpathcurveto{\pgfqpoint{2.481509in}{1.191568in}}{\pgfqpoint{2.477119in}{1.180969in}}{\pgfqpoint{2.477119in}{1.169918in}}%
\pgfpathcurveto{\pgfqpoint{2.477119in}{1.158868in}}{\pgfqpoint{2.481509in}{1.148269in}}{\pgfqpoint{2.489323in}{1.140456in}}%
\pgfpathcurveto{\pgfqpoint{2.497137in}{1.132642in}}{\pgfqpoint{2.507736in}{1.128252in}}{\pgfqpoint{2.518786in}{1.128252in}}%
\pgfpathclose%
\pgfusepath{stroke,fill}%
\end{pgfscope}%
\begin{pgfscope}%
\pgfpathrectangle{\pgfqpoint{0.800000in}{0.528000in}}{\pgfqpoint{4.960000in}{3.696000in}}%
\pgfusepath{clip}%
\pgfsetbuttcap%
\pgfsetroundjoin%
\definecolor{currentfill}{rgb}{0.000000,0.000000,0.000000}%
\pgfsetfillcolor{currentfill}%
\pgfsetlinewidth{1.003750pt}%
\definecolor{currentstroke}{rgb}{0.000000,0.000000,0.000000}%
\pgfsetstrokecolor{currentstroke}%
\pgfsetdash{}{0pt}%
\pgfpathmoveto{\pgfqpoint{2.518786in}{1.106754in}}%
\pgfpathcurveto{\pgfqpoint{2.529836in}{1.106754in}}{\pgfqpoint{2.540435in}{1.111145in}}{\pgfqpoint{2.548249in}{1.118958in}}%
\pgfpathcurveto{\pgfqpoint{2.556062in}{1.126772in}}{\pgfqpoint{2.560452in}{1.137371in}}{\pgfqpoint{2.560452in}{1.148421in}}%
\pgfpathcurveto{\pgfqpoint{2.560452in}{1.159471in}}{\pgfqpoint{2.556062in}{1.170070in}}{\pgfqpoint{2.548249in}{1.177884in}}%
\pgfpathcurveto{\pgfqpoint{2.540435in}{1.185697in}}{\pgfqpoint{2.529836in}{1.190088in}}{\pgfqpoint{2.518786in}{1.190088in}}%
\pgfpathcurveto{\pgfqpoint{2.507736in}{1.190088in}}{\pgfqpoint{2.497137in}{1.185697in}}{\pgfqpoint{2.489323in}{1.177884in}}%
\pgfpathcurveto{\pgfqpoint{2.481509in}{1.170070in}}{\pgfqpoint{2.477119in}{1.159471in}}{\pgfqpoint{2.477119in}{1.148421in}}%
\pgfpathcurveto{\pgfqpoint{2.477119in}{1.137371in}}{\pgfqpoint{2.481509in}{1.126772in}}{\pgfqpoint{2.489323in}{1.118958in}}%
\pgfpathcurveto{\pgfqpoint{2.497137in}{1.111145in}}{\pgfqpoint{2.507736in}{1.106754in}}{\pgfqpoint{2.518786in}{1.106754in}}%
\pgfpathclose%
\pgfusepath{stroke,fill}%
\end{pgfscope}%
\begin{pgfscope}%
\pgfpathrectangle{\pgfqpoint{0.800000in}{0.528000in}}{\pgfqpoint{4.960000in}{3.696000in}}%
\pgfusepath{clip}%
\pgfsetbuttcap%
\pgfsetroundjoin%
\definecolor{currentfill}{rgb}{0.000000,0.000000,0.000000}%
\pgfsetfillcolor{currentfill}%
\pgfsetlinewidth{1.003750pt}%
\definecolor{currentstroke}{rgb}{0.000000,0.000000,0.000000}%
\pgfsetstrokecolor{currentstroke}%
\pgfsetdash{}{0pt}%
\pgfpathmoveto{\pgfqpoint{2.518786in}{1.128252in}}%
\pgfpathcurveto{\pgfqpoint{2.529836in}{1.128252in}}{\pgfqpoint{2.540435in}{1.132642in}}{\pgfqpoint{2.548249in}{1.140456in}}%
\pgfpathcurveto{\pgfqpoint{2.556062in}{1.148269in}}{\pgfqpoint{2.560452in}{1.158868in}}{\pgfqpoint{2.560452in}{1.169918in}}%
\pgfpathcurveto{\pgfqpoint{2.560452in}{1.180969in}}{\pgfqpoint{2.556062in}{1.191568in}}{\pgfqpoint{2.548249in}{1.199381in}}%
\pgfpathcurveto{\pgfqpoint{2.540435in}{1.207195in}}{\pgfqpoint{2.529836in}{1.211585in}}{\pgfqpoint{2.518786in}{1.211585in}}%
\pgfpathcurveto{\pgfqpoint{2.507736in}{1.211585in}}{\pgfqpoint{2.497137in}{1.207195in}}{\pgfqpoint{2.489323in}{1.199381in}}%
\pgfpathcurveto{\pgfqpoint{2.481509in}{1.191568in}}{\pgfqpoint{2.477119in}{1.180969in}}{\pgfqpoint{2.477119in}{1.169918in}}%
\pgfpathcurveto{\pgfqpoint{2.477119in}{1.158868in}}{\pgfqpoint{2.481509in}{1.148269in}}{\pgfqpoint{2.489323in}{1.140456in}}%
\pgfpathcurveto{\pgfqpoint{2.497137in}{1.132642in}}{\pgfqpoint{2.507736in}{1.128252in}}{\pgfqpoint{2.518786in}{1.128252in}}%
\pgfpathclose%
\pgfusepath{stroke,fill}%
\end{pgfscope}%
\begin{pgfscope}%
\pgfpathrectangle{\pgfqpoint{0.800000in}{0.528000in}}{\pgfqpoint{4.960000in}{3.696000in}}%
\pgfusepath{clip}%
\pgfsetbuttcap%
\pgfsetroundjoin%
\definecolor{currentfill}{rgb}{0.000000,0.000000,0.000000}%
\pgfsetfillcolor{currentfill}%
\pgfsetlinewidth{1.003750pt}%
\definecolor{currentstroke}{rgb}{0.000000,0.000000,0.000000}%
\pgfsetstrokecolor{currentstroke}%
\pgfsetdash{}{0pt}%
\pgfpathmoveto{\pgfqpoint{2.518786in}{1.214242in}}%
\pgfpathcurveto{\pgfqpoint{2.529836in}{1.214242in}}{\pgfqpoint{2.540435in}{1.218632in}}{\pgfqpoint{2.548249in}{1.226446in}}%
\pgfpathcurveto{\pgfqpoint{2.556062in}{1.234259in}}{\pgfqpoint{2.560452in}{1.244858in}}{\pgfqpoint{2.560452in}{1.255908in}}%
\pgfpathcurveto{\pgfqpoint{2.560452in}{1.266959in}}{\pgfqpoint{2.556062in}{1.277558in}}{\pgfqpoint{2.548249in}{1.285371in}}%
\pgfpathcurveto{\pgfqpoint{2.540435in}{1.293185in}}{\pgfqpoint{2.529836in}{1.297575in}}{\pgfqpoint{2.518786in}{1.297575in}}%
\pgfpathcurveto{\pgfqpoint{2.507736in}{1.297575in}}{\pgfqpoint{2.497137in}{1.293185in}}{\pgfqpoint{2.489323in}{1.285371in}}%
\pgfpathcurveto{\pgfqpoint{2.481509in}{1.277558in}}{\pgfqpoint{2.477119in}{1.266959in}}{\pgfqpoint{2.477119in}{1.255908in}}%
\pgfpathcurveto{\pgfqpoint{2.477119in}{1.244858in}}{\pgfqpoint{2.481509in}{1.234259in}}{\pgfqpoint{2.489323in}{1.226446in}}%
\pgfpathcurveto{\pgfqpoint{2.497137in}{1.218632in}}{\pgfqpoint{2.507736in}{1.214242in}}{\pgfqpoint{2.518786in}{1.214242in}}%
\pgfpathclose%
\pgfusepath{stroke,fill}%
\end{pgfscope}%
\begin{pgfscope}%
\pgfpathrectangle{\pgfqpoint{0.800000in}{0.528000in}}{\pgfqpoint{4.960000in}{3.696000in}}%
\pgfusepath{clip}%
\pgfsetbuttcap%
\pgfsetroundjoin%
\definecolor{currentfill}{rgb}{0.000000,0.000000,0.000000}%
\pgfsetfillcolor{currentfill}%
\pgfsetlinewidth{1.003750pt}%
\definecolor{currentstroke}{rgb}{0.000000,0.000000,0.000000}%
\pgfsetstrokecolor{currentstroke}%
\pgfsetdash{}{0pt}%
\pgfpathmoveto{\pgfqpoint{2.518786in}{1.128252in}}%
\pgfpathcurveto{\pgfqpoint{2.529836in}{1.128252in}}{\pgfqpoint{2.540435in}{1.132642in}}{\pgfqpoint{2.548249in}{1.140456in}}%
\pgfpathcurveto{\pgfqpoint{2.556062in}{1.148269in}}{\pgfqpoint{2.560452in}{1.158868in}}{\pgfqpoint{2.560452in}{1.169918in}}%
\pgfpathcurveto{\pgfqpoint{2.560452in}{1.180969in}}{\pgfqpoint{2.556062in}{1.191568in}}{\pgfqpoint{2.548249in}{1.199381in}}%
\pgfpathcurveto{\pgfqpoint{2.540435in}{1.207195in}}{\pgfqpoint{2.529836in}{1.211585in}}{\pgfqpoint{2.518786in}{1.211585in}}%
\pgfpathcurveto{\pgfqpoint{2.507736in}{1.211585in}}{\pgfqpoint{2.497137in}{1.207195in}}{\pgfqpoint{2.489323in}{1.199381in}}%
\pgfpathcurveto{\pgfqpoint{2.481509in}{1.191568in}}{\pgfqpoint{2.477119in}{1.180969in}}{\pgfqpoint{2.477119in}{1.169918in}}%
\pgfpathcurveto{\pgfqpoint{2.477119in}{1.158868in}}{\pgfqpoint{2.481509in}{1.148269in}}{\pgfqpoint{2.489323in}{1.140456in}}%
\pgfpathcurveto{\pgfqpoint{2.497137in}{1.132642in}}{\pgfqpoint{2.507736in}{1.128252in}}{\pgfqpoint{2.518786in}{1.128252in}}%
\pgfpathclose%
\pgfusepath{stroke,fill}%
\end{pgfscope}%
\begin{pgfscope}%
\pgfpathrectangle{\pgfqpoint{0.800000in}{0.528000in}}{\pgfqpoint{4.960000in}{3.696000in}}%
\pgfusepath{clip}%
\pgfsetbuttcap%
\pgfsetroundjoin%
\definecolor{currentfill}{rgb}{0.000000,0.000000,0.000000}%
\pgfsetfillcolor{currentfill}%
\pgfsetlinewidth{1.003750pt}%
\definecolor{currentstroke}{rgb}{0.000000,0.000000,0.000000}%
\pgfsetstrokecolor{currentstroke}%
\pgfsetdash{}{0pt}%
\pgfpathmoveto{\pgfqpoint{2.518786in}{1.214242in}}%
\pgfpathcurveto{\pgfqpoint{2.529836in}{1.214242in}}{\pgfqpoint{2.540435in}{1.218632in}}{\pgfqpoint{2.548249in}{1.226446in}}%
\pgfpathcurveto{\pgfqpoint{2.556062in}{1.234259in}}{\pgfqpoint{2.560452in}{1.244858in}}{\pgfqpoint{2.560452in}{1.255908in}}%
\pgfpathcurveto{\pgfqpoint{2.560452in}{1.266959in}}{\pgfqpoint{2.556062in}{1.277558in}}{\pgfqpoint{2.548249in}{1.285371in}}%
\pgfpathcurveto{\pgfqpoint{2.540435in}{1.293185in}}{\pgfqpoint{2.529836in}{1.297575in}}{\pgfqpoint{2.518786in}{1.297575in}}%
\pgfpathcurveto{\pgfqpoint{2.507736in}{1.297575in}}{\pgfqpoint{2.497137in}{1.293185in}}{\pgfqpoint{2.489323in}{1.285371in}}%
\pgfpathcurveto{\pgfqpoint{2.481509in}{1.277558in}}{\pgfqpoint{2.477119in}{1.266959in}}{\pgfqpoint{2.477119in}{1.255908in}}%
\pgfpathcurveto{\pgfqpoint{2.477119in}{1.244858in}}{\pgfqpoint{2.481509in}{1.234259in}}{\pgfqpoint{2.489323in}{1.226446in}}%
\pgfpathcurveto{\pgfqpoint{2.497137in}{1.218632in}}{\pgfqpoint{2.507736in}{1.214242in}}{\pgfqpoint{2.518786in}{1.214242in}}%
\pgfpathclose%
\pgfusepath{stroke,fill}%
\end{pgfscope}%
\begin{pgfscope}%
\pgfpathrectangle{\pgfqpoint{0.800000in}{0.528000in}}{\pgfqpoint{4.960000in}{3.696000in}}%
\pgfusepath{clip}%
\pgfsetbuttcap%
\pgfsetroundjoin%
\definecolor{currentfill}{rgb}{0.000000,0.000000,0.000000}%
\pgfsetfillcolor{currentfill}%
\pgfsetlinewidth{1.003750pt}%
\definecolor{currentstroke}{rgb}{0.000000,0.000000,0.000000}%
\pgfsetstrokecolor{currentstroke}%
\pgfsetdash{}{0pt}%
\pgfpathmoveto{\pgfqpoint{2.518786in}{1.106754in}}%
\pgfpathcurveto{\pgfqpoint{2.529836in}{1.106754in}}{\pgfqpoint{2.540435in}{1.111145in}}{\pgfqpoint{2.548249in}{1.118958in}}%
\pgfpathcurveto{\pgfqpoint{2.556062in}{1.126772in}}{\pgfqpoint{2.560452in}{1.137371in}}{\pgfqpoint{2.560452in}{1.148421in}}%
\pgfpathcurveto{\pgfqpoint{2.560452in}{1.159471in}}{\pgfqpoint{2.556062in}{1.170070in}}{\pgfqpoint{2.548249in}{1.177884in}}%
\pgfpathcurveto{\pgfqpoint{2.540435in}{1.185697in}}{\pgfqpoint{2.529836in}{1.190088in}}{\pgfqpoint{2.518786in}{1.190088in}}%
\pgfpathcurveto{\pgfqpoint{2.507736in}{1.190088in}}{\pgfqpoint{2.497137in}{1.185697in}}{\pgfqpoint{2.489323in}{1.177884in}}%
\pgfpathcurveto{\pgfqpoint{2.481509in}{1.170070in}}{\pgfqpoint{2.477119in}{1.159471in}}{\pgfqpoint{2.477119in}{1.148421in}}%
\pgfpathcurveto{\pgfqpoint{2.477119in}{1.137371in}}{\pgfqpoint{2.481509in}{1.126772in}}{\pgfqpoint{2.489323in}{1.118958in}}%
\pgfpathcurveto{\pgfqpoint{2.497137in}{1.111145in}}{\pgfqpoint{2.507736in}{1.106754in}}{\pgfqpoint{2.518786in}{1.106754in}}%
\pgfpathclose%
\pgfusepath{stroke,fill}%
\end{pgfscope}%
\begin{pgfscope}%
\pgfpathrectangle{\pgfqpoint{0.800000in}{0.528000in}}{\pgfqpoint{4.960000in}{3.696000in}}%
\pgfusepath{clip}%
\pgfsetbuttcap%
\pgfsetroundjoin%
\definecolor{currentfill}{rgb}{0.000000,0.000000,0.000000}%
\pgfsetfillcolor{currentfill}%
\pgfsetlinewidth{1.003750pt}%
\definecolor{currentstroke}{rgb}{0.000000,0.000000,0.000000}%
\pgfsetstrokecolor{currentstroke}%
\pgfsetdash{}{0pt}%
\pgfpathmoveto{\pgfqpoint{2.518786in}{1.214242in}}%
\pgfpathcurveto{\pgfqpoint{2.529836in}{1.214242in}}{\pgfqpoint{2.540435in}{1.218632in}}{\pgfqpoint{2.548249in}{1.226446in}}%
\pgfpathcurveto{\pgfqpoint{2.556062in}{1.234259in}}{\pgfqpoint{2.560452in}{1.244858in}}{\pgfqpoint{2.560452in}{1.255908in}}%
\pgfpathcurveto{\pgfqpoint{2.560452in}{1.266959in}}{\pgfqpoint{2.556062in}{1.277558in}}{\pgfqpoint{2.548249in}{1.285371in}}%
\pgfpathcurveto{\pgfqpoint{2.540435in}{1.293185in}}{\pgfqpoint{2.529836in}{1.297575in}}{\pgfqpoint{2.518786in}{1.297575in}}%
\pgfpathcurveto{\pgfqpoint{2.507736in}{1.297575in}}{\pgfqpoint{2.497137in}{1.293185in}}{\pgfqpoint{2.489323in}{1.285371in}}%
\pgfpathcurveto{\pgfqpoint{2.481509in}{1.277558in}}{\pgfqpoint{2.477119in}{1.266959in}}{\pgfqpoint{2.477119in}{1.255908in}}%
\pgfpathcurveto{\pgfqpoint{2.477119in}{1.244858in}}{\pgfqpoint{2.481509in}{1.234259in}}{\pgfqpoint{2.489323in}{1.226446in}}%
\pgfpathcurveto{\pgfqpoint{2.497137in}{1.218632in}}{\pgfqpoint{2.507736in}{1.214242in}}{\pgfqpoint{2.518786in}{1.214242in}}%
\pgfpathclose%
\pgfusepath{stroke,fill}%
\end{pgfscope}%
\begin{pgfscope}%
\pgfpathrectangle{\pgfqpoint{0.800000in}{0.528000in}}{\pgfqpoint{4.960000in}{3.696000in}}%
\pgfusepath{clip}%
\pgfsetbuttcap%
\pgfsetroundjoin%
\definecolor{currentfill}{rgb}{0.000000,0.000000,0.000000}%
\pgfsetfillcolor{currentfill}%
\pgfsetlinewidth{1.003750pt}%
\definecolor{currentstroke}{rgb}{0.000000,0.000000,0.000000}%
\pgfsetstrokecolor{currentstroke}%
\pgfsetdash{}{0pt}%
\pgfpathmoveto{\pgfqpoint{2.518786in}{1.085257in}}%
\pgfpathcurveto{\pgfqpoint{2.529836in}{1.085257in}}{\pgfqpoint{2.540435in}{1.089647in}}{\pgfqpoint{2.548249in}{1.097461in}}%
\pgfpathcurveto{\pgfqpoint{2.556062in}{1.105274in}}{\pgfqpoint{2.560452in}{1.115873in}}{\pgfqpoint{2.560452in}{1.126923in}}%
\pgfpathcurveto{\pgfqpoint{2.560452in}{1.137974in}}{\pgfqpoint{2.556062in}{1.148573in}}{\pgfqpoint{2.548249in}{1.156386in}}%
\pgfpathcurveto{\pgfqpoint{2.540435in}{1.164200in}}{\pgfqpoint{2.529836in}{1.168590in}}{\pgfqpoint{2.518786in}{1.168590in}}%
\pgfpathcurveto{\pgfqpoint{2.507736in}{1.168590in}}{\pgfqpoint{2.497137in}{1.164200in}}{\pgfqpoint{2.489323in}{1.156386in}}%
\pgfpathcurveto{\pgfqpoint{2.481509in}{1.148573in}}{\pgfqpoint{2.477119in}{1.137974in}}{\pgfqpoint{2.477119in}{1.126923in}}%
\pgfpathcurveto{\pgfqpoint{2.477119in}{1.115873in}}{\pgfqpoint{2.481509in}{1.105274in}}{\pgfqpoint{2.489323in}{1.097461in}}%
\pgfpathcurveto{\pgfqpoint{2.497137in}{1.089647in}}{\pgfqpoint{2.507736in}{1.085257in}}{\pgfqpoint{2.518786in}{1.085257in}}%
\pgfpathclose%
\pgfusepath{stroke,fill}%
\end{pgfscope}%
\begin{pgfscope}%
\pgfpathrectangle{\pgfqpoint{0.800000in}{0.528000in}}{\pgfqpoint{4.960000in}{3.696000in}}%
\pgfusepath{clip}%
\pgfsetbuttcap%
\pgfsetroundjoin%
\definecolor{currentfill}{rgb}{0.000000,0.000000,0.000000}%
\pgfsetfillcolor{currentfill}%
\pgfsetlinewidth{1.003750pt}%
\definecolor{currentstroke}{rgb}{0.000000,0.000000,0.000000}%
\pgfsetstrokecolor{currentstroke}%
\pgfsetdash{}{0pt}%
\pgfpathmoveto{\pgfqpoint{2.518786in}{1.192744in}}%
\pgfpathcurveto{\pgfqpoint{2.529836in}{1.192744in}}{\pgfqpoint{2.540435in}{1.197135in}}{\pgfqpoint{2.548249in}{1.204948in}}%
\pgfpathcurveto{\pgfqpoint{2.556062in}{1.212762in}}{\pgfqpoint{2.560452in}{1.223361in}}{\pgfqpoint{2.560452in}{1.234411in}}%
\pgfpathcurveto{\pgfqpoint{2.560452in}{1.245461in}}{\pgfqpoint{2.556062in}{1.256060in}}{\pgfqpoint{2.548249in}{1.263874in}}%
\pgfpathcurveto{\pgfqpoint{2.540435in}{1.271687in}}{\pgfqpoint{2.529836in}{1.276078in}}{\pgfqpoint{2.518786in}{1.276078in}}%
\pgfpathcurveto{\pgfqpoint{2.507736in}{1.276078in}}{\pgfqpoint{2.497137in}{1.271687in}}{\pgfqpoint{2.489323in}{1.263874in}}%
\pgfpathcurveto{\pgfqpoint{2.481509in}{1.256060in}}{\pgfqpoint{2.477119in}{1.245461in}}{\pgfqpoint{2.477119in}{1.234411in}}%
\pgfpathcurveto{\pgfqpoint{2.477119in}{1.223361in}}{\pgfqpoint{2.481509in}{1.212762in}}{\pgfqpoint{2.489323in}{1.204948in}}%
\pgfpathcurveto{\pgfqpoint{2.497137in}{1.197135in}}{\pgfqpoint{2.507736in}{1.192744in}}{\pgfqpoint{2.518786in}{1.192744in}}%
\pgfpathclose%
\pgfusepath{stroke,fill}%
\end{pgfscope}%
\begin{pgfscope}%
\pgfpathrectangle{\pgfqpoint{0.800000in}{0.528000in}}{\pgfqpoint{4.960000in}{3.696000in}}%
\pgfusepath{clip}%
\pgfsetbuttcap%
\pgfsetroundjoin%
\definecolor{currentfill}{rgb}{0.000000,0.000000,0.000000}%
\pgfsetfillcolor{currentfill}%
\pgfsetlinewidth{1.003750pt}%
\definecolor{currentstroke}{rgb}{0.000000,0.000000,0.000000}%
\pgfsetstrokecolor{currentstroke}%
\pgfsetdash{}{0pt}%
\pgfpathmoveto{\pgfqpoint{2.518786in}{1.214242in}}%
\pgfpathcurveto{\pgfqpoint{2.529836in}{1.214242in}}{\pgfqpoint{2.540435in}{1.218632in}}{\pgfqpoint{2.548249in}{1.226446in}}%
\pgfpathcurveto{\pgfqpoint{2.556062in}{1.234259in}}{\pgfqpoint{2.560452in}{1.244858in}}{\pgfqpoint{2.560452in}{1.255908in}}%
\pgfpathcurveto{\pgfqpoint{2.560452in}{1.266959in}}{\pgfqpoint{2.556062in}{1.277558in}}{\pgfqpoint{2.548249in}{1.285371in}}%
\pgfpathcurveto{\pgfqpoint{2.540435in}{1.293185in}}{\pgfqpoint{2.529836in}{1.297575in}}{\pgfqpoint{2.518786in}{1.297575in}}%
\pgfpathcurveto{\pgfqpoint{2.507736in}{1.297575in}}{\pgfqpoint{2.497137in}{1.293185in}}{\pgfqpoint{2.489323in}{1.285371in}}%
\pgfpathcurveto{\pgfqpoint{2.481509in}{1.277558in}}{\pgfqpoint{2.477119in}{1.266959in}}{\pgfqpoint{2.477119in}{1.255908in}}%
\pgfpathcurveto{\pgfqpoint{2.477119in}{1.244858in}}{\pgfqpoint{2.481509in}{1.234259in}}{\pgfqpoint{2.489323in}{1.226446in}}%
\pgfpathcurveto{\pgfqpoint{2.497137in}{1.218632in}}{\pgfqpoint{2.507736in}{1.214242in}}{\pgfqpoint{2.518786in}{1.214242in}}%
\pgfpathclose%
\pgfusepath{stroke,fill}%
\end{pgfscope}%
\begin{pgfscope}%
\pgfpathrectangle{\pgfqpoint{0.800000in}{0.528000in}}{\pgfqpoint{4.960000in}{3.696000in}}%
\pgfusepath{clip}%
\pgfsetbuttcap%
\pgfsetroundjoin%
\definecolor{currentfill}{rgb}{0.000000,0.000000,0.000000}%
\pgfsetfillcolor{currentfill}%
\pgfsetlinewidth{1.003750pt}%
\definecolor{currentstroke}{rgb}{0.000000,0.000000,0.000000}%
\pgfsetstrokecolor{currentstroke}%
\pgfsetdash{}{0pt}%
\pgfpathmoveto{\pgfqpoint{2.518786in}{1.149749in}}%
\pgfpathcurveto{\pgfqpoint{2.529836in}{1.149749in}}{\pgfqpoint{2.540435in}{1.154140in}}{\pgfqpoint{2.548249in}{1.161953in}}%
\pgfpathcurveto{\pgfqpoint{2.556062in}{1.169767in}}{\pgfqpoint{2.560452in}{1.180366in}}{\pgfqpoint{2.560452in}{1.191416in}}%
\pgfpathcurveto{\pgfqpoint{2.560452in}{1.202466in}}{\pgfqpoint{2.556062in}{1.213065in}}{\pgfqpoint{2.548249in}{1.220879in}}%
\pgfpathcurveto{\pgfqpoint{2.540435in}{1.228692in}}{\pgfqpoint{2.529836in}{1.233083in}}{\pgfqpoint{2.518786in}{1.233083in}}%
\pgfpathcurveto{\pgfqpoint{2.507736in}{1.233083in}}{\pgfqpoint{2.497137in}{1.228692in}}{\pgfqpoint{2.489323in}{1.220879in}}%
\pgfpathcurveto{\pgfqpoint{2.481509in}{1.213065in}}{\pgfqpoint{2.477119in}{1.202466in}}{\pgfqpoint{2.477119in}{1.191416in}}%
\pgfpathcurveto{\pgfqpoint{2.477119in}{1.180366in}}{\pgfqpoint{2.481509in}{1.169767in}}{\pgfqpoint{2.489323in}{1.161953in}}%
\pgfpathcurveto{\pgfqpoint{2.497137in}{1.154140in}}{\pgfqpoint{2.507736in}{1.149749in}}{\pgfqpoint{2.518786in}{1.149749in}}%
\pgfpathclose%
\pgfusepath{stroke,fill}%
\end{pgfscope}%
\begin{pgfscope}%
\pgfpathrectangle{\pgfqpoint{0.800000in}{0.528000in}}{\pgfqpoint{4.960000in}{3.696000in}}%
\pgfusepath{clip}%
\pgfsetbuttcap%
\pgfsetroundjoin%
\definecolor{currentfill}{rgb}{0.000000,0.000000,0.000000}%
\pgfsetfillcolor{currentfill}%
\pgfsetlinewidth{1.003750pt}%
\definecolor{currentstroke}{rgb}{0.000000,0.000000,0.000000}%
\pgfsetstrokecolor{currentstroke}%
\pgfsetdash{}{0pt}%
\pgfpathmoveto{\pgfqpoint{2.518786in}{1.257237in}}%
\pgfpathcurveto{\pgfqpoint{2.529836in}{1.257237in}}{\pgfqpoint{2.540435in}{1.261627in}}{\pgfqpoint{2.548249in}{1.269441in}}%
\pgfpathcurveto{\pgfqpoint{2.556062in}{1.277254in}}{\pgfqpoint{2.560452in}{1.287853in}}{\pgfqpoint{2.560452in}{1.298903in}}%
\pgfpathcurveto{\pgfqpoint{2.560452in}{1.309954in}}{\pgfqpoint{2.556062in}{1.320553in}}{\pgfqpoint{2.548249in}{1.328366in}}%
\pgfpathcurveto{\pgfqpoint{2.540435in}{1.336180in}}{\pgfqpoint{2.529836in}{1.340570in}}{\pgfqpoint{2.518786in}{1.340570in}}%
\pgfpathcurveto{\pgfqpoint{2.507736in}{1.340570in}}{\pgfqpoint{2.497137in}{1.336180in}}{\pgfqpoint{2.489323in}{1.328366in}}%
\pgfpathcurveto{\pgfqpoint{2.481509in}{1.320553in}}{\pgfqpoint{2.477119in}{1.309954in}}{\pgfqpoint{2.477119in}{1.298903in}}%
\pgfpathcurveto{\pgfqpoint{2.477119in}{1.287853in}}{\pgfqpoint{2.481509in}{1.277254in}}{\pgfqpoint{2.489323in}{1.269441in}}%
\pgfpathcurveto{\pgfqpoint{2.497137in}{1.261627in}}{\pgfqpoint{2.507736in}{1.257237in}}{\pgfqpoint{2.518786in}{1.257237in}}%
\pgfpathclose%
\pgfusepath{stroke,fill}%
\end{pgfscope}%
\begin{pgfscope}%
\pgfpathrectangle{\pgfqpoint{0.800000in}{0.528000in}}{\pgfqpoint{4.960000in}{3.696000in}}%
\pgfusepath{clip}%
\pgfsetbuttcap%
\pgfsetroundjoin%
\definecolor{currentfill}{rgb}{0.000000,0.000000,0.000000}%
\pgfsetfillcolor{currentfill}%
\pgfsetlinewidth{1.003750pt}%
\definecolor{currentstroke}{rgb}{0.000000,0.000000,0.000000}%
\pgfsetstrokecolor{currentstroke}%
\pgfsetdash{}{0pt}%
\pgfpathmoveto{\pgfqpoint{2.518786in}{1.300232in}}%
\pgfpathcurveto{\pgfqpoint{2.529836in}{1.300232in}}{\pgfqpoint{2.540435in}{1.304622in}}{\pgfqpoint{2.548249in}{1.312436in}}%
\pgfpathcurveto{\pgfqpoint{2.556062in}{1.320249in}}{\pgfqpoint{2.560452in}{1.330848in}}{\pgfqpoint{2.560452in}{1.341898in}}%
\pgfpathcurveto{\pgfqpoint{2.560452in}{1.352949in}}{\pgfqpoint{2.556062in}{1.363548in}}{\pgfqpoint{2.548249in}{1.371361in}}%
\pgfpathcurveto{\pgfqpoint{2.540435in}{1.379175in}}{\pgfqpoint{2.529836in}{1.383565in}}{\pgfqpoint{2.518786in}{1.383565in}}%
\pgfpathcurveto{\pgfqpoint{2.507736in}{1.383565in}}{\pgfqpoint{2.497137in}{1.379175in}}{\pgfqpoint{2.489323in}{1.371361in}}%
\pgfpathcurveto{\pgfqpoint{2.481509in}{1.363548in}}{\pgfqpoint{2.477119in}{1.352949in}}{\pgfqpoint{2.477119in}{1.341898in}}%
\pgfpathcurveto{\pgfqpoint{2.477119in}{1.330848in}}{\pgfqpoint{2.481509in}{1.320249in}}{\pgfqpoint{2.489323in}{1.312436in}}%
\pgfpathcurveto{\pgfqpoint{2.497137in}{1.304622in}}{\pgfqpoint{2.507736in}{1.300232in}}{\pgfqpoint{2.518786in}{1.300232in}}%
\pgfpathclose%
\pgfusepath{stroke,fill}%
\end{pgfscope}%
\begin{pgfscope}%
\pgfpathrectangle{\pgfqpoint{0.800000in}{0.528000in}}{\pgfqpoint{4.960000in}{3.696000in}}%
\pgfusepath{clip}%
\pgfsetbuttcap%
\pgfsetroundjoin%
\definecolor{currentfill}{rgb}{0.000000,0.000000,0.000000}%
\pgfsetfillcolor{currentfill}%
\pgfsetlinewidth{1.003750pt}%
\definecolor{currentstroke}{rgb}{0.000000,0.000000,0.000000}%
\pgfsetstrokecolor{currentstroke}%
\pgfsetdash{}{0pt}%
\pgfpathmoveto{\pgfqpoint{2.518786in}{1.192744in}}%
\pgfpathcurveto{\pgfqpoint{2.529836in}{1.192744in}}{\pgfqpoint{2.540435in}{1.197135in}}{\pgfqpoint{2.548249in}{1.204948in}}%
\pgfpathcurveto{\pgfqpoint{2.556062in}{1.212762in}}{\pgfqpoint{2.560452in}{1.223361in}}{\pgfqpoint{2.560452in}{1.234411in}}%
\pgfpathcurveto{\pgfqpoint{2.560452in}{1.245461in}}{\pgfqpoint{2.556062in}{1.256060in}}{\pgfqpoint{2.548249in}{1.263874in}}%
\pgfpathcurveto{\pgfqpoint{2.540435in}{1.271687in}}{\pgfqpoint{2.529836in}{1.276078in}}{\pgfqpoint{2.518786in}{1.276078in}}%
\pgfpathcurveto{\pgfqpoint{2.507736in}{1.276078in}}{\pgfqpoint{2.497137in}{1.271687in}}{\pgfqpoint{2.489323in}{1.263874in}}%
\pgfpathcurveto{\pgfqpoint{2.481509in}{1.256060in}}{\pgfqpoint{2.477119in}{1.245461in}}{\pgfqpoint{2.477119in}{1.234411in}}%
\pgfpathcurveto{\pgfqpoint{2.477119in}{1.223361in}}{\pgfqpoint{2.481509in}{1.212762in}}{\pgfqpoint{2.489323in}{1.204948in}}%
\pgfpathcurveto{\pgfqpoint{2.497137in}{1.197135in}}{\pgfqpoint{2.507736in}{1.192744in}}{\pgfqpoint{2.518786in}{1.192744in}}%
\pgfpathclose%
\pgfusepath{stroke,fill}%
\end{pgfscope}%
\begin{pgfscope}%
\pgfpathrectangle{\pgfqpoint{0.800000in}{0.528000in}}{\pgfqpoint{4.960000in}{3.696000in}}%
\pgfusepath{clip}%
\pgfsetbuttcap%
\pgfsetroundjoin%
\definecolor{currentfill}{rgb}{0.000000,0.000000,0.000000}%
\pgfsetfillcolor{currentfill}%
\pgfsetlinewidth{1.003750pt}%
\definecolor{currentstroke}{rgb}{0.000000,0.000000,0.000000}%
\pgfsetstrokecolor{currentstroke}%
\pgfsetdash{}{0pt}%
\pgfpathmoveto{\pgfqpoint{2.518786in}{1.128252in}}%
\pgfpathcurveto{\pgfqpoint{2.529836in}{1.128252in}}{\pgfqpoint{2.540435in}{1.132642in}}{\pgfqpoint{2.548249in}{1.140456in}}%
\pgfpathcurveto{\pgfqpoint{2.556062in}{1.148269in}}{\pgfqpoint{2.560452in}{1.158868in}}{\pgfqpoint{2.560452in}{1.169918in}}%
\pgfpathcurveto{\pgfqpoint{2.560452in}{1.180969in}}{\pgfqpoint{2.556062in}{1.191568in}}{\pgfqpoint{2.548249in}{1.199381in}}%
\pgfpathcurveto{\pgfqpoint{2.540435in}{1.207195in}}{\pgfqpoint{2.529836in}{1.211585in}}{\pgfqpoint{2.518786in}{1.211585in}}%
\pgfpathcurveto{\pgfqpoint{2.507736in}{1.211585in}}{\pgfqpoint{2.497137in}{1.207195in}}{\pgfqpoint{2.489323in}{1.199381in}}%
\pgfpathcurveto{\pgfqpoint{2.481509in}{1.191568in}}{\pgfqpoint{2.477119in}{1.180969in}}{\pgfqpoint{2.477119in}{1.169918in}}%
\pgfpathcurveto{\pgfqpoint{2.477119in}{1.158868in}}{\pgfqpoint{2.481509in}{1.148269in}}{\pgfqpoint{2.489323in}{1.140456in}}%
\pgfpathcurveto{\pgfqpoint{2.497137in}{1.132642in}}{\pgfqpoint{2.507736in}{1.128252in}}{\pgfqpoint{2.518786in}{1.128252in}}%
\pgfpathclose%
\pgfusepath{stroke,fill}%
\end{pgfscope}%
\begin{pgfscope}%
\pgfpathrectangle{\pgfqpoint{0.800000in}{0.528000in}}{\pgfqpoint{4.960000in}{3.696000in}}%
\pgfusepath{clip}%
\pgfsetbuttcap%
\pgfsetroundjoin%
\definecolor{currentfill}{rgb}{0.000000,0.000000,0.000000}%
\pgfsetfillcolor{currentfill}%
\pgfsetlinewidth{1.003750pt}%
\definecolor{currentstroke}{rgb}{0.000000,0.000000,0.000000}%
\pgfsetstrokecolor{currentstroke}%
\pgfsetdash{}{0pt}%
\pgfpathmoveto{\pgfqpoint{2.518786in}{1.128252in}}%
\pgfpathcurveto{\pgfqpoint{2.529836in}{1.128252in}}{\pgfqpoint{2.540435in}{1.132642in}}{\pgfqpoint{2.548249in}{1.140456in}}%
\pgfpathcurveto{\pgfqpoint{2.556062in}{1.148269in}}{\pgfqpoint{2.560452in}{1.158868in}}{\pgfqpoint{2.560452in}{1.169918in}}%
\pgfpathcurveto{\pgfqpoint{2.560452in}{1.180969in}}{\pgfqpoint{2.556062in}{1.191568in}}{\pgfqpoint{2.548249in}{1.199381in}}%
\pgfpathcurveto{\pgfqpoint{2.540435in}{1.207195in}}{\pgfqpoint{2.529836in}{1.211585in}}{\pgfqpoint{2.518786in}{1.211585in}}%
\pgfpathcurveto{\pgfqpoint{2.507736in}{1.211585in}}{\pgfqpoint{2.497137in}{1.207195in}}{\pgfqpoint{2.489323in}{1.199381in}}%
\pgfpathcurveto{\pgfqpoint{2.481509in}{1.191568in}}{\pgfqpoint{2.477119in}{1.180969in}}{\pgfqpoint{2.477119in}{1.169918in}}%
\pgfpathcurveto{\pgfqpoint{2.477119in}{1.158868in}}{\pgfqpoint{2.481509in}{1.148269in}}{\pgfqpoint{2.489323in}{1.140456in}}%
\pgfpathcurveto{\pgfqpoint{2.497137in}{1.132642in}}{\pgfqpoint{2.507736in}{1.128252in}}{\pgfqpoint{2.518786in}{1.128252in}}%
\pgfpathclose%
\pgfusepath{stroke,fill}%
\end{pgfscope}%
\begin{pgfscope}%
\pgfpathrectangle{\pgfqpoint{0.800000in}{0.528000in}}{\pgfqpoint{4.960000in}{3.696000in}}%
\pgfusepath{clip}%
\pgfsetbuttcap%
\pgfsetroundjoin%
\definecolor{currentfill}{rgb}{0.000000,0.000000,0.000000}%
\pgfsetfillcolor{currentfill}%
\pgfsetlinewidth{1.003750pt}%
\definecolor{currentstroke}{rgb}{0.000000,0.000000,0.000000}%
\pgfsetstrokecolor{currentstroke}%
\pgfsetdash{}{0pt}%
\pgfpathmoveto{\pgfqpoint{2.518786in}{1.106754in}}%
\pgfpathcurveto{\pgfqpoint{2.529836in}{1.106754in}}{\pgfqpoint{2.540435in}{1.111145in}}{\pgfqpoint{2.548249in}{1.118958in}}%
\pgfpathcurveto{\pgfqpoint{2.556062in}{1.126772in}}{\pgfqpoint{2.560452in}{1.137371in}}{\pgfqpoint{2.560452in}{1.148421in}}%
\pgfpathcurveto{\pgfqpoint{2.560452in}{1.159471in}}{\pgfqpoint{2.556062in}{1.170070in}}{\pgfqpoint{2.548249in}{1.177884in}}%
\pgfpathcurveto{\pgfqpoint{2.540435in}{1.185697in}}{\pgfqpoint{2.529836in}{1.190088in}}{\pgfqpoint{2.518786in}{1.190088in}}%
\pgfpathcurveto{\pgfqpoint{2.507736in}{1.190088in}}{\pgfqpoint{2.497137in}{1.185697in}}{\pgfqpoint{2.489323in}{1.177884in}}%
\pgfpathcurveto{\pgfqpoint{2.481509in}{1.170070in}}{\pgfqpoint{2.477119in}{1.159471in}}{\pgfqpoint{2.477119in}{1.148421in}}%
\pgfpathcurveto{\pgfqpoint{2.477119in}{1.137371in}}{\pgfqpoint{2.481509in}{1.126772in}}{\pgfqpoint{2.489323in}{1.118958in}}%
\pgfpathcurveto{\pgfqpoint{2.497137in}{1.111145in}}{\pgfqpoint{2.507736in}{1.106754in}}{\pgfqpoint{2.518786in}{1.106754in}}%
\pgfpathclose%
\pgfusepath{stroke,fill}%
\end{pgfscope}%
\begin{pgfscope}%
\pgfpathrectangle{\pgfqpoint{0.800000in}{0.528000in}}{\pgfqpoint{4.960000in}{3.696000in}}%
\pgfusepath{clip}%
\pgfsetbuttcap%
\pgfsetroundjoin%
\definecolor{currentfill}{rgb}{0.000000,0.000000,0.000000}%
\pgfsetfillcolor{currentfill}%
\pgfsetlinewidth{1.003750pt}%
\definecolor{currentstroke}{rgb}{0.000000,0.000000,0.000000}%
\pgfsetstrokecolor{currentstroke}%
\pgfsetdash{}{0pt}%
\pgfpathmoveto{\pgfqpoint{2.518786in}{1.214242in}}%
\pgfpathcurveto{\pgfqpoint{2.529836in}{1.214242in}}{\pgfqpoint{2.540435in}{1.218632in}}{\pgfqpoint{2.548249in}{1.226446in}}%
\pgfpathcurveto{\pgfqpoint{2.556062in}{1.234259in}}{\pgfqpoint{2.560452in}{1.244858in}}{\pgfqpoint{2.560452in}{1.255908in}}%
\pgfpathcurveto{\pgfqpoint{2.560452in}{1.266959in}}{\pgfqpoint{2.556062in}{1.277558in}}{\pgfqpoint{2.548249in}{1.285371in}}%
\pgfpathcurveto{\pgfqpoint{2.540435in}{1.293185in}}{\pgfqpoint{2.529836in}{1.297575in}}{\pgfqpoint{2.518786in}{1.297575in}}%
\pgfpathcurveto{\pgfqpoint{2.507736in}{1.297575in}}{\pgfqpoint{2.497137in}{1.293185in}}{\pgfqpoint{2.489323in}{1.285371in}}%
\pgfpathcurveto{\pgfqpoint{2.481509in}{1.277558in}}{\pgfqpoint{2.477119in}{1.266959in}}{\pgfqpoint{2.477119in}{1.255908in}}%
\pgfpathcurveto{\pgfqpoint{2.477119in}{1.244858in}}{\pgfqpoint{2.481509in}{1.234259in}}{\pgfqpoint{2.489323in}{1.226446in}}%
\pgfpathcurveto{\pgfqpoint{2.497137in}{1.218632in}}{\pgfqpoint{2.507736in}{1.214242in}}{\pgfqpoint{2.518786in}{1.214242in}}%
\pgfpathclose%
\pgfusepath{stroke,fill}%
\end{pgfscope}%
\begin{pgfscope}%
\pgfpathrectangle{\pgfqpoint{0.800000in}{0.528000in}}{\pgfqpoint{4.960000in}{3.696000in}}%
\pgfusepath{clip}%
\pgfsetbuttcap%
\pgfsetroundjoin%
\definecolor{currentfill}{rgb}{0.000000,0.000000,0.000000}%
\pgfsetfillcolor{currentfill}%
\pgfsetlinewidth{1.003750pt}%
\definecolor{currentstroke}{rgb}{0.000000,0.000000,0.000000}%
\pgfsetstrokecolor{currentstroke}%
\pgfsetdash{}{0pt}%
\pgfpathmoveto{\pgfqpoint{2.518786in}{1.171247in}}%
\pgfpathcurveto{\pgfqpoint{2.529836in}{1.171247in}}{\pgfqpoint{2.540435in}{1.175637in}}{\pgfqpoint{2.548249in}{1.183451in}}%
\pgfpathcurveto{\pgfqpoint{2.556062in}{1.191264in}}{\pgfqpoint{2.560452in}{1.201863in}}{\pgfqpoint{2.560452in}{1.212913in}}%
\pgfpathcurveto{\pgfqpoint{2.560452in}{1.223964in}}{\pgfqpoint{2.556062in}{1.234563in}}{\pgfqpoint{2.548249in}{1.242376in}}%
\pgfpathcurveto{\pgfqpoint{2.540435in}{1.250190in}}{\pgfqpoint{2.529836in}{1.254580in}}{\pgfqpoint{2.518786in}{1.254580in}}%
\pgfpathcurveto{\pgfqpoint{2.507736in}{1.254580in}}{\pgfqpoint{2.497137in}{1.250190in}}{\pgfqpoint{2.489323in}{1.242376in}}%
\pgfpathcurveto{\pgfqpoint{2.481509in}{1.234563in}}{\pgfqpoint{2.477119in}{1.223964in}}{\pgfqpoint{2.477119in}{1.212913in}}%
\pgfpathcurveto{\pgfqpoint{2.477119in}{1.201863in}}{\pgfqpoint{2.481509in}{1.191264in}}{\pgfqpoint{2.489323in}{1.183451in}}%
\pgfpathcurveto{\pgfqpoint{2.497137in}{1.175637in}}{\pgfqpoint{2.507736in}{1.171247in}}{\pgfqpoint{2.518786in}{1.171247in}}%
\pgfpathclose%
\pgfusepath{stroke,fill}%
\end{pgfscope}%
\begin{pgfscope}%
\pgfpathrectangle{\pgfqpoint{0.800000in}{0.528000in}}{\pgfqpoint{4.960000in}{3.696000in}}%
\pgfusepath{clip}%
\pgfsetbuttcap%
\pgfsetroundjoin%
\definecolor{currentfill}{rgb}{0.000000,0.000000,0.000000}%
\pgfsetfillcolor{currentfill}%
\pgfsetlinewidth{1.003750pt}%
\definecolor{currentstroke}{rgb}{0.000000,0.000000,0.000000}%
\pgfsetstrokecolor{currentstroke}%
\pgfsetdash{}{0pt}%
\pgfpathmoveto{\pgfqpoint{2.518786in}{1.085257in}}%
\pgfpathcurveto{\pgfqpoint{2.529836in}{1.085257in}}{\pgfqpoint{2.540435in}{1.089647in}}{\pgfqpoint{2.548249in}{1.097461in}}%
\pgfpathcurveto{\pgfqpoint{2.556062in}{1.105274in}}{\pgfqpoint{2.560452in}{1.115873in}}{\pgfqpoint{2.560452in}{1.126923in}}%
\pgfpathcurveto{\pgfqpoint{2.560452in}{1.137974in}}{\pgfqpoint{2.556062in}{1.148573in}}{\pgfqpoint{2.548249in}{1.156386in}}%
\pgfpathcurveto{\pgfqpoint{2.540435in}{1.164200in}}{\pgfqpoint{2.529836in}{1.168590in}}{\pgfqpoint{2.518786in}{1.168590in}}%
\pgfpathcurveto{\pgfqpoint{2.507736in}{1.168590in}}{\pgfqpoint{2.497137in}{1.164200in}}{\pgfqpoint{2.489323in}{1.156386in}}%
\pgfpathcurveto{\pgfqpoint{2.481509in}{1.148573in}}{\pgfqpoint{2.477119in}{1.137974in}}{\pgfqpoint{2.477119in}{1.126923in}}%
\pgfpathcurveto{\pgfqpoint{2.477119in}{1.115873in}}{\pgfqpoint{2.481509in}{1.105274in}}{\pgfqpoint{2.489323in}{1.097461in}}%
\pgfpathcurveto{\pgfqpoint{2.497137in}{1.089647in}}{\pgfqpoint{2.507736in}{1.085257in}}{\pgfqpoint{2.518786in}{1.085257in}}%
\pgfpathclose%
\pgfusepath{stroke,fill}%
\end{pgfscope}%
\begin{pgfscope}%
\pgfpathrectangle{\pgfqpoint{0.800000in}{0.528000in}}{\pgfqpoint{4.960000in}{3.696000in}}%
\pgfusepath{clip}%
\pgfsetbuttcap%
\pgfsetroundjoin%
\definecolor{currentfill}{rgb}{0.000000,0.000000,0.000000}%
\pgfsetfillcolor{currentfill}%
\pgfsetlinewidth{1.003750pt}%
\definecolor{currentstroke}{rgb}{0.000000,0.000000,0.000000}%
\pgfsetstrokecolor{currentstroke}%
\pgfsetdash{}{0pt}%
\pgfpathmoveto{\pgfqpoint{2.518786in}{1.020764in}}%
\pgfpathcurveto{\pgfqpoint{2.529836in}{1.020764in}}{\pgfqpoint{2.540435in}{1.025155in}}{\pgfqpoint{2.548249in}{1.032968in}}%
\pgfpathcurveto{\pgfqpoint{2.556062in}{1.040782in}}{\pgfqpoint{2.560452in}{1.051381in}}{\pgfqpoint{2.560452in}{1.062431in}}%
\pgfpathcurveto{\pgfqpoint{2.560452in}{1.073481in}}{\pgfqpoint{2.556062in}{1.084080in}}{\pgfqpoint{2.548249in}{1.091894in}}%
\pgfpathcurveto{\pgfqpoint{2.540435in}{1.099707in}}{\pgfqpoint{2.529836in}{1.104098in}}{\pgfqpoint{2.518786in}{1.104098in}}%
\pgfpathcurveto{\pgfqpoint{2.507736in}{1.104098in}}{\pgfqpoint{2.497137in}{1.099707in}}{\pgfqpoint{2.489323in}{1.091894in}}%
\pgfpathcurveto{\pgfqpoint{2.481509in}{1.084080in}}{\pgfqpoint{2.477119in}{1.073481in}}{\pgfqpoint{2.477119in}{1.062431in}}%
\pgfpathcurveto{\pgfqpoint{2.477119in}{1.051381in}}{\pgfqpoint{2.481509in}{1.040782in}}{\pgfqpoint{2.489323in}{1.032968in}}%
\pgfpathcurveto{\pgfqpoint{2.497137in}{1.025155in}}{\pgfqpoint{2.507736in}{1.020764in}}{\pgfqpoint{2.518786in}{1.020764in}}%
\pgfpathclose%
\pgfusepath{stroke,fill}%
\end{pgfscope}%
\begin{pgfscope}%
\pgfpathrectangle{\pgfqpoint{0.800000in}{0.528000in}}{\pgfqpoint{4.960000in}{3.696000in}}%
\pgfusepath{clip}%
\pgfsetbuttcap%
\pgfsetroundjoin%
\definecolor{currentfill}{rgb}{0.000000,0.000000,0.000000}%
\pgfsetfillcolor{currentfill}%
\pgfsetlinewidth{1.003750pt}%
\definecolor{currentstroke}{rgb}{0.000000,0.000000,0.000000}%
\pgfsetstrokecolor{currentstroke}%
\pgfsetdash{}{0pt}%
\pgfpathmoveto{\pgfqpoint{2.518786in}{1.149749in}}%
\pgfpathcurveto{\pgfqpoint{2.529836in}{1.149749in}}{\pgfqpoint{2.540435in}{1.154140in}}{\pgfqpoint{2.548249in}{1.161953in}}%
\pgfpathcurveto{\pgfqpoint{2.556062in}{1.169767in}}{\pgfqpoint{2.560452in}{1.180366in}}{\pgfqpoint{2.560452in}{1.191416in}}%
\pgfpathcurveto{\pgfqpoint{2.560452in}{1.202466in}}{\pgfqpoint{2.556062in}{1.213065in}}{\pgfqpoint{2.548249in}{1.220879in}}%
\pgfpathcurveto{\pgfqpoint{2.540435in}{1.228692in}}{\pgfqpoint{2.529836in}{1.233083in}}{\pgfqpoint{2.518786in}{1.233083in}}%
\pgfpathcurveto{\pgfqpoint{2.507736in}{1.233083in}}{\pgfqpoint{2.497137in}{1.228692in}}{\pgfqpoint{2.489323in}{1.220879in}}%
\pgfpathcurveto{\pgfqpoint{2.481509in}{1.213065in}}{\pgfqpoint{2.477119in}{1.202466in}}{\pgfqpoint{2.477119in}{1.191416in}}%
\pgfpathcurveto{\pgfqpoint{2.477119in}{1.180366in}}{\pgfqpoint{2.481509in}{1.169767in}}{\pgfqpoint{2.489323in}{1.161953in}}%
\pgfpathcurveto{\pgfqpoint{2.497137in}{1.154140in}}{\pgfqpoint{2.507736in}{1.149749in}}{\pgfqpoint{2.518786in}{1.149749in}}%
\pgfpathclose%
\pgfusepath{stroke,fill}%
\end{pgfscope}%
\begin{pgfscope}%
\pgfpathrectangle{\pgfqpoint{0.800000in}{0.528000in}}{\pgfqpoint{4.960000in}{3.696000in}}%
\pgfusepath{clip}%
\pgfsetbuttcap%
\pgfsetroundjoin%
\definecolor{currentfill}{rgb}{0.000000,0.000000,0.000000}%
\pgfsetfillcolor{currentfill}%
\pgfsetlinewidth{1.003750pt}%
\definecolor{currentstroke}{rgb}{0.000000,0.000000,0.000000}%
\pgfsetstrokecolor{currentstroke}%
\pgfsetdash{}{0pt}%
\pgfpathmoveto{\pgfqpoint{2.518786in}{1.128252in}}%
\pgfpathcurveto{\pgfqpoint{2.529836in}{1.128252in}}{\pgfqpoint{2.540435in}{1.132642in}}{\pgfqpoint{2.548249in}{1.140456in}}%
\pgfpathcurveto{\pgfqpoint{2.556062in}{1.148269in}}{\pgfqpoint{2.560452in}{1.158868in}}{\pgfqpoint{2.560452in}{1.169918in}}%
\pgfpathcurveto{\pgfqpoint{2.560452in}{1.180969in}}{\pgfqpoint{2.556062in}{1.191568in}}{\pgfqpoint{2.548249in}{1.199381in}}%
\pgfpathcurveto{\pgfqpoint{2.540435in}{1.207195in}}{\pgfqpoint{2.529836in}{1.211585in}}{\pgfqpoint{2.518786in}{1.211585in}}%
\pgfpathcurveto{\pgfqpoint{2.507736in}{1.211585in}}{\pgfqpoint{2.497137in}{1.207195in}}{\pgfqpoint{2.489323in}{1.199381in}}%
\pgfpathcurveto{\pgfqpoint{2.481509in}{1.191568in}}{\pgfqpoint{2.477119in}{1.180969in}}{\pgfqpoint{2.477119in}{1.169918in}}%
\pgfpathcurveto{\pgfqpoint{2.477119in}{1.158868in}}{\pgfqpoint{2.481509in}{1.148269in}}{\pgfqpoint{2.489323in}{1.140456in}}%
\pgfpathcurveto{\pgfqpoint{2.497137in}{1.132642in}}{\pgfqpoint{2.507736in}{1.128252in}}{\pgfqpoint{2.518786in}{1.128252in}}%
\pgfpathclose%
\pgfusepath{stroke,fill}%
\end{pgfscope}%
\begin{pgfscope}%
\pgfpathrectangle{\pgfqpoint{0.800000in}{0.528000in}}{\pgfqpoint{4.960000in}{3.696000in}}%
\pgfusepath{clip}%
\pgfsetbuttcap%
\pgfsetroundjoin%
\definecolor{currentfill}{rgb}{0.000000,0.000000,0.000000}%
\pgfsetfillcolor{currentfill}%
\pgfsetlinewidth{1.003750pt}%
\definecolor{currentstroke}{rgb}{0.000000,0.000000,0.000000}%
\pgfsetstrokecolor{currentstroke}%
\pgfsetdash{}{0pt}%
\pgfpathmoveto{\pgfqpoint{2.518786in}{1.192744in}}%
\pgfpathcurveto{\pgfqpoint{2.529836in}{1.192744in}}{\pgfqpoint{2.540435in}{1.197135in}}{\pgfqpoint{2.548249in}{1.204948in}}%
\pgfpathcurveto{\pgfqpoint{2.556062in}{1.212762in}}{\pgfqpoint{2.560452in}{1.223361in}}{\pgfqpoint{2.560452in}{1.234411in}}%
\pgfpathcurveto{\pgfqpoint{2.560452in}{1.245461in}}{\pgfqpoint{2.556062in}{1.256060in}}{\pgfqpoint{2.548249in}{1.263874in}}%
\pgfpathcurveto{\pgfqpoint{2.540435in}{1.271687in}}{\pgfqpoint{2.529836in}{1.276078in}}{\pgfqpoint{2.518786in}{1.276078in}}%
\pgfpathcurveto{\pgfqpoint{2.507736in}{1.276078in}}{\pgfqpoint{2.497137in}{1.271687in}}{\pgfqpoint{2.489323in}{1.263874in}}%
\pgfpathcurveto{\pgfqpoint{2.481509in}{1.256060in}}{\pgfqpoint{2.477119in}{1.245461in}}{\pgfqpoint{2.477119in}{1.234411in}}%
\pgfpathcurveto{\pgfqpoint{2.477119in}{1.223361in}}{\pgfqpoint{2.481509in}{1.212762in}}{\pgfqpoint{2.489323in}{1.204948in}}%
\pgfpathcurveto{\pgfqpoint{2.497137in}{1.197135in}}{\pgfqpoint{2.507736in}{1.192744in}}{\pgfqpoint{2.518786in}{1.192744in}}%
\pgfpathclose%
\pgfusepath{stroke,fill}%
\end{pgfscope}%
\begin{pgfscope}%
\pgfpathrectangle{\pgfqpoint{0.800000in}{0.528000in}}{\pgfqpoint{4.960000in}{3.696000in}}%
\pgfusepath{clip}%
\pgfsetbuttcap%
\pgfsetroundjoin%
\definecolor{currentfill}{rgb}{0.000000,0.000000,0.000000}%
\pgfsetfillcolor{currentfill}%
\pgfsetlinewidth{1.003750pt}%
\definecolor{currentstroke}{rgb}{0.000000,0.000000,0.000000}%
\pgfsetstrokecolor{currentstroke}%
\pgfsetdash{}{0pt}%
\pgfpathmoveto{\pgfqpoint{2.518786in}{1.085257in}}%
\pgfpathcurveto{\pgfqpoint{2.529836in}{1.085257in}}{\pgfqpoint{2.540435in}{1.089647in}}{\pgfqpoint{2.548249in}{1.097461in}}%
\pgfpathcurveto{\pgfqpoint{2.556062in}{1.105274in}}{\pgfqpoint{2.560452in}{1.115873in}}{\pgfqpoint{2.560452in}{1.126923in}}%
\pgfpathcurveto{\pgfqpoint{2.560452in}{1.137974in}}{\pgfqpoint{2.556062in}{1.148573in}}{\pgfqpoint{2.548249in}{1.156386in}}%
\pgfpathcurveto{\pgfqpoint{2.540435in}{1.164200in}}{\pgfqpoint{2.529836in}{1.168590in}}{\pgfqpoint{2.518786in}{1.168590in}}%
\pgfpathcurveto{\pgfqpoint{2.507736in}{1.168590in}}{\pgfqpoint{2.497137in}{1.164200in}}{\pgfqpoint{2.489323in}{1.156386in}}%
\pgfpathcurveto{\pgfqpoint{2.481509in}{1.148573in}}{\pgfqpoint{2.477119in}{1.137974in}}{\pgfqpoint{2.477119in}{1.126923in}}%
\pgfpathcurveto{\pgfqpoint{2.477119in}{1.115873in}}{\pgfqpoint{2.481509in}{1.105274in}}{\pgfqpoint{2.489323in}{1.097461in}}%
\pgfpathcurveto{\pgfqpoint{2.497137in}{1.089647in}}{\pgfqpoint{2.507736in}{1.085257in}}{\pgfqpoint{2.518786in}{1.085257in}}%
\pgfpathclose%
\pgfusepath{stroke,fill}%
\end{pgfscope}%
\begin{pgfscope}%
\pgfpathrectangle{\pgfqpoint{0.800000in}{0.528000in}}{\pgfqpoint{4.960000in}{3.696000in}}%
\pgfusepath{clip}%
\pgfsetbuttcap%
\pgfsetroundjoin%
\definecolor{currentfill}{rgb}{0.000000,0.000000,0.000000}%
\pgfsetfillcolor{currentfill}%
\pgfsetlinewidth{1.003750pt}%
\definecolor{currentstroke}{rgb}{0.000000,0.000000,0.000000}%
\pgfsetstrokecolor{currentstroke}%
\pgfsetdash{}{0pt}%
\pgfpathmoveto{\pgfqpoint{2.518786in}{1.171247in}}%
\pgfpathcurveto{\pgfqpoint{2.529836in}{1.171247in}}{\pgfqpoint{2.540435in}{1.175637in}}{\pgfqpoint{2.548249in}{1.183451in}}%
\pgfpathcurveto{\pgfqpoint{2.556062in}{1.191264in}}{\pgfqpoint{2.560452in}{1.201863in}}{\pgfqpoint{2.560452in}{1.212913in}}%
\pgfpathcurveto{\pgfqpoint{2.560452in}{1.223964in}}{\pgfqpoint{2.556062in}{1.234563in}}{\pgfqpoint{2.548249in}{1.242376in}}%
\pgfpathcurveto{\pgfqpoint{2.540435in}{1.250190in}}{\pgfqpoint{2.529836in}{1.254580in}}{\pgfqpoint{2.518786in}{1.254580in}}%
\pgfpathcurveto{\pgfqpoint{2.507736in}{1.254580in}}{\pgfqpoint{2.497137in}{1.250190in}}{\pgfqpoint{2.489323in}{1.242376in}}%
\pgfpathcurveto{\pgfqpoint{2.481509in}{1.234563in}}{\pgfqpoint{2.477119in}{1.223964in}}{\pgfqpoint{2.477119in}{1.212913in}}%
\pgfpathcurveto{\pgfqpoint{2.477119in}{1.201863in}}{\pgfqpoint{2.481509in}{1.191264in}}{\pgfqpoint{2.489323in}{1.183451in}}%
\pgfpathcurveto{\pgfqpoint{2.497137in}{1.175637in}}{\pgfqpoint{2.507736in}{1.171247in}}{\pgfqpoint{2.518786in}{1.171247in}}%
\pgfpathclose%
\pgfusepath{stroke,fill}%
\end{pgfscope}%
\begin{pgfscope}%
\pgfpathrectangle{\pgfqpoint{0.800000in}{0.528000in}}{\pgfqpoint{4.960000in}{3.696000in}}%
\pgfusepath{clip}%
\pgfsetbuttcap%
\pgfsetroundjoin%
\definecolor{currentfill}{rgb}{0.000000,0.000000,0.000000}%
\pgfsetfillcolor{currentfill}%
\pgfsetlinewidth{1.003750pt}%
\definecolor{currentstroke}{rgb}{0.000000,0.000000,0.000000}%
\pgfsetstrokecolor{currentstroke}%
\pgfsetdash{}{0pt}%
\pgfpathmoveto{\pgfqpoint{2.518786in}{1.128252in}}%
\pgfpathcurveto{\pgfqpoint{2.529836in}{1.128252in}}{\pgfqpoint{2.540435in}{1.132642in}}{\pgfqpoint{2.548249in}{1.140456in}}%
\pgfpathcurveto{\pgfqpoint{2.556062in}{1.148269in}}{\pgfqpoint{2.560452in}{1.158868in}}{\pgfqpoint{2.560452in}{1.169918in}}%
\pgfpathcurveto{\pgfqpoint{2.560452in}{1.180969in}}{\pgfqpoint{2.556062in}{1.191568in}}{\pgfqpoint{2.548249in}{1.199381in}}%
\pgfpathcurveto{\pgfqpoint{2.540435in}{1.207195in}}{\pgfqpoint{2.529836in}{1.211585in}}{\pgfqpoint{2.518786in}{1.211585in}}%
\pgfpathcurveto{\pgfqpoint{2.507736in}{1.211585in}}{\pgfqpoint{2.497137in}{1.207195in}}{\pgfqpoint{2.489323in}{1.199381in}}%
\pgfpathcurveto{\pgfqpoint{2.481509in}{1.191568in}}{\pgfqpoint{2.477119in}{1.180969in}}{\pgfqpoint{2.477119in}{1.169918in}}%
\pgfpathcurveto{\pgfqpoint{2.477119in}{1.158868in}}{\pgfqpoint{2.481509in}{1.148269in}}{\pgfqpoint{2.489323in}{1.140456in}}%
\pgfpathcurveto{\pgfqpoint{2.497137in}{1.132642in}}{\pgfqpoint{2.507736in}{1.128252in}}{\pgfqpoint{2.518786in}{1.128252in}}%
\pgfpathclose%
\pgfusepath{stroke,fill}%
\end{pgfscope}%
\begin{pgfscope}%
\pgfpathrectangle{\pgfqpoint{0.800000in}{0.528000in}}{\pgfqpoint{4.960000in}{3.696000in}}%
\pgfusepath{clip}%
\pgfsetbuttcap%
\pgfsetroundjoin%
\definecolor{currentfill}{rgb}{0.000000,0.000000,0.000000}%
\pgfsetfillcolor{currentfill}%
\pgfsetlinewidth{1.003750pt}%
\definecolor{currentstroke}{rgb}{0.000000,0.000000,0.000000}%
\pgfsetstrokecolor{currentstroke}%
\pgfsetdash{}{0pt}%
\pgfpathmoveto{\pgfqpoint{2.518786in}{1.192744in}}%
\pgfpathcurveto{\pgfqpoint{2.529836in}{1.192744in}}{\pgfqpoint{2.540435in}{1.197135in}}{\pgfqpoint{2.548249in}{1.204948in}}%
\pgfpathcurveto{\pgfqpoint{2.556062in}{1.212762in}}{\pgfqpoint{2.560452in}{1.223361in}}{\pgfqpoint{2.560452in}{1.234411in}}%
\pgfpathcurveto{\pgfqpoint{2.560452in}{1.245461in}}{\pgfqpoint{2.556062in}{1.256060in}}{\pgfqpoint{2.548249in}{1.263874in}}%
\pgfpathcurveto{\pgfqpoint{2.540435in}{1.271687in}}{\pgfqpoint{2.529836in}{1.276078in}}{\pgfqpoint{2.518786in}{1.276078in}}%
\pgfpathcurveto{\pgfqpoint{2.507736in}{1.276078in}}{\pgfqpoint{2.497137in}{1.271687in}}{\pgfqpoint{2.489323in}{1.263874in}}%
\pgfpathcurveto{\pgfqpoint{2.481509in}{1.256060in}}{\pgfqpoint{2.477119in}{1.245461in}}{\pgfqpoint{2.477119in}{1.234411in}}%
\pgfpathcurveto{\pgfqpoint{2.477119in}{1.223361in}}{\pgfqpoint{2.481509in}{1.212762in}}{\pgfqpoint{2.489323in}{1.204948in}}%
\pgfpathcurveto{\pgfqpoint{2.497137in}{1.197135in}}{\pgfqpoint{2.507736in}{1.192744in}}{\pgfqpoint{2.518786in}{1.192744in}}%
\pgfpathclose%
\pgfusepath{stroke,fill}%
\end{pgfscope}%
\begin{pgfscope}%
\pgfpathrectangle{\pgfqpoint{0.800000in}{0.528000in}}{\pgfqpoint{4.960000in}{3.696000in}}%
\pgfusepath{clip}%
\pgfsetbuttcap%
\pgfsetroundjoin%
\definecolor{currentfill}{rgb}{0.000000,0.000000,0.000000}%
\pgfsetfillcolor{currentfill}%
\pgfsetlinewidth{1.003750pt}%
\definecolor{currentstroke}{rgb}{0.000000,0.000000,0.000000}%
\pgfsetstrokecolor{currentstroke}%
\pgfsetdash{}{0pt}%
\pgfpathmoveto{\pgfqpoint{2.518786in}{1.149749in}}%
\pgfpathcurveto{\pgfqpoint{2.529836in}{1.149749in}}{\pgfqpoint{2.540435in}{1.154140in}}{\pgfqpoint{2.548249in}{1.161953in}}%
\pgfpathcurveto{\pgfqpoint{2.556062in}{1.169767in}}{\pgfqpoint{2.560452in}{1.180366in}}{\pgfqpoint{2.560452in}{1.191416in}}%
\pgfpathcurveto{\pgfqpoint{2.560452in}{1.202466in}}{\pgfqpoint{2.556062in}{1.213065in}}{\pgfqpoint{2.548249in}{1.220879in}}%
\pgfpathcurveto{\pgfqpoint{2.540435in}{1.228692in}}{\pgfqpoint{2.529836in}{1.233083in}}{\pgfqpoint{2.518786in}{1.233083in}}%
\pgfpathcurveto{\pgfqpoint{2.507736in}{1.233083in}}{\pgfqpoint{2.497137in}{1.228692in}}{\pgfqpoint{2.489323in}{1.220879in}}%
\pgfpathcurveto{\pgfqpoint{2.481509in}{1.213065in}}{\pgfqpoint{2.477119in}{1.202466in}}{\pgfqpoint{2.477119in}{1.191416in}}%
\pgfpathcurveto{\pgfqpoint{2.477119in}{1.180366in}}{\pgfqpoint{2.481509in}{1.169767in}}{\pgfqpoint{2.489323in}{1.161953in}}%
\pgfpathcurveto{\pgfqpoint{2.497137in}{1.154140in}}{\pgfqpoint{2.507736in}{1.149749in}}{\pgfqpoint{2.518786in}{1.149749in}}%
\pgfpathclose%
\pgfusepath{stroke,fill}%
\end{pgfscope}%
\begin{pgfscope}%
\pgfpathrectangle{\pgfqpoint{0.800000in}{0.528000in}}{\pgfqpoint{4.960000in}{3.696000in}}%
\pgfusepath{clip}%
\pgfsetbuttcap%
\pgfsetroundjoin%
\definecolor{currentfill}{rgb}{0.000000,0.000000,0.000000}%
\pgfsetfillcolor{currentfill}%
\pgfsetlinewidth{1.003750pt}%
\definecolor{currentstroke}{rgb}{0.000000,0.000000,0.000000}%
\pgfsetstrokecolor{currentstroke}%
\pgfsetdash{}{0pt}%
\pgfpathmoveto{\pgfqpoint{2.518786in}{1.128252in}}%
\pgfpathcurveto{\pgfqpoint{2.529836in}{1.128252in}}{\pgfqpoint{2.540435in}{1.132642in}}{\pgfqpoint{2.548249in}{1.140456in}}%
\pgfpathcurveto{\pgfqpoint{2.556062in}{1.148269in}}{\pgfqpoint{2.560452in}{1.158868in}}{\pgfqpoint{2.560452in}{1.169918in}}%
\pgfpathcurveto{\pgfqpoint{2.560452in}{1.180969in}}{\pgfqpoint{2.556062in}{1.191568in}}{\pgfqpoint{2.548249in}{1.199381in}}%
\pgfpathcurveto{\pgfqpoint{2.540435in}{1.207195in}}{\pgfqpoint{2.529836in}{1.211585in}}{\pgfqpoint{2.518786in}{1.211585in}}%
\pgfpathcurveto{\pgfqpoint{2.507736in}{1.211585in}}{\pgfqpoint{2.497137in}{1.207195in}}{\pgfqpoint{2.489323in}{1.199381in}}%
\pgfpathcurveto{\pgfqpoint{2.481509in}{1.191568in}}{\pgfqpoint{2.477119in}{1.180969in}}{\pgfqpoint{2.477119in}{1.169918in}}%
\pgfpathcurveto{\pgfqpoint{2.477119in}{1.158868in}}{\pgfqpoint{2.481509in}{1.148269in}}{\pgfqpoint{2.489323in}{1.140456in}}%
\pgfpathcurveto{\pgfqpoint{2.497137in}{1.132642in}}{\pgfqpoint{2.507736in}{1.128252in}}{\pgfqpoint{2.518786in}{1.128252in}}%
\pgfpathclose%
\pgfusepath{stroke,fill}%
\end{pgfscope}%
\begin{pgfscope}%
\pgfpathrectangle{\pgfqpoint{0.800000in}{0.528000in}}{\pgfqpoint{4.960000in}{3.696000in}}%
\pgfusepath{clip}%
\pgfsetbuttcap%
\pgfsetroundjoin%
\definecolor{currentfill}{rgb}{0.000000,0.000000,0.000000}%
\pgfsetfillcolor{currentfill}%
\pgfsetlinewidth{1.003750pt}%
\definecolor{currentstroke}{rgb}{0.000000,0.000000,0.000000}%
\pgfsetstrokecolor{currentstroke}%
\pgfsetdash{}{0pt}%
\pgfpathmoveto{\pgfqpoint{2.518786in}{1.106754in}}%
\pgfpathcurveto{\pgfqpoint{2.529836in}{1.106754in}}{\pgfqpoint{2.540435in}{1.111145in}}{\pgfqpoint{2.548249in}{1.118958in}}%
\pgfpathcurveto{\pgfqpoint{2.556062in}{1.126772in}}{\pgfqpoint{2.560452in}{1.137371in}}{\pgfqpoint{2.560452in}{1.148421in}}%
\pgfpathcurveto{\pgfqpoint{2.560452in}{1.159471in}}{\pgfqpoint{2.556062in}{1.170070in}}{\pgfqpoint{2.548249in}{1.177884in}}%
\pgfpathcurveto{\pgfqpoint{2.540435in}{1.185697in}}{\pgfqpoint{2.529836in}{1.190088in}}{\pgfqpoint{2.518786in}{1.190088in}}%
\pgfpathcurveto{\pgfqpoint{2.507736in}{1.190088in}}{\pgfqpoint{2.497137in}{1.185697in}}{\pgfqpoint{2.489323in}{1.177884in}}%
\pgfpathcurveto{\pgfqpoint{2.481509in}{1.170070in}}{\pgfqpoint{2.477119in}{1.159471in}}{\pgfqpoint{2.477119in}{1.148421in}}%
\pgfpathcurveto{\pgfqpoint{2.477119in}{1.137371in}}{\pgfqpoint{2.481509in}{1.126772in}}{\pgfqpoint{2.489323in}{1.118958in}}%
\pgfpathcurveto{\pgfqpoint{2.497137in}{1.111145in}}{\pgfqpoint{2.507736in}{1.106754in}}{\pgfqpoint{2.518786in}{1.106754in}}%
\pgfpathclose%
\pgfusepath{stroke,fill}%
\end{pgfscope}%
\begin{pgfscope}%
\pgfpathrectangle{\pgfqpoint{0.800000in}{0.528000in}}{\pgfqpoint{4.960000in}{3.696000in}}%
\pgfusepath{clip}%
\pgfsetbuttcap%
\pgfsetroundjoin%
\definecolor{currentfill}{rgb}{0.000000,0.000000,0.000000}%
\pgfsetfillcolor{currentfill}%
\pgfsetlinewidth{1.003750pt}%
\definecolor{currentstroke}{rgb}{0.000000,0.000000,0.000000}%
\pgfsetstrokecolor{currentstroke}%
\pgfsetdash{}{0pt}%
\pgfpathmoveto{\pgfqpoint{2.518786in}{1.235739in}}%
\pgfpathcurveto{\pgfqpoint{2.529836in}{1.235739in}}{\pgfqpoint{2.540435in}{1.240130in}}{\pgfqpoint{2.548249in}{1.247943in}}%
\pgfpathcurveto{\pgfqpoint{2.556062in}{1.255757in}}{\pgfqpoint{2.560452in}{1.266356in}}{\pgfqpoint{2.560452in}{1.277406in}}%
\pgfpathcurveto{\pgfqpoint{2.560452in}{1.288456in}}{\pgfqpoint{2.556062in}{1.299055in}}{\pgfqpoint{2.548249in}{1.306869in}}%
\pgfpathcurveto{\pgfqpoint{2.540435in}{1.314682in}}{\pgfqpoint{2.529836in}{1.319073in}}{\pgfqpoint{2.518786in}{1.319073in}}%
\pgfpathcurveto{\pgfqpoint{2.507736in}{1.319073in}}{\pgfqpoint{2.497137in}{1.314682in}}{\pgfqpoint{2.489323in}{1.306869in}}%
\pgfpathcurveto{\pgfqpoint{2.481509in}{1.299055in}}{\pgfqpoint{2.477119in}{1.288456in}}{\pgfqpoint{2.477119in}{1.277406in}}%
\pgfpathcurveto{\pgfqpoint{2.477119in}{1.266356in}}{\pgfqpoint{2.481509in}{1.255757in}}{\pgfqpoint{2.489323in}{1.247943in}}%
\pgfpathcurveto{\pgfqpoint{2.497137in}{1.240130in}}{\pgfqpoint{2.507736in}{1.235739in}}{\pgfqpoint{2.518786in}{1.235739in}}%
\pgfpathclose%
\pgfusepath{stroke,fill}%
\end{pgfscope}%
\begin{pgfscope}%
\pgfpathrectangle{\pgfqpoint{0.800000in}{0.528000in}}{\pgfqpoint{4.960000in}{3.696000in}}%
\pgfusepath{clip}%
\pgfsetbuttcap%
\pgfsetroundjoin%
\definecolor{currentfill}{rgb}{0.000000,0.000000,0.000000}%
\pgfsetfillcolor{currentfill}%
\pgfsetlinewidth{1.003750pt}%
\definecolor{currentstroke}{rgb}{0.000000,0.000000,0.000000}%
\pgfsetstrokecolor{currentstroke}%
\pgfsetdash{}{0pt}%
\pgfpathmoveto{\pgfqpoint{2.518786in}{1.149749in}}%
\pgfpathcurveto{\pgfqpoint{2.529836in}{1.149749in}}{\pgfqpoint{2.540435in}{1.154140in}}{\pgfqpoint{2.548249in}{1.161953in}}%
\pgfpathcurveto{\pgfqpoint{2.556062in}{1.169767in}}{\pgfqpoint{2.560452in}{1.180366in}}{\pgfqpoint{2.560452in}{1.191416in}}%
\pgfpathcurveto{\pgfqpoint{2.560452in}{1.202466in}}{\pgfqpoint{2.556062in}{1.213065in}}{\pgfqpoint{2.548249in}{1.220879in}}%
\pgfpathcurveto{\pgfqpoint{2.540435in}{1.228692in}}{\pgfqpoint{2.529836in}{1.233083in}}{\pgfqpoint{2.518786in}{1.233083in}}%
\pgfpathcurveto{\pgfqpoint{2.507736in}{1.233083in}}{\pgfqpoint{2.497137in}{1.228692in}}{\pgfqpoint{2.489323in}{1.220879in}}%
\pgfpathcurveto{\pgfqpoint{2.481509in}{1.213065in}}{\pgfqpoint{2.477119in}{1.202466in}}{\pgfqpoint{2.477119in}{1.191416in}}%
\pgfpathcurveto{\pgfqpoint{2.477119in}{1.180366in}}{\pgfqpoint{2.481509in}{1.169767in}}{\pgfqpoint{2.489323in}{1.161953in}}%
\pgfpathcurveto{\pgfqpoint{2.497137in}{1.154140in}}{\pgfqpoint{2.507736in}{1.149749in}}{\pgfqpoint{2.518786in}{1.149749in}}%
\pgfpathclose%
\pgfusepath{stroke,fill}%
\end{pgfscope}%
\begin{pgfscope}%
\pgfpathrectangle{\pgfqpoint{0.800000in}{0.528000in}}{\pgfqpoint{4.960000in}{3.696000in}}%
\pgfusepath{clip}%
\pgfsetbuttcap%
\pgfsetroundjoin%
\definecolor{currentfill}{rgb}{0.000000,0.000000,0.000000}%
\pgfsetfillcolor{currentfill}%
\pgfsetlinewidth{1.003750pt}%
\definecolor{currentstroke}{rgb}{0.000000,0.000000,0.000000}%
\pgfsetstrokecolor{currentstroke}%
\pgfsetdash{}{0pt}%
\pgfpathmoveto{\pgfqpoint{2.518786in}{1.214242in}}%
\pgfpathcurveto{\pgfqpoint{2.529836in}{1.214242in}}{\pgfqpoint{2.540435in}{1.218632in}}{\pgfqpoint{2.548249in}{1.226446in}}%
\pgfpathcurveto{\pgfqpoint{2.556062in}{1.234259in}}{\pgfqpoint{2.560452in}{1.244858in}}{\pgfqpoint{2.560452in}{1.255908in}}%
\pgfpathcurveto{\pgfqpoint{2.560452in}{1.266959in}}{\pgfqpoint{2.556062in}{1.277558in}}{\pgfqpoint{2.548249in}{1.285371in}}%
\pgfpathcurveto{\pgfqpoint{2.540435in}{1.293185in}}{\pgfqpoint{2.529836in}{1.297575in}}{\pgfqpoint{2.518786in}{1.297575in}}%
\pgfpathcurveto{\pgfqpoint{2.507736in}{1.297575in}}{\pgfqpoint{2.497137in}{1.293185in}}{\pgfqpoint{2.489323in}{1.285371in}}%
\pgfpathcurveto{\pgfqpoint{2.481509in}{1.277558in}}{\pgfqpoint{2.477119in}{1.266959in}}{\pgfqpoint{2.477119in}{1.255908in}}%
\pgfpathcurveto{\pgfqpoint{2.477119in}{1.244858in}}{\pgfqpoint{2.481509in}{1.234259in}}{\pgfqpoint{2.489323in}{1.226446in}}%
\pgfpathcurveto{\pgfqpoint{2.497137in}{1.218632in}}{\pgfqpoint{2.507736in}{1.214242in}}{\pgfqpoint{2.518786in}{1.214242in}}%
\pgfpathclose%
\pgfusepath{stroke,fill}%
\end{pgfscope}%
\begin{pgfscope}%
\pgfpathrectangle{\pgfqpoint{0.800000in}{0.528000in}}{\pgfqpoint{4.960000in}{3.696000in}}%
\pgfusepath{clip}%
\pgfsetbuttcap%
\pgfsetroundjoin%
\definecolor{currentfill}{rgb}{0.000000,0.000000,0.000000}%
\pgfsetfillcolor{currentfill}%
\pgfsetlinewidth{1.003750pt}%
\definecolor{currentstroke}{rgb}{0.000000,0.000000,0.000000}%
\pgfsetstrokecolor{currentstroke}%
\pgfsetdash{}{0pt}%
\pgfpathmoveto{\pgfqpoint{2.518786in}{1.106754in}}%
\pgfpathcurveto{\pgfqpoint{2.529836in}{1.106754in}}{\pgfqpoint{2.540435in}{1.111145in}}{\pgfqpoint{2.548249in}{1.118958in}}%
\pgfpathcurveto{\pgfqpoint{2.556062in}{1.126772in}}{\pgfqpoint{2.560452in}{1.137371in}}{\pgfqpoint{2.560452in}{1.148421in}}%
\pgfpathcurveto{\pgfqpoint{2.560452in}{1.159471in}}{\pgfqpoint{2.556062in}{1.170070in}}{\pgfqpoint{2.548249in}{1.177884in}}%
\pgfpathcurveto{\pgfqpoint{2.540435in}{1.185697in}}{\pgfqpoint{2.529836in}{1.190088in}}{\pgfqpoint{2.518786in}{1.190088in}}%
\pgfpathcurveto{\pgfqpoint{2.507736in}{1.190088in}}{\pgfqpoint{2.497137in}{1.185697in}}{\pgfqpoint{2.489323in}{1.177884in}}%
\pgfpathcurveto{\pgfqpoint{2.481509in}{1.170070in}}{\pgfqpoint{2.477119in}{1.159471in}}{\pgfqpoint{2.477119in}{1.148421in}}%
\pgfpathcurveto{\pgfqpoint{2.477119in}{1.137371in}}{\pgfqpoint{2.481509in}{1.126772in}}{\pgfqpoint{2.489323in}{1.118958in}}%
\pgfpathcurveto{\pgfqpoint{2.497137in}{1.111145in}}{\pgfqpoint{2.507736in}{1.106754in}}{\pgfqpoint{2.518786in}{1.106754in}}%
\pgfpathclose%
\pgfusepath{stroke,fill}%
\end{pgfscope}%
\begin{pgfscope}%
\pgfpathrectangle{\pgfqpoint{0.800000in}{0.528000in}}{\pgfqpoint{4.960000in}{3.696000in}}%
\pgfusepath{clip}%
\pgfsetbuttcap%
\pgfsetroundjoin%
\definecolor{currentfill}{rgb}{0.000000,0.000000,0.000000}%
\pgfsetfillcolor{currentfill}%
\pgfsetlinewidth{1.003750pt}%
\definecolor{currentstroke}{rgb}{0.000000,0.000000,0.000000}%
\pgfsetstrokecolor{currentstroke}%
\pgfsetdash{}{0pt}%
\pgfpathmoveto{\pgfqpoint{2.518786in}{1.171247in}}%
\pgfpathcurveto{\pgfqpoint{2.529836in}{1.171247in}}{\pgfqpoint{2.540435in}{1.175637in}}{\pgfqpoint{2.548249in}{1.183451in}}%
\pgfpathcurveto{\pgfqpoint{2.556062in}{1.191264in}}{\pgfqpoint{2.560452in}{1.201863in}}{\pgfqpoint{2.560452in}{1.212913in}}%
\pgfpathcurveto{\pgfqpoint{2.560452in}{1.223964in}}{\pgfqpoint{2.556062in}{1.234563in}}{\pgfqpoint{2.548249in}{1.242376in}}%
\pgfpathcurveto{\pgfqpoint{2.540435in}{1.250190in}}{\pgfqpoint{2.529836in}{1.254580in}}{\pgfqpoint{2.518786in}{1.254580in}}%
\pgfpathcurveto{\pgfqpoint{2.507736in}{1.254580in}}{\pgfqpoint{2.497137in}{1.250190in}}{\pgfqpoint{2.489323in}{1.242376in}}%
\pgfpathcurveto{\pgfqpoint{2.481509in}{1.234563in}}{\pgfqpoint{2.477119in}{1.223964in}}{\pgfqpoint{2.477119in}{1.212913in}}%
\pgfpathcurveto{\pgfqpoint{2.477119in}{1.201863in}}{\pgfqpoint{2.481509in}{1.191264in}}{\pgfqpoint{2.489323in}{1.183451in}}%
\pgfpathcurveto{\pgfqpoint{2.497137in}{1.175637in}}{\pgfqpoint{2.507736in}{1.171247in}}{\pgfqpoint{2.518786in}{1.171247in}}%
\pgfpathclose%
\pgfusepath{stroke,fill}%
\end{pgfscope}%
\begin{pgfscope}%
\pgfpathrectangle{\pgfqpoint{0.800000in}{0.528000in}}{\pgfqpoint{4.960000in}{3.696000in}}%
\pgfusepath{clip}%
\pgfsetbuttcap%
\pgfsetroundjoin%
\definecolor{currentfill}{rgb}{0.000000,0.000000,0.000000}%
\pgfsetfillcolor{currentfill}%
\pgfsetlinewidth{1.003750pt}%
\definecolor{currentstroke}{rgb}{0.000000,0.000000,0.000000}%
\pgfsetstrokecolor{currentstroke}%
\pgfsetdash{}{0pt}%
\pgfpathmoveto{\pgfqpoint{2.518786in}{1.042262in}}%
\pgfpathcurveto{\pgfqpoint{2.529836in}{1.042262in}}{\pgfqpoint{2.540435in}{1.046652in}}{\pgfqpoint{2.548249in}{1.054466in}}%
\pgfpathcurveto{\pgfqpoint{2.556062in}{1.062279in}}{\pgfqpoint{2.560452in}{1.072878in}}{\pgfqpoint{2.560452in}{1.083928in}}%
\pgfpathcurveto{\pgfqpoint{2.560452in}{1.094979in}}{\pgfqpoint{2.556062in}{1.105578in}}{\pgfqpoint{2.548249in}{1.113391in}}%
\pgfpathcurveto{\pgfqpoint{2.540435in}{1.121205in}}{\pgfqpoint{2.529836in}{1.125595in}}{\pgfqpoint{2.518786in}{1.125595in}}%
\pgfpathcurveto{\pgfqpoint{2.507736in}{1.125595in}}{\pgfqpoint{2.497137in}{1.121205in}}{\pgfqpoint{2.489323in}{1.113391in}}%
\pgfpathcurveto{\pgfqpoint{2.481509in}{1.105578in}}{\pgfqpoint{2.477119in}{1.094979in}}{\pgfqpoint{2.477119in}{1.083928in}}%
\pgfpathcurveto{\pgfqpoint{2.477119in}{1.072878in}}{\pgfqpoint{2.481509in}{1.062279in}}{\pgfqpoint{2.489323in}{1.054466in}}%
\pgfpathcurveto{\pgfqpoint{2.497137in}{1.046652in}}{\pgfqpoint{2.507736in}{1.042262in}}{\pgfqpoint{2.518786in}{1.042262in}}%
\pgfpathclose%
\pgfusepath{stroke,fill}%
\end{pgfscope}%
\begin{pgfscope}%
\pgfpathrectangle{\pgfqpoint{0.800000in}{0.528000in}}{\pgfqpoint{4.960000in}{3.696000in}}%
\pgfusepath{clip}%
\pgfsetbuttcap%
\pgfsetroundjoin%
\definecolor{currentfill}{rgb}{0.000000,0.000000,0.000000}%
\pgfsetfillcolor{currentfill}%
\pgfsetlinewidth{1.003750pt}%
\definecolor{currentstroke}{rgb}{0.000000,0.000000,0.000000}%
\pgfsetstrokecolor{currentstroke}%
\pgfsetdash{}{0pt}%
\pgfpathmoveto{\pgfqpoint{2.518786in}{1.063759in}}%
\pgfpathcurveto{\pgfqpoint{2.529836in}{1.063759in}}{\pgfqpoint{2.540435in}{1.068150in}}{\pgfqpoint{2.548249in}{1.075963in}}%
\pgfpathcurveto{\pgfqpoint{2.556062in}{1.083777in}}{\pgfqpoint{2.560452in}{1.094376in}}{\pgfqpoint{2.560452in}{1.105426in}}%
\pgfpathcurveto{\pgfqpoint{2.560452in}{1.116476in}}{\pgfqpoint{2.556062in}{1.127075in}}{\pgfqpoint{2.548249in}{1.134889in}}%
\pgfpathcurveto{\pgfqpoint{2.540435in}{1.142702in}}{\pgfqpoint{2.529836in}{1.147093in}}{\pgfqpoint{2.518786in}{1.147093in}}%
\pgfpathcurveto{\pgfqpoint{2.507736in}{1.147093in}}{\pgfqpoint{2.497137in}{1.142702in}}{\pgfqpoint{2.489323in}{1.134889in}}%
\pgfpathcurveto{\pgfqpoint{2.481509in}{1.127075in}}{\pgfqpoint{2.477119in}{1.116476in}}{\pgfqpoint{2.477119in}{1.105426in}}%
\pgfpathcurveto{\pgfqpoint{2.477119in}{1.094376in}}{\pgfqpoint{2.481509in}{1.083777in}}{\pgfqpoint{2.489323in}{1.075963in}}%
\pgfpathcurveto{\pgfqpoint{2.497137in}{1.068150in}}{\pgfqpoint{2.507736in}{1.063759in}}{\pgfqpoint{2.518786in}{1.063759in}}%
\pgfpathclose%
\pgfusepath{stroke,fill}%
\end{pgfscope}%
\begin{pgfscope}%
\pgfpathrectangle{\pgfqpoint{0.800000in}{0.528000in}}{\pgfqpoint{4.960000in}{3.696000in}}%
\pgfusepath{clip}%
\pgfsetbuttcap%
\pgfsetroundjoin%
\definecolor{currentfill}{rgb}{0.000000,0.000000,0.000000}%
\pgfsetfillcolor{currentfill}%
\pgfsetlinewidth{1.003750pt}%
\definecolor{currentstroke}{rgb}{0.000000,0.000000,0.000000}%
\pgfsetstrokecolor{currentstroke}%
\pgfsetdash{}{0pt}%
\pgfpathmoveto{\pgfqpoint{2.518786in}{1.214242in}}%
\pgfpathcurveto{\pgfqpoint{2.529836in}{1.214242in}}{\pgfqpoint{2.540435in}{1.218632in}}{\pgfqpoint{2.548249in}{1.226446in}}%
\pgfpathcurveto{\pgfqpoint{2.556062in}{1.234259in}}{\pgfqpoint{2.560452in}{1.244858in}}{\pgfqpoint{2.560452in}{1.255908in}}%
\pgfpathcurveto{\pgfqpoint{2.560452in}{1.266959in}}{\pgfqpoint{2.556062in}{1.277558in}}{\pgfqpoint{2.548249in}{1.285371in}}%
\pgfpathcurveto{\pgfqpoint{2.540435in}{1.293185in}}{\pgfqpoint{2.529836in}{1.297575in}}{\pgfqpoint{2.518786in}{1.297575in}}%
\pgfpathcurveto{\pgfqpoint{2.507736in}{1.297575in}}{\pgfqpoint{2.497137in}{1.293185in}}{\pgfqpoint{2.489323in}{1.285371in}}%
\pgfpathcurveto{\pgfqpoint{2.481509in}{1.277558in}}{\pgfqpoint{2.477119in}{1.266959in}}{\pgfqpoint{2.477119in}{1.255908in}}%
\pgfpathcurveto{\pgfqpoint{2.477119in}{1.244858in}}{\pgfqpoint{2.481509in}{1.234259in}}{\pgfqpoint{2.489323in}{1.226446in}}%
\pgfpathcurveto{\pgfqpoint{2.497137in}{1.218632in}}{\pgfqpoint{2.507736in}{1.214242in}}{\pgfqpoint{2.518786in}{1.214242in}}%
\pgfpathclose%
\pgfusepath{stroke,fill}%
\end{pgfscope}%
\begin{pgfscope}%
\pgfpathrectangle{\pgfqpoint{0.800000in}{0.528000in}}{\pgfqpoint{4.960000in}{3.696000in}}%
\pgfusepath{clip}%
\pgfsetbuttcap%
\pgfsetroundjoin%
\definecolor{currentfill}{rgb}{0.000000,0.000000,0.000000}%
\pgfsetfillcolor{currentfill}%
\pgfsetlinewidth{1.003750pt}%
\definecolor{currentstroke}{rgb}{0.000000,0.000000,0.000000}%
\pgfsetstrokecolor{currentstroke}%
\pgfsetdash{}{0pt}%
\pgfpathmoveto{\pgfqpoint{2.518786in}{1.128252in}}%
\pgfpathcurveto{\pgfqpoint{2.529836in}{1.128252in}}{\pgfqpoint{2.540435in}{1.132642in}}{\pgfqpoint{2.548249in}{1.140456in}}%
\pgfpathcurveto{\pgfqpoint{2.556062in}{1.148269in}}{\pgfqpoint{2.560452in}{1.158868in}}{\pgfqpoint{2.560452in}{1.169918in}}%
\pgfpathcurveto{\pgfqpoint{2.560452in}{1.180969in}}{\pgfqpoint{2.556062in}{1.191568in}}{\pgfqpoint{2.548249in}{1.199381in}}%
\pgfpathcurveto{\pgfqpoint{2.540435in}{1.207195in}}{\pgfqpoint{2.529836in}{1.211585in}}{\pgfqpoint{2.518786in}{1.211585in}}%
\pgfpathcurveto{\pgfqpoint{2.507736in}{1.211585in}}{\pgfqpoint{2.497137in}{1.207195in}}{\pgfqpoint{2.489323in}{1.199381in}}%
\pgfpathcurveto{\pgfqpoint{2.481509in}{1.191568in}}{\pgfqpoint{2.477119in}{1.180969in}}{\pgfqpoint{2.477119in}{1.169918in}}%
\pgfpathcurveto{\pgfqpoint{2.477119in}{1.158868in}}{\pgfqpoint{2.481509in}{1.148269in}}{\pgfqpoint{2.489323in}{1.140456in}}%
\pgfpathcurveto{\pgfqpoint{2.497137in}{1.132642in}}{\pgfqpoint{2.507736in}{1.128252in}}{\pgfqpoint{2.518786in}{1.128252in}}%
\pgfpathclose%
\pgfusepath{stroke,fill}%
\end{pgfscope}%
\begin{pgfscope}%
\pgfpathrectangle{\pgfqpoint{0.800000in}{0.528000in}}{\pgfqpoint{4.960000in}{3.696000in}}%
\pgfusepath{clip}%
\pgfsetbuttcap%
\pgfsetroundjoin%
\definecolor{currentfill}{rgb}{0.000000,0.000000,0.000000}%
\pgfsetfillcolor{currentfill}%
\pgfsetlinewidth{1.003750pt}%
\definecolor{currentstroke}{rgb}{0.000000,0.000000,0.000000}%
\pgfsetstrokecolor{currentstroke}%
\pgfsetdash{}{0pt}%
\pgfpathmoveto{\pgfqpoint{4.011666in}{1.923659in}}%
\pgfpathcurveto{\pgfqpoint{4.022716in}{1.923659in}}{\pgfqpoint{4.033315in}{1.928049in}}{\pgfqpoint{4.041128in}{1.935863in}}%
\pgfpathcurveto{\pgfqpoint{4.048942in}{1.943677in}}{\pgfqpoint{4.053332in}{1.954276in}}{\pgfqpoint{4.053332in}{1.965326in}}%
\pgfpathcurveto{\pgfqpoint{4.053332in}{1.976376in}}{\pgfqpoint{4.048942in}{1.986975in}}{\pgfqpoint{4.041128in}{1.994788in}}%
\pgfpathcurveto{\pgfqpoint{4.033315in}{2.002602in}}{\pgfqpoint{4.022716in}{2.006992in}}{\pgfqpoint{4.011666in}{2.006992in}}%
\pgfpathcurveto{\pgfqpoint{4.000616in}{2.006992in}}{\pgfqpoint{3.990016in}{2.002602in}}{\pgfqpoint{3.982203in}{1.994788in}}%
\pgfpathcurveto{\pgfqpoint{3.974389in}{1.986975in}}{\pgfqpoint{3.969999in}{1.976376in}}{\pgfqpoint{3.969999in}{1.965326in}}%
\pgfpathcurveto{\pgfqpoint{3.969999in}{1.954276in}}{\pgfqpoint{3.974389in}{1.943677in}}{\pgfqpoint{3.982203in}{1.935863in}}%
\pgfpathcurveto{\pgfqpoint{3.990016in}{1.928049in}}{\pgfqpoint{4.000616in}{1.923659in}}{\pgfqpoint{4.011666in}{1.923659in}}%
\pgfpathclose%
\pgfusepath{stroke,fill}%
\end{pgfscope}%
\begin{pgfscope}%
\pgfpathrectangle{\pgfqpoint{0.800000in}{0.528000in}}{\pgfqpoint{4.960000in}{3.696000in}}%
\pgfusepath{clip}%
\pgfsetbuttcap%
\pgfsetroundjoin%
\definecolor{currentfill}{rgb}{0.000000,0.000000,0.000000}%
\pgfsetfillcolor{currentfill}%
\pgfsetlinewidth{1.003750pt}%
\definecolor{currentstroke}{rgb}{0.000000,0.000000,0.000000}%
\pgfsetstrokecolor{currentstroke}%
\pgfsetdash{}{0pt}%
\pgfpathmoveto{\pgfqpoint{4.011666in}{1.794674in}}%
\pgfpathcurveto{\pgfqpoint{4.022716in}{1.794674in}}{\pgfqpoint{4.033315in}{1.799064in}}{\pgfqpoint{4.041128in}{1.806878in}}%
\pgfpathcurveto{\pgfqpoint{4.048942in}{1.814692in}}{\pgfqpoint{4.053332in}{1.825291in}}{\pgfqpoint{4.053332in}{1.836341in}}%
\pgfpathcurveto{\pgfqpoint{4.053332in}{1.847391in}}{\pgfqpoint{4.048942in}{1.857990in}}{\pgfqpoint{4.041128in}{1.865804in}}%
\pgfpathcurveto{\pgfqpoint{4.033315in}{1.873617in}}{\pgfqpoint{4.022716in}{1.878007in}}{\pgfqpoint{4.011666in}{1.878007in}}%
\pgfpathcurveto{\pgfqpoint{4.000616in}{1.878007in}}{\pgfqpoint{3.990016in}{1.873617in}}{\pgfqpoint{3.982203in}{1.865804in}}%
\pgfpathcurveto{\pgfqpoint{3.974389in}{1.857990in}}{\pgfqpoint{3.969999in}{1.847391in}}{\pgfqpoint{3.969999in}{1.836341in}}%
\pgfpathcurveto{\pgfqpoint{3.969999in}{1.825291in}}{\pgfqpoint{3.974389in}{1.814692in}}{\pgfqpoint{3.982203in}{1.806878in}}%
\pgfpathcurveto{\pgfqpoint{3.990016in}{1.799064in}}{\pgfqpoint{4.000616in}{1.794674in}}{\pgfqpoint{4.011666in}{1.794674in}}%
\pgfpathclose%
\pgfusepath{stroke,fill}%
\end{pgfscope}%
\begin{pgfscope}%
\pgfpathrectangle{\pgfqpoint{0.800000in}{0.528000in}}{\pgfqpoint{4.960000in}{3.696000in}}%
\pgfusepath{clip}%
\pgfsetbuttcap%
\pgfsetroundjoin%
\definecolor{currentfill}{rgb}{0.000000,0.000000,0.000000}%
\pgfsetfillcolor{currentfill}%
\pgfsetlinewidth{1.003750pt}%
\definecolor{currentstroke}{rgb}{0.000000,0.000000,0.000000}%
\pgfsetstrokecolor{currentstroke}%
\pgfsetdash{}{0pt}%
\pgfpathmoveto{\pgfqpoint{4.011666in}{1.859167in}}%
\pgfpathcurveto{\pgfqpoint{4.022716in}{1.859167in}}{\pgfqpoint{4.033315in}{1.863557in}}{\pgfqpoint{4.041128in}{1.871370in}}%
\pgfpathcurveto{\pgfqpoint{4.048942in}{1.879184in}}{\pgfqpoint{4.053332in}{1.889783in}}{\pgfqpoint{4.053332in}{1.900833in}}%
\pgfpathcurveto{\pgfqpoint{4.053332in}{1.911883in}}{\pgfqpoint{4.048942in}{1.922482in}}{\pgfqpoint{4.041128in}{1.930296in}}%
\pgfpathcurveto{\pgfqpoint{4.033315in}{1.938110in}}{\pgfqpoint{4.022716in}{1.942500in}}{\pgfqpoint{4.011666in}{1.942500in}}%
\pgfpathcurveto{\pgfqpoint{4.000616in}{1.942500in}}{\pgfqpoint{3.990016in}{1.938110in}}{\pgfqpoint{3.982203in}{1.930296in}}%
\pgfpathcurveto{\pgfqpoint{3.974389in}{1.922482in}}{\pgfqpoint{3.969999in}{1.911883in}}{\pgfqpoint{3.969999in}{1.900833in}}%
\pgfpathcurveto{\pgfqpoint{3.969999in}{1.889783in}}{\pgfqpoint{3.974389in}{1.879184in}}{\pgfqpoint{3.982203in}{1.871370in}}%
\pgfpathcurveto{\pgfqpoint{3.990016in}{1.863557in}}{\pgfqpoint{4.000616in}{1.859167in}}{\pgfqpoint{4.011666in}{1.859167in}}%
\pgfpathclose%
\pgfusepath{stroke,fill}%
\end{pgfscope}%
\begin{pgfscope}%
\pgfpathrectangle{\pgfqpoint{0.800000in}{0.528000in}}{\pgfqpoint{4.960000in}{3.696000in}}%
\pgfusepath{clip}%
\pgfsetbuttcap%
\pgfsetroundjoin%
\definecolor{currentfill}{rgb}{0.000000,0.000000,0.000000}%
\pgfsetfillcolor{currentfill}%
\pgfsetlinewidth{1.003750pt}%
\definecolor{currentstroke}{rgb}{0.000000,0.000000,0.000000}%
\pgfsetstrokecolor{currentstroke}%
\pgfsetdash{}{0pt}%
\pgfpathmoveto{\pgfqpoint{4.011666in}{1.816172in}}%
\pgfpathcurveto{\pgfqpoint{4.022716in}{1.816172in}}{\pgfqpoint{4.033315in}{1.820562in}}{\pgfqpoint{4.041128in}{1.828375in}}%
\pgfpathcurveto{\pgfqpoint{4.048942in}{1.836189in}}{\pgfqpoint{4.053332in}{1.846788in}}{\pgfqpoint{4.053332in}{1.857838in}}%
\pgfpathcurveto{\pgfqpoint{4.053332in}{1.868888in}}{\pgfqpoint{4.048942in}{1.879487in}}{\pgfqpoint{4.041128in}{1.887301in}}%
\pgfpathcurveto{\pgfqpoint{4.033315in}{1.895115in}}{\pgfqpoint{4.022716in}{1.899505in}}{\pgfqpoint{4.011666in}{1.899505in}}%
\pgfpathcurveto{\pgfqpoint{4.000616in}{1.899505in}}{\pgfqpoint{3.990016in}{1.895115in}}{\pgfqpoint{3.982203in}{1.887301in}}%
\pgfpathcurveto{\pgfqpoint{3.974389in}{1.879487in}}{\pgfqpoint{3.969999in}{1.868888in}}{\pgfqpoint{3.969999in}{1.857838in}}%
\pgfpathcurveto{\pgfqpoint{3.969999in}{1.846788in}}{\pgfqpoint{3.974389in}{1.836189in}}{\pgfqpoint{3.982203in}{1.828375in}}%
\pgfpathcurveto{\pgfqpoint{3.990016in}{1.820562in}}{\pgfqpoint{4.000616in}{1.816172in}}{\pgfqpoint{4.011666in}{1.816172in}}%
\pgfpathclose%
\pgfusepath{stroke,fill}%
\end{pgfscope}%
\begin{pgfscope}%
\pgfpathrectangle{\pgfqpoint{0.800000in}{0.528000in}}{\pgfqpoint{4.960000in}{3.696000in}}%
\pgfusepath{clip}%
\pgfsetbuttcap%
\pgfsetroundjoin%
\definecolor{currentfill}{rgb}{0.000000,0.000000,0.000000}%
\pgfsetfillcolor{currentfill}%
\pgfsetlinewidth{1.003750pt}%
\definecolor{currentstroke}{rgb}{0.000000,0.000000,0.000000}%
\pgfsetstrokecolor{currentstroke}%
\pgfsetdash{}{0pt}%
\pgfpathmoveto{\pgfqpoint{4.011666in}{1.945157in}}%
\pgfpathcurveto{\pgfqpoint{4.022716in}{1.945157in}}{\pgfqpoint{4.033315in}{1.949547in}}{\pgfqpoint{4.041128in}{1.957360in}}%
\pgfpathcurveto{\pgfqpoint{4.048942in}{1.965174in}}{\pgfqpoint{4.053332in}{1.975773in}}{\pgfqpoint{4.053332in}{1.986823in}}%
\pgfpathcurveto{\pgfqpoint{4.053332in}{1.997873in}}{\pgfqpoint{4.048942in}{2.008472in}}{\pgfqpoint{4.041128in}{2.016286in}}%
\pgfpathcurveto{\pgfqpoint{4.033315in}{2.024100in}}{\pgfqpoint{4.022716in}{2.028490in}}{\pgfqpoint{4.011666in}{2.028490in}}%
\pgfpathcurveto{\pgfqpoint{4.000616in}{2.028490in}}{\pgfqpoint{3.990016in}{2.024100in}}{\pgfqpoint{3.982203in}{2.016286in}}%
\pgfpathcurveto{\pgfqpoint{3.974389in}{2.008472in}}{\pgfqpoint{3.969999in}{1.997873in}}{\pgfqpoint{3.969999in}{1.986823in}}%
\pgfpathcurveto{\pgfqpoint{3.969999in}{1.975773in}}{\pgfqpoint{3.974389in}{1.965174in}}{\pgfqpoint{3.982203in}{1.957360in}}%
\pgfpathcurveto{\pgfqpoint{3.990016in}{1.949547in}}{\pgfqpoint{4.000616in}{1.945157in}}{\pgfqpoint{4.011666in}{1.945157in}}%
\pgfpathclose%
\pgfusepath{stroke,fill}%
\end{pgfscope}%
\begin{pgfscope}%
\pgfpathrectangle{\pgfqpoint{0.800000in}{0.528000in}}{\pgfqpoint{4.960000in}{3.696000in}}%
\pgfusepath{clip}%
\pgfsetbuttcap%
\pgfsetroundjoin%
\definecolor{currentfill}{rgb}{0.000000,0.000000,0.000000}%
\pgfsetfillcolor{currentfill}%
\pgfsetlinewidth{1.003750pt}%
\definecolor{currentstroke}{rgb}{0.000000,0.000000,0.000000}%
\pgfsetstrokecolor{currentstroke}%
\pgfsetdash{}{0pt}%
\pgfpathmoveto{\pgfqpoint{4.011666in}{1.880664in}}%
\pgfpathcurveto{\pgfqpoint{4.022716in}{1.880664in}}{\pgfqpoint{4.033315in}{1.885054in}}{\pgfqpoint{4.041128in}{1.892868in}}%
\pgfpathcurveto{\pgfqpoint{4.048942in}{1.900682in}}{\pgfqpoint{4.053332in}{1.911281in}}{\pgfqpoint{4.053332in}{1.922331in}}%
\pgfpathcurveto{\pgfqpoint{4.053332in}{1.933381in}}{\pgfqpoint{4.048942in}{1.943980in}}{\pgfqpoint{4.041128in}{1.951793in}}%
\pgfpathcurveto{\pgfqpoint{4.033315in}{1.959607in}}{\pgfqpoint{4.022716in}{1.963997in}}{\pgfqpoint{4.011666in}{1.963997in}}%
\pgfpathcurveto{\pgfqpoint{4.000616in}{1.963997in}}{\pgfqpoint{3.990016in}{1.959607in}}{\pgfqpoint{3.982203in}{1.951793in}}%
\pgfpathcurveto{\pgfqpoint{3.974389in}{1.943980in}}{\pgfqpoint{3.969999in}{1.933381in}}{\pgfqpoint{3.969999in}{1.922331in}}%
\pgfpathcurveto{\pgfqpoint{3.969999in}{1.911281in}}{\pgfqpoint{3.974389in}{1.900682in}}{\pgfqpoint{3.982203in}{1.892868in}}%
\pgfpathcurveto{\pgfqpoint{3.990016in}{1.885054in}}{\pgfqpoint{4.000616in}{1.880664in}}{\pgfqpoint{4.011666in}{1.880664in}}%
\pgfpathclose%
\pgfusepath{stroke,fill}%
\end{pgfscope}%
\begin{pgfscope}%
\pgfpathrectangle{\pgfqpoint{0.800000in}{0.528000in}}{\pgfqpoint{4.960000in}{3.696000in}}%
\pgfusepath{clip}%
\pgfsetbuttcap%
\pgfsetroundjoin%
\definecolor{currentfill}{rgb}{0.000000,0.000000,0.000000}%
\pgfsetfillcolor{currentfill}%
\pgfsetlinewidth{1.003750pt}%
\definecolor{currentstroke}{rgb}{0.000000,0.000000,0.000000}%
\pgfsetstrokecolor{currentstroke}%
\pgfsetdash{}{0pt}%
\pgfpathmoveto{\pgfqpoint{4.011666in}{1.859167in}}%
\pgfpathcurveto{\pgfqpoint{4.022716in}{1.859167in}}{\pgfqpoint{4.033315in}{1.863557in}}{\pgfqpoint{4.041128in}{1.871370in}}%
\pgfpathcurveto{\pgfqpoint{4.048942in}{1.879184in}}{\pgfqpoint{4.053332in}{1.889783in}}{\pgfqpoint{4.053332in}{1.900833in}}%
\pgfpathcurveto{\pgfqpoint{4.053332in}{1.911883in}}{\pgfqpoint{4.048942in}{1.922482in}}{\pgfqpoint{4.041128in}{1.930296in}}%
\pgfpathcurveto{\pgfqpoint{4.033315in}{1.938110in}}{\pgfqpoint{4.022716in}{1.942500in}}{\pgfqpoint{4.011666in}{1.942500in}}%
\pgfpathcurveto{\pgfqpoint{4.000616in}{1.942500in}}{\pgfqpoint{3.990016in}{1.938110in}}{\pgfqpoint{3.982203in}{1.930296in}}%
\pgfpathcurveto{\pgfqpoint{3.974389in}{1.922482in}}{\pgfqpoint{3.969999in}{1.911883in}}{\pgfqpoint{3.969999in}{1.900833in}}%
\pgfpathcurveto{\pgfqpoint{3.969999in}{1.889783in}}{\pgfqpoint{3.974389in}{1.879184in}}{\pgfqpoint{3.982203in}{1.871370in}}%
\pgfpathcurveto{\pgfqpoint{3.990016in}{1.863557in}}{\pgfqpoint{4.000616in}{1.859167in}}{\pgfqpoint{4.011666in}{1.859167in}}%
\pgfpathclose%
\pgfusepath{stroke,fill}%
\end{pgfscope}%
\begin{pgfscope}%
\pgfpathrectangle{\pgfqpoint{0.800000in}{0.528000in}}{\pgfqpoint{4.960000in}{3.696000in}}%
\pgfusepath{clip}%
\pgfsetbuttcap%
\pgfsetroundjoin%
\definecolor{currentfill}{rgb}{0.000000,0.000000,0.000000}%
\pgfsetfillcolor{currentfill}%
\pgfsetlinewidth{1.003750pt}%
\definecolor{currentstroke}{rgb}{0.000000,0.000000,0.000000}%
\pgfsetstrokecolor{currentstroke}%
\pgfsetdash{}{0pt}%
\pgfpathmoveto{\pgfqpoint{4.011666in}{1.794674in}}%
\pgfpathcurveto{\pgfqpoint{4.022716in}{1.794674in}}{\pgfqpoint{4.033315in}{1.799064in}}{\pgfqpoint{4.041128in}{1.806878in}}%
\pgfpathcurveto{\pgfqpoint{4.048942in}{1.814692in}}{\pgfqpoint{4.053332in}{1.825291in}}{\pgfqpoint{4.053332in}{1.836341in}}%
\pgfpathcurveto{\pgfqpoint{4.053332in}{1.847391in}}{\pgfqpoint{4.048942in}{1.857990in}}{\pgfqpoint{4.041128in}{1.865804in}}%
\pgfpathcurveto{\pgfqpoint{4.033315in}{1.873617in}}{\pgfqpoint{4.022716in}{1.878007in}}{\pgfqpoint{4.011666in}{1.878007in}}%
\pgfpathcurveto{\pgfqpoint{4.000616in}{1.878007in}}{\pgfqpoint{3.990016in}{1.873617in}}{\pgfqpoint{3.982203in}{1.865804in}}%
\pgfpathcurveto{\pgfqpoint{3.974389in}{1.857990in}}{\pgfqpoint{3.969999in}{1.847391in}}{\pgfqpoint{3.969999in}{1.836341in}}%
\pgfpathcurveto{\pgfqpoint{3.969999in}{1.825291in}}{\pgfqpoint{3.974389in}{1.814692in}}{\pgfqpoint{3.982203in}{1.806878in}}%
\pgfpathcurveto{\pgfqpoint{3.990016in}{1.799064in}}{\pgfqpoint{4.000616in}{1.794674in}}{\pgfqpoint{4.011666in}{1.794674in}}%
\pgfpathclose%
\pgfusepath{stroke,fill}%
\end{pgfscope}%
\begin{pgfscope}%
\pgfpathrectangle{\pgfqpoint{0.800000in}{0.528000in}}{\pgfqpoint{4.960000in}{3.696000in}}%
\pgfusepath{clip}%
\pgfsetbuttcap%
\pgfsetroundjoin%
\definecolor{currentfill}{rgb}{0.000000,0.000000,0.000000}%
\pgfsetfillcolor{currentfill}%
\pgfsetlinewidth{1.003750pt}%
\definecolor{currentstroke}{rgb}{0.000000,0.000000,0.000000}%
\pgfsetstrokecolor{currentstroke}%
\pgfsetdash{}{0pt}%
\pgfpathmoveto{\pgfqpoint{4.011666in}{1.923659in}}%
\pgfpathcurveto{\pgfqpoint{4.022716in}{1.923659in}}{\pgfqpoint{4.033315in}{1.928049in}}{\pgfqpoint{4.041128in}{1.935863in}}%
\pgfpathcurveto{\pgfqpoint{4.048942in}{1.943677in}}{\pgfqpoint{4.053332in}{1.954276in}}{\pgfqpoint{4.053332in}{1.965326in}}%
\pgfpathcurveto{\pgfqpoint{4.053332in}{1.976376in}}{\pgfqpoint{4.048942in}{1.986975in}}{\pgfqpoint{4.041128in}{1.994788in}}%
\pgfpathcurveto{\pgfqpoint{4.033315in}{2.002602in}}{\pgfqpoint{4.022716in}{2.006992in}}{\pgfqpoint{4.011666in}{2.006992in}}%
\pgfpathcurveto{\pgfqpoint{4.000616in}{2.006992in}}{\pgfqpoint{3.990016in}{2.002602in}}{\pgfqpoint{3.982203in}{1.994788in}}%
\pgfpathcurveto{\pgfqpoint{3.974389in}{1.986975in}}{\pgfqpoint{3.969999in}{1.976376in}}{\pgfqpoint{3.969999in}{1.965326in}}%
\pgfpathcurveto{\pgfqpoint{3.969999in}{1.954276in}}{\pgfqpoint{3.974389in}{1.943677in}}{\pgfqpoint{3.982203in}{1.935863in}}%
\pgfpathcurveto{\pgfqpoint{3.990016in}{1.928049in}}{\pgfqpoint{4.000616in}{1.923659in}}{\pgfqpoint{4.011666in}{1.923659in}}%
\pgfpathclose%
\pgfusepath{stroke,fill}%
\end{pgfscope}%
\begin{pgfscope}%
\pgfpathrectangle{\pgfqpoint{0.800000in}{0.528000in}}{\pgfqpoint{4.960000in}{3.696000in}}%
\pgfusepath{clip}%
\pgfsetbuttcap%
\pgfsetroundjoin%
\definecolor{currentfill}{rgb}{0.000000,0.000000,0.000000}%
\pgfsetfillcolor{currentfill}%
\pgfsetlinewidth{1.003750pt}%
\definecolor{currentstroke}{rgb}{0.000000,0.000000,0.000000}%
\pgfsetstrokecolor{currentstroke}%
\pgfsetdash{}{0pt}%
\pgfpathmoveto{\pgfqpoint{4.011666in}{1.837669in}}%
\pgfpathcurveto{\pgfqpoint{4.022716in}{1.837669in}}{\pgfqpoint{4.033315in}{1.842059in}}{\pgfqpoint{4.041128in}{1.849873in}}%
\pgfpathcurveto{\pgfqpoint{4.048942in}{1.857687in}}{\pgfqpoint{4.053332in}{1.868286in}}{\pgfqpoint{4.053332in}{1.879336in}}%
\pgfpathcurveto{\pgfqpoint{4.053332in}{1.890386in}}{\pgfqpoint{4.048942in}{1.900985in}}{\pgfqpoint{4.041128in}{1.908798in}}%
\pgfpathcurveto{\pgfqpoint{4.033315in}{1.916612in}}{\pgfqpoint{4.022716in}{1.921002in}}{\pgfqpoint{4.011666in}{1.921002in}}%
\pgfpathcurveto{\pgfqpoint{4.000616in}{1.921002in}}{\pgfqpoint{3.990016in}{1.916612in}}{\pgfqpoint{3.982203in}{1.908798in}}%
\pgfpathcurveto{\pgfqpoint{3.974389in}{1.900985in}}{\pgfqpoint{3.969999in}{1.890386in}}{\pgfqpoint{3.969999in}{1.879336in}}%
\pgfpathcurveto{\pgfqpoint{3.969999in}{1.868286in}}{\pgfqpoint{3.974389in}{1.857687in}}{\pgfqpoint{3.982203in}{1.849873in}}%
\pgfpathcurveto{\pgfqpoint{3.990016in}{1.842059in}}{\pgfqpoint{4.000616in}{1.837669in}}{\pgfqpoint{4.011666in}{1.837669in}}%
\pgfpathclose%
\pgfusepath{stroke,fill}%
\end{pgfscope}%
\begin{pgfscope}%
\pgfpathrectangle{\pgfqpoint{0.800000in}{0.528000in}}{\pgfqpoint{4.960000in}{3.696000in}}%
\pgfusepath{clip}%
\pgfsetbuttcap%
\pgfsetroundjoin%
\definecolor{currentfill}{rgb}{0.000000,0.000000,0.000000}%
\pgfsetfillcolor{currentfill}%
\pgfsetlinewidth{1.003750pt}%
\definecolor{currentstroke}{rgb}{0.000000,0.000000,0.000000}%
\pgfsetstrokecolor{currentstroke}%
\pgfsetdash{}{0pt}%
\pgfpathmoveto{\pgfqpoint{4.011666in}{1.880664in}}%
\pgfpathcurveto{\pgfqpoint{4.022716in}{1.880664in}}{\pgfqpoint{4.033315in}{1.885054in}}{\pgfqpoint{4.041128in}{1.892868in}}%
\pgfpathcurveto{\pgfqpoint{4.048942in}{1.900682in}}{\pgfqpoint{4.053332in}{1.911281in}}{\pgfqpoint{4.053332in}{1.922331in}}%
\pgfpathcurveto{\pgfqpoint{4.053332in}{1.933381in}}{\pgfqpoint{4.048942in}{1.943980in}}{\pgfqpoint{4.041128in}{1.951793in}}%
\pgfpathcurveto{\pgfqpoint{4.033315in}{1.959607in}}{\pgfqpoint{4.022716in}{1.963997in}}{\pgfqpoint{4.011666in}{1.963997in}}%
\pgfpathcurveto{\pgfqpoint{4.000616in}{1.963997in}}{\pgfqpoint{3.990016in}{1.959607in}}{\pgfqpoint{3.982203in}{1.951793in}}%
\pgfpathcurveto{\pgfqpoint{3.974389in}{1.943980in}}{\pgfqpoint{3.969999in}{1.933381in}}{\pgfqpoint{3.969999in}{1.922331in}}%
\pgfpathcurveto{\pgfqpoint{3.969999in}{1.911281in}}{\pgfqpoint{3.974389in}{1.900682in}}{\pgfqpoint{3.982203in}{1.892868in}}%
\pgfpathcurveto{\pgfqpoint{3.990016in}{1.885054in}}{\pgfqpoint{4.000616in}{1.880664in}}{\pgfqpoint{4.011666in}{1.880664in}}%
\pgfpathclose%
\pgfusepath{stroke,fill}%
\end{pgfscope}%
\begin{pgfscope}%
\pgfpathrectangle{\pgfqpoint{0.800000in}{0.528000in}}{\pgfqpoint{4.960000in}{3.696000in}}%
\pgfusepath{clip}%
\pgfsetbuttcap%
\pgfsetroundjoin%
\definecolor{currentfill}{rgb}{0.000000,0.000000,0.000000}%
\pgfsetfillcolor{currentfill}%
\pgfsetlinewidth{1.003750pt}%
\definecolor{currentstroke}{rgb}{0.000000,0.000000,0.000000}%
\pgfsetstrokecolor{currentstroke}%
\pgfsetdash{}{0pt}%
\pgfpathmoveto{\pgfqpoint{4.011666in}{1.859167in}}%
\pgfpathcurveto{\pgfqpoint{4.022716in}{1.859167in}}{\pgfqpoint{4.033315in}{1.863557in}}{\pgfqpoint{4.041128in}{1.871370in}}%
\pgfpathcurveto{\pgfqpoint{4.048942in}{1.879184in}}{\pgfqpoint{4.053332in}{1.889783in}}{\pgfqpoint{4.053332in}{1.900833in}}%
\pgfpathcurveto{\pgfqpoint{4.053332in}{1.911883in}}{\pgfqpoint{4.048942in}{1.922482in}}{\pgfqpoint{4.041128in}{1.930296in}}%
\pgfpathcurveto{\pgfqpoint{4.033315in}{1.938110in}}{\pgfqpoint{4.022716in}{1.942500in}}{\pgfqpoint{4.011666in}{1.942500in}}%
\pgfpathcurveto{\pgfqpoint{4.000616in}{1.942500in}}{\pgfqpoint{3.990016in}{1.938110in}}{\pgfqpoint{3.982203in}{1.930296in}}%
\pgfpathcurveto{\pgfqpoint{3.974389in}{1.922482in}}{\pgfqpoint{3.969999in}{1.911883in}}{\pgfqpoint{3.969999in}{1.900833in}}%
\pgfpathcurveto{\pgfqpoint{3.969999in}{1.889783in}}{\pgfqpoint{3.974389in}{1.879184in}}{\pgfqpoint{3.982203in}{1.871370in}}%
\pgfpathcurveto{\pgfqpoint{3.990016in}{1.863557in}}{\pgfqpoint{4.000616in}{1.859167in}}{\pgfqpoint{4.011666in}{1.859167in}}%
\pgfpathclose%
\pgfusepath{stroke,fill}%
\end{pgfscope}%
\begin{pgfscope}%
\pgfpathrectangle{\pgfqpoint{0.800000in}{0.528000in}}{\pgfqpoint{4.960000in}{3.696000in}}%
\pgfusepath{clip}%
\pgfsetbuttcap%
\pgfsetroundjoin%
\definecolor{currentfill}{rgb}{0.000000,0.000000,0.000000}%
\pgfsetfillcolor{currentfill}%
\pgfsetlinewidth{1.003750pt}%
\definecolor{currentstroke}{rgb}{0.000000,0.000000,0.000000}%
\pgfsetstrokecolor{currentstroke}%
\pgfsetdash{}{0pt}%
\pgfpathmoveto{\pgfqpoint{4.011666in}{1.794674in}}%
\pgfpathcurveto{\pgfqpoint{4.022716in}{1.794674in}}{\pgfqpoint{4.033315in}{1.799064in}}{\pgfqpoint{4.041128in}{1.806878in}}%
\pgfpathcurveto{\pgfqpoint{4.048942in}{1.814692in}}{\pgfqpoint{4.053332in}{1.825291in}}{\pgfqpoint{4.053332in}{1.836341in}}%
\pgfpathcurveto{\pgfqpoint{4.053332in}{1.847391in}}{\pgfqpoint{4.048942in}{1.857990in}}{\pgfqpoint{4.041128in}{1.865804in}}%
\pgfpathcurveto{\pgfqpoint{4.033315in}{1.873617in}}{\pgfqpoint{4.022716in}{1.878007in}}{\pgfqpoint{4.011666in}{1.878007in}}%
\pgfpathcurveto{\pgfqpoint{4.000616in}{1.878007in}}{\pgfqpoint{3.990016in}{1.873617in}}{\pgfqpoint{3.982203in}{1.865804in}}%
\pgfpathcurveto{\pgfqpoint{3.974389in}{1.857990in}}{\pgfqpoint{3.969999in}{1.847391in}}{\pgfqpoint{3.969999in}{1.836341in}}%
\pgfpathcurveto{\pgfqpoint{3.969999in}{1.825291in}}{\pgfqpoint{3.974389in}{1.814692in}}{\pgfqpoint{3.982203in}{1.806878in}}%
\pgfpathcurveto{\pgfqpoint{3.990016in}{1.799064in}}{\pgfqpoint{4.000616in}{1.794674in}}{\pgfqpoint{4.011666in}{1.794674in}}%
\pgfpathclose%
\pgfusepath{stroke,fill}%
\end{pgfscope}%
\begin{pgfscope}%
\pgfpathrectangle{\pgfqpoint{0.800000in}{0.528000in}}{\pgfqpoint{4.960000in}{3.696000in}}%
\pgfusepath{clip}%
\pgfsetbuttcap%
\pgfsetroundjoin%
\definecolor{currentfill}{rgb}{0.000000,0.000000,0.000000}%
\pgfsetfillcolor{currentfill}%
\pgfsetlinewidth{1.003750pt}%
\definecolor{currentstroke}{rgb}{0.000000,0.000000,0.000000}%
\pgfsetstrokecolor{currentstroke}%
\pgfsetdash{}{0pt}%
\pgfpathmoveto{\pgfqpoint{4.011666in}{1.794674in}}%
\pgfpathcurveto{\pgfqpoint{4.022716in}{1.794674in}}{\pgfqpoint{4.033315in}{1.799064in}}{\pgfqpoint{4.041128in}{1.806878in}}%
\pgfpathcurveto{\pgfqpoint{4.048942in}{1.814692in}}{\pgfqpoint{4.053332in}{1.825291in}}{\pgfqpoint{4.053332in}{1.836341in}}%
\pgfpathcurveto{\pgfqpoint{4.053332in}{1.847391in}}{\pgfqpoint{4.048942in}{1.857990in}}{\pgfqpoint{4.041128in}{1.865804in}}%
\pgfpathcurveto{\pgfqpoint{4.033315in}{1.873617in}}{\pgfqpoint{4.022716in}{1.878007in}}{\pgfqpoint{4.011666in}{1.878007in}}%
\pgfpathcurveto{\pgfqpoint{4.000616in}{1.878007in}}{\pgfqpoint{3.990016in}{1.873617in}}{\pgfqpoint{3.982203in}{1.865804in}}%
\pgfpathcurveto{\pgfqpoint{3.974389in}{1.857990in}}{\pgfqpoint{3.969999in}{1.847391in}}{\pgfqpoint{3.969999in}{1.836341in}}%
\pgfpathcurveto{\pgfqpoint{3.969999in}{1.825291in}}{\pgfqpoint{3.974389in}{1.814692in}}{\pgfqpoint{3.982203in}{1.806878in}}%
\pgfpathcurveto{\pgfqpoint{3.990016in}{1.799064in}}{\pgfqpoint{4.000616in}{1.794674in}}{\pgfqpoint{4.011666in}{1.794674in}}%
\pgfpathclose%
\pgfusepath{stroke,fill}%
\end{pgfscope}%
\begin{pgfscope}%
\pgfpathrectangle{\pgfqpoint{0.800000in}{0.528000in}}{\pgfqpoint{4.960000in}{3.696000in}}%
\pgfusepath{clip}%
\pgfsetbuttcap%
\pgfsetroundjoin%
\definecolor{currentfill}{rgb}{0.000000,0.000000,0.000000}%
\pgfsetfillcolor{currentfill}%
\pgfsetlinewidth{1.003750pt}%
\definecolor{currentstroke}{rgb}{0.000000,0.000000,0.000000}%
\pgfsetstrokecolor{currentstroke}%
\pgfsetdash{}{0pt}%
\pgfpathmoveto{\pgfqpoint{4.011666in}{1.945157in}}%
\pgfpathcurveto{\pgfqpoint{4.022716in}{1.945157in}}{\pgfqpoint{4.033315in}{1.949547in}}{\pgfqpoint{4.041128in}{1.957360in}}%
\pgfpathcurveto{\pgfqpoint{4.048942in}{1.965174in}}{\pgfqpoint{4.053332in}{1.975773in}}{\pgfqpoint{4.053332in}{1.986823in}}%
\pgfpathcurveto{\pgfqpoint{4.053332in}{1.997873in}}{\pgfqpoint{4.048942in}{2.008472in}}{\pgfqpoint{4.041128in}{2.016286in}}%
\pgfpathcurveto{\pgfqpoint{4.033315in}{2.024100in}}{\pgfqpoint{4.022716in}{2.028490in}}{\pgfqpoint{4.011666in}{2.028490in}}%
\pgfpathcurveto{\pgfqpoint{4.000616in}{2.028490in}}{\pgfqpoint{3.990016in}{2.024100in}}{\pgfqpoint{3.982203in}{2.016286in}}%
\pgfpathcurveto{\pgfqpoint{3.974389in}{2.008472in}}{\pgfqpoint{3.969999in}{1.997873in}}{\pgfqpoint{3.969999in}{1.986823in}}%
\pgfpathcurveto{\pgfqpoint{3.969999in}{1.975773in}}{\pgfqpoint{3.974389in}{1.965174in}}{\pgfqpoint{3.982203in}{1.957360in}}%
\pgfpathcurveto{\pgfqpoint{3.990016in}{1.949547in}}{\pgfqpoint{4.000616in}{1.945157in}}{\pgfqpoint{4.011666in}{1.945157in}}%
\pgfpathclose%
\pgfusepath{stroke,fill}%
\end{pgfscope}%
\begin{pgfscope}%
\pgfpathrectangle{\pgfqpoint{0.800000in}{0.528000in}}{\pgfqpoint{4.960000in}{3.696000in}}%
\pgfusepath{clip}%
\pgfsetbuttcap%
\pgfsetroundjoin%
\definecolor{currentfill}{rgb}{0.000000,0.000000,0.000000}%
\pgfsetfillcolor{currentfill}%
\pgfsetlinewidth{1.003750pt}%
\definecolor{currentstroke}{rgb}{0.000000,0.000000,0.000000}%
\pgfsetstrokecolor{currentstroke}%
\pgfsetdash{}{0pt}%
\pgfpathmoveto{\pgfqpoint{4.011666in}{1.880664in}}%
\pgfpathcurveto{\pgfqpoint{4.022716in}{1.880664in}}{\pgfqpoint{4.033315in}{1.885054in}}{\pgfqpoint{4.041128in}{1.892868in}}%
\pgfpathcurveto{\pgfqpoint{4.048942in}{1.900682in}}{\pgfqpoint{4.053332in}{1.911281in}}{\pgfqpoint{4.053332in}{1.922331in}}%
\pgfpathcurveto{\pgfqpoint{4.053332in}{1.933381in}}{\pgfqpoint{4.048942in}{1.943980in}}{\pgfqpoint{4.041128in}{1.951793in}}%
\pgfpathcurveto{\pgfqpoint{4.033315in}{1.959607in}}{\pgfqpoint{4.022716in}{1.963997in}}{\pgfqpoint{4.011666in}{1.963997in}}%
\pgfpathcurveto{\pgfqpoint{4.000616in}{1.963997in}}{\pgfqpoint{3.990016in}{1.959607in}}{\pgfqpoint{3.982203in}{1.951793in}}%
\pgfpathcurveto{\pgfqpoint{3.974389in}{1.943980in}}{\pgfqpoint{3.969999in}{1.933381in}}{\pgfqpoint{3.969999in}{1.922331in}}%
\pgfpathcurveto{\pgfqpoint{3.969999in}{1.911281in}}{\pgfqpoint{3.974389in}{1.900682in}}{\pgfqpoint{3.982203in}{1.892868in}}%
\pgfpathcurveto{\pgfqpoint{3.990016in}{1.885054in}}{\pgfqpoint{4.000616in}{1.880664in}}{\pgfqpoint{4.011666in}{1.880664in}}%
\pgfpathclose%
\pgfusepath{stroke,fill}%
\end{pgfscope}%
\begin{pgfscope}%
\pgfpathrectangle{\pgfqpoint{0.800000in}{0.528000in}}{\pgfqpoint{4.960000in}{3.696000in}}%
\pgfusepath{clip}%
\pgfsetbuttcap%
\pgfsetroundjoin%
\definecolor{currentfill}{rgb}{0.000000,0.000000,0.000000}%
\pgfsetfillcolor{currentfill}%
\pgfsetlinewidth{1.003750pt}%
\definecolor{currentstroke}{rgb}{0.000000,0.000000,0.000000}%
\pgfsetstrokecolor{currentstroke}%
\pgfsetdash{}{0pt}%
\pgfpathmoveto{\pgfqpoint{4.011666in}{1.880664in}}%
\pgfpathcurveto{\pgfqpoint{4.022716in}{1.880664in}}{\pgfqpoint{4.033315in}{1.885054in}}{\pgfqpoint{4.041128in}{1.892868in}}%
\pgfpathcurveto{\pgfqpoint{4.048942in}{1.900682in}}{\pgfqpoint{4.053332in}{1.911281in}}{\pgfqpoint{4.053332in}{1.922331in}}%
\pgfpathcurveto{\pgfqpoint{4.053332in}{1.933381in}}{\pgfqpoint{4.048942in}{1.943980in}}{\pgfqpoint{4.041128in}{1.951793in}}%
\pgfpathcurveto{\pgfqpoint{4.033315in}{1.959607in}}{\pgfqpoint{4.022716in}{1.963997in}}{\pgfqpoint{4.011666in}{1.963997in}}%
\pgfpathcurveto{\pgfqpoint{4.000616in}{1.963997in}}{\pgfqpoint{3.990016in}{1.959607in}}{\pgfqpoint{3.982203in}{1.951793in}}%
\pgfpathcurveto{\pgfqpoint{3.974389in}{1.943980in}}{\pgfqpoint{3.969999in}{1.933381in}}{\pgfqpoint{3.969999in}{1.922331in}}%
\pgfpathcurveto{\pgfqpoint{3.969999in}{1.911281in}}{\pgfqpoint{3.974389in}{1.900682in}}{\pgfqpoint{3.982203in}{1.892868in}}%
\pgfpathcurveto{\pgfqpoint{3.990016in}{1.885054in}}{\pgfqpoint{4.000616in}{1.880664in}}{\pgfqpoint{4.011666in}{1.880664in}}%
\pgfpathclose%
\pgfusepath{stroke,fill}%
\end{pgfscope}%
\begin{pgfscope}%
\pgfpathrectangle{\pgfqpoint{0.800000in}{0.528000in}}{\pgfqpoint{4.960000in}{3.696000in}}%
\pgfusepath{clip}%
\pgfsetbuttcap%
\pgfsetroundjoin%
\definecolor{currentfill}{rgb}{0.000000,0.000000,0.000000}%
\pgfsetfillcolor{currentfill}%
\pgfsetlinewidth{1.003750pt}%
\definecolor{currentstroke}{rgb}{0.000000,0.000000,0.000000}%
\pgfsetstrokecolor{currentstroke}%
\pgfsetdash{}{0pt}%
\pgfpathmoveto{\pgfqpoint{4.011666in}{1.773177in}}%
\pgfpathcurveto{\pgfqpoint{4.022716in}{1.773177in}}{\pgfqpoint{4.033315in}{1.777567in}}{\pgfqpoint{4.041128in}{1.785380in}}%
\pgfpathcurveto{\pgfqpoint{4.048942in}{1.793194in}}{\pgfqpoint{4.053332in}{1.803793in}}{\pgfqpoint{4.053332in}{1.814843in}}%
\pgfpathcurveto{\pgfqpoint{4.053332in}{1.825893in}}{\pgfqpoint{4.048942in}{1.836492in}}{\pgfqpoint{4.041128in}{1.844306in}}%
\pgfpathcurveto{\pgfqpoint{4.033315in}{1.852120in}}{\pgfqpoint{4.022716in}{1.856510in}}{\pgfqpoint{4.011666in}{1.856510in}}%
\pgfpathcurveto{\pgfqpoint{4.000616in}{1.856510in}}{\pgfqpoint{3.990016in}{1.852120in}}{\pgfqpoint{3.982203in}{1.844306in}}%
\pgfpathcurveto{\pgfqpoint{3.974389in}{1.836492in}}{\pgfqpoint{3.969999in}{1.825893in}}{\pgfqpoint{3.969999in}{1.814843in}}%
\pgfpathcurveto{\pgfqpoint{3.969999in}{1.803793in}}{\pgfqpoint{3.974389in}{1.793194in}}{\pgfqpoint{3.982203in}{1.785380in}}%
\pgfpathcurveto{\pgfqpoint{3.990016in}{1.777567in}}{\pgfqpoint{4.000616in}{1.773177in}}{\pgfqpoint{4.011666in}{1.773177in}}%
\pgfpathclose%
\pgfusepath{stroke,fill}%
\end{pgfscope}%
\begin{pgfscope}%
\pgfpathrectangle{\pgfqpoint{0.800000in}{0.528000in}}{\pgfqpoint{4.960000in}{3.696000in}}%
\pgfusepath{clip}%
\pgfsetbuttcap%
\pgfsetroundjoin%
\definecolor{currentfill}{rgb}{0.000000,0.000000,0.000000}%
\pgfsetfillcolor{currentfill}%
\pgfsetlinewidth{1.003750pt}%
\definecolor{currentstroke}{rgb}{0.000000,0.000000,0.000000}%
\pgfsetstrokecolor{currentstroke}%
\pgfsetdash{}{0pt}%
\pgfpathmoveto{\pgfqpoint{4.011666in}{1.794674in}}%
\pgfpathcurveto{\pgfqpoint{4.022716in}{1.794674in}}{\pgfqpoint{4.033315in}{1.799064in}}{\pgfqpoint{4.041128in}{1.806878in}}%
\pgfpathcurveto{\pgfqpoint{4.048942in}{1.814692in}}{\pgfqpoint{4.053332in}{1.825291in}}{\pgfqpoint{4.053332in}{1.836341in}}%
\pgfpathcurveto{\pgfqpoint{4.053332in}{1.847391in}}{\pgfqpoint{4.048942in}{1.857990in}}{\pgfqpoint{4.041128in}{1.865804in}}%
\pgfpathcurveto{\pgfqpoint{4.033315in}{1.873617in}}{\pgfqpoint{4.022716in}{1.878007in}}{\pgfqpoint{4.011666in}{1.878007in}}%
\pgfpathcurveto{\pgfqpoint{4.000616in}{1.878007in}}{\pgfqpoint{3.990016in}{1.873617in}}{\pgfqpoint{3.982203in}{1.865804in}}%
\pgfpathcurveto{\pgfqpoint{3.974389in}{1.857990in}}{\pgfqpoint{3.969999in}{1.847391in}}{\pgfqpoint{3.969999in}{1.836341in}}%
\pgfpathcurveto{\pgfqpoint{3.969999in}{1.825291in}}{\pgfqpoint{3.974389in}{1.814692in}}{\pgfqpoint{3.982203in}{1.806878in}}%
\pgfpathcurveto{\pgfqpoint{3.990016in}{1.799064in}}{\pgfqpoint{4.000616in}{1.794674in}}{\pgfqpoint{4.011666in}{1.794674in}}%
\pgfpathclose%
\pgfusepath{stroke,fill}%
\end{pgfscope}%
\begin{pgfscope}%
\pgfpathrectangle{\pgfqpoint{0.800000in}{0.528000in}}{\pgfqpoint{4.960000in}{3.696000in}}%
\pgfusepath{clip}%
\pgfsetbuttcap%
\pgfsetroundjoin%
\definecolor{currentfill}{rgb}{0.000000,0.000000,0.000000}%
\pgfsetfillcolor{currentfill}%
\pgfsetlinewidth{1.003750pt}%
\definecolor{currentstroke}{rgb}{0.000000,0.000000,0.000000}%
\pgfsetstrokecolor{currentstroke}%
\pgfsetdash{}{0pt}%
\pgfpathmoveto{\pgfqpoint{4.011666in}{1.880664in}}%
\pgfpathcurveto{\pgfqpoint{4.022716in}{1.880664in}}{\pgfqpoint{4.033315in}{1.885054in}}{\pgfqpoint{4.041128in}{1.892868in}}%
\pgfpathcurveto{\pgfqpoint{4.048942in}{1.900682in}}{\pgfqpoint{4.053332in}{1.911281in}}{\pgfqpoint{4.053332in}{1.922331in}}%
\pgfpathcurveto{\pgfqpoint{4.053332in}{1.933381in}}{\pgfqpoint{4.048942in}{1.943980in}}{\pgfqpoint{4.041128in}{1.951793in}}%
\pgfpathcurveto{\pgfqpoint{4.033315in}{1.959607in}}{\pgfqpoint{4.022716in}{1.963997in}}{\pgfqpoint{4.011666in}{1.963997in}}%
\pgfpathcurveto{\pgfqpoint{4.000616in}{1.963997in}}{\pgfqpoint{3.990016in}{1.959607in}}{\pgfqpoint{3.982203in}{1.951793in}}%
\pgfpathcurveto{\pgfqpoint{3.974389in}{1.943980in}}{\pgfqpoint{3.969999in}{1.933381in}}{\pgfqpoint{3.969999in}{1.922331in}}%
\pgfpathcurveto{\pgfqpoint{3.969999in}{1.911281in}}{\pgfqpoint{3.974389in}{1.900682in}}{\pgfqpoint{3.982203in}{1.892868in}}%
\pgfpathcurveto{\pgfqpoint{3.990016in}{1.885054in}}{\pgfqpoint{4.000616in}{1.880664in}}{\pgfqpoint{4.011666in}{1.880664in}}%
\pgfpathclose%
\pgfusepath{stroke,fill}%
\end{pgfscope}%
\begin{pgfscope}%
\pgfpathrectangle{\pgfqpoint{0.800000in}{0.528000in}}{\pgfqpoint{4.960000in}{3.696000in}}%
\pgfusepath{clip}%
\pgfsetbuttcap%
\pgfsetroundjoin%
\definecolor{currentfill}{rgb}{0.000000,0.000000,0.000000}%
\pgfsetfillcolor{currentfill}%
\pgfsetlinewidth{1.003750pt}%
\definecolor{currentstroke}{rgb}{0.000000,0.000000,0.000000}%
\pgfsetstrokecolor{currentstroke}%
\pgfsetdash{}{0pt}%
\pgfpathmoveto{\pgfqpoint{4.011666in}{1.773177in}}%
\pgfpathcurveto{\pgfqpoint{4.022716in}{1.773177in}}{\pgfqpoint{4.033315in}{1.777567in}}{\pgfqpoint{4.041128in}{1.785380in}}%
\pgfpathcurveto{\pgfqpoint{4.048942in}{1.793194in}}{\pgfqpoint{4.053332in}{1.803793in}}{\pgfqpoint{4.053332in}{1.814843in}}%
\pgfpathcurveto{\pgfqpoint{4.053332in}{1.825893in}}{\pgfqpoint{4.048942in}{1.836492in}}{\pgfqpoint{4.041128in}{1.844306in}}%
\pgfpathcurveto{\pgfqpoint{4.033315in}{1.852120in}}{\pgfqpoint{4.022716in}{1.856510in}}{\pgfqpoint{4.011666in}{1.856510in}}%
\pgfpathcurveto{\pgfqpoint{4.000616in}{1.856510in}}{\pgfqpoint{3.990016in}{1.852120in}}{\pgfqpoint{3.982203in}{1.844306in}}%
\pgfpathcurveto{\pgfqpoint{3.974389in}{1.836492in}}{\pgfqpoint{3.969999in}{1.825893in}}{\pgfqpoint{3.969999in}{1.814843in}}%
\pgfpathcurveto{\pgfqpoint{3.969999in}{1.803793in}}{\pgfqpoint{3.974389in}{1.793194in}}{\pgfqpoint{3.982203in}{1.785380in}}%
\pgfpathcurveto{\pgfqpoint{3.990016in}{1.777567in}}{\pgfqpoint{4.000616in}{1.773177in}}{\pgfqpoint{4.011666in}{1.773177in}}%
\pgfpathclose%
\pgfusepath{stroke,fill}%
\end{pgfscope}%
\begin{pgfscope}%
\pgfpathrectangle{\pgfqpoint{0.800000in}{0.528000in}}{\pgfqpoint{4.960000in}{3.696000in}}%
\pgfusepath{clip}%
\pgfsetbuttcap%
\pgfsetroundjoin%
\definecolor{currentfill}{rgb}{0.000000,0.000000,0.000000}%
\pgfsetfillcolor{currentfill}%
\pgfsetlinewidth{1.003750pt}%
\definecolor{currentstroke}{rgb}{0.000000,0.000000,0.000000}%
\pgfsetstrokecolor{currentstroke}%
\pgfsetdash{}{0pt}%
\pgfpathmoveto{\pgfqpoint{4.011666in}{1.816172in}}%
\pgfpathcurveto{\pgfqpoint{4.022716in}{1.816172in}}{\pgfqpoint{4.033315in}{1.820562in}}{\pgfqpoint{4.041128in}{1.828375in}}%
\pgfpathcurveto{\pgfqpoint{4.048942in}{1.836189in}}{\pgfqpoint{4.053332in}{1.846788in}}{\pgfqpoint{4.053332in}{1.857838in}}%
\pgfpathcurveto{\pgfqpoint{4.053332in}{1.868888in}}{\pgfqpoint{4.048942in}{1.879487in}}{\pgfqpoint{4.041128in}{1.887301in}}%
\pgfpathcurveto{\pgfqpoint{4.033315in}{1.895115in}}{\pgfqpoint{4.022716in}{1.899505in}}{\pgfqpoint{4.011666in}{1.899505in}}%
\pgfpathcurveto{\pgfqpoint{4.000616in}{1.899505in}}{\pgfqpoint{3.990016in}{1.895115in}}{\pgfqpoint{3.982203in}{1.887301in}}%
\pgfpathcurveto{\pgfqpoint{3.974389in}{1.879487in}}{\pgfqpoint{3.969999in}{1.868888in}}{\pgfqpoint{3.969999in}{1.857838in}}%
\pgfpathcurveto{\pgfqpoint{3.969999in}{1.846788in}}{\pgfqpoint{3.974389in}{1.836189in}}{\pgfqpoint{3.982203in}{1.828375in}}%
\pgfpathcurveto{\pgfqpoint{3.990016in}{1.820562in}}{\pgfqpoint{4.000616in}{1.816172in}}{\pgfqpoint{4.011666in}{1.816172in}}%
\pgfpathclose%
\pgfusepath{stroke,fill}%
\end{pgfscope}%
\begin{pgfscope}%
\pgfpathrectangle{\pgfqpoint{0.800000in}{0.528000in}}{\pgfqpoint{4.960000in}{3.696000in}}%
\pgfusepath{clip}%
\pgfsetbuttcap%
\pgfsetroundjoin%
\definecolor{currentfill}{rgb}{0.000000,0.000000,0.000000}%
\pgfsetfillcolor{currentfill}%
\pgfsetlinewidth{1.003750pt}%
\definecolor{currentstroke}{rgb}{0.000000,0.000000,0.000000}%
\pgfsetstrokecolor{currentstroke}%
\pgfsetdash{}{0pt}%
\pgfpathmoveto{\pgfqpoint{4.011666in}{1.988151in}}%
\pgfpathcurveto{\pgfqpoint{4.022716in}{1.988151in}}{\pgfqpoint{4.033315in}{1.992542in}}{\pgfqpoint{4.041128in}{2.000355in}}%
\pgfpathcurveto{\pgfqpoint{4.048942in}{2.008169in}}{\pgfqpoint{4.053332in}{2.018768in}}{\pgfqpoint{4.053332in}{2.029818in}}%
\pgfpathcurveto{\pgfqpoint{4.053332in}{2.040868in}}{\pgfqpoint{4.048942in}{2.051467in}}{\pgfqpoint{4.041128in}{2.059281in}}%
\pgfpathcurveto{\pgfqpoint{4.033315in}{2.067095in}}{\pgfqpoint{4.022716in}{2.071485in}}{\pgfqpoint{4.011666in}{2.071485in}}%
\pgfpathcurveto{\pgfqpoint{4.000616in}{2.071485in}}{\pgfqpoint{3.990016in}{2.067095in}}{\pgfqpoint{3.982203in}{2.059281in}}%
\pgfpathcurveto{\pgfqpoint{3.974389in}{2.051467in}}{\pgfqpoint{3.969999in}{2.040868in}}{\pgfqpoint{3.969999in}{2.029818in}}%
\pgfpathcurveto{\pgfqpoint{3.969999in}{2.018768in}}{\pgfqpoint{3.974389in}{2.008169in}}{\pgfqpoint{3.982203in}{2.000355in}}%
\pgfpathcurveto{\pgfqpoint{3.990016in}{1.992542in}}{\pgfqpoint{4.000616in}{1.988151in}}{\pgfqpoint{4.011666in}{1.988151in}}%
\pgfpathclose%
\pgfusepath{stroke,fill}%
\end{pgfscope}%
\begin{pgfscope}%
\pgfpathrectangle{\pgfqpoint{0.800000in}{0.528000in}}{\pgfqpoint{4.960000in}{3.696000in}}%
\pgfusepath{clip}%
\pgfsetbuttcap%
\pgfsetroundjoin%
\definecolor{currentfill}{rgb}{0.000000,0.000000,0.000000}%
\pgfsetfillcolor{currentfill}%
\pgfsetlinewidth{1.003750pt}%
\definecolor{currentstroke}{rgb}{0.000000,0.000000,0.000000}%
\pgfsetstrokecolor{currentstroke}%
\pgfsetdash{}{0pt}%
\pgfpathmoveto{\pgfqpoint{4.011666in}{2.052644in}}%
\pgfpathcurveto{\pgfqpoint{4.022716in}{2.052644in}}{\pgfqpoint{4.033315in}{2.057034in}}{\pgfqpoint{4.041128in}{2.064848in}}%
\pgfpathcurveto{\pgfqpoint{4.048942in}{2.072661in}}{\pgfqpoint{4.053332in}{2.083261in}}{\pgfqpoint{4.053332in}{2.094311in}}%
\pgfpathcurveto{\pgfqpoint{4.053332in}{2.105361in}}{\pgfqpoint{4.048942in}{2.115960in}}{\pgfqpoint{4.041128in}{2.123773in}}%
\pgfpathcurveto{\pgfqpoint{4.033315in}{2.131587in}}{\pgfqpoint{4.022716in}{2.135977in}}{\pgfqpoint{4.011666in}{2.135977in}}%
\pgfpathcurveto{\pgfqpoint{4.000616in}{2.135977in}}{\pgfqpoint{3.990016in}{2.131587in}}{\pgfqpoint{3.982203in}{2.123773in}}%
\pgfpathcurveto{\pgfqpoint{3.974389in}{2.115960in}}{\pgfqpoint{3.969999in}{2.105361in}}{\pgfqpoint{3.969999in}{2.094311in}}%
\pgfpathcurveto{\pgfqpoint{3.969999in}{2.083261in}}{\pgfqpoint{3.974389in}{2.072661in}}{\pgfqpoint{3.982203in}{2.064848in}}%
\pgfpathcurveto{\pgfqpoint{3.990016in}{2.057034in}}{\pgfqpoint{4.000616in}{2.052644in}}{\pgfqpoint{4.011666in}{2.052644in}}%
\pgfpathclose%
\pgfusepath{stroke,fill}%
\end{pgfscope}%
\begin{pgfscope}%
\pgfpathrectangle{\pgfqpoint{0.800000in}{0.528000in}}{\pgfqpoint{4.960000in}{3.696000in}}%
\pgfusepath{clip}%
\pgfsetbuttcap%
\pgfsetroundjoin%
\definecolor{currentfill}{rgb}{0.000000,0.000000,0.000000}%
\pgfsetfillcolor{currentfill}%
\pgfsetlinewidth{1.003750pt}%
\definecolor{currentstroke}{rgb}{0.000000,0.000000,0.000000}%
\pgfsetstrokecolor{currentstroke}%
\pgfsetdash{}{0pt}%
\pgfpathmoveto{\pgfqpoint{4.011666in}{1.837669in}}%
\pgfpathcurveto{\pgfqpoint{4.022716in}{1.837669in}}{\pgfqpoint{4.033315in}{1.842059in}}{\pgfqpoint{4.041128in}{1.849873in}}%
\pgfpathcurveto{\pgfqpoint{4.048942in}{1.857687in}}{\pgfqpoint{4.053332in}{1.868286in}}{\pgfqpoint{4.053332in}{1.879336in}}%
\pgfpathcurveto{\pgfqpoint{4.053332in}{1.890386in}}{\pgfqpoint{4.048942in}{1.900985in}}{\pgfqpoint{4.041128in}{1.908798in}}%
\pgfpathcurveto{\pgfqpoint{4.033315in}{1.916612in}}{\pgfqpoint{4.022716in}{1.921002in}}{\pgfqpoint{4.011666in}{1.921002in}}%
\pgfpathcurveto{\pgfqpoint{4.000616in}{1.921002in}}{\pgfqpoint{3.990016in}{1.916612in}}{\pgfqpoint{3.982203in}{1.908798in}}%
\pgfpathcurveto{\pgfqpoint{3.974389in}{1.900985in}}{\pgfqpoint{3.969999in}{1.890386in}}{\pgfqpoint{3.969999in}{1.879336in}}%
\pgfpathcurveto{\pgfqpoint{3.969999in}{1.868286in}}{\pgfqpoint{3.974389in}{1.857687in}}{\pgfqpoint{3.982203in}{1.849873in}}%
\pgfpathcurveto{\pgfqpoint{3.990016in}{1.842059in}}{\pgfqpoint{4.000616in}{1.837669in}}{\pgfqpoint{4.011666in}{1.837669in}}%
\pgfpathclose%
\pgfusepath{stroke,fill}%
\end{pgfscope}%
\begin{pgfscope}%
\pgfpathrectangle{\pgfqpoint{0.800000in}{0.528000in}}{\pgfqpoint{4.960000in}{3.696000in}}%
\pgfusepath{clip}%
\pgfsetbuttcap%
\pgfsetroundjoin%
\definecolor{currentfill}{rgb}{0.000000,0.000000,0.000000}%
\pgfsetfillcolor{currentfill}%
\pgfsetlinewidth{1.003750pt}%
\definecolor{currentstroke}{rgb}{0.000000,0.000000,0.000000}%
\pgfsetstrokecolor{currentstroke}%
\pgfsetdash{}{0pt}%
\pgfpathmoveto{\pgfqpoint{4.011666in}{1.880664in}}%
\pgfpathcurveto{\pgfqpoint{4.022716in}{1.880664in}}{\pgfqpoint{4.033315in}{1.885054in}}{\pgfqpoint{4.041128in}{1.892868in}}%
\pgfpathcurveto{\pgfqpoint{4.048942in}{1.900682in}}{\pgfqpoint{4.053332in}{1.911281in}}{\pgfqpoint{4.053332in}{1.922331in}}%
\pgfpathcurveto{\pgfqpoint{4.053332in}{1.933381in}}{\pgfqpoint{4.048942in}{1.943980in}}{\pgfqpoint{4.041128in}{1.951793in}}%
\pgfpathcurveto{\pgfqpoint{4.033315in}{1.959607in}}{\pgfqpoint{4.022716in}{1.963997in}}{\pgfqpoint{4.011666in}{1.963997in}}%
\pgfpathcurveto{\pgfqpoint{4.000616in}{1.963997in}}{\pgfqpoint{3.990016in}{1.959607in}}{\pgfqpoint{3.982203in}{1.951793in}}%
\pgfpathcurveto{\pgfqpoint{3.974389in}{1.943980in}}{\pgfqpoint{3.969999in}{1.933381in}}{\pgfqpoint{3.969999in}{1.922331in}}%
\pgfpathcurveto{\pgfqpoint{3.969999in}{1.911281in}}{\pgfqpoint{3.974389in}{1.900682in}}{\pgfqpoint{3.982203in}{1.892868in}}%
\pgfpathcurveto{\pgfqpoint{3.990016in}{1.885054in}}{\pgfqpoint{4.000616in}{1.880664in}}{\pgfqpoint{4.011666in}{1.880664in}}%
\pgfpathclose%
\pgfusepath{stroke,fill}%
\end{pgfscope}%
\begin{pgfscope}%
\pgfpathrectangle{\pgfqpoint{0.800000in}{0.528000in}}{\pgfqpoint{4.960000in}{3.696000in}}%
\pgfusepath{clip}%
\pgfsetbuttcap%
\pgfsetroundjoin%
\definecolor{currentfill}{rgb}{0.000000,0.000000,0.000000}%
\pgfsetfillcolor{currentfill}%
\pgfsetlinewidth{1.003750pt}%
\definecolor{currentstroke}{rgb}{0.000000,0.000000,0.000000}%
\pgfsetstrokecolor{currentstroke}%
\pgfsetdash{}{0pt}%
\pgfpathmoveto{\pgfqpoint{4.011666in}{1.988151in}}%
\pgfpathcurveto{\pgfqpoint{4.022716in}{1.988151in}}{\pgfqpoint{4.033315in}{1.992542in}}{\pgfqpoint{4.041128in}{2.000355in}}%
\pgfpathcurveto{\pgfqpoint{4.048942in}{2.008169in}}{\pgfqpoint{4.053332in}{2.018768in}}{\pgfqpoint{4.053332in}{2.029818in}}%
\pgfpathcurveto{\pgfqpoint{4.053332in}{2.040868in}}{\pgfqpoint{4.048942in}{2.051467in}}{\pgfqpoint{4.041128in}{2.059281in}}%
\pgfpathcurveto{\pgfqpoint{4.033315in}{2.067095in}}{\pgfqpoint{4.022716in}{2.071485in}}{\pgfqpoint{4.011666in}{2.071485in}}%
\pgfpathcurveto{\pgfqpoint{4.000616in}{2.071485in}}{\pgfqpoint{3.990016in}{2.067095in}}{\pgfqpoint{3.982203in}{2.059281in}}%
\pgfpathcurveto{\pgfqpoint{3.974389in}{2.051467in}}{\pgfqpoint{3.969999in}{2.040868in}}{\pgfqpoint{3.969999in}{2.029818in}}%
\pgfpathcurveto{\pgfqpoint{3.969999in}{2.018768in}}{\pgfqpoint{3.974389in}{2.008169in}}{\pgfqpoint{3.982203in}{2.000355in}}%
\pgfpathcurveto{\pgfqpoint{3.990016in}{1.992542in}}{\pgfqpoint{4.000616in}{1.988151in}}{\pgfqpoint{4.011666in}{1.988151in}}%
\pgfpathclose%
\pgfusepath{stroke,fill}%
\end{pgfscope}%
\begin{pgfscope}%
\pgfpathrectangle{\pgfqpoint{0.800000in}{0.528000in}}{\pgfqpoint{4.960000in}{3.696000in}}%
\pgfusepath{clip}%
\pgfsetbuttcap%
\pgfsetroundjoin%
\definecolor{currentfill}{rgb}{0.000000,0.000000,0.000000}%
\pgfsetfillcolor{currentfill}%
\pgfsetlinewidth{1.003750pt}%
\definecolor{currentstroke}{rgb}{0.000000,0.000000,0.000000}%
\pgfsetstrokecolor{currentstroke}%
\pgfsetdash{}{0pt}%
\pgfpathmoveto{\pgfqpoint{4.011666in}{1.923659in}}%
\pgfpathcurveto{\pgfqpoint{4.022716in}{1.923659in}}{\pgfqpoint{4.033315in}{1.928049in}}{\pgfqpoint{4.041128in}{1.935863in}}%
\pgfpathcurveto{\pgfqpoint{4.048942in}{1.943677in}}{\pgfqpoint{4.053332in}{1.954276in}}{\pgfqpoint{4.053332in}{1.965326in}}%
\pgfpathcurveto{\pgfqpoint{4.053332in}{1.976376in}}{\pgfqpoint{4.048942in}{1.986975in}}{\pgfqpoint{4.041128in}{1.994788in}}%
\pgfpathcurveto{\pgfqpoint{4.033315in}{2.002602in}}{\pgfqpoint{4.022716in}{2.006992in}}{\pgfqpoint{4.011666in}{2.006992in}}%
\pgfpathcurveto{\pgfqpoint{4.000616in}{2.006992in}}{\pgfqpoint{3.990016in}{2.002602in}}{\pgfqpoint{3.982203in}{1.994788in}}%
\pgfpathcurveto{\pgfqpoint{3.974389in}{1.986975in}}{\pgfqpoint{3.969999in}{1.976376in}}{\pgfqpoint{3.969999in}{1.965326in}}%
\pgfpathcurveto{\pgfqpoint{3.969999in}{1.954276in}}{\pgfqpoint{3.974389in}{1.943677in}}{\pgfqpoint{3.982203in}{1.935863in}}%
\pgfpathcurveto{\pgfqpoint{3.990016in}{1.928049in}}{\pgfqpoint{4.000616in}{1.923659in}}{\pgfqpoint{4.011666in}{1.923659in}}%
\pgfpathclose%
\pgfusepath{stroke,fill}%
\end{pgfscope}%
\begin{pgfscope}%
\pgfpathrectangle{\pgfqpoint{0.800000in}{0.528000in}}{\pgfqpoint{4.960000in}{3.696000in}}%
\pgfusepath{clip}%
\pgfsetbuttcap%
\pgfsetroundjoin%
\definecolor{currentfill}{rgb}{0.000000,0.000000,0.000000}%
\pgfsetfillcolor{currentfill}%
\pgfsetlinewidth{1.003750pt}%
\definecolor{currentstroke}{rgb}{0.000000,0.000000,0.000000}%
\pgfsetstrokecolor{currentstroke}%
\pgfsetdash{}{0pt}%
\pgfpathmoveto{\pgfqpoint{4.011666in}{1.816172in}}%
\pgfpathcurveto{\pgfqpoint{4.022716in}{1.816172in}}{\pgfqpoint{4.033315in}{1.820562in}}{\pgfqpoint{4.041128in}{1.828375in}}%
\pgfpathcurveto{\pgfqpoint{4.048942in}{1.836189in}}{\pgfqpoint{4.053332in}{1.846788in}}{\pgfqpoint{4.053332in}{1.857838in}}%
\pgfpathcurveto{\pgfqpoint{4.053332in}{1.868888in}}{\pgfqpoint{4.048942in}{1.879487in}}{\pgfqpoint{4.041128in}{1.887301in}}%
\pgfpathcurveto{\pgfqpoint{4.033315in}{1.895115in}}{\pgfqpoint{4.022716in}{1.899505in}}{\pgfqpoint{4.011666in}{1.899505in}}%
\pgfpathcurveto{\pgfqpoint{4.000616in}{1.899505in}}{\pgfqpoint{3.990016in}{1.895115in}}{\pgfqpoint{3.982203in}{1.887301in}}%
\pgfpathcurveto{\pgfqpoint{3.974389in}{1.879487in}}{\pgfqpoint{3.969999in}{1.868888in}}{\pgfqpoint{3.969999in}{1.857838in}}%
\pgfpathcurveto{\pgfqpoint{3.969999in}{1.846788in}}{\pgfqpoint{3.974389in}{1.836189in}}{\pgfqpoint{3.982203in}{1.828375in}}%
\pgfpathcurveto{\pgfqpoint{3.990016in}{1.820562in}}{\pgfqpoint{4.000616in}{1.816172in}}{\pgfqpoint{4.011666in}{1.816172in}}%
\pgfpathclose%
\pgfusepath{stroke,fill}%
\end{pgfscope}%
\begin{pgfscope}%
\pgfpathrectangle{\pgfqpoint{0.800000in}{0.528000in}}{\pgfqpoint{4.960000in}{3.696000in}}%
\pgfusepath{clip}%
\pgfsetbuttcap%
\pgfsetroundjoin%
\definecolor{currentfill}{rgb}{0.000000,0.000000,0.000000}%
\pgfsetfillcolor{currentfill}%
\pgfsetlinewidth{1.003750pt}%
\definecolor{currentstroke}{rgb}{0.000000,0.000000,0.000000}%
\pgfsetstrokecolor{currentstroke}%
\pgfsetdash{}{0pt}%
\pgfpathmoveto{\pgfqpoint{4.011666in}{2.052644in}}%
\pgfpathcurveto{\pgfqpoint{4.022716in}{2.052644in}}{\pgfqpoint{4.033315in}{2.057034in}}{\pgfqpoint{4.041128in}{2.064848in}}%
\pgfpathcurveto{\pgfqpoint{4.048942in}{2.072661in}}{\pgfqpoint{4.053332in}{2.083261in}}{\pgfqpoint{4.053332in}{2.094311in}}%
\pgfpathcurveto{\pgfqpoint{4.053332in}{2.105361in}}{\pgfqpoint{4.048942in}{2.115960in}}{\pgfqpoint{4.041128in}{2.123773in}}%
\pgfpathcurveto{\pgfqpoint{4.033315in}{2.131587in}}{\pgfqpoint{4.022716in}{2.135977in}}{\pgfqpoint{4.011666in}{2.135977in}}%
\pgfpathcurveto{\pgfqpoint{4.000616in}{2.135977in}}{\pgfqpoint{3.990016in}{2.131587in}}{\pgfqpoint{3.982203in}{2.123773in}}%
\pgfpathcurveto{\pgfqpoint{3.974389in}{2.115960in}}{\pgfqpoint{3.969999in}{2.105361in}}{\pgfqpoint{3.969999in}{2.094311in}}%
\pgfpathcurveto{\pgfqpoint{3.969999in}{2.083261in}}{\pgfqpoint{3.974389in}{2.072661in}}{\pgfqpoint{3.982203in}{2.064848in}}%
\pgfpathcurveto{\pgfqpoint{3.990016in}{2.057034in}}{\pgfqpoint{4.000616in}{2.052644in}}{\pgfqpoint{4.011666in}{2.052644in}}%
\pgfpathclose%
\pgfusepath{stroke,fill}%
\end{pgfscope}%
\begin{pgfscope}%
\pgfpathrectangle{\pgfqpoint{0.800000in}{0.528000in}}{\pgfqpoint{4.960000in}{3.696000in}}%
\pgfusepath{clip}%
\pgfsetbuttcap%
\pgfsetroundjoin%
\definecolor{currentfill}{rgb}{0.000000,0.000000,0.000000}%
\pgfsetfillcolor{currentfill}%
\pgfsetlinewidth{1.003750pt}%
\definecolor{currentstroke}{rgb}{0.000000,0.000000,0.000000}%
\pgfsetstrokecolor{currentstroke}%
\pgfsetdash{}{0pt}%
\pgfpathmoveto{\pgfqpoint{4.011666in}{1.859167in}}%
\pgfpathcurveto{\pgfqpoint{4.022716in}{1.859167in}}{\pgfqpoint{4.033315in}{1.863557in}}{\pgfqpoint{4.041128in}{1.871370in}}%
\pgfpathcurveto{\pgfqpoint{4.048942in}{1.879184in}}{\pgfqpoint{4.053332in}{1.889783in}}{\pgfqpoint{4.053332in}{1.900833in}}%
\pgfpathcurveto{\pgfqpoint{4.053332in}{1.911883in}}{\pgfqpoint{4.048942in}{1.922482in}}{\pgfqpoint{4.041128in}{1.930296in}}%
\pgfpathcurveto{\pgfqpoint{4.033315in}{1.938110in}}{\pgfqpoint{4.022716in}{1.942500in}}{\pgfqpoint{4.011666in}{1.942500in}}%
\pgfpathcurveto{\pgfqpoint{4.000616in}{1.942500in}}{\pgfqpoint{3.990016in}{1.938110in}}{\pgfqpoint{3.982203in}{1.930296in}}%
\pgfpathcurveto{\pgfqpoint{3.974389in}{1.922482in}}{\pgfqpoint{3.969999in}{1.911883in}}{\pgfqpoint{3.969999in}{1.900833in}}%
\pgfpathcurveto{\pgfqpoint{3.969999in}{1.889783in}}{\pgfqpoint{3.974389in}{1.879184in}}{\pgfqpoint{3.982203in}{1.871370in}}%
\pgfpathcurveto{\pgfqpoint{3.990016in}{1.863557in}}{\pgfqpoint{4.000616in}{1.859167in}}{\pgfqpoint{4.011666in}{1.859167in}}%
\pgfpathclose%
\pgfusepath{stroke,fill}%
\end{pgfscope}%
\begin{pgfscope}%
\pgfpathrectangle{\pgfqpoint{0.800000in}{0.528000in}}{\pgfqpoint{4.960000in}{3.696000in}}%
\pgfusepath{clip}%
\pgfsetbuttcap%
\pgfsetroundjoin%
\definecolor{currentfill}{rgb}{0.000000,0.000000,0.000000}%
\pgfsetfillcolor{currentfill}%
\pgfsetlinewidth{1.003750pt}%
\definecolor{currentstroke}{rgb}{0.000000,0.000000,0.000000}%
\pgfsetstrokecolor{currentstroke}%
\pgfsetdash{}{0pt}%
\pgfpathmoveto{\pgfqpoint{4.011666in}{1.816172in}}%
\pgfpathcurveto{\pgfqpoint{4.022716in}{1.816172in}}{\pgfqpoint{4.033315in}{1.820562in}}{\pgfqpoint{4.041128in}{1.828375in}}%
\pgfpathcurveto{\pgfqpoint{4.048942in}{1.836189in}}{\pgfqpoint{4.053332in}{1.846788in}}{\pgfqpoint{4.053332in}{1.857838in}}%
\pgfpathcurveto{\pgfqpoint{4.053332in}{1.868888in}}{\pgfqpoint{4.048942in}{1.879487in}}{\pgfqpoint{4.041128in}{1.887301in}}%
\pgfpathcurveto{\pgfqpoint{4.033315in}{1.895115in}}{\pgfqpoint{4.022716in}{1.899505in}}{\pgfqpoint{4.011666in}{1.899505in}}%
\pgfpathcurveto{\pgfqpoint{4.000616in}{1.899505in}}{\pgfqpoint{3.990016in}{1.895115in}}{\pgfqpoint{3.982203in}{1.887301in}}%
\pgfpathcurveto{\pgfqpoint{3.974389in}{1.879487in}}{\pgfqpoint{3.969999in}{1.868888in}}{\pgfqpoint{3.969999in}{1.857838in}}%
\pgfpathcurveto{\pgfqpoint{3.969999in}{1.846788in}}{\pgfqpoint{3.974389in}{1.836189in}}{\pgfqpoint{3.982203in}{1.828375in}}%
\pgfpathcurveto{\pgfqpoint{3.990016in}{1.820562in}}{\pgfqpoint{4.000616in}{1.816172in}}{\pgfqpoint{4.011666in}{1.816172in}}%
\pgfpathclose%
\pgfusepath{stroke,fill}%
\end{pgfscope}%
\begin{pgfscope}%
\pgfpathrectangle{\pgfqpoint{0.800000in}{0.528000in}}{\pgfqpoint{4.960000in}{3.696000in}}%
\pgfusepath{clip}%
\pgfsetbuttcap%
\pgfsetroundjoin%
\definecolor{currentfill}{rgb}{0.000000,0.000000,0.000000}%
\pgfsetfillcolor{currentfill}%
\pgfsetlinewidth{1.003750pt}%
\definecolor{currentstroke}{rgb}{0.000000,0.000000,0.000000}%
\pgfsetstrokecolor{currentstroke}%
\pgfsetdash{}{0pt}%
\pgfpathmoveto{\pgfqpoint{4.011666in}{1.945157in}}%
\pgfpathcurveto{\pgfqpoint{4.022716in}{1.945157in}}{\pgfqpoint{4.033315in}{1.949547in}}{\pgfqpoint{4.041128in}{1.957360in}}%
\pgfpathcurveto{\pgfqpoint{4.048942in}{1.965174in}}{\pgfqpoint{4.053332in}{1.975773in}}{\pgfqpoint{4.053332in}{1.986823in}}%
\pgfpathcurveto{\pgfqpoint{4.053332in}{1.997873in}}{\pgfqpoint{4.048942in}{2.008472in}}{\pgfqpoint{4.041128in}{2.016286in}}%
\pgfpathcurveto{\pgfqpoint{4.033315in}{2.024100in}}{\pgfqpoint{4.022716in}{2.028490in}}{\pgfqpoint{4.011666in}{2.028490in}}%
\pgfpathcurveto{\pgfqpoint{4.000616in}{2.028490in}}{\pgfqpoint{3.990016in}{2.024100in}}{\pgfqpoint{3.982203in}{2.016286in}}%
\pgfpathcurveto{\pgfqpoint{3.974389in}{2.008472in}}{\pgfqpoint{3.969999in}{1.997873in}}{\pgfqpoint{3.969999in}{1.986823in}}%
\pgfpathcurveto{\pgfqpoint{3.969999in}{1.975773in}}{\pgfqpoint{3.974389in}{1.965174in}}{\pgfqpoint{3.982203in}{1.957360in}}%
\pgfpathcurveto{\pgfqpoint{3.990016in}{1.949547in}}{\pgfqpoint{4.000616in}{1.945157in}}{\pgfqpoint{4.011666in}{1.945157in}}%
\pgfpathclose%
\pgfusepath{stroke,fill}%
\end{pgfscope}%
\begin{pgfscope}%
\pgfpathrectangle{\pgfqpoint{0.800000in}{0.528000in}}{\pgfqpoint{4.960000in}{3.696000in}}%
\pgfusepath{clip}%
\pgfsetbuttcap%
\pgfsetroundjoin%
\definecolor{currentfill}{rgb}{0.000000,0.000000,0.000000}%
\pgfsetfillcolor{currentfill}%
\pgfsetlinewidth{1.003750pt}%
\definecolor{currentstroke}{rgb}{0.000000,0.000000,0.000000}%
\pgfsetstrokecolor{currentstroke}%
\pgfsetdash{}{0pt}%
\pgfpathmoveto{\pgfqpoint{4.011666in}{1.837669in}}%
\pgfpathcurveto{\pgfqpoint{4.022716in}{1.837669in}}{\pgfqpoint{4.033315in}{1.842059in}}{\pgfqpoint{4.041128in}{1.849873in}}%
\pgfpathcurveto{\pgfqpoint{4.048942in}{1.857687in}}{\pgfqpoint{4.053332in}{1.868286in}}{\pgfqpoint{4.053332in}{1.879336in}}%
\pgfpathcurveto{\pgfqpoint{4.053332in}{1.890386in}}{\pgfqpoint{4.048942in}{1.900985in}}{\pgfqpoint{4.041128in}{1.908798in}}%
\pgfpathcurveto{\pgfqpoint{4.033315in}{1.916612in}}{\pgfqpoint{4.022716in}{1.921002in}}{\pgfqpoint{4.011666in}{1.921002in}}%
\pgfpathcurveto{\pgfqpoint{4.000616in}{1.921002in}}{\pgfqpoint{3.990016in}{1.916612in}}{\pgfqpoint{3.982203in}{1.908798in}}%
\pgfpathcurveto{\pgfqpoint{3.974389in}{1.900985in}}{\pgfqpoint{3.969999in}{1.890386in}}{\pgfqpoint{3.969999in}{1.879336in}}%
\pgfpathcurveto{\pgfqpoint{3.969999in}{1.868286in}}{\pgfqpoint{3.974389in}{1.857687in}}{\pgfqpoint{3.982203in}{1.849873in}}%
\pgfpathcurveto{\pgfqpoint{3.990016in}{1.842059in}}{\pgfqpoint{4.000616in}{1.837669in}}{\pgfqpoint{4.011666in}{1.837669in}}%
\pgfpathclose%
\pgfusepath{stroke,fill}%
\end{pgfscope}%
\begin{pgfscope}%
\pgfpathrectangle{\pgfqpoint{0.800000in}{0.528000in}}{\pgfqpoint{4.960000in}{3.696000in}}%
\pgfusepath{clip}%
\pgfsetbuttcap%
\pgfsetroundjoin%
\definecolor{currentfill}{rgb}{0.000000,0.000000,0.000000}%
\pgfsetfillcolor{currentfill}%
\pgfsetlinewidth{1.003750pt}%
\definecolor{currentstroke}{rgb}{0.000000,0.000000,0.000000}%
\pgfsetstrokecolor{currentstroke}%
\pgfsetdash{}{0pt}%
\pgfpathmoveto{\pgfqpoint{4.011666in}{1.816172in}}%
\pgfpathcurveto{\pgfqpoint{4.022716in}{1.816172in}}{\pgfqpoint{4.033315in}{1.820562in}}{\pgfqpoint{4.041128in}{1.828375in}}%
\pgfpathcurveto{\pgfqpoint{4.048942in}{1.836189in}}{\pgfqpoint{4.053332in}{1.846788in}}{\pgfqpoint{4.053332in}{1.857838in}}%
\pgfpathcurveto{\pgfqpoint{4.053332in}{1.868888in}}{\pgfqpoint{4.048942in}{1.879487in}}{\pgfqpoint{4.041128in}{1.887301in}}%
\pgfpathcurveto{\pgfqpoint{4.033315in}{1.895115in}}{\pgfqpoint{4.022716in}{1.899505in}}{\pgfqpoint{4.011666in}{1.899505in}}%
\pgfpathcurveto{\pgfqpoint{4.000616in}{1.899505in}}{\pgfqpoint{3.990016in}{1.895115in}}{\pgfqpoint{3.982203in}{1.887301in}}%
\pgfpathcurveto{\pgfqpoint{3.974389in}{1.879487in}}{\pgfqpoint{3.969999in}{1.868888in}}{\pgfqpoint{3.969999in}{1.857838in}}%
\pgfpathcurveto{\pgfqpoint{3.969999in}{1.846788in}}{\pgfqpoint{3.974389in}{1.836189in}}{\pgfqpoint{3.982203in}{1.828375in}}%
\pgfpathcurveto{\pgfqpoint{3.990016in}{1.820562in}}{\pgfqpoint{4.000616in}{1.816172in}}{\pgfqpoint{4.011666in}{1.816172in}}%
\pgfpathclose%
\pgfusepath{stroke,fill}%
\end{pgfscope}%
\begin{pgfscope}%
\pgfpathrectangle{\pgfqpoint{0.800000in}{0.528000in}}{\pgfqpoint{4.960000in}{3.696000in}}%
\pgfusepath{clip}%
\pgfsetbuttcap%
\pgfsetroundjoin%
\definecolor{currentfill}{rgb}{0.000000,0.000000,0.000000}%
\pgfsetfillcolor{currentfill}%
\pgfsetlinewidth{1.003750pt}%
\definecolor{currentstroke}{rgb}{0.000000,0.000000,0.000000}%
\pgfsetstrokecolor{currentstroke}%
\pgfsetdash{}{0pt}%
\pgfpathmoveto{\pgfqpoint{4.011666in}{1.859167in}}%
\pgfpathcurveto{\pgfqpoint{4.022716in}{1.859167in}}{\pgfqpoint{4.033315in}{1.863557in}}{\pgfqpoint{4.041128in}{1.871370in}}%
\pgfpathcurveto{\pgfqpoint{4.048942in}{1.879184in}}{\pgfqpoint{4.053332in}{1.889783in}}{\pgfqpoint{4.053332in}{1.900833in}}%
\pgfpathcurveto{\pgfqpoint{4.053332in}{1.911883in}}{\pgfqpoint{4.048942in}{1.922482in}}{\pgfqpoint{4.041128in}{1.930296in}}%
\pgfpathcurveto{\pgfqpoint{4.033315in}{1.938110in}}{\pgfqpoint{4.022716in}{1.942500in}}{\pgfqpoint{4.011666in}{1.942500in}}%
\pgfpathcurveto{\pgfqpoint{4.000616in}{1.942500in}}{\pgfqpoint{3.990016in}{1.938110in}}{\pgfqpoint{3.982203in}{1.930296in}}%
\pgfpathcurveto{\pgfqpoint{3.974389in}{1.922482in}}{\pgfqpoint{3.969999in}{1.911883in}}{\pgfqpoint{3.969999in}{1.900833in}}%
\pgfpathcurveto{\pgfqpoint{3.969999in}{1.889783in}}{\pgfqpoint{3.974389in}{1.879184in}}{\pgfqpoint{3.982203in}{1.871370in}}%
\pgfpathcurveto{\pgfqpoint{3.990016in}{1.863557in}}{\pgfqpoint{4.000616in}{1.859167in}}{\pgfqpoint{4.011666in}{1.859167in}}%
\pgfpathclose%
\pgfusepath{stroke,fill}%
\end{pgfscope}%
\begin{pgfscope}%
\pgfpathrectangle{\pgfqpoint{0.800000in}{0.528000in}}{\pgfqpoint{4.960000in}{3.696000in}}%
\pgfusepath{clip}%
\pgfsetbuttcap%
\pgfsetroundjoin%
\definecolor{currentfill}{rgb}{0.000000,0.000000,0.000000}%
\pgfsetfillcolor{currentfill}%
\pgfsetlinewidth{1.003750pt}%
\definecolor{currentstroke}{rgb}{0.000000,0.000000,0.000000}%
\pgfsetstrokecolor{currentstroke}%
\pgfsetdash{}{0pt}%
\pgfpathmoveto{\pgfqpoint{4.011666in}{1.837669in}}%
\pgfpathcurveto{\pgfqpoint{4.022716in}{1.837669in}}{\pgfqpoint{4.033315in}{1.842059in}}{\pgfqpoint{4.041128in}{1.849873in}}%
\pgfpathcurveto{\pgfqpoint{4.048942in}{1.857687in}}{\pgfqpoint{4.053332in}{1.868286in}}{\pgfqpoint{4.053332in}{1.879336in}}%
\pgfpathcurveto{\pgfqpoint{4.053332in}{1.890386in}}{\pgfqpoint{4.048942in}{1.900985in}}{\pgfqpoint{4.041128in}{1.908798in}}%
\pgfpathcurveto{\pgfqpoint{4.033315in}{1.916612in}}{\pgfqpoint{4.022716in}{1.921002in}}{\pgfqpoint{4.011666in}{1.921002in}}%
\pgfpathcurveto{\pgfqpoint{4.000616in}{1.921002in}}{\pgfqpoint{3.990016in}{1.916612in}}{\pgfqpoint{3.982203in}{1.908798in}}%
\pgfpathcurveto{\pgfqpoint{3.974389in}{1.900985in}}{\pgfqpoint{3.969999in}{1.890386in}}{\pgfqpoint{3.969999in}{1.879336in}}%
\pgfpathcurveto{\pgfqpoint{3.969999in}{1.868286in}}{\pgfqpoint{3.974389in}{1.857687in}}{\pgfqpoint{3.982203in}{1.849873in}}%
\pgfpathcurveto{\pgfqpoint{3.990016in}{1.842059in}}{\pgfqpoint{4.000616in}{1.837669in}}{\pgfqpoint{4.011666in}{1.837669in}}%
\pgfpathclose%
\pgfusepath{stroke,fill}%
\end{pgfscope}%
\begin{pgfscope}%
\pgfpathrectangle{\pgfqpoint{0.800000in}{0.528000in}}{\pgfqpoint{4.960000in}{3.696000in}}%
\pgfusepath{clip}%
\pgfsetbuttcap%
\pgfsetroundjoin%
\definecolor{currentfill}{rgb}{0.000000,0.000000,0.000000}%
\pgfsetfillcolor{currentfill}%
\pgfsetlinewidth{1.003750pt}%
\definecolor{currentstroke}{rgb}{0.000000,0.000000,0.000000}%
\pgfsetstrokecolor{currentstroke}%
\pgfsetdash{}{0pt}%
\pgfpathmoveto{\pgfqpoint{4.011666in}{2.009649in}}%
\pgfpathcurveto{\pgfqpoint{4.022716in}{2.009649in}}{\pgfqpoint{4.033315in}{2.014039in}}{\pgfqpoint{4.041128in}{2.021853in}}%
\pgfpathcurveto{\pgfqpoint{4.048942in}{2.029666in}}{\pgfqpoint{4.053332in}{2.040266in}}{\pgfqpoint{4.053332in}{2.051316in}}%
\pgfpathcurveto{\pgfqpoint{4.053332in}{2.062366in}}{\pgfqpoint{4.048942in}{2.072965in}}{\pgfqpoint{4.041128in}{2.080778in}}%
\pgfpathcurveto{\pgfqpoint{4.033315in}{2.088592in}}{\pgfqpoint{4.022716in}{2.092982in}}{\pgfqpoint{4.011666in}{2.092982in}}%
\pgfpathcurveto{\pgfqpoint{4.000616in}{2.092982in}}{\pgfqpoint{3.990016in}{2.088592in}}{\pgfqpoint{3.982203in}{2.080778in}}%
\pgfpathcurveto{\pgfqpoint{3.974389in}{2.072965in}}{\pgfqpoint{3.969999in}{2.062366in}}{\pgfqpoint{3.969999in}{2.051316in}}%
\pgfpathcurveto{\pgfqpoint{3.969999in}{2.040266in}}{\pgfqpoint{3.974389in}{2.029666in}}{\pgfqpoint{3.982203in}{2.021853in}}%
\pgfpathcurveto{\pgfqpoint{3.990016in}{2.014039in}}{\pgfqpoint{4.000616in}{2.009649in}}{\pgfqpoint{4.011666in}{2.009649in}}%
\pgfpathclose%
\pgfusepath{stroke,fill}%
\end{pgfscope}%
\begin{pgfscope}%
\pgfpathrectangle{\pgfqpoint{0.800000in}{0.528000in}}{\pgfqpoint{4.960000in}{3.696000in}}%
\pgfusepath{clip}%
\pgfsetbuttcap%
\pgfsetroundjoin%
\definecolor{currentfill}{rgb}{0.000000,0.000000,0.000000}%
\pgfsetfillcolor{currentfill}%
\pgfsetlinewidth{1.003750pt}%
\definecolor{currentstroke}{rgb}{0.000000,0.000000,0.000000}%
\pgfsetstrokecolor{currentstroke}%
\pgfsetdash{}{0pt}%
\pgfpathmoveto{\pgfqpoint{4.011666in}{1.837669in}}%
\pgfpathcurveto{\pgfqpoint{4.022716in}{1.837669in}}{\pgfqpoint{4.033315in}{1.842059in}}{\pgfqpoint{4.041128in}{1.849873in}}%
\pgfpathcurveto{\pgfqpoint{4.048942in}{1.857687in}}{\pgfqpoint{4.053332in}{1.868286in}}{\pgfqpoint{4.053332in}{1.879336in}}%
\pgfpathcurveto{\pgfqpoint{4.053332in}{1.890386in}}{\pgfqpoint{4.048942in}{1.900985in}}{\pgfqpoint{4.041128in}{1.908798in}}%
\pgfpathcurveto{\pgfqpoint{4.033315in}{1.916612in}}{\pgfqpoint{4.022716in}{1.921002in}}{\pgfqpoint{4.011666in}{1.921002in}}%
\pgfpathcurveto{\pgfqpoint{4.000616in}{1.921002in}}{\pgfqpoint{3.990016in}{1.916612in}}{\pgfqpoint{3.982203in}{1.908798in}}%
\pgfpathcurveto{\pgfqpoint{3.974389in}{1.900985in}}{\pgfqpoint{3.969999in}{1.890386in}}{\pgfqpoint{3.969999in}{1.879336in}}%
\pgfpathcurveto{\pgfqpoint{3.969999in}{1.868286in}}{\pgfqpoint{3.974389in}{1.857687in}}{\pgfqpoint{3.982203in}{1.849873in}}%
\pgfpathcurveto{\pgfqpoint{3.990016in}{1.842059in}}{\pgfqpoint{4.000616in}{1.837669in}}{\pgfqpoint{4.011666in}{1.837669in}}%
\pgfpathclose%
\pgfusepath{stroke,fill}%
\end{pgfscope}%
\begin{pgfscope}%
\pgfpathrectangle{\pgfqpoint{0.800000in}{0.528000in}}{\pgfqpoint{4.960000in}{3.696000in}}%
\pgfusepath{clip}%
\pgfsetbuttcap%
\pgfsetroundjoin%
\definecolor{currentfill}{rgb}{0.000000,0.000000,0.000000}%
\pgfsetfillcolor{currentfill}%
\pgfsetlinewidth{1.003750pt}%
\definecolor{currentstroke}{rgb}{0.000000,0.000000,0.000000}%
\pgfsetstrokecolor{currentstroke}%
\pgfsetdash{}{0pt}%
\pgfpathmoveto{\pgfqpoint{4.011666in}{2.031146in}}%
\pgfpathcurveto{\pgfqpoint{4.022716in}{2.031146in}}{\pgfqpoint{4.033315in}{2.035537in}}{\pgfqpoint{4.041128in}{2.043350in}}%
\pgfpathcurveto{\pgfqpoint{4.048942in}{2.051164in}}{\pgfqpoint{4.053332in}{2.061763in}}{\pgfqpoint{4.053332in}{2.072813in}}%
\pgfpathcurveto{\pgfqpoint{4.053332in}{2.083863in}}{\pgfqpoint{4.048942in}{2.094462in}}{\pgfqpoint{4.041128in}{2.102276in}}%
\pgfpathcurveto{\pgfqpoint{4.033315in}{2.110090in}}{\pgfqpoint{4.022716in}{2.114480in}}{\pgfqpoint{4.011666in}{2.114480in}}%
\pgfpathcurveto{\pgfqpoint{4.000616in}{2.114480in}}{\pgfqpoint{3.990016in}{2.110090in}}{\pgfqpoint{3.982203in}{2.102276in}}%
\pgfpathcurveto{\pgfqpoint{3.974389in}{2.094462in}}{\pgfqpoint{3.969999in}{2.083863in}}{\pgfqpoint{3.969999in}{2.072813in}}%
\pgfpathcurveto{\pgfqpoint{3.969999in}{2.061763in}}{\pgfqpoint{3.974389in}{2.051164in}}{\pgfqpoint{3.982203in}{2.043350in}}%
\pgfpathcurveto{\pgfqpoint{3.990016in}{2.035537in}}{\pgfqpoint{4.000616in}{2.031146in}}{\pgfqpoint{4.011666in}{2.031146in}}%
\pgfpathclose%
\pgfusepath{stroke,fill}%
\end{pgfscope}%
\begin{pgfscope}%
\pgfpathrectangle{\pgfqpoint{0.800000in}{0.528000in}}{\pgfqpoint{4.960000in}{3.696000in}}%
\pgfusepath{clip}%
\pgfsetbuttcap%
\pgfsetroundjoin%
\definecolor{currentfill}{rgb}{0.000000,0.000000,0.000000}%
\pgfsetfillcolor{currentfill}%
\pgfsetlinewidth{1.003750pt}%
\definecolor{currentstroke}{rgb}{0.000000,0.000000,0.000000}%
\pgfsetstrokecolor{currentstroke}%
\pgfsetdash{}{0pt}%
\pgfpathmoveto{\pgfqpoint{4.011666in}{1.988151in}}%
\pgfpathcurveto{\pgfqpoint{4.022716in}{1.988151in}}{\pgfqpoint{4.033315in}{1.992542in}}{\pgfqpoint{4.041128in}{2.000355in}}%
\pgfpathcurveto{\pgfqpoint{4.048942in}{2.008169in}}{\pgfqpoint{4.053332in}{2.018768in}}{\pgfqpoint{4.053332in}{2.029818in}}%
\pgfpathcurveto{\pgfqpoint{4.053332in}{2.040868in}}{\pgfqpoint{4.048942in}{2.051467in}}{\pgfqpoint{4.041128in}{2.059281in}}%
\pgfpathcurveto{\pgfqpoint{4.033315in}{2.067095in}}{\pgfqpoint{4.022716in}{2.071485in}}{\pgfqpoint{4.011666in}{2.071485in}}%
\pgfpathcurveto{\pgfqpoint{4.000616in}{2.071485in}}{\pgfqpoint{3.990016in}{2.067095in}}{\pgfqpoint{3.982203in}{2.059281in}}%
\pgfpathcurveto{\pgfqpoint{3.974389in}{2.051467in}}{\pgfqpoint{3.969999in}{2.040868in}}{\pgfqpoint{3.969999in}{2.029818in}}%
\pgfpathcurveto{\pgfqpoint{3.969999in}{2.018768in}}{\pgfqpoint{3.974389in}{2.008169in}}{\pgfqpoint{3.982203in}{2.000355in}}%
\pgfpathcurveto{\pgfqpoint{3.990016in}{1.992542in}}{\pgfqpoint{4.000616in}{1.988151in}}{\pgfqpoint{4.011666in}{1.988151in}}%
\pgfpathclose%
\pgfusepath{stroke,fill}%
\end{pgfscope}%
\begin{pgfscope}%
\pgfpathrectangle{\pgfqpoint{0.800000in}{0.528000in}}{\pgfqpoint{4.960000in}{3.696000in}}%
\pgfusepath{clip}%
\pgfsetbuttcap%
\pgfsetroundjoin%
\definecolor{currentfill}{rgb}{0.000000,0.000000,0.000000}%
\pgfsetfillcolor{currentfill}%
\pgfsetlinewidth{1.003750pt}%
\definecolor{currentstroke}{rgb}{0.000000,0.000000,0.000000}%
\pgfsetstrokecolor{currentstroke}%
\pgfsetdash{}{0pt}%
\pgfpathmoveto{\pgfqpoint{4.011666in}{1.902162in}}%
\pgfpathcurveto{\pgfqpoint{4.022716in}{1.902162in}}{\pgfqpoint{4.033315in}{1.906552in}}{\pgfqpoint{4.041128in}{1.914365in}}%
\pgfpathcurveto{\pgfqpoint{4.048942in}{1.922179in}}{\pgfqpoint{4.053332in}{1.932778in}}{\pgfqpoint{4.053332in}{1.943828in}}%
\pgfpathcurveto{\pgfqpoint{4.053332in}{1.954878in}}{\pgfqpoint{4.048942in}{1.965477in}}{\pgfqpoint{4.041128in}{1.973291in}}%
\pgfpathcurveto{\pgfqpoint{4.033315in}{1.981105in}}{\pgfqpoint{4.022716in}{1.985495in}}{\pgfqpoint{4.011666in}{1.985495in}}%
\pgfpathcurveto{\pgfqpoint{4.000616in}{1.985495in}}{\pgfqpoint{3.990016in}{1.981105in}}{\pgfqpoint{3.982203in}{1.973291in}}%
\pgfpathcurveto{\pgfqpoint{3.974389in}{1.965477in}}{\pgfqpoint{3.969999in}{1.954878in}}{\pgfqpoint{3.969999in}{1.943828in}}%
\pgfpathcurveto{\pgfqpoint{3.969999in}{1.932778in}}{\pgfqpoint{3.974389in}{1.922179in}}{\pgfqpoint{3.982203in}{1.914365in}}%
\pgfpathcurveto{\pgfqpoint{3.990016in}{1.906552in}}{\pgfqpoint{4.000616in}{1.902162in}}{\pgfqpoint{4.011666in}{1.902162in}}%
\pgfpathclose%
\pgfusepath{stroke,fill}%
\end{pgfscope}%
\begin{pgfscope}%
\pgfpathrectangle{\pgfqpoint{0.800000in}{0.528000in}}{\pgfqpoint{4.960000in}{3.696000in}}%
\pgfusepath{clip}%
\pgfsetbuttcap%
\pgfsetroundjoin%
\definecolor{currentfill}{rgb}{0.000000,0.000000,0.000000}%
\pgfsetfillcolor{currentfill}%
\pgfsetlinewidth{1.003750pt}%
\definecolor{currentstroke}{rgb}{0.000000,0.000000,0.000000}%
\pgfsetstrokecolor{currentstroke}%
\pgfsetdash{}{0pt}%
\pgfpathmoveto{\pgfqpoint{4.011666in}{1.773177in}}%
\pgfpathcurveto{\pgfqpoint{4.022716in}{1.773177in}}{\pgfqpoint{4.033315in}{1.777567in}}{\pgfqpoint{4.041128in}{1.785380in}}%
\pgfpathcurveto{\pgfqpoint{4.048942in}{1.793194in}}{\pgfqpoint{4.053332in}{1.803793in}}{\pgfqpoint{4.053332in}{1.814843in}}%
\pgfpathcurveto{\pgfqpoint{4.053332in}{1.825893in}}{\pgfqpoint{4.048942in}{1.836492in}}{\pgfqpoint{4.041128in}{1.844306in}}%
\pgfpathcurveto{\pgfqpoint{4.033315in}{1.852120in}}{\pgfqpoint{4.022716in}{1.856510in}}{\pgfqpoint{4.011666in}{1.856510in}}%
\pgfpathcurveto{\pgfqpoint{4.000616in}{1.856510in}}{\pgfqpoint{3.990016in}{1.852120in}}{\pgfqpoint{3.982203in}{1.844306in}}%
\pgfpathcurveto{\pgfqpoint{3.974389in}{1.836492in}}{\pgfqpoint{3.969999in}{1.825893in}}{\pgfqpoint{3.969999in}{1.814843in}}%
\pgfpathcurveto{\pgfqpoint{3.969999in}{1.803793in}}{\pgfqpoint{3.974389in}{1.793194in}}{\pgfqpoint{3.982203in}{1.785380in}}%
\pgfpathcurveto{\pgfqpoint{3.990016in}{1.777567in}}{\pgfqpoint{4.000616in}{1.773177in}}{\pgfqpoint{4.011666in}{1.773177in}}%
\pgfpathclose%
\pgfusepath{stroke,fill}%
\end{pgfscope}%
\begin{pgfscope}%
\pgfpathrectangle{\pgfqpoint{0.800000in}{0.528000in}}{\pgfqpoint{4.960000in}{3.696000in}}%
\pgfusepath{clip}%
\pgfsetbuttcap%
\pgfsetroundjoin%
\definecolor{currentfill}{rgb}{0.000000,0.000000,0.000000}%
\pgfsetfillcolor{currentfill}%
\pgfsetlinewidth{1.003750pt}%
\definecolor{currentstroke}{rgb}{0.000000,0.000000,0.000000}%
\pgfsetstrokecolor{currentstroke}%
\pgfsetdash{}{0pt}%
\pgfpathmoveto{\pgfqpoint{4.011666in}{1.837669in}}%
\pgfpathcurveto{\pgfqpoint{4.022716in}{1.837669in}}{\pgfqpoint{4.033315in}{1.842059in}}{\pgfqpoint{4.041128in}{1.849873in}}%
\pgfpathcurveto{\pgfqpoint{4.048942in}{1.857687in}}{\pgfqpoint{4.053332in}{1.868286in}}{\pgfqpoint{4.053332in}{1.879336in}}%
\pgfpathcurveto{\pgfqpoint{4.053332in}{1.890386in}}{\pgfqpoint{4.048942in}{1.900985in}}{\pgfqpoint{4.041128in}{1.908798in}}%
\pgfpathcurveto{\pgfqpoint{4.033315in}{1.916612in}}{\pgfqpoint{4.022716in}{1.921002in}}{\pgfqpoint{4.011666in}{1.921002in}}%
\pgfpathcurveto{\pgfqpoint{4.000616in}{1.921002in}}{\pgfqpoint{3.990016in}{1.916612in}}{\pgfqpoint{3.982203in}{1.908798in}}%
\pgfpathcurveto{\pgfqpoint{3.974389in}{1.900985in}}{\pgfqpoint{3.969999in}{1.890386in}}{\pgfqpoint{3.969999in}{1.879336in}}%
\pgfpathcurveto{\pgfqpoint{3.969999in}{1.868286in}}{\pgfqpoint{3.974389in}{1.857687in}}{\pgfqpoint{3.982203in}{1.849873in}}%
\pgfpathcurveto{\pgfqpoint{3.990016in}{1.842059in}}{\pgfqpoint{4.000616in}{1.837669in}}{\pgfqpoint{4.011666in}{1.837669in}}%
\pgfpathclose%
\pgfusepath{stroke,fill}%
\end{pgfscope}%
\begin{pgfscope}%
\pgfpathrectangle{\pgfqpoint{0.800000in}{0.528000in}}{\pgfqpoint{4.960000in}{3.696000in}}%
\pgfusepath{clip}%
\pgfsetbuttcap%
\pgfsetroundjoin%
\definecolor{currentfill}{rgb}{0.000000,0.000000,0.000000}%
\pgfsetfillcolor{currentfill}%
\pgfsetlinewidth{1.003750pt}%
\definecolor{currentstroke}{rgb}{0.000000,0.000000,0.000000}%
\pgfsetstrokecolor{currentstroke}%
\pgfsetdash{}{0pt}%
\pgfpathmoveto{\pgfqpoint{4.011666in}{1.902162in}}%
\pgfpathcurveto{\pgfqpoint{4.022716in}{1.902162in}}{\pgfqpoint{4.033315in}{1.906552in}}{\pgfqpoint{4.041128in}{1.914365in}}%
\pgfpathcurveto{\pgfqpoint{4.048942in}{1.922179in}}{\pgfqpoint{4.053332in}{1.932778in}}{\pgfqpoint{4.053332in}{1.943828in}}%
\pgfpathcurveto{\pgfqpoint{4.053332in}{1.954878in}}{\pgfqpoint{4.048942in}{1.965477in}}{\pgfqpoint{4.041128in}{1.973291in}}%
\pgfpathcurveto{\pgfqpoint{4.033315in}{1.981105in}}{\pgfqpoint{4.022716in}{1.985495in}}{\pgfqpoint{4.011666in}{1.985495in}}%
\pgfpathcurveto{\pgfqpoint{4.000616in}{1.985495in}}{\pgfqpoint{3.990016in}{1.981105in}}{\pgfqpoint{3.982203in}{1.973291in}}%
\pgfpathcurveto{\pgfqpoint{3.974389in}{1.965477in}}{\pgfqpoint{3.969999in}{1.954878in}}{\pgfqpoint{3.969999in}{1.943828in}}%
\pgfpathcurveto{\pgfqpoint{3.969999in}{1.932778in}}{\pgfqpoint{3.974389in}{1.922179in}}{\pgfqpoint{3.982203in}{1.914365in}}%
\pgfpathcurveto{\pgfqpoint{3.990016in}{1.906552in}}{\pgfqpoint{4.000616in}{1.902162in}}{\pgfqpoint{4.011666in}{1.902162in}}%
\pgfpathclose%
\pgfusepath{stroke,fill}%
\end{pgfscope}%
\begin{pgfscope}%
\pgfpathrectangle{\pgfqpoint{0.800000in}{0.528000in}}{\pgfqpoint{4.960000in}{3.696000in}}%
\pgfusepath{clip}%
\pgfsetbuttcap%
\pgfsetroundjoin%
\definecolor{currentfill}{rgb}{0.000000,0.000000,0.000000}%
\pgfsetfillcolor{currentfill}%
\pgfsetlinewidth{1.003750pt}%
\definecolor{currentstroke}{rgb}{0.000000,0.000000,0.000000}%
\pgfsetstrokecolor{currentstroke}%
\pgfsetdash{}{0pt}%
\pgfpathmoveto{\pgfqpoint{4.011666in}{1.923659in}}%
\pgfpathcurveto{\pgfqpoint{4.022716in}{1.923659in}}{\pgfqpoint{4.033315in}{1.928049in}}{\pgfqpoint{4.041128in}{1.935863in}}%
\pgfpathcurveto{\pgfqpoint{4.048942in}{1.943677in}}{\pgfqpoint{4.053332in}{1.954276in}}{\pgfqpoint{4.053332in}{1.965326in}}%
\pgfpathcurveto{\pgfqpoint{4.053332in}{1.976376in}}{\pgfqpoint{4.048942in}{1.986975in}}{\pgfqpoint{4.041128in}{1.994788in}}%
\pgfpathcurveto{\pgfqpoint{4.033315in}{2.002602in}}{\pgfqpoint{4.022716in}{2.006992in}}{\pgfqpoint{4.011666in}{2.006992in}}%
\pgfpathcurveto{\pgfqpoint{4.000616in}{2.006992in}}{\pgfqpoint{3.990016in}{2.002602in}}{\pgfqpoint{3.982203in}{1.994788in}}%
\pgfpathcurveto{\pgfqpoint{3.974389in}{1.986975in}}{\pgfqpoint{3.969999in}{1.976376in}}{\pgfqpoint{3.969999in}{1.965326in}}%
\pgfpathcurveto{\pgfqpoint{3.969999in}{1.954276in}}{\pgfqpoint{3.974389in}{1.943677in}}{\pgfqpoint{3.982203in}{1.935863in}}%
\pgfpathcurveto{\pgfqpoint{3.990016in}{1.928049in}}{\pgfqpoint{4.000616in}{1.923659in}}{\pgfqpoint{4.011666in}{1.923659in}}%
\pgfpathclose%
\pgfusepath{stroke,fill}%
\end{pgfscope}%
\begin{pgfscope}%
\pgfpathrectangle{\pgfqpoint{0.800000in}{0.528000in}}{\pgfqpoint{4.960000in}{3.696000in}}%
\pgfusepath{clip}%
\pgfsetbuttcap%
\pgfsetroundjoin%
\definecolor{currentfill}{rgb}{0.000000,0.000000,0.000000}%
\pgfsetfillcolor{currentfill}%
\pgfsetlinewidth{1.003750pt}%
\definecolor{currentstroke}{rgb}{0.000000,0.000000,0.000000}%
\pgfsetstrokecolor{currentstroke}%
\pgfsetdash{}{0pt}%
\pgfpathmoveto{\pgfqpoint{4.011666in}{1.988151in}}%
\pgfpathcurveto{\pgfqpoint{4.022716in}{1.988151in}}{\pgfqpoint{4.033315in}{1.992542in}}{\pgfqpoint{4.041128in}{2.000355in}}%
\pgfpathcurveto{\pgfqpoint{4.048942in}{2.008169in}}{\pgfqpoint{4.053332in}{2.018768in}}{\pgfqpoint{4.053332in}{2.029818in}}%
\pgfpathcurveto{\pgfqpoint{4.053332in}{2.040868in}}{\pgfqpoint{4.048942in}{2.051467in}}{\pgfqpoint{4.041128in}{2.059281in}}%
\pgfpathcurveto{\pgfqpoint{4.033315in}{2.067095in}}{\pgfqpoint{4.022716in}{2.071485in}}{\pgfqpoint{4.011666in}{2.071485in}}%
\pgfpathcurveto{\pgfqpoint{4.000616in}{2.071485in}}{\pgfqpoint{3.990016in}{2.067095in}}{\pgfqpoint{3.982203in}{2.059281in}}%
\pgfpathcurveto{\pgfqpoint{3.974389in}{2.051467in}}{\pgfqpoint{3.969999in}{2.040868in}}{\pgfqpoint{3.969999in}{2.029818in}}%
\pgfpathcurveto{\pgfqpoint{3.969999in}{2.018768in}}{\pgfqpoint{3.974389in}{2.008169in}}{\pgfqpoint{3.982203in}{2.000355in}}%
\pgfpathcurveto{\pgfqpoint{3.990016in}{1.992542in}}{\pgfqpoint{4.000616in}{1.988151in}}{\pgfqpoint{4.011666in}{1.988151in}}%
\pgfpathclose%
\pgfusepath{stroke,fill}%
\end{pgfscope}%
\begin{pgfscope}%
\pgfpathrectangle{\pgfqpoint{0.800000in}{0.528000in}}{\pgfqpoint{4.960000in}{3.696000in}}%
\pgfusepath{clip}%
\pgfsetbuttcap%
\pgfsetroundjoin%
\definecolor{currentfill}{rgb}{0.000000,0.000000,0.000000}%
\pgfsetfillcolor{currentfill}%
\pgfsetlinewidth{1.003750pt}%
\definecolor{currentstroke}{rgb}{0.000000,0.000000,0.000000}%
\pgfsetstrokecolor{currentstroke}%
\pgfsetdash{}{0pt}%
\pgfpathmoveto{\pgfqpoint{4.011666in}{1.945157in}}%
\pgfpathcurveto{\pgfqpoint{4.022716in}{1.945157in}}{\pgfqpoint{4.033315in}{1.949547in}}{\pgfqpoint{4.041128in}{1.957360in}}%
\pgfpathcurveto{\pgfqpoint{4.048942in}{1.965174in}}{\pgfqpoint{4.053332in}{1.975773in}}{\pgfqpoint{4.053332in}{1.986823in}}%
\pgfpathcurveto{\pgfqpoint{4.053332in}{1.997873in}}{\pgfqpoint{4.048942in}{2.008472in}}{\pgfqpoint{4.041128in}{2.016286in}}%
\pgfpathcurveto{\pgfqpoint{4.033315in}{2.024100in}}{\pgfqpoint{4.022716in}{2.028490in}}{\pgfqpoint{4.011666in}{2.028490in}}%
\pgfpathcurveto{\pgfqpoint{4.000616in}{2.028490in}}{\pgfqpoint{3.990016in}{2.024100in}}{\pgfqpoint{3.982203in}{2.016286in}}%
\pgfpathcurveto{\pgfqpoint{3.974389in}{2.008472in}}{\pgfqpoint{3.969999in}{1.997873in}}{\pgfqpoint{3.969999in}{1.986823in}}%
\pgfpathcurveto{\pgfqpoint{3.969999in}{1.975773in}}{\pgfqpoint{3.974389in}{1.965174in}}{\pgfqpoint{3.982203in}{1.957360in}}%
\pgfpathcurveto{\pgfqpoint{3.990016in}{1.949547in}}{\pgfqpoint{4.000616in}{1.945157in}}{\pgfqpoint{4.011666in}{1.945157in}}%
\pgfpathclose%
\pgfusepath{stroke,fill}%
\end{pgfscope}%
\begin{pgfscope}%
\pgfpathrectangle{\pgfqpoint{0.800000in}{0.528000in}}{\pgfqpoint{4.960000in}{3.696000in}}%
\pgfusepath{clip}%
\pgfsetbuttcap%
\pgfsetroundjoin%
\definecolor{currentfill}{rgb}{0.000000,0.000000,0.000000}%
\pgfsetfillcolor{currentfill}%
\pgfsetlinewidth{1.003750pt}%
\definecolor{currentstroke}{rgb}{0.000000,0.000000,0.000000}%
\pgfsetstrokecolor{currentstroke}%
\pgfsetdash{}{0pt}%
\pgfpathmoveto{\pgfqpoint{4.011666in}{1.880664in}}%
\pgfpathcurveto{\pgfqpoint{4.022716in}{1.880664in}}{\pgfqpoint{4.033315in}{1.885054in}}{\pgfqpoint{4.041128in}{1.892868in}}%
\pgfpathcurveto{\pgfqpoint{4.048942in}{1.900682in}}{\pgfqpoint{4.053332in}{1.911281in}}{\pgfqpoint{4.053332in}{1.922331in}}%
\pgfpathcurveto{\pgfqpoint{4.053332in}{1.933381in}}{\pgfqpoint{4.048942in}{1.943980in}}{\pgfqpoint{4.041128in}{1.951793in}}%
\pgfpathcurveto{\pgfqpoint{4.033315in}{1.959607in}}{\pgfqpoint{4.022716in}{1.963997in}}{\pgfqpoint{4.011666in}{1.963997in}}%
\pgfpathcurveto{\pgfqpoint{4.000616in}{1.963997in}}{\pgfqpoint{3.990016in}{1.959607in}}{\pgfqpoint{3.982203in}{1.951793in}}%
\pgfpathcurveto{\pgfqpoint{3.974389in}{1.943980in}}{\pgfqpoint{3.969999in}{1.933381in}}{\pgfqpoint{3.969999in}{1.922331in}}%
\pgfpathcurveto{\pgfqpoint{3.969999in}{1.911281in}}{\pgfqpoint{3.974389in}{1.900682in}}{\pgfqpoint{3.982203in}{1.892868in}}%
\pgfpathcurveto{\pgfqpoint{3.990016in}{1.885054in}}{\pgfqpoint{4.000616in}{1.880664in}}{\pgfqpoint{4.011666in}{1.880664in}}%
\pgfpathclose%
\pgfusepath{stroke,fill}%
\end{pgfscope}%
\begin{pgfscope}%
\pgfpathrectangle{\pgfqpoint{0.800000in}{0.528000in}}{\pgfqpoint{4.960000in}{3.696000in}}%
\pgfusepath{clip}%
\pgfsetbuttcap%
\pgfsetroundjoin%
\definecolor{currentfill}{rgb}{0.000000,0.000000,0.000000}%
\pgfsetfillcolor{currentfill}%
\pgfsetlinewidth{1.003750pt}%
\definecolor{currentstroke}{rgb}{0.000000,0.000000,0.000000}%
\pgfsetstrokecolor{currentstroke}%
\pgfsetdash{}{0pt}%
\pgfpathmoveto{\pgfqpoint{4.011666in}{1.923659in}}%
\pgfpathcurveto{\pgfqpoint{4.022716in}{1.923659in}}{\pgfqpoint{4.033315in}{1.928049in}}{\pgfqpoint{4.041128in}{1.935863in}}%
\pgfpathcurveto{\pgfqpoint{4.048942in}{1.943677in}}{\pgfqpoint{4.053332in}{1.954276in}}{\pgfqpoint{4.053332in}{1.965326in}}%
\pgfpathcurveto{\pgfqpoint{4.053332in}{1.976376in}}{\pgfqpoint{4.048942in}{1.986975in}}{\pgfqpoint{4.041128in}{1.994788in}}%
\pgfpathcurveto{\pgfqpoint{4.033315in}{2.002602in}}{\pgfqpoint{4.022716in}{2.006992in}}{\pgfqpoint{4.011666in}{2.006992in}}%
\pgfpathcurveto{\pgfqpoint{4.000616in}{2.006992in}}{\pgfqpoint{3.990016in}{2.002602in}}{\pgfqpoint{3.982203in}{1.994788in}}%
\pgfpathcurveto{\pgfqpoint{3.974389in}{1.986975in}}{\pgfqpoint{3.969999in}{1.976376in}}{\pgfqpoint{3.969999in}{1.965326in}}%
\pgfpathcurveto{\pgfqpoint{3.969999in}{1.954276in}}{\pgfqpoint{3.974389in}{1.943677in}}{\pgfqpoint{3.982203in}{1.935863in}}%
\pgfpathcurveto{\pgfqpoint{3.990016in}{1.928049in}}{\pgfqpoint{4.000616in}{1.923659in}}{\pgfqpoint{4.011666in}{1.923659in}}%
\pgfpathclose%
\pgfusepath{stroke,fill}%
\end{pgfscope}%
\begin{pgfscope}%
\pgfpathrectangle{\pgfqpoint{0.800000in}{0.528000in}}{\pgfqpoint{4.960000in}{3.696000in}}%
\pgfusepath{clip}%
\pgfsetbuttcap%
\pgfsetroundjoin%
\definecolor{currentfill}{rgb}{0.000000,0.000000,0.000000}%
\pgfsetfillcolor{currentfill}%
\pgfsetlinewidth{1.003750pt}%
\definecolor{currentstroke}{rgb}{0.000000,0.000000,0.000000}%
\pgfsetstrokecolor{currentstroke}%
\pgfsetdash{}{0pt}%
\pgfpathmoveto{\pgfqpoint{4.011666in}{2.031146in}}%
\pgfpathcurveto{\pgfqpoint{4.022716in}{2.031146in}}{\pgfqpoint{4.033315in}{2.035537in}}{\pgfqpoint{4.041128in}{2.043350in}}%
\pgfpathcurveto{\pgfqpoint{4.048942in}{2.051164in}}{\pgfqpoint{4.053332in}{2.061763in}}{\pgfqpoint{4.053332in}{2.072813in}}%
\pgfpathcurveto{\pgfqpoint{4.053332in}{2.083863in}}{\pgfqpoint{4.048942in}{2.094462in}}{\pgfqpoint{4.041128in}{2.102276in}}%
\pgfpathcurveto{\pgfqpoint{4.033315in}{2.110090in}}{\pgfqpoint{4.022716in}{2.114480in}}{\pgfqpoint{4.011666in}{2.114480in}}%
\pgfpathcurveto{\pgfqpoint{4.000616in}{2.114480in}}{\pgfqpoint{3.990016in}{2.110090in}}{\pgfqpoint{3.982203in}{2.102276in}}%
\pgfpathcurveto{\pgfqpoint{3.974389in}{2.094462in}}{\pgfqpoint{3.969999in}{2.083863in}}{\pgfqpoint{3.969999in}{2.072813in}}%
\pgfpathcurveto{\pgfqpoint{3.969999in}{2.061763in}}{\pgfqpoint{3.974389in}{2.051164in}}{\pgfqpoint{3.982203in}{2.043350in}}%
\pgfpathcurveto{\pgfqpoint{3.990016in}{2.035537in}}{\pgfqpoint{4.000616in}{2.031146in}}{\pgfqpoint{4.011666in}{2.031146in}}%
\pgfpathclose%
\pgfusepath{stroke,fill}%
\end{pgfscope}%
\begin{pgfscope}%
\pgfpathrectangle{\pgfqpoint{0.800000in}{0.528000in}}{\pgfqpoint{4.960000in}{3.696000in}}%
\pgfusepath{clip}%
\pgfsetbuttcap%
\pgfsetroundjoin%
\definecolor{currentfill}{rgb}{0.000000,0.000000,0.000000}%
\pgfsetfillcolor{currentfill}%
\pgfsetlinewidth{1.003750pt}%
\definecolor{currentstroke}{rgb}{0.000000,0.000000,0.000000}%
\pgfsetstrokecolor{currentstroke}%
\pgfsetdash{}{0pt}%
\pgfpathmoveto{\pgfqpoint{4.011666in}{1.902162in}}%
\pgfpathcurveto{\pgfqpoint{4.022716in}{1.902162in}}{\pgfqpoint{4.033315in}{1.906552in}}{\pgfqpoint{4.041128in}{1.914365in}}%
\pgfpathcurveto{\pgfqpoint{4.048942in}{1.922179in}}{\pgfqpoint{4.053332in}{1.932778in}}{\pgfqpoint{4.053332in}{1.943828in}}%
\pgfpathcurveto{\pgfqpoint{4.053332in}{1.954878in}}{\pgfqpoint{4.048942in}{1.965477in}}{\pgfqpoint{4.041128in}{1.973291in}}%
\pgfpathcurveto{\pgfqpoint{4.033315in}{1.981105in}}{\pgfqpoint{4.022716in}{1.985495in}}{\pgfqpoint{4.011666in}{1.985495in}}%
\pgfpathcurveto{\pgfqpoint{4.000616in}{1.985495in}}{\pgfqpoint{3.990016in}{1.981105in}}{\pgfqpoint{3.982203in}{1.973291in}}%
\pgfpathcurveto{\pgfqpoint{3.974389in}{1.965477in}}{\pgfqpoint{3.969999in}{1.954878in}}{\pgfqpoint{3.969999in}{1.943828in}}%
\pgfpathcurveto{\pgfqpoint{3.969999in}{1.932778in}}{\pgfqpoint{3.974389in}{1.922179in}}{\pgfqpoint{3.982203in}{1.914365in}}%
\pgfpathcurveto{\pgfqpoint{3.990016in}{1.906552in}}{\pgfqpoint{4.000616in}{1.902162in}}{\pgfqpoint{4.011666in}{1.902162in}}%
\pgfpathclose%
\pgfusepath{stroke,fill}%
\end{pgfscope}%
\begin{pgfscope}%
\pgfpathrectangle{\pgfqpoint{0.800000in}{0.528000in}}{\pgfqpoint{4.960000in}{3.696000in}}%
\pgfusepath{clip}%
\pgfsetbuttcap%
\pgfsetroundjoin%
\definecolor{currentfill}{rgb}{0.000000,0.000000,0.000000}%
\pgfsetfillcolor{currentfill}%
\pgfsetlinewidth{1.003750pt}%
\definecolor{currentstroke}{rgb}{0.000000,0.000000,0.000000}%
\pgfsetstrokecolor{currentstroke}%
\pgfsetdash{}{0pt}%
\pgfpathmoveto{\pgfqpoint{4.011666in}{1.859167in}}%
\pgfpathcurveto{\pgfqpoint{4.022716in}{1.859167in}}{\pgfqpoint{4.033315in}{1.863557in}}{\pgfqpoint{4.041128in}{1.871370in}}%
\pgfpathcurveto{\pgfqpoint{4.048942in}{1.879184in}}{\pgfqpoint{4.053332in}{1.889783in}}{\pgfqpoint{4.053332in}{1.900833in}}%
\pgfpathcurveto{\pgfqpoint{4.053332in}{1.911883in}}{\pgfqpoint{4.048942in}{1.922482in}}{\pgfqpoint{4.041128in}{1.930296in}}%
\pgfpathcurveto{\pgfqpoint{4.033315in}{1.938110in}}{\pgfqpoint{4.022716in}{1.942500in}}{\pgfqpoint{4.011666in}{1.942500in}}%
\pgfpathcurveto{\pgfqpoint{4.000616in}{1.942500in}}{\pgfqpoint{3.990016in}{1.938110in}}{\pgfqpoint{3.982203in}{1.930296in}}%
\pgfpathcurveto{\pgfqpoint{3.974389in}{1.922482in}}{\pgfqpoint{3.969999in}{1.911883in}}{\pgfqpoint{3.969999in}{1.900833in}}%
\pgfpathcurveto{\pgfqpoint{3.969999in}{1.889783in}}{\pgfqpoint{3.974389in}{1.879184in}}{\pgfqpoint{3.982203in}{1.871370in}}%
\pgfpathcurveto{\pgfqpoint{3.990016in}{1.863557in}}{\pgfqpoint{4.000616in}{1.859167in}}{\pgfqpoint{4.011666in}{1.859167in}}%
\pgfpathclose%
\pgfusepath{stroke,fill}%
\end{pgfscope}%
\begin{pgfscope}%
\pgfpathrectangle{\pgfqpoint{0.800000in}{0.528000in}}{\pgfqpoint{4.960000in}{3.696000in}}%
\pgfusepath{clip}%
\pgfsetbuttcap%
\pgfsetroundjoin%
\definecolor{currentfill}{rgb}{0.000000,0.000000,0.000000}%
\pgfsetfillcolor{currentfill}%
\pgfsetlinewidth{1.003750pt}%
\definecolor{currentstroke}{rgb}{0.000000,0.000000,0.000000}%
\pgfsetstrokecolor{currentstroke}%
\pgfsetdash{}{0pt}%
\pgfpathmoveto{\pgfqpoint{4.011666in}{1.945157in}}%
\pgfpathcurveto{\pgfqpoint{4.022716in}{1.945157in}}{\pgfqpoint{4.033315in}{1.949547in}}{\pgfqpoint{4.041128in}{1.957360in}}%
\pgfpathcurveto{\pgfqpoint{4.048942in}{1.965174in}}{\pgfqpoint{4.053332in}{1.975773in}}{\pgfqpoint{4.053332in}{1.986823in}}%
\pgfpathcurveto{\pgfqpoint{4.053332in}{1.997873in}}{\pgfqpoint{4.048942in}{2.008472in}}{\pgfqpoint{4.041128in}{2.016286in}}%
\pgfpathcurveto{\pgfqpoint{4.033315in}{2.024100in}}{\pgfqpoint{4.022716in}{2.028490in}}{\pgfqpoint{4.011666in}{2.028490in}}%
\pgfpathcurveto{\pgfqpoint{4.000616in}{2.028490in}}{\pgfqpoint{3.990016in}{2.024100in}}{\pgfqpoint{3.982203in}{2.016286in}}%
\pgfpathcurveto{\pgfqpoint{3.974389in}{2.008472in}}{\pgfqpoint{3.969999in}{1.997873in}}{\pgfqpoint{3.969999in}{1.986823in}}%
\pgfpathcurveto{\pgfqpoint{3.969999in}{1.975773in}}{\pgfqpoint{3.974389in}{1.965174in}}{\pgfqpoint{3.982203in}{1.957360in}}%
\pgfpathcurveto{\pgfqpoint{3.990016in}{1.949547in}}{\pgfqpoint{4.000616in}{1.945157in}}{\pgfqpoint{4.011666in}{1.945157in}}%
\pgfpathclose%
\pgfusepath{stroke,fill}%
\end{pgfscope}%
\begin{pgfscope}%
\pgfpathrectangle{\pgfqpoint{0.800000in}{0.528000in}}{\pgfqpoint{4.960000in}{3.696000in}}%
\pgfusepath{clip}%
\pgfsetbuttcap%
\pgfsetroundjoin%
\definecolor{currentfill}{rgb}{0.000000,0.000000,0.000000}%
\pgfsetfillcolor{currentfill}%
\pgfsetlinewidth{1.003750pt}%
\definecolor{currentstroke}{rgb}{0.000000,0.000000,0.000000}%
\pgfsetstrokecolor{currentstroke}%
\pgfsetdash{}{0pt}%
\pgfpathmoveto{\pgfqpoint{4.011666in}{1.902162in}}%
\pgfpathcurveto{\pgfqpoint{4.022716in}{1.902162in}}{\pgfqpoint{4.033315in}{1.906552in}}{\pgfqpoint{4.041128in}{1.914365in}}%
\pgfpathcurveto{\pgfqpoint{4.048942in}{1.922179in}}{\pgfqpoint{4.053332in}{1.932778in}}{\pgfqpoint{4.053332in}{1.943828in}}%
\pgfpathcurveto{\pgfqpoint{4.053332in}{1.954878in}}{\pgfqpoint{4.048942in}{1.965477in}}{\pgfqpoint{4.041128in}{1.973291in}}%
\pgfpathcurveto{\pgfqpoint{4.033315in}{1.981105in}}{\pgfqpoint{4.022716in}{1.985495in}}{\pgfqpoint{4.011666in}{1.985495in}}%
\pgfpathcurveto{\pgfqpoint{4.000616in}{1.985495in}}{\pgfqpoint{3.990016in}{1.981105in}}{\pgfqpoint{3.982203in}{1.973291in}}%
\pgfpathcurveto{\pgfqpoint{3.974389in}{1.965477in}}{\pgfqpoint{3.969999in}{1.954878in}}{\pgfqpoint{3.969999in}{1.943828in}}%
\pgfpathcurveto{\pgfqpoint{3.969999in}{1.932778in}}{\pgfqpoint{3.974389in}{1.922179in}}{\pgfqpoint{3.982203in}{1.914365in}}%
\pgfpathcurveto{\pgfqpoint{3.990016in}{1.906552in}}{\pgfqpoint{4.000616in}{1.902162in}}{\pgfqpoint{4.011666in}{1.902162in}}%
\pgfpathclose%
\pgfusepath{stroke,fill}%
\end{pgfscope}%
\begin{pgfscope}%
\pgfpathrectangle{\pgfqpoint{0.800000in}{0.528000in}}{\pgfqpoint{4.960000in}{3.696000in}}%
\pgfusepath{clip}%
\pgfsetbuttcap%
\pgfsetroundjoin%
\definecolor{currentfill}{rgb}{0.000000,0.000000,0.000000}%
\pgfsetfillcolor{currentfill}%
\pgfsetlinewidth{1.003750pt}%
\definecolor{currentstroke}{rgb}{0.000000,0.000000,0.000000}%
\pgfsetstrokecolor{currentstroke}%
\pgfsetdash{}{0pt}%
\pgfpathmoveto{\pgfqpoint{4.011666in}{1.923659in}}%
\pgfpathcurveto{\pgfqpoint{4.022716in}{1.923659in}}{\pgfqpoint{4.033315in}{1.928049in}}{\pgfqpoint{4.041128in}{1.935863in}}%
\pgfpathcurveto{\pgfqpoint{4.048942in}{1.943677in}}{\pgfqpoint{4.053332in}{1.954276in}}{\pgfqpoint{4.053332in}{1.965326in}}%
\pgfpathcurveto{\pgfqpoint{4.053332in}{1.976376in}}{\pgfqpoint{4.048942in}{1.986975in}}{\pgfqpoint{4.041128in}{1.994788in}}%
\pgfpathcurveto{\pgfqpoint{4.033315in}{2.002602in}}{\pgfqpoint{4.022716in}{2.006992in}}{\pgfqpoint{4.011666in}{2.006992in}}%
\pgfpathcurveto{\pgfqpoint{4.000616in}{2.006992in}}{\pgfqpoint{3.990016in}{2.002602in}}{\pgfqpoint{3.982203in}{1.994788in}}%
\pgfpathcurveto{\pgfqpoint{3.974389in}{1.986975in}}{\pgfqpoint{3.969999in}{1.976376in}}{\pgfqpoint{3.969999in}{1.965326in}}%
\pgfpathcurveto{\pgfqpoint{3.969999in}{1.954276in}}{\pgfqpoint{3.974389in}{1.943677in}}{\pgfqpoint{3.982203in}{1.935863in}}%
\pgfpathcurveto{\pgfqpoint{3.990016in}{1.928049in}}{\pgfqpoint{4.000616in}{1.923659in}}{\pgfqpoint{4.011666in}{1.923659in}}%
\pgfpathclose%
\pgfusepath{stroke,fill}%
\end{pgfscope}%
\begin{pgfscope}%
\pgfpathrectangle{\pgfqpoint{0.800000in}{0.528000in}}{\pgfqpoint{4.960000in}{3.696000in}}%
\pgfusepath{clip}%
\pgfsetbuttcap%
\pgfsetroundjoin%
\definecolor{currentfill}{rgb}{0.000000,0.000000,0.000000}%
\pgfsetfillcolor{currentfill}%
\pgfsetlinewidth{1.003750pt}%
\definecolor{currentstroke}{rgb}{0.000000,0.000000,0.000000}%
\pgfsetstrokecolor{currentstroke}%
\pgfsetdash{}{0pt}%
\pgfpathmoveto{\pgfqpoint{4.011666in}{1.837669in}}%
\pgfpathcurveto{\pgfqpoint{4.022716in}{1.837669in}}{\pgfqpoint{4.033315in}{1.842059in}}{\pgfqpoint{4.041128in}{1.849873in}}%
\pgfpathcurveto{\pgfqpoint{4.048942in}{1.857687in}}{\pgfqpoint{4.053332in}{1.868286in}}{\pgfqpoint{4.053332in}{1.879336in}}%
\pgfpathcurveto{\pgfqpoint{4.053332in}{1.890386in}}{\pgfqpoint{4.048942in}{1.900985in}}{\pgfqpoint{4.041128in}{1.908798in}}%
\pgfpathcurveto{\pgfqpoint{4.033315in}{1.916612in}}{\pgfqpoint{4.022716in}{1.921002in}}{\pgfqpoint{4.011666in}{1.921002in}}%
\pgfpathcurveto{\pgfqpoint{4.000616in}{1.921002in}}{\pgfqpoint{3.990016in}{1.916612in}}{\pgfqpoint{3.982203in}{1.908798in}}%
\pgfpathcurveto{\pgfqpoint{3.974389in}{1.900985in}}{\pgfqpoint{3.969999in}{1.890386in}}{\pgfqpoint{3.969999in}{1.879336in}}%
\pgfpathcurveto{\pgfqpoint{3.969999in}{1.868286in}}{\pgfqpoint{3.974389in}{1.857687in}}{\pgfqpoint{3.982203in}{1.849873in}}%
\pgfpathcurveto{\pgfqpoint{3.990016in}{1.842059in}}{\pgfqpoint{4.000616in}{1.837669in}}{\pgfqpoint{4.011666in}{1.837669in}}%
\pgfpathclose%
\pgfusepath{stroke,fill}%
\end{pgfscope}%
\begin{pgfscope}%
\pgfpathrectangle{\pgfqpoint{0.800000in}{0.528000in}}{\pgfqpoint{4.960000in}{3.696000in}}%
\pgfusepath{clip}%
\pgfsetbuttcap%
\pgfsetroundjoin%
\definecolor{currentfill}{rgb}{0.000000,0.000000,0.000000}%
\pgfsetfillcolor{currentfill}%
\pgfsetlinewidth{1.003750pt}%
\definecolor{currentstroke}{rgb}{0.000000,0.000000,0.000000}%
\pgfsetstrokecolor{currentstroke}%
\pgfsetdash{}{0pt}%
\pgfpathmoveto{\pgfqpoint{4.011666in}{1.773177in}}%
\pgfpathcurveto{\pgfqpoint{4.022716in}{1.773177in}}{\pgfqpoint{4.033315in}{1.777567in}}{\pgfqpoint{4.041128in}{1.785380in}}%
\pgfpathcurveto{\pgfqpoint{4.048942in}{1.793194in}}{\pgfqpoint{4.053332in}{1.803793in}}{\pgfqpoint{4.053332in}{1.814843in}}%
\pgfpathcurveto{\pgfqpoint{4.053332in}{1.825893in}}{\pgfqpoint{4.048942in}{1.836492in}}{\pgfqpoint{4.041128in}{1.844306in}}%
\pgfpathcurveto{\pgfqpoint{4.033315in}{1.852120in}}{\pgfqpoint{4.022716in}{1.856510in}}{\pgfqpoint{4.011666in}{1.856510in}}%
\pgfpathcurveto{\pgfqpoint{4.000616in}{1.856510in}}{\pgfqpoint{3.990016in}{1.852120in}}{\pgfqpoint{3.982203in}{1.844306in}}%
\pgfpathcurveto{\pgfqpoint{3.974389in}{1.836492in}}{\pgfqpoint{3.969999in}{1.825893in}}{\pgfqpoint{3.969999in}{1.814843in}}%
\pgfpathcurveto{\pgfqpoint{3.969999in}{1.803793in}}{\pgfqpoint{3.974389in}{1.793194in}}{\pgfqpoint{3.982203in}{1.785380in}}%
\pgfpathcurveto{\pgfqpoint{3.990016in}{1.777567in}}{\pgfqpoint{4.000616in}{1.773177in}}{\pgfqpoint{4.011666in}{1.773177in}}%
\pgfpathclose%
\pgfusepath{stroke,fill}%
\end{pgfscope}%
\begin{pgfscope}%
\pgfpathrectangle{\pgfqpoint{0.800000in}{0.528000in}}{\pgfqpoint{4.960000in}{3.696000in}}%
\pgfusepath{clip}%
\pgfsetbuttcap%
\pgfsetroundjoin%
\definecolor{currentfill}{rgb}{0.000000,0.000000,0.000000}%
\pgfsetfillcolor{currentfill}%
\pgfsetlinewidth{1.003750pt}%
\definecolor{currentstroke}{rgb}{0.000000,0.000000,0.000000}%
\pgfsetstrokecolor{currentstroke}%
\pgfsetdash{}{0pt}%
\pgfpathmoveto{\pgfqpoint{4.011666in}{1.859167in}}%
\pgfpathcurveto{\pgfqpoint{4.022716in}{1.859167in}}{\pgfqpoint{4.033315in}{1.863557in}}{\pgfqpoint{4.041128in}{1.871370in}}%
\pgfpathcurveto{\pgfqpoint{4.048942in}{1.879184in}}{\pgfqpoint{4.053332in}{1.889783in}}{\pgfqpoint{4.053332in}{1.900833in}}%
\pgfpathcurveto{\pgfqpoint{4.053332in}{1.911883in}}{\pgfqpoint{4.048942in}{1.922482in}}{\pgfqpoint{4.041128in}{1.930296in}}%
\pgfpathcurveto{\pgfqpoint{4.033315in}{1.938110in}}{\pgfqpoint{4.022716in}{1.942500in}}{\pgfqpoint{4.011666in}{1.942500in}}%
\pgfpathcurveto{\pgfqpoint{4.000616in}{1.942500in}}{\pgfqpoint{3.990016in}{1.938110in}}{\pgfqpoint{3.982203in}{1.930296in}}%
\pgfpathcurveto{\pgfqpoint{3.974389in}{1.922482in}}{\pgfqpoint{3.969999in}{1.911883in}}{\pgfqpoint{3.969999in}{1.900833in}}%
\pgfpathcurveto{\pgfqpoint{3.969999in}{1.889783in}}{\pgfqpoint{3.974389in}{1.879184in}}{\pgfqpoint{3.982203in}{1.871370in}}%
\pgfpathcurveto{\pgfqpoint{3.990016in}{1.863557in}}{\pgfqpoint{4.000616in}{1.859167in}}{\pgfqpoint{4.011666in}{1.859167in}}%
\pgfpathclose%
\pgfusepath{stroke,fill}%
\end{pgfscope}%
\begin{pgfscope}%
\pgfpathrectangle{\pgfqpoint{0.800000in}{0.528000in}}{\pgfqpoint{4.960000in}{3.696000in}}%
\pgfusepath{clip}%
\pgfsetbuttcap%
\pgfsetroundjoin%
\definecolor{currentfill}{rgb}{0.000000,0.000000,0.000000}%
\pgfsetfillcolor{currentfill}%
\pgfsetlinewidth{1.003750pt}%
\definecolor{currentstroke}{rgb}{0.000000,0.000000,0.000000}%
\pgfsetstrokecolor{currentstroke}%
\pgfsetdash{}{0pt}%
\pgfpathmoveto{\pgfqpoint{4.011666in}{1.923659in}}%
\pgfpathcurveto{\pgfqpoint{4.022716in}{1.923659in}}{\pgfqpoint{4.033315in}{1.928049in}}{\pgfqpoint{4.041128in}{1.935863in}}%
\pgfpathcurveto{\pgfqpoint{4.048942in}{1.943677in}}{\pgfqpoint{4.053332in}{1.954276in}}{\pgfqpoint{4.053332in}{1.965326in}}%
\pgfpathcurveto{\pgfqpoint{4.053332in}{1.976376in}}{\pgfqpoint{4.048942in}{1.986975in}}{\pgfqpoint{4.041128in}{1.994788in}}%
\pgfpathcurveto{\pgfqpoint{4.033315in}{2.002602in}}{\pgfqpoint{4.022716in}{2.006992in}}{\pgfqpoint{4.011666in}{2.006992in}}%
\pgfpathcurveto{\pgfqpoint{4.000616in}{2.006992in}}{\pgfqpoint{3.990016in}{2.002602in}}{\pgfqpoint{3.982203in}{1.994788in}}%
\pgfpathcurveto{\pgfqpoint{3.974389in}{1.986975in}}{\pgfqpoint{3.969999in}{1.976376in}}{\pgfqpoint{3.969999in}{1.965326in}}%
\pgfpathcurveto{\pgfqpoint{3.969999in}{1.954276in}}{\pgfqpoint{3.974389in}{1.943677in}}{\pgfqpoint{3.982203in}{1.935863in}}%
\pgfpathcurveto{\pgfqpoint{3.990016in}{1.928049in}}{\pgfqpoint{4.000616in}{1.923659in}}{\pgfqpoint{4.011666in}{1.923659in}}%
\pgfpathclose%
\pgfusepath{stroke,fill}%
\end{pgfscope}%
\begin{pgfscope}%
\pgfpathrectangle{\pgfqpoint{0.800000in}{0.528000in}}{\pgfqpoint{4.960000in}{3.696000in}}%
\pgfusepath{clip}%
\pgfsetbuttcap%
\pgfsetroundjoin%
\definecolor{currentfill}{rgb}{0.000000,0.000000,0.000000}%
\pgfsetfillcolor{currentfill}%
\pgfsetlinewidth{1.003750pt}%
\definecolor{currentstroke}{rgb}{0.000000,0.000000,0.000000}%
\pgfsetstrokecolor{currentstroke}%
\pgfsetdash{}{0pt}%
\pgfpathmoveto{\pgfqpoint{4.011666in}{1.859167in}}%
\pgfpathcurveto{\pgfqpoint{4.022716in}{1.859167in}}{\pgfqpoint{4.033315in}{1.863557in}}{\pgfqpoint{4.041128in}{1.871370in}}%
\pgfpathcurveto{\pgfqpoint{4.048942in}{1.879184in}}{\pgfqpoint{4.053332in}{1.889783in}}{\pgfqpoint{4.053332in}{1.900833in}}%
\pgfpathcurveto{\pgfqpoint{4.053332in}{1.911883in}}{\pgfqpoint{4.048942in}{1.922482in}}{\pgfqpoint{4.041128in}{1.930296in}}%
\pgfpathcurveto{\pgfqpoint{4.033315in}{1.938110in}}{\pgfqpoint{4.022716in}{1.942500in}}{\pgfqpoint{4.011666in}{1.942500in}}%
\pgfpathcurveto{\pgfqpoint{4.000616in}{1.942500in}}{\pgfqpoint{3.990016in}{1.938110in}}{\pgfqpoint{3.982203in}{1.930296in}}%
\pgfpathcurveto{\pgfqpoint{3.974389in}{1.922482in}}{\pgfqpoint{3.969999in}{1.911883in}}{\pgfqpoint{3.969999in}{1.900833in}}%
\pgfpathcurveto{\pgfqpoint{3.969999in}{1.889783in}}{\pgfqpoint{3.974389in}{1.879184in}}{\pgfqpoint{3.982203in}{1.871370in}}%
\pgfpathcurveto{\pgfqpoint{3.990016in}{1.863557in}}{\pgfqpoint{4.000616in}{1.859167in}}{\pgfqpoint{4.011666in}{1.859167in}}%
\pgfpathclose%
\pgfusepath{stroke,fill}%
\end{pgfscope}%
\begin{pgfscope}%
\pgfpathrectangle{\pgfqpoint{0.800000in}{0.528000in}}{\pgfqpoint{4.960000in}{3.696000in}}%
\pgfusepath{clip}%
\pgfsetbuttcap%
\pgfsetroundjoin%
\definecolor{currentfill}{rgb}{0.000000,0.000000,0.000000}%
\pgfsetfillcolor{currentfill}%
\pgfsetlinewidth{1.003750pt}%
\definecolor{currentstroke}{rgb}{0.000000,0.000000,0.000000}%
\pgfsetstrokecolor{currentstroke}%
\pgfsetdash{}{0pt}%
\pgfpathmoveto{\pgfqpoint{4.011666in}{1.966654in}}%
\pgfpathcurveto{\pgfqpoint{4.022716in}{1.966654in}}{\pgfqpoint{4.033315in}{1.971044in}}{\pgfqpoint{4.041128in}{1.978858in}}%
\pgfpathcurveto{\pgfqpoint{4.048942in}{1.986672in}}{\pgfqpoint{4.053332in}{1.997271in}}{\pgfqpoint{4.053332in}{2.008321in}}%
\pgfpathcurveto{\pgfqpoint{4.053332in}{2.019371in}}{\pgfqpoint{4.048942in}{2.029970in}}{\pgfqpoint{4.041128in}{2.037783in}}%
\pgfpathcurveto{\pgfqpoint{4.033315in}{2.045597in}}{\pgfqpoint{4.022716in}{2.049987in}}{\pgfqpoint{4.011666in}{2.049987in}}%
\pgfpathcurveto{\pgfqpoint{4.000616in}{2.049987in}}{\pgfqpoint{3.990016in}{2.045597in}}{\pgfqpoint{3.982203in}{2.037783in}}%
\pgfpathcurveto{\pgfqpoint{3.974389in}{2.029970in}}{\pgfqpoint{3.969999in}{2.019371in}}{\pgfqpoint{3.969999in}{2.008321in}}%
\pgfpathcurveto{\pgfqpoint{3.969999in}{1.997271in}}{\pgfqpoint{3.974389in}{1.986672in}}{\pgfqpoint{3.982203in}{1.978858in}}%
\pgfpathcurveto{\pgfqpoint{3.990016in}{1.971044in}}{\pgfqpoint{4.000616in}{1.966654in}}{\pgfqpoint{4.011666in}{1.966654in}}%
\pgfpathclose%
\pgfusepath{stroke,fill}%
\end{pgfscope}%
\begin{pgfscope}%
\pgfpathrectangle{\pgfqpoint{0.800000in}{0.528000in}}{\pgfqpoint{4.960000in}{3.696000in}}%
\pgfusepath{clip}%
\pgfsetbuttcap%
\pgfsetroundjoin%
\definecolor{currentfill}{rgb}{0.000000,0.000000,0.000000}%
\pgfsetfillcolor{currentfill}%
\pgfsetlinewidth{1.003750pt}%
\definecolor{currentstroke}{rgb}{0.000000,0.000000,0.000000}%
\pgfsetstrokecolor{currentstroke}%
\pgfsetdash{}{0pt}%
\pgfpathmoveto{\pgfqpoint{4.011666in}{2.074141in}}%
\pgfpathcurveto{\pgfqpoint{4.022716in}{2.074141in}}{\pgfqpoint{4.033315in}{2.078532in}}{\pgfqpoint{4.041128in}{2.086345in}}%
\pgfpathcurveto{\pgfqpoint{4.048942in}{2.094159in}}{\pgfqpoint{4.053332in}{2.104758in}}{\pgfqpoint{4.053332in}{2.115808in}}%
\pgfpathcurveto{\pgfqpoint{4.053332in}{2.126858in}}{\pgfqpoint{4.048942in}{2.137457in}}{\pgfqpoint{4.041128in}{2.145271in}}%
\pgfpathcurveto{\pgfqpoint{4.033315in}{2.153085in}}{\pgfqpoint{4.022716in}{2.157475in}}{\pgfqpoint{4.011666in}{2.157475in}}%
\pgfpathcurveto{\pgfqpoint{4.000616in}{2.157475in}}{\pgfqpoint{3.990016in}{2.153085in}}{\pgfqpoint{3.982203in}{2.145271in}}%
\pgfpathcurveto{\pgfqpoint{3.974389in}{2.137457in}}{\pgfqpoint{3.969999in}{2.126858in}}{\pgfqpoint{3.969999in}{2.115808in}}%
\pgfpathcurveto{\pgfqpoint{3.969999in}{2.104758in}}{\pgfqpoint{3.974389in}{2.094159in}}{\pgfqpoint{3.982203in}{2.086345in}}%
\pgfpathcurveto{\pgfqpoint{3.990016in}{2.078532in}}{\pgfqpoint{4.000616in}{2.074141in}}{\pgfqpoint{4.011666in}{2.074141in}}%
\pgfpathclose%
\pgfusepath{stroke,fill}%
\end{pgfscope}%
\begin{pgfscope}%
\pgfpathrectangle{\pgfqpoint{0.800000in}{0.528000in}}{\pgfqpoint{4.960000in}{3.696000in}}%
\pgfusepath{clip}%
\pgfsetbuttcap%
\pgfsetroundjoin%
\definecolor{currentfill}{rgb}{0.000000,0.000000,0.000000}%
\pgfsetfillcolor{currentfill}%
\pgfsetlinewidth{1.003750pt}%
\definecolor{currentstroke}{rgb}{0.000000,0.000000,0.000000}%
\pgfsetstrokecolor{currentstroke}%
\pgfsetdash{}{0pt}%
\pgfpathmoveto{\pgfqpoint{4.011666in}{1.880664in}}%
\pgfpathcurveto{\pgfqpoint{4.022716in}{1.880664in}}{\pgfqpoint{4.033315in}{1.885054in}}{\pgfqpoint{4.041128in}{1.892868in}}%
\pgfpathcurveto{\pgfqpoint{4.048942in}{1.900682in}}{\pgfqpoint{4.053332in}{1.911281in}}{\pgfqpoint{4.053332in}{1.922331in}}%
\pgfpathcurveto{\pgfqpoint{4.053332in}{1.933381in}}{\pgfqpoint{4.048942in}{1.943980in}}{\pgfqpoint{4.041128in}{1.951793in}}%
\pgfpathcurveto{\pgfqpoint{4.033315in}{1.959607in}}{\pgfqpoint{4.022716in}{1.963997in}}{\pgfqpoint{4.011666in}{1.963997in}}%
\pgfpathcurveto{\pgfqpoint{4.000616in}{1.963997in}}{\pgfqpoint{3.990016in}{1.959607in}}{\pgfqpoint{3.982203in}{1.951793in}}%
\pgfpathcurveto{\pgfqpoint{3.974389in}{1.943980in}}{\pgfqpoint{3.969999in}{1.933381in}}{\pgfqpoint{3.969999in}{1.922331in}}%
\pgfpathcurveto{\pgfqpoint{3.969999in}{1.911281in}}{\pgfqpoint{3.974389in}{1.900682in}}{\pgfqpoint{3.982203in}{1.892868in}}%
\pgfpathcurveto{\pgfqpoint{3.990016in}{1.885054in}}{\pgfqpoint{4.000616in}{1.880664in}}{\pgfqpoint{4.011666in}{1.880664in}}%
\pgfpathclose%
\pgfusepath{stroke,fill}%
\end{pgfscope}%
\begin{pgfscope}%
\pgfpathrectangle{\pgfqpoint{0.800000in}{0.528000in}}{\pgfqpoint{4.960000in}{3.696000in}}%
\pgfusepath{clip}%
\pgfsetbuttcap%
\pgfsetroundjoin%
\definecolor{currentfill}{rgb}{0.000000,0.000000,0.000000}%
\pgfsetfillcolor{currentfill}%
\pgfsetlinewidth{1.003750pt}%
\definecolor{currentstroke}{rgb}{0.000000,0.000000,0.000000}%
\pgfsetstrokecolor{currentstroke}%
\pgfsetdash{}{0pt}%
\pgfpathmoveto{\pgfqpoint{4.011666in}{1.816172in}}%
\pgfpathcurveto{\pgfqpoint{4.022716in}{1.816172in}}{\pgfqpoint{4.033315in}{1.820562in}}{\pgfqpoint{4.041128in}{1.828375in}}%
\pgfpathcurveto{\pgfqpoint{4.048942in}{1.836189in}}{\pgfqpoint{4.053332in}{1.846788in}}{\pgfqpoint{4.053332in}{1.857838in}}%
\pgfpathcurveto{\pgfqpoint{4.053332in}{1.868888in}}{\pgfqpoint{4.048942in}{1.879487in}}{\pgfqpoint{4.041128in}{1.887301in}}%
\pgfpathcurveto{\pgfqpoint{4.033315in}{1.895115in}}{\pgfqpoint{4.022716in}{1.899505in}}{\pgfqpoint{4.011666in}{1.899505in}}%
\pgfpathcurveto{\pgfqpoint{4.000616in}{1.899505in}}{\pgfqpoint{3.990016in}{1.895115in}}{\pgfqpoint{3.982203in}{1.887301in}}%
\pgfpathcurveto{\pgfqpoint{3.974389in}{1.879487in}}{\pgfqpoint{3.969999in}{1.868888in}}{\pgfqpoint{3.969999in}{1.857838in}}%
\pgfpathcurveto{\pgfqpoint{3.969999in}{1.846788in}}{\pgfqpoint{3.974389in}{1.836189in}}{\pgfqpoint{3.982203in}{1.828375in}}%
\pgfpathcurveto{\pgfqpoint{3.990016in}{1.820562in}}{\pgfqpoint{4.000616in}{1.816172in}}{\pgfqpoint{4.011666in}{1.816172in}}%
\pgfpathclose%
\pgfusepath{stroke,fill}%
\end{pgfscope}%
\begin{pgfscope}%
\pgfpathrectangle{\pgfqpoint{0.800000in}{0.528000in}}{\pgfqpoint{4.960000in}{3.696000in}}%
\pgfusepath{clip}%
\pgfsetbuttcap%
\pgfsetroundjoin%
\definecolor{currentfill}{rgb}{0.000000,0.000000,0.000000}%
\pgfsetfillcolor{currentfill}%
\pgfsetlinewidth{1.003750pt}%
\definecolor{currentstroke}{rgb}{0.000000,0.000000,0.000000}%
\pgfsetstrokecolor{currentstroke}%
\pgfsetdash{}{0pt}%
\pgfpathmoveto{\pgfqpoint{4.011666in}{1.730182in}}%
\pgfpathcurveto{\pgfqpoint{4.022716in}{1.730182in}}{\pgfqpoint{4.033315in}{1.734572in}}{\pgfqpoint{4.041128in}{1.742385in}}%
\pgfpathcurveto{\pgfqpoint{4.048942in}{1.750199in}}{\pgfqpoint{4.053332in}{1.760798in}}{\pgfqpoint{4.053332in}{1.771848in}}%
\pgfpathcurveto{\pgfqpoint{4.053332in}{1.782898in}}{\pgfqpoint{4.048942in}{1.793497in}}{\pgfqpoint{4.041128in}{1.801311in}}%
\pgfpathcurveto{\pgfqpoint{4.033315in}{1.809125in}}{\pgfqpoint{4.022716in}{1.813515in}}{\pgfqpoint{4.011666in}{1.813515in}}%
\pgfpathcurveto{\pgfqpoint{4.000616in}{1.813515in}}{\pgfqpoint{3.990016in}{1.809125in}}{\pgfqpoint{3.982203in}{1.801311in}}%
\pgfpathcurveto{\pgfqpoint{3.974389in}{1.793497in}}{\pgfqpoint{3.969999in}{1.782898in}}{\pgfqpoint{3.969999in}{1.771848in}}%
\pgfpathcurveto{\pgfqpoint{3.969999in}{1.760798in}}{\pgfqpoint{3.974389in}{1.750199in}}{\pgfqpoint{3.982203in}{1.742385in}}%
\pgfpathcurveto{\pgfqpoint{3.990016in}{1.734572in}}{\pgfqpoint{4.000616in}{1.730182in}}{\pgfqpoint{4.011666in}{1.730182in}}%
\pgfpathclose%
\pgfusepath{stroke,fill}%
\end{pgfscope}%
\begin{pgfscope}%
\pgfpathrectangle{\pgfqpoint{0.800000in}{0.528000in}}{\pgfqpoint{4.960000in}{3.696000in}}%
\pgfusepath{clip}%
\pgfsetbuttcap%
\pgfsetroundjoin%
\definecolor{currentfill}{rgb}{0.000000,0.000000,0.000000}%
\pgfsetfillcolor{currentfill}%
\pgfsetlinewidth{1.003750pt}%
\definecolor{currentstroke}{rgb}{0.000000,0.000000,0.000000}%
\pgfsetstrokecolor{currentstroke}%
\pgfsetdash{}{0pt}%
\pgfpathmoveto{\pgfqpoint{4.011666in}{2.052644in}}%
\pgfpathcurveto{\pgfqpoint{4.022716in}{2.052644in}}{\pgfqpoint{4.033315in}{2.057034in}}{\pgfqpoint{4.041128in}{2.064848in}}%
\pgfpathcurveto{\pgfqpoint{4.048942in}{2.072661in}}{\pgfqpoint{4.053332in}{2.083261in}}{\pgfqpoint{4.053332in}{2.094311in}}%
\pgfpathcurveto{\pgfqpoint{4.053332in}{2.105361in}}{\pgfqpoint{4.048942in}{2.115960in}}{\pgfqpoint{4.041128in}{2.123773in}}%
\pgfpathcurveto{\pgfqpoint{4.033315in}{2.131587in}}{\pgfqpoint{4.022716in}{2.135977in}}{\pgfqpoint{4.011666in}{2.135977in}}%
\pgfpathcurveto{\pgfqpoint{4.000616in}{2.135977in}}{\pgfqpoint{3.990016in}{2.131587in}}{\pgfqpoint{3.982203in}{2.123773in}}%
\pgfpathcurveto{\pgfqpoint{3.974389in}{2.115960in}}{\pgfqpoint{3.969999in}{2.105361in}}{\pgfqpoint{3.969999in}{2.094311in}}%
\pgfpathcurveto{\pgfqpoint{3.969999in}{2.083261in}}{\pgfqpoint{3.974389in}{2.072661in}}{\pgfqpoint{3.982203in}{2.064848in}}%
\pgfpathcurveto{\pgfqpoint{3.990016in}{2.057034in}}{\pgfqpoint{4.000616in}{2.052644in}}{\pgfqpoint{4.011666in}{2.052644in}}%
\pgfpathclose%
\pgfusepath{stroke,fill}%
\end{pgfscope}%
\begin{pgfscope}%
\pgfpathrectangle{\pgfqpoint{0.800000in}{0.528000in}}{\pgfqpoint{4.960000in}{3.696000in}}%
\pgfusepath{clip}%
\pgfsetbuttcap%
\pgfsetroundjoin%
\definecolor{currentfill}{rgb}{0.000000,0.000000,0.000000}%
\pgfsetfillcolor{currentfill}%
\pgfsetlinewidth{1.003750pt}%
\definecolor{currentstroke}{rgb}{0.000000,0.000000,0.000000}%
\pgfsetstrokecolor{currentstroke}%
\pgfsetdash{}{0pt}%
\pgfpathmoveto{\pgfqpoint{4.011666in}{1.794674in}}%
\pgfpathcurveto{\pgfqpoint{4.022716in}{1.794674in}}{\pgfqpoint{4.033315in}{1.799064in}}{\pgfqpoint{4.041128in}{1.806878in}}%
\pgfpathcurveto{\pgfqpoint{4.048942in}{1.814692in}}{\pgfqpoint{4.053332in}{1.825291in}}{\pgfqpoint{4.053332in}{1.836341in}}%
\pgfpathcurveto{\pgfqpoint{4.053332in}{1.847391in}}{\pgfqpoint{4.048942in}{1.857990in}}{\pgfqpoint{4.041128in}{1.865804in}}%
\pgfpathcurveto{\pgfqpoint{4.033315in}{1.873617in}}{\pgfqpoint{4.022716in}{1.878007in}}{\pgfqpoint{4.011666in}{1.878007in}}%
\pgfpathcurveto{\pgfqpoint{4.000616in}{1.878007in}}{\pgfqpoint{3.990016in}{1.873617in}}{\pgfqpoint{3.982203in}{1.865804in}}%
\pgfpathcurveto{\pgfqpoint{3.974389in}{1.857990in}}{\pgfqpoint{3.969999in}{1.847391in}}{\pgfqpoint{3.969999in}{1.836341in}}%
\pgfpathcurveto{\pgfqpoint{3.969999in}{1.825291in}}{\pgfqpoint{3.974389in}{1.814692in}}{\pgfqpoint{3.982203in}{1.806878in}}%
\pgfpathcurveto{\pgfqpoint{3.990016in}{1.799064in}}{\pgfqpoint{4.000616in}{1.794674in}}{\pgfqpoint{4.011666in}{1.794674in}}%
\pgfpathclose%
\pgfusepath{stroke,fill}%
\end{pgfscope}%
\begin{pgfscope}%
\pgfpathrectangle{\pgfqpoint{0.800000in}{0.528000in}}{\pgfqpoint{4.960000in}{3.696000in}}%
\pgfusepath{clip}%
\pgfsetbuttcap%
\pgfsetroundjoin%
\definecolor{currentfill}{rgb}{0.000000,0.000000,0.000000}%
\pgfsetfillcolor{currentfill}%
\pgfsetlinewidth{1.003750pt}%
\definecolor{currentstroke}{rgb}{0.000000,0.000000,0.000000}%
\pgfsetstrokecolor{currentstroke}%
\pgfsetdash{}{0pt}%
\pgfpathmoveto{\pgfqpoint{4.011666in}{2.052644in}}%
\pgfpathcurveto{\pgfqpoint{4.022716in}{2.052644in}}{\pgfqpoint{4.033315in}{2.057034in}}{\pgfqpoint{4.041128in}{2.064848in}}%
\pgfpathcurveto{\pgfqpoint{4.048942in}{2.072661in}}{\pgfqpoint{4.053332in}{2.083261in}}{\pgfqpoint{4.053332in}{2.094311in}}%
\pgfpathcurveto{\pgfqpoint{4.053332in}{2.105361in}}{\pgfqpoint{4.048942in}{2.115960in}}{\pgfqpoint{4.041128in}{2.123773in}}%
\pgfpathcurveto{\pgfqpoint{4.033315in}{2.131587in}}{\pgfqpoint{4.022716in}{2.135977in}}{\pgfqpoint{4.011666in}{2.135977in}}%
\pgfpathcurveto{\pgfqpoint{4.000616in}{2.135977in}}{\pgfqpoint{3.990016in}{2.131587in}}{\pgfqpoint{3.982203in}{2.123773in}}%
\pgfpathcurveto{\pgfqpoint{3.974389in}{2.115960in}}{\pgfqpoint{3.969999in}{2.105361in}}{\pgfqpoint{3.969999in}{2.094311in}}%
\pgfpathcurveto{\pgfqpoint{3.969999in}{2.083261in}}{\pgfqpoint{3.974389in}{2.072661in}}{\pgfqpoint{3.982203in}{2.064848in}}%
\pgfpathcurveto{\pgfqpoint{3.990016in}{2.057034in}}{\pgfqpoint{4.000616in}{2.052644in}}{\pgfqpoint{4.011666in}{2.052644in}}%
\pgfpathclose%
\pgfusepath{stroke,fill}%
\end{pgfscope}%
\begin{pgfscope}%
\pgfpathrectangle{\pgfqpoint{0.800000in}{0.528000in}}{\pgfqpoint{4.960000in}{3.696000in}}%
\pgfusepath{clip}%
\pgfsetbuttcap%
\pgfsetroundjoin%
\definecolor{currentfill}{rgb}{0.000000,0.000000,0.000000}%
\pgfsetfillcolor{currentfill}%
\pgfsetlinewidth{1.003750pt}%
\definecolor{currentstroke}{rgb}{0.000000,0.000000,0.000000}%
\pgfsetstrokecolor{currentstroke}%
\pgfsetdash{}{0pt}%
\pgfpathmoveto{\pgfqpoint{4.011666in}{2.031146in}}%
\pgfpathcurveto{\pgfqpoint{4.022716in}{2.031146in}}{\pgfqpoint{4.033315in}{2.035537in}}{\pgfqpoint{4.041128in}{2.043350in}}%
\pgfpathcurveto{\pgfqpoint{4.048942in}{2.051164in}}{\pgfqpoint{4.053332in}{2.061763in}}{\pgfqpoint{4.053332in}{2.072813in}}%
\pgfpathcurveto{\pgfqpoint{4.053332in}{2.083863in}}{\pgfqpoint{4.048942in}{2.094462in}}{\pgfqpoint{4.041128in}{2.102276in}}%
\pgfpathcurveto{\pgfqpoint{4.033315in}{2.110090in}}{\pgfqpoint{4.022716in}{2.114480in}}{\pgfqpoint{4.011666in}{2.114480in}}%
\pgfpathcurveto{\pgfqpoint{4.000616in}{2.114480in}}{\pgfqpoint{3.990016in}{2.110090in}}{\pgfqpoint{3.982203in}{2.102276in}}%
\pgfpathcurveto{\pgfqpoint{3.974389in}{2.094462in}}{\pgfqpoint{3.969999in}{2.083863in}}{\pgfqpoint{3.969999in}{2.072813in}}%
\pgfpathcurveto{\pgfqpoint{3.969999in}{2.061763in}}{\pgfqpoint{3.974389in}{2.051164in}}{\pgfqpoint{3.982203in}{2.043350in}}%
\pgfpathcurveto{\pgfqpoint{3.990016in}{2.035537in}}{\pgfqpoint{4.000616in}{2.031146in}}{\pgfqpoint{4.011666in}{2.031146in}}%
\pgfpathclose%
\pgfusepath{stroke,fill}%
\end{pgfscope}%
\begin{pgfscope}%
\pgfpathrectangle{\pgfqpoint{0.800000in}{0.528000in}}{\pgfqpoint{4.960000in}{3.696000in}}%
\pgfusepath{clip}%
\pgfsetbuttcap%
\pgfsetroundjoin%
\definecolor{currentfill}{rgb}{0.000000,0.000000,0.000000}%
\pgfsetfillcolor{currentfill}%
\pgfsetlinewidth{1.003750pt}%
\definecolor{currentstroke}{rgb}{0.000000,0.000000,0.000000}%
\pgfsetstrokecolor{currentstroke}%
\pgfsetdash{}{0pt}%
\pgfpathmoveto{\pgfqpoint{4.011666in}{1.816172in}}%
\pgfpathcurveto{\pgfqpoint{4.022716in}{1.816172in}}{\pgfqpoint{4.033315in}{1.820562in}}{\pgfqpoint{4.041128in}{1.828375in}}%
\pgfpathcurveto{\pgfqpoint{4.048942in}{1.836189in}}{\pgfqpoint{4.053332in}{1.846788in}}{\pgfqpoint{4.053332in}{1.857838in}}%
\pgfpathcurveto{\pgfqpoint{4.053332in}{1.868888in}}{\pgfqpoint{4.048942in}{1.879487in}}{\pgfqpoint{4.041128in}{1.887301in}}%
\pgfpathcurveto{\pgfqpoint{4.033315in}{1.895115in}}{\pgfqpoint{4.022716in}{1.899505in}}{\pgfqpoint{4.011666in}{1.899505in}}%
\pgfpathcurveto{\pgfqpoint{4.000616in}{1.899505in}}{\pgfqpoint{3.990016in}{1.895115in}}{\pgfqpoint{3.982203in}{1.887301in}}%
\pgfpathcurveto{\pgfqpoint{3.974389in}{1.879487in}}{\pgfqpoint{3.969999in}{1.868888in}}{\pgfqpoint{3.969999in}{1.857838in}}%
\pgfpathcurveto{\pgfqpoint{3.969999in}{1.846788in}}{\pgfqpoint{3.974389in}{1.836189in}}{\pgfqpoint{3.982203in}{1.828375in}}%
\pgfpathcurveto{\pgfqpoint{3.990016in}{1.820562in}}{\pgfqpoint{4.000616in}{1.816172in}}{\pgfqpoint{4.011666in}{1.816172in}}%
\pgfpathclose%
\pgfusepath{stroke,fill}%
\end{pgfscope}%
\begin{pgfscope}%
\pgfpathrectangle{\pgfqpoint{0.800000in}{0.528000in}}{\pgfqpoint{4.960000in}{3.696000in}}%
\pgfusepath{clip}%
\pgfsetbuttcap%
\pgfsetroundjoin%
\definecolor{currentfill}{rgb}{0.000000,0.000000,0.000000}%
\pgfsetfillcolor{currentfill}%
\pgfsetlinewidth{1.003750pt}%
\definecolor{currentstroke}{rgb}{0.000000,0.000000,0.000000}%
\pgfsetstrokecolor{currentstroke}%
\pgfsetdash{}{0pt}%
\pgfpathmoveto{\pgfqpoint{4.011666in}{1.794674in}}%
\pgfpathcurveto{\pgfqpoint{4.022716in}{1.794674in}}{\pgfqpoint{4.033315in}{1.799064in}}{\pgfqpoint{4.041128in}{1.806878in}}%
\pgfpathcurveto{\pgfqpoint{4.048942in}{1.814692in}}{\pgfqpoint{4.053332in}{1.825291in}}{\pgfqpoint{4.053332in}{1.836341in}}%
\pgfpathcurveto{\pgfqpoint{4.053332in}{1.847391in}}{\pgfqpoint{4.048942in}{1.857990in}}{\pgfqpoint{4.041128in}{1.865804in}}%
\pgfpathcurveto{\pgfqpoint{4.033315in}{1.873617in}}{\pgfqpoint{4.022716in}{1.878007in}}{\pgfqpoint{4.011666in}{1.878007in}}%
\pgfpathcurveto{\pgfqpoint{4.000616in}{1.878007in}}{\pgfqpoint{3.990016in}{1.873617in}}{\pgfqpoint{3.982203in}{1.865804in}}%
\pgfpathcurveto{\pgfqpoint{3.974389in}{1.857990in}}{\pgfqpoint{3.969999in}{1.847391in}}{\pgfqpoint{3.969999in}{1.836341in}}%
\pgfpathcurveto{\pgfqpoint{3.969999in}{1.825291in}}{\pgfqpoint{3.974389in}{1.814692in}}{\pgfqpoint{3.982203in}{1.806878in}}%
\pgfpathcurveto{\pgfqpoint{3.990016in}{1.799064in}}{\pgfqpoint{4.000616in}{1.794674in}}{\pgfqpoint{4.011666in}{1.794674in}}%
\pgfpathclose%
\pgfusepath{stroke,fill}%
\end{pgfscope}%
\begin{pgfscope}%
\pgfpathrectangle{\pgfqpoint{0.800000in}{0.528000in}}{\pgfqpoint{4.960000in}{3.696000in}}%
\pgfusepath{clip}%
\pgfsetbuttcap%
\pgfsetroundjoin%
\definecolor{currentfill}{rgb}{0.000000,0.000000,0.000000}%
\pgfsetfillcolor{currentfill}%
\pgfsetlinewidth{1.003750pt}%
\definecolor{currentstroke}{rgb}{0.000000,0.000000,0.000000}%
\pgfsetstrokecolor{currentstroke}%
\pgfsetdash{}{0pt}%
\pgfpathmoveto{\pgfqpoint{4.011666in}{2.375106in}}%
\pgfpathcurveto{\pgfqpoint{4.022716in}{2.375106in}}{\pgfqpoint{4.033315in}{2.379497in}}{\pgfqpoint{4.041128in}{2.387310in}}%
\pgfpathcurveto{\pgfqpoint{4.048942in}{2.395124in}}{\pgfqpoint{4.053332in}{2.405723in}}{\pgfqpoint{4.053332in}{2.416773in}}%
\pgfpathcurveto{\pgfqpoint{4.053332in}{2.427823in}}{\pgfqpoint{4.048942in}{2.438422in}}{\pgfqpoint{4.041128in}{2.446236in}}%
\pgfpathcurveto{\pgfqpoint{4.033315in}{2.454049in}}{\pgfqpoint{4.022716in}{2.458440in}}{\pgfqpoint{4.011666in}{2.458440in}}%
\pgfpathcurveto{\pgfqpoint{4.000616in}{2.458440in}}{\pgfqpoint{3.990016in}{2.454049in}}{\pgfqpoint{3.982203in}{2.446236in}}%
\pgfpathcurveto{\pgfqpoint{3.974389in}{2.438422in}}{\pgfqpoint{3.969999in}{2.427823in}}{\pgfqpoint{3.969999in}{2.416773in}}%
\pgfpathcurveto{\pgfqpoint{3.969999in}{2.405723in}}{\pgfqpoint{3.974389in}{2.395124in}}{\pgfqpoint{3.982203in}{2.387310in}}%
\pgfpathcurveto{\pgfqpoint{3.990016in}{2.379497in}}{\pgfqpoint{4.000616in}{2.375106in}}{\pgfqpoint{4.011666in}{2.375106in}}%
\pgfpathclose%
\pgfusepath{stroke,fill}%
\end{pgfscope}%
\begin{pgfscope}%
\pgfpathrectangle{\pgfqpoint{0.800000in}{0.528000in}}{\pgfqpoint{4.960000in}{3.696000in}}%
\pgfusepath{clip}%
\pgfsetbuttcap%
\pgfsetroundjoin%
\definecolor{currentfill}{rgb}{0.000000,0.000000,0.000000}%
\pgfsetfillcolor{currentfill}%
\pgfsetlinewidth{1.003750pt}%
\definecolor{currentstroke}{rgb}{0.000000,0.000000,0.000000}%
\pgfsetstrokecolor{currentstroke}%
\pgfsetdash{}{0pt}%
\pgfpathmoveto{\pgfqpoint{4.011666in}{1.880664in}}%
\pgfpathcurveto{\pgfqpoint{4.022716in}{1.880664in}}{\pgfqpoint{4.033315in}{1.885054in}}{\pgfqpoint{4.041128in}{1.892868in}}%
\pgfpathcurveto{\pgfqpoint{4.048942in}{1.900682in}}{\pgfqpoint{4.053332in}{1.911281in}}{\pgfqpoint{4.053332in}{1.922331in}}%
\pgfpathcurveto{\pgfqpoint{4.053332in}{1.933381in}}{\pgfqpoint{4.048942in}{1.943980in}}{\pgfqpoint{4.041128in}{1.951793in}}%
\pgfpathcurveto{\pgfqpoint{4.033315in}{1.959607in}}{\pgfqpoint{4.022716in}{1.963997in}}{\pgfqpoint{4.011666in}{1.963997in}}%
\pgfpathcurveto{\pgfqpoint{4.000616in}{1.963997in}}{\pgfqpoint{3.990016in}{1.959607in}}{\pgfqpoint{3.982203in}{1.951793in}}%
\pgfpathcurveto{\pgfqpoint{3.974389in}{1.943980in}}{\pgfqpoint{3.969999in}{1.933381in}}{\pgfqpoint{3.969999in}{1.922331in}}%
\pgfpathcurveto{\pgfqpoint{3.969999in}{1.911281in}}{\pgfqpoint{3.974389in}{1.900682in}}{\pgfqpoint{3.982203in}{1.892868in}}%
\pgfpathcurveto{\pgfqpoint{3.990016in}{1.885054in}}{\pgfqpoint{4.000616in}{1.880664in}}{\pgfqpoint{4.011666in}{1.880664in}}%
\pgfpathclose%
\pgfusepath{stroke,fill}%
\end{pgfscope}%
\begin{pgfscope}%
\pgfpathrectangle{\pgfqpoint{0.800000in}{0.528000in}}{\pgfqpoint{4.960000in}{3.696000in}}%
\pgfusepath{clip}%
\pgfsetbuttcap%
\pgfsetroundjoin%
\definecolor{currentfill}{rgb}{0.000000,0.000000,0.000000}%
\pgfsetfillcolor{currentfill}%
\pgfsetlinewidth{1.003750pt}%
\definecolor{currentstroke}{rgb}{0.000000,0.000000,0.000000}%
\pgfsetstrokecolor{currentstroke}%
\pgfsetdash{}{0pt}%
\pgfpathmoveto{\pgfqpoint{4.011666in}{1.794674in}}%
\pgfpathcurveto{\pgfqpoint{4.022716in}{1.794674in}}{\pgfqpoint{4.033315in}{1.799064in}}{\pgfqpoint{4.041128in}{1.806878in}}%
\pgfpathcurveto{\pgfqpoint{4.048942in}{1.814692in}}{\pgfqpoint{4.053332in}{1.825291in}}{\pgfqpoint{4.053332in}{1.836341in}}%
\pgfpathcurveto{\pgfqpoint{4.053332in}{1.847391in}}{\pgfqpoint{4.048942in}{1.857990in}}{\pgfqpoint{4.041128in}{1.865804in}}%
\pgfpathcurveto{\pgfqpoint{4.033315in}{1.873617in}}{\pgfqpoint{4.022716in}{1.878007in}}{\pgfqpoint{4.011666in}{1.878007in}}%
\pgfpathcurveto{\pgfqpoint{4.000616in}{1.878007in}}{\pgfqpoint{3.990016in}{1.873617in}}{\pgfqpoint{3.982203in}{1.865804in}}%
\pgfpathcurveto{\pgfqpoint{3.974389in}{1.857990in}}{\pgfqpoint{3.969999in}{1.847391in}}{\pgfqpoint{3.969999in}{1.836341in}}%
\pgfpathcurveto{\pgfqpoint{3.969999in}{1.825291in}}{\pgfqpoint{3.974389in}{1.814692in}}{\pgfqpoint{3.982203in}{1.806878in}}%
\pgfpathcurveto{\pgfqpoint{3.990016in}{1.799064in}}{\pgfqpoint{4.000616in}{1.794674in}}{\pgfqpoint{4.011666in}{1.794674in}}%
\pgfpathclose%
\pgfusepath{stroke,fill}%
\end{pgfscope}%
\begin{pgfscope}%
\pgfpathrectangle{\pgfqpoint{0.800000in}{0.528000in}}{\pgfqpoint{4.960000in}{3.696000in}}%
\pgfusepath{clip}%
\pgfsetbuttcap%
\pgfsetroundjoin%
\definecolor{currentfill}{rgb}{0.000000,0.000000,0.000000}%
\pgfsetfillcolor{currentfill}%
\pgfsetlinewidth{1.003750pt}%
\definecolor{currentstroke}{rgb}{0.000000,0.000000,0.000000}%
\pgfsetstrokecolor{currentstroke}%
\pgfsetdash{}{0pt}%
\pgfpathmoveto{\pgfqpoint{4.011666in}{1.837669in}}%
\pgfpathcurveto{\pgfqpoint{4.022716in}{1.837669in}}{\pgfqpoint{4.033315in}{1.842059in}}{\pgfqpoint{4.041128in}{1.849873in}}%
\pgfpathcurveto{\pgfqpoint{4.048942in}{1.857687in}}{\pgfqpoint{4.053332in}{1.868286in}}{\pgfqpoint{4.053332in}{1.879336in}}%
\pgfpathcurveto{\pgfqpoint{4.053332in}{1.890386in}}{\pgfqpoint{4.048942in}{1.900985in}}{\pgfqpoint{4.041128in}{1.908798in}}%
\pgfpathcurveto{\pgfqpoint{4.033315in}{1.916612in}}{\pgfqpoint{4.022716in}{1.921002in}}{\pgfqpoint{4.011666in}{1.921002in}}%
\pgfpathcurveto{\pgfqpoint{4.000616in}{1.921002in}}{\pgfqpoint{3.990016in}{1.916612in}}{\pgfqpoint{3.982203in}{1.908798in}}%
\pgfpathcurveto{\pgfqpoint{3.974389in}{1.900985in}}{\pgfqpoint{3.969999in}{1.890386in}}{\pgfqpoint{3.969999in}{1.879336in}}%
\pgfpathcurveto{\pgfqpoint{3.969999in}{1.868286in}}{\pgfqpoint{3.974389in}{1.857687in}}{\pgfqpoint{3.982203in}{1.849873in}}%
\pgfpathcurveto{\pgfqpoint{3.990016in}{1.842059in}}{\pgfqpoint{4.000616in}{1.837669in}}{\pgfqpoint{4.011666in}{1.837669in}}%
\pgfpathclose%
\pgfusepath{stroke,fill}%
\end{pgfscope}%
\begin{pgfscope}%
\pgfpathrectangle{\pgfqpoint{0.800000in}{0.528000in}}{\pgfqpoint{4.960000in}{3.696000in}}%
\pgfusepath{clip}%
\pgfsetbuttcap%
\pgfsetroundjoin%
\definecolor{currentfill}{rgb}{0.000000,0.000000,0.000000}%
\pgfsetfillcolor{currentfill}%
\pgfsetlinewidth{1.003750pt}%
\definecolor{currentstroke}{rgb}{0.000000,0.000000,0.000000}%
\pgfsetstrokecolor{currentstroke}%
\pgfsetdash{}{0pt}%
\pgfpathmoveto{\pgfqpoint{4.011666in}{1.794674in}}%
\pgfpathcurveto{\pgfqpoint{4.022716in}{1.794674in}}{\pgfqpoint{4.033315in}{1.799064in}}{\pgfqpoint{4.041128in}{1.806878in}}%
\pgfpathcurveto{\pgfqpoint{4.048942in}{1.814692in}}{\pgfqpoint{4.053332in}{1.825291in}}{\pgfqpoint{4.053332in}{1.836341in}}%
\pgfpathcurveto{\pgfqpoint{4.053332in}{1.847391in}}{\pgfqpoint{4.048942in}{1.857990in}}{\pgfqpoint{4.041128in}{1.865804in}}%
\pgfpathcurveto{\pgfqpoint{4.033315in}{1.873617in}}{\pgfqpoint{4.022716in}{1.878007in}}{\pgfqpoint{4.011666in}{1.878007in}}%
\pgfpathcurveto{\pgfqpoint{4.000616in}{1.878007in}}{\pgfqpoint{3.990016in}{1.873617in}}{\pgfqpoint{3.982203in}{1.865804in}}%
\pgfpathcurveto{\pgfqpoint{3.974389in}{1.857990in}}{\pgfqpoint{3.969999in}{1.847391in}}{\pgfqpoint{3.969999in}{1.836341in}}%
\pgfpathcurveto{\pgfqpoint{3.969999in}{1.825291in}}{\pgfqpoint{3.974389in}{1.814692in}}{\pgfqpoint{3.982203in}{1.806878in}}%
\pgfpathcurveto{\pgfqpoint{3.990016in}{1.799064in}}{\pgfqpoint{4.000616in}{1.794674in}}{\pgfqpoint{4.011666in}{1.794674in}}%
\pgfpathclose%
\pgfusepath{stroke,fill}%
\end{pgfscope}%
\begin{pgfscope}%
\pgfpathrectangle{\pgfqpoint{0.800000in}{0.528000in}}{\pgfqpoint{4.960000in}{3.696000in}}%
\pgfusepath{clip}%
\pgfsetbuttcap%
\pgfsetroundjoin%
\definecolor{currentfill}{rgb}{0.000000,0.000000,0.000000}%
\pgfsetfillcolor{currentfill}%
\pgfsetlinewidth{1.003750pt}%
\definecolor{currentstroke}{rgb}{0.000000,0.000000,0.000000}%
\pgfsetstrokecolor{currentstroke}%
\pgfsetdash{}{0pt}%
\pgfpathmoveto{\pgfqpoint{4.011666in}{2.052644in}}%
\pgfpathcurveto{\pgfqpoint{4.022716in}{2.052644in}}{\pgfqpoint{4.033315in}{2.057034in}}{\pgfqpoint{4.041128in}{2.064848in}}%
\pgfpathcurveto{\pgfqpoint{4.048942in}{2.072661in}}{\pgfqpoint{4.053332in}{2.083261in}}{\pgfqpoint{4.053332in}{2.094311in}}%
\pgfpathcurveto{\pgfqpoint{4.053332in}{2.105361in}}{\pgfqpoint{4.048942in}{2.115960in}}{\pgfqpoint{4.041128in}{2.123773in}}%
\pgfpathcurveto{\pgfqpoint{4.033315in}{2.131587in}}{\pgfqpoint{4.022716in}{2.135977in}}{\pgfqpoint{4.011666in}{2.135977in}}%
\pgfpathcurveto{\pgfqpoint{4.000616in}{2.135977in}}{\pgfqpoint{3.990016in}{2.131587in}}{\pgfqpoint{3.982203in}{2.123773in}}%
\pgfpathcurveto{\pgfqpoint{3.974389in}{2.115960in}}{\pgfqpoint{3.969999in}{2.105361in}}{\pgfqpoint{3.969999in}{2.094311in}}%
\pgfpathcurveto{\pgfqpoint{3.969999in}{2.083261in}}{\pgfqpoint{3.974389in}{2.072661in}}{\pgfqpoint{3.982203in}{2.064848in}}%
\pgfpathcurveto{\pgfqpoint{3.990016in}{2.057034in}}{\pgfqpoint{4.000616in}{2.052644in}}{\pgfqpoint{4.011666in}{2.052644in}}%
\pgfpathclose%
\pgfusepath{stroke,fill}%
\end{pgfscope}%
\begin{pgfscope}%
\pgfpathrectangle{\pgfqpoint{0.800000in}{0.528000in}}{\pgfqpoint{4.960000in}{3.696000in}}%
\pgfusepath{clip}%
\pgfsetbuttcap%
\pgfsetroundjoin%
\definecolor{currentfill}{rgb}{0.000000,0.000000,0.000000}%
\pgfsetfillcolor{currentfill}%
\pgfsetlinewidth{1.003750pt}%
\definecolor{currentstroke}{rgb}{0.000000,0.000000,0.000000}%
\pgfsetstrokecolor{currentstroke}%
\pgfsetdash{}{0pt}%
\pgfpathmoveto{\pgfqpoint{4.011666in}{1.859167in}}%
\pgfpathcurveto{\pgfqpoint{4.022716in}{1.859167in}}{\pgfqpoint{4.033315in}{1.863557in}}{\pgfqpoint{4.041128in}{1.871370in}}%
\pgfpathcurveto{\pgfqpoint{4.048942in}{1.879184in}}{\pgfqpoint{4.053332in}{1.889783in}}{\pgfqpoint{4.053332in}{1.900833in}}%
\pgfpathcurveto{\pgfqpoint{4.053332in}{1.911883in}}{\pgfqpoint{4.048942in}{1.922482in}}{\pgfqpoint{4.041128in}{1.930296in}}%
\pgfpathcurveto{\pgfqpoint{4.033315in}{1.938110in}}{\pgfqpoint{4.022716in}{1.942500in}}{\pgfqpoint{4.011666in}{1.942500in}}%
\pgfpathcurveto{\pgfqpoint{4.000616in}{1.942500in}}{\pgfqpoint{3.990016in}{1.938110in}}{\pgfqpoint{3.982203in}{1.930296in}}%
\pgfpathcurveto{\pgfqpoint{3.974389in}{1.922482in}}{\pgfqpoint{3.969999in}{1.911883in}}{\pgfqpoint{3.969999in}{1.900833in}}%
\pgfpathcurveto{\pgfqpoint{3.969999in}{1.889783in}}{\pgfqpoint{3.974389in}{1.879184in}}{\pgfqpoint{3.982203in}{1.871370in}}%
\pgfpathcurveto{\pgfqpoint{3.990016in}{1.863557in}}{\pgfqpoint{4.000616in}{1.859167in}}{\pgfqpoint{4.011666in}{1.859167in}}%
\pgfpathclose%
\pgfusepath{stroke,fill}%
\end{pgfscope}%
\begin{pgfscope}%
\pgfpathrectangle{\pgfqpoint{0.800000in}{0.528000in}}{\pgfqpoint{4.960000in}{3.696000in}}%
\pgfusepath{clip}%
\pgfsetbuttcap%
\pgfsetroundjoin%
\definecolor{currentfill}{rgb}{0.000000,0.000000,0.000000}%
\pgfsetfillcolor{currentfill}%
\pgfsetlinewidth{1.003750pt}%
\definecolor{currentstroke}{rgb}{0.000000,0.000000,0.000000}%
\pgfsetstrokecolor{currentstroke}%
\pgfsetdash{}{0pt}%
\pgfpathmoveto{\pgfqpoint{4.011666in}{1.859167in}}%
\pgfpathcurveto{\pgfqpoint{4.022716in}{1.859167in}}{\pgfqpoint{4.033315in}{1.863557in}}{\pgfqpoint{4.041128in}{1.871370in}}%
\pgfpathcurveto{\pgfqpoint{4.048942in}{1.879184in}}{\pgfqpoint{4.053332in}{1.889783in}}{\pgfqpoint{4.053332in}{1.900833in}}%
\pgfpathcurveto{\pgfqpoint{4.053332in}{1.911883in}}{\pgfqpoint{4.048942in}{1.922482in}}{\pgfqpoint{4.041128in}{1.930296in}}%
\pgfpathcurveto{\pgfqpoint{4.033315in}{1.938110in}}{\pgfqpoint{4.022716in}{1.942500in}}{\pgfqpoint{4.011666in}{1.942500in}}%
\pgfpathcurveto{\pgfqpoint{4.000616in}{1.942500in}}{\pgfqpoint{3.990016in}{1.938110in}}{\pgfqpoint{3.982203in}{1.930296in}}%
\pgfpathcurveto{\pgfqpoint{3.974389in}{1.922482in}}{\pgfqpoint{3.969999in}{1.911883in}}{\pgfqpoint{3.969999in}{1.900833in}}%
\pgfpathcurveto{\pgfqpoint{3.969999in}{1.889783in}}{\pgfqpoint{3.974389in}{1.879184in}}{\pgfqpoint{3.982203in}{1.871370in}}%
\pgfpathcurveto{\pgfqpoint{3.990016in}{1.863557in}}{\pgfqpoint{4.000616in}{1.859167in}}{\pgfqpoint{4.011666in}{1.859167in}}%
\pgfpathclose%
\pgfusepath{stroke,fill}%
\end{pgfscope}%
\begin{pgfscope}%
\pgfpathrectangle{\pgfqpoint{0.800000in}{0.528000in}}{\pgfqpoint{4.960000in}{3.696000in}}%
\pgfusepath{clip}%
\pgfsetbuttcap%
\pgfsetroundjoin%
\definecolor{currentfill}{rgb}{0.000000,0.000000,0.000000}%
\pgfsetfillcolor{currentfill}%
\pgfsetlinewidth{1.003750pt}%
\definecolor{currentstroke}{rgb}{0.000000,0.000000,0.000000}%
\pgfsetstrokecolor{currentstroke}%
\pgfsetdash{}{0pt}%
\pgfpathmoveto{\pgfqpoint{4.011666in}{1.816172in}}%
\pgfpathcurveto{\pgfqpoint{4.022716in}{1.816172in}}{\pgfqpoint{4.033315in}{1.820562in}}{\pgfqpoint{4.041128in}{1.828375in}}%
\pgfpathcurveto{\pgfqpoint{4.048942in}{1.836189in}}{\pgfqpoint{4.053332in}{1.846788in}}{\pgfqpoint{4.053332in}{1.857838in}}%
\pgfpathcurveto{\pgfqpoint{4.053332in}{1.868888in}}{\pgfqpoint{4.048942in}{1.879487in}}{\pgfqpoint{4.041128in}{1.887301in}}%
\pgfpathcurveto{\pgfqpoint{4.033315in}{1.895115in}}{\pgfqpoint{4.022716in}{1.899505in}}{\pgfqpoint{4.011666in}{1.899505in}}%
\pgfpathcurveto{\pgfqpoint{4.000616in}{1.899505in}}{\pgfqpoint{3.990016in}{1.895115in}}{\pgfqpoint{3.982203in}{1.887301in}}%
\pgfpathcurveto{\pgfqpoint{3.974389in}{1.879487in}}{\pgfqpoint{3.969999in}{1.868888in}}{\pgfqpoint{3.969999in}{1.857838in}}%
\pgfpathcurveto{\pgfqpoint{3.969999in}{1.846788in}}{\pgfqpoint{3.974389in}{1.836189in}}{\pgfqpoint{3.982203in}{1.828375in}}%
\pgfpathcurveto{\pgfqpoint{3.990016in}{1.820562in}}{\pgfqpoint{4.000616in}{1.816172in}}{\pgfqpoint{4.011666in}{1.816172in}}%
\pgfpathclose%
\pgfusepath{stroke,fill}%
\end{pgfscope}%
\begin{pgfscope}%
\pgfpathrectangle{\pgfqpoint{0.800000in}{0.528000in}}{\pgfqpoint{4.960000in}{3.696000in}}%
\pgfusepath{clip}%
\pgfsetbuttcap%
\pgfsetroundjoin%
\definecolor{currentfill}{rgb}{0.000000,0.000000,0.000000}%
\pgfsetfillcolor{currentfill}%
\pgfsetlinewidth{1.003750pt}%
\definecolor{currentstroke}{rgb}{0.000000,0.000000,0.000000}%
\pgfsetstrokecolor{currentstroke}%
\pgfsetdash{}{0pt}%
\pgfpathmoveto{\pgfqpoint{4.011666in}{1.794674in}}%
\pgfpathcurveto{\pgfqpoint{4.022716in}{1.794674in}}{\pgfqpoint{4.033315in}{1.799064in}}{\pgfqpoint{4.041128in}{1.806878in}}%
\pgfpathcurveto{\pgfqpoint{4.048942in}{1.814692in}}{\pgfqpoint{4.053332in}{1.825291in}}{\pgfqpoint{4.053332in}{1.836341in}}%
\pgfpathcurveto{\pgfqpoint{4.053332in}{1.847391in}}{\pgfqpoint{4.048942in}{1.857990in}}{\pgfqpoint{4.041128in}{1.865804in}}%
\pgfpathcurveto{\pgfqpoint{4.033315in}{1.873617in}}{\pgfqpoint{4.022716in}{1.878007in}}{\pgfqpoint{4.011666in}{1.878007in}}%
\pgfpathcurveto{\pgfqpoint{4.000616in}{1.878007in}}{\pgfqpoint{3.990016in}{1.873617in}}{\pgfqpoint{3.982203in}{1.865804in}}%
\pgfpathcurveto{\pgfqpoint{3.974389in}{1.857990in}}{\pgfqpoint{3.969999in}{1.847391in}}{\pgfqpoint{3.969999in}{1.836341in}}%
\pgfpathcurveto{\pgfqpoint{3.969999in}{1.825291in}}{\pgfqpoint{3.974389in}{1.814692in}}{\pgfqpoint{3.982203in}{1.806878in}}%
\pgfpathcurveto{\pgfqpoint{3.990016in}{1.799064in}}{\pgfqpoint{4.000616in}{1.794674in}}{\pgfqpoint{4.011666in}{1.794674in}}%
\pgfpathclose%
\pgfusepath{stroke,fill}%
\end{pgfscope}%
\begin{pgfscope}%
\pgfpathrectangle{\pgfqpoint{0.800000in}{0.528000in}}{\pgfqpoint{4.960000in}{3.696000in}}%
\pgfusepath{clip}%
\pgfsetbuttcap%
\pgfsetroundjoin%
\definecolor{currentfill}{rgb}{0.000000,0.000000,0.000000}%
\pgfsetfillcolor{currentfill}%
\pgfsetlinewidth{1.003750pt}%
\definecolor{currentstroke}{rgb}{0.000000,0.000000,0.000000}%
\pgfsetstrokecolor{currentstroke}%
\pgfsetdash{}{0pt}%
\pgfpathmoveto{\pgfqpoint{4.011666in}{1.730182in}}%
\pgfpathcurveto{\pgfqpoint{4.022716in}{1.730182in}}{\pgfqpoint{4.033315in}{1.734572in}}{\pgfqpoint{4.041128in}{1.742385in}}%
\pgfpathcurveto{\pgfqpoint{4.048942in}{1.750199in}}{\pgfqpoint{4.053332in}{1.760798in}}{\pgfqpoint{4.053332in}{1.771848in}}%
\pgfpathcurveto{\pgfqpoint{4.053332in}{1.782898in}}{\pgfqpoint{4.048942in}{1.793497in}}{\pgfqpoint{4.041128in}{1.801311in}}%
\pgfpathcurveto{\pgfqpoint{4.033315in}{1.809125in}}{\pgfqpoint{4.022716in}{1.813515in}}{\pgfqpoint{4.011666in}{1.813515in}}%
\pgfpathcurveto{\pgfqpoint{4.000616in}{1.813515in}}{\pgfqpoint{3.990016in}{1.809125in}}{\pgfqpoint{3.982203in}{1.801311in}}%
\pgfpathcurveto{\pgfqpoint{3.974389in}{1.793497in}}{\pgfqpoint{3.969999in}{1.782898in}}{\pgfqpoint{3.969999in}{1.771848in}}%
\pgfpathcurveto{\pgfqpoint{3.969999in}{1.760798in}}{\pgfqpoint{3.974389in}{1.750199in}}{\pgfqpoint{3.982203in}{1.742385in}}%
\pgfpathcurveto{\pgfqpoint{3.990016in}{1.734572in}}{\pgfqpoint{4.000616in}{1.730182in}}{\pgfqpoint{4.011666in}{1.730182in}}%
\pgfpathclose%
\pgfusepath{stroke,fill}%
\end{pgfscope}%
\begin{pgfscope}%
\pgfpathrectangle{\pgfqpoint{0.800000in}{0.528000in}}{\pgfqpoint{4.960000in}{3.696000in}}%
\pgfusepath{clip}%
\pgfsetbuttcap%
\pgfsetroundjoin%
\definecolor{currentfill}{rgb}{0.000000,0.000000,0.000000}%
\pgfsetfillcolor{currentfill}%
\pgfsetlinewidth{1.003750pt}%
\definecolor{currentstroke}{rgb}{0.000000,0.000000,0.000000}%
\pgfsetstrokecolor{currentstroke}%
\pgfsetdash{}{0pt}%
\pgfpathmoveto{\pgfqpoint{4.011666in}{1.945157in}}%
\pgfpathcurveto{\pgfqpoint{4.022716in}{1.945157in}}{\pgfqpoint{4.033315in}{1.949547in}}{\pgfqpoint{4.041128in}{1.957360in}}%
\pgfpathcurveto{\pgfqpoint{4.048942in}{1.965174in}}{\pgfqpoint{4.053332in}{1.975773in}}{\pgfqpoint{4.053332in}{1.986823in}}%
\pgfpathcurveto{\pgfqpoint{4.053332in}{1.997873in}}{\pgfqpoint{4.048942in}{2.008472in}}{\pgfqpoint{4.041128in}{2.016286in}}%
\pgfpathcurveto{\pgfqpoint{4.033315in}{2.024100in}}{\pgfqpoint{4.022716in}{2.028490in}}{\pgfqpoint{4.011666in}{2.028490in}}%
\pgfpathcurveto{\pgfqpoint{4.000616in}{2.028490in}}{\pgfqpoint{3.990016in}{2.024100in}}{\pgfqpoint{3.982203in}{2.016286in}}%
\pgfpathcurveto{\pgfqpoint{3.974389in}{2.008472in}}{\pgfqpoint{3.969999in}{1.997873in}}{\pgfqpoint{3.969999in}{1.986823in}}%
\pgfpathcurveto{\pgfqpoint{3.969999in}{1.975773in}}{\pgfqpoint{3.974389in}{1.965174in}}{\pgfqpoint{3.982203in}{1.957360in}}%
\pgfpathcurveto{\pgfqpoint{3.990016in}{1.949547in}}{\pgfqpoint{4.000616in}{1.945157in}}{\pgfqpoint{4.011666in}{1.945157in}}%
\pgfpathclose%
\pgfusepath{stroke,fill}%
\end{pgfscope}%
\begin{pgfscope}%
\pgfpathrectangle{\pgfqpoint{0.800000in}{0.528000in}}{\pgfqpoint{4.960000in}{3.696000in}}%
\pgfusepath{clip}%
\pgfsetbuttcap%
\pgfsetroundjoin%
\definecolor{currentfill}{rgb}{0.000000,0.000000,0.000000}%
\pgfsetfillcolor{currentfill}%
\pgfsetlinewidth{1.003750pt}%
\definecolor{currentstroke}{rgb}{0.000000,0.000000,0.000000}%
\pgfsetstrokecolor{currentstroke}%
\pgfsetdash{}{0pt}%
\pgfpathmoveto{\pgfqpoint{4.011666in}{1.794674in}}%
\pgfpathcurveto{\pgfqpoint{4.022716in}{1.794674in}}{\pgfqpoint{4.033315in}{1.799064in}}{\pgfqpoint{4.041128in}{1.806878in}}%
\pgfpathcurveto{\pgfqpoint{4.048942in}{1.814692in}}{\pgfqpoint{4.053332in}{1.825291in}}{\pgfqpoint{4.053332in}{1.836341in}}%
\pgfpathcurveto{\pgfqpoint{4.053332in}{1.847391in}}{\pgfqpoint{4.048942in}{1.857990in}}{\pgfqpoint{4.041128in}{1.865804in}}%
\pgfpathcurveto{\pgfqpoint{4.033315in}{1.873617in}}{\pgfqpoint{4.022716in}{1.878007in}}{\pgfqpoint{4.011666in}{1.878007in}}%
\pgfpathcurveto{\pgfqpoint{4.000616in}{1.878007in}}{\pgfqpoint{3.990016in}{1.873617in}}{\pgfqpoint{3.982203in}{1.865804in}}%
\pgfpathcurveto{\pgfqpoint{3.974389in}{1.857990in}}{\pgfqpoint{3.969999in}{1.847391in}}{\pgfqpoint{3.969999in}{1.836341in}}%
\pgfpathcurveto{\pgfqpoint{3.969999in}{1.825291in}}{\pgfqpoint{3.974389in}{1.814692in}}{\pgfqpoint{3.982203in}{1.806878in}}%
\pgfpathcurveto{\pgfqpoint{3.990016in}{1.799064in}}{\pgfqpoint{4.000616in}{1.794674in}}{\pgfqpoint{4.011666in}{1.794674in}}%
\pgfpathclose%
\pgfusepath{stroke,fill}%
\end{pgfscope}%
\begin{pgfscope}%
\pgfpathrectangle{\pgfqpoint{0.800000in}{0.528000in}}{\pgfqpoint{4.960000in}{3.696000in}}%
\pgfusepath{clip}%
\pgfsetbuttcap%
\pgfsetroundjoin%
\definecolor{currentfill}{rgb}{0.000000,0.000000,0.000000}%
\pgfsetfillcolor{currentfill}%
\pgfsetlinewidth{1.003750pt}%
\definecolor{currentstroke}{rgb}{0.000000,0.000000,0.000000}%
\pgfsetstrokecolor{currentstroke}%
\pgfsetdash{}{0pt}%
\pgfpathmoveto{\pgfqpoint{4.011666in}{2.009649in}}%
\pgfpathcurveto{\pgfqpoint{4.022716in}{2.009649in}}{\pgfqpoint{4.033315in}{2.014039in}}{\pgfqpoint{4.041128in}{2.021853in}}%
\pgfpathcurveto{\pgfqpoint{4.048942in}{2.029666in}}{\pgfqpoint{4.053332in}{2.040266in}}{\pgfqpoint{4.053332in}{2.051316in}}%
\pgfpathcurveto{\pgfqpoint{4.053332in}{2.062366in}}{\pgfqpoint{4.048942in}{2.072965in}}{\pgfqpoint{4.041128in}{2.080778in}}%
\pgfpathcurveto{\pgfqpoint{4.033315in}{2.088592in}}{\pgfqpoint{4.022716in}{2.092982in}}{\pgfqpoint{4.011666in}{2.092982in}}%
\pgfpathcurveto{\pgfqpoint{4.000616in}{2.092982in}}{\pgfqpoint{3.990016in}{2.088592in}}{\pgfqpoint{3.982203in}{2.080778in}}%
\pgfpathcurveto{\pgfqpoint{3.974389in}{2.072965in}}{\pgfqpoint{3.969999in}{2.062366in}}{\pgfqpoint{3.969999in}{2.051316in}}%
\pgfpathcurveto{\pgfqpoint{3.969999in}{2.040266in}}{\pgfqpoint{3.974389in}{2.029666in}}{\pgfqpoint{3.982203in}{2.021853in}}%
\pgfpathcurveto{\pgfqpoint{3.990016in}{2.014039in}}{\pgfqpoint{4.000616in}{2.009649in}}{\pgfqpoint{4.011666in}{2.009649in}}%
\pgfpathclose%
\pgfusepath{stroke,fill}%
\end{pgfscope}%
\begin{pgfscope}%
\pgfpathrectangle{\pgfqpoint{0.800000in}{0.528000in}}{\pgfqpoint{4.960000in}{3.696000in}}%
\pgfusepath{clip}%
\pgfsetbuttcap%
\pgfsetroundjoin%
\definecolor{currentfill}{rgb}{0.000000,0.000000,0.000000}%
\pgfsetfillcolor{currentfill}%
\pgfsetlinewidth{1.003750pt}%
\definecolor{currentstroke}{rgb}{0.000000,0.000000,0.000000}%
\pgfsetstrokecolor{currentstroke}%
\pgfsetdash{}{0pt}%
\pgfpathmoveto{\pgfqpoint{4.011666in}{1.816172in}}%
\pgfpathcurveto{\pgfqpoint{4.022716in}{1.816172in}}{\pgfqpoint{4.033315in}{1.820562in}}{\pgfqpoint{4.041128in}{1.828375in}}%
\pgfpathcurveto{\pgfqpoint{4.048942in}{1.836189in}}{\pgfqpoint{4.053332in}{1.846788in}}{\pgfqpoint{4.053332in}{1.857838in}}%
\pgfpathcurveto{\pgfqpoint{4.053332in}{1.868888in}}{\pgfqpoint{4.048942in}{1.879487in}}{\pgfqpoint{4.041128in}{1.887301in}}%
\pgfpathcurveto{\pgfqpoint{4.033315in}{1.895115in}}{\pgfqpoint{4.022716in}{1.899505in}}{\pgfqpoint{4.011666in}{1.899505in}}%
\pgfpathcurveto{\pgfqpoint{4.000616in}{1.899505in}}{\pgfqpoint{3.990016in}{1.895115in}}{\pgfqpoint{3.982203in}{1.887301in}}%
\pgfpathcurveto{\pgfqpoint{3.974389in}{1.879487in}}{\pgfqpoint{3.969999in}{1.868888in}}{\pgfqpoint{3.969999in}{1.857838in}}%
\pgfpathcurveto{\pgfqpoint{3.969999in}{1.846788in}}{\pgfqpoint{3.974389in}{1.836189in}}{\pgfqpoint{3.982203in}{1.828375in}}%
\pgfpathcurveto{\pgfqpoint{3.990016in}{1.820562in}}{\pgfqpoint{4.000616in}{1.816172in}}{\pgfqpoint{4.011666in}{1.816172in}}%
\pgfpathclose%
\pgfusepath{stroke,fill}%
\end{pgfscope}%
\begin{pgfscope}%
\pgfpathrectangle{\pgfqpoint{0.800000in}{0.528000in}}{\pgfqpoint{4.960000in}{3.696000in}}%
\pgfusepath{clip}%
\pgfsetbuttcap%
\pgfsetroundjoin%
\definecolor{currentfill}{rgb}{0.000000,0.000000,0.000000}%
\pgfsetfillcolor{currentfill}%
\pgfsetlinewidth{1.003750pt}%
\definecolor{currentstroke}{rgb}{0.000000,0.000000,0.000000}%
\pgfsetstrokecolor{currentstroke}%
\pgfsetdash{}{0pt}%
\pgfpathmoveto{\pgfqpoint{4.011666in}{1.945157in}}%
\pgfpathcurveto{\pgfqpoint{4.022716in}{1.945157in}}{\pgfqpoint{4.033315in}{1.949547in}}{\pgfqpoint{4.041128in}{1.957360in}}%
\pgfpathcurveto{\pgfqpoint{4.048942in}{1.965174in}}{\pgfqpoint{4.053332in}{1.975773in}}{\pgfqpoint{4.053332in}{1.986823in}}%
\pgfpathcurveto{\pgfqpoint{4.053332in}{1.997873in}}{\pgfqpoint{4.048942in}{2.008472in}}{\pgfqpoint{4.041128in}{2.016286in}}%
\pgfpathcurveto{\pgfqpoint{4.033315in}{2.024100in}}{\pgfqpoint{4.022716in}{2.028490in}}{\pgfqpoint{4.011666in}{2.028490in}}%
\pgfpathcurveto{\pgfqpoint{4.000616in}{2.028490in}}{\pgfqpoint{3.990016in}{2.024100in}}{\pgfqpoint{3.982203in}{2.016286in}}%
\pgfpathcurveto{\pgfqpoint{3.974389in}{2.008472in}}{\pgfqpoint{3.969999in}{1.997873in}}{\pgfqpoint{3.969999in}{1.986823in}}%
\pgfpathcurveto{\pgfqpoint{3.969999in}{1.975773in}}{\pgfqpoint{3.974389in}{1.965174in}}{\pgfqpoint{3.982203in}{1.957360in}}%
\pgfpathcurveto{\pgfqpoint{3.990016in}{1.949547in}}{\pgfqpoint{4.000616in}{1.945157in}}{\pgfqpoint{4.011666in}{1.945157in}}%
\pgfpathclose%
\pgfusepath{stroke,fill}%
\end{pgfscope}%
\begin{pgfscope}%
\pgfpathrectangle{\pgfqpoint{0.800000in}{0.528000in}}{\pgfqpoint{4.960000in}{3.696000in}}%
\pgfusepath{clip}%
\pgfsetbuttcap%
\pgfsetroundjoin%
\definecolor{currentfill}{rgb}{0.000000,0.000000,0.000000}%
\pgfsetfillcolor{currentfill}%
\pgfsetlinewidth{1.003750pt}%
\definecolor{currentstroke}{rgb}{0.000000,0.000000,0.000000}%
\pgfsetstrokecolor{currentstroke}%
\pgfsetdash{}{0pt}%
\pgfpathmoveto{\pgfqpoint{4.011666in}{1.859167in}}%
\pgfpathcurveto{\pgfqpoint{4.022716in}{1.859167in}}{\pgfqpoint{4.033315in}{1.863557in}}{\pgfqpoint{4.041128in}{1.871370in}}%
\pgfpathcurveto{\pgfqpoint{4.048942in}{1.879184in}}{\pgfqpoint{4.053332in}{1.889783in}}{\pgfqpoint{4.053332in}{1.900833in}}%
\pgfpathcurveto{\pgfqpoint{4.053332in}{1.911883in}}{\pgfqpoint{4.048942in}{1.922482in}}{\pgfqpoint{4.041128in}{1.930296in}}%
\pgfpathcurveto{\pgfqpoint{4.033315in}{1.938110in}}{\pgfqpoint{4.022716in}{1.942500in}}{\pgfqpoint{4.011666in}{1.942500in}}%
\pgfpathcurveto{\pgfqpoint{4.000616in}{1.942500in}}{\pgfqpoint{3.990016in}{1.938110in}}{\pgfqpoint{3.982203in}{1.930296in}}%
\pgfpathcurveto{\pgfqpoint{3.974389in}{1.922482in}}{\pgfqpoint{3.969999in}{1.911883in}}{\pgfqpoint{3.969999in}{1.900833in}}%
\pgfpathcurveto{\pgfqpoint{3.969999in}{1.889783in}}{\pgfqpoint{3.974389in}{1.879184in}}{\pgfqpoint{3.982203in}{1.871370in}}%
\pgfpathcurveto{\pgfqpoint{3.990016in}{1.863557in}}{\pgfqpoint{4.000616in}{1.859167in}}{\pgfqpoint{4.011666in}{1.859167in}}%
\pgfpathclose%
\pgfusepath{stroke,fill}%
\end{pgfscope}%
\begin{pgfscope}%
\pgfpathrectangle{\pgfqpoint{0.800000in}{0.528000in}}{\pgfqpoint{4.960000in}{3.696000in}}%
\pgfusepath{clip}%
\pgfsetbuttcap%
\pgfsetroundjoin%
\definecolor{currentfill}{rgb}{0.000000,0.000000,0.000000}%
\pgfsetfillcolor{currentfill}%
\pgfsetlinewidth{1.003750pt}%
\definecolor{currentstroke}{rgb}{0.000000,0.000000,0.000000}%
\pgfsetstrokecolor{currentstroke}%
\pgfsetdash{}{0pt}%
\pgfpathmoveto{\pgfqpoint{4.011666in}{1.816172in}}%
\pgfpathcurveto{\pgfqpoint{4.022716in}{1.816172in}}{\pgfqpoint{4.033315in}{1.820562in}}{\pgfqpoint{4.041128in}{1.828375in}}%
\pgfpathcurveto{\pgfqpoint{4.048942in}{1.836189in}}{\pgfqpoint{4.053332in}{1.846788in}}{\pgfqpoint{4.053332in}{1.857838in}}%
\pgfpathcurveto{\pgfqpoint{4.053332in}{1.868888in}}{\pgfqpoint{4.048942in}{1.879487in}}{\pgfqpoint{4.041128in}{1.887301in}}%
\pgfpathcurveto{\pgfqpoint{4.033315in}{1.895115in}}{\pgfqpoint{4.022716in}{1.899505in}}{\pgfqpoint{4.011666in}{1.899505in}}%
\pgfpathcurveto{\pgfqpoint{4.000616in}{1.899505in}}{\pgfqpoint{3.990016in}{1.895115in}}{\pgfqpoint{3.982203in}{1.887301in}}%
\pgfpathcurveto{\pgfqpoint{3.974389in}{1.879487in}}{\pgfqpoint{3.969999in}{1.868888in}}{\pgfqpoint{3.969999in}{1.857838in}}%
\pgfpathcurveto{\pgfqpoint{3.969999in}{1.846788in}}{\pgfqpoint{3.974389in}{1.836189in}}{\pgfqpoint{3.982203in}{1.828375in}}%
\pgfpathcurveto{\pgfqpoint{3.990016in}{1.820562in}}{\pgfqpoint{4.000616in}{1.816172in}}{\pgfqpoint{4.011666in}{1.816172in}}%
\pgfpathclose%
\pgfusepath{stroke,fill}%
\end{pgfscope}%
\begin{pgfscope}%
\pgfpathrectangle{\pgfqpoint{0.800000in}{0.528000in}}{\pgfqpoint{4.960000in}{3.696000in}}%
\pgfusepath{clip}%
\pgfsetbuttcap%
\pgfsetroundjoin%
\definecolor{currentfill}{rgb}{0.000000,0.000000,0.000000}%
\pgfsetfillcolor{currentfill}%
\pgfsetlinewidth{1.003750pt}%
\definecolor{currentstroke}{rgb}{0.000000,0.000000,0.000000}%
\pgfsetstrokecolor{currentstroke}%
\pgfsetdash{}{0pt}%
\pgfpathmoveto{\pgfqpoint{4.011666in}{1.966654in}}%
\pgfpathcurveto{\pgfqpoint{4.022716in}{1.966654in}}{\pgfqpoint{4.033315in}{1.971044in}}{\pgfqpoint{4.041128in}{1.978858in}}%
\pgfpathcurveto{\pgfqpoint{4.048942in}{1.986672in}}{\pgfqpoint{4.053332in}{1.997271in}}{\pgfqpoint{4.053332in}{2.008321in}}%
\pgfpathcurveto{\pgfqpoint{4.053332in}{2.019371in}}{\pgfqpoint{4.048942in}{2.029970in}}{\pgfqpoint{4.041128in}{2.037783in}}%
\pgfpathcurveto{\pgfqpoint{4.033315in}{2.045597in}}{\pgfqpoint{4.022716in}{2.049987in}}{\pgfqpoint{4.011666in}{2.049987in}}%
\pgfpathcurveto{\pgfqpoint{4.000616in}{2.049987in}}{\pgfqpoint{3.990016in}{2.045597in}}{\pgfqpoint{3.982203in}{2.037783in}}%
\pgfpathcurveto{\pgfqpoint{3.974389in}{2.029970in}}{\pgfqpoint{3.969999in}{2.019371in}}{\pgfqpoint{3.969999in}{2.008321in}}%
\pgfpathcurveto{\pgfqpoint{3.969999in}{1.997271in}}{\pgfqpoint{3.974389in}{1.986672in}}{\pgfqpoint{3.982203in}{1.978858in}}%
\pgfpathcurveto{\pgfqpoint{3.990016in}{1.971044in}}{\pgfqpoint{4.000616in}{1.966654in}}{\pgfqpoint{4.011666in}{1.966654in}}%
\pgfpathclose%
\pgfusepath{stroke,fill}%
\end{pgfscope}%
\begin{pgfscope}%
\pgfpathrectangle{\pgfqpoint{0.800000in}{0.528000in}}{\pgfqpoint{4.960000in}{3.696000in}}%
\pgfusepath{clip}%
\pgfsetbuttcap%
\pgfsetroundjoin%
\definecolor{currentfill}{rgb}{0.000000,0.000000,0.000000}%
\pgfsetfillcolor{currentfill}%
\pgfsetlinewidth{1.003750pt}%
\definecolor{currentstroke}{rgb}{0.000000,0.000000,0.000000}%
\pgfsetstrokecolor{currentstroke}%
\pgfsetdash{}{0pt}%
\pgfpathmoveto{\pgfqpoint{4.011666in}{2.052644in}}%
\pgfpathcurveto{\pgfqpoint{4.022716in}{2.052644in}}{\pgfqpoint{4.033315in}{2.057034in}}{\pgfqpoint{4.041128in}{2.064848in}}%
\pgfpathcurveto{\pgfqpoint{4.048942in}{2.072661in}}{\pgfqpoint{4.053332in}{2.083261in}}{\pgfqpoint{4.053332in}{2.094311in}}%
\pgfpathcurveto{\pgfqpoint{4.053332in}{2.105361in}}{\pgfqpoint{4.048942in}{2.115960in}}{\pgfqpoint{4.041128in}{2.123773in}}%
\pgfpathcurveto{\pgfqpoint{4.033315in}{2.131587in}}{\pgfqpoint{4.022716in}{2.135977in}}{\pgfqpoint{4.011666in}{2.135977in}}%
\pgfpathcurveto{\pgfqpoint{4.000616in}{2.135977in}}{\pgfqpoint{3.990016in}{2.131587in}}{\pgfqpoint{3.982203in}{2.123773in}}%
\pgfpathcurveto{\pgfqpoint{3.974389in}{2.115960in}}{\pgfqpoint{3.969999in}{2.105361in}}{\pgfqpoint{3.969999in}{2.094311in}}%
\pgfpathcurveto{\pgfqpoint{3.969999in}{2.083261in}}{\pgfqpoint{3.974389in}{2.072661in}}{\pgfqpoint{3.982203in}{2.064848in}}%
\pgfpathcurveto{\pgfqpoint{3.990016in}{2.057034in}}{\pgfqpoint{4.000616in}{2.052644in}}{\pgfqpoint{4.011666in}{2.052644in}}%
\pgfpathclose%
\pgfusepath{stroke,fill}%
\end{pgfscope}%
\begin{pgfscope}%
\pgfpathrectangle{\pgfqpoint{0.800000in}{0.528000in}}{\pgfqpoint{4.960000in}{3.696000in}}%
\pgfusepath{clip}%
\pgfsetbuttcap%
\pgfsetroundjoin%
\definecolor{currentfill}{rgb}{0.000000,0.000000,0.000000}%
\pgfsetfillcolor{currentfill}%
\pgfsetlinewidth{1.003750pt}%
\definecolor{currentstroke}{rgb}{0.000000,0.000000,0.000000}%
\pgfsetstrokecolor{currentstroke}%
\pgfsetdash{}{0pt}%
\pgfpathmoveto{\pgfqpoint{4.011666in}{1.880664in}}%
\pgfpathcurveto{\pgfqpoint{4.022716in}{1.880664in}}{\pgfqpoint{4.033315in}{1.885054in}}{\pgfqpoint{4.041128in}{1.892868in}}%
\pgfpathcurveto{\pgfqpoint{4.048942in}{1.900682in}}{\pgfqpoint{4.053332in}{1.911281in}}{\pgfqpoint{4.053332in}{1.922331in}}%
\pgfpathcurveto{\pgfqpoint{4.053332in}{1.933381in}}{\pgfqpoint{4.048942in}{1.943980in}}{\pgfqpoint{4.041128in}{1.951793in}}%
\pgfpathcurveto{\pgfqpoint{4.033315in}{1.959607in}}{\pgfqpoint{4.022716in}{1.963997in}}{\pgfqpoint{4.011666in}{1.963997in}}%
\pgfpathcurveto{\pgfqpoint{4.000616in}{1.963997in}}{\pgfqpoint{3.990016in}{1.959607in}}{\pgfqpoint{3.982203in}{1.951793in}}%
\pgfpathcurveto{\pgfqpoint{3.974389in}{1.943980in}}{\pgfqpoint{3.969999in}{1.933381in}}{\pgfqpoint{3.969999in}{1.922331in}}%
\pgfpathcurveto{\pgfqpoint{3.969999in}{1.911281in}}{\pgfqpoint{3.974389in}{1.900682in}}{\pgfqpoint{3.982203in}{1.892868in}}%
\pgfpathcurveto{\pgfqpoint{3.990016in}{1.885054in}}{\pgfqpoint{4.000616in}{1.880664in}}{\pgfqpoint{4.011666in}{1.880664in}}%
\pgfpathclose%
\pgfusepath{stroke,fill}%
\end{pgfscope}%
\begin{pgfscope}%
\pgfpathrectangle{\pgfqpoint{0.800000in}{0.528000in}}{\pgfqpoint{4.960000in}{3.696000in}}%
\pgfusepath{clip}%
\pgfsetbuttcap%
\pgfsetroundjoin%
\definecolor{currentfill}{rgb}{0.000000,0.000000,0.000000}%
\pgfsetfillcolor{currentfill}%
\pgfsetlinewidth{1.003750pt}%
\definecolor{currentstroke}{rgb}{0.000000,0.000000,0.000000}%
\pgfsetstrokecolor{currentstroke}%
\pgfsetdash{}{0pt}%
\pgfpathmoveto{\pgfqpoint{4.011666in}{1.902162in}}%
\pgfpathcurveto{\pgfqpoint{4.022716in}{1.902162in}}{\pgfqpoint{4.033315in}{1.906552in}}{\pgfqpoint{4.041128in}{1.914365in}}%
\pgfpathcurveto{\pgfqpoint{4.048942in}{1.922179in}}{\pgfqpoint{4.053332in}{1.932778in}}{\pgfqpoint{4.053332in}{1.943828in}}%
\pgfpathcurveto{\pgfqpoint{4.053332in}{1.954878in}}{\pgfqpoint{4.048942in}{1.965477in}}{\pgfqpoint{4.041128in}{1.973291in}}%
\pgfpathcurveto{\pgfqpoint{4.033315in}{1.981105in}}{\pgfqpoint{4.022716in}{1.985495in}}{\pgfqpoint{4.011666in}{1.985495in}}%
\pgfpathcurveto{\pgfqpoint{4.000616in}{1.985495in}}{\pgfqpoint{3.990016in}{1.981105in}}{\pgfqpoint{3.982203in}{1.973291in}}%
\pgfpathcurveto{\pgfqpoint{3.974389in}{1.965477in}}{\pgfqpoint{3.969999in}{1.954878in}}{\pgfqpoint{3.969999in}{1.943828in}}%
\pgfpathcurveto{\pgfqpoint{3.969999in}{1.932778in}}{\pgfqpoint{3.974389in}{1.922179in}}{\pgfqpoint{3.982203in}{1.914365in}}%
\pgfpathcurveto{\pgfqpoint{3.990016in}{1.906552in}}{\pgfqpoint{4.000616in}{1.902162in}}{\pgfqpoint{4.011666in}{1.902162in}}%
\pgfpathclose%
\pgfusepath{stroke,fill}%
\end{pgfscope}%
\begin{pgfscope}%
\pgfpathrectangle{\pgfqpoint{0.800000in}{0.528000in}}{\pgfqpoint{4.960000in}{3.696000in}}%
\pgfusepath{clip}%
\pgfsetbuttcap%
\pgfsetroundjoin%
\definecolor{currentfill}{rgb}{0.000000,0.000000,0.000000}%
\pgfsetfillcolor{currentfill}%
\pgfsetlinewidth{1.003750pt}%
\definecolor{currentstroke}{rgb}{0.000000,0.000000,0.000000}%
\pgfsetstrokecolor{currentstroke}%
\pgfsetdash{}{0pt}%
\pgfpathmoveto{\pgfqpoint{4.011666in}{1.945157in}}%
\pgfpathcurveto{\pgfqpoint{4.022716in}{1.945157in}}{\pgfqpoint{4.033315in}{1.949547in}}{\pgfqpoint{4.041128in}{1.957360in}}%
\pgfpathcurveto{\pgfqpoint{4.048942in}{1.965174in}}{\pgfqpoint{4.053332in}{1.975773in}}{\pgfqpoint{4.053332in}{1.986823in}}%
\pgfpathcurveto{\pgfqpoint{4.053332in}{1.997873in}}{\pgfqpoint{4.048942in}{2.008472in}}{\pgfqpoint{4.041128in}{2.016286in}}%
\pgfpathcurveto{\pgfqpoint{4.033315in}{2.024100in}}{\pgfqpoint{4.022716in}{2.028490in}}{\pgfqpoint{4.011666in}{2.028490in}}%
\pgfpathcurveto{\pgfqpoint{4.000616in}{2.028490in}}{\pgfqpoint{3.990016in}{2.024100in}}{\pgfqpoint{3.982203in}{2.016286in}}%
\pgfpathcurveto{\pgfqpoint{3.974389in}{2.008472in}}{\pgfqpoint{3.969999in}{1.997873in}}{\pgfqpoint{3.969999in}{1.986823in}}%
\pgfpathcurveto{\pgfqpoint{3.969999in}{1.975773in}}{\pgfqpoint{3.974389in}{1.965174in}}{\pgfqpoint{3.982203in}{1.957360in}}%
\pgfpathcurveto{\pgfqpoint{3.990016in}{1.949547in}}{\pgfqpoint{4.000616in}{1.945157in}}{\pgfqpoint{4.011666in}{1.945157in}}%
\pgfpathclose%
\pgfusepath{stroke,fill}%
\end{pgfscope}%
\begin{pgfscope}%
\pgfpathrectangle{\pgfqpoint{0.800000in}{0.528000in}}{\pgfqpoint{4.960000in}{3.696000in}}%
\pgfusepath{clip}%
\pgfsetbuttcap%
\pgfsetroundjoin%
\definecolor{currentfill}{rgb}{0.000000,0.000000,0.000000}%
\pgfsetfillcolor{currentfill}%
\pgfsetlinewidth{1.003750pt}%
\definecolor{currentstroke}{rgb}{0.000000,0.000000,0.000000}%
\pgfsetstrokecolor{currentstroke}%
\pgfsetdash{}{0pt}%
\pgfpathmoveto{\pgfqpoint{4.011666in}{1.988151in}}%
\pgfpathcurveto{\pgfqpoint{4.022716in}{1.988151in}}{\pgfqpoint{4.033315in}{1.992542in}}{\pgfqpoint{4.041128in}{2.000355in}}%
\pgfpathcurveto{\pgfqpoint{4.048942in}{2.008169in}}{\pgfqpoint{4.053332in}{2.018768in}}{\pgfqpoint{4.053332in}{2.029818in}}%
\pgfpathcurveto{\pgfqpoint{4.053332in}{2.040868in}}{\pgfqpoint{4.048942in}{2.051467in}}{\pgfqpoint{4.041128in}{2.059281in}}%
\pgfpathcurveto{\pgfqpoint{4.033315in}{2.067095in}}{\pgfqpoint{4.022716in}{2.071485in}}{\pgfqpoint{4.011666in}{2.071485in}}%
\pgfpathcurveto{\pgfqpoint{4.000616in}{2.071485in}}{\pgfqpoint{3.990016in}{2.067095in}}{\pgfqpoint{3.982203in}{2.059281in}}%
\pgfpathcurveto{\pgfqpoint{3.974389in}{2.051467in}}{\pgfqpoint{3.969999in}{2.040868in}}{\pgfqpoint{3.969999in}{2.029818in}}%
\pgfpathcurveto{\pgfqpoint{3.969999in}{2.018768in}}{\pgfqpoint{3.974389in}{2.008169in}}{\pgfqpoint{3.982203in}{2.000355in}}%
\pgfpathcurveto{\pgfqpoint{3.990016in}{1.992542in}}{\pgfqpoint{4.000616in}{1.988151in}}{\pgfqpoint{4.011666in}{1.988151in}}%
\pgfpathclose%
\pgfusepath{stroke,fill}%
\end{pgfscope}%
\begin{pgfscope}%
\pgfpathrectangle{\pgfqpoint{0.800000in}{0.528000in}}{\pgfqpoint{4.960000in}{3.696000in}}%
\pgfusepath{clip}%
\pgfsetbuttcap%
\pgfsetroundjoin%
\definecolor{currentfill}{rgb}{0.000000,0.000000,0.000000}%
\pgfsetfillcolor{currentfill}%
\pgfsetlinewidth{1.003750pt}%
\definecolor{currentstroke}{rgb}{0.000000,0.000000,0.000000}%
\pgfsetstrokecolor{currentstroke}%
\pgfsetdash{}{0pt}%
\pgfpathmoveto{\pgfqpoint{4.011666in}{1.837669in}}%
\pgfpathcurveto{\pgfqpoint{4.022716in}{1.837669in}}{\pgfqpoint{4.033315in}{1.842059in}}{\pgfqpoint{4.041128in}{1.849873in}}%
\pgfpathcurveto{\pgfqpoint{4.048942in}{1.857687in}}{\pgfqpoint{4.053332in}{1.868286in}}{\pgfqpoint{4.053332in}{1.879336in}}%
\pgfpathcurveto{\pgfqpoint{4.053332in}{1.890386in}}{\pgfqpoint{4.048942in}{1.900985in}}{\pgfqpoint{4.041128in}{1.908798in}}%
\pgfpathcurveto{\pgfqpoint{4.033315in}{1.916612in}}{\pgfqpoint{4.022716in}{1.921002in}}{\pgfqpoint{4.011666in}{1.921002in}}%
\pgfpathcurveto{\pgfqpoint{4.000616in}{1.921002in}}{\pgfqpoint{3.990016in}{1.916612in}}{\pgfqpoint{3.982203in}{1.908798in}}%
\pgfpathcurveto{\pgfqpoint{3.974389in}{1.900985in}}{\pgfqpoint{3.969999in}{1.890386in}}{\pgfqpoint{3.969999in}{1.879336in}}%
\pgfpathcurveto{\pgfqpoint{3.969999in}{1.868286in}}{\pgfqpoint{3.974389in}{1.857687in}}{\pgfqpoint{3.982203in}{1.849873in}}%
\pgfpathcurveto{\pgfqpoint{3.990016in}{1.842059in}}{\pgfqpoint{4.000616in}{1.837669in}}{\pgfqpoint{4.011666in}{1.837669in}}%
\pgfpathclose%
\pgfusepath{stroke,fill}%
\end{pgfscope}%
\begin{pgfscope}%
\pgfpathrectangle{\pgfqpoint{0.800000in}{0.528000in}}{\pgfqpoint{4.960000in}{3.696000in}}%
\pgfusepath{clip}%
\pgfsetbuttcap%
\pgfsetroundjoin%
\definecolor{currentfill}{rgb}{0.000000,0.000000,0.000000}%
\pgfsetfillcolor{currentfill}%
\pgfsetlinewidth{1.003750pt}%
\definecolor{currentstroke}{rgb}{0.000000,0.000000,0.000000}%
\pgfsetstrokecolor{currentstroke}%
\pgfsetdash{}{0pt}%
\pgfpathmoveto{\pgfqpoint{4.011666in}{1.794674in}}%
\pgfpathcurveto{\pgfqpoint{4.022716in}{1.794674in}}{\pgfqpoint{4.033315in}{1.799064in}}{\pgfqpoint{4.041128in}{1.806878in}}%
\pgfpathcurveto{\pgfqpoint{4.048942in}{1.814692in}}{\pgfqpoint{4.053332in}{1.825291in}}{\pgfqpoint{4.053332in}{1.836341in}}%
\pgfpathcurveto{\pgfqpoint{4.053332in}{1.847391in}}{\pgfqpoint{4.048942in}{1.857990in}}{\pgfqpoint{4.041128in}{1.865804in}}%
\pgfpathcurveto{\pgfqpoint{4.033315in}{1.873617in}}{\pgfqpoint{4.022716in}{1.878007in}}{\pgfqpoint{4.011666in}{1.878007in}}%
\pgfpathcurveto{\pgfqpoint{4.000616in}{1.878007in}}{\pgfqpoint{3.990016in}{1.873617in}}{\pgfqpoint{3.982203in}{1.865804in}}%
\pgfpathcurveto{\pgfqpoint{3.974389in}{1.857990in}}{\pgfqpoint{3.969999in}{1.847391in}}{\pgfqpoint{3.969999in}{1.836341in}}%
\pgfpathcurveto{\pgfqpoint{3.969999in}{1.825291in}}{\pgfqpoint{3.974389in}{1.814692in}}{\pgfqpoint{3.982203in}{1.806878in}}%
\pgfpathcurveto{\pgfqpoint{3.990016in}{1.799064in}}{\pgfqpoint{4.000616in}{1.794674in}}{\pgfqpoint{4.011666in}{1.794674in}}%
\pgfpathclose%
\pgfusepath{stroke,fill}%
\end{pgfscope}%
\begin{pgfscope}%
\pgfpathrectangle{\pgfqpoint{0.800000in}{0.528000in}}{\pgfqpoint{4.960000in}{3.696000in}}%
\pgfusepath{clip}%
\pgfsetbuttcap%
\pgfsetroundjoin%
\definecolor{currentfill}{rgb}{0.000000,0.000000,0.000000}%
\pgfsetfillcolor{currentfill}%
\pgfsetlinewidth{1.003750pt}%
\definecolor{currentstroke}{rgb}{0.000000,0.000000,0.000000}%
\pgfsetstrokecolor{currentstroke}%
\pgfsetdash{}{0pt}%
\pgfpathmoveto{\pgfqpoint{4.011666in}{1.923659in}}%
\pgfpathcurveto{\pgfqpoint{4.022716in}{1.923659in}}{\pgfqpoint{4.033315in}{1.928049in}}{\pgfqpoint{4.041128in}{1.935863in}}%
\pgfpathcurveto{\pgfqpoint{4.048942in}{1.943677in}}{\pgfqpoint{4.053332in}{1.954276in}}{\pgfqpoint{4.053332in}{1.965326in}}%
\pgfpathcurveto{\pgfqpoint{4.053332in}{1.976376in}}{\pgfqpoint{4.048942in}{1.986975in}}{\pgfqpoint{4.041128in}{1.994788in}}%
\pgfpathcurveto{\pgfqpoint{4.033315in}{2.002602in}}{\pgfqpoint{4.022716in}{2.006992in}}{\pgfqpoint{4.011666in}{2.006992in}}%
\pgfpathcurveto{\pgfqpoint{4.000616in}{2.006992in}}{\pgfqpoint{3.990016in}{2.002602in}}{\pgfqpoint{3.982203in}{1.994788in}}%
\pgfpathcurveto{\pgfqpoint{3.974389in}{1.986975in}}{\pgfqpoint{3.969999in}{1.976376in}}{\pgfqpoint{3.969999in}{1.965326in}}%
\pgfpathcurveto{\pgfqpoint{3.969999in}{1.954276in}}{\pgfqpoint{3.974389in}{1.943677in}}{\pgfqpoint{3.982203in}{1.935863in}}%
\pgfpathcurveto{\pgfqpoint{3.990016in}{1.928049in}}{\pgfqpoint{4.000616in}{1.923659in}}{\pgfqpoint{4.011666in}{1.923659in}}%
\pgfpathclose%
\pgfusepath{stroke,fill}%
\end{pgfscope}%
\begin{pgfscope}%
\pgfpathrectangle{\pgfqpoint{0.800000in}{0.528000in}}{\pgfqpoint{4.960000in}{3.696000in}}%
\pgfusepath{clip}%
\pgfsetbuttcap%
\pgfsetroundjoin%
\definecolor{currentfill}{rgb}{0.000000,0.000000,0.000000}%
\pgfsetfillcolor{currentfill}%
\pgfsetlinewidth{1.003750pt}%
\definecolor{currentstroke}{rgb}{0.000000,0.000000,0.000000}%
\pgfsetstrokecolor{currentstroke}%
\pgfsetdash{}{0pt}%
\pgfpathmoveto{\pgfqpoint{4.011666in}{2.009649in}}%
\pgfpathcurveto{\pgfqpoint{4.022716in}{2.009649in}}{\pgfqpoint{4.033315in}{2.014039in}}{\pgfqpoint{4.041128in}{2.021853in}}%
\pgfpathcurveto{\pgfqpoint{4.048942in}{2.029666in}}{\pgfqpoint{4.053332in}{2.040266in}}{\pgfqpoint{4.053332in}{2.051316in}}%
\pgfpathcurveto{\pgfqpoint{4.053332in}{2.062366in}}{\pgfqpoint{4.048942in}{2.072965in}}{\pgfqpoint{4.041128in}{2.080778in}}%
\pgfpathcurveto{\pgfqpoint{4.033315in}{2.088592in}}{\pgfqpoint{4.022716in}{2.092982in}}{\pgfqpoint{4.011666in}{2.092982in}}%
\pgfpathcurveto{\pgfqpoint{4.000616in}{2.092982in}}{\pgfqpoint{3.990016in}{2.088592in}}{\pgfqpoint{3.982203in}{2.080778in}}%
\pgfpathcurveto{\pgfqpoint{3.974389in}{2.072965in}}{\pgfqpoint{3.969999in}{2.062366in}}{\pgfqpoint{3.969999in}{2.051316in}}%
\pgfpathcurveto{\pgfqpoint{3.969999in}{2.040266in}}{\pgfqpoint{3.974389in}{2.029666in}}{\pgfqpoint{3.982203in}{2.021853in}}%
\pgfpathcurveto{\pgfqpoint{3.990016in}{2.014039in}}{\pgfqpoint{4.000616in}{2.009649in}}{\pgfqpoint{4.011666in}{2.009649in}}%
\pgfpathclose%
\pgfusepath{stroke,fill}%
\end{pgfscope}%
\begin{pgfscope}%
\pgfpathrectangle{\pgfqpoint{0.800000in}{0.528000in}}{\pgfqpoint{4.960000in}{3.696000in}}%
\pgfusepath{clip}%
\pgfsetbuttcap%
\pgfsetroundjoin%
\definecolor{currentfill}{rgb}{0.000000,0.000000,0.000000}%
\pgfsetfillcolor{currentfill}%
\pgfsetlinewidth{1.003750pt}%
\definecolor{currentstroke}{rgb}{0.000000,0.000000,0.000000}%
\pgfsetstrokecolor{currentstroke}%
\pgfsetdash{}{0pt}%
\pgfpathmoveto{\pgfqpoint{5.504545in}{2.912544in}}%
\pgfpathcurveto{\pgfqpoint{5.515596in}{2.912544in}}{\pgfqpoint{5.526195in}{2.916934in}}{\pgfqpoint{5.534008in}{2.924748in}}%
\pgfpathcurveto{\pgfqpoint{5.541822in}{2.932561in}}{\pgfqpoint{5.546212in}{2.943160in}}{\pgfqpoint{5.546212in}{2.954210in}}%
\pgfpathcurveto{\pgfqpoint{5.546212in}{2.965260in}}{\pgfqpoint{5.541822in}{2.975859in}}{\pgfqpoint{5.534008in}{2.983673in}}%
\pgfpathcurveto{\pgfqpoint{5.526195in}{2.991487in}}{\pgfqpoint{5.515596in}{2.995877in}}{\pgfqpoint{5.504545in}{2.995877in}}%
\pgfpathcurveto{\pgfqpoint{5.493495in}{2.995877in}}{\pgfqpoint{5.482896in}{2.991487in}}{\pgfqpoint{5.475083in}{2.983673in}}%
\pgfpathcurveto{\pgfqpoint{5.467269in}{2.975859in}}{\pgfqpoint{5.462879in}{2.965260in}}{\pgfqpoint{5.462879in}{2.954210in}}%
\pgfpathcurveto{\pgfqpoint{5.462879in}{2.943160in}}{\pgfqpoint{5.467269in}{2.932561in}}{\pgfqpoint{5.475083in}{2.924748in}}%
\pgfpathcurveto{\pgfqpoint{5.482896in}{2.916934in}}{\pgfqpoint{5.493495in}{2.912544in}}{\pgfqpoint{5.504545in}{2.912544in}}%
\pgfpathclose%
\pgfusepath{stroke,fill}%
\end{pgfscope}%
\begin{pgfscope}%
\pgfpathrectangle{\pgfqpoint{0.800000in}{0.528000in}}{\pgfqpoint{4.960000in}{3.696000in}}%
\pgfusepath{clip}%
\pgfsetbuttcap%
\pgfsetroundjoin%
\definecolor{currentfill}{rgb}{0.000000,0.000000,0.000000}%
\pgfsetfillcolor{currentfill}%
\pgfsetlinewidth{1.003750pt}%
\definecolor{currentstroke}{rgb}{0.000000,0.000000,0.000000}%
\pgfsetstrokecolor{currentstroke}%
\pgfsetdash{}{0pt}%
\pgfpathmoveto{\pgfqpoint{5.504545in}{2.762061in}}%
\pgfpathcurveto{\pgfqpoint{5.515596in}{2.762061in}}{\pgfqpoint{5.526195in}{2.766451in}}{\pgfqpoint{5.534008in}{2.774265in}}%
\pgfpathcurveto{\pgfqpoint{5.541822in}{2.782079in}}{\pgfqpoint{5.546212in}{2.792678in}}{\pgfqpoint{5.546212in}{2.803728in}}%
\pgfpathcurveto{\pgfqpoint{5.546212in}{2.814778in}}{\pgfqpoint{5.541822in}{2.825377in}}{\pgfqpoint{5.534008in}{2.833191in}}%
\pgfpathcurveto{\pgfqpoint{5.526195in}{2.841004in}}{\pgfqpoint{5.515596in}{2.845395in}}{\pgfqpoint{5.504545in}{2.845395in}}%
\pgfpathcurveto{\pgfqpoint{5.493495in}{2.845395in}}{\pgfqpoint{5.482896in}{2.841004in}}{\pgfqpoint{5.475083in}{2.833191in}}%
\pgfpathcurveto{\pgfqpoint{5.467269in}{2.825377in}}{\pgfqpoint{5.462879in}{2.814778in}}{\pgfqpoint{5.462879in}{2.803728in}}%
\pgfpathcurveto{\pgfqpoint{5.462879in}{2.792678in}}{\pgfqpoint{5.467269in}{2.782079in}}{\pgfqpoint{5.475083in}{2.774265in}}%
\pgfpathcurveto{\pgfqpoint{5.482896in}{2.766451in}}{\pgfqpoint{5.493495in}{2.762061in}}{\pgfqpoint{5.504545in}{2.762061in}}%
\pgfpathclose%
\pgfusepath{stroke,fill}%
\end{pgfscope}%
\begin{pgfscope}%
\pgfpathrectangle{\pgfqpoint{0.800000in}{0.528000in}}{\pgfqpoint{4.960000in}{3.696000in}}%
\pgfusepath{clip}%
\pgfsetbuttcap%
\pgfsetroundjoin%
\definecolor{currentfill}{rgb}{0.000000,0.000000,0.000000}%
\pgfsetfillcolor{currentfill}%
\pgfsetlinewidth{1.003750pt}%
\definecolor{currentstroke}{rgb}{0.000000,0.000000,0.000000}%
\pgfsetstrokecolor{currentstroke}%
\pgfsetdash{}{0pt}%
\pgfpathmoveto{\pgfqpoint{5.504545in}{2.998534in}}%
\pgfpathcurveto{\pgfqpoint{5.515596in}{2.998534in}}{\pgfqpoint{5.526195in}{3.002924in}}{\pgfqpoint{5.534008in}{3.010738in}}%
\pgfpathcurveto{\pgfqpoint{5.541822in}{3.018551in}}{\pgfqpoint{5.546212in}{3.029150in}}{\pgfqpoint{5.546212in}{3.040200in}}%
\pgfpathcurveto{\pgfqpoint{5.546212in}{3.051250in}}{\pgfqpoint{5.541822in}{3.061849in}}{\pgfqpoint{5.534008in}{3.069663in}}%
\pgfpathcurveto{\pgfqpoint{5.526195in}{3.077477in}}{\pgfqpoint{5.515596in}{3.081867in}}{\pgfqpoint{5.504545in}{3.081867in}}%
\pgfpathcurveto{\pgfqpoint{5.493495in}{3.081867in}}{\pgfqpoint{5.482896in}{3.077477in}}{\pgfqpoint{5.475083in}{3.069663in}}%
\pgfpathcurveto{\pgfqpoint{5.467269in}{3.061849in}}{\pgfqpoint{5.462879in}{3.051250in}}{\pgfqpoint{5.462879in}{3.040200in}}%
\pgfpathcurveto{\pgfqpoint{5.462879in}{3.029150in}}{\pgfqpoint{5.467269in}{3.018551in}}{\pgfqpoint{5.475083in}{3.010738in}}%
\pgfpathcurveto{\pgfqpoint{5.482896in}{3.002924in}}{\pgfqpoint{5.493495in}{2.998534in}}{\pgfqpoint{5.504545in}{2.998534in}}%
\pgfpathclose%
\pgfusepath{stroke,fill}%
\end{pgfscope}%
\begin{pgfscope}%
\pgfpathrectangle{\pgfqpoint{0.800000in}{0.528000in}}{\pgfqpoint{4.960000in}{3.696000in}}%
\pgfusepath{clip}%
\pgfsetbuttcap%
\pgfsetroundjoin%
\definecolor{currentfill}{rgb}{0.000000,0.000000,0.000000}%
\pgfsetfillcolor{currentfill}%
\pgfsetlinewidth{1.003750pt}%
\definecolor{currentstroke}{rgb}{0.000000,0.000000,0.000000}%
\pgfsetstrokecolor{currentstroke}%
\pgfsetdash{}{0pt}%
\pgfpathmoveto{\pgfqpoint{5.504545in}{2.697569in}}%
\pgfpathcurveto{\pgfqpoint{5.515596in}{2.697569in}}{\pgfqpoint{5.526195in}{2.701959in}}{\pgfqpoint{5.534008in}{2.709773in}}%
\pgfpathcurveto{\pgfqpoint{5.541822in}{2.717586in}}{\pgfqpoint{5.546212in}{2.728185in}}{\pgfqpoint{5.546212in}{2.739235in}}%
\pgfpathcurveto{\pgfqpoint{5.546212in}{2.750286in}}{\pgfqpoint{5.541822in}{2.760885in}}{\pgfqpoint{5.534008in}{2.768698in}}%
\pgfpathcurveto{\pgfqpoint{5.526195in}{2.776512in}}{\pgfqpoint{5.515596in}{2.780902in}}{\pgfqpoint{5.504545in}{2.780902in}}%
\pgfpathcurveto{\pgfqpoint{5.493495in}{2.780902in}}{\pgfqpoint{5.482896in}{2.776512in}}{\pgfqpoint{5.475083in}{2.768698in}}%
\pgfpathcurveto{\pgfqpoint{5.467269in}{2.760885in}}{\pgfqpoint{5.462879in}{2.750286in}}{\pgfqpoint{5.462879in}{2.739235in}}%
\pgfpathcurveto{\pgfqpoint{5.462879in}{2.728185in}}{\pgfqpoint{5.467269in}{2.717586in}}{\pgfqpoint{5.475083in}{2.709773in}}%
\pgfpathcurveto{\pgfqpoint{5.482896in}{2.701959in}}{\pgfqpoint{5.493495in}{2.697569in}}{\pgfqpoint{5.504545in}{2.697569in}}%
\pgfpathclose%
\pgfusepath{stroke,fill}%
\end{pgfscope}%
\begin{pgfscope}%
\pgfpathrectangle{\pgfqpoint{0.800000in}{0.528000in}}{\pgfqpoint{4.960000in}{3.696000in}}%
\pgfusepath{clip}%
\pgfsetbuttcap%
\pgfsetroundjoin%
\definecolor{currentfill}{rgb}{0.000000,0.000000,0.000000}%
\pgfsetfillcolor{currentfill}%
\pgfsetlinewidth{1.003750pt}%
\definecolor{currentstroke}{rgb}{0.000000,0.000000,0.000000}%
\pgfsetstrokecolor{currentstroke}%
\pgfsetdash{}{0pt}%
\pgfpathmoveto{\pgfqpoint{5.504545in}{2.697569in}}%
\pgfpathcurveto{\pgfqpoint{5.515596in}{2.697569in}}{\pgfqpoint{5.526195in}{2.701959in}}{\pgfqpoint{5.534008in}{2.709773in}}%
\pgfpathcurveto{\pgfqpoint{5.541822in}{2.717586in}}{\pgfqpoint{5.546212in}{2.728185in}}{\pgfqpoint{5.546212in}{2.739235in}}%
\pgfpathcurveto{\pgfqpoint{5.546212in}{2.750286in}}{\pgfqpoint{5.541822in}{2.760885in}}{\pgfqpoint{5.534008in}{2.768698in}}%
\pgfpathcurveto{\pgfqpoint{5.526195in}{2.776512in}}{\pgfqpoint{5.515596in}{2.780902in}}{\pgfqpoint{5.504545in}{2.780902in}}%
\pgfpathcurveto{\pgfqpoint{5.493495in}{2.780902in}}{\pgfqpoint{5.482896in}{2.776512in}}{\pgfqpoint{5.475083in}{2.768698in}}%
\pgfpathcurveto{\pgfqpoint{5.467269in}{2.760885in}}{\pgfqpoint{5.462879in}{2.750286in}}{\pgfqpoint{5.462879in}{2.739235in}}%
\pgfpathcurveto{\pgfqpoint{5.462879in}{2.728185in}}{\pgfqpoint{5.467269in}{2.717586in}}{\pgfqpoint{5.475083in}{2.709773in}}%
\pgfpathcurveto{\pgfqpoint{5.482896in}{2.701959in}}{\pgfqpoint{5.493495in}{2.697569in}}{\pgfqpoint{5.504545in}{2.697569in}}%
\pgfpathclose%
\pgfusepath{stroke,fill}%
\end{pgfscope}%
\begin{pgfscope}%
\pgfpathrectangle{\pgfqpoint{0.800000in}{0.528000in}}{\pgfqpoint{4.960000in}{3.696000in}}%
\pgfusepath{clip}%
\pgfsetbuttcap%
\pgfsetroundjoin%
\definecolor{currentfill}{rgb}{0.000000,0.000000,0.000000}%
\pgfsetfillcolor{currentfill}%
\pgfsetlinewidth{1.003750pt}%
\definecolor{currentstroke}{rgb}{0.000000,0.000000,0.000000}%
\pgfsetstrokecolor{currentstroke}%
\pgfsetdash{}{0pt}%
\pgfpathmoveto{\pgfqpoint{5.504545in}{2.633076in}}%
\pgfpathcurveto{\pgfqpoint{5.515596in}{2.633076in}}{\pgfqpoint{5.526195in}{2.637467in}}{\pgfqpoint{5.534008in}{2.645280in}}%
\pgfpathcurveto{\pgfqpoint{5.541822in}{2.653094in}}{\pgfqpoint{5.546212in}{2.663693in}}{\pgfqpoint{5.546212in}{2.674743in}}%
\pgfpathcurveto{\pgfqpoint{5.546212in}{2.685793in}}{\pgfqpoint{5.541822in}{2.696392in}}{\pgfqpoint{5.534008in}{2.704206in}}%
\pgfpathcurveto{\pgfqpoint{5.526195in}{2.712019in}}{\pgfqpoint{5.515596in}{2.716410in}}{\pgfqpoint{5.504545in}{2.716410in}}%
\pgfpathcurveto{\pgfqpoint{5.493495in}{2.716410in}}{\pgfqpoint{5.482896in}{2.712019in}}{\pgfqpoint{5.475083in}{2.704206in}}%
\pgfpathcurveto{\pgfqpoint{5.467269in}{2.696392in}}{\pgfqpoint{5.462879in}{2.685793in}}{\pgfqpoint{5.462879in}{2.674743in}}%
\pgfpathcurveto{\pgfqpoint{5.462879in}{2.663693in}}{\pgfqpoint{5.467269in}{2.653094in}}{\pgfqpoint{5.475083in}{2.645280in}}%
\pgfpathcurveto{\pgfqpoint{5.482896in}{2.637467in}}{\pgfqpoint{5.493495in}{2.633076in}}{\pgfqpoint{5.504545in}{2.633076in}}%
\pgfpathclose%
\pgfusepath{stroke,fill}%
\end{pgfscope}%
\begin{pgfscope}%
\pgfpathrectangle{\pgfqpoint{0.800000in}{0.528000in}}{\pgfqpoint{4.960000in}{3.696000in}}%
\pgfusepath{clip}%
\pgfsetbuttcap%
\pgfsetroundjoin%
\definecolor{currentfill}{rgb}{0.000000,0.000000,0.000000}%
\pgfsetfillcolor{currentfill}%
\pgfsetlinewidth{1.003750pt}%
\definecolor{currentstroke}{rgb}{0.000000,0.000000,0.000000}%
\pgfsetstrokecolor{currentstroke}%
\pgfsetdash{}{0pt}%
\pgfpathmoveto{\pgfqpoint{5.504545in}{2.977036in}}%
\pgfpathcurveto{\pgfqpoint{5.515596in}{2.977036in}}{\pgfqpoint{5.526195in}{2.981426in}}{\pgfqpoint{5.534008in}{2.989240in}}%
\pgfpathcurveto{\pgfqpoint{5.541822in}{2.997054in}}{\pgfqpoint{5.546212in}{3.007653in}}{\pgfqpoint{5.546212in}{3.018703in}}%
\pgfpathcurveto{\pgfqpoint{5.546212in}{3.029753in}}{\pgfqpoint{5.541822in}{3.040352in}}{\pgfqpoint{5.534008in}{3.048166in}}%
\pgfpathcurveto{\pgfqpoint{5.526195in}{3.055979in}}{\pgfqpoint{5.515596in}{3.060369in}}{\pgfqpoint{5.504545in}{3.060369in}}%
\pgfpathcurveto{\pgfqpoint{5.493495in}{3.060369in}}{\pgfqpoint{5.482896in}{3.055979in}}{\pgfqpoint{5.475083in}{3.048166in}}%
\pgfpathcurveto{\pgfqpoint{5.467269in}{3.040352in}}{\pgfqpoint{5.462879in}{3.029753in}}{\pgfqpoint{5.462879in}{3.018703in}}%
\pgfpathcurveto{\pgfqpoint{5.462879in}{3.007653in}}{\pgfqpoint{5.467269in}{2.997054in}}{\pgfqpoint{5.475083in}{2.989240in}}%
\pgfpathcurveto{\pgfqpoint{5.482896in}{2.981426in}}{\pgfqpoint{5.493495in}{2.977036in}}{\pgfqpoint{5.504545in}{2.977036in}}%
\pgfpathclose%
\pgfusepath{stroke,fill}%
\end{pgfscope}%
\begin{pgfscope}%
\pgfpathrectangle{\pgfqpoint{0.800000in}{0.528000in}}{\pgfqpoint{4.960000in}{3.696000in}}%
\pgfusepath{clip}%
\pgfsetbuttcap%
\pgfsetroundjoin%
\definecolor{currentfill}{rgb}{0.000000,0.000000,0.000000}%
\pgfsetfillcolor{currentfill}%
\pgfsetlinewidth{1.003750pt}%
\definecolor{currentstroke}{rgb}{0.000000,0.000000,0.000000}%
\pgfsetstrokecolor{currentstroke}%
\pgfsetdash{}{0pt}%
\pgfpathmoveto{\pgfqpoint{5.504545in}{3.041529in}}%
\pgfpathcurveto{\pgfqpoint{5.515596in}{3.041529in}}{\pgfqpoint{5.526195in}{3.045919in}}{\pgfqpoint{5.534008in}{3.053732in}}%
\pgfpathcurveto{\pgfqpoint{5.541822in}{3.061546in}}{\pgfqpoint{5.546212in}{3.072145in}}{\pgfqpoint{5.546212in}{3.083195in}}%
\pgfpathcurveto{\pgfqpoint{5.546212in}{3.094245in}}{\pgfqpoint{5.541822in}{3.104844in}}{\pgfqpoint{5.534008in}{3.112658in}}%
\pgfpathcurveto{\pgfqpoint{5.526195in}{3.120472in}}{\pgfqpoint{5.515596in}{3.124862in}}{\pgfqpoint{5.504545in}{3.124862in}}%
\pgfpathcurveto{\pgfqpoint{5.493495in}{3.124862in}}{\pgfqpoint{5.482896in}{3.120472in}}{\pgfqpoint{5.475083in}{3.112658in}}%
\pgfpathcurveto{\pgfqpoint{5.467269in}{3.104844in}}{\pgfqpoint{5.462879in}{3.094245in}}{\pgfqpoint{5.462879in}{3.083195in}}%
\pgfpathcurveto{\pgfqpoint{5.462879in}{3.072145in}}{\pgfqpoint{5.467269in}{3.061546in}}{\pgfqpoint{5.475083in}{3.053732in}}%
\pgfpathcurveto{\pgfqpoint{5.482896in}{3.045919in}}{\pgfqpoint{5.493495in}{3.041529in}}{\pgfqpoint{5.504545in}{3.041529in}}%
\pgfpathclose%
\pgfusepath{stroke,fill}%
\end{pgfscope}%
\begin{pgfscope}%
\pgfpathrectangle{\pgfqpoint{0.800000in}{0.528000in}}{\pgfqpoint{4.960000in}{3.696000in}}%
\pgfusepath{clip}%
\pgfsetbuttcap%
\pgfsetroundjoin%
\definecolor{currentfill}{rgb}{0.000000,0.000000,0.000000}%
\pgfsetfillcolor{currentfill}%
\pgfsetlinewidth{1.003750pt}%
\definecolor{currentstroke}{rgb}{0.000000,0.000000,0.000000}%
\pgfsetstrokecolor{currentstroke}%
\pgfsetdash{}{0pt}%
\pgfpathmoveto{\pgfqpoint{5.504545in}{2.912544in}}%
\pgfpathcurveto{\pgfqpoint{5.515596in}{2.912544in}}{\pgfqpoint{5.526195in}{2.916934in}}{\pgfqpoint{5.534008in}{2.924748in}}%
\pgfpathcurveto{\pgfqpoint{5.541822in}{2.932561in}}{\pgfqpoint{5.546212in}{2.943160in}}{\pgfqpoint{5.546212in}{2.954210in}}%
\pgfpathcurveto{\pgfqpoint{5.546212in}{2.965260in}}{\pgfqpoint{5.541822in}{2.975859in}}{\pgfqpoint{5.534008in}{2.983673in}}%
\pgfpathcurveto{\pgfqpoint{5.526195in}{2.991487in}}{\pgfqpoint{5.515596in}{2.995877in}}{\pgfqpoint{5.504545in}{2.995877in}}%
\pgfpathcurveto{\pgfqpoint{5.493495in}{2.995877in}}{\pgfqpoint{5.482896in}{2.991487in}}{\pgfqpoint{5.475083in}{2.983673in}}%
\pgfpathcurveto{\pgfqpoint{5.467269in}{2.975859in}}{\pgfqpoint{5.462879in}{2.965260in}}{\pgfqpoint{5.462879in}{2.954210in}}%
\pgfpathcurveto{\pgfqpoint{5.462879in}{2.943160in}}{\pgfqpoint{5.467269in}{2.932561in}}{\pgfqpoint{5.475083in}{2.924748in}}%
\pgfpathcurveto{\pgfqpoint{5.482896in}{2.916934in}}{\pgfqpoint{5.493495in}{2.912544in}}{\pgfqpoint{5.504545in}{2.912544in}}%
\pgfpathclose%
\pgfusepath{stroke,fill}%
\end{pgfscope}%
\begin{pgfscope}%
\pgfpathrectangle{\pgfqpoint{0.800000in}{0.528000in}}{\pgfqpoint{4.960000in}{3.696000in}}%
\pgfusepath{clip}%
\pgfsetbuttcap%
\pgfsetroundjoin%
\definecolor{currentfill}{rgb}{0.000000,0.000000,0.000000}%
\pgfsetfillcolor{currentfill}%
\pgfsetlinewidth{1.003750pt}%
\definecolor{currentstroke}{rgb}{0.000000,0.000000,0.000000}%
\pgfsetstrokecolor{currentstroke}%
\pgfsetdash{}{0pt}%
\pgfpathmoveto{\pgfqpoint{5.504545in}{2.740564in}}%
\pgfpathcurveto{\pgfqpoint{5.515596in}{2.740564in}}{\pgfqpoint{5.526195in}{2.744954in}}{\pgfqpoint{5.534008in}{2.752768in}}%
\pgfpathcurveto{\pgfqpoint{5.541822in}{2.760581in}}{\pgfqpoint{5.546212in}{2.771180in}}{\pgfqpoint{5.546212in}{2.782230in}}%
\pgfpathcurveto{\pgfqpoint{5.546212in}{2.793281in}}{\pgfqpoint{5.541822in}{2.803880in}}{\pgfqpoint{5.534008in}{2.811693in}}%
\pgfpathcurveto{\pgfqpoint{5.526195in}{2.819507in}}{\pgfqpoint{5.515596in}{2.823897in}}{\pgfqpoint{5.504545in}{2.823897in}}%
\pgfpathcurveto{\pgfqpoint{5.493495in}{2.823897in}}{\pgfqpoint{5.482896in}{2.819507in}}{\pgfqpoint{5.475083in}{2.811693in}}%
\pgfpathcurveto{\pgfqpoint{5.467269in}{2.803880in}}{\pgfqpoint{5.462879in}{2.793281in}}{\pgfqpoint{5.462879in}{2.782230in}}%
\pgfpathcurveto{\pgfqpoint{5.462879in}{2.771180in}}{\pgfqpoint{5.467269in}{2.760581in}}{\pgfqpoint{5.475083in}{2.752768in}}%
\pgfpathcurveto{\pgfqpoint{5.482896in}{2.744954in}}{\pgfqpoint{5.493495in}{2.740564in}}{\pgfqpoint{5.504545in}{2.740564in}}%
\pgfpathclose%
\pgfusepath{stroke,fill}%
\end{pgfscope}%
\begin{pgfscope}%
\pgfpathrectangle{\pgfqpoint{0.800000in}{0.528000in}}{\pgfqpoint{4.960000in}{3.696000in}}%
\pgfusepath{clip}%
\pgfsetbuttcap%
\pgfsetroundjoin%
\definecolor{currentfill}{rgb}{0.000000,0.000000,0.000000}%
\pgfsetfillcolor{currentfill}%
\pgfsetlinewidth{1.003750pt}%
\definecolor{currentstroke}{rgb}{0.000000,0.000000,0.000000}%
\pgfsetstrokecolor{currentstroke}%
\pgfsetdash{}{0pt}%
\pgfpathmoveto{\pgfqpoint{5.504545in}{2.740564in}}%
\pgfpathcurveto{\pgfqpoint{5.515596in}{2.740564in}}{\pgfqpoint{5.526195in}{2.744954in}}{\pgfqpoint{5.534008in}{2.752768in}}%
\pgfpathcurveto{\pgfqpoint{5.541822in}{2.760581in}}{\pgfqpoint{5.546212in}{2.771180in}}{\pgfqpoint{5.546212in}{2.782230in}}%
\pgfpathcurveto{\pgfqpoint{5.546212in}{2.793281in}}{\pgfqpoint{5.541822in}{2.803880in}}{\pgfqpoint{5.534008in}{2.811693in}}%
\pgfpathcurveto{\pgfqpoint{5.526195in}{2.819507in}}{\pgfqpoint{5.515596in}{2.823897in}}{\pgfqpoint{5.504545in}{2.823897in}}%
\pgfpathcurveto{\pgfqpoint{5.493495in}{2.823897in}}{\pgfqpoint{5.482896in}{2.819507in}}{\pgfqpoint{5.475083in}{2.811693in}}%
\pgfpathcurveto{\pgfqpoint{5.467269in}{2.803880in}}{\pgfqpoint{5.462879in}{2.793281in}}{\pgfqpoint{5.462879in}{2.782230in}}%
\pgfpathcurveto{\pgfqpoint{5.462879in}{2.771180in}}{\pgfqpoint{5.467269in}{2.760581in}}{\pgfqpoint{5.475083in}{2.752768in}}%
\pgfpathcurveto{\pgfqpoint{5.482896in}{2.744954in}}{\pgfqpoint{5.493495in}{2.740564in}}{\pgfqpoint{5.504545in}{2.740564in}}%
\pgfpathclose%
\pgfusepath{stroke,fill}%
\end{pgfscope}%
\begin{pgfscope}%
\pgfpathrectangle{\pgfqpoint{0.800000in}{0.528000in}}{\pgfqpoint{4.960000in}{3.696000in}}%
\pgfusepath{clip}%
\pgfsetbuttcap%
\pgfsetroundjoin%
\definecolor{currentfill}{rgb}{0.000000,0.000000,0.000000}%
\pgfsetfillcolor{currentfill}%
\pgfsetlinewidth{1.003750pt}%
\definecolor{currentstroke}{rgb}{0.000000,0.000000,0.000000}%
\pgfsetstrokecolor{currentstroke}%
\pgfsetdash{}{0pt}%
\pgfpathmoveto{\pgfqpoint{5.504545in}{2.912544in}}%
\pgfpathcurveto{\pgfqpoint{5.515596in}{2.912544in}}{\pgfqpoint{5.526195in}{2.916934in}}{\pgfqpoint{5.534008in}{2.924748in}}%
\pgfpathcurveto{\pgfqpoint{5.541822in}{2.932561in}}{\pgfqpoint{5.546212in}{2.943160in}}{\pgfqpoint{5.546212in}{2.954210in}}%
\pgfpathcurveto{\pgfqpoint{5.546212in}{2.965260in}}{\pgfqpoint{5.541822in}{2.975859in}}{\pgfqpoint{5.534008in}{2.983673in}}%
\pgfpathcurveto{\pgfqpoint{5.526195in}{2.991487in}}{\pgfqpoint{5.515596in}{2.995877in}}{\pgfqpoint{5.504545in}{2.995877in}}%
\pgfpathcurveto{\pgfqpoint{5.493495in}{2.995877in}}{\pgfqpoint{5.482896in}{2.991487in}}{\pgfqpoint{5.475083in}{2.983673in}}%
\pgfpathcurveto{\pgfqpoint{5.467269in}{2.975859in}}{\pgfqpoint{5.462879in}{2.965260in}}{\pgfqpoint{5.462879in}{2.954210in}}%
\pgfpathcurveto{\pgfqpoint{5.462879in}{2.943160in}}{\pgfqpoint{5.467269in}{2.932561in}}{\pgfqpoint{5.475083in}{2.924748in}}%
\pgfpathcurveto{\pgfqpoint{5.482896in}{2.916934in}}{\pgfqpoint{5.493495in}{2.912544in}}{\pgfqpoint{5.504545in}{2.912544in}}%
\pgfpathclose%
\pgfusepath{stroke,fill}%
\end{pgfscope}%
\begin{pgfscope}%
\pgfpathrectangle{\pgfqpoint{0.800000in}{0.528000in}}{\pgfqpoint{4.960000in}{3.696000in}}%
\pgfusepath{clip}%
\pgfsetbuttcap%
\pgfsetroundjoin%
\definecolor{currentfill}{rgb}{0.000000,0.000000,0.000000}%
\pgfsetfillcolor{currentfill}%
\pgfsetlinewidth{1.003750pt}%
\definecolor{currentstroke}{rgb}{0.000000,0.000000,0.000000}%
\pgfsetstrokecolor{currentstroke}%
\pgfsetdash{}{0pt}%
\pgfpathmoveto{\pgfqpoint{5.504545in}{2.783559in}}%
\pgfpathcurveto{\pgfqpoint{5.515596in}{2.783559in}}{\pgfqpoint{5.526195in}{2.787949in}}{\pgfqpoint{5.534008in}{2.795763in}}%
\pgfpathcurveto{\pgfqpoint{5.541822in}{2.803576in}}{\pgfqpoint{5.546212in}{2.814175in}}{\pgfqpoint{5.546212in}{2.825225in}}%
\pgfpathcurveto{\pgfqpoint{5.546212in}{2.836275in}}{\pgfqpoint{5.541822in}{2.846875in}}{\pgfqpoint{5.534008in}{2.854688in}}%
\pgfpathcurveto{\pgfqpoint{5.526195in}{2.862502in}}{\pgfqpoint{5.515596in}{2.866892in}}{\pgfqpoint{5.504545in}{2.866892in}}%
\pgfpathcurveto{\pgfqpoint{5.493495in}{2.866892in}}{\pgfqpoint{5.482896in}{2.862502in}}{\pgfqpoint{5.475083in}{2.854688in}}%
\pgfpathcurveto{\pgfqpoint{5.467269in}{2.846875in}}{\pgfqpoint{5.462879in}{2.836275in}}{\pgfqpoint{5.462879in}{2.825225in}}%
\pgfpathcurveto{\pgfqpoint{5.462879in}{2.814175in}}{\pgfqpoint{5.467269in}{2.803576in}}{\pgfqpoint{5.475083in}{2.795763in}}%
\pgfpathcurveto{\pgfqpoint{5.482896in}{2.787949in}}{\pgfqpoint{5.493495in}{2.783559in}}{\pgfqpoint{5.504545in}{2.783559in}}%
\pgfpathclose%
\pgfusepath{stroke,fill}%
\end{pgfscope}%
\begin{pgfscope}%
\pgfpathrectangle{\pgfqpoint{0.800000in}{0.528000in}}{\pgfqpoint{4.960000in}{3.696000in}}%
\pgfusepath{clip}%
\pgfsetbuttcap%
\pgfsetroundjoin%
\definecolor{currentfill}{rgb}{0.000000,0.000000,0.000000}%
\pgfsetfillcolor{currentfill}%
\pgfsetlinewidth{1.003750pt}%
\definecolor{currentstroke}{rgb}{0.000000,0.000000,0.000000}%
\pgfsetstrokecolor{currentstroke}%
\pgfsetdash{}{0pt}%
\pgfpathmoveto{\pgfqpoint{5.504545in}{2.762061in}}%
\pgfpathcurveto{\pgfqpoint{5.515596in}{2.762061in}}{\pgfqpoint{5.526195in}{2.766451in}}{\pgfqpoint{5.534008in}{2.774265in}}%
\pgfpathcurveto{\pgfqpoint{5.541822in}{2.782079in}}{\pgfqpoint{5.546212in}{2.792678in}}{\pgfqpoint{5.546212in}{2.803728in}}%
\pgfpathcurveto{\pgfqpoint{5.546212in}{2.814778in}}{\pgfqpoint{5.541822in}{2.825377in}}{\pgfqpoint{5.534008in}{2.833191in}}%
\pgfpathcurveto{\pgfqpoint{5.526195in}{2.841004in}}{\pgfqpoint{5.515596in}{2.845395in}}{\pgfqpoint{5.504545in}{2.845395in}}%
\pgfpathcurveto{\pgfqpoint{5.493495in}{2.845395in}}{\pgfqpoint{5.482896in}{2.841004in}}{\pgfqpoint{5.475083in}{2.833191in}}%
\pgfpathcurveto{\pgfqpoint{5.467269in}{2.825377in}}{\pgfqpoint{5.462879in}{2.814778in}}{\pgfqpoint{5.462879in}{2.803728in}}%
\pgfpathcurveto{\pgfqpoint{5.462879in}{2.792678in}}{\pgfqpoint{5.467269in}{2.782079in}}{\pgfqpoint{5.475083in}{2.774265in}}%
\pgfpathcurveto{\pgfqpoint{5.482896in}{2.766451in}}{\pgfqpoint{5.493495in}{2.762061in}}{\pgfqpoint{5.504545in}{2.762061in}}%
\pgfpathclose%
\pgfusepath{stroke,fill}%
\end{pgfscope}%
\begin{pgfscope}%
\pgfpathrectangle{\pgfqpoint{0.800000in}{0.528000in}}{\pgfqpoint{4.960000in}{3.696000in}}%
\pgfusepath{clip}%
\pgfsetbuttcap%
\pgfsetroundjoin%
\definecolor{currentfill}{rgb}{0.000000,0.000000,0.000000}%
\pgfsetfillcolor{currentfill}%
\pgfsetlinewidth{1.003750pt}%
\definecolor{currentstroke}{rgb}{0.000000,0.000000,0.000000}%
\pgfsetstrokecolor{currentstroke}%
\pgfsetdash{}{0pt}%
\pgfpathmoveto{\pgfqpoint{5.504545in}{2.762061in}}%
\pgfpathcurveto{\pgfqpoint{5.515596in}{2.762061in}}{\pgfqpoint{5.526195in}{2.766451in}}{\pgfqpoint{5.534008in}{2.774265in}}%
\pgfpathcurveto{\pgfqpoint{5.541822in}{2.782079in}}{\pgfqpoint{5.546212in}{2.792678in}}{\pgfqpoint{5.546212in}{2.803728in}}%
\pgfpathcurveto{\pgfqpoint{5.546212in}{2.814778in}}{\pgfqpoint{5.541822in}{2.825377in}}{\pgfqpoint{5.534008in}{2.833191in}}%
\pgfpathcurveto{\pgfqpoint{5.526195in}{2.841004in}}{\pgfqpoint{5.515596in}{2.845395in}}{\pgfqpoint{5.504545in}{2.845395in}}%
\pgfpathcurveto{\pgfqpoint{5.493495in}{2.845395in}}{\pgfqpoint{5.482896in}{2.841004in}}{\pgfqpoint{5.475083in}{2.833191in}}%
\pgfpathcurveto{\pgfqpoint{5.467269in}{2.825377in}}{\pgfqpoint{5.462879in}{2.814778in}}{\pgfqpoint{5.462879in}{2.803728in}}%
\pgfpathcurveto{\pgfqpoint{5.462879in}{2.792678in}}{\pgfqpoint{5.467269in}{2.782079in}}{\pgfqpoint{5.475083in}{2.774265in}}%
\pgfpathcurveto{\pgfqpoint{5.482896in}{2.766451in}}{\pgfqpoint{5.493495in}{2.762061in}}{\pgfqpoint{5.504545in}{2.762061in}}%
\pgfpathclose%
\pgfusepath{stroke,fill}%
\end{pgfscope}%
\begin{pgfscope}%
\pgfpathrectangle{\pgfqpoint{0.800000in}{0.528000in}}{\pgfqpoint{4.960000in}{3.696000in}}%
\pgfusepath{clip}%
\pgfsetbuttcap%
\pgfsetroundjoin%
\definecolor{currentfill}{rgb}{0.000000,0.000000,0.000000}%
\pgfsetfillcolor{currentfill}%
\pgfsetlinewidth{1.003750pt}%
\definecolor{currentstroke}{rgb}{0.000000,0.000000,0.000000}%
\pgfsetstrokecolor{currentstroke}%
\pgfsetdash{}{0pt}%
\pgfpathmoveto{\pgfqpoint{5.504545in}{2.762061in}}%
\pgfpathcurveto{\pgfqpoint{5.515596in}{2.762061in}}{\pgfqpoint{5.526195in}{2.766451in}}{\pgfqpoint{5.534008in}{2.774265in}}%
\pgfpathcurveto{\pgfqpoint{5.541822in}{2.782079in}}{\pgfqpoint{5.546212in}{2.792678in}}{\pgfqpoint{5.546212in}{2.803728in}}%
\pgfpathcurveto{\pgfqpoint{5.546212in}{2.814778in}}{\pgfqpoint{5.541822in}{2.825377in}}{\pgfqpoint{5.534008in}{2.833191in}}%
\pgfpathcurveto{\pgfqpoint{5.526195in}{2.841004in}}{\pgfqpoint{5.515596in}{2.845395in}}{\pgfqpoint{5.504545in}{2.845395in}}%
\pgfpathcurveto{\pgfqpoint{5.493495in}{2.845395in}}{\pgfqpoint{5.482896in}{2.841004in}}{\pgfqpoint{5.475083in}{2.833191in}}%
\pgfpathcurveto{\pgfqpoint{5.467269in}{2.825377in}}{\pgfqpoint{5.462879in}{2.814778in}}{\pgfqpoint{5.462879in}{2.803728in}}%
\pgfpathcurveto{\pgfqpoint{5.462879in}{2.792678in}}{\pgfqpoint{5.467269in}{2.782079in}}{\pgfqpoint{5.475083in}{2.774265in}}%
\pgfpathcurveto{\pgfqpoint{5.482896in}{2.766451in}}{\pgfqpoint{5.493495in}{2.762061in}}{\pgfqpoint{5.504545in}{2.762061in}}%
\pgfpathclose%
\pgfusepath{stroke,fill}%
\end{pgfscope}%
\begin{pgfscope}%
\pgfpathrectangle{\pgfqpoint{0.800000in}{0.528000in}}{\pgfqpoint{4.960000in}{3.696000in}}%
\pgfusepath{clip}%
\pgfsetbuttcap%
\pgfsetroundjoin%
\definecolor{currentfill}{rgb}{0.000000,0.000000,0.000000}%
\pgfsetfillcolor{currentfill}%
\pgfsetlinewidth{1.003750pt}%
\definecolor{currentstroke}{rgb}{0.000000,0.000000,0.000000}%
\pgfsetstrokecolor{currentstroke}%
\pgfsetdash{}{0pt}%
\pgfpathmoveto{\pgfqpoint{5.504545in}{3.256504in}}%
\pgfpathcurveto{\pgfqpoint{5.515596in}{3.256504in}}{\pgfqpoint{5.526195in}{3.260894in}}{\pgfqpoint{5.534008in}{3.268707in}}%
\pgfpathcurveto{\pgfqpoint{5.541822in}{3.276521in}}{\pgfqpoint{5.546212in}{3.287120in}}{\pgfqpoint{5.546212in}{3.298170in}}%
\pgfpathcurveto{\pgfqpoint{5.546212in}{3.309220in}}{\pgfqpoint{5.541822in}{3.319819in}}{\pgfqpoint{5.534008in}{3.327633in}}%
\pgfpathcurveto{\pgfqpoint{5.526195in}{3.335447in}}{\pgfqpoint{5.515596in}{3.339837in}}{\pgfqpoint{5.504545in}{3.339837in}}%
\pgfpathcurveto{\pgfqpoint{5.493495in}{3.339837in}}{\pgfqpoint{5.482896in}{3.335447in}}{\pgfqpoint{5.475083in}{3.327633in}}%
\pgfpathcurveto{\pgfqpoint{5.467269in}{3.319819in}}{\pgfqpoint{5.462879in}{3.309220in}}{\pgfqpoint{5.462879in}{3.298170in}}%
\pgfpathcurveto{\pgfqpoint{5.462879in}{3.287120in}}{\pgfqpoint{5.467269in}{3.276521in}}{\pgfqpoint{5.475083in}{3.268707in}}%
\pgfpathcurveto{\pgfqpoint{5.482896in}{3.260894in}}{\pgfqpoint{5.493495in}{3.256504in}}{\pgfqpoint{5.504545in}{3.256504in}}%
\pgfpathclose%
\pgfusepath{stroke,fill}%
\end{pgfscope}%
\begin{pgfscope}%
\pgfpathrectangle{\pgfqpoint{0.800000in}{0.528000in}}{\pgfqpoint{4.960000in}{3.696000in}}%
\pgfusepath{clip}%
\pgfsetbuttcap%
\pgfsetroundjoin%
\definecolor{currentfill}{rgb}{0.000000,0.000000,0.000000}%
\pgfsetfillcolor{currentfill}%
\pgfsetlinewidth{1.003750pt}%
\definecolor{currentstroke}{rgb}{0.000000,0.000000,0.000000}%
\pgfsetstrokecolor{currentstroke}%
\pgfsetdash{}{0pt}%
\pgfpathmoveto{\pgfqpoint{5.504545in}{3.041529in}}%
\pgfpathcurveto{\pgfqpoint{5.515596in}{3.041529in}}{\pgfqpoint{5.526195in}{3.045919in}}{\pgfqpoint{5.534008in}{3.053732in}}%
\pgfpathcurveto{\pgfqpoint{5.541822in}{3.061546in}}{\pgfqpoint{5.546212in}{3.072145in}}{\pgfqpoint{5.546212in}{3.083195in}}%
\pgfpathcurveto{\pgfqpoint{5.546212in}{3.094245in}}{\pgfqpoint{5.541822in}{3.104844in}}{\pgfqpoint{5.534008in}{3.112658in}}%
\pgfpathcurveto{\pgfqpoint{5.526195in}{3.120472in}}{\pgfqpoint{5.515596in}{3.124862in}}{\pgfqpoint{5.504545in}{3.124862in}}%
\pgfpathcurveto{\pgfqpoint{5.493495in}{3.124862in}}{\pgfqpoint{5.482896in}{3.120472in}}{\pgfqpoint{5.475083in}{3.112658in}}%
\pgfpathcurveto{\pgfqpoint{5.467269in}{3.104844in}}{\pgfqpoint{5.462879in}{3.094245in}}{\pgfqpoint{5.462879in}{3.083195in}}%
\pgfpathcurveto{\pgfqpoint{5.462879in}{3.072145in}}{\pgfqpoint{5.467269in}{3.061546in}}{\pgfqpoint{5.475083in}{3.053732in}}%
\pgfpathcurveto{\pgfqpoint{5.482896in}{3.045919in}}{\pgfqpoint{5.493495in}{3.041529in}}{\pgfqpoint{5.504545in}{3.041529in}}%
\pgfpathclose%
\pgfusepath{stroke,fill}%
\end{pgfscope}%
\begin{pgfscope}%
\pgfpathrectangle{\pgfqpoint{0.800000in}{0.528000in}}{\pgfqpoint{4.960000in}{3.696000in}}%
\pgfusepath{clip}%
\pgfsetbuttcap%
\pgfsetroundjoin%
\definecolor{currentfill}{rgb}{0.000000,0.000000,0.000000}%
\pgfsetfillcolor{currentfill}%
\pgfsetlinewidth{1.003750pt}%
\definecolor{currentstroke}{rgb}{0.000000,0.000000,0.000000}%
\pgfsetstrokecolor{currentstroke}%
\pgfsetdash{}{0pt}%
\pgfpathmoveto{\pgfqpoint{5.504545in}{2.719066in}}%
\pgfpathcurveto{\pgfqpoint{5.515596in}{2.719066in}}{\pgfqpoint{5.526195in}{2.723456in}}{\pgfqpoint{5.534008in}{2.731270in}}%
\pgfpathcurveto{\pgfqpoint{5.541822in}{2.739084in}}{\pgfqpoint{5.546212in}{2.749683in}}{\pgfqpoint{5.546212in}{2.760733in}}%
\pgfpathcurveto{\pgfqpoint{5.546212in}{2.771783in}}{\pgfqpoint{5.541822in}{2.782382in}}{\pgfqpoint{5.534008in}{2.790196in}}%
\pgfpathcurveto{\pgfqpoint{5.526195in}{2.798009in}}{\pgfqpoint{5.515596in}{2.802400in}}{\pgfqpoint{5.504545in}{2.802400in}}%
\pgfpathcurveto{\pgfqpoint{5.493495in}{2.802400in}}{\pgfqpoint{5.482896in}{2.798009in}}{\pgfqpoint{5.475083in}{2.790196in}}%
\pgfpathcurveto{\pgfqpoint{5.467269in}{2.782382in}}{\pgfqpoint{5.462879in}{2.771783in}}{\pgfqpoint{5.462879in}{2.760733in}}%
\pgfpathcurveto{\pgfqpoint{5.462879in}{2.749683in}}{\pgfqpoint{5.467269in}{2.739084in}}{\pgfqpoint{5.475083in}{2.731270in}}%
\pgfpathcurveto{\pgfqpoint{5.482896in}{2.723456in}}{\pgfqpoint{5.493495in}{2.719066in}}{\pgfqpoint{5.504545in}{2.719066in}}%
\pgfpathclose%
\pgfusepath{stroke,fill}%
\end{pgfscope}%
\begin{pgfscope}%
\pgfpathrectangle{\pgfqpoint{0.800000in}{0.528000in}}{\pgfqpoint{4.960000in}{3.696000in}}%
\pgfusepath{clip}%
\pgfsetbuttcap%
\pgfsetroundjoin%
\definecolor{currentfill}{rgb}{0.000000,0.000000,0.000000}%
\pgfsetfillcolor{currentfill}%
\pgfsetlinewidth{1.003750pt}%
\definecolor{currentstroke}{rgb}{0.000000,0.000000,0.000000}%
\pgfsetstrokecolor{currentstroke}%
\pgfsetdash{}{0pt}%
\pgfpathmoveto{\pgfqpoint{5.504545in}{2.654574in}}%
\pgfpathcurveto{\pgfqpoint{5.515596in}{2.654574in}}{\pgfqpoint{5.526195in}{2.658964in}}{\pgfqpoint{5.534008in}{2.666778in}}%
\pgfpathcurveto{\pgfqpoint{5.541822in}{2.674591in}}{\pgfqpoint{5.546212in}{2.685190in}}{\pgfqpoint{5.546212in}{2.696240in}}%
\pgfpathcurveto{\pgfqpoint{5.546212in}{2.707291in}}{\pgfqpoint{5.541822in}{2.717890in}}{\pgfqpoint{5.534008in}{2.725703in}}%
\pgfpathcurveto{\pgfqpoint{5.526195in}{2.733517in}}{\pgfqpoint{5.515596in}{2.737907in}}{\pgfqpoint{5.504545in}{2.737907in}}%
\pgfpathcurveto{\pgfqpoint{5.493495in}{2.737907in}}{\pgfqpoint{5.482896in}{2.733517in}}{\pgfqpoint{5.475083in}{2.725703in}}%
\pgfpathcurveto{\pgfqpoint{5.467269in}{2.717890in}}{\pgfqpoint{5.462879in}{2.707291in}}{\pgfqpoint{5.462879in}{2.696240in}}%
\pgfpathcurveto{\pgfqpoint{5.462879in}{2.685190in}}{\pgfqpoint{5.467269in}{2.674591in}}{\pgfqpoint{5.475083in}{2.666778in}}%
\pgfpathcurveto{\pgfqpoint{5.482896in}{2.658964in}}{\pgfqpoint{5.493495in}{2.654574in}}{\pgfqpoint{5.504545in}{2.654574in}}%
\pgfpathclose%
\pgfusepath{stroke,fill}%
\end{pgfscope}%
\begin{pgfscope}%
\pgfpathrectangle{\pgfqpoint{0.800000in}{0.528000in}}{\pgfqpoint{4.960000in}{3.696000in}}%
\pgfusepath{clip}%
\pgfsetbuttcap%
\pgfsetroundjoin%
\definecolor{currentfill}{rgb}{0.000000,0.000000,0.000000}%
\pgfsetfillcolor{currentfill}%
\pgfsetlinewidth{1.003750pt}%
\definecolor{currentstroke}{rgb}{0.000000,0.000000,0.000000}%
\pgfsetstrokecolor{currentstroke}%
\pgfsetdash{}{0pt}%
\pgfpathmoveto{\pgfqpoint{5.504545in}{2.977036in}}%
\pgfpathcurveto{\pgfqpoint{5.515596in}{2.977036in}}{\pgfqpoint{5.526195in}{2.981426in}}{\pgfqpoint{5.534008in}{2.989240in}}%
\pgfpathcurveto{\pgfqpoint{5.541822in}{2.997054in}}{\pgfqpoint{5.546212in}{3.007653in}}{\pgfqpoint{5.546212in}{3.018703in}}%
\pgfpathcurveto{\pgfqpoint{5.546212in}{3.029753in}}{\pgfqpoint{5.541822in}{3.040352in}}{\pgfqpoint{5.534008in}{3.048166in}}%
\pgfpathcurveto{\pgfqpoint{5.526195in}{3.055979in}}{\pgfqpoint{5.515596in}{3.060369in}}{\pgfqpoint{5.504545in}{3.060369in}}%
\pgfpathcurveto{\pgfqpoint{5.493495in}{3.060369in}}{\pgfqpoint{5.482896in}{3.055979in}}{\pgfqpoint{5.475083in}{3.048166in}}%
\pgfpathcurveto{\pgfqpoint{5.467269in}{3.040352in}}{\pgfqpoint{5.462879in}{3.029753in}}{\pgfqpoint{5.462879in}{3.018703in}}%
\pgfpathcurveto{\pgfqpoint{5.462879in}{3.007653in}}{\pgfqpoint{5.467269in}{2.997054in}}{\pgfqpoint{5.475083in}{2.989240in}}%
\pgfpathcurveto{\pgfqpoint{5.482896in}{2.981426in}}{\pgfqpoint{5.493495in}{2.977036in}}{\pgfqpoint{5.504545in}{2.977036in}}%
\pgfpathclose%
\pgfusepath{stroke,fill}%
\end{pgfscope}%
\begin{pgfscope}%
\pgfpathrectangle{\pgfqpoint{0.800000in}{0.528000in}}{\pgfqpoint{4.960000in}{3.696000in}}%
\pgfusepath{clip}%
\pgfsetbuttcap%
\pgfsetroundjoin%
\definecolor{currentfill}{rgb}{0.000000,0.000000,0.000000}%
\pgfsetfillcolor{currentfill}%
\pgfsetlinewidth{1.003750pt}%
\definecolor{currentstroke}{rgb}{0.000000,0.000000,0.000000}%
\pgfsetstrokecolor{currentstroke}%
\pgfsetdash{}{0pt}%
\pgfpathmoveto{\pgfqpoint{5.504545in}{2.762061in}}%
\pgfpathcurveto{\pgfqpoint{5.515596in}{2.762061in}}{\pgfqpoint{5.526195in}{2.766451in}}{\pgfqpoint{5.534008in}{2.774265in}}%
\pgfpathcurveto{\pgfqpoint{5.541822in}{2.782079in}}{\pgfqpoint{5.546212in}{2.792678in}}{\pgfqpoint{5.546212in}{2.803728in}}%
\pgfpathcurveto{\pgfqpoint{5.546212in}{2.814778in}}{\pgfqpoint{5.541822in}{2.825377in}}{\pgfqpoint{5.534008in}{2.833191in}}%
\pgfpathcurveto{\pgfqpoint{5.526195in}{2.841004in}}{\pgfqpoint{5.515596in}{2.845395in}}{\pgfqpoint{5.504545in}{2.845395in}}%
\pgfpathcurveto{\pgfqpoint{5.493495in}{2.845395in}}{\pgfqpoint{5.482896in}{2.841004in}}{\pgfqpoint{5.475083in}{2.833191in}}%
\pgfpathcurveto{\pgfqpoint{5.467269in}{2.825377in}}{\pgfqpoint{5.462879in}{2.814778in}}{\pgfqpoint{5.462879in}{2.803728in}}%
\pgfpathcurveto{\pgfqpoint{5.462879in}{2.792678in}}{\pgfqpoint{5.467269in}{2.782079in}}{\pgfqpoint{5.475083in}{2.774265in}}%
\pgfpathcurveto{\pgfqpoint{5.482896in}{2.766451in}}{\pgfqpoint{5.493495in}{2.762061in}}{\pgfqpoint{5.504545in}{2.762061in}}%
\pgfpathclose%
\pgfusepath{stroke,fill}%
\end{pgfscope}%
\begin{pgfscope}%
\pgfpathrectangle{\pgfqpoint{0.800000in}{0.528000in}}{\pgfqpoint{4.960000in}{3.696000in}}%
\pgfusepath{clip}%
\pgfsetbuttcap%
\pgfsetroundjoin%
\definecolor{currentfill}{rgb}{0.000000,0.000000,0.000000}%
\pgfsetfillcolor{currentfill}%
\pgfsetlinewidth{1.003750pt}%
\definecolor{currentstroke}{rgb}{0.000000,0.000000,0.000000}%
\pgfsetstrokecolor{currentstroke}%
\pgfsetdash{}{0pt}%
\pgfpathmoveto{\pgfqpoint{5.504545in}{3.471478in}}%
\pgfpathcurveto{\pgfqpoint{5.515596in}{3.471478in}}{\pgfqpoint{5.526195in}{3.475869in}}{\pgfqpoint{5.534008in}{3.483682in}}%
\pgfpathcurveto{\pgfqpoint{5.541822in}{3.491496in}}{\pgfqpoint{5.546212in}{3.502095in}}{\pgfqpoint{5.546212in}{3.513145in}}%
\pgfpathcurveto{\pgfqpoint{5.546212in}{3.524195in}}{\pgfqpoint{5.541822in}{3.534794in}}{\pgfqpoint{5.534008in}{3.542608in}}%
\pgfpathcurveto{\pgfqpoint{5.526195in}{3.550422in}}{\pgfqpoint{5.515596in}{3.554812in}}{\pgfqpoint{5.504545in}{3.554812in}}%
\pgfpathcurveto{\pgfqpoint{5.493495in}{3.554812in}}{\pgfqpoint{5.482896in}{3.550422in}}{\pgfqpoint{5.475083in}{3.542608in}}%
\pgfpathcurveto{\pgfqpoint{5.467269in}{3.534794in}}{\pgfqpoint{5.462879in}{3.524195in}}{\pgfqpoint{5.462879in}{3.513145in}}%
\pgfpathcurveto{\pgfqpoint{5.462879in}{3.502095in}}{\pgfqpoint{5.467269in}{3.491496in}}{\pgfqpoint{5.475083in}{3.483682in}}%
\pgfpathcurveto{\pgfqpoint{5.482896in}{3.475869in}}{\pgfqpoint{5.493495in}{3.471478in}}{\pgfqpoint{5.504545in}{3.471478in}}%
\pgfpathclose%
\pgfusepath{stroke,fill}%
\end{pgfscope}%
\begin{pgfscope}%
\pgfpathrectangle{\pgfqpoint{0.800000in}{0.528000in}}{\pgfqpoint{4.960000in}{3.696000in}}%
\pgfusepath{clip}%
\pgfsetbuttcap%
\pgfsetroundjoin%
\definecolor{currentfill}{rgb}{0.000000,0.000000,0.000000}%
\pgfsetfillcolor{currentfill}%
\pgfsetlinewidth{1.003750pt}%
\definecolor{currentstroke}{rgb}{0.000000,0.000000,0.000000}%
\pgfsetstrokecolor{currentstroke}%
\pgfsetdash{}{0pt}%
\pgfpathmoveto{\pgfqpoint{5.504545in}{3.643458in}}%
\pgfpathcurveto{\pgfqpoint{5.515596in}{3.643458in}}{\pgfqpoint{5.526195in}{3.647849in}}{\pgfqpoint{5.534008in}{3.655662in}}%
\pgfpathcurveto{\pgfqpoint{5.541822in}{3.663476in}}{\pgfqpoint{5.546212in}{3.674075in}}{\pgfqpoint{5.546212in}{3.685125in}}%
\pgfpathcurveto{\pgfqpoint{5.546212in}{3.696175in}}{\pgfqpoint{5.541822in}{3.706774in}}{\pgfqpoint{5.534008in}{3.714588in}}%
\pgfpathcurveto{\pgfqpoint{5.526195in}{3.722401in}}{\pgfqpoint{5.515596in}{3.726792in}}{\pgfqpoint{5.504545in}{3.726792in}}%
\pgfpathcurveto{\pgfqpoint{5.493495in}{3.726792in}}{\pgfqpoint{5.482896in}{3.722401in}}{\pgfqpoint{5.475083in}{3.714588in}}%
\pgfpathcurveto{\pgfqpoint{5.467269in}{3.706774in}}{\pgfqpoint{5.462879in}{3.696175in}}{\pgfqpoint{5.462879in}{3.685125in}}%
\pgfpathcurveto{\pgfqpoint{5.462879in}{3.674075in}}{\pgfqpoint{5.467269in}{3.663476in}}{\pgfqpoint{5.475083in}{3.655662in}}%
\pgfpathcurveto{\pgfqpoint{5.482896in}{3.647849in}}{\pgfqpoint{5.493495in}{3.643458in}}{\pgfqpoint{5.504545in}{3.643458in}}%
\pgfpathclose%
\pgfusepath{stroke,fill}%
\end{pgfscope}%
\begin{pgfscope}%
\pgfpathrectangle{\pgfqpoint{0.800000in}{0.528000in}}{\pgfqpoint{4.960000in}{3.696000in}}%
\pgfusepath{clip}%
\pgfsetbuttcap%
\pgfsetroundjoin%
\definecolor{currentfill}{rgb}{0.000000,0.000000,0.000000}%
\pgfsetfillcolor{currentfill}%
\pgfsetlinewidth{1.003750pt}%
\definecolor{currentstroke}{rgb}{0.000000,0.000000,0.000000}%
\pgfsetstrokecolor{currentstroke}%
\pgfsetdash{}{0pt}%
\pgfpathmoveto{\pgfqpoint{5.504545in}{2.826554in}}%
\pgfpathcurveto{\pgfqpoint{5.515596in}{2.826554in}}{\pgfqpoint{5.526195in}{2.830944in}}{\pgfqpoint{5.534008in}{2.838758in}}%
\pgfpathcurveto{\pgfqpoint{5.541822in}{2.846571in}}{\pgfqpoint{5.546212in}{2.857170in}}{\pgfqpoint{5.546212in}{2.868220in}}%
\pgfpathcurveto{\pgfqpoint{5.546212in}{2.879270in}}{\pgfqpoint{5.541822in}{2.889870in}}{\pgfqpoint{5.534008in}{2.897683in}}%
\pgfpathcurveto{\pgfqpoint{5.526195in}{2.905497in}}{\pgfqpoint{5.515596in}{2.909887in}}{\pgfqpoint{5.504545in}{2.909887in}}%
\pgfpathcurveto{\pgfqpoint{5.493495in}{2.909887in}}{\pgfqpoint{5.482896in}{2.905497in}}{\pgfqpoint{5.475083in}{2.897683in}}%
\pgfpathcurveto{\pgfqpoint{5.467269in}{2.889870in}}{\pgfqpoint{5.462879in}{2.879270in}}{\pgfqpoint{5.462879in}{2.868220in}}%
\pgfpathcurveto{\pgfqpoint{5.462879in}{2.857170in}}{\pgfqpoint{5.467269in}{2.846571in}}{\pgfqpoint{5.475083in}{2.838758in}}%
\pgfpathcurveto{\pgfqpoint{5.482896in}{2.830944in}}{\pgfqpoint{5.493495in}{2.826554in}}{\pgfqpoint{5.504545in}{2.826554in}}%
\pgfpathclose%
\pgfusepath{stroke,fill}%
\end{pgfscope}%
\begin{pgfscope}%
\pgfpathrectangle{\pgfqpoint{0.800000in}{0.528000in}}{\pgfqpoint{4.960000in}{3.696000in}}%
\pgfusepath{clip}%
\pgfsetbuttcap%
\pgfsetroundjoin%
\definecolor{currentfill}{rgb}{0.000000,0.000000,0.000000}%
\pgfsetfillcolor{currentfill}%
\pgfsetlinewidth{1.003750pt}%
\definecolor{currentstroke}{rgb}{0.000000,0.000000,0.000000}%
\pgfsetstrokecolor{currentstroke}%
\pgfsetdash{}{0pt}%
\pgfpathmoveto{\pgfqpoint{5.504545in}{2.740564in}}%
\pgfpathcurveto{\pgfqpoint{5.515596in}{2.740564in}}{\pgfqpoint{5.526195in}{2.744954in}}{\pgfqpoint{5.534008in}{2.752768in}}%
\pgfpathcurveto{\pgfqpoint{5.541822in}{2.760581in}}{\pgfqpoint{5.546212in}{2.771180in}}{\pgfqpoint{5.546212in}{2.782230in}}%
\pgfpathcurveto{\pgfqpoint{5.546212in}{2.793281in}}{\pgfqpoint{5.541822in}{2.803880in}}{\pgfqpoint{5.534008in}{2.811693in}}%
\pgfpathcurveto{\pgfqpoint{5.526195in}{2.819507in}}{\pgfqpoint{5.515596in}{2.823897in}}{\pgfqpoint{5.504545in}{2.823897in}}%
\pgfpathcurveto{\pgfqpoint{5.493495in}{2.823897in}}{\pgfqpoint{5.482896in}{2.819507in}}{\pgfqpoint{5.475083in}{2.811693in}}%
\pgfpathcurveto{\pgfqpoint{5.467269in}{2.803880in}}{\pgfqpoint{5.462879in}{2.793281in}}{\pgfqpoint{5.462879in}{2.782230in}}%
\pgfpathcurveto{\pgfqpoint{5.462879in}{2.771180in}}{\pgfqpoint{5.467269in}{2.760581in}}{\pgfqpoint{5.475083in}{2.752768in}}%
\pgfpathcurveto{\pgfqpoint{5.482896in}{2.744954in}}{\pgfqpoint{5.493495in}{2.740564in}}{\pgfqpoint{5.504545in}{2.740564in}}%
\pgfpathclose%
\pgfusepath{stroke,fill}%
\end{pgfscope}%
\begin{pgfscope}%
\pgfpathrectangle{\pgfqpoint{0.800000in}{0.528000in}}{\pgfqpoint{4.960000in}{3.696000in}}%
\pgfusepath{clip}%
\pgfsetbuttcap%
\pgfsetroundjoin%
\definecolor{currentfill}{rgb}{0.000000,0.000000,0.000000}%
\pgfsetfillcolor{currentfill}%
\pgfsetlinewidth{1.003750pt}%
\definecolor{currentstroke}{rgb}{0.000000,0.000000,0.000000}%
\pgfsetstrokecolor{currentstroke}%
\pgfsetdash{}{0pt}%
\pgfpathmoveto{\pgfqpoint{5.504545in}{2.912544in}}%
\pgfpathcurveto{\pgfqpoint{5.515596in}{2.912544in}}{\pgfqpoint{5.526195in}{2.916934in}}{\pgfqpoint{5.534008in}{2.924748in}}%
\pgfpathcurveto{\pgfqpoint{5.541822in}{2.932561in}}{\pgfqpoint{5.546212in}{2.943160in}}{\pgfqpoint{5.546212in}{2.954210in}}%
\pgfpathcurveto{\pgfqpoint{5.546212in}{2.965260in}}{\pgfqpoint{5.541822in}{2.975859in}}{\pgfqpoint{5.534008in}{2.983673in}}%
\pgfpathcurveto{\pgfqpoint{5.526195in}{2.991487in}}{\pgfqpoint{5.515596in}{2.995877in}}{\pgfqpoint{5.504545in}{2.995877in}}%
\pgfpathcurveto{\pgfqpoint{5.493495in}{2.995877in}}{\pgfqpoint{5.482896in}{2.991487in}}{\pgfqpoint{5.475083in}{2.983673in}}%
\pgfpathcurveto{\pgfqpoint{5.467269in}{2.975859in}}{\pgfqpoint{5.462879in}{2.965260in}}{\pgfqpoint{5.462879in}{2.954210in}}%
\pgfpathcurveto{\pgfqpoint{5.462879in}{2.943160in}}{\pgfqpoint{5.467269in}{2.932561in}}{\pgfqpoint{5.475083in}{2.924748in}}%
\pgfpathcurveto{\pgfqpoint{5.482896in}{2.916934in}}{\pgfqpoint{5.493495in}{2.912544in}}{\pgfqpoint{5.504545in}{2.912544in}}%
\pgfpathclose%
\pgfusepath{stroke,fill}%
\end{pgfscope}%
\begin{pgfscope}%
\pgfpathrectangle{\pgfqpoint{0.800000in}{0.528000in}}{\pgfqpoint{4.960000in}{3.696000in}}%
\pgfusepath{clip}%
\pgfsetbuttcap%
\pgfsetroundjoin%
\definecolor{currentfill}{rgb}{0.000000,0.000000,0.000000}%
\pgfsetfillcolor{currentfill}%
\pgfsetlinewidth{1.003750pt}%
\definecolor{currentstroke}{rgb}{0.000000,0.000000,0.000000}%
\pgfsetstrokecolor{currentstroke}%
\pgfsetdash{}{0pt}%
\pgfpathmoveto{\pgfqpoint{5.504545in}{2.869549in}}%
\pgfpathcurveto{\pgfqpoint{5.515596in}{2.869549in}}{\pgfqpoint{5.526195in}{2.873939in}}{\pgfqpoint{5.534008in}{2.881753in}}%
\pgfpathcurveto{\pgfqpoint{5.541822in}{2.889566in}}{\pgfqpoint{5.546212in}{2.900165in}}{\pgfqpoint{5.546212in}{2.911215in}}%
\pgfpathcurveto{\pgfqpoint{5.546212in}{2.922265in}}{\pgfqpoint{5.541822in}{2.932864in}}{\pgfqpoint{5.534008in}{2.940678in}}%
\pgfpathcurveto{\pgfqpoint{5.526195in}{2.948492in}}{\pgfqpoint{5.515596in}{2.952882in}}{\pgfqpoint{5.504545in}{2.952882in}}%
\pgfpathcurveto{\pgfqpoint{5.493495in}{2.952882in}}{\pgfqpoint{5.482896in}{2.948492in}}{\pgfqpoint{5.475083in}{2.940678in}}%
\pgfpathcurveto{\pgfqpoint{5.467269in}{2.932864in}}{\pgfqpoint{5.462879in}{2.922265in}}{\pgfqpoint{5.462879in}{2.911215in}}%
\pgfpathcurveto{\pgfqpoint{5.462879in}{2.900165in}}{\pgfqpoint{5.467269in}{2.889566in}}{\pgfqpoint{5.475083in}{2.881753in}}%
\pgfpathcurveto{\pgfqpoint{5.482896in}{2.873939in}}{\pgfqpoint{5.493495in}{2.869549in}}{\pgfqpoint{5.504545in}{2.869549in}}%
\pgfpathclose%
\pgfusepath{stroke,fill}%
\end{pgfscope}%
\begin{pgfscope}%
\pgfpathrectangle{\pgfqpoint{0.800000in}{0.528000in}}{\pgfqpoint{4.960000in}{3.696000in}}%
\pgfusepath{clip}%
\pgfsetbuttcap%
\pgfsetroundjoin%
\definecolor{currentfill}{rgb}{0.000000,0.000000,0.000000}%
\pgfsetfillcolor{currentfill}%
\pgfsetlinewidth{1.003750pt}%
\definecolor{currentstroke}{rgb}{0.000000,0.000000,0.000000}%
\pgfsetstrokecolor{currentstroke}%
\pgfsetdash{}{0pt}%
\pgfpathmoveto{\pgfqpoint{5.504545in}{2.719066in}}%
\pgfpathcurveto{\pgfqpoint{5.515596in}{2.719066in}}{\pgfqpoint{5.526195in}{2.723456in}}{\pgfqpoint{5.534008in}{2.731270in}}%
\pgfpathcurveto{\pgfqpoint{5.541822in}{2.739084in}}{\pgfqpoint{5.546212in}{2.749683in}}{\pgfqpoint{5.546212in}{2.760733in}}%
\pgfpathcurveto{\pgfqpoint{5.546212in}{2.771783in}}{\pgfqpoint{5.541822in}{2.782382in}}{\pgfqpoint{5.534008in}{2.790196in}}%
\pgfpathcurveto{\pgfqpoint{5.526195in}{2.798009in}}{\pgfqpoint{5.515596in}{2.802400in}}{\pgfqpoint{5.504545in}{2.802400in}}%
\pgfpathcurveto{\pgfqpoint{5.493495in}{2.802400in}}{\pgfqpoint{5.482896in}{2.798009in}}{\pgfqpoint{5.475083in}{2.790196in}}%
\pgfpathcurveto{\pgfqpoint{5.467269in}{2.782382in}}{\pgfqpoint{5.462879in}{2.771783in}}{\pgfqpoint{5.462879in}{2.760733in}}%
\pgfpathcurveto{\pgfqpoint{5.462879in}{2.749683in}}{\pgfqpoint{5.467269in}{2.739084in}}{\pgfqpoint{5.475083in}{2.731270in}}%
\pgfpathcurveto{\pgfqpoint{5.482896in}{2.723456in}}{\pgfqpoint{5.493495in}{2.719066in}}{\pgfqpoint{5.504545in}{2.719066in}}%
\pgfpathclose%
\pgfusepath{stroke,fill}%
\end{pgfscope}%
\begin{pgfscope}%
\pgfpathrectangle{\pgfqpoint{0.800000in}{0.528000in}}{\pgfqpoint{4.960000in}{3.696000in}}%
\pgfusepath{clip}%
\pgfsetbuttcap%
\pgfsetroundjoin%
\definecolor{currentfill}{rgb}{0.000000,0.000000,0.000000}%
\pgfsetfillcolor{currentfill}%
\pgfsetlinewidth{1.003750pt}%
\definecolor{currentstroke}{rgb}{0.000000,0.000000,0.000000}%
\pgfsetstrokecolor{currentstroke}%
\pgfsetdash{}{0pt}%
\pgfpathmoveto{\pgfqpoint{5.504545in}{3.084524in}}%
\pgfpathcurveto{\pgfqpoint{5.515596in}{3.084524in}}{\pgfqpoint{5.526195in}{3.088914in}}{\pgfqpoint{5.534008in}{3.096727in}}%
\pgfpathcurveto{\pgfqpoint{5.541822in}{3.104541in}}{\pgfqpoint{5.546212in}{3.115140in}}{\pgfqpoint{5.546212in}{3.126190in}}%
\pgfpathcurveto{\pgfqpoint{5.546212in}{3.137240in}}{\pgfqpoint{5.541822in}{3.147839in}}{\pgfqpoint{5.534008in}{3.155653in}}%
\pgfpathcurveto{\pgfqpoint{5.526195in}{3.163467in}}{\pgfqpoint{5.515596in}{3.167857in}}{\pgfqpoint{5.504545in}{3.167857in}}%
\pgfpathcurveto{\pgfqpoint{5.493495in}{3.167857in}}{\pgfqpoint{5.482896in}{3.163467in}}{\pgfqpoint{5.475083in}{3.155653in}}%
\pgfpathcurveto{\pgfqpoint{5.467269in}{3.147839in}}{\pgfqpoint{5.462879in}{3.137240in}}{\pgfqpoint{5.462879in}{3.126190in}}%
\pgfpathcurveto{\pgfqpoint{5.462879in}{3.115140in}}{\pgfqpoint{5.467269in}{3.104541in}}{\pgfqpoint{5.475083in}{3.096727in}}%
\pgfpathcurveto{\pgfqpoint{5.482896in}{3.088914in}}{\pgfqpoint{5.493495in}{3.084524in}}{\pgfqpoint{5.504545in}{3.084524in}}%
\pgfpathclose%
\pgfusepath{stroke,fill}%
\end{pgfscope}%
\begin{pgfscope}%
\pgfpathrectangle{\pgfqpoint{0.800000in}{0.528000in}}{\pgfqpoint{4.960000in}{3.696000in}}%
\pgfusepath{clip}%
\pgfsetbuttcap%
\pgfsetroundjoin%
\definecolor{currentfill}{rgb}{0.000000,0.000000,0.000000}%
\pgfsetfillcolor{currentfill}%
\pgfsetlinewidth{1.003750pt}%
\definecolor{currentstroke}{rgb}{0.000000,0.000000,0.000000}%
\pgfsetstrokecolor{currentstroke}%
\pgfsetdash{}{0pt}%
\pgfpathmoveto{\pgfqpoint{5.504545in}{2.934041in}}%
\pgfpathcurveto{\pgfqpoint{5.515596in}{2.934041in}}{\pgfqpoint{5.526195in}{2.938431in}}{\pgfqpoint{5.534008in}{2.946245in}}%
\pgfpathcurveto{\pgfqpoint{5.541822in}{2.954059in}}{\pgfqpoint{5.546212in}{2.964658in}}{\pgfqpoint{5.546212in}{2.975708in}}%
\pgfpathcurveto{\pgfqpoint{5.546212in}{2.986758in}}{\pgfqpoint{5.541822in}{2.997357in}}{\pgfqpoint{5.534008in}{3.005171in}}%
\pgfpathcurveto{\pgfqpoint{5.526195in}{3.012984in}}{\pgfqpoint{5.515596in}{3.017374in}}{\pgfqpoint{5.504545in}{3.017374in}}%
\pgfpathcurveto{\pgfqpoint{5.493495in}{3.017374in}}{\pgfqpoint{5.482896in}{3.012984in}}{\pgfqpoint{5.475083in}{3.005171in}}%
\pgfpathcurveto{\pgfqpoint{5.467269in}{2.997357in}}{\pgfqpoint{5.462879in}{2.986758in}}{\pgfqpoint{5.462879in}{2.975708in}}%
\pgfpathcurveto{\pgfqpoint{5.462879in}{2.964658in}}{\pgfqpoint{5.467269in}{2.954059in}}{\pgfqpoint{5.475083in}{2.946245in}}%
\pgfpathcurveto{\pgfqpoint{5.482896in}{2.938431in}}{\pgfqpoint{5.493495in}{2.934041in}}{\pgfqpoint{5.504545in}{2.934041in}}%
\pgfpathclose%
\pgfusepath{stroke,fill}%
\end{pgfscope}%
\begin{pgfscope}%
\pgfpathrectangle{\pgfqpoint{0.800000in}{0.528000in}}{\pgfqpoint{4.960000in}{3.696000in}}%
\pgfusepath{clip}%
\pgfsetbuttcap%
\pgfsetroundjoin%
\definecolor{currentfill}{rgb}{0.000000,0.000000,0.000000}%
\pgfsetfillcolor{currentfill}%
\pgfsetlinewidth{1.003750pt}%
\definecolor{currentstroke}{rgb}{0.000000,0.000000,0.000000}%
\pgfsetstrokecolor{currentstroke}%
\pgfsetdash{}{0pt}%
\pgfpathmoveto{\pgfqpoint{5.504545in}{2.719066in}}%
\pgfpathcurveto{\pgfqpoint{5.515596in}{2.719066in}}{\pgfqpoint{5.526195in}{2.723456in}}{\pgfqpoint{5.534008in}{2.731270in}}%
\pgfpathcurveto{\pgfqpoint{5.541822in}{2.739084in}}{\pgfqpoint{5.546212in}{2.749683in}}{\pgfqpoint{5.546212in}{2.760733in}}%
\pgfpathcurveto{\pgfqpoint{5.546212in}{2.771783in}}{\pgfqpoint{5.541822in}{2.782382in}}{\pgfqpoint{5.534008in}{2.790196in}}%
\pgfpathcurveto{\pgfqpoint{5.526195in}{2.798009in}}{\pgfqpoint{5.515596in}{2.802400in}}{\pgfqpoint{5.504545in}{2.802400in}}%
\pgfpathcurveto{\pgfqpoint{5.493495in}{2.802400in}}{\pgfqpoint{5.482896in}{2.798009in}}{\pgfqpoint{5.475083in}{2.790196in}}%
\pgfpathcurveto{\pgfqpoint{5.467269in}{2.782382in}}{\pgfqpoint{5.462879in}{2.771783in}}{\pgfqpoint{5.462879in}{2.760733in}}%
\pgfpathcurveto{\pgfqpoint{5.462879in}{2.749683in}}{\pgfqpoint{5.467269in}{2.739084in}}{\pgfqpoint{5.475083in}{2.731270in}}%
\pgfpathcurveto{\pgfqpoint{5.482896in}{2.723456in}}{\pgfqpoint{5.493495in}{2.719066in}}{\pgfqpoint{5.504545in}{2.719066in}}%
\pgfpathclose%
\pgfusepath{stroke,fill}%
\end{pgfscope}%
\begin{pgfscope}%
\pgfpathrectangle{\pgfqpoint{0.800000in}{0.528000in}}{\pgfqpoint{4.960000in}{3.696000in}}%
\pgfusepath{clip}%
\pgfsetbuttcap%
\pgfsetroundjoin%
\definecolor{currentfill}{rgb}{0.000000,0.000000,0.000000}%
\pgfsetfillcolor{currentfill}%
\pgfsetlinewidth{1.003750pt}%
\definecolor{currentstroke}{rgb}{0.000000,0.000000,0.000000}%
\pgfsetstrokecolor{currentstroke}%
\pgfsetdash{}{0pt}%
\pgfpathmoveto{\pgfqpoint{5.504545in}{3.170514in}}%
\pgfpathcurveto{\pgfqpoint{5.515596in}{3.170514in}}{\pgfqpoint{5.526195in}{3.174904in}}{\pgfqpoint{5.534008in}{3.182717in}}%
\pgfpathcurveto{\pgfqpoint{5.541822in}{3.190531in}}{\pgfqpoint{5.546212in}{3.201130in}}{\pgfqpoint{5.546212in}{3.212180in}}%
\pgfpathcurveto{\pgfqpoint{5.546212in}{3.223230in}}{\pgfqpoint{5.541822in}{3.233829in}}{\pgfqpoint{5.534008in}{3.241643in}}%
\pgfpathcurveto{\pgfqpoint{5.526195in}{3.249457in}}{\pgfqpoint{5.515596in}{3.253847in}}{\pgfqpoint{5.504545in}{3.253847in}}%
\pgfpathcurveto{\pgfqpoint{5.493495in}{3.253847in}}{\pgfqpoint{5.482896in}{3.249457in}}{\pgfqpoint{5.475083in}{3.241643in}}%
\pgfpathcurveto{\pgfqpoint{5.467269in}{3.233829in}}{\pgfqpoint{5.462879in}{3.223230in}}{\pgfqpoint{5.462879in}{3.212180in}}%
\pgfpathcurveto{\pgfqpoint{5.462879in}{3.201130in}}{\pgfqpoint{5.467269in}{3.190531in}}{\pgfqpoint{5.475083in}{3.182717in}}%
\pgfpathcurveto{\pgfqpoint{5.482896in}{3.174904in}}{\pgfqpoint{5.493495in}{3.170514in}}{\pgfqpoint{5.504545in}{3.170514in}}%
\pgfpathclose%
\pgfusepath{stroke,fill}%
\end{pgfscope}%
\begin{pgfscope}%
\pgfpathrectangle{\pgfqpoint{0.800000in}{0.528000in}}{\pgfqpoint{4.960000in}{3.696000in}}%
\pgfusepath{clip}%
\pgfsetbuttcap%
\pgfsetroundjoin%
\definecolor{currentfill}{rgb}{0.000000,0.000000,0.000000}%
\pgfsetfillcolor{currentfill}%
\pgfsetlinewidth{1.003750pt}%
\definecolor{currentstroke}{rgb}{0.000000,0.000000,0.000000}%
\pgfsetstrokecolor{currentstroke}%
\pgfsetdash{}{0pt}%
\pgfpathmoveto{\pgfqpoint{5.504545in}{2.740564in}}%
\pgfpathcurveto{\pgfqpoint{5.515596in}{2.740564in}}{\pgfqpoint{5.526195in}{2.744954in}}{\pgfqpoint{5.534008in}{2.752768in}}%
\pgfpathcurveto{\pgfqpoint{5.541822in}{2.760581in}}{\pgfqpoint{5.546212in}{2.771180in}}{\pgfqpoint{5.546212in}{2.782230in}}%
\pgfpathcurveto{\pgfqpoint{5.546212in}{2.793281in}}{\pgfqpoint{5.541822in}{2.803880in}}{\pgfqpoint{5.534008in}{2.811693in}}%
\pgfpathcurveto{\pgfqpoint{5.526195in}{2.819507in}}{\pgfqpoint{5.515596in}{2.823897in}}{\pgfqpoint{5.504545in}{2.823897in}}%
\pgfpathcurveto{\pgfqpoint{5.493495in}{2.823897in}}{\pgfqpoint{5.482896in}{2.819507in}}{\pgfqpoint{5.475083in}{2.811693in}}%
\pgfpathcurveto{\pgfqpoint{5.467269in}{2.803880in}}{\pgfqpoint{5.462879in}{2.793281in}}{\pgfqpoint{5.462879in}{2.782230in}}%
\pgfpathcurveto{\pgfqpoint{5.462879in}{2.771180in}}{\pgfqpoint{5.467269in}{2.760581in}}{\pgfqpoint{5.475083in}{2.752768in}}%
\pgfpathcurveto{\pgfqpoint{5.482896in}{2.744954in}}{\pgfqpoint{5.493495in}{2.740564in}}{\pgfqpoint{5.504545in}{2.740564in}}%
\pgfpathclose%
\pgfusepath{stroke,fill}%
\end{pgfscope}%
\begin{pgfscope}%
\pgfpathrectangle{\pgfqpoint{0.800000in}{0.528000in}}{\pgfqpoint{4.960000in}{3.696000in}}%
\pgfusepath{clip}%
\pgfsetbuttcap%
\pgfsetroundjoin%
\definecolor{currentfill}{rgb}{0.000000,0.000000,0.000000}%
\pgfsetfillcolor{currentfill}%
\pgfsetlinewidth{1.003750pt}%
\definecolor{currentstroke}{rgb}{0.000000,0.000000,0.000000}%
\pgfsetstrokecolor{currentstroke}%
\pgfsetdash{}{0pt}%
\pgfpathmoveto{\pgfqpoint{5.504545in}{3.127519in}}%
\pgfpathcurveto{\pgfqpoint{5.515596in}{3.127519in}}{\pgfqpoint{5.526195in}{3.131909in}}{\pgfqpoint{5.534008in}{3.139722in}}%
\pgfpathcurveto{\pgfqpoint{5.541822in}{3.147536in}}{\pgfqpoint{5.546212in}{3.158135in}}{\pgfqpoint{5.546212in}{3.169185in}}%
\pgfpathcurveto{\pgfqpoint{5.546212in}{3.180235in}}{\pgfqpoint{5.541822in}{3.190834in}}{\pgfqpoint{5.534008in}{3.198648in}}%
\pgfpathcurveto{\pgfqpoint{5.526195in}{3.206462in}}{\pgfqpoint{5.515596in}{3.210852in}}{\pgfqpoint{5.504545in}{3.210852in}}%
\pgfpathcurveto{\pgfqpoint{5.493495in}{3.210852in}}{\pgfqpoint{5.482896in}{3.206462in}}{\pgfqpoint{5.475083in}{3.198648in}}%
\pgfpathcurveto{\pgfqpoint{5.467269in}{3.190834in}}{\pgfqpoint{5.462879in}{3.180235in}}{\pgfqpoint{5.462879in}{3.169185in}}%
\pgfpathcurveto{\pgfqpoint{5.462879in}{3.158135in}}{\pgfqpoint{5.467269in}{3.147536in}}{\pgfqpoint{5.475083in}{3.139722in}}%
\pgfpathcurveto{\pgfqpoint{5.482896in}{3.131909in}}{\pgfqpoint{5.493495in}{3.127519in}}{\pgfqpoint{5.504545in}{3.127519in}}%
\pgfpathclose%
\pgfusepath{stroke,fill}%
\end{pgfscope}%
\begin{pgfscope}%
\pgfpathrectangle{\pgfqpoint{0.800000in}{0.528000in}}{\pgfqpoint{4.960000in}{3.696000in}}%
\pgfusepath{clip}%
\pgfsetbuttcap%
\pgfsetroundjoin%
\definecolor{currentfill}{rgb}{0.000000,0.000000,0.000000}%
\pgfsetfillcolor{currentfill}%
\pgfsetlinewidth{1.003750pt}%
\definecolor{currentstroke}{rgb}{0.000000,0.000000,0.000000}%
\pgfsetstrokecolor{currentstroke}%
\pgfsetdash{}{0pt}%
\pgfpathmoveto{\pgfqpoint{5.504545in}{2.998534in}}%
\pgfpathcurveto{\pgfqpoint{5.515596in}{2.998534in}}{\pgfqpoint{5.526195in}{3.002924in}}{\pgfqpoint{5.534008in}{3.010738in}}%
\pgfpathcurveto{\pgfqpoint{5.541822in}{3.018551in}}{\pgfqpoint{5.546212in}{3.029150in}}{\pgfqpoint{5.546212in}{3.040200in}}%
\pgfpathcurveto{\pgfqpoint{5.546212in}{3.051250in}}{\pgfqpoint{5.541822in}{3.061849in}}{\pgfqpoint{5.534008in}{3.069663in}}%
\pgfpathcurveto{\pgfqpoint{5.526195in}{3.077477in}}{\pgfqpoint{5.515596in}{3.081867in}}{\pgfqpoint{5.504545in}{3.081867in}}%
\pgfpathcurveto{\pgfqpoint{5.493495in}{3.081867in}}{\pgfqpoint{5.482896in}{3.077477in}}{\pgfqpoint{5.475083in}{3.069663in}}%
\pgfpathcurveto{\pgfqpoint{5.467269in}{3.061849in}}{\pgfqpoint{5.462879in}{3.051250in}}{\pgfqpoint{5.462879in}{3.040200in}}%
\pgfpathcurveto{\pgfqpoint{5.462879in}{3.029150in}}{\pgfqpoint{5.467269in}{3.018551in}}{\pgfqpoint{5.475083in}{3.010738in}}%
\pgfpathcurveto{\pgfqpoint{5.482896in}{3.002924in}}{\pgfqpoint{5.493495in}{2.998534in}}{\pgfqpoint{5.504545in}{2.998534in}}%
\pgfpathclose%
\pgfusepath{stroke,fill}%
\end{pgfscope}%
\begin{pgfscope}%
\pgfpathrectangle{\pgfqpoint{0.800000in}{0.528000in}}{\pgfqpoint{4.960000in}{3.696000in}}%
\pgfusepath{clip}%
\pgfsetbuttcap%
\pgfsetroundjoin%
\definecolor{currentfill}{rgb}{0.000000,0.000000,0.000000}%
\pgfsetfillcolor{currentfill}%
\pgfsetlinewidth{1.003750pt}%
\definecolor{currentstroke}{rgb}{0.000000,0.000000,0.000000}%
\pgfsetstrokecolor{currentstroke}%
\pgfsetdash{}{0pt}%
\pgfpathmoveto{\pgfqpoint{5.504545in}{2.848051in}}%
\pgfpathcurveto{\pgfqpoint{5.515596in}{2.848051in}}{\pgfqpoint{5.526195in}{2.852441in}}{\pgfqpoint{5.534008in}{2.860255in}}%
\pgfpathcurveto{\pgfqpoint{5.541822in}{2.868069in}}{\pgfqpoint{5.546212in}{2.878668in}}{\pgfqpoint{5.546212in}{2.889718in}}%
\pgfpathcurveto{\pgfqpoint{5.546212in}{2.900768in}}{\pgfqpoint{5.541822in}{2.911367in}}{\pgfqpoint{5.534008in}{2.919181in}}%
\pgfpathcurveto{\pgfqpoint{5.526195in}{2.926994in}}{\pgfqpoint{5.515596in}{2.931385in}}{\pgfqpoint{5.504545in}{2.931385in}}%
\pgfpathcurveto{\pgfqpoint{5.493495in}{2.931385in}}{\pgfqpoint{5.482896in}{2.926994in}}{\pgfqpoint{5.475083in}{2.919181in}}%
\pgfpathcurveto{\pgfqpoint{5.467269in}{2.911367in}}{\pgfqpoint{5.462879in}{2.900768in}}{\pgfqpoint{5.462879in}{2.889718in}}%
\pgfpathcurveto{\pgfqpoint{5.462879in}{2.878668in}}{\pgfqpoint{5.467269in}{2.868069in}}{\pgfqpoint{5.475083in}{2.860255in}}%
\pgfpathcurveto{\pgfqpoint{5.482896in}{2.852441in}}{\pgfqpoint{5.493495in}{2.848051in}}{\pgfqpoint{5.504545in}{2.848051in}}%
\pgfpathclose%
\pgfusepath{stroke,fill}%
\end{pgfscope}%
\begin{pgfscope}%
\pgfpathrectangle{\pgfqpoint{0.800000in}{0.528000in}}{\pgfqpoint{4.960000in}{3.696000in}}%
\pgfusepath{clip}%
\pgfsetbuttcap%
\pgfsetroundjoin%
\definecolor{currentfill}{rgb}{0.000000,0.000000,0.000000}%
\pgfsetfillcolor{currentfill}%
\pgfsetlinewidth{1.003750pt}%
\definecolor{currentstroke}{rgb}{0.000000,0.000000,0.000000}%
\pgfsetstrokecolor{currentstroke}%
\pgfsetdash{}{0pt}%
\pgfpathmoveto{\pgfqpoint{5.504545in}{3.320996in}}%
\pgfpathcurveto{\pgfqpoint{5.515596in}{3.320996in}}{\pgfqpoint{5.526195in}{3.325386in}}{\pgfqpoint{5.534008in}{3.333200in}}%
\pgfpathcurveto{\pgfqpoint{5.541822in}{3.341014in}}{\pgfqpoint{5.546212in}{3.351613in}}{\pgfqpoint{5.546212in}{3.362663in}}%
\pgfpathcurveto{\pgfqpoint{5.546212in}{3.373713in}}{\pgfqpoint{5.541822in}{3.384312in}}{\pgfqpoint{5.534008in}{3.392125in}}%
\pgfpathcurveto{\pgfqpoint{5.526195in}{3.399939in}}{\pgfqpoint{5.515596in}{3.404329in}}{\pgfqpoint{5.504545in}{3.404329in}}%
\pgfpathcurveto{\pgfqpoint{5.493495in}{3.404329in}}{\pgfqpoint{5.482896in}{3.399939in}}{\pgfqpoint{5.475083in}{3.392125in}}%
\pgfpathcurveto{\pgfqpoint{5.467269in}{3.384312in}}{\pgfqpoint{5.462879in}{3.373713in}}{\pgfqpoint{5.462879in}{3.362663in}}%
\pgfpathcurveto{\pgfqpoint{5.462879in}{3.351613in}}{\pgfqpoint{5.467269in}{3.341014in}}{\pgfqpoint{5.475083in}{3.333200in}}%
\pgfpathcurveto{\pgfqpoint{5.482896in}{3.325386in}}{\pgfqpoint{5.493495in}{3.320996in}}{\pgfqpoint{5.504545in}{3.320996in}}%
\pgfpathclose%
\pgfusepath{stroke,fill}%
\end{pgfscope}%
\begin{pgfscope}%
\pgfpathrectangle{\pgfqpoint{0.800000in}{0.528000in}}{\pgfqpoint{4.960000in}{3.696000in}}%
\pgfusepath{clip}%
\pgfsetbuttcap%
\pgfsetroundjoin%
\definecolor{currentfill}{rgb}{0.000000,0.000000,0.000000}%
\pgfsetfillcolor{currentfill}%
\pgfsetlinewidth{1.003750pt}%
\definecolor{currentstroke}{rgb}{0.000000,0.000000,0.000000}%
\pgfsetstrokecolor{currentstroke}%
\pgfsetdash{}{0pt}%
\pgfpathmoveto{\pgfqpoint{5.504545in}{2.826554in}}%
\pgfpathcurveto{\pgfqpoint{5.515596in}{2.826554in}}{\pgfqpoint{5.526195in}{2.830944in}}{\pgfqpoint{5.534008in}{2.838758in}}%
\pgfpathcurveto{\pgfqpoint{5.541822in}{2.846571in}}{\pgfqpoint{5.546212in}{2.857170in}}{\pgfqpoint{5.546212in}{2.868220in}}%
\pgfpathcurveto{\pgfqpoint{5.546212in}{2.879270in}}{\pgfqpoint{5.541822in}{2.889870in}}{\pgfqpoint{5.534008in}{2.897683in}}%
\pgfpathcurveto{\pgfqpoint{5.526195in}{2.905497in}}{\pgfqpoint{5.515596in}{2.909887in}}{\pgfqpoint{5.504545in}{2.909887in}}%
\pgfpathcurveto{\pgfqpoint{5.493495in}{2.909887in}}{\pgfqpoint{5.482896in}{2.905497in}}{\pgfqpoint{5.475083in}{2.897683in}}%
\pgfpathcurveto{\pgfqpoint{5.467269in}{2.889870in}}{\pgfqpoint{5.462879in}{2.879270in}}{\pgfqpoint{5.462879in}{2.868220in}}%
\pgfpathcurveto{\pgfqpoint{5.462879in}{2.857170in}}{\pgfqpoint{5.467269in}{2.846571in}}{\pgfqpoint{5.475083in}{2.838758in}}%
\pgfpathcurveto{\pgfqpoint{5.482896in}{2.830944in}}{\pgfqpoint{5.493495in}{2.826554in}}{\pgfqpoint{5.504545in}{2.826554in}}%
\pgfpathclose%
\pgfusepath{stroke,fill}%
\end{pgfscope}%
\begin{pgfscope}%
\pgfpathrectangle{\pgfqpoint{0.800000in}{0.528000in}}{\pgfqpoint{4.960000in}{3.696000in}}%
\pgfusepath{clip}%
\pgfsetbuttcap%
\pgfsetroundjoin%
\definecolor{currentfill}{rgb}{0.000000,0.000000,0.000000}%
\pgfsetfillcolor{currentfill}%
\pgfsetlinewidth{1.003750pt}%
\definecolor{currentstroke}{rgb}{0.000000,0.000000,0.000000}%
\pgfsetstrokecolor{currentstroke}%
\pgfsetdash{}{0pt}%
\pgfpathmoveto{\pgfqpoint{5.504545in}{3.987418in}}%
\pgfpathcurveto{\pgfqpoint{5.515596in}{3.987418in}}{\pgfqpoint{5.526195in}{3.991809in}}{\pgfqpoint{5.534008in}{3.999622in}}%
\pgfpathcurveto{\pgfqpoint{5.541822in}{4.007436in}}{\pgfqpoint{5.546212in}{4.018035in}}{\pgfqpoint{5.546212in}{4.029085in}}%
\pgfpathcurveto{\pgfqpoint{5.546212in}{4.040135in}}{\pgfqpoint{5.541822in}{4.050734in}}{\pgfqpoint{5.534008in}{4.058548in}}%
\pgfpathcurveto{\pgfqpoint{5.526195in}{4.066361in}}{\pgfqpoint{5.515596in}{4.070752in}}{\pgfqpoint{5.504545in}{4.070752in}}%
\pgfpathcurveto{\pgfqpoint{5.493495in}{4.070752in}}{\pgfqpoint{5.482896in}{4.066361in}}{\pgfqpoint{5.475083in}{4.058548in}}%
\pgfpathcurveto{\pgfqpoint{5.467269in}{4.050734in}}{\pgfqpoint{5.462879in}{4.040135in}}{\pgfqpoint{5.462879in}{4.029085in}}%
\pgfpathcurveto{\pgfqpoint{5.462879in}{4.018035in}}{\pgfqpoint{5.467269in}{4.007436in}}{\pgfqpoint{5.475083in}{3.999622in}}%
\pgfpathcurveto{\pgfqpoint{5.482896in}{3.991809in}}{\pgfqpoint{5.493495in}{3.987418in}}{\pgfqpoint{5.504545in}{3.987418in}}%
\pgfpathclose%
\pgfusepath{stroke,fill}%
\end{pgfscope}%
\begin{pgfscope}%
\pgfpathrectangle{\pgfqpoint{0.800000in}{0.528000in}}{\pgfqpoint{4.960000in}{3.696000in}}%
\pgfusepath{clip}%
\pgfsetbuttcap%
\pgfsetroundjoin%
\definecolor{currentfill}{rgb}{0.000000,0.000000,0.000000}%
\pgfsetfillcolor{currentfill}%
\pgfsetlinewidth{1.003750pt}%
\definecolor{currentstroke}{rgb}{0.000000,0.000000,0.000000}%
\pgfsetstrokecolor{currentstroke}%
\pgfsetdash{}{0pt}%
\pgfpathmoveto{\pgfqpoint{5.504545in}{3.063026in}}%
\pgfpathcurveto{\pgfqpoint{5.515596in}{3.063026in}}{\pgfqpoint{5.526195in}{3.067416in}}{\pgfqpoint{5.534008in}{3.075230in}}%
\pgfpathcurveto{\pgfqpoint{5.541822in}{3.083044in}}{\pgfqpoint{5.546212in}{3.093643in}}{\pgfqpoint{5.546212in}{3.104693in}}%
\pgfpathcurveto{\pgfqpoint{5.546212in}{3.115743in}}{\pgfqpoint{5.541822in}{3.126342in}}{\pgfqpoint{5.534008in}{3.134156in}}%
\pgfpathcurveto{\pgfqpoint{5.526195in}{3.141969in}}{\pgfqpoint{5.515596in}{3.146359in}}{\pgfqpoint{5.504545in}{3.146359in}}%
\pgfpathcurveto{\pgfqpoint{5.493495in}{3.146359in}}{\pgfqpoint{5.482896in}{3.141969in}}{\pgfqpoint{5.475083in}{3.134156in}}%
\pgfpathcurveto{\pgfqpoint{5.467269in}{3.126342in}}{\pgfqpoint{5.462879in}{3.115743in}}{\pgfqpoint{5.462879in}{3.104693in}}%
\pgfpathcurveto{\pgfqpoint{5.462879in}{3.093643in}}{\pgfqpoint{5.467269in}{3.083044in}}{\pgfqpoint{5.475083in}{3.075230in}}%
\pgfpathcurveto{\pgfqpoint{5.482896in}{3.067416in}}{\pgfqpoint{5.493495in}{3.063026in}}{\pgfqpoint{5.504545in}{3.063026in}}%
\pgfpathclose%
\pgfusepath{stroke,fill}%
\end{pgfscope}%
\begin{pgfscope}%
\pgfpathrectangle{\pgfqpoint{0.800000in}{0.528000in}}{\pgfqpoint{4.960000in}{3.696000in}}%
\pgfusepath{clip}%
\pgfsetbuttcap%
\pgfsetroundjoin%
\definecolor{currentfill}{rgb}{0.000000,0.000000,0.000000}%
\pgfsetfillcolor{currentfill}%
\pgfsetlinewidth{1.003750pt}%
\definecolor{currentstroke}{rgb}{0.000000,0.000000,0.000000}%
\pgfsetstrokecolor{currentstroke}%
\pgfsetdash{}{0pt}%
\pgfpathmoveto{\pgfqpoint{5.504545in}{2.848051in}}%
\pgfpathcurveto{\pgfqpoint{5.515596in}{2.848051in}}{\pgfqpoint{5.526195in}{2.852441in}}{\pgfqpoint{5.534008in}{2.860255in}}%
\pgfpathcurveto{\pgfqpoint{5.541822in}{2.868069in}}{\pgfqpoint{5.546212in}{2.878668in}}{\pgfqpoint{5.546212in}{2.889718in}}%
\pgfpathcurveto{\pgfqpoint{5.546212in}{2.900768in}}{\pgfqpoint{5.541822in}{2.911367in}}{\pgfqpoint{5.534008in}{2.919181in}}%
\pgfpathcurveto{\pgfqpoint{5.526195in}{2.926994in}}{\pgfqpoint{5.515596in}{2.931385in}}{\pgfqpoint{5.504545in}{2.931385in}}%
\pgfpathcurveto{\pgfqpoint{5.493495in}{2.931385in}}{\pgfqpoint{5.482896in}{2.926994in}}{\pgfqpoint{5.475083in}{2.919181in}}%
\pgfpathcurveto{\pgfqpoint{5.467269in}{2.911367in}}{\pgfqpoint{5.462879in}{2.900768in}}{\pgfqpoint{5.462879in}{2.889718in}}%
\pgfpathcurveto{\pgfqpoint{5.462879in}{2.878668in}}{\pgfqpoint{5.467269in}{2.868069in}}{\pgfqpoint{5.475083in}{2.860255in}}%
\pgfpathcurveto{\pgfqpoint{5.482896in}{2.852441in}}{\pgfqpoint{5.493495in}{2.848051in}}{\pgfqpoint{5.504545in}{2.848051in}}%
\pgfpathclose%
\pgfusepath{stroke,fill}%
\end{pgfscope}%
\begin{pgfscope}%
\pgfpathrectangle{\pgfqpoint{0.800000in}{0.528000in}}{\pgfqpoint{4.960000in}{3.696000in}}%
\pgfusepath{clip}%
\pgfsetbuttcap%
\pgfsetroundjoin%
\definecolor{currentfill}{rgb}{0.000000,0.000000,0.000000}%
\pgfsetfillcolor{currentfill}%
\pgfsetlinewidth{1.003750pt}%
\definecolor{currentstroke}{rgb}{0.000000,0.000000,0.000000}%
\pgfsetstrokecolor{currentstroke}%
\pgfsetdash{}{0pt}%
\pgfpathmoveto{\pgfqpoint{5.504545in}{2.762061in}}%
\pgfpathcurveto{\pgfqpoint{5.515596in}{2.762061in}}{\pgfqpoint{5.526195in}{2.766451in}}{\pgfqpoint{5.534008in}{2.774265in}}%
\pgfpathcurveto{\pgfqpoint{5.541822in}{2.782079in}}{\pgfqpoint{5.546212in}{2.792678in}}{\pgfqpoint{5.546212in}{2.803728in}}%
\pgfpathcurveto{\pgfqpoint{5.546212in}{2.814778in}}{\pgfqpoint{5.541822in}{2.825377in}}{\pgfqpoint{5.534008in}{2.833191in}}%
\pgfpathcurveto{\pgfqpoint{5.526195in}{2.841004in}}{\pgfqpoint{5.515596in}{2.845395in}}{\pgfqpoint{5.504545in}{2.845395in}}%
\pgfpathcurveto{\pgfqpoint{5.493495in}{2.845395in}}{\pgfqpoint{5.482896in}{2.841004in}}{\pgfqpoint{5.475083in}{2.833191in}}%
\pgfpathcurveto{\pgfqpoint{5.467269in}{2.825377in}}{\pgfqpoint{5.462879in}{2.814778in}}{\pgfqpoint{5.462879in}{2.803728in}}%
\pgfpathcurveto{\pgfqpoint{5.462879in}{2.792678in}}{\pgfqpoint{5.467269in}{2.782079in}}{\pgfqpoint{5.475083in}{2.774265in}}%
\pgfpathcurveto{\pgfqpoint{5.482896in}{2.766451in}}{\pgfqpoint{5.493495in}{2.762061in}}{\pgfqpoint{5.504545in}{2.762061in}}%
\pgfpathclose%
\pgfusepath{stroke,fill}%
\end{pgfscope}%
\begin{pgfscope}%
\pgfpathrectangle{\pgfqpoint{0.800000in}{0.528000in}}{\pgfqpoint{4.960000in}{3.696000in}}%
\pgfusepath{clip}%
\pgfsetbuttcap%
\pgfsetroundjoin%
\definecolor{currentfill}{rgb}{0.000000,0.000000,0.000000}%
\pgfsetfillcolor{currentfill}%
\pgfsetlinewidth{1.003750pt}%
\definecolor{currentstroke}{rgb}{0.000000,0.000000,0.000000}%
\pgfsetstrokecolor{currentstroke}%
\pgfsetdash{}{0pt}%
\pgfpathmoveto{\pgfqpoint{5.504545in}{2.719066in}}%
\pgfpathcurveto{\pgfqpoint{5.515596in}{2.719066in}}{\pgfqpoint{5.526195in}{2.723456in}}{\pgfqpoint{5.534008in}{2.731270in}}%
\pgfpathcurveto{\pgfqpoint{5.541822in}{2.739084in}}{\pgfqpoint{5.546212in}{2.749683in}}{\pgfqpoint{5.546212in}{2.760733in}}%
\pgfpathcurveto{\pgfqpoint{5.546212in}{2.771783in}}{\pgfqpoint{5.541822in}{2.782382in}}{\pgfqpoint{5.534008in}{2.790196in}}%
\pgfpathcurveto{\pgfqpoint{5.526195in}{2.798009in}}{\pgfqpoint{5.515596in}{2.802400in}}{\pgfqpoint{5.504545in}{2.802400in}}%
\pgfpathcurveto{\pgfqpoint{5.493495in}{2.802400in}}{\pgfqpoint{5.482896in}{2.798009in}}{\pgfqpoint{5.475083in}{2.790196in}}%
\pgfpathcurveto{\pgfqpoint{5.467269in}{2.782382in}}{\pgfqpoint{5.462879in}{2.771783in}}{\pgfqpoint{5.462879in}{2.760733in}}%
\pgfpathcurveto{\pgfqpoint{5.462879in}{2.749683in}}{\pgfqpoint{5.467269in}{2.739084in}}{\pgfqpoint{5.475083in}{2.731270in}}%
\pgfpathcurveto{\pgfqpoint{5.482896in}{2.723456in}}{\pgfqpoint{5.493495in}{2.719066in}}{\pgfqpoint{5.504545in}{2.719066in}}%
\pgfpathclose%
\pgfusepath{stroke,fill}%
\end{pgfscope}%
\begin{pgfscope}%
\pgfpathrectangle{\pgfqpoint{0.800000in}{0.528000in}}{\pgfqpoint{4.960000in}{3.696000in}}%
\pgfusepath{clip}%
\pgfsetbuttcap%
\pgfsetroundjoin%
\definecolor{currentfill}{rgb}{0.000000,0.000000,0.000000}%
\pgfsetfillcolor{currentfill}%
\pgfsetlinewidth{1.003750pt}%
\definecolor{currentstroke}{rgb}{0.000000,0.000000,0.000000}%
\pgfsetstrokecolor{currentstroke}%
\pgfsetdash{}{0pt}%
\pgfpathmoveto{\pgfqpoint{5.504545in}{3.106021in}}%
\pgfpathcurveto{\pgfqpoint{5.515596in}{3.106021in}}{\pgfqpoint{5.526195in}{3.110411in}}{\pgfqpoint{5.534008in}{3.118225in}}%
\pgfpathcurveto{\pgfqpoint{5.541822in}{3.126039in}}{\pgfqpoint{5.546212in}{3.136638in}}{\pgfqpoint{5.546212in}{3.147688in}}%
\pgfpathcurveto{\pgfqpoint{5.546212in}{3.158738in}}{\pgfqpoint{5.541822in}{3.169337in}}{\pgfqpoint{5.534008in}{3.177151in}}%
\pgfpathcurveto{\pgfqpoint{5.526195in}{3.184964in}}{\pgfqpoint{5.515596in}{3.189354in}}{\pgfqpoint{5.504545in}{3.189354in}}%
\pgfpathcurveto{\pgfqpoint{5.493495in}{3.189354in}}{\pgfqpoint{5.482896in}{3.184964in}}{\pgfqpoint{5.475083in}{3.177151in}}%
\pgfpathcurveto{\pgfqpoint{5.467269in}{3.169337in}}{\pgfqpoint{5.462879in}{3.158738in}}{\pgfqpoint{5.462879in}{3.147688in}}%
\pgfpathcurveto{\pgfqpoint{5.462879in}{3.136638in}}{\pgfqpoint{5.467269in}{3.126039in}}{\pgfqpoint{5.475083in}{3.118225in}}%
\pgfpathcurveto{\pgfqpoint{5.482896in}{3.110411in}}{\pgfqpoint{5.493495in}{3.106021in}}{\pgfqpoint{5.504545in}{3.106021in}}%
\pgfpathclose%
\pgfusepath{stroke,fill}%
\end{pgfscope}%
\begin{pgfscope}%
\pgfpathrectangle{\pgfqpoint{0.800000in}{0.528000in}}{\pgfqpoint{4.960000in}{3.696000in}}%
\pgfusepath{clip}%
\pgfsetbuttcap%
\pgfsetroundjoin%
\definecolor{currentfill}{rgb}{0.000000,0.000000,0.000000}%
\pgfsetfillcolor{currentfill}%
\pgfsetlinewidth{1.003750pt}%
\definecolor{currentstroke}{rgb}{0.000000,0.000000,0.000000}%
\pgfsetstrokecolor{currentstroke}%
\pgfsetdash{}{0pt}%
\pgfpathmoveto{\pgfqpoint{5.504545in}{2.740564in}}%
\pgfpathcurveto{\pgfqpoint{5.515596in}{2.740564in}}{\pgfqpoint{5.526195in}{2.744954in}}{\pgfqpoint{5.534008in}{2.752768in}}%
\pgfpathcurveto{\pgfqpoint{5.541822in}{2.760581in}}{\pgfqpoint{5.546212in}{2.771180in}}{\pgfqpoint{5.546212in}{2.782230in}}%
\pgfpathcurveto{\pgfqpoint{5.546212in}{2.793281in}}{\pgfqpoint{5.541822in}{2.803880in}}{\pgfqpoint{5.534008in}{2.811693in}}%
\pgfpathcurveto{\pgfqpoint{5.526195in}{2.819507in}}{\pgfqpoint{5.515596in}{2.823897in}}{\pgfqpoint{5.504545in}{2.823897in}}%
\pgfpathcurveto{\pgfqpoint{5.493495in}{2.823897in}}{\pgfqpoint{5.482896in}{2.819507in}}{\pgfqpoint{5.475083in}{2.811693in}}%
\pgfpathcurveto{\pgfqpoint{5.467269in}{2.803880in}}{\pgfqpoint{5.462879in}{2.793281in}}{\pgfqpoint{5.462879in}{2.782230in}}%
\pgfpathcurveto{\pgfqpoint{5.462879in}{2.771180in}}{\pgfqpoint{5.467269in}{2.760581in}}{\pgfqpoint{5.475083in}{2.752768in}}%
\pgfpathcurveto{\pgfqpoint{5.482896in}{2.744954in}}{\pgfqpoint{5.493495in}{2.740564in}}{\pgfqpoint{5.504545in}{2.740564in}}%
\pgfpathclose%
\pgfusepath{stroke,fill}%
\end{pgfscope}%
\begin{pgfscope}%
\pgfpathrectangle{\pgfqpoint{0.800000in}{0.528000in}}{\pgfqpoint{4.960000in}{3.696000in}}%
\pgfusepath{clip}%
\pgfsetbuttcap%
\pgfsetroundjoin%
\definecolor{currentfill}{rgb}{0.000000,0.000000,0.000000}%
\pgfsetfillcolor{currentfill}%
\pgfsetlinewidth{1.003750pt}%
\definecolor{currentstroke}{rgb}{0.000000,0.000000,0.000000}%
\pgfsetstrokecolor{currentstroke}%
\pgfsetdash{}{0pt}%
\pgfpathmoveto{\pgfqpoint{5.504545in}{3.535971in}}%
\pgfpathcurveto{\pgfqpoint{5.515596in}{3.535971in}}{\pgfqpoint{5.526195in}{3.540361in}}{\pgfqpoint{5.534008in}{3.548175in}}%
\pgfpathcurveto{\pgfqpoint{5.541822in}{3.555988in}}{\pgfqpoint{5.546212in}{3.566587in}}{\pgfqpoint{5.546212in}{3.577638in}}%
\pgfpathcurveto{\pgfqpoint{5.546212in}{3.588688in}}{\pgfqpoint{5.541822in}{3.599287in}}{\pgfqpoint{5.534008in}{3.607100in}}%
\pgfpathcurveto{\pgfqpoint{5.526195in}{3.614914in}}{\pgfqpoint{5.515596in}{3.619304in}}{\pgfqpoint{5.504545in}{3.619304in}}%
\pgfpathcurveto{\pgfqpoint{5.493495in}{3.619304in}}{\pgfqpoint{5.482896in}{3.614914in}}{\pgfqpoint{5.475083in}{3.607100in}}%
\pgfpathcurveto{\pgfqpoint{5.467269in}{3.599287in}}{\pgfqpoint{5.462879in}{3.588688in}}{\pgfqpoint{5.462879in}{3.577638in}}%
\pgfpathcurveto{\pgfqpoint{5.462879in}{3.566587in}}{\pgfqpoint{5.467269in}{3.555988in}}{\pgfqpoint{5.475083in}{3.548175in}}%
\pgfpathcurveto{\pgfqpoint{5.482896in}{3.540361in}}{\pgfqpoint{5.493495in}{3.535971in}}{\pgfqpoint{5.504545in}{3.535971in}}%
\pgfpathclose%
\pgfusepath{stroke,fill}%
\end{pgfscope}%
\begin{pgfscope}%
\pgfpathrectangle{\pgfqpoint{0.800000in}{0.528000in}}{\pgfqpoint{4.960000in}{3.696000in}}%
\pgfusepath{clip}%
\pgfsetbuttcap%
\pgfsetroundjoin%
\definecolor{currentfill}{rgb}{0.000000,0.000000,0.000000}%
\pgfsetfillcolor{currentfill}%
\pgfsetlinewidth{1.003750pt}%
\definecolor{currentstroke}{rgb}{0.000000,0.000000,0.000000}%
\pgfsetstrokecolor{currentstroke}%
\pgfsetdash{}{0pt}%
\pgfpathmoveto{\pgfqpoint{5.504545in}{2.934041in}}%
\pgfpathcurveto{\pgfqpoint{5.515596in}{2.934041in}}{\pgfqpoint{5.526195in}{2.938431in}}{\pgfqpoint{5.534008in}{2.946245in}}%
\pgfpathcurveto{\pgfqpoint{5.541822in}{2.954059in}}{\pgfqpoint{5.546212in}{2.964658in}}{\pgfqpoint{5.546212in}{2.975708in}}%
\pgfpathcurveto{\pgfqpoint{5.546212in}{2.986758in}}{\pgfqpoint{5.541822in}{2.997357in}}{\pgfqpoint{5.534008in}{3.005171in}}%
\pgfpathcurveto{\pgfqpoint{5.526195in}{3.012984in}}{\pgfqpoint{5.515596in}{3.017374in}}{\pgfqpoint{5.504545in}{3.017374in}}%
\pgfpathcurveto{\pgfqpoint{5.493495in}{3.017374in}}{\pgfqpoint{5.482896in}{3.012984in}}{\pgfqpoint{5.475083in}{3.005171in}}%
\pgfpathcurveto{\pgfqpoint{5.467269in}{2.997357in}}{\pgfqpoint{5.462879in}{2.986758in}}{\pgfqpoint{5.462879in}{2.975708in}}%
\pgfpathcurveto{\pgfqpoint{5.462879in}{2.964658in}}{\pgfqpoint{5.467269in}{2.954059in}}{\pgfqpoint{5.475083in}{2.946245in}}%
\pgfpathcurveto{\pgfqpoint{5.482896in}{2.938431in}}{\pgfqpoint{5.493495in}{2.934041in}}{\pgfqpoint{5.504545in}{2.934041in}}%
\pgfpathclose%
\pgfusepath{stroke,fill}%
\end{pgfscope}%
\begin{pgfscope}%
\pgfpathrectangle{\pgfqpoint{0.800000in}{0.528000in}}{\pgfqpoint{4.960000in}{3.696000in}}%
\pgfusepath{clip}%
\pgfsetbuttcap%
\pgfsetroundjoin%
\definecolor{currentfill}{rgb}{0.000000,0.000000,0.000000}%
\pgfsetfillcolor{currentfill}%
\pgfsetlinewidth{1.003750pt}%
\definecolor{currentstroke}{rgb}{0.000000,0.000000,0.000000}%
\pgfsetstrokecolor{currentstroke}%
\pgfsetdash{}{0pt}%
\pgfpathmoveto{\pgfqpoint{5.504545in}{2.676071in}}%
\pgfpathcurveto{\pgfqpoint{5.515596in}{2.676071in}}{\pgfqpoint{5.526195in}{2.680462in}}{\pgfqpoint{5.534008in}{2.688275in}}%
\pgfpathcurveto{\pgfqpoint{5.541822in}{2.696089in}}{\pgfqpoint{5.546212in}{2.706688in}}{\pgfqpoint{5.546212in}{2.717738in}}%
\pgfpathcurveto{\pgfqpoint{5.546212in}{2.728788in}}{\pgfqpoint{5.541822in}{2.739387in}}{\pgfqpoint{5.534008in}{2.747201in}}%
\pgfpathcurveto{\pgfqpoint{5.526195in}{2.755014in}}{\pgfqpoint{5.515596in}{2.759405in}}{\pgfqpoint{5.504545in}{2.759405in}}%
\pgfpathcurveto{\pgfqpoint{5.493495in}{2.759405in}}{\pgfqpoint{5.482896in}{2.755014in}}{\pgfqpoint{5.475083in}{2.747201in}}%
\pgfpathcurveto{\pgfqpoint{5.467269in}{2.739387in}}{\pgfqpoint{5.462879in}{2.728788in}}{\pgfqpoint{5.462879in}{2.717738in}}%
\pgfpathcurveto{\pgfqpoint{5.462879in}{2.706688in}}{\pgfqpoint{5.467269in}{2.696089in}}{\pgfqpoint{5.475083in}{2.688275in}}%
\pgfpathcurveto{\pgfqpoint{5.482896in}{2.680462in}}{\pgfqpoint{5.493495in}{2.676071in}}{\pgfqpoint{5.504545in}{2.676071in}}%
\pgfpathclose%
\pgfusepath{stroke,fill}%
\end{pgfscope}%
\begin{pgfscope}%
\pgfpathrectangle{\pgfqpoint{0.800000in}{0.528000in}}{\pgfqpoint{4.960000in}{3.696000in}}%
\pgfusepath{clip}%
\pgfsetbuttcap%
\pgfsetroundjoin%
\definecolor{currentfill}{rgb}{0.000000,0.000000,0.000000}%
\pgfsetfillcolor{currentfill}%
\pgfsetlinewidth{1.003750pt}%
\definecolor{currentstroke}{rgb}{0.000000,0.000000,0.000000}%
\pgfsetstrokecolor{currentstroke}%
\pgfsetdash{}{0pt}%
\pgfpathmoveto{\pgfqpoint{5.504545in}{2.740564in}}%
\pgfpathcurveto{\pgfqpoint{5.515596in}{2.740564in}}{\pgfqpoint{5.526195in}{2.744954in}}{\pgfqpoint{5.534008in}{2.752768in}}%
\pgfpathcurveto{\pgfqpoint{5.541822in}{2.760581in}}{\pgfqpoint{5.546212in}{2.771180in}}{\pgfqpoint{5.546212in}{2.782230in}}%
\pgfpathcurveto{\pgfqpoint{5.546212in}{2.793281in}}{\pgfqpoint{5.541822in}{2.803880in}}{\pgfqpoint{5.534008in}{2.811693in}}%
\pgfpathcurveto{\pgfqpoint{5.526195in}{2.819507in}}{\pgfqpoint{5.515596in}{2.823897in}}{\pgfqpoint{5.504545in}{2.823897in}}%
\pgfpathcurveto{\pgfqpoint{5.493495in}{2.823897in}}{\pgfqpoint{5.482896in}{2.819507in}}{\pgfqpoint{5.475083in}{2.811693in}}%
\pgfpathcurveto{\pgfqpoint{5.467269in}{2.803880in}}{\pgfqpoint{5.462879in}{2.793281in}}{\pgfqpoint{5.462879in}{2.782230in}}%
\pgfpathcurveto{\pgfqpoint{5.462879in}{2.771180in}}{\pgfqpoint{5.467269in}{2.760581in}}{\pgfqpoint{5.475083in}{2.752768in}}%
\pgfpathcurveto{\pgfqpoint{5.482896in}{2.744954in}}{\pgfqpoint{5.493495in}{2.740564in}}{\pgfqpoint{5.504545in}{2.740564in}}%
\pgfpathclose%
\pgfusepath{stroke,fill}%
\end{pgfscope}%
\begin{pgfscope}%
\pgfpathrectangle{\pgfqpoint{0.800000in}{0.528000in}}{\pgfqpoint{4.960000in}{3.696000in}}%
\pgfusepath{clip}%
\pgfsetbuttcap%
\pgfsetroundjoin%
\definecolor{currentfill}{rgb}{0.000000,0.000000,0.000000}%
\pgfsetfillcolor{currentfill}%
\pgfsetlinewidth{1.003750pt}%
\definecolor{currentstroke}{rgb}{0.000000,0.000000,0.000000}%
\pgfsetstrokecolor{currentstroke}%
\pgfsetdash{}{0pt}%
\pgfpathmoveto{\pgfqpoint{5.504545in}{2.891046in}}%
\pgfpathcurveto{\pgfqpoint{5.515596in}{2.891046in}}{\pgfqpoint{5.526195in}{2.895436in}}{\pgfqpoint{5.534008in}{2.903250in}}%
\pgfpathcurveto{\pgfqpoint{5.541822in}{2.911064in}}{\pgfqpoint{5.546212in}{2.921663in}}{\pgfqpoint{5.546212in}{2.932713in}}%
\pgfpathcurveto{\pgfqpoint{5.546212in}{2.943763in}}{\pgfqpoint{5.541822in}{2.954362in}}{\pgfqpoint{5.534008in}{2.962176in}}%
\pgfpathcurveto{\pgfqpoint{5.526195in}{2.969989in}}{\pgfqpoint{5.515596in}{2.974379in}}{\pgfqpoint{5.504545in}{2.974379in}}%
\pgfpathcurveto{\pgfqpoint{5.493495in}{2.974379in}}{\pgfqpoint{5.482896in}{2.969989in}}{\pgfqpoint{5.475083in}{2.962176in}}%
\pgfpathcurveto{\pgfqpoint{5.467269in}{2.954362in}}{\pgfqpoint{5.462879in}{2.943763in}}{\pgfqpoint{5.462879in}{2.932713in}}%
\pgfpathcurveto{\pgfqpoint{5.462879in}{2.921663in}}{\pgfqpoint{5.467269in}{2.911064in}}{\pgfqpoint{5.475083in}{2.903250in}}%
\pgfpathcurveto{\pgfqpoint{5.482896in}{2.895436in}}{\pgfqpoint{5.493495in}{2.891046in}}{\pgfqpoint{5.504545in}{2.891046in}}%
\pgfpathclose%
\pgfusepath{stroke,fill}%
\end{pgfscope}%
\begin{pgfscope}%
\pgfpathrectangle{\pgfqpoint{0.800000in}{0.528000in}}{\pgfqpoint{4.960000in}{3.696000in}}%
\pgfusepath{clip}%
\pgfsetbuttcap%
\pgfsetroundjoin%
\definecolor{currentfill}{rgb}{0.000000,0.000000,0.000000}%
\pgfsetfillcolor{currentfill}%
\pgfsetlinewidth{1.003750pt}%
\definecolor{currentstroke}{rgb}{0.000000,0.000000,0.000000}%
\pgfsetstrokecolor{currentstroke}%
\pgfsetdash{}{0pt}%
\pgfpathmoveto{\pgfqpoint{5.504545in}{3.299499in}}%
\pgfpathcurveto{\pgfqpoint{5.515596in}{3.299499in}}{\pgfqpoint{5.526195in}{3.303889in}}{\pgfqpoint{5.534008in}{3.311702in}}%
\pgfpathcurveto{\pgfqpoint{5.541822in}{3.319516in}}{\pgfqpoint{5.546212in}{3.330115in}}{\pgfqpoint{5.546212in}{3.341165in}}%
\pgfpathcurveto{\pgfqpoint{5.546212in}{3.352215in}}{\pgfqpoint{5.541822in}{3.362814in}}{\pgfqpoint{5.534008in}{3.370628in}}%
\pgfpathcurveto{\pgfqpoint{5.526195in}{3.378442in}}{\pgfqpoint{5.515596in}{3.382832in}}{\pgfqpoint{5.504545in}{3.382832in}}%
\pgfpathcurveto{\pgfqpoint{5.493495in}{3.382832in}}{\pgfqpoint{5.482896in}{3.378442in}}{\pgfqpoint{5.475083in}{3.370628in}}%
\pgfpathcurveto{\pgfqpoint{5.467269in}{3.362814in}}{\pgfqpoint{5.462879in}{3.352215in}}{\pgfqpoint{5.462879in}{3.341165in}}%
\pgfpathcurveto{\pgfqpoint{5.462879in}{3.330115in}}{\pgfqpoint{5.467269in}{3.319516in}}{\pgfqpoint{5.475083in}{3.311702in}}%
\pgfpathcurveto{\pgfqpoint{5.482896in}{3.303889in}}{\pgfqpoint{5.493495in}{3.299499in}}{\pgfqpoint{5.504545in}{3.299499in}}%
\pgfpathclose%
\pgfusepath{stroke,fill}%
\end{pgfscope}%
\begin{pgfscope}%
\pgfpathrectangle{\pgfqpoint{0.800000in}{0.528000in}}{\pgfqpoint{4.960000in}{3.696000in}}%
\pgfusepath{clip}%
\pgfsetbuttcap%
\pgfsetroundjoin%
\definecolor{currentfill}{rgb}{0.000000,0.000000,0.000000}%
\pgfsetfillcolor{currentfill}%
\pgfsetlinewidth{1.003750pt}%
\definecolor{currentstroke}{rgb}{0.000000,0.000000,0.000000}%
\pgfsetstrokecolor{currentstroke}%
\pgfsetdash{}{0pt}%
\pgfpathmoveto{\pgfqpoint{5.504545in}{2.826554in}}%
\pgfpathcurveto{\pgfqpoint{5.515596in}{2.826554in}}{\pgfqpoint{5.526195in}{2.830944in}}{\pgfqpoint{5.534008in}{2.838758in}}%
\pgfpathcurveto{\pgfqpoint{5.541822in}{2.846571in}}{\pgfqpoint{5.546212in}{2.857170in}}{\pgfqpoint{5.546212in}{2.868220in}}%
\pgfpathcurveto{\pgfqpoint{5.546212in}{2.879270in}}{\pgfqpoint{5.541822in}{2.889870in}}{\pgfqpoint{5.534008in}{2.897683in}}%
\pgfpathcurveto{\pgfqpoint{5.526195in}{2.905497in}}{\pgfqpoint{5.515596in}{2.909887in}}{\pgfqpoint{5.504545in}{2.909887in}}%
\pgfpathcurveto{\pgfqpoint{5.493495in}{2.909887in}}{\pgfqpoint{5.482896in}{2.905497in}}{\pgfqpoint{5.475083in}{2.897683in}}%
\pgfpathcurveto{\pgfqpoint{5.467269in}{2.889870in}}{\pgfqpoint{5.462879in}{2.879270in}}{\pgfqpoint{5.462879in}{2.868220in}}%
\pgfpathcurveto{\pgfqpoint{5.462879in}{2.857170in}}{\pgfqpoint{5.467269in}{2.846571in}}{\pgfqpoint{5.475083in}{2.838758in}}%
\pgfpathcurveto{\pgfqpoint{5.482896in}{2.830944in}}{\pgfqpoint{5.493495in}{2.826554in}}{\pgfqpoint{5.504545in}{2.826554in}}%
\pgfpathclose%
\pgfusepath{stroke,fill}%
\end{pgfscope}%
\begin{pgfscope}%
\pgfpathrectangle{\pgfqpoint{0.800000in}{0.528000in}}{\pgfqpoint{4.960000in}{3.696000in}}%
\pgfusepath{clip}%
\pgfsetbuttcap%
\pgfsetroundjoin%
\definecolor{currentfill}{rgb}{0.000000,0.000000,0.000000}%
\pgfsetfillcolor{currentfill}%
\pgfsetlinewidth{1.003750pt}%
\definecolor{currentstroke}{rgb}{0.000000,0.000000,0.000000}%
\pgfsetstrokecolor{currentstroke}%
\pgfsetdash{}{0pt}%
\pgfpathmoveto{\pgfqpoint{5.504545in}{2.740564in}}%
\pgfpathcurveto{\pgfqpoint{5.515596in}{2.740564in}}{\pgfqpoint{5.526195in}{2.744954in}}{\pgfqpoint{5.534008in}{2.752768in}}%
\pgfpathcurveto{\pgfqpoint{5.541822in}{2.760581in}}{\pgfqpoint{5.546212in}{2.771180in}}{\pgfqpoint{5.546212in}{2.782230in}}%
\pgfpathcurveto{\pgfqpoint{5.546212in}{2.793281in}}{\pgfqpoint{5.541822in}{2.803880in}}{\pgfqpoint{5.534008in}{2.811693in}}%
\pgfpathcurveto{\pgfqpoint{5.526195in}{2.819507in}}{\pgfqpoint{5.515596in}{2.823897in}}{\pgfqpoint{5.504545in}{2.823897in}}%
\pgfpathcurveto{\pgfqpoint{5.493495in}{2.823897in}}{\pgfqpoint{5.482896in}{2.819507in}}{\pgfqpoint{5.475083in}{2.811693in}}%
\pgfpathcurveto{\pgfqpoint{5.467269in}{2.803880in}}{\pgfqpoint{5.462879in}{2.793281in}}{\pgfqpoint{5.462879in}{2.782230in}}%
\pgfpathcurveto{\pgfqpoint{5.462879in}{2.771180in}}{\pgfqpoint{5.467269in}{2.760581in}}{\pgfqpoint{5.475083in}{2.752768in}}%
\pgfpathcurveto{\pgfqpoint{5.482896in}{2.744954in}}{\pgfqpoint{5.493495in}{2.740564in}}{\pgfqpoint{5.504545in}{2.740564in}}%
\pgfpathclose%
\pgfusepath{stroke,fill}%
\end{pgfscope}%
\begin{pgfscope}%
\pgfpathrectangle{\pgfqpoint{0.800000in}{0.528000in}}{\pgfqpoint{4.960000in}{3.696000in}}%
\pgfusepath{clip}%
\pgfsetbuttcap%
\pgfsetroundjoin%
\definecolor{currentfill}{rgb}{0.000000,0.000000,0.000000}%
\pgfsetfillcolor{currentfill}%
\pgfsetlinewidth{1.003750pt}%
\definecolor{currentstroke}{rgb}{0.000000,0.000000,0.000000}%
\pgfsetstrokecolor{currentstroke}%
\pgfsetdash{}{0pt}%
\pgfpathmoveto{\pgfqpoint{5.504545in}{2.740564in}}%
\pgfpathcurveto{\pgfqpoint{5.515596in}{2.740564in}}{\pgfqpoint{5.526195in}{2.744954in}}{\pgfqpoint{5.534008in}{2.752768in}}%
\pgfpathcurveto{\pgfqpoint{5.541822in}{2.760581in}}{\pgfqpoint{5.546212in}{2.771180in}}{\pgfqpoint{5.546212in}{2.782230in}}%
\pgfpathcurveto{\pgfqpoint{5.546212in}{2.793281in}}{\pgfqpoint{5.541822in}{2.803880in}}{\pgfqpoint{5.534008in}{2.811693in}}%
\pgfpathcurveto{\pgfqpoint{5.526195in}{2.819507in}}{\pgfqpoint{5.515596in}{2.823897in}}{\pgfqpoint{5.504545in}{2.823897in}}%
\pgfpathcurveto{\pgfqpoint{5.493495in}{2.823897in}}{\pgfqpoint{5.482896in}{2.819507in}}{\pgfqpoint{5.475083in}{2.811693in}}%
\pgfpathcurveto{\pgfqpoint{5.467269in}{2.803880in}}{\pgfqpoint{5.462879in}{2.793281in}}{\pgfqpoint{5.462879in}{2.782230in}}%
\pgfpathcurveto{\pgfqpoint{5.462879in}{2.771180in}}{\pgfqpoint{5.467269in}{2.760581in}}{\pgfqpoint{5.475083in}{2.752768in}}%
\pgfpathcurveto{\pgfqpoint{5.482896in}{2.744954in}}{\pgfqpoint{5.493495in}{2.740564in}}{\pgfqpoint{5.504545in}{2.740564in}}%
\pgfpathclose%
\pgfusepath{stroke,fill}%
\end{pgfscope}%
\begin{pgfscope}%
\pgfpathrectangle{\pgfqpoint{0.800000in}{0.528000in}}{\pgfqpoint{4.960000in}{3.696000in}}%
\pgfusepath{clip}%
\pgfsetbuttcap%
\pgfsetroundjoin%
\definecolor{currentfill}{rgb}{0.000000,0.000000,0.000000}%
\pgfsetfillcolor{currentfill}%
\pgfsetlinewidth{1.003750pt}%
\definecolor{currentstroke}{rgb}{0.000000,0.000000,0.000000}%
\pgfsetstrokecolor{currentstroke}%
\pgfsetdash{}{0pt}%
\pgfpathmoveto{\pgfqpoint{5.504545in}{3.084524in}}%
\pgfpathcurveto{\pgfqpoint{5.515596in}{3.084524in}}{\pgfqpoint{5.526195in}{3.088914in}}{\pgfqpoint{5.534008in}{3.096727in}}%
\pgfpathcurveto{\pgfqpoint{5.541822in}{3.104541in}}{\pgfqpoint{5.546212in}{3.115140in}}{\pgfqpoint{5.546212in}{3.126190in}}%
\pgfpathcurveto{\pgfqpoint{5.546212in}{3.137240in}}{\pgfqpoint{5.541822in}{3.147839in}}{\pgfqpoint{5.534008in}{3.155653in}}%
\pgfpathcurveto{\pgfqpoint{5.526195in}{3.163467in}}{\pgfqpoint{5.515596in}{3.167857in}}{\pgfqpoint{5.504545in}{3.167857in}}%
\pgfpathcurveto{\pgfqpoint{5.493495in}{3.167857in}}{\pgfqpoint{5.482896in}{3.163467in}}{\pgfqpoint{5.475083in}{3.155653in}}%
\pgfpathcurveto{\pgfqpoint{5.467269in}{3.147839in}}{\pgfqpoint{5.462879in}{3.137240in}}{\pgfqpoint{5.462879in}{3.126190in}}%
\pgfpathcurveto{\pgfqpoint{5.462879in}{3.115140in}}{\pgfqpoint{5.467269in}{3.104541in}}{\pgfqpoint{5.475083in}{3.096727in}}%
\pgfpathcurveto{\pgfqpoint{5.482896in}{3.088914in}}{\pgfqpoint{5.493495in}{3.084524in}}{\pgfqpoint{5.504545in}{3.084524in}}%
\pgfpathclose%
\pgfusepath{stroke,fill}%
\end{pgfscope}%
\begin{pgfscope}%
\pgfpathrectangle{\pgfqpoint{0.800000in}{0.528000in}}{\pgfqpoint{4.960000in}{3.696000in}}%
\pgfusepath{clip}%
\pgfsetbuttcap%
\pgfsetroundjoin%
\definecolor{currentfill}{rgb}{0.000000,0.000000,0.000000}%
\pgfsetfillcolor{currentfill}%
\pgfsetlinewidth{1.003750pt}%
\definecolor{currentstroke}{rgb}{0.000000,0.000000,0.000000}%
\pgfsetstrokecolor{currentstroke}%
\pgfsetdash{}{0pt}%
\pgfpathmoveto{\pgfqpoint{5.504545in}{2.740564in}}%
\pgfpathcurveto{\pgfqpoint{5.515596in}{2.740564in}}{\pgfqpoint{5.526195in}{2.744954in}}{\pgfqpoint{5.534008in}{2.752768in}}%
\pgfpathcurveto{\pgfqpoint{5.541822in}{2.760581in}}{\pgfqpoint{5.546212in}{2.771180in}}{\pgfqpoint{5.546212in}{2.782230in}}%
\pgfpathcurveto{\pgfqpoint{5.546212in}{2.793281in}}{\pgfqpoint{5.541822in}{2.803880in}}{\pgfqpoint{5.534008in}{2.811693in}}%
\pgfpathcurveto{\pgfqpoint{5.526195in}{2.819507in}}{\pgfqpoint{5.515596in}{2.823897in}}{\pgfqpoint{5.504545in}{2.823897in}}%
\pgfpathcurveto{\pgfqpoint{5.493495in}{2.823897in}}{\pgfqpoint{5.482896in}{2.819507in}}{\pgfqpoint{5.475083in}{2.811693in}}%
\pgfpathcurveto{\pgfqpoint{5.467269in}{2.803880in}}{\pgfqpoint{5.462879in}{2.793281in}}{\pgfqpoint{5.462879in}{2.782230in}}%
\pgfpathcurveto{\pgfqpoint{5.462879in}{2.771180in}}{\pgfqpoint{5.467269in}{2.760581in}}{\pgfqpoint{5.475083in}{2.752768in}}%
\pgfpathcurveto{\pgfqpoint{5.482896in}{2.744954in}}{\pgfqpoint{5.493495in}{2.740564in}}{\pgfqpoint{5.504545in}{2.740564in}}%
\pgfpathclose%
\pgfusepath{stroke,fill}%
\end{pgfscope}%
\begin{pgfscope}%
\pgfpathrectangle{\pgfqpoint{0.800000in}{0.528000in}}{\pgfqpoint{4.960000in}{3.696000in}}%
\pgfusepath{clip}%
\pgfsetbuttcap%
\pgfsetroundjoin%
\definecolor{currentfill}{rgb}{0.000000,0.000000,0.000000}%
\pgfsetfillcolor{currentfill}%
\pgfsetlinewidth{1.003750pt}%
\definecolor{currentstroke}{rgb}{0.000000,0.000000,0.000000}%
\pgfsetstrokecolor{currentstroke}%
\pgfsetdash{}{0pt}%
\pgfpathmoveto{\pgfqpoint{5.504545in}{2.740564in}}%
\pgfpathcurveto{\pgfqpoint{5.515596in}{2.740564in}}{\pgfqpoint{5.526195in}{2.744954in}}{\pgfqpoint{5.534008in}{2.752768in}}%
\pgfpathcurveto{\pgfqpoint{5.541822in}{2.760581in}}{\pgfqpoint{5.546212in}{2.771180in}}{\pgfqpoint{5.546212in}{2.782230in}}%
\pgfpathcurveto{\pgfqpoint{5.546212in}{2.793281in}}{\pgfqpoint{5.541822in}{2.803880in}}{\pgfqpoint{5.534008in}{2.811693in}}%
\pgfpathcurveto{\pgfqpoint{5.526195in}{2.819507in}}{\pgfqpoint{5.515596in}{2.823897in}}{\pgfqpoint{5.504545in}{2.823897in}}%
\pgfpathcurveto{\pgfqpoint{5.493495in}{2.823897in}}{\pgfqpoint{5.482896in}{2.819507in}}{\pgfqpoint{5.475083in}{2.811693in}}%
\pgfpathcurveto{\pgfqpoint{5.467269in}{2.803880in}}{\pgfqpoint{5.462879in}{2.793281in}}{\pgfqpoint{5.462879in}{2.782230in}}%
\pgfpathcurveto{\pgfqpoint{5.462879in}{2.771180in}}{\pgfqpoint{5.467269in}{2.760581in}}{\pgfqpoint{5.475083in}{2.752768in}}%
\pgfpathcurveto{\pgfqpoint{5.482896in}{2.744954in}}{\pgfqpoint{5.493495in}{2.740564in}}{\pgfqpoint{5.504545in}{2.740564in}}%
\pgfpathclose%
\pgfusepath{stroke,fill}%
\end{pgfscope}%
\begin{pgfscope}%
\pgfpathrectangle{\pgfqpoint{0.800000in}{0.528000in}}{\pgfqpoint{4.960000in}{3.696000in}}%
\pgfusepath{clip}%
\pgfsetbuttcap%
\pgfsetroundjoin%
\definecolor{currentfill}{rgb}{0.000000,0.000000,0.000000}%
\pgfsetfillcolor{currentfill}%
\pgfsetlinewidth{1.003750pt}%
\definecolor{currentstroke}{rgb}{0.000000,0.000000,0.000000}%
\pgfsetstrokecolor{currentstroke}%
\pgfsetdash{}{0pt}%
\pgfpathmoveto{\pgfqpoint{5.504545in}{2.783559in}}%
\pgfpathcurveto{\pgfqpoint{5.515596in}{2.783559in}}{\pgfqpoint{5.526195in}{2.787949in}}{\pgfqpoint{5.534008in}{2.795763in}}%
\pgfpathcurveto{\pgfqpoint{5.541822in}{2.803576in}}{\pgfqpoint{5.546212in}{2.814175in}}{\pgfqpoint{5.546212in}{2.825225in}}%
\pgfpathcurveto{\pgfqpoint{5.546212in}{2.836275in}}{\pgfqpoint{5.541822in}{2.846875in}}{\pgfqpoint{5.534008in}{2.854688in}}%
\pgfpathcurveto{\pgfqpoint{5.526195in}{2.862502in}}{\pgfqpoint{5.515596in}{2.866892in}}{\pgfqpoint{5.504545in}{2.866892in}}%
\pgfpathcurveto{\pgfqpoint{5.493495in}{2.866892in}}{\pgfqpoint{5.482896in}{2.862502in}}{\pgfqpoint{5.475083in}{2.854688in}}%
\pgfpathcurveto{\pgfqpoint{5.467269in}{2.846875in}}{\pgfqpoint{5.462879in}{2.836275in}}{\pgfqpoint{5.462879in}{2.825225in}}%
\pgfpathcurveto{\pgfqpoint{5.462879in}{2.814175in}}{\pgfqpoint{5.467269in}{2.803576in}}{\pgfqpoint{5.475083in}{2.795763in}}%
\pgfpathcurveto{\pgfqpoint{5.482896in}{2.787949in}}{\pgfqpoint{5.493495in}{2.783559in}}{\pgfqpoint{5.504545in}{2.783559in}}%
\pgfpathclose%
\pgfusepath{stroke,fill}%
\end{pgfscope}%
\begin{pgfscope}%
\pgfpathrectangle{\pgfqpoint{0.800000in}{0.528000in}}{\pgfqpoint{4.960000in}{3.696000in}}%
\pgfusepath{clip}%
\pgfsetbuttcap%
\pgfsetroundjoin%
\definecolor{currentfill}{rgb}{0.000000,0.000000,0.000000}%
\pgfsetfillcolor{currentfill}%
\pgfsetlinewidth{1.003750pt}%
\definecolor{currentstroke}{rgb}{0.000000,0.000000,0.000000}%
\pgfsetstrokecolor{currentstroke}%
\pgfsetdash{}{0pt}%
\pgfpathmoveto{\pgfqpoint{5.504545in}{2.891046in}}%
\pgfpathcurveto{\pgfqpoint{5.515596in}{2.891046in}}{\pgfqpoint{5.526195in}{2.895436in}}{\pgfqpoint{5.534008in}{2.903250in}}%
\pgfpathcurveto{\pgfqpoint{5.541822in}{2.911064in}}{\pgfqpoint{5.546212in}{2.921663in}}{\pgfqpoint{5.546212in}{2.932713in}}%
\pgfpathcurveto{\pgfqpoint{5.546212in}{2.943763in}}{\pgfqpoint{5.541822in}{2.954362in}}{\pgfqpoint{5.534008in}{2.962176in}}%
\pgfpathcurveto{\pgfqpoint{5.526195in}{2.969989in}}{\pgfqpoint{5.515596in}{2.974379in}}{\pgfqpoint{5.504545in}{2.974379in}}%
\pgfpathcurveto{\pgfqpoint{5.493495in}{2.974379in}}{\pgfqpoint{5.482896in}{2.969989in}}{\pgfqpoint{5.475083in}{2.962176in}}%
\pgfpathcurveto{\pgfqpoint{5.467269in}{2.954362in}}{\pgfqpoint{5.462879in}{2.943763in}}{\pgfqpoint{5.462879in}{2.932713in}}%
\pgfpathcurveto{\pgfqpoint{5.462879in}{2.921663in}}{\pgfqpoint{5.467269in}{2.911064in}}{\pgfqpoint{5.475083in}{2.903250in}}%
\pgfpathcurveto{\pgfqpoint{5.482896in}{2.895436in}}{\pgfqpoint{5.493495in}{2.891046in}}{\pgfqpoint{5.504545in}{2.891046in}}%
\pgfpathclose%
\pgfusepath{stroke,fill}%
\end{pgfscope}%
\begin{pgfscope}%
\pgfpathrectangle{\pgfqpoint{0.800000in}{0.528000in}}{\pgfqpoint{4.960000in}{3.696000in}}%
\pgfusepath{clip}%
\pgfsetbuttcap%
\pgfsetroundjoin%
\definecolor{currentfill}{rgb}{0.000000,0.000000,0.000000}%
\pgfsetfillcolor{currentfill}%
\pgfsetlinewidth{1.003750pt}%
\definecolor{currentstroke}{rgb}{0.000000,0.000000,0.000000}%
\pgfsetstrokecolor{currentstroke}%
\pgfsetdash{}{0pt}%
\pgfpathmoveto{\pgfqpoint{5.504545in}{2.740564in}}%
\pgfpathcurveto{\pgfqpoint{5.515596in}{2.740564in}}{\pgfqpoint{5.526195in}{2.744954in}}{\pgfqpoint{5.534008in}{2.752768in}}%
\pgfpathcurveto{\pgfqpoint{5.541822in}{2.760581in}}{\pgfqpoint{5.546212in}{2.771180in}}{\pgfqpoint{5.546212in}{2.782230in}}%
\pgfpathcurveto{\pgfqpoint{5.546212in}{2.793281in}}{\pgfqpoint{5.541822in}{2.803880in}}{\pgfqpoint{5.534008in}{2.811693in}}%
\pgfpathcurveto{\pgfqpoint{5.526195in}{2.819507in}}{\pgfqpoint{5.515596in}{2.823897in}}{\pgfqpoint{5.504545in}{2.823897in}}%
\pgfpathcurveto{\pgfqpoint{5.493495in}{2.823897in}}{\pgfqpoint{5.482896in}{2.819507in}}{\pgfqpoint{5.475083in}{2.811693in}}%
\pgfpathcurveto{\pgfqpoint{5.467269in}{2.803880in}}{\pgfqpoint{5.462879in}{2.793281in}}{\pgfqpoint{5.462879in}{2.782230in}}%
\pgfpathcurveto{\pgfqpoint{5.462879in}{2.771180in}}{\pgfqpoint{5.467269in}{2.760581in}}{\pgfqpoint{5.475083in}{2.752768in}}%
\pgfpathcurveto{\pgfqpoint{5.482896in}{2.744954in}}{\pgfqpoint{5.493495in}{2.740564in}}{\pgfqpoint{5.504545in}{2.740564in}}%
\pgfpathclose%
\pgfusepath{stroke,fill}%
\end{pgfscope}%
\begin{pgfscope}%
\pgfpathrectangle{\pgfqpoint{0.800000in}{0.528000in}}{\pgfqpoint{4.960000in}{3.696000in}}%
\pgfusepath{clip}%
\pgfsetbuttcap%
\pgfsetroundjoin%
\definecolor{currentfill}{rgb}{0.000000,0.000000,0.000000}%
\pgfsetfillcolor{currentfill}%
\pgfsetlinewidth{1.003750pt}%
\definecolor{currentstroke}{rgb}{0.000000,0.000000,0.000000}%
\pgfsetstrokecolor{currentstroke}%
\pgfsetdash{}{0pt}%
\pgfpathmoveto{\pgfqpoint{5.504545in}{2.826554in}}%
\pgfpathcurveto{\pgfqpoint{5.515596in}{2.826554in}}{\pgfqpoint{5.526195in}{2.830944in}}{\pgfqpoint{5.534008in}{2.838758in}}%
\pgfpathcurveto{\pgfqpoint{5.541822in}{2.846571in}}{\pgfqpoint{5.546212in}{2.857170in}}{\pgfqpoint{5.546212in}{2.868220in}}%
\pgfpathcurveto{\pgfqpoint{5.546212in}{2.879270in}}{\pgfqpoint{5.541822in}{2.889870in}}{\pgfqpoint{5.534008in}{2.897683in}}%
\pgfpathcurveto{\pgfqpoint{5.526195in}{2.905497in}}{\pgfqpoint{5.515596in}{2.909887in}}{\pgfqpoint{5.504545in}{2.909887in}}%
\pgfpathcurveto{\pgfqpoint{5.493495in}{2.909887in}}{\pgfqpoint{5.482896in}{2.905497in}}{\pgfqpoint{5.475083in}{2.897683in}}%
\pgfpathcurveto{\pgfqpoint{5.467269in}{2.889870in}}{\pgfqpoint{5.462879in}{2.879270in}}{\pgfqpoint{5.462879in}{2.868220in}}%
\pgfpathcurveto{\pgfqpoint{5.462879in}{2.857170in}}{\pgfqpoint{5.467269in}{2.846571in}}{\pgfqpoint{5.475083in}{2.838758in}}%
\pgfpathcurveto{\pgfqpoint{5.482896in}{2.830944in}}{\pgfqpoint{5.493495in}{2.826554in}}{\pgfqpoint{5.504545in}{2.826554in}}%
\pgfpathclose%
\pgfusepath{stroke,fill}%
\end{pgfscope}%
\begin{pgfscope}%
\pgfpathrectangle{\pgfqpoint{0.800000in}{0.528000in}}{\pgfqpoint{4.960000in}{3.696000in}}%
\pgfusepath{clip}%
\pgfsetbuttcap%
\pgfsetroundjoin%
\definecolor{currentfill}{rgb}{0.000000,0.000000,0.000000}%
\pgfsetfillcolor{currentfill}%
\pgfsetlinewidth{1.003750pt}%
\definecolor{currentstroke}{rgb}{0.000000,0.000000,0.000000}%
\pgfsetstrokecolor{currentstroke}%
\pgfsetdash{}{0pt}%
\pgfpathmoveto{\pgfqpoint{5.504545in}{3.041529in}}%
\pgfpathcurveto{\pgfqpoint{5.515596in}{3.041529in}}{\pgfqpoint{5.526195in}{3.045919in}}{\pgfqpoint{5.534008in}{3.053732in}}%
\pgfpathcurveto{\pgfqpoint{5.541822in}{3.061546in}}{\pgfqpoint{5.546212in}{3.072145in}}{\pgfqpoint{5.546212in}{3.083195in}}%
\pgfpathcurveto{\pgfqpoint{5.546212in}{3.094245in}}{\pgfqpoint{5.541822in}{3.104844in}}{\pgfqpoint{5.534008in}{3.112658in}}%
\pgfpathcurveto{\pgfqpoint{5.526195in}{3.120472in}}{\pgfqpoint{5.515596in}{3.124862in}}{\pgfqpoint{5.504545in}{3.124862in}}%
\pgfpathcurveto{\pgfqpoint{5.493495in}{3.124862in}}{\pgfqpoint{5.482896in}{3.120472in}}{\pgfqpoint{5.475083in}{3.112658in}}%
\pgfpathcurveto{\pgfqpoint{5.467269in}{3.104844in}}{\pgfqpoint{5.462879in}{3.094245in}}{\pgfqpoint{5.462879in}{3.083195in}}%
\pgfpathcurveto{\pgfqpoint{5.462879in}{3.072145in}}{\pgfqpoint{5.467269in}{3.061546in}}{\pgfqpoint{5.475083in}{3.053732in}}%
\pgfpathcurveto{\pgfqpoint{5.482896in}{3.045919in}}{\pgfqpoint{5.493495in}{3.041529in}}{\pgfqpoint{5.504545in}{3.041529in}}%
\pgfpathclose%
\pgfusepath{stroke,fill}%
\end{pgfscope}%
\begin{pgfscope}%
\pgfpathrectangle{\pgfqpoint{0.800000in}{0.528000in}}{\pgfqpoint{4.960000in}{3.696000in}}%
\pgfusepath{clip}%
\pgfsetbuttcap%
\pgfsetroundjoin%
\definecolor{currentfill}{rgb}{0.000000,0.000000,0.000000}%
\pgfsetfillcolor{currentfill}%
\pgfsetlinewidth{1.003750pt}%
\definecolor{currentstroke}{rgb}{0.000000,0.000000,0.000000}%
\pgfsetstrokecolor{currentstroke}%
\pgfsetdash{}{0pt}%
\pgfpathmoveto{\pgfqpoint{5.504545in}{3.299499in}}%
\pgfpathcurveto{\pgfqpoint{5.515596in}{3.299499in}}{\pgfqpoint{5.526195in}{3.303889in}}{\pgfqpoint{5.534008in}{3.311702in}}%
\pgfpathcurveto{\pgfqpoint{5.541822in}{3.319516in}}{\pgfqpoint{5.546212in}{3.330115in}}{\pgfqpoint{5.546212in}{3.341165in}}%
\pgfpathcurveto{\pgfqpoint{5.546212in}{3.352215in}}{\pgfqpoint{5.541822in}{3.362814in}}{\pgfqpoint{5.534008in}{3.370628in}}%
\pgfpathcurveto{\pgfqpoint{5.526195in}{3.378442in}}{\pgfqpoint{5.515596in}{3.382832in}}{\pgfqpoint{5.504545in}{3.382832in}}%
\pgfpathcurveto{\pgfqpoint{5.493495in}{3.382832in}}{\pgfqpoint{5.482896in}{3.378442in}}{\pgfqpoint{5.475083in}{3.370628in}}%
\pgfpathcurveto{\pgfqpoint{5.467269in}{3.362814in}}{\pgfqpoint{5.462879in}{3.352215in}}{\pgfqpoint{5.462879in}{3.341165in}}%
\pgfpathcurveto{\pgfqpoint{5.462879in}{3.330115in}}{\pgfqpoint{5.467269in}{3.319516in}}{\pgfqpoint{5.475083in}{3.311702in}}%
\pgfpathcurveto{\pgfqpoint{5.482896in}{3.303889in}}{\pgfqpoint{5.493495in}{3.299499in}}{\pgfqpoint{5.504545in}{3.299499in}}%
\pgfpathclose%
\pgfusepath{stroke,fill}%
\end{pgfscope}%
\begin{pgfscope}%
\pgfpathrectangle{\pgfqpoint{0.800000in}{0.528000in}}{\pgfqpoint{4.960000in}{3.696000in}}%
\pgfusepath{clip}%
\pgfsetbuttcap%
\pgfsetroundjoin%
\definecolor{currentfill}{rgb}{0.000000,0.000000,0.000000}%
\pgfsetfillcolor{currentfill}%
\pgfsetlinewidth{1.003750pt}%
\definecolor{currentstroke}{rgb}{0.000000,0.000000,0.000000}%
\pgfsetstrokecolor{currentstroke}%
\pgfsetdash{}{0pt}%
\pgfpathmoveto{\pgfqpoint{5.504545in}{2.676071in}}%
\pgfpathcurveto{\pgfqpoint{5.515596in}{2.676071in}}{\pgfqpoint{5.526195in}{2.680462in}}{\pgfqpoint{5.534008in}{2.688275in}}%
\pgfpathcurveto{\pgfqpoint{5.541822in}{2.696089in}}{\pgfqpoint{5.546212in}{2.706688in}}{\pgfqpoint{5.546212in}{2.717738in}}%
\pgfpathcurveto{\pgfqpoint{5.546212in}{2.728788in}}{\pgfqpoint{5.541822in}{2.739387in}}{\pgfqpoint{5.534008in}{2.747201in}}%
\pgfpathcurveto{\pgfqpoint{5.526195in}{2.755014in}}{\pgfqpoint{5.515596in}{2.759405in}}{\pgfqpoint{5.504545in}{2.759405in}}%
\pgfpathcurveto{\pgfqpoint{5.493495in}{2.759405in}}{\pgfqpoint{5.482896in}{2.755014in}}{\pgfqpoint{5.475083in}{2.747201in}}%
\pgfpathcurveto{\pgfqpoint{5.467269in}{2.739387in}}{\pgfqpoint{5.462879in}{2.728788in}}{\pgfqpoint{5.462879in}{2.717738in}}%
\pgfpathcurveto{\pgfqpoint{5.462879in}{2.706688in}}{\pgfqpoint{5.467269in}{2.696089in}}{\pgfqpoint{5.475083in}{2.688275in}}%
\pgfpathcurveto{\pgfqpoint{5.482896in}{2.680462in}}{\pgfqpoint{5.493495in}{2.676071in}}{\pgfqpoint{5.504545in}{2.676071in}}%
\pgfpathclose%
\pgfusepath{stroke,fill}%
\end{pgfscope}%
\begin{pgfscope}%
\pgfpathrectangle{\pgfqpoint{0.800000in}{0.528000in}}{\pgfqpoint{4.960000in}{3.696000in}}%
\pgfusepath{clip}%
\pgfsetbuttcap%
\pgfsetroundjoin%
\definecolor{currentfill}{rgb}{0.000000,0.000000,0.000000}%
\pgfsetfillcolor{currentfill}%
\pgfsetlinewidth{1.003750pt}%
\definecolor{currentstroke}{rgb}{0.000000,0.000000,0.000000}%
\pgfsetstrokecolor{currentstroke}%
\pgfsetdash{}{0pt}%
\pgfpathmoveto{\pgfqpoint{5.504545in}{2.719066in}}%
\pgfpathcurveto{\pgfqpoint{5.515596in}{2.719066in}}{\pgfqpoint{5.526195in}{2.723456in}}{\pgfqpoint{5.534008in}{2.731270in}}%
\pgfpathcurveto{\pgfqpoint{5.541822in}{2.739084in}}{\pgfqpoint{5.546212in}{2.749683in}}{\pgfqpoint{5.546212in}{2.760733in}}%
\pgfpathcurveto{\pgfqpoint{5.546212in}{2.771783in}}{\pgfqpoint{5.541822in}{2.782382in}}{\pgfqpoint{5.534008in}{2.790196in}}%
\pgfpathcurveto{\pgfqpoint{5.526195in}{2.798009in}}{\pgfqpoint{5.515596in}{2.802400in}}{\pgfqpoint{5.504545in}{2.802400in}}%
\pgfpathcurveto{\pgfqpoint{5.493495in}{2.802400in}}{\pgfqpoint{5.482896in}{2.798009in}}{\pgfqpoint{5.475083in}{2.790196in}}%
\pgfpathcurveto{\pgfqpoint{5.467269in}{2.782382in}}{\pgfqpoint{5.462879in}{2.771783in}}{\pgfqpoint{5.462879in}{2.760733in}}%
\pgfpathcurveto{\pgfqpoint{5.462879in}{2.749683in}}{\pgfqpoint{5.467269in}{2.739084in}}{\pgfqpoint{5.475083in}{2.731270in}}%
\pgfpathcurveto{\pgfqpoint{5.482896in}{2.723456in}}{\pgfqpoint{5.493495in}{2.719066in}}{\pgfqpoint{5.504545in}{2.719066in}}%
\pgfpathclose%
\pgfusepath{stroke,fill}%
\end{pgfscope}%
\begin{pgfscope}%
\pgfpathrectangle{\pgfqpoint{0.800000in}{0.528000in}}{\pgfqpoint{4.960000in}{3.696000in}}%
\pgfusepath{clip}%
\pgfsetbuttcap%
\pgfsetroundjoin%
\definecolor{currentfill}{rgb}{0.000000,0.000000,0.000000}%
\pgfsetfillcolor{currentfill}%
\pgfsetlinewidth{1.003750pt}%
\definecolor{currentstroke}{rgb}{0.000000,0.000000,0.000000}%
\pgfsetstrokecolor{currentstroke}%
\pgfsetdash{}{0pt}%
\pgfpathmoveto{\pgfqpoint{5.504545in}{3.707951in}}%
\pgfpathcurveto{\pgfqpoint{5.515596in}{3.707951in}}{\pgfqpoint{5.526195in}{3.712341in}}{\pgfqpoint{5.534008in}{3.720155in}}%
\pgfpathcurveto{\pgfqpoint{5.541822in}{3.727968in}}{\pgfqpoint{5.546212in}{3.738567in}}{\pgfqpoint{5.546212in}{3.749618in}}%
\pgfpathcurveto{\pgfqpoint{5.546212in}{3.760668in}}{\pgfqpoint{5.541822in}{3.771267in}}{\pgfqpoint{5.534008in}{3.779080in}}%
\pgfpathcurveto{\pgfqpoint{5.526195in}{3.786894in}}{\pgfqpoint{5.515596in}{3.791284in}}{\pgfqpoint{5.504545in}{3.791284in}}%
\pgfpathcurveto{\pgfqpoint{5.493495in}{3.791284in}}{\pgfqpoint{5.482896in}{3.786894in}}{\pgfqpoint{5.475083in}{3.779080in}}%
\pgfpathcurveto{\pgfqpoint{5.467269in}{3.771267in}}{\pgfqpoint{5.462879in}{3.760668in}}{\pgfqpoint{5.462879in}{3.749618in}}%
\pgfpathcurveto{\pgfqpoint{5.462879in}{3.738567in}}{\pgfqpoint{5.467269in}{3.727968in}}{\pgfqpoint{5.475083in}{3.720155in}}%
\pgfpathcurveto{\pgfqpoint{5.482896in}{3.712341in}}{\pgfqpoint{5.493495in}{3.707951in}}{\pgfqpoint{5.504545in}{3.707951in}}%
\pgfpathclose%
\pgfusepath{stroke,fill}%
\end{pgfscope}%
\begin{pgfscope}%
\pgfpathrectangle{\pgfqpoint{0.800000in}{0.528000in}}{\pgfqpoint{4.960000in}{3.696000in}}%
\pgfusepath{clip}%
\pgfsetbuttcap%
\pgfsetroundjoin%
\definecolor{currentfill}{rgb}{0.000000,0.000000,0.000000}%
\pgfsetfillcolor{currentfill}%
\pgfsetlinewidth{1.003750pt}%
\definecolor{currentstroke}{rgb}{0.000000,0.000000,0.000000}%
\pgfsetstrokecolor{currentstroke}%
\pgfsetdash{}{0pt}%
\pgfpathmoveto{\pgfqpoint{5.504545in}{2.676071in}}%
\pgfpathcurveto{\pgfqpoint{5.515596in}{2.676071in}}{\pgfqpoint{5.526195in}{2.680462in}}{\pgfqpoint{5.534008in}{2.688275in}}%
\pgfpathcurveto{\pgfqpoint{5.541822in}{2.696089in}}{\pgfqpoint{5.546212in}{2.706688in}}{\pgfqpoint{5.546212in}{2.717738in}}%
\pgfpathcurveto{\pgfqpoint{5.546212in}{2.728788in}}{\pgfqpoint{5.541822in}{2.739387in}}{\pgfqpoint{5.534008in}{2.747201in}}%
\pgfpathcurveto{\pgfqpoint{5.526195in}{2.755014in}}{\pgfqpoint{5.515596in}{2.759405in}}{\pgfqpoint{5.504545in}{2.759405in}}%
\pgfpathcurveto{\pgfqpoint{5.493495in}{2.759405in}}{\pgfqpoint{5.482896in}{2.755014in}}{\pgfqpoint{5.475083in}{2.747201in}}%
\pgfpathcurveto{\pgfqpoint{5.467269in}{2.739387in}}{\pgfqpoint{5.462879in}{2.728788in}}{\pgfqpoint{5.462879in}{2.717738in}}%
\pgfpathcurveto{\pgfqpoint{5.462879in}{2.706688in}}{\pgfqpoint{5.467269in}{2.696089in}}{\pgfqpoint{5.475083in}{2.688275in}}%
\pgfpathcurveto{\pgfqpoint{5.482896in}{2.680462in}}{\pgfqpoint{5.493495in}{2.676071in}}{\pgfqpoint{5.504545in}{2.676071in}}%
\pgfpathclose%
\pgfusepath{stroke,fill}%
\end{pgfscope}%
\begin{pgfscope}%
\pgfpathrectangle{\pgfqpoint{0.800000in}{0.528000in}}{\pgfqpoint{4.960000in}{3.696000in}}%
\pgfusepath{clip}%
\pgfsetbuttcap%
\pgfsetroundjoin%
\definecolor{currentfill}{rgb}{0.000000,0.000000,0.000000}%
\pgfsetfillcolor{currentfill}%
\pgfsetlinewidth{1.003750pt}%
\definecolor{currentstroke}{rgb}{0.000000,0.000000,0.000000}%
\pgfsetstrokecolor{currentstroke}%
\pgfsetdash{}{0pt}%
\pgfpathmoveto{\pgfqpoint{5.504545in}{3.106021in}}%
\pgfpathcurveto{\pgfqpoint{5.515596in}{3.106021in}}{\pgfqpoint{5.526195in}{3.110411in}}{\pgfqpoint{5.534008in}{3.118225in}}%
\pgfpathcurveto{\pgfqpoint{5.541822in}{3.126039in}}{\pgfqpoint{5.546212in}{3.136638in}}{\pgfqpoint{5.546212in}{3.147688in}}%
\pgfpathcurveto{\pgfqpoint{5.546212in}{3.158738in}}{\pgfqpoint{5.541822in}{3.169337in}}{\pgfqpoint{5.534008in}{3.177151in}}%
\pgfpathcurveto{\pgfqpoint{5.526195in}{3.184964in}}{\pgfqpoint{5.515596in}{3.189354in}}{\pgfqpoint{5.504545in}{3.189354in}}%
\pgfpathcurveto{\pgfqpoint{5.493495in}{3.189354in}}{\pgfqpoint{5.482896in}{3.184964in}}{\pgfqpoint{5.475083in}{3.177151in}}%
\pgfpathcurveto{\pgfqpoint{5.467269in}{3.169337in}}{\pgfqpoint{5.462879in}{3.158738in}}{\pgfqpoint{5.462879in}{3.147688in}}%
\pgfpathcurveto{\pgfqpoint{5.462879in}{3.136638in}}{\pgfqpoint{5.467269in}{3.126039in}}{\pgfqpoint{5.475083in}{3.118225in}}%
\pgfpathcurveto{\pgfqpoint{5.482896in}{3.110411in}}{\pgfqpoint{5.493495in}{3.106021in}}{\pgfqpoint{5.504545in}{3.106021in}}%
\pgfpathclose%
\pgfusepath{stroke,fill}%
\end{pgfscope}%
\begin{pgfscope}%
\pgfpathrectangle{\pgfqpoint{0.800000in}{0.528000in}}{\pgfqpoint{4.960000in}{3.696000in}}%
\pgfusepath{clip}%
\pgfsetbuttcap%
\pgfsetroundjoin%
\definecolor{currentfill}{rgb}{0.000000,0.000000,0.000000}%
\pgfsetfillcolor{currentfill}%
\pgfsetlinewidth{1.003750pt}%
\definecolor{currentstroke}{rgb}{0.000000,0.000000,0.000000}%
\pgfsetstrokecolor{currentstroke}%
\pgfsetdash{}{0pt}%
\pgfpathmoveto{\pgfqpoint{5.504545in}{2.654574in}}%
\pgfpathcurveto{\pgfqpoint{5.515596in}{2.654574in}}{\pgfqpoint{5.526195in}{2.658964in}}{\pgfqpoint{5.534008in}{2.666778in}}%
\pgfpathcurveto{\pgfqpoint{5.541822in}{2.674591in}}{\pgfqpoint{5.546212in}{2.685190in}}{\pgfqpoint{5.546212in}{2.696240in}}%
\pgfpathcurveto{\pgfqpoint{5.546212in}{2.707291in}}{\pgfqpoint{5.541822in}{2.717890in}}{\pgfqpoint{5.534008in}{2.725703in}}%
\pgfpathcurveto{\pgfqpoint{5.526195in}{2.733517in}}{\pgfqpoint{5.515596in}{2.737907in}}{\pgfqpoint{5.504545in}{2.737907in}}%
\pgfpathcurveto{\pgfqpoint{5.493495in}{2.737907in}}{\pgfqpoint{5.482896in}{2.733517in}}{\pgfqpoint{5.475083in}{2.725703in}}%
\pgfpathcurveto{\pgfqpoint{5.467269in}{2.717890in}}{\pgfqpoint{5.462879in}{2.707291in}}{\pgfqpoint{5.462879in}{2.696240in}}%
\pgfpathcurveto{\pgfqpoint{5.462879in}{2.685190in}}{\pgfqpoint{5.467269in}{2.674591in}}{\pgfqpoint{5.475083in}{2.666778in}}%
\pgfpathcurveto{\pgfqpoint{5.482896in}{2.658964in}}{\pgfqpoint{5.493495in}{2.654574in}}{\pgfqpoint{5.504545in}{2.654574in}}%
\pgfpathclose%
\pgfusepath{stroke,fill}%
\end{pgfscope}%
\begin{pgfscope}%
\pgfpathrectangle{\pgfqpoint{0.800000in}{0.528000in}}{\pgfqpoint{4.960000in}{3.696000in}}%
\pgfusepath{clip}%
\pgfsetbuttcap%
\pgfsetroundjoin%
\definecolor{currentfill}{rgb}{0.000000,0.000000,0.000000}%
\pgfsetfillcolor{currentfill}%
\pgfsetlinewidth{1.003750pt}%
\definecolor{currentstroke}{rgb}{0.000000,0.000000,0.000000}%
\pgfsetstrokecolor{currentstroke}%
\pgfsetdash{}{0pt}%
\pgfpathmoveto{\pgfqpoint{5.504545in}{2.762061in}}%
\pgfpathcurveto{\pgfqpoint{5.515596in}{2.762061in}}{\pgfqpoint{5.526195in}{2.766451in}}{\pgfqpoint{5.534008in}{2.774265in}}%
\pgfpathcurveto{\pgfqpoint{5.541822in}{2.782079in}}{\pgfqpoint{5.546212in}{2.792678in}}{\pgfqpoint{5.546212in}{2.803728in}}%
\pgfpathcurveto{\pgfqpoint{5.546212in}{2.814778in}}{\pgfqpoint{5.541822in}{2.825377in}}{\pgfqpoint{5.534008in}{2.833191in}}%
\pgfpathcurveto{\pgfqpoint{5.526195in}{2.841004in}}{\pgfqpoint{5.515596in}{2.845395in}}{\pgfqpoint{5.504545in}{2.845395in}}%
\pgfpathcurveto{\pgfqpoint{5.493495in}{2.845395in}}{\pgfqpoint{5.482896in}{2.841004in}}{\pgfqpoint{5.475083in}{2.833191in}}%
\pgfpathcurveto{\pgfqpoint{5.467269in}{2.825377in}}{\pgfqpoint{5.462879in}{2.814778in}}{\pgfqpoint{5.462879in}{2.803728in}}%
\pgfpathcurveto{\pgfqpoint{5.462879in}{2.792678in}}{\pgfqpoint{5.467269in}{2.782079in}}{\pgfqpoint{5.475083in}{2.774265in}}%
\pgfpathcurveto{\pgfqpoint{5.482896in}{2.766451in}}{\pgfqpoint{5.493495in}{2.762061in}}{\pgfqpoint{5.504545in}{2.762061in}}%
\pgfpathclose%
\pgfusepath{stroke,fill}%
\end{pgfscope}%
\begin{pgfscope}%
\pgfpathrectangle{\pgfqpoint{0.800000in}{0.528000in}}{\pgfqpoint{4.960000in}{3.696000in}}%
\pgfusepath{clip}%
\pgfsetbuttcap%
\pgfsetroundjoin%
\definecolor{currentfill}{rgb}{0.000000,0.000000,0.000000}%
\pgfsetfillcolor{currentfill}%
\pgfsetlinewidth{1.003750pt}%
\definecolor{currentstroke}{rgb}{0.000000,0.000000,0.000000}%
\pgfsetstrokecolor{currentstroke}%
\pgfsetdash{}{0pt}%
\pgfpathmoveto{\pgfqpoint{5.504545in}{2.934041in}}%
\pgfpathcurveto{\pgfqpoint{5.515596in}{2.934041in}}{\pgfqpoint{5.526195in}{2.938431in}}{\pgfqpoint{5.534008in}{2.946245in}}%
\pgfpathcurveto{\pgfqpoint{5.541822in}{2.954059in}}{\pgfqpoint{5.546212in}{2.964658in}}{\pgfqpoint{5.546212in}{2.975708in}}%
\pgfpathcurveto{\pgfqpoint{5.546212in}{2.986758in}}{\pgfqpoint{5.541822in}{2.997357in}}{\pgfqpoint{5.534008in}{3.005171in}}%
\pgfpathcurveto{\pgfqpoint{5.526195in}{3.012984in}}{\pgfqpoint{5.515596in}{3.017374in}}{\pgfqpoint{5.504545in}{3.017374in}}%
\pgfpathcurveto{\pgfqpoint{5.493495in}{3.017374in}}{\pgfqpoint{5.482896in}{3.012984in}}{\pgfqpoint{5.475083in}{3.005171in}}%
\pgfpathcurveto{\pgfqpoint{5.467269in}{2.997357in}}{\pgfqpoint{5.462879in}{2.986758in}}{\pgfqpoint{5.462879in}{2.975708in}}%
\pgfpathcurveto{\pgfqpoint{5.462879in}{2.964658in}}{\pgfqpoint{5.467269in}{2.954059in}}{\pgfqpoint{5.475083in}{2.946245in}}%
\pgfpathcurveto{\pgfqpoint{5.482896in}{2.938431in}}{\pgfqpoint{5.493495in}{2.934041in}}{\pgfqpoint{5.504545in}{2.934041in}}%
\pgfpathclose%
\pgfusepath{stroke,fill}%
\end{pgfscope}%
\begin{pgfscope}%
\pgfpathrectangle{\pgfqpoint{0.800000in}{0.528000in}}{\pgfqpoint{4.960000in}{3.696000in}}%
\pgfusepath{clip}%
\pgfsetbuttcap%
\pgfsetroundjoin%
\definecolor{currentfill}{rgb}{0.000000,0.000000,0.000000}%
\pgfsetfillcolor{currentfill}%
\pgfsetlinewidth{1.003750pt}%
\definecolor{currentstroke}{rgb}{0.000000,0.000000,0.000000}%
\pgfsetstrokecolor{currentstroke}%
\pgfsetdash{}{0pt}%
\pgfpathmoveto{\pgfqpoint{5.504545in}{3.256504in}}%
\pgfpathcurveto{\pgfqpoint{5.515596in}{3.256504in}}{\pgfqpoint{5.526195in}{3.260894in}}{\pgfqpoint{5.534008in}{3.268707in}}%
\pgfpathcurveto{\pgfqpoint{5.541822in}{3.276521in}}{\pgfqpoint{5.546212in}{3.287120in}}{\pgfqpoint{5.546212in}{3.298170in}}%
\pgfpathcurveto{\pgfqpoint{5.546212in}{3.309220in}}{\pgfqpoint{5.541822in}{3.319819in}}{\pgfqpoint{5.534008in}{3.327633in}}%
\pgfpathcurveto{\pgfqpoint{5.526195in}{3.335447in}}{\pgfqpoint{5.515596in}{3.339837in}}{\pgfqpoint{5.504545in}{3.339837in}}%
\pgfpathcurveto{\pgfqpoint{5.493495in}{3.339837in}}{\pgfqpoint{5.482896in}{3.335447in}}{\pgfqpoint{5.475083in}{3.327633in}}%
\pgfpathcurveto{\pgfqpoint{5.467269in}{3.319819in}}{\pgfqpoint{5.462879in}{3.309220in}}{\pgfqpoint{5.462879in}{3.298170in}}%
\pgfpathcurveto{\pgfqpoint{5.462879in}{3.287120in}}{\pgfqpoint{5.467269in}{3.276521in}}{\pgfqpoint{5.475083in}{3.268707in}}%
\pgfpathcurveto{\pgfqpoint{5.482896in}{3.260894in}}{\pgfqpoint{5.493495in}{3.256504in}}{\pgfqpoint{5.504545in}{3.256504in}}%
\pgfpathclose%
\pgfusepath{stroke,fill}%
\end{pgfscope}%
\begin{pgfscope}%
\pgfpathrectangle{\pgfqpoint{0.800000in}{0.528000in}}{\pgfqpoint{4.960000in}{3.696000in}}%
\pgfusepath{clip}%
\pgfsetbuttcap%
\pgfsetroundjoin%
\definecolor{currentfill}{rgb}{0.000000,0.000000,0.000000}%
\pgfsetfillcolor{currentfill}%
\pgfsetlinewidth{1.003750pt}%
\definecolor{currentstroke}{rgb}{0.000000,0.000000,0.000000}%
\pgfsetstrokecolor{currentstroke}%
\pgfsetdash{}{0pt}%
\pgfpathmoveto{\pgfqpoint{5.504545in}{2.869549in}}%
\pgfpathcurveto{\pgfqpoint{5.515596in}{2.869549in}}{\pgfqpoint{5.526195in}{2.873939in}}{\pgfqpoint{5.534008in}{2.881753in}}%
\pgfpathcurveto{\pgfqpoint{5.541822in}{2.889566in}}{\pgfqpoint{5.546212in}{2.900165in}}{\pgfqpoint{5.546212in}{2.911215in}}%
\pgfpathcurveto{\pgfqpoint{5.546212in}{2.922265in}}{\pgfqpoint{5.541822in}{2.932864in}}{\pgfqpoint{5.534008in}{2.940678in}}%
\pgfpathcurveto{\pgfqpoint{5.526195in}{2.948492in}}{\pgfqpoint{5.515596in}{2.952882in}}{\pgfqpoint{5.504545in}{2.952882in}}%
\pgfpathcurveto{\pgfqpoint{5.493495in}{2.952882in}}{\pgfqpoint{5.482896in}{2.948492in}}{\pgfqpoint{5.475083in}{2.940678in}}%
\pgfpathcurveto{\pgfqpoint{5.467269in}{2.932864in}}{\pgfqpoint{5.462879in}{2.922265in}}{\pgfqpoint{5.462879in}{2.911215in}}%
\pgfpathcurveto{\pgfqpoint{5.462879in}{2.900165in}}{\pgfqpoint{5.467269in}{2.889566in}}{\pgfqpoint{5.475083in}{2.881753in}}%
\pgfpathcurveto{\pgfqpoint{5.482896in}{2.873939in}}{\pgfqpoint{5.493495in}{2.869549in}}{\pgfqpoint{5.504545in}{2.869549in}}%
\pgfpathclose%
\pgfusepath{stroke,fill}%
\end{pgfscope}%
\begin{pgfscope}%
\pgfpathrectangle{\pgfqpoint{0.800000in}{0.528000in}}{\pgfqpoint{4.960000in}{3.696000in}}%
\pgfusepath{clip}%
\pgfsetbuttcap%
\pgfsetroundjoin%
\definecolor{currentfill}{rgb}{0.000000,0.000000,0.000000}%
\pgfsetfillcolor{currentfill}%
\pgfsetlinewidth{1.003750pt}%
\definecolor{currentstroke}{rgb}{0.000000,0.000000,0.000000}%
\pgfsetstrokecolor{currentstroke}%
\pgfsetdash{}{0pt}%
\pgfpathmoveto{\pgfqpoint{5.504545in}{2.783559in}}%
\pgfpathcurveto{\pgfqpoint{5.515596in}{2.783559in}}{\pgfqpoint{5.526195in}{2.787949in}}{\pgfqpoint{5.534008in}{2.795763in}}%
\pgfpathcurveto{\pgfqpoint{5.541822in}{2.803576in}}{\pgfqpoint{5.546212in}{2.814175in}}{\pgfqpoint{5.546212in}{2.825225in}}%
\pgfpathcurveto{\pgfqpoint{5.546212in}{2.836275in}}{\pgfqpoint{5.541822in}{2.846875in}}{\pgfqpoint{5.534008in}{2.854688in}}%
\pgfpathcurveto{\pgfqpoint{5.526195in}{2.862502in}}{\pgfqpoint{5.515596in}{2.866892in}}{\pgfqpoint{5.504545in}{2.866892in}}%
\pgfpathcurveto{\pgfqpoint{5.493495in}{2.866892in}}{\pgfqpoint{5.482896in}{2.862502in}}{\pgfqpoint{5.475083in}{2.854688in}}%
\pgfpathcurveto{\pgfqpoint{5.467269in}{2.846875in}}{\pgfqpoint{5.462879in}{2.836275in}}{\pgfqpoint{5.462879in}{2.825225in}}%
\pgfpathcurveto{\pgfqpoint{5.462879in}{2.814175in}}{\pgfqpoint{5.467269in}{2.803576in}}{\pgfqpoint{5.475083in}{2.795763in}}%
\pgfpathcurveto{\pgfqpoint{5.482896in}{2.787949in}}{\pgfqpoint{5.493495in}{2.783559in}}{\pgfqpoint{5.504545in}{2.783559in}}%
\pgfpathclose%
\pgfusepath{stroke,fill}%
\end{pgfscope}%
\begin{pgfscope}%
\pgfpathrectangle{\pgfqpoint{0.800000in}{0.528000in}}{\pgfqpoint{4.960000in}{3.696000in}}%
\pgfusepath{clip}%
\pgfsetbuttcap%
\pgfsetroundjoin%
\definecolor{currentfill}{rgb}{0.000000,0.000000,0.000000}%
\pgfsetfillcolor{currentfill}%
\pgfsetlinewidth{1.003750pt}%
\definecolor{currentstroke}{rgb}{0.000000,0.000000,0.000000}%
\pgfsetstrokecolor{currentstroke}%
\pgfsetdash{}{0pt}%
\pgfpathmoveto{\pgfqpoint{5.504545in}{2.719066in}}%
\pgfpathcurveto{\pgfqpoint{5.515596in}{2.719066in}}{\pgfqpoint{5.526195in}{2.723456in}}{\pgfqpoint{5.534008in}{2.731270in}}%
\pgfpathcurveto{\pgfqpoint{5.541822in}{2.739084in}}{\pgfqpoint{5.546212in}{2.749683in}}{\pgfqpoint{5.546212in}{2.760733in}}%
\pgfpathcurveto{\pgfqpoint{5.546212in}{2.771783in}}{\pgfqpoint{5.541822in}{2.782382in}}{\pgfqpoint{5.534008in}{2.790196in}}%
\pgfpathcurveto{\pgfqpoint{5.526195in}{2.798009in}}{\pgfqpoint{5.515596in}{2.802400in}}{\pgfqpoint{5.504545in}{2.802400in}}%
\pgfpathcurveto{\pgfqpoint{5.493495in}{2.802400in}}{\pgfqpoint{5.482896in}{2.798009in}}{\pgfqpoint{5.475083in}{2.790196in}}%
\pgfpathcurveto{\pgfqpoint{5.467269in}{2.782382in}}{\pgfqpoint{5.462879in}{2.771783in}}{\pgfqpoint{5.462879in}{2.760733in}}%
\pgfpathcurveto{\pgfqpoint{5.462879in}{2.749683in}}{\pgfqpoint{5.467269in}{2.739084in}}{\pgfqpoint{5.475083in}{2.731270in}}%
\pgfpathcurveto{\pgfqpoint{5.482896in}{2.723456in}}{\pgfqpoint{5.493495in}{2.719066in}}{\pgfqpoint{5.504545in}{2.719066in}}%
\pgfpathclose%
\pgfusepath{stroke,fill}%
\end{pgfscope}%
\begin{pgfscope}%
\pgfpathrectangle{\pgfqpoint{0.800000in}{0.528000in}}{\pgfqpoint{4.960000in}{3.696000in}}%
\pgfusepath{clip}%
\pgfsetbuttcap%
\pgfsetroundjoin%
\definecolor{currentfill}{rgb}{0.000000,0.000000,0.000000}%
\pgfsetfillcolor{currentfill}%
\pgfsetlinewidth{1.003750pt}%
\definecolor{currentstroke}{rgb}{0.000000,0.000000,0.000000}%
\pgfsetstrokecolor{currentstroke}%
\pgfsetdash{}{0pt}%
\pgfpathmoveto{\pgfqpoint{5.504545in}{2.740564in}}%
\pgfpathcurveto{\pgfqpoint{5.515596in}{2.740564in}}{\pgfqpoint{5.526195in}{2.744954in}}{\pgfqpoint{5.534008in}{2.752768in}}%
\pgfpathcurveto{\pgfqpoint{5.541822in}{2.760581in}}{\pgfqpoint{5.546212in}{2.771180in}}{\pgfqpoint{5.546212in}{2.782230in}}%
\pgfpathcurveto{\pgfqpoint{5.546212in}{2.793281in}}{\pgfqpoint{5.541822in}{2.803880in}}{\pgfqpoint{5.534008in}{2.811693in}}%
\pgfpathcurveto{\pgfqpoint{5.526195in}{2.819507in}}{\pgfqpoint{5.515596in}{2.823897in}}{\pgfqpoint{5.504545in}{2.823897in}}%
\pgfpathcurveto{\pgfqpoint{5.493495in}{2.823897in}}{\pgfqpoint{5.482896in}{2.819507in}}{\pgfqpoint{5.475083in}{2.811693in}}%
\pgfpathcurveto{\pgfqpoint{5.467269in}{2.803880in}}{\pgfqpoint{5.462879in}{2.793281in}}{\pgfqpoint{5.462879in}{2.782230in}}%
\pgfpathcurveto{\pgfqpoint{5.462879in}{2.771180in}}{\pgfqpoint{5.467269in}{2.760581in}}{\pgfqpoint{5.475083in}{2.752768in}}%
\pgfpathcurveto{\pgfqpoint{5.482896in}{2.744954in}}{\pgfqpoint{5.493495in}{2.740564in}}{\pgfqpoint{5.504545in}{2.740564in}}%
\pgfpathclose%
\pgfusepath{stroke,fill}%
\end{pgfscope}%
\begin{pgfscope}%
\pgfpathrectangle{\pgfqpoint{0.800000in}{0.528000in}}{\pgfqpoint{4.960000in}{3.696000in}}%
\pgfusepath{clip}%
\pgfsetbuttcap%
\pgfsetroundjoin%
\definecolor{currentfill}{rgb}{0.000000,0.000000,0.000000}%
\pgfsetfillcolor{currentfill}%
\pgfsetlinewidth{1.003750pt}%
\definecolor{currentstroke}{rgb}{0.000000,0.000000,0.000000}%
\pgfsetstrokecolor{currentstroke}%
\pgfsetdash{}{0pt}%
\pgfpathmoveto{\pgfqpoint{5.504545in}{3.170514in}}%
\pgfpathcurveto{\pgfqpoint{5.515596in}{3.170514in}}{\pgfqpoint{5.526195in}{3.174904in}}{\pgfqpoint{5.534008in}{3.182717in}}%
\pgfpathcurveto{\pgfqpoint{5.541822in}{3.190531in}}{\pgfqpoint{5.546212in}{3.201130in}}{\pgfqpoint{5.546212in}{3.212180in}}%
\pgfpathcurveto{\pgfqpoint{5.546212in}{3.223230in}}{\pgfqpoint{5.541822in}{3.233829in}}{\pgfqpoint{5.534008in}{3.241643in}}%
\pgfpathcurveto{\pgfqpoint{5.526195in}{3.249457in}}{\pgfqpoint{5.515596in}{3.253847in}}{\pgfqpoint{5.504545in}{3.253847in}}%
\pgfpathcurveto{\pgfqpoint{5.493495in}{3.253847in}}{\pgfqpoint{5.482896in}{3.249457in}}{\pgfqpoint{5.475083in}{3.241643in}}%
\pgfpathcurveto{\pgfqpoint{5.467269in}{3.233829in}}{\pgfqpoint{5.462879in}{3.223230in}}{\pgfqpoint{5.462879in}{3.212180in}}%
\pgfpathcurveto{\pgfqpoint{5.462879in}{3.201130in}}{\pgfqpoint{5.467269in}{3.190531in}}{\pgfqpoint{5.475083in}{3.182717in}}%
\pgfpathcurveto{\pgfqpoint{5.482896in}{3.174904in}}{\pgfqpoint{5.493495in}{3.170514in}}{\pgfqpoint{5.504545in}{3.170514in}}%
\pgfpathclose%
\pgfusepath{stroke,fill}%
\end{pgfscope}%
\begin{pgfscope}%
\pgfpathrectangle{\pgfqpoint{0.800000in}{0.528000in}}{\pgfqpoint{4.960000in}{3.696000in}}%
\pgfusepath{clip}%
\pgfsetbuttcap%
\pgfsetroundjoin%
\definecolor{currentfill}{rgb}{0.000000,0.000000,0.000000}%
\pgfsetfillcolor{currentfill}%
\pgfsetlinewidth{1.003750pt}%
\definecolor{currentstroke}{rgb}{0.000000,0.000000,0.000000}%
\pgfsetstrokecolor{currentstroke}%
\pgfsetdash{}{0pt}%
\pgfpathmoveto{\pgfqpoint{5.504545in}{2.590081in}}%
\pgfpathcurveto{\pgfqpoint{5.515596in}{2.590081in}}{\pgfqpoint{5.526195in}{2.594472in}}{\pgfqpoint{5.534008in}{2.602285in}}%
\pgfpathcurveto{\pgfqpoint{5.541822in}{2.610099in}}{\pgfqpoint{5.546212in}{2.620698in}}{\pgfqpoint{5.546212in}{2.631748in}}%
\pgfpathcurveto{\pgfqpoint{5.546212in}{2.642798in}}{\pgfqpoint{5.541822in}{2.653397in}}{\pgfqpoint{5.534008in}{2.661211in}}%
\pgfpathcurveto{\pgfqpoint{5.526195in}{2.669024in}}{\pgfqpoint{5.515596in}{2.673415in}}{\pgfqpoint{5.504545in}{2.673415in}}%
\pgfpathcurveto{\pgfqpoint{5.493495in}{2.673415in}}{\pgfqpoint{5.482896in}{2.669024in}}{\pgfqpoint{5.475083in}{2.661211in}}%
\pgfpathcurveto{\pgfqpoint{5.467269in}{2.653397in}}{\pgfqpoint{5.462879in}{2.642798in}}{\pgfqpoint{5.462879in}{2.631748in}}%
\pgfpathcurveto{\pgfqpoint{5.462879in}{2.620698in}}{\pgfqpoint{5.467269in}{2.610099in}}{\pgfqpoint{5.475083in}{2.602285in}}%
\pgfpathcurveto{\pgfqpoint{5.482896in}{2.594472in}}{\pgfqpoint{5.493495in}{2.590081in}}{\pgfqpoint{5.504545in}{2.590081in}}%
\pgfpathclose%
\pgfusepath{stroke,fill}%
\end{pgfscope}%
\begin{pgfscope}%
\pgfpathrectangle{\pgfqpoint{0.800000in}{0.528000in}}{\pgfqpoint{4.960000in}{3.696000in}}%
\pgfusepath{clip}%
\pgfsetbuttcap%
\pgfsetroundjoin%
\definecolor{currentfill}{rgb}{0.000000,0.000000,0.000000}%
\pgfsetfillcolor{currentfill}%
\pgfsetlinewidth{1.003750pt}%
\definecolor{currentstroke}{rgb}{0.000000,0.000000,0.000000}%
\pgfsetstrokecolor{currentstroke}%
\pgfsetdash{}{0pt}%
\pgfpathmoveto{\pgfqpoint{5.504545in}{3.084524in}}%
\pgfpathcurveto{\pgfqpoint{5.515596in}{3.084524in}}{\pgfqpoint{5.526195in}{3.088914in}}{\pgfqpoint{5.534008in}{3.096727in}}%
\pgfpathcurveto{\pgfqpoint{5.541822in}{3.104541in}}{\pgfqpoint{5.546212in}{3.115140in}}{\pgfqpoint{5.546212in}{3.126190in}}%
\pgfpathcurveto{\pgfqpoint{5.546212in}{3.137240in}}{\pgfqpoint{5.541822in}{3.147839in}}{\pgfqpoint{5.534008in}{3.155653in}}%
\pgfpathcurveto{\pgfqpoint{5.526195in}{3.163467in}}{\pgfqpoint{5.515596in}{3.167857in}}{\pgfqpoint{5.504545in}{3.167857in}}%
\pgfpathcurveto{\pgfqpoint{5.493495in}{3.167857in}}{\pgfqpoint{5.482896in}{3.163467in}}{\pgfqpoint{5.475083in}{3.155653in}}%
\pgfpathcurveto{\pgfqpoint{5.467269in}{3.147839in}}{\pgfqpoint{5.462879in}{3.137240in}}{\pgfqpoint{5.462879in}{3.126190in}}%
\pgfpathcurveto{\pgfqpoint{5.462879in}{3.115140in}}{\pgfqpoint{5.467269in}{3.104541in}}{\pgfqpoint{5.475083in}{3.096727in}}%
\pgfpathcurveto{\pgfqpoint{5.482896in}{3.088914in}}{\pgfqpoint{5.493495in}{3.084524in}}{\pgfqpoint{5.504545in}{3.084524in}}%
\pgfpathclose%
\pgfusepath{stroke,fill}%
\end{pgfscope}%
\begin{pgfscope}%
\pgfpathrectangle{\pgfqpoint{0.800000in}{0.528000in}}{\pgfqpoint{4.960000in}{3.696000in}}%
\pgfusepath{clip}%
\pgfsetbuttcap%
\pgfsetroundjoin%
\definecolor{currentfill}{rgb}{0.000000,0.000000,0.000000}%
\pgfsetfillcolor{currentfill}%
\pgfsetlinewidth{1.003750pt}%
\definecolor{currentstroke}{rgb}{0.000000,0.000000,0.000000}%
\pgfsetstrokecolor{currentstroke}%
\pgfsetdash{}{0pt}%
\pgfpathmoveto{\pgfqpoint{5.504545in}{3.106021in}}%
\pgfpathcurveto{\pgfqpoint{5.515596in}{3.106021in}}{\pgfqpoint{5.526195in}{3.110411in}}{\pgfqpoint{5.534008in}{3.118225in}}%
\pgfpathcurveto{\pgfqpoint{5.541822in}{3.126039in}}{\pgfqpoint{5.546212in}{3.136638in}}{\pgfqpoint{5.546212in}{3.147688in}}%
\pgfpathcurveto{\pgfqpoint{5.546212in}{3.158738in}}{\pgfqpoint{5.541822in}{3.169337in}}{\pgfqpoint{5.534008in}{3.177151in}}%
\pgfpathcurveto{\pgfqpoint{5.526195in}{3.184964in}}{\pgfqpoint{5.515596in}{3.189354in}}{\pgfqpoint{5.504545in}{3.189354in}}%
\pgfpathcurveto{\pgfqpoint{5.493495in}{3.189354in}}{\pgfqpoint{5.482896in}{3.184964in}}{\pgfqpoint{5.475083in}{3.177151in}}%
\pgfpathcurveto{\pgfqpoint{5.467269in}{3.169337in}}{\pgfqpoint{5.462879in}{3.158738in}}{\pgfqpoint{5.462879in}{3.147688in}}%
\pgfpathcurveto{\pgfqpoint{5.462879in}{3.136638in}}{\pgfqpoint{5.467269in}{3.126039in}}{\pgfqpoint{5.475083in}{3.118225in}}%
\pgfpathcurveto{\pgfqpoint{5.482896in}{3.110411in}}{\pgfqpoint{5.493495in}{3.106021in}}{\pgfqpoint{5.504545in}{3.106021in}}%
\pgfpathclose%
\pgfusepath{stroke,fill}%
\end{pgfscope}%
\begin{pgfscope}%
\pgfpathrectangle{\pgfqpoint{0.800000in}{0.528000in}}{\pgfqpoint{4.960000in}{3.696000in}}%
\pgfusepath{clip}%
\pgfsetbuttcap%
\pgfsetroundjoin%
\definecolor{currentfill}{rgb}{0.000000,0.000000,0.000000}%
\pgfsetfillcolor{currentfill}%
\pgfsetlinewidth{1.003750pt}%
\definecolor{currentstroke}{rgb}{0.000000,0.000000,0.000000}%
\pgfsetstrokecolor{currentstroke}%
\pgfsetdash{}{0pt}%
\pgfpathmoveto{\pgfqpoint{5.504545in}{2.783559in}}%
\pgfpathcurveto{\pgfqpoint{5.515596in}{2.783559in}}{\pgfqpoint{5.526195in}{2.787949in}}{\pgfqpoint{5.534008in}{2.795763in}}%
\pgfpathcurveto{\pgfqpoint{5.541822in}{2.803576in}}{\pgfqpoint{5.546212in}{2.814175in}}{\pgfqpoint{5.546212in}{2.825225in}}%
\pgfpathcurveto{\pgfqpoint{5.546212in}{2.836275in}}{\pgfqpoint{5.541822in}{2.846875in}}{\pgfqpoint{5.534008in}{2.854688in}}%
\pgfpathcurveto{\pgfqpoint{5.526195in}{2.862502in}}{\pgfqpoint{5.515596in}{2.866892in}}{\pgfqpoint{5.504545in}{2.866892in}}%
\pgfpathcurveto{\pgfqpoint{5.493495in}{2.866892in}}{\pgfqpoint{5.482896in}{2.862502in}}{\pgfqpoint{5.475083in}{2.854688in}}%
\pgfpathcurveto{\pgfqpoint{5.467269in}{2.846875in}}{\pgfqpoint{5.462879in}{2.836275in}}{\pgfqpoint{5.462879in}{2.825225in}}%
\pgfpathcurveto{\pgfqpoint{5.462879in}{2.814175in}}{\pgfqpoint{5.467269in}{2.803576in}}{\pgfqpoint{5.475083in}{2.795763in}}%
\pgfpathcurveto{\pgfqpoint{5.482896in}{2.787949in}}{\pgfqpoint{5.493495in}{2.783559in}}{\pgfqpoint{5.504545in}{2.783559in}}%
\pgfpathclose%
\pgfusepath{stroke,fill}%
\end{pgfscope}%
\begin{pgfscope}%
\pgfpathrectangle{\pgfqpoint{0.800000in}{0.528000in}}{\pgfqpoint{4.960000in}{3.696000in}}%
\pgfusepath{clip}%
\pgfsetbuttcap%
\pgfsetroundjoin%
\definecolor{currentfill}{rgb}{0.000000,0.000000,0.000000}%
\pgfsetfillcolor{currentfill}%
\pgfsetlinewidth{1.003750pt}%
\definecolor{currentstroke}{rgb}{0.000000,0.000000,0.000000}%
\pgfsetstrokecolor{currentstroke}%
\pgfsetdash{}{0pt}%
\pgfpathmoveto{\pgfqpoint{5.504545in}{2.762061in}}%
\pgfpathcurveto{\pgfqpoint{5.515596in}{2.762061in}}{\pgfqpoint{5.526195in}{2.766451in}}{\pgfqpoint{5.534008in}{2.774265in}}%
\pgfpathcurveto{\pgfqpoint{5.541822in}{2.782079in}}{\pgfqpoint{5.546212in}{2.792678in}}{\pgfqpoint{5.546212in}{2.803728in}}%
\pgfpathcurveto{\pgfqpoint{5.546212in}{2.814778in}}{\pgfqpoint{5.541822in}{2.825377in}}{\pgfqpoint{5.534008in}{2.833191in}}%
\pgfpathcurveto{\pgfqpoint{5.526195in}{2.841004in}}{\pgfqpoint{5.515596in}{2.845395in}}{\pgfqpoint{5.504545in}{2.845395in}}%
\pgfpathcurveto{\pgfqpoint{5.493495in}{2.845395in}}{\pgfqpoint{5.482896in}{2.841004in}}{\pgfqpoint{5.475083in}{2.833191in}}%
\pgfpathcurveto{\pgfqpoint{5.467269in}{2.825377in}}{\pgfqpoint{5.462879in}{2.814778in}}{\pgfqpoint{5.462879in}{2.803728in}}%
\pgfpathcurveto{\pgfqpoint{5.462879in}{2.792678in}}{\pgfqpoint{5.467269in}{2.782079in}}{\pgfqpoint{5.475083in}{2.774265in}}%
\pgfpathcurveto{\pgfqpoint{5.482896in}{2.766451in}}{\pgfqpoint{5.493495in}{2.762061in}}{\pgfqpoint{5.504545in}{2.762061in}}%
\pgfpathclose%
\pgfusepath{stroke,fill}%
\end{pgfscope}%
\begin{pgfscope}%
\pgfpathrectangle{\pgfqpoint{0.800000in}{0.528000in}}{\pgfqpoint{4.960000in}{3.696000in}}%
\pgfusepath{clip}%
\pgfsetbuttcap%
\pgfsetroundjoin%
\definecolor{currentfill}{rgb}{0.000000,0.000000,0.000000}%
\pgfsetfillcolor{currentfill}%
\pgfsetlinewidth{1.003750pt}%
\definecolor{currentstroke}{rgb}{0.000000,0.000000,0.000000}%
\pgfsetstrokecolor{currentstroke}%
\pgfsetdash{}{0pt}%
\pgfpathmoveto{\pgfqpoint{5.504545in}{2.869549in}}%
\pgfpathcurveto{\pgfqpoint{5.515596in}{2.869549in}}{\pgfqpoint{5.526195in}{2.873939in}}{\pgfqpoint{5.534008in}{2.881753in}}%
\pgfpathcurveto{\pgfqpoint{5.541822in}{2.889566in}}{\pgfqpoint{5.546212in}{2.900165in}}{\pgfqpoint{5.546212in}{2.911215in}}%
\pgfpathcurveto{\pgfqpoint{5.546212in}{2.922265in}}{\pgfqpoint{5.541822in}{2.932864in}}{\pgfqpoint{5.534008in}{2.940678in}}%
\pgfpathcurveto{\pgfqpoint{5.526195in}{2.948492in}}{\pgfqpoint{5.515596in}{2.952882in}}{\pgfqpoint{5.504545in}{2.952882in}}%
\pgfpathcurveto{\pgfqpoint{5.493495in}{2.952882in}}{\pgfqpoint{5.482896in}{2.948492in}}{\pgfqpoint{5.475083in}{2.940678in}}%
\pgfpathcurveto{\pgfqpoint{5.467269in}{2.932864in}}{\pgfqpoint{5.462879in}{2.922265in}}{\pgfqpoint{5.462879in}{2.911215in}}%
\pgfpathcurveto{\pgfqpoint{5.462879in}{2.900165in}}{\pgfqpoint{5.467269in}{2.889566in}}{\pgfqpoint{5.475083in}{2.881753in}}%
\pgfpathcurveto{\pgfqpoint{5.482896in}{2.873939in}}{\pgfqpoint{5.493495in}{2.869549in}}{\pgfqpoint{5.504545in}{2.869549in}}%
\pgfpathclose%
\pgfusepath{stroke,fill}%
\end{pgfscope}%
\begin{pgfscope}%
\pgfpathrectangle{\pgfqpoint{0.800000in}{0.528000in}}{\pgfqpoint{4.960000in}{3.696000in}}%
\pgfusepath{clip}%
\pgfsetbuttcap%
\pgfsetroundjoin%
\definecolor{currentfill}{rgb}{0.000000,0.000000,0.000000}%
\pgfsetfillcolor{currentfill}%
\pgfsetlinewidth{1.003750pt}%
\definecolor{currentstroke}{rgb}{0.000000,0.000000,0.000000}%
\pgfsetstrokecolor{currentstroke}%
\pgfsetdash{}{0pt}%
\pgfpathmoveto{\pgfqpoint{5.504545in}{2.697569in}}%
\pgfpathcurveto{\pgfqpoint{5.515596in}{2.697569in}}{\pgfqpoint{5.526195in}{2.701959in}}{\pgfqpoint{5.534008in}{2.709773in}}%
\pgfpathcurveto{\pgfqpoint{5.541822in}{2.717586in}}{\pgfqpoint{5.546212in}{2.728185in}}{\pgfqpoint{5.546212in}{2.739235in}}%
\pgfpathcurveto{\pgfqpoint{5.546212in}{2.750286in}}{\pgfqpoint{5.541822in}{2.760885in}}{\pgfqpoint{5.534008in}{2.768698in}}%
\pgfpathcurveto{\pgfqpoint{5.526195in}{2.776512in}}{\pgfqpoint{5.515596in}{2.780902in}}{\pgfqpoint{5.504545in}{2.780902in}}%
\pgfpathcurveto{\pgfqpoint{5.493495in}{2.780902in}}{\pgfqpoint{5.482896in}{2.776512in}}{\pgfqpoint{5.475083in}{2.768698in}}%
\pgfpathcurveto{\pgfqpoint{5.467269in}{2.760885in}}{\pgfqpoint{5.462879in}{2.750286in}}{\pgfqpoint{5.462879in}{2.739235in}}%
\pgfpathcurveto{\pgfqpoint{5.462879in}{2.728185in}}{\pgfqpoint{5.467269in}{2.717586in}}{\pgfqpoint{5.475083in}{2.709773in}}%
\pgfpathcurveto{\pgfqpoint{5.482896in}{2.701959in}}{\pgfqpoint{5.493495in}{2.697569in}}{\pgfqpoint{5.504545in}{2.697569in}}%
\pgfpathclose%
\pgfusepath{stroke,fill}%
\end{pgfscope}%
\begin{pgfscope}%
\pgfpathrectangle{\pgfqpoint{0.800000in}{0.528000in}}{\pgfqpoint{4.960000in}{3.696000in}}%
\pgfusepath{clip}%
\pgfsetbuttcap%
\pgfsetroundjoin%
\definecolor{currentfill}{rgb}{0.000000,0.000000,0.000000}%
\pgfsetfillcolor{currentfill}%
\pgfsetlinewidth{1.003750pt}%
\definecolor{currentstroke}{rgb}{0.000000,0.000000,0.000000}%
\pgfsetstrokecolor{currentstroke}%
\pgfsetdash{}{0pt}%
\pgfpathmoveto{\pgfqpoint{5.504545in}{3.428483in}}%
\pgfpathcurveto{\pgfqpoint{5.515596in}{3.428483in}}{\pgfqpoint{5.526195in}{3.432874in}}{\pgfqpoint{5.534008in}{3.440687in}}%
\pgfpathcurveto{\pgfqpoint{5.541822in}{3.448501in}}{\pgfqpoint{5.546212in}{3.459100in}}{\pgfqpoint{5.546212in}{3.470150in}}%
\pgfpathcurveto{\pgfqpoint{5.546212in}{3.481200in}}{\pgfqpoint{5.541822in}{3.491799in}}{\pgfqpoint{5.534008in}{3.499613in}}%
\pgfpathcurveto{\pgfqpoint{5.526195in}{3.507427in}}{\pgfqpoint{5.515596in}{3.511817in}}{\pgfqpoint{5.504545in}{3.511817in}}%
\pgfpathcurveto{\pgfqpoint{5.493495in}{3.511817in}}{\pgfqpoint{5.482896in}{3.507427in}}{\pgfqpoint{5.475083in}{3.499613in}}%
\pgfpathcurveto{\pgfqpoint{5.467269in}{3.491799in}}{\pgfqpoint{5.462879in}{3.481200in}}{\pgfqpoint{5.462879in}{3.470150in}}%
\pgfpathcurveto{\pgfqpoint{5.462879in}{3.459100in}}{\pgfqpoint{5.467269in}{3.448501in}}{\pgfqpoint{5.475083in}{3.440687in}}%
\pgfpathcurveto{\pgfqpoint{5.482896in}{3.432874in}}{\pgfqpoint{5.493495in}{3.428483in}}{\pgfqpoint{5.504545in}{3.428483in}}%
\pgfpathclose%
\pgfusepath{stroke,fill}%
\end{pgfscope}%
\begin{pgfscope}%
\pgfpathrectangle{\pgfqpoint{0.800000in}{0.528000in}}{\pgfqpoint{4.960000in}{3.696000in}}%
\pgfusepath{clip}%
\pgfsetbuttcap%
\pgfsetroundjoin%
\definecolor{currentfill}{rgb}{0.000000,0.000000,0.000000}%
\pgfsetfillcolor{currentfill}%
\pgfsetlinewidth{1.003750pt}%
\definecolor{currentstroke}{rgb}{0.000000,0.000000,0.000000}%
\pgfsetstrokecolor{currentstroke}%
\pgfsetdash{}{0pt}%
\pgfpathmoveto{\pgfqpoint{5.504545in}{2.848051in}}%
\pgfpathcurveto{\pgfqpoint{5.515596in}{2.848051in}}{\pgfqpoint{5.526195in}{2.852441in}}{\pgfqpoint{5.534008in}{2.860255in}}%
\pgfpathcurveto{\pgfqpoint{5.541822in}{2.868069in}}{\pgfqpoint{5.546212in}{2.878668in}}{\pgfqpoint{5.546212in}{2.889718in}}%
\pgfpathcurveto{\pgfqpoint{5.546212in}{2.900768in}}{\pgfqpoint{5.541822in}{2.911367in}}{\pgfqpoint{5.534008in}{2.919181in}}%
\pgfpathcurveto{\pgfqpoint{5.526195in}{2.926994in}}{\pgfqpoint{5.515596in}{2.931385in}}{\pgfqpoint{5.504545in}{2.931385in}}%
\pgfpathcurveto{\pgfqpoint{5.493495in}{2.931385in}}{\pgfqpoint{5.482896in}{2.926994in}}{\pgfqpoint{5.475083in}{2.919181in}}%
\pgfpathcurveto{\pgfqpoint{5.467269in}{2.911367in}}{\pgfqpoint{5.462879in}{2.900768in}}{\pgfqpoint{5.462879in}{2.889718in}}%
\pgfpathcurveto{\pgfqpoint{5.462879in}{2.878668in}}{\pgfqpoint{5.467269in}{2.868069in}}{\pgfqpoint{5.475083in}{2.860255in}}%
\pgfpathcurveto{\pgfqpoint{5.482896in}{2.852441in}}{\pgfqpoint{5.493495in}{2.848051in}}{\pgfqpoint{5.504545in}{2.848051in}}%
\pgfpathclose%
\pgfusepath{stroke,fill}%
\end{pgfscope}%
\begin{pgfscope}%
\pgfpathrectangle{\pgfqpoint{0.800000in}{0.528000in}}{\pgfqpoint{4.960000in}{3.696000in}}%
\pgfusepath{clip}%
\pgfsetbuttcap%
\pgfsetroundjoin%
\definecolor{currentfill}{rgb}{0.000000,0.000000,0.000000}%
\pgfsetfillcolor{currentfill}%
\pgfsetlinewidth{1.003750pt}%
\definecolor{currentstroke}{rgb}{0.000000,0.000000,0.000000}%
\pgfsetstrokecolor{currentstroke}%
\pgfsetdash{}{0pt}%
\pgfpathmoveto{\pgfqpoint{5.504545in}{3.278001in}}%
\pgfpathcurveto{\pgfqpoint{5.515596in}{3.278001in}}{\pgfqpoint{5.526195in}{3.282391in}}{\pgfqpoint{5.534008in}{3.290205in}}%
\pgfpathcurveto{\pgfqpoint{5.541822in}{3.298019in}}{\pgfqpoint{5.546212in}{3.308618in}}{\pgfqpoint{5.546212in}{3.319668in}}%
\pgfpathcurveto{\pgfqpoint{5.546212in}{3.330718in}}{\pgfqpoint{5.541822in}{3.341317in}}{\pgfqpoint{5.534008in}{3.349130in}}%
\pgfpathcurveto{\pgfqpoint{5.526195in}{3.356944in}}{\pgfqpoint{5.515596in}{3.361334in}}{\pgfqpoint{5.504545in}{3.361334in}}%
\pgfpathcurveto{\pgfqpoint{5.493495in}{3.361334in}}{\pgfqpoint{5.482896in}{3.356944in}}{\pgfqpoint{5.475083in}{3.349130in}}%
\pgfpathcurveto{\pgfqpoint{5.467269in}{3.341317in}}{\pgfqpoint{5.462879in}{3.330718in}}{\pgfqpoint{5.462879in}{3.319668in}}%
\pgfpathcurveto{\pgfqpoint{5.462879in}{3.308618in}}{\pgfqpoint{5.467269in}{3.298019in}}{\pgfqpoint{5.475083in}{3.290205in}}%
\pgfpathcurveto{\pgfqpoint{5.482896in}{3.282391in}}{\pgfqpoint{5.493495in}{3.278001in}}{\pgfqpoint{5.504545in}{3.278001in}}%
\pgfpathclose%
\pgfusepath{stroke,fill}%
\end{pgfscope}%
\begin{pgfscope}%
\pgfpathrectangle{\pgfqpoint{0.800000in}{0.528000in}}{\pgfqpoint{4.960000in}{3.696000in}}%
\pgfusepath{clip}%
\pgfsetbuttcap%
\pgfsetroundjoin%
\definecolor{currentfill}{rgb}{0.000000,0.000000,0.000000}%
\pgfsetfillcolor{currentfill}%
\pgfsetlinewidth{1.003750pt}%
\definecolor{currentstroke}{rgb}{0.000000,0.000000,0.000000}%
\pgfsetstrokecolor{currentstroke}%
\pgfsetdash{}{0pt}%
\pgfpathmoveto{\pgfqpoint{5.504545in}{2.891046in}}%
\pgfpathcurveto{\pgfqpoint{5.515596in}{2.891046in}}{\pgfqpoint{5.526195in}{2.895436in}}{\pgfqpoint{5.534008in}{2.903250in}}%
\pgfpathcurveto{\pgfqpoint{5.541822in}{2.911064in}}{\pgfqpoint{5.546212in}{2.921663in}}{\pgfqpoint{5.546212in}{2.932713in}}%
\pgfpathcurveto{\pgfqpoint{5.546212in}{2.943763in}}{\pgfqpoint{5.541822in}{2.954362in}}{\pgfqpoint{5.534008in}{2.962176in}}%
\pgfpathcurveto{\pgfqpoint{5.526195in}{2.969989in}}{\pgfqpoint{5.515596in}{2.974379in}}{\pgfqpoint{5.504545in}{2.974379in}}%
\pgfpathcurveto{\pgfqpoint{5.493495in}{2.974379in}}{\pgfqpoint{5.482896in}{2.969989in}}{\pgfqpoint{5.475083in}{2.962176in}}%
\pgfpathcurveto{\pgfqpoint{5.467269in}{2.954362in}}{\pgfqpoint{5.462879in}{2.943763in}}{\pgfqpoint{5.462879in}{2.932713in}}%
\pgfpathcurveto{\pgfqpoint{5.462879in}{2.921663in}}{\pgfqpoint{5.467269in}{2.911064in}}{\pgfqpoint{5.475083in}{2.903250in}}%
\pgfpathcurveto{\pgfqpoint{5.482896in}{2.895436in}}{\pgfqpoint{5.493495in}{2.891046in}}{\pgfqpoint{5.504545in}{2.891046in}}%
\pgfpathclose%
\pgfusepath{stroke,fill}%
\end{pgfscope}%
\begin{pgfscope}%
\pgfpathrectangle{\pgfqpoint{0.800000in}{0.528000in}}{\pgfqpoint{4.960000in}{3.696000in}}%
\pgfusepath{clip}%
\pgfsetbuttcap%
\pgfsetroundjoin%
\definecolor{currentfill}{rgb}{0.000000,0.000000,0.000000}%
\pgfsetfillcolor{currentfill}%
\pgfsetlinewidth{1.003750pt}%
\definecolor{currentstroke}{rgb}{0.000000,0.000000,0.000000}%
\pgfsetstrokecolor{currentstroke}%
\pgfsetdash{}{0pt}%
\pgfpathmoveto{\pgfqpoint{5.504545in}{2.955539in}}%
\pgfpathcurveto{\pgfqpoint{5.515596in}{2.955539in}}{\pgfqpoint{5.526195in}{2.959929in}}{\pgfqpoint{5.534008in}{2.967743in}}%
\pgfpathcurveto{\pgfqpoint{5.541822in}{2.975556in}}{\pgfqpoint{5.546212in}{2.986155in}}{\pgfqpoint{5.546212in}{2.997205in}}%
\pgfpathcurveto{\pgfqpoint{5.546212in}{3.008255in}}{\pgfqpoint{5.541822in}{3.018854in}}{\pgfqpoint{5.534008in}{3.026668in}}%
\pgfpathcurveto{\pgfqpoint{5.526195in}{3.034482in}}{\pgfqpoint{5.515596in}{3.038872in}}{\pgfqpoint{5.504545in}{3.038872in}}%
\pgfpathcurveto{\pgfqpoint{5.493495in}{3.038872in}}{\pgfqpoint{5.482896in}{3.034482in}}{\pgfqpoint{5.475083in}{3.026668in}}%
\pgfpathcurveto{\pgfqpoint{5.467269in}{3.018854in}}{\pgfqpoint{5.462879in}{3.008255in}}{\pgfqpoint{5.462879in}{2.997205in}}%
\pgfpathcurveto{\pgfqpoint{5.462879in}{2.986155in}}{\pgfqpoint{5.467269in}{2.975556in}}{\pgfqpoint{5.475083in}{2.967743in}}%
\pgfpathcurveto{\pgfqpoint{5.482896in}{2.959929in}}{\pgfqpoint{5.493495in}{2.955539in}}{\pgfqpoint{5.504545in}{2.955539in}}%
\pgfpathclose%
\pgfusepath{stroke,fill}%
\end{pgfscope}%
\begin{pgfscope}%
\pgfpathrectangle{\pgfqpoint{0.800000in}{0.528000in}}{\pgfqpoint{4.960000in}{3.696000in}}%
\pgfusepath{clip}%
\pgfsetbuttcap%
\pgfsetroundjoin%
\definecolor{currentfill}{rgb}{0.000000,0.000000,0.000000}%
\pgfsetfillcolor{currentfill}%
\pgfsetlinewidth{1.003750pt}%
\definecolor{currentstroke}{rgb}{0.000000,0.000000,0.000000}%
\pgfsetstrokecolor{currentstroke}%
\pgfsetdash{}{0pt}%
\pgfpathmoveto{\pgfqpoint{5.504545in}{3.299499in}}%
\pgfpathcurveto{\pgfqpoint{5.515596in}{3.299499in}}{\pgfqpoint{5.526195in}{3.303889in}}{\pgfqpoint{5.534008in}{3.311702in}}%
\pgfpathcurveto{\pgfqpoint{5.541822in}{3.319516in}}{\pgfqpoint{5.546212in}{3.330115in}}{\pgfqpoint{5.546212in}{3.341165in}}%
\pgfpathcurveto{\pgfqpoint{5.546212in}{3.352215in}}{\pgfqpoint{5.541822in}{3.362814in}}{\pgfqpoint{5.534008in}{3.370628in}}%
\pgfpathcurveto{\pgfqpoint{5.526195in}{3.378442in}}{\pgfqpoint{5.515596in}{3.382832in}}{\pgfqpoint{5.504545in}{3.382832in}}%
\pgfpathcurveto{\pgfqpoint{5.493495in}{3.382832in}}{\pgfqpoint{5.482896in}{3.378442in}}{\pgfqpoint{5.475083in}{3.370628in}}%
\pgfpathcurveto{\pgfqpoint{5.467269in}{3.362814in}}{\pgfqpoint{5.462879in}{3.352215in}}{\pgfqpoint{5.462879in}{3.341165in}}%
\pgfpathcurveto{\pgfqpoint{5.462879in}{3.330115in}}{\pgfqpoint{5.467269in}{3.319516in}}{\pgfqpoint{5.475083in}{3.311702in}}%
\pgfpathcurveto{\pgfqpoint{5.482896in}{3.303889in}}{\pgfqpoint{5.493495in}{3.299499in}}{\pgfqpoint{5.504545in}{3.299499in}}%
\pgfpathclose%
\pgfusepath{stroke,fill}%
\end{pgfscope}%
\begin{pgfscope}%
\pgfpathrectangle{\pgfqpoint{0.800000in}{0.528000in}}{\pgfqpoint{4.960000in}{3.696000in}}%
\pgfusepath{clip}%
\pgfsetbuttcap%
\pgfsetroundjoin%
\definecolor{currentfill}{rgb}{0.000000,0.000000,0.000000}%
\pgfsetfillcolor{currentfill}%
\pgfsetlinewidth{1.003750pt}%
\definecolor{currentstroke}{rgb}{0.000000,0.000000,0.000000}%
\pgfsetstrokecolor{currentstroke}%
\pgfsetdash{}{0pt}%
\pgfpathmoveto{\pgfqpoint{5.504545in}{3.772443in}}%
\pgfpathcurveto{\pgfqpoint{5.515596in}{3.772443in}}{\pgfqpoint{5.526195in}{3.776834in}}{\pgfqpoint{5.534008in}{3.784647in}}%
\pgfpathcurveto{\pgfqpoint{5.541822in}{3.792461in}}{\pgfqpoint{5.546212in}{3.803060in}}{\pgfqpoint{5.546212in}{3.814110in}}%
\pgfpathcurveto{\pgfqpoint{5.546212in}{3.825160in}}{\pgfqpoint{5.541822in}{3.835759in}}{\pgfqpoint{5.534008in}{3.843573in}}%
\pgfpathcurveto{\pgfqpoint{5.526195in}{3.851386in}}{\pgfqpoint{5.515596in}{3.855777in}}{\pgfqpoint{5.504545in}{3.855777in}}%
\pgfpathcurveto{\pgfqpoint{5.493495in}{3.855777in}}{\pgfqpoint{5.482896in}{3.851386in}}{\pgfqpoint{5.475083in}{3.843573in}}%
\pgfpathcurveto{\pgfqpoint{5.467269in}{3.835759in}}{\pgfqpoint{5.462879in}{3.825160in}}{\pgfqpoint{5.462879in}{3.814110in}}%
\pgfpathcurveto{\pgfqpoint{5.462879in}{3.803060in}}{\pgfqpoint{5.467269in}{3.792461in}}{\pgfqpoint{5.475083in}{3.784647in}}%
\pgfpathcurveto{\pgfqpoint{5.482896in}{3.776834in}}{\pgfqpoint{5.493495in}{3.772443in}}{\pgfqpoint{5.504545in}{3.772443in}}%
\pgfpathclose%
\pgfusepath{stroke,fill}%
\end{pgfscope}%
\begin{pgfscope}%
\pgfpathrectangle{\pgfqpoint{0.800000in}{0.528000in}}{\pgfqpoint{4.960000in}{3.696000in}}%
\pgfusepath{clip}%
\pgfsetbuttcap%
\pgfsetroundjoin%
\definecolor{currentfill}{rgb}{0.000000,0.000000,0.000000}%
\pgfsetfillcolor{currentfill}%
\pgfsetlinewidth{1.003750pt}%
\definecolor{currentstroke}{rgb}{0.000000,0.000000,0.000000}%
\pgfsetstrokecolor{currentstroke}%
\pgfsetdash{}{0pt}%
\pgfpathmoveto{\pgfqpoint{5.504545in}{2.719066in}}%
\pgfpathcurveto{\pgfqpoint{5.515596in}{2.719066in}}{\pgfqpoint{5.526195in}{2.723456in}}{\pgfqpoint{5.534008in}{2.731270in}}%
\pgfpathcurveto{\pgfqpoint{5.541822in}{2.739084in}}{\pgfqpoint{5.546212in}{2.749683in}}{\pgfqpoint{5.546212in}{2.760733in}}%
\pgfpathcurveto{\pgfqpoint{5.546212in}{2.771783in}}{\pgfqpoint{5.541822in}{2.782382in}}{\pgfqpoint{5.534008in}{2.790196in}}%
\pgfpathcurveto{\pgfqpoint{5.526195in}{2.798009in}}{\pgfqpoint{5.515596in}{2.802400in}}{\pgfqpoint{5.504545in}{2.802400in}}%
\pgfpathcurveto{\pgfqpoint{5.493495in}{2.802400in}}{\pgfqpoint{5.482896in}{2.798009in}}{\pgfqpoint{5.475083in}{2.790196in}}%
\pgfpathcurveto{\pgfqpoint{5.467269in}{2.782382in}}{\pgfqpoint{5.462879in}{2.771783in}}{\pgfqpoint{5.462879in}{2.760733in}}%
\pgfpathcurveto{\pgfqpoint{5.462879in}{2.749683in}}{\pgfqpoint{5.467269in}{2.739084in}}{\pgfqpoint{5.475083in}{2.731270in}}%
\pgfpathcurveto{\pgfqpoint{5.482896in}{2.723456in}}{\pgfqpoint{5.493495in}{2.719066in}}{\pgfqpoint{5.504545in}{2.719066in}}%
\pgfpathclose%
\pgfusepath{stroke,fill}%
\end{pgfscope}%
\begin{pgfscope}%
\pgfpathrectangle{\pgfqpoint{0.800000in}{0.528000in}}{\pgfqpoint{4.960000in}{3.696000in}}%
\pgfusepath{clip}%
\pgfsetbuttcap%
\pgfsetroundjoin%
\definecolor{currentfill}{rgb}{0.000000,0.000000,0.000000}%
\pgfsetfillcolor{currentfill}%
\pgfsetlinewidth{1.003750pt}%
\definecolor{currentstroke}{rgb}{0.000000,0.000000,0.000000}%
\pgfsetstrokecolor{currentstroke}%
\pgfsetdash{}{0pt}%
\pgfpathmoveto{\pgfqpoint{5.504545in}{2.719066in}}%
\pgfpathcurveto{\pgfqpoint{5.515596in}{2.719066in}}{\pgfqpoint{5.526195in}{2.723456in}}{\pgfqpoint{5.534008in}{2.731270in}}%
\pgfpathcurveto{\pgfqpoint{5.541822in}{2.739084in}}{\pgfqpoint{5.546212in}{2.749683in}}{\pgfqpoint{5.546212in}{2.760733in}}%
\pgfpathcurveto{\pgfqpoint{5.546212in}{2.771783in}}{\pgfqpoint{5.541822in}{2.782382in}}{\pgfqpoint{5.534008in}{2.790196in}}%
\pgfpathcurveto{\pgfqpoint{5.526195in}{2.798009in}}{\pgfqpoint{5.515596in}{2.802400in}}{\pgfqpoint{5.504545in}{2.802400in}}%
\pgfpathcurveto{\pgfqpoint{5.493495in}{2.802400in}}{\pgfqpoint{5.482896in}{2.798009in}}{\pgfqpoint{5.475083in}{2.790196in}}%
\pgfpathcurveto{\pgfqpoint{5.467269in}{2.782382in}}{\pgfqpoint{5.462879in}{2.771783in}}{\pgfqpoint{5.462879in}{2.760733in}}%
\pgfpathcurveto{\pgfqpoint{5.462879in}{2.749683in}}{\pgfqpoint{5.467269in}{2.739084in}}{\pgfqpoint{5.475083in}{2.731270in}}%
\pgfpathcurveto{\pgfqpoint{5.482896in}{2.723456in}}{\pgfqpoint{5.493495in}{2.719066in}}{\pgfqpoint{5.504545in}{2.719066in}}%
\pgfpathclose%
\pgfusepath{stroke,fill}%
\end{pgfscope}%
\begin{pgfscope}%
\pgfpathrectangle{\pgfqpoint{0.800000in}{0.528000in}}{\pgfqpoint{4.960000in}{3.696000in}}%
\pgfusepath{clip}%
\pgfsetbuttcap%
\pgfsetroundjoin%
\definecolor{currentfill}{rgb}{0.000000,0.000000,0.000000}%
\pgfsetfillcolor{currentfill}%
\pgfsetlinewidth{1.003750pt}%
\definecolor{currentstroke}{rgb}{0.000000,0.000000,0.000000}%
\pgfsetstrokecolor{currentstroke}%
\pgfsetdash{}{0pt}%
\pgfpathmoveto{\pgfqpoint{5.504545in}{3.063026in}}%
\pgfpathcurveto{\pgfqpoint{5.515596in}{3.063026in}}{\pgfqpoint{5.526195in}{3.067416in}}{\pgfqpoint{5.534008in}{3.075230in}}%
\pgfpathcurveto{\pgfqpoint{5.541822in}{3.083044in}}{\pgfqpoint{5.546212in}{3.093643in}}{\pgfqpoint{5.546212in}{3.104693in}}%
\pgfpathcurveto{\pgfqpoint{5.546212in}{3.115743in}}{\pgfqpoint{5.541822in}{3.126342in}}{\pgfqpoint{5.534008in}{3.134156in}}%
\pgfpathcurveto{\pgfqpoint{5.526195in}{3.141969in}}{\pgfqpoint{5.515596in}{3.146359in}}{\pgfqpoint{5.504545in}{3.146359in}}%
\pgfpathcurveto{\pgfqpoint{5.493495in}{3.146359in}}{\pgfqpoint{5.482896in}{3.141969in}}{\pgfqpoint{5.475083in}{3.134156in}}%
\pgfpathcurveto{\pgfqpoint{5.467269in}{3.126342in}}{\pgfqpoint{5.462879in}{3.115743in}}{\pgfqpoint{5.462879in}{3.104693in}}%
\pgfpathcurveto{\pgfqpoint{5.462879in}{3.093643in}}{\pgfqpoint{5.467269in}{3.083044in}}{\pgfqpoint{5.475083in}{3.075230in}}%
\pgfpathcurveto{\pgfqpoint{5.482896in}{3.067416in}}{\pgfqpoint{5.493495in}{3.063026in}}{\pgfqpoint{5.504545in}{3.063026in}}%
\pgfpathclose%
\pgfusepath{stroke,fill}%
\end{pgfscope}%
\begin{pgfscope}%
\pgfpathrectangle{\pgfqpoint{0.800000in}{0.528000in}}{\pgfqpoint{4.960000in}{3.696000in}}%
\pgfusepath{clip}%
\pgfsetbuttcap%
\pgfsetroundjoin%
\definecolor{currentfill}{rgb}{0.000000,0.000000,0.000000}%
\pgfsetfillcolor{currentfill}%
\pgfsetlinewidth{1.003750pt}%
\definecolor{currentstroke}{rgb}{0.000000,0.000000,0.000000}%
\pgfsetstrokecolor{currentstroke}%
\pgfsetdash{}{0pt}%
\pgfpathmoveto{\pgfqpoint{5.504545in}{2.762061in}}%
\pgfpathcurveto{\pgfqpoint{5.515596in}{2.762061in}}{\pgfqpoint{5.526195in}{2.766451in}}{\pgfqpoint{5.534008in}{2.774265in}}%
\pgfpathcurveto{\pgfqpoint{5.541822in}{2.782079in}}{\pgfqpoint{5.546212in}{2.792678in}}{\pgfqpoint{5.546212in}{2.803728in}}%
\pgfpathcurveto{\pgfqpoint{5.546212in}{2.814778in}}{\pgfqpoint{5.541822in}{2.825377in}}{\pgfqpoint{5.534008in}{2.833191in}}%
\pgfpathcurveto{\pgfqpoint{5.526195in}{2.841004in}}{\pgfqpoint{5.515596in}{2.845395in}}{\pgfqpoint{5.504545in}{2.845395in}}%
\pgfpathcurveto{\pgfqpoint{5.493495in}{2.845395in}}{\pgfqpoint{5.482896in}{2.841004in}}{\pgfqpoint{5.475083in}{2.833191in}}%
\pgfpathcurveto{\pgfqpoint{5.467269in}{2.825377in}}{\pgfqpoint{5.462879in}{2.814778in}}{\pgfqpoint{5.462879in}{2.803728in}}%
\pgfpathcurveto{\pgfqpoint{5.462879in}{2.792678in}}{\pgfqpoint{5.467269in}{2.782079in}}{\pgfqpoint{5.475083in}{2.774265in}}%
\pgfpathcurveto{\pgfqpoint{5.482896in}{2.766451in}}{\pgfqpoint{5.493495in}{2.762061in}}{\pgfqpoint{5.504545in}{2.762061in}}%
\pgfpathclose%
\pgfusepath{stroke,fill}%
\end{pgfscope}%
\begin{pgfscope}%
\pgfpathrectangle{\pgfqpoint{0.800000in}{0.528000in}}{\pgfqpoint{4.960000in}{3.696000in}}%
\pgfusepath{clip}%
\pgfsetbuttcap%
\pgfsetroundjoin%
\definecolor{currentfill}{rgb}{0.000000,0.000000,0.000000}%
\pgfsetfillcolor{currentfill}%
\pgfsetlinewidth{1.003750pt}%
\definecolor{currentstroke}{rgb}{0.000000,0.000000,0.000000}%
\pgfsetstrokecolor{currentstroke}%
\pgfsetdash{}{0pt}%
\pgfpathmoveto{\pgfqpoint{5.504545in}{2.740564in}}%
\pgfpathcurveto{\pgfqpoint{5.515596in}{2.740564in}}{\pgfqpoint{5.526195in}{2.744954in}}{\pgfqpoint{5.534008in}{2.752768in}}%
\pgfpathcurveto{\pgfqpoint{5.541822in}{2.760581in}}{\pgfqpoint{5.546212in}{2.771180in}}{\pgfqpoint{5.546212in}{2.782230in}}%
\pgfpathcurveto{\pgfqpoint{5.546212in}{2.793281in}}{\pgfqpoint{5.541822in}{2.803880in}}{\pgfqpoint{5.534008in}{2.811693in}}%
\pgfpathcurveto{\pgfqpoint{5.526195in}{2.819507in}}{\pgfqpoint{5.515596in}{2.823897in}}{\pgfqpoint{5.504545in}{2.823897in}}%
\pgfpathcurveto{\pgfqpoint{5.493495in}{2.823897in}}{\pgfqpoint{5.482896in}{2.819507in}}{\pgfqpoint{5.475083in}{2.811693in}}%
\pgfpathcurveto{\pgfqpoint{5.467269in}{2.803880in}}{\pgfqpoint{5.462879in}{2.793281in}}{\pgfqpoint{5.462879in}{2.782230in}}%
\pgfpathcurveto{\pgfqpoint{5.462879in}{2.771180in}}{\pgfqpoint{5.467269in}{2.760581in}}{\pgfqpoint{5.475083in}{2.752768in}}%
\pgfpathcurveto{\pgfqpoint{5.482896in}{2.744954in}}{\pgfqpoint{5.493495in}{2.740564in}}{\pgfqpoint{5.504545in}{2.740564in}}%
\pgfpathclose%
\pgfusepath{stroke,fill}%
\end{pgfscope}%
\begin{pgfscope}%
\pgfpathrectangle{\pgfqpoint{0.800000in}{0.528000in}}{\pgfqpoint{4.960000in}{3.696000in}}%
\pgfusepath{clip}%
\pgfsetbuttcap%
\pgfsetroundjoin%
\definecolor{currentfill}{rgb}{0.000000,0.000000,0.000000}%
\pgfsetfillcolor{currentfill}%
\pgfsetlinewidth{1.003750pt}%
\definecolor{currentstroke}{rgb}{0.000000,0.000000,0.000000}%
\pgfsetstrokecolor{currentstroke}%
\pgfsetdash{}{0pt}%
\pgfpathmoveto{\pgfqpoint{5.504545in}{3.192011in}}%
\pgfpathcurveto{\pgfqpoint{5.515596in}{3.192011in}}{\pgfqpoint{5.526195in}{3.196401in}}{\pgfqpoint{5.534008in}{3.204215in}}%
\pgfpathcurveto{\pgfqpoint{5.541822in}{3.212029in}}{\pgfqpoint{5.546212in}{3.222628in}}{\pgfqpoint{5.546212in}{3.233678in}}%
\pgfpathcurveto{\pgfqpoint{5.546212in}{3.244728in}}{\pgfqpoint{5.541822in}{3.255327in}}{\pgfqpoint{5.534008in}{3.263141in}}%
\pgfpathcurveto{\pgfqpoint{5.526195in}{3.270954in}}{\pgfqpoint{5.515596in}{3.275344in}}{\pgfqpoint{5.504545in}{3.275344in}}%
\pgfpathcurveto{\pgfqpoint{5.493495in}{3.275344in}}{\pgfqpoint{5.482896in}{3.270954in}}{\pgfqpoint{5.475083in}{3.263141in}}%
\pgfpathcurveto{\pgfqpoint{5.467269in}{3.255327in}}{\pgfqpoint{5.462879in}{3.244728in}}{\pgfqpoint{5.462879in}{3.233678in}}%
\pgfpathcurveto{\pgfqpoint{5.462879in}{3.222628in}}{\pgfqpoint{5.467269in}{3.212029in}}{\pgfqpoint{5.475083in}{3.204215in}}%
\pgfpathcurveto{\pgfqpoint{5.482896in}{3.196401in}}{\pgfqpoint{5.493495in}{3.192011in}}{\pgfqpoint{5.504545in}{3.192011in}}%
\pgfpathclose%
\pgfusepath{stroke,fill}%
\end{pgfscope}%
\begin{pgfscope}%
\pgfpathrectangle{\pgfqpoint{0.800000in}{0.528000in}}{\pgfqpoint{4.960000in}{3.696000in}}%
\pgfusepath{clip}%
\pgfsetbuttcap%
\pgfsetroundjoin%
\definecolor{currentfill}{rgb}{0.000000,0.000000,0.000000}%
\pgfsetfillcolor{currentfill}%
\pgfsetlinewidth{1.003750pt}%
\definecolor{currentstroke}{rgb}{0.000000,0.000000,0.000000}%
\pgfsetstrokecolor{currentstroke}%
\pgfsetdash{}{0pt}%
\pgfpathmoveto{\pgfqpoint{5.504545in}{2.805056in}}%
\pgfpathcurveto{\pgfqpoint{5.515596in}{2.805056in}}{\pgfqpoint{5.526195in}{2.809446in}}{\pgfqpoint{5.534008in}{2.817260in}}%
\pgfpathcurveto{\pgfqpoint{5.541822in}{2.825074in}}{\pgfqpoint{5.546212in}{2.835673in}}{\pgfqpoint{5.546212in}{2.846723in}}%
\pgfpathcurveto{\pgfqpoint{5.546212in}{2.857773in}}{\pgfqpoint{5.541822in}{2.868372in}}{\pgfqpoint{5.534008in}{2.876186in}}%
\pgfpathcurveto{\pgfqpoint{5.526195in}{2.883999in}}{\pgfqpoint{5.515596in}{2.888390in}}{\pgfqpoint{5.504545in}{2.888390in}}%
\pgfpathcurveto{\pgfqpoint{5.493495in}{2.888390in}}{\pgfqpoint{5.482896in}{2.883999in}}{\pgfqpoint{5.475083in}{2.876186in}}%
\pgfpathcurveto{\pgfqpoint{5.467269in}{2.868372in}}{\pgfqpoint{5.462879in}{2.857773in}}{\pgfqpoint{5.462879in}{2.846723in}}%
\pgfpathcurveto{\pgfqpoint{5.462879in}{2.835673in}}{\pgfqpoint{5.467269in}{2.825074in}}{\pgfqpoint{5.475083in}{2.817260in}}%
\pgfpathcurveto{\pgfqpoint{5.482896in}{2.809446in}}{\pgfqpoint{5.493495in}{2.805056in}}{\pgfqpoint{5.504545in}{2.805056in}}%
\pgfpathclose%
\pgfusepath{stroke,fill}%
\end{pgfscope}%
\begin{pgfscope}%
\pgfpathrectangle{\pgfqpoint{0.800000in}{0.528000in}}{\pgfqpoint{4.960000in}{3.696000in}}%
\pgfusepath{clip}%
\pgfsetbuttcap%
\pgfsetroundjoin%
\definecolor{currentfill}{rgb}{0.000000,0.000000,0.000000}%
\pgfsetfillcolor{currentfill}%
\pgfsetlinewidth{1.003750pt}%
\definecolor{currentstroke}{rgb}{0.000000,0.000000,0.000000}%
\pgfsetstrokecolor{currentstroke}%
\pgfsetdash{}{0pt}%
\pgfpathmoveto{\pgfqpoint{5.504545in}{3.664956in}}%
\pgfpathcurveto{\pgfqpoint{5.515596in}{3.664956in}}{\pgfqpoint{5.526195in}{3.669346in}}{\pgfqpoint{5.534008in}{3.677160in}}%
\pgfpathcurveto{\pgfqpoint{5.541822in}{3.684973in}}{\pgfqpoint{5.546212in}{3.695572in}}{\pgfqpoint{5.546212in}{3.706623in}}%
\pgfpathcurveto{\pgfqpoint{5.546212in}{3.717673in}}{\pgfqpoint{5.541822in}{3.728272in}}{\pgfqpoint{5.534008in}{3.736085in}}%
\pgfpathcurveto{\pgfqpoint{5.526195in}{3.743899in}}{\pgfqpoint{5.515596in}{3.748289in}}{\pgfqpoint{5.504545in}{3.748289in}}%
\pgfpathcurveto{\pgfqpoint{5.493495in}{3.748289in}}{\pgfqpoint{5.482896in}{3.743899in}}{\pgfqpoint{5.475083in}{3.736085in}}%
\pgfpathcurveto{\pgfqpoint{5.467269in}{3.728272in}}{\pgfqpoint{5.462879in}{3.717673in}}{\pgfqpoint{5.462879in}{3.706623in}}%
\pgfpathcurveto{\pgfqpoint{5.462879in}{3.695572in}}{\pgfqpoint{5.467269in}{3.684973in}}{\pgfqpoint{5.475083in}{3.677160in}}%
\pgfpathcurveto{\pgfqpoint{5.482896in}{3.669346in}}{\pgfqpoint{5.493495in}{3.664956in}}{\pgfqpoint{5.504545in}{3.664956in}}%
\pgfpathclose%
\pgfusepath{stroke,fill}%
\end{pgfscope}%
\begin{pgfscope}%
\pgfsetbuttcap%
\pgfsetroundjoin%
\definecolor{currentfill}{rgb}{0.000000,0.000000,0.000000}%
\pgfsetfillcolor{currentfill}%
\pgfsetlinewidth{0.803000pt}%
\definecolor{currentstroke}{rgb}{0.000000,0.000000,0.000000}%
\pgfsetstrokecolor{currentstroke}%
\pgfsetdash{}{0pt}%
\pgfsys@defobject{currentmarker}{\pgfqpoint{0.000000in}{-0.048611in}}{\pgfqpoint{0.000000in}{0.000000in}}{%
\pgfpathmoveto{\pgfqpoint{0.000000in}{0.000000in}}%
\pgfpathlineto{\pgfqpoint{0.000000in}{-0.048611in}}%
\pgfusepath{stroke,fill}%
}%
\begin{pgfscope}%
\pgfsys@transformshift{1.025906in}{0.528000in}%
\pgfsys@useobject{currentmarker}{}%
\end{pgfscope}%
\end{pgfscope}%
\begin{pgfscope}%
\definecolor{textcolor}{rgb}{0.000000,0.000000,0.000000}%
\pgfsetstrokecolor{textcolor}%
\pgfsetfillcolor{textcolor}%
\pgftext[x=1.025906in,y=0.430778in,,top]{\color{textcolor}\sffamily\fontsize{10.000000}{12.000000}\selectfont 20}%
\end{pgfscope}%
\begin{pgfscope}%
\pgfsetbuttcap%
\pgfsetroundjoin%
\definecolor{currentfill}{rgb}{0.000000,0.000000,0.000000}%
\pgfsetfillcolor{currentfill}%
\pgfsetlinewidth{0.803000pt}%
\definecolor{currentstroke}{rgb}{0.000000,0.000000,0.000000}%
\pgfsetstrokecolor{currentstroke}%
\pgfsetdash{}{0pt}%
\pgfsys@defobject{currentmarker}{\pgfqpoint{0.000000in}{-0.048611in}}{\pgfqpoint{0.000000in}{0.000000in}}{%
\pgfpathmoveto{\pgfqpoint{0.000000in}{0.000000in}}%
\pgfpathlineto{\pgfqpoint{0.000000in}{-0.048611in}}%
\pgfusepath{stroke,fill}%
}%
\begin{pgfscope}%
\pgfsys@transformshift{2.518786in}{0.528000in}%
\pgfsys@useobject{currentmarker}{}%
\end{pgfscope}%
\end{pgfscope}%
\begin{pgfscope}%
\definecolor{textcolor}{rgb}{0.000000,0.000000,0.000000}%
\pgfsetstrokecolor{textcolor}%
\pgfsetfillcolor{textcolor}%
\pgftext[x=2.518786in,y=0.430778in,,top]{\color{textcolor}\sffamily\fontsize{10.000000}{12.000000}\selectfont 40}%
\end{pgfscope}%
\begin{pgfscope}%
\pgfsetbuttcap%
\pgfsetroundjoin%
\definecolor{currentfill}{rgb}{0.000000,0.000000,0.000000}%
\pgfsetfillcolor{currentfill}%
\pgfsetlinewidth{0.803000pt}%
\definecolor{currentstroke}{rgb}{0.000000,0.000000,0.000000}%
\pgfsetstrokecolor{currentstroke}%
\pgfsetdash{}{0pt}%
\pgfsys@defobject{currentmarker}{\pgfqpoint{0.000000in}{-0.048611in}}{\pgfqpoint{0.000000in}{0.000000in}}{%
\pgfpathmoveto{\pgfqpoint{0.000000in}{0.000000in}}%
\pgfpathlineto{\pgfqpoint{0.000000in}{-0.048611in}}%
\pgfusepath{stroke,fill}%
}%
\begin{pgfscope}%
\pgfsys@transformshift{4.011666in}{0.528000in}%
\pgfsys@useobject{currentmarker}{}%
\end{pgfscope}%
\end{pgfscope}%
\begin{pgfscope}%
\definecolor{textcolor}{rgb}{0.000000,0.000000,0.000000}%
\pgfsetstrokecolor{textcolor}%
\pgfsetfillcolor{textcolor}%
\pgftext[x=4.011666in,y=0.430778in,,top]{\color{textcolor}\sffamily\fontsize{10.000000}{12.000000}\selectfont 60}%
\end{pgfscope}%
\begin{pgfscope}%
\pgfsetbuttcap%
\pgfsetroundjoin%
\definecolor{currentfill}{rgb}{0.000000,0.000000,0.000000}%
\pgfsetfillcolor{currentfill}%
\pgfsetlinewidth{0.803000pt}%
\definecolor{currentstroke}{rgb}{0.000000,0.000000,0.000000}%
\pgfsetstrokecolor{currentstroke}%
\pgfsetdash{}{0pt}%
\pgfsys@defobject{currentmarker}{\pgfqpoint{0.000000in}{-0.048611in}}{\pgfqpoint{0.000000in}{0.000000in}}{%
\pgfpathmoveto{\pgfqpoint{0.000000in}{0.000000in}}%
\pgfpathlineto{\pgfqpoint{0.000000in}{-0.048611in}}%
\pgfusepath{stroke,fill}%
}%
\begin{pgfscope}%
\pgfsys@transformshift{5.504545in}{0.528000in}%
\pgfsys@useobject{currentmarker}{}%
\end{pgfscope}%
\end{pgfscope}%
\begin{pgfscope}%
\definecolor{textcolor}{rgb}{0.000000,0.000000,0.000000}%
\pgfsetstrokecolor{textcolor}%
\pgfsetfillcolor{textcolor}%
\pgftext[x=5.504545in,y=0.430778in,,top]{\color{textcolor}\sffamily\fontsize{10.000000}{12.000000}\selectfont 80}%
\end{pgfscope}%
\begin{pgfscope}%
\definecolor{textcolor}{rgb}{0.000000,0.000000,0.000000}%
\pgfsetstrokecolor{textcolor}%
\pgfsetfillcolor{textcolor}%
\pgftext[x=3.280000in,y=0.240809in,,top]{\color{textcolor}\sffamily\fontsize{10.000000}{12.000000}\selectfont \(\displaystyle k\)}%
\end{pgfscope}%
\begin{pgfscope}%
\pgfsetbuttcap%
\pgfsetroundjoin%
\definecolor{currentfill}{rgb}{0.000000,0.000000,0.000000}%
\pgfsetfillcolor{currentfill}%
\pgfsetlinewidth{0.803000pt}%
\definecolor{currentstroke}{rgb}{0.000000,0.000000,0.000000}%
\pgfsetstrokecolor{currentstroke}%
\pgfsetdash{}{0pt}%
\pgfsys@defobject{currentmarker}{\pgfqpoint{-0.048611in}{0.000000in}}{\pgfqpoint{0.000000in}{0.000000in}}{%
\pgfpathmoveto{\pgfqpoint{0.000000in}{0.000000in}}%
\pgfpathlineto{\pgfqpoint{-0.048611in}{0.000000in}}%
\pgfusepath{stroke,fill}%
}%
\begin{pgfscope}%
\pgfsys@transformshift{0.800000in}{0.675476in}%
\pgfsys@useobject{currentmarker}{}%
\end{pgfscope}%
\end{pgfscope}%
\begin{pgfscope}%
\definecolor{textcolor}{rgb}{0.000000,0.000000,0.000000}%
\pgfsetstrokecolor{textcolor}%
\pgfsetfillcolor{textcolor}%
\pgftext[x=0.526047in,y=0.622715in,left,base]{\color{textcolor}\sffamily\fontsize{10.000000}{12.000000}\selectfont 20}%
\end{pgfscope}%
\begin{pgfscope}%
\pgfsetbuttcap%
\pgfsetroundjoin%
\definecolor{currentfill}{rgb}{0.000000,0.000000,0.000000}%
\pgfsetfillcolor{currentfill}%
\pgfsetlinewidth{0.803000pt}%
\definecolor{currentstroke}{rgb}{0.000000,0.000000,0.000000}%
\pgfsetstrokecolor{currentstroke}%
\pgfsetdash{}{0pt}%
\pgfsys@defobject{currentmarker}{\pgfqpoint{-0.048611in}{0.000000in}}{\pgfqpoint{0.000000in}{0.000000in}}{%
\pgfpathmoveto{\pgfqpoint{0.000000in}{0.000000in}}%
\pgfpathlineto{\pgfqpoint{-0.048611in}{0.000000in}}%
\pgfusepath{stroke,fill}%
}%
\begin{pgfscope}%
\pgfsys@transformshift{0.800000in}{1.105426in}%
\pgfsys@useobject{currentmarker}{}%
\end{pgfscope}%
\end{pgfscope}%
\begin{pgfscope}%
\definecolor{textcolor}{rgb}{0.000000,0.000000,0.000000}%
\pgfsetstrokecolor{textcolor}%
\pgfsetfillcolor{textcolor}%
\pgftext[x=0.526047in,y=1.052664in,left,base]{\color{textcolor}\sffamily\fontsize{10.000000}{12.000000}\selectfont 40}%
\end{pgfscope}%
\begin{pgfscope}%
\pgfsetbuttcap%
\pgfsetroundjoin%
\definecolor{currentfill}{rgb}{0.000000,0.000000,0.000000}%
\pgfsetfillcolor{currentfill}%
\pgfsetlinewidth{0.803000pt}%
\definecolor{currentstroke}{rgb}{0.000000,0.000000,0.000000}%
\pgfsetstrokecolor{currentstroke}%
\pgfsetdash{}{0pt}%
\pgfsys@defobject{currentmarker}{\pgfqpoint{-0.048611in}{0.000000in}}{\pgfqpoint{0.000000in}{0.000000in}}{%
\pgfpathmoveto{\pgfqpoint{0.000000in}{0.000000in}}%
\pgfpathlineto{\pgfqpoint{-0.048611in}{0.000000in}}%
\pgfusepath{stroke,fill}%
}%
\begin{pgfscope}%
\pgfsys@transformshift{0.800000in}{1.535376in}%
\pgfsys@useobject{currentmarker}{}%
\end{pgfscope}%
\end{pgfscope}%
\begin{pgfscope}%
\definecolor{textcolor}{rgb}{0.000000,0.000000,0.000000}%
\pgfsetstrokecolor{textcolor}%
\pgfsetfillcolor{textcolor}%
\pgftext[x=0.526047in,y=1.482614in,left,base]{\color{textcolor}\sffamily\fontsize{10.000000}{12.000000}\selectfont 60}%
\end{pgfscope}%
\begin{pgfscope}%
\pgfsetbuttcap%
\pgfsetroundjoin%
\definecolor{currentfill}{rgb}{0.000000,0.000000,0.000000}%
\pgfsetfillcolor{currentfill}%
\pgfsetlinewidth{0.803000pt}%
\definecolor{currentstroke}{rgb}{0.000000,0.000000,0.000000}%
\pgfsetstrokecolor{currentstroke}%
\pgfsetdash{}{0pt}%
\pgfsys@defobject{currentmarker}{\pgfqpoint{-0.048611in}{0.000000in}}{\pgfqpoint{0.000000in}{0.000000in}}{%
\pgfpathmoveto{\pgfqpoint{0.000000in}{0.000000in}}%
\pgfpathlineto{\pgfqpoint{-0.048611in}{0.000000in}}%
\pgfusepath{stroke,fill}%
}%
\begin{pgfscope}%
\pgfsys@transformshift{0.800000in}{1.965326in}%
\pgfsys@useobject{currentmarker}{}%
\end{pgfscope}%
\end{pgfscope}%
\begin{pgfscope}%
\definecolor{textcolor}{rgb}{0.000000,0.000000,0.000000}%
\pgfsetstrokecolor{textcolor}%
\pgfsetfillcolor{textcolor}%
\pgftext[x=0.526047in,y=1.912564in,left,base]{\color{textcolor}\sffamily\fontsize{10.000000}{12.000000}\selectfont 80}%
\end{pgfscope}%
\begin{pgfscope}%
\pgfsetbuttcap%
\pgfsetroundjoin%
\definecolor{currentfill}{rgb}{0.000000,0.000000,0.000000}%
\pgfsetfillcolor{currentfill}%
\pgfsetlinewidth{0.803000pt}%
\definecolor{currentstroke}{rgb}{0.000000,0.000000,0.000000}%
\pgfsetstrokecolor{currentstroke}%
\pgfsetdash{}{0pt}%
\pgfsys@defobject{currentmarker}{\pgfqpoint{-0.048611in}{0.000000in}}{\pgfqpoint{0.000000in}{0.000000in}}{%
\pgfpathmoveto{\pgfqpoint{0.000000in}{0.000000in}}%
\pgfpathlineto{\pgfqpoint{-0.048611in}{0.000000in}}%
\pgfusepath{stroke,fill}%
}%
\begin{pgfscope}%
\pgfsys@transformshift{0.800000in}{2.395276in}%
\pgfsys@useobject{currentmarker}{}%
\end{pgfscope}%
\end{pgfscope}%
\begin{pgfscope}%
\definecolor{textcolor}{rgb}{0.000000,0.000000,0.000000}%
\pgfsetstrokecolor{textcolor}%
\pgfsetfillcolor{textcolor}%
\pgftext[x=0.437682in,y=2.342514in,left,base]{\color{textcolor}\sffamily\fontsize{10.000000}{12.000000}\selectfont 100}%
\end{pgfscope}%
\begin{pgfscope}%
\pgfsetbuttcap%
\pgfsetroundjoin%
\definecolor{currentfill}{rgb}{0.000000,0.000000,0.000000}%
\pgfsetfillcolor{currentfill}%
\pgfsetlinewidth{0.803000pt}%
\definecolor{currentstroke}{rgb}{0.000000,0.000000,0.000000}%
\pgfsetstrokecolor{currentstroke}%
\pgfsetdash{}{0pt}%
\pgfsys@defobject{currentmarker}{\pgfqpoint{-0.048611in}{0.000000in}}{\pgfqpoint{0.000000in}{0.000000in}}{%
\pgfpathmoveto{\pgfqpoint{0.000000in}{0.000000in}}%
\pgfpathlineto{\pgfqpoint{-0.048611in}{0.000000in}}%
\pgfusepath{stroke,fill}%
}%
\begin{pgfscope}%
\pgfsys@transformshift{0.800000in}{2.825225in}%
\pgfsys@useobject{currentmarker}{}%
\end{pgfscope}%
\end{pgfscope}%
\begin{pgfscope}%
\definecolor{textcolor}{rgb}{0.000000,0.000000,0.000000}%
\pgfsetstrokecolor{textcolor}%
\pgfsetfillcolor{textcolor}%
\pgftext[x=0.437682in,y=2.772464in,left,base]{\color{textcolor}\sffamily\fontsize{10.000000}{12.000000}\selectfont 120}%
\end{pgfscope}%
\begin{pgfscope}%
\pgfsetbuttcap%
\pgfsetroundjoin%
\definecolor{currentfill}{rgb}{0.000000,0.000000,0.000000}%
\pgfsetfillcolor{currentfill}%
\pgfsetlinewidth{0.803000pt}%
\definecolor{currentstroke}{rgb}{0.000000,0.000000,0.000000}%
\pgfsetstrokecolor{currentstroke}%
\pgfsetdash{}{0pt}%
\pgfsys@defobject{currentmarker}{\pgfqpoint{-0.048611in}{0.000000in}}{\pgfqpoint{0.000000in}{0.000000in}}{%
\pgfpathmoveto{\pgfqpoint{0.000000in}{0.000000in}}%
\pgfpathlineto{\pgfqpoint{-0.048611in}{0.000000in}}%
\pgfusepath{stroke,fill}%
}%
\begin{pgfscope}%
\pgfsys@transformshift{0.800000in}{3.255175in}%
\pgfsys@useobject{currentmarker}{}%
\end{pgfscope}%
\end{pgfscope}%
\begin{pgfscope}%
\definecolor{textcolor}{rgb}{0.000000,0.000000,0.000000}%
\pgfsetstrokecolor{textcolor}%
\pgfsetfillcolor{textcolor}%
\pgftext[x=0.437682in,y=3.202414in,left,base]{\color{textcolor}\sffamily\fontsize{10.000000}{12.000000}\selectfont 140}%
\end{pgfscope}%
\begin{pgfscope}%
\pgfsetbuttcap%
\pgfsetroundjoin%
\definecolor{currentfill}{rgb}{0.000000,0.000000,0.000000}%
\pgfsetfillcolor{currentfill}%
\pgfsetlinewidth{0.803000pt}%
\definecolor{currentstroke}{rgb}{0.000000,0.000000,0.000000}%
\pgfsetstrokecolor{currentstroke}%
\pgfsetdash{}{0pt}%
\pgfsys@defobject{currentmarker}{\pgfqpoint{-0.048611in}{0.000000in}}{\pgfqpoint{0.000000in}{0.000000in}}{%
\pgfpathmoveto{\pgfqpoint{0.000000in}{0.000000in}}%
\pgfpathlineto{\pgfqpoint{-0.048611in}{0.000000in}}%
\pgfusepath{stroke,fill}%
}%
\begin{pgfscope}%
\pgfsys@transformshift{0.800000in}{3.685125in}%
\pgfsys@useobject{currentmarker}{}%
\end{pgfscope}%
\end{pgfscope}%
\begin{pgfscope}%
\definecolor{textcolor}{rgb}{0.000000,0.000000,0.000000}%
\pgfsetstrokecolor{textcolor}%
\pgfsetfillcolor{textcolor}%
\pgftext[x=0.437682in,y=3.632364in,left,base]{\color{textcolor}\sffamily\fontsize{10.000000}{12.000000}\selectfont 160}%
\end{pgfscope}%
\begin{pgfscope}%
\pgfsetbuttcap%
\pgfsetroundjoin%
\definecolor{currentfill}{rgb}{0.000000,0.000000,0.000000}%
\pgfsetfillcolor{currentfill}%
\pgfsetlinewidth{0.803000pt}%
\definecolor{currentstroke}{rgb}{0.000000,0.000000,0.000000}%
\pgfsetstrokecolor{currentstroke}%
\pgfsetdash{}{0pt}%
\pgfsys@defobject{currentmarker}{\pgfqpoint{-0.048611in}{0.000000in}}{\pgfqpoint{0.000000in}{0.000000in}}{%
\pgfpathmoveto{\pgfqpoint{0.000000in}{0.000000in}}%
\pgfpathlineto{\pgfqpoint{-0.048611in}{0.000000in}}%
\pgfusepath{stroke,fill}%
}%
\begin{pgfscope}%
\pgfsys@transformshift{0.800000in}{4.115075in}%
\pgfsys@useobject{currentmarker}{}%
\end{pgfscope}%
\end{pgfscope}%
\begin{pgfscope}%
\definecolor{textcolor}{rgb}{0.000000,0.000000,0.000000}%
\pgfsetstrokecolor{textcolor}%
\pgfsetfillcolor{textcolor}%
\pgftext[x=0.437682in,y=4.062313in,left,base]{\color{textcolor}\sffamily\fontsize{10.000000}{12.000000}\selectfont 180}%
\end{pgfscope}%
\begin{pgfscope}%
\definecolor{textcolor}{rgb}{0.000000,0.000000,0.000000}%
\pgfsetstrokecolor{textcolor}%
\pgfsetfillcolor{textcolor}%
\pgftext[x=0.382126in,y=2.376000in,,bottom,rotate=90.000000]{\color{textcolor}\sffamily\fontsize{10.000000}{12.000000}\selectfont Number of GMRES Iterations}%
\end{pgfscope}%
\begin{pgfscope}%
\pgfsetrectcap%
\pgfsetmiterjoin%
\pgfsetlinewidth{0.803000pt}%
\definecolor{currentstroke}{rgb}{0.000000,0.000000,0.000000}%
\pgfsetstrokecolor{currentstroke}%
\pgfsetdash{}{0pt}%
\pgfpathmoveto{\pgfqpoint{0.800000in}{0.528000in}}%
\pgfpathlineto{\pgfqpoint{0.800000in}{4.224000in}}%
\pgfusepath{stroke}%
\end{pgfscope}%
\begin{pgfscope}%
\pgfsetrectcap%
\pgfsetmiterjoin%
\pgfsetlinewidth{0.803000pt}%
\definecolor{currentstroke}{rgb}{0.000000,0.000000,0.000000}%
\pgfsetstrokecolor{currentstroke}%
\pgfsetdash{}{0pt}%
\pgfpathmoveto{\pgfqpoint{5.760000in}{0.528000in}}%
\pgfpathlineto{\pgfqpoint{5.760000in}{4.224000in}}%
\pgfusepath{stroke}%
\end{pgfscope}%
\begin{pgfscope}%
\pgfsetrectcap%
\pgfsetmiterjoin%
\pgfsetlinewidth{0.803000pt}%
\definecolor{currentstroke}{rgb}{0.000000,0.000000,0.000000}%
\pgfsetstrokecolor{currentstroke}%
\pgfsetdash{}{0pt}%
\pgfpathmoveto{\pgfqpoint{0.800000in}{0.528000in}}%
\pgfpathlineto{\pgfqpoint{5.760000in}{0.528000in}}%
\pgfusepath{stroke}%
\end{pgfscope}%
\begin{pgfscope}%
\pgfsetrectcap%
\pgfsetmiterjoin%
\pgfsetlinewidth{0.803000pt}%
\definecolor{currentstroke}{rgb}{0.000000,0.000000,0.000000}%
\pgfsetstrokecolor{currentstroke}%
\pgfsetdash{}{0pt}%
\pgfpathmoveto{\pgfqpoint{0.800000in}{4.224000in}}%
\pgfpathlineto{\pgfqpoint{5.760000in}{4.224000in}}%
\pgfusepath{stroke}%
\end{pgfscope}%
\end{pgfpicture}%
\makeatother%
\endgroup%

\caption{GMRES iteration counts for $\alpha = 0.5$}\label{fig:linfinityA0}
    \end{subfigure}
    
    \begin{subfigure}{\textwidth}
      \centering
%% Creator: Matplotlib, PGF backend
%%
%% To include the figure in your LaTeX document, write
%%   \input{<filename>.pgf}
%%
%% Make sure the required packages are loaded in your preamble
%%   \usepackage{pgf}
%%
%% Figures using additional raster images can only be included by \input if
%% they are in the same directory as the main LaTeX file. For loading figures
%% from other directories you can use the `import` package
%%   \usepackage{import}
%% and then include the figures with
%%   \import{<path to file>}{<filename>.pgf}
%%
%% Matplotlib used the following preamble
%%   \usepackage{fontspec}
%%   \setmainfont{DejaVuSerif.ttf}[Path=/home/owen/progs/firedrake-complex/firedrake/lib/python3.5/site-packages/matplotlib/mpl-data/fonts/ttf/]
%%   \setsansfont{DejaVuSans.ttf}[Path=/home/owen/progs/firedrake-complex/firedrake/lib/python3.5/site-packages/matplotlib/mpl-data/fonts/ttf/]
%%   \setmonofont{DejaVuSansMono.ttf}[Path=/home/owen/progs/firedrake-complex/firedrake/lib/python3.5/site-packages/matplotlib/mpl-data/fonts/ttf/]
%%
\begingroup%
\makeatletter%
\begin{pgfpicture}%
\pgfpathrectangle{\pgfpointorigin}{\pgfqpoint{6.400000in}{4.800000in}}%
\pgfusepath{use as bounding box, clip}%
\begin{pgfscope}%
\pgfsetbuttcap%
\pgfsetmiterjoin%
\definecolor{currentfill}{rgb}{1.000000,1.000000,1.000000}%
\pgfsetfillcolor{currentfill}%
\pgfsetlinewidth{0.000000pt}%
\definecolor{currentstroke}{rgb}{1.000000,1.000000,1.000000}%
\pgfsetstrokecolor{currentstroke}%
\pgfsetdash{}{0pt}%
\pgfpathmoveto{\pgfqpoint{0.000000in}{0.000000in}}%
\pgfpathlineto{\pgfqpoint{6.400000in}{0.000000in}}%
\pgfpathlineto{\pgfqpoint{6.400000in}{4.800000in}}%
\pgfpathlineto{\pgfqpoint{0.000000in}{4.800000in}}%
\pgfpathclose%
\pgfusepath{fill}%
\end{pgfscope}%
\begin{pgfscope}%
\pgfsetbuttcap%
\pgfsetmiterjoin%
\definecolor{currentfill}{rgb}{1.000000,1.000000,1.000000}%
\pgfsetfillcolor{currentfill}%
\pgfsetlinewidth{0.000000pt}%
\definecolor{currentstroke}{rgb}{0.000000,0.000000,0.000000}%
\pgfsetstrokecolor{currentstroke}%
\pgfsetstrokeopacity{0.000000}%
\pgfsetdash{}{0pt}%
\pgfpathmoveto{\pgfqpoint{0.800000in}{0.528000in}}%
\pgfpathlineto{\pgfqpoint{5.760000in}{0.528000in}}%
\pgfpathlineto{\pgfqpoint{5.760000in}{4.224000in}}%
\pgfpathlineto{\pgfqpoint{0.800000in}{4.224000in}}%
\pgfpathclose%
\pgfusepath{fill}%
\end{pgfscope}%
\begin{pgfscope}%
\pgfpathrectangle{\pgfqpoint{0.800000in}{0.528000in}}{\pgfqpoint{4.960000in}{3.696000in}}%
\pgfusepath{clip}%
\pgfsetbuttcap%
\pgfsetroundjoin%
\definecolor{currentfill}{rgb}{0.000000,0.000000,0.000000}%
\pgfsetfillcolor{currentfill}%
\pgfsetlinewidth{1.003750pt}%
\definecolor{currentstroke}{rgb}{0.000000,0.000000,0.000000}%
\pgfsetstrokecolor{currentstroke}%
\pgfsetdash{}{0pt}%
\pgfpathmoveto{\pgfqpoint{1.025906in}{0.664394in}}%
\pgfpathcurveto{\pgfqpoint{1.036956in}{0.664394in}}{\pgfqpoint{1.047555in}{0.668784in}}{\pgfqpoint{1.055369in}{0.676598in}}%
\pgfpathcurveto{\pgfqpoint{1.063182in}{0.684411in}}{\pgfqpoint{1.067573in}{0.695010in}}{\pgfqpoint{1.067573in}{0.706060in}}%
\pgfpathcurveto{\pgfqpoint{1.067573in}{0.717111in}}{\pgfqpoint{1.063182in}{0.727710in}}{\pgfqpoint{1.055369in}{0.735523in}}%
\pgfpathcurveto{\pgfqpoint{1.047555in}{0.743337in}}{\pgfqpoint{1.036956in}{0.747727in}}{\pgfqpoint{1.025906in}{0.747727in}}%
\pgfpathcurveto{\pgfqpoint{1.014856in}{0.747727in}}{\pgfqpoint{1.004257in}{0.743337in}}{\pgfqpoint{0.996443in}{0.735523in}}%
\pgfpathcurveto{\pgfqpoint{0.988630in}{0.727710in}}{\pgfqpoint{0.984239in}{0.717111in}}{\pgfqpoint{0.984239in}{0.706060in}}%
\pgfpathcurveto{\pgfqpoint{0.984239in}{0.695010in}}{\pgfqpoint{0.988630in}{0.684411in}}{\pgfqpoint{0.996443in}{0.676598in}}%
\pgfpathcurveto{\pgfqpoint{1.004257in}{0.668784in}}{\pgfqpoint{1.014856in}{0.664394in}}{\pgfqpoint{1.025906in}{0.664394in}}%
\pgfpathclose%
\pgfusepath{stroke,fill}%
\end{pgfscope}%
\begin{pgfscope}%
\pgfpathrectangle{\pgfqpoint{0.800000in}{0.528000in}}{\pgfqpoint{4.960000in}{3.696000in}}%
\pgfusepath{clip}%
\pgfsetbuttcap%
\pgfsetroundjoin%
\definecolor{currentfill}{rgb}{0.000000,0.000000,0.000000}%
\pgfsetfillcolor{currentfill}%
\pgfsetlinewidth{1.003750pt}%
\definecolor{currentstroke}{rgb}{0.000000,0.000000,0.000000}%
\pgfsetstrokecolor{currentstroke}%
\pgfsetdash{}{0pt}%
\pgfpathmoveto{\pgfqpoint{1.025906in}{0.664394in}}%
\pgfpathcurveto{\pgfqpoint{1.036956in}{0.664394in}}{\pgfqpoint{1.047555in}{0.668784in}}{\pgfqpoint{1.055369in}{0.676598in}}%
\pgfpathcurveto{\pgfqpoint{1.063182in}{0.684411in}}{\pgfqpoint{1.067573in}{0.695010in}}{\pgfqpoint{1.067573in}{0.706060in}}%
\pgfpathcurveto{\pgfqpoint{1.067573in}{0.717111in}}{\pgfqpoint{1.063182in}{0.727710in}}{\pgfqpoint{1.055369in}{0.735523in}}%
\pgfpathcurveto{\pgfqpoint{1.047555in}{0.743337in}}{\pgfqpoint{1.036956in}{0.747727in}}{\pgfqpoint{1.025906in}{0.747727in}}%
\pgfpathcurveto{\pgfqpoint{1.014856in}{0.747727in}}{\pgfqpoint{1.004257in}{0.743337in}}{\pgfqpoint{0.996443in}{0.735523in}}%
\pgfpathcurveto{\pgfqpoint{0.988630in}{0.727710in}}{\pgfqpoint{0.984239in}{0.717111in}}{\pgfqpoint{0.984239in}{0.706060in}}%
\pgfpathcurveto{\pgfqpoint{0.984239in}{0.695010in}}{\pgfqpoint{0.988630in}{0.684411in}}{\pgfqpoint{0.996443in}{0.676598in}}%
\pgfpathcurveto{\pgfqpoint{1.004257in}{0.668784in}}{\pgfqpoint{1.014856in}{0.664394in}}{\pgfqpoint{1.025906in}{0.664394in}}%
\pgfpathclose%
\pgfusepath{stroke,fill}%
\end{pgfscope}%
\begin{pgfscope}%
\pgfpathrectangle{\pgfqpoint{0.800000in}{0.528000in}}{\pgfqpoint{4.960000in}{3.696000in}}%
\pgfusepath{clip}%
\pgfsetbuttcap%
\pgfsetroundjoin%
\definecolor{currentfill}{rgb}{0.000000,0.000000,0.000000}%
\pgfsetfillcolor{currentfill}%
\pgfsetlinewidth{1.003750pt}%
\definecolor{currentstroke}{rgb}{0.000000,0.000000,0.000000}%
\pgfsetstrokecolor{currentstroke}%
\pgfsetdash{}{0pt}%
\pgfpathmoveto{\pgfqpoint{1.025906in}{0.664394in}}%
\pgfpathcurveto{\pgfqpoint{1.036956in}{0.664394in}}{\pgfqpoint{1.047555in}{0.668784in}}{\pgfqpoint{1.055369in}{0.676598in}}%
\pgfpathcurveto{\pgfqpoint{1.063182in}{0.684411in}}{\pgfqpoint{1.067573in}{0.695010in}}{\pgfqpoint{1.067573in}{0.706060in}}%
\pgfpathcurveto{\pgfqpoint{1.067573in}{0.717111in}}{\pgfqpoint{1.063182in}{0.727710in}}{\pgfqpoint{1.055369in}{0.735523in}}%
\pgfpathcurveto{\pgfqpoint{1.047555in}{0.743337in}}{\pgfqpoint{1.036956in}{0.747727in}}{\pgfqpoint{1.025906in}{0.747727in}}%
\pgfpathcurveto{\pgfqpoint{1.014856in}{0.747727in}}{\pgfqpoint{1.004257in}{0.743337in}}{\pgfqpoint{0.996443in}{0.735523in}}%
\pgfpathcurveto{\pgfqpoint{0.988630in}{0.727710in}}{\pgfqpoint{0.984239in}{0.717111in}}{\pgfqpoint{0.984239in}{0.706060in}}%
\pgfpathcurveto{\pgfqpoint{0.984239in}{0.695010in}}{\pgfqpoint{0.988630in}{0.684411in}}{\pgfqpoint{0.996443in}{0.676598in}}%
\pgfpathcurveto{\pgfqpoint{1.004257in}{0.668784in}}{\pgfqpoint{1.014856in}{0.664394in}}{\pgfqpoint{1.025906in}{0.664394in}}%
\pgfpathclose%
\pgfusepath{stroke,fill}%
\end{pgfscope}%
\begin{pgfscope}%
\pgfpathrectangle{\pgfqpoint{0.800000in}{0.528000in}}{\pgfqpoint{4.960000in}{3.696000in}}%
\pgfusepath{clip}%
\pgfsetbuttcap%
\pgfsetroundjoin%
\definecolor{currentfill}{rgb}{0.000000,0.000000,0.000000}%
\pgfsetfillcolor{currentfill}%
\pgfsetlinewidth{1.003750pt}%
\definecolor{currentstroke}{rgb}{0.000000,0.000000,0.000000}%
\pgfsetstrokecolor{currentstroke}%
\pgfsetdash{}{0pt}%
\pgfpathmoveto{\pgfqpoint{1.025906in}{0.664394in}}%
\pgfpathcurveto{\pgfqpoint{1.036956in}{0.664394in}}{\pgfqpoint{1.047555in}{0.668784in}}{\pgfqpoint{1.055369in}{0.676598in}}%
\pgfpathcurveto{\pgfqpoint{1.063182in}{0.684411in}}{\pgfqpoint{1.067573in}{0.695010in}}{\pgfqpoint{1.067573in}{0.706060in}}%
\pgfpathcurveto{\pgfqpoint{1.067573in}{0.717111in}}{\pgfqpoint{1.063182in}{0.727710in}}{\pgfqpoint{1.055369in}{0.735523in}}%
\pgfpathcurveto{\pgfqpoint{1.047555in}{0.743337in}}{\pgfqpoint{1.036956in}{0.747727in}}{\pgfqpoint{1.025906in}{0.747727in}}%
\pgfpathcurveto{\pgfqpoint{1.014856in}{0.747727in}}{\pgfqpoint{1.004257in}{0.743337in}}{\pgfqpoint{0.996443in}{0.735523in}}%
\pgfpathcurveto{\pgfqpoint{0.988630in}{0.727710in}}{\pgfqpoint{0.984239in}{0.717111in}}{\pgfqpoint{0.984239in}{0.706060in}}%
\pgfpathcurveto{\pgfqpoint{0.984239in}{0.695010in}}{\pgfqpoint{0.988630in}{0.684411in}}{\pgfqpoint{0.996443in}{0.676598in}}%
\pgfpathcurveto{\pgfqpoint{1.004257in}{0.668784in}}{\pgfqpoint{1.014856in}{0.664394in}}{\pgfqpoint{1.025906in}{0.664394in}}%
\pgfpathclose%
\pgfusepath{stroke,fill}%
\end{pgfscope}%
\begin{pgfscope}%
\pgfpathrectangle{\pgfqpoint{0.800000in}{0.528000in}}{\pgfqpoint{4.960000in}{3.696000in}}%
\pgfusepath{clip}%
\pgfsetbuttcap%
\pgfsetroundjoin%
\definecolor{currentfill}{rgb}{0.000000,0.000000,0.000000}%
\pgfsetfillcolor{currentfill}%
\pgfsetlinewidth{1.003750pt}%
\definecolor{currentstroke}{rgb}{0.000000,0.000000,0.000000}%
\pgfsetstrokecolor{currentstroke}%
\pgfsetdash{}{0pt}%
\pgfpathmoveto{\pgfqpoint{1.025906in}{0.664394in}}%
\pgfpathcurveto{\pgfqpoint{1.036956in}{0.664394in}}{\pgfqpoint{1.047555in}{0.668784in}}{\pgfqpoint{1.055369in}{0.676598in}}%
\pgfpathcurveto{\pgfqpoint{1.063182in}{0.684411in}}{\pgfqpoint{1.067573in}{0.695010in}}{\pgfqpoint{1.067573in}{0.706060in}}%
\pgfpathcurveto{\pgfqpoint{1.067573in}{0.717111in}}{\pgfqpoint{1.063182in}{0.727710in}}{\pgfqpoint{1.055369in}{0.735523in}}%
\pgfpathcurveto{\pgfqpoint{1.047555in}{0.743337in}}{\pgfqpoint{1.036956in}{0.747727in}}{\pgfqpoint{1.025906in}{0.747727in}}%
\pgfpathcurveto{\pgfqpoint{1.014856in}{0.747727in}}{\pgfqpoint{1.004257in}{0.743337in}}{\pgfqpoint{0.996443in}{0.735523in}}%
\pgfpathcurveto{\pgfqpoint{0.988630in}{0.727710in}}{\pgfqpoint{0.984239in}{0.717111in}}{\pgfqpoint{0.984239in}{0.706060in}}%
\pgfpathcurveto{\pgfqpoint{0.984239in}{0.695010in}}{\pgfqpoint{0.988630in}{0.684411in}}{\pgfqpoint{0.996443in}{0.676598in}}%
\pgfpathcurveto{\pgfqpoint{1.004257in}{0.668784in}}{\pgfqpoint{1.014856in}{0.664394in}}{\pgfqpoint{1.025906in}{0.664394in}}%
\pgfpathclose%
\pgfusepath{stroke,fill}%
\end{pgfscope}%
\begin{pgfscope}%
\pgfpathrectangle{\pgfqpoint{0.800000in}{0.528000in}}{\pgfqpoint{4.960000in}{3.696000in}}%
\pgfusepath{clip}%
\pgfsetbuttcap%
\pgfsetroundjoin%
\definecolor{currentfill}{rgb}{0.000000,0.000000,0.000000}%
\pgfsetfillcolor{currentfill}%
\pgfsetlinewidth{1.003750pt}%
\definecolor{currentstroke}{rgb}{0.000000,0.000000,0.000000}%
\pgfsetstrokecolor{currentstroke}%
\pgfsetdash{}{0pt}%
\pgfpathmoveto{\pgfqpoint{1.025906in}{0.664394in}}%
\pgfpathcurveto{\pgfqpoint{1.036956in}{0.664394in}}{\pgfqpoint{1.047555in}{0.668784in}}{\pgfqpoint{1.055369in}{0.676598in}}%
\pgfpathcurveto{\pgfqpoint{1.063182in}{0.684411in}}{\pgfqpoint{1.067573in}{0.695010in}}{\pgfqpoint{1.067573in}{0.706060in}}%
\pgfpathcurveto{\pgfqpoint{1.067573in}{0.717111in}}{\pgfqpoint{1.063182in}{0.727710in}}{\pgfqpoint{1.055369in}{0.735523in}}%
\pgfpathcurveto{\pgfqpoint{1.047555in}{0.743337in}}{\pgfqpoint{1.036956in}{0.747727in}}{\pgfqpoint{1.025906in}{0.747727in}}%
\pgfpathcurveto{\pgfqpoint{1.014856in}{0.747727in}}{\pgfqpoint{1.004257in}{0.743337in}}{\pgfqpoint{0.996443in}{0.735523in}}%
\pgfpathcurveto{\pgfqpoint{0.988630in}{0.727710in}}{\pgfqpoint{0.984239in}{0.717111in}}{\pgfqpoint{0.984239in}{0.706060in}}%
\pgfpathcurveto{\pgfqpoint{0.984239in}{0.695010in}}{\pgfqpoint{0.988630in}{0.684411in}}{\pgfqpoint{0.996443in}{0.676598in}}%
\pgfpathcurveto{\pgfqpoint{1.004257in}{0.668784in}}{\pgfqpoint{1.014856in}{0.664394in}}{\pgfqpoint{1.025906in}{0.664394in}}%
\pgfpathclose%
\pgfusepath{stroke,fill}%
\end{pgfscope}%
\begin{pgfscope}%
\pgfpathrectangle{\pgfqpoint{0.800000in}{0.528000in}}{\pgfqpoint{4.960000in}{3.696000in}}%
\pgfusepath{clip}%
\pgfsetbuttcap%
\pgfsetroundjoin%
\definecolor{currentfill}{rgb}{0.000000,0.000000,0.000000}%
\pgfsetfillcolor{currentfill}%
\pgfsetlinewidth{1.003750pt}%
\definecolor{currentstroke}{rgb}{0.000000,0.000000,0.000000}%
\pgfsetstrokecolor{currentstroke}%
\pgfsetdash{}{0pt}%
\pgfpathmoveto{\pgfqpoint{1.025906in}{0.664394in}}%
\pgfpathcurveto{\pgfqpoint{1.036956in}{0.664394in}}{\pgfqpoint{1.047555in}{0.668784in}}{\pgfqpoint{1.055369in}{0.676598in}}%
\pgfpathcurveto{\pgfqpoint{1.063182in}{0.684411in}}{\pgfqpoint{1.067573in}{0.695010in}}{\pgfqpoint{1.067573in}{0.706060in}}%
\pgfpathcurveto{\pgfqpoint{1.067573in}{0.717111in}}{\pgfqpoint{1.063182in}{0.727710in}}{\pgfqpoint{1.055369in}{0.735523in}}%
\pgfpathcurveto{\pgfqpoint{1.047555in}{0.743337in}}{\pgfqpoint{1.036956in}{0.747727in}}{\pgfqpoint{1.025906in}{0.747727in}}%
\pgfpathcurveto{\pgfqpoint{1.014856in}{0.747727in}}{\pgfqpoint{1.004257in}{0.743337in}}{\pgfqpoint{0.996443in}{0.735523in}}%
\pgfpathcurveto{\pgfqpoint{0.988630in}{0.727710in}}{\pgfqpoint{0.984239in}{0.717111in}}{\pgfqpoint{0.984239in}{0.706060in}}%
\pgfpathcurveto{\pgfqpoint{0.984239in}{0.695010in}}{\pgfqpoint{0.988630in}{0.684411in}}{\pgfqpoint{0.996443in}{0.676598in}}%
\pgfpathcurveto{\pgfqpoint{1.004257in}{0.668784in}}{\pgfqpoint{1.014856in}{0.664394in}}{\pgfqpoint{1.025906in}{0.664394in}}%
\pgfpathclose%
\pgfusepath{stroke,fill}%
\end{pgfscope}%
\begin{pgfscope}%
\pgfpathrectangle{\pgfqpoint{0.800000in}{0.528000in}}{\pgfqpoint{4.960000in}{3.696000in}}%
\pgfusepath{clip}%
\pgfsetbuttcap%
\pgfsetroundjoin%
\definecolor{currentfill}{rgb}{0.000000,0.000000,0.000000}%
\pgfsetfillcolor{currentfill}%
\pgfsetlinewidth{1.003750pt}%
\definecolor{currentstroke}{rgb}{0.000000,0.000000,0.000000}%
\pgfsetstrokecolor{currentstroke}%
\pgfsetdash{}{0pt}%
\pgfpathmoveto{\pgfqpoint{1.025906in}{0.664394in}}%
\pgfpathcurveto{\pgfqpoint{1.036956in}{0.664394in}}{\pgfqpoint{1.047555in}{0.668784in}}{\pgfqpoint{1.055369in}{0.676598in}}%
\pgfpathcurveto{\pgfqpoint{1.063182in}{0.684411in}}{\pgfqpoint{1.067573in}{0.695010in}}{\pgfqpoint{1.067573in}{0.706060in}}%
\pgfpathcurveto{\pgfqpoint{1.067573in}{0.717111in}}{\pgfqpoint{1.063182in}{0.727710in}}{\pgfqpoint{1.055369in}{0.735523in}}%
\pgfpathcurveto{\pgfqpoint{1.047555in}{0.743337in}}{\pgfqpoint{1.036956in}{0.747727in}}{\pgfqpoint{1.025906in}{0.747727in}}%
\pgfpathcurveto{\pgfqpoint{1.014856in}{0.747727in}}{\pgfqpoint{1.004257in}{0.743337in}}{\pgfqpoint{0.996443in}{0.735523in}}%
\pgfpathcurveto{\pgfqpoint{0.988630in}{0.727710in}}{\pgfqpoint{0.984239in}{0.717111in}}{\pgfqpoint{0.984239in}{0.706060in}}%
\pgfpathcurveto{\pgfqpoint{0.984239in}{0.695010in}}{\pgfqpoint{0.988630in}{0.684411in}}{\pgfqpoint{0.996443in}{0.676598in}}%
\pgfpathcurveto{\pgfqpoint{1.004257in}{0.668784in}}{\pgfqpoint{1.014856in}{0.664394in}}{\pgfqpoint{1.025906in}{0.664394in}}%
\pgfpathclose%
\pgfusepath{stroke,fill}%
\end{pgfscope}%
\begin{pgfscope}%
\pgfpathrectangle{\pgfqpoint{0.800000in}{0.528000in}}{\pgfqpoint{4.960000in}{3.696000in}}%
\pgfusepath{clip}%
\pgfsetbuttcap%
\pgfsetroundjoin%
\definecolor{currentfill}{rgb}{0.000000,0.000000,0.000000}%
\pgfsetfillcolor{currentfill}%
\pgfsetlinewidth{1.003750pt}%
\definecolor{currentstroke}{rgb}{0.000000,0.000000,0.000000}%
\pgfsetstrokecolor{currentstroke}%
\pgfsetdash{}{0pt}%
\pgfpathmoveto{\pgfqpoint{1.025906in}{0.664394in}}%
\pgfpathcurveto{\pgfqpoint{1.036956in}{0.664394in}}{\pgfqpoint{1.047555in}{0.668784in}}{\pgfqpoint{1.055369in}{0.676598in}}%
\pgfpathcurveto{\pgfqpoint{1.063182in}{0.684411in}}{\pgfqpoint{1.067573in}{0.695010in}}{\pgfqpoint{1.067573in}{0.706060in}}%
\pgfpathcurveto{\pgfqpoint{1.067573in}{0.717111in}}{\pgfqpoint{1.063182in}{0.727710in}}{\pgfqpoint{1.055369in}{0.735523in}}%
\pgfpathcurveto{\pgfqpoint{1.047555in}{0.743337in}}{\pgfqpoint{1.036956in}{0.747727in}}{\pgfqpoint{1.025906in}{0.747727in}}%
\pgfpathcurveto{\pgfqpoint{1.014856in}{0.747727in}}{\pgfqpoint{1.004257in}{0.743337in}}{\pgfqpoint{0.996443in}{0.735523in}}%
\pgfpathcurveto{\pgfqpoint{0.988630in}{0.727710in}}{\pgfqpoint{0.984239in}{0.717111in}}{\pgfqpoint{0.984239in}{0.706060in}}%
\pgfpathcurveto{\pgfqpoint{0.984239in}{0.695010in}}{\pgfqpoint{0.988630in}{0.684411in}}{\pgfqpoint{0.996443in}{0.676598in}}%
\pgfpathcurveto{\pgfqpoint{1.004257in}{0.668784in}}{\pgfqpoint{1.014856in}{0.664394in}}{\pgfqpoint{1.025906in}{0.664394in}}%
\pgfpathclose%
\pgfusepath{stroke,fill}%
\end{pgfscope}%
\begin{pgfscope}%
\pgfpathrectangle{\pgfqpoint{0.800000in}{0.528000in}}{\pgfqpoint{4.960000in}{3.696000in}}%
\pgfusepath{clip}%
\pgfsetbuttcap%
\pgfsetroundjoin%
\definecolor{currentfill}{rgb}{0.000000,0.000000,0.000000}%
\pgfsetfillcolor{currentfill}%
\pgfsetlinewidth{1.003750pt}%
\definecolor{currentstroke}{rgb}{0.000000,0.000000,0.000000}%
\pgfsetstrokecolor{currentstroke}%
\pgfsetdash{}{0pt}%
\pgfpathmoveto{\pgfqpoint{1.025906in}{0.664394in}}%
\pgfpathcurveto{\pgfqpoint{1.036956in}{0.664394in}}{\pgfqpoint{1.047555in}{0.668784in}}{\pgfqpoint{1.055369in}{0.676598in}}%
\pgfpathcurveto{\pgfqpoint{1.063182in}{0.684411in}}{\pgfqpoint{1.067573in}{0.695010in}}{\pgfqpoint{1.067573in}{0.706060in}}%
\pgfpathcurveto{\pgfqpoint{1.067573in}{0.717111in}}{\pgfqpoint{1.063182in}{0.727710in}}{\pgfqpoint{1.055369in}{0.735523in}}%
\pgfpathcurveto{\pgfqpoint{1.047555in}{0.743337in}}{\pgfqpoint{1.036956in}{0.747727in}}{\pgfqpoint{1.025906in}{0.747727in}}%
\pgfpathcurveto{\pgfqpoint{1.014856in}{0.747727in}}{\pgfqpoint{1.004257in}{0.743337in}}{\pgfqpoint{0.996443in}{0.735523in}}%
\pgfpathcurveto{\pgfqpoint{0.988630in}{0.727710in}}{\pgfqpoint{0.984239in}{0.717111in}}{\pgfqpoint{0.984239in}{0.706060in}}%
\pgfpathcurveto{\pgfqpoint{0.984239in}{0.695010in}}{\pgfqpoint{0.988630in}{0.684411in}}{\pgfqpoint{0.996443in}{0.676598in}}%
\pgfpathcurveto{\pgfqpoint{1.004257in}{0.668784in}}{\pgfqpoint{1.014856in}{0.664394in}}{\pgfqpoint{1.025906in}{0.664394in}}%
\pgfpathclose%
\pgfusepath{stroke,fill}%
\end{pgfscope}%
\begin{pgfscope}%
\pgfpathrectangle{\pgfqpoint{0.800000in}{0.528000in}}{\pgfqpoint{4.960000in}{3.696000in}}%
\pgfusepath{clip}%
\pgfsetbuttcap%
\pgfsetroundjoin%
\definecolor{currentfill}{rgb}{0.000000,0.000000,0.000000}%
\pgfsetfillcolor{currentfill}%
\pgfsetlinewidth{1.003750pt}%
\definecolor{currentstroke}{rgb}{0.000000,0.000000,0.000000}%
\pgfsetstrokecolor{currentstroke}%
\pgfsetdash{}{0pt}%
\pgfpathmoveto{\pgfqpoint{1.025906in}{1.771040in}}%
\pgfpathcurveto{\pgfqpoint{1.036956in}{1.771040in}}{\pgfqpoint{1.047555in}{1.775431in}}{\pgfqpoint{1.055369in}{1.783244in}}%
\pgfpathcurveto{\pgfqpoint{1.063182in}{1.791058in}}{\pgfqpoint{1.067573in}{1.801657in}}{\pgfqpoint{1.067573in}{1.812707in}}%
\pgfpathcurveto{\pgfqpoint{1.067573in}{1.823757in}}{\pgfqpoint{1.063182in}{1.834356in}}{\pgfqpoint{1.055369in}{1.842170in}}%
\pgfpathcurveto{\pgfqpoint{1.047555in}{1.849983in}}{\pgfqpoint{1.036956in}{1.854374in}}{\pgfqpoint{1.025906in}{1.854374in}}%
\pgfpathcurveto{\pgfqpoint{1.014856in}{1.854374in}}{\pgfqpoint{1.004257in}{1.849983in}}{\pgfqpoint{0.996443in}{1.842170in}}%
\pgfpathcurveto{\pgfqpoint{0.988630in}{1.834356in}}{\pgfqpoint{0.984239in}{1.823757in}}{\pgfqpoint{0.984239in}{1.812707in}}%
\pgfpathcurveto{\pgfqpoint{0.984239in}{1.801657in}}{\pgfqpoint{0.988630in}{1.791058in}}{\pgfqpoint{0.996443in}{1.783244in}}%
\pgfpathcurveto{\pgfqpoint{1.004257in}{1.775431in}}{\pgfqpoint{1.014856in}{1.771040in}}{\pgfqpoint{1.025906in}{1.771040in}}%
\pgfpathclose%
\pgfusepath{stroke,fill}%
\end{pgfscope}%
\begin{pgfscope}%
\pgfpathrectangle{\pgfqpoint{0.800000in}{0.528000in}}{\pgfqpoint{4.960000in}{3.696000in}}%
\pgfusepath{clip}%
\pgfsetbuttcap%
\pgfsetroundjoin%
\definecolor{currentfill}{rgb}{0.000000,0.000000,0.000000}%
\pgfsetfillcolor{currentfill}%
\pgfsetlinewidth{1.003750pt}%
\definecolor{currentstroke}{rgb}{0.000000,0.000000,0.000000}%
\pgfsetstrokecolor{currentstroke}%
\pgfsetdash{}{0pt}%
\pgfpathmoveto{\pgfqpoint{1.025906in}{0.664394in}}%
\pgfpathcurveto{\pgfqpoint{1.036956in}{0.664394in}}{\pgfqpoint{1.047555in}{0.668784in}}{\pgfqpoint{1.055369in}{0.676598in}}%
\pgfpathcurveto{\pgfqpoint{1.063182in}{0.684411in}}{\pgfqpoint{1.067573in}{0.695010in}}{\pgfqpoint{1.067573in}{0.706060in}}%
\pgfpathcurveto{\pgfqpoint{1.067573in}{0.717111in}}{\pgfqpoint{1.063182in}{0.727710in}}{\pgfqpoint{1.055369in}{0.735523in}}%
\pgfpathcurveto{\pgfqpoint{1.047555in}{0.743337in}}{\pgfqpoint{1.036956in}{0.747727in}}{\pgfqpoint{1.025906in}{0.747727in}}%
\pgfpathcurveto{\pgfqpoint{1.014856in}{0.747727in}}{\pgfqpoint{1.004257in}{0.743337in}}{\pgfqpoint{0.996443in}{0.735523in}}%
\pgfpathcurveto{\pgfqpoint{0.988630in}{0.727710in}}{\pgfqpoint{0.984239in}{0.717111in}}{\pgfqpoint{0.984239in}{0.706060in}}%
\pgfpathcurveto{\pgfqpoint{0.984239in}{0.695010in}}{\pgfqpoint{0.988630in}{0.684411in}}{\pgfqpoint{0.996443in}{0.676598in}}%
\pgfpathcurveto{\pgfqpoint{1.004257in}{0.668784in}}{\pgfqpoint{1.014856in}{0.664394in}}{\pgfqpoint{1.025906in}{0.664394in}}%
\pgfpathclose%
\pgfusepath{stroke,fill}%
\end{pgfscope}%
\begin{pgfscope}%
\pgfpathrectangle{\pgfqpoint{0.800000in}{0.528000in}}{\pgfqpoint{4.960000in}{3.696000in}}%
\pgfusepath{clip}%
\pgfsetbuttcap%
\pgfsetroundjoin%
\definecolor{currentfill}{rgb}{0.000000,0.000000,0.000000}%
\pgfsetfillcolor{currentfill}%
\pgfsetlinewidth{1.003750pt}%
\definecolor{currentstroke}{rgb}{0.000000,0.000000,0.000000}%
\pgfsetstrokecolor{currentstroke}%
\pgfsetdash{}{0pt}%
\pgfpathmoveto{\pgfqpoint{1.025906in}{0.664394in}}%
\pgfpathcurveto{\pgfqpoint{1.036956in}{0.664394in}}{\pgfqpoint{1.047555in}{0.668784in}}{\pgfqpoint{1.055369in}{0.676598in}}%
\pgfpathcurveto{\pgfqpoint{1.063182in}{0.684411in}}{\pgfqpoint{1.067573in}{0.695010in}}{\pgfqpoint{1.067573in}{0.706060in}}%
\pgfpathcurveto{\pgfqpoint{1.067573in}{0.717111in}}{\pgfqpoint{1.063182in}{0.727710in}}{\pgfqpoint{1.055369in}{0.735523in}}%
\pgfpathcurveto{\pgfqpoint{1.047555in}{0.743337in}}{\pgfqpoint{1.036956in}{0.747727in}}{\pgfqpoint{1.025906in}{0.747727in}}%
\pgfpathcurveto{\pgfqpoint{1.014856in}{0.747727in}}{\pgfqpoint{1.004257in}{0.743337in}}{\pgfqpoint{0.996443in}{0.735523in}}%
\pgfpathcurveto{\pgfqpoint{0.988630in}{0.727710in}}{\pgfqpoint{0.984239in}{0.717111in}}{\pgfqpoint{0.984239in}{0.706060in}}%
\pgfpathcurveto{\pgfqpoint{0.984239in}{0.695010in}}{\pgfqpoint{0.988630in}{0.684411in}}{\pgfqpoint{0.996443in}{0.676598in}}%
\pgfpathcurveto{\pgfqpoint{1.004257in}{0.668784in}}{\pgfqpoint{1.014856in}{0.664394in}}{\pgfqpoint{1.025906in}{0.664394in}}%
\pgfpathclose%
\pgfusepath{stroke,fill}%
\end{pgfscope}%
\begin{pgfscope}%
\pgfpathrectangle{\pgfqpoint{0.800000in}{0.528000in}}{\pgfqpoint{4.960000in}{3.696000in}}%
\pgfusepath{clip}%
\pgfsetbuttcap%
\pgfsetroundjoin%
\definecolor{currentfill}{rgb}{0.000000,0.000000,0.000000}%
\pgfsetfillcolor{currentfill}%
\pgfsetlinewidth{1.003750pt}%
\definecolor{currentstroke}{rgb}{0.000000,0.000000,0.000000}%
\pgfsetstrokecolor{currentstroke}%
\pgfsetdash{}{0pt}%
\pgfpathmoveto{\pgfqpoint{1.025906in}{0.664394in}}%
\pgfpathcurveto{\pgfqpoint{1.036956in}{0.664394in}}{\pgfqpoint{1.047555in}{0.668784in}}{\pgfqpoint{1.055369in}{0.676598in}}%
\pgfpathcurveto{\pgfqpoint{1.063182in}{0.684411in}}{\pgfqpoint{1.067573in}{0.695010in}}{\pgfqpoint{1.067573in}{0.706060in}}%
\pgfpathcurveto{\pgfqpoint{1.067573in}{0.717111in}}{\pgfqpoint{1.063182in}{0.727710in}}{\pgfqpoint{1.055369in}{0.735523in}}%
\pgfpathcurveto{\pgfqpoint{1.047555in}{0.743337in}}{\pgfqpoint{1.036956in}{0.747727in}}{\pgfqpoint{1.025906in}{0.747727in}}%
\pgfpathcurveto{\pgfqpoint{1.014856in}{0.747727in}}{\pgfqpoint{1.004257in}{0.743337in}}{\pgfqpoint{0.996443in}{0.735523in}}%
\pgfpathcurveto{\pgfqpoint{0.988630in}{0.727710in}}{\pgfqpoint{0.984239in}{0.717111in}}{\pgfqpoint{0.984239in}{0.706060in}}%
\pgfpathcurveto{\pgfqpoint{0.984239in}{0.695010in}}{\pgfqpoint{0.988630in}{0.684411in}}{\pgfqpoint{0.996443in}{0.676598in}}%
\pgfpathcurveto{\pgfqpoint{1.004257in}{0.668784in}}{\pgfqpoint{1.014856in}{0.664394in}}{\pgfqpoint{1.025906in}{0.664394in}}%
\pgfpathclose%
\pgfusepath{stroke,fill}%
\end{pgfscope}%
\begin{pgfscope}%
\pgfpathrectangle{\pgfqpoint{0.800000in}{0.528000in}}{\pgfqpoint{4.960000in}{3.696000in}}%
\pgfusepath{clip}%
\pgfsetbuttcap%
\pgfsetroundjoin%
\definecolor{currentfill}{rgb}{0.000000,0.000000,0.000000}%
\pgfsetfillcolor{currentfill}%
\pgfsetlinewidth{1.003750pt}%
\definecolor{currentstroke}{rgb}{0.000000,0.000000,0.000000}%
\pgfsetstrokecolor{currentstroke}%
\pgfsetdash{}{0pt}%
\pgfpathmoveto{\pgfqpoint{1.025906in}{0.664394in}}%
\pgfpathcurveto{\pgfqpoint{1.036956in}{0.664394in}}{\pgfqpoint{1.047555in}{0.668784in}}{\pgfqpoint{1.055369in}{0.676598in}}%
\pgfpathcurveto{\pgfqpoint{1.063182in}{0.684411in}}{\pgfqpoint{1.067573in}{0.695010in}}{\pgfqpoint{1.067573in}{0.706060in}}%
\pgfpathcurveto{\pgfqpoint{1.067573in}{0.717111in}}{\pgfqpoint{1.063182in}{0.727710in}}{\pgfqpoint{1.055369in}{0.735523in}}%
\pgfpathcurveto{\pgfqpoint{1.047555in}{0.743337in}}{\pgfqpoint{1.036956in}{0.747727in}}{\pgfqpoint{1.025906in}{0.747727in}}%
\pgfpathcurveto{\pgfqpoint{1.014856in}{0.747727in}}{\pgfqpoint{1.004257in}{0.743337in}}{\pgfqpoint{0.996443in}{0.735523in}}%
\pgfpathcurveto{\pgfqpoint{0.988630in}{0.727710in}}{\pgfqpoint{0.984239in}{0.717111in}}{\pgfqpoint{0.984239in}{0.706060in}}%
\pgfpathcurveto{\pgfqpoint{0.984239in}{0.695010in}}{\pgfqpoint{0.988630in}{0.684411in}}{\pgfqpoint{0.996443in}{0.676598in}}%
\pgfpathcurveto{\pgfqpoint{1.004257in}{0.668784in}}{\pgfqpoint{1.014856in}{0.664394in}}{\pgfqpoint{1.025906in}{0.664394in}}%
\pgfpathclose%
\pgfusepath{stroke,fill}%
\end{pgfscope}%
\begin{pgfscope}%
\pgfpathrectangle{\pgfqpoint{0.800000in}{0.528000in}}{\pgfqpoint{4.960000in}{3.696000in}}%
\pgfusepath{clip}%
\pgfsetbuttcap%
\pgfsetroundjoin%
\definecolor{currentfill}{rgb}{0.000000,0.000000,0.000000}%
\pgfsetfillcolor{currentfill}%
\pgfsetlinewidth{1.003750pt}%
\definecolor{currentstroke}{rgb}{0.000000,0.000000,0.000000}%
\pgfsetstrokecolor{currentstroke}%
\pgfsetdash{}{0pt}%
\pgfpathmoveto{\pgfqpoint{1.025906in}{0.664394in}}%
\pgfpathcurveto{\pgfqpoint{1.036956in}{0.664394in}}{\pgfqpoint{1.047555in}{0.668784in}}{\pgfqpoint{1.055369in}{0.676598in}}%
\pgfpathcurveto{\pgfqpoint{1.063182in}{0.684411in}}{\pgfqpoint{1.067573in}{0.695010in}}{\pgfqpoint{1.067573in}{0.706060in}}%
\pgfpathcurveto{\pgfqpoint{1.067573in}{0.717111in}}{\pgfqpoint{1.063182in}{0.727710in}}{\pgfqpoint{1.055369in}{0.735523in}}%
\pgfpathcurveto{\pgfqpoint{1.047555in}{0.743337in}}{\pgfqpoint{1.036956in}{0.747727in}}{\pgfqpoint{1.025906in}{0.747727in}}%
\pgfpathcurveto{\pgfqpoint{1.014856in}{0.747727in}}{\pgfqpoint{1.004257in}{0.743337in}}{\pgfqpoint{0.996443in}{0.735523in}}%
\pgfpathcurveto{\pgfqpoint{0.988630in}{0.727710in}}{\pgfqpoint{0.984239in}{0.717111in}}{\pgfqpoint{0.984239in}{0.706060in}}%
\pgfpathcurveto{\pgfqpoint{0.984239in}{0.695010in}}{\pgfqpoint{0.988630in}{0.684411in}}{\pgfqpoint{0.996443in}{0.676598in}}%
\pgfpathcurveto{\pgfqpoint{1.004257in}{0.668784in}}{\pgfqpoint{1.014856in}{0.664394in}}{\pgfqpoint{1.025906in}{0.664394in}}%
\pgfpathclose%
\pgfusepath{stroke,fill}%
\end{pgfscope}%
\begin{pgfscope}%
\pgfpathrectangle{\pgfqpoint{0.800000in}{0.528000in}}{\pgfqpoint{4.960000in}{3.696000in}}%
\pgfusepath{clip}%
\pgfsetbuttcap%
\pgfsetroundjoin%
\definecolor{currentfill}{rgb}{0.000000,0.000000,0.000000}%
\pgfsetfillcolor{currentfill}%
\pgfsetlinewidth{1.003750pt}%
\definecolor{currentstroke}{rgb}{0.000000,0.000000,0.000000}%
\pgfsetstrokecolor{currentstroke}%
\pgfsetdash{}{0pt}%
\pgfpathmoveto{\pgfqpoint{1.025906in}{0.664394in}}%
\pgfpathcurveto{\pgfqpoint{1.036956in}{0.664394in}}{\pgfqpoint{1.047555in}{0.668784in}}{\pgfqpoint{1.055369in}{0.676598in}}%
\pgfpathcurveto{\pgfqpoint{1.063182in}{0.684411in}}{\pgfqpoint{1.067573in}{0.695010in}}{\pgfqpoint{1.067573in}{0.706060in}}%
\pgfpathcurveto{\pgfqpoint{1.067573in}{0.717111in}}{\pgfqpoint{1.063182in}{0.727710in}}{\pgfqpoint{1.055369in}{0.735523in}}%
\pgfpathcurveto{\pgfqpoint{1.047555in}{0.743337in}}{\pgfqpoint{1.036956in}{0.747727in}}{\pgfqpoint{1.025906in}{0.747727in}}%
\pgfpathcurveto{\pgfqpoint{1.014856in}{0.747727in}}{\pgfqpoint{1.004257in}{0.743337in}}{\pgfqpoint{0.996443in}{0.735523in}}%
\pgfpathcurveto{\pgfqpoint{0.988630in}{0.727710in}}{\pgfqpoint{0.984239in}{0.717111in}}{\pgfqpoint{0.984239in}{0.706060in}}%
\pgfpathcurveto{\pgfqpoint{0.984239in}{0.695010in}}{\pgfqpoint{0.988630in}{0.684411in}}{\pgfqpoint{0.996443in}{0.676598in}}%
\pgfpathcurveto{\pgfqpoint{1.004257in}{0.668784in}}{\pgfqpoint{1.014856in}{0.664394in}}{\pgfqpoint{1.025906in}{0.664394in}}%
\pgfpathclose%
\pgfusepath{stroke,fill}%
\end{pgfscope}%
\begin{pgfscope}%
\pgfpathrectangle{\pgfqpoint{0.800000in}{0.528000in}}{\pgfqpoint{4.960000in}{3.696000in}}%
\pgfusepath{clip}%
\pgfsetbuttcap%
\pgfsetroundjoin%
\definecolor{currentfill}{rgb}{0.000000,0.000000,0.000000}%
\pgfsetfillcolor{currentfill}%
\pgfsetlinewidth{1.003750pt}%
\definecolor{currentstroke}{rgb}{0.000000,0.000000,0.000000}%
\pgfsetstrokecolor{currentstroke}%
\pgfsetdash{}{0pt}%
\pgfpathmoveto{\pgfqpoint{1.025906in}{0.664394in}}%
\pgfpathcurveto{\pgfqpoint{1.036956in}{0.664394in}}{\pgfqpoint{1.047555in}{0.668784in}}{\pgfqpoint{1.055369in}{0.676598in}}%
\pgfpathcurveto{\pgfqpoint{1.063182in}{0.684411in}}{\pgfqpoint{1.067573in}{0.695010in}}{\pgfqpoint{1.067573in}{0.706060in}}%
\pgfpathcurveto{\pgfqpoint{1.067573in}{0.717111in}}{\pgfqpoint{1.063182in}{0.727710in}}{\pgfqpoint{1.055369in}{0.735523in}}%
\pgfpathcurveto{\pgfqpoint{1.047555in}{0.743337in}}{\pgfqpoint{1.036956in}{0.747727in}}{\pgfqpoint{1.025906in}{0.747727in}}%
\pgfpathcurveto{\pgfqpoint{1.014856in}{0.747727in}}{\pgfqpoint{1.004257in}{0.743337in}}{\pgfqpoint{0.996443in}{0.735523in}}%
\pgfpathcurveto{\pgfqpoint{0.988630in}{0.727710in}}{\pgfqpoint{0.984239in}{0.717111in}}{\pgfqpoint{0.984239in}{0.706060in}}%
\pgfpathcurveto{\pgfqpoint{0.984239in}{0.695010in}}{\pgfqpoint{0.988630in}{0.684411in}}{\pgfqpoint{0.996443in}{0.676598in}}%
\pgfpathcurveto{\pgfqpoint{1.004257in}{0.668784in}}{\pgfqpoint{1.014856in}{0.664394in}}{\pgfqpoint{1.025906in}{0.664394in}}%
\pgfpathclose%
\pgfusepath{stroke,fill}%
\end{pgfscope}%
\begin{pgfscope}%
\pgfpathrectangle{\pgfqpoint{0.800000in}{0.528000in}}{\pgfqpoint{4.960000in}{3.696000in}}%
\pgfusepath{clip}%
\pgfsetbuttcap%
\pgfsetroundjoin%
\definecolor{currentfill}{rgb}{0.000000,0.000000,0.000000}%
\pgfsetfillcolor{currentfill}%
\pgfsetlinewidth{1.003750pt}%
\definecolor{currentstroke}{rgb}{0.000000,0.000000,0.000000}%
\pgfsetstrokecolor{currentstroke}%
\pgfsetdash{}{0pt}%
\pgfpathmoveto{\pgfqpoint{1.025906in}{0.664394in}}%
\pgfpathcurveto{\pgfqpoint{1.036956in}{0.664394in}}{\pgfqpoint{1.047555in}{0.668784in}}{\pgfqpoint{1.055369in}{0.676598in}}%
\pgfpathcurveto{\pgfqpoint{1.063182in}{0.684411in}}{\pgfqpoint{1.067573in}{0.695010in}}{\pgfqpoint{1.067573in}{0.706060in}}%
\pgfpathcurveto{\pgfqpoint{1.067573in}{0.717111in}}{\pgfqpoint{1.063182in}{0.727710in}}{\pgfqpoint{1.055369in}{0.735523in}}%
\pgfpathcurveto{\pgfqpoint{1.047555in}{0.743337in}}{\pgfqpoint{1.036956in}{0.747727in}}{\pgfqpoint{1.025906in}{0.747727in}}%
\pgfpathcurveto{\pgfqpoint{1.014856in}{0.747727in}}{\pgfqpoint{1.004257in}{0.743337in}}{\pgfqpoint{0.996443in}{0.735523in}}%
\pgfpathcurveto{\pgfqpoint{0.988630in}{0.727710in}}{\pgfqpoint{0.984239in}{0.717111in}}{\pgfqpoint{0.984239in}{0.706060in}}%
\pgfpathcurveto{\pgfqpoint{0.984239in}{0.695010in}}{\pgfqpoint{0.988630in}{0.684411in}}{\pgfqpoint{0.996443in}{0.676598in}}%
\pgfpathcurveto{\pgfqpoint{1.004257in}{0.668784in}}{\pgfqpoint{1.014856in}{0.664394in}}{\pgfqpoint{1.025906in}{0.664394in}}%
\pgfpathclose%
\pgfusepath{stroke,fill}%
\end{pgfscope}%
\begin{pgfscope}%
\pgfpathrectangle{\pgfqpoint{0.800000in}{0.528000in}}{\pgfqpoint{4.960000in}{3.696000in}}%
\pgfusepath{clip}%
\pgfsetbuttcap%
\pgfsetroundjoin%
\definecolor{currentfill}{rgb}{0.000000,0.000000,0.000000}%
\pgfsetfillcolor{currentfill}%
\pgfsetlinewidth{1.003750pt}%
\definecolor{currentstroke}{rgb}{0.000000,0.000000,0.000000}%
\pgfsetstrokecolor{currentstroke}%
\pgfsetdash{}{0pt}%
\pgfpathmoveto{\pgfqpoint{1.025906in}{0.664394in}}%
\pgfpathcurveto{\pgfqpoint{1.036956in}{0.664394in}}{\pgfqpoint{1.047555in}{0.668784in}}{\pgfqpoint{1.055369in}{0.676598in}}%
\pgfpathcurveto{\pgfqpoint{1.063182in}{0.684411in}}{\pgfqpoint{1.067573in}{0.695010in}}{\pgfqpoint{1.067573in}{0.706060in}}%
\pgfpathcurveto{\pgfqpoint{1.067573in}{0.717111in}}{\pgfqpoint{1.063182in}{0.727710in}}{\pgfqpoint{1.055369in}{0.735523in}}%
\pgfpathcurveto{\pgfqpoint{1.047555in}{0.743337in}}{\pgfqpoint{1.036956in}{0.747727in}}{\pgfqpoint{1.025906in}{0.747727in}}%
\pgfpathcurveto{\pgfqpoint{1.014856in}{0.747727in}}{\pgfqpoint{1.004257in}{0.743337in}}{\pgfqpoint{0.996443in}{0.735523in}}%
\pgfpathcurveto{\pgfqpoint{0.988630in}{0.727710in}}{\pgfqpoint{0.984239in}{0.717111in}}{\pgfqpoint{0.984239in}{0.706060in}}%
\pgfpathcurveto{\pgfqpoint{0.984239in}{0.695010in}}{\pgfqpoint{0.988630in}{0.684411in}}{\pgfqpoint{0.996443in}{0.676598in}}%
\pgfpathcurveto{\pgfqpoint{1.004257in}{0.668784in}}{\pgfqpoint{1.014856in}{0.664394in}}{\pgfqpoint{1.025906in}{0.664394in}}%
\pgfpathclose%
\pgfusepath{stroke,fill}%
\end{pgfscope}%
\begin{pgfscope}%
\pgfpathrectangle{\pgfqpoint{0.800000in}{0.528000in}}{\pgfqpoint{4.960000in}{3.696000in}}%
\pgfusepath{clip}%
\pgfsetbuttcap%
\pgfsetroundjoin%
\definecolor{currentfill}{rgb}{0.000000,0.000000,0.000000}%
\pgfsetfillcolor{currentfill}%
\pgfsetlinewidth{1.003750pt}%
\definecolor{currentstroke}{rgb}{0.000000,0.000000,0.000000}%
\pgfsetstrokecolor{currentstroke}%
\pgfsetdash{}{0pt}%
\pgfpathmoveto{\pgfqpoint{1.025906in}{0.664394in}}%
\pgfpathcurveto{\pgfqpoint{1.036956in}{0.664394in}}{\pgfqpoint{1.047555in}{0.668784in}}{\pgfqpoint{1.055369in}{0.676598in}}%
\pgfpathcurveto{\pgfqpoint{1.063182in}{0.684411in}}{\pgfqpoint{1.067573in}{0.695010in}}{\pgfqpoint{1.067573in}{0.706060in}}%
\pgfpathcurveto{\pgfqpoint{1.067573in}{0.717111in}}{\pgfqpoint{1.063182in}{0.727710in}}{\pgfqpoint{1.055369in}{0.735523in}}%
\pgfpathcurveto{\pgfqpoint{1.047555in}{0.743337in}}{\pgfqpoint{1.036956in}{0.747727in}}{\pgfqpoint{1.025906in}{0.747727in}}%
\pgfpathcurveto{\pgfqpoint{1.014856in}{0.747727in}}{\pgfqpoint{1.004257in}{0.743337in}}{\pgfqpoint{0.996443in}{0.735523in}}%
\pgfpathcurveto{\pgfqpoint{0.988630in}{0.727710in}}{\pgfqpoint{0.984239in}{0.717111in}}{\pgfqpoint{0.984239in}{0.706060in}}%
\pgfpathcurveto{\pgfqpoint{0.984239in}{0.695010in}}{\pgfqpoint{0.988630in}{0.684411in}}{\pgfqpoint{0.996443in}{0.676598in}}%
\pgfpathcurveto{\pgfqpoint{1.004257in}{0.668784in}}{\pgfqpoint{1.014856in}{0.664394in}}{\pgfqpoint{1.025906in}{0.664394in}}%
\pgfpathclose%
\pgfusepath{stroke,fill}%
\end{pgfscope}%
\begin{pgfscope}%
\pgfpathrectangle{\pgfqpoint{0.800000in}{0.528000in}}{\pgfqpoint{4.960000in}{3.696000in}}%
\pgfusepath{clip}%
\pgfsetbuttcap%
\pgfsetroundjoin%
\definecolor{currentfill}{rgb}{0.000000,0.000000,0.000000}%
\pgfsetfillcolor{currentfill}%
\pgfsetlinewidth{1.003750pt}%
\definecolor{currentstroke}{rgb}{0.000000,0.000000,0.000000}%
\pgfsetstrokecolor{currentstroke}%
\pgfsetdash{}{0pt}%
\pgfpathmoveto{\pgfqpoint{1.025906in}{0.664394in}}%
\pgfpathcurveto{\pgfqpoint{1.036956in}{0.664394in}}{\pgfqpoint{1.047555in}{0.668784in}}{\pgfqpoint{1.055369in}{0.676598in}}%
\pgfpathcurveto{\pgfqpoint{1.063182in}{0.684411in}}{\pgfqpoint{1.067573in}{0.695010in}}{\pgfqpoint{1.067573in}{0.706060in}}%
\pgfpathcurveto{\pgfqpoint{1.067573in}{0.717111in}}{\pgfqpoint{1.063182in}{0.727710in}}{\pgfqpoint{1.055369in}{0.735523in}}%
\pgfpathcurveto{\pgfqpoint{1.047555in}{0.743337in}}{\pgfqpoint{1.036956in}{0.747727in}}{\pgfqpoint{1.025906in}{0.747727in}}%
\pgfpathcurveto{\pgfqpoint{1.014856in}{0.747727in}}{\pgfqpoint{1.004257in}{0.743337in}}{\pgfqpoint{0.996443in}{0.735523in}}%
\pgfpathcurveto{\pgfqpoint{0.988630in}{0.727710in}}{\pgfqpoint{0.984239in}{0.717111in}}{\pgfqpoint{0.984239in}{0.706060in}}%
\pgfpathcurveto{\pgfqpoint{0.984239in}{0.695010in}}{\pgfqpoint{0.988630in}{0.684411in}}{\pgfqpoint{0.996443in}{0.676598in}}%
\pgfpathcurveto{\pgfqpoint{1.004257in}{0.668784in}}{\pgfqpoint{1.014856in}{0.664394in}}{\pgfqpoint{1.025906in}{0.664394in}}%
\pgfpathclose%
\pgfusepath{stroke,fill}%
\end{pgfscope}%
\begin{pgfscope}%
\pgfpathrectangle{\pgfqpoint{0.800000in}{0.528000in}}{\pgfqpoint{4.960000in}{3.696000in}}%
\pgfusepath{clip}%
\pgfsetbuttcap%
\pgfsetroundjoin%
\definecolor{currentfill}{rgb}{0.000000,0.000000,0.000000}%
\pgfsetfillcolor{currentfill}%
\pgfsetlinewidth{1.003750pt}%
\definecolor{currentstroke}{rgb}{0.000000,0.000000,0.000000}%
\pgfsetstrokecolor{currentstroke}%
\pgfsetdash{}{0pt}%
\pgfpathmoveto{\pgfqpoint{1.025906in}{0.664394in}}%
\pgfpathcurveto{\pgfqpoint{1.036956in}{0.664394in}}{\pgfqpoint{1.047555in}{0.668784in}}{\pgfqpoint{1.055369in}{0.676598in}}%
\pgfpathcurveto{\pgfqpoint{1.063182in}{0.684411in}}{\pgfqpoint{1.067573in}{0.695010in}}{\pgfqpoint{1.067573in}{0.706060in}}%
\pgfpathcurveto{\pgfqpoint{1.067573in}{0.717111in}}{\pgfqpoint{1.063182in}{0.727710in}}{\pgfqpoint{1.055369in}{0.735523in}}%
\pgfpathcurveto{\pgfqpoint{1.047555in}{0.743337in}}{\pgfqpoint{1.036956in}{0.747727in}}{\pgfqpoint{1.025906in}{0.747727in}}%
\pgfpathcurveto{\pgfqpoint{1.014856in}{0.747727in}}{\pgfqpoint{1.004257in}{0.743337in}}{\pgfqpoint{0.996443in}{0.735523in}}%
\pgfpathcurveto{\pgfqpoint{0.988630in}{0.727710in}}{\pgfqpoint{0.984239in}{0.717111in}}{\pgfqpoint{0.984239in}{0.706060in}}%
\pgfpathcurveto{\pgfqpoint{0.984239in}{0.695010in}}{\pgfqpoint{0.988630in}{0.684411in}}{\pgfqpoint{0.996443in}{0.676598in}}%
\pgfpathcurveto{\pgfqpoint{1.004257in}{0.668784in}}{\pgfqpoint{1.014856in}{0.664394in}}{\pgfqpoint{1.025906in}{0.664394in}}%
\pgfpathclose%
\pgfusepath{stroke,fill}%
\end{pgfscope}%
\begin{pgfscope}%
\pgfpathrectangle{\pgfqpoint{0.800000in}{0.528000in}}{\pgfqpoint{4.960000in}{3.696000in}}%
\pgfusepath{clip}%
\pgfsetbuttcap%
\pgfsetroundjoin%
\definecolor{currentfill}{rgb}{0.000000,0.000000,0.000000}%
\pgfsetfillcolor{currentfill}%
\pgfsetlinewidth{1.003750pt}%
\definecolor{currentstroke}{rgb}{0.000000,0.000000,0.000000}%
\pgfsetstrokecolor{currentstroke}%
\pgfsetdash{}{0pt}%
\pgfpathmoveto{\pgfqpoint{1.025906in}{0.664394in}}%
\pgfpathcurveto{\pgfqpoint{1.036956in}{0.664394in}}{\pgfqpoint{1.047555in}{0.668784in}}{\pgfqpoint{1.055369in}{0.676598in}}%
\pgfpathcurveto{\pgfqpoint{1.063182in}{0.684411in}}{\pgfqpoint{1.067573in}{0.695010in}}{\pgfqpoint{1.067573in}{0.706060in}}%
\pgfpathcurveto{\pgfqpoint{1.067573in}{0.717111in}}{\pgfqpoint{1.063182in}{0.727710in}}{\pgfqpoint{1.055369in}{0.735523in}}%
\pgfpathcurveto{\pgfqpoint{1.047555in}{0.743337in}}{\pgfqpoint{1.036956in}{0.747727in}}{\pgfqpoint{1.025906in}{0.747727in}}%
\pgfpathcurveto{\pgfqpoint{1.014856in}{0.747727in}}{\pgfqpoint{1.004257in}{0.743337in}}{\pgfqpoint{0.996443in}{0.735523in}}%
\pgfpathcurveto{\pgfqpoint{0.988630in}{0.727710in}}{\pgfqpoint{0.984239in}{0.717111in}}{\pgfqpoint{0.984239in}{0.706060in}}%
\pgfpathcurveto{\pgfqpoint{0.984239in}{0.695010in}}{\pgfqpoint{0.988630in}{0.684411in}}{\pgfqpoint{0.996443in}{0.676598in}}%
\pgfpathcurveto{\pgfqpoint{1.004257in}{0.668784in}}{\pgfqpoint{1.014856in}{0.664394in}}{\pgfqpoint{1.025906in}{0.664394in}}%
\pgfpathclose%
\pgfusepath{stroke,fill}%
\end{pgfscope}%
\begin{pgfscope}%
\pgfpathrectangle{\pgfqpoint{0.800000in}{0.528000in}}{\pgfqpoint{4.960000in}{3.696000in}}%
\pgfusepath{clip}%
\pgfsetbuttcap%
\pgfsetroundjoin%
\definecolor{currentfill}{rgb}{0.000000,0.000000,0.000000}%
\pgfsetfillcolor{currentfill}%
\pgfsetlinewidth{1.003750pt}%
\definecolor{currentstroke}{rgb}{0.000000,0.000000,0.000000}%
\pgfsetstrokecolor{currentstroke}%
\pgfsetdash{}{0pt}%
\pgfpathmoveto{\pgfqpoint{1.025906in}{0.664394in}}%
\pgfpathcurveto{\pgfqpoint{1.036956in}{0.664394in}}{\pgfqpoint{1.047555in}{0.668784in}}{\pgfqpoint{1.055369in}{0.676598in}}%
\pgfpathcurveto{\pgfqpoint{1.063182in}{0.684411in}}{\pgfqpoint{1.067573in}{0.695010in}}{\pgfqpoint{1.067573in}{0.706060in}}%
\pgfpathcurveto{\pgfqpoint{1.067573in}{0.717111in}}{\pgfqpoint{1.063182in}{0.727710in}}{\pgfqpoint{1.055369in}{0.735523in}}%
\pgfpathcurveto{\pgfqpoint{1.047555in}{0.743337in}}{\pgfqpoint{1.036956in}{0.747727in}}{\pgfqpoint{1.025906in}{0.747727in}}%
\pgfpathcurveto{\pgfqpoint{1.014856in}{0.747727in}}{\pgfqpoint{1.004257in}{0.743337in}}{\pgfqpoint{0.996443in}{0.735523in}}%
\pgfpathcurveto{\pgfqpoint{0.988630in}{0.727710in}}{\pgfqpoint{0.984239in}{0.717111in}}{\pgfqpoint{0.984239in}{0.706060in}}%
\pgfpathcurveto{\pgfqpoint{0.984239in}{0.695010in}}{\pgfqpoint{0.988630in}{0.684411in}}{\pgfqpoint{0.996443in}{0.676598in}}%
\pgfpathcurveto{\pgfqpoint{1.004257in}{0.668784in}}{\pgfqpoint{1.014856in}{0.664394in}}{\pgfqpoint{1.025906in}{0.664394in}}%
\pgfpathclose%
\pgfusepath{stroke,fill}%
\end{pgfscope}%
\begin{pgfscope}%
\pgfpathrectangle{\pgfqpoint{0.800000in}{0.528000in}}{\pgfqpoint{4.960000in}{3.696000in}}%
\pgfusepath{clip}%
\pgfsetbuttcap%
\pgfsetroundjoin%
\definecolor{currentfill}{rgb}{0.000000,0.000000,0.000000}%
\pgfsetfillcolor{currentfill}%
\pgfsetlinewidth{1.003750pt}%
\definecolor{currentstroke}{rgb}{0.000000,0.000000,0.000000}%
\pgfsetstrokecolor{currentstroke}%
\pgfsetdash{}{0pt}%
\pgfpathmoveto{\pgfqpoint{1.025906in}{0.664394in}}%
\pgfpathcurveto{\pgfqpoint{1.036956in}{0.664394in}}{\pgfqpoint{1.047555in}{0.668784in}}{\pgfqpoint{1.055369in}{0.676598in}}%
\pgfpathcurveto{\pgfqpoint{1.063182in}{0.684411in}}{\pgfqpoint{1.067573in}{0.695010in}}{\pgfqpoint{1.067573in}{0.706060in}}%
\pgfpathcurveto{\pgfqpoint{1.067573in}{0.717111in}}{\pgfqpoint{1.063182in}{0.727710in}}{\pgfqpoint{1.055369in}{0.735523in}}%
\pgfpathcurveto{\pgfqpoint{1.047555in}{0.743337in}}{\pgfqpoint{1.036956in}{0.747727in}}{\pgfqpoint{1.025906in}{0.747727in}}%
\pgfpathcurveto{\pgfqpoint{1.014856in}{0.747727in}}{\pgfqpoint{1.004257in}{0.743337in}}{\pgfqpoint{0.996443in}{0.735523in}}%
\pgfpathcurveto{\pgfqpoint{0.988630in}{0.727710in}}{\pgfqpoint{0.984239in}{0.717111in}}{\pgfqpoint{0.984239in}{0.706060in}}%
\pgfpathcurveto{\pgfqpoint{0.984239in}{0.695010in}}{\pgfqpoint{0.988630in}{0.684411in}}{\pgfqpoint{0.996443in}{0.676598in}}%
\pgfpathcurveto{\pgfqpoint{1.004257in}{0.668784in}}{\pgfqpoint{1.014856in}{0.664394in}}{\pgfqpoint{1.025906in}{0.664394in}}%
\pgfpathclose%
\pgfusepath{stroke,fill}%
\end{pgfscope}%
\begin{pgfscope}%
\pgfpathrectangle{\pgfqpoint{0.800000in}{0.528000in}}{\pgfqpoint{4.960000in}{3.696000in}}%
\pgfusepath{clip}%
\pgfsetbuttcap%
\pgfsetroundjoin%
\definecolor{currentfill}{rgb}{0.000000,0.000000,0.000000}%
\pgfsetfillcolor{currentfill}%
\pgfsetlinewidth{1.003750pt}%
\definecolor{currentstroke}{rgb}{0.000000,0.000000,0.000000}%
\pgfsetstrokecolor{currentstroke}%
\pgfsetdash{}{0pt}%
\pgfpathmoveto{\pgfqpoint{1.025906in}{0.664394in}}%
\pgfpathcurveto{\pgfqpoint{1.036956in}{0.664394in}}{\pgfqpoint{1.047555in}{0.668784in}}{\pgfqpoint{1.055369in}{0.676598in}}%
\pgfpathcurveto{\pgfqpoint{1.063182in}{0.684411in}}{\pgfqpoint{1.067573in}{0.695010in}}{\pgfqpoint{1.067573in}{0.706060in}}%
\pgfpathcurveto{\pgfqpoint{1.067573in}{0.717111in}}{\pgfqpoint{1.063182in}{0.727710in}}{\pgfqpoint{1.055369in}{0.735523in}}%
\pgfpathcurveto{\pgfqpoint{1.047555in}{0.743337in}}{\pgfqpoint{1.036956in}{0.747727in}}{\pgfqpoint{1.025906in}{0.747727in}}%
\pgfpathcurveto{\pgfqpoint{1.014856in}{0.747727in}}{\pgfqpoint{1.004257in}{0.743337in}}{\pgfqpoint{0.996443in}{0.735523in}}%
\pgfpathcurveto{\pgfqpoint{0.988630in}{0.727710in}}{\pgfqpoint{0.984239in}{0.717111in}}{\pgfqpoint{0.984239in}{0.706060in}}%
\pgfpathcurveto{\pgfqpoint{0.984239in}{0.695010in}}{\pgfqpoint{0.988630in}{0.684411in}}{\pgfqpoint{0.996443in}{0.676598in}}%
\pgfpathcurveto{\pgfqpoint{1.004257in}{0.668784in}}{\pgfqpoint{1.014856in}{0.664394in}}{\pgfqpoint{1.025906in}{0.664394in}}%
\pgfpathclose%
\pgfusepath{stroke,fill}%
\end{pgfscope}%
\begin{pgfscope}%
\pgfpathrectangle{\pgfqpoint{0.800000in}{0.528000in}}{\pgfqpoint{4.960000in}{3.696000in}}%
\pgfusepath{clip}%
\pgfsetbuttcap%
\pgfsetroundjoin%
\definecolor{currentfill}{rgb}{0.000000,0.000000,0.000000}%
\pgfsetfillcolor{currentfill}%
\pgfsetlinewidth{1.003750pt}%
\definecolor{currentstroke}{rgb}{0.000000,0.000000,0.000000}%
\pgfsetstrokecolor{currentstroke}%
\pgfsetdash{}{0pt}%
\pgfpathmoveto{\pgfqpoint{1.025906in}{0.664394in}}%
\pgfpathcurveto{\pgfqpoint{1.036956in}{0.664394in}}{\pgfqpoint{1.047555in}{0.668784in}}{\pgfqpoint{1.055369in}{0.676598in}}%
\pgfpathcurveto{\pgfqpoint{1.063182in}{0.684411in}}{\pgfqpoint{1.067573in}{0.695010in}}{\pgfqpoint{1.067573in}{0.706060in}}%
\pgfpathcurveto{\pgfqpoint{1.067573in}{0.717111in}}{\pgfqpoint{1.063182in}{0.727710in}}{\pgfqpoint{1.055369in}{0.735523in}}%
\pgfpathcurveto{\pgfqpoint{1.047555in}{0.743337in}}{\pgfqpoint{1.036956in}{0.747727in}}{\pgfqpoint{1.025906in}{0.747727in}}%
\pgfpathcurveto{\pgfqpoint{1.014856in}{0.747727in}}{\pgfqpoint{1.004257in}{0.743337in}}{\pgfqpoint{0.996443in}{0.735523in}}%
\pgfpathcurveto{\pgfqpoint{0.988630in}{0.727710in}}{\pgfqpoint{0.984239in}{0.717111in}}{\pgfqpoint{0.984239in}{0.706060in}}%
\pgfpathcurveto{\pgfqpoint{0.984239in}{0.695010in}}{\pgfqpoint{0.988630in}{0.684411in}}{\pgfqpoint{0.996443in}{0.676598in}}%
\pgfpathcurveto{\pgfqpoint{1.004257in}{0.668784in}}{\pgfqpoint{1.014856in}{0.664394in}}{\pgfqpoint{1.025906in}{0.664394in}}%
\pgfpathclose%
\pgfusepath{stroke,fill}%
\end{pgfscope}%
\begin{pgfscope}%
\pgfpathrectangle{\pgfqpoint{0.800000in}{0.528000in}}{\pgfqpoint{4.960000in}{3.696000in}}%
\pgfusepath{clip}%
\pgfsetbuttcap%
\pgfsetroundjoin%
\definecolor{currentfill}{rgb}{0.000000,0.000000,0.000000}%
\pgfsetfillcolor{currentfill}%
\pgfsetlinewidth{1.003750pt}%
\definecolor{currentstroke}{rgb}{0.000000,0.000000,0.000000}%
\pgfsetstrokecolor{currentstroke}%
\pgfsetdash{}{0pt}%
\pgfpathmoveto{\pgfqpoint{1.025906in}{0.664394in}}%
\pgfpathcurveto{\pgfqpoint{1.036956in}{0.664394in}}{\pgfqpoint{1.047555in}{0.668784in}}{\pgfqpoint{1.055369in}{0.676598in}}%
\pgfpathcurveto{\pgfqpoint{1.063182in}{0.684411in}}{\pgfqpoint{1.067573in}{0.695010in}}{\pgfqpoint{1.067573in}{0.706060in}}%
\pgfpathcurveto{\pgfqpoint{1.067573in}{0.717111in}}{\pgfqpoint{1.063182in}{0.727710in}}{\pgfqpoint{1.055369in}{0.735523in}}%
\pgfpathcurveto{\pgfqpoint{1.047555in}{0.743337in}}{\pgfqpoint{1.036956in}{0.747727in}}{\pgfqpoint{1.025906in}{0.747727in}}%
\pgfpathcurveto{\pgfqpoint{1.014856in}{0.747727in}}{\pgfqpoint{1.004257in}{0.743337in}}{\pgfqpoint{0.996443in}{0.735523in}}%
\pgfpathcurveto{\pgfqpoint{0.988630in}{0.727710in}}{\pgfqpoint{0.984239in}{0.717111in}}{\pgfqpoint{0.984239in}{0.706060in}}%
\pgfpathcurveto{\pgfqpoint{0.984239in}{0.695010in}}{\pgfqpoint{0.988630in}{0.684411in}}{\pgfqpoint{0.996443in}{0.676598in}}%
\pgfpathcurveto{\pgfqpoint{1.004257in}{0.668784in}}{\pgfqpoint{1.014856in}{0.664394in}}{\pgfqpoint{1.025906in}{0.664394in}}%
\pgfpathclose%
\pgfusepath{stroke,fill}%
\end{pgfscope}%
\begin{pgfscope}%
\pgfpathrectangle{\pgfqpoint{0.800000in}{0.528000in}}{\pgfqpoint{4.960000in}{3.696000in}}%
\pgfusepath{clip}%
\pgfsetbuttcap%
\pgfsetroundjoin%
\definecolor{currentfill}{rgb}{0.000000,0.000000,0.000000}%
\pgfsetfillcolor{currentfill}%
\pgfsetlinewidth{1.003750pt}%
\definecolor{currentstroke}{rgb}{0.000000,0.000000,0.000000}%
\pgfsetstrokecolor{currentstroke}%
\pgfsetdash{}{0pt}%
\pgfpathmoveto{\pgfqpoint{1.025906in}{0.664394in}}%
\pgfpathcurveto{\pgfqpoint{1.036956in}{0.664394in}}{\pgfqpoint{1.047555in}{0.668784in}}{\pgfqpoint{1.055369in}{0.676598in}}%
\pgfpathcurveto{\pgfqpoint{1.063182in}{0.684411in}}{\pgfqpoint{1.067573in}{0.695010in}}{\pgfqpoint{1.067573in}{0.706060in}}%
\pgfpathcurveto{\pgfqpoint{1.067573in}{0.717111in}}{\pgfqpoint{1.063182in}{0.727710in}}{\pgfqpoint{1.055369in}{0.735523in}}%
\pgfpathcurveto{\pgfqpoint{1.047555in}{0.743337in}}{\pgfqpoint{1.036956in}{0.747727in}}{\pgfqpoint{1.025906in}{0.747727in}}%
\pgfpathcurveto{\pgfqpoint{1.014856in}{0.747727in}}{\pgfqpoint{1.004257in}{0.743337in}}{\pgfqpoint{0.996443in}{0.735523in}}%
\pgfpathcurveto{\pgfqpoint{0.988630in}{0.727710in}}{\pgfqpoint{0.984239in}{0.717111in}}{\pgfqpoint{0.984239in}{0.706060in}}%
\pgfpathcurveto{\pgfqpoint{0.984239in}{0.695010in}}{\pgfqpoint{0.988630in}{0.684411in}}{\pgfqpoint{0.996443in}{0.676598in}}%
\pgfpathcurveto{\pgfqpoint{1.004257in}{0.668784in}}{\pgfqpoint{1.014856in}{0.664394in}}{\pgfqpoint{1.025906in}{0.664394in}}%
\pgfpathclose%
\pgfusepath{stroke,fill}%
\end{pgfscope}%
\begin{pgfscope}%
\pgfpathrectangle{\pgfqpoint{0.800000in}{0.528000in}}{\pgfqpoint{4.960000in}{3.696000in}}%
\pgfusepath{clip}%
\pgfsetbuttcap%
\pgfsetroundjoin%
\definecolor{currentfill}{rgb}{0.000000,0.000000,0.000000}%
\pgfsetfillcolor{currentfill}%
\pgfsetlinewidth{1.003750pt}%
\definecolor{currentstroke}{rgb}{0.000000,0.000000,0.000000}%
\pgfsetstrokecolor{currentstroke}%
\pgfsetdash{}{0pt}%
\pgfpathmoveto{\pgfqpoint{1.025906in}{0.664394in}}%
\pgfpathcurveto{\pgfqpoint{1.036956in}{0.664394in}}{\pgfqpoint{1.047555in}{0.668784in}}{\pgfqpoint{1.055369in}{0.676598in}}%
\pgfpathcurveto{\pgfqpoint{1.063182in}{0.684411in}}{\pgfqpoint{1.067573in}{0.695010in}}{\pgfqpoint{1.067573in}{0.706060in}}%
\pgfpathcurveto{\pgfqpoint{1.067573in}{0.717111in}}{\pgfqpoint{1.063182in}{0.727710in}}{\pgfqpoint{1.055369in}{0.735523in}}%
\pgfpathcurveto{\pgfqpoint{1.047555in}{0.743337in}}{\pgfqpoint{1.036956in}{0.747727in}}{\pgfqpoint{1.025906in}{0.747727in}}%
\pgfpathcurveto{\pgfqpoint{1.014856in}{0.747727in}}{\pgfqpoint{1.004257in}{0.743337in}}{\pgfqpoint{0.996443in}{0.735523in}}%
\pgfpathcurveto{\pgfqpoint{0.988630in}{0.727710in}}{\pgfqpoint{0.984239in}{0.717111in}}{\pgfqpoint{0.984239in}{0.706060in}}%
\pgfpathcurveto{\pgfqpoint{0.984239in}{0.695010in}}{\pgfqpoint{0.988630in}{0.684411in}}{\pgfqpoint{0.996443in}{0.676598in}}%
\pgfpathcurveto{\pgfqpoint{1.004257in}{0.668784in}}{\pgfqpoint{1.014856in}{0.664394in}}{\pgfqpoint{1.025906in}{0.664394in}}%
\pgfpathclose%
\pgfusepath{stroke,fill}%
\end{pgfscope}%
\begin{pgfscope}%
\pgfpathrectangle{\pgfqpoint{0.800000in}{0.528000in}}{\pgfqpoint{4.960000in}{3.696000in}}%
\pgfusepath{clip}%
\pgfsetbuttcap%
\pgfsetroundjoin%
\definecolor{currentfill}{rgb}{0.000000,0.000000,0.000000}%
\pgfsetfillcolor{currentfill}%
\pgfsetlinewidth{1.003750pt}%
\definecolor{currentstroke}{rgb}{0.000000,0.000000,0.000000}%
\pgfsetstrokecolor{currentstroke}%
\pgfsetdash{}{0pt}%
\pgfpathmoveto{\pgfqpoint{1.025906in}{0.664394in}}%
\pgfpathcurveto{\pgfqpoint{1.036956in}{0.664394in}}{\pgfqpoint{1.047555in}{0.668784in}}{\pgfqpoint{1.055369in}{0.676598in}}%
\pgfpathcurveto{\pgfqpoint{1.063182in}{0.684411in}}{\pgfqpoint{1.067573in}{0.695010in}}{\pgfqpoint{1.067573in}{0.706060in}}%
\pgfpathcurveto{\pgfqpoint{1.067573in}{0.717111in}}{\pgfqpoint{1.063182in}{0.727710in}}{\pgfqpoint{1.055369in}{0.735523in}}%
\pgfpathcurveto{\pgfqpoint{1.047555in}{0.743337in}}{\pgfqpoint{1.036956in}{0.747727in}}{\pgfqpoint{1.025906in}{0.747727in}}%
\pgfpathcurveto{\pgfqpoint{1.014856in}{0.747727in}}{\pgfqpoint{1.004257in}{0.743337in}}{\pgfqpoint{0.996443in}{0.735523in}}%
\pgfpathcurveto{\pgfqpoint{0.988630in}{0.727710in}}{\pgfqpoint{0.984239in}{0.717111in}}{\pgfqpoint{0.984239in}{0.706060in}}%
\pgfpathcurveto{\pgfqpoint{0.984239in}{0.695010in}}{\pgfqpoint{0.988630in}{0.684411in}}{\pgfqpoint{0.996443in}{0.676598in}}%
\pgfpathcurveto{\pgfqpoint{1.004257in}{0.668784in}}{\pgfqpoint{1.014856in}{0.664394in}}{\pgfqpoint{1.025906in}{0.664394in}}%
\pgfpathclose%
\pgfusepath{stroke,fill}%
\end{pgfscope}%
\begin{pgfscope}%
\pgfpathrectangle{\pgfqpoint{0.800000in}{0.528000in}}{\pgfqpoint{4.960000in}{3.696000in}}%
\pgfusepath{clip}%
\pgfsetbuttcap%
\pgfsetroundjoin%
\definecolor{currentfill}{rgb}{0.000000,0.000000,0.000000}%
\pgfsetfillcolor{currentfill}%
\pgfsetlinewidth{1.003750pt}%
\definecolor{currentstroke}{rgb}{0.000000,0.000000,0.000000}%
\pgfsetstrokecolor{currentstroke}%
\pgfsetdash{}{0pt}%
\pgfpathmoveto{\pgfqpoint{1.025906in}{0.664394in}}%
\pgfpathcurveto{\pgfqpoint{1.036956in}{0.664394in}}{\pgfqpoint{1.047555in}{0.668784in}}{\pgfqpoint{1.055369in}{0.676598in}}%
\pgfpathcurveto{\pgfqpoint{1.063182in}{0.684411in}}{\pgfqpoint{1.067573in}{0.695010in}}{\pgfqpoint{1.067573in}{0.706060in}}%
\pgfpathcurveto{\pgfqpoint{1.067573in}{0.717111in}}{\pgfqpoint{1.063182in}{0.727710in}}{\pgfqpoint{1.055369in}{0.735523in}}%
\pgfpathcurveto{\pgfqpoint{1.047555in}{0.743337in}}{\pgfqpoint{1.036956in}{0.747727in}}{\pgfqpoint{1.025906in}{0.747727in}}%
\pgfpathcurveto{\pgfqpoint{1.014856in}{0.747727in}}{\pgfqpoint{1.004257in}{0.743337in}}{\pgfqpoint{0.996443in}{0.735523in}}%
\pgfpathcurveto{\pgfqpoint{0.988630in}{0.727710in}}{\pgfqpoint{0.984239in}{0.717111in}}{\pgfqpoint{0.984239in}{0.706060in}}%
\pgfpathcurveto{\pgfqpoint{0.984239in}{0.695010in}}{\pgfqpoint{0.988630in}{0.684411in}}{\pgfqpoint{0.996443in}{0.676598in}}%
\pgfpathcurveto{\pgfqpoint{1.004257in}{0.668784in}}{\pgfqpoint{1.014856in}{0.664394in}}{\pgfqpoint{1.025906in}{0.664394in}}%
\pgfpathclose%
\pgfusepath{stroke,fill}%
\end{pgfscope}%
\begin{pgfscope}%
\pgfpathrectangle{\pgfqpoint{0.800000in}{0.528000in}}{\pgfqpoint{4.960000in}{3.696000in}}%
\pgfusepath{clip}%
\pgfsetbuttcap%
\pgfsetroundjoin%
\definecolor{currentfill}{rgb}{0.000000,0.000000,0.000000}%
\pgfsetfillcolor{currentfill}%
\pgfsetlinewidth{1.003750pt}%
\definecolor{currentstroke}{rgb}{0.000000,0.000000,0.000000}%
\pgfsetstrokecolor{currentstroke}%
\pgfsetdash{}{0pt}%
\pgfpathmoveto{\pgfqpoint{1.025906in}{0.664394in}}%
\pgfpathcurveto{\pgfqpoint{1.036956in}{0.664394in}}{\pgfqpoint{1.047555in}{0.668784in}}{\pgfqpoint{1.055369in}{0.676598in}}%
\pgfpathcurveto{\pgfqpoint{1.063182in}{0.684411in}}{\pgfqpoint{1.067573in}{0.695010in}}{\pgfqpoint{1.067573in}{0.706060in}}%
\pgfpathcurveto{\pgfqpoint{1.067573in}{0.717111in}}{\pgfqpoint{1.063182in}{0.727710in}}{\pgfqpoint{1.055369in}{0.735523in}}%
\pgfpathcurveto{\pgfqpoint{1.047555in}{0.743337in}}{\pgfqpoint{1.036956in}{0.747727in}}{\pgfqpoint{1.025906in}{0.747727in}}%
\pgfpathcurveto{\pgfqpoint{1.014856in}{0.747727in}}{\pgfqpoint{1.004257in}{0.743337in}}{\pgfqpoint{0.996443in}{0.735523in}}%
\pgfpathcurveto{\pgfqpoint{0.988630in}{0.727710in}}{\pgfqpoint{0.984239in}{0.717111in}}{\pgfqpoint{0.984239in}{0.706060in}}%
\pgfpathcurveto{\pgfqpoint{0.984239in}{0.695010in}}{\pgfqpoint{0.988630in}{0.684411in}}{\pgfqpoint{0.996443in}{0.676598in}}%
\pgfpathcurveto{\pgfqpoint{1.004257in}{0.668784in}}{\pgfqpoint{1.014856in}{0.664394in}}{\pgfqpoint{1.025906in}{0.664394in}}%
\pgfpathclose%
\pgfusepath{stroke,fill}%
\end{pgfscope}%
\begin{pgfscope}%
\pgfpathrectangle{\pgfqpoint{0.800000in}{0.528000in}}{\pgfqpoint{4.960000in}{3.696000in}}%
\pgfusepath{clip}%
\pgfsetbuttcap%
\pgfsetroundjoin%
\definecolor{currentfill}{rgb}{0.000000,0.000000,0.000000}%
\pgfsetfillcolor{currentfill}%
\pgfsetlinewidth{1.003750pt}%
\definecolor{currentstroke}{rgb}{0.000000,0.000000,0.000000}%
\pgfsetstrokecolor{currentstroke}%
\pgfsetdash{}{0pt}%
\pgfpathmoveto{\pgfqpoint{1.025906in}{0.664394in}}%
\pgfpathcurveto{\pgfqpoint{1.036956in}{0.664394in}}{\pgfqpoint{1.047555in}{0.668784in}}{\pgfqpoint{1.055369in}{0.676598in}}%
\pgfpathcurveto{\pgfqpoint{1.063182in}{0.684411in}}{\pgfqpoint{1.067573in}{0.695010in}}{\pgfqpoint{1.067573in}{0.706060in}}%
\pgfpathcurveto{\pgfqpoint{1.067573in}{0.717111in}}{\pgfqpoint{1.063182in}{0.727710in}}{\pgfqpoint{1.055369in}{0.735523in}}%
\pgfpathcurveto{\pgfqpoint{1.047555in}{0.743337in}}{\pgfqpoint{1.036956in}{0.747727in}}{\pgfqpoint{1.025906in}{0.747727in}}%
\pgfpathcurveto{\pgfqpoint{1.014856in}{0.747727in}}{\pgfqpoint{1.004257in}{0.743337in}}{\pgfqpoint{0.996443in}{0.735523in}}%
\pgfpathcurveto{\pgfqpoint{0.988630in}{0.727710in}}{\pgfqpoint{0.984239in}{0.717111in}}{\pgfqpoint{0.984239in}{0.706060in}}%
\pgfpathcurveto{\pgfqpoint{0.984239in}{0.695010in}}{\pgfqpoint{0.988630in}{0.684411in}}{\pgfqpoint{0.996443in}{0.676598in}}%
\pgfpathcurveto{\pgfqpoint{1.004257in}{0.668784in}}{\pgfqpoint{1.014856in}{0.664394in}}{\pgfqpoint{1.025906in}{0.664394in}}%
\pgfpathclose%
\pgfusepath{stroke,fill}%
\end{pgfscope}%
\begin{pgfscope}%
\pgfpathrectangle{\pgfqpoint{0.800000in}{0.528000in}}{\pgfqpoint{4.960000in}{3.696000in}}%
\pgfusepath{clip}%
\pgfsetbuttcap%
\pgfsetroundjoin%
\definecolor{currentfill}{rgb}{0.000000,0.000000,0.000000}%
\pgfsetfillcolor{currentfill}%
\pgfsetlinewidth{1.003750pt}%
\definecolor{currentstroke}{rgb}{0.000000,0.000000,0.000000}%
\pgfsetstrokecolor{currentstroke}%
\pgfsetdash{}{0pt}%
\pgfpathmoveto{\pgfqpoint{1.025906in}{0.664394in}}%
\pgfpathcurveto{\pgfqpoint{1.036956in}{0.664394in}}{\pgfqpoint{1.047555in}{0.668784in}}{\pgfqpoint{1.055369in}{0.676598in}}%
\pgfpathcurveto{\pgfqpoint{1.063182in}{0.684411in}}{\pgfqpoint{1.067573in}{0.695010in}}{\pgfqpoint{1.067573in}{0.706060in}}%
\pgfpathcurveto{\pgfqpoint{1.067573in}{0.717111in}}{\pgfqpoint{1.063182in}{0.727710in}}{\pgfqpoint{1.055369in}{0.735523in}}%
\pgfpathcurveto{\pgfqpoint{1.047555in}{0.743337in}}{\pgfqpoint{1.036956in}{0.747727in}}{\pgfqpoint{1.025906in}{0.747727in}}%
\pgfpathcurveto{\pgfqpoint{1.014856in}{0.747727in}}{\pgfqpoint{1.004257in}{0.743337in}}{\pgfqpoint{0.996443in}{0.735523in}}%
\pgfpathcurveto{\pgfqpoint{0.988630in}{0.727710in}}{\pgfqpoint{0.984239in}{0.717111in}}{\pgfqpoint{0.984239in}{0.706060in}}%
\pgfpathcurveto{\pgfqpoint{0.984239in}{0.695010in}}{\pgfqpoint{0.988630in}{0.684411in}}{\pgfqpoint{0.996443in}{0.676598in}}%
\pgfpathcurveto{\pgfqpoint{1.004257in}{0.668784in}}{\pgfqpoint{1.014856in}{0.664394in}}{\pgfqpoint{1.025906in}{0.664394in}}%
\pgfpathclose%
\pgfusepath{stroke,fill}%
\end{pgfscope}%
\begin{pgfscope}%
\pgfpathrectangle{\pgfqpoint{0.800000in}{0.528000in}}{\pgfqpoint{4.960000in}{3.696000in}}%
\pgfusepath{clip}%
\pgfsetbuttcap%
\pgfsetroundjoin%
\definecolor{currentfill}{rgb}{0.000000,0.000000,0.000000}%
\pgfsetfillcolor{currentfill}%
\pgfsetlinewidth{1.003750pt}%
\definecolor{currentstroke}{rgb}{0.000000,0.000000,0.000000}%
\pgfsetstrokecolor{currentstroke}%
\pgfsetdash{}{0pt}%
\pgfpathmoveto{\pgfqpoint{1.025906in}{0.664394in}}%
\pgfpathcurveto{\pgfqpoint{1.036956in}{0.664394in}}{\pgfqpoint{1.047555in}{0.668784in}}{\pgfqpoint{1.055369in}{0.676598in}}%
\pgfpathcurveto{\pgfqpoint{1.063182in}{0.684411in}}{\pgfqpoint{1.067573in}{0.695010in}}{\pgfqpoint{1.067573in}{0.706060in}}%
\pgfpathcurveto{\pgfqpoint{1.067573in}{0.717111in}}{\pgfqpoint{1.063182in}{0.727710in}}{\pgfqpoint{1.055369in}{0.735523in}}%
\pgfpathcurveto{\pgfqpoint{1.047555in}{0.743337in}}{\pgfqpoint{1.036956in}{0.747727in}}{\pgfqpoint{1.025906in}{0.747727in}}%
\pgfpathcurveto{\pgfqpoint{1.014856in}{0.747727in}}{\pgfqpoint{1.004257in}{0.743337in}}{\pgfqpoint{0.996443in}{0.735523in}}%
\pgfpathcurveto{\pgfqpoint{0.988630in}{0.727710in}}{\pgfqpoint{0.984239in}{0.717111in}}{\pgfqpoint{0.984239in}{0.706060in}}%
\pgfpathcurveto{\pgfqpoint{0.984239in}{0.695010in}}{\pgfqpoint{0.988630in}{0.684411in}}{\pgfqpoint{0.996443in}{0.676598in}}%
\pgfpathcurveto{\pgfqpoint{1.004257in}{0.668784in}}{\pgfqpoint{1.014856in}{0.664394in}}{\pgfqpoint{1.025906in}{0.664394in}}%
\pgfpathclose%
\pgfusepath{stroke,fill}%
\end{pgfscope}%
\begin{pgfscope}%
\pgfpathrectangle{\pgfqpoint{0.800000in}{0.528000in}}{\pgfqpoint{4.960000in}{3.696000in}}%
\pgfusepath{clip}%
\pgfsetbuttcap%
\pgfsetroundjoin%
\definecolor{currentfill}{rgb}{0.000000,0.000000,0.000000}%
\pgfsetfillcolor{currentfill}%
\pgfsetlinewidth{1.003750pt}%
\definecolor{currentstroke}{rgb}{0.000000,0.000000,0.000000}%
\pgfsetstrokecolor{currentstroke}%
\pgfsetdash{}{0pt}%
\pgfpathmoveto{\pgfqpoint{1.025906in}{0.664394in}}%
\pgfpathcurveto{\pgfqpoint{1.036956in}{0.664394in}}{\pgfqpoint{1.047555in}{0.668784in}}{\pgfqpoint{1.055369in}{0.676598in}}%
\pgfpathcurveto{\pgfqpoint{1.063182in}{0.684411in}}{\pgfqpoint{1.067573in}{0.695010in}}{\pgfqpoint{1.067573in}{0.706060in}}%
\pgfpathcurveto{\pgfqpoint{1.067573in}{0.717111in}}{\pgfqpoint{1.063182in}{0.727710in}}{\pgfqpoint{1.055369in}{0.735523in}}%
\pgfpathcurveto{\pgfqpoint{1.047555in}{0.743337in}}{\pgfqpoint{1.036956in}{0.747727in}}{\pgfqpoint{1.025906in}{0.747727in}}%
\pgfpathcurveto{\pgfqpoint{1.014856in}{0.747727in}}{\pgfqpoint{1.004257in}{0.743337in}}{\pgfqpoint{0.996443in}{0.735523in}}%
\pgfpathcurveto{\pgfqpoint{0.988630in}{0.727710in}}{\pgfqpoint{0.984239in}{0.717111in}}{\pgfqpoint{0.984239in}{0.706060in}}%
\pgfpathcurveto{\pgfqpoint{0.984239in}{0.695010in}}{\pgfqpoint{0.988630in}{0.684411in}}{\pgfqpoint{0.996443in}{0.676598in}}%
\pgfpathcurveto{\pgfqpoint{1.004257in}{0.668784in}}{\pgfqpoint{1.014856in}{0.664394in}}{\pgfqpoint{1.025906in}{0.664394in}}%
\pgfpathclose%
\pgfusepath{stroke,fill}%
\end{pgfscope}%
\begin{pgfscope}%
\pgfpathrectangle{\pgfqpoint{0.800000in}{0.528000in}}{\pgfqpoint{4.960000in}{3.696000in}}%
\pgfusepath{clip}%
\pgfsetbuttcap%
\pgfsetroundjoin%
\definecolor{currentfill}{rgb}{0.000000,0.000000,0.000000}%
\pgfsetfillcolor{currentfill}%
\pgfsetlinewidth{1.003750pt}%
\definecolor{currentstroke}{rgb}{0.000000,0.000000,0.000000}%
\pgfsetstrokecolor{currentstroke}%
\pgfsetdash{}{0pt}%
\pgfpathmoveto{\pgfqpoint{1.025906in}{0.664394in}}%
\pgfpathcurveto{\pgfqpoint{1.036956in}{0.664394in}}{\pgfqpoint{1.047555in}{0.668784in}}{\pgfqpoint{1.055369in}{0.676598in}}%
\pgfpathcurveto{\pgfqpoint{1.063182in}{0.684411in}}{\pgfqpoint{1.067573in}{0.695010in}}{\pgfqpoint{1.067573in}{0.706060in}}%
\pgfpathcurveto{\pgfqpoint{1.067573in}{0.717111in}}{\pgfqpoint{1.063182in}{0.727710in}}{\pgfqpoint{1.055369in}{0.735523in}}%
\pgfpathcurveto{\pgfqpoint{1.047555in}{0.743337in}}{\pgfqpoint{1.036956in}{0.747727in}}{\pgfqpoint{1.025906in}{0.747727in}}%
\pgfpathcurveto{\pgfqpoint{1.014856in}{0.747727in}}{\pgfqpoint{1.004257in}{0.743337in}}{\pgfqpoint{0.996443in}{0.735523in}}%
\pgfpathcurveto{\pgfqpoint{0.988630in}{0.727710in}}{\pgfqpoint{0.984239in}{0.717111in}}{\pgfqpoint{0.984239in}{0.706060in}}%
\pgfpathcurveto{\pgfqpoint{0.984239in}{0.695010in}}{\pgfqpoint{0.988630in}{0.684411in}}{\pgfqpoint{0.996443in}{0.676598in}}%
\pgfpathcurveto{\pgfqpoint{1.004257in}{0.668784in}}{\pgfqpoint{1.014856in}{0.664394in}}{\pgfqpoint{1.025906in}{0.664394in}}%
\pgfpathclose%
\pgfusepath{stroke,fill}%
\end{pgfscope}%
\begin{pgfscope}%
\pgfpathrectangle{\pgfqpoint{0.800000in}{0.528000in}}{\pgfqpoint{4.960000in}{3.696000in}}%
\pgfusepath{clip}%
\pgfsetbuttcap%
\pgfsetroundjoin%
\definecolor{currentfill}{rgb}{0.000000,0.000000,0.000000}%
\pgfsetfillcolor{currentfill}%
\pgfsetlinewidth{1.003750pt}%
\definecolor{currentstroke}{rgb}{0.000000,0.000000,0.000000}%
\pgfsetstrokecolor{currentstroke}%
\pgfsetdash{}{0pt}%
\pgfpathmoveto{\pgfqpoint{1.025906in}{0.664394in}}%
\pgfpathcurveto{\pgfqpoint{1.036956in}{0.664394in}}{\pgfqpoint{1.047555in}{0.668784in}}{\pgfqpoint{1.055369in}{0.676598in}}%
\pgfpathcurveto{\pgfqpoint{1.063182in}{0.684411in}}{\pgfqpoint{1.067573in}{0.695010in}}{\pgfqpoint{1.067573in}{0.706060in}}%
\pgfpathcurveto{\pgfqpoint{1.067573in}{0.717111in}}{\pgfqpoint{1.063182in}{0.727710in}}{\pgfqpoint{1.055369in}{0.735523in}}%
\pgfpathcurveto{\pgfqpoint{1.047555in}{0.743337in}}{\pgfqpoint{1.036956in}{0.747727in}}{\pgfqpoint{1.025906in}{0.747727in}}%
\pgfpathcurveto{\pgfqpoint{1.014856in}{0.747727in}}{\pgfqpoint{1.004257in}{0.743337in}}{\pgfqpoint{0.996443in}{0.735523in}}%
\pgfpathcurveto{\pgfqpoint{0.988630in}{0.727710in}}{\pgfqpoint{0.984239in}{0.717111in}}{\pgfqpoint{0.984239in}{0.706060in}}%
\pgfpathcurveto{\pgfqpoint{0.984239in}{0.695010in}}{\pgfqpoint{0.988630in}{0.684411in}}{\pgfqpoint{0.996443in}{0.676598in}}%
\pgfpathcurveto{\pgfqpoint{1.004257in}{0.668784in}}{\pgfqpoint{1.014856in}{0.664394in}}{\pgfqpoint{1.025906in}{0.664394in}}%
\pgfpathclose%
\pgfusepath{stroke,fill}%
\end{pgfscope}%
\begin{pgfscope}%
\pgfpathrectangle{\pgfqpoint{0.800000in}{0.528000in}}{\pgfqpoint{4.960000in}{3.696000in}}%
\pgfusepath{clip}%
\pgfsetbuttcap%
\pgfsetroundjoin%
\definecolor{currentfill}{rgb}{0.000000,0.000000,0.000000}%
\pgfsetfillcolor{currentfill}%
\pgfsetlinewidth{1.003750pt}%
\definecolor{currentstroke}{rgb}{0.000000,0.000000,0.000000}%
\pgfsetstrokecolor{currentstroke}%
\pgfsetdash{}{0pt}%
\pgfpathmoveto{\pgfqpoint{1.025906in}{0.664394in}}%
\pgfpathcurveto{\pgfqpoint{1.036956in}{0.664394in}}{\pgfqpoint{1.047555in}{0.668784in}}{\pgfqpoint{1.055369in}{0.676598in}}%
\pgfpathcurveto{\pgfqpoint{1.063182in}{0.684411in}}{\pgfqpoint{1.067573in}{0.695010in}}{\pgfqpoint{1.067573in}{0.706060in}}%
\pgfpathcurveto{\pgfqpoint{1.067573in}{0.717111in}}{\pgfqpoint{1.063182in}{0.727710in}}{\pgfqpoint{1.055369in}{0.735523in}}%
\pgfpathcurveto{\pgfqpoint{1.047555in}{0.743337in}}{\pgfqpoint{1.036956in}{0.747727in}}{\pgfqpoint{1.025906in}{0.747727in}}%
\pgfpathcurveto{\pgfqpoint{1.014856in}{0.747727in}}{\pgfqpoint{1.004257in}{0.743337in}}{\pgfqpoint{0.996443in}{0.735523in}}%
\pgfpathcurveto{\pgfqpoint{0.988630in}{0.727710in}}{\pgfqpoint{0.984239in}{0.717111in}}{\pgfqpoint{0.984239in}{0.706060in}}%
\pgfpathcurveto{\pgfqpoint{0.984239in}{0.695010in}}{\pgfqpoint{0.988630in}{0.684411in}}{\pgfqpoint{0.996443in}{0.676598in}}%
\pgfpathcurveto{\pgfqpoint{1.004257in}{0.668784in}}{\pgfqpoint{1.014856in}{0.664394in}}{\pgfqpoint{1.025906in}{0.664394in}}%
\pgfpathclose%
\pgfusepath{stroke,fill}%
\end{pgfscope}%
\begin{pgfscope}%
\pgfpathrectangle{\pgfqpoint{0.800000in}{0.528000in}}{\pgfqpoint{4.960000in}{3.696000in}}%
\pgfusepath{clip}%
\pgfsetbuttcap%
\pgfsetroundjoin%
\definecolor{currentfill}{rgb}{0.000000,0.000000,0.000000}%
\pgfsetfillcolor{currentfill}%
\pgfsetlinewidth{1.003750pt}%
\definecolor{currentstroke}{rgb}{0.000000,0.000000,0.000000}%
\pgfsetstrokecolor{currentstroke}%
\pgfsetdash{}{0pt}%
\pgfpathmoveto{\pgfqpoint{1.025906in}{0.664394in}}%
\pgfpathcurveto{\pgfqpoint{1.036956in}{0.664394in}}{\pgfqpoint{1.047555in}{0.668784in}}{\pgfqpoint{1.055369in}{0.676598in}}%
\pgfpathcurveto{\pgfqpoint{1.063182in}{0.684411in}}{\pgfqpoint{1.067573in}{0.695010in}}{\pgfqpoint{1.067573in}{0.706060in}}%
\pgfpathcurveto{\pgfqpoint{1.067573in}{0.717111in}}{\pgfqpoint{1.063182in}{0.727710in}}{\pgfqpoint{1.055369in}{0.735523in}}%
\pgfpathcurveto{\pgfqpoint{1.047555in}{0.743337in}}{\pgfqpoint{1.036956in}{0.747727in}}{\pgfqpoint{1.025906in}{0.747727in}}%
\pgfpathcurveto{\pgfqpoint{1.014856in}{0.747727in}}{\pgfqpoint{1.004257in}{0.743337in}}{\pgfqpoint{0.996443in}{0.735523in}}%
\pgfpathcurveto{\pgfqpoint{0.988630in}{0.727710in}}{\pgfqpoint{0.984239in}{0.717111in}}{\pgfqpoint{0.984239in}{0.706060in}}%
\pgfpathcurveto{\pgfqpoint{0.984239in}{0.695010in}}{\pgfqpoint{0.988630in}{0.684411in}}{\pgfqpoint{0.996443in}{0.676598in}}%
\pgfpathcurveto{\pgfqpoint{1.004257in}{0.668784in}}{\pgfqpoint{1.014856in}{0.664394in}}{\pgfqpoint{1.025906in}{0.664394in}}%
\pgfpathclose%
\pgfusepath{stroke,fill}%
\end{pgfscope}%
\begin{pgfscope}%
\pgfpathrectangle{\pgfqpoint{0.800000in}{0.528000in}}{\pgfqpoint{4.960000in}{3.696000in}}%
\pgfusepath{clip}%
\pgfsetbuttcap%
\pgfsetroundjoin%
\definecolor{currentfill}{rgb}{0.000000,0.000000,0.000000}%
\pgfsetfillcolor{currentfill}%
\pgfsetlinewidth{1.003750pt}%
\definecolor{currentstroke}{rgb}{0.000000,0.000000,0.000000}%
\pgfsetstrokecolor{currentstroke}%
\pgfsetdash{}{0pt}%
\pgfpathmoveto{\pgfqpoint{1.025906in}{0.664394in}}%
\pgfpathcurveto{\pgfqpoint{1.036956in}{0.664394in}}{\pgfqpoint{1.047555in}{0.668784in}}{\pgfqpoint{1.055369in}{0.676598in}}%
\pgfpathcurveto{\pgfqpoint{1.063182in}{0.684411in}}{\pgfqpoint{1.067573in}{0.695010in}}{\pgfqpoint{1.067573in}{0.706060in}}%
\pgfpathcurveto{\pgfqpoint{1.067573in}{0.717111in}}{\pgfqpoint{1.063182in}{0.727710in}}{\pgfqpoint{1.055369in}{0.735523in}}%
\pgfpathcurveto{\pgfqpoint{1.047555in}{0.743337in}}{\pgfqpoint{1.036956in}{0.747727in}}{\pgfqpoint{1.025906in}{0.747727in}}%
\pgfpathcurveto{\pgfqpoint{1.014856in}{0.747727in}}{\pgfqpoint{1.004257in}{0.743337in}}{\pgfqpoint{0.996443in}{0.735523in}}%
\pgfpathcurveto{\pgfqpoint{0.988630in}{0.727710in}}{\pgfqpoint{0.984239in}{0.717111in}}{\pgfqpoint{0.984239in}{0.706060in}}%
\pgfpathcurveto{\pgfqpoint{0.984239in}{0.695010in}}{\pgfqpoint{0.988630in}{0.684411in}}{\pgfqpoint{0.996443in}{0.676598in}}%
\pgfpathcurveto{\pgfqpoint{1.004257in}{0.668784in}}{\pgfqpoint{1.014856in}{0.664394in}}{\pgfqpoint{1.025906in}{0.664394in}}%
\pgfpathclose%
\pgfusepath{stroke,fill}%
\end{pgfscope}%
\begin{pgfscope}%
\pgfpathrectangle{\pgfqpoint{0.800000in}{0.528000in}}{\pgfqpoint{4.960000in}{3.696000in}}%
\pgfusepath{clip}%
\pgfsetbuttcap%
\pgfsetroundjoin%
\definecolor{currentfill}{rgb}{0.000000,0.000000,0.000000}%
\pgfsetfillcolor{currentfill}%
\pgfsetlinewidth{1.003750pt}%
\definecolor{currentstroke}{rgb}{0.000000,0.000000,0.000000}%
\pgfsetstrokecolor{currentstroke}%
\pgfsetdash{}{0pt}%
\pgfpathmoveto{\pgfqpoint{1.025906in}{0.664394in}}%
\pgfpathcurveto{\pgfqpoint{1.036956in}{0.664394in}}{\pgfqpoint{1.047555in}{0.668784in}}{\pgfqpoint{1.055369in}{0.676598in}}%
\pgfpathcurveto{\pgfqpoint{1.063182in}{0.684411in}}{\pgfqpoint{1.067573in}{0.695010in}}{\pgfqpoint{1.067573in}{0.706060in}}%
\pgfpathcurveto{\pgfqpoint{1.067573in}{0.717111in}}{\pgfqpoint{1.063182in}{0.727710in}}{\pgfqpoint{1.055369in}{0.735523in}}%
\pgfpathcurveto{\pgfqpoint{1.047555in}{0.743337in}}{\pgfqpoint{1.036956in}{0.747727in}}{\pgfqpoint{1.025906in}{0.747727in}}%
\pgfpathcurveto{\pgfqpoint{1.014856in}{0.747727in}}{\pgfqpoint{1.004257in}{0.743337in}}{\pgfqpoint{0.996443in}{0.735523in}}%
\pgfpathcurveto{\pgfqpoint{0.988630in}{0.727710in}}{\pgfqpoint{0.984239in}{0.717111in}}{\pgfqpoint{0.984239in}{0.706060in}}%
\pgfpathcurveto{\pgfqpoint{0.984239in}{0.695010in}}{\pgfqpoint{0.988630in}{0.684411in}}{\pgfqpoint{0.996443in}{0.676598in}}%
\pgfpathcurveto{\pgfqpoint{1.004257in}{0.668784in}}{\pgfqpoint{1.014856in}{0.664394in}}{\pgfqpoint{1.025906in}{0.664394in}}%
\pgfpathclose%
\pgfusepath{stroke,fill}%
\end{pgfscope}%
\begin{pgfscope}%
\pgfpathrectangle{\pgfqpoint{0.800000in}{0.528000in}}{\pgfqpoint{4.960000in}{3.696000in}}%
\pgfusepath{clip}%
\pgfsetbuttcap%
\pgfsetroundjoin%
\definecolor{currentfill}{rgb}{0.000000,0.000000,0.000000}%
\pgfsetfillcolor{currentfill}%
\pgfsetlinewidth{1.003750pt}%
\definecolor{currentstroke}{rgb}{0.000000,0.000000,0.000000}%
\pgfsetstrokecolor{currentstroke}%
\pgfsetdash{}{0pt}%
\pgfpathmoveto{\pgfqpoint{1.025906in}{0.664394in}}%
\pgfpathcurveto{\pgfqpoint{1.036956in}{0.664394in}}{\pgfqpoint{1.047555in}{0.668784in}}{\pgfqpoint{1.055369in}{0.676598in}}%
\pgfpathcurveto{\pgfqpoint{1.063182in}{0.684411in}}{\pgfqpoint{1.067573in}{0.695010in}}{\pgfqpoint{1.067573in}{0.706060in}}%
\pgfpathcurveto{\pgfqpoint{1.067573in}{0.717111in}}{\pgfqpoint{1.063182in}{0.727710in}}{\pgfqpoint{1.055369in}{0.735523in}}%
\pgfpathcurveto{\pgfqpoint{1.047555in}{0.743337in}}{\pgfqpoint{1.036956in}{0.747727in}}{\pgfqpoint{1.025906in}{0.747727in}}%
\pgfpathcurveto{\pgfqpoint{1.014856in}{0.747727in}}{\pgfqpoint{1.004257in}{0.743337in}}{\pgfqpoint{0.996443in}{0.735523in}}%
\pgfpathcurveto{\pgfqpoint{0.988630in}{0.727710in}}{\pgfqpoint{0.984239in}{0.717111in}}{\pgfqpoint{0.984239in}{0.706060in}}%
\pgfpathcurveto{\pgfqpoint{0.984239in}{0.695010in}}{\pgfqpoint{0.988630in}{0.684411in}}{\pgfqpoint{0.996443in}{0.676598in}}%
\pgfpathcurveto{\pgfqpoint{1.004257in}{0.668784in}}{\pgfqpoint{1.014856in}{0.664394in}}{\pgfqpoint{1.025906in}{0.664394in}}%
\pgfpathclose%
\pgfusepath{stroke,fill}%
\end{pgfscope}%
\begin{pgfscope}%
\pgfpathrectangle{\pgfqpoint{0.800000in}{0.528000in}}{\pgfqpoint{4.960000in}{3.696000in}}%
\pgfusepath{clip}%
\pgfsetbuttcap%
\pgfsetroundjoin%
\definecolor{currentfill}{rgb}{0.000000,0.000000,0.000000}%
\pgfsetfillcolor{currentfill}%
\pgfsetlinewidth{1.003750pt}%
\definecolor{currentstroke}{rgb}{0.000000,0.000000,0.000000}%
\pgfsetstrokecolor{currentstroke}%
\pgfsetdash{}{0pt}%
\pgfpathmoveto{\pgfqpoint{1.025906in}{0.664394in}}%
\pgfpathcurveto{\pgfqpoint{1.036956in}{0.664394in}}{\pgfqpoint{1.047555in}{0.668784in}}{\pgfqpoint{1.055369in}{0.676598in}}%
\pgfpathcurveto{\pgfqpoint{1.063182in}{0.684411in}}{\pgfqpoint{1.067573in}{0.695010in}}{\pgfqpoint{1.067573in}{0.706060in}}%
\pgfpathcurveto{\pgfqpoint{1.067573in}{0.717111in}}{\pgfqpoint{1.063182in}{0.727710in}}{\pgfqpoint{1.055369in}{0.735523in}}%
\pgfpathcurveto{\pgfqpoint{1.047555in}{0.743337in}}{\pgfqpoint{1.036956in}{0.747727in}}{\pgfqpoint{1.025906in}{0.747727in}}%
\pgfpathcurveto{\pgfqpoint{1.014856in}{0.747727in}}{\pgfqpoint{1.004257in}{0.743337in}}{\pgfqpoint{0.996443in}{0.735523in}}%
\pgfpathcurveto{\pgfqpoint{0.988630in}{0.727710in}}{\pgfqpoint{0.984239in}{0.717111in}}{\pgfqpoint{0.984239in}{0.706060in}}%
\pgfpathcurveto{\pgfqpoint{0.984239in}{0.695010in}}{\pgfqpoint{0.988630in}{0.684411in}}{\pgfqpoint{0.996443in}{0.676598in}}%
\pgfpathcurveto{\pgfqpoint{1.004257in}{0.668784in}}{\pgfqpoint{1.014856in}{0.664394in}}{\pgfqpoint{1.025906in}{0.664394in}}%
\pgfpathclose%
\pgfusepath{stroke,fill}%
\end{pgfscope}%
\begin{pgfscope}%
\pgfpathrectangle{\pgfqpoint{0.800000in}{0.528000in}}{\pgfqpoint{4.960000in}{3.696000in}}%
\pgfusepath{clip}%
\pgfsetbuttcap%
\pgfsetroundjoin%
\definecolor{currentfill}{rgb}{0.000000,0.000000,0.000000}%
\pgfsetfillcolor{currentfill}%
\pgfsetlinewidth{1.003750pt}%
\definecolor{currentstroke}{rgb}{0.000000,0.000000,0.000000}%
\pgfsetstrokecolor{currentstroke}%
\pgfsetdash{}{0pt}%
\pgfpathmoveto{\pgfqpoint{1.025906in}{0.664394in}}%
\pgfpathcurveto{\pgfqpoint{1.036956in}{0.664394in}}{\pgfqpoint{1.047555in}{0.668784in}}{\pgfqpoint{1.055369in}{0.676598in}}%
\pgfpathcurveto{\pgfqpoint{1.063182in}{0.684411in}}{\pgfqpoint{1.067573in}{0.695010in}}{\pgfqpoint{1.067573in}{0.706060in}}%
\pgfpathcurveto{\pgfqpoint{1.067573in}{0.717111in}}{\pgfqpoint{1.063182in}{0.727710in}}{\pgfqpoint{1.055369in}{0.735523in}}%
\pgfpathcurveto{\pgfqpoint{1.047555in}{0.743337in}}{\pgfqpoint{1.036956in}{0.747727in}}{\pgfqpoint{1.025906in}{0.747727in}}%
\pgfpathcurveto{\pgfqpoint{1.014856in}{0.747727in}}{\pgfqpoint{1.004257in}{0.743337in}}{\pgfqpoint{0.996443in}{0.735523in}}%
\pgfpathcurveto{\pgfqpoint{0.988630in}{0.727710in}}{\pgfqpoint{0.984239in}{0.717111in}}{\pgfqpoint{0.984239in}{0.706060in}}%
\pgfpathcurveto{\pgfqpoint{0.984239in}{0.695010in}}{\pgfqpoint{0.988630in}{0.684411in}}{\pgfqpoint{0.996443in}{0.676598in}}%
\pgfpathcurveto{\pgfqpoint{1.004257in}{0.668784in}}{\pgfqpoint{1.014856in}{0.664394in}}{\pgfqpoint{1.025906in}{0.664394in}}%
\pgfpathclose%
\pgfusepath{stroke,fill}%
\end{pgfscope}%
\begin{pgfscope}%
\pgfpathrectangle{\pgfqpoint{0.800000in}{0.528000in}}{\pgfqpoint{4.960000in}{3.696000in}}%
\pgfusepath{clip}%
\pgfsetbuttcap%
\pgfsetroundjoin%
\definecolor{currentfill}{rgb}{0.000000,0.000000,0.000000}%
\pgfsetfillcolor{currentfill}%
\pgfsetlinewidth{1.003750pt}%
\definecolor{currentstroke}{rgb}{0.000000,0.000000,0.000000}%
\pgfsetstrokecolor{currentstroke}%
\pgfsetdash{}{0pt}%
\pgfpathmoveto{\pgfqpoint{1.025906in}{0.664394in}}%
\pgfpathcurveto{\pgfqpoint{1.036956in}{0.664394in}}{\pgfqpoint{1.047555in}{0.668784in}}{\pgfqpoint{1.055369in}{0.676598in}}%
\pgfpathcurveto{\pgfqpoint{1.063182in}{0.684411in}}{\pgfqpoint{1.067573in}{0.695010in}}{\pgfqpoint{1.067573in}{0.706060in}}%
\pgfpathcurveto{\pgfqpoint{1.067573in}{0.717111in}}{\pgfqpoint{1.063182in}{0.727710in}}{\pgfqpoint{1.055369in}{0.735523in}}%
\pgfpathcurveto{\pgfqpoint{1.047555in}{0.743337in}}{\pgfqpoint{1.036956in}{0.747727in}}{\pgfqpoint{1.025906in}{0.747727in}}%
\pgfpathcurveto{\pgfqpoint{1.014856in}{0.747727in}}{\pgfqpoint{1.004257in}{0.743337in}}{\pgfqpoint{0.996443in}{0.735523in}}%
\pgfpathcurveto{\pgfqpoint{0.988630in}{0.727710in}}{\pgfqpoint{0.984239in}{0.717111in}}{\pgfqpoint{0.984239in}{0.706060in}}%
\pgfpathcurveto{\pgfqpoint{0.984239in}{0.695010in}}{\pgfqpoint{0.988630in}{0.684411in}}{\pgfqpoint{0.996443in}{0.676598in}}%
\pgfpathcurveto{\pgfqpoint{1.004257in}{0.668784in}}{\pgfqpoint{1.014856in}{0.664394in}}{\pgfqpoint{1.025906in}{0.664394in}}%
\pgfpathclose%
\pgfusepath{stroke,fill}%
\end{pgfscope}%
\begin{pgfscope}%
\pgfpathrectangle{\pgfqpoint{0.800000in}{0.528000in}}{\pgfqpoint{4.960000in}{3.696000in}}%
\pgfusepath{clip}%
\pgfsetbuttcap%
\pgfsetroundjoin%
\definecolor{currentfill}{rgb}{0.000000,0.000000,0.000000}%
\pgfsetfillcolor{currentfill}%
\pgfsetlinewidth{1.003750pt}%
\definecolor{currentstroke}{rgb}{0.000000,0.000000,0.000000}%
\pgfsetstrokecolor{currentstroke}%
\pgfsetdash{}{0pt}%
\pgfpathmoveto{\pgfqpoint{1.025906in}{0.664394in}}%
\pgfpathcurveto{\pgfqpoint{1.036956in}{0.664394in}}{\pgfqpoint{1.047555in}{0.668784in}}{\pgfqpoint{1.055369in}{0.676598in}}%
\pgfpathcurveto{\pgfqpoint{1.063182in}{0.684411in}}{\pgfqpoint{1.067573in}{0.695010in}}{\pgfqpoint{1.067573in}{0.706060in}}%
\pgfpathcurveto{\pgfqpoint{1.067573in}{0.717111in}}{\pgfqpoint{1.063182in}{0.727710in}}{\pgfqpoint{1.055369in}{0.735523in}}%
\pgfpathcurveto{\pgfqpoint{1.047555in}{0.743337in}}{\pgfqpoint{1.036956in}{0.747727in}}{\pgfqpoint{1.025906in}{0.747727in}}%
\pgfpathcurveto{\pgfqpoint{1.014856in}{0.747727in}}{\pgfqpoint{1.004257in}{0.743337in}}{\pgfqpoint{0.996443in}{0.735523in}}%
\pgfpathcurveto{\pgfqpoint{0.988630in}{0.727710in}}{\pgfqpoint{0.984239in}{0.717111in}}{\pgfqpoint{0.984239in}{0.706060in}}%
\pgfpathcurveto{\pgfqpoint{0.984239in}{0.695010in}}{\pgfqpoint{0.988630in}{0.684411in}}{\pgfqpoint{0.996443in}{0.676598in}}%
\pgfpathcurveto{\pgfqpoint{1.004257in}{0.668784in}}{\pgfqpoint{1.014856in}{0.664394in}}{\pgfqpoint{1.025906in}{0.664394in}}%
\pgfpathclose%
\pgfusepath{stroke,fill}%
\end{pgfscope}%
\begin{pgfscope}%
\pgfpathrectangle{\pgfqpoint{0.800000in}{0.528000in}}{\pgfqpoint{4.960000in}{3.696000in}}%
\pgfusepath{clip}%
\pgfsetbuttcap%
\pgfsetroundjoin%
\definecolor{currentfill}{rgb}{0.000000,0.000000,0.000000}%
\pgfsetfillcolor{currentfill}%
\pgfsetlinewidth{1.003750pt}%
\definecolor{currentstroke}{rgb}{0.000000,0.000000,0.000000}%
\pgfsetstrokecolor{currentstroke}%
\pgfsetdash{}{0pt}%
\pgfpathmoveto{\pgfqpoint{1.025906in}{0.664394in}}%
\pgfpathcurveto{\pgfqpoint{1.036956in}{0.664394in}}{\pgfqpoint{1.047555in}{0.668784in}}{\pgfqpoint{1.055369in}{0.676598in}}%
\pgfpathcurveto{\pgfqpoint{1.063182in}{0.684411in}}{\pgfqpoint{1.067573in}{0.695010in}}{\pgfqpoint{1.067573in}{0.706060in}}%
\pgfpathcurveto{\pgfqpoint{1.067573in}{0.717111in}}{\pgfqpoint{1.063182in}{0.727710in}}{\pgfqpoint{1.055369in}{0.735523in}}%
\pgfpathcurveto{\pgfqpoint{1.047555in}{0.743337in}}{\pgfqpoint{1.036956in}{0.747727in}}{\pgfqpoint{1.025906in}{0.747727in}}%
\pgfpathcurveto{\pgfqpoint{1.014856in}{0.747727in}}{\pgfqpoint{1.004257in}{0.743337in}}{\pgfqpoint{0.996443in}{0.735523in}}%
\pgfpathcurveto{\pgfqpoint{0.988630in}{0.727710in}}{\pgfqpoint{0.984239in}{0.717111in}}{\pgfqpoint{0.984239in}{0.706060in}}%
\pgfpathcurveto{\pgfqpoint{0.984239in}{0.695010in}}{\pgfqpoint{0.988630in}{0.684411in}}{\pgfqpoint{0.996443in}{0.676598in}}%
\pgfpathcurveto{\pgfqpoint{1.004257in}{0.668784in}}{\pgfqpoint{1.014856in}{0.664394in}}{\pgfqpoint{1.025906in}{0.664394in}}%
\pgfpathclose%
\pgfusepath{stroke,fill}%
\end{pgfscope}%
\begin{pgfscope}%
\pgfpathrectangle{\pgfqpoint{0.800000in}{0.528000in}}{\pgfqpoint{4.960000in}{3.696000in}}%
\pgfusepath{clip}%
\pgfsetbuttcap%
\pgfsetroundjoin%
\definecolor{currentfill}{rgb}{0.000000,0.000000,0.000000}%
\pgfsetfillcolor{currentfill}%
\pgfsetlinewidth{1.003750pt}%
\definecolor{currentstroke}{rgb}{0.000000,0.000000,0.000000}%
\pgfsetstrokecolor{currentstroke}%
\pgfsetdash{}{0pt}%
\pgfpathmoveto{\pgfqpoint{1.025906in}{1.771040in}}%
\pgfpathcurveto{\pgfqpoint{1.036956in}{1.771040in}}{\pgfqpoint{1.047555in}{1.775431in}}{\pgfqpoint{1.055369in}{1.783244in}}%
\pgfpathcurveto{\pgfqpoint{1.063182in}{1.791058in}}{\pgfqpoint{1.067573in}{1.801657in}}{\pgfqpoint{1.067573in}{1.812707in}}%
\pgfpathcurveto{\pgfqpoint{1.067573in}{1.823757in}}{\pgfqpoint{1.063182in}{1.834356in}}{\pgfqpoint{1.055369in}{1.842170in}}%
\pgfpathcurveto{\pgfqpoint{1.047555in}{1.849983in}}{\pgfqpoint{1.036956in}{1.854374in}}{\pgfqpoint{1.025906in}{1.854374in}}%
\pgfpathcurveto{\pgfqpoint{1.014856in}{1.854374in}}{\pgfqpoint{1.004257in}{1.849983in}}{\pgfqpoint{0.996443in}{1.842170in}}%
\pgfpathcurveto{\pgfqpoint{0.988630in}{1.834356in}}{\pgfqpoint{0.984239in}{1.823757in}}{\pgfqpoint{0.984239in}{1.812707in}}%
\pgfpathcurveto{\pgfqpoint{0.984239in}{1.801657in}}{\pgfqpoint{0.988630in}{1.791058in}}{\pgfqpoint{0.996443in}{1.783244in}}%
\pgfpathcurveto{\pgfqpoint{1.004257in}{1.775431in}}{\pgfqpoint{1.014856in}{1.771040in}}{\pgfqpoint{1.025906in}{1.771040in}}%
\pgfpathclose%
\pgfusepath{stroke,fill}%
\end{pgfscope}%
\begin{pgfscope}%
\pgfpathrectangle{\pgfqpoint{0.800000in}{0.528000in}}{\pgfqpoint{4.960000in}{3.696000in}}%
\pgfusepath{clip}%
\pgfsetbuttcap%
\pgfsetroundjoin%
\definecolor{currentfill}{rgb}{0.000000,0.000000,0.000000}%
\pgfsetfillcolor{currentfill}%
\pgfsetlinewidth{1.003750pt}%
\definecolor{currentstroke}{rgb}{0.000000,0.000000,0.000000}%
\pgfsetstrokecolor{currentstroke}%
\pgfsetdash{}{0pt}%
\pgfpathmoveto{\pgfqpoint{1.025906in}{0.664394in}}%
\pgfpathcurveto{\pgfqpoint{1.036956in}{0.664394in}}{\pgfqpoint{1.047555in}{0.668784in}}{\pgfqpoint{1.055369in}{0.676598in}}%
\pgfpathcurveto{\pgfqpoint{1.063182in}{0.684411in}}{\pgfqpoint{1.067573in}{0.695010in}}{\pgfqpoint{1.067573in}{0.706060in}}%
\pgfpathcurveto{\pgfqpoint{1.067573in}{0.717111in}}{\pgfqpoint{1.063182in}{0.727710in}}{\pgfqpoint{1.055369in}{0.735523in}}%
\pgfpathcurveto{\pgfqpoint{1.047555in}{0.743337in}}{\pgfqpoint{1.036956in}{0.747727in}}{\pgfqpoint{1.025906in}{0.747727in}}%
\pgfpathcurveto{\pgfqpoint{1.014856in}{0.747727in}}{\pgfqpoint{1.004257in}{0.743337in}}{\pgfqpoint{0.996443in}{0.735523in}}%
\pgfpathcurveto{\pgfqpoint{0.988630in}{0.727710in}}{\pgfqpoint{0.984239in}{0.717111in}}{\pgfqpoint{0.984239in}{0.706060in}}%
\pgfpathcurveto{\pgfqpoint{0.984239in}{0.695010in}}{\pgfqpoint{0.988630in}{0.684411in}}{\pgfqpoint{0.996443in}{0.676598in}}%
\pgfpathcurveto{\pgfqpoint{1.004257in}{0.668784in}}{\pgfqpoint{1.014856in}{0.664394in}}{\pgfqpoint{1.025906in}{0.664394in}}%
\pgfpathclose%
\pgfusepath{stroke,fill}%
\end{pgfscope}%
\begin{pgfscope}%
\pgfpathrectangle{\pgfqpoint{0.800000in}{0.528000in}}{\pgfqpoint{4.960000in}{3.696000in}}%
\pgfusepath{clip}%
\pgfsetbuttcap%
\pgfsetroundjoin%
\definecolor{currentfill}{rgb}{0.000000,0.000000,0.000000}%
\pgfsetfillcolor{currentfill}%
\pgfsetlinewidth{1.003750pt}%
\definecolor{currentstroke}{rgb}{0.000000,0.000000,0.000000}%
\pgfsetstrokecolor{currentstroke}%
\pgfsetdash{}{0pt}%
\pgfpathmoveto{\pgfqpoint{1.025906in}{0.664394in}}%
\pgfpathcurveto{\pgfqpoint{1.036956in}{0.664394in}}{\pgfqpoint{1.047555in}{0.668784in}}{\pgfqpoint{1.055369in}{0.676598in}}%
\pgfpathcurveto{\pgfqpoint{1.063182in}{0.684411in}}{\pgfqpoint{1.067573in}{0.695010in}}{\pgfqpoint{1.067573in}{0.706060in}}%
\pgfpathcurveto{\pgfqpoint{1.067573in}{0.717111in}}{\pgfqpoint{1.063182in}{0.727710in}}{\pgfqpoint{1.055369in}{0.735523in}}%
\pgfpathcurveto{\pgfqpoint{1.047555in}{0.743337in}}{\pgfqpoint{1.036956in}{0.747727in}}{\pgfqpoint{1.025906in}{0.747727in}}%
\pgfpathcurveto{\pgfqpoint{1.014856in}{0.747727in}}{\pgfqpoint{1.004257in}{0.743337in}}{\pgfqpoint{0.996443in}{0.735523in}}%
\pgfpathcurveto{\pgfqpoint{0.988630in}{0.727710in}}{\pgfqpoint{0.984239in}{0.717111in}}{\pgfqpoint{0.984239in}{0.706060in}}%
\pgfpathcurveto{\pgfqpoint{0.984239in}{0.695010in}}{\pgfqpoint{0.988630in}{0.684411in}}{\pgfqpoint{0.996443in}{0.676598in}}%
\pgfpathcurveto{\pgfqpoint{1.004257in}{0.668784in}}{\pgfqpoint{1.014856in}{0.664394in}}{\pgfqpoint{1.025906in}{0.664394in}}%
\pgfpathclose%
\pgfusepath{stroke,fill}%
\end{pgfscope}%
\begin{pgfscope}%
\pgfpathrectangle{\pgfqpoint{0.800000in}{0.528000in}}{\pgfqpoint{4.960000in}{3.696000in}}%
\pgfusepath{clip}%
\pgfsetbuttcap%
\pgfsetroundjoin%
\definecolor{currentfill}{rgb}{0.000000,0.000000,0.000000}%
\pgfsetfillcolor{currentfill}%
\pgfsetlinewidth{1.003750pt}%
\definecolor{currentstroke}{rgb}{0.000000,0.000000,0.000000}%
\pgfsetstrokecolor{currentstroke}%
\pgfsetdash{}{0pt}%
\pgfpathmoveto{\pgfqpoint{1.025906in}{0.664394in}}%
\pgfpathcurveto{\pgfqpoint{1.036956in}{0.664394in}}{\pgfqpoint{1.047555in}{0.668784in}}{\pgfqpoint{1.055369in}{0.676598in}}%
\pgfpathcurveto{\pgfqpoint{1.063182in}{0.684411in}}{\pgfqpoint{1.067573in}{0.695010in}}{\pgfqpoint{1.067573in}{0.706060in}}%
\pgfpathcurveto{\pgfqpoint{1.067573in}{0.717111in}}{\pgfqpoint{1.063182in}{0.727710in}}{\pgfqpoint{1.055369in}{0.735523in}}%
\pgfpathcurveto{\pgfqpoint{1.047555in}{0.743337in}}{\pgfqpoint{1.036956in}{0.747727in}}{\pgfqpoint{1.025906in}{0.747727in}}%
\pgfpathcurveto{\pgfqpoint{1.014856in}{0.747727in}}{\pgfqpoint{1.004257in}{0.743337in}}{\pgfqpoint{0.996443in}{0.735523in}}%
\pgfpathcurveto{\pgfqpoint{0.988630in}{0.727710in}}{\pgfqpoint{0.984239in}{0.717111in}}{\pgfqpoint{0.984239in}{0.706060in}}%
\pgfpathcurveto{\pgfqpoint{0.984239in}{0.695010in}}{\pgfqpoint{0.988630in}{0.684411in}}{\pgfqpoint{0.996443in}{0.676598in}}%
\pgfpathcurveto{\pgfqpoint{1.004257in}{0.668784in}}{\pgfqpoint{1.014856in}{0.664394in}}{\pgfqpoint{1.025906in}{0.664394in}}%
\pgfpathclose%
\pgfusepath{stroke,fill}%
\end{pgfscope}%
\begin{pgfscope}%
\pgfpathrectangle{\pgfqpoint{0.800000in}{0.528000in}}{\pgfqpoint{4.960000in}{3.696000in}}%
\pgfusepath{clip}%
\pgfsetbuttcap%
\pgfsetroundjoin%
\definecolor{currentfill}{rgb}{0.000000,0.000000,0.000000}%
\pgfsetfillcolor{currentfill}%
\pgfsetlinewidth{1.003750pt}%
\definecolor{currentstroke}{rgb}{0.000000,0.000000,0.000000}%
\pgfsetstrokecolor{currentstroke}%
\pgfsetdash{}{0pt}%
\pgfpathmoveto{\pgfqpoint{1.025906in}{0.664394in}}%
\pgfpathcurveto{\pgfqpoint{1.036956in}{0.664394in}}{\pgfqpoint{1.047555in}{0.668784in}}{\pgfqpoint{1.055369in}{0.676598in}}%
\pgfpathcurveto{\pgfqpoint{1.063182in}{0.684411in}}{\pgfqpoint{1.067573in}{0.695010in}}{\pgfqpoint{1.067573in}{0.706060in}}%
\pgfpathcurveto{\pgfqpoint{1.067573in}{0.717111in}}{\pgfqpoint{1.063182in}{0.727710in}}{\pgfqpoint{1.055369in}{0.735523in}}%
\pgfpathcurveto{\pgfqpoint{1.047555in}{0.743337in}}{\pgfqpoint{1.036956in}{0.747727in}}{\pgfqpoint{1.025906in}{0.747727in}}%
\pgfpathcurveto{\pgfqpoint{1.014856in}{0.747727in}}{\pgfqpoint{1.004257in}{0.743337in}}{\pgfqpoint{0.996443in}{0.735523in}}%
\pgfpathcurveto{\pgfqpoint{0.988630in}{0.727710in}}{\pgfqpoint{0.984239in}{0.717111in}}{\pgfqpoint{0.984239in}{0.706060in}}%
\pgfpathcurveto{\pgfqpoint{0.984239in}{0.695010in}}{\pgfqpoint{0.988630in}{0.684411in}}{\pgfqpoint{0.996443in}{0.676598in}}%
\pgfpathcurveto{\pgfqpoint{1.004257in}{0.668784in}}{\pgfqpoint{1.014856in}{0.664394in}}{\pgfqpoint{1.025906in}{0.664394in}}%
\pgfpathclose%
\pgfusepath{stroke,fill}%
\end{pgfscope}%
\begin{pgfscope}%
\pgfpathrectangle{\pgfqpoint{0.800000in}{0.528000in}}{\pgfqpoint{4.960000in}{3.696000in}}%
\pgfusepath{clip}%
\pgfsetbuttcap%
\pgfsetroundjoin%
\definecolor{currentfill}{rgb}{0.000000,0.000000,0.000000}%
\pgfsetfillcolor{currentfill}%
\pgfsetlinewidth{1.003750pt}%
\definecolor{currentstroke}{rgb}{0.000000,0.000000,0.000000}%
\pgfsetstrokecolor{currentstroke}%
\pgfsetdash{}{0pt}%
\pgfpathmoveto{\pgfqpoint{1.025906in}{0.664394in}}%
\pgfpathcurveto{\pgfqpoint{1.036956in}{0.664394in}}{\pgfqpoint{1.047555in}{0.668784in}}{\pgfqpoint{1.055369in}{0.676598in}}%
\pgfpathcurveto{\pgfqpoint{1.063182in}{0.684411in}}{\pgfqpoint{1.067573in}{0.695010in}}{\pgfqpoint{1.067573in}{0.706060in}}%
\pgfpathcurveto{\pgfqpoint{1.067573in}{0.717111in}}{\pgfqpoint{1.063182in}{0.727710in}}{\pgfqpoint{1.055369in}{0.735523in}}%
\pgfpathcurveto{\pgfqpoint{1.047555in}{0.743337in}}{\pgfqpoint{1.036956in}{0.747727in}}{\pgfqpoint{1.025906in}{0.747727in}}%
\pgfpathcurveto{\pgfqpoint{1.014856in}{0.747727in}}{\pgfqpoint{1.004257in}{0.743337in}}{\pgfqpoint{0.996443in}{0.735523in}}%
\pgfpathcurveto{\pgfqpoint{0.988630in}{0.727710in}}{\pgfqpoint{0.984239in}{0.717111in}}{\pgfqpoint{0.984239in}{0.706060in}}%
\pgfpathcurveto{\pgfqpoint{0.984239in}{0.695010in}}{\pgfqpoint{0.988630in}{0.684411in}}{\pgfqpoint{0.996443in}{0.676598in}}%
\pgfpathcurveto{\pgfqpoint{1.004257in}{0.668784in}}{\pgfqpoint{1.014856in}{0.664394in}}{\pgfqpoint{1.025906in}{0.664394in}}%
\pgfpathclose%
\pgfusepath{stroke,fill}%
\end{pgfscope}%
\begin{pgfscope}%
\pgfpathrectangle{\pgfqpoint{0.800000in}{0.528000in}}{\pgfqpoint{4.960000in}{3.696000in}}%
\pgfusepath{clip}%
\pgfsetbuttcap%
\pgfsetroundjoin%
\definecolor{currentfill}{rgb}{0.000000,0.000000,0.000000}%
\pgfsetfillcolor{currentfill}%
\pgfsetlinewidth{1.003750pt}%
\definecolor{currentstroke}{rgb}{0.000000,0.000000,0.000000}%
\pgfsetstrokecolor{currentstroke}%
\pgfsetdash{}{0pt}%
\pgfpathmoveto{\pgfqpoint{1.025906in}{0.664394in}}%
\pgfpathcurveto{\pgfqpoint{1.036956in}{0.664394in}}{\pgfqpoint{1.047555in}{0.668784in}}{\pgfqpoint{1.055369in}{0.676598in}}%
\pgfpathcurveto{\pgfqpoint{1.063182in}{0.684411in}}{\pgfqpoint{1.067573in}{0.695010in}}{\pgfqpoint{1.067573in}{0.706060in}}%
\pgfpathcurveto{\pgfqpoint{1.067573in}{0.717111in}}{\pgfqpoint{1.063182in}{0.727710in}}{\pgfqpoint{1.055369in}{0.735523in}}%
\pgfpathcurveto{\pgfqpoint{1.047555in}{0.743337in}}{\pgfqpoint{1.036956in}{0.747727in}}{\pgfqpoint{1.025906in}{0.747727in}}%
\pgfpathcurveto{\pgfqpoint{1.014856in}{0.747727in}}{\pgfqpoint{1.004257in}{0.743337in}}{\pgfqpoint{0.996443in}{0.735523in}}%
\pgfpathcurveto{\pgfqpoint{0.988630in}{0.727710in}}{\pgfqpoint{0.984239in}{0.717111in}}{\pgfqpoint{0.984239in}{0.706060in}}%
\pgfpathcurveto{\pgfqpoint{0.984239in}{0.695010in}}{\pgfqpoint{0.988630in}{0.684411in}}{\pgfqpoint{0.996443in}{0.676598in}}%
\pgfpathcurveto{\pgfqpoint{1.004257in}{0.668784in}}{\pgfqpoint{1.014856in}{0.664394in}}{\pgfqpoint{1.025906in}{0.664394in}}%
\pgfpathclose%
\pgfusepath{stroke,fill}%
\end{pgfscope}%
\begin{pgfscope}%
\pgfpathrectangle{\pgfqpoint{0.800000in}{0.528000in}}{\pgfqpoint{4.960000in}{3.696000in}}%
\pgfusepath{clip}%
\pgfsetbuttcap%
\pgfsetroundjoin%
\definecolor{currentfill}{rgb}{0.000000,0.000000,0.000000}%
\pgfsetfillcolor{currentfill}%
\pgfsetlinewidth{1.003750pt}%
\definecolor{currentstroke}{rgb}{0.000000,0.000000,0.000000}%
\pgfsetstrokecolor{currentstroke}%
\pgfsetdash{}{0pt}%
\pgfpathmoveto{\pgfqpoint{1.025906in}{0.664394in}}%
\pgfpathcurveto{\pgfqpoint{1.036956in}{0.664394in}}{\pgfqpoint{1.047555in}{0.668784in}}{\pgfqpoint{1.055369in}{0.676598in}}%
\pgfpathcurveto{\pgfqpoint{1.063182in}{0.684411in}}{\pgfqpoint{1.067573in}{0.695010in}}{\pgfqpoint{1.067573in}{0.706060in}}%
\pgfpathcurveto{\pgfqpoint{1.067573in}{0.717111in}}{\pgfqpoint{1.063182in}{0.727710in}}{\pgfqpoint{1.055369in}{0.735523in}}%
\pgfpathcurveto{\pgfqpoint{1.047555in}{0.743337in}}{\pgfqpoint{1.036956in}{0.747727in}}{\pgfqpoint{1.025906in}{0.747727in}}%
\pgfpathcurveto{\pgfqpoint{1.014856in}{0.747727in}}{\pgfqpoint{1.004257in}{0.743337in}}{\pgfqpoint{0.996443in}{0.735523in}}%
\pgfpathcurveto{\pgfqpoint{0.988630in}{0.727710in}}{\pgfqpoint{0.984239in}{0.717111in}}{\pgfqpoint{0.984239in}{0.706060in}}%
\pgfpathcurveto{\pgfqpoint{0.984239in}{0.695010in}}{\pgfqpoint{0.988630in}{0.684411in}}{\pgfqpoint{0.996443in}{0.676598in}}%
\pgfpathcurveto{\pgfqpoint{1.004257in}{0.668784in}}{\pgfqpoint{1.014856in}{0.664394in}}{\pgfqpoint{1.025906in}{0.664394in}}%
\pgfpathclose%
\pgfusepath{stroke,fill}%
\end{pgfscope}%
\begin{pgfscope}%
\pgfpathrectangle{\pgfqpoint{0.800000in}{0.528000in}}{\pgfqpoint{4.960000in}{3.696000in}}%
\pgfusepath{clip}%
\pgfsetbuttcap%
\pgfsetroundjoin%
\definecolor{currentfill}{rgb}{0.000000,0.000000,0.000000}%
\pgfsetfillcolor{currentfill}%
\pgfsetlinewidth{1.003750pt}%
\definecolor{currentstroke}{rgb}{0.000000,0.000000,0.000000}%
\pgfsetstrokecolor{currentstroke}%
\pgfsetdash{}{0pt}%
\pgfpathmoveto{\pgfqpoint{1.025906in}{0.664394in}}%
\pgfpathcurveto{\pgfqpoint{1.036956in}{0.664394in}}{\pgfqpoint{1.047555in}{0.668784in}}{\pgfqpoint{1.055369in}{0.676598in}}%
\pgfpathcurveto{\pgfqpoint{1.063182in}{0.684411in}}{\pgfqpoint{1.067573in}{0.695010in}}{\pgfqpoint{1.067573in}{0.706060in}}%
\pgfpathcurveto{\pgfqpoint{1.067573in}{0.717111in}}{\pgfqpoint{1.063182in}{0.727710in}}{\pgfqpoint{1.055369in}{0.735523in}}%
\pgfpathcurveto{\pgfqpoint{1.047555in}{0.743337in}}{\pgfqpoint{1.036956in}{0.747727in}}{\pgfqpoint{1.025906in}{0.747727in}}%
\pgfpathcurveto{\pgfqpoint{1.014856in}{0.747727in}}{\pgfqpoint{1.004257in}{0.743337in}}{\pgfqpoint{0.996443in}{0.735523in}}%
\pgfpathcurveto{\pgfqpoint{0.988630in}{0.727710in}}{\pgfqpoint{0.984239in}{0.717111in}}{\pgfqpoint{0.984239in}{0.706060in}}%
\pgfpathcurveto{\pgfqpoint{0.984239in}{0.695010in}}{\pgfqpoint{0.988630in}{0.684411in}}{\pgfqpoint{0.996443in}{0.676598in}}%
\pgfpathcurveto{\pgfqpoint{1.004257in}{0.668784in}}{\pgfqpoint{1.014856in}{0.664394in}}{\pgfqpoint{1.025906in}{0.664394in}}%
\pgfpathclose%
\pgfusepath{stroke,fill}%
\end{pgfscope}%
\begin{pgfscope}%
\pgfpathrectangle{\pgfqpoint{0.800000in}{0.528000in}}{\pgfqpoint{4.960000in}{3.696000in}}%
\pgfusepath{clip}%
\pgfsetbuttcap%
\pgfsetroundjoin%
\definecolor{currentfill}{rgb}{0.000000,0.000000,0.000000}%
\pgfsetfillcolor{currentfill}%
\pgfsetlinewidth{1.003750pt}%
\definecolor{currentstroke}{rgb}{0.000000,0.000000,0.000000}%
\pgfsetstrokecolor{currentstroke}%
\pgfsetdash{}{0pt}%
\pgfpathmoveto{\pgfqpoint{1.025906in}{0.664394in}}%
\pgfpathcurveto{\pgfqpoint{1.036956in}{0.664394in}}{\pgfqpoint{1.047555in}{0.668784in}}{\pgfqpoint{1.055369in}{0.676598in}}%
\pgfpathcurveto{\pgfqpoint{1.063182in}{0.684411in}}{\pgfqpoint{1.067573in}{0.695010in}}{\pgfqpoint{1.067573in}{0.706060in}}%
\pgfpathcurveto{\pgfqpoint{1.067573in}{0.717111in}}{\pgfqpoint{1.063182in}{0.727710in}}{\pgfqpoint{1.055369in}{0.735523in}}%
\pgfpathcurveto{\pgfqpoint{1.047555in}{0.743337in}}{\pgfqpoint{1.036956in}{0.747727in}}{\pgfqpoint{1.025906in}{0.747727in}}%
\pgfpathcurveto{\pgfqpoint{1.014856in}{0.747727in}}{\pgfqpoint{1.004257in}{0.743337in}}{\pgfqpoint{0.996443in}{0.735523in}}%
\pgfpathcurveto{\pgfqpoint{0.988630in}{0.727710in}}{\pgfqpoint{0.984239in}{0.717111in}}{\pgfqpoint{0.984239in}{0.706060in}}%
\pgfpathcurveto{\pgfqpoint{0.984239in}{0.695010in}}{\pgfqpoint{0.988630in}{0.684411in}}{\pgfqpoint{0.996443in}{0.676598in}}%
\pgfpathcurveto{\pgfqpoint{1.004257in}{0.668784in}}{\pgfqpoint{1.014856in}{0.664394in}}{\pgfqpoint{1.025906in}{0.664394in}}%
\pgfpathclose%
\pgfusepath{stroke,fill}%
\end{pgfscope}%
\begin{pgfscope}%
\pgfpathrectangle{\pgfqpoint{0.800000in}{0.528000in}}{\pgfqpoint{4.960000in}{3.696000in}}%
\pgfusepath{clip}%
\pgfsetbuttcap%
\pgfsetroundjoin%
\definecolor{currentfill}{rgb}{0.000000,0.000000,0.000000}%
\pgfsetfillcolor{currentfill}%
\pgfsetlinewidth{1.003750pt}%
\definecolor{currentstroke}{rgb}{0.000000,0.000000,0.000000}%
\pgfsetstrokecolor{currentstroke}%
\pgfsetdash{}{0pt}%
\pgfpathmoveto{\pgfqpoint{1.025906in}{0.664394in}}%
\pgfpathcurveto{\pgfqpoint{1.036956in}{0.664394in}}{\pgfqpoint{1.047555in}{0.668784in}}{\pgfqpoint{1.055369in}{0.676598in}}%
\pgfpathcurveto{\pgfqpoint{1.063182in}{0.684411in}}{\pgfqpoint{1.067573in}{0.695010in}}{\pgfqpoint{1.067573in}{0.706060in}}%
\pgfpathcurveto{\pgfqpoint{1.067573in}{0.717111in}}{\pgfqpoint{1.063182in}{0.727710in}}{\pgfqpoint{1.055369in}{0.735523in}}%
\pgfpathcurveto{\pgfqpoint{1.047555in}{0.743337in}}{\pgfqpoint{1.036956in}{0.747727in}}{\pgfqpoint{1.025906in}{0.747727in}}%
\pgfpathcurveto{\pgfqpoint{1.014856in}{0.747727in}}{\pgfqpoint{1.004257in}{0.743337in}}{\pgfqpoint{0.996443in}{0.735523in}}%
\pgfpathcurveto{\pgfqpoint{0.988630in}{0.727710in}}{\pgfqpoint{0.984239in}{0.717111in}}{\pgfqpoint{0.984239in}{0.706060in}}%
\pgfpathcurveto{\pgfqpoint{0.984239in}{0.695010in}}{\pgfqpoint{0.988630in}{0.684411in}}{\pgfqpoint{0.996443in}{0.676598in}}%
\pgfpathcurveto{\pgfqpoint{1.004257in}{0.668784in}}{\pgfqpoint{1.014856in}{0.664394in}}{\pgfqpoint{1.025906in}{0.664394in}}%
\pgfpathclose%
\pgfusepath{stroke,fill}%
\end{pgfscope}%
\begin{pgfscope}%
\pgfpathrectangle{\pgfqpoint{0.800000in}{0.528000in}}{\pgfqpoint{4.960000in}{3.696000in}}%
\pgfusepath{clip}%
\pgfsetbuttcap%
\pgfsetroundjoin%
\definecolor{currentfill}{rgb}{0.000000,0.000000,0.000000}%
\pgfsetfillcolor{currentfill}%
\pgfsetlinewidth{1.003750pt}%
\definecolor{currentstroke}{rgb}{0.000000,0.000000,0.000000}%
\pgfsetstrokecolor{currentstroke}%
\pgfsetdash{}{0pt}%
\pgfpathmoveto{\pgfqpoint{1.025906in}{0.664394in}}%
\pgfpathcurveto{\pgfqpoint{1.036956in}{0.664394in}}{\pgfqpoint{1.047555in}{0.668784in}}{\pgfqpoint{1.055369in}{0.676598in}}%
\pgfpathcurveto{\pgfqpoint{1.063182in}{0.684411in}}{\pgfqpoint{1.067573in}{0.695010in}}{\pgfqpoint{1.067573in}{0.706060in}}%
\pgfpathcurveto{\pgfqpoint{1.067573in}{0.717111in}}{\pgfqpoint{1.063182in}{0.727710in}}{\pgfqpoint{1.055369in}{0.735523in}}%
\pgfpathcurveto{\pgfqpoint{1.047555in}{0.743337in}}{\pgfqpoint{1.036956in}{0.747727in}}{\pgfqpoint{1.025906in}{0.747727in}}%
\pgfpathcurveto{\pgfqpoint{1.014856in}{0.747727in}}{\pgfqpoint{1.004257in}{0.743337in}}{\pgfqpoint{0.996443in}{0.735523in}}%
\pgfpathcurveto{\pgfqpoint{0.988630in}{0.727710in}}{\pgfqpoint{0.984239in}{0.717111in}}{\pgfqpoint{0.984239in}{0.706060in}}%
\pgfpathcurveto{\pgfqpoint{0.984239in}{0.695010in}}{\pgfqpoint{0.988630in}{0.684411in}}{\pgfqpoint{0.996443in}{0.676598in}}%
\pgfpathcurveto{\pgfqpoint{1.004257in}{0.668784in}}{\pgfqpoint{1.014856in}{0.664394in}}{\pgfqpoint{1.025906in}{0.664394in}}%
\pgfpathclose%
\pgfusepath{stroke,fill}%
\end{pgfscope}%
\begin{pgfscope}%
\pgfpathrectangle{\pgfqpoint{0.800000in}{0.528000in}}{\pgfqpoint{4.960000in}{3.696000in}}%
\pgfusepath{clip}%
\pgfsetbuttcap%
\pgfsetroundjoin%
\definecolor{currentfill}{rgb}{0.000000,0.000000,0.000000}%
\pgfsetfillcolor{currentfill}%
\pgfsetlinewidth{1.003750pt}%
\definecolor{currentstroke}{rgb}{0.000000,0.000000,0.000000}%
\pgfsetstrokecolor{currentstroke}%
\pgfsetdash{}{0pt}%
\pgfpathmoveto{\pgfqpoint{1.025906in}{0.664394in}}%
\pgfpathcurveto{\pgfqpoint{1.036956in}{0.664394in}}{\pgfqpoint{1.047555in}{0.668784in}}{\pgfqpoint{1.055369in}{0.676598in}}%
\pgfpathcurveto{\pgfqpoint{1.063182in}{0.684411in}}{\pgfqpoint{1.067573in}{0.695010in}}{\pgfqpoint{1.067573in}{0.706060in}}%
\pgfpathcurveto{\pgfqpoint{1.067573in}{0.717111in}}{\pgfqpoint{1.063182in}{0.727710in}}{\pgfqpoint{1.055369in}{0.735523in}}%
\pgfpathcurveto{\pgfqpoint{1.047555in}{0.743337in}}{\pgfqpoint{1.036956in}{0.747727in}}{\pgfqpoint{1.025906in}{0.747727in}}%
\pgfpathcurveto{\pgfqpoint{1.014856in}{0.747727in}}{\pgfqpoint{1.004257in}{0.743337in}}{\pgfqpoint{0.996443in}{0.735523in}}%
\pgfpathcurveto{\pgfqpoint{0.988630in}{0.727710in}}{\pgfqpoint{0.984239in}{0.717111in}}{\pgfqpoint{0.984239in}{0.706060in}}%
\pgfpathcurveto{\pgfqpoint{0.984239in}{0.695010in}}{\pgfqpoint{0.988630in}{0.684411in}}{\pgfqpoint{0.996443in}{0.676598in}}%
\pgfpathcurveto{\pgfqpoint{1.004257in}{0.668784in}}{\pgfqpoint{1.014856in}{0.664394in}}{\pgfqpoint{1.025906in}{0.664394in}}%
\pgfpathclose%
\pgfusepath{stroke,fill}%
\end{pgfscope}%
\begin{pgfscope}%
\pgfpathrectangle{\pgfqpoint{0.800000in}{0.528000in}}{\pgfqpoint{4.960000in}{3.696000in}}%
\pgfusepath{clip}%
\pgfsetbuttcap%
\pgfsetroundjoin%
\definecolor{currentfill}{rgb}{0.000000,0.000000,0.000000}%
\pgfsetfillcolor{currentfill}%
\pgfsetlinewidth{1.003750pt}%
\definecolor{currentstroke}{rgb}{0.000000,0.000000,0.000000}%
\pgfsetstrokecolor{currentstroke}%
\pgfsetdash{}{0pt}%
\pgfpathmoveto{\pgfqpoint{1.025906in}{0.664394in}}%
\pgfpathcurveto{\pgfqpoint{1.036956in}{0.664394in}}{\pgfqpoint{1.047555in}{0.668784in}}{\pgfqpoint{1.055369in}{0.676598in}}%
\pgfpathcurveto{\pgfqpoint{1.063182in}{0.684411in}}{\pgfqpoint{1.067573in}{0.695010in}}{\pgfqpoint{1.067573in}{0.706060in}}%
\pgfpathcurveto{\pgfqpoint{1.067573in}{0.717111in}}{\pgfqpoint{1.063182in}{0.727710in}}{\pgfqpoint{1.055369in}{0.735523in}}%
\pgfpathcurveto{\pgfqpoint{1.047555in}{0.743337in}}{\pgfqpoint{1.036956in}{0.747727in}}{\pgfqpoint{1.025906in}{0.747727in}}%
\pgfpathcurveto{\pgfqpoint{1.014856in}{0.747727in}}{\pgfqpoint{1.004257in}{0.743337in}}{\pgfqpoint{0.996443in}{0.735523in}}%
\pgfpathcurveto{\pgfqpoint{0.988630in}{0.727710in}}{\pgfqpoint{0.984239in}{0.717111in}}{\pgfqpoint{0.984239in}{0.706060in}}%
\pgfpathcurveto{\pgfqpoint{0.984239in}{0.695010in}}{\pgfqpoint{0.988630in}{0.684411in}}{\pgfqpoint{0.996443in}{0.676598in}}%
\pgfpathcurveto{\pgfqpoint{1.004257in}{0.668784in}}{\pgfqpoint{1.014856in}{0.664394in}}{\pgfqpoint{1.025906in}{0.664394in}}%
\pgfpathclose%
\pgfusepath{stroke,fill}%
\end{pgfscope}%
\begin{pgfscope}%
\pgfpathrectangle{\pgfqpoint{0.800000in}{0.528000in}}{\pgfqpoint{4.960000in}{3.696000in}}%
\pgfusepath{clip}%
\pgfsetbuttcap%
\pgfsetroundjoin%
\definecolor{currentfill}{rgb}{0.000000,0.000000,0.000000}%
\pgfsetfillcolor{currentfill}%
\pgfsetlinewidth{1.003750pt}%
\definecolor{currentstroke}{rgb}{0.000000,0.000000,0.000000}%
\pgfsetstrokecolor{currentstroke}%
\pgfsetdash{}{0pt}%
\pgfpathmoveto{\pgfqpoint{1.025906in}{0.664394in}}%
\pgfpathcurveto{\pgfqpoint{1.036956in}{0.664394in}}{\pgfqpoint{1.047555in}{0.668784in}}{\pgfqpoint{1.055369in}{0.676598in}}%
\pgfpathcurveto{\pgfqpoint{1.063182in}{0.684411in}}{\pgfqpoint{1.067573in}{0.695010in}}{\pgfqpoint{1.067573in}{0.706060in}}%
\pgfpathcurveto{\pgfqpoint{1.067573in}{0.717111in}}{\pgfqpoint{1.063182in}{0.727710in}}{\pgfqpoint{1.055369in}{0.735523in}}%
\pgfpathcurveto{\pgfqpoint{1.047555in}{0.743337in}}{\pgfqpoint{1.036956in}{0.747727in}}{\pgfqpoint{1.025906in}{0.747727in}}%
\pgfpathcurveto{\pgfqpoint{1.014856in}{0.747727in}}{\pgfqpoint{1.004257in}{0.743337in}}{\pgfqpoint{0.996443in}{0.735523in}}%
\pgfpathcurveto{\pgfqpoint{0.988630in}{0.727710in}}{\pgfqpoint{0.984239in}{0.717111in}}{\pgfqpoint{0.984239in}{0.706060in}}%
\pgfpathcurveto{\pgfqpoint{0.984239in}{0.695010in}}{\pgfqpoint{0.988630in}{0.684411in}}{\pgfqpoint{0.996443in}{0.676598in}}%
\pgfpathcurveto{\pgfqpoint{1.004257in}{0.668784in}}{\pgfqpoint{1.014856in}{0.664394in}}{\pgfqpoint{1.025906in}{0.664394in}}%
\pgfpathclose%
\pgfusepath{stroke,fill}%
\end{pgfscope}%
\begin{pgfscope}%
\pgfpathrectangle{\pgfqpoint{0.800000in}{0.528000in}}{\pgfqpoint{4.960000in}{3.696000in}}%
\pgfusepath{clip}%
\pgfsetbuttcap%
\pgfsetroundjoin%
\definecolor{currentfill}{rgb}{0.000000,0.000000,0.000000}%
\pgfsetfillcolor{currentfill}%
\pgfsetlinewidth{1.003750pt}%
\definecolor{currentstroke}{rgb}{0.000000,0.000000,0.000000}%
\pgfsetstrokecolor{currentstroke}%
\pgfsetdash{}{0pt}%
\pgfpathmoveto{\pgfqpoint{1.025906in}{0.664394in}}%
\pgfpathcurveto{\pgfqpoint{1.036956in}{0.664394in}}{\pgfqpoint{1.047555in}{0.668784in}}{\pgfqpoint{1.055369in}{0.676598in}}%
\pgfpathcurveto{\pgfqpoint{1.063182in}{0.684411in}}{\pgfqpoint{1.067573in}{0.695010in}}{\pgfqpoint{1.067573in}{0.706060in}}%
\pgfpathcurveto{\pgfqpoint{1.067573in}{0.717111in}}{\pgfqpoint{1.063182in}{0.727710in}}{\pgfqpoint{1.055369in}{0.735523in}}%
\pgfpathcurveto{\pgfqpoint{1.047555in}{0.743337in}}{\pgfqpoint{1.036956in}{0.747727in}}{\pgfqpoint{1.025906in}{0.747727in}}%
\pgfpathcurveto{\pgfqpoint{1.014856in}{0.747727in}}{\pgfqpoint{1.004257in}{0.743337in}}{\pgfqpoint{0.996443in}{0.735523in}}%
\pgfpathcurveto{\pgfqpoint{0.988630in}{0.727710in}}{\pgfqpoint{0.984239in}{0.717111in}}{\pgfqpoint{0.984239in}{0.706060in}}%
\pgfpathcurveto{\pgfqpoint{0.984239in}{0.695010in}}{\pgfqpoint{0.988630in}{0.684411in}}{\pgfqpoint{0.996443in}{0.676598in}}%
\pgfpathcurveto{\pgfqpoint{1.004257in}{0.668784in}}{\pgfqpoint{1.014856in}{0.664394in}}{\pgfqpoint{1.025906in}{0.664394in}}%
\pgfpathclose%
\pgfusepath{stroke,fill}%
\end{pgfscope}%
\begin{pgfscope}%
\pgfpathrectangle{\pgfqpoint{0.800000in}{0.528000in}}{\pgfqpoint{4.960000in}{3.696000in}}%
\pgfusepath{clip}%
\pgfsetbuttcap%
\pgfsetroundjoin%
\definecolor{currentfill}{rgb}{0.000000,0.000000,0.000000}%
\pgfsetfillcolor{currentfill}%
\pgfsetlinewidth{1.003750pt}%
\definecolor{currentstroke}{rgb}{0.000000,0.000000,0.000000}%
\pgfsetstrokecolor{currentstroke}%
\pgfsetdash{}{0pt}%
\pgfpathmoveto{\pgfqpoint{1.025906in}{0.664394in}}%
\pgfpathcurveto{\pgfqpoint{1.036956in}{0.664394in}}{\pgfqpoint{1.047555in}{0.668784in}}{\pgfqpoint{1.055369in}{0.676598in}}%
\pgfpathcurveto{\pgfqpoint{1.063182in}{0.684411in}}{\pgfqpoint{1.067573in}{0.695010in}}{\pgfqpoint{1.067573in}{0.706060in}}%
\pgfpathcurveto{\pgfqpoint{1.067573in}{0.717111in}}{\pgfqpoint{1.063182in}{0.727710in}}{\pgfqpoint{1.055369in}{0.735523in}}%
\pgfpathcurveto{\pgfqpoint{1.047555in}{0.743337in}}{\pgfqpoint{1.036956in}{0.747727in}}{\pgfqpoint{1.025906in}{0.747727in}}%
\pgfpathcurveto{\pgfqpoint{1.014856in}{0.747727in}}{\pgfqpoint{1.004257in}{0.743337in}}{\pgfqpoint{0.996443in}{0.735523in}}%
\pgfpathcurveto{\pgfqpoint{0.988630in}{0.727710in}}{\pgfqpoint{0.984239in}{0.717111in}}{\pgfqpoint{0.984239in}{0.706060in}}%
\pgfpathcurveto{\pgfqpoint{0.984239in}{0.695010in}}{\pgfqpoint{0.988630in}{0.684411in}}{\pgfqpoint{0.996443in}{0.676598in}}%
\pgfpathcurveto{\pgfqpoint{1.004257in}{0.668784in}}{\pgfqpoint{1.014856in}{0.664394in}}{\pgfqpoint{1.025906in}{0.664394in}}%
\pgfpathclose%
\pgfusepath{stroke,fill}%
\end{pgfscope}%
\begin{pgfscope}%
\pgfpathrectangle{\pgfqpoint{0.800000in}{0.528000in}}{\pgfqpoint{4.960000in}{3.696000in}}%
\pgfusepath{clip}%
\pgfsetbuttcap%
\pgfsetroundjoin%
\definecolor{currentfill}{rgb}{0.000000,0.000000,0.000000}%
\pgfsetfillcolor{currentfill}%
\pgfsetlinewidth{1.003750pt}%
\definecolor{currentstroke}{rgb}{0.000000,0.000000,0.000000}%
\pgfsetstrokecolor{currentstroke}%
\pgfsetdash{}{0pt}%
\pgfpathmoveto{\pgfqpoint{1.025906in}{0.664394in}}%
\pgfpathcurveto{\pgfqpoint{1.036956in}{0.664394in}}{\pgfqpoint{1.047555in}{0.668784in}}{\pgfqpoint{1.055369in}{0.676598in}}%
\pgfpathcurveto{\pgfqpoint{1.063182in}{0.684411in}}{\pgfqpoint{1.067573in}{0.695010in}}{\pgfqpoint{1.067573in}{0.706060in}}%
\pgfpathcurveto{\pgfqpoint{1.067573in}{0.717111in}}{\pgfqpoint{1.063182in}{0.727710in}}{\pgfqpoint{1.055369in}{0.735523in}}%
\pgfpathcurveto{\pgfqpoint{1.047555in}{0.743337in}}{\pgfqpoint{1.036956in}{0.747727in}}{\pgfqpoint{1.025906in}{0.747727in}}%
\pgfpathcurveto{\pgfqpoint{1.014856in}{0.747727in}}{\pgfqpoint{1.004257in}{0.743337in}}{\pgfqpoint{0.996443in}{0.735523in}}%
\pgfpathcurveto{\pgfqpoint{0.988630in}{0.727710in}}{\pgfqpoint{0.984239in}{0.717111in}}{\pgfqpoint{0.984239in}{0.706060in}}%
\pgfpathcurveto{\pgfqpoint{0.984239in}{0.695010in}}{\pgfqpoint{0.988630in}{0.684411in}}{\pgfqpoint{0.996443in}{0.676598in}}%
\pgfpathcurveto{\pgfqpoint{1.004257in}{0.668784in}}{\pgfqpoint{1.014856in}{0.664394in}}{\pgfqpoint{1.025906in}{0.664394in}}%
\pgfpathclose%
\pgfusepath{stroke,fill}%
\end{pgfscope}%
\begin{pgfscope}%
\pgfpathrectangle{\pgfqpoint{0.800000in}{0.528000in}}{\pgfqpoint{4.960000in}{3.696000in}}%
\pgfusepath{clip}%
\pgfsetbuttcap%
\pgfsetroundjoin%
\definecolor{currentfill}{rgb}{0.000000,0.000000,0.000000}%
\pgfsetfillcolor{currentfill}%
\pgfsetlinewidth{1.003750pt}%
\definecolor{currentstroke}{rgb}{0.000000,0.000000,0.000000}%
\pgfsetstrokecolor{currentstroke}%
\pgfsetdash{}{0pt}%
\pgfpathmoveto{\pgfqpoint{1.025906in}{0.664394in}}%
\pgfpathcurveto{\pgfqpoint{1.036956in}{0.664394in}}{\pgfqpoint{1.047555in}{0.668784in}}{\pgfqpoint{1.055369in}{0.676598in}}%
\pgfpathcurveto{\pgfqpoint{1.063182in}{0.684411in}}{\pgfqpoint{1.067573in}{0.695010in}}{\pgfqpoint{1.067573in}{0.706060in}}%
\pgfpathcurveto{\pgfqpoint{1.067573in}{0.717111in}}{\pgfqpoint{1.063182in}{0.727710in}}{\pgfqpoint{1.055369in}{0.735523in}}%
\pgfpathcurveto{\pgfqpoint{1.047555in}{0.743337in}}{\pgfqpoint{1.036956in}{0.747727in}}{\pgfqpoint{1.025906in}{0.747727in}}%
\pgfpathcurveto{\pgfqpoint{1.014856in}{0.747727in}}{\pgfqpoint{1.004257in}{0.743337in}}{\pgfqpoint{0.996443in}{0.735523in}}%
\pgfpathcurveto{\pgfqpoint{0.988630in}{0.727710in}}{\pgfqpoint{0.984239in}{0.717111in}}{\pgfqpoint{0.984239in}{0.706060in}}%
\pgfpathcurveto{\pgfqpoint{0.984239in}{0.695010in}}{\pgfqpoint{0.988630in}{0.684411in}}{\pgfqpoint{0.996443in}{0.676598in}}%
\pgfpathcurveto{\pgfqpoint{1.004257in}{0.668784in}}{\pgfqpoint{1.014856in}{0.664394in}}{\pgfqpoint{1.025906in}{0.664394in}}%
\pgfpathclose%
\pgfusepath{stroke,fill}%
\end{pgfscope}%
\begin{pgfscope}%
\pgfpathrectangle{\pgfqpoint{0.800000in}{0.528000in}}{\pgfqpoint{4.960000in}{3.696000in}}%
\pgfusepath{clip}%
\pgfsetbuttcap%
\pgfsetroundjoin%
\definecolor{currentfill}{rgb}{0.000000,0.000000,0.000000}%
\pgfsetfillcolor{currentfill}%
\pgfsetlinewidth{1.003750pt}%
\definecolor{currentstroke}{rgb}{0.000000,0.000000,0.000000}%
\pgfsetstrokecolor{currentstroke}%
\pgfsetdash{}{0pt}%
\pgfpathmoveto{\pgfqpoint{1.025906in}{0.664394in}}%
\pgfpathcurveto{\pgfqpoint{1.036956in}{0.664394in}}{\pgfqpoint{1.047555in}{0.668784in}}{\pgfqpoint{1.055369in}{0.676598in}}%
\pgfpathcurveto{\pgfqpoint{1.063182in}{0.684411in}}{\pgfqpoint{1.067573in}{0.695010in}}{\pgfqpoint{1.067573in}{0.706060in}}%
\pgfpathcurveto{\pgfqpoint{1.067573in}{0.717111in}}{\pgfqpoint{1.063182in}{0.727710in}}{\pgfqpoint{1.055369in}{0.735523in}}%
\pgfpathcurveto{\pgfqpoint{1.047555in}{0.743337in}}{\pgfqpoint{1.036956in}{0.747727in}}{\pgfqpoint{1.025906in}{0.747727in}}%
\pgfpathcurveto{\pgfqpoint{1.014856in}{0.747727in}}{\pgfqpoint{1.004257in}{0.743337in}}{\pgfqpoint{0.996443in}{0.735523in}}%
\pgfpathcurveto{\pgfqpoint{0.988630in}{0.727710in}}{\pgfqpoint{0.984239in}{0.717111in}}{\pgfqpoint{0.984239in}{0.706060in}}%
\pgfpathcurveto{\pgfqpoint{0.984239in}{0.695010in}}{\pgfqpoint{0.988630in}{0.684411in}}{\pgfqpoint{0.996443in}{0.676598in}}%
\pgfpathcurveto{\pgfqpoint{1.004257in}{0.668784in}}{\pgfqpoint{1.014856in}{0.664394in}}{\pgfqpoint{1.025906in}{0.664394in}}%
\pgfpathclose%
\pgfusepath{stroke,fill}%
\end{pgfscope}%
\begin{pgfscope}%
\pgfpathrectangle{\pgfqpoint{0.800000in}{0.528000in}}{\pgfqpoint{4.960000in}{3.696000in}}%
\pgfusepath{clip}%
\pgfsetbuttcap%
\pgfsetroundjoin%
\definecolor{currentfill}{rgb}{0.000000,0.000000,0.000000}%
\pgfsetfillcolor{currentfill}%
\pgfsetlinewidth{1.003750pt}%
\definecolor{currentstroke}{rgb}{0.000000,0.000000,0.000000}%
\pgfsetstrokecolor{currentstroke}%
\pgfsetdash{}{0pt}%
\pgfpathmoveto{\pgfqpoint{1.025906in}{0.664394in}}%
\pgfpathcurveto{\pgfqpoint{1.036956in}{0.664394in}}{\pgfqpoint{1.047555in}{0.668784in}}{\pgfqpoint{1.055369in}{0.676598in}}%
\pgfpathcurveto{\pgfqpoint{1.063182in}{0.684411in}}{\pgfqpoint{1.067573in}{0.695010in}}{\pgfqpoint{1.067573in}{0.706060in}}%
\pgfpathcurveto{\pgfqpoint{1.067573in}{0.717111in}}{\pgfqpoint{1.063182in}{0.727710in}}{\pgfqpoint{1.055369in}{0.735523in}}%
\pgfpathcurveto{\pgfqpoint{1.047555in}{0.743337in}}{\pgfqpoint{1.036956in}{0.747727in}}{\pgfqpoint{1.025906in}{0.747727in}}%
\pgfpathcurveto{\pgfqpoint{1.014856in}{0.747727in}}{\pgfqpoint{1.004257in}{0.743337in}}{\pgfqpoint{0.996443in}{0.735523in}}%
\pgfpathcurveto{\pgfqpoint{0.988630in}{0.727710in}}{\pgfqpoint{0.984239in}{0.717111in}}{\pgfqpoint{0.984239in}{0.706060in}}%
\pgfpathcurveto{\pgfqpoint{0.984239in}{0.695010in}}{\pgfqpoint{0.988630in}{0.684411in}}{\pgfqpoint{0.996443in}{0.676598in}}%
\pgfpathcurveto{\pgfqpoint{1.004257in}{0.668784in}}{\pgfqpoint{1.014856in}{0.664394in}}{\pgfqpoint{1.025906in}{0.664394in}}%
\pgfpathclose%
\pgfusepath{stroke,fill}%
\end{pgfscope}%
\begin{pgfscope}%
\pgfpathrectangle{\pgfqpoint{0.800000in}{0.528000in}}{\pgfqpoint{4.960000in}{3.696000in}}%
\pgfusepath{clip}%
\pgfsetbuttcap%
\pgfsetroundjoin%
\definecolor{currentfill}{rgb}{0.000000,0.000000,0.000000}%
\pgfsetfillcolor{currentfill}%
\pgfsetlinewidth{1.003750pt}%
\definecolor{currentstroke}{rgb}{0.000000,0.000000,0.000000}%
\pgfsetstrokecolor{currentstroke}%
\pgfsetdash{}{0pt}%
\pgfpathmoveto{\pgfqpoint{1.025906in}{0.664394in}}%
\pgfpathcurveto{\pgfqpoint{1.036956in}{0.664394in}}{\pgfqpoint{1.047555in}{0.668784in}}{\pgfqpoint{1.055369in}{0.676598in}}%
\pgfpathcurveto{\pgfqpoint{1.063182in}{0.684411in}}{\pgfqpoint{1.067573in}{0.695010in}}{\pgfqpoint{1.067573in}{0.706060in}}%
\pgfpathcurveto{\pgfqpoint{1.067573in}{0.717111in}}{\pgfqpoint{1.063182in}{0.727710in}}{\pgfqpoint{1.055369in}{0.735523in}}%
\pgfpathcurveto{\pgfqpoint{1.047555in}{0.743337in}}{\pgfqpoint{1.036956in}{0.747727in}}{\pgfqpoint{1.025906in}{0.747727in}}%
\pgfpathcurveto{\pgfqpoint{1.014856in}{0.747727in}}{\pgfqpoint{1.004257in}{0.743337in}}{\pgfqpoint{0.996443in}{0.735523in}}%
\pgfpathcurveto{\pgfqpoint{0.988630in}{0.727710in}}{\pgfqpoint{0.984239in}{0.717111in}}{\pgfqpoint{0.984239in}{0.706060in}}%
\pgfpathcurveto{\pgfqpoint{0.984239in}{0.695010in}}{\pgfqpoint{0.988630in}{0.684411in}}{\pgfqpoint{0.996443in}{0.676598in}}%
\pgfpathcurveto{\pgfqpoint{1.004257in}{0.668784in}}{\pgfqpoint{1.014856in}{0.664394in}}{\pgfqpoint{1.025906in}{0.664394in}}%
\pgfpathclose%
\pgfusepath{stroke,fill}%
\end{pgfscope}%
\begin{pgfscope}%
\pgfpathrectangle{\pgfqpoint{0.800000in}{0.528000in}}{\pgfqpoint{4.960000in}{3.696000in}}%
\pgfusepath{clip}%
\pgfsetbuttcap%
\pgfsetroundjoin%
\definecolor{currentfill}{rgb}{0.000000,0.000000,0.000000}%
\pgfsetfillcolor{currentfill}%
\pgfsetlinewidth{1.003750pt}%
\definecolor{currentstroke}{rgb}{0.000000,0.000000,0.000000}%
\pgfsetstrokecolor{currentstroke}%
\pgfsetdash{}{0pt}%
\pgfpathmoveto{\pgfqpoint{1.025906in}{0.664394in}}%
\pgfpathcurveto{\pgfqpoint{1.036956in}{0.664394in}}{\pgfqpoint{1.047555in}{0.668784in}}{\pgfqpoint{1.055369in}{0.676598in}}%
\pgfpathcurveto{\pgfqpoint{1.063182in}{0.684411in}}{\pgfqpoint{1.067573in}{0.695010in}}{\pgfqpoint{1.067573in}{0.706060in}}%
\pgfpathcurveto{\pgfqpoint{1.067573in}{0.717111in}}{\pgfqpoint{1.063182in}{0.727710in}}{\pgfqpoint{1.055369in}{0.735523in}}%
\pgfpathcurveto{\pgfqpoint{1.047555in}{0.743337in}}{\pgfqpoint{1.036956in}{0.747727in}}{\pgfqpoint{1.025906in}{0.747727in}}%
\pgfpathcurveto{\pgfqpoint{1.014856in}{0.747727in}}{\pgfqpoint{1.004257in}{0.743337in}}{\pgfqpoint{0.996443in}{0.735523in}}%
\pgfpathcurveto{\pgfqpoint{0.988630in}{0.727710in}}{\pgfqpoint{0.984239in}{0.717111in}}{\pgfqpoint{0.984239in}{0.706060in}}%
\pgfpathcurveto{\pgfqpoint{0.984239in}{0.695010in}}{\pgfqpoint{0.988630in}{0.684411in}}{\pgfqpoint{0.996443in}{0.676598in}}%
\pgfpathcurveto{\pgfqpoint{1.004257in}{0.668784in}}{\pgfqpoint{1.014856in}{0.664394in}}{\pgfqpoint{1.025906in}{0.664394in}}%
\pgfpathclose%
\pgfusepath{stroke,fill}%
\end{pgfscope}%
\begin{pgfscope}%
\pgfpathrectangle{\pgfqpoint{0.800000in}{0.528000in}}{\pgfqpoint{4.960000in}{3.696000in}}%
\pgfusepath{clip}%
\pgfsetbuttcap%
\pgfsetroundjoin%
\definecolor{currentfill}{rgb}{0.000000,0.000000,0.000000}%
\pgfsetfillcolor{currentfill}%
\pgfsetlinewidth{1.003750pt}%
\definecolor{currentstroke}{rgb}{0.000000,0.000000,0.000000}%
\pgfsetstrokecolor{currentstroke}%
\pgfsetdash{}{0pt}%
\pgfpathmoveto{\pgfqpoint{1.025906in}{0.664394in}}%
\pgfpathcurveto{\pgfqpoint{1.036956in}{0.664394in}}{\pgfqpoint{1.047555in}{0.668784in}}{\pgfqpoint{1.055369in}{0.676598in}}%
\pgfpathcurveto{\pgfqpoint{1.063182in}{0.684411in}}{\pgfqpoint{1.067573in}{0.695010in}}{\pgfqpoint{1.067573in}{0.706060in}}%
\pgfpathcurveto{\pgfqpoint{1.067573in}{0.717111in}}{\pgfqpoint{1.063182in}{0.727710in}}{\pgfqpoint{1.055369in}{0.735523in}}%
\pgfpathcurveto{\pgfqpoint{1.047555in}{0.743337in}}{\pgfqpoint{1.036956in}{0.747727in}}{\pgfqpoint{1.025906in}{0.747727in}}%
\pgfpathcurveto{\pgfqpoint{1.014856in}{0.747727in}}{\pgfqpoint{1.004257in}{0.743337in}}{\pgfqpoint{0.996443in}{0.735523in}}%
\pgfpathcurveto{\pgfqpoint{0.988630in}{0.727710in}}{\pgfqpoint{0.984239in}{0.717111in}}{\pgfqpoint{0.984239in}{0.706060in}}%
\pgfpathcurveto{\pgfqpoint{0.984239in}{0.695010in}}{\pgfqpoint{0.988630in}{0.684411in}}{\pgfqpoint{0.996443in}{0.676598in}}%
\pgfpathcurveto{\pgfqpoint{1.004257in}{0.668784in}}{\pgfqpoint{1.014856in}{0.664394in}}{\pgfqpoint{1.025906in}{0.664394in}}%
\pgfpathclose%
\pgfusepath{stroke,fill}%
\end{pgfscope}%
\begin{pgfscope}%
\pgfpathrectangle{\pgfqpoint{0.800000in}{0.528000in}}{\pgfqpoint{4.960000in}{3.696000in}}%
\pgfusepath{clip}%
\pgfsetbuttcap%
\pgfsetroundjoin%
\definecolor{currentfill}{rgb}{0.000000,0.000000,0.000000}%
\pgfsetfillcolor{currentfill}%
\pgfsetlinewidth{1.003750pt}%
\definecolor{currentstroke}{rgb}{0.000000,0.000000,0.000000}%
\pgfsetstrokecolor{currentstroke}%
\pgfsetdash{}{0pt}%
\pgfpathmoveto{\pgfqpoint{1.025906in}{0.664394in}}%
\pgfpathcurveto{\pgfqpoint{1.036956in}{0.664394in}}{\pgfqpoint{1.047555in}{0.668784in}}{\pgfqpoint{1.055369in}{0.676598in}}%
\pgfpathcurveto{\pgfqpoint{1.063182in}{0.684411in}}{\pgfqpoint{1.067573in}{0.695010in}}{\pgfqpoint{1.067573in}{0.706060in}}%
\pgfpathcurveto{\pgfqpoint{1.067573in}{0.717111in}}{\pgfqpoint{1.063182in}{0.727710in}}{\pgfqpoint{1.055369in}{0.735523in}}%
\pgfpathcurveto{\pgfqpoint{1.047555in}{0.743337in}}{\pgfqpoint{1.036956in}{0.747727in}}{\pgfqpoint{1.025906in}{0.747727in}}%
\pgfpathcurveto{\pgfqpoint{1.014856in}{0.747727in}}{\pgfqpoint{1.004257in}{0.743337in}}{\pgfqpoint{0.996443in}{0.735523in}}%
\pgfpathcurveto{\pgfqpoint{0.988630in}{0.727710in}}{\pgfqpoint{0.984239in}{0.717111in}}{\pgfqpoint{0.984239in}{0.706060in}}%
\pgfpathcurveto{\pgfqpoint{0.984239in}{0.695010in}}{\pgfqpoint{0.988630in}{0.684411in}}{\pgfqpoint{0.996443in}{0.676598in}}%
\pgfpathcurveto{\pgfqpoint{1.004257in}{0.668784in}}{\pgfqpoint{1.014856in}{0.664394in}}{\pgfqpoint{1.025906in}{0.664394in}}%
\pgfpathclose%
\pgfusepath{stroke,fill}%
\end{pgfscope}%
\begin{pgfscope}%
\pgfpathrectangle{\pgfqpoint{0.800000in}{0.528000in}}{\pgfqpoint{4.960000in}{3.696000in}}%
\pgfusepath{clip}%
\pgfsetbuttcap%
\pgfsetroundjoin%
\definecolor{currentfill}{rgb}{0.000000,0.000000,0.000000}%
\pgfsetfillcolor{currentfill}%
\pgfsetlinewidth{1.003750pt}%
\definecolor{currentstroke}{rgb}{0.000000,0.000000,0.000000}%
\pgfsetstrokecolor{currentstroke}%
\pgfsetdash{}{0pt}%
\pgfpathmoveto{\pgfqpoint{1.025906in}{0.664394in}}%
\pgfpathcurveto{\pgfqpoint{1.036956in}{0.664394in}}{\pgfqpoint{1.047555in}{0.668784in}}{\pgfqpoint{1.055369in}{0.676598in}}%
\pgfpathcurveto{\pgfqpoint{1.063182in}{0.684411in}}{\pgfqpoint{1.067573in}{0.695010in}}{\pgfqpoint{1.067573in}{0.706060in}}%
\pgfpathcurveto{\pgfqpoint{1.067573in}{0.717111in}}{\pgfqpoint{1.063182in}{0.727710in}}{\pgfqpoint{1.055369in}{0.735523in}}%
\pgfpathcurveto{\pgfqpoint{1.047555in}{0.743337in}}{\pgfqpoint{1.036956in}{0.747727in}}{\pgfqpoint{1.025906in}{0.747727in}}%
\pgfpathcurveto{\pgfqpoint{1.014856in}{0.747727in}}{\pgfqpoint{1.004257in}{0.743337in}}{\pgfqpoint{0.996443in}{0.735523in}}%
\pgfpathcurveto{\pgfqpoint{0.988630in}{0.727710in}}{\pgfqpoint{0.984239in}{0.717111in}}{\pgfqpoint{0.984239in}{0.706060in}}%
\pgfpathcurveto{\pgfqpoint{0.984239in}{0.695010in}}{\pgfqpoint{0.988630in}{0.684411in}}{\pgfqpoint{0.996443in}{0.676598in}}%
\pgfpathcurveto{\pgfqpoint{1.004257in}{0.668784in}}{\pgfqpoint{1.014856in}{0.664394in}}{\pgfqpoint{1.025906in}{0.664394in}}%
\pgfpathclose%
\pgfusepath{stroke,fill}%
\end{pgfscope}%
\begin{pgfscope}%
\pgfpathrectangle{\pgfqpoint{0.800000in}{0.528000in}}{\pgfqpoint{4.960000in}{3.696000in}}%
\pgfusepath{clip}%
\pgfsetbuttcap%
\pgfsetroundjoin%
\definecolor{currentfill}{rgb}{0.000000,0.000000,0.000000}%
\pgfsetfillcolor{currentfill}%
\pgfsetlinewidth{1.003750pt}%
\definecolor{currentstroke}{rgb}{0.000000,0.000000,0.000000}%
\pgfsetstrokecolor{currentstroke}%
\pgfsetdash{}{0pt}%
\pgfpathmoveto{\pgfqpoint{1.025906in}{0.664394in}}%
\pgfpathcurveto{\pgfqpoint{1.036956in}{0.664394in}}{\pgfqpoint{1.047555in}{0.668784in}}{\pgfqpoint{1.055369in}{0.676598in}}%
\pgfpathcurveto{\pgfqpoint{1.063182in}{0.684411in}}{\pgfqpoint{1.067573in}{0.695010in}}{\pgfqpoint{1.067573in}{0.706060in}}%
\pgfpathcurveto{\pgfqpoint{1.067573in}{0.717111in}}{\pgfqpoint{1.063182in}{0.727710in}}{\pgfqpoint{1.055369in}{0.735523in}}%
\pgfpathcurveto{\pgfqpoint{1.047555in}{0.743337in}}{\pgfqpoint{1.036956in}{0.747727in}}{\pgfqpoint{1.025906in}{0.747727in}}%
\pgfpathcurveto{\pgfqpoint{1.014856in}{0.747727in}}{\pgfqpoint{1.004257in}{0.743337in}}{\pgfqpoint{0.996443in}{0.735523in}}%
\pgfpathcurveto{\pgfqpoint{0.988630in}{0.727710in}}{\pgfqpoint{0.984239in}{0.717111in}}{\pgfqpoint{0.984239in}{0.706060in}}%
\pgfpathcurveto{\pgfqpoint{0.984239in}{0.695010in}}{\pgfqpoint{0.988630in}{0.684411in}}{\pgfqpoint{0.996443in}{0.676598in}}%
\pgfpathcurveto{\pgfqpoint{1.004257in}{0.668784in}}{\pgfqpoint{1.014856in}{0.664394in}}{\pgfqpoint{1.025906in}{0.664394in}}%
\pgfpathclose%
\pgfusepath{stroke,fill}%
\end{pgfscope}%
\begin{pgfscope}%
\pgfpathrectangle{\pgfqpoint{0.800000in}{0.528000in}}{\pgfqpoint{4.960000in}{3.696000in}}%
\pgfusepath{clip}%
\pgfsetbuttcap%
\pgfsetroundjoin%
\definecolor{currentfill}{rgb}{0.000000,0.000000,0.000000}%
\pgfsetfillcolor{currentfill}%
\pgfsetlinewidth{1.003750pt}%
\definecolor{currentstroke}{rgb}{0.000000,0.000000,0.000000}%
\pgfsetstrokecolor{currentstroke}%
\pgfsetdash{}{0pt}%
\pgfpathmoveto{\pgfqpoint{1.025906in}{0.664394in}}%
\pgfpathcurveto{\pgfqpoint{1.036956in}{0.664394in}}{\pgfqpoint{1.047555in}{0.668784in}}{\pgfqpoint{1.055369in}{0.676598in}}%
\pgfpathcurveto{\pgfqpoint{1.063182in}{0.684411in}}{\pgfqpoint{1.067573in}{0.695010in}}{\pgfqpoint{1.067573in}{0.706060in}}%
\pgfpathcurveto{\pgfqpoint{1.067573in}{0.717111in}}{\pgfqpoint{1.063182in}{0.727710in}}{\pgfqpoint{1.055369in}{0.735523in}}%
\pgfpathcurveto{\pgfqpoint{1.047555in}{0.743337in}}{\pgfqpoint{1.036956in}{0.747727in}}{\pgfqpoint{1.025906in}{0.747727in}}%
\pgfpathcurveto{\pgfqpoint{1.014856in}{0.747727in}}{\pgfqpoint{1.004257in}{0.743337in}}{\pgfqpoint{0.996443in}{0.735523in}}%
\pgfpathcurveto{\pgfqpoint{0.988630in}{0.727710in}}{\pgfqpoint{0.984239in}{0.717111in}}{\pgfqpoint{0.984239in}{0.706060in}}%
\pgfpathcurveto{\pgfqpoint{0.984239in}{0.695010in}}{\pgfqpoint{0.988630in}{0.684411in}}{\pgfqpoint{0.996443in}{0.676598in}}%
\pgfpathcurveto{\pgfqpoint{1.004257in}{0.668784in}}{\pgfqpoint{1.014856in}{0.664394in}}{\pgfqpoint{1.025906in}{0.664394in}}%
\pgfpathclose%
\pgfusepath{stroke,fill}%
\end{pgfscope}%
\begin{pgfscope}%
\pgfpathrectangle{\pgfqpoint{0.800000in}{0.528000in}}{\pgfqpoint{4.960000in}{3.696000in}}%
\pgfusepath{clip}%
\pgfsetbuttcap%
\pgfsetroundjoin%
\definecolor{currentfill}{rgb}{0.000000,0.000000,0.000000}%
\pgfsetfillcolor{currentfill}%
\pgfsetlinewidth{1.003750pt}%
\definecolor{currentstroke}{rgb}{0.000000,0.000000,0.000000}%
\pgfsetstrokecolor{currentstroke}%
\pgfsetdash{}{0pt}%
\pgfpathmoveto{\pgfqpoint{1.025906in}{0.664394in}}%
\pgfpathcurveto{\pgfqpoint{1.036956in}{0.664394in}}{\pgfqpoint{1.047555in}{0.668784in}}{\pgfqpoint{1.055369in}{0.676598in}}%
\pgfpathcurveto{\pgfqpoint{1.063182in}{0.684411in}}{\pgfqpoint{1.067573in}{0.695010in}}{\pgfqpoint{1.067573in}{0.706060in}}%
\pgfpathcurveto{\pgfqpoint{1.067573in}{0.717111in}}{\pgfqpoint{1.063182in}{0.727710in}}{\pgfqpoint{1.055369in}{0.735523in}}%
\pgfpathcurveto{\pgfqpoint{1.047555in}{0.743337in}}{\pgfqpoint{1.036956in}{0.747727in}}{\pgfqpoint{1.025906in}{0.747727in}}%
\pgfpathcurveto{\pgfqpoint{1.014856in}{0.747727in}}{\pgfqpoint{1.004257in}{0.743337in}}{\pgfqpoint{0.996443in}{0.735523in}}%
\pgfpathcurveto{\pgfqpoint{0.988630in}{0.727710in}}{\pgfqpoint{0.984239in}{0.717111in}}{\pgfqpoint{0.984239in}{0.706060in}}%
\pgfpathcurveto{\pgfqpoint{0.984239in}{0.695010in}}{\pgfqpoint{0.988630in}{0.684411in}}{\pgfqpoint{0.996443in}{0.676598in}}%
\pgfpathcurveto{\pgfqpoint{1.004257in}{0.668784in}}{\pgfqpoint{1.014856in}{0.664394in}}{\pgfqpoint{1.025906in}{0.664394in}}%
\pgfpathclose%
\pgfusepath{stroke,fill}%
\end{pgfscope}%
\begin{pgfscope}%
\pgfpathrectangle{\pgfqpoint{0.800000in}{0.528000in}}{\pgfqpoint{4.960000in}{3.696000in}}%
\pgfusepath{clip}%
\pgfsetbuttcap%
\pgfsetroundjoin%
\definecolor{currentfill}{rgb}{0.000000,0.000000,0.000000}%
\pgfsetfillcolor{currentfill}%
\pgfsetlinewidth{1.003750pt}%
\definecolor{currentstroke}{rgb}{0.000000,0.000000,0.000000}%
\pgfsetstrokecolor{currentstroke}%
\pgfsetdash{}{0pt}%
\pgfpathmoveto{\pgfqpoint{1.025906in}{0.664394in}}%
\pgfpathcurveto{\pgfqpoint{1.036956in}{0.664394in}}{\pgfqpoint{1.047555in}{0.668784in}}{\pgfqpoint{1.055369in}{0.676598in}}%
\pgfpathcurveto{\pgfqpoint{1.063182in}{0.684411in}}{\pgfqpoint{1.067573in}{0.695010in}}{\pgfqpoint{1.067573in}{0.706060in}}%
\pgfpathcurveto{\pgfqpoint{1.067573in}{0.717111in}}{\pgfqpoint{1.063182in}{0.727710in}}{\pgfqpoint{1.055369in}{0.735523in}}%
\pgfpathcurveto{\pgfqpoint{1.047555in}{0.743337in}}{\pgfqpoint{1.036956in}{0.747727in}}{\pgfqpoint{1.025906in}{0.747727in}}%
\pgfpathcurveto{\pgfqpoint{1.014856in}{0.747727in}}{\pgfqpoint{1.004257in}{0.743337in}}{\pgfqpoint{0.996443in}{0.735523in}}%
\pgfpathcurveto{\pgfqpoint{0.988630in}{0.727710in}}{\pgfqpoint{0.984239in}{0.717111in}}{\pgfqpoint{0.984239in}{0.706060in}}%
\pgfpathcurveto{\pgfqpoint{0.984239in}{0.695010in}}{\pgfqpoint{0.988630in}{0.684411in}}{\pgfqpoint{0.996443in}{0.676598in}}%
\pgfpathcurveto{\pgfqpoint{1.004257in}{0.668784in}}{\pgfqpoint{1.014856in}{0.664394in}}{\pgfqpoint{1.025906in}{0.664394in}}%
\pgfpathclose%
\pgfusepath{stroke,fill}%
\end{pgfscope}%
\begin{pgfscope}%
\pgfpathrectangle{\pgfqpoint{0.800000in}{0.528000in}}{\pgfqpoint{4.960000in}{3.696000in}}%
\pgfusepath{clip}%
\pgfsetbuttcap%
\pgfsetroundjoin%
\definecolor{currentfill}{rgb}{0.000000,0.000000,0.000000}%
\pgfsetfillcolor{currentfill}%
\pgfsetlinewidth{1.003750pt}%
\definecolor{currentstroke}{rgb}{0.000000,0.000000,0.000000}%
\pgfsetstrokecolor{currentstroke}%
\pgfsetdash{}{0pt}%
\pgfpathmoveto{\pgfqpoint{1.025906in}{0.664394in}}%
\pgfpathcurveto{\pgfqpoint{1.036956in}{0.664394in}}{\pgfqpoint{1.047555in}{0.668784in}}{\pgfqpoint{1.055369in}{0.676598in}}%
\pgfpathcurveto{\pgfqpoint{1.063182in}{0.684411in}}{\pgfqpoint{1.067573in}{0.695010in}}{\pgfqpoint{1.067573in}{0.706060in}}%
\pgfpathcurveto{\pgfqpoint{1.067573in}{0.717111in}}{\pgfqpoint{1.063182in}{0.727710in}}{\pgfqpoint{1.055369in}{0.735523in}}%
\pgfpathcurveto{\pgfqpoint{1.047555in}{0.743337in}}{\pgfqpoint{1.036956in}{0.747727in}}{\pgfqpoint{1.025906in}{0.747727in}}%
\pgfpathcurveto{\pgfqpoint{1.014856in}{0.747727in}}{\pgfqpoint{1.004257in}{0.743337in}}{\pgfqpoint{0.996443in}{0.735523in}}%
\pgfpathcurveto{\pgfqpoint{0.988630in}{0.727710in}}{\pgfqpoint{0.984239in}{0.717111in}}{\pgfqpoint{0.984239in}{0.706060in}}%
\pgfpathcurveto{\pgfqpoint{0.984239in}{0.695010in}}{\pgfqpoint{0.988630in}{0.684411in}}{\pgfqpoint{0.996443in}{0.676598in}}%
\pgfpathcurveto{\pgfqpoint{1.004257in}{0.668784in}}{\pgfqpoint{1.014856in}{0.664394in}}{\pgfqpoint{1.025906in}{0.664394in}}%
\pgfpathclose%
\pgfusepath{stroke,fill}%
\end{pgfscope}%
\begin{pgfscope}%
\pgfpathrectangle{\pgfqpoint{0.800000in}{0.528000in}}{\pgfqpoint{4.960000in}{3.696000in}}%
\pgfusepath{clip}%
\pgfsetbuttcap%
\pgfsetroundjoin%
\definecolor{currentfill}{rgb}{0.000000,0.000000,0.000000}%
\pgfsetfillcolor{currentfill}%
\pgfsetlinewidth{1.003750pt}%
\definecolor{currentstroke}{rgb}{0.000000,0.000000,0.000000}%
\pgfsetstrokecolor{currentstroke}%
\pgfsetdash{}{0pt}%
\pgfpathmoveto{\pgfqpoint{1.025906in}{0.664394in}}%
\pgfpathcurveto{\pgfqpoint{1.036956in}{0.664394in}}{\pgfqpoint{1.047555in}{0.668784in}}{\pgfqpoint{1.055369in}{0.676598in}}%
\pgfpathcurveto{\pgfqpoint{1.063182in}{0.684411in}}{\pgfqpoint{1.067573in}{0.695010in}}{\pgfqpoint{1.067573in}{0.706060in}}%
\pgfpathcurveto{\pgfqpoint{1.067573in}{0.717111in}}{\pgfqpoint{1.063182in}{0.727710in}}{\pgfqpoint{1.055369in}{0.735523in}}%
\pgfpathcurveto{\pgfqpoint{1.047555in}{0.743337in}}{\pgfqpoint{1.036956in}{0.747727in}}{\pgfqpoint{1.025906in}{0.747727in}}%
\pgfpathcurveto{\pgfqpoint{1.014856in}{0.747727in}}{\pgfqpoint{1.004257in}{0.743337in}}{\pgfqpoint{0.996443in}{0.735523in}}%
\pgfpathcurveto{\pgfqpoint{0.988630in}{0.727710in}}{\pgfqpoint{0.984239in}{0.717111in}}{\pgfqpoint{0.984239in}{0.706060in}}%
\pgfpathcurveto{\pgfqpoint{0.984239in}{0.695010in}}{\pgfqpoint{0.988630in}{0.684411in}}{\pgfqpoint{0.996443in}{0.676598in}}%
\pgfpathcurveto{\pgfqpoint{1.004257in}{0.668784in}}{\pgfqpoint{1.014856in}{0.664394in}}{\pgfqpoint{1.025906in}{0.664394in}}%
\pgfpathclose%
\pgfusepath{stroke,fill}%
\end{pgfscope}%
\begin{pgfscope}%
\pgfpathrectangle{\pgfqpoint{0.800000in}{0.528000in}}{\pgfqpoint{4.960000in}{3.696000in}}%
\pgfusepath{clip}%
\pgfsetbuttcap%
\pgfsetroundjoin%
\definecolor{currentfill}{rgb}{0.000000,0.000000,0.000000}%
\pgfsetfillcolor{currentfill}%
\pgfsetlinewidth{1.003750pt}%
\definecolor{currentstroke}{rgb}{0.000000,0.000000,0.000000}%
\pgfsetstrokecolor{currentstroke}%
\pgfsetdash{}{0pt}%
\pgfpathmoveto{\pgfqpoint{1.025906in}{0.664394in}}%
\pgfpathcurveto{\pgfqpoint{1.036956in}{0.664394in}}{\pgfqpoint{1.047555in}{0.668784in}}{\pgfqpoint{1.055369in}{0.676598in}}%
\pgfpathcurveto{\pgfqpoint{1.063182in}{0.684411in}}{\pgfqpoint{1.067573in}{0.695010in}}{\pgfqpoint{1.067573in}{0.706060in}}%
\pgfpathcurveto{\pgfqpoint{1.067573in}{0.717111in}}{\pgfqpoint{1.063182in}{0.727710in}}{\pgfqpoint{1.055369in}{0.735523in}}%
\pgfpathcurveto{\pgfqpoint{1.047555in}{0.743337in}}{\pgfqpoint{1.036956in}{0.747727in}}{\pgfqpoint{1.025906in}{0.747727in}}%
\pgfpathcurveto{\pgfqpoint{1.014856in}{0.747727in}}{\pgfqpoint{1.004257in}{0.743337in}}{\pgfqpoint{0.996443in}{0.735523in}}%
\pgfpathcurveto{\pgfqpoint{0.988630in}{0.727710in}}{\pgfqpoint{0.984239in}{0.717111in}}{\pgfqpoint{0.984239in}{0.706060in}}%
\pgfpathcurveto{\pgfqpoint{0.984239in}{0.695010in}}{\pgfqpoint{0.988630in}{0.684411in}}{\pgfqpoint{0.996443in}{0.676598in}}%
\pgfpathcurveto{\pgfqpoint{1.004257in}{0.668784in}}{\pgfqpoint{1.014856in}{0.664394in}}{\pgfqpoint{1.025906in}{0.664394in}}%
\pgfpathclose%
\pgfusepath{stroke,fill}%
\end{pgfscope}%
\begin{pgfscope}%
\pgfpathrectangle{\pgfqpoint{0.800000in}{0.528000in}}{\pgfqpoint{4.960000in}{3.696000in}}%
\pgfusepath{clip}%
\pgfsetbuttcap%
\pgfsetroundjoin%
\definecolor{currentfill}{rgb}{0.000000,0.000000,0.000000}%
\pgfsetfillcolor{currentfill}%
\pgfsetlinewidth{1.003750pt}%
\definecolor{currentstroke}{rgb}{0.000000,0.000000,0.000000}%
\pgfsetstrokecolor{currentstroke}%
\pgfsetdash{}{0pt}%
\pgfpathmoveto{\pgfqpoint{1.025906in}{0.664394in}}%
\pgfpathcurveto{\pgfqpoint{1.036956in}{0.664394in}}{\pgfqpoint{1.047555in}{0.668784in}}{\pgfqpoint{1.055369in}{0.676598in}}%
\pgfpathcurveto{\pgfqpoint{1.063182in}{0.684411in}}{\pgfqpoint{1.067573in}{0.695010in}}{\pgfqpoint{1.067573in}{0.706060in}}%
\pgfpathcurveto{\pgfqpoint{1.067573in}{0.717111in}}{\pgfqpoint{1.063182in}{0.727710in}}{\pgfqpoint{1.055369in}{0.735523in}}%
\pgfpathcurveto{\pgfqpoint{1.047555in}{0.743337in}}{\pgfqpoint{1.036956in}{0.747727in}}{\pgfqpoint{1.025906in}{0.747727in}}%
\pgfpathcurveto{\pgfqpoint{1.014856in}{0.747727in}}{\pgfqpoint{1.004257in}{0.743337in}}{\pgfqpoint{0.996443in}{0.735523in}}%
\pgfpathcurveto{\pgfqpoint{0.988630in}{0.727710in}}{\pgfqpoint{0.984239in}{0.717111in}}{\pgfqpoint{0.984239in}{0.706060in}}%
\pgfpathcurveto{\pgfqpoint{0.984239in}{0.695010in}}{\pgfqpoint{0.988630in}{0.684411in}}{\pgfqpoint{0.996443in}{0.676598in}}%
\pgfpathcurveto{\pgfqpoint{1.004257in}{0.668784in}}{\pgfqpoint{1.014856in}{0.664394in}}{\pgfqpoint{1.025906in}{0.664394in}}%
\pgfpathclose%
\pgfusepath{stroke,fill}%
\end{pgfscope}%
\begin{pgfscope}%
\pgfpathrectangle{\pgfqpoint{0.800000in}{0.528000in}}{\pgfqpoint{4.960000in}{3.696000in}}%
\pgfusepath{clip}%
\pgfsetbuttcap%
\pgfsetroundjoin%
\definecolor{currentfill}{rgb}{0.000000,0.000000,0.000000}%
\pgfsetfillcolor{currentfill}%
\pgfsetlinewidth{1.003750pt}%
\definecolor{currentstroke}{rgb}{0.000000,0.000000,0.000000}%
\pgfsetstrokecolor{currentstroke}%
\pgfsetdash{}{0pt}%
\pgfpathmoveto{\pgfqpoint{1.025906in}{0.664394in}}%
\pgfpathcurveto{\pgfqpoint{1.036956in}{0.664394in}}{\pgfqpoint{1.047555in}{0.668784in}}{\pgfqpoint{1.055369in}{0.676598in}}%
\pgfpathcurveto{\pgfqpoint{1.063182in}{0.684411in}}{\pgfqpoint{1.067573in}{0.695010in}}{\pgfqpoint{1.067573in}{0.706060in}}%
\pgfpathcurveto{\pgfqpoint{1.067573in}{0.717111in}}{\pgfqpoint{1.063182in}{0.727710in}}{\pgfqpoint{1.055369in}{0.735523in}}%
\pgfpathcurveto{\pgfqpoint{1.047555in}{0.743337in}}{\pgfqpoint{1.036956in}{0.747727in}}{\pgfqpoint{1.025906in}{0.747727in}}%
\pgfpathcurveto{\pgfqpoint{1.014856in}{0.747727in}}{\pgfqpoint{1.004257in}{0.743337in}}{\pgfqpoint{0.996443in}{0.735523in}}%
\pgfpathcurveto{\pgfqpoint{0.988630in}{0.727710in}}{\pgfqpoint{0.984239in}{0.717111in}}{\pgfqpoint{0.984239in}{0.706060in}}%
\pgfpathcurveto{\pgfqpoint{0.984239in}{0.695010in}}{\pgfqpoint{0.988630in}{0.684411in}}{\pgfqpoint{0.996443in}{0.676598in}}%
\pgfpathcurveto{\pgfqpoint{1.004257in}{0.668784in}}{\pgfqpoint{1.014856in}{0.664394in}}{\pgfqpoint{1.025906in}{0.664394in}}%
\pgfpathclose%
\pgfusepath{stroke,fill}%
\end{pgfscope}%
\begin{pgfscope}%
\pgfpathrectangle{\pgfqpoint{0.800000in}{0.528000in}}{\pgfqpoint{4.960000in}{3.696000in}}%
\pgfusepath{clip}%
\pgfsetbuttcap%
\pgfsetroundjoin%
\definecolor{currentfill}{rgb}{0.000000,0.000000,0.000000}%
\pgfsetfillcolor{currentfill}%
\pgfsetlinewidth{1.003750pt}%
\definecolor{currentstroke}{rgb}{0.000000,0.000000,0.000000}%
\pgfsetstrokecolor{currentstroke}%
\pgfsetdash{}{0pt}%
\pgfpathmoveto{\pgfqpoint{1.025906in}{0.664394in}}%
\pgfpathcurveto{\pgfqpoint{1.036956in}{0.664394in}}{\pgfqpoint{1.047555in}{0.668784in}}{\pgfqpoint{1.055369in}{0.676598in}}%
\pgfpathcurveto{\pgfqpoint{1.063182in}{0.684411in}}{\pgfqpoint{1.067573in}{0.695010in}}{\pgfqpoint{1.067573in}{0.706060in}}%
\pgfpathcurveto{\pgfqpoint{1.067573in}{0.717111in}}{\pgfqpoint{1.063182in}{0.727710in}}{\pgfqpoint{1.055369in}{0.735523in}}%
\pgfpathcurveto{\pgfqpoint{1.047555in}{0.743337in}}{\pgfqpoint{1.036956in}{0.747727in}}{\pgfqpoint{1.025906in}{0.747727in}}%
\pgfpathcurveto{\pgfqpoint{1.014856in}{0.747727in}}{\pgfqpoint{1.004257in}{0.743337in}}{\pgfqpoint{0.996443in}{0.735523in}}%
\pgfpathcurveto{\pgfqpoint{0.988630in}{0.727710in}}{\pgfqpoint{0.984239in}{0.717111in}}{\pgfqpoint{0.984239in}{0.706060in}}%
\pgfpathcurveto{\pgfqpoint{0.984239in}{0.695010in}}{\pgfqpoint{0.988630in}{0.684411in}}{\pgfqpoint{0.996443in}{0.676598in}}%
\pgfpathcurveto{\pgfqpoint{1.004257in}{0.668784in}}{\pgfqpoint{1.014856in}{0.664394in}}{\pgfqpoint{1.025906in}{0.664394in}}%
\pgfpathclose%
\pgfusepath{stroke,fill}%
\end{pgfscope}%
\begin{pgfscope}%
\pgfpathrectangle{\pgfqpoint{0.800000in}{0.528000in}}{\pgfqpoint{4.960000in}{3.696000in}}%
\pgfusepath{clip}%
\pgfsetbuttcap%
\pgfsetroundjoin%
\definecolor{currentfill}{rgb}{0.000000,0.000000,0.000000}%
\pgfsetfillcolor{currentfill}%
\pgfsetlinewidth{1.003750pt}%
\definecolor{currentstroke}{rgb}{0.000000,0.000000,0.000000}%
\pgfsetstrokecolor{currentstroke}%
\pgfsetdash{}{0pt}%
\pgfpathmoveto{\pgfqpoint{1.025906in}{0.664394in}}%
\pgfpathcurveto{\pgfqpoint{1.036956in}{0.664394in}}{\pgfqpoint{1.047555in}{0.668784in}}{\pgfqpoint{1.055369in}{0.676598in}}%
\pgfpathcurveto{\pgfqpoint{1.063182in}{0.684411in}}{\pgfqpoint{1.067573in}{0.695010in}}{\pgfqpoint{1.067573in}{0.706060in}}%
\pgfpathcurveto{\pgfqpoint{1.067573in}{0.717111in}}{\pgfqpoint{1.063182in}{0.727710in}}{\pgfqpoint{1.055369in}{0.735523in}}%
\pgfpathcurveto{\pgfqpoint{1.047555in}{0.743337in}}{\pgfqpoint{1.036956in}{0.747727in}}{\pgfqpoint{1.025906in}{0.747727in}}%
\pgfpathcurveto{\pgfqpoint{1.014856in}{0.747727in}}{\pgfqpoint{1.004257in}{0.743337in}}{\pgfqpoint{0.996443in}{0.735523in}}%
\pgfpathcurveto{\pgfqpoint{0.988630in}{0.727710in}}{\pgfqpoint{0.984239in}{0.717111in}}{\pgfqpoint{0.984239in}{0.706060in}}%
\pgfpathcurveto{\pgfqpoint{0.984239in}{0.695010in}}{\pgfqpoint{0.988630in}{0.684411in}}{\pgfqpoint{0.996443in}{0.676598in}}%
\pgfpathcurveto{\pgfqpoint{1.004257in}{0.668784in}}{\pgfqpoint{1.014856in}{0.664394in}}{\pgfqpoint{1.025906in}{0.664394in}}%
\pgfpathclose%
\pgfusepath{stroke,fill}%
\end{pgfscope}%
\begin{pgfscope}%
\pgfpathrectangle{\pgfqpoint{0.800000in}{0.528000in}}{\pgfqpoint{4.960000in}{3.696000in}}%
\pgfusepath{clip}%
\pgfsetbuttcap%
\pgfsetroundjoin%
\definecolor{currentfill}{rgb}{0.000000,0.000000,0.000000}%
\pgfsetfillcolor{currentfill}%
\pgfsetlinewidth{1.003750pt}%
\definecolor{currentstroke}{rgb}{0.000000,0.000000,0.000000}%
\pgfsetstrokecolor{currentstroke}%
\pgfsetdash{}{0pt}%
\pgfpathmoveto{\pgfqpoint{1.025906in}{0.664394in}}%
\pgfpathcurveto{\pgfqpoint{1.036956in}{0.664394in}}{\pgfqpoint{1.047555in}{0.668784in}}{\pgfqpoint{1.055369in}{0.676598in}}%
\pgfpathcurveto{\pgfqpoint{1.063182in}{0.684411in}}{\pgfqpoint{1.067573in}{0.695010in}}{\pgfqpoint{1.067573in}{0.706060in}}%
\pgfpathcurveto{\pgfqpoint{1.067573in}{0.717111in}}{\pgfqpoint{1.063182in}{0.727710in}}{\pgfqpoint{1.055369in}{0.735523in}}%
\pgfpathcurveto{\pgfqpoint{1.047555in}{0.743337in}}{\pgfqpoint{1.036956in}{0.747727in}}{\pgfqpoint{1.025906in}{0.747727in}}%
\pgfpathcurveto{\pgfqpoint{1.014856in}{0.747727in}}{\pgfqpoint{1.004257in}{0.743337in}}{\pgfqpoint{0.996443in}{0.735523in}}%
\pgfpathcurveto{\pgfqpoint{0.988630in}{0.727710in}}{\pgfqpoint{0.984239in}{0.717111in}}{\pgfqpoint{0.984239in}{0.706060in}}%
\pgfpathcurveto{\pgfqpoint{0.984239in}{0.695010in}}{\pgfqpoint{0.988630in}{0.684411in}}{\pgfqpoint{0.996443in}{0.676598in}}%
\pgfpathcurveto{\pgfqpoint{1.004257in}{0.668784in}}{\pgfqpoint{1.014856in}{0.664394in}}{\pgfqpoint{1.025906in}{0.664394in}}%
\pgfpathclose%
\pgfusepath{stroke,fill}%
\end{pgfscope}%
\begin{pgfscope}%
\pgfpathrectangle{\pgfqpoint{0.800000in}{0.528000in}}{\pgfqpoint{4.960000in}{3.696000in}}%
\pgfusepath{clip}%
\pgfsetbuttcap%
\pgfsetroundjoin%
\definecolor{currentfill}{rgb}{0.000000,0.000000,0.000000}%
\pgfsetfillcolor{currentfill}%
\pgfsetlinewidth{1.003750pt}%
\definecolor{currentstroke}{rgb}{0.000000,0.000000,0.000000}%
\pgfsetstrokecolor{currentstroke}%
\pgfsetdash{}{0pt}%
\pgfpathmoveto{\pgfqpoint{1.025906in}{0.664394in}}%
\pgfpathcurveto{\pgfqpoint{1.036956in}{0.664394in}}{\pgfqpoint{1.047555in}{0.668784in}}{\pgfqpoint{1.055369in}{0.676598in}}%
\pgfpathcurveto{\pgfqpoint{1.063182in}{0.684411in}}{\pgfqpoint{1.067573in}{0.695010in}}{\pgfqpoint{1.067573in}{0.706060in}}%
\pgfpathcurveto{\pgfqpoint{1.067573in}{0.717111in}}{\pgfqpoint{1.063182in}{0.727710in}}{\pgfqpoint{1.055369in}{0.735523in}}%
\pgfpathcurveto{\pgfqpoint{1.047555in}{0.743337in}}{\pgfqpoint{1.036956in}{0.747727in}}{\pgfqpoint{1.025906in}{0.747727in}}%
\pgfpathcurveto{\pgfqpoint{1.014856in}{0.747727in}}{\pgfqpoint{1.004257in}{0.743337in}}{\pgfqpoint{0.996443in}{0.735523in}}%
\pgfpathcurveto{\pgfqpoint{0.988630in}{0.727710in}}{\pgfqpoint{0.984239in}{0.717111in}}{\pgfqpoint{0.984239in}{0.706060in}}%
\pgfpathcurveto{\pgfqpoint{0.984239in}{0.695010in}}{\pgfqpoint{0.988630in}{0.684411in}}{\pgfqpoint{0.996443in}{0.676598in}}%
\pgfpathcurveto{\pgfqpoint{1.004257in}{0.668784in}}{\pgfqpoint{1.014856in}{0.664394in}}{\pgfqpoint{1.025906in}{0.664394in}}%
\pgfpathclose%
\pgfusepath{stroke,fill}%
\end{pgfscope}%
\begin{pgfscope}%
\pgfpathrectangle{\pgfqpoint{0.800000in}{0.528000in}}{\pgfqpoint{4.960000in}{3.696000in}}%
\pgfusepath{clip}%
\pgfsetbuttcap%
\pgfsetroundjoin%
\definecolor{currentfill}{rgb}{0.000000,0.000000,0.000000}%
\pgfsetfillcolor{currentfill}%
\pgfsetlinewidth{1.003750pt}%
\definecolor{currentstroke}{rgb}{0.000000,0.000000,0.000000}%
\pgfsetstrokecolor{currentstroke}%
\pgfsetdash{}{0pt}%
\pgfpathmoveto{\pgfqpoint{1.025906in}{0.664394in}}%
\pgfpathcurveto{\pgfqpoint{1.036956in}{0.664394in}}{\pgfqpoint{1.047555in}{0.668784in}}{\pgfqpoint{1.055369in}{0.676598in}}%
\pgfpathcurveto{\pgfqpoint{1.063182in}{0.684411in}}{\pgfqpoint{1.067573in}{0.695010in}}{\pgfqpoint{1.067573in}{0.706060in}}%
\pgfpathcurveto{\pgfqpoint{1.067573in}{0.717111in}}{\pgfqpoint{1.063182in}{0.727710in}}{\pgfqpoint{1.055369in}{0.735523in}}%
\pgfpathcurveto{\pgfqpoint{1.047555in}{0.743337in}}{\pgfqpoint{1.036956in}{0.747727in}}{\pgfqpoint{1.025906in}{0.747727in}}%
\pgfpathcurveto{\pgfqpoint{1.014856in}{0.747727in}}{\pgfqpoint{1.004257in}{0.743337in}}{\pgfqpoint{0.996443in}{0.735523in}}%
\pgfpathcurveto{\pgfqpoint{0.988630in}{0.727710in}}{\pgfqpoint{0.984239in}{0.717111in}}{\pgfqpoint{0.984239in}{0.706060in}}%
\pgfpathcurveto{\pgfqpoint{0.984239in}{0.695010in}}{\pgfqpoint{0.988630in}{0.684411in}}{\pgfqpoint{0.996443in}{0.676598in}}%
\pgfpathcurveto{\pgfqpoint{1.004257in}{0.668784in}}{\pgfqpoint{1.014856in}{0.664394in}}{\pgfqpoint{1.025906in}{0.664394in}}%
\pgfpathclose%
\pgfusepath{stroke,fill}%
\end{pgfscope}%
\begin{pgfscope}%
\pgfpathrectangle{\pgfqpoint{0.800000in}{0.528000in}}{\pgfqpoint{4.960000in}{3.696000in}}%
\pgfusepath{clip}%
\pgfsetbuttcap%
\pgfsetroundjoin%
\definecolor{currentfill}{rgb}{0.000000,0.000000,0.000000}%
\pgfsetfillcolor{currentfill}%
\pgfsetlinewidth{1.003750pt}%
\definecolor{currentstroke}{rgb}{0.000000,0.000000,0.000000}%
\pgfsetstrokecolor{currentstroke}%
\pgfsetdash{}{0pt}%
\pgfpathmoveto{\pgfqpoint{1.025906in}{0.664394in}}%
\pgfpathcurveto{\pgfqpoint{1.036956in}{0.664394in}}{\pgfqpoint{1.047555in}{0.668784in}}{\pgfqpoint{1.055369in}{0.676598in}}%
\pgfpathcurveto{\pgfqpoint{1.063182in}{0.684411in}}{\pgfqpoint{1.067573in}{0.695010in}}{\pgfqpoint{1.067573in}{0.706060in}}%
\pgfpathcurveto{\pgfqpoint{1.067573in}{0.717111in}}{\pgfqpoint{1.063182in}{0.727710in}}{\pgfqpoint{1.055369in}{0.735523in}}%
\pgfpathcurveto{\pgfqpoint{1.047555in}{0.743337in}}{\pgfqpoint{1.036956in}{0.747727in}}{\pgfqpoint{1.025906in}{0.747727in}}%
\pgfpathcurveto{\pgfqpoint{1.014856in}{0.747727in}}{\pgfqpoint{1.004257in}{0.743337in}}{\pgfqpoint{0.996443in}{0.735523in}}%
\pgfpathcurveto{\pgfqpoint{0.988630in}{0.727710in}}{\pgfqpoint{0.984239in}{0.717111in}}{\pgfqpoint{0.984239in}{0.706060in}}%
\pgfpathcurveto{\pgfqpoint{0.984239in}{0.695010in}}{\pgfqpoint{0.988630in}{0.684411in}}{\pgfqpoint{0.996443in}{0.676598in}}%
\pgfpathcurveto{\pgfqpoint{1.004257in}{0.668784in}}{\pgfqpoint{1.014856in}{0.664394in}}{\pgfqpoint{1.025906in}{0.664394in}}%
\pgfpathclose%
\pgfusepath{stroke,fill}%
\end{pgfscope}%
\begin{pgfscope}%
\pgfpathrectangle{\pgfqpoint{0.800000in}{0.528000in}}{\pgfqpoint{4.960000in}{3.696000in}}%
\pgfusepath{clip}%
\pgfsetbuttcap%
\pgfsetroundjoin%
\definecolor{currentfill}{rgb}{0.000000,0.000000,0.000000}%
\pgfsetfillcolor{currentfill}%
\pgfsetlinewidth{1.003750pt}%
\definecolor{currentstroke}{rgb}{0.000000,0.000000,0.000000}%
\pgfsetstrokecolor{currentstroke}%
\pgfsetdash{}{0pt}%
\pgfpathmoveto{\pgfqpoint{1.025906in}{0.664394in}}%
\pgfpathcurveto{\pgfqpoint{1.036956in}{0.664394in}}{\pgfqpoint{1.047555in}{0.668784in}}{\pgfqpoint{1.055369in}{0.676598in}}%
\pgfpathcurveto{\pgfqpoint{1.063182in}{0.684411in}}{\pgfqpoint{1.067573in}{0.695010in}}{\pgfqpoint{1.067573in}{0.706060in}}%
\pgfpathcurveto{\pgfqpoint{1.067573in}{0.717111in}}{\pgfqpoint{1.063182in}{0.727710in}}{\pgfqpoint{1.055369in}{0.735523in}}%
\pgfpathcurveto{\pgfqpoint{1.047555in}{0.743337in}}{\pgfqpoint{1.036956in}{0.747727in}}{\pgfqpoint{1.025906in}{0.747727in}}%
\pgfpathcurveto{\pgfqpoint{1.014856in}{0.747727in}}{\pgfqpoint{1.004257in}{0.743337in}}{\pgfqpoint{0.996443in}{0.735523in}}%
\pgfpathcurveto{\pgfqpoint{0.988630in}{0.727710in}}{\pgfqpoint{0.984239in}{0.717111in}}{\pgfqpoint{0.984239in}{0.706060in}}%
\pgfpathcurveto{\pgfqpoint{0.984239in}{0.695010in}}{\pgfqpoint{0.988630in}{0.684411in}}{\pgfqpoint{0.996443in}{0.676598in}}%
\pgfpathcurveto{\pgfqpoint{1.004257in}{0.668784in}}{\pgfqpoint{1.014856in}{0.664394in}}{\pgfqpoint{1.025906in}{0.664394in}}%
\pgfpathclose%
\pgfusepath{stroke,fill}%
\end{pgfscope}%
\begin{pgfscope}%
\pgfpathrectangle{\pgfqpoint{0.800000in}{0.528000in}}{\pgfqpoint{4.960000in}{3.696000in}}%
\pgfusepath{clip}%
\pgfsetbuttcap%
\pgfsetroundjoin%
\definecolor{currentfill}{rgb}{0.000000,0.000000,0.000000}%
\pgfsetfillcolor{currentfill}%
\pgfsetlinewidth{1.003750pt}%
\definecolor{currentstroke}{rgb}{0.000000,0.000000,0.000000}%
\pgfsetstrokecolor{currentstroke}%
\pgfsetdash{}{0pt}%
\pgfpathmoveto{\pgfqpoint{1.025906in}{0.664394in}}%
\pgfpathcurveto{\pgfqpoint{1.036956in}{0.664394in}}{\pgfqpoint{1.047555in}{0.668784in}}{\pgfqpoint{1.055369in}{0.676598in}}%
\pgfpathcurveto{\pgfqpoint{1.063182in}{0.684411in}}{\pgfqpoint{1.067573in}{0.695010in}}{\pgfqpoint{1.067573in}{0.706060in}}%
\pgfpathcurveto{\pgfqpoint{1.067573in}{0.717111in}}{\pgfqpoint{1.063182in}{0.727710in}}{\pgfqpoint{1.055369in}{0.735523in}}%
\pgfpathcurveto{\pgfqpoint{1.047555in}{0.743337in}}{\pgfqpoint{1.036956in}{0.747727in}}{\pgfqpoint{1.025906in}{0.747727in}}%
\pgfpathcurveto{\pgfqpoint{1.014856in}{0.747727in}}{\pgfqpoint{1.004257in}{0.743337in}}{\pgfqpoint{0.996443in}{0.735523in}}%
\pgfpathcurveto{\pgfqpoint{0.988630in}{0.727710in}}{\pgfqpoint{0.984239in}{0.717111in}}{\pgfqpoint{0.984239in}{0.706060in}}%
\pgfpathcurveto{\pgfqpoint{0.984239in}{0.695010in}}{\pgfqpoint{0.988630in}{0.684411in}}{\pgfqpoint{0.996443in}{0.676598in}}%
\pgfpathcurveto{\pgfqpoint{1.004257in}{0.668784in}}{\pgfqpoint{1.014856in}{0.664394in}}{\pgfqpoint{1.025906in}{0.664394in}}%
\pgfpathclose%
\pgfusepath{stroke,fill}%
\end{pgfscope}%
\begin{pgfscope}%
\pgfpathrectangle{\pgfqpoint{0.800000in}{0.528000in}}{\pgfqpoint{4.960000in}{3.696000in}}%
\pgfusepath{clip}%
\pgfsetbuttcap%
\pgfsetroundjoin%
\definecolor{currentfill}{rgb}{0.000000,0.000000,0.000000}%
\pgfsetfillcolor{currentfill}%
\pgfsetlinewidth{1.003750pt}%
\definecolor{currentstroke}{rgb}{0.000000,0.000000,0.000000}%
\pgfsetstrokecolor{currentstroke}%
\pgfsetdash{}{0pt}%
\pgfpathmoveto{\pgfqpoint{1.025906in}{0.664394in}}%
\pgfpathcurveto{\pgfqpoint{1.036956in}{0.664394in}}{\pgfqpoint{1.047555in}{0.668784in}}{\pgfqpoint{1.055369in}{0.676598in}}%
\pgfpathcurveto{\pgfqpoint{1.063182in}{0.684411in}}{\pgfqpoint{1.067573in}{0.695010in}}{\pgfqpoint{1.067573in}{0.706060in}}%
\pgfpathcurveto{\pgfqpoint{1.067573in}{0.717111in}}{\pgfqpoint{1.063182in}{0.727710in}}{\pgfqpoint{1.055369in}{0.735523in}}%
\pgfpathcurveto{\pgfqpoint{1.047555in}{0.743337in}}{\pgfqpoint{1.036956in}{0.747727in}}{\pgfqpoint{1.025906in}{0.747727in}}%
\pgfpathcurveto{\pgfqpoint{1.014856in}{0.747727in}}{\pgfqpoint{1.004257in}{0.743337in}}{\pgfqpoint{0.996443in}{0.735523in}}%
\pgfpathcurveto{\pgfqpoint{0.988630in}{0.727710in}}{\pgfqpoint{0.984239in}{0.717111in}}{\pgfqpoint{0.984239in}{0.706060in}}%
\pgfpathcurveto{\pgfqpoint{0.984239in}{0.695010in}}{\pgfqpoint{0.988630in}{0.684411in}}{\pgfqpoint{0.996443in}{0.676598in}}%
\pgfpathcurveto{\pgfqpoint{1.004257in}{0.668784in}}{\pgfqpoint{1.014856in}{0.664394in}}{\pgfqpoint{1.025906in}{0.664394in}}%
\pgfpathclose%
\pgfusepath{stroke,fill}%
\end{pgfscope}%
\begin{pgfscope}%
\pgfpathrectangle{\pgfqpoint{0.800000in}{0.528000in}}{\pgfqpoint{4.960000in}{3.696000in}}%
\pgfusepath{clip}%
\pgfsetbuttcap%
\pgfsetroundjoin%
\definecolor{currentfill}{rgb}{0.000000,0.000000,0.000000}%
\pgfsetfillcolor{currentfill}%
\pgfsetlinewidth{1.003750pt}%
\definecolor{currentstroke}{rgb}{0.000000,0.000000,0.000000}%
\pgfsetstrokecolor{currentstroke}%
\pgfsetdash{}{0pt}%
\pgfpathmoveto{\pgfqpoint{1.025906in}{0.664394in}}%
\pgfpathcurveto{\pgfqpoint{1.036956in}{0.664394in}}{\pgfqpoint{1.047555in}{0.668784in}}{\pgfqpoint{1.055369in}{0.676598in}}%
\pgfpathcurveto{\pgfqpoint{1.063182in}{0.684411in}}{\pgfqpoint{1.067573in}{0.695010in}}{\pgfqpoint{1.067573in}{0.706060in}}%
\pgfpathcurveto{\pgfqpoint{1.067573in}{0.717111in}}{\pgfqpoint{1.063182in}{0.727710in}}{\pgfqpoint{1.055369in}{0.735523in}}%
\pgfpathcurveto{\pgfqpoint{1.047555in}{0.743337in}}{\pgfqpoint{1.036956in}{0.747727in}}{\pgfqpoint{1.025906in}{0.747727in}}%
\pgfpathcurveto{\pgfqpoint{1.014856in}{0.747727in}}{\pgfqpoint{1.004257in}{0.743337in}}{\pgfqpoint{0.996443in}{0.735523in}}%
\pgfpathcurveto{\pgfqpoint{0.988630in}{0.727710in}}{\pgfqpoint{0.984239in}{0.717111in}}{\pgfqpoint{0.984239in}{0.706060in}}%
\pgfpathcurveto{\pgfqpoint{0.984239in}{0.695010in}}{\pgfqpoint{0.988630in}{0.684411in}}{\pgfqpoint{0.996443in}{0.676598in}}%
\pgfpathcurveto{\pgfqpoint{1.004257in}{0.668784in}}{\pgfqpoint{1.014856in}{0.664394in}}{\pgfqpoint{1.025906in}{0.664394in}}%
\pgfpathclose%
\pgfusepath{stroke,fill}%
\end{pgfscope}%
\begin{pgfscope}%
\pgfpathrectangle{\pgfqpoint{0.800000in}{0.528000in}}{\pgfqpoint{4.960000in}{3.696000in}}%
\pgfusepath{clip}%
\pgfsetbuttcap%
\pgfsetroundjoin%
\definecolor{currentfill}{rgb}{0.000000,0.000000,0.000000}%
\pgfsetfillcolor{currentfill}%
\pgfsetlinewidth{1.003750pt}%
\definecolor{currentstroke}{rgb}{0.000000,0.000000,0.000000}%
\pgfsetstrokecolor{currentstroke}%
\pgfsetdash{}{0pt}%
\pgfpathmoveto{\pgfqpoint{1.025906in}{0.664394in}}%
\pgfpathcurveto{\pgfqpoint{1.036956in}{0.664394in}}{\pgfqpoint{1.047555in}{0.668784in}}{\pgfqpoint{1.055369in}{0.676598in}}%
\pgfpathcurveto{\pgfqpoint{1.063182in}{0.684411in}}{\pgfqpoint{1.067573in}{0.695010in}}{\pgfqpoint{1.067573in}{0.706060in}}%
\pgfpathcurveto{\pgfqpoint{1.067573in}{0.717111in}}{\pgfqpoint{1.063182in}{0.727710in}}{\pgfqpoint{1.055369in}{0.735523in}}%
\pgfpathcurveto{\pgfqpoint{1.047555in}{0.743337in}}{\pgfqpoint{1.036956in}{0.747727in}}{\pgfqpoint{1.025906in}{0.747727in}}%
\pgfpathcurveto{\pgfqpoint{1.014856in}{0.747727in}}{\pgfqpoint{1.004257in}{0.743337in}}{\pgfqpoint{0.996443in}{0.735523in}}%
\pgfpathcurveto{\pgfqpoint{0.988630in}{0.727710in}}{\pgfqpoint{0.984239in}{0.717111in}}{\pgfqpoint{0.984239in}{0.706060in}}%
\pgfpathcurveto{\pgfqpoint{0.984239in}{0.695010in}}{\pgfqpoint{0.988630in}{0.684411in}}{\pgfqpoint{0.996443in}{0.676598in}}%
\pgfpathcurveto{\pgfqpoint{1.004257in}{0.668784in}}{\pgfqpoint{1.014856in}{0.664394in}}{\pgfqpoint{1.025906in}{0.664394in}}%
\pgfpathclose%
\pgfusepath{stroke,fill}%
\end{pgfscope}%
\begin{pgfscope}%
\pgfpathrectangle{\pgfqpoint{0.800000in}{0.528000in}}{\pgfqpoint{4.960000in}{3.696000in}}%
\pgfusepath{clip}%
\pgfsetbuttcap%
\pgfsetroundjoin%
\definecolor{currentfill}{rgb}{0.000000,0.000000,0.000000}%
\pgfsetfillcolor{currentfill}%
\pgfsetlinewidth{1.003750pt}%
\definecolor{currentstroke}{rgb}{0.000000,0.000000,0.000000}%
\pgfsetstrokecolor{currentstroke}%
\pgfsetdash{}{0pt}%
\pgfpathmoveto{\pgfqpoint{1.025906in}{0.664394in}}%
\pgfpathcurveto{\pgfqpoint{1.036956in}{0.664394in}}{\pgfqpoint{1.047555in}{0.668784in}}{\pgfqpoint{1.055369in}{0.676598in}}%
\pgfpathcurveto{\pgfqpoint{1.063182in}{0.684411in}}{\pgfqpoint{1.067573in}{0.695010in}}{\pgfqpoint{1.067573in}{0.706060in}}%
\pgfpathcurveto{\pgfqpoint{1.067573in}{0.717111in}}{\pgfqpoint{1.063182in}{0.727710in}}{\pgfqpoint{1.055369in}{0.735523in}}%
\pgfpathcurveto{\pgfqpoint{1.047555in}{0.743337in}}{\pgfqpoint{1.036956in}{0.747727in}}{\pgfqpoint{1.025906in}{0.747727in}}%
\pgfpathcurveto{\pgfqpoint{1.014856in}{0.747727in}}{\pgfqpoint{1.004257in}{0.743337in}}{\pgfqpoint{0.996443in}{0.735523in}}%
\pgfpathcurveto{\pgfqpoint{0.988630in}{0.727710in}}{\pgfqpoint{0.984239in}{0.717111in}}{\pgfqpoint{0.984239in}{0.706060in}}%
\pgfpathcurveto{\pgfqpoint{0.984239in}{0.695010in}}{\pgfqpoint{0.988630in}{0.684411in}}{\pgfqpoint{0.996443in}{0.676598in}}%
\pgfpathcurveto{\pgfqpoint{1.004257in}{0.668784in}}{\pgfqpoint{1.014856in}{0.664394in}}{\pgfqpoint{1.025906in}{0.664394in}}%
\pgfpathclose%
\pgfusepath{stroke,fill}%
\end{pgfscope}%
\begin{pgfscope}%
\pgfpathrectangle{\pgfqpoint{0.800000in}{0.528000in}}{\pgfqpoint{4.960000in}{3.696000in}}%
\pgfusepath{clip}%
\pgfsetbuttcap%
\pgfsetroundjoin%
\definecolor{currentfill}{rgb}{0.000000,0.000000,0.000000}%
\pgfsetfillcolor{currentfill}%
\pgfsetlinewidth{1.003750pt}%
\definecolor{currentstroke}{rgb}{0.000000,0.000000,0.000000}%
\pgfsetstrokecolor{currentstroke}%
\pgfsetdash{}{0pt}%
\pgfpathmoveto{\pgfqpoint{1.025906in}{0.664394in}}%
\pgfpathcurveto{\pgfqpoint{1.036956in}{0.664394in}}{\pgfqpoint{1.047555in}{0.668784in}}{\pgfqpoint{1.055369in}{0.676598in}}%
\pgfpathcurveto{\pgfqpoint{1.063182in}{0.684411in}}{\pgfqpoint{1.067573in}{0.695010in}}{\pgfqpoint{1.067573in}{0.706060in}}%
\pgfpathcurveto{\pgfqpoint{1.067573in}{0.717111in}}{\pgfqpoint{1.063182in}{0.727710in}}{\pgfqpoint{1.055369in}{0.735523in}}%
\pgfpathcurveto{\pgfqpoint{1.047555in}{0.743337in}}{\pgfqpoint{1.036956in}{0.747727in}}{\pgfqpoint{1.025906in}{0.747727in}}%
\pgfpathcurveto{\pgfqpoint{1.014856in}{0.747727in}}{\pgfqpoint{1.004257in}{0.743337in}}{\pgfqpoint{0.996443in}{0.735523in}}%
\pgfpathcurveto{\pgfqpoint{0.988630in}{0.727710in}}{\pgfqpoint{0.984239in}{0.717111in}}{\pgfqpoint{0.984239in}{0.706060in}}%
\pgfpathcurveto{\pgfqpoint{0.984239in}{0.695010in}}{\pgfqpoint{0.988630in}{0.684411in}}{\pgfqpoint{0.996443in}{0.676598in}}%
\pgfpathcurveto{\pgfqpoint{1.004257in}{0.668784in}}{\pgfqpoint{1.014856in}{0.664394in}}{\pgfqpoint{1.025906in}{0.664394in}}%
\pgfpathclose%
\pgfusepath{stroke,fill}%
\end{pgfscope}%
\begin{pgfscope}%
\pgfpathrectangle{\pgfqpoint{0.800000in}{0.528000in}}{\pgfqpoint{4.960000in}{3.696000in}}%
\pgfusepath{clip}%
\pgfsetbuttcap%
\pgfsetroundjoin%
\definecolor{currentfill}{rgb}{0.000000,0.000000,0.000000}%
\pgfsetfillcolor{currentfill}%
\pgfsetlinewidth{1.003750pt}%
\definecolor{currentstroke}{rgb}{0.000000,0.000000,0.000000}%
\pgfsetstrokecolor{currentstroke}%
\pgfsetdash{}{0pt}%
\pgfpathmoveto{\pgfqpoint{1.025906in}{0.664394in}}%
\pgfpathcurveto{\pgfqpoint{1.036956in}{0.664394in}}{\pgfqpoint{1.047555in}{0.668784in}}{\pgfqpoint{1.055369in}{0.676598in}}%
\pgfpathcurveto{\pgfqpoint{1.063182in}{0.684411in}}{\pgfqpoint{1.067573in}{0.695010in}}{\pgfqpoint{1.067573in}{0.706060in}}%
\pgfpathcurveto{\pgfqpoint{1.067573in}{0.717111in}}{\pgfqpoint{1.063182in}{0.727710in}}{\pgfqpoint{1.055369in}{0.735523in}}%
\pgfpathcurveto{\pgfqpoint{1.047555in}{0.743337in}}{\pgfqpoint{1.036956in}{0.747727in}}{\pgfqpoint{1.025906in}{0.747727in}}%
\pgfpathcurveto{\pgfqpoint{1.014856in}{0.747727in}}{\pgfqpoint{1.004257in}{0.743337in}}{\pgfqpoint{0.996443in}{0.735523in}}%
\pgfpathcurveto{\pgfqpoint{0.988630in}{0.727710in}}{\pgfqpoint{0.984239in}{0.717111in}}{\pgfqpoint{0.984239in}{0.706060in}}%
\pgfpathcurveto{\pgfqpoint{0.984239in}{0.695010in}}{\pgfqpoint{0.988630in}{0.684411in}}{\pgfqpoint{0.996443in}{0.676598in}}%
\pgfpathcurveto{\pgfqpoint{1.004257in}{0.668784in}}{\pgfqpoint{1.014856in}{0.664394in}}{\pgfqpoint{1.025906in}{0.664394in}}%
\pgfpathclose%
\pgfusepath{stroke,fill}%
\end{pgfscope}%
\begin{pgfscope}%
\pgfpathrectangle{\pgfqpoint{0.800000in}{0.528000in}}{\pgfqpoint{4.960000in}{3.696000in}}%
\pgfusepath{clip}%
\pgfsetbuttcap%
\pgfsetroundjoin%
\definecolor{currentfill}{rgb}{0.000000,0.000000,0.000000}%
\pgfsetfillcolor{currentfill}%
\pgfsetlinewidth{1.003750pt}%
\definecolor{currentstroke}{rgb}{0.000000,0.000000,0.000000}%
\pgfsetstrokecolor{currentstroke}%
\pgfsetdash{}{0pt}%
\pgfpathmoveto{\pgfqpoint{1.025906in}{0.664394in}}%
\pgfpathcurveto{\pgfqpoint{1.036956in}{0.664394in}}{\pgfqpoint{1.047555in}{0.668784in}}{\pgfqpoint{1.055369in}{0.676598in}}%
\pgfpathcurveto{\pgfqpoint{1.063182in}{0.684411in}}{\pgfqpoint{1.067573in}{0.695010in}}{\pgfqpoint{1.067573in}{0.706060in}}%
\pgfpathcurveto{\pgfqpoint{1.067573in}{0.717111in}}{\pgfqpoint{1.063182in}{0.727710in}}{\pgfqpoint{1.055369in}{0.735523in}}%
\pgfpathcurveto{\pgfqpoint{1.047555in}{0.743337in}}{\pgfqpoint{1.036956in}{0.747727in}}{\pgfqpoint{1.025906in}{0.747727in}}%
\pgfpathcurveto{\pgfqpoint{1.014856in}{0.747727in}}{\pgfqpoint{1.004257in}{0.743337in}}{\pgfqpoint{0.996443in}{0.735523in}}%
\pgfpathcurveto{\pgfqpoint{0.988630in}{0.727710in}}{\pgfqpoint{0.984239in}{0.717111in}}{\pgfqpoint{0.984239in}{0.706060in}}%
\pgfpathcurveto{\pgfqpoint{0.984239in}{0.695010in}}{\pgfqpoint{0.988630in}{0.684411in}}{\pgfqpoint{0.996443in}{0.676598in}}%
\pgfpathcurveto{\pgfqpoint{1.004257in}{0.668784in}}{\pgfqpoint{1.014856in}{0.664394in}}{\pgfqpoint{1.025906in}{0.664394in}}%
\pgfpathclose%
\pgfusepath{stroke,fill}%
\end{pgfscope}%
\begin{pgfscope}%
\pgfpathrectangle{\pgfqpoint{0.800000in}{0.528000in}}{\pgfqpoint{4.960000in}{3.696000in}}%
\pgfusepath{clip}%
\pgfsetbuttcap%
\pgfsetroundjoin%
\definecolor{currentfill}{rgb}{0.000000,0.000000,0.000000}%
\pgfsetfillcolor{currentfill}%
\pgfsetlinewidth{1.003750pt}%
\definecolor{currentstroke}{rgb}{0.000000,0.000000,0.000000}%
\pgfsetstrokecolor{currentstroke}%
\pgfsetdash{}{0pt}%
\pgfpathmoveto{\pgfqpoint{2.518786in}{1.771040in}}%
\pgfpathcurveto{\pgfqpoint{2.529836in}{1.771040in}}{\pgfqpoint{2.540435in}{1.775431in}}{\pgfqpoint{2.548249in}{1.783244in}}%
\pgfpathcurveto{\pgfqpoint{2.556062in}{1.791058in}}{\pgfqpoint{2.560452in}{1.801657in}}{\pgfqpoint{2.560452in}{1.812707in}}%
\pgfpathcurveto{\pgfqpoint{2.560452in}{1.823757in}}{\pgfqpoint{2.556062in}{1.834356in}}{\pgfqpoint{2.548249in}{1.842170in}}%
\pgfpathcurveto{\pgfqpoint{2.540435in}{1.849983in}}{\pgfqpoint{2.529836in}{1.854374in}}{\pgfqpoint{2.518786in}{1.854374in}}%
\pgfpathcurveto{\pgfqpoint{2.507736in}{1.854374in}}{\pgfqpoint{2.497137in}{1.849983in}}{\pgfqpoint{2.489323in}{1.842170in}}%
\pgfpathcurveto{\pgfqpoint{2.481509in}{1.834356in}}{\pgfqpoint{2.477119in}{1.823757in}}{\pgfqpoint{2.477119in}{1.812707in}}%
\pgfpathcurveto{\pgfqpoint{2.477119in}{1.801657in}}{\pgfqpoint{2.481509in}{1.791058in}}{\pgfqpoint{2.489323in}{1.783244in}}%
\pgfpathcurveto{\pgfqpoint{2.497137in}{1.775431in}}{\pgfqpoint{2.507736in}{1.771040in}}{\pgfqpoint{2.518786in}{1.771040in}}%
\pgfpathclose%
\pgfusepath{stroke,fill}%
\end{pgfscope}%
\begin{pgfscope}%
\pgfpathrectangle{\pgfqpoint{0.800000in}{0.528000in}}{\pgfqpoint{4.960000in}{3.696000in}}%
\pgfusepath{clip}%
\pgfsetbuttcap%
\pgfsetroundjoin%
\definecolor{currentfill}{rgb}{0.000000,0.000000,0.000000}%
\pgfsetfillcolor{currentfill}%
\pgfsetlinewidth{1.003750pt}%
\definecolor{currentstroke}{rgb}{0.000000,0.000000,0.000000}%
\pgfsetstrokecolor{currentstroke}%
\pgfsetdash{}{0pt}%
\pgfpathmoveto{\pgfqpoint{2.518786in}{1.771040in}}%
\pgfpathcurveto{\pgfqpoint{2.529836in}{1.771040in}}{\pgfqpoint{2.540435in}{1.775431in}}{\pgfqpoint{2.548249in}{1.783244in}}%
\pgfpathcurveto{\pgfqpoint{2.556062in}{1.791058in}}{\pgfqpoint{2.560452in}{1.801657in}}{\pgfqpoint{2.560452in}{1.812707in}}%
\pgfpathcurveto{\pgfqpoint{2.560452in}{1.823757in}}{\pgfqpoint{2.556062in}{1.834356in}}{\pgfqpoint{2.548249in}{1.842170in}}%
\pgfpathcurveto{\pgfqpoint{2.540435in}{1.849983in}}{\pgfqpoint{2.529836in}{1.854374in}}{\pgfqpoint{2.518786in}{1.854374in}}%
\pgfpathcurveto{\pgfqpoint{2.507736in}{1.854374in}}{\pgfqpoint{2.497137in}{1.849983in}}{\pgfqpoint{2.489323in}{1.842170in}}%
\pgfpathcurveto{\pgfqpoint{2.481509in}{1.834356in}}{\pgfqpoint{2.477119in}{1.823757in}}{\pgfqpoint{2.477119in}{1.812707in}}%
\pgfpathcurveto{\pgfqpoint{2.477119in}{1.801657in}}{\pgfqpoint{2.481509in}{1.791058in}}{\pgfqpoint{2.489323in}{1.783244in}}%
\pgfpathcurveto{\pgfqpoint{2.497137in}{1.775431in}}{\pgfqpoint{2.507736in}{1.771040in}}{\pgfqpoint{2.518786in}{1.771040in}}%
\pgfpathclose%
\pgfusepath{stroke,fill}%
\end{pgfscope}%
\begin{pgfscope}%
\pgfpathrectangle{\pgfqpoint{0.800000in}{0.528000in}}{\pgfqpoint{4.960000in}{3.696000in}}%
\pgfusepath{clip}%
\pgfsetbuttcap%
\pgfsetroundjoin%
\definecolor{currentfill}{rgb}{0.000000,0.000000,0.000000}%
\pgfsetfillcolor{currentfill}%
\pgfsetlinewidth{1.003750pt}%
\definecolor{currentstroke}{rgb}{0.000000,0.000000,0.000000}%
\pgfsetstrokecolor{currentstroke}%
\pgfsetdash{}{0pt}%
\pgfpathmoveto{\pgfqpoint{2.518786in}{1.771040in}}%
\pgfpathcurveto{\pgfqpoint{2.529836in}{1.771040in}}{\pgfqpoint{2.540435in}{1.775431in}}{\pgfqpoint{2.548249in}{1.783244in}}%
\pgfpathcurveto{\pgfqpoint{2.556062in}{1.791058in}}{\pgfqpoint{2.560452in}{1.801657in}}{\pgfqpoint{2.560452in}{1.812707in}}%
\pgfpathcurveto{\pgfqpoint{2.560452in}{1.823757in}}{\pgfqpoint{2.556062in}{1.834356in}}{\pgfqpoint{2.548249in}{1.842170in}}%
\pgfpathcurveto{\pgfqpoint{2.540435in}{1.849983in}}{\pgfqpoint{2.529836in}{1.854374in}}{\pgfqpoint{2.518786in}{1.854374in}}%
\pgfpathcurveto{\pgfqpoint{2.507736in}{1.854374in}}{\pgfqpoint{2.497137in}{1.849983in}}{\pgfqpoint{2.489323in}{1.842170in}}%
\pgfpathcurveto{\pgfqpoint{2.481509in}{1.834356in}}{\pgfqpoint{2.477119in}{1.823757in}}{\pgfqpoint{2.477119in}{1.812707in}}%
\pgfpathcurveto{\pgfqpoint{2.477119in}{1.801657in}}{\pgfqpoint{2.481509in}{1.791058in}}{\pgfqpoint{2.489323in}{1.783244in}}%
\pgfpathcurveto{\pgfqpoint{2.497137in}{1.775431in}}{\pgfqpoint{2.507736in}{1.771040in}}{\pgfqpoint{2.518786in}{1.771040in}}%
\pgfpathclose%
\pgfusepath{stroke,fill}%
\end{pgfscope}%
\begin{pgfscope}%
\pgfpathrectangle{\pgfqpoint{0.800000in}{0.528000in}}{\pgfqpoint{4.960000in}{3.696000in}}%
\pgfusepath{clip}%
\pgfsetbuttcap%
\pgfsetroundjoin%
\definecolor{currentfill}{rgb}{0.000000,0.000000,0.000000}%
\pgfsetfillcolor{currentfill}%
\pgfsetlinewidth{1.003750pt}%
\definecolor{currentstroke}{rgb}{0.000000,0.000000,0.000000}%
\pgfsetstrokecolor{currentstroke}%
\pgfsetdash{}{0pt}%
\pgfpathmoveto{\pgfqpoint{2.518786in}{1.771040in}}%
\pgfpathcurveto{\pgfqpoint{2.529836in}{1.771040in}}{\pgfqpoint{2.540435in}{1.775431in}}{\pgfqpoint{2.548249in}{1.783244in}}%
\pgfpathcurveto{\pgfqpoint{2.556062in}{1.791058in}}{\pgfqpoint{2.560452in}{1.801657in}}{\pgfqpoint{2.560452in}{1.812707in}}%
\pgfpathcurveto{\pgfqpoint{2.560452in}{1.823757in}}{\pgfqpoint{2.556062in}{1.834356in}}{\pgfqpoint{2.548249in}{1.842170in}}%
\pgfpathcurveto{\pgfqpoint{2.540435in}{1.849983in}}{\pgfqpoint{2.529836in}{1.854374in}}{\pgfqpoint{2.518786in}{1.854374in}}%
\pgfpathcurveto{\pgfqpoint{2.507736in}{1.854374in}}{\pgfqpoint{2.497137in}{1.849983in}}{\pgfqpoint{2.489323in}{1.842170in}}%
\pgfpathcurveto{\pgfqpoint{2.481509in}{1.834356in}}{\pgfqpoint{2.477119in}{1.823757in}}{\pgfqpoint{2.477119in}{1.812707in}}%
\pgfpathcurveto{\pgfqpoint{2.477119in}{1.801657in}}{\pgfqpoint{2.481509in}{1.791058in}}{\pgfqpoint{2.489323in}{1.783244in}}%
\pgfpathcurveto{\pgfqpoint{2.497137in}{1.775431in}}{\pgfqpoint{2.507736in}{1.771040in}}{\pgfqpoint{2.518786in}{1.771040in}}%
\pgfpathclose%
\pgfusepath{stroke,fill}%
\end{pgfscope}%
\begin{pgfscope}%
\pgfpathrectangle{\pgfqpoint{0.800000in}{0.528000in}}{\pgfqpoint{4.960000in}{3.696000in}}%
\pgfusepath{clip}%
\pgfsetbuttcap%
\pgfsetroundjoin%
\definecolor{currentfill}{rgb}{0.000000,0.000000,0.000000}%
\pgfsetfillcolor{currentfill}%
\pgfsetlinewidth{1.003750pt}%
\definecolor{currentstroke}{rgb}{0.000000,0.000000,0.000000}%
\pgfsetstrokecolor{currentstroke}%
\pgfsetdash{}{0pt}%
\pgfpathmoveto{\pgfqpoint{2.518786in}{1.771040in}}%
\pgfpathcurveto{\pgfqpoint{2.529836in}{1.771040in}}{\pgfqpoint{2.540435in}{1.775431in}}{\pgfqpoint{2.548249in}{1.783244in}}%
\pgfpathcurveto{\pgfqpoint{2.556062in}{1.791058in}}{\pgfqpoint{2.560452in}{1.801657in}}{\pgfqpoint{2.560452in}{1.812707in}}%
\pgfpathcurveto{\pgfqpoint{2.560452in}{1.823757in}}{\pgfqpoint{2.556062in}{1.834356in}}{\pgfqpoint{2.548249in}{1.842170in}}%
\pgfpathcurveto{\pgfqpoint{2.540435in}{1.849983in}}{\pgfqpoint{2.529836in}{1.854374in}}{\pgfqpoint{2.518786in}{1.854374in}}%
\pgfpathcurveto{\pgfqpoint{2.507736in}{1.854374in}}{\pgfqpoint{2.497137in}{1.849983in}}{\pgfqpoint{2.489323in}{1.842170in}}%
\pgfpathcurveto{\pgfqpoint{2.481509in}{1.834356in}}{\pgfqpoint{2.477119in}{1.823757in}}{\pgfqpoint{2.477119in}{1.812707in}}%
\pgfpathcurveto{\pgfqpoint{2.477119in}{1.801657in}}{\pgfqpoint{2.481509in}{1.791058in}}{\pgfqpoint{2.489323in}{1.783244in}}%
\pgfpathcurveto{\pgfqpoint{2.497137in}{1.775431in}}{\pgfqpoint{2.507736in}{1.771040in}}{\pgfqpoint{2.518786in}{1.771040in}}%
\pgfpathclose%
\pgfusepath{stroke,fill}%
\end{pgfscope}%
\begin{pgfscope}%
\pgfpathrectangle{\pgfqpoint{0.800000in}{0.528000in}}{\pgfqpoint{4.960000in}{3.696000in}}%
\pgfusepath{clip}%
\pgfsetbuttcap%
\pgfsetroundjoin%
\definecolor{currentfill}{rgb}{0.000000,0.000000,0.000000}%
\pgfsetfillcolor{currentfill}%
\pgfsetlinewidth{1.003750pt}%
\definecolor{currentstroke}{rgb}{0.000000,0.000000,0.000000}%
\pgfsetstrokecolor{currentstroke}%
\pgfsetdash{}{0pt}%
\pgfpathmoveto{\pgfqpoint{2.518786in}{1.771040in}}%
\pgfpathcurveto{\pgfqpoint{2.529836in}{1.771040in}}{\pgfqpoint{2.540435in}{1.775431in}}{\pgfqpoint{2.548249in}{1.783244in}}%
\pgfpathcurveto{\pgfqpoint{2.556062in}{1.791058in}}{\pgfqpoint{2.560452in}{1.801657in}}{\pgfqpoint{2.560452in}{1.812707in}}%
\pgfpathcurveto{\pgfqpoint{2.560452in}{1.823757in}}{\pgfqpoint{2.556062in}{1.834356in}}{\pgfqpoint{2.548249in}{1.842170in}}%
\pgfpathcurveto{\pgfqpoint{2.540435in}{1.849983in}}{\pgfqpoint{2.529836in}{1.854374in}}{\pgfqpoint{2.518786in}{1.854374in}}%
\pgfpathcurveto{\pgfqpoint{2.507736in}{1.854374in}}{\pgfqpoint{2.497137in}{1.849983in}}{\pgfqpoint{2.489323in}{1.842170in}}%
\pgfpathcurveto{\pgfqpoint{2.481509in}{1.834356in}}{\pgfqpoint{2.477119in}{1.823757in}}{\pgfqpoint{2.477119in}{1.812707in}}%
\pgfpathcurveto{\pgfqpoint{2.477119in}{1.801657in}}{\pgfqpoint{2.481509in}{1.791058in}}{\pgfqpoint{2.489323in}{1.783244in}}%
\pgfpathcurveto{\pgfqpoint{2.497137in}{1.775431in}}{\pgfqpoint{2.507736in}{1.771040in}}{\pgfqpoint{2.518786in}{1.771040in}}%
\pgfpathclose%
\pgfusepath{stroke,fill}%
\end{pgfscope}%
\begin{pgfscope}%
\pgfpathrectangle{\pgfqpoint{0.800000in}{0.528000in}}{\pgfqpoint{4.960000in}{3.696000in}}%
\pgfusepath{clip}%
\pgfsetbuttcap%
\pgfsetroundjoin%
\definecolor{currentfill}{rgb}{0.000000,0.000000,0.000000}%
\pgfsetfillcolor{currentfill}%
\pgfsetlinewidth{1.003750pt}%
\definecolor{currentstroke}{rgb}{0.000000,0.000000,0.000000}%
\pgfsetstrokecolor{currentstroke}%
\pgfsetdash{}{0pt}%
\pgfpathmoveto{\pgfqpoint{2.518786in}{1.771040in}}%
\pgfpathcurveto{\pgfqpoint{2.529836in}{1.771040in}}{\pgfqpoint{2.540435in}{1.775431in}}{\pgfqpoint{2.548249in}{1.783244in}}%
\pgfpathcurveto{\pgfqpoint{2.556062in}{1.791058in}}{\pgfqpoint{2.560452in}{1.801657in}}{\pgfqpoint{2.560452in}{1.812707in}}%
\pgfpathcurveto{\pgfqpoint{2.560452in}{1.823757in}}{\pgfqpoint{2.556062in}{1.834356in}}{\pgfqpoint{2.548249in}{1.842170in}}%
\pgfpathcurveto{\pgfqpoint{2.540435in}{1.849983in}}{\pgfqpoint{2.529836in}{1.854374in}}{\pgfqpoint{2.518786in}{1.854374in}}%
\pgfpathcurveto{\pgfqpoint{2.507736in}{1.854374in}}{\pgfqpoint{2.497137in}{1.849983in}}{\pgfqpoint{2.489323in}{1.842170in}}%
\pgfpathcurveto{\pgfqpoint{2.481509in}{1.834356in}}{\pgfqpoint{2.477119in}{1.823757in}}{\pgfqpoint{2.477119in}{1.812707in}}%
\pgfpathcurveto{\pgfqpoint{2.477119in}{1.801657in}}{\pgfqpoint{2.481509in}{1.791058in}}{\pgfqpoint{2.489323in}{1.783244in}}%
\pgfpathcurveto{\pgfqpoint{2.497137in}{1.775431in}}{\pgfqpoint{2.507736in}{1.771040in}}{\pgfqpoint{2.518786in}{1.771040in}}%
\pgfpathclose%
\pgfusepath{stroke,fill}%
\end{pgfscope}%
\begin{pgfscope}%
\pgfpathrectangle{\pgfqpoint{0.800000in}{0.528000in}}{\pgfqpoint{4.960000in}{3.696000in}}%
\pgfusepath{clip}%
\pgfsetbuttcap%
\pgfsetroundjoin%
\definecolor{currentfill}{rgb}{0.000000,0.000000,0.000000}%
\pgfsetfillcolor{currentfill}%
\pgfsetlinewidth{1.003750pt}%
\definecolor{currentstroke}{rgb}{0.000000,0.000000,0.000000}%
\pgfsetstrokecolor{currentstroke}%
\pgfsetdash{}{0pt}%
\pgfpathmoveto{\pgfqpoint{2.518786in}{0.664394in}}%
\pgfpathcurveto{\pgfqpoint{2.529836in}{0.664394in}}{\pgfqpoint{2.540435in}{0.668784in}}{\pgfqpoint{2.548249in}{0.676598in}}%
\pgfpathcurveto{\pgfqpoint{2.556062in}{0.684411in}}{\pgfqpoint{2.560452in}{0.695010in}}{\pgfqpoint{2.560452in}{0.706060in}}%
\pgfpathcurveto{\pgfqpoint{2.560452in}{0.717111in}}{\pgfqpoint{2.556062in}{0.727710in}}{\pgfqpoint{2.548249in}{0.735523in}}%
\pgfpathcurveto{\pgfqpoint{2.540435in}{0.743337in}}{\pgfqpoint{2.529836in}{0.747727in}}{\pgfqpoint{2.518786in}{0.747727in}}%
\pgfpathcurveto{\pgfqpoint{2.507736in}{0.747727in}}{\pgfqpoint{2.497137in}{0.743337in}}{\pgfqpoint{2.489323in}{0.735523in}}%
\pgfpathcurveto{\pgfqpoint{2.481509in}{0.727710in}}{\pgfqpoint{2.477119in}{0.717111in}}{\pgfqpoint{2.477119in}{0.706060in}}%
\pgfpathcurveto{\pgfqpoint{2.477119in}{0.695010in}}{\pgfqpoint{2.481509in}{0.684411in}}{\pgfqpoint{2.489323in}{0.676598in}}%
\pgfpathcurveto{\pgfqpoint{2.497137in}{0.668784in}}{\pgfqpoint{2.507736in}{0.664394in}}{\pgfqpoint{2.518786in}{0.664394in}}%
\pgfpathclose%
\pgfusepath{stroke,fill}%
\end{pgfscope}%
\begin{pgfscope}%
\pgfpathrectangle{\pgfqpoint{0.800000in}{0.528000in}}{\pgfqpoint{4.960000in}{3.696000in}}%
\pgfusepath{clip}%
\pgfsetbuttcap%
\pgfsetroundjoin%
\definecolor{currentfill}{rgb}{0.000000,0.000000,0.000000}%
\pgfsetfillcolor{currentfill}%
\pgfsetlinewidth{1.003750pt}%
\definecolor{currentstroke}{rgb}{0.000000,0.000000,0.000000}%
\pgfsetstrokecolor{currentstroke}%
\pgfsetdash{}{0pt}%
\pgfpathmoveto{\pgfqpoint{2.518786in}{1.771040in}}%
\pgfpathcurveto{\pgfqpoint{2.529836in}{1.771040in}}{\pgfqpoint{2.540435in}{1.775431in}}{\pgfqpoint{2.548249in}{1.783244in}}%
\pgfpathcurveto{\pgfqpoint{2.556062in}{1.791058in}}{\pgfqpoint{2.560452in}{1.801657in}}{\pgfqpoint{2.560452in}{1.812707in}}%
\pgfpathcurveto{\pgfqpoint{2.560452in}{1.823757in}}{\pgfqpoint{2.556062in}{1.834356in}}{\pgfqpoint{2.548249in}{1.842170in}}%
\pgfpathcurveto{\pgfqpoint{2.540435in}{1.849983in}}{\pgfqpoint{2.529836in}{1.854374in}}{\pgfqpoint{2.518786in}{1.854374in}}%
\pgfpathcurveto{\pgfqpoint{2.507736in}{1.854374in}}{\pgfqpoint{2.497137in}{1.849983in}}{\pgfqpoint{2.489323in}{1.842170in}}%
\pgfpathcurveto{\pgfqpoint{2.481509in}{1.834356in}}{\pgfqpoint{2.477119in}{1.823757in}}{\pgfqpoint{2.477119in}{1.812707in}}%
\pgfpathcurveto{\pgfqpoint{2.477119in}{1.801657in}}{\pgfqpoint{2.481509in}{1.791058in}}{\pgfqpoint{2.489323in}{1.783244in}}%
\pgfpathcurveto{\pgfqpoint{2.497137in}{1.775431in}}{\pgfqpoint{2.507736in}{1.771040in}}{\pgfqpoint{2.518786in}{1.771040in}}%
\pgfpathclose%
\pgfusepath{stroke,fill}%
\end{pgfscope}%
\begin{pgfscope}%
\pgfpathrectangle{\pgfqpoint{0.800000in}{0.528000in}}{\pgfqpoint{4.960000in}{3.696000in}}%
\pgfusepath{clip}%
\pgfsetbuttcap%
\pgfsetroundjoin%
\definecolor{currentfill}{rgb}{0.000000,0.000000,0.000000}%
\pgfsetfillcolor{currentfill}%
\pgfsetlinewidth{1.003750pt}%
\definecolor{currentstroke}{rgb}{0.000000,0.000000,0.000000}%
\pgfsetstrokecolor{currentstroke}%
\pgfsetdash{}{0pt}%
\pgfpathmoveto{\pgfqpoint{2.518786in}{1.771040in}}%
\pgfpathcurveto{\pgfqpoint{2.529836in}{1.771040in}}{\pgfqpoint{2.540435in}{1.775431in}}{\pgfqpoint{2.548249in}{1.783244in}}%
\pgfpathcurveto{\pgfqpoint{2.556062in}{1.791058in}}{\pgfqpoint{2.560452in}{1.801657in}}{\pgfqpoint{2.560452in}{1.812707in}}%
\pgfpathcurveto{\pgfqpoint{2.560452in}{1.823757in}}{\pgfqpoint{2.556062in}{1.834356in}}{\pgfqpoint{2.548249in}{1.842170in}}%
\pgfpathcurveto{\pgfqpoint{2.540435in}{1.849983in}}{\pgfqpoint{2.529836in}{1.854374in}}{\pgfqpoint{2.518786in}{1.854374in}}%
\pgfpathcurveto{\pgfqpoint{2.507736in}{1.854374in}}{\pgfqpoint{2.497137in}{1.849983in}}{\pgfqpoint{2.489323in}{1.842170in}}%
\pgfpathcurveto{\pgfqpoint{2.481509in}{1.834356in}}{\pgfqpoint{2.477119in}{1.823757in}}{\pgfqpoint{2.477119in}{1.812707in}}%
\pgfpathcurveto{\pgfqpoint{2.477119in}{1.801657in}}{\pgfqpoint{2.481509in}{1.791058in}}{\pgfqpoint{2.489323in}{1.783244in}}%
\pgfpathcurveto{\pgfqpoint{2.497137in}{1.775431in}}{\pgfqpoint{2.507736in}{1.771040in}}{\pgfqpoint{2.518786in}{1.771040in}}%
\pgfpathclose%
\pgfusepath{stroke,fill}%
\end{pgfscope}%
\begin{pgfscope}%
\pgfpathrectangle{\pgfqpoint{0.800000in}{0.528000in}}{\pgfqpoint{4.960000in}{3.696000in}}%
\pgfusepath{clip}%
\pgfsetbuttcap%
\pgfsetroundjoin%
\definecolor{currentfill}{rgb}{0.000000,0.000000,0.000000}%
\pgfsetfillcolor{currentfill}%
\pgfsetlinewidth{1.003750pt}%
\definecolor{currentstroke}{rgb}{0.000000,0.000000,0.000000}%
\pgfsetstrokecolor{currentstroke}%
\pgfsetdash{}{0pt}%
\pgfpathmoveto{\pgfqpoint{2.518786in}{1.771040in}}%
\pgfpathcurveto{\pgfqpoint{2.529836in}{1.771040in}}{\pgfqpoint{2.540435in}{1.775431in}}{\pgfqpoint{2.548249in}{1.783244in}}%
\pgfpathcurveto{\pgfqpoint{2.556062in}{1.791058in}}{\pgfqpoint{2.560452in}{1.801657in}}{\pgfqpoint{2.560452in}{1.812707in}}%
\pgfpathcurveto{\pgfqpoint{2.560452in}{1.823757in}}{\pgfqpoint{2.556062in}{1.834356in}}{\pgfqpoint{2.548249in}{1.842170in}}%
\pgfpathcurveto{\pgfqpoint{2.540435in}{1.849983in}}{\pgfqpoint{2.529836in}{1.854374in}}{\pgfqpoint{2.518786in}{1.854374in}}%
\pgfpathcurveto{\pgfqpoint{2.507736in}{1.854374in}}{\pgfqpoint{2.497137in}{1.849983in}}{\pgfqpoint{2.489323in}{1.842170in}}%
\pgfpathcurveto{\pgfqpoint{2.481509in}{1.834356in}}{\pgfqpoint{2.477119in}{1.823757in}}{\pgfqpoint{2.477119in}{1.812707in}}%
\pgfpathcurveto{\pgfqpoint{2.477119in}{1.801657in}}{\pgfqpoint{2.481509in}{1.791058in}}{\pgfqpoint{2.489323in}{1.783244in}}%
\pgfpathcurveto{\pgfqpoint{2.497137in}{1.775431in}}{\pgfqpoint{2.507736in}{1.771040in}}{\pgfqpoint{2.518786in}{1.771040in}}%
\pgfpathclose%
\pgfusepath{stroke,fill}%
\end{pgfscope}%
\begin{pgfscope}%
\pgfpathrectangle{\pgfqpoint{0.800000in}{0.528000in}}{\pgfqpoint{4.960000in}{3.696000in}}%
\pgfusepath{clip}%
\pgfsetbuttcap%
\pgfsetroundjoin%
\definecolor{currentfill}{rgb}{0.000000,0.000000,0.000000}%
\pgfsetfillcolor{currentfill}%
\pgfsetlinewidth{1.003750pt}%
\definecolor{currentstroke}{rgb}{0.000000,0.000000,0.000000}%
\pgfsetstrokecolor{currentstroke}%
\pgfsetdash{}{0pt}%
\pgfpathmoveto{\pgfqpoint{2.518786in}{1.771040in}}%
\pgfpathcurveto{\pgfqpoint{2.529836in}{1.771040in}}{\pgfqpoint{2.540435in}{1.775431in}}{\pgfqpoint{2.548249in}{1.783244in}}%
\pgfpathcurveto{\pgfqpoint{2.556062in}{1.791058in}}{\pgfqpoint{2.560452in}{1.801657in}}{\pgfqpoint{2.560452in}{1.812707in}}%
\pgfpathcurveto{\pgfqpoint{2.560452in}{1.823757in}}{\pgfqpoint{2.556062in}{1.834356in}}{\pgfqpoint{2.548249in}{1.842170in}}%
\pgfpathcurveto{\pgfqpoint{2.540435in}{1.849983in}}{\pgfqpoint{2.529836in}{1.854374in}}{\pgfqpoint{2.518786in}{1.854374in}}%
\pgfpathcurveto{\pgfqpoint{2.507736in}{1.854374in}}{\pgfqpoint{2.497137in}{1.849983in}}{\pgfqpoint{2.489323in}{1.842170in}}%
\pgfpathcurveto{\pgfqpoint{2.481509in}{1.834356in}}{\pgfqpoint{2.477119in}{1.823757in}}{\pgfqpoint{2.477119in}{1.812707in}}%
\pgfpathcurveto{\pgfqpoint{2.477119in}{1.801657in}}{\pgfqpoint{2.481509in}{1.791058in}}{\pgfqpoint{2.489323in}{1.783244in}}%
\pgfpathcurveto{\pgfqpoint{2.497137in}{1.775431in}}{\pgfqpoint{2.507736in}{1.771040in}}{\pgfqpoint{2.518786in}{1.771040in}}%
\pgfpathclose%
\pgfusepath{stroke,fill}%
\end{pgfscope}%
\begin{pgfscope}%
\pgfpathrectangle{\pgfqpoint{0.800000in}{0.528000in}}{\pgfqpoint{4.960000in}{3.696000in}}%
\pgfusepath{clip}%
\pgfsetbuttcap%
\pgfsetroundjoin%
\definecolor{currentfill}{rgb}{0.000000,0.000000,0.000000}%
\pgfsetfillcolor{currentfill}%
\pgfsetlinewidth{1.003750pt}%
\definecolor{currentstroke}{rgb}{0.000000,0.000000,0.000000}%
\pgfsetstrokecolor{currentstroke}%
\pgfsetdash{}{0pt}%
\pgfpathmoveto{\pgfqpoint{2.518786in}{1.771040in}}%
\pgfpathcurveto{\pgfqpoint{2.529836in}{1.771040in}}{\pgfqpoint{2.540435in}{1.775431in}}{\pgfqpoint{2.548249in}{1.783244in}}%
\pgfpathcurveto{\pgfqpoint{2.556062in}{1.791058in}}{\pgfqpoint{2.560452in}{1.801657in}}{\pgfqpoint{2.560452in}{1.812707in}}%
\pgfpathcurveto{\pgfqpoint{2.560452in}{1.823757in}}{\pgfqpoint{2.556062in}{1.834356in}}{\pgfqpoint{2.548249in}{1.842170in}}%
\pgfpathcurveto{\pgfqpoint{2.540435in}{1.849983in}}{\pgfqpoint{2.529836in}{1.854374in}}{\pgfqpoint{2.518786in}{1.854374in}}%
\pgfpathcurveto{\pgfqpoint{2.507736in}{1.854374in}}{\pgfqpoint{2.497137in}{1.849983in}}{\pgfqpoint{2.489323in}{1.842170in}}%
\pgfpathcurveto{\pgfqpoint{2.481509in}{1.834356in}}{\pgfqpoint{2.477119in}{1.823757in}}{\pgfqpoint{2.477119in}{1.812707in}}%
\pgfpathcurveto{\pgfqpoint{2.477119in}{1.801657in}}{\pgfqpoint{2.481509in}{1.791058in}}{\pgfqpoint{2.489323in}{1.783244in}}%
\pgfpathcurveto{\pgfqpoint{2.497137in}{1.775431in}}{\pgfqpoint{2.507736in}{1.771040in}}{\pgfqpoint{2.518786in}{1.771040in}}%
\pgfpathclose%
\pgfusepath{stroke,fill}%
\end{pgfscope}%
\begin{pgfscope}%
\pgfpathrectangle{\pgfqpoint{0.800000in}{0.528000in}}{\pgfqpoint{4.960000in}{3.696000in}}%
\pgfusepath{clip}%
\pgfsetbuttcap%
\pgfsetroundjoin%
\definecolor{currentfill}{rgb}{0.000000,0.000000,0.000000}%
\pgfsetfillcolor{currentfill}%
\pgfsetlinewidth{1.003750pt}%
\definecolor{currentstroke}{rgb}{0.000000,0.000000,0.000000}%
\pgfsetstrokecolor{currentstroke}%
\pgfsetdash{}{0pt}%
\pgfpathmoveto{\pgfqpoint{2.518786in}{1.771040in}}%
\pgfpathcurveto{\pgfqpoint{2.529836in}{1.771040in}}{\pgfqpoint{2.540435in}{1.775431in}}{\pgfqpoint{2.548249in}{1.783244in}}%
\pgfpathcurveto{\pgfqpoint{2.556062in}{1.791058in}}{\pgfqpoint{2.560452in}{1.801657in}}{\pgfqpoint{2.560452in}{1.812707in}}%
\pgfpathcurveto{\pgfqpoint{2.560452in}{1.823757in}}{\pgfqpoint{2.556062in}{1.834356in}}{\pgfqpoint{2.548249in}{1.842170in}}%
\pgfpathcurveto{\pgfqpoint{2.540435in}{1.849983in}}{\pgfqpoint{2.529836in}{1.854374in}}{\pgfqpoint{2.518786in}{1.854374in}}%
\pgfpathcurveto{\pgfqpoint{2.507736in}{1.854374in}}{\pgfqpoint{2.497137in}{1.849983in}}{\pgfqpoint{2.489323in}{1.842170in}}%
\pgfpathcurveto{\pgfqpoint{2.481509in}{1.834356in}}{\pgfqpoint{2.477119in}{1.823757in}}{\pgfqpoint{2.477119in}{1.812707in}}%
\pgfpathcurveto{\pgfqpoint{2.477119in}{1.801657in}}{\pgfqpoint{2.481509in}{1.791058in}}{\pgfqpoint{2.489323in}{1.783244in}}%
\pgfpathcurveto{\pgfqpoint{2.497137in}{1.775431in}}{\pgfqpoint{2.507736in}{1.771040in}}{\pgfqpoint{2.518786in}{1.771040in}}%
\pgfpathclose%
\pgfusepath{stroke,fill}%
\end{pgfscope}%
\begin{pgfscope}%
\pgfpathrectangle{\pgfqpoint{0.800000in}{0.528000in}}{\pgfqpoint{4.960000in}{3.696000in}}%
\pgfusepath{clip}%
\pgfsetbuttcap%
\pgfsetroundjoin%
\definecolor{currentfill}{rgb}{0.000000,0.000000,0.000000}%
\pgfsetfillcolor{currentfill}%
\pgfsetlinewidth{1.003750pt}%
\definecolor{currentstroke}{rgb}{0.000000,0.000000,0.000000}%
\pgfsetstrokecolor{currentstroke}%
\pgfsetdash{}{0pt}%
\pgfpathmoveto{\pgfqpoint{2.518786in}{0.664394in}}%
\pgfpathcurveto{\pgfqpoint{2.529836in}{0.664394in}}{\pgfqpoint{2.540435in}{0.668784in}}{\pgfqpoint{2.548249in}{0.676598in}}%
\pgfpathcurveto{\pgfqpoint{2.556062in}{0.684411in}}{\pgfqpoint{2.560452in}{0.695010in}}{\pgfqpoint{2.560452in}{0.706060in}}%
\pgfpathcurveto{\pgfqpoint{2.560452in}{0.717111in}}{\pgfqpoint{2.556062in}{0.727710in}}{\pgfqpoint{2.548249in}{0.735523in}}%
\pgfpathcurveto{\pgfqpoint{2.540435in}{0.743337in}}{\pgfqpoint{2.529836in}{0.747727in}}{\pgfqpoint{2.518786in}{0.747727in}}%
\pgfpathcurveto{\pgfqpoint{2.507736in}{0.747727in}}{\pgfqpoint{2.497137in}{0.743337in}}{\pgfqpoint{2.489323in}{0.735523in}}%
\pgfpathcurveto{\pgfqpoint{2.481509in}{0.727710in}}{\pgfqpoint{2.477119in}{0.717111in}}{\pgfqpoint{2.477119in}{0.706060in}}%
\pgfpathcurveto{\pgfqpoint{2.477119in}{0.695010in}}{\pgfqpoint{2.481509in}{0.684411in}}{\pgfqpoint{2.489323in}{0.676598in}}%
\pgfpathcurveto{\pgfqpoint{2.497137in}{0.668784in}}{\pgfqpoint{2.507736in}{0.664394in}}{\pgfqpoint{2.518786in}{0.664394in}}%
\pgfpathclose%
\pgfusepath{stroke,fill}%
\end{pgfscope}%
\begin{pgfscope}%
\pgfpathrectangle{\pgfqpoint{0.800000in}{0.528000in}}{\pgfqpoint{4.960000in}{3.696000in}}%
\pgfusepath{clip}%
\pgfsetbuttcap%
\pgfsetroundjoin%
\definecolor{currentfill}{rgb}{0.000000,0.000000,0.000000}%
\pgfsetfillcolor{currentfill}%
\pgfsetlinewidth{1.003750pt}%
\definecolor{currentstroke}{rgb}{0.000000,0.000000,0.000000}%
\pgfsetstrokecolor{currentstroke}%
\pgfsetdash{}{0pt}%
\pgfpathmoveto{\pgfqpoint{2.518786in}{1.771040in}}%
\pgfpathcurveto{\pgfqpoint{2.529836in}{1.771040in}}{\pgfqpoint{2.540435in}{1.775431in}}{\pgfqpoint{2.548249in}{1.783244in}}%
\pgfpathcurveto{\pgfqpoint{2.556062in}{1.791058in}}{\pgfqpoint{2.560452in}{1.801657in}}{\pgfqpoint{2.560452in}{1.812707in}}%
\pgfpathcurveto{\pgfqpoint{2.560452in}{1.823757in}}{\pgfqpoint{2.556062in}{1.834356in}}{\pgfqpoint{2.548249in}{1.842170in}}%
\pgfpathcurveto{\pgfqpoint{2.540435in}{1.849983in}}{\pgfqpoint{2.529836in}{1.854374in}}{\pgfqpoint{2.518786in}{1.854374in}}%
\pgfpathcurveto{\pgfqpoint{2.507736in}{1.854374in}}{\pgfqpoint{2.497137in}{1.849983in}}{\pgfqpoint{2.489323in}{1.842170in}}%
\pgfpathcurveto{\pgfqpoint{2.481509in}{1.834356in}}{\pgfqpoint{2.477119in}{1.823757in}}{\pgfqpoint{2.477119in}{1.812707in}}%
\pgfpathcurveto{\pgfqpoint{2.477119in}{1.801657in}}{\pgfqpoint{2.481509in}{1.791058in}}{\pgfqpoint{2.489323in}{1.783244in}}%
\pgfpathcurveto{\pgfqpoint{2.497137in}{1.775431in}}{\pgfqpoint{2.507736in}{1.771040in}}{\pgfqpoint{2.518786in}{1.771040in}}%
\pgfpathclose%
\pgfusepath{stroke,fill}%
\end{pgfscope}%
\begin{pgfscope}%
\pgfpathrectangle{\pgfqpoint{0.800000in}{0.528000in}}{\pgfqpoint{4.960000in}{3.696000in}}%
\pgfusepath{clip}%
\pgfsetbuttcap%
\pgfsetroundjoin%
\definecolor{currentfill}{rgb}{0.000000,0.000000,0.000000}%
\pgfsetfillcolor{currentfill}%
\pgfsetlinewidth{1.003750pt}%
\definecolor{currentstroke}{rgb}{0.000000,0.000000,0.000000}%
\pgfsetstrokecolor{currentstroke}%
\pgfsetdash{}{0pt}%
\pgfpathmoveto{\pgfqpoint{2.518786in}{1.771040in}}%
\pgfpathcurveto{\pgfqpoint{2.529836in}{1.771040in}}{\pgfqpoint{2.540435in}{1.775431in}}{\pgfqpoint{2.548249in}{1.783244in}}%
\pgfpathcurveto{\pgfqpoint{2.556062in}{1.791058in}}{\pgfqpoint{2.560452in}{1.801657in}}{\pgfqpoint{2.560452in}{1.812707in}}%
\pgfpathcurveto{\pgfqpoint{2.560452in}{1.823757in}}{\pgfqpoint{2.556062in}{1.834356in}}{\pgfqpoint{2.548249in}{1.842170in}}%
\pgfpathcurveto{\pgfqpoint{2.540435in}{1.849983in}}{\pgfqpoint{2.529836in}{1.854374in}}{\pgfqpoint{2.518786in}{1.854374in}}%
\pgfpathcurveto{\pgfqpoint{2.507736in}{1.854374in}}{\pgfqpoint{2.497137in}{1.849983in}}{\pgfqpoint{2.489323in}{1.842170in}}%
\pgfpathcurveto{\pgfqpoint{2.481509in}{1.834356in}}{\pgfqpoint{2.477119in}{1.823757in}}{\pgfqpoint{2.477119in}{1.812707in}}%
\pgfpathcurveto{\pgfqpoint{2.477119in}{1.801657in}}{\pgfqpoint{2.481509in}{1.791058in}}{\pgfqpoint{2.489323in}{1.783244in}}%
\pgfpathcurveto{\pgfqpoint{2.497137in}{1.775431in}}{\pgfqpoint{2.507736in}{1.771040in}}{\pgfqpoint{2.518786in}{1.771040in}}%
\pgfpathclose%
\pgfusepath{stroke,fill}%
\end{pgfscope}%
\begin{pgfscope}%
\pgfpathrectangle{\pgfqpoint{0.800000in}{0.528000in}}{\pgfqpoint{4.960000in}{3.696000in}}%
\pgfusepath{clip}%
\pgfsetbuttcap%
\pgfsetroundjoin%
\definecolor{currentfill}{rgb}{0.000000,0.000000,0.000000}%
\pgfsetfillcolor{currentfill}%
\pgfsetlinewidth{1.003750pt}%
\definecolor{currentstroke}{rgb}{0.000000,0.000000,0.000000}%
\pgfsetstrokecolor{currentstroke}%
\pgfsetdash{}{0pt}%
\pgfpathmoveto{\pgfqpoint{2.518786in}{0.664394in}}%
\pgfpathcurveto{\pgfqpoint{2.529836in}{0.664394in}}{\pgfqpoint{2.540435in}{0.668784in}}{\pgfqpoint{2.548249in}{0.676598in}}%
\pgfpathcurveto{\pgfqpoint{2.556062in}{0.684411in}}{\pgfqpoint{2.560452in}{0.695010in}}{\pgfqpoint{2.560452in}{0.706060in}}%
\pgfpathcurveto{\pgfqpoint{2.560452in}{0.717111in}}{\pgfqpoint{2.556062in}{0.727710in}}{\pgfqpoint{2.548249in}{0.735523in}}%
\pgfpathcurveto{\pgfqpoint{2.540435in}{0.743337in}}{\pgfqpoint{2.529836in}{0.747727in}}{\pgfqpoint{2.518786in}{0.747727in}}%
\pgfpathcurveto{\pgfqpoint{2.507736in}{0.747727in}}{\pgfqpoint{2.497137in}{0.743337in}}{\pgfqpoint{2.489323in}{0.735523in}}%
\pgfpathcurveto{\pgfqpoint{2.481509in}{0.727710in}}{\pgfqpoint{2.477119in}{0.717111in}}{\pgfqpoint{2.477119in}{0.706060in}}%
\pgfpathcurveto{\pgfqpoint{2.477119in}{0.695010in}}{\pgfqpoint{2.481509in}{0.684411in}}{\pgfqpoint{2.489323in}{0.676598in}}%
\pgfpathcurveto{\pgfqpoint{2.497137in}{0.668784in}}{\pgfqpoint{2.507736in}{0.664394in}}{\pgfqpoint{2.518786in}{0.664394in}}%
\pgfpathclose%
\pgfusepath{stroke,fill}%
\end{pgfscope}%
\begin{pgfscope}%
\pgfpathrectangle{\pgfqpoint{0.800000in}{0.528000in}}{\pgfqpoint{4.960000in}{3.696000in}}%
\pgfusepath{clip}%
\pgfsetbuttcap%
\pgfsetroundjoin%
\definecolor{currentfill}{rgb}{0.000000,0.000000,0.000000}%
\pgfsetfillcolor{currentfill}%
\pgfsetlinewidth{1.003750pt}%
\definecolor{currentstroke}{rgb}{0.000000,0.000000,0.000000}%
\pgfsetstrokecolor{currentstroke}%
\pgfsetdash{}{0pt}%
\pgfpathmoveto{\pgfqpoint{2.518786in}{1.771040in}}%
\pgfpathcurveto{\pgfqpoint{2.529836in}{1.771040in}}{\pgfqpoint{2.540435in}{1.775431in}}{\pgfqpoint{2.548249in}{1.783244in}}%
\pgfpathcurveto{\pgfqpoint{2.556062in}{1.791058in}}{\pgfqpoint{2.560452in}{1.801657in}}{\pgfqpoint{2.560452in}{1.812707in}}%
\pgfpathcurveto{\pgfqpoint{2.560452in}{1.823757in}}{\pgfqpoint{2.556062in}{1.834356in}}{\pgfqpoint{2.548249in}{1.842170in}}%
\pgfpathcurveto{\pgfqpoint{2.540435in}{1.849983in}}{\pgfqpoint{2.529836in}{1.854374in}}{\pgfqpoint{2.518786in}{1.854374in}}%
\pgfpathcurveto{\pgfqpoint{2.507736in}{1.854374in}}{\pgfqpoint{2.497137in}{1.849983in}}{\pgfqpoint{2.489323in}{1.842170in}}%
\pgfpathcurveto{\pgfqpoint{2.481509in}{1.834356in}}{\pgfqpoint{2.477119in}{1.823757in}}{\pgfqpoint{2.477119in}{1.812707in}}%
\pgfpathcurveto{\pgfqpoint{2.477119in}{1.801657in}}{\pgfqpoint{2.481509in}{1.791058in}}{\pgfqpoint{2.489323in}{1.783244in}}%
\pgfpathcurveto{\pgfqpoint{2.497137in}{1.775431in}}{\pgfqpoint{2.507736in}{1.771040in}}{\pgfqpoint{2.518786in}{1.771040in}}%
\pgfpathclose%
\pgfusepath{stroke,fill}%
\end{pgfscope}%
\begin{pgfscope}%
\pgfpathrectangle{\pgfqpoint{0.800000in}{0.528000in}}{\pgfqpoint{4.960000in}{3.696000in}}%
\pgfusepath{clip}%
\pgfsetbuttcap%
\pgfsetroundjoin%
\definecolor{currentfill}{rgb}{0.000000,0.000000,0.000000}%
\pgfsetfillcolor{currentfill}%
\pgfsetlinewidth{1.003750pt}%
\definecolor{currentstroke}{rgb}{0.000000,0.000000,0.000000}%
\pgfsetstrokecolor{currentstroke}%
\pgfsetdash{}{0pt}%
\pgfpathmoveto{\pgfqpoint{2.518786in}{1.771040in}}%
\pgfpathcurveto{\pgfqpoint{2.529836in}{1.771040in}}{\pgfqpoint{2.540435in}{1.775431in}}{\pgfqpoint{2.548249in}{1.783244in}}%
\pgfpathcurveto{\pgfqpoint{2.556062in}{1.791058in}}{\pgfqpoint{2.560452in}{1.801657in}}{\pgfqpoint{2.560452in}{1.812707in}}%
\pgfpathcurveto{\pgfqpoint{2.560452in}{1.823757in}}{\pgfqpoint{2.556062in}{1.834356in}}{\pgfqpoint{2.548249in}{1.842170in}}%
\pgfpathcurveto{\pgfqpoint{2.540435in}{1.849983in}}{\pgfqpoint{2.529836in}{1.854374in}}{\pgfqpoint{2.518786in}{1.854374in}}%
\pgfpathcurveto{\pgfqpoint{2.507736in}{1.854374in}}{\pgfqpoint{2.497137in}{1.849983in}}{\pgfqpoint{2.489323in}{1.842170in}}%
\pgfpathcurveto{\pgfqpoint{2.481509in}{1.834356in}}{\pgfqpoint{2.477119in}{1.823757in}}{\pgfqpoint{2.477119in}{1.812707in}}%
\pgfpathcurveto{\pgfqpoint{2.477119in}{1.801657in}}{\pgfqpoint{2.481509in}{1.791058in}}{\pgfqpoint{2.489323in}{1.783244in}}%
\pgfpathcurveto{\pgfqpoint{2.497137in}{1.775431in}}{\pgfqpoint{2.507736in}{1.771040in}}{\pgfqpoint{2.518786in}{1.771040in}}%
\pgfpathclose%
\pgfusepath{stroke,fill}%
\end{pgfscope}%
\begin{pgfscope}%
\pgfpathrectangle{\pgfqpoint{0.800000in}{0.528000in}}{\pgfqpoint{4.960000in}{3.696000in}}%
\pgfusepath{clip}%
\pgfsetbuttcap%
\pgfsetroundjoin%
\definecolor{currentfill}{rgb}{0.000000,0.000000,0.000000}%
\pgfsetfillcolor{currentfill}%
\pgfsetlinewidth{1.003750pt}%
\definecolor{currentstroke}{rgb}{0.000000,0.000000,0.000000}%
\pgfsetstrokecolor{currentstroke}%
\pgfsetdash{}{0pt}%
\pgfpathmoveto{\pgfqpoint{2.518786in}{1.771040in}}%
\pgfpathcurveto{\pgfqpoint{2.529836in}{1.771040in}}{\pgfqpoint{2.540435in}{1.775431in}}{\pgfqpoint{2.548249in}{1.783244in}}%
\pgfpathcurveto{\pgfqpoint{2.556062in}{1.791058in}}{\pgfqpoint{2.560452in}{1.801657in}}{\pgfqpoint{2.560452in}{1.812707in}}%
\pgfpathcurveto{\pgfqpoint{2.560452in}{1.823757in}}{\pgfqpoint{2.556062in}{1.834356in}}{\pgfqpoint{2.548249in}{1.842170in}}%
\pgfpathcurveto{\pgfqpoint{2.540435in}{1.849983in}}{\pgfqpoint{2.529836in}{1.854374in}}{\pgfqpoint{2.518786in}{1.854374in}}%
\pgfpathcurveto{\pgfqpoint{2.507736in}{1.854374in}}{\pgfqpoint{2.497137in}{1.849983in}}{\pgfqpoint{2.489323in}{1.842170in}}%
\pgfpathcurveto{\pgfqpoint{2.481509in}{1.834356in}}{\pgfqpoint{2.477119in}{1.823757in}}{\pgfqpoint{2.477119in}{1.812707in}}%
\pgfpathcurveto{\pgfqpoint{2.477119in}{1.801657in}}{\pgfqpoint{2.481509in}{1.791058in}}{\pgfqpoint{2.489323in}{1.783244in}}%
\pgfpathcurveto{\pgfqpoint{2.497137in}{1.775431in}}{\pgfqpoint{2.507736in}{1.771040in}}{\pgfqpoint{2.518786in}{1.771040in}}%
\pgfpathclose%
\pgfusepath{stroke,fill}%
\end{pgfscope}%
\begin{pgfscope}%
\pgfpathrectangle{\pgfqpoint{0.800000in}{0.528000in}}{\pgfqpoint{4.960000in}{3.696000in}}%
\pgfusepath{clip}%
\pgfsetbuttcap%
\pgfsetroundjoin%
\definecolor{currentfill}{rgb}{0.000000,0.000000,0.000000}%
\pgfsetfillcolor{currentfill}%
\pgfsetlinewidth{1.003750pt}%
\definecolor{currentstroke}{rgb}{0.000000,0.000000,0.000000}%
\pgfsetstrokecolor{currentstroke}%
\pgfsetdash{}{0pt}%
\pgfpathmoveto{\pgfqpoint{2.518786in}{1.771040in}}%
\pgfpathcurveto{\pgfqpoint{2.529836in}{1.771040in}}{\pgfqpoint{2.540435in}{1.775431in}}{\pgfqpoint{2.548249in}{1.783244in}}%
\pgfpathcurveto{\pgfqpoint{2.556062in}{1.791058in}}{\pgfqpoint{2.560452in}{1.801657in}}{\pgfqpoint{2.560452in}{1.812707in}}%
\pgfpathcurveto{\pgfqpoint{2.560452in}{1.823757in}}{\pgfqpoint{2.556062in}{1.834356in}}{\pgfqpoint{2.548249in}{1.842170in}}%
\pgfpathcurveto{\pgfqpoint{2.540435in}{1.849983in}}{\pgfqpoint{2.529836in}{1.854374in}}{\pgfqpoint{2.518786in}{1.854374in}}%
\pgfpathcurveto{\pgfqpoint{2.507736in}{1.854374in}}{\pgfqpoint{2.497137in}{1.849983in}}{\pgfqpoint{2.489323in}{1.842170in}}%
\pgfpathcurveto{\pgfqpoint{2.481509in}{1.834356in}}{\pgfqpoint{2.477119in}{1.823757in}}{\pgfqpoint{2.477119in}{1.812707in}}%
\pgfpathcurveto{\pgfqpoint{2.477119in}{1.801657in}}{\pgfqpoint{2.481509in}{1.791058in}}{\pgfqpoint{2.489323in}{1.783244in}}%
\pgfpathcurveto{\pgfqpoint{2.497137in}{1.775431in}}{\pgfqpoint{2.507736in}{1.771040in}}{\pgfqpoint{2.518786in}{1.771040in}}%
\pgfpathclose%
\pgfusepath{stroke,fill}%
\end{pgfscope}%
\begin{pgfscope}%
\pgfpathrectangle{\pgfqpoint{0.800000in}{0.528000in}}{\pgfqpoint{4.960000in}{3.696000in}}%
\pgfusepath{clip}%
\pgfsetbuttcap%
\pgfsetroundjoin%
\definecolor{currentfill}{rgb}{0.000000,0.000000,0.000000}%
\pgfsetfillcolor{currentfill}%
\pgfsetlinewidth{1.003750pt}%
\definecolor{currentstroke}{rgb}{0.000000,0.000000,0.000000}%
\pgfsetstrokecolor{currentstroke}%
\pgfsetdash{}{0pt}%
\pgfpathmoveto{\pgfqpoint{2.518786in}{1.771040in}}%
\pgfpathcurveto{\pgfqpoint{2.529836in}{1.771040in}}{\pgfqpoint{2.540435in}{1.775431in}}{\pgfqpoint{2.548249in}{1.783244in}}%
\pgfpathcurveto{\pgfqpoint{2.556062in}{1.791058in}}{\pgfqpoint{2.560452in}{1.801657in}}{\pgfqpoint{2.560452in}{1.812707in}}%
\pgfpathcurveto{\pgfqpoint{2.560452in}{1.823757in}}{\pgfqpoint{2.556062in}{1.834356in}}{\pgfqpoint{2.548249in}{1.842170in}}%
\pgfpathcurveto{\pgfqpoint{2.540435in}{1.849983in}}{\pgfqpoint{2.529836in}{1.854374in}}{\pgfqpoint{2.518786in}{1.854374in}}%
\pgfpathcurveto{\pgfqpoint{2.507736in}{1.854374in}}{\pgfqpoint{2.497137in}{1.849983in}}{\pgfqpoint{2.489323in}{1.842170in}}%
\pgfpathcurveto{\pgfqpoint{2.481509in}{1.834356in}}{\pgfqpoint{2.477119in}{1.823757in}}{\pgfqpoint{2.477119in}{1.812707in}}%
\pgfpathcurveto{\pgfqpoint{2.477119in}{1.801657in}}{\pgfqpoint{2.481509in}{1.791058in}}{\pgfqpoint{2.489323in}{1.783244in}}%
\pgfpathcurveto{\pgfqpoint{2.497137in}{1.775431in}}{\pgfqpoint{2.507736in}{1.771040in}}{\pgfqpoint{2.518786in}{1.771040in}}%
\pgfpathclose%
\pgfusepath{stroke,fill}%
\end{pgfscope}%
\begin{pgfscope}%
\pgfpathrectangle{\pgfqpoint{0.800000in}{0.528000in}}{\pgfqpoint{4.960000in}{3.696000in}}%
\pgfusepath{clip}%
\pgfsetbuttcap%
\pgfsetroundjoin%
\definecolor{currentfill}{rgb}{0.000000,0.000000,0.000000}%
\pgfsetfillcolor{currentfill}%
\pgfsetlinewidth{1.003750pt}%
\definecolor{currentstroke}{rgb}{0.000000,0.000000,0.000000}%
\pgfsetstrokecolor{currentstroke}%
\pgfsetdash{}{0pt}%
\pgfpathmoveto{\pgfqpoint{2.518786in}{1.771040in}}%
\pgfpathcurveto{\pgfqpoint{2.529836in}{1.771040in}}{\pgfqpoint{2.540435in}{1.775431in}}{\pgfqpoint{2.548249in}{1.783244in}}%
\pgfpathcurveto{\pgfqpoint{2.556062in}{1.791058in}}{\pgfqpoint{2.560452in}{1.801657in}}{\pgfqpoint{2.560452in}{1.812707in}}%
\pgfpathcurveto{\pgfqpoint{2.560452in}{1.823757in}}{\pgfqpoint{2.556062in}{1.834356in}}{\pgfqpoint{2.548249in}{1.842170in}}%
\pgfpathcurveto{\pgfqpoint{2.540435in}{1.849983in}}{\pgfqpoint{2.529836in}{1.854374in}}{\pgfqpoint{2.518786in}{1.854374in}}%
\pgfpathcurveto{\pgfqpoint{2.507736in}{1.854374in}}{\pgfqpoint{2.497137in}{1.849983in}}{\pgfqpoint{2.489323in}{1.842170in}}%
\pgfpathcurveto{\pgfqpoint{2.481509in}{1.834356in}}{\pgfqpoint{2.477119in}{1.823757in}}{\pgfqpoint{2.477119in}{1.812707in}}%
\pgfpathcurveto{\pgfqpoint{2.477119in}{1.801657in}}{\pgfqpoint{2.481509in}{1.791058in}}{\pgfqpoint{2.489323in}{1.783244in}}%
\pgfpathcurveto{\pgfqpoint{2.497137in}{1.775431in}}{\pgfqpoint{2.507736in}{1.771040in}}{\pgfqpoint{2.518786in}{1.771040in}}%
\pgfpathclose%
\pgfusepath{stroke,fill}%
\end{pgfscope}%
\begin{pgfscope}%
\pgfpathrectangle{\pgfqpoint{0.800000in}{0.528000in}}{\pgfqpoint{4.960000in}{3.696000in}}%
\pgfusepath{clip}%
\pgfsetbuttcap%
\pgfsetroundjoin%
\definecolor{currentfill}{rgb}{0.000000,0.000000,0.000000}%
\pgfsetfillcolor{currentfill}%
\pgfsetlinewidth{1.003750pt}%
\definecolor{currentstroke}{rgb}{0.000000,0.000000,0.000000}%
\pgfsetstrokecolor{currentstroke}%
\pgfsetdash{}{0pt}%
\pgfpathmoveto{\pgfqpoint{2.518786in}{1.771040in}}%
\pgfpathcurveto{\pgfqpoint{2.529836in}{1.771040in}}{\pgfqpoint{2.540435in}{1.775431in}}{\pgfqpoint{2.548249in}{1.783244in}}%
\pgfpathcurveto{\pgfqpoint{2.556062in}{1.791058in}}{\pgfqpoint{2.560452in}{1.801657in}}{\pgfqpoint{2.560452in}{1.812707in}}%
\pgfpathcurveto{\pgfqpoint{2.560452in}{1.823757in}}{\pgfqpoint{2.556062in}{1.834356in}}{\pgfqpoint{2.548249in}{1.842170in}}%
\pgfpathcurveto{\pgfqpoint{2.540435in}{1.849983in}}{\pgfqpoint{2.529836in}{1.854374in}}{\pgfqpoint{2.518786in}{1.854374in}}%
\pgfpathcurveto{\pgfqpoint{2.507736in}{1.854374in}}{\pgfqpoint{2.497137in}{1.849983in}}{\pgfqpoint{2.489323in}{1.842170in}}%
\pgfpathcurveto{\pgfqpoint{2.481509in}{1.834356in}}{\pgfqpoint{2.477119in}{1.823757in}}{\pgfqpoint{2.477119in}{1.812707in}}%
\pgfpathcurveto{\pgfqpoint{2.477119in}{1.801657in}}{\pgfqpoint{2.481509in}{1.791058in}}{\pgfqpoint{2.489323in}{1.783244in}}%
\pgfpathcurveto{\pgfqpoint{2.497137in}{1.775431in}}{\pgfqpoint{2.507736in}{1.771040in}}{\pgfqpoint{2.518786in}{1.771040in}}%
\pgfpathclose%
\pgfusepath{stroke,fill}%
\end{pgfscope}%
\begin{pgfscope}%
\pgfpathrectangle{\pgfqpoint{0.800000in}{0.528000in}}{\pgfqpoint{4.960000in}{3.696000in}}%
\pgfusepath{clip}%
\pgfsetbuttcap%
\pgfsetroundjoin%
\definecolor{currentfill}{rgb}{0.000000,0.000000,0.000000}%
\pgfsetfillcolor{currentfill}%
\pgfsetlinewidth{1.003750pt}%
\definecolor{currentstroke}{rgb}{0.000000,0.000000,0.000000}%
\pgfsetstrokecolor{currentstroke}%
\pgfsetdash{}{0pt}%
\pgfpathmoveto{\pgfqpoint{2.518786in}{1.771040in}}%
\pgfpathcurveto{\pgfqpoint{2.529836in}{1.771040in}}{\pgfqpoint{2.540435in}{1.775431in}}{\pgfqpoint{2.548249in}{1.783244in}}%
\pgfpathcurveto{\pgfqpoint{2.556062in}{1.791058in}}{\pgfqpoint{2.560452in}{1.801657in}}{\pgfqpoint{2.560452in}{1.812707in}}%
\pgfpathcurveto{\pgfqpoint{2.560452in}{1.823757in}}{\pgfqpoint{2.556062in}{1.834356in}}{\pgfqpoint{2.548249in}{1.842170in}}%
\pgfpathcurveto{\pgfqpoint{2.540435in}{1.849983in}}{\pgfqpoint{2.529836in}{1.854374in}}{\pgfqpoint{2.518786in}{1.854374in}}%
\pgfpathcurveto{\pgfqpoint{2.507736in}{1.854374in}}{\pgfqpoint{2.497137in}{1.849983in}}{\pgfqpoint{2.489323in}{1.842170in}}%
\pgfpathcurveto{\pgfqpoint{2.481509in}{1.834356in}}{\pgfqpoint{2.477119in}{1.823757in}}{\pgfqpoint{2.477119in}{1.812707in}}%
\pgfpathcurveto{\pgfqpoint{2.477119in}{1.801657in}}{\pgfqpoint{2.481509in}{1.791058in}}{\pgfqpoint{2.489323in}{1.783244in}}%
\pgfpathcurveto{\pgfqpoint{2.497137in}{1.775431in}}{\pgfqpoint{2.507736in}{1.771040in}}{\pgfqpoint{2.518786in}{1.771040in}}%
\pgfpathclose%
\pgfusepath{stroke,fill}%
\end{pgfscope}%
\begin{pgfscope}%
\pgfpathrectangle{\pgfqpoint{0.800000in}{0.528000in}}{\pgfqpoint{4.960000in}{3.696000in}}%
\pgfusepath{clip}%
\pgfsetbuttcap%
\pgfsetroundjoin%
\definecolor{currentfill}{rgb}{0.000000,0.000000,0.000000}%
\pgfsetfillcolor{currentfill}%
\pgfsetlinewidth{1.003750pt}%
\definecolor{currentstroke}{rgb}{0.000000,0.000000,0.000000}%
\pgfsetstrokecolor{currentstroke}%
\pgfsetdash{}{0pt}%
\pgfpathmoveto{\pgfqpoint{2.518786in}{1.771040in}}%
\pgfpathcurveto{\pgfqpoint{2.529836in}{1.771040in}}{\pgfqpoint{2.540435in}{1.775431in}}{\pgfqpoint{2.548249in}{1.783244in}}%
\pgfpathcurveto{\pgfqpoint{2.556062in}{1.791058in}}{\pgfqpoint{2.560452in}{1.801657in}}{\pgfqpoint{2.560452in}{1.812707in}}%
\pgfpathcurveto{\pgfqpoint{2.560452in}{1.823757in}}{\pgfqpoint{2.556062in}{1.834356in}}{\pgfqpoint{2.548249in}{1.842170in}}%
\pgfpathcurveto{\pgfqpoint{2.540435in}{1.849983in}}{\pgfqpoint{2.529836in}{1.854374in}}{\pgfqpoint{2.518786in}{1.854374in}}%
\pgfpathcurveto{\pgfqpoint{2.507736in}{1.854374in}}{\pgfqpoint{2.497137in}{1.849983in}}{\pgfqpoint{2.489323in}{1.842170in}}%
\pgfpathcurveto{\pgfqpoint{2.481509in}{1.834356in}}{\pgfqpoint{2.477119in}{1.823757in}}{\pgfqpoint{2.477119in}{1.812707in}}%
\pgfpathcurveto{\pgfqpoint{2.477119in}{1.801657in}}{\pgfqpoint{2.481509in}{1.791058in}}{\pgfqpoint{2.489323in}{1.783244in}}%
\pgfpathcurveto{\pgfqpoint{2.497137in}{1.775431in}}{\pgfqpoint{2.507736in}{1.771040in}}{\pgfqpoint{2.518786in}{1.771040in}}%
\pgfpathclose%
\pgfusepath{stroke,fill}%
\end{pgfscope}%
\begin{pgfscope}%
\pgfpathrectangle{\pgfqpoint{0.800000in}{0.528000in}}{\pgfqpoint{4.960000in}{3.696000in}}%
\pgfusepath{clip}%
\pgfsetbuttcap%
\pgfsetroundjoin%
\definecolor{currentfill}{rgb}{0.000000,0.000000,0.000000}%
\pgfsetfillcolor{currentfill}%
\pgfsetlinewidth{1.003750pt}%
\definecolor{currentstroke}{rgb}{0.000000,0.000000,0.000000}%
\pgfsetstrokecolor{currentstroke}%
\pgfsetdash{}{0pt}%
\pgfpathmoveto{\pgfqpoint{2.518786in}{1.771040in}}%
\pgfpathcurveto{\pgfqpoint{2.529836in}{1.771040in}}{\pgfqpoint{2.540435in}{1.775431in}}{\pgfqpoint{2.548249in}{1.783244in}}%
\pgfpathcurveto{\pgfqpoint{2.556062in}{1.791058in}}{\pgfqpoint{2.560452in}{1.801657in}}{\pgfqpoint{2.560452in}{1.812707in}}%
\pgfpathcurveto{\pgfqpoint{2.560452in}{1.823757in}}{\pgfqpoint{2.556062in}{1.834356in}}{\pgfqpoint{2.548249in}{1.842170in}}%
\pgfpathcurveto{\pgfqpoint{2.540435in}{1.849983in}}{\pgfqpoint{2.529836in}{1.854374in}}{\pgfqpoint{2.518786in}{1.854374in}}%
\pgfpathcurveto{\pgfqpoint{2.507736in}{1.854374in}}{\pgfqpoint{2.497137in}{1.849983in}}{\pgfqpoint{2.489323in}{1.842170in}}%
\pgfpathcurveto{\pgfqpoint{2.481509in}{1.834356in}}{\pgfqpoint{2.477119in}{1.823757in}}{\pgfqpoint{2.477119in}{1.812707in}}%
\pgfpathcurveto{\pgfqpoint{2.477119in}{1.801657in}}{\pgfqpoint{2.481509in}{1.791058in}}{\pgfqpoint{2.489323in}{1.783244in}}%
\pgfpathcurveto{\pgfqpoint{2.497137in}{1.775431in}}{\pgfqpoint{2.507736in}{1.771040in}}{\pgfqpoint{2.518786in}{1.771040in}}%
\pgfpathclose%
\pgfusepath{stroke,fill}%
\end{pgfscope}%
\begin{pgfscope}%
\pgfpathrectangle{\pgfqpoint{0.800000in}{0.528000in}}{\pgfqpoint{4.960000in}{3.696000in}}%
\pgfusepath{clip}%
\pgfsetbuttcap%
\pgfsetroundjoin%
\definecolor{currentfill}{rgb}{0.000000,0.000000,0.000000}%
\pgfsetfillcolor{currentfill}%
\pgfsetlinewidth{1.003750pt}%
\definecolor{currentstroke}{rgb}{0.000000,0.000000,0.000000}%
\pgfsetstrokecolor{currentstroke}%
\pgfsetdash{}{0pt}%
\pgfpathmoveto{\pgfqpoint{2.518786in}{1.771040in}}%
\pgfpathcurveto{\pgfqpoint{2.529836in}{1.771040in}}{\pgfqpoint{2.540435in}{1.775431in}}{\pgfqpoint{2.548249in}{1.783244in}}%
\pgfpathcurveto{\pgfqpoint{2.556062in}{1.791058in}}{\pgfqpoint{2.560452in}{1.801657in}}{\pgfqpoint{2.560452in}{1.812707in}}%
\pgfpathcurveto{\pgfqpoint{2.560452in}{1.823757in}}{\pgfqpoint{2.556062in}{1.834356in}}{\pgfqpoint{2.548249in}{1.842170in}}%
\pgfpathcurveto{\pgfqpoint{2.540435in}{1.849983in}}{\pgfqpoint{2.529836in}{1.854374in}}{\pgfqpoint{2.518786in}{1.854374in}}%
\pgfpathcurveto{\pgfqpoint{2.507736in}{1.854374in}}{\pgfqpoint{2.497137in}{1.849983in}}{\pgfqpoint{2.489323in}{1.842170in}}%
\pgfpathcurveto{\pgfqpoint{2.481509in}{1.834356in}}{\pgfqpoint{2.477119in}{1.823757in}}{\pgfqpoint{2.477119in}{1.812707in}}%
\pgfpathcurveto{\pgfqpoint{2.477119in}{1.801657in}}{\pgfqpoint{2.481509in}{1.791058in}}{\pgfqpoint{2.489323in}{1.783244in}}%
\pgfpathcurveto{\pgfqpoint{2.497137in}{1.775431in}}{\pgfqpoint{2.507736in}{1.771040in}}{\pgfqpoint{2.518786in}{1.771040in}}%
\pgfpathclose%
\pgfusepath{stroke,fill}%
\end{pgfscope}%
\begin{pgfscope}%
\pgfpathrectangle{\pgfqpoint{0.800000in}{0.528000in}}{\pgfqpoint{4.960000in}{3.696000in}}%
\pgfusepath{clip}%
\pgfsetbuttcap%
\pgfsetroundjoin%
\definecolor{currentfill}{rgb}{0.000000,0.000000,0.000000}%
\pgfsetfillcolor{currentfill}%
\pgfsetlinewidth{1.003750pt}%
\definecolor{currentstroke}{rgb}{0.000000,0.000000,0.000000}%
\pgfsetstrokecolor{currentstroke}%
\pgfsetdash{}{0pt}%
\pgfpathmoveto{\pgfqpoint{2.518786in}{1.771040in}}%
\pgfpathcurveto{\pgfqpoint{2.529836in}{1.771040in}}{\pgfqpoint{2.540435in}{1.775431in}}{\pgfqpoint{2.548249in}{1.783244in}}%
\pgfpathcurveto{\pgfqpoint{2.556062in}{1.791058in}}{\pgfqpoint{2.560452in}{1.801657in}}{\pgfqpoint{2.560452in}{1.812707in}}%
\pgfpathcurveto{\pgfqpoint{2.560452in}{1.823757in}}{\pgfqpoint{2.556062in}{1.834356in}}{\pgfqpoint{2.548249in}{1.842170in}}%
\pgfpathcurveto{\pgfqpoint{2.540435in}{1.849983in}}{\pgfqpoint{2.529836in}{1.854374in}}{\pgfqpoint{2.518786in}{1.854374in}}%
\pgfpathcurveto{\pgfqpoint{2.507736in}{1.854374in}}{\pgfqpoint{2.497137in}{1.849983in}}{\pgfqpoint{2.489323in}{1.842170in}}%
\pgfpathcurveto{\pgfqpoint{2.481509in}{1.834356in}}{\pgfqpoint{2.477119in}{1.823757in}}{\pgfqpoint{2.477119in}{1.812707in}}%
\pgfpathcurveto{\pgfqpoint{2.477119in}{1.801657in}}{\pgfqpoint{2.481509in}{1.791058in}}{\pgfqpoint{2.489323in}{1.783244in}}%
\pgfpathcurveto{\pgfqpoint{2.497137in}{1.775431in}}{\pgfqpoint{2.507736in}{1.771040in}}{\pgfqpoint{2.518786in}{1.771040in}}%
\pgfpathclose%
\pgfusepath{stroke,fill}%
\end{pgfscope}%
\begin{pgfscope}%
\pgfpathrectangle{\pgfqpoint{0.800000in}{0.528000in}}{\pgfqpoint{4.960000in}{3.696000in}}%
\pgfusepath{clip}%
\pgfsetbuttcap%
\pgfsetroundjoin%
\definecolor{currentfill}{rgb}{0.000000,0.000000,0.000000}%
\pgfsetfillcolor{currentfill}%
\pgfsetlinewidth{1.003750pt}%
\definecolor{currentstroke}{rgb}{0.000000,0.000000,0.000000}%
\pgfsetstrokecolor{currentstroke}%
\pgfsetdash{}{0pt}%
\pgfpathmoveto{\pgfqpoint{2.518786in}{1.771040in}}%
\pgfpathcurveto{\pgfqpoint{2.529836in}{1.771040in}}{\pgfqpoint{2.540435in}{1.775431in}}{\pgfqpoint{2.548249in}{1.783244in}}%
\pgfpathcurveto{\pgfqpoint{2.556062in}{1.791058in}}{\pgfqpoint{2.560452in}{1.801657in}}{\pgfqpoint{2.560452in}{1.812707in}}%
\pgfpathcurveto{\pgfqpoint{2.560452in}{1.823757in}}{\pgfqpoint{2.556062in}{1.834356in}}{\pgfqpoint{2.548249in}{1.842170in}}%
\pgfpathcurveto{\pgfqpoint{2.540435in}{1.849983in}}{\pgfqpoint{2.529836in}{1.854374in}}{\pgfqpoint{2.518786in}{1.854374in}}%
\pgfpathcurveto{\pgfqpoint{2.507736in}{1.854374in}}{\pgfqpoint{2.497137in}{1.849983in}}{\pgfqpoint{2.489323in}{1.842170in}}%
\pgfpathcurveto{\pgfqpoint{2.481509in}{1.834356in}}{\pgfqpoint{2.477119in}{1.823757in}}{\pgfqpoint{2.477119in}{1.812707in}}%
\pgfpathcurveto{\pgfqpoint{2.477119in}{1.801657in}}{\pgfqpoint{2.481509in}{1.791058in}}{\pgfqpoint{2.489323in}{1.783244in}}%
\pgfpathcurveto{\pgfqpoint{2.497137in}{1.775431in}}{\pgfqpoint{2.507736in}{1.771040in}}{\pgfqpoint{2.518786in}{1.771040in}}%
\pgfpathclose%
\pgfusepath{stroke,fill}%
\end{pgfscope}%
\begin{pgfscope}%
\pgfpathrectangle{\pgfqpoint{0.800000in}{0.528000in}}{\pgfqpoint{4.960000in}{3.696000in}}%
\pgfusepath{clip}%
\pgfsetbuttcap%
\pgfsetroundjoin%
\definecolor{currentfill}{rgb}{0.000000,0.000000,0.000000}%
\pgfsetfillcolor{currentfill}%
\pgfsetlinewidth{1.003750pt}%
\definecolor{currentstroke}{rgb}{0.000000,0.000000,0.000000}%
\pgfsetstrokecolor{currentstroke}%
\pgfsetdash{}{0pt}%
\pgfpathmoveto{\pgfqpoint{2.518786in}{1.771040in}}%
\pgfpathcurveto{\pgfqpoint{2.529836in}{1.771040in}}{\pgfqpoint{2.540435in}{1.775431in}}{\pgfqpoint{2.548249in}{1.783244in}}%
\pgfpathcurveto{\pgfqpoint{2.556062in}{1.791058in}}{\pgfqpoint{2.560452in}{1.801657in}}{\pgfqpoint{2.560452in}{1.812707in}}%
\pgfpathcurveto{\pgfqpoint{2.560452in}{1.823757in}}{\pgfqpoint{2.556062in}{1.834356in}}{\pgfqpoint{2.548249in}{1.842170in}}%
\pgfpathcurveto{\pgfqpoint{2.540435in}{1.849983in}}{\pgfqpoint{2.529836in}{1.854374in}}{\pgfqpoint{2.518786in}{1.854374in}}%
\pgfpathcurveto{\pgfqpoint{2.507736in}{1.854374in}}{\pgfqpoint{2.497137in}{1.849983in}}{\pgfqpoint{2.489323in}{1.842170in}}%
\pgfpathcurveto{\pgfqpoint{2.481509in}{1.834356in}}{\pgfqpoint{2.477119in}{1.823757in}}{\pgfqpoint{2.477119in}{1.812707in}}%
\pgfpathcurveto{\pgfqpoint{2.477119in}{1.801657in}}{\pgfqpoint{2.481509in}{1.791058in}}{\pgfqpoint{2.489323in}{1.783244in}}%
\pgfpathcurveto{\pgfqpoint{2.497137in}{1.775431in}}{\pgfqpoint{2.507736in}{1.771040in}}{\pgfqpoint{2.518786in}{1.771040in}}%
\pgfpathclose%
\pgfusepath{stroke,fill}%
\end{pgfscope}%
\begin{pgfscope}%
\pgfpathrectangle{\pgfqpoint{0.800000in}{0.528000in}}{\pgfqpoint{4.960000in}{3.696000in}}%
\pgfusepath{clip}%
\pgfsetbuttcap%
\pgfsetroundjoin%
\definecolor{currentfill}{rgb}{0.000000,0.000000,0.000000}%
\pgfsetfillcolor{currentfill}%
\pgfsetlinewidth{1.003750pt}%
\definecolor{currentstroke}{rgb}{0.000000,0.000000,0.000000}%
\pgfsetstrokecolor{currentstroke}%
\pgfsetdash{}{0pt}%
\pgfpathmoveto{\pgfqpoint{2.518786in}{1.771040in}}%
\pgfpathcurveto{\pgfqpoint{2.529836in}{1.771040in}}{\pgfqpoint{2.540435in}{1.775431in}}{\pgfqpoint{2.548249in}{1.783244in}}%
\pgfpathcurveto{\pgfqpoint{2.556062in}{1.791058in}}{\pgfqpoint{2.560452in}{1.801657in}}{\pgfqpoint{2.560452in}{1.812707in}}%
\pgfpathcurveto{\pgfqpoint{2.560452in}{1.823757in}}{\pgfqpoint{2.556062in}{1.834356in}}{\pgfqpoint{2.548249in}{1.842170in}}%
\pgfpathcurveto{\pgfqpoint{2.540435in}{1.849983in}}{\pgfqpoint{2.529836in}{1.854374in}}{\pgfqpoint{2.518786in}{1.854374in}}%
\pgfpathcurveto{\pgfqpoint{2.507736in}{1.854374in}}{\pgfqpoint{2.497137in}{1.849983in}}{\pgfqpoint{2.489323in}{1.842170in}}%
\pgfpathcurveto{\pgfqpoint{2.481509in}{1.834356in}}{\pgfqpoint{2.477119in}{1.823757in}}{\pgfqpoint{2.477119in}{1.812707in}}%
\pgfpathcurveto{\pgfqpoint{2.477119in}{1.801657in}}{\pgfqpoint{2.481509in}{1.791058in}}{\pgfqpoint{2.489323in}{1.783244in}}%
\pgfpathcurveto{\pgfqpoint{2.497137in}{1.775431in}}{\pgfqpoint{2.507736in}{1.771040in}}{\pgfqpoint{2.518786in}{1.771040in}}%
\pgfpathclose%
\pgfusepath{stroke,fill}%
\end{pgfscope}%
\begin{pgfscope}%
\pgfpathrectangle{\pgfqpoint{0.800000in}{0.528000in}}{\pgfqpoint{4.960000in}{3.696000in}}%
\pgfusepath{clip}%
\pgfsetbuttcap%
\pgfsetroundjoin%
\definecolor{currentfill}{rgb}{0.000000,0.000000,0.000000}%
\pgfsetfillcolor{currentfill}%
\pgfsetlinewidth{1.003750pt}%
\definecolor{currentstroke}{rgb}{0.000000,0.000000,0.000000}%
\pgfsetstrokecolor{currentstroke}%
\pgfsetdash{}{0pt}%
\pgfpathmoveto{\pgfqpoint{2.518786in}{1.771040in}}%
\pgfpathcurveto{\pgfqpoint{2.529836in}{1.771040in}}{\pgfqpoint{2.540435in}{1.775431in}}{\pgfqpoint{2.548249in}{1.783244in}}%
\pgfpathcurveto{\pgfqpoint{2.556062in}{1.791058in}}{\pgfqpoint{2.560452in}{1.801657in}}{\pgfqpoint{2.560452in}{1.812707in}}%
\pgfpathcurveto{\pgfqpoint{2.560452in}{1.823757in}}{\pgfqpoint{2.556062in}{1.834356in}}{\pgfqpoint{2.548249in}{1.842170in}}%
\pgfpathcurveto{\pgfqpoint{2.540435in}{1.849983in}}{\pgfqpoint{2.529836in}{1.854374in}}{\pgfqpoint{2.518786in}{1.854374in}}%
\pgfpathcurveto{\pgfqpoint{2.507736in}{1.854374in}}{\pgfqpoint{2.497137in}{1.849983in}}{\pgfqpoint{2.489323in}{1.842170in}}%
\pgfpathcurveto{\pgfqpoint{2.481509in}{1.834356in}}{\pgfqpoint{2.477119in}{1.823757in}}{\pgfqpoint{2.477119in}{1.812707in}}%
\pgfpathcurveto{\pgfqpoint{2.477119in}{1.801657in}}{\pgfqpoint{2.481509in}{1.791058in}}{\pgfqpoint{2.489323in}{1.783244in}}%
\pgfpathcurveto{\pgfqpoint{2.497137in}{1.775431in}}{\pgfqpoint{2.507736in}{1.771040in}}{\pgfqpoint{2.518786in}{1.771040in}}%
\pgfpathclose%
\pgfusepath{stroke,fill}%
\end{pgfscope}%
\begin{pgfscope}%
\pgfpathrectangle{\pgfqpoint{0.800000in}{0.528000in}}{\pgfqpoint{4.960000in}{3.696000in}}%
\pgfusepath{clip}%
\pgfsetbuttcap%
\pgfsetroundjoin%
\definecolor{currentfill}{rgb}{0.000000,0.000000,0.000000}%
\pgfsetfillcolor{currentfill}%
\pgfsetlinewidth{1.003750pt}%
\definecolor{currentstroke}{rgb}{0.000000,0.000000,0.000000}%
\pgfsetstrokecolor{currentstroke}%
\pgfsetdash{}{0pt}%
\pgfpathmoveto{\pgfqpoint{2.518786in}{1.771040in}}%
\pgfpathcurveto{\pgfqpoint{2.529836in}{1.771040in}}{\pgfqpoint{2.540435in}{1.775431in}}{\pgfqpoint{2.548249in}{1.783244in}}%
\pgfpathcurveto{\pgfqpoint{2.556062in}{1.791058in}}{\pgfqpoint{2.560452in}{1.801657in}}{\pgfqpoint{2.560452in}{1.812707in}}%
\pgfpathcurveto{\pgfqpoint{2.560452in}{1.823757in}}{\pgfqpoint{2.556062in}{1.834356in}}{\pgfqpoint{2.548249in}{1.842170in}}%
\pgfpathcurveto{\pgfqpoint{2.540435in}{1.849983in}}{\pgfqpoint{2.529836in}{1.854374in}}{\pgfqpoint{2.518786in}{1.854374in}}%
\pgfpathcurveto{\pgfqpoint{2.507736in}{1.854374in}}{\pgfqpoint{2.497137in}{1.849983in}}{\pgfqpoint{2.489323in}{1.842170in}}%
\pgfpathcurveto{\pgfqpoint{2.481509in}{1.834356in}}{\pgfqpoint{2.477119in}{1.823757in}}{\pgfqpoint{2.477119in}{1.812707in}}%
\pgfpathcurveto{\pgfqpoint{2.477119in}{1.801657in}}{\pgfqpoint{2.481509in}{1.791058in}}{\pgfqpoint{2.489323in}{1.783244in}}%
\pgfpathcurveto{\pgfqpoint{2.497137in}{1.775431in}}{\pgfqpoint{2.507736in}{1.771040in}}{\pgfqpoint{2.518786in}{1.771040in}}%
\pgfpathclose%
\pgfusepath{stroke,fill}%
\end{pgfscope}%
\begin{pgfscope}%
\pgfpathrectangle{\pgfqpoint{0.800000in}{0.528000in}}{\pgfqpoint{4.960000in}{3.696000in}}%
\pgfusepath{clip}%
\pgfsetbuttcap%
\pgfsetroundjoin%
\definecolor{currentfill}{rgb}{0.000000,0.000000,0.000000}%
\pgfsetfillcolor{currentfill}%
\pgfsetlinewidth{1.003750pt}%
\definecolor{currentstroke}{rgb}{0.000000,0.000000,0.000000}%
\pgfsetstrokecolor{currentstroke}%
\pgfsetdash{}{0pt}%
\pgfpathmoveto{\pgfqpoint{2.518786in}{1.771040in}}%
\pgfpathcurveto{\pgfqpoint{2.529836in}{1.771040in}}{\pgfqpoint{2.540435in}{1.775431in}}{\pgfqpoint{2.548249in}{1.783244in}}%
\pgfpathcurveto{\pgfqpoint{2.556062in}{1.791058in}}{\pgfqpoint{2.560452in}{1.801657in}}{\pgfqpoint{2.560452in}{1.812707in}}%
\pgfpathcurveto{\pgfqpoint{2.560452in}{1.823757in}}{\pgfqpoint{2.556062in}{1.834356in}}{\pgfqpoint{2.548249in}{1.842170in}}%
\pgfpathcurveto{\pgfqpoint{2.540435in}{1.849983in}}{\pgfqpoint{2.529836in}{1.854374in}}{\pgfqpoint{2.518786in}{1.854374in}}%
\pgfpathcurveto{\pgfqpoint{2.507736in}{1.854374in}}{\pgfqpoint{2.497137in}{1.849983in}}{\pgfqpoint{2.489323in}{1.842170in}}%
\pgfpathcurveto{\pgfqpoint{2.481509in}{1.834356in}}{\pgfqpoint{2.477119in}{1.823757in}}{\pgfqpoint{2.477119in}{1.812707in}}%
\pgfpathcurveto{\pgfqpoint{2.477119in}{1.801657in}}{\pgfqpoint{2.481509in}{1.791058in}}{\pgfqpoint{2.489323in}{1.783244in}}%
\pgfpathcurveto{\pgfqpoint{2.497137in}{1.775431in}}{\pgfqpoint{2.507736in}{1.771040in}}{\pgfqpoint{2.518786in}{1.771040in}}%
\pgfpathclose%
\pgfusepath{stroke,fill}%
\end{pgfscope}%
\begin{pgfscope}%
\pgfpathrectangle{\pgfqpoint{0.800000in}{0.528000in}}{\pgfqpoint{4.960000in}{3.696000in}}%
\pgfusepath{clip}%
\pgfsetbuttcap%
\pgfsetroundjoin%
\definecolor{currentfill}{rgb}{0.000000,0.000000,0.000000}%
\pgfsetfillcolor{currentfill}%
\pgfsetlinewidth{1.003750pt}%
\definecolor{currentstroke}{rgb}{0.000000,0.000000,0.000000}%
\pgfsetstrokecolor{currentstroke}%
\pgfsetdash{}{0pt}%
\pgfpathmoveto{\pgfqpoint{2.518786in}{1.771040in}}%
\pgfpathcurveto{\pgfqpoint{2.529836in}{1.771040in}}{\pgfqpoint{2.540435in}{1.775431in}}{\pgfqpoint{2.548249in}{1.783244in}}%
\pgfpathcurveto{\pgfqpoint{2.556062in}{1.791058in}}{\pgfqpoint{2.560452in}{1.801657in}}{\pgfqpoint{2.560452in}{1.812707in}}%
\pgfpathcurveto{\pgfqpoint{2.560452in}{1.823757in}}{\pgfqpoint{2.556062in}{1.834356in}}{\pgfqpoint{2.548249in}{1.842170in}}%
\pgfpathcurveto{\pgfqpoint{2.540435in}{1.849983in}}{\pgfqpoint{2.529836in}{1.854374in}}{\pgfqpoint{2.518786in}{1.854374in}}%
\pgfpathcurveto{\pgfqpoint{2.507736in}{1.854374in}}{\pgfqpoint{2.497137in}{1.849983in}}{\pgfqpoint{2.489323in}{1.842170in}}%
\pgfpathcurveto{\pgfqpoint{2.481509in}{1.834356in}}{\pgfqpoint{2.477119in}{1.823757in}}{\pgfqpoint{2.477119in}{1.812707in}}%
\pgfpathcurveto{\pgfqpoint{2.477119in}{1.801657in}}{\pgfqpoint{2.481509in}{1.791058in}}{\pgfqpoint{2.489323in}{1.783244in}}%
\pgfpathcurveto{\pgfqpoint{2.497137in}{1.775431in}}{\pgfqpoint{2.507736in}{1.771040in}}{\pgfqpoint{2.518786in}{1.771040in}}%
\pgfpathclose%
\pgfusepath{stroke,fill}%
\end{pgfscope}%
\begin{pgfscope}%
\pgfpathrectangle{\pgfqpoint{0.800000in}{0.528000in}}{\pgfqpoint{4.960000in}{3.696000in}}%
\pgfusepath{clip}%
\pgfsetbuttcap%
\pgfsetroundjoin%
\definecolor{currentfill}{rgb}{0.000000,0.000000,0.000000}%
\pgfsetfillcolor{currentfill}%
\pgfsetlinewidth{1.003750pt}%
\definecolor{currentstroke}{rgb}{0.000000,0.000000,0.000000}%
\pgfsetstrokecolor{currentstroke}%
\pgfsetdash{}{0pt}%
\pgfpathmoveto{\pgfqpoint{2.518786in}{1.771040in}}%
\pgfpathcurveto{\pgfqpoint{2.529836in}{1.771040in}}{\pgfqpoint{2.540435in}{1.775431in}}{\pgfqpoint{2.548249in}{1.783244in}}%
\pgfpathcurveto{\pgfqpoint{2.556062in}{1.791058in}}{\pgfqpoint{2.560452in}{1.801657in}}{\pgfqpoint{2.560452in}{1.812707in}}%
\pgfpathcurveto{\pgfqpoint{2.560452in}{1.823757in}}{\pgfqpoint{2.556062in}{1.834356in}}{\pgfqpoint{2.548249in}{1.842170in}}%
\pgfpathcurveto{\pgfqpoint{2.540435in}{1.849983in}}{\pgfqpoint{2.529836in}{1.854374in}}{\pgfqpoint{2.518786in}{1.854374in}}%
\pgfpathcurveto{\pgfqpoint{2.507736in}{1.854374in}}{\pgfqpoint{2.497137in}{1.849983in}}{\pgfqpoint{2.489323in}{1.842170in}}%
\pgfpathcurveto{\pgfqpoint{2.481509in}{1.834356in}}{\pgfqpoint{2.477119in}{1.823757in}}{\pgfqpoint{2.477119in}{1.812707in}}%
\pgfpathcurveto{\pgfqpoint{2.477119in}{1.801657in}}{\pgfqpoint{2.481509in}{1.791058in}}{\pgfqpoint{2.489323in}{1.783244in}}%
\pgfpathcurveto{\pgfqpoint{2.497137in}{1.775431in}}{\pgfqpoint{2.507736in}{1.771040in}}{\pgfqpoint{2.518786in}{1.771040in}}%
\pgfpathclose%
\pgfusepath{stroke,fill}%
\end{pgfscope}%
\begin{pgfscope}%
\pgfpathrectangle{\pgfqpoint{0.800000in}{0.528000in}}{\pgfqpoint{4.960000in}{3.696000in}}%
\pgfusepath{clip}%
\pgfsetbuttcap%
\pgfsetroundjoin%
\definecolor{currentfill}{rgb}{0.000000,0.000000,0.000000}%
\pgfsetfillcolor{currentfill}%
\pgfsetlinewidth{1.003750pt}%
\definecolor{currentstroke}{rgb}{0.000000,0.000000,0.000000}%
\pgfsetstrokecolor{currentstroke}%
\pgfsetdash{}{0pt}%
\pgfpathmoveto{\pgfqpoint{2.518786in}{1.771040in}}%
\pgfpathcurveto{\pgfqpoint{2.529836in}{1.771040in}}{\pgfqpoint{2.540435in}{1.775431in}}{\pgfqpoint{2.548249in}{1.783244in}}%
\pgfpathcurveto{\pgfqpoint{2.556062in}{1.791058in}}{\pgfqpoint{2.560452in}{1.801657in}}{\pgfqpoint{2.560452in}{1.812707in}}%
\pgfpathcurveto{\pgfqpoint{2.560452in}{1.823757in}}{\pgfqpoint{2.556062in}{1.834356in}}{\pgfqpoint{2.548249in}{1.842170in}}%
\pgfpathcurveto{\pgfqpoint{2.540435in}{1.849983in}}{\pgfqpoint{2.529836in}{1.854374in}}{\pgfqpoint{2.518786in}{1.854374in}}%
\pgfpathcurveto{\pgfqpoint{2.507736in}{1.854374in}}{\pgfqpoint{2.497137in}{1.849983in}}{\pgfqpoint{2.489323in}{1.842170in}}%
\pgfpathcurveto{\pgfqpoint{2.481509in}{1.834356in}}{\pgfqpoint{2.477119in}{1.823757in}}{\pgfqpoint{2.477119in}{1.812707in}}%
\pgfpathcurveto{\pgfqpoint{2.477119in}{1.801657in}}{\pgfqpoint{2.481509in}{1.791058in}}{\pgfqpoint{2.489323in}{1.783244in}}%
\pgfpathcurveto{\pgfqpoint{2.497137in}{1.775431in}}{\pgfqpoint{2.507736in}{1.771040in}}{\pgfqpoint{2.518786in}{1.771040in}}%
\pgfpathclose%
\pgfusepath{stroke,fill}%
\end{pgfscope}%
\begin{pgfscope}%
\pgfpathrectangle{\pgfqpoint{0.800000in}{0.528000in}}{\pgfqpoint{4.960000in}{3.696000in}}%
\pgfusepath{clip}%
\pgfsetbuttcap%
\pgfsetroundjoin%
\definecolor{currentfill}{rgb}{0.000000,0.000000,0.000000}%
\pgfsetfillcolor{currentfill}%
\pgfsetlinewidth{1.003750pt}%
\definecolor{currentstroke}{rgb}{0.000000,0.000000,0.000000}%
\pgfsetstrokecolor{currentstroke}%
\pgfsetdash{}{0pt}%
\pgfpathmoveto{\pgfqpoint{2.518786in}{1.771040in}}%
\pgfpathcurveto{\pgfqpoint{2.529836in}{1.771040in}}{\pgfqpoint{2.540435in}{1.775431in}}{\pgfqpoint{2.548249in}{1.783244in}}%
\pgfpathcurveto{\pgfqpoint{2.556062in}{1.791058in}}{\pgfqpoint{2.560452in}{1.801657in}}{\pgfqpoint{2.560452in}{1.812707in}}%
\pgfpathcurveto{\pgfqpoint{2.560452in}{1.823757in}}{\pgfqpoint{2.556062in}{1.834356in}}{\pgfqpoint{2.548249in}{1.842170in}}%
\pgfpathcurveto{\pgfqpoint{2.540435in}{1.849983in}}{\pgfqpoint{2.529836in}{1.854374in}}{\pgfqpoint{2.518786in}{1.854374in}}%
\pgfpathcurveto{\pgfqpoint{2.507736in}{1.854374in}}{\pgfqpoint{2.497137in}{1.849983in}}{\pgfqpoint{2.489323in}{1.842170in}}%
\pgfpathcurveto{\pgfqpoint{2.481509in}{1.834356in}}{\pgfqpoint{2.477119in}{1.823757in}}{\pgfqpoint{2.477119in}{1.812707in}}%
\pgfpathcurveto{\pgfqpoint{2.477119in}{1.801657in}}{\pgfqpoint{2.481509in}{1.791058in}}{\pgfqpoint{2.489323in}{1.783244in}}%
\pgfpathcurveto{\pgfqpoint{2.497137in}{1.775431in}}{\pgfqpoint{2.507736in}{1.771040in}}{\pgfqpoint{2.518786in}{1.771040in}}%
\pgfpathclose%
\pgfusepath{stroke,fill}%
\end{pgfscope}%
\begin{pgfscope}%
\pgfpathrectangle{\pgfqpoint{0.800000in}{0.528000in}}{\pgfqpoint{4.960000in}{3.696000in}}%
\pgfusepath{clip}%
\pgfsetbuttcap%
\pgfsetroundjoin%
\definecolor{currentfill}{rgb}{0.000000,0.000000,0.000000}%
\pgfsetfillcolor{currentfill}%
\pgfsetlinewidth{1.003750pt}%
\definecolor{currentstroke}{rgb}{0.000000,0.000000,0.000000}%
\pgfsetstrokecolor{currentstroke}%
\pgfsetdash{}{0pt}%
\pgfpathmoveto{\pgfqpoint{2.518786in}{1.771040in}}%
\pgfpathcurveto{\pgfqpoint{2.529836in}{1.771040in}}{\pgfqpoint{2.540435in}{1.775431in}}{\pgfqpoint{2.548249in}{1.783244in}}%
\pgfpathcurveto{\pgfqpoint{2.556062in}{1.791058in}}{\pgfqpoint{2.560452in}{1.801657in}}{\pgfqpoint{2.560452in}{1.812707in}}%
\pgfpathcurveto{\pgfqpoint{2.560452in}{1.823757in}}{\pgfqpoint{2.556062in}{1.834356in}}{\pgfqpoint{2.548249in}{1.842170in}}%
\pgfpathcurveto{\pgfqpoint{2.540435in}{1.849983in}}{\pgfqpoint{2.529836in}{1.854374in}}{\pgfqpoint{2.518786in}{1.854374in}}%
\pgfpathcurveto{\pgfqpoint{2.507736in}{1.854374in}}{\pgfqpoint{2.497137in}{1.849983in}}{\pgfqpoint{2.489323in}{1.842170in}}%
\pgfpathcurveto{\pgfqpoint{2.481509in}{1.834356in}}{\pgfqpoint{2.477119in}{1.823757in}}{\pgfqpoint{2.477119in}{1.812707in}}%
\pgfpathcurveto{\pgfqpoint{2.477119in}{1.801657in}}{\pgfqpoint{2.481509in}{1.791058in}}{\pgfqpoint{2.489323in}{1.783244in}}%
\pgfpathcurveto{\pgfqpoint{2.497137in}{1.775431in}}{\pgfqpoint{2.507736in}{1.771040in}}{\pgfqpoint{2.518786in}{1.771040in}}%
\pgfpathclose%
\pgfusepath{stroke,fill}%
\end{pgfscope}%
\begin{pgfscope}%
\pgfpathrectangle{\pgfqpoint{0.800000in}{0.528000in}}{\pgfqpoint{4.960000in}{3.696000in}}%
\pgfusepath{clip}%
\pgfsetbuttcap%
\pgfsetroundjoin%
\definecolor{currentfill}{rgb}{0.000000,0.000000,0.000000}%
\pgfsetfillcolor{currentfill}%
\pgfsetlinewidth{1.003750pt}%
\definecolor{currentstroke}{rgb}{0.000000,0.000000,0.000000}%
\pgfsetstrokecolor{currentstroke}%
\pgfsetdash{}{0pt}%
\pgfpathmoveto{\pgfqpoint{2.518786in}{1.771040in}}%
\pgfpathcurveto{\pgfqpoint{2.529836in}{1.771040in}}{\pgfqpoint{2.540435in}{1.775431in}}{\pgfqpoint{2.548249in}{1.783244in}}%
\pgfpathcurveto{\pgfqpoint{2.556062in}{1.791058in}}{\pgfqpoint{2.560452in}{1.801657in}}{\pgfqpoint{2.560452in}{1.812707in}}%
\pgfpathcurveto{\pgfqpoint{2.560452in}{1.823757in}}{\pgfqpoint{2.556062in}{1.834356in}}{\pgfqpoint{2.548249in}{1.842170in}}%
\pgfpathcurveto{\pgfqpoint{2.540435in}{1.849983in}}{\pgfqpoint{2.529836in}{1.854374in}}{\pgfqpoint{2.518786in}{1.854374in}}%
\pgfpathcurveto{\pgfqpoint{2.507736in}{1.854374in}}{\pgfqpoint{2.497137in}{1.849983in}}{\pgfqpoint{2.489323in}{1.842170in}}%
\pgfpathcurveto{\pgfqpoint{2.481509in}{1.834356in}}{\pgfqpoint{2.477119in}{1.823757in}}{\pgfqpoint{2.477119in}{1.812707in}}%
\pgfpathcurveto{\pgfqpoint{2.477119in}{1.801657in}}{\pgfqpoint{2.481509in}{1.791058in}}{\pgfqpoint{2.489323in}{1.783244in}}%
\pgfpathcurveto{\pgfqpoint{2.497137in}{1.775431in}}{\pgfqpoint{2.507736in}{1.771040in}}{\pgfqpoint{2.518786in}{1.771040in}}%
\pgfpathclose%
\pgfusepath{stroke,fill}%
\end{pgfscope}%
\begin{pgfscope}%
\pgfpathrectangle{\pgfqpoint{0.800000in}{0.528000in}}{\pgfqpoint{4.960000in}{3.696000in}}%
\pgfusepath{clip}%
\pgfsetbuttcap%
\pgfsetroundjoin%
\definecolor{currentfill}{rgb}{0.000000,0.000000,0.000000}%
\pgfsetfillcolor{currentfill}%
\pgfsetlinewidth{1.003750pt}%
\definecolor{currentstroke}{rgb}{0.000000,0.000000,0.000000}%
\pgfsetstrokecolor{currentstroke}%
\pgfsetdash{}{0pt}%
\pgfpathmoveto{\pgfqpoint{2.518786in}{0.664394in}}%
\pgfpathcurveto{\pgfqpoint{2.529836in}{0.664394in}}{\pgfqpoint{2.540435in}{0.668784in}}{\pgfqpoint{2.548249in}{0.676598in}}%
\pgfpathcurveto{\pgfqpoint{2.556062in}{0.684411in}}{\pgfqpoint{2.560452in}{0.695010in}}{\pgfqpoint{2.560452in}{0.706060in}}%
\pgfpathcurveto{\pgfqpoint{2.560452in}{0.717111in}}{\pgfqpoint{2.556062in}{0.727710in}}{\pgfqpoint{2.548249in}{0.735523in}}%
\pgfpathcurveto{\pgfqpoint{2.540435in}{0.743337in}}{\pgfqpoint{2.529836in}{0.747727in}}{\pgfqpoint{2.518786in}{0.747727in}}%
\pgfpathcurveto{\pgfqpoint{2.507736in}{0.747727in}}{\pgfqpoint{2.497137in}{0.743337in}}{\pgfqpoint{2.489323in}{0.735523in}}%
\pgfpathcurveto{\pgfqpoint{2.481509in}{0.727710in}}{\pgfqpoint{2.477119in}{0.717111in}}{\pgfqpoint{2.477119in}{0.706060in}}%
\pgfpathcurveto{\pgfqpoint{2.477119in}{0.695010in}}{\pgfqpoint{2.481509in}{0.684411in}}{\pgfqpoint{2.489323in}{0.676598in}}%
\pgfpathcurveto{\pgfqpoint{2.497137in}{0.668784in}}{\pgfqpoint{2.507736in}{0.664394in}}{\pgfqpoint{2.518786in}{0.664394in}}%
\pgfpathclose%
\pgfusepath{stroke,fill}%
\end{pgfscope}%
\begin{pgfscope}%
\pgfpathrectangle{\pgfqpoint{0.800000in}{0.528000in}}{\pgfqpoint{4.960000in}{3.696000in}}%
\pgfusepath{clip}%
\pgfsetbuttcap%
\pgfsetroundjoin%
\definecolor{currentfill}{rgb}{0.000000,0.000000,0.000000}%
\pgfsetfillcolor{currentfill}%
\pgfsetlinewidth{1.003750pt}%
\definecolor{currentstroke}{rgb}{0.000000,0.000000,0.000000}%
\pgfsetstrokecolor{currentstroke}%
\pgfsetdash{}{0pt}%
\pgfpathmoveto{\pgfqpoint{2.518786in}{1.771040in}}%
\pgfpathcurveto{\pgfqpoint{2.529836in}{1.771040in}}{\pgfqpoint{2.540435in}{1.775431in}}{\pgfqpoint{2.548249in}{1.783244in}}%
\pgfpathcurveto{\pgfqpoint{2.556062in}{1.791058in}}{\pgfqpoint{2.560452in}{1.801657in}}{\pgfqpoint{2.560452in}{1.812707in}}%
\pgfpathcurveto{\pgfqpoint{2.560452in}{1.823757in}}{\pgfqpoint{2.556062in}{1.834356in}}{\pgfqpoint{2.548249in}{1.842170in}}%
\pgfpathcurveto{\pgfqpoint{2.540435in}{1.849983in}}{\pgfqpoint{2.529836in}{1.854374in}}{\pgfqpoint{2.518786in}{1.854374in}}%
\pgfpathcurveto{\pgfqpoint{2.507736in}{1.854374in}}{\pgfqpoint{2.497137in}{1.849983in}}{\pgfqpoint{2.489323in}{1.842170in}}%
\pgfpathcurveto{\pgfqpoint{2.481509in}{1.834356in}}{\pgfqpoint{2.477119in}{1.823757in}}{\pgfqpoint{2.477119in}{1.812707in}}%
\pgfpathcurveto{\pgfqpoint{2.477119in}{1.801657in}}{\pgfqpoint{2.481509in}{1.791058in}}{\pgfqpoint{2.489323in}{1.783244in}}%
\pgfpathcurveto{\pgfqpoint{2.497137in}{1.775431in}}{\pgfqpoint{2.507736in}{1.771040in}}{\pgfqpoint{2.518786in}{1.771040in}}%
\pgfpathclose%
\pgfusepath{stroke,fill}%
\end{pgfscope}%
\begin{pgfscope}%
\pgfpathrectangle{\pgfqpoint{0.800000in}{0.528000in}}{\pgfqpoint{4.960000in}{3.696000in}}%
\pgfusepath{clip}%
\pgfsetbuttcap%
\pgfsetroundjoin%
\definecolor{currentfill}{rgb}{0.000000,0.000000,0.000000}%
\pgfsetfillcolor{currentfill}%
\pgfsetlinewidth{1.003750pt}%
\definecolor{currentstroke}{rgb}{0.000000,0.000000,0.000000}%
\pgfsetstrokecolor{currentstroke}%
\pgfsetdash{}{0pt}%
\pgfpathmoveto{\pgfqpoint{2.518786in}{1.771040in}}%
\pgfpathcurveto{\pgfqpoint{2.529836in}{1.771040in}}{\pgfqpoint{2.540435in}{1.775431in}}{\pgfqpoint{2.548249in}{1.783244in}}%
\pgfpathcurveto{\pgfqpoint{2.556062in}{1.791058in}}{\pgfqpoint{2.560452in}{1.801657in}}{\pgfqpoint{2.560452in}{1.812707in}}%
\pgfpathcurveto{\pgfqpoint{2.560452in}{1.823757in}}{\pgfqpoint{2.556062in}{1.834356in}}{\pgfqpoint{2.548249in}{1.842170in}}%
\pgfpathcurveto{\pgfqpoint{2.540435in}{1.849983in}}{\pgfqpoint{2.529836in}{1.854374in}}{\pgfqpoint{2.518786in}{1.854374in}}%
\pgfpathcurveto{\pgfqpoint{2.507736in}{1.854374in}}{\pgfqpoint{2.497137in}{1.849983in}}{\pgfqpoint{2.489323in}{1.842170in}}%
\pgfpathcurveto{\pgfqpoint{2.481509in}{1.834356in}}{\pgfqpoint{2.477119in}{1.823757in}}{\pgfqpoint{2.477119in}{1.812707in}}%
\pgfpathcurveto{\pgfqpoint{2.477119in}{1.801657in}}{\pgfqpoint{2.481509in}{1.791058in}}{\pgfqpoint{2.489323in}{1.783244in}}%
\pgfpathcurveto{\pgfqpoint{2.497137in}{1.775431in}}{\pgfqpoint{2.507736in}{1.771040in}}{\pgfqpoint{2.518786in}{1.771040in}}%
\pgfpathclose%
\pgfusepath{stroke,fill}%
\end{pgfscope}%
\begin{pgfscope}%
\pgfpathrectangle{\pgfqpoint{0.800000in}{0.528000in}}{\pgfqpoint{4.960000in}{3.696000in}}%
\pgfusepath{clip}%
\pgfsetbuttcap%
\pgfsetroundjoin%
\definecolor{currentfill}{rgb}{0.000000,0.000000,0.000000}%
\pgfsetfillcolor{currentfill}%
\pgfsetlinewidth{1.003750pt}%
\definecolor{currentstroke}{rgb}{0.000000,0.000000,0.000000}%
\pgfsetstrokecolor{currentstroke}%
\pgfsetdash{}{0pt}%
\pgfpathmoveto{\pgfqpoint{2.518786in}{1.771040in}}%
\pgfpathcurveto{\pgfqpoint{2.529836in}{1.771040in}}{\pgfqpoint{2.540435in}{1.775431in}}{\pgfqpoint{2.548249in}{1.783244in}}%
\pgfpathcurveto{\pgfqpoint{2.556062in}{1.791058in}}{\pgfqpoint{2.560452in}{1.801657in}}{\pgfqpoint{2.560452in}{1.812707in}}%
\pgfpathcurveto{\pgfqpoint{2.560452in}{1.823757in}}{\pgfqpoint{2.556062in}{1.834356in}}{\pgfqpoint{2.548249in}{1.842170in}}%
\pgfpathcurveto{\pgfqpoint{2.540435in}{1.849983in}}{\pgfqpoint{2.529836in}{1.854374in}}{\pgfqpoint{2.518786in}{1.854374in}}%
\pgfpathcurveto{\pgfqpoint{2.507736in}{1.854374in}}{\pgfqpoint{2.497137in}{1.849983in}}{\pgfqpoint{2.489323in}{1.842170in}}%
\pgfpathcurveto{\pgfqpoint{2.481509in}{1.834356in}}{\pgfqpoint{2.477119in}{1.823757in}}{\pgfqpoint{2.477119in}{1.812707in}}%
\pgfpathcurveto{\pgfqpoint{2.477119in}{1.801657in}}{\pgfqpoint{2.481509in}{1.791058in}}{\pgfqpoint{2.489323in}{1.783244in}}%
\pgfpathcurveto{\pgfqpoint{2.497137in}{1.775431in}}{\pgfqpoint{2.507736in}{1.771040in}}{\pgfqpoint{2.518786in}{1.771040in}}%
\pgfpathclose%
\pgfusepath{stroke,fill}%
\end{pgfscope}%
\begin{pgfscope}%
\pgfpathrectangle{\pgfqpoint{0.800000in}{0.528000in}}{\pgfqpoint{4.960000in}{3.696000in}}%
\pgfusepath{clip}%
\pgfsetbuttcap%
\pgfsetroundjoin%
\definecolor{currentfill}{rgb}{0.000000,0.000000,0.000000}%
\pgfsetfillcolor{currentfill}%
\pgfsetlinewidth{1.003750pt}%
\definecolor{currentstroke}{rgb}{0.000000,0.000000,0.000000}%
\pgfsetstrokecolor{currentstroke}%
\pgfsetdash{}{0pt}%
\pgfpathmoveto{\pgfqpoint{2.518786in}{1.771040in}}%
\pgfpathcurveto{\pgfqpoint{2.529836in}{1.771040in}}{\pgfqpoint{2.540435in}{1.775431in}}{\pgfqpoint{2.548249in}{1.783244in}}%
\pgfpathcurveto{\pgfqpoint{2.556062in}{1.791058in}}{\pgfqpoint{2.560452in}{1.801657in}}{\pgfqpoint{2.560452in}{1.812707in}}%
\pgfpathcurveto{\pgfqpoint{2.560452in}{1.823757in}}{\pgfqpoint{2.556062in}{1.834356in}}{\pgfqpoint{2.548249in}{1.842170in}}%
\pgfpathcurveto{\pgfqpoint{2.540435in}{1.849983in}}{\pgfqpoint{2.529836in}{1.854374in}}{\pgfqpoint{2.518786in}{1.854374in}}%
\pgfpathcurveto{\pgfqpoint{2.507736in}{1.854374in}}{\pgfqpoint{2.497137in}{1.849983in}}{\pgfqpoint{2.489323in}{1.842170in}}%
\pgfpathcurveto{\pgfqpoint{2.481509in}{1.834356in}}{\pgfqpoint{2.477119in}{1.823757in}}{\pgfqpoint{2.477119in}{1.812707in}}%
\pgfpathcurveto{\pgfqpoint{2.477119in}{1.801657in}}{\pgfqpoint{2.481509in}{1.791058in}}{\pgfqpoint{2.489323in}{1.783244in}}%
\pgfpathcurveto{\pgfqpoint{2.497137in}{1.775431in}}{\pgfqpoint{2.507736in}{1.771040in}}{\pgfqpoint{2.518786in}{1.771040in}}%
\pgfpathclose%
\pgfusepath{stroke,fill}%
\end{pgfscope}%
\begin{pgfscope}%
\pgfpathrectangle{\pgfqpoint{0.800000in}{0.528000in}}{\pgfqpoint{4.960000in}{3.696000in}}%
\pgfusepath{clip}%
\pgfsetbuttcap%
\pgfsetroundjoin%
\definecolor{currentfill}{rgb}{0.000000,0.000000,0.000000}%
\pgfsetfillcolor{currentfill}%
\pgfsetlinewidth{1.003750pt}%
\definecolor{currentstroke}{rgb}{0.000000,0.000000,0.000000}%
\pgfsetstrokecolor{currentstroke}%
\pgfsetdash{}{0pt}%
\pgfpathmoveto{\pgfqpoint{2.518786in}{1.771040in}}%
\pgfpathcurveto{\pgfqpoint{2.529836in}{1.771040in}}{\pgfqpoint{2.540435in}{1.775431in}}{\pgfqpoint{2.548249in}{1.783244in}}%
\pgfpathcurveto{\pgfqpoint{2.556062in}{1.791058in}}{\pgfqpoint{2.560452in}{1.801657in}}{\pgfqpoint{2.560452in}{1.812707in}}%
\pgfpathcurveto{\pgfqpoint{2.560452in}{1.823757in}}{\pgfqpoint{2.556062in}{1.834356in}}{\pgfqpoint{2.548249in}{1.842170in}}%
\pgfpathcurveto{\pgfqpoint{2.540435in}{1.849983in}}{\pgfqpoint{2.529836in}{1.854374in}}{\pgfqpoint{2.518786in}{1.854374in}}%
\pgfpathcurveto{\pgfqpoint{2.507736in}{1.854374in}}{\pgfqpoint{2.497137in}{1.849983in}}{\pgfqpoint{2.489323in}{1.842170in}}%
\pgfpathcurveto{\pgfqpoint{2.481509in}{1.834356in}}{\pgfqpoint{2.477119in}{1.823757in}}{\pgfqpoint{2.477119in}{1.812707in}}%
\pgfpathcurveto{\pgfqpoint{2.477119in}{1.801657in}}{\pgfqpoint{2.481509in}{1.791058in}}{\pgfqpoint{2.489323in}{1.783244in}}%
\pgfpathcurveto{\pgfqpoint{2.497137in}{1.775431in}}{\pgfqpoint{2.507736in}{1.771040in}}{\pgfqpoint{2.518786in}{1.771040in}}%
\pgfpathclose%
\pgfusepath{stroke,fill}%
\end{pgfscope}%
\begin{pgfscope}%
\pgfpathrectangle{\pgfqpoint{0.800000in}{0.528000in}}{\pgfqpoint{4.960000in}{3.696000in}}%
\pgfusepath{clip}%
\pgfsetbuttcap%
\pgfsetroundjoin%
\definecolor{currentfill}{rgb}{0.000000,0.000000,0.000000}%
\pgfsetfillcolor{currentfill}%
\pgfsetlinewidth{1.003750pt}%
\definecolor{currentstroke}{rgb}{0.000000,0.000000,0.000000}%
\pgfsetstrokecolor{currentstroke}%
\pgfsetdash{}{0pt}%
\pgfpathmoveto{\pgfqpoint{2.518786in}{1.771040in}}%
\pgfpathcurveto{\pgfqpoint{2.529836in}{1.771040in}}{\pgfqpoint{2.540435in}{1.775431in}}{\pgfqpoint{2.548249in}{1.783244in}}%
\pgfpathcurveto{\pgfqpoint{2.556062in}{1.791058in}}{\pgfqpoint{2.560452in}{1.801657in}}{\pgfqpoint{2.560452in}{1.812707in}}%
\pgfpathcurveto{\pgfqpoint{2.560452in}{1.823757in}}{\pgfqpoint{2.556062in}{1.834356in}}{\pgfqpoint{2.548249in}{1.842170in}}%
\pgfpathcurveto{\pgfqpoint{2.540435in}{1.849983in}}{\pgfqpoint{2.529836in}{1.854374in}}{\pgfqpoint{2.518786in}{1.854374in}}%
\pgfpathcurveto{\pgfqpoint{2.507736in}{1.854374in}}{\pgfqpoint{2.497137in}{1.849983in}}{\pgfqpoint{2.489323in}{1.842170in}}%
\pgfpathcurveto{\pgfqpoint{2.481509in}{1.834356in}}{\pgfqpoint{2.477119in}{1.823757in}}{\pgfqpoint{2.477119in}{1.812707in}}%
\pgfpathcurveto{\pgfqpoint{2.477119in}{1.801657in}}{\pgfqpoint{2.481509in}{1.791058in}}{\pgfqpoint{2.489323in}{1.783244in}}%
\pgfpathcurveto{\pgfqpoint{2.497137in}{1.775431in}}{\pgfqpoint{2.507736in}{1.771040in}}{\pgfqpoint{2.518786in}{1.771040in}}%
\pgfpathclose%
\pgfusepath{stroke,fill}%
\end{pgfscope}%
\begin{pgfscope}%
\pgfpathrectangle{\pgfqpoint{0.800000in}{0.528000in}}{\pgfqpoint{4.960000in}{3.696000in}}%
\pgfusepath{clip}%
\pgfsetbuttcap%
\pgfsetroundjoin%
\definecolor{currentfill}{rgb}{0.000000,0.000000,0.000000}%
\pgfsetfillcolor{currentfill}%
\pgfsetlinewidth{1.003750pt}%
\definecolor{currentstroke}{rgb}{0.000000,0.000000,0.000000}%
\pgfsetstrokecolor{currentstroke}%
\pgfsetdash{}{0pt}%
\pgfpathmoveto{\pgfqpoint{2.518786in}{1.771040in}}%
\pgfpathcurveto{\pgfqpoint{2.529836in}{1.771040in}}{\pgfqpoint{2.540435in}{1.775431in}}{\pgfqpoint{2.548249in}{1.783244in}}%
\pgfpathcurveto{\pgfqpoint{2.556062in}{1.791058in}}{\pgfqpoint{2.560452in}{1.801657in}}{\pgfqpoint{2.560452in}{1.812707in}}%
\pgfpathcurveto{\pgfqpoint{2.560452in}{1.823757in}}{\pgfqpoint{2.556062in}{1.834356in}}{\pgfqpoint{2.548249in}{1.842170in}}%
\pgfpathcurveto{\pgfqpoint{2.540435in}{1.849983in}}{\pgfqpoint{2.529836in}{1.854374in}}{\pgfqpoint{2.518786in}{1.854374in}}%
\pgfpathcurveto{\pgfqpoint{2.507736in}{1.854374in}}{\pgfqpoint{2.497137in}{1.849983in}}{\pgfqpoint{2.489323in}{1.842170in}}%
\pgfpathcurveto{\pgfqpoint{2.481509in}{1.834356in}}{\pgfqpoint{2.477119in}{1.823757in}}{\pgfqpoint{2.477119in}{1.812707in}}%
\pgfpathcurveto{\pgfqpoint{2.477119in}{1.801657in}}{\pgfqpoint{2.481509in}{1.791058in}}{\pgfqpoint{2.489323in}{1.783244in}}%
\pgfpathcurveto{\pgfqpoint{2.497137in}{1.775431in}}{\pgfqpoint{2.507736in}{1.771040in}}{\pgfqpoint{2.518786in}{1.771040in}}%
\pgfpathclose%
\pgfusepath{stroke,fill}%
\end{pgfscope}%
\begin{pgfscope}%
\pgfpathrectangle{\pgfqpoint{0.800000in}{0.528000in}}{\pgfqpoint{4.960000in}{3.696000in}}%
\pgfusepath{clip}%
\pgfsetbuttcap%
\pgfsetroundjoin%
\definecolor{currentfill}{rgb}{0.000000,0.000000,0.000000}%
\pgfsetfillcolor{currentfill}%
\pgfsetlinewidth{1.003750pt}%
\definecolor{currentstroke}{rgb}{0.000000,0.000000,0.000000}%
\pgfsetstrokecolor{currentstroke}%
\pgfsetdash{}{0pt}%
\pgfpathmoveto{\pgfqpoint{2.518786in}{1.771040in}}%
\pgfpathcurveto{\pgfqpoint{2.529836in}{1.771040in}}{\pgfqpoint{2.540435in}{1.775431in}}{\pgfqpoint{2.548249in}{1.783244in}}%
\pgfpathcurveto{\pgfqpoint{2.556062in}{1.791058in}}{\pgfqpoint{2.560452in}{1.801657in}}{\pgfqpoint{2.560452in}{1.812707in}}%
\pgfpathcurveto{\pgfqpoint{2.560452in}{1.823757in}}{\pgfqpoint{2.556062in}{1.834356in}}{\pgfqpoint{2.548249in}{1.842170in}}%
\pgfpathcurveto{\pgfqpoint{2.540435in}{1.849983in}}{\pgfqpoint{2.529836in}{1.854374in}}{\pgfqpoint{2.518786in}{1.854374in}}%
\pgfpathcurveto{\pgfqpoint{2.507736in}{1.854374in}}{\pgfqpoint{2.497137in}{1.849983in}}{\pgfqpoint{2.489323in}{1.842170in}}%
\pgfpathcurveto{\pgfqpoint{2.481509in}{1.834356in}}{\pgfqpoint{2.477119in}{1.823757in}}{\pgfqpoint{2.477119in}{1.812707in}}%
\pgfpathcurveto{\pgfqpoint{2.477119in}{1.801657in}}{\pgfqpoint{2.481509in}{1.791058in}}{\pgfqpoint{2.489323in}{1.783244in}}%
\pgfpathcurveto{\pgfqpoint{2.497137in}{1.775431in}}{\pgfqpoint{2.507736in}{1.771040in}}{\pgfqpoint{2.518786in}{1.771040in}}%
\pgfpathclose%
\pgfusepath{stroke,fill}%
\end{pgfscope}%
\begin{pgfscope}%
\pgfpathrectangle{\pgfqpoint{0.800000in}{0.528000in}}{\pgfqpoint{4.960000in}{3.696000in}}%
\pgfusepath{clip}%
\pgfsetbuttcap%
\pgfsetroundjoin%
\definecolor{currentfill}{rgb}{0.000000,0.000000,0.000000}%
\pgfsetfillcolor{currentfill}%
\pgfsetlinewidth{1.003750pt}%
\definecolor{currentstroke}{rgb}{0.000000,0.000000,0.000000}%
\pgfsetstrokecolor{currentstroke}%
\pgfsetdash{}{0pt}%
\pgfpathmoveto{\pgfqpoint{2.518786in}{1.771040in}}%
\pgfpathcurveto{\pgfqpoint{2.529836in}{1.771040in}}{\pgfqpoint{2.540435in}{1.775431in}}{\pgfqpoint{2.548249in}{1.783244in}}%
\pgfpathcurveto{\pgfqpoint{2.556062in}{1.791058in}}{\pgfqpoint{2.560452in}{1.801657in}}{\pgfqpoint{2.560452in}{1.812707in}}%
\pgfpathcurveto{\pgfqpoint{2.560452in}{1.823757in}}{\pgfqpoint{2.556062in}{1.834356in}}{\pgfqpoint{2.548249in}{1.842170in}}%
\pgfpathcurveto{\pgfqpoint{2.540435in}{1.849983in}}{\pgfqpoint{2.529836in}{1.854374in}}{\pgfqpoint{2.518786in}{1.854374in}}%
\pgfpathcurveto{\pgfqpoint{2.507736in}{1.854374in}}{\pgfqpoint{2.497137in}{1.849983in}}{\pgfqpoint{2.489323in}{1.842170in}}%
\pgfpathcurveto{\pgfqpoint{2.481509in}{1.834356in}}{\pgfqpoint{2.477119in}{1.823757in}}{\pgfqpoint{2.477119in}{1.812707in}}%
\pgfpathcurveto{\pgfqpoint{2.477119in}{1.801657in}}{\pgfqpoint{2.481509in}{1.791058in}}{\pgfqpoint{2.489323in}{1.783244in}}%
\pgfpathcurveto{\pgfqpoint{2.497137in}{1.775431in}}{\pgfqpoint{2.507736in}{1.771040in}}{\pgfqpoint{2.518786in}{1.771040in}}%
\pgfpathclose%
\pgfusepath{stroke,fill}%
\end{pgfscope}%
\begin{pgfscope}%
\pgfpathrectangle{\pgfqpoint{0.800000in}{0.528000in}}{\pgfqpoint{4.960000in}{3.696000in}}%
\pgfusepath{clip}%
\pgfsetbuttcap%
\pgfsetroundjoin%
\definecolor{currentfill}{rgb}{0.000000,0.000000,0.000000}%
\pgfsetfillcolor{currentfill}%
\pgfsetlinewidth{1.003750pt}%
\definecolor{currentstroke}{rgb}{0.000000,0.000000,0.000000}%
\pgfsetstrokecolor{currentstroke}%
\pgfsetdash{}{0pt}%
\pgfpathmoveto{\pgfqpoint{2.518786in}{1.771040in}}%
\pgfpathcurveto{\pgfqpoint{2.529836in}{1.771040in}}{\pgfqpoint{2.540435in}{1.775431in}}{\pgfqpoint{2.548249in}{1.783244in}}%
\pgfpathcurveto{\pgfqpoint{2.556062in}{1.791058in}}{\pgfqpoint{2.560452in}{1.801657in}}{\pgfqpoint{2.560452in}{1.812707in}}%
\pgfpathcurveto{\pgfqpoint{2.560452in}{1.823757in}}{\pgfqpoint{2.556062in}{1.834356in}}{\pgfqpoint{2.548249in}{1.842170in}}%
\pgfpathcurveto{\pgfqpoint{2.540435in}{1.849983in}}{\pgfqpoint{2.529836in}{1.854374in}}{\pgfqpoint{2.518786in}{1.854374in}}%
\pgfpathcurveto{\pgfqpoint{2.507736in}{1.854374in}}{\pgfqpoint{2.497137in}{1.849983in}}{\pgfqpoint{2.489323in}{1.842170in}}%
\pgfpathcurveto{\pgfqpoint{2.481509in}{1.834356in}}{\pgfqpoint{2.477119in}{1.823757in}}{\pgfqpoint{2.477119in}{1.812707in}}%
\pgfpathcurveto{\pgfqpoint{2.477119in}{1.801657in}}{\pgfqpoint{2.481509in}{1.791058in}}{\pgfqpoint{2.489323in}{1.783244in}}%
\pgfpathcurveto{\pgfqpoint{2.497137in}{1.775431in}}{\pgfqpoint{2.507736in}{1.771040in}}{\pgfqpoint{2.518786in}{1.771040in}}%
\pgfpathclose%
\pgfusepath{stroke,fill}%
\end{pgfscope}%
\begin{pgfscope}%
\pgfpathrectangle{\pgfqpoint{0.800000in}{0.528000in}}{\pgfqpoint{4.960000in}{3.696000in}}%
\pgfusepath{clip}%
\pgfsetbuttcap%
\pgfsetroundjoin%
\definecolor{currentfill}{rgb}{0.000000,0.000000,0.000000}%
\pgfsetfillcolor{currentfill}%
\pgfsetlinewidth{1.003750pt}%
\definecolor{currentstroke}{rgb}{0.000000,0.000000,0.000000}%
\pgfsetstrokecolor{currentstroke}%
\pgfsetdash{}{0pt}%
\pgfpathmoveto{\pgfqpoint{2.518786in}{1.771040in}}%
\pgfpathcurveto{\pgfqpoint{2.529836in}{1.771040in}}{\pgfqpoint{2.540435in}{1.775431in}}{\pgfqpoint{2.548249in}{1.783244in}}%
\pgfpathcurveto{\pgfqpoint{2.556062in}{1.791058in}}{\pgfqpoint{2.560452in}{1.801657in}}{\pgfqpoint{2.560452in}{1.812707in}}%
\pgfpathcurveto{\pgfqpoint{2.560452in}{1.823757in}}{\pgfqpoint{2.556062in}{1.834356in}}{\pgfqpoint{2.548249in}{1.842170in}}%
\pgfpathcurveto{\pgfqpoint{2.540435in}{1.849983in}}{\pgfqpoint{2.529836in}{1.854374in}}{\pgfqpoint{2.518786in}{1.854374in}}%
\pgfpathcurveto{\pgfqpoint{2.507736in}{1.854374in}}{\pgfqpoint{2.497137in}{1.849983in}}{\pgfqpoint{2.489323in}{1.842170in}}%
\pgfpathcurveto{\pgfqpoint{2.481509in}{1.834356in}}{\pgfqpoint{2.477119in}{1.823757in}}{\pgfqpoint{2.477119in}{1.812707in}}%
\pgfpathcurveto{\pgfqpoint{2.477119in}{1.801657in}}{\pgfqpoint{2.481509in}{1.791058in}}{\pgfqpoint{2.489323in}{1.783244in}}%
\pgfpathcurveto{\pgfqpoint{2.497137in}{1.775431in}}{\pgfqpoint{2.507736in}{1.771040in}}{\pgfqpoint{2.518786in}{1.771040in}}%
\pgfpathclose%
\pgfusepath{stroke,fill}%
\end{pgfscope}%
\begin{pgfscope}%
\pgfpathrectangle{\pgfqpoint{0.800000in}{0.528000in}}{\pgfqpoint{4.960000in}{3.696000in}}%
\pgfusepath{clip}%
\pgfsetbuttcap%
\pgfsetroundjoin%
\definecolor{currentfill}{rgb}{0.000000,0.000000,0.000000}%
\pgfsetfillcolor{currentfill}%
\pgfsetlinewidth{1.003750pt}%
\definecolor{currentstroke}{rgb}{0.000000,0.000000,0.000000}%
\pgfsetstrokecolor{currentstroke}%
\pgfsetdash{}{0pt}%
\pgfpathmoveto{\pgfqpoint{2.518786in}{1.771040in}}%
\pgfpathcurveto{\pgfqpoint{2.529836in}{1.771040in}}{\pgfqpoint{2.540435in}{1.775431in}}{\pgfqpoint{2.548249in}{1.783244in}}%
\pgfpathcurveto{\pgfqpoint{2.556062in}{1.791058in}}{\pgfqpoint{2.560452in}{1.801657in}}{\pgfqpoint{2.560452in}{1.812707in}}%
\pgfpathcurveto{\pgfqpoint{2.560452in}{1.823757in}}{\pgfqpoint{2.556062in}{1.834356in}}{\pgfqpoint{2.548249in}{1.842170in}}%
\pgfpathcurveto{\pgfqpoint{2.540435in}{1.849983in}}{\pgfqpoint{2.529836in}{1.854374in}}{\pgfqpoint{2.518786in}{1.854374in}}%
\pgfpathcurveto{\pgfqpoint{2.507736in}{1.854374in}}{\pgfqpoint{2.497137in}{1.849983in}}{\pgfqpoint{2.489323in}{1.842170in}}%
\pgfpathcurveto{\pgfqpoint{2.481509in}{1.834356in}}{\pgfqpoint{2.477119in}{1.823757in}}{\pgfqpoint{2.477119in}{1.812707in}}%
\pgfpathcurveto{\pgfqpoint{2.477119in}{1.801657in}}{\pgfqpoint{2.481509in}{1.791058in}}{\pgfqpoint{2.489323in}{1.783244in}}%
\pgfpathcurveto{\pgfqpoint{2.497137in}{1.775431in}}{\pgfqpoint{2.507736in}{1.771040in}}{\pgfqpoint{2.518786in}{1.771040in}}%
\pgfpathclose%
\pgfusepath{stroke,fill}%
\end{pgfscope}%
\begin{pgfscope}%
\pgfpathrectangle{\pgfqpoint{0.800000in}{0.528000in}}{\pgfqpoint{4.960000in}{3.696000in}}%
\pgfusepath{clip}%
\pgfsetbuttcap%
\pgfsetroundjoin%
\definecolor{currentfill}{rgb}{0.000000,0.000000,0.000000}%
\pgfsetfillcolor{currentfill}%
\pgfsetlinewidth{1.003750pt}%
\definecolor{currentstroke}{rgb}{0.000000,0.000000,0.000000}%
\pgfsetstrokecolor{currentstroke}%
\pgfsetdash{}{0pt}%
\pgfpathmoveto{\pgfqpoint{2.518786in}{1.771040in}}%
\pgfpathcurveto{\pgfqpoint{2.529836in}{1.771040in}}{\pgfqpoint{2.540435in}{1.775431in}}{\pgfqpoint{2.548249in}{1.783244in}}%
\pgfpathcurveto{\pgfqpoint{2.556062in}{1.791058in}}{\pgfqpoint{2.560452in}{1.801657in}}{\pgfqpoint{2.560452in}{1.812707in}}%
\pgfpathcurveto{\pgfqpoint{2.560452in}{1.823757in}}{\pgfqpoint{2.556062in}{1.834356in}}{\pgfqpoint{2.548249in}{1.842170in}}%
\pgfpathcurveto{\pgfqpoint{2.540435in}{1.849983in}}{\pgfqpoint{2.529836in}{1.854374in}}{\pgfqpoint{2.518786in}{1.854374in}}%
\pgfpathcurveto{\pgfqpoint{2.507736in}{1.854374in}}{\pgfqpoint{2.497137in}{1.849983in}}{\pgfqpoint{2.489323in}{1.842170in}}%
\pgfpathcurveto{\pgfqpoint{2.481509in}{1.834356in}}{\pgfqpoint{2.477119in}{1.823757in}}{\pgfqpoint{2.477119in}{1.812707in}}%
\pgfpathcurveto{\pgfqpoint{2.477119in}{1.801657in}}{\pgfqpoint{2.481509in}{1.791058in}}{\pgfqpoint{2.489323in}{1.783244in}}%
\pgfpathcurveto{\pgfqpoint{2.497137in}{1.775431in}}{\pgfqpoint{2.507736in}{1.771040in}}{\pgfqpoint{2.518786in}{1.771040in}}%
\pgfpathclose%
\pgfusepath{stroke,fill}%
\end{pgfscope}%
\begin{pgfscope}%
\pgfpathrectangle{\pgfqpoint{0.800000in}{0.528000in}}{\pgfqpoint{4.960000in}{3.696000in}}%
\pgfusepath{clip}%
\pgfsetbuttcap%
\pgfsetroundjoin%
\definecolor{currentfill}{rgb}{0.000000,0.000000,0.000000}%
\pgfsetfillcolor{currentfill}%
\pgfsetlinewidth{1.003750pt}%
\definecolor{currentstroke}{rgb}{0.000000,0.000000,0.000000}%
\pgfsetstrokecolor{currentstroke}%
\pgfsetdash{}{0pt}%
\pgfpathmoveto{\pgfqpoint{2.518786in}{1.771040in}}%
\pgfpathcurveto{\pgfqpoint{2.529836in}{1.771040in}}{\pgfqpoint{2.540435in}{1.775431in}}{\pgfqpoint{2.548249in}{1.783244in}}%
\pgfpathcurveto{\pgfqpoint{2.556062in}{1.791058in}}{\pgfqpoint{2.560452in}{1.801657in}}{\pgfqpoint{2.560452in}{1.812707in}}%
\pgfpathcurveto{\pgfqpoint{2.560452in}{1.823757in}}{\pgfqpoint{2.556062in}{1.834356in}}{\pgfqpoint{2.548249in}{1.842170in}}%
\pgfpathcurveto{\pgfqpoint{2.540435in}{1.849983in}}{\pgfqpoint{2.529836in}{1.854374in}}{\pgfqpoint{2.518786in}{1.854374in}}%
\pgfpathcurveto{\pgfqpoint{2.507736in}{1.854374in}}{\pgfqpoint{2.497137in}{1.849983in}}{\pgfqpoint{2.489323in}{1.842170in}}%
\pgfpathcurveto{\pgfqpoint{2.481509in}{1.834356in}}{\pgfqpoint{2.477119in}{1.823757in}}{\pgfqpoint{2.477119in}{1.812707in}}%
\pgfpathcurveto{\pgfqpoint{2.477119in}{1.801657in}}{\pgfqpoint{2.481509in}{1.791058in}}{\pgfqpoint{2.489323in}{1.783244in}}%
\pgfpathcurveto{\pgfqpoint{2.497137in}{1.775431in}}{\pgfqpoint{2.507736in}{1.771040in}}{\pgfqpoint{2.518786in}{1.771040in}}%
\pgfpathclose%
\pgfusepath{stroke,fill}%
\end{pgfscope}%
\begin{pgfscope}%
\pgfpathrectangle{\pgfqpoint{0.800000in}{0.528000in}}{\pgfqpoint{4.960000in}{3.696000in}}%
\pgfusepath{clip}%
\pgfsetbuttcap%
\pgfsetroundjoin%
\definecolor{currentfill}{rgb}{0.000000,0.000000,0.000000}%
\pgfsetfillcolor{currentfill}%
\pgfsetlinewidth{1.003750pt}%
\definecolor{currentstroke}{rgb}{0.000000,0.000000,0.000000}%
\pgfsetstrokecolor{currentstroke}%
\pgfsetdash{}{0pt}%
\pgfpathmoveto{\pgfqpoint{2.518786in}{1.771040in}}%
\pgfpathcurveto{\pgfqpoint{2.529836in}{1.771040in}}{\pgfqpoint{2.540435in}{1.775431in}}{\pgfqpoint{2.548249in}{1.783244in}}%
\pgfpathcurveto{\pgfqpoint{2.556062in}{1.791058in}}{\pgfqpoint{2.560452in}{1.801657in}}{\pgfqpoint{2.560452in}{1.812707in}}%
\pgfpathcurveto{\pgfqpoint{2.560452in}{1.823757in}}{\pgfqpoint{2.556062in}{1.834356in}}{\pgfqpoint{2.548249in}{1.842170in}}%
\pgfpathcurveto{\pgfqpoint{2.540435in}{1.849983in}}{\pgfqpoint{2.529836in}{1.854374in}}{\pgfqpoint{2.518786in}{1.854374in}}%
\pgfpathcurveto{\pgfqpoint{2.507736in}{1.854374in}}{\pgfqpoint{2.497137in}{1.849983in}}{\pgfqpoint{2.489323in}{1.842170in}}%
\pgfpathcurveto{\pgfqpoint{2.481509in}{1.834356in}}{\pgfqpoint{2.477119in}{1.823757in}}{\pgfqpoint{2.477119in}{1.812707in}}%
\pgfpathcurveto{\pgfqpoint{2.477119in}{1.801657in}}{\pgfqpoint{2.481509in}{1.791058in}}{\pgfqpoint{2.489323in}{1.783244in}}%
\pgfpathcurveto{\pgfqpoint{2.497137in}{1.775431in}}{\pgfqpoint{2.507736in}{1.771040in}}{\pgfqpoint{2.518786in}{1.771040in}}%
\pgfpathclose%
\pgfusepath{stroke,fill}%
\end{pgfscope}%
\begin{pgfscope}%
\pgfpathrectangle{\pgfqpoint{0.800000in}{0.528000in}}{\pgfqpoint{4.960000in}{3.696000in}}%
\pgfusepath{clip}%
\pgfsetbuttcap%
\pgfsetroundjoin%
\definecolor{currentfill}{rgb}{0.000000,0.000000,0.000000}%
\pgfsetfillcolor{currentfill}%
\pgfsetlinewidth{1.003750pt}%
\definecolor{currentstroke}{rgb}{0.000000,0.000000,0.000000}%
\pgfsetstrokecolor{currentstroke}%
\pgfsetdash{}{0pt}%
\pgfpathmoveto{\pgfqpoint{2.518786in}{1.771040in}}%
\pgfpathcurveto{\pgfqpoint{2.529836in}{1.771040in}}{\pgfqpoint{2.540435in}{1.775431in}}{\pgfqpoint{2.548249in}{1.783244in}}%
\pgfpathcurveto{\pgfqpoint{2.556062in}{1.791058in}}{\pgfqpoint{2.560452in}{1.801657in}}{\pgfqpoint{2.560452in}{1.812707in}}%
\pgfpathcurveto{\pgfqpoint{2.560452in}{1.823757in}}{\pgfqpoint{2.556062in}{1.834356in}}{\pgfqpoint{2.548249in}{1.842170in}}%
\pgfpathcurveto{\pgfqpoint{2.540435in}{1.849983in}}{\pgfqpoint{2.529836in}{1.854374in}}{\pgfqpoint{2.518786in}{1.854374in}}%
\pgfpathcurveto{\pgfqpoint{2.507736in}{1.854374in}}{\pgfqpoint{2.497137in}{1.849983in}}{\pgfqpoint{2.489323in}{1.842170in}}%
\pgfpathcurveto{\pgfqpoint{2.481509in}{1.834356in}}{\pgfqpoint{2.477119in}{1.823757in}}{\pgfqpoint{2.477119in}{1.812707in}}%
\pgfpathcurveto{\pgfqpoint{2.477119in}{1.801657in}}{\pgfqpoint{2.481509in}{1.791058in}}{\pgfqpoint{2.489323in}{1.783244in}}%
\pgfpathcurveto{\pgfqpoint{2.497137in}{1.775431in}}{\pgfqpoint{2.507736in}{1.771040in}}{\pgfqpoint{2.518786in}{1.771040in}}%
\pgfpathclose%
\pgfusepath{stroke,fill}%
\end{pgfscope}%
\begin{pgfscope}%
\pgfpathrectangle{\pgfqpoint{0.800000in}{0.528000in}}{\pgfqpoint{4.960000in}{3.696000in}}%
\pgfusepath{clip}%
\pgfsetbuttcap%
\pgfsetroundjoin%
\definecolor{currentfill}{rgb}{0.000000,0.000000,0.000000}%
\pgfsetfillcolor{currentfill}%
\pgfsetlinewidth{1.003750pt}%
\definecolor{currentstroke}{rgb}{0.000000,0.000000,0.000000}%
\pgfsetstrokecolor{currentstroke}%
\pgfsetdash{}{0pt}%
\pgfpathmoveto{\pgfqpoint{2.518786in}{1.771040in}}%
\pgfpathcurveto{\pgfqpoint{2.529836in}{1.771040in}}{\pgfqpoint{2.540435in}{1.775431in}}{\pgfqpoint{2.548249in}{1.783244in}}%
\pgfpathcurveto{\pgfqpoint{2.556062in}{1.791058in}}{\pgfqpoint{2.560452in}{1.801657in}}{\pgfqpoint{2.560452in}{1.812707in}}%
\pgfpathcurveto{\pgfqpoint{2.560452in}{1.823757in}}{\pgfqpoint{2.556062in}{1.834356in}}{\pgfqpoint{2.548249in}{1.842170in}}%
\pgfpathcurveto{\pgfqpoint{2.540435in}{1.849983in}}{\pgfqpoint{2.529836in}{1.854374in}}{\pgfqpoint{2.518786in}{1.854374in}}%
\pgfpathcurveto{\pgfqpoint{2.507736in}{1.854374in}}{\pgfqpoint{2.497137in}{1.849983in}}{\pgfqpoint{2.489323in}{1.842170in}}%
\pgfpathcurveto{\pgfqpoint{2.481509in}{1.834356in}}{\pgfqpoint{2.477119in}{1.823757in}}{\pgfqpoint{2.477119in}{1.812707in}}%
\pgfpathcurveto{\pgfqpoint{2.477119in}{1.801657in}}{\pgfqpoint{2.481509in}{1.791058in}}{\pgfqpoint{2.489323in}{1.783244in}}%
\pgfpathcurveto{\pgfqpoint{2.497137in}{1.775431in}}{\pgfqpoint{2.507736in}{1.771040in}}{\pgfqpoint{2.518786in}{1.771040in}}%
\pgfpathclose%
\pgfusepath{stroke,fill}%
\end{pgfscope}%
\begin{pgfscope}%
\pgfpathrectangle{\pgfqpoint{0.800000in}{0.528000in}}{\pgfqpoint{4.960000in}{3.696000in}}%
\pgfusepath{clip}%
\pgfsetbuttcap%
\pgfsetroundjoin%
\definecolor{currentfill}{rgb}{0.000000,0.000000,0.000000}%
\pgfsetfillcolor{currentfill}%
\pgfsetlinewidth{1.003750pt}%
\definecolor{currentstroke}{rgb}{0.000000,0.000000,0.000000}%
\pgfsetstrokecolor{currentstroke}%
\pgfsetdash{}{0pt}%
\pgfpathmoveto{\pgfqpoint{2.518786in}{1.771040in}}%
\pgfpathcurveto{\pgfqpoint{2.529836in}{1.771040in}}{\pgfqpoint{2.540435in}{1.775431in}}{\pgfqpoint{2.548249in}{1.783244in}}%
\pgfpathcurveto{\pgfqpoint{2.556062in}{1.791058in}}{\pgfqpoint{2.560452in}{1.801657in}}{\pgfqpoint{2.560452in}{1.812707in}}%
\pgfpathcurveto{\pgfqpoint{2.560452in}{1.823757in}}{\pgfqpoint{2.556062in}{1.834356in}}{\pgfqpoint{2.548249in}{1.842170in}}%
\pgfpathcurveto{\pgfqpoint{2.540435in}{1.849983in}}{\pgfqpoint{2.529836in}{1.854374in}}{\pgfqpoint{2.518786in}{1.854374in}}%
\pgfpathcurveto{\pgfqpoint{2.507736in}{1.854374in}}{\pgfqpoint{2.497137in}{1.849983in}}{\pgfqpoint{2.489323in}{1.842170in}}%
\pgfpathcurveto{\pgfqpoint{2.481509in}{1.834356in}}{\pgfqpoint{2.477119in}{1.823757in}}{\pgfqpoint{2.477119in}{1.812707in}}%
\pgfpathcurveto{\pgfqpoint{2.477119in}{1.801657in}}{\pgfqpoint{2.481509in}{1.791058in}}{\pgfqpoint{2.489323in}{1.783244in}}%
\pgfpathcurveto{\pgfqpoint{2.497137in}{1.775431in}}{\pgfqpoint{2.507736in}{1.771040in}}{\pgfqpoint{2.518786in}{1.771040in}}%
\pgfpathclose%
\pgfusepath{stroke,fill}%
\end{pgfscope}%
\begin{pgfscope}%
\pgfpathrectangle{\pgfqpoint{0.800000in}{0.528000in}}{\pgfqpoint{4.960000in}{3.696000in}}%
\pgfusepath{clip}%
\pgfsetbuttcap%
\pgfsetroundjoin%
\definecolor{currentfill}{rgb}{0.000000,0.000000,0.000000}%
\pgfsetfillcolor{currentfill}%
\pgfsetlinewidth{1.003750pt}%
\definecolor{currentstroke}{rgb}{0.000000,0.000000,0.000000}%
\pgfsetstrokecolor{currentstroke}%
\pgfsetdash{}{0pt}%
\pgfpathmoveto{\pgfqpoint{2.518786in}{1.771040in}}%
\pgfpathcurveto{\pgfqpoint{2.529836in}{1.771040in}}{\pgfqpoint{2.540435in}{1.775431in}}{\pgfqpoint{2.548249in}{1.783244in}}%
\pgfpathcurveto{\pgfqpoint{2.556062in}{1.791058in}}{\pgfqpoint{2.560452in}{1.801657in}}{\pgfqpoint{2.560452in}{1.812707in}}%
\pgfpathcurveto{\pgfqpoint{2.560452in}{1.823757in}}{\pgfqpoint{2.556062in}{1.834356in}}{\pgfqpoint{2.548249in}{1.842170in}}%
\pgfpathcurveto{\pgfqpoint{2.540435in}{1.849983in}}{\pgfqpoint{2.529836in}{1.854374in}}{\pgfqpoint{2.518786in}{1.854374in}}%
\pgfpathcurveto{\pgfqpoint{2.507736in}{1.854374in}}{\pgfqpoint{2.497137in}{1.849983in}}{\pgfqpoint{2.489323in}{1.842170in}}%
\pgfpathcurveto{\pgfqpoint{2.481509in}{1.834356in}}{\pgfqpoint{2.477119in}{1.823757in}}{\pgfqpoint{2.477119in}{1.812707in}}%
\pgfpathcurveto{\pgfqpoint{2.477119in}{1.801657in}}{\pgfqpoint{2.481509in}{1.791058in}}{\pgfqpoint{2.489323in}{1.783244in}}%
\pgfpathcurveto{\pgfqpoint{2.497137in}{1.775431in}}{\pgfqpoint{2.507736in}{1.771040in}}{\pgfqpoint{2.518786in}{1.771040in}}%
\pgfpathclose%
\pgfusepath{stroke,fill}%
\end{pgfscope}%
\begin{pgfscope}%
\pgfpathrectangle{\pgfqpoint{0.800000in}{0.528000in}}{\pgfqpoint{4.960000in}{3.696000in}}%
\pgfusepath{clip}%
\pgfsetbuttcap%
\pgfsetroundjoin%
\definecolor{currentfill}{rgb}{0.000000,0.000000,0.000000}%
\pgfsetfillcolor{currentfill}%
\pgfsetlinewidth{1.003750pt}%
\definecolor{currentstroke}{rgb}{0.000000,0.000000,0.000000}%
\pgfsetstrokecolor{currentstroke}%
\pgfsetdash{}{0pt}%
\pgfpathmoveto{\pgfqpoint{2.518786in}{1.771040in}}%
\pgfpathcurveto{\pgfqpoint{2.529836in}{1.771040in}}{\pgfqpoint{2.540435in}{1.775431in}}{\pgfqpoint{2.548249in}{1.783244in}}%
\pgfpathcurveto{\pgfqpoint{2.556062in}{1.791058in}}{\pgfqpoint{2.560452in}{1.801657in}}{\pgfqpoint{2.560452in}{1.812707in}}%
\pgfpathcurveto{\pgfqpoint{2.560452in}{1.823757in}}{\pgfqpoint{2.556062in}{1.834356in}}{\pgfqpoint{2.548249in}{1.842170in}}%
\pgfpathcurveto{\pgfqpoint{2.540435in}{1.849983in}}{\pgfqpoint{2.529836in}{1.854374in}}{\pgfqpoint{2.518786in}{1.854374in}}%
\pgfpathcurveto{\pgfqpoint{2.507736in}{1.854374in}}{\pgfqpoint{2.497137in}{1.849983in}}{\pgfqpoint{2.489323in}{1.842170in}}%
\pgfpathcurveto{\pgfqpoint{2.481509in}{1.834356in}}{\pgfqpoint{2.477119in}{1.823757in}}{\pgfqpoint{2.477119in}{1.812707in}}%
\pgfpathcurveto{\pgfqpoint{2.477119in}{1.801657in}}{\pgfqpoint{2.481509in}{1.791058in}}{\pgfqpoint{2.489323in}{1.783244in}}%
\pgfpathcurveto{\pgfqpoint{2.497137in}{1.775431in}}{\pgfqpoint{2.507736in}{1.771040in}}{\pgfqpoint{2.518786in}{1.771040in}}%
\pgfpathclose%
\pgfusepath{stroke,fill}%
\end{pgfscope}%
\begin{pgfscope}%
\pgfpathrectangle{\pgfqpoint{0.800000in}{0.528000in}}{\pgfqpoint{4.960000in}{3.696000in}}%
\pgfusepath{clip}%
\pgfsetbuttcap%
\pgfsetroundjoin%
\definecolor{currentfill}{rgb}{0.000000,0.000000,0.000000}%
\pgfsetfillcolor{currentfill}%
\pgfsetlinewidth{1.003750pt}%
\definecolor{currentstroke}{rgb}{0.000000,0.000000,0.000000}%
\pgfsetstrokecolor{currentstroke}%
\pgfsetdash{}{0pt}%
\pgfpathmoveto{\pgfqpoint{2.518786in}{1.771040in}}%
\pgfpathcurveto{\pgfqpoint{2.529836in}{1.771040in}}{\pgfqpoint{2.540435in}{1.775431in}}{\pgfqpoint{2.548249in}{1.783244in}}%
\pgfpathcurveto{\pgfqpoint{2.556062in}{1.791058in}}{\pgfqpoint{2.560452in}{1.801657in}}{\pgfqpoint{2.560452in}{1.812707in}}%
\pgfpathcurveto{\pgfqpoint{2.560452in}{1.823757in}}{\pgfqpoint{2.556062in}{1.834356in}}{\pgfqpoint{2.548249in}{1.842170in}}%
\pgfpathcurveto{\pgfqpoint{2.540435in}{1.849983in}}{\pgfqpoint{2.529836in}{1.854374in}}{\pgfqpoint{2.518786in}{1.854374in}}%
\pgfpathcurveto{\pgfqpoint{2.507736in}{1.854374in}}{\pgfqpoint{2.497137in}{1.849983in}}{\pgfqpoint{2.489323in}{1.842170in}}%
\pgfpathcurveto{\pgfqpoint{2.481509in}{1.834356in}}{\pgfqpoint{2.477119in}{1.823757in}}{\pgfqpoint{2.477119in}{1.812707in}}%
\pgfpathcurveto{\pgfqpoint{2.477119in}{1.801657in}}{\pgfqpoint{2.481509in}{1.791058in}}{\pgfqpoint{2.489323in}{1.783244in}}%
\pgfpathcurveto{\pgfqpoint{2.497137in}{1.775431in}}{\pgfqpoint{2.507736in}{1.771040in}}{\pgfqpoint{2.518786in}{1.771040in}}%
\pgfpathclose%
\pgfusepath{stroke,fill}%
\end{pgfscope}%
\begin{pgfscope}%
\pgfpathrectangle{\pgfqpoint{0.800000in}{0.528000in}}{\pgfqpoint{4.960000in}{3.696000in}}%
\pgfusepath{clip}%
\pgfsetbuttcap%
\pgfsetroundjoin%
\definecolor{currentfill}{rgb}{0.000000,0.000000,0.000000}%
\pgfsetfillcolor{currentfill}%
\pgfsetlinewidth{1.003750pt}%
\definecolor{currentstroke}{rgb}{0.000000,0.000000,0.000000}%
\pgfsetstrokecolor{currentstroke}%
\pgfsetdash{}{0pt}%
\pgfpathmoveto{\pgfqpoint{2.518786in}{1.771040in}}%
\pgfpathcurveto{\pgfqpoint{2.529836in}{1.771040in}}{\pgfqpoint{2.540435in}{1.775431in}}{\pgfqpoint{2.548249in}{1.783244in}}%
\pgfpathcurveto{\pgfqpoint{2.556062in}{1.791058in}}{\pgfqpoint{2.560452in}{1.801657in}}{\pgfqpoint{2.560452in}{1.812707in}}%
\pgfpathcurveto{\pgfqpoint{2.560452in}{1.823757in}}{\pgfqpoint{2.556062in}{1.834356in}}{\pgfqpoint{2.548249in}{1.842170in}}%
\pgfpathcurveto{\pgfqpoint{2.540435in}{1.849983in}}{\pgfqpoint{2.529836in}{1.854374in}}{\pgfqpoint{2.518786in}{1.854374in}}%
\pgfpathcurveto{\pgfqpoint{2.507736in}{1.854374in}}{\pgfqpoint{2.497137in}{1.849983in}}{\pgfqpoint{2.489323in}{1.842170in}}%
\pgfpathcurveto{\pgfqpoint{2.481509in}{1.834356in}}{\pgfqpoint{2.477119in}{1.823757in}}{\pgfqpoint{2.477119in}{1.812707in}}%
\pgfpathcurveto{\pgfqpoint{2.477119in}{1.801657in}}{\pgfqpoint{2.481509in}{1.791058in}}{\pgfqpoint{2.489323in}{1.783244in}}%
\pgfpathcurveto{\pgfqpoint{2.497137in}{1.775431in}}{\pgfqpoint{2.507736in}{1.771040in}}{\pgfqpoint{2.518786in}{1.771040in}}%
\pgfpathclose%
\pgfusepath{stroke,fill}%
\end{pgfscope}%
\begin{pgfscope}%
\pgfpathrectangle{\pgfqpoint{0.800000in}{0.528000in}}{\pgfqpoint{4.960000in}{3.696000in}}%
\pgfusepath{clip}%
\pgfsetbuttcap%
\pgfsetroundjoin%
\definecolor{currentfill}{rgb}{0.000000,0.000000,0.000000}%
\pgfsetfillcolor{currentfill}%
\pgfsetlinewidth{1.003750pt}%
\definecolor{currentstroke}{rgb}{0.000000,0.000000,0.000000}%
\pgfsetstrokecolor{currentstroke}%
\pgfsetdash{}{0pt}%
\pgfpathmoveto{\pgfqpoint{2.518786in}{1.771040in}}%
\pgfpathcurveto{\pgfqpoint{2.529836in}{1.771040in}}{\pgfqpoint{2.540435in}{1.775431in}}{\pgfqpoint{2.548249in}{1.783244in}}%
\pgfpathcurveto{\pgfqpoint{2.556062in}{1.791058in}}{\pgfqpoint{2.560452in}{1.801657in}}{\pgfqpoint{2.560452in}{1.812707in}}%
\pgfpathcurveto{\pgfqpoint{2.560452in}{1.823757in}}{\pgfqpoint{2.556062in}{1.834356in}}{\pgfqpoint{2.548249in}{1.842170in}}%
\pgfpathcurveto{\pgfqpoint{2.540435in}{1.849983in}}{\pgfqpoint{2.529836in}{1.854374in}}{\pgfqpoint{2.518786in}{1.854374in}}%
\pgfpathcurveto{\pgfqpoint{2.507736in}{1.854374in}}{\pgfqpoint{2.497137in}{1.849983in}}{\pgfqpoint{2.489323in}{1.842170in}}%
\pgfpathcurveto{\pgfqpoint{2.481509in}{1.834356in}}{\pgfqpoint{2.477119in}{1.823757in}}{\pgfqpoint{2.477119in}{1.812707in}}%
\pgfpathcurveto{\pgfqpoint{2.477119in}{1.801657in}}{\pgfqpoint{2.481509in}{1.791058in}}{\pgfqpoint{2.489323in}{1.783244in}}%
\pgfpathcurveto{\pgfqpoint{2.497137in}{1.775431in}}{\pgfqpoint{2.507736in}{1.771040in}}{\pgfqpoint{2.518786in}{1.771040in}}%
\pgfpathclose%
\pgfusepath{stroke,fill}%
\end{pgfscope}%
\begin{pgfscope}%
\pgfpathrectangle{\pgfqpoint{0.800000in}{0.528000in}}{\pgfqpoint{4.960000in}{3.696000in}}%
\pgfusepath{clip}%
\pgfsetbuttcap%
\pgfsetroundjoin%
\definecolor{currentfill}{rgb}{0.000000,0.000000,0.000000}%
\pgfsetfillcolor{currentfill}%
\pgfsetlinewidth{1.003750pt}%
\definecolor{currentstroke}{rgb}{0.000000,0.000000,0.000000}%
\pgfsetstrokecolor{currentstroke}%
\pgfsetdash{}{0pt}%
\pgfpathmoveto{\pgfqpoint{2.518786in}{1.771040in}}%
\pgfpathcurveto{\pgfqpoint{2.529836in}{1.771040in}}{\pgfqpoint{2.540435in}{1.775431in}}{\pgfqpoint{2.548249in}{1.783244in}}%
\pgfpathcurveto{\pgfqpoint{2.556062in}{1.791058in}}{\pgfqpoint{2.560452in}{1.801657in}}{\pgfqpoint{2.560452in}{1.812707in}}%
\pgfpathcurveto{\pgfqpoint{2.560452in}{1.823757in}}{\pgfqpoint{2.556062in}{1.834356in}}{\pgfqpoint{2.548249in}{1.842170in}}%
\pgfpathcurveto{\pgfqpoint{2.540435in}{1.849983in}}{\pgfqpoint{2.529836in}{1.854374in}}{\pgfqpoint{2.518786in}{1.854374in}}%
\pgfpathcurveto{\pgfqpoint{2.507736in}{1.854374in}}{\pgfqpoint{2.497137in}{1.849983in}}{\pgfqpoint{2.489323in}{1.842170in}}%
\pgfpathcurveto{\pgfqpoint{2.481509in}{1.834356in}}{\pgfqpoint{2.477119in}{1.823757in}}{\pgfqpoint{2.477119in}{1.812707in}}%
\pgfpathcurveto{\pgfqpoint{2.477119in}{1.801657in}}{\pgfqpoint{2.481509in}{1.791058in}}{\pgfqpoint{2.489323in}{1.783244in}}%
\pgfpathcurveto{\pgfqpoint{2.497137in}{1.775431in}}{\pgfqpoint{2.507736in}{1.771040in}}{\pgfqpoint{2.518786in}{1.771040in}}%
\pgfpathclose%
\pgfusepath{stroke,fill}%
\end{pgfscope}%
\begin{pgfscope}%
\pgfpathrectangle{\pgfqpoint{0.800000in}{0.528000in}}{\pgfqpoint{4.960000in}{3.696000in}}%
\pgfusepath{clip}%
\pgfsetbuttcap%
\pgfsetroundjoin%
\definecolor{currentfill}{rgb}{0.000000,0.000000,0.000000}%
\pgfsetfillcolor{currentfill}%
\pgfsetlinewidth{1.003750pt}%
\definecolor{currentstroke}{rgb}{0.000000,0.000000,0.000000}%
\pgfsetstrokecolor{currentstroke}%
\pgfsetdash{}{0pt}%
\pgfpathmoveto{\pgfqpoint{2.518786in}{1.771040in}}%
\pgfpathcurveto{\pgfqpoint{2.529836in}{1.771040in}}{\pgfqpoint{2.540435in}{1.775431in}}{\pgfqpoint{2.548249in}{1.783244in}}%
\pgfpathcurveto{\pgfqpoint{2.556062in}{1.791058in}}{\pgfqpoint{2.560452in}{1.801657in}}{\pgfqpoint{2.560452in}{1.812707in}}%
\pgfpathcurveto{\pgfqpoint{2.560452in}{1.823757in}}{\pgfqpoint{2.556062in}{1.834356in}}{\pgfqpoint{2.548249in}{1.842170in}}%
\pgfpathcurveto{\pgfqpoint{2.540435in}{1.849983in}}{\pgfqpoint{2.529836in}{1.854374in}}{\pgfqpoint{2.518786in}{1.854374in}}%
\pgfpathcurveto{\pgfqpoint{2.507736in}{1.854374in}}{\pgfqpoint{2.497137in}{1.849983in}}{\pgfqpoint{2.489323in}{1.842170in}}%
\pgfpathcurveto{\pgfqpoint{2.481509in}{1.834356in}}{\pgfqpoint{2.477119in}{1.823757in}}{\pgfqpoint{2.477119in}{1.812707in}}%
\pgfpathcurveto{\pgfqpoint{2.477119in}{1.801657in}}{\pgfqpoint{2.481509in}{1.791058in}}{\pgfqpoint{2.489323in}{1.783244in}}%
\pgfpathcurveto{\pgfqpoint{2.497137in}{1.775431in}}{\pgfqpoint{2.507736in}{1.771040in}}{\pgfqpoint{2.518786in}{1.771040in}}%
\pgfpathclose%
\pgfusepath{stroke,fill}%
\end{pgfscope}%
\begin{pgfscope}%
\pgfpathrectangle{\pgfqpoint{0.800000in}{0.528000in}}{\pgfqpoint{4.960000in}{3.696000in}}%
\pgfusepath{clip}%
\pgfsetbuttcap%
\pgfsetroundjoin%
\definecolor{currentfill}{rgb}{0.000000,0.000000,0.000000}%
\pgfsetfillcolor{currentfill}%
\pgfsetlinewidth{1.003750pt}%
\definecolor{currentstroke}{rgb}{0.000000,0.000000,0.000000}%
\pgfsetstrokecolor{currentstroke}%
\pgfsetdash{}{0pt}%
\pgfpathmoveto{\pgfqpoint{2.518786in}{1.771040in}}%
\pgfpathcurveto{\pgfqpoint{2.529836in}{1.771040in}}{\pgfqpoint{2.540435in}{1.775431in}}{\pgfqpoint{2.548249in}{1.783244in}}%
\pgfpathcurveto{\pgfqpoint{2.556062in}{1.791058in}}{\pgfqpoint{2.560452in}{1.801657in}}{\pgfqpoint{2.560452in}{1.812707in}}%
\pgfpathcurveto{\pgfqpoint{2.560452in}{1.823757in}}{\pgfqpoint{2.556062in}{1.834356in}}{\pgfqpoint{2.548249in}{1.842170in}}%
\pgfpathcurveto{\pgfqpoint{2.540435in}{1.849983in}}{\pgfqpoint{2.529836in}{1.854374in}}{\pgfqpoint{2.518786in}{1.854374in}}%
\pgfpathcurveto{\pgfqpoint{2.507736in}{1.854374in}}{\pgfqpoint{2.497137in}{1.849983in}}{\pgfqpoint{2.489323in}{1.842170in}}%
\pgfpathcurveto{\pgfqpoint{2.481509in}{1.834356in}}{\pgfqpoint{2.477119in}{1.823757in}}{\pgfqpoint{2.477119in}{1.812707in}}%
\pgfpathcurveto{\pgfqpoint{2.477119in}{1.801657in}}{\pgfqpoint{2.481509in}{1.791058in}}{\pgfqpoint{2.489323in}{1.783244in}}%
\pgfpathcurveto{\pgfqpoint{2.497137in}{1.775431in}}{\pgfqpoint{2.507736in}{1.771040in}}{\pgfqpoint{2.518786in}{1.771040in}}%
\pgfpathclose%
\pgfusepath{stroke,fill}%
\end{pgfscope}%
\begin{pgfscope}%
\pgfpathrectangle{\pgfqpoint{0.800000in}{0.528000in}}{\pgfqpoint{4.960000in}{3.696000in}}%
\pgfusepath{clip}%
\pgfsetbuttcap%
\pgfsetroundjoin%
\definecolor{currentfill}{rgb}{0.000000,0.000000,0.000000}%
\pgfsetfillcolor{currentfill}%
\pgfsetlinewidth{1.003750pt}%
\definecolor{currentstroke}{rgb}{0.000000,0.000000,0.000000}%
\pgfsetstrokecolor{currentstroke}%
\pgfsetdash{}{0pt}%
\pgfpathmoveto{\pgfqpoint{2.518786in}{1.771040in}}%
\pgfpathcurveto{\pgfqpoint{2.529836in}{1.771040in}}{\pgfqpoint{2.540435in}{1.775431in}}{\pgfqpoint{2.548249in}{1.783244in}}%
\pgfpathcurveto{\pgfqpoint{2.556062in}{1.791058in}}{\pgfqpoint{2.560452in}{1.801657in}}{\pgfqpoint{2.560452in}{1.812707in}}%
\pgfpathcurveto{\pgfqpoint{2.560452in}{1.823757in}}{\pgfqpoint{2.556062in}{1.834356in}}{\pgfqpoint{2.548249in}{1.842170in}}%
\pgfpathcurveto{\pgfqpoint{2.540435in}{1.849983in}}{\pgfqpoint{2.529836in}{1.854374in}}{\pgfqpoint{2.518786in}{1.854374in}}%
\pgfpathcurveto{\pgfqpoint{2.507736in}{1.854374in}}{\pgfqpoint{2.497137in}{1.849983in}}{\pgfqpoint{2.489323in}{1.842170in}}%
\pgfpathcurveto{\pgfqpoint{2.481509in}{1.834356in}}{\pgfqpoint{2.477119in}{1.823757in}}{\pgfqpoint{2.477119in}{1.812707in}}%
\pgfpathcurveto{\pgfqpoint{2.477119in}{1.801657in}}{\pgfqpoint{2.481509in}{1.791058in}}{\pgfqpoint{2.489323in}{1.783244in}}%
\pgfpathcurveto{\pgfqpoint{2.497137in}{1.775431in}}{\pgfqpoint{2.507736in}{1.771040in}}{\pgfqpoint{2.518786in}{1.771040in}}%
\pgfpathclose%
\pgfusepath{stroke,fill}%
\end{pgfscope}%
\begin{pgfscope}%
\pgfpathrectangle{\pgfqpoint{0.800000in}{0.528000in}}{\pgfqpoint{4.960000in}{3.696000in}}%
\pgfusepath{clip}%
\pgfsetbuttcap%
\pgfsetroundjoin%
\definecolor{currentfill}{rgb}{0.000000,0.000000,0.000000}%
\pgfsetfillcolor{currentfill}%
\pgfsetlinewidth{1.003750pt}%
\definecolor{currentstroke}{rgb}{0.000000,0.000000,0.000000}%
\pgfsetstrokecolor{currentstroke}%
\pgfsetdash{}{0pt}%
\pgfpathmoveto{\pgfqpoint{2.518786in}{1.771040in}}%
\pgfpathcurveto{\pgfqpoint{2.529836in}{1.771040in}}{\pgfqpoint{2.540435in}{1.775431in}}{\pgfqpoint{2.548249in}{1.783244in}}%
\pgfpathcurveto{\pgfqpoint{2.556062in}{1.791058in}}{\pgfqpoint{2.560452in}{1.801657in}}{\pgfqpoint{2.560452in}{1.812707in}}%
\pgfpathcurveto{\pgfqpoint{2.560452in}{1.823757in}}{\pgfqpoint{2.556062in}{1.834356in}}{\pgfqpoint{2.548249in}{1.842170in}}%
\pgfpathcurveto{\pgfqpoint{2.540435in}{1.849983in}}{\pgfqpoint{2.529836in}{1.854374in}}{\pgfqpoint{2.518786in}{1.854374in}}%
\pgfpathcurveto{\pgfqpoint{2.507736in}{1.854374in}}{\pgfqpoint{2.497137in}{1.849983in}}{\pgfqpoint{2.489323in}{1.842170in}}%
\pgfpathcurveto{\pgfqpoint{2.481509in}{1.834356in}}{\pgfqpoint{2.477119in}{1.823757in}}{\pgfqpoint{2.477119in}{1.812707in}}%
\pgfpathcurveto{\pgfqpoint{2.477119in}{1.801657in}}{\pgfqpoint{2.481509in}{1.791058in}}{\pgfqpoint{2.489323in}{1.783244in}}%
\pgfpathcurveto{\pgfqpoint{2.497137in}{1.775431in}}{\pgfqpoint{2.507736in}{1.771040in}}{\pgfqpoint{2.518786in}{1.771040in}}%
\pgfpathclose%
\pgfusepath{stroke,fill}%
\end{pgfscope}%
\begin{pgfscope}%
\pgfpathrectangle{\pgfqpoint{0.800000in}{0.528000in}}{\pgfqpoint{4.960000in}{3.696000in}}%
\pgfusepath{clip}%
\pgfsetbuttcap%
\pgfsetroundjoin%
\definecolor{currentfill}{rgb}{0.000000,0.000000,0.000000}%
\pgfsetfillcolor{currentfill}%
\pgfsetlinewidth{1.003750pt}%
\definecolor{currentstroke}{rgb}{0.000000,0.000000,0.000000}%
\pgfsetstrokecolor{currentstroke}%
\pgfsetdash{}{0pt}%
\pgfpathmoveto{\pgfqpoint{2.518786in}{1.771040in}}%
\pgfpathcurveto{\pgfqpoint{2.529836in}{1.771040in}}{\pgfqpoint{2.540435in}{1.775431in}}{\pgfqpoint{2.548249in}{1.783244in}}%
\pgfpathcurveto{\pgfqpoint{2.556062in}{1.791058in}}{\pgfqpoint{2.560452in}{1.801657in}}{\pgfqpoint{2.560452in}{1.812707in}}%
\pgfpathcurveto{\pgfqpoint{2.560452in}{1.823757in}}{\pgfqpoint{2.556062in}{1.834356in}}{\pgfqpoint{2.548249in}{1.842170in}}%
\pgfpathcurveto{\pgfqpoint{2.540435in}{1.849983in}}{\pgfqpoint{2.529836in}{1.854374in}}{\pgfqpoint{2.518786in}{1.854374in}}%
\pgfpathcurveto{\pgfqpoint{2.507736in}{1.854374in}}{\pgfqpoint{2.497137in}{1.849983in}}{\pgfqpoint{2.489323in}{1.842170in}}%
\pgfpathcurveto{\pgfqpoint{2.481509in}{1.834356in}}{\pgfqpoint{2.477119in}{1.823757in}}{\pgfqpoint{2.477119in}{1.812707in}}%
\pgfpathcurveto{\pgfqpoint{2.477119in}{1.801657in}}{\pgfqpoint{2.481509in}{1.791058in}}{\pgfqpoint{2.489323in}{1.783244in}}%
\pgfpathcurveto{\pgfqpoint{2.497137in}{1.775431in}}{\pgfqpoint{2.507736in}{1.771040in}}{\pgfqpoint{2.518786in}{1.771040in}}%
\pgfpathclose%
\pgfusepath{stroke,fill}%
\end{pgfscope}%
\begin{pgfscope}%
\pgfpathrectangle{\pgfqpoint{0.800000in}{0.528000in}}{\pgfqpoint{4.960000in}{3.696000in}}%
\pgfusepath{clip}%
\pgfsetbuttcap%
\pgfsetroundjoin%
\definecolor{currentfill}{rgb}{0.000000,0.000000,0.000000}%
\pgfsetfillcolor{currentfill}%
\pgfsetlinewidth{1.003750pt}%
\definecolor{currentstroke}{rgb}{0.000000,0.000000,0.000000}%
\pgfsetstrokecolor{currentstroke}%
\pgfsetdash{}{0pt}%
\pgfpathmoveto{\pgfqpoint{2.518786in}{1.771040in}}%
\pgfpathcurveto{\pgfqpoint{2.529836in}{1.771040in}}{\pgfqpoint{2.540435in}{1.775431in}}{\pgfqpoint{2.548249in}{1.783244in}}%
\pgfpathcurveto{\pgfqpoint{2.556062in}{1.791058in}}{\pgfqpoint{2.560452in}{1.801657in}}{\pgfqpoint{2.560452in}{1.812707in}}%
\pgfpathcurveto{\pgfqpoint{2.560452in}{1.823757in}}{\pgfqpoint{2.556062in}{1.834356in}}{\pgfqpoint{2.548249in}{1.842170in}}%
\pgfpathcurveto{\pgfqpoint{2.540435in}{1.849983in}}{\pgfqpoint{2.529836in}{1.854374in}}{\pgfqpoint{2.518786in}{1.854374in}}%
\pgfpathcurveto{\pgfqpoint{2.507736in}{1.854374in}}{\pgfqpoint{2.497137in}{1.849983in}}{\pgfqpoint{2.489323in}{1.842170in}}%
\pgfpathcurveto{\pgfqpoint{2.481509in}{1.834356in}}{\pgfqpoint{2.477119in}{1.823757in}}{\pgfqpoint{2.477119in}{1.812707in}}%
\pgfpathcurveto{\pgfqpoint{2.477119in}{1.801657in}}{\pgfqpoint{2.481509in}{1.791058in}}{\pgfqpoint{2.489323in}{1.783244in}}%
\pgfpathcurveto{\pgfqpoint{2.497137in}{1.775431in}}{\pgfqpoint{2.507736in}{1.771040in}}{\pgfqpoint{2.518786in}{1.771040in}}%
\pgfpathclose%
\pgfusepath{stroke,fill}%
\end{pgfscope}%
\begin{pgfscope}%
\pgfpathrectangle{\pgfqpoint{0.800000in}{0.528000in}}{\pgfqpoint{4.960000in}{3.696000in}}%
\pgfusepath{clip}%
\pgfsetbuttcap%
\pgfsetroundjoin%
\definecolor{currentfill}{rgb}{0.000000,0.000000,0.000000}%
\pgfsetfillcolor{currentfill}%
\pgfsetlinewidth{1.003750pt}%
\definecolor{currentstroke}{rgb}{0.000000,0.000000,0.000000}%
\pgfsetstrokecolor{currentstroke}%
\pgfsetdash{}{0pt}%
\pgfpathmoveto{\pgfqpoint{2.518786in}{1.771040in}}%
\pgfpathcurveto{\pgfqpoint{2.529836in}{1.771040in}}{\pgfqpoint{2.540435in}{1.775431in}}{\pgfqpoint{2.548249in}{1.783244in}}%
\pgfpathcurveto{\pgfqpoint{2.556062in}{1.791058in}}{\pgfqpoint{2.560452in}{1.801657in}}{\pgfqpoint{2.560452in}{1.812707in}}%
\pgfpathcurveto{\pgfqpoint{2.560452in}{1.823757in}}{\pgfqpoint{2.556062in}{1.834356in}}{\pgfqpoint{2.548249in}{1.842170in}}%
\pgfpathcurveto{\pgfqpoint{2.540435in}{1.849983in}}{\pgfqpoint{2.529836in}{1.854374in}}{\pgfqpoint{2.518786in}{1.854374in}}%
\pgfpathcurveto{\pgfqpoint{2.507736in}{1.854374in}}{\pgfqpoint{2.497137in}{1.849983in}}{\pgfqpoint{2.489323in}{1.842170in}}%
\pgfpathcurveto{\pgfqpoint{2.481509in}{1.834356in}}{\pgfqpoint{2.477119in}{1.823757in}}{\pgfqpoint{2.477119in}{1.812707in}}%
\pgfpathcurveto{\pgfqpoint{2.477119in}{1.801657in}}{\pgfqpoint{2.481509in}{1.791058in}}{\pgfqpoint{2.489323in}{1.783244in}}%
\pgfpathcurveto{\pgfqpoint{2.497137in}{1.775431in}}{\pgfqpoint{2.507736in}{1.771040in}}{\pgfqpoint{2.518786in}{1.771040in}}%
\pgfpathclose%
\pgfusepath{stroke,fill}%
\end{pgfscope}%
\begin{pgfscope}%
\pgfpathrectangle{\pgfqpoint{0.800000in}{0.528000in}}{\pgfqpoint{4.960000in}{3.696000in}}%
\pgfusepath{clip}%
\pgfsetbuttcap%
\pgfsetroundjoin%
\definecolor{currentfill}{rgb}{0.000000,0.000000,0.000000}%
\pgfsetfillcolor{currentfill}%
\pgfsetlinewidth{1.003750pt}%
\definecolor{currentstroke}{rgb}{0.000000,0.000000,0.000000}%
\pgfsetstrokecolor{currentstroke}%
\pgfsetdash{}{0pt}%
\pgfpathmoveto{\pgfqpoint{2.518786in}{1.771040in}}%
\pgfpathcurveto{\pgfqpoint{2.529836in}{1.771040in}}{\pgfqpoint{2.540435in}{1.775431in}}{\pgfqpoint{2.548249in}{1.783244in}}%
\pgfpathcurveto{\pgfqpoint{2.556062in}{1.791058in}}{\pgfqpoint{2.560452in}{1.801657in}}{\pgfqpoint{2.560452in}{1.812707in}}%
\pgfpathcurveto{\pgfqpoint{2.560452in}{1.823757in}}{\pgfqpoint{2.556062in}{1.834356in}}{\pgfqpoint{2.548249in}{1.842170in}}%
\pgfpathcurveto{\pgfqpoint{2.540435in}{1.849983in}}{\pgfqpoint{2.529836in}{1.854374in}}{\pgfqpoint{2.518786in}{1.854374in}}%
\pgfpathcurveto{\pgfqpoint{2.507736in}{1.854374in}}{\pgfqpoint{2.497137in}{1.849983in}}{\pgfqpoint{2.489323in}{1.842170in}}%
\pgfpathcurveto{\pgfqpoint{2.481509in}{1.834356in}}{\pgfqpoint{2.477119in}{1.823757in}}{\pgfqpoint{2.477119in}{1.812707in}}%
\pgfpathcurveto{\pgfqpoint{2.477119in}{1.801657in}}{\pgfqpoint{2.481509in}{1.791058in}}{\pgfqpoint{2.489323in}{1.783244in}}%
\pgfpathcurveto{\pgfqpoint{2.497137in}{1.775431in}}{\pgfqpoint{2.507736in}{1.771040in}}{\pgfqpoint{2.518786in}{1.771040in}}%
\pgfpathclose%
\pgfusepath{stroke,fill}%
\end{pgfscope}%
\begin{pgfscope}%
\pgfpathrectangle{\pgfqpoint{0.800000in}{0.528000in}}{\pgfqpoint{4.960000in}{3.696000in}}%
\pgfusepath{clip}%
\pgfsetbuttcap%
\pgfsetroundjoin%
\definecolor{currentfill}{rgb}{0.000000,0.000000,0.000000}%
\pgfsetfillcolor{currentfill}%
\pgfsetlinewidth{1.003750pt}%
\definecolor{currentstroke}{rgb}{0.000000,0.000000,0.000000}%
\pgfsetstrokecolor{currentstroke}%
\pgfsetdash{}{0pt}%
\pgfpathmoveto{\pgfqpoint{2.518786in}{1.771040in}}%
\pgfpathcurveto{\pgfqpoint{2.529836in}{1.771040in}}{\pgfqpoint{2.540435in}{1.775431in}}{\pgfqpoint{2.548249in}{1.783244in}}%
\pgfpathcurveto{\pgfqpoint{2.556062in}{1.791058in}}{\pgfqpoint{2.560452in}{1.801657in}}{\pgfqpoint{2.560452in}{1.812707in}}%
\pgfpathcurveto{\pgfqpoint{2.560452in}{1.823757in}}{\pgfqpoint{2.556062in}{1.834356in}}{\pgfqpoint{2.548249in}{1.842170in}}%
\pgfpathcurveto{\pgfqpoint{2.540435in}{1.849983in}}{\pgfqpoint{2.529836in}{1.854374in}}{\pgfqpoint{2.518786in}{1.854374in}}%
\pgfpathcurveto{\pgfqpoint{2.507736in}{1.854374in}}{\pgfqpoint{2.497137in}{1.849983in}}{\pgfqpoint{2.489323in}{1.842170in}}%
\pgfpathcurveto{\pgfqpoint{2.481509in}{1.834356in}}{\pgfqpoint{2.477119in}{1.823757in}}{\pgfqpoint{2.477119in}{1.812707in}}%
\pgfpathcurveto{\pgfqpoint{2.477119in}{1.801657in}}{\pgfqpoint{2.481509in}{1.791058in}}{\pgfqpoint{2.489323in}{1.783244in}}%
\pgfpathcurveto{\pgfqpoint{2.497137in}{1.775431in}}{\pgfqpoint{2.507736in}{1.771040in}}{\pgfqpoint{2.518786in}{1.771040in}}%
\pgfpathclose%
\pgfusepath{stroke,fill}%
\end{pgfscope}%
\begin{pgfscope}%
\pgfpathrectangle{\pgfqpoint{0.800000in}{0.528000in}}{\pgfqpoint{4.960000in}{3.696000in}}%
\pgfusepath{clip}%
\pgfsetbuttcap%
\pgfsetroundjoin%
\definecolor{currentfill}{rgb}{0.000000,0.000000,0.000000}%
\pgfsetfillcolor{currentfill}%
\pgfsetlinewidth{1.003750pt}%
\definecolor{currentstroke}{rgb}{0.000000,0.000000,0.000000}%
\pgfsetstrokecolor{currentstroke}%
\pgfsetdash{}{0pt}%
\pgfpathmoveto{\pgfqpoint{2.518786in}{1.771040in}}%
\pgfpathcurveto{\pgfqpoint{2.529836in}{1.771040in}}{\pgfqpoint{2.540435in}{1.775431in}}{\pgfqpoint{2.548249in}{1.783244in}}%
\pgfpathcurveto{\pgfqpoint{2.556062in}{1.791058in}}{\pgfqpoint{2.560452in}{1.801657in}}{\pgfqpoint{2.560452in}{1.812707in}}%
\pgfpathcurveto{\pgfqpoint{2.560452in}{1.823757in}}{\pgfqpoint{2.556062in}{1.834356in}}{\pgfqpoint{2.548249in}{1.842170in}}%
\pgfpathcurveto{\pgfqpoint{2.540435in}{1.849983in}}{\pgfqpoint{2.529836in}{1.854374in}}{\pgfqpoint{2.518786in}{1.854374in}}%
\pgfpathcurveto{\pgfqpoint{2.507736in}{1.854374in}}{\pgfqpoint{2.497137in}{1.849983in}}{\pgfqpoint{2.489323in}{1.842170in}}%
\pgfpathcurveto{\pgfqpoint{2.481509in}{1.834356in}}{\pgfqpoint{2.477119in}{1.823757in}}{\pgfqpoint{2.477119in}{1.812707in}}%
\pgfpathcurveto{\pgfqpoint{2.477119in}{1.801657in}}{\pgfqpoint{2.481509in}{1.791058in}}{\pgfqpoint{2.489323in}{1.783244in}}%
\pgfpathcurveto{\pgfqpoint{2.497137in}{1.775431in}}{\pgfqpoint{2.507736in}{1.771040in}}{\pgfqpoint{2.518786in}{1.771040in}}%
\pgfpathclose%
\pgfusepath{stroke,fill}%
\end{pgfscope}%
\begin{pgfscope}%
\pgfpathrectangle{\pgfqpoint{0.800000in}{0.528000in}}{\pgfqpoint{4.960000in}{3.696000in}}%
\pgfusepath{clip}%
\pgfsetbuttcap%
\pgfsetroundjoin%
\definecolor{currentfill}{rgb}{0.000000,0.000000,0.000000}%
\pgfsetfillcolor{currentfill}%
\pgfsetlinewidth{1.003750pt}%
\definecolor{currentstroke}{rgb}{0.000000,0.000000,0.000000}%
\pgfsetstrokecolor{currentstroke}%
\pgfsetdash{}{0pt}%
\pgfpathmoveto{\pgfqpoint{2.518786in}{1.771040in}}%
\pgfpathcurveto{\pgfqpoint{2.529836in}{1.771040in}}{\pgfqpoint{2.540435in}{1.775431in}}{\pgfqpoint{2.548249in}{1.783244in}}%
\pgfpathcurveto{\pgfqpoint{2.556062in}{1.791058in}}{\pgfqpoint{2.560452in}{1.801657in}}{\pgfqpoint{2.560452in}{1.812707in}}%
\pgfpathcurveto{\pgfqpoint{2.560452in}{1.823757in}}{\pgfqpoint{2.556062in}{1.834356in}}{\pgfqpoint{2.548249in}{1.842170in}}%
\pgfpathcurveto{\pgfqpoint{2.540435in}{1.849983in}}{\pgfqpoint{2.529836in}{1.854374in}}{\pgfqpoint{2.518786in}{1.854374in}}%
\pgfpathcurveto{\pgfqpoint{2.507736in}{1.854374in}}{\pgfqpoint{2.497137in}{1.849983in}}{\pgfqpoint{2.489323in}{1.842170in}}%
\pgfpathcurveto{\pgfqpoint{2.481509in}{1.834356in}}{\pgfqpoint{2.477119in}{1.823757in}}{\pgfqpoint{2.477119in}{1.812707in}}%
\pgfpathcurveto{\pgfqpoint{2.477119in}{1.801657in}}{\pgfqpoint{2.481509in}{1.791058in}}{\pgfqpoint{2.489323in}{1.783244in}}%
\pgfpathcurveto{\pgfqpoint{2.497137in}{1.775431in}}{\pgfqpoint{2.507736in}{1.771040in}}{\pgfqpoint{2.518786in}{1.771040in}}%
\pgfpathclose%
\pgfusepath{stroke,fill}%
\end{pgfscope}%
\begin{pgfscope}%
\pgfpathrectangle{\pgfqpoint{0.800000in}{0.528000in}}{\pgfqpoint{4.960000in}{3.696000in}}%
\pgfusepath{clip}%
\pgfsetbuttcap%
\pgfsetroundjoin%
\definecolor{currentfill}{rgb}{0.000000,0.000000,0.000000}%
\pgfsetfillcolor{currentfill}%
\pgfsetlinewidth{1.003750pt}%
\definecolor{currentstroke}{rgb}{0.000000,0.000000,0.000000}%
\pgfsetstrokecolor{currentstroke}%
\pgfsetdash{}{0pt}%
\pgfpathmoveto{\pgfqpoint{2.518786in}{1.771040in}}%
\pgfpathcurveto{\pgfqpoint{2.529836in}{1.771040in}}{\pgfqpoint{2.540435in}{1.775431in}}{\pgfqpoint{2.548249in}{1.783244in}}%
\pgfpathcurveto{\pgfqpoint{2.556062in}{1.791058in}}{\pgfqpoint{2.560452in}{1.801657in}}{\pgfqpoint{2.560452in}{1.812707in}}%
\pgfpathcurveto{\pgfqpoint{2.560452in}{1.823757in}}{\pgfqpoint{2.556062in}{1.834356in}}{\pgfqpoint{2.548249in}{1.842170in}}%
\pgfpathcurveto{\pgfqpoint{2.540435in}{1.849983in}}{\pgfqpoint{2.529836in}{1.854374in}}{\pgfqpoint{2.518786in}{1.854374in}}%
\pgfpathcurveto{\pgfqpoint{2.507736in}{1.854374in}}{\pgfqpoint{2.497137in}{1.849983in}}{\pgfqpoint{2.489323in}{1.842170in}}%
\pgfpathcurveto{\pgfqpoint{2.481509in}{1.834356in}}{\pgfqpoint{2.477119in}{1.823757in}}{\pgfqpoint{2.477119in}{1.812707in}}%
\pgfpathcurveto{\pgfqpoint{2.477119in}{1.801657in}}{\pgfqpoint{2.481509in}{1.791058in}}{\pgfqpoint{2.489323in}{1.783244in}}%
\pgfpathcurveto{\pgfqpoint{2.497137in}{1.775431in}}{\pgfqpoint{2.507736in}{1.771040in}}{\pgfqpoint{2.518786in}{1.771040in}}%
\pgfpathclose%
\pgfusepath{stroke,fill}%
\end{pgfscope}%
\begin{pgfscope}%
\pgfpathrectangle{\pgfqpoint{0.800000in}{0.528000in}}{\pgfqpoint{4.960000in}{3.696000in}}%
\pgfusepath{clip}%
\pgfsetbuttcap%
\pgfsetroundjoin%
\definecolor{currentfill}{rgb}{0.000000,0.000000,0.000000}%
\pgfsetfillcolor{currentfill}%
\pgfsetlinewidth{1.003750pt}%
\definecolor{currentstroke}{rgb}{0.000000,0.000000,0.000000}%
\pgfsetstrokecolor{currentstroke}%
\pgfsetdash{}{0pt}%
\pgfpathmoveto{\pgfqpoint{2.518786in}{1.771040in}}%
\pgfpathcurveto{\pgfqpoint{2.529836in}{1.771040in}}{\pgfqpoint{2.540435in}{1.775431in}}{\pgfqpoint{2.548249in}{1.783244in}}%
\pgfpathcurveto{\pgfqpoint{2.556062in}{1.791058in}}{\pgfqpoint{2.560452in}{1.801657in}}{\pgfqpoint{2.560452in}{1.812707in}}%
\pgfpathcurveto{\pgfqpoint{2.560452in}{1.823757in}}{\pgfqpoint{2.556062in}{1.834356in}}{\pgfqpoint{2.548249in}{1.842170in}}%
\pgfpathcurveto{\pgfqpoint{2.540435in}{1.849983in}}{\pgfqpoint{2.529836in}{1.854374in}}{\pgfqpoint{2.518786in}{1.854374in}}%
\pgfpathcurveto{\pgfqpoint{2.507736in}{1.854374in}}{\pgfqpoint{2.497137in}{1.849983in}}{\pgfqpoint{2.489323in}{1.842170in}}%
\pgfpathcurveto{\pgfqpoint{2.481509in}{1.834356in}}{\pgfqpoint{2.477119in}{1.823757in}}{\pgfqpoint{2.477119in}{1.812707in}}%
\pgfpathcurveto{\pgfqpoint{2.477119in}{1.801657in}}{\pgfqpoint{2.481509in}{1.791058in}}{\pgfqpoint{2.489323in}{1.783244in}}%
\pgfpathcurveto{\pgfqpoint{2.497137in}{1.775431in}}{\pgfqpoint{2.507736in}{1.771040in}}{\pgfqpoint{2.518786in}{1.771040in}}%
\pgfpathclose%
\pgfusepath{stroke,fill}%
\end{pgfscope}%
\begin{pgfscope}%
\pgfpathrectangle{\pgfqpoint{0.800000in}{0.528000in}}{\pgfqpoint{4.960000in}{3.696000in}}%
\pgfusepath{clip}%
\pgfsetbuttcap%
\pgfsetroundjoin%
\definecolor{currentfill}{rgb}{0.000000,0.000000,0.000000}%
\pgfsetfillcolor{currentfill}%
\pgfsetlinewidth{1.003750pt}%
\definecolor{currentstroke}{rgb}{0.000000,0.000000,0.000000}%
\pgfsetstrokecolor{currentstroke}%
\pgfsetdash{}{0pt}%
\pgfpathmoveto{\pgfqpoint{2.518786in}{0.664394in}}%
\pgfpathcurveto{\pgfqpoint{2.529836in}{0.664394in}}{\pgfqpoint{2.540435in}{0.668784in}}{\pgfqpoint{2.548249in}{0.676598in}}%
\pgfpathcurveto{\pgfqpoint{2.556062in}{0.684411in}}{\pgfqpoint{2.560452in}{0.695010in}}{\pgfqpoint{2.560452in}{0.706060in}}%
\pgfpathcurveto{\pgfqpoint{2.560452in}{0.717111in}}{\pgfqpoint{2.556062in}{0.727710in}}{\pgfqpoint{2.548249in}{0.735523in}}%
\pgfpathcurveto{\pgfqpoint{2.540435in}{0.743337in}}{\pgfqpoint{2.529836in}{0.747727in}}{\pgfqpoint{2.518786in}{0.747727in}}%
\pgfpathcurveto{\pgfqpoint{2.507736in}{0.747727in}}{\pgfqpoint{2.497137in}{0.743337in}}{\pgfqpoint{2.489323in}{0.735523in}}%
\pgfpathcurveto{\pgfqpoint{2.481509in}{0.727710in}}{\pgfqpoint{2.477119in}{0.717111in}}{\pgfqpoint{2.477119in}{0.706060in}}%
\pgfpathcurveto{\pgfqpoint{2.477119in}{0.695010in}}{\pgfqpoint{2.481509in}{0.684411in}}{\pgfqpoint{2.489323in}{0.676598in}}%
\pgfpathcurveto{\pgfqpoint{2.497137in}{0.668784in}}{\pgfqpoint{2.507736in}{0.664394in}}{\pgfqpoint{2.518786in}{0.664394in}}%
\pgfpathclose%
\pgfusepath{stroke,fill}%
\end{pgfscope}%
\begin{pgfscope}%
\pgfpathrectangle{\pgfqpoint{0.800000in}{0.528000in}}{\pgfqpoint{4.960000in}{3.696000in}}%
\pgfusepath{clip}%
\pgfsetbuttcap%
\pgfsetroundjoin%
\definecolor{currentfill}{rgb}{0.000000,0.000000,0.000000}%
\pgfsetfillcolor{currentfill}%
\pgfsetlinewidth{1.003750pt}%
\definecolor{currentstroke}{rgb}{0.000000,0.000000,0.000000}%
\pgfsetstrokecolor{currentstroke}%
\pgfsetdash{}{0pt}%
\pgfpathmoveto{\pgfqpoint{2.518786in}{1.771040in}}%
\pgfpathcurveto{\pgfqpoint{2.529836in}{1.771040in}}{\pgfqpoint{2.540435in}{1.775431in}}{\pgfqpoint{2.548249in}{1.783244in}}%
\pgfpathcurveto{\pgfqpoint{2.556062in}{1.791058in}}{\pgfqpoint{2.560452in}{1.801657in}}{\pgfqpoint{2.560452in}{1.812707in}}%
\pgfpathcurveto{\pgfqpoint{2.560452in}{1.823757in}}{\pgfqpoint{2.556062in}{1.834356in}}{\pgfqpoint{2.548249in}{1.842170in}}%
\pgfpathcurveto{\pgfqpoint{2.540435in}{1.849983in}}{\pgfqpoint{2.529836in}{1.854374in}}{\pgfqpoint{2.518786in}{1.854374in}}%
\pgfpathcurveto{\pgfqpoint{2.507736in}{1.854374in}}{\pgfqpoint{2.497137in}{1.849983in}}{\pgfqpoint{2.489323in}{1.842170in}}%
\pgfpathcurveto{\pgfqpoint{2.481509in}{1.834356in}}{\pgfqpoint{2.477119in}{1.823757in}}{\pgfqpoint{2.477119in}{1.812707in}}%
\pgfpathcurveto{\pgfqpoint{2.477119in}{1.801657in}}{\pgfqpoint{2.481509in}{1.791058in}}{\pgfqpoint{2.489323in}{1.783244in}}%
\pgfpathcurveto{\pgfqpoint{2.497137in}{1.775431in}}{\pgfqpoint{2.507736in}{1.771040in}}{\pgfqpoint{2.518786in}{1.771040in}}%
\pgfpathclose%
\pgfusepath{stroke,fill}%
\end{pgfscope}%
\begin{pgfscope}%
\pgfpathrectangle{\pgfqpoint{0.800000in}{0.528000in}}{\pgfqpoint{4.960000in}{3.696000in}}%
\pgfusepath{clip}%
\pgfsetbuttcap%
\pgfsetroundjoin%
\definecolor{currentfill}{rgb}{0.000000,0.000000,0.000000}%
\pgfsetfillcolor{currentfill}%
\pgfsetlinewidth{1.003750pt}%
\definecolor{currentstroke}{rgb}{0.000000,0.000000,0.000000}%
\pgfsetstrokecolor{currentstroke}%
\pgfsetdash{}{0pt}%
\pgfpathmoveto{\pgfqpoint{2.518786in}{1.771040in}}%
\pgfpathcurveto{\pgfqpoint{2.529836in}{1.771040in}}{\pgfqpoint{2.540435in}{1.775431in}}{\pgfqpoint{2.548249in}{1.783244in}}%
\pgfpathcurveto{\pgfqpoint{2.556062in}{1.791058in}}{\pgfqpoint{2.560452in}{1.801657in}}{\pgfqpoint{2.560452in}{1.812707in}}%
\pgfpathcurveto{\pgfqpoint{2.560452in}{1.823757in}}{\pgfqpoint{2.556062in}{1.834356in}}{\pgfqpoint{2.548249in}{1.842170in}}%
\pgfpathcurveto{\pgfqpoint{2.540435in}{1.849983in}}{\pgfqpoint{2.529836in}{1.854374in}}{\pgfqpoint{2.518786in}{1.854374in}}%
\pgfpathcurveto{\pgfqpoint{2.507736in}{1.854374in}}{\pgfqpoint{2.497137in}{1.849983in}}{\pgfqpoint{2.489323in}{1.842170in}}%
\pgfpathcurveto{\pgfqpoint{2.481509in}{1.834356in}}{\pgfqpoint{2.477119in}{1.823757in}}{\pgfqpoint{2.477119in}{1.812707in}}%
\pgfpathcurveto{\pgfqpoint{2.477119in}{1.801657in}}{\pgfqpoint{2.481509in}{1.791058in}}{\pgfqpoint{2.489323in}{1.783244in}}%
\pgfpathcurveto{\pgfqpoint{2.497137in}{1.775431in}}{\pgfqpoint{2.507736in}{1.771040in}}{\pgfqpoint{2.518786in}{1.771040in}}%
\pgfpathclose%
\pgfusepath{stroke,fill}%
\end{pgfscope}%
\begin{pgfscope}%
\pgfpathrectangle{\pgfqpoint{0.800000in}{0.528000in}}{\pgfqpoint{4.960000in}{3.696000in}}%
\pgfusepath{clip}%
\pgfsetbuttcap%
\pgfsetroundjoin%
\definecolor{currentfill}{rgb}{0.000000,0.000000,0.000000}%
\pgfsetfillcolor{currentfill}%
\pgfsetlinewidth{1.003750pt}%
\definecolor{currentstroke}{rgb}{0.000000,0.000000,0.000000}%
\pgfsetstrokecolor{currentstroke}%
\pgfsetdash{}{0pt}%
\pgfpathmoveto{\pgfqpoint{2.518786in}{1.771040in}}%
\pgfpathcurveto{\pgfqpoint{2.529836in}{1.771040in}}{\pgfqpoint{2.540435in}{1.775431in}}{\pgfqpoint{2.548249in}{1.783244in}}%
\pgfpathcurveto{\pgfqpoint{2.556062in}{1.791058in}}{\pgfqpoint{2.560452in}{1.801657in}}{\pgfqpoint{2.560452in}{1.812707in}}%
\pgfpathcurveto{\pgfqpoint{2.560452in}{1.823757in}}{\pgfqpoint{2.556062in}{1.834356in}}{\pgfqpoint{2.548249in}{1.842170in}}%
\pgfpathcurveto{\pgfqpoint{2.540435in}{1.849983in}}{\pgfqpoint{2.529836in}{1.854374in}}{\pgfqpoint{2.518786in}{1.854374in}}%
\pgfpathcurveto{\pgfqpoint{2.507736in}{1.854374in}}{\pgfqpoint{2.497137in}{1.849983in}}{\pgfqpoint{2.489323in}{1.842170in}}%
\pgfpathcurveto{\pgfqpoint{2.481509in}{1.834356in}}{\pgfqpoint{2.477119in}{1.823757in}}{\pgfqpoint{2.477119in}{1.812707in}}%
\pgfpathcurveto{\pgfqpoint{2.477119in}{1.801657in}}{\pgfqpoint{2.481509in}{1.791058in}}{\pgfqpoint{2.489323in}{1.783244in}}%
\pgfpathcurveto{\pgfqpoint{2.497137in}{1.775431in}}{\pgfqpoint{2.507736in}{1.771040in}}{\pgfqpoint{2.518786in}{1.771040in}}%
\pgfpathclose%
\pgfusepath{stroke,fill}%
\end{pgfscope}%
\begin{pgfscope}%
\pgfpathrectangle{\pgfqpoint{0.800000in}{0.528000in}}{\pgfqpoint{4.960000in}{3.696000in}}%
\pgfusepath{clip}%
\pgfsetbuttcap%
\pgfsetroundjoin%
\definecolor{currentfill}{rgb}{0.000000,0.000000,0.000000}%
\pgfsetfillcolor{currentfill}%
\pgfsetlinewidth{1.003750pt}%
\definecolor{currentstroke}{rgb}{0.000000,0.000000,0.000000}%
\pgfsetstrokecolor{currentstroke}%
\pgfsetdash{}{0pt}%
\pgfpathmoveto{\pgfqpoint{2.518786in}{1.771040in}}%
\pgfpathcurveto{\pgfqpoint{2.529836in}{1.771040in}}{\pgfqpoint{2.540435in}{1.775431in}}{\pgfqpoint{2.548249in}{1.783244in}}%
\pgfpathcurveto{\pgfqpoint{2.556062in}{1.791058in}}{\pgfqpoint{2.560452in}{1.801657in}}{\pgfqpoint{2.560452in}{1.812707in}}%
\pgfpathcurveto{\pgfqpoint{2.560452in}{1.823757in}}{\pgfqpoint{2.556062in}{1.834356in}}{\pgfqpoint{2.548249in}{1.842170in}}%
\pgfpathcurveto{\pgfqpoint{2.540435in}{1.849983in}}{\pgfqpoint{2.529836in}{1.854374in}}{\pgfqpoint{2.518786in}{1.854374in}}%
\pgfpathcurveto{\pgfqpoint{2.507736in}{1.854374in}}{\pgfqpoint{2.497137in}{1.849983in}}{\pgfqpoint{2.489323in}{1.842170in}}%
\pgfpathcurveto{\pgfqpoint{2.481509in}{1.834356in}}{\pgfqpoint{2.477119in}{1.823757in}}{\pgfqpoint{2.477119in}{1.812707in}}%
\pgfpathcurveto{\pgfqpoint{2.477119in}{1.801657in}}{\pgfqpoint{2.481509in}{1.791058in}}{\pgfqpoint{2.489323in}{1.783244in}}%
\pgfpathcurveto{\pgfqpoint{2.497137in}{1.775431in}}{\pgfqpoint{2.507736in}{1.771040in}}{\pgfqpoint{2.518786in}{1.771040in}}%
\pgfpathclose%
\pgfusepath{stroke,fill}%
\end{pgfscope}%
\begin{pgfscope}%
\pgfpathrectangle{\pgfqpoint{0.800000in}{0.528000in}}{\pgfqpoint{4.960000in}{3.696000in}}%
\pgfusepath{clip}%
\pgfsetbuttcap%
\pgfsetroundjoin%
\definecolor{currentfill}{rgb}{0.000000,0.000000,0.000000}%
\pgfsetfillcolor{currentfill}%
\pgfsetlinewidth{1.003750pt}%
\definecolor{currentstroke}{rgb}{0.000000,0.000000,0.000000}%
\pgfsetstrokecolor{currentstroke}%
\pgfsetdash{}{0pt}%
\pgfpathmoveto{\pgfqpoint{2.518786in}{1.771040in}}%
\pgfpathcurveto{\pgfqpoint{2.529836in}{1.771040in}}{\pgfqpoint{2.540435in}{1.775431in}}{\pgfqpoint{2.548249in}{1.783244in}}%
\pgfpathcurveto{\pgfqpoint{2.556062in}{1.791058in}}{\pgfqpoint{2.560452in}{1.801657in}}{\pgfqpoint{2.560452in}{1.812707in}}%
\pgfpathcurveto{\pgfqpoint{2.560452in}{1.823757in}}{\pgfqpoint{2.556062in}{1.834356in}}{\pgfqpoint{2.548249in}{1.842170in}}%
\pgfpathcurveto{\pgfqpoint{2.540435in}{1.849983in}}{\pgfqpoint{2.529836in}{1.854374in}}{\pgfqpoint{2.518786in}{1.854374in}}%
\pgfpathcurveto{\pgfqpoint{2.507736in}{1.854374in}}{\pgfqpoint{2.497137in}{1.849983in}}{\pgfqpoint{2.489323in}{1.842170in}}%
\pgfpathcurveto{\pgfqpoint{2.481509in}{1.834356in}}{\pgfqpoint{2.477119in}{1.823757in}}{\pgfqpoint{2.477119in}{1.812707in}}%
\pgfpathcurveto{\pgfqpoint{2.477119in}{1.801657in}}{\pgfqpoint{2.481509in}{1.791058in}}{\pgfqpoint{2.489323in}{1.783244in}}%
\pgfpathcurveto{\pgfqpoint{2.497137in}{1.775431in}}{\pgfqpoint{2.507736in}{1.771040in}}{\pgfqpoint{2.518786in}{1.771040in}}%
\pgfpathclose%
\pgfusepath{stroke,fill}%
\end{pgfscope}%
\begin{pgfscope}%
\pgfpathrectangle{\pgfqpoint{0.800000in}{0.528000in}}{\pgfqpoint{4.960000in}{3.696000in}}%
\pgfusepath{clip}%
\pgfsetbuttcap%
\pgfsetroundjoin%
\definecolor{currentfill}{rgb}{0.000000,0.000000,0.000000}%
\pgfsetfillcolor{currentfill}%
\pgfsetlinewidth{1.003750pt}%
\definecolor{currentstroke}{rgb}{0.000000,0.000000,0.000000}%
\pgfsetstrokecolor{currentstroke}%
\pgfsetdash{}{0pt}%
\pgfpathmoveto{\pgfqpoint{2.518786in}{1.771040in}}%
\pgfpathcurveto{\pgfqpoint{2.529836in}{1.771040in}}{\pgfqpoint{2.540435in}{1.775431in}}{\pgfqpoint{2.548249in}{1.783244in}}%
\pgfpathcurveto{\pgfqpoint{2.556062in}{1.791058in}}{\pgfqpoint{2.560452in}{1.801657in}}{\pgfqpoint{2.560452in}{1.812707in}}%
\pgfpathcurveto{\pgfqpoint{2.560452in}{1.823757in}}{\pgfqpoint{2.556062in}{1.834356in}}{\pgfqpoint{2.548249in}{1.842170in}}%
\pgfpathcurveto{\pgfqpoint{2.540435in}{1.849983in}}{\pgfqpoint{2.529836in}{1.854374in}}{\pgfqpoint{2.518786in}{1.854374in}}%
\pgfpathcurveto{\pgfqpoint{2.507736in}{1.854374in}}{\pgfqpoint{2.497137in}{1.849983in}}{\pgfqpoint{2.489323in}{1.842170in}}%
\pgfpathcurveto{\pgfqpoint{2.481509in}{1.834356in}}{\pgfqpoint{2.477119in}{1.823757in}}{\pgfqpoint{2.477119in}{1.812707in}}%
\pgfpathcurveto{\pgfqpoint{2.477119in}{1.801657in}}{\pgfqpoint{2.481509in}{1.791058in}}{\pgfqpoint{2.489323in}{1.783244in}}%
\pgfpathcurveto{\pgfqpoint{2.497137in}{1.775431in}}{\pgfqpoint{2.507736in}{1.771040in}}{\pgfqpoint{2.518786in}{1.771040in}}%
\pgfpathclose%
\pgfusepath{stroke,fill}%
\end{pgfscope}%
\begin{pgfscope}%
\pgfpathrectangle{\pgfqpoint{0.800000in}{0.528000in}}{\pgfqpoint{4.960000in}{3.696000in}}%
\pgfusepath{clip}%
\pgfsetbuttcap%
\pgfsetroundjoin%
\definecolor{currentfill}{rgb}{0.000000,0.000000,0.000000}%
\pgfsetfillcolor{currentfill}%
\pgfsetlinewidth{1.003750pt}%
\definecolor{currentstroke}{rgb}{0.000000,0.000000,0.000000}%
\pgfsetstrokecolor{currentstroke}%
\pgfsetdash{}{0pt}%
\pgfpathmoveto{\pgfqpoint{2.518786in}{1.771040in}}%
\pgfpathcurveto{\pgfqpoint{2.529836in}{1.771040in}}{\pgfqpoint{2.540435in}{1.775431in}}{\pgfqpoint{2.548249in}{1.783244in}}%
\pgfpathcurveto{\pgfqpoint{2.556062in}{1.791058in}}{\pgfqpoint{2.560452in}{1.801657in}}{\pgfqpoint{2.560452in}{1.812707in}}%
\pgfpathcurveto{\pgfqpoint{2.560452in}{1.823757in}}{\pgfqpoint{2.556062in}{1.834356in}}{\pgfqpoint{2.548249in}{1.842170in}}%
\pgfpathcurveto{\pgfqpoint{2.540435in}{1.849983in}}{\pgfqpoint{2.529836in}{1.854374in}}{\pgfqpoint{2.518786in}{1.854374in}}%
\pgfpathcurveto{\pgfqpoint{2.507736in}{1.854374in}}{\pgfqpoint{2.497137in}{1.849983in}}{\pgfqpoint{2.489323in}{1.842170in}}%
\pgfpathcurveto{\pgfqpoint{2.481509in}{1.834356in}}{\pgfqpoint{2.477119in}{1.823757in}}{\pgfqpoint{2.477119in}{1.812707in}}%
\pgfpathcurveto{\pgfqpoint{2.477119in}{1.801657in}}{\pgfqpoint{2.481509in}{1.791058in}}{\pgfqpoint{2.489323in}{1.783244in}}%
\pgfpathcurveto{\pgfqpoint{2.497137in}{1.775431in}}{\pgfqpoint{2.507736in}{1.771040in}}{\pgfqpoint{2.518786in}{1.771040in}}%
\pgfpathclose%
\pgfusepath{stroke,fill}%
\end{pgfscope}%
\begin{pgfscope}%
\pgfpathrectangle{\pgfqpoint{0.800000in}{0.528000in}}{\pgfqpoint{4.960000in}{3.696000in}}%
\pgfusepath{clip}%
\pgfsetbuttcap%
\pgfsetroundjoin%
\definecolor{currentfill}{rgb}{0.000000,0.000000,0.000000}%
\pgfsetfillcolor{currentfill}%
\pgfsetlinewidth{1.003750pt}%
\definecolor{currentstroke}{rgb}{0.000000,0.000000,0.000000}%
\pgfsetstrokecolor{currentstroke}%
\pgfsetdash{}{0pt}%
\pgfpathmoveto{\pgfqpoint{2.518786in}{1.771040in}}%
\pgfpathcurveto{\pgfqpoint{2.529836in}{1.771040in}}{\pgfqpoint{2.540435in}{1.775431in}}{\pgfqpoint{2.548249in}{1.783244in}}%
\pgfpathcurveto{\pgfqpoint{2.556062in}{1.791058in}}{\pgfqpoint{2.560452in}{1.801657in}}{\pgfqpoint{2.560452in}{1.812707in}}%
\pgfpathcurveto{\pgfqpoint{2.560452in}{1.823757in}}{\pgfqpoint{2.556062in}{1.834356in}}{\pgfqpoint{2.548249in}{1.842170in}}%
\pgfpathcurveto{\pgfqpoint{2.540435in}{1.849983in}}{\pgfqpoint{2.529836in}{1.854374in}}{\pgfqpoint{2.518786in}{1.854374in}}%
\pgfpathcurveto{\pgfqpoint{2.507736in}{1.854374in}}{\pgfqpoint{2.497137in}{1.849983in}}{\pgfqpoint{2.489323in}{1.842170in}}%
\pgfpathcurveto{\pgfqpoint{2.481509in}{1.834356in}}{\pgfqpoint{2.477119in}{1.823757in}}{\pgfqpoint{2.477119in}{1.812707in}}%
\pgfpathcurveto{\pgfqpoint{2.477119in}{1.801657in}}{\pgfqpoint{2.481509in}{1.791058in}}{\pgfqpoint{2.489323in}{1.783244in}}%
\pgfpathcurveto{\pgfqpoint{2.497137in}{1.775431in}}{\pgfqpoint{2.507736in}{1.771040in}}{\pgfqpoint{2.518786in}{1.771040in}}%
\pgfpathclose%
\pgfusepath{stroke,fill}%
\end{pgfscope}%
\begin{pgfscope}%
\pgfpathrectangle{\pgfqpoint{0.800000in}{0.528000in}}{\pgfqpoint{4.960000in}{3.696000in}}%
\pgfusepath{clip}%
\pgfsetbuttcap%
\pgfsetroundjoin%
\definecolor{currentfill}{rgb}{0.000000,0.000000,0.000000}%
\pgfsetfillcolor{currentfill}%
\pgfsetlinewidth{1.003750pt}%
\definecolor{currentstroke}{rgb}{0.000000,0.000000,0.000000}%
\pgfsetstrokecolor{currentstroke}%
\pgfsetdash{}{0pt}%
\pgfpathmoveto{\pgfqpoint{2.518786in}{0.664394in}}%
\pgfpathcurveto{\pgfqpoint{2.529836in}{0.664394in}}{\pgfqpoint{2.540435in}{0.668784in}}{\pgfqpoint{2.548249in}{0.676598in}}%
\pgfpathcurveto{\pgfqpoint{2.556062in}{0.684411in}}{\pgfqpoint{2.560452in}{0.695010in}}{\pgfqpoint{2.560452in}{0.706060in}}%
\pgfpathcurveto{\pgfqpoint{2.560452in}{0.717111in}}{\pgfqpoint{2.556062in}{0.727710in}}{\pgfqpoint{2.548249in}{0.735523in}}%
\pgfpathcurveto{\pgfqpoint{2.540435in}{0.743337in}}{\pgfqpoint{2.529836in}{0.747727in}}{\pgfqpoint{2.518786in}{0.747727in}}%
\pgfpathcurveto{\pgfqpoint{2.507736in}{0.747727in}}{\pgfqpoint{2.497137in}{0.743337in}}{\pgfqpoint{2.489323in}{0.735523in}}%
\pgfpathcurveto{\pgfqpoint{2.481509in}{0.727710in}}{\pgfqpoint{2.477119in}{0.717111in}}{\pgfqpoint{2.477119in}{0.706060in}}%
\pgfpathcurveto{\pgfqpoint{2.477119in}{0.695010in}}{\pgfqpoint{2.481509in}{0.684411in}}{\pgfqpoint{2.489323in}{0.676598in}}%
\pgfpathcurveto{\pgfqpoint{2.497137in}{0.668784in}}{\pgfqpoint{2.507736in}{0.664394in}}{\pgfqpoint{2.518786in}{0.664394in}}%
\pgfpathclose%
\pgfusepath{stroke,fill}%
\end{pgfscope}%
\begin{pgfscope}%
\pgfpathrectangle{\pgfqpoint{0.800000in}{0.528000in}}{\pgfqpoint{4.960000in}{3.696000in}}%
\pgfusepath{clip}%
\pgfsetbuttcap%
\pgfsetroundjoin%
\definecolor{currentfill}{rgb}{0.000000,0.000000,0.000000}%
\pgfsetfillcolor{currentfill}%
\pgfsetlinewidth{1.003750pt}%
\definecolor{currentstroke}{rgb}{0.000000,0.000000,0.000000}%
\pgfsetstrokecolor{currentstroke}%
\pgfsetdash{}{0pt}%
\pgfpathmoveto{\pgfqpoint{2.518786in}{1.771040in}}%
\pgfpathcurveto{\pgfqpoint{2.529836in}{1.771040in}}{\pgfqpoint{2.540435in}{1.775431in}}{\pgfqpoint{2.548249in}{1.783244in}}%
\pgfpathcurveto{\pgfqpoint{2.556062in}{1.791058in}}{\pgfqpoint{2.560452in}{1.801657in}}{\pgfqpoint{2.560452in}{1.812707in}}%
\pgfpathcurveto{\pgfqpoint{2.560452in}{1.823757in}}{\pgfqpoint{2.556062in}{1.834356in}}{\pgfqpoint{2.548249in}{1.842170in}}%
\pgfpathcurveto{\pgfqpoint{2.540435in}{1.849983in}}{\pgfqpoint{2.529836in}{1.854374in}}{\pgfqpoint{2.518786in}{1.854374in}}%
\pgfpathcurveto{\pgfqpoint{2.507736in}{1.854374in}}{\pgfqpoint{2.497137in}{1.849983in}}{\pgfqpoint{2.489323in}{1.842170in}}%
\pgfpathcurveto{\pgfqpoint{2.481509in}{1.834356in}}{\pgfqpoint{2.477119in}{1.823757in}}{\pgfqpoint{2.477119in}{1.812707in}}%
\pgfpathcurveto{\pgfqpoint{2.477119in}{1.801657in}}{\pgfqpoint{2.481509in}{1.791058in}}{\pgfqpoint{2.489323in}{1.783244in}}%
\pgfpathcurveto{\pgfqpoint{2.497137in}{1.775431in}}{\pgfqpoint{2.507736in}{1.771040in}}{\pgfqpoint{2.518786in}{1.771040in}}%
\pgfpathclose%
\pgfusepath{stroke,fill}%
\end{pgfscope}%
\begin{pgfscope}%
\pgfpathrectangle{\pgfqpoint{0.800000in}{0.528000in}}{\pgfqpoint{4.960000in}{3.696000in}}%
\pgfusepath{clip}%
\pgfsetbuttcap%
\pgfsetroundjoin%
\definecolor{currentfill}{rgb}{0.000000,0.000000,0.000000}%
\pgfsetfillcolor{currentfill}%
\pgfsetlinewidth{1.003750pt}%
\definecolor{currentstroke}{rgb}{0.000000,0.000000,0.000000}%
\pgfsetstrokecolor{currentstroke}%
\pgfsetdash{}{0pt}%
\pgfpathmoveto{\pgfqpoint{2.518786in}{1.771040in}}%
\pgfpathcurveto{\pgfqpoint{2.529836in}{1.771040in}}{\pgfqpoint{2.540435in}{1.775431in}}{\pgfqpoint{2.548249in}{1.783244in}}%
\pgfpathcurveto{\pgfqpoint{2.556062in}{1.791058in}}{\pgfqpoint{2.560452in}{1.801657in}}{\pgfqpoint{2.560452in}{1.812707in}}%
\pgfpathcurveto{\pgfqpoint{2.560452in}{1.823757in}}{\pgfqpoint{2.556062in}{1.834356in}}{\pgfqpoint{2.548249in}{1.842170in}}%
\pgfpathcurveto{\pgfqpoint{2.540435in}{1.849983in}}{\pgfqpoint{2.529836in}{1.854374in}}{\pgfqpoint{2.518786in}{1.854374in}}%
\pgfpathcurveto{\pgfqpoint{2.507736in}{1.854374in}}{\pgfqpoint{2.497137in}{1.849983in}}{\pgfqpoint{2.489323in}{1.842170in}}%
\pgfpathcurveto{\pgfqpoint{2.481509in}{1.834356in}}{\pgfqpoint{2.477119in}{1.823757in}}{\pgfqpoint{2.477119in}{1.812707in}}%
\pgfpathcurveto{\pgfqpoint{2.477119in}{1.801657in}}{\pgfqpoint{2.481509in}{1.791058in}}{\pgfqpoint{2.489323in}{1.783244in}}%
\pgfpathcurveto{\pgfqpoint{2.497137in}{1.775431in}}{\pgfqpoint{2.507736in}{1.771040in}}{\pgfqpoint{2.518786in}{1.771040in}}%
\pgfpathclose%
\pgfusepath{stroke,fill}%
\end{pgfscope}%
\begin{pgfscope}%
\pgfpathrectangle{\pgfqpoint{0.800000in}{0.528000in}}{\pgfqpoint{4.960000in}{3.696000in}}%
\pgfusepath{clip}%
\pgfsetbuttcap%
\pgfsetroundjoin%
\definecolor{currentfill}{rgb}{0.000000,0.000000,0.000000}%
\pgfsetfillcolor{currentfill}%
\pgfsetlinewidth{1.003750pt}%
\definecolor{currentstroke}{rgb}{0.000000,0.000000,0.000000}%
\pgfsetstrokecolor{currentstroke}%
\pgfsetdash{}{0pt}%
\pgfpathmoveto{\pgfqpoint{2.518786in}{1.771040in}}%
\pgfpathcurveto{\pgfqpoint{2.529836in}{1.771040in}}{\pgfqpoint{2.540435in}{1.775431in}}{\pgfqpoint{2.548249in}{1.783244in}}%
\pgfpathcurveto{\pgfqpoint{2.556062in}{1.791058in}}{\pgfqpoint{2.560452in}{1.801657in}}{\pgfqpoint{2.560452in}{1.812707in}}%
\pgfpathcurveto{\pgfqpoint{2.560452in}{1.823757in}}{\pgfqpoint{2.556062in}{1.834356in}}{\pgfqpoint{2.548249in}{1.842170in}}%
\pgfpathcurveto{\pgfqpoint{2.540435in}{1.849983in}}{\pgfqpoint{2.529836in}{1.854374in}}{\pgfqpoint{2.518786in}{1.854374in}}%
\pgfpathcurveto{\pgfqpoint{2.507736in}{1.854374in}}{\pgfqpoint{2.497137in}{1.849983in}}{\pgfqpoint{2.489323in}{1.842170in}}%
\pgfpathcurveto{\pgfqpoint{2.481509in}{1.834356in}}{\pgfqpoint{2.477119in}{1.823757in}}{\pgfqpoint{2.477119in}{1.812707in}}%
\pgfpathcurveto{\pgfqpoint{2.477119in}{1.801657in}}{\pgfqpoint{2.481509in}{1.791058in}}{\pgfqpoint{2.489323in}{1.783244in}}%
\pgfpathcurveto{\pgfqpoint{2.497137in}{1.775431in}}{\pgfqpoint{2.507736in}{1.771040in}}{\pgfqpoint{2.518786in}{1.771040in}}%
\pgfpathclose%
\pgfusepath{stroke,fill}%
\end{pgfscope}%
\begin{pgfscope}%
\pgfpathrectangle{\pgfqpoint{0.800000in}{0.528000in}}{\pgfqpoint{4.960000in}{3.696000in}}%
\pgfusepath{clip}%
\pgfsetbuttcap%
\pgfsetroundjoin%
\definecolor{currentfill}{rgb}{0.000000,0.000000,0.000000}%
\pgfsetfillcolor{currentfill}%
\pgfsetlinewidth{1.003750pt}%
\definecolor{currentstroke}{rgb}{0.000000,0.000000,0.000000}%
\pgfsetstrokecolor{currentstroke}%
\pgfsetdash{}{0pt}%
\pgfpathmoveto{\pgfqpoint{2.518786in}{1.771040in}}%
\pgfpathcurveto{\pgfqpoint{2.529836in}{1.771040in}}{\pgfqpoint{2.540435in}{1.775431in}}{\pgfqpoint{2.548249in}{1.783244in}}%
\pgfpathcurveto{\pgfqpoint{2.556062in}{1.791058in}}{\pgfqpoint{2.560452in}{1.801657in}}{\pgfqpoint{2.560452in}{1.812707in}}%
\pgfpathcurveto{\pgfqpoint{2.560452in}{1.823757in}}{\pgfqpoint{2.556062in}{1.834356in}}{\pgfqpoint{2.548249in}{1.842170in}}%
\pgfpathcurveto{\pgfqpoint{2.540435in}{1.849983in}}{\pgfqpoint{2.529836in}{1.854374in}}{\pgfqpoint{2.518786in}{1.854374in}}%
\pgfpathcurveto{\pgfqpoint{2.507736in}{1.854374in}}{\pgfqpoint{2.497137in}{1.849983in}}{\pgfqpoint{2.489323in}{1.842170in}}%
\pgfpathcurveto{\pgfqpoint{2.481509in}{1.834356in}}{\pgfqpoint{2.477119in}{1.823757in}}{\pgfqpoint{2.477119in}{1.812707in}}%
\pgfpathcurveto{\pgfqpoint{2.477119in}{1.801657in}}{\pgfqpoint{2.481509in}{1.791058in}}{\pgfqpoint{2.489323in}{1.783244in}}%
\pgfpathcurveto{\pgfqpoint{2.497137in}{1.775431in}}{\pgfqpoint{2.507736in}{1.771040in}}{\pgfqpoint{2.518786in}{1.771040in}}%
\pgfpathclose%
\pgfusepath{stroke,fill}%
\end{pgfscope}%
\begin{pgfscope}%
\pgfpathrectangle{\pgfqpoint{0.800000in}{0.528000in}}{\pgfqpoint{4.960000in}{3.696000in}}%
\pgfusepath{clip}%
\pgfsetbuttcap%
\pgfsetroundjoin%
\definecolor{currentfill}{rgb}{0.000000,0.000000,0.000000}%
\pgfsetfillcolor{currentfill}%
\pgfsetlinewidth{1.003750pt}%
\definecolor{currentstroke}{rgb}{0.000000,0.000000,0.000000}%
\pgfsetstrokecolor{currentstroke}%
\pgfsetdash{}{0pt}%
\pgfpathmoveto{\pgfqpoint{2.518786in}{1.771040in}}%
\pgfpathcurveto{\pgfqpoint{2.529836in}{1.771040in}}{\pgfqpoint{2.540435in}{1.775431in}}{\pgfqpoint{2.548249in}{1.783244in}}%
\pgfpathcurveto{\pgfqpoint{2.556062in}{1.791058in}}{\pgfqpoint{2.560452in}{1.801657in}}{\pgfqpoint{2.560452in}{1.812707in}}%
\pgfpathcurveto{\pgfqpoint{2.560452in}{1.823757in}}{\pgfqpoint{2.556062in}{1.834356in}}{\pgfqpoint{2.548249in}{1.842170in}}%
\pgfpathcurveto{\pgfqpoint{2.540435in}{1.849983in}}{\pgfqpoint{2.529836in}{1.854374in}}{\pgfqpoint{2.518786in}{1.854374in}}%
\pgfpathcurveto{\pgfqpoint{2.507736in}{1.854374in}}{\pgfqpoint{2.497137in}{1.849983in}}{\pgfqpoint{2.489323in}{1.842170in}}%
\pgfpathcurveto{\pgfqpoint{2.481509in}{1.834356in}}{\pgfqpoint{2.477119in}{1.823757in}}{\pgfqpoint{2.477119in}{1.812707in}}%
\pgfpathcurveto{\pgfqpoint{2.477119in}{1.801657in}}{\pgfqpoint{2.481509in}{1.791058in}}{\pgfqpoint{2.489323in}{1.783244in}}%
\pgfpathcurveto{\pgfqpoint{2.497137in}{1.775431in}}{\pgfqpoint{2.507736in}{1.771040in}}{\pgfqpoint{2.518786in}{1.771040in}}%
\pgfpathclose%
\pgfusepath{stroke,fill}%
\end{pgfscope}%
\begin{pgfscope}%
\pgfpathrectangle{\pgfqpoint{0.800000in}{0.528000in}}{\pgfqpoint{4.960000in}{3.696000in}}%
\pgfusepath{clip}%
\pgfsetbuttcap%
\pgfsetroundjoin%
\definecolor{currentfill}{rgb}{0.000000,0.000000,0.000000}%
\pgfsetfillcolor{currentfill}%
\pgfsetlinewidth{1.003750pt}%
\definecolor{currentstroke}{rgb}{0.000000,0.000000,0.000000}%
\pgfsetstrokecolor{currentstroke}%
\pgfsetdash{}{0pt}%
\pgfpathmoveto{\pgfqpoint{2.518786in}{1.771040in}}%
\pgfpathcurveto{\pgfqpoint{2.529836in}{1.771040in}}{\pgfqpoint{2.540435in}{1.775431in}}{\pgfqpoint{2.548249in}{1.783244in}}%
\pgfpathcurveto{\pgfqpoint{2.556062in}{1.791058in}}{\pgfqpoint{2.560452in}{1.801657in}}{\pgfqpoint{2.560452in}{1.812707in}}%
\pgfpathcurveto{\pgfqpoint{2.560452in}{1.823757in}}{\pgfqpoint{2.556062in}{1.834356in}}{\pgfqpoint{2.548249in}{1.842170in}}%
\pgfpathcurveto{\pgfqpoint{2.540435in}{1.849983in}}{\pgfqpoint{2.529836in}{1.854374in}}{\pgfqpoint{2.518786in}{1.854374in}}%
\pgfpathcurveto{\pgfqpoint{2.507736in}{1.854374in}}{\pgfqpoint{2.497137in}{1.849983in}}{\pgfqpoint{2.489323in}{1.842170in}}%
\pgfpathcurveto{\pgfqpoint{2.481509in}{1.834356in}}{\pgfqpoint{2.477119in}{1.823757in}}{\pgfqpoint{2.477119in}{1.812707in}}%
\pgfpathcurveto{\pgfqpoint{2.477119in}{1.801657in}}{\pgfqpoint{2.481509in}{1.791058in}}{\pgfqpoint{2.489323in}{1.783244in}}%
\pgfpathcurveto{\pgfqpoint{2.497137in}{1.775431in}}{\pgfqpoint{2.507736in}{1.771040in}}{\pgfqpoint{2.518786in}{1.771040in}}%
\pgfpathclose%
\pgfusepath{stroke,fill}%
\end{pgfscope}%
\begin{pgfscope}%
\pgfpathrectangle{\pgfqpoint{0.800000in}{0.528000in}}{\pgfqpoint{4.960000in}{3.696000in}}%
\pgfusepath{clip}%
\pgfsetbuttcap%
\pgfsetroundjoin%
\definecolor{currentfill}{rgb}{0.000000,0.000000,0.000000}%
\pgfsetfillcolor{currentfill}%
\pgfsetlinewidth{1.003750pt}%
\definecolor{currentstroke}{rgb}{0.000000,0.000000,0.000000}%
\pgfsetstrokecolor{currentstroke}%
\pgfsetdash{}{0pt}%
\pgfpathmoveto{\pgfqpoint{2.518786in}{1.771040in}}%
\pgfpathcurveto{\pgfqpoint{2.529836in}{1.771040in}}{\pgfqpoint{2.540435in}{1.775431in}}{\pgfqpoint{2.548249in}{1.783244in}}%
\pgfpathcurveto{\pgfqpoint{2.556062in}{1.791058in}}{\pgfqpoint{2.560452in}{1.801657in}}{\pgfqpoint{2.560452in}{1.812707in}}%
\pgfpathcurveto{\pgfqpoint{2.560452in}{1.823757in}}{\pgfqpoint{2.556062in}{1.834356in}}{\pgfqpoint{2.548249in}{1.842170in}}%
\pgfpathcurveto{\pgfqpoint{2.540435in}{1.849983in}}{\pgfqpoint{2.529836in}{1.854374in}}{\pgfqpoint{2.518786in}{1.854374in}}%
\pgfpathcurveto{\pgfqpoint{2.507736in}{1.854374in}}{\pgfqpoint{2.497137in}{1.849983in}}{\pgfqpoint{2.489323in}{1.842170in}}%
\pgfpathcurveto{\pgfqpoint{2.481509in}{1.834356in}}{\pgfqpoint{2.477119in}{1.823757in}}{\pgfqpoint{2.477119in}{1.812707in}}%
\pgfpathcurveto{\pgfqpoint{2.477119in}{1.801657in}}{\pgfqpoint{2.481509in}{1.791058in}}{\pgfqpoint{2.489323in}{1.783244in}}%
\pgfpathcurveto{\pgfqpoint{2.497137in}{1.775431in}}{\pgfqpoint{2.507736in}{1.771040in}}{\pgfqpoint{2.518786in}{1.771040in}}%
\pgfpathclose%
\pgfusepath{stroke,fill}%
\end{pgfscope}%
\begin{pgfscope}%
\pgfpathrectangle{\pgfqpoint{0.800000in}{0.528000in}}{\pgfqpoint{4.960000in}{3.696000in}}%
\pgfusepath{clip}%
\pgfsetbuttcap%
\pgfsetroundjoin%
\definecolor{currentfill}{rgb}{0.000000,0.000000,0.000000}%
\pgfsetfillcolor{currentfill}%
\pgfsetlinewidth{1.003750pt}%
\definecolor{currentstroke}{rgb}{0.000000,0.000000,0.000000}%
\pgfsetstrokecolor{currentstroke}%
\pgfsetdash{}{0pt}%
\pgfpathmoveto{\pgfqpoint{2.518786in}{1.771040in}}%
\pgfpathcurveto{\pgfqpoint{2.529836in}{1.771040in}}{\pgfqpoint{2.540435in}{1.775431in}}{\pgfqpoint{2.548249in}{1.783244in}}%
\pgfpathcurveto{\pgfqpoint{2.556062in}{1.791058in}}{\pgfqpoint{2.560452in}{1.801657in}}{\pgfqpoint{2.560452in}{1.812707in}}%
\pgfpathcurveto{\pgfqpoint{2.560452in}{1.823757in}}{\pgfqpoint{2.556062in}{1.834356in}}{\pgfqpoint{2.548249in}{1.842170in}}%
\pgfpathcurveto{\pgfqpoint{2.540435in}{1.849983in}}{\pgfqpoint{2.529836in}{1.854374in}}{\pgfqpoint{2.518786in}{1.854374in}}%
\pgfpathcurveto{\pgfqpoint{2.507736in}{1.854374in}}{\pgfqpoint{2.497137in}{1.849983in}}{\pgfqpoint{2.489323in}{1.842170in}}%
\pgfpathcurveto{\pgfqpoint{2.481509in}{1.834356in}}{\pgfqpoint{2.477119in}{1.823757in}}{\pgfqpoint{2.477119in}{1.812707in}}%
\pgfpathcurveto{\pgfqpoint{2.477119in}{1.801657in}}{\pgfqpoint{2.481509in}{1.791058in}}{\pgfqpoint{2.489323in}{1.783244in}}%
\pgfpathcurveto{\pgfqpoint{2.497137in}{1.775431in}}{\pgfqpoint{2.507736in}{1.771040in}}{\pgfqpoint{2.518786in}{1.771040in}}%
\pgfpathclose%
\pgfusepath{stroke,fill}%
\end{pgfscope}%
\begin{pgfscope}%
\pgfpathrectangle{\pgfqpoint{0.800000in}{0.528000in}}{\pgfqpoint{4.960000in}{3.696000in}}%
\pgfusepath{clip}%
\pgfsetbuttcap%
\pgfsetroundjoin%
\definecolor{currentfill}{rgb}{0.000000,0.000000,0.000000}%
\pgfsetfillcolor{currentfill}%
\pgfsetlinewidth{1.003750pt}%
\definecolor{currentstroke}{rgb}{0.000000,0.000000,0.000000}%
\pgfsetstrokecolor{currentstroke}%
\pgfsetdash{}{0pt}%
\pgfpathmoveto{\pgfqpoint{2.518786in}{1.771040in}}%
\pgfpathcurveto{\pgfqpoint{2.529836in}{1.771040in}}{\pgfqpoint{2.540435in}{1.775431in}}{\pgfqpoint{2.548249in}{1.783244in}}%
\pgfpathcurveto{\pgfqpoint{2.556062in}{1.791058in}}{\pgfqpoint{2.560452in}{1.801657in}}{\pgfqpoint{2.560452in}{1.812707in}}%
\pgfpathcurveto{\pgfqpoint{2.560452in}{1.823757in}}{\pgfqpoint{2.556062in}{1.834356in}}{\pgfqpoint{2.548249in}{1.842170in}}%
\pgfpathcurveto{\pgfqpoint{2.540435in}{1.849983in}}{\pgfqpoint{2.529836in}{1.854374in}}{\pgfqpoint{2.518786in}{1.854374in}}%
\pgfpathcurveto{\pgfqpoint{2.507736in}{1.854374in}}{\pgfqpoint{2.497137in}{1.849983in}}{\pgfqpoint{2.489323in}{1.842170in}}%
\pgfpathcurveto{\pgfqpoint{2.481509in}{1.834356in}}{\pgfqpoint{2.477119in}{1.823757in}}{\pgfqpoint{2.477119in}{1.812707in}}%
\pgfpathcurveto{\pgfqpoint{2.477119in}{1.801657in}}{\pgfqpoint{2.481509in}{1.791058in}}{\pgfqpoint{2.489323in}{1.783244in}}%
\pgfpathcurveto{\pgfqpoint{2.497137in}{1.775431in}}{\pgfqpoint{2.507736in}{1.771040in}}{\pgfqpoint{2.518786in}{1.771040in}}%
\pgfpathclose%
\pgfusepath{stroke,fill}%
\end{pgfscope}%
\begin{pgfscope}%
\pgfpathrectangle{\pgfqpoint{0.800000in}{0.528000in}}{\pgfqpoint{4.960000in}{3.696000in}}%
\pgfusepath{clip}%
\pgfsetbuttcap%
\pgfsetroundjoin%
\definecolor{currentfill}{rgb}{0.000000,0.000000,0.000000}%
\pgfsetfillcolor{currentfill}%
\pgfsetlinewidth{1.003750pt}%
\definecolor{currentstroke}{rgb}{0.000000,0.000000,0.000000}%
\pgfsetstrokecolor{currentstroke}%
\pgfsetdash{}{0pt}%
\pgfpathmoveto{\pgfqpoint{2.518786in}{1.771040in}}%
\pgfpathcurveto{\pgfqpoint{2.529836in}{1.771040in}}{\pgfqpoint{2.540435in}{1.775431in}}{\pgfqpoint{2.548249in}{1.783244in}}%
\pgfpathcurveto{\pgfqpoint{2.556062in}{1.791058in}}{\pgfqpoint{2.560452in}{1.801657in}}{\pgfqpoint{2.560452in}{1.812707in}}%
\pgfpathcurveto{\pgfqpoint{2.560452in}{1.823757in}}{\pgfqpoint{2.556062in}{1.834356in}}{\pgfqpoint{2.548249in}{1.842170in}}%
\pgfpathcurveto{\pgfqpoint{2.540435in}{1.849983in}}{\pgfqpoint{2.529836in}{1.854374in}}{\pgfqpoint{2.518786in}{1.854374in}}%
\pgfpathcurveto{\pgfqpoint{2.507736in}{1.854374in}}{\pgfqpoint{2.497137in}{1.849983in}}{\pgfqpoint{2.489323in}{1.842170in}}%
\pgfpathcurveto{\pgfqpoint{2.481509in}{1.834356in}}{\pgfqpoint{2.477119in}{1.823757in}}{\pgfqpoint{2.477119in}{1.812707in}}%
\pgfpathcurveto{\pgfqpoint{2.477119in}{1.801657in}}{\pgfqpoint{2.481509in}{1.791058in}}{\pgfqpoint{2.489323in}{1.783244in}}%
\pgfpathcurveto{\pgfqpoint{2.497137in}{1.775431in}}{\pgfqpoint{2.507736in}{1.771040in}}{\pgfqpoint{2.518786in}{1.771040in}}%
\pgfpathclose%
\pgfusepath{stroke,fill}%
\end{pgfscope}%
\begin{pgfscope}%
\pgfpathrectangle{\pgfqpoint{0.800000in}{0.528000in}}{\pgfqpoint{4.960000in}{3.696000in}}%
\pgfusepath{clip}%
\pgfsetbuttcap%
\pgfsetroundjoin%
\definecolor{currentfill}{rgb}{0.000000,0.000000,0.000000}%
\pgfsetfillcolor{currentfill}%
\pgfsetlinewidth{1.003750pt}%
\definecolor{currentstroke}{rgb}{0.000000,0.000000,0.000000}%
\pgfsetstrokecolor{currentstroke}%
\pgfsetdash{}{0pt}%
\pgfpathmoveto{\pgfqpoint{4.011666in}{2.877687in}}%
\pgfpathcurveto{\pgfqpoint{4.022716in}{2.877687in}}{\pgfqpoint{4.033315in}{2.882077in}}{\pgfqpoint{4.041128in}{2.889891in}}%
\pgfpathcurveto{\pgfqpoint{4.048942in}{2.897704in}}{\pgfqpoint{4.053332in}{2.908303in}}{\pgfqpoint{4.053332in}{2.919353in}}%
\pgfpathcurveto{\pgfqpoint{4.053332in}{2.930404in}}{\pgfqpoint{4.048942in}{2.941003in}}{\pgfqpoint{4.041128in}{2.948816in}}%
\pgfpathcurveto{\pgfqpoint{4.033315in}{2.956630in}}{\pgfqpoint{4.022716in}{2.961020in}}{\pgfqpoint{4.011666in}{2.961020in}}%
\pgfpathcurveto{\pgfqpoint{4.000616in}{2.961020in}}{\pgfqpoint{3.990016in}{2.956630in}}{\pgfqpoint{3.982203in}{2.948816in}}%
\pgfpathcurveto{\pgfqpoint{3.974389in}{2.941003in}}{\pgfqpoint{3.969999in}{2.930404in}}{\pgfqpoint{3.969999in}{2.919353in}}%
\pgfpathcurveto{\pgfqpoint{3.969999in}{2.908303in}}{\pgfqpoint{3.974389in}{2.897704in}}{\pgfqpoint{3.982203in}{2.889891in}}%
\pgfpathcurveto{\pgfqpoint{3.990016in}{2.882077in}}{\pgfqpoint{4.000616in}{2.877687in}}{\pgfqpoint{4.011666in}{2.877687in}}%
\pgfpathclose%
\pgfusepath{stroke,fill}%
\end{pgfscope}%
\begin{pgfscope}%
\pgfpathrectangle{\pgfqpoint{0.800000in}{0.528000in}}{\pgfqpoint{4.960000in}{3.696000in}}%
\pgfusepath{clip}%
\pgfsetbuttcap%
\pgfsetroundjoin%
\definecolor{currentfill}{rgb}{0.000000,0.000000,0.000000}%
\pgfsetfillcolor{currentfill}%
\pgfsetlinewidth{1.003750pt}%
\definecolor{currentstroke}{rgb}{0.000000,0.000000,0.000000}%
\pgfsetstrokecolor{currentstroke}%
\pgfsetdash{}{0pt}%
\pgfpathmoveto{\pgfqpoint{4.011666in}{1.771040in}}%
\pgfpathcurveto{\pgfqpoint{4.022716in}{1.771040in}}{\pgfqpoint{4.033315in}{1.775431in}}{\pgfqpoint{4.041128in}{1.783244in}}%
\pgfpathcurveto{\pgfqpoint{4.048942in}{1.791058in}}{\pgfqpoint{4.053332in}{1.801657in}}{\pgfqpoint{4.053332in}{1.812707in}}%
\pgfpathcurveto{\pgfqpoint{4.053332in}{1.823757in}}{\pgfqpoint{4.048942in}{1.834356in}}{\pgfqpoint{4.041128in}{1.842170in}}%
\pgfpathcurveto{\pgfqpoint{4.033315in}{1.849983in}}{\pgfqpoint{4.022716in}{1.854374in}}{\pgfqpoint{4.011666in}{1.854374in}}%
\pgfpathcurveto{\pgfqpoint{4.000616in}{1.854374in}}{\pgfqpoint{3.990016in}{1.849983in}}{\pgfqpoint{3.982203in}{1.842170in}}%
\pgfpathcurveto{\pgfqpoint{3.974389in}{1.834356in}}{\pgfqpoint{3.969999in}{1.823757in}}{\pgfqpoint{3.969999in}{1.812707in}}%
\pgfpathcurveto{\pgfqpoint{3.969999in}{1.801657in}}{\pgfqpoint{3.974389in}{1.791058in}}{\pgfqpoint{3.982203in}{1.783244in}}%
\pgfpathcurveto{\pgfqpoint{3.990016in}{1.775431in}}{\pgfqpoint{4.000616in}{1.771040in}}{\pgfqpoint{4.011666in}{1.771040in}}%
\pgfpathclose%
\pgfusepath{stroke,fill}%
\end{pgfscope}%
\begin{pgfscope}%
\pgfpathrectangle{\pgfqpoint{0.800000in}{0.528000in}}{\pgfqpoint{4.960000in}{3.696000in}}%
\pgfusepath{clip}%
\pgfsetbuttcap%
\pgfsetroundjoin%
\definecolor{currentfill}{rgb}{0.000000,0.000000,0.000000}%
\pgfsetfillcolor{currentfill}%
\pgfsetlinewidth{1.003750pt}%
\definecolor{currentstroke}{rgb}{0.000000,0.000000,0.000000}%
\pgfsetstrokecolor{currentstroke}%
\pgfsetdash{}{0pt}%
\pgfpathmoveto{\pgfqpoint{4.011666in}{1.771040in}}%
\pgfpathcurveto{\pgfqpoint{4.022716in}{1.771040in}}{\pgfqpoint{4.033315in}{1.775431in}}{\pgfqpoint{4.041128in}{1.783244in}}%
\pgfpathcurveto{\pgfqpoint{4.048942in}{1.791058in}}{\pgfqpoint{4.053332in}{1.801657in}}{\pgfqpoint{4.053332in}{1.812707in}}%
\pgfpathcurveto{\pgfqpoint{4.053332in}{1.823757in}}{\pgfqpoint{4.048942in}{1.834356in}}{\pgfqpoint{4.041128in}{1.842170in}}%
\pgfpathcurveto{\pgfqpoint{4.033315in}{1.849983in}}{\pgfqpoint{4.022716in}{1.854374in}}{\pgfqpoint{4.011666in}{1.854374in}}%
\pgfpathcurveto{\pgfqpoint{4.000616in}{1.854374in}}{\pgfqpoint{3.990016in}{1.849983in}}{\pgfqpoint{3.982203in}{1.842170in}}%
\pgfpathcurveto{\pgfqpoint{3.974389in}{1.834356in}}{\pgfqpoint{3.969999in}{1.823757in}}{\pgfqpoint{3.969999in}{1.812707in}}%
\pgfpathcurveto{\pgfqpoint{3.969999in}{1.801657in}}{\pgfqpoint{3.974389in}{1.791058in}}{\pgfqpoint{3.982203in}{1.783244in}}%
\pgfpathcurveto{\pgfqpoint{3.990016in}{1.775431in}}{\pgfqpoint{4.000616in}{1.771040in}}{\pgfqpoint{4.011666in}{1.771040in}}%
\pgfpathclose%
\pgfusepath{stroke,fill}%
\end{pgfscope}%
\begin{pgfscope}%
\pgfpathrectangle{\pgfqpoint{0.800000in}{0.528000in}}{\pgfqpoint{4.960000in}{3.696000in}}%
\pgfusepath{clip}%
\pgfsetbuttcap%
\pgfsetroundjoin%
\definecolor{currentfill}{rgb}{0.000000,0.000000,0.000000}%
\pgfsetfillcolor{currentfill}%
\pgfsetlinewidth{1.003750pt}%
\definecolor{currentstroke}{rgb}{0.000000,0.000000,0.000000}%
\pgfsetstrokecolor{currentstroke}%
\pgfsetdash{}{0pt}%
\pgfpathmoveto{\pgfqpoint{4.011666in}{1.771040in}}%
\pgfpathcurveto{\pgfqpoint{4.022716in}{1.771040in}}{\pgfqpoint{4.033315in}{1.775431in}}{\pgfqpoint{4.041128in}{1.783244in}}%
\pgfpathcurveto{\pgfqpoint{4.048942in}{1.791058in}}{\pgfqpoint{4.053332in}{1.801657in}}{\pgfqpoint{4.053332in}{1.812707in}}%
\pgfpathcurveto{\pgfqpoint{4.053332in}{1.823757in}}{\pgfqpoint{4.048942in}{1.834356in}}{\pgfqpoint{4.041128in}{1.842170in}}%
\pgfpathcurveto{\pgfqpoint{4.033315in}{1.849983in}}{\pgfqpoint{4.022716in}{1.854374in}}{\pgfqpoint{4.011666in}{1.854374in}}%
\pgfpathcurveto{\pgfqpoint{4.000616in}{1.854374in}}{\pgfqpoint{3.990016in}{1.849983in}}{\pgfqpoint{3.982203in}{1.842170in}}%
\pgfpathcurveto{\pgfqpoint{3.974389in}{1.834356in}}{\pgfqpoint{3.969999in}{1.823757in}}{\pgfqpoint{3.969999in}{1.812707in}}%
\pgfpathcurveto{\pgfqpoint{3.969999in}{1.801657in}}{\pgfqpoint{3.974389in}{1.791058in}}{\pgfqpoint{3.982203in}{1.783244in}}%
\pgfpathcurveto{\pgfqpoint{3.990016in}{1.775431in}}{\pgfqpoint{4.000616in}{1.771040in}}{\pgfqpoint{4.011666in}{1.771040in}}%
\pgfpathclose%
\pgfusepath{stroke,fill}%
\end{pgfscope}%
\begin{pgfscope}%
\pgfpathrectangle{\pgfqpoint{0.800000in}{0.528000in}}{\pgfqpoint{4.960000in}{3.696000in}}%
\pgfusepath{clip}%
\pgfsetbuttcap%
\pgfsetroundjoin%
\definecolor{currentfill}{rgb}{0.000000,0.000000,0.000000}%
\pgfsetfillcolor{currentfill}%
\pgfsetlinewidth{1.003750pt}%
\definecolor{currentstroke}{rgb}{0.000000,0.000000,0.000000}%
\pgfsetstrokecolor{currentstroke}%
\pgfsetdash{}{0pt}%
\pgfpathmoveto{\pgfqpoint{4.011666in}{2.877687in}}%
\pgfpathcurveto{\pgfqpoint{4.022716in}{2.877687in}}{\pgfqpoint{4.033315in}{2.882077in}}{\pgfqpoint{4.041128in}{2.889891in}}%
\pgfpathcurveto{\pgfqpoint{4.048942in}{2.897704in}}{\pgfqpoint{4.053332in}{2.908303in}}{\pgfqpoint{4.053332in}{2.919353in}}%
\pgfpathcurveto{\pgfqpoint{4.053332in}{2.930404in}}{\pgfqpoint{4.048942in}{2.941003in}}{\pgfqpoint{4.041128in}{2.948816in}}%
\pgfpathcurveto{\pgfqpoint{4.033315in}{2.956630in}}{\pgfqpoint{4.022716in}{2.961020in}}{\pgfqpoint{4.011666in}{2.961020in}}%
\pgfpathcurveto{\pgfqpoint{4.000616in}{2.961020in}}{\pgfqpoint{3.990016in}{2.956630in}}{\pgfqpoint{3.982203in}{2.948816in}}%
\pgfpathcurveto{\pgfqpoint{3.974389in}{2.941003in}}{\pgfqpoint{3.969999in}{2.930404in}}{\pgfqpoint{3.969999in}{2.919353in}}%
\pgfpathcurveto{\pgfqpoint{3.969999in}{2.908303in}}{\pgfqpoint{3.974389in}{2.897704in}}{\pgfqpoint{3.982203in}{2.889891in}}%
\pgfpathcurveto{\pgfqpoint{3.990016in}{2.882077in}}{\pgfqpoint{4.000616in}{2.877687in}}{\pgfqpoint{4.011666in}{2.877687in}}%
\pgfpathclose%
\pgfusepath{stroke,fill}%
\end{pgfscope}%
\begin{pgfscope}%
\pgfpathrectangle{\pgfqpoint{0.800000in}{0.528000in}}{\pgfqpoint{4.960000in}{3.696000in}}%
\pgfusepath{clip}%
\pgfsetbuttcap%
\pgfsetroundjoin%
\definecolor{currentfill}{rgb}{0.000000,0.000000,0.000000}%
\pgfsetfillcolor{currentfill}%
\pgfsetlinewidth{1.003750pt}%
\definecolor{currentstroke}{rgb}{0.000000,0.000000,0.000000}%
\pgfsetstrokecolor{currentstroke}%
\pgfsetdash{}{0pt}%
\pgfpathmoveto{\pgfqpoint{4.011666in}{1.771040in}}%
\pgfpathcurveto{\pgfqpoint{4.022716in}{1.771040in}}{\pgfqpoint{4.033315in}{1.775431in}}{\pgfqpoint{4.041128in}{1.783244in}}%
\pgfpathcurveto{\pgfqpoint{4.048942in}{1.791058in}}{\pgfqpoint{4.053332in}{1.801657in}}{\pgfqpoint{4.053332in}{1.812707in}}%
\pgfpathcurveto{\pgfqpoint{4.053332in}{1.823757in}}{\pgfqpoint{4.048942in}{1.834356in}}{\pgfqpoint{4.041128in}{1.842170in}}%
\pgfpathcurveto{\pgfqpoint{4.033315in}{1.849983in}}{\pgfqpoint{4.022716in}{1.854374in}}{\pgfqpoint{4.011666in}{1.854374in}}%
\pgfpathcurveto{\pgfqpoint{4.000616in}{1.854374in}}{\pgfqpoint{3.990016in}{1.849983in}}{\pgfqpoint{3.982203in}{1.842170in}}%
\pgfpathcurveto{\pgfqpoint{3.974389in}{1.834356in}}{\pgfqpoint{3.969999in}{1.823757in}}{\pgfqpoint{3.969999in}{1.812707in}}%
\pgfpathcurveto{\pgfqpoint{3.969999in}{1.801657in}}{\pgfqpoint{3.974389in}{1.791058in}}{\pgfqpoint{3.982203in}{1.783244in}}%
\pgfpathcurveto{\pgfqpoint{3.990016in}{1.775431in}}{\pgfqpoint{4.000616in}{1.771040in}}{\pgfqpoint{4.011666in}{1.771040in}}%
\pgfpathclose%
\pgfusepath{stroke,fill}%
\end{pgfscope}%
\begin{pgfscope}%
\pgfpathrectangle{\pgfqpoint{0.800000in}{0.528000in}}{\pgfqpoint{4.960000in}{3.696000in}}%
\pgfusepath{clip}%
\pgfsetbuttcap%
\pgfsetroundjoin%
\definecolor{currentfill}{rgb}{0.000000,0.000000,0.000000}%
\pgfsetfillcolor{currentfill}%
\pgfsetlinewidth{1.003750pt}%
\definecolor{currentstroke}{rgb}{0.000000,0.000000,0.000000}%
\pgfsetstrokecolor{currentstroke}%
\pgfsetdash{}{0pt}%
\pgfpathmoveto{\pgfqpoint{4.011666in}{1.771040in}}%
\pgfpathcurveto{\pgfqpoint{4.022716in}{1.771040in}}{\pgfqpoint{4.033315in}{1.775431in}}{\pgfqpoint{4.041128in}{1.783244in}}%
\pgfpathcurveto{\pgfqpoint{4.048942in}{1.791058in}}{\pgfqpoint{4.053332in}{1.801657in}}{\pgfqpoint{4.053332in}{1.812707in}}%
\pgfpathcurveto{\pgfqpoint{4.053332in}{1.823757in}}{\pgfqpoint{4.048942in}{1.834356in}}{\pgfqpoint{4.041128in}{1.842170in}}%
\pgfpathcurveto{\pgfqpoint{4.033315in}{1.849983in}}{\pgfqpoint{4.022716in}{1.854374in}}{\pgfqpoint{4.011666in}{1.854374in}}%
\pgfpathcurveto{\pgfqpoint{4.000616in}{1.854374in}}{\pgfqpoint{3.990016in}{1.849983in}}{\pgfqpoint{3.982203in}{1.842170in}}%
\pgfpathcurveto{\pgfqpoint{3.974389in}{1.834356in}}{\pgfqpoint{3.969999in}{1.823757in}}{\pgfqpoint{3.969999in}{1.812707in}}%
\pgfpathcurveto{\pgfqpoint{3.969999in}{1.801657in}}{\pgfqpoint{3.974389in}{1.791058in}}{\pgfqpoint{3.982203in}{1.783244in}}%
\pgfpathcurveto{\pgfqpoint{3.990016in}{1.775431in}}{\pgfqpoint{4.000616in}{1.771040in}}{\pgfqpoint{4.011666in}{1.771040in}}%
\pgfpathclose%
\pgfusepath{stroke,fill}%
\end{pgfscope}%
\begin{pgfscope}%
\pgfpathrectangle{\pgfqpoint{0.800000in}{0.528000in}}{\pgfqpoint{4.960000in}{3.696000in}}%
\pgfusepath{clip}%
\pgfsetbuttcap%
\pgfsetroundjoin%
\definecolor{currentfill}{rgb}{0.000000,0.000000,0.000000}%
\pgfsetfillcolor{currentfill}%
\pgfsetlinewidth{1.003750pt}%
\definecolor{currentstroke}{rgb}{0.000000,0.000000,0.000000}%
\pgfsetstrokecolor{currentstroke}%
\pgfsetdash{}{0pt}%
\pgfpathmoveto{\pgfqpoint{4.011666in}{2.877687in}}%
\pgfpathcurveto{\pgfqpoint{4.022716in}{2.877687in}}{\pgfqpoint{4.033315in}{2.882077in}}{\pgfqpoint{4.041128in}{2.889891in}}%
\pgfpathcurveto{\pgfqpoint{4.048942in}{2.897704in}}{\pgfqpoint{4.053332in}{2.908303in}}{\pgfqpoint{4.053332in}{2.919353in}}%
\pgfpathcurveto{\pgfqpoint{4.053332in}{2.930404in}}{\pgfqpoint{4.048942in}{2.941003in}}{\pgfqpoint{4.041128in}{2.948816in}}%
\pgfpathcurveto{\pgfqpoint{4.033315in}{2.956630in}}{\pgfqpoint{4.022716in}{2.961020in}}{\pgfqpoint{4.011666in}{2.961020in}}%
\pgfpathcurveto{\pgfqpoint{4.000616in}{2.961020in}}{\pgfqpoint{3.990016in}{2.956630in}}{\pgfqpoint{3.982203in}{2.948816in}}%
\pgfpathcurveto{\pgfqpoint{3.974389in}{2.941003in}}{\pgfqpoint{3.969999in}{2.930404in}}{\pgfqpoint{3.969999in}{2.919353in}}%
\pgfpathcurveto{\pgfqpoint{3.969999in}{2.908303in}}{\pgfqpoint{3.974389in}{2.897704in}}{\pgfqpoint{3.982203in}{2.889891in}}%
\pgfpathcurveto{\pgfqpoint{3.990016in}{2.882077in}}{\pgfqpoint{4.000616in}{2.877687in}}{\pgfqpoint{4.011666in}{2.877687in}}%
\pgfpathclose%
\pgfusepath{stroke,fill}%
\end{pgfscope}%
\begin{pgfscope}%
\pgfpathrectangle{\pgfqpoint{0.800000in}{0.528000in}}{\pgfqpoint{4.960000in}{3.696000in}}%
\pgfusepath{clip}%
\pgfsetbuttcap%
\pgfsetroundjoin%
\definecolor{currentfill}{rgb}{0.000000,0.000000,0.000000}%
\pgfsetfillcolor{currentfill}%
\pgfsetlinewidth{1.003750pt}%
\definecolor{currentstroke}{rgb}{0.000000,0.000000,0.000000}%
\pgfsetstrokecolor{currentstroke}%
\pgfsetdash{}{0pt}%
\pgfpathmoveto{\pgfqpoint{4.011666in}{2.877687in}}%
\pgfpathcurveto{\pgfqpoint{4.022716in}{2.877687in}}{\pgfqpoint{4.033315in}{2.882077in}}{\pgfqpoint{4.041128in}{2.889891in}}%
\pgfpathcurveto{\pgfqpoint{4.048942in}{2.897704in}}{\pgfqpoint{4.053332in}{2.908303in}}{\pgfqpoint{4.053332in}{2.919353in}}%
\pgfpathcurveto{\pgfqpoint{4.053332in}{2.930404in}}{\pgfqpoint{4.048942in}{2.941003in}}{\pgfqpoint{4.041128in}{2.948816in}}%
\pgfpathcurveto{\pgfqpoint{4.033315in}{2.956630in}}{\pgfqpoint{4.022716in}{2.961020in}}{\pgfqpoint{4.011666in}{2.961020in}}%
\pgfpathcurveto{\pgfqpoint{4.000616in}{2.961020in}}{\pgfqpoint{3.990016in}{2.956630in}}{\pgfqpoint{3.982203in}{2.948816in}}%
\pgfpathcurveto{\pgfqpoint{3.974389in}{2.941003in}}{\pgfqpoint{3.969999in}{2.930404in}}{\pgfqpoint{3.969999in}{2.919353in}}%
\pgfpathcurveto{\pgfqpoint{3.969999in}{2.908303in}}{\pgfqpoint{3.974389in}{2.897704in}}{\pgfqpoint{3.982203in}{2.889891in}}%
\pgfpathcurveto{\pgfqpoint{3.990016in}{2.882077in}}{\pgfqpoint{4.000616in}{2.877687in}}{\pgfqpoint{4.011666in}{2.877687in}}%
\pgfpathclose%
\pgfusepath{stroke,fill}%
\end{pgfscope}%
\begin{pgfscope}%
\pgfpathrectangle{\pgfqpoint{0.800000in}{0.528000in}}{\pgfqpoint{4.960000in}{3.696000in}}%
\pgfusepath{clip}%
\pgfsetbuttcap%
\pgfsetroundjoin%
\definecolor{currentfill}{rgb}{0.000000,0.000000,0.000000}%
\pgfsetfillcolor{currentfill}%
\pgfsetlinewidth{1.003750pt}%
\definecolor{currentstroke}{rgb}{0.000000,0.000000,0.000000}%
\pgfsetstrokecolor{currentstroke}%
\pgfsetdash{}{0pt}%
\pgfpathmoveto{\pgfqpoint{4.011666in}{1.771040in}}%
\pgfpathcurveto{\pgfqpoint{4.022716in}{1.771040in}}{\pgfqpoint{4.033315in}{1.775431in}}{\pgfqpoint{4.041128in}{1.783244in}}%
\pgfpathcurveto{\pgfqpoint{4.048942in}{1.791058in}}{\pgfqpoint{4.053332in}{1.801657in}}{\pgfqpoint{4.053332in}{1.812707in}}%
\pgfpathcurveto{\pgfqpoint{4.053332in}{1.823757in}}{\pgfqpoint{4.048942in}{1.834356in}}{\pgfqpoint{4.041128in}{1.842170in}}%
\pgfpathcurveto{\pgfqpoint{4.033315in}{1.849983in}}{\pgfqpoint{4.022716in}{1.854374in}}{\pgfqpoint{4.011666in}{1.854374in}}%
\pgfpathcurveto{\pgfqpoint{4.000616in}{1.854374in}}{\pgfqpoint{3.990016in}{1.849983in}}{\pgfqpoint{3.982203in}{1.842170in}}%
\pgfpathcurveto{\pgfqpoint{3.974389in}{1.834356in}}{\pgfqpoint{3.969999in}{1.823757in}}{\pgfqpoint{3.969999in}{1.812707in}}%
\pgfpathcurveto{\pgfqpoint{3.969999in}{1.801657in}}{\pgfqpoint{3.974389in}{1.791058in}}{\pgfqpoint{3.982203in}{1.783244in}}%
\pgfpathcurveto{\pgfqpoint{3.990016in}{1.775431in}}{\pgfqpoint{4.000616in}{1.771040in}}{\pgfqpoint{4.011666in}{1.771040in}}%
\pgfpathclose%
\pgfusepath{stroke,fill}%
\end{pgfscope}%
\begin{pgfscope}%
\pgfpathrectangle{\pgfqpoint{0.800000in}{0.528000in}}{\pgfqpoint{4.960000in}{3.696000in}}%
\pgfusepath{clip}%
\pgfsetbuttcap%
\pgfsetroundjoin%
\definecolor{currentfill}{rgb}{0.000000,0.000000,0.000000}%
\pgfsetfillcolor{currentfill}%
\pgfsetlinewidth{1.003750pt}%
\definecolor{currentstroke}{rgb}{0.000000,0.000000,0.000000}%
\pgfsetstrokecolor{currentstroke}%
\pgfsetdash{}{0pt}%
\pgfpathmoveto{\pgfqpoint{4.011666in}{2.877687in}}%
\pgfpathcurveto{\pgfqpoint{4.022716in}{2.877687in}}{\pgfqpoint{4.033315in}{2.882077in}}{\pgfqpoint{4.041128in}{2.889891in}}%
\pgfpathcurveto{\pgfqpoint{4.048942in}{2.897704in}}{\pgfqpoint{4.053332in}{2.908303in}}{\pgfqpoint{4.053332in}{2.919353in}}%
\pgfpathcurveto{\pgfqpoint{4.053332in}{2.930404in}}{\pgfqpoint{4.048942in}{2.941003in}}{\pgfqpoint{4.041128in}{2.948816in}}%
\pgfpathcurveto{\pgfqpoint{4.033315in}{2.956630in}}{\pgfqpoint{4.022716in}{2.961020in}}{\pgfqpoint{4.011666in}{2.961020in}}%
\pgfpathcurveto{\pgfqpoint{4.000616in}{2.961020in}}{\pgfqpoint{3.990016in}{2.956630in}}{\pgfqpoint{3.982203in}{2.948816in}}%
\pgfpathcurveto{\pgfqpoint{3.974389in}{2.941003in}}{\pgfqpoint{3.969999in}{2.930404in}}{\pgfqpoint{3.969999in}{2.919353in}}%
\pgfpathcurveto{\pgfqpoint{3.969999in}{2.908303in}}{\pgfqpoint{3.974389in}{2.897704in}}{\pgfqpoint{3.982203in}{2.889891in}}%
\pgfpathcurveto{\pgfqpoint{3.990016in}{2.882077in}}{\pgfqpoint{4.000616in}{2.877687in}}{\pgfqpoint{4.011666in}{2.877687in}}%
\pgfpathclose%
\pgfusepath{stroke,fill}%
\end{pgfscope}%
\begin{pgfscope}%
\pgfpathrectangle{\pgfqpoint{0.800000in}{0.528000in}}{\pgfqpoint{4.960000in}{3.696000in}}%
\pgfusepath{clip}%
\pgfsetbuttcap%
\pgfsetroundjoin%
\definecolor{currentfill}{rgb}{0.000000,0.000000,0.000000}%
\pgfsetfillcolor{currentfill}%
\pgfsetlinewidth{1.003750pt}%
\definecolor{currentstroke}{rgb}{0.000000,0.000000,0.000000}%
\pgfsetstrokecolor{currentstroke}%
\pgfsetdash{}{0pt}%
\pgfpathmoveto{\pgfqpoint{4.011666in}{1.771040in}}%
\pgfpathcurveto{\pgfqpoint{4.022716in}{1.771040in}}{\pgfqpoint{4.033315in}{1.775431in}}{\pgfqpoint{4.041128in}{1.783244in}}%
\pgfpathcurveto{\pgfqpoint{4.048942in}{1.791058in}}{\pgfqpoint{4.053332in}{1.801657in}}{\pgfqpoint{4.053332in}{1.812707in}}%
\pgfpathcurveto{\pgfqpoint{4.053332in}{1.823757in}}{\pgfqpoint{4.048942in}{1.834356in}}{\pgfqpoint{4.041128in}{1.842170in}}%
\pgfpathcurveto{\pgfqpoint{4.033315in}{1.849983in}}{\pgfqpoint{4.022716in}{1.854374in}}{\pgfqpoint{4.011666in}{1.854374in}}%
\pgfpathcurveto{\pgfqpoint{4.000616in}{1.854374in}}{\pgfqpoint{3.990016in}{1.849983in}}{\pgfqpoint{3.982203in}{1.842170in}}%
\pgfpathcurveto{\pgfqpoint{3.974389in}{1.834356in}}{\pgfqpoint{3.969999in}{1.823757in}}{\pgfqpoint{3.969999in}{1.812707in}}%
\pgfpathcurveto{\pgfqpoint{3.969999in}{1.801657in}}{\pgfqpoint{3.974389in}{1.791058in}}{\pgfqpoint{3.982203in}{1.783244in}}%
\pgfpathcurveto{\pgfqpoint{3.990016in}{1.775431in}}{\pgfqpoint{4.000616in}{1.771040in}}{\pgfqpoint{4.011666in}{1.771040in}}%
\pgfpathclose%
\pgfusepath{stroke,fill}%
\end{pgfscope}%
\begin{pgfscope}%
\pgfpathrectangle{\pgfqpoint{0.800000in}{0.528000in}}{\pgfqpoint{4.960000in}{3.696000in}}%
\pgfusepath{clip}%
\pgfsetbuttcap%
\pgfsetroundjoin%
\definecolor{currentfill}{rgb}{0.000000,0.000000,0.000000}%
\pgfsetfillcolor{currentfill}%
\pgfsetlinewidth{1.003750pt}%
\definecolor{currentstroke}{rgb}{0.000000,0.000000,0.000000}%
\pgfsetstrokecolor{currentstroke}%
\pgfsetdash{}{0pt}%
\pgfpathmoveto{\pgfqpoint{4.011666in}{1.771040in}}%
\pgfpathcurveto{\pgfqpoint{4.022716in}{1.771040in}}{\pgfqpoint{4.033315in}{1.775431in}}{\pgfqpoint{4.041128in}{1.783244in}}%
\pgfpathcurveto{\pgfqpoint{4.048942in}{1.791058in}}{\pgfqpoint{4.053332in}{1.801657in}}{\pgfqpoint{4.053332in}{1.812707in}}%
\pgfpathcurveto{\pgfqpoint{4.053332in}{1.823757in}}{\pgfqpoint{4.048942in}{1.834356in}}{\pgfqpoint{4.041128in}{1.842170in}}%
\pgfpathcurveto{\pgfqpoint{4.033315in}{1.849983in}}{\pgfqpoint{4.022716in}{1.854374in}}{\pgfqpoint{4.011666in}{1.854374in}}%
\pgfpathcurveto{\pgfqpoint{4.000616in}{1.854374in}}{\pgfqpoint{3.990016in}{1.849983in}}{\pgfqpoint{3.982203in}{1.842170in}}%
\pgfpathcurveto{\pgfqpoint{3.974389in}{1.834356in}}{\pgfqpoint{3.969999in}{1.823757in}}{\pgfqpoint{3.969999in}{1.812707in}}%
\pgfpathcurveto{\pgfqpoint{3.969999in}{1.801657in}}{\pgfqpoint{3.974389in}{1.791058in}}{\pgfqpoint{3.982203in}{1.783244in}}%
\pgfpathcurveto{\pgfqpoint{3.990016in}{1.775431in}}{\pgfqpoint{4.000616in}{1.771040in}}{\pgfqpoint{4.011666in}{1.771040in}}%
\pgfpathclose%
\pgfusepath{stroke,fill}%
\end{pgfscope}%
\begin{pgfscope}%
\pgfpathrectangle{\pgfqpoint{0.800000in}{0.528000in}}{\pgfqpoint{4.960000in}{3.696000in}}%
\pgfusepath{clip}%
\pgfsetbuttcap%
\pgfsetroundjoin%
\definecolor{currentfill}{rgb}{0.000000,0.000000,0.000000}%
\pgfsetfillcolor{currentfill}%
\pgfsetlinewidth{1.003750pt}%
\definecolor{currentstroke}{rgb}{0.000000,0.000000,0.000000}%
\pgfsetstrokecolor{currentstroke}%
\pgfsetdash{}{0pt}%
\pgfpathmoveto{\pgfqpoint{4.011666in}{1.771040in}}%
\pgfpathcurveto{\pgfqpoint{4.022716in}{1.771040in}}{\pgfqpoint{4.033315in}{1.775431in}}{\pgfqpoint{4.041128in}{1.783244in}}%
\pgfpathcurveto{\pgfqpoint{4.048942in}{1.791058in}}{\pgfqpoint{4.053332in}{1.801657in}}{\pgfqpoint{4.053332in}{1.812707in}}%
\pgfpathcurveto{\pgfqpoint{4.053332in}{1.823757in}}{\pgfqpoint{4.048942in}{1.834356in}}{\pgfqpoint{4.041128in}{1.842170in}}%
\pgfpathcurveto{\pgfqpoint{4.033315in}{1.849983in}}{\pgfqpoint{4.022716in}{1.854374in}}{\pgfqpoint{4.011666in}{1.854374in}}%
\pgfpathcurveto{\pgfqpoint{4.000616in}{1.854374in}}{\pgfqpoint{3.990016in}{1.849983in}}{\pgfqpoint{3.982203in}{1.842170in}}%
\pgfpathcurveto{\pgfqpoint{3.974389in}{1.834356in}}{\pgfqpoint{3.969999in}{1.823757in}}{\pgfqpoint{3.969999in}{1.812707in}}%
\pgfpathcurveto{\pgfqpoint{3.969999in}{1.801657in}}{\pgfqpoint{3.974389in}{1.791058in}}{\pgfqpoint{3.982203in}{1.783244in}}%
\pgfpathcurveto{\pgfqpoint{3.990016in}{1.775431in}}{\pgfqpoint{4.000616in}{1.771040in}}{\pgfqpoint{4.011666in}{1.771040in}}%
\pgfpathclose%
\pgfusepath{stroke,fill}%
\end{pgfscope}%
\begin{pgfscope}%
\pgfpathrectangle{\pgfqpoint{0.800000in}{0.528000in}}{\pgfqpoint{4.960000in}{3.696000in}}%
\pgfusepath{clip}%
\pgfsetbuttcap%
\pgfsetroundjoin%
\definecolor{currentfill}{rgb}{0.000000,0.000000,0.000000}%
\pgfsetfillcolor{currentfill}%
\pgfsetlinewidth{1.003750pt}%
\definecolor{currentstroke}{rgb}{0.000000,0.000000,0.000000}%
\pgfsetstrokecolor{currentstroke}%
\pgfsetdash{}{0pt}%
\pgfpathmoveto{\pgfqpoint{4.011666in}{1.771040in}}%
\pgfpathcurveto{\pgfqpoint{4.022716in}{1.771040in}}{\pgfqpoint{4.033315in}{1.775431in}}{\pgfqpoint{4.041128in}{1.783244in}}%
\pgfpathcurveto{\pgfqpoint{4.048942in}{1.791058in}}{\pgfqpoint{4.053332in}{1.801657in}}{\pgfqpoint{4.053332in}{1.812707in}}%
\pgfpathcurveto{\pgfqpoint{4.053332in}{1.823757in}}{\pgfqpoint{4.048942in}{1.834356in}}{\pgfqpoint{4.041128in}{1.842170in}}%
\pgfpathcurveto{\pgfqpoint{4.033315in}{1.849983in}}{\pgfqpoint{4.022716in}{1.854374in}}{\pgfqpoint{4.011666in}{1.854374in}}%
\pgfpathcurveto{\pgfqpoint{4.000616in}{1.854374in}}{\pgfqpoint{3.990016in}{1.849983in}}{\pgfqpoint{3.982203in}{1.842170in}}%
\pgfpathcurveto{\pgfqpoint{3.974389in}{1.834356in}}{\pgfqpoint{3.969999in}{1.823757in}}{\pgfqpoint{3.969999in}{1.812707in}}%
\pgfpathcurveto{\pgfqpoint{3.969999in}{1.801657in}}{\pgfqpoint{3.974389in}{1.791058in}}{\pgfqpoint{3.982203in}{1.783244in}}%
\pgfpathcurveto{\pgfqpoint{3.990016in}{1.775431in}}{\pgfqpoint{4.000616in}{1.771040in}}{\pgfqpoint{4.011666in}{1.771040in}}%
\pgfpathclose%
\pgfusepath{stroke,fill}%
\end{pgfscope}%
\begin{pgfscope}%
\pgfpathrectangle{\pgfqpoint{0.800000in}{0.528000in}}{\pgfqpoint{4.960000in}{3.696000in}}%
\pgfusepath{clip}%
\pgfsetbuttcap%
\pgfsetroundjoin%
\definecolor{currentfill}{rgb}{0.000000,0.000000,0.000000}%
\pgfsetfillcolor{currentfill}%
\pgfsetlinewidth{1.003750pt}%
\definecolor{currentstroke}{rgb}{0.000000,0.000000,0.000000}%
\pgfsetstrokecolor{currentstroke}%
\pgfsetdash{}{0pt}%
\pgfpathmoveto{\pgfqpoint{4.011666in}{2.877687in}}%
\pgfpathcurveto{\pgfqpoint{4.022716in}{2.877687in}}{\pgfqpoint{4.033315in}{2.882077in}}{\pgfqpoint{4.041128in}{2.889891in}}%
\pgfpathcurveto{\pgfqpoint{4.048942in}{2.897704in}}{\pgfqpoint{4.053332in}{2.908303in}}{\pgfqpoint{4.053332in}{2.919353in}}%
\pgfpathcurveto{\pgfqpoint{4.053332in}{2.930404in}}{\pgfqpoint{4.048942in}{2.941003in}}{\pgfqpoint{4.041128in}{2.948816in}}%
\pgfpathcurveto{\pgfqpoint{4.033315in}{2.956630in}}{\pgfqpoint{4.022716in}{2.961020in}}{\pgfqpoint{4.011666in}{2.961020in}}%
\pgfpathcurveto{\pgfqpoint{4.000616in}{2.961020in}}{\pgfqpoint{3.990016in}{2.956630in}}{\pgfqpoint{3.982203in}{2.948816in}}%
\pgfpathcurveto{\pgfqpoint{3.974389in}{2.941003in}}{\pgfqpoint{3.969999in}{2.930404in}}{\pgfqpoint{3.969999in}{2.919353in}}%
\pgfpathcurveto{\pgfqpoint{3.969999in}{2.908303in}}{\pgfqpoint{3.974389in}{2.897704in}}{\pgfqpoint{3.982203in}{2.889891in}}%
\pgfpathcurveto{\pgfqpoint{3.990016in}{2.882077in}}{\pgfqpoint{4.000616in}{2.877687in}}{\pgfqpoint{4.011666in}{2.877687in}}%
\pgfpathclose%
\pgfusepath{stroke,fill}%
\end{pgfscope}%
\begin{pgfscope}%
\pgfpathrectangle{\pgfqpoint{0.800000in}{0.528000in}}{\pgfqpoint{4.960000in}{3.696000in}}%
\pgfusepath{clip}%
\pgfsetbuttcap%
\pgfsetroundjoin%
\definecolor{currentfill}{rgb}{0.000000,0.000000,0.000000}%
\pgfsetfillcolor{currentfill}%
\pgfsetlinewidth{1.003750pt}%
\definecolor{currentstroke}{rgb}{0.000000,0.000000,0.000000}%
\pgfsetstrokecolor{currentstroke}%
\pgfsetdash{}{0pt}%
\pgfpathmoveto{\pgfqpoint{4.011666in}{1.771040in}}%
\pgfpathcurveto{\pgfqpoint{4.022716in}{1.771040in}}{\pgfqpoint{4.033315in}{1.775431in}}{\pgfqpoint{4.041128in}{1.783244in}}%
\pgfpathcurveto{\pgfqpoint{4.048942in}{1.791058in}}{\pgfqpoint{4.053332in}{1.801657in}}{\pgfqpoint{4.053332in}{1.812707in}}%
\pgfpathcurveto{\pgfqpoint{4.053332in}{1.823757in}}{\pgfqpoint{4.048942in}{1.834356in}}{\pgfqpoint{4.041128in}{1.842170in}}%
\pgfpathcurveto{\pgfqpoint{4.033315in}{1.849983in}}{\pgfqpoint{4.022716in}{1.854374in}}{\pgfqpoint{4.011666in}{1.854374in}}%
\pgfpathcurveto{\pgfqpoint{4.000616in}{1.854374in}}{\pgfqpoint{3.990016in}{1.849983in}}{\pgfqpoint{3.982203in}{1.842170in}}%
\pgfpathcurveto{\pgfqpoint{3.974389in}{1.834356in}}{\pgfqpoint{3.969999in}{1.823757in}}{\pgfqpoint{3.969999in}{1.812707in}}%
\pgfpathcurveto{\pgfqpoint{3.969999in}{1.801657in}}{\pgfqpoint{3.974389in}{1.791058in}}{\pgfqpoint{3.982203in}{1.783244in}}%
\pgfpathcurveto{\pgfqpoint{3.990016in}{1.775431in}}{\pgfqpoint{4.000616in}{1.771040in}}{\pgfqpoint{4.011666in}{1.771040in}}%
\pgfpathclose%
\pgfusepath{stroke,fill}%
\end{pgfscope}%
\begin{pgfscope}%
\pgfpathrectangle{\pgfqpoint{0.800000in}{0.528000in}}{\pgfqpoint{4.960000in}{3.696000in}}%
\pgfusepath{clip}%
\pgfsetbuttcap%
\pgfsetroundjoin%
\definecolor{currentfill}{rgb}{0.000000,0.000000,0.000000}%
\pgfsetfillcolor{currentfill}%
\pgfsetlinewidth{1.003750pt}%
\definecolor{currentstroke}{rgb}{0.000000,0.000000,0.000000}%
\pgfsetstrokecolor{currentstroke}%
\pgfsetdash{}{0pt}%
\pgfpathmoveto{\pgfqpoint{4.011666in}{1.771040in}}%
\pgfpathcurveto{\pgfqpoint{4.022716in}{1.771040in}}{\pgfqpoint{4.033315in}{1.775431in}}{\pgfqpoint{4.041128in}{1.783244in}}%
\pgfpathcurveto{\pgfqpoint{4.048942in}{1.791058in}}{\pgfqpoint{4.053332in}{1.801657in}}{\pgfqpoint{4.053332in}{1.812707in}}%
\pgfpathcurveto{\pgfqpoint{4.053332in}{1.823757in}}{\pgfqpoint{4.048942in}{1.834356in}}{\pgfqpoint{4.041128in}{1.842170in}}%
\pgfpathcurveto{\pgfqpoint{4.033315in}{1.849983in}}{\pgfqpoint{4.022716in}{1.854374in}}{\pgfqpoint{4.011666in}{1.854374in}}%
\pgfpathcurveto{\pgfqpoint{4.000616in}{1.854374in}}{\pgfqpoint{3.990016in}{1.849983in}}{\pgfqpoint{3.982203in}{1.842170in}}%
\pgfpathcurveto{\pgfqpoint{3.974389in}{1.834356in}}{\pgfqpoint{3.969999in}{1.823757in}}{\pgfqpoint{3.969999in}{1.812707in}}%
\pgfpathcurveto{\pgfqpoint{3.969999in}{1.801657in}}{\pgfqpoint{3.974389in}{1.791058in}}{\pgfqpoint{3.982203in}{1.783244in}}%
\pgfpathcurveto{\pgfqpoint{3.990016in}{1.775431in}}{\pgfqpoint{4.000616in}{1.771040in}}{\pgfqpoint{4.011666in}{1.771040in}}%
\pgfpathclose%
\pgfusepath{stroke,fill}%
\end{pgfscope}%
\begin{pgfscope}%
\pgfpathrectangle{\pgfqpoint{0.800000in}{0.528000in}}{\pgfqpoint{4.960000in}{3.696000in}}%
\pgfusepath{clip}%
\pgfsetbuttcap%
\pgfsetroundjoin%
\definecolor{currentfill}{rgb}{0.000000,0.000000,0.000000}%
\pgfsetfillcolor{currentfill}%
\pgfsetlinewidth{1.003750pt}%
\definecolor{currentstroke}{rgb}{0.000000,0.000000,0.000000}%
\pgfsetstrokecolor{currentstroke}%
\pgfsetdash{}{0pt}%
\pgfpathmoveto{\pgfqpoint{4.011666in}{1.771040in}}%
\pgfpathcurveto{\pgfqpoint{4.022716in}{1.771040in}}{\pgfqpoint{4.033315in}{1.775431in}}{\pgfqpoint{4.041128in}{1.783244in}}%
\pgfpathcurveto{\pgfqpoint{4.048942in}{1.791058in}}{\pgfqpoint{4.053332in}{1.801657in}}{\pgfqpoint{4.053332in}{1.812707in}}%
\pgfpathcurveto{\pgfqpoint{4.053332in}{1.823757in}}{\pgfqpoint{4.048942in}{1.834356in}}{\pgfqpoint{4.041128in}{1.842170in}}%
\pgfpathcurveto{\pgfqpoint{4.033315in}{1.849983in}}{\pgfqpoint{4.022716in}{1.854374in}}{\pgfqpoint{4.011666in}{1.854374in}}%
\pgfpathcurveto{\pgfqpoint{4.000616in}{1.854374in}}{\pgfqpoint{3.990016in}{1.849983in}}{\pgfqpoint{3.982203in}{1.842170in}}%
\pgfpathcurveto{\pgfqpoint{3.974389in}{1.834356in}}{\pgfqpoint{3.969999in}{1.823757in}}{\pgfqpoint{3.969999in}{1.812707in}}%
\pgfpathcurveto{\pgfqpoint{3.969999in}{1.801657in}}{\pgfqpoint{3.974389in}{1.791058in}}{\pgfqpoint{3.982203in}{1.783244in}}%
\pgfpathcurveto{\pgfqpoint{3.990016in}{1.775431in}}{\pgfqpoint{4.000616in}{1.771040in}}{\pgfqpoint{4.011666in}{1.771040in}}%
\pgfpathclose%
\pgfusepath{stroke,fill}%
\end{pgfscope}%
\begin{pgfscope}%
\pgfpathrectangle{\pgfqpoint{0.800000in}{0.528000in}}{\pgfqpoint{4.960000in}{3.696000in}}%
\pgfusepath{clip}%
\pgfsetbuttcap%
\pgfsetroundjoin%
\definecolor{currentfill}{rgb}{0.000000,0.000000,0.000000}%
\pgfsetfillcolor{currentfill}%
\pgfsetlinewidth{1.003750pt}%
\definecolor{currentstroke}{rgb}{0.000000,0.000000,0.000000}%
\pgfsetstrokecolor{currentstroke}%
\pgfsetdash{}{0pt}%
\pgfpathmoveto{\pgfqpoint{4.011666in}{1.771040in}}%
\pgfpathcurveto{\pgfqpoint{4.022716in}{1.771040in}}{\pgfqpoint{4.033315in}{1.775431in}}{\pgfqpoint{4.041128in}{1.783244in}}%
\pgfpathcurveto{\pgfqpoint{4.048942in}{1.791058in}}{\pgfqpoint{4.053332in}{1.801657in}}{\pgfqpoint{4.053332in}{1.812707in}}%
\pgfpathcurveto{\pgfqpoint{4.053332in}{1.823757in}}{\pgfqpoint{4.048942in}{1.834356in}}{\pgfqpoint{4.041128in}{1.842170in}}%
\pgfpathcurveto{\pgfqpoint{4.033315in}{1.849983in}}{\pgfqpoint{4.022716in}{1.854374in}}{\pgfqpoint{4.011666in}{1.854374in}}%
\pgfpathcurveto{\pgfqpoint{4.000616in}{1.854374in}}{\pgfqpoint{3.990016in}{1.849983in}}{\pgfqpoint{3.982203in}{1.842170in}}%
\pgfpathcurveto{\pgfqpoint{3.974389in}{1.834356in}}{\pgfqpoint{3.969999in}{1.823757in}}{\pgfqpoint{3.969999in}{1.812707in}}%
\pgfpathcurveto{\pgfqpoint{3.969999in}{1.801657in}}{\pgfqpoint{3.974389in}{1.791058in}}{\pgfqpoint{3.982203in}{1.783244in}}%
\pgfpathcurveto{\pgfqpoint{3.990016in}{1.775431in}}{\pgfqpoint{4.000616in}{1.771040in}}{\pgfqpoint{4.011666in}{1.771040in}}%
\pgfpathclose%
\pgfusepath{stroke,fill}%
\end{pgfscope}%
\begin{pgfscope}%
\pgfpathrectangle{\pgfqpoint{0.800000in}{0.528000in}}{\pgfqpoint{4.960000in}{3.696000in}}%
\pgfusepath{clip}%
\pgfsetbuttcap%
\pgfsetroundjoin%
\definecolor{currentfill}{rgb}{0.000000,0.000000,0.000000}%
\pgfsetfillcolor{currentfill}%
\pgfsetlinewidth{1.003750pt}%
\definecolor{currentstroke}{rgb}{0.000000,0.000000,0.000000}%
\pgfsetstrokecolor{currentstroke}%
\pgfsetdash{}{0pt}%
\pgfpathmoveto{\pgfqpoint{4.011666in}{1.771040in}}%
\pgfpathcurveto{\pgfqpoint{4.022716in}{1.771040in}}{\pgfqpoint{4.033315in}{1.775431in}}{\pgfqpoint{4.041128in}{1.783244in}}%
\pgfpathcurveto{\pgfqpoint{4.048942in}{1.791058in}}{\pgfqpoint{4.053332in}{1.801657in}}{\pgfqpoint{4.053332in}{1.812707in}}%
\pgfpathcurveto{\pgfqpoint{4.053332in}{1.823757in}}{\pgfqpoint{4.048942in}{1.834356in}}{\pgfqpoint{4.041128in}{1.842170in}}%
\pgfpathcurveto{\pgfqpoint{4.033315in}{1.849983in}}{\pgfqpoint{4.022716in}{1.854374in}}{\pgfqpoint{4.011666in}{1.854374in}}%
\pgfpathcurveto{\pgfqpoint{4.000616in}{1.854374in}}{\pgfqpoint{3.990016in}{1.849983in}}{\pgfqpoint{3.982203in}{1.842170in}}%
\pgfpathcurveto{\pgfqpoint{3.974389in}{1.834356in}}{\pgfqpoint{3.969999in}{1.823757in}}{\pgfqpoint{3.969999in}{1.812707in}}%
\pgfpathcurveto{\pgfqpoint{3.969999in}{1.801657in}}{\pgfqpoint{3.974389in}{1.791058in}}{\pgfqpoint{3.982203in}{1.783244in}}%
\pgfpathcurveto{\pgfqpoint{3.990016in}{1.775431in}}{\pgfqpoint{4.000616in}{1.771040in}}{\pgfqpoint{4.011666in}{1.771040in}}%
\pgfpathclose%
\pgfusepath{stroke,fill}%
\end{pgfscope}%
\begin{pgfscope}%
\pgfpathrectangle{\pgfqpoint{0.800000in}{0.528000in}}{\pgfqpoint{4.960000in}{3.696000in}}%
\pgfusepath{clip}%
\pgfsetbuttcap%
\pgfsetroundjoin%
\definecolor{currentfill}{rgb}{0.000000,0.000000,0.000000}%
\pgfsetfillcolor{currentfill}%
\pgfsetlinewidth{1.003750pt}%
\definecolor{currentstroke}{rgb}{0.000000,0.000000,0.000000}%
\pgfsetstrokecolor{currentstroke}%
\pgfsetdash{}{0pt}%
\pgfpathmoveto{\pgfqpoint{4.011666in}{1.771040in}}%
\pgfpathcurveto{\pgfqpoint{4.022716in}{1.771040in}}{\pgfqpoint{4.033315in}{1.775431in}}{\pgfqpoint{4.041128in}{1.783244in}}%
\pgfpathcurveto{\pgfqpoint{4.048942in}{1.791058in}}{\pgfqpoint{4.053332in}{1.801657in}}{\pgfqpoint{4.053332in}{1.812707in}}%
\pgfpathcurveto{\pgfqpoint{4.053332in}{1.823757in}}{\pgfqpoint{4.048942in}{1.834356in}}{\pgfqpoint{4.041128in}{1.842170in}}%
\pgfpathcurveto{\pgfqpoint{4.033315in}{1.849983in}}{\pgfqpoint{4.022716in}{1.854374in}}{\pgfqpoint{4.011666in}{1.854374in}}%
\pgfpathcurveto{\pgfqpoint{4.000616in}{1.854374in}}{\pgfqpoint{3.990016in}{1.849983in}}{\pgfqpoint{3.982203in}{1.842170in}}%
\pgfpathcurveto{\pgfqpoint{3.974389in}{1.834356in}}{\pgfqpoint{3.969999in}{1.823757in}}{\pgfqpoint{3.969999in}{1.812707in}}%
\pgfpathcurveto{\pgfqpoint{3.969999in}{1.801657in}}{\pgfqpoint{3.974389in}{1.791058in}}{\pgfqpoint{3.982203in}{1.783244in}}%
\pgfpathcurveto{\pgfqpoint{3.990016in}{1.775431in}}{\pgfqpoint{4.000616in}{1.771040in}}{\pgfqpoint{4.011666in}{1.771040in}}%
\pgfpathclose%
\pgfusepath{stroke,fill}%
\end{pgfscope}%
\begin{pgfscope}%
\pgfpathrectangle{\pgfqpoint{0.800000in}{0.528000in}}{\pgfqpoint{4.960000in}{3.696000in}}%
\pgfusepath{clip}%
\pgfsetbuttcap%
\pgfsetroundjoin%
\definecolor{currentfill}{rgb}{0.000000,0.000000,0.000000}%
\pgfsetfillcolor{currentfill}%
\pgfsetlinewidth{1.003750pt}%
\definecolor{currentstroke}{rgb}{0.000000,0.000000,0.000000}%
\pgfsetstrokecolor{currentstroke}%
\pgfsetdash{}{0pt}%
\pgfpathmoveto{\pgfqpoint{4.011666in}{2.877687in}}%
\pgfpathcurveto{\pgfqpoint{4.022716in}{2.877687in}}{\pgfqpoint{4.033315in}{2.882077in}}{\pgfqpoint{4.041128in}{2.889891in}}%
\pgfpathcurveto{\pgfqpoint{4.048942in}{2.897704in}}{\pgfqpoint{4.053332in}{2.908303in}}{\pgfqpoint{4.053332in}{2.919353in}}%
\pgfpathcurveto{\pgfqpoint{4.053332in}{2.930404in}}{\pgfqpoint{4.048942in}{2.941003in}}{\pgfqpoint{4.041128in}{2.948816in}}%
\pgfpathcurveto{\pgfqpoint{4.033315in}{2.956630in}}{\pgfqpoint{4.022716in}{2.961020in}}{\pgfqpoint{4.011666in}{2.961020in}}%
\pgfpathcurveto{\pgfqpoint{4.000616in}{2.961020in}}{\pgfqpoint{3.990016in}{2.956630in}}{\pgfqpoint{3.982203in}{2.948816in}}%
\pgfpathcurveto{\pgfqpoint{3.974389in}{2.941003in}}{\pgfqpoint{3.969999in}{2.930404in}}{\pgfqpoint{3.969999in}{2.919353in}}%
\pgfpathcurveto{\pgfqpoint{3.969999in}{2.908303in}}{\pgfqpoint{3.974389in}{2.897704in}}{\pgfqpoint{3.982203in}{2.889891in}}%
\pgfpathcurveto{\pgfqpoint{3.990016in}{2.882077in}}{\pgfqpoint{4.000616in}{2.877687in}}{\pgfqpoint{4.011666in}{2.877687in}}%
\pgfpathclose%
\pgfusepath{stroke,fill}%
\end{pgfscope}%
\begin{pgfscope}%
\pgfpathrectangle{\pgfqpoint{0.800000in}{0.528000in}}{\pgfqpoint{4.960000in}{3.696000in}}%
\pgfusepath{clip}%
\pgfsetbuttcap%
\pgfsetroundjoin%
\definecolor{currentfill}{rgb}{0.000000,0.000000,0.000000}%
\pgfsetfillcolor{currentfill}%
\pgfsetlinewidth{1.003750pt}%
\definecolor{currentstroke}{rgb}{0.000000,0.000000,0.000000}%
\pgfsetstrokecolor{currentstroke}%
\pgfsetdash{}{0pt}%
\pgfpathmoveto{\pgfqpoint{4.011666in}{1.771040in}}%
\pgfpathcurveto{\pgfqpoint{4.022716in}{1.771040in}}{\pgfqpoint{4.033315in}{1.775431in}}{\pgfqpoint{4.041128in}{1.783244in}}%
\pgfpathcurveto{\pgfqpoint{4.048942in}{1.791058in}}{\pgfqpoint{4.053332in}{1.801657in}}{\pgfqpoint{4.053332in}{1.812707in}}%
\pgfpathcurveto{\pgfqpoint{4.053332in}{1.823757in}}{\pgfqpoint{4.048942in}{1.834356in}}{\pgfqpoint{4.041128in}{1.842170in}}%
\pgfpathcurveto{\pgfqpoint{4.033315in}{1.849983in}}{\pgfqpoint{4.022716in}{1.854374in}}{\pgfqpoint{4.011666in}{1.854374in}}%
\pgfpathcurveto{\pgfqpoint{4.000616in}{1.854374in}}{\pgfqpoint{3.990016in}{1.849983in}}{\pgfqpoint{3.982203in}{1.842170in}}%
\pgfpathcurveto{\pgfqpoint{3.974389in}{1.834356in}}{\pgfqpoint{3.969999in}{1.823757in}}{\pgfqpoint{3.969999in}{1.812707in}}%
\pgfpathcurveto{\pgfqpoint{3.969999in}{1.801657in}}{\pgfqpoint{3.974389in}{1.791058in}}{\pgfqpoint{3.982203in}{1.783244in}}%
\pgfpathcurveto{\pgfqpoint{3.990016in}{1.775431in}}{\pgfqpoint{4.000616in}{1.771040in}}{\pgfqpoint{4.011666in}{1.771040in}}%
\pgfpathclose%
\pgfusepath{stroke,fill}%
\end{pgfscope}%
\begin{pgfscope}%
\pgfpathrectangle{\pgfqpoint{0.800000in}{0.528000in}}{\pgfqpoint{4.960000in}{3.696000in}}%
\pgfusepath{clip}%
\pgfsetbuttcap%
\pgfsetroundjoin%
\definecolor{currentfill}{rgb}{0.000000,0.000000,0.000000}%
\pgfsetfillcolor{currentfill}%
\pgfsetlinewidth{1.003750pt}%
\definecolor{currentstroke}{rgb}{0.000000,0.000000,0.000000}%
\pgfsetstrokecolor{currentstroke}%
\pgfsetdash{}{0pt}%
\pgfpathmoveto{\pgfqpoint{4.011666in}{1.771040in}}%
\pgfpathcurveto{\pgfqpoint{4.022716in}{1.771040in}}{\pgfqpoint{4.033315in}{1.775431in}}{\pgfqpoint{4.041128in}{1.783244in}}%
\pgfpathcurveto{\pgfqpoint{4.048942in}{1.791058in}}{\pgfqpoint{4.053332in}{1.801657in}}{\pgfqpoint{4.053332in}{1.812707in}}%
\pgfpathcurveto{\pgfqpoint{4.053332in}{1.823757in}}{\pgfqpoint{4.048942in}{1.834356in}}{\pgfqpoint{4.041128in}{1.842170in}}%
\pgfpathcurveto{\pgfqpoint{4.033315in}{1.849983in}}{\pgfqpoint{4.022716in}{1.854374in}}{\pgfqpoint{4.011666in}{1.854374in}}%
\pgfpathcurveto{\pgfqpoint{4.000616in}{1.854374in}}{\pgfqpoint{3.990016in}{1.849983in}}{\pgfqpoint{3.982203in}{1.842170in}}%
\pgfpathcurveto{\pgfqpoint{3.974389in}{1.834356in}}{\pgfqpoint{3.969999in}{1.823757in}}{\pgfqpoint{3.969999in}{1.812707in}}%
\pgfpathcurveto{\pgfqpoint{3.969999in}{1.801657in}}{\pgfqpoint{3.974389in}{1.791058in}}{\pgfqpoint{3.982203in}{1.783244in}}%
\pgfpathcurveto{\pgfqpoint{3.990016in}{1.775431in}}{\pgfqpoint{4.000616in}{1.771040in}}{\pgfqpoint{4.011666in}{1.771040in}}%
\pgfpathclose%
\pgfusepath{stroke,fill}%
\end{pgfscope}%
\begin{pgfscope}%
\pgfpathrectangle{\pgfqpoint{0.800000in}{0.528000in}}{\pgfqpoint{4.960000in}{3.696000in}}%
\pgfusepath{clip}%
\pgfsetbuttcap%
\pgfsetroundjoin%
\definecolor{currentfill}{rgb}{0.000000,0.000000,0.000000}%
\pgfsetfillcolor{currentfill}%
\pgfsetlinewidth{1.003750pt}%
\definecolor{currentstroke}{rgb}{0.000000,0.000000,0.000000}%
\pgfsetstrokecolor{currentstroke}%
\pgfsetdash{}{0pt}%
\pgfpathmoveto{\pgfqpoint{4.011666in}{2.877687in}}%
\pgfpathcurveto{\pgfqpoint{4.022716in}{2.877687in}}{\pgfqpoint{4.033315in}{2.882077in}}{\pgfqpoint{4.041128in}{2.889891in}}%
\pgfpathcurveto{\pgfqpoint{4.048942in}{2.897704in}}{\pgfqpoint{4.053332in}{2.908303in}}{\pgfqpoint{4.053332in}{2.919353in}}%
\pgfpathcurveto{\pgfqpoint{4.053332in}{2.930404in}}{\pgfqpoint{4.048942in}{2.941003in}}{\pgfqpoint{4.041128in}{2.948816in}}%
\pgfpathcurveto{\pgfqpoint{4.033315in}{2.956630in}}{\pgfqpoint{4.022716in}{2.961020in}}{\pgfqpoint{4.011666in}{2.961020in}}%
\pgfpathcurveto{\pgfqpoint{4.000616in}{2.961020in}}{\pgfqpoint{3.990016in}{2.956630in}}{\pgfqpoint{3.982203in}{2.948816in}}%
\pgfpathcurveto{\pgfqpoint{3.974389in}{2.941003in}}{\pgfqpoint{3.969999in}{2.930404in}}{\pgfqpoint{3.969999in}{2.919353in}}%
\pgfpathcurveto{\pgfqpoint{3.969999in}{2.908303in}}{\pgfqpoint{3.974389in}{2.897704in}}{\pgfqpoint{3.982203in}{2.889891in}}%
\pgfpathcurveto{\pgfqpoint{3.990016in}{2.882077in}}{\pgfqpoint{4.000616in}{2.877687in}}{\pgfqpoint{4.011666in}{2.877687in}}%
\pgfpathclose%
\pgfusepath{stroke,fill}%
\end{pgfscope}%
\begin{pgfscope}%
\pgfpathrectangle{\pgfqpoint{0.800000in}{0.528000in}}{\pgfqpoint{4.960000in}{3.696000in}}%
\pgfusepath{clip}%
\pgfsetbuttcap%
\pgfsetroundjoin%
\definecolor{currentfill}{rgb}{0.000000,0.000000,0.000000}%
\pgfsetfillcolor{currentfill}%
\pgfsetlinewidth{1.003750pt}%
\definecolor{currentstroke}{rgb}{0.000000,0.000000,0.000000}%
\pgfsetstrokecolor{currentstroke}%
\pgfsetdash{}{0pt}%
\pgfpathmoveto{\pgfqpoint{4.011666in}{2.877687in}}%
\pgfpathcurveto{\pgfqpoint{4.022716in}{2.877687in}}{\pgfqpoint{4.033315in}{2.882077in}}{\pgfqpoint{4.041128in}{2.889891in}}%
\pgfpathcurveto{\pgfqpoint{4.048942in}{2.897704in}}{\pgfqpoint{4.053332in}{2.908303in}}{\pgfqpoint{4.053332in}{2.919353in}}%
\pgfpathcurveto{\pgfqpoint{4.053332in}{2.930404in}}{\pgfqpoint{4.048942in}{2.941003in}}{\pgfqpoint{4.041128in}{2.948816in}}%
\pgfpathcurveto{\pgfqpoint{4.033315in}{2.956630in}}{\pgfqpoint{4.022716in}{2.961020in}}{\pgfqpoint{4.011666in}{2.961020in}}%
\pgfpathcurveto{\pgfqpoint{4.000616in}{2.961020in}}{\pgfqpoint{3.990016in}{2.956630in}}{\pgfqpoint{3.982203in}{2.948816in}}%
\pgfpathcurveto{\pgfqpoint{3.974389in}{2.941003in}}{\pgfqpoint{3.969999in}{2.930404in}}{\pgfqpoint{3.969999in}{2.919353in}}%
\pgfpathcurveto{\pgfqpoint{3.969999in}{2.908303in}}{\pgfqpoint{3.974389in}{2.897704in}}{\pgfqpoint{3.982203in}{2.889891in}}%
\pgfpathcurveto{\pgfqpoint{3.990016in}{2.882077in}}{\pgfqpoint{4.000616in}{2.877687in}}{\pgfqpoint{4.011666in}{2.877687in}}%
\pgfpathclose%
\pgfusepath{stroke,fill}%
\end{pgfscope}%
\begin{pgfscope}%
\pgfpathrectangle{\pgfqpoint{0.800000in}{0.528000in}}{\pgfqpoint{4.960000in}{3.696000in}}%
\pgfusepath{clip}%
\pgfsetbuttcap%
\pgfsetroundjoin%
\definecolor{currentfill}{rgb}{0.000000,0.000000,0.000000}%
\pgfsetfillcolor{currentfill}%
\pgfsetlinewidth{1.003750pt}%
\definecolor{currentstroke}{rgb}{0.000000,0.000000,0.000000}%
\pgfsetstrokecolor{currentstroke}%
\pgfsetdash{}{0pt}%
\pgfpathmoveto{\pgfqpoint{4.011666in}{1.771040in}}%
\pgfpathcurveto{\pgfqpoint{4.022716in}{1.771040in}}{\pgfqpoint{4.033315in}{1.775431in}}{\pgfqpoint{4.041128in}{1.783244in}}%
\pgfpathcurveto{\pgfqpoint{4.048942in}{1.791058in}}{\pgfqpoint{4.053332in}{1.801657in}}{\pgfqpoint{4.053332in}{1.812707in}}%
\pgfpathcurveto{\pgfqpoint{4.053332in}{1.823757in}}{\pgfqpoint{4.048942in}{1.834356in}}{\pgfqpoint{4.041128in}{1.842170in}}%
\pgfpathcurveto{\pgfqpoint{4.033315in}{1.849983in}}{\pgfqpoint{4.022716in}{1.854374in}}{\pgfqpoint{4.011666in}{1.854374in}}%
\pgfpathcurveto{\pgfqpoint{4.000616in}{1.854374in}}{\pgfqpoint{3.990016in}{1.849983in}}{\pgfqpoint{3.982203in}{1.842170in}}%
\pgfpathcurveto{\pgfqpoint{3.974389in}{1.834356in}}{\pgfqpoint{3.969999in}{1.823757in}}{\pgfqpoint{3.969999in}{1.812707in}}%
\pgfpathcurveto{\pgfqpoint{3.969999in}{1.801657in}}{\pgfqpoint{3.974389in}{1.791058in}}{\pgfqpoint{3.982203in}{1.783244in}}%
\pgfpathcurveto{\pgfqpoint{3.990016in}{1.775431in}}{\pgfqpoint{4.000616in}{1.771040in}}{\pgfqpoint{4.011666in}{1.771040in}}%
\pgfpathclose%
\pgfusepath{stroke,fill}%
\end{pgfscope}%
\begin{pgfscope}%
\pgfpathrectangle{\pgfqpoint{0.800000in}{0.528000in}}{\pgfqpoint{4.960000in}{3.696000in}}%
\pgfusepath{clip}%
\pgfsetbuttcap%
\pgfsetroundjoin%
\definecolor{currentfill}{rgb}{0.000000,0.000000,0.000000}%
\pgfsetfillcolor{currentfill}%
\pgfsetlinewidth{1.003750pt}%
\definecolor{currentstroke}{rgb}{0.000000,0.000000,0.000000}%
\pgfsetstrokecolor{currentstroke}%
\pgfsetdash{}{0pt}%
\pgfpathmoveto{\pgfqpoint{4.011666in}{2.877687in}}%
\pgfpathcurveto{\pgfqpoint{4.022716in}{2.877687in}}{\pgfqpoint{4.033315in}{2.882077in}}{\pgfqpoint{4.041128in}{2.889891in}}%
\pgfpathcurveto{\pgfqpoint{4.048942in}{2.897704in}}{\pgfqpoint{4.053332in}{2.908303in}}{\pgfqpoint{4.053332in}{2.919353in}}%
\pgfpathcurveto{\pgfqpoint{4.053332in}{2.930404in}}{\pgfqpoint{4.048942in}{2.941003in}}{\pgfqpoint{4.041128in}{2.948816in}}%
\pgfpathcurveto{\pgfqpoint{4.033315in}{2.956630in}}{\pgfqpoint{4.022716in}{2.961020in}}{\pgfqpoint{4.011666in}{2.961020in}}%
\pgfpathcurveto{\pgfqpoint{4.000616in}{2.961020in}}{\pgfqpoint{3.990016in}{2.956630in}}{\pgfqpoint{3.982203in}{2.948816in}}%
\pgfpathcurveto{\pgfqpoint{3.974389in}{2.941003in}}{\pgfqpoint{3.969999in}{2.930404in}}{\pgfqpoint{3.969999in}{2.919353in}}%
\pgfpathcurveto{\pgfqpoint{3.969999in}{2.908303in}}{\pgfqpoint{3.974389in}{2.897704in}}{\pgfqpoint{3.982203in}{2.889891in}}%
\pgfpathcurveto{\pgfqpoint{3.990016in}{2.882077in}}{\pgfqpoint{4.000616in}{2.877687in}}{\pgfqpoint{4.011666in}{2.877687in}}%
\pgfpathclose%
\pgfusepath{stroke,fill}%
\end{pgfscope}%
\begin{pgfscope}%
\pgfpathrectangle{\pgfqpoint{0.800000in}{0.528000in}}{\pgfqpoint{4.960000in}{3.696000in}}%
\pgfusepath{clip}%
\pgfsetbuttcap%
\pgfsetroundjoin%
\definecolor{currentfill}{rgb}{0.000000,0.000000,0.000000}%
\pgfsetfillcolor{currentfill}%
\pgfsetlinewidth{1.003750pt}%
\definecolor{currentstroke}{rgb}{0.000000,0.000000,0.000000}%
\pgfsetstrokecolor{currentstroke}%
\pgfsetdash{}{0pt}%
\pgfpathmoveto{\pgfqpoint{4.011666in}{2.877687in}}%
\pgfpathcurveto{\pgfqpoint{4.022716in}{2.877687in}}{\pgfqpoint{4.033315in}{2.882077in}}{\pgfqpoint{4.041128in}{2.889891in}}%
\pgfpathcurveto{\pgfqpoint{4.048942in}{2.897704in}}{\pgfqpoint{4.053332in}{2.908303in}}{\pgfqpoint{4.053332in}{2.919353in}}%
\pgfpathcurveto{\pgfqpoint{4.053332in}{2.930404in}}{\pgfqpoint{4.048942in}{2.941003in}}{\pgfqpoint{4.041128in}{2.948816in}}%
\pgfpathcurveto{\pgfqpoint{4.033315in}{2.956630in}}{\pgfqpoint{4.022716in}{2.961020in}}{\pgfqpoint{4.011666in}{2.961020in}}%
\pgfpathcurveto{\pgfqpoint{4.000616in}{2.961020in}}{\pgfqpoint{3.990016in}{2.956630in}}{\pgfqpoint{3.982203in}{2.948816in}}%
\pgfpathcurveto{\pgfqpoint{3.974389in}{2.941003in}}{\pgfqpoint{3.969999in}{2.930404in}}{\pgfqpoint{3.969999in}{2.919353in}}%
\pgfpathcurveto{\pgfqpoint{3.969999in}{2.908303in}}{\pgfqpoint{3.974389in}{2.897704in}}{\pgfqpoint{3.982203in}{2.889891in}}%
\pgfpathcurveto{\pgfqpoint{3.990016in}{2.882077in}}{\pgfqpoint{4.000616in}{2.877687in}}{\pgfqpoint{4.011666in}{2.877687in}}%
\pgfpathclose%
\pgfusepath{stroke,fill}%
\end{pgfscope}%
\begin{pgfscope}%
\pgfpathrectangle{\pgfqpoint{0.800000in}{0.528000in}}{\pgfqpoint{4.960000in}{3.696000in}}%
\pgfusepath{clip}%
\pgfsetbuttcap%
\pgfsetroundjoin%
\definecolor{currentfill}{rgb}{0.000000,0.000000,0.000000}%
\pgfsetfillcolor{currentfill}%
\pgfsetlinewidth{1.003750pt}%
\definecolor{currentstroke}{rgb}{0.000000,0.000000,0.000000}%
\pgfsetstrokecolor{currentstroke}%
\pgfsetdash{}{0pt}%
\pgfpathmoveto{\pgfqpoint{4.011666in}{1.771040in}}%
\pgfpathcurveto{\pgfqpoint{4.022716in}{1.771040in}}{\pgfqpoint{4.033315in}{1.775431in}}{\pgfqpoint{4.041128in}{1.783244in}}%
\pgfpathcurveto{\pgfqpoint{4.048942in}{1.791058in}}{\pgfqpoint{4.053332in}{1.801657in}}{\pgfqpoint{4.053332in}{1.812707in}}%
\pgfpathcurveto{\pgfqpoint{4.053332in}{1.823757in}}{\pgfqpoint{4.048942in}{1.834356in}}{\pgfqpoint{4.041128in}{1.842170in}}%
\pgfpathcurveto{\pgfqpoint{4.033315in}{1.849983in}}{\pgfqpoint{4.022716in}{1.854374in}}{\pgfqpoint{4.011666in}{1.854374in}}%
\pgfpathcurveto{\pgfqpoint{4.000616in}{1.854374in}}{\pgfqpoint{3.990016in}{1.849983in}}{\pgfqpoint{3.982203in}{1.842170in}}%
\pgfpathcurveto{\pgfqpoint{3.974389in}{1.834356in}}{\pgfqpoint{3.969999in}{1.823757in}}{\pgfqpoint{3.969999in}{1.812707in}}%
\pgfpathcurveto{\pgfqpoint{3.969999in}{1.801657in}}{\pgfqpoint{3.974389in}{1.791058in}}{\pgfqpoint{3.982203in}{1.783244in}}%
\pgfpathcurveto{\pgfqpoint{3.990016in}{1.775431in}}{\pgfqpoint{4.000616in}{1.771040in}}{\pgfqpoint{4.011666in}{1.771040in}}%
\pgfpathclose%
\pgfusepath{stroke,fill}%
\end{pgfscope}%
\begin{pgfscope}%
\pgfpathrectangle{\pgfqpoint{0.800000in}{0.528000in}}{\pgfqpoint{4.960000in}{3.696000in}}%
\pgfusepath{clip}%
\pgfsetbuttcap%
\pgfsetroundjoin%
\definecolor{currentfill}{rgb}{0.000000,0.000000,0.000000}%
\pgfsetfillcolor{currentfill}%
\pgfsetlinewidth{1.003750pt}%
\definecolor{currentstroke}{rgb}{0.000000,0.000000,0.000000}%
\pgfsetstrokecolor{currentstroke}%
\pgfsetdash{}{0pt}%
\pgfpathmoveto{\pgfqpoint{4.011666in}{2.877687in}}%
\pgfpathcurveto{\pgfqpoint{4.022716in}{2.877687in}}{\pgfqpoint{4.033315in}{2.882077in}}{\pgfqpoint{4.041128in}{2.889891in}}%
\pgfpathcurveto{\pgfqpoint{4.048942in}{2.897704in}}{\pgfqpoint{4.053332in}{2.908303in}}{\pgfqpoint{4.053332in}{2.919353in}}%
\pgfpathcurveto{\pgfqpoint{4.053332in}{2.930404in}}{\pgfqpoint{4.048942in}{2.941003in}}{\pgfqpoint{4.041128in}{2.948816in}}%
\pgfpathcurveto{\pgfqpoint{4.033315in}{2.956630in}}{\pgfqpoint{4.022716in}{2.961020in}}{\pgfqpoint{4.011666in}{2.961020in}}%
\pgfpathcurveto{\pgfqpoint{4.000616in}{2.961020in}}{\pgfqpoint{3.990016in}{2.956630in}}{\pgfqpoint{3.982203in}{2.948816in}}%
\pgfpathcurveto{\pgfqpoint{3.974389in}{2.941003in}}{\pgfqpoint{3.969999in}{2.930404in}}{\pgfqpoint{3.969999in}{2.919353in}}%
\pgfpathcurveto{\pgfqpoint{3.969999in}{2.908303in}}{\pgfqpoint{3.974389in}{2.897704in}}{\pgfqpoint{3.982203in}{2.889891in}}%
\pgfpathcurveto{\pgfqpoint{3.990016in}{2.882077in}}{\pgfqpoint{4.000616in}{2.877687in}}{\pgfqpoint{4.011666in}{2.877687in}}%
\pgfpathclose%
\pgfusepath{stroke,fill}%
\end{pgfscope}%
\begin{pgfscope}%
\pgfpathrectangle{\pgfqpoint{0.800000in}{0.528000in}}{\pgfqpoint{4.960000in}{3.696000in}}%
\pgfusepath{clip}%
\pgfsetbuttcap%
\pgfsetroundjoin%
\definecolor{currentfill}{rgb}{0.000000,0.000000,0.000000}%
\pgfsetfillcolor{currentfill}%
\pgfsetlinewidth{1.003750pt}%
\definecolor{currentstroke}{rgb}{0.000000,0.000000,0.000000}%
\pgfsetstrokecolor{currentstroke}%
\pgfsetdash{}{0pt}%
\pgfpathmoveto{\pgfqpoint{4.011666in}{1.771040in}}%
\pgfpathcurveto{\pgfqpoint{4.022716in}{1.771040in}}{\pgfqpoint{4.033315in}{1.775431in}}{\pgfqpoint{4.041128in}{1.783244in}}%
\pgfpathcurveto{\pgfqpoint{4.048942in}{1.791058in}}{\pgfqpoint{4.053332in}{1.801657in}}{\pgfqpoint{4.053332in}{1.812707in}}%
\pgfpathcurveto{\pgfqpoint{4.053332in}{1.823757in}}{\pgfqpoint{4.048942in}{1.834356in}}{\pgfqpoint{4.041128in}{1.842170in}}%
\pgfpathcurveto{\pgfqpoint{4.033315in}{1.849983in}}{\pgfqpoint{4.022716in}{1.854374in}}{\pgfqpoint{4.011666in}{1.854374in}}%
\pgfpathcurveto{\pgfqpoint{4.000616in}{1.854374in}}{\pgfqpoint{3.990016in}{1.849983in}}{\pgfqpoint{3.982203in}{1.842170in}}%
\pgfpathcurveto{\pgfqpoint{3.974389in}{1.834356in}}{\pgfqpoint{3.969999in}{1.823757in}}{\pgfqpoint{3.969999in}{1.812707in}}%
\pgfpathcurveto{\pgfqpoint{3.969999in}{1.801657in}}{\pgfqpoint{3.974389in}{1.791058in}}{\pgfqpoint{3.982203in}{1.783244in}}%
\pgfpathcurveto{\pgfqpoint{3.990016in}{1.775431in}}{\pgfqpoint{4.000616in}{1.771040in}}{\pgfqpoint{4.011666in}{1.771040in}}%
\pgfpathclose%
\pgfusepath{stroke,fill}%
\end{pgfscope}%
\begin{pgfscope}%
\pgfpathrectangle{\pgfqpoint{0.800000in}{0.528000in}}{\pgfqpoint{4.960000in}{3.696000in}}%
\pgfusepath{clip}%
\pgfsetbuttcap%
\pgfsetroundjoin%
\definecolor{currentfill}{rgb}{0.000000,0.000000,0.000000}%
\pgfsetfillcolor{currentfill}%
\pgfsetlinewidth{1.003750pt}%
\definecolor{currentstroke}{rgb}{0.000000,0.000000,0.000000}%
\pgfsetstrokecolor{currentstroke}%
\pgfsetdash{}{0pt}%
\pgfpathmoveto{\pgfqpoint{4.011666in}{1.771040in}}%
\pgfpathcurveto{\pgfqpoint{4.022716in}{1.771040in}}{\pgfqpoint{4.033315in}{1.775431in}}{\pgfqpoint{4.041128in}{1.783244in}}%
\pgfpathcurveto{\pgfqpoint{4.048942in}{1.791058in}}{\pgfqpoint{4.053332in}{1.801657in}}{\pgfqpoint{4.053332in}{1.812707in}}%
\pgfpathcurveto{\pgfqpoint{4.053332in}{1.823757in}}{\pgfqpoint{4.048942in}{1.834356in}}{\pgfqpoint{4.041128in}{1.842170in}}%
\pgfpathcurveto{\pgfqpoint{4.033315in}{1.849983in}}{\pgfqpoint{4.022716in}{1.854374in}}{\pgfqpoint{4.011666in}{1.854374in}}%
\pgfpathcurveto{\pgfqpoint{4.000616in}{1.854374in}}{\pgfqpoint{3.990016in}{1.849983in}}{\pgfqpoint{3.982203in}{1.842170in}}%
\pgfpathcurveto{\pgfqpoint{3.974389in}{1.834356in}}{\pgfqpoint{3.969999in}{1.823757in}}{\pgfqpoint{3.969999in}{1.812707in}}%
\pgfpathcurveto{\pgfqpoint{3.969999in}{1.801657in}}{\pgfqpoint{3.974389in}{1.791058in}}{\pgfqpoint{3.982203in}{1.783244in}}%
\pgfpathcurveto{\pgfqpoint{3.990016in}{1.775431in}}{\pgfqpoint{4.000616in}{1.771040in}}{\pgfqpoint{4.011666in}{1.771040in}}%
\pgfpathclose%
\pgfusepath{stroke,fill}%
\end{pgfscope}%
\begin{pgfscope}%
\pgfpathrectangle{\pgfqpoint{0.800000in}{0.528000in}}{\pgfqpoint{4.960000in}{3.696000in}}%
\pgfusepath{clip}%
\pgfsetbuttcap%
\pgfsetroundjoin%
\definecolor{currentfill}{rgb}{0.000000,0.000000,0.000000}%
\pgfsetfillcolor{currentfill}%
\pgfsetlinewidth{1.003750pt}%
\definecolor{currentstroke}{rgb}{0.000000,0.000000,0.000000}%
\pgfsetstrokecolor{currentstroke}%
\pgfsetdash{}{0pt}%
\pgfpathmoveto{\pgfqpoint{4.011666in}{1.771040in}}%
\pgfpathcurveto{\pgfqpoint{4.022716in}{1.771040in}}{\pgfqpoint{4.033315in}{1.775431in}}{\pgfqpoint{4.041128in}{1.783244in}}%
\pgfpathcurveto{\pgfqpoint{4.048942in}{1.791058in}}{\pgfqpoint{4.053332in}{1.801657in}}{\pgfqpoint{4.053332in}{1.812707in}}%
\pgfpathcurveto{\pgfqpoint{4.053332in}{1.823757in}}{\pgfqpoint{4.048942in}{1.834356in}}{\pgfqpoint{4.041128in}{1.842170in}}%
\pgfpathcurveto{\pgfqpoint{4.033315in}{1.849983in}}{\pgfqpoint{4.022716in}{1.854374in}}{\pgfqpoint{4.011666in}{1.854374in}}%
\pgfpathcurveto{\pgfqpoint{4.000616in}{1.854374in}}{\pgfqpoint{3.990016in}{1.849983in}}{\pgfqpoint{3.982203in}{1.842170in}}%
\pgfpathcurveto{\pgfqpoint{3.974389in}{1.834356in}}{\pgfqpoint{3.969999in}{1.823757in}}{\pgfqpoint{3.969999in}{1.812707in}}%
\pgfpathcurveto{\pgfqpoint{3.969999in}{1.801657in}}{\pgfqpoint{3.974389in}{1.791058in}}{\pgfqpoint{3.982203in}{1.783244in}}%
\pgfpathcurveto{\pgfqpoint{3.990016in}{1.775431in}}{\pgfqpoint{4.000616in}{1.771040in}}{\pgfqpoint{4.011666in}{1.771040in}}%
\pgfpathclose%
\pgfusepath{stroke,fill}%
\end{pgfscope}%
\begin{pgfscope}%
\pgfpathrectangle{\pgfqpoint{0.800000in}{0.528000in}}{\pgfqpoint{4.960000in}{3.696000in}}%
\pgfusepath{clip}%
\pgfsetbuttcap%
\pgfsetroundjoin%
\definecolor{currentfill}{rgb}{0.000000,0.000000,0.000000}%
\pgfsetfillcolor{currentfill}%
\pgfsetlinewidth{1.003750pt}%
\definecolor{currentstroke}{rgb}{0.000000,0.000000,0.000000}%
\pgfsetstrokecolor{currentstroke}%
\pgfsetdash{}{0pt}%
\pgfpathmoveto{\pgfqpoint{4.011666in}{2.877687in}}%
\pgfpathcurveto{\pgfqpoint{4.022716in}{2.877687in}}{\pgfqpoint{4.033315in}{2.882077in}}{\pgfqpoint{4.041128in}{2.889891in}}%
\pgfpathcurveto{\pgfqpoint{4.048942in}{2.897704in}}{\pgfqpoint{4.053332in}{2.908303in}}{\pgfqpoint{4.053332in}{2.919353in}}%
\pgfpathcurveto{\pgfqpoint{4.053332in}{2.930404in}}{\pgfqpoint{4.048942in}{2.941003in}}{\pgfqpoint{4.041128in}{2.948816in}}%
\pgfpathcurveto{\pgfqpoint{4.033315in}{2.956630in}}{\pgfqpoint{4.022716in}{2.961020in}}{\pgfqpoint{4.011666in}{2.961020in}}%
\pgfpathcurveto{\pgfqpoint{4.000616in}{2.961020in}}{\pgfqpoint{3.990016in}{2.956630in}}{\pgfqpoint{3.982203in}{2.948816in}}%
\pgfpathcurveto{\pgfqpoint{3.974389in}{2.941003in}}{\pgfqpoint{3.969999in}{2.930404in}}{\pgfqpoint{3.969999in}{2.919353in}}%
\pgfpathcurveto{\pgfqpoint{3.969999in}{2.908303in}}{\pgfqpoint{3.974389in}{2.897704in}}{\pgfqpoint{3.982203in}{2.889891in}}%
\pgfpathcurveto{\pgfqpoint{3.990016in}{2.882077in}}{\pgfqpoint{4.000616in}{2.877687in}}{\pgfqpoint{4.011666in}{2.877687in}}%
\pgfpathclose%
\pgfusepath{stroke,fill}%
\end{pgfscope}%
\begin{pgfscope}%
\pgfpathrectangle{\pgfqpoint{0.800000in}{0.528000in}}{\pgfqpoint{4.960000in}{3.696000in}}%
\pgfusepath{clip}%
\pgfsetbuttcap%
\pgfsetroundjoin%
\definecolor{currentfill}{rgb}{0.000000,0.000000,0.000000}%
\pgfsetfillcolor{currentfill}%
\pgfsetlinewidth{1.003750pt}%
\definecolor{currentstroke}{rgb}{0.000000,0.000000,0.000000}%
\pgfsetstrokecolor{currentstroke}%
\pgfsetdash{}{0pt}%
\pgfpathmoveto{\pgfqpoint{4.011666in}{1.771040in}}%
\pgfpathcurveto{\pgfqpoint{4.022716in}{1.771040in}}{\pgfqpoint{4.033315in}{1.775431in}}{\pgfqpoint{4.041128in}{1.783244in}}%
\pgfpathcurveto{\pgfqpoint{4.048942in}{1.791058in}}{\pgfqpoint{4.053332in}{1.801657in}}{\pgfqpoint{4.053332in}{1.812707in}}%
\pgfpathcurveto{\pgfqpoint{4.053332in}{1.823757in}}{\pgfqpoint{4.048942in}{1.834356in}}{\pgfqpoint{4.041128in}{1.842170in}}%
\pgfpathcurveto{\pgfqpoint{4.033315in}{1.849983in}}{\pgfqpoint{4.022716in}{1.854374in}}{\pgfqpoint{4.011666in}{1.854374in}}%
\pgfpathcurveto{\pgfqpoint{4.000616in}{1.854374in}}{\pgfqpoint{3.990016in}{1.849983in}}{\pgfqpoint{3.982203in}{1.842170in}}%
\pgfpathcurveto{\pgfqpoint{3.974389in}{1.834356in}}{\pgfqpoint{3.969999in}{1.823757in}}{\pgfqpoint{3.969999in}{1.812707in}}%
\pgfpathcurveto{\pgfqpoint{3.969999in}{1.801657in}}{\pgfqpoint{3.974389in}{1.791058in}}{\pgfqpoint{3.982203in}{1.783244in}}%
\pgfpathcurveto{\pgfqpoint{3.990016in}{1.775431in}}{\pgfqpoint{4.000616in}{1.771040in}}{\pgfqpoint{4.011666in}{1.771040in}}%
\pgfpathclose%
\pgfusepath{stroke,fill}%
\end{pgfscope}%
\begin{pgfscope}%
\pgfpathrectangle{\pgfqpoint{0.800000in}{0.528000in}}{\pgfqpoint{4.960000in}{3.696000in}}%
\pgfusepath{clip}%
\pgfsetbuttcap%
\pgfsetroundjoin%
\definecolor{currentfill}{rgb}{0.000000,0.000000,0.000000}%
\pgfsetfillcolor{currentfill}%
\pgfsetlinewidth{1.003750pt}%
\definecolor{currentstroke}{rgb}{0.000000,0.000000,0.000000}%
\pgfsetstrokecolor{currentstroke}%
\pgfsetdash{}{0pt}%
\pgfpathmoveto{\pgfqpoint{4.011666in}{2.877687in}}%
\pgfpathcurveto{\pgfqpoint{4.022716in}{2.877687in}}{\pgfqpoint{4.033315in}{2.882077in}}{\pgfqpoint{4.041128in}{2.889891in}}%
\pgfpathcurveto{\pgfqpoint{4.048942in}{2.897704in}}{\pgfqpoint{4.053332in}{2.908303in}}{\pgfqpoint{4.053332in}{2.919353in}}%
\pgfpathcurveto{\pgfqpoint{4.053332in}{2.930404in}}{\pgfqpoint{4.048942in}{2.941003in}}{\pgfqpoint{4.041128in}{2.948816in}}%
\pgfpathcurveto{\pgfqpoint{4.033315in}{2.956630in}}{\pgfqpoint{4.022716in}{2.961020in}}{\pgfqpoint{4.011666in}{2.961020in}}%
\pgfpathcurveto{\pgfqpoint{4.000616in}{2.961020in}}{\pgfqpoint{3.990016in}{2.956630in}}{\pgfqpoint{3.982203in}{2.948816in}}%
\pgfpathcurveto{\pgfqpoint{3.974389in}{2.941003in}}{\pgfqpoint{3.969999in}{2.930404in}}{\pgfqpoint{3.969999in}{2.919353in}}%
\pgfpathcurveto{\pgfqpoint{3.969999in}{2.908303in}}{\pgfqpoint{3.974389in}{2.897704in}}{\pgfqpoint{3.982203in}{2.889891in}}%
\pgfpathcurveto{\pgfqpoint{3.990016in}{2.882077in}}{\pgfqpoint{4.000616in}{2.877687in}}{\pgfqpoint{4.011666in}{2.877687in}}%
\pgfpathclose%
\pgfusepath{stroke,fill}%
\end{pgfscope}%
\begin{pgfscope}%
\pgfpathrectangle{\pgfqpoint{0.800000in}{0.528000in}}{\pgfqpoint{4.960000in}{3.696000in}}%
\pgfusepath{clip}%
\pgfsetbuttcap%
\pgfsetroundjoin%
\definecolor{currentfill}{rgb}{0.000000,0.000000,0.000000}%
\pgfsetfillcolor{currentfill}%
\pgfsetlinewidth{1.003750pt}%
\definecolor{currentstroke}{rgb}{0.000000,0.000000,0.000000}%
\pgfsetstrokecolor{currentstroke}%
\pgfsetdash{}{0pt}%
\pgfpathmoveto{\pgfqpoint{4.011666in}{1.771040in}}%
\pgfpathcurveto{\pgfqpoint{4.022716in}{1.771040in}}{\pgfqpoint{4.033315in}{1.775431in}}{\pgfqpoint{4.041128in}{1.783244in}}%
\pgfpathcurveto{\pgfqpoint{4.048942in}{1.791058in}}{\pgfqpoint{4.053332in}{1.801657in}}{\pgfqpoint{4.053332in}{1.812707in}}%
\pgfpathcurveto{\pgfqpoint{4.053332in}{1.823757in}}{\pgfqpoint{4.048942in}{1.834356in}}{\pgfqpoint{4.041128in}{1.842170in}}%
\pgfpathcurveto{\pgfqpoint{4.033315in}{1.849983in}}{\pgfqpoint{4.022716in}{1.854374in}}{\pgfqpoint{4.011666in}{1.854374in}}%
\pgfpathcurveto{\pgfqpoint{4.000616in}{1.854374in}}{\pgfqpoint{3.990016in}{1.849983in}}{\pgfqpoint{3.982203in}{1.842170in}}%
\pgfpathcurveto{\pgfqpoint{3.974389in}{1.834356in}}{\pgfqpoint{3.969999in}{1.823757in}}{\pgfqpoint{3.969999in}{1.812707in}}%
\pgfpathcurveto{\pgfqpoint{3.969999in}{1.801657in}}{\pgfqpoint{3.974389in}{1.791058in}}{\pgfqpoint{3.982203in}{1.783244in}}%
\pgfpathcurveto{\pgfqpoint{3.990016in}{1.775431in}}{\pgfqpoint{4.000616in}{1.771040in}}{\pgfqpoint{4.011666in}{1.771040in}}%
\pgfpathclose%
\pgfusepath{stroke,fill}%
\end{pgfscope}%
\begin{pgfscope}%
\pgfpathrectangle{\pgfqpoint{0.800000in}{0.528000in}}{\pgfqpoint{4.960000in}{3.696000in}}%
\pgfusepath{clip}%
\pgfsetbuttcap%
\pgfsetroundjoin%
\definecolor{currentfill}{rgb}{0.000000,0.000000,0.000000}%
\pgfsetfillcolor{currentfill}%
\pgfsetlinewidth{1.003750pt}%
\definecolor{currentstroke}{rgb}{0.000000,0.000000,0.000000}%
\pgfsetstrokecolor{currentstroke}%
\pgfsetdash{}{0pt}%
\pgfpathmoveto{\pgfqpoint{4.011666in}{2.877687in}}%
\pgfpathcurveto{\pgfqpoint{4.022716in}{2.877687in}}{\pgfqpoint{4.033315in}{2.882077in}}{\pgfqpoint{4.041128in}{2.889891in}}%
\pgfpathcurveto{\pgfqpoint{4.048942in}{2.897704in}}{\pgfqpoint{4.053332in}{2.908303in}}{\pgfqpoint{4.053332in}{2.919353in}}%
\pgfpathcurveto{\pgfqpoint{4.053332in}{2.930404in}}{\pgfqpoint{4.048942in}{2.941003in}}{\pgfqpoint{4.041128in}{2.948816in}}%
\pgfpathcurveto{\pgfqpoint{4.033315in}{2.956630in}}{\pgfqpoint{4.022716in}{2.961020in}}{\pgfqpoint{4.011666in}{2.961020in}}%
\pgfpathcurveto{\pgfqpoint{4.000616in}{2.961020in}}{\pgfqpoint{3.990016in}{2.956630in}}{\pgfqpoint{3.982203in}{2.948816in}}%
\pgfpathcurveto{\pgfqpoint{3.974389in}{2.941003in}}{\pgfqpoint{3.969999in}{2.930404in}}{\pgfqpoint{3.969999in}{2.919353in}}%
\pgfpathcurveto{\pgfqpoint{3.969999in}{2.908303in}}{\pgfqpoint{3.974389in}{2.897704in}}{\pgfqpoint{3.982203in}{2.889891in}}%
\pgfpathcurveto{\pgfqpoint{3.990016in}{2.882077in}}{\pgfqpoint{4.000616in}{2.877687in}}{\pgfqpoint{4.011666in}{2.877687in}}%
\pgfpathclose%
\pgfusepath{stroke,fill}%
\end{pgfscope}%
\begin{pgfscope}%
\pgfpathrectangle{\pgfqpoint{0.800000in}{0.528000in}}{\pgfqpoint{4.960000in}{3.696000in}}%
\pgfusepath{clip}%
\pgfsetbuttcap%
\pgfsetroundjoin%
\definecolor{currentfill}{rgb}{0.000000,0.000000,0.000000}%
\pgfsetfillcolor{currentfill}%
\pgfsetlinewidth{1.003750pt}%
\definecolor{currentstroke}{rgb}{0.000000,0.000000,0.000000}%
\pgfsetstrokecolor{currentstroke}%
\pgfsetdash{}{0pt}%
\pgfpathmoveto{\pgfqpoint{4.011666in}{1.771040in}}%
\pgfpathcurveto{\pgfqpoint{4.022716in}{1.771040in}}{\pgfqpoint{4.033315in}{1.775431in}}{\pgfqpoint{4.041128in}{1.783244in}}%
\pgfpathcurveto{\pgfqpoint{4.048942in}{1.791058in}}{\pgfqpoint{4.053332in}{1.801657in}}{\pgfqpoint{4.053332in}{1.812707in}}%
\pgfpathcurveto{\pgfqpoint{4.053332in}{1.823757in}}{\pgfqpoint{4.048942in}{1.834356in}}{\pgfqpoint{4.041128in}{1.842170in}}%
\pgfpathcurveto{\pgfqpoint{4.033315in}{1.849983in}}{\pgfqpoint{4.022716in}{1.854374in}}{\pgfqpoint{4.011666in}{1.854374in}}%
\pgfpathcurveto{\pgfqpoint{4.000616in}{1.854374in}}{\pgfqpoint{3.990016in}{1.849983in}}{\pgfqpoint{3.982203in}{1.842170in}}%
\pgfpathcurveto{\pgfqpoint{3.974389in}{1.834356in}}{\pgfqpoint{3.969999in}{1.823757in}}{\pgfqpoint{3.969999in}{1.812707in}}%
\pgfpathcurveto{\pgfqpoint{3.969999in}{1.801657in}}{\pgfqpoint{3.974389in}{1.791058in}}{\pgfqpoint{3.982203in}{1.783244in}}%
\pgfpathcurveto{\pgfqpoint{3.990016in}{1.775431in}}{\pgfqpoint{4.000616in}{1.771040in}}{\pgfqpoint{4.011666in}{1.771040in}}%
\pgfpathclose%
\pgfusepath{stroke,fill}%
\end{pgfscope}%
\begin{pgfscope}%
\pgfpathrectangle{\pgfqpoint{0.800000in}{0.528000in}}{\pgfqpoint{4.960000in}{3.696000in}}%
\pgfusepath{clip}%
\pgfsetbuttcap%
\pgfsetroundjoin%
\definecolor{currentfill}{rgb}{0.000000,0.000000,0.000000}%
\pgfsetfillcolor{currentfill}%
\pgfsetlinewidth{1.003750pt}%
\definecolor{currentstroke}{rgb}{0.000000,0.000000,0.000000}%
\pgfsetstrokecolor{currentstroke}%
\pgfsetdash{}{0pt}%
\pgfpathmoveto{\pgfqpoint{4.011666in}{1.771040in}}%
\pgfpathcurveto{\pgfqpoint{4.022716in}{1.771040in}}{\pgfqpoint{4.033315in}{1.775431in}}{\pgfqpoint{4.041128in}{1.783244in}}%
\pgfpathcurveto{\pgfqpoint{4.048942in}{1.791058in}}{\pgfqpoint{4.053332in}{1.801657in}}{\pgfqpoint{4.053332in}{1.812707in}}%
\pgfpathcurveto{\pgfqpoint{4.053332in}{1.823757in}}{\pgfqpoint{4.048942in}{1.834356in}}{\pgfqpoint{4.041128in}{1.842170in}}%
\pgfpathcurveto{\pgfqpoint{4.033315in}{1.849983in}}{\pgfqpoint{4.022716in}{1.854374in}}{\pgfqpoint{4.011666in}{1.854374in}}%
\pgfpathcurveto{\pgfqpoint{4.000616in}{1.854374in}}{\pgfqpoint{3.990016in}{1.849983in}}{\pgfqpoint{3.982203in}{1.842170in}}%
\pgfpathcurveto{\pgfqpoint{3.974389in}{1.834356in}}{\pgfqpoint{3.969999in}{1.823757in}}{\pgfqpoint{3.969999in}{1.812707in}}%
\pgfpathcurveto{\pgfqpoint{3.969999in}{1.801657in}}{\pgfqpoint{3.974389in}{1.791058in}}{\pgfqpoint{3.982203in}{1.783244in}}%
\pgfpathcurveto{\pgfqpoint{3.990016in}{1.775431in}}{\pgfqpoint{4.000616in}{1.771040in}}{\pgfqpoint{4.011666in}{1.771040in}}%
\pgfpathclose%
\pgfusepath{stroke,fill}%
\end{pgfscope}%
\begin{pgfscope}%
\pgfpathrectangle{\pgfqpoint{0.800000in}{0.528000in}}{\pgfqpoint{4.960000in}{3.696000in}}%
\pgfusepath{clip}%
\pgfsetbuttcap%
\pgfsetroundjoin%
\definecolor{currentfill}{rgb}{0.000000,0.000000,0.000000}%
\pgfsetfillcolor{currentfill}%
\pgfsetlinewidth{1.003750pt}%
\definecolor{currentstroke}{rgb}{0.000000,0.000000,0.000000}%
\pgfsetstrokecolor{currentstroke}%
\pgfsetdash{}{0pt}%
\pgfpathmoveto{\pgfqpoint{4.011666in}{1.771040in}}%
\pgfpathcurveto{\pgfqpoint{4.022716in}{1.771040in}}{\pgfqpoint{4.033315in}{1.775431in}}{\pgfqpoint{4.041128in}{1.783244in}}%
\pgfpathcurveto{\pgfqpoint{4.048942in}{1.791058in}}{\pgfqpoint{4.053332in}{1.801657in}}{\pgfqpoint{4.053332in}{1.812707in}}%
\pgfpathcurveto{\pgfqpoint{4.053332in}{1.823757in}}{\pgfqpoint{4.048942in}{1.834356in}}{\pgfqpoint{4.041128in}{1.842170in}}%
\pgfpathcurveto{\pgfqpoint{4.033315in}{1.849983in}}{\pgfqpoint{4.022716in}{1.854374in}}{\pgfqpoint{4.011666in}{1.854374in}}%
\pgfpathcurveto{\pgfqpoint{4.000616in}{1.854374in}}{\pgfqpoint{3.990016in}{1.849983in}}{\pgfqpoint{3.982203in}{1.842170in}}%
\pgfpathcurveto{\pgfqpoint{3.974389in}{1.834356in}}{\pgfqpoint{3.969999in}{1.823757in}}{\pgfqpoint{3.969999in}{1.812707in}}%
\pgfpathcurveto{\pgfqpoint{3.969999in}{1.801657in}}{\pgfqpoint{3.974389in}{1.791058in}}{\pgfqpoint{3.982203in}{1.783244in}}%
\pgfpathcurveto{\pgfqpoint{3.990016in}{1.775431in}}{\pgfqpoint{4.000616in}{1.771040in}}{\pgfqpoint{4.011666in}{1.771040in}}%
\pgfpathclose%
\pgfusepath{stroke,fill}%
\end{pgfscope}%
\begin{pgfscope}%
\pgfpathrectangle{\pgfqpoint{0.800000in}{0.528000in}}{\pgfqpoint{4.960000in}{3.696000in}}%
\pgfusepath{clip}%
\pgfsetbuttcap%
\pgfsetroundjoin%
\definecolor{currentfill}{rgb}{0.000000,0.000000,0.000000}%
\pgfsetfillcolor{currentfill}%
\pgfsetlinewidth{1.003750pt}%
\definecolor{currentstroke}{rgb}{0.000000,0.000000,0.000000}%
\pgfsetstrokecolor{currentstroke}%
\pgfsetdash{}{0pt}%
\pgfpathmoveto{\pgfqpoint{4.011666in}{1.771040in}}%
\pgfpathcurveto{\pgfqpoint{4.022716in}{1.771040in}}{\pgfqpoint{4.033315in}{1.775431in}}{\pgfqpoint{4.041128in}{1.783244in}}%
\pgfpathcurveto{\pgfqpoint{4.048942in}{1.791058in}}{\pgfqpoint{4.053332in}{1.801657in}}{\pgfqpoint{4.053332in}{1.812707in}}%
\pgfpathcurveto{\pgfqpoint{4.053332in}{1.823757in}}{\pgfqpoint{4.048942in}{1.834356in}}{\pgfqpoint{4.041128in}{1.842170in}}%
\pgfpathcurveto{\pgfqpoint{4.033315in}{1.849983in}}{\pgfqpoint{4.022716in}{1.854374in}}{\pgfqpoint{4.011666in}{1.854374in}}%
\pgfpathcurveto{\pgfqpoint{4.000616in}{1.854374in}}{\pgfqpoint{3.990016in}{1.849983in}}{\pgfqpoint{3.982203in}{1.842170in}}%
\pgfpathcurveto{\pgfqpoint{3.974389in}{1.834356in}}{\pgfqpoint{3.969999in}{1.823757in}}{\pgfqpoint{3.969999in}{1.812707in}}%
\pgfpathcurveto{\pgfqpoint{3.969999in}{1.801657in}}{\pgfqpoint{3.974389in}{1.791058in}}{\pgfqpoint{3.982203in}{1.783244in}}%
\pgfpathcurveto{\pgfqpoint{3.990016in}{1.775431in}}{\pgfqpoint{4.000616in}{1.771040in}}{\pgfqpoint{4.011666in}{1.771040in}}%
\pgfpathclose%
\pgfusepath{stroke,fill}%
\end{pgfscope}%
\begin{pgfscope}%
\pgfpathrectangle{\pgfqpoint{0.800000in}{0.528000in}}{\pgfqpoint{4.960000in}{3.696000in}}%
\pgfusepath{clip}%
\pgfsetbuttcap%
\pgfsetroundjoin%
\definecolor{currentfill}{rgb}{0.000000,0.000000,0.000000}%
\pgfsetfillcolor{currentfill}%
\pgfsetlinewidth{1.003750pt}%
\definecolor{currentstroke}{rgb}{0.000000,0.000000,0.000000}%
\pgfsetstrokecolor{currentstroke}%
\pgfsetdash{}{0pt}%
\pgfpathmoveto{\pgfqpoint{4.011666in}{1.771040in}}%
\pgfpathcurveto{\pgfqpoint{4.022716in}{1.771040in}}{\pgfqpoint{4.033315in}{1.775431in}}{\pgfqpoint{4.041128in}{1.783244in}}%
\pgfpathcurveto{\pgfqpoint{4.048942in}{1.791058in}}{\pgfqpoint{4.053332in}{1.801657in}}{\pgfqpoint{4.053332in}{1.812707in}}%
\pgfpathcurveto{\pgfqpoint{4.053332in}{1.823757in}}{\pgfqpoint{4.048942in}{1.834356in}}{\pgfqpoint{4.041128in}{1.842170in}}%
\pgfpathcurveto{\pgfqpoint{4.033315in}{1.849983in}}{\pgfqpoint{4.022716in}{1.854374in}}{\pgfqpoint{4.011666in}{1.854374in}}%
\pgfpathcurveto{\pgfqpoint{4.000616in}{1.854374in}}{\pgfqpoint{3.990016in}{1.849983in}}{\pgfqpoint{3.982203in}{1.842170in}}%
\pgfpathcurveto{\pgfqpoint{3.974389in}{1.834356in}}{\pgfqpoint{3.969999in}{1.823757in}}{\pgfqpoint{3.969999in}{1.812707in}}%
\pgfpathcurveto{\pgfqpoint{3.969999in}{1.801657in}}{\pgfqpoint{3.974389in}{1.791058in}}{\pgfqpoint{3.982203in}{1.783244in}}%
\pgfpathcurveto{\pgfqpoint{3.990016in}{1.775431in}}{\pgfqpoint{4.000616in}{1.771040in}}{\pgfqpoint{4.011666in}{1.771040in}}%
\pgfpathclose%
\pgfusepath{stroke,fill}%
\end{pgfscope}%
\begin{pgfscope}%
\pgfpathrectangle{\pgfqpoint{0.800000in}{0.528000in}}{\pgfqpoint{4.960000in}{3.696000in}}%
\pgfusepath{clip}%
\pgfsetbuttcap%
\pgfsetroundjoin%
\definecolor{currentfill}{rgb}{0.000000,0.000000,0.000000}%
\pgfsetfillcolor{currentfill}%
\pgfsetlinewidth{1.003750pt}%
\definecolor{currentstroke}{rgb}{0.000000,0.000000,0.000000}%
\pgfsetstrokecolor{currentstroke}%
\pgfsetdash{}{0pt}%
\pgfpathmoveto{\pgfqpoint{4.011666in}{2.877687in}}%
\pgfpathcurveto{\pgfqpoint{4.022716in}{2.877687in}}{\pgfqpoint{4.033315in}{2.882077in}}{\pgfqpoint{4.041128in}{2.889891in}}%
\pgfpathcurveto{\pgfqpoint{4.048942in}{2.897704in}}{\pgfqpoint{4.053332in}{2.908303in}}{\pgfqpoint{4.053332in}{2.919353in}}%
\pgfpathcurveto{\pgfqpoint{4.053332in}{2.930404in}}{\pgfqpoint{4.048942in}{2.941003in}}{\pgfqpoint{4.041128in}{2.948816in}}%
\pgfpathcurveto{\pgfqpoint{4.033315in}{2.956630in}}{\pgfqpoint{4.022716in}{2.961020in}}{\pgfqpoint{4.011666in}{2.961020in}}%
\pgfpathcurveto{\pgfqpoint{4.000616in}{2.961020in}}{\pgfqpoint{3.990016in}{2.956630in}}{\pgfqpoint{3.982203in}{2.948816in}}%
\pgfpathcurveto{\pgfqpoint{3.974389in}{2.941003in}}{\pgfqpoint{3.969999in}{2.930404in}}{\pgfqpoint{3.969999in}{2.919353in}}%
\pgfpathcurveto{\pgfqpoint{3.969999in}{2.908303in}}{\pgfqpoint{3.974389in}{2.897704in}}{\pgfqpoint{3.982203in}{2.889891in}}%
\pgfpathcurveto{\pgfqpoint{3.990016in}{2.882077in}}{\pgfqpoint{4.000616in}{2.877687in}}{\pgfqpoint{4.011666in}{2.877687in}}%
\pgfpathclose%
\pgfusepath{stroke,fill}%
\end{pgfscope}%
\begin{pgfscope}%
\pgfpathrectangle{\pgfqpoint{0.800000in}{0.528000in}}{\pgfqpoint{4.960000in}{3.696000in}}%
\pgfusepath{clip}%
\pgfsetbuttcap%
\pgfsetroundjoin%
\definecolor{currentfill}{rgb}{0.000000,0.000000,0.000000}%
\pgfsetfillcolor{currentfill}%
\pgfsetlinewidth{1.003750pt}%
\definecolor{currentstroke}{rgb}{0.000000,0.000000,0.000000}%
\pgfsetstrokecolor{currentstroke}%
\pgfsetdash{}{0pt}%
\pgfpathmoveto{\pgfqpoint{4.011666in}{1.771040in}}%
\pgfpathcurveto{\pgfqpoint{4.022716in}{1.771040in}}{\pgfqpoint{4.033315in}{1.775431in}}{\pgfqpoint{4.041128in}{1.783244in}}%
\pgfpathcurveto{\pgfqpoint{4.048942in}{1.791058in}}{\pgfqpoint{4.053332in}{1.801657in}}{\pgfqpoint{4.053332in}{1.812707in}}%
\pgfpathcurveto{\pgfqpoint{4.053332in}{1.823757in}}{\pgfqpoint{4.048942in}{1.834356in}}{\pgfqpoint{4.041128in}{1.842170in}}%
\pgfpathcurveto{\pgfqpoint{4.033315in}{1.849983in}}{\pgfqpoint{4.022716in}{1.854374in}}{\pgfqpoint{4.011666in}{1.854374in}}%
\pgfpathcurveto{\pgfqpoint{4.000616in}{1.854374in}}{\pgfqpoint{3.990016in}{1.849983in}}{\pgfqpoint{3.982203in}{1.842170in}}%
\pgfpathcurveto{\pgfqpoint{3.974389in}{1.834356in}}{\pgfqpoint{3.969999in}{1.823757in}}{\pgfqpoint{3.969999in}{1.812707in}}%
\pgfpathcurveto{\pgfqpoint{3.969999in}{1.801657in}}{\pgfqpoint{3.974389in}{1.791058in}}{\pgfqpoint{3.982203in}{1.783244in}}%
\pgfpathcurveto{\pgfqpoint{3.990016in}{1.775431in}}{\pgfqpoint{4.000616in}{1.771040in}}{\pgfqpoint{4.011666in}{1.771040in}}%
\pgfpathclose%
\pgfusepath{stroke,fill}%
\end{pgfscope}%
\begin{pgfscope}%
\pgfpathrectangle{\pgfqpoint{0.800000in}{0.528000in}}{\pgfqpoint{4.960000in}{3.696000in}}%
\pgfusepath{clip}%
\pgfsetbuttcap%
\pgfsetroundjoin%
\definecolor{currentfill}{rgb}{0.000000,0.000000,0.000000}%
\pgfsetfillcolor{currentfill}%
\pgfsetlinewidth{1.003750pt}%
\definecolor{currentstroke}{rgb}{0.000000,0.000000,0.000000}%
\pgfsetstrokecolor{currentstroke}%
\pgfsetdash{}{0pt}%
\pgfpathmoveto{\pgfqpoint{4.011666in}{1.771040in}}%
\pgfpathcurveto{\pgfqpoint{4.022716in}{1.771040in}}{\pgfqpoint{4.033315in}{1.775431in}}{\pgfqpoint{4.041128in}{1.783244in}}%
\pgfpathcurveto{\pgfqpoint{4.048942in}{1.791058in}}{\pgfqpoint{4.053332in}{1.801657in}}{\pgfqpoint{4.053332in}{1.812707in}}%
\pgfpathcurveto{\pgfqpoint{4.053332in}{1.823757in}}{\pgfqpoint{4.048942in}{1.834356in}}{\pgfqpoint{4.041128in}{1.842170in}}%
\pgfpathcurveto{\pgfqpoint{4.033315in}{1.849983in}}{\pgfqpoint{4.022716in}{1.854374in}}{\pgfqpoint{4.011666in}{1.854374in}}%
\pgfpathcurveto{\pgfqpoint{4.000616in}{1.854374in}}{\pgfqpoint{3.990016in}{1.849983in}}{\pgfqpoint{3.982203in}{1.842170in}}%
\pgfpathcurveto{\pgfqpoint{3.974389in}{1.834356in}}{\pgfqpoint{3.969999in}{1.823757in}}{\pgfqpoint{3.969999in}{1.812707in}}%
\pgfpathcurveto{\pgfqpoint{3.969999in}{1.801657in}}{\pgfqpoint{3.974389in}{1.791058in}}{\pgfqpoint{3.982203in}{1.783244in}}%
\pgfpathcurveto{\pgfqpoint{3.990016in}{1.775431in}}{\pgfqpoint{4.000616in}{1.771040in}}{\pgfqpoint{4.011666in}{1.771040in}}%
\pgfpathclose%
\pgfusepath{stroke,fill}%
\end{pgfscope}%
\begin{pgfscope}%
\pgfpathrectangle{\pgfqpoint{0.800000in}{0.528000in}}{\pgfqpoint{4.960000in}{3.696000in}}%
\pgfusepath{clip}%
\pgfsetbuttcap%
\pgfsetroundjoin%
\definecolor{currentfill}{rgb}{0.000000,0.000000,0.000000}%
\pgfsetfillcolor{currentfill}%
\pgfsetlinewidth{1.003750pt}%
\definecolor{currentstroke}{rgb}{0.000000,0.000000,0.000000}%
\pgfsetstrokecolor{currentstroke}%
\pgfsetdash{}{0pt}%
\pgfpathmoveto{\pgfqpoint{4.011666in}{2.877687in}}%
\pgfpathcurveto{\pgfqpoint{4.022716in}{2.877687in}}{\pgfqpoint{4.033315in}{2.882077in}}{\pgfqpoint{4.041128in}{2.889891in}}%
\pgfpathcurveto{\pgfqpoint{4.048942in}{2.897704in}}{\pgfqpoint{4.053332in}{2.908303in}}{\pgfqpoint{4.053332in}{2.919353in}}%
\pgfpathcurveto{\pgfqpoint{4.053332in}{2.930404in}}{\pgfqpoint{4.048942in}{2.941003in}}{\pgfqpoint{4.041128in}{2.948816in}}%
\pgfpathcurveto{\pgfqpoint{4.033315in}{2.956630in}}{\pgfqpoint{4.022716in}{2.961020in}}{\pgfqpoint{4.011666in}{2.961020in}}%
\pgfpathcurveto{\pgfqpoint{4.000616in}{2.961020in}}{\pgfqpoint{3.990016in}{2.956630in}}{\pgfqpoint{3.982203in}{2.948816in}}%
\pgfpathcurveto{\pgfqpoint{3.974389in}{2.941003in}}{\pgfqpoint{3.969999in}{2.930404in}}{\pgfqpoint{3.969999in}{2.919353in}}%
\pgfpathcurveto{\pgfqpoint{3.969999in}{2.908303in}}{\pgfqpoint{3.974389in}{2.897704in}}{\pgfqpoint{3.982203in}{2.889891in}}%
\pgfpathcurveto{\pgfqpoint{3.990016in}{2.882077in}}{\pgfqpoint{4.000616in}{2.877687in}}{\pgfqpoint{4.011666in}{2.877687in}}%
\pgfpathclose%
\pgfusepath{stroke,fill}%
\end{pgfscope}%
\begin{pgfscope}%
\pgfpathrectangle{\pgfqpoint{0.800000in}{0.528000in}}{\pgfqpoint{4.960000in}{3.696000in}}%
\pgfusepath{clip}%
\pgfsetbuttcap%
\pgfsetroundjoin%
\definecolor{currentfill}{rgb}{0.000000,0.000000,0.000000}%
\pgfsetfillcolor{currentfill}%
\pgfsetlinewidth{1.003750pt}%
\definecolor{currentstroke}{rgb}{0.000000,0.000000,0.000000}%
\pgfsetstrokecolor{currentstroke}%
\pgfsetdash{}{0pt}%
\pgfpathmoveto{\pgfqpoint{4.011666in}{2.877687in}}%
\pgfpathcurveto{\pgfqpoint{4.022716in}{2.877687in}}{\pgfqpoint{4.033315in}{2.882077in}}{\pgfqpoint{4.041128in}{2.889891in}}%
\pgfpathcurveto{\pgfqpoint{4.048942in}{2.897704in}}{\pgfqpoint{4.053332in}{2.908303in}}{\pgfqpoint{4.053332in}{2.919353in}}%
\pgfpathcurveto{\pgfqpoint{4.053332in}{2.930404in}}{\pgfqpoint{4.048942in}{2.941003in}}{\pgfqpoint{4.041128in}{2.948816in}}%
\pgfpathcurveto{\pgfqpoint{4.033315in}{2.956630in}}{\pgfqpoint{4.022716in}{2.961020in}}{\pgfqpoint{4.011666in}{2.961020in}}%
\pgfpathcurveto{\pgfqpoint{4.000616in}{2.961020in}}{\pgfqpoint{3.990016in}{2.956630in}}{\pgfqpoint{3.982203in}{2.948816in}}%
\pgfpathcurveto{\pgfqpoint{3.974389in}{2.941003in}}{\pgfqpoint{3.969999in}{2.930404in}}{\pgfqpoint{3.969999in}{2.919353in}}%
\pgfpathcurveto{\pgfqpoint{3.969999in}{2.908303in}}{\pgfqpoint{3.974389in}{2.897704in}}{\pgfqpoint{3.982203in}{2.889891in}}%
\pgfpathcurveto{\pgfqpoint{3.990016in}{2.882077in}}{\pgfqpoint{4.000616in}{2.877687in}}{\pgfqpoint{4.011666in}{2.877687in}}%
\pgfpathclose%
\pgfusepath{stroke,fill}%
\end{pgfscope}%
\begin{pgfscope}%
\pgfpathrectangle{\pgfqpoint{0.800000in}{0.528000in}}{\pgfqpoint{4.960000in}{3.696000in}}%
\pgfusepath{clip}%
\pgfsetbuttcap%
\pgfsetroundjoin%
\definecolor{currentfill}{rgb}{0.000000,0.000000,0.000000}%
\pgfsetfillcolor{currentfill}%
\pgfsetlinewidth{1.003750pt}%
\definecolor{currentstroke}{rgb}{0.000000,0.000000,0.000000}%
\pgfsetstrokecolor{currentstroke}%
\pgfsetdash{}{0pt}%
\pgfpathmoveto{\pgfqpoint{4.011666in}{1.771040in}}%
\pgfpathcurveto{\pgfqpoint{4.022716in}{1.771040in}}{\pgfqpoint{4.033315in}{1.775431in}}{\pgfqpoint{4.041128in}{1.783244in}}%
\pgfpathcurveto{\pgfqpoint{4.048942in}{1.791058in}}{\pgfqpoint{4.053332in}{1.801657in}}{\pgfqpoint{4.053332in}{1.812707in}}%
\pgfpathcurveto{\pgfqpoint{4.053332in}{1.823757in}}{\pgfqpoint{4.048942in}{1.834356in}}{\pgfqpoint{4.041128in}{1.842170in}}%
\pgfpathcurveto{\pgfqpoint{4.033315in}{1.849983in}}{\pgfqpoint{4.022716in}{1.854374in}}{\pgfqpoint{4.011666in}{1.854374in}}%
\pgfpathcurveto{\pgfqpoint{4.000616in}{1.854374in}}{\pgfqpoint{3.990016in}{1.849983in}}{\pgfqpoint{3.982203in}{1.842170in}}%
\pgfpathcurveto{\pgfqpoint{3.974389in}{1.834356in}}{\pgfqpoint{3.969999in}{1.823757in}}{\pgfqpoint{3.969999in}{1.812707in}}%
\pgfpathcurveto{\pgfqpoint{3.969999in}{1.801657in}}{\pgfqpoint{3.974389in}{1.791058in}}{\pgfqpoint{3.982203in}{1.783244in}}%
\pgfpathcurveto{\pgfqpoint{3.990016in}{1.775431in}}{\pgfqpoint{4.000616in}{1.771040in}}{\pgfqpoint{4.011666in}{1.771040in}}%
\pgfpathclose%
\pgfusepath{stroke,fill}%
\end{pgfscope}%
\begin{pgfscope}%
\pgfpathrectangle{\pgfqpoint{0.800000in}{0.528000in}}{\pgfqpoint{4.960000in}{3.696000in}}%
\pgfusepath{clip}%
\pgfsetbuttcap%
\pgfsetroundjoin%
\definecolor{currentfill}{rgb}{0.000000,0.000000,0.000000}%
\pgfsetfillcolor{currentfill}%
\pgfsetlinewidth{1.003750pt}%
\definecolor{currentstroke}{rgb}{0.000000,0.000000,0.000000}%
\pgfsetstrokecolor{currentstroke}%
\pgfsetdash{}{0pt}%
\pgfpathmoveto{\pgfqpoint{4.011666in}{1.771040in}}%
\pgfpathcurveto{\pgfqpoint{4.022716in}{1.771040in}}{\pgfqpoint{4.033315in}{1.775431in}}{\pgfqpoint{4.041128in}{1.783244in}}%
\pgfpathcurveto{\pgfqpoint{4.048942in}{1.791058in}}{\pgfqpoint{4.053332in}{1.801657in}}{\pgfqpoint{4.053332in}{1.812707in}}%
\pgfpathcurveto{\pgfqpoint{4.053332in}{1.823757in}}{\pgfqpoint{4.048942in}{1.834356in}}{\pgfqpoint{4.041128in}{1.842170in}}%
\pgfpathcurveto{\pgfqpoint{4.033315in}{1.849983in}}{\pgfqpoint{4.022716in}{1.854374in}}{\pgfqpoint{4.011666in}{1.854374in}}%
\pgfpathcurveto{\pgfqpoint{4.000616in}{1.854374in}}{\pgfqpoint{3.990016in}{1.849983in}}{\pgfqpoint{3.982203in}{1.842170in}}%
\pgfpathcurveto{\pgfqpoint{3.974389in}{1.834356in}}{\pgfqpoint{3.969999in}{1.823757in}}{\pgfqpoint{3.969999in}{1.812707in}}%
\pgfpathcurveto{\pgfqpoint{3.969999in}{1.801657in}}{\pgfqpoint{3.974389in}{1.791058in}}{\pgfqpoint{3.982203in}{1.783244in}}%
\pgfpathcurveto{\pgfqpoint{3.990016in}{1.775431in}}{\pgfqpoint{4.000616in}{1.771040in}}{\pgfqpoint{4.011666in}{1.771040in}}%
\pgfpathclose%
\pgfusepath{stroke,fill}%
\end{pgfscope}%
\begin{pgfscope}%
\pgfpathrectangle{\pgfqpoint{0.800000in}{0.528000in}}{\pgfqpoint{4.960000in}{3.696000in}}%
\pgfusepath{clip}%
\pgfsetbuttcap%
\pgfsetroundjoin%
\definecolor{currentfill}{rgb}{0.000000,0.000000,0.000000}%
\pgfsetfillcolor{currentfill}%
\pgfsetlinewidth{1.003750pt}%
\definecolor{currentstroke}{rgb}{0.000000,0.000000,0.000000}%
\pgfsetstrokecolor{currentstroke}%
\pgfsetdash{}{0pt}%
\pgfpathmoveto{\pgfqpoint{4.011666in}{1.771040in}}%
\pgfpathcurveto{\pgfqpoint{4.022716in}{1.771040in}}{\pgfqpoint{4.033315in}{1.775431in}}{\pgfqpoint{4.041128in}{1.783244in}}%
\pgfpathcurveto{\pgfqpoint{4.048942in}{1.791058in}}{\pgfqpoint{4.053332in}{1.801657in}}{\pgfqpoint{4.053332in}{1.812707in}}%
\pgfpathcurveto{\pgfqpoint{4.053332in}{1.823757in}}{\pgfqpoint{4.048942in}{1.834356in}}{\pgfqpoint{4.041128in}{1.842170in}}%
\pgfpathcurveto{\pgfqpoint{4.033315in}{1.849983in}}{\pgfqpoint{4.022716in}{1.854374in}}{\pgfqpoint{4.011666in}{1.854374in}}%
\pgfpathcurveto{\pgfqpoint{4.000616in}{1.854374in}}{\pgfqpoint{3.990016in}{1.849983in}}{\pgfqpoint{3.982203in}{1.842170in}}%
\pgfpathcurveto{\pgfqpoint{3.974389in}{1.834356in}}{\pgfqpoint{3.969999in}{1.823757in}}{\pgfqpoint{3.969999in}{1.812707in}}%
\pgfpathcurveto{\pgfqpoint{3.969999in}{1.801657in}}{\pgfqpoint{3.974389in}{1.791058in}}{\pgfqpoint{3.982203in}{1.783244in}}%
\pgfpathcurveto{\pgfqpoint{3.990016in}{1.775431in}}{\pgfqpoint{4.000616in}{1.771040in}}{\pgfqpoint{4.011666in}{1.771040in}}%
\pgfpathclose%
\pgfusepath{stroke,fill}%
\end{pgfscope}%
\begin{pgfscope}%
\pgfpathrectangle{\pgfqpoint{0.800000in}{0.528000in}}{\pgfqpoint{4.960000in}{3.696000in}}%
\pgfusepath{clip}%
\pgfsetbuttcap%
\pgfsetroundjoin%
\definecolor{currentfill}{rgb}{0.000000,0.000000,0.000000}%
\pgfsetfillcolor{currentfill}%
\pgfsetlinewidth{1.003750pt}%
\definecolor{currentstroke}{rgb}{0.000000,0.000000,0.000000}%
\pgfsetstrokecolor{currentstroke}%
\pgfsetdash{}{0pt}%
\pgfpathmoveto{\pgfqpoint{4.011666in}{1.771040in}}%
\pgfpathcurveto{\pgfqpoint{4.022716in}{1.771040in}}{\pgfqpoint{4.033315in}{1.775431in}}{\pgfqpoint{4.041128in}{1.783244in}}%
\pgfpathcurveto{\pgfqpoint{4.048942in}{1.791058in}}{\pgfqpoint{4.053332in}{1.801657in}}{\pgfqpoint{4.053332in}{1.812707in}}%
\pgfpathcurveto{\pgfqpoint{4.053332in}{1.823757in}}{\pgfqpoint{4.048942in}{1.834356in}}{\pgfqpoint{4.041128in}{1.842170in}}%
\pgfpathcurveto{\pgfqpoint{4.033315in}{1.849983in}}{\pgfqpoint{4.022716in}{1.854374in}}{\pgfqpoint{4.011666in}{1.854374in}}%
\pgfpathcurveto{\pgfqpoint{4.000616in}{1.854374in}}{\pgfqpoint{3.990016in}{1.849983in}}{\pgfqpoint{3.982203in}{1.842170in}}%
\pgfpathcurveto{\pgfqpoint{3.974389in}{1.834356in}}{\pgfqpoint{3.969999in}{1.823757in}}{\pgfqpoint{3.969999in}{1.812707in}}%
\pgfpathcurveto{\pgfqpoint{3.969999in}{1.801657in}}{\pgfqpoint{3.974389in}{1.791058in}}{\pgfqpoint{3.982203in}{1.783244in}}%
\pgfpathcurveto{\pgfqpoint{3.990016in}{1.775431in}}{\pgfqpoint{4.000616in}{1.771040in}}{\pgfqpoint{4.011666in}{1.771040in}}%
\pgfpathclose%
\pgfusepath{stroke,fill}%
\end{pgfscope}%
\begin{pgfscope}%
\pgfpathrectangle{\pgfqpoint{0.800000in}{0.528000in}}{\pgfqpoint{4.960000in}{3.696000in}}%
\pgfusepath{clip}%
\pgfsetbuttcap%
\pgfsetroundjoin%
\definecolor{currentfill}{rgb}{0.000000,0.000000,0.000000}%
\pgfsetfillcolor{currentfill}%
\pgfsetlinewidth{1.003750pt}%
\definecolor{currentstroke}{rgb}{0.000000,0.000000,0.000000}%
\pgfsetstrokecolor{currentstroke}%
\pgfsetdash{}{0pt}%
\pgfpathmoveto{\pgfqpoint{4.011666in}{2.877687in}}%
\pgfpathcurveto{\pgfqpoint{4.022716in}{2.877687in}}{\pgfqpoint{4.033315in}{2.882077in}}{\pgfqpoint{4.041128in}{2.889891in}}%
\pgfpathcurveto{\pgfqpoint{4.048942in}{2.897704in}}{\pgfqpoint{4.053332in}{2.908303in}}{\pgfqpoint{4.053332in}{2.919353in}}%
\pgfpathcurveto{\pgfqpoint{4.053332in}{2.930404in}}{\pgfqpoint{4.048942in}{2.941003in}}{\pgfqpoint{4.041128in}{2.948816in}}%
\pgfpathcurveto{\pgfqpoint{4.033315in}{2.956630in}}{\pgfqpoint{4.022716in}{2.961020in}}{\pgfqpoint{4.011666in}{2.961020in}}%
\pgfpathcurveto{\pgfqpoint{4.000616in}{2.961020in}}{\pgfqpoint{3.990016in}{2.956630in}}{\pgfqpoint{3.982203in}{2.948816in}}%
\pgfpathcurveto{\pgfqpoint{3.974389in}{2.941003in}}{\pgfqpoint{3.969999in}{2.930404in}}{\pgfqpoint{3.969999in}{2.919353in}}%
\pgfpathcurveto{\pgfqpoint{3.969999in}{2.908303in}}{\pgfqpoint{3.974389in}{2.897704in}}{\pgfqpoint{3.982203in}{2.889891in}}%
\pgfpathcurveto{\pgfqpoint{3.990016in}{2.882077in}}{\pgfqpoint{4.000616in}{2.877687in}}{\pgfqpoint{4.011666in}{2.877687in}}%
\pgfpathclose%
\pgfusepath{stroke,fill}%
\end{pgfscope}%
\begin{pgfscope}%
\pgfpathrectangle{\pgfqpoint{0.800000in}{0.528000in}}{\pgfqpoint{4.960000in}{3.696000in}}%
\pgfusepath{clip}%
\pgfsetbuttcap%
\pgfsetroundjoin%
\definecolor{currentfill}{rgb}{0.000000,0.000000,0.000000}%
\pgfsetfillcolor{currentfill}%
\pgfsetlinewidth{1.003750pt}%
\definecolor{currentstroke}{rgb}{0.000000,0.000000,0.000000}%
\pgfsetstrokecolor{currentstroke}%
\pgfsetdash{}{0pt}%
\pgfpathmoveto{\pgfqpoint{4.011666in}{2.877687in}}%
\pgfpathcurveto{\pgfqpoint{4.022716in}{2.877687in}}{\pgfqpoint{4.033315in}{2.882077in}}{\pgfqpoint{4.041128in}{2.889891in}}%
\pgfpathcurveto{\pgfqpoint{4.048942in}{2.897704in}}{\pgfqpoint{4.053332in}{2.908303in}}{\pgfqpoint{4.053332in}{2.919353in}}%
\pgfpathcurveto{\pgfqpoint{4.053332in}{2.930404in}}{\pgfqpoint{4.048942in}{2.941003in}}{\pgfqpoint{4.041128in}{2.948816in}}%
\pgfpathcurveto{\pgfqpoint{4.033315in}{2.956630in}}{\pgfqpoint{4.022716in}{2.961020in}}{\pgfqpoint{4.011666in}{2.961020in}}%
\pgfpathcurveto{\pgfqpoint{4.000616in}{2.961020in}}{\pgfqpoint{3.990016in}{2.956630in}}{\pgfqpoint{3.982203in}{2.948816in}}%
\pgfpathcurveto{\pgfqpoint{3.974389in}{2.941003in}}{\pgfqpoint{3.969999in}{2.930404in}}{\pgfqpoint{3.969999in}{2.919353in}}%
\pgfpathcurveto{\pgfqpoint{3.969999in}{2.908303in}}{\pgfqpoint{3.974389in}{2.897704in}}{\pgfqpoint{3.982203in}{2.889891in}}%
\pgfpathcurveto{\pgfqpoint{3.990016in}{2.882077in}}{\pgfqpoint{4.000616in}{2.877687in}}{\pgfqpoint{4.011666in}{2.877687in}}%
\pgfpathclose%
\pgfusepath{stroke,fill}%
\end{pgfscope}%
\begin{pgfscope}%
\pgfpathrectangle{\pgfqpoint{0.800000in}{0.528000in}}{\pgfqpoint{4.960000in}{3.696000in}}%
\pgfusepath{clip}%
\pgfsetbuttcap%
\pgfsetroundjoin%
\definecolor{currentfill}{rgb}{0.000000,0.000000,0.000000}%
\pgfsetfillcolor{currentfill}%
\pgfsetlinewidth{1.003750pt}%
\definecolor{currentstroke}{rgb}{0.000000,0.000000,0.000000}%
\pgfsetstrokecolor{currentstroke}%
\pgfsetdash{}{0pt}%
\pgfpathmoveto{\pgfqpoint{4.011666in}{1.771040in}}%
\pgfpathcurveto{\pgfqpoint{4.022716in}{1.771040in}}{\pgfqpoint{4.033315in}{1.775431in}}{\pgfqpoint{4.041128in}{1.783244in}}%
\pgfpathcurveto{\pgfqpoint{4.048942in}{1.791058in}}{\pgfqpoint{4.053332in}{1.801657in}}{\pgfqpoint{4.053332in}{1.812707in}}%
\pgfpathcurveto{\pgfqpoint{4.053332in}{1.823757in}}{\pgfqpoint{4.048942in}{1.834356in}}{\pgfqpoint{4.041128in}{1.842170in}}%
\pgfpathcurveto{\pgfqpoint{4.033315in}{1.849983in}}{\pgfqpoint{4.022716in}{1.854374in}}{\pgfqpoint{4.011666in}{1.854374in}}%
\pgfpathcurveto{\pgfqpoint{4.000616in}{1.854374in}}{\pgfqpoint{3.990016in}{1.849983in}}{\pgfqpoint{3.982203in}{1.842170in}}%
\pgfpathcurveto{\pgfqpoint{3.974389in}{1.834356in}}{\pgfqpoint{3.969999in}{1.823757in}}{\pgfqpoint{3.969999in}{1.812707in}}%
\pgfpathcurveto{\pgfqpoint{3.969999in}{1.801657in}}{\pgfqpoint{3.974389in}{1.791058in}}{\pgfqpoint{3.982203in}{1.783244in}}%
\pgfpathcurveto{\pgfqpoint{3.990016in}{1.775431in}}{\pgfqpoint{4.000616in}{1.771040in}}{\pgfqpoint{4.011666in}{1.771040in}}%
\pgfpathclose%
\pgfusepath{stroke,fill}%
\end{pgfscope}%
\begin{pgfscope}%
\pgfpathrectangle{\pgfqpoint{0.800000in}{0.528000in}}{\pgfqpoint{4.960000in}{3.696000in}}%
\pgfusepath{clip}%
\pgfsetbuttcap%
\pgfsetroundjoin%
\definecolor{currentfill}{rgb}{0.000000,0.000000,0.000000}%
\pgfsetfillcolor{currentfill}%
\pgfsetlinewidth{1.003750pt}%
\definecolor{currentstroke}{rgb}{0.000000,0.000000,0.000000}%
\pgfsetstrokecolor{currentstroke}%
\pgfsetdash{}{0pt}%
\pgfpathmoveto{\pgfqpoint{4.011666in}{1.771040in}}%
\pgfpathcurveto{\pgfqpoint{4.022716in}{1.771040in}}{\pgfqpoint{4.033315in}{1.775431in}}{\pgfqpoint{4.041128in}{1.783244in}}%
\pgfpathcurveto{\pgfqpoint{4.048942in}{1.791058in}}{\pgfqpoint{4.053332in}{1.801657in}}{\pgfqpoint{4.053332in}{1.812707in}}%
\pgfpathcurveto{\pgfqpoint{4.053332in}{1.823757in}}{\pgfqpoint{4.048942in}{1.834356in}}{\pgfqpoint{4.041128in}{1.842170in}}%
\pgfpathcurveto{\pgfqpoint{4.033315in}{1.849983in}}{\pgfqpoint{4.022716in}{1.854374in}}{\pgfqpoint{4.011666in}{1.854374in}}%
\pgfpathcurveto{\pgfqpoint{4.000616in}{1.854374in}}{\pgfqpoint{3.990016in}{1.849983in}}{\pgfqpoint{3.982203in}{1.842170in}}%
\pgfpathcurveto{\pgfqpoint{3.974389in}{1.834356in}}{\pgfqpoint{3.969999in}{1.823757in}}{\pgfqpoint{3.969999in}{1.812707in}}%
\pgfpathcurveto{\pgfqpoint{3.969999in}{1.801657in}}{\pgfqpoint{3.974389in}{1.791058in}}{\pgfqpoint{3.982203in}{1.783244in}}%
\pgfpathcurveto{\pgfqpoint{3.990016in}{1.775431in}}{\pgfqpoint{4.000616in}{1.771040in}}{\pgfqpoint{4.011666in}{1.771040in}}%
\pgfpathclose%
\pgfusepath{stroke,fill}%
\end{pgfscope}%
\begin{pgfscope}%
\pgfpathrectangle{\pgfqpoint{0.800000in}{0.528000in}}{\pgfqpoint{4.960000in}{3.696000in}}%
\pgfusepath{clip}%
\pgfsetbuttcap%
\pgfsetroundjoin%
\definecolor{currentfill}{rgb}{0.000000,0.000000,0.000000}%
\pgfsetfillcolor{currentfill}%
\pgfsetlinewidth{1.003750pt}%
\definecolor{currentstroke}{rgb}{0.000000,0.000000,0.000000}%
\pgfsetstrokecolor{currentstroke}%
\pgfsetdash{}{0pt}%
\pgfpathmoveto{\pgfqpoint{4.011666in}{2.877687in}}%
\pgfpathcurveto{\pgfqpoint{4.022716in}{2.877687in}}{\pgfqpoint{4.033315in}{2.882077in}}{\pgfqpoint{4.041128in}{2.889891in}}%
\pgfpathcurveto{\pgfqpoint{4.048942in}{2.897704in}}{\pgfqpoint{4.053332in}{2.908303in}}{\pgfqpoint{4.053332in}{2.919353in}}%
\pgfpathcurveto{\pgfqpoint{4.053332in}{2.930404in}}{\pgfqpoint{4.048942in}{2.941003in}}{\pgfqpoint{4.041128in}{2.948816in}}%
\pgfpathcurveto{\pgfqpoint{4.033315in}{2.956630in}}{\pgfqpoint{4.022716in}{2.961020in}}{\pgfqpoint{4.011666in}{2.961020in}}%
\pgfpathcurveto{\pgfqpoint{4.000616in}{2.961020in}}{\pgfqpoint{3.990016in}{2.956630in}}{\pgfqpoint{3.982203in}{2.948816in}}%
\pgfpathcurveto{\pgfqpoint{3.974389in}{2.941003in}}{\pgfqpoint{3.969999in}{2.930404in}}{\pgfqpoint{3.969999in}{2.919353in}}%
\pgfpathcurveto{\pgfqpoint{3.969999in}{2.908303in}}{\pgfqpoint{3.974389in}{2.897704in}}{\pgfqpoint{3.982203in}{2.889891in}}%
\pgfpathcurveto{\pgfqpoint{3.990016in}{2.882077in}}{\pgfqpoint{4.000616in}{2.877687in}}{\pgfqpoint{4.011666in}{2.877687in}}%
\pgfpathclose%
\pgfusepath{stroke,fill}%
\end{pgfscope}%
\begin{pgfscope}%
\pgfpathrectangle{\pgfqpoint{0.800000in}{0.528000in}}{\pgfqpoint{4.960000in}{3.696000in}}%
\pgfusepath{clip}%
\pgfsetbuttcap%
\pgfsetroundjoin%
\definecolor{currentfill}{rgb}{0.000000,0.000000,0.000000}%
\pgfsetfillcolor{currentfill}%
\pgfsetlinewidth{1.003750pt}%
\definecolor{currentstroke}{rgb}{0.000000,0.000000,0.000000}%
\pgfsetstrokecolor{currentstroke}%
\pgfsetdash{}{0pt}%
\pgfpathmoveto{\pgfqpoint{4.011666in}{1.771040in}}%
\pgfpathcurveto{\pgfqpoint{4.022716in}{1.771040in}}{\pgfqpoint{4.033315in}{1.775431in}}{\pgfqpoint{4.041128in}{1.783244in}}%
\pgfpathcurveto{\pgfqpoint{4.048942in}{1.791058in}}{\pgfqpoint{4.053332in}{1.801657in}}{\pgfqpoint{4.053332in}{1.812707in}}%
\pgfpathcurveto{\pgfqpoint{4.053332in}{1.823757in}}{\pgfqpoint{4.048942in}{1.834356in}}{\pgfqpoint{4.041128in}{1.842170in}}%
\pgfpathcurveto{\pgfqpoint{4.033315in}{1.849983in}}{\pgfqpoint{4.022716in}{1.854374in}}{\pgfqpoint{4.011666in}{1.854374in}}%
\pgfpathcurveto{\pgfqpoint{4.000616in}{1.854374in}}{\pgfqpoint{3.990016in}{1.849983in}}{\pgfqpoint{3.982203in}{1.842170in}}%
\pgfpathcurveto{\pgfqpoint{3.974389in}{1.834356in}}{\pgfqpoint{3.969999in}{1.823757in}}{\pgfqpoint{3.969999in}{1.812707in}}%
\pgfpathcurveto{\pgfqpoint{3.969999in}{1.801657in}}{\pgfqpoint{3.974389in}{1.791058in}}{\pgfqpoint{3.982203in}{1.783244in}}%
\pgfpathcurveto{\pgfqpoint{3.990016in}{1.775431in}}{\pgfqpoint{4.000616in}{1.771040in}}{\pgfqpoint{4.011666in}{1.771040in}}%
\pgfpathclose%
\pgfusepath{stroke,fill}%
\end{pgfscope}%
\begin{pgfscope}%
\pgfpathrectangle{\pgfqpoint{0.800000in}{0.528000in}}{\pgfqpoint{4.960000in}{3.696000in}}%
\pgfusepath{clip}%
\pgfsetbuttcap%
\pgfsetroundjoin%
\definecolor{currentfill}{rgb}{0.000000,0.000000,0.000000}%
\pgfsetfillcolor{currentfill}%
\pgfsetlinewidth{1.003750pt}%
\definecolor{currentstroke}{rgb}{0.000000,0.000000,0.000000}%
\pgfsetstrokecolor{currentstroke}%
\pgfsetdash{}{0pt}%
\pgfpathmoveto{\pgfqpoint{4.011666in}{1.771040in}}%
\pgfpathcurveto{\pgfqpoint{4.022716in}{1.771040in}}{\pgfqpoint{4.033315in}{1.775431in}}{\pgfqpoint{4.041128in}{1.783244in}}%
\pgfpathcurveto{\pgfqpoint{4.048942in}{1.791058in}}{\pgfqpoint{4.053332in}{1.801657in}}{\pgfqpoint{4.053332in}{1.812707in}}%
\pgfpathcurveto{\pgfqpoint{4.053332in}{1.823757in}}{\pgfqpoint{4.048942in}{1.834356in}}{\pgfqpoint{4.041128in}{1.842170in}}%
\pgfpathcurveto{\pgfqpoint{4.033315in}{1.849983in}}{\pgfqpoint{4.022716in}{1.854374in}}{\pgfqpoint{4.011666in}{1.854374in}}%
\pgfpathcurveto{\pgfqpoint{4.000616in}{1.854374in}}{\pgfqpoint{3.990016in}{1.849983in}}{\pgfqpoint{3.982203in}{1.842170in}}%
\pgfpathcurveto{\pgfqpoint{3.974389in}{1.834356in}}{\pgfqpoint{3.969999in}{1.823757in}}{\pgfqpoint{3.969999in}{1.812707in}}%
\pgfpathcurveto{\pgfqpoint{3.969999in}{1.801657in}}{\pgfqpoint{3.974389in}{1.791058in}}{\pgfqpoint{3.982203in}{1.783244in}}%
\pgfpathcurveto{\pgfqpoint{3.990016in}{1.775431in}}{\pgfqpoint{4.000616in}{1.771040in}}{\pgfqpoint{4.011666in}{1.771040in}}%
\pgfpathclose%
\pgfusepath{stroke,fill}%
\end{pgfscope}%
\begin{pgfscope}%
\pgfpathrectangle{\pgfqpoint{0.800000in}{0.528000in}}{\pgfqpoint{4.960000in}{3.696000in}}%
\pgfusepath{clip}%
\pgfsetbuttcap%
\pgfsetroundjoin%
\definecolor{currentfill}{rgb}{0.000000,0.000000,0.000000}%
\pgfsetfillcolor{currentfill}%
\pgfsetlinewidth{1.003750pt}%
\definecolor{currentstroke}{rgb}{0.000000,0.000000,0.000000}%
\pgfsetstrokecolor{currentstroke}%
\pgfsetdash{}{0pt}%
\pgfpathmoveto{\pgfqpoint{4.011666in}{1.771040in}}%
\pgfpathcurveto{\pgfqpoint{4.022716in}{1.771040in}}{\pgfqpoint{4.033315in}{1.775431in}}{\pgfqpoint{4.041128in}{1.783244in}}%
\pgfpathcurveto{\pgfqpoint{4.048942in}{1.791058in}}{\pgfqpoint{4.053332in}{1.801657in}}{\pgfqpoint{4.053332in}{1.812707in}}%
\pgfpathcurveto{\pgfqpoint{4.053332in}{1.823757in}}{\pgfqpoint{4.048942in}{1.834356in}}{\pgfqpoint{4.041128in}{1.842170in}}%
\pgfpathcurveto{\pgfqpoint{4.033315in}{1.849983in}}{\pgfqpoint{4.022716in}{1.854374in}}{\pgfqpoint{4.011666in}{1.854374in}}%
\pgfpathcurveto{\pgfqpoint{4.000616in}{1.854374in}}{\pgfqpoint{3.990016in}{1.849983in}}{\pgfqpoint{3.982203in}{1.842170in}}%
\pgfpathcurveto{\pgfqpoint{3.974389in}{1.834356in}}{\pgfqpoint{3.969999in}{1.823757in}}{\pgfqpoint{3.969999in}{1.812707in}}%
\pgfpathcurveto{\pgfqpoint{3.969999in}{1.801657in}}{\pgfqpoint{3.974389in}{1.791058in}}{\pgfqpoint{3.982203in}{1.783244in}}%
\pgfpathcurveto{\pgfqpoint{3.990016in}{1.775431in}}{\pgfqpoint{4.000616in}{1.771040in}}{\pgfqpoint{4.011666in}{1.771040in}}%
\pgfpathclose%
\pgfusepath{stroke,fill}%
\end{pgfscope}%
\begin{pgfscope}%
\pgfpathrectangle{\pgfqpoint{0.800000in}{0.528000in}}{\pgfqpoint{4.960000in}{3.696000in}}%
\pgfusepath{clip}%
\pgfsetbuttcap%
\pgfsetroundjoin%
\definecolor{currentfill}{rgb}{0.000000,0.000000,0.000000}%
\pgfsetfillcolor{currentfill}%
\pgfsetlinewidth{1.003750pt}%
\definecolor{currentstroke}{rgb}{0.000000,0.000000,0.000000}%
\pgfsetstrokecolor{currentstroke}%
\pgfsetdash{}{0pt}%
\pgfpathmoveto{\pgfqpoint{4.011666in}{2.877687in}}%
\pgfpathcurveto{\pgfqpoint{4.022716in}{2.877687in}}{\pgfqpoint{4.033315in}{2.882077in}}{\pgfqpoint{4.041128in}{2.889891in}}%
\pgfpathcurveto{\pgfqpoint{4.048942in}{2.897704in}}{\pgfqpoint{4.053332in}{2.908303in}}{\pgfqpoint{4.053332in}{2.919353in}}%
\pgfpathcurveto{\pgfqpoint{4.053332in}{2.930404in}}{\pgfqpoint{4.048942in}{2.941003in}}{\pgfqpoint{4.041128in}{2.948816in}}%
\pgfpathcurveto{\pgfqpoint{4.033315in}{2.956630in}}{\pgfqpoint{4.022716in}{2.961020in}}{\pgfqpoint{4.011666in}{2.961020in}}%
\pgfpathcurveto{\pgfqpoint{4.000616in}{2.961020in}}{\pgfqpoint{3.990016in}{2.956630in}}{\pgfqpoint{3.982203in}{2.948816in}}%
\pgfpathcurveto{\pgfqpoint{3.974389in}{2.941003in}}{\pgfqpoint{3.969999in}{2.930404in}}{\pgfqpoint{3.969999in}{2.919353in}}%
\pgfpathcurveto{\pgfqpoint{3.969999in}{2.908303in}}{\pgfqpoint{3.974389in}{2.897704in}}{\pgfqpoint{3.982203in}{2.889891in}}%
\pgfpathcurveto{\pgfqpoint{3.990016in}{2.882077in}}{\pgfqpoint{4.000616in}{2.877687in}}{\pgfqpoint{4.011666in}{2.877687in}}%
\pgfpathclose%
\pgfusepath{stroke,fill}%
\end{pgfscope}%
\begin{pgfscope}%
\pgfpathrectangle{\pgfqpoint{0.800000in}{0.528000in}}{\pgfqpoint{4.960000in}{3.696000in}}%
\pgfusepath{clip}%
\pgfsetbuttcap%
\pgfsetroundjoin%
\definecolor{currentfill}{rgb}{0.000000,0.000000,0.000000}%
\pgfsetfillcolor{currentfill}%
\pgfsetlinewidth{1.003750pt}%
\definecolor{currentstroke}{rgb}{0.000000,0.000000,0.000000}%
\pgfsetstrokecolor{currentstroke}%
\pgfsetdash{}{0pt}%
\pgfpathmoveto{\pgfqpoint{4.011666in}{1.771040in}}%
\pgfpathcurveto{\pgfqpoint{4.022716in}{1.771040in}}{\pgfqpoint{4.033315in}{1.775431in}}{\pgfqpoint{4.041128in}{1.783244in}}%
\pgfpathcurveto{\pgfqpoint{4.048942in}{1.791058in}}{\pgfqpoint{4.053332in}{1.801657in}}{\pgfqpoint{4.053332in}{1.812707in}}%
\pgfpathcurveto{\pgfqpoint{4.053332in}{1.823757in}}{\pgfqpoint{4.048942in}{1.834356in}}{\pgfqpoint{4.041128in}{1.842170in}}%
\pgfpathcurveto{\pgfqpoint{4.033315in}{1.849983in}}{\pgfqpoint{4.022716in}{1.854374in}}{\pgfqpoint{4.011666in}{1.854374in}}%
\pgfpathcurveto{\pgfqpoint{4.000616in}{1.854374in}}{\pgfqpoint{3.990016in}{1.849983in}}{\pgfqpoint{3.982203in}{1.842170in}}%
\pgfpathcurveto{\pgfqpoint{3.974389in}{1.834356in}}{\pgfqpoint{3.969999in}{1.823757in}}{\pgfqpoint{3.969999in}{1.812707in}}%
\pgfpathcurveto{\pgfqpoint{3.969999in}{1.801657in}}{\pgfqpoint{3.974389in}{1.791058in}}{\pgfqpoint{3.982203in}{1.783244in}}%
\pgfpathcurveto{\pgfqpoint{3.990016in}{1.775431in}}{\pgfqpoint{4.000616in}{1.771040in}}{\pgfqpoint{4.011666in}{1.771040in}}%
\pgfpathclose%
\pgfusepath{stroke,fill}%
\end{pgfscope}%
\begin{pgfscope}%
\pgfpathrectangle{\pgfqpoint{0.800000in}{0.528000in}}{\pgfqpoint{4.960000in}{3.696000in}}%
\pgfusepath{clip}%
\pgfsetbuttcap%
\pgfsetroundjoin%
\definecolor{currentfill}{rgb}{0.000000,0.000000,0.000000}%
\pgfsetfillcolor{currentfill}%
\pgfsetlinewidth{1.003750pt}%
\definecolor{currentstroke}{rgb}{0.000000,0.000000,0.000000}%
\pgfsetstrokecolor{currentstroke}%
\pgfsetdash{}{0pt}%
\pgfpathmoveto{\pgfqpoint{4.011666in}{1.771040in}}%
\pgfpathcurveto{\pgfqpoint{4.022716in}{1.771040in}}{\pgfqpoint{4.033315in}{1.775431in}}{\pgfqpoint{4.041128in}{1.783244in}}%
\pgfpathcurveto{\pgfqpoint{4.048942in}{1.791058in}}{\pgfqpoint{4.053332in}{1.801657in}}{\pgfqpoint{4.053332in}{1.812707in}}%
\pgfpathcurveto{\pgfqpoint{4.053332in}{1.823757in}}{\pgfqpoint{4.048942in}{1.834356in}}{\pgfqpoint{4.041128in}{1.842170in}}%
\pgfpathcurveto{\pgfqpoint{4.033315in}{1.849983in}}{\pgfqpoint{4.022716in}{1.854374in}}{\pgfqpoint{4.011666in}{1.854374in}}%
\pgfpathcurveto{\pgfqpoint{4.000616in}{1.854374in}}{\pgfqpoint{3.990016in}{1.849983in}}{\pgfqpoint{3.982203in}{1.842170in}}%
\pgfpathcurveto{\pgfqpoint{3.974389in}{1.834356in}}{\pgfqpoint{3.969999in}{1.823757in}}{\pgfqpoint{3.969999in}{1.812707in}}%
\pgfpathcurveto{\pgfqpoint{3.969999in}{1.801657in}}{\pgfqpoint{3.974389in}{1.791058in}}{\pgfqpoint{3.982203in}{1.783244in}}%
\pgfpathcurveto{\pgfqpoint{3.990016in}{1.775431in}}{\pgfqpoint{4.000616in}{1.771040in}}{\pgfqpoint{4.011666in}{1.771040in}}%
\pgfpathclose%
\pgfusepath{stroke,fill}%
\end{pgfscope}%
\begin{pgfscope}%
\pgfpathrectangle{\pgfqpoint{0.800000in}{0.528000in}}{\pgfqpoint{4.960000in}{3.696000in}}%
\pgfusepath{clip}%
\pgfsetbuttcap%
\pgfsetroundjoin%
\definecolor{currentfill}{rgb}{0.000000,0.000000,0.000000}%
\pgfsetfillcolor{currentfill}%
\pgfsetlinewidth{1.003750pt}%
\definecolor{currentstroke}{rgb}{0.000000,0.000000,0.000000}%
\pgfsetstrokecolor{currentstroke}%
\pgfsetdash{}{0pt}%
\pgfpathmoveto{\pgfqpoint{4.011666in}{1.771040in}}%
\pgfpathcurveto{\pgfqpoint{4.022716in}{1.771040in}}{\pgfqpoint{4.033315in}{1.775431in}}{\pgfqpoint{4.041128in}{1.783244in}}%
\pgfpathcurveto{\pgfqpoint{4.048942in}{1.791058in}}{\pgfqpoint{4.053332in}{1.801657in}}{\pgfqpoint{4.053332in}{1.812707in}}%
\pgfpathcurveto{\pgfqpoint{4.053332in}{1.823757in}}{\pgfqpoint{4.048942in}{1.834356in}}{\pgfqpoint{4.041128in}{1.842170in}}%
\pgfpathcurveto{\pgfqpoint{4.033315in}{1.849983in}}{\pgfqpoint{4.022716in}{1.854374in}}{\pgfqpoint{4.011666in}{1.854374in}}%
\pgfpathcurveto{\pgfqpoint{4.000616in}{1.854374in}}{\pgfqpoint{3.990016in}{1.849983in}}{\pgfqpoint{3.982203in}{1.842170in}}%
\pgfpathcurveto{\pgfqpoint{3.974389in}{1.834356in}}{\pgfqpoint{3.969999in}{1.823757in}}{\pgfqpoint{3.969999in}{1.812707in}}%
\pgfpathcurveto{\pgfqpoint{3.969999in}{1.801657in}}{\pgfqpoint{3.974389in}{1.791058in}}{\pgfqpoint{3.982203in}{1.783244in}}%
\pgfpathcurveto{\pgfqpoint{3.990016in}{1.775431in}}{\pgfqpoint{4.000616in}{1.771040in}}{\pgfqpoint{4.011666in}{1.771040in}}%
\pgfpathclose%
\pgfusepath{stroke,fill}%
\end{pgfscope}%
\begin{pgfscope}%
\pgfpathrectangle{\pgfqpoint{0.800000in}{0.528000in}}{\pgfqpoint{4.960000in}{3.696000in}}%
\pgfusepath{clip}%
\pgfsetbuttcap%
\pgfsetroundjoin%
\definecolor{currentfill}{rgb}{0.000000,0.000000,0.000000}%
\pgfsetfillcolor{currentfill}%
\pgfsetlinewidth{1.003750pt}%
\definecolor{currentstroke}{rgb}{0.000000,0.000000,0.000000}%
\pgfsetstrokecolor{currentstroke}%
\pgfsetdash{}{0pt}%
\pgfpathmoveto{\pgfqpoint{4.011666in}{2.877687in}}%
\pgfpathcurveto{\pgfqpoint{4.022716in}{2.877687in}}{\pgfqpoint{4.033315in}{2.882077in}}{\pgfqpoint{4.041128in}{2.889891in}}%
\pgfpathcurveto{\pgfqpoint{4.048942in}{2.897704in}}{\pgfqpoint{4.053332in}{2.908303in}}{\pgfqpoint{4.053332in}{2.919353in}}%
\pgfpathcurveto{\pgfqpoint{4.053332in}{2.930404in}}{\pgfqpoint{4.048942in}{2.941003in}}{\pgfqpoint{4.041128in}{2.948816in}}%
\pgfpathcurveto{\pgfqpoint{4.033315in}{2.956630in}}{\pgfqpoint{4.022716in}{2.961020in}}{\pgfqpoint{4.011666in}{2.961020in}}%
\pgfpathcurveto{\pgfqpoint{4.000616in}{2.961020in}}{\pgfqpoint{3.990016in}{2.956630in}}{\pgfqpoint{3.982203in}{2.948816in}}%
\pgfpathcurveto{\pgfqpoint{3.974389in}{2.941003in}}{\pgfqpoint{3.969999in}{2.930404in}}{\pgfqpoint{3.969999in}{2.919353in}}%
\pgfpathcurveto{\pgfqpoint{3.969999in}{2.908303in}}{\pgfqpoint{3.974389in}{2.897704in}}{\pgfqpoint{3.982203in}{2.889891in}}%
\pgfpathcurveto{\pgfqpoint{3.990016in}{2.882077in}}{\pgfqpoint{4.000616in}{2.877687in}}{\pgfqpoint{4.011666in}{2.877687in}}%
\pgfpathclose%
\pgfusepath{stroke,fill}%
\end{pgfscope}%
\begin{pgfscope}%
\pgfpathrectangle{\pgfqpoint{0.800000in}{0.528000in}}{\pgfqpoint{4.960000in}{3.696000in}}%
\pgfusepath{clip}%
\pgfsetbuttcap%
\pgfsetroundjoin%
\definecolor{currentfill}{rgb}{0.000000,0.000000,0.000000}%
\pgfsetfillcolor{currentfill}%
\pgfsetlinewidth{1.003750pt}%
\definecolor{currentstroke}{rgb}{0.000000,0.000000,0.000000}%
\pgfsetstrokecolor{currentstroke}%
\pgfsetdash{}{0pt}%
\pgfpathmoveto{\pgfqpoint{4.011666in}{1.771040in}}%
\pgfpathcurveto{\pgfqpoint{4.022716in}{1.771040in}}{\pgfqpoint{4.033315in}{1.775431in}}{\pgfqpoint{4.041128in}{1.783244in}}%
\pgfpathcurveto{\pgfqpoint{4.048942in}{1.791058in}}{\pgfqpoint{4.053332in}{1.801657in}}{\pgfqpoint{4.053332in}{1.812707in}}%
\pgfpathcurveto{\pgfqpoint{4.053332in}{1.823757in}}{\pgfqpoint{4.048942in}{1.834356in}}{\pgfqpoint{4.041128in}{1.842170in}}%
\pgfpathcurveto{\pgfqpoint{4.033315in}{1.849983in}}{\pgfqpoint{4.022716in}{1.854374in}}{\pgfqpoint{4.011666in}{1.854374in}}%
\pgfpathcurveto{\pgfqpoint{4.000616in}{1.854374in}}{\pgfqpoint{3.990016in}{1.849983in}}{\pgfqpoint{3.982203in}{1.842170in}}%
\pgfpathcurveto{\pgfqpoint{3.974389in}{1.834356in}}{\pgfqpoint{3.969999in}{1.823757in}}{\pgfqpoint{3.969999in}{1.812707in}}%
\pgfpathcurveto{\pgfqpoint{3.969999in}{1.801657in}}{\pgfqpoint{3.974389in}{1.791058in}}{\pgfqpoint{3.982203in}{1.783244in}}%
\pgfpathcurveto{\pgfqpoint{3.990016in}{1.775431in}}{\pgfqpoint{4.000616in}{1.771040in}}{\pgfqpoint{4.011666in}{1.771040in}}%
\pgfpathclose%
\pgfusepath{stroke,fill}%
\end{pgfscope}%
\begin{pgfscope}%
\pgfpathrectangle{\pgfqpoint{0.800000in}{0.528000in}}{\pgfqpoint{4.960000in}{3.696000in}}%
\pgfusepath{clip}%
\pgfsetbuttcap%
\pgfsetroundjoin%
\definecolor{currentfill}{rgb}{0.000000,0.000000,0.000000}%
\pgfsetfillcolor{currentfill}%
\pgfsetlinewidth{1.003750pt}%
\definecolor{currentstroke}{rgb}{0.000000,0.000000,0.000000}%
\pgfsetstrokecolor{currentstroke}%
\pgfsetdash{}{0pt}%
\pgfpathmoveto{\pgfqpoint{4.011666in}{2.877687in}}%
\pgfpathcurveto{\pgfqpoint{4.022716in}{2.877687in}}{\pgfqpoint{4.033315in}{2.882077in}}{\pgfqpoint{4.041128in}{2.889891in}}%
\pgfpathcurveto{\pgfqpoint{4.048942in}{2.897704in}}{\pgfqpoint{4.053332in}{2.908303in}}{\pgfqpoint{4.053332in}{2.919353in}}%
\pgfpathcurveto{\pgfqpoint{4.053332in}{2.930404in}}{\pgfqpoint{4.048942in}{2.941003in}}{\pgfqpoint{4.041128in}{2.948816in}}%
\pgfpathcurveto{\pgfqpoint{4.033315in}{2.956630in}}{\pgfqpoint{4.022716in}{2.961020in}}{\pgfqpoint{4.011666in}{2.961020in}}%
\pgfpathcurveto{\pgfqpoint{4.000616in}{2.961020in}}{\pgfqpoint{3.990016in}{2.956630in}}{\pgfqpoint{3.982203in}{2.948816in}}%
\pgfpathcurveto{\pgfqpoint{3.974389in}{2.941003in}}{\pgfqpoint{3.969999in}{2.930404in}}{\pgfqpoint{3.969999in}{2.919353in}}%
\pgfpathcurveto{\pgfqpoint{3.969999in}{2.908303in}}{\pgfqpoint{3.974389in}{2.897704in}}{\pgfqpoint{3.982203in}{2.889891in}}%
\pgfpathcurveto{\pgfqpoint{3.990016in}{2.882077in}}{\pgfqpoint{4.000616in}{2.877687in}}{\pgfqpoint{4.011666in}{2.877687in}}%
\pgfpathclose%
\pgfusepath{stroke,fill}%
\end{pgfscope}%
\begin{pgfscope}%
\pgfpathrectangle{\pgfqpoint{0.800000in}{0.528000in}}{\pgfqpoint{4.960000in}{3.696000in}}%
\pgfusepath{clip}%
\pgfsetbuttcap%
\pgfsetroundjoin%
\definecolor{currentfill}{rgb}{0.000000,0.000000,0.000000}%
\pgfsetfillcolor{currentfill}%
\pgfsetlinewidth{1.003750pt}%
\definecolor{currentstroke}{rgb}{0.000000,0.000000,0.000000}%
\pgfsetstrokecolor{currentstroke}%
\pgfsetdash{}{0pt}%
\pgfpathmoveto{\pgfqpoint{4.011666in}{1.771040in}}%
\pgfpathcurveto{\pgfqpoint{4.022716in}{1.771040in}}{\pgfqpoint{4.033315in}{1.775431in}}{\pgfqpoint{4.041128in}{1.783244in}}%
\pgfpathcurveto{\pgfqpoint{4.048942in}{1.791058in}}{\pgfqpoint{4.053332in}{1.801657in}}{\pgfqpoint{4.053332in}{1.812707in}}%
\pgfpathcurveto{\pgfqpoint{4.053332in}{1.823757in}}{\pgfqpoint{4.048942in}{1.834356in}}{\pgfqpoint{4.041128in}{1.842170in}}%
\pgfpathcurveto{\pgfqpoint{4.033315in}{1.849983in}}{\pgfqpoint{4.022716in}{1.854374in}}{\pgfqpoint{4.011666in}{1.854374in}}%
\pgfpathcurveto{\pgfqpoint{4.000616in}{1.854374in}}{\pgfqpoint{3.990016in}{1.849983in}}{\pgfqpoint{3.982203in}{1.842170in}}%
\pgfpathcurveto{\pgfqpoint{3.974389in}{1.834356in}}{\pgfqpoint{3.969999in}{1.823757in}}{\pgfqpoint{3.969999in}{1.812707in}}%
\pgfpathcurveto{\pgfqpoint{3.969999in}{1.801657in}}{\pgfqpoint{3.974389in}{1.791058in}}{\pgfqpoint{3.982203in}{1.783244in}}%
\pgfpathcurveto{\pgfqpoint{3.990016in}{1.775431in}}{\pgfqpoint{4.000616in}{1.771040in}}{\pgfqpoint{4.011666in}{1.771040in}}%
\pgfpathclose%
\pgfusepath{stroke,fill}%
\end{pgfscope}%
\begin{pgfscope}%
\pgfpathrectangle{\pgfqpoint{0.800000in}{0.528000in}}{\pgfqpoint{4.960000in}{3.696000in}}%
\pgfusepath{clip}%
\pgfsetbuttcap%
\pgfsetroundjoin%
\definecolor{currentfill}{rgb}{0.000000,0.000000,0.000000}%
\pgfsetfillcolor{currentfill}%
\pgfsetlinewidth{1.003750pt}%
\definecolor{currentstroke}{rgb}{0.000000,0.000000,0.000000}%
\pgfsetstrokecolor{currentstroke}%
\pgfsetdash{}{0pt}%
\pgfpathmoveto{\pgfqpoint{4.011666in}{1.771040in}}%
\pgfpathcurveto{\pgfqpoint{4.022716in}{1.771040in}}{\pgfqpoint{4.033315in}{1.775431in}}{\pgfqpoint{4.041128in}{1.783244in}}%
\pgfpathcurveto{\pgfqpoint{4.048942in}{1.791058in}}{\pgfqpoint{4.053332in}{1.801657in}}{\pgfqpoint{4.053332in}{1.812707in}}%
\pgfpathcurveto{\pgfqpoint{4.053332in}{1.823757in}}{\pgfqpoint{4.048942in}{1.834356in}}{\pgfqpoint{4.041128in}{1.842170in}}%
\pgfpathcurveto{\pgfqpoint{4.033315in}{1.849983in}}{\pgfqpoint{4.022716in}{1.854374in}}{\pgfqpoint{4.011666in}{1.854374in}}%
\pgfpathcurveto{\pgfqpoint{4.000616in}{1.854374in}}{\pgfqpoint{3.990016in}{1.849983in}}{\pgfqpoint{3.982203in}{1.842170in}}%
\pgfpathcurveto{\pgfqpoint{3.974389in}{1.834356in}}{\pgfqpoint{3.969999in}{1.823757in}}{\pgfqpoint{3.969999in}{1.812707in}}%
\pgfpathcurveto{\pgfqpoint{3.969999in}{1.801657in}}{\pgfqpoint{3.974389in}{1.791058in}}{\pgfqpoint{3.982203in}{1.783244in}}%
\pgfpathcurveto{\pgfqpoint{3.990016in}{1.775431in}}{\pgfqpoint{4.000616in}{1.771040in}}{\pgfqpoint{4.011666in}{1.771040in}}%
\pgfpathclose%
\pgfusepath{stroke,fill}%
\end{pgfscope}%
\begin{pgfscope}%
\pgfpathrectangle{\pgfqpoint{0.800000in}{0.528000in}}{\pgfqpoint{4.960000in}{3.696000in}}%
\pgfusepath{clip}%
\pgfsetbuttcap%
\pgfsetroundjoin%
\definecolor{currentfill}{rgb}{0.000000,0.000000,0.000000}%
\pgfsetfillcolor{currentfill}%
\pgfsetlinewidth{1.003750pt}%
\definecolor{currentstroke}{rgb}{0.000000,0.000000,0.000000}%
\pgfsetstrokecolor{currentstroke}%
\pgfsetdash{}{0pt}%
\pgfpathmoveto{\pgfqpoint{4.011666in}{1.771040in}}%
\pgfpathcurveto{\pgfqpoint{4.022716in}{1.771040in}}{\pgfqpoint{4.033315in}{1.775431in}}{\pgfqpoint{4.041128in}{1.783244in}}%
\pgfpathcurveto{\pgfqpoint{4.048942in}{1.791058in}}{\pgfqpoint{4.053332in}{1.801657in}}{\pgfqpoint{4.053332in}{1.812707in}}%
\pgfpathcurveto{\pgfqpoint{4.053332in}{1.823757in}}{\pgfqpoint{4.048942in}{1.834356in}}{\pgfqpoint{4.041128in}{1.842170in}}%
\pgfpathcurveto{\pgfqpoint{4.033315in}{1.849983in}}{\pgfqpoint{4.022716in}{1.854374in}}{\pgfqpoint{4.011666in}{1.854374in}}%
\pgfpathcurveto{\pgfqpoint{4.000616in}{1.854374in}}{\pgfqpoint{3.990016in}{1.849983in}}{\pgfqpoint{3.982203in}{1.842170in}}%
\pgfpathcurveto{\pgfqpoint{3.974389in}{1.834356in}}{\pgfqpoint{3.969999in}{1.823757in}}{\pgfqpoint{3.969999in}{1.812707in}}%
\pgfpathcurveto{\pgfqpoint{3.969999in}{1.801657in}}{\pgfqpoint{3.974389in}{1.791058in}}{\pgfqpoint{3.982203in}{1.783244in}}%
\pgfpathcurveto{\pgfqpoint{3.990016in}{1.775431in}}{\pgfqpoint{4.000616in}{1.771040in}}{\pgfqpoint{4.011666in}{1.771040in}}%
\pgfpathclose%
\pgfusepath{stroke,fill}%
\end{pgfscope}%
\begin{pgfscope}%
\pgfpathrectangle{\pgfqpoint{0.800000in}{0.528000in}}{\pgfqpoint{4.960000in}{3.696000in}}%
\pgfusepath{clip}%
\pgfsetbuttcap%
\pgfsetroundjoin%
\definecolor{currentfill}{rgb}{0.000000,0.000000,0.000000}%
\pgfsetfillcolor{currentfill}%
\pgfsetlinewidth{1.003750pt}%
\definecolor{currentstroke}{rgb}{0.000000,0.000000,0.000000}%
\pgfsetstrokecolor{currentstroke}%
\pgfsetdash{}{0pt}%
\pgfpathmoveto{\pgfqpoint{4.011666in}{2.877687in}}%
\pgfpathcurveto{\pgfqpoint{4.022716in}{2.877687in}}{\pgfqpoint{4.033315in}{2.882077in}}{\pgfqpoint{4.041128in}{2.889891in}}%
\pgfpathcurveto{\pgfqpoint{4.048942in}{2.897704in}}{\pgfqpoint{4.053332in}{2.908303in}}{\pgfqpoint{4.053332in}{2.919353in}}%
\pgfpathcurveto{\pgfqpoint{4.053332in}{2.930404in}}{\pgfqpoint{4.048942in}{2.941003in}}{\pgfqpoint{4.041128in}{2.948816in}}%
\pgfpathcurveto{\pgfqpoint{4.033315in}{2.956630in}}{\pgfqpoint{4.022716in}{2.961020in}}{\pgfqpoint{4.011666in}{2.961020in}}%
\pgfpathcurveto{\pgfqpoint{4.000616in}{2.961020in}}{\pgfqpoint{3.990016in}{2.956630in}}{\pgfqpoint{3.982203in}{2.948816in}}%
\pgfpathcurveto{\pgfqpoint{3.974389in}{2.941003in}}{\pgfqpoint{3.969999in}{2.930404in}}{\pgfqpoint{3.969999in}{2.919353in}}%
\pgfpathcurveto{\pgfqpoint{3.969999in}{2.908303in}}{\pgfqpoint{3.974389in}{2.897704in}}{\pgfqpoint{3.982203in}{2.889891in}}%
\pgfpathcurveto{\pgfqpoint{3.990016in}{2.882077in}}{\pgfqpoint{4.000616in}{2.877687in}}{\pgfqpoint{4.011666in}{2.877687in}}%
\pgfpathclose%
\pgfusepath{stroke,fill}%
\end{pgfscope}%
\begin{pgfscope}%
\pgfpathrectangle{\pgfqpoint{0.800000in}{0.528000in}}{\pgfqpoint{4.960000in}{3.696000in}}%
\pgfusepath{clip}%
\pgfsetbuttcap%
\pgfsetroundjoin%
\definecolor{currentfill}{rgb}{0.000000,0.000000,0.000000}%
\pgfsetfillcolor{currentfill}%
\pgfsetlinewidth{1.003750pt}%
\definecolor{currentstroke}{rgb}{0.000000,0.000000,0.000000}%
\pgfsetstrokecolor{currentstroke}%
\pgfsetdash{}{0pt}%
\pgfpathmoveto{\pgfqpoint{4.011666in}{1.771040in}}%
\pgfpathcurveto{\pgfqpoint{4.022716in}{1.771040in}}{\pgfqpoint{4.033315in}{1.775431in}}{\pgfqpoint{4.041128in}{1.783244in}}%
\pgfpathcurveto{\pgfqpoint{4.048942in}{1.791058in}}{\pgfqpoint{4.053332in}{1.801657in}}{\pgfqpoint{4.053332in}{1.812707in}}%
\pgfpathcurveto{\pgfqpoint{4.053332in}{1.823757in}}{\pgfqpoint{4.048942in}{1.834356in}}{\pgfqpoint{4.041128in}{1.842170in}}%
\pgfpathcurveto{\pgfqpoint{4.033315in}{1.849983in}}{\pgfqpoint{4.022716in}{1.854374in}}{\pgfqpoint{4.011666in}{1.854374in}}%
\pgfpathcurveto{\pgfqpoint{4.000616in}{1.854374in}}{\pgfqpoint{3.990016in}{1.849983in}}{\pgfqpoint{3.982203in}{1.842170in}}%
\pgfpathcurveto{\pgfqpoint{3.974389in}{1.834356in}}{\pgfqpoint{3.969999in}{1.823757in}}{\pgfqpoint{3.969999in}{1.812707in}}%
\pgfpathcurveto{\pgfqpoint{3.969999in}{1.801657in}}{\pgfqpoint{3.974389in}{1.791058in}}{\pgfqpoint{3.982203in}{1.783244in}}%
\pgfpathcurveto{\pgfqpoint{3.990016in}{1.775431in}}{\pgfqpoint{4.000616in}{1.771040in}}{\pgfqpoint{4.011666in}{1.771040in}}%
\pgfpathclose%
\pgfusepath{stroke,fill}%
\end{pgfscope}%
\begin{pgfscope}%
\pgfpathrectangle{\pgfqpoint{0.800000in}{0.528000in}}{\pgfqpoint{4.960000in}{3.696000in}}%
\pgfusepath{clip}%
\pgfsetbuttcap%
\pgfsetroundjoin%
\definecolor{currentfill}{rgb}{0.000000,0.000000,0.000000}%
\pgfsetfillcolor{currentfill}%
\pgfsetlinewidth{1.003750pt}%
\definecolor{currentstroke}{rgb}{0.000000,0.000000,0.000000}%
\pgfsetstrokecolor{currentstroke}%
\pgfsetdash{}{0pt}%
\pgfpathmoveto{\pgfqpoint{4.011666in}{1.771040in}}%
\pgfpathcurveto{\pgfqpoint{4.022716in}{1.771040in}}{\pgfqpoint{4.033315in}{1.775431in}}{\pgfqpoint{4.041128in}{1.783244in}}%
\pgfpathcurveto{\pgfqpoint{4.048942in}{1.791058in}}{\pgfqpoint{4.053332in}{1.801657in}}{\pgfqpoint{4.053332in}{1.812707in}}%
\pgfpathcurveto{\pgfqpoint{4.053332in}{1.823757in}}{\pgfqpoint{4.048942in}{1.834356in}}{\pgfqpoint{4.041128in}{1.842170in}}%
\pgfpathcurveto{\pgfqpoint{4.033315in}{1.849983in}}{\pgfqpoint{4.022716in}{1.854374in}}{\pgfqpoint{4.011666in}{1.854374in}}%
\pgfpathcurveto{\pgfqpoint{4.000616in}{1.854374in}}{\pgfqpoint{3.990016in}{1.849983in}}{\pgfqpoint{3.982203in}{1.842170in}}%
\pgfpathcurveto{\pgfqpoint{3.974389in}{1.834356in}}{\pgfqpoint{3.969999in}{1.823757in}}{\pgfqpoint{3.969999in}{1.812707in}}%
\pgfpathcurveto{\pgfqpoint{3.969999in}{1.801657in}}{\pgfqpoint{3.974389in}{1.791058in}}{\pgfqpoint{3.982203in}{1.783244in}}%
\pgfpathcurveto{\pgfqpoint{3.990016in}{1.775431in}}{\pgfqpoint{4.000616in}{1.771040in}}{\pgfqpoint{4.011666in}{1.771040in}}%
\pgfpathclose%
\pgfusepath{stroke,fill}%
\end{pgfscope}%
\begin{pgfscope}%
\pgfpathrectangle{\pgfqpoint{0.800000in}{0.528000in}}{\pgfqpoint{4.960000in}{3.696000in}}%
\pgfusepath{clip}%
\pgfsetbuttcap%
\pgfsetroundjoin%
\definecolor{currentfill}{rgb}{0.000000,0.000000,0.000000}%
\pgfsetfillcolor{currentfill}%
\pgfsetlinewidth{1.003750pt}%
\definecolor{currentstroke}{rgb}{0.000000,0.000000,0.000000}%
\pgfsetstrokecolor{currentstroke}%
\pgfsetdash{}{0pt}%
\pgfpathmoveto{\pgfqpoint{4.011666in}{1.771040in}}%
\pgfpathcurveto{\pgfqpoint{4.022716in}{1.771040in}}{\pgfqpoint{4.033315in}{1.775431in}}{\pgfqpoint{4.041128in}{1.783244in}}%
\pgfpathcurveto{\pgfqpoint{4.048942in}{1.791058in}}{\pgfqpoint{4.053332in}{1.801657in}}{\pgfqpoint{4.053332in}{1.812707in}}%
\pgfpathcurveto{\pgfqpoint{4.053332in}{1.823757in}}{\pgfqpoint{4.048942in}{1.834356in}}{\pgfqpoint{4.041128in}{1.842170in}}%
\pgfpathcurveto{\pgfqpoint{4.033315in}{1.849983in}}{\pgfqpoint{4.022716in}{1.854374in}}{\pgfqpoint{4.011666in}{1.854374in}}%
\pgfpathcurveto{\pgfqpoint{4.000616in}{1.854374in}}{\pgfqpoint{3.990016in}{1.849983in}}{\pgfqpoint{3.982203in}{1.842170in}}%
\pgfpathcurveto{\pgfqpoint{3.974389in}{1.834356in}}{\pgfqpoint{3.969999in}{1.823757in}}{\pgfqpoint{3.969999in}{1.812707in}}%
\pgfpathcurveto{\pgfqpoint{3.969999in}{1.801657in}}{\pgfqpoint{3.974389in}{1.791058in}}{\pgfqpoint{3.982203in}{1.783244in}}%
\pgfpathcurveto{\pgfqpoint{3.990016in}{1.775431in}}{\pgfqpoint{4.000616in}{1.771040in}}{\pgfqpoint{4.011666in}{1.771040in}}%
\pgfpathclose%
\pgfusepath{stroke,fill}%
\end{pgfscope}%
\begin{pgfscope}%
\pgfpathrectangle{\pgfqpoint{0.800000in}{0.528000in}}{\pgfqpoint{4.960000in}{3.696000in}}%
\pgfusepath{clip}%
\pgfsetbuttcap%
\pgfsetroundjoin%
\definecolor{currentfill}{rgb}{0.000000,0.000000,0.000000}%
\pgfsetfillcolor{currentfill}%
\pgfsetlinewidth{1.003750pt}%
\definecolor{currentstroke}{rgb}{0.000000,0.000000,0.000000}%
\pgfsetstrokecolor{currentstroke}%
\pgfsetdash{}{0pt}%
\pgfpathmoveto{\pgfqpoint{4.011666in}{1.771040in}}%
\pgfpathcurveto{\pgfqpoint{4.022716in}{1.771040in}}{\pgfqpoint{4.033315in}{1.775431in}}{\pgfqpoint{4.041128in}{1.783244in}}%
\pgfpathcurveto{\pgfqpoint{4.048942in}{1.791058in}}{\pgfqpoint{4.053332in}{1.801657in}}{\pgfqpoint{4.053332in}{1.812707in}}%
\pgfpathcurveto{\pgfqpoint{4.053332in}{1.823757in}}{\pgfqpoint{4.048942in}{1.834356in}}{\pgfqpoint{4.041128in}{1.842170in}}%
\pgfpathcurveto{\pgfqpoint{4.033315in}{1.849983in}}{\pgfqpoint{4.022716in}{1.854374in}}{\pgfqpoint{4.011666in}{1.854374in}}%
\pgfpathcurveto{\pgfqpoint{4.000616in}{1.854374in}}{\pgfqpoint{3.990016in}{1.849983in}}{\pgfqpoint{3.982203in}{1.842170in}}%
\pgfpathcurveto{\pgfqpoint{3.974389in}{1.834356in}}{\pgfqpoint{3.969999in}{1.823757in}}{\pgfqpoint{3.969999in}{1.812707in}}%
\pgfpathcurveto{\pgfqpoint{3.969999in}{1.801657in}}{\pgfqpoint{3.974389in}{1.791058in}}{\pgfqpoint{3.982203in}{1.783244in}}%
\pgfpathcurveto{\pgfqpoint{3.990016in}{1.775431in}}{\pgfqpoint{4.000616in}{1.771040in}}{\pgfqpoint{4.011666in}{1.771040in}}%
\pgfpathclose%
\pgfusepath{stroke,fill}%
\end{pgfscope}%
\begin{pgfscope}%
\pgfpathrectangle{\pgfqpoint{0.800000in}{0.528000in}}{\pgfqpoint{4.960000in}{3.696000in}}%
\pgfusepath{clip}%
\pgfsetbuttcap%
\pgfsetroundjoin%
\definecolor{currentfill}{rgb}{0.000000,0.000000,0.000000}%
\pgfsetfillcolor{currentfill}%
\pgfsetlinewidth{1.003750pt}%
\definecolor{currentstroke}{rgb}{0.000000,0.000000,0.000000}%
\pgfsetstrokecolor{currentstroke}%
\pgfsetdash{}{0pt}%
\pgfpathmoveto{\pgfqpoint{4.011666in}{2.877687in}}%
\pgfpathcurveto{\pgfqpoint{4.022716in}{2.877687in}}{\pgfqpoint{4.033315in}{2.882077in}}{\pgfqpoint{4.041128in}{2.889891in}}%
\pgfpathcurveto{\pgfqpoint{4.048942in}{2.897704in}}{\pgfqpoint{4.053332in}{2.908303in}}{\pgfqpoint{4.053332in}{2.919353in}}%
\pgfpathcurveto{\pgfqpoint{4.053332in}{2.930404in}}{\pgfqpoint{4.048942in}{2.941003in}}{\pgfqpoint{4.041128in}{2.948816in}}%
\pgfpathcurveto{\pgfqpoint{4.033315in}{2.956630in}}{\pgfqpoint{4.022716in}{2.961020in}}{\pgfqpoint{4.011666in}{2.961020in}}%
\pgfpathcurveto{\pgfqpoint{4.000616in}{2.961020in}}{\pgfqpoint{3.990016in}{2.956630in}}{\pgfqpoint{3.982203in}{2.948816in}}%
\pgfpathcurveto{\pgfqpoint{3.974389in}{2.941003in}}{\pgfqpoint{3.969999in}{2.930404in}}{\pgfqpoint{3.969999in}{2.919353in}}%
\pgfpathcurveto{\pgfqpoint{3.969999in}{2.908303in}}{\pgfqpoint{3.974389in}{2.897704in}}{\pgfqpoint{3.982203in}{2.889891in}}%
\pgfpathcurveto{\pgfqpoint{3.990016in}{2.882077in}}{\pgfqpoint{4.000616in}{2.877687in}}{\pgfqpoint{4.011666in}{2.877687in}}%
\pgfpathclose%
\pgfusepath{stroke,fill}%
\end{pgfscope}%
\begin{pgfscope}%
\pgfpathrectangle{\pgfqpoint{0.800000in}{0.528000in}}{\pgfqpoint{4.960000in}{3.696000in}}%
\pgfusepath{clip}%
\pgfsetbuttcap%
\pgfsetroundjoin%
\definecolor{currentfill}{rgb}{0.000000,0.000000,0.000000}%
\pgfsetfillcolor{currentfill}%
\pgfsetlinewidth{1.003750pt}%
\definecolor{currentstroke}{rgb}{0.000000,0.000000,0.000000}%
\pgfsetstrokecolor{currentstroke}%
\pgfsetdash{}{0pt}%
\pgfpathmoveto{\pgfqpoint{4.011666in}{2.877687in}}%
\pgfpathcurveto{\pgfqpoint{4.022716in}{2.877687in}}{\pgfqpoint{4.033315in}{2.882077in}}{\pgfqpoint{4.041128in}{2.889891in}}%
\pgfpathcurveto{\pgfqpoint{4.048942in}{2.897704in}}{\pgfqpoint{4.053332in}{2.908303in}}{\pgfqpoint{4.053332in}{2.919353in}}%
\pgfpathcurveto{\pgfqpoint{4.053332in}{2.930404in}}{\pgfqpoint{4.048942in}{2.941003in}}{\pgfqpoint{4.041128in}{2.948816in}}%
\pgfpathcurveto{\pgfqpoint{4.033315in}{2.956630in}}{\pgfqpoint{4.022716in}{2.961020in}}{\pgfqpoint{4.011666in}{2.961020in}}%
\pgfpathcurveto{\pgfqpoint{4.000616in}{2.961020in}}{\pgfqpoint{3.990016in}{2.956630in}}{\pgfqpoint{3.982203in}{2.948816in}}%
\pgfpathcurveto{\pgfqpoint{3.974389in}{2.941003in}}{\pgfqpoint{3.969999in}{2.930404in}}{\pgfqpoint{3.969999in}{2.919353in}}%
\pgfpathcurveto{\pgfqpoint{3.969999in}{2.908303in}}{\pgfqpoint{3.974389in}{2.897704in}}{\pgfqpoint{3.982203in}{2.889891in}}%
\pgfpathcurveto{\pgfqpoint{3.990016in}{2.882077in}}{\pgfqpoint{4.000616in}{2.877687in}}{\pgfqpoint{4.011666in}{2.877687in}}%
\pgfpathclose%
\pgfusepath{stroke,fill}%
\end{pgfscope}%
\begin{pgfscope}%
\pgfpathrectangle{\pgfqpoint{0.800000in}{0.528000in}}{\pgfqpoint{4.960000in}{3.696000in}}%
\pgfusepath{clip}%
\pgfsetbuttcap%
\pgfsetroundjoin%
\definecolor{currentfill}{rgb}{0.000000,0.000000,0.000000}%
\pgfsetfillcolor{currentfill}%
\pgfsetlinewidth{1.003750pt}%
\definecolor{currentstroke}{rgb}{0.000000,0.000000,0.000000}%
\pgfsetstrokecolor{currentstroke}%
\pgfsetdash{}{0pt}%
\pgfpathmoveto{\pgfqpoint{4.011666in}{2.877687in}}%
\pgfpathcurveto{\pgfqpoint{4.022716in}{2.877687in}}{\pgfqpoint{4.033315in}{2.882077in}}{\pgfqpoint{4.041128in}{2.889891in}}%
\pgfpathcurveto{\pgfqpoint{4.048942in}{2.897704in}}{\pgfqpoint{4.053332in}{2.908303in}}{\pgfqpoint{4.053332in}{2.919353in}}%
\pgfpathcurveto{\pgfqpoint{4.053332in}{2.930404in}}{\pgfqpoint{4.048942in}{2.941003in}}{\pgfqpoint{4.041128in}{2.948816in}}%
\pgfpathcurveto{\pgfqpoint{4.033315in}{2.956630in}}{\pgfqpoint{4.022716in}{2.961020in}}{\pgfqpoint{4.011666in}{2.961020in}}%
\pgfpathcurveto{\pgfqpoint{4.000616in}{2.961020in}}{\pgfqpoint{3.990016in}{2.956630in}}{\pgfqpoint{3.982203in}{2.948816in}}%
\pgfpathcurveto{\pgfqpoint{3.974389in}{2.941003in}}{\pgfqpoint{3.969999in}{2.930404in}}{\pgfqpoint{3.969999in}{2.919353in}}%
\pgfpathcurveto{\pgfqpoint{3.969999in}{2.908303in}}{\pgfqpoint{3.974389in}{2.897704in}}{\pgfqpoint{3.982203in}{2.889891in}}%
\pgfpathcurveto{\pgfqpoint{3.990016in}{2.882077in}}{\pgfqpoint{4.000616in}{2.877687in}}{\pgfqpoint{4.011666in}{2.877687in}}%
\pgfpathclose%
\pgfusepath{stroke,fill}%
\end{pgfscope}%
\begin{pgfscope}%
\pgfpathrectangle{\pgfqpoint{0.800000in}{0.528000in}}{\pgfqpoint{4.960000in}{3.696000in}}%
\pgfusepath{clip}%
\pgfsetbuttcap%
\pgfsetroundjoin%
\definecolor{currentfill}{rgb}{0.000000,0.000000,0.000000}%
\pgfsetfillcolor{currentfill}%
\pgfsetlinewidth{1.003750pt}%
\definecolor{currentstroke}{rgb}{0.000000,0.000000,0.000000}%
\pgfsetstrokecolor{currentstroke}%
\pgfsetdash{}{0pt}%
\pgfpathmoveto{\pgfqpoint{4.011666in}{1.771040in}}%
\pgfpathcurveto{\pgfqpoint{4.022716in}{1.771040in}}{\pgfqpoint{4.033315in}{1.775431in}}{\pgfqpoint{4.041128in}{1.783244in}}%
\pgfpathcurveto{\pgfqpoint{4.048942in}{1.791058in}}{\pgfqpoint{4.053332in}{1.801657in}}{\pgfqpoint{4.053332in}{1.812707in}}%
\pgfpathcurveto{\pgfqpoint{4.053332in}{1.823757in}}{\pgfqpoint{4.048942in}{1.834356in}}{\pgfqpoint{4.041128in}{1.842170in}}%
\pgfpathcurveto{\pgfqpoint{4.033315in}{1.849983in}}{\pgfqpoint{4.022716in}{1.854374in}}{\pgfqpoint{4.011666in}{1.854374in}}%
\pgfpathcurveto{\pgfqpoint{4.000616in}{1.854374in}}{\pgfqpoint{3.990016in}{1.849983in}}{\pgfqpoint{3.982203in}{1.842170in}}%
\pgfpathcurveto{\pgfqpoint{3.974389in}{1.834356in}}{\pgfqpoint{3.969999in}{1.823757in}}{\pgfqpoint{3.969999in}{1.812707in}}%
\pgfpathcurveto{\pgfqpoint{3.969999in}{1.801657in}}{\pgfqpoint{3.974389in}{1.791058in}}{\pgfqpoint{3.982203in}{1.783244in}}%
\pgfpathcurveto{\pgfqpoint{3.990016in}{1.775431in}}{\pgfqpoint{4.000616in}{1.771040in}}{\pgfqpoint{4.011666in}{1.771040in}}%
\pgfpathclose%
\pgfusepath{stroke,fill}%
\end{pgfscope}%
\begin{pgfscope}%
\pgfpathrectangle{\pgfqpoint{0.800000in}{0.528000in}}{\pgfqpoint{4.960000in}{3.696000in}}%
\pgfusepath{clip}%
\pgfsetbuttcap%
\pgfsetroundjoin%
\definecolor{currentfill}{rgb}{0.000000,0.000000,0.000000}%
\pgfsetfillcolor{currentfill}%
\pgfsetlinewidth{1.003750pt}%
\definecolor{currentstroke}{rgb}{0.000000,0.000000,0.000000}%
\pgfsetstrokecolor{currentstroke}%
\pgfsetdash{}{0pt}%
\pgfpathmoveto{\pgfqpoint{4.011666in}{2.877687in}}%
\pgfpathcurveto{\pgfqpoint{4.022716in}{2.877687in}}{\pgfqpoint{4.033315in}{2.882077in}}{\pgfqpoint{4.041128in}{2.889891in}}%
\pgfpathcurveto{\pgfqpoint{4.048942in}{2.897704in}}{\pgfqpoint{4.053332in}{2.908303in}}{\pgfqpoint{4.053332in}{2.919353in}}%
\pgfpathcurveto{\pgfqpoint{4.053332in}{2.930404in}}{\pgfqpoint{4.048942in}{2.941003in}}{\pgfqpoint{4.041128in}{2.948816in}}%
\pgfpathcurveto{\pgfqpoint{4.033315in}{2.956630in}}{\pgfqpoint{4.022716in}{2.961020in}}{\pgfqpoint{4.011666in}{2.961020in}}%
\pgfpathcurveto{\pgfqpoint{4.000616in}{2.961020in}}{\pgfqpoint{3.990016in}{2.956630in}}{\pgfqpoint{3.982203in}{2.948816in}}%
\pgfpathcurveto{\pgfqpoint{3.974389in}{2.941003in}}{\pgfqpoint{3.969999in}{2.930404in}}{\pgfqpoint{3.969999in}{2.919353in}}%
\pgfpathcurveto{\pgfqpoint{3.969999in}{2.908303in}}{\pgfqpoint{3.974389in}{2.897704in}}{\pgfqpoint{3.982203in}{2.889891in}}%
\pgfpathcurveto{\pgfqpoint{3.990016in}{2.882077in}}{\pgfqpoint{4.000616in}{2.877687in}}{\pgfqpoint{4.011666in}{2.877687in}}%
\pgfpathclose%
\pgfusepath{stroke,fill}%
\end{pgfscope}%
\begin{pgfscope}%
\pgfpathrectangle{\pgfqpoint{0.800000in}{0.528000in}}{\pgfqpoint{4.960000in}{3.696000in}}%
\pgfusepath{clip}%
\pgfsetbuttcap%
\pgfsetroundjoin%
\definecolor{currentfill}{rgb}{0.000000,0.000000,0.000000}%
\pgfsetfillcolor{currentfill}%
\pgfsetlinewidth{1.003750pt}%
\definecolor{currentstroke}{rgb}{0.000000,0.000000,0.000000}%
\pgfsetstrokecolor{currentstroke}%
\pgfsetdash{}{0pt}%
\pgfpathmoveto{\pgfqpoint{4.011666in}{1.771040in}}%
\pgfpathcurveto{\pgfqpoint{4.022716in}{1.771040in}}{\pgfqpoint{4.033315in}{1.775431in}}{\pgfqpoint{4.041128in}{1.783244in}}%
\pgfpathcurveto{\pgfqpoint{4.048942in}{1.791058in}}{\pgfqpoint{4.053332in}{1.801657in}}{\pgfqpoint{4.053332in}{1.812707in}}%
\pgfpathcurveto{\pgfqpoint{4.053332in}{1.823757in}}{\pgfqpoint{4.048942in}{1.834356in}}{\pgfqpoint{4.041128in}{1.842170in}}%
\pgfpathcurveto{\pgfqpoint{4.033315in}{1.849983in}}{\pgfqpoint{4.022716in}{1.854374in}}{\pgfqpoint{4.011666in}{1.854374in}}%
\pgfpathcurveto{\pgfqpoint{4.000616in}{1.854374in}}{\pgfqpoint{3.990016in}{1.849983in}}{\pgfqpoint{3.982203in}{1.842170in}}%
\pgfpathcurveto{\pgfqpoint{3.974389in}{1.834356in}}{\pgfqpoint{3.969999in}{1.823757in}}{\pgfqpoint{3.969999in}{1.812707in}}%
\pgfpathcurveto{\pgfqpoint{3.969999in}{1.801657in}}{\pgfqpoint{3.974389in}{1.791058in}}{\pgfqpoint{3.982203in}{1.783244in}}%
\pgfpathcurveto{\pgfqpoint{3.990016in}{1.775431in}}{\pgfqpoint{4.000616in}{1.771040in}}{\pgfqpoint{4.011666in}{1.771040in}}%
\pgfpathclose%
\pgfusepath{stroke,fill}%
\end{pgfscope}%
\begin{pgfscope}%
\pgfpathrectangle{\pgfqpoint{0.800000in}{0.528000in}}{\pgfqpoint{4.960000in}{3.696000in}}%
\pgfusepath{clip}%
\pgfsetbuttcap%
\pgfsetroundjoin%
\definecolor{currentfill}{rgb}{0.000000,0.000000,0.000000}%
\pgfsetfillcolor{currentfill}%
\pgfsetlinewidth{1.003750pt}%
\definecolor{currentstroke}{rgb}{0.000000,0.000000,0.000000}%
\pgfsetstrokecolor{currentstroke}%
\pgfsetdash{}{0pt}%
\pgfpathmoveto{\pgfqpoint{4.011666in}{1.771040in}}%
\pgfpathcurveto{\pgfqpoint{4.022716in}{1.771040in}}{\pgfqpoint{4.033315in}{1.775431in}}{\pgfqpoint{4.041128in}{1.783244in}}%
\pgfpathcurveto{\pgfqpoint{4.048942in}{1.791058in}}{\pgfqpoint{4.053332in}{1.801657in}}{\pgfqpoint{4.053332in}{1.812707in}}%
\pgfpathcurveto{\pgfqpoint{4.053332in}{1.823757in}}{\pgfqpoint{4.048942in}{1.834356in}}{\pgfqpoint{4.041128in}{1.842170in}}%
\pgfpathcurveto{\pgfqpoint{4.033315in}{1.849983in}}{\pgfqpoint{4.022716in}{1.854374in}}{\pgfqpoint{4.011666in}{1.854374in}}%
\pgfpathcurveto{\pgfqpoint{4.000616in}{1.854374in}}{\pgfqpoint{3.990016in}{1.849983in}}{\pgfqpoint{3.982203in}{1.842170in}}%
\pgfpathcurveto{\pgfqpoint{3.974389in}{1.834356in}}{\pgfqpoint{3.969999in}{1.823757in}}{\pgfqpoint{3.969999in}{1.812707in}}%
\pgfpathcurveto{\pgfqpoint{3.969999in}{1.801657in}}{\pgfqpoint{3.974389in}{1.791058in}}{\pgfqpoint{3.982203in}{1.783244in}}%
\pgfpathcurveto{\pgfqpoint{3.990016in}{1.775431in}}{\pgfqpoint{4.000616in}{1.771040in}}{\pgfqpoint{4.011666in}{1.771040in}}%
\pgfpathclose%
\pgfusepath{stroke,fill}%
\end{pgfscope}%
\begin{pgfscope}%
\pgfpathrectangle{\pgfqpoint{0.800000in}{0.528000in}}{\pgfqpoint{4.960000in}{3.696000in}}%
\pgfusepath{clip}%
\pgfsetbuttcap%
\pgfsetroundjoin%
\definecolor{currentfill}{rgb}{0.000000,0.000000,0.000000}%
\pgfsetfillcolor{currentfill}%
\pgfsetlinewidth{1.003750pt}%
\definecolor{currentstroke}{rgb}{0.000000,0.000000,0.000000}%
\pgfsetstrokecolor{currentstroke}%
\pgfsetdash{}{0pt}%
\pgfpathmoveto{\pgfqpoint{4.011666in}{1.771040in}}%
\pgfpathcurveto{\pgfqpoint{4.022716in}{1.771040in}}{\pgfqpoint{4.033315in}{1.775431in}}{\pgfqpoint{4.041128in}{1.783244in}}%
\pgfpathcurveto{\pgfqpoint{4.048942in}{1.791058in}}{\pgfqpoint{4.053332in}{1.801657in}}{\pgfqpoint{4.053332in}{1.812707in}}%
\pgfpathcurveto{\pgfqpoint{4.053332in}{1.823757in}}{\pgfqpoint{4.048942in}{1.834356in}}{\pgfqpoint{4.041128in}{1.842170in}}%
\pgfpathcurveto{\pgfqpoint{4.033315in}{1.849983in}}{\pgfqpoint{4.022716in}{1.854374in}}{\pgfqpoint{4.011666in}{1.854374in}}%
\pgfpathcurveto{\pgfqpoint{4.000616in}{1.854374in}}{\pgfqpoint{3.990016in}{1.849983in}}{\pgfqpoint{3.982203in}{1.842170in}}%
\pgfpathcurveto{\pgfqpoint{3.974389in}{1.834356in}}{\pgfqpoint{3.969999in}{1.823757in}}{\pgfqpoint{3.969999in}{1.812707in}}%
\pgfpathcurveto{\pgfqpoint{3.969999in}{1.801657in}}{\pgfqpoint{3.974389in}{1.791058in}}{\pgfqpoint{3.982203in}{1.783244in}}%
\pgfpathcurveto{\pgfqpoint{3.990016in}{1.775431in}}{\pgfqpoint{4.000616in}{1.771040in}}{\pgfqpoint{4.011666in}{1.771040in}}%
\pgfpathclose%
\pgfusepath{stroke,fill}%
\end{pgfscope}%
\begin{pgfscope}%
\pgfpathrectangle{\pgfqpoint{0.800000in}{0.528000in}}{\pgfqpoint{4.960000in}{3.696000in}}%
\pgfusepath{clip}%
\pgfsetbuttcap%
\pgfsetroundjoin%
\definecolor{currentfill}{rgb}{0.000000,0.000000,0.000000}%
\pgfsetfillcolor{currentfill}%
\pgfsetlinewidth{1.003750pt}%
\definecolor{currentstroke}{rgb}{0.000000,0.000000,0.000000}%
\pgfsetstrokecolor{currentstroke}%
\pgfsetdash{}{0pt}%
\pgfpathmoveto{\pgfqpoint{4.011666in}{1.771040in}}%
\pgfpathcurveto{\pgfqpoint{4.022716in}{1.771040in}}{\pgfqpoint{4.033315in}{1.775431in}}{\pgfqpoint{4.041128in}{1.783244in}}%
\pgfpathcurveto{\pgfqpoint{4.048942in}{1.791058in}}{\pgfqpoint{4.053332in}{1.801657in}}{\pgfqpoint{4.053332in}{1.812707in}}%
\pgfpathcurveto{\pgfqpoint{4.053332in}{1.823757in}}{\pgfqpoint{4.048942in}{1.834356in}}{\pgfqpoint{4.041128in}{1.842170in}}%
\pgfpathcurveto{\pgfqpoint{4.033315in}{1.849983in}}{\pgfqpoint{4.022716in}{1.854374in}}{\pgfqpoint{4.011666in}{1.854374in}}%
\pgfpathcurveto{\pgfqpoint{4.000616in}{1.854374in}}{\pgfqpoint{3.990016in}{1.849983in}}{\pgfqpoint{3.982203in}{1.842170in}}%
\pgfpathcurveto{\pgfqpoint{3.974389in}{1.834356in}}{\pgfqpoint{3.969999in}{1.823757in}}{\pgfqpoint{3.969999in}{1.812707in}}%
\pgfpathcurveto{\pgfqpoint{3.969999in}{1.801657in}}{\pgfqpoint{3.974389in}{1.791058in}}{\pgfqpoint{3.982203in}{1.783244in}}%
\pgfpathcurveto{\pgfqpoint{3.990016in}{1.775431in}}{\pgfqpoint{4.000616in}{1.771040in}}{\pgfqpoint{4.011666in}{1.771040in}}%
\pgfpathclose%
\pgfusepath{stroke,fill}%
\end{pgfscope}%
\begin{pgfscope}%
\pgfpathrectangle{\pgfqpoint{0.800000in}{0.528000in}}{\pgfqpoint{4.960000in}{3.696000in}}%
\pgfusepath{clip}%
\pgfsetbuttcap%
\pgfsetroundjoin%
\definecolor{currentfill}{rgb}{0.000000,0.000000,0.000000}%
\pgfsetfillcolor{currentfill}%
\pgfsetlinewidth{1.003750pt}%
\definecolor{currentstroke}{rgb}{0.000000,0.000000,0.000000}%
\pgfsetstrokecolor{currentstroke}%
\pgfsetdash{}{0pt}%
\pgfpathmoveto{\pgfqpoint{4.011666in}{1.771040in}}%
\pgfpathcurveto{\pgfqpoint{4.022716in}{1.771040in}}{\pgfqpoint{4.033315in}{1.775431in}}{\pgfqpoint{4.041128in}{1.783244in}}%
\pgfpathcurveto{\pgfqpoint{4.048942in}{1.791058in}}{\pgfqpoint{4.053332in}{1.801657in}}{\pgfqpoint{4.053332in}{1.812707in}}%
\pgfpathcurveto{\pgfqpoint{4.053332in}{1.823757in}}{\pgfqpoint{4.048942in}{1.834356in}}{\pgfqpoint{4.041128in}{1.842170in}}%
\pgfpathcurveto{\pgfqpoint{4.033315in}{1.849983in}}{\pgfqpoint{4.022716in}{1.854374in}}{\pgfqpoint{4.011666in}{1.854374in}}%
\pgfpathcurveto{\pgfqpoint{4.000616in}{1.854374in}}{\pgfqpoint{3.990016in}{1.849983in}}{\pgfqpoint{3.982203in}{1.842170in}}%
\pgfpathcurveto{\pgfqpoint{3.974389in}{1.834356in}}{\pgfqpoint{3.969999in}{1.823757in}}{\pgfqpoint{3.969999in}{1.812707in}}%
\pgfpathcurveto{\pgfqpoint{3.969999in}{1.801657in}}{\pgfqpoint{3.974389in}{1.791058in}}{\pgfqpoint{3.982203in}{1.783244in}}%
\pgfpathcurveto{\pgfqpoint{3.990016in}{1.775431in}}{\pgfqpoint{4.000616in}{1.771040in}}{\pgfqpoint{4.011666in}{1.771040in}}%
\pgfpathclose%
\pgfusepath{stroke,fill}%
\end{pgfscope}%
\begin{pgfscope}%
\pgfpathrectangle{\pgfqpoint{0.800000in}{0.528000in}}{\pgfqpoint{4.960000in}{3.696000in}}%
\pgfusepath{clip}%
\pgfsetbuttcap%
\pgfsetroundjoin%
\definecolor{currentfill}{rgb}{0.000000,0.000000,0.000000}%
\pgfsetfillcolor{currentfill}%
\pgfsetlinewidth{1.003750pt}%
\definecolor{currentstroke}{rgb}{0.000000,0.000000,0.000000}%
\pgfsetstrokecolor{currentstroke}%
\pgfsetdash{}{0pt}%
\pgfpathmoveto{\pgfqpoint{4.011666in}{2.877687in}}%
\pgfpathcurveto{\pgfqpoint{4.022716in}{2.877687in}}{\pgfqpoint{4.033315in}{2.882077in}}{\pgfqpoint{4.041128in}{2.889891in}}%
\pgfpathcurveto{\pgfqpoint{4.048942in}{2.897704in}}{\pgfqpoint{4.053332in}{2.908303in}}{\pgfqpoint{4.053332in}{2.919353in}}%
\pgfpathcurveto{\pgfqpoint{4.053332in}{2.930404in}}{\pgfqpoint{4.048942in}{2.941003in}}{\pgfqpoint{4.041128in}{2.948816in}}%
\pgfpathcurveto{\pgfqpoint{4.033315in}{2.956630in}}{\pgfqpoint{4.022716in}{2.961020in}}{\pgfqpoint{4.011666in}{2.961020in}}%
\pgfpathcurveto{\pgfqpoint{4.000616in}{2.961020in}}{\pgfqpoint{3.990016in}{2.956630in}}{\pgfqpoint{3.982203in}{2.948816in}}%
\pgfpathcurveto{\pgfqpoint{3.974389in}{2.941003in}}{\pgfqpoint{3.969999in}{2.930404in}}{\pgfqpoint{3.969999in}{2.919353in}}%
\pgfpathcurveto{\pgfqpoint{3.969999in}{2.908303in}}{\pgfqpoint{3.974389in}{2.897704in}}{\pgfqpoint{3.982203in}{2.889891in}}%
\pgfpathcurveto{\pgfqpoint{3.990016in}{2.882077in}}{\pgfqpoint{4.000616in}{2.877687in}}{\pgfqpoint{4.011666in}{2.877687in}}%
\pgfpathclose%
\pgfusepath{stroke,fill}%
\end{pgfscope}%
\begin{pgfscope}%
\pgfpathrectangle{\pgfqpoint{0.800000in}{0.528000in}}{\pgfqpoint{4.960000in}{3.696000in}}%
\pgfusepath{clip}%
\pgfsetbuttcap%
\pgfsetroundjoin%
\definecolor{currentfill}{rgb}{0.000000,0.000000,0.000000}%
\pgfsetfillcolor{currentfill}%
\pgfsetlinewidth{1.003750pt}%
\definecolor{currentstroke}{rgb}{0.000000,0.000000,0.000000}%
\pgfsetstrokecolor{currentstroke}%
\pgfsetdash{}{0pt}%
\pgfpathmoveto{\pgfqpoint{4.011666in}{2.877687in}}%
\pgfpathcurveto{\pgfqpoint{4.022716in}{2.877687in}}{\pgfqpoint{4.033315in}{2.882077in}}{\pgfqpoint{4.041128in}{2.889891in}}%
\pgfpathcurveto{\pgfqpoint{4.048942in}{2.897704in}}{\pgfqpoint{4.053332in}{2.908303in}}{\pgfqpoint{4.053332in}{2.919353in}}%
\pgfpathcurveto{\pgfqpoint{4.053332in}{2.930404in}}{\pgfqpoint{4.048942in}{2.941003in}}{\pgfqpoint{4.041128in}{2.948816in}}%
\pgfpathcurveto{\pgfqpoint{4.033315in}{2.956630in}}{\pgfqpoint{4.022716in}{2.961020in}}{\pgfqpoint{4.011666in}{2.961020in}}%
\pgfpathcurveto{\pgfqpoint{4.000616in}{2.961020in}}{\pgfqpoint{3.990016in}{2.956630in}}{\pgfqpoint{3.982203in}{2.948816in}}%
\pgfpathcurveto{\pgfqpoint{3.974389in}{2.941003in}}{\pgfqpoint{3.969999in}{2.930404in}}{\pgfqpoint{3.969999in}{2.919353in}}%
\pgfpathcurveto{\pgfqpoint{3.969999in}{2.908303in}}{\pgfqpoint{3.974389in}{2.897704in}}{\pgfqpoint{3.982203in}{2.889891in}}%
\pgfpathcurveto{\pgfqpoint{3.990016in}{2.882077in}}{\pgfqpoint{4.000616in}{2.877687in}}{\pgfqpoint{4.011666in}{2.877687in}}%
\pgfpathclose%
\pgfusepath{stroke,fill}%
\end{pgfscope}%
\begin{pgfscope}%
\pgfpathrectangle{\pgfqpoint{0.800000in}{0.528000in}}{\pgfqpoint{4.960000in}{3.696000in}}%
\pgfusepath{clip}%
\pgfsetbuttcap%
\pgfsetroundjoin%
\definecolor{currentfill}{rgb}{0.000000,0.000000,0.000000}%
\pgfsetfillcolor{currentfill}%
\pgfsetlinewidth{1.003750pt}%
\definecolor{currentstroke}{rgb}{0.000000,0.000000,0.000000}%
\pgfsetstrokecolor{currentstroke}%
\pgfsetdash{}{0pt}%
\pgfpathmoveto{\pgfqpoint{4.011666in}{1.771040in}}%
\pgfpathcurveto{\pgfqpoint{4.022716in}{1.771040in}}{\pgfqpoint{4.033315in}{1.775431in}}{\pgfqpoint{4.041128in}{1.783244in}}%
\pgfpathcurveto{\pgfqpoint{4.048942in}{1.791058in}}{\pgfqpoint{4.053332in}{1.801657in}}{\pgfqpoint{4.053332in}{1.812707in}}%
\pgfpathcurveto{\pgfqpoint{4.053332in}{1.823757in}}{\pgfqpoint{4.048942in}{1.834356in}}{\pgfqpoint{4.041128in}{1.842170in}}%
\pgfpathcurveto{\pgfqpoint{4.033315in}{1.849983in}}{\pgfqpoint{4.022716in}{1.854374in}}{\pgfqpoint{4.011666in}{1.854374in}}%
\pgfpathcurveto{\pgfqpoint{4.000616in}{1.854374in}}{\pgfqpoint{3.990016in}{1.849983in}}{\pgfqpoint{3.982203in}{1.842170in}}%
\pgfpathcurveto{\pgfqpoint{3.974389in}{1.834356in}}{\pgfqpoint{3.969999in}{1.823757in}}{\pgfqpoint{3.969999in}{1.812707in}}%
\pgfpathcurveto{\pgfqpoint{3.969999in}{1.801657in}}{\pgfqpoint{3.974389in}{1.791058in}}{\pgfqpoint{3.982203in}{1.783244in}}%
\pgfpathcurveto{\pgfqpoint{3.990016in}{1.775431in}}{\pgfqpoint{4.000616in}{1.771040in}}{\pgfqpoint{4.011666in}{1.771040in}}%
\pgfpathclose%
\pgfusepath{stroke,fill}%
\end{pgfscope}%
\begin{pgfscope}%
\pgfpathrectangle{\pgfqpoint{0.800000in}{0.528000in}}{\pgfqpoint{4.960000in}{3.696000in}}%
\pgfusepath{clip}%
\pgfsetbuttcap%
\pgfsetroundjoin%
\definecolor{currentfill}{rgb}{0.000000,0.000000,0.000000}%
\pgfsetfillcolor{currentfill}%
\pgfsetlinewidth{1.003750pt}%
\definecolor{currentstroke}{rgb}{0.000000,0.000000,0.000000}%
\pgfsetstrokecolor{currentstroke}%
\pgfsetdash{}{0pt}%
\pgfpathmoveto{\pgfqpoint{4.011666in}{1.771040in}}%
\pgfpathcurveto{\pgfqpoint{4.022716in}{1.771040in}}{\pgfqpoint{4.033315in}{1.775431in}}{\pgfqpoint{4.041128in}{1.783244in}}%
\pgfpathcurveto{\pgfqpoint{4.048942in}{1.791058in}}{\pgfqpoint{4.053332in}{1.801657in}}{\pgfqpoint{4.053332in}{1.812707in}}%
\pgfpathcurveto{\pgfqpoint{4.053332in}{1.823757in}}{\pgfqpoint{4.048942in}{1.834356in}}{\pgfqpoint{4.041128in}{1.842170in}}%
\pgfpathcurveto{\pgfqpoint{4.033315in}{1.849983in}}{\pgfqpoint{4.022716in}{1.854374in}}{\pgfqpoint{4.011666in}{1.854374in}}%
\pgfpathcurveto{\pgfqpoint{4.000616in}{1.854374in}}{\pgfqpoint{3.990016in}{1.849983in}}{\pgfqpoint{3.982203in}{1.842170in}}%
\pgfpathcurveto{\pgfqpoint{3.974389in}{1.834356in}}{\pgfqpoint{3.969999in}{1.823757in}}{\pgfqpoint{3.969999in}{1.812707in}}%
\pgfpathcurveto{\pgfqpoint{3.969999in}{1.801657in}}{\pgfqpoint{3.974389in}{1.791058in}}{\pgfqpoint{3.982203in}{1.783244in}}%
\pgfpathcurveto{\pgfqpoint{3.990016in}{1.775431in}}{\pgfqpoint{4.000616in}{1.771040in}}{\pgfqpoint{4.011666in}{1.771040in}}%
\pgfpathclose%
\pgfusepath{stroke,fill}%
\end{pgfscope}%
\begin{pgfscope}%
\pgfpathrectangle{\pgfqpoint{0.800000in}{0.528000in}}{\pgfqpoint{4.960000in}{3.696000in}}%
\pgfusepath{clip}%
\pgfsetbuttcap%
\pgfsetroundjoin%
\definecolor{currentfill}{rgb}{0.000000,0.000000,0.000000}%
\pgfsetfillcolor{currentfill}%
\pgfsetlinewidth{1.003750pt}%
\definecolor{currentstroke}{rgb}{0.000000,0.000000,0.000000}%
\pgfsetstrokecolor{currentstroke}%
\pgfsetdash{}{0pt}%
\pgfpathmoveto{\pgfqpoint{4.011666in}{1.771040in}}%
\pgfpathcurveto{\pgfqpoint{4.022716in}{1.771040in}}{\pgfqpoint{4.033315in}{1.775431in}}{\pgfqpoint{4.041128in}{1.783244in}}%
\pgfpathcurveto{\pgfqpoint{4.048942in}{1.791058in}}{\pgfqpoint{4.053332in}{1.801657in}}{\pgfqpoint{4.053332in}{1.812707in}}%
\pgfpathcurveto{\pgfqpoint{4.053332in}{1.823757in}}{\pgfqpoint{4.048942in}{1.834356in}}{\pgfqpoint{4.041128in}{1.842170in}}%
\pgfpathcurveto{\pgfqpoint{4.033315in}{1.849983in}}{\pgfqpoint{4.022716in}{1.854374in}}{\pgfqpoint{4.011666in}{1.854374in}}%
\pgfpathcurveto{\pgfqpoint{4.000616in}{1.854374in}}{\pgfqpoint{3.990016in}{1.849983in}}{\pgfqpoint{3.982203in}{1.842170in}}%
\pgfpathcurveto{\pgfqpoint{3.974389in}{1.834356in}}{\pgfqpoint{3.969999in}{1.823757in}}{\pgfqpoint{3.969999in}{1.812707in}}%
\pgfpathcurveto{\pgfqpoint{3.969999in}{1.801657in}}{\pgfqpoint{3.974389in}{1.791058in}}{\pgfqpoint{3.982203in}{1.783244in}}%
\pgfpathcurveto{\pgfqpoint{3.990016in}{1.775431in}}{\pgfqpoint{4.000616in}{1.771040in}}{\pgfqpoint{4.011666in}{1.771040in}}%
\pgfpathclose%
\pgfusepath{stroke,fill}%
\end{pgfscope}%
\begin{pgfscope}%
\pgfpathrectangle{\pgfqpoint{0.800000in}{0.528000in}}{\pgfqpoint{4.960000in}{3.696000in}}%
\pgfusepath{clip}%
\pgfsetbuttcap%
\pgfsetroundjoin%
\definecolor{currentfill}{rgb}{0.000000,0.000000,0.000000}%
\pgfsetfillcolor{currentfill}%
\pgfsetlinewidth{1.003750pt}%
\definecolor{currentstroke}{rgb}{0.000000,0.000000,0.000000}%
\pgfsetstrokecolor{currentstroke}%
\pgfsetdash{}{0pt}%
\pgfpathmoveto{\pgfqpoint{4.011666in}{1.771040in}}%
\pgfpathcurveto{\pgfqpoint{4.022716in}{1.771040in}}{\pgfqpoint{4.033315in}{1.775431in}}{\pgfqpoint{4.041128in}{1.783244in}}%
\pgfpathcurveto{\pgfqpoint{4.048942in}{1.791058in}}{\pgfqpoint{4.053332in}{1.801657in}}{\pgfqpoint{4.053332in}{1.812707in}}%
\pgfpathcurveto{\pgfqpoint{4.053332in}{1.823757in}}{\pgfqpoint{4.048942in}{1.834356in}}{\pgfqpoint{4.041128in}{1.842170in}}%
\pgfpathcurveto{\pgfqpoint{4.033315in}{1.849983in}}{\pgfqpoint{4.022716in}{1.854374in}}{\pgfqpoint{4.011666in}{1.854374in}}%
\pgfpathcurveto{\pgfqpoint{4.000616in}{1.854374in}}{\pgfqpoint{3.990016in}{1.849983in}}{\pgfqpoint{3.982203in}{1.842170in}}%
\pgfpathcurveto{\pgfqpoint{3.974389in}{1.834356in}}{\pgfqpoint{3.969999in}{1.823757in}}{\pgfqpoint{3.969999in}{1.812707in}}%
\pgfpathcurveto{\pgfqpoint{3.969999in}{1.801657in}}{\pgfqpoint{3.974389in}{1.791058in}}{\pgfqpoint{3.982203in}{1.783244in}}%
\pgfpathcurveto{\pgfqpoint{3.990016in}{1.775431in}}{\pgfqpoint{4.000616in}{1.771040in}}{\pgfqpoint{4.011666in}{1.771040in}}%
\pgfpathclose%
\pgfusepath{stroke,fill}%
\end{pgfscope}%
\begin{pgfscope}%
\pgfpathrectangle{\pgfqpoint{0.800000in}{0.528000in}}{\pgfqpoint{4.960000in}{3.696000in}}%
\pgfusepath{clip}%
\pgfsetbuttcap%
\pgfsetroundjoin%
\definecolor{currentfill}{rgb}{0.000000,0.000000,0.000000}%
\pgfsetfillcolor{currentfill}%
\pgfsetlinewidth{1.003750pt}%
\definecolor{currentstroke}{rgb}{0.000000,0.000000,0.000000}%
\pgfsetstrokecolor{currentstroke}%
\pgfsetdash{}{0pt}%
\pgfpathmoveto{\pgfqpoint{4.011666in}{2.877687in}}%
\pgfpathcurveto{\pgfqpoint{4.022716in}{2.877687in}}{\pgfqpoint{4.033315in}{2.882077in}}{\pgfqpoint{4.041128in}{2.889891in}}%
\pgfpathcurveto{\pgfqpoint{4.048942in}{2.897704in}}{\pgfqpoint{4.053332in}{2.908303in}}{\pgfqpoint{4.053332in}{2.919353in}}%
\pgfpathcurveto{\pgfqpoint{4.053332in}{2.930404in}}{\pgfqpoint{4.048942in}{2.941003in}}{\pgfqpoint{4.041128in}{2.948816in}}%
\pgfpathcurveto{\pgfqpoint{4.033315in}{2.956630in}}{\pgfqpoint{4.022716in}{2.961020in}}{\pgfqpoint{4.011666in}{2.961020in}}%
\pgfpathcurveto{\pgfqpoint{4.000616in}{2.961020in}}{\pgfqpoint{3.990016in}{2.956630in}}{\pgfqpoint{3.982203in}{2.948816in}}%
\pgfpathcurveto{\pgfqpoint{3.974389in}{2.941003in}}{\pgfqpoint{3.969999in}{2.930404in}}{\pgfqpoint{3.969999in}{2.919353in}}%
\pgfpathcurveto{\pgfqpoint{3.969999in}{2.908303in}}{\pgfqpoint{3.974389in}{2.897704in}}{\pgfqpoint{3.982203in}{2.889891in}}%
\pgfpathcurveto{\pgfqpoint{3.990016in}{2.882077in}}{\pgfqpoint{4.000616in}{2.877687in}}{\pgfqpoint{4.011666in}{2.877687in}}%
\pgfpathclose%
\pgfusepath{stroke,fill}%
\end{pgfscope}%
\begin{pgfscope}%
\pgfpathrectangle{\pgfqpoint{0.800000in}{0.528000in}}{\pgfqpoint{4.960000in}{3.696000in}}%
\pgfusepath{clip}%
\pgfsetbuttcap%
\pgfsetroundjoin%
\definecolor{currentfill}{rgb}{0.000000,0.000000,0.000000}%
\pgfsetfillcolor{currentfill}%
\pgfsetlinewidth{1.003750pt}%
\definecolor{currentstroke}{rgb}{0.000000,0.000000,0.000000}%
\pgfsetstrokecolor{currentstroke}%
\pgfsetdash{}{0pt}%
\pgfpathmoveto{\pgfqpoint{4.011666in}{2.877687in}}%
\pgfpathcurveto{\pgfqpoint{4.022716in}{2.877687in}}{\pgfqpoint{4.033315in}{2.882077in}}{\pgfqpoint{4.041128in}{2.889891in}}%
\pgfpathcurveto{\pgfqpoint{4.048942in}{2.897704in}}{\pgfqpoint{4.053332in}{2.908303in}}{\pgfqpoint{4.053332in}{2.919353in}}%
\pgfpathcurveto{\pgfqpoint{4.053332in}{2.930404in}}{\pgfqpoint{4.048942in}{2.941003in}}{\pgfqpoint{4.041128in}{2.948816in}}%
\pgfpathcurveto{\pgfqpoint{4.033315in}{2.956630in}}{\pgfqpoint{4.022716in}{2.961020in}}{\pgfqpoint{4.011666in}{2.961020in}}%
\pgfpathcurveto{\pgfqpoint{4.000616in}{2.961020in}}{\pgfqpoint{3.990016in}{2.956630in}}{\pgfqpoint{3.982203in}{2.948816in}}%
\pgfpathcurveto{\pgfqpoint{3.974389in}{2.941003in}}{\pgfqpoint{3.969999in}{2.930404in}}{\pgfqpoint{3.969999in}{2.919353in}}%
\pgfpathcurveto{\pgfqpoint{3.969999in}{2.908303in}}{\pgfqpoint{3.974389in}{2.897704in}}{\pgfqpoint{3.982203in}{2.889891in}}%
\pgfpathcurveto{\pgfqpoint{3.990016in}{2.882077in}}{\pgfqpoint{4.000616in}{2.877687in}}{\pgfqpoint{4.011666in}{2.877687in}}%
\pgfpathclose%
\pgfusepath{stroke,fill}%
\end{pgfscope}%
\begin{pgfscope}%
\pgfpathrectangle{\pgfqpoint{0.800000in}{0.528000in}}{\pgfqpoint{4.960000in}{3.696000in}}%
\pgfusepath{clip}%
\pgfsetbuttcap%
\pgfsetroundjoin%
\definecolor{currentfill}{rgb}{0.000000,0.000000,0.000000}%
\pgfsetfillcolor{currentfill}%
\pgfsetlinewidth{1.003750pt}%
\definecolor{currentstroke}{rgb}{0.000000,0.000000,0.000000}%
\pgfsetstrokecolor{currentstroke}%
\pgfsetdash{}{0pt}%
\pgfpathmoveto{\pgfqpoint{4.011666in}{2.877687in}}%
\pgfpathcurveto{\pgfqpoint{4.022716in}{2.877687in}}{\pgfqpoint{4.033315in}{2.882077in}}{\pgfqpoint{4.041128in}{2.889891in}}%
\pgfpathcurveto{\pgfqpoint{4.048942in}{2.897704in}}{\pgfqpoint{4.053332in}{2.908303in}}{\pgfqpoint{4.053332in}{2.919353in}}%
\pgfpathcurveto{\pgfqpoint{4.053332in}{2.930404in}}{\pgfqpoint{4.048942in}{2.941003in}}{\pgfqpoint{4.041128in}{2.948816in}}%
\pgfpathcurveto{\pgfqpoint{4.033315in}{2.956630in}}{\pgfqpoint{4.022716in}{2.961020in}}{\pgfqpoint{4.011666in}{2.961020in}}%
\pgfpathcurveto{\pgfqpoint{4.000616in}{2.961020in}}{\pgfqpoint{3.990016in}{2.956630in}}{\pgfqpoint{3.982203in}{2.948816in}}%
\pgfpathcurveto{\pgfqpoint{3.974389in}{2.941003in}}{\pgfqpoint{3.969999in}{2.930404in}}{\pgfqpoint{3.969999in}{2.919353in}}%
\pgfpathcurveto{\pgfqpoint{3.969999in}{2.908303in}}{\pgfqpoint{3.974389in}{2.897704in}}{\pgfqpoint{3.982203in}{2.889891in}}%
\pgfpathcurveto{\pgfqpoint{3.990016in}{2.882077in}}{\pgfqpoint{4.000616in}{2.877687in}}{\pgfqpoint{4.011666in}{2.877687in}}%
\pgfpathclose%
\pgfusepath{stroke,fill}%
\end{pgfscope}%
\begin{pgfscope}%
\pgfpathrectangle{\pgfqpoint{0.800000in}{0.528000in}}{\pgfqpoint{4.960000in}{3.696000in}}%
\pgfusepath{clip}%
\pgfsetbuttcap%
\pgfsetroundjoin%
\definecolor{currentfill}{rgb}{0.000000,0.000000,0.000000}%
\pgfsetfillcolor{currentfill}%
\pgfsetlinewidth{1.003750pt}%
\definecolor{currentstroke}{rgb}{0.000000,0.000000,0.000000}%
\pgfsetstrokecolor{currentstroke}%
\pgfsetdash{}{0pt}%
\pgfpathmoveto{\pgfqpoint{4.011666in}{2.877687in}}%
\pgfpathcurveto{\pgfqpoint{4.022716in}{2.877687in}}{\pgfqpoint{4.033315in}{2.882077in}}{\pgfqpoint{4.041128in}{2.889891in}}%
\pgfpathcurveto{\pgfqpoint{4.048942in}{2.897704in}}{\pgfqpoint{4.053332in}{2.908303in}}{\pgfqpoint{4.053332in}{2.919353in}}%
\pgfpathcurveto{\pgfqpoint{4.053332in}{2.930404in}}{\pgfqpoint{4.048942in}{2.941003in}}{\pgfqpoint{4.041128in}{2.948816in}}%
\pgfpathcurveto{\pgfqpoint{4.033315in}{2.956630in}}{\pgfqpoint{4.022716in}{2.961020in}}{\pgfqpoint{4.011666in}{2.961020in}}%
\pgfpathcurveto{\pgfqpoint{4.000616in}{2.961020in}}{\pgfqpoint{3.990016in}{2.956630in}}{\pgfqpoint{3.982203in}{2.948816in}}%
\pgfpathcurveto{\pgfqpoint{3.974389in}{2.941003in}}{\pgfqpoint{3.969999in}{2.930404in}}{\pgfqpoint{3.969999in}{2.919353in}}%
\pgfpathcurveto{\pgfqpoint{3.969999in}{2.908303in}}{\pgfqpoint{3.974389in}{2.897704in}}{\pgfqpoint{3.982203in}{2.889891in}}%
\pgfpathcurveto{\pgfqpoint{3.990016in}{2.882077in}}{\pgfqpoint{4.000616in}{2.877687in}}{\pgfqpoint{4.011666in}{2.877687in}}%
\pgfpathclose%
\pgfusepath{stroke,fill}%
\end{pgfscope}%
\begin{pgfscope}%
\pgfpathrectangle{\pgfqpoint{0.800000in}{0.528000in}}{\pgfqpoint{4.960000in}{3.696000in}}%
\pgfusepath{clip}%
\pgfsetbuttcap%
\pgfsetroundjoin%
\definecolor{currentfill}{rgb}{0.000000,0.000000,0.000000}%
\pgfsetfillcolor{currentfill}%
\pgfsetlinewidth{1.003750pt}%
\definecolor{currentstroke}{rgb}{0.000000,0.000000,0.000000}%
\pgfsetstrokecolor{currentstroke}%
\pgfsetdash{}{0pt}%
\pgfpathmoveto{\pgfqpoint{4.011666in}{1.771040in}}%
\pgfpathcurveto{\pgfqpoint{4.022716in}{1.771040in}}{\pgfqpoint{4.033315in}{1.775431in}}{\pgfqpoint{4.041128in}{1.783244in}}%
\pgfpathcurveto{\pgfqpoint{4.048942in}{1.791058in}}{\pgfqpoint{4.053332in}{1.801657in}}{\pgfqpoint{4.053332in}{1.812707in}}%
\pgfpathcurveto{\pgfqpoint{4.053332in}{1.823757in}}{\pgfqpoint{4.048942in}{1.834356in}}{\pgfqpoint{4.041128in}{1.842170in}}%
\pgfpathcurveto{\pgfqpoint{4.033315in}{1.849983in}}{\pgfqpoint{4.022716in}{1.854374in}}{\pgfqpoint{4.011666in}{1.854374in}}%
\pgfpathcurveto{\pgfqpoint{4.000616in}{1.854374in}}{\pgfqpoint{3.990016in}{1.849983in}}{\pgfqpoint{3.982203in}{1.842170in}}%
\pgfpathcurveto{\pgfqpoint{3.974389in}{1.834356in}}{\pgfqpoint{3.969999in}{1.823757in}}{\pgfqpoint{3.969999in}{1.812707in}}%
\pgfpathcurveto{\pgfqpoint{3.969999in}{1.801657in}}{\pgfqpoint{3.974389in}{1.791058in}}{\pgfqpoint{3.982203in}{1.783244in}}%
\pgfpathcurveto{\pgfqpoint{3.990016in}{1.775431in}}{\pgfqpoint{4.000616in}{1.771040in}}{\pgfqpoint{4.011666in}{1.771040in}}%
\pgfpathclose%
\pgfusepath{stroke,fill}%
\end{pgfscope}%
\begin{pgfscope}%
\pgfpathrectangle{\pgfqpoint{0.800000in}{0.528000in}}{\pgfqpoint{4.960000in}{3.696000in}}%
\pgfusepath{clip}%
\pgfsetbuttcap%
\pgfsetroundjoin%
\definecolor{currentfill}{rgb}{0.000000,0.000000,0.000000}%
\pgfsetfillcolor{currentfill}%
\pgfsetlinewidth{1.003750pt}%
\definecolor{currentstroke}{rgb}{0.000000,0.000000,0.000000}%
\pgfsetstrokecolor{currentstroke}%
\pgfsetdash{}{0pt}%
\pgfpathmoveto{\pgfqpoint{4.011666in}{2.877687in}}%
\pgfpathcurveto{\pgfqpoint{4.022716in}{2.877687in}}{\pgfqpoint{4.033315in}{2.882077in}}{\pgfqpoint{4.041128in}{2.889891in}}%
\pgfpathcurveto{\pgfqpoint{4.048942in}{2.897704in}}{\pgfqpoint{4.053332in}{2.908303in}}{\pgfqpoint{4.053332in}{2.919353in}}%
\pgfpathcurveto{\pgfqpoint{4.053332in}{2.930404in}}{\pgfqpoint{4.048942in}{2.941003in}}{\pgfqpoint{4.041128in}{2.948816in}}%
\pgfpathcurveto{\pgfqpoint{4.033315in}{2.956630in}}{\pgfqpoint{4.022716in}{2.961020in}}{\pgfqpoint{4.011666in}{2.961020in}}%
\pgfpathcurveto{\pgfqpoint{4.000616in}{2.961020in}}{\pgfqpoint{3.990016in}{2.956630in}}{\pgfqpoint{3.982203in}{2.948816in}}%
\pgfpathcurveto{\pgfqpoint{3.974389in}{2.941003in}}{\pgfqpoint{3.969999in}{2.930404in}}{\pgfqpoint{3.969999in}{2.919353in}}%
\pgfpathcurveto{\pgfqpoint{3.969999in}{2.908303in}}{\pgfqpoint{3.974389in}{2.897704in}}{\pgfqpoint{3.982203in}{2.889891in}}%
\pgfpathcurveto{\pgfqpoint{3.990016in}{2.882077in}}{\pgfqpoint{4.000616in}{2.877687in}}{\pgfqpoint{4.011666in}{2.877687in}}%
\pgfpathclose%
\pgfusepath{stroke,fill}%
\end{pgfscope}%
\begin{pgfscope}%
\pgfpathrectangle{\pgfqpoint{0.800000in}{0.528000in}}{\pgfqpoint{4.960000in}{3.696000in}}%
\pgfusepath{clip}%
\pgfsetbuttcap%
\pgfsetroundjoin%
\definecolor{currentfill}{rgb}{0.000000,0.000000,0.000000}%
\pgfsetfillcolor{currentfill}%
\pgfsetlinewidth{1.003750pt}%
\definecolor{currentstroke}{rgb}{0.000000,0.000000,0.000000}%
\pgfsetstrokecolor{currentstroke}%
\pgfsetdash{}{0pt}%
\pgfpathmoveto{\pgfqpoint{4.011666in}{2.877687in}}%
\pgfpathcurveto{\pgfqpoint{4.022716in}{2.877687in}}{\pgfqpoint{4.033315in}{2.882077in}}{\pgfqpoint{4.041128in}{2.889891in}}%
\pgfpathcurveto{\pgfqpoint{4.048942in}{2.897704in}}{\pgfqpoint{4.053332in}{2.908303in}}{\pgfqpoint{4.053332in}{2.919353in}}%
\pgfpathcurveto{\pgfqpoint{4.053332in}{2.930404in}}{\pgfqpoint{4.048942in}{2.941003in}}{\pgfqpoint{4.041128in}{2.948816in}}%
\pgfpathcurveto{\pgfqpoint{4.033315in}{2.956630in}}{\pgfqpoint{4.022716in}{2.961020in}}{\pgfqpoint{4.011666in}{2.961020in}}%
\pgfpathcurveto{\pgfqpoint{4.000616in}{2.961020in}}{\pgfqpoint{3.990016in}{2.956630in}}{\pgfqpoint{3.982203in}{2.948816in}}%
\pgfpathcurveto{\pgfqpoint{3.974389in}{2.941003in}}{\pgfqpoint{3.969999in}{2.930404in}}{\pgfqpoint{3.969999in}{2.919353in}}%
\pgfpathcurveto{\pgfqpoint{3.969999in}{2.908303in}}{\pgfqpoint{3.974389in}{2.897704in}}{\pgfqpoint{3.982203in}{2.889891in}}%
\pgfpathcurveto{\pgfqpoint{3.990016in}{2.882077in}}{\pgfqpoint{4.000616in}{2.877687in}}{\pgfqpoint{4.011666in}{2.877687in}}%
\pgfpathclose%
\pgfusepath{stroke,fill}%
\end{pgfscope}%
\begin{pgfscope}%
\pgfpathrectangle{\pgfqpoint{0.800000in}{0.528000in}}{\pgfqpoint{4.960000in}{3.696000in}}%
\pgfusepath{clip}%
\pgfsetbuttcap%
\pgfsetroundjoin%
\definecolor{currentfill}{rgb}{0.000000,0.000000,0.000000}%
\pgfsetfillcolor{currentfill}%
\pgfsetlinewidth{1.003750pt}%
\definecolor{currentstroke}{rgb}{0.000000,0.000000,0.000000}%
\pgfsetstrokecolor{currentstroke}%
\pgfsetdash{}{0pt}%
\pgfpathmoveto{\pgfqpoint{5.504545in}{2.877687in}}%
\pgfpathcurveto{\pgfqpoint{5.515596in}{2.877687in}}{\pgfqpoint{5.526195in}{2.882077in}}{\pgfqpoint{5.534008in}{2.889891in}}%
\pgfpathcurveto{\pgfqpoint{5.541822in}{2.897704in}}{\pgfqpoint{5.546212in}{2.908303in}}{\pgfqpoint{5.546212in}{2.919353in}}%
\pgfpathcurveto{\pgfqpoint{5.546212in}{2.930404in}}{\pgfqpoint{5.541822in}{2.941003in}}{\pgfqpoint{5.534008in}{2.948816in}}%
\pgfpathcurveto{\pgfqpoint{5.526195in}{2.956630in}}{\pgfqpoint{5.515596in}{2.961020in}}{\pgfqpoint{5.504545in}{2.961020in}}%
\pgfpathcurveto{\pgfqpoint{5.493495in}{2.961020in}}{\pgfqpoint{5.482896in}{2.956630in}}{\pgfqpoint{5.475083in}{2.948816in}}%
\pgfpathcurveto{\pgfqpoint{5.467269in}{2.941003in}}{\pgfqpoint{5.462879in}{2.930404in}}{\pgfqpoint{5.462879in}{2.919353in}}%
\pgfpathcurveto{\pgfqpoint{5.462879in}{2.908303in}}{\pgfqpoint{5.467269in}{2.897704in}}{\pgfqpoint{5.475083in}{2.889891in}}%
\pgfpathcurveto{\pgfqpoint{5.482896in}{2.882077in}}{\pgfqpoint{5.493495in}{2.877687in}}{\pgfqpoint{5.504545in}{2.877687in}}%
\pgfpathclose%
\pgfusepath{stroke,fill}%
\end{pgfscope}%
\begin{pgfscope}%
\pgfpathrectangle{\pgfqpoint{0.800000in}{0.528000in}}{\pgfqpoint{4.960000in}{3.696000in}}%
\pgfusepath{clip}%
\pgfsetbuttcap%
\pgfsetroundjoin%
\definecolor{currentfill}{rgb}{0.000000,0.000000,0.000000}%
\pgfsetfillcolor{currentfill}%
\pgfsetlinewidth{1.003750pt}%
\definecolor{currentstroke}{rgb}{0.000000,0.000000,0.000000}%
\pgfsetstrokecolor{currentstroke}%
\pgfsetdash{}{0pt}%
\pgfpathmoveto{\pgfqpoint{5.504545in}{2.877687in}}%
\pgfpathcurveto{\pgfqpoint{5.515596in}{2.877687in}}{\pgfqpoint{5.526195in}{2.882077in}}{\pgfqpoint{5.534008in}{2.889891in}}%
\pgfpathcurveto{\pgfqpoint{5.541822in}{2.897704in}}{\pgfqpoint{5.546212in}{2.908303in}}{\pgfqpoint{5.546212in}{2.919353in}}%
\pgfpathcurveto{\pgfqpoint{5.546212in}{2.930404in}}{\pgfqpoint{5.541822in}{2.941003in}}{\pgfqpoint{5.534008in}{2.948816in}}%
\pgfpathcurveto{\pgfqpoint{5.526195in}{2.956630in}}{\pgfqpoint{5.515596in}{2.961020in}}{\pgfqpoint{5.504545in}{2.961020in}}%
\pgfpathcurveto{\pgfqpoint{5.493495in}{2.961020in}}{\pgfqpoint{5.482896in}{2.956630in}}{\pgfqpoint{5.475083in}{2.948816in}}%
\pgfpathcurveto{\pgfqpoint{5.467269in}{2.941003in}}{\pgfqpoint{5.462879in}{2.930404in}}{\pgfqpoint{5.462879in}{2.919353in}}%
\pgfpathcurveto{\pgfqpoint{5.462879in}{2.908303in}}{\pgfqpoint{5.467269in}{2.897704in}}{\pgfqpoint{5.475083in}{2.889891in}}%
\pgfpathcurveto{\pgfqpoint{5.482896in}{2.882077in}}{\pgfqpoint{5.493495in}{2.877687in}}{\pgfqpoint{5.504545in}{2.877687in}}%
\pgfpathclose%
\pgfusepath{stroke,fill}%
\end{pgfscope}%
\begin{pgfscope}%
\pgfpathrectangle{\pgfqpoint{0.800000in}{0.528000in}}{\pgfqpoint{4.960000in}{3.696000in}}%
\pgfusepath{clip}%
\pgfsetbuttcap%
\pgfsetroundjoin%
\definecolor{currentfill}{rgb}{0.000000,0.000000,0.000000}%
\pgfsetfillcolor{currentfill}%
\pgfsetlinewidth{1.003750pt}%
\definecolor{currentstroke}{rgb}{0.000000,0.000000,0.000000}%
\pgfsetstrokecolor{currentstroke}%
\pgfsetdash{}{0pt}%
\pgfpathmoveto{\pgfqpoint{5.504545in}{2.877687in}}%
\pgfpathcurveto{\pgfqpoint{5.515596in}{2.877687in}}{\pgfqpoint{5.526195in}{2.882077in}}{\pgfqpoint{5.534008in}{2.889891in}}%
\pgfpathcurveto{\pgfqpoint{5.541822in}{2.897704in}}{\pgfqpoint{5.546212in}{2.908303in}}{\pgfqpoint{5.546212in}{2.919353in}}%
\pgfpathcurveto{\pgfqpoint{5.546212in}{2.930404in}}{\pgfqpoint{5.541822in}{2.941003in}}{\pgfqpoint{5.534008in}{2.948816in}}%
\pgfpathcurveto{\pgfqpoint{5.526195in}{2.956630in}}{\pgfqpoint{5.515596in}{2.961020in}}{\pgfqpoint{5.504545in}{2.961020in}}%
\pgfpathcurveto{\pgfqpoint{5.493495in}{2.961020in}}{\pgfqpoint{5.482896in}{2.956630in}}{\pgfqpoint{5.475083in}{2.948816in}}%
\pgfpathcurveto{\pgfqpoint{5.467269in}{2.941003in}}{\pgfqpoint{5.462879in}{2.930404in}}{\pgfqpoint{5.462879in}{2.919353in}}%
\pgfpathcurveto{\pgfqpoint{5.462879in}{2.908303in}}{\pgfqpoint{5.467269in}{2.897704in}}{\pgfqpoint{5.475083in}{2.889891in}}%
\pgfpathcurveto{\pgfqpoint{5.482896in}{2.882077in}}{\pgfqpoint{5.493495in}{2.877687in}}{\pgfqpoint{5.504545in}{2.877687in}}%
\pgfpathclose%
\pgfusepath{stroke,fill}%
\end{pgfscope}%
\begin{pgfscope}%
\pgfpathrectangle{\pgfqpoint{0.800000in}{0.528000in}}{\pgfqpoint{4.960000in}{3.696000in}}%
\pgfusepath{clip}%
\pgfsetbuttcap%
\pgfsetroundjoin%
\definecolor{currentfill}{rgb}{0.000000,0.000000,0.000000}%
\pgfsetfillcolor{currentfill}%
\pgfsetlinewidth{1.003750pt}%
\definecolor{currentstroke}{rgb}{0.000000,0.000000,0.000000}%
\pgfsetstrokecolor{currentstroke}%
\pgfsetdash{}{0pt}%
\pgfpathmoveto{\pgfqpoint{5.504545in}{2.877687in}}%
\pgfpathcurveto{\pgfqpoint{5.515596in}{2.877687in}}{\pgfqpoint{5.526195in}{2.882077in}}{\pgfqpoint{5.534008in}{2.889891in}}%
\pgfpathcurveto{\pgfqpoint{5.541822in}{2.897704in}}{\pgfqpoint{5.546212in}{2.908303in}}{\pgfqpoint{5.546212in}{2.919353in}}%
\pgfpathcurveto{\pgfqpoint{5.546212in}{2.930404in}}{\pgfqpoint{5.541822in}{2.941003in}}{\pgfqpoint{5.534008in}{2.948816in}}%
\pgfpathcurveto{\pgfqpoint{5.526195in}{2.956630in}}{\pgfqpoint{5.515596in}{2.961020in}}{\pgfqpoint{5.504545in}{2.961020in}}%
\pgfpathcurveto{\pgfqpoint{5.493495in}{2.961020in}}{\pgfqpoint{5.482896in}{2.956630in}}{\pgfqpoint{5.475083in}{2.948816in}}%
\pgfpathcurveto{\pgfqpoint{5.467269in}{2.941003in}}{\pgfqpoint{5.462879in}{2.930404in}}{\pgfqpoint{5.462879in}{2.919353in}}%
\pgfpathcurveto{\pgfqpoint{5.462879in}{2.908303in}}{\pgfqpoint{5.467269in}{2.897704in}}{\pgfqpoint{5.475083in}{2.889891in}}%
\pgfpathcurveto{\pgfqpoint{5.482896in}{2.882077in}}{\pgfqpoint{5.493495in}{2.877687in}}{\pgfqpoint{5.504545in}{2.877687in}}%
\pgfpathclose%
\pgfusepath{stroke,fill}%
\end{pgfscope}%
\begin{pgfscope}%
\pgfpathrectangle{\pgfqpoint{0.800000in}{0.528000in}}{\pgfqpoint{4.960000in}{3.696000in}}%
\pgfusepath{clip}%
\pgfsetbuttcap%
\pgfsetroundjoin%
\definecolor{currentfill}{rgb}{0.000000,0.000000,0.000000}%
\pgfsetfillcolor{currentfill}%
\pgfsetlinewidth{1.003750pt}%
\definecolor{currentstroke}{rgb}{0.000000,0.000000,0.000000}%
\pgfsetstrokecolor{currentstroke}%
\pgfsetdash{}{0pt}%
\pgfpathmoveto{\pgfqpoint{5.504545in}{2.877687in}}%
\pgfpathcurveto{\pgfqpoint{5.515596in}{2.877687in}}{\pgfqpoint{5.526195in}{2.882077in}}{\pgfqpoint{5.534008in}{2.889891in}}%
\pgfpathcurveto{\pgfqpoint{5.541822in}{2.897704in}}{\pgfqpoint{5.546212in}{2.908303in}}{\pgfqpoint{5.546212in}{2.919353in}}%
\pgfpathcurveto{\pgfqpoint{5.546212in}{2.930404in}}{\pgfqpoint{5.541822in}{2.941003in}}{\pgfqpoint{5.534008in}{2.948816in}}%
\pgfpathcurveto{\pgfqpoint{5.526195in}{2.956630in}}{\pgfqpoint{5.515596in}{2.961020in}}{\pgfqpoint{5.504545in}{2.961020in}}%
\pgfpathcurveto{\pgfqpoint{5.493495in}{2.961020in}}{\pgfqpoint{5.482896in}{2.956630in}}{\pgfqpoint{5.475083in}{2.948816in}}%
\pgfpathcurveto{\pgfqpoint{5.467269in}{2.941003in}}{\pgfqpoint{5.462879in}{2.930404in}}{\pgfqpoint{5.462879in}{2.919353in}}%
\pgfpathcurveto{\pgfqpoint{5.462879in}{2.908303in}}{\pgfqpoint{5.467269in}{2.897704in}}{\pgfqpoint{5.475083in}{2.889891in}}%
\pgfpathcurveto{\pgfqpoint{5.482896in}{2.882077in}}{\pgfqpoint{5.493495in}{2.877687in}}{\pgfqpoint{5.504545in}{2.877687in}}%
\pgfpathclose%
\pgfusepath{stroke,fill}%
\end{pgfscope}%
\begin{pgfscope}%
\pgfpathrectangle{\pgfqpoint{0.800000in}{0.528000in}}{\pgfqpoint{4.960000in}{3.696000in}}%
\pgfusepath{clip}%
\pgfsetbuttcap%
\pgfsetroundjoin%
\definecolor{currentfill}{rgb}{0.000000,0.000000,0.000000}%
\pgfsetfillcolor{currentfill}%
\pgfsetlinewidth{1.003750pt}%
\definecolor{currentstroke}{rgb}{0.000000,0.000000,0.000000}%
\pgfsetstrokecolor{currentstroke}%
\pgfsetdash{}{0pt}%
\pgfpathmoveto{\pgfqpoint{5.504545in}{2.877687in}}%
\pgfpathcurveto{\pgfqpoint{5.515596in}{2.877687in}}{\pgfqpoint{5.526195in}{2.882077in}}{\pgfqpoint{5.534008in}{2.889891in}}%
\pgfpathcurveto{\pgfqpoint{5.541822in}{2.897704in}}{\pgfqpoint{5.546212in}{2.908303in}}{\pgfqpoint{5.546212in}{2.919353in}}%
\pgfpathcurveto{\pgfqpoint{5.546212in}{2.930404in}}{\pgfqpoint{5.541822in}{2.941003in}}{\pgfqpoint{5.534008in}{2.948816in}}%
\pgfpathcurveto{\pgfqpoint{5.526195in}{2.956630in}}{\pgfqpoint{5.515596in}{2.961020in}}{\pgfqpoint{5.504545in}{2.961020in}}%
\pgfpathcurveto{\pgfqpoint{5.493495in}{2.961020in}}{\pgfqpoint{5.482896in}{2.956630in}}{\pgfqpoint{5.475083in}{2.948816in}}%
\pgfpathcurveto{\pgfqpoint{5.467269in}{2.941003in}}{\pgfqpoint{5.462879in}{2.930404in}}{\pgfqpoint{5.462879in}{2.919353in}}%
\pgfpathcurveto{\pgfqpoint{5.462879in}{2.908303in}}{\pgfqpoint{5.467269in}{2.897704in}}{\pgfqpoint{5.475083in}{2.889891in}}%
\pgfpathcurveto{\pgfqpoint{5.482896in}{2.882077in}}{\pgfqpoint{5.493495in}{2.877687in}}{\pgfqpoint{5.504545in}{2.877687in}}%
\pgfpathclose%
\pgfusepath{stroke,fill}%
\end{pgfscope}%
\begin{pgfscope}%
\pgfpathrectangle{\pgfqpoint{0.800000in}{0.528000in}}{\pgfqpoint{4.960000in}{3.696000in}}%
\pgfusepath{clip}%
\pgfsetbuttcap%
\pgfsetroundjoin%
\definecolor{currentfill}{rgb}{0.000000,0.000000,0.000000}%
\pgfsetfillcolor{currentfill}%
\pgfsetlinewidth{1.003750pt}%
\definecolor{currentstroke}{rgb}{0.000000,0.000000,0.000000}%
\pgfsetstrokecolor{currentstroke}%
\pgfsetdash{}{0pt}%
\pgfpathmoveto{\pgfqpoint{5.504545in}{2.877687in}}%
\pgfpathcurveto{\pgfqpoint{5.515596in}{2.877687in}}{\pgfqpoint{5.526195in}{2.882077in}}{\pgfqpoint{5.534008in}{2.889891in}}%
\pgfpathcurveto{\pgfqpoint{5.541822in}{2.897704in}}{\pgfqpoint{5.546212in}{2.908303in}}{\pgfqpoint{5.546212in}{2.919353in}}%
\pgfpathcurveto{\pgfqpoint{5.546212in}{2.930404in}}{\pgfqpoint{5.541822in}{2.941003in}}{\pgfqpoint{5.534008in}{2.948816in}}%
\pgfpathcurveto{\pgfqpoint{5.526195in}{2.956630in}}{\pgfqpoint{5.515596in}{2.961020in}}{\pgfqpoint{5.504545in}{2.961020in}}%
\pgfpathcurveto{\pgfqpoint{5.493495in}{2.961020in}}{\pgfqpoint{5.482896in}{2.956630in}}{\pgfqpoint{5.475083in}{2.948816in}}%
\pgfpathcurveto{\pgfqpoint{5.467269in}{2.941003in}}{\pgfqpoint{5.462879in}{2.930404in}}{\pgfqpoint{5.462879in}{2.919353in}}%
\pgfpathcurveto{\pgfqpoint{5.462879in}{2.908303in}}{\pgfqpoint{5.467269in}{2.897704in}}{\pgfqpoint{5.475083in}{2.889891in}}%
\pgfpathcurveto{\pgfqpoint{5.482896in}{2.882077in}}{\pgfqpoint{5.493495in}{2.877687in}}{\pgfqpoint{5.504545in}{2.877687in}}%
\pgfpathclose%
\pgfusepath{stroke,fill}%
\end{pgfscope}%
\begin{pgfscope}%
\pgfpathrectangle{\pgfqpoint{0.800000in}{0.528000in}}{\pgfqpoint{4.960000in}{3.696000in}}%
\pgfusepath{clip}%
\pgfsetbuttcap%
\pgfsetroundjoin%
\definecolor{currentfill}{rgb}{0.000000,0.000000,0.000000}%
\pgfsetfillcolor{currentfill}%
\pgfsetlinewidth{1.003750pt}%
\definecolor{currentstroke}{rgb}{0.000000,0.000000,0.000000}%
\pgfsetstrokecolor{currentstroke}%
\pgfsetdash{}{0pt}%
\pgfpathmoveto{\pgfqpoint{5.504545in}{2.877687in}}%
\pgfpathcurveto{\pgfqpoint{5.515596in}{2.877687in}}{\pgfqpoint{5.526195in}{2.882077in}}{\pgfqpoint{5.534008in}{2.889891in}}%
\pgfpathcurveto{\pgfqpoint{5.541822in}{2.897704in}}{\pgfqpoint{5.546212in}{2.908303in}}{\pgfqpoint{5.546212in}{2.919353in}}%
\pgfpathcurveto{\pgfqpoint{5.546212in}{2.930404in}}{\pgfqpoint{5.541822in}{2.941003in}}{\pgfqpoint{5.534008in}{2.948816in}}%
\pgfpathcurveto{\pgfqpoint{5.526195in}{2.956630in}}{\pgfqpoint{5.515596in}{2.961020in}}{\pgfqpoint{5.504545in}{2.961020in}}%
\pgfpathcurveto{\pgfqpoint{5.493495in}{2.961020in}}{\pgfqpoint{5.482896in}{2.956630in}}{\pgfqpoint{5.475083in}{2.948816in}}%
\pgfpathcurveto{\pgfqpoint{5.467269in}{2.941003in}}{\pgfqpoint{5.462879in}{2.930404in}}{\pgfqpoint{5.462879in}{2.919353in}}%
\pgfpathcurveto{\pgfqpoint{5.462879in}{2.908303in}}{\pgfqpoint{5.467269in}{2.897704in}}{\pgfqpoint{5.475083in}{2.889891in}}%
\pgfpathcurveto{\pgfqpoint{5.482896in}{2.882077in}}{\pgfqpoint{5.493495in}{2.877687in}}{\pgfqpoint{5.504545in}{2.877687in}}%
\pgfpathclose%
\pgfusepath{stroke,fill}%
\end{pgfscope}%
\begin{pgfscope}%
\pgfpathrectangle{\pgfqpoint{0.800000in}{0.528000in}}{\pgfqpoint{4.960000in}{3.696000in}}%
\pgfusepath{clip}%
\pgfsetbuttcap%
\pgfsetroundjoin%
\definecolor{currentfill}{rgb}{0.000000,0.000000,0.000000}%
\pgfsetfillcolor{currentfill}%
\pgfsetlinewidth{1.003750pt}%
\definecolor{currentstroke}{rgb}{0.000000,0.000000,0.000000}%
\pgfsetstrokecolor{currentstroke}%
\pgfsetdash{}{0pt}%
\pgfpathmoveto{\pgfqpoint{5.504545in}{2.877687in}}%
\pgfpathcurveto{\pgfqpoint{5.515596in}{2.877687in}}{\pgfqpoint{5.526195in}{2.882077in}}{\pgfqpoint{5.534008in}{2.889891in}}%
\pgfpathcurveto{\pgfqpoint{5.541822in}{2.897704in}}{\pgfqpoint{5.546212in}{2.908303in}}{\pgfqpoint{5.546212in}{2.919353in}}%
\pgfpathcurveto{\pgfqpoint{5.546212in}{2.930404in}}{\pgfqpoint{5.541822in}{2.941003in}}{\pgfqpoint{5.534008in}{2.948816in}}%
\pgfpathcurveto{\pgfqpoint{5.526195in}{2.956630in}}{\pgfqpoint{5.515596in}{2.961020in}}{\pgfqpoint{5.504545in}{2.961020in}}%
\pgfpathcurveto{\pgfqpoint{5.493495in}{2.961020in}}{\pgfqpoint{5.482896in}{2.956630in}}{\pgfqpoint{5.475083in}{2.948816in}}%
\pgfpathcurveto{\pgfqpoint{5.467269in}{2.941003in}}{\pgfqpoint{5.462879in}{2.930404in}}{\pgfqpoint{5.462879in}{2.919353in}}%
\pgfpathcurveto{\pgfqpoint{5.462879in}{2.908303in}}{\pgfqpoint{5.467269in}{2.897704in}}{\pgfqpoint{5.475083in}{2.889891in}}%
\pgfpathcurveto{\pgfqpoint{5.482896in}{2.882077in}}{\pgfqpoint{5.493495in}{2.877687in}}{\pgfqpoint{5.504545in}{2.877687in}}%
\pgfpathclose%
\pgfusepath{stroke,fill}%
\end{pgfscope}%
\begin{pgfscope}%
\pgfpathrectangle{\pgfqpoint{0.800000in}{0.528000in}}{\pgfqpoint{4.960000in}{3.696000in}}%
\pgfusepath{clip}%
\pgfsetbuttcap%
\pgfsetroundjoin%
\definecolor{currentfill}{rgb}{0.000000,0.000000,0.000000}%
\pgfsetfillcolor{currentfill}%
\pgfsetlinewidth{1.003750pt}%
\definecolor{currentstroke}{rgb}{0.000000,0.000000,0.000000}%
\pgfsetstrokecolor{currentstroke}%
\pgfsetdash{}{0pt}%
\pgfpathmoveto{\pgfqpoint{5.504545in}{2.877687in}}%
\pgfpathcurveto{\pgfqpoint{5.515596in}{2.877687in}}{\pgfqpoint{5.526195in}{2.882077in}}{\pgfqpoint{5.534008in}{2.889891in}}%
\pgfpathcurveto{\pgfqpoint{5.541822in}{2.897704in}}{\pgfqpoint{5.546212in}{2.908303in}}{\pgfqpoint{5.546212in}{2.919353in}}%
\pgfpathcurveto{\pgfqpoint{5.546212in}{2.930404in}}{\pgfqpoint{5.541822in}{2.941003in}}{\pgfqpoint{5.534008in}{2.948816in}}%
\pgfpathcurveto{\pgfqpoint{5.526195in}{2.956630in}}{\pgfqpoint{5.515596in}{2.961020in}}{\pgfqpoint{5.504545in}{2.961020in}}%
\pgfpathcurveto{\pgfqpoint{5.493495in}{2.961020in}}{\pgfqpoint{5.482896in}{2.956630in}}{\pgfqpoint{5.475083in}{2.948816in}}%
\pgfpathcurveto{\pgfqpoint{5.467269in}{2.941003in}}{\pgfqpoint{5.462879in}{2.930404in}}{\pgfqpoint{5.462879in}{2.919353in}}%
\pgfpathcurveto{\pgfqpoint{5.462879in}{2.908303in}}{\pgfqpoint{5.467269in}{2.897704in}}{\pgfqpoint{5.475083in}{2.889891in}}%
\pgfpathcurveto{\pgfqpoint{5.482896in}{2.882077in}}{\pgfqpoint{5.493495in}{2.877687in}}{\pgfqpoint{5.504545in}{2.877687in}}%
\pgfpathclose%
\pgfusepath{stroke,fill}%
\end{pgfscope}%
\begin{pgfscope}%
\pgfpathrectangle{\pgfqpoint{0.800000in}{0.528000in}}{\pgfqpoint{4.960000in}{3.696000in}}%
\pgfusepath{clip}%
\pgfsetbuttcap%
\pgfsetroundjoin%
\definecolor{currentfill}{rgb}{0.000000,0.000000,0.000000}%
\pgfsetfillcolor{currentfill}%
\pgfsetlinewidth{1.003750pt}%
\definecolor{currentstroke}{rgb}{0.000000,0.000000,0.000000}%
\pgfsetstrokecolor{currentstroke}%
\pgfsetdash{}{0pt}%
\pgfpathmoveto{\pgfqpoint{5.504545in}{2.877687in}}%
\pgfpathcurveto{\pgfqpoint{5.515596in}{2.877687in}}{\pgfqpoint{5.526195in}{2.882077in}}{\pgfqpoint{5.534008in}{2.889891in}}%
\pgfpathcurveto{\pgfqpoint{5.541822in}{2.897704in}}{\pgfqpoint{5.546212in}{2.908303in}}{\pgfqpoint{5.546212in}{2.919353in}}%
\pgfpathcurveto{\pgfqpoint{5.546212in}{2.930404in}}{\pgfqpoint{5.541822in}{2.941003in}}{\pgfqpoint{5.534008in}{2.948816in}}%
\pgfpathcurveto{\pgfqpoint{5.526195in}{2.956630in}}{\pgfqpoint{5.515596in}{2.961020in}}{\pgfqpoint{5.504545in}{2.961020in}}%
\pgfpathcurveto{\pgfqpoint{5.493495in}{2.961020in}}{\pgfqpoint{5.482896in}{2.956630in}}{\pgfqpoint{5.475083in}{2.948816in}}%
\pgfpathcurveto{\pgfqpoint{5.467269in}{2.941003in}}{\pgfqpoint{5.462879in}{2.930404in}}{\pgfqpoint{5.462879in}{2.919353in}}%
\pgfpathcurveto{\pgfqpoint{5.462879in}{2.908303in}}{\pgfqpoint{5.467269in}{2.897704in}}{\pgfqpoint{5.475083in}{2.889891in}}%
\pgfpathcurveto{\pgfqpoint{5.482896in}{2.882077in}}{\pgfqpoint{5.493495in}{2.877687in}}{\pgfqpoint{5.504545in}{2.877687in}}%
\pgfpathclose%
\pgfusepath{stroke,fill}%
\end{pgfscope}%
\begin{pgfscope}%
\pgfpathrectangle{\pgfqpoint{0.800000in}{0.528000in}}{\pgfqpoint{4.960000in}{3.696000in}}%
\pgfusepath{clip}%
\pgfsetbuttcap%
\pgfsetroundjoin%
\definecolor{currentfill}{rgb}{0.000000,0.000000,0.000000}%
\pgfsetfillcolor{currentfill}%
\pgfsetlinewidth{1.003750pt}%
\definecolor{currentstroke}{rgb}{0.000000,0.000000,0.000000}%
\pgfsetstrokecolor{currentstroke}%
\pgfsetdash{}{0pt}%
\pgfpathmoveto{\pgfqpoint{5.504545in}{2.877687in}}%
\pgfpathcurveto{\pgfqpoint{5.515596in}{2.877687in}}{\pgfqpoint{5.526195in}{2.882077in}}{\pgfqpoint{5.534008in}{2.889891in}}%
\pgfpathcurveto{\pgfqpoint{5.541822in}{2.897704in}}{\pgfqpoint{5.546212in}{2.908303in}}{\pgfqpoint{5.546212in}{2.919353in}}%
\pgfpathcurveto{\pgfqpoint{5.546212in}{2.930404in}}{\pgfqpoint{5.541822in}{2.941003in}}{\pgfqpoint{5.534008in}{2.948816in}}%
\pgfpathcurveto{\pgfqpoint{5.526195in}{2.956630in}}{\pgfqpoint{5.515596in}{2.961020in}}{\pgfqpoint{5.504545in}{2.961020in}}%
\pgfpathcurveto{\pgfqpoint{5.493495in}{2.961020in}}{\pgfqpoint{5.482896in}{2.956630in}}{\pgfqpoint{5.475083in}{2.948816in}}%
\pgfpathcurveto{\pgfqpoint{5.467269in}{2.941003in}}{\pgfqpoint{5.462879in}{2.930404in}}{\pgfqpoint{5.462879in}{2.919353in}}%
\pgfpathcurveto{\pgfqpoint{5.462879in}{2.908303in}}{\pgfqpoint{5.467269in}{2.897704in}}{\pgfqpoint{5.475083in}{2.889891in}}%
\pgfpathcurveto{\pgfqpoint{5.482896in}{2.882077in}}{\pgfqpoint{5.493495in}{2.877687in}}{\pgfqpoint{5.504545in}{2.877687in}}%
\pgfpathclose%
\pgfusepath{stroke,fill}%
\end{pgfscope}%
\begin{pgfscope}%
\pgfpathrectangle{\pgfqpoint{0.800000in}{0.528000in}}{\pgfqpoint{4.960000in}{3.696000in}}%
\pgfusepath{clip}%
\pgfsetbuttcap%
\pgfsetroundjoin%
\definecolor{currentfill}{rgb}{0.000000,0.000000,0.000000}%
\pgfsetfillcolor{currentfill}%
\pgfsetlinewidth{1.003750pt}%
\definecolor{currentstroke}{rgb}{0.000000,0.000000,0.000000}%
\pgfsetstrokecolor{currentstroke}%
\pgfsetdash{}{0pt}%
\pgfpathmoveto{\pgfqpoint{5.504545in}{2.877687in}}%
\pgfpathcurveto{\pgfqpoint{5.515596in}{2.877687in}}{\pgfqpoint{5.526195in}{2.882077in}}{\pgfqpoint{5.534008in}{2.889891in}}%
\pgfpathcurveto{\pgfqpoint{5.541822in}{2.897704in}}{\pgfqpoint{5.546212in}{2.908303in}}{\pgfqpoint{5.546212in}{2.919353in}}%
\pgfpathcurveto{\pgfqpoint{5.546212in}{2.930404in}}{\pgfqpoint{5.541822in}{2.941003in}}{\pgfqpoint{5.534008in}{2.948816in}}%
\pgfpathcurveto{\pgfqpoint{5.526195in}{2.956630in}}{\pgfqpoint{5.515596in}{2.961020in}}{\pgfqpoint{5.504545in}{2.961020in}}%
\pgfpathcurveto{\pgfqpoint{5.493495in}{2.961020in}}{\pgfqpoint{5.482896in}{2.956630in}}{\pgfqpoint{5.475083in}{2.948816in}}%
\pgfpathcurveto{\pgfqpoint{5.467269in}{2.941003in}}{\pgfqpoint{5.462879in}{2.930404in}}{\pgfqpoint{5.462879in}{2.919353in}}%
\pgfpathcurveto{\pgfqpoint{5.462879in}{2.908303in}}{\pgfqpoint{5.467269in}{2.897704in}}{\pgfqpoint{5.475083in}{2.889891in}}%
\pgfpathcurveto{\pgfqpoint{5.482896in}{2.882077in}}{\pgfqpoint{5.493495in}{2.877687in}}{\pgfqpoint{5.504545in}{2.877687in}}%
\pgfpathclose%
\pgfusepath{stroke,fill}%
\end{pgfscope}%
\begin{pgfscope}%
\pgfpathrectangle{\pgfqpoint{0.800000in}{0.528000in}}{\pgfqpoint{4.960000in}{3.696000in}}%
\pgfusepath{clip}%
\pgfsetbuttcap%
\pgfsetroundjoin%
\definecolor{currentfill}{rgb}{0.000000,0.000000,0.000000}%
\pgfsetfillcolor{currentfill}%
\pgfsetlinewidth{1.003750pt}%
\definecolor{currentstroke}{rgb}{0.000000,0.000000,0.000000}%
\pgfsetstrokecolor{currentstroke}%
\pgfsetdash{}{0pt}%
\pgfpathmoveto{\pgfqpoint{5.504545in}{2.877687in}}%
\pgfpathcurveto{\pgfqpoint{5.515596in}{2.877687in}}{\pgfqpoint{5.526195in}{2.882077in}}{\pgfqpoint{5.534008in}{2.889891in}}%
\pgfpathcurveto{\pgfqpoint{5.541822in}{2.897704in}}{\pgfqpoint{5.546212in}{2.908303in}}{\pgfqpoint{5.546212in}{2.919353in}}%
\pgfpathcurveto{\pgfqpoint{5.546212in}{2.930404in}}{\pgfqpoint{5.541822in}{2.941003in}}{\pgfqpoint{5.534008in}{2.948816in}}%
\pgfpathcurveto{\pgfqpoint{5.526195in}{2.956630in}}{\pgfqpoint{5.515596in}{2.961020in}}{\pgfqpoint{5.504545in}{2.961020in}}%
\pgfpathcurveto{\pgfqpoint{5.493495in}{2.961020in}}{\pgfqpoint{5.482896in}{2.956630in}}{\pgfqpoint{5.475083in}{2.948816in}}%
\pgfpathcurveto{\pgfqpoint{5.467269in}{2.941003in}}{\pgfqpoint{5.462879in}{2.930404in}}{\pgfqpoint{5.462879in}{2.919353in}}%
\pgfpathcurveto{\pgfqpoint{5.462879in}{2.908303in}}{\pgfqpoint{5.467269in}{2.897704in}}{\pgfqpoint{5.475083in}{2.889891in}}%
\pgfpathcurveto{\pgfqpoint{5.482896in}{2.882077in}}{\pgfqpoint{5.493495in}{2.877687in}}{\pgfqpoint{5.504545in}{2.877687in}}%
\pgfpathclose%
\pgfusepath{stroke,fill}%
\end{pgfscope}%
\begin{pgfscope}%
\pgfpathrectangle{\pgfqpoint{0.800000in}{0.528000in}}{\pgfqpoint{4.960000in}{3.696000in}}%
\pgfusepath{clip}%
\pgfsetbuttcap%
\pgfsetroundjoin%
\definecolor{currentfill}{rgb}{0.000000,0.000000,0.000000}%
\pgfsetfillcolor{currentfill}%
\pgfsetlinewidth{1.003750pt}%
\definecolor{currentstroke}{rgb}{0.000000,0.000000,0.000000}%
\pgfsetstrokecolor{currentstroke}%
\pgfsetdash{}{0pt}%
\pgfpathmoveto{\pgfqpoint{5.504545in}{2.877687in}}%
\pgfpathcurveto{\pgfqpoint{5.515596in}{2.877687in}}{\pgfqpoint{5.526195in}{2.882077in}}{\pgfqpoint{5.534008in}{2.889891in}}%
\pgfpathcurveto{\pgfqpoint{5.541822in}{2.897704in}}{\pgfqpoint{5.546212in}{2.908303in}}{\pgfqpoint{5.546212in}{2.919353in}}%
\pgfpathcurveto{\pgfqpoint{5.546212in}{2.930404in}}{\pgfqpoint{5.541822in}{2.941003in}}{\pgfqpoint{5.534008in}{2.948816in}}%
\pgfpathcurveto{\pgfqpoint{5.526195in}{2.956630in}}{\pgfqpoint{5.515596in}{2.961020in}}{\pgfqpoint{5.504545in}{2.961020in}}%
\pgfpathcurveto{\pgfqpoint{5.493495in}{2.961020in}}{\pgfqpoint{5.482896in}{2.956630in}}{\pgfqpoint{5.475083in}{2.948816in}}%
\pgfpathcurveto{\pgfqpoint{5.467269in}{2.941003in}}{\pgfqpoint{5.462879in}{2.930404in}}{\pgfqpoint{5.462879in}{2.919353in}}%
\pgfpathcurveto{\pgfqpoint{5.462879in}{2.908303in}}{\pgfqpoint{5.467269in}{2.897704in}}{\pgfqpoint{5.475083in}{2.889891in}}%
\pgfpathcurveto{\pgfqpoint{5.482896in}{2.882077in}}{\pgfqpoint{5.493495in}{2.877687in}}{\pgfqpoint{5.504545in}{2.877687in}}%
\pgfpathclose%
\pgfusepath{stroke,fill}%
\end{pgfscope}%
\begin{pgfscope}%
\pgfpathrectangle{\pgfqpoint{0.800000in}{0.528000in}}{\pgfqpoint{4.960000in}{3.696000in}}%
\pgfusepath{clip}%
\pgfsetbuttcap%
\pgfsetroundjoin%
\definecolor{currentfill}{rgb}{0.000000,0.000000,0.000000}%
\pgfsetfillcolor{currentfill}%
\pgfsetlinewidth{1.003750pt}%
\definecolor{currentstroke}{rgb}{0.000000,0.000000,0.000000}%
\pgfsetstrokecolor{currentstroke}%
\pgfsetdash{}{0pt}%
\pgfpathmoveto{\pgfqpoint{5.504545in}{2.877687in}}%
\pgfpathcurveto{\pgfqpoint{5.515596in}{2.877687in}}{\pgfqpoint{5.526195in}{2.882077in}}{\pgfqpoint{5.534008in}{2.889891in}}%
\pgfpathcurveto{\pgfqpoint{5.541822in}{2.897704in}}{\pgfqpoint{5.546212in}{2.908303in}}{\pgfqpoint{5.546212in}{2.919353in}}%
\pgfpathcurveto{\pgfqpoint{5.546212in}{2.930404in}}{\pgfqpoint{5.541822in}{2.941003in}}{\pgfqpoint{5.534008in}{2.948816in}}%
\pgfpathcurveto{\pgfqpoint{5.526195in}{2.956630in}}{\pgfqpoint{5.515596in}{2.961020in}}{\pgfqpoint{5.504545in}{2.961020in}}%
\pgfpathcurveto{\pgfqpoint{5.493495in}{2.961020in}}{\pgfqpoint{5.482896in}{2.956630in}}{\pgfqpoint{5.475083in}{2.948816in}}%
\pgfpathcurveto{\pgfqpoint{5.467269in}{2.941003in}}{\pgfqpoint{5.462879in}{2.930404in}}{\pgfqpoint{5.462879in}{2.919353in}}%
\pgfpathcurveto{\pgfqpoint{5.462879in}{2.908303in}}{\pgfqpoint{5.467269in}{2.897704in}}{\pgfqpoint{5.475083in}{2.889891in}}%
\pgfpathcurveto{\pgfqpoint{5.482896in}{2.882077in}}{\pgfqpoint{5.493495in}{2.877687in}}{\pgfqpoint{5.504545in}{2.877687in}}%
\pgfpathclose%
\pgfusepath{stroke,fill}%
\end{pgfscope}%
\begin{pgfscope}%
\pgfpathrectangle{\pgfqpoint{0.800000in}{0.528000in}}{\pgfqpoint{4.960000in}{3.696000in}}%
\pgfusepath{clip}%
\pgfsetbuttcap%
\pgfsetroundjoin%
\definecolor{currentfill}{rgb}{0.000000,0.000000,0.000000}%
\pgfsetfillcolor{currentfill}%
\pgfsetlinewidth{1.003750pt}%
\definecolor{currentstroke}{rgb}{0.000000,0.000000,0.000000}%
\pgfsetstrokecolor{currentstroke}%
\pgfsetdash{}{0pt}%
\pgfpathmoveto{\pgfqpoint{5.504545in}{2.877687in}}%
\pgfpathcurveto{\pgfqpoint{5.515596in}{2.877687in}}{\pgfqpoint{5.526195in}{2.882077in}}{\pgfqpoint{5.534008in}{2.889891in}}%
\pgfpathcurveto{\pgfqpoint{5.541822in}{2.897704in}}{\pgfqpoint{5.546212in}{2.908303in}}{\pgfqpoint{5.546212in}{2.919353in}}%
\pgfpathcurveto{\pgfqpoint{5.546212in}{2.930404in}}{\pgfqpoint{5.541822in}{2.941003in}}{\pgfqpoint{5.534008in}{2.948816in}}%
\pgfpathcurveto{\pgfqpoint{5.526195in}{2.956630in}}{\pgfqpoint{5.515596in}{2.961020in}}{\pgfqpoint{5.504545in}{2.961020in}}%
\pgfpathcurveto{\pgfqpoint{5.493495in}{2.961020in}}{\pgfqpoint{5.482896in}{2.956630in}}{\pgfqpoint{5.475083in}{2.948816in}}%
\pgfpathcurveto{\pgfqpoint{5.467269in}{2.941003in}}{\pgfqpoint{5.462879in}{2.930404in}}{\pgfqpoint{5.462879in}{2.919353in}}%
\pgfpathcurveto{\pgfqpoint{5.462879in}{2.908303in}}{\pgfqpoint{5.467269in}{2.897704in}}{\pgfqpoint{5.475083in}{2.889891in}}%
\pgfpathcurveto{\pgfqpoint{5.482896in}{2.882077in}}{\pgfqpoint{5.493495in}{2.877687in}}{\pgfqpoint{5.504545in}{2.877687in}}%
\pgfpathclose%
\pgfusepath{stroke,fill}%
\end{pgfscope}%
\begin{pgfscope}%
\pgfpathrectangle{\pgfqpoint{0.800000in}{0.528000in}}{\pgfqpoint{4.960000in}{3.696000in}}%
\pgfusepath{clip}%
\pgfsetbuttcap%
\pgfsetroundjoin%
\definecolor{currentfill}{rgb}{0.000000,0.000000,0.000000}%
\pgfsetfillcolor{currentfill}%
\pgfsetlinewidth{1.003750pt}%
\definecolor{currentstroke}{rgb}{0.000000,0.000000,0.000000}%
\pgfsetstrokecolor{currentstroke}%
\pgfsetdash{}{0pt}%
\pgfpathmoveto{\pgfqpoint{5.504545in}{2.877687in}}%
\pgfpathcurveto{\pgfqpoint{5.515596in}{2.877687in}}{\pgfqpoint{5.526195in}{2.882077in}}{\pgfqpoint{5.534008in}{2.889891in}}%
\pgfpathcurveto{\pgfqpoint{5.541822in}{2.897704in}}{\pgfqpoint{5.546212in}{2.908303in}}{\pgfqpoint{5.546212in}{2.919353in}}%
\pgfpathcurveto{\pgfqpoint{5.546212in}{2.930404in}}{\pgfqpoint{5.541822in}{2.941003in}}{\pgfqpoint{5.534008in}{2.948816in}}%
\pgfpathcurveto{\pgfqpoint{5.526195in}{2.956630in}}{\pgfqpoint{5.515596in}{2.961020in}}{\pgfqpoint{5.504545in}{2.961020in}}%
\pgfpathcurveto{\pgfqpoint{5.493495in}{2.961020in}}{\pgfqpoint{5.482896in}{2.956630in}}{\pgfqpoint{5.475083in}{2.948816in}}%
\pgfpathcurveto{\pgfqpoint{5.467269in}{2.941003in}}{\pgfqpoint{5.462879in}{2.930404in}}{\pgfqpoint{5.462879in}{2.919353in}}%
\pgfpathcurveto{\pgfqpoint{5.462879in}{2.908303in}}{\pgfqpoint{5.467269in}{2.897704in}}{\pgfqpoint{5.475083in}{2.889891in}}%
\pgfpathcurveto{\pgfqpoint{5.482896in}{2.882077in}}{\pgfqpoint{5.493495in}{2.877687in}}{\pgfqpoint{5.504545in}{2.877687in}}%
\pgfpathclose%
\pgfusepath{stroke,fill}%
\end{pgfscope}%
\begin{pgfscope}%
\pgfpathrectangle{\pgfqpoint{0.800000in}{0.528000in}}{\pgfqpoint{4.960000in}{3.696000in}}%
\pgfusepath{clip}%
\pgfsetbuttcap%
\pgfsetroundjoin%
\definecolor{currentfill}{rgb}{0.000000,0.000000,0.000000}%
\pgfsetfillcolor{currentfill}%
\pgfsetlinewidth{1.003750pt}%
\definecolor{currentstroke}{rgb}{0.000000,0.000000,0.000000}%
\pgfsetstrokecolor{currentstroke}%
\pgfsetdash{}{0pt}%
\pgfpathmoveto{\pgfqpoint{5.504545in}{2.877687in}}%
\pgfpathcurveto{\pgfqpoint{5.515596in}{2.877687in}}{\pgfqpoint{5.526195in}{2.882077in}}{\pgfqpoint{5.534008in}{2.889891in}}%
\pgfpathcurveto{\pgfqpoint{5.541822in}{2.897704in}}{\pgfqpoint{5.546212in}{2.908303in}}{\pgfqpoint{5.546212in}{2.919353in}}%
\pgfpathcurveto{\pgfqpoint{5.546212in}{2.930404in}}{\pgfqpoint{5.541822in}{2.941003in}}{\pgfqpoint{5.534008in}{2.948816in}}%
\pgfpathcurveto{\pgfqpoint{5.526195in}{2.956630in}}{\pgfqpoint{5.515596in}{2.961020in}}{\pgfqpoint{5.504545in}{2.961020in}}%
\pgfpathcurveto{\pgfqpoint{5.493495in}{2.961020in}}{\pgfqpoint{5.482896in}{2.956630in}}{\pgfqpoint{5.475083in}{2.948816in}}%
\pgfpathcurveto{\pgfqpoint{5.467269in}{2.941003in}}{\pgfqpoint{5.462879in}{2.930404in}}{\pgfqpoint{5.462879in}{2.919353in}}%
\pgfpathcurveto{\pgfqpoint{5.462879in}{2.908303in}}{\pgfqpoint{5.467269in}{2.897704in}}{\pgfqpoint{5.475083in}{2.889891in}}%
\pgfpathcurveto{\pgfqpoint{5.482896in}{2.882077in}}{\pgfqpoint{5.493495in}{2.877687in}}{\pgfqpoint{5.504545in}{2.877687in}}%
\pgfpathclose%
\pgfusepath{stroke,fill}%
\end{pgfscope}%
\begin{pgfscope}%
\pgfpathrectangle{\pgfqpoint{0.800000in}{0.528000in}}{\pgfqpoint{4.960000in}{3.696000in}}%
\pgfusepath{clip}%
\pgfsetbuttcap%
\pgfsetroundjoin%
\definecolor{currentfill}{rgb}{0.000000,0.000000,0.000000}%
\pgfsetfillcolor{currentfill}%
\pgfsetlinewidth{1.003750pt}%
\definecolor{currentstroke}{rgb}{0.000000,0.000000,0.000000}%
\pgfsetstrokecolor{currentstroke}%
\pgfsetdash{}{0pt}%
\pgfpathmoveto{\pgfqpoint{5.504545in}{2.877687in}}%
\pgfpathcurveto{\pgfqpoint{5.515596in}{2.877687in}}{\pgfqpoint{5.526195in}{2.882077in}}{\pgfqpoint{5.534008in}{2.889891in}}%
\pgfpathcurveto{\pgfqpoint{5.541822in}{2.897704in}}{\pgfqpoint{5.546212in}{2.908303in}}{\pgfqpoint{5.546212in}{2.919353in}}%
\pgfpathcurveto{\pgfqpoint{5.546212in}{2.930404in}}{\pgfqpoint{5.541822in}{2.941003in}}{\pgfqpoint{5.534008in}{2.948816in}}%
\pgfpathcurveto{\pgfqpoint{5.526195in}{2.956630in}}{\pgfqpoint{5.515596in}{2.961020in}}{\pgfqpoint{5.504545in}{2.961020in}}%
\pgfpathcurveto{\pgfqpoint{5.493495in}{2.961020in}}{\pgfqpoint{5.482896in}{2.956630in}}{\pgfqpoint{5.475083in}{2.948816in}}%
\pgfpathcurveto{\pgfqpoint{5.467269in}{2.941003in}}{\pgfqpoint{5.462879in}{2.930404in}}{\pgfqpoint{5.462879in}{2.919353in}}%
\pgfpathcurveto{\pgfqpoint{5.462879in}{2.908303in}}{\pgfqpoint{5.467269in}{2.897704in}}{\pgfqpoint{5.475083in}{2.889891in}}%
\pgfpathcurveto{\pgfqpoint{5.482896in}{2.882077in}}{\pgfqpoint{5.493495in}{2.877687in}}{\pgfqpoint{5.504545in}{2.877687in}}%
\pgfpathclose%
\pgfusepath{stroke,fill}%
\end{pgfscope}%
\begin{pgfscope}%
\pgfpathrectangle{\pgfqpoint{0.800000in}{0.528000in}}{\pgfqpoint{4.960000in}{3.696000in}}%
\pgfusepath{clip}%
\pgfsetbuttcap%
\pgfsetroundjoin%
\definecolor{currentfill}{rgb}{0.000000,0.000000,0.000000}%
\pgfsetfillcolor{currentfill}%
\pgfsetlinewidth{1.003750pt}%
\definecolor{currentstroke}{rgb}{0.000000,0.000000,0.000000}%
\pgfsetstrokecolor{currentstroke}%
\pgfsetdash{}{0pt}%
\pgfpathmoveto{\pgfqpoint{5.504545in}{2.877687in}}%
\pgfpathcurveto{\pgfqpoint{5.515596in}{2.877687in}}{\pgfqpoint{5.526195in}{2.882077in}}{\pgfqpoint{5.534008in}{2.889891in}}%
\pgfpathcurveto{\pgfqpoint{5.541822in}{2.897704in}}{\pgfqpoint{5.546212in}{2.908303in}}{\pgfqpoint{5.546212in}{2.919353in}}%
\pgfpathcurveto{\pgfqpoint{5.546212in}{2.930404in}}{\pgfqpoint{5.541822in}{2.941003in}}{\pgfqpoint{5.534008in}{2.948816in}}%
\pgfpathcurveto{\pgfqpoint{5.526195in}{2.956630in}}{\pgfqpoint{5.515596in}{2.961020in}}{\pgfqpoint{5.504545in}{2.961020in}}%
\pgfpathcurveto{\pgfqpoint{5.493495in}{2.961020in}}{\pgfqpoint{5.482896in}{2.956630in}}{\pgfqpoint{5.475083in}{2.948816in}}%
\pgfpathcurveto{\pgfqpoint{5.467269in}{2.941003in}}{\pgfqpoint{5.462879in}{2.930404in}}{\pgfqpoint{5.462879in}{2.919353in}}%
\pgfpathcurveto{\pgfqpoint{5.462879in}{2.908303in}}{\pgfqpoint{5.467269in}{2.897704in}}{\pgfqpoint{5.475083in}{2.889891in}}%
\pgfpathcurveto{\pgfqpoint{5.482896in}{2.882077in}}{\pgfqpoint{5.493495in}{2.877687in}}{\pgfqpoint{5.504545in}{2.877687in}}%
\pgfpathclose%
\pgfusepath{stroke,fill}%
\end{pgfscope}%
\begin{pgfscope}%
\pgfpathrectangle{\pgfqpoint{0.800000in}{0.528000in}}{\pgfqpoint{4.960000in}{3.696000in}}%
\pgfusepath{clip}%
\pgfsetbuttcap%
\pgfsetroundjoin%
\definecolor{currentfill}{rgb}{0.000000,0.000000,0.000000}%
\pgfsetfillcolor{currentfill}%
\pgfsetlinewidth{1.003750pt}%
\definecolor{currentstroke}{rgb}{0.000000,0.000000,0.000000}%
\pgfsetstrokecolor{currentstroke}%
\pgfsetdash{}{0pt}%
\pgfpathmoveto{\pgfqpoint{5.504545in}{2.877687in}}%
\pgfpathcurveto{\pgfqpoint{5.515596in}{2.877687in}}{\pgfqpoint{5.526195in}{2.882077in}}{\pgfqpoint{5.534008in}{2.889891in}}%
\pgfpathcurveto{\pgfqpoint{5.541822in}{2.897704in}}{\pgfqpoint{5.546212in}{2.908303in}}{\pgfqpoint{5.546212in}{2.919353in}}%
\pgfpathcurveto{\pgfqpoint{5.546212in}{2.930404in}}{\pgfqpoint{5.541822in}{2.941003in}}{\pgfqpoint{5.534008in}{2.948816in}}%
\pgfpathcurveto{\pgfqpoint{5.526195in}{2.956630in}}{\pgfqpoint{5.515596in}{2.961020in}}{\pgfqpoint{5.504545in}{2.961020in}}%
\pgfpathcurveto{\pgfqpoint{5.493495in}{2.961020in}}{\pgfqpoint{5.482896in}{2.956630in}}{\pgfqpoint{5.475083in}{2.948816in}}%
\pgfpathcurveto{\pgfqpoint{5.467269in}{2.941003in}}{\pgfqpoint{5.462879in}{2.930404in}}{\pgfqpoint{5.462879in}{2.919353in}}%
\pgfpathcurveto{\pgfqpoint{5.462879in}{2.908303in}}{\pgfqpoint{5.467269in}{2.897704in}}{\pgfqpoint{5.475083in}{2.889891in}}%
\pgfpathcurveto{\pgfqpoint{5.482896in}{2.882077in}}{\pgfqpoint{5.493495in}{2.877687in}}{\pgfqpoint{5.504545in}{2.877687in}}%
\pgfpathclose%
\pgfusepath{stroke,fill}%
\end{pgfscope}%
\begin{pgfscope}%
\pgfpathrectangle{\pgfqpoint{0.800000in}{0.528000in}}{\pgfqpoint{4.960000in}{3.696000in}}%
\pgfusepath{clip}%
\pgfsetbuttcap%
\pgfsetroundjoin%
\definecolor{currentfill}{rgb}{0.000000,0.000000,0.000000}%
\pgfsetfillcolor{currentfill}%
\pgfsetlinewidth{1.003750pt}%
\definecolor{currentstroke}{rgb}{0.000000,0.000000,0.000000}%
\pgfsetstrokecolor{currentstroke}%
\pgfsetdash{}{0pt}%
\pgfpathmoveto{\pgfqpoint{5.504545in}{3.984333in}}%
\pgfpathcurveto{\pgfqpoint{5.515596in}{3.984333in}}{\pgfqpoint{5.526195in}{3.988724in}}{\pgfqpoint{5.534008in}{3.996537in}}%
\pgfpathcurveto{\pgfqpoint{5.541822in}{4.004351in}}{\pgfqpoint{5.546212in}{4.014950in}}{\pgfqpoint{5.546212in}{4.026000in}}%
\pgfpathcurveto{\pgfqpoint{5.546212in}{4.037050in}}{\pgfqpoint{5.541822in}{4.047649in}}{\pgfqpoint{5.534008in}{4.055463in}}%
\pgfpathcurveto{\pgfqpoint{5.526195in}{4.063276in}}{\pgfqpoint{5.515596in}{4.067667in}}{\pgfqpoint{5.504545in}{4.067667in}}%
\pgfpathcurveto{\pgfqpoint{5.493495in}{4.067667in}}{\pgfqpoint{5.482896in}{4.063276in}}{\pgfqpoint{5.475083in}{4.055463in}}%
\pgfpathcurveto{\pgfqpoint{5.467269in}{4.047649in}}{\pgfqpoint{5.462879in}{4.037050in}}{\pgfqpoint{5.462879in}{4.026000in}}%
\pgfpathcurveto{\pgfqpoint{5.462879in}{4.014950in}}{\pgfqpoint{5.467269in}{4.004351in}}{\pgfqpoint{5.475083in}{3.996537in}}%
\pgfpathcurveto{\pgfqpoint{5.482896in}{3.988724in}}{\pgfqpoint{5.493495in}{3.984333in}}{\pgfqpoint{5.504545in}{3.984333in}}%
\pgfpathclose%
\pgfusepath{stroke,fill}%
\end{pgfscope}%
\begin{pgfscope}%
\pgfpathrectangle{\pgfqpoint{0.800000in}{0.528000in}}{\pgfqpoint{4.960000in}{3.696000in}}%
\pgfusepath{clip}%
\pgfsetbuttcap%
\pgfsetroundjoin%
\definecolor{currentfill}{rgb}{0.000000,0.000000,0.000000}%
\pgfsetfillcolor{currentfill}%
\pgfsetlinewidth{1.003750pt}%
\definecolor{currentstroke}{rgb}{0.000000,0.000000,0.000000}%
\pgfsetstrokecolor{currentstroke}%
\pgfsetdash{}{0pt}%
\pgfpathmoveto{\pgfqpoint{5.504545in}{2.877687in}}%
\pgfpathcurveto{\pgfqpoint{5.515596in}{2.877687in}}{\pgfqpoint{5.526195in}{2.882077in}}{\pgfqpoint{5.534008in}{2.889891in}}%
\pgfpathcurveto{\pgfqpoint{5.541822in}{2.897704in}}{\pgfqpoint{5.546212in}{2.908303in}}{\pgfqpoint{5.546212in}{2.919353in}}%
\pgfpathcurveto{\pgfqpoint{5.546212in}{2.930404in}}{\pgfqpoint{5.541822in}{2.941003in}}{\pgfqpoint{5.534008in}{2.948816in}}%
\pgfpathcurveto{\pgfqpoint{5.526195in}{2.956630in}}{\pgfqpoint{5.515596in}{2.961020in}}{\pgfqpoint{5.504545in}{2.961020in}}%
\pgfpathcurveto{\pgfqpoint{5.493495in}{2.961020in}}{\pgfqpoint{5.482896in}{2.956630in}}{\pgfqpoint{5.475083in}{2.948816in}}%
\pgfpathcurveto{\pgfqpoint{5.467269in}{2.941003in}}{\pgfqpoint{5.462879in}{2.930404in}}{\pgfqpoint{5.462879in}{2.919353in}}%
\pgfpathcurveto{\pgfqpoint{5.462879in}{2.908303in}}{\pgfqpoint{5.467269in}{2.897704in}}{\pgfqpoint{5.475083in}{2.889891in}}%
\pgfpathcurveto{\pgfqpoint{5.482896in}{2.882077in}}{\pgfqpoint{5.493495in}{2.877687in}}{\pgfqpoint{5.504545in}{2.877687in}}%
\pgfpathclose%
\pgfusepath{stroke,fill}%
\end{pgfscope}%
\begin{pgfscope}%
\pgfpathrectangle{\pgfqpoint{0.800000in}{0.528000in}}{\pgfqpoint{4.960000in}{3.696000in}}%
\pgfusepath{clip}%
\pgfsetbuttcap%
\pgfsetroundjoin%
\definecolor{currentfill}{rgb}{0.000000,0.000000,0.000000}%
\pgfsetfillcolor{currentfill}%
\pgfsetlinewidth{1.003750pt}%
\definecolor{currentstroke}{rgb}{0.000000,0.000000,0.000000}%
\pgfsetstrokecolor{currentstroke}%
\pgfsetdash{}{0pt}%
\pgfpathmoveto{\pgfqpoint{5.504545in}{2.877687in}}%
\pgfpathcurveto{\pgfqpoint{5.515596in}{2.877687in}}{\pgfqpoint{5.526195in}{2.882077in}}{\pgfqpoint{5.534008in}{2.889891in}}%
\pgfpathcurveto{\pgfqpoint{5.541822in}{2.897704in}}{\pgfqpoint{5.546212in}{2.908303in}}{\pgfqpoint{5.546212in}{2.919353in}}%
\pgfpathcurveto{\pgfqpoint{5.546212in}{2.930404in}}{\pgfqpoint{5.541822in}{2.941003in}}{\pgfqpoint{5.534008in}{2.948816in}}%
\pgfpathcurveto{\pgfqpoint{5.526195in}{2.956630in}}{\pgfqpoint{5.515596in}{2.961020in}}{\pgfqpoint{5.504545in}{2.961020in}}%
\pgfpathcurveto{\pgfqpoint{5.493495in}{2.961020in}}{\pgfqpoint{5.482896in}{2.956630in}}{\pgfqpoint{5.475083in}{2.948816in}}%
\pgfpathcurveto{\pgfqpoint{5.467269in}{2.941003in}}{\pgfqpoint{5.462879in}{2.930404in}}{\pgfqpoint{5.462879in}{2.919353in}}%
\pgfpathcurveto{\pgfqpoint{5.462879in}{2.908303in}}{\pgfqpoint{5.467269in}{2.897704in}}{\pgfqpoint{5.475083in}{2.889891in}}%
\pgfpathcurveto{\pgfqpoint{5.482896in}{2.882077in}}{\pgfqpoint{5.493495in}{2.877687in}}{\pgfqpoint{5.504545in}{2.877687in}}%
\pgfpathclose%
\pgfusepath{stroke,fill}%
\end{pgfscope}%
\begin{pgfscope}%
\pgfpathrectangle{\pgfqpoint{0.800000in}{0.528000in}}{\pgfqpoint{4.960000in}{3.696000in}}%
\pgfusepath{clip}%
\pgfsetbuttcap%
\pgfsetroundjoin%
\definecolor{currentfill}{rgb}{0.000000,0.000000,0.000000}%
\pgfsetfillcolor{currentfill}%
\pgfsetlinewidth{1.003750pt}%
\definecolor{currentstroke}{rgb}{0.000000,0.000000,0.000000}%
\pgfsetstrokecolor{currentstroke}%
\pgfsetdash{}{0pt}%
\pgfpathmoveto{\pgfqpoint{5.504545in}{2.877687in}}%
\pgfpathcurveto{\pgfqpoint{5.515596in}{2.877687in}}{\pgfqpoint{5.526195in}{2.882077in}}{\pgfqpoint{5.534008in}{2.889891in}}%
\pgfpathcurveto{\pgfqpoint{5.541822in}{2.897704in}}{\pgfqpoint{5.546212in}{2.908303in}}{\pgfqpoint{5.546212in}{2.919353in}}%
\pgfpathcurveto{\pgfqpoint{5.546212in}{2.930404in}}{\pgfqpoint{5.541822in}{2.941003in}}{\pgfqpoint{5.534008in}{2.948816in}}%
\pgfpathcurveto{\pgfqpoint{5.526195in}{2.956630in}}{\pgfqpoint{5.515596in}{2.961020in}}{\pgfqpoint{5.504545in}{2.961020in}}%
\pgfpathcurveto{\pgfqpoint{5.493495in}{2.961020in}}{\pgfqpoint{5.482896in}{2.956630in}}{\pgfqpoint{5.475083in}{2.948816in}}%
\pgfpathcurveto{\pgfqpoint{5.467269in}{2.941003in}}{\pgfqpoint{5.462879in}{2.930404in}}{\pgfqpoint{5.462879in}{2.919353in}}%
\pgfpathcurveto{\pgfqpoint{5.462879in}{2.908303in}}{\pgfqpoint{5.467269in}{2.897704in}}{\pgfqpoint{5.475083in}{2.889891in}}%
\pgfpathcurveto{\pgfqpoint{5.482896in}{2.882077in}}{\pgfqpoint{5.493495in}{2.877687in}}{\pgfqpoint{5.504545in}{2.877687in}}%
\pgfpathclose%
\pgfusepath{stroke,fill}%
\end{pgfscope}%
\begin{pgfscope}%
\pgfpathrectangle{\pgfqpoint{0.800000in}{0.528000in}}{\pgfqpoint{4.960000in}{3.696000in}}%
\pgfusepath{clip}%
\pgfsetbuttcap%
\pgfsetroundjoin%
\definecolor{currentfill}{rgb}{0.000000,0.000000,0.000000}%
\pgfsetfillcolor{currentfill}%
\pgfsetlinewidth{1.003750pt}%
\definecolor{currentstroke}{rgb}{0.000000,0.000000,0.000000}%
\pgfsetstrokecolor{currentstroke}%
\pgfsetdash{}{0pt}%
\pgfpathmoveto{\pgfqpoint{5.504545in}{2.877687in}}%
\pgfpathcurveto{\pgfqpoint{5.515596in}{2.877687in}}{\pgfqpoint{5.526195in}{2.882077in}}{\pgfqpoint{5.534008in}{2.889891in}}%
\pgfpathcurveto{\pgfqpoint{5.541822in}{2.897704in}}{\pgfqpoint{5.546212in}{2.908303in}}{\pgfqpoint{5.546212in}{2.919353in}}%
\pgfpathcurveto{\pgfqpoint{5.546212in}{2.930404in}}{\pgfqpoint{5.541822in}{2.941003in}}{\pgfqpoint{5.534008in}{2.948816in}}%
\pgfpathcurveto{\pgfqpoint{5.526195in}{2.956630in}}{\pgfqpoint{5.515596in}{2.961020in}}{\pgfqpoint{5.504545in}{2.961020in}}%
\pgfpathcurveto{\pgfqpoint{5.493495in}{2.961020in}}{\pgfqpoint{5.482896in}{2.956630in}}{\pgfqpoint{5.475083in}{2.948816in}}%
\pgfpathcurveto{\pgfqpoint{5.467269in}{2.941003in}}{\pgfqpoint{5.462879in}{2.930404in}}{\pgfqpoint{5.462879in}{2.919353in}}%
\pgfpathcurveto{\pgfqpoint{5.462879in}{2.908303in}}{\pgfqpoint{5.467269in}{2.897704in}}{\pgfqpoint{5.475083in}{2.889891in}}%
\pgfpathcurveto{\pgfqpoint{5.482896in}{2.882077in}}{\pgfqpoint{5.493495in}{2.877687in}}{\pgfqpoint{5.504545in}{2.877687in}}%
\pgfpathclose%
\pgfusepath{stroke,fill}%
\end{pgfscope}%
\begin{pgfscope}%
\pgfpathrectangle{\pgfqpoint{0.800000in}{0.528000in}}{\pgfqpoint{4.960000in}{3.696000in}}%
\pgfusepath{clip}%
\pgfsetbuttcap%
\pgfsetroundjoin%
\definecolor{currentfill}{rgb}{0.000000,0.000000,0.000000}%
\pgfsetfillcolor{currentfill}%
\pgfsetlinewidth{1.003750pt}%
\definecolor{currentstroke}{rgb}{0.000000,0.000000,0.000000}%
\pgfsetstrokecolor{currentstroke}%
\pgfsetdash{}{0pt}%
\pgfpathmoveto{\pgfqpoint{5.504545in}{2.877687in}}%
\pgfpathcurveto{\pgfqpoint{5.515596in}{2.877687in}}{\pgfqpoint{5.526195in}{2.882077in}}{\pgfqpoint{5.534008in}{2.889891in}}%
\pgfpathcurveto{\pgfqpoint{5.541822in}{2.897704in}}{\pgfqpoint{5.546212in}{2.908303in}}{\pgfqpoint{5.546212in}{2.919353in}}%
\pgfpathcurveto{\pgfqpoint{5.546212in}{2.930404in}}{\pgfqpoint{5.541822in}{2.941003in}}{\pgfqpoint{5.534008in}{2.948816in}}%
\pgfpathcurveto{\pgfqpoint{5.526195in}{2.956630in}}{\pgfqpoint{5.515596in}{2.961020in}}{\pgfqpoint{5.504545in}{2.961020in}}%
\pgfpathcurveto{\pgfqpoint{5.493495in}{2.961020in}}{\pgfqpoint{5.482896in}{2.956630in}}{\pgfqpoint{5.475083in}{2.948816in}}%
\pgfpathcurveto{\pgfqpoint{5.467269in}{2.941003in}}{\pgfqpoint{5.462879in}{2.930404in}}{\pgfqpoint{5.462879in}{2.919353in}}%
\pgfpathcurveto{\pgfqpoint{5.462879in}{2.908303in}}{\pgfqpoint{5.467269in}{2.897704in}}{\pgfqpoint{5.475083in}{2.889891in}}%
\pgfpathcurveto{\pgfqpoint{5.482896in}{2.882077in}}{\pgfqpoint{5.493495in}{2.877687in}}{\pgfqpoint{5.504545in}{2.877687in}}%
\pgfpathclose%
\pgfusepath{stroke,fill}%
\end{pgfscope}%
\begin{pgfscope}%
\pgfpathrectangle{\pgfqpoint{0.800000in}{0.528000in}}{\pgfqpoint{4.960000in}{3.696000in}}%
\pgfusepath{clip}%
\pgfsetbuttcap%
\pgfsetroundjoin%
\definecolor{currentfill}{rgb}{0.000000,0.000000,0.000000}%
\pgfsetfillcolor{currentfill}%
\pgfsetlinewidth{1.003750pt}%
\definecolor{currentstroke}{rgb}{0.000000,0.000000,0.000000}%
\pgfsetstrokecolor{currentstroke}%
\pgfsetdash{}{0pt}%
\pgfpathmoveto{\pgfqpoint{5.504545in}{2.877687in}}%
\pgfpathcurveto{\pgfqpoint{5.515596in}{2.877687in}}{\pgfqpoint{5.526195in}{2.882077in}}{\pgfqpoint{5.534008in}{2.889891in}}%
\pgfpathcurveto{\pgfqpoint{5.541822in}{2.897704in}}{\pgfqpoint{5.546212in}{2.908303in}}{\pgfqpoint{5.546212in}{2.919353in}}%
\pgfpathcurveto{\pgfqpoint{5.546212in}{2.930404in}}{\pgfqpoint{5.541822in}{2.941003in}}{\pgfqpoint{5.534008in}{2.948816in}}%
\pgfpathcurveto{\pgfqpoint{5.526195in}{2.956630in}}{\pgfqpoint{5.515596in}{2.961020in}}{\pgfqpoint{5.504545in}{2.961020in}}%
\pgfpathcurveto{\pgfqpoint{5.493495in}{2.961020in}}{\pgfqpoint{5.482896in}{2.956630in}}{\pgfqpoint{5.475083in}{2.948816in}}%
\pgfpathcurveto{\pgfqpoint{5.467269in}{2.941003in}}{\pgfqpoint{5.462879in}{2.930404in}}{\pgfqpoint{5.462879in}{2.919353in}}%
\pgfpathcurveto{\pgfqpoint{5.462879in}{2.908303in}}{\pgfqpoint{5.467269in}{2.897704in}}{\pgfqpoint{5.475083in}{2.889891in}}%
\pgfpathcurveto{\pgfqpoint{5.482896in}{2.882077in}}{\pgfqpoint{5.493495in}{2.877687in}}{\pgfqpoint{5.504545in}{2.877687in}}%
\pgfpathclose%
\pgfusepath{stroke,fill}%
\end{pgfscope}%
\begin{pgfscope}%
\pgfpathrectangle{\pgfqpoint{0.800000in}{0.528000in}}{\pgfqpoint{4.960000in}{3.696000in}}%
\pgfusepath{clip}%
\pgfsetbuttcap%
\pgfsetroundjoin%
\definecolor{currentfill}{rgb}{0.000000,0.000000,0.000000}%
\pgfsetfillcolor{currentfill}%
\pgfsetlinewidth{1.003750pt}%
\definecolor{currentstroke}{rgb}{0.000000,0.000000,0.000000}%
\pgfsetstrokecolor{currentstroke}%
\pgfsetdash{}{0pt}%
\pgfpathmoveto{\pgfqpoint{5.504545in}{2.877687in}}%
\pgfpathcurveto{\pgfqpoint{5.515596in}{2.877687in}}{\pgfqpoint{5.526195in}{2.882077in}}{\pgfqpoint{5.534008in}{2.889891in}}%
\pgfpathcurveto{\pgfqpoint{5.541822in}{2.897704in}}{\pgfqpoint{5.546212in}{2.908303in}}{\pgfqpoint{5.546212in}{2.919353in}}%
\pgfpathcurveto{\pgfqpoint{5.546212in}{2.930404in}}{\pgfqpoint{5.541822in}{2.941003in}}{\pgfqpoint{5.534008in}{2.948816in}}%
\pgfpathcurveto{\pgfqpoint{5.526195in}{2.956630in}}{\pgfqpoint{5.515596in}{2.961020in}}{\pgfqpoint{5.504545in}{2.961020in}}%
\pgfpathcurveto{\pgfqpoint{5.493495in}{2.961020in}}{\pgfqpoint{5.482896in}{2.956630in}}{\pgfqpoint{5.475083in}{2.948816in}}%
\pgfpathcurveto{\pgfqpoint{5.467269in}{2.941003in}}{\pgfqpoint{5.462879in}{2.930404in}}{\pgfqpoint{5.462879in}{2.919353in}}%
\pgfpathcurveto{\pgfqpoint{5.462879in}{2.908303in}}{\pgfqpoint{5.467269in}{2.897704in}}{\pgfqpoint{5.475083in}{2.889891in}}%
\pgfpathcurveto{\pgfqpoint{5.482896in}{2.882077in}}{\pgfqpoint{5.493495in}{2.877687in}}{\pgfqpoint{5.504545in}{2.877687in}}%
\pgfpathclose%
\pgfusepath{stroke,fill}%
\end{pgfscope}%
\begin{pgfscope}%
\pgfpathrectangle{\pgfqpoint{0.800000in}{0.528000in}}{\pgfqpoint{4.960000in}{3.696000in}}%
\pgfusepath{clip}%
\pgfsetbuttcap%
\pgfsetroundjoin%
\definecolor{currentfill}{rgb}{0.000000,0.000000,0.000000}%
\pgfsetfillcolor{currentfill}%
\pgfsetlinewidth{1.003750pt}%
\definecolor{currentstroke}{rgb}{0.000000,0.000000,0.000000}%
\pgfsetstrokecolor{currentstroke}%
\pgfsetdash{}{0pt}%
\pgfpathmoveto{\pgfqpoint{5.504545in}{2.877687in}}%
\pgfpathcurveto{\pgfqpoint{5.515596in}{2.877687in}}{\pgfqpoint{5.526195in}{2.882077in}}{\pgfqpoint{5.534008in}{2.889891in}}%
\pgfpathcurveto{\pgfqpoint{5.541822in}{2.897704in}}{\pgfqpoint{5.546212in}{2.908303in}}{\pgfqpoint{5.546212in}{2.919353in}}%
\pgfpathcurveto{\pgfqpoint{5.546212in}{2.930404in}}{\pgfqpoint{5.541822in}{2.941003in}}{\pgfqpoint{5.534008in}{2.948816in}}%
\pgfpathcurveto{\pgfqpoint{5.526195in}{2.956630in}}{\pgfqpoint{5.515596in}{2.961020in}}{\pgfqpoint{5.504545in}{2.961020in}}%
\pgfpathcurveto{\pgfqpoint{5.493495in}{2.961020in}}{\pgfqpoint{5.482896in}{2.956630in}}{\pgfqpoint{5.475083in}{2.948816in}}%
\pgfpathcurveto{\pgfqpoint{5.467269in}{2.941003in}}{\pgfqpoint{5.462879in}{2.930404in}}{\pgfqpoint{5.462879in}{2.919353in}}%
\pgfpathcurveto{\pgfqpoint{5.462879in}{2.908303in}}{\pgfqpoint{5.467269in}{2.897704in}}{\pgfqpoint{5.475083in}{2.889891in}}%
\pgfpathcurveto{\pgfqpoint{5.482896in}{2.882077in}}{\pgfqpoint{5.493495in}{2.877687in}}{\pgfqpoint{5.504545in}{2.877687in}}%
\pgfpathclose%
\pgfusepath{stroke,fill}%
\end{pgfscope}%
\begin{pgfscope}%
\pgfpathrectangle{\pgfqpoint{0.800000in}{0.528000in}}{\pgfqpoint{4.960000in}{3.696000in}}%
\pgfusepath{clip}%
\pgfsetbuttcap%
\pgfsetroundjoin%
\definecolor{currentfill}{rgb}{0.000000,0.000000,0.000000}%
\pgfsetfillcolor{currentfill}%
\pgfsetlinewidth{1.003750pt}%
\definecolor{currentstroke}{rgb}{0.000000,0.000000,0.000000}%
\pgfsetstrokecolor{currentstroke}%
\pgfsetdash{}{0pt}%
\pgfpathmoveto{\pgfqpoint{5.504545in}{2.877687in}}%
\pgfpathcurveto{\pgfqpoint{5.515596in}{2.877687in}}{\pgfqpoint{5.526195in}{2.882077in}}{\pgfqpoint{5.534008in}{2.889891in}}%
\pgfpathcurveto{\pgfqpoint{5.541822in}{2.897704in}}{\pgfqpoint{5.546212in}{2.908303in}}{\pgfqpoint{5.546212in}{2.919353in}}%
\pgfpathcurveto{\pgfqpoint{5.546212in}{2.930404in}}{\pgfqpoint{5.541822in}{2.941003in}}{\pgfqpoint{5.534008in}{2.948816in}}%
\pgfpathcurveto{\pgfqpoint{5.526195in}{2.956630in}}{\pgfqpoint{5.515596in}{2.961020in}}{\pgfqpoint{5.504545in}{2.961020in}}%
\pgfpathcurveto{\pgfqpoint{5.493495in}{2.961020in}}{\pgfqpoint{5.482896in}{2.956630in}}{\pgfqpoint{5.475083in}{2.948816in}}%
\pgfpathcurveto{\pgfqpoint{5.467269in}{2.941003in}}{\pgfqpoint{5.462879in}{2.930404in}}{\pgfqpoint{5.462879in}{2.919353in}}%
\pgfpathcurveto{\pgfqpoint{5.462879in}{2.908303in}}{\pgfqpoint{5.467269in}{2.897704in}}{\pgfqpoint{5.475083in}{2.889891in}}%
\pgfpathcurveto{\pgfqpoint{5.482896in}{2.882077in}}{\pgfqpoint{5.493495in}{2.877687in}}{\pgfqpoint{5.504545in}{2.877687in}}%
\pgfpathclose%
\pgfusepath{stroke,fill}%
\end{pgfscope}%
\begin{pgfscope}%
\pgfpathrectangle{\pgfqpoint{0.800000in}{0.528000in}}{\pgfqpoint{4.960000in}{3.696000in}}%
\pgfusepath{clip}%
\pgfsetbuttcap%
\pgfsetroundjoin%
\definecolor{currentfill}{rgb}{0.000000,0.000000,0.000000}%
\pgfsetfillcolor{currentfill}%
\pgfsetlinewidth{1.003750pt}%
\definecolor{currentstroke}{rgb}{0.000000,0.000000,0.000000}%
\pgfsetstrokecolor{currentstroke}%
\pgfsetdash{}{0pt}%
\pgfpathmoveto{\pgfqpoint{5.504545in}{2.877687in}}%
\pgfpathcurveto{\pgfqpoint{5.515596in}{2.877687in}}{\pgfqpoint{5.526195in}{2.882077in}}{\pgfqpoint{5.534008in}{2.889891in}}%
\pgfpathcurveto{\pgfqpoint{5.541822in}{2.897704in}}{\pgfqpoint{5.546212in}{2.908303in}}{\pgfqpoint{5.546212in}{2.919353in}}%
\pgfpathcurveto{\pgfqpoint{5.546212in}{2.930404in}}{\pgfqpoint{5.541822in}{2.941003in}}{\pgfqpoint{5.534008in}{2.948816in}}%
\pgfpathcurveto{\pgfqpoint{5.526195in}{2.956630in}}{\pgfqpoint{5.515596in}{2.961020in}}{\pgfqpoint{5.504545in}{2.961020in}}%
\pgfpathcurveto{\pgfqpoint{5.493495in}{2.961020in}}{\pgfqpoint{5.482896in}{2.956630in}}{\pgfqpoint{5.475083in}{2.948816in}}%
\pgfpathcurveto{\pgfqpoint{5.467269in}{2.941003in}}{\pgfqpoint{5.462879in}{2.930404in}}{\pgfqpoint{5.462879in}{2.919353in}}%
\pgfpathcurveto{\pgfqpoint{5.462879in}{2.908303in}}{\pgfqpoint{5.467269in}{2.897704in}}{\pgfqpoint{5.475083in}{2.889891in}}%
\pgfpathcurveto{\pgfqpoint{5.482896in}{2.882077in}}{\pgfqpoint{5.493495in}{2.877687in}}{\pgfqpoint{5.504545in}{2.877687in}}%
\pgfpathclose%
\pgfusepath{stroke,fill}%
\end{pgfscope}%
\begin{pgfscope}%
\pgfpathrectangle{\pgfqpoint{0.800000in}{0.528000in}}{\pgfqpoint{4.960000in}{3.696000in}}%
\pgfusepath{clip}%
\pgfsetbuttcap%
\pgfsetroundjoin%
\definecolor{currentfill}{rgb}{0.000000,0.000000,0.000000}%
\pgfsetfillcolor{currentfill}%
\pgfsetlinewidth{1.003750pt}%
\definecolor{currentstroke}{rgb}{0.000000,0.000000,0.000000}%
\pgfsetstrokecolor{currentstroke}%
\pgfsetdash{}{0pt}%
\pgfpathmoveto{\pgfqpoint{5.504545in}{2.877687in}}%
\pgfpathcurveto{\pgfqpoint{5.515596in}{2.877687in}}{\pgfqpoint{5.526195in}{2.882077in}}{\pgfqpoint{5.534008in}{2.889891in}}%
\pgfpathcurveto{\pgfqpoint{5.541822in}{2.897704in}}{\pgfqpoint{5.546212in}{2.908303in}}{\pgfqpoint{5.546212in}{2.919353in}}%
\pgfpathcurveto{\pgfqpoint{5.546212in}{2.930404in}}{\pgfqpoint{5.541822in}{2.941003in}}{\pgfqpoint{5.534008in}{2.948816in}}%
\pgfpathcurveto{\pgfqpoint{5.526195in}{2.956630in}}{\pgfqpoint{5.515596in}{2.961020in}}{\pgfqpoint{5.504545in}{2.961020in}}%
\pgfpathcurveto{\pgfqpoint{5.493495in}{2.961020in}}{\pgfqpoint{5.482896in}{2.956630in}}{\pgfqpoint{5.475083in}{2.948816in}}%
\pgfpathcurveto{\pgfqpoint{5.467269in}{2.941003in}}{\pgfqpoint{5.462879in}{2.930404in}}{\pgfqpoint{5.462879in}{2.919353in}}%
\pgfpathcurveto{\pgfqpoint{5.462879in}{2.908303in}}{\pgfqpoint{5.467269in}{2.897704in}}{\pgfqpoint{5.475083in}{2.889891in}}%
\pgfpathcurveto{\pgfqpoint{5.482896in}{2.882077in}}{\pgfqpoint{5.493495in}{2.877687in}}{\pgfqpoint{5.504545in}{2.877687in}}%
\pgfpathclose%
\pgfusepath{stroke,fill}%
\end{pgfscope}%
\begin{pgfscope}%
\pgfpathrectangle{\pgfqpoint{0.800000in}{0.528000in}}{\pgfqpoint{4.960000in}{3.696000in}}%
\pgfusepath{clip}%
\pgfsetbuttcap%
\pgfsetroundjoin%
\definecolor{currentfill}{rgb}{0.000000,0.000000,0.000000}%
\pgfsetfillcolor{currentfill}%
\pgfsetlinewidth{1.003750pt}%
\definecolor{currentstroke}{rgb}{0.000000,0.000000,0.000000}%
\pgfsetstrokecolor{currentstroke}%
\pgfsetdash{}{0pt}%
\pgfpathmoveto{\pgfqpoint{5.504545in}{2.877687in}}%
\pgfpathcurveto{\pgfqpoint{5.515596in}{2.877687in}}{\pgfqpoint{5.526195in}{2.882077in}}{\pgfqpoint{5.534008in}{2.889891in}}%
\pgfpathcurveto{\pgfqpoint{5.541822in}{2.897704in}}{\pgfqpoint{5.546212in}{2.908303in}}{\pgfqpoint{5.546212in}{2.919353in}}%
\pgfpathcurveto{\pgfqpoint{5.546212in}{2.930404in}}{\pgfqpoint{5.541822in}{2.941003in}}{\pgfqpoint{5.534008in}{2.948816in}}%
\pgfpathcurveto{\pgfqpoint{5.526195in}{2.956630in}}{\pgfqpoint{5.515596in}{2.961020in}}{\pgfqpoint{5.504545in}{2.961020in}}%
\pgfpathcurveto{\pgfqpoint{5.493495in}{2.961020in}}{\pgfqpoint{5.482896in}{2.956630in}}{\pgfqpoint{5.475083in}{2.948816in}}%
\pgfpathcurveto{\pgfqpoint{5.467269in}{2.941003in}}{\pgfqpoint{5.462879in}{2.930404in}}{\pgfqpoint{5.462879in}{2.919353in}}%
\pgfpathcurveto{\pgfqpoint{5.462879in}{2.908303in}}{\pgfqpoint{5.467269in}{2.897704in}}{\pgfqpoint{5.475083in}{2.889891in}}%
\pgfpathcurveto{\pgfqpoint{5.482896in}{2.882077in}}{\pgfqpoint{5.493495in}{2.877687in}}{\pgfqpoint{5.504545in}{2.877687in}}%
\pgfpathclose%
\pgfusepath{stroke,fill}%
\end{pgfscope}%
\begin{pgfscope}%
\pgfpathrectangle{\pgfqpoint{0.800000in}{0.528000in}}{\pgfqpoint{4.960000in}{3.696000in}}%
\pgfusepath{clip}%
\pgfsetbuttcap%
\pgfsetroundjoin%
\definecolor{currentfill}{rgb}{0.000000,0.000000,0.000000}%
\pgfsetfillcolor{currentfill}%
\pgfsetlinewidth{1.003750pt}%
\definecolor{currentstroke}{rgb}{0.000000,0.000000,0.000000}%
\pgfsetstrokecolor{currentstroke}%
\pgfsetdash{}{0pt}%
\pgfpathmoveto{\pgfqpoint{5.504545in}{2.877687in}}%
\pgfpathcurveto{\pgfqpoint{5.515596in}{2.877687in}}{\pgfqpoint{5.526195in}{2.882077in}}{\pgfqpoint{5.534008in}{2.889891in}}%
\pgfpathcurveto{\pgfqpoint{5.541822in}{2.897704in}}{\pgfqpoint{5.546212in}{2.908303in}}{\pgfqpoint{5.546212in}{2.919353in}}%
\pgfpathcurveto{\pgfqpoint{5.546212in}{2.930404in}}{\pgfqpoint{5.541822in}{2.941003in}}{\pgfqpoint{5.534008in}{2.948816in}}%
\pgfpathcurveto{\pgfqpoint{5.526195in}{2.956630in}}{\pgfqpoint{5.515596in}{2.961020in}}{\pgfqpoint{5.504545in}{2.961020in}}%
\pgfpathcurveto{\pgfqpoint{5.493495in}{2.961020in}}{\pgfqpoint{5.482896in}{2.956630in}}{\pgfqpoint{5.475083in}{2.948816in}}%
\pgfpathcurveto{\pgfqpoint{5.467269in}{2.941003in}}{\pgfqpoint{5.462879in}{2.930404in}}{\pgfqpoint{5.462879in}{2.919353in}}%
\pgfpathcurveto{\pgfqpoint{5.462879in}{2.908303in}}{\pgfqpoint{5.467269in}{2.897704in}}{\pgfqpoint{5.475083in}{2.889891in}}%
\pgfpathcurveto{\pgfqpoint{5.482896in}{2.882077in}}{\pgfqpoint{5.493495in}{2.877687in}}{\pgfqpoint{5.504545in}{2.877687in}}%
\pgfpathclose%
\pgfusepath{stroke,fill}%
\end{pgfscope}%
\begin{pgfscope}%
\pgfpathrectangle{\pgfqpoint{0.800000in}{0.528000in}}{\pgfqpoint{4.960000in}{3.696000in}}%
\pgfusepath{clip}%
\pgfsetbuttcap%
\pgfsetroundjoin%
\definecolor{currentfill}{rgb}{0.000000,0.000000,0.000000}%
\pgfsetfillcolor{currentfill}%
\pgfsetlinewidth{1.003750pt}%
\definecolor{currentstroke}{rgb}{0.000000,0.000000,0.000000}%
\pgfsetstrokecolor{currentstroke}%
\pgfsetdash{}{0pt}%
\pgfpathmoveto{\pgfqpoint{5.504545in}{3.984333in}}%
\pgfpathcurveto{\pgfqpoint{5.515596in}{3.984333in}}{\pgfqpoint{5.526195in}{3.988724in}}{\pgfqpoint{5.534008in}{3.996537in}}%
\pgfpathcurveto{\pgfqpoint{5.541822in}{4.004351in}}{\pgfqpoint{5.546212in}{4.014950in}}{\pgfqpoint{5.546212in}{4.026000in}}%
\pgfpathcurveto{\pgfqpoint{5.546212in}{4.037050in}}{\pgfqpoint{5.541822in}{4.047649in}}{\pgfqpoint{5.534008in}{4.055463in}}%
\pgfpathcurveto{\pgfqpoint{5.526195in}{4.063276in}}{\pgfqpoint{5.515596in}{4.067667in}}{\pgfqpoint{5.504545in}{4.067667in}}%
\pgfpathcurveto{\pgfqpoint{5.493495in}{4.067667in}}{\pgfqpoint{5.482896in}{4.063276in}}{\pgfqpoint{5.475083in}{4.055463in}}%
\pgfpathcurveto{\pgfqpoint{5.467269in}{4.047649in}}{\pgfqpoint{5.462879in}{4.037050in}}{\pgfqpoint{5.462879in}{4.026000in}}%
\pgfpathcurveto{\pgfqpoint{5.462879in}{4.014950in}}{\pgfqpoint{5.467269in}{4.004351in}}{\pgfqpoint{5.475083in}{3.996537in}}%
\pgfpathcurveto{\pgfqpoint{5.482896in}{3.988724in}}{\pgfqpoint{5.493495in}{3.984333in}}{\pgfqpoint{5.504545in}{3.984333in}}%
\pgfpathclose%
\pgfusepath{stroke,fill}%
\end{pgfscope}%
\begin{pgfscope}%
\pgfpathrectangle{\pgfqpoint{0.800000in}{0.528000in}}{\pgfqpoint{4.960000in}{3.696000in}}%
\pgfusepath{clip}%
\pgfsetbuttcap%
\pgfsetroundjoin%
\definecolor{currentfill}{rgb}{0.000000,0.000000,0.000000}%
\pgfsetfillcolor{currentfill}%
\pgfsetlinewidth{1.003750pt}%
\definecolor{currentstroke}{rgb}{0.000000,0.000000,0.000000}%
\pgfsetstrokecolor{currentstroke}%
\pgfsetdash{}{0pt}%
\pgfpathmoveto{\pgfqpoint{5.504545in}{2.877687in}}%
\pgfpathcurveto{\pgfqpoint{5.515596in}{2.877687in}}{\pgfqpoint{5.526195in}{2.882077in}}{\pgfqpoint{5.534008in}{2.889891in}}%
\pgfpathcurveto{\pgfqpoint{5.541822in}{2.897704in}}{\pgfqpoint{5.546212in}{2.908303in}}{\pgfqpoint{5.546212in}{2.919353in}}%
\pgfpathcurveto{\pgfqpoint{5.546212in}{2.930404in}}{\pgfqpoint{5.541822in}{2.941003in}}{\pgfqpoint{5.534008in}{2.948816in}}%
\pgfpathcurveto{\pgfqpoint{5.526195in}{2.956630in}}{\pgfqpoint{5.515596in}{2.961020in}}{\pgfqpoint{5.504545in}{2.961020in}}%
\pgfpathcurveto{\pgfqpoint{5.493495in}{2.961020in}}{\pgfqpoint{5.482896in}{2.956630in}}{\pgfqpoint{5.475083in}{2.948816in}}%
\pgfpathcurveto{\pgfqpoint{5.467269in}{2.941003in}}{\pgfqpoint{5.462879in}{2.930404in}}{\pgfqpoint{5.462879in}{2.919353in}}%
\pgfpathcurveto{\pgfqpoint{5.462879in}{2.908303in}}{\pgfqpoint{5.467269in}{2.897704in}}{\pgfqpoint{5.475083in}{2.889891in}}%
\pgfpathcurveto{\pgfqpoint{5.482896in}{2.882077in}}{\pgfqpoint{5.493495in}{2.877687in}}{\pgfqpoint{5.504545in}{2.877687in}}%
\pgfpathclose%
\pgfusepath{stroke,fill}%
\end{pgfscope}%
\begin{pgfscope}%
\pgfpathrectangle{\pgfqpoint{0.800000in}{0.528000in}}{\pgfqpoint{4.960000in}{3.696000in}}%
\pgfusepath{clip}%
\pgfsetbuttcap%
\pgfsetroundjoin%
\definecolor{currentfill}{rgb}{0.000000,0.000000,0.000000}%
\pgfsetfillcolor{currentfill}%
\pgfsetlinewidth{1.003750pt}%
\definecolor{currentstroke}{rgb}{0.000000,0.000000,0.000000}%
\pgfsetstrokecolor{currentstroke}%
\pgfsetdash{}{0pt}%
\pgfpathmoveto{\pgfqpoint{5.504545in}{2.877687in}}%
\pgfpathcurveto{\pgfqpoint{5.515596in}{2.877687in}}{\pgfqpoint{5.526195in}{2.882077in}}{\pgfqpoint{5.534008in}{2.889891in}}%
\pgfpathcurveto{\pgfqpoint{5.541822in}{2.897704in}}{\pgfqpoint{5.546212in}{2.908303in}}{\pgfqpoint{5.546212in}{2.919353in}}%
\pgfpathcurveto{\pgfqpoint{5.546212in}{2.930404in}}{\pgfqpoint{5.541822in}{2.941003in}}{\pgfqpoint{5.534008in}{2.948816in}}%
\pgfpathcurveto{\pgfqpoint{5.526195in}{2.956630in}}{\pgfqpoint{5.515596in}{2.961020in}}{\pgfqpoint{5.504545in}{2.961020in}}%
\pgfpathcurveto{\pgfqpoint{5.493495in}{2.961020in}}{\pgfqpoint{5.482896in}{2.956630in}}{\pgfqpoint{5.475083in}{2.948816in}}%
\pgfpathcurveto{\pgfqpoint{5.467269in}{2.941003in}}{\pgfqpoint{5.462879in}{2.930404in}}{\pgfqpoint{5.462879in}{2.919353in}}%
\pgfpathcurveto{\pgfqpoint{5.462879in}{2.908303in}}{\pgfqpoint{5.467269in}{2.897704in}}{\pgfqpoint{5.475083in}{2.889891in}}%
\pgfpathcurveto{\pgfqpoint{5.482896in}{2.882077in}}{\pgfqpoint{5.493495in}{2.877687in}}{\pgfqpoint{5.504545in}{2.877687in}}%
\pgfpathclose%
\pgfusepath{stroke,fill}%
\end{pgfscope}%
\begin{pgfscope}%
\pgfpathrectangle{\pgfqpoint{0.800000in}{0.528000in}}{\pgfqpoint{4.960000in}{3.696000in}}%
\pgfusepath{clip}%
\pgfsetbuttcap%
\pgfsetroundjoin%
\definecolor{currentfill}{rgb}{0.000000,0.000000,0.000000}%
\pgfsetfillcolor{currentfill}%
\pgfsetlinewidth{1.003750pt}%
\definecolor{currentstroke}{rgb}{0.000000,0.000000,0.000000}%
\pgfsetstrokecolor{currentstroke}%
\pgfsetdash{}{0pt}%
\pgfpathmoveto{\pgfqpoint{5.504545in}{3.984333in}}%
\pgfpathcurveto{\pgfqpoint{5.515596in}{3.984333in}}{\pgfqpoint{5.526195in}{3.988724in}}{\pgfqpoint{5.534008in}{3.996537in}}%
\pgfpathcurveto{\pgfqpoint{5.541822in}{4.004351in}}{\pgfqpoint{5.546212in}{4.014950in}}{\pgfqpoint{5.546212in}{4.026000in}}%
\pgfpathcurveto{\pgfqpoint{5.546212in}{4.037050in}}{\pgfqpoint{5.541822in}{4.047649in}}{\pgfqpoint{5.534008in}{4.055463in}}%
\pgfpathcurveto{\pgfqpoint{5.526195in}{4.063276in}}{\pgfqpoint{5.515596in}{4.067667in}}{\pgfqpoint{5.504545in}{4.067667in}}%
\pgfpathcurveto{\pgfqpoint{5.493495in}{4.067667in}}{\pgfqpoint{5.482896in}{4.063276in}}{\pgfqpoint{5.475083in}{4.055463in}}%
\pgfpathcurveto{\pgfqpoint{5.467269in}{4.047649in}}{\pgfqpoint{5.462879in}{4.037050in}}{\pgfqpoint{5.462879in}{4.026000in}}%
\pgfpathcurveto{\pgfqpoint{5.462879in}{4.014950in}}{\pgfqpoint{5.467269in}{4.004351in}}{\pgfqpoint{5.475083in}{3.996537in}}%
\pgfpathcurveto{\pgfqpoint{5.482896in}{3.988724in}}{\pgfqpoint{5.493495in}{3.984333in}}{\pgfqpoint{5.504545in}{3.984333in}}%
\pgfpathclose%
\pgfusepath{stroke,fill}%
\end{pgfscope}%
\begin{pgfscope}%
\pgfpathrectangle{\pgfqpoint{0.800000in}{0.528000in}}{\pgfqpoint{4.960000in}{3.696000in}}%
\pgfusepath{clip}%
\pgfsetbuttcap%
\pgfsetroundjoin%
\definecolor{currentfill}{rgb}{0.000000,0.000000,0.000000}%
\pgfsetfillcolor{currentfill}%
\pgfsetlinewidth{1.003750pt}%
\definecolor{currentstroke}{rgb}{0.000000,0.000000,0.000000}%
\pgfsetstrokecolor{currentstroke}%
\pgfsetdash{}{0pt}%
\pgfpathmoveto{\pgfqpoint{5.504545in}{2.877687in}}%
\pgfpathcurveto{\pgfqpoint{5.515596in}{2.877687in}}{\pgfqpoint{5.526195in}{2.882077in}}{\pgfqpoint{5.534008in}{2.889891in}}%
\pgfpathcurveto{\pgfqpoint{5.541822in}{2.897704in}}{\pgfqpoint{5.546212in}{2.908303in}}{\pgfqpoint{5.546212in}{2.919353in}}%
\pgfpathcurveto{\pgfqpoint{5.546212in}{2.930404in}}{\pgfqpoint{5.541822in}{2.941003in}}{\pgfqpoint{5.534008in}{2.948816in}}%
\pgfpathcurveto{\pgfqpoint{5.526195in}{2.956630in}}{\pgfqpoint{5.515596in}{2.961020in}}{\pgfqpoint{5.504545in}{2.961020in}}%
\pgfpathcurveto{\pgfqpoint{5.493495in}{2.961020in}}{\pgfqpoint{5.482896in}{2.956630in}}{\pgfqpoint{5.475083in}{2.948816in}}%
\pgfpathcurveto{\pgfqpoint{5.467269in}{2.941003in}}{\pgfqpoint{5.462879in}{2.930404in}}{\pgfqpoint{5.462879in}{2.919353in}}%
\pgfpathcurveto{\pgfqpoint{5.462879in}{2.908303in}}{\pgfqpoint{5.467269in}{2.897704in}}{\pgfqpoint{5.475083in}{2.889891in}}%
\pgfpathcurveto{\pgfqpoint{5.482896in}{2.882077in}}{\pgfqpoint{5.493495in}{2.877687in}}{\pgfqpoint{5.504545in}{2.877687in}}%
\pgfpathclose%
\pgfusepath{stroke,fill}%
\end{pgfscope}%
\begin{pgfscope}%
\pgfpathrectangle{\pgfqpoint{0.800000in}{0.528000in}}{\pgfqpoint{4.960000in}{3.696000in}}%
\pgfusepath{clip}%
\pgfsetbuttcap%
\pgfsetroundjoin%
\definecolor{currentfill}{rgb}{0.000000,0.000000,0.000000}%
\pgfsetfillcolor{currentfill}%
\pgfsetlinewidth{1.003750pt}%
\definecolor{currentstroke}{rgb}{0.000000,0.000000,0.000000}%
\pgfsetstrokecolor{currentstroke}%
\pgfsetdash{}{0pt}%
\pgfpathmoveto{\pgfqpoint{5.504545in}{2.877687in}}%
\pgfpathcurveto{\pgfqpoint{5.515596in}{2.877687in}}{\pgfqpoint{5.526195in}{2.882077in}}{\pgfqpoint{5.534008in}{2.889891in}}%
\pgfpathcurveto{\pgfqpoint{5.541822in}{2.897704in}}{\pgfqpoint{5.546212in}{2.908303in}}{\pgfqpoint{5.546212in}{2.919353in}}%
\pgfpathcurveto{\pgfqpoint{5.546212in}{2.930404in}}{\pgfqpoint{5.541822in}{2.941003in}}{\pgfqpoint{5.534008in}{2.948816in}}%
\pgfpathcurveto{\pgfqpoint{5.526195in}{2.956630in}}{\pgfqpoint{5.515596in}{2.961020in}}{\pgfqpoint{5.504545in}{2.961020in}}%
\pgfpathcurveto{\pgfqpoint{5.493495in}{2.961020in}}{\pgfqpoint{5.482896in}{2.956630in}}{\pgfqpoint{5.475083in}{2.948816in}}%
\pgfpathcurveto{\pgfqpoint{5.467269in}{2.941003in}}{\pgfqpoint{5.462879in}{2.930404in}}{\pgfqpoint{5.462879in}{2.919353in}}%
\pgfpathcurveto{\pgfqpoint{5.462879in}{2.908303in}}{\pgfqpoint{5.467269in}{2.897704in}}{\pgfqpoint{5.475083in}{2.889891in}}%
\pgfpathcurveto{\pgfqpoint{5.482896in}{2.882077in}}{\pgfqpoint{5.493495in}{2.877687in}}{\pgfqpoint{5.504545in}{2.877687in}}%
\pgfpathclose%
\pgfusepath{stroke,fill}%
\end{pgfscope}%
\begin{pgfscope}%
\pgfpathrectangle{\pgfqpoint{0.800000in}{0.528000in}}{\pgfqpoint{4.960000in}{3.696000in}}%
\pgfusepath{clip}%
\pgfsetbuttcap%
\pgfsetroundjoin%
\definecolor{currentfill}{rgb}{0.000000,0.000000,0.000000}%
\pgfsetfillcolor{currentfill}%
\pgfsetlinewidth{1.003750pt}%
\definecolor{currentstroke}{rgb}{0.000000,0.000000,0.000000}%
\pgfsetstrokecolor{currentstroke}%
\pgfsetdash{}{0pt}%
\pgfpathmoveto{\pgfqpoint{5.504545in}{2.877687in}}%
\pgfpathcurveto{\pgfqpoint{5.515596in}{2.877687in}}{\pgfqpoint{5.526195in}{2.882077in}}{\pgfqpoint{5.534008in}{2.889891in}}%
\pgfpathcurveto{\pgfqpoint{5.541822in}{2.897704in}}{\pgfqpoint{5.546212in}{2.908303in}}{\pgfqpoint{5.546212in}{2.919353in}}%
\pgfpathcurveto{\pgfqpoint{5.546212in}{2.930404in}}{\pgfqpoint{5.541822in}{2.941003in}}{\pgfqpoint{5.534008in}{2.948816in}}%
\pgfpathcurveto{\pgfqpoint{5.526195in}{2.956630in}}{\pgfqpoint{5.515596in}{2.961020in}}{\pgfqpoint{5.504545in}{2.961020in}}%
\pgfpathcurveto{\pgfqpoint{5.493495in}{2.961020in}}{\pgfqpoint{5.482896in}{2.956630in}}{\pgfqpoint{5.475083in}{2.948816in}}%
\pgfpathcurveto{\pgfqpoint{5.467269in}{2.941003in}}{\pgfqpoint{5.462879in}{2.930404in}}{\pgfqpoint{5.462879in}{2.919353in}}%
\pgfpathcurveto{\pgfqpoint{5.462879in}{2.908303in}}{\pgfqpoint{5.467269in}{2.897704in}}{\pgfqpoint{5.475083in}{2.889891in}}%
\pgfpathcurveto{\pgfqpoint{5.482896in}{2.882077in}}{\pgfqpoint{5.493495in}{2.877687in}}{\pgfqpoint{5.504545in}{2.877687in}}%
\pgfpathclose%
\pgfusepath{stroke,fill}%
\end{pgfscope}%
\begin{pgfscope}%
\pgfpathrectangle{\pgfqpoint{0.800000in}{0.528000in}}{\pgfqpoint{4.960000in}{3.696000in}}%
\pgfusepath{clip}%
\pgfsetbuttcap%
\pgfsetroundjoin%
\definecolor{currentfill}{rgb}{0.000000,0.000000,0.000000}%
\pgfsetfillcolor{currentfill}%
\pgfsetlinewidth{1.003750pt}%
\definecolor{currentstroke}{rgb}{0.000000,0.000000,0.000000}%
\pgfsetstrokecolor{currentstroke}%
\pgfsetdash{}{0pt}%
\pgfpathmoveto{\pgfqpoint{5.504545in}{2.877687in}}%
\pgfpathcurveto{\pgfqpoint{5.515596in}{2.877687in}}{\pgfqpoint{5.526195in}{2.882077in}}{\pgfqpoint{5.534008in}{2.889891in}}%
\pgfpathcurveto{\pgfqpoint{5.541822in}{2.897704in}}{\pgfqpoint{5.546212in}{2.908303in}}{\pgfqpoint{5.546212in}{2.919353in}}%
\pgfpathcurveto{\pgfqpoint{5.546212in}{2.930404in}}{\pgfqpoint{5.541822in}{2.941003in}}{\pgfqpoint{5.534008in}{2.948816in}}%
\pgfpathcurveto{\pgfqpoint{5.526195in}{2.956630in}}{\pgfqpoint{5.515596in}{2.961020in}}{\pgfqpoint{5.504545in}{2.961020in}}%
\pgfpathcurveto{\pgfqpoint{5.493495in}{2.961020in}}{\pgfqpoint{5.482896in}{2.956630in}}{\pgfqpoint{5.475083in}{2.948816in}}%
\pgfpathcurveto{\pgfqpoint{5.467269in}{2.941003in}}{\pgfqpoint{5.462879in}{2.930404in}}{\pgfqpoint{5.462879in}{2.919353in}}%
\pgfpathcurveto{\pgfqpoint{5.462879in}{2.908303in}}{\pgfqpoint{5.467269in}{2.897704in}}{\pgfqpoint{5.475083in}{2.889891in}}%
\pgfpathcurveto{\pgfqpoint{5.482896in}{2.882077in}}{\pgfqpoint{5.493495in}{2.877687in}}{\pgfqpoint{5.504545in}{2.877687in}}%
\pgfpathclose%
\pgfusepath{stroke,fill}%
\end{pgfscope}%
\begin{pgfscope}%
\pgfpathrectangle{\pgfqpoint{0.800000in}{0.528000in}}{\pgfqpoint{4.960000in}{3.696000in}}%
\pgfusepath{clip}%
\pgfsetbuttcap%
\pgfsetroundjoin%
\definecolor{currentfill}{rgb}{0.000000,0.000000,0.000000}%
\pgfsetfillcolor{currentfill}%
\pgfsetlinewidth{1.003750pt}%
\definecolor{currentstroke}{rgb}{0.000000,0.000000,0.000000}%
\pgfsetstrokecolor{currentstroke}%
\pgfsetdash{}{0pt}%
\pgfpathmoveto{\pgfqpoint{5.504545in}{2.877687in}}%
\pgfpathcurveto{\pgfqpoint{5.515596in}{2.877687in}}{\pgfqpoint{5.526195in}{2.882077in}}{\pgfqpoint{5.534008in}{2.889891in}}%
\pgfpathcurveto{\pgfqpoint{5.541822in}{2.897704in}}{\pgfqpoint{5.546212in}{2.908303in}}{\pgfqpoint{5.546212in}{2.919353in}}%
\pgfpathcurveto{\pgfqpoint{5.546212in}{2.930404in}}{\pgfqpoint{5.541822in}{2.941003in}}{\pgfqpoint{5.534008in}{2.948816in}}%
\pgfpathcurveto{\pgfqpoint{5.526195in}{2.956630in}}{\pgfqpoint{5.515596in}{2.961020in}}{\pgfqpoint{5.504545in}{2.961020in}}%
\pgfpathcurveto{\pgfqpoint{5.493495in}{2.961020in}}{\pgfqpoint{5.482896in}{2.956630in}}{\pgfqpoint{5.475083in}{2.948816in}}%
\pgfpathcurveto{\pgfqpoint{5.467269in}{2.941003in}}{\pgfqpoint{5.462879in}{2.930404in}}{\pgfqpoint{5.462879in}{2.919353in}}%
\pgfpathcurveto{\pgfqpoint{5.462879in}{2.908303in}}{\pgfqpoint{5.467269in}{2.897704in}}{\pgfqpoint{5.475083in}{2.889891in}}%
\pgfpathcurveto{\pgfqpoint{5.482896in}{2.882077in}}{\pgfqpoint{5.493495in}{2.877687in}}{\pgfqpoint{5.504545in}{2.877687in}}%
\pgfpathclose%
\pgfusepath{stroke,fill}%
\end{pgfscope}%
\begin{pgfscope}%
\pgfpathrectangle{\pgfqpoint{0.800000in}{0.528000in}}{\pgfqpoint{4.960000in}{3.696000in}}%
\pgfusepath{clip}%
\pgfsetbuttcap%
\pgfsetroundjoin%
\definecolor{currentfill}{rgb}{0.000000,0.000000,0.000000}%
\pgfsetfillcolor{currentfill}%
\pgfsetlinewidth{1.003750pt}%
\definecolor{currentstroke}{rgb}{0.000000,0.000000,0.000000}%
\pgfsetstrokecolor{currentstroke}%
\pgfsetdash{}{0pt}%
\pgfpathmoveto{\pgfqpoint{5.504545in}{2.877687in}}%
\pgfpathcurveto{\pgfqpoint{5.515596in}{2.877687in}}{\pgfqpoint{5.526195in}{2.882077in}}{\pgfqpoint{5.534008in}{2.889891in}}%
\pgfpathcurveto{\pgfqpoint{5.541822in}{2.897704in}}{\pgfqpoint{5.546212in}{2.908303in}}{\pgfqpoint{5.546212in}{2.919353in}}%
\pgfpathcurveto{\pgfqpoint{5.546212in}{2.930404in}}{\pgfqpoint{5.541822in}{2.941003in}}{\pgfqpoint{5.534008in}{2.948816in}}%
\pgfpathcurveto{\pgfqpoint{5.526195in}{2.956630in}}{\pgfqpoint{5.515596in}{2.961020in}}{\pgfqpoint{5.504545in}{2.961020in}}%
\pgfpathcurveto{\pgfqpoint{5.493495in}{2.961020in}}{\pgfqpoint{5.482896in}{2.956630in}}{\pgfqpoint{5.475083in}{2.948816in}}%
\pgfpathcurveto{\pgfqpoint{5.467269in}{2.941003in}}{\pgfqpoint{5.462879in}{2.930404in}}{\pgfqpoint{5.462879in}{2.919353in}}%
\pgfpathcurveto{\pgfqpoint{5.462879in}{2.908303in}}{\pgfqpoint{5.467269in}{2.897704in}}{\pgfqpoint{5.475083in}{2.889891in}}%
\pgfpathcurveto{\pgfqpoint{5.482896in}{2.882077in}}{\pgfqpoint{5.493495in}{2.877687in}}{\pgfqpoint{5.504545in}{2.877687in}}%
\pgfpathclose%
\pgfusepath{stroke,fill}%
\end{pgfscope}%
\begin{pgfscope}%
\pgfpathrectangle{\pgfqpoint{0.800000in}{0.528000in}}{\pgfqpoint{4.960000in}{3.696000in}}%
\pgfusepath{clip}%
\pgfsetbuttcap%
\pgfsetroundjoin%
\definecolor{currentfill}{rgb}{0.000000,0.000000,0.000000}%
\pgfsetfillcolor{currentfill}%
\pgfsetlinewidth{1.003750pt}%
\definecolor{currentstroke}{rgb}{0.000000,0.000000,0.000000}%
\pgfsetstrokecolor{currentstroke}%
\pgfsetdash{}{0pt}%
\pgfpathmoveto{\pgfqpoint{5.504545in}{2.877687in}}%
\pgfpathcurveto{\pgfqpoint{5.515596in}{2.877687in}}{\pgfqpoint{5.526195in}{2.882077in}}{\pgfqpoint{5.534008in}{2.889891in}}%
\pgfpathcurveto{\pgfqpoint{5.541822in}{2.897704in}}{\pgfqpoint{5.546212in}{2.908303in}}{\pgfqpoint{5.546212in}{2.919353in}}%
\pgfpathcurveto{\pgfqpoint{5.546212in}{2.930404in}}{\pgfqpoint{5.541822in}{2.941003in}}{\pgfqpoint{5.534008in}{2.948816in}}%
\pgfpathcurveto{\pgfqpoint{5.526195in}{2.956630in}}{\pgfqpoint{5.515596in}{2.961020in}}{\pgfqpoint{5.504545in}{2.961020in}}%
\pgfpathcurveto{\pgfqpoint{5.493495in}{2.961020in}}{\pgfqpoint{5.482896in}{2.956630in}}{\pgfqpoint{5.475083in}{2.948816in}}%
\pgfpathcurveto{\pgfqpoint{5.467269in}{2.941003in}}{\pgfqpoint{5.462879in}{2.930404in}}{\pgfqpoint{5.462879in}{2.919353in}}%
\pgfpathcurveto{\pgfqpoint{5.462879in}{2.908303in}}{\pgfqpoint{5.467269in}{2.897704in}}{\pgfqpoint{5.475083in}{2.889891in}}%
\pgfpathcurveto{\pgfqpoint{5.482896in}{2.882077in}}{\pgfqpoint{5.493495in}{2.877687in}}{\pgfqpoint{5.504545in}{2.877687in}}%
\pgfpathclose%
\pgfusepath{stroke,fill}%
\end{pgfscope}%
\begin{pgfscope}%
\pgfpathrectangle{\pgfqpoint{0.800000in}{0.528000in}}{\pgfqpoint{4.960000in}{3.696000in}}%
\pgfusepath{clip}%
\pgfsetbuttcap%
\pgfsetroundjoin%
\definecolor{currentfill}{rgb}{0.000000,0.000000,0.000000}%
\pgfsetfillcolor{currentfill}%
\pgfsetlinewidth{1.003750pt}%
\definecolor{currentstroke}{rgb}{0.000000,0.000000,0.000000}%
\pgfsetstrokecolor{currentstroke}%
\pgfsetdash{}{0pt}%
\pgfpathmoveto{\pgfqpoint{5.504545in}{2.877687in}}%
\pgfpathcurveto{\pgfqpoint{5.515596in}{2.877687in}}{\pgfqpoint{5.526195in}{2.882077in}}{\pgfqpoint{5.534008in}{2.889891in}}%
\pgfpathcurveto{\pgfqpoint{5.541822in}{2.897704in}}{\pgfqpoint{5.546212in}{2.908303in}}{\pgfqpoint{5.546212in}{2.919353in}}%
\pgfpathcurveto{\pgfqpoint{5.546212in}{2.930404in}}{\pgfqpoint{5.541822in}{2.941003in}}{\pgfqpoint{5.534008in}{2.948816in}}%
\pgfpathcurveto{\pgfqpoint{5.526195in}{2.956630in}}{\pgfqpoint{5.515596in}{2.961020in}}{\pgfqpoint{5.504545in}{2.961020in}}%
\pgfpathcurveto{\pgfqpoint{5.493495in}{2.961020in}}{\pgfqpoint{5.482896in}{2.956630in}}{\pgfqpoint{5.475083in}{2.948816in}}%
\pgfpathcurveto{\pgfqpoint{5.467269in}{2.941003in}}{\pgfqpoint{5.462879in}{2.930404in}}{\pgfqpoint{5.462879in}{2.919353in}}%
\pgfpathcurveto{\pgfqpoint{5.462879in}{2.908303in}}{\pgfqpoint{5.467269in}{2.897704in}}{\pgfqpoint{5.475083in}{2.889891in}}%
\pgfpathcurveto{\pgfqpoint{5.482896in}{2.882077in}}{\pgfqpoint{5.493495in}{2.877687in}}{\pgfqpoint{5.504545in}{2.877687in}}%
\pgfpathclose%
\pgfusepath{stroke,fill}%
\end{pgfscope}%
\begin{pgfscope}%
\pgfpathrectangle{\pgfqpoint{0.800000in}{0.528000in}}{\pgfqpoint{4.960000in}{3.696000in}}%
\pgfusepath{clip}%
\pgfsetbuttcap%
\pgfsetroundjoin%
\definecolor{currentfill}{rgb}{0.000000,0.000000,0.000000}%
\pgfsetfillcolor{currentfill}%
\pgfsetlinewidth{1.003750pt}%
\definecolor{currentstroke}{rgb}{0.000000,0.000000,0.000000}%
\pgfsetstrokecolor{currentstroke}%
\pgfsetdash{}{0pt}%
\pgfpathmoveto{\pgfqpoint{5.504545in}{2.877687in}}%
\pgfpathcurveto{\pgfqpoint{5.515596in}{2.877687in}}{\pgfqpoint{5.526195in}{2.882077in}}{\pgfqpoint{5.534008in}{2.889891in}}%
\pgfpathcurveto{\pgfqpoint{5.541822in}{2.897704in}}{\pgfqpoint{5.546212in}{2.908303in}}{\pgfqpoint{5.546212in}{2.919353in}}%
\pgfpathcurveto{\pgfqpoint{5.546212in}{2.930404in}}{\pgfqpoint{5.541822in}{2.941003in}}{\pgfqpoint{5.534008in}{2.948816in}}%
\pgfpathcurveto{\pgfqpoint{5.526195in}{2.956630in}}{\pgfqpoint{5.515596in}{2.961020in}}{\pgfqpoint{5.504545in}{2.961020in}}%
\pgfpathcurveto{\pgfqpoint{5.493495in}{2.961020in}}{\pgfqpoint{5.482896in}{2.956630in}}{\pgfqpoint{5.475083in}{2.948816in}}%
\pgfpathcurveto{\pgfqpoint{5.467269in}{2.941003in}}{\pgfqpoint{5.462879in}{2.930404in}}{\pgfqpoint{5.462879in}{2.919353in}}%
\pgfpathcurveto{\pgfqpoint{5.462879in}{2.908303in}}{\pgfqpoint{5.467269in}{2.897704in}}{\pgfqpoint{5.475083in}{2.889891in}}%
\pgfpathcurveto{\pgfqpoint{5.482896in}{2.882077in}}{\pgfqpoint{5.493495in}{2.877687in}}{\pgfqpoint{5.504545in}{2.877687in}}%
\pgfpathclose%
\pgfusepath{stroke,fill}%
\end{pgfscope}%
\begin{pgfscope}%
\pgfpathrectangle{\pgfqpoint{0.800000in}{0.528000in}}{\pgfqpoint{4.960000in}{3.696000in}}%
\pgfusepath{clip}%
\pgfsetbuttcap%
\pgfsetroundjoin%
\definecolor{currentfill}{rgb}{0.000000,0.000000,0.000000}%
\pgfsetfillcolor{currentfill}%
\pgfsetlinewidth{1.003750pt}%
\definecolor{currentstroke}{rgb}{0.000000,0.000000,0.000000}%
\pgfsetstrokecolor{currentstroke}%
\pgfsetdash{}{0pt}%
\pgfpathmoveto{\pgfqpoint{5.504545in}{2.877687in}}%
\pgfpathcurveto{\pgfqpoint{5.515596in}{2.877687in}}{\pgfqpoint{5.526195in}{2.882077in}}{\pgfqpoint{5.534008in}{2.889891in}}%
\pgfpathcurveto{\pgfqpoint{5.541822in}{2.897704in}}{\pgfqpoint{5.546212in}{2.908303in}}{\pgfqpoint{5.546212in}{2.919353in}}%
\pgfpathcurveto{\pgfqpoint{5.546212in}{2.930404in}}{\pgfqpoint{5.541822in}{2.941003in}}{\pgfqpoint{5.534008in}{2.948816in}}%
\pgfpathcurveto{\pgfqpoint{5.526195in}{2.956630in}}{\pgfqpoint{5.515596in}{2.961020in}}{\pgfqpoint{5.504545in}{2.961020in}}%
\pgfpathcurveto{\pgfqpoint{5.493495in}{2.961020in}}{\pgfqpoint{5.482896in}{2.956630in}}{\pgfqpoint{5.475083in}{2.948816in}}%
\pgfpathcurveto{\pgfqpoint{5.467269in}{2.941003in}}{\pgfqpoint{5.462879in}{2.930404in}}{\pgfqpoint{5.462879in}{2.919353in}}%
\pgfpathcurveto{\pgfqpoint{5.462879in}{2.908303in}}{\pgfqpoint{5.467269in}{2.897704in}}{\pgfqpoint{5.475083in}{2.889891in}}%
\pgfpathcurveto{\pgfqpoint{5.482896in}{2.882077in}}{\pgfqpoint{5.493495in}{2.877687in}}{\pgfqpoint{5.504545in}{2.877687in}}%
\pgfpathclose%
\pgfusepath{stroke,fill}%
\end{pgfscope}%
\begin{pgfscope}%
\pgfpathrectangle{\pgfqpoint{0.800000in}{0.528000in}}{\pgfqpoint{4.960000in}{3.696000in}}%
\pgfusepath{clip}%
\pgfsetbuttcap%
\pgfsetroundjoin%
\definecolor{currentfill}{rgb}{0.000000,0.000000,0.000000}%
\pgfsetfillcolor{currentfill}%
\pgfsetlinewidth{1.003750pt}%
\definecolor{currentstroke}{rgb}{0.000000,0.000000,0.000000}%
\pgfsetstrokecolor{currentstroke}%
\pgfsetdash{}{0pt}%
\pgfpathmoveto{\pgfqpoint{5.504545in}{2.877687in}}%
\pgfpathcurveto{\pgfqpoint{5.515596in}{2.877687in}}{\pgfqpoint{5.526195in}{2.882077in}}{\pgfqpoint{5.534008in}{2.889891in}}%
\pgfpathcurveto{\pgfqpoint{5.541822in}{2.897704in}}{\pgfqpoint{5.546212in}{2.908303in}}{\pgfqpoint{5.546212in}{2.919353in}}%
\pgfpathcurveto{\pgfqpoint{5.546212in}{2.930404in}}{\pgfqpoint{5.541822in}{2.941003in}}{\pgfqpoint{5.534008in}{2.948816in}}%
\pgfpathcurveto{\pgfqpoint{5.526195in}{2.956630in}}{\pgfqpoint{5.515596in}{2.961020in}}{\pgfqpoint{5.504545in}{2.961020in}}%
\pgfpathcurveto{\pgfqpoint{5.493495in}{2.961020in}}{\pgfqpoint{5.482896in}{2.956630in}}{\pgfqpoint{5.475083in}{2.948816in}}%
\pgfpathcurveto{\pgfqpoint{5.467269in}{2.941003in}}{\pgfqpoint{5.462879in}{2.930404in}}{\pgfqpoint{5.462879in}{2.919353in}}%
\pgfpathcurveto{\pgfqpoint{5.462879in}{2.908303in}}{\pgfqpoint{5.467269in}{2.897704in}}{\pgfqpoint{5.475083in}{2.889891in}}%
\pgfpathcurveto{\pgfqpoint{5.482896in}{2.882077in}}{\pgfqpoint{5.493495in}{2.877687in}}{\pgfqpoint{5.504545in}{2.877687in}}%
\pgfpathclose%
\pgfusepath{stroke,fill}%
\end{pgfscope}%
\begin{pgfscope}%
\pgfpathrectangle{\pgfqpoint{0.800000in}{0.528000in}}{\pgfqpoint{4.960000in}{3.696000in}}%
\pgfusepath{clip}%
\pgfsetbuttcap%
\pgfsetroundjoin%
\definecolor{currentfill}{rgb}{0.000000,0.000000,0.000000}%
\pgfsetfillcolor{currentfill}%
\pgfsetlinewidth{1.003750pt}%
\definecolor{currentstroke}{rgb}{0.000000,0.000000,0.000000}%
\pgfsetstrokecolor{currentstroke}%
\pgfsetdash{}{0pt}%
\pgfpathmoveto{\pgfqpoint{5.504545in}{2.877687in}}%
\pgfpathcurveto{\pgfqpoint{5.515596in}{2.877687in}}{\pgfqpoint{5.526195in}{2.882077in}}{\pgfqpoint{5.534008in}{2.889891in}}%
\pgfpathcurveto{\pgfqpoint{5.541822in}{2.897704in}}{\pgfqpoint{5.546212in}{2.908303in}}{\pgfqpoint{5.546212in}{2.919353in}}%
\pgfpathcurveto{\pgfqpoint{5.546212in}{2.930404in}}{\pgfqpoint{5.541822in}{2.941003in}}{\pgfqpoint{5.534008in}{2.948816in}}%
\pgfpathcurveto{\pgfqpoint{5.526195in}{2.956630in}}{\pgfqpoint{5.515596in}{2.961020in}}{\pgfqpoint{5.504545in}{2.961020in}}%
\pgfpathcurveto{\pgfqpoint{5.493495in}{2.961020in}}{\pgfqpoint{5.482896in}{2.956630in}}{\pgfqpoint{5.475083in}{2.948816in}}%
\pgfpathcurveto{\pgfqpoint{5.467269in}{2.941003in}}{\pgfqpoint{5.462879in}{2.930404in}}{\pgfqpoint{5.462879in}{2.919353in}}%
\pgfpathcurveto{\pgfqpoint{5.462879in}{2.908303in}}{\pgfqpoint{5.467269in}{2.897704in}}{\pgfqpoint{5.475083in}{2.889891in}}%
\pgfpathcurveto{\pgfqpoint{5.482896in}{2.882077in}}{\pgfqpoint{5.493495in}{2.877687in}}{\pgfqpoint{5.504545in}{2.877687in}}%
\pgfpathclose%
\pgfusepath{stroke,fill}%
\end{pgfscope}%
\begin{pgfscope}%
\pgfpathrectangle{\pgfqpoint{0.800000in}{0.528000in}}{\pgfqpoint{4.960000in}{3.696000in}}%
\pgfusepath{clip}%
\pgfsetbuttcap%
\pgfsetroundjoin%
\definecolor{currentfill}{rgb}{0.000000,0.000000,0.000000}%
\pgfsetfillcolor{currentfill}%
\pgfsetlinewidth{1.003750pt}%
\definecolor{currentstroke}{rgb}{0.000000,0.000000,0.000000}%
\pgfsetstrokecolor{currentstroke}%
\pgfsetdash{}{0pt}%
\pgfpathmoveto{\pgfqpoint{5.504545in}{2.877687in}}%
\pgfpathcurveto{\pgfqpoint{5.515596in}{2.877687in}}{\pgfqpoint{5.526195in}{2.882077in}}{\pgfqpoint{5.534008in}{2.889891in}}%
\pgfpathcurveto{\pgfqpoint{5.541822in}{2.897704in}}{\pgfqpoint{5.546212in}{2.908303in}}{\pgfqpoint{5.546212in}{2.919353in}}%
\pgfpathcurveto{\pgfqpoint{5.546212in}{2.930404in}}{\pgfqpoint{5.541822in}{2.941003in}}{\pgfqpoint{5.534008in}{2.948816in}}%
\pgfpathcurveto{\pgfqpoint{5.526195in}{2.956630in}}{\pgfqpoint{5.515596in}{2.961020in}}{\pgfqpoint{5.504545in}{2.961020in}}%
\pgfpathcurveto{\pgfqpoint{5.493495in}{2.961020in}}{\pgfqpoint{5.482896in}{2.956630in}}{\pgfqpoint{5.475083in}{2.948816in}}%
\pgfpathcurveto{\pgfqpoint{5.467269in}{2.941003in}}{\pgfqpoint{5.462879in}{2.930404in}}{\pgfqpoint{5.462879in}{2.919353in}}%
\pgfpathcurveto{\pgfqpoint{5.462879in}{2.908303in}}{\pgfqpoint{5.467269in}{2.897704in}}{\pgfqpoint{5.475083in}{2.889891in}}%
\pgfpathcurveto{\pgfqpoint{5.482896in}{2.882077in}}{\pgfqpoint{5.493495in}{2.877687in}}{\pgfqpoint{5.504545in}{2.877687in}}%
\pgfpathclose%
\pgfusepath{stroke,fill}%
\end{pgfscope}%
\begin{pgfscope}%
\pgfpathrectangle{\pgfqpoint{0.800000in}{0.528000in}}{\pgfqpoint{4.960000in}{3.696000in}}%
\pgfusepath{clip}%
\pgfsetbuttcap%
\pgfsetroundjoin%
\definecolor{currentfill}{rgb}{0.000000,0.000000,0.000000}%
\pgfsetfillcolor{currentfill}%
\pgfsetlinewidth{1.003750pt}%
\definecolor{currentstroke}{rgb}{0.000000,0.000000,0.000000}%
\pgfsetstrokecolor{currentstroke}%
\pgfsetdash{}{0pt}%
\pgfpathmoveto{\pgfqpoint{5.504545in}{2.877687in}}%
\pgfpathcurveto{\pgfqpoint{5.515596in}{2.877687in}}{\pgfqpoint{5.526195in}{2.882077in}}{\pgfqpoint{5.534008in}{2.889891in}}%
\pgfpathcurveto{\pgfqpoint{5.541822in}{2.897704in}}{\pgfqpoint{5.546212in}{2.908303in}}{\pgfqpoint{5.546212in}{2.919353in}}%
\pgfpathcurveto{\pgfqpoint{5.546212in}{2.930404in}}{\pgfqpoint{5.541822in}{2.941003in}}{\pgfqpoint{5.534008in}{2.948816in}}%
\pgfpathcurveto{\pgfqpoint{5.526195in}{2.956630in}}{\pgfqpoint{5.515596in}{2.961020in}}{\pgfqpoint{5.504545in}{2.961020in}}%
\pgfpathcurveto{\pgfqpoint{5.493495in}{2.961020in}}{\pgfqpoint{5.482896in}{2.956630in}}{\pgfqpoint{5.475083in}{2.948816in}}%
\pgfpathcurveto{\pgfqpoint{5.467269in}{2.941003in}}{\pgfqpoint{5.462879in}{2.930404in}}{\pgfqpoint{5.462879in}{2.919353in}}%
\pgfpathcurveto{\pgfqpoint{5.462879in}{2.908303in}}{\pgfqpoint{5.467269in}{2.897704in}}{\pgfqpoint{5.475083in}{2.889891in}}%
\pgfpathcurveto{\pgfqpoint{5.482896in}{2.882077in}}{\pgfqpoint{5.493495in}{2.877687in}}{\pgfqpoint{5.504545in}{2.877687in}}%
\pgfpathclose%
\pgfusepath{stroke,fill}%
\end{pgfscope}%
\begin{pgfscope}%
\pgfpathrectangle{\pgfqpoint{0.800000in}{0.528000in}}{\pgfqpoint{4.960000in}{3.696000in}}%
\pgfusepath{clip}%
\pgfsetbuttcap%
\pgfsetroundjoin%
\definecolor{currentfill}{rgb}{0.000000,0.000000,0.000000}%
\pgfsetfillcolor{currentfill}%
\pgfsetlinewidth{1.003750pt}%
\definecolor{currentstroke}{rgb}{0.000000,0.000000,0.000000}%
\pgfsetstrokecolor{currentstroke}%
\pgfsetdash{}{0pt}%
\pgfpathmoveto{\pgfqpoint{5.504545in}{2.877687in}}%
\pgfpathcurveto{\pgfqpoint{5.515596in}{2.877687in}}{\pgfqpoint{5.526195in}{2.882077in}}{\pgfqpoint{5.534008in}{2.889891in}}%
\pgfpathcurveto{\pgfqpoint{5.541822in}{2.897704in}}{\pgfqpoint{5.546212in}{2.908303in}}{\pgfqpoint{5.546212in}{2.919353in}}%
\pgfpathcurveto{\pgfqpoint{5.546212in}{2.930404in}}{\pgfqpoint{5.541822in}{2.941003in}}{\pgfqpoint{5.534008in}{2.948816in}}%
\pgfpathcurveto{\pgfqpoint{5.526195in}{2.956630in}}{\pgfqpoint{5.515596in}{2.961020in}}{\pgfqpoint{5.504545in}{2.961020in}}%
\pgfpathcurveto{\pgfqpoint{5.493495in}{2.961020in}}{\pgfqpoint{5.482896in}{2.956630in}}{\pgfqpoint{5.475083in}{2.948816in}}%
\pgfpathcurveto{\pgfqpoint{5.467269in}{2.941003in}}{\pgfqpoint{5.462879in}{2.930404in}}{\pgfqpoint{5.462879in}{2.919353in}}%
\pgfpathcurveto{\pgfqpoint{5.462879in}{2.908303in}}{\pgfqpoint{5.467269in}{2.897704in}}{\pgfqpoint{5.475083in}{2.889891in}}%
\pgfpathcurveto{\pgfqpoint{5.482896in}{2.882077in}}{\pgfqpoint{5.493495in}{2.877687in}}{\pgfqpoint{5.504545in}{2.877687in}}%
\pgfpathclose%
\pgfusepath{stroke,fill}%
\end{pgfscope}%
\begin{pgfscope}%
\pgfpathrectangle{\pgfqpoint{0.800000in}{0.528000in}}{\pgfqpoint{4.960000in}{3.696000in}}%
\pgfusepath{clip}%
\pgfsetbuttcap%
\pgfsetroundjoin%
\definecolor{currentfill}{rgb}{0.000000,0.000000,0.000000}%
\pgfsetfillcolor{currentfill}%
\pgfsetlinewidth{1.003750pt}%
\definecolor{currentstroke}{rgb}{0.000000,0.000000,0.000000}%
\pgfsetstrokecolor{currentstroke}%
\pgfsetdash{}{0pt}%
\pgfpathmoveto{\pgfqpoint{5.504545in}{2.877687in}}%
\pgfpathcurveto{\pgfqpoint{5.515596in}{2.877687in}}{\pgfqpoint{5.526195in}{2.882077in}}{\pgfqpoint{5.534008in}{2.889891in}}%
\pgfpathcurveto{\pgfqpoint{5.541822in}{2.897704in}}{\pgfqpoint{5.546212in}{2.908303in}}{\pgfqpoint{5.546212in}{2.919353in}}%
\pgfpathcurveto{\pgfqpoint{5.546212in}{2.930404in}}{\pgfqpoint{5.541822in}{2.941003in}}{\pgfqpoint{5.534008in}{2.948816in}}%
\pgfpathcurveto{\pgfqpoint{5.526195in}{2.956630in}}{\pgfqpoint{5.515596in}{2.961020in}}{\pgfqpoint{5.504545in}{2.961020in}}%
\pgfpathcurveto{\pgfqpoint{5.493495in}{2.961020in}}{\pgfqpoint{5.482896in}{2.956630in}}{\pgfqpoint{5.475083in}{2.948816in}}%
\pgfpathcurveto{\pgfqpoint{5.467269in}{2.941003in}}{\pgfqpoint{5.462879in}{2.930404in}}{\pgfqpoint{5.462879in}{2.919353in}}%
\pgfpathcurveto{\pgfqpoint{5.462879in}{2.908303in}}{\pgfqpoint{5.467269in}{2.897704in}}{\pgfqpoint{5.475083in}{2.889891in}}%
\pgfpathcurveto{\pgfqpoint{5.482896in}{2.882077in}}{\pgfqpoint{5.493495in}{2.877687in}}{\pgfqpoint{5.504545in}{2.877687in}}%
\pgfpathclose%
\pgfusepath{stroke,fill}%
\end{pgfscope}%
\begin{pgfscope}%
\pgfpathrectangle{\pgfqpoint{0.800000in}{0.528000in}}{\pgfqpoint{4.960000in}{3.696000in}}%
\pgfusepath{clip}%
\pgfsetbuttcap%
\pgfsetroundjoin%
\definecolor{currentfill}{rgb}{0.000000,0.000000,0.000000}%
\pgfsetfillcolor{currentfill}%
\pgfsetlinewidth{1.003750pt}%
\definecolor{currentstroke}{rgb}{0.000000,0.000000,0.000000}%
\pgfsetstrokecolor{currentstroke}%
\pgfsetdash{}{0pt}%
\pgfpathmoveto{\pgfqpoint{5.504545in}{2.877687in}}%
\pgfpathcurveto{\pgfqpoint{5.515596in}{2.877687in}}{\pgfqpoint{5.526195in}{2.882077in}}{\pgfqpoint{5.534008in}{2.889891in}}%
\pgfpathcurveto{\pgfqpoint{5.541822in}{2.897704in}}{\pgfqpoint{5.546212in}{2.908303in}}{\pgfqpoint{5.546212in}{2.919353in}}%
\pgfpathcurveto{\pgfqpoint{5.546212in}{2.930404in}}{\pgfqpoint{5.541822in}{2.941003in}}{\pgfqpoint{5.534008in}{2.948816in}}%
\pgfpathcurveto{\pgfqpoint{5.526195in}{2.956630in}}{\pgfqpoint{5.515596in}{2.961020in}}{\pgfqpoint{5.504545in}{2.961020in}}%
\pgfpathcurveto{\pgfqpoint{5.493495in}{2.961020in}}{\pgfqpoint{5.482896in}{2.956630in}}{\pgfqpoint{5.475083in}{2.948816in}}%
\pgfpathcurveto{\pgfqpoint{5.467269in}{2.941003in}}{\pgfqpoint{5.462879in}{2.930404in}}{\pgfqpoint{5.462879in}{2.919353in}}%
\pgfpathcurveto{\pgfqpoint{5.462879in}{2.908303in}}{\pgfqpoint{5.467269in}{2.897704in}}{\pgfqpoint{5.475083in}{2.889891in}}%
\pgfpathcurveto{\pgfqpoint{5.482896in}{2.882077in}}{\pgfqpoint{5.493495in}{2.877687in}}{\pgfqpoint{5.504545in}{2.877687in}}%
\pgfpathclose%
\pgfusepath{stroke,fill}%
\end{pgfscope}%
\begin{pgfscope}%
\pgfpathrectangle{\pgfqpoint{0.800000in}{0.528000in}}{\pgfqpoint{4.960000in}{3.696000in}}%
\pgfusepath{clip}%
\pgfsetbuttcap%
\pgfsetroundjoin%
\definecolor{currentfill}{rgb}{0.000000,0.000000,0.000000}%
\pgfsetfillcolor{currentfill}%
\pgfsetlinewidth{1.003750pt}%
\definecolor{currentstroke}{rgb}{0.000000,0.000000,0.000000}%
\pgfsetstrokecolor{currentstroke}%
\pgfsetdash{}{0pt}%
\pgfpathmoveto{\pgfqpoint{5.504545in}{2.877687in}}%
\pgfpathcurveto{\pgfqpoint{5.515596in}{2.877687in}}{\pgfqpoint{5.526195in}{2.882077in}}{\pgfqpoint{5.534008in}{2.889891in}}%
\pgfpathcurveto{\pgfqpoint{5.541822in}{2.897704in}}{\pgfqpoint{5.546212in}{2.908303in}}{\pgfqpoint{5.546212in}{2.919353in}}%
\pgfpathcurveto{\pgfqpoint{5.546212in}{2.930404in}}{\pgfqpoint{5.541822in}{2.941003in}}{\pgfqpoint{5.534008in}{2.948816in}}%
\pgfpathcurveto{\pgfqpoint{5.526195in}{2.956630in}}{\pgfqpoint{5.515596in}{2.961020in}}{\pgfqpoint{5.504545in}{2.961020in}}%
\pgfpathcurveto{\pgfqpoint{5.493495in}{2.961020in}}{\pgfqpoint{5.482896in}{2.956630in}}{\pgfqpoint{5.475083in}{2.948816in}}%
\pgfpathcurveto{\pgfqpoint{5.467269in}{2.941003in}}{\pgfqpoint{5.462879in}{2.930404in}}{\pgfqpoint{5.462879in}{2.919353in}}%
\pgfpathcurveto{\pgfqpoint{5.462879in}{2.908303in}}{\pgfqpoint{5.467269in}{2.897704in}}{\pgfqpoint{5.475083in}{2.889891in}}%
\pgfpathcurveto{\pgfqpoint{5.482896in}{2.882077in}}{\pgfqpoint{5.493495in}{2.877687in}}{\pgfqpoint{5.504545in}{2.877687in}}%
\pgfpathclose%
\pgfusepath{stroke,fill}%
\end{pgfscope}%
\begin{pgfscope}%
\pgfpathrectangle{\pgfqpoint{0.800000in}{0.528000in}}{\pgfqpoint{4.960000in}{3.696000in}}%
\pgfusepath{clip}%
\pgfsetbuttcap%
\pgfsetroundjoin%
\definecolor{currentfill}{rgb}{0.000000,0.000000,0.000000}%
\pgfsetfillcolor{currentfill}%
\pgfsetlinewidth{1.003750pt}%
\definecolor{currentstroke}{rgb}{0.000000,0.000000,0.000000}%
\pgfsetstrokecolor{currentstroke}%
\pgfsetdash{}{0pt}%
\pgfpathmoveto{\pgfqpoint{5.504545in}{2.877687in}}%
\pgfpathcurveto{\pgfqpoint{5.515596in}{2.877687in}}{\pgfqpoint{5.526195in}{2.882077in}}{\pgfqpoint{5.534008in}{2.889891in}}%
\pgfpathcurveto{\pgfqpoint{5.541822in}{2.897704in}}{\pgfqpoint{5.546212in}{2.908303in}}{\pgfqpoint{5.546212in}{2.919353in}}%
\pgfpathcurveto{\pgfqpoint{5.546212in}{2.930404in}}{\pgfqpoint{5.541822in}{2.941003in}}{\pgfqpoint{5.534008in}{2.948816in}}%
\pgfpathcurveto{\pgfqpoint{5.526195in}{2.956630in}}{\pgfqpoint{5.515596in}{2.961020in}}{\pgfqpoint{5.504545in}{2.961020in}}%
\pgfpathcurveto{\pgfqpoint{5.493495in}{2.961020in}}{\pgfqpoint{5.482896in}{2.956630in}}{\pgfqpoint{5.475083in}{2.948816in}}%
\pgfpathcurveto{\pgfqpoint{5.467269in}{2.941003in}}{\pgfqpoint{5.462879in}{2.930404in}}{\pgfqpoint{5.462879in}{2.919353in}}%
\pgfpathcurveto{\pgfqpoint{5.462879in}{2.908303in}}{\pgfqpoint{5.467269in}{2.897704in}}{\pgfqpoint{5.475083in}{2.889891in}}%
\pgfpathcurveto{\pgfqpoint{5.482896in}{2.882077in}}{\pgfqpoint{5.493495in}{2.877687in}}{\pgfqpoint{5.504545in}{2.877687in}}%
\pgfpathclose%
\pgfusepath{stroke,fill}%
\end{pgfscope}%
\begin{pgfscope}%
\pgfpathrectangle{\pgfqpoint{0.800000in}{0.528000in}}{\pgfqpoint{4.960000in}{3.696000in}}%
\pgfusepath{clip}%
\pgfsetbuttcap%
\pgfsetroundjoin%
\definecolor{currentfill}{rgb}{0.000000,0.000000,0.000000}%
\pgfsetfillcolor{currentfill}%
\pgfsetlinewidth{1.003750pt}%
\definecolor{currentstroke}{rgb}{0.000000,0.000000,0.000000}%
\pgfsetstrokecolor{currentstroke}%
\pgfsetdash{}{0pt}%
\pgfpathmoveto{\pgfqpoint{5.504545in}{2.877687in}}%
\pgfpathcurveto{\pgfqpoint{5.515596in}{2.877687in}}{\pgfqpoint{5.526195in}{2.882077in}}{\pgfqpoint{5.534008in}{2.889891in}}%
\pgfpathcurveto{\pgfqpoint{5.541822in}{2.897704in}}{\pgfqpoint{5.546212in}{2.908303in}}{\pgfqpoint{5.546212in}{2.919353in}}%
\pgfpathcurveto{\pgfqpoint{5.546212in}{2.930404in}}{\pgfqpoint{5.541822in}{2.941003in}}{\pgfqpoint{5.534008in}{2.948816in}}%
\pgfpathcurveto{\pgfqpoint{5.526195in}{2.956630in}}{\pgfqpoint{5.515596in}{2.961020in}}{\pgfqpoint{5.504545in}{2.961020in}}%
\pgfpathcurveto{\pgfqpoint{5.493495in}{2.961020in}}{\pgfqpoint{5.482896in}{2.956630in}}{\pgfqpoint{5.475083in}{2.948816in}}%
\pgfpathcurveto{\pgfqpoint{5.467269in}{2.941003in}}{\pgfqpoint{5.462879in}{2.930404in}}{\pgfqpoint{5.462879in}{2.919353in}}%
\pgfpathcurveto{\pgfqpoint{5.462879in}{2.908303in}}{\pgfqpoint{5.467269in}{2.897704in}}{\pgfqpoint{5.475083in}{2.889891in}}%
\pgfpathcurveto{\pgfqpoint{5.482896in}{2.882077in}}{\pgfqpoint{5.493495in}{2.877687in}}{\pgfqpoint{5.504545in}{2.877687in}}%
\pgfpathclose%
\pgfusepath{stroke,fill}%
\end{pgfscope}%
\begin{pgfscope}%
\pgfpathrectangle{\pgfqpoint{0.800000in}{0.528000in}}{\pgfqpoint{4.960000in}{3.696000in}}%
\pgfusepath{clip}%
\pgfsetbuttcap%
\pgfsetroundjoin%
\definecolor{currentfill}{rgb}{0.000000,0.000000,0.000000}%
\pgfsetfillcolor{currentfill}%
\pgfsetlinewidth{1.003750pt}%
\definecolor{currentstroke}{rgb}{0.000000,0.000000,0.000000}%
\pgfsetstrokecolor{currentstroke}%
\pgfsetdash{}{0pt}%
\pgfpathmoveto{\pgfqpoint{5.504545in}{2.877687in}}%
\pgfpathcurveto{\pgfqpoint{5.515596in}{2.877687in}}{\pgfqpoint{5.526195in}{2.882077in}}{\pgfqpoint{5.534008in}{2.889891in}}%
\pgfpathcurveto{\pgfqpoint{5.541822in}{2.897704in}}{\pgfqpoint{5.546212in}{2.908303in}}{\pgfqpoint{5.546212in}{2.919353in}}%
\pgfpathcurveto{\pgfqpoint{5.546212in}{2.930404in}}{\pgfqpoint{5.541822in}{2.941003in}}{\pgfqpoint{5.534008in}{2.948816in}}%
\pgfpathcurveto{\pgfqpoint{5.526195in}{2.956630in}}{\pgfqpoint{5.515596in}{2.961020in}}{\pgfqpoint{5.504545in}{2.961020in}}%
\pgfpathcurveto{\pgfqpoint{5.493495in}{2.961020in}}{\pgfqpoint{5.482896in}{2.956630in}}{\pgfqpoint{5.475083in}{2.948816in}}%
\pgfpathcurveto{\pgfqpoint{5.467269in}{2.941003in}}{\pgfqpoint{5.462879in}{2.930404in}}{\pgfqpoint{5.462879in}{2.919353in}}%
\pgfpathcurveto{\pgfqpoint{5.462879in}{2.908303in}}{\pgfqpoint{5.467269in}{2.897704in}}{\pgfqpoint{5.475083in}{2.889891in}}%
\pgfpathcurveto{\pgfqpoint{5.482896in}{2.882077in}}{\pgfqpoint{5.493495in}{2.877687in}}{\pgfqpoint{5.504545in}{2.877687in}}%
\pgfpathclose%
\pgfusepath{stroke,fill}%
\end{pgfscope}%
\begin{pgfscope}%
\pgfpathrectangle{\pgfqpoint{0.800000in}{0.528000in}}{\pgfqpoint{4.960000in}{3.696000in}}%
\pgfusepath{clip}%
\pgfsetbuttcap%
\pgfsetroundjoin%
\definecolor{currentfill}{rgb}{0.000000,0.000000,0.000000}%
\pgfsetfillcolor{currentfill}%
\pgfsetlinewidth{1.003750pt}%
\definecolor{currentstroke}{rgb}{0.000000,0.000000,0.000000}%
\pgfsetstrokecolor{currentstroke}%
\pgfsetdash{}{0pt}%
\pgfpathmoveto{\pgfqpoint{5.504545in}{2.877687in}}%
\pgfpathcurveto{\pgfqpoint{5.515596in}{2.877687in}}{\pgfqpoint{5.526195in}{2.882077in}}{\pgfqpoint{5.534008in}{2.889891in}}%
\pgfpathcurveto{\pgfqpoint{5.541822in}{2.897704in}}{\pgfqpoint{5.546212in}{2.908303in}}{\pgfqpoint{5.546212in}{2.919353in}}%
\pgfpathcurveto{\pgfqpoint{5.546212in}{2.930404in}}{\pgfqpoint{5.541822in}{2.941003in}}{\pgfqpoint{5.534008in}{2.948816in}}%
\pgfpathcurveto{\pgfqpoint{5.526195in}{2.956630in}}{\pgfqpoint{5.515596in}{2.961020in}}{\pgfqpoint{5.504545in}{2.961020in}}%
\pgfpathcurveto{\pgfqpoint{5.493495in}{2.961020in}}{\pgfqpoint{5.482896in}{2.956630in}}{\pgfqpoint{5.475083in}{2.948816in}}%
\pgfpathcurveto{\pgfqpoint{5.467269in}{2.941003in}}{\pgfqpoint{5.462879in}{2.930404in}}{\pgfqpoint{5.462879in}{2.919353in}}%
\pgfpathcurveto{\pgfqpoint{5.462879in}{2.908303in}}{\pgfqpoint{5.467269in}{2.897704in}}{\pgfqpoint{5.475083in}{2.889891in}}%
\pgfpathcurveto{\pgfqpoint{5.482896in}{2.882077in}}{\pgfqpoint{5.493495in}{2.877687in}}{\pgfqpoint{5.504545in}{2.877687in}}%
\pgfpathclose%
\pgfusepath{stroke,fill}%
\end{pgfscope}%
\begin{pgfscope}%
\pgfpathrectangle{\pgfqpoint{0.800000in}{0.528000in}}{\pgfqpoint{4.960000in}{3.696000in}}%
\pgfusepath{clip}%
\pgfsetbuttcap%
\pgfsetroundjoin%
\definecolor{currentfill}{rgb}{0.000000,0.000000,0.000000}%
\pgfsetfillcolor{currentfill}%
\pgfsetlinewidth{1.003750pt}%
\definecolor{currentstroke}{rgb}{0.000000,0.000000,0.000000}%
\pgfsetstrokecolor{currentstroke}%
\pgfsetdash{}{0pt}%
\pgfpathmoveto{\pgfqpoint{5.504545in}{2.877687in}}%
\pgfpathcurveto{\pgfqpoint{5.515596in}{2.877687in}}{\pgfqpoint{5.526195in}{2.882077in}}{\pgfqpoint{5.534008in}{2.889891in}}%
\pgfpathcurveto{\pgfqpoint{5.541822in}{2.897704in}}{\pgfqpoint{5.546212in}{2.908303in}}{\pgfqpoint{5.546212in}{2.919353in}}%
\pgfpathcurveto{\pgfqpoint{5.546212in}{2.930404in}}{\pgfqpoint{5.541822in}{2.941003in}}{\pgfqpoint{5.534008in}{2.948816in}}%
\pgfpathcurveto{\pgfqpoint{5.526195in}{2.956630in}}{\pgfqpoint{5.515596in}{2.961020in}}{\pgfqpoint{5.504545in}{2.961020in}}%
\pgfpathcurveto{\pgfqpoint{5.493495in}{2.961020in}}{\pgfqpoint{5.482896in}{2.956630in}}{\pgfqpoint{5.475083in}{2.948816in}}%
\pgfpathcurveto{\pgfqpoint{5.467269in}{2.941003in}}{\pgfqpoint{5.462879in}{2.930404in}}{\pgfqpoint{5.462879in}{2.919353in}}%
\pgfpathcurveto{\pgfqpoint{5.462879in}{2.908303in}}{\pgfqpoint{5.467269in}{2.897704in}}{\pgfqpoint{5.475083in}{2.889891in}}%
\pgfpathcurveto{\pgfqpoint{5.482896in}{2.882077in}}{\pgfqpoint{5.493495in}{2.877687in}}{\pgfqpoint{5.504545in}{2.877687in}}%
\pgfpathclose%
\pgfusepath{stroke,fill}%
\end{pgfscope}%
\begin{pgfscope}%
\pgfpathrectangle{\pgfqpoint{0.800000in}{0.528000in}}{\pgfqpoint{4.960000in}{3.696000in}}%
\pgfusepath{clip}%
\pgfsetbuttcap%
\pgfsetroundjoin%
\definecolor{currentfill}{rgb}{0.000000,0.000000,0.000000}%
\pgfsetfillcolor{currentfill}%
\pgfsetlinewidth{1.003750pt}%
\definecolor{currentstroke}{rgb}{0.000000,0.000000,0.000000}%
\pgfsetstrokecolor{currentstroke}%
\pgfsetdash{}{0pt}%
\pgfpathmoveto{\pgfqpoint{5.504545in}{3.984333in}}%
\pgfpathcurveto{\pgfqpoint{5.515596in}{3.984333in}}{\pgfqpoint{5.526195in}{3.988724in}}{\pgfqpoint{5.534008in}{3.996537in}}%
\pgfpathcurveto{\pgfqpoint{5.541822in}{4.004351in}}{\pgfqpoint{5.546212in}{4.014950in}}{\pgfqpoint{5.546212in}{4.026000in}}%
\pgfpathcurveto{\pgfqpoint{5.546212in}{4.037050in}}{\pgfqpoint{5.541822in}{4.047649in}}{\pgfqpoint{5.534008in}{4.055463in}}%
\pgfpathcurveto{\pgfqpoint{5.526195in}{4.063276in}}{\pgfqpoint{5.515596in}{4.067667in}}{\pgfqpoint{5.504545in}{4.067667in}}%
\pgfpathcurveto{\pgfqpoint{5.493495in}{4.067667in}}{\pgfqpoint{5.482896in}{4.063276in}}{\pgfqpoint{5.475083in}{4.055463in}}%
\pgfpathcurveto{\pgfqpoint{5.467269in}{4.047649in}}{\pgfqpoint{5.462879in}{4.037050in}}{\pgfqpoint{5.462879in}{4.026000in}}%
\pgfpathcurveto{\pgfqpoint{5.462879in}{4.014950in}}{\pgfqpoint{5.467269in}{4.004351in}}{\pgfqpoint{5.475083in}{3.996537in}}%
\pgfpathcurveto{\pgfqpoint{5.482896in}{3.988724in}}{\pgfqpoint{5.493495in}{3.984333in}}{\pgfqpoint{5.504545in}{3.984333in}}%
\pgfpathclose%
\pgfusepath{stroke,fill}%
\end{pgfscope}%
\begin{pgfscope}%
\pgfpathrectangle{\pgfqpoint{0.800000in}{0.528000in}}{\pgfqpoint{4.960000in}{3.696000in}}%
\pgfusepath{clip}%
\pgfsetbuttcap%
\pgfsetroundjoin%
\definecolor{currentfill}{rgb}{0.000000,0.000000,0.000000}%
\pgfsetfillcolor{currentfill}%
\pgfsetlinewidth{1.003750pt}%
\definecolor{currentstroke}{rgb}{0.000000,0.000000,0.000000}%
\pgfsetstrokecolor{currentstroke}%
\pgfsetdash{}{0pt}%
\pgfpathmoveto{\pgfqpoint{5.504545in}{2.877687in}}%
\pgfpathcurveto{\pgfqpoint{5.515596in}{2.877687in}}{\pgfqpoint{5.526195in}{2.882077in}}{\pgfqpoint{5.534008in}{2.889891in}}%
\pgfpathcurveto{\pgfqpoint{5.541822in}{2.897704in}}{\pgfqpoint{5.546212in}{2.908303in}}{\pgfqpoint{5.546212in}{2.919353in}}%
\pgfpathcurveto{\pgfqpoint{5.546212in}{2.930404in}}{\pgfqpoint{5.541822in}{2.941003in}}{\pgfqpoint{5.534008in}{2.948816in}}%
\pgfpathcurveto{\pgfqpoint{5.526195in}{2.956630in}}{\pgfqpoint{5.515596in}{2.961020in}}{\pgfqpoint{5.504545in}{2.961020in}}%
\pgfpathcurveto{\pgfqpoint{5.493495in}{2.961020in}}{\pgfqpoint{5.482896in}{2.956630in}}{\pgfqpoint{5.475083in}{2.948816in}}%
\pgfpathcurveto{\pgfqpoint{5.467269in}{2.941003in}}{\pgfqpoint{5.462879in}{2.930404in}}{\pgfqpoint{5.462879in}{2.919353in}}%
\pgfpathcurveto{\pgfqpoint{5.462879in}{2.908303in}}{\pgfqpoint{5.467269in}{2.897704in}}{\pgfqpoint{5.475083in}{2.889891in}}%
\pgfpathcurveto{\pgfqpoint{5.482896in}{2.882077in}}{\pgfqpoint{5.493495in}{2.877687in}}{\pgfqpoint{5.504545in}{2.877687in}}%
\pgfpathclose%
\pgfusepath{stroke,fill}%
\end{pgfscope}%
\begin{pgfscope}%
\pgfpathrectangle{\pgfqpoint{0.800000in}{0.528000in}}{\pgfqpoint{4.960000in}{3.696000in}}%
\pgfusepath{clip}%
\pgfsetbuttcap%
\pgfsetroundjoin%
\definecolor{currentfill}{rgb}{0.000000,0.000000,0.000000}%
\pgfsetfillcolor{currentfill}%
\pgfsetlinewidth{1.003750pt}%
\definecolor{currentstroke}{rgb}{0.000000,0.000000,0.000000}%
\pgfsetstrokecolor{currentstroke}%
\pgfsetdash{}{0pt}%
\pgfpathmoveto{\pgfqpoint{5.504545in}{2.877687in}}%
\pgfpathcurveto{\pgfqpoint{5.515596in}{2.877687in}}{\pgfqpoint{5.526195in}{2.882077in}}{\pgfqpoint{5.534008in}{2.889891in}}%
\pgfpathcurveto{\pgfqpoint{5.541822in}{2.897704in}}{\pgfqpoint{5.546212in}{2.908303in}}{\pgfqpoint{5.546212in}{2.919353in}}%
\pgfpathcurveto{\pgfqpoint{5.546212in}{2.930404in}}{\pgfqpoint{5.541822in}{2.941003in}}{\pgfqpoint{5.534008in}{2.948816in}}%
\pgfpathcurveto{\pgfqpoint{5.526195in}{2.956630in}}{\pgfqpoint{5.515596in}{2.961020in}}{\pgfqpoint{5.504545in}{2.961020in}}%
\pgfpathcurveto{\pgfqpoint{5.493495in}{2.961020in}}{\pgfqpoint{5.482896in}{2.956630in}}{\pgfqpoint{5.475083in}{2.948816in}}%
\pgfpathcurveto{\pgfqpoint{5.467269in}{2.941003in}}{\pgfqpoint{5.462879in}{2.930404in}}{\pgfqpoint{5.462879in}{2.919353in}}%
\pgfpathcurveto{\pgfqpoint{5.462879in}{2.908303in}}{\pgfqpoint{5.467269in}{2.897704in}}{\pgfqpoint{5.475083in}{2.889891in}}%
\pgfpathcurveto{\pgfqpoint{5.482896in}{2.882077in}}{\pgfqpoint{5.493495in}{2.877687in}}{\pgfqpoint{5.504545in}{2.877687in}}%
\pgfpathclose%
\pgfusepath{stroke,fill}%
\end{pgfscope}%
\begin{pgfscope}%
\pgfpathrectangle{\pgfqpoint{0.800000in}{0.528000in}}{\pgfqpoint{4.960000in}{3.696000in}}%
\pgfusepath{clip}%
\pgfsetbuttcap%
\pgfsetroundjoin%
\definecolor{currentfill}{rgb}{0.000000,0.000000,0.000000}%
\pgfsetfillcolor{currentfill}%
\pgfsetlinewidth{1.003750pt}%
\definecolor{currentstroke}{rgb}{0.000000,0.000000,0.000000}%
\pgfsetstrokecolor{currentstroke}%
\pgfsetdash{}{0pt}%
\pgfpathmoveto{\pgfqpoint{5.504545in}{3.984333in}}%
\pgfpathcurveto{\pgfqpoint{5.515596in}{3.984333in}}{\pgfqpoint{5.526195in}{3.988724in}}{\pgfqpoint{5.534008in}{3.996537in}}%
\pgfpathcurveto{\pgfqpoint{5.541822in}{4.004351in}}{\pgfqpoint{5.546212in}{4.014950in}}{\pgfqpoint{5.546212in}{4.026000in}}%
\pgfpathcurveto{\pgfqpoint{5.546212in}{4.037050in}}{\pgfqpoint{5.541822in}{4.047649in}}{\pgfqpoint{5.534008in}{4.055463in}}%
\pgfpathcurveto{\pgfqpoint{5.526195in}{4.063276in}}{\pgfqpoint{5.515596in}{4.067667in}}{\pgfqpoint{5.504545in}{4.067667in}}%
\pgfpathcurveto{\pgfqpoint{5.493495in}{4.067667in}}{\pgfqpoint{5.482896in}{4.063276in}}{\pgfqpoint{5.475083in}{4.055463in}}%
\pgfpathcurveto{\pgfqpoint{5.467269in}{4.047649in}}{\pgfqpoint{5.462879in}{4.037050in}}{\pgfqpoint{5.462879in}{4.026000in}}%
\pgfpathcurveto{\pgfqpoint{5.462879in}{4.014950in}}{\pgfqpoint{5.467269in}{4.004351in}}{\pgfqpoint{5.475083in}{3.996537in}}%
\pgfpathcurveto{\pgfqpoint{5.482896in}{3.988724in}}{\pgfqpoint{5.493495in}{3.984333in}}{\pgfqpoint{5.504545in}{3.984333in}}%
\pgfpathclose%
\pgfusepath{stroke,fill}%
\end{pgfscope}%
\begin{pgfscope}%
\pgfpathrectangle{\pgfqpoint{0.800000in}{0.528000in}}{\pgfqpoint{4.960000in}{3.696000in}}%
\pgfusepath{clip}%
\pgfsetbuttcap%
\pgfsetroundjoin%
\definecolor{currentfill}{rgb}{0.000000,0.000000,0.000000}%
\pgfsetfillcolor{currentfill}%
\pgfsetlinewidth{1.003750pt}%
\definecolor{currentstroke}{rgb}{0.000000,0.000000,0.000000}%
\pgfsetstrokecolor{currentstroke}%
\pgfsetdash{}{0pt}%
\pgfpathmoveto{\pgfqpoint{5.504545in}{2.877687in}}%
\pgfpathcurveto{\pgfqpoint{5.515596in}{2.877687in}}{\pgfqpoint{5.526195in}{2.882077in}}{\pgfqpoint{5.534008in}{2.889891in}}%
\pgfpathcurveto{\pgfqpoint{5.541822in}{2.897704in}}{\pgfqpoint{5.546212in}{2.908303in}}{\pgfqpoint{5.546212in}{2.919353in}}%
\pgfpathcurveto{\pgfqpoint{5.546212in}{2.930404in}}{\pgfqpoint{5.541822in}{2.941003in}}{\pgfqpoint{5.534008in}{2.948816in}}%
\pgfpathcurveto{\pgfqpoint{5.526195in}{2.956630in}}{\pgfqpoint{5.515596in}{2.961020in}}{\pgfqpoint{5.504545in}{2.961020in}}%
\pgfpathcurveto{\pgfqpoint{5.493495in}{2.961020in}}{\pgfqpoint{5.482896in}{2.956630in}}{\pgfqpoint{5.475083in}{2.948816in}}%
\pgfpathcurveto{\pgfqpoint{5.467269in}{2.941003in}}{\pgfqpoint{5.462879in}{2.930404in}}{\pgfqpoint{5.462879in}{2.919353in}}%
\pgfpathcurveto{\pgfqpoint{5.462879in}{2.908303in}}{\pgfqpoint{5.467269in}{2.897704in}}{\pgfqpoint{5.475083in}{2.889891in}}%
\pgfpathcurveto{\pgfqpoint{5.482896in}{2.882077in}}{\pgfqpoint{5.493495in}{2.877687in}}{\pgfqpoint{5.504545in}{2.877687in}}%
\pgfpathclose%
\pgfusepath{stroke,fill}%
\end{pgfscope}%
\begin{pgfscope}%
\pgfpathrectangle{\pgfqpoint{0.800000in}{0.528000in}}{\pgfqpoint{4.960000in}{3.696000in}}%
\pgfusepath{clip}%
\pgfsetbuttcap%
\pgfsetroundjoin%
\definecolor{currentfill}{rgb}{0.000000,0.000000,0.000000}%
\pgfsetfillcolor{currentfill}%
\pgfsetlinewidth{1.003750pt}%
\definecolor{currentstroke}{rgb}{0.000000,0.000000,0.000000}%
\pgfsetstrokecolor{currentstroke}%
\pgfsetdash{}{0pt}%
\pgfpathmoveto{\pgfqpoint{5.504545in}{2.877687in}}%
\pgfpathcurveto{\pgfqpoint{5.515596in}{2.877687in}}{\pgfqpoint{5.526195in}{2.882077in}}{\pgfqpoint{5.534008in}{2.889891in}}%
\pgfpathcurveto{\pgfqpoint{5.541822in}{2.897704in}}{\pgfqpoint{5.546212in}{2.908303in}}{\pgfqpoint{5.546212in}{2.919353in}}%
\pgfpathcurveto{\pgfqpoint{5.546212in}{2.930404in}}{\pgfqpoint{5.541822in}{2.941003in}}{\pgfqpoint{5.534008in}{2.948816in}}%
\pgfpathcurveto{\pgfqpoint{5.526195in}{2.956630in}}{\pgfqpoint{5.515596in}{2.961020in}}{\pgfqpoint{5.504545in}{2.961020in}}%
\pgfpathcurveto{\pgfqpoint{5.493495in}{2.961020in}}{\pgfqpoint{5.482896in}{2.956630in}}{\pgfqpoint{5.475083in}{2.948816in}}%
\pgfpathcurveto{\pgfqpoint{5.467269in}{2.941003in}}{\pgfqpoint{5.462879in}{2.930404in}}{\pgfqpoint{5.462879in}{2.919353in}}%
\pgfpathcurveto{\pgfqpoint{5.462879in}{2.908303in}}{\pgfqpoint{5.467269in}{2.897704in}}{\pgfqpoint{5.475083in}{2.889891in}}%
\pgfpathcurveto{\pgfqpoint{5.482896in}{2.882077in}}{\pgfqpoint{5.493495in}{2.877687in}}{\pgfqpoint{5.504545in}{2.877687in}}%
\pgfpathclose%
\pgfusepath{stroke,fill}%
\end{pgfscope}%
\begin{pgfscope}%
\pgfpathrectangle{\pgfqpoint{0.800000in}{0.528000in}}{\pgfqpoint{4.960000in}{3.696000in}}%
\pgfusepath{clip}%
\pgfsetbuttcap%
\pgfsetroundjoin%
\definecolor{currentfill}{rgb}{0.000000,0.000000,0.000000}%
\pgfsetfillcolor{currentfill}%
\pgfsetlinewidth{1.003750pt}%
\definecolor{currentstroke}{rgb}{0.000000,0.000000,0.000000}%
\pgfsetstrokecolor{currentstroke}%
\pgfsetdash{}{0pt}%
\pgfpathmoveto{\pgfqpoint{5.504545in}{2.877687in}}%
\pgfpathcurveto{\pgfqpoint{5.515596in}{2.877687in}}{\pgfqpoint{5.526195in}{2.882077in}}{\pgfqpoint{5.534008in}{2.889891in}}%
\pgfpathcurveto{\pgfqpoint{5.541822in}{2.897704in}}{\pgfqpoint{5.546212in}{2.908303in}}{\pgfqpoint{5.546212in}{2.919353in}}%
\pgfpathcurveto{\pgfqpoint{5.546212in}{2.930404in}}{\pgfqpoint{5.541822in}{2.941003in}}{\pgfqpoint{5.534008in}{2.948816in}}%
\pgfpathcurveto{\pgfqpoint{5.526195in}{2.956630in}}{\pgfqpoint{5.515596in}{2.961020in}}{\pgfqpoint{5.504545in}{2.961020in}}%
\pgfpathcurveto{\pgfqpoint{5.493495in}{2.961020in}}{\pgfqpoint{5.482896in}{2.956630in}}{\pgfqpoint{5.475083in}{2.948816in}}%
\pgfpathcurveto{\pgfqpoint{5.467269in}{2.941003in}}{\pgfqpoint{5.462879in}{2.930404in}}{\pgfqpoint{5.462879in}{2.919353in}}%
\pgfpathcurveto{\pgfqpoint{5.462879in}{2.908303in}}{\pgfqpoint{5.467269in}{2.897704in}}{\pgfqpoint{5.475083in}{2.889891in}}%
\pgfpathcurveto{\pgfqpoint{5.482896in}{2.882077in}}{\pgfqpoint{5.493495in}{2.877687in}}{\pgfqpoint{5.504545in}{2.877687in}}%
\pgfpathclose%
\pgfusepath{stroke,fill}%
\end{pgfscope}%
\begin{pgfscope}%
\pgfpathrectangle{\pgfqpoint{0.800000in}{0.528000in}}{\pgfqpoint{4.960000in}{3.696000in}}%
\pgfusepath{clip}%
\pgfsetbuttcap%
\pgfsetroundjoin%
\definecolor{currentfill}{rgb}{0.000000,0.000000,0.000000}%
\pgfsetfillcolor{currentfill}%
\pgfsetlinewidth{1.003750pt}%
\definecolor{currentstroke}{rgb}{0.000000,0.000000,0.000000}%
\pgfsetstrokecolor{currentstroke}%
\pgfsetdash{}{0pt}%
\pgfpathmoveto{\pgfqpoint{5.504545in}{2.877687in}}%
\pgfpathcurveto{\pgfqpoint{5.515596in}{2.877687in}}{\pgfqpoint{5.526195in}{2.882077in}}{\pgfqpoint{5.534008in}{2.889891in}}%
\pgfpathcurveto{\pgfqpoint{5.541822in}{2.897704in}}{\pgfqpoint{5.546212in}{2.908303in}}{\pgfqpoint{5.546212in}{2.919353in}}%
\pgfpathcurveto{\pgfqpoint{5.546212in}{2.930404in}}{\pgfqpoint{5.541822in}{2.941003in}}{\pgfqpoint{5.534008in}{2.948816in}}%
\pgfpathcurveto{\pgfqpoint{5.526195in}{2.956630in}}{\pgfqpoint{5.515596in}{2.961020in}}{\pgfqpoint{5.504545in}{2.961020in}}%
\pgfpathcurveto{\pgfqpoint{5.493495in}{2.961020in}}{\pgfqpoint{5.482896in}{2.956630in}}{\pgfqpoint{5.475083in}{2.948816in}}%
\pgfpathcurveto{\pgfqpoint{5.467269in}{2.941003in}}{\pgfqpoint{5.462879in}{2.930404in}}{\pgfqpoint{5.462879in}{2.919353in}}%
\pgfpathcurveto{\pgfqpoint{5.462879in}{2.908303in}}{\pgfqpoint{5.467269in}{2.897704in}}{\pgfqpoint{5.475083in}{2.889891in}}%
\pgfpathcurveto{\pgfqpoint{5.482896in}{2.882077in}}{\pgfqpoint{5.493495in}{2.877687in}}{\pgfqpoint{5.504545in}{2.877687in}}%
\pgfpathclose%
\pgfusepath{stroke,fill}%
\end{pgfscope}%
\begin{pgfscope}%
\pgfpathrectangle{\pgfqpoint{0.800000in}{0.528000in}}{\pgfqpoint{4.960000in}{3.696000in}}%
\pgfusepath{clip}%
\pgfsetbuttcap%
\pgfsetroundjoin%
\definecolor{currentfill}{rgb}{0.000000,0.000000,0.000000}%
\pgfsetfillcolor{currentfill}%
\pgfsetlinewidth{1.003750pt}%
\definecolor{currentstroke}{rgb}{0.000000,0.000000,0.000000}%
\pgfsetstrokecolor{currentstroke}%
\pgfsetdash{}{0pt}%
\pgfpathmoveto{\pgfqpoint{5.504545in}{2.877687in}}%
\pgfpathcurveto{\pgfqpoint{5.515596in}{2.877687in}}{\pgfqpoint{5.526195in}{2.882077in}}{\pgfqpoint{5.534008in}{2.889891in}}%
\pgfpathcurveto{\pgfqpoint{5.541822in}{2.897704in}}{\pgfqpoint{5.546212in}{2.908303in}}{\pgfqpoint{5.546212in}{2.919353in}}%
\pgfpathcurveto{\pgfqpoint{5.546212in}{2.930404in}}{\pgfqpoint{5.541822in}{2.941003in}}{\pgfqpoint{5.534008in}{2.948816in}}%
\pgfpathcurveto{\pgfqpoint{5.526195in}{2.956630in}}{\pgfqpoint{5.515596in}{2.961020in}}{\pgfqpoint{5.504545in}{2.961020in}}%
\pgfpathcurveto{\pgfqpoint{5.493495in}{2.961020in}}{\pgfqpoint{5.482896in}{2.956630in}}{\pgfqpoint{5.475083in}{2.948816in}}%
\pgfpathcurveto{\pgfqpoint{5.467269in}{2.941003in}}{\pgfqpoint{5.462879in}{2.930404in}}{\pgfqpoint{5.462879in}{2.919353in}}%
\pgfpathcurveto{\pgfqpoint{5.462879in}{2.908303in}}{\pgfqpoint{5.467269in}{2.897704in}}{\pgfqpoint{5.475083in}{2.889891in}}%
\pgfpathcurveto{\pgfqpoint{5.482896in}{2.882077in}}{\pgfqpoint{5.493495in}{2.877687in}}{\pgfqpoint{5.504545in}{2.877687in}}%
\pgfpathclose%
\pgfusepath{stroke,fill}%
\end{pgfscope}%
\begin{pgfscope}%
\pgfpathrectangle{\pgfqpoint{0.800000in}{0.528000in}}{\pgfqpoint{4.960000in}{3.696000in}}%
\pgfusepath{clip}%
\pgfsetbuttcap%
\pgfsetroundjoin%
\definecolor{currentfill}{rgb}{0.000000,0.000000,0.000000}%
\pgfsetfillcolor{currentfill}%
\pgfsetlinewidth{1.003750pt}%
\definecolor{currentstroke}{rgb}{0.000000,0.000000,0.000000}%
\pgfsetstrokecolor{currentstroke}%
\pgfsetdash{}{0pt}%
\pgfpathmoveto{\pgfqpoint{5.504545in}{2.877687in}}%
\pgfpathcurveto{\pgfqpoint{5.515596in}{2.877687in}}{\pgfqpoint{5.526195in}{2.882077in}}{\pgfqpoint{5.534008in}{2.889891in}}%
\pgfpathcurveto{\pgfqpoint{5.541822in}{2.897704in}}{\pgfqpoint{5.546212in}{2.908303in}}{\pgfqpoint{5.546212in}{2.919353in}}%
\pgfpathcurveto{\pgfqpoint{5.546212in}{2.930404in}}{\pgfqpoint{5.541822in}{2.941003in}}{\pgfqpoint{5.534008in}{2.948816in}}%
\pgfpathcurveto{\pgfqpoint{5.526195in}{2.956630in}}{\pgfqpoint{5.515596in}{2.961020in}}{\pgfqpoint{5.504545in}{2.961020in}}%
\pgfpathcurveto{\pgfqpoint{5.493495in}{2.961020in}}{\pgfqpoint{5.482896in}{2.956630in}}{\pgfqpoint{5.475083in}{2.948816in}}%
\pgfpathcurveto{\pgfqpoint{5.467269in}{2.941003in}}{\pgfqpoint{5.462879in}{2.930404in}}{\pgfqpoint{5.462879in}{2.919353in}}%
\pgfpathcurveto{\pgfqpoint{5.462879in}{2.908303in}}{\pgfqpoint{5.467269in}{2.897704in}}{\pgfqpoint{5.475083in}{2.889891in}}%
\pgfpathcurveto{\pgfqpoint{5.482896in}{2.882077in}}{\pgfqpoint{5.493495in}{2.877687in}}{\pgfqpoint{5.504545in}{2.877687in}}%
\pgfpathclose%
\pgfusepath{stroke,fill}%
\end{pgfscope}%
\begin{pgfscope}%
\pgfpathrectangle{\pgfqpoint{0.800000in}{0.528000in}}{\pgfqpoint{4.960000in}{3.696000in}}%
\pgfusepath{clip}%
\pgfsetbuttcap%
\pgfsetroundjoin%
\definecolor{currentfill}{rgb}{0.000000,0.000000,0.000000}%
\pgfsetfillcolor{currentfill}%
\pgfsetlinewidth{1.003750pt}%
\definecolor{currentstroke}{rgb}{0.000000,0.000000,0.000000}%
\pgfsetstrokecolor{currentstroke}%
\pgfsetdash{}{0pt}%
\pgfpathmoveto{\pgfqpoint{5.504545in}{2.877687in}}%
\pgfpathcurveto{\pgfqpoint{5.515596in}{2.877687in}}{\pgfqpoint{5.526195in}{2.882077in}}{\pgfqpoint{5.534008in}{2.889891in}}%
\pgfpathcurveto{\pgfqpoint{5.541822in}{2.897704in}}{\pgfqpoint{5.546212in}{2.908303in}}{\pgfqpoint{5.546212in}{2.919353in}}%
\pgfpathcurveto{\pgfqpoint{5.546212in}{2.930404in}}{\pgfqpoint{5.541822in}{2.941003in}}{\pgfqpoint{5.534008in}{2.948816in}}%
\pgfpathcurveto{\pgfqpoint{5.526195in}{2.956630in}}{\pgfqpoint{5.515596in}{2.961020in}}{\pgfqpoint{5.504545in}{2.961020in}}%
\pgfpathcurveto{\pgfqpoint{5.493495in}{2.961020in}}{\pgfqpoint{5.482896in}{2.956630in}}{\pgfqpoint{5.475083in}{2.948816in}}%
\pgfpathcurveto{\pgfqpoint{5.467269in}{2.941003in}}{\pgfqpoint{5.462879in}{2.930404in}}{\pgfqpoint{5.462879in}{2.919353in}}%
\pgfpathcurveto{\pgfqpoint{5.462879in}{2.908303in}}{\pgfqpoint{5.467269in}{2.897704in}}{\pgfqpoint{5.475083in}{2.889891in}}%
\pgfpathcurveto{\pgfqpoint{5.482896in}{2.882077in}}{\pgfqpoint{5.493495in}{2.877687in}}{\pgfqpoint{5.504545in}{2.877687in}}%
\pgfpathclose%
\pgfusepath{stroke,fill}%
\end{pgfscope}%
\begin{pgfscope}%
\pgfpathrectangle{\pgfqpoint{0.800000in}{0.528000in}}{\pgfqpoint{4.960000in}{3.696000in}}%
\pgfusepath{clip}%
\pgfsetbuttcap%
\pgfsetroundjoin%
\definecolor{currentfill}{rgb}{0.000000,0.000000,0.000000}%
\pgfsetfillcolor{currentfill}%
\pgfsetlinewidth{1.003750pt}%
\definecolor{currentstroke}{rgb}{0.000000,0.000000,0.000000}%
\pgfsetstrokecolor{currentstroke}%
\pgfsetdash{}{0pt}%
\pgfpathmoveto{\pgfqpoint{5.504545in}{2.877687in}}%
\pgfpathcurveto{\pgfqpoint{5.515596in}{2.877687in}}{\pgfqpoint{5.526195in}{2.882077in}}{\pgfqpoint{5.534008in}{2.889891in}}%
\pgfpathcurveto{\pgfqpoint{5.541822in}{2.897704in}}{\pgfqpoint{5.546212in}{2.908303in}}{\pgfqpoint{5.546212in}{2.919353in}}%
\pgfpathcurveto{\pgfqpoint{5.546212in}{2.930404in}}{\pgfqpoint{5.541822in}{2.941003in}}{\pgfqpoint{5.534008in}{2.948816in}}%
\pgfpathcurveto{\pgfqpoint{5.526195in}{2.956630in}}{\pgfqpoint{5.515596in}{2.961020in}}{\pgfqpoint{5.504545in}{2.961020in}}%
\pgfpathcurveto{\pgfqpoint{5.493495in}{2.961020in}}{\pgfqpoint{5.482896in}{2.956630in}}{\pgfqpoint{5.475083in}{2.948816in}}%
\pgfpathcurveto{\pgfqpoint{5.467269in}{2.941003in}}{\pgfqpoint{5.462879in}{2.930404in}}{\pgfqpoint{5.462879in}{2.919353in}}%
\pgfpathcurveto{\pgfqpoint{5.462879in}{2.908303in}}{\pgfqpoint{5.467269in}{2.897704in}}{\pgfqpoint{5.475083in}{2.889891in}}%
\pgfpathcurveto{\pgfqpoint{5.482896in}{2.882077in}}{\pgfqpoint{5.493495in}{2.877687in}}{\pgfqpoint{5.504545in}{2.877687in}}%
\pgfpathclose%
\pgfusepath{stroke,fill}%
\end{pgfscope}%
\begin{pgfscope}%
\pgfpathrectangle{\pgfqpoint{0.800000in}{0.528000in}}{\pgfqpoint{4.960000in}{3.696000in}}%
\pgfusepath{clip}%
\pgfsetbuttcap%
\pgfsetroundjoin%
\definecolor{currentfill}{rgb}{0.000000,0.000000,0.000000}%
\pgfsetfillcolor{currentfill}%
\pgfsetlinewidth{1.003750pt}%
\definecolor{currentstroke}{rgb}{0.000000,0.000000,0.000000}%
\pgfsetstrokecolor{currentstroke}%
\pgfsetdash{}{0pt}%
\pgfpathmoveto{\pgfqpoint{5.504545in}{2.877687in}}%
\pgfpathcurveto{\pgfqpoint{5.515596in}{2.877687in}}{\pgfqpoint{5.526195in}{2.882077in}}{\pgfqpoint{5.534008in}{2.889891in}}%
\pgfpathcurveto{\pgfqpoint{5.541822in}{2.897704in}}{\pgfqpoint{5.546212in}{2.908303in}}{\pgfqpoint{5.546212in}{2.919353in}}%
\pgfpathcurveto{\pgfqpoint{5.546212in}{2.930404in}}{\pgfqpoint{5.541822in}{2.941003in}}{\pgfqpoint{5.534008in}{2.948816in}}%
\pgfpathcurveto{\pgfqpoint{5.526195in}{2.956630in}}{\pgfqpoint{5.515596in}{2.961020in}}{\pgfqpoint{5.504545in}{2.961020in}}%
\pgfpathcurveto{\pgfqpoint{5.493495in}{2.961020in}}{\pgfqpoint{5.482896in}{2.956630in}}{\pgfqpoint{5.475083in}{2.948816in}}%
\pgfpathcurveto{\pgfqpoint{5.467269in}{2.941003in}}{\pgfqpoint{5.462879in}{2.930404in}}{\pgfqpoint{5.462879in}{2.919353in}}%
\pgfpathcurveto{\pgfqpoint{5.462879in}{2.908303in}}{\pgfqpoint{5.467269in}{2.897704in}}{\pgfqpoint{5.475083in}{2.889891in}}%
\pgfpathcurveto{\pgfqpoint{5.482896in}{2.882077in}}{\pgfqpoint{5.493495in}{2.877687in}}{\pgfqpoint{5.504545in}{2.877687in}}%
\pgfpathclose%
\pgfusepath{stroke,fill}%
\end{pgfscope}%
\begin{pgfscope}%
\pgfpathrectangle{\pgfqpoint{0.800000in}{0.528000in}}{\pgfqpoint{4.960000in}{3.696000in}}%
\pgfusepath{clip}%
\pgfsetbuttcap%
\pgfsetroundjoin%
\definecolor{currentfill}{rgb}{0.000000,0.000000,0.000000}%
\pgfsetfillcolor{currentfill}%
\pgfsetlinewidth{1.003750pt}%
\definecolor{currentstroke}{rgb}{0.000000,0.000000,0.000000}%
\pgfsetstrokecolor{currentstroke}%
\pgfsetdash{}{0pt}%
\pgfpathmoveto{\pgfqpoint{5.504545in}{2.877687in}}%
\pgfpathcurveto{\pgfqpoint{5.515596in}{2.877687in}}{\pgfqpoint{5.526195in}{2.882077in}}{\pgfqpoint{5.534008in}{2.889891in}}%
\pgfpathcurveto{\pgfqpoint{5.541822in}{2.897704in}}{\pgfqpoint{5.546212in}{2.908303in}}{\pgfqpoint{5.546212in}{2.919353in}}%
\pgfpathcurveto{\pgfqpoint{5.546212in}{2.930404in}}{\pgfqpoint{5.541822in}{2.941003in}}{\pgfqpoint{5.534008in}{2.948816in}}%
\pgfpathcurveto{\pgfqpoint{5.526195in}{2.956630in}}{\pgfqpoint{5.515596in}{2.961020in}}{\pgfqpoint{5.504545in}{2.961020in}}%
\pgfpathcurveto{\pgfqpoint{5.493495in}{2.961020in}}{\pgfqpoint{5.482896in}{2.956630in}}{\pgfqpoint{5.475083in}{2.948816in}}%
\pgfpathcurveto{\pgfqpoint{5.467269in}{2.941003in}}{\pgfqpoint{5.462879in}{2.930404in}}{\pgfqpoint{5.462879in}{2.919353in}}%
\pgfpathcurveto{\pgfqpoint{5.462879in}{2.908303in}}{\pgfqpoint{5.467269in}{2.897704in}}{\pgfqpoint{5.475083in}{2.889891in}}%
\pgfpathcurveto{\pgfqpoint{5.482896in}{2.882077in}}{\pgfqpoint{5.493495in}{2.877687in}}{\pgfqpoint{5.504545in}{2.877687in}}%
\pgfpathclose%
\pgfusepath{stroke,fill}%
\end{pgfscope}%
\begin{pgfscope}%
\pgfpathrectangle{\pgfqpoint{0.800000in}{0.528000in}}{\pgfqpoint{4.960000in}{3.696000in}}%
\pgfusepath{clip}%
\pgfsetbuttcap%
\pgfsetroundjoin%
\definecolor{currentfill}{rgb}{0.000000,0.000000,0.000000}%
\pgfsetfillcolor{currentfill}%
\pgfsetlinewidth{1.003750pt}%
\definecolor{currentstroke}{rgb}{0.000000,0.000000,0.000000}%
\pgfsetstrokecolor{currentstroke}%
\pgfsetdash{}{0pt}%
\pgfpathmoveto{\pgfqpoint{5.504545in}{2.877687in}}%
\pgfpathcurveto{\pgfqpoint{5.515596in}{2.877687in}}{\pgfqpoint{5.526195in}{2.882077in}}{\pgfqpoint{5.534008in}{2.889891in}}%
\pgfpathcurveto{\pgfqpoint{5.541822in}{2.897704in}}{\pgfqpoint{5.546212in}{2.908303in}}{\pgfqpoint{5.546212in}{2.919353in}}%
\pgfpathcurveto{\pgfqpoint{5.546212in}{2.930404in}}{\pgfqpoint{5.541822in}{2.941003in}}{\pgfqpoint{5.534008in}{2.948816in}}%
\pgfpathcurveto{\pgfqpoint{5.526195in}{2.956630in}}{\pgfqpoint{5.515596in}{2.961020in}}{\pgfqpoint{5.504545in}{2.961020in}}%
\pgfpathcurveto{\pgfqpoint{5.493495in}{2.961020in}}{\pgfqpoint{5.482896in}{2.956630in}}{\pgfqpoint{5.475083in}{2.948816in}}%
\pgfpathcurveto{\pgfqpoint{5.467269in}{2.941003in}}{\pgfqpoint{5.462879in}{2.930404in}}{\pgfqpoint{5.462879in}{2.919353in}}%
\pgfpathcurveto{\pgfqpoint{5.462879in}{2.908303in}}{\pgfqpoint{5.467269in}{2.897704in}}{\pgfqpoint{5.475083in}{2.889891in}}%
\pgfpathcurveto{\pgfqpoint{5.482896in}{2.882077in}}{\pgfqpoint{5.493495in}{2.877687in}}{\pgfqpoint{5.504545in}{2.877687in}}%
\pgfpathclose%
\pgfusepath{stroke,fill}%
\end{pgfscope}%
\begin{pgfscope}%
\pgfpathrectangle{\pgfqpoint{0.800000in}{0.528000in}}{\pgfqpoint{4.960000in}{3.696000in}}%
\pgfusepath{clip}%
\pgfsetbuttcap%
\pgfsetroundjoin%
\definecolor{currentfill}{rgb}{0.000000,0.000000,0.000000}%
\pgfsetfillcolor{currentfill}%
\pgfsetlinewidth{1.003750pt}%
\definecolor{currentstroke}{rgb}{0.000000,0.000000,0.000000}%
\pgfsetstrokecolor{currentstroke}%
\pgfsetdash{}{0pt}%
\pgfpathmoveto{\pgfqpoint{5.504545in}{2.877687in}}%
\pgfpathcurveto{\pgfqpoint{5.515596in}{2.877687in}}{\pgfqpoint{5.526195in}{2.882077in}}{\pgfqpoint{5.534008in}{2.889891in}}%
\pgfpathcurveto{\pgfqpoint{5.541822in}{2.897704in}}{\pgfqpoint{5.546212in}{2.908303in}}{\pgfqpoint{5.546212in}{2.919353in}}%
\pgfpathcurveto{\pgfqpoint{5.546212in}{2.930404in}}{\pgfqpoint{5.541822in}{2.941003in}}{\pgfqpoint{5.534008in}{2.948816in}}%
\pgfpathcurveto{\pgfqpoint{5.526195in}{2.956630in}}{\pgfqpoint{5.515596in}{2.961020in}}{\pgfqpoint{5.504545in}{2.961020in}}%
\pgfpathcurveto{\pgfqpoint{5.493495in}{2.961020in}}{\pgfqpoint{5.482896in}{2.956630in}}{\pgfqpoint{5.475083in}{2.948816in}}%
\pgfpathcurveto{\pgfqpoint{5.467269in}{2.941003in}}{\pgfqpoint{5.462879in}{2.930404in}}{\pgfqpoint{5.462879in}{2.919353in}}%
\pgfpathcurveto{\pgfqpoint{5.462879in}{2.908303in}}{\pgfqpoint{5.467269in}{2.897704in}}{\pgfqpoint{5.475083in}{2.889891in}}%
\pgfpathcurveto{\pgfqpoint{5.482896in}{2.882077in}}{\pgfqpoint{5.493495in}{2.877687in}}{\pgfqpoint{5.504545in}{2.877687in}}%
\pgfpathclose%
\pgfusepath{stroke,fill}%
\end{pgfscope}%
\begin{pgfscope}%
\pgfpathrectangle{\pgfqpoint{0.800000in}{0.528000in}}{\pgfqpoint{4.960000in}{3.696000in}}%
\pgfusepath{clip}%
\pgfsetbuttcap%
\pgfsetroundjoin%
\definecolor{currentfill}{rgb}{0.000000,0.000000,0.000000}%
\pgfsetfillcolor{currentfill}%
\pgfsetlinewidth{1.003750pt}%
\definecolor{currentstroke}{rgb}{0.000000,0.000000,0.000000}%
\pgfsetstrokecolor{currentstroke}%
\pgfsetdash{}{0pt}%
\pgfpathmoveto{\pgfqpoint{5.504545in}{2.877687in}}%
\pgfpathcurveto{\pgfqpoint{5.515596in}{2.877687in}}{\pgfqpoint{5.526195in}{2.882077in}}{\pgfqpoint{5.534008in}{2.889891in}}%
\pgfpathcurveto{\pgfqpoint{5.541822in}{2.897704in}}{\pgfqpoint{5.546212in}{2.908303in}}{\pgfqpoint{5.546212in}{2.919353in}}%
\pgfpathcurveto{\pgfqpoint{5.546212in}{2.930404in}}{\pgfqpoint{5.541822in}{2.941003in}}{\pgfqpoint{5.534008in}{2.948816in}}%
\pgfpathcurveto{\pgfqpoint{5.526195in}{2.956630in}}{\pgfqpoint{5.515596in}{2.961020in}}{\pgfqpoint{5.504545in}{2.961020in}}%
\pgfpathcurveto{\pgfqpoint{5.493495in}{2.961020in}}{\pgfqpoint{5.482896in}{2.956630in}}{\pgfqpoint{5.475083in}{2.948816in}}%
\pgfpathcurveto{\pgfqpoint{5.467269in}{2.941003in}}{\pgfqpoint{5.462879in}{2.930404in}}{\pgfqpoint{5.462879in}{2.919353in}}%
\pgfpathcurveto{\pgfqpoint{5.462879in}{2.908303in}}{\pgfqpoint{5.467269in}{2.897704in}}{\pgfqpoint{5.475083in}{2.889891in}}%
\pgfpathcurveto{\pgfqpoint{5.482896in}{2.882077in}}{\pgfqpoint{5.493495in}{2.877687in}}{\pgfqpoint{5.504545in}{2.877687in}}%
\pgfpathclose%
\pgfusepath{stroke,fill}%
\end{pgfscope}%
\begin{pgfscope}%
\pgfpathrectangle{\pgfqpoint{0.800000in}{0.528000in}}{\pgfqpoint{4.960000in}{3.696000in}}%
\pgfusepath{clip}%
\pgfsetbuttcap%
\pgfsetroundjoin%
\definecolor{currentfill}{rgb}{0.000000,0.000000,0.000000}%
\pgfsetfillcolor{currentfill}%
\pgfsetlinewidth{1.003750pt}%
\definecolor{currentstroke}{rgb}{0.000000,0.000000,0.000000}%
\pgfsetstrokecolor{currentstroke}%
\pgfsetdash{}{0pt}%
\pgfpathmoveto{\pgfqpoint{5.504545in}{2.877687in}}%
\pgfpathcurveto{\pgfqpoint{5.515596in}{2.877687in}}{\pgfqpoint{5.526195in}{2.882077in}}{\pgfqpoint{5.534008in}{2.889891in}}%
\pgfpathcurveto{\pgfqpoint{5.541822in}{2.897704in}}{\pgfqpoint{5.546212in}{2.908303in}}{\pgfqpoint{5.546212in}{2.919353in}}%
\pgfpathcurveto{\pgfqpoint{5.546212in}{2.930404in}}{\pgfqpoint{5.541822in}{2.941003in}}{\pgfqpoint{5.534008in}{2.948816in}}%
\pgfpathcurveto{\pgfqpoint{5.526195in}{2.956630in}}{\pgfqpoint{5.515596in}{2.961020in}}{\pgfqpoint{5.504545in}{2.961020in}}%
\pgfpathcurveto{\pgfqpoint{5.493495in}{2.961020in}}{\pgfqpoint{5.482896in}{2.956630in}}{\pgfqpoint{5.475083in}{2.948816in}}%
\pgfpathcurveto{\pgfqpoint{5.467269in}{2.941003in}}{\pgfqpoint{5.462879in}{2.930404in}}{\pgfqpoint{5.462879in}{2.919353in}}%
\pgfpathcurveto{\pgfqpoint{5.462879in}{2.908303in}}{\pgfqpoint{5.467269in}{2.897704in}}{\pgfqpoint{5.475083in}{2.889891in}}%
\pgfpathcurveto{\pgfqpoint{5.482896in}{2.882077in}}{\pgfqpoint{5.493495in}{2.877687in}}{\pgfqpoint{5.504545in}{2.877687in}}%
\pgfpathclose%
\pgfusepath{stroke,fill}%
\end{pgfscope}%
\begin{pgfscope}%
\pgfpathrectangle{\pgfqpoint{0.800000in}{0.528000in}}{\pgfqpoint{4.960000in}{3.696000in}}%
\pgfusepath{clip}%
\pgfsetbuttcap%
\pgfsetroundjoin%
\definecolor{currentfill}{rgb}{0.000000,0.000000,0.000000}%
\pgfsetfillcolor{currentfill}%
\pgfsetlinewidth{1.003750pt}%
\definecolor{currentstroke}{rgb}{0.000000,0.000000,0.000000}%
\pgfsetstrokecolor{currentstroke}%
\pgfsetdash{}{0pt}%
\pgfpathmoveto{\pgfqpoint{5.504545in}{2.877687in}}%
\pgfpathcurveto{\pgfqpoint{5.515596in}{2.877687in}}{\pgfqpoint{5.526195in}{2.882077in}}{\pgfqpoint{5.534008in}{2.889891in}}%
\pgfpathcurveto{\pgfqpoint{5.541822in}{2.897704in}}{\pgfqpoint{5.546212in}{2.908303in}}{\pgfqpoint{5.546212in}{2.919353in}}%
\pgfpathcurveto{\pgfqpoint{5.546212in}{2.930404in}}{\pgfqpoint{5.541822in}{2.941003in}}{\pgfqpoint{5.534008in}{2.948816in}}%
\pgfpathcurveto{\pgfqpoint{5.526195in}{2.956630in}}{\pgfqpoint{5.515596in}{2.961020in}}{\pgfqpoint{5.504545in}{2.961020in}}%
\pgfpathcurveto{\pgfqpoint{5.493495in}{2.961020in}}{\pgfqpoint{5.482896in}{2.956630in}}{\pgfqpoint{5.475083in}{2.948816in}}%
\pgfpathcurveto{\pgfqpoint{5.467269in}{2.941003in}}{\pgfqpoint{5.462879in}{2.930404in}}{\pgfqpoint{5.462879in}{2.919353in}}%
\pgfpathcurveto{\pgfqpoint{5.462879in}{2.908303in}}{\pgfqpoint{5.467269in}{2.897704in}}{\pgfqpoint{5.475083in}{2.889891in}}%
\pgfpathcurveto{\pgfqpoint{5.482896in}{2.882077in}}{\pgfqpoint{5.493495in}{2.877687in}}{\pgfqpoint{5.504545in}{2.877687in}}%
\pgfpathclose%
\pgfusepath{stroke,fill}%
\end{pgfscope}%
\begin{pgfscope}%
\pgfpathrectangle{\pgfqpoint{0.800000in}{0.528000in}}{\pgfqpoint{4.960000in}{3.696000in}}%
\pgfusepath{clip}%
\pgfsetbuttcap%
\pgfsetroundjoin%
\definecolor{currentfill}{rgb}{0.000000,0.000000,0.000000}%
\pgfsetfillcolor{currentfill}%
\pgfsetlinewidth{1.003750pt}%
\definecolor{currentstroke}{rgb}{0.000000,0.000000,0.000000}%
\pgfsetstrokecolor{currentstroke}%
\pgfsetdash{}{0pt}%
\pgfpathmoveto{\pgfqpoint{5.504545in}{2.877687in}}%
\pgfpathcurveto{\pgfqpoint{5.515596in}{2.877687in}}{\pgfqpoint{5.526195in}{2.882077in}}{\pgfqpoint{5.534008in}{2.889891in}}%
\pgfpathcurveto{\pgfqpoint{5.541822in}{2.897704in}}{\pgfqpoint{5.546212in}{2.908303in}}{\pgfqpoint{5.546212in}{2.919353in}}%
\pgfpathcurveto{\pgfqpoint{5.546212in}{2.930404in}}{\pgfqpoint{5.541822in}{2.941003in}}{\pgfqpoint{5.534008in}{2.948816in}}%
\pgfpathcurveto{\pgfqpoint{5.526195in}{2.956630in}}{\pgfqpoint{5.515596in}{2.961020in}}{\pgfqpoint{5.504545in}{2.961020in}}%
\pgfpathcurveto{\pgfqpoint{5.493495in}{2.961020in}}{\pgfqpoint{5.482896in}{2.956630in}}{\pgfqpoint{5.475083in}{2.948816in}}%
\pgfpathcurveto{\pgfqpoint{5.467269in}{2.941003in}}{\pgfqpoint{5.462879in}{2.930404in}}{\pgfqpoint{5.462879in}{2.919353in}}%
\pgfpathcurveto{\pgfqpoint{5.462879in}{2.908303in}}{\pgfqpoint{5.467269in}{2.897704in}}{\pgfqpoint{5.475083in}{2.889891in}}%
\pgfpathcurveto{\pgfqpoint{5.482896in}{2.882077in}}{\pgfqpoint{5.493495in}{2.877687in}}{\pgfqpoint{5.504545in}{2.877687in}}%
\pgfpathclose%
\pgfusepath{stroke,fill}%
\end{pgfscope}%
\begin{pgfscope}%
\pgfpathrectangle{\pgfqpoint{0.800000in}{0.528000in}}{\pgfqpoint{4.960000in}{3.696000in}}%
\pgfusepath{clip}%
\pgfsetbuttcap%
\pgfsetroundjoin%
\definecolor{currentfill}{rgb}{0.000000,0.000000,0.000000}%
\pgfsetfillcolor{currentfill}%
\pgfsetlinewidth{1.003750pt}%
\definecolor{currentstroke}{rgb}{0.000000,0.000000,0.000000}%
\pgfsetstrokecolor{currentstroke}%
\pgfsetdash{}{0pt}%
\pgfpathmoveto{\pgfqpoint{5.504545in}{2.877687in}}%
\pgfpathcurveto{\pgfqpoint{5.515596in}{2.877687in}}{\pgfqpoint{5.526195in}{2.882077in}}{\pgfqpoint{5.534008in}{2.889891in}}%
\pgfpathcurveto{\pgfqpoint{5.541822in}{2.897704in}}{\pgfqpoint{5.546212in}{2.908303in}}{\pgfqpoint{5.546212in}{2.919353in}}%
\pgfpathcurveto{\pgfqpoint{5.546212in}{2.930404in}}{\pgfqpoint{5.541822in}{2.941003in}}{\pgfqpoint{5.534008in}{2.948816in}}%
\pgfpathcurveto{\pgfqpoint{5.526195in}{2.956630in}}{\pgfqpoint{5.515596in}{2.961020in}}{\pgfqpoint{5.504545in}{2.961020in}}%
\pgfpathcurveto{\pgfqpoint{5.493495in}{2.961020in}}{\pgfqpoint{5.482896in}{2.956630in}}{\pgfqpoint{5.475083in}{2.948816in}}%
\pgfpathcurveto{\pgfqpoint{5.467269in}{2.941003in}}{\pgfqpoint{5.462879in}{2.930404in}}{\pgfqpoint{5.462879in}{2.919353in}}%
\pgfpathcurveto{\pgfqpoint{5.462879in}{2.908303in}}{\pgfqpoint{5.467269in}{2.897704in}}{\pgfqpoint{5.475083in}{2.889891in}}%
\pgfpathcurveto{\pgfqpoint{5.482896in}{2.882077in}}{\pgfqpoint{5.493495in}{2.877687in}}{\pgfqpoint{5.504545in}{2.877687in}}%
\pgfpathclose%
\pgfusepath{stroke,fill}%
\end{pgfscope}%
\begin{pgfscope}%
\pgfpathrectangle{\pgfqpoint{0.800000in}{0.528000in}}{\pgfqpoint{4.960000in}{3.696000in}}%
\pgfusepath{clip}%
\pgfsetbuttcap%
\pgfsetroundjoin%
\definecolor{currentfill}{rgb}{0.000000,0.000000,0.000000}%
\pgfsetfillcolor{currentfill}%
\pgfsetlinewidth{1.003750pt}%
\definecolor{currentstroke}{rgb}{0.000000,0.000000,0.000000}%
\pgfsetstrokecolor{currentstroke}%
\pgfsetdash{}{0pt}%
\pgfpathmoveto{\pgfqpoint{5.504545in}{2.877687in}}%
\pgfpathcurveto{\pgfqpoint{5.515596in}{2.877687in}}{\pgfqpoint{5.526195in}{2.882077in}}{\pgfqpoint{5.534008in}{2.889891in}}%
\pgfpathcurveto{\pgfqpoint{5.541822in}{2.897704in}}{\pgfqpoint{5.546212in}{2.908303in}}{\pgfqpoint{5.546212in}{2.919353in}}%
\pgfpathcurveto{\pgfqpoint{5.546212in}{2.930404in}}{\pgfqpoint{5.541822in}{2.941003in}}{\pgfqpoint{5.534008in}{2.948816in}}%
\pgfpathcurveto{\pgfqpoint{5.526195in}{2.956630in}}{\pgfqpoint{5.515596in}{2.961020in}}{\pgfqpoint{5.504545in}{2.961020in}}%
\pgfpathcurveto{\pgfqpoint{5.493495in}{2.961020in}}{\pgfqpoint{5.482896in}{2.956630in}}{\pgfqpoint{5.475083in}{2.948816in}}%
\pgfpathcurveto{\pgfqpoint{5.467269in}{2.941003in}}{\pgfqpoint{5.462879in}{2.930404in}}{\pgfqpoint{5.462879in}{2.919353in}}%
\pgfpathcurveto{\pgfqpoint{5.462879in}{2.908303in}}{\pgfqpoint{5.467269in}{2.897704in}}{\pgfqpoint{5.475083in}{2.889891in}}%
\pgfpathcurveto{\pgfqpoint{5.482896in}{2.882077in}}{\pgfqpoint{5.493495in}{2.877687in}}{\pgfqpoint{5.504545in}{2.877687in}}%
\pgfpathclose%
\pgfusepath{stroke,fill}%
\end{pgfscope}%
\begin{pgfscope}%
\pgfpathrectangle{\pgfqpoint{0.800000in}{0.528000in}}{\pgfqpoint{4.960000in}{3.696000in}}%
\pgfusepath{clip}%
\pgfsetbuttcap%
\pgfsetroundjoin%
\definecolor{currentfill}{rgb}{0.000000,0.000000,0.000000}%
\pgfsetfillcolor{currentfill}%
\pgfsetlinewidth{1.003750pt}%
\definecolor{currentstroke}{rgb}{0.000000,0.000000,0.000000}%
\pgfsetstrokecolor{currentstroke}%
\pgfsetdash{}{0pt}%
\pgfpathmoveto{\pgfqpoint{5.504545in}{2.877687in}}%
\pgfpathcurveto{\pgfqpoint{5.515596in}{2.877687in}}{\pgfqpoint{5.526195in}{2.882077in}}{\pgfqpoint{5.534008in}{2.889891in}}%
\pgfpathcurveto{\pgfqpoint{5.541822in}{2.897704in}}{\pgfqpoint{5.546212in}{2.908303in}}{\pgfqpoint{5.546212in}{2.919353in}}%
\pgfpathcurveto{\pgfqpoint{5.546212in}{2.930404in}}{\pgfqpoint{5.541822in}{2.941003in}}{\pgfqpoint{5.534008in}{2.948816in}}%
\pgfpathcurveto{\pgfqpoint{5.526195in}{2.956630in}}{\pgfqpoint{5.515596in}{2.961020in}}{\pgfqpoint{5.504545in}{2.961020in}}%
\pgfpathcurveto{\pgfqpoint{5.493495in}{2.961020in}}{\pgfqpoint{5.482896in}{2.956630in}}{\pgfqpoint{5.475083in}{2.948816in}}%
\pgfpathcurveto{\pgfqpoint{5.467269in}{2.941003in}}{\pgfqpoint{5.462879in}{2.930404in}}{\pgfqpoint{5.462879in}{2.919353in}}%
\pgfpathcurveto{\pgfqpoint{5.462879in}{2.908303in}}{\pgfqpoint{5.467269in}{2.897704in}}{\pgfqpoint{5.475083in}{2.889891in}}%
\pgfpathcurveto{\pgfqpoint{5.482896in}{2.882077in}}{\pgfqpoint{5.493495in}{2.877687in}}{\pgfqpoint{5.504545in}{2.877687in}}%
\pgfpathclose%
\pgfusepath{stroke,fill}%
\end{pgfscope}%
\begin{pgfscope}%
\pgfpathrectangle{\pgfqpoint{0.800000in}{0.528000in}}{\pgfqpoint{4.960000in}{3.696000in}}%
\pgfusepath{clip}%
\pgfsetbuttcap%
\pgfsetroundjoin%
\definecolor{currentfill}{rgb}{0.000000,0.000000,0.000000}%
\pgfsetfillcolor{currentfill}%
\pgfsetlinewidth{1.003750pt}%
\definecolor{currentstroke}{rgb}{0.000000,0.000000,0.000000}%
\pgfsetstrokecolor{currentstroke}%
\pgfsetdash{}{0pt}%
\pgfpathmoveto{\pgfqpoint{5.504545in}{2.877687in}}%
\pgfpathcurveto{\pgfqpoint{5.515596in}{2.877687in}}{\pgfqpoint{5.526195in}{2.882077in}}{\pgfqpoint{5.534008in}{2.889891in}}%
\pgfpathcurveto{\pgfqpoint{5.541822in}{2.897704in}}{\pgfqpoint{5.546212in}{2.908303in}}{\pgfqpoint{5.546212in}{2.919353in}}%
\pgfpathcurveto{\pgfqpoint{5.546212in}{2.930404in}}{\pgfqpoint{5.541822in}{2.941003in}}{\pgfqpoint{5.534008in}{2.948816in}}%
\pgfpathcurveto{\pgfqpoint{5.526195in}{2.956630in}}{\pgfqpoint{5.515596in}{2.961020in}}{\pgfqpoint{5.504545in}{2.961020in}}%
\pgfpathcurveto{\pgfqpoint{5.493495in}{2.961020in}}{\pgfqpoint{5.482896in}{2.956630in}}{\pgfqpoint{5.475083in}{2.948816in}}%
\pgfpathcurveto{\pgfqpoint{5.467269in}{2.941003in}}{\pgfqpoint{5.462879in}{2.930404in}}{\pgfqpoint{5.462879in}{2.919353in}}%
\pgfpathcurveto{\pgfqpoint{5.462879in}{2.908303in}}{\pgfqpoint{5.467269in}{2.897704in}}{\pgfqpoint{5.475083in}{2.889891in}}%
\pgfpathcurveto{\pgfqpoint{5.482896in}{2.882077in}}{\pgfqpoint{5.493495in}{2.877687in}}{\pgfqpoint{5.504545in}{2.877687in}}%
\pgfpathclose%
\pgfusepath{stroke,fill}%
\end{pgfscope}%
\begin{pgfscope}%
\pgfpathrectangle{\pgfqpoint{0.800000in}{0.528000in}}{\pgfqpoint{4.960000in}{3.696000in}}%
\pgfusepath{clip}%
\pgfsetbuttcap%
\pgfsetroundjoin%
\definecolor{currentfill}{rgb}{0.000000,0.000000,0.000000}%
\pgfsetfillcolor{currentfill}%
\pgfsetlinewidth{1.003750pt}%
\definecolor{currentstroke}{rgb}{0.000000,0.000000,0.000000}%
\pgfsetstrokecolor{currentstroke}%
\pgfsetdash{}{0pt}%
\pgfpathmoveto{\pgfqpoint{5.504545in}{2.877687in}}%
\pgfpathcurveto{\pgfqpoint{5.515596in}{2.877687in}}{\pgfqpoint{5.526195in}{2.882077in}}{\pgfqpoint{5.534008in}{2.889891in}}%
\pgfpathcurveto{\pgfqpoint{5.541822in}{2.897704in}}{\pgfqpoint{5.546212in}{2.908303in}}{\pgfqpoint{5.546212in}{2.919353in}}%
\pgfpathcurveto{\pgfqpoint{5.546212in}{2.930404in}}{\pgfqpoint{5.541822in}{2.941003in}}{\pgfqpoint{5.534008in}{2.948816in}}%
\pgfpathcurveto{\pgfqpoint{5.526195in}{2.956630in}}{\pgfqpoint{5.515596in}{2.961020in}}{\pgfqpoint{5.504545in}{2.961020in}}%
\pgfpathcurveto{\pgfqpoint{5.493495in}{2.961020in}}{\pgfqpoint{5.482896in}{2.956630in}}{\pgfqpoint{5.475083in}{2.948816in}}%
\pgfpathcurveto{\pgfqpoint{5.467269in}{2.941003in}}{\pgfqpoint{5.462879in}{2.930404in}}{\pgfqpoint{5.462879in}{2.919353in}}%
\pgfpathcurveto{\pgfqpoint{5.462879in}{2.908303in}}{\pgfqpoint{5.467269in}{2.897704in}}{\pgfqpoint{5.475083in}{2.889891in}}%
\pgfpathcurveto{\pgfqpoint{5.482896in}{2.882077in}}{\pgfqpoint{5.493495in}{2.877687in}}{\pgfqpoint{5.504545in}{2.877687in}}%
\pgfpathclose%
\pgfusepath{stroke,fill}%
\end{pgfscope}%
\begin{pgfscope}%
\pgfpathrectangle{\pgfqpoint{0.800000in}{0.528000in}}{\pgfqpoint{4.960000in}{3.696000in}}%
\pgfusepath{clip}%
\pgfsetbuttcap%
\pgfsetroundjoin%
\definecolor{currentfill}{rgb}{0.000000,0.000000,0.000000}%
\pgfsetfillcolor{currentfill}%
\pgfsetlinewidth{1.003750pt}%
\definecolor{currentstroke}{rgb}{0.000000,0.000000,0.000000}%
\pgfsetstrokecolor{currentstroke}%
\pgfsetdash{}{0pt}%
\pgfpathmoveto{\pgfqpoint{5.504545in}{2.877687in}}%
\pgfpathcurveto{\pgfqpoint{5.515596in}{2.877687in}}{\pgfqpoint{5.526195in}{2.882077in}}{\pgfqpoint{5.534008in}{2.889891in}}%
\pgfpathcurveto{\pgfqpoint{5.541822in}{2.897704in}}{\pgfqpoint{5.546212in}{2.908303in}}{\pgfqpoint{5.546212in}{2.919353in}}%
\pgfpathcurveto{\pgfqpoint{5.546212in}{2.930404in}}{\pgfqpoint{5.541822in}{2.941003in}}{\pgfqpoint{5.534008in}{2.948816in}}%
\pgfpathcurveto{\pgfqpoint{5.526195in}{2.956630in}}{\pgfqpoint{5.515596in}{2.961020in}}{\pgfqpoint{5.504545in}{2.961020in}}%
\pgfpathcurveto{\pgfqpoint{5.493495in}{2.961020in}}{\pgfqpoint{5.482896in}{2.956630in}}{\pgfqpoint{5.475083in}{2.948816in}}%
\pgfpathcurveto{\pgfqpoint{5.467269in}{2.941003in}}{\pgfqpoint{5.462879in}{2.930404in}}{\pgfqpoint{5.462879in}{2.919353in}}%
\pgfpathcurveto{\pgfqpoint{5.462879in}{2.908303in}}{\pgfqpoint{5.467269in}{2.897704in}}{\pgfqpoint{5.475083in}{2.889891in}}%
\pgfpathcurveto{\pgfqpoint{5.482896in}{2.882077in}}{\pgfqpoint{5.493495in}{2.877687in}}{\pgfqpoint{5.504545in}{2.877687in}}%
\pgfpathclose%
\pgfusepath{stroke,fill}%
\end{pgfscope}%
\begin{pgfscope}%
\pgfpathrectangle{\pgfqpoint{0.800000in}{0.528000in}}{\pgfqpoint{4.960000in}{3.696000in}}%
\pgfusepath{clip}%
\pgfsetbuttcap%
\pgfsetroundjoin%
\definecolor{currentfill}{rgb}{0.000000,0.000000,0.000000}%
\pgfsetfillcolor{currentfill}%
\pgfsetlinewidth{1.003750pt}%
\definecolor{currentstroke}{rgb}{0.000000,0.000000,0.000000}%
\pgfsetstrokecolor{currentstroke}%
\pgfsetdash{}{0pt}%
\pgfpathmoveto{\pgfqpoint{5.504545in}{2.877687in}}%
\pgfpathcurveto{\pgfqpoint{5.515596in}{2.877687in}}{\pgfqpoint{5.526195in}{2.882077in}}{\pgfqpoint{5.534008in}{2.889891in}}%
\pgfpathcurveto{\pgfqpoint{5.541822in}{2.897704in}}{\pgfqpoint{5.546212in}{2.908303in}}{\pgfqpoint{5.546212in}{2.919353in}}%
\pgfpathcurveto{\pgfqpoint{5.546212in}{2.930404in}}{\pgfqpoint{5.541822in}{2.941003in}}{\pgfqpoint{5.534008in}{2.948816in}}%
\pgfpathcurveto{\pgfqpoint{5.526195in}{2.956630in}}{\pgfqpoint{5.515596in}{2.961020in}}{\pgfqpoint{5.504545in}{2.961020in}}%
\pgfpathcurveto{\pgfqpoint{5.493495in}{2.961020in}}{\pgfqpoint{5.482896in}{2.956630in}}{\pgfqpoint{5.475083in}{2.948816in}}%
\pgfpathcurveto{\pgfqpoint{5.467269in}{2.941003in}}{\pgfqpoint{5.462879in}{2.930404in}}{\pgfqpoint{5.462879in}{2.919353in}}%
\pgfpathcurveto{\pgfqpoint{5.462879in}{2.908303in}}{\pgfqpoint{5.467269in}{2.897704in}}{\pgfqpoint{5.475083in}{2.889891in}}%
\pgfpathcurveto{\pgfqpoint{5.482896in}{2.882077in}}{\pgfqpoint{5.493495in}{2.877687in}}{\pgfqpoint{5.504545in}{2.877687in}}%
\pgfpathclose%
\pgfusepath{stroke,fill}%
\end{pgfscope}%
\begin{pgfscope}%
\pgfpathrectangle{\pgfqpoint{0.800000in}{0.528000in}}{\pgfqpoint{4.960000in}{3.696000in}}%
\pgfusepath{clip}%
\pgfsetbuttcap%
\pgfsetroundjoin%
\definecolor{currentfill}{rgb}{0.000000,0.000000,0.000000}%
\pgfsetfillcolor{currentfill}%
\pgfsetlinewidth{1.003750pt}%
\definecolor{currentstroke}{rgb}{0.000000,0.000000,0.000000}%
\pgfsetstrokecolor{currentstroke}%
\pgfsetdash{}{0pt}%
\pgfpathmoveto{\pgfqpoint{5.504545in}{3.984333in}}%
\pgfpathcurveto{\pgfqpoint{5.515596in}{3.984333in}}{\pgfqpoint{5.526195in}{3.988724in}}{\pgfqpoint{5.534008in}{3.996537in}}%
\pgfpathcurveto{\pgfqpoint{5.541822in}{4.004351in}}{\pgfqpoint{5.546212in}{4.014950in}}{\pgfqpoint{5.546212in}{4.026000in}}%
\pgfpathcurveto{\pgfqpoint{5.546212in}{4.037050in}}{\pgfqpoint{5.541822in}{4.047649in}}{\pgfqpoint{5.534008in}{4.055463in}}%
\pgfpathcurveto{\pgfqpoint{5.526195in}{4.063276in}}{\pgfqpoint{5.515596in}{4.067667in}}{\pgfqpoint{5.504545in}{4.067667in}}%
\pgfpathcurveto{\pgfqpoint{5.493495in}{4.067667in}}{\pgfqpoint{5.482896in}{4.063276in}}{\pgfqpoint{5.475083in}{4.055463in}}%
\pgfpathcurveto{\pgfqpoint{5.467269in}{4.047649in}}{\pgfqpoint{5.462879in}{4.037050in}}{\pgfqpoint{5.462879in}{4.026000in}}%
\pgfpathcurveto{\pgfqpoint{5.462879in}{4.014950in}}{\pgfqpoint{5.467269in}{4.004351in}}{\pgfqpoint{5.475083in}{3.996537in}}%
\pgfpathcurveto{\pgfqpoint{5.482896in}{3.988724in}}{\pgfqpoint{5.493495in}{3.984333in}}{\pgfqpoint{5.504545in}{3.984333in}}%
\pgfpathclose%
\pgfusepath{stroke,fill}%
\end{pgfscope}%
\begin{pgfscope}%
\pgfpathrectangle{\pgfqpoint{0.800000in}{0.528000in}}{\pgfqpoint{4.960000in}{3.696000in}}%
\pgfusepath{clip}%
\pgfsetbuttcap%
\pgfsetroundjoin%
\definecolor{currentfill}{rgb}{0.000000,0.000000,0.000000}%
\pgfsetfillcolor{currentfill}%
\pgfsetlinewidth{1.003750pt}%
\definecolor{currentstroke}{rgb}{0.000000,0.000000,0.000000}%
\pgfsetstrokecolor{currentstroke}%
\pgfsetdash{}{0pt}%
\pgfpathmoveto{\pgfqpoint{5.504545in}{2.877687in}}%
\pgfpathcurveto{\pgfqpoint{5.515596in}{2.877687in}}{\pgfqpoint{5.526195in}{2.882077in}}{\pgfqpoint{5.534008in}{2.889891in}}%
\pgfpathcurveto{\pgfqpoint{5.541822in}{2.897704in}}{\pgfqpoint{5.546212in}{2.908303in}}{\pgfqpoint{5.546212in}{2.919353in}}%
\pgfpathcurveto{\pgfqpoint{5.546212in}{2.930404in}}{\pgfqpoint{5.541822in}{2.941003in}}{\pgfqpoint{5.534008in}{2.948816in}}%
\pgfpathcurveto{\pgfqpoint{5.526195in}{2.956630in}}{\pgfqpoint{5.515596in}{2.961020in}}{\pgfqpoint{5.504545in}{2.961020in}}%
\pgfpathcurveto{\pgfqpoint{5.493495in}{2.961020in}}{\pgfqpoint{5.482896in}{2.956630in}}{\pgfqpoint{5.475083in}{2.948816in}}%
\pgfpathcurveto{\pgfqpoint{5.467269in}{2.941003in}}{\pgfqpoint{5.462879in}{2.930404in}}{\pgfqpoint{5.462879in}{2.919353in}}%
\pgfpathcurveto{\pgfqpoint{5.462879in}{2.908303in}}{\pgfqpoint{5.467269in}{2.897704in}}{\pgfqpoint{5.475083in}{2.889891in}}%
\pgfpathcurveto{\pgfqpoint{5.482896in}{2.882077in}}{\pgfqpoint{5.493495in}{2.877687in}}{\pgfqpoint{5.504545in}{2.877687in}}%
\pgfpathclose%
\pgfusepath{stroke,fill}%
\end{pgfscope}%
\begin{pgfscope}%
\pgfpathrectangle{\pgfqpoint{0.800000in}{0.528000in}}{\pgfqpoint{4.960000in}{3.696000in}}%
\pgfusepath{clip}%
\pgfsetbuttcap%
\pgfsetroundjoin%
\definecolor{currentfill}{rgb}{0.000000,0.000000,0.000000}%
\pgfsetfillcolor{currentfill}%
\pgfsetlinewidth{1.003750pt}%
\definecolor{currentstroke}{rgb}{0.000000,0.000000,0.000000}%
\pgfsetstrokecolor{currentstroke}%
\pgfsetdash{}{0pt}%
\pgfpathmoveto{\pgfqpoint{5.504545in}{2.877687in}}%
\pgfpathcurveto{\pgfqpoint{5.515596in}{2.877687in}}{\pgfqpoint{5.526195in}{2.882077in}}{\pgfqpoint{5.534008in}{2.889891in}}%
\pgfpathcurveto{\pgfqpoint{5.541822in}{2.897704in}}{\pgfqpoint{5.546212in}{2.908303in}}{\pgfqpoint{5.546212in}{2.919353in}}%
\pgfpathcurveto{\pgfqpoint{5.546212in}{2.930404in}}{\pgfqpoint{5.541822in}{2.941003in}}{\pgfqpoint{5.534008in}{2.948816in}}%
\pgfpathcurveto{\pgfqpoint{5.526195in}{2.956630in}}{\pgfqpoint{5.515596in}{2.961020in}}{\pgfqpoint{5.504545in}{2.961020in}}%
\pgfpathcurveto{\pgfqpoint{5.493495in}{2.961020in}}{\pgfqpoint{5.482896in}{2.956630in}}{\pgfqpoint{5.475083in}{2.948816in}}%
\pgfpathcurveto{\pgfqpoint{5.467269in}{2.941003in}}{\pgfqpoint{5.462879in}{2.930404in}}{\pgfqpoint{5.462879in}{2.919353in}}%
\pgfpathcurveto{\pgfqpoint{5.462879in}{2.908303in}}{\pgfqpoint{5.467269in}{2.897704in}}{\pgfqpoint{5.475083in}{2.889891in}}%
\pgfpathcurveto{\pgfqpoint{5.482896in}{2.882077in}}{\pgfqpoint{5.493495in}{2.877687in}}{\pgfqpoint{5.504545in}{2.877687in}}%
\pgfpathclose%
\pgfusepath{stroke,fill}%
\end{pgfscope}%
\begin{pgfscope}%
\pgfpathrectangle{\pgfqpoint{0.800000in}{0.528000in}}{\pgfqpoint{4.960000in}{3.696000in}}%
\pgfusepath{clip}%
\pgfsetbuttcap%
\pgfsetroundjoin%
\definecolor{currentfill}{rgb}{0.000000,0.000000,0.000000}%
\pgfsetfillcolor{currentfill}%
\pgfsetlinewidth{1.003750pt}%
\definecolor{currentstroke}{rgb}{0.000000,0.000000,0.000000}%
\pgfsetstrokecolor{currentstroke}%
\pgfsetdash{}{0pt}%
\pgfpathmoveto{\pgfqpoint{5.504545in}{2.877687in}}%
\pgfpathcurveto{\pgfqpoint{5.515596in}{2.877687in}}{\pgfqpoint{5.526195in}{2.882077in}}{\pgfqpoint{5.534008in}{2.889891in}}%
\pgfpathcurveto{\pgfqpoint{5.541822in}{2.897704in}}{\pgfqpoint{5.546212in}{2.908303in}}{\pgfqpoint{5.546212in}{2.919353in}}%
\pgfpathcurveto{\pgfqpoint{5.546212in}{2.930404in}}{\pgfqpoint{5.541822in}{2.941003in}}{\pgfqpoint{5.534008in}{2.948816in}}%
\pgfpathcurveto{\pgfqpoint{5.526195in}{2.956630in}}{\pgfqpoint{5.515596in}{2.961020in}}{\pgfqpoint{5.504545in}{2.961020in}}%
\pgfpathcurveto{\pgfqpoint{5.493495in}{2.961020in}}{\pgfqpoint{5.482896in}{2.956630in}}{\pgfqpoint{5.475083in}{2.948816in}}%
\pgfpathcurveto{\pgfqpoint{5.467269in}{2.941003in}}{\pgfqpoint{5.462879in}{2.930404in}}{\pgfqpoint{5.462879in}{2.919353in}}%
\pgfpathcurveto{\pgfqpoint{5.462879in}{2.908303in}}{\pgfqpoint{5.467269in}{2.897704in}}{\pgfqpoint{5.475083in}{2.889891in}}%
\pgfpathcurveto{\pgfqpoint{5.482896in}{2.882077in}}{\pgfqpoint{5.493495in}{2.877687in}}{\pgfqpoint{5.504545in}{2.877687in}}%
\pgfpathclose%
\pgfusepath{stroke,fill}%
\end{pgfscope}%
\begin{pgfscope}%
\pgfpathrectangle{\pgfqpoint{0.800000in}{0.528000in}}{\pgfqpoint{4.960000in}{3.696000in}}%
\pgfusepath{clip}%
\pgfsetbuttcap%
\pgfsetroundjoin%
\definecolor{currentfill}{rgb}{0.000000,0.000000,0.000000}%
\pgfsetfillcolor{currentfill}%
\pgfsetlinewidth{1.003750pt}%
\definecolor{currentstroke}{rgb}{0.000000,0.000000,0.000000}%
\pgfsetstrokecolor{currentstroke}%
\pgfsetdash{}{0pt}%
\pgfpathmoveto{\pgfqpoint{5.504545in}{2.877687in}}%
\pgfpathcurveto{\pgfqpoint{5.515596in}{2.877687in}}{\pgfqpoint{5.526195in}{2.882077in}}{\pgfqpoint{5.534008in}{2.889891in}}%
\pgfpathcurveto{\pgfqpoint{5.541822in}{2.897704in}}{\pgfqpoint{5.546212in}{2.908303in}}{\pgfqpoint{5.546212in}{2.919353in}}%
\pgfpathcurveto{\pgfqpoint{5.546212in}{2.930404in}}{\pgfqpoint{5.541822in}{2.941003in}}{\pgfqpoint{5.534008in}{2.948816in}}%
\pgfpathcurveto{\pgfqpoint{5.526195in}{2.956630in}}{\pgfqpoint{5.515596in}{2.961020in}}{\pgfqpoint{5.504545in}{2.961020in}}%
\pgfpathcurveto{\pgfqpoint{5.493495in}{2.961020in}}{\pgfqpoint{5.482896in}{2.956630in}}{\pgfqpoint{5.475083in}{2.948816in}}%
\pgfpathcurveto{\pgfqpoint{5.467269in}{2.941003in}}{\pgfqpoint{5.462879in}{2.930404in}}{\pgfqpoint{5.462879in}{2.919353in}}%
\pgfpathcurveto{\pgfqpoint{5.462879in}{2.908303in}}{\pgfqpoint{5.467269in}{2.897704in}}{\pgfqpoint{5.475083in}{2.889891in}}%
\pgfpathcurveto{\pgfqpoint{5.482896in}{2.882077in}}{\pgfqpoint{5.493495in}{2.877687in}}{\pgfqpoint{5.504545in}{2.877687in}}%
\pgfpathclose%
\pgfusepath{stroke,fill}%
\end{pgfscope}%
\begin{pgfscope}%
\pgfpathrectangle{\pgfqpoint{0.800000in}{0.528000in}}{\pgfqpoint{4.960000in}{3.696000in}}%
\pgfusepath{clip}%
\pgfsetbuttcap%
\pgfsetroundjoin%
\definecolor{currentfill}{rgb}{0.000000,0.000000,0.000000}%
\pgfsetfillcolor{currentfill}%
\pgfsetlinewidth{1.003750pt}%
\definecolor{currentstroke}{rgb}{0.000000,0.000000,0.000000}%
\pgfsetstrokecolor{currentstroke}%
\pgfsetdash{}{0pt}%
\pgfpathmoveto{\pgfqpoint{5.504545in}{2.877687in}}%
\pgfpathcurveto{\pgfqpoint{5.515596in}{2.877687in}}{\pgfqpoint{5.526195in}{2.882077in}}{\pgfqpoint{5.534008in}{2.889891in}}%
\pgfpathcurveto{\pgfqpoint{5.541822in}{2.897704in}}{\pgfqpoint{5.546212in}{2.908303in}}{\pgfqpoint{5.546212in}{2.919353in}}%
\pgfpathcurveto{\pgfqpoint{5.546212in}{2.930404in}}{\pgfqpoint{5.541822in}{2.941003in}}{\pgfqpoint{5.534008in}{2.948816in}}%
\pgfpathcurveto{\pgfqpoint{5.526195in}{2.956630in}}{\pgfqpoint{5.515596in}{2.961020in}}{\pgfqpoint{5.504545in}{2.961020in}}%
\pgfpathcurveto{\pgfqpoint{5.493495in}{2.961020in}}{\pgfqpoint{5.482896in}{2.956630in}}{\pgfqpoint{5.475083in}{2.948816in}}%
\pgfpathcurveto{\pgfqpoint{5.467269in}{2.941003in}}{\pgfqpoint{5.462879in}{2.930404in}}{\pgfqpoint{5.462879in}{2.919353in}}%
\pgfpathcurveto{\pgfqpoint{5.462879in}{2.908303in}}{\pgfqpoint{5.467269in}{2.897704in}}{\pgfqpoint{5.475083in}{2.889891in}}%
\pgfpathcurveto{\pgfqpoint{5.482896in}{2.882077in}}{\pgfqpoint{5.493495in}{2.877687in}}{\pgfqpoint{5.504545in}{2.877687in}}%
\pgfpathclose%
\pgfusepath{stroke,fill}%
\end{pgfscope}%
\begin{pgfscope}%
\pgfpathrectangle{\pgfqpoint{0.800000in}{0.528000in}}{\pgfqpoint{4.960000in}{3.696000in}}%
\pgfusepath{clip}%
\pgfsetbuttcap%
\pgfsetroundjoin%
\definecolor{currentfill}{rgb}{0.000000,0.000000,0.000000}%
\pgfsetfillcolor{currentfill}%
\pgfsetlinewidth{1.003750pt}%
\definecolor{currentstroke}{rgb}{0.000000,0.000000,0.000000}%
\pgfsetstrokecolor{currentstroke}%
\pgfsetdash{}{0pt}%
\pgfpathmoveto{\pgfqpoint{5.504545in}{2.877687in}}%
\pgfpathcurveto{\pgfqpoint{5.515596in}{2.877687in}}{\pgfqpoint{5.526195in}{2.882077in}}{\pgfqpoint{5.534008in}{2.889891in}}%
\pgfpathcurveto{\pgfqpoint{5.541822in}{2.897704in}}{\pgfqpoint{5.546212in}{2.908303in}}{\pgfqpoint{5.546212in}{2.919353in}}%
\pgfpathcurveto{\pgfqpoint{5.546212in}{2.930404in}}{\pgfqpoint{5.541822in}{2.941003in}}{\pgfqpoint{5.534008in}{2.948816in}}%
\pgfpathcurveto{\pgfqpoint{5.526195in}{2.956630in}}{\pgfqpoint{5.515596in}{2.961020in}}{\pgfqpoint{5.504545in}{2.961020in}}%
\pgfpathcurveto{\pgfqpoint{5.493495in}{2.961020in}}{\pgfqpoint{5.482896in}{2.956630in}}{\pgfqpoint{5.475083in}{2.948816in}}%
\pgfpathcurveto{\pgfqpoint{5.467269in}{2.941003in}}{\pgfqpoint{5.462879in}{2.930404in}}{\pgfqpoint{5.462879in}{2.919353in}}%
\pgfpathcurveto{\pgfqpoint{5.462879in}{2.908303in}}{\pgfqpoint{5.467269in}{2.897704in}}{\pgfqpoint{5.475083in}{2.889891in}}%
\pgfpathcurveto{\pgfqpoint{5.482896in}{2.882077in}}{\pgfqpoint{5.493495in}{2.877687in}}{\pgfqpoint{5.504545in}{2.877687in}}%
\pgfpathclose%
\pgfusepath{stroke,fill}%
\end{pgfscope}%
\begin{pgfscope}%
\pgfpathrectangle{\pgfqpoint{0.800000in}{0.528000in}}{\pgfqpoint{4.960000in}{3.696000in}}%
\pgfusepath{clip}%
\pgfsetbuttcap%
\pgfsetroundjoin%
\definecolor{currentfill}{rgb}{0.000000,0.000000,0.000000}%
\pgfsetfillcolor{currentfill}%
\pgfsetlinewidth{1.003750pt}%
\definecolor{currentstroke}{rgb}{0.000000,0.000000,0.000000}%
\pgfsetstrokecolor{currentstroke}%
\pgfsetdash{}{0pt}%
\pgfpathmoveto{\pgfqpoint{5.504545in}{2.877687in}}%
\pgfpathcurveto{\pgfqpoint{5.515596in}{2.877687in}}{\pgfqpoint{5.526195in}{2.882077in}}{\pgfqpoint{5.534008in}{2.889891in}}%
\pgfpathcurveto{\pgfqpoint{5.541822in}{2.897704in}}{\pgfqpoint{5.546212in}{2.908303in}}{\pgfqpoint{5.546212in}{2.919353in}}%
\pgfpathcurveto{\pgfqpoint{5.546212in}{2.930404in}}{\pgfqpoint{5.541822in}{2.941003in}}{\pgfqpoint{5.534008in}{2.948816in}}%
\pgfpathcurveto{\pgfqpoint{5.526195in}{2.956630in}}{\pgfqpoint{5.515596in}{2.961020in}}{\pgfqpoint{5.504545in}{2.961020in}}%
\pgfpathcurveto{\pgfqpoint{5.493495in}{2.961020in}}{\pgfqpoint{5.482896in}{2.956630in}}{\pgfqpoint{5.475083in}{2.948816in}}%
\pgfpathcurveto{\pgfqpoint{5.467269in}{2.941003in}}{\pgfqpoint{5.462879in}{2.930404in}}{\pgfqpoint{5.462879in}{2.919353in}}%
\pgfpathcurveto{\pgfqpoint{5.462879in}{2.908303in}}{\pgfqpoint{5.467269in}{2.897704in}}{\pgfqpoint{5.475083in}{2.889891in}}%
\pgfpathcurveto{\pgfqpoint{5.482896in}{2.882077in}}{\pgfqpoint{5.493495in}{2.877687in}}{\pgfqpoint{5.504545in}{2.877687in}}%
\pgfpathclose%
\pgfusepath{stroke,fill}%
\end{pgfscope}%
\begin{pgfscope}%
\pgfpathrectangle{\pgfqpoint{0.800000in}{0.528000in}}{\pgfqpoint{4.960000in}{3.696000in}}%
\pgfusepath{clip}%
\pgfsetbuttcap%
\pgfsetroundjoin%
\definecolor{currentfill}{rgb}{0.000000,0.000000,0.000000}%
\pgfsetfillcolor{currentfill}%
\pgfsetlinewidth{1.003750pt}%
\definecolor{currentstroke}{rgb}{0.000000,0.000000,0.000000}%
\pgfsetstrokecolor{currentstroke}%
\pgfsetdash{}{0pt}%
\pgfpathmoveto{\pgfqpoint{5.504545in}{2.877687in}}%
\pgfpathcurveto{\pgfqpoint{5.515596in}{2.877687in}}{\pgfqpoint{5.526195in}{2.882077in}}{\pgfqpoint{5.534008in}{2.889891in}}%
\pgfpathcurveto{\pgfqpoint{5.541822in}{2.897704in}}{\pgfqpoint{5.546212in}{2.908303in}}{\pgfqpoint{5.546212in}{2.919353in}}%
\pgfpathcurveto{\pgfqpoint{5.546212in}{2.930404in}}{\pgfqpoint{5.541822in}{2.941003in}}{\pgfqpoint{5.534008in}{2.948816in}}%
\pgfpathcurveto{\pgfqpoint{5.526195in}{2.956630in}}{\pgfqpoint{5.515596in}{2.961020in}}{\pgfqpoint{5.504545in}{2.961020in}}%
\pgfpathcurveto{\pgfqpoint{5.493495in}{2.961020in}}{\pgfqpoint{5.482896in}{2.956630in}}{\pgfqpoint{5.475083in}{2.948816in}}%
\pgfpathcurveto{\pgfqpoint{5.467269in}{2.941003in}}{\pgfqpoint{5.462879in}{2.930404in}}{\pgfqpoint{5.462879in}{2.919353in}}%
\pgfpathcurveto{\pgfqpoint{5.462879in}{2.908303in}}{\pgfqpoint{5.467269in}{2.897704in}}{\pgfqpoint{5.475083in}{2.889891in}}%
\pgfpathcurveto{\pgfqpoint{5.482896in}{2.882077in}}{\pgfqpoint{5.493495in}{2.877687in}}{\pgfqpoint{5.504545in}{2.877687in}}%
\pgfpathclose%
\pgfusepath{stroke,fill}%
\end{pgfscope}%
\begin{pgfscope}%
\pgfpathrectangle{\pgfqpoint{0.800000in}{0.528000in}}{\pgfqpoint{4.960000in}{3.696000in}}%
\pgfusepath{clip}%
\pgfsetbuttcap%
\pgfsetroundjoin%
\definecolor{currentfill}{rgb}{0.000000,0.000000,0.000000}%
\pgfsetfillcolor{currentfill}%
\pgfsetlinewidth{1.003750pt}%
\definecolor{currentstroke}{rgb}{0.000000,0.000000,0.000000}%
\pgfsetstrokecolor{currentstroke}%
\pgfsetdash{}{0pt}%
\pgfpathmoveto{\pgfqpoint{5.504545in}{3.984333in}}%
\pgfpathcurveto{\pgfqpoint{5.515596in}{3.984333in}}{\pgfqpoint{5.526195in}{3.988724in}}{\pgfqpoint{5.534008in}{3.996537in}}%
\pgfpathcurveto{\pgfqpoint{5.541822in}{4.004351in}}{\pgfqpoint{5.546212in}{4.014950in}}{\pgfqpoint{5.546212in}{4.026000in}}%
\pgfpathcurveto{\pgfqpoint{5.546212in}{4.037050in}}{\pgfqpoint{5.541822in}{4.047649in}}{\pgfqpoint{5.534008in}{4.055463in}}%
\pgfpathcurveto{\pgfqpoint{5.526195in}{4.063276in}}{\pgfqpoint{5.515596in}{4.067667in}}{\pgfqpoint{5.504545in}{4.067667in}}%
\pgfpathcurveto{\pgfqpoint{5.493495in}{4.067667in}}{\pgfqpoint{5.482896in}{4.063276in}}{\pgfqpoint{5.475083in}{4.055463in}}%
\pgfpathcurveto{\pgfqpoint{5.467269in}{4.047649in}}{\pgfqpoint{5.462879in}{4.037050in}}{\pgfqpoint{5.462879in}{4.026000in}}%
\pgfpathcurveto{\pgfqpoint{5.462879in}{4.014950in}}{\pgfqpoint{5.467269in}{4.004351in}}{\pgfqpoint{5.475083in}{3.996537in}}%
\pgfpathcurveto{\pgfqpoint{5.482896in}{3.988724in}}{\pgfqpoint{5.493495in}{3.984333in}}{\pgfqpoint{5.504545in}{3.984333in}}%
\pgfpathclose%
\pgfusepath{stroke,fill}%
\end{pgfscope}%
\begin{pgfscope}%
\pgfsetbuttcap%
\pgfsetroundjoin%
\definecolor{currentfill}{rgb}{0.000000,0.000000,0.000000}%
\pgfsetfillcolor{currentfill}%
\pgfsetlinewidth{0.803000pt}%
\definecolor{currentstroke}{rgb}{0.000000,0.000000,0.000000}%
\pgfsetstrokecolor{currentstroke}%
\pgfsetdash{}{0pt}%
\pgfsys@defobject{currentmarker}{\pgfqpoint{0.000000in}{-0.048611in}}{\pgfqpoint{0.000000in}{0.000000in}}{%
\pgfpathmoveto{\pgfqpoint{0.000000in}{0.000000in}}%
\pgfpathlineto{\pgfqpoint{0.000000in}{-0.048611in}}%
\pgfusepath{stroke,fill}%
}%
\begin{pgfscope}%
\pgfsys@transformshift{1.025906in}{0.528000in}%
\pgfsys@useobject{currentmarker}{}%
\end{pgfscope}%
\end{pgfscope}%
\begin{pgfscope}%
\definecolor{textcolor}{rgb}{0.000000,0.000000,0.000000}%
\pgfsetstrokecolor{textcolor}%
\pgfsetfillcolor{textcolor}%
\pgftext[x=1.025906in,y=0.430778in,,top]{\color{textcolor}\sffamily\fontsize{10.000000}{12.000000}\selectfont 20}%
\end{pgfscope}%
\begin{pgfscope}%
\pgfsetbuttcap%
\pgfsetroundjoin%
\definecolor{currentfill}{rgb}{0.000000,0.000000,0.000000}%
\pgfsetfillcolor{currentfill}%
\pgfsetlinewidth{0.803000pt}%
\definecolor{currentstroke}{rgb}{0.000000,0.000000,0.000000}%
\pgfsetstrokecolor{currentstroke}%
\pgfsetdash{}{0pt}%
\pgfsys@defobject{currentmarker}{\pgfqpoint{0.000000in}{-0.048611in}}{\pgfqpoint{0.000000in}{0.000000in}}{%
\pgfpathmoveto{\pgfqpoint{0.000000in}{0.000000in}}%
\pgfpathlineto{\pgfqpoint{0.000000in}{-0.048611in}}%
\pgfusepath{stroke,fill}%
}%
\begin{pgfscope}%
\pgfsys@transformshift{2.518786in}{0.528000in}%
\pgfsys@useobject{currentmarker}{}%
\end{pgfscope}%
\end{pgfscope}%
\begin{pgfscope}%
\definecolor{textcolor}{rgb}{0.000000,0.000000,0.000000}%
\pgfsetstrokecolor{textcolor}%
\pgfsetfillcolor{textcolor}%
\pgftext[x=2.518786in,y=0.430778in,,top]{\color{textcolor}\sffamily\fontsize{10.000000}{12.000000}\selectfont 40}%
\end{pgfscope}%
\begin{pgfscope}%
\pgfsetbuttcap%
\pgfsetroundjoin%
\definecolor{currentfill}{rgb}{0.000000,0.000000,0.000000}%
\pgfsetfillcolor{currentfill}%
\pgfsetlinewidth{0.803000pt}%
\definecolor{currentstroke}{rgb}{0.000000,0.000000,0.000000}%
\pgfsetstrokecolor{currentstroke}%
\pgfsetdash{}{0pt}%
\pgfsys@defobject{currentmarker}{\pgfqpoint{0.000000in}{-0.048611in}}{\pgfqpoint{0.000000in}{0.000000in}}{%
\pgfpathmoveto{\pgfqpoint{0.000000in}{0.000000in}}%
\pgfpathlineto{\pgfqpoint{0.000000in}{-0.048611in}}%
\pgfusepath{stroke,fill}%
}%
\begin{pgfscope}%
\pgfsys@transformshift{4.011666in}{0.528000in}%
\pgfsys@useobject{currentmarker}{}%
\end{pgfscope}%
\end{pgfscope}%
\begin{pgfscope}%
\definecolor{textcolor}{rgb}{0.000000,0.000000,0.000000}%
\pgfsetstrokecolor{textcolor}%
\pgfsetfillcolor{textcolor}%
\pgftext[x=4.011666in,y=0.430778in,,top]{\color{textcolor}\sffamily\fontsize{10.000000}{12.000000}\selectfont 60}%
\end{pgfscope}%
\begin{pgfscope}%
\pgfsetbuttcap%
\pgfsetroundjoin%
\definecolor{currentfill}{rgb}{0.000000,0.000000,0.000000}%
\pgfsetfillcolor{currentfill}%
\pgfsetlinewidth{0.803000pt}%
\definecolor{currentstroke}{rgb}{0.000000,0.000000,0.000000}%
\pgfsetstrokecolor{currentstroke}%
\pgfsetdash{}{0pt}%
\pgfsys@defobject{currentmarker}{\pgfqpoint{0.000000in}{-0.048611in}}{\pgfqpoint{0.000000in}{0.000000in}}{%
\pgfpathmoveto{\pgfqpoint{0.000000in}{0.000000in}}%
\pgfpathlineto{\pgfqpoint{0.000000in}{-0.048611in}}%
\pgfusepath{stroke,fill}%
}%
\begin{pgfscope}%
\pgfsys@transformshift{5.504545in}{0.528000in}%
\pgfsys@useobject{currentmarker}{}%
\end{pgfscope}%
\end{pgfscope}%
\begin{pgfscope}%
\definecolor{textcolor}{rgb}{0.000000,0.000000,0.000000}%
\pgfsetstrokecolor{textcolor}%
\pgfsetfillcolor{textcolor}%
\pgftext[x=5.504545in,y=0.430778in,,top]{\color{textcolor}\sffamily\fontsize{10.000000}{12.000000}\selectfont 80}%
\end{pgfscope}%
\begin{pgfscope}%
\definecolor{textcolor}{rgb}{0.000000,0.000000,0.000000}%
\pgfsetstrokecolor{textcolor}%
\pgfsetfillcolor{textcolor}%
\pgftext[x=3.280000in,y=0.240809in,,top]{\color{textcolor}\sffamily\fontsize{10.000000}{12.000000}\selectfont \(\displaystyle k\)}%
\end{pgfscope}%
\begin{pgfscope}%
\pgfsetbuttcap%
\pgfsetroundjoin%
\definecolor{currentfill}{rgb}{0.000000,0.000000,0.000000}%
\pgfsetfillcolor{currentfill}%
\pgfsetlinewidth{0.803000pt}%
\definecolor{currentstroke}{rgb}{0.000000,0.000000,0.000000}%
\pgfsetstrokecolor{currentstroke}%
\pgfsetdash{}{0pt}%
\pgfsys@defobject{currentmarker}{\pgfqpoint{-0.048611in}{0.000000in}}{\pgfqpoint{0.000000in}{0.000000in}}{%
\pgfpathmoveto{\pgfqpoint{0.000000in}{0.000000in}}%
\pgfpathlineto{\pgfqpoint{-0.048611in}{0.000000in}}%
\pgfusepath{stroke,fill}%
}%
\begin{pgfscope}%
\pgfsys@transformshift{0.800000in}{0.706060in}%
\pgfsys@useobject{currentmarker}{}%
\end{pgfscope}%
\end{pgfscope}%
\begin{pgfscope}%
\definecolor{textcolor}{rgb}{0.000000,0.000000,0.000000}%
\pgfsetstrokecolor{textcolor}%
\pgfsetfillcolor{textcolor}%
\pgftext[x=0.614413in,y=0.653299in,left,base]{\color{textcolor}\sffamily\fontsize{10.000000}{12.000000}\selectfont 9}%
\end{pgfscope}%
\begin{pgfscope}%
\pgfsetbuttcap%
\pgfsetroundjoin%
\definecolor{currentfill}{rgb}{0.000000,0.000000,0.000000}%
\pgfsetfillcolor{currentfill}%
\pgfsetlinewidth{0.803000pt}%
\definecolor{currentstroke}{rgb}{0.000000,0.000000,0.000000}%
\pgfsetstrokecolor{currentstroke}%
\pgfsetdash{}{0pt}%
\pgfsys@defobject{currentmarker}{\pgfqpoint{-0.048611in}{0.000000in}}{\pgfqpoint{0.000000in}{0.000000in}}{%
\pgfpathmoveto{\pgfqpoint{0.000000in}{0.000000in}}%
\pgfpathlineto{\pgfqpoint{-0.048611in}{0.000000in}}%
\pgfusepath{stroke,fill}%
}%
\begin{pgfscope}%
\pgfsys@transformshift{0.800000in}{1.812707in}%
\pgfsys@useobject{currentmarker}{}%
\end{pgfscope}%
\end{pgfscope}%
\begin{pgfscope}%
\definecolor{textcolor}{rgb}{0.000000,0.000000,0.000000}%
\pgfsetstrokecolor{textcolor}%
\pgfsetfillcolor{textcolor}%
\pgftext[x=0.526047in,y=1.759945in,left,base]{\color{textcolor}\sffamily\fontsize{10.000000}{12.000000}\selectfont 10}%
\end{pgfscope}%
\begin{pgfscope}%
\pgfsetbuttcap%
\pgfsetroundjoin%
\definecolor{currentfill}{rgb}{0.000000,0.000000,0.000000}%
\pgfsetfillcolor{currentfill}%
\pgfsetlinewidth{0.803000pt}%
\definecolor{currentstroke}{rgb}{0.000000,0.000000,0.000000}%
\pgfsetstrokecolor{currentstroke}%
\pgfsetdash{}{0pt}%
\pgfsys@defobject{currentmarker}{\pgfqpoint{-0.048611in}{0.000000in}}{\pgfqpoint{0.000000in}{0.000000in}}{%
\pgfpathmoveto{\pgfqpoint{0.000000in}{0.000000in}}%
\pgfpathlineto{\pgfqpoint{-0.048611in}{0.000000in}}%
\pgfusepath{stroke,fill}%
}%
\begin{pgfscope}%
\pgfsys@transformshift{0.800000in}{2.919353in}%
\pgfsys@useobject{currentmarker}{}%
\end{pgfscope}%
\end{pgfscope}%
\begin{pgfscope}%
\definecolor{textcolor}{rgb}{0.000000,0.000000,0.000000}%
\pgfsetstrokecolor{textcolor}%
\pgfsetfillcolor{textcolor}%
\pgftext[x=0.526047in,y=2.866592in,left,base]{\color{textcolor}\sffamily\fontsize{10.000000}{12.000000}\selectfont 11}%
\end{pgfscope}%
\begin{pgfscope}%
\pgfsetbuttcap%
\pgfsetroundjoin%
\definecolor{currentfill}{rgb}{0.000000,0.000000,0.000000}%
\pgfsetfillcolor{currentfill}%
\pgfsetlinewidth{0.803000pt}%
\definecolor{currentstroke}{rgb}{0.000000,0.000000,0.000000}%
\pgfsetstrokecolor{currentstroke}%
\pgfsetdash{}{0pt}%
\pgfsys@defobject{currentmarker}{\pgfqpoint{-0.048611in}{0.000000in}}{\pgfqpoint{0.000000in}{0.000000in}}{%
\pgfpathmoveto{\pgfqpoint{0.000000in}{0.000000in}}%
\pgfpathlineto{\pgfqpoint{-0.048611in}{0.000000in}}%
\pgfusepath{stroke,fill}%
}%
\begin{pgfscope}%
\pgfsys@transformshift{0.800000in}{4.026000in}%
\pgfsys@useobject{currentmarker}{}%
\end{pgfscope}%
\end{pgfscope}%
\begin{pgfscope}%
\definecolor{textcolor}{rgb}{0.000000,0.000000,0.000000}%
\pgfsetstrokecolor{textcolor}%
\pgfsetfillcolor{textcolor}%
\pgftext[x=0.526047in,y=3.973238in,left,base]{\color{textcolor}\sffamily\fontsize{10.000000}{12.000000}\selectfont 12}%
\end{pgfscope}%
\begin{pgfscope}%
\definecolor{textcolor}{rgb}{0.000000,0.000000,0.000000}%
\pgfsetstrokecolor{textcolor}%
\pgfsetfillcolor{textcolor}%
\pgftext[x=0.470492in,y=2.376000in,,bottom,rotate=90.000000]{\color{textcolor}\sffamily\fontsize{10.000000}{12.000000}\selectfont Number of GMRES Iterations}%
\end{pgfscope}%
\begin{pgfscope}%
\pgfsetrectcap%
\pgfsetmiterjoin%
\pgfsetlinewidth{0.803000pt}%
\definecolor{currentstroke}{rgb}{0.000000,0.000000,0.000000}%
\pgfsetstrokecolor{currentstroke}%
\pgfsetdash{}{0pt}%
\pgfpathmoveto{\pgfqpoint{0.800000in}{0.528000in}}%
\pgfpathlineto{\pgfqpoint{0.800000in}{4.224000in}}%
\pgfusepath{stroke}%
\end{pgfscope}%
\begin{pgfscope}%
\pgfsetrectcap%
\pgfsetmiterjoin%
\pgfsetlinewidth{0.803000pt}%
\definecolor{currentstroke}{rgb}{0.000000,0.000000,0.000000}%
\pgfsetstrokecolor{currentstroke}%
\pgfsetdash{}{0pt}%
\pgfpathmoveto{\pgfqpoint{5.760000in}{0.528000in}}%
\pgfpathlineto{\pgfqpoint{5.760000in}{4.224000in}}%
\pgfusepath{stroke}%
\end{pgfscope}%
\begin{pgfscope}%
\pgfsetrectcap%
\pgfsetmiterjoin%
\pgfsetlinewidth{0.803000pt}%
\definecolor{currentstroke}{rgb}{0.000000,0.000000,0.000000}%
\pgfsetstrokecolor{currentstroke}%
\pgfsetdash{}{0pt}%
\pgfpathmoveto{\pgfqpoint{0.800000in}{0.528000in}}%
\pgfpathlineto{\pgfqpoint{5.760000in}{0.528000in}}%
\pgfusepath{stroke}%
\end{pgfscope}%
\begin{pgfscope}%
\pgfsetrectcap%
\pgfsetmiterjoin%
\pgfsetlinewidth{0.803000pt}%
\definecolor{currentstroke}{rgb}{0.000000,0.000000,0.000000}%
\pgfsetstrokecolor{currentstroke}%
\pgfsetdash{}{0pt}%
\pgfpathmoveto{\pgfqpoint{0.800000in}{4.224000in}}%
\pgfpathlineto{\pgfqpoint{5.760000in}{4.224000in}}%
\pgfusepath{stroke}%
\end{pgfscope}%
\end{pgfpicture}%
\makeatother%
\endgroup%

  \caption{GMRES iteration counts for $\alpha = 0.5/k^{1/2}$}\label{fig:linfinityA1}
\end{subfigure}

    \begin{subfigure}{\textwidth}
      \centering
%% Creator: Matplotlib, PGF backend
%%
%% To include the figure in your LaTeX document, write
%%   \input{<filename>.pgf}
%%
%% Make sure the required packages are loaded in your preamble
%%   \usepackage{pgf}
%%
%% Figures using additional raster images can only be included by \input if
%% they are in the same directory as the main LaTeX file. For loading figures
%% from other directories you can use the `import` package
%%   \usepackage{import}
%% and then include the figures with
%%   \import{<path to file>}{<filename>.pgf}
%%
%% Matplotlib used the following preamble
%%   \usepackage{fontspec}
%%   \setmainfont{DejaVuSerif.ttf}[Path=/home/owen/progs/firedrake-complex/firedrake/lib/python3.5/site-packages/matplotlib/mpl-data/fonts/ttf/]
%%   \setsansfont{DejaVuSans.ttf}[Path=/home/owen/progs/firedrake-complex/firedrake/lib/python3.5/site-packages/matplotlib/mpl-data/fonts/ttf/]
%%   \setmonofont{DejaVuSansMono.ttf}[Path=/home/owen/progs/firedrake-complex/firedrake/lib/python3.5/site-packages/matplotlib/mpl-data/fonts/ttf/]
%%
\begingroup%
\makeatletter%
\begin{pgfpicture}%
\pgfpathrectangle{\pgfpointorigin}{\pgfqpoint{4.500000in}{2.500000in}}%
\pgfusepath{use as bounding box, clip}%
\begin{pgfscope}%
\pgfsetbuttcap%
\pgfsetmiterjoin%
\definecolor{currentfill}{rgb}{1.000000,1.000000,1.000000}%
\pgfsetfillcolor{currentfill}%
\pgfsetlinewidth{0.000000pt}%
\definecolor{currentstroke}{rgb}{1.000000,1.000000,1.000000}%
\pgfsetstrokecolor{currentstroke}%
\pgfsetdash{}{0pt}%
\pgfpathmoveto{\pgfqpoint{0.000000in}{0.000000in}}%
\pgfpathlineto{\pgfqpoint{4.500000in}{0.000000in}}%
\pgfpathlineto{\pgfqpoint{4.500000in}{2.500000in}}%
\pgfpathlineto{\pgfqpoint{0.000000in}{2.500000in}}%
\pgfpathclose%
\pgfusepath{fill}%
\end{pgfscope}%
\begin{pgfscope}%
\pgfsetbuttcap%
\pgfsetmiterjoin%
\definecolor{currentfill}{rgb}{1.000000,1.000000,1.000000}%
\pgfsetfillcolor{currentfill}%
\pgfsetlinewidth{0.000000pt}%
\definecolor{currentstroke}{rgb}{0.000000,0.000000,0.000000}%
\pgfsetstrokecolor{currentstroke}%
\pgfsetstrokeopacity{0.000000}%
\pgfsetdash{}{0pt}%
\pgfpathmoveto{\pgfqpoint{0.562500in}{0.275000in}}%
\pgfpathlineto{\pgfqpoint{4.050000in}{0.275000in}}%
\pgfpathlineto{\pgfqpoint{4.050000in}{2.200000in}}%
\pgfpathlineto{\pgfqpoint{0.562500in}{2.200000in}}%
\pgfpathclose%
\pgfusepath{fill}%
\end{pgfscope}%
\begin{pgfscope}%
\pgfpathrectangle{\pgfqpoint{0.562500in}{0.275000in}}{\pgfqpoint{3.487500in}{1.925000in}}%
\pgfusepath{clip}%
\pgfsetbuttcap%
\pgfsetroundjoin%
\definecolor{currentfill}{rgb}{0.000000,0.000000,0.000000}%
\pgfsetfillcolor{currentfill}%
\pgfsetlinewidth{1.003750pt}%
\definecolor{currentstroke}{rgb}{0.000000,0.000000,0.000000}%
\pgfsetstrokecolor{currentstroke}%
\pgfsetdash{}{0pt}%
\pgfpathmoveto{\pgfqpoint{0.721249in}{0.356667in}}%
\pgfpathcurveto{\pgfqpoint{0.726774in}{0.356667in}}{\pgfqpoint{0.732073in}{0.358862in}}{\pgfqpoint{0.735980in}{0.362769in}}%
\pgfpathcurveto{\pgfqpoint{0.739887in}{0.366675in}}{\pgfqpoint{0.742082in}{0.371975in}}{\pgfqpoint{0.742082in}{0.377500in}}%
\pgfpathcurveto{\pgfqpoint{0.742082in}{0.383025in}}{\pgfqpoint{0.739887in}{0.388325in}}{\pgfqpoint{0.735980in}{0.392231in}}%
\pgfpathcurveto{\pgfqpoint{0.732073in}{0.396138in}}{\pgfqpoint{0.726774in}{0.398333in}}{\pgfqpoint{0.721249in}{0.398333in}}%
\pgfpathcurveto{\pgfqpoint{0.715724in}{0.398333in}}{\pgfqpoint{0.710424in}{0.396138in}}{\pgfqpoint{0.706518in}{0.392231in}}%
\pgfpathcurveto{\pgfqpoint{0.702611in}{0.388325in}}{\pgfqpoint{0.700416in}{0.383025in}}{\pgfqpoint{0.700416in}{0.377500in}}%
\pgfpathcurveto{\pgfqpoint{0.700416in}{0.371975in}}{\pgfqpoint{0.702611in}{0.366675in}}{\pgfqpoint{0.706518in}{0.362769in}}%
\pgfpathcurveto{\pgfqpoint{0.710424in}{0.358862in}}{\pgfqpoint{0.715724in}{0.356667in}}{\pgfqpoint{0.721249in}{0.356667in}}%
\pgfpathclose%
\pgfusepath{stroke,fill}%
\end{pgfscope}%
\begin{pgfscope}%
\pgfpathrectangle{\pgfqpoint{0.562500in}{0.275000in}}{\pgfqpoint{3.487500in}{1.925000in}}%
\pgfusepath{clip}%
\pgfsetbuttcap%
\pgfsetroundjoin%
\definecolor{currentfill}{rgb}{0.000000,0.000000,0.000000}%
\pgfsetfillcolor{currentfill}%
\pgfsetlinewidth{1.003750pt}%
\definecolor{currentstroke}{rgb}{0.000000,0.000000,0.000000}%
\pgfsetstrokecolor{currentstroke}%
\pgfsetdash{}{0pt}%
\pgfpathmoveto{\pgfqpoint{0.721249in}{0.356667in}}%
\pgfpathcurveto{\pgfqpoint{0.726774in}{0.356667in}}{\pgfqpoint{0.732073in}{0.358862in}}{\pgfqpoint{0.735980in}{0.362769in}}%
\pgfpathcurveto{\pgfqpoint{0.739887in}{0.366675in}}{\pgfqpoint{0.742082in}{0.371975in}}{\pgfqpoint{0.742082in}{0.377500in}}%
\pgfpathcurveto{\pgfqpoint{0.742082in}{0.383025in}}{\pgfqpoint{0.739887in}{0.388325in}}{\pgfqpoint{0.735980in}{0.392231in}}%
\pgfpathcurveto{\pgfqpoint{0.732073in}{0.396138in}}{\pgfqpoint{0.726774in}{0.398333in}}{\pgfqpoint{0.721249in}{0.398333in}}%
\pgfpathcurveto{\pgfqpoint{0.715724in}{0.398333in}}{\pgfqpoint{0.710424in}{0.396138in}}{\pgfqpoint{0.706518in}{0.392231in}}%
\pgfpathcurveto{\pgfqpoint{0.702611in}{0.388325in}}{\pgfqpoint{0.700416in}{0.383025in}}{\pgfqpoint{0.700416in}{0.377500in}}%
\pgfpathcurveto{\pgfqpoint{0.700416in}{0.371975in}}{\pgfqpoint{0.702611in}{0.366675in}}{\pgfqpoint{0.706518in}{0.362769in}}%
\pgfpathcurveto{\pgfqpoint{0.710424in}{0.358862in}}{\pgfqpoint{0.715724in}{0.356667in}}{\pgfqpoint{0.721249in}{0.356667in}}%
\pgfpathclose%
\pgfusepath{stroke,fill}%
\end{pgfscope}%
\begin{pgfscope}%
\pgfpathrectangle{\pgfqpoint{0.562500in}{0.275000in}}{\pgfqpoint{3.487500in}{1.925000in}}%
\pgfusepath{clip}%
\pgfsetbuttcap%
\pgfsetroundjoin%
\definecolor{currentfill}{rgb}{0.000000,0.000000,0.000000}%
\pgfsetfillcolor{currentfill}%
\pgfsetlinewidth{1.003750pt}%
\definecolor{currentstroke}{rgb}{0.000000,0.000000,0.000000}%
\pgfsetstrokecolor{currentstroke}%
\pgfsetdash{}{0pt}%
\pgfpathmoveto{\pgfqpoint{0.721249in}{0.356667in}}%
\pgfpathcurveto{\pgfqpoint{0.726774in}{0.356667in}}{\pgfqpoint{0.732073in}{0.358862in}}{\pgfqpoint{0.735980in}{0.362769in}}%
\pgfpathcurveto{\pgfqpoint{0.739887in}{0.366675in}}{\pgfqpoint{0.742082in}{0.371975in}}{\pgfqpoint{0.742082in}{0.377500in}}%
\pgfpathcurveto{\pgfqpoint{0.742082in}{0.383025in}}{\pgfqpoint{0.739887in}{0.388325in}}{\pgfqpoint{0.735980in}{0.392231in}}%
\pgfpathcurveto{\pgfqpoint{0.732073in}{0.396138in}}{\pgfqpoint{0.726774in}{0.398333in}}{\pgfqpoint{0.721249in}{0.398333in}}%
\pgfpathcurveto{\pgfqpoint{0.715724in}{0.398333in}}{\pgfqpoint{0.710424in}{0.396138in}}{\pgfqpoint{0.706518in}{0.392231in}}%
\pgfpathcurveto{\pgfqpoint{0.702611in}{0.388325in}}{\pgfqpoint{0.700416in}{0.383025in}}{\pgfqpoint{0.700416in}{0.377500in}}%
\pgfpathcurveto{\pgfqpoint{0.700416in}{0.371975in}}{\pgfqpoint{0.702611in}{0.366675in}}{\pgfqpoint{0.706518in}{0.362769in}}%
\pgfpathcurveto{\pgfqpoint{0.710424in}{0.358862in}}{\pgfqpoint{0.715724in}{0.356667in}}{\pgfqpoint{0.721249in}{0.356667in}}%
\pgfpathclose%
\pgfusepath{stroke,fill}%
\end{pgfscope}%
\begin{pgfscope}%
\pgfpathrectangle{\pgfqpoint{0.562500in}{0.275000in}}{\pgfqpoint{3.487500in}{1.925000in}}%
\pgfusepath{clip}%
\pgfsetbuttcap%
\pgfsetroundjoin%
\definecolor{currentfill}{rgb}{0.000000,0.000000,0.000000}%
\pgfsetfillcolor{currentfill}%
\pgfsetlinewidth{1.003750pt}%
\definecolor{currentstroke}{rgb}{0.000000,0.000000,0.000000}%
\pgfsetstrokecolor{currentstroke}%
\pgfsetdash{}{0pt}%
\pgfpathmoveto{\pgfqpoint{0.721249in}{0.356667in}}%
\pgfpathcurveto{\pgfqpoint{0.726774in}{0.356667in}}{\pgfqpoint{0.732073in}{0.358862in}}{\pgfqpoint{0.735980in}{0.362769in}}%
\pgfpathcurveto{\pgfqpoint{0.739887in}{0.366675in}}{\pgfqpoint{0.742082in}{0.371975in}}{\pgfqpoint{0.742082in}{0.377500in}}%
\pgfpathcurveto{\pgfqpoint{0.742082in}{0.383025in}}{\pgfqpoint{0.739887in}{0.388325in}}{\pgfqpoint{0.735980in}{0.392231in}}%
\pgfpathcurveto{\pgfqpoint{0.732073in}{0.396138in}}{\pgfqpoint{0.726774in}{0.398333in}}{\pgfqpoint{0.721249in}{0.398333in}}%
\pgfpathcurveto{\pgfqpoint{0.715724in}{0.398333in}}{\pgfqpoint{0.710424in}{0.396138in}}{\pgfqpoint{0.706518in}{0.392231in}}%
\pgfpathcurveto{\pgfqpoint{0.702611in}{0.388325in}}{\pgfqpoint{0.700416in}{0.383025in}}{\pgfqpoint{0.700416in}{0.377500in}}%
\pgfpathcurveto{\pgfqpoint{0.700416in}{0.371975in}}{\pgfqpoint{0.702611in}{0.366675in}}{\pgfqpoint{0.706518in}{0.362769in}}%
\pgfpathcurveto{\pgfqpoint{0.710424in}{0.358862in}}{\pgfqpoint{0.715724in}{0.356667in}}{\pgfqpoint{0.721249in}{0.356667in}}%
\pgfpathclose%
\pgfusepath{stroke,fill}%
\end{pgfscope}%
\begin{pgfscope}%
\pgfpathrectangle{\pgfqpoint{0.562500in}{0.275000in}}{\pgfqpoint{3.487500in}{1.925000in}}%
\pgfusepath{clip}%
\pgfsetbuttcap%
\pgfsetroundjoin%
\definecolor{currentfill}{rgb}{0.000000,0.000000,0.000000}%
\pgfsetfillcolor{currentfill}%
\pgfsetlinewidth{1.003750pt}%
\definecolor{currentstroke}{rgb}{0.000000,0.000000,0.000000}%
\pgfsetstrokecolor{currentstroke}%
\pgfsetdash{}{0pt}%
\pgfpathmoveto{\pgfqpoint{0.721249in}{0.356667in}}%
\pgfpathcurveto{\pgfqpoint{0.726774in}{0.356667in}}{\pgfqpoint{0.732073in}{0.358862in}}{\pgfqpoint{0.735980in}{0.362769in}}%
\pgfpathcurveto{\pgfqpoint{0.739887in}{0.366675in}}{\pgfqpoint{0.742082in}{0.371975in}}{\pgfqpoint{0.742082in}{0.377500in}}%
\pgfpathcurveto{\pgfqpoint{0.742082in}{0.383025in}}{\pgfqpoint{0.739887in}{0.388325in}}{\pgfqpoint{0.735980in}{0.392231in}}%
\pgfpathcurveto{\pgfqpoint{0.732073in}{0.396138in}}{\pgfqpoint{0.726774in}{0.398333in}}{\pgfqpoint{0.721249in}{0.398333in}}%
\pgfpathcurveto{\pgfqpoint{0.715724in}{0.398333in}}{\pgfqpoint{0.710424in}{0.396138in}}{\pgfqpoint{0.706518in}{0.392231in}}%
\pgfpathcurveto{\pgfqpoint{0.702611in}{0.388325in}}{\pgfqpoint{0.700416in}{0.383025in}}{\pgfqpoint{0.700416in}{0.377500in}}%
\pgfpathcurveto{\pgfqpoint{0.700416in}{0.371975in}}{\pgfqpoint{0.702611in}{0.366675in}}{\pgfqpoint{0.706518in}{0.362769in}}%
\pgfpathcurveto{\pgfqpoint{0.710424in}{0.358862in}}{\pgfqpoint{0.715724in}{0.356667in}}{\pgfqpoint{0.721249in}{0.356667in}}%
\pgfpathclose%
\pgfusepath{stroke,fill}%
\end{pgfscope}%
\begin{pgfscope}%
\pgfpathrectangle{\pgfqpoint{0.562500in}{0.275000in}}{\pgfqpoint{3.487500in}{1.925000in}}%
\pgfusepath{clip}%
\pgfsetbuttcap%
\pgfsetroundjoin%
\definecolor{currentfill}{rgb}{0.000000,0.000000,0.000000}%
\pgfsetfillcolor{currentfill}%
\pgfsetlinewidth{1.003750pt}%
\definecolor{currentstroke}{rgb}{0.000000,0.000000,0.000000}%
\pgfsetstrokecolor{currentstroke}%
\pgfsetdash{}{0pt}%
\pgfpathmoveto{\pgfqpoint{0.721249in}{0.356667in}}%
\pgfpathcurveto{\pgfqpoint{0.726774in}{0.356667in}}{\pgfqpoint{0.732073in}{0.358862in}}{\pgfqpoint{0.735980in}{0.362769in}}%
\pgfpathcurveto{\pgfqpoint{0.739887in}{0.366675in}}{\pgfqpoint{0.742082in}{0.371975in}}{\pgfqpoint{0.742082in}{0.377500in}}%
\pgfpathcurveto{\pgfqpoint{0.742082in}{0.383025in}}{\pgfqpoint{0.739887in}{0.388325in}}{\pgfqpoint{0.735980in}{0.392231in}}%
\pgfpathcurveto{\pgfqpoint{0.732073in}{0.396138in}}{\pgfqpoint{0.726774in}{0.398333in}}{\pgfqpoint{0.721249in}{0.398333in}}%
\pgfpathcurveto{\pgfqpoint{0.715724in}{0.398333in}}{\pgfqpoint{0.710424in}{0.396138in}}{\pgfqpoint{0.706518in}{0.392231in}}%
\pgfpathcurveto{\pgfqpoint{0.702611in}{0.388325in}}{\pgfqpoint{0.700416in}{0.383025in}}{\pgfqpoint{0.700416in}{0.377500in}}%
\pgfpathcurveto{\pgfqpoint{0.700416in}{0.371975in}}{\pgfqpoint{0.702611in}{0.366675in}}{\pgfqpoint{0.706518in}{0.362769in}}%
\pgfpathcurveto{\pgfqpoint{0.710424in}{0.358862in}}{\pgfqpoint{0.715724in}{0.356667in}}{\pgfqpoint{0.721249in}{0.356667in}}%
\pgfpathclose%
\pgfusepath{stroke,fill}%
\end{pgfscope}%
\begin{pgfscope}%
\pgfpathrectangle{\pgfqpoint{0.562500in}{0.275000in}}{\pgfqpoint{3.487500in}{1.925000in}}%
\pgfusepath{clip}%
\pgfsetbuttcap%
\pgfsetroundjoin%
\definecolor{currentfill}{rgb}{0.000000,0.000000,0.000000}%
\pgfsetfillcolor{currentfill}%
\pgfsetlinewidth{1.003750pt}%
\definecolor{currentstroke}{rgb}{0.000000,0.000000,0.000000}%
\pgfsetstrokecolor{currentstroke}%
\pgfsetdash{}{0pt}%
\pgfpathmoveto{\pgfqpoint{0.721249in}{0.356667in}}%
\pgfpathcurveto{\pgfqpoint{0.726774in}{0.356667in}}{\pgfqpoint{0.732073in}{0.358862in}}{\pgfqpoint{0.735980in}{0.362769in}}%
\pgfpathcurveto{\pgfqpoint{0.739887in}{0.366675in}}{\pgfqpoint{0.742082in}{0.371975in}}{\pgfqpoint{0.742082in}{0.377500in}}%
\pgfpathcurveto{\pgfqpoint{0.742082in}{0.383025in}}{\pgfqpoint{0.739887in}{0.388325in}}{\pgfqpoint{0.735980in}{0.392231in}}%
\pgfpathcurveto{\pgfqpoint{0.732073in}{0.396138in}}{\pgfqpoint{0.726774in}{0.398333in}}{\pgfqpoint{0.721249in}{0.398333in}}%
\pgfpathcurveto{\pgfqpoint{0.715724in}{0.398333in}}{\pgfqpoint{0.710424in}{0.396138in}}{\pgfqpoint{0.706518in}{0.392231in}}%
\pgfpathcurveto{\pgfqpoint{0.702611in}{0.388325in}}{\pgfqpoint{0.700416in}{0.383025in}}{\pgfqpoint{0.700416in}{0.377500in}}%
\pgfpathcurveto{\pgfqpoint{0.700416in}{0.371975in}}{\pgfqpoint{0.702611in}{0.366675in}}{\pgfqpoint{0.706518in}{0.362769in}}%
\pgfpathcurveto{\pgfqpoint{0.710424in}{0.358862in}}{\pgfqpoint{0.715724in}{0.356667in}}{\pgfqpoint{0.721249in}{0.356667in}}%
\pgfpathclose%
\pgfusepath{stroke,fill}%
\end{pgfscope}%
\begin{pgfscope}%
\pgfpathrectangle{\pgfqpoint{0.562500in}{0.275000in}}{\pgfqpoint{3.487500in}{1.925000in}}%
\pgfusepath{clip}%
\pgfsetbuttcap%
\pgfsetroundjoin%
\definecolor{currentfill}{rgb}{0.000000,0.000000,0.000000}%
\pgfsetfillcolor{currentfill}%
\pgfsetlinewidth{1.003750pt}%
\definecolor{currentstroke}{rgb}{0.000000,0.000000,0.000000}%
\pgfsetstrokecolor{currentstroke}%
\pgfsetdash{}{0pt}%
\pgfpathmoveto{\pgfqpoint{0.721249in}{0.356667in}}%
\pgfpathcurveto{\pgfqpoint{0.726774in}{0.356667in}}{\pgfqpoint{0.732073in}{0.358862in}}{\pgfqpoint{0.735980in}{0.362769in}}%
\pgfpathcurveto{\pgfqpoint{0.739887in}{0.366675in}}{\pgfqpoint{0.742082in}{0.371975in}}{\pgfqpoint{0.742082in}{0.377500in}}%
\pgfpathcurveto{\pgfqpoint{0.742082in}{0.383025in}}{\pgfqpoint{0.739887in}{0.388325in}}{\pgfqpoint{0.735980in}{0.392231in}}%
\pgfpathcurveto{\pgfqpoint{0.732073in}{0.396138in}}{\pgfqpoint{0.726774in}{0.398333in}}{\pgfqpoint{0.721249in}{0.398333in}}%
\pgfpathcurveto{\pgfqpoint{0.715724in}{0.398333in}}{\pgfqpoint{0.710424in}{0.396138in}}{\pgfqpoint{0.706518in}{0.392231in}}%
\pgfpathcurveto{\pgfqpoint{0.702611in}{0.388325in}}{\pgfqpoint{0.700416in}{0.383025in}}{\pgfqpoint{0.700416in}{0.377500in}}%
\pgfpathcurveto{\pgfqpoint{0.700416in}{0.371975in}}{\pgfqpoint{0.702611in}{0.366675in}}{\pgfqpoint{0.706518in}{0.362769in}}%
\pgfpathcurveto{\pgfqpoint{0.710424in}{0.358862in}}{\pgfqpoint{0.715724in}{0.356667in}}{\pgfqpoint{0.721249in}{0.356667in}}%
\pgfpathclose%
\pgfusepath{stroke,fill}%
\end{pgfscope}%
\begin{pgfscope}%
\pgfpathrectangle{\pgfqpoint{0.562500in}{0.275000in}}{\pgfqpoint{3.487500in}{1.925000in}}%
\pgfusepath{clip}%
\pgfsetbuttcap%
\pgfsetroundjoin%
\definecolor{currentfill}{rgb}{0.000000,0.000000,0.000000}%
\pgfsetfillcolor{currentfill}%
\pgfsetlinewidth{1.003750pt}%
\definecolor{currentstroke}{rgb}{0.000000,0.000000,0.000000}%
\pgfsetstrokecolor{currentstroke}%
\pgfsetdash{}{0pt}%
\pgfpathmoveto{\pgfqpoint{0.721249in}{0.356667in}}%
\pgfpathcurveto{\pgfqpoint{0.726774in}{0.356667in}}{\pgfqpoint{0.732073in}{0.358862in}}{\pgfqpoint{0.735980in}{0.362769in}}%
\pgfpathcurveto{\pgfqpoint{0.739887in}{0.366675in}}{\pgfqpoint{0.742082in}{0.371975in}}{\pgfqpoint{0.742082in}{0.377500in}}%
\pgfpathcurveto{\pgfqpoint{0.742082in}{0.383025in}}{\pgfqpoint{0.739887in}{0.388325in}}{\pgfqpoint{0.735980in}{0.392231in}}%
\pgfpathcurveto{\pgfqpoint{0.732073in}{0.396138in}}{\pgfqpoint{0.726774in}{0.398333in}}{\pgfqpoint{0.721249in}{0.398333in}}%
\pgfpathcurveto{\pgfqpoint{0.715724in}{0.398333in}}{\pgfqpoint{0.710424in}{0.396138in}}{\pgfqpoint{0.706518in}{0.392231in}}%
\pgfpathcurveto{\pgfqpoint{0.702611in}{0.388325in}}{\pgfqpoint{0.700416in}{0.383025in}}{\pgfqpoint{0.700416in}{0.377500in}}%
\pgfpathcurveto{\pgfqpoint{0.700416in}{0.371975in}}{\pgfqpoint{0.702611in}{0.366675in}}{\pgfqpoint{0.706518in}{0.362769in}}%
\pgfpathcurveto{\pgfqpoint{0.710424in}{0.358862in}}{\pgfqpoint{0.715724in}{0.356667in}}{\pgfqpoint{0.721249in}{0.356667in}}%
\pgfpathclose%
\pgfusepath{stroke,fill}%
\end{pgfscope}%
\begin{pgfscope}%
\pgfpathrectangle{\pgfqpoint{0.562500in}{0.275000in}}{\pgfqpoint{3.487500in}{1.925000in}}%
\pgfusepath{clip}%
\pgfsetbuttcap%
\pgfsetroundjoin%
\definecolor{currentfill}{rgb}{0.000000,0.000000,0.000000}%
\pgfsetfillcolor{currentfill}%
\pgfsetlinewidth{1.003750pt}%
\definecolor{currentstroke}{rgb}{0.000000,0.000000,0.000000}%
\pgfsetstrokecolor{currentstroke}%
\pgfsetdash{}{0pt}%
\pgfpathmoveto{\pgfqpoint{0.721249in}{0.356667in}}%
\pgfpathcurveto{\pgfqpoint{0.726774in}{0.356667in}}{\pgfqpoint{0.732073in}{0.358862in}}{\pgfqpoint{0.735980in}{0.362769in}}%
\pgfpathcurveto{\pgfqpoint{0.739887in}{0.366675in}}{\pgfqpoint{0.742082in}{0.371975in}}{\pgfqpoint{0.742082in}{0.377500in}}%
\pgfpathcurveto{\pgfqpoint{0.742082in}{0.383025in}}{\pgfqpoint{0.739887in}{0.388325in}}{\pgfqpoint{0.735980in}{0.392231in}}%
\pgfpathcurveto{\pgfqpoint{0.732073in}{0.396138in}}{\pgfqpoint{0.726774in}{0.398333in}}{\pgfqpoint{0.721249in}{0.398333in}}%
\pgfpathcurveto{\pgfqpoint{0.715724in}{0.398333in}}{\pgfqpoint{0.710424in}{0.396138in}}{\pgfqpoint{0.706518in}{0.392231in}}%
\pgfpathcurveto{\pgfqpoint{0.702611in}{0.388325in}}{\pgfqpoint{0.700416in}{0.383025in}}{\pgfqpoint{0.700416in}{0.377500in}}%
\pgfpathcurveto{\pgfqpoint{0.700416in}{0.371975in}}{\pgfqpoint{0.702611in}{0.366675in}}{\pgfqpoint{0.706518in}{0.362769in}}%
\pgfpathcurveto{\pgfqpoint{0.710424in}{0.358862in}}{\pgfqpoint{0.715724in}{0.356667in}}{\pgfqpoint{0.721249in}{0.356667in}}%
\pgfpathclose%
\pgfusepath{stroke,fill}%
\end{pgfscope}%
\begin{pgfscope}%
\pgfpathrectangle{\pgfqpoint{0.562500in}{0.275000in}}{\pgfqpoint{3.487500in}{1.925000in}}%
\pgfusepath{clip}%
\pgfsetbuttcap%
\pgfsetroundjoin%
\definecolor{currentfill}{rgb}{0.000000,0.000000,0.000000}%
\pgfsetfillcolor{currentfill}%
\pgfsetlinewidth{1.003750pt}%
\definecolor{currentstroke}{rgb}{0.000000,0.000000,0.000000}%
\pgfsetstrokecolor{currentstroke}%
\pgfsetdash{}{0pt}%
\pgfpathmoveto{\pgfqpoint{0.721249in}{2.076667in}}%
\pgfpathcurveto{\pgfqpoint{0.726774in}{2.076667in}}{\pgfqpoint{0.732073in}{2.078862in}}{\pgfqpoint{0.735980in}{2.082769in}}%
\pgfpathcurveto{\pgfqpoint{0.739887in}{2.086675in}}{\pgfqpoint{0.742082in}{2.091975in}}{\pgfqpoint{0.742082in}{2.097500in}}%
\pgfpathcurveto{\pgfqpoint{0.742082in}{2.103025in}}{\pgfqpoint{0.739887in}{2.108325in}}{\pgfqpoint{0.735980in}{2.112231in}}%
\pgfpathcurveto{\pgfqpoint{0.732073in}{2.116138in}}{\pgfqpoint{0.726774in}{2.118333in}}{\pgfqpoint{0.721249in}{2.118333in}}%
\pgfpathcurveto{\pgfqpoint{0.715724in}{2.118333in}}{\pgfqpoint{0.710424in}{2.116138in}}{\pgfqpoint{0.706518in}{2.112231in}}%
\pgfpathcurveto{\pgfqpoint{0.702611in}{2.108325in}}{\pgfqpoint{0.700416in}{2.103025in}}{\pgfqpoint{0.700416in}{2.097500in}}%
\pgfpathcurveto{\pgfqpoint{0.700416in}{2.091975in}}{\pgfqpoint{0.702611in}{2.086675in}}{\pgfqpoint{0.706518in}{2.082769in}}%
\pgfpathcurveto{\pgfqpoint{0.710424in}{2.078862in}}{\pgfqpoint{0.715724in}{2.076667in}}{\pgfqpoint{0.721249in}{2.076667in}}%
\pgfpathclose%
\pgfusepath{stroke,fill}%
\end{pgfscope}%
\begin{pgfscope}%
\pgfpathrectangle{\pgfqpoint{0.562500in}{0.275000in}}{\pgfqpoint{3.487500in}{1.925000in}}%
\pgfusepath{clip}%
\pgfsetbuttcap%
\pgfsetroundjoin%
\definecolor{currentfill}{rgb}{0.000000,0.000000,0.000000}%
\pgfsetfillcolor{currentfill}%
\pgfsetlinewidth{1.003750pt}%
\definecolor{currentstroke}{rgb}{0.000000,0.000000,0.000000}%
\pgfsetstrokecolor{currentstroke}%
\pgfsetdash{}{0pt}%
\pgfpathmoveto{\pgfqpoint{0.721249in}{0.356667in}}%
\pgfpathcurveto{\pgfqpoint{0.726774in}{0.356667in}}{\pgfqpoint{0.732073in}{0.358862in}}{\pgfqpoint{0.735980in}{0.362769in}}%
\pgfpathcurveto{\pgfqpoint{0.739887in}{0.366675in}}{\pgfqpoint{0.742082in}{0.371975in}}{\pgfqpoint{0.742082in}{0.377500in}}%
\pgfpathcurveto{\pgfqpoint{0.742082in}{0.383025in}}{\pgfqpoint{0.739887in}{0.388325in}}{\pgfqpoint{0.735980in}{0.392231in}}%
\pgfpathcurveto{\pgfqpoint{0.732073in}{0.396138in}}{\pgfqpoint{0.726774in}{0.398333in}}{\pgfqpoint{0.721249in}{0.398333in}}%
\pgfpathcurveto{\pgfqpoint{0.715724in}{0.398333in}}{\pgfqpoint{0.710424in}{0.396138in}}{\pgfqpoint{0.706518in}{0.392231in}}%
\pgfpathcurveto{\pgfqpoint{0.702611in}{0.388325in}}{\pgfqpoint{0.700416in}{0.383025in}}{\pgfqpoint{0.700416in}{0.377500in}}%
\pgfpathcurveto{\pgfqpoint{0.700416in}{0.371975in}}{\pgfqpoint{0.702611in}{0.366675in}}{\pgfqpoint{0.706518in}{0.362769in}}%
\pgfpathcurveto{\pgfqpoint{0.710424in}{0.358862in}}{\pgfqpoint{0.715724in}{0.356667in}}{\pgfqpoint{0.721249in}{0.356667in}}%
\pgfpathclose%
\pgfusepath{stroke,fill}%
\end{pgfscope}%
\begin{pgfscope}%
\pgfpathrectangle{\pgfqpoint{0.562500in}{0.275000in}}{\pgfqpoint{3.487500in}{1.925000in}}%
\pgfusepath{clip}%
\pgfsetbuttcap%
\pgfsetroundjoin%
\definecolor{currentfill}{rgb}{0.000000,0.000000,0.000000}%
\pgfsetfillcolor{currentfill}%
\pgfsetlinewidth{1.003750pt}%
\definecolor{currentstroke}{rgb}{0.000000,0.000000,0.000000}%
\pgfsetstrokecolor{currentstroke}%
\pgfsetdash{}{0pt}%
\pgfpathmoveto{\pgfqpoint{0.721249in}{0.356667in}}%
\pgfpathcurveto{\pgfqpoint{0.726774in}{0.356667in}}{\pgfqpoint{0.732073in}{0.358862in}}{\pgfqpoint{0.735980in}{0.362769in}}%
\pgfpathcurveto{\pgfqpoint{0.739887in}{0.366675in}}{\pgfqpoint{0.742082in}{0.371975in}}{\pgfqpoint{0.742082in}{0.377500in}}%
\pgfpathcurveto{\pgfqpoint{0.742082in}{0.383025in}}{\pgfqpoint{0.739887in}{0.388325in}}{\pgfqpoint{0.735980in}{0.392231in}}%
\pgfpathcurveto{\pgfqpoint{0.732073in}{0.396138in}}{\pgfqpoint{0.726774in}{0.398333in}}{\pgfqpoint{0.721249in}{0.398333in}}%
\pgfpathcurveto{\pgfqpoint{0.715724in}{0.398333in}}{\pgfqpoint{0.710424in}{0.396138in}}{\pgfqpoint{0.706518in}{0.392231in}}%
\pgfpathcurveto{\pgfqpoint{0.702611in}{0.388325in}}{\pgfqpoint{0.700416in}{0.383025in}}{\pgfqpoint{0.700416in}{0.377500in}}%
\pgfpathcurveto{\pgfqpoint{0.700416in}{0.371975in}}{\pgfqpoint{0.702611in}{0.366675in}}{\pgfqpoint{0.706518in}{0.362769in}}%
\pgfpathcurveto{\pgfqpoint{0.710424in}{0.358862in}}{\pgfqpoint{0.715724in}{0.356667in}}{\pgfqpoint{0.721249in}{0.356667in}}%
\pgfpathclose%
\pgfusepath{stroke,fill}%
\end{pgfscope}%
\begin{pgfscope}%
\pgfpathrectangle{\pgfqpoint{0.562500in}{0.275000in}}{\pgfqpoint{3.487500in}{1.925000in}}%
\pgfusepath{clip}%
\pgfsetbuttcap%
\pgfsetroundjoin%
\definecolor{currentfill}{rgb}{0.000000,0.000000,0.000000}%
\pgfsetfillcolor{currentfill}%
\pgfsetlinewidth{1.003750pt}%
\definecolor{currentstroke}{rgb}{0.000000,0.000000,0.000000}%
\pgfsetstrokecolor{currentstroke}%
\pgfsetdash{}{0pt}%
\pgfpathmoveto{\pgfqpoint{0.721249in}{0.356667in}}%
\pgfpathcurveto{\pgfqpoint{0.726774in}{0.356667in}}{\pgfqpoint{0.732073in}{0.358862in}}{\pgfqpoint{0.735980in}{0.362769in}}%
\pgfpathcurveto{\pgfqpoint{0.739887in}{0.366675in}}{\pgfqpoint{0.742082in}{0.371975in}}{\pgfqpoint{0.742082in}{0.377500in}}%
\pgfpathcurveto{\pgfqpoint{0.742082in}{0.383025in}}{\pgfqpoint{0.739887in}{0.388325in}}{\pgfqpoint{0.735980in}{0.392231in}}%
\pgfpathcurveto{\pgfqpoint{0.732073in}{0.396138in}}{\pgfqpoint{0.726774in}{0.398333in}}{\pgfqpoint{0.721249in}{0.398333in}}%
\pgfpathcurveto{\pgfqpoint{0.715724in}{0.398333in}}{\pgfqpoint{0.710424in}{0.396138in}}{\pgfqpoint{0.706518in}{0.392231in}}%
\pgfpathcurveto{\pgfqpoint{0.702611in}{0.388325in}}{\pgfqpoint{0.700416in}{0.383025in}}{\pgfqpoint{0.700416in}{0.377500in}}%
\pgfpathcurveto{\pgfqpoint{0.700416in}{0.371975in}}{\pgfqpoint{0.702611in}{0.366675in}}{\pgfqpoint{0.706518in}{0.362769in}}%
\pgfpathcurveto{\pgfqpoint{0.710424in}{0.358862in}}{\pgfqpoint{0.715724in}{0.356667in}}{\pgfqpoint{0.721249in}{0.356667in}}%
\pgfpathclose%
\pgfusepath{stroke,fill}%
\end{pgfscope}%
\begin{pgfscope}%
\pgfpathrectangle{\pgfqpoint{0.562500in}{0.275000in}}{\pgfqpoint{3.487500in}{1.925000in}}%
\pgfusepath{clip}%
\pgfsetbuttcap%
\pgfsetroundjoin%
\definecolor{currentfill}{rgb}{0.000000,0.000000,0.000000}%
\pgfsetfillcolor{currentfill}%
\pgfsetlinewidth{1.003750pt}%
\definecolor{currentstroke}{rgb}{0.000000,0.000000,0.000000}%
\pgfsetstrokecolor{currentstroke}%
\pgfsetdash{}{0pt}%
\pgfpathmoveto{\pgfqpoint{0.721249in}{0.356667in}}%
\pgfpathcurveto{\pgfqpoint{0.726774in}{0.356667in}}{\pgfqpoint{0.732073in}{0.358862in}}{\pgfqpoint{0.735980in}{0.362769in}}%
\pgfpathcurveto{\pgfqpoint{0.739887in}{0.366675in}}{\pgfqpoint{0.742082in}{0.371975in}}{\pgfqpoint{0.742082in}{0.377500in}}%
\pgfpathcurveto{\pgfqpoint{0.742082in}{0.383025in}}{\pgfqpoint{0.739887in}{0.388325in}}{\pgfqpoint{0.735980in}{0.392231in}}%
\pgfpathcurveto{\pgfqpoint{0.732073in}{0.396138in}}{\pgfqpoint{0.726774in}{0.398333in}}{\pgfqpoint{0.721249in}{0.398333in}}%
\pgfpathcurveto{\pgfqpoint{0.715724in}{0.398333in}}{\pgfqpoint{0.710424in}{0.396138in}}{\pgfqpoint{0.706518in}{0.392231in}}%
\pgfpathcurveto{\pgfqpoint{0.702611in}{0.388325in}}{\pgfqpoint{0.700416in}{0.383025in}}{\pgfqpoint{0.700416in}{0.377500in}}%
\pgfpathcurveto{\pgfqpoint{0.700416in}{0.371975in}}{\pgfqpoint{0.702611in}{0.366675in}}{\pgfqpoint{0.706518in}{0.362769in}}%
\pgfpathcurveto{\pgfqpoint{0.710424in}{0.358862in}}{\pgfqpoint{0.715724in}{0.356667in}}{\pgfqpoint{0.721249in}{0.356667in}}%
\pgfpathclose%
\pgfusepath{stroke,fill}%
\end{pgfscope}%
\begin{pgfscope}%
\pgfpathrectangle{\pgfqpoint{0.562500in}{0.275000in}}{\pgfqpoint{3.487500in}{1.925000in}}%
\pgfusepath{clip}%
\pgfsetbuttcap%
\pgfsetroundjoin%
\definecolor{currentfill}{rgb}{0.000000,0.000000,0.000000}%
\pgfsetfillcolor{currentfill}%
\pgfsetlinewidth{1.003750pt}%
\definecolor{currentstroke}{rgb}{0.000000,0.000000,0.000000}%
\pgfsetstrokecolor{currentstroke}%
\pgfsetdash{}{0pt}%
\pgfpathmoveto{\pgfqpoint{0.721249in}{0.356667in}}%
\pgfpathcurveto{\pgfqpoint{0.726774in}{0.356667in}}{\pgfqpoint{0.732073in}{0.358862in}}{\pgfqpoint{0.735980in}{0.362769in}}%
\pgfpathcurveto{\pgfqpoint{0.739887in}{0.366675in}}{\pgfqpoint{0.742082in}{0.371975in}}{\pgfqpoint{0.742082in}{0.377500in}}%
\pgfpathcurveto{\pgfqpoint{0.742082in}{0.383025in}}{\pgfqpoint{0.739887in}{0.388325in}}{\pgfqpoint{0.735980in}{0.392231in}}%
\pgfpathcurveto{\pgfqpoint{0.732073in}{0.396138in}}{\pgfqpoint{0.726774in}{0.398333in}}{\pgfqpoint{0.721249in}{0.398333in}}%
\pgfpathcurveto{\pgfqpoint{0.715724in}{0.398333in}}{\pgfqpoint{0.710424in}{0.396138in}}{\pgfqpoint{0.706518in}{0.392231in}}%
\pgfpathcurveto{\pgfqpoint{0.702611in}{0.388325in}}{\pgfqpoint{0.700416in}{0.383025in}}{\pgfqpoint{0.700416in}{0.377500in}}%
\pgfpathcurveto{\pgfqpoint{0.700416in}{0.371975in}}{\pgfqpoint{0.702611in}{0.366675in}}{\pgfqpoint{0.706518in}{0.362769in}}%
\pgfpathcurveto{\pgfqpoint{0.710424in}{0.358862in}}{\pgfqpoint{0.715724in}{0.356667in}}{\pgfqpoint{0.721249in}{0.356667in}}%
\pgfpathclose%
\pgfusepath{stroke,fill}%
\end{pgfscope}%
\begin{pgfscope}%
\pgfpathrectangle{\pgfqpoint{0.562500in}{0.275000in}}{\pgfqpoint{3.487500in}{1.925000in}}%
\pgfusepath{clip}%
\pgfsetbuttcap%
\pgfsetroundjoin%
\definecolor{currentfill}{rgb}{0.000000,0.000000,0.000000}%
\pgfsetfillcolor{currentfill}%
\pgfsetlinewidth{1.003750pt}%
\definecolor{currentstroke}{rgb}{0.000000,0.000000,0.000000}%
\pgfsetstrokecolor{currentstroke}%
\pgfsetdash{}{0pt}%
\pgfpathmoveto{\pgfqpoint{0.721249in}{0.356667in}}%
\pgfpathcurveto{\pgfqpoint{0.726774in}{0.356667in}}{\pgfqpoint{0.732073in}{0.358862in}}{\pgfqpoint{0.735980in}{0.362769in}}%
\pgfpathcurveto{\pgfqpoint{0.739887in}{0.366675in}}{\pgfqpoint{0.742082in}{0.371975in}}{\pgfqpoint{0.742082in}{0.377500in}}%
\pgfpathcurveto{\pgfqpoint{0.742082in}{0.383025in}}{\pgfqpoint{0.739887in}{0.388325in}}{\pgfqpoint{0.735980in}{0.392231in}}%
\pgfpathcurveto{\pgfqpoint{0.732073in}{0.396138in}}{\pgfqpoint{0.726774in}{0.398333in}}{\pgfqpoint{0.721249in}{0.398333in}}%
\pgfpathcurveto{\pgfqpoint{0.715724in}{0.398333in}}{\pgfqpoint{0.710424in}{0.396138in}}{\pgfqpoint{0.706518in}{0.392231in}}%
\pgfpathcurveto{\pgfqpoint{0.702611in}{0.388325in}}{\pgfqpoint{0.700416in}{0.383025in}}{\pgfqpoint{0.700416in}{0.377500in}}%
\pgfpathcurveto{\pgfqpoint{0.700416in}{0.371975in}}{\pgfqpoint{0.702611in}{0.366675in}}{\pgfqpoint{0.706518in}{0.362769in}}%
\pgfpathcurveto{\pgfqpoint{0.710424in}{0.358862in}}{\pgfqpoint{0.715724in}{0.356667in}}{\pgfqpoint{0.721249in}{0.356667in}}%
\pgfpathclose%
\pgfusepath{stroke,fill}%
\end{pgfscope}%
\begin{pgfscope}%
\pgfpathrectangle{\pgfqpoint{0.562500in}{0.275000in}}{\pgfqpoint{3.487500in}{1.925000in}}%
\pgfusepath{clip}%
\pgfsetbuttcap%
\pgfsetroundjoin%
\definecolor{currentfill}{rgb}{0.000000,0.000000,0.000000}%
\pgfsetfillcolor{currentfill}%
\pgfsetlinewidth{1.003750pt}%
\definecolor{currentstroke}{rgb}{0.000000,0.000000,0.000000}%
\pgfsetstrokecolor{currentstroke}%
\pgfsetdash{}{0pt}%
\pgfpathmoveto{\pgfqpoint{0.721249in}{0.356667in}}%
\pgfpathcurveto{\pgfqpoint{0.726774in}{0.356667in}}{\pgfqpoint{0.732073in}{0.358862in}}{\pgfqpoint{0.735980in}{0.362769in}}%
\pgfpathcurveto{\pgfqpoint{0.739887in}{0.366675in}}{\pgfqpoint{0.742082in}{0.371975in}}{\pgfqpoint{0.742082in}{0.377500in}}%
\pgfpathcurveto{\pgfqpoint{0.742082in}{0.383025in}}{\pgfqpoint{0.739887in}{0.388325in}}{\pgfqpoint{0.735980in}{0.392231in}}%
\pgfpathcurveto{\pgfqpoint{0.732073in}{0.396138in}}{\pgfqpoint{0.726774in}{0.398333in}}{\pgfqpoint{0.721249in}{0.398333in}}%
\pgfpathcurveto{\pgfqpoint{0.715724in}{0.398333in}}{\pgfqpoint{0.710424in}{0.396138in}}{\pgfqpoint{0.706518in}{0.392231in}}%
\pgfpathcurveto{\pgfqpoint{0.702611in}{0.388325in}}{\pgfqpoint{0.700416in}{0.383025in}}{\pgfqpoint{0.700416in}{0.377500in}}%
\pgfpathcurveto{\pgfqpoint{0.700416in}{0.371975in}}{\pgfqpoint{0.702611in}{0.366675in}}{\pgfqpoint{0.706518in}{0.362769in}}%
\pgfpathcurveto{\pgfqpoint{0.710424in}{0.358862in}}{\pgfqpoint{0.715724in}{0.356667in}}{\pgfqpoint{0.721249in}{0.356667in}}%
\pgfpathclose%
\pgfusepath{stroke,fill}%
\end{pgfscope}%
\begin{pgfscope}%
\pgfpathrectangle{\pgfqpoint{0.562500in}{0.275000in}}{\pgfqpoint{3.487500in}{1.925000in}}%
\pgfusepath{clip}%
\pgfsetbuttcap%
\pgfsetroundjoin%
\definecolor{currentfill}{rgb}{0.000000,0.000000,0.000000}%
\pgfsetfillcolor{currentfill}%
\pgfsetlinewidth{1.003750pt}%
\definecolor{currentstroke}{rgb}{0.000000,0.000000,0.000000}%
\pgfsetstrokecolor{currentstroke}%
\pgfsetdash{}{0pt}%
\pgfpathmoveto{\pgfqpoint{0.721249in}{0.356667in}}%
\pgfpathcurveto{\pgfqpoint{0.726774in}{0.356667in}}{\pgfqpoint{0.732073in}{0.358862in}}{\pgfqpoint{0.735980in}{0.362769in}}%
\pgfpathcurveto{\pgfqpoint{0.739887in}{0.366675in}}{\pgfqpoint{0.742082in}{0.371975in}}{\pgfqpoint{0.742082in}{0.377500in}}%
\pgfpathcurveto{\pgfqpoint{0.742082in}{0.383025in}}{\pgfqpoint{0.739887in}{0.388325in}}{\pgfqpoint{0.735980in}{0.392231in}}%
\pgfpathcurveto{\pgfqpoint{0.732073in}{0.396138in}}{\pgfqpoint{0.726774in}{0.398333in}}{\pgfqpoint{0.721249in}{0.398333in}}%
\pgfpathcurveto{\pgfqpoint{0.715724in}{0.398333in}}{\pgfqpoint{0.710424in}{0.396138in}}{\pgfqpoint{0.706518in}{0.392231in}}%
\pgfpathcurveto{\pgfqpoint{0.702611in}{0.388325in}}{\pgfqpoint{0.700416in}{0.383025in}}{\pgfqpoint{0.700416in}{0.377500in}}%
\pgfpathcurveto{\pgfqpoint{0.700416in}{0.371975in}}{\pgfqpoint{0.702611in}{0.366675in}}{\pgfqpoint{0.706518in}{0.362769in}}%
\pgfpathcurveto{\pgfqpoint{0.710424in}{0.358862in}}{\pgfqpoint{0.715724in}{0.356667in}}{\pgfqpoint{0.721249in}{0.356667in}}%
\pgfpathclose%
\pgfusepath{stroke,fill}%
\end{pgfscope}%
\begin{pgfscope}%
\pgfpathrectangle{\pgfqpoint{0.562500in}{0.275000in}}{\pgfqpoint{3.487500in}{1.925000in}}%
\pgfusepath{clip}%
\pgfsetbuttcap%
\pgfsetroundjoin%
\definecolor{currentfill}{rgb}{0.000000,0.000000,0.000000}%
\pgfsetfillcolor{currentfill}%
\pgfsetlinewidth{1.003750pt}%
\definecolor{currentstroke}{rgb}{0.000000,0.000000,0.000000}%
\pgfsetstrokecolor{currentstroke}%
\pgfsetdash{}{0pt}%
\pgfpathmoveto{\pgfqpoint{0.721249in}{0.356667in}}%
\pgfpathcurveto{\pgfqpoint{0.726774in}{0.356667in}}{\pgfqpoint{0.732073in}{0.358862in}}{\pgfqpoint{0.735980in}{0.362769in}}%
\pgfpathcurveto{\pgfqpoint{0.739887in}{0.366675in}}{\pgfqpoint{0.742082in}{0.371975in}}{\pgfqpoint{0.742082in}{0.377500in}}%
\pgfpathcurveto{\pgfqpoint{0.742082in}{0.383025in}}{\pgfqpoint{0.739887in}{0.388325in}}{\pgfqpoint{0.735980in}{0.392231in}}%
\pgfpathcurveto{\pgfqpoint{0.732073in}{0.396138in}}{\pgfqpoint{0.726774in}{0.398333in}}{\pgfqpoint{0.721249in}{0.398333in}}%
\pgfpathcurveto{\pgfqpoint{0.715724in}{0.398333in}}{\pgfqpoint{0.710424in}{0.396138in}}{\pgfqpoint{0.706518in}{0.392231in}}%
\pgfpathcurveto{\pgfqpoint{0.702611in}{0.388325in}}{\pgfqpoint{0.700416in}{0.383025in}}{\pgfqpoint{0.700416in}{0.377500in}}%
\pgfpathcurveto{\pgfqpoint{0.700416in}{0.371975in}}{\pgfqpoint{0.702611in}{0.366675in}}{\pgfqpoint{0.706518in}{0.362769in}}%
\pgfpathcurveto{\pgfqpoint{0.710424in}{0.358862in}}{\pgfqpoint{0.715724in}{0.356667in}}{\pgfqpoint{0.721249in}{0.356667in}}%
\pgfpathclose%
\pgfusepath{stroke,fill}%
\end{pgfscope}%
\begin{pgfscope}%
\pgfpathrectangle{\pgfqpoint{0.562500in}{0.275000in}}{\pgfqpoint{3.487500in}{1.925000in}}%
\pgfusepath{clip}%
\pgfsetbuttcap%
\pgfsetroundjoin%
\definecolor{currentfill}{rgb}{0.000000,0.000000,0.000000}%
\pgfsetfillcolor{currentfill}%
\pgfsetlinewidth{1.003750pt}%
\definecolor{currentstroke}{rgb}{0.000000,0.000000,0.000000}%
\pgfsetstrokecolor{currentstroke}%
\pgfsetdash{}{0pt}%
\pgfpathmoveto{\pgfqpoint{0.721249in}{0.356667in}}%
\pgfpathcurveto{\pgfqpoint{0.726774in}{0.356667in}}{\pgfqpoint{0.732073in}{0.358862in}}{\pgfqpoint{0.735980in}{0.362769in}}%
\pgfpathcurveto{\pgfqpoint{0.739887in}{0.366675in}}{\pgfqpoint{0.742082in}{0.371975in}}{\pgfqpoint{0.742082in}{0.377500in}}%
\pgfpathcurveto{\pgfqpoint{0.742082in}{0.383025in}}{\pgfqpoint{0.739887in}{0.388325in}}{\pgfqpoint{0.735980in}{0.392231in}}%
\pgfpathcurveto{\pgfqpoint{0.732073in}{0.396138in}}{\pgfqpoint{0.726774in}{0.398333in}}{\pgfqpoint{0.721249in}{0.398333in}}%
\pgfpathcurveto{\pgfqpoint{0.715724in}{0.398333in}}{\pgfqpoint{0.710424in}{0.396138in}}{\pgfqpoint{0.706518in}{0.392231in}}%
\pgfpathcurveto{\pgfqpoint{0.702611in}{0.388325in}}{\pgfqpoint{0.700416in}{0.383025in}}{\pgfqpoint{0.700416in}{0.377500in}}%
\pgfpathcurveto{\pgfqpoint{0.700416in}{0.371975in}}{\pgfqpoint{0.702611in}{0.366675in}}{\pgfqpoint{0.706518in}{0.362769in}}%
\pgfpathcurveto{\pgfqpoint{0.710424in}{0.358862in}}{\pgfqpoint{0.715724in}{0.356667in}}{\pgfqpoint{0.721249in}{0.356667in}}%
\pgfpathclose%
\pgfusepath{stroke,fill}%
\end{pgfscope}%
\begin{pgfscope}%
\pgfpathrectangle{\pgfqpoint{0.562500in}{0.275000in}}{\pgfqpoint{3.487500in}{1.925000in}}%
\pgfusepath{clip}%
\pgfsetbuttcap%
\pgfsetroundjoin%
\definecolor{currentfill}{rgb}{0.000000,0.000000,0.000000}%
\pgfsetfillcolor{currentfill}%
\pgfsetlinewidth{1.003750pt}%
\definecolor{currentstroke}{rgb}{0.000000,0.000000,0.000000}%
\pgfsetstrokecolor{currentstroke}%
\pgfsetdash{}{0pt}%
\pgfpathmoveto{\pgfqpoint{0.721249in}{0.356667in}}%
\pgfpathcurveto{\pgfqpoint{0.726774in}{0.356667in}}{\pgfqpoint{0.732073in}{0.358862in}}{\pgfqpoint{0.735980in}{0.362769in}}%
\pgfpathcurveto{\pgfqpoint{0.739887in}{0.366675in}}{\pgfqpoint{0.742082in}{0.371975in}}{\pgfqpoint{0.742082in}{0.377500in}}%
\pgfpathcurveto{\pgfqpoint{0.742082in}{0.383025in}}{\pgfqpoint{0.739887in}{0.388325in}}{\pgfqpoint{0.735980in}{0.392231in}}%
\pgfpathcurveto{\pgfqpoint{0.732073in}{0.396138in}}{\pgfqpoint{0.726774in}{0.398333in}}{\pgfqpoint{0.721249in}{0.398333in}}%
\pgfpathcurveto{\pgfqpoint{0.715724in}{0.398333in}}{\pgfqpoint{0.710424in}{0.396138in}}{\pgfqpoint{0.706518in}{0.392231in}}%
\pgfpathcurveto{\pgfqpoint{0.702611in}{0.388325in}}{\pgfqpoint{0.700416in}{0.383025in}}{\pgfqpoint{0.700416in}{0.377500in}}%
\pgfpathcurveto{\pgfqpoint{0.700416in}{0.371975in}}{\pgfqpoint{0.702611in}{0.366675in}}{\pgfqpoint{0.706518in}{0.362769in}}%
\pgfpathcurveto{\pgfqpoint{0.710424in}{0.358862in}}{\pgfqpoint{0.715724in}{0.356667in}}{\pgfqpoint{0.721249in}{0.356667in}}%
\pgfpathclose%
\pgfusepath{stroke,fill}%
\end{pgfscope}%
\begin{pgfscope}%
\pgfpathrectangle{\pgfqpoint{0.562500in}{0.275000in}}{\pgfqpoint{3.487500in}{1.925000in}}%
\pgfusepath{clip}%
\pgfsetbuttcap%
\pgfsetroundjoin%
\definecolor{currentfill}{rgb}{0.000000,0.000000,0.000000}%
\pgfsetfillcolor{currentfill}%
\pgfsetlinewidth{1.003750pt}%
\definecolor{currentstroke}{rgb}{0.000000,0.000000,0.000000}%
\pgfsetstrokecolor{currentstroke}%
\pgfsetdash{}{0pt}%
\pgfpathmoveto{\pgfqpoint{0.721249in}{0.356667in}}%
\pgfpathcurveto{\pgfqpoint{0.726774in}{0.356667in}}{\pgfqpoint{0.732073in}{0.358862in}}{\pgfqpoint{0.735980in}{0.362769in}}%
\pgfpathcurveto{\pgfqpoint{0.739887in}{0.366675in}}{\pgfqpoint{0.742082in}{0.371975in}}{\pgfqpoint{0.742082in}{0.377500in}}%
\pgfpathcurveto{\pgfqpoint{0.742082in}{0.383025in}}{\pgfqpoint{0.739887in}{0.388325in}}{\pgfqpoint{0.735980in}{0.392231in}}%
\pgfpathcurveto{\pgfqpoint{0.732073in}{0.396138in}}{\pgfqpoint{0.726774in}{0.398333in}}{\pgfqpoint{0.721249in}{0.398333in}}%
\pgfpathcurveto{\pgfqpoint{0.715724in}{0.398333in}}{\pgfqpoint{0.710424in}{0.396138in}}{\pgfqpoint{0.706518in}{0.392231in}}%
\pgfpathcurveto{\pgfqpoint{0.702611in}{0.388325in}}{\pgfqpoint{0.700416in}{0.383025in}}{\pgfqpoint{0.700416in}{0.377500in}}%
\pgfpathcurveto{\pgfqpoint{0.700416in}{0.371975in}}{\pgfqpoint{0.702611in}{0.366675in}}{\pgfqpoint{0.706518in}{0.362769in}}%
\pgfpathcurveto{\pgfqpoint{0.710424in}{0.358862in}}{\pgfqpoint{0.715724in}{0.356667in}}{\pgfqpoint{0.721249in}{0.356667in}}%
\pgfpathclose%
\pgfusepath{stroke,fill}%
\end{pgfscope}%
\begin{pgfscope}%
\pgfpathrectangle{\pgfqpoint{0.562500in}{0.275000in}}{\pgfqpoint{3.487500in}{1.925000in}}%
\pgfusepath{clip}%
\pgfsetbuttcap%
\pgfsetroundjoin%
\definecolor{currentfill}{rgb}{0.000000,0.000000,0.000000}%
\pgfsetfillcolor{currentfill}%
\pgfsetlinewidth{1.003750pt}%
\definecolor{currentstroke}{rgb}{0.000000,0.000000,0.000000}%
\pgfsetstrokecolor{currentstroke}%
\pgfsetdash{}{0pt}%
\pgfpathmoveto{\pgfqpoint{0.721249in}{0.356667in}}%
\pgfpathcurveto{\pgfqpoint{0.726774in}{0.356667in}}{\pgfqpoint{0.732073in}{0.358862in}}{\pgfqpoint{0.735980in}{0.362769in}}%
\pgfpathcurveto{\pgfqpoint{0.739887in}{0.366675in}}{\pgfqpoint{0.742082in}{0.371975in}}{\pgfqpoint{0.742082in}{0.377500in}}%
\pgfpathcurveto{\pgfqpoint{0.742082in}{0.383025in}}{\pgfqpoint{0.739887in}{0.388325in}}{\pgfqpoint{0.735980in}{0.392231in}}%
\pgfpathcurveto{\pgfqpoint{0.732073in}{0.396138in}}{\pgfqpoint{0.726774in}{0.398333in}}{\pgfqpoint{0.721249in}{0.398333in}}%
\pgfpathcurveto{\pgfqpoint{0.715724in}{0.398333in}}{\pgfqpoint{0.710424in}{0.396138in}}{\pgfqpoint{0.706518in}{0.392231in}}%
\pgfpathcurveto{\pgfqpoint{0.702611in}{0.388325in}}{\pgfqpoint{0.700416in}{0.383025in}}{\pgfqpoint{0.700416in}{0.377500in}}%
\pgfpathcurveto{\pgfqpoint{0.700416in}{0.371975in}}{\pgfqpoint{0.702611in}{0.366675in}}{\pgfqpoint{0.706518in}{0.362769in}}%
\pgfpathcurveto{\pgfqpoint{0.710424in}{0.358862in}}{\pgfqpoint{0.715724in}{0.356667in}}{\pgfqpoint{0.721249in}{0.356667in}}%
\pgfpathclose%
\pgfusepath{stroke,fill}%
\end{pgfscope}%
\begin{pgfscope}%
\pgfpathrectangle{\pgfqpoint{0.562500in}{0.275000in}}{\pgfqpoint{3.487500in}{1.925000in}}%
\pgfusepath{clip}%
\pgfsetbuttcap%
\pgfsetroundjoin%
\definecolor{currentfill}{rgb}{0.000000,0.000000,0.000000}%
\pgfsetfillcolor{currentfill}%
\pgfsetlinewidth{1.003750pt}%
\definecolor{currentstroke}{rgb}{0.000000,0.000000,0.000000}%
\pgfsetstrokecolor{currentstroke}%
\pgfsetdash{}{0pt}%
\pgfpathmoveto{\pgfqpoint{0.721249in}{0.356667in}}%
\pgfpathcurveto{\pgfqpoint{0.726774in}{0.356667in}}{\pgfqpoint{0.732073in}{0.358862in}}{\pgfqpoint{0.735980in}{0.362769in}}%
\pgfpathcurveto{\pgfqpoint{0.739887in}{0.366675in}}{\pgfqpoint{0.742082in}{0.371975in}}{\pgfqpoint{0.742082in}{0.377500in}}%
\pgfpathcurveto{\pgfqpoint{0.742082in}{0.383025in}}{\pgfqpoint{0.739887in}{0.388325in}}{\pgfqpoint{0.735980in}{0.392231in}}%
\pgfpathcurveto{\pgfqpoint{0.732073in}{0.396138in}}{\pgfqpoint{0.726774in}{0.398333in}}{\pgfqpoint{0.721249in}{0.398333in}}%
\pgfpathcurveto{\pgfqpoint{0.715724in}{0.398333in}}{\pgfqpoint{0.710424in}{0.396138in}}{\pgfqpoint{0.706518in}{0.392231in}}%
\pgfpathcurveto{\pgfqpoint{0.702611in}{0.388325in}}{\pgfqpoint{0.700416in}{0.383025in}}{\pgfqpoint{0.700416in}{0.377500in}}%
\pgfpathcurveto{\pgfqpoint{0.700416in}{0.371975in}}{\pgfqpoint{0.702611in}{0.366675in}}{\pgfqpoint{0.706518in}{0.362769in}}%
\pgfpathcurveto{\pgfqpoint{0.710424in}{0.358862in}}{\pgfqpoint{0.715724in}{0.356667in}}{\pgfqpoint{0.721249in}{0.356667in}}%
\pgfpathclose%
\pgfusepath{stroke,fill}%
\end{pgfscope}%
\begin{pgfscope}%
\pgfpathrectangle{\pgfqpoint{0.562500in}{0.275000in}}{\pgfqpoint{3.487500in}{1.925000in}}%
\pgfusepath{clip}%
\pgfsetbuttcap%
\pgfsetroundjoin%
\definecolor{currentfill}{rgb}{0.000000,0.000000,0.000000}%
\pgfsetfillcolor{currentfill}%
\pgfsetlinewidth{1.003750pt}%
\definecolor{currentstroke}{rgb}{0.000000,0.000000,0.000000}%
\pgfsetstrokecolor{currentstroke}%
\pgfsetdash{}{0pt}%
\pgfpathmoveto{\pgfqpoint{0.721249in}{0.356667in}}%
\pgfpathcurveto{\pgfqpoint{0.726774in}{0.356667in}}{\pgfqpoint{0.732073in}{0.358862in}}{\pgfqpoint{0.735980in}{0.362769in}}%
\pgfpathcurveto{\pgfqpoint{0.739887in}{0.366675in}}{\pgfqpoint{0.742082in}{0.371975in}}{\pgfqpoint{0.742082in}{0.377500in}}%
\pgfpathcurveto{\pgfqpoint{0.742082in}{0.383025in}}{\pgfqpoint{0.739887in}{0.388325in}}{\pgfqpoint{0.735980in}{0.392231in}}%
\pgfpathcurveto{\pgfqpoint{0.732073in}{0.396138in}}{\pgfqpoint{0.726774in}{0.398333in}}{\pgfqpoint{0.721249in}{0.398333in}}%
\pgfpathcurveto{\pgfqpoint{0.715724in}{0.398333in}}{\pgfqpoint{0.710424in}{0.396138in}}{\pgfqpoint{0.706518in}{0.392231in}}%
\pgfpathcurveto{\pgfqpoint{0.702611in}{0.388325in}}{\pgfqpoint{0.700416in}{0.383025in}}{\pgfqpoint{0.700416in}{0.377500in}}%
\pgfpathcurveto{\pgfqpoint{0.700416in}{0.371975in}}{\pgfqpoint{0.702611in}{0.366675in}}{\pgfqpoint{0.706518in}{0.362769in}}%
\pgfpathcurveto{\pgfqpoint{0.710424in}{0.358862in}}{\pgfqpoint{0.715724in}{0.356667in}}{\pgfqpoint{0.721249in}{0.356667in}}%
\pgfpathclose%
\pgfusepath{stroke,fill}%
\end{pgfscope}%
\begin{pgfscope}%
\pgfpathrectangle{\pgfqpoint{0.562500in}{0.275000in}}{\pgfqpoint{3.487500in}{1.925000in}}%
\pgfusepath{clip}%
\pgfsetbuttcap%
\pgfsetroundjoin%
\definecolor{currentfill}{rgb}{0.000000,0.000000,0.000000}%
\pgfsetfillcolor{currentfill}%
\pgfsetlinewidth{1.003750pt}%
\definecolor{currentstroke}{rgb}{0.000000,0.000000,0.000000}%
\pgfsetstrokecolor{currentstroke}%
\pgfsetdash{}{0pt}%
\pgfpathmoveto{\pgfqpoint{0.721249in}{0.356667in}}%
\pgfpathcurveto{\pgfqpoint{0.726774in}{0.356667in}}{\pgfqpoint{0.732073in}{0.358862in}}{\pgfqpoint{0.735980in}{0.362769in}}%
\pgfpathcurveto{\pgfqpoint{0.739887in}{0.366675in}}{\pgfqpoint{0.742082in}{0.371975in}}{\pgfqpoint{0.742082in}{0.377500in}}%
\pgfpathcurveto{\pgfqpoint{0.742082in}{0.383025in}}{\pgfqpoint{0.739887in}{0.388325in}}{\pgfqpoint{0.735980in}{0.392231in}}%
\pgfpathcurveto{\pgfqpoint{0.732073in}{0.396138in}}{\pgfqpoint{0.726774in}{0.398333in}}{\pgfqpoint{0.721249in}{0.398333in}}%
\pgfpathcurveto{\pgfqpoint{0.715724in}{0.398333in}}{\pgfqpoint{0.710424in}{0.396138in}}{\pgfqpoint{0.706518in}{0.392231in}}%
\pgfpathcurveto{\pgfqpoint{0.702611in}{0.388325in}}{\pgfqpoint{0.700416in}{0.383025in}}{\pgfqpoint{0.700416in}{0.377500in}}%
\pgfpathcurveto{\pgfqpoint{0.700416in}{0.371975in}}{\pgfqpoint{0.702611in}{0.366675in}}{\pgfqpoint{0.706518in}{0.362769in}}%
\pgfpathcurveto{\pgfqpoint{0.710424in}{0.358862in}}{\pgfqpoint{0.715724in}{0.356667in}}{\pgfqpoint{0.721249in}{0.356667in}}%
\pgfpathclose%
\pgfusepath{stroke,fill}%
\end{pgfscope}%
\begin{pgfscope}%
\pgfpathrectangle{\pgfqpoint{0.562500in}{0.275000in}}{\pgfqpoint{3.487500in}{1.925000in}}%
\pgfusepath{clip}%
\pgfsetbuttcap%
\pgfsetroundjoin%
\definecolor{currentfill}{rgb}{0.000000,0.000000,0.000000}%
\pgfsetfillcolor{currentfill}%
\pgfsetlinewidth{1.003750pt}%
\definecolor{currentstroke}{rgb}{0.000000,0.000000,0.000000}%
\pgfsetstrokecolor{currentstroke}%
\pgfsetdash{}{0pt}%
\pgfpathmoveto{\pgfqpoint{0.721249in}{0.356667in}}%
\pgfpathcurveto{\pgfqpoint{0.726774in}{0.356667in}}{\pgfqpoint{0.732073in}{0.358862in}}{\pgfqpoint{0.735980in}{0.362769in}}%
\pgfpathcurveto{\pgfqpoint{0.739887in}{0.366675in}}{\pgfqpoint{0.742082in}{0.371975in}}{\pgfqpoint{0.742082in}{0.377500in}}%
\pgfpathcurveto{\pgfqpoint{0.742082in}{0.383025in}}{\pgfqpoint{0.739887in}{0.388325in}}{\pgfqpoint{0.735980in}{0.392231in}}%
\pgfpathcurveto{\pgfqpoint{0.732073in}{0.396138in}}{\pgfqpoint{0.726774in}{0.398333in}}{\pgfqpoint{0.721249in}{0.398333in}}%
\pgfpathcurveto{\pgfqpoint{0.715724in}{0.398333in}}{\pgfqpoint{0.710424in}{0.396138in}}{\pgfqpoint{0.706518in}{0.392231in}}%
\pgfpathcurveto{\pgfqpoint{0.702611in}{0.388325in}}{\pgfqpoint{0.700416in}{0.383025in}}{\pgfqpoint{0.700416in}{0.377500in}}%
\pgfpathcurveto{\pgfqpoint{0.700416in}{0.371975in}}{\pgfqpoint{0.702611in}{0.366675in}}{\pgfqpoint{0.706518in}{0.362769in}}%
\pgfpathcurveto{\pgfqpoint{0.710424in}{0.358862in}}{\pgfqpoint{0.715724in}{0.356667in}}{\pgfqpoint{0.721249in}{0.356667in}}%
\pgfpathclose%
\pgfusepath{stroke,fill}%
\end{pgfscope}%
\begin{pgfscope}%
\pgfpathrectangle{\pgfqpoint{0.562500in}{0.275000in}}{\pgfqpoint{3.487500in}{1.925000in}}%
\pgfusepath{clip}%
\pgfsetbuttcap%
\pgfsetroundjoin%
\definecolor{currentfill}{rgb}{0.000000,0.000000,0.000000}%
\pgfsetfillcolor{currentfill}%
\pgfsetlinewidth{1.003750pt}%
\definecolor{currentstroke}{rgb}{0.000000,0.000000,0.000000}%
\pgfsetstrokecolor{currentstroke}%
\pgfsetdash{}{0pt}%
\pgfpathmoveto{\pgfqpoint{0.721249in}{0.356667in}}%
\pgfpathcurveto{\pgfqpoint{0.726774in}{0.356667in}}{\pgfqpoint{0.732073in}{0.358862in}}{\pgfqpoint{0.735980in}{0.362769in}}%
\pgfpathcurveto{\pgfqpoint{0.739887in}{0.366675in}}{\pgfqpoint{0.742082in}{0.371975in}}{\pgfqpoint{0.742082in}{0.377500in}}%
\pgfpathcurveto{\pgfqpoint{0.742082in}{0.383025in}}{\pgfqpoint{0.739887in}{0.388325in}}{\pgfqpoint{0.735980in}{0.392231in}}%
\pgfpathcurveto{\pgfqpoint{0.732073in}{0.396138in}}{\pgfqpoint{0.726774in}{0.398333in}}{\pgfqpoint{0.721249in}{0.398333in}}%
\pgfpathcurveto{\pgfqpoint{0.715724in}{0.398333in}}{\pgfqpoint{0.710424in}{0.396138in}}{\pgfqpoint{0.706518in}{0.392231in}}%
\pgfpathcurveto{\pgfqpoint{0.702611in}{0.388325in}}{\pgfqpoint{0.700416in}{0.383025in}}{\pgfqpoint{0.700416in}{0.377500in}}%
\pgfpathcurveto{\pgfqpoint{0.700416in}{0.371975in}}{\pgfqpoint{0.702611in}{0.366675in}}{\pgfqpoint{0.706518in}{0.362769in}}%
\pgfpathcurveto{\pgfqpoint{0.710424in}{0.358862in}}{\pgfqpoint{0.715724in}{0.356667in}}{\pgfqpoint{0.721249in}{0.356667in}}%
\pgfpathclose%
\pgfusepath{stroke,fill}%
\end{pgfscope}%
\begin{pgfscope}%
\pgfpathrectangle{\pgfqpoint{0.562500in}{0.275000in}}{\pgfqpoint{3.487500in}{1.925000in}}%
\pgfusepath{clip}%
\pgfsetbuttcap%
\pgfsetroundjoin%
\definecolor{currentfill}{rgb}{0.000000,0.000000,0.000000}%
\pgfsetfillcolor{currentfill}%
\pgfsetlinewidth{1.003750pt}%
\definecolor{currentstroke}{rgb}{0.000000,0.000000,0.000000}%
\pgfsetstrokecolor{currentstroke}%
\pgfsetdash{}{0pt}%
\pgfpathmoveto{\pgfqpoint{0.721249in}{0.356667in}}%
\pgfpathcurveto{\pgfqpoint{0.726774in}{0.356667in}}{\pgfqpoint{0.732073in}{0.358862in}}{\pgfqpoint{0.735980in}{0.362769in}}%
\pgfpathcurveto{\pgfqpoint{0.739887in}{0.366675in}}{\pgfqpoint{0.742082in}{0.371975in}}{\pgfqpoint{0.742082in}{0.377500in}}%
\pgfpathcurveto{\pgfqpoint{0.742082in}{0.383025in}}{\pgfqpoint{0.739887in}{0.388325in}}{\pgfqpoint{0.735980in}{0.392231in}}%
\pgfpathcurveto{\pgfqpoint{0.732073in}{0.396138in}}{\pgfqpoint{0.726774in}{0.398333in}}{\pgfqpoint{0.721249in}{0.398333in}}%
\pgfpathcurveto{\pgfqpoint{0.715724in}{0.398333in}}{\pgfqpoint{0.710424in}{0.396138in}}{\pgfqpoint{0.706518in}{0.392231in}}%
\pgfpathcurveto{\pgfqpoint{0.702611in}{0.388325in}}{\pgfqpoint{0.700416in}{0.383025in}}{\pgfqpoint{0.700416in}{0.377500in}}%
\pgfpathcurveto{\pgfqpoint{0.700416in}{0.371975in}}{\pgfqpoint{0.702611in}{0.366675in}}{\pgfqpoint{0.706518in}{0.362769in}}%
\pgfpathcurveto{\pgfqpoint{0.710424in}{0.358862in}}{\pgfqpoint{0.715724in}{0.356667in}}{\pgfqpoint{0.721249in}{0.356667in}}%
\pgfpathclose%
\pgfusepath{stroke,fill}%
\end{pgfscope}%
\begin{pgfscope}%
\pgfpathrectangle{\pgfqpoint{0.562500in}{0.275000in}}{\pgfqpoint{3.487500in}{1.925000in}}%
\pgfusepath{clip}%
\pgfsetbuttcap%
\pgfsetroundjoin%
\definecolor{currentfill}{rgb}{0.000000,0.000000,0.000000}%
\pgfsetfillcolor{currentfill}%
\pgfsetlinewidth{1.003750pt}%
\definecolor{currentstroke}{rgb}{0.000000,0.000000,0.000000}%
\pgfsetstrokecolor{currentstroke}%
\pgfsetdash{}{0pt}%
\pgfpathmoveto{\pgfqpoint{0.721249in}{0.356667in}}%
\pgfpathcurveto{\pgfqpoint{0.726774in}{0.356667in}}{\pgfqpoint{0.732073in}{0.358862in}}{\pgfqpoint{0.735980in}{0.362769in}}%
\pgfpathcurveto{\pgfqpoint{0.739887in}{0.366675in}}{\pgfqpoint{0.742082in}{0.371975in}}{\pgfqpoint{0.742082in}{0.377500in}}%
\pgfpathcurveto{\pgfqpoint{0.742082in}{0.383025in}}{\pgfqpoint{0.739887in}{0.388325in}}{\pgfqpoint{0.735980in}{0.392231in}}%
\pgfpathcurveto{\pgfqpoint{0.732073in}{0.396138in}}{\pgfqpoint{0.726774in}{0.398333in}}{\pgfqpoint{0.721249in}{0.398333in}}%
\pgfpathcurveto{\pgfqpoint{0.715724in}{0.398333in}}{\pgfqpoint{0.710424in}{0.396138in}}{\pgfqpoint{0.706518in}{0.392231in}}%
\pgfpathcurveto{\pgfqpoint{0.702611in}{0.388325in}}{\pgfqpoint{0.700416in}{0.383025in}}{\pgfqpoint{0.700416in}{0.377500in}}%
\pgfpathcurveto{\pgfqpoint{0.700416in}{0.371975in}}{\pgfqpoint{0.702611in}{0.366675in}}{\pgfqpoint{0.706518in}{0.362769in}}%
\pgfpathcurveto{\pgfqpoint{0.710424in}{0.358862in}}{\pgfqpoint{0.715724in}{0.356667in}}{\pgfqpoint{0.721249in}{0.356667in}}%
\pgfpathclose%
\pgfusepath{stroke,fill}%
\end{pgfscope}%
\begin{pgfscope}%
\pgfpathrectangle{\pgfqpoint{0.562500in}{0.275000in}}{\pgfqpoint{3.487500in}{1.925000in}}%
\pgfusepath{clip}%
\pgfsetbuttcap%
\pgfsetroundjoin%
\definecolor{currentfill}{rgb}{0.000000,0.000000,0.000000}%
\pgfsetfillcolor{currentfill}%
\pgfsetlinewidth{1.003750pt}%
\definecolor{currentstroke}{rgb}{0.000000,0.000000,0.000000}%
\pgfsetstrokecolor{currentstroke}%
\pgfsetdash{}{0pt}%
\pgfpathmoveto{\pgfqpoint{0.721249in}{0.356667in}}%
\pgfpathcurveto{\pgfqpoint{0.726774in}{0.356667in}}{\pgfqpoint{0.732073in}{0.358862in}}{\pgfqpoint{0.735980in}{0.362769in}}%
\pgfpathcurveto{\pgfqpoint{0.739887in}{0.366675in}}{\pgfqpoint{0.742082in}{0.371975in}}{\pgfqpoint{0.742082in}{0.377500in}}%
\pgfpathcurveto{\pgfqpoint{0.742082in}{0.383025in}}{\pgfqpoint{0.739887in}{0.388325in}}{\pgfqpoint{0.735980in}{0.392231in}}%
\pgfpathcurveto{\pgfqpoint{0.732073in}{0.396138in}}{\pgfqpoint{0.726774in}{0.398333in}}{\pgfqpoint{0.721249in}{0.398333in}}%
\pgfpathcurveto{\pgfqpoint{0.715724in}{0.398333in}}{\pgfqpoint{0.710424in}{0.396138in}}{\pgfqpoint{0.706518in}{0.392231in}}%
\pgfpathcurveto{\pgfqpoint{0.702611in}{0.388325in}}{\pgfqpoint{0.700416in}{0.383025in}}{\pgfqpoint{0.700416in}{0.377500in}}%
\pgfpathcurveto{\pgfqpoint{0.700416in}{0.371975in}}{\pgfqpoint{0.702611in}{0.366675in}}{\pgfqpoint{0.706518in}{0.362769in}}%
\pgfpathcurveto{\pgfqpoint{0.710424in}{0.358862in}}{\pgfqpoint{0.715724in}{0.356667in}}{\pgfqpoint{0.721249in}{0.356667in}}%
\pgfpathclose%
\pgfusepath{stroke,fill}%
\end{pgfscope}%
\begin{pgfscope}%
\pgfpathrectangle{\pgfqpoint{0.562500in}{0.275000in}}{\pgfqpoint{3.487500in}{1.925000in}}%
\pgfusepath{clip}%
\pgfsetbuttcap%
\pgfsetroundjoin%
\definecolor{currentfill}{rgb}{0.000000,0.000000,0.000000}%
\pgfsetfillcolor{currentfill}%
\pgfsetlinewidth{1.003750pt}%
\definecolor{currentstroke}{rgb}{0.000000,0.000000,0.000000}%
\pgfsetstrokecolor{currentstroke}%
\pgfsetdash{}{0pt}%
\pgfpathmoveto{\pgfqpoint{0.721249in}{0.356667in}}%
\pgfpathcurveto{\pgfqpoint{0.726774in}{0.356667in}}{\pgfqpoint{0.732073in}{0.358862in}}{\pgfqpoint{0.735980in}{0.362769in}}%
\pgfpathcurveto{\pgfqpoint{0.739887in}{0.366675in}}{\pgfqpoint{0.742082in}{0.371975in}}{\pgfqpoint{0.742082in}{0.377500in}}%
\pgfpathcurveto{\pgfqpoint{0.742082in}{0.383025in}}{\pgfqpoint{0.739887in}{0.388325in}}{\pgfqpoint{0.735980in}{0.392231in}}%
\pgfpathcurveto{\pgfqpoint{0.732073in}{0.396138in}}{\pgfqpoint{0.726774in}{0.398333in}}{\pgfqpoint{0.721249in}{0.398333in}}%
\pgfpathcurveto{\pgfqpoint{0.715724in}{0.398333in}}{\pgfqpoint{0.710424in}{0.396138in}}{\pgfqpoint{0.706518in}{0.392231in}}%
\pgfpathcurveto{\pgfqpoint{0.702611in}{0.388325in}}{\pgfqpoint{0.700416in}{0.383025in}}{\pgfqpoint{0.700416in}{0.377500in}}%
\pgfpathcurveto{\pgfqpoint{0.700416in}{0.371975in}}{\pgfqpoint{0.702611in}{0.366675in}}{\pgfqpoint{0.706518in}{0.362769in}}%
\pgfpathcurveto{\pgfqpoint{0.710424in}{0.358862in}}{\pgfqpoint{0.715724in}{0.356667in}}{\pgfqpoint{0.721249in}{0.356667in}}%
\pgfpathclose%
\pgfusepath{stroke,fill}%
\end{pgfscope}%
\begin{pgfscope}%
\pgfpathrectangle{\pgfqpoint{0.562500in}{0.275000in}}{\pgfqpoint{3.487500in}{1.925000in}}%
\pgfusepath{clip}%
\pgfsetbuttcap%
\pgfsetroundjoin%
\definecolor{currentfill}{rgb}{0.000000,0.000000,0.000000}%
\pgfsetfillcolor{currentfill}%
\pgfsetlinewidth{1.003750pt}%
\definecolor{currentstroke}{rgb}{0.000000,0.000000,0.000000}%
\pgfsetstrokecolor{currentstroke}%
\pgfsetdash{}{0pt}%
\pgfpathmoveto{\pgfqpoint{0.721249in}{0.356667in}}%
\pgfpathcurveto{\pgfqpoint{0.726774in}{0.356667in}}{\pgfqpoint{0.732073in}{0.358862in}}{\pgfqpoint{0.735980in}{0.362769in}}%
\pgfpathcurveto{\pgfqpoint{0.739887in}{0.366675in}}{\pgfqpoint{0.742082in}{0.371975in}}{\pgfqpoint{0.742082in}{0.377500in}}%
\pgfpathcurveto{\pgfqpoint{0.742082in}{0.383025in}}{\pgfqpoint{0.739887in}{0.388325in}}{\pgfqpoint{0.735980in}{0.392231in}}%
\pgfpathcurveto{\pgfqpoint{0.732073in}{0.396138in}}{\pgfqpoint{0.726774in}{0.398333in}}{\pgfqpoint{0.721249in}{0.398333in}}%
\pgfpathcurveto{\pgfqpoint{0.715724in}{0.398333in}}{\pgfqpoint{0.710424in}{0.396138in}}{\pgfqpoint{0.706518in}{0.392231in}}%
\pgfpathcurveto{\pgfqpoint{0.702611in}{0.388325in}}{\pgfqpoint{0.700416in}{0.383025in}}{\pgfqpoint{0.700416in}{0.377500in}}%
\pgfpathcurveto{\pgfqpoint{0.700416in}{0.371975in}}{\pgfqpoint{0.702611in}{0.366675in}}{\pgfqpoint{0.706518in}{0.362769in}}%
\pgfpathcurveto{\pgfqpoint{0.710424in}{0.358862in}}{\pgfqpoint{0.715724in}{0.356667in}}{\pgfqpoint{0.721249in}{0.356667in}}%
\pgfpathclose%
\pgfusepath{stroke,fill}%
\end{pgfscope}%
\begin{pgfscope}%
\pgfpathrectangle{\pgfqpoint{0.562500in}{0.275000in}}{\pgfqpoint{3.487500in}{1.925000in}}%
\pgfusepath{clip}%
\pgfsetbuttcap%
\pgfsetroundjoin%
\definecolor{currentfill}{rgb}{0.000000,0.000000,0.000000}%
\pgfsetfillcolor{currentfill}%
\pgfsetlinewidth{1.003750pt}%
\definecolor{currentstroke}{rgb}{0.000000,0.000000,0.000000}%
\pgfsetstrokecolor{currentstroke}%
\pgfsetdash{}{0pt}%
\pgfpathmoveto{\pgfqpoint{0.721249in}{0.356667in}}%
\pgfpathcurveto{\pgfqpoint{0.726774in}{0.356667in}}{\pgfqpoint{0.732073in}{0.358862in}}{\pgfqpoint{0.735980in}{0.362769in}}%
\pgfpathcurveto{\pgfqpoint{0.739887in}{0.366675in}}{\pgfqpoint{0.742082in}{0.371975in}}{\pgfqpoint{0.742082in}{0.377500in}}%
\pgfpathcurveto{\pgfqpoint{0.742082in}{0.383025in}}{\pgfqpoint{0.739887in}{0.388325in}}{\pgfqpoint{0.735980in}{0.392231in}}%
\pgfpathcurveto{\pgfqpoint{0.732073in}{0.396138in}}{\pgfqpoint{0.726774in}{0.398333in}}{\pgfqpoint{0.721249in}{0.398333in}}%
\pgfpathcurveto{\pgfqpoint{0.715724in}{0.398333in}}{\pgfqpoint{0.710424in}{0.396138in}}{\pgfqpoint{0.706518in}{0.392231in}}%
\pgfpathcurveto{\pgfqpoint{0.702611in}{0.388325in}}{\pgfqpoint{0.700416in}{0.383025in}}{\pgfqpoint{0.700416in}{0.377500in}}%
\pgfpathcurveto{\pgfqpoint{0.700416in}{0.371975in}}{\pgfqpoint{0.702611in}{0.366675in}}{\pgfqpoint{0.706518in}{0.362769in}}%
\pgfpathcurveto{\pgfqpoint{0.710424in}{0.358862in}}{\pgfqpoint{0.715724in}{0.356667in}}{\pgfqpoint{0.721249in}{0.356667in}}%
\pgfpathclose%
\pgfusepath{stroke,fill}%
\end{pgfscope}%
\begin{pgfscope}%
\pgfpathrectangle{\pgfqpoint{0.562500in}{0.275000in}}{\pgfqpoint{3.487500in}{1.925000in}}%
\pgfusepath{clip}%
\pgfsetbuttcap%
\pgfsetroundjoin%
\definecolor{currentfill}{rgb}{0.000000,0.000000,0.000000}%
\pgfsetfillcolor{currentfill}%
\pgfsetlinewidth{1.003750pt}%
\definecolor{currentstroke}{rgb}{0.000000,0.000000,0.000000}%
\pgfsetstrokecolor{currentstroke}%
\pgfsetdash{}{0pt}%
\pgfpathmoveto{\pgfqpoint{0.721249in}{0.356667in}}%
\pgfpathcurveto{\pgfqpoint{0.726774in}{0.356667in}}{\pgfqpoint{0.732073in}{0.358862in}}{\pgfqpoint{0.735980in}{0.362769in}}%
\pgfpathcurveto{\pgfqpoint{0.739887in}{0.366675in}}{\pgfqpoint{0.742082in}{0.371975in}}{\pgfqpoint{0.742082in}{0.377500in}}%
\pgfpathcurveto{\pgfqpoint{0.742082in}{0.383025in}}{\pgfqpoint{0.739887in}{0.388325in}}{\pgfqpoint{0.735980in}{0.392231in}}%
\pgfpathcurveto{\pgfqpoint{0.732073in}{0.396138in}}{\pgfqpoint{0.726774in}{0.398333in}}{\pgfqpoint{0.721249in}{0.398333in}}%
\pgfpathcurveto{\pgfqpoint{0.715724in}{0.398333in}}{\pgfqpoint{0.710424in}{0.396138in}}{\pgfqpoint{0.706518in}{0.392231in}}%
\pgfpathcurveto{\pgfqpoint{0.702611in}{0.388325in}}{\pgfqpoint{0.700416in}{0.383025in}}{\pgfqpoint{0.700416in}{0.377500in}}%
\pgfpathcurveto{\pgfqpoint{0.700416in}{0.371975in}}{\pgfqpoint{0.702611in}{0.366675in}}{\pgfqpoint{0.706518in}{0.362769in}}%
\pgfpathcurveto{\pgfqpoint{0.710424in}{0.358862in}}{\pgfqpoint{0.715724in}{0.356667in}}{\pgfqpoint{0.721249in}{0.356667in}}%
\pgfpathclose%
\pgfusepath{stroke,fill}%
\end{pgfscope}%
\begin{pgfscope}%
\pgfpathrectangle{\pgfqpoint{0.562500in}{0.275000in}}{\pgfqpoint{3.487500in}{1.925000in}}%
\pgfusepath{clip}%
\pgfsetbuttcap%
\pgfsetroundjoin%
\definecolor{currentfill}{rgb}{0.000000,0.000000,0.000000}%
\pgfsetfillcolor{currentfill}%
\pgfsetlinewidth{1.003750pt}%
\definecolor{currentstroke}{rgb}{0.000000,0.000000,0.000000}%
\pgfsetstrokecolor{currentstroke}%
\pgfsetdash{}{0pt}%
\pgfpathmoveto{\pgfqpoint{0.721249in}{0.356667in}}%
\pgfpathcurveto{\pgfqpoint{0.726774in}{0.356667in}}{\pgfqpoint{0.732073in}{0.358862in}}{\pgfqpoint{0.735980in}{0.362769in}}%
\pgfpathcurveto{\pgfqpoint{0.739887in}{0.366675in}}{\pgfqpoint{0.742082in}{0.371975in}}{\pgfqpoint{0.742082in}{0.377500in}}%
\pgfpathcurveto{\pgfqpoint{0.742082in}{0.383025in}}{\pgfqpoint{0.739887in}{0.388325in}}{\pgfqpoint{0.735980in}{0.392231in}}%
\pgfpathcurveto{\pgfqpoint{0.732073in}{0.396138in}}{\pgfqpoint{0.726774in}{0.398333in}}{\pgfqpoint{0.721249in}{0.398333in}}%
\pgfpathcurveto{\pgfqpoint{0.715724in}{0.398333in}}{\pgfqpoint{0.710424in}{0.396138in}}{\pgfqpoint{0.706518in}{0.392231in}}%
\pgfpathcurveto{\pgfqpoint{0.702611in}{0.388325in}}{\pgfqpoint{0.700416in}{0.383025in}}{\pgfqpoint{0.700416in}{0.377500in}}%
\pgfpathcurveto{\pgfqpoint{0.700416in}{0.371975in}}{\pgfqpoint{0.702611in}{0.366675in}}{\pgfqpoint{0.706518in}{0.362769in}}%
\pgfpathcurveto{\pgfqpoint{0.710424in}{0.358862in}}{\pgfqpoint{0.715724in}{0.356667in}}{\pgfqpoint{0.721249in}{0.356667in}}%
\pgfpathclose%
\pgfusepath{stroke,fill}%
\end{pgfscope}%
\begin{pgfscope}%
\pgfpathrectangle{\pgfqpoint{0.562500in}{0.275000in}}{\pgfqpoint{3.487500in}{1.925000in}}%
\pgfusepath{clip}%
\pgfsetbuttcap%
\pgfsetroundjoin%
\definecolor{currentfill}{rgb}{0.000000,0.000000,0.000000}%
\pgfsetfillcolor{currentfill}%
\pgfsetlinewidth{1.003750pt}%
\definecolor{currentstroke}{rgb}{0.000000,0.000000,0.000000}%
\pgfsetstrokecolor{currentstroke}%
\pgfsetdash{}{0pt}%
\pgfpathmoveto{\pgfqpoint{0.721249in}{0.356667in}}%
\pgfpathcurveto{\pgfqpoint{0.726774in}{0.356667in}}{\pgfqpoint{0.732073in}{0.358862in}}{\pgfqpoint{0.735980in}{0.362769in}}%
\pgfpathcurveto{\pgfqpoint{0.739887in}{0.366675in}}{\pgfqpoint{0.742082in}{0.371975in}}{\pgfqpoint{0.742082in}{0.377500in}}%
\pgfpathcurveto{\pgfqpoint{0.742082in}{0.383025in}}{\pgfqpoint{0.739887in}{0.388325in}}{\pgfqpoint{0.735980in}{0.392231in}}%
\pgfpathcurveto{\pgfqpoint{0.732073in}{0.396138in}}{\pgfqpoint{0.726774in}{0.398333in}}{\pgfqpoint{0.721249in}{0.398333in}}%
\pgfpathcurveto{\pgfqpoint{0.715724in}{0.398333in}}{\pgfqpoint{0.710424in}{0.396138in}}{\pgfqpoint{0.706518in}{0.392231in}}%
\pgfpathcurveto{\pgfqpoint{0.702611in}{0.388325in}}{\pgfqpoint{0.700416in}{0.383025in}}{\pgfqpoint{0.700416in}{0.377500in}}%
\pgfpathcurveto{\pgfqpoint{0.700416in}{0.371975in}}{\pgfqpoint{0.702611in}{0.366675in}}{\pgfqpoint{0.706518in}{0.362769in}}%
\pgfpathcurveto{\pgfqpoint{0.710424in}{0.358862in}}{\pgfqpoint{0.715724in}{0.356667in}}{\pgfqpoint{0.721249in}{0.356667in}}%
\pgfpathclose%
\pgfusepath{stroke,fill}%
\end{pgfscope}%
\begin{pgfscope}%
\pgfpathrectangle{\pgfqpoint{0.562500in}{0.275000in}}{\pgfqpoint{3.487500in}{1.925000in}}%
\pgfusepath{clip}%
\pgfsetbuttcap%
\pgfsetroundjoin%
\definecolor{currentfill}{rgb}{0.000000,0.000000,0.000000}%
\pgfsetfillcolor{currentfill}%
\pgfsetlinewidth{1.003750pt}%
\definecolor{currentstroke}{rgb}{0.000000,0.000000,0.000000}%
\pgfsetstrokecolor{currentstroke}%
\pgfsetdash{}{0pt}%
\pgfpathmoveto{\pgfqpoint{0.721249in}{0.356667in}}%
\pgfpathcurveto{\pgfqpoint{0.726774in}{0.356667in}}{\pgfqpoint{0.732073in}{0.358862in}}{\pgfqpoint{0.735980in}{0.362769in}}%
\pgfpathcurveto{\pgfqpoint{0.739887in}{0.366675in}}{\pgfqpoint{0.742082in}{0.371975in}}{\pgfqpoint{0.742082in}{0.377500in}}%
\pgfpathcurveto{\pgfqpoint{0.742082in}{0.383025in}}{\pgfqpoint{0.739887in}{0.388325in}}{\pgfqpoint{0.735980in}{0.392231in}}%
\pgfpathcurveto{\pgfqpoint{0.732073in}{0.396138in}}{\pgfqpoint{0.726774in}{0.398333in}}{\pgfqpoint{0.721249in}{0.398333in}}%
\pgfpathcurveto{\pgfqpoint{0.715724in}{0.398333in}}{\pgfqpoint{0.710424in}{0.396138in}}{\pgfqpoint{0.706518in}{0.392231in}}%
\pgfpathcurveto{\pgfqpoint{0.702611in}{0.388325in}}{\pgfqpoint{0.700416in}{0.383025in}}{\pgfqpoint{0.700416in}{0.377500in}}%
\pgfpathcurveto{\pgfqpoint{0.700416in}{0.371975in}}{\pgfqpoint{0.702611in}{0.366675in}}{\pgfqpoint{0.706518in}{0.362769in}}%
\pgfpathcurveto{\pgfqpoint{0.710424in}{0.358862in}}{\pgfqpoint{0.715724in}{0.356667in}}{\pgfqpoint{0.721249in}{0.356667in}}%
\pgfpathclose%
\pgfusepath{stroke,fill}%
\end{pgfscope}%
\begin{pgfscope}%
\pgfpathrectangle{\pgfqpoint{0.562500in}{0.275000in}}{\pgfqpoint{3.487500in}{1.925000in}}%
\pgfusepath{clip}%
\pgfsetbuttcap%
\pgfsetroundjoin%
\definecolor{currentfill}{rgb}{0.000000,0.000000,0.000000}%
\pgfsetfillcolor{currentfill}%
\pgfsetlinewidth{1.003750pt}%
\definecolor{currentstroke}{rgb}{0.000000,0.000000,0.000000}%
\pgfsetstrokecolor{currentstroke}%
\pgfsetdash{}{0pt}%
\pgfpathmoveto{\pgfqpoint{0.721249in}{0.356667in}}%
\pgfpathcurveto{\pgfqpoint{0.726774in}{0.356667in}}{\pgfqpoint{0.732073in}{0.358862in}}{\pgfqpoint{0.735980in}{0.362769in}}%
\pgfpathcurveto{\pgfqpoint{0.739887in}{0.366675in}}{\pgfqpoint{0.742082in}{0.371975in}}{\pgfqpoint{0.742082in}{0.377500in}}%
\pgfpathcurveto{\pgfqpoint{0.742082in}{0.383025in}}{\pgfqpoint{0.739887in}{0.388325in}}{\pgfqpoint{0.735980in}{0.392231in}}%
\pgfpathcurveto{\pgfqpoint{0.732073in}{0.396138in}}{\pgfqpoint{0.726774in}{0.398333in}}{\pgfqpoint{0.721249in}{0.398333in}}%
\pgfpathcurveto{\pgfqpoint{0.715724in}{0.398333in}}{\pgfqpoint{0.710424in}{0.396138in}}{\pgfqpoint{0.706518in}{0.392231in}}%
\pgfpathcurveto{\pgfqpoint{0.702611in}{0.388325in}}{\pgfqpoint{0.700416in}{0.383025in}}{\pgfqpoint{0.700416in}{0.377500in}}%
\pgfpathcurveto{\pgfqpoint{0.700416in}{0.371975in}}{\pgfqpoint{0.702611in}{0.366675in}}{\pgfqpoint{0.706518in}{0.362769in}}%
\pgfpathcurveto{\pgfqpoint{0.710424in}{0.358862in}}{\pgfqpoint{0.715724in}{0.356667in}}{\pgfqpoint{0.721249in}{0.356667in}}%
\pgfpathclose%
\pgfusepath{stroke,fill}%
\end{pgfscope}%
\begin{pgfscope}%
\pgfpathrectangle{\pgfqpoint{0.562500in}{0.275000in}}{\pgfqpoint{3.487500in}{1.925000in}}%
\pgfusepath{clip}%
\pgfsetbuttcap%
\pgfsetroundjoin%
\definecolor{currentfill}{rgb}{0.000000,0.000000,0.000000}%
\pgfsetfillcolor{currentfill}%
\pgfsetlinewidth{1.003750pt}%
\definecolor{currentstroke}{rgb}{0.000000,0.000000,0.000000}%
\pgfsetstrokecolor{currentstroke}%
\pgfsetdash{}{0pt}%
\pgfpathmoveto{\pgfqpoint{0.721249in}{0.356667in}}%
\pgfpathcurveto{\pgfqpoint{0.726774in}{0.356667in}}{\pgfqpoint{0.732073in}{0.358862in}}{\pgfqpoint{0.735980in}{0.362769in}}%
\pgfpathcurveto{\pgfqpoint{0.739887in}{0.366675in}}{\pgfqpoint{0.742082in}{0.371975in}}{\pgfqpoint{0.742082in}{0.377500in}}%
\pgfpathcurveto{\pgfqpoint{0.742082in}{0.383025in}}{\pgfqpoint{0.739887in}{0.388325in}}{\pgfqpoint{0.735980in}{0.392231in}}%
\pgfpathcurveto{\pgfqpoint{0.732073in}{0.396138in}}{\pgfqpoint{0.726774in}{0.398333in}}{\pgfqpoint{0.721249in}{0.398333in}}%
\pgfpathcurveto{\pgfqpoint{0.715724in}{0.398333in}}{\pgfqpoint{0.710424in}{0.396138in}}{\pgfqpoint{0.706518in}{0.392231in}}%
\pgfpathcurveto{\pgfqpoint{0.702611in}{0.388325in}}{\pgfqpoint{0.700416in}{0.383025in}}{\pgfqpoint{0.700416in}{0.377500in}}%
\pgfpathcurveto{\pgfqpoint{0.700416in}{0.371975in}}{\pgfqpoint{0.702611in}{0.366675in}}{\pgfqpoint{0.706518in}{0.362769in}}%
\pgfpathcurveto{\pgfqpoint{0.710424in}{0.358862in}}{\pgfqpoint{0.715724in}{0.356667in}}{\pgfqpoint{0.721249in}{0.356667in}}%
\pgfpathclose%
\pgfusepath{stroke,fill}%
\end{pgfscope}%
\begin{pgfscope}%
\pgfpathrectangle{\pgfqpoint{0.562500in}{0.275000in}}{\pgfqpoint{3.487500in}{1.925000in}}%
\pgfusepath{clip}%
\pgfsetbuttcap%
\pgfsetroundjoin%
\definecolor{currentfill}{rgb}{0.000000,0.000000,0.000000}%
\pgfsetfillcolor{currentfill}%
\pgfsetlinewidth{1.003750pt}%
\definecolor{currentstroke}{rgb}{0.000000,0.000000,0.000000}%
\pgfsetstrokecolor{currentstroke}%
\pgfsetdash{}{0pt}%
\pgfpathmoveto{\pgfqpoint{0.721249in}{0.356667in}}%
\pgfpathcurveto{\pgfqpoint{0.726774in}{0.356667in}}{\pgfqpoint{0.732073in}{0.358862in}}{\pgfqpoint{0.735980in}{0.362769in}}%
\pgfpathcurveto{\pgfqpoint{0.739887in}{0.366675in}}{\pgfqpoint{0.742082in}{0.371975in}}{\pgfqpoint{0.742082in}{0.377500in}}%
\pgfpathcurveto{\pgfqpoint{0.742082in}{0.383025in}}{\pgfqpoint{0.739887in}{0.388325in}}{\pgfqpoint{0.735980in}{0.392231in}}%
\pgfpathcurveto{\pgfqpoint{0.732073in}{0.396138in}}{\pgfqpoint{0.726774in}{0.398333in}}{\pgfqpoint{0.721249in}{0.398333in}}%
\pgfpathcurveto{\pgfqpoint{0.715724in}{0.398333in}}{\pgfqpoint{0.710424in}{0.396138in}}{\pgfqpoint{0.706518in}{0.392231in}}%
\pgfpathcurveto{\pgfqpoint{0.702611in}{0.388325in}}{\pgfqpoint{0.700416in}{0.383025in}}{\pgfqpoint{0.700416in}{0.377500in}}%
\pgfpathcurveto{\pgfqpoint{0.700416in}{0.371975in}}{\pgfqpoint{0.702611in}{0.366675in}}{\pgfqpoint{0.706518in}{0.362769in}}%
\pgfpathcurveto{\pgfqpoint{0.710424in}{0.358862in}}{\pgfqpoint{0.715724in}{0.356667in}}{\pgfqpoint{0.721249in}{0.356667in}}%
\pgfpathclose%
\pgfusepath{stroke,fill}%
\end{pgfscope}%
\begin{pgfscope}%
\pgfpathrectangle{\pgfqpoint{0.562500in}{0.275000in}}{\pgfqpoint{3.487500in}{1.925000in}}%
\pgfusepath{clip}%
\pgfsetbuttcap%
\pgfsetroundjoin%
\definecolor{currentfill}{rgb}{0.000000,0.000000,0.000000}%
\pgfsetfillcolor{currentfill}%
\pgfsetlinewidth{1.003750pt}%
\definecolor{currentstroke}{rgb}{0.000000,0.000000,0.000000}%
\pgfsetstrokecolor{currentstroke}%
\pgfsetdash{}{0pt}%
\pgfpathmoveto{\pgfqpoint{0.721249in}{0.356667in}}%
\pgfpathcurveto{\pgfqpoint{0.726774in}{0.356667in}}{\pgfqpoint{0.732073in}{0.358862in}}{\pgfqpoint{0.735980in}{0.362769in}}%
\pgfpathcurveto{\pgfqpoint{0.739887in}{0.366675in}}{\pgfqpoint{0.742082in}{0.371975in}}{\pgfqpoint{0.742082in}{0.377500in}}%
\pgfpathcurveto{\pgfqpoint{0.742082in}{0.383025in}}{\pgfqpoint{0.739887in}{0.388325in}}{\pgfqpoint{0.735980in}{0.392231in}}%
\pgfpathcurveto{\pgfqpoint{0.732073in}{0.396138in}}{\pgfqpoint{0.726774in}{0.398333in}}{\pgfqpoint{0.721249in}{0.398333in}}%
\pgfpathcurveto{\pgfqpoint{0.715724in}{0.398333in}}{\pgfqpoint{0.710424in}{0.396138in}}{\pgfqpoint{0.706518in}{0.392231in}}%
\pgfpathcurveto{\pgfqpoint{0.702611in}{0.388325in}}{\pgfqpoint{0.700416in}{0.383025in}}{\pgfqpoint{0.700416in}{0.377500in}}%
\pgfpathcurveto{\pgfqpoint{0.700416in}{0.371975in}}{\pgfqpoint{0.702611in}{0.366675in}}{\pgfqpoint{0.706518in}{0.362769in}}%
\pgfpathcurveto{\pgfqpoint{0.710424in}{0.358862in}}{\pgfqpoint{0.715724in}{0.356667in}}{\pgfqpoint{0.721249in}{0.356667in}}%
\pgfpathclose%
\pgfusepath{stroke,fill}%
\end{pgfscope}%
\begin{pgfscope}%
\pgfpathrectangle{\pgfqpoint{0.562500in}{0.275000in}}{\pgfqpoint{3.487500in}{1.925000in}}%
\pgfusepath{clip}%
\pgfsetbuttcap%
\pgfsetroundjoin%
\definecolor{currentfill}{rgb}{0.000000,0.000000,0.000000}%
\pgfsetfillcolor{currentfill}%
\pgfsetlinewidth{1.003750pt}%
\definecolor{currentstroke}{rgb}{0.000000,0.000000,0.000000}%
\pgfsetstrokecolor{currentstroke}%
\pgfsetdash{}{0pt}%
\pgfpathmoveto{\pgfqpoint{0.721249in}{0.356667in}}%
\pgfpathcurveto{\pgfqpoint{0.726774in}{0.356667in}}{\pgfqpoint{0.732073in}{0.358862in}}{\pgfqpoint{0.735980in}{0.362769in}}%
\pgfpathcurveto{\pgfqpoint{0.739887in}{0.366675in}}{\pgfqpoint{0.742082in}{0.371975in}}{\pgfqpoint{0.742082in}{0.377500in}}%
\pgfpathcurveto{\pgfqpoint{0.742082in}{0.383025in}}{\pgfqpoint{0.739887in}{0.388325in}}{\pgfqpoint{0.735980in}{0.392231in}}%
\pgfpathcurveto{\pgfqpoint{0.732073in}{0.396138in}}{\pgfqpoint{0.726774in}{0.398333in}}{\pgfqpoint{0.721249in}{0.398333in}}%
\pgfpathcurveto{\pgfqpoint{0.715724in}{0.398333in}}{\pgfqpoint{0.710424in}{0.396138in}}{\pgfqpoint{0.706518in}{0.392231in}}%
\pgfpathcurveto{\pgfqpoint{0.702611in}{0.388325in}}{\pgfqpoint{0.700416in}{0.383025in}}{\pgfqpoint{0.700416in}{0.377500in}}%
\pgfpathcurveto{\pgfqpoint{0.700416in}{0.371975in}}{\pgfqpoint{0.702611in}{0.366675in}}{\pgfqpoint{0.706518in}{0.362769in}}%
\pgfpathcurveto{\pgfqpoint{0.710424in}{0.358862in}}{\pgfqpoint{0.715724in}{0.356667in}}{\pgfqpoint{0.721249in}{0.356667in}}%
\pgfpathclose%
\pgfusepath{stroke,fill}%
\end{pgfscope}%
\begin{pgfscope}%
\pgfpathrectangle{\pgfqpoint{0.562500in}{0.275000in}}{\pgfqpoint{3.487500in}{1.925000in}}%
\pgfusepath{clip}%
\pgfsetbuttcap%
\pgfsetroundjoin%
\definecolor{currentfill}{rgb}{0.000000,0.000000,0.000000}%
\pgfsetfillcolor{currentfill}%
\pgfsetlinewidth{1.003750pt}%
\definecolor{currentstroke}{rgb}{0.000000,0.000000,0.000000}%
\pgfsetstrokecolor{currentstroke}%
\pgfsetdash{}{0pt}%
\pgfpathmoveto{\pgfqpoint{0.721249in}{0.356667in}}%
\pgfpathcurveto{\pgfqpoint{0.726774in}{0.356667in}}{\pgfqpoint{0.732073in}{0.358862in}}{\pgfqpoint{0.735980in}{0.362769in}}%
\pgfpathcurveto{\pgfqpoint{0.739887in}{0.366675in}}{\pgfqpoint{0.742082in}{0.371975in}}{\pgfqpoint{0.742082in}{0.377500in}}%
\pgfpathcurveto{\pgfqpoint{0.742082in}{0.383025in}}{\pgfqpoint{0.739887in}{0.388325in}}{\pgfqpoint{0.735980in}{0.392231in}}%
\pgfpathcurveto{\pgfqpoint{0.732073in}{0.396138in}}{\pgfqpoint{0.726774in}{0.398333in}}{\pgfqpoint{0.721249in}{0.398333in}}%
\pgfpathcurveto{\pgfqpoint{0.715724in}{0.398333in}}{\pgfqpoint{0.710424in}{0.396138in}}{\pgfqpoint{0.706518in}{0.392231in}}%
\pgfpathcurveto{\pgfqpoint{0.702611in}{0.388325in}}{\pgfqpoint{0.700416in}{0.383025in}}{\pgfqpoint{0.700416in}{0.377500in}}%
\pgfpathcurveto{\pgfqpoint{0.700416in}{0.371975in}}{\pgfqpoint{0.702611in}{0.366675in}}{\pgfqpoint{0.706518in}{0.362769in}}%
\pgfpathcurveto{\pgfqpoint{0.710424in}{0.358862in}}{\pgfqpoint{0.715724in}{0.356667in}}{\pgfqpoint{0.721249in}{0.356667in}}%
\pgfpathclose%
\pgfusepath{stroke,fill}%
\end{pgfscope}%
\begin{pgfscope}%
\pgfpathrectangle{\pgfqpoint{0.562500in}{0.275000in}}{\pgfqpoint{3.487500in}{1.925000in}}%
\pgfusepath{clip}%
\pgfsetbuttcap%
\pgfsetroundjoin%
\definecolor{currentfill}{rgb}{0.000000,0.000000,0.000000}%
\pgfsetfillcolor{currentfill}%
\pgfsetlinewidth{1.003750pt}%
\definecolor{currentstroke}{rgb}{0.000000,0.000000,0.000000}%
\pgfsetstrokecolor{currentstroke}%
\pgfsetdash{}{0pt}%
\pgfpathmoveto{\pgfqpoint{0.721249in}{0.356667in}}%
\pgfpathcurveto{\pgfqpoint{0.726774in}{0.356667in}}{\pgfqpoint{0.732073in}{0.358862in}}{\pgfqpoint{0.735980in}{0.362769in}}%
\pgfpathcurveto{\pgfqpoint{0.739887in}{0.366675in}}{\pgfqpoint{0.742082in}{0.371975in}}{\pgfqpoint{0.742082in}{0.377500in}}%
\pgfpathcurveto{\pgfqpoint{0.742082in}{0.383025in}}{\pgfqpoint{0.739887in}{0.388325in}}{\pgfqpoint{0.735980in}{0.392231in}}%
\pgfpathcurveto{\pgfqpoint{0.732073in}{0.396138in}}{\pgfqpoint{0.726774in}{0.398333in}}{\pgfqpoint{0.721249in}{0.398333in}}%
\pgfpathcurveto{\pgfqpoint{0.715724in}{0.398333in}}{\pgfqpoint{0.710424in}{0.396138in}}{\pgfqpoint{0.706518in}{0.392231in}}%
\pgfpathcurveto{\pgfqpoint{0.702611in}{0.388325in}}{\pgfqpoint{0.700416in}{0.383025in}}{\pgfqpoint{0.700416in}{0.377500in}}%
\pgfpathcurveto{\pgfqpoint{0.700416in}{0.371975in}}{\pgfqpoint{0.702611in}{0.366675in}}{\pgfqpoint{0.706518in}{0.362769in}}%
\pgfpathcurveto{\pgfqpoint{0.710424in}{0.358862in}}{\pgfqpoint{0.715724in}{0.356667in}}{\pgfqpoint{0.721249in}{0.356667in}}%
\pgfpathclose%
\pgfusepath{stroke,fill}%
\end{pgfscope}%
\begin{pgfscope}%
\pgfpathrectangle{\pgfqpoint{0.562500in}{0.275000in}}{\pgfqpoint{3.487500in}{1.925000in}}%
\pgfusepath{clip}%
\pgfsetbuttcap%
\pgfsetroundjoin%
\definecolor{currentfill}{rgb}{0.000000,0.000000,0.000000}%
\pgfsetfillcolor{currentfill}%
\pgfsetlinewidth{1.003750pt}%
\definecolor{currentstroke}{rgb}{0.000000,0.000000,0.000000}%
\pgfsetstrokecolor{currentstroke}%
\pgfsetdash{}{0pt}%
\pgfpathmoveto{\pgfqpoint{0.721249in}{0.356667in}}%
\pgfpathcurveto{\pgfqpoint{0.726774in}{0.356667in}}{\pgfqpoint{0.732073in}{0.358862in}}{\pgfqpoint{0.735980in}{0.362769in}}%
\pgfpathcurveto{\pgfqpoint{0.739887in}{0.366675in}}{\pgfqpoint{0.742082in}{0.371975in}}{\pgfqpoint{0.742082in}{0.377500in}}%
\pgfpathcurveto{\pgfqpoint{0.742082in}{0.383025in}}{\pgfqpoint{0.739887in}{0.388325in}}{\pgfqpoint{0.735980in}{0.392231in}}%
\pgfpathcurveto{\pgfqpoint{0.732073in}{0.396138in}}{\pgfqpoint{0.726774in}{0.398333in}}{\pgfqpoint{0.721249in}{0.398333in}}%
\pgfpathcurveto{\pgfqpoint{0.715724in}{0.398333in}}{\pgfqpoint{0.710424in}{0.396138in}}{\pgfqpoint{0.706518in}{0.392231in}}%
\pgfpathcurveto{\pgfqpoint{0.702611in}{0.388325in}}{\pgfqpoint{0.700416in}{0.383025in}}{\pgfqpoint{0.700416in}{0.377500in}}%
\pgfpathcurveto{\pgfqpoint{0.700416in}{0.371975in}}{\pgfqpoint{0.702611in}{0.366675in}}{\pgfqpoint{0.706518in}{0.362769in}}%
\pgfpathcurveto{\pgfqpoint{0.710424in}{0.358862in}}{\pgfqpoint{0.715724in}{0.356667in}}{\pgfqpoint{0.721249in}{0.356667in}}%
\pgfpathclose%
\pgfusepath{stroke,fill}%
\end{pgfscope}%
\begin{pgfscope}%
\pgfpathrectangle{\pgfqpoint{0.562500in}{0.275000in}}{\pgfqpoint{3.487500in}{1.925000in}}%
\pgfusepath{clip}%
\pgfsetbuttcap%
\pgfsetroundjoin%
\definecolor{currentfill}{rgb}{0.000000,0.000000,0.000000}%
\pgfsetfillcolor{currentfill}%
\pgfsetlinewidth{1.003750pt}%
\definecolor{currentstroke}{rgb}{0.000000,0.000000,0.000000}%
\pgfsetstrokecolor{currentstroke}%
\pgfsetdash{}{0pt}%
\pgfpathmoveto{\pgfqpoint{0.721249in}{0.356667in}}%
\pgfpathcurveto{\pgfqpoint{0.726774in}{0.356667in}}{\pgfqpoint{0.732073in}{0.358862in}}{\pgfqpoint{0.735980in}{0.362769in}}%
\pgfpathcurveto{\pgfqpoint{0.739887in}{0.366675in}}{\pgfqpoint{0.742082in}{0.371975in}}{\pgfqpoint{0.742082in}{0.377500in}}%
\pgfpathcurveto{\pgfqpoint{0.742082in}{0.383025in}}{\pgfqpoint{0.739887in}{0.388325in}}{\pgfqpoint{0.735980in}{0.392231in}}%
\pgfpathcurveto{\pgfqpoint{0.732073in}{0.396138in}}{\pgfqpoint{0.726774in}{0.398333in}}{\pgfqpoint{0.721249in}{0.398333in}}%
\pgfpathcurveto{\pgfqpoint{0.715724in}{0.398333in}}{\pgfqpoint{0.710424in}{0.396138in}}{\pgfqpoint{0.706518in}{0.392231in}}%
\pgfpathcurveto{\pgfqpoint{0.702611in}{0.388325in}}{\pgfqpoint{0.700416in}{0.383025in}}{\pgfqpoint{0.700416in}{0.377500in}}%
\pgfpathcurveto{\pgfqpoint{0.700416in}{0.371975in}}{\pgfqpoint{0.702611in}{0.366675in}}{\pgfqpoint{0.706518in}{0.362769in}}%
\pgfpathcurveto{\pgfqpoint{0.710424in}{0.358862in}}{\pgfqpoint{0.715724in}{0.356667in}}{\pgfqpoint{0.721249in}{0.356667in}}%
\pgfpathclose%
\pgfusepath{stroke,fill}%
\end{pgfscope}%
\begin{pgfscope}%
\pgfpathrectangle{\pgfqpoint{0.562500in}{0.275000in}}{\pgfqpoint{3.487500in}{1.925000in}}%
\pgfusepath{clip}%
\pgfsetbuttcap%
\pgfsetroundjoin%
\definecolor{currentfill}{rgb}{0.000000,0.000000,0.000000}%
\pgfsetfillcolor{currentfill}%
\pgfsetlinewidth{1.003750pt}%
\definecolor{currentstroke}{rgb}{0.000000,0.000000,0.000000}%
\pgfsetstrokecolor{currentstroke}%
\pgfsetdash{}{0pt}%
\pgfpathmoveto{\pgfqpoint{0.721249in}{2.076667in}}%
\pgfpathcurveto{\pgfqpoint{0.726774in}{2.076667in}}{\pgfqpoint{0.732073in}{2.078862in}}{\pgfqpoint{0.735980in}{2.082769in}}%
\pgfpathcurveto{\pgfqpoint{0.739887in}{2.086675in}}{\pgfqpoint{0.742082in}{2.091975in}}{\pgfqpoint{0.742082in}{2.097500in}}%
\pgfpathcurveto{\pgfqpoint{0.742082in}{2.103025in}}{\pgfqpoint{0.739887in}{2.108325in}}{\pgfqpoint{0.735980in}{2.112231in}}%
\pgfpathcurveto{\pgfqpoint{0.732073in}{2.116138in}}{\pgfqpoint{0.726774in}{2.118333in}}{\pgfqpoint{0.721249in}{2.118333in}}%
\pgfpathcurveto{\pgfqpoint{0.715724in}{2.118333in}}{\pgfqpoint{0.710424in}{2.116138in}}{\pgfqpoint{0.706518in}{2.112231in}}%
\pgfpathcurveto{\pgfqpoint{0.702611in}{2.108325in}}{\pgfqpoint{0.700416in}{2.103025in}}{\pgfqpoint{0.700416in}{2.097500in}}%
\pgfpathcurveto{\pgfqpoint{0.700416in}{2.091975in}}{\pgfqpoint{0.702611in}{2.086675in}}{\pgfqpoint{0.706518in}{2.082769in}}%
\pgfpathcurveto{\pgfqpoint{0.710424in}{2.078862in}}{\pgfqpoint{0.715724in}{2.076667in}}{\pgfqpoint{0.721249in}{2.076667in}}%
\pgfpathclose%
\pgfusepath{stroke,fill}%
\end{pgfscope}%
\begin{pgfscope}%
\pgfpathrectangle{\pgfqpoint{0.562500in}{0.275000in}}{\pgfqpoint{3.487500in}{1.925000in}}%
\pgfusepath{clip}%
\pgfsetbuttcap%
\pgfsetroundjoin%
\definecolor{currentfill}{rgb}{0.000000,0.000000,0.000000}%
\pgfsetfillcolor{currentfill}%
\pgfsetlinewidth{1.003750pt}%
\definecolor{currentstroke}{rgb}{0.000000,0.000000,0.000000}%
\pgfsetstrokecolor{currentstroke}%
\pgfsetdash{}{0pt}%
\pgfpathmoveto{\pgfqpoint{0.721249in}{0.356667in}}%
\pgfpathcurveto{\pgfqpoint{0.726774in}{0.356667in}}{\pgfqpoint{0.732073in}{0.358862in}}{\pgfqpoint{0.735980in}{0.362769in}}%
\pgfpathcurveto{\pgfqpoint{0.739887in}{0.366675in}}{\pgfqpoint{0.742082in}{0.371975in}}{\pgfqpoint{0.742082in}{0.377500in}}%
\pgfpathcurveto{\pgfqpoint{0.742082in}{0.383025in}}{\pgfqpoint{0.739887in}{0.388325in}}{\pgfqpoint{0.735980in}{0.392231in}}%
\pgfpathcurveto{\pgfqpoint{0.732073in}{0.396138in}}{\pgfqpoint{0.726774in}{0.398333in}}{\pgfqpoint{0.721249in}{0.398333in}}%
\pgfpathcurveto{\pgfqpoint{0.715724in}{0.398333in}}{\pgfqpoint{0.710424in}{0.396138in}}{\pgfqpoint{0.706518in}{0.392231in}}%
\pgfpathcurveto{\pgfqpoint{0.702611in}{0.388325in}}{\pgfqpoint{0.700416in}{0.383025in}}{\pgfqpoint{0.700416in}{0.377500in}}%
\pgfpathcurveto{\pgfqpoint{0.700416in}{0.371975in}}{\pgfqpoint{0.702611in}{0.366675in}}{\pgfqpoint{0.706518in}{0.362769in}}%
\pgfpathcurveto{\pgfqpoint{0.710424in}{0.358862in}}{\pgfqpoint{0.715724in}{0.356667in}}{\pgfqpoint{0.721249in}{0.356667in}}%
\pgfpathclose%
\pgfusepath{stroke,fill}%
\end{pgfscope}%
\begin{pgfscope}%
\pgfpathrectangle{\pgfqpoint{0.562500in}{0.275000in}}{\pgfqpoint{3.487500in}{1.925000in}}%
\pgfusepath{clip}%
\pgfsetbuttcap%
\pgfsetroundjoin%
\definecolor{currentfill}{rgb}{0.000000,0.000000,0.000000}%
\pgfsetfillcolor{currentfill}%
\pgfsetlinewidth{1.003750pt}%
\definecolor{currentstroke}{rgb}{0.000000,0.000000,0.000000}%
\pgfsetstrokecolor{currentstroke}%
\pgfsetdash{}{0pt}%
\pgfpathmoveto{\pgfqpoint{0.721249in}{2.076667in}}%
\pgfpathcurveto{\pgfqpoint{0.726774in}{2.076667in}}{\pgfqpoint{0.732073in}{2.078862in}}{\pgfqpoint{0.735980in}{2.082769in}}%
\pgfpathcurveto{\pgfqpoint{0.739887in}{2.086675in}}{\pgfqpoint{0.742082in}{2.091975in}}{\pgfqpoint{0.742082in}{2.097500in}}%
\pgfpathcurveto{\pgfqpoint{0.742082in}{2.103025in}}{\pgfqpoint{0.739887in}{2.108325in}}{\pgfqpoint{0.735980in}{2.112231in}}%
\pgfpathcurveto{\pgfqpoint{0.732073in}{2.116138in}}{\pgfqpoint{0.726774in}{2.118333in}}{\pgfqpoint{0.721249in}{2.118333in}}%
\pgfpathcurveto{\pgfqpoint{0.715724in}{2.118333in}}{\pgfqpoint{0.710424in}{2.116138in}}{\pgfqpoint{0.706518in}{2.112231in}}%
\pgfpathcurveto{\pgfqpoint{0.702611in}{2.108325in}}{\pgfqpoint{0.700416in}{2.103025in}}{\pgfqpoint{0.700416in}{2.097500in}}%
\pgfpathcurveto{\pgfqpoint{0.700416in}{2.091975in}}{\pgfqpoint{0.702611in}{2.086675in}}{\pgfqpoint{0.706518in}{2.082769in}}%
\pgfpathcurveto{\pgfqpoint{0.710424in}{2.078862in}}{\pgfqpoint{0.715724in}{2.076667in}}{\pgfqpoint{0.721249in}{2.076667in}}%
\pgfpathclose%
\pgfusepath{stroke,fill}%
\end{pgfscope}%
\begin{pgfscope}%
\pgfpathrectangle{\pgfqpoint{0.562500in}{0.275000in}}{\pgfqpoint{3.487500in}{1.925000in}}%
\pgfusepath{clip}%
\pgfsetbuttcap%
\pgfsetroundjoin%
\definecolor{currentfill}{rgb}{0.000000,0.000000,0.000000}%
\pgfsetfillcolor{currentfill}%
\pgfsetlinewidth{1.003750pt}%
\definecolor{currentstroke}{rgb}{0.000000,0.000000,0.000000}%
\pgfsetstrokecolor{currentstroke}%
\pgfsetdash{}{0pt}%
\pgfpathmoveto{\pgfqpoint{0.721249in}{0.356667in}}%
\pgfpathcurveto{\pgfqpoint{0.726774in}{0.356667in}}{\pgfqpoint{0.732073in}{0.358862in}}{\pgfqpoint{0.735980in}{0.362769in}}%
\pgfpathcurveto{\pgfqpoint{0.739887in}{0.366675in}}{\pgfqpoint{0.742082in}{0.371975in}}{\pgfqpoint{0.742082in}{0.377500in}}%
\pgfpathcurveto{\pgfqpoint{0.742082in}{0.383025in}}{\pgfqpoint{0.739887in}{0.388325in}}{\pgfqpoint{0.735980in}{0.392231in}}%
\pgfpathcurveto{\pgfqpoint{0.732073in}{0.396138in}}{\pgfqpoint{0.726774in}{0.398333in}}{\pgfqpoint{0.721249in}{0.398333in}}%
\pgfpathcurveto{\pgfqpoint{0.715724in}{0.398333in}}{\pgfqpoint{0.710424in}{0.396138in}}{\pgfqpoint{0.706518in}{0.392231in}}%
\pgfpathcurveto{\pgfqpoint{0.702611in}{0.388325in}}{\pgfqpoint{0.700416in}{0.383025in}}{\pgfqpoint{0.700416in}{0.377500in}}%
\pgfpathcurveto{\pgfqpoint{0.700416in}{0.371975in}}{\pgfqpoint{0.702611in}{0.366675in}}{\pgfqpoint{0.706518in}{0.362769in}}%
\pgfpathcurveto{\pgfqpoint{0.710424in}{0.358862in}}{\pgfqpoint{0.715724in}{0.356667in}}{\pgfqpoint{0.721249in}{0.356667in}}%
\pgfpathclose%
\pgfusepath{stroke,fill}%
\end{pgfscope}%
\begin{pgfscope}%
\pgfpathrectangle{\pgfqpoint{0.562500in}{0.275000in}}{\pgfqpoint{3.487500in}{1.925000in}}%
\pgfusepath{clip}%
\pgfsetbuttcap%
\pgfsetroundjoin%
\definecolor{currentfill}{rgb}{0.000000,0.000000,0.000000}%
\pgfsetfillcolor{currentfill}%
\pgfsetlinewidth{1.003750pt}%
\definecolor{currentstroke}{rgb}{0.000000,0.000000,0.000000}%
\pgfsetstrokecolor{currentstroke}%
\pgfsetdash{}{0pt}%
\pgfpathmoveto{\pgfqpoint{0.721249in}{0.356667in}}%
\pgfpathcurveto{\pgfqpoint{0.726774in}{0.356667in}}{\pgfqpoint{0.732073in}{0.358862in}}{\pgfqpoint{0.735980in}{0.362769in}}%
\pgfpathcurveto{\pgfqpoint{0.739887in}{0.366675in}}{\pgfqpoint{0.742082in}{0.371975in}}{\pgfqpoint{0.742082in}{0.377500in}}%
\pgfpathcurveto{\pgfqpoint{0.742082in}{0.383025in}}{\pgfqpoint{0.739887in}{0.388325in}}{\pgfqpoint{0.735980in}{0.392231in}}%
\pgfpathcurveto{\pgfqpoint{0.732073in}{0.396138in}}{\pgfqpoint{0.726774in}{0.398333in}}{\pgfqpoint{0.721249in}{0.398333in}}%
\pgfpathcurveto{\pgfqpoint{0.715724in}{0.398333in}}{\pgfqpoint{0.710424in}{0.396138in}}{\pgfqpoint{0.706518in}{0.392231in}}%
\pgfpathcurveto{\pgfqpoint{0.702611in}{0.388325in}}{\pgfqpoint{0.700416in}{0.383025in}}{\pgfqpoint{0.700416in}{0.377500in}}%
\pgfpathcurveto{\pgfqpoint{0.700416in}{0.371975in}}{\pgfqpoint{0.702611in}{0.366675in}}{\pgfqpoint{0.706518in}{0.362769in}}%
\pgfpathcurveto{\pgfqpoint{0.710424in}{0.358862in}}{\pgfqpoint{0.715724in}{0.356667in}}{\pgfqpoint{0.721249in}{0.356667in}}%
\pgfpathclose%
\pgfusepath{stroke,fill}%
\end{pgfscope}%
\begin{pgfscope}%
\pgfpathrectangle{\pgfqpoint{0.562500in}{0.275000in}}{\pgfqpoint{3.487500in}{1.925000in}}%
\pgfusepath{clip}%
\pgfsetbuttcap%
\pgfsetroundjoin%
\definecolor{currentfill}{rgb}{0.000000,0.000000,0.000000}%
\pgfsetfillcolor{currentfill}%
\pgfsetlinewidth{1.003750pt}%
\definecolor{currentstroke}{rgb}{0.000000,0.000000,0.000000}%
\pgfsetstrokecolor{currentstroke}%
\pgfsetdash{}{0pt}%
\pgfpathmoveto{\pgfqpoint{0.721249in}{0.356667in}}%
\pgfpathcurveto{\pgfqpoint{0.726774in}{0.356667in}}{\pgfqpoint{0.732073in}{0.358862in}}{\pgfqpoint{0.735980in}{0.362769in}}%
\pgfpathcurveto{\pgfqpoint{0.739887in}{0.366675in}}{\pgfqpoint{0.742082in}{0.371975in}}{\pgfqpoint{0.742082in}{0.377500in}}%
\pgfpathcurveto{\pgfqpoint{0.742082in}{0.383025in}}{\pgfqpoint{0.739887in}{0.388325in}}{\pgfqpoint{0.735980in}{0.392231in}}%
\pgfpathcurveto{\pgfqpoint{0.732073in}{0.396138in}}{\pgfqpoint{0.726774in}{0.398333in}}{\pgfqpoint{0.721249in}{0.398333in}}%
\pgfpathcurveto{\pgfqpoint{0.715724in}{0.398333in}}{\pgfqpoint{0.710424in}{0.396138in}}{\pgfqpoint{0.706518in}{0.392231in}}%
\pgfpathcurveto{\pgfqpoint{0.702611in}{0.388325in}}{\pgfqpoint{0.700416in}{0.383025in}}{\pgfqpoint{0.700416in}{0.377500in}}%
\pgfpathcurveto{\pgfqpoint{0.700416in}{0.371975in}}{\pgfqpoint{0.702611in}{0.366675in}}{\pgfqpoint{0.706518in}{0.362769in}}%
\pgfpathcurveto{\pgfqpoint{0.710424in}{0.358862in}}{\pgfqpoint{0.715724in}{0.356667in}}{\pgfqpoint{0.721249in}{0.356667in}}%
\pgfpathclose%
\pgfusepath{stroke,fill}%
\end{pgfscope}%
\begin{pgfscope}%
\pgfpathrectangle{\pgfqpoint{0.562500in}{0.275000in}}{\pgfqpoint{3.487500in}{1.925000in}}%
\pgfusepath{clip}%
\pgfsetbuttcap%
\pgfsetroundjoin%
\definecolor{currentfill}{rgb}{0.000000,0.000000,0.000000}%
\pgfsetfillcolor{currentfill}%
\pgfsetlinewidth{1.003750pt}%
\definecolor{currentstroke}{rgb}{0.000000,0.000000,0.000000}%
\pgfsetstrokecolor{currentstroke}%
\pgfsetdash{}{0pt}%
\pgfpathmoveto{\pgfqpoint{0.721249in}{2.076667in}}%
\pgfpathcurveto{\pgfqpoint{0.726774in}{2.076667in}}{\pgfqpoint{0.732073in}{2.078862in}}{\pgfqpoint{0.735980in}{2.082769in}}%
\pgfpathcurveto{\pgfqpoint{0.739887in}{2.086675in}}{\pgfqpoint{0.742082in}{2.091975in}}{\pgfqpoint{0.742082in}{2.097500in}}%
\pgfpathcurveto{\pgfqpoint{0.742082in}{2.103025in}}{\pgfqpoint{0.739887in}{2.108325in}}{\pgfqpoint{0.735980in}{2.112231in}}%
\pgfpathcurveto{\pgfqpoint{0.732073in}{2.116138in}}{\pgfqpoint{0.726774in}{2.118333in}}{\pgfqpoint{0.721249in}{2.118333in}}%
\pgfpathcurveto{\pgfqpoint{0.715724in}{2.118333in}}{\pgfqpoint{0.710424in}{2.116138in}}{\pgfqpoint{0.706518in}{2.112231in}}%
\pgfpathcurveto{\pgfqpoint{0.702611in}{2.108325in}}{\pgfqpoint{0.700416in}{2.103025in}}{\pgfqpoint{0.700416in}{2.097500in}}%
\pgfpathcurveto{\pgfqpoint{0.700416in}{2.091975in}}{\pgfqpoint{0.702611in}{2.086675in}}{\pgfqpoint{0.706518in}{2.082769in}}%
\pgfpathcurveto{\pgfqpoint{0.710424in}{2.078862in}}{\pgfqpoint{0.715724in}{2.076667in}}{\pgfqpoint{0.721249in}{2.076667in}}%
\pgfpathclose%
\pgfusepath{stroke,fill}%
\end{pgfscope}%
\begin{pgfscope}%
\pgfpathrectangle{\pgfqpoint{0.562500in}{0.275000in}}{\pgfqpoint{3.487500in}{1.925000in}}%
\pgfusepath{clip}%
\pgfsetbuttcap%
\pgfsetroundjoin%
\definecolor{currentfill}{rgb}{0.000000,0.000000,0.000000}%
\pgfsetfillcolor{currentfill}%
\pgfsetlinewidth{1.003750pt}%
\definecolor{currentstroke}{rgb}{0.000000,0.000000,0.000000}%
\pgfsetstrokecolor{currentstroke}%
\pgfsetdash{}{0pt}%
\pgfpathmoveto{\pgfqpoint{0.721249in}{0.356667in}}%
\pgfpathcurveto{\pgfqpoint{0.726774in}{0.356667in}}{\pgfqpoint{0.732073in}{0.358862in}}{\pgfqpoint{0.735980in}{0.362769in}}%
\pgfpathcurveto{\pgfqpoint{0.739887in}{0.366675in}}{\pgfqpoint{0.742082in}{0.371975in}}{\pgfqpoint{0.742082in}{0.377500in}}%
\pgfpathcurveto{\pgfqpoint{0.742082in}{0.383025in}}{\pgfqpoint{0.739887in}{0.388325in}}{\pgfqpoint{0.735980in}{0.392231in}}%
\pgfpathcurveto{\pgfqpoint{0.732073in}{0.396138in}}{\pgfqpoint{0.726774in}{0.398333in}}{\pgfqpoint{0.721249in}{0.398333in}}%
\pgfpathcurveto{\pgfqpoint{0.715724in}{0.398333in}}{\pgfqpoint{0.710424in}{0.396138in}}{\pgfqpoint{0.706518in}{0.392231in}}%
\pgfpathcurveto{\pgfqpoint{0.702611in}{0.388325in}}{\pgfqpoint{0.700416in}{0.383025in}}{\pgfqpoint{0.700416in}{0.377500in}}%
\pgfpathcurveto{\pgfqpoint{0.700416in}{0.371975in}}{\pgfqpoint{0.702611in}{0.366675in}}{\pgfqpoint{0.706518in}{0.362769in}}%
\pgfpathcurveto{\pgfqpoint{0.710424in}{0.358862in}}{\pgfqpoint{0.715724in}{0.356667in}}{\pgfqpoint{0.721249in}{0.356667in}}%
\pgfpathclose%
\pgfusepath{stroke,fill}%
\end{pgfscope}%
\begin{pgfscope}%
\pgfpathrectangle{\pgfqpoint{0.562500in}{0.275000in}}{\pgfqpoint{3.487500in}{1.925000in}}%
\pgfusepath{clip}%
\pgfsetbuttcap%
\pgfsetroundjoin%
\definecolor{currentfill}{rgb}{0.000000,0.000000,0.000000}%
\pgfsetfillcolor{currentfill}%
\pgfsetlinewidth{1.003750pt}%
\definecolor{currentstroke}{rgb}{0.000000,0.000000,0.000000}%
\pgfsetstrokecolor{currentstroke}%
\pgfsetdash{}{0pt}%
\pgfpathmoveto{\pgfqpoint{0.721249in}{0.356667in}}%
\pgfpathcurveto{\pgfqpoint{0.726774in}{0.356667in}}{\pgfqpoint{0.732073in}{0.358862in}}{\pgfqpoint{0.735980in}{0.362769in}}%
\pgfpathcurveto{\pgfqpoint{0.739887in}{0.366675in}}{\pgfqpoint{0.742082in}{0.371975in}}{\pgfqpoint{0.742082in}{0.377500in}}%
\pgfpathcurveto{\pgfqpoint{0.742082in}{0.383025in}}{\pgfqpoint{0.739887in}{0.388325in}}{\pgfqpoint{0.735980in}{0.392231in}}%
\pgfpathcurveto{\pgfqpoint{0.732073in}{0.396138in}}{\pgfqpoint{0.726774in}{0.398333in}}{\pgfqpoint{0.721249in}{0.398333in}}%
\pgfpathcurveto{\pgfqpoint{0.715724in}{0.398333in}}{\pgfqpoint{0.710424in}{0.396138in}}{\pgfqpoint{0.706518in}{0.392231in}}%
\pgfpathcurveto{\pgfqpoint{0.702611in}{0.388325in}}{\pgfqpoint{0.700416in}{0.383025in}}{\pgfqpoint{0.700416in}{0.377500in}}%
\pgfpathcurveto{\pgfqpoint{0.700416in}{0.371975in}}{\pgfqpoint{0.702611in}{0.366675in}}{\pgfqpoint{0.706518in}{0.362769in}}%
\pgfpathcurveto{\pgfqpoint{0.710424in}{0.358862in}}{\pgfqpoint{0.715724in}{0.356667in}}{\pgfqpoint{0.721249in}{0.356667in}}%
\pgfpathclose%
\pgfusepath{stroke,fill}%
\end{pgfscope}%
\begin{pgfscope}%
\pgfpathrectangle{\pgfqpoint{0.562500in}{0.275000in}}{\pgfqpoint{3.487500in}{1.925000in}}%
\pgfusepath{clip}%
\pgfsetbuttcap%
\pgfsetroundjoin%
\definecolor{currentfill}{rgb}{0.000000,0.000000,0.000000}%
\pgfsetfillcolor{currentfill}%
\pgfsetlinewidth{1.003750pt}%
\definecolor{currentstroke}{rgb}{0.000000,0.000000,0.000000}%
\pgfsetstrokecolor{currentstroke}%
\pgfsetdash{}{0pt}%
\pgfpathmoveto{\pgfqpoint{0.721249in}{0.356667in}}%
\pgfpathcurveto{\pgfqpoint{0.726774in}{0.356667in}}{\pgfqpoint{0.732073in}{0.358862in}}{\pgfqpoint{0.735980in}{0.362769in}}%
\pgfpathcurveto{\pgfqpoint{0.739887in}{0.366675in}}{\pgfqpoint{0.742082in}{0.371975in}}{\pgfqpoint{0.742082in}{0.377500in}}%
\pgfpathcurveto{\pgfqpoint{0.742082in}{0.383025in}}{\pgfqpoint{0.739887in}{0.388325in}}{\pgfqpoint{0.735980in}{0.392231in}}%
\pgfpathcurveto{\pgfqpoint{0.732073in}{0.396138in}}{\pgfqpoint{0.726774in}{0.398333in}}{\pgfqpoint{0.721249in}{0.398333in}}%
\pgfpathcurveto{\pgfqpoint{0.715724in}{0.398333in}}{\pgfqpoint{0.710424in}{0.396138in}}{\pgfqpoint{0.706518in}{0.392231in}}%
\pgfpathcurveto{\pgfqpoint{0.702611in}{0.388325in}}{\pgfqpoint{0.700416in}{0.383025in}}{\pgfqpoint{0.700416in}{0.377500in}}%
\pgfpathcurveto{\pgfqpoint{0.700416in}{0.371975in}}{\pgfqpoint{0.702611in}{0.366675in}}{\pgfqpoint{0.706518in}{0.362769in}}%
\pgfpathcurveto{\pgfqpoint{0.710424in}{0.358862in}}{\pgfqpoint{0.715724in}{0.356667in}}{\pgfqpoint{0.721249in}{0.356667in}}%
\pgfpathclose%
\pgfusepath{stroke,fill}%
\end{pgfscope}%
\begin{pgfscope}%
\pgfpathrectangle{\pgfqpoint{0.562500in}{0.275000in}}{\pgfqpoint{3.487500in}{1.925000in}}%
\pgfusepath{clip}%
\pgfsetbuttcap%
\pgfsetroundjoin%
\definecolor{currentfill}{rgb}{0.000000,0.000000,0.000000}%
\pgfsetfillcolor{currentfill}%
\pgfsetlinewidth{1.003750pt}%
\definecolor{currentstroke}{rgb}{0.000000,0.000000,0.000000}%
\pgfsetstrokecolor{currentstroke}%
\pgfsetdash{}{0pt}%
\pgfpathmoveto{\pgfqpoint{0.721249in}{0.356667in}}%
\pgfpathcurveto{\pgfqpoint{0.726774in}{0.356667in}}{\pgfqpoint{0.732073in}{0.358862in}}{\pgfqpoint{0.735980in}{0.362769in}}%
\pgfpathcurveto{\pgfqpoint{0.739887in}{0.366675in}}{\pgfqpoint{0.742082in}{0.371975in}}{\pgfqpoint{0.742082in}{0.377500in}}%
\pgfpathcurveto{\pgfqpoint{0.742082in}{0.383025in}}{\pgfqpoint{0.739887in}{0.388325in}}{\pgfqpoint{0.735980in}{0.392231in}}%
\pgfpathcurveto{\pgfqpoint{0.732073in}{0.396138in}}{\pgfqpoint{0.726774in}{0.398333in}}{\pgfqpoint{0.721249in}{0.398333in}}%
\pgfpathcurveto{\pgfqpoint{0.715724in}{0.398333in}}{\pgfqpoint{0.710424in}{0.396138in}}{\pgfqpoint{0.706518in}{0.392231in}}%
\pgfpathcurveto{\pgfqpoint{0.702611in}{0.388325in}}{\pgfqpoint{0.700416in}{0.383025in}}{\pgfqpoint{0.700416in}{0.377500in}}%
\pgfpathcurveto{\pgfqpoint{0.700416in}{0.371975in}}{\pgfqpoint{0.702611in}{0.366675in}}{\pgfqpoint{0.706518in}{0.362769in}}%
\pgfpathcurveto{\pgfqpoint{0.710424in}{0.358862in}}{\pgfqpoint{0.715724in}{0.356667in}}{\pgfqpoint{0.721249in}{0.356667in}}%
\pgfpathclose%
\pgfusepath{stroke,fill}%
\end{pgfscope}%
\begin{pgfscope}%
\pgfpathrectangle{\pgfqpoint{0.562500in}{0.275000in}}{\pgfqpoint{3.487500in}{1.925000in}}%
\pgfusepath{clip}%
\pgfsetbuttcap%
\pgfsetroundjoin%
\definecolor{currentfill}{rgb}{0.000000,0.000000,0.000000}%
\pgfsetfillcolor{currentfill}%
\pgfsetlinewidth{1.003750pt}%
\definecolor{currentstroke}{rgb}{0.000000,0.000000,0.000000}%
\pgfsetstrokecolor{currentstroke}%
\pgfsetdash{}{0pt}%
\pgfpathmoveto{\pgfqpoint{0.721249in}{0.356667in}}%
\pgfpathcurveto{\pgfqpoint{0.726774in}{0.356667in}}{\pgfqpoint{0.732073in}{0.358862in}}{\pgfqpoint{0.735980in}{0.362769in}}%
\pgfpathcurveto{\pgfqpoint{0.739887in}{0.366675in}}{\pgfqpoint{0.742082in}{0.371975in}}{\pgfqpoint{0.742082in}{0.377500in}}%
\pgfpathcurveto{\pgfqpoint{0.742082in}{0.383025in}}{\pgfqpoint{0.739887in}{0.388325in}}{\pgfqpoint{0.735980in}{0.392231in}}%
\pgfpathcurveto{\pgfqpoint{0.732073in}{0.396138in}}{\pgfqpoint{0.726774in}{0.398333in}}{\pgfqpoint{0.721249in}{0.398333in}}%
\pgfpathcurveto{\pgfqpoint{0.715724in}{0.398333in}}{\pgfqpoint{0.710424in}{0.396138in}}{\pgfqpoint{0.706518in}{0.392231in}}%
\pgfpathcurveto{\pgfqpoint{0.702611in}{0.388325in}}{\pgfqpoint{0.700416in}{0.383025in}}{\pgfqpoint{0.700416in}{0.377500in}}%
\pgfpathcurveto{\pgfqpoint{0.700416in}{0.371975in}}{\pgfqpoint{0.702611in}{0.366675in}}{\pgfqpoint{0.706518in}{0.362769in}}%
\pgfpathcurveto{\pgfqpoint{0.710424in}{0.358862in}}{\pgfqpoint{0.715724in}{0.356667in}}{\pgfqpoint{0.721249in}{0.356667in}}%
\pgfpathclose%
\pgfusepath{stroke,fill}%
\end{pgfscope}%
\begin{pgfscope}%
\pgfpathrectangle{\pgfqpoint{0.562500in}{0.275000in}}{\pgfqpoint{3.487500in}{1.925000in}}%
\pgfusepath{clip}%
\pgfsetbuttcap%
\pgfsetroundjoin%
\definecolor{currentfill}{rgb}{0.000000,0.000000,0.000000}%
\pgfsetfillcolor{currentfill}%
\pgfsetlinewidth{1.003750pt}%
\definecolor{currentstroke}{rgb}{0.000000,0.000000,0.000000}%
\pgfsetstrokecolor{currentstroke}%
\pgfsetdash{}{0pt}%
\pgfpathmoveto{\pgfqpoint{0.721249in}{0.356667in}}%
\pgfpathcurveto{\pgfqpoint{0.726774in}{0.356667in}}{\pgfqpoint{0.732073in}{0.358862in}}{\pgfqpoint{0.735980in}{0.362769in}}%
\pgfpathcurveto{\pgfqpoint{0.739887in}{0.366675in}}{\pgfqpoint{0.742082in}{0.371975in}}{\pgfqpoint{0.742082in}{0.377500in}}%
\pgfpathcurveto{\pgfqpoint{0.742082in}{0.383025in}}{\pgfqpoint{0.739887in}{0.388325in}}{\pgfqpoint{0.735980in}{0.392231in}}%
\pgfpathcurveto{\pgfqpoint{0.732073in}{0.396138in}}{\pgfqpoint{0.726774in}{0.398333in}}{\pgfqpoint{0.721249in}{0.398333in}}%
\pgfpathcurveto{\pgfqpoint{0.715724in}{0.398333in}}{\pgfqpoint{0.710424in}{0.396138in}}{\pgfqpoint{0.706518in}{0.392231in}}%
\pgfpathcurveto{\pgfqpoint{0.702611in}{0.388325in}}{\pgfqpoint{0.700416in}{0.383025in}}{\pgfqpoint{0.700416in}{0.377500in}}%
\pgfpathcurveto{\pgfqpoint{0.700416in}{0.371975in}}{\pgfqpoint{0.702611in}{0.366675in}}{\pgfqpoint{0.706518in}{0.362769in}}%
\pgfpathcurveto{\pgfqpoint{0.710424in}{0.358862in}}{\pgfqpoint{0.715724in}{0.356667in}}{\pgfqpoint{0.721249in}{0.356667in}}%
\pgfpathclose%
\pgfusepath{stroke,fill}%
\end{pgfscope}%
\begin{pgfscope}%
\pgfpathrectangle{\pgfqpoint{0.562500in}{0.275000in}}{\pgfqpoint{3.487500in}{1.925000in}}%
\pgfusepath{clip}%
\pgfsetbuttcap%
\pgfsetroundjoin%
\definecolor{currentfill}{rgb}{0.000000,0.000000,0.000000}%
\pgfsetfillcolor{currentfill}%
\pgfsetlinewidth{1.003750pt}%
\definecolor{currentstroke}{rgb}{0.000000,0.000000,0.000000}%
\pgfsetstrokecolor{currentstroke}%
\pgfsetdash{}{0pt}%
\pgfpathmoveto{\pgfqpoint{0.721249in}{0.356667in}}%
\pgfpathcurveto{\pgfqpoint{0.726774in}{0.356667in}}{\pgfqpoint{0.732073in}{0.358862in}}{\pgfqpoint{0.735980in}{0.362769in}}%
\pgfpathcurveto{\pgfqpoint{0.739887in}{0.366675in}}{\pgfqpoint{0.742082in}{0.371975in}}{\pgfqpoint{0.742082in}{0.377500in}}%
\pgfpathcurveto{\pgfqpoint{0.742082in}{0.383025in}}{\pgfqpoint{0.739887in}{0.388325in}}{\pgfqpoint{0.735980in}{0.392231in}}%
\pgfpathcurveto{\pgfqpoint{0.732073in}{0.396138in}}{\pgfqpoint{0.726774in}{0.398333in}}{\pgfqpoint{0.721249in}{0.398333in}}%
\pgfpathcurveto{\pgfqpoint{0.715724in}{0.398333in}}{\pgfqpoint{0.710424in}{0.396138in}}{\pgfqpoint{0.706518in}{0.392231in}}%
\pgfpathcurveto{\pgfqpoint{0.702611in}{0.388325in}}{\pgfqpoint{0.700416in}{0.383025in}}{\pgfqpoint{0.700416in}{0.377500in}}%
\pgfpathcurveto{\pgfqpoint{0.700416in}{0.371975in}}{\pgfqpoint{0.702611in}{0.366675in}}{\pgfqpoint{0.706518in}{0.362769in}}%
\pgfpathcurveto{\pgfqpoint{0.710424in}{0.358862in}}{\pgfqpoint{0.715724in}{0.356667in}}{\pgfqpoint{0.721249in}{0.356667in}}%
\pgfpathclose%
\pgfusepath{stroke,fill}%
\end{pgfscope}%
\begin{pgfscope}%
\pgfpathrectangle{\pgfqpoint{0.562500in}{0.275000in}}{\pgfqpoint{3.487500in}{1.925000in}}%
\pgfusepath{clip}%
\pgfsetbuttcap%
\pgfsetroundjoin%
\definecolor{currentfill}{rgb}{0.000000,0.000000,0.000000}%
\pgfsetfillcolor{currentfill}%
\pgfsetlinewidth{1.003750pt}%
\definecolor{currentstroke}{rgb}{0.000000,0.000000,0.000000}%
\pgfsetstrokecolor{currentstroke}%
\pgfsetdash{}{0pt}%
\pgfpathmoveto{\pgfqpoint{0.721249in}{0.356667in}}%
\pgfpathcurveto{\pgfqpoint{0.726774in}{0.356667in}}{\pgfqpoint{0.732073in}{0.358862in}}{\pgfqpoint{0.735980in}{0.362769in}}%
\pgfpathcurveto{\pgfqpoint{0.739887in}{0.366675in}}{\pgfqpoint{0.742082in}{0.371975in}}{\pgfqpoint{0.742082in}{0.377500in}}%
\pgfpathcurveto{\pgfqpoint{0.742082in}{0.383025in}}{\pgfqpoint{0.739887in}{0.388325in}}{\pgfqpoint{0.735980in}{0.392231in}}%
\pgfpathcurveto{\pgfqpoint{0.732073in}{0.396138in}}{\pgfqpoint{0.726774in}{0.398333in}}{\pgfqpoint{0.721249in}{0.398333in}}%
\pgfpathcurveto{\pgfqpoint{0.715724in}{0.398333in}}{\pgfqpoint{0.710424in}{0.396138in}}{\pgfqpoint{0.706518in}{0.392231in}}%
\pgfpathcurveto{\pgfqpoint{0.702611in}{0.388325in}}{\pgfqpoint{0.700416in}{0.383025in}}{\pgfqpoint{0.700416in}{0.377500in}}%
\pgfpathcurveto{\pgfqpoint{0.700416in}{0.371975in}}{\pgfqpoint{0.702611in}{0.366675in}}{\pgfqpoint{0.706518in}{0.362769in}}%
\pgfpathcurveto{\pgfqpoint{0.710424in}{0.358862in}}{\pgfqpoint{0.715724in}{0.356667in}}{\pgfqpoint{0.721249in}{0.356667in}}%
\pgfpathclose%
\pgfusepath{stroke,fill}%
\end{pgfscope}%
\begin{pgfscope}%
\pgfpathrectangle{\pgfqpoint{0.562500in}{0.275000in}}{\pgfqpoint{3.487500in}{1.925000in}}%
\pgfusepath{clip}%
\pgfsetbuttcap%
\pgfsetroundjoin%
\definecolor{currentfill}{rgb}{0.000000,0.000000,0.000000}%
\pgfsetfillcolor{currentfill}%
\pgfsetlinewidth{1.003750pt}%
\definecolor{currentstroke}{rgb}{0.000000,0.000000,0.000000}%
\pgfsetstrokecolor{currentstroke}%
\pgfsetdash{}{0pt}%
\pgfpathmoveto{\pgfqpoint{0.721249in}{0.356667in}}%
\pgfpathcurveto{\pgfqpoint{0.726774in}{0.356667in}}{\pgfqpoint{0.732073in}{0.358862in}}{\pgfqpoint{0.735980in}{0.362769in}}%
\pgfpathcurveto{\pgfqpoint{0.739887in}{0.366675in}}{\pgfqpoint{0.742082in}{0.371975in}}{\pgfqpoint{0.742082in}{0.377500in}}%
\pgfpathcurveto{\pgfqpoint{0.742082in}{0.383025in}}{\pgfqpoint{0.739887in}{0.388325in}}{\pgfqpoint{0.735980in}{0.392231in}}%
\pgfpathcurveto{\pgfqpoint{0.732073in}{0.396138in}}{\pgfqpoint{0.726774in}{0.398333in}}{\pgfqpoint{0.721249in}{0.398333in}}%
\pgfpathcurveto{\pgfqpoint{0.715724in}{0.398333in}}{\pgfqpoint{0.710424in}{0.396138in}}{\pgfqpoint{0.706518in}{0.392231in}}%
\pgfpathcurveto{\pgfqpoint{0.702611in}{0.388325in}}{\pgfqpoint{0.700416in}{0.383025in}}{\pgfqpoint{0.700416in}{0.377500in}}%
\pgfpathcurveto{\pgfqpoint{0.700416in}{0.371975in}}{\pgfqpoint{0.702611in}{0.366675in}}{\pgfqpoint{0.706518in}{0.362769in}}%
\pgfpathcurveto{\pgfqpoint{0.710424in}{0.358862in}}{\pgfqpoint{0.715724in}{0.356667in}}{\pgfqpoint{0.721249in}{0.356667in}}%
\pgfpathclose%
\pgfusepath{stroke,fill}%
\end{pgfscope}%
\begin{pgfscope}%
\pgfpathrectangle{\pgfqpoint{0.562500in}{0.275000in}}{\pgfqpoint{3.487500in}{1.925000in}}%
\pgfusepath{clip}%
\pgfsetbuttcap%
\pgfsetroundjoin%
\definecolor{currentfill}{rgb}{0.000000,0.000000,0.000000}%
\pgfsetfillcolor{currentfill}%
\pgfsetlinewidth{1.003750pt}%
\definecolor{currentstroke}{rgb}{0.000000,0.000000,0.000000}%
\pgfsetstrokecolor{currentstroke}%
\pgfsetdash{}{0pt}%
\pgfpathmoveto{\pgfqpoint{0.721249in}{0.356667in}}%
\pgfpathcurveto{\pgfqpoint{0.726774in}{0.356667in}}{\pgfqpoint{0.732073in}{0.358862in}}{\pgfqpoint{0.735980in}{0.362769in}}%
\pgfpathcurveto{\pgfqpoint{0.739887in}{0.366675in}}{\pgfqpoint{0.742082in}{0.371975in}}{\pgfqpoint{0.742082in}{0.377500in}}%
\pgfpathcurveto{\pgfqpoint{0.742082in}{0.383025in}}{\pgfqpoint{0.739887in}{0.388325in}}{\pgfqpoint{0.735980in}{0.392231in}}%
\pgfpathcurveto{\pgfqpoint{0.732073in}{0.396138in}}{\pgfqpoint{0.726774in}{0.398333in}}{\pgfqpoint{0.721249in}{0.398333in}}%
\pgfpathcurveto{\pgfqpoint{0.715724in}{0.398333in}}{\pgfqpoint{0.710424in}{0.396138in}}{\pgfqpoint{0.706518in}{0.392231in}}%
\pgfpathcurveto{\pgfqpoint{0.702611in}{0.388325in}}{\pgfqpoint{0.700416in}{0.383025in}}{\pgfqpoint{0.700416in}{0.377500in}}%
\pgfpathcurveto{\pgfqpoint{0.700416in}{0.371975in}}{\pgfqpoint{0.702611in}{0.366675in}}{\pgfqpoint{0.706518in}{0.362769in}}%
\pgfpathcurveto{\pgfqpoint{0.710424in}{0.358862in}}{\pgfqpoint{0.715724in}{0.356667in}}{\pgfqpoint{0.721249in}{0.356667in}}%
\pgfpathclose%
\pgfusepath{stroke,fill}%
\end{pgfscope}%
\begin{pgfscope}%
\pgfpathrectangle{\pgfqpoint{0.562500in}{0.275000in}}{\pgfqpoint{3.487500in}{1.925000in}}%
\pgfusepath{clip}%
\pgfsetbuttcap%
\pgfsetroundjoin%
\definecolor{currentfill}{rgb}{0.000000,0.000000,0.000000}%
\pgfsetfillcolor{currentfill}%
\pgfsetlinewidth{1.003750pt}%
\definecolor{currentstroke}{rgb}{0.000000,0.000000,0.000000}%
\pgfsetstrokecolor{currentstroke}%
\pgfsetdash{}{0pt}%
\pgfpathmoveto{\pgfqpoint{0.721249in}{0.356667in}}%
\pgfpathcurveto{\pgfqpoint{0.726774in}{0.356667in}}{\pgfqpoint{0.732073in}{0.358862in}}{\pgfqpoint{0.735980in}{0.362769in}}%
\pgfpathcurveto{\pgfqpoint{0.739887in}{0.366675in}}{\pgfqpoint{0.742082in}{0.371975in}}{\pgfqpoint{0.742082in}{0.377500in}}%
\pgfpathcurveto{\pgfqpoint{0.742082in}{0.383025in}}{\pgfqpoint{0.739887in}{0.388325in}}{\pgfqpoint{0.735980in}{0.392231in}}%
\pgfpathcurveto{\pgfqpoint{0.732073in}{0.396138in}}{\pgfqpoint{0.726774in}{0.398333in}}{\pgfqpoint{0.721249in}{0.398333in}}%
\pgfpathcurveto{\pgfqpoint{0.715724in}{0.398333in}}{\pgfqpoint{0.710424in}{0.396138in}}{\pgfqpoint{0.706518in}{0.392231in}}%
\pgfpathcurveto{\pgfqpoint{0.702611in}{0.388325in}}{\pgfqpoint{0.700416in}{0.383025in}}{\pgfqpoint{0.700416in}{0.377500in}}%
\pgfpathcurveto{\pgfqpoint{0.700416in}{0.371975in}}{\pgfqpoint{0.702611in}{0.366675in}}{\pgfqpoint{0.706518in}{0.362769in}}%
\pgfpathcurveto{\pgfqpoint{0.710424in}{0.358862in}}{\pgfqpoint{0.715724in}{0.356667in}}{\pgfqpoint{0.721249in}{0.356667in}}%
\pgfpathclose%
\pgfusepath{stroke,fill}%
\end{pgfscope}%
\begin{pgfscope}%
\pgfpathrectangle{\pgfqpoint{0.562500in}{0.275000in}}{\pgfqpoint{3.487500in}{1.925000in}}%
\pgfusepath{clip}%
\pgfsetbuttcap%
\pgfsetroundjoin%
\definecolor{currentfill}{rgb}{0.000000,0.000000,0.000000}%
\pgfsetfillcolor{currentfill}%
\pgfsetlinewidth{1.003750pt}%
\definecolor{currentstroke}{rgb}{0.000000,0.000000,0.000000}%
\pgfsetstrokecolor{currentstroke}%
\pgfsetdash{}{0pt}%
\pgfpathmoveto{\pgfqpoint{0.721249in}{0.356667in}}%
\pgfpathcurveto{\pgfqpoint{0.726774in}{0.356667in}}{\pgfqpoint{0.732073in}{0.358862in}}{\pgfqpoint{0.735980in}{0.362769in}}%
\pgfpathcurveto{\pgfqpoint{0.739887in}{0.366675in}}{\pgfqpoint{0.742082in}{0.371975in}}{\pgfqpoint{0.742082in}{0.377500in}}%
\pgfpathcurveto{\pgfqpoint{0.742082in}{0.383025in}}{\pgfqpoint{0.739887in}{0.388325in}}{\pgfqpoint{0.735980in}{0.392231in}}%
\pgfpathcurveto{\pgfqpoint{0.732073in}{0.396138in}}{\pgfqpoint{0.726774in}{0.398333in}}{\pgfqpoint{0.721249in}{0.398333in}}%
\pgfpathcurveto{\pgfqpoint{0.715724in}{0.398333in}}{\pgfqpoint{0.710424in}{0.396138in}}{\pgfqpoint{0.706518in}{0.392231in}}%
\pgfpathcurveto{\pgfqpoint{0.702611in}{0.388325in}}{\pgfqpoint{0.700416in}{0.383025in}}{\pgfqpoint{0.700416in}{0.377500in}}%
\pgfpathcurveto{\pgfqpoint{0.700416in}{0.371975in}}{\pgfqpoint{0.702611in}{0.366675in}}{\pgfqpoint{0.706518in}{0.362769in}}%
\pgfpathcurveto{\pgfqpoint{0.710424in}{0.358862in}}{\pgfqpoint{0.715724in}{0.356667in}}{\pgfqpoint{0.721249in}{0.356667in}}%
\pgfpathclose%
\pgfusepath{stroke,fill}%
\end{pgfscope}%
\begin{pgfscope}%
\pgfpathrectangle{\pgfqpoint{0.562500in}{0.275000in}}{\pgfqpoint{3.487500in}{1.925000in}}%
\pgfusepath{clip}%
\pgfsetbuttcap%
\pgfsetroundjoin%
\definecolor{currentfill}{rgb}{0.000000,0.000000,0.000000}%
\pgfsetfillcolor{currentfill}%
\pgfsetlinewidth{1.003750pt}%
\definecolor{currentstroke}{rgb}{0.000000,0.000000,0.000000}%
\pgfsetstrokecolor{currentstroke}%
\pgfsetdash{}{0pt}%
\pgfpathmoveto{\pgfqpoint{0.721249in}{0.356667in}}%
\pgfpathcurveto{\pgfqpoint{0.726774in}{0.356667in}}{\pgfqpoint{0.732073in}{0.358862in}}{\pgfqpoint{0.735980in}{0.362769in}}%
\pgfpathcurveto{\pgfqpoint{0.739887in}{0.366675in}}{\pgfqpoint{0.742082in}{0.371975in}}{\pgfqpoint{0.742082in}{0.377500in}}%
\pgfpathcurveto{\pgfqpoint{0.742082in}{0.383025in}}{\pgfqpoint{0.739887in}{0.388325in}}{\pgfqpoint{0.735980in}{0.392231in}}%
\pgfpathcurveto{\pgfqpoint{0.732073in}{0.396138in}}{\pgfqpoint{0.726774in}{0.398333in}}{\pgfqpoint{0.721249in}{0.398333in}}%
\pgfpathcurveto{\pgfqpoint{0.715724in}{0.398333in}}{\pgfqpoint{0.710424in}{0.396138in}}{\pgfqpoint{0.706518in}{0.392231in}}%
\pgfpathcurveto{\pgfqpoint{0.702611in}{0.388325in}}{\pgfqpoint{0.700416in}{0.383025in}}{\pgfqpoint{0.700416in}{0.377500in}}%
\pgfpathcurveto{\pgfqpoint{0.700416in}{0.371975in}}{\pgfqpoint{0.702611in}{0.366675in}}{\pgfqpoint{0.706518in}{0.362769in}}%
\pgfpathcurveto{\pgfqpoint{0.710424in}{0.358862in}}{\pgfqpoint{0.715724in}{0.356667in}}{\pgfqpoint{0.721249in}{0.356667in}}%
\pgfpathclose%
\pgfusepath{stroke,fill}%
\end{pgfscope}%
\begin{pgfscope}%
\pgfpathrectangle{\pgfqpoint{0.562500in}{0.275000in}}{\pgfqpoint{3.487500in}{1.925000in}}%
\pgfusepath{clip}%
\pgfsetbuttcap%
\pgfsetroundjoin%
\definecolor{currentfill}{rgb}{0.000000,0.000000,0.000000}%
\pgfsetfillcolor{currentfill}%
\pgfsetlinewidth{1.003750pt}%
\definecolor{currentstroke}{rgb}{0.000000,0.000000,0.000000}%
\pgfsetstrokecolor{currentstroke}%
\pgfsetdash{}{0pt}%
\pgfpathmoveto{\pgfqpoint{0.721249in}{0.356667in}}%
\pgfpathcurveto{\pgfqpoint{0.726774in}{0.356667in}}{\pgfqpoint{0.732073in}{0.358862in}}{\pgfqpoint{0.735980in}{0.362769in}}%
\pgfpathcurveto{\pgfqpoint{0.739887in}{0.366675in}}{\pgfqpoint{0.742082in}{0.371975in}}{\pgfqpoint{0.742082in}{0.377500in}}%
\pgfpathcurveto{\pgfqpoint{0.742082in}{0.383025in}}{\pgfqpoint{0.739887in}{0.388325in}}{\pgfqpoint{0.735980in}{0.392231in}}%
\pgfpathcurveto{\pgfqpoint{0.732073in}{0.396138in}}{\pgfqpoint{0.726774in}{0.398333in}}{\pgfqpoint{0.721249in}{0.398333in}}%
\pgfpathcurveto{\pgfqpoint{0.715724in}{0.398333in}}{\pgfqpoint{0.710424in}{0.396138in}}{\pgfqpoint{0.706518in}{0.392231in}}%
\pgfpathcurveto{\pgfqpoint{0.702611in}{0.388325in}}{\pgfqpoint{0.700416in}{0.383025in}}{\pgfqpoint{0.700416in}{0.377500in}}%
\pgfpathcurveto{\pgfqpoint{0.700416in}{0.371975in}}{\pgfqpoint{0.702611in}{0.366675in}}{\pgfqpoint{0.706518in}{0.362769in}}%
\pgfpathcurveto{\pgfqpoint{0.710424in}{0.358862in}}{\pgfqpoint{0.715724in}{0.356667in}}{\pgfqpoint{0.721249in}{0.356667in}}%
\pgfpathclose%
\pgfusepath{stroke,fill}%
\end{pgfscope}%
\begin{pgfscope}%
\pgfpathrectangle{\pgfqpoint{0.562500in}{0.275000in}}{\pgfqpoint{3.487500in}{1.925000in}}%
\pgfusepath{clip}%
\pgfsetbuttcap%
\pgfsetroundjoin%
\definecolor{currentfill}{rgb}{0.000000,0.000000,0.000000}%
\pgfsetfillcolor{currentfill}%
\pgfsetlinewidth{1.003750pt}%
\definecolor{currentstroke}{rgb}{0.000000,0.000000,0.000000}%
\pgfsetstrokecolor{currentstroke}%
\pgfsetdash{}{0pt}%
\pgfpathmoveto{\pgfqpoint{0.721249in}{0.356667in}}%
\pgfpathcurveto{\pgfqpoint{0.726774in}{0.356667in}}{\pgfqpoint{0.732073in}{0.358862in}}{\pgfqpoint{0.735980in}{0.362769in}}%
\pgfpathcurveto{\pgfqpoint{0.739887in}{0.366675in}}{\pgfqpoint{0.742082in}{0.371975in}}{\pgfqpoint{0.742082in}{0.377500in}}%
\pgfpathcurveto{\pgfqpoint{0.742082in}{0.383025in}}{\pgfqpoint{0.739887in}{0.388325in}}{\pgfqpoint{0.735980in}{0.392231in}}%
\pgfpathcurveto{\pgfqpoint{0.732073in}{0.396138in}}{\pgfqpoint{0.726774in}{0.398333in}}{\pgfqpoint{0.721249in}{0.398333in}}%
\pgfpathcurveto{\pgfqpoint{0.715724in}{0.398333in}}{\pgfqpoint{0.710424in}{0.396138in}}{\pgfqpoint{0.706518in}{0.392231in}}%
\pgfpathcurveto{\pgfqpoint{0.702611in}{0.388325in}}{\pgfqpoint{0.700416in}{0.383025in}}{\pgfqpoint{0.700416in}{0.377500in}}%
\pgfpathcurveto{\pgfqpoint{0.700416in}{0.371975in}}{\pgfqpoint{0.702611in}{0.366675in}}{\pgfqpoint{0.706518in}{0.362769in}}%
\pgfpathcurveto{\pgfqpoint{0.710424in}{0.358862in}}{\pgfqpoint{0.715724in}{0.356667in}}{\pgfqpoint{0.721249in}{0.356667in}}%
\pgfpathclose%
\pgfusepath{stroke,fill}%
\end{pgfscope}%
\begin{pgfscope}%
\pgfpathrectangle{\pgfqpoint{0.562500in}{0.275000in}}{\pgfqpoint{3.487500in}{1.925000in}}%
\pgfusepath{clip}%
\pgfsetbuttcap%
\pgfsetroundjoin%
\definecolor{currentfill}{rgb}{0.000000,0.000000,0.000000}%
\pgfsetfillcolor{currentfill}%
\pgfsetlinewidth{1.003750pt}%
\definecolor{currentstroke}{rgb}{0.000000,0.000000,0.000000}%
\pgfsetstrokecolor{currentstroke}%
\pgfsetdash{}{0pt}%
\pgfpathmoveto{\pgfqpoint{0.721249in}{0.356667in}}%
\pgfpathcurveto{\pgfqpoint{0.726774in}{0.356667in}}{\pgfqpoint{0.732073in}{0.358862in}}{\pgfqpoint{0.735980in}{0.362769in}}%
\pgfpathcurveto{\pgfqpoint{0.739887in}{0.366675in}}{\pgfqpoint{0.742082in}{0.371975in}}{\pgfqpoint{0.742082in}{0.377500in}}%
\pgfpathcurveto{\pgfqpoint{0.742082in}{0.383025in}}{\pgfqpoint{0.739887in}{0.388325in}}{\pgfqpoint{0.735980in}{0.392231in}}%
\pgfpathcurveto{\pgfqpoint{0.732073in}{0.396138in}}{\pgfqpoint{0.726774in}{0.398333in}}{\pgfqpoint{0.721249in}{0.398333in}}%
\pgfpathcurveto{\pgfqpoint{0.715724in}{0.398333in}}{\pgfqpoint{0.710424in}{0.396138in}}{\pgfqpoint{0.706518in}{0.392231in}}%
\pgfpathcurveto{\pgfqpoint{0.702611in}{0.388325in}}{\pgfqpoint{0.700416in}{0.383025in}}{\pgfqpoint{0.700416in}{0.377500in}}%
\pgfpathcurveto{\pgfqpoint{0.700416in}{0.371975in}}{\pgfqpoint{0.702611in}{0.366675in}}{\pgfqpoint{0.706518in}{0.362769in}}%
\pgfpathcurveto{\pgfqpoint{0.710424in}{0.358862in}}{\pgfqpoint{0.715724in}{0.356667in}}{\pgfqpoint{0.721249in}{0.356667in}}%
\pgfpathclose%
\pgfusepath{stroke,fill}%
\end{pgfscope}%
\begin{pgfscope}%
\pgfpathrectangle{\pgfqpoint{0.562500in}{0.275000in}}{\pgfqpoint{3.487500in}{1.925000in}}%
\pgfusepath{clip}%
\pgfsetbuttcap%
\pgfsetroundjoin%
\definecolor{currentfill}{rgb}{0.000000,0.000000,0.000000}%
\pgfsetfillcolor{currentfill}%
\pgfsetlinewidth{1.003750pt}%
\definecolor{currentstroke}{rgb}{0.000000,0.000000,0.000000}%
\pgfsetstrokecolor{currentstroke}%
\pgfsetdash{}{0pt}%
\pgfpathmoveto{\pgfqpoint{0.721249in}{0.356667in}}%
\pgfpathcurveto{\pgfqpoint{0.726774in}{0.356667in}}{\pgfqpoint{0.732073in}{0.358862in}}{\pgfqpoint{0.735980in}{0.362769in}}%
\pgfpathcurveto{\pgfqpoint{0.739887in}{0.366675in}}{\pgfqpoint{0.742082in}{0.371975in}}{\pgfqpoint{0.742082in}{0.377500in}}%
\pgfpathcurveto{\pgfqpoint{0.742082in}{0.383025in}}{\pgfqpoint{0.739887in}{0.388325in}}{\pgfqpoint{0.735980in}{0.392231in}}%
\pgfpathcurveto{\pgfqpoint{0.732073in}{0.396138in}}{\pgfqpoint{0.726774in}{0.398333in}}{\pgfqpoint{0.721249in}{0.398333in}}%
\pgfpathcurveto{\pgfqpoint{0.715724in}{0.398333in}}{\pgfqpoint{0.710424in}{0.396138in}}{\pgfqpoint{0.706518in}{0.392231in}}%
\pgfpathcurveto{\pgfqpoint{0.702611in}{0.388325in}}{\pgfqpoint{0.700416in}{0.383025in}}{\pgfqpoint{0.700416in}{0.377500in}}%
\pgfpathcurveto{\pgfqpoint{0.700416in}{0.371975in}}{\pgfqpoint{0.702611in}{0.366675in}}{\pgfqpoint{0.706518in}{0.362769in}}%
\pgfpathcurveto{\pgfqpoint{0.710424in}{0.358862in}}{\pgfqpoint{0.715724in}{0.356667in}}{\pgfqpoint{0.721249in}{0.356667in}}%
\pgfpathclose%
\pgfusepath{stroke,fill}%
\end{pgfscope}%
\begin{pgfscope}%
\pgfpathrectangle{\pgfqpoint{0.562500in}{0.275000in}}{\pgfqpoint{3.487500in}{1.925000in}}%
\pgfusepath{clip}%
\pgfsetbuttcap%
\pgfsetroundjoin%
\definecolor{currentfill}{rgb}{0.000000,0.000000,0.000000}%
\pgfsetfillcolor{currentfill}%
\pgfsetlinewidth{1.003750pt}%
\definecolor{currentstroke}{rgb}{0.000000,0.000000,0.000000}%
\pgfsetstrokecolor{currentstroke}%
\pgfsetdash{}{0pt}%
\pgfpathmoveto{\pgfqpoint{0.721249in}{0.356667in}}%
\pgfpathcurveto{\pgfqpoint{0.726774in}{0.356667in}}{\pgfqpoint{0.732073in}{0.358862in}}{\pgfqpoint{0.735980in}{0.362769in}}%
\pgfpathcurveto{\pgfqpoint{0.739887in}{0.366675in}}{\pgfqpoint{0.742082in}{0.371975in}}{\pgfqpoint{0.742082in}{0.377500in}}%
\pgfpathcurveto{\pgfqpoint{0.742082in}{0.383025in}}{\pgfqpoint{0.739887in}{0.388325in}}{\pgfqpoint{0.735980in}{0.392231in}}%
\pgfpathcurveto{\pgfqpoint{0.732073in}{0.396138in}}{\pgfqpoint{0.726774in}{0.398333in}}{\pgfqpoint{0.721249in}{0.398333in}}%
\pgfpathcurveto{\pgfqpoint{0.715724in}{0.398333in}}{\pgfqpoint{0.710424in}{0.396138in}}{\pgfqpoint{0.706518in}{0.392231in}}%
\pgfpathcurveto{\pgfqpoint{0.702611in}{0.388325in}}{\pgfqpoint{0.700416in}{0.383025in}}{\pgfqpoint{0.700416in}{0.377500in}}%
\pgfpathcurveto{\pgfqpoint{0.700416in}{0.371975in}}{\pgfqpoint{0.702611in}{0.366675in}}{\pgfqpoint{0.706518in}{0.362769in}}%
\pgfpathcurveto{\pgfqpoint{0.710424in}{0.358862in}}{\pgfqpoint{0.715724in}{0.356667in}}{\pgfqpoint{0.721249in}{0.356667in}}%
\pgfpathclose%
\pgfusepath{stroke,fill}%
\end{pgfscope}%
\begin{pgfscope}%
\pgfpathrectangle{\pgfqpoint{0.562500in}{0.275000in}}{\pgfqpoint{3.487500in}{1.925000in}}%
\pgfusepath{clip}%
\pgfsetbuttcap%
\pgfsetroundjoin%
\definecolor{currentfill}{rgb}{0.000000,0.000000,0.000000}%
\pgfsetfillcolor{currentfill}%
\pgfsetlinewidth{1.003750pt}%
\definecolor{currentstroke}{rgb}{0.000000,0.000000,0.000000}%
\pgfsetstrokecolor{currentstroke}%
\pgfsetdash{}{0pt}%
\pgfpathmoveto{\pgfqpoint{0.721249in}{0.356667in}}%
\pgfpathcurveto{\pgfqpoint{0.726774in}{0.356667in}}{\pgfqpoint{0.732073in}{0.358862in}}{\pgfqpoint{0.735980in}{0.362769in}}%
\pgfpathcurveto{\pgfqpoint{0.739887in}{0.366675in}}{\pgfqpoint{0.742082in}{0.371975in}}{\pgfqpoint{0.742082in}{0.377500in}}%
\pgfpathcurveto{\pgfqpoint{0.742082in}{0.383025in}}{\pgfqpoint{0.739887in}{0.388325in}}{\pgfqpoint{0.735980in}{0.392231in}}%
\pgfpathcurveto{\pgfqpoint{0.732073in}{0.396138in}}{\pgfqpoint{0.726774in}{0.398333in}}{\pgfqpoint{0.721249in}{0.398333in}}%
\pgfpathcurveto{\pgfqpoint{0.715724in}{0.398333in}}{\pgfqpoint{0.710424in}{0.396138in}}{\pgfqpoint{0.706518in}{0.392231in}}%
\pgfpathcurveto{\pgfqpoint{0.702611in}{0.388325in}}{\pgfqpoint{0.700416in}{0.383025in}}{\pgfqpoint{0.700416in}{0.377500in}}%
\pgfpathcurveto{\pgfqpoint{0.700416in}{0.371975in}}{\pgfqpoint{0.702611in}{0.366675in}}{\pgfqpoint{0.706518in}{0.362769in}}%
\pgfpathcurveto{\pgfqpoint{0.710424in}{0.358862in}}{\pgfqpoint{0.715724in}{0.356667in}}{\pgfqpoint{0.721249in}{0.356667in}}%
\pgfpathclose%
\pgfusepath{stroke,fill}%
\end{pgfscope}%
\begin{pgfscope}%
\pgfpathrectangle{\pgfqpoint{0.562500in}{0.275000in}}{\pgfqpoint{3.487500in}{1.925000in}}%
\pgfusepath{clip}%
\pgfsetbuttcap%
\pgfsetroundjoin%
\definecolor{currentfill}{rgb}{0.000000,0.000000,0.000000}%
\pgfsetfillcolor{currentfill}%
\pgfsetlinewidth{1.003750pt}%
\definecolor{currentstroke}{rgb}{0.000000,0.000000,0.000000}%
\pgfsetstrokecolor{currentstroke}%
\pgfsetdash{}{0pt}%
\pgfpathmoveto{\pgfqpoint{0.721249in}{0.356667in}}%
\pgfpathcurveto{\pgfqpoint{0.726774in}{0.356667in}}{\pgfqpoint{0.732073in}{0.358862in}}{\pgfqpoint{0.735980in}{0.362769in}}%
\pgfpathcurveto{\pgfqpoint{0.739887in}{0.366675in}}{\pgfqpoint{0.742082in}{0.371975in}}{\pgfqpoint{0.742082in}{0.377500in}}%
\pgfpathcurveto{\pgfqpoint{0.742082in}{0.383025in}}{\pgfqpoint{0.739887in}{0.388325in}}{\pgfqpoint{0.735980in}{0.392231in}}%
\pgfpathcurveto{\pgfqpoint{0.732073in}{0.396138in}}{\pgfqpoint{0.726774in}{0.398333in}}{\pgfqpoint{0.721249in}{0.398333in}}%
\pgfpathcurveto{\pgfqpoint{0.715724in}{0.398333in}}{\pgfqpoint{0.710424in}{0.396138in}}{\pgfqpoint{0.706518in}{0.392231in}}%
\pgfpathcurveto{\pgfqpoint{0.702611in}{0.388325in}}{\pgfqpoint{0.700416in}{0.383025in}}{\pgfqpoint{0.700416in}{0.377500in}}%
\pgfpathcurveto{\pgfqpoint{0.700416in}{0.371975in}}{\pgfqpoint{0.702611in}{0.366675in}}{\pgfqpoint{0.706518in}{0.362769in}}%
\pgfpathcurveto{\pgfqpoint{0.710424in}{0.358862in}}{\pgfqpoint{0.715724in}{0.356667in}}{\pgfqpoint{0.721249in}{0.356667in}}%
\pgfpathclose%
\pgfusepath{stroke,fill}%
\end{pgfscope}%
\begin{pgfscope}%
\pgfpathrectangle{\pgfqpoint{0.562500in}{0.275000in}}{\pgfqpoint{3.487500in}{1.925000in}}%
\pgfusepath{clip}%
\pgfsetbuttcap%
\pgfsetroundjoin%
\definecolor{currentfill}{rgb}{0.000000,0.000000,0.000000}%
\pgfsetfillcolor{currentfill}%
\pgfsetlinewidth{1.003750pt}%
\definecolor{currentstroke}{rgb}{0.000000,0.000000,0.000000}%
\pgfsetstrokecolor{currentstroke}%
\pgfsetdash{}{0pt}%
\pgfpathmoveto{\pgfqpoint{0.721249in}{0.356667in}}%
\pgfpathcurveto{\pgfqpoint{0.726774in}{0.356667in}}{\pgfqpoint{0.732073in}{0.358862in}}{\pgfqpoint{0.735980in}{0.362769in}}%
\pgfpathcurveto{\pgfqpoint{0.739887in}{0.366675in}}{\pgfqpoint{0.742082in}{0.371975in}}{\pgfqpoint{0.742082in}{0.377500in}}%
\pgfpathcurveto{\pgfqpoint{0.742082in}{0.383025in}}{\pgfqpoint{0.739887in}{0.388325in}}{\pgfqpoint{0.735980in}{0.392231in}}%
\pgfpathcurveto{\pgfqpoint{0.732073in}{0.396138in}}{\pgfqpoint{0.726774in}{0.398333in}}{\pgfqpoint{0.721249in}{0.398333in}}%
\pgfpathcurveto{\pgfqpoint{0.715724in}{0.398333in}}{\pgfqpoint{0.710424in}{0.396138in}}{\pgfqpoint{0.706518in}{0.392231in}}%
\pgfpathcurveto{\pgfqpoint{0.702611in}{0.388325in}}{\pgfqpoint{0.700416in}{0.383025in}}{\pgfqpoint{0.700416in}{0.377500in}}%
\pgfpathcurveto{\pgfqpoint{0.700416in}{0.371975in}}{\pgfqpoint{0.702611in}{0.366675in}}{\pgfqpoint{0.706518in}{0.362769in}}%
\pgfpathcurveto{\pgfqpoint{0.710424in}{0.358862in}}{\pgfqpoint{0.715724in}{0.356667in}}{\pgfqpoint{0.721249in}{0.356667in}}%
\pgfpathclose%
\pgfusepath{stroke,fill}%
\end{pgfscope}%
\begin{pgfscope}%
\pgfpathrectangle{\pgfqpoint{0.562500in}{0.275000in}}{\pgfqpoint{3.487500in}{1.925000in}}%
\pgfusepath{clip}%
\pgfsetbuttcap%
\pgfsetroundjoin%
\definecolor{currentfill}{rgb}{0.000000,0.000000,0.000000}%
\pgfsetfillcolor{currentfill}%
\pgfsetlinewidth{1.003750pt}%
\definecolor{currentstroke}{rgb}{0.000000,0.000000,0.000000}%
\pgfsetstrokecolor{currentstroke}%
\pgfsetdash{}{0pt}%
\pgfpathmoveto{\pgfqpoint{0.721249in}{0.356667in}}%
\pgfpathcurveto{\pgfqpoint{0.726774in}{0.356667in}}{\pgfqpoint{0.732073in}{0.358862in}}{\pgfqpoint{0.735980in}{0.362769in}}%
\pgfpathcurveto{\pgfqpoint{0.739887in}{0.366675in}}{\pgfqpoint{0.742082in}{0.371975in}}{\pgfqpoint{0.742082in}{0.377500in}}%
\pgfpathcurveto{\pgfqpoint{0.742082in}{0.383025in}}{\pgfqpoint{0.739887in}{0.388325in}}{\pgfqpoint{0.735980in}{0.392231in}}%
\pgfpathcurveto{\pgfqpoint{0.732073in}{0.396138in}}{\pgfqpoint{0.726774in}{0.398333in}}{\pgfqpoint{0.721249in}{0.398333in}}%
\pgfpathcurveto{\pgfqpoint{0.715724in}{0.398333in}}{\pgfqpoint{0.710424in}{0.396138in}}{\pgfqpoint{0.706518in}{0.392231in}}%
\pgfpathcurveto{\pgfqpoint{0.702611in}{0.388325in}}{\pgfqpoint{0.700416in}{0.383025in}}{\pgfqpoint{0.700416in}{0.377500in}}%
\pgfpathcurveto{\pgfqpoint{0.700416in}{0.371975in}}{\pgfqpoint{0.702611in}{0.366675in}}{\pgfqpoint{0.706518in}{0.362769in}}%
\pgfpathcurveto{\pgfqpoint{0.710424in}{0.358862in}}{\pgfqpoint{0.715724in}{0.356667in}}{\pgfqpoint{0.721249in}{0.356667in}}%
\pgfpathclose%
\pgfusepath{stroke,fill}%
\end{pgfscope}%
\begin{pgfscope}%
\pgfpathrectangle{\pgfqpoint{0.562500in}{0.275000in}}{\pgfqpoint{3.487500in}{1.925000in}}%
\pgfusepath{clip}%
\pgfsetbuttcap%
\pgfsetroundjoin%
\definecolor{currentfill}{rgb}{0.000000,0.000000,0.000000}%
\pgfsetfillcolor{currentfill}%
\pgfsetlinewidth{1.003750pt}%
\definecolor{currentstroke}{rgb}{0.000000,0.000000,0.000000}%
\pgfsetstrokecolor{currentstroke}%
\pgfsetdash{}{0pt}%
\pgfpathmoveto{\pgfqpoint{0.721249in}{0.356667in}}%
\pgfpathcurveto{\pgfqpoint{0.726774in}{0.356667in}}{\pgfqpoint{0.732073in}{0.358862in}}{\pgfqpoint{0.735980in}{0.362769in}}%
\pgfpathcurveto{\pgfqpoint{0.739887in}{0.366675in}}{\pgfqpoint{0.742082in}{0.371975in}}{\pgfqpoint{0.742082in}{0.377500in}}%
\pgfpathcurveto{\pgfqpoint{0.742082in}{0.383025in}}{\pgfqpoint{0.739887in}{0.388325in}}{\pgfqpoint{0.735980in}{0.392231in}}%
\pgfpathcurveto{\pgfqpoint{0.732073in}{0.396138in}}{\pgfqpoint{0.726774in}{0.398333in}}{\pgfqpoint{0.721249in}{0.398333in}}%
\pgfpathcurveto{\pgfqpoint{0.715724in}{0.398333in}}{\pgfqpoint{0.710424in}{0.396138in}}{\pgfqpoint{0.706518in}{0.392231in}}%
\pgfpathcurveto{\pgfqpoint{0.702611in}{0.388325in}}{\pgfqpoint{0.700416in}{0.383025in}}{\pgfqpoint{0.700416in}{0.377500in}}%
\pgfpathcurveto{\pgfqpoint{0.700416in}{0.371975in}}{\pgfqpoint{0.702611in}{0.366675in}}{\pgfqpoint{0.706518in}{0.362769in}}%
\pgfpathcurveto{\pgfqpoint{0.710424in}{0.358862in}}{\pgfqpoint{0.715724in}{0.356667in}}{\pgfqpoint{0.721249in}{0.356667in}}%
\pgfpathclose%
\pgfusepath{stroke,fill}%
\end{pgfscope}%
\begin{pgfscope}%
\pgfpathrectangle{\pgfqpoint{0.562500in}{0.275000in}}{\pgfqpoint{3.487500in}{1.925000in}}%
\pgfusepath{clip}%
\pgfsetbuttcap%
\pgfsetroundjoin%
\definecolor{currentfill}{rgb}{0.000000,0.000000,0.000000}%
\pgfsetfillcolor{currentfill}%
\pgfsetlinewidth{1.003750pt}%
\definecolor{currentstroke}{rgb}{0.000000,0.000000,0.000000}%
\pgfsetstrokecolor{currentstroke}%
\pgfsetdash{}{0pt}%
\pgfpathmoveto{\pgfqpoint{0.721249in}{0.356667in}}%
\pgfpathcurveto{\pgfqpoint{0.726774in}{0.356667in}}{\pgfqpoint{0.732073in}{0.358862in}}{\pgfqpoint{0.735980in}{0.362769in}}%
\pgfpathcurveto{\pgfqpoint{0.739887in}{0.366675in}}{\pgfqpoint{0.742082in}{0.371975in}}{\pgfqpoint{0.742082in}{0.377500in}}%
\pgfpathcurveto{\pgfqpoint{0.742082in}{0.383025in}}{\pgfqpoint{0.739887in}{0.388325in}}{\pgfqpoint{0.735980in}{0.392231in}}%
\pgfpathcurveto{\pgfqpoint{0.732073in}{0.396138in}}{\pgfqpoint{0.726774in}{0.398333in}}{\pgfqpoint{0.721249in}{0.398333in}}%
\pgfpathcurveto{\pgfqpoint{0.715724in}{0.398333in}}{\pgfqpoint{0.710424in}{0.396138in}}{\pgfqpoint{0.706518in}{0.392231in}}%
\pgfpathcurveto{\pgfqpoint{0.702611in}{0.388325in}}{\pgfqpoint{0.700416in}{0.383025in}}{\pgfqpoint{0.700416in}{0.377500in}}%
\pgfpathcurveto{\pgfqpoint{0.700416in}{0.371975in}}{\pgfqpoint{0.702611in}{0.366675in}}{\pgfqpoint{0.706518in}{0.362769in}}%
\pgfpathcurveto{\pgfqpoint{0.710424in}{0.358862in}}{\pgfqpoint{0.715724in}{0.356667in}}{\pgfqpoint{0.721249in}{0.356667in}}%
\pgfpathclose%
\pgfusepath{stroke,fill}%
\end{pgfscope}%
\begin{pgfscope}%
\pgfpathrectangle{\pgfqpoint{0.562500in}{0.275000in}}{\pgfqpoint{3.487500in}{1.925000in}}%
\pgfusepath{clip}%
\pgfsetbuttcap%
\pgfsetroundjoin%
\definecolor{currentfill}{rgb}{0.000000,0.000000,0.000000}%
\pgfsetfillcolor{currentfill}%
\pgfsetlinewidth{1.003750pt}%
\definecolor{currentstroke}{rgb}{0.000000,0.000000,0.000000}%
\pgfsetstrokecolor{currentstroke}%
\pgfsetdash{}{0pt}%
\pgfpathmoveto{\pgfqpoint{0.721249in}{0.356667in}}%
\pgfpathcurveto{\pgfqpoint{0.726774in}{0.356667in}}{\pgfqpoint{0.732073in}{0.358862in}}{\pgfqpoint{0.735980in}{0.362769in}}%
\pgfpathcurveto{\pgfqpoint{0.739887in}{0.366675in}}{\pgfqpoint{0.742082in}{0.371975in}}{\pgfqpoint{0.742082in}{0.377500in}}%
\pgfpathcurveto{\pgfqpoint{0.742082in}{0.383025in}}{\pgfqpoint{0.739887in}{0.388325in}}{\pgfqpoint{0.735980in}{0.392231in}}%
\pgfpathcurveto{\pgfqpoint{0.732073in}{0.396138in}}{\pgfqpoint{0.726774in}{0.398333in}}{\pgfqpoint{0.721249in}{0.398333in}}%
\pgfpathcurveto{\pgfqpoint{0.715724in}{0.398333in}}{\pgfqpoint{0.710424in}{0.396138in}}{\pgfqpoint{0.706518in}{0.392231in}}%
\pgfpathcurveto{\pgfqpoint{0.702611in}{0.388325in}}{\pgfqpoint{0.700416in}{0.383025in}}{\pgfqpoint{0.700416in}{0.377500in}}%
\pgfpathcurveto{\pgfqpoint{0.700416in}{0.371975in}}{\pgfqpoint{0.702611in}{0.366675in}}{\pgfqpoint{0.706518in}{0.362769in}}%
\pgfpathcurveto{\pgfqpoint{0.710424in}{0.358862in}}{\pgfqpoint{0.715724in}{0.356667in}}{\pgfqpoint{0.721249in}{0.356667in}}%
\pgfpathclose%
\pgfusepath{stroke,fill}%
\end{pgfscope}%
\begin{pgfscope}%
\pgfpathrectangle{\pgfqpoint{0.562500in}{0.275000in}}{\pgfqpoint{3.487500in}{1.925000in}}%
\pgfusepath{clip}%
\pgfsetbuttcap%
\pgfsetroundjoin%
\definecolor{currentfill}{rgb}{0.000000,0.000000,0.000000}%
\pgfsetfillcolor{currentfill}%
\pgfsetlinewidth{1.003750pt}%
\definecolor{currentstroke}{rgb}{0.000000,0.000000,0.000000}%
\pgfsetstrokecolor{currentstroke}%
\pgfsetdash{}{0pt}%
\pgfpathmoveto{\pgfqpoint{0.721249in}{0.356667in}}%
\pgfpathcurveto{\pgfqpoint{0.726774in}{0.356667in}}{\pgfqpoint{0.732073in}{0.358862in}}{\pgfqpoint{0.735980in}{0.362769in}}%
\pgfpathcurveto{\pgfqpoint{0.739887in}{0.366675in}}{\pgfqpoint{0.742082in}{0.371975in}}{\pgfqpoint{0.742082in}{0.377500in}}%
\pgfpathcurveto{\pgfqpoint{0.742082in}{0.383025in}}{\pgfqpoint{0.739887in}{0.388325in}}{\pgfqpoint{0.735980in}{0.392231in}}%
\pgfpathcurveto{\pgfqpoint{0.732073in}{0.396138in}}{\pgfqpoint{0.726774in}{0.398333in}}{\pgfqpoint{0.721249in}{0.398333in}}%
\pgfpathcurveto{\pgfqpoint{0.715724in}{0.398333in}}{\pgfqpoint{0.710424in}{0.396138in}}{\pgfqpoint{0.706518in}{0.392231in}}%
\pgfpathcurveto{\pgfqpoint{0.702611in}{0.388325in}}{\pgfqpoint{0.700416in}{0.383025in}}{\pgfqpoint{0.700416in}{0.377500in}}%
\pgfpathcurveto{\pgfqpoint{0.700416in}{0.371975in}}{\pgfqpoint{0.702611in}{0.366675in}}{\pgfqpoint{0.706518in}{0.362769in}}%
\pgfpathcurveto{\pgfqpoint{0.710424in}{0.358862in}}{\pgfqpoint{0.715724in}{0.356667in}}{\pgfqpoint{0.721249in}{0.356667in}}%
\pgfpathclose%
\pgfusepath{stroke,fill}%
\end{pgfscope}%
\begin{pgfscope}%
\pgfpathrectangle{\pgfqpoint{0.562500in}{0.275000in}}{\pgfqpoint{3.487500in}{1.925000in}}%
\pgfusepath{clip}%
\pgfsetbuttcap%
\pgfsetroundjoin%
\definecolor{currentfill}{rgb}{0.000000,0.000000,0.000000}%
\pgfsetfillcolor{currentfill}%
\pgfsetlinewidth{1.003750pt}%
\definecolor{currentstroke}{rgb}{0.000000,0.000000,0.000000}%
\pgfsetstrokecolor{currentstroke}%
\pgfsetdash{}{0pt}%
\pgfpathmoveto{\pgfqpoint{0.721249in}{0.356667in}}%
\pgfpathcurveto{\pgfqpoint{0.726774in}{0.356667in}}{\pgfqpoint{0.732073in}{0.358862in}}{\pgfqpoint{0.735980in}{0.362769in}}%
\pgfpathcurveto{\pgfqpoint{0.739887in}{0.366675in}}{\pgfqpoint{0.742082in}{0.371975in}}{\pgfqpoint{0.742082in}{0.377500in}}%
\pgfpathcurveto{\pgfqpoint{0.742082in}{0.383025in}}{\pgfqpoint{0.739887in}{0.388325in}}{\pgfqpoint{0.735980in}{0.392231in}}%
\pgfpathcurveto{\pgfqpoint{0.732073in}{0.396138in}}{\pgfqpoint{0.726774in}{0.398333in}}{\pgfqpoint{0.721249in}{0.398333in}}%
\pgfpathcurveto{\pgfqpoint{0.715724in}{0.398333in}}{\pgfqpoint{0.710424in}{0.396138in}}{\pgfqpoint{0.706518in}{0.392231in}}%
\pgfpathcurveto{\pgfqpoint{0.702611in}{0.388325in}}{\pgfqpoint{0.700416in}{0.383025in}}{\pgfqpoint{0.700416in}{0.377500in}}%
\pgfpathcurveto{\pgfqpoint{0.700416in}{0.371975in}}{\pgfqpoint{0.702611in}{0.366675in}}{\pgfqpoint{0.706518in}{0.362769in}}%
\pgfpathcurveto{\pgfqpoint{0.710424in}{0.358862in}}{\pgfqpoint{0.715724in}{0.356667in}}{\pgfqpoint{0.721249in}{0.356667in}}%
\pgfpathclose%
\pgfusepath{stroke,fill}%
\end{pgfscope}%
\begin{pgfscope}%
\pgfpathrectangle{\pgfqpoint{0.562500in}{0.275000in}}{\pgfqpoint{3.487500in}{1.925000in}}%
\pgfusepath{clip}%
\pgfsetbuttcap%
\pgfsetroundjoin%
\definecolor{currentfill}{rgb}{0.000000,0.000000,0.000000}%
\pgfsetfillcolor{currentfill}%
\pgfsetlinewidth{1.003750pt}%
\definecolor{currentstroke}{rgb}{0.000000,0.000000,0.000000}%
\pgfsetstrokecolor{currentstroke}%
\pgfsetdash{}{0pt}%
\pgfpathmoveto{\pgfqpoint{0.721249in}{0.356667in}}%
\pgfpathcurveto{\pgfqpoint{0.726774in}{0.356667in}}{\pgfqpoint{0.732073in}{0.358862in}}{\pgfqpoint{0.735980in}{0.362769in}}%
\pgfpathcurveto{\pgfqpoint{0.739887in}{0.366675in}}{\pgfqpoint{0.742082in}{0.371975in}}{\pgfqpoint{0.742082in}{0.377500in}}%
\pgfpathcurveto{\pgfqpoint{0.742082in}{0.383025in}}{\pgfqpoint{0.739887in}{0.388325in}}{\pgfqpoint{0.735980in}{0.392231in}}%
\pgfpathcurveto{\pgfqpoint{0.732073in}{0.396138in}}{\pgfqpoint{0.726774in}{0.398333in}}{\pgfqpoint{0.721249in}{0.398333in}}%
\pgfpathcurveto{\pgfqpoint{0.715724in}{0.398333in}}{\pgfqpoint{0.710424in}{0.396138in}}{\pgfqpoint{0.706518in}{0.392231in}}%
\pgfpathcurveto{\pgfqpoint{0.702611in}{0.388325in}}{\pgfqpoint{0.700416in}{0.383025in}}{\pgfqpoint{0.700416in}{0.377500in}}%
\pgfpathcurveto{\pgfqpoint{0.700416in}{0.371975in}}{\pgfqpoint{0.702611in}{0.366675in}}{\pgfqpoint{0.706518in}{0.362769in}}%
\pgfpathcurveto{\pgfqpoint{0.710424in}{0.358862in}}{\pgfqpoint{0.715724in}{0.356667in}}{\pgfqpoint{0.721249in}{0.356667in}}%
\pgfpathclose%
\pgfusepath{stroke,fill}%
\end{pgfscope}%
\begin{pgfscope}%
\pgfpathrectangle{\pgfqpoint{0.562500in}{0.275000in}}{\pgfqpoint{3.487500in}{1.925000in}}%
\pgfusepath{clip}%
\pgfsetbuttcap%
\pgfsetroundjoin%
\definecolor{currentfill}{rgb}{0.000000,0.000000,0.000000}%
\pgfsetfillcolor{currentfill}%
\pgfsetlinewidth{1.003750pt}%
\definecolor{currentstroke}{rgb}{0.000000,0.000000,0.000000}%
\pgfsetstrokecolor{currentstroke}%
\pgfsetdash{}{0pt}%
\pgfpathmoveto{\pgfqpoint{0.721249in}{0.356667in}}%
\pgfpathcurveto{\pgfqpoint{0.726774in}{0.356667in}}{\pgfqpoint{0.732073in}{0.358862in}}{\pgfqpoint{0.735980in}{0.362769in}}%
\pgfpathcurveto{\pgfqpoint{0.739887in}{0.366675in}}{\pgfqpoint{0.742082in}{0.371975in}}{\pgfqpoint{0.742082in}{0.377500in}}%
\pgfpathcurveto{\pgfqpoint{0.742082in}{0.383025in}}{\pgfqpoint{0.739887in}{0.388325in}}{\pgfqpoint{0.735980in}{0.392231in}}%
\pgfpathcurveto{\pgfqpoint{0.732073in}{0.396138in}}{\pgfqpoint{0.726774in}{0.398333in}}{\pgfqpoint{0.721249in}{0.398333in}}%
\pgfpathcurveto{\pgfqpoint{0.715724in}{0.398333in}}{\pgfqpoint{0.710424in}{0.396138in}}{\pgfqpoint{0.706518in}{0.392231in}}%
\pgfpathcurveto{\pgfqpoint{0.702611in}{0.388325in}}{\pgfqpoint{0.700416in}{0.383025in}}{\pgfqpoint{0.700416in}{0.377500in}}%
\pgfpathcurveto{\pgfqpoint{0.700416in}{0.371975in}}{\pgfqpoint{0.702611in}{0.366675in}}{\pgfqpoint{0.706518in}{0.362769in}}%
\pgfpathcurveto{\pgfqpoint{0.710424in}{0.358862in}}{\pgfqpoint{0.715724in}{0.356667in}}{\pgfqpoint{0.721249in}{0.356667in}}%
\pgfpathclose%
\pgfusepath{stroke,fill}%
\end{pgfscope}%
\begin{pgfscope}%
\pgfpathrectangle{\pgfqpoint{0.562500in}{0.275000in}}{\pgfqpoint{3.487500in}{1.925000in}}%
\pgfusepath{clip}%
\pgfsetbuttcap%
\pgfsetroundjoin%
\definecolor{currentfill}{rgb}{0.000000,0.000000,0.000000}%
\pgfsetfillcolor{currentfill}%
\pgfsetlinewidth{1.003750pt}%
\definecolor{currentstroke}{rgb}{0.000000,0.000000,0.000000}%
\pgfsetstrokecolor{currentstroke}%
\pgfsetdash{}{0pt}%
\pgfpathmoveto{\pgfqpoint{0.721249in}{0.356667in}}%
\pgfpathcurveto{\pgfqpoint{0.726774in}{0.356667in}}{\pgfqpoint{0.732073in}{0.358862in}}{\pgfqpoint{0.735980in}{0.362769in}}%
\pgfpathcurveto{\pgfqpoint{0.739887in}{0.366675in}}{\pgfqpoint{0.742082in}{0.371975in}}{\pgfqpoint{0.742082in}{0.377500in}}%
\pgfpathcurveto{\pgfqpoint{0.742082in}{0.383025in}}{\pgfqpoint{0.739887in}{0.388325in}}{\pgfqpoint{0.735980in}{0.392231in}}%
\pgfpathcurveto{\pgfqpoint{0.732073in}{0.396138in}}{\pgfqpoint{0.726774in}{0.398333in}}{\pgfqpoint{0.721249in}{0.398333in}}%
\pgfpathcurveto{\pgfqpoint{0.715724in}{0.398333in}}{\pgfqpoint{0.710424in}{0.396138in}}{\pgfqpoint{0.706518in}{0.392231in}}%
\pgfpathcurveto{\pgfqpoint{0.702611in}{0.388325in}}{\pgfqpoint{0.700416in}{0.383025in}}{\pgfqpoint{0.700416in}{0.377500in}}%
\pgfpathcurveto{\pgfqpoint{0.700416in}{0.371975in}}{\pgfqpoint{0.702611in}{0.366675in}}{\pgfqpoint{0.706518in}{0.362769in}}%
\pgfpathcurveto{\pgfqpoint{0.710424in}{0.358862in}}{\pgfqpoint{0.715724in}{0.356667in}}{\pgfqpoint{0.721249in}{0.356667in}}%
\pgfpathclose%
\pgfusepath{stroke,fill}%
\end{pgfscope}%
\begin{pgfscope}%
\pgfpathrectangle{\pgfqpoint{0.562500in}{0.275000in}}{\pgfqpoint{3.487500in}{1.925000in}}%
\pgfusepath{clip}%
\pgfsetbuttcap%
\pgfsetroundjoin%
\definecolor{currentfill}{rgb}{0.000000,0.000000,0.000000}%
\pgfsetfillcolor{currentfill}%
\pgfsetlinewidth{1.003750pt}%
\definecolor{currentstroke}{rgb}{0.000000,0.000000,0.000000}%
\pgfsetstrokecolor{currentstroke}%
\pgfsetdash{}{0pt}%
\pgfpathmoveto{\pgfqpoint{0.721249in}{0.356667in}}%
\pgfpathcurveto{\pgfqpoint{0.726774in}{0.356667in}}{\pgfqpoint{0.732073in}{0.358862in}}{\pgfqpoint{0.735980in}{0.362769in}}%
\pgfpathcurveto{\pgfqpoint{0.739887in}{0.366675in}}{\pgfqpoint{0.742082in}{0.371975in}}{\pgfqpoint{0.742082in}{0.377500in}}%
\pgfpathcurveto{\pgfqpoint{0.742082in}{0.383025in}}{\pgfqpoint{0.739887in}{0.388325in}}{\pgfqpoint{0.735980in}{0.392231in}}%
\pgfpathcurveto{\pgfqpoint{0.732073in}{0.396138in}}{\pgfqpoint{0.726774in}{0.398333in}}{\pgfqpoint{0.721249in}{0.398333in}}%
\pgfpathcurveto{\pgfqpoint{0.715724in}{0.398333in}}{\pgfqpoint{0.710424in}{0.396138in}}{\pgfqpoint{0.706518in}{0.392231in}}%
\pgfpathcurveto{\pgfqpoint{0.702611in}{0.388325in}}{\pgfqpoint{0.700416in}{0.383025in}}{\pgfqpoint{0.700416in}{0.377500in}}%
\pgfpathcurveto{\pgfqpoint{0.700416in}{0.371975in}}{\pgfqpoint{0.702611in}{0.366675in}}{\pgfqpoint{0.706518in}{0.362769in}}%
\pgfpathcurveto{\pgfqpoint{0.710424in}{0.358862in}}{\pgfqpoint{0.715724in}{0.356667in}}{\pgfqpoint{0.721249in}{0.356667in}}%
\pgfpathclose%
\pgfusepath{stroke,fill}%
\end{pgfscope}%
\begin{pgfscope}%
\pgfpathrectangle{\pgfqpoint{0.562500in}{0.275000in}}{\pgfqpoint{3.487500in}{1.925000in}}%
\pgfusepath{clip}%
\pgfsetbuttcap%
\pgfsetroundjoin%
\definecolor{currentfill}{rgb}{0.000000,0.000000,0.000000}%
\pgfsetfillcolor{currentfill}%
\pgfsetlinewidth{1.003750pt}%
\definecolor{currentstroke}{rgb}{0.000000,0.000000,0.000000}%
\pgfsetstrokecolor{currentstroke}%
\pgfsetdash{}{0pt}%
\pgfpathmoveto{\pgfqpoint{0.721249in}{0.356667in}}%
\pgfpathcurveto{\pgfqpoint{0.726774in}{0.356667in}}{\pgfqpoint{0.732073in}{0.358862in}}{\pgfqpoint{0.735980in}{0.362769in}}%
\pgfpathcurveto{\pgfqpoint{0.739887in}{0.366675in}}{\pgfqpoint{0.742082in}{0.371975in}}{\pgfqpoint{0.742082in}{0.377500in}}%
\pgfpathcurveto{\pgfqpoint{0.742082in}{0.383025in}}{\pgfqpoint{0.739887in}{0.388325in}}{\pgfqpoint{0.735980in}{0.392231in}}%
\pgfpathcurveto{\pgfqpoint{0.732073in}{0.396138in}}{\pgfqpoint{0.726774in}{0.398333in}}{\pgfqpoint{0.721249in}{0.398333in}}%
\pgfpathcurveto{\pgfqpoint{0.715724in}{0.398333in}}{\pgfqpoint{0.710424in}{0.396138in}}{\pgfqpoint{0.706518in}{0.392231in}}%
\pgfpathcurveto{\pgfqpoint{0.702611in}{0.388325in}}{\pgfqpoint{0.700416in}{0.383025in}}{\pgfqpoint{0.700416in}{0.377500in}}%
\pgfpathcurveto{\pgfqpoint{0.700416in}{0.371975in}}{\pgfqpoint{0.702611in}{0.366675in}}{\pgfqpoint{0.706518in}{0.362769in}}%
\pgfpathcurveto{\pgfqpoint{0.710424in}{0.358862in}}{\pgfqpoint{0.715724in}{0.356667in}}{\pgfqpoint{0.721249in}{0.356667in}}%
\pgfpathclose%
\pgfusepath{stroke,fill}%
\end{pgfscope}%
\begin{pgfscope}%
\pgfpathrectangle{\pgfqpoint{0.562500in}{0.275000in}}{\pgfqpoint{3.487500in}{1.925000in}}%
\pgfusepath{clip}%
\pgfsetbuttcap%
\pgfsetroundjoin%
\definecolor{currentfill}{rgb}{0.000000,0.000000,0.000000}%
\pgfsetfillcolor{currentfill}%
\pgfsetlinewidth{1.003750pt}%
\definecolor{currentstroke}{rgb}{0.000000,0.000000,0.000000}%
\pgfsetstrokecolor{currentstroke}%
\pgfsetdash{}{0pt}%
\pgfpathmoveto{\pgfqpoint{0.721249in}{0.356667in}}%
\pgfpathcurveto{\pgfqpoint{0.726774in}{0.356667in}}{\pgfqpoint{0.732073in}{0.358862in}}{\pgfqpoint{0.735980in}{0.362769in}}%
\pgfpathcurveto{\pgfqpoint{0.739887in}{0.366675in}}{\pgfqpoint{0.742082in}{0.371975in}}{\pgfqpoint{0.742082in}{0.377500in}}%
\pgfpathcurveto{\pgfqpoint{0.742082in}{0.383025in}}{\pgfqpoint{0.739887in}{0.388325in}}{\pgfqpoint{0.735980in}{0.392231in}}%
\pgfpathcurveto{\pgfqpoint{0.732073in}{0.396138in}}{\pgfqpoint{0.726774in}{0.398333in}}{\pgfqpoint{0.721249in}{0.398333in}}%
\pgfpathcurveto{\pgfqpoint{0.715724in}{0.398333in}}{\pgfqpoint{0.710424in}{0.396138in}}{\pgfqpoint{0.706518in}{0.392231in}}%
\pgfpathcurveto{\pgfqpoint{0.702611in}{0.388325in}}{\pgfqpoint{0.700416in}{0.383025in}}{\pgfqpoint{0.700416in}{0.377500in}}%
\pgfpathcurveto{\pgfqpoint{0.700416in}{0.371975in}}{\pgfqpoint{0.702611in}{0.366675in}}{\pgfqpoint{0.706518in}{0.362769in}}%
\pgfpathcurveto{\pgfqpoint{0.710424in}{0.358862in}}{\pgfqpoint{0.715724in}{0.356667in}}{\pgfqpoint{0.721249in}{0.356667in}}%
\pgfpathclose%
\pgfusepath{stroke,fill}%
\end{pgfscope}%
\begin{pgfscope}%
\pgfpathrectangle{\pgfqpoint{0.562500in}{0.275000in}}{\pgfqpoint{3.487500in}{1.925000in}}%
\pgfusepath{clip}%
\pgfsetbuttcap%
\pgfsetroundjoin%
\definecolor{currentfill}{rgb}{0.000000,0.000000,0.000000}%
\pgfsetfillcolor{currentfill}%
\pgfsetlinewidth{1.003750pt}%
\definecolor{currentstroke}{rgb}{0.000000,0.000000,0.000000}%
\pgfsetstrokecolor{currentstroke}%
\pgfsetdash{}{0pt}%
\pgfpathmoveto{\pgfqpoint{0.721249in}{0.356667in}}%
\pgfpathcurveto{\pgfqpoint{0.726774in}{0.356667in}}{\pgfqpoint{0.732073in}{0.358862in}}{\pgfqpoint{0.735980in}{0.362769in}}%
\pgfpathcurveto{\pgfqpoint{0.739887in}{0.366675in}}{\pgfqpoint{0.742082in}{0.371975in}}{\pgfqpoint{0.742082in}{0.377500in}}%
\pgfpathcurveto{\pgfqpoint{0.742082in}{0.383025in}}{\pgfqpoint{0.739887in}{0.388325in}}{\pgfqpoint{0.735980in}{0.392231in}}%
\pgfpathcurveto{\pgfqpoint{0.732073in}{0.396138in}}{\pgfqpoint{0.726774in}{0.398333in}}{\pgfqpoint{0.721249in}{0.398333in}}%
\pgfpathcurveto{\pgfqpoint{0.715724in}{0.398333in}}{\pgfqpoint{0.710424in}{0.396138in}}{\pgfqpoint{0.706518in}{0.392231in}}%
\pgfpathcurveto{\pgfqpoint{0.702611in}{0.388325in}}{\pgfqpoint{0.700416in}{0.383025in}}{\pgfqpoint{0.700416in}{0.377500in}}%
\pgfpathcurveto{\pgfqpoint{0.700416in}{0.371975in}}{\pgfqpoint{0.702611in}{0.366675in}}{\pgfqpoint{0.706518in}{0.362769in}}%
\pgfpathcurveto{\pgfqpoint{0.710424in}{0.358862in}}{\pgfqpoint{0.715724in}{0.356667in}}{\pgfqpoint{0.721249in}{0.356667in}}%
\pgfpathclose%
\pgfusepath{stroke,fill}%
\end{pgfscope}%
\begin{pgfscope}%
\pgfpathrectangle{\pgfqpoint{0.562500in}{0.275000in}}{\pgfqpoint{3.487500in}{1.925000in}}%
\pgfusepath{clip}%
\pgfsetbuttcap%
\pgfsetroundjoin%
\definecolor{currentfill}{rgb}{0.000000,0.000000,0.000000}%
\pgfsetfillcolor{currentfill}%
\pgfsetlinewidth{1.003750pt}%
\definecolor{currentstroke}{rgb}{0.000000,0.000000,0.000000}%
\pgfsetstrokecolor{currentstroke}%
\pgfsetdash{}{0pt}%
\pgfpathmoveto{\pgfqpoint{0.721249in}{0.356667in}}%
\pgfpathcurveto{\pgfqpoint{0.726774in}{0.356667in}}{\pgfqpoint{0.732073in}{0.358862in}}{\pgfqpoint{0.735980in}{0.362769in}}%
\pgfpathcurveto{\pgfqpoint{0.739887in}{0.366675in}}{\pgfqpoint{0.742082in}{0.371975in}}{\pgfqpoint{0.742082in}{0.377500in}}%
\pgfpathcurveto{\pgfqpoint{0.742082in}{0.383025in}}{\pgfqpoint{0.739887in}{0.388325in}}{\pgfqpoint{0.735980in}{0.392231in}}%
\pgfpathcurveto{\pgfqpoint{0.732073in}{0.396138in}}{\pgfqpoint{0.726774in}{0.398333in}}{\pgfqpoint{0.721249in}{0.398333in}}%
\pgfpathcurveto{\pgfqpoint{0.715724in}{0.398333in}}{\pgfqpoint{0.710424in}{0.396138in}}{\pgfqpoint{0.706518in}{0.392231in}}%
\pgfpathcurveto{\pgfqpoint{0.702611in}{0.388325in}}{\pgfqpoint{0.700416in}{0.383025in}}{\pgfqpoint{0.700416in}{0.377500in}}%
\pgfpathcurveto{\pgfqpoint{0.700416in}{0.371975in}}{\pgfqpoint{0.702611in}{0.366675in}}{\pgfqpoint{0.706518in}{0.362769in}}%
\pgfpathcurveto{\pgfqpoint{0.710424in}{0.358862in}}{\pgfqpoint{0.715724in}{0.356667in}}{\pgfqpoint{0.721249in}{0.356667in}}%
\pgfpathclose%
\pgfusepath{stroke,fill}%
\end{pgfscope}%
\begin{pgfscope}%
\pgfpathrectangle{\pgfqpoint{0.562500in}{0.275000in}}{\pgfqpoint{3.487500in}{1.925000in}}%
\pgfusepath{clip}%
\pgfsetbuttcap%
\pgfsetroundjoin%
\definecolor{currentfill}{rgb}{0.000000,0.000000,0.000000}%
\pgfsetfillcolor{currentfill}%
\pgfsetlinewidth{1.003750pt}%
\definecolor{currentstroke}{rgb}{0.000000,0.000000,0.000000}%
\pgfsetstrokecolor{currentstroke}%
\pgfsetdash{}{0pt}%
\pgfpathmoveto{\pgfqpoint{0.721249in}{0.356667in}}%
\pgfpathcurveto{\pgfqpoint{0.726774in}{0.356667in}}{\pgfqpoint{0.732073in}{0.358862in}}{\pgfqpoint{0.735980in}{0.362769in}}%
\pgfpathcurveto{\pgfqpoint{0.739887in}{0.366675in}}{\pgfqpoint{0.742082in}{0.371975in}}{\pgfqpoint{0.742082in}{0.377500in}}%
\pgfpathcurveto{\pgfqpoint{0.742082in}{0.383025in}}{\pgfqpoint{0.739887in}{0.388325in}}{\pgfqpoint{0.735980in}{0.392231in}}%
\pgfpathcurveto{\pgfqpoint{0.732073in}{0.396138in}}{\pgfqpoint{0.726774in}{0.398333in}}{\pgfqpoint{0.721249in}{0.398333in}}%
\pgfpathcurveto{\pgfqpoint{0.715724in}{0.398333in}}{\pgfqpoint{0.710424in}{0.396138in}}{\pgfqpoint{0.706518in}{0.392231in}}%
\pgfpathcurveto{\pgfqpoint{0.702611in}{0.388325in}}{\pgfqpoint{0.700416in}{0.383025in}}{\pgfqpoint{0.700416in}{0.377500in}}%
\pgfpathcurveto{\pgfqpoint{0.700416in}{0.371975in}}{\pgfqpoint{0.702611in}{0.366675in}}{\pgfqpoint{0.706518in}{0.362769in}}%
\pgfpathcurveto{\pgfqpoint{0.710424in}{0.358862in}}{\pgfqpoint{0.715724in}{0.356667in}}{\pgfqpoint{0.721249in}{0.356667in}}%
\pgfpathclose%
\pgfusepath{stroke,fill}%
\end{pgfscope}%
\begin{pgfscope}%
\pgfpathrectangle{\pgfqpoint{0.562500in}{0.275000in}}{\pgfqpoint{3.487500in}{1.925000in}}%
\pgfusepath{clip}%
\pgfsetbuttcap%
\pgfsetroundjoin%
\definecolor{currentfill}{rgb}{0.000000,0.000000,0.000000}%
\pgfsetfillcolor{currentfill}%
\pgfsetlinewidth{1.003750pt}%
\definecolor{currentstroke}{rgb}{0.000000,0.000000,0.000000}%
\pgfsetstrokecolor{currentstroke}%
\pgfsetdash{}{0pt}%
\pgfpathmoveto{\pgfqpoint{0.721249in}{0.356667in}}%
\pgfpathcurveto{\pgfqpoint{0.726774in}{0.356667in}}{\pgfqpoint{0.732073in}{0.358862in}}{\pgfqpoint{0.735980in}{0.362769in}}%
\pgfpathcurveto{\pgfqpoint{0.739887in}{0.366675in}}{\pgfqpoint{0.742082in}{0.371975in}}{\pgfqpoint{0.742082in}{0.377500in}}%
\pgfpathcurveto{\pgfqpoint{0.742082in}{0.383025in}}{\pgfqpoint{0.739887in}{0.388325in}}{\pgfqpoint{0.735980in}{0.392231in}}%
\pgfpathcurveto{\pgfqpoint{0.732073in}{0.396138in}}{\pgfqpoint{0.726774in}{0.398333in}}{\pgfqpoint{0.721249in}{0.398333in}}%
\pgfpathcurveto{\pgfqpoint{0.715724in}{0.398333in}}{\pgfqpoint{0.710424in}{0.396138in}}{\pgfqpoint{0.706518in}{0.392231in}}%
\pgfpathcurveto{\pgfqpoint{0.702611in}{0.388325in}}{\pgfqpoint{0.700416in}{0.383025in}}{\pgfqpoint{0.700416in}{0.377500in}}%
\pgfpathcurveto{\pgfqpoint{0.700416in}{0.371975in}}{\pgfqpoint{0.702611in}{0.366675in}}{\pgfqpoint{0.706518in}{0.362769in}}%
\pgfpathcurveto{\pgfqpoint{0.710424in}{0.358862in}}{\pgfqpoint{0.715724in}{0.356667in}}{\pgfqpoint{0.721249in}{0.356667in}}%
\pgfpathclose%
\pgfusepath{stroke,fill}%
\end{pgfscope}%
\begin{pgfscope}%
\pgfpathrectangle{\pgfqpoint{0.562500in}{0.275000in}}{\pgfqpoint{3.487500in}{1.925000in}}%
\pgfusepath{clip}%
\pgfsetbuttcap%
\pgfsetroundjoin%
\definecolor{currentfill}{rgb}{0.000000,0.000000,0.000000}%
\pgfsetfillcolor{currentfill}%
\pgfsetlinewidth{1.003750pt}%
\definecolor{currentstroke}{rgb}{0.000000,0.000000,0.000000}%
\pgfsetstrokecolor{currentstroke}%
\pgfsetdash{}{0pt}%
\pgfpathmoveto{\pgfqpoint{0.721249in}{0.356667in}}%
\pgfpathcurveto{\pgfqpoint{0.726774in}{0.356667in}}{\pgfqpoint{0.732073in}{0.358862in}}{\pgfqpoint{0.735980in}{0.362769in}}%
\pgfpathcurveto{\pgfqpoint{0.739887in}{0.366675in}}{\pgfqpoint{0.742082in}{0.371975in}}{\pgfqpoint{0.742082in}{0.377500in}}%
\pgfpathcurveto{\pgfqpoint{0.742082in}{0.383025in}}{\pgfqpoint{0.739887in}{0.388325in}}{\pgfqpoint{0.735980in}{0.392231in}}%
\pgfpathcurveto{\pgfqpoint{0.732073in}{0.396138in}}{\pgfqpoint{0.726774in}{0.398333in}}{\pgfqpoint{0.721249in}{0.398333in}}%
\pgfpathcurveto{\pgfqpoint{0.715724in}{0.398333in}}{\pgfqpoint{0.710424in}{0.396138in}}{\pgfqpoint{0.706518in}{0.392231in}}%
\pgfpathcurveto{\pgfqpoint{0.702611in}{0.388325in}}{\pgfqpoint{0.700416in}{0.383025in}}{\pgfqpoint{0.700416in}{0.377500in}}%
\pgfpathcurveto{\pgfqpoint{0.700416in}{0.371975in}}{\pgfqpoint{0.702611in}{0.366675in}}{\pgfqpoint{0.706518in}{0.362769in}}%
\pgfpathcurveto{\pgfqpoint{0.710424in}{0.358862in}}{\pgfqpoint{0.715724in}{0.356667in}}{\pgfqpoint{0.721249in}{0.356667in}}%
\pgfpathclose%
\pgfusepath{stroke,fill}%
\end{pgfscope}%
\begin{pgfscope}%
\pgfpathrectangle{\pgfqpoint{0.562500in}{0.275000in}}{\pgfqpoint{3.487500in}{1.925000in}}%
\pgfusepath{clip}%
\pgfsetbuttcap%
\pgfsetroundjoin%
\definecolor{currentfill}{rgb}{0.000000,0.000000,0.000000}%
\pgfsetfillcolor{currentfill}%
\pgfsetlinewidth{1.003750pt}%
\definecolor{currentstroke}{rgb}{0.000000,0.000000,0.000000}%
\pgfsetstrokecolor{currentstroke}%
\pgfsetdash{}{0pt}%
\pgfpathmoveto{\pgfqpoint{0.721249in}{0.356667in}}%
\pgfpathcurveto{\pgfqpoint{0.726774in}{0.356667in}}{\pgfqpoint{0.732073in}{0.358862in}}{\pgfqpoint{0.735980in}{0.362769in}}%
\pgfpathcurveto{\pgfqpoint{0.739887in}{0.366675in}}{\pgfqpoint{0.742082in}{0.371975in}}{\pgfqpoint{0.742082in}{0.377500in}}%
\pgfpathcurveto{\pgfqpoint{0.742082in}{0.383025in}}{\pgfqpoint{0.739887in}{0.388325in}}{\pgfqpoint{0.735980in}{0.392231in}}%
\pgfpathcurveto{\pgfqpoint{0.732073in}{0.396138in}}{\pgfqpoint{0.726774in}{0.398333in}}{\pgfqpoint{0.721249in}{0.398333in}}%
\pgfpathcurveto{\pgfqpoint{0.715724in}{0.398333in}}{\pgfqpoint{0.710424in}{0.396138in}}{\pgfqpoint{0.706518in}{0.392231in}}%
\pgfpathcurveto{\pgfqpoint{0.702611in}{0.388325in}}{\pgfqpoint{0.700416in}{0.383025in}}{\pgfqpoint{0.700416in}{0.377500in}}%
\pgfpathcurveto{\pgfqpoint{0.700416in}{0.371975in}}{\pgfqpoint{0.702611in}{0.366675in}}{\pgfqpoint{0.706518in}{0.362769in}}%
\pgfpathcurveto{\pgfqpoint{0.710424in}{0.358862in}}{\pgfqpoint{0.715724in}{0.356667in}}{\pgfqpoint{0.721249in}{0.356667in}}%
\pgfpathclose%
\pgfusepath{stroke,fill}%
\end{pgfscope}%
\begin{pgfscope}%
\pgfpathrectangle{\pgfqpoint{0.562500in}{0.275000in}}{\pgfqpoint{3.487500in}{1.925000in}}%
\pgfusepath{clip}%
\pgfsetbuttcap%
\pgfsetroundjoin%
\definecolor{currentfill}{rgb}{0.000000,0.000000,0.000000}%
\pgfsetfillcolor{currentfill}%
\pgfsetlinewidth{1.003750pt}%
\definecolor{currentstroke}{rgb}{0.000000,0.000000,0.000000}%
\pgfsetstrokecolor{currentstroke}%
\pgfsetdash{}{0pt}%
\pgfpathmoveto{\pgfqpoint{0.721249in}{0.356667in}}%
\pgfpathcurveto{\pgfqpoint{0.726774in}{0.356667in}}{\pgfqpoint{0.732073in}{0.358862in}}{\pgfqpoint{0.735980in}{0.362769in}}%
\pgfpathcurveto{\pgfqpoint{0.739887in}{0.366675in}}{\pgfqpoint{0.742082in}{0.371975in}}{\pgfqpoint{0.742082in}{0.377500in}}%
\pgfpathcurveto{\pgfqpoint{0.742082in}{0.383025in}}{\pgfqpoint{0.739887in}{0.388325in}}{\pgfqpoint{0.735980in}{0.392231in}}%
\pgfpathcurveto{\pgfqpoint{0.732073in}{0.396138in}}{\pgfqpoint{0.726774in}{0.398333in}}{\pgfqpoint{0.721249in}{0.398333in}}%
\pgfpathcurveto{\pgfqpoint{0.715724in}{0.398333in}}{\pgfqpoint{0.710424in}{0.396138in}}{\pgfqpoint{0.706518in}{0.392231in}}%
\pgfpathcurveto{\pgfqpoint{0.702611in}{0.388325in}}{\pgfqpoint{0.700416in}{0.383025in}}{\pgfqpoint{0.700416in}{0.377500in}}%
\pgfpathcurveto{\pgfqpoint{0.700416in}{0.371975in}}{\pgfqpoint{0.702611in}{0.366675in}}{\pgfqpoint{0.706518in}{0.362769in}}%
\pgfpathcurveto{\pgfqpoint{0.710424in}{0.358862in}}{\pgfqpoint{0.715724in}{0.356667in}}{\pgfqpoint{0.721249in}{0.356667in}}%
\pgfpathclose%
\pgfusepath{stroke,fill}%
\end{pgfscope}%
\begin{pgfscope}%
\pgfpathrectangle{\pgfqpoint{0.562500in}{0.275000in}}{\pgfqpoint{3.487500in}{1.925000in}}%
\pgfusepath{clip}%
\pgfsetbuttcap%
\pgfsetroundjoin%
\definecolor{currentfill}{rgb}{0.000000,0.000000,0.000000}%
\pgfsetfillcolor{currentfill}%
\pgfsetlinewidth{1.003750pt}%
\definecolor{currentstroke}{rgb}{0.000000,0.000000,0.000000}%
\pgfsetstrokecolor{currentstroke}%
\pgfsetdash{}{0pt}%
\pgfpathmoveto{\pgfqpoint{0.721249in}{0.356667in}}%
\pgfpathcurveto{\pgfqpoint{0.726774in}{0.356667in}}{\pgfqpoint{0.732073in}{0.358862in}}{\pgfqpoint{0.735980in}{0.362769in}}%
\pgfpathcurveto{\pgfqpoint{0.739887in}{0.366675in}}{\pgfqpoint{0.742082in}{0.371975in}}{\pgfqpoint{0.742082in}{0.377500in}}%
\pgfpathcurveto{\pgfqpoint{0.742082in}{0.383025in}}{\pgfqpoint{0.739887in}{0.388325in}}{\pgfqpoint{0.735980in}{0.392231in}}%
\pgfpathcurveto{\pgfqpoint{0.732073in}{0.396138in}}{\pgfqpoint{0.726774in}{0.398333in}}{\pgfqpoint{0.721249in}{0.398333in}}%
\pgfpathcurveto{\pgfqpoint{0.715724in}{0.398333in}}{\pgfqpoint{0.710424in}{0.396138in}}{\pgfqpoint{0.706518in}{0.392231in}}%
\pgfpathcurveto{\pgfqpoint{0.702611in}{0.388325in}}{\pgfqpoint{0.700416in}{0.383025in}}{\pgfqpoint{0.700416in}{0.377500in}}%
\pgfpathcurveto{\pgfqpoint{0.700416in}{0.371975in}}{\pgfqpoint{0.702611in}{0.366675in}}{\pgfqpoint{0.706518in}{0.362769in}}%
\pgfpathcurveto{\pgfqpoint{0.710424in}{0.358862in}}{\pgfqpoint{0.715724in}{0.356667in}}{\pgfqpoint{0.721249in}{0.356667in}}%
\pgfpathclose%
\pgfusepath{stroke,fill}%
\end{pgfscope}%
\begin{pgfscope}%
\pgfpathrectangle{\pgfqpoint{0.562500in}{0.275000in}}{\pgfqpoint{3.487500in}{1.925000in}}%
\pgfusepath{clip}%
\pgfsetbuttcap%
\pgfsetroundjoin%
\definecolor{currentfill}{rgb}{0.000000,0.000000,0.000000}%
\pgfsetfillcolor{currentfill}%
\pgfsetlinewidth{1.003750pt}%
\definecolor{currentstroke}{rgb}{0.000000,0.000000,0.000000}%
\pgfsetstrokecolor{currentstroke}%
\pgfsetdash{}{0pt}%
\pgfpathmoveto{\pgfqpoint{0.721249in}{0.356667in}}%
\pgfpathcurveto{\pgfqpoint{0.726774in}{0.356667in}}{\pgfqpoint{0.732073in}{0.358862in}}{\pgfqpoint{0.735980in}{0.362769in}}%
\pgfpathcurveto{\pgfqpoint{0.739887in}{0.366675in}}{\pgfqpoint{0.742082in}{0.371975in}}{\pgfqpoint{0.742082in}{0.377500in}}%
\pgfpathcurveto{\pgfqpoint{0.742082in}{0.383025in}}{\pgfqpoint{0.739887in}{0.388325in}}{\pgfqpoint{0.735980in}{0.392231in}}%
\pgfpathcurveto{\pgfqpoint{0.732073in}{0.396138in}}{\pgfqpoint{0.726774in}{0.398333in}}{\pgfqpoint{0.721249in}{0.398333in}}%
\pgfpathcurveto{\pgfqpoint{0.715724in}{0.398333in}}{\pgfqpoint{0.710424in}{0.396138in}}{\pgfqpoint{0.706518in}{0.392231in}}%
\pgfpathcurveto{\pgfqpoint{0.702611in}{0.388325in}}{\pgfqpoint{0.700416in}{0.383025in}}{\pgfqpoint{0.700416in}{0.377500in}}%
\pgfpathcurveto{\pgfqpoint{0.700416in}{0.371975in}}{\pgfqpoint{0.702611in}{0.366675in}}{\pgfqpoint{0.706518in}{0.362769in}}%
\pgfpathcurveto{\pgfqpoint{0.710424in}{0.358862in}}{\pgfqpoint{0.715724in}{0.356667in}}{\pgfqpoint{0.721249in}{0.356667in}}%
\pgfpathclose%
\pgfusepath{stroke,fill}%
\end{pgfscope}%
\begin{pgfscope}%
\pgfpathrectangle{\pgfqpoint{0.562500in}{0.275000in}}{\pgfqpoint{3.487500in}{1.925000in}}%
\pgfusepath{clip}%
\pgfsetbuttcap%
\pgfsetroundjoin%
\definecolor{currentfill}{rgb}{0.000000,0.000000,0.000000}%
\pgfsetfillcolor{currentfill}%
\pgfsetlinewidth{1.003750pt}%
\definecolor{currentstroke}{rgb}{0.000000,0.000000,0.000000}%
\pgfsetstrokecolor{currentstroke}%
\pgfsetdash{}{0pt}%
\pgfpathmoveto{\pgfqpoint{0.721249in}{0.356667in}}%
\pgfpathcurveto{\pgfqpoint{0.726774in}{0.356667in}}{\pgfqpoint{0.732073in}{0.358862in}}{\pgfqpoint{0.735980in}{0.362769in}}%
\pgfpathcurveto{\pgfqpoint{0.739887in}{0.366675in}}{\pgfqpoint{0.742082in}{0.371975in}}{\pgfqpoint{0.742082in}{0.377500in}}%
\pgfpathcurveto{\pgfqpoint{0.742082in}{0.383025in}}{\pgfqpoint{0.739887in}{0.388325in}}{\pgfqpoint{0.735980in}{0.392231in}}%
\pgfpathcurveto{\pgfqpoint{0.732073in}{0.396138in}}{\pgfqpoint{0.726774in}{0.398333in}}{\pgfqpoint{0.721249in}{0.398333in}}%
\pgfpathcurveto{\pgfqpoint{0.715724in}{0.398333in}}{\pgfqpoint{0.710424in}{0.396138in}}{\pgfqpoint{0.706518in}{0.392231in}}%
\pgfpathcurveto{\pgfqpoint{0.702611in}{0.388325in}}{\pgfqpoint{0.700416in}{0.383025in}}{\pgfqpoint{0.700416in}{0.377500in}}%
\pgfpathcurveto{\pgfqpoint{0.700416in}{0.371975in}}{\pgfqpoint{0.702611in}{0.366675in}}{\pgfqpoint{0.706518in}{0.362769in}}%
\pgfpathcurveto{\pgfqpoint{0.710424in}{0.358862in}}{\pgfqpoint{0.715724in}{0.356667in}}{\pgfqpoint{0.721249in}{0.356667in}}%
\pgfpathclose%
\pgfusepath{stroke,fill}%
\end{pgfscope}%
\begin{pgfscope}%
\pgfpathrectangle{\pgfqpoint{0.562500in}{0.275000in}}{\pgfqpoint{3.487500in}{1.925000in}}%
\pgfusepath{clip}%
\pgfsetbuttcap%
\pgfsetroundjoin%
\definecolor{currentfill}{rgb}{0.000000,0.000000,0.000000}%
\pgfsetfillcolor{currentfill}%
\pgfsetlinewidth{1.003750pt}%
\definecolor{currentstroke}{rgb}{0.000000,0.000000,0.000000}%
\pgfsetstrokecolor{currentstroke}%
\pgfsetdash{}{0pt}%
\pgfpathmoveto{\pgfqpoint{0.721249in}{0.356667in}}%
\pgfpathcurveto{\pgfqpoint{0.726774in}{0.356667in}}{\pgfqpoint{0.732073in}{0.358862in}}{\pgfqpoint{0.735980in}{0.362769in}}%
\pgfpathcurveto{\pgfqpoint{0.739887in}{0.366675in}}{\pgfqpoint{0.742082in}{0.371975in}}{\pgfqpoint{0.742082in}{0.377500in}}%
\pgfpathcurveto{\pgfqpoint{0.742082in}{0.383025in}}{\pgfqpoint{0.739887in}{0.388325in}}{\pgfqpoint{0.735980in}{0.392231in}}%
\pgfpathcurveto{\pgfqpoint{0.732073in}{0.396138in}}{\pgfqpoint{0.726774in}{0.398333in}}{\pgfqpoint{0.721249in}{0.398333in}}%
\pgfpathcurveto{\pgfqpoint{0.715724in}{0.398333in}}{\pgfqpoint{0.710424in}{0.396138in}}{\pgfqpoint{0.706518in}{0.392231in}}%
\pgfpathcurveto{\pgfqpoint{0.702611in}{0.388325in}}{\pgfqpoint{0.700416in}{0.383025in}}{\pgfqpoint{0.700416in}{0.377500in}}%
\pgfpathcurveto{\pgfqpoint{0.700416in}{0.371975in}}{\pgfqpoint{0.702611in}{0.366675in}}{\pgfqpoint{0.706518in}{0.362769in}}%
\pgfpathcurveto{\pgfqpoint{0.710424in}{0.358862in}}{\pgfqpoint{0.715724in}{0.356667in}}{\pgfqpoint{0.721249in}{0.356667in}}%
\pgfpathclose%
\pgfusepath{stroke,fill}%
\end{pgfscope}%
\begin{pgfscope}%
\pgfpathrectangle{\pgfqpoint{0.562500in}{0.275000in}}{\pgfqpoint{3.487500in}{1.925000in}}%
\pgfusepath{clip}%
\pgfsetbuttcap%
\pgfsetroundjoin%
\definecolor{currentfill}{rgb}{0.000000,0.000000,0.000000}%
\pgfsetfillcolor{currentfill}%
\pgfsetlinewidth{1.003750pt}%
\definecolor{currentstroke}{rgb}{0.000000,0.000000,0.000000}%
\pgfsetstrokecolor{currentstroke}%
\pgfsetdash{}{0pt}%
\pgfpathmoveto{\pgfqpoint{0.721249in}{0.356667in}}%
\pgfpathcurveto{\pgfqpoint{0.726774in}{0.356667in}}{\pgfqpoint{0.732073in}{0.358862in}}{\pgfqpoint{0.735980in}{0.362769in}}%
\pgfpathcurveto{\pgfqpoint{0.739887in}{0.366675in}}{\pgfqpoint{0.742082in}{0.371975in}}{\pgfqpoint{0.742082in}{0.377500in}}%
\pgfpathcurveto{\pgfqpoint{0.742082in}{0.383025in}}{\pgfqpoint{0.739887in}{0.388325in}}{\pgfqpoint{0.735980in}{0.392231in}}%
\pgfpathcurveto{\pgfqpoint{0.732073in}{0.396138in}}{\pgfqpoint{0.726774in}{0.398333in}}{\pgfqpoint{0.721249in}{0.398333in}}%
\pgfpathcurveto{\pgfqpoint{0.715724in}{0.398333in}}{\pgfqpoint{0.710424in}{0.396138in}}{\pgfqpoint{0.706518in}{0.392231in}}%
\pgfpathcurveto{\pgfqpoint{0.702611in}{0.388325in}}{\pgfqpoint{0.700416in}{0.383025in}}{\pgfqpoint{0.700416in}{0.377500in}}%
\pgfpathcurveto{\pgfqpoint{0.700416in}{0.371975in}}{\pgfqpoint{0.702611in}{0.366675in}}{\pgfqpoint{0.706518in}{0.362769in}}%
\pgfpathcurveto{\pgfqpoint{0.710424in}{0.358862in}}{\pgfqpoint{0.715724in}{0.356667in}}{\pgfqpoint{0.721249in}{0.356667in}}%
\pgfpathclose%
\pgfusepath{stroke,fill}%
\end{pgfscope}%
\begin{pgfscope}%
\pgfpathrectangle{\pgfqpoint{0.562500in}{0.275000in}}{\pgfqpoint{3.487500in}{1.925000in}}%
\pgfusepath{clip}%
\pgfsetbuttcap%
\pgfsetroundjoin%
\definecolor{currentfill}{rgb}{0.000000,0.000000,0.000000}%
\pgfsetfillcolor{currentfill}%
\pgfsetlinewidth{1.003750pt}%
\definecolor{currentstroke}{rgb}{0.000000,0.000000,0.000000}%
\pgfsetstrokecolor{currentstroke}%
\pgfsetdash{}{0pt}%
\pgfpathmoveto{\pgfqpoint{1.772992in}{0.356667in}}%
\pgfpathcurveto{\pgfqpoint{1.778517in}{0.356667in}}{\pgfqpoint{1.783816in}{0.358862in}}{\pgfqpoint{1.787723in}{0.362769in}}%
\pgfpathcurveto{\pgfqpoint{1.791630in}{0.366675in}}{\pgfqpoint{1.793825in}{0.371975in}}{\pgfqpoint{1.793825in}{0.377500in}}%
\pgfpathcurveto{\pgfqpoint{1.793825in}{0.383025in}}{\pgfqpoint{1.791630in}{0.388325in}}{\pgfqpoint{1.787723in}{0.392231in}}%
\pgfpathcurveto{\pgfqpoint{1.783816in}{0.396138in}}{\pgfqpoint{1.778517in}{0.398333in}}{\pgfqpoint{1.772992in}{0.398333in}}%
\pgfpathcurveto{\pgfqpoint{1.767467in}{0.398333in}}{\pgfqpoint{1.762167in}{0.396138in}}{\pgfqpoint{1.758260in}{0.392231in}}%
\pgfpathcurveto{\pgfqpoint{1.754353in}{0.388325in}}{\pgfqpoint{1.752158in}{0.383025in}}{\pgfqpoint{1.752158in}{0.377500in}}%
\pgfpathcurveto{\pgfqpoint{1.752158in}{0.371975in}}{\pgfqpoint{1.754353in}{0.366675in}}{\pgfqpoint{1.758260in}{0.362769in}}%
\pgfpathcurveto{\pgfqpoint{1.762167in}{0.358862in}}{\pgfqpoint{1.767467in}{0.356667in}}{\pgfqpoint{1.772992in}{0.356667in}}%
\pgfpathclose%
\pgfusepath{stroke,fill}%
\end{pgfscope}%
\begin{pgfscope}%
\pgfpathrectangle{\pgfqpoint{0.562500in}{0.275000in}}{\pgfqpoint{3.487500in}{1.925000in}}%
\pgfusepath{clip}%
\pgfsetbuttcap%
\pgfsetroundjoin%
\definecolor{currentfill}{rgb}{0.000000,0.000000,0.000000}%
\pgfsetfillcolor{currentfill}%
\pgfsetlinewidth{1.003750pt}%
\definecolor{currentstroke}{rgb}{0.000000,0.000000,0.000000}%
\pgfsetstrokecolor{currentstroke}%
\pgfsetdash{}{0pt}%
\pgfpathmoveto{\pgfqpoint{1.772992in}{0.356667in}}%
\pgfpathcurveto{\pgfqpoint{1.778517in}{0.356667in}}{\pgfqpoint{1.783816in}{0.358862in}}{\pgfqpoint{1.787723in}{0.362769in}}%
\pgfpathcurveto{\pgfqpoint{1.791630in}{0.366675in}}{\pgfqpoint{1.793825in}{0.371975in}}{\pgfqpoint{1.793825in}{0.377500in}}%
\pgfpathcurveto{\pgfqpoint{1.793825in}{0.383025in}}{\pgfqpoint{1.791630in}{0.388325in}}{\pgfqpoint{1.787723in}{0.392231in}}%
\pgfpathcurveto{\pgfqpoint{1.783816in}{0.396138in}}{\pgfqpoint{1.778517in}{0.398333in}}{\pgfqpoint{1.772992in}{0.398333in}}%
\pgfpathcurveto{\pgfqpoint{1.767467in}{0.398333in}}{\pgfqpoint{1.762167in}{0.396138in}}{\pgfqpoint{1.758260in}{0.392231in}}%
\pgfpathcurveto{\pgfqpoint{1.754353in}{0.388325in}}{\pgfqpoint{1.752158in}{0.383025in}}{\pgfqpoint{1.752158in}{0.377500in}}%
\pgfpathcurveto{\pgfqpoint{1.752158in}{0.371975in}}{\pgfqpoint{1.754353in}{0.366675in}}{\pgfqpoint{1.758260in}{0.362769in}}%
\pgfpathcurveto{\pgfqpoint{1.762167in}{0.358862in}}{\pgfqpoint{1.767467in}{0.356667in}}{\pgfqpoint{1.772992in}{0.356667in}}%
\pgfpathclose%
\pgfusepath{stroke,fill}%
\end{pgfscope}%
\begin{pgfscope}%
\pgfpathrectangle{\pgfqpoint{0.562500in}{0.275000in}}{\pgfqpoint{3.487500in}{1.925000in}}%
\pgfusepath{clip}%
\pgfsetbuttcap%
\pgfsetroundjoin%
\definecolor{currentfill}{rgb}{0.000000,0.000000,0.000000}%
\pgfsetfillcolor{currentfill}%
\pgfsetlinewidth{1.003750pt}%
\definecolor{currentstroke}{rgb}{0.000000,0.000000,0.000000}%
\pgfsetstrokecolor{currentstroke}%
\pgfsetdash{}{0pt}%
\pgfpathmoveto{\pgfqpoint{1.772992in}{0.356667in}}%
\pgfpathcurveto{\pgfqpoint{1.778517in}{0.356667in}}{\pgfqpoint{1.783816in}{0.358862in}}{\pgfqpoint{1.787723in}{0.362769in}}%
\pgfpathcurveto{\pgfqpoint{1.791630in}{0.366675in}}{\pgfqpoint{1.793825in}{0.371975in}}{\pgfqpoint{1.793825in}{0.377500in}}%
\pgfpathcurveto{\pgfqpoint{1.793825in}{0.383025in}}{\pgfqpoint{1.791630in}{0.388325in}}{\pgfqpoint{1.787723in}{0.392231in}}%
\pgfpathcurveto{\pgfqpoint{1.783816in}{0.396138in}}{\pgfqpoint{1.778517in}{0.398333in}}{\pgfqpoint{1.772992in}{0.398333in}}%
\pgfpathcurveto{\pgfqpoint{1.767467in}{0.398333in}}{\pgfqpoint{1.762167in}{0.396138in}}{\pgfqpoint{1.758260in}{0.392231in}}%
\pgfpathcurveto{\pgfqpoint{1.754353in}{0.388325in}}{\pgfqpoint{1.752158in}{0.383025in}}{\pgfqpoint{1.752158in}{0.377500in}}%
\pgfpathcurveto{\pgfqpoint{1.752158in}{0.371975in}}{\pgfqpoint{1.754353in}{0.366675in}}{\pgfqpoint{1.758260in}{0.362769in}}%
\pgfpathcurveto{\pgfqpoint{1.762167in}{0.358862in}}{\pgfqpoint{1.767467in}{0.356667in}}{\pgfqpoint{1.772992in}{0.356667in}}%
\pgfpathclose%
\pgfusepath{stroke,fill}%
\end{pgfscope}%
\begin{pgfscope}%
\pgfpathrectangle{\pgfqpoint{0.562500in}{0.275000in}}{\pgfqpoint{3.487500in}{1.925000in}}%
\pgfusepath{clip}%
\pgfsetbuttcap%
\pgfsetroundjoin%
\definecolor{currentfill}{rgb}{0.000000,0.000000,0.000000}%
\pgfsetfillcolor{currentfill}%
\pgfsetlinewidth{1.003750pt}%
\definecolor{currentstroke}{rgb}{0.000000,0.000000,0.000000}%
\pgfsetstrokecolor{currentstroke}%
\pgfsetdash{}{0pt}%
\pgfpathmoveto{\pgfqpoint{1.772992in}{0.356667in}}%
\pgfpathcurveto{\pgfqpoint{1.778517in}{0.356667in}}{\pgfqpoint{1.783816in}{0.358862in}}{\pgfqpoint{1.787723in}{0.362769in}}%
\pgfpathcurveto{\pgfqpoint{1.791630in}{0.366675in}}{\pgfqpoint{1.793825in}{0.371975in}}{\pgfqpoint{1.793825in}{0.377500in}}%
\pgfpathcurveto{\pgfqpoint{1.793825in}{0.383025in}}{\pgfqpoint{1.791630in}{0.388325in}}{\pgfqpoint{1.787723in}{0.392231in}}%
\pgfpathcurveto{\pgfqpoint{1.783816in}{0.396138in}}{\pgfqpoint{1.778517in}{0.398333in}}{\pgfqpoint{1.772992in}{0.398333in}}%
\pgfpathcurveto{\pgfqpoint{1.767467in}{0.398333in}}{\pgfqpoint{1.762167in}{0.396138in}}{\pgfqpoint{1.758260in}{0.392231in}}%
\pgfpathcurveto{\pgfqpoint{1.754353in}{0.388325in}}{\pgfqpoint{1.752158in}{0.383025in}}{\pgfqpoint{1.752158in}{0.377500in}}%
\pgfpathcurveto{\pgfqpoint{1.752158in}{0.371975in}}{\pgfqpoint{1.754353in}{0.366675in}}{\pgfqpoint{1.758260in}{0.362769in}}%
\pgfpathcurveto{\pgfqpoint{1.762167in}{0.358862in}}{\pgfqpoint{1.767467in}{0.356667in}}{\pgfqpoint{1.772992in}{0.356667in}}%
\pgfpathclose%
\pgfusepath{stroke,fill}%
\end{pgfscope}%
\begin{pgfscope}%
\pgfpathrectangle{\pgfqpoint{0.562500in}{0.275000in}}{\pgfqpoint{3.487500in}{1.925000in}}%
\pgfusepath{clip}%
\pgfsetbuttcap%
\pgfsetroundjoin%
\definecolor{currentfill}{rgb}{0.000000,0.000000,0.000000}%
\pgfsetfillcolor{currentfill}%
\pgfsetlinewidth{1.003750pt}%
\definecolor{currentstroke}{rgb}{0.000000,0.000000,0.000000}%
\pgfsetstrokecolor{currentstroke}%
\pgfsetdash{}{0pt}%
\pgfpathmoveto{\pgfqpoint{1.772992in}{0.356667in}}%
\pgfpathcurveto{\pgfqpoint{1.778517in}{0.356667in}}{\pgfqpoint{1.783816in}{0.358862in}}{\pgfqpoint{1.787723in}{0.362769in}}%
\pgfpathcurveto{\pgfqpoint{1.791630in}{0.366675in}}{\pgfqpoint{1.793825in}{0.371975in}}{\pgfqpoint{1.793825in}{0.377500in}}%
\pgfpathcurveto{\pgfqpoint{1.793825in}{0.383025in}}{\pgfqpoint{1.791630in}{0.388325in}}{\pgfqpoint{1.787723in}{0.392231in}}%
\pgfpathcurveto{\pgfqpoint{1.783816in}{0.396138in}}{\pgfqpoint{1.778517in}{0.398333in}}{\pgfqpoint{1.772992in}{0.398333in}}%
\pgfpathcurveto{\pgfqpoint{1.767467in}{0.398333in}}{\pgfqpoint{1.762167in}{0.396138in}}{\pgfqpoint{1.758260in}{0.392231in}}%
\pgfpathcurveto{\pgfqpoint{1.754353in}{0.388325in}}{\pgfqpoint{1.752158in}{0.383025in}}{\pgfqpoint{1.752158in}{0.377500in}}%
\pgfpathcurveto{\pgfqpoint{1.752158in}{0.371975in}}{\pgfqpoint{1.754353in}{0.366675in}}{\pgfqpoint{1.758260in}{0.362769in}}%
\pgfpathcurveto{\pgfqpoint{1.762167in}{0.358862in}}{\pgfqpoint{1.767467in}{0.356667in}}{\pgfqpoint{1.772992in}{0.356667in}}%
\pgfpathclose%
\pgfusepath{stroke,fill}%
\end{pgfscope}%
\begin{pgfscope}%
\pgfpathrectangle{\pgfqpoint{0.562500in}{0.275000in}}{\pgfqpoint{3.487500in}{1.925000in}}%
\pgfusepath{clip}%
\pgfsetbuttcap%
\pgfsetroundjoin%
\definecolor{currentfill}{rgb}{0.000000,0.000000,0.000000}%
\pgfsetfillcolor{currentfill}%
\pgfsetlinewidth{1.003750pt}%
\definecolor{currentstroke}{rgb}{0.000000,0.000000,0.000000}%
\pgfsetstrokecolor{currentstroke}%
\pgfsetdash{}{0pt}%
\pgfpathmoveto{\pgfqpoint{1.772992in}{0.356667in}}%
\pgfpathcurveto{\pgfqpoint{1.778517in}{0.356667in}}{\pgfqpoint{1.783816in}{0.358862in}}{\pgfqpoint{1.787723in}{0.362769in}}%
\pgfpathcurveto{\pgfqpoint{1.791630in}{0.366675in}}{\pgfqpoint{1.793825in}{0.371975in}}{\pgfqpoint{1.793825in}{0.377500in}}%
\pgfpathcurveto{\pgfqpoint{1.793825in}{0.383025in}}{\pgfqpoint{1.791630in}{0.388325in}}{\pgfqpoint{1.787723in}{0.392231in}}%
\pgfpathcurveto{\pgfqpoint{1.783816in}{0.396138in}}{\pgfqpoint{1.778517in}{0.398333in}}{\pgfqpoint{1.772992in}{0.398333in}}%
\pgfpathcurveto{\pgfqpoint{1.767467in}{0.398333in}}{\pgfqpoint{1.762167in}{0.396138in}}{\pgfqpoint{1.758260in}{0.392231in}}%
\pgfpathcurveto{\pgfqpoint{1.754353in}{0.388325in}}{\pgfqpoint{1.752158in}{0.383025in}}{\pgfqpoint{1.752158in}{0.377500in}}%
\pgfpathcurveto{\pgfqpoint{1.752158in}{0.371975in}}{\pgfqpoint{1.754353in}{0.366675in}}{\pgfqpoint{1.758260in}{0.362769in}}%
\pgfpathcurveto{\pgfqpoint{1.762167in}{0.358862in}}{\pgfqpoint{1.767467in}{0.356667in}}{\pgfqpoint{1.772992in}{0.356667in}}%
\pgfpathclose%
\pgfusepath{stroke,fill}%
\end{pgfscope}%
\begin{pgfscope}%
\pgfpathrectangle{\pgfqpoint{0.562500in}{0.275000in}}{\pgfqpoint{3.487500in}{1.925000in}}%
\pgfusepath{clip}%
\pgfsetbuttcap%
\pgfsetroundjoin%
\definecolor{currentfill}{rgb}{0.000000,0.000000,0.000000}%
\pgfsetfillcolor{currentfill}%
\pgfsetlinewidth{1.003750pt}%
\definecolor{currentstroke}{rgb}{0.000000,0.000000,0.000000}%
\pgfsetstrokecolor{currentstroke}%
\pgfsetdash{}{0pt}%
\pgfpathmoveto{\pgfqpoint{1.772992in}{0.356667in}}%
\pgfpathcurveto{\pgfqpoint{1.778517in}{0.356667in}}{\pgfqpoint{1.783816in}{0.358862in}}{\pgfqpoint{1.787723in}{0.362769in}}%
\pgfpathcurveto{\pgfqpoint{1.791630in}{0.366675in}}{\pgfqpoint{1.793825in}{0.371975in}}{\pgfqpoint{1.793825in}{0.377500in}}%
\pgfpathcurveto{\pgfqpoint{1.793825in}{0.383025in}}{\pgfqpoint{1.791630in}{0.388325in}}{\pgfqpoint{1.787723in}{0.392231in}}%
\pgfpathcurveto{\pgfqpoint{1.783816in}{0.396138in}}{\pgfqpoint{1.778517in}{0.398333in}}{\pgfqpoint{1.772992in}{0.398333in}}%
\pgfpathcurveto{\pgfqpoint{1.767467in}{0.398333in}}{\pgfqpoint{1.762167in}{0.396138in}}{\pgfqpoint{1.758260in}{0.392231in}}%
\pgfpathcurveto{\pgfqpoint{1.754353in}{0.388325in}}{\pgfqpoint{1.752158in}{0.383025in}}{\pgfqpoint{1.752158in}{0.377500in}}%
\pgfpathcurveto{\pgfqpoint{1.752158in}{0.371975in}}{\pgfqpoint{1.754353in}{0.366675in}}{\pgfqpoint{1.758260in}{0.362769in}}%
\pgfpathcurveto{\pgfqpoint{1.762167in}{0.358862in}}{\pgfqpoint{1.767467in}{0.356667in}}{\pgfqpoint{1.772992in}{0.356667in}}%
\pgfpathclose%
\pgfusepath{stroke,fill}%
\end{pgfscope}%
\begin{pgfscope}%
\pgfpathrectangle{\pgfqpoint{0.562500in}{0.275000in}}{\pgfqpoint{3.487500in}{1.925000in}}%
\pgfusepath{clip}%
\pgfsetbuttcap%
\pgfsetroundjoin%
\definecolor{currentfill}{rgb}{0.000000,0.000000,0.000000}%
\pgfsetfillcolor{currentfill}%
\pgfsetlinewidth{1.003750pt}%
\definecolor{currentstroke}{rgb}{0.000000,0.000000,0.000000}%
\pgfsetstrokecolor{currentstroke}%
\pgfsetdash{}{0pt}%
\pgfpathmoveto{\pgfqpoint{1.772992in}{0.356667in}}%
\pgfpathcurveto{\pgfqpoint{1.778517in}{0.356667in}}{\pgfqpoint{1.783816in}{0.358862in}}{\pgfqpoint{1.787723in}{0.362769in}}%
\pgfpathcurveto{\pgfqpoint{1.791630in}{0.366675in}}{\pgfqpoint{1.793825in}{0.371975in}}{\pgfqpoint{1.793825in}{0.377500in}}%
\pgfpathcurveto{\pgfqpoint{1.793825in}{0.383025in}}{\pgfqpoint{1.791630in}{0.388325in}}{\pgfqpoint{1.787723in}{0.392231in}}%
\pgfpathcurveto{\pgfqpoint{1.783816in}{0.396138in}}{\pgfqpoint{1.778517in}{0.398333in}}{\pgfqpoint{1.772992in}{0.398333in}}%
\pgfpathcurveto{\pgfqpoint{1.767467in}{0.398333in}}{\pgfqpoint{1.762167in}{0.396138in}}{\pgfqpoint{1.758260in}{0.392231in}}%
\pgfpathcurveto{\pgfqpoint{1.754353in}{0.388325in}}{\pgfqpoint{1.752158in}{0.383025in}}{\pgfqpoint{1.752158in}{0.377500in}}%
\pgfpathcurveto{\pgfqpoint{1.752158in}{0.371975in}}{\pgfqpoint{1.754353in}{0.366675in}}{\pgfqpoint{1.758260in}{0.362769in}}%
\pgfpathcurveto{\pgfqpoint{1.762167in}{0.358862in}}{\pgfqpoint{1.767467in}{0.356667in}}{\pgfqpoint{1.772992in}{0.356667in}}%
\pgfpathclose%
\pgfusepath{stroke,fill}%
\end{pgfscope}%
\begin{pgfscope}%
\pgfpathrectangle{\pgfqpoint{0.562500in}{0.275000in}}{\pgfqpoint{3.487500in}{1.925000in}}%
\pgfusepath{clip}%
\pgfsetbuttcap%
\pgfsetroundjoin%
\definecolor{currentfill}{rgb}{0.000000,0.000000,0.000000}%
\pgfsetfillcolor{currentfill}%
\pgfsetlinewidth{1.003750pt}%
\definecolor{currentstroke}{rgb}{0.000000,0.000000,0.000000}%
\pgfsetstrokecolor{currentstroke}%
\pgfsetdash{}{0pt}%
\pgfpathmoveto{\pgfqpoint{1.772992in}{0.356667in}}%
\pgfpathcurveto{\pgfqpoint{1.778517in}{0.356667in}}{\pgfqpoint{1.783816in}{0.358862in}}{\pgfqpoint{1.787723in}{0.362769in}}%
\pgfpathcurveto{\pgfqpoint{1.791630in}{0.366675in}}{\pgfqpoint{1.793825in}{0.371975in}}{\pgfqpoint{1.793825in}{0.377500in}}%
\pgfpathcurveto{\pgfqpoint{1.793825in}{0.383025in}}{\pgfqpoint{1.791630in}{0.388325in}}{\pgfqpoint{1.787723in}{0.392231in}}%
\pgfpathcurveto{\pgfqpoint{1.783816in}{0.396138in}}{\pgfqpoint{1.778517in}{0.398333in}}{\pgfqpoint{1.772992in}{0.398333in}}%
\pgfpathcurveto{\pgfqpoint{1.767467in}{0.398333in}}{\pgfqpoint{1.762167in}{0.396138in}}{\pgfqpoint{1.758260in}{0.392231in}}%
\pgfpathcurveto{\pgfqpoint{1.754353in}{0.388325in}}{\pgfqpoint{1.752158in}{0.383025in}}{\pgfqpoint{1.752158in}{0.377500in}}%
\pgfpathcurveto{\pgfqpoint{1.752158in}{0.371975in}}{\pgfqpoint{1.754353in}{0.366675in}}{\pgfqpoint{1.758260in}{0.362769in}}%
\pgfpathcurveto{\pgfqpoint{1.762167in}{0.358862in}}{\pgfqpoint{1.767467in}{0.356667in}}{\pgfqpoint{1.772992in}{0.356667in}}%
\pgfpathclose%
\pgfusepath{stroke,fill}%
\end{pgfscope}%
\begin{pgfscope}%
\pgfpathrectangle{\pgfqpoint{0.562500in}{0.275000in}}{\pgfqpoint{3.487500in}{1.925000in}}%
\pgfusepath{clip}%
\pgfsetbuttcap%
\pgfsetroundjoin%
\definecolor{currentfill}{rgb}{0.000000,0.000000,0.000000}%
\pgfsetfillcolor{currentfill}%
\pgfsetlinewidth{1.003750pt}%
\definecolor{currentstroke}{rgb}{0.000000,0.000000,0.000000}%
\pgfsetstrokecolor{currentstroke}%
\pgfsetdash{}{0pt}%
\pgfpathmoveto{\pgfqpoint{1.772992in}{0.356667in}}%
\pgfpathcurveto{\pgfqpoint{1.778517in}{0.356667in}}{\pgfqpoint{1.783816in}{0.358862in}}{\pgfqpoint{1.787723in}{0.362769in}}%
\pgfpathcurveto{\pgfqpoint{1.791630in}{0.366675in}}{\pgfqpoint{1.793825in}{0.371975in}}{\pgfqpoint{1.793825in}{0.377500in}}%
\pgfpathcurveto{\pgfqpoint{1.793825in}{0.383025in}}{\pgfqpoint{1.791630in}{0.388325in}}{\pgfqpoint{1.787723in}{0.392231in}}%
\pgfpathcurveto{\pgfqpoint{1.783816in}{0.396138in}}{\pgfqpoint{1.778517in}{0.398333in}}{\pgfqpoint{1.772992in}{0.398333in}}%
\pgfpathcurveto{\pgfqpoint{1.767467in}{0.398333in}}{\pgfqpoint{1.762167in}{0.396138in}}{\pgfqpoint{1.758260in}{0.392231in}}%
\pgfpathcurveto{\pgfqpoint{1.754353in}{0.388325in}}{\pgfqpoint{1.752158in}{0.383025in}}{\pgfqpoint{1.752158in}{0.377500in}}%
\pgfpathcurveto{\pgfqpoint{1.752158in}{0.371975in}}{\pgfqpoint{1.754353in}{0.366675in}}{\pgfqpoint{1.758260in}{0.362769in}}%
\pgfpathcurveto{\pgfqpoint{1.762167in}{0.358862in}}{\pgfqpoint{1.767467in}{0.356667in}}{\pgfqpoint{1.772992in}{0.356667in}}%
\pgfpathclose%
\pgfusepath{stroke,fill}%
\end{pgfscope}%
\begin{pgfscope}%
\pgfpathrectangle{\pgfqpoint{0.562500in}{0.275000in}}{\pgfqpoint{3.487500in}{1.925000in}}%
\pgfusepath{clip}%
\pgfsetbuttcap%
\pgfsetroundjoin%
\definecolor{currentfill}{rgb}{0.000000,0.000000,0.000000}%
\pgfsetfillcolor{currentfill}%
\pgfsetlinewidth{1.003750pt}%
\definecolor{currentstroke}{rgb}{0.000000,0.000000,0.000000}%
\pgfsetstrokecolor{currentstroke}%
\pgfsetdash{}{0pt}%
\pgfpathmoveto{\pgfqpoint{1.772992in}{0.356667in}}%
\pgfpathcurveto{\pgfqpoint{1.778517in}{0.356667in}}{\pgfqpoint{1.783816in}{0.358862in}}{\pgfqpoint{1.787723in}{0.362769in}}%
\pgfpathcurveto{\pgfqpoint{1.791630in}{0.366675in}}{\pgfqpoint{1.793825in}{0.371975in}}{\pgfqpoint{1.793825in}{0.377500in}}%
\pgfpathcurveto{\pgfqpoint{1.793825in}{0.383025in}}{\pgfqpoint{1.791630in}{0.388325in}}{\pgfqpoint{1.787723in}{0.392231in}}%
\pgfpathcurveto{\pgfqpoint{1.783816in}{0.396138in}}{\pgfqpoint{1.778517in}{0.398333in}}{\pgfqpoint{1.772992in}{0.398333in}}%
\pgfpathcurveto{\pgfqpoint{1.767467in}{0.398333in}}{\pgfqpoint{1.762167in}{0.396138in}}{\pgfqpoint{1.758260in}{0.392231in}}%
\pgfpathcurveto{\pgfqpoint{1.754353in}{0.388325in}}{\pgfqpoint{1.752158in}{0.383025in}}{\pgfqpoint{1.752158in}{0.377500in}}%
\pgfpathcurveto{\pgfqpoint{1.752158in}{0.371975in}}{\pgfqpoint{1.754353in}{0.366675in}}{\pgfqpoint{1.758260in}{0.362769in}}%
\pgfpathcurveto{\pgfqpoint{1.762167in}{0.358862in}}{\pgfqpoint{1.767467in}{0.356667in}}{\pgfqpoint{1.772992in}{0.356667in}}%
\pgfpathclose%
\pgfusepath{stroke,fill}%
\end{pgfscope}%
\begin{pgfscope}%
\pgfpathrectangle{\pgfqpoint{0.562500in}{0.275000in}}{\pgfqpoint{3.487500in}{1.925000in}}%
\pgfusepath{clip}%
\pgfsetbuttcap%
\pgfsetroundjoin%
\definecolor{currentfill}{rgb}{0.000000,0.000000,0.000000}%
\pgfsetfillcolor{currentfill}%
\pgfsetlinewidth{1.003750pt}%
\definecolor{currentstroke}{rgb}{0.000000,0.000000,0.000000}%
\pgfsetstrokecolor{currentstroke}%
\pgfsetdash{}{0pt}%
\pgfpathmoveto{\pgfqpoint{1.772992in}{0.356667in}}%
\pgfpathcurveto{\pgfqpoint{1.778517in}{0.356667in}}{\pgfqpoint{1.783816in}{0.358862in}}{\pgfqpoint{1.787723in}{0.362769in}}%
\pgfpathcurveto{\pgfqpoint{1.791630in}{0.366675in}}{\pgfqpoint{1.793825in}{0.371975in}}{\pgfqpoint{1.793825in}{0.377500in}}%
\pgfpathcurveto{\pgfqpoint{1.793825in}{0.383025in}}{\pgfqpoint{1.791630in}{0.388325in}}{\pgfqpoint{1.787723in}{0.392231in}}%
\pgfpathcurveto{\pgfqpoint{1.783816in}{0.396138in}}{\pgfqpoint{1.778517in}{0.398333in}}{\pgfqpoint{1.772992in}{0.398333in}}%
\pgfpathcurveto{\pgfqpoint{1.767467in}{0.398333in}}{\pgfqpoint{1.762167in}{0.396138in}}{\pgfqpoint{1.758260in}{0.392231in}}%
\pgfpathcurveto{\pgfqpoint{1.754353in}{0.388325in}}{\pgfqpoint{1.752158in}{0.383025in}}{\pgfqpoint{1.752158in}{0.377500in}}%
\pgfpathcurveto{\pgfqpoint{1.752158in}{0.371975in}}{\pgfqpoint{1.754353in}{0.366675in}}{\pgfqpoint{1.758260in}{0.362769in}}%
\pgfpathcurveto{\pgfqpoint{1.762167in}{0.358862in}}{\pgfqpoint{1.767467in}{0.356667in}}{\pgfqpoint{1.772992in}{0.356667in}}%
\pgfpathclose%
\pgfusepath{stroke,fill}%
\end{pgfscope}%
\begin{pgfscope}%
\pgfpathrectangle{\pgfqpoint{0.562500in}{0.275000in}}{\pgfqpoint{3.487500in}{1.925000in}}%
\pgfusepath{clip}%
\pgfsetbuttcap%
\pgfsetroundjoin%
\definecolor{currentfill}{rgb}{0.000000,0.000000,0.000000}%
\pgfsetfillcolor{currentfill}%
\pgfsetlinewidth{1.003750pt}%
\definecolor{currentstroke}{rgb}{0.000000,0.000000,0.000000}%
\pgfsetstrokecolor{currentstroke}%
\pgfsetdash{}{0pt}%
\pgfpathmoveto{\pgfqpoint{1.772992in}{0.356667in}}%
\pgfpathcurveto{\pgfqpoint{1.778517in}{0.356667in}}{\pgfqpoint{1.783816in}{0.358862in}}{\pgfqpoint{1.787723in}{0.362769in}}%
\pgfpathcurveto{\pgfqpoint{1.791630in}{0.366675in}}{\pgfqpoint{1.793825in}{0.371975in}}{\pgfqpoint{1.793825in}{0.377500in}}%
\pgfpathcurveto{\pgfqpoint{1.793825in}{0.383025in}}{\pgfqpoint{1.791630in}{0.388325in}}{\pgfqpoint{1.787723in}{0.392231in}}%
\pgfpathcurveto{\pgfqpoint{1.783816in}{0.396138in}}{\pgfqpoint{1.778517in}{0.398333in}}{\pgfqpoint{1.772992in}{0.398333in}}%
\pgfpathcurveto{\pgfqpoint{1.767467in}{0.398333in}}{\pgfqpoint{1.762167in}{0.396138in}}{\pgfqpoint{1.758260in}{0.392231in}}%
\pgfpathcurveto{\pgfqpoint{1.754353in}{0.388325in}}{\pgfqpoint{1.752158in}{0.383025in}}{\pgfqpoint{1.752158in}{0.377500in}}%
\pgfpathcurveto{\pgfqpoint{1.752158in}{0.371975in}}{\pgfqpoint{1.754353in}{0.366675in}}{\pgfqpoint{1.758260in}{0.362769in}}%
\pgfpathcurveto{\pgfqpoint{1.762167in}{0.358862in}}{\pgfqpoint{1.767467in}{0.356667in}}{\pgfqpoint{1.772992in}{0.356667in}}%
\pgfpathclose%
\pgfusepath{stroke,fill}%
\end{pgfscope}%
\begin{pgfscope}%
\pgfpathrectangle{\pgfqpoint{0.562500in}{0.275000in}}{\pgfqpoint{3.487500in}{1.925000in}}%
\pgfusepath{clip}%
\pgfsetbuttcap%
\pgfsetroundjoin%
\definecolor{currentfill}{rgb}{0.000000,0.000000,0.000000}%
\pgfsetfillcolor{currentfill}%
\pgfsetlinewidth{1.003750pt}%
\definecolor{currentstroke}{rgb}{0.000000,0.000000,0.000000}%
\pgfsetstrokecolor{currentstroke}%
\pgfsetdash{}{0pt}%
\pgfpathmoveto{\pgfqpoint{1.772992in}{0.356667in}}%
\pgfpathcurveto{\pgfqpoint{1.778517in}{0.356667in}}{\pgfqpoint{1.783816in}{0.358862in}}{\pgfqpoint{1.787723in}{0.362769in}}%
\pgfpathcurveto{\pgfqpoint{1.791630in}{0.366675in}}{\pgfqpoint{1.793825in}{0.371975in}}{\pgfqpoint{1.793825in}{0.377500in}}%
\pgfpathcurveto{\pgfqpoint{1.793825in}{0.383025in}}{\pgfqpoint{1.791630in}{0.388325in}}{\pgfqpoint{1.787723in}{0.392231in}}%
\pgfpathcurveto{\pgfqpoint{1.783816in}{0.396138in}}{\pgfqpoint{1.778517in}{0.398333in}}{\pgfqpoint{1.772992in}{0.398333in}}%
\pgfpathcurveto{\pgfqpoint{1.767467in}{0.398333in}}{\pgfqpoint{1.762167in}{0.396138in}}{\pgfqpoint{1.758260in}{0.392231in}}%
\pgfpathcurveto{\pgfqpoint{1.754353in}{0.388325in}}{\pgfqpoint{1.752158in}{0.383025in}}{\pgfqpoint{1.752158in}{0.377500in}}%
\pgfpathcurveto{\pgfqpoint{1.752158in}{0.371975in}}{\pgfqpoint{1.754353in}{0.366675in}}{\pgfqpoint{1.758260in}{0.362769in}}%
\pgfpathcurveto{\pgfqpoint{1.762167in}{0.358862in}}{\pgfqpoint{1.767467in}{0.356667in}}{\pgfqpoint{1.772992in}{0.356667in}}%
\pgfpathclose%
\pgfusepath{stroke,fill}%
\end{pgfscope}%
\begin{pgfscope}%
\pgfpathrectangle{\pgfqpoint{0.562500in}{0.275000in}}{\pgfqpoint{3.487500in}{1.925000in}}%
\pgfusepath{clip}%
\pgfsetbuttcap%
\pgfsetroundjoin%
\definecolor{currentfill}{rgb}{0.000000,0.000000,0.000000}%
\pgfsetfillcolor{currentfill}%
\pgfsetlinewidth{1.003750pt}%
\definecolor{currentstroke}{rgb}{0.000000,0.000000,0.000000}%
\pgfsetstrokecolor{currentstroke}%
\pgfsetdash{}{0pt}%
\pgfpathmoveto{\pgfqpoint{1.772992in}{0.356667in}}%
\pgfpathcurveto{\pgfqpoint{1.778517in}{0.356667in}}{\pgfqpoint{1.783816in}{0.358862in}}{\pgfqpoint{1.787723in}{0.362769in}}%
\pgfpathcurveto{\pgfqpoint{1.791630in}{0.366675in}}{\pgfqpoint{1.793825in}{0.371975in}}{\pgfqpoint{1.793825in}{0.377500in}}%
\pgfpathcurveto{\pgfqpoint{1.793825in}{0.383025in}}{\pgfqpoint{1.791630in}{0.388325in}}{\pgfqpoint{1.787723in}{0.392231in}}%
\pgfpathcurveto{\pgfqpoint{1.783816in}{0.396138in}}{\pgfqpoint{1.778517in}{0.398333in}}{\pgfqpoint{1.772992in}{0.398333in}}%
\pgfpathcurveto{\pgfqpoint{1.767467in}{0.398333in}}{\pgfqpoint{1.762167in}{0.396138in}}{\pgfqpoint{1.758260in}{0.392231in}}%
\pgfpathcurveto{\pgfqpoint{1.754353in}{0.388325in}}{\pgfqpoint{1.752158in}{0.383025in}}{\pgfqpoint{1.752158in}{0.377500in}}%
\pgfpathcurveto{\pgfqpoint{1.752158in}{0.371975in}}{\pgfqpoint{1.754353in}{0.366675in}}{\pgfqpoint{1.758260in}{0.362769in}}%
\pgfpathcurveto{\pgfqpoint{1.762167in}{0.358862in}}{\pgfqpoint{1.767467in}{0.356667in}}{\pgfqpoint{1.772992in}{0.356667in}}%
\pgfpathclose%
\pgfusepath{stroke,fill}%
\end{pgfscope}%
\begin{pgfscope}%
\pgfpathrectangle{\pgfqpoint{0.562500in}{0.275000in}}{\pgfqpoint{3.487500in}{1.925000in}}%
\pgfusepath{clip}%
\pgfsetbuttcap%
\pgfsetroundjoin%
\definecolor{currentfill}{rgb}{0.000000,0.000000,0.000000}%
\pgfsetfillcolor{currentfill}%
\pgfsetlinewidth{1.003750pt}%
\definecolor{currentstroke}{rgb}{0.000000,0.000000,0.000000}%
\pgfsetstrokecolor{currentstroke}%
\pgfsetdash{}{0pt}%
\pgfpathmoveto{\pgfqpoint{1.772992in}{0.356667in}}%
\pgfpathcurveto{\pgfqpoint{1.778517in}{0.356667in}}{\pgfqpoint{1.783816in}{0.358862in}}{\pgfqpoint{1.787723in}{0.362769in}}%
\pgfpathcurveto{\pgfqpoint{1.791630in}{0.366675in}}{\pgfqpoint{1.793825in}{0.371975in}}{\pgfqpoint{1.793825in}{0.377500in}}%
\pgfpathcurveto{\pgfqpoint{1.793825in}{0.383025in}}{\pgfqpoint{1.791630in}{0.388325in}}{\pgfqpoint{1.787723in}{0.392231in}}%
\pgfpathcurveto{\pgfqpoint{1.783816in}{0.396138in}}{\pgfqpoint{1.778517in}{0.398333in}}{\pgfqpoint{1.772992in}{0.398333in}}%
\pgfpathcurveto{\pgfqpoint{1.767467in}{0.398333in}}{\pgfqpoint{1.762167in}{0.396138in}}{\pgfqpoint{1.758260in}{0.392231in}}%
\pgfpathcurveto{\pgfqpoint{1.754353in}{0.388325in}}{\pgfqpoint{1.752158in}{0.383025in}}{\pgfqpoint{1.752158in}{0.377500in}}%
\pgfpathcurveto{\pgfqpoint{1.752158in}{0.371975in}}{\pgfqpoint{1.754353in}{0.366675in}}{\pgfqpoint{1.758260in}{0.362769in}}%
\pgfpathcurveto{\pgfqpoint{1.762167in}{0.358862in}}{\pgfqpoint{1.767467in}{0.356667in}}{\pgfqpoint{1.772992in}{0.356667in}}%
\pgfpathclose%
\pgfusepath{stroke,fill}%
\end{pgfscope}%
\begin{pgfscope}%
\pgfpathrectangle{\pgfqpoint{0.562500in}{0.275000in}}{\pgfqpoint{3.487500in}{1.925000in}}%
\pgfusepath{clip}%
\pgfsetbuttcap%
\pgfsetroundjoin%
\definecolor{currentfill}{rgb}{0.000000,0.000000,0.000000}%
\pgfsetfillcolor{currentfill}%
\pgfsetlinewidth{1.003750pt}%
\definecolor{currentstroke}{rgb}{0.000000,0.000000,0.000000}%
\pgfsetstrokecolor{currentstroke}%
\pgfsetdash{}{0pt}%
\pgfpathmoveto{\pgfqpoint{1.772992in}{0.356667in}}%
\pgfpathcurveto{\pgfqpoint{1.778517in}{0.356667in}}{\pgfqpoint{1.783816in}{0.358862in}}{\pgfqpoint{1.787723in}{0.362769in}}%
\pgfpathcurveto{\pgfqpoint{1.791630in}{0.366675in}}{\pgfqpoint{1.793825in}{0.371975in}}{\pgfqpoint{1.793825in}{0.377500in}}%
\pgfpathcurveto{\pgfqpoint{1.793825in}{0.383025in}}{\pgfqpoint{1.791630in}{0.388325in}}{\pgfqpoint{1.787723in}{0.392231in}}%
\pgfpathcurveto{\pgfqpoint{1.783816in}{0.396138in}}{\pgfqpoint{1.778517in}{0.398333in}}{\pgfqpoint{1.772992in}{0.398333in}}%
\pgfpathcurveto{\pgfqpoint{1.767467in}{0.398333in}}{\pgfqpoint{1.762167in}{0.396138in}}{\pgfqpoint{1.758260in}{0.392231in}}%
\pgfpathcurveto{\pgfqpoint{1.754353in}{0.388325in}}{\pgfqpoint{1.752158in}{0.383025in}}{\pgfqpoint{1.752158in}{0.377500in}}%
\pgfpathcurveto{\pgfqpoint{1.752158in}{0.371975in}}{\pgfqpoint{1.754353in}{0.366675in}}{\pgfqpoint{1.758260in}{0.362769in}}%
\pgfpathcurveto{\pgfqpoint{1.762167in}{0.358862in}}{\pgfqpoint{1.767467in}{0.356667in}}{\pgfqpoint{1.772992in}{0.356667in}}%
\pgfpathclose%
\pgfusepath{stroke,fill}%
\end{pgfscope}%
\begin{pgfscope}%
\pgfpathrectangle{\pgfqpoint{0.562500in}{0.275000in}}{\pgfqpoint{3.487500in}{1.925000in}}%
\pgfusepath{clip}%
\pgfsetbuttcap%
\pgfsetroundjoin%
\definecolor{currentfill}{rgb}{0.000000,0.000000,0.000000}%
\pgfsetfillcolor{currentfill}%
\pgfsetlinewidth{1.003750pt}%
\definecolor{currentstroke}{rgb}{0.000000,0.000000,0.000000}%
\pgfsetstrokecolor{currentstroke}%
\pgfsetdash{}{0pt}%
\pgfpathmoveto{\pgfqpoint{1.772992in}{0.356667in}}%
\pgfpathcurveto{\pgfqpoint{1.778517in}{0.356667in}}{\pgfqpoint{1.783816in}{0.358862in}}{\pgfqpoint{1.787723in}{0.362769in}}%
\pgfpathcurveto{\pgfqpoint{1.791630in}{0.366675in}}{\pgfqpoint{1.793825in}{0.371975in}}{\pgfqpoint{1.793825in}{0.377500in}}%
\pgfpathcurveto{\pgfqpoint{1.793825in}{0.383025in}}{\pgfqpoint{1.791630in}{0.388325in}}{\pgfqpoint{1.787723in}{0.392231in}}%
\pgfpathcurveto{\pgfqpoint{1.783816in}{0.396138in}}{\pgfqpoint{1.778517in}{0.398333in}}{\pgfqpoint{1.772992in}{0.398333in}}%
\pgfpathcurveto{\pgfqpoint{1.767467in}{0.398333in}}{\pgfqpoint{1.762167in}{0.396138in}}{\pgfqpoint{1.758260in}{0.392231in}}%
\pgfpathcurveto{\pgfqpoint{1.754353in}{0.388325in}}{\pgfqpoint{1.752158in}{0.383025in}}{\pgfqpoint{1.752158in}{0.377500in}}%
\pgfpathcurveto{\pgfqpoint{1.752158in}{0.371975in}}{\pgfqpoint{1.754353in}{0.366675in}}{\pgfqpoint{1.758260in}{0.362769in}}%
\pgfpathcurveto{\pgfqpoint{1.762167in}{0.358862in}}{\pgfqpoint{1.767467in}{0.356667in}}{\pgfqpoint{1.772992in}{0.356667in}}%
\pgfpathclose%
\pgfusepath{stroke,fill}%
\end{pgfscope}%
\begin{pgfscope}%
\pgfpathrectangle{\pgfqpoint{0.562500in}{0.275000in}}{\pgfqpoint{3.487500in}{1.925000in}}%
\pgfusepath{clip}%
\pgfsetbuttcap%
\pgfsetroundjoin%
\definecolor{currentfill}{rgb}{0.000000,0.000000,0.000000}%
\pgfsetfillcolor{currentfill}%
\pgfsetlinewidth{1.003750pt}%
\definecolor{currentstroke}{rgb}{0.000000,0.000000,0.000000}%
\pgfsetstrokecolor{currentstroke}%
\pgfsetdash{}{0pt}%
\pgfpathmoveto{\pgfqpoint{1.772992in}{0.356667in}}%
\pgfpathcurveto{\pgfqpoint{1.778517in}{0.356667in}}{\pgfqpoint{1.783816in}{0.358862in}}{\pgfqpoint{1.787723in}{0.362769in}}%
\pgfpathcurveto{\pgfqpoint{1.791630in}{0.366675in}}{\pgfqpoint{1.793825in}{0.371975in}}{\pgfqpoint{1.793825in}{0.377500in}}%
\pgfpathcurveto{\pgfqpoint{1.793825in}{0.383025in}}{\pgfqpoint{1.791630in}{0.388325in}}{\pgfqpoint{1.787723in}{0.392231in}}%
\pgfpathcurveto{\pgfqpoint{1.783816in}{0.396138in}}{\pgfqpoint{1.778517in}{0.398333in}}{\pgfqpoint{1.772992in}{0.398333in}}%
\pgfpathcurveto{\pgfqpoint{1.767467in}{0.398333in}}{\pgfqpoint{1.762167in}{0.396138in}}{\pgfqpoint{1.758260in}{0.392231in}}%
\pgfpathcurveto{\pgfqpoint{1.754353in}{0.388325in}}{\pgfqpoint{1.752158in}{0.383025in}}{\pgfqpoint{1.752158in}{0.377500in}}%
\pgfpathcurveto{\pgfqpoint{1.752158in}{0.371975in}}{\pgfqpoint{1.754353in}{0.366675in}}{\pgfqpoint{1.758260in}{0.362769in}}%
\pgfpathcurveto{\pgfqpoint{1.762167in}{0.358862in}}{\pgfqpoint{1.767467in}{0.356667in}}{\pgfqpoint{1.772992in}{0.356667in}}%
\pgfpathclose%
\pgfusepath{stroke,fill}%
\end{pgfscope}%
\begin{pgfscope}%
\pgfpathrectangle{\pgfqpoint{0.562500in}{0.275000in}}{\pgfqpoint{3.487500in}{1.925000in}}%
\pgfusepath{clip}%
\pgfsetbuttcap%
\pgfsetroundjoin%
\definecolor{currentfill}{rgb}{0.000000,0.000000,0.000000}%
\pgfsetfillcolor{currentfill}%
\pgfsetlinewidth{1.003750pt}%
\definecolor{currentstroke}{rgb}{0.000000,0.000000,0.000000}%
\pgfsetstrokecolor{currentstroke}%
\pgfsetdash{}{0pt}%
\pgfpathmoveto{\pgfqpoint{1.772992in}{0.356667in}}%
\pgfpathcurveto{\pgfqpoint{1.778517in}{0.356667in}}{\pgfqpoint{1.783816in}{0.358862in}}{\pgfqpoint{1.787723in}{0.362769in}}%
\pgfpathcurveto{\pgfqpoint{1.791630in}{0.366675in}}{\pgfqpoint{1.793825in}{0.371975in}}{\pgfqpoint{1.793825in}{0.377500in}}%
\pgfpathcurveto{\pgfqpoint{1.793825in}{0.383025in}}{\pgfqpoint{1.791630in}{0.388325in}}{\pgfqpoint{1.787723in}{0.392231in}}%
\pgfpathcurveto{\pgfqpoint{1.783816in}{0.396138in}}{\pgfqpoint{1.778517in}{0.398333in}}{\pgfqpoint{1.772992in}{0.398333in}}%
\pgfpathcurveto{\pgfqpoint{1.767467in}{0.398333in}}{\pgfqpoint{1.762167in}{0.396138in}}{\pgfqpoint{1.758260in}{0.392231in}}%
\pgfpathcurveto{\pgfqpoint{1.754353in}{0.388325in}}{\pgfqpoint{1.752158in}{0.383025in}}{\pgfqpoint{1.752158in}{0.377500in}}%
\pgfpathcurveto{\pgfqpoint{1.752158in}{0.371975in}}{\pgfqpoint{1.754353in}{0.366675in}}{\pgfqpoint{1.758260in}{0.362769in}}%
\pgfpathcurveto{\pgfqpoint{1.762167in}{0.358862in}}{\pgfqpoint{1.767467in}{0.356667in}}{\pgfqpoint{1.772992in}{0.356667in}}%
\pgfpathclose%
\pgfusepath{stroke,fill}%
\end{pgfscope}%
\begin{pgfscope}%
\pgfpathrectangle{\pgfqpoint{0.562500in}{0.275000in}}{\pgfqpoint{3.487500in}{1.925000in}}%
\pgfusepath{clip}%
\pgfsetbuttcap%
\pgfsetroundjoin%
\definecolor{currentfill}{rgb}{0.000000,0.000000,0.000000}%
\pgfsetfillcolor{currentfill}%
\pgfsetlinewidth{1.003750pt}%
\definecolor{currentstroke}{rgb}{0.000000,0.000000,0.000000}%
\pgfsetstrokecolor{currentstroke}%
\pgfsetdash{}{0pt}%
\pgfpathmoveto{\pgfqpoint{1.772992in}{0.356667in}}%
\pgfpathcurveto{\pgfqpoint{1.778517in}{0.356667in}}{\pgfqpoint{1.783816in}{0.358862in}}{\pgfqpoint{1.787723in}{0.362769in}}%
\pgfpathcurveto{\pgfqpoint{1.791630in}{0.366675in}}{\pgfqpoint{1.793825in}{0.371975in}}{\pgfqpoint{1.793825in}{0.377500in}}%
\pgfpathcurveto{\pgfqpoint{1.793825in}{0.383025in}}{\pgfqpoint{1.791630in}{0.388325in}}{\pgfqpoint{1.787723in}{0.392231in}}%
\pgfpathcurveto{\pgfqpoint{1.783816in}{0.396138in}}{\pgfqpoint{1.778517in}{0.398333in}}{\pgfqpoint{1.772992in}{0.398333in}}%
\pgfpathcurveto{\pgfqpoint{1.767467in}{0.398333in}}{\pgfqpoint{1.762167in}{0.396138in}}{\pgfqpoint{1.758260in}{0.392231in}}%
\pgfpathcurveto{\pgfqpoint{1.754353in}{0.388325in}}{\pgfqpoint{1.752158in}{0.383025in}}{\pgfqpoint{1.752158in}{0.377500in}}%
\pgfpathcurveto{\pgfqpoint{1.752158in}{0.371975in}}{\pgfqpoint{1.754353in}{0.366675in}}{\pgfqpoint{1.758260in}{0.362769in}}%
\pgfpathcurveto{\pgfqpoint{1.762167in}{0.358862in}}{\pgfqpoint{1.767467in}{0.356667in}}{\pgfqpoint{1.772992in}{0.356667in}}%
\pgfpathclose%
\pgfusepath{stroke,fill}%
\end{pgfscope}%
\begin{pgfscope}%
\pgfpathrectangle{\pgfqpoint{0.562500in}{0.275000in}}{\pgfqpoint{3.487500in}{1.925000in}}%
\pgfusepath{clip}%
\pgfsetbuttcap%
\pgfsetroundjoin%
\definecolor{currentfill}{rgb}{0.000000,0.000000,0.000000}%
\pgfsetfillcolor{currentfill}%
\pgfsetlinewidth{1.003750pt}%
\definecolor{currentstroke}{rgb}{0.000000,0.000000,0.000000}%
\pgfsetstrokecolor{currentstroke}%
\pgfsetdash{}{0pt}%
\pgfpathmoveto{\pgfqpoint{1.772992in}{0.356667in}}%
\pgfpathcurveto{\pgfqpoint{1.778517in}{0.356667in}}{\pgfqpoint{1.783816in}{0.358862in}}{\pgfqpoint{1.787723in}{0.362769in}}%
\pgfpathcurveto{\pgfqpoint{1.791630in}{0.366675in}}{\pgfqpoint{1.793825in}{0.371975in}}{\pgfqpoint{1.793825in}{0.377500in}}%
\pgfpathcurveto{\pgfqpoint{1.793825in}{0.383025in}}{\pgfqpoint{1.791630in}{0.388325in}}{\pgfqpoint{1.787723in}{0.392231in}}%
\pgfpathcurveto{\pgfqpoint{1.783816in}{0.396138in}}{\pgfqpoint{1.778517in}{0.398333in}}{\pgfqpoint{1.772992in}{0.398333in}}%
\pgfpathcurveto{\pgfqpoint{1.767467in}{0.398333in}}{\pgfqpoint{1.762167in}{0.396138in}}{\pgfqpoint{1.758260in}{0.392231in}}%
\pgfpathcurveto{\pgfqpoint{1.754353in}{0.388325in}}{\pgfqpoint{1.752158in}{0.383025in}}{\pgfqpoint{1.752158in}{0.377500in}}%
\pgfpathcurveto{\pgfqpoint{1.752158in}{0.371975in}}{\pgfqpoint{1.754353in}{0.366675in}}{\pgfqpoint{1.758260in}{0.362769in}}%
\pgfpathcurveto{\pgfqpoint{1.762167in}{0.358862in}}{\pgfqpoint{1.767467in}{0.356667in}}{\pgfqpoint{1.772992in}{0.356667in}}%
\pgfpathclose%
\pgfusepath{stroke,fill}%
\end{pgfscope}%
\begin{pgfscope}%
\pgfpathrectangle{\pgfqpoint{0.562500in}{0.275000in}}{\pgfqpoint{3.487500in}{1.925000in}}%
\pgfusepath{clip}%
\pgfsetbuttcap%
\pgfsetroundjoin%
\definecolor{currentfill}{rgb}{0.000000,0.000000,0.000000}%
\pgfsetfillcolor{currentfill}%
\pgfsetlinewidth{1.003750pt}%
\definecolor{currentstroke}{rgb}{0.000000,0.000000,0.000000}%
\pgfsetstrokecolor{currentstroke}%
\pgfsetdash{}{0pt}%
\pgfpathmoveto{\pgfqpoint{1.772992in}{0.356667in}}%
\pgfpathcurveto{\pgfqpoint{1.778517in}{0.356667in}}{\pgfqpoint{1.783816in}{0.358862in}}{\pgfqpoint{1.787723in}{0.362769in}}%
\pgfpathcurveto{\pgfqpoint{1.791630in}{0.366675in}}{\pgfqpoint{1.793825in}{0.371975in}}{\pgfqpoint{1.793825in}{0.377500in}}%
\pgfpathcurveto{\pgfqpoint{1.793825in}{0.383025in}}{\pgfqpoint{1.791630in}{0.388325in}}{\pgfqpoint{1.787723in}{0.392231in}}%
\pgfpathcurveto{\pgfqpoint{1.783816in}{0.396138in}}{\pgfqpoint{1.778517in}{0.398333in}}{\pgfqpoint{1.772992in}{0.398333in}}%
\pgfpathcurveto{\pgfqpoint{1.767467in}{0.398333in}}{\pgfqpoint{1.762167in}{0.396138in}}{\pgfqpoint{1.758260in}{0.392231in}}%
\pgfpathcurveto{\pgfqpoint{1.754353in}{0.388325in}}{\pgfqpoint{1.752158in}{0.383025in}}{\pgfqpoint{1.752158in}{0.377500in}}%
\pgfpathcurveto{\pgfqpoint{1.752158in}{0.371975in}}{\pgfqpoint{1.754353in}{0.366675in}}{\pgfqpoint{1.758260in}{0.362769in}}%
\pgfpathcurveto{\pgfqpoint{1.762167in}{0.358862in}}{\pgfqpoint{1.767467in}{0.356667in}}{\pgfqpoint{1.772992in}{0.356667in}}%
\pgfpathclose%
\pgfusepath{stroke,fill}%
\end{pgfscope}%
\begin{pgfscope}%
\pgfpathrectangle{\pgfqpoint{0.562500in}{0.275000in}}{\pgfqpoint{3.487500in}{1.925000in}}%
\pgfusepath{clip}%
\pgfsetbuttcap%
\pgfsetroundjoin%
\definecolor{currentfill}{rgb}{0.000000,0.000000,0.000000}%
\pgfsetfillcolor{currentfill}%
\pgfsetlinewidth{1.003750pt}%
\definecolor{currentstroke}{rgb}{0.000000,0.000000,0.000000}%
\pgfsetstrokecolor{currentstroke}%
\pgfsetdash{}{0pt}%
\pgfpathmoveto{\pgfqpoint{1.772992in}{0.356667in}}%
\pgfpathcurveto{\pgfqpoint{1.778517in}{0.356667in}}{\pgfqpoint{1.783816in}{0.358862in}}{\pgfqpoint{1.787723in}{0.362769in}}%
\pgfpathcurveto{\pgfqpoint{1.791630in}{0.366675in}}{\pgfqpoint{1.793825in}{0.371975in}}{\pgfqpoint{1.793825in}{0.377500in}}%
\pgfpathcurveto{\pgfqpoint{1.793825in}{0.383025in}}{\pgfqpoint{1.791630in}{0.388325in}}{\pgfqpoint{1.787723in}{0.392231in}}%
\pgfpathcurveto{\pgfqpoint{1.783816in}{0.396138in}}{\pgfqpoint{1.778517in}{0.398333in}}{\pgfqpoint{1.772992in}{0.398333in}}%
\pgfpathcurveto{\pgfqpoint{1.767467in}{0.398333in}}{\pgfqpoint{1.762167in}{0.396138in}}{\pgfqpoint{1.758260in}{0.392231in}}%
\pgfpathcurveto{\pgfqpoint{1.754353in}{0.388325in}}{\pgfqpoint{1.752158in}{0.383025in}}{\pgfqpoint{1.752158in}{0.377500in}}%
\pgfpathcurveto{\pgfqpoint{1.752158in}{0.371975in}}{\pgfqpoint{1.754353in}{0.366675in}}{\pgfqpoint{1.758260in}{0.362769in}}%
\pgfpathcurveto{\pgfqpoint{1.762167in}{0.358862in}}{\pgfqpoint{1.767467in}{0.356667in}}{\pgfqpoint{1.772992in}{0.356667in}}%
\pgfpathclose%
\pgfusepath{stroke,fill}%
\end{pgfscope}%
\begin{pgfscope}%
\pgfpathrectangle{\pgfqpoint{0.562500in}{0.275000in}}{\pgfqpoint{3.487500in}{1.925000in}}%
\pgfusepath{clip}%
\pgfsetbuttcap%
\pgfsetroundjoin%
\definecolor{currentfill}{rgb}{0.000000,0.000000,0.000000}%
\pgfsetfillcolor{currentfill}%
\pgfsetlinewidth{1.003750pt}%
\definecolor{currentstroke}{rgb}{0.000000,0.000000,0.000000}%
\pgfsetstrokecolor{currentstroke}%
\pgfsetdash{}{0pt}%
\pgfpathmoveto{\pgfqpoint{1.772992in}{0.356667in}}%
\pgfpathcurveto{\pgfqpoint{1.778517in}{0.356667in}}{\pgfqpoint{1.783816in}{0.358862in}}{\pgfqpoint{1.787723in}{0.362769in}}%
\pgfpathcurveto{\pgfqpoint{1.791630in}{0.366675in}}{\pgfqpoint{1.793825in}{0.371975in}}{\pgfqpoint{1.793825in}{0.377500in}}%
\pgfpathcurveto{\pgfqpoint{1.793825in}{0.383025in}}{\pgfqpoint{1.791630in}{0.388325in}}{\pgfqpoint{1.787723in}{0.392231in}}%
\pgfpathcurveto{\pgfqpoint{1.783816in}{0.396138in}}{\pgfqpoint{1.778517in}{0.398333in}}{\pgfqpoint{1.772992in}{0.398333in}}%
\pgfpathcurveto{\pgfqpoint{1.767467in}{0.398333in}}{\pgfqpoint{1.762167in}{0.396138in}}{\pgfqpoint{1.758260in}{0.392231in}}%
\pgfpathcurveto{\pgfqpoint{1.754353in}{0.388325in}}{\pgfqpoint{1.752158in}{0.383025in}}{\pgfqpoint{1.752158in}{0.377500in}}%
\pgfpathcurveto{\pgfqpoint{1.752158in}{0.371975in}}{\pgfqpoint{1.754353in}{0.366675in}}{\pgfqpoint{1.758260in}{0.362769in}}%
\pgfpathcurveto{\pgfqpoint{1.762167in}{0.358862in}}{\pgfqpoint{1.767467in}{0.356667in}}{\pgfqpoint{1.772992in}{0.356667in}}%
\pgfpathclose%
\pgfusepath{stroke,fill}%
\end{pgfscope}%
\begin{pgfscope}%
\pgfpathrectangle{\pgfqpoint{0.562500in}{0.275000in}}{\pgfqpoint{3.487500in}{1.925000in}}%
\pgfusepath{clip}%
\pgfsetbuttcap%
\pgfsetroundjoin%
\definecolor{currentfill}{rgb}{0.000000,0.000000,0.000000}%
\pgfsetfillcolor{currentfill}%
\pgfsetlinewidth{1.003750pt}%
\definecolor{currentstroke}{rgb}{0.000000,0.000000,0.000000}%
\pgfsetstrokecolor{currentstroke}%
\pgfsetdash{}{0pt}%
\pgfpathmoveto{\pgfqpoint{1.772992in}{0.356667in}}%
\pgfpathcurveto{\pgfqpoint{1.778517in}{0.356667in}}{\pgfqpoint{1.783816in}{0.358862in}}{\pgfqpoint{1.787723in}{0.362769in}}%
\pgfpathcurveto{\pgfqpoint{1.791630in}{0.366675in}}{\pgfqpoint{1.793825in}{0.371975in}}{\pgfqpoint{1.793825in}{0.377500in}}%
\pgfpathcurveto{\pgfqpoint{1.793825in}{0.383025in}}{\pgfqpoint{1.791630in}{0.388325in}}{\pgfqpoint{1.787723in}{0.392231in}}%
\pgfpathcurveto{\pgfqpoint{1.783816in}{0.396138in}}{\pgfqpoint{1.778517in}{0.398333in}}{\pgfqpoint{1.772992in}{0.398333in}}%
\pgfpathcurveto{\pgfqpoint{1.767467in}{0.398333in}}{\pgfqpoint{1.762167in}{0.396138in}}{\pgfqpoint{1.758260in}{0.392231in}}%
\pgfpathcurveto{\pgfqpoint{1.754353in}{0.388325in}}{\pgfqpoint{1.752158in}{0.383025in}}{\pgfqpoint{1.752158in}{0.377500in}}%
\pgfpathcurveto{\pgfqpoint{1.752158in}{0.371975in}}{\pgfqpoint{1.754353in}{0.366675in}}{\pgfqpoint{1.758260in}{0.362769in}}%
\pgfpathcurveto{\pgfqpoint{1.762167in}{0.358862in}}{\pgfqpoint{1.767467in}{0.356667in}}{\pgfqpoint{1.772992in}{0.356667in}}%
\pgfpathclose%
\pgfusepath{stroke,fill}%
\end{pgfscope}%
\begin{pgfscope}%
\pgfpathrectangle{\pgfqpoint{0.562500in}{0.275000in}}{\pgfqpoint{3.487500in}{1.925000in}}%
\pgfusepath{clip}%
\pgfsetbuttcap%
\pgfsetroundjoin%
\definecolor{currentfill}{rgb}{0.000000,0.000000,0.000000}%
\pgfsetfillcolor{currentfill}%
\pgfsetlinewidth{1.003750pt}%
\definecolor{currentstroke}{rgb}{0.000000,0.000000,0.000000}%
\pgfsetstrokecolor{currentstroke}%
\pgfsetdash{}{0pt}%
\pgfpathmoveto{\pgfqpoint{1.772992in}{0.356667in}}%
\pgfpathcurveto{\pgfqpoint{1.778517in}{0.356667in}}{\pgfqpoint{1.783816in}{0.358862in}}{\pgfqpoint{1.787723in}{0.362769in}}%
\pgfpathcurveto{\pgfqpoint{1.791630in}{0.366675in}}{\pgfqpoint{1.793825in}{0.371975in}}{\pgfqpoint{1.793825in}{0.377500in}}%
\pgfpathcurveto{\pgfqpoint{1.793825in}{0.383025in}}{\pgfqpoint{1.791630in}{0.388325in}}{\pgfqpoint{1.787723in}{0.392231in}}%
\pgfpathcurveto{\pgfqpoint{1.783816in}{0.396138in}}{\pgfqpoint{1.778517in}{0.398333in}}{\pgfqpoint{1.772992in}{0.398333in}}%
\pgfpathcurveto{\pgfqpoint{1.767467in}{0.398333in}}{\pgfqpoint{1.762167in}{0.396138in}}{\pgfqpoint{1.758260in}{0.392231in}}%
\pgfpathcurveto{\pgfqpoint{1.754353in}{0.388325in}}{\pgfqpoint{1.752158in}{0.383025in}}{\pgfqpoint{1.752158in}{0.377500in}}%
\pgfpathcurveto{\pgfqpoint{1.752158in}{0.371975in}}{\pgfqpoint{1.754353in}{0.366675in}}{\pgfqpoint{1.758260in}{0.362769in}}%
\pgfpathcurveto{\pgfqpoint{1.762167in}{0.358862in}}{\pgfqpoint{1.767467in}{0.356667in}}{\pgfqpoint{1.772992in}{0.356667in}}%
\pgfpathclose%
\pgfusepath{stroke,fill}%
\end{pgfscope}%
\begin{pgfscope}%
\pgfpathrectangle{\pgfqpoint{0.562500in}{0.275000in}}{\pgfqpoint{3.487500in}{1.925000in}}%
\pgfusepath{clip}%
\pgfsetbuttcap%
\pgfsetroundjoin%
\definecolor{currentfill}{rgb}{0.000000,0.000000,0.000000}%
\pgfsetfillcolor{currentfill}%
\pgfsetlinewidth{1.003750pt}%
\definecolor{currentstroke}{rgb}{0.000000,0.000000,0.000000}%
\pgfsetstrokecolor{currentstroke}%
\pgfsetdash{}{0pt}%
\pgfpathmoveto{\pgfqpoint{1.772992in}{0.356667in}}%
\pgfpathcurveto{\pgfqpoint{1.778517in}{0.356667in}}{\pgfqpoint{1.783816in}{0.358862in}}{\pgfqpoint{1.787723in}{0.362769in}}%
\pgfpathcurveto{\pgfqpoint{1.791630in}{0.366675in}}{\pgfqpoint{1.793825in}{0.371975in}}{\pgfqpoint{1.793825in}{0.377500in}}%
\pgfpathcurveto{\pgfqpoint{1.793825in}{0.383025in}}{\pgfqpoint{1.791630in}{0.388325in}}{\pgfqpoint{1.787723in}{0.392231in}}%
\pgfpathcurveto{\pgfqpoint{1.783816in}{0.396138in}}{\pgfqpoint{1.778517in}{0.398333in}}{\pgfqpoint{1.772992in}{0.398333in}}%
\pgfpathcurveto{\pgfqpoint{1.767467in}{0.398333in}}{\pgfqpoint{1.762167in}{0.396138in}}{\pgfqpoint{1.758260in}{0.392231in}}%
\pgfpathcurveto{\pgfqpoint{1.754353in}{0.388325in}}{\pgfqpoint{1.752158in}{0.383025in}}{\pgfqpoint{1.752158in}{0.377500in}}%
\pgfpathcurveto{\pgfqpoint{1.752158in}{0.371975in}}{\pgfqpoint{1.754353in}{0.366675in}}{\pgfqpoint{1.758260in}{0.362769in}}%
\pgfpathcurveto{\pgfqpoint{1.762167in}{0.358862in}}{\pgfqpoint{1.767467in}{0.356667in}}{\pgfqpoint{1.772992in}{0.356667in}}%
\pgfpathclose%
\pgfusepath{stroke,fill}%
\end{pgfscope}%
\begin{pgfscope}%
\pgfpathrectangle{\pgfqpoint{0.562500in}{0.275000in}}{\pgfqpoint{3.487500in}{1.925000in}}%
\pgfusepath{clip}%
\pgfsetbuttcap%
\pgfsetroundjoin%
\definecolor{currentfill}{rgb}{0.000000,0.000000,0.000000}%
\pgfsetfillcolor{currentfill}%
\pgfsetlinewidth{1.003750pt}%
\definecolor{currentstroke}{rgb}{0.000000,0.000000,0.000000}%
\pgfsetstrokecolor{currentstroke}%
\pgfsetdash{}{0pt}%
\pgfpathmoveto{\pgfqpoint{1.772992in}{0.356667in}}%
\pgfpathcurveto{\pgfqpoint{1.778517in}{0.356667in}}{\pgfqpoint{1.783816in}{0.358862in}}{\pgfqpoint{1.787723in}{0.362769in}}%
\pgfpathcurveto{\pgfqpoint{1.791630in}{0.366675in}}{\pgfqpoint{1.793825in}{0.371975in}}{\pgfqpoint{1.793825in}{0.377500in}}%
\pgfpathcurveto{\pgfqpoint{1.793825in}{0.383025in}}{\pgfqpoint{1.791630in}{0.388325in}}{\pgfqpoint{1.787723in}{0.392231in}}%
\pgfpathcurveto{\pgfqpoint{1.783816in}{0.396138in}}{\pgfqpoint{1.778517in}{0.398333in}}{\pgfqpoint{1.772992in}{0.398333in}}%
\pgfpathcurveto{\pgfqpoint{1.767467in}{0.398333in}}{\pgfqpoint{1.762167in}{0.396138in}}{\pgfqpoint{1.758260in}{0.392231in}}%
\pgfpathcurveto{\pgfqpoint{1.754353in}{0.388325in}}{\pgfqpoint{1.752158in}{0.383025in}}{\pgfqpoint{1.752158in}{0.377500in}}%
\pgfpathcurveto{\pgfqpoint{1.752158in}{0.371975in}}{\pgfqpoint{1.754353in}{0.366675in}}{\pgfqpoint{1.758260in}{0.362769in}}%
\pgfpathcurveto{\pgfqpoint{1.762167in}{0.358862in}}{\pgfqpoint{1.767467in}{0.356667in}}{\pgfqpoint{1.772992in}{0.356667in}}%
\pgfpathclose%
\pgfusepath{stroke,fill}%
\end{pgfscope}%
\begin{pgfscope}%
\pgfpathrectangle{\pgfqpoint{0.562500in}{0.275000in}}{\pgfqpoint{3.487500in}{1.925000in}}%
\pgfusepath{clip}%
\pgfsetbuttcap%
\pgfsetroundjoin%
\definecolor{currentfill}{rgb}{0.000000,0.000000,0.000000}%
\pgfsetfillcolor{currentfill}%
\pgfsetlinewidth{1.003750pt}%
\definecolor{currentstroke}{rgb}{0.000000,0.000000,0.000000}%
\pgfsetstrokecolor{currentstroke}%
\pgfsetdash{}{0pt}%
\pgfpathmoveto{\pgfqpoint{1.772992in}{0.356667in}}%
\pgfpathcurveto{\pgfqpoint{1.778517in}{0.356667in}}{\pgfqpoint{1.783816in}{0.358862in}}{\pgfqpoint{1.787723in}{0.362769in}}%
\pgfpathcurveto{\pgfqpoint{1.791630in}{0.366675in}}{\pgfqpoint{1.793825in}{0.371975in}}{\pgfqpoint{1.793825in}{0.377500in}}%
\pgfpathcurveto{\pgfqpoint{1.793825in}{0.383025in}}{\pgfqpoint{1.791630in}{0.388325in}}{\pgfqpoint{1.787723in}{0.392231in}}%
\pgfpathcurveto{\pgfqpoint{1.783816in}{0.396138in}}{\pgfqpoint{1.778517in}{0.398333in}}{\pgfqpoint{1.772992in}{0.398333in}}%
\pgfpathcurveto{\pgfqpoint{1.767467in}{0.398333in}}{\pgfqpoint{1.762167in}{0.396138in}}{\pgfqpoint{1.758260in}{0.392231in}}%
\pgfpathcurveto{\pgfqpoint{1.754353in}{0.388325in}}{\pgfqpoint{1.752158in}{0.383025in}}{\pgfqpoint{1.752158in}{0.377500in}}%
\pgfpathcurveto{\pgfqpoint{1.752158in}{0.371975in}}{\pgfqpoint{1.754353in}{0.366675in}}{\pgfqpoint{1.758260in}{0.362769in}}%
\pgfpathcurveto{\pgfqpoint{1.762167in}{0.358862in}}{\pgfqpoint{1.767467in}{0.356667in}}{\pgfqpoint{1.772992in}{0.356667in}}%
\pgfpathclose%
\pgfusepath{stroke,fill}%
\end{pgfscope}%
\begin{pgfscope}%
\pgfpathrectangle{\pgfqpoint{0.562500in}{0.275000in}}{\pgfqpoint{3.487500in}{1.925000in}}%
\pgfusepath{clip}%
\pgfsetbuttcap%
\pgfsetroundjoin%
\definecolor{currentfill}{rgb}{0.000000,0.000000,0.000000}%
\pgfsetfillcolor{currentfill}%
\pgfsetlinewidth{1.003750pt}%
\definecolor{currentstroke}{rgb}{0.000000,0.000000,0.000000}%
\pgfsetstrokecolor{currentstroke}%
\pgfsetdash{}{0pt}%
\pgfpathmoveto{\pgfqpoint{1.772992in}{0.356667in}}%
\pgfpathcurveto{\pgfqpoint{1.778517in}{0.356667in}}{\pgfqpoint{1.783816in}{0.358862in}}{\pgfqpoint{1.787723in}{0.362769in}}%
\pgfpathcurveto{\pgfqpoint{1.791630in}{0.366675in}}{\pgfqpoint{1.793825in}{0.371975in}}{\pgfqpoint{1.793825in}{0.377500in}}%
\pgfpathcurveto{\pgfqpoint{1.793825in}{0.383025in}}{\pgfqpoint{1.791630in}{0.388325in}}{\pgfqpoint{1.787723in}{0.392231in}}%
\pgfpathcurveto{\pgfqpoint{1.783816in}{0.396138in}}{\pgfqpoint{1.778517in}{0.398333in}}{\pgfqpoint{1.772992in}{0.398333in}}%
\pgfpathcurveto{\pgfqpoint{1.767467in}{0.398333in}}{\pgfqpoint{1.762167in}{0.396138in}}{\pgfqpoint{1.758260in}{0.392231in}}%
\pgfpathcurveto{\pgfqpoint{1.754353in}{0.388325in}}{\pgfqpoint{1.752158in}{0.383025in}}{\pgfqpoint{1.752158in}{0.377500in}}%
\pgfpathcurveto{\pgfqpoint{1.752158in}{0.371975in}}{\pgfqpoint{1.754353in}{0.366675in}}{\pgfqpoint{1.758260in}{0.362769in}}%
\pgfpathcurveto{\pgfqpoint{1.762167in}{0.358862in}}{\pgfqpoint{1.767467in}{0.356667in}}{\pgfqpoint{1.772992in}{0.356667in}}%
\pgfpathclose%
\pgfusepath{stroke,fill}%
\end{pgfscope}%
\begin{pgfscope}%
\pgfpathrectangle{\pgfqpoint{0.562500in}{0.275000in}}{\pgfqpoint{3.487500in}{1.925000in}}%
\pgfusepath{clip}%
\pgfsetbuttcap%
\pgfsetroundjoin%
\definecolor{currentfill}{rgb}{0.000000,0.000000,0.000000}%
\pgfsetfillcolor{currentfill}%
\pgfsetlinewidth{1.003750pt}%
\definecolor{currentstroke}{rgb}{0.000000,0.000000,0.000000}%
\pgfsetstrokecolor{currentstroke}%
\pgfsetdash{}{0pt}%
\pgfpathmoveto{\pgfqpoint{1.772992in}{0.356667in}}%
\pgfpathcurveto{\pgfqpoint{1.778517in}{0.356667in}}{\pgfqpoint{1.783816in}{0.358862in}}{\pgfqpoint{1.787723in}{0.362769in}}%
\pgfpathcurveto{\pgfqpoint{1.791630in}{0.366675in}}{\pgfqpoint{1.793825in}{0.371975in}}{\pgfqpoint{1.793825in}{0.377500in}}%
\pgfpathcurveto{\pgfqpoint{1.793825in}{0.383025in}}{\pgfqpoint{1.791630in}{0.388325in}}{\pgfqpoint{1.787723in}{0.392231in}}%
\pgfpathcurveto{\pgfqpoint{1.783816in}{0.396138in}}{\pgfqpoint{1.778517in}{0.398333in}}{\pgfqpoint{1.772992in}{0.398333in}}%
\pgfpathcurveto{\pgfqpoint{1.767467in}{0.398333in}}{\pgfqpoint{1.762167in}{0.396138in}}{\pgfqpoint{1.758260in}{0.392231in}}%
\pgfpathcurveto{\pgfqpoint{1.754353in}{0.388325in}}{\pgfqpoint{1.752158in}{0.383025in}}{\pgfqpoint{1.752158in}{0.377500in}}%
\pgfpathcurveto{\pgfqpoint{1.752158in}{0.371975in}}{\pgfqpoint{1.754353in}{0.366675in}}{\pgfqpoint{1.758260in}{0.362769in}}%
\pgfpathcurveto{\pgfqpoint{1.762167in}{0.358862in}}{\pgfqpoint{1.767467in}{0.356667in}}{\pgfqpoint{1.772992in}{0.356667in}}%
\pgfpathclose%
\pgfusepath{stroke,fill}%
\end{pgfscope}%
\begin{pgfscope}%
\pgfpathrectangle{\pgfqpoint{0.562500in}{0.275000in}}{\pgfqpoint{3.487500in}{1.925000in}}%
\pgfusepath{clip}%
\pgfsetbuttcap%
\pgfsetroundjoin%
\definecolor{currentfill}{rgb}{0.000000,0.000000,0.000000}%
\pgfsetfillcolor{currentfill}%
\pgfsetlinewidth{1.003750pt}%
\definecolor{currentstroke}{rgb}{0.000000,0.000000,0.000000}%
\pgfsetstrokecolor{currentstroke}%
\pgfsetdash{}{0pt}%
\pgfpathmoveto{\pgfqpoint{1.772992in}{0.356667in}}%
\pgfpathcurveto{\pgfqpoint{1.778517in}{0.356667in}}{\pgfqpoint{1.783816in}{0.358862in}}{\pgfqpoint{1.787723in}{0.362769in}}%
\pgfpathcurveto{\pgfqpoint{1.791630in}{0.366675in}}{\pgfqpoint{1.793825in}{0.371975in}}{\pgfqpoint{1.793825in}{0.377500in}}%
\pgfpathcurveto{\pgfqpoint{1.793825in}{0.383025in}}{\pgfqpoint{1.791630in}{0.388325in}}{\pgfqpoint{1.787723in}{0.392231in}}%
\pgfpathcurveto{\pgfqpoint{1.783816in}{0.396138in}}{\pgfqpoint{1.778517in}{0.398333in}}{\pgfqpoint{1.772992in}{0.398333in}}%
\pgfpathcurveto{\pgfqpoint{1.767467in}{0.398333in}}{\pgfqpoint{1.762167in}{0.396138in}}{\pgfqpoint{1.758260in}{0.392231in}}%
\pgfpathcurveto{\pgfqpoint{1.754353in}{0.388325in}}{\pgfqpoint{1.752158in}{0.383025in}}{\pgfqpoint{1.752158in}{0.377500in}}%
\pgfpathcurveto{\pgfqpoint{1.752158in}{0.371975in}}{\pgfqpoint{1.754353in}{0.366675in}}{\pgfqpoint{1.758260in}{0.362769in}}%
\pgfpathcurveto{\pgfqpoint{1.762167in}{0.358862in}}{\pgfqpoint{1.767467in}{0.356667in}}{\pgfqpoint{1.772992in}{0.356667in}}%
\pgfpathclose%
\pgfusepath{stroke,fill}%
\end{pgfscope}%
\begin{pgfscope}%
\pgfpathrectangle{\pgfqpoint{0.562500in}{0.275000in}}{\pgfqpoint{3.487500in}{1.925000in}}%
\pgfusepath{clip}%
\pgfsetbuttcap%
\pgfsetroundjoin%
\definecolor{currentfill}{rgb}{0.000000,0.000000,0.000000}%
\pgfsetfillcolor{currentfill}%
\pgfsetlinewidth{1.003750pt}%
\definecolor{currentstroke}{rgb}{0.000000,0.000000,0.000000}%
\pgfsetstrokecolor{currentstroke}%
\pgfsetdash{}{0pt}%
\pgfpathmoveto{\pgfqpoint{1.772992in}{0.356667in}}%
\pgfpathcurveto{\pgfqpoint{1.778517in}{0.356667in}}{\pgfqpoint{1.783816in}{0.358862in}}{\pgfqpoint{1.787723in}{0.362769in}}%
\pgfpathcurveto{\pgfqpoint{1.791630in}{0.366675in}}{\pgfqpoint{1.793825in}{0.371975in}}{\pgfqpoint{1.793825in}{0.377500in}}%
\pgfpathcurveto{\pgfqpoint{1.793825in}{0.383025in}}{\pgfqpoint{1.791630in}{0.388325in}}{\pgfqpoint{1.787723in}{0.392231in}}%
\pgfpathcurveto{\pgfqpoint{1.783816in}{0.396138in}}{\pgfqpoint{1.778517in}{0.398333in}}{\pgfqpoint{1.772992in}{0.398333in}}%
\pgfpathcurveto{\pgfqpoint{1.767467in}{0.398333in}}{\pgfqpoint{1.762167in}{0.396138in}}{\pgfqpoint{1.758260in}{0.392231in}}%
\pgfpathcurveto{\pgfqpoint{1.754353in}{0.388325in}}{\pgfqpoint{1.752158in}{0.383025in}}{\pgfqpoint{1.752158in}{0.377500in}}%
\pgfpathcurveto{\pgfqpoint{1.752158in}{0.371975in}}{\pgfqpoint{1.754353in}{0.366675in}}{\pgfqpoint{1.758260in}{0.362769in}}%
\pgfpathcurveto{\pgfqpoint{1.762167in}{0.358862in}}{\pgfqpoint{1.767467in}{0.356667in}}{\pgfqpoint{1.772992in}{0.356667in}}%
\pgfpathclose%
\pgfusepath{stroke,fill}%
\end{pgfscope}%
\begin{pgfscope}%
\pgfpathrectangle{\pgfqpoint{0.562500in}{0.275000in}}{\pgfqpoint{3.487500in}{1.925000in}}%
\pgfusepath{clip}%
\pgfsetbuttcap%
\pgfsetroundjoin%
\definecolor{currentfill}{rgb}{0.000000,0.000000,0.000000}%
\pgfsetfillcolor{currentfill}%
\pgfsetlinewidth{1.003750pt}%
\definecolor{currentstroke}{rgb}{0.000000,0.000000,0.000000}%
\pgfsetstrokecolor{currentstroke}%
\pgfsetdash{}{0pt}%
\pgfpathmoveto{\pgfqpoint{1.772992in}{0.356667in}}%
\pgfpathcurveto{\pgfqpoint{1.778517in}{0.356667in}}{\pgfqpoint{1.783816in}{0.358862in}}{\pgfqpoint{1.787723in}{0.362769in}}%
\pgfpathcurveto{\pgfqpoint{1.791630in}{0.366675in}}{\pgfqpoint{1.793825in}{0.371975in}}{\pgfqpoint{1.793825in}{0.377500in}}%
\pgfpathcurveto{\pgfqpoint{1.793825in}{0.383025in}}{\pgfqpoint{1.791630in}{0.388325in}}{\pgfqpoint{1.787723in}{0.392231in}}%
\pgfpathcurveto{\pgfqpoint{1.783816in}{0.396138in}}{\pgfqpoint{1.778517in}{0.398333in}}{\pgfqpoint{1.772992in}{0.398333in}}%
\pgfpathcurveto{\pgfqpoint{1.767467in}{0.398333in}}{\pgfqpoint{1.762167in}{0.396138in}}{\pgfqpoint{1.758260in}{0.392231in}}%
\pgfpathcurveto{\pgfqpoint{1.754353in}{0.388325in}}{\pgfqpoint{1.752158in}{0.383025in}}{\pgfqpoint{1.752158in}{0.377500in}}%
\pgfpathcurveto{\pgfqpoint{1.752158in}{0.371975in}}{\pgfqpoint{1.754353in}{0.366675in}}{\pgfqpoint{1.758260in}{0.362769in}}%
\pgfpathcurveto{\pgfqpoint{1.762167in}{0.358862in}}{\pgfqpoint{1.767467in}{0.356667in}}{\pgfqpoint{1.772992in}{0.356667in}}%
\pgfpathclose%
\pgfusepath{stroke,fill}%
\end{pgfscope}%
\begin{pgfscope}%
\pgfpathrectangle{\pgfqpoint{0.562500in}{0.275000in}}{\pgfqpoint{3.487500in}{1.925000in}}%
\pgfusepath{clip}%
\pgfsetbuttcap%
\pgfsetroundjoin%
\definecolor{currentfill}{rgb}{0.000000,0.000000,0.000000}%
\pgfsetfillcolor{currentfill}%
\pgfsetlinewidth{1.003750pt}%
\definecolor{currentstroke}{rgb}{0.000000,0.000000,0.000000}%
\pgfsetstrokecolor{currentstroke}%
\pgfsetdash{}{0pt}%
\pgfpathmoveto{\pgfqpoint{1.772992in}{0.356667in}}%
\pgfpathcurveto{\pgfqpoint{1.778517in}{0.356667in}}{\pgfqpoint{1.783816in}{0.358862in}}{\pgfqpoint{1.787723in}{0.362769in}}%
\pgfpathcurveto{\pgfqpoint{1.791630in}{0.366675in}}{\pgfqpoint{1.793825in}{0.371975in}}{\pgfqpoint{1.793825in}{0.377500in}}%
\pgfpathcurveto{\pgfqpoint{1.793825in}{0.383025in}}{\pgfqpoint{1.791630in}{0.388325in}}{\pgfqpoint{1.787723in}{0.392231in}}%
\pgfpathcurveto{\pgfqpoint{1.783816in}{0.396138in}}{\pgfqpoint{1.778517in}{0.398333in}}{\pgfqpoint{1.772992in}{0.398333in}}%
\pgfpathcurveto{\pgfqpoint{1.767467in}{0.398333in}}{\pgfqpoint{1.762167in}{0.396138in}}{\pgfqpoint{1.758260in}{0.392231in}}%
\pgfpathcurveto{\pgfqpoint{1.754353in}{0.388325in}}{\pgfqpoint{1.752158in}{0.383025in}}{\pgfqpoint{1.752158in}{0.377500in}}%
\pgfpathcurveto{\pgfqpoint{1.752158in}{0.371975in}}{\pgfqpoint{1.754353in}{0.366675in}}{\pgfqpoint{1.758260in}{0.362769in}}%
\pgfpathcurveto{\pgfqpoint{1.762167in}{0.358862in}}{\pgfqpoint{1.767467in}{0.356667in}}{\pgfqpoint{1.772992in}{0.356667in}}%
\pgfpathclose%
\pgfusepath{stroke,fill}%
\end{pgfscope}%
\begin{pgfscope}%
\pgfpathrectangle{\pgfqpoint{0.562500in}{0.275000in}}{\pgfqpoint{3.487500in}{1.925000in}}%
\pgfusepath{clip}%
\pgfsetbuttcap%
\pgfsetroundjoin%
\definecolor{currentfill}{rgb}{0.000000,0.000000,0.000000}%
\pgfsetfillcolor{currentfill}%
\pgfsetlinewidth{1.003750pt}%
\definecolor{currentstroke}{rgb}{0.000000,0.000000,0.000000}%
\pgfsetstrokecolor{currentstroke}%
\pgfsetdash{}{0pt}%
\pgfpathmoveto{\pgfqpoint{1.772992in}{0.356667in}}%
\pgfpathcurveto{\pgfqpoint{1.778517in}{0.356667in}}{\pgfqpoint{1.783816in}{0.358862in}}{\pgfqpoint{1.787723in}{0.362769in}}%
\pgfpathcurveto{\pgfqpoint{1.791630in}{0.366675in}}{\pgfqpoint{1.793825in}{0.371975in}}{\pgfqpoint{1.793825in}{0.377500in}}%
\pgfpathcurveto{\pgfqpoint{1.793825in}{0.383025in}}{\pgfqpoint{1.791630in}{0.388325in}}{\pgfqpoint{1.787723in}{0.392231in}}%
\pgfpathcurveto{\pgfqpoint{1.783816in}{0.396138in}}{\pgfqpoint{1.778517in}{0.398333in}}{\pgfqpoint{1.772992in}{0.398333in}}%
\pgfpathcurveto{\pgfqpoint{1.767467in}{0.398333in}}{\pgfqpoint{1.762167in}{0.396138in}}{\pgfqpoint{1.758260in}{0.392231in}}%
\pgfpathcurveto{\pgfqpoint{1.754353in}{0.388325in}}{\pgfqpoint{1.752158in}{0.383025in}}{\pgfqpoint{1.752158in}{0.377500in}}%
\pgfpathcurveto{\pgfqpoint{1.752158in}{0.371975in}}{\pgfqpoint{1.754353in}{0.366675in}}{\pgfqpoint{1.758260in}{0.362769in}}%
\pgfpathcurveto{\pgfqpoint{1.762167in}{0.358862in}}{\pgfqpoint{1.767467in}{0.356667in}}{\pgfqpoint{1.772992in}{0.356667in}}%
\pgfpathclose%
\pgfusepath{stroke,fill}%
\end{pgfscope}%
\begin{pgfscope}%
\pgfpathrectangle{\pgfqpoint{0.562500in}{0.275000in}}{\pgfqpoint{3.487500in}{1.925000in}}%
\pgfusepath{clip}%
\pgfsetbuttcap%
\pgfsetroundjoin%
\definecolor{currentfill}{rgb}{0.000000,0.000000,0.000000}%
\pgfsetfillcolor{currentfill}%
\pgfsetlinewidth{1.003750pt}%
\definecolor{currentstroke}{rgb}{0.000000,0.000000,0.000000}%
\pgfsetstrokecolor{currentstroke}%
\pgfsetdash{}{0pt}%
\pgfpathmoveto{\pgfqpoint{1.772992in}{0.356667in}}%
\pgfpathcurveto{\pgfqpoint{1.778517in}{0.356667in}}{\pgfqpoint{1.783816in}{0.358862in}}{\pgfqpoint{1.787723in}{0.362769in}}%
\pgfpathcurveto{\pgfqpoint{1.791630in}{0.366675in}}{\pgfqpoint{1.793825in}{0.371975in}}{\pgfqpoint{1.793825in}{0.377500in}}%
\pgfpathcurveto{\pgfqpoint{1.793825in}{0.383025in}}{\pgfqpoint{1.791630in}{0.388325in}}{\pgfqpoint{1.787723in}{0.392231in}}%
\pgfpathcurveto{\pgfqpoint{1.783816in}{0.396138in}}{\pgfqpoint{1.778517in}{0.398333in}}{\pgfqpoint{1.772992in}{0.398333in}}%
\pgfpathcurveto{\pgfqpoint{1.767467in}{0.398333in}}{\pgfqpoint{1.762167in}{0.396138in}}{\pgfqpoint{1.758260in}{0.392231in}}%
\pgfpathcurveto{\pgfqpoint{1.754353in}{0.388325in}}{\pgfqpoint{1.752158in}{0.383025in}}{\pgfqpoint{1.752158in}{0.377500in}}%
\pgfpathcurveto{\pgfqpoint{1.752158in}{0.371975in}}{\pgfqpoint{1.754353in}{0.366675in}}{\pgfqpoint{1.758260in}{0.362769in}}%
\pgfpathcurveto{\pgfqpoint{1.762167in}{0.358862in}}{\pgfqpoint{1.767467in}{0.356667in}}{\pgfqpoint{1.772992in}{0.356667in}}%
\pgfpathclose%
\pgfusepath{stroke,fill}%
\end{pgfscope}%
\begin{pgfscope}%
\pgfpathrectangle{\pgfqpoint{0.562500in}{0.275000in}}{\pgfqpoint{3.487500in}{1.925000in}}%
\pgfusepath{clip}%
\pgfsetbuttcap%
\pgfsetroundjoin%
\definecolor{currentfill}{rgb}{0.000000,0.000000,0.000000}%
\pgfsetfillcolor{currentfill}%
\pgfsetlinewidth{1.003750pt}%
\definecolor{currentstroke}{rgb}{0.000000,0.000000,0.000000}%
\pgfsetstrokecolor{currentstroke}%
\pgfsetdash{}{0pt}%
\pgfpathmoveto{\pgfqpoint{1.772992in}{0.356667in}}%
\pgfpathcurveto{\pgfqpoint{1.778517in}{0.356667in}}{\pgfqpoint{1.783816in}{0.358862in}}{\pgfqpoint{1.787723in}{0.362769in}}%
\pgfpathcurveto{\pgfqpoint{1.791630in}{0.366675in}}{\pgfqpoint{1.793825in}{0.371975in}}{\pgfqpoint{1.793825in}{0.377500in}}%
\pgfpathcurveto{\pgfqpoint{1.793825in}{0.383025in}}{\pgfqpoint{1.791630in}{0.388325in}}{\pgfqpoint{1.787723in}{0.392231in}}%
\pgfpathcurveto{\pgfqpoint{1.783816in}{0.396138in}}{\pgfqpoint{1.778517in}{0.398333in}}{\pgfqpoint{1.772992in}{0.398333in}}%
\pgfpathcurveto{\pgfqpoint{1.767467in}{0.398333in}}{\pgfqpoint{1.762167in}{0.396138in}}{\pgfqpoint{1.758260in}{0.392231in}}%
\pgfpathcurveto{\pgfqpoint{1.754353in}{0.388325in}}{\pgfqpoint{1.752158in}{0.383025in}}{\pgfqpoint{1.752158in}{0.377500in}}%
\pgfpathcurveto{\pgfqpoint{1.752158in}{0.371975in}}{\pgfqpoint{1.754353in}{0.366675in}}{\pgfqpoint{1.758260in}{0.362769in}}%
\pgfpathcurveto{\pgfqpoint{1.762167in}{0.358862in}}{\pgfqpoint{1.767467in}{0.356667in}}{\pgfqpoint{1.772992in}{0.356667in}}%
\pgfpathclose%
\pgfusepath{stroke,fill}%
\end{pgfscope}%
\begin{pgfscope}%
\pgfpathrectangle{\pgfqpoint{0.562500in}{0.275000in}}{\pgfqpoint{3.487500in}{1.925000in}}%
\pgfusepath{clip}%
\pgfsetbuttcap%
\pgfsetroundjoin%
\definecolor{currentfill}{rgb}{0.000000,0.000000,0.000000}%
\pgfsetfillcolor{currentfill}%
\pgfsetlinewidth{1.003750pt}%
\definecolor{currentstroke}{rgb}{0.000000,0.000000,0.000000}%
\pgfsetstrokecolor{currentstroke}%
\pgfsetdash{}{0pt}%
\pgfpathmoveto{\pgfqpoint{1.772992in}{0.356667in}}%
\pgfpathcurveto{\pgfqpoint{1.778517in}{0.356667in}}{\pgfqpoint{1.783816in}{0.358862in}}{\pgfqpoint{1.787723in}{0.362769in}}%
\pgfpathcurveto{\pgfqpoint{1.791630in}{0.366675in}}{\pgfqpoint{1.793825in}{0.371975in}}{\pgfqpoint{1.793825in}{0.377500in}}%
\pgfpathcurveto{\pgfqpoint{1.793825in}{0.383025in}}{\pgfqpoint{1.791630in}{0.388325in}}{\pgfqpoint{1.787723in}{0.392231in}}%
\pgfpathcurveto{\pgfqpoint{1.783816in}{0.396138in}}{\pgfqpoint{1.778517in}{0.398333in}}{\pgfqpoint{1.772992in}{0.398333in}}%
\pgfpathcurveto{\pgfqpoint{1.767467in}{0.398333in}}{\pgfqpoint{1.762167in}{0.396138in}}{\pgfqpoint{1.758260in}{0.392231in}}%
\pgfpathcurveto{\pgfqpoint{1.754353in}{0.388325in}}{\pgfqpoint{1.752158in}{0.383025in}}{\pgfqpoint{1.752158in}{0.377500in}}%
\pgfpathcurveto{\pgfqpoint{1.752158in}{0.371975in}}{\pgfqpoint{1.754353in}{0.366675in}}{\pgfqpoint{1.758260in}{0.362769in}}%
\pgfpathcurveto{\pgfqpoint{1.762167in}{0.358862in}}{\pgfqpoint{1.767467in}{0.356667in}}{\pgfqpoint{1.772992in}{0.356667in}}%
\pgfpathclose%
\pgfusepath{stroke,fill}%
\end{pgfscope}%
\begin{pgfscope}%
\pgfpathrectangle{\pgfqpoint{0.562500in}{0.275000in}}{\pgfqpoint{3.487500in}{1.925000in}}%
\pgfusepath{clip}%
\pgfsetbuttcap%
\pgfsetroundjoin%
\definecolor{currentfill}{rgb}{0.000000,0.000000,0.000000}%
\pgfsetfillcolor{currentfill}%
\pgfsetlinewidth{1.003750pt}%
\definecolor{currentstroke}{rgb}{0.000000,0.000000,0.000000}%
\pgfsetstrokecolor{currentstroke}%
\pgfsetdash{}{0pt}%
\pgfpathmoveto{\pgfqpoint{1.772992in}{0.356667in}}%
\pgfpathcurveto{\pgfqpoint{1.778517in}{0.356667in}}{\pgfqpoint{1.783816in}{0.358862in}}{\pgfqpoint{1.787723in}{0.362769in}}%
\pgfpathcurveto{\pgfqpoint{1.791630in}{0.366675in}}{\pgfqpoint{1.793825in}{0.371975in}}{\pgfqpoint{1.793825in}{0.377500in}}%
\pgfpathcurveto{\pgfqpoint{1.793825in}{0.383025in}}{\pgfqpoint{1.791630in}{0.388325in}}{\pgfqpoint{1.787723in}{0.392231in}}%
\pgfpathcurveto{\pgfqpoint{1.783816in}{0.396138in}}{\pgfqpoint{1.778517in}{0.398333in}}{\pgfqpoint{1.772992in}{0.398333in}}%
\pgfpathcurveto{\pgfqpoint{1.767467in}{0.398333in}}{\pgfqpoint{1.762167in}{0.396138in}}{\pgfqpoint{1.758260in}{0.392231in}}%
\pgfpathcurveto{\pgfqpoint{1.754353in}{0.388325in}}{\pgfqpoint{1.752158in}{0.383025in}}{\pgfqpoint{1.752158in}{0.377500in}}%
\pgfpathcurveto{\pgfqpoint{1.752158in}{0.371975in}}{\pgfqpoint{1.754353in}{0.366675in}}{\pgfqpoint{1.758260in}{0.362769in}}%
\pgfpathcurveto{\pgfqpoint{1.762167in}{0.358862in}}{\pgfqpoint{1.767467in}{0.356667in}}{\pgfqpoint{1.772992in}{0.356667in}}%
\pgfpathclose%
\pgfusepath{stroke,fill}%
\end{pgfscope}%
\begin{pgfscope}%
\pgfpathrectangle{\pgfqpoint{0.562500in}{0.275000in}}{\pgfqpoint{3.487500in}{1.925000in}}%
\pgfusepath{clip}%
\pgfsetbuttcap%
\pgfsetroundjoin%
\definecolor{currentfill}{rgb}{0.000000,0.000000,0.000000}%
\pgfsetfillcolor{currentfill}%
\pgfsetlinewidth{1.003750pt}%
\definecolor{currentstroke}{rgb}{0.000000,0.000000,0.000000}%
\pgfsetstrokecolor{currentstroke}%
\pgfsetdash{}{0pt}%
\pgfpathmoveto{\pgfqpoint{1.772992in}{0.356667in}}%
\pgfpathcurveto{\pgfqpoint{1.778517in}{0.356667in}}{\pgfqpoint{1.783816in}{0.358862in}}{\pgfqpoint{1.787723in}{0.362769in}}%
\pgfpathcurveto{\pgfqpoint{1.791630in}{0.366675in}}{\pgfqpoint{1.793825in}{0.371975in}}{\pgfqpoint{1.793825in}{0.377500in}}%
\pgfpathcurveto{\pgfqpoint{1.793825in}{0.383025in}}{\pgfqpoint{1.791630in}{0.388325in}}{\pgfqpoint{1.787723in}{0.392231in}}%
\pgfpathcurveto{\pgfqpoint{1.783816in}{0.396138in}}{\pgfqpoint{1.778517in}{0.398333in}}{\pgfqpoint{1.772992in}{0.398333in}}%
\pgfpathcurveto{\pgfqpoint{1.767467in}{0.398333in}}{\pgfqpoint{1.762167in}{0.396138in}}{\pgfqpoint{1.758260in}{0.392231in}}%
\pgfpathcurveto{\pgfqpoint{1.754353in}{0.388325in}}{\pgfqpoint{1.752158in}{0.383025in}}{\pgfqpoint{1.752158in}{0.377500in}}%
\pgfpathcurveto{\pgfqpoint{1.752158in}{0.371975in}}{\pgfqpoint{1.754353in}{0.366675in}}{\pgfqpoint{1.758260in}{0.362769in}}%
\pgfpathcurveto{\pgfqpoint{1.762167in}{0.358862in}}{\pgfqpoint{1.767467in}{0.356667in}}{\pgfqpoint{1.772992in}{0.356667in}}%
\pgfpathclose%
\pgfusepath{stroke,fill}%
\end{pgfscope}%
\begin{pgfscope}%
\pgfpathrectangle{\pgfqpoint{0.562500in}{0.275000in}}{\pgfqpoint{3.487500in}{1.925000in}}%
\pgfusepath{clip}%
\pgfsetbuttcap%
\pgfsetroundjoin%
\definecolor{currentfill}{rgb}{0.000000,0.000000,0.000000}%
\pgfsetfillcolor{currentfill}%
\pgfsetlinewidth{1.003750pt}%
\definecolor{currentstroke}{rgb}{0.000000,0.000000,0.000000}%
\pgfsetstrokecolor{currentstroke}%
\pgfsetdash{}{0pt}%
\pgfpathmoveto{\pgfqpoint{1.772992in}{0.356667in}}%
\pgfpathcurveto{\pgfqpoint{1.778517in}{0.356667in}}{\pgfqpoint{1.783816in}{0.358862in}}{\pgfqpoint{1.787723in}{0.362769in}}%
\pgfpathcurveto{\pgfqpoint{1.791630in}{0.366675in}}{\pgfqpoint{1.793825in}{0.371975in}}{\pgfqpoint{1.793825in}{0.377500in}}%
\pgfpathcurveto{\pgfqpoint{1.793825in}{0.383025in}}{\pgfqpoint{1.791630in}{0.388325in}}{\pgfqpoint{1.787723in}{0.392231in}}%
\pgfpathcurveto{\pgfqpoint{1.783816in}{0.396138in}}{\pgfqpoint{1.778517in}{0.398333in}}{\pgfqpoint{1.772992in}{0.398333in}}%
\pgfpathcurveto{\pgfqpoint{1.767467in}{0.398333in}}{\pgfqpoint{1.762167in}{0.396138in}}{\pgfqpoint{1.758260in}{0.392231in}}%
\pgfpathcurveto{\pgfqpoint{1.754353in}{0.388325in}}{\pgfqpoint{1.752158in}{0.383025in}}{\pgfqpoint{1.752158in}{0.377500in}}%
\pgfpathcurveto{\pgfqpoint{1.752158in}{0.371975in}}{\pgfqpoint{1.754353in}{0.366675in}}{\pgfqpoint{1.758260in}{0.362769in}}%
\pgfpathcurveto{\pgfqpoint{1.762167in}{0.358862in}}{\pgfqpoint{1.767467in}{0.356667in}}{\pgfqpoint{1.772992in}{0.356667in}}%
\pgfpathclose%
\pgfusepath{stroke,fill}%
\end{pgfscope}%
\begin{pgfscope}%
\pgfpathrectangle{\pgfqpoint{0.562500in}{0.275000in}}{\pgfqpoint{3.487500in}{1.925000in}}%
\pgfusepath{clip}%
\pgfsetbuttcap%
\pgfsetroundjoin%
\definecolor{currentfill}{rgb}{0.000000,0.000000,0.000000}%
\pgfsetfillcolor{currentfill}%
\pgfsetlinewidth{1.003750pt}%
\definecolor{currentstroke}{rgb}{0.000000,0.000000,0.000000}%
\pgfsetstrokecolor{currentstroke}%
\pgfsetdash{}{0pt}%
\pgfpathmoveto{\pgfqpoint{1.772992in}{0.356667in}}%
\pgfpathcurveto{\pgfqpoint{1.778517in}{0.356667in}}{\pgfqpoint{1.783816in}{0.358862in}}{\pgfqpoint{1.787723in}{0.362769in}}%
\pgfpathcurveto{\pgfqpoint{1.791630in}{0.366675in}}{\pgfqpoint{1.793825in}{0.371975in}}{\pgfqpoint{1.793825in}{0.377500in}}%
\pgfpathcurveto{\pgfqpoint{1.793825in}{0.383025in}}{\pgfqpoint{1.791630in}{0.388325in}}{\pgfqpoint{1.787723in}{0.392231in}}%
\pgfpathcurveto{\pgfqpoint{1.783816in}{0.396138in}}{\pgfqpoint{1.778517in}{0.398333in}}{\pgfqpoint{1.772992in}{0.398333in}}%
\pgfpathcurveto{\pgfqpoint{1.767467in}{0.398333in}}{\pgfqpoint{1.762167in}{0.396138in}}{\pgfqpoint{1.758260in}{0.392231in}}%
\pgfpathcurveto{\pgfqpoint{1.754353in}{0.388325in}}{\pgfqpoint{1.752158in}{0.383025in}}{\pgfqpoint{1.752158in}{0.377500in}}%
\pgfpathcurveto{\pgfqpoint{1.752158in}{0.371975in}}{\pgfqpoint{1.754353in}{0.366675in}}{\pgfqpoint{1.758260in}{0.362769in}}%
\pgfpathcurveto{\pgfqpoint{1.762167in}{0.358862in}}{\pgfqpoint{1.767467in}{0.356667in}}{\pgfqpoint{1.772992in}{0.356667in}}%
\pgfpathclose%
\pgfusepath{stroke,fill}%
\end{pgfscope}%
\begin{pgfscope}%
\pgfpathrectangle{\pgfqpoint{0.562500in}{0.275000in}}{\pgfqpoint{3.487500in}{1.925000in}}%
\pgfusepath{clip}%
\pgfsetbuttcap%
\pgfsetroundjoin%
\definecolor{currentfill}{rgb}{0.000000,0.000000,0.000000}%
\pgfsetfillcolor{currentfill}%
\pgfsetlinewidth{1.003750pt}%
\definecolor{currentstroke}{rgb}{0.000000,0.000000,0.000000}%
\pgfsetstrokecolor{currentstroke}%
\pgfsetdash{}{0pt}%
\pgfpathmoveto{\pgfqpoint{1.772992in}{0.356667in}}%
\pgfpathcurveto{\pgfqpoint{1.778517in}{0.356667in}}{\pgfqpoint{1.783816in}{0.358862in}}{\pgfqpoint{1.787723in}{0.362769in}}%
\pgfpathcurveto{\pgfqpoint{1.791630in}{0.366675in}}{\pgfqpoint{1.793825in}{0.371975in}}{\pgfqpoint{1.793825in}{0.377500in}}%
\pgfpathcurveto{\pgfqpoint{1.793825in}{0.383025in}}{\pgfqpoint{1.791630in}{0.388325in}}{\pgfqpoint{1.787723in}{0.392231in}}%
\pgfpathcurveto{\pgfqpoint{1.783816in}{0.396138in}}{\pgfqpoint{1.778517in}{0.398333in}}{\pgfqpoint{1.772992in}{0.398333in}}%
\pgfpathcurveto{\pgfqpoint{1.767467in}{0.398333in}}{\pgfqpoint{1.762167in}{0.396138in}}{\pgfqpoint{1.758260in}{0.392231in}}%
\pgfpathcurveto{\pgfqpoint{1.754353in}{0.388325in}}{\pgfqpoint{1.752158in}{0.383025in}}{\pgfqpoint{1.752158in}{0.377500in}}%
\pgfpathcurveto{\pgfqpoint{1.752158in}{0.371975in}}{\pgfqpoint{1.754353in}{0.366675in}}{\pgfqpoint{1.758260in}{0.362769in}}%
\pgfpathcurveto{\pgfqpoint{1.762167in}{0.358862in}}{\pgfqpoint{1.767467in}{0.356667in}}{\pgfqpoint{1.772992in}{0.356667in}}%
\pgfpathclose%
\pgfusepath{stroke,fill}%
\end{pgfscope}%
\begin{pgfscope}%
\pgfpathrectangle{\pgfqpoint{0.562500in}{0.275000in}}{\pgfqpoint{3.487500in}{1.925000in}}%
\pgfusepath{clip}%
\pgfsetbuttcap%
\pgfsetroundjoin%
\definecolor{currentfill}{rgb}{0.000000,0.000000,0.000000}%
\pgfsetfillcolor{currentfill}%
\pgfsetlinewidth{1.003750pt}%
\definecolor{currentstroke}{rgb}{0.000000,0.000000,0.000000}%
\pgfsetstrokecolor{currentstroke}%
\pgfsetdash{}{0pt}%
\pgfpathmoveto{\pgfqpoint{1.772992in}{0.356667in}}%
\pgfpathcurveto{\pgfqpoint{1.778517in}{0.356667in}}{\pgfqpoint{1.783816in}{0.358862in}}{\pgfqpoint{1.787723in}{0.362769in}}%
\pgfpathcurveto{\pgfqpoint{1.791630in}{0.366675in}}{\pgfqpoint{1.793825in}{0.371975in}}{\pgfqpoint{1.793825in}{0.377500in}}%
\pgfpathcurveto{\pgfqpoint{1.793825in}{0.383025in}}{\pgfqpoint{1.791630in}{0.388325in}}{\pgfqpoint{1.787723in}{0.392231in}}%
\pgfpathcurveto{\pgfqpoint{1.783816in}{0.396138in}}{\pgfqpoint{1.778517in}{0.398333in}}{\pgfqpoint{1.772992in}{0.398333in}}%
\pgfpathcurveto{\pgfqpoint{1.767467in}{0.398333in}}{\pgfqpoint{1.762167in}{0.396138in}}{\pgfqpoint{1.758260in}{0.392231in}}%
\pgfpathcurveto{\pgfqpoint{1.754353in}{0.388325in}}{\pgfqpoint{1.752158in}{0.383025in}}{\pgfqpoint{1.752158in}{0.377500in}}%
\pgfpathcurveto{\pgfqpoint{1.752158in}{0.371975in}}{\pgfqpoint{1.754353in}{0.366675in}}{\pgfqpoint{1.758260in}{0.362769in}}%
\pgfpathcurveto{\pgfqpoint{1.762167in}{0.358862in}}{\pgfqpoint{1.767467in}{0.356667in}}{\pgfqpoint{1.772992in}{0.356667in}}%
\pgfpathclose%
\pgfusepath{stroke,fill}%
\end{pgfscope}%
\begin{pgfscope}%
\pgfpathrectangle{\pgfqpoint{0.562500in}{0.275000in}}{\pgfqpoint{3.487500in}{1.925000in}}%
\pgfusepath{clip}%
\pgfsetbuttcap%
\pgfsetroundjoin%
\definecolor{currentfill}{rgb}{0.000000,0.000000,0.000000}%
\pgfsetfillcolor{currentfill}%
\pgfsetlinewidth{1.003750pt}%
\definecolor{currentstroke}{rgb}{0.000000,0.000000,0.000000}%
\pgfsetstrokecolor{currentstroke}%
\pgfsetdash{}{0pt}%
\pgfpathmoveto{\pgfqpoint{1.772992in}{0.356667in}}%
\pgfpathcurveto{\pgfqpoint{1.778517in}{0.356667in}}{\pgfqpoint{1.783816in}{0.358862in}}{\pgfqpoint{1.787723in}{0.362769in}}%
\pgfpathcurveto{\pgfqpoint{1.791630in}{0.366675in}}{\pgfqpoint{1.793825in}{0.371975in}}{\pgfqpoint{1.793825in}{0.377500in}}%
\pgfpathcurveto{\pgfqpoint{1.793825in}{0.383025in}}{\pgfqpoint{1.791630in}{0.388325in}}{\pgfqpoint{1.787723in}{0.392231in}}%
\pgfpathcurveto{\pgfqpoint{1.783816in}{0.396138in}}{\pgfqpoint{1.778517in}{0.398333in}}{\pgfqpoint{1.772992in}{0.398333in}}%
\pgfpathcurveto{\pgfqpoint{1.767467in}{0.398333in}}{\pgfqpoint{1.762167in}{0.396138in}}{\pgfqpoint{1.758260in}{0.392231in}}%
\pgfpathcurveto{\pgfqpoint{1.754353in}{0.388325in}}{\pgfqpoint{1.752158in}{0.383025in}}{\pgfqpoint{1.752158in}{0.377500in}}%
\pgfpathcurveto{\pgfqpoint{1.752158in}{0.371975in}}{\pgfqpoint{1.754353in}{0.366675in}}{\pgfqpoint{1.758260in}{0.362769in}}%
\pgfpathcurveto{\pgfqpoint{1.762167in}{0.358862in}}{\pgfqpoint{1.767467in}{0.356667in}}{\pgfqpoint{1.772992in}{0.356667in}}%
\pgfpathclose%
\pgfusepath{stroke,fill}%
\end{pgfscope}%
\begin{pgfscope}%
\pgfpathrectangle{\pgfqpoint{0.562500in}{0.275000in}}{\pgfqpoint{3.487500in}{1.925000in}}%
\pgfusepath{clip}%
\pgfsetbuttcap%
\pgfsetroundjoin%
\definecolor{currentfill}{rgb}{0.000000,0.000000,0.000000}%
\pgfsetfillcolor{currentfill}%
\pgfsetlinewidth{1.003750pt}%
\definecolor{currentstroke}{rgb}{0.000000,0.000000,0.000000}%
\pgfsetstrokecolor{currentstroke}%
\pgfsetdash{}{0pt}%
\pgfpathmoveto{\pgfqpoint{1.772992in}{0.356667in}}%
\pgfpathcurveto{\pgfqpoint{1.778517in}{0.356667in}}{\pgfqpoint{1.783816in}{0.358862in}}{\pgfqpoint{1.787723in}{0.362769in}}%
\pgfpathcurveto{\pgfqpoint{1.791630in}{0.366675in}}{\pgfqpoint{1.793825in}{0.371975in}}{\pgfqpoint{1.793825in}{0.377500in}}%
\pgfpathcurveto{\pgfqpoint{1.793825in}{0.383025in}}{\pgfqpoint{1.791630in}{0.388325in}}{\pgfqpoint{1.787723in}{0.392231in}}%
\pgfpathcurveto{\pgfqpoint{1.783816in}{0.396138in}}{\pgfqpoint{1.778517in}{0.398333in}}{\pgfqpoint{1.772992in}{0.398333in}}%
\pgfpathcurveto{\pgfqpoint{1.767467in}{0.398333in}}{\pgfqpoint{1.762167in}{0.396138in}}{\pgfqpoint{1.758260in}{0.392231in}}%
\pgfpathcurveto{\pgfqpoint{1.754353in}{0.388325in}}{\pgfqpoint{1.752158in}{0.383025in}}{\pgfqpoint{1.752158in}{0.377500in}}%
\pgfpathcurveto{\pgfqpoint{1.752158in}{0.371975in}}{\pgfqpoint{1.754353in}{0.366675in}}{\pgfqpoint{1.758260in}{0.362769in}}%
\pgfpathcurveto{\pgfqpoint{1.762167in}{0.358862in}}{\pgfqpoint{1.767467in}{0.356667in}}{\pgfqpoint{1.772992in}{0.356667in}}%
\pgfpathclose%
\pgfusepath{stroke,fill}%
\end{pgfscope}%
\begin{pgfscope}%
\pgfpathrectangle{\pgfqpoint{0.562500in}{0.275000in}}{\pgfqpoint{3.487500in}{1.925000in}}%
\pgfusepath{clip}%
\pgfsetbuttcap%
\pgfsetroundjoin%
\definecolor{currentfill}{rgb}{0.000000,0.000000,0.000000}%
\pgfsetfillcolor{currentfill}%
\pgfsetlinewidth{1.003750pt}%
\definecolor{currentstroke}{rgb}{0.000000,0.000000,0.000000}%
\pgfsetstrokecolor{currentstroke}%
\pgfsetdash{}{0pt}%
\pgfpathmoveto{\pgfqpoint{1.772992in}{0.356667in}}%
\pgfpathcurveto{\pgfqpoint{1.778517in}{0.356667in}}{\pgfqpoint{1.783816in}{0.358862in}}{\pgfqpoint{1.787723in}{0.362769in}}%
\pgfpathcurveto{\pgfqpoint{1.791630in}{0.366675in}}{\pgfqpoint{1.793825in}{0.371975in}}{\pgfqpoint{1.793825in}{0.377500in}}%
\pgfpathcurveto{\pgfqpoint{1.793825in}{0.383025in}}{\pgfqpoint{1.791630in}{0.388325in}}{\pgfqpoint{1.787723in}{0.392231in}}%
\pgfpathcurveto{\pgfqpoint{1.783816in}{0.396138in}}{\pgfqpoint{1.778517in}{0.398333in}}{\pgfqpoint{1.772992in}{0.398333in}}%
\pgfpathcurveto{\pgfqpoint{1.767467in}{0.398333in}}{\pgfqpoint{1.762167in}{0.396138in}}{\pgfqpoint{1.758260in}{0.392231in}}%
\pgfpathcurveto{\pgfqpoint{1.754353in}{0.388325in}}{\pgfqpoint{1.752158in}{0.383025in}}{\pgfqpoint{1.752158in}{0.377500in}}%
\pgfpathcurveto{\pgfqpoint{1.752158in}{0.371975in}}{\pgfqpoint{1.754353in}{0.366675in}}{\pgfqpoint{1.758260in}{0.362769in}}%
\pgfpathcurveto{\pgfqpoint{1.762167in}{0.358862in}}{\pgfqpoint{1.767467in}{0.356667in}}{\pgfqpoint{1.772992in}{0.356667in}}%
\pgfpathclose%
\pgfusepath{stroke,fill}%
\end{pgfscope}%
\begin{pgfscope}%
\pgfpathrectangle{\pgfqpoint{0.562500in}{0.275000in}}{\pgfqpoint{3.487500in}{1.925000in}}%
\pgfusepath{clip}%
\pgfsetbuttcap%
\pgfsetroundjoin%
\definecolor{currentfill}{rgb}{0.000000,0.000000,0.000000}%
\pgfsetfillcolor{currentfill}%
\pgfsetlinewidth{1.003750pt}%
\definecolor{currentstroke}{rgb}{0.000000,0.000000,0.000000}%
\pgfsetstrokecolor{currentstroke}%
\pgfsetdash{}{0pt}%
\pgfpathmoveto{\pgfqpoint{1.772992in}{0.356667in}}%
\pgfpathcurveto{\pgfqpoint{1.778517in}{0.356667in}}{\pgfqpoint{1.783816in}{0.358862in}}{\pgfqpoint{1.787723in}{0.362769in}}%
\pgfpathcurveto{\pgfqpoint{1.791630in}{0.366675in}}{\pgfqpoint{1.793825in}{0.371975in}}{\pgfqpoint{1.793825in}{0.377500in}}%
\pgfpathcurveto{\pgfqpoint{1.793825in}{0.383025in}}{\pgfqpoint{1.791630in}{0.388325in}}{\pgfqpoint{1.787723in}{0.392231in}}%
\pgfpathcurveto{\pgfqpoint{1.783816in}{0.396138in}}{\pgfqpoint{1.778517in}{0.398333in}}{\pgfqpoint{1.772992in}{0.398333in}}%
\pgfpathcurveto{\pgfqpoint{1.767467in}{0.398333in}}{\pgfqpoint{1.762167in}{0.396138in}}{\pgfqpoint{1.758260in}{0.392231in}}%
\pgfpathcurveto{\pgfqpoint{1.754353in}{0.388325in}}{\pgfqpoint{1.752158in}{0.383025in}}{\pgfqpoint{1.752158in}{0.377500in}}%
\pgfpathcurveto{\pgfqpoint{1.752158in}{0.371975in}}{\pgfqpoint{1.754353in}{0.366675in}}{\pgfqpoint{1.758260in}{0.362769in}}%
\pgfpathcurveto{\pgfqpoint{1.762167in}{0.358862in}}{\pgfqpoint{1.767467in}{0.356667in}}{\pgfqpoint{1.772992in}{0.356667in}}%
\pgfpathclose%
\pgfusepath{stroke,fill}%
\end{pgfscope}%
\begin{pgfscope}%
\pgfpathrectangle{\pgfqpoint{0.562500in}{0.275000in}}{\pgfqpoint{3.487500in}{1.925000in}}%
\pgfusepath{clip}%
\pgfsetbuttcap%
\pgfsetroundjoin%
\definecolor{currentfill}{rgb}{0.000000,0.000000,0.000000}%
\pgfsetfillcolor{currentfill}%
\pgfsetlinewidth{1.003750pt}%
\definecolor{currentstroke}{rgb}{0.000000,0.000000,0.000000}%
\pgfsetstrokecolor{currentstroke}%
\pgfsetdash{}{0pt}%
\pgfpathmoveto{\pgfqpoint{1.772992in}{0.356667in}}%
\pgfpathcurveto{\pgfqpoint{1.778517in}{0.356667in}}{\pgfqpoint{1.783816in}{0.358862in}}{\pgfqpoint{1.787723in}{0.362769in}}%
\pgfpathcurveto{\pgfqpoint{1.791630in}{0.366675in}}{\pgfqpoint{1.793825in}{0.371975in}}{\pgfqpoint{1.793825in}{0.377500in}}%
\pgfpathcurveto{\pgfqpoint{1.793825in}{0.383025in}}{\pgfqpoint{1.791630in}{0.388325in}}{\pgfqpoint{1.787723in}{0.392231in}}%
\pgfpathcurveto{\pgfqpoint{1.783816in}{0.396138in}}{\pgfqpoint{1.778517in}{0.398333in}}{\pgfqpoint{1.772992in}{0.398333in}}%
\pgfpathcurveto{\pgfqpoint{1.767467in}{0.398333in}}{\pgfqpoint{1.762167in}{0.396138in}}{\pgfqpoint{1.758260in}{0.392231in}}%
\pgfpathcurveto{\pgfqpoint{1.754353in}{0.388325in}}{\pgfqpoint{1.752158in}{0.383025in}}{\pgfqpoint{1.752158in}{0.377500in}}%
\pgfpathcurveto{\pgfqpoint{1.752158in}{0.371975in}}{\pgfqpoint{1.754353in}{0.366675in}}{\pgfqpoint{1.758260in}{0.362769in}}%
\pgfpathcurveto{\pgfqpoint{1.762167in}{0.358862in}}{\pgfqpoint{1.767467in}{0.356667in}}{\pgfqpoint{1.772992in}{0.356667in}}%
\pgfpathclose%
\pgfusepath{stroke,fill}%
\end{pgfscope}%
\begin{pgfscope}%
\pgfpathrectangle{\pgfqpoint{0.562500in}{0.275000in}}{\pgfqpoint{3.487500in}{1.925000in}}%
\pgfusepath{clip}%
\pgfsetbuttcap%
\pgfsetroundjoin%
\definecolor{currentfill}{rgb}{0.000000,0.000000,0.000000}%
\pgfsetfillcolor{currentfill}%
\pgfsetlinewidth{1.003750pt}%
\definecolor{currentstroke}{rgb}{0.000000,0.000000,0.000000}%
\pgfsetstrokecolor{currentstroke}%
\pgfsetdash{}{0pt}%
\pgfpathmoveto{\pgfqpoint{1.772992in}{0.356667in}}%
\pgfpathcurveto{\pgfqpoint{1.778517in}{0.356667in}}{\pgfqpoint{1.783816in}{0.358862in}}{\pgfqpoint{1.787723in}{0.362769in}}%
\pgfpathcurveto{\pgfqpoint{1.791630in}{0.366675in}}{\pgfqpoint{1.793825in}{0.371975in}}{\pgfqpoint{1.793825in}{0.377500in}}%
\pgfpathcurveto{\pgfqpoint{1.793825in}{0.383025in}}{\pgfqpoint{1.791630in}{0.388325in}}{\pgfqpoint{1.787723in}{0.392231in}}%
\pgfpathcurveto{\pgfqpoint{1.783816in}{0.396138in}}{\pgfqpoint{1.778517in}{0.398333in}}{\pgfqpoint{1.772992in}{0.398333in}}%
\pgfpathcurveto{\pgfqpoint{1.767467in}{0.398333in}}{\pgfqpoint{1.762167in}{0.396138in}}{\pgfqpoint{1.758260in}{0.392231in}}%
\pgfpathcurveto{\pgfqpoint{1.754353in}{0.388325in}}{\pgfqpoint{1.752158in}{0.383025in}}{\pgfqpoint{1.752158in}{0.377500in}}%
\pgfpathcurveto{\pgfqpoint{1.752158in}{0.371975in}}{\pgfqpoint{1.754353in}{0.366675in}}{\pgfqpoint{1.758260in}{0.362769in}}%
\pgfpathcurveto{\pgfqpoint{1.762167in}{0.358862in}}{\pgfqpoint{1.767467in}{0.356667in}}{\pgfqpoint{1.772992in}{0.356667in}}%
\pgfpathclose%
\pgfusepath{stroke,fill}%
\end{pgfscope}%
\begin{pgfscope}%
\pgfpathrectangle{\pgfqpoint{0.562500in}{0.275000in}}{\pgfqpoint{3.487500in}{1.925000in}}%
\pgfusepath{clip}%
\pgfsetbuttcap%
\pgfsetroundjoin%
\definecolor{currentfill}{rgb}{0.000000,0.000000,0.000000}%
\pgfsetfillcolor{currentfill}%
\pgfsetlinewidth{1.003750pt}%
\definecolor{currentstroke}{rgb}{0.000000,0.000000,0.000000}%
\pgfsetstrokecolor{currentstroke}%
\pgfsetdash{}{0pt}%
\pgfpathmoveto{\pgfqpoint{1.772992in}{0.356667in}}%
\pgfpathcurveto{\pgfqpoint{1.778517in}{0.356667in}}{\pgfqpoint{1.783816in}{0.358862in}}{\pgfqpoint{1.787723in}{0.362769in}}%
\pgfpathcurveto{\pgfqpoint{1.791630in}{0.366675in}}{\pgfqpoint{1.793825in}{0.371975in}}{\pgfqpoint{1.793825in}{0.377500in}}%
\pgfpathcurveto{\pgfqpoint{1.793825in}{0.383025in}}{\pgfqpoint{1.791630in}{0.388325in}}{\pgfqpoint{1.787723in}{0.392231in}}%
\pgfpathcurveto{\pgfqpoint{1.783816in}{0.396138in}}{\pgfqpoint{1.778517in}{0.398333in}}{\pgfqpoint{1.772992in}{0.398333in}}%
\pgfpathcurveto{\pgfqpoint{1.767467in}{0.398333in}}{\pgfqpoint{1.762167in}{0.396138in}}{\pgfqpoint{1.758260in}{0.392231in}}%
\pgfpathcurveto{\pgfqpoint{1.754353in}{0.388325in}}{\pgfqpoint{1.752158in}{0.383025in}}{\pgfqpoint{1.752158in}{0.377500in}}%
\pgfpathcurveto{\pgfqpoint{1.752158in}{0.371975in}}{\pgfqpoint{1.754353in}{0.366675in}}{\pgfqpoint{1.758260in}{0.362769in}}%
\pgfpathcurveto{\pgfqpoint{1.762167in}{0.358862in}}{\pgfqpoint{1.767467in}{0.356667in}}{\pgfqpoint{1.772992in}{0.356667in}}%
\pgfpathclose%
\pgfusepath{stroke,fill}%
\end{pgfscope}%
\begin{pgfscope}%
\pgfpathrectangle{\pgfqpoint{0.562500in}{0.275000in}}{\pgfqpoint{3.487500in}{1.925000in}}%
\pgfusepath{clip}%
\pgfsetbuttcap%
\pgfsetroundjoin%
\definecolor{currentfill}{rgb}{0.000000,0.000000,0.000000}%
\pgfsetfillcolor{currentfill}%
\pgfsetlinewidth{1.003750pt}%
\definecolor{currentstroke}{rgb}{0.000000,0.000000,0.000000}%
\pgfsetstrokecolor{currentstroke}%
\pgfsetdash{}{0pt}%
\pgfpathmoveto{\pgfqpoint{1.772992in}{0.356667in}}%
\pgfpathcurveto{\pgfqpoint{1.778517in}{0.356667in}}{\pgfqpoint{1.783816in}{0.358862in}}{\pgfqpoint{1.787723in}{0.362769in}}%
\pgfpathcurveto{\pgfqpoint{1.791630in}{0.366675in}}{\pgfqpoint{1.793825in}{0.371975in}}{\pgfqpoint{1.793825in}{0.377500in}}%
\pgfpathcurveto{\pgfqpoint{1.793825in}{0.383025in}}{\pgfqpoint{1.791630in}{0.388325in}}{\pgfqpoint{1.787723in}{0.392231in}}%
\pgfpathcurveto{\pgfqpoint{1.783816in}{0.396138in}}{\pgfqpoint{1.778517in}{0.398333in}}{\pgfqpoint{1.772992in}{0.398333in}}%
\pgfpathcurveto{\pgfqpoint{1.767467in}{0.398333in}}{\pgfqpoint{1.762167in}{0.396138in}}{\pgfqpoint{1.758260in}{0.392231in}}%
\pgfpathcurveto{\pgfqpoint{1.754353in}{0.388325in}}{\pgfqpoint{1.752158in}{0.383025in}}{\pgfqpoint{1.752158in}{0.377500in}}%
\pgfpathcurveto{\pgfqpoint{1.752158in}{0.371975in}}{\pgfqpoint{1.754353in}{0.366675in}}{\pgfqpoint{1.758260in}{0.362769in}}%
\pgfpathcurveto{\pgfqpoint{1.762167in}{0.358862in}}{\pgfqpoint{1.767467in}{0.356667in}}{\pgfqpoint{1.772992in}{0.356667in}}%
\pgfpathclose%
\pgfusepath{stroke,fill}%
\end{pgfscope}%
\begin{pgfscope}%
\pgfpathrectangle{\pgfqpoint{0.562500in}{0.275000in}}{\pgfqpoint{3.487500in}{1.925000in}}%
\pgfusepath{clip}%
\pgfsetbuttcap%
\pgfsetroundjoin%
\definecolor{currentfill}{rgb}{0.000000,0.000000,0.000000}%
\pgfsetfillcolor{currentfill}%
\pgfsetlinewidth{1.003750pt}%
\definecolor{currentstroke}{rgb}{0.000000,0.000000,0.000000}%
\pgfsetstrokecolor{currentstroke}%
\pgfsetdash{}{0pt}%
\pgfpathmoveto{\pgfqpoint{1.772992in}{0.356667in}}%
\pgfpathcurveto{\pgfqpoint{1.778517in}{0.356667in}}{\pgfqpoint{1.783816in}{0.358862in}}{\pgfqpoint{1.787723in}{0.362769in}}%
\pgfpathcurveto{\pgfqpoint{1.791630in}{0.366675in}}{\pgfqpoint{1.793825in}{0.371975in}}{\pgfqpoint{1.793825in}{0.377500in}}%
\pgfpathcurveto{\pgfqpoint{1.793825in}{0.383025in}}{\pgfqpoint{1.791630in}{0.388325in}}{\pgfqpoint{1.787723in}{0.392231in}}%
\pgfpathcurveto{\pgfqpoint{1.783816in}{0.396138in}}{\pgfqpoint{1.778517in}{0.398333in}}{\pgfqpoint{1.772992in}{0.398333in}}%
\pgfpathcurveto{\pgfqpoint{1.767467in}{0.398333in}}{\pgfqpoint{1.762167in}{0.396138in}}{\pgfqpoint{1.758260in}{0.392231in}}%
\pgfpathcurveto{\pgfqpoint{1.754353in}{0.388325in}}{\pgfqpoint{1.752158in}{0.383025in}}{\pgfqpoint{1.752158in}{0.377500in}}%
\pgfpathcurveto{\pgfqpoint{1.752158in}{0.371975in}}{\pgfqpoint{1.754353in}{0.366675in}}{\pgfqpoint{1.758260in}{0.362769in}}%
\pgfpathcurveto{\pgfqpoint{1.762167in}{0.358862in}}{\pgfqpoint{1.767467in}{0.356667in}}{\pgfqpoint{1.772992in}{0.356667in}}%
\pgfpathclose%
\pgfusepath{stroke,fill}%
\end{pgfscope}%
\begin{pgfscope}%
\pgfpathrectangle{\pgfqpoint{0.562500in}{0.275000in}}{\pgfqpoint{3.487500in}{1.925000in}}%
\pgfusepath{clip}%
\pgfsetbuttcap%
\pgfsetroundjoin%
\definecolor{currentfill}{rgb}{0.000000,0.000000,0.000000}%
\pgfsetfillcolor{currentfill}%
\pgfsetlinewidth{1.003750pt}%
\definecolor{currentstroke}{rgb}{0.000000,0.000000,0.000000}%
\pgfsetstrokecolor{currentstroke}%
\pgfsetdash{}{0pt}%
\pgfpathmoveto{\pgfqpoint{1.772992in}{0.356667in}}%
\pgfpathcurveto{\pgfqpoint{1.778517in}{0.356667in}}{\pgfqpoint{1.783816in}{0.358862in}}{\pgfqpoint{1.787723in}{0.362769in}}%
\pgfpathcurveto{\pgfqpoint{1.791630in}{0.366675in}}{\pgfqpoint{1.793825in}{0.371975in}}{\pgfqpoint{1.793825in}{0.377500in}}%
\pgfpathcurveto{\pgfqpoint{1.793825in}{0.383025in}}{\pgfqpoint{1.791630in}{0.388325in}}{\pgfqpoint{1.787723in}{0.392231in}}%
\pgfpathcurveto{\pgfqpoint{1.783816in}{0.396138in}}{\pgfqpoint{1.778517in}{0.398333in}}{\pgfqpoint{1.772992in}{0.398333in}}%
\pgfpathcurveto{\pgfqpoint{1.767467in}{0.398333in}}{\pgfqpoint{1.762167in}{0.396138in}}{\pgfqpoint{1.758260in}{0.392231in}}%
\pgfpathcurveto{\pgfqpoint{1.754353in}{0.388325in}}{\pgfqpoint{1.752158in}{0.383025in}}{\pgfqpoint{1.752158in}{0.377500in}}%
\pgfpathcurveto{\pgfqpoint{1.752158in}{0.371975in}}{\pgfqpoint{1.754353in}{0.366675in}}{\pgfqpoint{1.758260in}{0.362769in}}%
\pgfpathcurveto{\pgfqpoint{1.762167in}{0.358862in}}{\pgfqpoint{1.767467in}{0.356667in}}{\pgfqpoint{1.772992in}{0.356667in}}%
\pgfpathclose%
\pgfusepath{stroke,fill}%
\end{pgfscope}%
\begin{pgfscope}%
\pgfpathrectangle{\pgfqpoint{0.562500in}{0.275000in}}{\pgfqpoint{3.487500in}{1.925000in}}%
\pgfusepath{clip}%
\pgfsetbuttcap%
\pgfsetroundjoin%
\definecolor{currentfill}{rgb}{0.000000,0.000000,0.000000}%
\pgfsetfillcolor{currentfill}%
\pgfsetlinewidth{1.003750pt}%
\definecolor{currentstroke}{rgb}{0.000000,0.000000,0.000000}%
\pgfsetstrokecolor{currentstroke}%
\pgfsetdash{}{0pt}%
\pgfpathmoveto{\pgfqpoint{1.772992in}{0.356667in}}%
\pgfpathcurveto{\pgfqpoint{1.778517in}{0.356667in}}{\pgfqpoint{1.783816in}{0.358862in}}{\pgfqpoint{1.787723in}{0.362769in}}%
\pgfpathcurveto{\pgfqpoint{1.791630in}{0.366675in}}{\pgfqpoint{1.793825in}{0.371975in}}{\pgfqpoint{1.793825in}{0.377500in}}%
\pgfpathcurveto{\pgfqpoint{1.793825in}{0.383025in}}{\pgfqpoint{1.791630in}{0.388325in}}{\pgfqpoint{1.787723in}{0.392231in}}%
\pgfpathcurveto{\pgfqpoint{1.783816in}{0.396138in}}{\pgfqpoint{1.778517in}{0.398333in}}{\pgfqpoint{1.772992in}{0.398333in}}%
\pgfpathcurveto{\pgfqpoint{1.767467in}{0.398333in}}{\pgfqpoint{1.762167in}{0.396138in}}{\pgfqpoint{1.758260in}{0.392231in}}%
\pgfpathcurveto{\pgfqpoint{1.754353in}{0.388325in}}{\pgfqpoint{1.752158in}{0.383025in}}{\pgfqpoint{1.752158in}{0.377500in}}%
\pgfpathcurveto{\pgfqpoint{1.752158in}{0.371975in}}{\pgfqpoint{1.754353in}{0.366675in}}{\pgfqpoint{1.758260in}{0.362769in}}%
\pgfpathcurveto{\pgfqpoint{1.762167in}{0.358862in}}{\pgfqpoint{1.767467in}{0.356667in}}{\pgfqpoint{1.772992in}{0.356667in}}%
\pgfpathclose%
\pgfusepath{stroke,fill}%
\end{pgfscope}%
\begin{pgfscope}%
\pgfpathrectangle{\pgfqpoint{0.562500in}{0.275000in}}{\pgfqpoint{3.487500in}{1.925000in}}%
\pgfusepath{clip}%
\pgfsetbuttcap%
\pgfsetroundjoin%
\definecolor{currentfill}{rgb}{0.000000,0.000000,0.000000}%
\pgfsetfillcolor{currentfill}%
\pgfsetlinewidth{1.003750pt}%
\definecolor{currentstroke}{rgb}{0.000000,0.000000,0.000000}%
\pgfsetstrokecolor{currentstroke}%
\pgfsetdash{}{0pt}%
\pgfpathmoveto{\pgfqpoint{1.772992in}{0.356667in}}%
\pgfpathcurveto{\pgfqpoint{1.778517in}{0.356667in}}{\pgfqpoint{1.783816in}{0.358862in}}{\pgfqpoint{1.787723in}{0.362769in}}%
\pgfpathcurveto{\pgfqpoint{1.791630in}{0.366675in}}{\pgfqpoint{1.793825in}{0.371975in}}{\pgfqpoint{1.793825in}{0.377500in}}%
\pgfpathcurveto{\pgfqpoint{1.793825in}{0.383025in}}{\pgfqpoint{1.791630in}{0.388325in}}{\pgfqpoint{1.787723in}{0.392231in}}%
\pgfpathcurveto{\pgfqpoint{1.783816in}{0.396138in}}{\pgfqpoint{1.778517in}{0.398333in}}{\pgfqpoint{1.772992in}{0.398333in}}%
\pgfpathcurveto{\pgfqpoint{1.767467in}{0.398333in}}{\pgfqpoint{1.762167in}{0.396138in}}{\pgfqpoint{1.758260in}{0.392231in}}%
\pgfpathcurveto{\pgfqpoint{1.754353in}{0.388325in}}{\pgfqpoint{1.752158in}{0.383025in}}{\pgfqpoint{1.752158in}{0.377500in}}%
\pgfpathcurveto{\pgfqpoint{1.752158in}{0.371975in}}{\pgfqpoint{1.754353in}{0.366675in}}{\pgfqpoint{1.758260in}{0.362769in}}%
\pgfpathcurveto{\pgfqpoint{1.762167in}{0.358862in}}{\pgfqpoint{1.767467in}{0.356667in}}{\pgfqpoint{1.772992in}{0.356667in}}%
\pgfpathclose%
\pgfusepath{stroke,fill}%
\end{pgfscope}%
\begin{pgfscope}%
\pgfpathrectangle{\pgfqpoint{0.562500in}{0.275000in}}{\pgfqpoint{3.487500in}{1.925000in}}%
\pgfusepath{clip}%
\pgfsetbuttcap%
\pgfsetroundjoin%
\definecolor{currentfill}{rgb}{0.000000,0.000000,0.000000}%
\pgfsetfillcolor{currentfill}%
\pgfsetlinewidth{1.003750pt}%
\definecolor{currentstroke}{rgb}{0.000000,0.000000,0.000000}%
\pgfsetstrokecolor{currentstroke}%
\pgfsetdash{}{0pt}%
\pgfpathmoveto{\pgfqpoint{1.772992in}{0.356667in}}%
\pgfpathcurveto{\pgfqpoint{1.778517in}{0.356667in}}{\pgfqpoint{1.783816in}{0.358862in}}{\pgfqpoint{1.787723in}{0.362769in}}%
\pgfpathcurveto{\pgfqpoint{1.791630in}{0.366675in}}{\pgfqpoint{1.793825in}{0.371975in}}{\pgfqpoint{1.793825in}{0.377500in}}%
\pgfpathcurveto{\pgfqpoint{1.793825in}{0.383025in}}{\pgfqpoint{1.791630in}{0.388325in}}{\pgfqpoint{1.787723in}{0.392231in}}%
\pgfpathcurveto{\pgfqpoint{1.783816in}{0.396138in}}{\pgfqpoint{1.778517in}{0.398333in}}{\pgfqpoint{1.772992in}{0.398333in}}%
\pgfpathcurveto{\pgfqpoint{1.767467in}{0.398333in}}{\pgfqpoint{1.762167in}{0.396138in}}{\pgfqpoint{1.758260in}{0.392231in}}%
\pgfpathcurveto{\pgfqpoint{1.754353in}{0.388325in}}{\pgfqpoint{1.752158in}{0.383025in}}{\pgfqpoint{1.752158in}{0.377500in}}%
\pgfpathcurveto{\pgfqpoint{1.752158in}{0.371975in}}{\pgfqpoint{1.754353in}{0.366675in}}{\pgfqpoint{1.758260in}{0.362769in}}%
\pgfpathcurveto{\pgfqpoint{1.762167in}{0.358862in}}{\pgfqpoint{1.767467in}{0.356667in}}{\pgfqpoint{1.772992in}{0.356667in}}%
\pgfpathclose%
\pgfusepath{stroke,fill}%
\end{pgfscope}%
\begin{pgfscope}%
\pgfpathrectangle{\pgfqpoint{0.562500in}{0.275000in}}{\pgfqpoint{3.487500in}{1.925000in}}%
\pgfusepath{clip}%
\pgfsetbuttcap%
\pgfsetroundjoin%
\definecolor{currentfill}{rgb}{0.000000,0.000000,0.000000}%
\pgfsetfillcolor{currentfill}%
\pgfsetlinewidth{1.003750pt}%
\definecolor{currentstroke}{rgb}{0.000000,0.000000,0.000000}%
\pgfsetstrokecolor{currentstroke}%
\pgfsetdash{}{0pt}%
\pgfpathmoveto{\pgfqpoint{1.772992in}{0.356667in}}%
\pgfpathcurveto{\pgfqpoint{1.778517in}{0.356667in}}{\pgfqpoint{1.783816in}{0.358862in}}{\pgfqpoint{1.787723in}{0.362769in}}%
\pgfpathcurveto{\pgfqpoint{1.791630in}{0.366675in}}{\pgfqpoint{1.793825in}{0.371975in}}{\pgfqpoint{1.793825in}{0.377500in}}%
\pgfpathcurveto{\pgfqpoint{1.793825in}{0.383025in}}{\pgfqpoint{1.791630in}{0.388325in}}{\pgfqpoint{1.787723in}{0.392231in}}%
\pgfpathcurveto{\pgfqpoint{1.783816in}{0.396138in}}{\pgfqpoint{1.778517in}{0.398333in}}{\pgfqpoint{1.772992in}{0.398333in}}%
\pgfpathcurveto{\pgfqpoint{1.767467in}{0.398333in}}{\pgfqpoint{1.762167in}{0.396138in}}{\pgfqpoint{1.758260in}{0.392231in}}%
\pgfpathcurveto{\pgfqpoint{1.754353in}{0.388325in}}{\pgfqpoint{1.752158in}{0.383025in}}{\pgfqpoint{1.752158in}{0.377500in}}%
\pgfpathcurveto{\pgfqpoint{1.752158in}{0.371975in}}{\pgfqpoint{1.754353in}{0.366675in}}{\pgfqpoint{1.758260in}{0.362769in}}%
\pgfpathcurveto{\pgfqpoint{1.762167in}{0.358862in}}{\pgfqpoint{1.767467in}{0.356667in}}{\pgfqpoint{1.772992in}{0.356667in}}%
\pgfpathclose%
\pgfusepath{stroke,fill}%
\end{pgfscope}%
\begin{pgfscope}%
\pgfpathrectangle{\pgfqpoint{0.562500in}{0.275000in}}{\pgfqpoint{3.487500in}{1.925000in}}%
\pgfusepath{clip}%
\pgfsetbuttcap%
\pgfsetroundjoin%
\definecolor{currentfill}{rgb}{0.000000,0.000000,0.000000}%
\pgfsetfillcolor{currentfill}%
\pgfsetlinewidth{1.003750pt}%
\definecolor{currentstroke}{rgb}{0.000000,0.000000,0.000000}%
\pgfsetstrokecolor{currentstroke}%
\pgfsetdash{}{0pt}%
\pgfpathmoveto{\pgfqpoint{1.772992in}{0.356667in}}%
\pgfpathcurveto{\pgfqpoint{1.778517in}{0.356667in}}{\pgfqpoint{1.783816in}{0.358862in}}{\pgfqpoint{1.787723in}{0.362769in}}%
\pgfpathcurveto{\pgfqpoint{1.791630in}{0.366675in}}{\pgfqpoint{1.793825in}{0.371975in}}{\pgfqpoint{1.793825in}{0.377500in}}%
\pgfpathcurveto{\pgfqpoint{1.793825in}{0.383025in}}{\pgfqpoint{1.791630in}{0.388325in}}{\pgfqpoint{1.787723in}{0.392231in}}%
\pgfpathcurveto{\pgfqpoint{1.783816in}{0.396138in}}{\pgfqpoint{1.778517in}{0.398333in}}{\pgfqpoint{1.772992in}{0.398333in}}%
\pgfpathcurveto{\pgfqpoint{1.767467in}{0.398333in}}{\pgfqpoint{1.762167in}{0.396138in}}{\pgfqpoint{1.758260in}{0.392231in}}%
\pgfpathcurveto{\pgfqpoint{1.754353in}{0.388325in}}{\pgfqpoint{1.752158in}{0.383025in}}{\pgfqpoint{1.752158in}{0.377500in}}%
\pgfpathcurveto{\pgfqpoint{1.752158in}{0.371975in}}{\pgfqpoint{1.754353in}{0.366675in}}{\pgfqpoint{1.758260in}{0.362769in}}%
\pgfpathcurveto{\pgfqpoint{1.762167in}{0.358862in}}{\pgfqpoint{1.767467in}{0.356667in}}{\pgfqpoint{1.772992in}{0.356667in}}%
\pgfpathclose%
\pgfusepath{stroke,fill}%
\end{pgfscope}%
\begin{pgfscope}%
\pgfpathrectangle{\pgfqpoint{0.562500in}{0.275000in}}{\pgfqpoint{3.487500in}{1.925000in}}%
\pgfusepath{clip}%
\pgfsetbuttcap%
\pgfsetroundjoin%
\definecolor{currentfill}{rgb}{0.000000,0.000000,0.000000}%
\pgfsetfillcolor{currentfill}%
\pgfsetlinewidth{1.003750pt}%
\definecolor{currentstroke}{rgb}{0.000000,0.000000,0.000000}%
\pgfsetstrokecolor{currentstroke}%
\pgfsetdash{}{0pt}%
\pgfpathmoveto{\pgfqpoint{1.772992in}{0.356667in}}%
\pgfpathcurveto{\pgfqpoint{1.778517in}{0.356667in}}{\pgfqpoint{1.783816in}{0.358862in}}{\pgfqpoint{1.787723in}{0.362769in}}%
\pgfpathcurveto{\pgfqpoint{1.791630in}{0.366675in}}{\pgfqpoint{1.793825in}{0.371975in}}{\pgfqpoint{1.793825in}{0.377500in}}%
\pgfpathcurveto{\pgfqpoint{1.793825in}{0.383025in}}{\pgfqpoint{1.791630in}{0.388325in}}{\pgfqpoint{1.787723in}{0.392231in}}%
\pgfpathcurveto{\pgfqpoint{1.783816in}{0.396138in}}{\pgfqpoint{1.778517in}{0.398333in}}{\pgfqpoint{1.772992in}{0.398333in}}%
\pgfpathcurveto{\pgfqpoint{1.767467in}{0.398333in}}{\pgfqpoint{1.762167in}{0.396138in}}{\pgfqpoint{1.758260in}{0.392231in}}%
\pgfpathcurveto{\pgfqpoint{1.754353in}{0.388325in}}{\pgfqpoint{1.752158in}{0.383025in}}{\pgfqpoint{1.752158in}{0.377500in}}%
\pgfpathcurveto{\pgfqpoint{1.752158in}{0.371975in}}{\pgfqpoint{1.754353in}{0.366675in}}{\pgfqpoint{1.758260in}{0.362769in}}%
\pgfpathcurveto{\pgfqpoint{1.762167in}{0.358862in}}{\pgfqpoint{1.767467in}{0.356667in}}{\pgfqpoint{1.772992in}{0.356667in}}%
\pgfpathclose%
\pgfusepath{stroke,fill}%
\end{pgfscope}%
\begin{pgfscope}%
\pgfpathrectangle{\pgfqpoint{0.562500in}{0.275000in}}{\pgfqpoint{3.487500in}{1.925000in}}%
\pgfusepath{clip}%
\pgfsetbuttcap%
\pgfsetroundjoin%
\definecolor{currentfill}{rgb}{0.000000,0.000000,0.000000}%
\pgfsetfillcolor{currentfill}%
\pgfsetlinewidth{1.003750pt}%
\definecolor{currentstroke}{rgb}{0.000000,0.000000,0.000000}%
\pgfsetstrokecolor{currentstroke}%
\pgfsetdash{}{0pt}%
\pgfpathmoveto{\pgfqpoint{1.772992in}{0.356667in}}%
\pgfpathcurveto{\pgfqpoint{1.778517in}{0.356667in}}{\pgfqpoint{1.783816in}{0.358862in}}{\pgfqpoint{1.787723in}{0.362769in}}%
\pgfpathcurveto{\pgfqpoint{1.791630in}{0.366675in}}{\pgfqpoint{1.793825in}{0.371975in}}{\pgfqpoint{1.793825in}{0.377500in}}%
\pgfpathcurveto{\pgfqpoint{1.793825in}{0.383025in}}{\pgfqpoint{1.791630in}{0.388325in}}{\pgfqpoint{1.787723in}{0.392231in}}%
\pgfpathcurveto{\pgfqpoint{1.783816in}{0.396138in}}{\pgfqpoint{1.778517in}{0.398333in}}{\pgfqpoint{1.772992in}{0.398333in}}%
\pgfpathcurveto{\pgfqpoint{1.767467in}{0.398333in}}{\pgfqpoint{1.762167in}{0.396138in}}{\pgfqpoint{1.758260in}{0.392231in}}%
\pgfpathcurveto{\pgfqpoint{1.754353in}{0.388325in}}{\pgfqpoint{1.752158in}{0.383025in}}{\pgfqpoint{1.752158in}{0.377500in}}%
\pgfpathcurveto{\pgfqpoint{1.752158in}{0.371975in}}{\pgfqpoint{1.754353in}{0.366675in}}{\pgfqpoint{1.758260in}{0.362769in}}%
\pgfpathcurveto{\pgfqpoint{1.762167in}{0.358862in}}{\pgfqpoint{1.767467in}{0.356667in}}{\pgfqpoint{1.772992in}{0.356667in}}%
\pgfpathclose%
\pgfusepath{stroke,fill}%
\end{pgfscope}%
\begin{pgfscope}%
\pgfpathrectangle{\pgfqpoint{0.562500in}{0.275000in}}{\pgfqpoint{3.487500in}{1.925000in}}%
\pgfusepath{clip}%
\pgfsetbuttcap%
\pgfsetroundjoin%
\definecolor{currentfill}{rgb}{0.000000,0.000000,0.000000}%
\pgfsetfillcolor{currentfill}%
\pgfsetlinewidth{1.003750pt}%
\definecolor{currentstroke}{rgb}{0.000000,0.000000,0.000000}%
\pgfsetstrokecolor{currentstroke}%
\pgfsetdash{}{0pt}%
\pgfpathmoveto{\pgfqpoint{1.772992in}{0.356667in}}%
\pgfpathcurveto{\pgfqpoint{1.778517in}{0.356667in}}{\pgfqpoint{1.783816in}{0.358862in}}{\pgfqpoint{1.787723in}{0.362769in}}%
\pgfpathcurveto{\pgfqpoint{1.791630in}{0.366675in}}{\pgfqpoint{1.793825in}{0.371975in}}{\pgfqpoint{1.793825in}{0.377500in}}%
\pgfpathcurveto{\pgfqpoint{1.793825in}{0.383025in}}{\pgfqpoint{1.791630in}{0.388325in}}{\pgfqpoint{1.787723in}{0.392231in}}%
\pgfpathcurveto{\pgfqpoint{1.783816in}{0.396138in}}{\pgfqpoint{1.778517in}{0.398333in}}{\pgfqpoint{1.772992in}{0.398333in}}%
\pgfpathcurveto{\pgfqpoint{1.767467in}{0.398333in}}{\pgfqpoint{1.762167in}{0.396138in}}{\pgfqpoint{1.758260in}{0.392231in}}%
\pgfpathcurveto{\pgfqpoint{1.754353in}{0.388325in}}{\pgfqpoint{1.752158in}{0.383025in}}{\pgfqpoint{1.752158in}{0.377500in}}%
\pgfpathcurveto{\pgfqpoint{1.752158in}{0.371975in}}{\pgfqpoint{1.754353in}{0.366675in}}{\pgfqpoint{1.758260in}{0.362769in}}%
\pgfpathcurveto{\pgfqpoint{1.762167in}{0.358862in}}{\pgfqpoint{1.767467in}{0.356667in}}{\pgfqpoint{1.772992in}{0.356667in}}%
\pgfpathclose%
\pgfusepath{stroke,fill}%
\end{pgfscope}%
\begin{pgfscope}%
\pgfpathrectangle{\pgfqpoint{0.562500in}{0.275000in}}{\pgfqpoint{3.487500in}{1.925000in}}%
\pgfusepath{clip}%
\pgfsetbuttcap%
\pgfsetroundjoin%
\definecolor{currentfill}{rgb}{0.000000,0.000000,0.000000}%
\pgfsetfillcolor{currentfill}%
\pgfsetlinewidth{1.003750pt}%
\definecolor{currentstroke}{rgb}{0.000000,0.000000,0.000000}%
\pgfsetstrokecolor{currentstroke}%
\pgfsetdash{}{0pt}%
\pgfpathmoveto{\pgfqpoint{1.772992in}{0.356667in}}%
\pgfpathcurveto{\pgfqpoint{1.778517in}{0.356667in}}{\pgfqpoint{1.783816in}{0.358862in}}{\pgfqpoint{1.787723in}{0.362769in}}%
\pgfpathcurveto{\pgfqpoint{1.791630in}{0.366675in}}{\pgfqpoint{1.793825in}{0.371975in}}{\pgfqpoint{1.793825in}{0.377500in}}%
\pgfpathcurveto{\pgfqpoint{1.793825in}{0.383025in}}{\pgfqpoint{1.791630in}{0.388325in}}{\pgfqpoint{1.787723in}{0.392231in}}%
\pgfpathcurveto{\pgfqpoint{1.783816in}{0.396138in}}{\pgfqpoint{1.778517in}{0.398333in}}{\pgfqpoint{1.772992in}{0.398333in}}%
\pgfpathcurveto{\pgfqpoint{1.767467in}{0.398333in}}{\pgfqpoint{1.762167in}{0.396138in}}{\pgfqpoint{1.758260in}{0.392231in}}%
\pgfpathcurveto{\pgfqpoint{1.754353in}{0.388325in}}{\pgfqpoint{1.752158in}{0.383025in}}{\pgfqpoint{1.752158in}{0.377500in}}%
\pgfpathcurveto{\pgfqpoint{1.752158in}{0.371975in}}{\pgfqpoint{1.754353in}{0.366675in}}{\pgfqpoint{1.758260in}{0.362769in}}%
\pgfpathcurveto{\pgfqpoint{1.762167in}{0.358862in}}{\pgfqpoint{1.767467in}{0.356667in}}{\pgfqpoint{1.772992in}{0.356667in}}%
\pgfpathclose%
\pgfusepath{stroke,fill}%
\end{pgfscope}%
\begin{pgfscope}%
\pgfpathrectangle{\pgfqpoint{0.562500in}{0.275000in}}{\pgfqpoint{3.487500in}{1.925000in}}%
\pgfusepath{clip}%
\pgfsetbuttcap%
\pgfsetroundjoin%
\definecolor{currentfill}{rgb}{0.000000,0.000000,0.000000}%
\pgfsetfillcolor{currentfill}%
\pgfsetlinewidth{1.003750pt}%
\definecolor{currentstroke}{rgb}{0.000000,0.000000,0.000000}%
\pgfsetstrokecolor{currentstroke}%
\pgfsetdash{}{0pt}%
\pgfpathmoveto{\pgfqpoint{1.772992in}{0.356667in}}%
\pgfpathcurveto{\pgfqpoint{1.778517in}{0.356667in}}{\pgfqpoint{1.783816in}{0.358862in}}{\pgfqpoint{1.787723in}{0.362769in}}%
\pgfpathcurveto{\pgfqpoint{1.791630in}{0.366675in}}{\pgfqpoint{1.793825in}{0.371975in}}{\pgfqpoint{1.793825in}{0.377500in}}%
\pgfpathcurveto{\pgfqpoint{1.793825in}{0.383025in}}{\pgfqpoint{1.791630in}{0.388325in}}{\pgfqpoint{1.787723in}{0.392231in}}%
\pgfpathcurveto{\pgfqpoint{1.783816in}{0.396138in}}{\pgfqpoint{1.778517in}{0.398333in}}{\pgfqpoint{1.772992in}{0.398333in}}%
\pgfpathcurveto{\pgfqpoint{1.767467in}{0.398333in}}{\pgfqpoint{1.762167in}{0.396138in}}{\pgfqpoint{1.758260in}{0.392231in}}%
\pgfpathcurveto{\pgfqpoint{1.754353in}{0.388325in}}{\pgfqpoint{1.752158in}{0.383025in}}{\pgfqpoint{1.752158in}{0.377500in}}%
\pgfpathcurveto{\pgfqpoint{1.752158in}{0.371975in}}{\pgfqpoint{1.754353in}{0.366675in}}{\pgfqpoint{1.758260in}{0.362769in}}%
\pgfpathcurveto{\pgfqpoint{1.762167in}{0.358862in}}{\pgfqpoint{1.767467in}{0.356667in}}{\pgfqpoint{1.772992in}{0.356667in}}%
\pgfpathclose%
\pgfusepath{stroke,fill}%
\end{pgfscope}%
\begin{pgfscope}%
\pgfpathrectangle{\pgfqpoint{0.562500in}{0.275000in}}{\pgfqpoint{3.487500in}{1.925000in}}%
\pgfusepath{clip}%
\pgfsetbuttcap%
\pgfsetroundjoin%
\definecolor{currentfill}{rgb}{0.000000,0.000000,0.000000}%
\pgfsetfillcolor{currentfill}%
\pgfsetlinewidth{1.003750pt}%
\definecolor{currentstroke}{rgb}{0.000000,0.000000,0.000000}%
\pgfsetstrokecolor{currentstroke}%
\pgfsetdash{}{0pt}%
\pgfpathmoveto{\pgfqpoint{1.772992in}{0.356667in}}%
\pgfpathcurveto{\pgfqpoint{1.778517in}{0.356667in}}{\pgfqpoint{1.783816in}{0.358862in}}{\pgfqpoint{1.787723in}{0.362769in}}%
\pgfpathcurveto{\pgfqpoint{1.791630in}{0.366675in}}{\pgfqpoint{1.793825in}{0.371975in}}{\pgfqpoint{1.793825in}{0.377500in}}%
\pgfpathcurveto{\pgfqpoint{1.793825in}{0.383025in}}{\pgfqpoint{1.791630in}{0.388325in}}{\pgfqpoint{1.787723in}{0.392231in}}%
\pgfpathcurveto{\pgfqpoint{1.783816in}{0.396138in}}{\pgfqpoint{1.778517in}{0.398333in}}{\pgfqpoint{1.772992in}{0.398333in}}%
\pgfpathcurveto{\pgfqpoint{1.767467in}{0.398333in}}{\pgfqpoint{1.762167in}{0.396138in}}{\pgfqpoint{1.758260in}{0.392231in}}%
\pgfpathcurveto{\pgfqpoint{1.754353in}{0.388325in}}{\pgfqpoint{1.752158in}{0.383025in}}{\pgfqpoint{1.752158in}{0.377500in}}%
\pgfpathcurveto{\pgfqpoint{1.752158in}{0.371975in}}{\pgfqpoint{1.754353in}{0.366675in}}{\pgfqpoint{1.758260in}{0.362769in}}%
\pgfpathcurveto{\pgfqpoint{1.762167in}{0.358862in}}{\pgfqpoint{1.767467in}{0.356667in}}{\pgfqpoint{1.772992in}{0.356667in}}%
\pgfpathclose%
\pgfusepath{stroke,fill}%
\end{pgfscope}%
\begin{pgfscope}%
\pgfpathrectangle{\pgfqpoint{0.562500in}{0.275000in}}{\pgfqpoint{3.487500in}{1.925000in}}%
\pgfusepath{clip}%
\pgfsetbuttcap%
\pgfsetroundjoin%
\definecolor{currentfill}{rgb}{0.000000,0.000000,0.000000}%
\pgfsetfillcolor{currentfill}%
\pgfsetlinewidth{1.003750pt}%
\definecolor{currentstroke}{rgb}{0.000000,0.000000,0.000000}%
\pgfsetstrokecolor{currentstroke}%
\pgfsetdash{}{0pt}%
\pgfpathmoveto{\pgfqpoint{1.772992in}{0.356667in}}%
\pgfpathcurveto{\pgfqpoint{1.778517in}{0.356667in}}{\pgfqpoint{1.783816in}{0.358862in}}{\pgfqpoint{1.787723in}{0.362769in}}%
\pgfpathcurveto{\pgfqpoint{1.791630in}{0.366675in}}{\pgfqpoint{1.793825in}{0.371975in}}{\pgfqpoint{1.793825in}{0.377500in}}%
\pgfpathcurveto{\pgfqpoint{1.793825in}{0.383025in}}{\pgfqpoint{1.791630in}{0.388325in}}{\pgfqpoint{1.787723in}{0.392231in}}%
\pgfpathcurveto{\pgfqpoint{1.783816in}{0.396138in}}{\pgfqpoint{1.778517in}{0.398333in}}{\pgfqpoint{1.772992in}{0.398333in}}%
\pgfpathcurveto{\pgfqpoint{1.767467in}{0.398333in}}{\pgfqpoint{1.762167in}{0.396138in}}{\pgfqpoint{1.758260in}{0.392231in}}%
\pgfpathcurveto{\pgfqpoint{1.754353in}{0.388325in}}{\pgfqpoint{1.752158in}{0.383025in}}{\pgfqpoint{1.752158in}{0.377500in}}%
\pgfpathcurveto{\pgfqpoint{1.752158in}{0.371975in}}{\pgfqpoint{1.754353in}{0.366675in}}{\pgfqpoint{1.758260in}{0.362769in}}%
\pgfpathcurveto{\pgfqpoint{1.762167in}{0.358862in}}{\pgfqpoint{1.767467in}{0.356667in}}{\pgfqpoint{1.772992in}{0.356667in}}%
\pgfpathclose%
\pgfusepath{stroke,fill}%
\end{pgfscope}%
\begin{pgfscope}%
\pgfpathrectangle{\pgfqpoint{0.562500in}{0.275000in}}{\pgfqpoint{3.487500in}{1.925000in}}%
\pgfusepath{clip}%
\pgfsetbuttcap%
\pgfsetroundjoin%
\definecolor{currentfill}{rgb}{0.000000,0.000000,0.000000}%
\pgfsetfillcolor{currentfill}%
\pgfsetlinewidth{1.003750pt}%
\definecolor{currentstroke}{rgb}{0.000000,0.000000,0.000000}%
\pgfsetstrokecolor{currentstroke}%
\pgfsetdash{}{0pt}%
\pgfpathmoveto{\pgfqpoint{1.772992in}{0.356667in}}%
\pgfpathcurveto{\pgfqpoint{1.778517in}{0.356667in}}{\pgfqpoint{1.783816in}{0.358862in}}{\pgfqpoint{1.787723in}{0.362769in}}%
\pgfpathcurveto{\pgfqpoint{1.791630in}{0.366675in}}{\pgfqpoint{1.793825in}{0.371975in}}{\pgfqpoint{1.793825in}{0.377500in}}%
\pgfpathcurveto{\pgfqpoint{1.793825in}{0.383025in}}{\pgfqpoint{1.791630in}{0.388325in}}{\pgfqpoint{1.787723in}{0.392231in}}%
\pgfpathcurveto{\pgfqpoint{1.783816in}{0.396138in}}{\pgfqpoint{1.778517in}{0.398333in}}{\pgfqpoint{1.772992in}{0.398333in}}%
\pgfpathcurveto{\pgfqpoint{1.767467in}{0.398333in}}{\pgfqpoint{1.762167in}{0.396138in}}{\pgfqpoint{1.758260in}{0.392231in}}%
\pgfpathcurveto{\pgfqpoint{1.754353in}{0.388325in}}{\pgfqpoint{1.752158in}{0.383025in}}{\pgfqpoint{1.752158in}{0.377500in}}%
\pgfpathcurveto{\pgfqpoint{1.752158in}{0.371975in}}{\pgfqpoint{1.754353in}{0.366675in}}{\pgfqpoint{1.758260in}{0.362769in}}%
\pgfpathcurveto{\pgfqpoint{1.762167in}{0.358862in}}{\pgfqpoint{1.767467in}{0.356667in}}{\pgfqpoint{1.772992in}{0.356667in}}%
\pgfpathclose%
\pgfusepath{stroke,fill}%
\end{pgfscope}%
\begin{pgfscope}%
\pgfpathrectangle{\pgfqpoint{0.562500in}{0.275000in}}{\pgfqpoint{3.487500in}{1.925000in}}%
\pgfusepath{clip}%
\pgfsetbuttcap%
\pgfsetroundjoin%
\definecolor{currentfill}{rgb}{0.000000,0.000000,0.000000}%
\pgfsetfillcolor{currentfill}%
\pgfsetlinewidth{1.003750pt}%
\definecolor{currentstroke}{rgb}{0.000000,0.000000,0.000000}%
\pgfsetstrokecolor{currentstroke}%
\pgfsetdash{}{0pt}%
\pgfpathmoveto{\pgfqpoint{1.772992in}{0.356667in}}%
\pgfpathcurveto{\pgfqpoint{1.778517in}{0.356667in}}{\pgfqpoint{1.783816in}{0.358862in}}{\pgfqpoint{1.787723in}{0.362769in}}%
\pgfpathcurveto{\pgfqpoint{1.791630in}{0.366675in}}{\pgfqpoint{1.793825in}{0.371975in}}{\pgfqpoint{1.793825in}{0.377500in}}%
\pgfpathcurveto{\pgfqpoint{1.793825in}{0.383025in}}{\pgfqpoint{1.791630in}{0.388325in}}{\pgfqpoint{1.787723in}{0.392231in}}%
\pgfpathcurveto{\pgfqpoint{1.783816in}{0.396138in}}{\pgfqpoint{1.778517in}{0.398333in}}{\pgfqpoint{1.772992in}{0.398333in}}%
\pgfpathcurveto{\pgfqpoint{1.767467in}{0.398333in}}{\pgfqpoint{1.762167in}{0.396138in}}{\pgfqpoint{1.758260in}{0.392231in}}%
\pgfpathcurveto{\pgfqpoint{1.754353in}{0.388325in}}{\pgfqpoint{1.752158in}{0.383025in}}{\pgfqpoint{1.752158in}{0.377500in}}%
\pgfpathcurveto{\pgfqpoint{1.752158in}{0.371975in}}{\pgfqpoint{1.754353in}{0.366675in}}{\pgfqpoint{1.758260in}{0.362769in}}%
\pgfpathcurveto{\pgfqpoint{1.762167in}{0.358862in}}{\pgfqpoint{1.767467in}{0.356667in}}{\pgfqpoint{1.772992in}{0.356667in}}%
\pgfpathclose%
\pgfusepath{stroke,fill}%
\end{pgfscope}%
\begin{pgfscope}%
\pgfpathrectangle{\pgfqpoint{0.562500in}{0.275000in}}{\pgfqpoint{3.487500in}{1.925000in}}%
\pgfusepath{clip}%
\pgfsetbuttcap%
\pgfsetroundjoin%
\definecolor{currentfill}{rgb}{0.000000,0.000000,0.000000}%
\pgfsetfillcolor{currentfill}%
\pgfsetlinewidth{1.003750pt}%
\definecolor{currentstroke}{rgb}{0.000000,0.000000,0.000000}%
\pgfsetstrokecolor{currentstroke}%
\pgfsetdash{}{0pt}%
\pgfpathmoveto{\pgfqpoint{1.772992in}{0.356667in}}%
\pgfpathcurveto{\pgfqpoint{1.778517in}{0.356667in}}{\pgfqpoint{1.783816in}{0.358862in}}{\pgfqpoint{1.787723in}{0.362769in}}%
\pgfpathcurveto{\pgfqpoint{1.791630in}{0.366675in}}{\pgfqpoint{1.793825in}{0.371975in}}{\pgfqpoint{1.793825in}{0.377500in}}%
\pgfpathcurveto{\pgfqpoint{1.793825in}{0.383025in}}{\pgfqpoint{1.791630in}{0.388325in}}{\pgfqpoint{1.787723in}{0.392231in}}%
\pgfpathcurveto{\pgfqpoint{1.783816in}{0.396138in}}{\pgfqpoint{1.778517in}{0.398333in}}{\pgfqpoint{1.772992in}{0.398333in}}%
\pgfpathcurveto{\pgfqpoint{1.767467in}{0.398333in}}{\pgfqpoint{1.762167in}{0.396138in}}{\pgfqpoint{1.758260in}{0.392231in}}%
\pgfpathcurveto{\pgfqpoint{1.754353in}{0.388325in}}{\pgfqpoint{1.752158in}{0.383025in}}{\pgfqpoint{1.752158in}{0.377500in}}%
\pgfpathcurveto{\pgfqpoint{1.752158in}{0.371975in}}{\pgfqpoint{1.754353in}{0.366675in}}{\pgfqpoint{1.758260in}{0.362769in}}%
\pgfpathcurveto{\pgfqpoint{1.762167in}{0.358862in}}{\pgfqpoint{1.767467in}{0.356667in}}{\pgfqpoint{1.772992in}{0.356667in}}%
\pgfpathclose%
\pgfusepath{stroke,fill}%
\end{pgfscope}%
\begin{pgfscope}%
\pgfpathrectangle{\pgfqpoint{0.562500in}{0.275000in}}{\pgfqpoint{3.487500in}{1.925000in}}%
\pgfusepath{clip}%
\pgfsetbuttcap%
\pgfsetroundjoin%
\definecolor{currentfill}{rgb}{0.000000,0.000000,0.000000}%
\pgfsetfillcolor{currentfill}%
\pgfsetlinewidth{1.003750pt}%
\definecolor{currentstroke}{rgb}{0.000000,0.000000,0.000000}%
\pgfsetstrokecolor{currentstroke}%
\pgfsetdash{}{0pt}%
\pgfpathmoveto{\pgfqpoint{1.772992in}{0.356667in}}%
\pgfpathcurveto{\pgfqpoint{1.778517in}{0.356667in}}{\pgfqpoint{1.783816in}{0.358862in}}{\pgfqpoint{1.787723in}{0.362769in}}%
\pgfpathcurveto{\pgfqpoint{1.791630in}{0.366675in}}{\pgfqpoint{1.793825in}{0.371975in}}{\pgfqpoint{1.793825in}{0.377500in}}%
\pgfpathcurveto{\pgfqpoint{1.793825in}{0.383025in}}{\pgfqpoint{1.791630in}{0.388325in}}{\pgfqpoint{1.787723in}{0.392231in}}%
\pgfpathcurveto{\pgfqpoint{1.783816in}{0.396138in}}{\pgfqpoint{1.778517in}{0.398333in}}{\pgfqpoint{1.772992in}{0.398333in}}%
\pgfpathcurveto{\pgfqpoint{1.767467in}{0.398333in}}{\pgfqpoint{1.762167in}{0.396138in}}{\pgfqpoint{1.758260in}{0.392231in}}%
\pgfpathcurveto{\pgfqpoint{1.754353in}{0.388325in}}{\pgfqpoint{1.752158in}{0.383025in}}{\pgfqpoint{1.752158in}{0.377500in}}%
\pgfpathcurveto{\pgfqpoint{1.752158in}{0.371975in}}{\pgfqpoint{1.754353in}{0.366675in}}{\pgfqpoint{1.758260in}{0.362769in}}%
\pgfpathcurveto{\pgfqpoint{1.762167in}{0.358862in}}{\pgfqpoint{1.767467in}{0.356667in}}{\pgfqpoint{1.772992in}{0.356667in}}%
\pgfpathclose%
\pgfusepath{stroke,fill}%
\end{pgfscope}%
\begin{pgfscope}%
\pgfpathrectangle{\pgfqpoint{0.562500in}{0.275000in}}{\pgfqpoint{3.487500in}{1.925000in}}%
\pgfusepath{clip}%
\pgfsetbuttcap%
\pgfsetroundjoin%
\definecolor{currentfill}{rgb}{0.000000,0.000000,0.000000}%
\pgfsetfillcolor{currentfill}%
\pgfsetlinewidth{1.003750pt}%
\definecolor{currentstroke}{rgb}{0.000000,0.000000,0.000000}%
\pgfsetstrokecolor{currentstroke}%
\pgfsetdash{}{0pt}%
\pgfpathmoveto{\pgfqpoint{1.772992in}{0.356667in}}%
\pgfpathcurveto{\pgfqpoint{1.778517in}{0.356667in}}{\pgfqpoint{1.783816in}{0.358862in}}{\pgfqpoint{1.787723in}{0.362769in}}%
\pgfpathcurveto{\pgfqpoint{1.791630in}{0.366675in}}{\pgfqpoint{1.793825in}{0.371975in}}{\pgfqpoint{1.793825in}{0.377500in}}%
\pgfpathcurveto{\pgfqpoint{1.793825in}{0.383025in}}{\pgfqpoint{1.791630in}{0.388325in}}{\pgfqpoint{1.787723in}{0.392231in}}%
\pgfpathcurveto{\pgfqpoint{1.783816in}{0.396138in}}{\pgfqpoint{1.778517in}{0.398333in}}{\pgfqpoint{1.772992in}{0.398333in}}%
\pgfpathcurveto{\pgfqpoint{1.767467in}{0.398333in}}{\pgfqpoint{1.762167in}{0.396138in}}{\pgfqpoint{1.758260in}{0.392231in}}%
\pgfpathcurveto{\pgfqpoint{1.754353in}{0.388325in}}{\pgfqpoint{1.752158in}{0.383025in}}{\pgfqpoint{1.752158in}{0.377500in}}%
\pgfpathcurveto{\pgfqpoint{1.752158in}{0.371975in}}{\pgfqpoint{1.754353in}{0.366675in}}{\pgfqpoint{1.758260in}{0.362769in}}%
\pgfpathcurveto{\pgfqpoint{1.762167in}{0.358862in}}{\pgfqpoint{1.767467in}{0.356667in}}{\pgfqpoint{1.772992in}{0.356667in}}%
\pgfpathclose%
\pgfusepath{stroke,fill}%
\end{pgfscope}%
\begin{pgfscope}%
\pgfpathrectangle{\pgfqpoint{0.562500in}{0.275000in}}{\pgfqpoint{3.487500in}{1.925000in}}%
\pgfusepath{clip}%
\pgfsetbuttcap%
\pgfsetroundjoin%
\definecolor{currentfill}{rgb}{0.000000,0.000000,0.000000}%
\pgfsetfillcolor{currentfill}%
\pgfsetlinewidth{1.003750pt}%
\definecolor{currentstroke}{rgb}{0.000000,0.000000,0.000000}%
\pgfsetstrokecolor{currentstroke}%
\pgfsetdash{}{0pt}%
\pgfpathmoveto{\pgfqpoint{1.772992in}{0.356667in}}%
\pgfpathcurveto{\pgfqpoint{1.778517in}{0.356667in}}{\pgfqpoint{1.783816in}{0.358862in}}{\pgfqpoint{1.787723in}{0.362769in}}%
\pgfpathcurveto{\pgfqpoint{1.791630in}{0.366675in}}{\pgfqpoint{1.793825in}{0.371975in}}{\pgfqpoint{1.793825in}{0.377500in}}%
\pgfpathcurveto{\pgfqpoint{1.793825in}{0.383025in}}{\pgfqpoint{1.791630in}{0.388325in}}{\pgfqpoint{1.787723in}{0.392231in}}%
\pgfpathcurveto{\pgfqpoint{1.783816in}{0.396138in}}{\pgfqpoint{1.778517in}{0.398333in}}{\pgfqpoint{1.772992in}{0.398333in}}%
\pgfpathcurveto{\pgfqpoint{1.767467in}{0.398333in}}{\pgfqpoint{1.762167in}{0.396138in}}{\pgfqpoint{1.758260in}{0.392231in}}%
\pgfpathcurveto{\pgfqpoint{1.754353in}{0.388325in}}{\pgfqpoint{1.752158in}{0.383025in}}{\pgfqpoint{1.752158in}{0.377500in}}%
\pgfpathcurveto{\pgfqpoint{1.752158in}{0.371975in}}{\pgfqpoint{1.754353in}{0.366675in}}{\pgfqpoint{1.758260in}{0.362769in}}%
\pgfpathcurveto{\pgfqpoint{1.762167in}{0.358862in}}{\pgfqpoint{1.767467in}{0.356667in}}{\pgfqpoint{1.772992in}{0.356667in}}%
\pgfpathclose%
\pgfusepath{stroke,fill}%
\end{pgfscope}%
\begin{pgfscope}%
\pgfpathrectangle{\pgfqpoint{0.562500in}{0.275000in}}{\pgfqpoint{3.487500in}{1.925000in}}%
\pgfusepath{clip}%
\pgfsetbuttcap%
\pgfsetroundjoin%
\definecolor{currentfill}{rgb}{0.000000,0.000000,0.000000}%
\pgfsetfillcolor{currentfill}%
\pgfsetlinewidth{1.003750pt}%
\definecolor{currentstroke}{rgb}{0.000000,0.000000,0.000000}%
\pgfsetstrokecolor{currentstroke}%
\pgfsetdash{}{0pt}%
\pgfpathmoveto{\pgfqpoint{1.772992in}{0.356667in}}%
\pgfpathcurveto{\pgfqpoint{1.778517in}{0.356667in}}{\pgfqpoint{1.783816in}{0.358862in}}{\pgfqpoint{1.787723in}{0.362769in}}%
\pgfpathcurveto{\pgfqpoint{1.791630in}{0.366675in}}{\pgfqpoint{1.793825in}{0.371975in}}{\pgfqpoint{1.793825in}{0.377500in}}%
\pgfpathcurveto{\pgfqpoint{1.793825in}{0.383025in}}{\pgfqpoint{1.791630in}{0.388325in}}{\pgfqpoint{1.787723in}{0.392231in}}%
\pgfpathcurveto{\pgfqpoint{1.783816in}{0.396138in}}{\pgfqpoint{1.778517in}{0.398333in}}{\pgfqpoint{1.772992in}{0.398333in}}%
\pgfpathcurveto{\pgfqpoint{1.767467in}{0.398333in}}{\pgfqpoint{1.762167in}{0.396138in}}{\pgfqpoint{1.758260in}{0.392231in}}%
\pgfpathcurveto{\pgfqpoint{1.754353in}{0.388325in}}{\pgfqpoint{1.752158in}{0.383025in}}{\pgfqpoint{1.752158in}{0.377500in}}%
\pgfpathcurveto{\pgfqpoint{1.752158in}{0.371975in}}{\pgfqpoint{1.754353in}{0.366675in}}{\pgfqpoint{1.758260in}{0.362769in}}%
\pgfpathcurveto{\pgfqpoint{1.762167in}{0.358862in}}{\pgfqpoint{1.767467in}{0.356667in}}{\pgfqpoint{1.772992in}{0.356667in}}%
\pgfpathclose%
\pgfusepath{stroke,fill}%
\end{pgfscope}%
\begin{pgfscope}%
\pgfpathrectangle{\pgfqpoint{0.562500in}{0.275000in}}{\pgfqpoint{3.487500in}{1.925000in}}%
\pgfusepath{clip}%
\pgfsetbuttcap%
\pgfsetroundjoin%
\definecolor{currentfill}{rgb}{0.000000,0.000000,0.000000}%
\pgfsetfillcolor{currentfill}%
\pgfsetlinewidth{1.003750pt}%
\definecolor{currentstroke}{rgb}{0.000000,0.000000,0.000000}%
\pgfsetstrokecolor{currentstroke}%
\pgfsetdash{}{0pt}%
\pgfpathmoveto{\pgfqpoint{1.772992in}{0.356667in}}%
\pgfpathcurveto{\pgfqpoint{1.778517in}{0.356667in}}{\pgfqpoint{1.783816in}{0.358862in}}{\pgfqpoint{1.787723in}{0.362769in}}%
\pgfpathcurveto{\pgfqpoint{1.791630in}{0.366675in}}{\pgfqpoint{1.793825in}{0.371975in}}{\pgfqpoint{1.793825in}{0.377500in}}%
\pgfpathcurveto{\pgfqpoint{1.793825in}{0.383025in}}{\pgfqpoint{1.791630in}{0.388325in}}{\pgfqpoint{1.787723in}{0.392231in}}%
\pgfpathcurveto{\pgfqpoint{1.783816in}{0.396138in}}{\pgfqpoint{1.778517in}{0.398333in}}{\pgfqpoint{1.772992in}{0.398333in}}%
\pgfpathcurveto{\pgfqpoint{1.767467in}{0.398333in}}{\pgfqpoint{1.762167in}{0.396138in}}{\pgfqpoint{1.758260in}{0.392231in}}%
\pgfpathcurveto{\pgfqpoint{1.754353in}{0.388325in}}{\pgfqpoint{1.752158in}{0.383025in}}{\pgfqpoint{1.752158in}{0.377500in}}%
\pgfpathcurveto{\pgfqpoint{1.752158in}{0.371975in}}{\pgfqpoint{1.754353in}{0.366675in}}{\pgfqpoint{1.758260in}{0.362769in}}%
\pgfpathcurveto{\pgfqpoint{1.762167in}{0.358862in}}{\pgfqpoint{1.767467in}{0.356667in}}{\pgfqpoint{1.772992in}{0.356667in}}%
\pgfpathclose%
\pgfusepath{stroke,fill}%
\end{pgfscope}%
\begin{pgfscope}%
\pgfpathrectangle{\pgfqpoint{0.562500in}{0.275000in}}{\pgfqpoint{3.487500in}{1.925000in}}%
\pgfusepath{clip}%
\pgfsetbuttcap%
\pgfsetroundjoin%
\definecolor{currentfill}{rgb}{0.000000,0.000000,0.000000}%
\pgfsetfillcolor{currentfill}%
\pgfsetlinewidth{1.003750pt}%
\definecolor{currentstroke}{rgb}{0.000000,0.000000,0.000000}%
\pgfsetstrokecolor{currentstroke}%
\pgfsetdash{}{0pt}%
\pgfpathmoveto{\pgfqpoint{1.772992in}{0.356667in}}%
\pgfpathcurveto{\pgfqpoint{1.778517in}{0.356667in}}{\pgfqpoint{1.783816in}{0.358862in}}{\pgfqpoint{1.787723in}{0.362769in}}%
\pgfpathcurveto{\pgfqpoint{1.791630in}{0.366675in}}{\pgfqpoint{1.793825in}{0.371975in}}{\pgfqpoint{1.793825in}{0.377500in}}%
\pgfpathcurveto{\pgfqpoint{1.793825in}{0.383025in}}{\pgfqpoint{1.791630in}{0.388325in}}{\pgfqpoint{1.787723in}{0.392231in}}%
\pgfpathcurveto{\pgfqpoint{1.783816in}{0.396138in}}{\pgfqpoint{1.778517in}{0.398333in}}{\pgfqpoint{1.772992in}{0.398333in}}%
\pgfpathcurveto{\pgfqpoint{1.767467in}{0.398333in}}{\pgfqpoint{1.762167in}{0.396138in}}{\pgfqpoint{1.758260in}{0.392231in}}%
\pgfpathcurveto{\pgfqpoint{1.754353in}{0.388325in}}{\pgfqpoint{1.752158in}{0.383025in}}{\pgfqpoint{1.752158in}{0.377500in}}%
\pgfpathcurveto{\pgfqpoint{1.752158in}{0.371975in}}{\pgfqpoint{1.754353in}{0.366675in}}{\pgfqpoint{1.758260in}{0.362769in}}%
\pgfpathcurveto{\pgfqpoint{1.762167in}{0.358862in}}{\pgfqpoint{1.767467in}{0.356667in}}{\pgfqpoint{1.772992in}{0.356667in}}%
\pgfpathclose%
\pgfusepath{stroke,fill}%
\end{pgfscope}%
\begin{pgfscope}%
\pgfpathrectangle{\pgfqpoint{0.562500in}{0.275000in}}{\pgfqpoint{3.487500in}{1.925000in}}%
\pgfusepath{clip}%
\pgfsetbuttcap%
\pgfsetroundjoin%
\definecolor{currentfill}{rgb}{0.000000,0.000000,0.000000}%
\pgfsetfillcolor{currentfill}%
\pgfsetlinewidth{1.003750pt}%
\definecolor{currentstroke}{rgb}{0.000000,0.000000,0.000000}%
\pgfsetstrokecolor{currentstroke}%
\pgfsetdash{}{0pt}%
\pgfpathmoveto{\pgfqpoint{1.772992in}{0.356667in}}%
\pgfpathcurveto{\pgfqpoint{1.778517in}{0.356667in}}{\pgfqpoint{1.783816in}{0.358862in}}{\pgfqpoint{1.787723in}{0.362769in}}%
\pgfpathcurveto{\pgfqpoint{1.791630in}{0.366675in}}{\pgfqpoint{1.793825in}{0.371975in}}{\pgfqpoint{1.793825in}{0.377500in}}%
\pgfpathcurveto{\pgfqpoint{1.793825in}{0.383025in}}{\pgfqpoint{1.791630in}{0.388325in}}{\pgfqpoint{1.787723in}{0.392231in}}%
\pgfpathcurveto{\pgfqpoint{1.783816in}{0.396138in}}{\pgfqpoint{1.778517in}{0.398333in}}{\pgfqpoint{1.772992in}{0.398333in}}%
\pgfpathcurveto{\pgfqpoint{1.767467in}{0.398333in}}{\pgfqpoint{1.762167in}{0.396138in}}{\pgfqpoint{1.758260in}{0.392231in}}%
\pgfpathcurveto{\pgfqpoint{1.754353in}{0.388325in}}{\pgfqpoint{1.752158in}{0.383025in}}{\pgfqpoint{1.752158in}{0.377500in}}%
\pgfpathcurveto{\pgfqpoint{1.752158in}{0.371975in}}{\pgfqpoint{1.754353in}{0.366675in}}{\pgfqpoint{1.758260in}{0.362769in}}%
\pgfpathcurveto{\pgfqpoint{1.762167in}{0.358862in}}{\pgfqpoint{1.767467in}{0.356667in}}{\pgfqpoint{1.772992in}{0.356667in}}%
\pgfpathclose%
\pgfusepath{stroke,fill}%
\end{pgfscope}%
\begin{pgfscope}%
\pgfpathrectangle{\pgfqpoint{0.562500in}{0.275000in}}{\pgfqpoint{3.487500in}{1.925000in}}%
\pgfusepath{clip}%
\pgfsetbuttcap%
\pgfsetroundjoin%
\definecolor{currentfill}{rgb}{0.000000,0.000000,0.000000}%
\pgfsetfillcolor{currentfill}%
\pgfsetlinewidth{1.003750pt}%
\definecolor{currentstroke}{rgb}{0.000000,0.000000,0.000000}%
\pgfsetstrokecolor{currentstroke}%
\pgfsetdash{}{0pt}%
\pgfpathmoveto{\pgfqpoint{1.772992in}{0.356667in}}%
\pgfpathcurveto{\pgfqpoint{1.778517in}{0.356667in}}{\pgfqpoint{1.783816in}{0.358862in}}{\pgfqpoint{1.787723in}{0.362769in}}%
\pgfpathcurveto{\pgfqpoint{1.791630in}{0.366675in}}{\pgfqpoint{1.793825in}{0.371975in}}{\pgfqpoint{1.793825in}{0.377500in}}%
\pgfpathcurveto{\pgfqpoint{1.793825in}{0.383025in}}{\pgfqpoint{1.791630in}{0.388325in}}{\pgfqpoint{1.787723in}{0.392231in}}%
\pgfpathcurveto{\pgfqpoint{1.783816in}{0.396138in}}{\pgfqpoint{1.778517in}{0.398333in}}{\pgfqpoint{1.772992in}{0.398333in}}%
\pgfpathcurveto{\pgfqpoint{1.767467in}{0.398333in}}{\pgfqpoint{1.762167in}{0.396138in}}{\pgfqpoint{1.758260in}{0.392231in}}%
\pgfpathcurveto{\pgfqpoint{1.754353in}{0.388325in}}{\pgfqpoint{1.752158in}{0.383025in}}{\pgfqpoint{1.752158in}{0.377500in}}%
\pgfpathcurveto{\pgfqpoint{1.752158in}{0.371975in}}{\pgfqpoint{1.754353in}{0.366675in}}{\pgfqpoint{1.758260in}{0.362769in}}%
\pgfpathcurveto{\pgfqpoint{1.762167in}{0.358862in}}{\pgfqpoint{1.767467in}{0.356667in}}{\pgfqpoint{1.772992in}{0.356667in}}%
\pgfpathclose%
\pgfusepath{stroke,fill}%
\end{pgfscope}%
\begin{pgfscope}%
\pgfpathrectangle{\pgfqpoint{0.562500in}{0.275000in}}{\pgfqpoint{3.487500in}{1.925000in}}%
\pgfusepath{clip}%
\pgfsetbuttcap%
\pgfsetroundjoin%
\definecolor{currentfill}{rgb}{0.000000,0.000000,0.000000}%
\pgfsetfillcolor{currentfill}%
\pgfsetlinewidth{1.003750pt}%
\definecolor{currentstroke}{rgb}{0.000000,0.000000,0.000000}%
\pgfsetstrokecolor{currentstroke}%
\pgfsetdash{}{0pt}%
\pgfpathmoveto{\pgfqpoint{1.772992in}{0.356667in}}%
\pgfpathcurveto{\pgfqpoint{1.778517in}{0.356667in}}{\pgfqpoint{1.783816in}{0.358862in}}{\pgfqpoint{1.787723in}{0.362769in}}%
\pgfpathcurveto{\pgfqpoint{1.791630in}{0.366675in}}{\pgfqpoint{1.793825in}{0.371975in}}{\pgfqpoint{1.793825in}{0.377500in}}%
\pgfpathcurveto{\pgfqpoint{1.793825in}{0.383025in}}{\pgfqpoint{1.791630in}{0.388325in}}{\pgfqpoint{1.787723in}{0.392231in}}%
\pgfpathcurveto{\pgfqpoint{1.783816in}{0.396138in}}{\pgfqpoint{1.778517in}{0.398333in}}{\pgfqpoint{1.772992in}{0.398333in}}%
\pgfpathcurveto{\pgfqpoint{1.767467in}{0.398333in}}{\pgfqpoint{1.762167in}{0.396138in}}{\pgfqpoint{1.758260in}{0.392231in}}%
\pgfpathcurveto{\pgfqpoint{1.754353in}{0.388325in}}{\pgfqpoint{1.752158in}{0.383025in}}{\pgfqpoint{1.752158in}{0.377500in}}%
\pgfpathcurveto{\pgfqpoint{1.752158in}{0.371975in}}{\pgfqpoint{1.754353in}{0.366675in}}{\pgfqpoint{1.758260in}{0.362769in}}%
\pgfpathcurveto{\pgfqpoint{1.762167in}{0.358862in}}{\pgfqpoint{1.767467in}{0.356667in}}{\pgfqpoint{1.772992in}{0.356667in}}%
\pgfpathclose%
\pgfusepath{stroke,fill}%
\end{pgfscope}%
\begin{pgfscope}%
\pgfpathrectangle{\pgfqpoint{0.562500in}{0.275000in}}{\pgfqpoint{3.487500in}{1.925000in}}%
\pgfusepath{clip}%
\pgfsetbuttcap%
\pgfsetroundjoin%
\definecolor{currentfill}{rgb}{0.000000,0.000000,0.000000}%
\pgfsetfillcolor{currentfill}%
\pgfsetlinewidth{1.003750pt}%
\definecolor{currentstroke}{rgb}{0.000000,0.000000,0.000000}%
\pgfsetstrokecolor{currentstroke}%
\pgfsetdash{}{0pt}%
\pgfpathmoveto{\pgfqpoint{1.772992in}{0.356667in}}%
\pgfpathcurveto{\pgfqpoint{1.778517in}{0.356667in}}{\pgfqpoint{1.783816in}{0.358862in}}{\pgfqpoint{1.787723in}{0.362769in}}%
\pgfpathcurveto{\pgfqpoint{1.791630in}{0.366675in}}{\pgfqpoint{1.793825in}{0.371975in}}{\pgfqpoint{1.793825in}{0.377500in}}%
\pgfpathcurveto{\pgfqpoint{1.793825in}{0.383025in}}{\pgfqpoint{1.791630in}{0.388325in}}{\pgfqpoint{1.787723in}{0.392231in}}%
\pgfpathcurveto{\pgfqpoint{1.783816in}{0.396138in}}{\pgfqpoint{1.778517in}{0.398333in}}{\pgfqpoint{1.772992in}{0.398333in}}%
\pgfpathcurveto{\pgfqpoint{1.767467in}{0.398333in}}{\pgfqpoint{1.762167in}{0.396138in}}{\pgfqpoint{1.758260in}{0.392231in}}%
\pgfpathcurveto{\pgfqpoint{1.754353in}{0.388325in}}{\pgfqpoint{1.752158in}{0.383025in}}{\pgfqpoint{1.752158in}{0.377500in}}%
\pgfpathcurveto{\pgfqpoint{1.752158in}{0.371975in}}{\pgfqpoint{1.754353in}{0.366675in}}{\pgfqpoint{1.758260in}{0.362769in}}%
\pgfpathcurveto{\pgfqpoint{1.762167in}{0.358862in}}{\pgfqpoint{1.767467in}{0.356667in}}{\pgfqpoint{1.772992in}{0.356667in}}%
\pgfpathclose%
\pgfusepath{stroke,fill}%
\end{pgfscope}%
\begin{pgfscope}%
\pgfpathrectangle{\pgfqpoint{0.562500in}{0.275000in}}{\pgfqpoint{3.487500in}{1.925000in}}%
\pgfusepath{clip}%
\pgfsetbuttcap%
\pgfsetroundjoin%
\definecolor{currentfill}{rgb}{0.000000,0.000000,0.000000}%
\pgfsetfillcolor{currentfill}%
\pgfsetlinewidth{1.003750pt}%
\definecolor{currentstroke}{rgb}{0.000000,0.000000,0.000000}%
\pgfsetstrokecolor{currentstroke}%
\pgfsetdash{}{0pt}%
\pgfpathmoveto{\pgfqpoint{1.772992in}{0.356667in}}%
\pgfpathcurveto{\pgfqpoint{1.778517in}{0.356667in}}{\pgfqpoint{1.783816in}{0.358862in}}{\pgfqpoint{1.787723in}{0.362769in}}%
\pgfpathcurveto{\pgfqpoint{1.791630in}{0.366675in}}{\pgfqpoint{1.793825in}{0.371975in}}{\pgfqpoint{1.793825in}{0.377500in}}%
\pgfpathcurveto{\pgfqpoint{1.793825in}{0.383025in}}{\pgfqpoint{1.791630in}{0.388325in}}{\pgfqpoint{1.787723in}{0.392231in}}%
\pgfpathcurveto{\pgfqpoint{1.783816in}{0.396138in}}{\pgfqpoint{1.778517in}{0.398333in}}{\pgfqpoint{1.772992in}{0.398333in}}%
\pgfpathcurveto{\pgfqpoint{1.767467in}{0.398333in}}{\pgfqpoint{1.762167in}{0.396138in}}{\pgfqpoint{1.758260in}{0.392231in}}%
\pgfpathcurveto{\pgfqpoint{1.754353in}{0.388325in}}{\pgfqpoint{1.752158in}{0.383025in}}{\pgfqpoint{1.752158in}{0.377500in}}%
\pgfpathcurveto{\pgfqpoint{1.752158in}{0.371975in}}{\pgfqpoint{1.754353in}{0.366675in}}{\pgfqpoint{1.758260in}{0.362769in}}%
\pgfpathcurveto{\pgfqpoint{1.762167in}{0.358862in}}{\pgfqpoint{1.767467in}{0.356667in}}{\pgfqpoint{1.772992in}{0.356667in}}%
\pgfpathclose%
\pgfusepath{stroke,fill}%
\end{pgfscope}%
\begin{pgfscope}%
\pgfpathrectangle{\pgfqpoint{0.562500in}{0.275000in}}{\pgfqpoint{3.487500in}{1.925000in}}%
\pgfusepath{clip}%
\pgfsetbuttcap%
\pgfsetroundjoin%
\definecolor{currentfill}{rgb}{0.000000,0.000000,0.000000}%
\pgfsetfillcolor{currentfill}%
\pgfsetlinewidth{1.003750pt}%
\definecolor{currentstroke}{rgb}{0.000000,0.000000,0.000000}%
\pgfsetstrokecolor{currentstroke}%
\pgfsetdash{}{0pt}%
\pgfpathmoveto{\pgfqpoint{1.772992in}{0.356667in}}%
\pgfpathcurveto{\pgfqpoint{1.778517in}{0.356667in}}{\pgfqpoint{1.783816in}{0.358862in}}{\pgfqpoint{1.787723in}{0.362769in}}%
\pgfpathcurveto{\pgfqpoint{1.791630in}{0.366675in}}{\pgfqpoint{1.793825in}{0.371975in}}{\pgfqpoint{1.793825in}{0.377500in}}%
\pgfpathcurveto{\pgfqpoint{1.793825in}{0.383025in}}{\pgfqpoint{1.791630in}{0.388325in}}{\pgfqpoint{1.787723in}{0.392231in}}%
\pgfpathcurveto{\pgfqpoint{1.783816in}{0.396138in}}{\pgfqpoint{1.778517in}{0.398333in}}{\pgfqpoint{1.772992in}{0.398333in}}%
\pgfpathcurveto{\pgfqpoint{1.767467in}{0.398333in}}{\pgfqpoint{1.762167in}{0.396138in}}{\pgfqpoint{1.758260in}{0.392231in}}%
\pgfpathcurveto{\pgfqpoint{1.754353in}{0.388325in}}{\pgfqpoint{1.752158in}{0.383025in}}{\pgfqpoint{1.752158in}{0.377500in}}%
\pgfpathcurveto{\pgfqpoint{1.752158in}{0.371975in}}{\pgfqpoint{1.754353in}{0.366675in}}{\pgfqpoint{1.758260in}{0.362769in}}%
\pgfpathcurveto{\pgfqpoint{1.762167in}{0.358862in}}{\pgfqpoint{1.767467in}{0.356667in}}{\pgfqpoint{1.772992in}{0.356667in}}%
\pgfpathclose%
\pgfusepath{stroke,fill}%
\end{pgfscope}%
\begin{pgfscope}%
\pgfpathrectangle{\pgfqpoint{0.562500in}{0.275000in}}{\pgfqpoint{3.487500in}{1.925000in}}%
\pgfusepath{clip}%
\pgfsetbuttcap%
\pgfsetroundjoin%
\definecolor{currentfill}{rgb}{0.000000,0.000000,0.000000}%
\pgfsetfillcolor{currentfill}%
\pgfsetlinewidth{1.003750pt}%
\definecolor{currentstroke}{rgb}{0.000000,0.000000,0.000000}%
\pgfsetstrokecolor{currentstroke}%
\pgfsetdash{}{0pt}%
\pgfpathmoveto{\pgfqpoint{1.772992in}{0.356667in}}%
\pgfpathcurveto{\pgfqpoint{1.778517in}{0.356667in}}{\pgfqpoint{1.783816in}{0.358862in}}{\pgfqpoint{1.787723in}{0.362769in}}%
\pgfpathcurveto{\pgfqpoint{1.791630in}{0.366675in}}{\pgfqpoint{1.793825in}{0.371975in}}{\pgfqpoint{1.793825in}{0.377500in}}%
\pgfpathcurveto{\pgfqpoint{1.793825in}{0.383025in}}{\pgfqpoint{1.791630in}{0.388325in}}{\pgfqpoint{1.787723in}{0.392231in}}%
\pgfpathcurveto{\pgfqpoint{1.783816in}{0.396138in}}{\pgfqpoint{1.778517in}{0.398333in}}{\pgfqpoint{1.772992in}{0.398333in}}%
\pgfpathcurveto{\pgfqpoint{1.767467in}{0.398333in}}{\pgfqpoint{1.762167in}{0.396138in}}{\pgfqpoint{1.758260in}{0.392231in}}%
\pgfpathcurveto{\pgfqpoint{1.754353in}{0.388325in}}{\pgfqpoint{1.752158in}{0.383025in}}{\pgfqpoint{1.752158in}{0.377500in}}%
\pgfpathcurveto{\pgfqpoint{1.752158in}{0.371975in}}{\pgfqpoint{1.754353in}{0.366675in}}{\pgfqpoint{1.758260in}{0.362769in}}%
\pgfpathcurveto{\pgfqpoint{1.762167in}{0.358862in}}{\pgfqpoint{1.767467in}{0.356667in}}{\pgfqpoint{1.772992in}{0.356667in}}%
\pgfpathclose%
\pgfusepath{stroke,fill}%
\end{pgfscope}%
\begin{pgfscope}%
\pgfpathrectangle{\pgfqpoint{0.562500in}{0.275000in}}{\pgfqpoint{3.487500in}{1.925000in}}%
\pgfusepath{clip}%
\pgfsetbuttcap%
\pgfsetroundjoin%
\definecolor{currentfill}{rgb}{0.000000,0.000000,0.000000}%
\pgfsetfillcolor{currentfill}%
\pgfsetlinewidth{1.003750pt}%
\definecolor{currentstroke}{rgb}{0.000000,0.000000,0.000000}%
\pgfsetstrokecolor{currentstroke}%
\pgfsetdash{}{0pt}%
\pgfpathmoveto{\pgfqpoint{1.772992in}{0.356667in}}%
\pgfpathcurveto{\pgfqpoint{1.778517in}{0.356667in}}{\pgfqpoint{1.783816in}{0.358862in}}{\pgfqpoint{1.787723in}{0.362769in}}%
\pgfpathcurveto{\pgfqpoint{1.791630in}{0.366675in}}{\pgfqpoint{1.793825in}{0.371975in}}{\pgfqpoint{1.793825in}{0.377500in}}%
\pgfpathcurveto{\pgfqpoint{1.793825in}{0.383025in}}{\pgfqpoint{1.791630in}{0.388325in}}{\pgfqpoint{1.787723in}{0.392231in}}%
\pgfpathcurveto{\pgfqpoint{1.783816in}{0.396138in}}{\pgfqpoint{1.778517in}{0.398333in}}{\pgfqpoint{1.772992in}{0.398333in}}%
\pgfpathcurveto{\pgfqpoint{1.767467in}{0.398333in}}{\pgfqpoint{1.762167in}{0.396138in}}{\pgfqpoint{1.758260in}{0.392231in}}%
\pgfpathcurveto{\pgfqpoint{1.754353in}{0.388325in}}{\pgfqpoint{1.752158in}{0.383025in}}{\pgfqpoint{1.752158in}{0.377500in}}%
\pgfpathcurveto{\pgfqpoint{1.752158in}{0.371975in}}{\pgfqpoint{1.754353in}{0.366675in}}{\pgfqpoint{1.758260in}{0.362769in}}%
\pgfpathcurveto{\pgfqpoint{1.762167in}{0.358862in}}{\pgfqpoint{1.767467in}{0.356667in}}{\pgfqpoint{1.772992in}{0.356667in}}%
\pgfpathclose%
\pgfusepath{stroke,fill}%
\end{pgfscope}%
\begin{pgfscope}%
\pgfpathrectangle{\pgfqpoint{0.562500in}{0.275000in}}{\pgfqpoint{3.487500in}{1.925000in}}%
\pgfusepath{clip}%
\pgfsetbuttcap%
\pgfsetroundjoin%
\definecolor{currentfill}{rgb}{0.000000,0.000000,0.000000}%
\pgfsetfillcolor{currentfill}%
\pgfsetlinewidth{1.003750pt}%
\definecolor{currentstroke}{rgb}{0.000000,0.000000,0.000000}%
\pgfsetstrokecolor{currentstroke}%
\pgfsetdash{}{0pt}%
\pgfpathmoveto{\pgfqpoint{1.772992in}{0.356667in}}%
\pgfpathcurveto{\pgfqpoint{1.778517in}{0.356667in}}{\pgfqpoint{1.783816in}{0.358862in}}{\pgfqpoint{1.787723in}{0.362769in}}%
\pgfpathcurveto{\pgfqpoint{1.791630in}{0.366675in}}{\pgfqpoint{1.793825in}{0.371975in}}{\pgfqpoint{1.793825in}{0.377500in}}%
\pgfpathcurveto{\pgfqpoint{1.793825in}{0.383025in}}{\pgfqpoint{1.791630in}{0.388325in}}{\pgfqpoint{1.787723in}{0.392231in}}%
\pgfpathcurveto{\pgfqpoint{1.783816in}{0.396138in}}{\pgfqpoint{1.778517in}{0.398333in}}{\pgfqpoint{1.772992in}{0.398333in}}%
\pgfpathcurveto{\pgfqpoint{1.767467in}{0.398333in}}{\pgfqpoint{1.762167in}{0.396138in}}{\pgfqpoint{1.758260in}{0.392231in}}%
\pgfpathcurveto{\pgfqpoint{1.754353in}{0.388325in}}{\pgfqpoint{1.752158in}{0.383025in}}{\pgfqpoint{1.752158in}{0.377500in}}%
\pgfpathcurveto{\pgfqpoint{1.752158in}{0.371975in}}{\pgfqpoint{1.754353in}{0.366675in}}{\pgfqpoint{1.758260in}{0.362769in}}%
\pgfpathcurveto{\pgfqpoint{1.762167in}{0.358862in}}{\pgfqpoint{1.767467in}{0.356667in}}{\pgfqpoint{1.772992in}{0.356667in}}%
\pgfpathclose%
\pgfusepath{stroke,fill}%
\end{pgfscope}%
\begin{pgfscope}%
\pgfpathrectangle{\pgfqpoint{0.562500in}{0.275000in}}{\pgfqpoint{3.487500in}{1.925000in}}%
\pgfusepath{clip}%
\pgfsetbuttcap%
\pgfsetroundjoin%
\definecolor{currentfill}{rgb}{0.000000,0.000000,0.000000}%
\pgfsetfillcolor{currentfill}%
\pgfsetlinewidth{1.003750pt}%
\definecolor{currentstroke}{rgb}{0.000000,0.000000,0.000000}%
\pgfsetstrokecolor{currentstroke}%
\pgfsetdash{}{0pt}%
\pgfpathmoveto{\pgfqpoint{1.772992in}{0.356667in}}%
\pgfpathcurveto{\pgfqpoint{1.778517in}{0.356667in}}{\pgfqpoint{1.783816in}{0.358862in}}{\pgfqpoint{1.787723in}{0.362769in}}%
\pgfpathcurveto{\pgfqpoint{1.791630in}{0.366675in}}{\pgfqpoint{1.793825in}{0.371975in}}{\pgfqpoint{1.793825in}{0.377500in}}%
\pgfpathcurveto{\pgfqpoint{1.793825in}{0.383025in}}{\pgfqpoint{1.791630in}{0.388325in}}{\pgfqpoint{1.787723in}{0.392231in}}%
\pgfpathcurveto{\pgfqpoint{1.783816in}{0.396138in}}{\pgfqpoint{1.778517in}{0.398333in}}{\pgfqpoint{1.772992in}{0.398333in}}%
\pgfpathcurveto{\pgfqpoint{1.767467in}{0.398333in}}{\pgfqpoint{1.762167in}{0.396138in}}{\pgfqpoint{1.758260in}{0.392231in}}%
\pgfpathcurveto{\pgfqpoint{1.754353in}{0.388325in}}{\pgfqpoint{1.752158in}{0.383025in}}{\pgfqpoint{1.752158in}{0.377500in}}%
\pgfpathcurveto{\pgfqpoint{1.752158in}{0.371975in}}{\pgfqpoint{1.754353in}{0.366675in}}{\pgfqpoint{1.758260in}{0.362769in}}%
\pgfpathcurveto{\pgfqpoint{1.762167in}{0.358862in}}{\pgfqpoint{1.767467in}{0.356667in}}{\pgfqpoint{1.772992in}{0.356667in}}%
\pgfpathclose%
\pgfusepath{stroke,fill}%
\end{pgfscope}%
\begin{pgfscope}%
\pgfpathrectangle{\pgfqpoint{0.562500in}{0.275000in}}{\pgfqpoint{3.487500in}{1.925000in}}%
\pgfusepath{clip}%
\pgfsetbuttcap%
\pgfsetroundjoin%
\definecolor{currentfill}{rgb}{0.000000,0.000000,0.000000}%
\pgfsetfillcolor{currentfill}%
\pgfsetlinewidth{1.003750pt}%
\definecolor{currentstroke}{rgb}{0.000000,0.000000,0.000000}%
\pgfsetstrokecolor{currentstroke}%
\pgfsetdash{}{0pt}%
\pgfpathmoveto{\pgfqpoint{1.772992in}{0.356667in}}%
\pgfpathcurveto{\pgfqpoint{1.778517in}{0.356667in}}{\pgfqpoint{1.783816in}{0.358862in}}{\pgfqpoint{1.787723in}{0.362769in}}%
\pgfpathcurveto{\pgfqpoint{1.791630in}{0.366675in}}{\pgfqpoint{1.793825in}{0.371975in}}{\pgfqpoint{1.793825in}{0.377500in}}%
\pgfpathcurveto{\pgfqpoint{1.793825in}{0.383025in}}{\pgfqpoint{1.791630in}{0.388325in}}{\pgfqpoint{1.787723in}{0.392231in}}%
\pgfpathcurveto{\pgfqpoint{1.783816in}{0.396138in}}{\pgfqpoint{1.778517in}{0.398333in}}{\pgfqpoint{1.772992in}{0.398333in}}%
\pgfpathcurveto{\pgfqpoint{1.767467in}{0.398333in}}{\pgfqpoint{1.762167in}{0.396138in}}{\pgfqpoint{1.758260in}{0.392231in}}%
\pgfpathcurveto{\pgfqpoint{1.754353in}{0.388325in}}{\pgfqpoint{1.752158in}{0.383025in}}{\pgfqpoint{1.752158in}{0.377500in}}%
\pgfpathcurveto{\pgfqpoint{1.752158in}{0.371975in}}{\pgfqpoint{1.754353in}{0.366675in}}{\pgfqpoint{1.758260in}{0.362769in}}%
\pgfpathcurveto{\pgfqpoint{1.762167in}{0.358862in}}{\pgfqpoint{1.767467in}{0.356667in}}{\pgfqpoint{1.772992in}{0.356667in}}%
\pgfpathclose%
\pgfusepath{stroke,fill}%
\end{pgfscope}%
\begin{pgfscope}%
\pgfpathrectangle{\pgfqpoint{0.562500in}{0.275000in}}{\pgfqpoint{3.487500in}{1.925000in}}%
\pgfusepath{clip}%
\pgfsetbuttcap%
\pgfsetroundjoin%
\definecolor{currentfill}{rgb}{0.000000,0.000000,0.000000}%
\pgfsetfillcolor{currentfill}%
\pgfsetlinewidth{1.003750pt}%
\definecolor{currentstroke}{rgb}{0.000000,0.000000,0.000000}%
\pgfsetstrokecolor{currentstroke}%
\pgfsetdash{}{0pt}%
\pgfpathmoveto{\pgfqpoint{1.772992in}{0.356667in}}%
\pgfpathcurveto{\pgfqpoint{1.778517in}{0.356667in}}{\pgfqpoint{1.783816in}{0.358862in}}{\pgfqpoint{1.787723in}{0.362769in}}%
\pgfpathcurveto{\pgfqpoint{1.791630in}{0.366675in}}{\pgfqpoint{1.793825in}{0.371975in}}{\pgfqpoint{1.793825in}{0.377500in}}%
\pgfpathcurveto{\pgfqpoint{1.793825in}{0.383025in}}{\pgfqpoint{1.791630in}{0.388325in}}{\pgfqpoint{1.787723in}{0.392231in}}%
\pgfpathcurveto{\pgfqpoint{1.783816in}{0.396138in}}{\pgfqpoint{1.778517in}{0.398333in}}{\pgfqpoint{1.772992in}{0.398333in}}%
\pgfpathcurveto{\pgfqpoint{1.767467in}{0.398333in}}{\pgfqpoint{1.762167in}{0.396138in}}{\pgfqpoint{1.758260in}{0.392231in}}%
\pgfpathcurveto{\pgfqpoint{1.754353in}{0.388325in}}{\pgfqpoint{1.752158in}{0.383025in}}{\pgfqpoint{1.752158in}{0.377500in}}%
\pgfpathcurveto{\pgfqpoint{1.752158in}{0.371975in}}{\pgfqpoint{1.754353in}{0.366675in}}{\pgfqpoint{1.758260in}{0.362769in}}%
\pgfpathcurveto{\pgfqpoint{1.762167in}{0.358862in}}{\pgfqpoint{1.767467in}{0.356667in}}{\pgfqpoint{1.772992in}{0.356667in}}%
\pgfpathclose%
\pgfusepath{stroke,fill}%
\end{pgfscope}%
\begin{pgfscope}%
\pgfpathrectangle{\pgfqpoint{0.562500in}{0.275000in}}{\pgfqpoint{3.487500in}{1.925000in}}%
\pgfusepath{clip}%
\pgfsetbuttcap%
\pgfsetroundjoin%
\definecolor{currentfill}{rgb}{0.000000,0.000000,0.000000}%
\pgfsetfillcolor{currentfill}%
\pgfsetlinewidth{1.003750pt}%
\definecolor{currentstroke}{rgb}{0.000000,0.000000,0.000000}%
\pgfsetstrokecolor{currentstroke}%
\pgfsetdash{}{0pt}%
\pgfpathmoveto{\pgfqpoint{1.772992in}{0.356667in}}%
\pgfpathcurveto{\pgfqpoint{1.778517in}{0.356667in}}{\pgfqpoint{1.783816in}{0.358862in}}{\pgfqpoint{1.787723in}{0.362769in}}%
\pgfpathcurveto{\pgfqpoint{1.791630in}{0.366675in}}{\pgfqpoint{1.793825in}{0.371975in}}{\pgfqpoint{1.793825in}{0.377500in}}%
\pgfpathcurveto{\pgfqpoint{1.793825in}{0.383025in}}{\pgfqpoint{1.791630in}{0.388325in}}{\pgfqpoint{1.787723in}{0.392231in}}%
\pgfpathcurveto{\pgfqpoint{1.783816in}{0.396138in}}{\pgfqpoint{1.778517in}{0.398333in}}{\pgfqpoint{1.772992in}{0.398333in}}%
\pgfpathcurveto{\pgfqpoint{1.767467in}{0.398333in}}{\pgfqpoint{1.762167in}{0.396138in}}{\pgfqpoint{1.758260in}{0.392231in}}%
\pgfpathcurveto{\pgfqpoint{1.754353in}{0.388325in}}{\pgfqpoint{1.752158in}{0.383025in}}{\pgfqpoint{1.752158in}{0.377500in}}%
\pgfpathcurveto{\pgfqpoint{1.752158in}{0.371975in}}{\pgfqpoint{1.754353in}{0.366675in}}{\pgfqpoint{1.758260in}{0.362769in}}%
\pgfpathcurveto{\pgfqpoint{1.762167in}{0.358862in}}{\pgfqpoint{1.767467in}{0.356667in}}{\pgfqpoint{1.772992in}{0.356667in}}%
\pgfpathclose%
\pgfusepath{stroke,fill}%
\end{pgfscope}%
\begin{pgfscope}%
\pgfpathrectangle{\pgfqpoint{0.562500in}{0.275000in}}{\pgfqpoint{3.487500in}{1.925000in}}%
\pgfusepath{clip}%
\pgfsetbuttcap%
\pgfsetroundjoin%
\definecolor{currentfill}{rgb}{0.000000,0.000000,0.000000}%
\pgfsetfillcolor{currentfill}%
\pgfsetlinewidth{1.003750pt}%
\definecolor{currentstroke}{rgb}{0.000000,0.000000,0.000000}%
\pgfsetstrokecolor{currentstroke}%
\pgfsetdash{}{0pt}%
\pgfpathmoveto{\pgfqpoint{1.772992in}{0.356667in}}%
\pgfpathcurveto{\pgfqpoint{1.778517in}{0.356667in}}{\pgfqpoint{1.783816in}{0.358862in}}{\pgfqpoint{1.787723in}{0.362769in}}%
\pgfpathcurveto{\pgfqpoint{1.791630in}{0.366675in}}{\pgfqpoint{1.793825in}{0.371975in}}{\pgfqpoint{1.793825in}{0.377500in}}%
\pgfpathcurveto{\pgfqpoint{1.793825in}{0.383025in}}{\pgfqpoint{1.791630in}{0.388325in}}{\pgfqpoint{1.787723in}{0.392231in}}%
\pgfpathcurveto{\pgfqpoint{1.783816in}{0.396138in}}{\pgfqpoint{1.778517in}{0.398333in}}{\pgfqpoint{1.772992in}{0.398333in}}%
\pgfpathcurveto{\pgfqpoint{1.767467in}{0.398333in}}{\pgfqpoint{1.762167in}{0.396138in}}{\pgfqpoint{1.758260in}{0.392231in}}%
\pgfpathcurveto{\pgfqpoint{1.754353in}{0.388325in}}{\pgfqpoint{1.752158in}{0.383025in}}{\pgfqpoint{1.752158in}{0.377500in}}%
\pgfpathcurveto{\pgfqpoint{1.752158in}{0.371975in}}{\pgfqpoint{1.754353in}{0.366675in}}{\pgfqpoint{1.758260in}{0.362769in}}%
\pgfpathcurveto{\pgfqpoint{1.762167in}{0.358862in}}{\pgfqpoint{1.767467in}{0.356667in}}{\pgfqpoint{1.772992in}{0.356667in}}%
\pgfpathclose%
\pgfusepath{stroke,fill}%
\end{pgfscope}%
\begin{pgfscope}%
\pgfpathrectangle{\pgfqpoint{0.562500in}{0.275000in}}{\pgfqpoint{3.487500in}{1.925000in}}%
\pgfusepath{clip}%
\pgfsetbuttcap%
\pgfsetroundjoin%
\definecolor{currentfill}{rgb}{0.000000,0.000000,0.000000}%
\pgfsetfillcolor{currentfill}%
\pgfsetlinewidth{1.003750pt}%
\definecolor{currentstroke}{rgb}{0.000000,0.000000,0.000000}%
\pgfsetstrokecolor{currentstroke}%
\pgfsetdash{}{0pt}%
\pgfpathmoveto{\pgfqpoint{1.772992in}{0.356667in}}%
\pgfpathcurveto{\pgfqpoint{1.778517in}{0.356667in}}{\pgfqpoint{1.783816in}{0.358862in}}{\pgfqpoint{1.787723in}{0.362769in}}%
\pgfpathcurveto{\pgfqpoint{1.791630in}{0.366675in}}{\pgfqpoint{1.793825in}{0.371975in}}{\pgfqpoint{1.793825in}{0.377500in}}%
\pgfpathcurveto{\pgfqpoint{1.793825in}{0.383025in}}{\pgfqpoint{1.791630in}{0.388325in}}{\pgfqpoint{1.787723in}{0.392231in}}%
\pgfpathcurveto{\pgfqpoint{1.783816in}{0.396138in}}{\pgfqpoint{1.778517in}{0.398333in}}{\pgfqpoint{1.772992in}{0.398333in}}%
\pgfpathcurveto{\pgfqpoint{1.767467in}{0.398333in}}{\pgfqpoint{1.762167in}{0.396138in}}{\pgfqpoint{1.758260in}{0.392231in}}%
\pgfpathcurveto{\pgfqpoint{1.754353in}{0.388325in}}{\pgfqpoint{1.752158in}{0.383025in}}{\pgfqpoint{1.752158in}{0.377500in}}%
\pgfpathcurveto{\pgfqpoint{1.752158in}{0.371975in}}{\pgfqpoint{1.754353in}{0.366675in}}{\pgfqpoint{1.758260in}{0.362769in}}%
\pgfpathcurveto{\pgfqpoint{1.762167in}{0.358862in}}{\pgfqpoint{1.767467in}{0.356667in}}{\pgfqpoint{1.772992in}{0.356667in}}%
\pgfpathclose%
\pgfusepath{stroke,fill}%
\end{pgfscope}%
\begin{pgfscope}%
\pgfpathrectangle{\pgfqpoint{0.562500in}{0.275000in}}{\pgfqpoint{3.487500in}{1.925000in}}%
\pgfusepath{clip}%
\pgfsetbuttcap%
\pgfsetroundjoin%
\definecolor{currentfill}{rgb}{0.000000,0.000000,0.000000}%
\pgfsetfillcolor{currentfill}%
\pgfsetlinewidth{1.003750pt}%
\definecolor{currentstroke}{rgb}{0.000000,0.000000,0.000000}%
\pgfsetstrokecolor{currentstroke}%
\pgfsetdash{}{0pt}%
\pgfpathmoveto{\pgfqpoint{1.772992in}{0.356667in}}%
\pgfpathcurveto{\pgfqpoint{1.778517in}{0.356667in}}{\pgfqpoint{1.783816in}{0.358862in}}{\pgfqpoint{1.787723in}{0.362769in}}%
\pgfpathcurveto{\pgfqpoint{1.791630in}{0.366675in}}{\pgfqpoint{1.793825in}{0.371975in}}{\pgfqpoint{1.793825in}{0.377500in}}%
\pgfpathcurveto{\pgfqpoint{1.793825in}{0.383025in}}{\pgfqpoint{1.791630in}{0.388325in}}{\pgfqpoint{1.787723in}{0.392231in}}%
\pgfpathcurveto{\pgfqpoint{1.783816in}{0.396138in}}{\pgfqpoint{1.778517in}{0.398333in}}{\pgfqpoint{1.772992in}{0.398333in}}%
\pgfpathcurveto{\pgfqpoint{1.767467in}{0.398333in}}{\pgfqpoint{1.762167in}{0.396138in}}{\pgfqpoint{1.758260in}{0.392231in}}%
\pgfpathcurveto{\pgfqpoint{1.754353in}{0.388325in}}{\pgfqpoint{1.752158in}{0.383025in}}{\pgfqpoint{1.752158in}{0.377500in}}%
\pgfpathcurveto{\pgfqpoint{1.752158in}{0.371975in}}{\pgfqpoint{1.754353in}{0.366675in}}{\pgfqpoint{1.758260in}{0.362769in}}%
\pgfpathcurveto{\pgfqpoint{1.762167in}{0.358862in}}{\pgfqpoint{1.767467in}{0.356667in}}{\pgfqpoint{1.772992in}{0.356667in}}%
\pgfpathclose%
\pgfusepath{stroke,fill}%
\end{pgfscope}%
\begin{pgfscope}%
\pgfpathrectangle{\pgfqpoint{0.562500in}{0.275000in}}{\pgfqpoint{3.487500in}{1.925000in}}%
\pgfusepath{clip}%
\pgfsetbuttcap%
\pgfsetroundjoin%
\definecolor{currentfill}{rgb}{0.000000,0.000000,0.000000}%
\pgfsetfillcolor{currentfill}%
\pgfsetlinewidth{1.003750pt}%
\definecolor{currentstroke}{rgb}{0.000000,0.000000,0.000000}%
\pgfsetstrokecolor{currentstroke}%
\pgfsetdash{}{0pt}%
\pgfpathmoveto{\pgfqpoint{1.772992in}{0.356667in}}%
\pgfpathcurveto{\pgfqpoint{1.778517in}{0.356667in}}{\pgfqpoint{1.783816in}{0.358862in}}{\pgfqpoint{1.787723in}{0.362769in}}%
\pgfpathcurveto{\pgfqpoint{1.791630in}{0.366675in}}{\pgfqpoint{1.793825in}{0.371975in}}{\pgfqpoint{1.793825in}{0.377500in}}%
\pgfpathcurveto{\pgfqpoint{1.793825in}{0.383025in}}{\pgfqpoint{1.791630in}{0.388325in}}{\pgfqpoint{1.787723in}{0.392231in}}%
\pgfpathcurveto{\pgfqpoint{1.783816in}{0.396138in}}{\pgfqpoint{1.778517in}{0.398333in}}{\pgfqpoint{1.772992in}{0.398333in}}%
\pgfpathcurveto{\pgfqpoint{1.767467in}{0.398333in}}{\pgfqpoint{1.762167in}{0.396138in}}{\pgfqpoint{1.758260in}{0.392231in}}%
\pgfpathcurveto{\pgfqpoint{1.754353in}{0.388325in}}{\pgfqpoint{1.752158in}{0.383025in}}{\pgfqpoint{1.752158in}{0.377500in}}%
\pgfpathcurveto{\pgfqpoint{1.752158in}{0.371975in}}{\pgfqpoint{1.754353in}{0.366675in}}{\pgfqpoint{1.758260in}{0.362769in}}%
\pgfpathcurveto{\pgfqpoint{1.762167in}{0.358862in}}{\pgfqpoint{1.767467in}{0.356667in}}{\pgfqpoint{1.772992in}{0.356667in}}%
\pgfpathclose%
\pgfusepath{stroke,fill}%
\end{pgfscope}%
\begin{pgfscope}%
\pgfpathrectangle{\pgfqpoint{0.562500in}{0.275000in}}{\pgfqpoint{3.487500in}{1.925000in}}%
\pgfusepath{clip}%
\pgfsetbuttcap%
\pgfsetroundjoin%
\definecolor{currentfill}{rgb}{0.000000,0.000000,0.000000}%
\pgfsetfillcolor{currentfill}%
\pgfsetlinewidth{1.003750pt}%
\definecolor{currentstroke}{rgb}{0.000000,0.000000,0.000000}%
\pgfsetstrokecolor{currentstroke}%
\pgfsetdash{}{0pt}%
\pgfpathmoveto{\pgfqpoint{1.772992in}{0.356667in}}%
\pgfpathcurveto{\pgfqpoint{1.778517in}{0.356667in}}{\pgfqpoint{1.783816in}{0.358862in}}{\pgfqpoint{1.787723in}{0.362769in}}%
\pgfpathcurveto{\pgfqpoint{1.791630in}{0.366675in}}{\pgfqpoint{1.793825in}{0.371975in}}{\pgfqpoint{1.793825in}{0.377500in}}%
\pgfpathcurveto{\pgfqpoint{1.793825in}{0.383025in}}{\pgfqpoint{1.791630in}{0.388325in}}{\pgfqpoint{1.787723in}{0.392231in}}%
\pgfpathcurveto{\pgfqpoint{1.783816in}{0.396138in}}{\pgfqpoint{1.778517in}{0.398333in}}{\pgfqpoint{1.772992in}{0.398333in}}%
\pgfpathcurveto{\pgfqpoint{1.767467in}{0.398333in}}{\pgfqpoint{1.762167in}{0.396138in}}{\pgfqpoint{1.758260in}{0.392231in}}%
\pgfpathcurveto{\pgfqpoint{1.754353in}{0.388325in}}{\pgfqpoint{1.752158in}{0.383025in}}{\pgfqpoint{1.752158in}{0.377500in}}%
\pgfpathcurveto{\pgfqpoint{1.752158in}{0.371975in}}{\pgfqpoint{1.754353in}{0.366675in}}{\pgfqpoint{1.758260in}{0.362769in}}%
\pgfpathcurveto{\pgfqpoint{1.762167in}{0.358862in}}{\pgfqpoint{1.767467in}{0.356667in}}{\pgfqpoint{1.772992in}{0.356667in}}%
\pgfpathclose%
\pgfusepath{stroke,fill}%
\end{pgfscope}%
\begin{pgfscope}%
\pgfpathrectangle{\pgfqpoint{0.562500in}{0.275000in}}{\pgfqpoint{3.487500in}{1.925000in}}%
\pgfusepath{clip}%
\pgfsetbuttcap%
\pgfsetroundjoin%
\definecolor{currentfill}{rgb}{0.000000,0.000000,0.000000}%
\pgfsetfillcolor{currentfill}%
\pgfsetlinewidth{1.003750pt}%
\definecolor{currentstroke}{rgb}{0.000000,0.000000,0.000000}%
\pgfsetstrokecolor{currentstroke}%
\pgfsetdash{}{0pt}%
\pgfpathmoveto{\pgfqpoint{1.772992in}{0.356667in}}%
\pgfpathcurveto{\pgfqpoint{1.778517in}{0.356667in}}{\pgfqpoint{1.783816in}{0.358862in}}{\pgfqpoint{1.787723in}{0.362769in}}%
\pgfpathcurveto{\pgfqpoint{1.791630in}{0.366675in}}{\pgfqpoint{1.793825in}{0.371975in}}{\pgfqpoint{1.793825in}{0.377500in}}%
\pgfpathcurveto{\pgfqpoint{1.793825in}{0.383025in}}{\pgfqpoint{1.791630in}{0.388325in}}{\pgfqpoint{1.787723in}{0.392231in}}%
\pgfpathcurveto{\pgfqpoint{1.783816in}{0.396138in}}{\pgfqpoint{1.778517in}{0.398333in}}{\pgfqpoint{1.772992in}{0.398333in}}%
\pgfpathcurveto{\pgfqpoint{1.767467in}{0.398333in}}{\pgfqpoint{1.762167in}{0.396138in}}{\pgfqpoint{1.758260in}{0.392231in}}%
\pgfpathcurveto{\pgfqpoint{1.754353in}{0.388325in}}{\pgfqpoint{1.752158in}{0.383025in}}{\pgfqpoint{1.752158in}{0.377500in}}%
\pgfpathcurveto{\pgfqpoint{1.752158in}{0.371975in}}{\pgfqpoint{1.754353in}{0.366675in}}{\pgfqpoint{1.758260in}{0.362769in}}%
\pgfpathcurveto{\pgfqpoint{1.762167in}{0.358862in}}{\pgfqpoint{1.767467in}{0.356667in}}{\pgfqpoint{1.772992in}{0.356667in}}%
\pgfpathclose%
\pgfusepath{stroke,fill}%
\end{pgfscope}%
\begin{pgfscope}%
\pgfpathrectangle{\pgfqpoint{0.562500in}{0.275000in}}{\pgfqpoint{3.487500in}{1.925000in}}%
\pgfusepath{clip}%
\pgfsetbuttcap%
\pgfsetroundjoin%
\definecolor{currentfill}{rgb}{0.000000,0.000000,0.000000}%
\pgfsetfillcolor{currentfill}%
\pgfsetlinewidth{1.003750pt}%
\definecolor{currentstroke}{rgb}{0.000000,0.000000,0.000000}%
\pgfsetstrokecolor{currentstroke}%
\pgfsetdash{}{0pt}%
\pgfpathmoveto{\pgfqpoint{1.772992in}{0.356667in}}%
\pgfpathcurveto{\pgfqpoint{1.778517in}{0.356667in}}{\pgfqpoint{1.783816in}{0.358862in}}{\pgfqpoint{1.787723in}{0.362769in}}%
\pgfpathcurveto{\pgfqpoint{1.791630in}{0.366675in}}{\pgfqpoint{1.793825in}{0.371975in}}{\pgfqpoint{1.793825in}{0.377500in}}%
\pgfpathcurveto{\pgfqpoint{1.793825in}{0.383025in}}{\pgfqpoint{1.791630in}{0.388325in}}{\pgfqpoint{1.787723in}{0.392231in}}%
\pgfpathcurveto{\pgfqpoint{1.783816in}{0.396138in}}{\pgfqpoint{1.778517in}{0.398333in}}{\pgfqpoint{1.772992in}{0.398333in}}%
\pgfpathcurveto{\pgfqpoint{1.767467in}{0.398333in}}{\pgfqpoint{1.762167in}{0.396138in}}{\pgfqpoint{1.758260in}{0.392231in}}%
\pgfpathcurveto{\pgfqpoint{1.754353in}{0.388325in}}{\pgfqpoint{1.752158in}{0.383025in}}{\pgfqpoint{1.752158in}{0.377500in}}%
\pgfpathcurveto{\pgfqpoint{1.752158in}{0.371975in}}{\pgfqpoint{1.754353in}{0.366675in}}{\pgfqpoint{1.758260in}{0.362769in}}%
\pgfpathcurveto{\pgfqpoint{1.762167in}{0.358862in}}{\pgfqpoint{1.767467in}{0.356667in}}{\pgfqpoint{1.772992in}{0.356667in}}%
\pgfpathclose%
\pgfusepath{stroke,fill}%
\end{pgfscope}%
\begin{pgfscope}%
\pgfpathrectangle{\pgfqpoint{0.562500in}{0.275000in}}{\pgfqpoint{3.487500in}{1.925000in}}%
\pgfusepath{clip}%
\pgfsetbuttcap%
\pgfsetroundjoin%
\definecolor{currentfill}{rgb}{0.000000,0.000000,0.000000}%
\pgfsetfillcolor{currentfill}%
\pgfsetlinewidth{1.003750pt}%
\definecolor{currentstroke}{rgb}{0.000000,0.000000,0.000000}%
\pgfsetstrokecolor{currentstroke}%
\pgfsetdash{}{0pt}%
\pgfpathmoveto{\pgfqpoint{1.772992in}{0.356667in}}%
\pgfpathcurveto{\pgfqpoint{1.778517in}{0.356667in}}{\pgfqpoint{1.783816in}{0.358862in}}{\pgfqpoint{1.787723in}{0.362769in}}%
\pgfpathcurveto{\pgfqpoint{1.791630in}{0.366675in}}{\pgfqpoint{1.793825in}{0.371975in}}{\pgfqpoint{1.793825in}{0.377500in}}%
\pgfpathcurveto{\pgfqpoint{1.793825in}{0.383025in}}{\pgfqpoint{1.791630in}{0.388325in}}{\pgfqpoint{1.787723in}{0.392231in}}%
\pgfpathcurveto{\pgfqpoint{1.783816in}{0.396138in}}{\pgfqpoint{1.778517in}{0.398333in}}{\pgfqpoint{1.772992in}{0.398333in}}%
\pgfpathcurveto{\pgfqpoint{1.767467in}{0.398333in}}{\pgfqpoint{1.762167in}{0.396138in}}{\pgfqpoint{1.758260in}{0.392231in}}%
\pgfpathcurveto{\pgfqpoint{1.754353in}{0.388325in}}{\pgfqpoint{1.752158in}{0.383025in}}{\pgfqpoint{1.752158in}{0.377500in}}%
\pgfpathcurveto{\pgfqpoint{1.752158in}{0.371975in}}{\pgfqpoint{1.754353in}{0.366675in}}{\pgfqpoint{1.758260in}{0.362769in}}%
\pgfpathcurveto{\pgfqpoint{1.762167in}{0.358862in}}{\pgfqpoint{1.767467in}{0.356667in}}{\pgfqpoint{1.772992in}{0.356667in}}%
\pgfpathclose%
\pgfusepath{stroke,fill}%
\end{pgfscope}%
\begin{pgfscope}%
\pgfpathrectangle{\pgfqpoint{0.562500in}{0.275000in}}{\pgfqpoint{3.487500in}{1.925000in}}%
\pgfusepath{clip}%
\pgfsetbuttcap%
\pgfsetroundjoin%
\definecolor{currentfill}{rgb}{0.000000,0.000000,0.000000}%
\pgfsetfillcolor{currentfill}%
\pgfsetlinewidth{1.003750pt}%
\definecolor{currentstroke}{rgb}{0.000000,0.000000,0.000000}%
\pgfsetstrokecolor{currentstroke}%
\pgfsetdash{}{0pt}%
\pgfpathmoveto{\pgfqpoint{1.772992in}{0.356667in}}%
\pgfpathcurveto{\pgfqpoint{1.778517in}{0.356667in}}{\pgfqpoint{1.783816in}{0.358862in}}{\pgfqpoint{1.787723in}{0.362769in}}%
\pgfpathcurveto{\pgfqpoint{1.791630in}{0.366675in}}{\pgfqpoint{1.793825in}{0.371975in}}{\pgfqpoint{1.793825in}{0.377500in}}%
\pgfpathcurveto{\pgfqpoint{1.793825in}{0.383025in}}{\pgfqpoint{1.791630in}{0.388325in}}{\pgfqpoint{1.787723in}{0.392231in}}%
\pgfpathcurveto{\pgfqpoint{1.783816in}{0.396138in}}{\pgfqpoint{1.778517in}{0.398333in}}{\pgfqpoint{1.772992in}{0.398333in}}%
\pgfpathcurveto{\pgfqpoint{1.767467in}{0.398333in}}{\pgfqpoint{1.762167in}{0.396138in}}{\pgfqpoint{1.758260in}{0.392231in}}%
\pgfpathcurveto{\pgfqpoint{1.754353in}{0.388325in}}{\pgfqpoint{1.752158in}{0.383025in}}{\pgfqpoint{1.752158in}{0.377500in}}%
\pgfpathcurveto{\pgfqpoint{1.752158in}{0.371975in}}{\pgfqpoint{1.754353in}{0.366675in}}{\pgfqpoint{1.758260in}{0.362769in}}%
\pgfpathcurveto{\pgfqpoint{1.762167in}{0.358862in}}{\pgfqpoint{1.767467in}{0.356667in}}{\pgfqpoint{1.772992in}{0.356667in}}%
\pgfpathclose%
\pgfusepath{stroke,fill}%
\end{pgfscope}%
\begin{pgfscope}%
\pgfpathrectangle{\pgfqpoint{0.562500in}{0.275000in}}{\pgfqpoint{3.487500in}{1.925000in}}%
\pgfusepath{clip}%
\pgfsetbuttcap%
\pgfsetroundjoin%
\definecolor{currentfill}{rgb}{0.000000,0.000000,0.000000}%
\pgfsetfillcolor{currentfill}%
\pgfsetlinewidth{1.003750pt}%
\definecolor{currentstroke}{rgb}{0.000000,0.000000,0.000000}%
\pgfsetstrokecolor{currentstroke}%
\pgfsetdash{}{0pt}%
\pgfpathmoveto{\pgfqpoint{1.772992in}{0.356667in}}%
\pgfpathcurveto{\pgfqpoint{1.778517in}{0.356667in}}{\pgfqpoint{1.783816in}{0.358862in}}{\pgfqpoint{1.787723in}{0.362769in}}%
\pgfpathcurveto{\pgfqpoint{1.791630in}{0.366675in}}{\pgfqpoint{1.793825in}{0.371975in}}{\pgfqpoint{1.793825in}{0.377500in}}%
\pgfpathcurveto{\pgfqpoint{1.793825in}{0.383025in}}{\pgfqpoint{1.791630in}{0.388325in}}{\pgfqpoint{1.787723in}{0.392231in}}%
\pgfpathcurveto{\pgfqpoint{1.783816in}{0.396138in}}{\pgfqpoint{1.778517in}{0.398333in}}{\pgfqpoint{1.772992in}{0.398333in}}%
\pgfpathcurveto{\pgfqpoint{1.767467in}{0.398333in}}{\pgfqpoint{1.762167in}{0.396138in}}{\pgfqpoint{1.758260in}{0.392231in}}%
\pgfpathcurveto{\pgfqpoint{1.754353in}{0.388325in}}{\pgfqpoint{1.752158in}{0.383025in}}{\pgfqpoint{1.752158in}{0.377500in}}%
\pgfpathcurveto{\pgfqpoint{1.752158in}{0.371975in}}{\pgfqpoint{1.754353in}{0.366675in}}{\pgfqpoint{1.758260in}{0.362769in}}%
\pgfpathcurveto{\pgfqpoint{1.762167in}{0.358862in}}{\pgfqpoint{1.767467in}{0.356667in}}{\pgfqpoint{1.772992in}{0.356667in}}%
\pgfpathclose%
\pgfusepath{stroke,fill}%
\end{pgfscope}%
\begin{pgfscope}%
\pgfpathrectangle{\pgfqpoint{0.562500in}{0.275000in}}{\pgfqpoint{3.487500in}{1.925000in}}%
\pgfusepath{clip}%
\pgfsetbuttcap%
\pgfsetroundjoin%
\definecolor{currentfill}{rgb}{0.000000,0.000000,0.000000}%
\pgfsetfillcolor{currentfill}%
\pgfsetlinewidth{1.003750pt}%
\definecolor{currentstroke}{rgb}{0.000000,0.000000,0.000000}%
\pgfsetstrokecolor{currentstroke}%
\pgfsetdash{}{0pt}%
\pgfpathmoveto{\pgfqpoint{2.824734in}{0.356667in}}%
\pgfpathcurveto{\pgfqpoint{2.830260in}{0.356667in}}{\pgfqpoint{2.835559in}{0.358862in}}{\pgfqpoint{2.839466in}{0.362769in}}%
\pgfpathcurveto{\pgfqpoint{2.843373in}{0.366675in}}{\pgfqpoint{2.845568in}{0.371975in}}{\pgfqpoint{2.845568in}{0.377500in}}%
\pgfpathcurveto{\pgfqpoint{2.845568in}{0.383025in}}{\pgfqpoint{2.843373in}{0.388325in}}{\pgfqpoint{2.839466in}{0.392231in}}%
\pgfpathcurveto{\pgfqpoint{2.835559in}{0.396138in}}{\pgfqpoint{2.830260in}{0.398333in}}{\pgfqpoint{2.824734in}{0.398333in}}%
\pgfpathcurveto{\pgfqpoint{2.819209in}{0.398333in}}{\pgfqpoint{2.813910in}{0.396138in}}{\pgfqpoint{2.810003in}{0.392231in}}%
\pgfpathcurveto{\pgfqpoint{2.806096in}{0.388325in}}{\pgfqpoint{2.803901in}{0.383025in}}{\pgfqpoint{2.803901in}{0.377500in}}%
\pgfpathcurveto{\pgfqpoint{2.803901in}{0.371975in}}{\pgfqpoint{2.806096in}{0.366675in}}{\pgfqpoint{2.810003in}{0.362769in}}%
\pgfpathcurveto{\pgfqpoint{2.813910in}{0.358862in}}{\pgfqpoint{2.819209in}{0.356667in}}{\pgfqpoint{2.824734in}{0.356667in}}%
\pgfpathclose%
\pgfusepath{stroke,fill}%
\end{pgfscope}%
\begin{pgfscope}%
\pgfpathrectangle{\pgfqpoint{0.562500in}{0.275000in}}{\pgfqpoint{3.487500in}{1.925000in}}%
\pgfusepath{clip}%
\pgfsetbuttcap%
\pgfsetroundjoin%
\definecolor{currentfill}{rgb}{0.000000,0.000000,0.000000}%
\pgfsetfillcolor{currentfill}%
\pgfsetlinewidth{1.003750pt}%
\definecolor{currentstroke}{rgb}{0.000000,0.000000,0.000000}%
\pgfsetstrokecolor{currentstroke}%
\pgfsetdash{}{0pt}%
\pgfpathmoveto{\pgfqpoint{2.824734in}{0.356667in}}%
\pgfpathcurveto{\pgfqpoint{2.830260in}{0.356667in}}{\pgfqpoint{2.835559in}{0.358862in}}{\pgfqpoint{2.839466in}{0.362769in}}%
\pgfpathcurveto{\pgfqpoint{2.843373in}{0.366675in}}{\pgfqpoint{2.845568in}{0.371975in}}{\pgfqpoint{2.845568in}{0.377500in}}%
\pgfpathcurveto{\pgfqpoint{2.845568in}{0.383025in}}{\pgfqpoint{2.843373in}{0.388325in}}{\pgfqpoint{2.839466in}{0.392231in}}%
\pgfpathcurveto{\pgfqpoint{2.835559in}{0.396138in}}{\pgfqpoint{2.830260in}{0.398333in}}{\pgfqpoint{2.824734in}{0.398333in}}%
\pgfpathcurveto{\pgfqpoint{2.819209in}{0.398333in}}{\pgfqpoint{2.813910in}{0.396138in}}{\pgfqpoint{2.810003in}{0.392231in}}%
\pgfpathcurveto{\pgfqpoint{2.806096in}{0.388325in}}{\pgfqpoint{2.803901in}{0.383025in}}{\pgfqpoint{2.803901in}{0.377500in}}%
\pgfpathcurveto{\pgfqpoint{2.803901in}{0.371975in}}{\pgfqpoint{2.806096in}{0.366675in}}{\pgfqpoint{2.810003in}{0.362769in}}%
\pgfpathcurveto{\pgfqpoint{2.813910in}{0.358862in}}{\pgfqpoint{2.819209in}{0.356667in}}{\pgfqpoint{2.824734in}{0.356667in}}%
\pgfpathclose%
\pgfusepath{stroke,fill}%
\end{pgfscope}%
\begin{pgfscope}%
\pgfpathrectangle{\pgfqpoint{0.562500in}{0.275000in}}{\pgfqpoint{3.487500in}{1.925000in}}%
\pgfusepath{clip}%
\pgfsetbuttcap%
\pgfsetroundjoin%
\definecolor{currentfill}{rgb}{0.000000,0.000000,0.000000}%
\pgfsetfillcolor{currentfill}%
\pgfsetlinewidth{1.003750pt}%
\definecolor{currentstroke}{rgb}{0.000000,0.000000,0.000000}%
\pgfsetstrokecolor{currentstroke}%
\pgfsetdash{}{0pt}%
\pgfpathmoveto{\pgfqpoint{2.824734in}{0.356667in}}%
\pgfpathcurveto{\pgfqpoint{2.830260in}{0.356667in}}{\pgfqpoint{2.835559in}{0.358862in}}{\pgfqpoint{2.839466in}{0.362769in}}%
\pgfpathcurveto{\pgfqpoint{2.843373in}{0.366675in}}{\pgfqpoint{2.845568in}{0.371975in}}{\pgfqpoint{2.845568in}{0.377500in}}%
\pgfpathcurveto{\pgfqpoint{2.845568in}{0.383025in}}{\pgfqpoint{2.843373in}{0.388325in}}{\pgfqpoint{2.839466in}{0.392231in}}%
\pgfpathcurveto{\pgfqpoint{2.835559in}{0.396138in}}{\pgfqpoint{2.830260in}{0.398333in}}{\pgfqpoint{2.824734in}{0.398333in}}%
\pgfpathcurveto{\pgfqpoint{2.819209in}{0.398333in}}{\pgfqpoint{2.813910in}{0.396138in}}{\pgfqpoint{2.810003in}{0.392231in}}%
\pgfpathcurveto{\pgfqpoint{2.806096in}{0.388325in}}{\pgfqpoint{2.803901in}{0.383025in}}{\pgfqpoint{2.803901in}{0.377500in}}%
\pgfpathcurveto{\pgfqpoint{2.803901in}{0.371975in}}{\pgfqpoint{2.806096in}{0.366675in}}{\pgfqpoint{2.810003in}{0.362769in}}%
\pgfpathcurveto{\pgfqpoint{2.813910in}{0.358862in}}{\pgfqpoint{2.819209in}{0.356667in}}{\pgfqpoint{2.824734in}{0.356667in}}%
\pgfpathclose%
\pgfusepath{stroke,fill}%
\end{pgfscope}%
\begin{pgfscope}%
\pgfpathrectangle{\pgfqpoint{0.562500in}{0.275000in}}{\pgfqpoint{3.487500in}{1.925000in}}%
\pgfusepath{clip}%
\pgfsetbuttcap%
\pgfsetroundjoin%
\definecolor{currentfill}{rgb}{0.000000,0.000000,0.000000}%
\pgfsetfillcolor{currentfill}%
\pgfsetlinewidth{1.003750pt}%
\definecolor{currentstroke}{rgb}{0.000000,0.000000,0.000000}%
\pgfsetstrokecolor{currentstroke}%
\pgfsetdash{}{0pt}%
\pgfpathmoveto{\pgfqpoint{2.824734in}{0.356667in}}%
\pgfpathcurveto{\pgfqpoint{2.830260in}{0.356667in}}{\pgfqpoint{2.835559in}{0.358862in}}{\pgfqpoint{2.839466in}{0.362769in}}%
\pgfpathcurveto{\pgfqpoint{2.843373in}{0.366675in}}{\pgfqpoint{2.845568in}{0.371975in}}{\pgfqpoint{2.845568in}{0.377500in}}%
\pgfpathcurveto{\pgfqpoint{2.845568in}{0.383025in}}{\pgfqpoint{2.843373in}{0.388325in}}{\pgfqpoint{2.839466in}{0.392231in}}%
\pgfpathcurveto{\pgfqpoint{2.835559in}{0.396138in}}{\pgfqpoint{2.830260in}{0.398333in}}{\pgfqpoint{2.824734in}{0.398333in}}%
\pgfpathcurveto{\pgfqpoint{2.819209in}{0.398333in}}{\pgfqpoint{2.813910in}{0.396138in}}{\pgfqpoint{2.810003in}{0.392231in}}%
\pgfpathcurveto{\pgfqpoint{2.806096in}{0.388325in}}{\pgfqpoint{2.803901in}{0.383025in}}{\pgfqpoint{2.803901in}{0.377500in}}%
\pgfpathcurveto{\pgfqpoint{2.803901in}{0.371975in}}{\pgfqpoint{2.806096in}{0.366675in}}{\pgfqpoint{2.810003in}{0.362769in}}%
\pgfpathcurveto{\pgfqpoint{2.813910in}{0.358862in}}{\pgfqpoint{2.819209in}{0.356667in}}{\pgfqpoint{2.824734in}{0.356667in}}%
\pgfpathclose%
\pgfusepath{stroke,fill}%
\end{pgfscope}%
\begin{pgfscope}%
\pgfpathrectangle{\pgfqpoint{0.562500in}{0.275000in}}{\pgfqpoint{3.487500in}{1.925000in}}%
\pgfusepath{clip}%
\pgfsetbuttcap%
\pgfsetroundjoin%
\definecolor{currentfill}{rgb}{0.000000,0.000000,0.000000}%
\pgfsetfillcolor{currentfill}%
\pgfsetlinewidth{1.003750pt}%
\definecolor{currentstroke}{rgb}{0.000000,0.000000,0.000000}%
\pgfsetstrokecolor{currentstroke}%
\pgfsetdash{}{0pt}%
\pgfpathmoveto{\pgfqpoint{2.824734in}{0.356667in}}%
\pgfpathcurveto{\pgfqpoint{2.830260in}{0.356667in}}{\pgfqpoint{2.835559in}{0.358862in}}{\pgfqpoint{2.839466in}{0.362769in}}%
\pgfpathcurveto{\pgfqpoint{2.843373in}{0.366675in}}{\pgfqpoint{2.845568in}{0.371975in}}{\pgfqpoint{2.845568in}{0.377500in}}%
\pgfpathcurveto{\pgfqpoint{2.845568in}{0.383025in}}{\pgfqpoint{2.843373in}{0.388325in}}{\pgfqpoint{2.839466in}{0.392231in}}%
\pgfpathcurveto{\pgfqpoint{2.835559in}{0.396138in}}{\pgfqpoint{2.830260in}{0.398333in}}{\pgfqpoint{2.824734in}{0.398333in}}%
\pgfpathcurveto{\pgfqpoint{2.819209in}{0.398333in}}{\pgfqpoint{2.813910in}{0.396138in}}{\pgfqpoint{2.810003in}{0.392231in}}%
\pgfpathcurveto{\pgfqpoint{2.806096in}{0.388325in}}{\pgfqpoint{2.803901in}{0.383025in}}{\pgfqpoint{2.803901in}{0.377500in}}%
\pgfpathcurveto{\pgfqpoint{2.803901in}{0.371975in}}{\pgfqpoint{2.806096in}{0.366675in}}{\pgfqpoint{2.810003in}{0.362769in}}%
\pgfpathcurveto{\pgfqpoint{2.813910in}{0.358862in}}{\pgfqpoint{2.819209in}{0.356667in}}{\pgfqpoint{2.824734in}{0.356667in}}%
\pgfpathclose%
\pgfusepath{stroke,fill}%
\end{pgfscope}%
\begin{pgfscope}%
\pgfpathrectangle{\pgfqpoint{0.562500in}{0.275000in}}{\pgfqpoint{3.487500in}{1.925000in}}%
\pgfusepath{clip}%
\pgfsetbuttcap%
\pgfsetroundjoin%
\definecolor{currentfill}{rgb}{0.000000,0.000000,0.000000}%
\pgfsetfillcolor{currentfill}%
\pgfsetlinewidth{1.003750pt}%
\definecolor{currentstroke}{rgb}{0.000000,0.000000,0.000000}%
\pgfsetstrokecolor{currentstroke}%
\pgfsetdash{}{0pt}%
\pgfpathmoveto{\pgfqpoint{2.824734in}{0.356667in}}%
\pgfpathcurveto{\pgfqpoint{2.830260in}{0.356667in}}{\pgfqpoint{2.835559in}{0.358862in}}{\pgfqpoint{2.839466in}{0.362769in}}%
\pgfpathcurveto{\pgfqpoint{2.843373in}{0.366675in}}{\pgfqpoint{2.845568in}{0.371975in}}{\pgfqpoint{2.845568in}{0.377500in}}%
\pgfpathcurveto{\pgfqpoint{2.845568in}{0.383025in}}{\pgfqpoint{2.843373in}{0.388325in}}{\pgfqpoint{2.839466in}{0.392231in}}%
\pgfpathcurveto{\pgfqpoint{2.835559in}{0.396138in}}{\pgfqpoint{2.830260in}{0.398333in}}{\pgfqpoint{2.824734in}{0.398333in}}%
\pgfpathcurveto{\pgfqpoint{2.819209in}{0.398333in}}{\pgfqpoint{2.813910in}{0.396138in}}{\pgfqpoint{2.810003in}{0.392231in}}%
\pgfpathcurveto{\pgfqpoint{2.806096in}{0.388325in}}{\pgfqpoint{2.803901in}{0.383025in}}{\pgfqpoint{2.803901in}{0.377500in}}%
\pgfpathcurveto{\pgfqpoint{2.803901in}{0.371975in}}{\pgfqpoint{2.806096in}{0.366675in}}{\pgfqpoint{2.810003in}{0.362769in}}%
\pgfpathcurveto{\pgfqpoint{2.813910in}{0.358862in}}{\pgfqpoint{2.819209in}{0.356667in}}{\pgfqpoint{2.824734in}{0.356667in}}%
\pgfpathclose%
\pgfusepath{stroke,fill}%
\end{pgfscope}%
\begin{pgfscope}%
\pgfpathrectangle{\pgfqpoint{0.562500in}{0.275000in}}{\pgfqpoint{3.487500in}{1.925000in}}%
\pgfusepath{clip}%
\pgfsetbuttcap%
\pgfsetroundjoin%
\definecolor{currentfill}{rgb}{0.000000,0.000000,0.000000}%
\pgfsetfillcolor{currentfill}%
\pgfsetlinewidth{1.003750pt}%
\definecolor{currentstroke}{rgb}{0.000000,0.000000,0.000000}%
\pgfsetstrokecolor{currentstroke}%
\pgfsetdash{}{0pt}%
\pgfpathmoveto{\pgfqpoint{2.824734in}{0.356667in}}%
\pgfpathcurveto{\pgfqpoint{2.830260in}{0.356667in}}{\pgfqpoint{2.835559in}{0.358862in}}{\pgfqpoint{2.839466in}{0.362769in}}%
\pgfpathcurveto{\pgfqpoint{2.843373in}{0.366675in}}{\pgfqpoint{2.845568in}{0.371975in}}{\pgfqpoint{2.845568in}{0.377500in}}%
\pgfpathcurveto{\pgfqpoint{2.845568in}{0.383025in}}{\pgfqpoint{2.843373in}{0.388325in}}{\pgfqpoint{2.839466in}{0.392231in}}%
\pgfpathcurveto{\pgfqpoint{2.835559in}{0.396138in}}{\pgfqpoint{2.830260in}{0.398333in}}{\pgfqpoint{2.824734in}{0.398333in}}%
\pgfpathcurveto{\pgfqpoint{2.819209in}{0.398333in}}{\pgfqpoint{2.813910in}{0.396138in}}{\pgfqpoint{2.810003in}{0.392231in}}%
\pgfpathcurveto{\pgfqpoint{2.806096in}{0.388325in}}{\pgfqpoint{2.803901in}{0.383025in}}{\pgfqpoint{2.803901in}{0.377500in}}%
\pgfpathcurveto{\pgfqpoint{2.803901in}{0.371975in}}{\pgfqpoint{2.806096in}{0.366675in}}{\pgfqpoint{2.810003in}{0.362769in}}%
\pgfpathcurveto{\pgfqpoint{2.813910in}{0.358862in}}{\pgfqpoint{2.819209in}{0.356667in}}{\pgfqpoint{2.824734in}{0.356667in}}%
\pgfpathclose%
\pgfusepath{stroke,fill}%
\end{pgfscope}%
\begin{pgfscope}%
\pgfpathrectangle{\pgfqpoint{0.562500in}{0.275000in}}{\pgfqpoint{3.487500in}{1.925000in}}%
\pgfusepath{clip}%
\pgfsetbuttcap%
\pgfsetroundjoin%
\definecolor{currentfill}{rgb}{0.000000,0.000000,0.000000}%
\pgfsetfillcolor{currentfill}%
\pgfsetlinewidth{1.003750pt}%
\definecolor{currentstroke}{rgb}{0.000000,0.000000,0.000000}%
\pgfsetstrokecolor{currentstroke}%
\pgfsetdash{}{0pt}%
\pgfpathmoveto{\pgfqpoint{2.824734in}{0.356667in}}%
\pgfpathcurveto{\pgfqpoint{2.830260in}{0.356667in}}{\pgfqpoint{2.835559in}{0.358862in}}{\pgfqpoint{2.839466in}{0.362769in}}%
\pgfpathcurveto{\pgfqpoint{2.843373in}{0.366675in}}{\pgfqpoint{2.845568in}{0.371975in}}{\pgfqpoint{2.845568in}{0.377500in}}%
\pgfpathcurveto{\pgfqpoint{2.845568in}{0.383025in}}{\pgfqpoint{2.843373in}{0.388325in}}{\pgfqpoint{2.839466in}{0.392231in}}%
\pgfpathcurveto{\pgfqpoint{2.835559in}{0.396138in}}{\pgfqpoint{2.830260in}{0.398333in}}{\pgfqpoint{2.824734in}{0.398333in}}%
\pgfpathcurveto{\pgfqpoint{2.819209in}{0.398333in}}{\pgfqpoint{2.813910in}{0.396138in}}{\pgfqpoint{2.810003in}{0.392231in}}%
\pgfpathcurveto{\pgfqpoint{2.806096in}{0.388325in}}{\pgfqpoint{2.803901in}{0.383025in}}{\pgfqpoint{2.803901in}{0.377500in}}%
\pgfpathcurveto{\pgfqpoint{2.803901in}{0.371975in}}{\pgfqpoint{2.806096in}{0.366675in}}{\pgfqpoint{2.810003in}{0.362769in}}%
\pgfpathcurveto{\pgfqpoint{2.813910in}{0.358862in}}{\pgfqpoint{2.819209in}{0.356667in}}{\pgfqpoint{2.824734in}{0.356667in}}%
\pgfpathclose%
\pgfusepath{stroke,fill}%
\end{pgfscope}%
\begin{pgfscope}%
\pgfpathrectangle{\pgfqpoint{0.562500in}{0.275000in}}{\pgfqpoint{3.487500in}{1.925000in}}%
\pgfusepath{clip}%
\pgfsetbuttcap%
\pgfsetroundjoin%
\definecolor{currentfill}{rgb}{0.000000,0.000000,0.000000}%
\pgfsetfillcolor{currentfill}%
\pgfsetlinewidth{1.003750pt}%
\definecolor{currentstroke}{rgb}{0.000000,0.000000,0.000000}%
\pgfsetstrokecolor{currentstroke}%
\pgfsetdash{}{0pt}%
\pgfpathmoveto{\pgfqpoint{2.824734in}{0.356667in}}%
\pgfpathcurveto{\pgfqpoint{2.830260in}{0.356667in}}{\pgfqpoint{2.835559in}{0.358862in}}{\pgfqpoint{2.839466in}{0.362769in}}%
\pgfpathcurveto{\pgfqpoint{2.843373in}{0.366675in}}{\pgfqpoint{2.845568in}{0.371975in}}{\pgfqpoint{2.845568in}{0.377500in}}%
\pgfpathcurveto{\pgfqpoint{2.845568in}{0.383025in}}{\pgfqpoint{2.843373in}{0.388325in}}{\pgfqpoint{2.839466in}{0.392231in}}%
\pgfpathcurveto{\pgfqpoint{2.835559in}{0.396138in}}{\pgfqpoint{2.830260in}{0.398333in}}{\pgfqpoint{2.824734in}{0.398333in}}%
\pgfpathcurveto{\pgfqpoint{2.819209in}{0.398333in}}{\pgfqpoint{2.813910in}{0.396138in}}{\pgfqpoint{2.810003in}{0.392231in}}%
\pgfpathcurveto{\pgfqpoint{2.806096in}{0.388325in}}{\pgfqpoint{2.803901in}{0.383025in}}{\pgfqpoint{2.803901in}{0.377500in}}%
\pgfpathcurveto{\pgfqpoint{2.803901in}{0.371975in}}{\pgfqpoint{2.806096in}{0.366675in}}{\pgfqpoint{2.810003in}{0.362769in}}%
\pgfpathcurveto{\pgfqpoint{2.813910in}{0.358862in}}{\pgfqpoint{2.819209in}{0.356667in}}{\pgfqpoint{2.824734in}{0.356667in}}%
\pgfpathclose%
\pgfusepath{stroke,fill}%
\end{pgfscope}%
\begin{pgfscope}%
\pgfpathrectangle{\pgfqpoint{0.562500in}{0.275000in}}{\pgfqpoint{3.487500in}{1.925000in}}%
\pgfusepath{clip}%
\pgfsetbuttcap%
\pgfsetroundjoin%
\definecolor{currentfill}{rgb}{0.000000,0.000000,0.000000}%
\pgfsetfillcolor{currentfill}%
\pgfsetlinewidth{1.003750pt}%
\definecolor{currentstroke}{rgb}{0.000000,0.000000,0.000000}%
\pgfsetstrokecolor{currentstroke}%
\pgfsetdash{}{0pt}%
\pgfpathmoveto{\pgfqpoint{2.824734in}{0.356667in}}%
\pgfpathcurveto{\pgfqpoint{2.830260in}{0.356667in}}{\pgfqpoint{2.835559in}{0.358862in}}{\pgfqpoint{2.839466in}{0.362769in}}%
\pgfpathcurveto{\pgfqpoint{2.843373in}{0.366675in}}{\pgfqpoint{2.845568in}{0.371975in}}{\pgfqpoint{2.845568in}{0.377500in}}%
\pgfpathcurveto{\pgfqpoint{2.845568in}{0.383025in}}{\pgfqpoint{2.843373in}{0.388325in}}{\pgfqpoint{2.839466in}{0.392231in}}%
\pgfpathcurveto{\pgfqpoint{2.835559in}{0.396138in}}{\pgfqpoint{2.830260in}{0.398333in}}{\pgfqpoint{2.824734in}{0.398333in}}%
\pgfpathcurveto{\pgfqpoint{2.819209in}{0.398333in}}{\pgfqpoint{2.813910in}{0.396138in}}{\pgfqpoint{2.810003in}{0.392231in}}%
\pgfpathcurveto{\pgfqpoint{2.806096in}{0.388325in}}{\pgfqpoint{2.803901in}{0.383025in}}{\pgfqpoint{2.803901in}{0.377500in}}%
\pgfpathcurveto{\pgfqpoint{2.803901in}{0.371975in}}{\pgfqpoint{2.806096in}{0.366675in}}{\pgfqpoint{2.810003in}{0.362769in}}%
\pgfpathcurveto{\pgfqpoint{2.813910in}{0.358862in}}{\pgfqpoint{2.819209in}{0.356667in}}{\pgfqpoint{2.824734in}{0.356667in}}%
\pgfpathclose%
\pgfusepath{stroke,fill}%
\end{pgfscope}%
\begin{pgfscope}%
\pgfpathrectangle{\pgfqpoint{0.562500in}{0.275000in}}{\pgfqpoint{3.487500in}{1.925000in}}%
\pgfusepath{clip}%
\pgfsetbuttcap%
\pgfsetroundjoin%
\definecolor{currentfill}{rgb}{0.000000,0.000000,0.000000}%
\pgfsetfillcolor{currentfill}%
\pgfsetlinewidth{1.003750pt}%
\definecolor{currentstroke}{rgb}{0.000000,0.000000,0.000000}%
\pgfsetstrokecolor{currentstroke}%
\pgfsetdash{}{0pt}%
\pgfpathmoveto{\pgfqpoint{2.824734in}{0.356667in}}%
\pgfpathcurveto{\pgfqpoint{2.830260in}{0.356667in}}{\pgfqpoint{2.835559in}{0.358862in}}{\pgfqpoint{2.839466in}{0.362769in}}%
\pgfpathcurveto{\pgfqpoint{2.843373in}{0.366675in}}{\pgfqpoint{2.845568in}{0.371975in}}{\pgfqpoint{2.845568in}{0.377500in}}%
\pgfpathcurveto{\pgfqpoint{2.845568in}{0.383025in}}{\pgfqpoint{2.843373in}{0.388325in}}{\pgfqpoint{2.839466in}{0.392231in}}%
\pgfpathcurveto{\pgfqpoint{2.835559in}{0.396138in}}{\pgfqpoint{2.830260in}{0.398333in}}{\pgfqpoint{2.824734in}{0.398333in}}%
\pgfpathcurveto{\pgfqpoint{2.819209in}{0.398333in}}{\pgfqpoint{2.813910in}{0.396138in}}{\pgfqpoint{2.810003in}{0.392231in}}%
\pgfpathcurveto{\pgfqpoint{2.806096in}{0.388325in}}{\pgfqpoint{2.803901in}{0.383025in}}{\pgfqpoint{2.803901in}{0.377500in}}%
\pgfpathcurveto{\pgfqpoint{2.803901in}{0.371975in}}{\pgfqpoint{2.806096in}{0.366675in}}{\pgfqpoint{2.810003in}{0.362769in}}%
\pgfpathcurveto{\pgfqpoint{2.813910in}{0.358862in}}{\pgfqpoint{2.819209in}{0.356667in}}{\pgfqpoint{2.824734in}{0.356667in}}%
\pgfpathclose%
\pgfusepath{stroke,fill}%
\end{pgfscope}%
\begin{pgfscope}%
\pgfpathrectangle{\pgfqpoint{0.562500in}{0.275000in}}{\pgfqpoint{3.487500in}{1.925000in}}%
\pgfusepath{clip}%
\pgfsetbuttcap%
\pgfsetroundjoin%
\definecolor{currentfill}{rgb}{0.000000,0.000000,0.000000}%
\pgfsetfillcolor{currentfill}%
\pgfsetlinewidth{1.003750pt}%
\definecolor{currentstroke}{rgb}{0.000000,0.000000,0.000000}%
\pgfsetstrokecolor{currentstroke}%
\pgfsetdash{}{0pt}%
\pgfpathmoveto{\pgfqpoint{2.824734in}{0.356667in}}%
\pgfpathcurveto{\pgfqpoint{2.830260in}{0.356667in}}{\pgfqpoint{2.835559in}{0.358862in}}{\pgfqpoint{2.839466in}{0.362769in}}%
\pgfpathcurveto{\pgfqpoint{2.843373in}{0.366675in}}{\pgfqpoint{2.845568in}{0.371975in}}{\pgfqpoint{2.845568in}{0.377500in}}%
\pgfpathcurveto{\pgfqpoint{2.845568in}{0.383025in}}{\pgfqpoint{2.843373in}{0.388325in}}{\pgfqpoint{2.839466in}{0.392231in}}%
\pgfpathcurveto{\pgfqpoint{2.835559in}{0.396138in}}{\pgfqpoint{2.830260in}{0.398333in}}{\pgfqpoint{2.824734in}{0.398333in}}%
\pgfpathcurveto{\pgfqpoint{2.819209in}{0.398333in}}{\pgfqpoint{2.813910in}{0.396138in}}{\pgfqpoint{2.810003in}{0.392231in}}%
\pgfpathcurveto{\pgfqpoint{2.806096in}{0.388325in}}{\pgfqpoint{2.803901in}{0.383025in}}{\pgfqpoint{2.803901in}{0.377500in}}%
\pgfpathcurveto{\pgfqpoint{2.803901in}{0.371975in}}{\pgfqpoint{2.806096in}{0.366675in}}{\pgfqpoint{2.810003in}{0.362769in}}%
\pgfpathcurveto{\pgfqpoint{2.813910in}{0.358862in}}{\pgfqpoint{2.819209in}{0.356667in}}{\pgfqpoint{2.824734in}{0.356667in}}%
\pgfpathclose%
\pgfusepath{stroke,fill}%
\end{pgfscope}%
\begin{pgfscope}%
\pgfpathrectangle{\pgfqpoint{0.562500in}{0.275000in}}{\pgfqpoint{3.487500in}{1.925000in}}%
\pgfusepath{clip}%
\pgfsetbuttcap%
\pgfsetroundjoin%
\definecolor{currentfill}{rgb}{0.000000,0.000000,0.000000}%
\pgfsetfillcolor{currentfill}%
\pgfsetlinewidth{1.003750pt}%
\definecolor{currentstroke}{rgb}{0.000000,0.000000,0.000000}%
\pgfsetstrokecolor{currentstroke}%
\pgfsetdash{}{0pt}%
\pgfpathmoveto{\pgfqpoint{2.824734in}{0.356667in}}%
\pgfpathcurveto{\pgfqpoint{2.830260in}{0.356667in}}{\pgfqpoint{2.835559in}{0.358862in}}{\pgfqpoint{2.839466in}{0.362769in}}%
\pgfpathcurveto{\pgfqpoint{2.843373in}{0.366675in}}{\pgfqpoint{2.845568in}{0.371975in}}{\pgfqpoint{2.845568in}{0.377500in}}%
\pgfpathcurveto{\pgfqpoint{2.845568in}{0.383025in}}{\pgfqpoint{2.843373in}{0.388325in}}{\pgfqpoint{2.839466in}{0.392231in}}%
\pgfpathcurveto{\pgfqpoint{2.835559in}{0.396138in}}{\pgfqpoint{2.830260in}{0.398333in}}{\pgfqpoint{2.824734in}{0.398333in}}%
\pgfpathcurveto{\pgfqpoint{2.819209in}{0.398333in}}{\pgfqpoint{2.813910in}{0.396138in}}{\pgfqpoint{2.810003in}{0.392231in}}%
\pgfpathcurveto{\pgfqpoint{2.806096in}{0.388325in}}{\pgfqpoint{2.803901in}{0.383025in}}{\pgfqpoint{2.803901in}{0.377500in}}%
\pgfpathcurveto{\pgfqpoint{2.803901in}{0.371975in}}{\pgfqpoint{2.806096in}{0.366675in}}{\pgfqpoint{2.810003in}{0.362769in}}%
\pgfpathcurveto{\pgfqpoint{2.813910in}{0.358862in}}{\pgfqpoint{2.819209in}{0.356667in}}{\pgfqpoint{2.824734in}{0.356667in}}%
\pgfpathclose%
\pgfusepath{stroke,fill}%
\end{pgfscope}%
\begin{pgfscope}%
\pgfpathrectangle{\pgfqpoint{0.562500in}{0.275000in}}{\pgfqpoint{3.487500in}{1.925000in}}%
\pgfusepath{clip}%
\pgfsetbuttcap%
\pgfsetroundjoin%
\definecolor{currentfill}{rgb}{0.000000,0.000000,0.000000}%
\pgfsetfillcolor{currentfill}%
\pgfsetlinewidth{1.003750pt}%
\definecolor{currentstroke}{rgb}{0.000000,0.000000,0.000000}%
\pgfsetstrokecolor{currentstroke}%
\pgfsetdash{}{0pt}%
\pgfpathmoveto{\pgfqpoint{2.824734in}{0.356667in}}%
\pgfpathcurveto{\pgfqpoint{2.830260in}{0.356667in}}{\pgfqpoint{2.835559in}{0.358862in}}{\pgfqpoint{2.839466in}{0.362769in}}%
\pgfpathcurveto{\pgfqpoint{2.843373in}{0.366675in}}{\pgfqpoint{2.845568in}{0.371975in}}{\pgfqpoint{2.845568in}{0.377500in}}%
\pgfpathcurveto{\pgfqpoint{2.845568in}{0.383025in}}{\pgfqpoint{2.843373in}{0.388325in}}{\pgfqpoint{2.839466in}{0.392231in}}%
\pgfpathcurveto{\pgfqpoint{2.835559in}{0.396138in}}{\pgfqpoint{2.830260in}{0.398333in}}{\pgfqpoint{2.824734in}{0.398333in}}%
\pgfpathcurveto{\pgfqpoint{2.819209in}{0.398333in}}{\pgfqpoint{2.813910in}{0.396138in}}{\pgfqpoint{2.810003in}{0.392231in}}%
\pgfpathcurveto{\pgfqpoint{2.806096in}{0.388325in}}{\pgfqpoint{2.803901in}{0.383025in}}{\pgfqpoint{2.803901in}{0.377500in}}%
\pgfpathcurveto{\pgfqpoint{2.803901in}{0.371975in}}{\pgfqpoint{2.806096in}{0.366675in}}{\pgfqpoint{2.810003in}{0.362769in}}%
\pgfpathcurveto{\pgfqpoint{2.813910in}{0.358862in}}{\pgfqpoint{2.819209in}{0.356667in}}{\pgfqpoint{2.824734in}{0.356667in}}%
\pgfpathclose%
\pgfusepath{stroke,fill}%
\end{pgfscope}%
\begin{pgfscope}%
\pgfpathrectangle{\pgfqpoint{0.562500in}{0.275000in}}{\pgfqpoint{3.487500in}{1.925000in}}%
\pgfusepath{clip}%
\pgfsetbuttcap%
\pgfsetroundjoin%
\definecolor{currentfill}{rgb}{0.000000,0.000000,0.000000}%
\pgfsetfillcolor{currentfill}%
\pgfsetlinewidth{1.003750pt}%
\definecolor{currentstroke}{rgb}{0.000000,0.000000,0.000000}%
\pgfsetstrokecolor{currentstroke}%
\pgfsetdash{}{0pt}%
\pgfpathmoveto{\pgfqpoint{2.824734in}{0.356667in}}%
\pgfpathcurveto{\pgfqpoint{2.830260in}{0.356667in}}{\pgfqpoint{2.835559in}{0.358862in}}{\pgfqpoint{2.839466in}{0.362769in}}%
\pgfpathcurveto{\pgfqpoint{2.843373in}{0.366675in}}{\pgfqpoint{2.845568in}{0.371975in}}{\pgfqpoint{2.845568in}{0.377500in}}%
\pgfpathcurveto{\pgfqpoint{2.845568in}{0.383025in}}{\pgfqpoint{2.843373in}{0.388325in}}{\pgfqpoint{2.839466in}{0.392231in}}%
\pgfpathcurveto{\pgfqpoint{2.835559in}{0.396138in}}{\pgfqpoint{2.830260in}{0.398333in}}{\pgfqpoint{2.824734in}{0.398333in}}%
\pgfpathcurveto{\pgfqpoint{2.819209in}{0.398333in}}{\pgfqpoint{2.813910in}{0.396138in}}{\pgfqpoint{2.810003in}{0.392231in}}%
\pgfpathcurveto{\pgfqpoint{2.806096in}{0.388325in}}{\pgfqpoint{2.803901in}{0.383025in}}{\pgfqpoint{2.803901in}{0.377500in}}%
\pgfpathcurveto{\pgfqpoint{2.803901in}{0.371975in}}{\pgfqpoint{2.806096in}{0.366675in}}{\pgfqpoint{2.810003in}{0.362769in}}%
\pgfpathcurveto{\pgfqpoint{2.813910in}{0.358862in}}{\pgfqpoint{2.819209in}{0.356667in}}{\pgfqpoint{2.824734in}{0.356667in}}%
\pgfpathclose%
\pgfusepath{stroke,fill}%
\end{pgfscope}%
\begin{pgfscope}%
\pgfpathrectangle{\pgfqpoint{0.562500in}{0.275000in}}{\pgfqpoint{3.487500in}{1.925000in}}%
\pgfusepath{clip}%
\pgfsetbuttcap%
\pgfsetroundjoin%
\definecolor{currentfill}{rgb}{0.000000,0.000000,0.000000}%
\pgfsetfillcolor{currentfill}%
\pgfsetlinewidth{1.003750pt}%
\definecolor{currentstroke}{rgb}{0.000000,0.000000,0.000000}%
\pgfsetstrokecolor{currentstroke}%
\pgfsetdash{}{0pt}%
\pgfpathmoveto{\pgfqpoint{2.824734in}{0.356667in}}%
\pgfpathcurveto{\pgfqpoint{2.830260in}{0.356667in}}{\pgfqpoint{2.835559in}{0.358862in}}{\pgfqpoint{2.839466in}{0.362769in}}%
\pgfpathcurveto{\pgfqpoint{2.843373in}{0.366675in}}{\pgfqpoint{2.845568in}{0.371975in}}{\pgfqpoint{2.845568in}{0.377500in}}%
\pgfpathcurveto{\pgfqpoint{2.845568in}{0.383025in}}{\pgfqpoint{2.843373in}{0.388325in}}{\pgfqpoint{2.839466in}{0.392231in}}%
\pgfpathcurveto{\pgfqpoint{2.835559in}{0.396138in}}{\pgfqpoint{2.830260in}{0.398333in}}{\pgfqpoint{2.824734in}{0.398333in}}%
\pgfpathcurveto{\pgfqpoint{2.819209in}{0.398333in}}{\pgfqpoint{2.813910in}{0.396138in}}{\pgfqpoint{2.810003in}{0.392231in}}%
\pgfpathcurveto{\pgfqpoint{2.806096in}{0.388325in}}{\pgfqpoint{2.803901in}{0.383025in}}{\pgfqpoint{2.803901in}{0.377500in}}%
\pgfpathcurveto{\pgfqpoint{2.803901in}{0.371975in}}{\pgfqpoint{2.806096in}{0.366675in}}{\pgfqpoint{2.810003in}{0.362769in}}%
\pgfpathcurveto{\pgfqpoint{2.813910in}{0.358862in}}{\pgfqpoint{2.819209in}{0.356667in}}{\pgfqpoint{2.824734in}{0.356667in}}%
\pgfpathclose%
\pgfusepath{stroke,fill}%
\end{pgfscope}%
\begin{pgfscope}%
\pgfpathrectangle{\pgfqpoint{0.562500in}{0.275000in}}{\pgfqpoint{3.487500in}{1.925000in}}%
\pgfusepath{clip}%
\pgfsetbuttcap%
\pgfsetroundjoin%
\definecolor{currentfill}{rgb}{0.000000,0.000000,0.000000}%
\pgfsetfillcolor{currentfill}%
\pgfsetlinewidth{1.003750pt}%
\definecolor{currentstroke}{rgb}{0.000000,0.000000,0.000000}%
\pgfsetstrokecolor{currentstroke}%
\pgfsetdash{}{0pt}%
\pgfpathmoveto{\pgfqpoint{2.824734in}{0.356667in}}%
\pgfpathcurveto{\pgfqpoint{2.830260in}{0.356667in}}{\pgfqpoint{2.835559in}{0.358862in}}{\pgfqpoint{2.839466in}{0.362769in}}%
\pgfpathcurveto{\pgfqpoint{2.843373in}{0.366675in}}{\pgfqpoint{2.845568in}{0.371975in}}{\pgfqpoint{2.845568in}{0.377500in}}%
\pgfpathcurveto{\pgfqpoint{2.845568in}{0.383025in}}{\pgfqpoint{2.843373in}{0.388325in}}{\pgfqpoint{2.839466in}{0.392231in}}%
\pgfpathcurveto{\pgfqpoint{2.835559in}{0.396138in}}{\pgfqpoint{2.830260in}{0.398333in}}{\pgfqpoint{2.824734in}{0.398333in}}%
\pgfpathcurveto{\pgfqpoint{2.819209in}{0.398333in}}{\pgfqpoint{2.813910in}{0.396138in}}{\pgfqpoint{2.810003in}{0.392231in}}%
\pgfpathcurveto{\pgfqpoint{2.806096in}{0.388325in}}{\pgfqpoint{2.803901in}{0.383025in}}{\pgfqpoint{2.803901in}{0.377500in}}%
\pgfpathcurveto{\pgfqpoint{2.803901in}{0.371975in}}{\pgfqpoint{2.806096in}{0.366675in}}{\pgfqpoint{2.810003in}{0.362769in}}%
\pgfpathcurveto{\pgfqpoint{2.813910in}{0.358862in}}{\pgfqpoint{2.819209in}{0.356667in}}{\pgfqpoint{2.824734in}{0.356667in}}%
\pgfpathclose%
\pgfusepath{stroke,fill}%
\end{pgfscope}%
\begin{pgfscope}%
\pgfpathrectangle{\pgfqpoint{0.562500in}{0.275000in}}{\pgfqpoint{3.487500in}{1.925000in}}%
\pgfusepath{clip}%
\pgfsetbuttcap%
\pgfsetroundjoin%
\definecolor{currentfill}{rgb}{0.000000,0.000000,0.000000}%
\pgfsetfillcolor{currentfill}%
\pgfsetlinewidth{1.003750pt}%
\definecolor{currentstroke}{rgb}{0.000000,0.000000,0.000000}%
\pgfsetstrokecolor{currentstroke}%
\pgfsetdash{}{0pt}%
\pgfpathmoveto{\pgfqpoint{2.824734in}{0.356667in}}%
\pgfpathcurveto{\pgfqpoint{2.830260in}{0.356667in}}{\pgfqpoint{2.835559in}{0.358862in}}{\pgfqpoint{2.839466in}{0.362769in}}%
\pgfpathcurveto{\pgfqpoint{2.843373in}{0.366675in}}{\pgfqpoint{2.845568in}{0.371975in}}{\pgfqpoint{2.845568in}{0.377500in}}%
\pgfpathcurveto{\pgfqpoint{2.845568in}{0.383025in}}{\pgfqpoint{2.843373in}{0.388325in}}{\pgfqpoint{2.839466in}{0.392231in}}%
\pgfpathcurveto{\pgfqpoint{2.835559in}{0.396138in}}{\pgfqpoint{2.830260in}{0.398333in}}{\pgfqpoint{2.824734in}{0.398333in}}%
\pgfpathcurveto{\pgfqpoint{2.819209in}{0.398333in}}{\pgfqpoint{2.813910in}{0.396138in}}{\pgfqpoint{2.810003in}{0.392231in}}%
\pgfpathcurveto{\pgfqpoint{2.806096in}{0.388325in}}{\pgfqpoint{2.803901in}{0.383025in}}{\pgfqpoint{2.803901in}{0.377500in}}%
\pgfpathcurveto{\pgfqpoint{2.803901in}{0.371975in}}{\pgfqpoint{2.806096in}{0.366675in}}{\pgfqpoint{2.810003in}{0.362769in}}%
\pgfpathcurveto{\pgfqpoint{2.813910in}{0.358862in}}{\pgfqpoint{2.819209in}{0.356667in}}{\pgfqpoint{2.824734in}{0.356667in}}%
\pgfpathclose%
\pgfusepath{stroke,fill}%
\end{pgfscope}%
\begin{pgfscope}%
\pgfpathrectangle{\pgfqpoint{0.562500in}{0.275000in}}{\pgfqpoint{3.487500in}{1.925000in}}%
\pgfusepath{clip}%
\pgfsetbuttcap%
\pgfsetroundjoin%
\definecolor{currentfill}{rgb}{0.000000,0.000000,0.000000}%
\pgfsetfillcolor{currentfill}%
\pgfsetlinewidth{1.003750pt}%
\definecolor{currentstroke}{rgb}{0.000000,0.000000,0.000000}%
\pgfsetstrokecolor{currentstroke}%
\pgfsetdash{}{0pt}%
\pgfpathmoveto{\pgfqpoint{2.824734in}{0.356667in}}%
\pgfpathcurveto{\pgfqpoint{2.830260in}{0.356667in}}{\pgfqpoint{2.835559in}{0.358862in}}{\pgfqpoint{2.839466in}{0.362769in}}%
\pgfpathcurveto{\pgfqpoint{2.843373in}{0.366675in}}{\pgfqpoint{2.845568in}{0.371975in}}{\pgfqpoint{2.845568in}{0.377500in}}%
\pgfpathcurveto{\pgfqpoint{2.845568in}{0.383025in}}{\pgfqpoint{2.843373in}{0.388325in}}{\pgfqpoint{2.839466in}{0.392231in}}%
\pgfpathcurveto{\pgfqpoint{2.835559in}{0.396138in}}{\pgfqpoint{2.830260in}{0.398333in}}{\pgfqpoint{2.824734in}{0.398333in}}%
\pgfpathcurveto{\pgfqpoint{2.819209in}{0.398333in}}{\pgfqpoint{2.813910in}{0.396138in}}{\pgfqpoint{2.810003in}{0.392231in}}%
\pgfpathcurveto{\pgfqpoint{2.806096in}{0.388325in}}{\pgfqpoint{2.803901in}{0.383025in}}{\pgfqpoint{2.803901in}{0.377500in}}%
\pgfpathcurveto{\pgfqpoint{2.803901in}{0.371975in}}{\pgfqpoint{2.806096in}{0.366675in}}{\pgfqpoint{2.810003in}{0.362769in}}%
\pgfpathcurveto{\pgfqpoint{2.813910in}{0.358862in}}{\pgfqpoint{2.819209in}{0.356667in}}{\pgfqpoint{2.824734in}{0.356667in}}%
\pgfpathclose%
\pgfusepath{stroke,fill}%
\end{pgfscope}%
\begin{pgfscope}%
\pgfpathrectangle{\pgfqpoint{0.562500in}{0.275000in}}{\pgfqpoint{3.487500in}{1.925000in}}%
\pgfusepath{clip}%
\pgfsetbuttcap%
\pgfsetroundjoin%
\definecolor{currentfill}{rgb}{0.000000,0.000000,0.000000}%
\pgfsetfillcolor{currentfill}%
\pgfsetlinewidth{1.003750pt}%
\definecolor{currentstroke}{rgb}{0.000000,0.000000,0.000000}%
\pgfsetstrokecolor{currentstroke}%
\pgfsetdash{}{0pt}%
\pgfpathmoveto{\pgfqpoint{2.824734in}{0.356667in}}%
\pgfpathcurveto{\pgfqpoint{2.830260in}{0.356667in}}{\pgfqpoint{2.835559in}{0.358862in}}{\pgfqpoint{2.839466in}{0.362769in}}%
\pgfpathcurveto{\pgfqpoint{2.843373in}{0.366675in}}{\pgfqpoint{2.845568in}{0.371975in}}{\pgfqpoint{2.845568in}{0.377500in}}%
\pgfpathcurveto{\pgfqpoint{2.845568in}{0.383025in}}{\pgfqpoint{2.843373in}{0.388325in}}{\pgfqpoint{2.839466in}{0.392231in}}%
\pgfpathcurveto{\pgfqpoint{2.835559in}{0.396138in}}{\pgfqpoint{2.830260in}{0.398333in}}{\pgfqpoint{2.824734in}{0.398333in}}%
\pgfpathcurveto{\pgfqpoint{2.819209in}{0.398333in}}{\pgfqpoint{2.813910in}{0.396138in}}{\pgfqpoint{2.810003in}{0.392231in}}%
\pgfpathcurveto{\pgfqpoint{2.806096in}{0.388325in}}{\pgfqpoint{2.803901in}{0.383025in}}{\pgfqpoint{2.803901in}{0.377500in}}%
\pgfpathcurveto{\pgfqpoint{2.803901in}{0.371975in}}{\pgfqpoint{2.806096in}{0.366675in}}{\pgfqpoint{2.810003in}{0.362769in}}%
\pgfpathcurveto{\pgfqpoint{2.813910in}{0.358862in}}{\pgfqpoint{2.819209in}{0.356667in}}{\pgfqpoint{2.824734in}{0.356667in}}%
\pgfpathclose%
\pgfusepath{stroke,fill}%
\end{pgfscope}%
\begin{pgfscope}%
\pgfpathrectangle{\pgfqpoint{0.562500in}{0.275000in}}{\pgfqpoint{3.487500in}{1.925000in}}%
\pgfusepath{clip}%
\pgfsetbuttcap%
\pgfsetroundjoin%
\definecolor{currentfill}{rgb}{0.000000,0.000000,0.000000}%
\pgfsetfillcolor{currentfill}%
\pgfsetlinewidth{1.003750pt}%
\definecolor{currentstroke}{rgb}{0.000000,0.000000,0.000000}%
\pgfsetstrokecolor{currentstroke}%
\pgfsetdash{}{0pt}%
\pgfpathmoveto{\pgfqpoint{2.824734in}{0.356667in}}%
\pgfpathcurveto{\pgfqpoint{2.830260in}{0.356667in}}{\pgfqpoint{2.835559in}{0.358862in}}{\pgfqpoint{2.839466in}{0.362769in}}%
\pgfpathcurveto{\pgfqpoint{2.843373in}{0.366675in}}{\pgfqpoint{2.845568in}{0.371975in}}{\pgfqpoint{2.845568in}{0.377500in}}%
\pgfpathcurveto{\pgfqpoint{2.845568in}{0.383025in}}{\pgfqpoint{2.843373in}{0.388325in}}{\pgfqpoint{2.839466in}{0.392231in}}%
\pgfpathcurveto{\pgfqpoint{2.835559in}{0.396138in}}{\pgfqpoint{2.830260in}{0.398333in}}{\pgfqpoint{2.824734in}{0.398333in}}%
\pgfpathcurveto{\pgfqpoint{2.819209in}{0.398333in}}{\pgfqpoint{2.813910in}{0.396138in}}{\pgfqpoint{2.810003in}{0.392231in}}%
\pgfpathcurveto{\pgfqpoint{2.806096in}{0.388325in}}{\pgfqpoint{2.803901in}{0.383025in}}{\pgfqpoint{2.803901in}{0.377500in}}%
\pgfpathcurveto{\pgfqpoint{2.803901in}{0.371975in}}{\pgfqpoint{2.806096in}{0.366675in}}{\pgfqpoint{2.810003in}{0.362769in}}%
\pgfpathcurveto{\pgfqpoint{2.813910in}{0.358862in}}{\pgfqpoint{2.819209in}{0.356667in}}{\pgfqpoint{2.824734in}{0.356667in}}%
\pgfpathclose%
\pgfusepath{stroke,fill}%
\end{pgfscope}%
\begin{pgfscope}%
\pgfpathrectangle{\pgfqpoint{0.562500in}{0.275000in}}{\pgfqpoint{3.487500in}{1.925000in}}%
\pgfusepath{clip}%
\pgfsetbuttcap%
\pgfsetroundjoin%
\definecolor{currentfill}{rgb}{0.000000,0.000000,0.000000}%
\pgfsetfillcolor{currentfill}%
\pgfsetlinewidth{1.003750pt}%
\definecolor{currentstroke}{rgb}{0.000000,0.000000,0.000000}%
\pgfsetstrokecolor{currentstroke}%
\pgfsetdash{}{0pt}%
\pgfpathmoveto{\pgfqpoint{2.824734in}{0.356667in}}%
\pgfpathcurveto{\pgfqpoint{2.830260in}{0.356667in}}{\pgfqpoint{2.835559in}{0.358862in}}{\pgfqpoint{2.839466in}{0.362769in}}%
\pgfpathcurveto{\pgfqpoint{2.843373in}{0.366675in}}{\pgfqpoint{2.845568in}{0.371975in}}{\pgfqpoint{2.845568in}{0.377500in}}%
\pgfpathcurveto{\pgfqpoint{2.845568in}{0.383025in}}{\pgfqpoint{2.843373in}{0.388325in}}{\pgfqpoint{2.839466in}{0.392231in}}%
\pgfpathcurveto{\pgfqpoint{2.835559in}{0.396138in}}{\pgfqpoint{2.830260in}{0.398333in}}{\pgfqpoint{2.824734in}{0.398333in}}%
\pgfpathcurveto{\pgfqpoint{2.819209in}{0.398333in}}{\pgfqpoint{2.813910in}{0.396138in}}{\pgfqpoint{2.810003in}{0.392231in}}%
\pgfpathcurveto{\pgfqpoint{2.806096in}{0.388325in}}{\pgfqpoint{2.803901in}{0.383025in}}{\pgfqpoint{2.803901in}{0.377500in}}%
\pgfpathcurveto{\pgfqpoint{2.803901in}{0.371975in}}{\pgfqpoint{2.806096in}{0.366675in}}{\pgfqpoint{2.810003in}{0.362769in}}%
\pgfpathcurveto{\pgfqpoint{2.813910in}{0.358862in}}{\pgfqpoint{2.819209in}{0.356667in}}{\pgfqpoint{2.824734in}{0.356667in}}%
\pgfpathclose%
\pgfusepath{stroke,fill}%
\end{pgfscope}%
\begin{pgfscope}%
\pgfpathrectangle{\pgfqpoint{0.562500in}{0.275000in}}{\pgfqpoint{3.487500in}{1.925000in}}%
\pgfusepath{clip}%
\pgfsetbuttcap%
\pgfsetroundjoin%
\definecolor{currentfill}{rgb}{0.000000,0.000000,0.000000}%
\pgfsetfillcolor{currentfill}%
\pgfsetlinewidth{1.003750pt}%
\definecolor{currentstroke}{rgb}{0.000000,0.000000,0.000000}%
\pgfsetstrokecolor{currentstroke}%
\pgfsetdash{}{0pt}%
\pgfpathmoveto{\pgfqpoint{2.824734in}{0.356667in}}%
\pgfpathcurveto{\pgfqpoint{2.830260in}{0.356667in}}{\pgfqpoint{2.835559in}{0.358862in}}{\pgfqpoint{2.839466in}{0.362769in}}%
\pgfpathcurveto{\pgfqpoint{2.843373in}{0.366675in}}{\pgfqpoint{2.845568in}{0.371975in}}{\pgfqpoint{2.845568in}{0.377500in}}%
\pgfpathcurveto{\pgfqpoint{2.845568in}{0.383025in}}{\pgfqpoint{2.843373in}{0.388325in}}{\pgfqpoint{2.839466in}{0.392231in}}%
\pgfpathcurveto{\pgfqpoint{2.835559in}{0.396138in}}{\pgfqpoint{2.830260in}{0.398333in}}{\pgfqpoint{2.824734in}{0.398333in}}%
\pgfpathcurveto{\pgfqpoint{2.819209in}{0.398333in}}{\pgfqpoint{2.813910in}{0.396138in}}{\pgfqpoint{2.810003in}{0.392231in}}%
\pgfpathcurveto{\pgfqpoint{2.806096in}{0.388325in}}{\pgfqpoint{2.803901in}{0.383025in}}{\pgfqpoint{2.803901in}{0.377500in}}%
\pgfpathcurveto{\pgfqpoint{2.803901in}{0.371975in}}{\pgfqpoint{2.806096in}{0.366675in}}{\pgfqpoint{2.810003in}{0.362769in}}%
\pgfpathcurveto{\pgfqpoint{2.813910in}{0.358862in}}{\pgfqpoint{2.819209in}{0.356667in}}{\pgfqpoint{2.824734in}{0.356667in}}%
\pgfpathclose%
\pgfusepath{stroke,fill}%
\end{pgfscope}%
\begin{pgfscope}%
\pgfpathrectangle{\pgfqpoint{0.562500in}{0.275000in}}{\pgfqpoint{3.487500in}{1.925000in}}%
\pgfusepath{clip}%
\pgfsetbuttcap%
\pgfsetroundjoin%
\definecolor{currentfill}{rgb}{0.000000,0.000000,0.000000}%
\pgfsetfillcolor{currentfill}%
\pgfsetlinewidth{1.003750pt}%
\definecolor{currentstroke}{rgb}{0.000000,0.000000,0.000000}%
\pgfsetstrokecolor{currentstroke}%
\pgfsetdash{}{0pt}%
\pgfpathmoveto{\pgfqpoint{2.824734in}{0.356667in}}%
\pgfpathcurveto{\pgfqpoint{2.830260in}{0.356667in}}{\pgfqpoint{2.835559in}{0.358862in}}{\pgfqpoint{2.839466in}{0.362769in}}%
\pgfpathcurveto{\pgfqpoint{2.843373in}{0.366675in}}{\pgfqpoint{2.845568in}{0.371975in}}{\pgfqpoint{2.845568in}{0.377500in}}%
\pgfpathcurveto{\pgfqpoint{2.845568in}{0.383025in}}{\pgfqpoint{2.843373in}{0.388325in}}{\pgfqpoint{2.839466in}{0.392231in}}%
\pgfpathcurveto{\pgfqpoint{2.835559in}{0.396138in}}{\pgfqpoint{2.830260in}{0.398333in}}{\pgfqpoint{2.824734in}{0.398333in}}%
\pgfpathcurveto{\pgfqpoint{2.819209in}{0.398333in}}{\pgfqpoint{2.813910in}{0.396138in}}{\pgfqpoint{2.810003in}{0.392231in}}%
\pgfpathcurveto{\pgfqpoint{2.806096in}{0.388325in}}{\pgfqpoint{2.803901in}{0.383025in}}{\pgfqpoint{2.803901in}{0.377500in}}%
\pgfpathcurveto{\pgfqpoint{2.803901in}{0.371975in}}{\pgfqpoint{2.806096in}{0.366675in}}{\pgfqpoint{2.810003in}{0.362769in}}%
\pgfpathcurveto{\pgfqpoint{2.813910in}{0.358862in}}{\pgfqpoint{2.819209in}{0.356667in}}{\pgfqpoint{2.824734in}{0.356667in}}%
\pgfpathclose%
\pgfusepath{stroke,fill}%
\end{pgfscope}%
\begin{pgfscope}%
\pgfpathrectangle{\pgfqpoint{0.562500in}{0.275000in}}{\pgfqpoint{3.487500in}{1.925000in}}%
\pgfusepath{clip}%
\pgfsetbuttcap%
\pgfsetroundjoin%
\definecolor{currentfill}{rgb}{0.000000,0.000000,0.000000}%
\pgfsetfillcolor{currentfill}%
\pgfsetlinewidth{1.003750pt}%
\definecolor{currentstroke}{rgb}{0.000000,0.000000,0.000000}%
\pgfsetstrokecolor{currentstroke}%
\pgfsetdash{}{0pt}%
\pgfpathmoveto{\pgfqpoint{2.824734in}{0.356667in}}%
\pgfpathcurveto{\pgfqpoint{2.830260in}{0.356667in}}{\pgfqpoint{2.835559in}{0.358862in}}{\pgfqpoint{2.839466in}{0.362769in}}%
\pgfpathcurveto{\pgfqpoint{2.843373in}{0.366675in}}{\pgfqpoint{2.845568in}{0.371975in}}{\pgfqpoint{2.845568in}{0.377500in}}%
\pgfpathcurveto{\pgfqpoint{2.845568in}{0.383025in}}{\pgfqpoint{2.843373in}{0.388325in}}{\pgfqpoint{2.839466in}{0.392231in}}%
\pgfpathcurveto{\pgfqpoint{2.835559in}{0.396138in}}{\pgfqpoint{2.830260in}{0.398333in}}{\pgfqpoint{2.824734in}{0.398333in}}%
\pgfpathcurveto{\pgfqpoint{2.819209in}{0.398333in}}{\pgfqpoint{2.813910in}{0.396138in}}{\pgfqpoint{2.810003in}{0.392231in}}%
\pgfpathcurveto{\pgfqpoint{2.806096in}{0.388325in}}{\pgfqpoint{2.803901in}{0.383025in}}{\pgfqpoint{2.803901in}{0.377500in}}%
\pgfpathcurveto{\pgfqpoint{2.803901in}{0.371975in}}{\pgfqpoint{2.806096in}{0.366675in}}{\pgfqpoint{2.810003in}{0.362769in}}%
\pgfpathcurveto{\pgfqpoint{2.813910in}{0.358862in}}{\pgfqpoint{2.819209in}{0.356667in}}{\pgfqpoint{2.824734in}{0.356667in}}%
\pgfpathclose%
\pgfusepath{stroke,fill}%
\end{pgfscope}%
\begin{pgfscope}%
\pgfpathrectangle{\pgfqpoint{0.562500in}{0.275000in}}{\pgfqpoint{3.487500in}{1.925000in}}%
\pgfusepath{clip}%
\pgfsetbuttcap%
\pgfsetroundjoin%
\definecolor{currentfill}{rgb}{0.000000,0.000000,0.000000}%
\pgfsetfillcolor{currentfill}%
\pgfsetlinewidth{1.003750pt}%
\definecolor{currentstroke}{rgb}{0.000000,0.000000,0.000000}%
\pgfsetstrokecolor{currentstroke}%
\pgfsetdash{}{0pt}%
\pgfpathmoveto{\pgfqpoint{2.824734in}{0.356667in}}%
\pgfpathcurveto{\pgfqpoint{2.830260in}{0.356667in}}{\pgfqpoint{2.835559in}{0.358862in}}{\pgfqpoint{2.839466in}{0.362769in}}%
\pgfpathcurveto{\pgfqpoint{2.843373in}{0.366675in}}{\pgfqpoint{2.845568in}{0.371975in}}{\pgfqpoint{2.845568in}{0.377500in}}%
\pgfpathcurveto{\pgfqpoint{2.845568in}{0.383025in}}{\pgfqpoint{2.843373in}{0.388325in}}{\pgfqpoint{2.839466in}{0.392231in}}%
\pgfpathcurveto{\pgfqpoint{2.835559in}{0.396138in}}{\pgfqpoint{2.830260in}{0.398333in}}{\pgfqpoint{2.824734in}{0.398333in}}%
\pgfpathcurveto{\pgfqpoint{2.819209in}{0.398333in}}{\pgfqpoint{2.813910in}{0.396138in}}{\pgfqpoint{2.810003in}{0.392231in}}%
\pgfpathcurveto{\pgfqpoint{2.806096in}{0.388325in}}{\pgfqpoint{2.803901in}{0.383025in}}{\pgfqpoint{2.803901in}{0.377500in}}%
\pgfpathcurveto{\pgfqpoint{2.803901in}{0.371975in}}{\pgfqpoint{2.806096in}{0.366675in}}{\pgfqpoint{2.810003in}{0.362769in}}%
\pgfpathcurveto{\pgfqpoint{2.813910in}{0.358862in}}{\pgfqpoint{2.819209in}{0.356667in}}{\pgfqpoint{2.824734in}{0.356667in}}%
\pgfpathclose%
\pgfusepath{stroke,fill}%
\end{pgfscope}%
\begin{pgfscope}%
\pgfpathrectangle{\pgfqpoint{0.562500in}{0.275000in}}{\pgfqpoint{3.487500in}{1.925000in}}%
\pgfusepath{clip}%
\pgfsetbuttcap%
\pgfsetroundjoin%
\definecolor{currentfill}{rgb}{0.000000,0.000000,0.000000}%
\pgfsetfillcolor{currentfill}%
\pgfsetlinewidth{1.003750pt}%
\definecolor{currentstroke}{rgb}{0.000000,0.000000,0.000000}%
\pgfsetstrokecolor{currentstroke}%
\pgfsetdash{}{0pt}%
\pgfpathmoveto{\pgfqpoint{2.824734in}{0.356667in}}%
\pgfpathcurveto{\pgfqpoint{2.830260in}{0.356667in}}{\pgfqpoint{2.835559in}{0.358862in}}{\pgfqpoint{2.839466in}{0.362769in}}%
\pgfpathcurveto{\pgfqpoint{2.843373in}{0.366675in}}{\pgfqpoint{2.845568in}{0.371975in}}{\pgfqpoint{2.845568in}{0.377500in}}%
\pgfpathcurveto{\pgfqpoint{2.845568in}{0.383025in}}{\pgfqpoint{2.843373in}{0.388325in}}{\pgfqpoint{2.839466in}{0.392231in}}%
\pgfpathcurveto{\pgfqpoint{2.835559in}{0.396138in}}{\pgfqpoint{2.830260in}{0.398333in}}{\pgfqpoint{2.824734in}{0.398333in}}%
\pgfpathcurveto{\pgfqpoint{2.819209in}{0.398333in}}{\pgfqpoint{2.813910in}{0.396138in}}{\pgfqpoint{2.810003in}{0.392231in}}%
\pgfpathcurveto{\pgfqpoint{2.806096in}{0.388325in}}{\pgfqpoint{2.803901in}{0.383025in}}{\pgfqpoint{2.803901in}{0.377500in}}%
\pgfpathcurveto{\pgfqpoint{2.803901in}{0.371975in}}{\pgfqpoint{2.806096in}{0.366675in}}{\pgfqpoint{2.810003in}{0.362769in}}%
\pgfpathcurveto{\pgfqpoint{2.813910in}{0.358862in}}{\pgfqpoint{2.819209in}{0.356667in}}{\pgfqpoint{2.824734in}{0.356667in}}%
\pgfpathclose%
\pgfusepath{stroke,fill}%
\end{pgfscope}%
\begin{pgfscope}%
\pgfpathrectangle{\pgfqpoint{0.562500in}{0.275000in}}{\pgfqpoint{3.487500in}{1.925000in}}%
\pgfusepath{clip}%
\pgfsetbuttcap%
\pgfsetroundjoin%
\definecolor{currentfill}{rgb}{0.000000,0.000000,0.000000}%
\pgfsetfillcolor{currentfill}%
\pgfsetlinewidth{1.003750pt}%
\definecolor{currentstroke}{rgb}{0.000000,0.000000,0.000000}%
\pgfsetstrokecolor{currentstroke}%
\pgfsetdash{}{0pt}%
\pgfpathmoveto{\pgfqpoint{2.824734in}{0.356667in}}%
\pgfpathcurveto{\pgfqpoint{2.830260in}{0.356667in}}{\pgfqpoint{2.835559in}{0.358862in}}{\pgfqpoint{2.839466in}{0.362769in}}%
\pgfpathcurveto{\pgfqpoint{2.843373in}{0.366675in}}{\pgfqpoint{2.845568in}{0.371975in}}{\pgfqpoint{2.845568in}{0.377500in}}%
\pgfpathcurveto{\pgfqpoint{2.845568in}{0.383025in}}{\pgfqpoint{2.843373in}{0.388325in}}{\pgfqpoint{2.839466in}{0.392231in}}%
\pgfpathcurveto{\pgfqpoint{2.835559in}{0.396138in}}{\pgfqpoint{2.830260in}{0.398333in}}{\pgfqpoint{2.824734in}{0.398333in}}%
\pgfpathcurveto{\pgfqpoint{2.819209in}{0.398333in}}{\pgfqpoint{2.813910in}{0.396138in}}{\pgfqpoint{2.810003in}{0.392231in}}%
\pgfpathcurveto{\pgfqpoint{2.806096in}{0.388325in}}{\pgfqpoint{2.803901in}{0.383025in}}{\pgfqpoint{2.803901in}{0.377500in}}%
\pgfpathcurveto{\pgfqpoint{2.803901in}{0.371975in}}{\pgfqpoint{2.806096in}{0.366675in}}{\pgfqpoint{2.810003in}{0.362769in}}%
\pgfpathcurveto{\pgfqpoint{2.813910in}{0.358862in}}{\pgfqpoint{2.819209in}{0.356667in}}{\pgfqpoint{2.824734in}{0.356667in}}%
\pgfpathclose%
\pgfusepath{stroke,fill}%
\end{pgfscope}%
\begin{pgfscope}%
\pgfpathrectangle{\pgfqpoint{0.562500in}{0.275000in}}{\pgfqpoint{3.487500in}{1.925000in}}%
\pgfusepath{clip}%
\pgfsetbuttcap%
\pgfsetroundjoin%
\definecolor{currentfill}{rgb}{0.000000,0.000000,0.000000}%
\pgfsetfillcolor{currentfill}%
\pgfsetlinewidth{1.003750pt}%
\definecolor{currentstroke}{rgb}{0.000000,0.000000,0.000000}%
\pgfsetstrokecolor{currentstroke}%
\pgfsetdash{}{0pt}%
\pgfpathmoveto{\pgfqpoint{2.824734in}{0.356667in}}%
\pgfpathcurveto{\pgfqpoint{2.830260in}{0.356667in}}{\pgfqpoint{2.835559in}{0.358862in}}{\pgfqpoint{2.839466in}{0.362769in}}%
\pgfpathcurveto{\pgfqpoint{2.843373in}{0.366675in}}{\pgfqpoint{2.845568in}{0.371975in}}{\pgfqpoint{2.845568in}{0.377500in}}%
\pgfpathcurveto{\pgfqpoint{2.845568in}{0.383025in}}{\pgfqpoint{2.843373in}{0.388325in}}{\pgfqpoint{2.839466in}{0.392231in}}%
\pgfpathcurveto{\pgfqpoint{2.835559in}{0.396138in}}{\pgfqpoint{2.830260in}{0.398333in}}{\pgfqpoint{2.824734in}{0.398333in}}%
\pgfpathcurveto{\pgfqpoint{2.819209in}{0.398333in}}{\pgfqpoint{2.813910in}{0.396138in}}{\pgfqpoint{2.810003in}{0.392231in}}%
\pgfpathcurveto{\pgfqpoint{2.806096in}{0.388325in}}{\pgfqpoint{2.803901in}{0.383025in}}{\pgfqpoint{2.803901in}{0.377500in}}%
\pgfpathcurveto{\pgfqpoint{2.803901in}{0.371975in}}{\pgfqpoint{2.806096in}{0.366675in}}{\pgfqpoint{2.810003in}{0.362769in}}%
\pgfpathcurveto{\pgfqpoint{2.813910in}{0.358862in}}{\pgfqpoint{2.819209in}{0.356667in}}{\pgfqpoint{2.824734in}{0.356667in}}%
\pgfpathclose%
\pgfusepath{stroke,fill}%
\end{pgfscope}%
\begin{pgfscope}%
\pgfpathrectangle{\pgfqpoint{0.562500in}{0.275000in}}{\pgfqpoint{3.487500in}{1.925000in}}%
\pgfusepath{clip}%
\pgfsetbuttcap%
\pgfsetroundjoin%
\definecolor{currentfill}{rgb}{0.000000,0.000000,0.000000}%
\pgfsetfillcolor{currentfill}%
\pgfsetlinewidth{1.003750pt}%
\definecolor{currentstroke}{rgb}{0.000000,0.000000,0.000000}%
\pgfsetstrokecolor{currentstroke}%
\pgfsetdash{}{0pt}%
\pgfpathmoveto{\pgfqpoint{2.824734in}{0.356667in}}%
\pgfpathcurveto{\pgfqpoint{2.830260in}{0.356667in}}{\pgfqpoint{2.835559in}{0.358862in}}{\pgfqpoint{2.839466in}{0.362769in}}%
\pgfpathcurveto{\pgfqpoint{2.843373in}{0.366675in}}{\pgfqpoint{2.845568in}{0.371975in}}{\pgfqpoint{2.845568in}{0.377500in}}%
\pgfpathcurveto{\pgfqpoint{2.845568in}{0.383025in}}{\pgfqpoint{2.843373in}{0.388325in}}{\pgfqpoint{2.839466in}{0.392231in}}%
\pgfpathcurveto{\pgfqpoint{2.835559in}{0.396138in}}{\pgfqpoint{2.830260in}{0.398333in}}{\pgfqpoint{2.824734in}{0.398333in}}%
\pgfpathcurveto{\pgfqpoint{2.819209in}{0.398333in}}{\pgfqpoint{2.813910in}{0.396138in}}{\pgfqpoint{2.810003in}{0.392231in}}%
\pgfpathcurveto{\pgfqpoint{2.806096in}{0.388325in}}{\pgfqpoint{2.803901in}{0.383025in}}{\pgfqpoint{2.803901in}{0.377500in}}%
\pgfpathcurveto{\pgfqpoint{2.803901in}{0.371975in}}{\pgfqpoint{2.806096in}{0.366675in}}{\pgfqpoint{2.810003in}{0.362769in}}%
\pgfpathcurveto{\pgfqpoint{2.813910in}{0.358862in}}{\pgfqpoint{2.819209in}{0.356667in}}{\pgfqpoint{2.824734in}{0.356667in}}%
\pgfpathclose%
\pgfusepath{stroke,fill}%
\end{pgfscope}%
\begin{pgfscope}%
\pgfpathrectangle{\pgfqpoint{0.562500in}{0.275000in}}{\pgfqpoint{3.487500in}{1.925000in}}%
\pgfusepath{clip}%
\pgfsetbuttcap%
\pgfsetroundjoin%
\definecolor{currentfill}{rgb}{0.000000,0.000000,0.000000}%
\pgfsetfillcolor{currentfill}%
\pgfsetlinewidth{1.003750pt}%
\definecolor{currentstroke}{rgb}{0.000000,0.000000,0.000000}%
\pgfsetstrokecolor{currentstroke}%
\pgfsetdash{}{0pt}%
\pgfpathmoveto{\pgfqpoint{2.824734in}{0.356667in}}%
\pgfpathcurveto{\pgfqpoint{2.830260in}{0.356667in}}{\pgfqpoint{2.835559in}{0.358862in}}{\pgfqpoint{2.839466in}{0.362769in}}%
\pgfpathcurveto{\pgfqpoint{2.843373in}{0.366675in}}{\pgfqpoint{2.845568in}{0.371975in}}{\pgfqpoint{2.845568in}{0.377500in}}%
\pgfpathcurveto{\pgfqpoint{2.845568in}{0.383025in}}{\pgfqpoint{2.843373in}{0.388325in}}{\pgfqpoint{2.839466in}{0.392231in}}%
\pgfpathcurveto{\pgfqpoint{2.835559in}{0.396138in}}{\pgfqpoint{2.830260in}{0.398333in}}{\pgfqpoint{2.824734in}{0.398333in}}%
\pgfpathcurveto{\pgfqpoint{2.819209in}{0.398333in}}{\pgfqpoint{2.813910in}{0.396138in}}{\pgfqpoint{2.810003in}{0.392231in}}%
\pgfpathcurveto{\pgfqpoint{2.806096in}{0.388325in}}{\pgfqpoint{2.803901in}{0.383025in}}{\pgfqpoint{2.803901in}{0.377500in}}%
\pgfpathcurveto{\pgfqpoint{2.803901in}{0.371975in}}{\pgfqpoint{2.806096in}{0.366675in}}{\pgfqpoint{2.810003in}{0.362769in}}%
\pgfpathcurveto{\pgfqpoint{2.813910in}{0.358862in}}{\pgfqpoint{2.819209in}{0.356667in}}{\pgfqpoint{2.824734in}{0.356667in}}%
\pgfpathclose%
\pgfusepath{stroke,fill}%
\end{pgfscope}%
\begin{pgfscope}%
\pgfpathrectangle{\pgfqpoint{0.562500in}{0.275000in}}{\pgfqpoint{3.487500in}{1.925000in}}%
\pgfusepath{clip}%
\pgfsetbuttcap%
\pgfsetroundjoin%
\definecolor{currentfill}{rgb}{0.000000,0.000000,0.000000}%
\pgfsetfillcolor{currentfill}%
\pgfsetlinewidth{1.003750pt}%
\definecolor{currentstroke}{rgb}{0.000000,0.000000,0.000000}%
\pgfsetstrokecolor{currentstroke}%
\pgfsetdash{}{0pt}%
\pgfpathmoveto{\pgfqpoint{2.824734in}{0.356667in}}%
\pgfpathcurveto{\pgfqpoint{2.830260in}{0.356667in}}{\pgfqpoint{2.835559in}{0.358862in}}{\pgfqpoint{2.839466in}{0.362769in}}%
\pgfpathcurveto{\pgfqpoint{2.843373in}{0.366675in}}{\pgfqpoint{2.845568in}{0.371975in}}{\pgfqpoint{2.845568in}{0.377500in}}%
\pgfpathcurveto{\pgfqpoint{2.845568in}{0.383025in}}{\pgfqpoint{2.843373in}{0.388325in}}{\pgfqpoint{2.839466in}{0.392231in}}%
\pgfpathcurveto{\pgfqpoint{2.835559in}{0.396138in}}{\pgfqpoint{2.830260in}{0.398333in}}{\pgfqpoint{2.824734in}{0.398333in}}%
\pgfpathcurveto{\pgfqpoint{2.819209in}{0.398333in}}{\pgfqpoint{2.813910in}{0.396138in}}{\pgfqpoint{2.810003in}{0.392231in}}%
\pgfpathcurveto{\pgfqpoint{2.806096in}{0.388325in}}{\pgfqpoint{2.803901in}{0.383025in}}{\pgfqpoint{2.803901in}{0.377500in}}%
\pgfpathcurveto{\pgfqpoint{2.803901in}{0.371975in}}{\pgfqpoint{2.806096in}{0.366675in}}{\pgfqpoint{2.810003in}{0.362769in}}%
\pgfpathcurveto{\pgfqpoint{2.813910in}{0.358862in}}{\pgfqpoint{2.819209in}{0.356667in}}{\pgfqpoint{2.824734in}{0.356667in}}%
\pgfpathclose%
\pgfusepath{stroke,fill}%
\end{pgfscope}%
\begin{pgfscope}%
\pgfpathrectangle{\pgfqpoint{0.562500in}{0.275000in}}{\pgfqpoint{3.487500in}{1.925000in}}%
\pgfusepath{clip}%
\pgfsetbuttcap%
\pgfsetroundjoin%
\definecolor{currentfill}{rgb}{0.000000,0.000000,0.000000}%
\pgfsetfillcolor{currentfill}%
\pgfsetlinewidth{1.003750pt}%
\definecolor{currentstroke}{rgb}{0.000000,0.000000,0.000000}%
\pgfsetstrokecolor{currentstroke}%
\pgfsetdash{}{0pt}%
\pgfpathmoveto{\pgfqpoint{2.824734in}{0.356667in}}%
\pgfpathcurveto{\pgfqpoint{2.830260in}{0.356667in}}{\pgfqpoint{2.835559in}{0.358862in}}{\pgfqpoint{2.839466in}{0.362769in}}%
\pgfpathcurveto{\pgfqpoint{2.843373in}{0.366675in}}{\pgfqpoint{2.845568in}{0.371975in}}{\pgfqpoint{2.845568in}{0.377500in}}%
\pgfpathcurveto{\pgfqpoint{2.845568in}{0.383025in}}{\pgfqpoint{2.843373in}{0.388325in}}{\pgfqpoint{2.839466in}{0.392231in}}%
\pgfpathcurveto{\pgfqpoint{2.835559in}{0.396138in}}{\pgfqpoint{2.830260in}{0.398333in}}{\pgfqpoint{2.824734in}{0.398333in}}%
\pgfpathcurveto{\pgfqpoint{2.819209in}{0.398333in}}{\pgfqpoint{2.813910in}{0.396138in}}{\pgfqpoint{2.810003in}{0.392231in}}%
\pgfpathcurveto{\pgfqpoint{2.806096in}{0.388325in}}{\pgfqpoint{2.803901in}{0.383025in}}{\pgfqpoint{2.803901in}{0.377500in}}%
\pgfpathcurveto{\pgfqpoint{2.803901in}{0.371975in}}{\pgfqpoint{2.806096in}{0.366675in}}{\pgfqpoint{2.810003in}{0.362769in}}%
\pgfpathcurveto{\pgfqpoint{2.813910in}{0.358862in}}{\pgfqpoint{2.819209in}{0.356667in}}{\pgfqpoint{2.824734in}{0.356667in}}%
\pgfpathclose%
\pgfusepath{stroke,fill}%
\end{pgfscope}%
\begin{pgfscope}%
\pgfpathrectangle{\pgfqpoint{0.562500in}{0.275000in}}{\pgfqpoint{3.487500in}{1.925000in}}%
\pgfusepath{clip}%
\pgfsetbuttcap%
\pgfsetroundjoin%
\definecolor{currentfill}{rgb}{0.000000,0.000000,0.000000}%
\pgfsetfillcolor{currentfill}%
\pgfsetlinewidth{1.003750pt}%
\definecolor{currentstroke}{rgb}{0.000000,0.000000,0.000000}%
\pgfsetstrokecolor{currentstroke}%
\pgfsetdash{}{0pt}%
\pgfpathmoveto{\pgfqpoint{2.824734in}{0.356667in}}%
\pgfpathcurveto{\pgfqpoint{2.830260in}{0.356667in}}{\pgfqpoint{2.835559in}{0.358862in}}{\pgfqpoint{2.839466in}{0.362769in}}%
\pgfpathcurveto{\pgfqpoint{2.843373in}{0.366675in}}{\pgfqpoint{2.845568in}{0.371975in}}{\pgfqpoint{2.845568in}{0.377500in}}%
\pgfpathcurveto{\pgfqpoint{2.845568in}{0.383025in}}{\pgfqpoint{2.843373in}{0.388325in}}{\pgfqpoint{2.839466in}{0.392231in}}%
\pgfpathcurveto{\pgfqpoint{2.835559in}{0.396138in}}{\pgfqpoint{2.830260in}{0.398333in}}{\pgfqpoint{2.824734in}{0.398333in}}%
\pgfpathcurveto{\pgfqpoint{2.819209in}{0.398333in}}{\pgfqpoint{2.813910in}{0.396138in}}{\pgfqpoint{2.810003in}{0.392231in}}%
\pgfpathcurveto{\pgfqpoint{2.806096in}{0.388325in}}{\pgfqpoint{2.803901in}{0.383025in}}{\pgfqpoint{2.803901in}{0.377500in}}%
\pgfpathcurveto{\pgfqpoint{2.803901in}{0.371975in}}{\pgfqpoint{2.806096in}{0.366675in}}{\pgfqpoint{2.810003in}{0.362769in}}%
\pgfpathcurveto{\pgfqpoint{2.813910in}{0.358862in}}{\pgfqpoint{2.819209in}{0.356667in}}{\pgfqpoint{2.824734in}{0.356667in}}%
\pgfpathclose%
\pgfusepath{stroke,fill}%
\end{pgfscope}%
\begin{pgfscope}%
\pgfpathrectangle{\pgfqpoint{0.562500in}{0.275000in}}{\pgfqpoint{3.487500in}{1.925000in}}%
\pgfusepath{clip}%
\pgfsetbuttcap%
\pgfsetroundjoin%
\definecolor{currentfill}{rgb}{0.000000,0.000000,0.000000}%
\pgfsetfillcolor{currentfill}%
\pgfsetlinewidth{1.003750pt}%
\definecolor{currentstroke}{rgb}{0.000000,0.000000,0.000000}%
\pgfsetstrokecolor{currentstroke}%
\pgfsetdash{}{0pt}%
\pgfpathmoveto{\pgfqpoint{2.824734in}{0.356667in}}%
\pgfpathcurveto{\pgfqpoint{2.830260in}{0.356667in}}{\pgfqpoint{2.835559in}{0.358862in}}{\pgfqpoint{2.839466in}{0.362769in}}%
\pgfpathcurveto{\pgfqpoint{2.843373in}{0.366675in}}{\pgfqpoint{2.845568in}{0.371975in}}{\pgfqpoint{2.845568in}{0.377500in}}%
\pgfpathcurveto{\pgfqpoint{2.845568in}{0.383025in}}{\pgfqpoint{2.843373in}{0.388325in}}{\pgfqpoint{2.839466in}{0.392231in}}%
\pgfpathcurveto{\pgfqpoint{2.835559in}{0.396138in}}{\pgfqpoint{2.830260in}{0.398333in}}{\pgfqpoint{2.824734in}{0.398333in}}%
\pgfpathcurveto{\pgfqpoint{2.819209in}{0.398333in}}{\pgfqpoint{2.813910in}{0.396138in}}{\pgfqpoint{2.810003in}{0.392231in}}%
\pgfpathcurveto{\pgfqpoint{2.806096in}{0.388325in}}{\pgfqpoint{2.803901in}{0.383025in}}{\pgfqpoint{2.803901in}{0.377500in}}%
\pgfpathcurveto{\pgfqpoint{2.803901in}{0.371975in}}{\pgfqpoint{2.806096in}{0.366675in}}{\pgfqpoint{2.810003in}{0.362769in}}%
\pgfpathcurveto{\pgfqpoint{2.813910in}{0.358862in}}{\pgfqpoint{2.819209in}{0.356667in}}{\pgfqpoint{2.824734in}{0.356667in}}%
\pgfpathclose%
\pgfusepath{stroke,fill}%
\end{pgfscope}%
\begin{pgfscope}%
\pgfpathrectangle{\pgfqpoint{0.562500in}{0.275000in}}{\pgfqpoint{3.487500in}{1.925000in}}%
\pgfusepath{clip}%
\pgfsetbuttcap%
\pgfsetroundjoin%
\definecolor{currentfill}{rgb}{0.000000,0.000000,0.000000}%
\pgfsetfillcolor{currentfill}%
\pgfsetlinewidth{1.003750pt}%
\definecolor{currentstroke}{rgb}{0.000000,0.000000,0.000000}%
\pgfsetstrokecolor{currentstroke}%
\pgfsetdash{}{0pt}%
\pgfpathmoveto{\pgfqpoint{2.824734in}{0.356667in}}%
\pgfpathcurveto{\pgfqpoint{2.830260in}{0.356667in}}{\pgfqpoint{2.835559in}{0.358862in}}{\pgfqpoint{2.839466in}{0.362769in}}%
\pgfpathcurveto{\pgfqpoint{2.843373in}{0.366675in}}{\pgfqpoint{2.845568in}{0.371975in}}{\pgfqpoint{2.845568in}{0.377500in}}%
\pgfpathcurveto{\pgfqpoint{2.845568in}{0.383025in}}{\pgfqpoint{2.843373in}{0.388325in}}{\pgfqpoint{2.839466in}{0.392231in}}%
\pgfpathcurveto{\pgfqpoint{2.835559in}{0.396138in}}{\pgfqpoint{2.830260in}{0.398333in}}{\pgfqpoint{2.824734in}{0.398333in}}%
\pgfpathcurveto{\pgfqpoint{2.819209in}{0.398333in}}{\pgfqpoint{2.813910in}{0.396138in}}{\pgfqpoint{2.810003in}{0.392231in}}%
\pgfpathcurveto{\pgfqpoint{2.806096in}{0.388325in}}{\pgfqpoint{2.803901in}{0.383025in}}{\pgfqpoint{2.803901in}{0.377500in}}%
\pgfpathcurveto{\pgfqpoint{2.803901in}{0.371975in}}{\pgfqpoint{2.806096in}{0.366675in}}{\pgfqpoint{2.810003in}{0.362769in}}%
\pgfpathcurveto{\pgfqpoint{2.813910in}{0.358862in}}{\pgfqpoint{2.819209in}{0.356667in}}{\pgfqpoint{2.824734in}{0.356667in}}%
\pgfpathclose%
\pgfusepath{stroke,fill}%
\end{pgfscope}%
\begin{pgfscope}%
\pgfpathrectangle{\pgfqpoint{0.562500in}{0.275000in}}{\pgfqpoint{3.487500in}{1.925000in}}%
\pgfusepath{clip}%
\pgfsetbuttcap%
\pgfsetroundjoin%
\definecolor{currentfill}{rgb}{0.000000,0.000000,0.000000}%
\pgfsetfillcolor{currentfill}%
\pgfsetlinewidth{1.003750pt}%
\definecolor{currentstroke}{rgb}{0.000000,0.000000,0.000000}%
\pgfsetstrokecolor{currentstroke}%
\pgfsetdash{}{0pt}%
\pgfpathmoveto{\pgfqpoint{2.824734in}{0.356667in}}%
\pgfpathcurveto{\pgfqpoint{2.830260in}{0.356667in}}{\pgfqpoint{2.835559in}{0.358862in}}{\pgfqpoint{2.839466in}{0.362769in}}%
\pgfpathcurveto{\pgfqpoint{2.843373in}{0.366675in}}{\pgfqpoint{2.845568in}{0.371975in}}{\pgfqpoint{2.845568in}{0.377500in}}%
\pgfpathcurveto{\pgfqpoint{2.845568in}{0.383025in}}{\pgfqpoint{2.843373in}{0.388325in}}{\pgfqpoint{2.839466in}{0.392231in}}%
\pgfpathcurveto{\pgfqpoint{2.835559in}{0.396138in}}{\pgfqpoint{2.830260in}{0.398333in}}{\pgfqpoint{2.824734in}{0.398333in}}%
\pgfpathcurveto{\pgfqpoint{2.819209in}{0.398333in}}{\pgfqpoint{2.813910in}{0.396138in}}{\pgfqpoint{2.810003in}{0.392231in}}%
\pgfpathcurveto{\pgfqpoint{2.806096in}{0.388325in}}{\pgfqpoint{2.803901in}{0.383025in}}{\pgfqpoint{2.803901in}{0.377500in}}%
\pgfpathcurveto{\pgfqpoint{2.803901in}{0.371975in}}{\pgfqpoint{2.806096in}{0.366675in}}{\pgfqpoint{2.810003in}{0.362769in}}%
\pgfpathcurveto{\pgfqpoint{2.813910in}{0.358862in}}{\pgfqpoint{2.819209in}{0.356667in}}{\pgfqpoint{2.824734in}{0.356667in}}%
\pgfpathclose%
\pgfusepath{stroke,fill}%
\end{pgfscope}%
\begin{pgfscope}%
\pgfpathrectangle{\pgfqpoint{0.562500in}{0.275000in}}{\pgfqpoint{3.487500in}{1.925000in}}%
\pgfusepath{clip}%
\pgfsetbuttcap%
\pgfsetroundjoin%
\definecolor{currentfill}{rgb}{0.000000,0.000000,0.000000}%
\pgfsetfillcolor{currentfill}%
\pgfsetlinewidth{1.003750pt}%
\definecolor{currentstroke}{rgb}{0.000000,0.000000,0.000000}%
\pgfsetstrokecolor{currentstroke}%
\pgfsetdash{}{0pt}%
\pgfpathmoveto{\pgfqpoint{2.824734in}{0.356667in}}%
\pgfpathcurveto{\pgfqpoint{2.830260in}{0.356667in}}{\pgfqpoint{2.835559in}{0.358862in}}{\pgfqpoint{2.839466in}{0.362769in}}%
\pgfpathcurveto{\pgfqpoint{2.843373in}{0.366675in}}{\pgfqpoint{2.845568in}{0.371975in}}{\pgfqpoint{2.845568in}{0.377500in}}%
\pgfpathcurveto{\pgfqpoint{2.845568in}{0.383025in}}{\pgfqpoint{2.843373in}{0.388325in}}{\pgfqpoint{2.839466in}{0.392231in}}%
\pgfpathcurveto{\pgfqpoint{2.835559in}{0.396138in}}{\pgfqpoint{2.830260in}{0.398333in}}{\pgfqpoint{2.824734in}{0.398333in}}%
\pgfpathcurveto{\pgfqpoint{2.819209in}{0.398333in}}{\pgfqpoint{2.813910in}{0.396138in}}{\pgfqpoint{2.810003in}{0.392231in}}%
\pgfpathcurveto{\pgfqpoint{2.806096in}{0.388325in}}{\pgfqpoint{2.803901in}{0.383025in}}{\pgfqpoint{2.803901in}{0.377500in}}%
\pgfpathcurveto{\pgfqpoint{2.803901in}{0.371975in}}{\pgfqpoint{2.806096in}{0.366675in}}{\pgfqpoint{2.810003in}{0.362769in}}%
\pgfpathcurveto{\pgfqpoint{2.813910in}{0.358862in}}{\pgfqpoint{2.819209in}{0.356667in}}{\pgfqpoint{2.824734in}{0.356667in}}%
\pgfpathclose%
\pgfusepath{stroke,fill}%
\end{pgfscope}%
\begin{pgfscope}%
\pgfpathrectangle{\pgfqpoint{0.562500in}{0.275000in}}{\pgfqpoint{3.487500in}{1.925000in}}%
\pgfusepath{clip}%
\pgfsetbuttcap%
\pgfsetroundjoin%
\definecolor{currentfill}{rgb}{0.000000,0.000000,0.000000}%
\pgfsetfillcolor{currentfill}%
\pgfsetlinewidth{1.003750pt}%
\definecolor{currentstroke}{rgb}{0.000000,0.000000,0.000000}%
\pgfsetstrokecolor{currentstroke}%
\pgfsetdash{}{0pt}%
\pgfpathmoveto{\pgfqpoint{2.824734in}{0.356667in}}%
\pgfpathcurveto{\pgfqpoint{2.830260in}{0.356667in}}{\pgfqpoint{2.835559in}{0.358862in}}{\pgfqpoint{2.839466in}{0.362769in}}%
\pgfpathcurveto{\pgfqpoint{2.843373in}{0.366675in}}{\pgfqpoint{2.845568in}{0.371975in}}{\pgfqpoint{2.845568in}{0.377500in}}%
\pgfpathcurveto{\pgfqpoint{2.845568in}{0.383025in}}{\pgfqpoint{2.843373in}{0.388325in}}{\pgfqpoint{2.839466in}{0.392231in}}%
\pgfpathcurveto{\pgfqpoint{2.835559in}{0.396138in}}{\pgfqpoint{2.830260in}{0.398333in}}{\pgfqpoint{2.824734in}{0.398333in}}%
\pgfpathcurveto{\pgfqpoint{2.819209in}{0.398333in}}{\pgfqpoint{2.813910in}{0.396138in}}{\pgfqpoint{2.810003in}{0.392231in}}%
\pgfpathcurveto{\pgfqpoint{2.806096in}{0.388325in}}{\pgfqpoint{2.803901in}{0.383025in}}{\pgfqpoint{2.803901in}{0.377500in}}%
\pgfpathcurveto{\pgfqpoint{2.803901in}{0.371975in}}{\pgfqpoint{2.806096in}{0.366675in}}{\pgfqpoint{2.810003in}{0.362769in}}%
\pgfpathcurveto{\pgfqpoint{2.813910in}{0.358862in}}{\pgfqpoint{2.819209in}{0.356667in}}{\pgfqpoint{2.824734in}{0.356667in}}%
\pgfpathclose%
\pgfusepath{stroke,fill}%
\end{pgfscope}%
\begin{pgfscope}%
\pgfpathrectangle{\pgfqpoint{0.562500in}{0.275000in}}{\pgfqpoint{3.487500in}{1.925000in}}%
\pgfusepath{clip}%
\pgfsetbuttcap%
\pgfsetroundjoin%
\definecolor{currentfill}{rgb}{0.000000,0.000000,0.000000}%
\pgfsetfillcolor{currentfill}%
\pgfsetlinewidth{1.003750pt}%
\definecolor{currentstroke}{rgb}{0.000000,0.000000,0.000000}%
\pgfsetstrokecolor{currentstroke}%
\pgfsetdash{}{0pt}%
\pgfpathmoveto{\pgfqpoint{2.824734in}{0.356667in}}%
\pgfpathcurveto{\pgfqpoint{2.830260in}{0.356667in}}{\pgfqpoint{2.835559in}{0.358862in}}{\pgfqpoint{2.839466in}{0.362769in}}%
\pgfpathcurveto{\pgfqpoint{2.843373in}{0.366675in}}{\pgfqpoint{2.845568in}{0.371975in}}{\pgfqpoint{2.845568in}{0.377500in}}%
\pgfpathcurveto{\pgfqpoint{2.845568in}{0.383025in}}{\pgfqpoint{2.843373in}{0.388325in}}{\pgfqpoint{2.839466in}{0.392231in}}%
\pgfpathcurveto{\pgfqpoint{2.835559in}{0.396138in}}{\pgfqpoint{2.830260in}{0.398333in}}{\pgfqpoint{2.824734in}{0.398333in}}%
\pgfpathcurveto{\pgfqpoint{2.819209in}{0.398333in}}{\pgfqpoint{2.813910in}{0.396138in}}{\pgfqpoint{2.810003in}{0.392231in}}%
\pgfpathcurveto{\pgfqpoint{2.806096in}{0.388325in}}{\pgfqpoint{2.803901in}{0.383025in}}{\pgfqpoint{2.803901in}{0.377500in}}%
\pgfpathcurveto{\pgfqpoint{2.803901in}{0.371975in}}{\pgfqpoint{2.806096in}{0.366675in}}{\pgfqpoint{2.810003in}{0.362769in}}%
\pgfpathcurveto{\pgfqpoint{2.813910in}{0.358862in}}{\pgfqpoint{2.819209in}{0.356667in}}{\pgfqpoint{2.824734in}{0.356667in}}%
\pgfpathclose%
\pgfusepath{stroke,fill}%
\end{pgfscope}%
\begin{pgfscope}%
\pgfpathrectangle{\pgfqpoint{0.562500in}{0.275000in}}{\pgfqpoint{3.487500in}{1.925000in}}%
\pgfusepath{clip}%
\pgfsetbuttcap%
\pgfsetroundjoin%
\definecolor{currentfill}{rgb}{0.000000,0.000000,0.000000}%
\pgfsetfillcolor{currentfill}%
\pgfsetlinewidth{1.003750pt}%
\definecolor{currentstroke}{rgb}{0.000000,0.000000,0.000000}%
\pgfsetstrokecolor{currentstroke}%
\pgfsetdash{}{0pt}%
\pgfpathmoveto{\pgfqpoint{2.824734in}{0.356667in}}%
\pgfpathcurveto{\pgfqpoint{2.830260in}{0.356667in}}{\pgfqpoint{2.835559in}{0.358862in}}{\pgfqpoint{2.839466in}{0.362769in}}%
\pgfpathcurveto{\pgfqpoint{2.843373in}{0.366675in}}{\pgfqpoint{2.845568in}{0.371975in}}{\pgfqpoint{2.845568in}{0.377500in}}%
\pgfpathcurveto{\pgfqpoint{2.845568in}{0.383025in}}{\pgfqpoint{2.843373in}{0.388325in}}{\pgfqpoint{2.839466in}{0.392231in}}%
\pgfpathcurveto{\pgfqpoint{2.835559in}{0.396138in}}{\pgfqpoint{2.830260in}{0.398333in}}{\pgfqpoint{2.824734in}{0.398333in}}%
\pgfpathcurveto{\pgfqpoint{2.819209in}{0.398333in}}{\pgfqpoint{2.813910in}{0.396138in}}{\pgfqpoint{2.810003in}{0.392231in}}%
\pgfpathcurveto{\pgfqpoint{2.806096in}{0.388325in}}{\pgfqpoint{2.803901in}{0.383025in}}{\pgfqpoint{2.803901in}{0.377500in}}%
\pgfpathcurveto{\pgfqpoint{2.803901in}{0.371975in}}{\pgfqpoint{2.806096in}{0.366675in}}{\pgfqpoint{2.810003in}{0.362769in}}%
\pgfpathcurveto{\pgfqpoint{2.813910in}{0.358862in}}{\pgfqpoint{2.819209in}{0.356667in}}{\pgfqpoint{2.824734in}{0.356667in}}%
\pgfpathclose%
\pgfusepath{stroke,fill}%
\end{pgfscope}%
\begin{pgfscope}%
\pgfpathrectangle{\pgfqpoint{0.562500in}{0.275000in}}{\pgfqpoint{3.487500in}{1.925000in}}%
\pgfusepath{clip}%
\pgfsetbuttcap%
\pgfsetroundjoin%
\definecolor{currentfill}{rgb}{0.000000,0.000000,0.000000}%
\pgfsetfillcolor{currentfill}%
\pgfsetlinewidth{1.003750pt}%
\definecolor{currentstroke}{rgb}{0.000000,0.000000,0.000000}%
\pgfsetstrokecolor{currentstroke}%
\pgfsetdash{}{0pt}%
\pgfpathmoveto{\pgfqpoint{2.824734in}{0.356667in}}%
\pgfpathcurveto{\pgfqpoint{2.830260in}{0.356667in}}{\pgfqpoint{2.835559in}{0.358862in}}{\pgfqpoint{2.839466in}{0.362769in}}%
\pgfpathcurveto{\pgfqpoint{2.843373in}{0.366675in}}{\pgfqpoint{2.845568in}{0.371975in}}{\pgfqpoint{2.845568in}{0.377500in}}%
\pgfpathcurveto{\pgfqpoint{2.845568in}{0.383025in}}{\pgfqpoint{2.843373in}{0.388325in}}{\pgfqpoint{2.839466in}{0.392231in}}%
\pgfpathcurveto{\pgfqpoint{2.835559in}{0.396138in}}{\pgfqpoint{2.830260in}{0.398333in}}{\pgfqpoint{2.824734in}{0.398333in}}%
\pgfpathcurveto{\pgfqpoint{2.819209in}{0.398333in}}{\pgfqpoint{2.813910in}{0.396138in}}{\pgfqpoint{2.810003in}{0.392231in}}%
\pgfpathcurveto{\pgfqpoint{2.806096in}{0.388325in}}{\pgfqpoint{2.803901in}{0.383025in}}{\pgfqpoint{2.803901in}{0.377500in}}%
\pgfpathcurveto{\pgfqpoint{2.803901in}{0.371975in}}{\pgfqpoint{2.806096in}{0.366675in}}{\pgfqpoint{2.810003in}{0.362769in}}%
\pgfpathcurveto{\pgfqpoint{2.813910in}{0.358862in}}{\pgfqpoint{2.819209in}{0.356667in}}{\pgfqpoint{2.824734in}{0.356667in}}%
\pgfpathclose%
\pgfusepath{stroke,fill}%
\end{pgfscope}%
\begin{pgfscope}%
\pgfpathrectangle{\pgfqpoint{0.562500in}{0.275000in}}{\pgfqpoint{3.487500in}{1.925000in}}%
\pgfusepath{clip}%
\pgfsetbuttcap%
\pgfsetroundjoin%
\definecolor{currentfill}{rgb}{0.000000,0.000000,0.000000}%
\pgfsetfillcolor{currentfill}%
\pgfsetlinewidth{1.003750pt}%
\definecolor{currentstroke}{rgb}{0.000000,0.000000,0.000000}%
\pgfsetstrokecolor{currentstroke}%
\pgfsetdash{}{0pt}%
\pgfpathmoveto{\pgfqpoint{2.824734in}{0.356667in}}%
\pgfpathcurveto{\pgfqpoint{2.830260in}{0.356667in}}{\pgfqpoint{2.835559in}{0.358862in}}{\pgfqpoint{2.839466in}{0.362769in}}%
\pgfpathcurveto{\pgfqpoint{2.843373in}{0.366675in}}{\pgfqpoint{2.845568in}{0.371975in}}{\pgfqpoint{2.845568in}{0.377500in}}%
\pgfpathcurveto{\pgfqpoint{2.845568in}{0.383025in}}{\pgfqpoint{2.843373in}{0.388325in}}{\pgfqpoint{2.839466in}{0.392231in}}%
\pgfpathcurveto{\pgfqpoint{2.835559in}{0.396138in}}{\pgfqpoint{2.830260in}{0.398333in}}{\pgfqpoint{2.824734in}{0.398333in}}%
\pgfpathcurveto{\pgfqpoint{2.819209in}{0.398333in}}{\pgfqpoint{2.813910in}{0.396138in}}{\pgfqpoint{2.810003in}{0.392231in}}%
\pgfpathcurveto{\pgfqpoint{2.806096in}{0.388325in}}{\pgfqpoint{2.803901in}{0.383025in}}{\pgfqpoint{2.803901in}{0.377500in}}%
\pgfpathcurveto{\pgfqpoint{2.803901in}{0.371975in}}{\pgfqpoint{2.806096in}{0.366675in}}{\pgfqpoint{2.810003in}{0.362769in}}%
\pgfpathcurveto{\pgfqpoint{2.813910in}{0.358862in}}{\pgfqpoint{2.819209in}{0.356667in}}{\pgfqpoint{2.824734in}{0.356667in}}%
\pgfpathclose%
\pgfusepath{stroke,fill}%
\end{pgfscope}%
\begin{pgfscope}%
\pgfpathrectangle{\pgfqpoint{0.562500in}{0.275000in}}{\pgfqpoint{3.487500in}{1.925000in}}%
\pgfusepath{clip}%
\pgfsetbuttcap%
\pgfsetroundjoin%
\definecolor{currentfill}{rgb}{0.000000,0.000000,0.000000}%
\pgfsetfillcolor{currentfill}%
\pgfsetlinewidth{1.003750pt}%
\definecolor{currentstroke}{rgb}{0.000000,0.000000,0.000000}%
\pgfsetstrokecolor{currentstroke}%
\pgfsetdash{}{0pt}%
\pgfpathmoveto{\pgfqpoint{2.824734in}{0.356667in}}%
\pgfpathcurveto{\pgfqpoint{2.830260in}{0.356667in}}{\pgfqpoint{2.835559in}{0.358862in}}{\pgfqpoint{2.839466in}{0.362769in}}%
\pgfpathcurveto{\pgfqpoint{2.843373in}{0.366675in}}{\pgfqpoint{2.845568in}{0.371975in}}{\pgfqpoint{2.845568in}{0.377500in}}%
\pgfpathcurveto{\pgfqpoint{2.845568in}{0.383025in}}{\pgfqpoint{2.843373in}{0.388325in}}{\pgfqpoint{2.839466in}{0.392231in}}%
\pgfpathcurveto{\pgfqpoint{2.835559in}{0.396138in}}{\pgfqpoint{2.830260in}{0.398333in}}{\pgfqpoint{2.824734in}{0.398333in}}%
\pgfpathcurveto{\pgfqpoint{2.819209in}{0.398333in}}{\pgfqpoint{2.813910in}{0.396138in}}{\pgfqpoint{2.810003in}{0.392231in}}%
\pgfpathcurveto{\pgfqpoint{2.806096in}{0.388325in}}{\pgfqpoint{2.803901in}{0.383025in}}{\pgfqpoint{2.803901in}{0.377500in}}%
\pgfpathcurveto{\pgfqpoint{2.803901in}{0.371975in}}{\pgfqpoint{2.806096in}{0.366675in}}{\pgfqpoint{2.810003in}{0.362769in}}%
\pgfpathcurveto{\pgfqpoint{2.813910in}{0.358862in}}{\pgfqpoint{2.819209in}{0.356667in}}{\pgfqpoint{2.824734in}{0.356667in}}%
\pgfpathclose%
\pgfusepath{stroke,fill}%
\end{pgfscope}%
\begin{pgfscope}%
\pgfpathrectangle{\pgfqpoint{0.562500in}{0.275000in}}{\pgfqpoint{3.487500in}{1.925000in}}%
\pgfusepath{clip}%
\pgfsetbuttcap%
\pgfsetroundjoin%
\definecolor{currentfill}{rgb}{0.000000,0.000000,0.000000}%
\pgfsetfillcolor{currentfill}%
\pgfsetlinewidth{1.003750pt}%
\definecolor{currentstroke}{rgb}{0.000000,0.000000,0.000000}%
\pgfsetstrokecolor{currentstroke}%
\pgfsetdash{}{0pt}%
\pgfpathmoveto{\pgfqpoint{2.824734in}{0.356667in}}%
\pgfpathcurveto{\pgfqpoint{2.830260in}{0.356667in}}{\pgfqpoint{2.835559in}{0.358862in}}{\pgfqpoint{2.839466in}{0.362769in}}%
\pgfpathcurveto{\pgfqpoint{2.843373in}{0.366675in}}{\pgfqpoint{2.845568in}{0.371975in}}{\pgfqpoint{2.845568in}{0.377500in}}%
\pgfpathcurveto{\pgfqpoint{2.845568in}{0.383025in}}{\pgfqpoint{2.843373in}{0.388325in}}{\pgfqpoint{2.839466in}{0.392231in}}%
\pgfpathcurveto{\pgfqpoint{2.835559in}{0.396138in}}{\pgfqpoint{2.830260in}{0.398333in}}{\pgfqpoint{2.824734in}{0.398333in}}%
\pgfpathcurveto{\pgfqpoint{2.819209in}{0.398333in}}{\pgfqpoint{2.813910in}{0.396138in}}{\pgfqpoint{2.810003in}{0.392231in}}%
\pgfpathcurveto{\pgfqpoint{2.806096in}{0.388325in}}{\pgfqpoint{2.803901in}{0.383025in}}{\pgfqpoint{2.803901in}{0.377500in}}%
\pgfpathcurveto{\pgfqpoint{2.803901in}{0.371975in}}{\pgfqpoint{2.806096in}{0.366675in}}{\pgfqpoint{2.810003in}{0.362769in}}%
\pgfpathcurveto{\pgfqpoint{2.813910in}{0.358862in}}{\pgfqpoint{2.819209in}{0.356667in}}{\pgfqpoint{2.824734in}{0.356667in}}%
\pgfpathclose%
\pgfusepath{stroke,fill}%
\end{pgfscope}%
\begin{pgfscope}%
\pgfpathrectangle{\pgfqpoint{0.562500in}{0.275000in}}{\pgfqpoint{3.487500in}{1.925000in}}%
\pgfusepath{clip}%
\pgfsetbuttcap%
\pgfsetroundjoin%
\definecolor{currentfill}{rgb}{0.000000,0.000000,0.000000}%
\pgfsetfillcolor{currentfill}%
\pgfsetlinewidth{1.003750pt}%
\definecolor{currentstroke}{rgb}{0.000000,0.000000,0.000000}%
\pgfsetstrokecolor{currentstroke}%
\pgfsetdash{}{0pt}%
\pgfpathmoveto{\pgfqpoint{2.824734in}{0.356667in}}%
\pgfpathcurveto{\pgfqpoint{2.830260in}{0.356667in}}{\pgfqpoint{2.835559in}{0.358862in}}{\pgfqpoint{2.839466in}{0.362769in}}%
\pgfpathcurveto{\pgfqpoint{2.843373in}{0.366675in}}{\pgfqpoint{2.845568in}{0.371975in}}{\pgfqpoint{2.845568in}{0.377500in}}%
\pgfpathcurveto{\pgfqpoint{2.845568in}{0.383025in}}{\pgfqpoint{2.843373in}{0.388325in}}{\pgfqpoint{2.839466in}{0.392231in}}%
\pgfpathcurveto{\pgfqpoint{2.835559in}{0.396138in}}{\pgfqpoint{2.830260in}{0.398333in}}{\pgfqpoint{2.824734in}{0.398333in}}%
\pgfpathcurveto{\pgfqpoint{2.819209in}{0.398333in}}{\pgfqpoint{2.813910in}{0.396138in}}{\pgfqpoint{2.810003in}{0.392231in}}%
\pgfpathcurveto{\pgfqpoint{2.806096in}{0.388325in}}{\pgfqpoint{2.803901in}{0.383025in}}{\pgfqpoint{2.803901in}{0.377500in}}%
\pgfpathcurveto{\pgfqpoint{2.803901in}{0.371975in}}{\pgfqpoint{2.806096in}{0.366675in}}{\pgfqpoint{2.810003in}{0.362769in}}%
\pgfpathcurveto{\pgfqpoint{2.813910in}{0.358862in}}{\pgfqpoint{2.819209in}{0.356667in}}{\pgfqpoint{2.824734in}{0.356667in}}%
\pgfpathclose%
\pgfusepath{stroke,fill}%
\end{pgfscope}%
\begin{pgfscope}%
\pgfpathrectangle{\pgfqpoint{0.562500in}{0.275000in}}{\pgfqpoint{3.487500in}{1.925000in}}%
\pgfusepath{clip}%
\pgfsetbuttcap%
\pgfsetroundjoin%
\definecolor{currentfill}{rgb}{0.000000,0.000000,0.000000}%
\pgfsetfillcolor{currentfill}%
\pgfsetlinewidth{1.003750pt}%
\definecolor{currentstroke}{rgb}{0.000000,0.000000,0.000000}%
\pgfsetstrokecolor{currentstroke}%
\pgfsetdash{}{0pt}%
\pgfpathmoveto{\pgfqpoint{2.824734in}{0.356667in}}%
\pgfpathcurveto{\pgfqpoint{2.830260in}{0.356667in}}{\pgfqpoint{2.835559in}{0.358862in}}{\pgfqpoint{2.839466in}{0.362769in}}%
\pgfpathcurveto{\pgfqpoint{2.843373in}{0.366675in}}{\pgfqpoint{2.845568in}{0.371975in}}{\pgfqpoint{2.845568in}{0.377500in}}%
\pgfpathcurveto{\pgfqpoint{2.845568in}{0.383025in}}{\pgfqpoint{2.843373in}{0.388325in}}{\pgfqpoint{2.839466in}{0.392231in}}%
\pgfpathcurveto{\pgfqpoint{2.835559in}{0.396138in}}{\pgfqpoint{2.830260in}{0.398333in}}{\pgfqpoint{2.824734in}{0.398333in}}%
\pgfpathcurveto{\pgfqpoint{2.819209in}{0.398333in}}{\pgfqpoint{2.813910in}{0.396138in}}{\pgfqpoint{2.810003in}{0.392231in}}%
\pgfpathcurveto{\pgfqpoint{2.806096in}{0.388325in}}{\pgfqpoint{2.803901in}{0.383025in}}{\pgfqpoint{2.803901in}{0.377500in}}%
\pgfpathcurveto{\pgfqpoint{2.803901in}{0.371975in}}{\pgfqpoint{2.806096in}{0.366675in}}{\pgfqpoint{2.810003in}{0.362769in}}%
\pgfpathcurveto{\pgfqpoint{2.813910in}{0.358862in}}{\pgfqpoint{2.819209in}{0.356667in}}{\pgfqpoint{2.824734in}{0.356667in}}%
\pgfpathclose%
\pgfusepath{stroke,fill}%
\end{pgfscope}%
\begin{pgfscope}%
\pgfpathrectangle{\pgfqpoint{0.562500in}{0.275000in}}{\pgfqpoint{3.487500in}{1.925000in}}%
\pgfusepath{clip}%
\pgfsetbuttcap%
\pgfsetroundjoin%
\definecolor{currentfill}{rgb}{0.000000,0.000000,0.000000}%
\pgfsetfillcolor{currentfill}%
\pgfsetlinewidth{1.003750pt}%
\definecolor{currentstroke}{rgb}{0.000000,0.000000,0.000000}%
\pgfsetstrokecolor{currentstroke}%
\pgfsetdash{}{0pt}%
\pgfpathmoveto{\pgfqpoint{2.824734in}{0.356667in}}%
\pgfpathcurveto{\pgfqpoint{2.830260in}{0.356667in}}{\pgfqpoint{2.835559in}{0.358862in}}{\pgfqpoint{2.839466in}{0.362769in}}%
\pgfpathcurveto{\pgfqpoint{2.843373in}{0.366675in}}{\pgfqpoint{2.845568in}{0.371975in}}{\pgfqpoint{2.845568in}{0.377500in}}%
\pgfpathcurveto{\pgfqpoint{2.845568in}{0.383025in}}{\pgfqpoint{2.843373in}{0.388325in}}{\pgfqpoint{2.839466in}{0.392231in}}%
\pgfpathcurveto{\pgfqpoint{2.835559in}{0.396138in}}{\pgfqpoint{2.830260in}{0.398333in}}{\pgfqpoint{2.824734in}{0.398333in}}%
\pgfpathcurveto{\pgfqpoint{2.819209in}{0.398333in}}{\pgfqpoint{2.813910in}{0.396138in}}{\pgfqpoint{2.810003in}{0.392231in}}%
\pgfpathcurveto{\pgfqpoint{2.806096in}{0.388325in}}{\pgfqpoint{2.803901in}{0.383025in}}{\pgfqpoint{2.803901in}{0.377500in}}%
\pgfpathcurveto{\pgfqpoint{2.803901in}{0.371975in}}{\pgfqpoint{2.806096in}{0.366675in}}{\pgfqpoint{2.810003in}{0.362769in}}%
\pgfpathcurveto{\pgfqpoint{2.813910in}{0.358862in}}{\pgfqpoint{2.819209in}{0.356667in}}{\pgfqpoint{2.824734in}{0.356667in}}%
\pgfpathclose%
\pgfusepath{stroke,fill}%
\end{pgfscope}%
\begin{pgfscope}%
\pgfpathrectangle{\pgfqpoint{0.562500in}{0.275000in}}{\pgfqpoint{3.487500in}{1.925000in}}%
\pgfusepath{clip}%
\pgfsetbuttcap%
\pgfsetroundjoin%
\definecolor{currentfill}{rgb}{0.000000,0.000000,0.000000}%
\pgfsetfillcolor{currentfill}%
\pgfsetlinewidth{1.003750pt}%
\definecolor{currentstroke}{rgb}{0.000000,0.000000,0.000000}%
\pgfsetstrokecolor{currentstroke}%
\pgfsetdash{}{0pt}%
\pgfpathmoveto{\pgfqpoint{2.824734in}{0.356667in}}%
\pgfpathcurveto{\pgfqpoint{2.830260in}{0.356667in}}{\pgfqpoint{2.835559in}{0.358862in}}{\pgfqpoint{2.839466in}{0.362769in}}%
\pgfpathcurveto{\pgfqpoint{2.843373in}{0.366675in}}{\pgfqpoint{2.845568in}{0.371975in}}{\pgfqpoint{2.845568in}{0.377500in}}%
\pgfpathcurveto{\pgfqpoint{2.845568in}{0.383025in}}{\pgfqpoint{2.843373in}{0.388325in}}{\pgfqpoint{2.839466in}{0.392231in}}%
\pgfpathcurveto{\pgfqpoint{2.835559in}{0.396138in}}{\pgfqpoint{2.830260in}{0.398333in}}{\pgfqpoint{2.824734in}{0.398333in}}%
\pgfpathcurveto{\pgfqpoint{2.819209in}{0.398333in}}{\pgfqpoint{2.813910in}{0.396138in}}{\pgfqpoint{2.810003in}{0.392231in}}%
\pgfpathcurveto{\pgfqpoint{2.806096in}{0.388325in}}{\pgfqpoint{2.803901in}{0.383025in}}{\pgfqpoint{2.803901in}{0.377500in}}%
\pgfpathcurveto{\pgfqpoint{2.803901in}{0.371975in}}{\pgfqpoint{2.806096in}{0.366675in}}{\pgfqpoint{2.810003in}{0.362769in}}%
\pgfpathcurveto{\pgfqpoint{2.813910in}{0.358862in}}{\pgfqpoint{2.819209in}{0.356667in}}{\pgfqpoint{2.824734in}{0.356667in}}%
\pgfpathclose%
\pgfusepath{stroke,fill}%
\end{pgfscope}%
\begin{pgfscope}%
\pgfpathrectangle{\pgfqpoint{0.562500in}{0.275000in}}{\pgfqpoint{3.487500in}{1.925000in}}%
\pgfusepath{clip}%
\pgfsetbuttcap%
\pgfsetroundjoin%
\definecolor{currentfill}{rgb}{0.000000,0.000000,0.000000}%
\pgfsetfillcolor{currentfill}%
\pgfsetlinewidth{1.003750pt}%
\definecolor{currentstroke}{rgb}{0.000000,0.000000,0.000000}%
\pgfsetstrokecolor{currentstroke}%
\pgfsetdash{}{0pt}%
\pgfpathmoveto{\pgfqpoint{2.824734in}{0.356667in}}%
\pgfpathcurveto{\pgfqpoint{2.830260in}{0.356667in}}{\pgfqpoint{2.835559in}{0.358862in}}{\pgfqpoint{2.839466in}{0.362769in}}%
\pgfpathcurveto{\pgfqpoint{2.843373in}{0.366675in}}{\pgfqpoint{2.845568in}{0.371975in}}{\pgfqpoint{2.845568in}{0.377500in}}%
\pgfpathcurveto{\pgfqpoint{2.845568in}{0.383025in}}{\pgfqpoint{2.843373in}{0.388325in}}{\pgfqpoint{2.839466in}{0.392231in}}%
\pgfpathcurveto{\pgfqpoint{2.835559in}{0.396138in}}{\pgfqpoint{2.830260in}{0.398333in}}{\pgfqpoint{2.824734in}{0.398333in}}%
\pgfpathcurveto{\pgfqpoint{2.819209in}{0.398333in}}{\pgfqpoint{2.813910in}{0.396138in}}{\pgfqpoint{2.810003in}{0.392231in}}%
\pgfpathcurveto{\pgfqpoint{2.806096in}{0.388325in}}{\pgfqpoint{2.803901in}{0.383025in}}{\pgfqpoint{2.803901in}{0.377500in}}%
\pgfpathcurveto{\pgfqpoint{2.803901in}{0.371975in}}{\pgfqpoint{2.806096in}{0.366675in}}{\pgfqpoint{2.810003in}{0.362769in}}%
\pgfpathcurveto{\pgfqpoint{2.813910in}{0.358862in}}{\pgfqpoint{2.819209in}{0.356667in}}{\pgfqpoint{2.824734in}{0.356667in}}%
\pgfpathclose%
\pgfusepath{stroke,fill}%
\end{pgfscope}%
\begin{pgfscope}%
\pgfpathrectangle{\pgfqpoint{0.562500in}{0.275000in}}{\pgfqpoint{3.487500in}{1.925000in}}%
\pgfusepath{clip}%
\pgfsetbuttcap%
\pgfsetroundjoin%
\definecolor{currentfill}{rgb}{0.000000,0.000000,0.000000}%
\pgfsetfillcolor{currentfill}%
\pgfsetlinewidth{1.003750pt}%
\definecolor{currentstroke}{rgb}{0.000000,0.000000,0.000000}%
\pgfsetstrokecolor{currentstroke}%
\pgfsetdash{}{0pt}%
\pgfpathmoveto{\pgfqpoint{2.824734in}{0.356667in}}%
\pgfpathcurveto{\pgfqpoint{2.830260in}{0.356667in}}{\pgfqpoint{2.835559in}{0.358862in}}{\pgfqpoint{2.839466in}{0.362769in}}%
\pgfpathcurveto{\pgfqpoint{2.843373in}{0.366675in}}{\pgfqpoint{2.845568in}{0.371975in}}{\pgfqpoint{2.845568in}{0.377500in}}%
\pgfpathcurveto{\pgfqpoint{2.845568in}{0.383025in}}{\pgfqpoint{2.843373in}{0.388325in}}{\pgfqpoint{2.839466in}{0.392231in}}%
\pgfpathcurveto{\pgfqpoint{2.835559in}{0.396138in}}{\pgfqpoint{2.830260in}{0.398333in}}{\pgfqpoint{2.824734in}{0.398333in}}%
\pgfpathcurveto{\pgfqpoint{2.819209in}{0.398333in}}{\pgfqpoint{2.813910in}{0.396138in}}{\pgfqpoint{2.810003in}{0.392231in}}%
\pgfpathcurveto{\pgfqpoint{2.806096in}{0.388325in}}{\pgfqpoint{2.803901in}{0.383025in}}{\pgfqpoint{2.803901in}{0.377500in}}%
\pgfpathcurveto{\pgfqpoint{2.803901in}{0.371975in}}{\pgfqpoint{2.806096in}{0.366675in}}{\pgfqpoint{2.810003in}{0.362769in}}%
\pgfpathcurveto{\pgfqpoint{2.813910in}{0.358862in}}{\pgfqpoint{2.819209in}{0.356667in}}{\pgfqpoint{2.824734in}{0.356667in}}%
\pgfpathclose%
\pgfusepath{stroke,fill}%
\end{pgfscope}%
\begin{pgfscope}%
\pgfpathrectangle{\pgfqpoint{0.562500in}{0.275000in}}{\pgfqpoint{3.487500in}{1.925000in}}%
\pgfusepath{clip}%
\pgfsetbuttcap%
\pgfsetroundjoin%
\definecolor{currentfill}{rgb}{0.000000,0.000000,0.000000}%
\pgfsetfillcolor{currentfill}%
\pgfsetlinewidth{1.003750pt}%
\definecolor{currentstroke}{rgb}{0.000000,0.000000,0.000000}%
\pgfsetstrokecolor{currentstroke}%
\pgfsetdash{}{0pt}%
\pgfpathmoveto{\pgfqpoint{2.824734in}{0.356667in}}%
\pgfpathcurveto{\pgfqpoint{2.830260in}{0.356667in}}{\pgfqpoint{2.835559in}{0.358862in}}{\pgfqpoint{2.839466in}{0.362769in}}%
\pgfpathcurveto{\pgfqpoint{2.843373in}{0.366675in}}{\pgfqpoint{2.845568in}{0.371975in}}{\pgfqpoint{2.845568in}{0.377500in}}%
\pgfpathcurveto{\pgfqpoint{2.845568in}{0.383025in}}{\pgfqpoint{2.843373in}{0.388325in}}{\pgfqpoint{2.839466in}{0.392231in}}%
\pgfpathcurveto{\pgfqpoint{2.835559in}{0.396138in}}{\pgfqpoint{2.830260in}{0.398333in}}{\pgfqpoint{2.824734in}{0.398333in}}%
\pgfpathcurveto{\pgfqpoint{2.819209in}{0.398333in}}{\pgfqpoint{2.813910in}{0.396138in}}{\pgfqpoint{2.810003in}{0.392231in}}%
\pgfpathcurveto{\pgfqpoint{2.806096in}{0.388325in}}{\pgfqpoint{2.803901in}{0.383025in}}{\pgfqpoint{2.803901in}{0.377500in}}%
\pgfpathcurveto{\pgfqpoint{2.803901in}{0.371975in}}{\pgfqpoint{2.806096in}{0.366675in}}{\pgfqpoint{2.810003in}{0.362769in}}%
\pgfpathcurveto{\pgfqpoint{2.813910in}{0.358862in}}{\pgfqpoint{2.819209in}{0.356667in}}{\pgfqpoint{2.824734in}{0.356667in}}%
\pgfpathclose%
\pgfusepath{stroke,fill}%
\end{pgfscope}%
\begin{pgfscope}%
\pgfpathrectangle{\pgfqpoint{0.562500in}{0.275000in}}{\pgfqpoint{3.487500in}{1.925000in}}%
\pgfusepath{clip}%
\pgfsetbuttcap%
\pgfsetroundjoin%
\definecolor{currentfill}{rgb}{0.000000,0.000000,0.000000}%
\pgfsetfillcolor{currentfill}%
\pgfsetlinewidth{1.003750pt}%
\definecolor{currentstroke}{rgb}{0.000000,0.000000,0.000000}%
\pgfsetstrokecolor{currentstroke}%
\pgfsetdash{}{0pt}%
\pgfpathmoveto{\pgfqpoint{2.824734in}{0.356667in}}%
\pgfpathcurveto{\pgfqpoint{2.830260in}{0.356667in}}{\pgfqpoint{2.835559in}{0.358862in}}{\pgfqpoint{2.839466in}{0.362769in}}%
\pgfpathcurveto{\pgfqpoint{2.843373in}{0.366675in}}{\pgfqpoint{2.845568in}{0.371975in}}{\pgfqpoint{2.845568in}{0.377500in}}%
\pgfpathcurveto{\pgfqpoint{2.845568in}{0.383025in}}{\pgfqpoint{2.843373in}{0.388325in}}{\pgfqpoint{2.839466in}{0.392231in}}%
\pgfpathcurveto{\pgfqpoint{2.835559in}{0.396138in}}{\pgfqpoint{2.830260in}{0.398333in}}{\pgfqpoint{2.824734in}{0.398333in}}%
\pgfpathcurveto{\pgfqpoint{2.819209in}{0.398333in}}{\pgfqpoint{2.813910in}{0.396138in}}{\pgfqpoint{2.810003in}{0.392231in}}%
\pgfpathcurveto{\pgfqpoint{2.806096in}{0.388325in}}{\pgfqpoint{2.803901in}{0.383025in}}{\pgfqpoint{2.803901in}{0.377500in}}%
\pgfpathcurveto{\pgfqpoint{2.803901in}{0.371975in}}{\pgfqpoint{2.806096in}{0.366675in}}{\pgfqpoint{2.810003in}{0.362769in}}%
\pgfpathcurveto{\pgfqpoint{2.813910in}{0.358862in}}{\pgfqpoint{2.819209in}{0.356667in}}{\pgfqpoint{2.824734in}{0.356667in}}%
\pgfpathclose%
\pgfusepath{stroke,fill}%
\end{pgfscope}%
\begin{pgfscope}%
\pgfpathrectangle{\pgfqpoint{0.562500in}{0.275000in}}{\pgfqpoint{3.487500in}{1.925000in}}%
\pgfusepath{clip}%
\pgfsetbuttcap%
\pgfsetroundjoin%
\definecolor{currentfill}{rgb}{0.000000,0.000000,0.000000}%
\pgfsetfillcolor{currentfill}%
\pgfsetlinewidth{1.003750pt}%
\definecolor{currentstroke}{rgb}{0.000000,0.000000,0.000000}%
\pgfsetstrokecolor{currentstroke}%
\pgfsetdash{}{0pt}%
\pgfpathmoveto{\pgfqpoint{2.824734in}{0.356667in}}%
\pgfpathcurveto{\pgfqpoint{2.830260in}{0.356667in}}{\pgfqpoint{2.835559in}{0.358862in}}{\pgfqpoint{2.839466in}{0.362769in}}%
\pgfpathcurveto{\pgfqpoint{2.843373in}{0.366675in}}{\pgfqpoint{2.845568in}{0.371975in}}{\pgfqpoint{2.845568in}{0.377500in}}%
\pgfpathcurveto{\pgfqpoint{2.845568in}{0.383025in}}{\pgfqpoint{2.843373in}{0.388325in}}{\pgfqpoint{2.839466in}{0.392231in}}%
\pgfpathcurveto{\pgfqpoint{2.835559in}{0.396138in}}{\pgfqpoint{2.830260in}{0.398333in}}{\pgfqpoint{2.824734in}{0.398333in}}%
\pgfpathcurveto{\pgfqpoint{2.819209in}{0.398333in}}{\pgfqpoint{2.813910in}{0.396138in}}{\pgfqpoint{2.810003in}{0.392231in}}%
\pgfpathcurveto{\pgfqpoint{2.806096in}{0.388325in}}{\pgfqpoint{2.803901in}{0.383025in}}{\pgfqpoint{2.803901in}{0.377500in}}%
\pgfpathcurveto{\pgfqpoint{2.803901in}{0.371975in}}{\pgfqpoint{2.806096in}{0.366675in}}{\pgfqpoint{2.810003in}{0.362769in}}%
\pgfpathcurveto{\pgfqpoint{2.813910in}{0.358862in}}{\pgfqpoint{2.819209in}{0.356667in}}{\pgfqpoint{2.824734in}{0.356667in}}%
\pgfpathclose%
\pgfusepath{stroke,fill}%
\end{pgfscope}%
\begin{pgfscope}%
\pgfpathrectangle{\pgfqpoint{0.562500in}{0.275000in}}{\pgfqpoint{3.487500in}{1.925000in}}%
\pgfusepath{clip}%
\pgfsetbuttcap%
\pgfsetroundjoin%
\definecolor{currentfill}{rgb}{0.000000,0.000000,0.000000}%
\pgfsetfillcolor{currentfill}%
\pgfsetlinewidth{1.003750pt}%
\definecolor{currentstroke}{rgb}{0.000000,0.000000,0.000000}%
\pgfsetstrokecolor{currentstroke}%
\pgfsetdash{}{0pt}%
\pgfpathmoveto{\pgfqpoint{2.824734in}{0.356667in}}%
\pgfpathcurveto{\pgfqpoint{2.830260in}{0.356667in}}{\pgfqpoint{2.835559in}{0.358862in}}{\pgfqpoint{2.839466in}{0.362769in}}%
\pgfpathcurveto{\pgfqpoint{2.843373in}{0.366675in}}{\pgfqpoint{2.845568in}{0.371975in}}{\pgfqpoint{2.845568in}{0.377500in}}%
\pgfpathcurveto{\pgfqpoint{2.845568in}{0.383025in}}{\pgfqpoint{2.843373in}{0.388325in}}{\pgfqpoint{2.839466in}{0.392231in}}%
\pgfpathcurveto{\pgfqpoint{2.835559in}{0.396138in}}{\pgfqpoint{2.830260in}{0.398333in}}{\pgfqpoint{2.824734in}{0.398333in}}%
\pgfpathcurveto{\pgfqpoint{2.819209in}{0.398333in}}{\pgfqpoint{2.813910in}{0.396138in}}{\pgfqpoint{2.810003in}{0.392231in}}%
\pgfpathcurveto{\pgfqpoint{2.806096in}{0.388325in}}{\pgfqpoint{2.803901in}{0.383025in}}{\pgfqpoint{2.803901in}{0.377500in}}%
\pgfpathcurveto{\pgfqpoint{2.803901in}{0.371975in}}{\pgfqpoint{2.806096in}{0.366675in}}{\pgfqpoint{2.810003in}{0.362769in}}%
\pgfpathcurveto{\pgfqpoint{2.813910in}{0.358862in}}{\pgfqpoint{2.819209in}{0.356667in}}{\pgfqpoint{2.824734in}{0.356667in}}%
\pgfpathclose%
\pgfusepath{stroke,fill}%
\end{pgfscope}%
\begin{pgfscope}%
\pgfpathrectangle{\pgfqpoint{0.562500in}{0.275000in}}{\pgfqpoint{3.487500in}{1.925000in}}%
\pgfusepath{clip}%
\pgfsetbuttcap%
\pgfsetroundjoin%
\definecolor{currentfill}{rgb}{0.000000,0.000000,0.000000}%
\pgfsetfillcolor{currentfill}%
\pgfsetlinewidth{1.003750pt}%
\definecolor{currentstroke}{rgb}{0.000000,0.000000,0.000000}%
\pgfsetstrokecolor{currentstroke}%
\pgfsetdash{}{0pt}%
\pgfpathmoveto{\pgfqpoint{2.824734in}{0.356667in}}%
\pgfpathcurveto{\pgfqpoint{2.830260in}{0.356667in}}{\pgfqpoint{2.835559in}{0.358862in}}{\pgfqpoint{2.839466in}{0.362769in}}%
\pgfpathcurveto{\pgfqpoint{2.843373in}{0.366675in}}{\pgfqpoint{2.845568in}{0.371975in}}{\pgfqpoint{2.845568in}{0.377500in}}%
\pgfpathcurveto{\pgfqpoint{2.845568in}{0.383025in}}{\pgfqpoint{2.843373in}{0.388325in}}{\pgfqpoint{2.839466in}{0.392231in}}%
\pgfpathcurveto{\pgfqpoint{2.835559in}{0.396138in}}{\pgfqpoint{2.830260in}{0.398333in}}{\pgfqpoint{2.824734in}{0.398333in}}%
\pgfpathcurveto{\pgfqpoint{2.819209in}{0.398333in}}{\pgfqpoint{2.813910in}{0.396138in}}{\pgfqpoint{2.810003in}{0.392231in}}%
\pgfpathcurveto{\pgfqpoint{2.806096in}{0.388325in}}{\pgfqpoint{2.803901in}{0.383025in}}{\pgfqpoint{2.803901in}{0.377500in}}%
\pgfpathcurveto{\pgfqpoint{2.803901in}{0.371975in}}{\pgfqpoint{2.806096in}{0.366675in}}{\pgfqpoint{2.810003in}{0.362769in}}%
\pgfpathcurveto{\pgfqpoint{2.813910in}{0.358862in}}{\pgfqpoint{2.819209in}{0.356667in}}{\pgfqpoint{2.824734in}{0.356667in}}%
\pgfpathclose%
\pgfusepath{stroke,fill}%
\end{pgfscope}%
\begin{pgfscope}%
\pgfpathrectangle{\pgfqpoint{0.562500in}{0.275000in}}{\pgfqpoint{3.487500in}{1.925000in}}%
\pgfusepath{clip}%
\pgfsetbuttcap%
\pgfsetroundjoin%
\definecolor{currentfill}{rgb}{0.000000,0.000000,0.000000}%
\pgfsetfillcolor{currentfill}%
\pgfsetlinewidth{1.003750pt}%
\definecolor{currentstroke}{rgb}{0.000000,0.000000,0.000000}%
\pgfsetstrokecolor{currentstroke}%
\pgfsetdash{}{0pt}%
\pgfpathmoveto{\pgfqpoint{2.824734in}{0.356667in}}%
\pgfpathcurveto{\pgfqpoint{2.830260in}{0.356667in}}{\pgfqpoint{2.835559in}{0.358862in}}{\pgfqpoint{2.839466in}{0.362769in}}%
\pgfpathcurveto{\pgfqpoint{2.843373in}{0.366675in}}{\pgfqpoint{2.845568in}{0.371975in}}{\pgfqpoint{2.845568in}{0.377500in}}%
\pgfpathcurveto{\pgfqpoint{2.845568in}{0.383025in}}{\pgfqpoint{2.843373in}{0.388325in}}{\pgfqpoint{2.839466in}{0.392231in}}%
\pgfpathcurveto{\pgfqpoint{2.835559in}{0.396138in}}{\pgfqpoint{2.830260in}{0.398333in}}{\pgfqpoint{2.824734in}{0.398333in}}%
\pgfpathcurveto{\pgfqpoint{2.819209in}{0.398333in}}{\pgfqpoint{2.813910in}{0.396138in}}{\pgfqpoint{2.810003in}{0.392231in}}%
\pgfpathcurveto{\pgfqpoint{2.806096in}{0.388325in}}{\pgfqpoint{2.803901in}{0.383025in}}{\pgfqpoint{2.803901in}{0.377500in}}%
\pgfpathcurveto{\pgfqpoint{2.803901in}{0.371975in}}{\pgfqpoint{2.806096in}{0.366675in}}{\pgfqpoint{2.810003in}{0.362769in}}%
\pgfpathcurveto{\pgfqpoint{2.813910in}{0.358862in}}{\pgfqpoint{2.819209in}{0.356667in}}{\pgfqpoint{2.824734in}{0.356667in}}%
\pgfpathclose%
\pgfusepath{stroke,fill}%
\end{pgfscope}%
\begin{pgfscope}%
\pgfpathrectangle{\pgfqpoint{0.562500in}{0.275000in}}{\pgfqpoint{3.487500in}{1.925000in}}%
\pgfusepath{clip}%
\pgfsetbuttcap%
\pgfsetroundjoin%
\definecolor{currentfill}{rgb}{0.000000,0.000000,0.000000}%
\pgfsetfillcolor{currentfill}%
\pgfsetlinewidth{1.003750pt}%
\definecolor{currentstroke}{rgb}{0.000000,0.000000,0.000000}%
\pgfsetstrokecolor{currentstroke}%
\pgfsetdash{}{0pt}%
\pgfpathmoveto{\pgfqpoint{2.824734in}{0.356667in}}%
\pgfpathcurveto{\pgfqpoint{2.830260in}{0.356667in}}{\pgfqpoint{2.835559in}{0.358862in}}{\pgfqpoint{2.839466in}{0.362769in}}%
\pgfpathcurveto{\pgfqpoint{2.843373in}{0.366675in}}{\pgfqpoint{2.845568in}{0.371975in}}{\pgfqpoint{2.845568in}{0.377500in}}%
\pgfpathcurveto{\pgfqpoint{2.845568in}{0.383025in}}{\pgfqpoint{2.843373in}{0.388325in}}{\pgfqpoint{2.839466in}{0.392231in}}%
\pgfpathcurveto{\pgfqpoint{2.835559in}{0.396138in}}{\pgfqpoint{2.830260in}{0.398333in}}{\pgfqpoint{2.824734in}{0.398333in}}%
\pgfpathcurveto{\pgfqpoint{2.819209in}{0.398333in}}{\pgfqpoint{2.813910in}{0.396138in}}{\pgfqpoint{2.810003in}{0.392231in}}%
\pgfpathcurveto{\pgfqpoint{2.806096in}{0.388325in}}{\pgfqpoint{2.803901in}{0.383025in}}{\pgfqpoint{2.803901in}{0.377500in}}%
\pgfpathcurveto{\pgfqpoint{2.803901in}{0.371975in}}{\pgfqpoint{2.806096in}{0.366675in}}{\pgfqpoint{2.810003in}{0.362769in}}%
\pgfpathcurveto{\pgfqpoint{2.813910in}{0.358862in}}{\pgfqpoint{2.819209in}{0.356667in}}{\pgfqpoint{2.824734in}{0.356667in}}%
\pgfpathclose%
\pgfusepath{stroke,fill}%
\end{pgfscope}%
\begin{pgfscope}%
\pgfpathrectangle{\pgfqpoint{0.562500in}{0.275000in}}{\pgfqpoint{3.487500in}{1.925000in}}%
\pgfusepath{clip}%
\pgfsetbuttcap%
\pgfsetroundjoin%
\definecolor{currentfill}{rgb}{0.000000,0.000000,0.000000}%
\pgfsetfillcolor{currentfill}%
\pgfsetlinewidth{1.003750pt}%
\definecolor{currentstroke}{rgb}{0.000000,0.000000,0.000000}%
\pgfsetstrokecolor{currentstroke}%
\pgfsetdash{}{0pt}%
\pgfpathmoveto{\pgfqpoint{2.824734in}{0.356667in}}%
\pgfpathcurveto{\pgfqpoint{2.830260in}{0.356667in}}{\pgfqpoint{2.835559in}{0.358862in}}{\pgfqpoint{2.839466in}{0.362769in}}%
\pgfpathcurveto{\pgfqpoint{2.843373in}{0.366675in}}{\pgfqpoint{2.845568in}{0.371975in}}{\pgfqpoint{2.845568in}{0.377500in}}%
\pgfpathcurveto{\pgfqpoint{2.845568in}{0.383025in}}{\pgfqpoint{2.843373in}{0.388325in}}{\pgfqpoint{2.839466in}{0.392231in}}%
\pgfpathcurveto{\pgfqpoint{2.835559in}{0.396138in}}{\pgfqpoint{2.830260in}{0.398333in}}{\pgfqpoint{2.824734in}{0.398333in}}%
\pgfpathcurveto{\pgfqpoint{2.819209in}{0.398333in}}{\pgfqpoint{2.813910in}{0.396138in}}{\pgfqpoint{2.810003in}{0.392231in}}%
\pgfpathcurveto{\pgfqpoint{2.806096in}{0.388325in}}{\pgfqpoint{2.803901in}{0.383025in}}{\pgfqpoint{2.803901in}{0.377500in}}%
\pgfpathcurveto{\pgfqpoint{2.803901in}{0.371975in}}{\pgfqpoint{2.806096in}{0.366675in}}{\pgfqpoint{2.810003in}{0.362769in}}%
\pgfpathcurveto{\pgfqpoint{2.813910in}{0.358862in}}{\pgfqpoint{2.819209in}{0.356667in}}{\pgfqpoint{2.824734in}{0.356667in}}%
\pgfpathclose%
\pgfusepath{stroke,fill}%
\end{pgfscope}%
\begin{pgfscope}%
\pgfpathrectangle{\pgfqpoint{0.562500in}{0.275000in}}{\pgfqpoint{3.487500in}{1.925000in}}%
\pgfusepath{clip}%
\pgfsetbuttcap%
\pgfsetroundjoin%
\definecolor{currentfill}{rgb}{0.000000,0.000000,0.000000}%
\pgfsetfillcolor{currentfill}%
\pgfsetlinewidth{1.003750pt}%
\definecolor{currentstroke}{rgb}{0.000000,0.000000,0.000000}%
\pgfsetstrokecolor{currentstroke}%
\pgfsetdash{}{0pt}%
\pgfpathmoveto{\pgfqpoint{2.824734in}{0.356667in}}%
\pgfpathcurveto{\pgfqpoint{2.830260in}{0.356667in}}{\pgfqpoint{2.835559in}{0.358862in}}{\pgfqpoint{2.839466in}{0.362769in}}%
\pgfpathcurveto{\pgfqpoint{2.843373in}{0.366675in}}{\pgfqpoint{2.845568in}{0.371975in}}{\pgfqpoint{2.845568in}{0.377500in}}%
\pgfpathcurveto{\pgfqpoint{2.845568in}{0.383025in}}{\pgfqpoint{2.843373in}{0.388325in}}{\pgfqpoint{2.839466in}{0.392231in}}%
\pgfpathcurveto{\pgfqpoint{2.835559in}{0.396138in}}{\pgfqpoint{2.830260in}{0.398333in}}{\pgfqpoint{2.824734in}{0.398333in}}%
\pgfpathcurveto{\pgfqpoint{2.819209in}{0.398333in}}{\pgfqpoint{2.813910in}{0.396138in}}{\pgfqpoint{2.810003in}{0.392231in}}%
\pgfpathcurveto{\pgfqpoint{2.806096in}{0.388325in}}{\pgfqpoint{2.803901in}{0.383025in}}{\pgfqpoint{2.803901in}{0.377500in}}%
\pgfpathcurveto{\pgfqpoint{2.803901in}{0.371975in}}{\pgfqpoint{2.806096in}{0.366675in}}{\pgfqpoint{2.810003in}{0.362769in}}%
\pgfpathcurveto{\pgfqpoint{2.813910in}{0.358862in}}{\pgfqpoint{2.819209in}{0.356667in}}{\pgfqpoint{2.824734in}{0.356667in}}%
\pgfpathclose%
\pgfusepath{stroke,fill}%
\end{pgfscope}%
\begin{pgfscope}%
\pgfpathrectangle{\pgfqpoint{0.562500in}{0.275000in}}{\pgfqpoint{3.487500in}{1.925000in}}%
\pgfusepath{clip}%
\pgfsetbuttcap%
\pgfsetroundjoin%
\definecolor{currentfill}{rgb}{0.000000,0.000000,0.000000}%
\pgfsetfillcolor{currentfill}%
\pgfsetlinewidth{1.003750pt}%
\definecolor{currentstroke}{rgb}{0.000000,0.000000,0.000000}%
\pgfsetstrokecolor{currentstroke}%
\pgfsetdash{}{0pt}%
\pgfpathmoveto{\pgfqpoint{2.824734in}{0.356667in}}%
\pgfpathcurveto{\pgfqpoint{2.830260in}{0.356667in}}{\pgfqpoint{2.835559in}{0.358862in}}{\pgfqpoint{2.839466in}{0.362769in}}%
\pgfpathcurveto{\pgfqpoint{2.843373in}{0.366675in}}{\pgfqpoint{2.845568in}{0.371975in}}{\pgfqpoint{2.845568in}{0.377500in}}%
\pgfpathcurveto{\pgfqpoint{2.845568in}{0.383025in}}{\pgfqpoint{2.843373in}{0.388325in}}{\pgfqpoint{2.839466in}{0.392231in}}%
\pgfpathcurveto{\pgfqpoint{2.835559in}{0.396138in}}{\pgfqpoint{2.830260in}{0.398333in}}{\pgfqpoint{2.824734in}{0.398333in}}%
\pgfpathcurveto{\pgfqpoint{2.819209in}{0.398333in}}{\pgfqpoint{2.813910in}{0.396138in}}{\pgfqpoint{2.810003in}{0.392231in}}%
\pgfpathcurveto{\pgfqpoint{2.806096in}{0.388325in}}{\pgfqpoint{2.803901in}{0.383025in}}{\pgfqpoint{2.803901in}{0.377500in}}%
\pgfpathcurveto{\pgfqpoint{2.803901in}{0.371975in}}{\pgfqpoint{2.806096in}{0.366675in}}{\pgfqpoint{2.810003in}{0.362769in}}%
\pgfpathcurveto{\pgfqpoint{2.813910in}{0.358862in}}{\pgfqpoint{2.819209in}{0.356667in}}{\pgfqpoint{2.824734in}{0.356667in}}%
\pgfpathclose%
\pgfusepath{stroke,fill}%
\end{pgfscope}%
\begin{pgfscope}%
\pgfpathrectangle{\pgfqpoint{0.562500in}{0.275000in}}{\pgfqpoint{3.487500in}{1.925000in}}%
\pgfusepath{clip}%
\pgfsetbuttcap%
\pgfsetroundjoin%
\definecolor{currentfill}{rgb}{0.000000,0.000000,0.000000}%
\pgfsetfillcolor{currentfill}%
\pgfsetlinewidth{1.003750pt}%
\definecolor{currentstroke}{rgb}{0.000000,0.000000,0.000000}%
\pgfsetstrokecolor{currentstroke}%
\pgfsetdash{}{0pt}%
\pgfpathmoveto{\pgfqpoint{2.824734in}{0.356667in}}%
\pgfpathcurveto{\pgfqpoint{2.830260in}{0.356667in}}{\pgfqpoint{2.835559in}{0.358862in}}{\pgfqpoint{2.839466in}{0.362769in}}%
\pgfpathcurveto{\pgfqpoint{2.843373in}{0.366675in}}{\pgfqpoint{2.845568in}{0.371975in}}{\pgfqpoint{2.845568in}{0.377500in}}%
\pgfpathcurveto{\pgfqpoint{2.845568in}{0.383025in}}{\pgfqpoint{2.843373in}{0.388325in}}{\pgfqpoint{2.839466in}{0.392231in}}%
\pgfpathcurveto{\pgfqpoint{2.835559in}{0.396138in}}{\pgfqpoint{2.830260in}{0.398333in}}{\pgfqpoint{2.824734in}{0.398333in}}%
\pgfpathcurveto{\pgfqpoint{2.819209in}{0.398333in}}{\pgfqpoint{2.813910in}{0.396138in}}{\pgfqpoint{2.810003in}{0.392231in}}%
\pgfpathcurveto{\pgfqpoint{2.806096in}{0.388325in}}{\pgfqpoint{2.803901in}{0.383025in}}{\pgfqpoint{2.803901in}{0.377500in}}%
\pgfpathcurveto{\pgfqpoint{2.803901in}{0.371975in}}{\pgfqpoint{2.806096in}{0.366675in}}{\pgfqpoint{2.810003in}{0.362769in}}%
\pgfpathcurveto{\pgfqpoint{2.813910in}{0.358862in}}{\pgfqpoint{2.819209in}{0.356667in}}{\pgfqpoint{2.824734in}{0.356667in}}%
\pgfpathclose%
\pgfusepath{stroke,fill}%
\end{pgfscope}%
\begin{pgfscope}%
\pgfpathrectangle{\pgfqpoint{0.562500in}{0.275000in}}{\pgfqpoint{3.487500in}{1.925000in}}%
\pgfusepath{clip}%
\pgfsetbuttcap%
\pgfsetroundjoin%
\definecolor{currentfill}{rgb}{0.000000,0.000000,0.000000}%
\pgfsetfillcolor{currentfill}%
\pgfsetlinewidth{1.003750pt}%
\definecolor{currentstroke}{rgb}{0.000000,0.000000,0.000000}%
\pgfsetstrokecolor{currentstroke}%
\pgfsetdash{}{0pt}%
\pgfpathmoveto{\pgfqpoint{2.824734in}{0.356667in}}%
\pgfpathcurveto{\pgfqpoint{2.830260in}{0.356667in}}{\pgfqpoint{2.835559in}{0.358862in}}{\pgfqpoint{2.839466in}{0.362769in}}%
\pgfpathcurveto{\pgfqpoint{2.843373in}{0.366675in}}{\pgfqpoint{2.845568in}{0.371975in}}{\pgfqpoint{2.845568in}{0.377500in}}%
\pgfpathcurveto{\pgfqpoint{2.845568in}{0.383025in}}{\pgfqpoint{2.843373in}{0.388325in}}{\pgfqpoint{2.839466in}{0.392231in}}%
\pgfpathcurveto{\pgfqpoint{2.835559in}{0.396138in}}{\pgfqpoint{2.830260in}{0.398333in}}{\pgfqpoint{2.824734in}{0.398333in}}%
\pgfpathcurveto{\pgfqpoint{2.819209in}{0.398333in}}{\pgfqpoint{2.813910in}{0.396138in}}{\pgfqpoint{2.810003in}{0.392231in}}%
\pgfpathcurveto{\pgfqpoint{2.806096in}{0.388325in}}{\pgfqpoint{2.803901in}{0.383025in}}{\pgfqpoint{2.803901in}{0.377500in}}%
\pgfpathcurveto{\pgfqpoint{2.803901in}{0.371975in}}{\pgfqpoint{2.806096in}{0.366675in}}{\pgfqpoint{2.810003in}{0.362769in}}%
\pgfpathcurveto{\pgfqpoint{2.813910in}{0.358862in}}{\pgfqpoint{2.819209in}{0.356667in}}{\pgfqpoint{2.824734in}{0.356667in}}%
\pgfpathclose%
\pgfusepath{stroke,fill}%
\end{pgfscope}%
\begin{pgfscope}%
\pgfpathrectangle{\pgfqpoint{0.562500in}{0.275000in}}{\pgfqpoint{3.487500in}{1.925000in}}%
\pgfusepath{clip}%
\pgfsetbuttcap%
\pgfsetroundjoin%
\definecolor{currentfill}{rgb}{0.000000,0.000000,0.000000}%
\pgfsetfillcolor{currentfill}%
\pgfsetlinewidth{1.003750pt}%
\definecolor{currentstroke}{rgb}{0.000000,0.000000,0.000000}%
\pgfsetstrokecolor{currentstroke}%
\pgfsetdash{}{0pt}%
\pgfpathmoveto{\pgfqpoint{2.824734in}{0.356667in}}%
\pgfpathcurveto{\pgfqpoint{2.830260in}{0.356667in}}{\pgfqpoint{2.835559in}{0.358862in}}{\pgfqpoint{2.839466in}{0.362769in}}%
\pgfpathcurveto{\pgfqpoint{2.843373in}{0.366675in}}{\pgfqpoint{2.845568in}{0.371975in}}{\pgfqpoint{2.845568in}{0.377500in}}%
\pgfpathcurveto{\pgfqpoint{2.845568in}{0.383025in}}{\pgfqpoint{2.843373in}{0.388325in}}{\pgfqpoint{2.839466in}{0.392231in}}%
\pgfpathcurveto{\pgfqpoint{2.835559in}{0.396138in}}{\pgfqpoint{2.830260in}{0.398333in}}{\pgfqpoint{2.824734in}{0.398333in}}%
\pgfpathcurveto{\pgfqpoint{2.819209in}{0.398333in}}{\pgfqpoint{2.813910in}{0.396138in}}{\pgfqpoint{2.810003in}{0.392231in}}%
\pgfpathcurveto{\pgfqpoint{2.806096in}{0.388325in}}{\pgfqpoint{2.803901in}{0.383025in}}{\pgfqpoint{2.803901in}{0.377500in}}%
\pgfpathcurveto{\pgfqpoint{2.803901in}{0.371975in}}{\pgfqpoint{2.806096in}{0.366675in}}{\pgfqpoint{2.810003in}{0.362769in}}%
\pgfpathcurveto{\pgfqpoint{2.813910in}{0.358862in}}{\pgfqpoint{2.819209in}{0.356667in}}{\pgfqpoint{2.824734in}{0.356667in}}%
\pgfpathclose%
\pgfusepath{stroke,fill}%
\end{pgfscope}%
\begin{pgfscope}%
\pgfpathrectangle{\pgfqpoint{0.562500in}{0.275000in}}{\pgfqpoint{3.487500in}{1.925000in}}%
\pgfusepath{clip}%
\pgfsetbuttcap%
\pgfsetroundjoin%
\definecolor{currentfill}{rgb}{0.000000,0.000000,0.000000}%
\pgfsetfillcolor{currentfill}%
\pgfsetlinewidth{1.003750pt}%
\definecolor{currentstroke}{rgb}{0.000000,0.000000,0.000000}%
\pgfsetstrokecolor{currentstroke}%
\pgfsetdash{}{0pt}%
\pgfpathmoveto{\pgfqpoint{2.824734in}{0.356667in}}%
\pgfpathcurveto{\pgfqpoint{2.830260in}{0.356667in}}{\pgfqpoint{2.835559in}{0.358862in}}{\pgfqpoint{2.839466in}{0.362769in}}%
\pgfpathcurveto{\pgfqpoint{2.843373in}{0.366675in}}{\pgfqpoint{2.845568in}{0.371975in}}{\pgfqpoint{2.845568in}{0.377500in}}%
\pgfpathcurveto{\pgfqpoint{2.845568in}{0.383025in}}{\pgfqpoint{2.843373in}{0.388325in}}{\pgfqpoint{2.839466in}{0.392231in}}%
\pgfpathcurveto{\pgfqpoint{2.835559in}{0.396138in}}{\pgfqpoint{2.830260in}{0.398333in}}{\pgfqpoint{2.824734in}{0.398333in}}%
\pgfpathcurveto{\pgfqpoint{2.819209in}{0.398333in}}{\pgfqpoint{2.813910in}{0.396138in}}{\pgfqpoint{2.810003in}{0.392231in}}%
\pgfpathcurveto{\pgfqpoint{2.806096in}{0.388325in}}{\pgfqpoint{2.803901in}{0.383025in}}{\pgfqpoint{2.803901in}{0.377500in}}%
\pgfpathcurveto{\pgfqpoint{2.803901in}{0.371975in}}{\pgfqpoint{2.806096in}{0.366675in}}{\pgfqpoint{2.810003in}{0.362769in}}%
\pgfpathcurveto{\pgfqpoint{2.813910in}{0.358862in}}{\pgfqpoint{2.819209in}{0.356667in}}{\pgfqpoint{2.824734in}{0.356667in}}%
\pgfpathclose%
\pgfusepath{stroke,fill}%
\end{pgfscope}%
\begin{pgfscope}%
\pgfpathrectangle{\pgfqpoint{0.562500in}{0.275000in}}{\pgfqpoint{3.487500in}{1.925000in}}%
\pgfusepath{clip}%
\pgfsetbuttcap%
\pgfsetroundjoin%
\definecolor{currentfill}{rgb}{0.000000,0.000000,0.000000}%
\pgfsetfillcolor{currentfill}%
\pgfsetlinewidth{1.003750pt}%
\definecolor{currentstroke}{rgb}{0.000000,0.000000,0.000000}%
\pgfsetstrokecolor{currentstroke}%
\pgfsetdash{}{0pt}%
\pgfpathmoveto{\pgfqpoint{2.824734in}{0.356667in}}%
\pgfpathcurveto{\pgfqpoint{2.830260in}{0.356667in}}{\pgfqpoint{2.835559in}{0.358862in}}{\pgfqpoint{2.839466in}{0.362769in}}%
\pgfpathcurveto{\pgfqpoint{2.843373in}{0.366675in}}{\pgfqpoint{2.845568in}{0.371975in}}{\pgfqpoint{2.845568in}{0.377500in}}%
\pgfpathcurveto{\pgfqpoint{2.845568in}{0.383025in}}{\pgfqpoint{2.843373in}{0.388325in}}{\pgfqpoint{2.839466in}{0.392231in}}%
\pgfpathcurveto{\pgfqpoint{2.835559in}{0.396138in}}{\pgfqpoint{2.830260in}{0.398333in}}{\pgfqpoint{2.824734in}{0.398333in}}%
\pgfpathcurveto{\pgfqpoint{2.819209in}{0.398333in}}{\pgfqpoint{2.813910in}{0.396138in}}{\pgfqpoint{2.810003in}{0.392231in}}%
\pgfpathcurveto{\pgfqpoint{2.806096in}{0.388325in}}{\pgfqpoint{2.803901in}{0.383025in}}{\pgfqpoint{2.803901in}{0.377500in}}%
\pgfpathcurveto{\pgfqpoint{2.803901in}{0.371975in}}{\pgfqpoint{2.806096in}{0.366675in}}{\pgfqpoint{2.810003in}{0.362769in}}%
\pgfpathcurveto{\pgfqpoint{2.813910in}{0.358862in}}{\pgfqpoint{2.819209in}{0.356667in}}{\pgfqpoint{2.824734in}{0.356667in}}%
\pgfpathclose%
\pgfusepath{stroke,fill}%
\end{pgfscope}%
\begin{pgfscope}%
\pgfpathrectangle{\pgfqpoint{0.562500in}{0.275000in}}{\pgfqpoint{3.487500in}{1.925000in}}%
\pgfusepath{clip}%
\pgfsetbuttcap%
\pgfsetroundjoin%
\definecolor{currentfill}{rgb}{0.000000,0.000000,0.000000}%
\pgfsetfillcolor{currentfill}%
\pgfsetlinewidth{1.003750pt}%
\definecolor{currentstroke}{rgb}{0.000000,0.000000,0.000000}%
\pgfsetstrokecolor{currentstroke}%
\pgfsetdash{}{0pt}%
\pgfpathmoveto{\pgfqpoint{2.824734in}{0.356667in}}%
\pgfpathcurveto{\pgfqpoint{2.830260in}{0.356667in}}{\pgfqpoint{2.835559in}{0.358862in}}{\pgfqpoint{2.839466in}{0.362769in}}%
\pgfpathcurveto{\pgfqpoint{2.843373in}{0.366675in}}{\pgfqpoint{2.845568in}{0.371975in}}{\pgfqpoint{2.845568in}{0.377500in}}%
\pgfpathcurveto{\pgfqpoint{2.845568in}{0.383025in}}{\pgfqpoint{2.843373in}{0.388325in}}{\pgfqpoint{2.839466in}{0.392231in}}%
\pgfpathcurveto{\pgfqpoint{2.835559in}{0.396138in}}{\pgfqpoint{2.830260in}{0.398333in}}{\pgfqpoint{2.824734in}{0.398333in}}%
\pgfpathcurveto{\pgfqpoint{2.819209in}{0.398333in}}{\pgfqpoint{2.813910in}{0.396138in}}{\pgfqpoint{2.810003in}{0.392231in}}%
\pgfpathcurveto{\pgfqpoint{2.806096in}{0.388325in}}{\pgfqpoint{2.803901in}{0.383025in}}{\pgfqpoint{2.803901in}{0.377500in}}%
\pgfpathcurveto{\pgfqpoint{2.803901in}{0.371975in}}{\pgfqpoint{2.806096in}{0.366675in}}{\pgfqpoint{2.810003in}{0.362769in}}%
\pgfpathcurveto{\pgfqpoint{2.813910in}{0.358862in}}{\pgfqpoint{2.819209in}{0.356667in}}{\pgfqpoint{2.824734in}{0.356667in}}%
\pgfpathclose%
\pgfusepath{stroke,fill}%
\end{pgfscope}%
\begin{pgfscope}%
\pgfpathrectangle{\pgfqpoint{0.562500in}{0.275000in}}{\pgfqpoint{3.487500in}{1.925000in}}%
\pgfusepath{clip}%
\pgfsetbuttcap%
\pgfsetroundjoin%
\definecolor{currentfill}{rgb}{0.000000,0.000000,0.000000}%
\pgfsetfillcolor{currentfill}%
\pgfsetlinewidth{1.003750pt}%
\definecolor{currentstroke}{rgb}{0.000000,0.000000,0.000000}%
\pgfsetstrokecolor{currentstroke}%
\pgfsetdash{}{0pt}%
\pgfpathmoveto{\pgfqpoint{2.824734in}{0.356667in}}%
\pgfpathcurveto{\pgfqpoint{2.830260in}{0.356667in}}{\pgfqpoint{2.835559in}{0.358862in}}{\pgfqpoint{2.839466in}{0.362769in}}%
\pgfpathcurveto{\pgfqpoint{2.843373in}{0.366675in}}{\pgfqpoint{2.845568in}{0.371975in}}{\pgfqpoint{2.845568in}{0.377500in}}%
\pgfpathcurveto{\pgfqpoint{2.845568in}{0.383025in}}{\pgfqpoint{2.843373in}{0.388325in}}{\pgfqpoint{2.839466in}{0.392231in}}%
\pgfpathcurveto{\pgfqpoint{2.835559in}{0.396138in}}{\pgfqpoint{2.830260in}{0.398333in}}{\pgfqpoint{2.824734in}{0.398333in}}%
\pgfpathcurveto{\pgfqpoint{2.819209in}{0.398333in}}{\pgfqpoint{2.813910in}{0.396138in}}{\pgfqpoint{2.810003in}{0.392231in}}%
\pgfpathcurveto{\pgfqpoint{2.806096in}{0.388325in}}{\pgfqpoint{2.803901in}{0.383025in}}{\pgfqpoint{2.803901in}{0.377500in}}%
\pgfpathcurveto{\pgfqpoint{2.803901in}{0.371975in}}{\pgfqpoint{2.806096in}{0.366675in}}{\pgfqpoint{2.810003in}{0.362769in}}%
\pgfpathcurveto{\pgfqpoint{2.813910in}{0.358862in}}{\pgfqpoint{2.819209in}{0.356667in}}{\pgfqpoint{2.824734in}{0.356667in}}%
\pgfpathclose%
\pgfusepath{stroke,fill}%
\end{pgfscope}%
\begin{pgfscope}%
\pgfpathrectangle{\pgfqpoint{0.562500in}{0.275000in}}{\pgfqpoint{3.487500in}{1.925000in}}%
\pgfusepath{clip}%
\pgfsetbuttcap%
\pgfsetroundjoin%
\definecolor{currentfill}{rgb}{0.000000,0.000000,0.000000}%
\pgfsetfillcolor{currentfill}%
\pgfsetlinewidth{1.003750pt}%
\definecolor{currentstroke}{rgb}{0.000000,0.000000,0.000000}%
\pgfsetstrokecolor{currentstroke}%
\pgfsetdash{}{0pt}%
\pgfpathmoveto{\pgfqpoint{2.824734in}{0.356667in}}%
\pgfpathcurveto{\pgfqpoint{2.830260in}{0.356667in}}{\pgfqpoint{2.835559in}{0.358862in}}{\pgfqpoint{2.839466in}{0.362769in}}%
\pgfpathcurveto{\pgfqpoint{2.843373in}{0.366675in}}{\pgfqpoint{2.845568in}{0.371975in}}{\pgfqpoint{2.845568in}{0.377500in}}%
\pgfpathcurveto{\pgfqpoint{2.845568in}{0.383025in}}{\pgfqpoint{2.843373in}{0.388325in}}{\pgfqpoint{2.839466in}{0.392231in}}%
\pgfpathcurveto{\pgfqpoint{2.835559in}{0.396138in}}{\pgfqpoint{2.830260in}{0.398333in}}{\pgfqpoint{2.824734in}{0.398333in}}%
\pgfpathcurveto{\pgfqpoint{2.819209in}{0.398333in}}{\pgfqpoint{2.813910in}{0.396138in}}{\pgfqpoint{2.810003in}{0.392231in}}%
\pgfpathcurveto{\pgfqpoint{2.806096in}{0.388325in}}{\pgfqpoint{2.803901in}{0.383025in}}{\pgfqpoint{2.803901in}{0.377500in}}%
\pgfpathcurveto{\pgfqpoint{2.803901in}{0.371975in}}{\pgfqpoint{2.806096in}{0.366675in}}{\pgfqpoint{2.810003in}{0.362769in}}%
\pgfpathcurveto{\pgfqpoint{2.813910in}{0.358862in}}{\pgfqpoint{2.819209in}{0.356667in}}{\pgfqpoint{2.824734in}{0.356667in}}%
\pgfpathclose%
\pgfusepath{stroke,fill}%
\end{pgfscope}%
\begin{pgfscope}%
\pgfpathrectangle{\pgfqpoint{0.562500in}{0.275000in}}{\pgfqpoint{3.487500in}{1.925000in}}%
\pgfusepath{clip}%
\pgfsetbuttcap%
\pgfsetroundjoin%
\definecolor{currentfill}{rgb}{0.000000,0.000000,0.000000}%
\pgfsetfillcolor{currentfill}%
\pgfsetlinewidth{1.003750pt}%
\definecolor{currentstroke}{rgb}{0.000000,0.000000,0.000000}%
\pgfsetstrokecolor{currentstroke}%
\pgfsetdash{}{0pt}%
\pgfpathmoveto{\pgfqpoint{2.824734in}{0.356667in}}%
\pgfpathcurveto{\pgfqpoint{2.830260in}{0.356667in}}{\pgfqpoint{2.835559in}{0.358862in}}{\pgfqpoint{2.839466in}{0.362769in}}%
\pgfpathcurveto{\pgfqpoint{2.843373in}{0.366675in}}{\pgfqpoint{2.845568in}{0.371975in}}{\pgfqpoint{2.845568in}{0.377500in}}%
\pgfpathcurveto{\pgfqpoint{2.845568in}{0.383025in}}{\pgfqpoint{2.843373in}{0.388325in}}{\pgfqpoint{2.839466in}{0.392231in}}%
\pgfpathcurveto{\pgfqpoint{2.835559in}{0.396138in}}{\pgfqpoint{2.830260in}{0.398333in}}{\pgfqpoint{2.824734in}{0.398333in}}%
\pgfpathcurveto{\pgfqpoint{2.819209in}{0.398333in}}{\pgfqpoint{2.813910in}{0.396138in}}{\pgfqpoint{2.810003in}{0.392231in}}%
\pgfpathcurveto{\pgfqpoint{2.806096in}{0.388325in}}{\pgfqpoint{2.803901in}{0.383025in}}{\pgfqpoint{2.803901in}{0.377500in}}%
\pgfpathcurveto{\pgfqpoint{2.803901in}{0.371975in}}{\pgfqpoint{2.806096in}{0.366675in}}{\pgfqpoint{2.810003in}{0.362769in}}%
\pgfpathcurveto{\pgfqpoint{2.813910in}{0.358862in}}{\pgfqpoint{2.819209in}{0.356667in}}{\pgfqpoint{2.824734in}{0.356667in}}%
\pgfpathclose%
\pgfusepath{stroke,fill}%
\end{pgfscope}%
\begin{pgfscope}%
\pgfpathrectangle{\pgfqpoint{0.562500in}{0.275000in}}{\pgfqpoint{3.487500in}{1.925000in}}%
\pgfusepath{clip}%
\pgfsetbuttcap%
\pgfsetroundjoin%
\definecolor{currentfill}{rgb}{0.000000,0.000000,0.000000}%
\pgfsetfillcolor{currentfill}%
\pgfsetlinewidth{1.003750pt}%
\definecolor{currentstroke}{rgb}{0.000000,0.000000,0.000000}%
\pgfsetstrokecolor{currentstroke}%
\pgfsetdash{}{0pt}%
\pgfpathmoveto{\pgfqpoint{2.824734in}{0.356667in}}%
\pgfpathcurveto{\pgfqpoint{2.830260in}{0.356667in}}{\pgfqpoint{2.835559in}{0.358862in}}{\pgfqpoint{2.839466in}{0.362769in}}%
\pgfpathcurveto{\pgfqpoint{2.843373in}{0.366675in}}{\pgfqpoint{2.845568in}{0.371975in}}{\pgfqpoint{2.845568in}{0.377500in}}%
\pgfpathcurveto{\pgfqpoint{2.845568in}{0.383025in}}{\pgfqpoint{2.843373in}{0.388325in}}{\pgfqpoint{2.839466in}{0.392231in}}%
\pgfpathcurveto{\pgfqpoint{2.835559in}{0.396138in}}{\pgfqpoint{2.830260in}{0.398333in}}{\pgfqpoint{2.824734in}{0.398333in}}%
\pgfpathcurveto{\pgfqpoint{2.819209in}{0.398333in}}{\pgfqpoint{2.813910in}{0.396138in}}{\pgfqpoint{2.810003in}{0.392231in}}%
\pgfpathcurveto{\pgfqpoint{2.806096in}{0.388325in}}{\pgfqpoint{2.803901in}{0.383025in}}{\pgfqpoint{2.803901in}{0.377500in}}%
\pgfpathcurveto{\pgfqpoint{2.803901in}{0.371975in}}{\pgfqpoint{2.806096in}{0.366675in}}{\pgfqpoint{2.810003in}{0.362769in}}%
\pgfpathcurveto{\pgfqpoint{2.813910in}{0.358862in}}{\pgfqpoint{2.819209in}{0.356667in}}{\pgfqpoint{2.824734in}{0.356667in}}%
\pgfpathclose%
\pgfusepath{stroke,fill}%
\end{pgfscope}%
\begin{pgfscope}%
\pgfpathrectangle{\pgfqpoint{0.562500in}{0.275000in}}{\pgfqpoint{3.487500in}{1.925000in}}%
\pgfusepath{clip}%
\pgfsetbuttcap%
\pgfsetroundjoin%
\definecolor{currentfill}{rgb}{0.000000,0.000000,0.000000}%
\pgfsetfillcolor{currentfill}%
\pgfsetlinewidth{1.003750pt}%
\definecolor{currentstroke}{rgb}{0.000000,0.000000,0.000000}%
\pgfsetstrokecolor{currentstroke}%
\pgfsetdash{}{0pt}%
\pgfpathmoveto{\pgfqpoint{2.824734in}{0.356667in}}%
\pgfpathcurveto{\pgfqpoint{2.830260in}{0.356667in}}{\pgfqpoint{2.835559in}{0.358862in}}{\pgfqpoint{2.839466in}{0.362769in}}%
\pgfpathcurveto{\pgfqpoint{2.843373in}{0.366675in}}{\pgfqpoint{2.845568in}{0.371975in}}{\pgfqpoint{2.845568in}{0.377500in}}%
\pgfpathcurveto{\pgfqpoint{2.845568in}{0.383025in}}{\pgfqpoint{2.843373in}{0.388325in}}{\pgfqpoint{2.839466in}{0.392231in}}%
\pgfpathcurveto{\pgfqpoint{2.835559in}{0.396138in}}{\pgfqpoint{2.830260in}{0.398333in}}{\pgfqpoint{2.824734in}{0.398333in}}%
\pgfpathcurveto{\pgfqpoint{2.819209in}{0.398333in}}{\pgfqpoint{2.813910in}{0.396138in}}{\pgfqpoint{2.810003in}{0.392231in}}%
\pgfpathcurveto{\pgfqpoint{2.806096in}{0.388325in}}{\pgfqpoint{2.803901in}{0.383025in}}{\pgfqpoint{2.803901in}{0.377500in}}%
\pgfpathcurveto{\pgfqpoint{2.803901in}{0.371975in}}{\pgfqpoint{2.806096in}{0.366675in}}{\pgfqpoint{2.810003in}{0.362769in}}%
\pgfpathcurveto{\pgfqpoint{2.813910in}{0.358862in}}{\pgfqpoint{2.819209in}{0.356667in}}{\pgfqpoint{2.824734in}{0.356667in}}%
\pgfpathclose%
\pgfusepath{stroke,fill}%
\end{pgfscope}%
\begin{pgfscope}%
\pgfpathrectangle{\pgfqpoint{0.562500in}{0.275000in}}{\pgfqpoint{3.487500in}{1.925000in}}%
\pgfusepath{clip}%
\pgfsetbuttcap%
\pgfsetroundjoin%
\definecolor{currentfill}{rgb}{0.000000,0.000000,0.000000}%
\pgfsetfillcolor{currentfill}%
\pgfsetlinewidth{1.003750pt}%
\definecolor{currentstroke}{rgb}{0.000000,0.000000,0.000000}%
\pgfsetstrokecolor{currentstroke}%
\pgfsetdash{}{0pt}%
\pgfpathmoveto{\pgfqpoint{2.824734in}{0.356667in}}%
\pgfpathcurveto{\pgfqpoint{2.830260in}{0.356667in}}{\pgfqpoint{2.835559in}{0.358862in}}{\pgfqpoint{2.839466in}{0.362769in}}%
\pgfpathcurveto{\pgfqpoint{2.843373in}{0.366675in}}{\pgfqpoint{2.845568in}{0.371975in}}{\pgfqpoint{2.845568in}{0.377500in}}%
\pgfpathcurveto{\pgfqpoint{2.845568in}{0.383025in}}{\pgfqpoint{2.843373in}{0.388325in}}{\pgfqpoint{2.839466in}{0.392231in}}%
\pgfpathcurveto{\pgfqpoint{2.835559in}{0.396138in}}{\pgfqpoint{2.830260in}{0.398333in}}{\pgfqpoint{2.824734in}{0.398333in}}%
\pgfpathcurveto{\pgfqpoint{2.819209in}{0.398333in}}{\pgfqpoint{2.813910in}{0.396138in}}{\pgfqpoint{2.810003in}{0.392231in}}%
\pgfpathcurveto{\pgfqpoint{2.806096in}{0.388325in}}{\pgfqpoint{2.803901in}{0.383025in}}{\pgfqpoint{2.803901in}{0.377500in}}%
\pgfpathcurveto{\pgfqpoint{2.803901in}{0.371975in}}{\pgfqpoint{2.806096in}{0.366675in}}{\pgfqpoint{2.810003in}{0.362769in}}%
\pgfpathcurveto{\pgfqpoint{2.813910in}{0.358862in}}{\pgfqpoint{2.819209in}{0.356667in}}{\pgfqpoint{2.824734in}{0.356667in}}%
\pgfpathclose%
\pgfusepath{stroke,fill}%
\end{pgfscope}%
\begin{pgfscope}%
\pgfpathrectangle{\pgfqpoint{0.562500in}{0.275000in}}{\pgfqpoint{3.487500in}{1.925000in}}%
\pgfusepath{clip}%
\pgfsetbuttcap%
\pgfsetroundjoin%
\definecolor{currentfill}{rgb}{0.000000,0.000000,0.000000}%
\pgfsetfillcolor{currentfill}%
\pgfsetlinewidth{1.003750pt}%
\definecolor{currentstroke}{rgb}{0.000000,0.000000,0.000000}%
\pgfsetstrokecolor{currentstroke}%
\pgfsetdash{}{0pt}%
\pgfpathmoveto{\pgfqpoint{2.824734in}{0.356667in}}%
\pgfpathcurveto{\pgfqpoint{2.830260in}{0.356667in}}{\pgfqpoint{2.835559in}{0.358862in}}{\pgfqpoint{2.839466in}{0.362769in}}%
\pgfpathcurveto{\pgfqpoint{2.843373in}{0.366675in}}{\pgfqpoint{2.845568in}{0.371975in}}{\pgfqpoint{2.845568in}{0.377500in}}%
\pgfpathcurveto{\pgfqpoint{2.845568in}{0.383025in}}{\pgfqpoint{2.843373in}{0.388325in}}{\pgfqpoint{2.839466in}{0.392231in}}%
\pgfpathcurveto{\pgfqpoint{2.835559in}{0.396138in}}{\pgfqpoint{2.830260in}{0.398333in}}{\pgfqpoint{2.824734in}{0.398333in}}%
\pgfpathcurveto{\pgfqpoint{2.819209in}{0.398333in}}{\pgfqpoint{2.813910in}{0.396138in}}{\pgfqpoint{2.810003in}{0.392231in}}%
\pgfpathcurveto{\pgfqpoint{2.806096in}{0.388325in}}{\pgfqpoint{2.803901in}{0.383025in}}{\pgfqpoint{2.803901in}{0.377500in}}%
\pgfpathcurveto{\pgfqpoint{2.803901in}{0.371975in}}{\pgfqpoint{2.806096in}{0.366675in}}{\pgfqpoint{2.810003in}{0.362769in}}%
\pgfpathcurveto{\pgfqpoint{2.813910in}{0.358862in}}{\pgfqpoint{2.819209in}{0.356667in}}{\pgfqpoint{2.824734in}{0.356667in}}%
\pgfpathclose%
\pgfusepath{stroke,fill}%
\end{pgfscope}%
\begin{pgfscope}%
\pgfpathrectangle{\pgfqpoint{0.562500in}{0.275000in}}{\pgfqpoint{3.487500in}{1.925000in}}%
\pgfusepath{clip}%
\pgfsetbuttcap%
\pgfsetroundjoin%
\definecolor{currentfill}{rgb}{0.000000,0.000000,0.000000}%
\pgfsetfillcolor{currentfill}%
\pgfsetlinewidth{1.003750pt}%
\definecolor{currentstroke}{rgb}{0.000000,0.000000,0.000000}%
\pgfsetstrokecolor{currentstroke}%
\pgfsetdash{}{0pt}%
\pgfpathmoveto{\pgfqpoint{2.824734in}{0.356667in}}%
\pgfpathcurveto{\pgfqpoint{2.830260in}{0.356667in}}{\pgfqpoint{2.835559in}{0.358862in}}{\pgfqpoint{2.839466in}{0.362769in}}%
\pgfpathcurveto{\pgfqpoint{2.843373in}{0.366675in}}{\pgfqpoint{2.845568in}{0.371975in}}{\pgfqpoint{2.845568in}{0.377500in}}%
\pgfpathcurveto{\pgfqpoint{2.845568in}{0.383025in}}{\pgfqpoint{2.843373in}{0.388325in}}{\pgfqpoint{2.839466in}{0.392231in}}%
\pgfpathcurveto{\pgfqpoint{2.835559in}{0.396138in}}{\pgfqpoint{2.830260in}{0.398333in}}{\pgfqpoint{2.824734in}{0.398333in}}%
\pgfpathcurveto{\pgfqpoint{2.819209in}{0.398333in}}{\pgfqpoint{2.813910in}{0.396138in}}{\pgfqpoint{2.810003in}{0.392231in}}%
\pgfpathcurveto{\pgfqpoint{2.806096in}{0.388325in}}{\pgfqpoint{2.803901in}{0.383025in}}{\pgfqpoint{2.803901in}{0.377500in}}%
\pgfpathcurveto{\pgfqpoint{2.803901in}{0.371975in}}{\pgfqpoint{2.806096in}{0.366675in}}{\pgfqpoint{2.810003in}{0.362769in}}%
\pgfpathcurveto{\pgfqpoint{2.813910in}{0.358862in}}{\pgfqpoint{2.819209in}{0.356667in}}{\pgfqpoint{2.824734in}{0.356667in}}%
\pgfpathclose%
\pgfusepath{stroke,fill}%
\end{pgfscope}%
\begin{pgfscope}%
\pgfpathrectangle{\pgfqpoint{0.562500in}{0.275000in}}{\pgfqpoint{3.487500in}{1.925000in}}%
\pgfusepath{clip}%
\pgfsetbuttcap%
\pgfsetroundjoin%
\definecolor{currentfill}{rgb}{0.000000,0.000000,0.000000}%
\pgfsetfillcolor{currentfill}%
\pgfsetlinewidth{1.003750pt}%
\definecolor{currentstroke}{rgb}{0.000000,0.000000,0.000000}%
\pgfsetstrokecolor{currentstroke}%
\pgfsetdash{}{0pt}%
\pgfpathmoveto{\pgfqpoint{2.824734in}{0.356667in}}%
\pgfpathcurveto{\pgfqpoint{2.830260in}{0.356667in}}{\pgfqpoint{2.835559in}{0.358862in}}{\pgfqpoint{2.839466in}{0.362769in}}%
\pgfpathcurveto{\pgfqpoint{2.843373in}{0.366675in}}{\pgfqpoint{2.845568in}{0.371975in}}{\pgfqpoint{2.845568in}{0.377500in}}%
\pgfpathcurveto{\pgfqpoint{2.845568in}{0.383025in}}{\pgfqpoint{2.843373in}{0.388325in}}{\pgfqpoint{2.839466in}{0.392231in}}%
\pgfpathcurveto{\pgfqpoint{2.835559in}{0.396138in}}{\pgfqpoint{2.830260in}{0.398333in}}{\pgfqpoint{2.824734in}{0.398333in}}%
\pgfpathcurveto{\pgfqpoint{2.819209in}{0.398333in}}{\pgfqpoint{2.813910in}{0.396138in}}{\pgfqpoint{2.810003in}{0.392231in}}%
\pgfpathcurveto{\pgfqpoint{2.806096in}{0.388325in}}{\pgfqpoint{2.803901in}{0.383025in}}{\pgfqpoint{2.803901in}{0.377500in}}%
\pgfpathcurveto{\pgfqpoint{2.803901in}{0.371975in}}{\pgfqpoint{2.806096in}{0.366675in}}{\pgfqpoint{2.810003in}{0.362769in}}%
\pgfpathcurveto{\pgfqpoint{2.813910in}{0.358862in}}{\pgfqpoint{2.819209in}{0.356667in}}{\pgfqpoint{2.824734in}{0.356667in}}%
\pgfpathclose%
\pgfusepath{stroke,fill}%
\end{pgfscope}%
\begin{pgfscope}%
\pgfpathrectangle{\pgfqpoint{0.562500in}{0.275000in}}{\pgfqpoint{3.487500in}{1.925000in}}%
\pgfusepath{clip}%
\pgfsetbuttcap%
\pgfsetroundjoin%
\definecolor{currentfill}{rgb}{0.000000,0.000000,0.000000}%
\pgfsetfillcolor{currentfill}%
\pgfsetlinewidth{1.003750pt}%
\definecolor{currentstroke}{rgb}{0.000000,0.000000,0.000000}%
\pgfsetstrokecolor{currentstroke}%
\pgfsetdash{}{0pt}%
\pgfpathmoveto{\pgfqpoint{2.824734in}{0.356667in}}%
\pgfpathcurveto{\pgfqpoint{2.830260in}{0.356667in}}{\pgfqpoint{2.835559in}{0.358862in}}{\pgfqpoint{2.839466in}{0.362769in}}%
\pgfpathcurveto{\pgfqpoint{2.843373in}{0.366675in}}{\pgfqpoint{2.845568in}{0.371975in}}{\pgfqpoint{2.845568in}{0.377500in}}%
\pgfpathcurveto{\pgfqpoint{2.845568in}{0.383025in}}{\pgfqpoint{2.843373in}{0.388325in}}{\pgfqpoint{2.839466in}{0.392231in}}%
\pgfpathcurveto{\pgfqpoint{2.835559in}{0.396138in}}{\pgfqpoint{2.830260in}{0.398333in}}{\pgfqpoint{2.824734in}{0.398333in}}%
\pgfpathcurveto{\pgfqpoint{2.819209in}{0.398333in}}{\pgfqpoint{2.813910in}{0.396138in}}{\pgfqpoint{2.810003in}{0.392231in}}%
\pgfpathcurveto{\pgfqpoint{2.806096in}{0.388325in}}{\pgfqpoint{2.803901in}{0.383025in}}{\pgfqpoint{2.803901in}{0.377500in}}%
\pgfpathcurveto{\pgfqpoint{2.803901in}{0.371975in}}{\pgfqpoint{2.806096in}{0.366675in}}{\pgfqpoint{2.810003in}{0.362769in}}%
\pgfpathcurveto{\pgfqpoint{2.813910in}{0.358862in}}{\pgfqpoint{2.819209in}{0.356667in}}{\pgfqpoint{2.824734in}{0.356667in}}%
\pgfpathclose%
\pgfusepath{stroke,fill}%
\end{pgfscope}%
\begin{pgfscope}%
\pgfpathrectangle{\pgfqpoint{0.562500in}{0.275000in}}{\pgfqpoint{3.487500in}{1.925000in}}%
\pgfusepath{clip}%
\pgfsetbuttcap%
\pgfsetroundjoin%
\definecolor{currentfill}{rgb}{0.000000,0.000000,0.000000}%
\pgfsetfillcolor{currentfill}%
\pgfsetlinewidth{1.003750pt}%
\definecolor{currentstroke}{rgb}{0.000000,0.000000,0.000000}%
\pgfsetstrokecolor{currentstroke}%
\pgfsetdash{}{0pt}%
\pgfpathmoveto{\pgfqpoint{2.824734in}{0.356667in}}%
\pgfpathcurveto{\pgfqpoint{2.830260in}{0.356667in}}{\pgfqpoint{2.835559in}{0.358862in}}{\pgfqpoint{2.839466in}{0.362769in}}%
\pgfpathcurveto{\pgfqpoint{2.843373in}{0.366675in}}{\pgfqpoint{2.845568in}{0.371975in}}{\pgfqpoint{2.845568in}{0.377500in}}%
\pgfpathcurveto{\pgfqpoint{2.845568in}{0.383025in}}{\pgfqpoint{2.843373in}{0.388325in}}{\pgfqpoint{2.839466in}{0.392231in}}%
\pgfpathcurveto{\pgfqpoint{2.835559in}{0.396138in}}{\pgfqpoint{2.830260in}{0.398333in}}{\pgfqpoint{2.824734in}{0.398333in}}%
\pgfpathcurveto{\pgfqpoint{2.819209in}{0.398333in}}{\pgfqpoint{2.813910in}{0.396138in}}{\pgfqpoint{2.810003in}{0.392231in}}%
\pgfpathcurveto{\pgfqpoint{2.806096in}{0.388325in}}{\pgfqpoint{2.803901in}{0.383025in}}{\pgfqpoint{2.803901in}{0.377500in}}%
\pgfpathcurveto{\pgfqpoint{2.803901in}{0.371975in}}{\pgfqpoint{2.806096in}{0.366675in}}{\pgfqpoint{2.810003in}{0.362769in}}%
\pgfpathcurveto{\pgfqpoint{2.813910in}{0.358862in}}{\pgfqpoint{2.819209in}{0.356667in}}{\pgfqpoint{2.824734in}{0.356667in}}%
\pgfpathclose%
\pgfusepath{stroke,fill}%
\end{pgfscope}%
\begin{pgfscope}%
\pgfpathrectangle{\pgfqpoint{0.562500in}{0.275000in}}{\pgfqpoint{3.487500in}{1.925000in}}%
\pgfusepath{clip}%
\pgfsetbuttcap%
\pgfsetroundjoin%
\definecolor{currentfill}{rgb}{0.000000,0.000000,0.000000}%
\pgfsetfillcolor{currentfill}%
\pgfsetlinewidth{1.003750pt}%
\definecolor{currentstroke}{rgb}{0.000000,0.000000,0.000000}%
\pgfsetstrokecolor{currentstroke}%
\pgfsetdash{}{0pt}%
\pgfpathmoveto{\pgfqpoint{2.824734in}{0.356667in}}%
\pgfpathcurveto{\pgfqpoint{2.830260in}{0.356667in}}{\pgfqpoint{2.835559in}{0.358862in}}{\pgfqpoint{2.839466in}{0.362769in}}%
\pgfpathcurveto{\pgfqpoint{2.843373in}{0.366675in}}{\pgfqpoint{2.845568in}{0.371975in}}{\pgfqpoint{2.845568in}{0.377500in}}%
\pgfpathcurveto{\pgfqpoint{2.845568in}{0.383025in}}{\pgfqpoint{2.843373in}{0.388325in}}{\pgfqpoint{2.839466in}{0.392231in}}%
\pgfpathcurveto{\pgfqpoint{2.835559in}{0.396138in}}{\pgfqpoint{2.830260in}{0.398333in}}{\pgfqpoint{2.824734in}{0.398333in}}%
\pgfpathcurveto{\pgfqpoint{2.819209in}{0.398333in}}{\pgfqpoint{2.813910in}{0.396138in}}{\pgfqpoint{2.810003in}{0.392231in}}%
\pgfpathcurveto{\pgfqpoint{2.806096in}{0.388325in}}{\pgfqpoint{2.803901in}{0.383025in}}{\pgfqpoint{2.803901in}{0.377500in}}%
\pgfpathcurveto{\pgfqpoint{2.803901in}{0.371975in}}{\pgfqpoint{2.806096in}{0.366675in}}{\pgfqpoint{2.810003in}{0.362769in}}%
\pgfpathcurveto{\pgfqpoint{2.813910in}{0.358862in}}{\pgfqpoint{2.819209in}{0.356667in}}{\pgfqpoint{2.824734in}{0.356667in}}%
\pgfpathclose%
\pgfusepath{stroke,fill}%
\end{pgfscope}%
\begin{pgfscope}%
\pgfpathrectangle{\pgfqpoint{0.562500in}{0.275000in}}{\pgfqpoint{3.487500in}{1.925000in}}%
\pgfusepath{clip}%
\pgfsetbuttcap%
\pgfsetroundjoin%
\definecolor{currentfill}{rgb}{0.000000,0.000000,0.000000}%
\pgfsetfillcolor{currentfill}%
\pgfsetlinewidth{1.003750pt}%
\definecolor{currentstroke}{rgb}{0.000000,0.000000,0.000000}%
\pgfsetstrokecolor{currentstroke}%
\pgfsetdash{}{0pt}%
\pgfpathmoveto{\pgfqpoint{2.824734in}{0.356667in}}%
\pgfpathcurveto{\pgfqpoint{2.830260in}{0.356667in}}{\pgfqpoint{2.835559in}{0.358862in}}{\pgfqpoint{2.839466in}{0.362769in}}%
\pgfpathcurveto{\pgfqpoint{2.843373in}{0.366675in}}{\pgfqpoint{2.845568in}{0.371975in}}{\pgfqpoint{2.845568in}{0.377500in}}%
\pgfpathcurveto{\pgfqpoint{2.845568in}{0.383025in}}{\pgfqpoint{2.843373in}{0.388325in}}{\pgfqpoint{2.839466in}{0.392231in}}%
\pgfpathcurveto{\pgfqpoint{2.835559in}{0.396138in}}{\pgfqpoint{2.830260in}{0.398333in}}{\pgfqpoint{2.824734in}{0.398333in}}%
\pgfpathcurveto{\pgfqpoint{2.819209in}{0.398333in}}{\pgfqpoint{2.813910in}{0.396138in}}{\pgfqpoint{2.810003in}{0.392231in}}%
\pgfpathcurveto{\pgfqpoint{2.806096in}{0.388325in}}{\pgfqpoint{2.803901in}{0.383025in}}{\pgfqpoint{2.803901in}{0.377500in}}%
\pgfpathcurveto{\pgfqpoint{2.803901in}{0.371975in}}{\pgfqpoint{2.806096in}{0.366675in}}{\pgfqpoint{2.810003in}{0.362769in}}%
\pgfpathcurveto{\pgfqpoint{2.813910in}{0.358862in}}{\pgfqpoint{2.819209in}{0.356667in}}{\pgfqpoint{2.824734in}{0.356667in}}%
\pgfpathclose%
\pgfusepath{stroke,fill}%
\end{pgfscope}%
\begin{pgfscope}%
\pgfpathrectangle{\pgfqpoint{0.562500in}{0.275000in}}{\pgfqpoint{3.487500in}{1.925000in}}%
\pgfusepath{clip}%
\pgfsetbuttcap%
\pgfsetroundjoin%
\definecolor{currentfill}{rgb}{0.000000,0.000000,0.000000}%
\pgfsetfillcolor{currentfill}%
\pgfsetlinewidth{1.003750pt}%
\definecolor{currentstroke}{rgb}{0.000000,0.000000,0.000000}%
\pgfsetstrokecolor{currentstroke}%
\pgfsetdash{}{0pt}%
\pgfpathmoveto{\pgfqpoint{2.824734in}{0.356667in}}%
\pgfpathcurveto{\pgfqpoint{2.830260in}{0.356667in}}{\pgfqpoint{2.835559in}{0.358862in}}{\pgfqpoint{2.839466in}{0.362769in}}%
\pgfpathcurveto{\pgfqpoint{2.843373in}{0.366675in}}{\pgfqpoint{2.845568in}{0.371975in}}{\pgfqpoint{2.845568in}{0.377500in}}%
\pgfpathcurveto{\pgfqpoint{2.845568in}{0.383025in}}{\pgfqpoint{2.843373in}{0.388325in}}{\pgfqpoint{2.839466in}{0.392231in}}%
\pgfpathcurveto{\pgfqpoint{2.835559in}{0.396138in}}{\pgfqpoint{2.830260in}{0.398333in}}{\pgfqpoint{2.824734in}{0.398333in}}%
\pgfpathcurveto{\pgfqpoint{2.819209in}{0.398333in}}{\pgfqpoint{2.813910in}{0.396138in}}{\pgfqpoint{2.810003in}{0.392231in}}%
\pgfpathcurveto{\pgfqpoint{2.806096in}{0.388325in}}{\pgfqpoint{2.803901in}{0.383025in}}{\pgfqpoint{2.803901in}{0.377500in}}%
\pgfpathcurveto{\pgfqpoint{2.803901in}{0.371975in}}{\pgfqpoint{2.806096in}{0.366675in}}{\pgfqpoint{2.810003in}{0.362769in}}%
\pgfpathcurveto{\pgfqpoint{2.813910in}{0.358862in}}{\pgfqpoint{2.819209in}{0.356667in}}{\pgfqpoint{2.824734in}{0.356667in}}%
\pgfpathclose%
\pgfusepath{stroke,fill}%
\end{pgfscope}%
\begin{pgfscope}%
\pgfpathrectangle{\pgfqpoint{0.562500in}{0.275000in}}{\pgfqpoint{3.487500in}{1.925000in}}%
\pgfusepath{clip}%
\pgfsetbuttcap%
\pgfsetroundjoin%
\definecolor{currentfill}{rgb}{0.000000,0.000000,0.000000}%
\pgfsetfillcolor{currentfill}%
\pgfsetlinewidth{1.003750pt}%
\definecolor{currentstroke}{rgb}{0.000000,0.000000,0.000000}%
\pgfsetstrokecolor{currentstroke}%
\pgfsetdash{}{0pt}%
\pgfpathmoveto{\pgfqpoint{2.824734in}{0.356667in}}%
\pgfpathcurveto{\pgfqpoint{2.830260in}{0.356667in}}{\pgfqpoint{2.835559in}{0.358862in}}{\pgfqpoint{2.839466in}{0.362769in}}%
\pgfpathcurveto{\pgfqpoint{2.843373in}{0.366675in}}{\pgfqpoint{2.845568in}{0.371975in}}{\pgfqpoint{2.845568in}{0.377500in}}%
\pgfpathcurveto{\pgfqpoint{2.845568in}{0.383025in}}{\pgfqpoint{2.843373in}{0.388325in}}{\pgfqpoint{2.839466in}{0.392231in}}%
\pgfpathcurveto{\pgfqpoint{2.835559in}{0.396138in}}{\pgfqpoint{2.830260in}{0.398333in}}{\pgfqpoint{2.824734in}{0.398333in}}%
\pgfpathcurveto{\pgfqpoint{2.819209in}{0.398333in}}{\pgfqpoint{2.813910in}{0.396138in}}{\pgfqpoint{2.810003in}{0.392231in}}%
\pgfpathcurveto{\pgfqpoint{2.806096in}{0.388325in}}{\pgfqpoint{2.803901in}{0.383025in}}{\pgfqpoint{2.803901in}{0.377500in}}%
\pgfpathcurveto{\pgfqpoint{2.803901in}{0.371975in}}{\pgfqpoint{2.806096in}{0.366675in}}{\pgfqpoint{2.810003in}{0.362769in}}%
\pgfpathcurveto{\pgfqpoint{2.813910in}{0.358862in}}{\pgfqpoint{2.819209in}{0.356667in}}{\pgfqpoint{2.824734in}{0.356667in}}%
\pgfpathclose%
\pgfusepath{stroke,fill}%
\end{pgfscope}%
\begin{pgfscope}%
\pgfpathrectangle{\pgfqpoint{0.562500in}{0.275000in}}{\pgfqpoint{3.487500in}{1.925000in}}%
\pgfusepath{clip}%
\pgfsetbuttcap%
\pgfsetroundjoin%
\definecolor{currentfill}{rgb}{0.000000,0.000000,0.000000}%
\pgfsetfillcolor{currentfill}%
\pgfsetlinewidth{1.003750pt}%
\definecolor{currentstroke}{rgb}{0.000000,0.000000,0.000000}%
\pgfsetstrokecolor{currentstroke}%
\pgfsetdash{}{0pt}%
\pgfpathmoveto{\pgfqpoint{2.824734in}{0.356667in}}%
\pgfpathcurveto{\pgfqpoint{2.830260in}{0.356667in}}{\pgfqpoint{2.835559in}{0.358862in}}{\pgfqpoint{2.839466in}{0.362769in}}%
\pgfpathcurveto{\pgfqpoint{2.843373in}{0.366675in}}{\pgfqpoint{2.845568in}{0.371975in}}{\pgfqpoint{2.845568in}{0.377500in}}%
\pgfpathcurveto{\pgfqpoint{2.845568in}{0.383025in}}{\pgfqpoint{2.843373in}{0.388325in}}{\pgfqpoint{2.839466in}{0.392231in}}%
\pgfpathcurveto{\pgfqpoint{2.835559in}{0.396138in}}{\pgfqpoint{2.830260in}{0.398333in}}{\pgfqpoint{2.824734in}{0.398333in}}%
\pgfpathcurveto{\pgfqpoint{2.819209in}{0.398333in}}{\pgfqpoint{2.813910in}{0.396138in}}{\pgfqpoint{2.810003in}{0.392231in}}%
\pgfpathcurveto{\pgfqpoint{2.806096in}{0.388325in}}{\pgfqpoint{2.803901in}{0.383025in}}{\pgfqpoint{2.803901in}{0.377500in}}%
\pgfpathcurveto{\pgfqpoint{2.803901in}{0.371975in}}{\pgfqpoint{2.806096in}{0.366675in}}{\pgfqpoint{2.810003in}{0.362769in}}%
\pgfpathcurveto{\pgfqpoint{2.813910in}{0.358862in}}{\pgfqpoint{2.819209in}{0.356667in}}{\pgfqpoint{2.824734in}{0.356667in}}%
\pgfpathclose%
\pgfusepath{stroke,fill}%
\end{pgfscope}%
\begin{pgfscope}%
\pgfpathrectangle{\pgfqpoint{0.562500in}{0.275000in}}{\pgfqpoint{3.487500in}{1.925000in}}%
\pgfusepath{clip}%
\pgfsetbuttcap%
\pgfsetroundjoin%
\definecolor{currentfill}{rgb}{0.000000,0.000000,0.000000}%
\pgfsetfillcolor{currentfill}%
\pgfsetlinewidth{1.003750pt}%
\definecolor{currentstroke}{rgb}{0.000000,0.000000,0.000000}%
\pgfsetstrokecolor{currentstroke}%
\pgfsetdash{}{0pt}%
\pgfpathmoveto{\pgfqpoint{2.824734in}{0.356667in}}%
\pgfpathcurveto{\pgfqpoint{2.830260in}{0.356667in}}{\pgfqpoint{2.835559in}{0.358862in}}{\pgfqpoint{2.839466in}{0.362769in}}%
\pgfpathcurveto{\pgfqpoint{2.843373in}{0.366675in}}{\pgfqpoint{2.845568in}{0.371975in}}{\pgfqpoint{2.845568in}{0.377500in}}%
\pgfpathcurveto{\pgfqpoint{2.845568in}{0.383025in}}{\pgfqpoint{2.843373in}{0.388325in}}{\pgfqpoint{2.839466in}{0.392231in}}%
\pgfpathcurveto{\pgfqpoint{2.835559in}{0.396138in}}{\pgfqpoint{2.830260in}{0.398333in}}{\pgfqpoint{2.824734in}{0.398333in}}%
\pgfpathcurveto{\pgfqpoint{2.819209in}{0.398333in}}{\pgfqpoint{2.813910in}{0.396138in}}{\pgfqpoint{2.810003in}{0.392231in}}%
\pgfpathcurveto{\pgfqpoint{2.806096in}{0.388325in}}{\pgfqpoint{2.803901in}{0.383025in}}{\pgfqpoint{2.803901in}{0.377500in}}%
\pgfpathcurveto{\pgfqpoint{2.803901in}{0.371975in}}{\pgfqpoint{2.806096in}{0.366675in}}{\pgfqpoint{2.810003in}{0.362769in}}%
\pgfpathcurveto{\pgfqpoint{2.813910in}{0.358862in}}{\pgfqpoint{2.819209in}{0.356667in}}{\pgfqpoint{2.824734in}{0.356667in}}%
\pgfpathclose%
\pgfusepath{stroke,fill}%
\end{pgfscope}%
\begin{pgfscope}%
\pgfpathrectangle{\pgfqpoint{0.562500in}{0.275000in}}{\pgfqpoint{3.487500in}{1.925000in}}%
\pgfusepath{clip}%
\pgfsetbuttcap%
\pgfsetroundjoin%
\definecolor{currentfill}{rgb}{0.000000,0.000000,0.000000}%
\pgfsetfillcolor{currentfill}%
\pgfsetlinewidth{1.003750pt}%
\definecolor{currentstroke}{rgb}{0.000000,0.000000,0.000000}%
\pgfsetstrokecolor{currentstroke}%
\pgfsetdash{}{0pt}%
\pgfpathmoveto{\pgfqpoint{2.824734in}{0.356667in}}%
\pgfpathcurveto{\pgfqpoint{2.830260in}{0.356667in}}{\pgfqpoint{2.835559in}{0.358862in}}{\pgfqpoint{2.839466in}{0.362769in}}%
\pgfpathcurveto{\pgfqpoint{2.843373in}{0.366675in}}{\pgfqpoint{2.845568in}{0.371975in}}{\pgfqpoint{2.845568in}{0.377500in}}%
\pgfpathcurveto{\pgfqpoint{2.845568in}{0.383025in}}{\pgfqpoint{2.843373in}{0.388325in}}{\pgfqpoint{2.839466in}{0.392231in}}%
\pgfpathcurveto{\pgfqpoint{2.835559in}{0.396138in}}{\pgfqpoint{2.830260in}{0.398333in}}{\pgfqpoint{2.824734in}{0.398333in}}%
\pgfpathcurveto{\pgfqpoint{2.819209in}{0.398333in}}{\pgfqpoint{2.813910in}{0.396138in}}{\pgfqpoint{2.810003in}{0.392231in}}%
\pgfpathcurveto{\pgfqpoint{2.806096in}{0.388325in}}{\pgfqpoint{2.803901in}{0.383025in}}{\pgfqpoint{2.803901in}{0.377500in}}%
\pgfpathcurveto{\pgfqpoint{2.803901in}{0.371975in}}{\pgfqpoint{2.806096in}{0.366675in}}{\pgfqpoint{2.810003in}{0.362769in}}%
\pgfpathcurveto{\pgfqpoint{2.813910in}{0.358862in}}{\pgfqpoint{2.819209in}{0.356667in}}{\pgfqpoint{2.824734in}{0.356667in}}%
\pgfpathclose%
\pgfusepath{stroke,fill}%
\end{pgfscope}%
\begin{pgfscope}%
\pgfpathrectangle{\pgfqpoint{0.562500in}{0.275000in}}{\pgfqpoint{3.487500in}{1.925000in}}%
\pgfusepath{clip}%
\pgfsetbuttcap%
\pgfsetroundjoin%
\definecolor{currentfill}{rgb}{0.000000,0.000000,0.000000}%
\pgfsetfillcolor{currentfill}%
\pgfsetlinewidth{1.003750pt}%
\definecolor{currentstroke}{rgb}{0.000000,0.000000,0.000000}%
\pgfsetstrokecolor{currentstroke}%
\pgfsetdash{}{0pt}%
\pgfpathmoveto{\pgfqpoint{2.824734in}{0.356667in}}%
\pgfpathcurveto{\pgfqpoint{2.830260in}{0.356667in}}{\pgfqpoint{2.835559in}{0.358862in}}{\pgfqpoint{2.839466in}{0.362769in}}%
\pgfpathcurveto{\pgfqpoint{2.843373in}{0.366675in}}{\pgfqpoint{2.845568in}{0.371975in}}{\pgfqpoint{2.845568in}{0.377500in}}%
\pgfpathcurveto{\pgfqpoint{2.845568in}{0.383025in}}{\pgfqpoint{2.843373in}{0.388325in}}{\pgfqpoint{2.839466in}{0.392231in}}%
\pgfpathcurveto{\pgfqpoint{2.835559in}{0.396138in}}{\pgfqpoint{2.830260in}{0.398333in}}{\pgfqpoint{2.824734in}{0.398333in}}%
\pgfpathcurveto{\pgfqpoint{2.819209in}{0.398333in}}{\pgfqpoint{2.813910in}{0.396138in}}{\pgfqpoint{2.810003in}{0.392231in}}%
\pgfpathcurveto{\pgfqpoint{2.806096in}{0.388325in}}{\pgfqpoint{2.803901in}{0.383025in}}{\pgfqpoint{2.803901in}{0.377500in}}%
\pgfpathcurveto{\pgfqpoint{2.803901in}{0.371975in}}{\pgfqpoint{2.806096in}{0.366675in}}{\pgfqpoint{2.810003in}{0.362769in}}%
\pgfpathcurveto{\pgfqpoint{2.813910in}{0.358862in}}{\pgfqpoint{2.819209in}{0.356667in}}{\pgfqpoint{2.824734in}{0.356667in}}%
\pgfpathclose%
\pgfusepath{stroke,fill}%
\end{pgfscope}%
\begin{pgfscope}%
\pgfpathrectangle{\pgfqpoint{0.562500in}{0.275000in}}{\pgfqpoint{3.487500in}{1.925000in}}%
\pgfusepath{clip}%
\pgfsetbuttcap%
\pgfsetroundjoin%
\definecolor{currentfill}{rgb}{0.000000,0.000000,0.000000}%
\pgfsetfillcolor{currentfill}%
\pgfsetlinewidth{1.003750pt}%
\definecolor{currentstroke}{rgb}{0.000000,0.000000,0.000000}%
\pgfsetstrokecolor{currentstroke}%
\pgfsetdash{}{0pt}%
\pgfpathmoveto{\pgfqpoint{2.824734in}{0.356667in}}%
\pgfpathcurveto{\pgfqpoint{2.830260in}{0.356667in}}{\pgfqpoint{2.835559in}{0.358862in}}{\pgfqpoint{2.839466in}{0.362769in}}%
\pgfpathcurveto{\pgfqpoint{2.843373in}{0.366675in}}{\pgfqpoint{2.845568in}{0.371975in}}{\pgfqpoint{2.845568in}{0.377500in}}%
\pgfpathcurveto{\pgfqpoint{2.845568in}{0.383025in}}{\pgfqpoint{2.843373in}{0.388325in}}{\pgfqpoint{2.839466in}{0.392231in}}%
\pgfpathcurveto{\pgfqpoint{2.835559in}{0.396138in}}{\pgfqpoint{2.830260in}{0.398333in}}{\pgfqpoint{2.824734in}{0.398333in}}%
\pgfpathcurveto{\pgfqpoint{2.819209in}{0.398333in}}{\pgfqpoint{2.813910in}{0.396138in}}{\pgfqpoint{2.810003in}{0.392231in}}%
\pgfpathcurveto{\pgfqpoint{2.806096in}{0.388325in}}{\pgfqpoint{2.803901in}{0.383025in}}{\pgfqpoint{2.803901in}{0.377500in}}%
\pgfpathcurveto{\pgfqpoint{2.803901in}{0.371975in}}{\pgfqpoint{2.806096in}{0.366675in}}{\pgfqpoint{2.810003in}{0.362769in}}%
\pgfpathcurveto{\pgfqpoint{2.813910in}{0.358862in}}{\pgfqpoint{2.819209in}{0.356667in}}{\pgfqpoint{2.824734in}{0.356667in}}%
\pgfpathclose%
\pgfusepath{stroke,fill}%
\end{pgfscope}%
\begin{pgfscope}%
\pgfpathrectangle{\pgfqpoint{0.562500in}{0.275000in}}{\pgfqpoint{3.487500in}{1.925000in}}%
\pgfusepath{clip}%
\pgfsetbuttcap%
\pgfsetroundjoin%
\definecolor{currentfill}{rgb}{0.000000,0.000000,0.000000}%
\pgfsetfillcolor{currentfill}%
\pgfsetlinewidth{1.003750pt}%
\definecolor{currentstroke}{rgb}{0.000000,0.000000,0.000000}%
\pgfsetstrokecolor{currentstroke}%
\pgfsetdash{}{0pt}%
\pgfpathmoveto{\pgfqpoint{2.824734in}{0.356667in}}%
\pgfpathcurveto{\pgfqpoint{2.830260in}{0.356667in}}{\pgfqpoint{2.835559in}{0.358862in}}{\pgfqpoint{2.839466in}{0.362769in}}%
\pgfpathcurveto{\pgfqpoint{2.843373in}{0.366675in}}{\pgfqpoint{2.845568in}{0.371975in}}{\pgfqpoint{2.845568in}{0.377500in}}%
\pgfpathcurveto{\pgfqpoint{2.845568in}{0.383025in}}{\pgfqpoint{2.843373in}{0.388325in}}{\pgfqpoint{2.839466in}{0.392231in}}%
\pgfpathcurveto{\pgfqpoint{2.835559in}{0.396138in}}{\pgfqpoint{2.830260in}{0.398333in}}{\pgfqpoint{2.824734in}{0.398333in}}%
\pgfpathcurveto{\pgfqpoint{2.819209in}{0.398333in}}{\pgfqpoint{2.813910in}{0.396138in}}{\pgfqpoint{2.810003in}{0.392231in}}%
\pgfpathcurveto{\pgfqpoint{2.806096in}{0.388325in}}{\pgfqpoint{2.803901in}{0.383025in}}{\pgfqpoint{2.803901in}{0.377500in}}%
\pgfpathcurveto{\pgfqpoint{2.803901in}{0.371975in}}{\pgfqpoint{2.806096in}{0.366675in}}{\pgfqpoint{2.810003in}{0.362769in}}%
\pgfpathcurveto{\pgfqpoint{2.813910in}{0.358862in}}{\pgfqpoint{2.819209in}{0.356667in}}{\pgfqpoint{2.824734in}{0.356667in}}%
\pgfpathclose%
\pgfusepath{stroke,fill}%
\end{pgfscope}%
\begin{pgfscope}%
\pgfpathrectangle{\pgfqpoint{0.562500in}{0.275000in}}{\pgfqpoint{3.487500in}{1.925000in}}%
\pgfusepath{clip}%
\pgfsetbuttcap%
\pgfsetroundjoin%
\definecolor{currentfill}{rgb}{0.000000,0.000000,0.000000}%
\pgfsetfillcolor{currentfill}%
\pgfsetlinewidth{1.003750pt}%
\definecolor{currentstroke}{rgb}{0.000000,0.000000,0.000000}%
\pgfsetstrokecolor{currentstroke}%
\pgfsetdash{}{0pt}%
\pgfpathmoveto{\pgfqpoint{2.824734in}{0.356667in}}%
\pgfpathcurveto{\pgfqpoint{2.830260in}{0.356667in}}{\pgfqpoint{2.835559in}{0.358862in}}{\pgfqpoint{2.839466in}{0.362769in}}%
\pgfpathcurveto{\pgfqpoint{2.843373in}{0.366675in}}{\pgfqpoint{2.845568in}{0.371975in}}{\pgfqpoint{2.845568in}{0.377500in}}%
\pgfpathcurveto{\pgfqpoint{2.845568in}{0.383025in}}{\pgfqpoint{2.843373in}{0.388325in}}{\pgfqpoint{2.839466in}{0.392231in}}%
\pgfpathcurveto{\pgfqpoint{2.835559in}{0.396138in}}{\pgfqpoint{2.830260in}{0.398333in}}{\pgfqpoint{2.824734in}{0.398333in}}%
\pgfpathcurveto{\pgfqpoint{2.819209in}{0.398333in}}{\pgfqpoint{2.813910in}{0.396138in}}{\pgfqpoint{2.810003in}{0.392231in}}%
\pgfpathcurveto{\pgfqpoint{2.806096in}{0.388325in}}{\pgfqpoint{2.803901in}{0.383025in}}{\pgfqpoint{2.803901in}{0.377500in}}%
\pgfpathcurveto{\pgfqpoint{2.803901in}{0.371975in}}{\pgfqpoint{2.806096in}{0.366675in}}{\pgfqpoint{2.810003in}{0.362769in}}%
\pgfpathcurveto{\pgfqpoint{2.813910in}{0.358862in}}{\pgfqpoint{2.819209in}{0.356667in}}{\pgfqpoint{2.824734in}{0.356667in}}%
\pgfpathclose%
\pgfusepath{stroke,fill}%
\end{pgfscope}%
\begin{pgfscope}%
\pgfpathrectangle{\pgfqpoint{0.562500in}{0.275000in}}{\pgfqpoint{3.487500in}{1.925000in}}%
\pgfusepath{clip}%
\pgfsetbuttcap%
\pgfsetroundjoin%
\definecolor{currentfill}{rgb}{0.000000,0.000000,0.000000}%
\pgfsetfillcolor{currentfill}%
\pgfsetlinewidth{1.003750pt}%
\definecolor{currentstroke}{rgb}{0.000000,0.000000,0.000000}%
\pgfsetstrokecolor{currentstroke}%
\pgfsetdash{}{0pt}%
\pgfpathmoveto{\pgfqpoint{2.824734in}{0.356667in}}%
\pgfpathcurveto{\pgfqpoint{2.830260in}{0.356667in}}{\pgfqpoint{2.835559in}{0.358862in}}{\pgfqpoint{2.839466in}{0.362769in}}%
\pgfpathcurveto{\pgfqpoint{2.843373in}{0.366675in}}{\pgfqpoint{2.845568in}{0.371975in}}{\pgfqpoint{2.845568in}{0.377500in}}%
\pgfpathcurveto{\pgfqpoint{2.845568in}{0.383025in}}{\pgfqpoint{2.843373in}{0.388325in}}{\pgfqpoint{2.839466in}{0.392231in}}%
\pgfpathcurveto{\pgfqpoint{2.835559in}{0.396138in}}{\pgfqpoint{2.830260in}{0.398333in}}{\pgfqpoint{2.824734in}{0.398333in}}%
\pgfpathcurveto{\pgfqpoint{2.819209in}{0.398333in}}{\pgfqpoint{2.813910in}{0.396138in}}{\pgfqpoint{2.810003in}{0.392231in}}%
\pgfpathcurveto{\pgfqpoint{2.806096in}{0.388325in}}{\pgfqpoint{2.803901in}{0.383025in}}{\pgfqpoint{2.803901in}{0.377500in}}%
\pgfpathcurveto{\pgfqpoint{2.803901in}{0.371975in}}{\pgfqpoint{2.806096in}{0.366675in}}{\pgfqpoint{2.810003in}{0.362769in}}%
\pgfpathcurveto{\pgfqpoint{2.813910in}{0.358862in}}{\pgfqpoint{2.819209in}{0.356667in}}{\pgfqpoint{2.824734in}{0.356667in}}%
\pgfpathclose%
\pgfusepath{stroke,fill}%
\end{pgfscope}%
\begin{pgfscope}%
\pgfpathrectangle{\pgfqpoint{0.562500in}{0.275000in}}{\pgfqpoint{3.487500in}{1.925000in}}%
\pgfusepath{clip}%
\pgfsetbuttcap%
\pgfsetroundjoin%
\definecolor{currentfill}{rgb}{0.000000,0.000000,0.000000}%
\pgfsetfillcolor{currentfill}%
\pgfsetlinewidth{1.003750pt}%
\definecolor{currentstroke}{rgb}{0.000000,0.000000,0.000000}%
\pgfsetstrokecolor{currentstroke}%
\pgfsetdash{}{0pt}%
\pgfpathmoveto{\pgfqpoint{2.824734in}{0.356667in}}%
\pgfpathcurveto{\pgfqpoint{2.830260in}{0.356667in}}{\pgfqpoint{2.835559in}{0.358862in}}{\pgfqpoint{2.839466in}{0.362769in}}%
\pgfpathcurveto{\pgfqpoint{2.843373in}{0.366675in}}{\pgfqpoint{2.845568in}{0.371975in}}{\pgfqpoint{2.845568in}{0.377500in}}%
\pgfpathcurveto{\pgfqpoint{2.845568in}{0.383025in}}{\pgfqpoint{2.843373in}{0.388325in}}{\pgfqpoint{2.839466in}{0.392231in}}%
\pgfpathcurveto{\pgfqpoint{2.835559in}{0.396138in}}{\pgfqpoint{2.830260in}{0.398333in}}{\pgfqpoint{2.824734in}{0.398333in}}%
\pgfpathcurveto{\pgfqpoint{2.819209in}{0.398333in}}{\pgfqpoint{2.813910in}{0.396138in}}{\pgfqpoint{2.810003in}{0.392231in}}%
\pgfpathcurveto{\pgfqpoint{2.806096in}{0.388325in}}{\pgfqpoint{2.803901in}{0.383025in}}{\pgfqpoint{2.803901in}{0.377500in}}%
\pgfpathcurveto{\pgfqpoint{2.803901in}{0.371975in}}{\pgfqpoint{2.806096in}{0.366675in}}{\pgfqpoint{2.810003in}{0.362769in}}%
\pgfpathcurveto{\pgfqpoint{2.813910in}{0.358862in}}{\pgfqpoint{2.819209in}{0.356667in}}{\pgfqpoint{2.824734in}{0.356667in}}%
\pgfpathclose%
\pgfusepath{stroke,fill}%
\end{pgfscope}%
\begin{pgfscope}%
\pgfpathrectangle{\pgfqpoint{0.562500in}{0.275000in}}{\pgfqpoint{3.487500in}{1.925000in}}%
\pgfusepath{clip}%
\pgfsetbuttcap%
\pgfsetroundjoin%
\definecolor{currentfill}{rgb}{0.000000,0.000000,0.000000}%
\pgfsetfillcolor{currentfill}%
\pgfsetlinewidth{1.003750pt}%
\definecolor{currentstroke}{rgb}{0.000000,0.000000,0.000000}%
\pgfsetstrokecolor{currentstroke}%
\pgfsetdash{}{0pt}%
\pgfpathmoveto{\pgfqpoint{2.824734in}{0.356667in}}%
\pgfpathcurveto{\pgfqpoint{2.830260in}{0.356667in}}{\pgfqpoint{2.835559in}{0.358862in}}{\pgfqpoint{2.839466in}{0.362769in}}%
\pgfpathcurveto{\pgfqpoint{2.843373in}{0.366675in}}{\pgfqpoint{2.845568in}{0.371975in}}{\pgfqpoint{2.845568in}{0.377500in}}%
\pgfpathcurveto{\pgfqpoint{2.845568in}{0.383025in}}{\pgfqpoint{2.843373in}{0.388325in}}{\pgfqpoint{2.839466in}{0.392231in}}%
\pgfpathcurveto{\pgfqpoint{2.835559in}{0.396138in}}{\pgfqpoint{2.830260in}{0.398333in}}{\pgfqpoint{2.824734in}{0.398333in}}%
\pgfpathcurveto{\pgfqpoint{2.819209in}{0.398333in}}{\pgfqpoint{2.813910in}{0.396138in}}{\pgfqpoint{2.810003in}{0.392231in}}%
\pgfpathcurveto{\pgfqpoint{2.806096in}{0.388325in}}{\pgfqpoint{2.803901in}{0.383025in}}{\pgfqpoint{2.803901in}{0.377500in}}%
\pgfpathcurveto{\pgfqpoint{2.803901in}{0.371975in}}{\pgfqpoint{2.806096in}{0.366675in}}{\pgfqpoint{2.810003in}{0.362769in}}%
\pgfpathcurveto{\pgfqpoint{2.813910in}{0.358862in}}{\pgfqpoint{2.819209in}{0.356667in}}{\pgfqpoint{2.824734in}{0.356667in}}%
\pgfpathclose%
\pgfusepath{stroke,fill}%
\end{pgfscope}%
\begin{pgfscope}%
\pgfpathrectangle{\pgfqpoint{0.562500in}{0.275000in}}{\pgfqpoint{3.487500in}{1.925000in}}%
\pgfusepath{clip}%
\pgfsetbuttcap%
\pgfsetroundjoin%
\definecolor{currentfill}{rgb}{0.000000,0.000000,0.000000}%
\pgfsetfillcolor{currentfill}%
\pgfsetlinewidth{1.003750pt}%
\definecolor{currentstroke}{rgb}{0.000000,0.000000,0.000000}%
\pgfsetstrokecolor{currentstroke}%
\pgfsetdash{}{0pt}%
\pgfpathmoveto{\pgfqpoint{2.824734in}{0.356667in}}%
\pgfpathcurveto{\pgfqpoint{2.830260in}{0.356667in}}{\pgfqpoint{2.835559in}{0.358862in}}{\pgfqpoint{2.839466in}{0.362769in}}%
\pgfpathcurveto{\pgfqpoint{2.843373in}{0.366675in}}{\pgfqpoint{2.845568in}{0.371975in}}{\pgfqpoint{2.845568in}{0.377500in}}%
\pgfpathcurveto{\pgfqpoint{2.845568in}{0.383025in}}{\pgfqpoint{2.843373in}{0.388325in}}{\pgfqpoint{2.839466in}{0.392231in}}%
\pgfpathcurveto{\pgfqpoint{2.835559in}{0.396138in}}{\pgfqpoint{2.830260in}{0.398333in}}{\pgfqpoint{2.824734in}{0.398333in}}%
\pgfpathcurveto{\pgfqpoint{2.819209in}{0.398333in}}{\pgfqpoint{2.813910in}{0.396138in}}{\pgfqpoint{2.810003in}{0.392231in}}%
\pgfpathcurveto{\pgfqpoint{2.806096in}{0.388325in}}{\pgfqpoint{2.803901in}{0.383025in}}{\pgfqpoint{2.803901in}{0.377500in}}%
\pgfpathcurveto{\pgfqpoint{2.803901in}{0.371975in}}{\pgfqpoint{2.806096in}{0.366675in}}{\pgfqpoint{2.810003in}{0.362769in}}%
\pgfpathcurveto{\pgfqpoint{2.813910in}{0.358862in}}{\pgfqpoint{2.819209in}{0.356667in}}{\pgfqpoint{2.824734in}{0.356667in}}%
\pgfpathclose%
\pgfusepath{stroke,fill}%
\end{pgfscope}%
\begin{pgfscope}%
\pgfpathrectangle{\pgfqpoint{0.562500in}{0.275000in}}{\pgfqpoint{3.487500in}{1.925000in}}%
\pgfusepath{clip}%
\pgfsetbuttcap%
\pgfsetroundjoin%
\definecolor{currentfill}{rgb}{0.000000,0.000000,0.000000}%
\pgfsetfillcolor{currentfill}%
\pgfsetlinewidth{1.003750pt}%
\definecolor{currentstroke}{rgb}{0.000000,0.000000,0.000000}%
\pgfsetstrokecolor{currentstroke}%
\pgfsetdash{}{0pt}%
\pgfpathmoveto{\pgfqpoint{2.824734in}{0.356667in}}%
\pgfpathcurveto{\pgfqpoint{2.830260in}{0.356667in}}{\pgfqpoint{2.835559in}{0.358862in}}{\pgfqpoint{2.839466in}{0.362769in}}%
\pgfpathcurveto{\pgfqpoint{2.843373in}{0.366675in}}{\pgfqpoint{2.845568in}{0.371975in}}{\pgfqpoint{2.845568in}{0.377500in}}%
\pgfpathcurveto{\pgfqpoint{2.845568in}{0.383025in}}{\pgfqpoint{2.843373in}{0.388325in}}{\pgfqpoint{2.839466in}{0.392231in}}%
\pgfpathcurveto{\pgfqpoint{2.835559in}{0.396138in}}{\pgfqpoint{2.830260in}{0.398333in}}{\pgfqpoint{2.824734in}{0.398333in}}%
\pgfpathcurveto{\pgfqpoint{2.819209in}{0.398333in}}{\pgfqpoint{2.813910in}{0.396138in}}{\pgfqpoint{2.810003in}{0.392231in}}%
\pgfpathcurveto{\pgfqpoint{2.806096in}{0.388325in}}{\pgfqpoint{2.803901in}{0.383025in}}{\pgfqpoint{2.803901in}{0.377500in}}%
\pgfpathcurveto{\pgfqpoint{2.803901in}{0.371975in}}{\pgfqpoint{2.806096in}{0.366675in}}{\pgfqpoint{2.810003in}{0.362769in}}%
\pgfpathcurveto{\pgfqpoint{2.813910in}{0.358862in}}{\pgfqpoint{2.819209in}{0.356667in}}{\pgfqpoint{2.824734in}{0.356667in}}%
\pgfpathclose%
\pgfusepath{stroke,fill}%
\end{pgfscope}%
\begin{pgfscope}%
\pgfpathrectangle{\pgfqpoint{0.562500in}{0.275000in}}{\pgfqpoint{3.487500in}{1.925000in}}%
\pgfusepath{clip}%
\pgfsetbuttcap%
\pgfsetroundjoin%
\definecolor{currentfill}{rgb}{0.000000,0.000000,0.000000}%
\pgfsetfillcolor{currentfill}%
\pgfsetlinewidth{1.003750pt}%
\definecolor{currentstroke}{rgb}{0.000000,0.000000,0.000000}%
\pgfsetstrokecolor{currentstroke}%
\pgfsetdash{}{0pt}%
\pgfpathmoveto{\pgfqpoint{2.824734in}{0.356667in}}%
\pgfpathcurveto{\pgfqpoint{2.830260in}{0.356667in}}{\pgfqpoint{2.835559in}{0.358862in}}{\pgfqpoint{2.839466in}{0.362769in}}%
\pgfpathcurveto{\pgfqpoint{2.843373in}{0.366675in}}{\pgfqpoint{2.845568in}{0.371975in}}{\pgfqpoint{2.845568in}{0.377500in}}%
\pgfpathcurveto{\pgfqpoint{2.845568in}{0.383025in}}{\pgfqpoint{2.843373in}{0.388325in}}{\pgfqpoint{2.839466in}{0.392231in}}%
\pgfpathcurveto{\pgfqpoint{2.835559in}{0.396138in}}{\pgfqpoint{2.830260in}{0.398333in}}{\pgfqpoint{2.824734in}{0.398333in}}%
\pgfpathcurveto{\pgfqpoint{2.819209in}{0.398333in}}{\pgfqpoint{2.813910in}{0.396138in}}{\pgfqpoint{2.810003in}{0.392231in}}%
\pgfpathcurveto{\pgfqpoint{2.806096in}{0.388325in}}{\pgfqpoint{2.803901in}{0.383025in}}{\pgfqpoint{2.803901in}{0.377500in}}%
\pgfpathcurveto{\pgfqpoint{2.803901in}{0.371975in}}{\pgfqpoint{2.806096in}{0.366675in}}{\pgfqpoint{2.810003in}{0.362769in}}%
\pgfpathcurveto{\pgfqpoint{2.813910in}{0.358862in}}{\pgfqpoint{2.819209in}{0.356667in}}{\pgfqpoint{2.824734in}{0.356667in}}%
\pgfpathclose%
\pgfusepath{stroke,fill}%
\end{pgfscope}%
\begin{pgfscope}%
\pgfpathrectangle{\pgfqpoint{0.562500in}{0.275000in}}{\pgfqpoint{3.487500in}{1.925000in}}%
\pgfusepath{clip}%
\pgfsetbuttcap%
\pgfsetroundjoin%
\definecolor{currentfill}{rgb}{0.000000,0.000000,0.000000}%
\pgfsetfillcolor{currentfill}%
\pgfsetlinewidth{1.003750pt}%
\definecolor{currentstroke}{rgb}{0.000000,0.000000,0.000000}%
\pgfsetstrokecolor{currentstroke}%
\pgfsetdash{}{0pt}%
\pgfpathmoveto{\pgfqpoint{2.824734in}{0.356667in}}%
\pgfpathcurveto{\pgfqpoint{2.830260in}{0.356667in}}{\pgfqpoint{2.835559in}{0.358862in}}{\pgfqpoint{2.839466in}{0.362769in}}%
\pgfpathcurveto{\pgfqpoint{2.843373in}{0.366675in}}{\pgfqpoint{2.845568in}{0.371975in}}{\pgfqpoint{2.845568in}{0.377500in}}%
\pgfpathcurveto{\pgfqpoint{2.845568in}{0.383025in}}{\pgfqpoint{2.843373in}{0.388325in}}{\pgfqpoint{2.839466in}{0.392231in}}%
\pgfpathcurveto{\pgfqpoint{2.835559in}{0.396138in}}{\pgfqpoint{2.830260in}{0.398333in}}{\pgfqpoint{2.824734in}{0.398333in}}%
\pgfpathcurveto{\pgfqpoint{2.819209in}{0.398333in}}{\pgfqpoint{2.813910in}{0.396138in}}{\pgfqpoint{2.810003in}{0.392231in}}%
\pgfpathcurveto{\pgfqpoint{2.806096in}{0.388325in}}{\pgfqpoint{2.803901in}{0.383025in}}{\pgfqpoint{2.803901in}{0.377500in}}%
\pgfpathcurveto{\pgfqpoint{2.803901in}{0.371975in}}{\pgfqpoint{2.806096in}{0.366675in}}{\pgfqpoint{2.810003in}{0.362769in}}%
\pgfpathcurveto{\pgfqpoint{2.813910in}{0.358862in}}{\pgfqpoint{2.819209in}{0.356667in}}{\pgfqpoint{2.824734in}{0.356667in}}%
\pgfpathclose%
\pgfusepath{stroke,fill}%
\end{pgfscope}%
\begin{pgfscope}%
\pgfpathrectangle{\pgfqpoint{0.562500in}{0.275000in}}{\pgfqpoint{3.487500in}{1.925000in}}%
\pgfusepath{clip}%
\pgfsetbuttcap%
\pgfsetroundjoin%
\definecolor{currentfill}{rgb}{0.000000,0.000000,0.000000}%
\pgfsetfillcolor{currentfill}%
\pgfsetlinewidth{1.003750pt}%
\definecolor{currentstroke}{rgb}{0.000000,0.000000,0.000000}%
\pgfsetstrokecolor{currentstroke}%
\pgfsetdash{}{0pt}%
\pgfpathmoveto{\pgfqpoint{2.824734in}{0.356667in}}%
\pgfpathcurveto{\pgfqpoint{2.830260in}{0.356667in}}{\pgfqpoint{2.835559in}{0.358862in}}{\pgfqpoint{2.839466in}{0.362769in}}%
\pgfpathcurveto{\pgfqpoint{2.843373in}{0.366675in}}{\pgfqpoint{2.845568in}{0.371975in}}{\pgfqpoint{2.845568in}{0.377500in}}%
\pgfpathcurveto{\pgfqpoint{2.845568in}{0.383025in}}{\pgfqpoint{2.843373in}{0.388325in}}{\pgfqpoint{2.839466in}{0.392231in}}%
\pgfpathcurveto{\pgfqpoint{2.835559in}{0.396138in}}{\pgfqpoint{2.830260in}{0.398333in}}{\pgfqpoint{2.824734in}{0.398333in}}%
\pgfpathcurveto{\pgfqpoint{2.819209in}{0.398333in}}{\pgfqpoint{2.813910in}{0.396138in}}{\pgfqpoint{2.810003in}{0.392231in}}%
\pgfpathcurveto{\pgfqpoint{2.806096in}{0.388325in}}{\pgfqpoint{2.803901in}{0.383025in}}{\pgfqpoint{2.803901in}{0.377500in}}%
\pgfpathcurveto{\pgfqpoint{2.803901in}{0.371975in}}{\pgfqpoint{2.806096in}{0.366675in}}{\pgfqpoint{2.810003in}{0.362769in}}%
\pgfpathcurveto{\pgfqpoint{2.813910in}{0.358862in}}{\pgfqpoint{2.819209in}{0.356667in}}{\pgfqpoint{2.824734in}{0.356667in}}%
\pgfpathclose%
\pgfusepath{stroke,fill}%
\end{pgfscope}%
\begin{pgfscope}%
\pgfpathrectangle{\pgfqpoint{0.562500in}{0.275000in}}{\pgfqpoint{3.487500in}{1.925000in}}%
\pgfusepath{clip}%
\pgfsetbuttcap%
\pgfsetroundjoin%
\definecolor{currentfill}{rgb}{0.000000,0.000000,0.000000}%
\pgfsetfillcolor{currentfill}%
\pgfsetlinewidth{1.003750pt}%
\definecolor{currentstroke}{rgb}{0.000000,0.000000,0.000000}%
\pgfsetstrokecolor{currentstroke}%
\pgfsetdash{}{0pt}%
\pgfpathmoveto{\pgfqpoint{2.824734in}{0.356667in}}%
\pgfpathcurveto{\pgfqpoint{2.830260in}{0.356667in}}{\pgfqpoint{2.835559in}{0.358862in}}{\pgfqpoint{2.839466in}{0.362769in}}%
\pgfpathcurveto{\pgfqpoint{2.843373in}{0.366675in}}{\pgfqpoint{2.845568in}{0.371975in}}{\pgfqpoint{2.845568in}{0.377500in}}%
\pgfpathcurveto{\pgfqpoint{2.845568in}{0.383025in}}{\pgfqpoint{2.843373in}{0.388325in}}{\pgfqpoint{2.839466in}{0.392231in}}%
\pgfpathcurveto{\pgfqpoint{2.835559in}{0.396138in}}{\pgfqpoint{2.830260in}{0.398333in}}{\pgfqpoint{2.824734in}{0.398333in}}%
\pgfpathcurveto{\pgfqpoint{2.819209in}{0.398333in}}{\pgfqpoint{2.813910in}{0.396138in}}{\pgfqpoint{2.810003in}{0.392231in}}%
\pgfpathcurveto{\pgfqpoint{2.806096in}{0.388325in}}{\pgfqpoint{2.803901in}{0.383025in}}{\pgfqpoint{2.803901in}{0.377500in}}%
\pgfpathcurveto{\pgfqpoint{2.803901in}{0.371975in}}{\pgfqpoint{2.806096in}{0.366675in}}{\pgfqpoint{2.810003in}{0.362769in}}%
\pgfpathcurveto{\pgfqpoint{2.813910in}{0.358862in}}{\pgfqpoint{2.819209in}{0.356667in}}{\pgfqpoint{2.824734in}{0.356667in}}%
\pgfpathclose%
\pgfusepath{stroke,fill}%
\end{pgfscope}%
\begin{pgfscope}%
\pgfpathrectangle{\pgfqpoint{0.562500in}{0.275000in}}{\pgfqpoint{3.487500in}{1.925000in}}%
\pgfusepath{clip}%
\pgfsetbuttcap%
\pgfsetroundjoin%
\definecolor{currentfill}{rgb}{0.000000,0.000000,0.000000}%
\pgfsetfillcolor{currentfill}%
\pgfsetlinewidth{1.003750pt}%
\definecolor{currentstroke}{rgb}{0.000000,0.000000,0.000000}%
\pgfsetstrokecolor{currentstroke}%
\pgfsetdash{}{0pt}%
\pgfpathmoveto{\pgfqpoint{2.824734in}{0.356667in}}%
\pgfpathcurveto{\pgfqpoint{2.830260in}{0.356667in}}{\pgfqpoint{2.835559in}{0.358862in}}{\pgfqpoint{2.839466in}{0.362769in}}%
\pgfpathcurveto{\pgfqpoint{2.843373in}{0.366675in}}{\pgfqpoint{2.845568in}{0.371975in}}{\pgfqpoint{2.845568in}{0.377500in}}%
\pgfpathcurveto{\pgfqpoint{2.845568in}{0.383025in}}{\pgfqpoint{2.843373in}{0.388325in}}{\pgfqpoint{2.839466in}{0.392231in}}%
\pgfpathcurveto{\pgfqpoint{2.835559in}{0.396138in}}{\pgfqpoint{2.830260in}{0.398333in}}{\pgfqpoint{2.824734in}{0.398333in}}%
\pgfpathcurveto{\pgfqpoint{2.819209in}{0.398333in}}{\pgfqpoint{2.813910in}{0.396138in}}{\pgfqpoint{2.810003in}{0.392231in}}%
\pgfpathcurveto{\pgfqpoint{2.806096in}{0.388325in}}{\pgfqpoint{2.803901in}{0.383025in}}{\pgfqpoint{2.803901in}{0.377500in}}%
\pgfpathcurveto{\pgfqpoint{2.803901in}{0.371975in}}{\pgfqpoint{2.806096in}{0.366675in}}{\pgfqpoint{2.810003in}{0.362769in}}%
\pgfpathcurveto{\pgfqpoint{2.813910in}{0.358862in}}{\pgfqpoint{2.819209in}{0.356667in}}{\pgfqpoint{2.824734in}{0.356667in}}%
\pgfpathclose%
\pgfusepath{stroke,fill}%
\end{pgfscope}%
\begin{pgfscope}%
\pgfpathrectangle{\pgfqpoint{0.562500in}{0.275000in}}{\pgfqpoint{3.487500in}{1.925000in}}%
\pgfusepath{clip}%
\pgfsetbuttcap%
\pgfsetroundjoin%
\definecolor{currentfill}{rgb}{0.000000,0.000000,0.000000}%
\pgfsetfillcolor{currentfill}%
\pgfsetlinewidth{1.003750pt}%
\definecolor{currentstroke}{rgb}{0.000000,0.000000,0.000000}%
\pgfsetstrokecolor{currentstroke}%
\pgfsetdash{}{0pt}%
\pgfpathmoveto{\pgfqpoint{2.824734in}{0.356667in}}%
\pgfpathcurveto{\pgfqpoint{2.830260in}{0.356667in}}{\pgfqpoint{2.835559in}{0.358862in}}{\pgfqpoint{2.839466in}{0.362769in}}%
\pgfpathcurveto{\pgfqpoint{2.843373in}{0.366675in}}{\pgfqpoint{2.845568in}{0.371975in}}{\pgfqpoint{2.845568in}{0.377500in}}%
\pgfpathcurveto{\pgfqpoint{2.845568in}{0.383025in}}{\pgfqpoint{2.843373in}{0.388325in}}{\pgfqpoint{2.839466in}{0.392231in}}%
\pgfpathcurveto{\pgfqpoint{2.835559in}{0.396138in}}{\pgfqpoint{2.830260in}{0.398333in}}{\pgfqpoint{2.824734in}{0.398333in}}%
\pgfpathcurveto{\pgfqpoint{2.819209in}{0.398333in}}{\pgfqpoint{2.813910in}{0.396138in}}{\pgfqpoint{2.810003in}{0.392231in}}%
\pgfpathcurveto{\pgfqpoint{2.806096in}{0.388325in}}{\pgfqpoint{2.803901in}{0.383025in}}{\pgfqpoint{2.803901in}{0.377500in}}%
\pgfpathcurveto{\pgfqpoint{2.803901in}{0.371975in}}{\pgfqpoint{2.806096in}{0.366675in}}{\pgfqpoint{2.810003in}{0.362769in}}%
\pgfpathcurveto{\pgfqpoint{2.813910in}{0.358862in}}{\pgfqpoint{2.819209in}{0.356667in}}{\pgfqpoint{2.824734in}{0.356667in}}%
\pgfpathclose%
\pgfusepath{stroke,fill}%
\end{pgfscope}%
\begin{pgfscope}%
\pgfpathrectangle{\pgfqpoint{0.562500in}{0.275000in}}{\pgfqpoint{3.487500in}{1.925000in}}%
\pgfusepath{clip}%
\pgfsetbuttcap%
\pgfsetroundjoin%
\definecolor{currentfill}{rgb}{0.000000,0.000000,0.000000}%
\pgfsetfillcolor{currentfill}%
\pgfsetlinewidth{1.003750pt}%
\definecolor{currentstroke}{rgb}{0.000000,0.000000,0.000000}%
\pgfsetstrokecolor{currentstroke}%
\pgfsetdash{}{0pt}%
\pgfpathmoveto{\pgfqpoint{3.876477in}{0.356667in}}%
\pgfpathcurveto{\pgfqpoint{3.882002in}{0.356667in}}{\pgfqpoint{3.887302in}{0.358862in}}{\pgfqpoint{3.891209in}{0.362769in}}%
\pgfpathcurveto{\pgfqpoint{3.895115in}{0.366675in}}{\pgfqpoint{3.897311in}{0.371975in}}{\pgfqpoint{3.897311in}{0.377500in}}%
\pgfpathcurveto{\pgfqpoint{3.897311in}{0.383025in}}{\pgfqpoint{3.895115in}{0.388325in}}{\pgfqpoint{3.891209in}{0.392231in}}%
\pgfpathcurveto{\pgfqpoint{3.887302in}{0.396138in}}{\pgfqpoint{3.882002in}{0.398333in}}{\pgfqpoint{3.876477in}{0.398333in}}%
\pgfpathcurveto{\pgfqpoint{3.870952in}{0.398333in}}{\pgfqpoint{3.865653in}{0.396138in}}{\pgfqpoint{3.861746in}{0.392231in}}%
\pgfpathcurveto{\pgfqpoint{3.857839in}{0.388325in}}{\pgfqpoint{3.855644in}{0.383025in}}{\pgfqpoint{3.855644in}{0.377500in}}%
\pgfpathcurveto{\pgfqpoint{3.855644in}{0.371975in}}{\pgfqpoint{3.857839in}{0.366675in}}{\pgfqpoint{3.861746in}{0.362769in}}%
\pgfpathcurveto{\pgfqpoint{3.865653in}{0.358862in}}{\pgfqpoint{3.870952in}{0.356667in}}{\pgfqpoint{3.876477in}{0.356667in}}%
\pgfpathclose%
\pgfusepath{stroke,fill}%
\end{pgfscope}%
\begin{pgfscope}%
\pgfpathrectangle{\pgfqpoint{0.562500in}{0.275000in}}{\pgfqpoint{3.487500in}{1.925000in}}%
\pgfusepath{clip}%
\pgfsetbuttcap%
\pgfsetroundjoin%
\definecolor{currentfill}{rgb}{0.000000,0.000000,0.000000}%
\pgfsetfillcolor{currentfill}%
\pgfsetlinewidth{1.003750pt}%
\definecolor{currentstroke}{rgb}{0.000000,0.000000,0.000000}%
\pgfsetstrokecolor{currentstroke}%
\pgfsetdash{}{0pt}%
\pgfpathmoveto{\pgfqpoint{3.876477in}{0.356667in}}%
\pgfpathcurveto{\pgfqpoint{3.882002in}{0.356667in}}{\pgfqpoint{3.887302in}{0.358862in}}{\pgfqpoint{3.891209in}{0.362769in}}%
\pgfpathcurveto{\pgfqpoint{3.895115in}{0.366675in}}{\pgfqpoint{3.897311in}{0.371975in}}{\pgfqpoint{3.897311in}{0.377500in}}%
\pgfpathcurveto{\pgfqpoint{3.897311in}{0.383025in}}{\pgfqpoint{3.895115in}{0.388325in}}{\pgfqpoint{3.891209in}{0.392231in}}%
\pgfpathcurveto{\pgfqpoint{3.887302in}{0.396138in}}{\pgfqpoint{3.882002in}{0.398333in}}{\pgfqpoint{3.876477in}{0.398333in}}%
\pgfpathcurveto{\pgfqpoint{3.870952in}{0.398333in}}{\pgfqpoint{3.865653in}{0.396138in}}{\pgfqpoint{3.861746in}{0.392231in}}%
\pgfpathcurveto{\pgfqpoint{3.857839in}{0.388325in}}{\pgfqpoint{3.855644in}{0.383025in}}{\pgfqpoint{3.855644in}{0.377500in}}%
\pgfpathcurveto{\pgfqpoint{3.855644in}{0.371975in}}{\pgfqpoint{3.857839in}{0.366675in}}{\pgfqpoint{3.861746in}{0.362769in}}%
\pgfpathcurveto{\pgfqpoint{3.865653in}{0.358862in}}{\pgfqpoint{3.870952in}{0.356667in}}{\pgfqpoint{3.876477in}{0.356667in}}%
\pgfpathclose%
\pgfusepath{stroke,fill}%
\end{pgfscope}%
\begin{pgfscope}%
\pgfpathrectangle{\pgfqpoint{0.562500in}{0.275000in}}{\pgfqpoint{3.487500in}{1.925000in}}%
\pgfusepath{clip}%
\pgfsetbuttcap%
\pgfsetroundjoin%
\definecolor{currentfill}{rgb}{0.000000,0.000000,0.000000}%
\pgfsetfillcolor{currentfill}%
\pgfsetlinewidth{1.003750pt}%
\definecolor{currentstroke}{rgb}{0.000000,0.000000,0.000000}%
\pgfsetstrokecolor{currentstroke}%
\pgfsetdash{}{0pt}%
\pgfpathmoveto{\pgfqpoint{3.876477in}{0.356667in}}%
\pgfpathcurveto{\pgfqpoint{3.882002in}{0.356667in}}{\pgfqpoint{3.887302in}{0.358862in}}{\pgfqpoint{3.891209in}{0.362769in}}%
\pgfpathcurveto{\pgfqpoint{3.895115in}{0.366675in}}{\pgfqpoint{3.897311in}{0.371975in}}{\pgfqpoint{3.897311in}{0.377500in}}%
\pgfpathcurveto{\pgfqpoint{3.897311in}{0.383025in}}{\pgfqpoint{3.895115in}{0.388325in}}{\pgfqpoint{3.891209in}{0.392231in}}%
\pgfpathcurveto{\pgfqpoint{3.887302in}{0.396138in}}{\pgfqpoint{3.882002in}{0.398333in}}{\pgfqpoint{3.876477in}{0.398333in}}%
\pgfpathcurveto{\pgfqpoint{3.870952in}{0.398333in}}{\pgfqpoint{3.865653in}{0.396138in}}{\pgfqpoint{3.861746in}{0.392231in}}%
\pgfpathcurveto{\pgfqpoint{3.857839in}{0.388325in}}{\pgfqpoint{3.855644in}{0.383025in}}{\pgfqpoint{3.855644in}{0.377500in}}%
\pgfpathcurveto{\pgfqpoint{3.855644in}{0.371975in}}{\pgfqpoint{3.857839in}{0.366675in}}{\pgfqpoint{3.861746in}{0.362769in}}%
\pgfpathcurveto{\pgfqpoint{3.865653in}{0.358862in}}{\pgfqpoint{3.870952in}{0.356667in}}{\pgfqpoint{3.876477in}{0.356667in}}%
\pgfpathclose%
\pgfusepath{stroke,fill}%
\end{pgfscope}%
\begin{pgfscope}%
\pgfpathrectangle{\pgfqpoint{0.562500in}{0.275000in}}{\pgfqpoint{3.487500in}{1.925000in}}%
\pgfusepath{clip}%
\pgfsetbuttcap%
\pgfsetroundjoin%
\definecolor{currentfill}{rgb}{0.000000,0.000000,0.000000}%
\pgfsetfillcolor{currentfill}%
\pgfsetlinewidth{1.003750pt}%
\definecolor{currentstroke}{rgb}{0.000000,0.000000,0.000000}%
\pgfsetstrokecolor{currentstroke}%
\pgfsetdash{}{0pt}%
\pgfpathmoveto{\pgfqpoint{3.876477in}{0.356667in}}%
\pgfpathcurveto{\pgfqpoint{3.882002in}{0.356667in}}{\pgfqpoint{3.887302in}{0.358862in}}{\pgfqpoint{3.891209in}{0.362769in}}%
\pgfpathcurveto{\pgfqpoint{3.895115in}{0.366675in}}{\pgfqpoint{3.897311in}{0.371975in}}{\pgfqpoint{3.897311in}{0.377500in}}%
\pgfpathcurveto{\pgfqpoint{3.897311in}{0.383025in}}{\pgfqpoint{3.895115in}{0.388325in}}{\pgfqpoint{3.891209in}{0.392231in}}%
\pgfpathcurveto{\pgfqpoint{3.887302in}{0.396138in}}{\pgfqpoint{3.882002in}{0.398333in}}{\pgfqpoint{3.876477in}{0.398333in}}%
\pgfpathcurveto{\pgfqpoint{3.870952in}{0.398333in}}{\pgfqpoint{3.865653in}{0.396138in}}{\pgfqpoint{3.861746in}{0.392231in}}%
\pgfpathcurveto{\pgfqpoint{3.857839in}{0.388325in}}{\pgfqpoint{3.855644in}{0.383025in}}{\pgfqpoint{3.855644in}{0.377500in}}%
\pgfpathcurveto{\pgfqpoint{3.855644in}{0.371975in}}{\pgfqpoint{3.857839in}{0.366675in}}{\pgfqpoint{3.861746in}{0.362769in}}%
\pgfpathcurveto{\pgfqpoint{3.865653in}{0.358862in}}{\pgfqpoint{3.870952in}{0.356667in}}{\pgfqpoint{3.876477in}{0.356667in}}%
\pgfpathclose%
\pgfusepath{stroke,fill}%
\end{pgfscope}%
\begin{pgfscope}%
\pgfpathrectangle{\pgfqpoint{0.562500in}{0.275000in}}{\pgfqpoint{3.487500in}{1.925000in}}%
\pgfusepath{clip}%
\pgfsetbuttcap%
\pgfsetroundjoin%
\definecolor{currentfill}{rgb}{0.000000,0.000000,0.000000}%
\pgfsetfillcolor{currentfill}%
\pgfsetlinewidth{1.003750pt}%
\definecolor{currentstroke}{rgb}{0.000000,0.000000,0.000000}%
\pgfsetstrokecolor{currentstroke}%
\pgfsetdash{}{0pt}%
\pgfpathmoveto{\pgfqpoint{3.876477in}{0.356667in}}%
\pgfpathcurveto{\pgfqpoint{3.882002in}{0.356667in}}{\pgfqpoint{3.887302in}{0.358862in}}{\pgfqpoint{3.891209in}{0.362769in}}%
\pgfpathcurveto{\pgfqpoint{3.895115in}{0.366675in}}{\pgfqpoint{3.897311in}{0.371975in}}{\pgfqpoint{3.897311in}{0.377500in}}%
\pgfpathcurveto{\pgfqpoint{3.897311in}{0.383025in}}{\pgfqpoint{3.895115in}{0.388325in}}{\pgfqpoint{3.891209in}{0.392231in}}%
\pgfpathcurveto{\pgfqpoint{3.887302in}{0.396138in}}{\pgfqpoint{3.882002in}{0.398333in}}{\pgfqpoint{3.876477in}{0.398333in}}%
\pgfpathcurveto{\pgfqpoint{3.870952in}{0.398333in}}{\pgfqpoint{3.865653in}{0.396138in}}{\pgfqpoint{3.861746in}{0.392231in}}%
\pgfpathcurveto{\pgfqpoint{3.857839in}{0.388325in}}{\pgfqpoint{3.855644in}{0.383025in}}{\pgfqpoint{3.855644in}{0.377500in}}%
\pgfpathcurveto{\pgfqpoint{3.855644in}{0.371975in}}{\pgfqpoint{3.857839in}{0.366675in}}{\pgfqpoint{3.861746in}{0.362769in}}%
\pgfpathcurveto{\pgfqpoint{3.865653in}{0.358862in}}{\pgfqpoint{3.870952in}{0.356667in}}{\pgfqpoint{3.876477in}{0.356667in}}%
\pgfpathclose%
\pgfusepath{stroke,fill}%
\end{pgfscope}%
\begin{pgfscope}%
\pgfpathrectangle{\pgfqpoint{0.562500in}{0.275000in}}{\pgfqpoint{3.487500in}{1.925000in}}%
\pgfusepath{clip}%
\pgfsetbuttcap%
\pgfsetroundjoin%
\definecolor{currentfill}{rgb}{0.000000,0.000000,0.000000}%
\pgfsetfillcolor{currentfill}%
\pgfsetlinewidth{1.003750pt}%
\definecolor{currentstroke}{rgb}{0.000000,0.000000,0.000000}%
\pgfsetstrokecolor{currentstroke}%
\pgfsetdash{}{0pt}%
\pgfpathmoveto{\pgfqpoint{3.876477in}{0.356667in}}%
\pgfpathcurveto{\pgfqpoint{3.882002in}{0.356667in}}{\pgfqpoint{3.887302in}{0.358862in}}{\pgfqpoint{3.891209in}{0.362769in}}%
\pgfpathcurveto{\pgfqpoint{3.895115in}{0.366675in}}{\pgfqpoint{3.897311in}{0.371975in}}{\pgfqpoint{3.897311in}{0.377500in}}%
\pgfpathcurveto{\pgfqpoint{3.897311in}{0.383025in}}{\pgfqpoint{3.895115in}{0.388325in}}{\pgfqpoint{3.891209in}{0.392231in}}%
\pgfpathcurveto{\pgfqpoint{3.887302in}{0.396138in}}{\pgfqpoint{3.882002in}{0.398333in}}{\pgfqpoint{3.876477in}{0.398333in}}%
\pgfpathcurveto{\pgfqpoint{3.870952in}{0.398333in}}{\pgfqpoint{3.865653in}{0.396138in}}{\pgfqpoint{3.861746in}{0.392231in}}%
\pgfpathcurveto{\pgfqpoint{3.857839in}{0.388325in}}{\pgfqpoint{3.855644in}{0.383025in}}{\pgfqpoint{3.855644in}{0.377500in}}%
\pgfpathcurveto{\pgfqpoint{3.855644in}{0.371975in}}{\pgfqpoint{3.857839in}{0.366675in}}{\pgfqpoint{3.861746in}{0.362769in}}%
\pgfpathcurveto{\pgfqpoint{3.865653in}{0.358862in}}{\pgfqpoint{3.870952in}{0.356667in}}{\pgfqpoint{3.876477in}{0.356667in}}%
\pgfpathclose%
\pgfusepath{stroke,fill}%
\end{pgfscope}%
\begin{pgfscope}%
\pgfpathrectangle{\pgfqpoint{0.562500in}{0.275000in}}{\pgfqpoint{3.487500in}{1.925000in}}%
\pgfusepath{clip}%
\pgfsetbuttcap%
\pgfsetroundjoin%
\definecolor{currentfill}{rgb}{0.000000,0.000000,0.000000}%
\pgfsetfillcolor{currentfill}%
\pgfsetlinewidth{1.003750pt}%
\definecolor{currentstroke}{rgb}{0.000000,0.000000,0.000000}%
\pgfsetstrokecolor{currentstroke}%
\pgfsetdash{}{0pt}%
\pgfpathmoveto{\pgfqpoint{3.876477in}{0.356667in}}%
\pgfpathcurveto{\pgfqpoint{3.882002in}{0.356667in}}{\pgfqpoint{3.887302in}{0.358862in}}{\pgfqpoint{3.891209in}{0.362769in}}%
\pgfpathcurveto{\pgfqpoint{3.895115in}{0.366675in}}{\pgfqpoint{3.897311in}{0.371975in}}{\pgfqpoint{3.897311in}{0.377500in}}%
\pgfpathcurveto{\pgfqpoint{3.897311in}{0.383025in}}{\pgfqpoint{3.895115in}{0.388325in}}{\pgfqpoint{3.891209in}{0.392231in}}%
\pgfpathcurveto{\pgfqpoint{3.887302in}{0.396138in}}{\pgfqpoint{3.882002in}{0.398333in}}{\pgfqpoint{3.876477in}{0.398333in}}%
\pgfpathcurveto{\pgfqpoint{3.870952in}{0.398333in}}{\pgfqpoint{3.865653in}{0.396138in}}{\pgfqpoint{3.861746in}{0.392231in}}%
\pgfpathcurveto{\pgfqpoint{3.857839in}{0.388325in}}{\pgfqpoint{3.855644in}{0.383025in}}{\pgfqpoint{3.855644in}{0.377500in}}%
\pgfpathcurveto{\pgfqpoint{3.855644in}{0.371975in}}{\pgfqpoint{3.857839in}{0.366675in}}{\pgfqpoint{3.861746in}{0.362769in}}%
\pgfpathcurveto{\pgfqpoint{3.865653in}{0.358862in}}{\pgfqpoint{3.870952in}{0.356667in}}{\pgfqpoint{3.876477in}{0.356667in}}%
\pgfpathclose%
\pgfusepath{stroke,fill}%
\end{pgfscope}%
\begin{pgfscope}%
\pgfpathrectangle{\pgfqpoint{0.562500in}{0.275000in}}{\pgfqpoint{3.487500in}{1.925000in}}%
\pgfusepath{clip}%
\pgfsetbuttcap%
\pgfsetroundjoin%
\definecolor{currentfill}{rgb}{0.000000,0.000000,0.000000}%
\pgfsetfillcolor{currentfill}%
\pgfsetlinewidth{1.003750pt}%
\definecolor{currentstroke}{rgb}{0.000000,0.000000,0.000000}%
\pgfsetstrokecolor{currentstroke}%
\pgfsetdash{}{0pt}%
\pgfpathmoveto{\pgfqpoint{3.876477in}{0.356667in}}%
\pgfpathcurveto{\pgfqpoint{3.882002in}{0.356667in}}{\pgfqpoint{3.887302in}{0.358862in}}{\pgfqpoint{3.891209in}{0.362769in}}%
\pgfpathcurveto{\pgfqpoint{3.895115in}{0.366675in}}{\pgfqpoint{3.897311in}{0.371975in}}{\pgfqpoint{3.897311in}{0.377500in}}%
\pgfpathcurveto{\pgfqpoint{3.897311in}{0.383025in}}{\pgfqpoint{3.895115in}{0.388325in}}{\pgfqpoint{3.891209in}{0.392231in}}%
\pgfpathcurveto{\pgfqpoint{3.887302in}{0.396138in}}{\pgfqpoint{3.882002in}{0.398333in}}{\pgfqpoint{3.876477in}{0.398333in}}%
\pgfpathcurveto{\pgfqpoint{3.870952in}{0.398333in}}{\pgfqpoint{3.865653in}{0.396138in}}{\pgfqpoint{3.861746in}{0.392231in}}%
\pgfpathcurveto{\pgfqpoint{3.857839in}{0.388325in}}{\pgfqpoint{3.855644in}{0.383025in}}{\pgfqpoint{3.855644in}{0.377500in}}%
\pgfpathcurveto{\pgfqpoint{3.855644in}{0.371975in}}{\pgfqpoint{3.857839in}{0.366675in}}{\pgfqpoint{3.861746in}{0.362769in}}%
\pgfpathcurveto{\pgfqpoint{3.865653in}{0.358862in}}{\pgfqpoint{3.870952in}{0.356667in}}{\pgfqpoint{3.876477in}{0.356667in}}%
\pgfpathclose%
\pgfusepath{stroke,fill}%
\end{pgfscope}%
\begin{pgfscope}%
\pgfpathrectangle{\pgfqpoint{0.562500in}{0.275000in}}{\pgfqpoint{3.487500in}{1.925000in}}%
\pgfusepath{clip}%
\pgfsetbuttcap%
\pgfsetroundjoin%
\definecolor{currentfill}{rgb}{0.000000,0.000000,0.000000}%
\pgfsetfillcolor{currentfill}%
\pgfsetlinewidth{1.003750pt}%
\definecolor{currentstroke}{rgb}{0.000000,0.000000,0.000000}%
\pgfsetstrokecolor{currentstroke}%
\pgfsetdash{}{0pt}%
\pgfpathmoveto{\pgfqpoint{3.876477in}{0.356667in}}%
\pgfpathcurveto{\pgfqpoint{3.882002in}{0.356667in}}{\pgfqpoint{3.887302in}{0.358862in}}{\pgfqpoint{3.891209in}{0.362769in}}%
\pgfpathcurveto{\pgfqpoint{3.895115in}{0.366675in}}{\pgfqpoint{3.897311in}{0.371975in}}{\pgfqpoint{3.897311in}{0.377500in}}%
\pgfpathcurveto{\pgfqpoint{3.897311in}{0.383025in}}{\pgfqpoint{3.895115in}{0.388325in}}{\pgfqpoint{3.891209in}{0.392231in}}%
\pgfpathcurveto{\pgfqpoint{3.887302in}{0.396138in}}{\pgfqpoint{3.882002in}{0.398333in}}{\pgfqpoint{3.876477in}{0.398333in}}%
\pgfpathcurveto{\pgfqpoint{3.870952in}{0.398333in}}{\pgfqpoint{3.865653in}{0.396138in}}{\pgfqpoint{3.861746in}{0.392231in}}%
\pgfpathcurveto{\pgfqpoint{3.857839in}{0.388325in}}{\pgfqpoint{3.855644in}{0.383025in}}{\pgfqpoint{3.855644in}{0.377500in}}%
\pgfpathcurveto{\pgfqpoint{3.855644in}{0.371975in}}{\pgfqpoint{3.857839in}{0.366675in}}{\pgfqpoint{3.861746in}{0.362769in}}%
\pgfpathcurveto{\pgfqpoint{3.865653in}{0.358862in}}{\pgfqpoint{3.870952in}{0.356667in}}{\pgfqpoint{3.876477in}{0.356667in}}%
\pgfpathclose%
\pgfusepath{stroke,fill}%
\end{pgfscope}%
\begin{pgfscope}%
\pgfpathrectangle{\pgfqpoint{0.562500in}{0.275000in}}{\pgfqpoint{3.487500in}{1.925000in}}%
\pgfusepath{clip}%
\pgfsetbuttcap%
\pgfsetroundjoin%
\definecolor{currentfill}{rgb}{0.000000,0.000000,0.000000}%
\pgfsetfillcolor{currentfill}%
\pgfsetlinewidth{1.003750pt}%
\definecolor{currentstroke}{rgb}{0.000000,0.000000,0.000000}%
\pgfsetstrokecolor{currentstroke}%
\pgfsetdash{}{0pt}%
\pgfpathmoveto{\pgfqpoint{3.876477in}{0.356667in}}%
\pgfpathcurveto{\pgfqpoint{3.882002in}{0.356667in}}{\pgfqpoint{3.887302in}{0.358862in}}{\pgfqpoint{3.891209in}{0.362769in}}%
\pgfpathcurveto{\pgfqpoint{3.895115in}{0.366675in}}{\pgfqpoint{3.897311in}{0.371975in}}{\pgfqpoint{3.897311in}{0.377500in}}%
\pgfpathcurveto{\pgfqpoint{3.897311in}{0.383025in}}{\pgfqpoint{3.895115in}{0.388325in}}{\pgfqpoint{3.891209in}{0.392231in}}%
\pgfpathcurveto{\pgfqpoint{3.887302in}{0.396138in}}{\pgfqpoint{3.882002in}{0.398333in}}{\pgfqpoint{3.876477in}{0.398333in}}%
\pgfpathcurveto{\pgfqpoint{3.870952in}{0.398333in}}{\pgfqpoint{3.865653in}{0.396138in}}{\pgfqpoint{3.861746in}{0.392231in}}%
\pgfpathcurveto{\pgfqpoint{3.857839in}{0.388325in}}{\pgfqpoint{3.855644in}{0.383025in}}{\pgfqpoint{3.855644in}{0.377500in}}%
\pgfpathcurveto{\pgfqpoint{3.855644in}{0.371975in}}{\pgfqpoint{3.857839in}{0.366675in}}{\pgfqpoint{3.861746in}{0.362769in}}%
\pgfpathcurveto{\pgfqpoint{3.865653in}{0.358862in}}{\pgfqpoint{3.870952in}{0.356667in}}{\pgfqpoint{3.876477in}{0.356667in}}%
\pgfpathclose%
\pgfusepath{stroke,fill}%
\end{pgfscope}%
\begin{pgfscope}%
\pgfpathrectangle{\pgfqpoint{0.562500in}{0.275000in}}{\pgfqpoint{3.487500in}{1.925000in}}%
\pgfusepath{clip}%
\pgfsetbuttcap%
\pgfsetroundjoin%
\definecolor{currentfill}{rgb}{0.000000,0.000000,0.000000}%
\pgfsetfillcolor{currentfill}%
\pgfsetlinewidth{1.003750pt}%
\definecolor{currentstroke}{rgb}{0.000000,0.000000,0.000000}%
\pgfsetstrokecolor{currentstroke}%
\pgfsetdash{}{0pt}%
\pgfpathmoveto{\pgfqpoint{3.876477in}{0.356667in}}%
\pgfpathcurveto{\pgfqpoint{3.882002in}{0.356667in}}{\pgfqpoint{3.887302in}{0.358862in}}{\pgfqpoint{3.891209in}{0.362769in}}%
\pgfpathcurveto{\pgfqpoint{3.895115in}{0.366675in}}{\pgfqpoint{3.897311in}{0.371975in}}{\pgfqpoint{3.897311in}{0.377500in}}%
\pgfpathcurveto{\pgfqpoint{3.897311in}{0.383025in}}{\pgfqpoint{3.895115in}{0.388325in}}{\pgfqpoint{3.891209in}{0.392231in}}%
\pgfpathcurveto{\pgfqpoint{3.887302in}{0.396138in}}{\pgfqpoint{3.882002in}{0.398333in}}{\pgfqpoint{3.876477in}{0.398333in}}%
\pgfpathcurveto{\pgfqpoint{3.870952in}{0.398333in}}{\pgfqpoint{3.865653in}{0.396138in}}{\pgfqpoint{3.861746in}{0.392231in}}%
\pgfpathcurveto{\pgfqpoint{3.857839in}{0.388325in}}{\pgfqpoint{3.855644in}{0.383025in}}{\pgfqpoint{3.855644in}{0.377500in}}%
\pgfpathcurveto{\pgfqpoint{3.855644in}{0.371975in}}{\pgfqpoint{3.857839in}{0.366675in}}{\pgfqpoint{3.861746in}{0.362769in}}%
\pgfpathcurveto{\pgfqpoint{3.865653in}{0.358862in}}{\pgfqpoint{3.870952in}{0.356667in}}{\pgfqpoint{3.876477in}{0.356667in}}%
\pgfpathclose%
\pgfusepath{stroke,fill}%
\end{pgfscope}%
\begin{pgfscope}%
\pgfpathrectangle{\pgfqpoint{0.562500in}{0.275000in}}{\pgfqpoint{3.487500in}{1.925000in}}%
\pgfusepath{clip}%
\pgfsetbuttcap%
\pgfsetroundjoin%
\definecolor{currentfill}{rgb}{0.000000,0.000000,0.000000}%
\pgfsetfillcolor{currentfill}%
\pgfsetlinewidth{1.003750pt}%
\definecolor{currentstroke}{rgb}{0.000000,0.000000,0.000000}%
\pgfsetstrokecolor{currentstroke}%
\pgfsetdash{}{0pt}%
\pgfpathmoveto{\pgfqpoint{3.876477in}{0.356667in}}%
\pgfpathcurveto{\pgfqpoint{3.882002in}{0.356667in}}{\pgfqpoint{3.887302in}{0.358862in}}{\pgfqpoint{3.891209in}{0.362769in}}%
\pgfpathcurveto{\pgfqpoint{3.895115in}{0.366675in}}{\pgfqpoint{3.897311in}{0.371975in}}{\pgfqpoint{3.897311in}{0.377500in}}%
\pgfpathcurveto{\pgfqpoint{3.897311in}{0.383025in}}{\pgfqpoint{3.895115in}{0.388325in}}{\pgfqpoint{3.891209in}{0.392231in}}%
\pgfpathcurveto{\pgfqpoint{3.887302in}{0.396138in}}{\pgfqpoint{3.882002in}{0.398333in}}{\pgfqpoint{3.876477in}{0.398333in}}%
\pgfpathcurveto{\pgfqpoint{3.870952in}{0.398333in}}{\pgfqpoint{3.865653in}{0.396138in}}{\pgfqpoint{3.861746in}{0.392231in}}%
\pgfpathcurveto{\pgfqpoint{3.857839in}{0.388325in}}{\pgfqpoint{3.855644in}{0.383025in}}{\pgfqpoint{3.855644in}{0.377500in}}%
\pgfpathcurveto{\pgfqpoint{3.855644in}{0.371975in}}{\pgfqpoint{3.857839in}{0.366675in}}{\pgfqpoint{3.861746in}{0.362769in}}%
\pgfpathcurveto{\pgfqpoint{3.865653in}{0.358862in}}{\pgfqpoint{3.870952in}{0.356667in}}{\pgfqpoint{3.876477in}{0.356667in}}%
\pgfpathclose%
\pgfusepath{stroke,fill}%
\end{pgfscope}%
\begin{pgfscope}%
\pgfpathrectangle{\pgfqpoint{0.562500in}{0.275000in}}{\pgfqpoint{3.487500in}{1.925000in}}%
\pgfusepath{clip}%
\pgfsetbuttcap%
\pgfsetroundjoin%
\definecolor{currentfill}{rgb}{0.000000,0.000000,0.000000}%
\pgfsetfillcolor{currentfill}%
\pgfsetlinewidth{1.003750pt}%
\definecolor{currentstroke}{rgb}{0.000000,0.000000,0.000000}%
\pgfsetstrokecolor{currentstroke}%
\pgfsetdash{}{0pt}%
\pgfpathmoveto{\pgfqpoint{3.876477in}{0.356667in}}%
\pgfpathcurveto{\pgfqpoint{3.882002in}{0.356667in}}{\pgfqpoint{3.887302in}{0.358862in}}{\pgfqpoint{3.891209in}{0.362769in}}%
\pgfpathcurveto{\pgfqpoint{3.895115in}{0.366675in}}{\pgfqpoint{3.897311in}{0.371975in}}{\pgfqpoint{3.897311in}{0.377500in}}%
\pgfpathcurveto{\pgfqpoint{3.897311in}{0.383025in}}{\pgfqpoint{3.895115in}{0.388325in}}{\pgfqpoint{3.891209in}{0.392231in}}%
\pgfpathcurveto{\pgfqpoint{3.887302in}{0.396138in}}{\pgfqpoint{3.882002in}{0.398333in}}{\pgfqpoint{3.876477in}{0.398333in}}%
\pgfpathcurveto{\pgfqpoint{3.870952in}{0.398333in}}{\pgfqpoint{3.865653in}{0.396138in}}{\pgfqpoint{3.861746in}{0.392231in}}%
\pgfpathcurveto{\pgfqpoint{3.857839in}{0.388325in}}{\pgfqpoint{3.855644in}{0.383025in}}{\pgfqpoint{3.855644in}{0.377500in}}%
\pgfpathcurveto{\pgfqpoint{3.855644in}{0.371975in}}{\pgfqpoint{3.857839in}{0.366675in}}{\pgfqpoint{3.861746in}{0.362769in}}%
\pgfpathcurveto{\pgfqpoint{3.865653in}{0.358862in}}{\pgfqpoint{3.870952in}{0.356667in}}{\pgfqpoint{3.876477in}{0.356667in}}%
\pgfpathclose%
\pgfusepath{stroke,fill}%
\end{pgfscope}%
\begin{pgfscope}%
\pgfpathrectangle{\pgfqpoint{0.562500in}{0.275000in}}{\pgfqpoint{3.487500in}{1.925000in}}%
\pgfusepath{clip}%
\pgfsetbuttcap%
\pgfsetroundjoin%
\definecolor{currentfill}{rgb}{0.000000,0.000000,0.000000}%
\pgfsetfillcolor{currentfill}%
\pgfsetlinewidth{1.003750pt}%
\definecolor{currentstroke}{rgb}{0.000000,0.000000,0.000000}%
\pgfsetstrokecolor{currentstroke}%
\pgfsetdash{}{0pt}%
\pgfpathmoveto{\pgfqpoint{3.876477in}{0.356667in}}%
\pgfpathcurveto{\pgfqpoint{3.882002in}{0.356667in}}{\pgfqpoint{3.887302in}{0.358862in}}{\pgfqpoint{3.891209in}{0.362769in}}%
\pgfpathcurveto{\pgfqpoint{3.895115in}{0.366675in}}{\pgfqpoint{3.897311in}{0.371975in}}{\pgfqpoint{3.897311in}{0.377500in}}%
\pgfpathcurveto{\pgfqpoint{3.897311in}{0.383025in}}{\pgfqpoint{3.895115in}{0.388325in}}{\pgfqpoint{3.891209in}{0.392231in}}%
\pgfpathcurveto{\pgfqpoint{3.887302in}{0.396138in}}{\pgfqpoint{3.882002in}{0.398333in}}{\pgfqpoint{3.876477in}{0.398333in}}%
\pgfpathcurveto{\pgfqpoint{3.870952in}{0.398333in}}{\pgfqpoint{3.865653in}{0.396138in}}{\pgfqpoint{3.861746in}{0.392231in}}%
\pgfpathcurveto{\pgfqpoint{3.857839in}{0.388325in}}{\pgfqpoint{3.855644in}{0.383025in}}{\pgfqpoint{3.855644in}{0.377500in}}%
\pgfpathcurveto{\pgfqpoint{3.855644in}{0.371975in}}{\pgfqpoint{3.857839in}{0.366675in}}{\pgfqpoint{3.861746in}{0.362769in}}%
\pgfpathcurveto{\pgfqpoint{3.865653in}{0.358862in}}{\pgfqpoint{3.870952in}{0.356667in}}{\pgfqpoint{3.876477in}{0.356667in}}%
\pgfpathclose%
\pgfusepath{stroke,fill}%
\end{pgfscope}%
\begin{pgfscope}%
\pgfpathrectangle{\pgfqpoint{0.562500in}{0.275000in}}{\pgfqpoint{3.487500in}{1.925000in}}%
\pgfusepath{clip}%
\pgfsetbuttcap%
\pgfsetroundjoin%
\definecolor{currentfill}{rgb}{0.000000,0.000000,0.000000}%
\pgfsetfillcolor{currentfill}%
\pgfsetlinewidth{1.003750pt}%
\definecolor{currentstroke}{rgb}{0.000000,0.000000,0.000000}%
\pgfsetstrokecolor{currentstroke}%
\pgfsetdash{}{0pt}%
\pgfpathmoveto{\pgfqpoint{3.876477in}{0.356667in}}%
\pgfpathcurveto{\pgfqpoint{3.882002in}{0.356667in}}{\pgfqpoint{3.887302in}{0.358862in}}{\pgfqpoint{3.891209in}{0.362769in}}%
\pgfpathcurveto{\pgfqpoint{3.895115in}{0.366675in}}{\pgfqpoint{3.897311in}{0.371975in}}{\pgfqpoint{3.897311in}{0.377500in}}%
\pgfpathcurveto{\pgfqpoint{3.897311in}{0.383025in}}{\pgfqpoint{3.895115in}{0.388325in}}{\pgfqpoint{3.891209in}{0.392231in}}%
\pgfpathcurveto{\pgfqpoint{3.887302in}{0.396138in}}{\pgfqpoint{3.882002in}{0.398333in}}{\pgfqpoint{3.876477in}{0.398333in}}%
\pgfpathcurveto{\pgfqpoint{3.870952in}{0.398333in}}{\pgfqpoint{3.865653in}{0.396138in}}{\pgfqpoint{3.861746in}{0.392231in}}%
\pgfpathcurveto{\pgfqpoint{3.857839in}{0.388325in}}{\pgfqpoint{3.855644in}{0.383025in}}{\pgfqpoint{3.855644in}{0.377500in}}%
\pgfpathcurveto{\pgfqpoint{3.855644in}{0.371975in}}{\pgfqpoint{3.857839in}{0.366675in}}{\pgfqpoint{3.861746in}{0.362769in}}%
\pgfpathcurveto{\pgfqpoint{3.865653in}{0.358862in}}{\pgfqpoint{3.870952in}{0.356667in}}{\pgfqpoint{3.876477in}{0.356667in}}%
\pgfpathclose%
\pgfusepath{stroke,fill}%
\end{pgfscope}%
\begin{pgfscope}%
\pgfpathrectangle{\pgfqpoint{0.562500in}{0.275000in}}{\pgfqpoint{3.487500in}{1.925000in}}%
\pgfusepath{clip}%
\pgfsetbuttcap%
\pgfsetroundjoin%
\definecolor{currentfill}{rgb}{0.000000,0.000000,0.000000}%
\pgfsetfillcolor{currentfill}%
\pgfsetlinewidth{1.003750pt}%
\definecolor{currentstroke}{rgb}{0.000000,0.000000,0.000000}%
\pgfsetstrokecolor{currentstroke}%
\pgfsetdash{}{0pt}%
\pgfpathmoveto{\pgfqpoint{3.876477in}{0.356667in}}%
\pgfpathcurveto{\pgfqpoint{3.882002in}{0.356667in}}{\pgfqpoint{3.887302in}{0.358862in}}{\pgfqpoint{3.891209in}{0.362769in}}%
\pgfpathcurveto{\pgfqpoint{3.895115in}{0.366675in}}{\pgfqpoint{3.897311in}{0.371975in}}{\pgfqpoint{3.897311in}{0.377500in}}%
\pgfpathcurveto{\pgfqpoint{3.897311in}{0.383025in}}{\pgfqpoint{3.895115in}{0.388325in}}{\pgfqpoint{3.891209in}{0.392231in}}%
\pgfpathcurveto{\pgfqpoint{3.887302in}{0.396138in}}{\pgfqpoint{3.882002in}{0.398333in}}{\pgfqpoint{3.876477in}{0.398333in}}%
\pgfpathcurveto{\pgfqpoint{3.870952in}{0.398333in}}{\pgfqpoint{3.865653in}{0.396138in}}{\pgfqpoint{3.861746in}{0.392231in}}%
\pgfpathcurveto{\pgfqpoint{3.857839in}{0.388325in}}{\pgfqpoint{3.855644in}{0.383025in}}{\pgfqpoint{3.855644in}{0.377500in}}%
\pgfpathcurveto{\pgfqpoint{3.855644in}{0.371975in}}{\pgfqpoint{3.857839in}{0.366675in}}{\pgfqpoint{3.861746in}{0.362769in}}%
\pgfpathcurveto{\pgfqpoint{3.865653in}{0.358862in}}{\pgfqpoint{3.870952in}{0.356667in}}{\pgfqpoint{3.876477in}{0.356667in}}%
\pgfpathclose%
\pgfusepath{stroke,fill}%
\end{pgfscope}%
\begin{pgfscope}%
\pgfpathrectangle{\pgfqpoint{0.562500in}{0.275000in}}{\pgfqpoint{3.487500in}{1.925000in}}%
\pgfusepath{clip}%
\pgfsetbuttcap%
\pgfsetroundjoin%
\definecolor{currentfill}{rgb}{0.000000,0.000000,0.000000}%
\pgfsetfillcolor{currentfill}%
\pgfsetlinewidth{1.003750pt}%
\definecolor{currentstroke}{rgb}{0.000000,0.000000,0.000000}%
\pgfsetstrokecolor{currentstroke}%
\pgfsetdash{}{0pt}%
\pgfpathmoveto{\pgfqpoint{3.876477in}{0.356667in}}%
\pgfpathcurveto{\pgfqpoint{3.882002in}{0.356667in}}{\pgfqpoint{3.887302in}{0.358862in}}{\pgfqpoint{3.891209in}{0.362769in}}%
\pgfpathcurveto{\pgfqpoint{3.895115in}{0.366675in}}{\pgfqpoint{3.897311in}{0.371975in}}{\pgfqpoint{3.897311in}{0.377500in}}%
\pgfpathcurveto{\pgfqpoint{3.897311in}{0.383025in}}{\pgfqpoint{3.895115in}{0.388325in}}{\pgfqpoint{3.891209in}{0.392231in}}%
\pgfpathcurveto{\pgfqpoint{3.887302in}{0.396138in}}{\pgfqpoint{3.882002in}{0.398333in}}{\pgfqpoint{3.876477in}{0.398333in}}%
\pgfpathcurveto{\pgfqpoint{3.870952in}{0.398333in}}{\pgfqpoint{3.865653in}{0.396138in}}{\pgfqpoint{3.861746in}{0.392231in}}%
\pgfpathcurveto{\pgfqpoint{3.857839in}{0.388325in}}{\pgfqpoint{3.855644in}{0.383025in}}{\pgfqpoint{3.855644in}{0.377500in}}%
\pgfpathcurveto{\pgfqpoint{3.855644in}{0.371975in}}{\pgfqpoint{3.857839in}{0.366675in}}{\pgfqpoint{3.861746in}{0.362769in}}%
\pgfpathcurveto{\pgfqpoint{3.865653in}{0.358862in}}{\pgfqpoint{3.870952in}{0.356667in}}{\pgfqpoint{3.876477in}{0.356667in}}%
\pgfpathclose%
\pgfusepath{stroke,fill}%
\end{pgfscope}%
\begin{pgfscope}%
\pgfpathrectangle{\pgfqpoint{0.562500in}{0.275000in}}{\pgfqpoint{3.487500in}{1.925000in}}%
\pgfusepath{clip}%
\pgfsetbuttcap%
\pgfsetroundjoin%
\definecolor{currentfill}{rgb}{0.000000,0.000000,0.000000}%
\pgfsetfillcolor{currentfill}%
\pgfsetlinewidth{1.003750pt}%
\definecolor{currentstroke}{rgb}{0.000000,0.000000,0.000000}%
\pgfsetstrokecolor{currentstroke}%
\pgfsetdash{}{0pt}%
\pgfpathmoveto{\pgfqpoint{3.876477in}{0.356667in}}%
\pgfpathcurveto{\pgfqpoint{3.882002in}{0.356667in}}{\pgfqpoint{3.887302in}{0.358862in}}{\pgfqpoint{3.891209in}{0.362769in}}%
\pgfpathcurveto{\pgfqpoint{3.895115in}{0.366675in}}{\pgfqpoint{3.897311in}{0.371975in}}{\pgfqpoint{3.897311in}{0.377500in}}%
\pgfpathcurveto{\pgfqpoint{3.897311in}{0.383025in}}{\pgfqpoint{3.895115in}{0.388325in}}{\pgfqpoint{3.891209in}{0.392231in}}%
\pgfpathcurveto{\pgfqpoint{3.887302in}{0.396138in}}{\pgfqpoint{3.882002in}{0.398333in}}{\pgfqpoint{3.876477in}{0.398333in}}%
\pgfpathcurveto{\pgfqpoint{3.870952in}{0.398333in}}{\pgfqpoint{3.865653in}{0.396138in}}{\pgfqpoint{3.861746in}{0.392231in}}%
\pgfpathcurveto{\pgfqpoint{3.857839in}{0.388325in}}{\pgfqpoint{3.855644in}{0.383025in}}{\pgfqpoint{3.855644in}{0.377500in}}%
\pgfpathcurveto{\pgfqpoint{3.855644in}{0.371975in}}{\pgfqpoint{3.857839in}{0.366675in}}{\pgfqpoint{3.861746in}{0.362769in}}%
\pgfpathcurveto{\pgfqpoint{3.865653in}{0.358862in}}{\pgfqpoint{3.870952in}{0.356667in}}{\pgfqpoint{3.876477in}{0.356667in}}%
\pgfpathclose%
\pgfusepath{stroke,fill}%
\end{pgfscope}%
\begin{pgfscope}%
\pgfpathrectangle{\pgfqpoint{0.562500in}{0.275000in}}{\pgfqpoint{3.487500in}{1.925000in}}%
\pgfusepath{clip}%
\pgfsetbuttcap%
\pgfsetroundjoin%
\definecolor{currentfill}{rgb}{0.000000,0.000000,0.000000}%
\pgfsetfillcolor{currentfill}%
\pgfsetlinewidth{1.003750pt}%
\definecolor{currentstroke}{rgb}{0.000000,0.000000,0.000000}%
\pgfsetstrokecolor{currentstroke}%
\pgfsetdash{}{0pt}%
\pgfpathmoveto{\pgfqpoint{3.876477in}{0.356667in}}%
\pgfpathcurveto{\pgfqpoint{3.882002in}{0.356667in}}{\pgfqpoint{3.887302in}{0.358862in}}{\pgfqpoint{3.891209in}{0.362769in}}%
\pgfpathcurveto{\pgfqpoint{3.895115in}{0.366675in}}{\pgfqpoint{3.897311in}{0.371975in}}{\pgfqpoint{3.897311in}{0.377500in}}%
\pgfpathcurveto{\pgfqpoint{3.897311in}{0.383025in}}{\pgfqpoint{3.895115in}{0.388325in}}{\pgfqpoint{3.891209in}{0.392231in}}%
\pgfpathcurveto{\pgfqpoint{3.887302in}{0.396138in}}{\pgfqpoint{3.882002in}{0.398333in}}{\pgfqpoint{3.876477in}{0.398333in}}%
\pgfpathcurveto{\pgfqpoint{3.870952in}{0.398333in}}{\pgfqpoint{3.865653in}{0.396138in}}{\pgfqpoint{3.861746in}{0.392231in}}%
\pgfpathcurveto{\pgfqpoint{3.857839in}{0.388325in}}{\pgfqpoint{3.855644in}{0.383025in}}{\pgfqpoint{3.855644in}{0.377500in}}%
\pgfpathcurveto{\pgfqpoint{3.855644in}{0.371975in}}{\pgfqpoint{3.857839in}{0.366675in}}{\pgfqpoint{3.861746in}{0.362769in}}%
\pgfpathcurveto{\pgfqpoint{3.865653in}{0.358862in}}{\pgfqpoint{3.870952in}{0.356667in}}{\pgfqpoint{3.876477in}{0.356667in}}%
\pgfpathclose%
\pgfusepath{stroke,fill}%
\end{pgfscope}%
\begin{pgfscope}%
\pgfpathrectangle{\pgfqpoint{0.562500in}{0.275000in}}{\pgfqpoint{3.487500in}{1.925000in}}%
\pgfusepath{clip}%
\pgfsetbuttcap%
\pgfsetroundjoin%
\definecolor{currentfill}{rgb}{0.000000,0.000000,0.000000}%
\pgfsetfillcolor{currentfill}%
\pgfsetlinewidth{1.003750pt}%
\definecolor{currentstroke}{rgb}{0.000000,0.000000,0.000000}%
\pgfsetstrokecolor{currentstroke}%
\pgfsetdash{}{0pt}%
\pgfpathmoveto{\pgfqpoint{3.876477in}{0.356667in}}%
\pgfpathcurveto{\pgfqpoint{3.882002in}{0.356667in}}{\pgfqpoint{3.887302in}{0.358862in}}{\pgfqpoint{3.891209in}{0.362769in}}%
\pgfpathcurveto{\pgfqpoint{3.895115in}{0.366675in}}{\pgfqpoint{3.897311in}{0.371975in}}{\pgfqpoint{3.897311in}{0.377500in}}%
\pgfpathcurveto{\pgfqpoint{3.897311in}{0.383025in}}{\pgfqpoint{3.895115in}{0.388325in}}{\pgfqpoint{3.891209in}{0.392231in}}%
\pgfpathcurveto{\pgfqpoint{3.887302in}{0.396138in}}{\pgfqpoint{3.882002in}{0.398333in}}{\pgfqpoint{3.876477in}{0.398333in}}%
\pgfpathcurveto{\pgfqpoint{3.870952in}{0.398333in}}{\pgfqpoint{3.865653in}{0.396138in}}{\pgfqpoint{3.861746in}{0.392231in}}%
\pgfpathcurveto{\pgfqpoint{3.857839in}{0.388325in}}{\pgfqpoint{3.855644in}{0.383025in}}{\pgfqpoint{3.855644in}{0.377500in}}%
\pgfpathcurveto{\pgfqpoint{3.855644in}{0.371975in}}{\pgfqpoint{3.857839in}{0.366675in}}{\pgfqpoint{3.861746in}{0.362769in}}%
\pgfpathcurveto{\pgfqpoint{3.865653in}{0.358862in}}{\pgfqpoint{3.870952in}{0.356667in}}{\pgfqpoint{3.876477in}{0.356667in}}%
\pgfpathclose%
\pgfusepath{stroke,fill}%
\end{pgfscope}%
\begin{pgfscope}%
\pgfpathrectangle{\pgfqpoint{0.562500in}{0.275000in}}{\pgfqpoint{3.487500in}{1.925000in}}%
\pgfusepath{clip}%
\pgfsetbuttcap%
\pgfsetroundjoin%
\definecolor{currentfill}{rgb}{0.000000,0.000000,0.000000}%
\pgfsetfillcolor{currentfill}%
\pgfsetlinewidth{1.003750pt}%
\definecolor{currentstroke}{rgb}{0.000000,0.000000,0.000000}%
\pgfsetstrokecolor{currentstroke}%
\pgfsetdash{}{0pt}%
\pgfpathmoveto{\pgfqpoint{3.876477in}{0.356667in}}%
\pgfpathcurveto{\pgfqpoint{3.882002in}{0.356667in}}{\pgfqpoint{3.887302in}{0.358862in}}{\pgfqpoint{3.891209in}{0.362769in}}%
\pgfpathcurveto{\pgfqpoint{3.895115in}{0.366675in}}{\pgfqpoint{3.897311in}{0.371975in}}{\pgfqpoint{3.897311in}{0.377500in}}%
\pgfpathcurveto{\pgfqpoint{3.897311in}{0.383025in}}{\pgfqpoint{3.895115in}{0.388325in}}{\pgfqpoint{3.891209in}{0.392231in}}%
\pgfpathcurveto{\pgfqpoint{3.887302in}{0.396138in}}{\pgfqpoint{3.882002in}{0.398333in}}{\pgfqpoint{3.876477in}{0.398333in}}%
\pgfpathcurveto{\pgfqpoint{3.870952in}{0.398333in}}{\pgfqpoint{3.865653in}{0.396138in}}{\pgfqpoint{3.861746in}{0.392231in}}%
\pgfpathcurveto{\pgfqpoint{3.857839in}{0.388325in}}{\pgfqpoint{3.855644in}{0.383025in}}{\pgfqpoint{3.855644in}{0.377500in}}%
\pgfpathcurveto{\pgfqpoint{3.855644in}{0.371975in}}{\pgfqpoint{3.857839in}{0.366675in}}{\pgfqpoint{3.861746in}{0.362769in}}%
\pgfpathcurveto{\pgfqpoint{3.865653in}{0.358862in}}{\pgfqpoint{3.870952in}{0.356667in}}{\pgfqpoint{3.876477in}{0.356667in}}%
\pgfpathclose%
\pgfusepath{stroke,fill}%
\end{pgfscope}%
\begin{pgfscope}%
\pgfpathrectangle{\pgfqpoint{0.562500in}{0.275000in}}{\pgfqpoint{3.487500in}{1.925000in}}%
\pgfusepath{clip}%
\pgfsetbuttcap%
\pgfsetroundjoin%
\definecolor{currentfill}{rgb}{0.000000,0.000000,0.000000}%
\pgfsetfillcolor{currentfill}%
\pgfsetlinewidth{1.003750pt}%
\definecolor{currentstroke}{rgb}{0.000000,0.000000,0.000000}%
\pgfsetstrokecolor{currentstroke}%
\pgfsetdash{}{0pt}%
\pgfpathmoveto{\pgfqpoint{3.876477in}{0.356667in}}%
\pgfpathcurveto{\pgfqpoint{3.882002in}{0.356667in}}{\pgfqpoint{3.887302in}{0.358862in}}{\pgfqpoint{3.891209in}{0.362769in}}%
\pgfpathcurveto{\pgfqpoint{3.895115in}{0.366675in}}{\pgfqpoint{3.897311in}{0.371975in}}{\pgfqpoint{3.897311in}{0.377500in}}%
\pgfpathcurveto{\pgfqpoint{3.897311in}{0.383025in}}{\pgfqpoint{3.895115in}{0.388325in}}{\pgfqpoint{3.891209in}{0.392231in}}%
\pgfpathcurveto{\pgfqpoint{3.887302in}{0.396138in}}{\pgfqpoint{3.882002in}{0.398333in}}{\pgfqpoint{3.876477in}{0.398333in}}%
\pgfpathcurveto{\pgfqpoint{3.870952in}{0.398333in}}{\pgfqpoint{3.865653in}{0.396138in}}{\pgfqpoint{3.861746in}{0.392231in}}%
\pgfpathcurveto{\pgfqpoint{3.857839in}{0.388325in}}{\pgfqpoint{3.855644in}{0.383025in}}{\pgfqpoint{3.855644in}{0.377500in}}%
\pgfpathcurveto{\pgfqpoint{3.855644in}{0.371975in}}{\pgfqpoint{3.857839in}{0.366675in}}{\pgfqpoint{3.861746in}{0.362769in}}%
\pgfpathcurveto{\pgfqpoint{3.865653in}{0.358862in}}{\pgfqpoint{3.870952in}{0.356667in}}{\pgfqpoint{3.876477in}{0.356667in}}%
\pgfpathclose%
\pgfusepath{stroke,fill}%
\end{pgfscope}%
\begin{pgfscope}%
\pgfpathrectangle{\pgfqpoint{0.562500in}{0.275000in}}{\pgfqpoint{3.487500in}{1.925000in}}%
\pgfusepath{clip}%
\pgfsetbuttcap%
\pgfsetroundjoin%
\definecolor{currentfill}{rgb}{0.000000,0.000000,0.000000}%
\pgfsetfillcolor{currentfill}%
\pgfsetlinewidth{1.003750pt}%
\definecolor{currentstroke}{rgb}{0.000000,0.000000,0.000000}%
\pgfsetstrokecolor{currentstroke}%
\pgfsetdash{}{0pt}%
\pgfpathmoveto{\pgfqpoint{3.876477in}{0.356667in}}%
\pgfpathcurveto{\pgfqpoint{3.882002in}{0.356667in}}{\pgfqpoint{3.887302in}{0.358862in}}{\pgfqpoint{3.891209in}{0.362769in}}%
\pgfpathcurveto{\pgfqpoint{3.895115in}{0.366675in}}{\pgfqpoint{3.897311in}{0.371975in}}{\pgfqpoint{3.897311in}{0.377500in}}%
\pgfpathcurveto{\pgfqpoint{3.897311in}{0.383025in}}{\pgfqpoint{3.895115in}{0.388325in}}{\pgfqpoint{3.891209in}{0.392231in}}%
\pgfpathcurveto{\pgfqpoint{3.887302in}{0.396138in}}{\pgfqpoint{3.882002in}{0.398333in}}{\pgfqpoint{3.876477in}{0.398333in}}%
\pgfpathcurveto{\pgfqpoint{3.870952in}{0.398333in}}{\pgfqpoint{3.865653in}{0.396138in}}{\pgfqpoint{3.861746in}{0.392231in}}%
\pgfpathcurveto{\pgfqpoint{3.857839in}{0.388325in}}{\pgfqpoint{3.855644in}{0.383025in}}{\pgfqpoint{3.855644in}{0.377500in}}%
\pgfpathcurveto{\pgfqpoint{3.855644in}{0.371975in}}{\pgfqpoint{3.857839in}{0.366675in}}{\pgfqpoint{3.861746in}{0.362769in}}%
\pgfpathcurveto{\pgfqpoint{3.865653in}{0.358862in}}{\pgfqpoint{3.870952in}{0.356667in}}{\pgfqpoint{3.876477in}{0.356667in}}%
\pgfpathclose%
\pgfusepath{stroke,fill}%
\end{pgfscope}%
\begin{pgfscope}%
\pgfpathrectangle{\pgfqpoint{0.562500in}{0.275000in}}{\pgfqpoint{3.487500in}{1.925000in}}%
\pgfusepath{clip}%
\pgfsetbuttcap%
\pgfsetroundjoin%
\definecolor{currentfill}{rgb}{0.000000,0.000000,0.000000}%
\pgfsetfillcolor{currentfill}%
\pgfsetlinewidth{1.003750pt}%
\definecolor{currentstroke}{rgb}{0.000000,0.000000,0.000000}%
\pgfsetstrokecolor{currentstroke}%
\pgfsetdash{}{0pt}%
\pgfpathmoveto{\pgfqpoint{3.876477in}{0.356667in}}%
\pgfpathcurveto{\pgfqpoint{3.882002in}{0.356667in}}{\pgfqpoint{3.887302in}{0.358862in}}{\pgfqpoint{3.891209in}{0.362769in}}%
\pgfpathcurveto{\pgfqpoint{3.895115in}{0.366675in}}{\pgfqpoint{3.897311in}{0.371975in}}{\pgfqpoint{3.897311in}{0.377500in}}%
\pgfpathcurveto{\pgfqpoint{3.897311in}{0.383025in}}{\pgfqpoint{3.895115in}{0.388325in}}{\pgfqpoint{3.891209in}{0.392231in}}%
\pgfpathcurveto{\pgfqpoint{3.887302in}{0.396138in}}{\pgfqpoint{3.882002in}{0.398333in}}{\pgfqpoint{3.876477in}{0.398333in}}%
\pgfpathcurveto{\pgfqpoint{3.870952in}{0.398333in}}{\pgfqpoint{3.865653in}{0.396138in}}{\pgfqpoint{3.861746in}{0.392231in}}%
\pgfpathcurveto{\pgfqpoint{3.857839in}{0.388325in}}{\pgfqpoint{3.855644in}{0.383025in}}{\pgfqpoint{3.855644in}{0.377500in}}%
\pgfpathcurveto{\pgfqpoint{3.855644in}{0.371975in}}{\pgfqpoint{3.857839in}{0.366675in}}{\pgfqpoint{3.861746in}{0.362769in}}%
\pgfpathcurveto{\pgfqpoint{3.865653in}{0.358862in}}{\pgfqpoint{3.870952in}{0.356667in}}{\pgfqpoint{3.876477in}{0.356667in}}%
\pgfpathclose%
\pgfusepath{stroke,fill}%
\end{pgfscope}%
\begin{pgfscope}%
\pgfpathrectangle{\pgfqpoint{0.562500in}{0.275000in}}{\pgfqpoint{3.487500in}{1.925000in}}%
\pgfusepath{clip}%
\pgfsetbuttcap%
\pgfsetroundjoin%
\definecolor{currentfill}{rgb}{0.000000,0.000000,0.000000}%
\pgfsetfillcolor{currentfill}%
\pgfsetlinewidth{1.003750pt}%
\definecolor{currentstroke}{rgb}{0.000000,0.000000,0.000000}%
\pgfsetstrokecolor{currentstroke}%
\pgfsetdash{}{0pt}%
\pgfpathmoveto{\pgfqpoint{3.876477in}{0.356667in}}%
\pgfpathcurveto{\pgfqpoint{3.882002in}{0.356667in}}{\pgfqpoint{3.887302in}{0.358862in}}{\pgfqpoint{3.891209in}{0.362769in}}%
\pgfpathcurveto{\pgfqpoint{3.895115in}{0.366675in}}{\pgfqpoint{3.897311in}{0.371975in}}{\pgfqpoint{3.897311in}{0.377500in}}%
\pgfpathcurveto{\pgfqpoint{3.897311in}{0.383025in}}{\pgfqpoint{3.895115in}{0.388325in}}{\pgfqpoint{3.891209in}{0.392231in}}%
\pgfpathcurveto{\pgfqpoint{3.887302in}{0.396138in}}{\pgfqpoint{3.882002in}{0.398333in}}{\pgfqpoint{3.876477in}{0.398333in}}%
\pgfpathcurveto{\pgfqpoint{3.870952in}{0.398333in}}{\pgfqpoint{3.865653in}{0.396138in}}{\pgfqpoint{3.861746in}{0.392231in}}%
\pgfpathcurveto{\pgfqpoint{3.857839in}{0.388325in}}{\pgfqpoint{3.855644in}{0.383025in}}{\pgfqpoint{3.855644in}{0.377500in}}%
\pgfpathcurveto{\pgfqpoint{3.855644in}{0.371975in}}{\pgfqpoint{3.857839in}{0.366675in}}{\pgfqpoint{3.861746in}{0.362769in}}%
\pgfpathcurveto{\pgfqpoint{3.865653in}{0.358862in}}{\pgfqpoint{3.870952in}{0.356667in}}{\pgfqpoint{3.876477in}{0.356667in}}%
\pgfpathclose%
\pgfusepath{stroke,fill}%
\end{pgfscope}%
\begin{pgfscope}%
\pgfpathrectangle{\pgfqpoint{0.562500in}{0.275000in}}{\pgfqpoint{3.487500in}{1.925000in}}%
\pgfusepath{clip}%
\pgfsetbuttcap%
\pgfsetroundjoin%
\definecolor{currentfill}{rgb}{0.000000,0.000000,0.000000}%
\pgfsetfillcolor{currentfill}%
\pgfsetlinewidth{1.003750pt}%
\definecolor{currentstroke}{rgb}{0.000000,0.000000,0.000000}%
\pgfsetstrokecolor{currentstroke}%
\pgfsetdash{}{0pt}%
\pgfpathmoveto{\pgfqpoint{3.876477in}{0.356667in}}%
\pgfpathcurveto{\pgfqpoint{3.882002in}{0.356667in}}{\pgfqpoint{3.887302in}{0.358862in}}{\pgfqpoint{3.891209in}{0.362769in}}%
\pgfpathcurveto{\pgfqpoint{3.895115in}{0.366675in}}{\pgfqpoint{3.897311in}{0.371975in}}{\pgfqpoint{3.897311in}{0.377500in}}%
\pgfpathcurveto{\pgfqpoint{3.897311in}{0.383025in}}{\pgfqpoint{3.895115in}{0.388325in}}{\pgfqpoint{3.891209in}{0.392231in}}%
\pgfpathcurveto{\pgfqpoint{3.887302in}{0.396138in}}{\pgfqpoint{3.882002in}{0.398333in}}{\pgfqpoint{3.876477in}{0.398333in}}%
\pgfpathcurveto{\pgfqpoint{3.870952in}{0.398333in}}{\pgfqpoint{3.865653in}{0.396138in}}{\pgfqpoint{3.861746in}{0.392231in}}%
\pgfpathcurveto{\pgfqpoint{3.857839in}{0.388325in}}{\pgfqpoint{3.855644in}{0.383025in}}{\pgfqpoint{3.855644in}{0.377500in}}%
\pgfpathcurveto{\pgfqpoint{3.855644in}{0.371975in}}{\pgfqpoint{3.857839in}{0.366675in}}{\pgfqpoint{3.861746in}{0.362769in}}%
\pgfpathcurveto{\pgfqpoint{3.865653in}{0.358862in}}{\pgfqpoint{3.870952in}{0.356667in}}{\pgfqpoint{3.876477in}{0.356667in}}%
\pgfpathclose%
\pgfusepath{stroke,fill}%
\end{pgfscope}%
\begin{pgfscope}%
\pgfpathrectangle{\pgfqpoint{0.562500in}{0.275000in}}{\pgfqpoint{3.487500in}{1.925000in}}%
\pgfusepath{clip}%
\pgfsetbuttcap%
\pgfsetroundjoin%
\definecolor{currentfill}{rgb}{0.000000,0.000000,0.000000}%
\pgfsetfillcolor{currentfill}%
\pgfsetlinewidth{1.003750pt}%
\definecolor{currentstroke}{rgb}{0.000000,0.000000,0.000000}%
\pgfsetstrokecolor{currentstroke}%
\pgfsetdash{}{0pt}%
\pgfpathmoveto{\pgfqpoint{3.876477in}{0.356667in}}%
\pgfpathcurveto{\pgfqpoint{3.882002in}{0.356667in}}{\pgfqpoint{3.887302in}{0.358862in}}{\pgfqpoint{3.891209in}{0.362769in}}%
\pgfpathcurveto{\pgfqpoint{3.895115in}{0.366675in}}{\pgfqpoint{3.897311in}{0.371975in}}{\pgfqpoint{3.897311in}{0.377500in}}%
\pgfpathcurveto{\pgfqpoint{3.897311in}{0.383025in}}{\pgfqpoint{3.895115in}{0.388325in}}{\pgfqpoint{3.891209in}{0.392231in}}%
\pgfpathcurveto{\pgfqpoint{3.887302in}{0.396138in}}{\pgfqpoint{3.882002in}{0.398333in}}{\pgfqpoint{3.876477in}{0.398333in}}%
\pgfpathcurveto{\pgfqpoint{3.870952in}{0.398333in}}{\pgfqpoint{3.865653in}{0.396138in}}{\pgfqpoint{3.861746in}{0.392231in}}%
\pgfpathcurveto{\pgfqpoint{3.857839in}{0.388325in}}{\pgfqpoint{3.855644in}{0.383025in}}{\pgfqpoint{3.855644in}{0.377500in}}%
\pgfpathcurveto{\pgfqpoint{3.855644in}{0.371975in}}{\pgfqpoint{3.857839in}{0.366675in}}{\pgfqpoint{3.861746in}{0.362769in}}%
\pgfpathcurveto{\pgfqpoint{3.865653in}{0.358862in}}{\pgfqpoint{3.870952in}{0.356667in}}{\pgfqpoint{3.876477in}{0.356667in}}%
\pgfpathclose%
\pgfusepath{stroke,fill}%
\end{pgfscope}%
\begin{pgfscope}%
\pgfpathrectangle{\pgfqpoint{0.562500in}{0.275000in}}{\pgfqpoint{3.487500in}{1.925000in}}%
\pgfusepath{clip}%
\pgfsetbuttcap%
\pgfsetroundjoin%
\definecolor{currentfill}{rgb}{0.000000,0.000000,0.000000}%
\pgfsetfillcolor{currentfill}%
\pgfsetlinewidth{1.003750pt}%
\definecolor{currentstroke}{rgb}{0.000000,0.000000,0.000000}%
\pgfsetstrokecolor{currentstroke}%
\pgfsetdash{}{0pt}%
\pgfpathmoveto{\pgfqpoint{3.876477in}{0.356667in}}%
\pgfpathcurveto{\pgfqpoint{3.882002in}{0.356667in}}{\pgfqpoint{3.887302in}{0.358862in}}{\pgfqpoint{3.891209in}{0.362769in}}%
\pgfpathcurveto{\pgfqpoint{3.895115in}{0.366675in}}{\pgfqpoint{3.897311in}{0.371975in}}{\pgfqpoint{3.897311in}{0.377500in}}%
\pgfpathcurveto{\pgfqpoint{3.897311in}{0.383025in}}{\pgfqpoint{3.895115in}{0.388325in}}{\pgfqpoint{3.891209in}{0.392231in}}%
\pgfpathcurveto{\pgfqpoint{3.887302in}{0.396138in}}{\pgfqpoint{3.882002in}{0.398333in}}{\pgfqpoint{3.876477in}{0.398333in}}%
\pgfpathcurveto{\pgfqpoint{3.870952in}{0.398333in}}{\pgfqpoint{3.865653in}{0.396138in}}{\pgfqpoint{3.861746in}{0.392231in}}%
\pgfpathcurveto{\pgfqpoint{3.857839in}{0.388325in}}{\pgfqpoint{3.855644in}{0.383025in}}{\pgfqpoint{3.855644in}{0.377500in}}%
\pgfpathcurveto{\pgfqpoint{3.855644in}{0.371975in}}{\pgfqpoint{3.857839in}{0.366675in}}{\pgfqpoint{3.861746in}{0.362769in}}%
\pgfpathcurveto{\pgfqpoint{3.865653in}{0.358862in}}{\pgfqpoint{3.870952in}{0.356667in}}{\pgfqpoint{3.876477in}{0.356667in}}%
\pgfpathclose%
\pgfusepath{stroke,fill}%
\end{pgfscope}%
\begin{pgfscope}%
\pgfpathrectangle{\pgfqpoint{0.562500in}{0.275000in}}{\pgfqpoint{3.487500in}{1.925000in}}%
\pgfusepath{clip}%
\pgfsetbuttcap%
\pgfsetroundjoin%
\definecolor{currentfill}{rgb}{0.000000,0.000000,0.000000}%
\pgfsetfillcolor{currentfill}%
\pgfsetlinewidth{1.003750pt}%
\definecolor{currentstroke}{rgb}{0.000000,0.000000,0.000000}%
\pgfsetstrokecolor{currentstroke}%
\pgfsetdash{}{0pt}%
\pgfpathmoveto{\pgfqpoint{3.876477in}{0.356667in}}%
\pgfpathcurveto{\pgfqpoint{3.882002in}{0.356667in}}{\pgfqpoint{3.887302in}{0.358862in}}{\pgfqpoint{3.891209in}{0.362769in}}%
\pgfpathcurveto{\pgfqpoint{3.895115in}{0.366675in}}{\pgfqpoint{3.897311in}{0.371975in}}{\pgfqpoint{3.897311in}{0.377500in}}%
\pgfpathcurveto{\pgfqpoint{3.897311in}{0.383025in}}{\pgfqpoint{3.895115in}{0.388325in}}{\pgfqpoint{3.891209in}{0.392231in}}%
\pgfpathcurveto{\pgfqpoint{3.887302in}{0.396138in}}{\pgfqpoint{3.882002in}{0.398333in}}{\pgfqpoint{3.876477in}{0.398333in}}%
\pgfpathcurveto{\pgfqpoint{3.870952in}{0.398333in}}{\pgfqpoint{3.865653in}{0.396138in}}{\pgfqpoint{3.861746in}{0.392231in}}%
\pgfpathcurveto{\pgfqpoint{3.857839in}{0.388325in}}{\pgfqpoint{3.855644in}{0.383025in}}{\pgfqpoint{3.855644in}{0.377500in}}%
\pgfpathcurveto{\pgfqpoint{3.855644in}{0.371975in}}{\pgfqpoint{3.857839in}{0.366675in}}{\pgfqpoint{3.861746in}{0.362769in}}%
\pgfpathcurveto{\pgfqpoint{3.865653in}{0.358862in}}{\pgfqpoint{3.870952in}{0.356667in}}{\pgfqpoint{3.876477in}{0.356667in}}%
\pgfpathclose%
\pgfusepath{stroke,fill}%
\end{pgfscope}%
\begin{pgfscope}%
\pgfpathrectangle{\pgfqpoint{0.562500in}{0.275000in}}{\pgfqpoint{3.487500in}{1.925000in}}%
\pgfusepath{clip}%
\pgfsetbuttcap%
\pgfsetroundjoin%
\definecolor{currentfill}{rgb}{0.000000,0.000000,0.000000}%
\pgfsetfillcolor{currentfill}%
\pgfsetlinewidth{1.003750pt}%
\definecolor{currentstroke}{rgb}{0.000000,0.000000,0.000000}%
\pgfsetstrokecolor{currentstroke}%
\pgfsetdash{}{0pt}%
\pgfpathmoveto{\pgfqpoint{3.876477in}{0.356667in}}%
\pgfpathcurveto{\pgfqpoint{3.882002in}{0.356667in}}{\pgfqpoint{3.887302in}{0.358862in}}{\pgfqpoint{3.891209in}{0.362769in}}%
\pgfpathcurveto{\pgfqpoint{3.895115in}{0.366675in}}{\pgfqpoint{3.897311in}{0.371975in}}{\pgfqpoint{3.897311in}{0.377500in}}%
\pgfpathcurveto{\pgfqpoint{3.897311in}{0.383025in}}{\pgfqpoint{3.895115in}{0.388325in}}{\pgfqpoint{3.891209in}{0.392231in}}%
\pgfpathcurveto{\pgfqpoint{3.887302in}{0.396138in}}{\pgfqpoint{3.882002in}{0.398333in}}{\pgfqpoint{3.876477in}{0.398333in}}%
\pgfpathcurveto{\pgfqpoint{3.870952in}{0.398333in}}{\pgfqpoint{3.865653in}{0.396138in}}{\pgfqpoint{3.861746in}{0.392231in}}%
\pgfpathcurveto{\pgfqpoint{3.857839in}{0.388325in}}{\pgfqpoint{3.855644in}{0.383025in}}{\pgfqpoint{3.855644in}{0.377500in}}%
\pgfpathcurveto{\pgfqpoint{3.855644in}{0.371975in}}{\pgfqpoint{3.857839in}{0.366675in}}{\pgfqpoint{3.861746in}{0.362769in}}%
\pgfpathcurveto{\pgfqpoint{3.865653in}{0.358862in}}{\pgfqpoint{3.870952in}{0.356667in}}{\pgfqpoint{3.876477in}{0.356667in}}%
\pgfpathclose%
\pgfusepath{stroke,fill}%
\end{pgfscope}%
\begin{pgfscope}%
\pgfpathrectangle{\pgfqpoint{0.562500in}{0.275000in}}{\pgfqpoint{3.487500in}{1.925000in}}%
\pgfusepath{clip}%
\pgfsetbuttcap%
\pgfsetroundjoin%
\definecolor{currentfill}{rgb}{0.000000,0.000000,0.000000}%
\pgfsetfillcolor{currentfill}%
\pgfsetlinewidth{1.003750pt}%
\definecolor{currentstroke}{rgb}{0.000000,0.000000,0.000000}%
\pgfsetstrokecolor{currentstroke}%
\pgfsetdash{}{0pt}%
\pgfpathmoveto{\pgfqpoint{3.876477in}{0.356667in}}%
\pgfpathcurveto{\pgfqpoint{3.882002in}{0.356667in}}{\pgfqpoint{3.887302in}{0.358862in}}{\pgfqpoint{3.891209in}{0.362769in}}%
\pgfpathcurveto{\pgfqpoint{3.895115in}{0.366675in}}{\pgfqpoint{3.897311in}{0.371975in}}{\pgfqpoint{3.897311in}{0.377500in}}%
\pgfpathcurveto{\pgfqpoint{3.897311in}{0.383025in}}{\pgfqpoint{3.895115in}{0.388325in}}{\pgfqpoint{3.891209in}{0.392231in}}%
\pgfpathcurveto{\pgfqpoint{3.887302in}{0.396138in}}{\pgfqpoint{3.882002in}{0.398333in}}{\pgfqpoint{3.876477in}{0.398333in}}%
\pgfpathcurveto{\pgfqpoint{3.870952in}{0.398333in}}{\pgfqpoint{3.865653in}{0.396138in}}{\pgfqpoint{3.861746in}{0.392231in}}%
\pgfpathcurveto{\pgfqpoint{3.857839in}{0.388325in}}{\pgfqpoint{3.855644in}{0.383025in}}{\pgfqpoint{3.855644in}{0.377500in}}%
\pgfpathcurveto{\pgfqpoint{3.855644in}{0.371975in}}{\pgfqpoint{3.857839in}{0.366675in}}{\pgfqpoint{3.861746in}{0.362769in}}%
\pgfpathcurveto{\pgfqpoint{3.865653in}{0.358862in}}{\pgfqpoint{3.870952in}{0.356667in}}{\pgfqpoint{3.876477in}{0.356667in}}%
\pgfpathclose%
\pgfusepath{stroke,fill}%
\end{pgfscope}%
\begin{pgfscope}%
\pgfpathrectangle{\pgfqpoint{0.562500in}{0.275000in}}{\pgfqpoint{3.487500in}{1.925000in}}%
\pgfusepath{clip}%
\pgfsetbuttcap%
\pgfsetroundjoin%
\definecolor{currentfill}{rgb}{0.000000,0.000000,0.000000}%
\pgfsetfillcolor{currentfill}%
\pgfsetlinewidth{1.003750pt}%
\definecolor{currentstroke}{rgb}{0.000000,0.000000,0.000000}%
\pgfsetstrokecolor{currentstroke}%
\pgfsetdash{}{0pt}%
\pgfpathmoveto{\pgfqpoint{3.876477in}{0.356667in}}%
\pgfpathcurveto{\pgfqpoint{3.882002in}{0.356667in}}{\pgfqpoint{3.887302in}{0.358862in}}{\pgfqpoint{3.891209in}{0.362769in}}%
\pgfpathcurveto{\pgfqpoint{3.895115in}{0.366675in}}{\pgfqpoint{3.897311in}{0.371975in}}{\pgfqpoint{3.897311in}{0.377500in}}%
\pgfpathcurveto{\pgfqpoint{3.897311in}{0.383025in}}{\pgfqpoint{3.895115in}{0.388325in}}{\pgfqpoint{3.891209in}{0.392231in}}%
\pgfpathcurveto{\pgfqpoint{3.887302in}{0.396138in}}{\pgfqpoint{3.882002in}{0.398333in}}{\pgfqpoint{3.876477in}{0.398333in}}%
\pgfpathcurveto{\pgfqpoint{3.870952in}{0.398333in}}{\pgfqpoint{3.865653in}{0.396138in}}{\pgfqpoint{3.861746in}{0.392231in}}%
\pgfpathcurveto{\pgfqpoint{3.857839in}{0.388325in}}{\pgfqpoint{3.855644in}{0.383025in}}{\pgfqpoint{3.855644in}{0.377500in}}%
\pgfpathcurveto{\pgfqpoint{3.855644in}{0.371975in}}{\pgfqpoint{3.857839in}{0.366675in}}{\pgfqpoint{3.861746in}{0.362769in}}%
\pgfpathcurveto{\pgfqpoint{3.865653in}{0.358862in}}{\pgfqpoint{3.870952in}{0.356667in}}{\pgfqpoint{3.876477in}{0.356667in}}%
\pgfpathclose%
\pgfusepath{stroke,fill}%
\end{pgfscope}%
\begin{pgfscope}%
\pgfpathrectangle{\pgfqpoint{0.562500in}{0.275000in}}{\pgfqpoint{3.487500in}{1.925000in}}%
\pgfusepath{clip}%
\pgfsetbuttcap%
\pgfsetroundjoin%
\definecolor{currentfill}{rgb}{0.000000,0.000000,0.000000}%
\pgfsetfillcolor{currentfill}%
\pgfsetlinewidth{1.003750pt}%
\definecolor{currentstroke}{rgb}{0.000000,0.000000,0.000000}%
\pgfsetstrokecolor{currentstroke}%
\pgfsetdash{}{0pt}%
\pgfpathmoveto{\pgfqpoint{3.876477in}{0.356667in}}%
\pgfpathcurveto{\pgfqpoint{3.882002in}{0.356667in}}{\pgfqpoint{3.887302in}{0.358862in}}{\pgfqpoint{3.891209in}{0.362769in}}%
\pgfpathcurveto{\pgfqpoint{3.895115in}{0.366675in}}{\pgfqpoint{3.897311in}{0.371975in}}{\pgfqpoint{3.897311in}{0.377500in}}%
\pgfpathcurveto{\pgfqpoint{3.897311in}{0.383025in}}{\pgfqpoint{3.895115in}{0.388325in}}{\pgfqpoint{3.891209in}{0.392231in}}%
\pgfpathcurveto{\pgfqpoint{3.887302in}{0.396138in}}{\pgfqpoint{3.882002in}{0.398333in}}{\pgfqpoint{3.876477in}{0.398333in}}%
\pgfpathcurveto{\pgfqpoint{3.870952in}{0.398333in}}{\pgfqpoint{3.865653in}{0.396138in}}{\pgfqpoint{3.861746in}{0.392231in}}%
\pgfpathcurveto{\pgfqpoint{3.857839in}{0.388325in}}{\pgfqpoint{3.855644in}{0.383025in}}{\pgfqpoint{3.855644in}{0.377500in}}%
\pgfpathcurveto{\pgfqpoint{3.855644in}{0.371975in}}{\pgfqpoint{3.857839in}{0.366675in}}{\pgfqpoint{3.861746in}{0.362769in}}%
\pgfpathcurveto{\pgfqpoint{3.865653in}{0.358862in}}{\pgfqpoint{3.870952in}{0.356667in}}{\pgfqpoint{3.876477in}{0.356667in}}%
\pgfpathclose%
\pgfusepath{stroke,fill}%
\end{pgfscope}%
\begin{pgfscope}%
\pgfpathrectangle{\pgfqpoint{0.562500in}{0.275000in}}{\pgfqpoint{3.487500in}{1.925000in}}%
\pgfusepath{clip}%
\pgfsetbuttcap%
\pgfsetroundjoin%
\definecolor{currentfill}{rgb}{0.000000,0.000000,0.000000}%
\pgfsetfillcolor{currentfill}%
\pgfsetlinewidth{1.003750pt}%
\definecolor{currentstroke}{rgb}{0.000000,0.000000,0.000000}%
\pgfsetstrokecolor{currentstroke}%
\pgfsetdash{}{0pt}%
\pgfpathmoveto{\pgfqpoint{3.876477in}{0.356667in}}%
\pgfpathcurveto{\pgfqpoint{3.882002in}{0.356667in}}{\pgfqpoint{3.887302in}{0.358862in}}{\pgfqpoint{3.891209in}{0.362769in}}%
\pgfpathcurveto{\pgfqpoint{3.895115in}{0.366675in}}{\pgfqpoint{3.897311in}{0.371975in}}{\pgfqpoint{3.897311in}{0.377500in}}%
\pgfpathcurveto{\pgfqpoint{3.897311in}{0.383025in}}{\pgfqpoint{3.895115in}{0.388325in}}{\pgfqpoint{3.891209in}{0.392231in}}%
\pgfpathcurveto{\pgfqpoint{3.887302in}{0.396138in}}{\pgfqpoint{3.882002in}{0.398333in}}{\pgfqpoint{3.876477in}{0.398333in}}%
\pgfpathcurveto{\pgfqpoint{3.870952in}{0.398333in}}{\pgfqpoint{3.865653in}{0.396138in}}{\pgfqpoint{3.861746in}{0.392231in}}%
\pgfpathcurveto{\pgfqpoint{3.857839in}{0.388325in}}{\pgfqpoint{3.855644in}{0.383025in}}{\pgfqpoint{3.855644in}{0.377500in}}%
\pgfpathcurveto{\pgfqpoint{3.855644in}{0.371975in}}{\pgfqpoint{3.857839in}{0.366675in}}{\pgfqpoint{3.861746in}{0.362769in}}%
\pgfpathcurveto{\pgfqpoint{3.865653in}{0.358862in}}{\pgfqpoint{3.870952in}{0.356667in}}{\pgfqpoint{3.876477in}{0.356667in}}%
\pgfpathclose%
\pgfusepath{stroke,fill}%
\end{pgfscope}%
\begin{pgfscope}%
\pgfpathrectangle{\pgfqpoint{0.562500in}{0.275000in}}{\pgfqpoint{3.487500in}{1.925000in}}%
\pgfusepath{clip}%
\pgfsetbuttcap%
\pgfsetroundjoin%
\definecolor{currentfill}{rgb}{0.000000,0.000000,0.000000}%
\pgfsetfillcolor{currentfill}%
\pgfsetlinewidth{1.003750pt}%
\definecolor{currentstroke}{rgb}{0.000000,0.000000,0.000000}%
\pgfsetstrokecolor{currentstroke}%
\pgfsetdash{}{0pt}%
\pgfpathmoveto{\pgfqpoint{3.876477in}{0.356667in}}%
\pgfpathcurveto{\pgfqpoint{3.882002in}{0.356667in}}{\pgfqpoint{3.887302in}{0.358862in}}{\pgfqpoint{3.891209in}{0.362769in}}%
\pgfpathcurveto{\pgfqpoint{3.895115in}{0.366675in}}{\pgfqpoint{3.897311in}{0.371975in}}{\pgfqpoint{3.897311in}{0.377500in}}%
\pgfpathcurveto{\pgfqpoint{3.897311in}{0.383025in}}{\pgfqpoint{3.895115in}{0.388325in}}{\pgfqpoint{3.891209in}{0.392231in}}%
\pgfpathcurveto{\pgfqpoint{3.887302in}{0.396138in}}{\pgfqpoint{3.882002in}{0.398333in}}{\pgfqpoint{3.876477in}{0.398333in}}%
\pgfpathcurveto{\pgfqpoint{3.870952in}{0.398333in}}{\pgfqpoint{3.865653in}{0.396138in}}{\pgfqpoint{3.861746in}{0.392231in}}%
\pgfpathcurveto{\pgfqpoint{3.857839in}{0.388325in}}{\pgfqpoint{3.855644in}{0.383025in}}{\pgfqpoint{3.855644in}{0.377500in}}%
\pgfpathcurveto{\pgfqpoint{3.855644in}{0.371975in}}{\pgfqpoint{3.857839in}{0.366675in}}{\pgfqpoint{3.861746in}{0.362769in}}%
\pgfpathcurveto{\pgfqpoint{3.865653in}{0.358862in}}{\pgfqpoint{3.870952in}{0.356667in}}{\pgfqpoint{3.876477in}{0.356667in}}%
\pgfpathclose%
\pgfusepath{stroke,fill}%
\end{pgfscope}%
\begin{pgfscope}%
\pgfpathrectangle{\pgfqpoint{0.562500in}{0.275000in}}{\pgfqpoint{3.487500in}{1.925000in}}%
\pgfusepath{clip}%
\pgfsetbuttcap%
\pgfsetroundjoin%
\definecolor{currentfill}{rgb}{0.000000,0.000000,0.000000}%
\pgfsetfillcolor{currentfill}%
\pgfsetlinewidth{1.003750pt}%
\definecolor{currentstroke}{rgb}{0.000000,0.000000,0.000000}%
\pgfsetstrokecolor{currentstroke}%
\pgfsetdash{}{0pt}%
\pgfpathmoveto{\pgfqpoint{3.876477in}{0.356667in}}%
\pgfpathcurveto{\pgfqpoint{3.882002in}{0.356667in}}{\pgfqpoint{3.887302in}{0.358862in}}{\pgfqpoint{3.891209in}{0.362769in}}%
\pgfpathcurveto{\pgfqpoint{3.895115in}{0.366675in}}{\pgfqpoint{3.897311in}{0.371975in}}{\pgfqpoint{3.897311in}{0.377500in}}%
\pgfpathcurveto{\pgfqpoint{3.897311in}{0.383025in}}{\pgfqpoint{3.895115in}{0.388325in}}{\pgfqpoint{3.891209in}{0.392231in}}%
\pgfpathcurveto{\pgfqpoint{3.887302in}{0.396138in}}{\pgfqpoint{3.882002in}{0.398333in}}{\pgfqpoint{3.876477in}{0.398333in}}%
\pgfpathcurveto{\pgfqpoint{3.870952in}{0.398333in}}{\pgfqpoint{3.865653in}{0.396138in}}{\pgfqpoint{3.861746in}{0.392231in}}%
\pgfpathcurveto{\pgfqpoint{3.857839in}{0.388325in}}{\pgfqpoint{3.855644in}{0.383025in}}{\pgfqpoint{3.855644in}{0.377500in}}%
\pgfpathcurveto{\pgfqpoint{3.855644in}{0.371975in}}{\pgfqpoint{3.857839in}{0.366675in}}{\pgfqpoint{3.861746in}{0.362769in}}%
\pgfpathcurveto{\pgfqpoint{3.865653in}{0.358862in}}{\pgfqpoint{3.870952in}{0.356667in}}{\pgfqpoint{3.876477in}{0.356667in}}%
\pgfpathclose%
\pgfusepath{stroke,fill}%
\end{pgfscope}%
\begin{pgfscope}%
\pgfpathrectangle{\pgfqpoint{0.562500in}{0.275000in}}{\pgfqpoint{3.487500in}{1.925000in}}%
\pgfusepath{clip}%
\pgfsetbuttcap%
\pgfsetroundjoin%
\definecolor{currentfill}{rgb}{0.000000,0.000000,0.000000}%
\pgfsetfillcolor{currentfill}%
\pgfsetlinewidth{1.003750pt}%
\definecolor{currentstroke}{rgb}{0.000000,0.000000,0.000000}%
\pgfsetstrokecolor{currentstroke}%
\pgfsetdash{}{0pt}%
\pgfpathmoveto{\pgfqpoint{3.876477in}{0.356667in}}%
\pgfpathcurveto{\pgfqpoint{3.882002in}{0.356667in}}{\pgfqpoint{3.887302in}{0.358862in}}{\pgfqpoint{3.891209in}{0.362769in}}%
\pgfpathcurveto{\pgfqpoint{3.895115in}{0.366675in}}{\pgfqpoint{3.897311in}{0.371975in}}{\pgfqpoint{3.897311in}{0.377500in}}%
\pgfpathcurveto{\pgfqpoint{3.897311in}{0.383025in}}{\pgfqpoint{3.895115in}{0.388325in}}{\pgfqpoint{3.891209in}{0.392231in}}%
\pgfpathcurveto{\pgfqpoint{3.887302in}{0.396138in}}{\pgfqpoint{3.882002in}{0.398333in}}{\pgfqpoint{3.876477in}{0.398333in}}%
\pgfpathcurveto{\pgfqpoint{3.870952in}{0.398333in}}{\pgfqpoint{3.865653in}{0.396138in}}{\pgfqpoint{3.861746in}{0.392231in}}%
\pgfpathcurveto{\pgfqpoint{3.857839in}{0.388325in}}{\pgfqpoint{3.855644in}{0.383025in}}{\pgfqpoint{3.855644in}{0.377500in}}%
\pgfpathcurveto{\pgfqpoint{3.855644in}{0.371975in}}{\pgfqpoint{3.857839in}{0.366675in}}{\pgfqpoint{3.861746in}{0.362769in}}%
\pgfpathcurveto{\pgfqpoint{3.865653in}{0.358862in}}{\pgfqpoint{3.870952in}{0.356667in}}{\pgfqpoint{3.876477in}{0.356667in}}%
\pgfpathclose%
\pgfusepath{stroke,fill}%
\end{pgfscope}%
\begin{pgfscope}%
\pgfpathrectangle{\pgfqpoint{0.562500in}{0.275000in}}{\pgfqpoint{3.487500in}{1.925000in}}%
\pgfusepath{clip}%
\pgfsetbuttcap%
\pgfsetroundjoin%
\definecolor{currentfill}{rgb}{0.000000,0.000000,0.000000}%
\pgfsetfillcolor{currentfill}%
\pgfsetlinewidth{1.003750pt}%
\definecolor{currentstroke}{rgb}{0.000000,0.000000,0.000000}%
\pgfsetstrokecolor{currentstroke}%
\pgfsetdash{}{0pt}%
\pgfpathmoveto{\pgfqpoint{3.876477in}{0.356667in}}%
\pgfpathcurveto{\pgfqpoint{3.882002in}{0.356667in}}{\pgfqpoint{3.887302in}{0.358862in}}{\pgfqpoint{3.891209in}{0.362769in}}%
\pgfpathcurveto{\pgfqpoint{3.895115in}{0.366675in}}{\pgfqpoint{3.897311in}{0.371975in}}{\pgfqpoint{3.897311in}{0.377500in}}%
\pgfpathcurveto{\pgfqpoint{3.897311in}{0.383025in}}{\pgfqpoint{3.895115in}{0.388325in}}{\pgfqpoint{3.891209in}{0.392231in}}%
\pgfpathcurveto{\pgfqpoint{3.887302in}{0.396138in}}{\pgfqpoint{3.882002in}{0.398333in}}{\pgfqpoint{3.876477in}{0.398333in}}%
\pgfpathcurveto{\pgfqpoint{3.870952in}{0.398333in}}{\pgfqpoint{3.865653in}{0.396138in}}{\pgfqpoint{3.861746in}{0.392231in}}%
\pgfpathcurveto{\pgfqpoint{3.857839in}{0.388325in}}{\pgfqpoint{3.855644in}{0.383025in}}{\pgfqpoint{3.855644in}{0.377500in}}%
\pgfpathcurveto{\pgfqpoint{3.855644in}{0.371975in}}{\pgfqpoint{3.857839in}{0.366675in}}{\pgfqpoint{3.861746in}{0.362769in}}%
\pgfpathcurveto{\pgfqpoint{3.865653in}{0.358862in}}{\pgfqpoint{3.870952in}{0.356667in}}{\pgfqpoint{3.876477in}{0.356667in}}%
\pgfpathclose%
\pgfusepath{stroke,fill}%
\end{pgfscope}%
\begin{pgfscope}%
\pgfpathrectangle{\pgfqpoint{0.562500in}{0.275000in}}{\pgfqpoint{3.487500in}{1.925000in}}%
\pgfusepath{clip}%
\pgfsetbuttcap%
\pgfsetroundjoin%
\definecolor{currentfill}{rgb}{0.000000,0.000000,0.000000}%
\pgfsetfillcolor{currentfill}%
\pgfsetlinewidth{1.003750pt}%
\definecolor{currentstroke}{rgb}{0.000000,0.000000,0.000000}%
\pgfsetstrokecolor{currentstroke}%
\pgfsetdash{}{0pt}%
\pgfpathmoveto{\pgfqpoint{3.876477in}{0.356667in}}%
\pgfpathcurveto{\pgfqpoint{3.882002in}{0.356667in}}{\pgfqpoint{3.887302in}{0.358862in}}{\pgfqpoint{3.891209in}{0.362769in}}%
\pgfpathcurveto{\pgfqpoint{3.895115in}{0.366675in}}{\pgfqpoint{3.897311in}{0.371975in}}{\pgfqpoint{3.897311in}{0.377500in}}%
\pgfpathcurveto{\pgfqpoint{3.897311in}{0.383025in}}{\pgfqpoint{3.895115in}{0.388325in}}{\pgfqpoint{3.891209in}{0.392231in}}%
\pgfpathcurveto{\pgfqpoint{3.887302in}{0.396138in}}{\pgfqpoint{3.882002in}{0.398333in}}{\pgfqpoint{3.876477in}{0.398333in}}%
\pgfpathcurveto{\pgfqpoint{3.870952in}{0.398333in}}{\pgfqpoint{3.865653in}{0.396138in}}{\pgfqpoint{3.861746in}{0.392231in}}%
\pgfpathcurveto{\pgfqpoint{3.857839in}{0.388325in}}{\pgfqpoint{3.855644in}{0.383025in}}{\pgfqpoint{3.855644in}{0.377500in}}%
\pgfpathcurveto{\pgfqpoint{3.855644in}{0.371975in}}{\pgfqpoint{3.857839in}{0.366675in}}{\pgfqpoint{3.861746in}{0.362769in}}%
\pgfpathcurveto{\pgfqpoint{3.865653in}{0.358862in}}{\pgfqpoint{3.870952in}{0.356667in}}{\pgfqpoint{3.876477in}{0.356667in}}%
\pgfpathclose%
\pgfusepath{stroke,fill}%
\end{pgfscope}%
\begin{pgfscope}%
\pgfpathrectangle{\pgfqpoint{0.562500in}{0.275000in}}{\pgfqpoint{3.487500in}{1.925000in}}%
\pgfusepath{clip}%
\pgfsetbuttcap%
\pgfsetroundjoin%
\definecolor{currentfill}{rgb}{0.000000,0.000000,0.000000}%
\pgfsetfillcolor{currentfill}%
\pgfsetlinewidth{1.003750pt}%
\definecolor{currentstroke}{rgb}{0.000000,0.000000,0.000000}%
\pgfsetstrokecolor{currentstroke}%
\pgfsetdash{}{0pt}%
\pgfpathmoveto{\pgfqpoint{3.876477in}{0.356667in}}%
\pgfpathcurveto{\pgfqpoint{3.882002in}{0.356667in}}{\pgfqpoint{3.887302in}{0.358862in}}{\pgfqpoint{3.891209in}{0.362769in}}%
\pgfpathcurveto{\pgfqpoint{3.895115in}{0.366675in}}{\pgfqpoint{3.897311in}{0.371975in}}{\pgfqpoint{3.897311in}{0.377500in}}%
\pgfpathcurveto{\pgfqpoint{3.897311in}{0.383025in}}{\pgfqpoint{3.895115in}{0.388325in}}{\pgfqpoint{3.891209in}{0.392231in}}%
\pgfpathcurveto{\pgfqpoint{3.887302in}{0.396138in}}{\pgfqpoint{3.882002in}{0.398333in}}{\pgfqpoint{3.876477in}{0.398333in}}%
\pgfpathcurveto{\pgfqpoint{3.870952in}{0.398333in}}{\pgfqpoint{3.865653in}{0.396138in}}{\pgfqpoint{3.861746in}{0.392231in}}%
\pgfpathcurveto{\pgfqpoint{3.857839in}{0.388325in}}{\pgfqpoint{3.855644in}{0.383025in}}{\pgfqpoint{3.855644in}{0.377500in}}%
\pgfpathcurveto{\pgfqpoint{3.855644in}{0.371975in}}{\pgfqpoint{3.857839in}{0.366675in}}{\pgfqpoint{3.861746in}{0.362769in}}%
\pgfpathcurveto{\pgfqpoint{3.865653in}{0.358862in}}{\pgfqpoint{3.870952in}{0.356667in}}{\pgfqpoint{3.876477in}{0.356667in}}%
\pgfpathclose%
\pgfusepath{stroke,fill}%
\end{pgfscope}%
\begin{pgfscope}%
\pgfpathrectangle{\pgfqpoint{0.562500in}{0.275000in}}{\pgfqpoint{3.487500in}{1.925000in}}%
\pgfusepath{clip}%
\pgfsetbuttcap%
\pgfsetroundjoin%
\definecolor{currentfill}{rgb}{0.000000,0.000000,0.000000}%
\pgfsetfillcolor{currentfill}%
\pgfsetlinewidth{1.003750pt}%
\definecolor{currentstroke}{rgb}{0.000000,0.000000,0.000000}%
\pgfsetstrokecolor{currentstroke}%
\pgfsetdash{}{0pt}%
\pgfpathmoveto{\pgfqpoint{3.876477in}{0.356667in}}%
\pgfpathcurveto{\pgfqpoint{3.882002in}{0.356667in}}{\pgfqpoint{3.887302in}{0.358862in}}{\pgfqpoint{3.891209in}{0.362769in}}%
\pgfpathcurveto{\pgfqpoint{3.895115in}{0.366675in}}{\pgfqpoint{3.897311in}{0.371975in}}{\pgfqpoint{3.897311in}{0.377500in}}%
\pgfpathcurveto{\pgfqpoint{3.897311in}{0.383025in}}{\pgfqpoint{3.895115in}{0.388325in}}{\pgfqpoint{3.891209in}{0.392231in}}%
\pgfpathcurveto{\pgfqpoint{3.887302in}{0.396138in}}{\pgfqpoint{3.882002in}{0.398333in}}{\pgfqpoint{3.876477in}{0.398333in}}%
\pgfpathcurveto{\pgfqpoint{3.870952in}{0.398333in}}{\pgfqpoint{3.865653in}{0.396138in}}{\pgfqpoint{3.861746in}{0.392231in}}%
\pgfpathcurveto{\pgfqpoint{3.857839in}{0.388325in}}{\pgfqpoint{3.855644in}{0.383025in}}{\pgfqpoint{3.855644in}{0.377500in}}%
\pgfpathcurveto{\pgfqpoint{3.855644in}{0.371975in}}{\pgfqpoint{3.857839in}{0.366675in}}{\pgfqpoint{3.861746in}{0.362769in}}%
\pgfpathcurveto{\pgfqpoint{3.865653in}{0.358862in}}{\pgfqpoint{3.870952in}{0.356667in}}{\pgfqpoint{3.876477in}{0.356667in}}%
\pgfpathclose%
\pgfusepath{stroke,fill}%
\end{pgfscope}%
\begin{pgfscope}%
\pgfpathrectangle{\pgfqpoint{0.562500in}{0.275000in}}{\pgfqpoint{3.487500in}{1.925000in}}%
\pgfusepath{clip}%
\pgfsetbuttcap%
\pgfsetroundjoin%
\definecolor{currentfill}{rgb}{0.000000,0.000000,0.000000}%
\pgfsetfillcolor{currentfill}%
\pgfsetlinewidth{1.003750pt}%
\definecolor{currentstroke}{rgb}{0.000000,0.000000,0.000000}%
\pgfsetstrokecolor{currentstroke}%
\pgfsetdash{}{0pt}%
\pgfpathmoveto{\pgfqpoint{3.876477in}{0.356667in}}%
\pgfpathcurveto{\pgfqpoint{3.882002in}{0.356667in}}{\pgfqpoint{3.887302in}{0.358862in}}{\pgfqpoint{3.891209in}{0.362769in}}%
\pgfpathcurveto{\pgfqpoint{3.895115in}{0.366675in}}{\pgfqpoint{3.897311in}{0.371975in}}{\pgfqpoint{3.897311in}{0.377500in}}%
\pgfpathcurveto{\pgfqpoint{3.897311in}{0.383025in}}{\pgfqpoint{3.895115in}{0.388325in}}{\pgfqpoint{3.891209in}{0.392231in}}%
\pgfpathcurveto{\pgfqpoint{3.887302in}{0.396138in}}{\pgfqpoint{3.882002in}{0.398333in}}{\pgfqpoint{3.876477in}{0.398333in}}%
\pgfpathcurveto{\pgfqpoint{3.870952in}{0.398333in}}{\pgfqpoint{3.865653in}{0.396138in}}{\pgfqpoint{3.861746in}{0.392231in}}%
\pgfpathcurveto{\pgfqpoint{3.857839in}{0.388325in}}{\pgfqpoint{3.855644in}{0.383025in}}{\pgfqpoint{3.855644in}{0.377500in}}%
\pgfpathcurveto{\pgfqpoint{3.855644in}{0.371975in}}{\pgfqpoint{3.857839in}{0.366675in}}{\pgfqpoint{3.861746in}{0.362769in}}%
\pgfpathcurveto{\pgfqpoint{3.865653in}{0.358862in}}{\pgfqpoint{3.870952in}{0.356667in}}{\pgfqpoint{3.876477in}{0.356667in}}%
\pgfpathclose%
\pgfusepath{stroke,fill}%
\end{pgfscope}%
\begin{pgfscope}%
\pgfpathrectangle{\pgfqpoint{0.562500in}{0.275000in}}{\pgfqpoint{3.487500in}{1.925000in}}%
\pgfusepath{clip}%
\pgfsetbuttcap%
\pgfsetroundjoin%
\definecolor{currentfill}{rgb}{0.000000,0.000000,0.000000}%
\pgfsetfillcolor{currentfill}%
\pgfsetlinewidth{1.003750pt}%
\definecolor{currentstroke}{rgb}{0.000000,0.000000,0.000000}%
\pgfsetstrokecolor{currentstroke}%
\pgfsetdash{}{0pt}%
\pgfpathmoveto{\pgfqpoint{3.876477in}{0.356667in}}%
\pgfpathcurveto{\pgfqpoint{3.882002in}{0.356667in}}{\pgfqpoint{3.887302in}{0.358862in}}{\pgfqpoint{3.891209in}{0.362769in}}%
\pgfpathcurveto{\pgfqpoint{3.895115in}{0.366675in}}{\pgfqpoint{3.897311in}{0.371975in}}{\pgfqpoint{3.897311in}{0.377500in}}%
\pgfpathcurveto{\pgfqpoint{3.897311in}{0.383025in}}{\pgfqpoint{3.895115in}{0.388325in}}{\pgfqpoint{3.891209in}{0.392231in}}%
\pgfpathcurveto{\pgfqpoint{3.887302in}{0.396138in}}{\pgfqpoint{3.882002in}{0.398333in}}{\pgfqpoint{3.876477in}{0.398333in}}%
\pgfpathcurveto{\pgfqpoint{3.870952in}{0.398333in}}{\pgfqpoint{3.865653in}{0.396138in}}{\pgfqpoint{3.861746in}{0.392231in}}%
\pgfpathcurveto{\pgfqpoint{3.857839in}{0.388325in}}{\pgfqpoint{3.855644in}{0.383025in}}{\pgfqpoint{3.855644in}{0.377500in}}%
\pgfpathcurveto{\pgfqpoint{3.855644in}{0.371975in}}{\pgfqpoint{3.857839in}{0.366675in}}{\pgfqpoint{3.861746in}{0.362769in}}%
\pgfpathcurveto{\pgfqpoint{3.865653in}{0.358862in}}{\pgfqpoint{3.870952in}{0.356667in}}{\pgfqpoint{3.876477in}{0.356667in}}%
\pgfpathclose%
\pgfusepath{stroke,fill}%
\end{pgfscope}%
\begin{pgfscope}%
\pgfpathrectangle{\pgfqpoint{0.562500in}{0.275000in}}{\pgfqpoint{3.487500in}{1.925000in}}%
\pgfusepath{clip}%
\pgfsetbuttcap%
\pgfsetroundjoin%
\definecolor{currentfill}{rgb}{0.000000,0.000000,0.000000}%
\pgfsetfillcolor{currentfill}%
\pgfsetlinewidth{1.003750pt}%
\definecolor{currentstroke}{rgb}{0.000000,0.000000,0.000000}%
\pgfsetstrokecolor{currentstroke}%
\pgfsetdash{}{0pt}%
\pgfpathmoveto{\pgfqpoint{3.876477in}{0.356667in}}%
\pgfpathcurveto{\pgfqpoint{3.882002in}{0.356667in}}{\pgfqpoint{3.887302in}{0.358862in}}{\pgfqpoint{3.891209in}{0.362769in}}%
\pgfpathcurveto{\pgfqpoint{3.895115in}{0.366675in}}{\pgfqpoint{3.897311in}{0.371975in}}{\pgfqpoint{3.897311in}{0.377500in}}%
\pgfpathcurveto{\pgfqpoint{3.897311in}{0.383025in}}{\pgfqpoint{3.895115in}{0.388325in}}{\pgfqpoint{3.891209in}{0.392231in}}%
\pgfpathcurveto{\pgfqpoint{3.887302in}{0.396138in}}{\pgfqpoint{3.882002in}{0.398333in}}{\pgfqpoint{3.876477in}{0.398333in}}%
\pgfpathcurveto{\pgfqpoint{3.870952in}{0.398333in}}{\pgfqpoint{3.865653in}{0.396138in}}{\pgfqpoint{3.861746in}{0.392231in}}%
\pgfpathcurveto{\pgfqpoint{3.857839in}{0.388325in}}{\pgfqpoint{3.855644in}{0.383025in}}{\pgfqpoint{3.855644in}{0.377500in}}%
\pgfpathcurveto{\pgfqpoint{3.855644in}{0.371975in}}{\pgfqpoint{3.857839in}{0.366675in}}{\pgfqpoint{3.861746in}{0.362769in}}%
\pgfpathcurveto{\pgfqpoint{3.865653in}{0.358862in}}{\pgfqpoint{3.870952in}{0.356667in}}{\pgfqpoint{3.876477in}{0.356667in}}%
\pgfpathclose%
\pgfusepath{stroke,fill}%
\end{pgfscope}%
\begin{pgfscope}%
\pgfpathrectangle{\pgfqpoint{0.562500in}{0.275000in}}{\pgfqpoint{3.487500in}{1.925000in}}%
\pgfusepath{clip}%
\pgfsetbuttcap%
\pgfsetroundjoin%
\definecolor{currentfill}{rgb}{0.000000,0.000000,0.000000}%
\pgfsetfillcolor{currentfill}%
\pgfsetlinewidth{1.003750pt}%
\definecolor{currentstroke}{rgb}{0.000000,0.000000,0.000000}%
\pgfsetstrokecolor{currentstroke}%
\pgfsetdash{}{0pt}%
\pgfpathmoveto{\pgfqpoint{3.876477in}{0.356667in}}%
\pgfpathcurveto{\pgfqpoint{3.882002in}{0.356667in}}{\pgfqpoint{3.887302in}{0.358862in}}{\pgfqpoint{3.891209in}{0.362769in}}%
\pgfpathcurveto{\pgfqpoint{3.895115in}{0.366675in}}{\pgfqpoint{3.897311in}{0.371975in}}{\pgfqpoint{3.897311in}{0.377500in}}%
\pgfpathcurveto{\pgfqpoint{3.897311in}{0.383025in}}{\pgfqpoint{3.895115in}{0.388325in}}{\pgfqpoint{3.891209in}{0.392231in}}%
\pgfpathcurveto{\pgfqpoint{3.887302in}{0.396138in}}{\pgfqpoint{3.882002in}{0.398333in}}{\pgfqpoint{3.876477in}{0.398333in}}%
\pgfpathcurveto{\pgfqpoint{3.870952in}{0.398333in}}{\pgfqpoint{3.865653in}{0.396138in}}{\pgfqpoint{3.861746in}{0.392231in}}%
\pgfpathcurveto{\pgfqpoint{3.857839in}{0.388325in}}{\pgfqpoint{3.855644in}{0.383025in}}{\pgfqpoint{3.855644in}{0.377500in}}%
\pgfpathcurveto{\pgfqpoint{3.855644in}{0.371975in}}{\pgfqpoint{3.857839in}{0.366675in}}{\pgfqpoint{3.861746in}{0.362769in}}%
\pgfpathcurveto{\pgfqpoint{3.865653in}{0.358862in}}{\pgfqpoint{3.870952in}{0.356667in}}{\pgfqpoint{3.876477in}{0.356667in}}%
\pgfpathclose%
\pgfusepath{stroke,fill}%
\end{pgfscope}%
\begin{pgfscope}%
\pgfpathrectangle{\pgfqpoint{0.562500in}{0.275000in}}{\pgfqpoint{3.487500in}{1.925000in}}%
\pgfusepath{clip}%
\pgfsetbuttcap%
\pgfsetroundjoin%
\definecolor{currentfill}{rgb}{0.000000,0.000000,0.000000}%
\pgfsetfillcolor{currentfill}%
\pgfsetlinewidth{1.003750pt}%
\definecolor{currentstroke}{rgb}{0.000000,0.000000,0.000000}%
\pgfsetstrokecolor{currentstroke}%
\pgfsetdash{}{0pt}%
\pgfpathmoveto{\pgfqpoint{3.876477in}{0.356667in}}%
\pgfpathcurveto{\pgfqpoint{3.882002in}{0.356667in}}{\pgfqpoint{3.887302in}{0.358862in}}{\pgfqpoint{3.891209in}{0.362769in}}%
\pgfpathcurveto{\pgfqpoint{3.895115in}{0.366675in}}{\pgfqpoint{3.897311in}{0.371975in}}{\pgfqpoint{3.897311in}{0.377500in}}%
\pgfpathcurveto{\pgfqpoint{3.897311in}{0.383025in}}{\pgfqpoint{3.895115in}{0.388325in}}{\pgfqpoint{3.891209in}{0.392231in}}%
\pgfpathcurveto{\pgfqpoint{3.887302in}{0.396138in}}{\pgfqpoint{3.882002in}{0.398333in}}{\pgfqpoint{3.876477in}{0.398333in}}%
\pgfpathcurveto{\pgfqpoint{3.870952in}{0.398333in}}{\pgfqpoint{3.865653in}{0.396138in}}{\pgfqpoint{3.861746in}{0.392231in}}%
\pgfpathcurveto{\pgfqpoint{3.857839in}{0.388325in}}{\pgfqpoint{3.855644in}{0.383025in}}{\pgfqpoint{3.855644in}{0.377500in}}%
\pgfpathcurveto{\pgfqpoint{3.855644in}{0.371975in}}{\pgfqpoint{3.857839in}{0.366675in}}{\pgfqpoint{3.861746in}{0.362769in}}%
\pgfpathcurveto{\pgfqpoint{3.865653in}{0.358862in}}{\pgfqpoint{3.870952in}{0.356667in}}{\pgfqpoint{3.876477in}{0.356667in}}%
\pgfpathclose%
\pgfusepath{stroke,fill}%
\end{pgfscope}%
\begin{pgfscope}%
\pgfpathrectangle{\pgfqpoint{0.562500in}{0.275000in}}{\pgfqpoint{3.487500in}{1.925000in}}%
\pgfusepath{clip}%
\pgfsetbuttcap%
\pgfsetroundjoin%
\definecolor{currentfill}{rgb}{0.000000,0.000000,0.000000}%
\pgfsetfillcolor{currentfill}%
\pgfsetlinewidth{1.003750pt}%
\definecolor{currentstroke}{rgb}{0.000000,0.000000,0.000000}%
\pgfsetstrokecolor{currentstroke}%
\pgfsetdash{}{0pt}%
\pgfpathmoveto{\pgfqpoint{3.876477in}{0.356667in}}%
\pgfpathcurveto{\pgfqpoint{3.882002in}{0.356667in}}{\pgfqpoint{3.887302in}{0.358862in}}{\pgfqpoint{3.891209in}{0.362769in}}%
\pgfpathcurveto{\pgfqpoint{3.895115in}{0.366675in}}{\pgfqpoint{3.897311in}{0.371975in}}{\pgfqpoint{3.897311in}{0.377500in}}%
\pgfpathcurveto{\pgfqpoint{3.897311in}{0.383025in}}{\pgfqpoint{3.895115in}{0.388325in}}{\pgfqpoint{3.891209in}{0.392231in}}%
\pgfpathcurveto{\pgfqpoint{3.887302in}{0.396138in}}{\pgfqpoint{3.882002in}{0.398333in}}{\pgfqpoint{3.876477in}{0.398333in}}%
\pgfpathcurveto{\pgfqpoint{3.870952in}{0.398333in}}{\pgfqpoint{3.865653in}{0.396138in}}{\pgfqpoint{3.861746in}{0.392231in}}%
\pgfpathcurveto{\pgfqpoint{3.857839in}{0.388325in}}{\pgfqpoint{3.855644in}{0.383025in}}{\pgfqpoint{3.855644in}{0.377500in}}%
\pgfpathcurveto{\pgfqpoint{3.855644in}{0.371975in}}{\pgfqpoint{3.857839in}{0.366675in}}{\pgfqpoint{3.861746in}{0.362769in}}%
\pgfpathcurveto{\pgfqpoint{3.865653in}{0.358862in}}{\pgfqpoint{3.870952in}{0.356667in}}{\pgfqpoint{3.876477in}{0.356667in}}%
\pgfpathclose%
\pgfusepath{stroke,fill}%
\end{pgfscope}%
\begin{pgfscope}%
\pgfpathrectangle{\pgfqpoint{0.562500in}{0.275000in}}{\pgfqpoint{3.487500in}{1.925000in}}%
\pgfusepath{clip}%
\pgfsetbuttcap%
\pgfsetroundjoin%
\definecolor{currentfill}{rgb}{0.000000,0.000000,0.000000}%
\pgfsetfillcolor{currentfill}%
\pgfsetlinewidth{1.003750pt}%
\definecolor{currentstroke}{rgb}{0.000000,0.000000,0.000000}%
\pgfsetstrokecolor{currentstroke}%
\pgfsetdash{}{0pt}%
\pgfpathmoveto{\pgfqpoint{3.876477in}{0.356667in}}%
\pgfpathcurveto{\pgfqpoint{3.882002in}{0.356667in}}{\pgfqpoint{3.887302in}{0.358862in}}{\pgfqpoint{3.891209in}{0.362769in}}%
\pgfpathcurveto{\pgfqpoint{3.895115in}{0.366675in}}{\pgfqpoint{3.897311in}{0.371975in}}{\pgfqpoint{3.897311in}{0.377500in}}%
\pgfpathcurveto{\pgfqpoint{3.897311in}{0.383025in}}{\pgfqpoint{3.895115in}{0.388325in}}{\pgfqpoint{3.891209in}{0.392231in}}%
\pgfpathcurveto{\pgfqpoint{3.887302in}{0.396138in}}{\pgfqpoint{3.882002in}{0.398333in}}{\pgfqpoint{3.876477in}{0.398333in}}%
\pgfpathcurveto{\pgfqpoint{3.870952in}{0.398333in}}{\pgfqpoint{3.865653in}{0.396138in}}{\pgfqpoint{3.861746in}{0.392231in}}%
\pgfpathcurveto{\pgfqpoint{3.857839in}{0.388325in}}{\pgfqpoint{3.855644in}{0.383025in}}{\pgfqpoint{3.855644in}{0.377500in}}%
\pgfpathcurveto{\pgfqpoint{3.855644in}{0.371975in}}{\pgfqpoint{3.857839in}{0.366675in}}{\pgfqpoint{3.861746in}{0.362769in}}%
\pgfpathcurveto{\pgfqpoint{3.865653in}{0.358862in}}{\pgfqpoint{3.870952in}{0.356667in}}{\pgfqpoint{3.876477in}{0.356667in}}%
\pgfpathclose%
\pgfusepath{stroke,fill}%
\end{pgfscope}%
\begin{pgfscope}%
\pgfpathrectangle{\pgfqpoint{0.562500in}{0.275000in}}{\pgfqpoint{3.487500in}{1.925000in}}%
\pgfusepath{clip}%
\pgfsetbuttcap%
\pgfsetroundjoin%
\definecolor{currentfill}{rgb}{0.000000,0.000000,0.000000}%
\pgfsetfillcolor{currentfill}%
\pgfsetlinewidth{1.003750pt}%
\definecolor{currentstroke}{rgb}{0.000000,0.000000,0.000000}%
\pgfsetstrokecolor{currentstroke}%
\pgfsetdash{}{0pt}%
\pgfpathmoveto{\pgfqpoint{3.876477in}{0.356667in}}%
\pgfpathcurveto{\pgfqpoint{3.882002in}{0.356667in}}{\pgfqpoint{3.887302in}{0.358862in}}{\pgfqpoint{3.891209in}{0.362769in}}%
\pgfpathcurveto{\pgfqpoint{3.895115in}{0.366675in}}{\pgfqpoint{3.897311in}{0.371975in}}{\pgfqpoint{3.897311in}{0.377500in}}%
\pgfpathcurveto{\pgfqpoint{3.897311in}{0.383025in}}{\pgfqpoint{3.895115in}{0.388325in}}{\pgfqpoint{3.891209in}{0.392231in}}%
\pgfpathcurveto{\pgfqpoint{3.887302in}{0.396138in}}{\pgfqpoint{3.882002in}{0.398333in}}{\pgfqpoint{3.876477in}{0.398333in}}%
\pgfpathcurveto{\pgfqpoint{3.870952in}{0.398333in}}{\pgfqpoint{3.865653in}{0.396138in}}{\pgfqpoint{3.861746in}{0.392231in}}%
\pgfpathcurveto{\pgfqpoint{3.857839in}{0.388325in}}{\pgfqpoint{3.855644in}{0.383025in}}{\pgfqpoint{3.855644in}{0.377500in}}%
\pgfpathcurveto{\pgfqpoint{3.855644in}{0.371975in}}{\pgfqpoint{3.857839in}{0.366675in}}{\pgfqpoint{3.861746in}{0.362769in}}%
\pgfpathcurveto{\pgfqpoint{3.865653in}{0.358862in}}{\pgfqpoint{3.870952in}{0.356667in}}{\pgfqpoint{3.876477in}{0.356667in}}%
\pgfpathclose%
\pgfusepath{stroke,fill}%
\end{pgfscope}%
\begin{pgfscope}%
\pgfpathrectangle{\pgfqpoint{0.562500in}{0.275000in}}{\pgfqpoint{3.487500in}{1.925000in}}%
\pgfusepath{clip}%
\pgfsetbuttcap%
\pgfsetroundjoin%
\definecolor{currentfill}{rgb}{0.000000,0.000000,0.000000}%
\pgfsetfillcolor{currentfill}%
\pgfsetlinewidth{1.003750pt}%
\definecolor{currentstroke}{rgb}{0.000000,0.000000,0.000000}%
\pgfsetstrokecolor{currentstroke}%
\pgfsetdash{}{0pt}%
\pgfpathmoveto{\pgfqpoint{3.876477in}{0.356667in}}%
\pgfpathcurveto{\pgfqpoint{3.882002in}{0.356667in}}{\pgfqpoint{3.887302in}{0.358862in}}{\pgfqpoint{3.891209in}{0.362769in}}%
\pgfpathcurveto{\pgfqpoint{3.895115in}{0.366675in}}{\pgfqpoint{3.897311in}{0.371975in}}{\pgfqpoint{3.897311in}{0.377500in}}%
\pgfpathcurveto{\pgfqpoint{3.897311in}{0.383025in}}{\pgfqpoint{3.895115in}{0.388325in}}{\pgfqpoint{3.891209in}{0.392231in}}%
\pgfpathcurveto{\pgfqpoint{3.887302in}{0.396138in}}{\pgfqpoint{3.882002in}{0.398333in}}{\pgfqpoint{3.876477in}{0.398333in}}%
\pgfpathcurveto{\pgfqpoint{3.870952in}{0.398333in}}{\pgfqpoint{3.865653in}{0.396138in}}{\pgfqpoint{3.861746in}{0.392231in}}%
\pgfpathcurveto{\pgfqpoint{3.857839in}{0.388325in}}{\pgfqpoint{3.855644in}{0.383025in}}{\pgfqpoint{3.855644in}{0.377500in}}%
\pgfpathcurveto{\pgfqpoint{3.855644in}{0.371975in}}{\pgfqpoint{3.857839in}{0.366675in}}{\pgfqpoint{3.861746in}{0.362769in}}%
\pgfpathcurveto{\pgfqpoint{3.865653in}{0.358862in}}{\pgfqpoint{3.870952in}{0.356667in}}{\pgfqpoint{3.876477in}{0.356667in}}%
\pgfpathclose%
\pgfusepath{stroke,fill}%
\end{pgfscope}%
\begin{pgfscope}%
\pgfpathrectangle{\pgfqpoint{0.562500in}{0.275000in}}{\pgfqpoint{3.487500in}{1.925000in}}%
\pgfusepath{clip}%
\pgfsetbuttcap%
\pgfsetroundjoin%
\definecolor{currentfill}{rgb}{0.000000,0.000000,0.000000}%
\pgfsetfillcolor{currentfill}%
\pgfsetlinewidth{1.003750pt}%
\definecolor{currentstroke}{rgb}{0.000000,0.000000,0.000000}%
\pgfsetstrokecolor{currentstroke}%
\pgfsetdash{}{0pt}%
\pgfpathmoveto{\pgfqpoint{3.876477in}{0.356667in}}%
\pgfpathcurveto{\pgfqpoint{3.882002in}{0.356667in}}{\pgfqpoint{3.887302in}{0.358862in}}{\pgfqpoint{3.891209in}{0.362769in}}%
\pgfpathcurveto{\pgfqpoint{3.895115in}{0.366675in}}{\pgfqpoint{3.897311in}{0.371975in}}{\pgfqpoint{3.897311in}{0.377500in}}%
\pgfpathcurveto{\pgfqpoint{3.897311in}{0.383025in}}{\pgfqpoint{3.895115in}{0.388325in}}{\pgfqpoint{3.891209in}{0.392231in}}%
\pgfpathcurveto{\pgfqpoint{3.887302in}{0.396138in}}{\pgfqpoint{3.882002in}{0.398333in}}{\pgfqpoint{3.876477in}{0.398333in}}%
\pgfpathcurveto{\pgfqpoint{3.870952in}{0.398333in}}{\pgfqpoint{3.865653in}{0.396138in}}{\pgfqpoint{3.861746in}{0.392231in}}%
\pgfpathcurveto{\pgfqpoint{3.857839in}{0.388325in}}{\pgfqpoint{3.855644in}{0.383025in}}{\pgfqpoint{3.855644in}{0.377500in}}%
\pgfpathcurveto{\pgfqpoint{3.855644in}{0.371975in}}{\pgfqpoint{3.857839in}{0.366675in}}{\pgfqpoint{3.861746in}{0.362769in}}%
\pgfpathcurveto{\pgfqpoint{3.865653in}{0.358862in}}{\pgfqpoint{3.870952in}{0.356667in}}{\pgfqpoint{3.876477in}{0.356667in}}%
\pgfpathclose%
\pgfusepath{stroke,fill}%
\end{pgfscope}%
\begin{pgfscope}%
\pgfpathrectangle{\pgfqpoint{0.562500in}{0.275000in}}{\pgfqpoint{3.487500in}{1.925000in}}%
\pgfusepath{clip}%
\pgfsetbuttcap%
\pgfsetroundjoin%
\definecolor{currentfill}{rgb}{0.000000,0.000000,0.000000}%
\pgfsetfillcolor{currentfill}%
\pgfsetlinewidth{1.003750pt}%
\definecolor{currentstroke}{rgb}{0.000000,0.000000,0.000000}%
\pgfsetstrokecolor{currentstroke}%
\pgfsetdash{}{0pt}%
\pgfpathmoveto{\pgfqpoint{3.876477in}{0.356667in}}%
\pgfpathcurveto{\pgfqpoint{3.882002in}{0.356667in}}{\pgfqpoint{3.887302in}{0.358862in}}{\pgfqpoint{3.891209in}{0.362769in}}%
\pgfpathcurveto{\pgfqpoint{3.895115in}{0.366675in}}{\pgfqpoint{3.897311in}{0.371975in}}{\pgfqpoint{3.897311in}{0.377500in}}%
\pgfpathcurveto{\pgfqpoint{3.897311in}{0.383025in}}{\pgfqpoint{3.895115in}{0.388325in}}{\pgfqpoint{3.891209in}{0.392231in}}%
\pgfpathcurveto{\pgfqpoint{3.887302in}{0.396138in}}{\pgfqpoint{3.882002in}{0.398333in}}{\pgfqpoint{3.876477in}{0.398333in}}%
\pgfpathcurveto{\pgfqpoint{3.870952in}{0.398333in}}{\pgfqpoint{3.865653in}{0.396138in}}{\pgfqpoint{3.861746in}{0.392231in}}%
\pgfpathcurveto{\pgfqpoint{3.857839in}{0.388325in}}{\pgfqpoint{3.855644in}{0.383025in}}{\pgfqpoint{3.855644in}{0.377500in}}%
\pgfpathcurveto{\pgfqpoint{3.855644in}{0.371975in}}{\pgfqpoint{3.857839in}{0.366675in}}{\pgfqpoint{3.861746in}{0.362769in}}%
\pgfpathcurveto{\pgfqpoint{3.865653in}{0.358862in}}{\pgfqpoint{3.870952in}{0.356667in}}{\pgfqpoint{3.876477in}{0.356667in}}%
\pgfpathclose%
\pgfusepath{stroke,fill}%
\end{pgfscope}%
\begin{pgfscope}%
\pgfpathrectangle{\pgfqpoint{0.562500in}{0.275000in}}{\pgfqpoint{3.487500in}{1.925000in}}%
\pgfusepath{clip}%
\pgfsetbuttcap%
\pgfsetroundjoin%
\definecolor{currentfill}{rgb}{0.000000,0.000000,0.000000}%
\pgfsetfillcolor{currentfill}%
\pgfsetlinewidth{1.003750pt}%
\definecolor{currentstroke}{rgb}{0.000000,0.000000,0.000000}%
\pgfsetstrokecolor{currentstroke}%
\pgfsetdash{}{0pt}%
\pgfpathmoveto{\pgfqpoint{3.876477in}{0.356667in}}%
\pgfpathcurveto{\pgfqpoint{3.882002in}{0.356667in}}{\pgfqpoint{3.887302in}{0.358862in}}{\pgfqpoint{3.891209in}{0.362769in}}%
\pgfpathcurveto{\pgfqpoint{3.895115in}{0.366675in}}{\pgfqpoint{3.897311in}{0.371975in}}{\pgfqpoint{3.897311in}{0.377500in}}%
\pgfpathcurveto{\pgfqpoint{3.897311in}{0.383025in}}{\pgfqpoint{3.895115in}{0.388325in}}{\pgfqpoint{3.891209in}{0.392231in}}%
\pgfpathcurveto{\pgfqpoint{3.887302in}{0.396138in}}{\pgfqpoint{3.882002in}{0.398333in}}{\pgfqpoint{3.876477in}{0.398333in}}%
\pgfpathcurveto{\pgfqpoint{3.870952in}{0.398333in}}{\pgfqpoint{3.865653in}{0.396138in}}{\pgfqpoint{3.861746in}{0.392231in}}%
\pgfpathcurveto{\pgfqpoint{3.857839in}{0.388325in}}{\pgfqpoint{3.855644in}{0.383025in}}{\pgfqpoint{3.855644in}{0.377500in}}%
\pgfpathcurveto{\pgfqpoint{3.855644in}{0.371975in}}{\pgfqpoint{3.857839in}{0.366675in}}{\pgfqpoint{3.861746in}{0.362769in}}%
\pgfpathcurveto{\pgfqpoint{3.865653in}{0.358862in}}{\pgfqpoint{3.870952in}{0.356667in}}{\pgfqpoint{3.876477in}{0.356667in}}%
\pgfpathclose%
\pgfusepath{stroke,fill}%
\end{pgfscope}%
\begin{pgfscope}%
\pgfpathrectangle{\pgfqpoint{0.562500in}{0.275000in}}{\pgfqpoint{3.487500in}{1.925000in}}%
\pgfusepath{clip}%
\pgfsetbuttcap%
\pgfsetroundjoin%
\definecolor{currentfill}{rgb}{0.000000,0.000000,0.000000}%
\pgfsetfillcolor{currentfill}%
\pgfsetlinewidth{1.003750pt}%
\definecolor{currentstroke}{rgb}{0.000000,0.000000,0.000000}%
\pgfsetstrokecolor{currentstroke}%
\pgfsetdash{}{0pt}%
\pgfpathmoveto{\pgfqpoint{3.876477in}{0.356667in}}%
\pgfpathcurveto{\pgfqpoint{3.882002in}{0.356667in}}{\pgfqpoint{3.887302in}{0.358862in}}{\pgfqpoint{3.891209in}{0.362769in}}%
\pgfpathcurveto{\pgfqpoint{3.895115in}{0.366675in}}{\pgfqpoint{3.897311in}{0.371975in}}{\pgfqpoint{3.897311in}{0.377500in}}%
\pgfpathcurveto{\pgfqpoint{3.897311in}{0.383025in}}{\pgfqpoint{3.895115in}{0.388325in}}{\pgfqpoint{3.891209in}{0.392231in}}%
\pgfpathcurveto{\pgfqpoint{3.887302in}{0.396138in}}{\pgfqpoint{3.882002in}{0.398333in}}{\pgfqpoint{3.876477in}{0.398333in}}%
\pgfpathcurveto{\pgfqpoint{3.870952in}{0.398333in}}{\pgfqpoint{3.865653in}{0.396138in}}{\pgfqpoint{3.861746in}{0.392231in}}%
\pgfpathcurveto{\pgfqpoint{3.857839in}{0.388325in}}{\pgfqpoint{3.855644in}{0.383025in}}{\pgfqpoint{3.855644in}{0.377500in}}%
\pgfpathcurveto{\pgfqpoint{3.855644in}{0.371975in}}{\pgfqpoint{3.857839in}{0.366675in}}{\pgfqpoint{3.861746in}{0.362769in}}%
\pgfpathcurveto{\pgfqpoint{3.865653in}{0.358862in}}{\pgfqpoint{3.870952in}{0.356667in}}{\pgfqpoint{3.876477in}{0.356667in}}%
\pgfpathclose%
\pgfusepath{stroke,fill}%
\end{pgfscope}%
\begin{pgfscope}%
\pgfpathrectangle{\pgfqpoint{0.562500in}{0.275000in}}{\pgfqpoint{3.487500in}{1.925000in}}%
\pgfusepath{clip}%
\pgfsetbuttcap%
\pgfsetroundjoin%
\definecolor{currentfill}{rgb}{0.000000,0.000000,0.000000}%
\pgfsetfillcolor{currentfill}%
\pgfsetlinewidth{1.003750pt}%
\definecolor{currentstroke}{rgb}{0.000000,0.000000,0.000000}%
\pgfsetstrokecolor{currentstroke}%
\pgfsetdash{}{0pt}%
\pgfpathmoveto{\pgfqpoint{3.876477in}{0.356667in}}%
\pgfpathcurveto{\pgfqpoint{3.882002in}{0.356667in}}{\pgfqpoint{3.887302in}{0.358862in}}{\pgfqpoint{3.891209in}{0.362769in}}%
\pgfpathcurveto{\pgfqpoint{3.895115in}{0.366675in}}{\pgfqpoint{3.897311in}{0.371975in}}{\pgfqpoint{3.897311in}{0.377500in}}%
\pgfpathcurveto{\pgfqpoint{3.897311in}{0.383025in}}{\pgfqpoint{3.895115in}{0.388325in}}{\pgfqpoint{3.891209in}{0.392231in}}%
\pgfpathcurveto{\pgfqpoint{3.887302in}{0.396138in}}{\pgfqpoint{3.882002in}{0.398333in}}{\pgfqpoint{3.876477in}{0.398333in}}%
\pgfpathcurveto{\pgfqpoint{3.870952in}{0.398333in}}{\pgfqpoint{3.865653in}{0.396138in}}{\pgfqpoint{3.861746in}{0.392231in}}%
\pgfpathcurveto{\pgfqpoint{3.857839in}{0.388325in}}{\pgfqpoint{3.855644in}{0.383025in}}{\pgfqpoint{3.855644in}{0.377500in}}%
\pgfpathcurveto{\pgfqpoint{3.855644in}{0.371975in}}{\pgfqpoint{3.857839in}{0.366675in}}{\pgfqpoint{3.861746in}{0.362769in}}%
\pgfpathcurveto{\pgfqpoint{3.865653in}{0.358862in}}{\pgfqpoint{3.870952in}{0.356667in}}{\pgfqpoint{3.876477in}{0.356667in}}%
\pgfpathclose%
\pgfusepath{stroke,fill}%
\end{pgfscope}%
\begin{pgfscope}%
\pgfpathrectangle{\pgfqpoint{0.562500in}{0.275000in}}{\pgfqpoint{3.487500in}{1.925000in}}%
\pgfusepath{clip}%
\pgfsetbuttcap%
\pgfsetroundjoin%
\definecolor{currentfill}{rgb}{0.000000,0.000000,0.000000}%
\pgfsetfillcolor{currentfill}%
\pgfsetlinewidth{1.003750pt}%
\definecolor{currentstroke}{rgb}{0.000000,0.000000,0.000000}%
\pgfsetstrokecolor{currentstroke}%
\pgfsetdash{}{0pt}%
\pgfpathmoveto{\pgfqpoint{3.876477in}{0.356667in}}%
\pgfpathcurveto{\pgfqpoint{3.882002in}{0.356667in}}{\pgfqpoint{3.887302in}{0.358862in}}{\pgfqpoint{3.891209in}{0.362769in}}%
\pgfpathcurveto{\pgfqpoint{3.895115in}{0.366675in}}{\pgfqpoint{3.897311in}{0.371975in}}{\pgfqpoint{3.897311in}{0.377500in}}%
\pgfpathcurveto{\pgfqpoint{3.897311in}{0.383025in}}{\pgfqpoint{3.895115in}{0.388325in}}{\pgfqpoint{3.891209in}{0.392231in}}%
\pgfpathcurveto{\pgfqpoint{3.887302in}{0.396138in}}{\pgfqpoint{3.882002in}{0.398333in}}{\pgfqpoint{3.876477in}{0.398333in}}%
\pgfpathcurveto{\pgfqpoint{3.870952in}{0.398333in}}{\pgfqpoint{3.865653in}{0.396138in}}{\pgfqpoint{3.861746in}{0.392231in}}%
\pgfpathcurveto{\pgfqpoint{3.857839in}{0.388325in}}{\pgfqpoint{3.855644in}{0.383025in}}{\pgfqpoint{3.855644in}{0.377500in}}%
\pgfpathcurveto{\pgfqpoint{3.855644in}{0.371975in}}{\pgfqpoint{3.857839in}{0.366675in}}{\pgfqpoint{3.861746in}{0.362769in}}%
\pgfpathcurveto{\pgfqpoint{3.865653in}{0.358862in}}{\pgfqpoint{3.870952in}{0.356667in}}{\pgfqpoint{3.876477in}{0.356667in}}%
\pgfpathclose%
\pgfusepath{stroke,fill}%
\end{pgfscope}%
\begin{pgfscope}%
\pgfpathrectangle{\pgfqpoint{0.562500in}{0.275000in}}{\pgfqpoint{3.487500in}{1.925000in}}%
\pgfusepath{clip}%
\pgfsetbuttcap%
\pgfsetroundjoin%
\definecolor{currentfill}{rgb}{0.000000,0.000000,0.000000}%
\pgfsetfillcolor{currentfill}%
\pgfsetlinewidth{1.003750pt}%
\definecolor{currentstroke}{rgb}{0.000000,0.000000,0.000000}%
\pgfsetstrokecolor{currentstroke}%
\pgfsetdash{}{0pt}%
\pgfpathmoveto{\pgfqpoint{3.876477in}{0.356667in}}%
\pgfpathcurveto{\pgfqpoint{3.882002in}{0.356667in}}{\pgfqpoint{3.887302in}{0.358862in}}{\pgfqpoint{3.891209in}{0.362769in}}%
\pgfpathcurveto{\pgfqpoint{3.895115in}{0.366675in}}{\pgfqpoint{3.897311in}{0.371975in}}{\pgfqpoint{3.897311in}{0.377500in}}%
\pgfpathcurveto{\pgfqpoint{3.897311in}{0.383025in}}{\pgfqpoint{3.895115in}{0.388325in}}{\pgfqpoint{3.891209in}{0.392231in}}%
\pgfpathcurveto{\pgfqpoint{3.887302in}{0.396138in}}{\pgfqpoint{3.882002in}{0.398333in}}{\pgfqpoint{3.876477in}{0.398333in}}%
\pgfpathcurveto{\pgfqpoint{3.870952in}{0.398333in}}{\pgfqpoint{3.865653in}{0.396138in}}{\pgfqpoint{3.861746in}{0.392231in}}%
\pgfpathcurveto{\pgfqpoint{3.857839in}{0.388325in}}{\pgfqpoint{3.855644in}{0.383025in}}{\pgfqpoint{3.855644in}{0.377500in}}%
\pgfpathcurveto{\pgfqpoint{3.855644in}{0.371975in}}{\pgfqpoint{3.857839in}{0.366675in}}{\pgfqpoint{3.861746in}{0.362769in}}%
\pgfpathcurveto{\pgfqpoint{3.865653in}{0.358862in}}{\pgfqpoint{3.870952in}{0.356667in}}{\pgfqpoint{3.876477in}{0.356667in}}%
\pgfpathclose%
\pgfusepath{stroke,fill}%
\end{pgfscope}%
\begin{pgfscope}%
\pgfpathrectangle{\pgfqpoint{0.562500in}{0.275000in}}{\pgfqpoint{3.487500in}{1.925000in}}%
\pgfusepath{clip}%
\pgfsetbuttcap%
\pgfsetroundjoin%
\definecolor{currentfill}{rgb}{0.000000,0.000000,0.000000}%
\pgfsetfillcolor{currentfill}%
\pgfsetlinewidth{1.003750pt}%
\definecolor{currentstroke}{rgb}{0.000000,0.000000,0.000000}%
\pgfsetstrokecolor{currentstroke}%
\pgfsetdash{}{0pt}%
\pgfpathmoveto{\pgfqpoint{3.876477in}{0.356667in}}%
\pgfpathcurveto{\pgfqpoint{3.882002in}{0.356667in}}{\pgfqpoint{3.887302in}{0.358862in}}{\pgfqpoint{3.891209in}{0.362769in}}%
\pgfpathcurveto{\pgfqpoint{3.895115in}{0.366675in}}{\pgfqpoint{3.897311in}{0.371975in}}{\pgfqpoint{3.897311in}{0.377500in}}%
\pgfpathcurveto{\pgfqpoint{3.897311in}{0.383025in}}{\pgfqpoint{3.895115in}{0.388325in}}{\pgfqpoint{3.891209in}{0.392231in}}%
\pgfpathcurveto{\pgfqpoint{3.887302in}{0.396138in}}{\pgfqpoint{3.882002in}{0.398333in}}{\pgfqpoint{3.876477in}{0.398333in}}%
\pgfpathcurveto{\pgfqpoint{3.870952in}{0.398333in}}{\pgfqpoint{3.865653in}{0.396138in}}{\pgfqpoint{3.861746in}{0.392231in}}%
\pgfpathcurveto{\pgfqpoint{3.857839in}{0.388325in}}{\pgfqpoint{3.855644in}{0.383025in}}{\pgfqpoint{3.855644in}{0.377500in}}%
\pgfpathcurveto{\pgfqpoint{3.855644in}{0.371975in}}{\pgfqpoint{3.857839in}{0.366675in}}{\pgfqpoint{3.861746in}{0.362769in}}%
\pgfpathcurveto{\pgfqpoint{3.865653in}{0.358862in}}{\pgfqpoint{3.870952in}{0.356667in}}{\pgfqpoint{3.876477in}{0.356667in}}%
\pgfpathclose%
\pgfusepath{stroke,fill}%
\end{pgfscope}%
\begin{pgfscope}%
\pgfpathrectangle{\pgfqpoint{0.562500in}{0.275000in}}{\pgfqpoint{3.487500in}{1.925000in}}%
\pgfusepath{clip}%
\pgfsetbuttcap%
\pgfsetroundjoin%
\definecolor{currentfill}{rgb}{0.000000,0.000000,0.000000}%
\pgfsetfillcolor{currentfill}%
\pgfsetlinewidth{1.003750pt}%
\definecolor{currentstroke}{rgb}{0.000000,0.000000,0.000000}%
\pgfsetstrokecolor{currentstroke}%
\pgfsetdash{}{0pt}%
\pgfpathmoveto{\pgfqpoint{3.876477in}{0.356667in}}%
\pgfpathcurveto{\pgfqpoint{3.882002in}{0.356667in}}{\pgfqpoint{3.887302in}{0.358862in}}{\pgfqpoint{3.891209in}{0.362769in}}%
\pgfpathcurveto{\pgfqpoint{3.895115in}{0.366675in}}{\pgfqpoint{3.897311in}{0.371975in}}{\pgfqpoint{3.897311in}{0.377500in}}%
\pgfpathcurveto{\pgfqpoint{3.897311in}{0.383025in}}{\pgfqpoint{3.895115in}{0.388325in}}{\pgfqpoint{3.891209in}{0.392231in}}%
\pgfpathcurveto{\pgfqpoint{3.887302in}{0.396138in}}{\pgfqpoint{3.882002in}{0.398333in}}{\pgfqpoint{3.876477in}{0.398333in}}%
\pgfpathcurveto{\pgfqpoint{3.870952in}{0.398333in}}{\pgfqpoint{3.865653in}{0.396138in}}{\pgfqpoint{3.861746in}{0.392231in}}%
\pgfpathcurveto{\pgfqpoint{3.857839in}{0.388325in}}{\pgfqpoint{3.855644in}{0.383025in}}{\pgfqpoint{3.855644in}{0.377500in}}%
\pgfpathcurveto{\pgfqpoint{3.855644in}{0.371975in}}{\pgfqpoint{3.857839in}{0.366675in}}{\pgfqpoint{3.861746in}{0.362769in}}%
\pgfpathcurveto{\pgfqpoint{3.865653in}{0.358862in}}{\pgfqpoint{3.870952in}{0.356667in}}{\pgfqpoint{3.876477in}{0.356667in}}%
\pgfpathclose%
\pgfusepath{stroke,fill}%
\end{pgfscope}%
\begin{pgfscope}%
\pgfpathrectangle{\pgfqpoint{0.562500in}{0.275000in}}{\pgfqpoint{3.487500in}{1.925000in}}%
\pgfusepath{clip}%
\pgfsetbuttcap%
\pgfsetroundjoin%
\definecolor{currentfill}{rgb}{0.000000,0.000000,0.000000}%
\pgfsetfillcolor{currentfill}%
\pgfsetlinewidth{1.003750pt}%
\definecolor{currentstroke}{rgb}{0.000000,0.000000,0.000000}%
\pgfsetstrokecolor{currentstroke}%
\pgfsetdash{}{0pt}%
\pgfpathmoveto{\pgfqpoint{3.876477in}{0.356667in}}%
\pgfpathcurveto{\pgfqpoint{3.882002in}{0.356667in}}{\pgfqpoint{3.887302in}{0.358862in}}{\pgfqpoint{3.891209in}{0.362769in}}%
\pgfpathcurveto{\pgfqpoint{3.895115in}{0.366675in}}{\pgfqpoint{3.897311in}{0.371975in}}{\pgfqpoint{3.897311in}{0.377500in}}%
\pgfpathcurveto{\pgfqpoint{3.897311in}{0.383025in}}{\pgfqpoint{3.895115in}{0.388325in}}{\pgfqpoint{3.891209in}{0.392231in}}%
\pgfpathcurveto{\pgfqpoint{3.887302in}{0.396138in}}{\pgfqpoint{3.882002in}{0.398333in}}{\pgfqpoint{3.876477in}{0.398333in}}%
\pgfpathcurveto{\pgfqpoint{3.870952in}{0.398333in}}{\pgfqpoint{3.865653in}{0.396138in}}{\pgfqpoint{3.861746in}{0.392231in}}%
\pgfpathcurveto{\pgfqpoint{3.857839in}{0.388325in}}{\pgfqpoint{3.855644in}{0.383025in}}{\pgfqpoint{3.855644in}{0.377500in}}%
\pgfpathcurveto{\pgfqpoint{3.855644in}{0.371975in}}{\pgfqpoint{3.857839in}{0.366675in}}{\pgfqpoint{3.861746in}{0.362769in}}%
\pgfpathcurveto{\pgfqpoint{3.865653in}{0.358862in}}{\pgfqpoint{3.870952in}{0.356667in}}{\pgfqpoint{3.876477in}{0.356667in}}%
\pgfpathclose%
\pgfusepath{stroke,fill}%
\end{pgfscope}%
\begin{pgfscope}%
\pgfpathrectangle{\pgfqpoint{0.562500in}{0.275000in}}{\pgfqpoint{3.487500in}{1.925000in}}%
\pgfusepath{clip}%
\pgfsetbuttcap%
\pgfsetroundjoin%
\definecolor{currentfill}{rgb}{0.000000,0.000000,0.000000}%
\pgfsetfillcolor{currentfill}%
\pgfsetlinewidth{1.003750pt}%
\definecolor{currentstroke}{rgb}{0.000000,0.000000,0.000000}%
\pgfsetstrokecolor{currentstroke}%
\pgfsetdash{}{0pt}%
\pgfpathmoveto{\pgfqpoint{3.876477in}{0.356667in}}%
\pgfpathcurveto{\pgfqpoint{3.882002in}{0.356667in}}{\pgfqpoint{3.887302in}{0.358862in}}{\pgfqpoint{3.891209in}{0.362769in}}%
\pgfpathcurveto{\pgfqpoint{3.895115in}{0.366675in}}{\pgfqpoint{3.897311in}{0.371975in}}{\pgfqpoint{3.897311in}{0.377500in}}%
\pgfpathcurveto{\pgfqpoint{3.897311in}{0.383025in}}{\pgfqpoint{3.895115in}{0.388325in}}{\pgfqpoint{3.891209in}{0.392231in}}%
\pgfpathcurveto{\pgfqpoint{3.887302in}{0.396138in}}{\pgfqpoint{3.882002in}{0.398333in}}{\pgfqpoint{3.876477in}{0.398333in}}%
\pgfpathcurveto{\pgfqpoint{3.870952in}{0.398333in}}{\pgfqpoint{3.865653in}{0.396138in}}{\pgfqpoint{3.861746in}{0.392231in}}%
\pgfpathcurveto{\pgfqpoint{3.857839in}{0.388325in}}{\pgfqpoint{3.855644in}{0.383025in}}{\pgfqpoint{3.855644in}{0.377500in}}%
\pgfpathcurveto{\pgfqpoint{3.855644in}{0.371975in}}{\pgfqpoint{3.857839in}{0.366675in}}{\pgfqpoint{3.861746in}{0.362769in}}%
\pgfpathcurveto{\pgfqpoint{3.865653in}{0.358862in}}{\pgfqpoint{3.870952in}{0.356667in}}{\pgfqpoint{3.876477in}{0.356667in}}%
\pgfpathclose%
\pgfusepath{stroke,fill}%
\end{pgfscope}%
\begin{pgfscope}%
\pgfpathrectangle{\pgfqpoint{0.562500in}{0.275000in}}{\pgfqpoint{3.487500in}{1.925000in}}%
\pgfusepath{clip}%
\pgfsetbuttcap%
\pgfsetroundjoin%
\definecolor{currentfill}{rgb}{0.000000,0.000000,0.000000}%
\pgfsetfillcolor{currentfill}%
\pgfsetlinewidth{1.003750pt}%
\definecolor{currentstroke}{rgb}{0.000000,0.000000,0.000000}%
\pgfsetstrokecolor{currentstroke}%
\pgfsetdash{}{0pt}%
\pgfpathmoveto{\pgfqpoint{3.876477in}{0.356667in}}%
\pgfpathcurveto{\pgfqpoint{3.882002in}{0.356667in}}{\pgfqpoint{3.887302in}{0.358862in}}{\pgfqpoint{3.891209in}{0.362769in}}%
\pgfpathcurveto{\pgfqpoint{3.895115in}{0.366675in}}{\pgfqpoint{3.897311in}{0.371975in}}{\pgfqpoint{3.897311in}{0.377500in}}%
\pgfpathcurveto{\pgfqpoint{3.897311in}{0.383025in}}{\pgfqpoint{3.895115in}{0.388325in}}{\pgfqpoint{3.891209in}{0.392231in}}%
\pgfpathcurveto{\pgfqpoint{3.887302in}{0.396138in}}{\pgfqpoint{3.882002in}{0.398333in}}{\pgfqpoint{3.876477in}{0.398333in}}%
\pgfpathcurveto{\pgfqpoint{3.870952in}{0.398333in}}{\pgfqpoint{3.865653in}{0.396138in}}{\pgfqpoint{3.861746in}{0.392231in}}%
\pgfpathcurveto{\pgfqpoint{3.857839in}{0.388325in}}{\pgfqpoint{3.855644in}{0.383025in}}{\pgfqpoint{3.855644in}{0.377500in}}%
\pgfpathcurveto{\pgfqpoint{3.855644in}{0.371975in}}{\pgfqpoint{3.857839in}{0.366675in}}{\pgfqpoint{3.861746in}{0.362769in}}%
\pgfpathcurveto{\pgfqpoint{3.865653in}{0.358862in}}{\pgfqpoint{3.870952in}{0.356667in}}{\pgfqpoint{3.876477in}{0.356667in}}%
\pgfpathclose%
\pgfusepath{stroke,fill}%
\end{pgfscope}%
\begin{pgfscope}%
\pgfpathrectangle{\pgfqpoint{0.562500in}{0.275000in}}{\pgfqpoint{3.487500in}{1.925000in}}%
\pgfusepath{clip}%
\pgfsetbuttcap%
\pgfsetroundjoin%
\definecolor{currentfill}{rgb}{0.000000,0.000000,0.000000}%
\pgfsetfillcolor{currentfill}%
\pgfsetlinewidth{1.003750pt}%
\definecolor{currentstroke}{rgb}{0.000000,0.000000,0.000000}%
\pgfsetstrokecolor{currentstroke}%
\pgfsetdash{}{0pt}%
\pgfpathmoveto{\pgfqpoint{3.876477in}{0.356667in}}%
\pgfpathcurveto{\pgfqpoint{3.882002in}{0.356667in}}{\pgfqpoint{3.887302in}{0.358862in}}{\pgfqpoint{3.891209in}{0.362769in}}%
\pgfpathcurveto{\pgfqpoint{3.895115in}{0.366675in}}{\pgfqpoint{3.897311in}{0.371975in}}{\pgfqpoint{3.897311in}{0.377500in}}%
\pgfpathcurveto{\pgfqpoint{3.897311in}{0.383025in}}{\pgfqpoint{3.895115in}{0.388325in}}{\pgfqpoint{3.891209in}{0.392231in}}%
\pgfpathcurveto{\pgfqpoint{3.887302in}{0.396138in}}{\pgfqpoint{3.882002in}{0.398333in}}{\pgfqpoint{3.876477in}{0.398333in}}%
\pgfpathcurveto{\pgfqpoint{3.870952in}{0.398333in}}{\pgfqpoint{3.865653in}{0.396138in}}{\pgfqpoint{3.861746in}{0.392231in}}%
\pgfpathcurveto{\pgfqpoint{3.857839in}{0.388325in}}{\pgfqpoint{3.855644in}{0.383025in}}{\pgfqpoint{3.855644in}{0.377500in}}%
\pgfpathcurveto{\pgfqpoint{3.855644in}{0.371975in}}{\pgfqpoint{3.857839in}{0.366675in}}{\pgfqpoint{3.861746in}{0.362769in}}%
\pgfpathcurveto{\pgfqpoint{3.865653in}{0.358862in}}{\pgfqpoint{3.870952in}{0.356667in}}{\pgfqpoint{3.876477in}{0.356667in}}%
\pgfpathclose%
\pgfusepath{stroke,fill}%
\end{pgfscope}%
\begin{pgfscope}%
\pgfpathrectangle{\pgfqpoint{0.562500in}{0.275000in}}{\pgfqpoint{3.487500in}{1.925000in}}%
\pgfusepath{clip}%
\pgfsetbuttcap%
\pgfsetroundjoin%
\definecolor{currentfill}{rgb}{0.000000,0.000000,0.000000}%
\pgfsetfillcolor{currentfill}%
\pgfsetlinewidth{1.003750pt}%
\definecolor{currentstroke}{rgb}{0.000000,0.000000,0.000000}%
\pgfsetstrokecolor{currentstroke}%
\pgfsetdash{}{0pt}%
\pgfpathmoveto{\pgfqpoint{3.876477in}{0.356667in}}%
\pgfpathcurveto{\pgfqpoint{3.882002in}{0.356667in}}{\pgfqpoint{3.887302in}{0.358862in}}{\pgfqpoint{3.891209in}{0.362769in}}%
\pgfpathcurveto{\pgfqpoint{3.895115in}{0.366675in}}{\pgfqpoint{3.897311in}{0.371975in}}{\pgfqpoint{3.897311in}{0.377500in}}%
\pgfpathcurveto{\pgfqpoint{3.897311in}{0.383025in}}{\pgfqpoint{3.895115in}{0.388325in}}{\pgfqpoint{3.891209in}{0.392231in}}%
\pgfpathcurveto{\pgfqpoint{3.887302in}{0.396138in}}{\pgfqpoint{3.882002in}{0.398333in}}{\pgfqpoint{3.876477in}{0.398333in}}%
\pgfpathcurveto{\pgfqpoint{3.870952in}{0.398333in}}{\pgfqpoint{3.865653in}{0.396138in}}{\pgfqpoint{3.861746in}{0.392231in}}%
\pgfpathcurveto{\pgfqpoint{3.857839in}{0.388325in}}{\pgfqpoint{3.855644in}{0.383025in}}{\pgfqpoint{3.855644in}{0.377500in}}%
\pgfpathcurveto{\pgfqpoint{3.855644in}{0.371975in}}{\pgfqpoint{3.857839in}{0.366675in}}{\pgfqpoint{3.861746in}{0.362769in}}%
\pgfpathcurveto{\pgfqpoint{3.865653in}{0.358862in}}{\pgfqpoint{3.870952in}{0.356667in}}{\pgfqpoint{3.876477in}{0.356667in}}%
\pgfpathclose%
\pgfusepath{stroke,fill}%
\end{pgfscope}%
\begin{pgfscope}%
\pgfpathrectangle{\pgfqpoint{0.562500in}{0.275000in}}{\pgfqpoint{3.487500in}{1.925000in}}%
\pgfusepath{clip}%
\pgfsetbuttcap%
\pgfsetroundjoin%
\definecolor{currentfill}{rgb}{0.000000,0.000000,0.000000}%
\pgfsetfillcolor{currentfill}%
\pgfsetlinewidth{1.003750pt}%
\definecolor{currentstroke}{rgb}{0.000000,0.000000,0.000000}%
\pgfsetstrokecolor{currentstroke}%
\pgfsetdash{}{0pt}%
\pgfpathmoveto{\pgfqpoint{3.876477in}{0.356667in}}%
\pgfpathcurveto{\pgfqpoint{3.882002in}{0.356667in}}{\pgfqpoint{3.887302in}{0.358862in}}{\pgfqpoint{3.891209in}{0.362769in}}%
\pgfpathcurveto{\pgfqpoint{3.895115in}{0.366675in}}{\pgfqpoint{3.897311in}{0.371975in}}{\pgfqpoint{3.897311in}{0.377500in}}%
\pgfpathcurveto{\pgfqpoint{3.897311in}{0.383025in}}{\pgfqpoint{3.895115in}{0.388325in}}{\pgfqpoint{3.891209in}{0.392231in}}%
\pgfpathcurveto{\pgfqpoint{3.887302in}{0.396138in}}{\pgfqpoint{3.882002in}{0.398333in}}{\pgfqpoint{3.876477in}{0.398333in}}%
\pgfpathcurveto{\pgfqpoint{3.870952in}{0.398333in}}{\pgfqpoint{3.865653in}{0.396138in}}{\pgfqpoint{3.861746in}{0.392231in}}%
\pgfpathcurveto{\pgfqpoint{3.857839in}{0.388325in}}{\pgfqpoint{3.855644in}{0.383025in}}{\pgfqpoint{3.855644in}{0.377500in}}%
\pgfpathcurveto{\pgfqpoint{3.855644in}{0.371975in}}{\pgfqpoint{3.857839in}{0.366675in}}{\pgfqpoint{3.861746in}{0.362769in}}%
\pgfpathcurveto{\pgfqpoint{3.865653in}{0.358862in}}{\pgfqpoint{3.870952in}{0.356667in}}{\pgfqpoint{3.876477in}{0.356667in}}%
\pgfpathclose%
\pgfusepath{stroke,fill}%
\end{pgfscope}%
\begin{pgfscope}%
\pgfpathrectangle{\pgfqpoint{0.562500in}{0.275000in}}{\pgfqpoint{3.487500in}{1.925000in}}%
\pgfusepath{clip}%
\pgfsetbuttcap%
\pgfsetroundjoin%
\definecolor{currentfill}{rgb}{0.000000,0.000000,0.000000}%
\pgfsetfillcolor{currentfill}%
\pgfsetlinewidth{1.003750pt}%
\definecolor{currentstroke}{rgb}{0.000000,0.000000,0.000000}%
\pgfsetstrokecolor{currentstroke}%
\pgfsetdash{}{0pt}%
\pgfpathmoveto{\pgfqpoint{3.876477in}{0.356667in}}%
\pgfpathcurveto{\pgfqpoint{3.882002in}{0.356667in}}{\pgfqpoint{3.887302in}{0.358862in}}{\pgfqpoint{3.891209in}{0.362769in}}%
\pgfpathcurveto{\pgfqpoint{3.895115in}{0.366675in}}{\pgfqpoint{3.897311in}{0.371975in}}{\pgfqpoint{3.897311in}{0.377500in}}%
\pgfpathcurveto{\pgfqpoint{3.897311in}{0.383025in}}{\pgfqpoint{3.895115in}{0.388325in}}{\pgfqpoint{3.891209in}{0.392231in}}%
\pgfpathcurveto{\pgfqpoint{3.887302in}{0.396138in}}{\pgfqpoint{3.882002in}{0.398333in}}{\pgfqpoint{3.876477in}{0.398333in}}%
\pgfpathcurveto{\pgfqpoint{3.870952in}{0.398333in}}{\pgfqpoint{3.865653in}{0.396138in}}{\pgfqpoint{3.861746in}{0.392231in}}%
\pgfpathcurveto{\pgfqpoint{3.857839in}{0.388325in}}{\pgfqpoint{3.855644in}{0.383025in}}{\pgfqpoint{3.855644in}{0.377500in}}%
\pgfpathcurveto{\pgfqpoint{3.855644in}{0.371975in}}{\pgfqpoint{3.857839in}{0.366675in}}{\pgfqpoint{3.861746in}{0.362769in}}%
\pgfpathcurveto{\pgfqpoint{3.865653in}{0.358862in}}{\pgfqpoint{3.870952in}{0.356667in}}{\pgfqpoint{3.876477in}{0.356667in}}%
\pgfpathclose%
\pgfusepath{stroke,fill}%
\end{pgfscope}%
\begin{pgfscope}%
\pgfpathrectangle{\pgfqpoint{0.562500in}{0.275000in}}{\pgfqpoint{3.487500in}{1.925000in}}%
\pgfusepath{clip}%
\pgfsetbuttcap%
\pgfsetroundjoin%
\definecolor{currentfill}{rgb}{0.000000,0.000000,0.000000}%
\pgfsetfillcolor{currentfill}%
\pgfsetlinewidth{1.003750pt}%
\definecolor{currentstroke}{rgb}{0.000000,0.000000,0.000000}%
\pgfsetstrokecolor{currentstroke}%
\pgfsetdash{}{0pt}%
\pgfpathmoveto{\pgfqpoint{3.876477in}{0.356667in}}%
\pgfpathcurveto{\pgfqpoint{3.882002in}{0.356667in}}{\pgfqpoint{3.887302in}{0.358862in}}{\pgfqpoint{3.891209in}{0.362769in}}%
\pgfpathcurveto{\pgfqpoint{3.895115in}{0.366675in}}{\pgfqpoint{3.897311in}{0.371975in}}{\pgfqpoint{3.897311in}{0.377500in}}%
\pgfpathcurveto{\pgfqpoint{3.897311in}{0.383025in}}{\pgfqpoint{3.895115in}{0.388325in}}{\pgfqpoint{3.891209in}{0.392231in}}%
\pgfpathcurveto{\pgfqpoint{3.887302in}{0.396138in}}{\pgfqpoint{3.882002in}{0.398333in}}{\pgfqpoint{3.876477in}{0.398333in}}%
\pgfpathcurveto{\pgfqpoint{3.870952in}{0.398333in}}{\pgfqpoint{3.865653in}{0.396138in}}{\pgfqpoint{3.861746in}{0.392231in}}%
\pgfpathcurveto{\pgfqpoint{3.857839in}{0.388325in}}{\pgfqpoint{3.855644in}{0.383025in}}{\pgfqpoint{3.855644in}{0.377500in}}%
\pgfpathcurveto{\pgfqpoint{3.855644in}{0.371975in}}{\pgfqpoint{3.857839in}{0.366675in}}{\pgfqpoint{3.861746in}{0.362769in}}%
\pgfpathcurveto{\pgfqpoint{3.865653in}{0.358862in}}{\pgfqpoint{3.870952in}{0.356667in}}{\pgfqpoint{3.876477in}{0.356667in}}%
\pgfpathclose%
\pgfusepath{stroke,fill}%
\end{pgfscope}%
\begin{pgfscope}%
\pgfpathrectangle{\pgfqpoint{0.562500in}{0.275000in}}{\pgfqpoint{3.487500in}{1.925000in}}%
\pgfusepath{clip}%
\pgfsetbuttcap%
\pgfsetroundjoin%
\definecolor{currentfill}{rgb}{0.000000,0.000000,0.000000}%
\pgfsetfillcolor{currentfill}%
\pgfsetlinewidth{1.003750pt}%
\definecolor{currentstroke}{rgb}{0.000000,0.000000,0.000000}%
\pgfsetstrokecolor{currentstroke}%
\pgfsetdash{}{0pt}%
\pgfpathmoveto{\pgfqpoint{3.876477in}{0.356667in}}%
\pgfpathcurveto{\pgfqpoint{3.882002in}{0.356667in}}{\pgfqpoint{3.887302in}{0.358862in}}{\pgfqpoint{3.891209in}{0.362769in}}%
\pgfpathcurveto{\pgfqpoint{3.895115in}{0.366675in}}{\pgfqpoint{3.897311in}{0.371975in}}{\pgfqpoint{3.897311in}{0.377500in}}%
\pgfpathcurveto{\pgfqpoint{3.897311in}{0.383025in}}{\pgfqpoint{3.895115in}{0.388325in}}{\pgfqpoint{3.891209in}{0.392231in}}%
\pgfpathcurveto{\pgfqpoint{3.887302in}{0.396138in}}{\pgfqpoint{3.882002in}{0.398333in}}{\pgfqpoint{3.876477in}{0.398333in}}%
\pgfpathcurveto{\pgfqpoint{3.870952in}{0.398333in}}{\pgfqpoint{3.865653in}{0.396138in}}{\pgfqpoint{3.861746in}{0.392231in}}%
\pgfpathcurveto{\pgfqpoint{3.857839in}{0.388325in}}{\pgfqpoint{3.855644in}{0.383025in}}{\pgfqpoint{3.855644in}{0.377500in}}%
\pgfpathcurveto{\pgfqpoint{3.855644in}{0.371975in}}{\pgfqpoint{3.857839in}{0.366675in}}{\pgfqpoint{3.861746in}{0.362769in}}%
\pgfpathcurveto{\pgfqpoint{3.865653in}{0.358862in}}{\pgfqpoint{3.870952in}{0.356667in}}{\pgfqpoint{3.876477in}{0.356667in}}%
\pgfpathclose%
\pgfusepath{stroke,fill}%
\end{pgfscope}%
\begin{pgfscope}%
\pgfpathrectangle{\pgfqpoint{0.562500in}{0.275000in}}{\pgfqpoint{3.487500in}{1.925000in}}%
\pgfusepath{clip}%
\pgfsetbuttcap%
\pgfsetroundjoin%
\definecolor{currentfill}{rgb}{0.000000,0.000000,0.000000}%
\pgfsetfillcolor{currentfill}%
\pgfsetlinewidth{1.003750pt}%
\definecolor{currentstroke}{rgb}{0.000000,0.000000,0.000000}%
\pgfsetstrokecolor{currentstroke}%
\pgfsetdash{}{0pt}%
\pgfpathmoveto{\pgfqpoint{3.876477in}{0.356667in}}%
\pgfpathcurveto{\pgfqpoint{3.882002in}{0.356667in}}{\pgfqpoint{3.887302in}{0.358862in}}{\pgfqpoint{3.891209in}{0.362769in}}%
\pgfpathcurveto{\pgfqpoint{3.895115in}{0.366675in}}{\pgfqpoint{3.897311in}{0.371975in}}{\pgfqpoint{3.897311in}{0.377500in}}%
\pgfpathcurveto{\pgfqpoint{3.897311in}{0.383025in}}{\pgfqpoint{3.895115in}{0.388325in}}{\pgfqpoint{3.891209in}{0.392231in}}%
\pgfpathcurveto{\pgfqpoint{3.887302in}{0.396138in}}{\pgfqpoint{3.882002in}{0.398333in}}{\pgfqpoint{3.876477in}{0.398333in}}%
\pgfpathcurveto{\pgfqpoint{3.870952in}{0.398333in}}{\pgfqpoint{3.865653in}{0.396138in}}{\pgfqpoint{3.861746in}{0.392231in}}%
\pgfpathcurveto{\pgfqpoint{3.857839in}{0.388325in}}{\pgfqpoint{3.855644in}{0.383025in}}{\pgfqpoint{3.855644in}{0.377500in}}%
\pgfpathcurveto{\pgfqpoint{3.855644in}{0.371975in}}{\pgfqpoint{3.857839in}{0.366675in}}{\pgfqpoint{3.861746in}{0.362769in}}%
\pgfpathcurveto{\pgfqpoint{3.865653in}{0.358862in}}{\pgfqpoint{3.870952in}{0.356667in}}{\pgfqpoint{3.876477in}{0.356667in}}%
\pgfpathclose%
\pgfusepath{stroke,fill}%
\end{pgfscope}%
\begin{pgfscope}%
\pgfpathrectangle{\pgfqpoint{0.562500in}{0.275000in}}{\pgfqpoint{3.487500in}{1.925000in}}%
\pgfusepath{clip}%
\pgfsetbuttcap%
\pgfsetroundjoin%
\definecolor{currentfill}{rgb}{0.000000,0.000000,0.000000}%
\pgfsetfillcolor{currentfill}%
\pgfsetlinewidth{1.003750pt}%
\definecolor{currentstroke}{rgb}{0.000000,0.000000,0.000000}%
\pgfsetstrokecolor{currentstroke}%
\pgfsetdash{}{0pt}%
\pgfpathmoveto{\pgfqpoint{3.876477in}{0.356667in}}%
\pgfpathcurveto{\pgfqpoint{3.882002in}{0.356667in}}{\pgfqpoint{3.887302in}{0.358862in}}{\pgfqpoint{3.891209in}{0.362769in}}%
\pgfpathcurveto{\pgfqpoint{3.895115in}{0.366675in}}{\pgfqpoint{3.897311in}{0.371975in}}{\pgfqpoint{3.897311in}{0.377500in}}%
\pgfpathcurveto{\pgfqpoint{3.897311in}{0.383025in}}{\pgfqpoint{3.895115in}{0.388325in}}{\pgfqpoint{3.891209in}{0.392231in}}%
\pgfpathcurveto{\pgfqpoint{3.887302in}{0.396138in}}{\pgfqpoint{3.882002in}{0.398333in}}{\pgfqpoint{3.876477in}{0.398333in}}%
\pgfpathcurveto{\pgfqpoint{3.870952in}{0.398333in}}{\pgfqpoint{3.865653in}{0.396138in}}{\pgfqpoint{3.861746in}{0.392231in}}%
\pgfpathcurveto{\pgfqpoint{3.857839in}{0.388325in}}{\pgfqpoint{3.855644in}{0.383025in}}{\pgfqpoint{3.855644in}{0.377500in}}%
\pgfpathcurveto{\pgfqpoint{3.855644in}{0.371975in}}{\pgfqpoint{3.857839in}{0.366675in}}{\pgfqpoint{3.861746in}{0.362769in}}%
\pgfpathcurveto{\pgfqpoint{3.865653in}{0.358862in}}{\pgfqpoint{3.870952in}{0.356667in}}{\pgfqpoint{3.876477in}{0.356667in}}%
\pgfpathclose%
\pgfusepath{stroke,fill}%
\end{pgfscope}%
\begin{pgfscope}%
\pgfpathrectangle{\pgfqpoint{0.562500in}{0.275000in}}{\pgfqpoint{3.487500in}{1.925000in}}%
\pgfusepath{clip}%
\pgfsetbuttcap%
\pgfsetroundjoin%
\definecolor{currentfill}{rgb}{0.000000,0.000000,0.000000}%
\pgfsetfillcolor{currentfill}%
\pgfsetlinewidth{1.003750pt}%
\definecolor{currentstroke}{rgb}{0.000000,0.000000,0.000000}%
\pgfsetstrokecolor{currentstroke}%
\pgfsetdash{}{0pt}%
\pgfpathmoveto{\pgfqpoint{3.876477in}{0.356667in}}%
\pgfpathcurveto{\pgfqpoint{3.882002in}{0.356667in}}{\pgfqpoint{3.887302in}{0.358862in}}{\pgfqpoint{3.891209in}{0.362769in}}%
\pgfpathcurveto{\pgfqpoint{3.895115in}{0.366675in}}{\pgfqpoint{3.897311in}{0.371975in}}{\pgfqpoint{3.897311in}{0.377500in}}%
\pgfpathcurveto{\pgfqpoint{3.897311in}{0.383025in}}{\pgfqpoint{3.895115in}{0.388325in}}{\pgfqpoint{3.891209in}{0.392231in}}%
\pgfpathcurveto{\pgfqpoint{3.887302in}{0.396138in}}{\pgfqpoint{3.882002in}{0.398333in}}{\pgfqpoint{3.876477in}{0.398333in}}%
\pgfpathcurveto{\pgfqpoint{3.870952in}{0.398333in}}{\pgfqpoint{3.865653in}{0.396138in}}{\pgfqpoint{3.861746in}{0.392231in}}%
\pgfpathcurveto{\pgfqpoint{3.857839in}{0.388325in}}{\pgfqpoint{3.855644in}{0.383025in}}{\pgfqpoint{3.855644in}{0.377500in}}%
\pgfpathcurveto{\pgfqpoint{3.855644in}{0.371975in}}{\pgfqpoint{3.857839in}{0.366675in}}{\pgfqpoint{3.861746in}{0.362769in}}%
\pgfpathcurveto{\pgfqpoint{3.865653in}{0.358862in}}{\pgfqpoint{3.870952in}{0.356667in}}{\pgfqpoint{3.876477in}{0.356667in}}%
\pgfpathclose%
\pgfusepath{stroke,fill}%
\end{pgfscope}%
\begin{pgfscope}%
\pgfpathrectangle{\pgfqpoint{0.562500in}{0.275000in}}{\pgfqpoint{3.487500in}{1.925000in}}%
\pgfusepath{clip}%
\pgfsetbuttcap%
\pgfsetroundjoin%
\definecolor{currentfill}{rgb}{0.000000,0.000000,0.000000}%
\pgfsetfillcolor{currentfill}%
\pgfsetlinewidth{1.003750pt}%
\definecolor{currentstroke}{rgb}{0.000000,0.000000,0.000000}%
\pgfsetstrokecolor{currentstroke}%
\pgfsetdash{}{0pt}%
\pgfpathmoveto{\pgfqpoint{3.876477in}{0.356667in}}%
\pgfpathcurveto{\pgfqpoint{3.882002in}{0.356667in}}{\pgfqpoint{3.887302in}{0.358862in}}{\pgfqpoint{3.891209in}{0.362769in}}%
\pgfpathcurveto{\pgfqpoint{3.895115in}{0.366675in}}{\pgfqpoint{3.897311in}{0.371975in}}{\pgfqpoint{3.897311in}{0.377500in}}%
\pgfpathcurveto{\pgfqpoint{3.897311in}{0.383025in}}{\pgfqpoint{3.895115in}{0.388325in}}{\pgfqpoint{3.891209in}{0.392231in}}%
\pgfpathcurveto{\pgfqpoint{3.887302in}{0.396138in}}{\pgfqpoint{3.882002in}{0.398333in}}{\pgfqpoint{3.876477in}{0.398333in}}%
\pgfpathcurveto{\pgfqpoint{3.870952in}{0.398333in}}{\pgfqpoint{3.865653in}{0.396138in}}{\pgfqpoint{3.861746in}{0.392231in}}%
\pgfpathcurveto{\pgfqpoint{3.857839in}{0.388325in}}{\pgfqpoint{3.855644in}{0.383025in}}{\pgfqpoint{3.855644in}{0.377500in}}%
\pgfpathcurveto{\pgfqpoint{3.855644in}{0.371975in}}{\pgfqpoint{3.857839in}{0.366675in}}{\pgfqpoint{3.861746in}{0.362769in}}%
\pgfpathcurveto{\pgfqpoint{3.865653in}{0.358862in}}{\pgfqpoint{3.870952in}{0.356667in}}{\pgfqpoint{3.876477in}{0.356667in}}%
\pgfpathclose%
\pgfusepath{stroke,fill}%
\end{pgfscope}%
\begin{pgfscope}%
\pgfpathrectangle{\pgfqpoint{0.562500in}{0.275000in}}{\pgfqpoint{3.487500in}{1.925000in}}%
\pgfusepath{clip}%
\pgfsetbuttcap%
\pgfsetroundjoin%
\definecolor{currentfill}{rgb}{0.000000,0.000000,0.000000}%
\pgfsetfillcolor{currentfill}%
\pgfsetlinewidth{1.003750pt}%
\definecolor{currentstroke}{rgb}{0.000000,0.000000,0.000000}%
\pgfsetstrokecolor{currentstroke}%
\pgfsetdash{}{0pt}%
\pgfpathmoveto{\pgfqpoint{3.876477in}{0.356667in}}%
\pgfpathcurveto{\pgfqpoint{3.882002in}{0.356667in}}{\pgfqpoint{3.887302in}{0.358862in}}{\pgfqpoint{3.891209in}{0.362769in}}%
\pgfpathcurveto{\pgfqpoint{3.895115in}{0.366675in}}{\pgfqpoint{3.897311in}{0.371975in}}{\pgfqpoint{3.897311in}{0.377500in}}%
\pgfpathcurveto{\pgfqpoint{3.897311in}{0.383025in}}{\pgfqpoint{3.895115in}{0.388325in}}{\pgfqpoint{3.891209in}{0.392231in}}%
\pgfpathcurveto{\pgfqpoint{3.887302in}{0.396138in}}{\pgfqpoint{3.882002in}{0.398333in}}{\pgfqpoint{3.876477in}{0.398333in}}%
\pgfpathcurveto{\pgfqpoint{3.870952in}{0.398333in}}{\pgfqpoint{3.865653in}{0.396138in}}{\pgfqpoint{3.861746in}{0.392231in}}%
\pgfpathcurveto{\pgfqpoint{3.857839in}{0.388325in}}{\pgfqpoint{3.855644in}{0.383025in}}{\pgfqpoint{3.855644in}{0.377500in}}%
\pgfpathcurveto{\pgfqpoint{3.855644in}{0.371975in}}{\pgfqpoint{3.857839in}{0.366675in}}{\pgfqpoint{3.861746in}{0.362769in}}%
\pgfpathcurveto{\pgfqpoint{3.865653in}{0.358862in}}{\pgfqpoint{3.870952in}{0.356667in}}{\pgfqpoint{3.876477in}{0.356667in}}%
\pgfpathclose%
\pgfusepath{stroke,fill}%
\end{pgfscope}%
\begin{pgfscope}%
\pgfpathrectangle{\pgfqpoint{0.562500in}{0.275000in}}{\pgfqpoint{3.487500in}{1.925000in}}%
\pgfusepath{clip}%
\pgfsetbuttcap%
\pgfsetroundjoin%
\definecolor{currentfill}{rgb}{0.000000,0.000000,0.000000}%
\pgfsetfillcolor{currentfill}%
\pgfsetlinewidth{1.003750pt}%
\definecolor{currentstroke}{rgb}{0.000000,0.000000,0.000000}%
\pgfsetstrokecolor{currentstroke}%
\pgfsetdash{}{0pt}%
\pgfpathmoveto{\pgfqpoint{3.876477in}{0.356667in}}%
\pgfpathcurveto{\pgfqpoint{3.882002in}{0.356667in}}{\pgfqpoint{3.887302in}{0.358862in}}{\pgfqpoint{3.891209in}{0.362769in}}%
\pgfpathcurveto{\pgfqpoint{3.895115in}{0.366675in}}{\pgfqpoint{3.897311in}{0.371975in}}{\pgfqpoint{3.897311in}{0.377500in}}%
\pgfpathcurveto{\pgfqpoint{3.897311in}{0.383025in}}{\pgfqpoint{3.895115in}{0.388325in}}{\pgfqpoint{3.891209in}{0.392231in}}%
\pgfpathcurveto{\pgfqpoint{3.887302in}{0.396138in}}{\pgfqpoint{3.882002in}{0.398333in}}{\pgfqpoint{3.876477in}{0.398333in}}%
\pgfpathcurveto{\pgfqpoint{3.870952in}{0.398333in}}{\pgfqpoint{3.865653in}{0.396138in}}{\pgfqpoint{3.861746in}{0.392231in}}%
\pgfpathcurveto{\pgfqpoint{3.857839in}{0.388325in}}{\pgfqpoint{3.855644in}{0.383025in}}{\pgfqpoint{3.855644in}{0.377500in}}%
\pgfpathcurveto{\pgfqpoint{3.855644in}{0.371975in}}{\pgfqpoint{3.857839in}{0.366675in}}{\pgfqpoint{3.861746in}{0.362769in}}%
\pgfpathcurveto{\pgfqpoint{3.865653in}{0.358862in}}{\pgfqpoint{3.870952in}{0.356667in}}{\pgfqpoint{3.876477in}{0.356667in}}%
\pgfpathclose%
\pgfusepath{stroke,fill}%
\end{pgfscope}%
\begin{pgfscope}%
\pgfpathrectangle{\pgfqpoint{0.562500in}{0.275000in}}{\pgfqpoint{3.487500in}{1.925000in}}%
\pgfusepath{clip}%
\pgfsetbuttcap%
\pgfsetroundjoin%
\definecolor{currentfill}{rgb}{0.000000,0.000000,0.000000}%
\pgfsetfillcolor{currentfill}%
\pgfsetlinewidth{1.003750pt}%
\definecolor{currentstroke}{rgb}{0.000000,0.000000,0.000000}%
\pgfsetstrokecolor{currentstroke}%
\pgfsetdash{}{0pt}%
\pgfpathmoveto{\pgfqpoint{3.876477in}{0.356667in}}%
\pgfpathcurveto{\pgfqpoint{3.882002in}{0.356667in}}{\pgfqpoint{3.887302in}{0.358862in}}{\pgfqpoint{3.891209in}{0.362769in}}%
\pgfpathcurveto{\pgfqpoint{3.895115in}{0.366675in}}{\pgfqpoint{3.897311in}{0.371975in}}{\pgfqpoint{3.897311in}{0.377500in}}%
\pgfpathcurveto{\pgfqpoint{3.897311in}{0.383025in}}{\pgfqpoint{3.895115in}{0.388325in}}{\pgfqpoint{3.891209in}{0.392231in}}%
\pgfpathcurveto{\pgfqpoint{3.887302in}{0.396138in}}{\pgfqpoint{3.882002in}{0.398333in}}{\pgfqpoint{3.876477in}{0.398333in}}%
\pgfpathcurveto{\pgfqpoint{3.870952in}{0.398333in}}{\pgfqpoint{3.865653in}{0.396138in}}{\pgfqpoint{3.861746in}{0.392231in}}%
\pgfpathcurveto{\pgfqpoint{3.857839in}{0.388325in}}{\pgfqpoint{3.855644in}{0.383025in}}{\pgfqpoint{3.855644in}{0.377500in}}%
\pgfpathcurveto{\pgfqpoint{3.855644in}{0.371975in}}{\pgfqpoint{3.857839in}{0.366675in}}{\pgfqpoint{3.861746in}{0.362769in}}%
\pgfpathcurveto{\pgfqpoint{3.865653in}{0.358862in}}{\pgfqpoint{3.870952in}{0.356667in}}{\pgfqpoint{3.876477in}{0.356667in}}%
\pgfpathclose%
\pgfusepath{stroke,fill}%
\end{pgfscope}%
\begin{pgfscope}%
\pgfpathrectangle{\pgfqpoint{0.562500in}{0.275000in}}{\pgfqpoint{3.487500in}{1.925000in}}%
\pgfusepath{clip}%
\pgfsetbuttcap%
\pgfsetroundjoin%
\definecolor{currentfill}{rgb}{0.000000,0.000000,0.000000}%
\pgfsetfillcolor{currentfill}%
\pgfsetlinewidth{1.003750pt}%
\definecolor{currentstroke}{rgb}{0.000000,0.000000,0.000000}%
\pgfsetstrokecolor{currentstroke}%
\pgfsetdash{}{0pt}%
\pgfpathmoveto{\pgfqpoint{3.876477in}{0.356667in}}%
\pgfpathcurveto{\pgfqpoint{3.882002in}{0.356667in}}{\pgfqpoint{3.887302in}{0.358862in}}{\pgfqpoint{3.891209in}{0.362769in}}%
\pgfpathcurveto{\pgfqpoint{3.895115in}{0.366675in}}{\pgfqpoint{3.897311in}{0.371975in}}{\pgfqpoint{3.897311in}{0.377500in}}%
\pgfpathcurveto{\pgfqpoint{3.897311in}{0.383025in}}{\pgfqpoint{3.895115in}{0.388325in}}{\pgfqpoint{3.891209in}{0.392231in}}%
\pgfpathcurveto{\pgfqpoint{3.887302in}{0.396138in}}{\pgfqpoint{3.882002in}{0.398333in}}{\pgfqpoint{3.876477in}{0.398333in}}%
\pgfpathcurveto{\pgfqpoint{3.870952in}{0.398333in}}{\pgfqpoint{3.865653in}{0.396138in}}{\pgfqpoint{3.861746in}{0.392231in}}%
\pgfpathcurveto{\pgfqpoint{3.857839in}{0.388325in}}{\pgfqpoint{3.855644in}{0.383025in}}{\pgfqpoint{3.855644in}{0.377500in}}%
\pgfpathcurveto{\pgfqpoint{3.855644in}{0.371975in}}{\pgfqpoint{3.857839in}{0.366675in}}{\pgfqpoint{3.861746in}{0.362769in}}%
\pgfpathcurveto{\pgfqpoint{3.865653in}{0.358862in}}{\pgfqpoint{3.870952in}{0.356667in}}{\pgfqpoint{3.876477in}{0.356667in}}%
\pgfpathclose%
\pgfusepath{stroke,fill}%
\end{pgfscope}%
\begin{pgfscope}%
\pgfpathrectangle{\pgfqpoint{0.562500in}{0.275000in}}{\pgfqpoint{3.487500in}{1.925000in}}%
\pgfusepath{clip}%
\pgfsetbuttcap%
\pgfsetroundjoin%
\definecolor{currentfill}{rgb}{0.000000,0.000000,0.000000}%
\pgfsetfillcolor{currentfill}%
\pgfsetlinewidth{1.003750pt}%
\definecolor{currentstroke}{rgb}{0.000000,0.000000,0.000000}%
\pgfsetstrokecolor{currentstroke}%
\pgfsetdash{}{0pt}%
\pgfpathmoveto{\pgfqpoint{3.876477in}{0.356667in}}%
\pgfpathcurveto{\pgfqpoint{3.882002in}{0.356667in}}{\pgfqpoint{3.887302in}{0.358862in}}{\pgfqpoint{3.891209in}{0.362769in}}%
\pgfpathcurveto{\pgfqpoint{3.895115in}{0.366675in}}{\pgfqpoint{3.897311in}{0.371975in}}{\pgfqpoint{3.897311in}{0.377500in}}%
\pgfpathcurveto{\pgfqpoint{3.897311in}{0.383025in}}{\pgfqpoint{3.895115in}{0.388325in}}{\pgfqpoint{3.891209in}{0.392231in}}%
\pgfpathcurveto{\pgfqpoint{3.887302in}{0.396138in}}{\pgfqpoint{3.882002in}{0.398333in}}{\pgfqpoint{3.876477in}{0.398333in}}%
\pgfpathcurveto{\pgfqpoint{3.870952in}{0.398333in}}{\pgfqpoint{3.865653in}{0.396138in}}{\pgfqpoint{3.861746in}{0.392231in}}%
\pgfpathcurveto{\pgfqpoint{3.857839in}{0.388325in}}{\pgfqpoint{3.855644in}{0.383025in}}{\pgfqpoint{3.855644in}{0.377500in}}%
\pgfpathcurveto{\pgfqpoint{3.855644in}{0.371975in}}{\pgfqpoint{3.857839in}{0.366675in}}{\pgfqpoint{3.861746in}{0.362769in}}%
\pgfpathcurveto{\pgfqpoint{3.865653in}{0.358862in}}{\pgfqpoint{3.870952in}{0.356667in}}{\pgfqpoint{3.876477in}{0.356667in}}%
\pgfpathclose%
\pgfusepath{stroke,fill}%
\end{pgfscope}%
\begin{pgfscope}%
\pgfpathrectangle{\pgfqpoint{0.562500in}{0.275000in}}{\pgfqpoint{3.487500in}{1.925000in}}%
\pgfusepath{clip}%
\pgfsetbuttcap%
\pgfsetroundjoin%
\definecolor{currentfill}{rgb}{0.000000,0.000000,0.000000}%
\pgfsetfillcolor{currentfill}%
\pgfsetlinewidth{1.003750pt}%
\definecolor{currentstroke}{rgb}{0.000000,0.000000,0.000000}%
\pgfsetstrokecolor{currentstroke}%
\pgfsetdash{}{0pt}%
\pgfpathmoveto{\pgfqpoint{3.876477in}{0.356667in}}%
\pgfpathcurveto{\pgfqpoint{3.882002in}{0.356667in}}{\pgfqpoint{3.887302in}{0.358862in}}{\pgfqpoint{3.891209in}{0.362769in}}%
\pgfpathcurveto{\pgfqpoint{3.895115in}{0.366675in}}{\pgfqpoint{3.897311in}{0.371975in}}{\pgfqpoint{3.897311in}{0.377500in}}%
\pgfpathcurveto{\pgfqpoint{3.897311in}{0.383025in}}{\pgfqpoint{3.895115in}{0.388325in}}{\pgfqpoint{3.891209in}{0.392231in}}%
\pgfpathcurveto{\pgfqpoint{3.887302in}{0.396138in}}{\pgfqpoint{3.882002in}{0.398333in}}{\pgfqpoint{3.876477in}{0.398333in}}%
\pgfpathcurveto{\pgfqpoint{3.870952in}{0.398333in}}{\pgfqpoint{3.865653in}{0.396138in}}{\pgfqpoint{3.861746in}{0.392231in}}%
\pgfpathcurveto{\pgfqpoint{3.857839in}{0.388325in}}{\pgfqpoint{3.855644in}{0.383025in}}{\pgfqpoint{3.855644in}{0.377500in}}%
\pgfpathcurveto{\pgfqpoint{3.855644in}{0.371975in}}{\pgfqpoint{3.857839in}{0.366675in}}{\pgfqpoint{3.861746in}{0.362769in}}%
\pgfpathcurveto{\pgfqpoint{3.865653in}{0.358862in}}{\pgfqpoint{3.870952in}{0.356667in}}{\pgfqpoint{3.876477in}{0.356667in}}%
\pgfpathclose%
\pgfusepath{stroke,fill}%
\end{pgfscope}%
\begin{pgfscope}%
\pgfpathrectangle{\pgfqpoint{0.562500in}{0.275000in}}{\pgfqpoint{3.487500in}{1.925000in}}%
\pgfusepath{clip}%
\pgfsetbuttcap%
\pgfsetroundjoin%
\definecolor{currentfill}{rgb}{0.000000,0.000000,0.000000}%
\pgfsetfillcolor{currentfill}%
\pgfsetlinewidth{1.003750pt}%
\definecolor{currentstroke}{rgb}{0.000000,0.000000,0.000000}%
\pgfsetstrokecolor{currentstroke}%
\pgfsetdash{}{0pt}%
\pgfpathmoveto{\pgfqpoint{3.876477in}{0.356667in}}%
\pgfpathcurveto{\pgfqpoint{3.882002in}{0.356667in}}{\pgfqpoint{3.887302in}{0.358862in}}{\pgfqpoint{3.891209in}{0.362769in}}%
\pgfpathcurveto{\pgfqpoint{3.895115in}{0.366675in}}{\pgfqpoint{3.897311in}{0.371975in}}{\pgfqpoint{3.897311in}{0.377500in}}%
\pgfpathcurveto{\pgfqpoint{3.897311in}{0.383025in}}{\pgfqpoint{3.895115in}{0.388325in}}{\pgfqpoint{3.891209in}{0.392231in}}%
\pgfpathcurveto{\pgfqpoint{3.887302in}{0.396138in}}{\pgfqpoint{3.882002in}{0.398333in}}{\pgfqpoint{3.876477in}{0.398333in}}%
\pgfpathcurveto{\pgfqpoint{3.870952in}{0.398333in}}{\pgfqpoint{3.865653in}{0.396138in}}{\pgfqpoint{3.861746in}{0.392231in}}%
\pgfpathcurveto{\pgfqpoint{3.857839in}{0.388325in}}{\pgfqpoint{3.855644in}{0.383025in}}{\pgfqpoint{3.855644in}{0.377500in}}%
\pgfpathcurveto{\pgfqpoint{3.855644in}{0.371975in}}{\pgfqpoint{3.857839in}{0.366675in}}{\pgfqpoint{3.861746in}{0.362769in}}%
\pgfpathcurveto{\pgfqpoint{3.865653in}{0.358862in}}{\pgfqpoint{3.870952in}{0.356667in}}{\pgfqpoint{3.876477in}{0.356667in}}%
\pgfpathclose%
\pgfusepath{stroke,fill}%
\end{pgfscope}%
\begin{pgfscope}%
\pgfpathrectangle{\pgfqpoint{0.562500in}{0.275000in}}{\pgfqpoint{3.487500in}{1.925000in}}%
\pgfusepath{clip}%
\pgfsetbuttcap%
\pgfsetroundjoin%
\definecolor{currentfill}{rgb}{0.000000,0.000000,0.000000}%
\pgfsetfillcolor{currentfill}%
\pgfsetlinewidth{1.003750pt}%
\definecolor{currentstroke}{rgb}{0.000000,0.000000,0.000000}%
\pgfsetstrokecolor{currentstroke}%
\pgfsetdash{}{0pt}%
\pgfpathmoveto{\pgfqpoint{3.876477in}{0.356667in}}%
\pgfpathcurveto{\pgfqpoint{3.882002in}{0.356667in}}{\pgfqpoint{3.887302in}{0.358862in}}{\pgfqpoint{3.891209in}{0.362769in}}%
\pgfpathcurveto{\pgfqpoint{3.895115in}{0.366675in}}{\pgfqpoint{3.897311in}{0.371975in}}{\pgfqpoint{3.897311in}{0.377500in}}%
\pgfpathcurveto{\pgfqpoint{3.897311in}{0.383025in}}{\pgfqpoint{3.895115in}{0.388325in}}{\pgfqpoint{3.891209in}{0.392231in}}%
\pgfpathcurveto{\pgfqpoint{3.887302in}{0.396138in}}{\pgfqpoint{3.882002in}{0.398333in}}{\pgfqpoint{3.876477in}{0.398333in}}%
\pgfpathcurveto{\pgfqpoint{3.870952in}{0.398333in}}{\pgfqpoint{3.865653in}{0.396138in}}{\pgfqpoint{3.861746in}{0.392231in}}%
\pgfpathcurveto{\pgfqpoint{3.857839in}{0.388325in}}{\pgfqpoint{3.855644in}{0.383025in}}{\pgfqpoint{3.855644in}{0.377500in}}%
\pgfpathcurveto{\pgfqpoint{3.855644in}{0.371975in}}{\pgfqpoint{3.857839in}{0.366675in}}{\pgfqpoint{3.861746in}{0.362769in}}%
\pgfpathcurveto{\pgfqpoint{3.865653in}{0.358862in}}{\pgfqpoint{3.870952in}{0.356667in}}{\pgfqpoint{3.876477in}{0.356667in}}%
\pgfpathclose%
\pgfusepath{stroke,fill}%
\end{pgfscope}%
\begin{pgfscope}%
\pgfpathrectangle{\pgfqpoint{0.562500in}{0.275000in}}{\pgfqpoint{3.487500in}{1.925000in}}%
\pgfusepath{clip}%
\pgfsetbuttcap%
\pgfsetroundjoin%
\definecolor{currentfill}{rgb}{0.000000,0.000000,0.000000}%
\pgfsetfillcolor{currentfill}%
\pgfsetlinewidth{1.003750pt}%
\definecolor{currentstroke}{rgb}{0.000000,0.000000,0.000000}%
\pgfsetstrokecolor{currentstroke}%
\pgfsetdash{}{0pt}%
\pgfpathmoveto{\pgfqpoint{3.876477in}{0.356667in}}%
\pgfpathcurveto{\pgfqpoint{3.882002in}{0.356667in}}{\pgfqpoint{3.887302in}{0.358862in}}{\pgfqpoint{3.891209in}{0.362769in}}%
\pgfpathcurveto{\pgfqpoint{3.895115in}{0.366675in}}{\pgfqpoint{3.897311in}{0.371975in}}{\pgfqpoint{3.897311in}{0.377500in}}%
\pgfpathcurveto{\pgfqpoint{3.897311in}{0.383025in}}{\pgfqpoint{3.895115in}{0.388325in}}{\pgfqpoint{3.891209in}{0.392231in}}%
\pgfpathcurveto{\pgfqpoint{3.887302in}{0.396138in}}{\pgfqpoint{3.882002in}{0.398333in}}{\pgfqpoint{3.876477in}{0.398333in}}%
\pgfpathcurveto{\pgfqpoint{3.870952in}{0.398333in}}{\pgfqpoint{3.865653in}{0.396138in}}{\pgfqpoint{3.861746in}{0.392231in}}%
\pgfpathcurveto{\pgfqpoint{3.857839in}{0.388325in}}{\pgfqpoint{3.855644in}{0.383025in}}{\pgfqpoint{3.855644in}{0.377500in}}%
\pgfpathcurveto{\pgfqpoint{3.855644in}{0.371975in}}{\pgfqpoint{3.857839in}{0.366675in}}{\pgfqpoint{3.861746in}{0.362769in}}%
\pgfpathcurveto{\pgfqpoint{3.865653in}{0.358862in}}{\pgfqpoint{3.870952in}{0.356667in}}{\pgfqpoint{3.876477in}{0.356667in}}%
\pgfpathclose%
\pgfusepath{stroke,fill}%
\end{pgfscope}%
\begin{pgfscope}%
\pgfpathrectangle{\pgfqpoint{0.562500in}{0.275000in}}{\pgfqpoint{3.487500in}{1.925000in}}%
\pgfusepath{clip}%
\pgfsetbuttcap%
\pgfsetroundjoin%
\definecolor{currentfill}{rgb}{0.000000,0.000000,0.000000}%
\pgfsetfillcolor{currentfill}%
\pgfsetlinewidth{1.003750pt}%
\definecolor{currentstroke}{rgb}{0.000000,0.000000,0.000000}%
\pgfsetstrokecolor{currentstroke}%
\pgfsetdash{}{0pt}%
\pgfpathmoveto{\pgfqpoint{3.876477in}{0.356667in}}%
\pgfpathcurveto{\pgfqpoint{3.882002in}{0.356667in}}{\pgfqpoint{3.887302in}{0.358862in}}{\pgfqpoint{3.891209in}{0.362769in}}%
\pgfpathcurveto{\pgfqpoint{3.895115in}{0.366675in}}{\pgfqpoint{3.897311in}{0.371975in}}{\pgfqpoint{3.897311in}{0.377500in}}%
\pgfpathcurveto{\pgfqpoint{3.897311in}{0.383025in}}{\pgfqpoint{3.895115in}{0.388325in}}{\pgfqpoint{3.891209in}{0.392231in}}%
\pgfpathcurveto{\pgfqpoint{3.887302in}{0.396138in}}{\pgfqpoint{3.882002in}{0.398333in}}{\pgfqpoint{3.876477in}{0.398333in}}%
\pgfpathcurveto{\pgfqpoint{3.870952in}{0.398333in}}{\pgfqpoint{3.865653in}{0.396138in}}{\pgfqpoint{3.861746in}{0.392231in}}%
\pgfpathcurveto{\pgfqpoint{3.857839in}{0.388325in}}{\pgfqpoint{3.855644in}{0.383025in}}{\pgfqpoint{3.855644in}{0.377500in}}%
\pgfpathcurveto{\pgfqpoint{3.855644in}{0.371975in}}{\pgfqpoint{3.857839in}{0.366675in}}{\pgfqpoint{3.861746in}{0.362769in}}%
\pgfpathcurveto{\pgfqpoint{3.865653in}{0.358862in}}{\pgfqpoint{3.870952in}{0.356667in}}{\pgfqpoint{3.876477in}{0.356667in}}%
\pgfpathclose%
\pgfusepath{stroke,fill}%
\end{pgfscope}%
\begin{pgfscope}%
\pgfpathrectangle{\pgfqpoint{0.562500in}{0.275000in}}{\pgfqpoint{3.487500in}{1.925000in}}%
\pgfusepath{clip}%
\pgfsetbuttcap%
\pgfsetroundjoin%
\definecolor{currentfill}{rgb}{0.000000,0.000000,0.000000}%
\pgfsetfillcolor{currentfill}%
\pgfsetlinewidth{1.003750pt}%
\definecolor{currentstroke}{rgb}{0.000000,0.000000,0.000000}%
\pgfsetstrokecolor{currentstroke}%
\pgfsetdash{}{0pt}%
\pgfpathmoveto{\pgfqpoint{3.876477in}{0.356667in}}%
\pgfpathcurveto{\pgfqpoint{3.882002in}{0.356667in}}{\pgfqpoint{3.887302in}{0.358862in}}{\pgfqpoint{3.891209in}{0.362769in}}%
\pgfpathcurveto{\pgfqpoint{3.895115in}{0.366675in}}{\pgfqpoint{3.897311in}{0.371975in}}{\pgfqpoint{3.897311in}{0.377500in}}%
\pgfpathcurveto{\pgfqpoint{3.897311in}{0.383025in}}{\pgfqpoint{3.895115in}{0.388325in}}{\pgfqpoint{3.891209in}{0.392231in}}%
\pgfpathcurveto{\pgfqpoint{3.887302in}{0.396138in}}{\pgfqpoint{3.882002in}{0.398333in}}{\pgfqpoint{3.876477in}{0.398333in}}%
\pgfpathcurveto{\pgfqpoint{3.870952in}{0.398333in}}{\pgfqpoint{3.865653in}{0.396138in}}{\pgfqpoint{3.861746in}{0.392231in}}%
\pgfpathcurveto{\pgfqpoint{3.857839in}{0.388325in}}{\pgfqpoint{3.855644in}{0.383025in}}{\pgfqpoint{3.855644in}{0.377500in}}%
\pgfpathcurveto{\pgfqpoint{3.855644in}{0.371975in}}{\pgfqpoint{3.857839in}{0.366675in}}{\pgfqpoint{3.861746in}{0.362769in}}%
\pgfpathcurveto{\pgfqpoint{3.865653in}{0.358862in}}{\pgfqpoint{3.870952in}{0.356667in}}{\pgfqpoint{3.876477in}{0.356667in}}%
\pgfpathclose%
\pgfusepath{stroke,fill}%
\end{pgfscope}%
\begin{pgfscope}%
\pgfpathrectangle{\pgfqpoint{0.562500in}{0.275000in}}{\pgfqpoint{3.487500in}{1.925000in}}%
\pgfusepath{clip}%
\pgfsetbuttcap%
\pgfsetroundjoin%
\definecolor{currentfill}{rgb}{0.000000,0.000000,0.000000}%
\pgfsetfillcolor{currentfill}%
\pgfsetlinewidth{1.003750pt}%
\definecolor{currentstroke}{rgb}{0.000000,0.000000,0.000000}%
\pgfsetstrokecolor{currentstroke}%
\pgfsetdash{}{0pt}%
\pgfpathmoveto{\pgfqpoint{3.876477in}{0.356667in}}%
\pgfpathcurveto{\pgfqpoint{3.882002in}{0.356667in}}{\pgfqpoint{3.887302in}{0.358862in}}{\pgfqpoint{3.891209in}{0.362769in}}%
\pgfpathcurveto{\pgfqpoint{3.895115in}{0.366675in}}{\pgfqpoint{3.897311in}{0.371975in}}{\pgfqpoint{3.897311in}{0.377500in}}%
\pgfpathcurveto{\pgfqpoint{3.897311in}{0.383025in}}{\pgfqpoint{3.895115in}{0.388325in}}{\pgfqpoint{3.891209in}{0.392231in}}%
\pgfpathcurveto{\pgfqpoint{3.887302in}{0.396138in}}{\pgfqpoint{3.882002in}{0.398333in}}{\pgfqpoint{3.876477in}{0.398333in}}%
\pgfpathcurveto{\pgfqpoint{3.870952in}{0.398333in}}{\pgfqpoint{3.865653in}{0.396138in}}{\pgfqpoint{3.861746in}{0.392231in}}%
\pgfpathcurveto{\pgfqpoint{3.857839in}{0.388325in}}{\pgfqpoint{3.855644in}{0.383025in}}{\pgfqpoint{3.855644in}{0.377500in}}%
\pgfpathcurveto{\pgfqpoint{3.855644in}{0.371975in}}{\pgfqpoint{3.857839in}{0.366675in}}{\pgfqpoint{3.861746in}{0.362769in}}%
\pgfpathcurveto{\pgfqpoint{3.865653in}{0.358862in}}{\pgfqpoint{3.870952in}{0.356667in}}{\pgfqpoint{3.876477in}{0.356667in}}%
\pgfpathclose%
\pgfusepath{stroke,fill}%
\end{pgfscope}%
\begin{pgfscope}%
\pgfpathrectangle{\pgfqpoint{0.562500in}{0.275000in}}{\pgfqpoint{3.487500in}{1.925000in}}%
\pgfusepath{clip}%
\pgfsetbuttcap%
\pgfsetroundjoin%
\definecolor{currentfill}{rgb}{0.000000,0.000000,0.000000}%
\pgfsetfillcolor{currentfill}%
\pgfsetlinewidth{1.003750pt}%
\definecolor{currentstroke}{rgb}{0.000000,0.000000,0.000000}%
\pgfsetstrokecolor{currentstroke}%
\pgfsetdash{}{0pt}%
\pgfpathmoveto{\pgfqpoint{3.876477in}{0.356667in}}%
\pgfpathcurveto{\pgfqpoint{3.882002in}{0.356667in}}{\pgfqpoint{3.887302in}{0.358862in}}{\pgfqpoint{3.891209in}{0.362769in}}%
\pgfpathcurveto{\pgfqpoint{3.895115in}{0.366675in}}{\pgfqpoint{3.897311in}{0.371975in}}{\pgfqpoint{3.897311in}{0.377500in}}%
\pgfpathcurveto{\pgfqpoint{3.897311in}{0.383025in}}{\pgfqpoint{3.895115in}{0.388325in}}{\pgfqpoint{3.891209in}{0.392231in}}%
\pgfpathcurveto{\pgfqpoint{3.887302in}{0.396138in}}{\pgfqpoint{3.882002in}{0.398333in}}{\pgfqpoint{3.876477in}{0.398333in}}%
\pgfpathcurveto{\pgfqpoint{3.870952in}{0.398333in}}{\pgfqpoint{3.865653in}{0.396138in}}{\pgfqpoint{3.861746in}{0.392231in}}%
\pgfpathcurveto{\pgfqpoint{3.857839in}{0.388325in}}{\pgfqpoint{3.855644in}{0.383025in}}{\pgfqpoint{3.855644in}{0.377500in}}%
\pgfpathcurveto{\pgfqpoint{3.855644in}{0.371975in}}{\pgfqpoint{3.857839in}{0.366675in}}{\pgfqpoint{3.861746in}{0.362769in}}%
\pgfpathcurveto{\pgfqpoint{3.865653in}{0.358862in}}{\pgfqpoint{3.870952in}{0.356667in}}{\pgfqpoint{3.876477in}{0.356667in}}%
\pgfpathclose%
\pgfusepath{stroke,fill}%
\end{pgfscope}%
\begin{pgfscope}%
\pgfpathrectangle{\pgfqpoint{0.562500in}{0.275000in}}{\pgfqpoint{3.487500in}{1.925000in}}%
\pgfusepath{clip}%
\pgfsetbuttcap%
\pgfsetroundjoin%
\definecolor{currentfill}{rgb}{0.000000,0.000000,0.000000}%
\pgfsetfillcolor{currentfill}%
\pgfsetlinewidth{1.003750pt}%
\definecolor{currentstroke}{rgb}{0.000000,0.000000,0.000000}%
\pgfsetstrokecolor{currentstroke}%
\pgfsetdash{}{0pt}%
\pgfpathmoveto{\pgfqpoint{3.876477in}{0.356667in}}%
\pgfpathcurveto{\pgfqpoint{3.882002in}{0.356667in}}{\pgfqpoint{3.887302in}{0.358862in}}{\pgfqpoint{3.891209in}{0.362769in}}%
\pgfpathcurveto{\pgfqpoint{3.895115in}{0.366675in}}{\pgfqpoint{3.897311in}{0.371975in}}{\pgfqpoint{3.897311in}{0.377500in}}%
\pgfpathcurveto{\pgfqpoint{3.897311in}{0.383025in}}{\pgfqpoint{3.895115in}{0.388325in}}{\pgfqpoint{3.891209in}{0.392231in}}%
\pgfpathcurveto{\pgfqpoint{3.887302in}{0.396138in}}{\pgfqpoint{3.882002in}{0.398333in}}{\pgfqpoint{3.876477in}{0.398333in}}%
\pgfpathcurveto{\pgfqpoint{3.870952in}{0.398333in}}{\pgfqpoint{3.865653in}{0.396138in}}{\pgfqpoint{3.861746in}{0.392231in}}%
\pgfpathcurveto{\pgfqpoint{3.857839in}{0.388325in}}{\pgfqpoint{3.855644in}{0.383025in}}{\pgfqpoint{3.855644in}{0.377500in}}%
\pgfpathcurveto{\pgfqpoint{3.855644in}{0.371975in}}{\pgfqpoint{3.857839in}{0.366675in}}{\pgfqpoint{3.861746in}{0.362769in}}%
\pgfpathcurveto{\pgfqpoint{3.865653in}{0.358862in}}{\pgfqpoint{3.870952in}{0.356667in}}{\pgfqpoint{3.876477in}{0.356667in}}%
\pgfpathclose%
\pgfusepath{stroke,fill}%
\end{pgfscope}%
\begin{pgfscope}%
\pgfpathrectangle{\pgfqpoint{0.562500in}{0.275000in}}{\pgfqpoint{3.487500in}{1.925000in}}%
\pgfusepath{clip}%
\pgfsetbuttcap%
\pgfsetroundjoin%
\definecolor{currentfill}{rgb}{0.000000,0.000000,0.000000}%
\pgfsetfillcolor{currentfill}%
\pgfsetlinewidth{1.003750pt}%
\definecolor{currentstroke}{rgb}{0.000000,0.000000,0.000000}%
\pgfsetstrokecolor{currentstroke}%
\pgfsetdash{}{0pt}%
\pgfpathmoveto{\pgfqpoint{3.876477in}{0.356667in}}%
\pgfpathcurveto{\pgfqpoint{3.882002in}{0.356667in}}{\pgfqpoint{3.887302in}{0.358862in}}{\pgfqpoint{3.891209in}{0.362769in}}%
\pgfpathcurveto{\pgfqpoint{3.895115in}{0.366675in}}{\pgfqpoint{3.897311in}{0.371975in}}{\pgfqpoint{3.897311in}{0.377500in}}%
\pgfpathcurveto{\pgfqpoint{3.897311in}{0.383025in}}{\pgfqpoint{3.895115in}{0.388325in}}{\pgfqpoint{3.891209in}{0.392231in}}%
\pgfpathcurveto{\pgfqpoint{3.887302in}{0.396138in}}{\pgfqpoint{3.882002in}{0.398333in}}{\pgfqpoint{3.876477in}{0.398333in}}%
\pgfpathcurveto{\pgfqpoint{3.870952in}{0.398333in}}{\pgfqpoint{3.865653in}{0.396138in}}{\pgfqpoint{3.861746in}{0.392231in}}%
\pgfpathcurveto{\pgfqpoint{3.857839in}{0.388325in}}{\pgfqpoint{3.855644in}{0.383025in}}{\pgfqpoint{3.855644in}{0.377500in}}%
\pgfpathcurveto{\pgfqpoint{3.855644in}{0.371975in}}{\pgfqpoint{3.857839in}{0.366675in}}{\pgfqpoint{3.861746in}{0.362769in}}%
\pgfpathcurveto{\pgfqpoint{3.865653in}{0.358862in}}{\pgfqpoint{3.870952in}{0.356667in}}{\pgfqpoint{3.876477in}{0.356667in}}%
\pgfpathclose%
\pgfusepath{stroke,fill}%
\end{pgfscope}%
\begin{pgfscope}%
\pgfpathrectangle{\pgfqpoint{0.562500in}{0.275000in}}{\pgfqpoint{3.487500in}{1.925000in}}%
\pgfusepath{clip}%
\pgfsetbuttcap%
\pgfsetroundjoin%
\definecolor{currentfill}{rgb}{0.000000,0.000000,0.000000}%
\pgfsetfillcolor{currentfill}%
\pgfsetlinewidth{1.003750pt}%
\definecolor{currentstroke}{rgb}{0.000000,0.000000,0.000000}%
\pgfsetstrokecolor{currentstroke}%
\pgfsetdash{}{0pt}%
\pgfpathmoveto{\pgfqpoint{3.876477in}{0.356667in}}%
\pgfpathcurveto{\pgfqpoint{3.882002in}{0.356667in}}{\pgfqpoint{3.887302in}{0.358862in}}{\pgfqpoint{3.891209in}{0.362769in}}%
\pgfpathcurveto{\pgfqpoint{3.895115in}{0.366675in}}{\pgfqpoint{3.897311in}{0.371975in}}{\pgfqpoint{3.897311in}{0.377500in}}%
\pgfpathcurveto{\pgfqpoint{3.897311in}{0.383025in}}{\pgfqpoint{3.895115in}{0.388325in}}{\pgfqpoint{3.891209in}{0.392231in}}%
\pgfpathcurveto{\pgfqpoint{3.887302in}{0.396138in}}{\pgfqpoint{3.882002in}{0.398333in}}{\pgfqpoint{3.876477in}{0.398333in}}%
\pgfpathcurveto{\pgfqpoint{3.870952in}{0.398333in}}{\pgfqpoint{3.865653in}{0.396138in}}{\pgfqpoint{3.861746in}{0.392231in}}%
\pgfpathcurveto{\pgfqpoint{3.857839in}{0.388325in}}{\pgfqpoint{3.855644in}{0.383025in}}{\pgfqpoint{3.855644in}{0.377500in}}%
\pgfpathcurveto{\pgfqpoint{3.855644in}{0.371975in}}{\pgfqpoint{3.857839in}{0.366675in}}{\pgfqpoint{3.861746in}{0.362769in}}%
\pgfpathcurveto{\pgfqpoint{3.865653in}{0.358862in}}{\pgfqpoint{3.870952in}{0.356667in}}{\pgfqpoint{3.876477in}{0.356667in}}%
\pgfpathclose%
\pgfusepath{stroke,fill}%
\end{pgfscope}%
\begin{pgfscope}%
\pgfpathrectangle{\pgfqpoint{0.562500in}{0.275000in}}{\pgfqpoint{3.487500in}{1.925000in}}%
\pgfusepath{clip}%
\pgfsetbuttcap%
\pgfsetroundjoin%
\definecolor{currentfill}{rgb}{0.000000,0.000000,0.000000}%
\pgfsetfillcolor{currentfill}%
\pgfsetlinewidth{1.003750pt}%
\definecolor{currentstroke}{rgb}{0.000000,0.000000,0.000000}%
\pgfsetstrokecolor{currentstroke}%
\pgfsetdash{}{0pt}%
\pgfpathmoveto{\pgfqpoint{3.876477in}{0.356667in}}%
\pgfpathcurveto{\pgfqpoint{3.882002in}{0.356667in}}{\pgfqpoint{3.887302in}{0.358862in}}{\pgfqpoint{3.891209in}{0.362769in}}%
\pgfpathcurveto{\pgfqpoint{3.895115in}{0.366675in}}{\pgfqpoint{3.897311in}{0.371975in}}{\pgfqpoint{3.897311in}{0.377500in}}%
\pgfpathcurveto{\pgfqpoint{3.897311in}{0.383025in}}{\pgfqpoint{3.895115in}{0.388325in}}{\pgfqpoint{3.891209in}{0.392231in}}%
\pgfpathcurveto{\pgfqpoint{3.887302in}{0.396138in}}{\pgfqpoint{3.882002in}{0.398333in}}{\pgfqpoint{3.876477in}{0.398333in}}%
\pgfpathcurveto{\pgfqpoint{3.870952in}{0.398333in}}{\pgfqpoint{3.865653in}{0.396138in}}{\pgfqpoint{3.861746in}{0.392231in}}%
\pgfpathcurveto{\pgfqpoint{3.857839in}{0.388325in}}{\pgfqpoint{3.855644in}{0.383025in}}{\pgfqpoint{3.855644in}{0.377500in}}%
\pgfpathcurveto{\pgfqpoint{3.855644in}{0.371975in}}{\pgfqpoint{3.857839in}{0.366675in}}{\pgfqpoint{3.861746in}{0.362769in}}%
\pgfpathcurveto{\pgfqpoint{3.865653in}{0.358862in}}{\pgfqpoint{3.870952in}{0.356667in}}{\pgfqpoint{3.876477in}{0.356667in}}%
\pgfpathclose%
\pgfusepath{stroke,fill}%
\end{pgfscope}%
\begin{pgfscope}%
\pgfpathrectangle{\pgfqpoint{0.562500in}{0.275000in}}{\pgfqpoint{3.487500in}{1.925000in}}%
\pgfusepath{clip}%
\pgfsetbuttcap%
\pgfsetroundjoin%
\definecolor{currentfill}{rgb}{0.000000,0.000000,0.000000}%
\pgfsetfillcolor{currentfill}%
\pgfsetlinewidth{1.003750pt}%
\definecolor{currentstroke}{rgb}{0.000000,0.000000,0.000000}%
\pgfsetstrokecolor{currentstroke}%
\pgfsetdash{}{0pt}%
\pgfpathmoveto{\pgfqpoint{3.876477in}{0.356667in}}%
\pgfpathcurveto{\pgfqpoint{3.882002in}{0.356667in}}{\pgfqpoint{3.887302in}{0.358862in}}{\pgfqpoint{3.891209in}{0.362769in}}%
\pgfpathcurveto{\pgfqpoint{3.895115in}{0.366675in}}{\pgfqpoint{3.897311in}{0.371975in}}{\pgfqpoint{3.897311in}{0.377500in}}%
\pgfpathcurveto{\pgfqpoint{3.897311in}{0.383025in}}{\pgfqpoint{3.895115in}{0.388325in}}{\pgfqpoint{3.891209in}{0.392231in}}%
\pgfpathcurveto{\pgfqpoint{3.887302in}{0.396138in}}{\pgfqpoint{3.882002in}{0.398333in}}{\pgfqpoint{3.876477in}{0.398333in}}%
\pgfpathcurveto{\pgfqpoint{3.870952in}{0.398333in}}{\pgfqpoint{3.865653in}{0.396138in}}{\pgfqpoint{3.861746in}{0.392231in}}%
\pgfpathcurveto{\pgfqpoint{3.857839in}{0.388325in}}{\pgfqpoint{3.855644in}{0.383025in}}{\pgfqpoint{3.855644in}{0.377500in}}%
\pgfpathcurveto{\pgfqpoint{3.855644in}{0.371975in}}{\pgfqpoint{3.857839in}{0.366675in}}{\pgfqpoint{3.861746in}{0.362769in}}%
\pgfpathcurveto{\pgfqpoint{3.865653in}{0.358862in}}{\pgfqpoint{3.870952in}{0.356667in}}{\pgfqpoint{3.876477in}{0.356667in}}%
\pgfpathclose%
\pgfusepath{stroke,fill}%
\end{pgfscope}%
\begin{pgfscope}%
\pgfpathrectangle{\pgfqpoint{0.562500in}{0.275000in}}{\pgfqpoint{3.487500in}{1.925000in}}%
\pgfusepath{clip}%
\pgfsetbuttcap%
\pgfsetroundjoin%
\definecolor{currentfill}{rgb}{0.000000,0.000000,0.000000}%
\pgfsetfillcolor{currentfill}%
\pgfsetlinewidth{1.003750pt}%
\definecolor{currentstroke}{rgb}{0.000000,0.000000,0.000000}%
\pgfsetstrokecolor{currentstroke}%
\pgfsetdash{}{0pt}%
\pgfpathmoveto{\pgfqpoint{3.876477in}{0.356667in}}%
\pgfpathcurveto{\pgfqpoint{3.882002in}{0.356667in}}{\pgfqpoint{3.887302in}{0.358862in}}{\pgfqpoint{3.891209in}{0.362769in}}%
\pgfpathcurveto{\pgfqpoint{3.895115in}{0.366675in}}{\pgfqpoint{3.897311in}{0.371975in}}{\pgfqpoint{3.897311in}{0.377500in}}%
\pgfpathcurveto{\pgfqpoint{3.897311in}{0.383025in}}{\pgfqpoint{3.895115in}{0.388325in}}{\pgfqpoint{3.891209in}{0.392231in}}%
\pgfpathcurveto{\pgfqpoint{3.887302in}{0.396138in}}{\pgfqpoint{3.882002in}{0.398333in}}{\pgfqpoint{3.876477in}{0.398333in}}%
\pgfpathcurveto{\pgfqpoint{3.870952in}{0.398333in}}{\pgfqpoint{3.865653in}{0.396138in}}{\pgfqpoint{3.861746in}{0.392231in}}%
\pgfpathcurveto{\pgfqpoint{3.857839in}{0.388325in}}{\pgfqpoint{3.855644in}{0.383025in}}{\pgfqpoint{3.855644in}{0.377500in}}%
\pgfpathcurveto{\pgfqpoint{3.855644in}{0.371975in}}{\pgfqpoint{3.857839in}{0.366675in}}{\pgfqpoint{3.861746in}{0.362769in}}%
\pgfpathcurveto{\pgfqpoint{3.865653in}{0.358862in}}{\pgfqpoint{3.870952in}{0.356667in}}{\pgfqpoint{3.876477in}{0.356667in}}%
\pgfpathclose%
\pgfusepath{stroke,fill}%
\end{pgfscope}%
\begin{pgfscope}%
\pgfpathrectangle{\pgfqpoint{0.562500in}{0.275000in}}{\pgfqpoint{3.487500in}{1.925000in}}%
\pgfusepath{clip}%
\pgfsetbuttcap%
\pgfsetroundjoin%
\definecolor{currentfill}{rgb}{0.000000,0.000000,0.000000}%
\pgfsetfillcolor{currentfill}%
\pgfsetlinewidth{1.003750pt}%
\definecolor{currentstroke}{rgb}{0.000000,0.000000,0.000000}%
\pgfsetstrokecolor{currentstroke}%
\pgfsetdash{}{0pt}%
\pgfpathmoveto{\pgfqpoint{3.876477in}{0.356667in}}%
\pgfpathcurveto{\pgfqpoint{3.882002in}{0.356667in}}{\pgfqpoint{3.887302in}{0.358862in}}{\pgfqpoint{3.891209in}{0.362769in}}%
\pgfpathcurveto{\pgfqpoint{3.895115in}{0.366675in}}{\pgfqpoint{3.897311in}{0.371975in}}{\pgfqpoint{3.897311in}{0.377500in}}%
\pgfpathcurveto{\pgfqpoint{3.897311in}{0.383025in}}{\pgfqpoint{3.895115in}{0.388325in}}{\pgfqpoint{3.891209in}{0.392231in}}%
\pgfpathcurveto{\pgfqpoint{3.887302in}{0.396138in}}{\pgfqpoint{3.882002in}{0.398333in}}{\pgfqpoint{3.876477in}{0.398333in}}%
\pgfpathcurveto{\pgfqpoint{3.870952in}{0.398333in}}{\pgfqpoint{3.865653in}{0.396138in}}{\pgfqpoint{3.861746in}{0.392231in}}%
\pgfpathcurveto{\pgfqpoint{3.857839in}{0.388325in}}{\pgfqpoint{3.855644in}{0.383025in}}{\pgfqpoint{3.855644in}{0.377500in}}%
\pgfpathcurveto{\pgfqpoint{3.855644in}{0.371975in}}{\pgfqpoint{3.857839in}{0.366675in}}{\pgfqpoint{3.861746in}{0.362769in}}%
\pgfpathcurveto{\pgfqpoint{3.865653in}{0.358862in}}{\pgfqpoint{3.870952in}{0.356667in}}{\pgfqpoint{3.876477in}{0.356667in}}%
\pgfpathclose%
\pgfusepath{stroke,fill}%
\end{pgfscope}%
\begin{pgfscope}%
\pgfpathrectangle{\pgfqpoint{0.562500in}{0.275000in}}{\pgfqpoint{3.487500in}{1.925000in}}%
\pgfusepath{clip}%
\pgfsetbuttcap%
\pgfsetroundjoin%
\definecolor{currentfill}{rgb}{0.000000,0.000000,0.000000}%
\pgfsetfillcolor{currentfill}%
\pgfsetlinewidth{1.003750pt}%
\definecolor{currentstroke}{rgb}{0.000000,0.000000,0.000000}%
\pgfsetstrokecolor{currentstroke}%
\pgfsetdash{}{0pt}%
\pgfpathmoveto{\pgfqpoint{3.876477in}{0.356667in}}%
\pgfpathcurveto{\pgfqpoint{3.882002in}{0.356667in}}{\pgfqpoint{3.887302in}{0.358862in}}{\pgfqpoint{3.891209in}{0.362769in}}%
\pgfpathcurveto{\pgfqpoint{3.895115in}{0.366675in}}{\pgfqpoint{3.897311in}{0.371975in}}{\pgfqpoint{3.897311in}{0.377500in}}%
\pgfpathcurveto{\pgfqpoint{3.897311in}{0.383025in}}{\pgfqpoint{3.895115in}{0.388325in}}{\pgfqpoint{3.891209in}{0.392231in}}%
\pgfpathcurveto{\pgfqpoint{3.887302in}{0.396138in}}{\pgfqpoint{3.882002in}{0.398333in}}{\pgfqpoint{3.876477in}{0.398333in}}%
\pgfpathcurveto{\pgfqpoint{3.870952in}{0.398333in}}{\pgfqpoint{3.865653in}{0.396138in}}{\pgfqpoint{3.861746in}{0.392231in}}%
\pgfpathcurveto{\pgfqpoint{3.857839in}{0.388325in}}{\pgfqpoint{3.855644in}{0.383025in}}{\pgfqpoint{3.855644in}{0.377500in}}%
\pgfpathcurveto{\pgfqpoint{3.855644in}{0.371975in}}{\pgfqpoint{3.857839in}{0.366675in}}{\pgfqpoint{3.861746in}{0.362769in}}%
\pgfpathcurveto{\pgfqpoint{3.865653in}{0.358862in}}{\pgfqpoint{3.870952in}{0.356667in}}{\pgfqpoint{3.876477in}{0.356667in}}%
\pgfpathclose%
\pgfusepath{stroke,fill}%
\end{pgfscope}%
\begin{pgfscope}%
\pgfpathrectangle{\pgfqpoint{0.562500in}{0.275000in}}{\pgfqpoint{3.487500in}{1.925000in}}%
\pgfusepath{clip}%
\pgfsetbuttcap%
\pgfsetroundjoin%
\definecolor{currentfill}{rgb}{0.000000,0.000000,0.000000}%
\pgfsetfillcolor{currentfill}%
\pgfsetlinewidth{1.003750pt}%
\definecolor{currentstroke}{rgb}{0.000000,0.000000,0.000000}%
\pgfsetstrokecolor{currentstroke}%
\pgfsetdash{}{0pt}%
\pgfpathmoveto{\pgfqpoint{3.876477in}{0.356667in}}%
\pgfpathcurveto{\pgfqpoint{3.882002in}{0.356667in}}{\pgfqpoint{3.887302in}{0.358862in}}{\pgfqpoint{3.891209in}{0.362769in}}%
\pgfpathcurveto{\pgfqpoint{3.895115in}{0.366675in}}{\pgfqpoint{3.897311in}{0.371975in}}{\pgfqpoint{3.897311in}{0.377500in}}%
\pgfpathcurveto{\pgfqpoint{3.897311in}{0.383025in}}{\pgfqpoint{3.895115in}{0.388325in}}{\pgfqpoint{3.891209in}{0.392231in}}%
\pgfpathcurveto{\pgfqpoint{3.887302in}{0.396138in}}{\pgfqpoint{3.882002in}{0.398333in}}{\pgfqpoint{3.876477in}{0.398333in}}%
\pgfpathcurveto{\pgfqpoint{3.870952in}{0.398333in}}{\pgfqpoint{3.865653in}{0.396138in}}{\pgfqpoint{3.861746in}{0.392231in}}%
\pgfpathcurveto{\pgfqpoint{3.857839in}{0.388325in}}{\pgfqpoint{3.855644in}{0.383025in}}{\pgfqpoint{3.855644in}{0.377500in}}%
\pgfpathcurveto{\pgfqpoint{3.855644in}{0.371975in}}{\pgfqpoint{3.857839in}{0.366675in}}{\pgfqpoint{3.861746in}{0.362769in}}%
\pgfpathcurveto{\pgfqpoint{3.865653in}{0.358862in}}{\pgfqpoint{3.870952in}{0.356667in}}{\pgfqpoint{3.876477in}{0.356667in}}%
\pgfpathclose%
\pgfusepath{stroke,fill}%
\end{pgfscope}%
\begin{pgfscope}%
\pgfpathrectangle{\pgfqpoint{0.562500in}{0.275000in}}{\pgfqpoint{3.487500in}{1.925000in}}%
\pgfusepath{clip}%
\pgfsetbuttcap%
\pgfsetroundjoin%
\definecolor{currentfill}{rgb}{0.000000,0.000000,0.000000}%
\pgfsetfillcolor{currentfill}%
\pgfsetlinewidth{1.003750pt}%
\definecolor{currentstroke}{rgb}{0.000000,0.000000,0.000000}%
\pgfsetstrokecolor{currentstroke}%
\pgfsetdash{}{0pt}%
\pgfpathmoveto{\pgfqpoint{3.876477in}{0.356667in}}%
\pgfpathcurveto{\pgfqpoint{3.882002in}{0.356667in}}{\pgfqpoint{3.887302in}{0.358862in}}{\pgfqpoint{3.891209in}{0.362769in}}%
\pgfpathcurveto{\pgfqpoint{3.895115in}{0.366675in}}{\pgfqpoint{3.897311in}{0.371975in}}{\pgfqpoint{3.897311in}{0.377500in}}%
\pgfpathcurveto{\pgfqpoint{3.897311in}{0.383025in}}{\pgfqpoint{3.895115in}{0.388325in}}{\pgfqpoint{3.891209in}{0.392231in}}%
\pgfpathcurveto{\pgfqpoint{3.887302in}{0.396138in}}{\pgfqpoint{3.882002in}{0.398333in}}{\pgfqpoint{3.876477in}{0.398333in}}%
\pgfpathcurveto{\pgfqpoint{3.870952in}{0.398333in}}{\pgfqpoint{3.865653in}{0.396138in}}{\pgfqpoint{3.861746in}{0.392231in}}%
\pgfpathcurveto{\pgfqpoint{3.857839in}{0.388325in}}{\pgfqpoint{3.855644in}{0.383025in}}{\pgfqpoint{3.855644in}{0.377500in}}%
\pgfpathcurveto{\pgfqpoint{3.855644in}{0.371975in}}{\pgfqpoint{3.857839in}{0.366675in}}{\pgfqpoint{3.861746in}{0.362769in}}%
\pgfpathcurveto{\pgfqpoint{3.865653in}{0.358862in}}{\pgfqpoint{3.870952in}{0.356667in}}{\pgfqpoint{3.876477in}{0.356667in}}%
\pgfpathclose%
\pgfusepath{stroke,fill}%
\end{pgfscope}%
\begin{pgfscope}%
\pgfpathrectangle{\pgfqpoint{0.562500in}{0.275000in}}{\pgfqpoint{3.487500in}{1.925000in}}%
\pgfusepath{clip}%
\pgfsetbuttcap%
\pgfsetroundjoin%
\definecolor{currentfill}{rgb}{0.000000,0.000000,0.000000}%
\pgfsetfillcolor{currentfill}%
\pgfsetlinewidth{1.003750pt}%
\definecolor{currentstroke}{rgb}{0.000000,0.000000,0.000000}%
\pgfsetstrokecolor{currentstroke}%
\pgfsetdash{}{0pt}%
\pgfpathmoveto{\pgfqpoint{3.876477in}{0.356667in}}%
\pgfpathcurveto{\pgfqpoint{3.882002in}{0.356667in}}{\pgfqpoint{3.887302in}{0.358862in}}{\pgfqpoint{3.891209in}{0.362769in}}%
\pgfpathcurveto{\pgfqpoint{3.895115in}{0.366675in}}{\pgfqpoint{3.897311in}{0.371975in}}{\pgfqpoint{3.897311in}{0.377500in}}%
\pgfpathcurveto{\pgfqpoint{3.897311in}{0.383025in}}{\pgfqpoint{3.895115in}{0.388325in}}{\pgfqpoint{3.891209in}{0.392231in}}%
\pgfpathcurveto{\pgfqpoint{3.887302in}{0.396138in}}{\pgfqpoint{3.882002in}{0.398333in}}{\pgfqpoint{3.876477in}{0.398333in}}%
\pgfpathcurveto{\pgfqpoint{3.870952in}{0.398333in}}{\pgfqpoint{3.865653in}{0.396138in}}{\pgfqpoint{3.861746in}{0.392231in}}%
\pgfpathcurveto{\pgfqpoint{3.857839in}{0.388325in}}{\pgfqpoint{3.855644in}{0.383025in}}{\pgfqpoint{3.855644in}{0.377500in}}%
\pgfpathcurveto{\pgfqpoint{3.855644in}{0.371975in}}{\pgfqpoint{3.857839in}{0.366675in}}{\pgfqpoint{3.861746in}{0.362769in}}%
\pgfpathcurveto{\pgfqpoint{3.865653in}{0.358862in}}{\pgfqpoint{3.870952in}{0.356667in}}{\pgfqpoint{3.876477in}{0.356667in}}%
\pgfpathclose%
\pgfusepath{stroke,fill}%
\end{pgfscope}%
\begin{pgfscope}%
\pgfpathrectangle{\pgfqpoint{0.562500in}{0.275000in}}{\pgfqpoint{3.487500in}{1.925000in}}%
\pgfusepath{clip}%
\pgfsetbuttcap%
\pgfsetroundjoin%
\definecolor{currentfill}{rgb}{0.000000,0.000000,0.000000}%
\pgfsetfillcolor{currentfill}%
\pgfsetlinewidth{1.003750pt}%
\definecolor{currentstroke}{rgb}{0.000000,0.000000,0.000000}%
\pgfsetstrokecolor{currentstroke}%
\pgfsetdash{}{0pt}%
\pgfpathmoveto{\pgfqpoint{3.876477in}{0.356667in}}%
\pgfpathcurveto{\pgfqpoint{3.882002in}{0.356667in}}{\pgfqpoint{3.887302in}{0.358862in}}{\pgfqpoint{3.891209in}{0.362769in}}%
\pgfpathcurveto{\pgfqpoint{3.895115in}{0.366675in}}{\pgfqpoint{3.897311in}{0.371975in}}{\pgfqpoint{3.897311in}{0.377500in}}%
\pgfpathcurveto{\pgfqpoint{3.897311in}{0.383025in}}{\pgfqpoint{3.895115in}{0.388325in}}{\pgfqpoint{3.891209in}{0.392231in}}%
\pgfpathcurveto{\pgfqpoint{3.887302in}{0.396138in}}{\pgfqpoint{3.882002in}{0.398333in}}{\pgfqpoint{3.876477in}{0.398333in}}%
\pgfpathcurveto{\pgfqpoint{3.870952in}{0.398333in}}{\pgfqpoint{3.865653in}{0.396138in}}{\pgfqpoint{3.861746in}{0.392231in}}%
\pgfpathcurveto{\pgfqpoint{3.857839in}{0.388325in}}{\pgfqpoint{3.855644in}{0.383025in}}{\pgfqpoint{3.855644in}{0.377500in}}%
\pgfpathcurveto{\pgfqpoint{3.855644in}{0.371975in}}{\pgfqpoint{3.857839in}{0.366675in}}{\pgfqpoint{3.861746in}{0.362769in}}%
\pgfpathcurveto{\pgfqpoint{3.865653in}{0.358862in}}{\pgfqpoint{3.870952in}{0.356667in}}{\pgfqpoint{3.876477in}{0.356667in}}%
\pgfpathclose%
\pgfusepath{stroke,fill}%
\end{pgfscope}%
\begin{pgfscope}%
\pgfpathrectangle{\pgfqpoint{0.562500in}{0.275000in}}{\pgfqpoint{3.487500in}{1.925000in}}%
\pgfusepath{clip}%
\pgfsetbuttcap%
\pgfsetroundjoin%
\definecolor{currentfill}{rgb}{0.000000,0.000000,0.000000}%
\pgfsetfillcolor{currentfill}%
\pgfsetlinewidth{1.003750pt}%
\definecolor{currentstroke}{rgb}{0.000000,0.000000,0.000000}%
\pgfsetstrokecolor{currentstroke}%
\pgfsetdash{}{0pt}%
\pgfpathmoveto{\pgfqpoint{3.876477in}{0.356667in}}%
\pgfpathcurveto{\pgfqpoint{3.882002in}{0.356667in}}{\pgfqpoint{3.887302in}{0.358862in}}{\pgfqpoint{3.891209in}{0.362769in}}%
\pgfpathcurveto{\pgfqpoint{3.895115in}{0.366675in}}{\pgfqpoint{3.897311in}{0.371975in}}{\pgfqpoint{3.897311in}{0.377500in}}%
\pgfpathcurveto{\pgfqpoint{3.897311in}{0.383025in}}{\pgfqpoint{3.895115in}{0.388325in}}{\pgfqpoint{3.891209in}{0.392231in}}%
\pgfpathcurveto{\pgfqpoint{3.887302in}{0.396138in}}{\pgfqpoint{3.882002in}{0.398333in}}{\pgfqpoint{3.876477in}{0.398333in}}%
\pgfpathcurveto{\pgfqpoint{3.870952in}{0.398333in}}{\pgfqpoint{3.865653in}{0.396138in}}{\pgfqpoint{3.861746in}{0.392231in}}%
\pgfpathcurveto{\pgfqpoint{3.857839in}{0.388325in}}{\pgfqpoint{3.855644in}{0.383025in}}{\pgfqpoint{3.855644in}{0.377500in}}%
\pgfpathcurveto{\pgfqpoint{3.855644in}{0.371975in}}{\pgfqpoint{3.857839in}{0.366675in}}{\pgfqpoint{3.861746in}{0.362769in}}%
\pgfpathcurveto{\pgfqpoint{3.865653in}{0.358862in}}{\pgfqpoint{3.870952in}{0.356667in}}{\pgfqpoint{3.876477in}{0.356667in}}%
\pgfpathclose%
\pgfusepath{stroke,fill}%
\end{pgfscope}%
\begin{pgfscope}%
\pgfpathrectangle{\pgfqpoint{0.562500in}{0.275000in}}{\pgfqpoint{3.487500in}{1.925000in}}%
\pgfusepath{clip}%
\pgfsetbuttcap%
\pgfsetroundjoin%
\definecolor{currentfill}{rgb}{0.000000,0.000000,0.000000}%
\pgfsetfillcolor{currentfill}%
\pgfsetlinewidth{1.003750pt}%
\definecolor{currentstroke}{rgb}{0.000000,0.000000,0.000000}%
\pgfsetstrokecolor{currentstroke}%
\pgfsetdash{}{0pt}%
\pgfpathmoveto{\pgfqpoint{3.876477in}{0.356667in}}%
\pgfpathcurveto{\pgfqpoint{3.882002in}{0.356667in}}{\pgfqpoint{3.887302in}{0.358862in}}{\pgfqpoint{3.891209in}{0.362769in}}%
\pgfpathcurveto{\pgfqpoint{3.895115in}{0.366675in}}{\pgfqpoint{3.897311in}{0.371975in}}{\pgfqpoint{3.897311in}{0.377500in}}%
\pgfpathcurveto{\pgfqpoint{3.897311in}{0.383025in}}{\pgfqpoint{3.895115in}{0.388325in}}{\pgfqpoint{3.891209in}{0.392231in}}%
\pgfpathcurveto{\pgfqpoint{3.887302in}{0.396138in}}{\pgfqpoint{3.882002in}{0.398333in}}{\pgfqpoint{3.876477in}{0.398333in}}%
\pgfpathcurveto{\pgfqpoint{3.870952in}{0.398333in}}{\pgfqpoint{3.865653in}{0.396138in}}{\pgfqpoint{3.861746in}{0.392231in}}%
\pgfpathcurveto{\pgfqpoint{3.857839in}{0.388325in}}{\pgfqpoint{3.855644in}{0.383025in}}{\pgfqpoint{3.855644in}{0.377500in}}%
\pgfpathcurveto{\pgfqpoint{3.855644in}{0.371975in}}{\pgfqpoint{3.857839in}{0.366675in}}{\pgfqpoint{3.861746in}{0.362769in}}%
\pgfpathcurveto{\pgfqpoint{3.865653in}{0.358862in}}{\pgfqpoint{3.870952in}{0.356667in}}{\pgfqpoint{3.876477in}{0.356667in}}%
\pgfpathclose%
\pgfusepath{stroke,fill}%
\end{pgfscope}%
\begin{pgfscope}%
\pgfpathrectangle{\pgfqpoint{0.562500in}{0.275000in}}{\pgfqpoint{3.487500in}{1.925000in}}%
\pgfusepath{clip}%
\pgfsetbuttcap%
\pgfsetroundjoin%
\definecolor{currentfill}{rgb}{0.000000,0.000000,0.000000}%
\pgfsetfillcolor{currentfill}%
\pgfsetlinewidth{1.003750pt}%
\definecolor{currentstroke}{rgb}{0.000000,0.000000,0.000000}%
\pgfsetstrokecolor{currentstroke}%
\pgfsetdash{}{0pt}%
\pgfpathmoveto{\pgfqpoint{3.876477in}{0.356667in}}%
\pgfpathcurveto{\pgfqpoint{3.882002in}{0.356667in}}{\pgfqpoint{3.887302in}{0.358862in}}{\pgfqpoint{3.891209in}{0.362769in}}%
\pgfpathcurveto{\pgfqpoint{3.895115in}{0.366675in}}{\pgfqpoint{3.897311in}{0.371975in}}{\pgfqpoint{3.897311in}{0.377500in}}%
\pgfpathcurveto{\pgfqpoint{3.897311in}{0.383025in}}{\pgfqpoint{3.895115in}{0.388325in}}{\pgfqpoint{3.891209in}{0.392231in}}%
\pgfpathcurveto{\pgfqpoint{3.887302in}{0.396138in}}{\pgfqpoint{3.882002in}{0.398333in}}{\pgfqpoint{3.876477in}{0.398333in}}%
\pgfpathcurveto{\pgfqpoint{3.870952in}{0.398333in}}{\pgfqpoint{3.865653in}{0.396138in}}{\pgfqpoint{3.861746in}{0.392231in}}%
\pgfpathcurveto{\pgfqpoint{3.857839in}{0.388325in}}{\pgfqpoint{3.855644in}{0.383025in}}{\pgfqpoint{3.855644in}{0.377500in}}%
\pgfpathcurveto{\pgfqpoint{3.855644in}{0.371975in}}{\pgfqpoint{3.857839in}{0.366675in}}{\pgfqpoint{3.861746in}{0.362769in}}%
\pgfpathcurveto{\pgfqpoint{3.865653in}{0.358862in}}{\pgfqpoint{3.870952in}{0.356667in}}{\pgfqpoint{3.876477in}{0.356667in}}%
\pgfpathclose%
\pgfusepath{stroke,fill}%
\end{pgfscope}%
\begin{pgfscope}%
\pgfsetbuttcap%
\pgfsetroundjoin%
\definecolor{currentfill}{rgb}{0.000000,0.000000,0.000000}%
\pgfsetfillcolor{currentfill}%
\pgfsetlinewidth{0.803000pt}%
\definecolor{currentstroke}{rgb}{0.000000,0.000000,0.000000}%
\pgfsetstrokecolor{currentstroke}%
\pgfsetdash{}{0pt}%
\pgfsys@defobject{currentmarker}{\pgfqpoint{0.000000in}{-0.048611in}}{\pgfqpoint{0.000000in}{0.000000in}}{%
\pgfpathmoveto{\pgfqpoint{0.000000in}{0.000000in}}%
\pgfpathlineto{\pgfqpoint{0.000000in}{-0.048611in}}%
\pgfusepath{stroke,fill}%
}%
\begin{pgfscope}%
\pgfsys@transformshift{0.721249in}{0.275000in}%
\pgfsys@useobject{currentmarker}{}%
\end{pgfscope}%
\end{pgfscope}%
\begin{pgfscope}%
\definecolor{textcolor}{rgb}{0.000000,0.000000,0.000000}%
\pgfsetstrokecolor{textcolor}%
\pgfsetfillcolor{textcolor}%
\pgftext[x=0.721249in,y=0.177778in,,top]{\color{textcolor}\sffamily\fontsize{10.000000}{12.000000}\selectfont 20}%
\end{pgfscope}%
\begin{pgfscope}%
\pgfsetbuttcap%
\pgfsetroundjoin%
\definecolor{currentfill}{rgb}{0.000000,0.000000,0.000000}%
\pgfsetfillcolor{currentfill}%
\pgfsetlinewidth{0.803000pt}%
\definecolor{currentstroke}{rgb}{0.000000,0.000000,0.000000}%
\pgfsetstrokecolor{currentstroke}%
\pgfsetdash{}{0pt}%
\pgfsys@defobject{currentmarker}{\pgfqpoint{0.000000in}{-0.048611in}}{\pgfqpoint{0.000000in}{0.000000in}}{%
\pgfpathmoveto{\pgfqpoint{0.000000in}{0.000000in}}%
\pgfpathlineto{\pgfqpoint{0.000000in}{-0.048611in}}%
\pgfusepath{stroke,fill}%
}%
\begin{pgfscope}%
\pgfsys@transformshift{1.772992in}{0.275000in}%
\pgfsys@useobject{currentmarker}{}%
\end{pgfscope}%
\end{pgfscope}%
\begin{pgfscope}%
\definecolor{textcolor}{rgb}{0.000000,0.000000,0.000000}%
\pgfsetstrokecolor{textcolor}%
\pgfsetfillcolor{textcolor}%
\pgftext[x=1.772992in,y=0.177778in,,top]{\color{textcolor}\sffamily\fontsize{10.000000}{12.000000}\selectfont 40}%
\end{pgfscope}%
\begin{pgfscope}%
\pgfsetbuttcap%
\pgfsetroundjoin%
\definecolor{currentfill}{rgb}{0.000000,0.000000,0.000000}%
\pgfsetfillcolor{currentfill}%
\pgfsetlinewidth{0.803000pt}%
\definecolor{currentstroke}{rgb}{0.000000,0.000000,0.000000}%
\pgfsetstrokecolor{currentstroke}%
\pgfsetdash{}{0pt}%
\pgfsys@defobject{currentmarker}{\pgfqpoint{0.000000in}{-0.048611in}}{\pgfqpoint{0.000000in}{0.000000in}}{%
\pgfpathmoveto{\pgfqpoint{0.000000in}{0.000000in}}%
\pgfpathlineto{\pgfqpoint{0.000000in}{-0.048611in}}%
\pgfusepath{stroke,fill}%
}%
\begin{pgfscope}%
\pgfsys@transformshift{2.824734in}{0.275000in}%
\pgfsys@useobject{currentmarker}{}%
\end{pgfscope}%
\end{pgfscope}%
\begin{pgfscope}%
\definecolor{textcolor}{rgb}{0.000000,0.000000,0.000000}%
\pgfsetstrokecolor{textcolor}%
\pgfsetfillcolor{textcolor}%
\pgftext[x=2.824734in,y=0.177778in,,top]{\color{textcolor}\sffamily\fontsize{10.000000}{12.000000}\selectfont 60}%
\end{pgfscope}%
\begin{pgfscope}%
\pgfsetbuttcap%
\pgfsetroundjoin%
\definecolor{currentfill}{rgb}{0.000000,0.000000,0.000000}%
\pgfsetfillcolor{currentfill}%
\pgfsetlinewidth{0.803000pt}%
\definecolor{currentstroke}{rgb}{0.000000,0.000000,0.000000}%
\pgfsetstrokecolor{currentstroke}%
\pgfsetdash{}{0pt}%
\pgfsys@defobject{currentmarker}{\pgfqpoint{0.000000in}{-0.048611in}}{\pgfqpoint{0.000000in}{0.000000in}}{%
\pgfpathmoveto{\pgfqpoint{0.000000in}{0.000000in}}%
\pgfpathlineto{\pgfqpoint{0.000000in}{-0.048611in}}%
\pgfusepath{stroke,fill}%
}%
\begin{pgfscope}%
\pgfsys@transformshift{3.876477in}{0.275000in}%
\pgfsys@useobject{currentmarker}{}%
\end{pgfscope}%
\end{pgfscope}%
\begin{pgfscope}%
\definecolor{textcolor}{rgb}{0.000000,0.000000,0.000000}%
\pgfsetstrokecolor{textcolor}%
\pgfsetfillcolor{textcolor}%
\pgftext[x=3.876477in,y=0.177778in,,top]{\color{textcolor}\sffamily\fontsize{10.000000}{12.000000}\selectfont 80}%
\end{pgfscope}%
\begin{pgfscope}%
\definecolor{textcolor}{rgb}{0.000000,0.000000,0.000000}%
\pgfsetstrokecolor{textcolor}%
\pgfsetfillcolor{textcolor}%
\pgftext[x=2.306250in,y=-0.012191in,,top]{\color{textcolor}\sffamily\fontsize{10.000000}{12.000000}\selectfont \(\displaystyle k\)}%
\end{pgfscope}%
\begin{pgfscope}%
\pgfsetbuttcap%
\pgfsetroundjoin%
\definecolor{currentfill}{rgb}{0.000000,0.000000,0.000000}%
\pgfsetfillcolor{currentfill}%
\pgfsetlinewidth{0.803000pt}%
\definecolor{currentstroke}{rgb}{0.000000,0.000000,0.000000}%
\pgfsetstrokecolor{currentstroke}%
\pgfsetdash{}{0pt}%
\pgfsys@defobject{currentmarker}{\pgfqpoint{-0.048611in}{0.000000in}}{\pgfqpoint{0.000000in}{0.000000in}}{%
\pgfpathmoveto{\pgfqpoint{0.000000in}{0.000000in}}%
\pgfpathlineto{\pgfqpoint{-0.048611in}{0.000000in}}%
\pgfusepath{stroke,fill}%
}%
\begin{pgfscope}%
\pgfsys@transformshift{0.562500in}{0.377500in}%
\pgfsys@useobject{currentmarker}{}%
\end{pgfscope}%
\end{pgfscope}%
\begin{pgfscope}%
\definecolor{textcolor}{rgb}{0.000000,0.000000,0.000000}%
\pgfsetstrokecolor{textcolor}%
\pgfsetfillcolor{textcolor}%
\pgftext[x=0.376912in,y=0.324738in,left,base]{\color{textcolor}\sffamily\fontsize{10.000000}{12.000000}\selectfont 5}%
\end{pgfscope}%
\begin{pgfscope}%
\pgfsetbuttcap%
\pgfsetroundjoin%
\definecolor{currentfill}{rgb}{0.000000,0.000000,0.000000}%
\pgfsetfillcolor{currentfill}%
\pgfsetlinewidth{0.803000pt}%
\definecolor{currentstroke}{rgb}{0.000000,0.000000,0.000000}%
\pgfsetstrokecolor{currentstroke}%
\pgfsetdash{}{0pt}%
\pgfsys@defobject{currentmarker}{\pgfqpoint{-0.048611in}{0.000000in}}{\pgfqpoint{0.000000in}{0.000000in}}{%
\pgfpathmoveto{\pgfqpoint{0.000000in}{0.000000in}}%
\pgfpathlineto{\pgfqpoint{-0.048611in}{0.000000in}}%
\pgfusepath{stroke,fill}%
}%
\begin{pgfscope}%
\pgfsys@transformshift{0.562500in}{2.097500in}%
\pgfsys@useobject{currentmarker}{}%
\end{pgfscope}%
\end{pgfscope}%
\begin{pgfscope}%
\definecolor{textcolor}{rgb}{0.000000,0.000000,0.000000}%
\pgfsetstrokecolor{textcolor}%
\pgfsetfillcolor{textcolor}%
\pgftext[x=0.376912in,y=2.044738in,left,base]{\color{textcolor}\sffamily\fontsize{10.000000}{12.000000}\selectfont 6}%
\end{pgfscope}%
\begin{pgfscope}%
\definecolor{textcolor}{rgb}{0.000000,0.000000,0.000000}%
\pgfsetstrokecolor{textcolor}%
\pgfsetfillcolor{textcolor}%
\pgftext[x=0.321357in,y=1.237500in,,bottom,rotate=90.000000]{\color{textcolor}\sffamily\fontsize{10.000000}{12.000000}\selectfont Number of GMRES Iterations}%
\end{pgfscope}%
\begin{pgfscope}%
\pgfsetrectcap%
\pgfsetmiterjoin%
\pgfsetlinewidth{0.803000pt}%
\definecolor{currentstroke}{rgb}{0.000000,0.000000,0.000000}%
\pgfsetstrokecolor{currentstroke}%
\pgfsetdash{}{0pt}%
\pgfpathmoveto{\pgfqpoint{0.562500in}{0.275000in}}%
\pgfpathlineto{\pgfqpoint{0.562500in}{2.200000in}}%
\pgfusepath{stroke}%
\end{pgfscope}%
\begin{pgfscope}%
\pgfsetrectcap%
\pgfsetmiterjoin%
\pgfsetlinewidth{0.803000pt}%
\definecolor{currentstroke}{rgb}{0.000000,0.000000,0.000000}%
\pgfsetstrokecolor{currentstroke}%
\pgfsetdash{}{0pt}%
\pgfpathmoveto{\pgfqpoint{4.050000in}{0.275000in}}%
\pgfpathlineto{\pgfqpoint{4.050000in}{2.200000in}}%
\pgfusepath{stroke}%
\end{pgfscope}%
\begin{pgfscope}%
\pgfsetrectcap%
\pgfsetmiterjoin%
\pgfsetlinewidth{0.803000pt}%
\definecolor{currentstroke}{rgb}{0.000000,0.000000,0.000000}%
\pgfsetstrokecolor{currentstroke}%
\pgfsetdash{}{0pt}%
\pgfpathmoveto{\pgfqpoint{0.562500in}{0.275000in}}%
\pgfpathlineto{\pgfqpoint{4.050000in}{0.275000in}}%
\pgfusepath{stroke}%
\end{pgfscope}%
\begin{pgfscope}%
\pgfsetrectcap%
\pgfsetmiterjoin%
\pgfsetlinewidth{0.803000pt}%
\definecolor{currentstroke}{rgb}{0.000000,0.000000,0.000000}%
\pgfsetstrokecolor{currentstroke}%
\pgfsetdash{}{0pt}%
\pgfpathmoveto{\pgfqpoint{0.562500in}{2.200000in}}%
\pgfpathlineto{\pgfqpoint{4.050000in}{2.200000in}}%
\pgfusepath{stroke}%
\end{pgfscope}%
\end{pgfpicture}%
\makeatother%
\endgroup%

  \caption{GMRES iteration counts for $\alpha = 0.5/k$}\label{fig:linfinityA2}
\end{subfigure}
\caption{GMRES iteration counts for $\AmatoI\Amatt$ where $\nso=\nst=1$ and $\NLiDRRRdtd{\Aso-\Ast} = \alpha$ as described in \cref{sec:num}.}
\end{figure}

  \begin{figure}
    \centering
    \begin{subfigure}{\textwidth}
      \centering
%% Creator: Matplotlib, PGF backend
%%
%% To include the figure in your LaTeX document, write
%%   \input{<filename>.pgf}
%%
%% Make sure the required packages are loaded in your preamble
%%   \usepackage{pgf}
%%
%% Figures using additional raster images can only be included by \input if
%% they are in the same directory as the main LaTeX file. For loading figures
%% from other directories you can use the `import` package
%%   \usepackage{import}
%% and then include the figures with
%%   \import{<path to file>}{<filename>.pgf}
%%
%% Matplotlib used the following preamble
%%   \usepackage{fontspec}
%%   \setmainfont{DejaVuSerif.ttf}[Path=/home/owen/progs/firedrake-complex/firedrake/lib/python3.5/site-packages/matplotlib/mpl-data/fonts/ttf/]
%%   \setsansfont{DejaVuSans.ttf}[Path=/home/owen/progs/firedrake-complex/firedrake/lib/python3.5/site-packages/matplotlib/mpl-data/fonts/ttf/]
%%   \setmonofont{DejaVuSansMono.ttf}[Path=/home/owen/progs/firedrake-complex/firedrake/lib/python3.5/site-packages/matplotlib/mpl-data/fonts/ttf/]
%%
\begingroup%
\makeatletter%
\begin{pgfpicture}%
\pgfpathrectangle{\pgfpointorigin}{\pgfqpoint{6.400000in}{4.800000in}}%
\pgfusepath{use as bounding box, clip}%
\begin{pgfscope}%
\pgfsetbuttcap%
\pgfsetmiterjoin%
\definecolor{currentfill}{rgb}{1.000000,1.000000,1.000000}%
\pgfsetfillcolor{currentfill}%
\pgfsetlinewidth{0.000000pt}%
\definecolor{currentstroke}{rgb}{1.000000,1.000000,1.000000}%
\pgfsetstrokecolor{currentstroke}%
\pgfsetdash{}{0pt}%
\pgfpathmoveto{\pgfqpoint{0.000000in}{0.000000in}}%
\pgfpathlineto{\pgfqpoint{6.400000in}{0.000000in}}%
\pgfpathlineto{\pgfqpoint{6.400000in}{4.800000in}}%
\pgfpathlineto{\pgfqpoint{0.000000in}{4.800000in}}%
\pgfpathclose%
\pgfusepath{fill}%
\end{pgfscope}%
\begin{pgfscope}%
\pgfsetbuttcap%
\pgfsetmiterjoin%
\definecolor{currentfill}{rgb}{1.000000,1.000000,1.000000}%
\pgfsetfillcolor{currentfill}%
\pgfsetlinewidth{0.000000pt}%
\definecolor{currentstroke}{rgb}{0.000000,0.000000,0.000000}%
\pgfsetstrokecolor{currentstroke}%
\pgfsetstrokeopacity{0.000000}%
\pgfsetdash{}{0pt}%
\pgfpathmoveto{\pgfqpoint{0.800000in}{0.528000in}}%
\pgfpathlineto{\pgfqpoint{5.760000in}{0.528000in}}%
\pgfpathlineto{\pgfqpoint{5.760000in}{4.224000in}}%
\pgfpathlineto{\pgfqpoint{0.800000in}{4.224000in}}%
\pgfpathclose%
\pgfusepath{fill}%
\end{pgfscope}%
\begin{pgfscope}%
\pgfpathrectangle{\pgfqpoint{0.800000in}{0.528000in}}{\pgfqpoint{4.960000in}{3.696000in}}%
\pgfusepath{clip}%
\pgfsetbuttcap%
\pgfsetroundjoin%
\definecolor{currentfill}{rgb}{0.000000,0.000000,0.000000}%
\pgfsetfillcolor{currentfill}%
\pgfsetlinewidth{1.003750pt}%
\definecolor{currentstroke}{rgb}{0.000000,0.000000,0.000000}%
\pgfsetstrokecolor{currentstroke}%
\pgfsetdash{}{0pt}%
\pgfpathmoveto{\pgfqpoint{1.025906in}{0.693696in}}%
\pgfpathcurveto{\pgfqpoint{1.036956in}{0.693696in}}{\pgfqpoint{1.047555in}{0.698086in}}{\pgfqpoint{1.055369in}{0.705900in}}%
\pgfpathcurveto{\pgfqpoint{1.063182in}{0.713713in}}{\pgfqpoint{1.067573in}{0.724312in}}{\pgfqpoint{1.067573in}{0.735362in}}%
\pgfpathcurveto{\pgfqpoint{1.067573in}{0.746412in}}{\pgfqpoint{1.063182in}{0.757011in}}{\pgfqpoint{1.055369in}{0.764825in}}%
\pgfpathcurveto{\pgfqpoint{1.047555in}{0.772639in}}{\pgfqpoint{1.036956in}{0.777029in}}{\pgfqpoint{1.025906in}{0.777029in}}%
\pgfpathcurveto{\pgfqpoint{1.014856in}{0.777029in}}{\pgfqpoint{1.004257in}{0.772639in}}{\pgfqpoint{0.996443in}{0.764825in}}%
\pgfpathcurveto{\pgfqpoint{0.988630in}{0.757011in}}{\pgfqpoint{0.984239in}{0.746412in}}{\pgfqpoint{0.984239in}{0.735362in}}%
\pgfpathcurveto{\pgfqpoint{0.984239in}{0.724312in}}{\pgfqpoint{0.988630in}{0.713713in}}{\pgfqpoint{0.996443in}{0.705900in}}%
\pgfpathcurveto{\pgfqpoint{1.004257in}{0.698086in}}{\pgfqpoint{1.014856in}{0.693696in}}{\pgfqpoint{1.025906in}{0.693696in}}%
\pgfpathclose%
\pgfusepath{stroke,fill}%
\end{pgfscope}%
\begin{pgfscope}%
\pgfpathrectangle{\pgfqpoint{0.800000in}{0.528000in}}{\pgfqpoint{4.960000in}{3.696000in}}%
\pgfusepath{clip}%
\pgfsetbuttcap%
\pgfsetroundjoin%
\definecolor{currentfill}{rgb}{0.000000,0.000000,0.000000}%
\pgfsetfillcolor{currentfill}%
\pgfsetlinewidth{1.003750pt}%
\definecolor{currentstroke}{rgb}{0.000000,0.000000,0.000000}%
\pgfsetstrokecolor{currentstroke}%
\pgfsetdash{}{0pt}%
\pgfpathmoveto{\pgfqpoint{1.025906in}{0.683987in}}%
\pgfpathcurveto{\pgfqpoint{1.036956in}{0.683987in}}{\pgfqpoint{1.047555in}{0.688377in}}{\pgfqpoint{1.055369in}{0.696191in}}%
\pgfpathcurveto{\pgfqpoint{1.063182in}{0.704004in}}{\pgfqpoint{1.067573in}{0.714603in}}{\pgfqpoint{1.067573in}{0.725653in}}%
\pgfpathcurveto{\pgfqpoint{1.067573in}{0.736703in}}{\pgfqpoint{1.063182in}{0.747302in}}{\pgfqpoint{1.055369in}{0.755116in}}%
\pgfpathcurveto{\pgfqpoint{1.047555in}{0.762930in}}{\pgfqpoint{1.036956in}{0.767320in}}{\pgfqpoint{1.025906in}{0.767320in}}%
\pgfpathcurveto{\pgfqpoint{1.014856in}{0.767320in}}{\pgfqpoint{1.004257in}{0.762930in}}{\pgfqpoint{0.996443in}{0.755116in}}%
\pgfpathcurveto{\pgfqpoint{0.988630in}{0.747302in}}{\pgfqpoint{0.984239in}{0.736703in}}{\pgfqpoint{0.984239in}{0.725653in}}%
\pgfpathcurveto{\pgfqpoint{0.984239in}{0.714603in}}{\pgfqpoint{0.988630in}{0.704004in}}{\pgfqpoint{0.996443in}{0.696191in}}%
\pgfpathcurveto{\pgfqpoint{1.004257in}{0.688377in}}{\pgfqpoint{1.014856in}{0.683987in}}{\pgfqpoint{1.025906in}{0.683987in}}%
\pgfpathclose%
\pgfusepath{stroke,fill}%
\end{pgfscope}%
\begin{pgfscope}%
\pgfpathrectangle{\pgfqpoint{0.800000in}{0.528000in}}{\pgfqpoint{4.960000in}{3.696000in}}%
\pgfusepath{clip}%
\pgfsetbuttcap%
\pgfsetroundjoin%
\definecolor{currentfill}{rgb}{0.000000,0.000000,0.000000}%
\pgfsetfillcolor{currentfill}%
\pgfsetlinewidth{1.003750pt}%
\definecolor{currentstroke}{rgb}{0.000000,0.000000,0.000000}%
\pgfsetstrokecolor{currentstroke}%
\pgfsetdash{}{0pt}%
\pgfpathmoveto{\pgfqpoint{1.025906in}{0.674278in}}%
\pgfpathcurveto{\pgfqpoint{1.036956in}{0.674278in}}{\pgfqpoint{1.047555in}{0.678668in}}{\pgfqpoint{1.055369in}{0.686482in}}%
\pgfpathcurveto{\pgfqpoint{1.063182in}{0.694295in}}{\pgfqpoint{1.067573in}{0.704894in}}{\pgfqpoint{1.067573in}{0.715944in}}%
\pgfpathcurveto{\pgfqpoint{1.067573in}{0.726994in}}{\pgfqpoint{1.063182in}{0.737594in}}{\pgfqpoint{1.055369in}{0.745407in}}%
\pgfpathcurveto{\pgfqpoint{1.047555in}{0.753221in}}{\pgfqpoint{1.036956in}{0.757611in}}{\pgfqpoint{1.025906in}{0.757611in}}%
\pgfpathcurveto{\pgfqpoint{1.014856in}{0.757611in}}{\pgfqpoint{1.004257in}{0.753221in}}{\pgfqpoint{0.996443in}{0.745407in}}%
\pgfpathcurveto{\pgfqpoint{0.988630in}{0.737594in}}{\pgfqpoint{0.984239in}{0.726994in}}{\pgfqpoint{0.984239in}{0.715944in}}%
\pgfpathcurveto{\pgfqpoint{0.984239in}{0.704894in}}{\pgfqpoint{0.988630in}{0.694295in}}{\pgfqpoint{0.996443in}{0.686482in}}%
\pgfpathcurveto{\pgfqpoint{1.004257in}{0.678668in}}{\pgfqpoint{1.014856in}{0.674278in}}{\pgfqpoint{1.025906in}{0.674278in}}%
\pgfpathclose%
\pgfusepath{stroke,fill}%
\end{pgfscope}%
\begin{pgfscope}%
\pgfpathrectangle{\pgfqpoint{0.800000in}{0.528000in}}{\pgfqpoint{4.960000in}{3.696000in}}%
\pgfusepath{clip}%
\pgfsetbuttcap%
\pgfsetroundjoin%
\definecolor{currentfill}{rgb}{0.000000,0.000000,0.000000}%
\pgfsetfillcolor{currentfill}%
\pgfsetlinewidth{1.003750pt}%
\definecolor{currentstroke}{rgb}{0.000000,0.000000,0.000000}%
\pgfsetstrokecolor{currentstroke}%
\pgfsetdash{}{0pt}%
\pgfpathmoveto{\pgfqpoint{1.025906in}{0.683987in}}%
\pgfpathcurveto{\pgfqpoint{1.036956in}{0.683987in}}{\pgfqpoint{1.047555in}{0.688377in}}{\pgfqpoint{1.055369in}{0.696191in}}%
\pgfpathcurveto{\pgfqpoint{1.063182in}{0.704004in}}{\pgfqpoint{1.067573in}{0.714603in}}{\pgfqpoint{1.067573in}{0.725653in}}%
\pgfpathcurveto{\pgfqpoint{1.067573in}{0.736703in}}{\pgfqpoint{1.063182in}{0.747302in}}{\pgfqpoint{1.055369in}{0.755116in}}%
\pgfpathcurveto{\pgfqpoint{1.047555in}{0.762930in}}{\pgfqpoint{1.036956in}{0.767320in}}{\pgfqpoint{1.025906in}{0.767320in}}%
\pgfpathcurveto{\pgfqpoint{1.014856in}{0.767320in}}{\pgfqpoint{1.004257in}{0.762930in}}{\pgfqpoint{0.996443in}{0.755116in}}%
\pgfpathcurveto{\pgfqpoint{0.988630in}{0.747302in}}{\pgfqpoint{0.984239in}{0.736703in}}{\pgfqpoint{0.984239in}{0.725653in}}%
\pgfpathcurveto{\pgfqpoint{0.984239in}{0.714603in}}{\pgfqpoint{0.988630in}{0.704004in}}{\pgfqpoint{0.996443in}{0.696191in}}%
\pgfpathcurveto{\pgfqpoint{1.004257in}{0.688377in}}{\pgfqpoint{1.014856in}{0.683987in}}{\pgfqpoint{1.025906in}{0.683987in}}%
\pgfpathclose%
\pgfusepath{stroke,fill}%
\end{pgfscope}%
\begin{pgfscope}%
\pgfpathrectangle{\pgfqpoint{0.800000in}{0.528000in}}{\pgfqpoint{4.960000in}{3.696000in}}%
\pgfusepath{clip}%
\pgfsetbuttcap%
\pgfsetroundjoin%
\definecolor{currentfill}{rgb}{0.000000,0.000000,0.000000}%
\pgfsetfillcolor{currentfill}%
\pgfsetlinewidth{1.003750pt}%
\definecolor{currentstroke}{rgb}{0.000000,0.000000,0.000000}%
\pgfsetstrokecolor{currentstroke}%
\pgfsetdash{}{0pt}%
\pgfpathmoveto{\pgfqpoint{1.025906in}{0.703405in}}%
\pgfpathcurveto{\pgfqpoint{1.036956in}{0.703405in}}{\pgfqpoint{1.047555in}{0.707795in}}{\pgfqpoint{1.055369in}{0.715608in}}%
\pgfpathcurveto{\pgfqpoint{1.063182in}{0.723422in}}{\pgfqpoint{1.067573in}{0.734021in}}{\pgfqpoint{1.067573in}{0.745071in}}%
\pgfpathcurveto{\pgfqpoint{1.067573in}{0.756121in}}{\pgfqpoint{1.063182in}{0.766720in}}{\pgfqpoint{1.055369in}{0.774534in}}%
\pgfpathcurveto{\pgfqpoint{1.047555in}{0.782348in}}{\pgfqpoint{1.036956in}{0.786738in}}{\pgfqpoint{1.025906in}{0.786738in}}%
\pgfpathcurveto{\pgfqpoint{1.014856in}{0.786738in}}{\pgfqpoint{1.004257in}{0.782348in}}{\pgfqpoint{0.996443in}{0.774534in}}%
\pgfpathcurveto{\pgfqpoint{0.988630in}{0.766720in}}{\pgfqpoint{0.984239in}{0.756121in}}{\pgfqpoint{0.984239in}{0.745071in}}%
\pgfpathcurveto{\pgfqpoint{0.984239in}{0.734021in}}{\pgfqpoint{0.988630in}{0.723422in}}{\pgfqpoint{0.996443in}{0.715608in}}%
\pgfpathcurveto{\pgfqpoint{1.004257in}{0.707795in}}{\pgfqpoint{1.014856in}{0.703405in}}{\pgfqpoint{1.025906in}{0.703405in}}%
\pgfpathclose%
\pgfusepath{stroke,fill}%
\end{pgfscope}%
\begin{pgfscope}%
\pgfpathrectangle{\pgfqpoint{0.800000in}{0.528000in}}{\pgfqpoint{4.960000in}{3.696000in}}%
\pgfusepath{clip}%
\pgfsetbuttcap%
\pgfsetroundjoin%
\definecolor{currentfill}{rgb}{0.000000,0.000000,0.000000}%
\pgfsetfillcolor{currentfill}%
\pgfsetlinewidth{1.003750pt}%
\definecolor{currentstroke}{rgb}{0.000000,0.000000,0.000000}%
\pgfsetstrokecolor{currentstroke}%
\pgfsetdash{}{0pt}%
\pgfpathmoveto{\pgfqpoint{1.025906in}{0.683987in}}%
\pgfpathcurveto{\pgfqpoint{1.036956in}{0.683987in}}{\pgfqpoint{1.047555in}{0.688377in}}{\pgfqpoint{1.055369in}{0.696191in}}%
\pgfpathcurveto{\pgfqpoint{1.063182in}{0.704004in}}{\pgfqpoint{1.067573in}{0.714603in}}{\pgfqpoint{1.067573in}{0.725653in}}%
\pgfpathcurveto{\pgfqpoint{1.067573in}{0.736703in}}{\pgfqpoint{1.063182in}{0.747302in}}{\pgfqpoint{1.055369in}{0.755116in}}%
\pgfpathcurveto{\pgfqpoint{1.047555in}{0.762930in}}{\pgfqpoint{1.036956in}{0.767320in}}{\pgfqpoint{1.025906in}{0.767320in}}%
\pgfpathcurveto{\pgfqpoint{1.014856in}{0.767320in}}{\pgfqpoint{1.004257in}{0.762930in}}{\pgfqpoint{0.996443in}{0.755116in}}%
\pgfpathcurveto{\pgfqpoint{0.988630in}{0.747302in}}{\pgfqpoint{0.984239in}{0.736703in}}{\pgfqpoint{0.984239in}{0.725653in}}%
\pgfpathcurveto{\pgfqpoint{0.984239in}{0.714603in}}{\pgfqpoint{0.988630in}{0.704004in}}{\pgfqpoint{0.996443in}{0.696191in}}%
\pgfpathcurveto{\pgfqpoint{1.004257in}{0.688377in}}{\pgfqpoint{1.014856in}{0.683987in}}{\pgfqpoint{1.025906in}{0.683987in}}%
\pgfpathclose%
\pgfusepath{stroke,fill}%
\end{pgfscope}%
\begin{pgfscope}%
\pgfpathrectangle{\pgfqpoint{0.800000in}{0.528000in}}{\pgfqpoint{4.960000in}{3.696000in}}%
\pgfusepath{clip}%
\pgfsetbuttcap%
\pgfsetroundjoin%
\definecolor{currentfill}{rgb}{0.000000,0.000000,0.000000}%
\pgfsetfillcolor{currentfill}%
\pgfsetlinewidth{1.003750pt}%
\definecolor{currentstroke}{rgb}{0.000000,0.000000,0.000000}%
\pgfsetstrokecolor{currentstroke}%
\pgfsetdash{}{0pt}%
\pgfpathmoveto{\pgfqpoint{1.025906in}{0.674278in}}%
\pgfpathcurveto{\pgfqpoint{1.036956in}{0.674278in}}{\pgfqpoint{1.047555in}{0.678668in}}{\pgfqpoint{1.055369in}{0.686482in}}%
\pgfpathcurveto{\pgfqpoint{1.063182in}{0.694295in}}{\pgfqpoint{1.067573in}{0.704894in}}{\pgfqpoint{1.067573in}{0.715944in}}%
\pgfpathcurveto{\pgfqpoint{1.067573in}{0.726994in}}{\pgfqpoint{1.063182in}{0.737594in}}{\pgfqpoint{1.055369in}{0.745407in}}%
\pgfpathcurveto{\pgfqpoint{1.047555in}{0.753221in}}{\pgfqpoint{1.036956in}{0.757611in}}{\pgfqpoint{1.025906in}{0.757611in}}%
\pgfpathcurveto{\pgfqpoint{1.014856in}{0.757611in}}{\pgfqpoint{1.004257in}{0.753221in}}{\pgfqpoint{0.996443in}{0.745407in}}%
\pgfpathcurveto{\pgfqpoint{0.988630in}{0.737594in}}{\pgfqpoint{0.984239in}{0.726994in}}{\pgfqpoint{0.984239in}{0.715944in}}%
\pgfpathcurveto{\pgfqpoint{0.984239in}{0.704894in}}{\pgfqpoint{0.988630in}{0.694295in}}{\pgfqpoint{0.996443in}{0.686482in}}%
\pgfpathcurveto{\pgfqpoint{1.004257in}{0.678668in}}{\pgfqpoint{1.014856in}{0.674278in}}{\pgfqpoint{1.025906in}{0.674278in}}%
\pgfpathclose%
\pgfusepath{stroke,fill}%
\end{pgfscope}%
\begin{pgfscope}%
\pgfpathrectangle{\pgfqpoint{0.800000in}{0.528000in}}{\pgfqpoint{4.960000in}{3.696000in}}%
\pgfusepath{clip}%
\pgfsetbuttcap%
\pgfsetroundjoin%
\definecolor{currentfill}{rgb}{0.000000,0.000000,0.000000}%
\pgfsetfillcolor{currentfill}%
\pgfsetlinewidth{1.003750pt}%
\definecolor{currentstroke}{rgb}{0.000000,0.000000,0.000000}%
\pgfsetstrokecolor{currentstroke}%
\pgfsetdash{}{0pt}%
\pgfpathmoveto{\pgfqpoint{1.025906in}{0.693696in}}%
\pgfpathcurveto{\pgfqpoint{1.036956in}{0.693696in}}{\pgfqpoint{1.047555in}{0.698086in}}{\pgfqpoint{1.055369in}{0.705900in}}%
\pgfpathcurveto{\pgfqpoint{1.063182in}{0.713713in}}{\pgfqpoint{1.067573in}{0.724312in}}{\pgfqpoint{1.067573in}{0.735362in}}%
\pgfpathcurveto{\pgfqpoint{1.067573in}{0.746412in}}{\pgfqpoint{1.063182in}{0.757011in}}{\pgfqpoint{1.055369in}{0.764825in}}%
\pgfpathcurveto{\pgfqpoint{1.047555in}{0.772639in}}{\pgfqpoint{1.036956in}{0.777029in}}{\pgfqpoint{1.025906in}{0.777029in}}%
\pgfpathcurveto{\pgfqpoint{1.014856in}{0.777029in}}{\pgfqpoint{1.004257in}{0.772639in}}{\pgfqpoint{0.996443in}{0.764825in}}%
\pgfpathcurveto{\pgfqpoint{0.988630in}{0.757011in}}{\pgfqpoint{0.984239in}{0.746412in}}{\pgfqpoint{0.984239in}{0.735362in}}%
\pgfpathcurveto{\pgfqpoint{0.984239in}{0.724312in}}{\pgfqpoint{0.988630in}{0.713713in}}{\pgfqpoint{0.996443in}{0.705900in}}%
\pgfpathcurveto{\pgfqpoint{1.004257in}{0.698086in}}{\pgfqpoint{1.014856in}{0.693696in}}{\pgfqpoint{1.025906in}{0.693696in}}%
\pgfpathclose%
\pgfusepath{stroke,fill}%
\end{pgfscope}%
\begin{pgfscope}%
\pgfpathrectangle{\pgfqpoint{0.800000in}{0.528000in}}{\pgfqpoint{4.960000in}{3.696000in}}%
\pgfusepath{clip}%
\pgfsetbuttcap%
\pgfsetroundjoin%
\definecolor{currentfill}{rgb}{0.000000,0.000000,0.000000}%
\pgfsetfillcolor{currentfill}%
\pgfsetlinewidth{1.003750pt}%
\definecolor{currentstroke}{rgb}{0.000000,0.000000,0.000000}%
\pgfsetstrokecolor{currentstroke}%
\pgfsetdash{}{0pt}%
\pgfpathmoveto{\pgfqpoint{1.025906in}{0.683987in}}%
\pgfpathcurveto{\pgfqpoint{1.036956in}{0.683987in}}{\pgfqpoint{1.047555in}{0.688377in}}{\pgfqpoint{1.055369in}{0.696191in}}%
\pgfpathcurveto{\pgfqpoint{1.063182in}{0.704004in}}{\pgfqpoint{1.067573in}{0.714603in}}{\pgfqpoint{1.067573in}{0.725653in}}%
\pgfpathcurveto{\pgfqpoint{1.067573in}{0.736703in}}{\pgfqpoint{1.063182in}{0.747302in}}{\pgfqpoint{1.055369in}{0.755116in}}%
\pgfpathcurveto{\pgfqpoint{1.047555in}{0.762930in}}{\pgfqpoint{1.036956in}{0.767320in}}{\pgfqpoint{1.025906in}{0.767320in}}%
\pgfpathcurveto{\pgfqpoint{1.014856in}{0.767320in}}{\pgfqpoint{1.004257in}{0.762930in}}{\pgfqpoint{0.996443in}{0.755116in}}%
\pgfpathcurveto{\pgfqpoint{0.988630in}{0.747302in}}{\pgfqpoint{0.984239in}{0.736703in}}{\pgfqpoint{0.984239in}{0.725653in}}%
\pgfpathcurveto{\pgfqpoint{0.984239in}{0.714603in}}{\pgfqpoint{0.988630in}{0.704004in}}{\pgfqpoint{0.996443in}{0.696191in}}%
\pgfpathcurveto{\pgfqpoint{1.004257in}{0.688377in}}{\pgfqpoint{1.014856in}{0.683987in}}{\pgfqpoint{1.025906in}{0.683987in}}%
\pgfpathclose%
\pgfusepath{stroke,fill}%
\end{pgfscope}%
\begin{pgfscope}%
\pgfpathrectangle{\pgfqpoint{0.800000in}{0.528000in}}{\pgfqpoint{4.960000in}{3.696000in}}%
\pgfusepath{clip}%
\pgfsetbuttcap%
\pgfsetroundjoin%
\definecolor{currentfill}{rgb}{0.000000,0.000000,0.000000}%
\pgfsetfillcolor{currentfill}%
\pgfsetlinewidth{1.003750pt}%
\definecolor{currentstroke}{rgb}{0.000000,0.000000,0.000000}%
\pgfsetstrokecolor{currentstroke}%
\pgfsetdash{}{0pt}%
\pgfpathmoveto{\pgfqpoint{1.025906in}{0.693696in}}%
\pgfpathcurveto{\pgfqpoint{1.036956in}{0.693696in}}{\pgfqpoint{1.047555in}{0.698086in}}{\pgfqpoint{1.055369in}{0.705900in}}%
\pgfpathcurveto{\pgfqpoint{1.063182in}{0.713713in}}{\pgfqpoint{1.067573in}{0.724312in}}{\pgfqpoint{1.067573in}{0.735362in}}%
\pgfpathcurveto{\pgfqpoint{1.067573in}{0.746412in}}{\pgfqpoint{1.063182in}{0.757011in}}{\pgfqpoint{1.055369in}{0.764825in}}%
\pgfpathcurveto{\pgfqpoint{1.047555in}{0.772639in}}{\pgfqpoint{1.036956in}{0.777029in}}{\pgfqpoint{1.025906in}{0.777029in}}%
\pgfpathcurveto{\pgfqpoint{1.014856in}{0.777029in}}{\pgfqpoint{1.004257in}{0.772639in}}{\pgfqpoint{0.996443in}{0.764825in}}%
\pgfpathcurveto{\pgfqpoint{0.988630in}{0.757011in}}{\pgfqpoint{0.984239in}{0.746412in}}{\pgfqpoint{0.984239in}{0.735362in}}%
\pgfpathcurveto{\pgfqpoint{0.984239in}{0.724312in}}{\pgfqpoint{0.988630in}{0.713713in}}{\pgfqpoint{0.996443in}{0.705900in}}%
\pgfpathcurveto{\pgfqpoint{1.004257in}{0.698086in}}{\pgfqpoint{1.014856in}{0.693696in}}{\pgfqpoint{1.025906in}{0.693696in}}%
\pgfpathclose%
\pgfusepath{stroke,fill}%
\end{pgfscope}%
\begin{pgfscope}%
\pgfpathrectangle{\pgfqpoint{0.800000in}{0.528000in}}{\pgfqpoint{4.960000in}{3.696000in}}%
\pgfusepath{clip}%
\pgfsetbuttcap%
\pgfsetroundjoin%
\definecolor{currentfill}{rgb}{0.000000,0.000000,0.000000}%
\pgfsetfillcolor{currentfill}%
\pgfsetlinewidth{1.003750pt}%
\definecolor{currentstroke}{rgb}{0.000000,0.000000,0.000000}%
\pgfsetstrokecolor{currentstroke}%
\pgfsetdash{}{0pt}%
\pgfpathmoveto{\pgfqpoint{1.025906in}{0.722823in}}%
\pgfpathcurveto{\pgfqpoint{1.036956in}{0.722823in}}{\pgfqpoint{1.047555in}{0.727213in}}{\pgfqpoint{1.055369in}{0.735026in}}%
\pgfpathcurveto{\pgfqpoint{1.063182in}{0.742840in}}{\pgfqpoint{1.067573in}{0.753439in}}{\pgfqpoint{1.067573in}{0.764489in}}%
\pgfpathcurveto{\pgfqpoint{1.067573in}{0.775539in}}{\pgfqpoint{1.063182in}{0.786138in}}{\pgfqpoint{1.055369in}{0.793952in}}%
\pgfpathcurveto{\pgfqpoint{1.047555in}{0.801766in}}{\pgfqpoint{1.036956in}{0.806156in}}{\pgfqpoint{1.025906in}{0.806156in}}%
\pgfpathcurveto{\pgfqpoint{1.014856in}{0.806156in}}{\pgfqpoint{1.004257in}{0.801766in}}{\pgfqpoint{0.996443in}{0.793952in}}%
\pgfpathcurveto{\pgfqpoint{0.988630in}{0.786138in}}{\pgfqpoint{0.984239in}{0.775539in}}{\pgfqpoint{0.984239in}{0.764489in}}%
\pgfpathcurveto{\pgfqpoint{0.984239in}{0.753439in}}{\pgfqpoint{0.988630in}{0.742840in}}{\pgfqpoint{0.996443in}{0.735026in}}%
\pgfpathcurveto{\pgfqpoint{1.004257in}{0.727213in}}{\pgfqpoint{1.014856in}{0.722823in}}{\pgfqpoint{1.025906in}{0.722823in}}%
\pgfpathclose%
\pgfusepath{stroke,fill}%
\end{pgfscope}%
\begin{pgfscope}%
\pgfpathrectangle{\pgfqpoint{0.800000in}{0.528000in}}{\pgfqpoint{4.960000in}{3.696000in}}%
\pgfusepath{clip}%
\pgfsetbuttcap%
\pgfsetroundjoin%
\definecolor{currentfill}{rgb}{0.000000,0.000000,0.000000}%
\pgfsetfillcolor{currentfill}%
\pgfsetlinewidth{1.003750pt}%
\definecolor{currentstroke}{rgb}{0.000000,0.000000,0.000000}%
\pgfsetstrokecolor{currentstroke}%
\pgfsetdash{}{0pt}%
\pgfpathmoveto{\pgfqpoint{1.025906in}{0.703405in}}%
\pgfpathcurveto{\pgfqpoint{1.036956in}{0.703405in}}{\pgfqpoint{1.047555in}{0.707795in}}{\pgfqpoint{1.055369in}{0.715608in}}%
\pgfpathcurveto{\pgfqpoint{1.063182in}{0.723422in}}{\pgfqpoint{1.067573in}{0.734021in}}{\pgfqpoint{1.067573in}{0.745071in}}%
\pgfpathcurveto{\pgfqpoint{1.067573in}{0.756121in}}{\pgfqpoint{1.063182in}{0.766720in}}{\pgfqpoint{1.055369in}{0.774534in}}%
\pgfpathcurveto{\pgfqpoint{1.047555in}{0.782348in}}{\pgfqpoint{1.036956in}{0.786738in}}{\pgfqpoint{1.025906in}{0.786738in}}%
\pgfpathcurveto{\pgfqpoint{1.014856in}{0.786738in}}{\pgfqpoint{1.004257in}{0.782348in}}{\pgfqpoint{0.996443in}{0.774534in}}%
\pgfpathcurveto{\pgfqpoint{0.988630in}{0.766720in}}{\pgfqpoint{0.984239in}{0.756121in}}{\pgfqpoint{0.984239in}{0.745071in}}%
\pgfpathcurveto{\pgfqpoint{0.984239in}{0.734021in}}{\pgfqpoint{0.988630in}{0.723422in}}{\pgfqpoint{0.996443in}{0.715608in}}%
\pgfpathcurveto{\pgfqpoint{1.004257in}{0.707795in}}{\pgfqpoint{1.014856in}{0.703405in}}{\pgfqpoint{1.025906in}{0.703405in}}%
\pgfpathclose%
\pgfusepath{stroke,fill}%
\end{pgfscope}%
\begin{pgfscope}%
\pgfpathrectangle{\pgfqpoint{0.800000in}{0.528000in}}{\pgfqpoint{4.960000in}{3.696000in}}%
\pgfusepath{clip}%
\pgfsetbuttcap%
\pgfsetroundjoin%
\definecolor{currentfill}{rgb}{0.000000,0.000000,0.000000}%
\pgfsetfillcolor{currentfill}%
\pgfsetlinewidth{1.003750pt}%
\definecolor{currentstroke}{rgb}{0.000000,0.000000,0.000000}%
\pgfsetstrokecolor{currentstroke}%
\pgfsetdash{}{0pt}%
\pgfpathmoveto{\pgfqpoint{1.025906in}{0.683987in}}%
\pgfpathcurveto{\pgfqpoint{1.036956in}{0.683987in}}{\pgfqpoint{1.047555in}{0.688377in}}{\pgfqpoint{1.055369in}{0.696191in}}%
\pgfpathcurveto{\pgfqpoint{1.063182in}{0.704004in}}{\pgfqpoint{1.067573in}{0.714603in}}{\pgfqpoint{1.067573in}{0.725653in}}%
\pgfpathcurveto{\pgfqpoint{1.067573in}{0.736703in}}{\pgfqpoint{1.063182in}{0.747302in}}{\pgfqpoint{1.055369in}{0.755116in}}%
\pgfpathcurveto{\pgfqpoint{1.047555in}{0.762930in}}{\pgfqpoint{1.036956in}{0.767320in}}{\pgfqpoint{1.025906in}{0.767320in}}%
\pgfpathcurveto{\pgfqpoint{1.014856in}{0.767320in}}{\pgfqpoint{1.004257in}{0.762930in}}{\pgfqpoint{0.996443in}{0.755116in}}%
\pgfpathcurveto{\pgfqpoint{0.988630in}{0.747302in}}{\pgfqpoint{0.984239in}{0.736703in}}{\pgfqpoint{0.984239in}{0.725653in}}%
\pgfpathcurveto{\pgfqpoint{0.984239in}{0.714603in}}{\pgfqpoint{0.988630in}{0.704004in}}{\pgfqpoint{0.996443in}{0.696191in}}%
\pgfpathcurveto{\pgfqpoint{1.004257in}{0.688377in}}{\pgfqpoint{1.014856in}{0.683987in}}{\pgfqpoint{1.025906in}{0.683987in}}%
\pgfpathclose%
\pgfusepath{stroke,fill}%
\end{pgfscope}%
\begin{pgfscope}%
\pgfpathrectangle{\pgfqpoint{0.800000in}{0.528000in}}{\pgfqpoint{4.960000in}{3.696000in}}%
\pgfusepath{clip}%
\pgfsetbuttcap%
\pgfsetroundjoin%
\definecolor{currentfill}{rgb}{0.000000,0.000000,0.000000}%
\pgfsetfillcolor{currentfill}%
\pgfsetlinewidth{1.003750pt}%
\definecolor{currentstroke}{rgb}{0.000000,0.000000,0.000000}%
\pgfsetstrokecolor{currentstroke}%
\pgfsetdash{}{0pt}%
\pgfpathmoveto{\pgfqpoint{1.025906in}{0.674278in}}%
\pgfpathcurveto{\pgfqpoint{1.036956in}{0.674278in}}{\pgfqpoint{1.047555in}{0.678668in}}{\pgfqpoint{1.055369in}{0.686482in}}%
\pgfpathcurveto{\pgfqpoint{1.063182in}{0.694295in}}{\pgfqpoint{1.067573in}{0.704894in}}{\pgfqpoint{1.067573in}{0.715944in}}%
\pgfpathcurveto{\pgfqpoint{1.067573in}{0.726994in}}{\pgfqpoint{1.063182in}{0.737594in}}{\pgfqpoint{1.055369in}{0.745407in}}%
\pgfpathcurveto{\pgfqpoint{1.047555in}{0.753221in}}{\pgfqpoint{1.036956in}{0.757611in}}{\pgfqpoint{1.025906in}{0.757611in}}%
\pgfpathcurveto{\pgfqpoint{1.014856in}{0.757611in}}{\pgfqpoint{1.004257in}{0.753221in}}{\pgfqpoint{0.996443in}{0.745407in}}%
\pgfpathcurveto{\pgfqpoint{0.988630in}{0.737594in}}{\pgfqpoint{0.984239in}{0.726994in}}{\pgfqpoint{0.984239in}{0.715944in}}%
\pgfpathcurveto{\pgfqpoint{0.984239in}{0.704894in}}{\pgfqpoint{0.988630in}{0.694295in}}{\pgfqpoint{0.996443in}{0.686482in}}%
\pgfpathcurveto{\pgfqpoint{1.004257in}{0.678668in}}{\pgfqpoint{1.014856in}{0.674278in}}{\pgfqpoint{1.025906in}{0.674278in}}%
\pgfpathclose%
\pgfusepath{stroke,fill}%
\end{pgfscope}%
\begin{pgfscope}%
\pgfpathrectangle{\pgfqpoint{0.800000in}{0.528000in}}{\pgfqpoint{4.960000in}{3.696000in}}%
\pgfusepath{clip}%
\pgfsetbuttcap%
\pgfsetroundjoin%
\definecolor{currentfill}{rgb}{0.000000,0.000000,0.000000}%
\pgfsetfillcolor{currentfill}%
\pgfsetlinewidth{1.003750pt}%
\definecolor{currentstroke}{rgb}{0.000000,0.000000,0.000000}%
\pgfsetstrokecolor{currentstroke}%
\pgfsetdash{}{0pt}%
\pgfpathmoveto{\pgfqpoint{1.025906in}{0.683987in}}%
\pgfpathcurveto{\pgfqpoint{1.036956in}{0.683987in}}{\pgfqpoint{1.047555in}{0.688377in}}{\pgfqpoint{1.055369in}{0.696191in}}%
\pgfpathcurveto{\pgfqpoint{1.063182in}{0.704004in}}{\pgfqpoint{1.067573in}{0.714603in}}{\pgfqpoint{1.067573in}{0.725653in}}%
\pgfpathcurveto{\pgfqpoint{1.067573in}{0.736703in}}{\pgfqpoint{1.063182in}{0.747302in}}{\pgfqpoint{1.055369in}{0.755116in}}%
\pgfpathcurveto{\pgfqpoint{1.047555in}{0.762930in}}{\pgfqpoint{1.036956in}{0.767320in}}{\pgfqpoint{1.025906in}{0.767320in}}%
\pgfpathcurveto{\pgfqpoint{1.014856in}{0.767320in}}{\pgfqpoint{1.004257in}{0.762930in}}{\pgfqpoint{0.996443in}{0.755116in}}%
\pgfpathcurveto{\pgfqpoint{0.988630in}{0.747302in}}{\pgfqpoint{0.984239in}{0.736703in}}{\pgfqpoint{0.984239in}{0.725653in}}%
\pgfpathcurveto{\pgfqpoint{0.984239in}{0.714603in}}{\pgfqpoint{0.988630in}{0.704004in}}{\pgfqpoint{0.996443in}{0.696191in}}%
\pgfpathcurveto{\pgfqpoint{1.004257in}{0.688377in}}{\pgfqpoint{1.014856in}{0.683987in}}{\pgfqpoint{1.025906in}{0.683987in}}%
\pgfpathclose%
\pgfusepath{stroke,fill}%
\end{pgfscope}%
\begin{pgfscope}%
\pgfpathrectangle{\pgfqpoint{0.800000in}{0.528000in}}{\pgfqpoint{4.960000in}{3.696000in}}%
\pgfusepath{clip}%
\pgfsetbuttcap%
\pgfsetroundjoin%
\definecolor{currentfill}{rgb}{0.000000,0.000000,0.000000}%
\pgfsetfillcolor{currentfill}%
\pgfsetlinewidth{1.003750pt}%
\definecolor{currentstroke}{rgb}{0.000000,0.000000,0.000000}%
\pgfsetstrokecolor{currentstroke}%
\pgfsetdash{}{0pt}%
\pgfpathmoveto{\pgfqpoint{1.025906in}{0.674278in}}%
\pgfpathcurveto{\pgfqpoint{1.036956in}{0.674278in}}{\pgfqpoint{1.047555in}{0.678668in}}{\pgfqpoint{1.055369in}{0.686482in}}%
\pgfpathcurveto{\pgfqpoint{1.063182in}{0.694295in}}{\pgfqpoint{1.067573in}{0.704894in}}{\pgfqpoint{1.067573in}{0.715944in}}%
\pgfpathcurveto{\pgfqpoint{1.067573in}{0.726994in}}{\pgfqpoint{1.063182in}{0.737594in}}{\pgfqpoint{1.055369in}{0.745407in}}%
\pgfpathcurveto{\pgfqpoint{1.047555in}{0.753221in}}{\pgfqpoint{1.036956in}{0.757611in}}{\pgfqpoint{1.025906in}{0.757611in}}%
\pgfpathcurveto{\pgfqpoint{1.014856in}{0.757611in}}{\pgfqpoint{1.004257in}{0.753221in}}{\pgfqpoint{0.996443in}{0.745407in}}%
\pgfpathcurveto{\pgfqpoint{0.988630in}{0.737594in}}{\pgfqpoint{0.984239in}{0.726994in}}{\pgfqpoint{0.984239in}{0.715944in}}%
\pgfpathcurveto{\pgfqpoint{0.984239in}{0.704894in}}{\pgfqpoint{0.988630in}{0.694295in}}{\pgfqpoint{0.996443in}{0.686482in}}%
\pgfpathcurveto{\pgfqpoint{1.004257in}{0.678668in}}{\pgfqpoint{1.014856in}{0.674278in}}{\pgfqpoint{1.025906in}{0.674278in}}%
\pgfpathclose%
\pgfusepath{stroke,fill}%
\end{pgfscope}%
\begin{pgfscope}%
\pgfpathrectangle{\pgfqpoint{0.800000in}{0.528000in}}{\pgfqpoint{4.960000in}{3.696000in}}%
\pgfusepath{clip}%
\pgfsetbuttcap%
\pgfsetroundjoin%
\definecolor{currentfill}{rgb}{0.000000,0.000000,0.000000}%
\pgfsetfillcolor{currentfill}%
\pgfsetlinewidth{1.003750pt}%
\definecolor{currentstroke}{rgb}{0.000000,0.000000,0.000000}%
\pgfsetstrokecolor{currentstroke}%
\pgfsetdash{}{0pt}%
\pgfpathmoveto{\pgfqpoint{1.025906in}{0.683987in}}%
\pgfpathcurveto{\pgfqpoint{1.036956in}{0.683987in}}{\pgfqpoint{1.047555in}{0.688377in}}{\pgfqpoint{1.055369in}{0.696191in}}%
\pgfpathcurveto{\pgfqpoint{1.063182in}{0.704004in}}{\pgfqpoint{1.067573in}{0.714603in}}{\pgfqpoint{1.067573in}{0.725653in}}%
\pgfpathcurveto{\pgfqpoint{1.067573in}{0.736703in}}{\pgfqpoint{1.063182in}{0.747302in}}{\pgfqpoint{1.055369in}{0.755116in}}%
\pgfpathcurveto{\pgfqpoint{1.047555in}{0.762930in}}{\pgfqpoint{1.036956in}{0.767320in}}{\pgfqpoint{1.025906in}{0.767320in}}%
\pgfpathcurveto{\pgfqpoint{1.014856in}{0.767320in}}{\pgfqpoint{1.004257in}{0.762930in}}{\pgfqpoint{0.996443in}{0.755116in}}%
\pgfpathcurveto{\pgfqpoint{0.988630in}{0.747302in}}{\pgfqpoint{0.984239in}{0.736703in}}{\pgfqpoint{0.984239in}{0.725653in}}%
\pgfpathcurveto{\pgfqpoint{0.984239in}{0.714603in}}{\pgfqpoint{0.988630in}{0.704004in}}{\pgfqpoint{0.996443in}{0.696191in}}%
\pgfpathcurveto{\pgfqpoint{1.004257in}{0.688377in}}{\pgfqpoint{1.014856in}{0.683987in}}{\pgfqpoint{1.025906in}{0.683987in}}%
\pgfpathclose%
\pgfusepath{stroke,fill}%
\end{pgfscope}%
\begin{pgfscope}%
\pgfpathrectangle{\pgfqpoint{0.800000in}{0.528000in}}{\pgfqpoint{4.960000in}{3.696000in}}%
\pgfusepath{clip}%
\pgfsetbuttcap%
\pgfsetroundjoin%
\definecolor{currentfill}{rgb}{0.000000,0.000000,0.000000}%
\pgfsetfillcolor{currentfill}%
\pgfsetlinewidth{1.003750pt}%
\definecolor{currentstroke}{rgb}{0.000000,0.000000,0.000000}%
\pgfsetstrokecolor{currentstroke}%
\pgfsetdash{}{0pt}%
\pgfpathmoveto{\pgfqpoint{1.025906in}{0.683987in}}%
\pgfpathcurveto{\pgfqpoint{1.036956in}{0.683987in}}{\pgfqpoint{1.047555in}{0.688377in}}{\pgfqpoint{1.055369in}{0.696191in}}%
\pgfpathcurveto{\pgfqpoint{1.063182in}{0.704004in}}{\pgfqpoint{1.067573in}{0.714603in}}{\pgfqpoint{1.067573in}{0.725653in}}%
\pgfpathcurveto{\pgfqpoint{1.067573in}{0.736703in}}{\pgfqpoint{1.063182in}{0.747302in}}{\pgfqpoint{1.055369in}{0.755116in}}%
\pgfpathcurveto{\pgfqpoint{1.047555in}{0.762930in}}{\pgfqpoint{1.036956in}{0.767320in}}{\pgfqpoint{1.025906in}{0.767320in}}%
\pgfpathcurveto{\pgfqpoint{1.014856in}{0.767320in}}{\pgfqpoint{1.004257in}{0.762930in}}{\pgfqpoint{0.996443in}{0.755116in}}%
\pgfpathcurveto{\pgfqpoint{0.988630in}{0.747302in}}{\pgfqpoint{0.984239in}{0.736703in}}{\pgfqpoint{0.984239in}{0.725653in}}%
\pgfpathcurveto{\pgfqpoint{0.984239in}{0.714603in}}{\pgfqpoint{0.988630in}{0.704004in}}{\pgfqpoint{0.996443in}{0.696191in}}%
\pgfpathcurveto{\pgfqpoint{1.004257in}{0.688377in}}{\pgfqpoint{1.014856in}{0.683987in}}{\pgfqpoint{1.025906in}{0.683987in}}%
\pgfpathclose%
\pgfusepath{stroke,fill}%
\end{pgfscope}%
\begin{pgfscope}%
\pgfpathrectangle{\pgfqpoint{0.800000in}{0.528000in}}{\pgfqpoint{4.960000in}{3.696000in}}%
\pgfusepath{clip}%
\pgfsetbuttcap%
\pgfsetroundjoin%
\definecolor{currentfill}{rgb}{0.000000,0.000000,0.000000}%
\pgfsetfillcolor{currentfill}%
\pgfsetlinewidth{1.003750pt}%
\definecolor{currentstroke}{rgb}{0.000000,0.000000,0.000000}%
\pgfsetstrokecolor{currentstroke}%
\pgfsetdash{}{0pt}%
\pgfpathmoveto{\pgfqpoint{1.025906in}{0.703405in}}%
\pgfpathcurveto{\pgfqpoint{1.036956in}{0.703405in}}{\pgfqpoint{1.047555in}{0.707795in}}{\pgfqpoint{1.055369in}{0.715608in}}%
\pgfpathcurveto{\pgfqpoint{1.063182in}{0.723422in}}{\pgfqpoint{1.067573in}{0.734021in}}{\pgfqpoint{1.067573in}{0.745071in}}%
\pgfpathcurveto{\pgfqpoint{1.067573in}{0.756121in}}{\pgfqpoint{1.063182in}{0.766720in}}{\pgfqpoint{1.055369in}{0.774534in}}%
\pgfpathcurveto{\pgfqpoint{1.047555in}{0.782348in}}{\pgfqpoint{1.036956in}{0.786738in}}{\pgfqpoint{1.025906in}{0.786738in}}%
\pgfpathcurveto{\pgfqpoint{1.014856in}{0.786738in}}{\pgfqpoint{1.004257in}{0.782348in}}{\pgfqpoint{0.996443in}{0.774534in}}%
\pgfpathcurveto{\pgfqpoint{0.988630in}{0.766720in}}{\pgfqpoint{0.984239in}{0.756121in}}{\pgfqpoint{0.984239in}{0.745071in}}%
\pgfpathcurveto{\pgfqpoint{0.984239in}{0.734021in}}{\pgfqpoint{0.988630in}{0.723422in}}{\pgfqpoint{0.996443in}{0.715608in}}%
\pgfpathcurveto{\pgfqpoint{1.004257in}{0.707795in}}{\pgfqpoint{1.014856in}{0.703405in}}{\pgfqpoint{1.025906in}{0.703405in}}%
\pgfpathclose%
\pgfusepath{stroke,fill}%
\end{pgfscope}%
\begin{pgfscope}%
\pgfpathrectangle{\pgfqpoint{0.800000in}{0.528000in}}{\pgfqpoint{4.960000in}{3.696000in}}%
\pgfusepath{clip}%
\pgfsetbuttcap%
\pgfsetroundjoin%
\definecolor{currentfill}{rgb}{0.000000,0.000000,0.000000}%
\pgfsetfillcolor{currentfill}%
\pgfsetlinewidth{1.003750pt}%
\definecolor{currentstroke}{rgb}{0.000000,0.000000,0.000000}%
\pgfsetstrokecolor{currentstroke}%
\pgfsetdash{}{0pt}%
\pgfpathmoveto{\pgfqpoint{1.025906in}{0.703405in}}%
\pgfpathcurveto{\pgfqpoint{1.036956in}{0.703405in}}{\pgfqpoint{1.047555in}{0.707795in}}{\pgfqpoint{1.055369in}{0.715608in}}%
\pgfpathcurveto{\pgfqpoint{1.063182in}{0.723422in}}{\pgfqpoint{1.067573in}{0.734021in}}{\pgfqpoint{1.067573in}{0.745071in}}%
\pgfpathcurveto{\pgfqpoint{1.067573in}{0.756121in}}{\pgfqpoint{1.063182in}{0.766720in}}{\pgfqpoint{1.055369in}{0.774534in}}%
\pgfpathcurveto{\pgfqpoint{1.047555in}{0.782348in}}{\pgfqpoint{1.036956in}{0.786738in}}{\pgfqpoint{1.025906in}{0.786738in}}%
\pgfpathcurveto{\pgfqpoint{1.014856in}{0.786738in}}{\pgfqpoint{1.004257in}{0.782348in}}{\pgfqpoint{0.996443in}{0.774534in}}%
\pgfpathcurveto{\pgfqpoint{0.988630in}{0.766720in}}{\pgfqpoint{0.984239in}{0.756121in}}{\pgfqpoint{0.984239in}{0.745071in}}%
\pgfpathcurveto{\pgfqpoint{0.984239in}{0.734021in}}{\pgfqpoint{0.988630in}{0.723422in}}{\pgfqpoint{0.996443in}{0.715608in}}%
\pgfpathcurveto{\pgfqpoint{1.004257in}{0.707795in}}{\pgfqpoint{1.014856in}{0.703405in}}{\pgfqpoint{1.025906in}{0.703405in}}%
\pgfpathclose%
\pgfusepath{stroke,fill}%
\end{pgfscope}%
\begin{pgfscope}%
\pgfpathrectangle{\pgfqpoint{0.800000in}{0.528000in}}{\pgfqpoint{4.960000in}{3.696000in}}%
\pgfusepath{clip}%
\pgfsetbuttcap%
\pgfsetroundjoin%
\definecolor{currentfill}{rgb}{0.000000,0.000000,0.000000}%
\pgfsetfillcolor{currentfill}%
\pgfsetlinewidth{1.003750pt}%
\definecolor{currentstroke}{rgb}{0.000000,0.000000,0.000000}%
\pgfsetstrokecolor{currentstroke}%
\pgfsetdash{}{0pt}%
\pgfpathmoveto{\pgfqpoint{1.025906in}{0.693696in}}%
\pgfpathcurveto{\pgfqpoint{1.036956in}{0.693696in}}{\pgfqpoint{1.047555in}{0.698086in}}{\pgfqpoint{1.055369in}{0.705900in}}%
\pgfpathcurveto{\pgfqpoint{1.063182in}{0.713713in}}{\pgfqpoint{1.067573in}{0.724312in}}{\pgfqpoint{1.067573in}{0.735362in}}%
\pgfpathcurveto{\pgfqpoint{1.067573in}{0.746412in}}{\pgfqpoint{1.063182in}{0.757011in}}{\pgfqpoint{1.055369in}{0.764825in}}%
\pgfpathcurveto{\pgfqpoint{1.047555in}{0.772639in}}{\pgfqpoint{1.036956in}{0.777029in}}{\pgfqpoint{1.025906in}{0.777029in}}%
\pgfpathcurveto{\pgfqpoint{1.014856in}{0.777029in}}{\pgfqpoint{1.004257in}{0.772639in}}{\pgfqpoint{0.996443in}{0.764825in}}%
\pgfpathcurveto{\pgfqpoint{0.988630in}{0.757011in}}{\pgfqpoint{0.984239in}{0.746412in}}{\pgfqpoint{0.984239in}{0.735362in}}%
\pgfpathcurveto{\pgfqpoint{0.984239in}{0.724312in}}{\pgfqpoint{0.988630in}{0.713713in}}{\pgfqpoint{0.996443in}{0.705900in}}%
\pgfpathcurveto{\pgfqpoint{1.004257in}{0.698086in}}{\pgfqpoint{1.014856in}{0.693696in}}{\pgfqpoint{1.025906in}{0.693696in}}%
\pgfpathclose%
\pgfusepath{stroke,fill}%
\end{pgfscope}%
\begin{pgfscope}%
\pgfpathrectangle{\pgfqpoint{0.800000in}{0.528000in}}{\pgfqpoint{4.960000in}{3.696000in}}%
\pgfusepath{clip}%
\pgfsetbuttcap%
\pgfsetroundjoin%
\definecolor{currentfill}{rgb}{0.000000,0.000000,0.000000}%
\pgfsetfillcolor{currentfill}%
\pgfsetlinewidth{1.003750pt}%
\definecolor{currentstroke}{rgb}{0.000000,0.000000,0.000000}%
\pgfsetstrokecolor{currentstroke}%
\pgfsetdash{}{0pt}%
\pgfpathmoveto{\pgfqpoint{1.025906in}{0.683987in}}%
\pgfpathcurveto{\pgfqpoint{1.036956in}{0.683987in}}{\pgfqpoint{1.047555in}{0.688377in}}{\pgfqpoint{1.055369in}{0.696191in}}%
\pgfpathcurveto{\pgfqpoint{1.063182in}{0.704004in}}{\pgfqpoint{1.067573in}{0.714603in}}{\pgfqpoint{1.067573in}{0.725653in}}%
\pgfpathcurveto{\pgfqpoint{1.067573in}{0.736703in}}{\pgfqpoint{1.063182in}{0.747302in}}{\pgfqpoint{1.055369in}{0.755116in}}%
\pgfpathcurveto{\pgfqpoint{1.047555in}{0.762930in}}{\pgfqpoint{1.036956in}{0.767320in}}{\pgfqpoint{1.025906in}{0.767320in}}%
\pgfpathcurveto{\pgfqpoint{1.014856in}{0.767320in}}{\pgfqpoint{1.004257in}{0.762930in}}{\pgfqpoint{0.996443in}{0.755116in}}%
\pgfpathcurveto{\pgfqpoint{0.988630in}{0.747302in}}{\pgfqpoint{0.984239in}{0.736703in}}{\pgfqpoint{0.984239in}{0.725653in}}%
\pgfpathcurveto{\pgfqpoint{0.984239in}{0.714603in}}{\pgfqpoint{0.988630in}{0.704004in}}{\pgfqpoint{0.996443in}{0.696191in}}%
\pgfpathcurveto{\pgfqpoint{1.004257in}{0.688377in}}{\pgfqpoint{1.014856in}{0.683987in}}{\pgfqpoint{1.025906in}{0.683987in}}%
\pgfpathclose%
\pgfusepath{stroke,fill}%
\end{pgfscope}%
\begin{pgfscope}%
\pgfpathrectangle{\pgfqpoint{0.800000in}{0.528000in}}{\pgfqpoint{4.960000in}{3.696000in}}%
\pgfusepath{clip}%
\pgfsetbuttcap%
\pgfsetroundjoin%
\definecolor{currentfill}{rgb}{0.000000,0.000000,0.000000}%
\pgfsetfillcolor{currentfill}%
\pgfsetlinewidth{1.003750pt}%
\definecolor{currentstroke}{rgb}{0.000000,0.000000,0.000000}%
\pgfsetstrokecolor{currentstroke}%
\pgfsetdash{}{0pt}%
\pgfpathmoveto{\pgfqpoint{1.025906in}{0.693696in}}%
\pgfpathcurveto{\pgfqpoint{1.036956in}{0.693696in}}{\pgfqpoint{1.047555in}{0.698086in}}{\pgfqpoint{1.055369in}{0.705900in}}%
\pgfpathcurveto{\pgfqpoint{1.063182in}{0.713713in}}{\pgfqpoint{1.067573in}{0.724312in}}{\pgfqpoint{1.067573in}{0.735362in}}%
\pgfpathcurveto{\pgfqpoint{1.067573in}{0.746412in}}{\pgfqpoint{1.063182in}{0.757011in}}{\pgfqpoint{1.055369in}{0.764825in}}%
\pgfpathcurveto{\pgfqpoint{1.047555in}{0.772639in}}{\pgfqpoint{1.036956in}{0.777029in}}{\pgfqpoint{1.025906in}{0.777029in}}%
\pgfpathcurveto{\pgfqpoint{1.014856in}{0.777029in}}{\pgfqpoint{1.004257in}{0.772639in}}{\pgfqpoint{0.996443in}{0.764825in}}%
\pgfpathcurveto{\pgfqpoint{0.988630in}{0.757011in}}{\pgfqpoint{0.984239in}{0.746412in}}{\pgfqpoint{0.984239in}{0.735362in}}%
\pgfpathcurveto{\pgfqpoint{0.984239in}{0.724312in}}{\pgfqpoint{0.988630in}{0.713713in}}{\pgfqpoint{0.996443in}{0.705900in}}%
\pgfpathcurveto{\pgfqpoint{1.004257in}{0.698086in}}{\pgfqpoint{1.014856in}{0.693696in}}{\pgfqpoint{1.025906in}{0.693696in}}%
\pgfpathclose%
\pgfusepath{stroke,fill}%
\end{pgfscope}%
\begin{pgfscope}%
\pgfpathrectangle{\pgfqpoint{0.800000in}{0.528000in}}{\pgfqpoint{4.960000in}{3.696000in}}%
\pgfusepath{clip}%
\pgfsetbuttcap%
\pgfsetroundjoin%
\definecolor{currentfill}{rgb}{0.000000,0.000000,0.000000}%
\pgfsetfillcolor{currentfill}%
\pgfsetlinewidth{1.003750pt}%
\definecolor{currentstroke}{rgb}{0.000000,0.000000,0.000000}%
\pgfsetstrokecolor{currentstroke}%
\pgfsetdash{}{0pt}%
\pgfpathmoveto{\pgfqpoint{1.025906in}{0.703405in}}%
\pgfpathcurveto{\pgfqpoint{1.036956in}{0.703405in}}{\pgfqpoint{1.047555in}{0.707795in}}{\pgfqpoint{1.055369in}{0.715608in}}%
\pgfpathcurveto{\pgfqpoint{1.063182in}{0.723422in}}{\pgfqpoint{1.067573in}{0.734021in}}{\pgfqpoint{1.067573in}{0.745071in}}%
\pgfpathcurveto{\pgfqpoint{1.067573in}{0.756121in}}{\pgfqpoint{1.063182in}{0.766720in}}{\pgfqpoint{1.055369in}{0.774534in}}%
\pgfpathcurveto{\pgfqpoint{1.047555in}{0.782348in}}{\pgfqpoint{1.036956in}{0.786738in}}{\pgfqpoint{1.025906in}{0.786738in}}%
\pgfpathcurveto{\pgfqpoint{1.014856in}{0.786738in}}{\pgfqpoint{1.004257in}{0.782348in}}{\pgfqpoint{0.996443in}{0.774534in}}%
\pgfpathcurveto{\pgfqpoint{0.988630in}{0.766720in}}{\pgfqpoint{0.984239in}{0.756121in}}{\pgfqpoint{0.984239in}{0.745071in}}%
\pgfpathcurveto{\pgfqpoint{0.984239in}{0.734021in}}{\pgfqpoint{0.988630in}{0.723422in}}{\pgfqpoint{0.996443in}{0.715608in}}%
\pgfpathcurveto{\pgfqpoint{1.004257in}{0.707795in}}{\pgfqpoint{1.014856in}{0.703405in}}{\pgfqpoint{1.025906in}{0.703405in}}%
\pgfpathclose%
\pgfusepath{stroke,fill}%
\end{pgfscope}%
\begin{pgfscope}%
\pgfpathrectangle{\pgfqpoint{0.800000in}{0.528000in}}{\pgfqpoint{4.960000in}{3.696000in}}%
\pgfusepath{clip}%
\pgfsetbuttcap%
\pgfsetroundjoin%
\definecolor{currentfill}{rgb}{0.000000,0.000000,0.000000}%
\pgfsetfillcolor{currentfill}%
\pgfsetlinewidth{1.003750pt}%
\definecolor{currentstroke}{rgb}{0.000000,0.000000,0.000000}%
\pgfsetstrokecolor{currentstroke}%
\pgfsetdash{}{0pt}%
\pgfpathmoveto{\pgfqpoint{1.025906in}{0.683987in}}%
\pgfpathcurveto{\pgfqpoint{1.036956in}{0.683987in}}{\pgfqpoint{1.047555in}{0.688377in}}{\pgfqpoint{1.055369in}{0.696191in}}%
\pgfpathcurveto{\pgfqpoint{1.063182in}{0.704004in}}{\pgfqpoint{1.067573in}{0.714603in}}{\pgfqpoint{1.067573in}{0.725653in}}%
\pgfpathcurveto{\pgfqpoint{1.067573in}{0.736703in}}{\pgfqpoint{1.063182in}{0.747302in}}{\pgfqpoint{1.055369in}{0.755116in}}%
\pgfpathcurveto{\pgfqpoint{1.047555in}{0.762930in}}{\pgfqpoint{1.036956in}{0.767320in}}{\pgfqpoint{1.025906in}{0.767320in}}%
\pgfpathcurveto{\pgfqpoint{1.014856in}{0.767320in}}{\pgfqpoint{1.004257in}{0.762930in}}{\pgfqpoint{0.996443in}{0.755116in}}%
\pgfpathcurveto{\pgfqpoint{0.988630in}{0.747302in}}{\pgfqpoint{0.984239in}{0.736703in}}{\pgfqpoint{0.984239in}{0.725653in}}%
\pgfpathcurveto{\pgfqpoint{0.984239in}{0.714603in}}{\pgfqpoint{0.988630in}{0.704004in}}{\pgfqpoint{0.996443in}{0.696191in}}%
\pgfpathcurveto{\pgfqpoint{1.004257in}{0.688377in}}{\pgfqpoint{1.014856in}{0.683987in}}{\pgfqpoint{1.025906in}{0.683987in}}%
\pgfpathclose%
\pgfusepath{stroke,fill}%
\end{pgfscope}%
\begin{pgfscope}%
\pgfpathrectangle{\pgfqpoint{0.800000in}{0.528000in}}{\pgfqpoint{4.960000in}{3.696000in}}%
\pgfusepath{clip}%
\pgfsetbuttcap%
\pgfsetroundjoin%
\definecolor{currentfill}{rgb}{0.000000,0.000000,0.000000}%
\pgfsetfillcolor{currentfill}%
\pgfsetlinewidth{1.003750pt}%
\definecolor{currentstroke}{rgb}{0.000000,0.000000,0.000000}%
\pgfsetstrokecolor{currentstroke}%
\pgfsetdash{}{0pt}%
\pgfpathmoveto{\pgfqpoint{1.025906in}{0.674278in}}%
\pgfpathcurveto{\pgfqpoint{1.036956in}{0.674278in}}{\pgfqpoint{1.047555in}{0.678668in}}{\pgfqpoint{1.055369in}{0.686482in}}%
\pgfpathcurveto{\pgfqpoint{1.063182in}{0.694295in}}{\pgfqpoint{1.067573in}{0.704894in}}{\pgfqpoint{1.067573in}{0.715944in}}%
\pgfpathcurveto{\pgfqpoint{1.067573in}{0.726994in}}{\pgfqpoint{1.063182in}{0.737594in}}{\pgfqpoint{1.055369in}{0.745407in}}%
\pgfpathcurveto{\pgfqpoint{1.047555in}{0.753221in}}{\pgfqpoint{1.036956in}{0.757611in}}{\pgfqpoint{1.025906in}{0.757611in}}%
\pgfpathcurveto{\pgfqpoint{1.014856in}{0.757611in}}{\pgfqpoint{1.004257in}{0.753221in}}{\pgfqpoint{0.996443in}{0.745407in}}%
\pgfpathcurveto{\pgfqpoint{0.988630in}{0.737594in}}{\pgfqpoint{0.984239in}{0.726994in}}{\pgfqpoint{0.984239in}{0.715944in}}%
\pgfpathcurveto{\pgfqpoint{0.984239in}{0.704894in}}{\pgfqpoint{0.988630in}{0.694295in}}{\pgfqpoint{0.996443in}{0.686482in}}%
\pgfpathcurveto{\pgfqpoint{1.004257in}{0.678668in}}{\pgfqpoint{1.014856in}{0.674278in}}{\pgfqpoint{1.025906in}{0.674278in}}%
\pgfpathclose%
\pgfusepath{stroke,fill}%
\end{pgfscope}%
\begin{pgfscope}%
\pgfpathrectangle{\pgfqpoint{0.800000in}{0.528000in}}{\pgfqpoint{4.960000in}{3.696000in}}%
\pgfusepath{clip}%
\pgfsetbuttcap%
\pgfsetroundjoin%
\definecolor{currentfill}{rgb}{0.000000,0.000000,0.000000}%
\pgfsetfillcolor{currentfill}%
\pgfsetlinewidth{1.003750pt}%
\definecolor{currentstroke}{rgb}{0.000000,0.000000,0.000000}%
\pgfsetstrokecolor{currentstroke}%
\pgfsetdash{}{0pt}%
\pgfpathmoveto{\pgfqpoint{1.025906in}{0.664569in}}%
\pgfpathcurveto{\pgfqpoint{1.036956in}{0.664569in}}{\pgfqpoint{1.047555in}{0.668959in}}{\pgfqpoint{1.055369in}{0.676773in}}%
\pgfpathcurveto{\pgfqpoint{1.063182in}{0.684586in}}{\pgfqpoint{1.067573in}{0.695185in}}{\pgfqpoint{1.067573in}{0.706235in}}%
\pgfpathcurveto{\pgfqpoint{1.067573in}{0.717286in}}{\pgfqpoint{1.063182in}{0.727885in}}{\pgfqpoint{1.055369in}{0.735698in}}%
\pgfpathcurveto{\pgfqpoint{1.047555in}{0.743512in}}{\pgfqpoint{1.036956in}{0.747902in}}{\pgfqpoint{1.025906in}{0.747902in}}%
\pgfpathcurveto{\pgfqpoint{1.014856in}{0.747902in}}{\pgfqpoint{1.004257in}{0.743512in}}{\pgfqpoint{0.996443in}{0.735698in}}%
\pgfpathcurveto{\pgfqpoint{0.988630in}{0.727885in}}{\pgfqpoint{0.984239in}{0.717286in}}{\pgfqpoint{0.984239in}{0.706235in}}%
\pgfpathcurveto{\pgfqpoint{0.984239in}{0.695185in}}{\pgfqpoint{0.988630in}{0.684586in}}{\pgfqpoint{0.996443in}{0.676773in}}%
\pgfpathcurveto{\pgfqpoint{1.004257in}{0.668959in}}{\pgfqpoint{1.014856in}{0.664569in}}{\pgfqpoint{1.025906in}{0.664569in}}%
\pgfpathclose%
\pgfusepath{stroke,fill}%
\end{pgfscope}%
\begin{pgfscope}%
\pgfpathrectangle{\pgfqpoint{0.800000in}{0.528000in}}{\pgfqpoint{4.960000in}{3.696000in}}%
\pgfusepath{clip}%
\pgfsetbuttcap%
\pgfsetroundjoin%
\definecolor{currentfill}{rgb}{0.000000,0.000000,0.000000}%
\pgfsetfillcolor{currentfill}%
\pgfsetlinewidth{1.003750pt}%
\definecolor{currentstroke}{rgb}{0.000000,0.000000,0.000000}%
\pgfsetstrokecolor{currentstroke}%
\pgfsetdash{}{0pt}%
\pgfpathmoveto{\pgfqpoint{1.025906in}{0.683987in}}%
\pgfpathcurveto{\pgfqpoint{1.036956in}{0.683987in}}{\pgfqpoint{1.047555in}{0.688377in}}{\pgfqpoint{1.055369in}{0.696191in}}%
\pgfpathcurveto{\pgfqpoint{1.063182in}{0.704004in}}{\pgfqpoint{1.067573in}{0.714603in}}{\pgfqpoint{1.067573in}{0.725653in}}%
\pgfpathcurveto{\pgfqpoint{1.067573in}{0.736703in}}{\pgfqpoint{1.063182in}{0.747302in}}{\pgfqpoint{1.055369in}{0.755116in}}%
\pgfpathcurveto{\pgfqpoint{1.047555in}{0.762930in}}{\pgfqpoint{1.036956in}{0.767320in}}{\pgfqpoint{1.025906in}{0.767320in}}%
\pgfpathcurveto{\pgfqpoint{1.014856in}{0.767320in}}{\pgfqpoint{1.004257in}{0.762930in}}{\pgfqpoint{0.996443in}{0.755116in}}%
\pgfpathcurveto{\pgfqpoint{0.988630in}{0.747302in}}{\pgfqpoint{0.984239in}{0.736703in}}{\pgfqpoint{0.984239in}{0.725653in}}%
\pgfpathcurveto{\pgfqpoint{0.984239in}{0.714603in}}{\pgfqpoint{0.988630in}{0.704004in}}{\pgfqpoint{0.996443in}{0.696191in}}%
\pgfpathcurveto{\pgfqpoint{1.004257in}{0.688377in}}{\pgfqpoint{1.014856in}{0.683987in}}{\pgfqpoint{1.025906in}{0.683987in}}%
\pgfpathclose%
\pgfusepath{stroke,fill}%
\end{pgfscope}%
\begin{pgfscope}%
\pgfpathrectangle{\pgfqpoint{0.800000in}{0.528000in}}{\pgfqpoint{4.960000in}{3.696000in}}%
\pgfusepath{clip}%
\pgfsetbuttcap%
\pgfsetroundjoin%
\definecolor{currentfill}{rgb}{0.000000,0.000000,0.000000}%
\pgfsetfillcolor{currentfill}%
\pgfsetlinewidth{1.003750pt}%
\definecolor{currentstroke}{rgb}{0.000000,0.000000,0.000000}%
\pgfsetstrokecolor{currentstroke}%
\pgfsetdash{}{0pt}%
\pgfpathmoveto{\pgfqpoint{1.025906in}{0.703405in}}%
\pgfpathcurveto{\pgfqpoint{1.036956in}{0.703405in}}{\pgfqpoint{1.047555in}{0.707795in}}{\pgfqpoint{1.055369in}{0.715608in}}%
\pgfpathcurveto{\pgfqpoint{1.063182in}{0.723422in}}{\pgfqpoint{1.067573in}{0.734021in}}{\pgfqpoint{1.067573in}{0.745071in}}%
\pgfpathcurveto{\pgfqpoint{1.067573in}{0.756121in}}{\pgfqpoint{1.063182in}{0.766720in}}{\pgfqpoint{1.055369in}{0.774534in}}%
\pgfpathcurveto{\pgfqpoint{1.047555in}{0.782348in}}{\pgfqpoint{1.036956in}{0.786738in}}{\pgfqpoint{1.025906in}{0.786738in}}%
\pgfpathcurveto{\pgfqpoint{1.014856in}{0.786738in}}{\pgfqpoint{1.004257in}{0.782348in}}{\pgfqpoint{0.996443in}{0.774534in}}%
\pgfpathcurveto{\pgfqpoint{0.988630in}{0.766720in}}{\pgfqpoint{0.984239in}{0.756121in}}{\pgfqpoint{0.984239in}{0.745071in}}%
\pgfpathcurveto{\pgfqpoint{0.984239in}{0.734021in}}{\pgfqpoint{0.988630in}{0.723422in}}{\pgfqpoint{0.996443in}{0.715608in}}%
\pgfpathcurveto{\pgfqpoint{1.004257in}{0.707795in}}{\pgfqpoint{1.014856in}{0.703405in}}{\pgfqpoint{1.025906in}{0.703405in}}%
\pgfpathclose%
\pgfusepath{stroke,fill}%
\end{pgfscope}%
\begin{pgfscope}%
\pgfpathrectangle{\pgfqpoint{0.800000in}{0.528000in}}{\pgfqpoint{4.960000in}{3.696000in}}%
\pgfusepath{clip}%
\pgfsetbuttcap%
\pgfsetroundjoin%
\definecolor{currentfill}{rgb}{0.000000,0.000000,0.000000}%
\pgfsetfillcolor{currentfill}%
\pgfsetlinewidth{1.003750pt}%
\definecolor{currentstroke}{rgb}{0.000000,0.000000,0.000000}%
\pgfsetstrokecolor{currentstroke}%
\pgfsetdash{}{0pt}%
\pgfpathmoveto{\pgfqpoint{1.025906in}{0.703405in}}%
\pgfpathcurveto{\pgfqpoint{1.036956in}{0.703405in}}{\pgfqpoint{1.047555in}{0.707795in}}{\pgfqpoint{1.055369in}{0.715608in}}%
\pgfpathcurveto{\pgfqpoint{1.063182in}{0.723422in}}{\pgfqpoint{1.067573in}{0.734021in}}{\pgfqpoint{1.067573in}{0.745071in}}%
\pgfpathcurveto{\pgfqpoint{1.067573in}{0.756121in}}{\pgfqpoint{1.063182in}{0.766720in}}{\pgfqpoint{1.055369in}{0.774534in}}%
\pgfpathcurveto{\pgfqpoint{1.047555in}{0.782348in}}{\pgfqpoint{1.036956in}{0.786738in}}{\pgfqpoint{1.025906in}{0.786738in}}%
\pgfpathcurveto{\pgfqpoint{1.014856in}{0.786738in}}{\pgfqpoint{1.004257in}{0.782348in}}{\pgfqpoint{0.996443in}{0.774534in}}%
\pgfpathcurveto{\pgfqpoint{0.988630in}{0.766720in}}{\pgfqpoint{0.984239in}{0.756121in}}{\pgfqpoint{0.984239in}{0.745071in}}%
\pgfpathcurveto{\pgfqpoint{0.984239in}{0.734021in}}{\pgfqpoint{0.988630in}{0.723422in}}{\pgfqpoint{0.996443in}{0.715608in}}%
\pgfpathcurveto{\pgfqpoint{1.004257in}{0.707795in}}{\pgfqpoint{1.014856in}{0.703405in}}{\pgfqpoint{1.025906in}{0.703405in}}%
\pgfpathclose%
\pgfusepath{stroke,fill}%
\end{pgfscope}%
\begin{pgfscope}%
\pgfpathrectangle{\pgfqpoint{0.800000in}{0.528000in}}{\pgfqpoint{4.960000in}{3.696000in}}%
\pgfusepath{clip}%
\pgfsetbuttcap%
\pgfsetroundjoin%
\definecolor{currentfill}{rgb}{0.000000,0.000000,0.000000}%
\pgfsetfillcolor{currentfill}%
\pgfsetlinewidth{1.003750pt}%
\definecolor{currentstroke}{rgb}{0.000000,0.000000,0.000000}%
\pgfsetstrokecolor{currentstroke}%
\pgfsetdash{}{0pt}%
\pgfpathmoveto{\pgfqpoint{1.025906in}{0.703405in}}%
\pgfpathcurveto{\pgfqpoint{1.036956in}{0.703405in}}{\pgfqpoint{1.047555in}{0.707795in}}{\pgfqpoint{1.055369in}{0.715608in}}%
\pgfpathcurveto{\pgfqpoint{1.063182in}{0.723422in}}{\pgfqpoint{1.067573in}{0.734021in}}{\pgfqpoint{1.067573in}{0.745071in}}%
\pgfpathcurveto{\pgfqpoint{1.067573in}{0.756121in}}{\pgfqpoint{1.063182in}{0.766720in}}{\pgfqpoint{1.055369in}{0.774534in}}%
\pgfpathcurveto{\pgfqpoint{1.047555in}{0.782348in}}{\pgfqpoint{1.036956in}{0.786738in}}{\pgfqpoint{1.025906in}{0.786738in}}%
\pgfpathcurveto{\pgfqpoint{1.014856in}{0.786738in}}{\pgfqpoint{1.004257in}{0.782348in}}{\pgfqpoint{0.996443in}{0.774534in}}%
\pgfpathcurveto{\pgfqpoint{0.988630in}{0.766720in}}{\pgfqpoint{0.984239in}{0.756121in}}{\pgfqpoint{0.984239in}{0.745071in}}%
\pgfpathcurveto{\pgfqpoint{0.984239in}{0.734021in}}{\pgfqpoint{0.988630in}{0.723422in}}{\pgfqpoint{0.996443in}{0.715608in}}%
\pgfpathcurveto{\pgfqpoint{1.004257in}{0.707795in}}{\pgfqpoint{1.014856in}{0.703405in}}{\pgfqpoint{1.025906in}{0.703405in}}%
\pgfpathclose%
\pgfusepath{stroke,fill}%
\end{pgfscope}%
\begin{pgfscope}%
\pgfpathrectangle{\pgfqpoint{0.800000in}{0.528000in}}{\pgfqpoint{4.960000in}{3.696000in}}%
\pgfusepath{clip}%
\pgfsetbuttcap%
\pgfsetroundjoin%
\definecolor{currentfill}{rgb}{0.000000,0.000000,0.000000}%
\pgfsetfillcolor{currentfill}%
\pgfsetlinewidth{1.003750pt}%
\definecolor{currentstroke}{rgb}{0.000000,0.000000,0.000000}%
\pgfsetstrokecolor{currentstroke}%
\pgfsetdash{}{0pt}%
\pgfpathmoveto{\pgfqpoint{1.025906in}{0.703405in}}%
\pgfpathcurveto{\pgfqpoint{1.036956in}{0.703405in}}{\pgfqpoint{1.047555in}{0.707795in}}{\pgfqpoint{1.055369in}{0.715608in}}%
\pgfpathcurveto{\pgfqpoint{1.063182in}{0.723422in}}{\pgfqpoint{1.067573in}{0.734021in}}{\pgfqpoint{1.067573in}{0.745071in}}%
\pgfpathcurveto{\pgfqpoint{1.067573in}{0.756121in}}{\pgfqpoint{1.063182in}{0.766720in}}{\pgfqpoint{1.055369in}{0.774534in}}%
\pgfpathcurveto{\pgfqpoint{1.047555in}{0.782348in}}{\pgfqpoint{1.036956in}{0.786738in}}{\pgfqpoint{1.025906in}{0.786738in}}%
\pgfpathcurveto{\pgfqpoint{1.014856in}{0.786738in}}{\pgfqpoint{1.004257in}{0.782348in}}{\pgfqpoint{0.996443in}{0.774534in}}%
\pgfpathcurveto{\pgfqpoint{0.988630in}{0.766720in}}{\pgfqpoint{0.984239in}{0.756121in}}{\pgfqpoint{0.984239in}{0.745071in}}%
\pgfpathcurveto{\pgfqpoint{0.984239in}{0.734021in}}{\pgfqpoint{0.988630in}{0.723422in}}{\pgfqpoint{0.996443in}{0.715608in}}%
\pgfpathcurveto{\pgfqpoint{1.004257in}{0.707795in}}{\pgfqpoint{1.014856in}{0.703405in}}{\pgfqpoint{1.025906in}{0.703405in}}%
\pgfpathclose%
\pgfusepath{stroke,fill}%
\end{pgfscope}%
\begin{pgfscope}%
\pgfpathrectangle{\pgfqpoint{0.800000in}{0.528000in}}{\pgfqpoint{4.960000in}{3.696000in}}%
\pgfusepath{clip}%
\pgfsetbuttcap%
\pgfsetroundjoin%
\definecolor{currentfill}{rgb}{0.000000,0.000000,0.000000}%
\pgfsetfillcolor{currentfill}%
\pgfsetlinewidth{1.003750pt}%
\definecolor{currentstroke}{rgb}{0.000000,0.000000,0.000000}%
\pgfsetstrokecolor{currentstroke}%
\pgfsetdash{}{0pt}%
\pgfpathmoveto{\pgfqpoint{1.025906in}{0.693696in}}%
\pgfpathcurveto{\pgfqpoint{1.036956in}{0.693696in}}{\pgfqpoint{1.047555in}{0.698086in}}{\pgfqpoint{1.055369in}{0.705900in}}%
\pgfpathcurveto{\pgfqpoint{1.063182in}{0.713713in}}{\pgfqpoint{1.067573in}{0.724312in}}{\pgfqpoint{1.067573in}{0.735362in}}%
\pgfpathcurveto{\pgfqpoint{1.067573in}{0.746412in}}{\pgfqpoint{1.063182in}{0.757011in}}{\pgfqpoint{1.055369in}{0.764825in}}%
\pgfpathcurveto{\pgfqpoint{1.047555in}{0.772639in}}{\pgfqpoint{1.036956in}{0.777029in}}{\pgfqpoint{1.025906in}{0.777029in}}%
\pgfpathcurveto{\pgfqpoint{1.014856in}{0.777029in}}{\pgfqpoint{1.004257in}{0.772639in}}{\pgfqpoint{0.996443in}{0.764825in}}%
\pgfpathcurveto{\pgfqpoint{0.988630in}{0.757011in}}{\pgfqpoint{0.984239in}{0.746412in}}{\pgfqpoint{0.984239in}{0.735362in}}%
\pgfpathcurveto{\pgfqpoint{0.984239in}{0.724312in}}{\pgfqpoint{0.988630in}{0.713713in}}{\pgfqpoint{0.996443in}{0.705900in}}%
\pgfpathcurveto{\pgfqpoint{1.004257in}{0.698086in}}{\pgfqpoint{1.014856in}{0.693696in}}{\pgfqpoint{1.025906in}{0.693696in}}%
\pgfpathclose%
\pgfusepath{stroke,fill}%
\end{pgfscope}%
\begin{pgfscope}%
\pgfpathrectangle{\pgfqpoint{0.800000in}{0.528000in}}{\pgfqpoint{4.960000in}{3.696000in}}%
\pgfusepath{clip}%
\pgfsetbuttcap%
\pgfsetroundjoin%
\definecolor{currentfill}{rgb}{0.000000,0.000000,0.000000}%
\pgfsetfillcolor{currentfill}%
\pgfsetlinewidth{1.003750pt}%
\definecolor{currentstroke}{rgb}{0.000000,0.000000,0.000000}%
\pgfsetstrokecolor{currentstroke}%
\pgfsetdash{}{0pt}%
\pgfpathmoveto{\pgfqpoint{1.025906in}{0.703405in}}%
\pgfpathcurveto{\pgfqpoint{1.036956in}{0.703405in}}{\pgfqpoint{1.047555in}{0.707795in}}{\pgfqpoint{1.055369in}{0.715608in}}%
\pgfpathcurveto{\pgfqpoint{1.063182in}{0.723422in}}{\pgfqpoint{1.067573in}{0.734021in}}{\pgfqpoint{1.067573in}{0.745071in}}%
\pgfpathcurveto{\pgfqpoint{1.067573in}{0.756121in}}{\pgfqpoint{1.063182in}{0.766720in}}{\pgfqpoint{1.055369in}{0.774534in}}%
\pgfpathcurveto{\pgfqpoint{1.047555in}{0.782348in}}{\pgfqpoint{1.036956in}{0.786738in}}{\pgfqpoint{1.025906in}{0.786738in}}%
\pgfpathcurveto{\pgfqpoint{1.014856in}{0.786738in}}{\pgfqpoint{1.004257in}{0.782348in}}{\pgfqpoint{0.996443in}{0.774534in}}%
\pgfpathcurveto{\pgfqpoint{0.988630in}{0.766720in}}{\pgfqpoint{0.984239in}{0.756121in}}{\pgfqpoint{0.984239in}{0.745071in}}%
\pgfpathcurveto{\pgfqpoint{0.984239in}{0.734021in}}{\pgfqpoint{0.988630in}{0.723422in}}{\pgfqpoint{0.996443in}{0.715608in}}%
\pgfpathcurveto{\pgfqpoint{1.004257in}{0.707795in}}{\pgfqpoint{1.014856in}{0.703405in}}{\pgfqpoint{1.025906in}{0.703405in}}%
\pgfpathclose%
\pgfusepath{stroke,fill}%
\end{pgfscope}%
\begin{pgfscope}%
\pgfpathrectangle{\pgfqpoint{0.800000in}{0.528000in}}{\pgfqpoint{4.960000in}{3.696000in}}%
\pgfusepath{clip}%
\pgfsetbuttcap%
\pgfsetroundjoin%
\definecolor{currentfill}{rgb}{0.000000,0.000000,0.000000}%
\pgfsetfillcolor{currentfill}%
\pgfsetlinewidth{1.003750pt}%
\definecolor{currentstroke}{rgb}{0.000000,0.000000,0.000000}%
\pgfsetstrokecolor{currentstroke}%
\pgfsetdash{}{0pt}%
\pgfpathmoveto{\pgfqpoint{1.025906in}{0.674278in}}%
\pgfpathcurveto{\pgfqpoint{1.036956in}{0.674278in}}{\pgfqpoint{1.047555in}{0.678668in}}{\pgfqpoint{1.055369in}{0.686482in}}%
\pgfpathcurveto{\pgfqpoint{1.063182in}{0.694295in}}{\pgfqpoint{1.067573in}{0.704894in}}{\pgfqpoint{1.067573in}{0.715944in}}%
\pgfpathcurveto{\pgfqpoint{1.067573in}{0.726994in}}{\pgfqpoint{1.063182in}{0.737594in}}{\pgfqpoint{1.055369in}{0.745407in}}%
\pgfpathcurveto{\pgfqpoint{1.047555in}{0.753221in}}{\pgfqpoint{1.036956in}{0.757611in}}{\pgfqpoint{1.025906in}{0.757611in}}%
\pgfpathcurveto{\pgfqpoint{1.014856in}{0.757611in}}{\pgfqpoint{1.004257in}{0.753221in}}{\pgfqpoint{0.996443in}{0.745407in}}%
\pgfpathcurveto{\pgfqpoint{0.988630in}{0.737594in}}{\pgfqpoint{0.984239in}{0.726994in}}{\pgfqpoint{0.984239in}{0.715944in}}%
\pgfpathcurveto{\pgfqpoint{0.984239in}{0.704894in}}{\pgfqpoint{0.988630in}{0.694295in}}{\pgfqpoint{0.996443in}{0.686482in}}%
\pgfpathcurveto{\pgfqpoint{1.004257in}{0.678668in}}{\pgfqpoint{1.014856in}{0.674278in}}{\pgfqpoint{1.025906in}{0.674278in}}%
\pgfpathclose%
\pgfusepath{stroke,fill}%
\end{pgfscope}%
\begin{pgfscope}%
\pgfpathrectangle{\pgfqpoint{0.800000in}{0.528000in}}{\pgfqpoint{4.960000in}{3.696000in}}%
\pgfusepath{clip}%
\pgfsetbuttcap%
\pgfsetroundjoin%
\definecolor{currentfill}{rgb}{0.000000,0.000000,0.000000}%
\pgfsetfillcolor{currentfill}%
\pgfsetlinewidth{1.003750pt}%
\definecolor{currentstroke}{rgb}{0.000000,0.000000,0.000000}%
\pgfsetstrokecolor{currentstroke}%
\pgfsetdash{}{0pt}%
\pgfpathmoveto{\pgfqpoint{1.025906in}{0.674278in}}%
\pgfpathcurveto{\pgfqpoint{1.036956in}{0.674278in}}{\pgfqpoint{1.047555in}{0.678668in}}{\pgfqpoint{1.055369in}{0.686482in}}%
\pgfpathcurveto{\pgfqpoint{1.063182in}{0.694295in}}{\pgfqpoint{1.067573in}{0.704894in}}{\pgfqpoint{1.067573in}{0.715944in}}%
\pgfpathcurveto{\pgfqpoint{1.067573in}{0.726994in}}{\pgfqpoint{1.063182in}{0.737594in}}{\pgfqpoint{1.055369in}{0.745407in}}%
\pgfpathcurveto{\pgfqpoint{1.047555in}{0.753221in}}{\pgfqpoint{1.036956in}{0.757611in}}{\pgfqpoint{1.025906in}{0.757611in}}%
\pgfpathcurveto{\pgfqpoint{1.014856in}{0.757611in}}{\pgfqpoint{1.004257in}{0.753221in}}{\pgfqpoint{0.996443in}{0.745407in}}%
\pgfpathcurveto{\pgfqpoint{0.988630in}{0.737594in}}{\pgfqpoint{0.984239in}{0.726994in}}{\pgfqpoint{0.984239in}{0.715944in}}%
\pgfpathcurveto{\pgfqpoint{0.984239in}{0.704894in}}{\pgfqpoint{0.988630in}{0.694295in}}{\pgfqpoint{0.996443in}{0.686482in}}%
\pgfpathcurveto{\pgfqpoint{1.004257in}{0.678668in}}{\pgfqpoint{1.014856in}{0.674278in}}{\pgfqpoint{1.025906in}{0.674278in}}%
\pgfpathclose%
\pgfusepath{stroke,fill}%
\end{pgfscope}%
\begin{pgfscope}%
\pgfpathrectangle{\pgfqpoint{0.800000in}{0.528000in}}{\pgfqpoint{4.960000in}{3.696000in}}%
\pgfusepath{clip}%
\pgfsetbuttcap%
\pgfsetroundjoin%
\definecolor{currentfill}{rgb}{0.000000,0.000000,0.000000}%
\pgfsetfillcolor{currentfill}%
\pgfsetlinewidth{1.003750pt}%
\definecolor{currentstroke}{rgb}{0.000000,0.000000,0.000000}%
\pgfsetstrokecolor{currentstroke}%
\pgfsetdash{}{0pt}%
\pgfpathmoveto{\pgfqpoint{1.025906in}{0.683987in}}%
\pgfpathcurveto{\pgfqpoint{1.036956in}{0.683987in}}{\pgfqpoint{1.047555in}{0.688377in}}{\pgfqpoint{1.055369in}{0.696191in}}%
\pgfpathcurveto{\pgfqpoint{1.063182in}{0.704004in}}{\pgfqpoint{1.067573in}{0.714603in}}{\pgfqpoint{1.067573in}{0.725653in}}%
\pgfpathcurveto{\pgfqpoint{1.067573in}{0.736703in}}{\pgfqpoint{1.063182in}{0.747302in}}{\pgfqpoint{1.055369in}{0.755116in}}%
\pgfpathcurveto{\pgfqpoint{1.047555in}{0.762930in}}{\pgfqpoint{1.036956in}{0.767320in}}{\pgfqpoint{1.025906in}{0.767320in}}%
\pgfpathcurveto{\pgfqpoint{1.014856in}{0.767320in}}{\pgfqpoint{1.004257in}{0.762930in}}{\pgfqpoint{0.996443in}{0.755116in}}%
\pgfpathcurveto{\pgfqpoint{0.988630in}{0.747302in}}{\pgfqpoint{0.984239in}{0.736703in}}{\pgfqpoint{0.984239in}{0.725653in}}%
\pgfpathcurveto{\pgfqpoint{0.984239in}{0.714603in}}{\pgfqpoint{0.988630in}{0.704004in}}{\pgfqpoint{0.996443in}{0.696191in}}%
\pgfpathcurveto{\pgfqpoint{1.004257in}{0.688377in}}{\pgfqpoint{1.014856in}{0.683987in}}{\pgfqpoint{1.025906in}{0.683987in}}%
\pgfpathclose%
\pgfusepath{stroke,fill}%
\end{pgfscope}%
\begin{pgfscope}%
\pgfpathrectangle{\pgfqpoint{0.800000in}{0.528000in}}{\pgfqpoint{4.960000in}{3.696000in}}%
\pgfusepath{clip}%
\pgfsetbuttcap%
\pgfsetroundjoin%
\definecolor{currentfill}{rgb}{0.000000,0.000000,0.000000}%
\pgfsetfillcolor{currentfill}%
\pgfsetlinewidth{1.003750pt}%
\definecolor{currentstroke}{rgb}{0.000000,0.000000,0.000000}%
\pgfsetstrokecolor{currentstroke}%
\pgfsetdash{}{0pt}%
\pgfpathmoveto{\pgfqpoint{1.025906in}{0.683987in}}%
\pgfpathcurveto{\pgfqpoint{1.036956in}{0.683987in}}{\pgfqpoint{1.047555in}{0.688377in}}{\pgfqpoint{1.055369in}{0.696191in}}%
\pgfpathcurveto{\pgfqpoint{1.063182in}{0.704004in}}{\pgfqpoint{1.067573in}{0.714603in}}{\pgfqpoint{1.067573in}{0.725653in}}%
\pgfpathcurveto{\pgfqpoint{1.067573in}{0.736703in}}{\pgfqpoint{1.063182in}{0.747302in}}{\pgfqpoint{1.055369in}{0.755116in}}%
\pgfpathcurveto{\pgfqpoint{1.047555in}{0.762930in}}{\pgfqpoint{1.036956in}{0.767320in}}{\pgfqpoint{1.025906in}{0.767320in}}%
\pgfpathcurveto{\pgfqpoint{1.014856in}{0.767320in}}{\pgfqpoint{1.004257in}{0.762930in}}{\pgfqpoint{0.996443in}{0.755116in}}%
\pgfpathcurveto{\pgfqpoint{0.988630in}{0.747302in}}{\pgfqpoint{0.984239in}{0.736703in}}{\pgfqpoint{0.984239in}{0.725653in}}%
\pgfpathcurveto{\pgfqpoint{0.984239in}{0.714603in}}{\pgfqpoint{0.988630in}{0.704004in}}{\pgfqpoint{0.996443in}{0.696191in}}%
\pgfpathcurveto{\pgfqpoint{1.004257in}{0.688377in}}{\pgfqpoint{1.014856in}{0.683987in}}{\pgfqpoint{1.025906in}{0.683987in}}%
\pgfpathclose%
\pgfusepath{stroke,fill}%
\end{pgfscope}%
\begin{pgfscope}%
\pgfpathrectangle{\pgfqpoint{0.800000in}{0.528000in}}{\pgfqpoint{4.960000in}{3.696000in}}%
\pgfusepath{clip}%
\pgfsetbuttcap%
\pgfsetroundjoin%
\definecolor{currentfill}{rgb}{0.000000,0.000000,0.000000}%
\pgfsetfillcolor{currentfill}%
\pgfsetlinewidth{1.003750pt}%
\definecolor{currentstroke}{rgb}{0.000000,0.000000,0.000000}%
\pgfsetstrokecolor{currentstroke}%
\pgfsetdash{}{0pt}%
\pgfpathmoveto{\pgfqpoint{1.025906in}{0.703405in}}%
\pgfpathcurveto{\pgfqpoint{1.036956in}{0.703405in}}{\pgfqpoint{1.047555in}{0.707795in}}{\pgfqpoint{1.055369in}{0.715608in}}%
\pgfpathcurveto{\pgfqpoint{1.063182in}{0.723422in}}{\pgfqpoint{1.067573in}{0.734021in}}{\pgfqpoint{1.067573in}{0.745071in}}%
\pgfpathcurveto{\pgfqpoint{1.067573in}{0.756121in}}{\pgfqpoint{1.063182in}{0.766720in}}{\pgfqpoint{1.055369in}{0.774534in}}%
\pgfpathcurveto{\pgfqpoint{1.047555in}{0.782348in}}{\pgfqpoint{1.036956in}{0.786738in}}{\pgfqpoint{1.025906in}{0.786738in}}%
\pgfpathcurveto{\pgfqpoint{1.014856in}{0.786738in}}{\pgfqpoint{1.004257in}{0.782348in}}{\pgfqpoint{0.996443in}{0.774534in}}%
\pgfpathcurveto{\pgfqpoint{0.988630in}{0.766720in}}{\pgfqpoint{0.984239in}{0.756121in}}{\pgfqpoint{0.984239in}{0.745071in}}%
\pgfpathcurveto{\pgfqpoint{0.984239in}{0.734021in}}{\pgfqpoint{0.988630in}{0.723422in}}{\pgfqpoint{0.996443in}{0.715608in}}%
\pgfpathcurveto{\pgfqpoint{1.004257in}{0.707795in}}{\pgfqpoint{1.014856in}{0.703405in}}{\pgfqpoint{1.025906in}{0.703405in}}%
\pgfpathclose%
\pgfusepath{stroke,fill}%
\end{pgfscope}%
\begin{pgfscope}%
\pgfpathrectangle{\pgfqpoint{0.800000in}{0.528000in}}{\pgfqpoint{4.960000in}{3.696000in}}%
\pgfusepath{clip}%
\pgfsetbuttcap%
\pgfsetroundjoin%
\definecolor{currentfill}{rgb}{0.000000,0.000000,0.000000}%
\pgfsetfillcolor{currentfill}%
\pgfsetlinewidth{1.003750pt}%
\definecolor{currentstroke}{rgb}{0.000000,0.000000,0.000000}%
\pgfsetstrokecolor{currentstroke}%
\pgfsetdash{}{0pt}%
\pgfpathmoveto{\pgfqpoint{1.025906in}{0.703405in}}%
\pgfpathcurveto{\pgfqpoint{1.036956in}{0.703405in}}{\pgfqpoint{1.047555in}{0.707795in}}{\pgfqpoint{1.055369in}{0.715608in}}%
\pgfpathcurveto{\pgfqpoint{1.063182in}{0.723422in}}{\pgfqpoint{1.067573in}{0.734021in}}{\pgfqpoint{1.067573in}{0.745071in}}%
\pgfpathcurveto{\pgfqpoint{1.067573in}{0.756121in}}{\pgfqpoint{1.063182in}{0.766720in}}{\pgfqpoint{1.055369in}{0.774534in}}%
\pgfpathcurveto{\pgfqpoint{1.047555in}{0.782348in}}{\pgfqpoint{1.036956in}{0.786738in}}{\pgfqpoint{1.025906in}{0.786738in}}%
\pgfpathcurveto{\pgfqpoint{1.014856in}{0.786738in}}{\pgfqpoint{1.004257in}{0.782348in}}{\pgfqpoint{0.996443in}{0.774534in}}%
\pgfpathcurveto{\pgfqpoint{0.988630in}{0.766720in}}{\pgfqpoint{0.984239in}{0.756121in}}{\pgfqpoint{0.984239in}{0.745071in}}%
\pgfpathcurveto{\pgfqpoint{0.984239in}{0.734021in}}{\pgfqpoint{0.988630in}{0.723422in}}{\pgfqpoint{0.996443in}{0.715608in}}%
\pgfpathcurveto{\pgfqpoint{1.004257in}{0.707795in}}{\pgfqpoint{1.014856in}{0.703405in}}{\pgfqpoint{1.025906in}{0.703405in}}%
\pgfpathclose%
\pgfusepath{stroke,fill}%
\end{pgfscope}%
\begin{pgfscope}%
\pgfpathrectangle{\pgfqpoint{0.800000in}{0.528000in}}{\pgfqpoint{4.960000in}{3.696000in}}%
\pgfusepath{clip}%
\pgfsetbuttcap%
\pgfsetroundjoin%
\definecolor{currentfill}{rgb}{0.000000,0.000000,0.000000}%
\pgfsetfillcolor{currentfill}%
\pgfsetlinewidth{1.003750pt}%
\definecolor{currentstroke}{rgb}{0.000000,0.000000,0.000000}%
\pgfsetstrokecolor{currentstroke}%
\pgfsetdash{}{0pt}%
\pgfpathmoveto{\pgfqpoint{1.025906in}{0.703405in}}%
\pgfpathcurveto{\pgfqpoint{1.036956in}{0.703405in}}{\pgfqpoint{1.047555in}{0.707795in}}{\pgfqpoint{1.055369in}{0.715608in}}%
\pgfpathcurveto{\pgfqpoint{1.063182in}{0.723422in}}{\pgfqpoint{1.067573in}{0.734021in}}{\pgfqpoint{1.067573in}{0.745071in}}%
\pgfpathcurveto{\pgfqpoint{1.067573in}{0.756121in}}{\pgfqpoint{1.063182in}{0.766720in}}{\pgfqpoint{1.055369in}{0.774534in}}%
\pgfpathcurveto{\pgfqpoint{1.047555in}{0.782348in}}{\pgfqpoint{1.036956in}{0.786738in}}{\pgfqpoint{1.025906in}{0.786738in}}%
\pgfpathcurveto{\pgfqpoint{1.014856in}{0.786738in}}{\pgfqpoint{1.004257in}{0.782348in}}{\pgfqpoint{0.996443in}{0.774534in}}%
\pgfpathcurveto{\pgfqpoint{0.988630in}{0.766720in}}{\pgfqpoint{0.984239in}{0.756121in}}{\pgfqpoint{0.984239in}{0.745071in}}%
\pgfpathcurveto{\pgfqpoint{0.984239in}{0.734021in}}{\pgfqpoint{0.988630in}{0.723422in}}{\pgfqpoint{0.996443in}{0.715608in}}%
\pgfpathcurveto{\pgfqpoint{1.004257in}{0.707795in}}{\pgfqpoint{1.014856in}{0.703405in}}{\pgfqpoint{1.025906in}{0.703405in}}%
\pgfpathclose%
\pgfusepath{stroke,fill}%
\end{pgfscope}%
\begin{pgfscope}%
\pgfpathrectangle{\pgfqpoint{0.800000in}{0.528000in}}{\pgfqpoint{4.960000in}{3.696000in}}%
\pgfusepath{clip}%
\pgfsetbuttcap%
\pgfsetroundjoin%
\definecolor{currentfill}{rgb}{0.000000,0.000000,0.000000}%
\pgfsetfillcolor{currentfill}%
\pgfsetlinewidth{1.003750pt}%
\definecolor{currentstroke}{rgb}{0.000000,0.000000,0.000000}%
\pgfsetstrokecolor{currentstroke}%
\pgfsetdash{}{0pt}%
\pgfpathmoveto{\pgfqpoint{1.025906in}{0.683987in}}%
\pgfpathcurveto{\pgfqpoint{1.036956in}{0.683987in}}{\pgfqpoint{1.047555in}{0.688377in}}{\pgfqpoint{1.055369in}{0.696191in}}%
\pgfpathcurveto{\pgfqpoint{1.063182in}{0.704004in}}{\pgfqpoint{1.067573in}{0.714603in}}{\pgfqpoint{1.067573in}{0.725653in}}%
\pgfpathcurveto{\pgfqpoint{1.067573in}{0.736703in}}{\pgfqpoint{1.063182in}{0.747302in}}{\pgfqpoint{1.055369in}{0.755116in}}%
\pgfpathcurveto{\pgfqpoint{1.047555in}{0.762930in}}{\pgfqpoint{1.036956in}{0.767320in}}{\pgfqpoint{1.025906in}{0.767320in}}%
\pgfpathcurveto{\pgfqpoint{1.014856in}{0.767320in}}{\pgfqpoint{1.004257in}{0.762930in}}{\pgfqpoint{0.996443in}{0.755116in}}%
\pgfpathcurveto{\pgfqpoint{0.988630in}{0.747302in}}{\pgfqpoint{0.984239in}{0.736703in}}{\pgfqpoint{0.984239in}{0.725653in}}%
\pgfpathcurveto{\pgfqpoint{0.984239in}{0.714603in}}{\pgfqpoint{0.988630in}{0.704004in}}{\pgfqpoint{0.996443in}{0.696191in}}%
\pgfpathcurveto{\pgfqpoint{1.004257in}{0.688377in}}{\pgfqpoint{1.014856in}{0.683987in}}{\pgfqpoint{1.025906in}{0.683987in}}%
\pgfpathclose%
\pgfusepath{stroke,fill}%
\end{pgfscope}%
\begin{pgfscope}%
\pgfpathrectangle{\pgfqpoint{0.800000in}{0.528000in}}{\pgfqpoint{4.960000in}{3.696000in}}%
\pgfusepath{clip}%
\pgfsetbuttcap%
\pgfsetroundjoin%
\definecolor{currentfill}{rgb}{0.000000,0.000000,0.000000}%
\pgfsetfillcolor{currentfill}%
\pgfsetlinewidth{1.003750pt}%
\definecolor{currentstroke}{rgb}{0.000000,0.000000,0.000000}%
\pgfsetstrokecolor{currentstroke}%
\pgfsetdash{}{0pt}%
\pgfpathmoveto{\pgfqpoint{1.025906in}{0.683987in}}%
\pgfpathcurveto{\pgfqpoint{1.036956in}{0.683987in}}{\pgfqpoint{1.047555in}{0.688377in}}{\pgfqpoint{1.055369in}{0.696191in}}%
\pgfpathcurveto{\pgfqpoint{1.063182in}{0.704004in}}{\pgfqpoint{1.067573in}{0.714603in}}{\pgfqpoint{1.067573in}{0.725653in}}%
\pgfpathcurveto{\pgfqpoint{1.067573in}{0.736703in}}{\pgfqpoint{1.063182in}{0.747302in}}{\pgfqpoint{1.055369in}{0.755116in}}%
\pgfpathcurveto{\pgfqpoint{1.047555in}{0.762930in}}{\pgfqpoint{1.036956in}{0.767320in}}{\pgfqpoint{1.025906in}{0.767320in}}%
\pgfpathcurveto{\pgfqpoint{1.014856in}{0.767320in}}{\pgfqpoint{1.004257in}{0.762930in}}{\pgfqpoint{0.996443in}{0.755116in}}%
\pgfpathcurveto{\pgfqpoint{0.988630in}{0.747302in}}{\pgfqpoint{0.984239in}{0.736703in}}{\pgfqpoint{0.984239in}{0.725653in}}%
\pgfpathcurveto{\pgfqpoint{0.984239in}{0.714603in}}{\pgfqpoint{0.988630in}{0.704004in}}{\pgfqpoint{0.996443in}{0.696191in}}%
\pgfpathcurveto{\pgfqpoint{1.004257in}{0.688377in}}{\pgfqpoint{1.014856in}{0.683987in}}{\pgfqpoint{1.025906in}{0.683987in}}%
\pgfpathclose%
\pgfusepath{stroke,fill}%
\end{pgfscope}%
\begin{pgfscope}%
\pgfpathrectangle{\pgfqpoint{0.800000in}{0.528000in}}{\pgfqpoint{4.960000in}{3.696000in}}%
\pgfusepath{clip}%
\pgfsetbuttcap%
\pgfsetroundjoin%
\definecolor{currentfill}{rgb}{0.000000,0.000000,0.000000}%
\pgfsetfillcolor{currentfill}%
\pgfsetlinewidth{1.003750pt}%
\definecolor{currentstroke}{rgb}{0.000000,0.000000,0.000000}%
\pgfsetstrokecolor{currentstroke}%
\pgfsetdash{}{0pt}%
\pgfpathmoveto{\pgfqpoint{1.025906in}{0.693696in}}%
\pgfpathcurveto{\pgfqpoint{1.036956in}{0.693696in}}{\pgfqpoint{1.047555in}{0.698086in}}{\pgfqpoint{1.055369in}{0.705900in}}%
\pgfpathcurveto{\pgfqpoint{1.063182in}{0.713713in}}{\pgfqpoint{1.067573in}{0.724312in}}{\pgfqpoint{1.067573in}{0.735362in}}%
\pgfpathcurveto{\pgfqpoint{1.067573in}{0.746412in}}{\pgfqpoint{1.063182in}{0.757011in}}{\pgfqpoint{1.055369in}{0.764825in}}%
\pgfpathcurveto{\pgfqpoint{1.047555in}{0.772639in}}{\pgfqpoint{1.036956in}{0.777029in}}{\pgfqpoint{1.025906in}{0.777029in}}%
\pgfpathcurveto{\pgfqpoint{1.014856in}{0.777029in}}{\pgfqpoint{1.004257in}{0.772639in}}{\pgfqpoint{0.996443in}{0.764825in}}%
\pgfpathcurveto{\pgfqpoint{0.988630in}{0.757011in}}{\pgfqpoint{0.984239in}{0.746412in}}{\pgfqpoint{0.984239in}{0.735362in}}%
\pgfpathcurveto{\pgfqpoint{0.984239in}{0.724312in}}{\pgfqpoint{0.988630in}{0.713713in}}{\pgfqpoint{0.996443in}{0.705900in}}%
\pgfpathcurveto{\pgfqpoint{1.004257in}{0.698086in}}{\pgfqpoint{1.014856in}{0.693696in}}{\pgfqpoint{1.025906in}{0.693696in}}%
\pgfpathclose%
\pgfusepath{stroke,fill}%
\end{pgfscope}%
\begin{pgfscope}%
\pgfpathrectangle{\pgfqpoint{0.800000in}{0.528000in}}{\pgfqpoint{4.960000in}{3.696000in}}%
\pgfusepath{clip}%
\pgfsetbuttcap%
\pgfsetroundjoin%
\definecolor{currentfill}{rgb}{0.000000,0.000000,0.000000}%
\pgfsetfillcolor{currentfill}%
\pgfsetlinewidth{1.003750pt}%
\definecolor{currentstroke}{rgb}{0.000000,0.000000,0.000000}%
\pgfsetstrokecolor{currentstroke}%
\pgfsetdash{}{0pt}%
\pgfpathmoveto{\pgfqpoint{1.025906in}{0.683987in}}%
\pgfpathcurveto{\pgfqpoint{1.036956in}{0.683987in}}{\pgfqpoint{1.047555in}{0.688377in}}{\pgfqpoint{1.055369in}{0.696191in}}%
\pgfpathcurveto{\pgfqpoint{1.063182in}{0.704004in}}{\pgfqpoint{1.067573in}{0.714603in}}{\pgfqpoint{1.067573in}{0.725653in}}%
\pgfpathcurveto{\pgfqpoint{1.067573in}{0.736703in}}{\pgfqpoint{1.063182in}{0.747302in}}{\pgfqpoint{1.055369in}{0.755116in}}%
\pgfpathcurveto{\pgfqpoint{1.047555in}{0.762930in}}{\pgfqpoint{1.036956in}{0.767320in}}{\pgfqpoint{1.025906in}{0.767320in}}%
\pgfpathcurveto{\pgfqpoint{1.014856in}{0.767320in}}{\pgfqpoint{1.004257in}{0.762930in}}{\pgfqpoint{0.996443in}{0.755116in}}%
\pgfpathcurveto{\pgfqpoint{0.988630in}{0.747302in}}{\pgfqpoint{0.984239in}{0.736703in}}{\pgfqpoint{0.984239in}{0.725653in}}%
\pgfpathcurveto{\pgfqpoint{0.984239in}{0.714603in}}{\pgfqpoint{0.988630in}{0.704004in}}{\pgfqpoint{0.996443in}{0.696191in}}%
\pgfpathcurveto{\pgfqpoint{1.004257in}{0.688377in}}{\pgfqpoint{1.014856in}{0.683987in}}{\pgfqpoint{1.025906in}{0.683987in}}%
\pgfpathclose%
\pgfusepath{stroke,fill}%
\end{pgfscope}%
\begin{pgfscope}%
\pgfpathrectangle{\pgfqpoint{0.800000in}{0.528000in}}{\pgfqpoint{4.960000in}{3.696000in}}%
\pgfusepath{clip}%
\pgfsetbuttcap%
\pgfsetroundjoin%
\definecolor{currentfill}{rgb}{0.000000,0.000000,0.000000}%
\pgfsetfillcolor{currentfill}%
\pgfsetlinewidth{1.003750pt}%
\definecolor{currentstroke}{rgb}{0.000000,0.000000,0.000000}%
\pgfsetstrokecolor{currentstroke}%
\pgfsetdash{}{0pt}%
\pgfpathmoveto{\pgfqpoint{1.025906in}{0.683987in}}%
\pgfpathcurveto{\pgfqpoint{1.036956in}{0.683987in}}{\pgfqpoint{1.047555in}{0.688377in}}{\pgfqpoint{1.055369in}{0.696191in}}%
\pgfpathcurveto{\pgfqpoint{1.063182in}{0.704004in}}{\pgfqpoint{1.067573in}{0.714603in}}{\pgfqpoint{1.067573in}{0.725653in}}%
\pgfpathcurveto{\pgfqpoint{1.067573in}{0.736703in}}{\pgfqpoint{1.063182in}{0.747302in}}{\pgfqpoint{1.055369in}{0.755116in}}%
\pgfpathcurveto{\pgfqpoint{1.047555in}{0.762930in}}{\pgfqpoint{1.036956in}{0.767320in}}{\pgfqpoint{1.025906in}{0.767320in}}%
\pgfpathcurveto{\pgfqpoint{1.014856in}{0.767320in}}{\pgfqpoint{1.004257in}{0.762930in}}{\pgfqpoint{0.996443in}{0.755116in}}%
\pgfpathcurveto{\pgfqpoint{0.988630in}{0.747302in}}{\pgfqpoint{0.984239in}{0.736703in}}{\pgfqpoint{0.984239in}{0.725653in}}%
\pgfpathcurveto{\pgfqpoint{0.984239in}{0.714603in}}{\pgfqpoint{0.988630in}{0.704004in}}{\pgfqpoint{0.996443in}{0.696191in}}%
\pgfpathcurveto{\pgfqpoint{1.004257in}{0.688377in}}{\pgfqpoint{1.014856in}{0.683987in}}{\pgfqpoint{1.025906in}{0.683987in}}%
\pgfpathclose%
\pgfusepath{stroke,fill}%
\end{pgfscope}%
\begin{pgfscope}%
\pgfpathrectangle{\pgfqpoint{0.800000in}{0.528000in}}{\pgfqpoint{4.960000in}{3.696000in}}%
\pgfusepath{clip}%
\pgfsetbuttcap%
\pgfsetroundjoin%
\definecolor{currentfill}{rgb}{0.000000,0.000000,0.000000}%
\pgfsetfillcolor{currentfill}%
\pgfsetlinewidth{1.003750pt}%
\definecolor{currentstroke}{rgb}{0.000000,0.000000,0.000000}%
\pgfsetstrokecolor{currentstroke}%
\pgfsetdash{}{0pt}%
\pgfpathmoveto{\pgfqpoint{1.025906in}{0.693696in}}%
\pgfpathcurveto{\pgfqpoint{1.036956in}{0.693696in}}{\pgfqpoint{1.047555in}{0.698086in}}{\pgfqpoint{1.055369in}{0.705900in}}%
\pgfpathcurveto{\pgfqpoint{1.063182in}{0.713713in}}{\pgfqpoint{1.067573in}{0.724312in}}{\pgfqpoint{1.067573in}{0.735362in}}%
\pgfpathcurveto{\pgfqpoint{1.067573in}{0.746412in}}{\pgfqpoint{1.063182in}{0.757011in}}{\pgfqpoint{1.055369in}{0.764825in}}%
\pgfpathcurveto{\pgfqpoint{1.047555in}{0.772639in}}{\pgfqpoint{1.036956in}{0.777029in}}{\pgfqpoint{1.025906in}{0.777029in}}%
\pgfpathcurveto{\pgfqpoint{1.014856in}{0.777029in}}{\pgfqpoint{1.004257in}{0.772639in}}{\pgfqpoint{0.996443in}{0.764825in}}%
\pgfpathcurveto{\pgfqpoint{0.988630in}{0.757011in}}{\pgfqpoint{0.984239in}{0.746412in}}{\pgfqpoint{0.984239in}{0.735362in}}%
\pgfpathcurveto{\pgfqpoint{0.984239in}{0.724312in}}{\pgfqpoint{0.988630in}{0.713713in}}{\pgfqpoint{0.996443in}{0.705900in}}%
\pgfpathcurveto{\pgfqpoint{1.004257in}{0.698086in}}{\pgfqpoint{1.014856in}{0.693696in}}{\pgfqpoint{1.025906in}{0.693696in}}%
\pgfpathclose%
\pgfusepath{stroke,fill}%
\end{pgfscope}%
\begin{pgfscope}%
\pgfpathrectangle{\pgfqpoint{0.800000in}{0.528000in}}{\pgfqpoint{4.960000in}{3.696000in}}%
\pgfusepath{clip}%
\pgfsetbuttcap%
\pgfsetroundjoin%
\definecolor{currentfill}{rgb}{0.000000,0.000000,0.000000}%
\pgfsetfillcolor{currentfill}%
\pgfsetlinewidth{1.003750pt}%
\definecolor{currentstroke}{rgb}{0.000000,0.000000,0.000000}%
\pgfsetstrokecolor{currentstroke}%
\pgfsetdash{}{0pt}%
\pgfpathmoveto{\pgfqpoint{1.025906in}{0.703405in}}%
\pgfpathcurveto{\pgfqpoint{1.036956in}{0.703405in}}{\pgfqpoint{1.047555in}{0.707795in}}{\pgfqpoint{1.055369in}{0.715608in}}%
\pgfpathcurveto{\pgfqpoint{1.063182in}{0.723422in}}{\pgfqpoint{1.067573in}{0.734021in}}{\pgfqpoint{1.067573in}{0.745071in}}%
\pgfpathcurveto{\pgfqpoint{1.067573in}{0.756121in}}{\pgfqpoint{1.063182in}{0.766720in}}{\pgfqpoint{1.055369in}{0.774534in}}%
\pgfpathcurveto{\pgfqpoint{1.047555in}{0.782348in}}{\pgfqpoint{1.036956in}{0.786738in}}{\pgfqpoint{1.025906in}{0.786738in}}%
\pgfpathcurveto{\pgfqpoint{1.014856in}{0.786738in}}{\pgfqpoint{1.004257in}{0.782348in}}{\pgfqpoint{0.996443in}{0.774534in}}%
\pgfpathcurveto{\pgfqpoint{0.988630in}{0.766720in}}{\pgfqpoint{0.984239in}{0.756121in}}{\pgfqpoint{0.984239in}{0.745071in}}%
\pgfpathcurveto{\pgfqpoint{0.984239in}{0.734021in}}{\pgfqpoint{0.988630in}{0.723422in}}{\pgfqpoint{0.996443in}{0.715608in}}%
\pgfpathcurveto{\pgfqpoint{1.004257in}{0.707795in}}{\pgfqpoint{1.014856in}{0.703405in}}{\pgfqpoint{1.025906in}{0.703405in}}%
\pgfpathclose%
\pgfusepath{stroke,fill}%
\end{pgfscope}%
\begin{pgfscope}%
\pgfpathrectangle{\pgfqpoint{0.800000in}{0.528000in}}{\pgfqpoint{4.960000in}{3.696000in}}%
\pgfusepath{clip}%
\pgfsetbuttcap%
\pgfsetroundjoin%
\definecolor{currentfill}{rgb}{0.000000,0.000000,0.000000}%
\pgfsetfillcolor{currentfill}%
\pgfsetlinewidth{1.003750pt}%
\definecolor{currentstroke}{rgb}{0.000000,0.000000,0.000000}%
\pgfsetstrokecolor{currentstroke}%
\pgfsetdash{}{0pt}%
\pgfpathmoveto{\pgfqpoint{1.025906in}{0.674278in}}%
\pgfpathcurveto{\pgfqpoint{1.036956in}{0.674278in}}{\pgfqpoint{1.047555in}{0.678668in}}{\pgfqpoint{1.055369in}{0.686482in}}%
\pgfpathcurveto{\pgfqpoint{1.063182in}{0.694295in}}{\pgfqpoint{1.067573in}{0.704894in}}{\pgfqpoint{1.067573in}{0.715944in}}%
\pgfpathcurveto{\pgfqpoint{1.067573in}{0.726994in}}{\pgfqpoint{1.063182in}{0.737594in}}{\pgfqpoint{1.055369in}{0.745407in}}%
\pgfpathcurveto{\pgfqpoint{1.047555in}{0.753221in}}{\pgfqpoint{1.036956in}{0.757611in}}{\pgfqpoint{1.025906in}{0.757611in}}%
\pgfpathcurveto{\pgfqpoint{1.014856in}{0.757611in}}{\pgfqpoint{1.004257in}{0.753221in}}{\pgfqpoint{0.996443in}{0.745407in}}%
\pgfpathcurveto{\pgfqpoint{0.988630in}{0.737594in}}{\pgfqpoint{0.984239in}{0.726994in}}{\pgfqpoint{0.984239in}{0.715944in}}%
\pgfpathcurveto{\pgfqpoint{0.984239in}{0.704894in}}{\pgfqpoint{0.988630in}{0.694295in}}{\pgfqpoint{0.996443in}{0.686482in}}%
\pgfpathcurveto{\pgfqpoint{1.004257in}{0.678668in}}{\pgfqpoint{1.014856in}{0.674278in}}{\pgfqpoint{1.025906in}{0.674278in}}%
\pgfpathclose%
\pgfusepath{stroke,fill}%
\end{pgfscope}%
\begin{pgfscope}%
\pgfpathrectangle{\pgfqpoint{0.800000in}{0.528000in}}{\pgfqpoint{4.960000in}{3.696000in}}%
\pgfusepath{clip}%
\pgfsetbuttcap%
\pgfsetroundjoin%
\definecolor{currentfill}{rgb}{0.000000,0.000000,0.000000}%
\pgfsetfillcolor{currentfill}%
\pgfsetlinewidth{1.003750pt}%
\definecolor{currentstroke}{rgb}{0.000000,0.000000,0.000000}%
\pgfsetstrokecolor{currentstroke}%
\pgfsetdash{}{0pt}%
\pgfpathmoveto{\pgfqpoint{1.025906in}{0.703405in}}%
\pgfpathcurveto{\pgfqpoint{1.036956in}{0.703405in}}{\pgfqpoint{1.047555in}{0.707795in}}{\pgfqpoint{1.055369in}{0.715608in}}%
\pgfpathcurveto{\pgfqpoint{1.063182in}{0.723422in}}{\pgfqpoint{1.067573in}{0.734021in}}{\pgfqpoint{1.067573in}{0.745071in}}%
\pgfpathcurveto{\pgfqpoint{1.067573in}{0.756121in}}{\pgfqpoint{1.063182in}{0.766720in}}{\pgfqpoint{1.055369in}{0.774534in}}%
\pgfpathcurveto{\pgfqpoint{1.047555in}{0.782348in}}{\pgfqpoint{1.036956in}{0.786738in}}{\pgfqpoint{1.025906in}{0.786738in}}%
\pgfpathcurveto{\pgfqpoint{1.014856in}{0.786738in}}{\pgfqpoint{1.004257in}{0.782348in}}{\pgfqpoint{0.996443in}{0.774534in}}%
\pgfpathcurveto{\pgfqpoint{0.988630in}{0.766720in}}{\pgfqpoint{0.984239in}{0.756121in}}{\pgfqpoint{0.984239in}{0.745071in}}%
\pgfpathcurveto{\pgfqpoint{0.984239in}{0.734021in}}{\pgfqpoint{0.988630in}{0.723422in}}{\pgfqpoint{0.996443in}{0.715608in}}%
\pgfpathcurveto{\pgfqpoint{1.004257in}{0.707795in}}{\pgfqpoint{1.014856in}{0.703405in}}{\pgfqpoint{1.025906in}{0.703405in}}%
\pgfpathclose%
\pgfusepath{stroke,fill}%
\end{pgfscope}%
\begin{pgfscope}%
\pgfpathrectangle{\pgfqpoint{0.800000in}{0.528000in}}{\pgfqpoint{4.960000in}{3.696000in}}%
\pgfusepath{clip}%
\pgfsetbuttcap%
\pgfsetroundjoin%
\definecolor{currentfill}{rgb}{0.000000,0.000000,0.000000}%
\pgfsetfillcolor{currentfill}%
\pgfsetlinewidth{1.003750pt}%
\definecolor{currentstroke}{rgb}{0.000000,0.000000,0.000000}%
\pgfsetstrokecolor{currentstroke}%
\pgfsetdash{}{0pt}%
\pgfpathmoveto{\pgfqpoint{1.025906in}{0.683987in}}%
\pgfpathcurveto{\pgfqpoint{1.036956in}{0.683987in}}{\pgfqpoint{1.047555in}{0.688377in}}{\pgfqpoint{1.055369in}{0.696191in}}%
\pgfpathcurveto{\pgfqpoint{1.063182in}{0.704004in}}{\pgfqpoint{1.067573in}{0.714603in}}{\pgfqpoint{1.067573in}{0.725653in}}%
\pgfpathcurveto{\pgfqpoint{1.067573in}{0.736703in}}{\pgfqpoint{1.063182in}{0.747302in}}{\pgfqpoint{1.055369in}{0.755116in}}%
\pgfpathcurveto{\pgfqpoint{1.047555in}{0.762930in}}{\pgfqpoint{1.036956in}{0.767320in}}{\pgfqpoint{1.025906in}{0.767320in}}%
\pgfpathcurveto{\pgfqpoint{1.014856in}{0.767320in}}{\pgfqpoint{1.004257in}{0.762930in}}{\pgfqpoint{0.996443in}{0.755116in}}%
\pgfpathcurveto{\pgfqpoint{0.988630in}{0.747302in}}{\pgfqpoint{0.984239in}{0.736703in}}{\pgfqpoint{0.984239in}{0.725653in}}%
\pgfpathcurveto{\pgfqpoint{0.984239in}{0.714603in}}{\pgfqpoint{0.988630in}{0.704004in}}{\pgfqpoint{0.996443in}{0.696191in}}%
\pgfpathcurveto{\pgfqpoint{1.004257in}{0.688377in}}{\pgfqpoint{1.014856in}{0.683987in}}{\pgfqpoint{1.025906in}{0.683987in}}%
\pgfpathclose%
\pgfusepath{stroke,fill}%
\end{pgfscope}%
\begin{pgfscope}%
\pgfpathrectangle{\pgfqpoint{0.800000in}{0.528000in}}{\pgfqpoint{4.960000in}{3.696000in}}%
\pgfusepath{clip}%
\pgfsetbuttcap%
\pgfsetroundjoin%
\definecolor{currentfill}{rgb}{0.000000,0.000000,0.000000}%
\pgfsetfillcolor{currentfill}%
\pgfsetlinewidth{1.003750pt}%
\definecolor{currentstroke}{rgb}{0.000000,0.000000,0.000000}%
\pgfsetstrokecolor{currentstroke}%
\pgfsetdash{}{0pt}%
\pgfpathmoveto{\pgfqpoint{1.025906in}{0.654860in}}%
\pgfpathcurveto{\pgfqpoint{1.036956in}{0.654860in}}{\pgfqpoint{1.047555in}{0.659250in}}{\pgfqpoint{1.055369in}{0.667064in}}%
\pgfpathcurveto{\pgfqpoint{1.063182in}{0.674877in}}{\pgfqpoint{1.067573in}{0.685476in}}{\pgfqpoint{1.067573in}{0.696526in}}%
\pgfpathcurveto{\pgfqpoint{1.067573in}{0.707577in}}{\pgfqpoint{1.063182in}{0.718176in}}{\pgfqpoint{1.055369in}{0.725989in}}%
\pgfpathcurveto{\pgfqpoint{1.047555in}{0.733803in}}{\pgfqpoint{1.036956in}{0.738193in}}{\pgfqpoint{1.025906in}{0.738193in}}%
\pgfpathcurveto{\pgfqpoint{1.014856in}{0.738193in}}{\pgfqpoint{1.004257in}{0.733803in}}{\pgfqpoint{0.996443in}{0.725989in}}%
\pgfpathcurveto{\pgfqpoint{0.988630in}{0.718176in}}{\pgfqpoint{0.984239in}{0.707577in}}{\pgfqpoint{0.984239in}{0.696526in}}%
\pgfpathcurveto{\pgfqpoint{0.984239in}{0.685476in}}{\pgfqpoint{0.988630in}{0.674877in}}{\pgfqpoint{0.996443in}{0.667064in}}%
\pgfpathcurveto{\pgfqpoint{1.004257in}{0.659250in}}{\pgfqpoint{1.014856in}{0.654860in}}{\pgfqpoint{1.025906in}{0.654860in}}%
\pgfpathclose%
\pgfusepath{stroke,fill}%
\end{pgfscope}%
\begin{pgfscope}%
\pgfpathrectangle{\pgfqpoint{0.800000in}{0.528000in}}{\pgfqpoint{4.960000in}{3.696000in}}%
\pgfusepath{clip}%
\pgfsetbuttcap%
\pgfsetroundjoin%
\definecolor{currentfill}{rgb}{0.000000,0.000000,0.000000}%
\pgfsetfillcolor{currentfill}%
\pgfsetlinewidth{1.003750pt}%
\definecolor{currentstroke}{rgb}{0.000000,0.000000,0.000000}%
\pgfsetstrokecolor{currentstroke}%
\pgfsetdash{}{0pt}%
\pgfpathmoveto{\pgfqpoint{1.025906in}{0.713114in}}%
\pgfpathcurveto{\pgfqpoint{1.036956in}{0.713114in}}{\pgfqpoint{1.047555in}{0.717504in}}{\pgfqpoint{1.055369in}{0.725317in}}%
\pgfpathcurveto{\pgfqpoint{1.063182in}{0.733131in}}{\pgfqpoint{1.067573in}{0.743730in}}{\pgfqpoint{1.067573in}{0.754780in}}%
\pgfpathcurveto{\pgfqpoint{1.067573in}{0.765830in}}{\pgfqpoint{1.063182in}{0.776429in}}{\pgfqpoint{1.055369in}{0.784243in}}%
\pgfpathcurveto{\pgfqpoint{1.047555in}{0.792057in}}{\pgfqpoint{1.036956in}{0.796447in}}{\pgfqpoint{1.025906in}{0.796447in}}%
\pgfpathcurveto{\pgfqpoint{1.014856in}{0.796447in}}{\pgfqpoint{1.004257in}{0.792057in}}{\pgfqpoint{0.996443in}{0.784243in}}%
\pgfpathcurveto{\pgfqpoint{0.988630in}{0.776429in}}{\pgfqpoint{0.984239in}{0.765830in}}{\pgfqpoint{0.984239in}{0.754780in}}%
\pgfpathcurveto{\pgfqpoint{0.984239in}{0.743730in}}{\pgfqpoint{0.988630in}{0.733131in}}{\pgfqpoint{0.996443in}{0.725317in}}%
\pgfpathcurveto{\pgfqpoint{1.004257in}{0.717504in}}{\pgfqpoint{1.014856in}{0.713114in}}{\pgfqpoint{1.025906in}{0.713114in}}%
\pgfpathclose%
\pgfusepath{stroke,fill}%
\end{pgfscope}%
\begin{pgfscope}%
\pgfpathrectangle{\pgfqpoint{0.800000in}{0.528000in}}{\pgfqpoint{4.960000in}{3.696000in}}%
\pgfusepath{clip}%
\pgfsetbuttcap%
\pgfsetroundjoin%
\definecolor{currentfill}{rgb}{0.000000,0.000000,0.000000}%
\pgfsetfillcolor{currentfill}%
\pgfsetlinewidth{1.003750pt}%
\definecolor{currentstroke}{rgb}{0.000000,0.000000,0.000000}%
\pgfsetstrokecolor{currentstroke}%
\pgfsetdash{}{0pt}%
\pgfpathmoveto{\pgfqpoint{1.025906in}{0.703405in}}%
\pgfpathcurveto{\pgfqpoint{1.036956in}{0.703405in}}{\pgfqpoint{1.047555in}{0.707795in}}{\pgfqpoint{1.055369in}{0.715608in}}%
\pgfpathcurveto{\pgfqpoint{1.063182in}{0.723422in}}{\pgfqpoint{1.067573in}{0.734021in}}{\pgfqpoint{1.067573in}{0.745071in}}%
\pgfpathcurveto{\pgfqpoint{1.067573in}{0.756121in}}{\pgfqpoint{1.063182in}{0.766720in}}{\pgfqpoint{1.055369in}{0.774534in}}%
\pgfpathcurveto{\pgfqpoint{1.047555in}{0.782348in}}{\pgfqpoint{1.036956in}{0.786738in}}{\pgfqpoint{1.025906in}{0.786738in}}%
\pgfpathcurveto{\pgfqpoint{1.014856in}{0.786738in}}{\pgfqpoint{1.004257in}{0.782348in}}{\pgfqpoint{0.996443in}{0.774534in}}%
\pgfpathcurveto{\pgfqpoint{0.988630in}{0.766720in}}{\pgfqpoint{0.984239in}{0.756121in}}{\pgfqpoint{0.984239in}{0.745071in}}%
\pgfpathcurveto{\pgfqpoint{0.984239in}{0.734021in}}{\pgfqpoint{0.988630in}{0.723422in}}{\pgfqpoint{0.996443in}{0.715608in}}%
\pgfpathcurveto{\pgfqpoint{1.004257in}{0.707795in}}{\pgfqpoint{1.014856in}{0.703405in}}{\pgfqpoint{1.025906in}{0.703405in}}%
\pgfpathclose%
\pgfusepath{stroke,fill}%
\end{pgfscope}%
\begin{pgfscope}%
\pgfpathrectangle{\pgfqpoint{0.800000in}{0.528000in}}{\pgfqpoint{4.960000in}{3.696000in}}%
\pgfusepath{clip}%
\pgfsetbuttcap%
\pgfsetroundjoin%
\definecolor{currentfill}{rgb}{0.000000,0.000000,0.000000}%
\pgfsetfillcolor{currentfill}%
\pgfsetlinewidth{1.003750pt}%
\definecolor{currentstroke}{rgb}{0.000000,0.000000,0.000000}%
\pgfsetstrokecolor{currentstroke}%
\pgfsetdash{}{0pt}%
\pgfpathmoveto{\pgfqpoint{1.025906in}{0.693696in}}%
\pgfpathcurveto{\pgfqpoint{1.036956in}{0.693696in}}{\pgfqpoint{1.047555in}{0.698086in}}{\pgfqpoint{1.055369in}{0.705900in}}%
\pgfpathcurveto{\pgfqpoint{1.063182in}{0.713713in}}{\pgfqpoint{1.067573in}{0.724312in}}{\pgfqpoint{1.067573in}{0.735362in}}%
\pgfpathcurveto{\pgfqpoint{1.067573in}{0.746412in}}{\pgfqpoint{1.063182in}{0.757011in}}{\pgfqpoint{1.055369in}{0.764825in}}%
\pgfpathcurveto{\pgfqpoint{1.047555in}{0.772639in}}{\pgfqpoint{1.036956in}{0.777029in}}{\pgfqpoint{1.025906in}{0.777029in}}%
\pgfpathcurveto{\pgfqpoint{1.014856in}{0.777029in}}{\pgfqpoint{1.004257in}{0.772639in}}{\pgfqpoint{0.996443in}{0.764825in}}%
\pgfpathcurveto{\pgfqpoint{0.988630in}{0.757011in}}{\pgfqpoint{0.984239in}{0.746412in}}{\pgfqpoint{0.984239in}{0.735362in}}%
\pgfpathcurveto{\pgfqpoint{0.984239in}{0.724312in}}{\pgfqpoint{0.988630in}{0.713713in}}{\pgfqpoint{0.996443in}{0.705900in}}%
\pgfpathcurveto{\pgfqpoint{1.004257in}{0.698086in}}{\pgfqpoint{1.014856in}{0.693696in}}{\pgfqpoint{1.025906in}{0.693696in}}%
\pgfpathclose%
\pgfusepath{stroke,fill}%
\end{pgfscope}%
\begin{pgfscope}%
\pgfpathrectangle{\pgfqpoint{0.800000in}{0.528000in}}{\pgfqpoint{4.960000in}{3.696000in}}%
\pgfusepath{clip}%
\pgfsetbuttcap%
\pgfsetroundjoin%
\definecolor{currentfill}{rgb}{0.000000,0.000000,0.000000}%
\pgfsetfillcolor{currentfill}%
\pgfsetlinewidth{1.003750pt}%
\definecolor{currentstroke}{rgb}{0.000000,0.000000,0.000000}%
\pgfsetstrokecolor{currentstroke}%
\pgfsetdash{}{0pt}%
\pgfpathmoveto{\pgfqpoint{1.025906in}{0.683987in}}%
\pgfpathcurveto{\pgfqpoint{1.036956in}{0.683987in}}{\pgfqpoint{1.047555in}{0.688377in}}{\pgfqpoint{1.055369in}{0.696191in}}%
\pgfpathcurveto{\pgfqpoint{1.063182in}{0.704004in}}{\pgfqpoint{1.067573in}{0.714603in}}{\pgfqpoint{1.067573in}{0.725653in}}%
\pgfpathcurveto{\pgfqpoint{1.067573in}{0.736703in}}{\pgfqpoint{1.063182in}{0.747302in}}{\pgfqpoint{1.055369in}{0.755116in}}%
\pgfpathcurveto{\pgfqpoint{1.047555in}{0.762930in}}{\pgfqpoint{1.036956in}{0.767320in}}{\pgfqpoint{1.025906in}{0.767320in}}%
\pgfpathcurveto{\pgfqpoint{1.014856in}{0.767320in}}{\pgfqpoint{1.004257in}{0.762930in}}{\pgfqpoint{0.996443in}{0.755116in}}%
\pgfpathcurveto{\pgfqpoint{0.988630in}{0.747302in}}{\pgfqpoint{0.984239in}{0.736703in}}{\pgfqpoint{0.984239in}{0.725653in}}%
\pgfpathcurveto{\pgfqpoint{0.984239in}{0.714603in}}{\pgfqpoint{0.988630in}{0.704004in}}{\pgfqpoint{0.996443in}{0.696191in}}%
\pgfpathcurveto{\pgfqpoint{1.004257in}{0.688377in}}{\pgfqpoint{1.014856in}{0.683987in}}{\pgfqpoint{1.025906in}{0.683987in}}%
\pgfpathclose%
\pgfusepath{stroke,fill}%
\end{pgfscope}%
\begin{pgfscope}%
\pgfpathrectangle{\pgfqpoint{0.800000in}{0.528000in}}{\pgfqpoint{4.960000in}{3.696000in}}%
\pgfusepath{clip}%
\pgfsetbuttcap%
\pgfsetroundjoin%
\definecolor{currentfill}{rgb}{0.000000,0.000000,0.000000}%
\pgfsetfillcolor{currentfill}%
\pgfsetlinewidth{1.003750pt}%
\definecolor{currentstroke}{rgb}{0.000000,0.000000,0.000000}%
\pgfsetstrokecolor{currentstroke}%
\pgfsetdash{}{0pt}%
\pgfpathmoveto{\pgfqpoint{1.025906in}{0.693696in}}%
\pgfpathcurveto{\pgfqpoint{1.036956in}{0.693696in}}{\pgfqpoint{1.047555in}{0.698086in}}{\pgfqpoint{1.055369in}{0.705900in}}%
\pgfpathcurveto{\pgfqpoint{1.063182in}{0.713713in}}{\pgfqpoint{1.067573in}{0.724312in}}{\pgfqpoint{1.067573in}{0.735362in}}%
\pgfpathcurveto{\pgfqpoint{1.067573in}{0.746412in}}{\pgfqpoint{1.063182in}{0.757011in}}{\pgfqpoint{1.055369in}{0.764825in}}%
\pgfpathcurveto{\pgfqpoint{1.047555in}{0.772639in}}{\pgfqpoint{1.036956in}{0.777029in}}{\pgfqpoint{1.025906in}{0.777029in}}%
\pgfpathcurveto{\pgfqpoint{1.014856in}{0.777029in}}{\pgfqpoint{1.004257in}{0.772639in}}{\pgfqpoint{0.996443in}{0.764825in}}%
\pgfpathcurveto{\pgfqpoint{0.988630in}{0.757011in}}{\pgfqpoint{0.984239in}{0.746412in}}{\pgfqpoint{0.984239in}{0.735362in}}%
\pgfpathcurveto{\pgfqpoint{0.984239in}{0.724312in}}{\pgfqpoint{0.988630in}{0.713713in}}{\pgfqpoint{0.996443in}{0.705900in}}%
\pgfpathcurveto{\pgfqpoint{1.004257in}{0.698086in}}{\pgfqpoint{1.014856in}{0.693696in}}{\pgfqpoint{1.025906in}{0.693696in}}%
\pgfpathclose%
\pgfusepath{stroke,fill}%
\end{pgfscope}%
\begin{pgfscope}%
\pgfpathrectangle{\pgfqpoint{0.800000in}{0.528000in}}{\pgfqpoint{4.960000in}{3.696000in}}%
\pgfusepath{clip}%
\pgfsetbuttcap%
\pgfsetroundjoin%
\definecolor{currentfill}{rgb}{0.000000,0.000000,0.000000}%
\pgfsetfillcolor{currentfill}%
\pgfsetlinewidth{1.003750pt}%
\definecolor{currentstroke}{rgb}{0.000000,0.000000,0.000000}%
\pgfsetstrokecolor{currentstroke}%
\pgfsetdash{}{0pt}%
\pgfpathmoveto{\pgfqpoint{1.025906in}{0.664569in}}%
\pgfpathcurveto{\pgfqpoint{1.036956in}{0.664569in}}{\pgfqpoint{1.047555in}{0.668959in}}{\pgfqpoint{1.055369in}{0.676773in}}%
\pgfpathcurveto{\pgfqpoint{1.063182in}{0.684586in}}{\pgfqpoint{1.067573in}{0.695185in}}{\pgfqpoint{1.067573in}{0.706235in}}%
\pgfpathcurveto{\pgfqpoint{1.067573in}{0.717286in}}{\pgfqpoint{1.063182in}{0.727885in}}{\pgfqpoint{1.055369in}{0.735698in}}%
\pgfpathcurveto{\pgfqpoint{1.047555in}{0.743512in}}{\pgfqpoint{1.036956in}{0.747902in}}{\pgfqpoint{1.025906in}{0.747902in}}%
\pgfpathcurveto{\pgfqpoint{1.014856in}{0.747902in}}{\pgfqpoint{1.004257in}{0.743512in}}{\pgfqpoint{0.996443in}{0.735698in}}%
\pgfpathcurveto{\pgfqpoint{0.988630in}{0.727885in}}{\pgfqpoint{0.984239in}{0.717286in}}{\pgfqpoint{0.984239in}{0.706235in}}%
\pgfpathcurveto{\pgfqpoint{0.984239in}{0.695185in}}{\pgfqpoint{0.988630in}{0.684586in}}{\pgfqpoint{0.996443in}{0.676773in}}%
\pgfpathcurveto{\pgfqpoint{1.004257in}{0.668959in}}{\pgfqpoint{1.014856in}{0.664569in}}{\pgfqpoint{1.025906in}{0.664569in}}%
\pgfpathclose%
\pgfusepath{stroke,fill}%
\end{pgfscope}%
\begin{pgfscope}%
\pgfpathrectangle{\pgfqpoint{0.800000in}{0.528000in}}{\pgfqpoint{4.960000in}{3.696000in}}%
\pgfusepath{clip}%
\pgfsetbuttcap%
\pgfsetroundjoin%
\definecolor{currentfill}{rgb}{0.000000,0.000000,0.000000}%
\pgfsetfillcolor{currentfill}%
\pgfsetlinewidth{1.003750pt}%
\definecolor{currentstroke}{rgb}{0.000000,0.000000,0.000000}%
\pgfsetstrokecolor{currentstroke}%
\pgfsetdash{}{0pt}%
\pgfpathmoveto{\pgfqpoint{1.025906in}{0.693696in}}%
\pgfpathcurveto{\pgfqpoint{1.036956in}{0.693696in}}{\pgfqpoint{1.047555in}{0.698086in}}{\pgfqpoint{1.055369in}{0.705900in}}%
\pgfpathcurveto{\pgfqpoint{1.063182in}{0.713713in}}{\pgfqpoint{1.067573in}{0.724312in}}{\pgfqpoint{1.067573in}{0.735362in}}%
\pgfpathcurveto{\pgfqpoint{1.067573in}{0.746412in}}{\pgfqpoint{1.063182in}{0.757011in}}{\pgfqpoint{1.055369in}{0.764825in}}%
\pgfpathcurveto{\pgfqpoint{1.047555in}{0.772639in}}{\pgfqpoint{1.036956in}{0.777029in}}{\pgfqpoint{1.025906in}{0.777029in}}%
\pgfpathcurveto{\pgfqpoint{1.014856in}{0.777029in}}{\pgfqpoint{1.004257in}{0.772639in}}{\pgfqpoint{0.996443in}{0.764825in}}%
\pgfpathcurveto{\pgfqpoint{0.988630in}{0.757011in}}{\pgfqpoint{0.984239in}{0.746412in}}{\pgfqpoint{0.984239in}{0.735362in}}%
\pgfpathcurveto{\pgfqpoint{0.984239in}{0.724312in}}{\pgfqpoint{0.988630in}{0.713713in}}{\pgfqpoint{0.996443in}{0.705900in}}%
\pgfpathcurveto{\pgfqpoint{1.004257in}{0.698086in}}{\pgfqpoint{1.014856in}{0.693696in}}{\pgfqpoint{1.025906in}{0.693696in}}%
\pgfpathclose%
\pgfusepath{stroke,fill}%
\end{pgfscope}%
\begin{pgfscope}%
\pgfpathrectangle{\pgfqpoint{0.800000in}{0.528000in}}{\pgfqpoint{4.960000in}{3.696000in}}%
\pgfusepath{clip}%
\pgfsetbuttcap%
\pgfsetroundjoin%
\definecolor{currentfill}{rgb}{0.000000,0.000000,0.000000}%
\pgfsetfillcolor{currentfill}%
\pgfsetlinewidth{1.003750pt}%
\definecolor{currentstroke}{rgb}{0.000000,0.000000,0.000000}%
\pgfsetstrokecolor{currentstroke}%
\pgfsetdash{}{0pt}%
\pgfpathmoveto{\pgfqpoint{1.025906in}{0.703405in}}%
\pgfpathcurveto{\pgfqpoint{1.036956in}{0.703405in}}{\pgfqpoint{1.047555in}{0.707795in}}{\pgfqpoint{1.055369in}{0.715608in}}%
\pgfpathcurveto{\pgfqpoint{1.063182in}{0.723422in}}{\pgfqpoint{1.067573in}{0.734021in}}{\pgfqpoint{1.067573in}{0.745071in}}%
\pgfpathcurveto{\pgfqpoint{1.067573in}{0.756121in}}{\pgfqpoint{1.063182in}{0.766720in}}{\pgfqpoint{1.055369in}{0.774534in}}%
\pgfpathcurveto{\pgfqpoint{1.047555in}{0.782348in}}{\pgfqpoint{1.036956in}{0.786738in}}{\pgfqpoint{1.025906in}{0.786738in}}%
\pgfpathcurveto{\pgfqpoint{1.014856in}{0.786738in}}{\pgfqpoint{1.004257in}{0.782348in}}{\pgfqpoint{0.996443in}{0.774534in}}%
\pgfpathcurveto{\pgfqpoint{0.988630in}{0.766720in}}{\pgfqpoint{0.984239in}{0.756121in}}{\pgfqpoint{0.984239in}{0.745071in}}%
\pgfpathcurveto{\pgfqpoint{0.984239in}{0.734021in}}{\pgfqpoint{0.988630in}{0.723422in}}{\pgfqpoint{0.996443in}{0.715608in}}%
\pgfpathcurveto{\pgfqpoint{1.004257in}{0.707795in}}{\pgfqpoint{1.014856in}{0.703405in}}{\pgfqpoint{1.025906in}{0.703405in}}%
\pgfpathclose%
\pgfusepath{stroke,fill}%
\end{pgfscope}%
\begin{pgfscope}%
\pgfpathrectangle{\pgfqpoint{0.800000in}{0.528000in}}{\pgfqpoint{4.960000in}{3.696000in}}%
\pgfusepath{clip}%
\pgfsetbuttcap%
\pgfsetroundjoin%
\definecolor{currentfill}{rgb}{0.000000,0.000000,0.000000}%
\pgfsetfillcolor{currentfill}%
\pgfsetlinewidth{1.003750pt}%
\definecolor{currentstroke}{rgb}{0.000000,0.000000,0.000000}%
\pgfsetstrokecolor{currentstroke}%
\pgfsetdash{}{0pt}%
\pgfpathmoveto{\pgfqpoint{1.025906in}{0.703405in}}%
\pgfpathcurveto{\pgfqpoint{1.036956in}{0.703405in}}{\pgfqpoint{1.047555in}{0.707795in}}{\pgfqpoint{1.055369in}{0.715608in}}%
\pgfpathcurveto{\pgfqpoint{1.063182in}{0.723422in}}{\pgfqpoint{1.067573in}{0.734021in}}{\pgfqpoint{1.067573in}{0.745071in}}%
\pgfpathcurveto{\pgfqpoint{1.067573in}{0.756121in}}{\pgfqpoint{1.063182in}{0.766720in}}{\pgfqpoint{1.055369in}{0.774534in}}%
\pgfpathcurveto{\pgfqpoint{1.047555in}{0.782348in}}{\pgfqpoint{1.036956in}{0.786738in}}{\pgfqpoint{1.025906in}{0.786738in}}%
\pgfpathcurveto{\pgfqpoint{1.014856in}{0.786738in}}{\pgfqpoint{1.004257in}{0.782348in}}{\pgfqpoint{0.996443in}{0.774534in}}%
\pgfpathcurveto{\pgfqpoint{0.988630in}{0.766720in}}{\pgfqpoint{0.984239in}{0.756121in}}{\pgfqpoint{0.984239in}{0.745071in}}%
\pgfpathcurveto{\pgfqpoint{0.984239in}{0.734021in}}{\pgfqpoint{0.988630in}{0.723422in}}{\pgfqpoint{0.996443in}{0.715608in}}%
\pgfpathcurveto{\pgfqpoint{1.004257in}{0.707795in}}{\pgfqpoint{1.014856in}{0.703405in}}{\pgfqpoint{1.025906in}{0.703405in}}%
\pgfpathclose%
\pgfusepath{stroke,fill}%
\end{pgfscope}%
\begin{pgfscope}%
\pgfpathrectangle{\pgfqpoint{0.800000in}{0.528000in}}{\pgfqpoint{4.960000in}{3.696000in}}%
\pgfusepath{clip}%
\pgfsetbuttcap%
\pgfsetroundjoin%
\definecolor{currentfill}{rgb}{0.000000,0.000000,0.000000}%
\pgfsetfillcolor{currentfill}%
\pgfsetlinewidth{1.003750pt}%
\definecolor{currentstroke}{rgb}{0.000000,0.000000,0.000000}%
\pgfsetstrokecolor{currentstroke}%
\pgfsetdash{}{0pt}%
\pgfpathmoveto{\pgfqpoint{1.025906in}{0.683987in}}%
\pgfpathcurveto{\pgfqpoint{1.036956in}{0.683987in}}{\pgfqpoint{1.047555in}{0.688377in}}{\pgfqpoint{1.055369in}{0.696191in}}%
\pgfpathcurveto{\pgfqpoint{1.063182in}{0.704004in}}{\pgfqpoint{1.067573in}{0.714603in}}{\pgfqpoint{1.067573in}{0.725653in}}%
\pgfpathcurveto{\pgfqpoint{1.067573in}{0.736703in}}{\pgfqpoint{1.063182in}{0.747302in}}{\pgfqpoint{1.055369in}{0.755116in}}%
\pgfpathcurveto{\pgfqpoint{1.047555in}{0.762930in}}{\pgfqpoint{1.036956in}{0.767320in}}{\pgfqpoint{1.025906in}{0.767320in}}%
\pgfpathcurveto{\pgfqpoint{1.014856in}{0.767320in}}{\pgfqpoint{1.004257in}{0.762930in}}{\pgfqpoint{0.996443in}{0.755116in}}%
\pgfpathcurveto{\pgfqpoint{0.988630in}{0.747302in}}{\pgfqpoint{0.984239in}{0.736703in}}{\pgfqpoint{0.984239in}{0.725653in}}%
\pgfpathcurveto{\pgfqpoint{0.984239in}{0.714603in}}{\pgfqpoint{0.988630in}{0.704004in}}{\pgfqpoint{0.996443in}{0.696191in}}%
\pgfpathcurveto{\pgfqpoint{1.004257in}{0.688377in}}{\pgfqpoint{1.014856in}{0.683987in}}{\pgfqpoint{1.025906in}{0.683987in}}%
\pgfpathclose%
\pgfusepath{stroke,fill}%
\end{pgfscope}%
\begin{pgfscope}%
\pgfpathrectangle{\pgfqpoint{0.800000in}{0.528000in}}{\pgfqpoint{4.960000in}{3.696000in}}%
\pgfusepath{clip}%
\pgfsetbuttcap%
\pgfsetroundjoin%
\definecolor{currentfill}{rgb}{0.000000,0.000000,0.000000}%
\pgfsetfillcolor{currentfill}%
\pgfsetlinewidth{1.003750pt}%
\definecolor{currentstroke}{rgb}{0.000000,0.000000,0.000000}%
\pgfsetstrokecolor{currentstroke}%
\pgfsetdash{}{0pt}%
\pgfpathmoveto{\pgfqpoint{1.025906in}{0.703405in}}%
\pgfpathcurveto{\pgfqpoint{1.036956in}{0.703405in}}{\pgfqpoint{1.047555in}{0.707795in}}{\pgfqpoint{1.055369in}{0.715608in}}%
\pgfpathcurveto{\pgfqpoint{1.063182in}{0.723422in}}{\pgfqpoint{1.067573in}{0.734021in}}{\pgfqpoint{1.067573in}{0.745071in}}%
\pgfpathcurveto{\pgfqpoint{1.067573in}{0.756121in}}{\pgfqpoint{1.063182in}{0.766720in}}{\pgfqpoint{1.055369in}{0.774534in}}%
\pgfpathcurveto{\pgfqpoint{1.047555in}{0.782348in}}{\pgfqpoint{1.036956in}{0.786738in}}{\pgfqpoint{1.025906in}{0.786738in}}%
\pgfpathcurveto{\pgfqpoint{1.014856in}{0.786738in}}{\pgfqpoint{1.004257in}{0.782348in}}{\pgfqpoint{0.996443in}{0.774534in}}%
\pgfpathcurveto{\pgfqpoint{0.988630in}{0.766720in}}{\pgfqpoint{0.984239in}{0.756121in}}{\pgfqpoint{0.984239in}{0.745071in}}%
\pgfpathcurveto{\pgfqpoint{0.984239in}{0.734021in}}{\pgfqpoint{0.988630in}{0.723422in}}{\pgfqpoint{0.996443in}{0.715608in}}%
\pgfpathcurveto{\pgfqpoint{1.004257in}{0.707795in}}{\pgfqpoint{1.014856in}{0.703405in}}{\pgfqpoint{1.025906in}{0.703405in}}%
\pgfpathclose%
\pgfusepath{stroke,fill}%
\end{pgfscope}%
\begin{pgfscope}%
\pgfpathrectangle{\pgfqpoint{0.800000in}{0.528000in}}{\pgfqpoint{4.960000in}{3.696000in}}%
\pgfusepath{clip}%
\pgfsetbuttcap%
\pgfsetroundjoin%
\definecolor{currentfill}{rgb}{0.000000,0.000000,0.000000}%
\pgfsetfillcolor{currentfill}%
\pgfsetlinewidth{1.003750pt}%
\definecolor{currentstroke}{rgb}{0.000000,0.000000,0.000000}%
\pgfsetstrokecolor{currentstroke}%
\pgfsetdash{}{0pt}%
\pgfpathmoveto{\pgfqpoint{1.025906in}{0.674278in}}%
\pgfpathcurveto{\pgfqpoint{1.036956in}{0.674278in}}{\pgfqpoint{1.047555in}{0.678668in}}{\pgfqpoint{1.055369in}{0.686482in}}%
\pgfpathcurveto{\pgfqpoint{1.063182in}{0.694295in}}{\pgfqpoint{1.067573in}{0.704894in}}{\pgfqpoint{1.067573in}{0.715944in}}%
\pgfpathcurveto{\pgfqpoint{1.067573in}{0.726994in}}{\pgfqpoint{1.063182in}{0.737594in}}{\pgfqpoint{1.055369in}{0.745407in}}%
\pgfpathcurveto{\pgfqpoint{1.047555in}{0.753221in}}{\pgfqpoint{1.036956in}{0.757611in}}{\pgfqpoint{1.025906in}{0.757611in}}%
\pgfpathcurveto{\pgfqpoint{1.014856in}{0.757611in}}{\pgfqpoint{1.004257in}{0.753221in}}{\pgfqpoint{0.996443in}{0.745407in}}%
\pgfpathcurveto{\pgfqpoint{0.988630in}{0.737594in}}{\pgfqpoint{0.984239in}{0.726994in}}{\pgfqpoint{0.984239in}{0.715944in}}%
\pgfpathcurveto{\pgfqpoint{0.984239in}{0.704894in}}{\pgfqpoint{0.988630in}{0.694295in}}{\pgfqpoint{0.996443in}{0.686482in}}%
\pgfpathcurveto{\pgfqpoint{1.004257in}{0.678668in}}{\pgfqpoint{1.014856in}{0.674278in}}{\pgfqpoint{1.025906in}{0.674278in}}%
\pgfpathclose%
\pgfusepath{stroke,fill}%
\end{pgfscope}%
\begin{pgfscope}%
\pgfpathrectangle{\pgfqpoint{0.800000in}{0.528000in}}{\pgfqpoint{4.960000in}{3.696000in}}%
\pgfusepath{clip}%
\pgfsetbuttcap%
\pgfsetroundjoin%
\definecolor{currentfill}{rgb}{0.000000,0.000000,0.000000}%
\pgfsetfillcolor{currentfill}%
\pgfsetlinewidth{1.003750pt}%
\definecolor{currentstroke}{rgb}{0.000000,0.000000,0.000000}%
\pgfsetstrokecolor{currentstroke}%
\pgfsetdash{}{0pt}%
\pgfpathmoveto{\pgfqpoint{1.025906in}{0.674278in}}%
\pgfpathcurveto{\pgfqpoint{1.036956in}{0.674278in}}{\pgfqpoint{1.047555in}{0.678668in}}{\pgfqpoint{1.055369in}{0.686482in}}%
\pgfpathcurveto{\pgfqpoint{1.063182in}{0.694295in}}{\pgfqpoint{1.067573in}{0.704894in}}{\pgfqpoint{1.067573in}{0.715944in}}%
\pgfpathcurveto{\pgfqpoint{1.067573in}{0.726994in}}{\pgfqpoint{1.063182in}{0.737594in}}{\pgfqpoint{1.055369in}{0.745407in}}%
\pgfpathcurveto{\pgfqpoint{1.047555in}{0.753221in}}{\pgfqpoint{1.036956in}{0.757611in}}{\pgfqpoint{1.025906in}{0.757611in}}%
\pgfpathcurveto{\pgfqpoint{1.014856in}{0.757611in}}{\pgfqpoint{1.004257in}{0.753221in}}{\pgfqpoint{0.996443in}{0.745407in}}%
\pgfpathcurveto{\pgfqpoint{0.988630in}{0.737594in}}{\pgfqpoint{0.984239in}{0.726994in}}{\pgfqpoint{0.984239in}{0.715944in}}%
\pgfpathcurveto{\pgfqpoint{0.984239in}{0.704894in}}{\pgfqpoint{0.988630in}{0.694295in}}{\pgfqpoint{0.996443in}{0.686482in}}%
\pgfpathcurveto{\pgfqpoint{1.004257in}{0.678668in}}{\pgfqpoint{1.014856in}{0.674278in}}{\pgfqpoint{1.025906in}{0.674278in}}%
\pgfpathclose%
\pgfusepath{stroke,fill}%
\end{pgfscope}%
\begin{pgfscope}%
\pgfpathrectangle{\pgfqpoint{0.800000in}{0.528000in}}{\pgfqpoint{4.960000in}{3.696000in}}%
\pgfusepath{clip}%
\pgfsetbuttcap%
\pgfsetroundjoin%
\definecolor{currentfill}{rgb}{0.000000,0.000000,0.000000}%
\pgfsetfillcolor{currentfill}%
\pgfsetlinewidth{1.003750pt}%
\definecolor{currentstroke}{rgb}{0.000000,0.000000,0.000000}%
\pgfsetstrokecolor{currentstroke}%
\pgfsetdash{}{0pt}%
\pgfpathmoveto{\pgfqpoint{1.025906in}{0.683987in}}%
\pgfpathcurveto{\pgfqpoint{1.036956in}{0.683987in}}{\pgfqpoint{1.047555in}{0.688377in}}{\pgfqpoint{1.055369in}{0.696191in}}%
\pgfpathcurveto{\pgfqpoint{1.063182in}{0.704004in}}{\pgfqpoint{1.067573in}{0.714603in}}{\pgfqpoint{1.067573in}{0.725653in}}%
\pgfpathcurveto{\pgfqpoint{1.067573in}{0.736703in}}{\pgfqpoint{1.063182in}{0.747302in}}{\pgfqpoint{1.055369in}{0.755116in}}%
\pgfpathcurveto{\pgfqpoint{1.047555in}{0.762930in}}{\pgfqpoint{1.036956in}{0.767320in}}{\pgfqpoint{1.025906in}{0.767320in}}%
\pgfpathcurveto{\pgfqpoint{1.014856in}{0.767320in}}{\pgfqpoint{1.004257in}{0.762930in}}{\pgfqpoint{0.996443in}{0.755116in}}%
\pgfpathcurveto{\pgfqpoint{0.988630in}{0.747302in}}{\pgfqpoint{0.984239in}{0.736703in}}{\pgfqpoint{0.984239in}{0.725653in}}%
\pgfpathcurveto{\pgfqpoint{0.984239in}{0.714603in}}{\pgfqpoint{0.988630in}{0.704004in}}{\pgfqpoint{0.996443in}{0.696191in}}%
\pgfpathcurveto{\pgfqpoint{1.004257in}{0.688377in}}{\pgfqpoint{1.014856in}{0.683987in}}{\pgfqpoint{1.025906in}{0.683987in}}%
\pgfpathclose%
\pgfusepath{stroke,fill}%
\end{pgfscope}%
\begin{pgfscope}%
\pgfpathrectangle{\pgfqpoint{0.800000in}{0.528000in}}{\pgfqpoint{4.960000in}{3.696000in}}%
\pgfusepath{clip}%
\pgfsetbuttcap%
\pgfsetroundjoin%
\definecolor{currentfill}{rgb}{0.000000,0.000000,0.000000}%
\pgfsetfillcolor{currentfill}%
\pgfsetlinewidth{1.003750pt}%
\definecolor{currentstroke}{rgb}{0.000000,0.000000,0.000000}%
\pgfsetstrokecolor{currentstroke}%
\pgfsetdash{}{0pt}%
\pgfpathmoveto{\pgfqpoint{1.025906in}{0.722823in}}%
\pgfpathcurveto{\pgfqpoint{1.036956in}{0.722823in}}{\pgfqpoint{1.047555in}{0.727213in}}{\pgfqpoint{1.055369in}{0.735026in}}%
\pgfpathcurveto{\pgfqpoint{1.063182in}{0.742840in}}{\pgfqpoint{1.067573in}{0.753439in}}{\pgfqpoint{1.067573in}{0.764489in}}%
\pgfpathcurveto{\pgfqpoint{1.067573in}{0.775539in}}{\pgfqpoint{1.063182in}{0.786138in}}{\pgfqpoint{1.055369in}{0.793952in}}%
\pgfpathcurveto{\pgfqpoint{1.047555in}{0.801766in}}{\pgfqpoint{1.036956in}{0.806156in}}{\pgfqpoint{1.025906in}{0.806156in}}%
\pgfpathcurveto{\pgfqpoint{1.014856in}{0.806156in}}{\pgfqpoint{1.004257in}{0.801766in}}{\pgfqpoint{0.996443in}{0.793952in}}%
\pgfpathcurveto{\pgfqpoint{0.988630in}{0.786138in}}{\pgfqpoint{0.984239in}{0.775539in}}{\pgfqpoint{0.984239in}{0.764489in}}%
\pgfpathcurveto{\pgfqpoint{0.984239in}{0.753439in}}{\pgfqpoint{0.988630in}{0.742840in}}{\pgfqpoint{0.996443in}{0.735026in}}%
\pgfpathcurveto{\pgfqpoint{1.004257in}{0.727213in}}{\pgfqpoint{1.014856in}{0.722823in}}{\pgfqpoint{1.025906in}{0.722823in}}%
\pgfpathclose%
\pgfusepath{stroke,fill}%
\end{pgfscope}%
\begin{pgfscope}%
\pgfpathrectangle{\pgfqpoint{0.800000in}{0.528000in}}{\pgfqpoint{4.960000in}{3.696000in}}%
\pgfusepath{clip}%
\pgfsetbuttcap%
\pgfsetroundjoin%
\definecolor{currentfill}{rgb}{0.000000,0.000000,0.000000}%
\pgfsetfillcolor{currentfill}%
\pgfsetlinewidth{1.003750pt}%
\definecolor{currentstroke}{rgb}{0.000000,0.000000,0.000000}%
\pgfsetstrokecolor{currentstroke}%
\pgfsetdash{}{0pt}%
\pgfpathmoveto{\pgfqpoint{1.025906in}{0.693696in}}%
\pgfpathcurveto{\pgfqpoint{1.036956in}{0.693696in}}{\pgfqpoint{1.047555in}{0.698086in}}{\pgfqpoint{1.055369in}{0.705900in}}%
\pgfpathcurveto{\pgfqpoint{1.063182in}{0.713713in}}{\pgfqpoint{1.067573in}{0.724312in}}{\pgfqpoint{1.067573in}{0.735362in}}%
\pgfpathcurveto{\pgfqpoint{1.067573in}{0.746412in}}{\pgfqpoint{1.063182in}{0.757011in}}{\pgfqpoint{1.055369in}{0.764825in}}%
\pgfpathcurveto{\pgfqpoint{1.047555in}{0.772639in}}{\pgfqpoint{1.036956in}{0.777029in}}{\pgfqpoint{1.025906in}{0.777029in}}%
\pgfpathcurveto{\pgfqpoint{1.014856in}{0.777029in}}{\pgfqpoint{1.004257in}{0.772639in}}{\pgfqpoint{0.996443in}{0.764825in}}%
\pgfpathcurveto{\pgfqpoint{0.988630in}{0.757011in}}{\pgfqpoint{0.984239in}{0.746412in}}{\pgfqpoint{0.984239in}{0.735362in}}%
\pgfpathcurveto{\pgfqpoint{0.984239in}{0.724312in}}{\pgfqpoint{0.988630in}{0.713713in}}{\pgfqpoint{0.996443in}{0.705900in}}%
\pgfpathcurveto{\pgfqpoint{1.004257in}{0.698086in}}{\pgfqpoint{1.014856in}{0.693696in}}{\pgfqpoint{1.025906in}{0.693696in}}%
\pgfpathclose%
\pgfusepath{stroke,fill}%
\end{pgfscope}%
\begin{pgfscope}%
\pgfpathrectangle{\pgfqpoint{0.800000in}{0.528000in}}{\pgfqpoint{4.960000in}{3.696000in}}%
\pgfusepath{clip}%
\pgfsetbuttcap%
\pgfsetroundjoin%
\definecolor{currentfill}{rgb}{0.000000,0.000000,0.000000}%
\pgfsetfillcolor{currentfill}%
\pgfsetlinewidth{1.003750pt}%
\definecolor{currentstroke}{rgb}{0.000000,0.000000,0.000000}%
\pgfsetstrokecolor{currentstroke}%
\pgfsetdash{}{0pt}%
\pgfpathmoveto{\pgfqpoint{1.025906in}{0.664569in}}%
\pgfpathcurveto{\pgfqpoint{1.036956in}{0.664569in}}{\pgfqpoint{1.047555in}{0.668959in}}{\pgfqpoint{1.055369in}{0.676773in}}%
\pgfpathcurveto{\pgfqpoint{1.063182in}{0.684586in}}{\pgfqpoint{1.067573in}{0.695185in}}{\pgfqpoint{1.067573in}{0.706235in}}%
\pgfpathcurveto{\pgfqpoint{1.067573in}{0.717286in}}{\pgfqpoint{1.063182in}{0.727885in}}{\pgfqpoint{1.055369in}{0.735698in}}%
\pgfpathcurveto{\pgfqpoint{1.047555in}{0.743512in}}{\pgfqpoint{1.036956in}{0.747902in}}{\pgfqpoint{1.025906in}{0.747902in}}%
\pgfpathcurveto{\pgfqpoint{1.014856in}{0.747902in}}{\pgfqpoint{1.004257in}{0.743512in}}{\pgfqpoint{0.996443in}{0.735698in}}%
\pgfpathcurveto{\pgfqpoint{0.988630in}{0.727885in}}{\pgfqpoint{0.984239in}{0.717286in}}{\pgfqpoint{0.984239in}{0.706235in}}%
\pgfpathcurveto{\pgfqpoint{0.984239in}{0.695185in}}{\pgfqpoint{0.988630in}{0.684586in}}{\pgfqpoint{0.996443in}{0.676773in}}%
\pgfpathcurveto{\pgfqpoint{1.004257in}{0.668959in}}{\pgfqpoint{1.014856in}{0.664569in}}{\pgfqpoint{1.025906in}{0.664569in}}%
\pgfpathclose%
\pgfusepath{stroke,fill}%
\end{pgfscope}%
\begin{pgfscope}%
\pgfpathrectangle{\pgfqpoint{0.800000in}{0.528000in}}{\pgfqpoint{4.960000in}{3.696000in}}%
\pgfusepath{clip}%
\pgfsetbuttcap%
\pgfsetroundjoin%
\definecolor{currentfill}{rgb}{0.000000,0.000000,0.000000}%
\pgfsetfillcolor{currentfill}%
\pgfsetlinewidth{1.003750pt}%
\definecolor{currentstroke}{rgb}{0.000000,0.000000,0.000000}%
\pgfsetstrokecolor{currentstroke}%
\pgfsetdash{}{0pt}%
\pgfpathmoveto{\pgfqpoint{1.025906in}{0.683987in}}%
\pgfpathcurveto{\pgfqpoint{1.036956in}{0.683987in}}{\pgfqpoint{1.047555in}{0.688377in}}{\pgfqpoint{1.055369in}{0.696191in}}%
\pgfpathcurveto{\pgfqpoint{1.063182in}{0.704004in}}{\pgfqpoint{1.067573in}{0.714603in}}{\pgfqpoint{1.067573in}{0.725653in}}%
\pgfpathcurveto{\pgfqpoint{1.067573in}{0.736703in}}{\pgfqpoint{1.063182in}{0.747302in}}{\pgfqpoint{1.055369in}{0.755116in}}%
\pgfpathcurveto{\pgfqpoint{1.047555in}{0.762930in}}{\pgfqpoint{1.036956in}{0.767320in}}{\pgfqpoint{1.025906in}{0.767320in}}%
\pgfpathcurveto{\pgfqpoint{1.014856in}{0.767320in}}{\pgfqpoint{1.004257in}{0.762930in}}{\pgfqpoint{0.996443in}{0.755116in}}%
\pgfpathcurveto{\pgfqpoint{0.988630in}{0.747302in}}{\pgfqpoint{0.984239in}{0.736703in}}{\pgfqpoint{0.984239in}{0.725653in}}%
\pgfpathcurveto{\pgfqpoint{0.984239in}{0.714603in}}{\pgfqpoint{0.988630in}{0.704004in}}{\pgfqpoint{0.996443in}{0.696191in}}%
\pgfpathcurveto{\pgfqpoint{1.004257in}{0.688377in}}{\pgfqpoint{1.014856in}{0.683987in}}{\pgfqpoint{1.025906in}{0.683987in}}%
\pgfpathclose%
\pgfusepath{stroke,fill}%
\end{pgfscope}%
\begin{pgfscope}%
\pgfpathrectangle{\pgfqpoint{0.800000in}{0.528000in}}{\pgfqpoint{4.960000in}{3.696000in}}%
\pgfusepath{clip}%
\pgfsetbuttcap%
\pgfsetroundjoin%
\definecolor{currentfill}{rgb}{0.000000,0.000000,0.000000}%
\pgfsetfillcolor{currentfill}%
\pgfsetlinewidth{1.003750pt}%
\definecolor{currentstroke}{rgb}{0.000000,0.000000,0.000000}%
\pgfsetstrokecolor{currentstroke}%
\pgfsetdash{}{0pt}%
\pgfpathmoveto{\pgfqpoint{1.025906in}{0.713114in}}%
\pgfpathcurveto{\pgfqpoint{1.036956in}{0.713114in}}{\pgfqpoint{1.047555in}{0.717504in}}{\pgfqpoint{1.055369in}{0.725317in}}%
\pgfpathcurveto{\pgfqpoint{1.063182in}{0.733131in}}{\pgfqpoint{1.067573in}{0.743730in}}{\pgfqpoint{1.067573in}{0.754780in}}%
\pgfpathcurveto{\pgfqpoint{1.067573in}{0.765830in}}{\pgfqpoint{1.063182in}{0.776429in}}{\pgfqpoint{1.055369in}{0.784243in}}%
\pgfpathcurveto{\pgfqpoint{1.047555in}{0.792057in}}{\pgfqpoint{1.036956in}{0.796447in}}{\pgfqpoint{1.025906in}{0.796447in}}%
\pgfpathcurveto{\pgfqpoint{1.014856in}{0.796447in}}{\pgfqpoint{1.004257in}{0.792057in}}{\pgfqpoint{0.996443in}{0.784243in}}%
\pgfpathcurveto{\pgfqpoint{0.988630in}{0.776429in}}{\pgfqpoint{0.984239in}{0.765830in}}{\pgfqpoint{0.984239in}{0.754780in}}%
\pgfpathcurveto{\pgfqpoint{0.984239in}{0.743730in}}{\pgfqpoint{0.988630in}{0.733131in}}{\pgfqpoint{0.996443in}{0.725317in}}%
\pgfpathcurveto{\pgfqpoint{1.004257in}{0.717504in}}{\pgfqpoint{1.014856in}{0.713114in}}{\pgfqpoint{1.025906in}{0.713114in}}%
\pgfpathclose%
\pgfusepath{stroke,fill}%
\end{pgfscope}%
\begin{pgfscope}%
\pgfpathrectangle{\pgfqpoint{0.800000in}{0.528000in}}{\pgfqpoint{4.960000in}{3.696000in}}%
\pgfusepath{clip}%
\pgfsetbuttcap%
\pgfsetroundjoin%
\definecolor{currentfill}{rgb}{0.000000,0.000000,0.000000}%
\pgfsetfillcolor{currentfill}%
\pgfsetlinewidth{1.003750pt}%
\definecolor{currentstroke}{rgb}{0.000000,0.000000,0.000000}%
\pgfsetstrokecolor{currentstroke}%
\pgfsetdash{}{0pt}%
\pgfpathmoveto{\pgfqpoint{1.025906in}{0.683987in}}%
\pgfpathcurveto{\pgfqpoint{1.036956in}{0.683987in}}{\pgfqpoint{1.047555in}{0.688377in}}{\pgfqpoint{1.055369in}{0.696191in}}%
\pgfpathcurveto{\pgfqpoint{1.063182in}{0.704004in}}{\pgfqpoint{1.067573in}{0.714603in}}{\pgfqpoint{1.067573in}{0.725653in}}%
\pgfpathcurveto{\pgfqpoint{1.067573in}{0.736703in}}{\pgfqpoint{1.063182in}{0.747302in}}{\pgfqpoint{1.055369in}{0.755116in}}%
\pgfpathcurveto{\pgfqpoint{1.047555in}{0.762930in}}{\pgfqpoint{1.036956in}{0.767320in}}{\pgfqpoint{1.025906in}{0.767320in}}%
\pgfpathcurveto{\pgfqpoint{1.014856in}{0.767320in}}{\pgfqpoint{1.004257in}{0.762930in}}{\pgfqpoint{0.996443in}{0.755116in}}%
\pgfpathcurveto{\pgfqpoint{0.988630in}{0.747302in}}{\pgfqpoint{0.984239in}{0.736703in}}{\pgfqpoint{0.984239in}{0.725653in}}%
\pgfpathcurveto{\pgfqpoint{0.984239in}{0.714603in}}{\pgfqpoint{0.988630in}{0.704004in}}{\pgfqpoint{0.996443in}{0.696191in}}%
\pgfpathcurveto{\pgfqpoint{1.004257in}{0.688377in}}{\pgfqpoint{1.014856in}{0.683987in}}{\pgfqpoint{1.025906in}{0.683987in}}%
\pgfpathclose%
\pgfusepath{stroke,fill}%
\end{pgfscope}%
\begin{pgfscope}%
\pgfpathrectangle{\pgfqpoint{0.800000in}{0.528000in}}{\pgfqpoint{4.960000in}{3.696000in}}%
\pgfusepath{clip}%
\pgfsetbuttcap%
\pgfsetroundjoin%
\definecolor{currentfill}{rgb}{0.000000,0.000000,0.000000}%
\pgfsetfillcolor{currentfill}%
\pgfsetlinewidth{1.003750pt}%
\definecolor{currentstroke}{rgb}{0.000000,0.000000,0.000000}%
\pgfsetstrokecolor{currentstroke}%
\pgfsetdash{}{0pt}%
\pgfpathmoveto{\pgfqpoint{1.025906in}{0.713114in}}%
\pgfpathcurveto{\pgfqpoint{1.036956in}{0.713114in}}{\pgfqpoint{1.047555in}{0.717504in}}{\pgfqpoint{1.055369in}{0.725317in}}%
\pgfpathcurveto{\pgfqpoint{1.063182in}{0.733131in}}{\pgfqpoint{1.067573in}{0.743730in}}{\pgfqpoint{1.067573in}{0.754780in}}%
\pgfpathcurveto{\pgfqpoint{1.067573in}{0.765830in}}{\pgfqpoint{1.063182in}{0.776429in}}{\pgfqpoint{1.055369in}{0.784243in}}%
\pgfpathcurveto{\pgfqpoint{1.047555in}{0.792057in}}{\pgfqpoint{1.036956in}{0.796447in}}{\pgfqpoint{1.025906in}{0.796447in}}%
\pgfpathcurveto{\pgfqpoint{1.014856in}{0.796447in}}{\pgfqpoint{1.004257in}{0.792057in}}{\pgfqpoint{0.996443in}{0.784243in}}%
\pgfpathcurveto{\pgfqpoint{0.988630in}{0.776429in}}{\pgfqpoint{0.984239in}{0.765830in}}{\pgfqpoint{0.984239in}{0.754780in}}%
\pgfpathcurveto{\pgfqpoint{0.984239in}{0.743730in}}{\pgfqpoint{0.988630in}{0.733131in}}{\pgfqpoint{0.996443in}{0.725317in}}%
\pgfpathcurveto{\pgfqpoint{1.004257in}{0.717504in}}{\pgfqpoint{1.014856in}{0.713114in}}{\pgfqpoint{1.025906in}{0.713114in}}%
\pgfpathclose%
\pgfusepath{stroke,fill}%
\end{pgfscope}%
\begin{pgfscope}%
\pgfpathrectangle{\pgfqpoint{0.800000in}{0.528000in}}{\pgfqpoint{4.960000in}{3.696000in}}%
\pgfusepath{clip}%
\pgfsetbuttcap%
\pgfsetroundjoin%
\definecolor{currentfill}{rgb}{0.000000,0.000000,0.000000}%
\pgfsetfillcolor{currentfill}%
\pgfsetlinewidth{1.003750pt}%
\definecolor{currentstroke}{rgb}{0.000000,0.000000,0.000000}%
\pgfsetstrokecolor{currentstroke}%
\pgfsetdash{}{0pt}%
\pgfpathmoveto{\pgfqpoint{1.025906in}{0.703405in}}%
\pgfpathcurveto{\pgfqpoint{1.036956in}{0.703405in}}{\pgfqpoint{1.047555in}{0.707795in}}{\pgfqpoint{1.055369in}{0.715608in}}%
\pgfpathcurveto{\pgfqpoint{1.063182in}{0.723422in}}{\pgfqpoint{1.067573in}{0.734021in}}{\pgfqpoint{1.067573in}{0.745071in}}%
\pgfpathcurveto{\pgfqpoint{1.067573in}{0.756121in}}{\pgfqpoint{1.063182in}{0.766720in}}{\pgfqpoint{1.055369in}{0.774534in}}%
\pgfpathcurveto{\pgfqpoint{1.047555in}{0.782348in}}{\pgfqpoint{1.036956in}{0.786738in}}{\pgfqpoint{1.025906in}{0.786738in}}%
\pgfpathcurveto{\pgfqpoint{1.014856in}{0.786738in}}{\pgfqpoint{1.004257in}{0.782348in}}{\pgfqpoint{0.996443in}{0.774534in}}%
\pgfpathcurveto{\pgfqpoint{0.988630in}{0.766720in}}{\pgfqpoint{0.984239in}{0.756121in}}{\pgfqpoint{0.984239in}{0.745071in}}%
\pgfpathcurveto{\pgfqpoint{0.984239in}{0.734021in}}{\pgfqpoint{0.988630in}{0.723422in}}{\pgfqpoint{0.996443in}{0.715608in}}%
\pgfpathcurveto{\pgfqpoint{1.004257in}{0.707795in}}{\pgfqpoint{1.014856in}{0.703405in}}{\pgfqpoint{1.025906in}{0.703405in}}%
\pgfpathclose%
\pgfusepath{stroke,fill}%
\end{pgfscope}%
\begin{pgfscope}%
\pgfpathrectangle{\pgfqpoint{0.800000in}{0.528000in}}{\pgfqpoint{4.960000in}{3.696000in}}%
\pgfusepath{clip}%
\pgfsetbuttcap%
\pgfsetroundjoin%
\definecolor{currentfill}{rgb}{0.000000,0.000000,0.000000}%
\pgfsetfillcolor{currentfill}%
\pgfsetlinewidth{1.003750pt}%
\definecolor{currentstroke}{rgb}{0.000000,0.000000,0.000000}%
\pgfsetstrokecolor{currentstroke}%
\pgfsetdash{}{0pt}%
\pgfpathmoveto{\pgfqpoint{1.025906in}{0.683987in}}%
\pgfpathcurveto{\pgfqpoint{1.036956in}{0.683987in}}{\pgfqpoint{1.047555in}{0.688377in}}{\pgfqpoint{1.055369in}{0.696191in}}%
\pgfpathcurveto{\pgfqpoint{1.063182in}{0.704004in}}{\pgfqpoint{1.067573in}{0.714603in}}{\pgfqpoint{1.067573in}{0.725653in}}%
\pgfpathcurveto{\pgfqpoint{1.067573in}{0.736703in}}{\pgfqpoint{1.063182in}{0.747302in}}{\pgfqpoint{1.055369in}{0.755116in}}%
\pgfpathcurveto{\pgfqpoint{1.047555in}{0.762930in}}{\pgfqpoint{1.036956in}{0.767320in}}{\pgfqpoint{1.025906in}{0.767320in}}%
\pgfpathcurveto{\pgfqpoint{1.014856in}{0.767320in}}{\pgfqpoint{1.004257in}{0.762930in}}{\pgfqpoint{0.996443in}{0.755116in}}%
\pgfpathcurveto{\pgfqpoint{0.988630in}{0.747302in}}{\pgfqpoint{0.984239in}{0.736703in}}{\pgfqpoint{0.984239in}{0.725653in}}%
\pgfpathcurveto{\pgfqpoint{0.984239in}{0.714603in}}{\pgfqpoint{0.988630in}{0.704004in}}{\pgfqpoint{0.996443in}{0.696191in}}%
\pgfpathcurveto{\pgfqpoint{1.004257in}{0.688377in}}{\pgfqpoint{1.014856in}{0.683987in}}{\pgfqpoint{1.025906in}{0.683987in}}%
\pgfpathclose%
\pgfusepath{stroke,fill}%
\end{pgfscope}%
\begin{pgfscope}%
\pgfpathrectangle{\pgfqpoint{0.800000in}{0.528000in}}{\pgfqpoint{4.960000in}{3.696000in}}%
\pgfusepath{clip}%
\pgfsetbuttcap%
\pgfsetroundjoin%
\definecolor{currentfill}{rgb}{0.000000,0.000000,0.000000}%
\pgfsetfillcolor{currentfill}%
\pgfsetlinewidth{1.003750pt}%
\definecolor{currentstroke}{rgb}{0.000000,0.000000,0.000000}%
\pgfsetstrokecolor{currentstroke}%
\pgfsetdash{}{0pt}%
\pgfpathmoveto{\pgfqpoint{1.025906in}{0.683987in}}%
\pgfpathcurveto{\pgfqpoint{1.036956in}{0.683987in}}{\pgfqpoint{1.047555in}{0.688377in}}{\pgfqpoint{1.055369in}{0.696191in}}%
\pgfpathcurveto{\pgfqpoint{1.063182in}{0.704004in}}{\pgfqpoint{1.067573in}{0.714603in}}{\pgfqpoint{1.067573in}{0.725653in}}%
\pgfpathcurveto{\pgfqpoint{1.067573in}{0.736703in}}{\pgfqpoint{1.063182in}{0.747302in}}{\pgfqpoint{1.055369in}{0.755116in}}%
\pgfpathcurveto{\pgfqpoint{1.047555in}{0.762930in}}{\pgfqpoint{1.036956in}{0.767320in}}{\pgfqpoint{1.025906in}{0.767320in}}%
\pgfpathcurveto{\pgfqpoint{1.014856in}{0.767320in}}{\pgfqpoint{1.004257in}{0.762930in}}{\pgfqpoint{0.996443in}{0.755116in}}%
\pgfpathcurveto{\pgfqpoint{0.988630in}{0.747302in}}{\pgfqpoint{0.984239in}{0.736703in}}{\pgfqpoint{0.984239in}{0.725653in}}%
\pgfpathcurveto{\pgfqpoint{0.984239in}{0.714603in}}{\pgfqpoint{0.988630in}{0.704004in}}{\pgfqpoint{0.996443in}{0.696191in}}%
\pgfpathcurveto{\pgfqpoint{1.004257in}{0.688377in}}{\pgfqpoint{1.014856in}{0.683987in}}{\pgfqpoint{1.025906in}{0.683987in}}%
\pgfpathclose%
\pgfusepath{stroke,fill}%
\end{pgfscope}%
\begin{pgfscope}%
\pgfpathrectangle{\pgfqpoint{0.800000in}{0.528000in}}{\pgfqpoint{4.960000in}{3.696000in}}%
\pgfusepath{clip}%
\pgfsetbuttcap%
\pgfsetroundjoin%
\definecolor{currentfill}{rgb}{0.000000,0.000000,0.000000}%
\pgfsetfillcolor{currentfill}%
\pgfsetlinewidth{1.003750pt}%
\definecolor{currentstroke}{rgb}{0.000000,0.000000,0.000000}%
\pgfsetstrokecolor{currentstroke}%
\pgfsetdash{}{0pt}%
\pgfpathmoveto{\pgfqpoint{1.025906in}{0.703405in}}%
\pgfpathcurveto{\pgfqpoint{1.036956in}{0.703405in}}{\pgfqpoint{1.047555in}{0.707795in}}{\pgfqpoint{1.055369in}{0.715608in}}%
\pgfpathcurveto{\pgfqpoint{1.063182in}{0.723422in}}{\pgfqpoint{1.067573in}{0.734021in}}{\pgfqpoint{1.067573in}{0.745071in}}%
\pgfpathcurveto{\pgfqpoint{1.067573in}{0.756121in}}{\pgfqpoint{1.063182in}{0.766720in}}{\pgfqpoint{1.055369in}{0.774534in}}%
\pgfpathcurveto{\pgfqpoint{1.047555in}{0.782348in}}{\pgfqpoint{1.036956in}{0.786738in}}{\pgfqpoint{1.025906in}{0.786738in}}%
\pgfpathcurveto{\pgfqpoint{1.014856in}{0.786738in}}{\pgfqpoint{1.004257in}{0.782348in}}{\pgfqpoint{0.996443in}{0.774534in}}%
\pgfpathcurveto{\pgfqpoint{0.988630in}{0.766720in}}{\pgfqpoint{0.984239in}{0.756121in}}{\pgfqpoint{0.984239in}{0.745071in}}%
\pgfpathcurveto{\pgfqpoint{0.984239in}{0.734021in}}{\pgfqpoint{0.988630in}{0.723422in}}{\pgfqpoint{0.996443in}{0.715608in}}%
\pgfpathcurveto{\pgfqpoint{1.004257in}{0.707795in}}{\pgfqpoint{1.014856in}{0.703405in}}{\pgfqpoint{1.025906in}{0.703405in}}%
\pgfpathclose%
\pgfusepath{stroke,fill}%
\end{pgfscope}%
\begin{pgfscope}%
\pgfpathrectangle{\pgfqpoint{0.800000in}{0.528000in}}{\pgfqpoint{4.960000in}{3.696000in}}%
\pgfusepath{clip}%
\pgfsetbuttcap%
\pgfsetroundjoin%
\definecolor{currentfill}{rgb}{0.000000,0.000000,0.000000}%
\pgfsetfillcolor{currentfill}%
\pgfsetlinewidth{1.003750pt}%
\definecolor{currentstroke}{rgb}{0.000000,0.000000,0.000000}%
\pgfsetstrokecolor{currentstroke}%
\pgfsetdash{}{0pt}%
\pgfpathmoveto{\pgfqpoint{1.025906in}{0.683987in}}%
\pgfpathcurveto{\pgfqpoint{1.036956in}{0.683987in}}{\pgfqpoint{1.047555in}{0.688377in}}{\pgfqpoint{1.055369in}{0.696191in}}%
\pgfpathcurveto{\pgfqpoint{1.063182in}{0.704004in}}{\pgfqpoint{1.067573in}{0.714603in}}{\pgfqpoint{1.067573in}{0.725653in}}%
\pgfpathcurveto{\pgfqpoint{1.067573in}{0.736703in}}{\pgfqpoint{1.063182in}{0.747302in}}{\pgfqpoint{1.055369in}{0.755116in}}%
\pgfpathcurveto{\pgfqpoint{1.047555in}{0.762930in}}{\pgfqpoint{1.036956in}{0.767320in}}{\pgfqpoint{1.025906in}{0.767320in}}%
\pgfpathcurveto{\pgfqpoint{1.014856in}{0.767320in}}{\pgfqpoint{1.004257in}{0.762930in}}{\pgfqpoint{0.996443in}{0.755116in}}%
\pgfpathcurveto{\pgfqpoint{0.988630in}{0.747302in}}{\pgfqpoint{0.984239in}{0.736703in}}{\pgfqpoint{0.984239in}{0.725653in}}%
\pgfpathcurveto{\pgfqpoint{0.984239in}{0.714603in}}{\pgfqpoint{0.988630in}{0.704004in}}{\pgfqpoint{0.996443in}{0.696191in}}%
\pgfpathcurveto{\pgfqpoint{1.004257in}{0.688377in}}{\pgfqpoint{1.014856in}{0.683987in}}{\pgfqpoint{1.025906in}{0.683987in}}%
\pgfpathclose%
\pgfusepath{stroke,fill}%
\end{pgfscope}%
\begin{pgfscope}%
\pgfpathrectangle{\pgfqpoint{0.800000in}{0.528000in}}{\pgfqpoint{4.960000in}{3.696000in}}%
\pgfusepath{clip}%
\pgfsetbuttcap%
\pgfsetroundjoin%
\definecolor{currentfill}{rgb}{0.000000,0.000000,0.000000}%
\pgfsetfillcolor{currentfill}%
\pgfsetlinewidth{1.003750pt}%
\definecolor{currentstroke}{rgb}{0.000000,0.000000,0.000000}%
\pgfsetstrokecolor{currentstroke}%
\pgfsetdash{}{0pt}%
\pgfpathmoveto{\pgfqpoint{1.025906in}{0.693696in}}%
\pgfpathcurveto{\pgfqpoint{1.036956in}{0.693696in}}{\pgfqpoint{1.047555in}{0.698086in}}{\pgfqpoint{1.055369in}{0.705900in}}%
\pgfpathcurveto{\pgfqpoint{1.063182in}{0.713713in}}{\pgfqpoint{1.067573in}{0.724312in}}{\pgfqpoint{1.067573in}{0.735362in}}%
\pgfpathcurveto{\pgfqpoint{1.067573in}{0.746412in}}{\pgfqpoint{1.063182in}{0.757011in}}{\pgfqpoint{1.055369in}{0.764825in}}%
\pgfpathcurveto{\pgfqpoint{1.047555in}{0.772639in}}{\pgfqpoint{1.036956in}{0.777029in}}{\pgfqpoint{1.025906in}{0.777029in}}%
\pgfpathcurveto{\pgfqpoint{1.014856in}{0.777029in}}{\pgfqpoint{1.004257in}{0.772639in}}{\pgfqpoint{0.996443in}{0.764825in}}%
\pgfpathcurveto{\pgfqpoint{0.988630in}{0.757011in}}{\pgfqpoint{0.984239in}{0.746412in}}{\pgfqpoint{0.984239in}{0.735362in}}%
\pgfpathcurveto{\pgfqpoint{0.984239in}{0.724312in}}{\pgfqpoint{0.988630in}{0.713713in}}{\pgfqpoint{0.996443in}{0.705900in}}%
\pgfpathcurveto{\pgfqpoint{1.004257in}{0.698086in}}{\pgfqpoint{1.014856in}{0.693696in}}{\pgfqpoint{1.025906in}{0.693696in}}%
\pgfpathclose%
\pgfusepath{stroke,fill}%
\end{pgfscope}%
\begin{pgfscope}%
\pgfpathrectangle{\pgfqpoint{0.800000in}{0.528000in}}{\pgfqpoint{4.960000in}{3.696000in}}%
\pgfusepath{clip}%
\pgfsetbuttcap%
\pgfsetroundjoin%
\definecolor{currentfill}{rgb}{0.000000,0.000000,0.000000}%
\pgfsetfillcolor{currentfill}%
\pgfsetlinewidth{1.003750pt}%
\definecolor{currentstroke}{rgb}{0.000000,0.000000,0.000000}%
\pgfsetstrokecolor{currentstroke}%
\pgfsetdash{}{0pt}%
\pgfpathmoveto{\pgfqpoint{1.025906in}{0.693696in}}%
\pgfpathcurveto{\pgfqpoint{1.036956in}{0.693696in}}{\pgfqpoint{1.047555in}{0.698086in}}{\pgfqpoint{1.055369in}{0.705900in}}%
\pgfpathcurveto{\pgfqpoint{1.063182in}{0.713713in}}{\pgfqpoint{1.067573in}{0.724312in}}{\pgfqpoint{1.067573in}{0.735362in}}%
\pgfpathcurveto{\pgfqpoint{1.067573in}{0.746412in}}{\pgfqpoint{1.063182in}{0.757011in}}{\pgfqpoint{1.055369in}{0.764825in}}%
\pgfpathcurveto{\pgfqpoint{1.047555in}{0.772639in}}{\pgfqpoint{1.036956in}{0.777029in}}{\pgfqpoint{1.025906in}{0.777029in}}%
\pgfpathcurveto{\pgfqpoint{1.014856in}{0.777029in}}{\pgfqpoint{1.004257in}{0.772639in}}{\pgfqpoint{0.996443in}{0.764825in}}%
\pgfpathcurveto{\pgfqpoint{0.988630in}{0.757011in}}{\pgfqpoint{0.984239in}{0.746412in}}{\pgfqpoint{0.984239in}{0.735362in}}%
\pgfpathcurveto{\pgfqpoint{0.984239in}{0.724312in}}{\pgfqpoint{0.988630in}{0.713713in}}{\pgfqpoint{0.996443in}{0.705900in}}%
\pgfpathcurveto{\pgfqpoint{1.004257in}{0.698086in}}{\pgfqpoint{1.014856in}{0.693696in}}{\pgfqpoint{1.025906in}{0.693696in}}%
\pgfpathclose%
\pgfusepath{stroke,fill}%
\end{pgfscope}%
\begin{pgfscope}%
\pgfpathrectangle{\pgfqpoint{0.800000in}{0.528000in}}{\pgfqpoint{4.960000in}{3.696000in}}%
\pgfusepath{clip}%
\pgfsetbuttcap%
\pgfsetroundjoin%
\definecolor{currentfill}{rgb}{0.000000,0.000000,0.000000}%
\pgfsetfillcolor{currentfill}%
\pgfsetlinewidth{1.003750pt}%
\definecolor{currentstroke}{rgb}{0.000000,0.000000,0.000000}%
\pgfsetstrokecolor{currentstroke}%
\pgfsetdash{}{0pt}%
\pgfpathmoveto{\pgfqpoint{1.025906in}{0.683987in}}%
\pgfpathcurveto{\pgfqpoint{1.036956in}{0.683987in}}{\pgfqpoint{1.047555in}{0.688377in}}{\pgfqpoint{1.055369in}{0.696191in}}%
\pgfpathcurveto{\pgfqpoint{1.063182in}{0.704004in}}{\pgfqpoint{1.067573in}{0.714603in}}{\pgfqpoint{1.067573in}{0.725653in}}%
\pgfpathcurveto{\pgfqpoint{1.067573in}{0.736703in}}{\pgfqpoint{1.063182in}{0.747302in}}{\pgfqpoint{1.055369in}{0.755116in}}%
\pgfpathcurveto{\pgfqpoint{1.047555in}{0.762930in}}{\pgfqpoint{1.036956in}{0.767320in}}{\pgfqpoint{1.025906in}{0.767320in}}%
\pgfpathcurveto{\pgfqpoint{1.014856in}{0.767320in}}{\pgfqpoint{1.004257in}{0.762930in}}{\pgfqpoint{0.996443in}{0.755116in}}%
\pgfpathcurveto{\pgfqpoint{0.988630in}{0.747302in}}{\pgfqpoint{0.984239in}{0.736703in}}{\pgfqpoint{0.984239in}{0.725653in}}%
\pgfpathcurveto{\pgfqpoint{0.984239in}{0.714603in}}{\pgfqpoint{0.988630in}{0.704004in}}{\pgfqpoint{0.996443in}{0.696191in}}%
\pgfpathcurveto{\pgfqpoint{1.004257in}{0.688377in}}{\pgfqpoint{1.014856in}{0.683987in}}{\pgfqpoint{1.025906in}{0.683987in}}%
\pgfpathclose%
\pgfusepath{stroke,fill}%
\end{pgfscope}%
\begin{pgfscope}%
\pgfpathrectangle{\pgfqpoint{0.800000in}{0.528000in}}{\pgfqpoint{4.960000in}{3.696000in}}%
\pgfusepath{clip}%
\pgfsetbuttcap%
\pgfsetroundjoin%
\definecolor{currentfill}{rgb}{0.000000,0.000000,0.000000}%
\pgfsetfillcolor{currentfill}%
\pgfsetlinewidth{1.003750pt}%
\definecolor{currentstroke}{rgb}{0.000000,0.000000,0.000000}%
\pgfsetstrokecolor{currentstroke}%
\pgfsetdash{}{0pt}%
\pgfpathmoveto{\pgfqpoint{1.025906in}{0.693696in}}%
\pgfpathcurveto{\pgfqpoint{1.036956in}{0.693696in}}{\pgfqpoint{1.047555in}{0.698086in}}{\pgfqpoint{1.055369in}{0.705900in}}%
\pgfpathcurveto{\pgfqpoint{1.063182in}{0.713713in}}{\pgfqpoint{1.067573in}{0.724312in}}{\pgfqpoint{1.067573in}{0.735362in}}%
\pgfpathcurveto{\pgfqpoint{1.067573in}{0.746412in}}{\pgfqpoint{1.063182in}{0.757011in}}{\pgfqpoint{1.055369in}{0.764825in}}%
\pgfpathcurveto{\pgfqpoint{1.047555in}{0.772639in}}{\pgfqpoint{1.036956in}{0.777029in}}{\pgfqpoint{1.025906in}{0.777029in}}%
\pgfpathcurveto{\pgfqpoint{1.014856in}{0.777029in}}{\pgfqpoint{1.004257in}{0.772639in}}{\pgfqpoint{0.996443in}{0.764825in}}%
\pgfpathcurveto{\pgfqpoint{0.988630in}{0.757011in}}{\pgfqpoint{0.984239in}{0.746412in}}{\pgfqpoint{0.984239in}{0.735362in}}%
\pgfpathcurveto{\pgfqpoint{0.984239in}{0.724312in}}{\pgfqpoint{0.988630in}{0.713713in}}{\pgfqpoint{0.996443in}{0.705900in}}%
\pgfpathcurveto{\pgfqpoint{1.004257in}{0.698086in}}{\pgfqpoint{1.014856in}{0.693696in}}{\pgfqpoint{1.025906in}{0.693696in}}%
\pgfpathclose%
\pgfusepath{stroke,fill}%
\end{pgfscope}%
\begin{pgfscope}%
\pgfpathrectangle{\pgfqpoint{0.800000in}{0.528000in}}{\pgfqpoint{4.960000in}{3.696000in}}%
\pgfusepath{clip}%
\pgfsetbuttcap%
\pgfsetroundjoin%
\definecolor{currentfill}{rgb}{0.000000,0.000000,0.000000}%
\pgfsetfillcolor{currentfill}%
\pgfsetlinewidth{1.003750pt}%
\definecolor{currentstroke}{rgb}{0.000000,0.000000,0.000000}%
\pgfsetstrokecolor{currentstroke}%
\pgfsetdash{}{0pt}%
\pgfpathmoveto{\pgfqpoint{1.025906in}{0.713114in}}%
\pgfpathcurveto{\pgfqpoint{1.036956in}{0.713114in}}{\pgfqpoint{1.047555in}{0.717504in}}{\pgfqpoint{1.055369in}{0.725317in}}%
\pgfpathcurveto{\pgfqpoint{1.063182in}{0.733131in}}{\pgfqpoint{1.067573in}{0.743730in}}{\pgfqpoint{1.067573in}{0.754780in}}%
\pgfpathcurveto{\pgfqpoint{1.067573in}{0.765830in}}{\pgfqpoint{1.063182in}{0.776429in}}{\pgfqpoint{1.055369in}{0.784243in}}%
\pgfpathcurveto{\pgfqpoint{1.047555in}{0.792057in}}{\pgfqpoint{1.036956in}{0.796447in}}{\pgfqpoint{1.025906in}{0.796447in}}%
\pgfpathcurveto{\pgfqpoint{1.014856in}{0.796447in}}{\pgfqpoint{1.004257in}{0.792057in}}{\pgfqpoint{0.996443in}{0.784243in}}%
\pgfpathcurveto{\pgfqpoint{0.988630in}{0.776429in}}{\pgfqpoint{0.984239in}{0.765830in}}{\pgfqpoint{0.984239in}{0.754780in}}%
\pgfpathcurveto{\pgfqpoint{0.984239in}{0.743730in}}{\pgfqpoint{0.988630in}{0.733131in}}{\pgfqpoint{0.996443in}{0.725317in}}%
\pgfpathcurveto{\pgfqpoint{1.004257in}{0.717504in}}{\pgfqpoint{1.014856in}{0.713114in}}{\pgfqpoint{1.025906in}{0.713114in}}%
\pgfpathclose%
\pgfusepath{stroke,fill}%
\end{pgfscope}%
\begin{pgfscope}%
\pgfpathrectangle{\pgfqpoint{0.800000in}{0.528000in}}{\pgfqpoint{4.960000in}{3.696000in}}%
\pgfusepath{clip}%
\pgfsetbuttcap%
\pgfsetroundjoin%
\definecolor{currentfill}{rgb}{0.000000,0.000000,0.000000}%
\pgfsetfillcolor{currentfill}%
\pgfsetlinewidth{1.003750pt}%
\definecolor{currentstroke}{rgb}{0.000000,0.000000,0.000000}%
\pgfsetstrokecolor{currentstroke}%
\pgfsetdash{}{0pt}%
\pgfpathmoveto{\pgfqpoint{1.025906in}{0.683987in}}%
\pgfpathcurveto{\pgfqpoint{1.036956in}{0.683987in}}{\pgfqpoint{1.047555in}{0.688377in}}{\pgfqpoint{1.055369in}{0.696191in}}%
\pgfpathcurveto{\pgfqpoint{1.063182in}{0.704004in}}{\pgfqpoint{1.067573in}{0.714603in}}{\pgfqpoint{1.067573in}{0.725653in}}%
\pgfpathcurveto{\pgfqpoint{1.067573in}{0.736703in}}{\pgfqpoint{1.063182in}{0.747302in}}{\pgfqpoint{1.055369in}{0.755116in}}%
\pgfpathcurveto{\pgfqpoint{1.047555in}{0.762930in}}{\pgfqpoint{1.036956in}{0.767320in}}{\pgfqpoint{1.025906in}{0.767320in}}%
\pgfpathcurveto{\pgfqpoint{1.014856in}{0.767320in}}{\pgfqpoint{1.004257in}{0.762930in}}{\pgfqpoint{0.996443in}{0.755116in}}%
\pgfpathcurveto{\pgfqpoint{0.988630in}{0.747302in}}{\pgfqpoint{0.984239in}{0.736703in}}{\pgfqpoint{0.984239in}{0.725653in}}%
\pgfpathcurveto{\pgfqpoint{0.984239in}{0.714603in}}{\pgfqpoint{0.988630in}{0.704004in}}{\pgfqpoint{0.996443in}{0.696191in}}%
\pgfpathcurveto{\pgfqpoint{1.004257in}{0.688377in}}{\pgfqpoint{1.014856in}{0.683987in}}{\pgfqpoint{1.025906in}{0.683987in}}%
\pgfpathclose%
\pgfusepath{stroke,fill}%
\end{pgfscope}%
\begin{pgfscope}%
\pgfpathrectangle{\pgfqpoint{0.800000in}{0.528000in}}{\pgfqpoint{4.960000in}{3.696000in}}%
\pgfusepath{clip}%
\pgfsetbuttcap%
\pgfsetroundjoin%
\definecolor{currentfill}{rgb}{0.000000,0.000000,0.000000}%
\pgfsetfillcolor{currentfill}%
\pgfsetlinewidth{1.003750pt}%
\definecolor{currentstroke}{rgb}{0.000000,0.000000,0.000000}%
\pgfsetstrokecolor{currentstroke}%
\pgfsetdash{}{0pt}%
\pgfpathmoveto{\pgfqpoint{1.025906in}{0.664569in}}%
\pgfpathcurveto{\pgfqpoint{1.036956in}{0.664569in}}{\pgfqpoint{1.047555in}{0.668959in}}{\pgfqpoint{1.055369in}{0.676773in}}%
\pgfpathcurveto{\pgfqpoint{1.063182in}{0.684586in}}{\pgfqpoint{1.067573in}{0.695185in}}{\pgfqpoint{1.067573in}{0.706235in}}%
\pgfpathcurveto{\pgfqpoint{1.067573in}{0.717286in}}{\pgfqpoint{1.063182in}{0.727885in}}{\pgfqpoint{1.055369in}{0.735698in}}%
\pgfpathcurveto{\pgfqpoint{1.047555in}{0.743512in}}{\pgfqpoint{1.036956in}{0.747902in}}{\pgfqpoint{1.025906in}{0.747902in}}%
\pgfpathcurveto{\pgfqpoint{1.014856in}{0.747902in}}{\pgfqpoint{1.004257in}{0.743512in}}{\pgfqpoint{0.996443in}{0.735698in}}%
\pgfpathcurveto{\pgfqpoint{0.988630in}{0.727885in}}{\pgfqpoint{0.984239in}{0.717286in}}{\pgfqpoint{0.984239in}{0.706235in}}%
\pgfpathcurveto{\pgfqpoint{0.984239in}{0.695185in}}{\pgfqpoint{0.988630in}{0.684586in}}{\pgfqpoint{0.996443in}{0.676773in}}%
\pgfpathcurveto{\pgfqpoint{1.004257in}{0.668959in}}{\pgfqpoint{1.014856in}{0.664569in}}{\pgfqpoint{1.025906in}{0.664569in}}%
\pgfpathclose%
\pgfusepath{stroke,fill}%
\end{pgfscope}%
\begin{pgfscope}%
\pgfpathrectangle{\pgfqpoint{0.800000in}{0.528000in}}{\pgfqpoint{4.960000in}{3.696000in}}%
\pgfusepath{clip}%
\pgfsetbuttcap%
\pgfsetroundjoin%
\definecolor{currentfill}{rgb}{0.000000,0.000000,0.000000}%
\pgfsetfillcolor{currentfill}%
\pgfsetlinewidth{1.003750pt}%
\definecolor{currentstroke}{rgb}{0.000000,0.000000,0.000000}%
\pgfsetstrokecolor{currentstroke}%
\pgfsetdash{}{0pt}%
\pgfpathmoveto{\pgfqpoint{1.025906in}{0.683987in}}%
\pgfpathcurveto{\pgfqpoint{1.036956in}{0.683987in}}{\pgfqpoint{1.047555in}{0.688377in}}{\pgfqpoint{1.055369in}{0.696191in}}%
\pgfpathcurveto{\pgfqpoint{1.063182in}{0.704004in}}{\pgfqpoint{1.067573in}{0.714603in}}{\pgfqpoint{1.067573in}{0.725653in}}%
\pgfpathcurveto{\pgfqpoint{1.067573in}{0.736703in}}{\pgfqpoint{1.063182in}{0.747302in}}{\pgfqpoint{1.055369in}{0.755116in}}%
\pgfpathcurveto{\pgfqpoint{1.047555in}{0.762930in}}{\pgfqpoint{1.036956in}{0.767320in}}{\pgfqpoint{1.025906in}{0.767320in}}%
\pgfpathcurveto{\pgfqpoint{1.014856in}{0.767320in}}{\pgfqpoint{1.004257in}{0.762930in}}{\pgfqpoint{0.996443in}{0.755116in}}%
\pgfpathcurveto{\pgfqpoint{0.988630in}{0.747302in}}{\pgfqpoint{0.984239in}{0.736703in}}{\pgfqpoint{0.984239in}{0.725653in}}%
\pgfpathcurveto{\pgfqpoint{0.984239in}{0.714603in}}{\pgfqpoint{0.988630in}{0.704004in}}{\pgfqpoint{0.996443in}{0.696191in}}%
\pgfpathcurveto{\pgfqpoint{1.004257in}{0.688377in}}{\pgfqpoint{1.014856in}{0.683987in}}{\pgfqpoint{1.025906in}{0.683987in}}%
\pgfpathclose%
\pgfusepath{stroke,fill}%
\end{pgfscope}%
\begin{pgfscope}%
\pgfpathrectangle{\pgfqpoint{0.800000in}{0.528000in}}{\pgfqpoint{4.960000in}{3.696000in}}%
\pgfusepath{clip}%
\pgfsetbuttcap%
\pgfsetroundjoin%
\definecolor{currentfill}{rgb}{0.000000,0.000000,0.000000}%
\pgfsetfillcolor{currentfill}%
\pgfsetlinewidth{1.003750pt}%
\definecolor{currentstroke}{rgb}{0.000000,0.000000,0.000000}%
\pgfsetstrokecolor{currentstroke}%
\pgfsetdash{}{0pt}%
\pgfpathmoveto{\pgfqpoint{1.025906in}{0.703405in}}%
\pgfpathcurveto{\pgfqpoint{1.036956in}{0.703405in}}{\pgfqpoint{1.047555in}{0.707795in}}{\pgfqpoint{1.055369in}{0.715608in}}%
\pgfpathcurveto{\pgfqpoint{1.063182in}{0.723422in}}{\pgfqpoint{1.067573in}{0.734021in}}{\pgfqpoint{1.067573in}{0.745071in}}%
\pgfpathcurveto{\pgfqpoint{1.067573in}{0.756121in}}{\pgfqpoint{1.063182in}{0.766720in}}{\pgfqpoint{1.055369in}{0.774534in}}%
\pgfpathcurveto{\pgfqpoint{1.047555in}{0.782348in}}{\pgfqpoint{1.036956in}{0.786738in}}{\pgfqpoint{1.025906in}{0.786738in}}%
\pgfpathcurveto{\pgfqpoint{1.014856in}{0.786738in}}{\pgfqpoint{1.004257in}{0.782348in}}{\pgfqpoint{0.996443in}{0.774534in}}%
\pgfpathcurveto{\pgfqpoint{0.988630in}{0.766720in}}{\pgfqpoint{0.984239in}{0.756121in}}{\pgfqpoint{0.984239in}{0.745071in}}%
\pgfpathcurveto{\pgfqpoint{0.984239in}{0.734021in}}{\pgfqpoint{0.988630in}{0.723422in}}{\pgfqpoint{0.996443in}{0.715608in}}%
\pgfpathcurveto{\pgfqpoint{1.004257in}{0.707795in}}{\pgfqpoint{1.014856in}{0.703405in}}{\pgfqpoint{1.025906in}{0.703405in}}%
\pgfpathclose%
\pgfusepath{stroke,fill}%
\end{pgfscope}%
\begin{pgfscope}%
\pgfpathrectangle{\pgfqpoint{0.800000in}{0.528000in}}{\pgfqpoint{4.960000in}{3.696000in}}%
\pgfusepath{clip}%
\pgfsetbuttcap%
\pgfsetroundjoin%
\definecolor{currentfill}{rgb}{0.000000,0.000000,0.000000}%
\pgfsetfillcolor{currentfill}%
\pgfsetlinewidth{1.003750pt}%
\definecolor{currentstroke}{rgb}{0.000000,0.000000,0.000000}%
\pgfsetstrokecolor{currentstroke}%
\pgfsetdash{}{0pt}%
\pgfpathmoveto{\pgfqpoint{1.025906in}{0.693696in}}%
\pgfpathcurveto{\pgfqpoint{1.036956in}{0.693696in}}{\pgfqpoint{1.047555in}{0.698086in}}{\pgfqpoint{1.055369in}{0.705900in}}%
\pgfpathcurveto{\pgfqpoint{1.063182in}{0.713713in}}{\pgfqpoint{1.067573in}{0.724312in}}{\pgfqpoint{1.067573in}{0.735362in}}%
\pgfpathcurveto{\pgfqpoint{1.067573in}{0.746412in}}{\pgfqpoint{1.063182in}{0.757011in}}{\pgfqpoint{1.055369in}{0.764825in}}%
\pgfpathcurveto{\pgfqpoint{1.047555in}{0.772639in}}{\pgfqpoint{1.036956in}{0.777029in}}{\pgfqpoint{1.025906in}{0.777029in}}%
\pgfpathcurveto{\pgfqpoint{1.014856in}{0.777029in}}{\pgfqpoint{1.004257in}{0.772639in}}{\pgfqpoint{0.996443in}{0.764825in}}%
\pgfpathcurveto{\pgfqpoint{0.988630in}{0.757011in}}{\pgfqpoint{0.984239in}{0.746412in}}{\pgfqpoint{0.984239in}{0.735362in}}%
\pgfpathcurveto{\pgfqpoint{0.984239in}{0.724312in}}{\pgfqpoint{0.988630in}{0.713713in}}{\pgfqpoint{0.996443in}{0.705900in}}%
\pgfpathcurveto{\pgfqpoint{1.004257in}{0.698086in}}{\pgfqpoint{1.014856in}{0.693696in}}{\pgfqpoint{1.025906in}{0.693696in}}%
\pgfpathclose%
\pgfusepath{stroke,fill}%
\end{pgfscope}%
\begin{pgfscope}%
\pgfpathrectangle{\pgfqpoint{0.800000in}{0.528000in}}{\pgfqpoint{4.960000in}{3.696000in}}%
\pgfusepath{clip}%
\pgfsetbuttcap%
\pgfsetroundjoin%
\definecolor{currentfill}{rgb}{0.000000,0.000000,0.000000}%
\pgfsetfillcolor{currentfill}%
\pgfsetlinewidth{1.003750pt}%
\definecolor{currentstroke}{rgb}{0.000000,0.000000,0.000000}%
\pgfsetstrokecolor{currentstroke}%
\pgfsetdash{}{0pt}%
\pgfpathmoveto{\pgfqpoint{1.025906in}{0.703405in}}%
\pgfpathcurveto{\pgfqpoint{1.036956in}{0.703405in}}{\pgfqpoint{1.047555in}{0.707795in}}{\pgfqpoint{1.055369in}{0.715608in}}%
\pgfpathcurveto{\pgfqpoint{1.063182in}{0.723422in}}{\pgfqpoint{1.067573in}{0.734021in}}{\pgfqpoint{1.067573in}{0.745071in}}%
\pgfpathcurveto{\pgfqpoint{1.067573in}{0.756121in}}{\pgfqpoint{1.063182in}{0.766720in}}{\pgfqpoint{1.055369in}{0.774534in}}%
\pgfpathcurveto{\pgfqpoint{1.047555in}{0.782348in}}{\pgfqpoint{1.036956in}{0.786738in}}{\pgfqpoint{1.025906in}{0.786738in}}%
\pgfpathcurveto{\pgfqpoint{1.014856in}{0.786738in}}{\pgfqpoint{1.004257in}{0.782348in}}{\pgfqpoint{0.996443in}{0.774534in}}%
\pgfpathcurveto{\pgfqpoint{0.988630in}{0.766720in}}{\pgfqpoint{0.984239in}{0.756121in}}{\pgfqpoint{0.984239in}{0.745071in}}%
\pgfpathcurveto{\pgfqpoint{0.984239in}{0.734021in}}{\pgfqpoint{0.988630in}{0.723422in}}{\pgfqpoint{0.996443in}{0.715608in}}%
\pgfpathcurveto{\pgfqpoint{1.004257in}{0.707795in}}{\pgfqpoint{1.014856in}{0.703405in}}{\pgfqpoint{1.025906in}{0.703405in}}%
\pgfpathclose%
\pgfusepath{stroke,fill}%
\end{pgfscope}%
\begin{pgfscope}%
\pgfpathrectangle{\pgfqpoint{0.800000in}{0.528000in}}{\pgfqpoint{4.960000in}{3.696000in}}%
\pgfusepath{clip}%
\pgfsetbuttcap%
\pgfsetroundjoin%
\definecolor{currentfill}{rgb}{0.000000,0.000000,0.000000}%
\pgfsetfillcolor{currentfill}%
\pgfsetlinewidth{1.003750pt}%
\definecolor{currentstroke}{rgb}{0.000000,0.000000,0.000000}%
\pgfsetstrokecolor{currentstroke}%
\pgfsetdash{}{0pt}%
\pgfpathmoveto{\pgfqpoint{1.025906in}{0.674278in}}%
\pgfpathcurveto{\pgfqpoint{1.036956in}{0.674278in}}{\pgfqpoint{1.047555in}{0.678668in}}{\pgfqpoint{1.055369in}{0.686482in}}%
\pgfpathcurveto{\pgfqpoint{1.063182in}{0.694295in}}{\pgfqpoint{1.067573in}{0.704894in}}{\pgfqpoint{1.067573in}{0.715944in}}%
\pgfpathcurveto{\pgfqpoint{1.067573in}{0.726994in}}{\pgfqpoint{1.063182in}{0.737594in}}{\pgfqpoint{1.055369in}{0.745407in}}%
\pgfpathcurveto{\pgfqpoint{1.047555in}{0.753221in}}{\pgfqpoint{1.036956in}{0.757611in}}{\pgfqpoint{1.025906in}{0.757611in}}%
\pgfpathcurveto{\pgfqpoint{1.014856in}{0.757611in}}{\pgfqpoint{1.004257in}{0.753221in}}{\pgfqpoint{0.996443in}{0.745407in}}%
\pgfpathcurveto{\pgfqpoint{0.988630in}{0.737594in}}{\pgfqpoint{0.984239in}{0.726994in}}{\pgfqpoint{0.984239in}{0.715944in}}%
\pgfpathcurveto{\pgfqpoint{0.984239in}{0.704894in}}{\pgfqpoint{0.988630in}{0.694295in}}{\pgfqpoint{0.996443in}{0.686482in}}%
\pgfpathcurveto{\pgfqpoint{1.004257in}{0.678668in}}{\pgfqpoint{1.014856in}{0.674278in}}{\pgfqpoint{1.025906in}{0.674278in}}%
\pgfpathclose%
\pgfusepath{stroke,fill}%
\end{pgfscope}%
\begin{pgfscope}%
\pgfpathrectangle{\pgfqpoint{0.800000in}{0.528000in}}{\pgfqpoint{4.960000in}{3.696000in}}%
\pgfusepath{clip}%
\pgfsetbuttcap%
\pgfsetroundjoin%
\definecolor{currentfill}{rgb}{0.000000,0.000000,0.000000}%
\pgfsetfillcolor{currentfill}%
\pgfsetlinewidth{1.003750pt}%
\definecolor{currentstroke}{rgb}{0.000000,0.000000,0.000000}%
\pgfsetstrokecolor{currentstroke}%
\pgfsetdash{}{0pt}%
\pgfpathmoveto{\pgfqpoint{1.025906in}{0.693696in}}%
\pgfpathcurveto{\pgfqpoint{1.036956in}{0.693696in}}{\pgfqpoint{1.047555in}{0.698086in}}{\pgfqpoint{1.055369in}{0.705900in}}%
\pgfpathcurveto{\pgfqpoint{1.063182in}{0.713713in}}{\pgfqpoint{1.067573in}{0.724312in}}{\pgfqpoint{1.067573in}{0.735362in}}%
\pgfpathcurveto{\pgfqpoint{1.067573in}{0.746412in}}{\pgfqpoint{1.063182in}{0.757011in}}{\pgfqpoint{1.055369in}{0.764825in}}%
\pgfpathcurveto{\pgfqpoint{1.047555in}{0.772639in}}{\pgfqpoint{1.036956in}{0.777029in}}{\pgfqpoint{1.025906in}{0.777029in}}%
\pgfpathcurveto{\pgfqpoint{1.014856in}{0.777029in}}{\pgfqpoint{1.004257in}{0.772639in}}{\pgfqpoint{0.996443in}{0.764825in}}%
\pgfpathcurveto{\pgfqpoint{0.988630in}{0.757011in}}{\pgfqpoint{0.984239in}{0.746412in}}{\pgfqpoint{0.984239in}{0.735362in}}%
\pgfpathcurveto{\pgfqpoint{0.984239in}{0.724312in}}{\pgfqpoint{0.988630in}{0.713713in}}{\pgfqpoint{0.996443in}{0.705900in}}%
\pgfpathcurveto{\pgfqpoint{1.004257in}{0.698086in}}{\pgfqpoint{1.014856in}{0.693696in}}{\pgfqpoint{1.025906in}{0.693696in}}%
\pgfpathclose%
\pgfusepath{stroke,fill}%
\end{pgfscope}%
\begin{pgfscope}%
\pgfpathrectangle{\pgfqpoint{0.800000in}{0.528000in}}{\pgfqpoint{4.960000in}{3.696000in}}%
\pgfusepath{clip}%
\pgfsetbuttcap%
\pgfsetroundjoin%
\definecolor{currentfill}{rgb}{0.000000,0.000000,0.000000}%
\pgfsetfillcolor{currentfill}%
\pgfsetlinewidth{1.003750pt}%
\definecolor{currentstroke}{rgb}{0.000000,0.000000,0.000000}%
\pgfsetstrokecolor{currentstroke}%
\pgfsetdash{}{0pt}%
\pgfpathmoveto{\pgfqpoint{1.025906in}{0.693696in}}%
\pgfpathcurveto{\pgfqpoint{1.036956in}{0.693696in}}{\pgfqpoint{1.047555in}{0.698086in}}{\pgfqpoint{1.055369in}{0.705900in}}%
\pgfpathcurveto{\pgfqpoint{1.063182in}{0.713713in}}{\pgfqpoint{1.067573in}{0.724312in}}{\pgfqpoint{1.067573in}{0.735362in}}%
\pgfpathcurveto{\pgfqpoint{1.067573in}{0.746412in}}{\pgfqpoint{1.063182in}{0.757011in}}{\pgfqpoint{1.055369in}{0.764825in}}%
\pgfpathcurveto{\pgfqpoint{1.047555in}{0.772639in}}{\pgfqpoint{1.036956in}{0.777029in}}{\pgfqpoint{1.025906in}{0.777029in}}%
\pgfpathcurveto{\pgfqpoint{1.014856in}{0.777029in}}{\pgfqpoint{1.004257in}{0.772639in}}{\pgfqpoint{0.996443in}{0.764825in}}%
\pgfpathcurveto{\pgfqpoint{0.988630in}{0.757011in}}{\pgfqpoint{0.984239in}{0.746412in}}{\pgfqpoint{0.984239in}{0.735362in}}%
\pgfpathcurveto{\pgfqpoint{0.984239in}{0.724312in}}{\pgfqpoint{0.988630in}{0.713713in}}{\pgfqpoint{0.996443in}{0.705900in}}%
\pgfpathcurveto{\pgfqpoint{1.004257in}{0.698086in}}{\pgfqpoint{1.014856in}{0.693696in}}{\pgfqpoint{1.025906in}{0.693696in}}%
\pgfpathclose%
\pgfusepath{stroke,fill}%
\end{pgfscope}%
\begin{pgfscope}%
\pgfpathrectangle{\pgfqpoint{0.800000in}{0.528000in}}{\pgfqpoint{4.960000in}{3.696000in}}%
\pgfusepath{clip}%
\pgfsetbuttcap%
\pgfsetroundjoin%
\definecolor{currentfill}{rgb}{0.000000,0.000000,0.000000}%
\pgfsetfillcolor{currentfill}%
\pgfsetlinewidth{1.003750pt}%
\definecolor{currentstroke}{rgb}{0.000000,0.000000,0.000000}%
\pgfsetstrokecolor{currentstroke}%
\pgfsetdash{}{0pt}%
\pgfpathmoveto{\pgfqpoint{1.025906in}{0.683987in}}%
\pgfpathcurveto{\pgfqpoint{1.036956in}{0.683987in}}{\pgfqpoint{1.047555in}{0.688377in}}{\pgfqpoint{1.055369in}{0.696191in}}%
\pgfpathcurveto{\pgfqpoint{1.063182in}{0.704004in}}{\pgfqpoint{1.067573in}{0.714603in}}{\pgfqpoint{1.067573in}{0.725653in}}%
\pgfpathcurveto{\pgfqpoint{1.067573in}{0.736703in}}{\pgfqpoint{1.063182in}{0.747302in}}{\pgfqpoint{1.055369in}{0.755116in}}%
\pgfpathcurveto{\pgfqpoint{1.047555in}{0.762930in}}{\pgfqpoint{1.036956in}{0.767320in}}{\pgfqpoint{1.025906in}{0.767320in}}%
\pgfpathcurveto{\pgfqpoint{1.014856in}{0.767320in}}{\pgfqpoint{1.004257in}{0.762930in}}{\pgfqpoint{0.996443in}{0.755116in}}%
\pgfpathcurveto{\pgfqpoint{0.988630in}{0.747302in}}{\pgfqpoint{0.984239in}{0.736703in}}{\pgfqpoint{0.984239in}{0.725653in}}%
\pgfpathcurveto{\pgfqpoint{0.984239in}{0.714603in}}{\pgfqpoint{0.988630in}{0.704004in}}{\pgfqpoint{0.996443in}{0.696191in}}%
\pgfpathcurveto{\pgfqpoint{1.004257in}{0.688377in}}{\pgfqpoint{1.014856in}{0.683987in}}{\pgfqpoint{1.025906in}{0.683987in}}%
\pgfpathclose%
\pgfusepath{stroke,fill}%
\end{pgfscope}%
\begin{pgfscope}%
\pgfpathrectangle{\pgfqpoint{0.800000in}{0.528000in}}{\pgfqpoint{4.960000in}{3.696000in}}%
\pgfusepath{clip}%
\pgfsetbuttcap%
\pgfsetroundjoin%
\definecolor{currentfill}{rgb}{0.000000,0.000000,0.000000}%
\pgfsetfillcolor{currentfill}%
\pgfsetlinewidth{1.003750pt}%
\definecolor{currentstroke}{rgb}{0.000000,0.000000,0.000000}%
\pgfsetstrokecolor{currentstroke}%
\pgfsetdash{}{0pt}%
\pgfpathmoveto{\pgfqpoint{1.025906in}{0.683987in}}%
\pgfpathcurveto{\pgfqpoint{1.036956in}{0.683987in}}{\pgfqpoint{1.047555in}{0.688377in}}{\pgfqpoint{1.055369in}{0.696191in}}%
\pgfpathcurveto{\pgfqpoint{1.063182in}{0.704004in}}{\pgfqpoint{1.067573in}{0.714603in}}{\pgfqpoint{1.067573in}{0.725653in}}%
\pgfpathcurveto{\pgfqpoint{1.067573in}{0.736703in}}{\pgfqpoint{1.063182in}{0.747302in}}{\pgfqpoint{1.055369in}{0.755116in}}%
\pgfpathcurveto{\pgfqpoint{1.047555in}{0.762930in}}{\pgfqpoint{1.036956in}{0.767320in}}{\pgfqpoint{1.025906in}{0.767320in}}%
\pgfpathcurveto{\pgfqpoint{1.014856in}{0.767320in}}{\pgfqpoint{1.004257in}{0.762930in}}{\pgfqpoint{0.996443in}{0.755116in}}%
\pgfpathcurveto{\pgfqpoint{0.988630in}{0.747302in}}{\pgfqpoint{0.984239in}{0.736703in}}{\pgfqpoint{0.984239in}{0.725653in}}%
\pgfpathcurveto{\pgfqpoint{0.984239in}{0.714603in}}{\pgfqpoint{0.988630in}{0.704004in}}{\pgfqpoint{0.996443in}{0.696191in}}%
\pgfpathcurveto{\pgfqpoint{1.004257in}{0.688377in}}{\pgfqpoint{1.014856in}{0.683987in}}{\pgfqpoint{1.025906in}{0.683987in}}%
\pgfpathclose%
\pgfusepath{stroke,fill}%
\end{pgfscope}%
\begin{pgfscope}%
\pgfpathrectangle{\pgfqpoint{0.800000in}{0.528000in}}{\pgfqpoint{4.960000in}{3.696000in}}%
\pgfusepath{clip}%
\pgfsetbuttcap%
\pgfsetroundjoin%
\definecolor{currentfill}{rgb}{0.000000,0.000000,0.000000}%
\pgfsetfillcolor{currentfill}%
\pgfsetlinewidth{1.003750pt}%
\definecolor{currentstroke}{rgb}{0.000000,0.000000,0.000000}%
\pgfsetstrokecolor{currentstroke}%
\pgfsetdash{}{0pt}%
\pgfpathmoveto{\pgfqpoint{1.025906in}{0.713114in}}%
\pgfpathcurveto{\pgfqpoint{1.036956in}{0.713114in}}{\pgfqpoint{1.047555in}{0.717504in}}{\pgfqpoint{1.055369in}{0.725317in}}%
\pgfpathcurveto{\pgfqpoint{1.063182in}{0.733131in}}{\pgfqpoint{1.067573in}{0.743730in}}{\pgfqpoint{1.067573in}{0.754780in}}%
\pgfpathcurveto{\pgfqpoint{1.067573in}{0.765830in}}{\pgfqpoint{1.063182in}{0.776429in}}{\pgfqpoint{1.055369in}{0.784243in}}%
\pgfpathcurveto{\pgfqpoint{1.047555in}{0.792057in}}{\pgfqpoint{1.036956in}{0.796447in}}{\pgfqpoint{1.025906in}{0.796447in}}%
\pgfpathcurveto{\pgfqpoint{1.014856in}{0.796447in}}{\pgfqpoint{1.004257in}{0.792057in}}{\pgfqpoint{0.996443in}{0.784243in}}%
\pgfpathcurveto{\pgfqpoint{0.988630in}{0.776429in}}{\pgfqpoint{0.984239in}{0.765830in}}{\pgfqpoint{0.984239in}{0.754780in}}%
\pgfpathcurveto{\pgfqpoint{0.984239in}{0.743730in}}{\pgfqpoint{0.988630in}{0.733131in}}{\pgfqpoint{0.996443in}{0.725317in}}%
\pgfpathcurveto{\pgfqpoint{1.004257in}{0.717504in}}{\pgfqpoint{1.014856in}{0.713114in}}{\pgfqpoint{1.025906in}{0.713114in}}%
\pgfpathclose%
\pgfusepath{stroke,fill}%
\end{pgfscope}%
\begin{pgfscope}%
\pgfpathrectangle{\pgfqpoint{0.800000in}{0.528000in}}{\pgfqpoint{4.960000in}{3.696000in}}%
\pgfusepath{clip}%
\pgfsetbuttcap%
\pgfsetroundjoin%
\definecolor{currentfill}{rgb}{0.000000,0.000000,0.000000}%
\pgfsetfillcolor{currentfill}%
\pgfsetlinewidth{1.003750pt}%
\definecolor{currentstroke}{rgb}{0.000000,0.000000,0.000000}%
\pgfsetstrokecolor{currentstroke}%
\pgfsetdash{}{0pt}%
\pgfpathmoveto{\pgfqpoint{1.025906in}{0.674278in}}%
\pgfpathcurveto{\pgfqpoint{1.036956in}{0.674278in}}{\pgfqpoint{1.047555in}{0.678668in}}{\pgfqpoint{1.055369in}{0.686482in}}%
\pgfpathcurveto{\pgfqpoint{1.063182in}{0.694295in}}{\pgfqpoint{1.067573in}{0.704894in}}{\pgfqpoint{1.067573in}{0.715944in}}%
\pgfpathcurveto{\pgfqpoint{1.067573in}{0.726994in}}{\pgfqpoint{1.063182in}{0.737594in}}{\pgfqpoint{1.055369in}{0.745407in}}%
\pgfpathcurveto{\pgfqpoint{1.047555in}{0.753221in}}{\pgfqpoint{1.036956in}{0.757611in}}{\pgfqpoint{1.025906in}{0.757611in}}%
\pgfpathcurveto{\pgfqpoint{1.014856in}{0.757611in}}{\pgfqpoint{1.004257in}{0.753221in}}{\pgfqpoint{0.996443in}{0.745407in}}%
\pgfpathcurveto{\pgfqpoint{0.988630in}{0.737594in}}{\pgfqpoint{0.984239in}{0.726994in}}{\pgfqpoint{0.984239in}{0.715944in}}%
\pgfpathcurveto{\pgfqpoint{0.984239in}{0.704894in}}{\pgfqpoint{0.988630in}{0.694295in}}{\pgfqpoint{0.996443in}{0.686482in}}%
\pgfpathcurveto{\pgfqpoint{1.004257in}{0.678668in}}{\pgfqpoint{1.014856in}{0.674278in}}{\pgfqpoint{1.025906in}{0.674278in}}%
\pgfpathclose%
\pgfusepath{stroke,fill}%
\end{pgfscope}%
\begin{pgfscope}%
\pgfpathrectangle{\pgfqpoint{0.800000in}{0.528000in}}{\pgfqpoint{4.960000in}{3.696000in}}%
\pgfusepath{clip}%
\pgfsetbuttcap%
\pgfsetroundjoin%
\definecolor{currentfill}{rgb}{0.000000,0.000000,0.000000}%
\pgfsetfillcolor{currentfill}%
\pgfsetlinewidth{1.003750pt}%
\definecolor{currentstroke}{rgb}{0.000000,0.000000,0.000000}%
\pgfsetstrokecolor{currentstroke}%
\pgfsetdash{}{0pt}%
\pgfpathmoveto{\pgfqpoint{1.025906in}{0.674278in}}%
\pgfpathcurveto{\pgfqpoint{1.036956in}{0.674278in}}{\pgfqpoint{1.047555in}{0.678668in}}{\pgfqpoint{1.055369in}{0.686482in}}%
\pgfpathcurveto{\pgfqpoint{1.063182in}{0.694295in}}{\pgfqpoint{1.067573in}{0.704894in}}{\pgfqpoint{1.067573in}{0.715944in}}%
\pgfpathcurveto{\pgfqpoint{1.067573in}{0.726994in}}{\pgfqpoint{1.063182in}{0.737594in}}{\pgfqpoint{1.055369in}{0.745407in}}%
\pgfpathcurveto{\pgfqpoint{1.047555in}{0.753221in}}{\pgfqpoint{1.036956in}{0.757611in}}{\pgfqpoint{1.025906in}{0.757611in}}%
\pgfpathcurveto{\pgfqpoint{1.014856in}{0.757611in}}{\pgfqpoint{1.004257in}{0.753221in}}{\pgfqpoint{0.996443in}{0.745407in}}%
\pgfpathcurveto{\pgfqpoint{0.988630in}{0.737594in}}{\pgfqpoint{0.984239in}{0.726994in}}{\pgfqpoint{0.984239in}{0.715944in}}%
\pgfpathcurveto{\pgfqpoint{0.984239in}{0.704894in}}{\pgfqpoint{0.988630in}{0.694295in}}{\pgfqpoint{0.996443in}{0.686482in}}%
\pgfpathcurveto{\pgfqpoint{1.004257in}{0.678668in}}{\pgfqpoint{1.014856in}{0.674278in}}{\pgfqpoint{1.025906in}{0.674278in}}%
\pgfpathclose%
\pgfusepath{stroke,fill}%
\end{pgfscope}%
\begin{pgfscope}%
\pgfpathrectangle{\pgfqpoint{0.800000in}{0.528000in}}{\pgfqpoint{4.960000in}{3.696000in}}%
\pgfusepath{clip}%
\pgfsetbuttcap%
\pgfsetroundjoin%
\definecolor{currentfill}{rgb}{0.000000,0.000000,0.000000}%
\pgfsetfillcolor{currentfill}%
\pgfsetlinewidth{1.003750pt}%
\definecolor{currentstroke}{rgb}{0.000000,0.000000,0.000000}%
\pgfsetstrokecolor{currentstroke}%
\pgfsetdash{}{0pt}%
\pgfpathmoveto{\pgfqpoint{1.025906in}{0.683987in}}%
\pgfpathcurveto{\pgfqpoint{1.036956in}{0.683987in}}{\pgfqpoint{1.047555in}{0.688377in}}{\pgfqpoint{1.055369in}{0.696191in}}%
\pgfpathcurveto{\pgfqpoint{1.063182in}{0.704004in}}{\pgfqpoint{1.067573in}{0.714603in}}{\pgfqpoint{1.067573in}{0.725653in}}%
\pgfpathcurveto{\pgfqpoint{1.067573in}{0.736703in}}{\pgfqpoint{1.063182in}{0.747302in}}{\pgfqpoint{1.055369in}{0.755116in}}%
\pgfpathcurveto{\pgfqpoint{1.047555in}{0.762930in}}{\pgfqpoint{1.036956in}{0.767320in}}{\pgfqpoint{1.025906in}{0.767320in}}%
\pgfpathcurveto{\pgfqpoint{1.014856in}{0.767320in}}{\pgfqpoint{1.004257in}{0.762930in}}{\pgfqpoint{0.996443in}{0.755116in}}%
\pgfpathcurveto{\pgfqpoint{0.988630in}{0.747302in}}{\pgfqpoint{0.984239in}{0.736703in}}{\pgfqpoint{0.984239in}{0.725653in}}%
\pgfpathcurveto{\pgfqpoint{0.984239in}{0.714603in}}{\pgfqpoint{0.988630in}{0.704004in}}{\pgfqpoint{0.996443in}{0.696191in}}%
\pgfpathcurveto{\pgfqpoint{1.004257in}{0.688377in}}{\pgfqpoint{1.014856in}{0.683987in}}{\pgfqpoint{1.025906in}{0.683987in}}%
\pgfpathclose%
\pgfusepath{stroke,fill}%
\end{pgfscope}%
\begin{pgfscope}%
\pgfpathrectangle{\pgfqpoint{0.800000in}{0.528000in}}{\pgfqpoint{4.960000in}{3.696000in}}%
\pgfusepath{clip}%
\pgfsetbuttcap%
\pgfsetroundjoin%
\definecolor{currentfill}{rgb}{0.000000,0.000000,0.000000}%
\pgfsetfillcolor{currentfill}%
\pgfsetlinewidth{1.003750pt}%
\definecolor{currentstroke}{rgb}{0.000000,0.000000,0.000000}%
\pgfsetstrokecolor{currentstroke}%
\pgfsetdash{}{0pt}%
\pgfpathmoveto{\pgfqpoint{1.025906in}{0.693696in}}%
\pgfpathcurveto{\pgfqpoint{1.036956in}{0.693696in}}{\pgfqpoint{1.047555in}{0.698086in}}{\pgfqpoint{1.055369in}{0.705900in}}%
\pgfpathcurveto{\pgfqpoint{1.063182in}{0.713713in}}{\pgfqpoint{1.067573in}{0.724312in}}{\pgfqpoint{1.067573in}{0.735362in}}%
\pgfpathcurveto{\pgfqpoint{1.067573in}{0.746412in}}{\pgfqpoint{1.063182in}{0.757011in}}{\pgfqpoint{1.055369in}{0.764825in}}%
\pgfpathcurveto{\pgfqpoint{1.047555in}{0.772639in}}{\pgfqpoint{1.036956in}{0.777029in}}{\pgfqpoint{1.025906in}{0.777029in}}%
\pgfpathcurveto{\pgfqpoint{1.014856in}{0.777029in}}{\pgfqpoint{1.004257in}{0.772639in}}{\pgfqpoint{0.996443in}{0.764825in}}%
\pgfpathcurveto{\pgfqpoint{0.988630in}{0.757011in}}{\pgfqpoint{0.984239in}{0.746412in}}{\pgfqpoint{0.984239in}{0.735362in}}%
\pgfpathcurveto{\pgfqpoint{0.984239in}{0.724312in}}{\pgfqpoint{0.988630in}{0.713713in}}{\pgfqpoint{0.996443in}{0.705900in}}%
\pgfpathcurveto{\pgfqpoint{1.004257in}{0.698086in}}{\pgfqpoint{1.014856in}{0.693696in}}{\pgfqpoint{1.025906in}{0.693696in}}%
\pgfpathclose%
\pgfusepath{stroke,fill}%
\end{pgfscope}%
\begin{pgfscope}%
\pgfpathrectangle{\pgfqpoint{0.800000in}{0.528000in}}{\pgfqpoint{4.960000in}{3.696000in}}%
\pgfusepath{clip}%
\pgfsetbuttcap%
\pgfsetroundjoin%
\definecolor{currentfill}{rgb}{0.000000,0.000000,0.000000}%
\pgfsetfillcolor{currentfill}%
\pgfsetlinewidth{1.003750pt}%
\definecolor{currentstroke}{rgb}{0.000000,0.000000,0.000000}%
\pgfsetstrokecolor{currentstroke}%
\pgfsetdash{}{0pt}%
\pgfpathmoveto{\pgfqpoint{1.025906in}{0.674278in}}%
\pgfpathcurveto{\pgfqpoint{1.036956in}{0.674278in}}{\pgfqpoint{1.047555in}{0.678668in}}{\pgfqpoint{1.055369in}{0.686482in}}%
\pgfpathcurveto{\pgfqpoint{1.063182in}{0.694295in}}{\pgfqpoint{1.067573in}{0.704894in}}{\pgfqpoint{1.067573in}{0.715944in}}%
\pgfpathcurveto{\pgfqpoint{1.067573in}{0.726994in}}{\pgfqpoint{1.063182in}{0.737594in}}{\pgfqpoint{1.055369in}{0.745407in}}%
\pgfpathcurveto{\pgfqpoint{1.047555in}{0.753221in}}{\pgfqpoint{1.036956in}{0.757611in}}{\pgfqpoint{1.025906in}{0.757611in}}%
\pgfpathcurveto{\pgfqpoint{1.014856in}{0.757611in}}{\pgfqpoint{1.004257in}{0.753221in}}{\pgfqpoint{0.996443in}{0.745407in}}%
\pgfpathcurveto{\pgfqpoint{0.988630in}{0.737594in}}{\pgfqpoint{0.984239in}{0.726994in}}{\pgfqpoint{0.984239in}{0.715944in}}%
\pgfpathcurveto{\pgfqpoint{0.984239in}{0.704894in}}{\pgfqpoint{0.988630in}{0.694295in}}{\pgfqpoint{0.996443in}{0.686482in}}%
\pgfpathcurveto{\pgfqpoint{1.004257in}{0.678668in}}{\pgfqpoint{1.014856in}{0.674278in}}{\pgfqpoint{1.025906in}{0.674278in}}%
\pgfpathclose%
\pgfusepath{stroke,fill}%
\end{pgfscope}%
\begin{pgfscope}%
\pgfpathrectangle{\pgfqpoint{0.800000in}{0.528000in}}{\pgfqpoint{4.960000in}{3.696000in}}%
\pgfusepath{clip}%
\pgfsetbuttcap%
\pgfsetroundjoin%
\definecolor{currentfill}{rgb}{0.000000,0.000000,0.000000}%
\pgfsetfillcolor{currentfill}%
\pgfsetlinewidth{1.003750pt}%
\definecolor{currentstroke}{rgb}{0.000000,0.000000,0.000000}%
\pgfsetstrokecolor{currentstroke}%
\pgfsetdash{}{0pt}%
\pgfpathmoveto{\pgfqpoint{2.518786in}{1.033510in}}%
\pgfpathcurveto{\pgfqpoint{2.529836in}{1.033510in}}{\pgfqpoint{2.540435in}{1.037900in}}{\pgfqpoint{2.548249in}{1.045713in}}%
\pgfpathcurveto{\pgfqpoint{2.556062in}{1.053527in}}{\pgfqpoint{2.560452in}{1.064126in}}{\pgfqpoint{2.560452in}{1.075176in}}%
\pgfpathcurveto{\pgfqpoint{2.560452in}{1.086226in}}{\pgfqpoint{2.556062in}{1.096825in}}{\pgfqpoint{2.548249in}{1.104639in}}%
\pgfpathcurveto{\pgfqpoint{2.540435in}{1.112453in}}{\pgfqpoint{2.529836in}{1.116843in}}{\pgfqpoint{2.518786in}{1.116843in}}%
\pgfpathcurveto{\pgfqpoint{2.507736in}{1.116843in}}{\pgfqpoint{2.497137in}{1.112453in}}{\pgfqpoint{2.489323in}{1.104639in}}%
\pgfpathcurveto{\pgfqpoint{2.481509in}{1.096825in}}{\pgfqpoint{2.477119in}{1.086226in}}{\pgfqpoint{2.477119in}{1.075176in}}%
\pgfpathcurveto{\pgfqpoint{2.477119in}{1.064126in}}{\pgfqpoint{2.481509in}{1.053527in}}{\pgfqpoint{2.489323in}{1.045713in}}%
\pgfpathcurveto{\pgfqpoint{2.497137in}{1.037900in}}{\pgfqpoint{2.507736in}{1.033510in}}{\pgfqpoint{2.518786in}{1.033510in}}%
\pgfpathclose%
\pgfusepath{stroke,fill}%
\end{pgfscope}%
\begin{pgfscope}%
\pgfpathrectangle{\pgfqpoint{0.800000in}{0.528000in}}{\pgfqpoint{4.960000in}{3.696000in}}%
\pgfusepath{clip}%
\pgfsetbuttcap%
\pgfsetroundjoin%
\definecolor{currentfill}{rgb}{0.000000,0.000000,0.000000}%
\pgfsetfillcolor{currentfill}%
\pgfsetlinewidth{1.003750pt}%
\definecolor{currentstroke}{rgb}{0.000000,0.000000,0.000000}%
\pgfsetstrokecolor{currentstroke}%
\pgfsetdash{}{0pt}%
\pgfpathmoveto{\pgfqpoint{2.518786in}{0.984965in}}%
\pgfpathcurveto{\pgfqpoint{2.529836in}{0.984965in}}{\pgfqpoint{2.540435in}{0.989355in}}{\pgfqpoint{2.548249in}{0.997169in}}%
\pgfpathcurveto{\pgfqpoint{2.556062in}{1.004982in}}{\pgfqpoint{2.560452in}{1.015581in}}{\pgfqpoint{2.560452in}{1.026631in}}%
\pgfpathcurveto{\pgfqpoint{2.560452in}{1.037681in}}{\pgfqpoint{2.556062in}{1.048281in}}{\pgfqpoint{2.548249in}{1.056094in}}%
\pgfpathcurveto{\pgfqpoint{2.540435in}{1.063908in}}{\pgfqpoint{2.529836in}{1.068298in}}{\pgfqpoint{2.518786in}{1.068298in}}%
\pgfpathcurveto{\pgfqpoint{2.507736in}{1.068298in}}{\pgfqpoint{2.497137in}{1.063908in}}{\pgfqpoint{2.489323in}{1.056094in}}%
\pgfpathcurveto{\pgfqpoint{2.481509in}{1.048281in}}{\pgfqpoint{2.477119in}{1.037681in}}{\pgfqpoint{2.477119in}{1.026631in}}%
\pgfpathcurveto{\pgfqpoint{2.477119in}{1.015581in}}{\pgfqpoint{2.481509in}{1.004982in}}{\pgfqpoint{2.489323in}{0.997169in}}%
\pgfpathcurveto{\pgfqpoint{2.497137in}{0.989355in}}{\pgfqpoint{2.507736in}{0.984965in}}{\pgfqpoint{2.518786in}{0.984965in}}%
\pgfpathclose%
\pgfusepath{stroke,fill}%
\end{pgfscope}%
\begin{pgfscope}%
\pgfpathrectangle{\pgfqpoint{0.800000in}{0.528000in}}{\pgfqpoint{4.960000in}{3.696000in}}%
\pgfusepath{clip}%
\pgfsetbuttcap%
\pgfsetroundjoin%
\definecolor{currentfill}{rgb}{0.000000,0.000000,0.000000}%
\pgfsetfillcolor{currentfill}%
\pgfsetlinewidth{1.003750pt}%
\definecolor{currentstroke}{rgb}{0.000000,0.000000,0.000000}%
\pgfsetstrokecolor{currentstroke}%
\pgfsetdash{}{0pt}%
\pgfpathmoveto{\pgfqpoint{2.518786in}{0.936420in}}%
\pgfpathcurveto{\pgfqpoint{2.529836in}{0.936420in}}{\pgfqpoint{2.540435in}{0.940810in}}{\pgfqpoint{2.548249in}{0.948624in}}%
\pgfpathcurveto{\pgfqpoint{2.556062in}{0.956437in}}{\pgfqpoint{2.560452in}{0.967036in}}{\pgfqpoint{2.560452in}{0.978087in}}%
\pgfpathcurveto{\pgfqpoint{2.560452in}{0.989137in}}{\pgfqpoint{2.556062in}{0.999736in}}{\pgfqpoint{2.548249in}{1.007549in}}%
\pgfpathcurveto{\pgfqpoint{2.540435in}{1.015363in}}{\pgfqpoint{2.529836in}{1.019753in}}{\pgfqpoint{2.518786in}{1.019753in}}%
\pgfpathcurveto{\pgfqpoint{2.507736in}{1.019753in}}{\pgfqpoint{2.497137in}{1.015363in}}{\pgfqpoint{2.489323in}{1.007549in}}%
\pgfpathcurveto{\pgfqpoint{2.481509in}{0.999736in}}{\pgfqpoint{2.477119in}{0.989137in}}{\pgfqpoint{2.477119in}{0.978087in}}%
\pgfpathcurveto{\pgfqpoint{2.477119in}{0.967036in}}{\pgfqpoint{2.481509in}{0.956437in}}{\pgfqpoint{2.489323in}{0.948624in}}%
\pgfpathcurveto{\pgfqpoint{2.497137in}{0.940810in}}{\pgfqpoint{2.507736in}{0.936420in}}{\pgfqpoint{2.518786in}{0.936420in}}%
\pgfpathclose%
\pgfusepath{stroke,fill}%
\end{pgfscope}%
\begin{pgfscope}%
\pgfpathrectangle{\pgfqpoint{0.800000in}{0.528000in}}{\pgfqpoint{4.960000in}{3.696000in}}%
\pgfusepath{clip}%
\pgfsetbuttcap%
\pgfsetroundjoin%
\definecolor{currentfill}{rgb}{0.000000,0.000000,0.000000}%
\pgfsetfillcolor{currentfill}%
\pgfsetlinewidth{1.003750pt}%
\definecolor{currentstroke}{rgb}{0.000000,0.000000,0.000000}%
\pgfsetstrokecolor{currentstroke}%
\pgfsetdash{}{0pt}%
\pgfpathmoveto{\pgfqpoint{2.518786in}{1.043219in}}%
\pgfpathcurveto{\pgfqpoint{2.529836in}{1.043219in}}{\pgfqpoint{2.540435in}{1.047609in}}{\pgfqpoint{2.548249in}{1.055422in}}%
\pgfpathcurveto{\pgfqpoint{2.556062in}{1.063236in}}{\pgfqpoint{2.560452in}{1.073835in}}{\pgfqpoint{2.560452in}{1.084885in}}%
\pgfpathcurveto{\pgfqpoint{2.560452in}{1.095935in}}{\pgfqpoint{2.556062in}{1.106534in}}{\pgfqpoint{2.548249in}{1.114348in}}%
\pgfpathcurveto{\pgfqpoint{2.540435in}{1.122162in}}{\pgfqpoint{2.529836in}{1.126552in}}{\pgfqpoint{2.518786in}{1.126552in}}%
\pgfpathcurveto{\pgfqpoint{2.507736in}{1.126552in}}{\pgfqpoint{2.497137in}{1.122162in}}{\pgfqpoint{2.489323in}{1.114348in}}%
\pgfpathcurveto{\pgfqpoint{2.481509in}{1.106534in}}{\pgfqpoint{2.477119in}{1.095935in}}{\pgfqpoint{2.477119in}{1.084885in}}%
\pgfpathcurveto{\pgfqpoint{2.477119in}{1.073835in}}{\pgfqpoint{2.481509in}{1.063236in}}{\pgfqpoint{2.489323in}{1.055422in}}%
\pgfpathcurveto{\pgfqpoint{2.497137in}{1.047609in}}{\pgfqpoint{2.507736in}{1.043219in}}{\pgfqpoint{2.518786in}{1.043219in}}%
\pgfpathclose%
\pgfusepath{stroke,fill}%
\end{pgfscope}%
\begin{pgfscope}%
\pgfpathrectangle{\pgfqpoint{0.800000in}{0.528000in}}{\pgfqpoint{4.960000in}{3.696000in}}%
\pgfusepath{clip}%
\pgfsetbuttcap%
\pgfsetroundjoin%
\definecolor{currentfill}{rgb}{0.000000,0.000000,0.000000}%
\pgfsetfillcolor{currentfill}%
\pgfsetlinewidth{1.003750pt}%
\definecolor{currentstroke}{rgb}{0.000000,0.000000,0.000000}%
\pgfsetstrokecolor{currentstroke}%
\pgfsetdash{}{0pt}%
\pgfpathmoveto{\pgfqpoint{2.518786in}{1.072345in}}%
\pgfpathcurveto{\pgfqpoint{2.529836in}{1.072345in}}{\pgfqpoint{2.540435in}{1.076736in}}{\pgfqpoint{2.548249in}{1.084549in}}%
\pgfpathcurveto{\pgfqpoint{2.556062in}{1.092363in}}{\pgfqpoint{2.560452in}{1.102962in}}{\pgfqpoint{2.560452in}{1.114012in}}%
\pgfpathcurveto{\pgfqpoint{2.560452in}{1.125062in}}{\pgfqpoint{2.556062in}{1.135661in}}{\pgfqpoint{2.548249in}{1.143475in}}%
\pgfpathcurveto{\pgfqpoint{2.540435in}{1.151288in}}{\pgfqpoint{2.529836in}{1.155679in}}{\pgfqpoint{2.518786in}{1.155679in}}%
\pgfpathcurveto{\pgfqpoint{2.507736in}{1.155679in}}{\pgfqpoint{2.497137in}{1.151288in}}{\pgfqpoint{2.489323in}{1.143475in}}%
\pgfpathcurveto{\pgfqpoint{2.481509in}{1.135661in}}{\pgfqpoint{2.477119in}{1.125062in}}{\pgfqpoint{2.477119in}{1.114012in}}%
\pgfpathcurveto{\pgfqpoint{2.477119in}{1.102962in}}{\pgfqpoint{2.481509in}{1.092363in}}{\pgfqpoint{2.489323in}{1.084549in}}%
\pgfpathcurveto{\pgfqpoint{2.497137in}{1.076736in}}{\pgfqpoint{2.507736in}{1.072345in}}{\pgfqpoint{2.518786in}{1.072345in}}%
\pgfpathclose%
\pgfusepath{stroke,fill}%
\end{pgfscope}%
\begin{pgfscope}%
\pgfpathrectangle{\pgfqpoint{0.800000in}{0.528000in}}{\pgfqpoint{4.960000in}{3.696000in}}%
\pgfusepath{clip}%
\pgfsetbuttcap%
\pgfsetroundjoin%
\definecolor{currentfill}{rgb}{0.000000,0.000000,0.000000}%
\pgfsetfillcolor{currentfill}%
\pgfsetlinewidth{1.003750pt}%
\definecolor{currentstroke}{rgb}{0.000000,0.000000,0.000000}%
\pgfsetstrokecolor{currentstroke}%
\pgfsetdash{}{0pt}%
\pgfpathmoveto{\pgfqpoint{2.518786in}{0.984965in}}%
\pgfpathcurveto{\pgfqpoint{2.529836in}{0.984965in}}{\pgfqpoint{2.540435in}{0.989355in}}{\pgfqpoint{2.548249in}{0.997169in}}%
\pgfpathcurveto{\pgfqpoint{2.556062in}{1.004982in}}{\pgfqpoint{2.560452in}{1.015581in}}{\pgfqpoint{2.560452in}{1.026631in}}%
\pgfpathcurveto{\pgfqpoint{2.560452in}{1.037681in}}{\pgfqpoint{2.556062in}{1.048281in}}{\pgfqpoint{2.548249in}{1.056094in}}%
\pgfpathcurveto{\pgfqpoint{2.540435in}{1.063908in}}{\pgfqpoint{2.529836in}{1.068298in}}{\pgfqpoint{2.518786in}{1.068298in}}%
\pgfpathcurveto{\pgfqpoint{2.507736in}{1.068298in}}{\pgfqpoint{2.497137in}{1.063908in}}{\pgfqpoint{2.489323in}{1.056094in}}%
\pgfpathcurveto{\pgfqpoint{2.481509in}{1.048281in}}{\pgfqpoint{2.477119in}{1.037681in}}{\pgfqpoint{2.477119in}{1.026631in}}%
\pgfpathcurveto{\pgfqpoint{2.477119in}{1.015581in}}{\pgfqpoint{2.481509in}{1.004982in}}{\pgfqpoint{2.489323in}{0.997169in}}%
\pgfpathcurveto{\pgfqpoint{2.497137in}{0.989355in}}{\pgfqpoint{2.507736in}{0.984965in}}{\pgfqpoint{2.518786in}{0.984965in}}%
\pgfpathclose%
\pgfusepath{stroke,fill}%
\end{pgfscope}%
\begin{pgfscope}%
\pgfpathrectangle{\pgfqpoint{0.800000in}{0.528000in}}{\pgfqpoint{4.960000in}{3.696000in}}%
\pgfusepath{clip}%
\pgfsetbuttcap%
\pgfsetroundjoin%
\definecolor{currentfill}{rgb}{0.000000,0.000000,0.000000}%
\pgfsetfillcolor{currentfill}%
\pgfsetlinewidth{1.003750pt}%
\definecolor{currentstroke}{rgb}{0.000000,0.000000,0.000000}%
\pgfsetstrokecolor{currentstroke}%
\pgfsetdash{}{0pt}%
\pgfpathmoveto{\pgfqpoint{2.518786in}{1.023801in}}%
\pgfpathcurveto{\pgfqpoint{2.529836in}{1.023801in}}{\pgfqpoint{2.540435in}{1.028191in}}{\pgfqpoint{2.548249in}{1.036004in}}%
\pgfpathcurveto{\pgfqpoint{2.556062in}{1.043818in}}{\pgfqpoint{2.560452in}{1.054417in}}{\pgfqpoint{2.560452in}{1.065467in}}%
\pgfpathcurveto{\pgfqpoint{2.560452in}{1.076517in}}{\pgfqpoint{2.556062in}{1.087116in}}{\pgfqpoint{2.548249in}{1.094930in}}%
\pgfpathcurveto{\pgfqpoint{2.540435in}{1.102744in}}{\pgfqpoint{2.529836in}{1.107134in}}{\pgfqpoint{2.518786in}{1.107134in}}%
\pgfpathcurveto{\pgfqpoint{2.507736in}{1.107134in}}{\pgfqpoint{2.497137in}{1.102744in}}{\pgfqpoint{2.489323in}{1.094930in}}%
\pgfpathcurveto{\pgfqpoint{2.481509in}{1.087116in}}{\pgfqpoint{2.477119in}{1.076517in}}{\pgfqpoint{2.477119in}{1.065467in}}%
\pgfpathcurveto{\pgfqpoint{2.477119in}{1.054417in}}{\pgfqpoint{2.481509in}{1.043818in}}{\pgfqpoint{2.489323in}{1.036004in}}%
\pgfpathcurveto{\pgfqpoint{2.497137in}{1.028191in}}{\pgfqpoint{2.507736in}{1.023801in}}{\pgfqpoint{2.518786in}{1.023801in}}%
\pgfpathclose%
\pgfusepath{stroke,fill}%
\end{pgfscope}%
\begin{pgfscope}%
\pgfpathrectangle{\pgfqpoint{0.800000in}{0.528000in}}{\pgfqpoint{4.960000in}{3.696000in}}%
\pgfusepath{clip}%
\pgfsetbuttcap%
\pgfsetroundjoin%
\definecolor{currentfill}{rgb}{0.000000,0.000000,0.000000}%
\pgfsetfillcolor{currentfill}%
\pgfsetlinewidth{1.003750pt}%
\definecolor{currentstroke}{rgb}{0.000000,0.000000,0.000000}%
\pgfsetstrokecolor{currentstroke}%
\pgfsetdash{}{0pt}%
\pgfpathmoveto{\pgfqpoint{2.518786in}{1.023801in}}%
\pgfpathcurveto{\pgfqpoint{2.529836in}{1.023801in}}{\pgfqpoint{2.540435in}{1.028191in}}{\pgfqpoint{2.548249in}{1.036004in}}%
\pgfpathcurveto{\pgfqpoint{2.556062in}{1.043818in}}{\pgfqpoint{2.560452in}{1.054417in}}{\pgfqpoint{2.560452in}{1.065467in}}%
\pgfpathcurveto{\pgfqpoint{2.560452in}{1.076517in}}{\pgfqpoint{2.556062in}{1.087116in}}{\pgfqpoint{2.548249in}{1.094930in}}%
\pgfpathcurveto{\pgfqpoint{2.540435in}{1.102744in}}{\pgfqpoint{2.529836in}{1.107134in}}{\pgfqpoint{2.518786in}{1.107134in}}%
\pgfpathcurveto{\pgfqpoint{2.507736in}{1.107134in}}{\pgfqpoint{2.497137in}{1.102744in}}{\pgfqpoint{2.489323in}{1.094930in}}%
\pgfpathcurveto{\pgfqpoint{2.481509in}{1.087116in}}{\pgfqpoint{2.477119in}{1.076517in}}{\pgfqpoint{2.477119in}{1.065467in}}%
\pgfpathcurveto{\pgfqpoint{2.477119in}{1.054417in}}{\pgfqpoint{2.481509in}{1.043818in}}{\pgfqpoint{2.489323in}{1.036004in}}%
\pgfpathcurveto{\pgfqpoint{2.497137in}{1.028191in}}{\pgfqpoint{2.507736in}{1.023801in}}{\pgfqpoint{2.518786in}{1.023801in}}%
\pgfpathclose%
\pgfusepath{stroke,fill}%
\end{pgfscope}%
\begin{pgfscope}%
\pgfpathrectangle{\pgfqpoint{0.800000in}{0.528000in}}{\pgfqpoint{4.960000in}{3.696000in}}%
\pgfusepath{clip}%
\pgfsetbuttcap%
\pgfsetroundjoin%
\definecolor{currentfill}{rgb}{0.000000,0.000000,0.000000}%
\pgfsetfillcolor{currentfill}%
\pgfsetlinewidth{1.003750pt}%
\definecolor{currentstroke}{rgb}{0.000000,0.000000,0.000000}%
\pgfsetstrokecolor{currentstroke}%
\pgfsetdash{}{0pt}%
\pgfpathmoveto{\pgfqpoint{2.518786in}{1.052927in}}%
\pgfpathcurveto{\pgfqpoint{2.529836in}{1.052927in}}{\pgfqpoint{2.540435in}{1.057318in}}{\pgfqpoint{2.548249in}{1.065131in}}%
\pgfpathcurveto{\pgfqpoint{2.556062in}{1.072945in}}{\pgfqpoint{2.560452in}{1.083544in}}{\pgfqpoint{2.560452in}{1.094594in}}%
\pgfpathcurveto{\pgfqpoint{2.560452in}{1.105644in}}{\pgfqpoint{2.556062in}{1.116243in}}{\pgfqpoint{2.548249in}{1.124057in}}%
\pgfpathcurveto{\pgfqpoint{2.540435in}{1.131871in}}{\pgfqpoint{2.529836in}{1.136261in}}{\pgfqpoint{2.518786in}{1.136261in}}%
\pgfpathcurveto{\pgfqpoint{2.507736in}{1.136261in}}{\pgfqpoint{2.497137in}{1.131871in}}{\pgfqpoint{2.489323in}{1.124057in}}%
\pgfpathcurveto{\pgfqpoint{2.481509in}{1.116243in}}{\pgfqpoint{2.477119in}{1.105644in}}{\pgfqpoint{2.477119in}{1.094594in}}%
\pgfpathcurveto{\pgfqpoint{2.477119in}{1.083544in}}{\pgfqpoint{2.481509in}{1.072945in}}{\pgfqpoint{2.489323in}{1.065131in}}%
\pgfpathcurveto{\pgfqpoint{2.497137in}{1.057318in}}{\pgfqpoint{2.507736in}{1.052927in}}{\pgfqpoint{2.518786in}{1.052927in}}%
\pgfpathclose%
\pgfusepath{stroke,fill}%
\end{pgfscope}%
\begin{pgfscope}%
\pgfpathrectangle{\pgfqpoint{0.800000in}{0.528000in}}{\pgfqpoint{4.960000in}{3.696000in}}%
\pgfusepath{clip}%
\pgfsetbuttcap%
\pgfsetroundjoin%
\definecolor{currentfill}{rgb}{0.000000,0.000000,0.000000}%
\pgfsetfillcolor{currentfill}%
\pgfsetlinewidth{1.003750pt}%
\definecolor{currentstroke}{rgb}{0.000000,0.000000,0.000000}%
\pgfsetstrokecolor{currentstroke}%
\pgfsetdash{}{0pt}%
\pgfpathmoveto{\pgfqpoint{2.518786in}{1.052927in}}%
\pgfpathcurveto{\pgfqpoint{2.529836in}{1.052927in}}{\pgfqpoint{2.540435in}{1.057318in}}{\pgfqpoint{2.548249in}{1.065131in}}%
\pgfpathcurveto{\pgfqpoint{2.556062in}{1.072945in}}{\pgfqpoint{2.560452in}{1.083544in}}{\pgfqpoint{2.560452in}{1.094594in}}%
\pgfpathcurveto{\pgfqpoint{2.560452in}{1.105644in}}{\pgfqpoint{2.556062in}{1.116243in}}{\pgfqpoint{2.548249in}{1.124057in}}%
\pgfpathcurveto{\pgfqpoint{2.540435in}{1.131871in}}{\pgfqpoint{2.529836in}{1.136261in}}{\pgfqpoint{2.518786in}{1.136261in}}%
\pgfpathcurveto{\pgfqpoint{2.507736in}{1.136261in}}{\pgfqpoint{2.497137in}{1.131871in}}{\pgfqpoint{2.489323in}{1.124057in}}%
\pgfpathcurveto{\pgfqpoint{2.481509in}{1.116243in}}{\pgfqpoint{2.477119in}{1.105644in}}{\pgfqpoint{2.477119in}{1.094594in}}%
\pgfpathcurveto{\pgfqpoint{2.477119in}{1.083544in}}{\pgfqpoint{2.481509in}{1.072945in}}{\pgfqpoint{2.489323in}{1.065131in}}%
\pgfpathcurveto{\pgfqpoint{2.497137in}{1.057318in}}{\pgfqpoint{2.507736in}{1.052927in}}{\pgfqpoint{2.518786in}{1.052927in}}%
\pgfpathclose%
\pgfusepath{stroke,fill}%
\end{pgfscope}%
\begin{pgfscope}%
\pgfpathrectangle{\pgfqpoint{0.800000in}{0.528000in}}{\pgfqpoint{4.960000in}{3.696000in}}%
\pgfusepath{clip}%
\pgfsetbuttcap%
\pgfsetroundjoin%
\definecolor{currentfill}{rgb}{0.000000,0.000000,0.000000}%
\pgfsetfillcolor{currentfill}%
\pgfsetlinewidth{1.003750pt}%
\definecolor{currentstroke}{rgb}{0.000000,0.000000,0.000000}%
\pgfsetstrokecolor{currentstroke}%
\pgfsetdash{}{0pt}%
\pgfpathmoveto{\pgfqpoint{2.518786in}{1.266525in}}%
\pgfpathcurveto{\pgfqpoint{2.529836in}{1.266525in}}{\pgfqpoint{2.540435in}{1.270915in}}{\pgfqpoint{2.548249in}{1.278729in}}%
\pgfpathcurveto{\pgfqpoint{2.556062in}{1.286542in}}{\pgfqpoint{2.560452in}{1.297141in}}{\pgfqpoint{2.560452in}{1.308191in}}%
\pgfpathcurveto{\pgfqpoint{2.560452in}{1.319242in}}{\pgfqpoint{2.556062in}{1.329841in}}{\pgfqpoint{2.548249in}{1.337654in}}%
\pgfpathcurveto{\pgfqpoint{2.540435in}{1.345468in}}{\pgfqpoint{2.529836in}{1.349858in}}{\pgfqpoint{2.518786in}{1.349858in}}%
\pgfpathcurveto{\pgfqpoint{2.507736in}{1.349858in}}{\pgfqpoint{2.497137in}{1.345468in}}{\pgfqpoint{2.489323in}{1.337654in}}%
\pgfpathcurveto{\pgfqpoint{2.481509in}{1.329841in}}{\pgfqpoint{2.477119in}{1.319242in}}{\pgfqpoint{2.477119in}{1.308191in}}%
\pgfpathcurveto{\pgfqpoint{2.477119in}{1.297141in}}{\pgfqpoint{2.481509in}{1.286542in}}{\pgfqpoint{2.489323in}{1.278729in}}%
\pgfpathcurveto{\pgfqpoint{2.497137in}{1.270915in}}{\pgfqpoint{2.507736in}{1.266525in}}{\pgfqpoint{2.518786in}{1.266525in}}%
\pgfpathclose%
\pgfusepath{stroke,fill}%
\end{pgfscope}%
\begin{pgfscope}%
\pgfpathrectangle{\pgfqpoint{0.800000in}{0.528000in}}{\pgfqpoint{4.960000in}{3.696000in}}%
\pgfusepath{clip}%
\pgfsetbuttcap%
\pgfsetroundjoin%
\definecolor{currentfill}{rgb}{0.000000,0.000000,0.000000}%
\pgfsetfillcolor{currentfill}%
\pgfsetlinewidth{1.003750pt}%
\definecolor{currentstroke}{rgb}{0.000000,0.000000,0.000000}%
\pgfsetstrokecolor{currentstroke}%
\pgfsetdash{}{0pt}%
\pgfpathmoveto{\pgfqpoint{2.518786in}{1.052927in}}%
\pgfpathcurveto{\pgfqpoint{2.529836in}{1.052927in}}{\pgfqpoint{2.540435in}{1.057318in}}{\pgfqpoint{2.548249in}{1.065131in}}%
\pgfpathcurveto{\pgfqpoint{2.556062in}{1.072945in}}{\pgfqpoint{2.560452in}{1.083544in}}{\pgfqpoint{2.560452in}{1.094594in}}%
\pgfpathcurveto{\pgfqpoint{2.560452in}{1.105644in}}{\pgfqpoint{2.556062in}{1.116243in}}{\pgfqpoint{2.548249in}{1.124057in}}%
\pgfpathcurveto{\pgfqpoint{2.540435in}{1.131871in}}{\pgfqpoint{2.529836in}{1.136261in}}{\pgfqpoint{2.518786in}{1.136261in}}%
\pgfpathcurveto{\pgfqpoint{2.507736in}{1.136261in}}{\pgfqpoint{2.497137in}{1.131871in}}{\pgfqpoint{2.489323in}{1.124057in}}%
\pgfpathcurveto{\pgfqpoint{2.481509in}{1.116243in}}{\pgfqpoint{2.477119in}{1.105644in}}{\pgfqpoint{2.477119in}{1.094594in}}%
\pgfpathcurveto{\pgfqpoint{2.477119in}{1.083544in}}{\pgfqpoint{2.481509in}{1.072945in}}{\pgfqpoint{2.489323in}{1.065131in}}%
\pgfpathcurveto{\pgfqpoint{2.497137in}{1.057318in}}{\pgfqpoint{2.507736in}{1.052927in}}{\pgfqpoint{2.518786in}{1.052927in}}%
\pgfpathclose%
\pgfusepath{stroke,fill}%
\end{pgfscope}%
\begin{pgfscope}%
\pgfpathrectangle{\pgfqpoint{0.800000in}{0.528000in}}{\pgfqpoint{4.960000in}{3.696000in}}%
\pgfusepath{clip}%
\pgfsetbuttcap%
\pgfsetroundjoin%
\definecolor{currentfill}{rgb}{0.000000,0.000000,0.000000}%
\pgfsetfillcolor{currentfill}%
\pgfsetlinewidth{1.003750pt}%
\definecolor{currentstroke}{rgb}{0.000000,0.000000,0.000000}%
\pgfsetstrokecolor{currentstroke}%
\pgfsetdash{}{0pt}%
\pgfpathmoveto{\pgfqpoint{2.518786in}{1.033510in}}%
\pgfpathcurveto{\pgfqpoint{2.529836in}{1.033510in}}{\pgfqpoint{2.540435in}{1.037900in}}{\pgfqpoint{2.548249in}{1.045713in}}%
\pgfpathcurveto{\pgfqpoint{2.556062in}{1.053527in}}{\pgfqpoint{2.560452in}{1.064126in}}{\pgfqpoint{2.560452in}{1.075176in}}%
\pgfpathcurveto{\pgfqpoint{2.560452in}{1.086226in}}{\pgfqpoint{2.556062in}{1.096825in}}{\pgfqpoint{2.548249in}{1.104639in}}%
\pgfpathcurveto{\pgfqpoint{2.540435in}{1.112453in}}{\pgfqpoint{2.529836in}{1.116843in}}{\pgfqpoint{2.518786in}{1.116843in}}%
\pgfpathcurveto{\pgfqpoint{2.507736in}{1.116843in}}{\pgfqpoint{2.497137in}{1.112453in}}{\pgfqpoint{2.489323in}{1.104639in}}%
\pgfpathcurveto{\pgfqpoint{2.481509in}{1.096825in}}{\pgfqpoint{2.477119in}{1.086226in}}{\pgfqpoint{2.477119in}{1.075176in}}%
\pgfpathcurveto{\pgfqpoint{2.477119in}{1.064126in}}{\pgfqpoint{2.481509in}{1.053527in}}{\pgfqpoint{2.489323in}{1.045713in}}%
\pgfpathcurveto{\pgfqpoint{2.497137in}{1.037900in}}{\pgfqpoint{2.507736in}{1.033510in}}{\pgfqpoint{2.518786in}{1.033510in}}%
\pgfpathclose%
\pgfusepath{stroke,fill}%
\end{pgfscope}%
\begin{pgfscope}%
\pgfpathrectangle{\pgfqpoint{0.800000in}{0.528000in}}{\pgfqpoint{4.960000in}{3.696000in}}%
\pgfusepath{clip}%
\pgfsetbuttcap%
\pgfsetroundjoin%
\definecolor{currentfill}{rgb}{0.000000,0.000000,0.000000}%
\pgfsetfillcolor{currentfill}%
\pgfsetlinewidth{1.003750pt}%
\definecolor{currentstroke}{rgb}{0.000000,0.000000,0.000000}%
\pgfsetstrokecolor{currentstroke}%
\pgfsetdash{}{0pt}%
\pgfpathmoveto{\pgfqpoint{2.518786in}{1.014092in}}%
\pgfpathcurveto{\pgfqpoint{2.529836in}{1.014092in}}{\pgfqpoint{2.540435in}{1.018482in}}{\pgfqpoint{2.548249in}{1.026295in}}%
\pgfpathcurveto{\pgfqpoint{2.556062in}{1.034109in}}{\pgfqpoint{2.560452in}{1.044708in}}{\pgfqpoint{2.560452in}{1.055758in}}%
\pgfpathcurveto{\pgfqpoint{2.560452in}{1.066808in}}{\pgfqpoint{2.556062in}{1.077407in}}{\pgfqpoint{2.548249in}{1.085221in}}%
\pgfpathcurveto{\pgfqpoint{2.540435in}{1.093035in}}{\pgfqpoint{2.529836in}{1.097425in}}{\pgfqpoint{2.518786in}{1.097425in}}%
\pgfpathcurveto{\pgfqpoint{2.507736in}{1.097425in}}{\pgfqpoint{2.497137in}{1.093035in}}{\pgfqpoint{2.489323in}{1.085221in}}%
\pgfpathcurveto{\pgfqpoint{2.481509in}{1.077407in}}{\pgfqpoint{2.477119in}{1.066808in}}{\pgfqpoint{2.477119in}{1.055758in}}%
\pgfpathcurveto{\pgfqpoint{2.477119in}{1.044708in}}{\pgfqpoint{2.481509in}{1.034109in}}{\pgfqpoint{2.489323in}{1.026295in}}%
\pgfpathcurveto{\pgfqpoint{2.497137in}{1.018482in}}{\pgfqpoint{2.507736in}{1.014092in}}{\pgfqpoint{2.518786in}{1.014092in}}%
\pgfpathclose%
\pgfusepath{stroke,fill}%
\end{pgfscope}%
\begin{pgfscope}%
\pgfpathrectangle{\pgfqpoint{0.800000in}{0.528000in}}{\pgfqpoint{4.960000in}{3.696000in}}%
\pgfusepath{clip}%
\pgfsetbuttcap%
\pgfsetroundjoin%
\definecolor{currentfill}{rgb}{0.000000,0.000000,0.000000}%
\pgfsetfillcolor{currentfill}%
\pgfsetlinewidth{1.003750pt}%
\definecolor{currentstroke}{rgb}{0.000000,0.000000,0.000000}%
\pgfsetstrokecolor{currentstroke}%
\pgfsetdash{}{0pt}%
\pgfpathmoveto{\pgfqpoint{2.518786in}{0.975256in}}%
\pgfpathcurveto{\pgfqpoint{2.529836in}{0.975256in}}{\pgfqpoint{2.540435in}{0.979646in}}{\pgfqpoint{2.548249in}{0.987460in}}%
\pgfpathcurveto{\pgfqpoint{2.556062in}{0.995273in}}{\pgfqpoint{2.560452in}{1.005872in}}{\pgfqpoint{2.560452in}{1.016922in}}%
\pgfpathcurveto{\pgfqpoint{2.560452in}{1.027973in}}{\pgfqpoint{2.556062in}{1.038572in}}{\pgfqpoint{2.548249in}{1.046385in}}%
\pgfpathcurveto{\pgfqpoint{2.540435in}{1.054199in}}{\pgfqpoint{2.529836in}{1.058589in}}{\pgfqpoint{2.518786in}{1.058589in}}%
\pgfpathcurveto{\pgfqpoint{2.507736in}{1.058589in}}{\pgfqpoint{2.497137in}{1.054199in}}{\pgfqpoint{2.489323in}{1.046385in}}%
\pgfpathcurveto{\pgfqpoint{2.481509in}{1.038572in}}{\pgfqpoint{2.477119in}{1.027973in}}{\pgfqpoint{2.477119in}{1.016922in}}%
\pgfpathcurveto{\pgfqpoint{2.477119in}{1.005872in}}{\pgfqpoint{2.481509in}{0.995273in}}{\pgfqpoint{2.489323in}{0.987460in}}%
\pgfpathcurveto{\pgfqpoint{2.497137in}{0.979646in}}{\pgfqpoint{2.507736in}{0.975256in}}{\pgfqpoint{2.518786in}{0.975256in}}%
\pgfpathclose%
\pgfusepath{stroke,fill}%
\end{pgfscope}%
\begin{pgfscope}%
\pgfpathrectangle{\pgfqpoint{0.800000in}{0.528000in}}{\pgfqpoint{4.960000in}{3.696000in}}%
\pgfusepath{clip}%
\pgfsetbuttcap%
\pgfsetroundjoin%
\definecolor{currentfill}{rgb}{0.000000,0.000000,0.000000}%
\pgfsetfillcolor{currentfill}%
\pgfsetlinewidth{1.003750pt}%
\definecolor{currentstroke}{rgb}{0.000000,0.000000,0.000000}%
\pgfsetstrokecolor{currentstroke}%
\pgfsetdash{}{0pt}%
\pgfpathmoveto{\pgfqpoint{2.518786in}{0.984965in}}%
\pgfpathcurveto{\pgfqpoint{2.529836in}{0.984965in}}{\pgfqpoint{2.540435in}{0.989355in}}{\pgfqpoint{2.548249in}{0.997169in}}%
\pgfpathcurveto{\pgfqpoint{2.556062in}{1.004982in}}{\pgfqpoint{2.560452in}{1.015581in}}{\pgfqpoint{2.560452in}{1.026631in}}%
\pgfpathcurveto{\pgfqpoint{2.560452in}{1.037681in}}{\pgfqpoint{2.556062in}{1.048281in}}{\pgfqpoint{2.548249in}{1.056094in}}%
\pgfpathcurveto{\pgfqpoint{2.540435in}{1.063908in}}{\pgfqpoint{2.529836in}{1.068298in}}{\pgfqpoint{2.518786in}{1.068298in}}%
\pgfpathcurveto{\pgfqpoint{2.507736in}{1.068298in}}{\pgfqpoint{2.497137in}{1.063908in}}{\pgfqpoint{2.489323in}{1.056094in}}%
\pgfpathcurveto{\pgfqpoint{2.481509in}{1.048281in}}{\pgfqpoint{2.477119in}{1.037681in}}{\pgfqpoint{2.477119in}{1.026631in}}%
\pgfpathcurveto{\pgfqpoint{2.477119in}{1.015581in}}{\pgfqpoint{2.481509in}{1.004982in}}{\pgfqpoint{2.489323in}{0.997169in}}%
\pgfpathcurveto{\pgfqpoint{2.497137in}{0.989355in}}{\pgfqpoint{2.507736in}{0.984965in}}{\pgfqpoint{2.518786in}{0.984965in}}%
\pgfpathclose%
\pgfusepath{stroke,fill}%
\end{pgfscope}%
\begin{pgfscope}%
\pgfpathrectangle{\pgfqpoint{0.800000in}{0.528000in}}{\pgfqpoint{4.960000in}{3.696000in}}%
\pgfusepath{clip}%
\pgfsetbuttcap%
\pgfsetroundjoin%
\definecolor{currentfill}{rgb}{0.000000,0.000000,0.000000}%
\pgfsetfillcolor{currentfill}%
\pgfsetlinewidth{1.003750pt}%
\definecolor{currentstroke}{rgb}{0.000000,0.000000,0.000000}%
\pgfsetstrokecolor{currentstroke}%
\pgfsetdash{}{0pt}%
\pgfpathmoveto{\pgfqpoint{2.518786in}{0.965547in}}%
\pgfpathcurveto{\pgfqpoint{2.529836in}{0.965547in}}{\pgfqpoint{2.540435in}{0.969937in}}{\pgfqpoint{2.548249in}{0.977751in}}%
\pgfpathcurveto{\pgfqpoint{2.556062in}{0.985564in}}{\pgfqpoint{2.560452in}{0.996163in}}{\pgfqpoint{2.560452in}{1.007213in}}%
\pgfpathcurveto{\pgfqpoint{2.560452in}{1.018264in}}{\pgfqpoint{2.556062in}{1.028863in}}{\pgfqpoint{2.548249in}{1.036676in}}%
\pgfpathcurveto{\pgfqpoint{2.540435in}{1.044490in}}{\pgfqpoint{2.529836in}{1.048880in}}{\pgfqpoint{2.518786in}{1.048880in}}%
\pgfpathcurveto{\pgfqpoint{2.507736in}{1.048880in}}{\pgfqpoint{2.497137in}{1.044490in}}{\pgfqpoint{2.489323in}{1.036676in}}%
\pgfpathcurveto{\pgfqpoint{2.481509in}{1.028863in}}{\pgfqpoint{2.477119in}{1.018264in}}{\pgfqpoint{2.477119in}{1.007213in}}%
\pgfpathcurveto{\pgfqpoint{2.477119in}{0.996163in}}{\pgfqpoint{2.481509in}{0.985564in}}{\pgfqpoint{2.489323in}{0.977751in}}%
\pgfpathcurveto{\pgfqpoint{2.497137in}{0.969937in}}{\pgfqpoint{2.507736in}{0.965547in}}{\pgfqpoint{2.518786in}{0.965547in}}%
\pgfpathclose%
\pgfusepath{stroke,fill}%
\end{pgfscope}%
\begin{pgfscope}%
\pgfpathrectangle{\pgfqpoint{0.800000in}{0.528000in}}{\pgfqpoint{4.960000in}{3.696000in}}%
\pgfusepath{clip}%
\pgfsetbuttcap%
\pgfsetroundjoin%
\definecolor{currentfill}{rgb}{0.000000,0.000000,0.000000}%
\pgfsetfillcolor{currentfill}%
\pgfsetlinewidth{1.003750pt}%
\definecolor{currentstroke}{rgb}{0.000000,0.000000,0.000000}%
\pgfsetstrokecolor{currentstroke}%
\pgfsetdash{}{0pt}%
\pgfpathmoveto{\pgfqpoint{2.518786in}{1.033510in}}%
\pgfpathcurveto{\pgfqpoint{2.529836in}{1.033510in}}{\pgfqpoint{2.540435in}{1.037900in}}{\pgfqpoint{2.548249in}{1.045713in}}%
\pgfpathcurveto{\pgfqpoint{2.556062in}{1.053527in}}{\pgfqpoint{2.560452in}{1.064126in}}{\pgfqpoint{2.560452in}{1.075176in}}%
\pgfpathcurveto{\pgfqpoint{2.560452in}{1.086226in}}{\pgfqpoint{2.556062in}{1.096825in}}{\pgfqpoint{2.548249in}{1.104639in}}%
\pgfpathcurveto{\pgfqpoint{2.540435in}{1.112453in}}{\pgfqpoint{2.529836in}{1.116843in}}{\pgfqpoint{2.518786in}{1.116843in}}%
\pgfpathcurveto{\pgfqpoint{2.507736in}{1.116843in}}{\pgfqpoint{2.497137in}{1.112453in}}{\pgfqpoint{2.489323in}{1.104639in}}%
\pgfpathcurveto{\pgfqpoint{2.481509in}{1.096825in}}{\pgfqpoint{2.477119in}{1.086226in}}{\pgfqpoint{2.477119in}{1.075176in}}%
\pgfpathcurveto{\pgfqpoint{2.477119in}{1.064126in}}{\pgfqpoint{2.481509in}{1.053527in}}{\pgfqpoint{2.489323in}{1.045713in}}%
\pgfpathcurveto{\pgfqpoint{2.497137in}{1.037900in}}{\pgfqpoint{2.507736in}{1.033510in}}{\pgfqpoint{2.518786in}{1.033510in}}%
\pgfpathclose%
\pgfusepath{stroke,fill}%
\end{pgfscope}%
\begin{pgfscope}%
\pgfpathrectangle{\pgfqpoint{0.800000in}{0.528000in}}{\pgfqpoint{4.960000in}{3.696000in}}%
\pgfusepath{clip}%
\pgfsetbuttcap%
\pgfsetroundjoin%
\definecolor{currentfill}{rgb}{0.000000,0.000000,0.000000}%
\pgfsetfillcolor{currentfill}%
\pgfsetlinewidth{1.003750pt}%
\definecolor{currentstroke}{rgb}{0.000000,0.000000,0.000000}%
\pgfsetstrokecolor{currentstroke}%
\pgfsetdash{}{0pt}%
\pgfpathmoveto{\pgfqpoint{2.518786in}{1.101472in}}%
\pgfpathcurveto{\pgfqpoint{2.529836in}{1.101472in}}{\pgfqpoint{2.540435in}{1.105863in}}{\pgfqpoint{2.548249in}{1.113676in}}%
\pgfpathcurveto{\pgfqpoint{2.556062in}{1.121490in}}{\pgfqpoint{2.560452in}{1.132089in}}{\pgfqpoint{2.560452in}{1.143139in}}%
\pgfpathcurveto{\pgfqpoint{2.560452in}{1.154189in}}{\pgfqpoint{2.556062in}{1.164788in}}{\pgfqpoint{2.548249in}{1.172602in}}%
\pgfpathcurveto{\pgfqpoint{2.540435in}{1.180415in}}{\pgfqpoint{2.529836in}{1.184806in}}{\pgfqpoint{2.518786in}{1.184806in}}%
\pgfpathcurveto{\pgfqpoint{2.507736in}{1.184806in}}{\pgfqpoint{2.497137in}{1.180415in}}{\pgfqpoint{2.489323in}{1.172602in}}%
\pgfpathcurveto{\pgfqpoint{2.481509in}{1.164788in}}{\pgfqpoint{2.477119in}{1.154189in}}{\pgfqpoint{2.477119in}{1.143139in}}%
\pgfpathcurveto{\pgfqpoint{2.477119in}{1.132089in}}{\pgfqpoint{2.481509in}{1.121490in}}{\pgfqpoint{2.489323in}{1.113676in}}%
\pgfpathcurveto{\pgfqpoint{2.497137in}{1.105863in}}{\pgfqpoint{2.507736in}{1.101472in}}{\pgfqpoint{2.518786in}{1.101472in}}%
\pgfpathclose%
\pgfusepath{stroke,fill}%
\end{pgfscope}%
\begin{pgfscope}%
\pgfpathrectangle{\pgfqpoint{0.800000in}{0.528000in}}{\pgfqpoint{4.960000in}{3.696000in}}%
\pgfusepath{clip}%
\pgfsetbuttcap%
\pgfsetroundjoin%
\definecolor{currentfill}{rgb}{0.000000,0.000000,0.000000}%
\pgfsetfillcolor{currentfill}%
\pgfsetlinewidth{1.003750pt}%
\definecolor{currentstroke}{rgb}{0.000000,0.000000,0.000000}%
\pgfsetstrokecolor{currentstroke}%
\pgfsetdash{}{0pt}%
\pgfpathmoveto{\pgfqpoint{2.518786in}{1.150017in}}%
\pgfpathcurveto{\pgfqpoint{2.529836in}{1.150017in}}{\pgfqpoint{2.540435in}{1.154407in}}{\pgfqpoint{2.548249in}{1.162221in}}%
\pgfpathcurveto{\pgfqpoint{2.556062in}{1.170035in}}{\pgfqpoint{2.560452in}{1.180634in}}{\pgfqpoint{2.560452in}{1.191684in}}%
\pgfpathcurveto{\pgfqpoint{2.560452in}{1.202734in}}{\pgfqpoint{2.556062in}{1.213333in}}{\pgfqpoint{2.548249in}{1.221147in}}%
\pgfpathcurveto{\pgfqpoint{2.540435in}{1.228960in}}{\pgfqpoint{2.529836in}{1.233350in}}{\pgfqpoint{2.518786in}{1.233350in}}%
\pgfpathcurveto{\pgfqpoint{2.507736in}{1.233350in}}{\pgfqpoint{2.497137in}{1.228960in}}{\pgfqpoint{2.489323in}{1.221147in}}%
\pgfpathcurveto{\pgfqpoint{2.481509in}{1.213333in}}{\pgfqpoint{2.477119in}{1.202734in}}{\pgfqpoint{2.477119in}{1.191684in}}%
\pgfpathcurveto{\pgfqpoint{2.477119in}{1.180634in}}{\pgfqpoint{2.481509in}{1.170035in}}{\pgfqpoint{2.489323in}{1.162221in}}%
\pgfpathcurveto{\pgfqpoint{2.497137in}{1.154407in}}{\pgfqpoint{2.507736in}{1.150017in}}{\pgfqpoint{2.518786in}{1.150017in}}%
\pgfpathclose%
\pgfusepath{stroke,fill}%
\end{pgfscope}%
\begin{pgfscope}%
\pgfpathrectangle{\pgfqpoint{0.800000in}{0.528000in}}{\pgfqpoint{4.960000in}{3.696000in}}%
\pgfusepath{clip}%
\pgfsetbuttcap%
\pgfsetroundjoin%
\definecolor{currentfill}{rgb}{0.000000,0.000000,0.000000}%
\pgfsetfillcolor{currentfill}%
\pgfsetlinewidth{1.003750pt}%
\definecolor{currentstroke}{rgb}{0.000000,0.000000,0.000000}%
\pgfsetstrokecolor{currentstroke}%
\pgfsetdash{}{0pt}%
\pgfpathmoveto{\pgfqpoint{2.518786in}{1.043219in}}%
\pgfpathcurveto{\pgfqpoint{2.529836in}{1.043219in}}{\pgfqpoint{2.540435in}{1.047609in}}{\pgfqpoint{2.548249in}{1.055422in}}%
\pgfpathcurveto{\pgfqpoint{2.556062in}{1.063236in}}{\pgfqpoint{2.560452in}{1.073835in}}{\pgfqpoint{2.560452in}{1.084885in}}%
\pgfpathcurveto{\pgfqpoint{2.560452in}{1.095935in}}{\pgfqpoint{2.556062in}{1.106534in}}{\pgfqpoint{2.548249in}{1.114348in}}%
\pgfpathcurveto{\pgfqpoint{2.540435in}{1.122162in}}{\pgfqpoint{2.529836in}{1.126552in}}{\pgfqpoint{2.518786in}{1.126552in}}%
\pgfpathcurveto{\pgfqpoint{2.507736in}{1.126552in}}{\pgfqpoint{2.497137in}{1.122162in}}{\pgfqpoint{2.489323in}{1.114348in}}%
\pgfpathcurveto{\pgfqpoint{2.481509in}{1.106534in}}{\pgfqpoint{2.477119in}{1.095935in}}{\pgfqpoint{2.477119in}{1.084885in}}%
\pgfpathcurveto{\pgfqpoint{2.477119in}{1.073835in}}{\pgfqpoint{2.481509in}{1.063236in}}{\pgfqpoint{2.489323in}{1.055422in}}%
\pgfpathcurveto{\pgfqpoint{2.497137in}{1.047609in}}{\pgfqpoint{2.507736in}{1.043219in}}{\pgfqpoint{2.518786in}{1.043219in}}%
\pgfpathclose%
\pgfusepath{stroke,fill}%
\end{pgfscope}%
\begin{pgfscope}%
\pgfpathrectangle{\pgfqpoint{0.800000in}{0.528000in}}{\pgfqpoint{4.960000in}{3.696000in}}%
\pgfusepath{clip}%
\pgfsetbuttcap%
\pgfsetroundjoin%
\definecolor{currentfill}{rgb}{0.000000,0.000000,0.000000}%
\pgfsetfillcolor{currentfill}%
\pgfsetlinewidth{1.003750pt}%
\definecolor{currentstroke}{rgb}{0.000000,0.000000,0.000000}%
\pgfsetstrokecolor{currentstroke}%
\pgfsetdash{}{0pt}%
\pgfpathmoveto{\pgfqpoint{2.518786in}{1.004383in}}%
\pgfpathcurveto{\pgfqpoint{2.529836in}{1.004383in}}{\pgfqpoint{2.540435in}{1.008773in}}{\pgfqpoint{2.548249in}{1.016587in}}%
\pgfpathcurveto{\pgfqpoint{2.556062in}{1.024400in}}{\pgfqpoint{2.560452in}{1.034999in}}{\pgfqpoint{2.560452in}{1.046049in}}%
\pgfpathcurveto{\pgfqpoint{2.560452in}{1.057099in}}{\pgfqpoint{2.556062in}{1.067698in}}{\pgfqpoint{2.548249in}{1.075512in}}%
\pgfpathcurveto{\pgfqpoint{2.540435in}{1.083326in}}{\pgfqpoint{2.529836in}{1.087716in}}{\pgfqpoint{2.518786in}{1.087716in}}%
\pgfpathcurveto{\pgfqpoint{2.507736in}{1.087716in}}{\pgfqpoint{2.497137in}{1.083326in}}{\pgfqpoint{2.489323in}{1.075512in}}%
\pgfpathcurveto{\pgfqpoint{2.481509in}{1.067698in}}{\pgfqpoint{2.477119in}{1.057099in}}{\pgfqpoint{2.477119in}{1.046049in}}%
\pgfpathcurveto{\pgfqpoint{2.477119in}{1.034999in}}{\pgfqpoint{2.481509in}{1.024400in}}{\pgfqpoint{2.489323in}{1.016587in}}%
\pgfpathcurveto{\pgfqpoint{2.497137in}{1.008773in}}{\pgfqpoint{2.507736in}{1.004383in}}{\pgfqpoint{2.518786in}{1.004383in}}%
\pgfpathclose%
\pgfusepath{stroke,fill}%
\end{pgfscope}%
\begin{pgfscope}%
\pgfpathrectangle{\pgfqpoint{0.800000in}{0.528000in}}{\pgfqpoint{4.960000in}{3.696000in}}%
\pgfusepath{clip}%
\pgfsetbuttcap%
\pgfsetroundjoin%
\definecolor{currentfill}{rgb}{0.000000,0.000000,0.000000}%
\pgfsetfillcolor{currentfill}%
\pgfsetlinewidth{1.003750pt}%
\definecolor{currentstroke}{rgb}{0.000000,0.000000,0.000000}%
\pgfsetstrokecolor{currentstroke}%
\pgfsetdash{}{0pt}%
\pgfpathmoveto{\pgfqpoint{2.518786in}{1.091763in}}%
\pgfpathcurveto{\pgfqpoint{2.529836in}{1.091763in}}{\pgfqpoint{2.540435in}{1.096154in}}{\pgfqpoint{2.548249in}{1.103967in}}%
\pgfpathcurveto{\pgfqpoint{2.556062in}{1.111781in}}{\pgfqpoint{2.560452in}{1.122380in}}{\pgfqpoint{2.560452in}{1.133430in}}%
\pgfpathcurveto{\pgfqpoint{2.560452in}{1.144480in}}{\pgfqpoint{2.556062in}{1.155079in}}{\pgfqpoint{2.548249in}{1.162893in}}%
\pgfpathcurveto{\pgfqpoint{2.540435in}{1.170706in}}{\pgfqpoint{2.529836in}{1.175097in}}{\pgfqpoint{2.518786in}{1.175097in}}%
\pgfpathcurveto{\pgfqpoint{2.507736in}{1.175097in}}{\pgfqpoint{2.497137in}{1.170706in}}{\pgfqpoint{2.489323in}{1.162893in}}%
\pgfpathcurveto{\pgfqpoint{2.481509in}{1.155079in}}{\pgfqpoint{2.477119in}{1.144480in}}{\pgfqpoint{2.477119in}{1.133430in}}%
\pgfpathcurveto{\pgfqpoint{2.477119in}{1.122380in}}{\pgfqpoint{2.481509in}{1.111781in}}{\pgfqpoint{2.489323in}{1.103967in}}%
\pgfpathcurveto{\pgfqpoint{2.497137in}{1.096154in}}{\pgfqpoint{2.507736in}{1.091763in}}{\pgfqpoint{2.518786in}{1.091763in}}%
\pgfpathclose%
\pgfusepath{stroke,fill}%
\end{pgfscope}%
\begin{pgfscope}%
\pgfpathrectangle{\pgfqpoint{0.800000in}{0.528000in}}{\pgfqpoint{4.960000in}{3.696000in}}%
\pgfusepath{clip}%
\pgfsetbuttcap%
\pgfsetroundjoin%
\definecolor{currentfill}{rgb}{0.000000,0.000000,0.000000}%
\pgfsetfillcolor{currentfill}%
\pgfsetlinewidth{1.003750pt}%
\definecolor{currentstroke}{rgb}{0.000000,0.000000,0.000000}%
\pgfsetstrokecolor{currentstroke}%
\pgfsetdash{}{0pt}%
\pgfpathmoveto{\pgfqpoint{2.518786in}{1.111181in}}%
\pgfpathcurveto{\pgfqpoint{2.529836in}{1.111181in}}{\pgfqpoint{2.540435in}{1.115572in}}{\pgfqpoint{2.548249in}{1.123385in}}%
\pgfpathcurveto{\pgfqpoint{2.556062in}{1.131199in}}{\pgfqpoint{2.560452in}{1.141798in}}{\pgfqpoint{2.560452in}{1.152848in}}%
\pgfpathcurveto{\pgfqpoint{2.560452in}{1.163898in}}{\pgfqpoint{2.556062in}{1.174497in}}{\pgfqpoint{2.548249in}{1.182311in}}%
\pgfpathcurveto{\pgfqpoint{2.540435in}{1.190124in}}{\pgfqpoint{2.529836in}{1.194515in}}{\pgfqpoint{2.518786in}{1.194515in}}%
\pgfpathcurveto{\pgfqpoint{2.507736in}{1.194515in}}{\pgfqpoint{2.497137in}{1.190124in}}{\pgfqpoint{2.489323in}{1.182311in}}%
\pgfpathcurveto{\pgfqpoint{2.481509in}{1.174497in}}{\pgfqpoint{2.477119in}{1.163898in}}{\pgfqpoint{2.477119in}{1.152848in}}%
\pgfpathcurveto{\pgfqpoint{2.477119in}{1.141798in}}{\pgfqpoint{2.481509in}{1.131199in}}{\pgfqpoint{2.489323in}{1.123385in}}%
\pgfpathcurveto{\pgfqpoint{2.497137in}{1.115572in}}{\pgfqpoint{2.507736in}{1.111181in}}{\pgfqpoint{2.518786in}{1.111181in}}%
\pgfpathclose%
\pgfusepath{stroke,fill}%
\end{pgfscope}%
\begin{pgfscope}%
\pgfpathrectangle{\pgfqpoint{0.800000in}{0.528000in}}{\pgfqpoint{4.960000in}{3.696000in}}%
\pgfusepath{clip}%
\pgfsetbuttcap%
\pgfsetroundjoin%
\definecolor{currentfill}{rgb}{0.000000,0.000000,0.000000}%
\pgfsetfillcolor{currentfill}%
\pgfsetlinewidth{1.003750pt}%
\definecolor{currentstroke}{rgb}{0.000000,0.000000,0.000000}%
\pgfsetstrokecolor{currentstroke}%
\pgfsetdash{}{0pt}%
\pgfpathmoveto{\pgfqpoint{2.518786in}{1.004383in}}%
\pgfpathcurveto{\pgfqpoint{2.529836in}{1.004383in}}{\pgfqpoint{2.540435in}{1.008773in}}{\pgfqpoint{2.548249in}{1.016587in}}%
\pgfpathcurveto{\pgfqpoint{2.556062in}{1.024400in}}{\pgfqpoint{2.560452in}{1.034999in}}{\pgfqpoint{2.560452in}{1.046049in}}%
\pgfpathcurveto{\pgfqpoint{2.560452in}{1.057099in}}{\pgfqpoint{2.556062in}{1.067698in}}{\pgfqpoint{2.548249in}{1.075512in}}%
\pgfpathcurveto{\pgfqpoint{2.540435in}{1.083326in}}{\pgfqpoint{2.529836in}{1.087716in}}{\pgfqpoint{2.518786in}{1.087716in}}%
\pgfpathcurveto{\pgfqpoint{2.507736in}{1.087716in}}{\pgfqpoint{2.497137in}{1.083326in}}{\pgfqpoint{2.489323in}{1.075512in}}%
\pgfpathcurveto{\pgfqpoint{2.481509in}{1.067698in}}{\pgfqpoint{2.477119in}{1.057099in}}{\pgfqpoint{2.477119in}{1.046049in}}%
\pgfpathcurveto{\pgfqpoint{2.477119in}{1.034999in}}{\pgfqpoint{2.481509in}{1.024400in}}{\pgfqpoint{2.489323in}{1.016587in}}%
\pgfpathcurveto{\pgfqpoint{2.497137in}{1.008773in}}{\pgfqpoint{2.507736in}{1.004383in}}{\pgfqpoint{2.518786in}{1.004383in}}%
\pgfpathclose%
\pgfusepath{stroke,fill}%
\end{pgfscope}%
\begin{pgfscope}%
\pgfpathrectangle{\pgfqpoint{0.800000in}{0.528000in}}{\pgfqpoint{4.960000in}{3.696000in}}%
\pgfusepath{clip}%
\pgfsetbuttcap%
\pgfsetroundjoin%
\definecolor{currentfill}{rgb}{0.000000,0.000000,0.000000}%
\pgfsetfillcolor{currentfill}%
\pgfsetlinewidth{1.003750pt}%
\definecolor{currentstroke}{rgb}{0.000000,0.000000,0.000000}%
\pgfsetstrokecolor{currentstroke}%
\pgfsetdash{}{0pt}%
\pgfpathmoveto{\pgfqpoint{2.518786in}{0.946129in}}%
\pgfpathcurveto{\pgfqpoint{2.529836in}{0.946129in}}{\pgfqpoint{2.540435in}{0.950519in}}{\pgfqpoint{2.548249in}{0.958333in}}%
\pgfpathcurveto{\pgfqpoint{2.556062in}{0.966146in}}{\pgfqpoint{2.560452in}{0.976745in}}{\pgfqpoint{2.560452in}{0.987795in}}%
\pgfpathcurveto{\pgfqpoint{2.560452in}{0.998846in}}{\pgfqpoint{2.556062in}{1.009445in}}{\pgfqpoint{2.548249in}{1.017258in}}%
\pgfpathcurveto{\pgfqpoint{2.540435in}{1.025072in}}{\pgfqpoint{2.529836in}{1.029462in}}{\pgfqpoint{2.518786in}{1.029462in}}%
\pgfpathcurveto{\pgfqpoint{2.507736in}{1.029462in}}{\pgfqpoint{2.497137in}{1.025072in}}{\pgfqpoint{2.489323in}{1.017258in}}%
\pgfpathcurveto{\pgfqpoint{2.481509in}{1.009445in}}{\pgfqpoint{2.477119in}{0.998846in}}{\pgfqpoint{2.477119in}{0.987795in}}%
\pgfpathcurveto{\pgfqpoint{2.477119in}{0.976745in}}{\pgfqpoint{2.481509in}{0.966146in}}{\pgfqpoint{2.489323in}{0.958333in}}%
\pgfpathcurveto{\pgfqpoint{2.497137in}{0.950519in}}{\pgfqpoint{2.507736in}{0.946129in}}{\pgfqpoint{2.518786in}{0.946129in}}%
\pgfpathclose%
\pgfusepath{stroke,fill}%
\end{pgfscope}%
\begin{pgfscope}%
\pgfpathrectangle{\pgfqpoint{0.800000in}{0.528000in}}{\pgfqpoint{4.960000in}{3.696000in}}%
\pgfusepath{clip}%
\pgfsetbuttcap%
\pgfsetroundjoin%
\definecolor{currentfill}{rgb}{0.000000,0.000000,0.000000}%
\pgfsetfillcolor{currentfill}%
\pgfsetlinewidth{1.003750pt}%
\definecolor{currentstroke}{rgb}{0.000000,0.000000,0.000000}%
\pgfsetstrokecolor{currentstroke}%
\pgfsetdash{}{0pt}%
\pgfpathmoveto{\pgfqpoint{2.518786in}{0.946129in}}%
\pgfpathcurveto{\pgfqpoint{2.529836in}{0.946129in}}{\pgfqpoint{2.540435in}{0.950519in}}{\pgfqpoint{2.548249in}{0.958333in}}%
\pgfpathcurveto{\pgfqpoint{2.556062in}{0.966146in}}{\pgfqpoint{2.560452in}{0.976745in}}{\pgfqpoint{2.560452in}{0.987795in}}%
\pgfpathcurveto{\pgfqpoint{2.560452in}{0.998846in}}{\pgfqpoint{2.556062in}{1.009445in}}{\pgfqpoint{2.548249in}{1.017258in}}%
\pgfpathcurveto{\pgfqpoint{2.540435in}{1.025072in}}{\pgfqpoint{2.529836in}{1.029462in}}{\pgfqpoint{2.518786in}{1.029462in}}%
\pgfpathcurveto{\pgfqpoint{2.507736in}{1.029462in}}{\pgfqpoint{2.497137in}{1.025072in}}{\pgfqpoint{2.489323in}{1.017258in}}%
\pgfpathcurveto{\pgfqpoint{2.481509in}{1.009445in}}{\pgfqpoint{2.477119in}{0.998846in}}{\pgfqpoint{2.477119in}{0.987795in}}%
\pgfpathcurveto{\pgfqpoint{2.477119in}{0.976745in}}{\pgfqpoint{2.481509in}{0.966146in}}{\pgfqpoint{2.489323in}{0.958333in}}%
\pgfpathcurveto{\pgfqpoint{2.497137in}{0.950519in}}{\pgfqpoint{2.507736in}{0.946129in}}{\pgfqpoint{2.518786in}{0.946129in}}%
\pgfpathclose%
\pgfusepath{stroke,fill}%
\end{pgfscope}%
\begin{pgfscope}%
\pgfpathrectangle{\pgfqpoint{0.800000in}{0.528000in}}{\pgfqpoint{4.960000in}{3.696000in}}%
\pgfusepath{clip}%
\pgfsetbuttcap%
\pgfsetroundjoin%
\definecolor{currentfill}{rgb}{0.000000,0.000000,0.000000}%
\pgfsetfillcolor{currentfill}%
\pgfsetlinewidth{1.003750pt}%
\definecolor{currentstroke}{rgb}{0.000000,0.000000,0.000000}%
\pgfsetstrokecolor{currentstroke}%
\pgfsetdash{}{0pt}%
\pgfpathmoveto{\pgfqpoint{2.518786in}{1.033510in}}%
\pgfpathcurveto{\pgfqpoint{2.529836in}{1.033510in}}{\pgfqpoint{2.540435in}{1.037900in}}{\pgfqpoint{2.548249in}{1.045713in}}%
\pgfpathcurveto{\pgfqpoint{2.556062in}{1.053527in}}{\pgfqpoint{2.560452in}{1.064126in}}{\pgfqpoint{2.560452in}{1.075176in}}%
\pgfpathcurveto{\pgfqpoint{2.560452in}{1.086226in}}{\pgfqpoint{2.556062in}{1.096825in}}{\pgfqpoint{2.548249in}{1.104639in}}%
\pgfpathcurveto{\pgfqpoint{2.540435in}{1.112453in}}{\pgfqpoint{2.529836in}{1.116843in}}{\pgfqpoint{2.518786in}{1.116843in}}%
\pgfpathcurveto{\pgfqpoint{2.507736in}{1.116843in}}{\pgfqpoint{2.497137in}{1.112453in}}{\pgfqpoint{2.489323in}{1.104639in}}%
\pgfpathcurveto{\pgfqpoint{2.481509in}{1.096825in}}{\pgfqpoint{2.477119in}{1.086226in}}{\pgfqpoint{2.477119in}{1.075176in}}%
\pgfpathcurveto{\pgfqpoint{2.477119in}{1.064126in}}{\pgfqpoint{2.481509in}{1.053527in}}{\pgfqpoint{2.489323in}{1.045713in}}%
\pgfpathcurveto{\pgfqpoint{2.497137in}{1.037900in}}{\pgfqpoint{2.507736in}{1.033510in}}{\pgfqpoint{2.518786in}{1.033510in}}%
\pgfpathclose%
\pgfusepath{stroke,fill}%
\end{pgfscope}%
\begin{pgfscope}%
\pgfpathrectangle{\pgfqpoint{0.800000in}{0.528000in}}{\pgfqpoint{4.960000in}{3.696000in}}%
\pgfusepath{clip}%
\pgfsetbuttcap%
\pgfsetroundjoin%
\definecolor{currentfill}{rgb}{0.000000,0.000000,0.000000}%
\pgfsetfillcolor{currentfill}%
\pgfsetlinewidth{1.003750pt}%
\definecolor{currentstroke}{rgb}{0.000000,0.000000,0.000000}%
\pgfsetstrokecolor{currentstroke}%
\pgfsetdash{}{0pt}%
\pgfpathmoveto{\pgfqpoint{2.518786in}{1.091763in}}%
\pgfpathcurveto{\pgfqpoint{2.529836in}{1.091763in}}{\pgfqpoint{2.540435in}{1.096154in}}{\pgfqpoint{2.548249in}{1.103967in}}%
\pgfpathcurveto{\pgfqpoint{2.556062in}{1.111781in}}{\pgfqpoint{2.560452in}{1.122380in}}{\pgfqpoint{2.560452in}{1.133430in}}%
\pgfpathcurveto{\pgfqpoint{2.560452in}{1.144480in}}{\pgfqpoint{2.556062in}{1.155079in}}{\pgfqpoint{2.548249in}{1.162893in}}%
\pgfpathcurveto{\pgfqpoint{2.540435in}{1.170706in}}{\pgfqpoint{2.529836in}{1.175097in}}{\pgfqpoint{2.518786in}{1.175097in}}%
\pgfpathcurveto{\pgfqpoint{2.507736in}{1.175097in}}{\pgfqpoint{2.497137in}{1.170706in}}{\pgfqpoint{2.489323in}{1.162893in}}%
\pgfpathcurveto{\pgfqpoint{2.481509in}{1.155079in}}{\pgfqpoint{2.477119in}{1.144480in}}{\pgfqpoint{2.477119in}{1.133430in}}%
\pgfpathcurveto{\pgfqpoint{2.477119in}{1.122380in}}{\pgfqpoint{2.481509in}{1.111781in}}{\pgfqpoint{2.489323in}{1.103967in}}%
\pgfpathcurveto{\pgfqpoint{2.497137in}{1.096154in}}{\pgfqpoint{2.507736in}{1.091763in}}{\pgfqpoint{2.518786in}{1.091763in}}%
\pgfpathclose%
\pgfusepath{stroke,fill}%
\end{pgfscope}%
\begin{pgfscope}%
\pgfpathrectangle{\pgfqpoint{0.800000in}{0.528000in}}{\pgfqpoint{4.960000in}{3.696000in}}%
\pgfusepath{clip}%
\pgfsetbuttcap%
\pgfsetroundjoin%
\definecolor{currentfill}{rgb}{0.000000,0.000000,0.000000}%
\pgfsetfillcolor{currentfill}%
\pgfsetlinewidth{1.003750pt}%
\definecolor{currentstroke}{rgb}{0.000000,0.000000,0.000000}%
\pgfsetstrokecolor{currentstroke}%
\pgfsetdash{}{0pt}%
\pgfpathmoveto{\pgfqpoint{2.518786in}{1.120890in}}%
\pgfpathcurveto{\pgfqpoint{2.529836in}{1.120890in}}{\pgfqpoint{2.540435in}{1.125281in}}{\pgfqpoint{2.548249in}{1.133094in}}%
\pgfpathcurveto{\pgfqpoint{2.556062in}{1.140908in}}{\pgfqpoint{2.560452in}{1.151507in}}{\pgfqpoint{2.560452in}{1.162557in}}%
\pgfpathcurveto{\pgfqpoint{2.560452in}{1.173607in}}{\pgfqpoint{2.556062in}{1.184206in}}{\pgfqpoint{2.548249in}{1.192020in}}%
\pgfpathcurveto{\pgfqpoint{2.540435in}{1.199833in}}{\pgfqpoint{2.529836in}{1.204224in}}{\pgfqpoint{2.518786in}{1.204224in}}%
\pgfpathcurveto{\pgfqpoint{2.507736in}{1.204224in}}{\pgfqpoint{2.497137in}{1.199833in}}{\pgfqpoint{2.489323in}{1.192020in}}%
\pgfpathcurveto{\pgfqpoint{2.481509in}{1.184206in}}{\pgfqpoint{2.477119in}{1.173607in}}{\pgfqpoint{2.477119in}{1.162557in}}%
\pgfpathcurveto{\pgfqpoint{2.477119in}{1.151507in}}{\pgfqpoint{2.481509in}{1.140908in}}{\pgfqpoint{2.489323in}{1.133094in}}%
\pgfpathcurveto{\pgfqpoint{2.497137in}{1.125281in}}{\pgfqpoint{2.507736in}{1.120890in}}{\pgfqpoint{2.518786in}{1.120890in}}%
\pgfpathclose%
\pgfusepath{stroke,fill}%
\end{pgfscope}%
\begin{pgfscope}%
\pgfpathrectangle{\pgfqpoint{0.800000in}{0.528000in}}{\pgfqpoint{4.960000in}{3.696000in}}%
\pgfusepath{clip}%
\pgfsetbuttcap%
\pgfsetroundjoin%
\definecolor{currentfill}{rgb}{0.000000,0.000000,0.000000}%
\pgfsetfillcolor{currentfill}%
\pgfsetlinewidth{1.003750pt}%
\definecolor{currentstroke}{rgb}{0.000000,0.000000,0.000000}%
\pgfsetstrokecolor{currentstroke}%
\pgfsetdash{}{0pt}%
\pgfpathmoveto{\pgfqpoint{2.518786in}{1.052927in}}%
\pgfpathcurveto{\pgfqpoint{2.529836in}{1.052927in}}{\pgfqpoint{2.540435in}{1.057318in}}{\pgfqpoint{2.548249in}{1.065131in}}%
\pgfpathcurveto{\pgfqpoint{2.556062in}{1.072945in}}{\pgfqpoint{2.560452in}{1.083544in}}{\pgfqpoint{2.560452in}{1.094594in}}%
\pgfpathcurveto{\pgfqpoint{2.560452in}{1.105644in}}{\pgfqpoint{2.556062in}{1.116243in}}{\pgfqpoint{2.548249in}{1.124057in}}%
\pgfpathcurveto{\pgfqpoint{2.540435in}{1.131871in}}{\pgfqpoint{2.529836in}{1.136261in}}{\pgfqpoint{2.518786in}{1.136261in}}%
\pgfpathcurveto{\pgfqpoint{2.507736in}{1.136261in}}{\pgfqpoint{2.497137in}{1.131871in}}{\pgfqpoint{2.489323in}{1.124057in}}%
\pgfpathcurveto{\pgfqpoint{2.481509in}{1.116243in}}{\pgfqpoint{2.477119in}{1.105644in}}{\pgfqpoint{2.477119in}{1.094594in}}%
\pgfpathcurveto{\pgfqpoint{2.477119in}{1.083544in}}{\pgfqpoint{2.481509in}{1.072945in}}{\pgfqpoint{2.489323in}{1.065131in}}%
\pgfpathcurveto{\pgfqpoint{2.497137in}{1.057318in}}{\pgfqpoint{2.507736in}{1.052927in}}{\pgfqpoint{2.518786in}{1.052927in}}%
\pgfpathclose%
\pgfusepath{stroke,fill}%
\end{pgfscope}%
\begin{pgfscope}%
\pgfpathrectangle{\pgfqpoint{0.800000in}{0.528000in}}{\pgfqpoint{4.960000in}{3.696000in}}%
\pgfusepath{clip}%
\pgfsetbuttcap%
\pgfsetroundjoin%
\definecolor{currentfill}{rgb}{0.000000,0.000000,0.000000}%
\pgfsetfillcolor{currentfill}%
\pgfsetlinewidth{1.003750pt}%
\definecolor{currentstroke}{rgb}{0.000000,0.000000,0.000000}%
\pgfsetstrokecolor{currentstroke}%
\pgfsetdash{}{0pt}%
\pgfpathmoveto{\pgfqpoint{2.518786in}{1.130599in}}%
\pgfpathcurveto{\pgfqpoint{2.529836in}{1.130599in}}{\pgfqpoint{2.540435in}{1.134989in}}{\pgfqpoint{2.548249in}{1.142803in}}%
\pgfpathcurveto{\pgfqpoint{2.556062in}{1.150617in}}{\pgfqpoint{2.560452in}{1.161216in}}{\pgfqpoint{2.560452in}{1.172266in}}%
\pgfpathcurveto{\pgfqpoint{2.560452in}{1.183316in}}{\pgfqpoint{2.556062in}{1.193915in}}{\pgfqpoint{2.548249in}{1.201729in}}%
\pgfpathcurveto{\pgfqpoint{2.540435in}{1.209542in}}{\pgfqpoint{2.529836in}{1.213933in}}{\pgfqpoint{2.518786in}{1.213933in}}%
\pgfpathcurveto{\pgfqpoint{2.507736in}{1.213933in}}{\pgfqpoint{2.497137in}{1.209542in}}{\pgfqpoint{2.489323in}{1.201729in}}%
\pgfpathcurveto{\pgfqpoint{2.481509in}{1.193915in}}{\pgfqpoint{2.477119in}{1.183316in}}{\pgfqpoint{2.477119in}{1.172266in}}%
\pgfpathcurveto{\pgfqpoint{2.477119in}{1.161216in}}{\pgfqpoint{2.481509in}{1.150617in}}{\pgfqpoint{2.489323in}{1.142803in}}%
\pgfpathcurveto{\pgfqpoint{2.497137in}{1.134989in}}{\pgfqpoint{2.507736in}{1.130599in}}{\pgfqpoint{2.518786in}{1.130599in}}%
\pgfpathclose%
\pgfusepath{stroke,fill}%
\end{pgfscope}%
\begin{pgfscope}%
\pgfpathrectangle{\pgfqpoint{0.800000in}{0.528000in}}{\pgfqpoint{4.960000in}{3.696000in}}%
\pgfusepath{clip}%
\pgfsetbuttcap%
\pgfsetroundjoin%
\definecolor{currentfill}{rgb}{0.000000,0.000000,0.000000}%
\pgfsetfillcolor{currentfill}%
\pgfsetlinewidth{1.003750pt}%
\definecolor{currentstroke}{rgb}{0.000000,0.000000,0.000000}%
\pgfsetstrokecolor{currentstroke}%
\pgfsetdash{}{0pt}%
\pgfpathmoveto{\pgfqpoint{2.518786in}{1.023801in}}%
\pgfpathcurveto{\pgfqpoint{2.529836in}{1.023801in}}{\pgfqpoint{2.540435in}{1.028191in}}{\pgfqpoint{2.548249in}{1.036004in}}%
\pgfpathcurveto{\pgfqpoint{2.556062in}{1.043818in}}{\pgfqpoint{2.560452in}{1.054417in}}{\pgfqpoint{2.560452in}{1.065467in}}%
\pgfpathcurveto{\pgfqpoint{2.560452in}{1.076517in}}{\pgfqpoint{2.556062in}{1.087116in}}{\pgfqpoint{2.548249in}{1.094930in}}%
\pgfpathcurveto{\pgfqpoint{2.540435in}{1.102744in}}{\pgfqpoint{2.529836in}{1.107134in}}{\pgfqpoint{2.518786in}{1.107134in}}%
\pgfpathcurveto{\pgfqpoint{2.507736in}{1.107134in}}{\pgfqpoint{2.497137in}{1.102744in}}{\pgfqpoint{2.489323in}{1.094930in}}%
\pgfpathcurveto{\pgfqpoint{2.481509in}{1.087116in}}{\pgfqpoint{2.477119in}{1.076517in}}{\pgfqpoint{2.477119in}{1.065467in}}%
\pgfpathcurveto{\pgfqpoint{2.477119in}{1.054417in}}{\pgfqpoint{2.481509in}{1.043818in}}{\pgfqpoint{2.489323in}{1.036004in}}%
\pgfpathcurveto{\pgfqpoint{2.497137in}{1.028191in}}{\pgfqpoint{2.507736in}{1.023801in}}{\pgfqpoint{2.518786in}{1.023801in}}%
\pgfpathclose%
\pgfusepath{stroke,fill}%
\end{pgfscope}%
\begin{pgfscope}%
\pgfpathrectangle{\pgfqpoint{0.800000in}{0.528000in}}{\pgfqpoint{4.960000in}{3.696000in}}%
\pgfusepath{clip}%
\pgfsetbuttcap%
\pgfsetroundjoin%
\definecolor{currentfill}{rgb}{0.000000,0.000000,0.000000}%
\pgfsetfillcolor{currentfill}%
\pgfsetlinewidth{1.003750pt}%
\definecolor{currentstroke}{rgb}{0.000000,0.000000,0.000000}%
\pgfsetstrokecolor{currentstroke}%
\pgfsetdash{}{0pt}%
\pgfpathmoveto{\pgfqpoint{2.518786in}{1.023801in}}%
\pgfpathcurveto{\pgfqpoint{2.529836in}{1.023801in}}{\pgfqpoint{2.540435in}{1.028191in}}{\pgfqpoint{2.548249in}{1.036004in}}%
\pgfpathcurveto{\pgfqpoint{2.556062in}{1.043818in}}{\pgfqpoint{2.560452in}{1.054417in}}{\pgfqpoint{2.560452in}{1.065467in}}%
\pgfpathcurveto{\pgfqpoint{2.560452in}{1.076517in}}{\pgfqpoint{2.556062in}{1.087116in}}{\pgfqpoint{2.548249in}{1.094930in}}%
\pgfpathcurveto{\pgfqpoint{2.540435in}{1.102744in}}{\pgfqpoint{2.529836in}{1.107134in}}{\pgfqpoint{2.518786in}{1.107134in}}%
\pgfpathcurveto{\pgfqpoint{2.507736in}{1.107134in}}{\pgfqpoint{2.497137in}{1.102744in}}{\pgfqpoint{2.489323in}{1.094930in}}%
\pgfpathcurveto{\pgfqpoint{2.481509in}{1.087116in}}{\pgfqpoint{2.477119in}{1.076517in}}{\pgfqpoint{2.477119in}{1.065467in}}%
\pgfpathcurveto{\pgfqpoint{2.477119in}{1.054417in}}{\pgfqpoint{2.481509in}{1.043818in}}{\pgfqpoint{2.489323in}{1.036004in}}%
\pgfpathcurveto{\pgfqpoint{2.497137in}{1.028191in}}{\pgfqpoint{2.507736in}{1.023801in}}{\pgfqpoint{2.518786in}{1.023801in}}%
\pgfpathclose%
\pgfusepath{stroke,fill}%
\end{pgfscope}%
\begin{pgfscope}%
\pgfpathrectangle{\pgfqpoint{0.800000in}{0.528000in}}{\pgfqpoint{4.960000in}{3.696000in}}%
\pgfusepath{clip}%
\pgfsetbuttcap%
\pgfsetroundjoin%
\definecolor{currentfill}{rgb}{0.000000,0.000000,0.000000}%
\pgfsetfillcolor{currentfill}%
\pgfsetlinewidth{1.003750pt}%
\definecolor{currentstroke}{rgb}{0.000000,0.000000,0.000000}%
\pgfsetstrokecolor{currentstroke}%
\pgfsetdash{}{0pt}%
\pgfpathmoveto{\pgfqpoint{2.518786in}{0.965547in}}%
\pgfpathcurveto{\pgfqpoint{2.529836in}{0.965547in}}{\pgfqpoint{2.540435in}{0.969937in}}{\pgfqpoint{2.548249in}{0.977751in}}%
\pgfpathcurveto{\pgfqpoint{2.556062in}{0.985564in}}{\pgfqpoint{2.560452in}{0.996163in}}{\pgfqpoint{2.560452in}{1.007213in}}%
\pgfpathcurveto{\pgfqpoint{2.560452in}{1.018264in}}{\pgfqpoint{2.556062in}{1.028863in}}{\pgfqpoint{2.548249in}{1.036676in}}%
\pgfpathcurveto{\pgfqpoint{2.540435in}{1.044490in}}{\pgfqpoint{2.529836in}{1.048880in}}{\pgfqpoint{2.518786in}{1.048880in}}%
\pgfpathcurveto{\pgfqpoint{2.507736in}{1.048880in}}{\pgfqpoint{2.497137in}{1.044490in}}{\pgfqpoint{2.489323in}{1.036676in}}%
\pgfpathcurveto{\pgfqpoint{2.481509in}{1.028863in}}{\pgfqpoint{2.477119in}{1.018264in}}{\pgfqpoint{2.477119in}{1.007213in}}%
\pgfpathcurveto{\pgfqpoint{2.477119in}{0.996163in}}{\pgfqpoint{2.481509in}{0.985564in}}{\pgfqpoint{2.489323in}{0.977751in}}%
\pgfpathcurveto{\pgfqpoint{2.497137in}{0.969937in}}{\pgfqpoint{2.507736in}{0.965547in}}{\pgfqpoint{2.518786in}{0.965547in}}%
\pgfpathclose%
\pgfusepath{stroke,fill}%
\end{pgfscope}%
\begin{pgfscope}%
\pgfpathrectangle{\pgfqpoint{0.800000in}{0.528000in}}{\pgfqpoint{4.960000in}{3.696000in}}%
\pgfusepath{clip}%
\pgfsetbuttcap%
\pgfsetroundjoin%
\definecolor{currentfill}{rgb}{0.000000,0.000000,0.000000}%
\pgfsetfillcolor{currentfill}%
\pgfsetlinewidth{1.003750pt}%
\definecolor{currentstroke}{rgb}{0.000000,0.000000,0.000000}%
\pgfsetstrokecolor{currentstroke}%
\pgfsetdash{}{0pt}%
\pgfpathmoveto{\pgfqpoint{2.518786in}{0.975256in}}%
\pgfpathcurveto{\pgfqpoint{2.529836in}{0.975256in}}{\pgfqpoint{2.540435in}{0.979646in}}{\pgfqpoint{2.548249in}{0.987460in}}%
\pgfpathcurveto{\pgfqpoint{2.556062in}{0.995273in}}{\pgfqpoint{2.560452in}{1.005872in}}{\pgfqpoint{2.560452in}{1.016922in}}%
\pgfpathcurveto{\pgfqpoint{2.560452in}{1.027973in}}{\pgfqpoint{2.556062in}{1.038572in}}{\pgfqpoint{2.548249in}{1.046385in}}%
\pgfpathcurveto{\pgfqpoint{2.540435in}{1.054199in}}{\pgfqpoint{2.529836in}{1.058589in}}{\pgfqpoint{2.518786in}{1.058589in}}%
\pgfpathcurveto{\pgfqpoint{2.507736in}{1.058589in}}{\pgfqpoint{2.497137in}{1.054199in}}{\pgfqpoint{2.489323in}{1.046385in}}%
\pgfpathcurveto{\pgfqpoint{2.481509in}{1.038572in}}{\pgfqpoint{2.477119in}{1.027973in}}{\pgfqpoint{2.477119in}{1.016922in}}%
\pgfpathcurveto{\pgfqpoint{2.477119in}{1.005872in}}{\pgfqpoint{2.481509in}{0.995273in}}{\pgfqpoint{2.489323in}{0.987460in}}%
\pgfpathcurveto{\pgfqpoint{2.497137in}{0.979646in}}{\pgfqpoint{2.507736in}{0.975256in}}{\pgfqpoint{2.518786in}{0.975256in}}%
\pgfpathclose%
\pgfusepath{stroke,fill}%
\end{pgfscope}%
\begin{pgfscope}%
\pgfpathrectangle{\pgfqpoint{0.800000in}{0.528000in}}{\pgfqpoint{4.960000in}{3.696000in}}%
\pgfusepath{clip}%
\pgfsetbuttcap%
\pgfsetroundjoin%
\definecolor{currentfill}{rgb}{0.000000,0.000000,0.000000}%
\pgfsetfillcolor{currentfill}%
\pgfsetlinewidth{1.003750pt}%
\definecolor{currentstroke}{rgb}{0.000000,0.000000,0.000000}%
\pgfsetstrokecolor{currentstroke}%
\pgfsetdash{}{0pt}%
\pgfpathmoveto{\pgfqpoint{2.518786in}{0.984965in}}%
\pgfpathcurveto{\pgfqpoint{2.529836in}{0.984965in}}{\pgfqpoint{2.540435in}{0.989355in}}{\pgfqpoint{2.548249in}{0.997169in}}%
\pgfpathcurveto{\pgfqpoint{2.556062in}{1.004982in}}{\pgfqpoint{2.560452in}{1.015581in}}{\pgfqpoint{2.560452in}{1.026631in}}%
\pgfpathcurveto{\pgfqpoint{2.560452in}{1.037681in}}{\pgfqpoint{2.556062in}{1.048281in}}{\pgfqpoint{2.548249in}{1.056094in}}%
\pgfpathcurveto{\pgfqpoint{2.540435in}{1.063908in}}{\pgfqpoint{2.529836in}{1.068298in}}{\pgfqpoint{2.518786in}{1.068298in}}%
\pgfpathcurveto{\pgfqpoint{2.507736in}{1.068298in}}{\pgfqpoint{2.497137in}{1.063908in}}{\pgfqpoint{2.489323in}{1.056094in}}%
\pgfpathcurveto{\pgfqpoint{2.481509in}{1.048281in}}{\pgfqpoint{2.477119in}{1.037681in}}{\pgfqpoint{2.477119in}{1.026631in}}%
\pgfpathcurveto{\pgfqpoint{2.477119in}{1.015581in}}{\pgfqpoint{2.481509in}{1.004982in}}{\pgfqpoint{2.489323in}{0.997169in}}%
\pgfpathcurveto{\pgfqpoint{2.497137in}{0.989355in}}{\pgfqpoint{2.507736in}{0.984965in}}{\pgfqpoint{2.518786in}{0.984965in}}%
\pgfpathclose%
\pgfusepath{stroke,fill}%
\end{pgfscope}%
\begin{pgfscope}%
\pgfpathrectangle{\pgfqpoint{0.800000in}{0.528000in}}{\pgfqpoint{4.960000in}{3.696000in}}%
\pgfusepath{clip}%
\pgfsetbuttcap%
\pgfsetroundjoin%
\definecolor{currentfill}{rgb}{0.000000,0.000000,0.000000}%
\pgfsetfillcolor{currentfill}%
\pgfsetlinewidth{1.003750pt}%
\definecolor{currentstroke}{rgb}{0.000000,0.000000,0.000000}%
\pgfsetstrokecolor{currentstroke}%
\pgfsetdash{}{0pt}%
\pgfpathmoveto{\pgfqpoint{2.518786in}{1.052927in}}%
\pgfpathcurveto{\pgfqpoint{2.529836in}{1.052927in}}{\pgfqpoint{2.540435in}{1.057318in}}{\pgfqpoint{2.548249in}{1.065131in}}%
\pgfpathcurveto{\pgfqpoint{2.556062in}{1.072945in}}{\pgfqpoint{2.560452in}{1.083544in}}{\pgfqpoint{2.560452in}{1.094594in}}%
\pgfpathcurveto{\pgfqpoint{2.560452in}{1.105644in}}{\pgfqpoint{2.556062in}{1.116243in}}{\pgfqpoint{2.548249in}{1.124057in}}%
\pgfpathcurveto{\pgfqpoint{2.540435in}{1.131871in}}{\pgfqpoint{2.529836in}{1.136261in}}{\pgfqpoint{2.518786in}{1.136261in}}%
\pgfpathcurveto{\pgfqpoint{2.507736in}{1.136261in}}{\pgfqpoint{2.497137in}{1.131871in}}{\pgfqpoint{2.489323in}{1.124057in}}%
\pgfpathcurveto{\pgfqpoint{2.481509in}{1.116243in}}{\pgfqpoint{2.477119in}{1.105644in}}{\pgfqpoint{2.477119in}{1.094594in}}%
\pgfpathcurveto{\pgfqpoint{2.477119in}{1.083544in}}{\pgfqpoint{2.481509in}{1.072945in}}{\pgfqpoint{2.489323in}{1.065131in}}%
\pgfpathcurveto{\pgfqpoint{2.497137in}{1.057318in}}{\pgfqpoint{2.507736in}{1.052927in}}{\pgfqpoint{2.518786in}{1.052927in}}%
\pgfpathclose%
\pgfusepath{stroke,fill}%
\end{pgfscope}%
\begin{pgfscope}%
\pgfpathrectangle{\pgfqpoint{0.800000in}{0.528000in}}{\pgfqpoint{4.960000in}{3.696000in}}%
\pgfusepath{clip}%
\pgfsetbuttcap%
\pgfsetroundjoin%
\definecolor{currentfill}{rgb}{0.000000,0.000000,0.000000}%
\pgfsetfillcolor{currentfill}%
\pgfsetlinewidth{1.003750pt}%
\definecolor{currentstroke}{rgb}{0.000000,0.000000,0.000000}%
\pgfsetstrokecolor{currentstroke}%
\pgfsetdash{}{0pt}%
\pgfpathmoveto{\pgfqpoint{2.518786in}{1.082054in}}%
\pgfpathcurveto{\pgfqpoint{2.529836in}{1.082054in}}{\pgfqpoint{2.540435in}{1.086445in}}{\pgfqpoint{2.548249in}{1.094258in}}%
\pgfpathcurveto{\pgfqpoint{2.556062in}{1.102072in}}{\pgfqpoint{2.560452in}{1.112671in}}{\pgfqpoint{2.560452in}{1.123721in}}%
\pgfpathcurveto{\pgfqpoint{2.560452in}{1.134771in}}{\pgfqpoint{2.556062in}{1.145370in}}{\pgfqpoint{2.548249in}{1.153184in}}%
\pgfpathcurveto{\pgfqpoint{2.540435in}{1.160997in}}{\pgfqpoint{2.529836in}{1.165388in}}{\pgfqpoint{2.518786in}{1.165388in}}%
\pgfpathcurveto{\pgfqpoint{2.507736in}{1.165388in}}{\pgfqpoint{2.497137in}{1.160997in}}{\pgfqpoint{2.489323in}{1.153184in}}%
\pgfpathcurveto{\pgfqpoint{2.481509in}{1.145370in}}{\pgfqpoint{2.477119in}{1.134771in}}{\pgfqpoint{2.477119in}{1.123721in}}%
\pgfpathcurveto{\pgfqpoint{2.477119in}{1.112671in}}{\pgfqpoint{2.481509in}{1.102072in}}{\pgfqpoint{2.489323in}{1.094258in}}%
\pgfpathcurveto{\pgfqpoint{2.497137in}{1.086445in}}{\pgfqpoint{2.507736in}{1.082054in}}{\pgfqpoint{2.518786in}{1.082054in}}%
\pgfpathclose%
\pgfusepath{stroke,fill}%
\end{pgfscope}%
\begin{pgfscope}%
\pgfpathrectangle{\pgfqpoint{0.800000in}{0.528000in}}{\pgfqpoint{4.960000in}{3.696000in}}%
\pgfusepath{clip}%
\pgfsetbuttcap%
\pgfsetroundjoin%
\definecolor{currentfill}{rgb}{0.000000,0.000000,0.000000}%
\pgfsetfillcolor{currentfill}%
\pgfsetlinewidth{1.003750pt}%
\definecolor{currentstroke}{rgb}{0.000000,0.000000,0.000000}%
\pgfsetstrokecolor{currentstroke}%
\pgfsetdash{}{0pt}%
\pgfpathmoveto{\pgfqpoint{2.518786in}{1.101472in}}%
\pgfpathcurveto{\pgfqpoint{2.529836in}{1.101472in}}{\pgfqpoint{2.540435in}{1.105863in}}{\pgfqpoint{2.548249in}{1.113676in}}%
\pgfpathcurveto{\pgfqpoint{2.556062in}{1.121490in}}{\pgfqpoint{2.560452in}{1.132089in}}{\pgfqpoint{2.560452in}{1.143139in}}%
\pgfpathcurveto{\pgfqpoint{2.560452in}{1.154189in}}{\pgfqpoint{2.556062in}{1.164788in}}{\pgfqpoint{2.548249in}{1.172602in}}%
\pgfpathcurveto{\pgfqpoint{2.540435in}{1.180415in}}{\pgfqpoint{2.529836in}{1.184806in}}{\pgfqpoint{2.518786in}{1.184806in}}%
\pgfpathcurveto{\pgfqpoint{2.507736in}{1.184806in}}{\pgfqpoint{2.497137in}{1.180415in}}{\pgfqpoint{2.489323in}{1.172602in}}%
\pgfpathcurveto{\pgfqpoint{2.481509in}{1.164788in}}{\pgfqpoint{2.477119in}{1.154189in}}{\pgfqpoint{2.477119in}{1.143139in}}%
\pgfpathcurveto{\pgfqpoint{2.477119in}{1.132089in}}{\pgfqpoint{2.481509in}{1.121490in}}{\pgfqpoint{2.489323in}{1.113676in}}%
\pgfpathcurveto{\pgfqpoint{2.497137in}{1.105863in}}{\pgfqpoint{2.507736in}{1.101472in}}{\pgfqpoint{2.518786in}{1.101472in}}%
\pgfpathclose%
\pgfusepath{stroke,fill}%
\end{pgfscope}%
\begin{pgfscope}%
\pgfpathrectangle{\pgfqpoint{0.800000in}{0.528000in}}{\pgfqpoint{4.960000in}{3.696000in}}%
\pgfusepath{clip}%
\pgfsetbuttcap%
\pgfsetroundjoin%
\definecolor{currentfill}{rgb}{0.000000,0.000000,0.000000}%
\pgfsetfillcolor{currentfill}%
\pgfsetlinewidth{1.003750pt}%
\definecolor{currentstroke}{rgb}{0.000000,0.000000,0.000000}%
\pgfsetstrokecolor{currentstroke}%
\pgfsetdash{}{0pt}%
\pgfpathmoveto{\pgfqpoint{2.518786in}{1.062636in}}%
\pgfpathcurveto{\pgfqpoint{2.529836in}{1.062636in}}{\pgfqpoint{2.540435in}{1.067027in}}{\pgfqpoint{2.548249in}{1.074840in}}%
\pgfpathcurveto{\pgfqpoint{2.556062in}{1.082654in}}{\pgfqpoint{2.560452in}{1.093253in}}{\pgfqpoint{2.560452in}{1.104303in}}%
\pgfpathcurveto{\pgfqpoint{2.560452in}{1.115353in}}{\pgfqpoint{2.556062in}{1.125952in}}{\pgfqpoint{2.548249in}{1.133766in}}%
\pgfpathcurveto{\pgfqpoint{2.540435in}{1.141580in}}{\pgfqpoint{2.529836in}{1.145970in}}{\pgfqpoint{2.518786in}{1.145970in}}%
\pgfpathcurveto{\pgfqpoint{2.507736in}{1.145970in}}{\pgfqpoint{2.497137in}{1.141580in}}{\pgfqpoint{2.489323in}{1.133766in}}%
\pgfpathcurveto{\pgfqpoint{2.481509in}{1.125952in}}{\pgfqpoint{2.477119in}{1.115353in}}{\pgfqpoint{2.477119in}{1.104303in}}%
\pgfpathcurveto{\pgfqpoint{2.477119in}{1.093253in}}{\pgfqpoint{2.481509in}{1.082654in}}{\pgfqpoint{2.489323in}{1.074840in}}%
\pgfpathcurveto{\pgfqpoint{2.497137in}{1.067027in}}{\pgfqpoint{2.507736in}{1.062636in}}{\pgfqpoint{2.518786in}{1.062636in}}%
\pgfpathclose%
\pgfusepath{stroke,fill}%
\end{pgfscope}%
\begin{pgfscope}%
\pgfpathrectangle{\pgfqpoint{0.800000in}{0.528000in}}{\pgfqpoint{4.960000in}{3.696000in}}%
\pgfusepath{clip}%
\pgfsetbuttcap%
\pgfsetroundjoin%
\definecolor{currentfill}{rgb}{0.000000,0.000000,0.000000}%
\pgfsetfillcolor{currentfill}%
\pgfsetlinewidth{1.003750pt}%
\definecolor{currentstroke}{rgb}{0.000000,0.000000,0.000000}%
\pgfsetstrokecolor{currentstroke}%
\pgfsetdash{}{0pt}%
\pgfpathmoveto{\pgfqpoint{2.518786in}{1.004383in}}%
\pgfpathcurveto{\pgfqpoint{2.529836in}{1.004383in}}{\pgfqpoint{2.540435in}{1.008773in}}{\pgfqpoint{2.548249in}{1.016587in}}%
\pgfpathcurveto{\pgfqpoint{2.556062in}{1.024400in}}{\pgfqpoint{2.560452in}{1.034999in}}{\pgfqpoint{2.560452in}{1.046049in}}%
\pgfpathcurveto{\pgfqpoint{2.560452in}{1.057099in}}{\pgfqpoint{2.556062in}{1.067698in}}{\pgfqpoint{2.548249in}{1.075512in}}%
\pgfpathcurveto{\pgfqpoint{2.540435in}{1.083326in}}{\pgfqpoint{2.529836in}{1.087716in}}{\pgfqpoint{2.518786in}{1.087716in}}%
\pgfpathcurveto{\pgfqpoint{2.507736in}{1.087716in}}{\pgfqpoint{2.497137in}{1.083326in}}{\pgfqpoint{2.489323in}{1.075512in}}%
\pgfpathcurveto{\pgfqpoint{2.481509in}{1.067698in}}{\pgfqpoint{2.477119in}{1.057099in}}{\pgfqpoint{2.477119in}{1.046049in}}%
\pgfpathcurveto{\pgfqpoint{2.477119in}{1.034999in}}{\pgfqpoint{2.481509in}{1.024400in}}{\pgfqpoint{2.489323in}{1.016587in}}%
\pgfpathcurveto{\pgfqpoint{2.497137in}{1.008773in}}{\pgfqpoint{2.507736in}{1.004383in}}{\pgfqpoint{2.518786in}{1.004383in}}%
\pgfpathclose%
\pgfusepath{stroke,fill}%
\end{pgfscope}%
\begin{pgfscope}%
\pgfpathrectangle{\pgfqpoint{0.800000in}{0.528000in}}{\pgfqpoint{4.960000in}{3.696000in}}%
\pgfusepath{clip}%
\pgfsetbuttcap%
\pgfsetroundjoin%
\definecolor{currentfill}{rgb}{0.000000,0.000000,0.000000}%
\pgfsetfillcolor{currentfill}%
\pgfsetlinewidth{1.003750pt}%
\definecolor{currentstroke}{rgb}{0.000000,0.000000,0.000000}%
\pgfsetstrokecolor{currentstroke}%
\pgfsetdash{}{0pt}%
\pgfpathmoveto{\pgfqpoint{2.518786in}{0.984965in}}%
\pgfpathcurveto{\pgfqpoint{2.529836in}{0.984965in}}{\pgfqpoint{2.540435in}{0.989355in}}{\pgfqpoint{2.548249in}{0.997169in}}%
\pgfpathcurveto{\pgfqpoint{2.556062in}{1.004982in}}{\pgfqpoint{2.560452in}{1.015581in}}{\pgfqpoint{2.560452in}{1.026631in}}%
\pgfpathcurveto{\pgfqpoint{2.560452in}{1.037681in}}{\pgfqpoint{2.556062in}{1.048281in}}{\pgfqpoint{2.548249in}{1.056094in}}%
\pgfpathcurveto{\pgfqpoint{2.540435in}{1.063908in}}{\pgfqpoint{2.529836in}{1.068298in}}{\pgfqpoint{2.518786in}{1.068298in}}%
\pgfpathcurveto{\pgfqpoint{2.507736in}{1.068298in}}{\pgfqpoint{2.497137in}{1.063908in}}{\pgfqpoint{2.489323in}{1.056094in}}%
\pgfpathcurveto{\pgfqpoint{2.481509in}{1.048281in}}{\pgfqpoint{2.477119in}{1.037681in}}{\pgfqpoint{2.477119in}{1.026631in}}%
\pgfpathcurveto{\pgfqpoint{2.477119in}{1.015581in}}{\pgfqpoint{2.481509in}{1.004982in}}{\pgfqpoint{2.489323in}{0.997169in}}%
\pgfpathcurveto{\pgfqpoint{2.497137in}{0.989355in}}{\pgfqpoint{2.507736in}{0.984965in}}{\pgfqpoint{2.518786in}{0.984965in}}%
\pgfpathclose%
\pgfusepath{stroke,fill}%
\end{pgfscope}%
\begin{pgfscope}%
\pgfpathrectangle{\pgfqpoint{0.800000in}{0.528000in}}{\pgfqpoint{4.960000in}{3.696000in}}%
\pgfusepath{clip}%
\pgfsetbuttcap%
\pgfsetroundjoin%
\definecolor{currentfill}{rgb}{0.000000,0.000000,0.000000}%
\pgfsetfillcolor{currentfill}%
\pgfsetlinewidth{1.003750pt}%
\definecolor{currentstroke}{rgb}{0.000000,0.000000,0.000000}%
\pgfsetstrokecolor{currentstroke}%
\pgfsetdash{}{0pt}%
\pgfpathmoveto{\pgfqpoint{2.518786in}{1.033510in}}%
\pgfpathcurveto{\pgfqpoint{2.529836in}{1.033510in}}{\pgfqpoint{2.540435in}{1.037900in}}{\pgfqpoint{2.548249in}{1.045713in}}%
\pgfpathcurveto{\pgfqpoint{2.556062in}{1.053527in}}{\pgfqpoint{2.560452in}{1.064126in}}{\pgfqpoint{2.560452in}{1.075176in}}%
\pgfpathcurveto{\pgfqpoint{2.560452in}{1.086226in}}{\pgfqpoint{2.556062in}{1.096825in}}{\pgfqpoint{2.548249in}{1.104639in}}%
\pgfpathcurveto{\pgfqpoint{2.540435in}{1.112453in}}{\pgfqpoint{2.529836in}{1.116843in}}{\pgfqpoint{2.518786in}{1.116843in}}%
\pgfpathcurveto{\pgfqpoint{2.507736in}{1.116843in}}{\pgfqpoint{2.497137in}{1.112453in}}{\pgfqpoint{2.489323in}{1.104639in}}%
\pgfpathcurveto{\pgfqpoint{2.481509in}{1.096825in}}{\pgfqpoint{2.477119in}{1.086226in}}{\pgfqpoint{2.477119in}{1.075176in}}%
\pgfpathcurveto{\pgfqpoint{2.477119in}{1.064126in}}{\pgfqpoint{2.481509in}{1.053527in}}{\pgfqpoint{2.489323in}{1.045713in}}%
\pgfpathcurveto{\pgfqpoint{2.497137in}{1.037900in}}{\pgfqpoint{2.507736in}{1.033510in}}{\pgfqpoint{2.518786in}{1.033510in}}%
\pgfpathclose%
\pgfusepath{stroke,fill}%
\end{pgfscope}%
\begin{pgfscope}%
\pgfpathrectangle{\pgfqpoint{0.800000in}{0.528000in}}{\pgfqpoint{4.960000in}{3.696000in}}%
\pgfusepath{clip}%
\pgfsetbuttcap%
\pgfsetroundjoin%
\definecolor{currentfill}{rgb}{0.000000,0.000000,0.000000}%
\pgfsetfillcolor{currentfill}%
\pgfsetlinewidth{1.003750pt}%
\definecolor{currentstroke}{rgb}{0.000000,0.000000,0.000000}%
\pgfsetstrokecolor{currentstroke}%
\pgfsetdash{}{0pt}%
\pgfpathmoveto{\pgfqpoint{2.518786in}{0.975256in}}%
\pgfpathcurveto{\pgfqpoint{2.529836in}{0.975256in}}{\pgfqpoint{2.540435in}{0.979646in}}{\pgfqpoint{2.548249in}{0.987460in}}%
\pgfpathcurveto{\pgfqpoint{2.556062in}{0.995273in}}{\pgfqpoint{2.560452in}{1.005872in}}{\pgfqpoint{2.560452in}{1.016922in}}%
\pgfpathcurveto{\pgfqpoint{2.560452in}{1.027973in}}{\pgfqpoint{2.556062in}{1.038572in}}{\pgfqpoint{2.548249in}{1.046385in}}%
\pgfpathcurveto{\pgfqpoint{2.540435in}{1.054199in}}{\pgfqpoint{2.529836in}{1.058589in}}{\pgfqpoint{2.518786in}{1.058589in}}%
\pgfpathcurveto{\pgfqpoint{2.507736in}{1.058589in}}{\pgfqpoint{2.497137in}{1.054199in}}{\pgfqpoint{2.489323in}{1.046385in}}%
\pgfpathcurveto{\pgfqpoint{2.481509in}{1.038572in}}{\pgfqpoint{2.477119in}{1.027973in}}{\pgfqpoint{2.477119in}{1.016922in}}%
\pgfpathcurveto{\pgfqpoint{2.477119in}{1.005872in}}{\pgfqpoint{2.481509in}{0.995273in}}{\pgfqpoint{2.489323in}{0.987460in}}%
\pgfpathcurveto{\pgfqpoint{2.497137in}{0.979646in}}{\pgfqpoint{2.507736in}{0.975256in}}{\pgfqpoint{2.518786in}{0.975256in}}%
\pgfpathclose%
\pgfusepath{stroke,fill}%
\end{pgfscope}%
\begin{pgfscope}%
\pgfpathrectangle{\pgfqpoint{0.800000in}{0.528000in}}{\pgfqpoint{4.960000in}{3.696000in}}%
\pgfusepath{clip}%
\pgfsetbuttcap%
\pgfsetroundjoin%
\definecolor{currentfill}{rgb}{0.000000,0.000000,0.000000}%
\pgfsetfillcolor{currentfill}%
\pgfsetlinewidth{1.003750pt}%
\definecolor{currentstroke}{rgb}{0.000000,0.000000,0.000000}%
\pgfsetstrokecolor{currentstroke}%
\pgfsetdash{}{0pt}%
\pgfpathmoveto{\pgfqpoint{2.518786in}{1.033510in}}%
\pgfpathcurveto{\pgfqpoint{2.529836in}{1.033510in}}{\pgfqpoint{2.540435in}{1.037900in}}{\pgfqpoint{2.548249in}{1.045713in}}%
\pgfpathcurveto{\pgfqpoint{2.556062in}{1.053527in}}{\pgfqpoint{2.560452in}{1.064126in}}{\pgfqpoint{2.560452in}{1.075176in}}%
\pgfpathcurveto{\pgfqpoint{2.560452in}{1.086226in}}{\pgfqpoint{2.556062in}{1.096825in}}{\pgfqpoint{2.548249in}{1.104639in}}%
\pgfpathcurveto{\pgfqpoint{2.540435in}{1.112453in}}{\pgfqpoint{2.529836in}{1.116843in}}{\pgfqpoint{2.518786in}{1.116843in}}%
\pgfpathcurveto{\pgfqpoint{2.507736in}{1.116843in}}{\pgfqpoint{2.497137in}{1.112453in}}{\pgfqpoint{2.489323in}{1.104639in}}%
\pgfpathcurveto{\pgfqpoint{2.481509in}{1.096825in}}{\pgfqpoint{2.477119in}{1.086226in}}{\pgfqpoint{2.477119in}{1.075176in}}%
\pgfpathcurveto{\pgfqpoint{2.477119in}{1.064126in}}{\pgfqpoint{2.481509in}{1.053527in}}{\pgfqpoint{2.489323in}{1.045713in}}%
\pgfpathcurveto{\pgfqpoint{2.497137in}{1.037900in}}{\pgfqpoint{2.507736in}{1.033510in}}{\pgfqpoint{2.518786in}{1.033510in}}%
\pgfpathclose%
\pgfusepath{stroke,fill}%
\end{pgfscope}%
\begin{pgfscope}%
\pgfpathrectangle{\pgfqpoint{0.800000in}{0.528000in}}{\pgfqpoint{4.960000in}{3.696000in}}%
\pgfusepath{clip}%
\pgfsetbuttcap%
\pgfsetroundjoin%
\definecolor{currentfill}{rgb}{0.000000,0.000000,0.000000}%
\pgfsetfillcolor{currentfill}%
\pgfsetlinewidth{1.003750pt}%
\definecolor{currentstroke}{rgb}{0.000000,0.000000,0.000000}%
\pgfsetstrokecolor{currentstroke}%
\pgfsetdash{}{0pt}%
\pgfpathmoveto{\pgfqpoint{2.518786in}{1.004383in}}%
\pgfpathcurveto{\pgfqpoint{2.529836in}{1.004383in}}{\pgfqpoint{2.540435in}{1.008773in}}{\pgfqpoint{2.548249in}{1.016587in}}%
\pgfpathcurveto{\pgfqpoint{2.556062in}{1.024400in}}{\pgfqpoint{2.560452in}{1.034999in}}{\pgfqpoint{2.560452in}{1.046049in}}%
\pgfpathcurveto{\pgfqpoint{2.560452in}{1.057099in}}{\pgfqpoint{2.556062in}{1.067698in}}{\pgfqpoint{2.548249in}{1.075512in}}%
\pgfpathcurveto{\pgfqpoint{2.540435in}{1.083326in}}{\pgfqpoint{2.529836in}{1.087716in}}{\pgfqpoint{2.518786in}{1.087716in}}%
\pgfpathcurveto{\pgfqpoint{2.507736in}{1.087716in}}{\pgfqpoint{2.497137in}{1.083326in}}{\pgfqpoint{2.489323in}{1.075512in}}%
\pgfpathcurveto{\pgfqpoint{2.481509in}{1.067698in}}{\pgfqpoint{2.477119in}{1.057099in}}{\pgfqpoint{2.477119in}{1.046049in}}%
\pgfpathcurveto{\pgfqpoint{2.477119in}{1.034999in}}{\pgfqpoint{2.481509in}{1.024400in}}{\pgfqpoint{2.489323in}{1.016587in}}%
\pgfpathcurveto{\pgfqpoint{2.497137in}{1.008773in}}{\pgfqpoint{2.507736in}{1.004383in}}{\pgfqpoint{2.518786in}{1.004383in}}%
\pgfpathclose%
\pgfusepath{stroke,fill}%
\end{pgfscope}%
\begin{pgfscope}%
\pgfpathrectangle{\pgfqpoint{0.800000in}{0.528000in}}{\pgfqpoint{4.960000in}{3.696000in}}%
\pgfusepath{clip}%
\pgfsetbuttcap%
\pgfsetroundjoin%
\definecolor{currentfill}{rgb}{0.000000,0.000000,0.000000}%
\pgfsetfillcolor{currentfill}%
\pgfsetlinewidth{1.003750pt}%
\definecolor{currentstroke}{rgb}{0.000000,0.000000,0.000000}%
\pgfsetstrokecolor{currentstroke}%
\pgfsetdash{}{0pt}%
\pgfpathmoveto{\pgfqpoint{2.518786in}{1.101472in}}%
\pgfpathcurveto{\pgfqpoint{2.529836in}{1.101472in}}{\pgfqpoint{2.540435in}{1.105863in}}{\pgfqpoint{2.548249in}{1.113676in}}%
\pgfpathcurveto{\pgfqpoint{2.556062in}{1.121490in}}{\pgfqpoint{2.560452in}{1.132089in}}{\pgfqpoint{2.560452in}{1.143139in}}%
\pgfpathcurveto{\pgfqpoint{2.560452in}{1.154189in}}{\pgfqpoint{2.556062in}{1.164788in}}{\pgfqpoint{2.548249in}{1.172602in}}%
\pgfpathcurveto{\pgfqpoint{2.540435in}{1.180415in}}{\pgfqpoint{2.529836in}{1.184806in}}{\pgfqpoint{2.518786in}{1.184806in}}%
\pgfpathcurveto{\pgfqpoint{2.507736in}{1.184806in}}{\pgfqpoint{2.497137in}{1.180415in}}{\pgfqpoint{2.489323in}{1.172602in}}%
\pgfpathcurveto{\pgfqpoint{2.481509in}{1.164788in}}{\pgfqpoint{2.477119in}{1.154189in}}{\pgfqpoint{2.477119in}{1.143139in}}%
\pgfpathcurveto{\pgfqpoint{2.477119in}{1.132089in}}{\pgfqpoint{2.481509in}{1.121490in}}{\pgfqpoint{2.489323in}{1.113676in}}%
\pgfpathcurveto{\pgfqpoint{2.497137in}{1.105863in}}{\pgfqpoint{2.507736in}{1.101472in}}{\pgfqpoint{2.518786in}{1.101472in}}%
\pgfpathclose%
\pgfusepath{stroke,fill}%
\end{pgfscope}%
\begin{pgfscope}%
\pgfpathrectangle{\pgfqpoint{0.800000in}{0.528000in}}{\pgfqpoint{4.960000in}{3.696000in}}%
\pgfusepath{clip}%
\pgfsetbuttcap%
\pgfsetroundjoin%
\definecolor{currentfill}{rgb}{0.000000,0.000000,0.000000}%
\pgfsetfillcolor{currentfill}%
\pgfsetlinewidth{1.003750pt}%
\definecolor{currentstroke}{rgb}{0.000000,0.000000,0.000000}%
\pgfsetstrokecolor{currentstroke}%
\pgfsetdash{}{0pt}%
\pgfpathmoveto{\pgfqpoint{2.518786in}{0.965547in}}%
\pgfpathcurveto{\pgfqpoint{2.529836in}{0.965547in}}{\pgfqpoint{2.540435in}{0.969937in}}{\pgfqpoint{2.548249in}{0.977751in}}%
\pgfpathcurveto{\pgfqpoint{2.556062in}{0.985564in}}{\pgfqpoint{2.560452in}{0.996163in}}{\pgfqpoint{2.560452in}{1.007213in}}%
\pgfpathcurveto{\pgfqpoint{2.560452in}{1.018264in}}{\pgfqpoint{2.556062in}{1.028863in}}{\pgfqpoint{2.548249in}{1.036676in}}%
\pgfpathcurveto{\pgfqpoint{2.540435in}{1.044490in}}{\pgfqpoint{2.529836in}{1.048880in}}{\pgfqpoint{2.518786in}{1.048880in}}%
\pgfpathcurveto{\pgfqpoint{2.507736in}{1.048880in}}{\pgfqpoint{2.497137in}{1.044490in}}{\pgfqpoint{2.489323in}{1.036676in}}%
\pgfpathcurveto{\pgfqpoint{2.481509in}{1.028863in}}{\pgfqpoint{2.477119in}{1.018264in}}{\pgfqpoint{2.477119in}{1.007213in}}%
\pgfpathcurveto{\pgfqpoint{2.477119in}{0.996163in}}{\pgfqpoint{2.481509in}{0.985564in}}{\pgfqpoint{2.489323in}{0.977751in}}%
\pgfpathcurveto{\pgfqpoint{2.497137in}{0.969937in}}{\pgfqpoint{2.507736in}{0.965547in}}{\pgfqpoint{2.518786in}{0.965547in}}%
\pgfpathclose%
\pgfusepath{stroke,fill}%
\end{pgfscope}%
\begin{pgfscope}%
\pgfpathrectangle{\pgfqpoint{0.800000in}{0.528000in}}{\pgfqpoint{4.960000in}{3.696000in}}%
\pgfusepath{clip}%
\pgfsetbuttcap%
\pgfsetroundjoin%
\definecolor{currentfill}{rgb}{0.000000,0.000000,0.000000}%
\pgfsetfillcolor{currentfill}%
\pgfsetlinewidth{1.003750pt}%
\definecolor{currentstroke}{rgb}{0.000000,0.000000,0.000000}%
\pgfsetstrokecolor{currentstroke}%
\pgfsetdash{}{0pt}%
\pgfpathmoveto{\pgfqpoint{2.518786in}{1.043219in}}%
\pgfpathcurveto{\pgfqpoint{2.529836in}{1.043219in}}{\pgfqpoint{2.540435in}{1.047609in}}{\pgfqpoint{2.548249in}{1.055422in}}%
\pgfpathcurveto{\pgfqpoint{2.556062in}{1.063236in}}{\pgfqpoint{2.560452in}{1.073835in}}{\pgfqpoint{2.560452in}{1.084885in}}%
\pgfpathcurveto{\pgfqpoint{2.560452in}{1.095935in}}{\pgfqpoint{2.556062in}{1.106534in}}{\pgfqpoint{2.548249in}{1.114348in}}%
\pgfpathcurveto{\pgfqpoint{2.540435in}{1.122162in}}{\pgfqpoint{2.529836in}{1.126552in}}{\pgfqpoint{2.518786in}{1.126552in}}%
\pgfpathcurveto{\pgfqpoint{2.507736in}{1.126552in}}{\pgfqpoint{2.497137in}{1.122162in}}{\pgfqpoint{2.489323in}{1.114348in}}%
\pgfpathcurveto{\pgfqpoint{2.481509in}{1.106534in}}{\pgfqpoint{2.477119in}{1.095935in}}{\pgfqpoint{2.477119in}{1.084885in}}%
\pgfpathcurveto{\pgfqpoint{2.477119in}{1.073835in}}{\pgfqpoint{2.481509in}{1.063236in}}{\pgfqpoint{2.489323in}{1.055422in}}%
\pgfpathcurveto{\pgfqpoint{2.497137in}{1.047609in}}{\pgfqpoint{2.507736in}{1.043219in}}{\pgfqpoint{2.518786in}{1.043219in}}%
\pgfpathclose%
\pgfusepath{stroke,fill}%
\end{pgfscope}%
\begin{pgfscope}%
\pgfpathrectangle{\pgfqpoint{0.800000in}{0.528000in}}{\pgfqpoint{4.960000in}{3.696000in}}%
\pgfusepath{clip}%
\pgfsetbuttcap%
\pgfsetroundjoin%
\definecolor{currentfill}{rgb}{0.000000,0.000000,0.000000}%
\pgfsetfillcolor{currentfill}%
\pgfsetlinewidth{1.003750pt}%
\definecolor{currentstroke}{rgb}{0.000000,0.000000,0.000000}%
\pgfsetstrokecolor{currentstroke}%
\pgfsetdash{}{0pt}%
\pgfpathmoveto{\pgfqpoint{2.518786in}{1.004383in}}%
\pgfpathcurveto{\pgfqpoint{2.529836in}{1.004383in}}{\pgfqpoint{2.540435in}{1.008773in}}{\pgfqpoint{2.548249in}{1.016587in}}%
\pgfpathcurveto{\pgfqpoint{2.556062in}{1.024400in}}{\pgfqpoint{2.560452in}{1.034999in}}{\pgfqpoint{2.560452in}{1.046049in}}%
\pgfpathcurveto{\pgfqpoint{2.560452in}{1.057099in}}{\pgfqpoint{2.556062in}{1.067698in}}{\pgfqpoint{2.548249in}{1.075512in}}%
\pgfpathcurveto{\pgfqpoint{2.540435in}{1.083326in}}{\pgfqpoint{2.529836in}{1.087716in}}{\pgfqpoint{2.518786in}{1.087716in}}%
\pgfpathcurveto{\pgfqpoint{2.507736in}{1.087716in}}{\pgfqpoint{2.497137in}{1.083326in}}{\pgfqpoint{2.489323in}{1.075512in}}%
\pgfpathcurveto{\pgfqpoint{2.481509in}{1.067698in}}{\pgfqpoint{2.477119in}{1.057099in}}{\pgfqpoint{2.477119in}{1.046049in}}%
\pgfpathcurveto{\pgfqpoint{2.477119in}{1.034999in}}{\pgfqpoint{2.481509in}{1.024400in}}{\pgfqpoint{2.489323in}{1.016587in}}%
\pgfpathcurveto{\pgfqpoint{2.497137in}{1.008773in}}{\pgfqpoint{2.507736in}{1.004383in}}{\pgfqpoint{2.518786in}{1.004383in}}%
\pgfpathclose%
\pgfusepath{stroke,fill}%
\end{pgfscope}%
\begin{pgfscope}%
\pgfpathrectangle{\pgfqpoint{0.800000in}{0.528000in}}{\pgfqpoint{4.960000in}{3.696000in}}%
\pgfusepath{clip}%
\pgfsetbuttcap%
\pgfsetroundjoin%
\definecolor{currentfill}{rgb}{0.000000,0.000000,0.000000}%
\pgfsetfillcolor{currentfill}%
\pgfsetlinewidth{1.003750pt}%
\definecolor{currentstroke}{rgb}{0.000000,0.000000,0.000000}%
\pgfsetstrokecolor{currentstroke}%
\pgfsetdash{}{0pt}%
\pgfpathmoveto{\pgfqpoint{2.518786in}{0.917002in}}%
\pgfpathcurveto{\pgfqpoint{2.529836in}{0.917002in}}{\pgfqpoint{2.540435in}{0.921392in}}{\pgfqpoint{2.548249in}{0.929206in}}%
\pgfpathcurveto{\pgfqpoint{2.556062in}{0.937019in}}{\pgfqpoint{2.560452in}{0.947618in}}{\pgfqpoint{2.560452in}{0.958669in}}%
\pgfpathcurveto{\pgfqpoint{2.560452in}{0.969719in}}{\pgfqpoint{2.556062in}{0.980318in}}{\pgfqpoint{2.548249in}{0.988131in}}%
\pgfpathcurveto{\pgfqpoint{2.540435in}{0.995945in}}{\pgfqpoint{2.529836in}{1.000335in}}{\pgfqpoint{2.518786in}{1.000335in}}%
\pgfpathcurveto{\pgfqpoint{2.507736in}{1.000335in}}{\pgfqpoint{2.497137in}{0.995945in}}{\pgfqpoint{2.489323in}{0.988131in}}%
\pgfpathcurveto{\pgfqpoint{2.481509in}{0.980318in}}{\pgfqpoint{2.477119in}{0.969719in}}{\pgfqpoint{2.477119in}{0.958669in}}%
\pgfpathcurveto{\pgfqpoint{2.477119in}{0.947618in}}{\pgfqpoint{2.481509in}{0.937019in}}{\pgfqpoint{2.489323in}{0.929206in}}%
\pgfpathcurveto{\pgfqpoint{2.497137in}{0.921392in}}{\pgfqpoint{2.507736in}{0.917002in}}{\pgfqpoint{2.518786in}{0.917002in}}%
\pgfpathclose%
\pgfusepath{stroke,fill}%
\end{pgfscope}%
\begin{pgfscope}%
\pgfpathrectangle{\pgfqpoint{0.800000in}{0.528000in}}{\pgfqpoint{4.960000in}{3.696000in}}%
\pgfusepath{clip}%
\pgfsetbuttcap%
\pgfsetroundjoin%
\definecolor{currentfill}{rgb}{0.000000,0.000000,0.000000}%
\pgfsetfillcolor{currentfill}%
\pgfsetlinewidth{1.003750pt}%
\definecolor{currentstroke}{rgb}{0.000000,0.000000,0.000000}%
\pgfsetstrokecolor{currentstroke}%
\pgfsetdash{}{0pt}%
\pgfpathmoveto{\pgfqpoint{2.518786in}{1.208271in}}%
\pgfpathcurveto{\pgfqpoint{2.529836in}{1.208271in}}{\pgfqpoint{2.540435in}{1.212661in}}{\pgfqpoint{2.548249in}{1.220475in}}%
\pgfpathcurveto{\pgfqpoint{2.556062in}{1.228288in}}{\pgfqpoint{2.560452in}{1.238888in}}{\pgfqpoint{2.560452in}{1.249938in}}%
\pgfpathcurveto{\pgfqpoint{2.560452in}{1.260988in}}{\pgfqpoint{2.556062in}{1.271587in}}{\pgfqpoint{2.548249in}{1.279400in}}%
\pgfpathcurveto{\pgfqpoint{2.540435in}{1.287214in}}{\pgfqpoint{2.529836in}{1.291604in}}{\pgfqpoint{2.518786in}{1.291604in}}%
\pgfpathcurveto{\pgfqpoint{2.507736in}{1.291604in}}{\pgfqpoint{2.497137in}{1.287214in}}{\pgfqpoint{2.489323in}{1.279400in}}%
\pgfpathcurveto{\pgfqpoint{2.481509in}{1.271587in}}{\pgfqpoint{2.477119in}{1.260988in}}{\pgfqpoint{2.477119in}{1.249938in}}%
\pgfpathcurveto{\pgfqpoint{2.477119in}{1.238888in}}{\pgfqpoint{2.481509in}{1.228288in}}{\pgfqpoint{2.489323in}{1.220475in}}%
\pgfpathcurveto{\pgfqpoint{2.497137in}{1.212661in}}{\pgfqpoint{2.507736in}{1.208271in}}{\pgfqpoint{2.518786in}{1.208271in}}%
\pgfpathclose%
\pgfusepath{stroke,fill}%
\end{pgfscope}%
\begin{pgfscope}%
\pgfpathrectangle{\pgfqpoint{0.800000in}{0.528000in}}{\pgfqpoint{4.960000in}{3.696000in}}%
\pgfusepath{clip}%
\pgfsetbuttcap%
\pgfsetroundjoin%
\definecolor{currentfill}{rgb}{0.000000,0.000000,0.000000}%
\pgfsetfillcolor{currentfill}%
\pgfsetlinewidth{1.003750pt}%
\definecolor{currentstroke}{rgb}{0.000000,0.000000,0.000000}%
\pgfsetstrokecolor{currentstroke}%
\pgfsetdash{}{0pt}%
\pgfpathmoveto{\pgfqpoint{2.518786in}{1.120890in}}%
\pgfpathcurveto{\pgfqpoint{2.529836in}{1.120890in}}{\pgfqpoint{2.540435in}{1.125281in}}{\pgfqpoint{2.548249in}{1.133094in}}%
\pgfpathcurveto{\pgfqpoint{2.556062in}{1.140908in}}{\pgfqpoint{2.560452in}{1.151507in}}{\pgfqpoint{2.560452in}{1.162557in}}%
\pgfpathcurveto{\pgfqpoint{2.560452in}{1.173607in}}{\pgfqpoint{2.556062in}{1.184206in}}{\pgfqpoint{2.548249in}{1.192020in}}%
\pgfpathcurveto{\pgfqpoint{2.540435in}{1.199833in}}{\pgfqpoint{2.529836in}{1.204224in}}{\pgfqpoint{2.518786in}{1.204224in}}%
\pgfpathcurveto{\pgfqpoint{2.507736in}{1.204224in}}{\pgfqpoint{2.497137in}{1.199833in}}{\pgfqpoint{2.489323in}{1.192020in}}%
\pgfpathcurveto{\pgfqpoint{2.481509in}{1.184206in}}{\pgfqpoint{2.477119in}{1.173607in}}{\pgfqpoint{2.477119in}{1.162557in}}%
\pgfpathcurveto{\pgfqpoint{2.477119in}{1.151507in}}{\pgfqpoint{2.481509in}{1.140908in}}{\pgfqpoint{2.489323in}{1.133094in}}%
\pgfpathcurveto{\pgfqpoint{2.497137in}{1.125281in}}{\pgfqpoint{2.507736in}{1.120890in}}{\pgfqpoint{2.518786in}{1.120890in}}%
\pgfpathclose%
\pgfusepath{stroke,fill}%
\end{pgfscope}%
\begin{pgfscope}%
\pgfpathrectangle{\pgfqpoint{0.800000in}{0.528000in}}{\pgfqpoint{4.960000in}{3.696000in}}%
\pgfusepath{clip}%
\pgfsetbuttcap%
\pgfsetroundjoin%
\definecolor{currentfill}{rgb}{0.000000,0.000000,0.000000}%
\pgfsetfillcolor{currentfill}%
\pgfsetlinewidth{1.003750pt}%
\definecolor{currentstroke}{rgb}{0.000000,0.000000,0.000000}%
\pgfsetstrokecolor{currentstroke}%
\pgfsetdash{}{0pt}%
\pgfpathmoveto{\pgfqpoint{2.518786in}{0.994674in}}%
\pgfpathcurveto{\pgfqpoint{2.529836in}{0.994674in}}{\pgfqpoint{2.540435in}{0.999064in}}{\pgfqpoint{2.548249in}{1.006878in}}%
\pgfpathcurveto{\pgfqpoint{2.556062in}{1.014691in}}{\pgfqpoint{2.560452in}{1.025290in}}{\pgfqpoint{2.560452in}{1.036340in}}%
\pgfpathcurveto{\pgfqpoint{2.560452in}{1.047390in}}{\pgfqpoint{2.556062in}{1.057989in}}{\pgfqpoint{2.548249in}{1.065803in}}%
\pgfpathcurveto{\pgfqpoint{2.540435in}{1.073617in}}{\pgfqpoint{2.529836in}{1.078007in}}{\pgfqpoint{2.518786in}{1.078007in}}%
\pgfpathcurveto{\pgfqpoint{2.507736in}{1.078007in}}{\pgfqpoint{2.497137in}{1.073617in}}{\pgfqpoint{2.489323in}{1.065803in}}%
\pgfpathcurveto{\pgfqpoint{2.481509in}{1.057989in}}{\pgfqpoint{2.477119in}{1.047390in}}{\pgfqpoint{2.477119in}{1.036340in}}%
\pgfpathcurveto{\pgfqpoint{2.477119in}{1.025290in}}{\pgfqpoint{2.481509in}{1.014691in}}{\pgfqpoint{2.489323in}{1.006878in}}%
\pgfpathcurveto{\pgfqpoint{2.497137in}{0.999064in}}{\pgfqpoint{2.507736in}{0.994674in}}{\pgfqpoint{2.518786in}{0.994674in}}%
\pgfpathclose%
\pgfusepath{stroke,fill}%
\end{pgfscope}%
\begin{pgfscope}%
\pgfpathrectangle{\pgfqpoint{0.800000in}{0.528000in}}{\pgfqpoint{4.960000in}{3.696000in}}%
\pgfusepath{clip}%
\pgfsetbuttcap%
\pgfsetroundjoin%
\definecolor{currentfill}{rgb}{0.000000,0.000000,0.000000}%
\pgfsetfillcolor{currentfill}%
\pgfsetlinewidth{1.003750pt}%
\definecolor{currentstroke}{rgb}{0.000000,0.000000,0.000000}%
\pgfsetstrokecolor{currentstroke}%
\pgfsetdash{}{0pt}%
\pgfpathmoveto{\pgfqpoint{2.518786in}{1.033510in}}%
\pgfpathcurveto{\pgfqpoint{2.529836in}{1.033510in}}{\pgfqpoint{2.540435in}{1.037900in}}{\pgfqpoint{2.548249in}{1.045713in}}%
\pgfpathcurveto{\pgfqpoint{2.556062in}{1.053527in}}{\pgfqpoint{2.560452in}{1.064126in}}{\pgfqpoint{2.560452in}{1.075176in}}%
\pgfpathcurveto{\pgfqpoint{2.560452in}{1.086226in}}{\pgfqpoint{2.556062in}{1.096825in}}{\pgfqpoint{2.548249in}{1.104639in}}%
\pgfpathcurveto{\pgfqpoint{2.540435in}{1.112453in}}{\pgfqpoint{2.529836in}{1.116843in}}{\pgfqpoint{2.518786in}{1.116843in}}%
\pgfpathcurveto{\pgfqpoint{2.507736in}{1.116843in}}{\pgfqpoint{2.497137in}{1.112453in}}{\pgfqpoint{2.489323in}{1.104639in}}%
\pgfpathcurveto{\pgfqpoint{2.481509in}{1.096825in}}{\pgfqpoint{2.477119in}{1.086226in}}{\pgfqpoint{2.477119in}{1.075176in}}%
\pgfpathcurveto{\pgfqpoint{2.477119in}{1.064126in}}{\pgfqpoint{2.481509in}{1.053527in}}{\pgfqpoint{2.489323in}{1.045713in}}%
\pgfpathcurveto{\pgfqpoint{2.497137in}{1.037900in}}{\pgfqpoint{2.507736in}{1.033510in}}{\pgfqpoint{2.518786in}{1.033510in}}%
\pgfpathclose%
\pgfusepath{stroke,fill}%
\end{pgfscope}%
\begin{pgfscope}%
\pgfpathrectangle{\pgfqpoint{0.800000in}{0.528000in}}{\pgfqpoint{4.960000in}{3.696000in}}%
\pgfusepath{clip}%
\pgfsetbuttcap%
\pgfsetroundjoin%
\definecolor{currentfill}{rgb}{0.000000,0.000000,0.000000}%
\pgfsetfillcolor{currentfill}%
\pgfsetlinewidth{1.003750pt}%
\definecolor{currentstroke}{rgb}{0.000000,0.000000,0.000000}%
\pgfsetstrokecolor{currentstroke}%
\pgfsetdash{}{0pt}%
\pgfpathmoveto{\pgfqpoint{2.518786in}{0.984965in}}%
\pgfpathcurveto{\pgfqpoint{2.529836in}{0.984965in}}{\pgfqpoint{2.540435in}{0.989355in}}{\pgfqpoint{2.548249in}{0.997169in}}%
\pgfpathcurveto{\pgfqpoint{2.556062in}{1.004982in}}{\pgfqpoint{2.560452in}{1.015581in}}{\pgfqpoint{2.560452in}{1.026631in}}%
\pgfpathcurveto{\pgfqpoint{2.560452in}{1.037681in}}{\pgfqpoint{2.556062in}{1.048281in}}{\pgfqpoint{2.548249in}{1.056094in}}%
\pgfpathcurveto{\pgfqpoint{2.540435in}{1.063908in}}{\pgfqpoint{2.529836in}{1.068298in}}{\pgfqpoint{2.518786in}{1.068298in}}%
\pgfpathcurveto{\pgfqpoint{2.507736in}{1.068298in}}{\pgfqpoint{2.497137in}{1.063908in}}{\pgfqpoint{2.489323in}{1.056094in}}%
\pgfpathcurveto{\pgfqpoint{2.481509in}{1.048281in}}{\pgfqpoint{2.477119in}{1.037681in}}{\pgfqpoint{2.477119in}{1.026631in}}%
\pgfpathcurveto{\pgfqpoint{2.477119in}{1.015581in}}{\pgfqpoint{2.481509in}{1.004982in}}{\pgfqpoint{2.489323in}{0.997169in}}%
\pgfpathcurveto{\pgfqpoint{2.497137in}{0.989355in}}{\pgfqpoint{2.507736in}{0.984965in}}{\pgfqpoint{2.518786in}{0.984965in}}%
\pgfpathclose%
\pgfusepath{stroke,fill}%
\end{pgfscope}%
\begin{pgfscope}%
\pgfpathrectangle{\pgfqpoint{0.800000in}{0.528000in}}{\pgfqpoint{4.960000in}{3.696000in}}%
\pgfusepath{clip}%
\pgfsetbuttcap%
\pgfsetroundjoin%
\definecolor{currentfill}{rgb}{0.000000,0.000000,0.000000}%
\pgfsetfillcolor{currentfill}%
\pgfsetlinewidth{1.003750pt}%
\definecolor{currentstroke}{rgb}{0.000000,0.000000,0.000000}%
\pgfsetstrokecolor{currentstroke}%
\pgfsetdash{}{0pt}%
\pgfpathmoveto{\pgfqpoint{2.518786in}{0.878166in}}%
\pgfpathcurveto{\pgfqpoint{2.529836in}{0.878166in}}{\pgfqpoint{2.540435in}{0.882556in}}{\pgfqpoint{2.548249in}{0.890370in}}%
\pgfpathcurveto{\pgfqpoint{2.556062in}{0.898184in}}{\pgfqpoint{2.560452in}{0.908783in}}{\pgfqpoint{2.560452in}{0.919833in}}%
\pgfpathcurveto{\pgfqpoint{2.560452in}{0.930883in}}{\pgfqpoint{2.556062in}{0.941482in}}{\pgfqpoint{2.548249in}{0.949295in}}%
\pgfpathcurveto{\pgfqpoint{2.540435in}{0.957109in}}{\pgfqpoint{2.529836in}{0.961499in}}{\pgfqpoint{2.518786in}{0.961499in}}%
\pgfpathcurveto{\pgfqpoint{2.507736in}{0.961499in}}{\pgfqpoint{2.497137in}{0.957109in}}{\pgfqpoint{2.489323in}{0.949295in}}%
\pgfpathcurveto{\pgfqpoint{2.481509in}{0.941482in}}{\pgfqpoint{2.477119in}{0.930883in}}{\pgfqpoint{2.477119in}{0.919833in}}%
\pgfpathcurveto{\pgfqpoint{2.477119in}{0.908783in}}{\pgfqpoint{2.481509in}{0.898184in}}{\pgfqpoint{2.489323in}{0.890370in}}%
\pgfpathcurveto{\pgfqpoint{2.497137in}{0.882556in}}{\pgfqpoint{2.507736in}{0.878166in}}{\pgfqpoint{2.518786in}{0.878166in}}%
\pgfpathclose%
\pgfusepath{stroke,fill}%
\end{pgfscope}%
\begin{pgfscope}%
\pgfpathrectangle{\pgfqpoint{0.800000in}{0.528000in}}{\pgfqpoint{4.960000in}{3.696000in}}%
\pgfusepath{clip}%
\pgfsetbuttcap%
\pgfsetroundjoin%
\definecolor{currentfill}{rgb}{0.000000,0.000000,0.000000}%
\pgfsetfillcolor{currentfill}%
\pgfsetlinewidth{1.003750pt}%
\definecolor{currentstroke}{rgb}{0.000000,0.000000,0.000000}%
\pgfsetstrokecolor{currentstroke}%
\pgfsetdash{}{0pt}%
\pgfpathmoveto{\pgfqpoint{2.518786in}{1.033510in}}%
\pgfpathcurveto{\pgfqpoint{2.529836in}{1.033510in}}{\pgfqpoint{2.540435in}{1.037900in}}{\pgfqpoint{2.548249in}{1.045713in}}%
\pgfpathcurveto{\pgfqpoint{2.556062in}{1.053527in}}{\pgfqpoint{2.560452in}{1.064126in}}{\pgfqpoint{2.560452in}{1.075176in}}%
\pgfpathcurveto{\pgfqpoint{2.560452in}{1.086226in}}{\pgfqpoint{2.556062in}{1.096825in}}{\pgfqpoint{2.548249in}{1.104639in}}%
\pgfpathcurveto{\pgfqpoint{2.540435in}{1.112453in}}{\pgfqpoint{2.529836in}{1.116843in}}{\pgfqpoint{2.518786in}{1.116843in}}%
\pgfpathcurveto{\pgfqpoint{2.507736in}{1.116843in}}{\pgfqpoint{2.497137in}{1.112453in}}{\pgfqpoint{2.489323in}{1.104639in}}%
\pgfpathcurveto{\pgfqpoint{2.481509in}{1.096825in}}{\pgfqpoint{2.477119in}{1.086226in}}{\pgfqpoint{2.477119in}{1.075176in}}%
\pgfpathcurveto{\pgfqpoint{2.477119in}{1.064126in}}{\pgfqpoint{2.481509in}{1.053527in}}{\pgfqpoint{2.489323in}{1.045713in}}%
\pgfpathcurveto{\pgfqpoint{2.497137in}{1.037900in}}{\pgfqpoint{2.507736in}{1.033510in}}{\pgfqpoint{2.518786in}{1.033510in}}%
\pgfpathclose%
\pgfusepath{stroke,fill}%
\end{pgfscope}%
\begin{pgfscope}%
\pgfpathrectangle{\pgfqpoint{0.800000in}{0.528000in}}{\pgfqpoint{4.960000in}{3.696000in}}%
\pgfusepath{clip}%
\pgfsetbuttcap%
\pgfsetroundjoin%
\definecolor{currentfill}{rgb}{0.000000,0.000000,0.000000}%
\pgfsetfillcolor{currentfill}%
\pgfsetlinewidth{1.003750pt}%
\definecolor{currentstroke}{rgb}{0.000000,0.000000,0.000000}%
\pgfsetstrokecolor{currentstroke}%
\pgfsetdash{}{0pt}%
\pgfpathmoveto{\pgfqpoint{2.518786in}{1.159726in}}%
\pgfpathcurveto{\pgfqpoint{2.529836in}{1.159726in}}{\pgfqpoint{2.540435in}{1.164116in}}{\pgfqpoint{2.548249in}{1.171930in}}%
\pgfpathcurveto{\pgfqpoint{2.556062in}{1.179744in}}{\pgfqpoint{2.560452in}{1.190343in}}{\pgfqpoint{2.560452in}{1.201393in}}%
\pgfpathcurveto{\pgfqpoint{2.560452in}{1.212443in}}{\pgfqpoint{2.556062in}{1.223042in}}{\pgfqpoint{2.548249in}{1.230856in}}%
\pgfpathcurveto{\pgfqpoint{2.540435in}{1.238669in}}{\pgfqpoint{2.529836in}{1.243059in}}{\pgfqpoint{2.518786in}{1.243059in}}%
\pgfpathcurveto{\pgfqpoint{2.507736in}{1.243059in}}{\pgfqpoint{2.497137in}{1.238669in}}{\pgfqpoint{2.489323in}{1.230856in}}%
\pgfpathcurveto{\pgfqpoint{2.481509in}{1.223042in}}{\pgfqpoint{2.477119in}{1.212443in}}{\pgfqpoint{2.477119in}{1.201393in}}%
\pgfpathcurveto{\pgfqpoint{2.477119in}{1.190343in}}{\pgfqpoint{2.481509in}{1.179744in}}{\pgfqpoint{2.489323in}{1.171930in}}%
\pgfpathcurveto{\pgfqpoint{2.497137in}{1.164116in}}{\pgfqpoint{2.507736in}{1.159726in}}{\pgfqpoint{2.518786in}{1.159726in}}%
\pgfpathclose%
\pgfusepath{stroke,fill}%
\end{pgfscope}%
\begin{pgfscope}%
\pgfpathrectangle{\pgfqpoint{0.800000in}{0.528000in}}{\pgfqpoint{4.960000in}{3.696000in}}%
\pgfusepath{clip}%
\pgfsetbuttcap%
\pgfsetroundjoin%
\definecolor{currentfill}{rgb}{0.000000,0.000000,0.000000}%
\pgfsetfillcolor{currentfill}%
\pgfsetlinewidth{1.003750pt}%
\definecolor{currentstroke}{rgb}{0.000000,0.000000,0.000000}%
\pgfsetstrokecolor{currentstroke}%
\pgfsetdash{}{0pt}%
\pgfpathmoveto{\pgfqpoint{2.518786in}{1.043219in}}%
\pgfpathcurveto{\pgfqpoint{2.529836in}{1.043219in}}{\pgfqpoint{2.540435in}{1.047609in}}{\pgfqpoint{2.548249in}{1.055422in}}%
\pgfpathcurveto{\pgfqpoint{2.556062in}{1.063236in}}{\pgfqpoint{2.560452in}{1.073835in}}{\pgfqpoint{2.560452in}{1.084885in}}%
\pgfpathcurveto{\pgfqpoint{2.560452in}{1.095935in}}{\pgfqpoint{2.556062in}{1.106534in}}{\pgfqpoint{2.548249in}{1.114348in}}%
\pgfpathcurveto{\pgfqpoint{2.540435in}{1.122162in}}{\pgfqpoint{2.529836in}{1.126552in}}{\pgfqpoint{2.518786in}{1.126552in}}%
\pgfpathcurveto{\pgfqpoint{2.507736in}{1.126552in}}{\pgfqpoint{2.497137in}{1.122162in}}{\pgfqpoint{2.489323in}{1.114348in}}%
\pgfpathcurveto{\pgfqpoint{2.481509in}{1.106534in}}{\pgfqpoint{2.477119in}{1.095935in}}{\pgfqpoint{2.477119in}{1.084885in}}%
\pgfpathcurveto{\pgfqpoint{2.477119in}{1.073835in}}{\pgfqpoint{2.481509in}{1.063236in}}{\pgfqpoint{2.489323in}{1.055422in}}%
\pgfpathcurveto{\pgfqpoint{2.497137in}{1.047609in}}{\pgfqpoint{2.507736in}{1.043219in}}{\pgfqpoint{2.518786in}{1.043219in}}%
\pgfpathclose%
\pgfusepath{stroke,fill}%
\end{pgfscope}%
\begin{pgfscope}%
\pgfpathrectangle{\pgfqpoint{0.800000in}{0.528000in}}{\pgfqpoint{4.960000in}{3.696000in}}%
\pgfusepath{clip}%
\pgfsetbuttcap%
\pgfsetroundjoin%
\definecolor{currentfill}{rgb}{0.000000,0.000000,0.000000}%
\pgfsetfillcolor{currentfill}%
\pgfsetlinewidth{1.003750pt}%
\definecolor{currentstroke}{rgb}{0.000000,0.000000,0.000000}%
\pgfsetstrokecolor{currentstroke}%
\pgfsetdash{}{0pt}%
\pgfpathmoveto{\pgfqpoint{2.518786in}{0.955838in}}%
\pgfpathcurveto{\pgfqpoint{2.529836in}{0.955838in}}{\pgfqpoint{2.540435in}{0.960228in}}{\pgfqpoint{2.548249in}{0.968042in}}%
\pgfpathcurveto{\pgfqpoint{2.556062in}{0.975855in}}{\pgfqpoint{2.560452in}{0.986454in}}{\pgfqpoint{2.560452in}{0.997504in}}%
\pgfpathcurveto{\pgfqpoint{2.560452in}{1.008555in}}{\pgfqpoint{2.556062in}{1.019154in}}{\pgfqpoint{2.548249in}{1.026967in}}%
\pgfpathcurveto{\pgfqpoint{2.540435in}{1.034781in}}{\pgfqpoint{2.529836in}{1.039171in}}{\pgfqpoint{2.518786in}{1.039171in}}%
\pgfpathcurveto{\pgfqpoint{2.507736in}{1.039171in}}{\pgfqpoint{2.497137in}{1.034781in}}{\pgfqpoint{2.489323in}{1.026967in}}%
\pgfpathcurveto{\pgfqpoint{2.481509in}{1.019154in}}{\pgfqpoint{2.477119in}{1.008555in}}{\pgfqpoint{2.477119in}{0.997504in}}%
\pgfpathcurveto{\pgfqpoint{2.477119in}{0.986454in}}{\pgfqpoint{2.481509in}{0.975855in}}{\pgfqpoint{2.489323in}{0.968042in}}%
\pgfpathcurveto{\pgfqpoint{2.497137in}{0.960228in}}{\pgfqpoint{2.507736in}{0.955838in}}{\pgfqpoint{2.518786in}{0.955838in}}%
\pgfpathclose%
\pgfusepath{stroke,fill}%
\end{pgfscope}%
\begin{pgfscope}%
\pgfpathrectangle{\pgfqpoint{0.800000in}{0.528000in}}{\pgfqpoint{4.960000in}{3.696000in}}%
\pgfusepath{clip}%
\pgfsetbuttcap%
\pgfsetroundjoin%
\definecolor{currentfill}{rgb}{0.000000,0.000000,0.000000}%
\pgfsetfillcolor{currentfill}%
\pgfsetlinewidth{1.003750pt}%
\definecolor{currentstroke}{rgb}{0.000000,0.000000,0.000000}%
\pgfsetstrokecolor{currentstroke}%
\pgfsetdash{}{0pt}%
\pgfpathmoveto{\pgfqpoint{2.518786in}{1.091763in}}%
\pgfpathcurveto{\pgfqpoint{2.529836in}{1.091763in}}{\pgfqpoint{2.540435in}{1.096154in}}{\pgfqpoint{2.548249in}{1.103967in}}%
\pgfpathcurveto{\pgfqpoint{2.556062in}{1.111781in}}{\pgfqpoint{2.560452in}{1.122380in}}{\pgfqpoint{2.560452in}{1.133430in}}%
\pgfpathcurveto{\pgfqpoint{2.560452in}{1.144480in}}{\pgfqpoint{2.556062in}{1.155079in}}{\pgfqpoint{2.548249in}{1.162893in}}%
\pgfpathcurveto{\pgfqpoint{2.540435in}{1.170706in}}{\pgfqpoint{2.529836in}{1.175097in}}{\pgfqpoint{2.518786in}{1.175097in}}%
\pgfpathcurveto{\pgfqpoint{2.507736in}{1.175097in}}{\pgfqpoint{2.497137in}{1.170706in}}{\pgfqpoint{2.489323in}{1.162893in}}%
\pgfpathcurveto{\pgfqpoint{2.481509in}{1.155079in}}{\pgfqpoint{2.477119in}{1.144480in}}{\pgfqpoint{2.477119in}{1.133430in}}%
\pgfpathcurveto{\pgfqpoint{2.477119in}{1.122380in}}{\pgfqpoint{2.481509in}{1.111781in}}{\pgfqpoint{2.489323in}{1.103967in}}%
\pgfpathcurveto{\pgfqpoint{2.497137in}{1.096154in}}{\pgfqpoint{2.507736in}{1.091763in}}{\pgfqpoint{2.518786in}{1.091763in}}%
\pgfpathclose%
\pgfusepath{stroke,fill}%
\end{pgfscope}%
\begin{pgfscope}%
\pgfpathrectangle{\pgfqpoint{0.800000in}{0.528000in}}{\pgfqpoint{4.960000in}{3.696000in}}%
\pgfusepath{clip}%
\pgfsetbuttcap%
\pgfsetroundjoin%
\definecolor{currentfill}{rgb}{0.000000,0.000000,0.000000}%
\pgfsetfillcolor{currentfill}%
\pgfsetlinewidth{1.003750pt}%
\definecolor{currentstroke}{rgb}{0.000000,0.000000,0.000000}%
\pgfsetstrokecolor{currentstroke}%
\pgfsetdash{}{0pt}%
\pgfpathmoveto{\pgfqpoint{2.518786in}{0.994674in}}%
\pgfpathcurveto{\pgfqpoint{2.529836in}{0.994674in}}{\pgfqpoint{2.540435in}{0.999064in}}{\pgfqpoint{2.548249in}{1.006878in}}%
\pgfpathcurveto{\pgfqpoint{2.556062in}{1.014691in}}{\pgfqpoint{2.560452in}{1.025290in}}{\pgfqpoint{2.560452in}{1.036340in}}%
\pgfpathcurveto{\pgfqpoint{2.560452in}{1.047390in}}{\pgfqpoint{2.556062in}{1.057989in}}{\pgfqpoint{2.548249in}{1.065803in}}%
\pgfpathcurveto{\pgfqpoint{2.540435in}{1.073617in}}{\pgfqpoint{2.529836in}{1.078007in}}{\pgfqpoint{2.518786in}{1.078007in}}%
\pgfpathcurveto{\pgfqpoint{2.507736in}{1.078007in}}{\pgfqpoint{2.497137in}{1.073617in}}{\pgfqpoint{2.489323in}{1.065803in}}%
\pgfpathcurveto{\pgfqpoint{2.481509in}{1.057989in}}{\pgfqpoint{2.477119in}{1.047390in}}{\pgfqpoint{2.477119in}{1.036340in}}%
\pgfpathcurveto{\pgfqpoint{2.477119in}{1.025290in}}{\pgfqpoint{2.481509in}{1.014691in}}{\pgfqpoint{2.489323in}{1.006878in}}%
\pgfpathcurveto{\pgfqpoint{2.497137in}{0.999064in}}{\pgfqpoint{2.507736in}{0.994674in}}{\pgfqpoint{2.518786in}{0.994674in}}%
\pgfpathclose%
\pgfusepath{stroke,fill}%
\end{pgfscope}%
\begin{pgfscope}%
\pgfpathrectangle{\pgfqpoint{0.800000in}{0.528000in}}{\pgfqpoint{4.960000in}{3.696000in}}%
\pgfusepath{clip}%
\pgfsetbuttcap%
\pgfsetroundjoin%
\definecolor{currentfill}{rgb}{0.000000,0.000000,0.000000}%
\pgfsetfillcolor{currentfill}%
\pgfsetlinewidth{1.003750pt}%
\definecolor{currentstroke}{rgb}{0.000000,0.000000,0.000000}%
\pgfsetstrokecolor{currentstroke}%
\pgfsetdash{}{0pt}%
\pgfpathmoveto{\pgfqpoint{2.518786in}{0.984965in}}%
\pgfpathcurveto{\pgfqpoint{2.529836in}{0.984965in}}{\pgfqpoint{2.540435in}{0.989355in}}{\pgfqpoint{2.548249in}{0.997169in}}%
\pgfpathcurveto{\pgfqpoint{2.556062in}{1.004982in}}{\pgfqpoint{2.560452in}{1.015581in}}{\pgfqpoint{2.560452in}{1.026631in}}%
\pgfpathcurveto{\pgfqpoint{2.560452in}{1.037681in}}{\pgfqpoint{2.556062in}{1.048281in}}{\pgfqpoint{2.548249in}{1.056094in}}%
\pgfpathcurveto{\pgfqpoint{2.540435in}{1.063908in}}{\pgfqpoint{2.529836in}{1.068298in}}{\pgfqpoint{2.518786in}{1.068298in}}%
\pgfpathcurveto{\pgfqpoint{2.507736in}{1.068298in}}{\pgfqpoint{2.497137in}{1.063908in}}{\pgfqpoint{2.489323in}{1.056094in}}%
\pgfpathcurveto{\pgfqpoint{2.481509in}{1.048281in}}{\pgfqpoint{2.477119in}{1.037681in}}{\pgfqpoint{2.477119in}{1.026631in}}%
\pgfpathcurveto{\pgfqpoint{2.477119in}{1.015581in}}{\pgfqpoint{2.481509in}{1.004982in}}{\pgfqpoint{2.489323in}{0.997169in}}%
\pgfpathcurveto{\pgfqpoint{2.497137in}{0.989355in}}{\pgfqpoint{2.507736in}{0.984965in}}{\pgfqpoint{2.518786in}{0.984965in}}%
\pgfpathclose%
\pgfusepath{stroke,fill}%
\end{pgfscope}%
\begin{pgfscope}%
\pgfpathrectangle{\pgfqpoint{0.800000in}{0.528000in}}{\pgfqpoint{4.960000in}{3.696000in}}%
\pgfusepath{clip}%
\pgfsetbuttcap%
\pgfsetroundjoin%
\definecolor{currentfill}{rgb}{0.000000,0.000000,0.000000}%
\pgfsetfillcolor{currentfill}%
\pgfsetlinewidth{1.003750pt}%
\definecolor{currentstroke}{rgb}{0.000000,0.000000,0.000000}%
\pgfsetstrokecolor{currentstroke}%
\pgfsetdash{}{0pt}%
\pgfpathmoveto{\pgfqpoint{2.518786in}{0.984965in}}%
\pgfpathcurveto{\pgfqpoint{2.529836in}{0.984965in}}{\pgfqpoint{2.540435in}{0.989355in}}{\pgfqpoint{2.548249in}{0.997169in}}%
\pgfpathcurveto{\pgfqpoint{2.556062in}{1.004982in}}{\pgfqpoint{2.560452in}{1.015581in}}{\pgfqpoint{2.560452in}{1.026631in}}%
\pgfpathcurveto{\pgfqpoint{2.560452in}{1.037681in}}{\pgfqpoint{2.556062in}{1.048281in}}{\pgfqpoint{2.548249in}{1.056094in}}%
\pgfpathcurveto{\pgfqpoint{2.540435in}{1.063908in}}{\pgfqpoint{2.529836in}{1.068298in}}{\pgfqpoint{2.518786in}{1.068298in}}%
\pgfpathcurveto{\pgfqpoint{2.507736in}{1.068298in}}{\pgfqpoint{2.497137in}{1.063908in}}{\pgfqpoint{2.489323in}{1.056094in}}%
\pgfpathcurveto{\pgfqpoint{2.481509in}{1.048281in}}{\pgfqpoint{2.477119in}{1.037681in}}{\pgfqpoint{2.477119in}{1.026631in}}%
\pgfpathcurveto{\pgfqpoint{2.477119in}{1.015581in}}{\pgfqpoint{2.481509in}{1.004982in}}{\pgfqpoint{2.489323in}{0.997169in}}%
\pgfpathcurveto{\pgfqpoint{2.497137in}{0.989355in}}{\pgfqpoint{2.507736in}{0.984965in}}{\pgfqpoint{2.518786in}{0.984965in}}%
\pgfpathclose%
\pgfusepath{stroke,fill}%
\end{pgfscope}%
\begin{pgfscope}%
\pgfpathrectangle{\pgfqpoint{0.800000in}{0.528000in}}{\pgfqpoint{4.960000in}{3.696000in}}%
\pgfusepath{clip}%
\pgfsetbuttcap%
\pgfsetroundjoin%
\definecolor{currentfill}{rgb}{0.000000,0.000000,0.000000}%
\pgfsetfillcolor{currentfill}%
\pgfsetlinewidth{1.003750pt}%
\definecolor{currentstroke}{rgb}{0.000000,0.000000,0.000000}%
\pgfsetstrokecolor{currentstroke}%
\pgfsetdash{}{0pt}%
\pgfpathmoveto{\pgfqpoint{2.518786in}{1.159726in}}%
\pgfpathcurveto{\pgfqpoint{2.529836in}{1.159726in}}{\pgfqpoint{2.540435in}{1.164116in}}{\pgfqpoint{2.548249in}{1.171930in}}%
\pgfpathcurveto{\pgfqpoint{2.556062in}{1.179744in}}{\pgfqpoint{2.560452in}{1.190343in}}{\pgfqpoint{2.560452in}{1.201393in}}%
\pgfpathcurveto{\pgfqpoint{2.560452in}{1.212443in}}{\pgfqpoint{2.556062in}{1.223042in}}{\pgfqpoint{2.548249in}{1.230856in}}%
\pgfpathcurveto{\pgfqpoint{2.540435in}{1.238669in}}{\pgfqpoint{2.529836in}{1.243059in}}{\pgfqpoint{2.518786in}{1.243059in}}%
\pgfpathcurveto{\pgfqpoint{2.507736in}{1.243059in}}{\pgfqpoint{2.497137in}{1.238669in}}{\pgfqpoint{2.489323in}{1.230856in}}%
\pgfpathcurveto{\pgfqpoint{2.481509in}{1.223042in}}{\pgfqpoint{2.477119in}{1.212443in}}{\pgfqpoint{2.477119in}{1.201393in}}%
\pgfpathcurveto{\pgfqpoint{2.477119in}{1.190343in}}{\pgfqpoint{2.481509in}{1.179744in}}{\pgfqpoint{2.489323in}{1.171930in}}%
\pgfpathcurveto{\pgfqpoint{2.497137in}{1.164116in}}{\pgfqpoint{2.507736in}{1.159726in}}{\pgfqpoint{2.518786in}{1.159726in}}%
\pgfpathclose%
\pgfusepath{stroke,fill}%
\end{pgfscope}%
\begin{pgfscope}%
\pgfpathrectangle{\pgfqpoint{0.800000in}{0.528000in}}{\pgfqpoint{4.960000in}{3.696000in}}%
\pgfusepath{clip}%
\pgfsetbuttcap%
\pgfsetroundjoin%
\definecolor{currentfill}{rgb}{0.000000,0.000000,0.000000}%
\pgfsetfillcolor{currentfill}%
\pgfsetlinewidth{1.003750pt}%
\definecolor{currentstroke}{rgb}{0.000000,0.000000,0.000000}%
\pgfsetstrokecolor{currentstroke}%
\pgfsetdash{}{0pt}%
\pgfpathmoveto{\pgfqpoint{2.518786in}{1.043219in}}%
\pgfpathcurveto{\pgfqpoint{2.529836in}{1.043219in}}{\pgfqpoint{2.540435in}{1.047609in}}{\pgfqpoint{2.548249in}{1.055422in}}%
\pgfpathcurveto{\pgfqpoint{2.556062in}{1.063236in}}{\pgfqpoint{2.560452in}{1.073835in}}{\pgfqpoint{2.560452in}{1.084885in}}%
\pgfpathcurveto{\pgfqpoint{2.560452in}{1.095935in}}{\pgfqpoint{2.556062in}{1.106534in}}{\pgfqpoint{2.548249in}{1.114348in}}%
\pgfpathcurveto{\pgfqpoint{2.540435in}{1.122162in}}{\pgfqpoint{2.529836in}{1.126552in}}{\pgfqpoint{2.518786in}{1.126552in}}%
\pgfpathcurveto{\pgfqpoint{2.507736in}{1.126552in}}{\pgfqpoint{2.497137in}{1.122162in}}{\pgfqpoint{2.489323in}{1.114348in}}%
\pgfpathcurveto{\pgfqpoint{2.481509in}{1.106534in}}{\pgfqpoint{2.477119in}{1.095935in}}{\pgfqpoint{2.477119in}{1.084885in}}%
\pgfpathcurveto{\pgfqpoint{2.477119in}{1.073835in}}{\pgfqpoint{2.481509in}{1.063236in}}{\pgfqpoint{2.489323in}{1.055422in}}%
\pgfpathcurveto{\pgfqpoint{2.497137in}{1.047609in}}{\pgfqpoint{2.507736in}{1.043219in}}{\pgfqpoint{2.518786in}{1.043219in}}%
\pgfpathclose%
\pgfusepath{stroke,fill}%
\end{pgfscope}%
\begin{pgfscope}%
\pgfpathrectangle{\pgfqpoint{0.800000in}{0.528000in}}{\pgfqpoint{4.960000in}{3.696000in}}%
\pgfusepath{clip}%
\pgfsetbuttcap%
\pgfsetroundjoin%
\definecolor{currentfill}{rgb}{0.000000,0.000000,0.000000}%
\pgfsetfillcolor{currentfill}%
\pgfsetlinewidth{1.003750pt}%
\definecolor{currentstroke}{rgb}{0.000000,0.000000,0.000000}%
\pgfsetstrokecolor{currentstroke}%
\pgfsetdash{}{0pt}%
\pgfpathmoveto{\pgfqpoint{2.518786in}{0.936420in}}%
\pgfpathcurveto{\pgfqpoint{2.529836in}{0.936420in}}{\pgfqpoint{2.540435in}{0.940810in}}{\pgfqpoint{2.548249in}{0.948624in}}%
\pgfpathcurveto{\pgfqpoint{2.556062in}{0.956437in}}{\pgfqpoint{2.560452in}{0.967036in}}{\pgfqpoint{2.560452in}{0.978087in}}%
\pgfpathcurveto{\pgfqpoint{2.560452in}{0.989137in}}{\pgfqpoint{2.556062in}{0.999736in}}{\pgfqpoint{2.548249in}{1.007549in}}%
\pgfpathcurveto{\pgfqpoint{2.540435in}{1.015363in}}{\pgfqpoint{2.529836in}{1.019753in}}{\pgfqpoint{2.518786in}{1.019753in}}%
\pgfpathcurveto{\pgfqpoint{2.507736in}{1.019753in}}{\pgfqpoint{2.497137in}{1.015363in}}{\pgfqpoint{2.489323in}{1.007549in}}%
\pgfpathcurveto{\pgfqpoint{2.481509in}{0.999736in}}{\pgfqpoint{2.477119in}{0.989137in}}{\pgfqpoint{2.477119in}{0.978087in}}%
\pgfpathcurveto{\pgfqpoint{2.477119in}{0.967036in}}{\pgfqpoint{2.481509in}{0.956437in}}{\pgfqpoint{2.489323in}{0.948624in}}%
\pgfpathcurveto{\pgfqpoint{2.497137in}{0.940810in}}{\pgfqpoint{2.507736in}{0.936420in}}{\pgfqpoint{2.518786in}{0.936420in}}%
\pgfpathclose%
\pgfusepath{stroke,fill}%
\end{pgfscope}%
\begin{pgfscope}%
\pgfpathrectangle{\pgfqpoint{0.800000in}{0.528000in}}{\pgfqpoint{4.960000in}{3.696000in}}%
\pgfusepath{clip}%
\pgfsetbuttcap%
\pgfsetroundjoin%
\definecolor{currentfill}{rgb}{0.000000,0.000000,0.000000}%
\pgfsetfillcolor{currentfill}%
\pgfsetlinewidth{1.003750pt}%
\definecolor{currentstroke}{rgb}{0.000000,0.000000,0.000000}%
\pgfsetstrokecolor{currentstroke}%
\pgfsetdash{}{0pt}%
\pgfpathmoveto{\pgfqpoint{2.518786in}{1.004383in}}%
\pgfpathcurveto{\pgfqpoint{2.529836in}{1.004383in}}{\pgfqpoint{2.540435in}{1.008773in}}{\pgfqpoint{2.548249in}{1.016587in}}%
\pgfpathcurveto{\pgfqpoint{2.556062in}{1.024400in}}{\pgfqpoint{2.560452in}{1.034999in}}{\pgfqpoint{2.560452in}{1.046049in}}%
\pgfpathcurveto{\pgfqpoint{2.560452in}{1.057099in}}{\pgfqpoint{2.556062in}{1.067698in}}{\pgfqpoint{2.548249in}{1.075512in}}%
\pgfpathcurveto{\pgfqpoint{2.540435in}{1.083326in}}{\pgfqpoint{2.529836in}{1.087716in}}{\pgfqpoint{2.518786in}{1.087716in}}%
\pgfpathcurveto{\pgfqpoint{2.507736in}{1.087716in}}{\pgfqpoint{2.497137in}{1.083326in}}{\pgfqpoint{2.489323in}{1.075512in}}%
\pgfpathcurveto{\pgfqpoint{2.481509in}{1.067698in}}{\pgfqpoint{2.477119in}{1.057099in}}{\pgfqpoint{2.477119in}{1.046049in}}%
\pgfpathcurveto{\pgfqpoint{2.477119in}{1.034999in}}{\pgfqpoint{2.481509in}{1.024400in}}{\pgfqpoint{2.489323in}{1.016587in}}%
\pgfpathcurveto{\pgfqpoint{2.497137in}{1.008773in}}{\pgfqpoint{2.507736in}{1.004383in}}{\pgfqpoint{2.518786in}{1.004383in}}%
\pgfpathclose%
\pgfusepath{stroke,fill}%
\end{pgfscope}%
\begin{pgfscope}%
\pgfpathrectangle{\pgfqpoint{0.800000in}{0.528000in}}{\pgfqpoint{4.960000in}{3.696000in}}%
\pgfusepath{clip}%
\pgfsetbuttcap%
\pgfsetroundjoin%
\definecolor{currentfill}{rgb}{0.000000,0.000000,0.000000}%
\pgfsetfillcolor{currentfill}%
\pgfsetlinewidth{1.003750pt}%
\definecolor{currentstroke}{rgb}{0.000000,0.000000,0.000000}%
\pgfsetstrokecolor{currentstroke}%
\pgfsetdash{}{0pt}%
\pgfpathmoveto{\pgfqpoint{2.518786in}{1.140308in}}%
\pgfpathcurveto{\pgfqpoint{2.529836in}{1.140308in}}{\pgfqpoint{2.540435in}{1.144698in}}{\pgfqpoint{2.548249in}{1.152512in}}%
\pgfpathcurveto{\pgfqpoint{2.556062in}{1.160326in}}{\pgfqpoint{2.560452in}{1.170925in}}{\pgfqpoint{2.560452in}{1.181975in}}%
\pgfpathcurveto{\pgfqpoint{2.560452in}{1.193025in}}{\pgfqpoint{2.556062in}{1.203624in}}{\pgfqpoint{2.548249in}{1.211438in}}%
\pgfpathcurveto{\pgfqpoint{2.540435in}{1.219251in}}{\pgfqpoint{2.529836in}{1.223642in}}{\pgfqpoint{2.518786in}{1.223642in}}%
\pgfpathcurveto{\pgfqpoint{2.507736in}{1.223642in}}{\pgfqpoint{2.497137in}{1.219251in}}{\pgfqpoint{2.489323in}{1.211438in}}%
\pgfpathcurveto{\pgfqpoint{2.481509in}{1.203624in}}{\pgfqpoint{2.477119in}{1.193025in}}{\pgfqpoint{2.477119in}{1.181975in}}%
\pgfpathcurveto{\pgfqpoint{2.477119in}{1.170925in}}{\pgfqpoint{2.481509in}{1.160326in}}{\pgfqpoint{2.489323in}{1.152512in}}%
\pgfpathcurveto{\pgfqpoint{2.497137in}{1.144698in}}{\pgfqpoint{2.507736in}{1.140308in}}{\pgfqpoint{2.518786in}{1.140308in}}%
\pgfpathclose%
\pgfusepath{stroke,fill}%
\end{pgfscope}%
\begin{pgfscope}%
\pgfpathrectangle{\pgfqpoint{0.800000in}{0.528000in}}{\pgfqpoint{4.960000in}{3.696000in}}%
\pgfusepath{clip}%
\pgfsetbuttcap%
\pgfsetroundjoin%
\definecolor{currentfill}{rgb}{0.000000,0.000000,0.000000}%
\pgfsetfillcolor{currentfill}%
\pgfsetlinewidth{1.003750pt}%
\definecolor{currentstroke}{rgb}{0.000000,0.000000,0.000000}%
\pgfsetstrokecolor{currentstroke}%
\pgfsetdash{}{0pt}%
\pgfpathmoveto{\pgfqpoint{2.518786in}{1.014092in}}%
\pgfpathcurveto{\pgfqpoint{2.529836in}{1.014092in}}{\pgfqpoint{2.540435in}{1.018482in}}{\pgfqpoint{2.548249in}{1.026295in}}%
\pgfpathcurveto{\pgfqpoint{2.556062in}{1.034109in}}{\pgfqpoint{2.560452in}{1.044708in}}{\pgfqpoint{2.560452in}{1.055758in}}%
\pgfpathcurveto{\pgfqpoint{2.560452in}{1.066808in}}{\pgfqpoint{2.556062in}{1.077407in}}{\pgfqpoint{2.548249in}{1.085221in}}%
\pgfpathcurveto{\pgfqpoint{2.540435in}{1.093035in}}{\pgfqpoint{2.529836in}{1.097425in}}{\pgfqpoint{2.518786in}{1.097425in}}%
\pgfpathcurveto{\pgfqpoint{2.507736in}{1.097425in}}{\pgfqpoint{2.497137in}{1.093035in}}{\pgfqpoint{2.489323in}{1.085221in}}%
\pgfpathcurveto{\pgfqpoint{2.481509in}{1.077407in}}{\pgfqpoint{2.477119in}{1.066808in}}{\pgfqpoint{2.477119in}{1.055758in}}%
\pgfpathcurveto{\pgfqpoint{2.477119in}{1.044708in}}{\pgfqpoint{2.481509in}{1.034109in}}{\pgfqpoint{2.489323in}{1.026295in}}%
\pgfpathcurveto{\pgfqpoint{2.497137in}{1.018482in}}{\pgfqpoint{2.507736in}{1.014092in}}{\pgfqpoint{2.518786in}{1.014092in}}%
\pgfpathclose%
\pgfusepath{stroke,fill}%
\end{pgfscope}%
\begin{pgfscope}%
\pgfpathrectangle{\pgfqpoint{0.800000in}{0.528000in}}{\pgfqpoint{4.960000in}{3.696000in}}%
\pgfusepath{clip}%
\pgfsetbuttcap%
\pgfsetroundjoin%
\definecolor{currentfill}{rgb}{0.000000,0.000000,0.000000}%
\pgfsetfillcolor{currentfill}%
\pgfsetlinewidth{1.003750pt}%
\definecolor{currentstroke}{rgb}{0.000000,0.000000,0.000000}%
\pgfsetstrokecolor{currentstroke}%
\pgfsetdash{}{0pt}%
\pgfpathmoveto{\pgfqpoint{2.518786in}{1.091763in}}%
\pgfpathcurveto{\pgfqpoint{2.529836in}{1.091763in}}{\pgfqpoint{2.540435in}{1.096154in}}{\pgfqpoint{2.548249in}{1.103967in}}%
\pgfpathcurveto{\pgfqpoint{2.556062in}{1.111781in}}{\pgfqpoint{2.560452in}{1.122380in}}{\pgfqpoint{2.560452in}{1.133430in}}%
\pgfpathcurveto{\pgfqpoint{2.560452in}{1.144480in}}{\pgfqpoint{2.556062in}{1.155079in}}{\pgfqpoint{2.548249in}{1.162893in}}%
\pgfpathcurveto{\pgfqpoint{2.540435in}{1.170706in}}{\pgfqpoint{2.529836in}{1.175097in}}{\pgfqpoint{2.518786in}{1.175097in}}%
\pgfpathcurveto{\pgfqpoint{2.507736in}{1.175097in}}{\pgfqpoint{2.497137in}{1.170706in}}{\pgfqpoint{2.489323in}{1.162893in}}%
\pgfpathcurveto{\pgfqpoint{2.481509in}{1.155079in}}{\pgfqpoint{2.477119in}{1.144480in}}{\pgfqpoint{2.477119in}{1.133430in}}%
\pgfpathcurveto{\pgfqpoint{2.477119in}{1.122380in}}{\pgfqpoint{2.481509in}{1.111781in}}{\pgfqpoint{2.489323in}{1.103967in}}%
\pgfpathcurveto{\pgfqpoint{2.497137in}{1.096154in}}{\pgfqpoint{2.507736in}{1.091763in}}{\pgfqpoint{2.518786in}{1.091763in}}%
\pgfpathclose%
\pgfusepath{stroke,fill}%
\end{pgfscope}%
\begin{pgfscope}%
\pgfpathrectangle{\pgfqpoint{0.800000in}{0.528000in}}{\pgfqpoint{4.960000in}{3.696000in}}%
\pgfusepath{clip}%
\pgfsetbuttcap%
\pgfsetroundjoin%
\definecolor{currentfill}{rgb}{0.000000,0.000000,0.000000}%
\pgfsetfillcolor{currentfill}%
\pgfsetlinewidth{1.003750pt}%
\definecolor{currentstroke}{rgb}{0.000000,0.000000,0.000000}%
\pgfsetstrokecolor{currentstroke}%
\pgfsetdash{}{0pt}%
\pgfpathmoveto{\pgfqpoint{2.518786in}{1.052927in}}%
\pgfpathcurveto{\pgfqpoint{2.529836in}{1.052927in}}{\pgfqpoint{2.540435in}{1.057318in}}{\pgfqpoint{2.548249in}{1.065131in}}%
\pgfpathcurveto{\pgfqpoint{2.556062in}{1.072945in}}{\pgfqpoint{2.560452in}{1.083544in}}{\pgfqpoint{2.560452in}{1.094594in}}%
\pgfpathcurveto{\pgfqpoint{2.560452in}{1.105644in}}{\pgfqpoint{2.556062in}{1.116243in}}{\pgfqpoint{2.548249in}{1.124057in}}%
\pgfpathcurveto{\pgfqpoint{2.540435in}{1.131871in}}{\pgfqpoint{2.529836in}{1.136261in}}{\pgfqpoint{2.518786in}{1.136261in}}%
\pgfpathcurveto{\pgfqpoint{2.507736in}{1.136261in}}{\pgfqpoint{2.497137in}{1.131871in}}{\pgfqpoint{2.489323in}{1.124057in}}%
\pgfpathcurveto{\pgfqpoint{2.481509in}{1.116243in}}{\pgfqpoint{2.477119in}{1.105644in}}{\pgfqpoint{2.477119in}{1.094594in}}%
\pgfpathcurveto{\pgfqpoint{2.477119in}{1.083544in}}{\pgfqpoint{2.481509in}{1.072945in}}{\pgfqpoint{2.489323in}{1.065131in}}%
\pgfpathcurveto{\pgfqpoint{2.497137in}{1.057318in}}{\pgfqpoint{2.507736in}{1.052927in}}{\pgfqpoint{2.518786in}{1.052927in}}%
\pgfpathclose%
\pgfusepath{stroke,fill}%
\end{pgfscope}%
\begin{pgfscope}%
\pgfpathrectangle{\pgfqpoint{0.800000in}{0.528000in}}{\pgfqpoint{4.960000in}{3.696000in}}%
\pgfusepath{clip}%
\pgfsetbuttcap%
\pgfsetroundjoin%
\definecolor{currentfill}{rgb}{0.000000,0.000000,0.000000}%
\pgfsetfillcolor{currentfill}%
\pgfsetlinewidth{1.003750pt}%
\definecolor{currentstroke}{rgb}{0.000000,0.000000,0.000000}%
\pgfsetstrokecolor{currentstroke}%
\pgfsetdash{}{0pt}%
\pgfpathmoveto{\pgfqpoint{2.518786in}{1.004383in}}%
\pgfpathcurveto{\pgfqpoint{2.529836in}{1.004383in}}{\pgfqpoint{2.540435in}{1.008773in}}{\pgfqpoint{2.548249in}{1.016587in}}%
\pgfpathcurveto{\pgfqpoint{2.556062in}{1.024400in}}{\pgfqpoint{2.560452in}{1.034999in}}{\pgfqpoint{2.560452in}{1.046049in}}%
\pgfpathcurveto{\pgfqpoint{2.560452in}{1.057099in}}{\pgfqpoint{2.556062in}{1.067698in}}{\pgfqpoint{2.548249in}{1.075512in}}%
\pgfpathcurveto{\pgfqpoint{2.540435in}{1.083326in}}{\pgfqpoint{2.529836in}{1.087716in}}{\pgfqpoint{2.518786in}{1.087716in}}%
\pgfpathcurveto{\pgfqpoint{2.507736in}{1.087716in}}{\pgfqpoint{2.497137in}{1.083326in}}{\pgfqpoint{2.489323in}{1.075512in}}%
\pgfpathcurveto{\pgfqpoint{2.481509in}{1.067698in}}{\pgfqpoint{2.477119in}{1.057099in}}{\pgfqpoint{2.477119in}{1.046049in}}%
\pgfpathcurveto{\pgfqpoint{2.477119in}{1.034999in}}{\pgfqpoint{2.481509in}{1.024400in}}{\pgfqpoint{2.489323in}{1.016587in}}%
\pgfpathcurveto{\pgfqpoint{2.497137in}{1.008773in}}{\pgfqpoint{2.507736in}{1.004383in}}{\pgfqpoint{2.518786in}{1.004383in}}%
\pgfpathclose%
\pgfusepath{stroke,fill}%
\end{pgfscope}%
\begin{pgfscope}%
\pgfpathrectangle{\pgfqpoint{0.800000in}{0.528000in}}{\pgfqpoint{4.960000in}{3.696000in}}%
\pgfusepath{clip}%
\pgfsetbuttcap%
\pgfsetroundjoin%
\definecolor{currentfill}{rgb}{0.000000,0.000000,0.000000}%
\pgfsetfillcolor{currentfill}%
\pgfsetlinewidth{1.003750pt}%
\definecolor{currentstroke}{rgb}{0.000000,0.000000,0.000000}%
\pgfsetstrokecolor{currentstroke}%
\pgfsetdash{}{0pt}%
\pgfpathmoveto{\pgfqpoint{2.518786in}{0.994674in}}%
\pgfpathcurveto{\pgfqpoint{2.529836in}{0.994674in}}{\pgfqpoint{2.540435in}{0.999064in}}{\pgfqpoint{2.548249in}{1.006878in}}%
\pgfpathcurveto{\pgfqpoint{2.556062in}{1.014691in}}{\pgfqpoint{2.560452in}{1.025290in}}{\pgfqpoint{2.560452in}{1.036340in}}%
\pgfpathcurveto{\pgfqpoint{2.560452in}{1.047390in}}{\pgfqpoint{2.556062in}{1.057989in}}{\pgfqpoint{2.548249in}{1.065803in}}%
\pgfpathcurveto{\pgfqpoint{2.540435in}{1.073617in}}{\pgfqpoint{2.529836in}{1.078007in}}{\pgfqpoint{2.518786in}{1.078007in}}%
\pgfpathcurveto{\pgfqpoint{2.507736in}{1.078007in}}{\pgfqpoint{2.497137in}{1.073617in}}{\pgfqpoint{2.489323in}{1.065803in}}%
\pgfpathcurveto{\pgfqpoint{2.481509in}{1.057989in}}{\pgfqpoint{2.477119in}{1.047390in}}{\pgfqpoint{2.477119in}{1.036340in}}%
\pgfpathcurveto{\pgfqpoint{2.477119in}{1.025290in}}{\pgfqpoint{2.481509in}{1.014691in}}{\pgfqpoint{2.489323in}{1.006878in}}%
\pgfpathcurveto{\pgfqpoint{2.497137in}{0.999064in}}{\pgfqpoint{2.507736in}{0.994674in}}{\pgfqpoint{2.518786in}{0.994674in}}%
\pgfpathclose%
\pgfusepath{stroke,fill}%
\end{pgfscope}%
\begin{pgfscope}%
\pgfpathrectangle{\pgfqpoint{0.800000in}{0.528000in}}{\pgfqpoint{4.960000in}{3.696000in}}%
\pgfusepath{clip}%
\pgfsetbuttcap%
\pgfsetroundjoin%
\definecolor{currentfill}{rgb}{0.000000,0.000000,0.000000}%
\pgfsetfillcolor{currentfill}%
\pgfsetlinewidth{1.003750pt}%
\definecolor{currentstroke}{rgb}{0.000000,0.000000,0.000000}%
\pgfsetstrokecolor{currentstroke}%
\pgfsetdash{}{0pt}%
\pgfpathmoveto{\pgfqpoint{2.518786in}{1.091763in}}%
\pgfpathcurveto{\pgfqpoint{2.529836in}{1.091763in}}{\pgfqpoint{2.540435in}{1.096154in}}{\pgfqpoint{2.548249in}{1.103967in}}%
\pgfpathcurveto{\pgfqpoint{2.556062in}{1.111781in}}{\pgfqpoint{2.560452in}{1.122380in}}{\pgfqpoint{2.560452in}{1.133430in}}%
\pgfpathcurveto{\pgfqpoint{2.560452in}{1.144480in}}{\pgfqpoint{2.556062in}{1.155079in}}{\pgfqpoint{2.548249in}{1.162893in}}%
\pgfpathcurveto{\pgfqpoint{2.540435in}{1.170706in}}{\pgfqpoint{2.529836in}{1.175097in}}{\pgfqpoint{2.518786in}{1.175097in}}%
\pgfpathcurveto{\pgfqpoint{2.507736in}{1.175097in}}{\pgfqpoint{2.497137in}{1.170706in}}{\pgfqpoint{2.489323in}{1.162893in}}%
\pgfpathcurveto{\pgfqpoint{2.481509in}{1.155079in}}{\pgfqpoint{2.477119in}{1.144480in}}{\pgfqpoint{2.477119in}{1.133430in}}%
\pgfpathcurveto{\pgfqpoint{2.477119in}{1.122380in}}{\pgfqpoint{2.481509in}{1.111781in}}{\pgfqpoint{2.489323in}{1.103967in}}%
\pgfpathcurveto{\pgfqpoint{2.497137in}{1.096154in}}{\pgfqpoint{2.507736in}{1.091763in}}{\pgfqpoint{2.518786in}{1.091763in}}%
\pgfpathclose%
\pgfusepath{stroke,fill}%
\end{pgfscope}%
\begin{pgfscope}%
\pgfpathrectangle{\pgfqpoint{0.800000in}{0.528000in}}{\pgfqpoint{4.960000in}{3.696000in}}%
\pgfusepath{clip}%
\pgfsetbuttcap%
\pgfsetroundjoin%
\definecolor{currentfill}{rgb}{0.000000,0.000000,0.000000}%
\pgfsetfillcolor{currentfill}%
\pgfsetlinewidth{1.003750pt}%
\definecolor{currentstroke}{rgb}{0.000000,0.000000,0.000000}%
\pgfsetstrokecolor{currentstroke}%
\pgfsetdash{}{0pt}%
\pgfpathmoveto{\pgfqpoint{2.518786in}{0.965547in}}%
\pgfpathcurveto{\pgfqpoint{2.529836in}{0.965547in}}{\pgfqpoint{2.540435in}{0.969937in}}{\pgfqpoint{2.548249in}{0.977751in}}%
\pgfpathcurveto{\pgfqpoint{2.556062in}{0.985564in}}{\pgfqpoint{2.560452in}{0.996163in}}{\pgfqpoint{2.560452in}{1.007213in}}%
\pgfpathcurveto{\pgfqpoint{2.560452in}{1.018264in}}{\pgfqpoint{2.556062in}{1.028863in}}{\pgfqpoint{2.548249in}{1.036676in}}%
\pgfpathcurveto{\pgfqpoint{2.540435in}{1.044490in}}{\pgfqpoint{2.529836in}{1.048880in}}{\pgfqpoint{2.518786in}{1.048880in}}%
\pgfpathcurveto{\pgfqpoint{2.507736in}{1.048880in}}{\pgfqpoint{2.497137in}{1.044490in}}{\pgfqpoint{2.489323in}{1.036676in}}%
\pgfpathcurveto{\pgfqpoint{2.481509in}{1.028863in}}{\pgfqpoint{2.477119in}{1.018264in}}{\pgfqpoint{2.477119in}{1.007213in}}%
\pgfpathcurveto{\pgfqpoint{2.477119in}{0.996163in}}{\pgfqpoint{2.481509in}{0.985564in}}{\pgfqpoint{2.489323in}{0.977751in}}%
\pgfpathcurveto{\pgfqpoint{2.497137in}{0.969937in}}{\pgfqpoint{2.507736in}{0.965547in}}{\pgfqpoint{2.518786in}{0.965547in}}%
\pgfpathclose%
\pgfusepath{stroke,fill}%
\end{pgfscope}%
\begin{pgfscope}%
\pgfpathrectangle{\pgfqpoint{0.800000in}{0.528000in}}{\pgfqpoint{4.960000in}{3.696000in}}%
\pgfusepath{clip}%
\pgfsetbuttcap%
\pgfsetroundjoin%
\definecolor{currentfill}{rgb}{0.000000,0.000000,0.000000}%
\pgfsetfillcolor{currentfill}%
\pgfsetlinewidth{1.003750pt}%
\definecolor{currentstroke}{rgb}{0.000000,0.000000,0.000000}%
\pgfsetstrokecolor{currentstroke}%
\pgfsetdash{}{0pt}%
\pgfpathmoveto{\pgfqpoint{2.518786in}{1.052927in}}%
\pgfpathcurveto{\pgfqpoint{2.529836in}{1.052927in}}{\pgfqpoint{2.540435in}{1.057318in}}{\pgfqpoint{2.548249in}{1.065131in}}%
\pgfpathcurveto{\pgfqpoint{2.556062in}{1.072945in}}{\pgfqpoint{2.560452in}{1.083544in}}{\pgfqpoint{2.560452in}{1.094594in}}%
\pgfpathcurveto{\pgfqpoint{2.560452in}{1.105644in}}{\pgfqpoint{2.556062in}{1.116243in}}{\pgfqpoint{2.548249in}{1.124057in}}%
\pgfpathcurveto{\pgfqpoint{2.540435in}{1.131871in}}{\pgfqpoint{2.529836in}{1.136261in}}{\pgfqpoint{2.518786in}{1.136261in}}%
\pgfpathcurveto{\pgfqpoint{2.507736in}{1.136261in}}{\pgfqpoint{2.497137in}{1.131871in}}{\pgfqpoint{2.489323in}{1.124057in}}%
\pgfpathcurveto{\pgfqpoint{2.481509in}{1.116243in}}{\pgfqpoint{2.477119in}{1.105644in}}{\pgfqpoint{2.477119in}{1.094594in}}%
\pgfpathcurveto{\pgfqpoint{2.477119in}{1.083544in}}{\pgfqpoint{2.481509in}{1.072945in}}{\pgfqpoint{2.489323in}{1.065131in}}%
\pgfpathcurveto{\pgfqpoint{2.497137in}{1.057318in}}{\pgfqpoint{2.507736in}{1.052927in}}{\pgfqpoint{2.518786in}{1.052927in}}%
\pgfpathclose%
\pgfusepath{stroke,fill}%
\end{pgfscope}%
\begin{pgfscope}%
\pgfpathrectangle{\pgfqpoint{0.800000in}{0.528000in}}{\pgfqpoint{4.960000in}{3.696000in}}%
\pgfusepath{clip}%
\pgfsetbuttcap%
\pgfsetroundjoin%
\definecolor{currentfill}{rgb}{0.000000,0.000000,0.000000}%
\pgfsetfillcolor{currentfill}%
\pgfsetlinewidth{1.003750pt}%
\definecolor{currentstroke}{rgb}{0.000000,0.000000,0.000000}%
\pgfsetstrokecolor{currentstroke}%
\pgfsetdash{}{0pt}%
\pgfpathmoveto{\pgfqpoint{2.518786in}{1.014092in}}%
\pgfpathcurveto{\pgfqpoint{2.529836in}{1.014092in}}{\pgfqpoint{2.540435in}{1.018482in}}{\pgfqpoint{2.548249in}{1.026295in}}%
\pgfpathcurveto{\pgfqpoint{2.556062in}{1.034109in}}{\pgfqpoint{2.560452in}{1.044708in}}{\pgfqpoint{2.560452in}{1.055758in}}%
\pgfpathcurveto{\pgfqpoint{2.560452in}{1.066808in}}{\pgfqpoint{2.556062in}{1.077407in}}{\pgfqpoint{2.548249in}{1.085221in}}%
\pgfpathcurveto{\pgfqpoint{2.540435in}{1.093035in}}{\pgfqpoint{2.529836in}{1.097425in}}{\pgfqpoint{2.518786in}{1.097425in}}%
\pgfpathcurveto{\pgfqpoint{2.507736in}{1.097425in}}{\pgfqpoint{2.497137in}{1.093035in}}{\pgfqpoint{2.489323in}{1.085221in}}%
\pgfpathcurveto{\pgfqpoint{2.481509in}{1.077407in}}{\pgfqpoint{2.477119in}{1.066808in}}{\pgfqpoint{2.477119in}{1.055758in}}%
\pgfpathcurveto{\pgfqpoint{2.477119in}{1.044708in}}{\pgfqpoint{2.481509in}{1.034109in}}{\pgfqpoint{2.489323in}{1.026295in}}%
\pgfpathcurveto{\pgfqpoint{2.497137in}{1.018482in}}{\pgfqpoint{2.507736in}{1.014092in}}{\pgfqpoint{2.518786in}{1.014092in}}%
\pgfpathclose%
\pgfusepath{stroke,fill}%
\end{pgfscope}%
\begin{pgfscope}%
\pgfpathrectangle{\pgfqpoint{0.800000in}{0.528000in}}{\pgfqpoint{4.960000in}{3.696000in}}%
\pgfusepath{clip}%
\pgfsetbuttcap%
\pgfsetroundjoin%
\definecolor{currentfill}{rgb}{0.000000,0.000000,0.000000}%
\pgfsetfillcolor{currentfill}%
\pgfsetlinewidth{1.003750pt}%
\definecolor{currentstroke}{rgb}{0.000000,0.000000,0.000000}%
\pgfsetstrokecolor{currentstroke}%
\pgfsetdash{}{0pt}%
\pgfpathmoveto{\pgfqpoint{2.518786in}{1.043219in}}%
\pgfpathcurveto{\pgfqpoint{2.529836in}{1.043219in}}{\pgfqpoint{2.540435in}{1.047609in}}{\pgfqpoint{2.548249in}{1.055422in}}%
\pgfpathcurveto{\pgfqpoint{2.556062in}{1.063236in}}{\pgfqpoint{2.560452in}{1.073835in}}{\pgfqpoint{2.560452in}{1.084885in}}%
\pgfpathcurveto{\pgfqpoint{2.560452in}{1.095935in}}{\pgfqpoint{2.556062in}{1.106534in}}{\pgfqpoint{2.548249in}{1.114348in}}%
\pgfpathcurveto{\pgfqpoint{2.540435in}{1.122162in}}{\pgfqpoint{2.529836in}{1.126552in}}{\pgfqpoint{2.518786in}{1.126552in}}%
\pgfpathcurveto{\pgfqpoint{2.507736in}{1.126552in}}{\pgfqpoint{2.497137in}{1.122162in}}{\pgfqpoint{2.489323in}{1.114348in}}%
\pgfpathcurveto{\pgfqpoint{2.481509in}{1.106534in}}{\pgfqpoint{2.477119in}{1.095935in}}{\pgfqpoint{2.477119in}{1.084885in}}%
\pgfpathcurveto{\pgfqpoint{2.477119in}{1.073835in}}{\pgfqpoint{2.481509in}{1.063236in}}{\pgfqpoint{2.489323in}{1.055422in}}%
\pgfpathcurveto{\pgfqpoint{2.497137in}{1.047609in}}{\pgfqpoint{2.507736in}{1.043219in}}{\pgfqpoint{2.518786in}{1.043219in}}%
\pgfpathclose%
\pgfusepath{stroke,fill}%
\end{pgfscope}%
\begin{pgfscope}%
\pgfpathrectangle{\pgfqpoint{0.800000in}{0.528000in}}{\pgfqpoint{4.960000in}{3.696000in}}%
\pgfusepath{clip}%
\pgfsetbuttcap%
\pgfsetroundjoin%
\definecolor{currentfill}{rgb}{0.000000,0.000000,0.000000}%
\pgfsetfillcolor{currentfill}%
\pgfsetlinewidth{1.003750pt}%
\definecolor{currentstroke}{rgb}{0.000000,0.000000,0.000000}%
\pgfsetstrokecolor{currentstroke}%
\pgfsetdash{}{0pt}%
\pgfpathmoveto{\pgfqpoint{2.518786in}{1.043219in}}%
\pgfpathcurveto{\pgfqpoint{2.529836in}{1.043219in}}{\pgfqpoint{2.540435in}{1.047609in}}{\pgfqpoint{2.548249in}{1.055422in}}%
\pgfpathcurveto{\pgfqpoint{2.556062in}{1.063236in}}{\pgfqpoint{2.560452in}{1.073835in}}{\pgfqpoint{2.560452in}{1.084885in}}%
\pgfpathcurveto{\pgfqpoint{2.560452in}{1.095935in}}{\pgfqpoint{2.556062in}{1.106534in}}{\pgfqpoint{2.548249in}{1.114348in}}%
\pgfpathcurveto{\pgfqpoint{2.540435in}{1.122162in}}{\pgfqpoint{2.529836in}{1.126552in}}{\pgfqpoint{2.518786in}{1.126552in}}%
\pgfpathcurveto{\pgfqpoint{2.507736in}{1.126552in}}{\pgfqpoint{2.497137in}{1.122162in}}{\pgfqpoint{2.489323in}{1.114348in}}%
\pgfpathcurveto{\pgfqpoint{2.481509in}{1.106534in}}{\pgfqpoint{2.477119in}{1.095935in}}{\pgfqpoint{2.477119in}{1.084885in}}%
\pgfpathcurveto{\pgfqpoint{2.477119in}{1.073835in}}{\pgfqpoint{2.481509in}{1.063236in}}{\pgfqpoint{2.489323in}{1.055422in}}%
\pgfpathcurveto{\pgfqpoint{2.497137in}{1.047609in}}{\pgfqpoint{2.507736in}{1.043219in}}{\pgfqpoint{2.518786in}{1.043219in}}%
\pgfpathclose%
\pgfusepath{stroke,fill}%
\end{pgfscope}%
\begin{pgfscope}%
\pgfpathrectangle{\pgfqpoint{0.800000in}{0.528000in}}{\pgfqpoint{4.960000in}{3.696000in}}%
\pgfusepath{clip}%
\pgfsetbuttcap%
\pgfsetroundjoin%
\definecolor{currentfill}{rgb}{0.000000,0.000000,0.000000}%
\pgfsetfillcolor{currentfill}%
\pgfsetlinewidth{1.003750pt}%
\definecolor{currentstroke}{rgb}{0.000000,0.000000,0.000000}%
\pgfsetstrokecolor{currentstroke}%
\pgfsetdash{}{0pt}%
\pgfpathmoveto{\pgfqpoint{2.518786in}{1.140308in}}%
\pgfpathcurveto{\pgfqpoint{2.529836in}{1.140308in}}{\pgfqpoint{2.540435in}{1.144698in}}{\pgfqpoint{2.548249in}{1.152512in}}%
\pgfpathcurveto{\pgfqpoint{2.556062in}{1.160326in}}{\pgfqpoint{2.560452in}{1.170925in}}{\pgfqpoint{2.560452in}{1.181975in}}%
\pgfpathcurveto{\pgfqpoint{2.560452in}{1.193025in}}{\pgfqpoint{2.556062in}{1.203624in}}{\pgfqpoint{2.548249in}{1.211438in}}%
\pgfpathcurveto{\pgfqpoint{2.540435in}{1.219251in}}{\pgfqpoint{2.529836in}{1.223642in}}{\pgfqpoint{2.518786in}{1.223642in}}%
\pgfpathcurveto{\pgfqpoint{2.507736in}{1.223642in}}{\pgfqpoint{2.497137in}{1.219251in}}{\pgfqpoint{2.489323in}{1.211438in}}%
\pgfpathcurveto{\pgfqpoint{2.481509in}{1.203624in}}{\pgfqpoint{2.477119in}{1.193025in}}{\pgfqpoint{2.477119in}{1.181975in}}%
\pgfpathcurveto{\pgfqpoint{2.477119in}{1.170925in}}{\pgfqpoint{2.481509in}{1.160326in}}{\pgfqpoint{2.489323in}{1.152512in}}%
\pgfpathcurveto{\pgfqpoint{2.497137in}{1.144698in}}{\pgfqpoint{2.507736in}{1.140308in}}{\pgfqpoint{2.518786in}{1.140308in}}%
\pgfpathclose%
\pgfusepath{stroke,fill}%
\end{pgfscope}%
\begin{pgfscope}%
\pgfpathrectangle{\pgfqpoint{0.800000in}{0.528000in}}{\pgfqpoint{4.960000in}{3.696000in}}%
\pgfusepath{clip}%
\pgfsetbuttcap%
\pgfsetroundjoin%
\definecolor{currentfill}{rgb}{0.000000,0.000000,0.000000}%
\pgfsetfillcolor{currentfill}%
\pgfsetlinewidth{1.003750pt}%
\definecolor{currentstroke}{rgb}{0.000000,0.000000,0.000000}%
\pgfsetstrokecolor{currentstroke}%
\pgfsetdash{}{0pt}%
\pgfpathmoveto{\pgfqpoint{2.518786in}{1.014092in}}%
\pgfpathcurveto{\pgfqpoint{2.529836in}{1.014092in}}{\pgfqpoint{2.540435in}{1.018482in}}{\pgfqpoint{2.548249in}{1.026295in}}%
\pgfpathcurveto{\pgfqpoint{2.556062in}{1.034109in}}{\pgfqpoint{2.560452in}{1.044708in}}{\pgfqpoint{2.560452in}{1.055758in}}%
\pgfpathcurveto{\pgfqpoint{2.560452in}{1.066808in}}{\pgfqpoint{2.556062in}{1.077407in}}{\pgfqpoint{2.548249in}{1.085221in}}%
\pgfpathcurveto{\pgfqpoint{2.540435in}{1.093035in}}{\pgfqpoint{2.529836in}{1.097425in}}{\pgfqpoint{2.518786in}{1.097425in}}%
\pgfpathcurveto{\pgfqpoint{2.507736in}{1.097425in}}{\pgfqpoint{2.497137in}{1.093035in}}{\pgfqpoint{2.489323in}{1.085221in}}%
\pgfpathcurveto{\pgfqpoint{2.481509in}{1.077407in}}{\pgfqpoint{2.477119in}{1.066808in}}{\pgfqpoint{2.477119in}{1.055758in}}%
\pgfpathcurveto{\pgfqpoint{2.477119in}{1.044708in}}{\pgfqpoint{2.481509in}{1.034109in}}{\pgfqpoint{2.489323in}{1.026295in}}%
\pgfpathcurveto{\pgfqpoint{2.497137in}{1.018482in}}{\pgfqpoint{2.507736in}{1.014092in}}{\pgfqpoint{2.518786in}{1.014092in}}%
\pgfpathclose%
\pgfusepath{stroke,fill}%
\end{pgfscope}%
\begin{pgfscope}%
\pgfpathrectangle{\pgfqpoint{0.800000in}{0.528000in}}{\pgfqpoint{4.960000in}{3.696000in}}%
\pgfusepath{clip}%
\pgfsetbuttcap%
\pgfsetroundjoin%
\definecolor{currentfill}{rgb}{0.000000,0.000000,0.000000}%
\pgfsetfillcolor{currentfill}%
\pgfsetlinewidth{1.003750pt}%
\definecolor{currentstroke}{rgb}{0.000000,0.000000,0.000000}%
\pgfsetstrokecolor{currentstroke}%
\pgfsetdash{}{0pt}%
\pgfpathmoveto{\pgfqpoint{2.518786in}{0.965547in}}%
\pgfpathcurveto{\pgfqpoint{2.529836in}{0.965547in}}{\pgfqpoint{2.540435in}{0.969937in}}{\pgfqpoint{2.548249in}{0.977751in}}%
\pgfpathcurveto{\pgfqpoint{2.556062in}{0.985564in}}{\pgfqpoint{2.560452in}{0.996163in}}{\pgfqpoint{2.560452in}{1.007213in}}%
\pgfpathcurveto{\pgfqpoint{2.560452in}{1.018264in}}{\pgfqpoint{2.556062in}{1.028863in}}{\pgfqpoint{2.548249in}{1.036676in}}%
\pgfpathcurveto{\pgfqpoint{2.540435in}{1.044490in}}{\pgfqpoint{2.529836in}{1.048880in}}{\pgfqpoint{2.518786in}{1.048880in}}%
\pgfpathcurveto{\pgfqpoint{2.507736in}{1.048880in}}{\pgfqpoint{2.497137in}{1.044490in}}{\pgfqpoint{2.489323in}{1.036676in}}%
\pgfpathcurveto{\pgfqpoint{2.481509in}{1.028863in}}{\pgfqpoint{2.477119in}{1.018264in}}{\pgfqpoint{2.477119in}{1.007213in}}%
\pgfpathcurveto{\pgfqpoint{2.477119in}{0.996163in}}{\pgfqpoint{2.481509in}{0.985564in}}{\pgfqpoint{2.489323in}{0.977751in}}%
\pgfpathcurveto{\pgfqpoint{2.497137in}{0.969937in}}{\pgfqpoint{2.507736in}{0.965547in}}{\pgfqpoint{2.518786in}{0.965547in}}%
\pgfpathclose%
\pgfusepath{stroke,fill}%
\end{pgfscope}%
\begin{pgfscope}%
\pgfpathrectangle{\pgfqpoint{0.800000in}{0.528000in}}{\pgfqpoint{4.960000in}{3.696000in}}%
\pgfusepath{clip}%
\pgfsetbuttcap%
\pgfsetroundjoin%
\definecolor{currentfill}{rgb}{0.000000,0.000000,0.000000}%
\pgfsetfillcolor{currentfill}%
\pgfsetlinewidth{1.003750pt}%
\definecolor{currentstroke}{rgb}{0.000000,0.000000,0.000000}%
\pgfsetstrokecolor{currentstroke}%
\pgfsetdash{}{0pt}%
\pgfpathmoveto{\pgfqpoint{2.518786in}{0.975256in}}%
\pgfpathcurveto{\pgfqpoint{2.529836in}{0.975256in}}{\pgfqpoint{2.540435in}{0.979646in}}{\pgfqpoint{2.548249in}{0.987460in}}%
\pgfpathcurveto{\pgfqpoint{2.556062in}{0.995273in}}{\pgfqpoint{2.560452in}{1.005872in}}{\pgfqpoint{2.560452in}{1.016922in}}%
\pgfpathcurveto{\pgfqpoint{2.560452in}{1.027973in}}{\pgfqpoint{2.556062in}{1.038572in}}{\pgfqpoint{2.548249in}{1.046385in}}%
\pgfpathcurveto{\pgfqpoint{2.540435in}{1.054199in}}{\pgfqpoint{2.529836in}{1.058589in}}{\pgfqpoint{2.518786in}{1.058589in}}%
\pgfpathcurveto{\pgfqpoint{2.507736in}{1.058589in}}{\pgfqpoint{2.497137in}{1.054199in}}{\pgfqpoint{2.489323in}{1.046385in}}%
\pgfpathcurveto{\pgfqpoint{2.481509in}{1.038572in}}{\pgfqpoint{2.477119in}{1.027973in}}{\pgfqpoint{2.477119in}{1.016922in}}%
\pgfpathcurveto{\pgfqpoint{2.477119in}{1.005872in}}{\pgfqpoint{2.481509in}{0.995273in}}{\pgfqpoint{2.489323in}{0.987460in}}%
\pgfpathcurveto{\pgfqpoint{2.497137in}{0.979646in}}{\pgfqpoint{2.507736in}{0.975256in}}{\pgfqpoint{2.518786in}{0.975256in}}%
\pgfpathclose%
\pgfusepath{stroke,fill}%
\end{pgfscope}%
\begin{pgfscope}%
\pgfpathrectangle{\pgfqpoint{0.800000in}{0.528000in}}{\pgfqpoint{4.960000in}{3.696000in}}%
\pgfusepath{clip}%
\pgfsetbuttcap%
\pgfsetroundjoin%
\definecolor{currentfill}{rgb}{0.000000,0.000000,0.000000}%
\pgfsetfillcolor{currentfill}%
\pgfsetlinewidth{1.003750pt}%
\definecolor{currentstroke}{rgb}{0.000000,0.000000,0.000000}%
\pgfsetstrokecolor{currentstroke}%
\pgfsetdash{}{0pt}%
\pgfpathmoveto{\pgfqpoint{2.518786in}{1.062636in}}%
\pgfpathcurveto{\pgfqpoint{2.529836in}{1.062636in}}{\pgfqpoint{2.540435in}{1.067027in}}{\pgfqpoint{2.548249in}{1.074840in}}%
\pgfpathcurveto{\pgfqpoint{2.556062in}{1.082654in}}{\pgfqpoint{2.560452in}{1.093253in}}{\pgfqpoint{2.560452in}{1.104303in}}%
\pgfpathcurveto{\pgfqpoint{2.560452in}{1.115353in}}{\pgfqpoint{2.556062in}{1.125952in}}{\pgfqpoint{2.548249in}{1.133766in}}%
\pgfpathcurveto{\pgfqpoint{2.540435in}{1.141580in}}{\pgfqpoint{2.529836in}{1.145970in}}{\pgfqpoint{2.518786in}{1.145970in}}%
\pgfpathcurveto{\pgfqpoint{2.507736in}{1.145970in}}{\pgfqpoint{2.497137in}{1.141580in}}{\pgfqpoint{2.489323in}{1.133766in}}%
\pgfpathcurveto{\pgfqpoint{2.481509in}{1.125952in}}{\pgfqpoint{2.477119in}{1.115353in}}{\pgfqpoint{2.477119in}{1.104303in}}%
\pgfpathcurveto{\pgfqpoint{2.477119in}{1.093253in}}{\pgfqpoint{2.481509in}{1.082654in}}{\pgfqpoint{2.489323in}{1.074840in}}%
\pgfpathcurveto{\pgfqpoint{2.497137in}{1.067027in}}{\pgfqpoint{2.507736in}{1.062636in}}{\pgfqpoint{2.518786in}{1.062636in}}%
\pgfpathclose%
\pgfusepath{stroke,fill}%
\end{pgfscope}%
\begin{pgfscope}%
\pgfpathrectangle{\pgfqpoint{0.800000in}{0.528000in}}{\pgfqpoint{4.960000in}{3.696000in}}%
\pgfusepath{clip}%
\pgfsetbuttcap%
\pgfsetroundjoin%
\definecolor{currentfill}{rgb}{0.000000,0.000000,0.000000}%
\pgfsetfillcolor{currentfill}%
\pgfsetlinewidth{1.003750pt}%
\definecolor{currentstroke}{rgb}{0.000000,0.000000,0.000000}%
\pgfsetstrokecolor{currentstroke}%
\pgfsetdash{}{0pt}%
\pgfpathmoveto{\pgfqpoint{2.518786in}{0.994674in}}%
\pgfpathcurveto{\pgfqpoint{2.529836in}{0.994674in}}{\pgfqpoint{2.540435in}{0.999064in}}{\pgfqpoint{2.548249in}{1.006878in}}%
\pgfpathcurveto{\pgfqpoint{2.556062in}{1.014691in}}{\pgfqpoint{2.560452in}{1.025290in}}{\pgfqpoint{2.560452in}{1.036340in}}%
\pgfpathcurveto{\pgfqpoint{2.560452in}{1.047390in}}{\pgfqpoint{2.556062in}{1.057989in}}{\pgfqpoint{2.548249in}{1.065803in}}%
\pgfpathcurveto{\pgfqpoint{2.540435in}{1.073617in}}{\pgfqpoint{2.529836in}{1.078007in}}{\pgfqpoint{2.518786in}{1.078007in}}%
\pgfpathcurveto{\pgfqpoint{2.507736in}{1.078007in}}{\pgfqpoint{2.497137in}{1.073617in}}{\pgfqpoint{2.489323in}{1.065803in}}%
\pgfpathcurveto{\pgfqpoint{2.481509in}{1.057989in}}{\pgfqpoint{2.477119in}{1.047390in}}{\pgfqpoint{2.477119in}{1.036340in}}%
\pgfpathcurveto{\pgfqpoint{2.477119in}{1.025290in}}{\pgfqpoint{2.481509in}{1.014691in}}{\pgfqpoint{2.489323in}{1.006878in}}%
\pgfpathcurveto{\pgfqpoint{2.497137in}{0.999064in}}{\pgfqpoint{2.507736in}{0.994674in}}{\pgfqpoint{2.518786in}{0.994674in}}%
\pgfpathclose%
\pgfusepath{stroke,fill}%
\end{pgfscope}%
\begin{pgfscope}%
\pgfpathrectangle{\pgfqpoint{0.800000in}{0.528000in}}{\pgfqpoint{4.960000in}{3.696000in}}%
\pgfusepath{clip}%
\pgfsetbuttcap%
\pgfsetroundjoin%
\definecolor{currentfill}{rgb}{0.000000,0.000000,0.000000}%
\pgfsetfillcolor{currentfill}%
\pgfsetlinewidth{1.003750pt}%
\definecolor{currentstroke}{rgb}{0.000000,0.000000,0.000000}%
\pgfsetstrokecolor{currentstroke}%
\pgfsetdash{}{0pt}%
\pgfpathmoveto{\pgfqpoint{2.518786in}{1.052927in}}%
\pgfpathcurveto{\pgfqpoint{2.529836in}{1.052927in}}{\pgfqpoint{2.540435in}{1.057318in}}{\pgfqpoint{2.548249in}{1.065131in}}%
\pgfpathcurveto{\pgfqpoint{2.556062in}{1.072945in}}{\pgfqpoint{2.560452in}{1.083544in}}{\pgfqpoint{2.560452in}{1.094594in}}%
\pgfpathcurveto{\pgfqpoint{2.560452in}{1.105644in}}{\pgfqpoint{2.556062in}{1.116243in}}{\pgfqpoint{2.548249in}{1.124057in}}%
\pgfpathcurveto{\pgfqpoint{2.540435in}{1.131871in}}{\pgfqpoint{2.529836in}{1.136261in}}{\pgfqpoint{2.518786in}{1.136261in}}%
\pgfpathcurveto{\pgfqpoint{2.507736in}{1.136261in}}{\pgfqpoint{2.497137in}{1.131871in}}{\pgfqpoint{2.489323in}{1.124057in}}%
\pgfpathcurveto{\pgfqpoint{2.481509in}{1.116243in}}{\pgfqpoint{2.477119in}{1.105644in}}{\pgfqpoint{2.477119in}{1.094594in}}%
\pgfpathcurveto{\pgfqpoint{2.477119in}{1.083544in}}{\pgfqpoint{2.481509in}{1.072945in}}{\pgfqpoint{2.489323in}{1.065131in}}%
\pgfpathcurveto{\pgfqpoint{2.497137in}{1.057318in}}{\pgfqpoint{2.507736in}{1.052927in}}{\pgfqpoint{2.518786in}{1.052927in}}%
\pgfpathclose%
\pgfusepath{stroke,fill}%
\end{pgfscope}%
\begin{pgfscope}%
\pgfpathrectangle{\pgfqpoint{0.800000in}{0.528000in}}{\pgfqpoint{4.960000in}{3.696000in}}%
\pgfusepath{clip}%
\pgfsetbuttcap%
\pgfsetroundjoin%
\definecolor{currentfill}{rgb}{0.000000,0.000000,0.000000}%
\pgfsetfillcolor{currentfill}%
\pgfsetlinewidth{1.003750pt}%
\definecolor{currentstroke}{rgb}{0.000000,0.000000,0.000000}%
\pgfsetstrokecolor{currentstroke}%
\pgfsetdash{}{0pt}%
\pgfpathmoveto{\pgfqpoint{2.518786in}{0.955838in}}%
\pgfpathcurveto{\pgfqpoint{2.529836in}{0.955838in}}{\pgfqpoint{2.540435in}{0.960228in}}{\pgfqpoint{2.548249in}{0.968042in}}%
\pgfpathcurveto{\pgfqpoint{2.556062in}{0.975855in}}{\pgfqpoint{2.560452in}{0.986454in}}{\pgfqpoint{2.560452in}{0.997504in}}%
\pgfpathcurveto{\pgfqpoint{2.560452in}{1.008555in}}{\pgfqpoint{2.556062in}{1.019154in}}{\pgfqpoint{2.548249in}{1.026967in}}%
\pgfpathcurveto{\pgfqpoint{2.540435in}{1.034781in}}{\pgfqpoint{2.529836in}{1.039171in}}{\pgfqpoint{2.518786in}{1.039171in}}%
\pgfpathcurveto{\pgfqpoint{2.507736in}{1.039171in}}{\pgfqpoint{2.497137in}{1.034781in}}{\pgfqpoint{2.489323in}{1.026967in}}%
\pgfpathcurveto{\pgfqpoint{2.481509in}{1.019154in}}{\pgfqpoint{2.477119in}{1.008555in}}{\pgfqpoint{2.477119in}{0.997504in}}%
\pgfpathcurveto{\pgfqpoint{2.477119in}{0.986454in}}{\pgfqpoint{2.481509in}{0.975855in}}{\pgfqpoint{2.489323in}{0.968042in}}%
\pgfpathcurveto{\pgfqpoint{2.497137in}{0.960228in}}{\pgfqpoint{2.507736in}{0.955838in}}{\pgfqpoint{2.518786in}{0.955838in}}%
\pgfpathclose%
\pgfusepath{stroke,fill}%
\end{pgfscope}%
\begin{pgfscope}%
\pgfpathrectangle{\pgfqpoint{0.800000in}{0.528000in}}{\pgfqpoint{4.960000in}{3.696000in}}%
\pgfusepath{clip}%
\pgfsetbuttcap%
\pgfsetroundjoin%
\definecolor{currentfill}{rgb}{0.000000,0.000000,0.000000}%
\pgfsetfillcolor{currentfill}%
\pgfsetlinewidth{1.003750pt}%
\definecolor{currentstroke}{rgb}{0.000000,0.000000,0.000000}%
\pgfsetstrokecolor{currentstroke}%
\pgfsetdash{}{0pt}%
\pgfpathmoveto{\pgfqpoint{2.518786in}{1.082054in}}%
\pgfpathcurveto{\pgfqpoint{2.529836in}{1.082054in}}{\pgfqpoint{2.540435in}{1.086445in}}{\pgfqpoint{2.548249in}{1.094258in}}%
\pgfpathcurveto{\pgfqpoint{2.556062in}{1.102072in}}{\pgfqpoint{2.560452in}{1.112671in}}{\pgfqpoint{2.560452in}{1.123721in}}%
\pgfpathcurveto{\pgfqpoint{2.560452in}{1.134771in}}{\pgfqpoint{2.556062in}{1.145370in}}{\pgfqpoint{2.548249in}{1.153184in}}%
\pgfpathcurveto{\pgfqpoint{2.540435in}{1.160997in}}{\pgfqpoint{2.529836in}{1.165388in}}{\pgfqpoint{2.518786in}{1.165388in}}%
\pgfpathcurveto{\pgfqpoint{2.507736in}{1.165388in}}{\pgfqpoint{2.497137in}{1.160997in}}{\pgfqpoint{2.489323in}{1.153184in}}%
\pgfpathcurveto{\pgfqpoint{2.481509in}{1.145370in}}{\pgfqpoint{2.477119in}{1.134771in}}{\pgfqpoint{2.477119in}{1.123721in}}%
\pgfpathcurveto{\pgfqpoint{2.477119in}{1.112671in}}{\pgfqpoint{2.481509in}{1.102072in}}{\pgfqpoint{2.489323in}{1.094258in}}%
\pgfpathcurveto{\pgfqpoint{2.497137in}{1.086445in}}{\pgfqpoint{2.507736in}{1.082054in}}{\pgfqpoint{2.518786in}{1.082054in}}%
\pgfpathclose%
\pgfusepath{stroke,fill}%
\end{pgfscope}%
\begin{pgfscope}%
\pgfpathrectangle{\pgfqpoint{0.800000in}{0.528000in}}{\pgfqpoint{4.960000in}{3.696000in}}%
\pgfusepath{clip}%
\pgfsetbuttcap%
\pgfsetroundjoin%
\definecolor{currentfill}{rgb}{0.000000,0.000000,0.000000}%
\pgfsetfillcolor{currentfill}%
\pgfsetlinewidth{1.003750pt}%
\definecolor{currentstroke}{rgb}{0.000000,0.000000,0.000000}%
\pgfsetstrokecolor{currentstroke}%
\pgfsetdash{}{0pt}%
\pgfpathmoveto{\pgfqpoint{2.518786in}{1.043219in}}%
\pgfpathcurveto{\pgfqpoint{2.529836in}{1.043219in}}{\pgfqpoint{2.540435in}{1.047609in}}{\pgfqpoint{2.548249in}{1.055422in}}%
\pgfpathcurveto{\pgfqpoint{2.556062in}{1.063236in}}{\pgfqpoint{2.560452in}{1.073835in}}{\pgfqpoint{2.560452in}{1.084885in}}%
\pgfpathcurveto{\pgfqpoint{2.560452in}{1.095935in}}{\pgfqpoint{2.556062in}{1.106534in}}{\pgfqpoint{2.548249in}{1.114348in}}%
\pgfpathcurveto{\pgfqpoint{2.540435in}{1.122162in}}{\pgfqpoint{2.529836in}{1.126552in}}{\pgfqpoint{2.518786in}{1.126552in}}%
\pgfpathcurveto{\pgfqpoint{2.507736in}{1.126552in}}{\pgfqpoint{2.497137in}{1.122162in}}{\pgfqpoint{2.489323in}{1.114348in}}%
\pgfpathcurveto{\pgfqpoint{2.481509in}{1.106534in}}{\pgfqpoint{2.477119in}{1.095935in}}{\pgfqpoint{2.477119in}{1.084885in}}%
\pgfpathcurveto{\pgfqpoint{2.477119in}{1.073835in}}{\pgfqpoint{2.481509in}{1.063236in}}{\pgfqpoint{2.489323in}{1.055422in}}%
\pgfpathcurveto{\pgfqpoint{2.497137in}{1.047609in}}{\pgfqpoint{2.507736in}{1.043219in}}{\pgfqpoint{2.518786in}{1.043219in}}%
\pgfpathclose%
\pgfusepath{stroke,fill}%
\end{pgfscope}%
\begin{pgfscope}%
\pgfpathrectangle{\pgfqpoint{0.800000in}{0.528000in}}{\pgfqpoint{4.960000in}{3.696000in}}%
\pgfusepath{clip}%
\pgfsetbuttcap%
\pgfsetroundjoin%
\definecolor{currentfill}{rgb}{0.000000,0.000000,0.000000}%
\pgfsetfillcolor{currentfill}%
\pgfsetlinewidth{1.003750pt}%
\definecolor{currentstroke}{rgb}{0.000000,0.000000,0.000000}%
\pgfsetstrokecolor{currentstroke}%
\pgfsetdash{}{0pt}%
\pgfpathmoveto{\pgfqpoint{2.518786in}{1.043219in}}%
\pgfpathcurveto{\pgfqpoint{2.529836in}{1.043219in}}{\pgfqpoint{2.540435in}{1.047609in}}{\pgfqpoint{2.548249in}{1.055422in}}%
\pgfpathcurveto{\pgfqpoint{2.556062in}{1.063236in}}{\pgfqpoint{2.560452in}{1.073835in}}{\pgfqpoint{2.560452in}{1.084885in}}%
\pgfpathcurveto{\pgfqpoint{2.560452in}{1.095935in}}{\pgfqpoint{2.556062in}{1.106534in}}{\pgfqpoint{2.548249in}{1.114348in}}%
\pgfpathcurveto{\pgfqpoint{2.540435in}{1.122162in}}{\pgfqpoint{2.529836in}{1.126552in}}{\pgfqpoint{2.518786in}{1.126552in}}%
\pgfpathcurveto{\pgfqpoint{2.507736in}{1.126552in}}{\pgfqpoint{2.497137in}{1.122162in}}{\pgfqpoint{2.489323in}{1.114348in}}%
\pgfpathcurveto{\pgfqpoint{2.481509in}{1.106534in}}{\pgfqpoint{2.477119in}{1.095935in}}{\pgfqpoint{2.477119in}{1.084885in}}%
\pgfpathcurveto{\pgfqpoint{2.477119in}{1.073835in}}{\pgfqpoint{2.481509in}{1.063236in}}{\pgfqpoint{2.489323in}{1.055422in}}%
\pgfpathcurveto{\pgfqpoint{2.497137in}{1.047609in}}{\pgfqpoint{2.507736in}{1.043219in}}{\pgfqpoint{2.518786in}{1.043219in}}%
\pgfpathclose%
\pgfusepath{stroke,fill}%
\end{pgfscope}%
\begin{pgfscope}%
\pgfpathrectangle{\pgfqpoint{0.800000in}{0.528000in}}{\pgfqpoint{4.960000in}{3.696000in}}%
\pgfusepath{clip}%
\pgfsetbuttcap%
\pgfsetroundjoin%
\definecolor{currentfill}{rgb}{0.000000,0.000000,0.000000}%
\pgfsetfillcolor{currentfill}%
\pgfsetlinewidth{1.003750pt}%
\definecolor{currentstroke}{rgb}{0.000000,0.000000,0.000000}%
\pgfsetstrokecolor{currentstroke}%
\pgfsetdash{}{0pt}%
\pgfpathmoveto{\pgfqpoint{2.518786in}{0.994674in}}%
\pgfpathcurveto{\pgfqpoint{2.529836in}{0.994674in}}{\pgfqpoint{2.540435in}{0.999064in}}{\pgfqpoint{2.548249in}{1.006878in}}%
\pgfpathcurveto{\pgfqpoint{2.556062in}{1.014691in}}{\pgfqpoint{2.560452in}{1.025290in}}{\pgfqpoint{2.560452in}{1.036340in}}%
\pgfpathcurveto{\pgfqpoint{2.560452in}{1.047390in}}{\pgfqpoint{2.556062in}{1.057989in}}{\pgfqpoint{2.548249in}{1.065803in}}%
\pgfpathcurveto{\pgfqpoint{2.540435in}{1.073617in}}{\pgfqpoint{2.529836in}{1.078007in}}{\pgfqpoint{2.518786in}{1.078007in}}%
\pgfpathcurveto{\pgfqpoint{2.507736in}{1.078007in}}{\pgfqpoint{2.497137in}{1.073617in}}{\pgfqpoint{2.489323in}{1.065803in}}%
\pgfpathcurveto{\pgfqpoint{2.481509in}{1.057989in}}{\pgfqpoint{2.477119in}{1.047390in}}{\pgfqpoint{2.477119in}{1.036340in}}%
\pgfpathcurveto{\pgfqpoint{2.477119in}{1.025290in}}{\pgfqpoint{2.481509in}{1.014691in}}{\pgfqpoint{2.489323in}{1.006878in}}%
\pgfpathcurveto{\pgfqpoint{2.497137in}{0.999064in}}{\pgfqpoint{2.507736in}{0.994674in}}{\pgfqpoint{2.518786in}{0.994674in}}%
\pgfpathclose%
\pgfusepath{stroke,fill}%
\end{pgfscope}%
\begin{pgfscope}%
\pgfpathrectangle{\pgfqpoint{0.800000in}{0.528000in}}{\pgfqpoint{4.960000in}{3.696000in}}%
\pgfusepath{clip}%
\pgfsetbuttcap%
\pgfsetroundjoin%
\definecolor{currentfill}{rgb}{0.000000,0.000000,0.000000}%
\pgfsetfillcolor{currentfill}%
\pgfsetlinewidth{1.003750pt}%
\definecolor{currentstroke}{rgb}{0.000000,0.000000,0.000000}%
\pgfsetstrokecolor{currentstroke}%
\pgfsetdash{}{0pt}%
\pgfpathmoveto{\pgfqpoint{2.518786in}{1.101472in}}%
\pgfpathcurveto{\pgfqpoint{2.529836in}{1.101472in}}{\pgfqpoint{2.540435in}{1.105863in}}{\pgfqpoint{2.548249in}{1.113676in}}%
\pgfpathcurveto{\pgfqpoint{2.556062in}{1.121490in}}{\pgfqpoint{2.560452in}{1.132089in}}{\pgfqpoint{2.560452in}{1.143139in}}%
\pgfpathcurveto{\pgfqpoint{2.560452in}{1.154189in}}{\pgfqpoint{2.556062in}{1.164788in}}{\pgfqpoint{2.548249in}{1.172602in}}%
\pgfpathcurveto{\pgfqpoint{2.540435in}{1.180415in}}{\pgfqpoint{2.529836in}{1.184806in}}{\pgfqpoint{2.518786in}{1.184806in}}%
\pgfpathcurveto{\pgfqpoint{2.507736in}{1.184806in}}{\pgfqpoint{2.497137in}{1.180415in}}{\pgfqpoint{2.489323in}{1.172602in}}%
\pgfpathcurveto{\pgfqpoint{2.481509in}{1.164788in}}{\pgfqpoint{2.477119in}{1.154189in}}{\pgfqpoint{2.477119in}{1.143139in}}%
\pgfpathcurveto{\pgfqpoint{2.477119in}{1.132089in}}{\pgfqpoint{2.481509in}{1.121490in}}{\pgfqpoint{2.489323in}{1.113676in}}%
\pgfpathcurveto{\pgfqpoint{2.497137in}{1.105863in}}{\pgfqpoint{2.507736in}{1.101472in}}{\pgfqpoint{2.518786in}{1.101472in}}%
\pgfpathclose%
\pgfusepath{stroke,fill}%
\end{pgfscope}%
\begin{pgfscope}%
\pgfpathrectangle{\pgfqpoint{0.800000in}{0.528000in}}{\pgfqpoint{4.960000in}{3.696000in}}%
\pgfusepath{clip}%
\pgfsetbuttcap%
\pgfsetroundjoin%
\definecolor{currentfill}{rgb}{0.000000,0.000000,0.000000}%
\pgfsetfillcolor{currentfill}%
\pgfsetlinewidth{1.003750pt}%
\definecolor{currentstroke}{rgb}{0.000000,0.000000,0.000000}%
\pgfsetstrokecolor{currentstroke}%
\pgfsetdash{}{0pt}%
\pgfpathmoveto{\pgfqpoint{2.518786in}{0.975256in}}%
\pgfpathcurveto{\pgfqpoint{2.529836in}{0.975256in}}{\pgfqpoint{2.540435in}{0.979646in}}{\pgfqpoint{2.548249in}{0.987460in}}%
\pgfpathcurveto{\pgfqpoint{2.556062in}{0.995273in}}{\pgfqpoint{2.560452in}{1.005872in}}{\pgfqpoint{2.560452in}{1.016922in}}%
\pgfpathcurveto{\pgfqpoint{2.560452in}{1.027973in}}{\pgfqpoint{2.556062in}{1.038572in}}{\pgfqpoint{2.548249in}{1.046385in}}%
\pgfpathcurveto{\pgfqpoint{2.540435in}{1.054199in}}{\pgfqpoint{2.529836in}{1.058589in}}{\pgfqpoint{2.518786in}{1.058589in}}%
\pgfpathcurveto{\pgfqpoint{2.507736in}{1.058589in}}{\pgfqpoint{2.497137in}{1.054199in}}{\pgfqpoint{2.489323in}{1.046385in}}%
\pgfpathcurveto{\pgfqpoint{2.481509in}{1.038572in}}{\pgfqpoint{2.477119in}{1.027973in}}{\pgfqpoint{2.477119in}{1.016922in}}%
\pgfpathcurveto{\pgfqpoint{2.477119in}{1.005872in}}{\pgfqpoint{2.481509in}{0.995273in}}{\pgfqpoint{2.489323in}{0.987460in}}%
\pgfpathcurveto{\pgfqpoint{2.497137in}{0.979646in}}{\pgfqpoint{2.507736in}{0.975256in}}{\pgfqpoint{2.518786in}{0.975256in}}%
\pgfpathclose%
\pgfusepath{stroke,fill}%
\end{pgfscope}%
\begin{pgfscope}%
\pgfpathrectangle{\pgfqpoint{0.800000in}{0.528000in}}{\pgfqpoint{4.960000in}{3.696000in}}%
\pgfusepath{clip}%
\pgfsetbuttcap%
\pgfsetroundjoin%
\definecolor{currentfill}{rgb}{0.000000,0.000000,0.000000}%
\pgfsetfillcolor{currentfill}%
\pgfsetlinewidth{1.003750pt}%
\definecolor{currentstroke}{rgb}{0.000000,0.000000,0.000000}%
\pgfsetstrokecolor{currentstroke}%
\pgfsetdash{}{0pt}%
\pgfpathmoveto{\pgfqpoint{2.518786in}{0.984965in}}%
\pgfpathcurveto{\pgfqpoint{2.529836in}{0.984965in}}{\pgfqpoint{2.540435in}{0.989355in}}{\pgfqpoint{2.548249in}{0.997169in}}%
\pgfpathcurveto{\pgfqpoint{2.556062in}{1.004982in}}{\pgfqpoint{2.560452in}{1.015581in}}{\pgfqpoint{2.560452in}{1.026631in}}%
\pgfpathcurveto{\pgfqpoint{2.560452in}{1.037681in}}{\pgfqpoint{2.556062in}{1.048281in}}{\pgfqpoint{2.548249in}{1.056094in}}%
\pgfpathcurveto{\pgfqpoint{2.540435in}{1.063908in}}{\pgfqpoint{2.529836in}{1.068298in}}{\pgfqpoint{2.518786in}{1.068298in}}%
\pgfpathcurveto{\pgfqpoint{2.507736in}{1.068298in}}{\pgfqpoint{2.497137in}{1.063908in}}{\pgfqpoint{2.489323in}{1.056094in}}%
\pgfpathcurveto{\pgfqpoint{2.481509in}{1.048281in}}{\pgfqpoint{2.477119in}{1.037681in}}{\pgfqpoint{2.477119in}{1.026631in}}%
\pgfpathcurveto{\pgfqpoint{2.477119in}{1.015581in}}{\pgfqpoint{2.481509in}{1.004982in}}{\pgfqpoint{2.489323in}{0.997169in}}%
\pgfpathcurveto{\pgfqpoint{2.497137in}{0.989355in}}{\pgfqpoint{2.507736in}{0.984965in}}{\pgfqpoint{2.518786in}{0.984965in}}%
\pgfpathclose%
\pgfusepath{stroke,fill}%
\end{pgfscope}%
\begin{pgfscope}%
\pgfpathrectangle{\pgfqpoint{0.800000in}{0.528000in}}{\pgfqpoint{4.960000in}{3.696000in}}%
\pgfusepath{clip}%
\pgfsetbuttcap%
\pgfsetroundjoin%
\definecolor{currentfill}{rgb}{0.000000,0.000000,0.000000}%
\pgfsetfillcolor{currentfill}%
\pgfsetlinewidth{1.003750pt}%
\definecolor{currentstroke}{rgb}{0.000000,0.000000,0.000000}%
\pgfsetstrokecolor{currentstroke}%
\pgfsetdash{}{0pt}%
\pgfpathmoveto{\pgfqpoint{2.518786in}{1.023801in}}%
\pgfpathcurveto{\pgfqpoint{2.529836in}{1.023801in}}{\pgfqpoint{2.540435in}{1.028191in}}{\pgfqpoint{2.548249in}{1.036004in}}%
\pgfpathcurveto{\pgfqpoint{2.556062in}{1.043818in}}{\pgfqpoint{2.560452in}{1.054417in}}{\pgfqpoint{2.560452in}{1.065467in}}%
\pgfpathcurveto{\pgfqpoint{2.560452in}{1.076517in}}{\pgfqpoint{2.556062in}{1.087116in}}{\pgfqpoint{2.548249in}{1.094930in}}%
\pgfpathcurveto{\pgfqpoint{2.540435in}{1.102744in}}{\pgfqpoint{2.529836in}{1.107134in}}{\pgfqpoint{2.518786in}{1.107134in}}%
\pgfpathcurveto{\pgfqpoint{2.507736in}{1.107134in}}{\pgfqpoint{2.497137in}{1.102744in}}{\pgfqpoint{2.489323in}{1.094930in}}%
\pgfpathcurveto{\pgfqpoint{2.481509in}{1.087116in}}{\pgfqpoint{2.477119in}{1.076517in}}{\pgfqpoint{2.477119in}{1.065467in}}%
\pgfpathcurveto{\pgfqpoint{2.477119in}{1.054417in}}{\pgfqpoint{2.481509in}{1.043818in}}{\pgfqpoint{2.489323in}{1.036004in}}%
\pgfpathcurveto{\pgfqpoint{2.497137in}{1.028191in}}{\pgfqpoint{2.507736in}{1.023801in}}{\pgfqpoint{2.518786in}{1.023801in}}%
\pgfpathclose%
\pgfusepath{stroke,fill}%
\end{pgfscope}%
\begin{pgfscope}%
\pgfpathrectangle{\pgfqpoint{0.800000in}{0.528000in}}{\pgfqpoint{4.960000in}{3.696000in}}%
\pgfusepath{clip}%
\pgfsetbuttcap%
\pgfsetroundjoin%
\definecolor{currentfill}{rgb}{0.000000,0.000000,0.000000}%
\pgfsetfillcolor{currentfill}%
\pgfsetlinewidth{1.003750pt}%
\definecolor{currentstroke}{rgb}{0.000000,0.000000,0.000000}%
\pgfsetstrokecolor{currentstroke}%
\pgfsetdash{}{0pt}%
\pgfpathmoveto{\pgfqpoint{2.518786in}{1.033510in}}%
\pgfpathcurveto{\pgfqpoint{2.529836in}{1.033510in}}{\pgfqpoint{2.540435in}{1.037900in}}{\pgfqpoint{2.548249in}{1.045713in}}%
\pgfpathcurveto{\pgfqpoint{2.556062in}{1.053527in}}{\pgfqpoint{2.560452in}{1.064126in}}{\pgfqpoint{2.560452in}{1.075176in}}%
\pgfpathcurveto{\pgfqpoint{2.560452in}{1.086226in}}{\pgfqpoint{2.556062in}{1.096825in}}{\pgfqpoint{2.548249in}{1.104639in}}%
\pgfpathcurveto{\pgfqpoint{2.540435in}{1.112453in}}{\pgfqpoint{2.529836in}{1.116843in}}{\pgfqpoint{2.518786in}{1.116843in}}%
\pgfpathcurveto{\pgfqpoint{2.507736in}{1.116843in}}{\pgfqpoint{2.497137in}{1.112453in}}{\pgfqpoint{2.489323in}{1.104639in}}%
\pgfpathcurveto{\pgfqpoint{2.481509in}{1.096825in}}{\pgfqpoint{2.477119in}{1.086226in}}{\pgfqpoint{2.477119in}{1.075176in}}%
\pgfpathcurveto{\pgfqpoint{2.477119in}{1.064126in}}{\pgfqpoint{2.481509in}{1.053527in}}{\pgfqpoint{2.489323in}{1.045713in}}%
\pgfpathcurveto{\pgfqpoint{2.497137in}{1.037900in}}{\pgfqpoint{2.507736in}{1.033510in}}{\pgfqpoint{2.518786in}{1.033510in}}%
\pgfpathclose%
\pgfusepath{stroke,fill}%
\end{pgfscope}%
\begin{pgfscope}%
\pgfpathrectangle{\pgfqpoint{0.800000in}{0.528000in}}{\pgfqpoint{4.960000in}{3.696000in}}%
\pgfusepath{clip}%
\pgfsetbuttcap%
\pgfsetroundjoin%
\definecolor{currentfill}{rgb}{0.000000,0.000000,0.000000}%
\pgfsetfillcolor{currentfill}%
\pgfsetlinewidth{1.003750pt}%
\definecolor{currentstroke}{rgb}{0.000000,0.000000,0.000000}%
\pgfsetstrokecolor{currentstroke}%
\pgfsetdash{}{0pt}%
\pgfpathmoveto{\pgfqpoint{2.518786in}{0.936420in}}%
\pgfpathcurveto{\pgfqpoint{2.529836in}{0.936420in}}{\pgfqpoint{2.540435in}{0.940810in}}{\pgfqpoint{2.548249in}{0.948624in}}%
\pgfpathcurveto{\pgfqpoint{2.556062in}{0.956437in}}{\pgfqpoint{2.560452in}{0.967036in}}{\pgfqpoint{2.560452in}{0.978087in}}%
\pgfpathcurveto{\pgfqpoint{2.560452in}{0.989137in}}{\pgfqpoint{2.556062in}{0.999736in}}{\pgfqpoint{2.548249in}{1.007549in}}%
\pgfpathcurveto{\pgfqpoint{2.540435in}{1.015363in}}{\pgfqpoint{2.529836in}{1.019753in}}{\pgfqpoint{2.518786in}{1.019753in}}%
\pgfpathcurveto{\pgfqpoint{2.507736in}{1.019753in}}{\pgfqpoint{2.497137in}{1.015363in}}{\pgfqpoint{2.489323in}{1.007549in}}%
\pgfpathcurveto{\pgfqpoint{2.481509in}{0.999736in}}{\pgfqpoint{2.477119in}{0.989137in}}{\pgfqpoint{2.477119in}{0.978087in}}%
\pgfpathcurveto{\pgfqpoint{2.477119in}{0.967036in}}{\pgfqpoint{2.481509in}{0.956437in}}{\pgfqpoint{2.489323in}{0.948624in}}%
\pgfpathcurveto{\pgfqpoint{2.497137in}{0.940810in}}{\pgfqpoint{2.507736in}{0.936420in}}{\pgfqpoint{2.518786in}{0.936420in}}%
\pgfpathclose%
\pgfusepath{stroke,fill}%
\end{pgfscope}%
\begin{pgfscope}%
\pgfpathrectangle{\pgfqpoint{0.800000in}{0.528000in}}{\pgfqpoint{4.960000in}{3.696000in}}%
\pgfusepath{clip}%
\pgfsetbuttcap%
\pgfsetroundjoin%
\definecolor{currentfill}{rgb}{0.000000,0.000000,0.000000}%
\pgfsetfillcolor{currentfill}%
\pgfsetlinewidth{1.003750pt}%
\definecolor{currentstroke}{rgb}{0.000000,0.000000,0.000000}%
\pgfsetstrokecolor{currentstroke}%
\pgfsetdash{}{0pt}%
\pgfpathmoveto{\pgfqpoint{4.011666in}{1.625757in}}%
\pgfpathcurveto{\pgfqpoint{4.022716in}{1.625757in}}{\pgfqpoint{4.033315in}{1.630147in}}{\pgfqpoint{4.041128in}{1.637961in}}%
\pgfpathcurveto{\pgfqpoint{4.048942in}{1.645774in}}{\pgfqpoint{4.053332in}{1.656373in}}{\pgfqpoint{4.053332in}{1.667423in}}%
\pgfpathcurveto{\pgfqpoint{4.053332in}{1.678473in}}{\pgfqpoint{4.048942in}{1.689072in}}{\pgfqpoint{4.041128in}{1.696886in}}%
\pgfpathcurveto{\pgfqpoint{4.033315in}{1.704700in}}{\pgfqpoint{4.022716in}{1.709090in}}{\pgfqpoint{4.011666in}{1.709090in}}%
\pgfpathcurveto{\pgfqpoint{4.000616in}{1.709090in}}{\pgfqpoint{3.990016in}{1.704700in}}{\pgfqpoint{3.982203in}{1.696886in}}%
\pgfpathcurveto{\pgfqpoint{3.974389in}{1.689072in}}{\pgfqpoint{3.969999in}{1.678473in}}{\pgfqpoint{3.969999in}{1.667423in}}%
\pgfpathcurveto{\pgfqpoint{3.969999in}{1.656373in}}{\pgfqpoint{3.974389in}{1.645774in}}{\pgfqpoint{3.982203in}{1.637961in}}%
\pgfpathcurveto{\pgfqpoint{3.990016in}{1.630147in}}{\pgfqpoint{4.000616in}{1.625757in}}{\pgfqpoint{4.011666in}{1.625757in}}%
\pgfpathclose%
\pgfusepath{stroke,fill}%
\end{pgfscope}%
\begin{pgfscope}%
\pgfpathrectangle{\pgfqpoint{0.800000in}{0.528000in}}{\pgfqpoint{4.960000in}{3.696000in}}%
\pgfusepath{clip}%
\pgfsetbuttcap%
\pgfsetroundjoin%
\definecolor{currentfill}{rgb}{0.000000,0.000000,0.000000}%
\pgfsetfillcolor{currentfill}%
\pgfsetlinewidth{1.003750pt}%
\definecolor{currentstroke}{rgb}{0.000000,0.000000,0.000000}%
\pgfsetstrokecolor{currentstroke}%
\pgfsetdash{}{0pt}%
\pgfpathmoveto{\pgfqpoint{4.011666in}{1.596630in}}%
\pgfpathcurveto{\pgfqpoint{4.022716in}{1.596630in}}{\pgfqpoint{4.033315in}{1.601020in}}{\pgfqpoint{4.041128in}{1.608834in}}%
\pgfpathcurveto{\pgfqpoint{4.048942in}{1.616647in}}{\pgfqpoint{4.053332in}{1.627246in}}{\pgfqpoint{4.053332in}{1.638296in}}%
\pgfpathcurveto{\pgfqpoint{4.053332in}{1.649347in}}{\pgfqpoint{4.048942in}{1.659946in}}{\pgfqpoint{4.041128in}{1.667759in}}%
\pgfpathcurveto{\pgfqpoint{4.033315in}{1.675573in}}{\pgfqpoint{4.022716in}{1.679963in}}{\pgfqpoint{4.011666in}{1.679963in}}%
\pgfpathcurveto{\pgfqpoint{4.000616in}{1.679963in}}{\pgfqpoint{3.990016in}{1.675573in}}{\pgfqpoint{3.982203in}{1.667759in}}%
\pgfpathcurveto{\pgfqpoint{3.974389in}{1.659946in}}{\pgfqpoint{3.969999in}{1.649347in}}{\pgfqpoint{3.969999in}{1.638296in}}%
\pgfpathcurveto{\pgfqpoint{3.969999in}{1.627246in}}{\pgfqpoint{3.974389in}{1.616647in}}{\pgfqpoint{3.982203in}{1.608834in}}%
\pgfpathcurveto{\pgfqpoint{3.990016in}{1.601020in}}{\pgfqpoint{4.000616in}{1.596630in}}{\pgfqpoint{4.011666in}{1.596630in}}%
\pgfpathclose%
\pgfusepath{stroke,fill}%
\end{pgfscope}%
\begin{pgfscope}%
\pgfpathrectangle{\pgfqpoint{0.800000in}{0.528000in}}{\pgfqpoint{4.960000in}{3.696000in}}%
\pgfusepath{clip}%
\pgfsetbuttcap%
\pgfsetroundjoin%
\definecolor{currentfill}{rgb}{0.000000,0.000000,0.000000}%
\pgfsetfillcolor{currentfill}%
\pgfsetlinewidth{1.003750pt}%
\definecolor{currentstroke}{rgb}{0.000000,0.000000,0.000000}%
\pgfsetstrokecolor{currentstroke}%
\pgfsetdash{}{0pt}%
\pgfpathmoveto{\pgfqpoint{4.011666in}{1.557794in}}%
\pgfpathcurveto{\pgfqpoint{4.022716in}{1.557794in}}{\pgfqpoint{4.033315in}{1.562184in}}{\pgfqpoint{4.041128in}{1.569998in}}%
\pgfpathcurveto{\pgfqpoint{4.048942in}{1.577811in}}{\pgfqpoint{4.053332in}{1.588410in}}{\pgfqpoint{4.053332in}{1.599461in}}%
\pgfpathcurveto{\pgfqpoint{4.053332in}{1.610511in}}{\pgfqpoint{4.048942in}{1.621110in}}{\pgfqpoint{4.041128in}{1.628923in}}%
\pgfpathcurveto{\pgfqpoint{4.033315in}{1.636737in}}{\pgfqpoint{4.022716in}{1.641127in}}{\pgfqpoint{4.011666in}{1.641127in}}%
\pgfpathcurveto{\pgfqpoint{4.000616in}{1.641127in}}{\pgfqpoint{3.990016in}{1.636737in}}{\pgfqpoint{3.982203in}{1.628923in}}%
\pgfpathcurveto{\pgfqpoint{3.974389in}{1.621110in}}{\pgfqpoint{3.969999in}{1.610511in}}{\pgfqpoint{3.969999in}{1.599461in}}%
\pgfpathcurveto{\pgfqpoint{3.969999in}{1.588410in}}{\pgfqpoint{3.974389in}{1.577811in}}{\pgfqpoint{3.982203in}{1.569998in}}%
\pgfpathcurveto{\pgfqpoint{3.990016in}{1.562184in}}{\pgfqpoint{4.000616in}{1.557794in}}{\pgfqpoint{4.011666in}{1.557794in}}%
\pgfpathclose%
\pgfusepath{stroke,fill}%
\end{pgfscope}%
\begin{pgfscope}%
\pgfpathrectangle{\pgfqpoint{0.800000in}{0.528000in}}{\pgfqpoint{4.960000in}{3.696000in}}%
\pgfusepath{clip}%
\pgfsetbuttcap%
\pgfsetroundjoin%
\definecolor{currentfill}{rgb}{0.000000,0.000000,0.000000}%
\pgfsetfillcolor{currentfill}%
\pgfsetlinewidth{1.003750pt}%
\definecolor{currentstroke}{rgb}{0.000000,0.000000,0.000000}%
\pgfsetstrokecolor{currentstroke}%
\pgfsetdash{}{0pt}%
\pgfpathmoveto{\pgfqpoint{4.011666in}{1.751973in}}%
\pgfpathcurveto{\pgfqpoint{4.022716in}{1.751973in}}{\pgfqpoint{4.033315in}{1.756364in}}{\pgfqpoint{4.041128in}{1.764177in}}%
\pgfpathcurveto{\pgfqpoint{4.048942in}{1.771991in}}{\pgfqpoint{4.053332in}{1.782590in}}{\pgfqpoint{4.053332in}{1.793640in}}%
\pgfpathcurveto{\pgfqpoint{4.053332in}{1.804690in}}{\pgfqpoint{4.048942in}{1.815289in}}{\pgfqpoint{4.041128in}{1.823103in}}%
\pgfpathcurveto{\pgfqpoint{4.033315in}{1.830916in}}{\pgfqpoint{4.022716in}{1.835307in}}{\pgfqpoint{4.011666in}{1.835307in}}%
\pgfpathcurveto{\pgfqpoint{4.000616in}{1.835307in}}{\pgfqpoint{3.990016in}{1.830916in}}{\pgfqpoint{3.982203in}{1.823103in}}%
\pgfpathcurveto{\pgfqpoint{3.974389in}{1.815289in}}{\pgfqpoint{3.969999in}{1.804690in}}{\pgfqpoint{3.969999in}{1.793640in}}%
\pgfpathcurveto{\pgfqpoint{3.969999in}{1.782590in}}{\pgfqpoint{3.974389in}{1.771991in}}{\pgfqpoint{3.982203in}{1.764177in}}%
\pgfpathcurveto{\pgfqpoint{3.990016in}{1.756364in}}{\pgfqpoint{4.000616in}{1.751973in}}{\pgfqpoint{4.011666in}{1.751973in}}%
\pgfpathclose%
\pgfusepath{stroke,fill}%
\end{pgfscope}%
\begin{pgfscope}%
\pgfpathrectangle{\pgfqpoint{0.800000in}{0.528000in}}{\pgfqpoint{4.960000in}{3.696000in}}%
\pgfusepath{clip}%
\pgfsetbuttcap%
\pgfsetroundjoin%
\definecolor{currentfill}{rgb}{0.000000,0.000000,0.000000}%
\pgfsetfillcolor{currentfill}%
\pgfsetlinewidth{1.003750pt}%
\definecolor{currentstroke}{rgb}{0.000000,0.000000,0.000000}%
\pgfsetstrokecolor{currentstroke}%
\pgfsetdash{}{0pt}%
\pgfpathmoveto{\pgfqpoint{4.011666in}{1.625757in}}%
\pgfpathcurveto{\pgfqpoint{4.022716in}{1.625757in}}{\pgfqpoint{4.033315in}{1.630147in}}{\pgfqpoint{4.041128in}{1.637961in}}%
\pgfpathcurveto{\pgfqpoint{4.048942in}{1.645774in}}{\pgfqpoint{4.053332in}{1.656373in}}{\pgfqpoint{4.053332in}{1.667423in}}%
\pgfpathcurveto{\pgfqpoint{4.053332in}{1.678473in}}{\pgfqpoint{4.048942in}{1.689072in}}{\pgfqpoint{4.041128in}{1.696886in}}%
\pgfpathcurveto{\pgfqpoint{4.033315in}{1.704700in}}{\pgfqpoint{4.022716in}{1.709090in}}{\pgfqpoint{4.011666in}{1.709090in}}%
\pgfpathcurveto{\pgfqpoint{4.000616in}{1.709090in}}{\pgfqpoint{3.990016in}{1.704700in}}{\pgfqpoint{3.982203in}{1.696886in}}%
\pgfpathcurveto{\pgfqpoint{3.974389in}{1.689072in}}{\pgfqpoint{3.969999in}{1.678473in}}{\pgfqpoint{3.969999in}{1.667423in}}%
\pgfpathcurveto{\pgfqpoint{3.969999in}{1.656373in}}{\pgfqpoint{3.974389in}{1.645774in}}{\pgfqpoint{3.982203in}{1.637961in}}%
\pgfpathcurveto{\pgfqpoint{3.990016in}{1.630147in}}{\pgfqpoint{4.000616in}{1.625757in}}{\pgfqpoint{4.011666in}{1.625757in}}%
\pgfpathclose%
\pgfusepath{stroke,fill}%
\end{pgfscope}%
\begin{pgfscope}%
\pgfpathrectangle{\pgfqpoint{0.800000in}{0.528000in}}{\pgfqpoint{4.960000in}{3.696000in}}%
\pgfusepath{clip}%
\pgfsetbuttcap%
\pgfsetroundjoin%
\definecolor{currentfill}{rgb}{0.000000,0.000000,0.000000}%
\pgfsetfillcolor{currentfill}%
\pgfsetlinewidth{1.003750pt}%
\definecolor{currentstroke}{rgb}{0.000000,0.000000,0.000000}%
\pgfsetstrokecolor{currentstroke}%
\pgfsetdash{}{0pt}%
\pgfpathmoveto{\pgfqpoint{4.011666in}{1.567503in}}%
\pgfpathcurveto{\pgfqpoint{4.022716in}{1.567503in}}{\pgfqpoint{4.033315in}{1.571893in}}{\pgfqpoint{4.041128in}{1.579707in}}%
\pgfpathcurveto{\pgfqpoint{4.048942in}{1.587520in}}{\pgfqpoint{4.053332in}{1.598119in}}{\pgfqpoint{4.053332in}{1.609170in}}%
\pgfpathcurveto{\pgfqpoint{4.053332in}{1.620220in}}{\pgfqpoint{4.048942in}{1.630819in}}{\pgfqpoint{4.041128in}{1.638632in}}%
\pgfpathcurveto{\pgfqpoint{4.033315in}{1.646446in}}{\pgfqpoint{4.022716in}{1.650836in}}{\pgfqpoint{4.011666in}{1.650836in}}%
\pgfpathcurveto{\pgfqpoint{4.000616in}{1.650836in}}{\pgfqpoint{3.990016in}{1.646446in}}{\pgfqpoint{3.982203in}{1.638632in}}%
\pgfpathcurveto{\pgfqpoint{3.974389in}{1.630819in}}{\pgfqpoint{3.969999in}{1.620220in}}{\pgfqpoint{3.969999in}{1.609170in}}%
\pgfpathcurveto{\pgfqpoint{3.969999in}{1.598119in}}{\pgfqpoint{3.974389in}{1.587520in}}{\pgfqpoint{3.982203in}{1.579707in}}%
\pgfpathcurveto{\pgfqpoint{3.990016in}{1.571893in}}{\pgfqpoint{4.000616in}{1.567503in}}{\pgfqpoint{4.011666in}{1.567503in}}%
\pgfpathclose%
\pgfusepath{stroke,fill}%
\end{pgfscope}%
\begin{pgfscope}%
\pgfpathrectangle{\pgfqpoint{0.800000in}{0.528000in}}{\pgfqpoint{4.960000in}{3.696000in}}%
\pgfusepath{clip}%
\pgfsetbuttcap%
\pgfsetroundjoin%
\definecolor{currentfill}{rgb}{0.000000,0.000000,0.000000}%
\pgfsetfillcolor{currentfill}%
\pgfsetlinewidth{1.003750pt}%
\definecolor{currentstroke}{rgb}{0.000000,0.000000,0.000000}%
\pgfsetstrokecolor{currentstroke}%
\pgfsetdash{}{0pt}%
\pgfpathmoveto{\pgfqpoint{4.011666in}{1.955862in}}%
\pgfpathcurveto{\pgfqpoint{4.022716in}{1.955862in}}{\pgfqpoint{4.033315in}{1.960252in}}{\pgfqpoint{4.041128in}{1.968065in}}%
\pgfpathcurveto{\pgfqpoint{4.048942in}{1.975879in}}{\pgfqpoint{4.053332in}{1.986478in}}{\pgfqpoint{4.053332in}{1.997528in}}%
\pgfpathcurveto{\pgfqpoint{4.053332in}{2.008578in}}{\pgfqpoint{4.048942in}{2.019177in}}{\pgfqpoint{4.041128in}{2.026991in}}%
\pgfpathcurveto{\pgfqpoint{4.033315in}{2.034805in}}{\pgfqpoint{4.022716in}{2.039195in}}{\pgfqpoint{4.011666in}{2.039195in}}%
\pgfpathcurveto{\pgfqpoint{4.000616in}{2.039195in}}{\pgfqpoint{3.990016in}{2.034805in}}{\pgfqpoint{3.982203in}{2.026991in}}%
\pgfpathcurveto{\pgfqpoint{3.974389in}{2.019177in}}{\pgfqpoint{3.969999in}{2.008578in}}{\pgfqpoint{3.969999in}{1.997528in}}%
\pgfpathcurveto{\pgfqpoint{3.969999in}{1.986478in}}{\pgfqpoint{3.974389in}{1.975879in}}{\pgfqpoint{3.982203in}{1.968065in}}%
\pgfpathcurveto{\pgfqpoint{3.990016in}{1.960252in}}{\pgfqpoint{4.000616in}{1.955862in}}{\pgfqpoint{4.011666in}{1.955862in}}%
\pgfpathclose%
\pgfusepath{stroke,fill}%
\end{pgfscope}%
\begin{pgfscope}%
\pgfpathrectangle{\pgfqpoint{0.800000in}{0.528000in}}{\pgfqpoint{4.960000in}{3.696000in}}%
\pgfusepath{clip}%
\pgfsetbuttcap%
\pgfsetroundjoin%
\definecolor{currentfill}{rgb}{0.000000,0.000000,0.000000}%
\pgfsetfillcolor{currentfill}%
\pgfsetlinewidth{1.003750pt}%
\definecolor{currentstroke}{rgb}{0.000000,0.000000,0.000000}%
\pgfsetstrokecolor{currentstroke}%
\pgfsetdash{}{0pt}%
\pgfpathmoveto{\pgfqpoint{4.011666in}{1.577212in}}%
\pgfpathcurveto{\pgfqpoint{4.022716in}{1.577212in}}{\pgfqpoint{4.033315in}{1.581602in}}{\pgfqpoint{4.041128in}{1.589416in}}%
\pgfpathcurveto{\pgfqpoint{4.048942in}{1.597229in}}{\pgfqpoint{4.053332in}{1.607828in}}{\pgfqpoint{4.053332in}{1.618878in}}%
\pgfpathcurveto{\pgfqpoint{4.053332in}{1.629929in}}{\pgfqpoint{4.048942in}{1.640528in}}{\pgfqpoint{4.041128in}{1.648341in}}%
\pgfpathcurveto{\pgfqpoint{4.033315in}{1.656155in}}{\pgfqpoint{4.022716in}{1.660545in}}{\pgfqpoint{4.011666in}{1.660545in}}%
\pgfpathcurveto{\pgfqpoint{4.000616in}{1.660545in}}{\pgfqpoint{3.990016in}{1.656155in}}{\pgfqpoint{3.982203in}{1.648341in}}%
\pgfpathcurveto{\pgfqpoint{3.974389in}{1.640528in}}{\pgfqpoint{3.969999in}{1.629929in}}{\pgfqpoint{3.969999in}{1.618878in}}%
\pgfpathcurveto{\pgfqpoint{3.969999in}{1.607828in}}{\pgfqpoint{3.974389in}{1.597229in}}{\pgfqpoint{3.982203in}{1.589416in}}%
\pgfpathcurveto{\pgfqpoint{3.990016in}{1.581602in}}{\pgfqpoint{4.000616in}{1.577212in}}{\pgfqpoint{4.011666in}{1.577212in}}%
\pgfpathclose%
\pgfusepath{stroke,fill}%
\end{pgfscope}%
\begin{pgfscope}%
\pgfpathrectangle{\pgfqpoint{0.800000in}{0.528000in}}{\pgfqpoint{4.960000in}{3.696000in}}%
\pgfusepath{clip}%
\pgfsetbuttcap%
\pgfsetroundjoin%
\definecolor{currentfill}{rgb}{0.000000,0.000000,0.000000}%
\pgfsetfillcolor{currentfill}%
\pgfsetlinewidth{1.003750pt}%
\definecolor{currentstroke}{rgb}{0.000000,0.000000,0.000000}%
\pgfsetstrokecolor{currentstroke}%
\pgfsetdash{}{0pt}%
\pgfpathmoveto{\pgfqpoint{4.011666in}{1.606339in}}%
\pgfpathcurveto{\pgfqpoint{4.022716in}{1.606339in}}{\pgfqpoint{4.033315in}{1.610729in}}{\pgfqpoint{4.041128in}{1.618543in}}%
\pgfpathcurveto{\pgfqpoint{4.048942in}{1.626356in}}{\pgfqpoint{4.053332in}{1.636955in}}{\pgfqpoint{4.053332in}{1.648005in}}%
\pgfpathcurveto{\pgfqpoint{4.053332in}{1.659056in}}{\pgfqpoint{4.048942in}{1.669655in}}{\pgfqpoint{4.041128in}{1.677468in}}%
\pgfpathcurveto{\pgfqpoint{4.033315in}{1.685282in}}{\pgfqpoint{4.022716in}{1.689672in}}{\pgfqpoint{4.011666in}{1.689672in}}%
\pgfpathcurveto{\pgfqpoint{4.000616in}{1.689672in}}{\pgfqpoint{3.990016in}{1.685282in}}{\pgfqpoint{3.982203in}{1.677468in}}%
\pgfpathcurveto{\pgfqpoint{3.974389in}{1.669655in}}{\pgfqpoint{3.969999in}{1.659056in}}{\pgfqpoint{3.969999in}{1.648005in}}%
\pgfpathcurveto{\pgfqpoint{3.969999in}{1.636955in}}{\pgfqpoint{3.974389in}{1.626356in}}{\pgfqpoint{3.982203in}{1.618543in}}%
\pgfpathcurveto{\pgfqpoint{3.990016in}{1.610729in}}{\pgfqpoint{4.000616in}{1.606339in}}{\pgfqpoint{4.011666in}{1.606339in}}%
\pgfpathclose%
\pgfusepath{stroke,fill}%
\end{pgfscope}%
\begin{pgfscope}%
\pgfpathrectangle{\pgfqpoint{0.800000in}{0.528000in}}{\pgfqpoint{4.960000in}{3.696000in}}%
\pgfusepath{clip}%
\pgfsetbuttcap%
\pgfsetroundjoin%
\definecolor{currentfill}{rgb}{0.000000,0.000000,0.000000}%
\pgfsetfillcolor{currentfill}%
\pgfsetlinewidth{1.003750pt}%
\definecolor{currentstroke}{rgb}{0.000000,0.000000,0.000000}%
\pgfsetstrokecolor{currentstroke}%
\pgfsetdash{}{0pt}%
\pgfpathmoveto{\pgfqpoint{4.011666in}{1.577212in}}%
\pgfpathcurveto{\pgfqpoint{4.022716in}{1.577212in}}{\pgfqpoint{4.033315in}{1.581602in}}{\pgfqpoint{4.041128in}{1.589416in}}%
\pgfpathcurveto{\pgfqpoint{4.048942in}{1.597229in}}{\pgfqpoint{4.053332in}{1.607828in}}{\pgfqpoint{4.053332in}{1.618878in}}%
\pgfpathcurveto{\pgfqpoint{4.053332in}{1.629929in}}{\pgfqpoint{4.048942in}{1.640528in}}{\pgfqpoint{4.041128in}{1.648341in}}%
\pgfpathcurveto{\pgfqpoint{4.033315in}{1.656155in}}{\pgfqpoint{4.022716in}{1.660545in}}{\pgfqpoint{4.011666in}{1.660545in}}%
\pgfpathcurveto{\pgfqpoint{4.000616in}{1.660545in}}{\pgfqpoint{3.990016in}{1.656155in}}{\pgfqpoint{3.982203in}{1.648341in}}%
\pgfpathcurveto{\pgfqpoint{3.974389in}{1.640528in}}{\pgfqpoint{3.969999in}{1.629929in}}{\pgfqpoint{3.969999in}{1.618878in}}%
\pgfpathcurveto{\pgfqpoint{3.969999in}{1.607828in}}{\pgfqpoint{3.974389in}{1.597229in}}{\pgfqpoint{3.982203in}{1.589416in}}%
\pgfpathcurveto{\pgfqpoint{3.990016in}{1.581602in}}{\pgfqpoint{4.000616in}{1.577212in}}{\pgfqpoint{4.011666in}{1.577212in}}%
\pgfpathclose%
\pgfusepath{stroke,fill}%
\end{pgfscope}%
\begin{pgfscope}%
\pgfpathrectangle{\pgfqpoint{0.800000in}{0.528000in}}{\pgfqpoint{4.960000in}{3.696000in}}%
\pgfusepath{clip}%
\pgfsetbuttcap%
\pgfsetroundjoin%
\definecolor{currentfill}{rgb}{0.000000,0.000000,0.000000}%
\pgfsetfillcolor{currentfill}%
\pgfsetlinewidth{1.003750pt}%
\definecolor{currentstroke}{rgb}{0.000000,0.000000,0.000000}%
\pgfsetstrokecolor{currentstroke}%
\pgfsetdash{}{0pt}%
\pgfpathmoveto{\pgfqpoint{4.011666in}{1.907317in}}%
\pgfpathcurveto{\pgfqpoint{4.022716in}{1.907317in}}{\pgfqpoint{4.033315in}{1.911707in}}{\pgfqpoint{4.041128in}{1.919521in}}%
\pgfpathcurveto{\pgfqpoint{4.048942in}{1.927334in}}{\pgfqpoint{4.053332in}{1.937933in}}{\pgfqpoint{4.053332in}{1.948983in}}%
\pgfpathcurveto{\pgfqpoint{4.053332in}{1.960034in}}{\pgfqpoint{4.048942in}{1.970633in}}{\pgfqpoint{4.041128in}{1.978446in}}%
\pgfpathcurveto{\pgfqpoint{4.033315in}{1.986260in}}{\pgfqpoint{4.022716in}{1.990650in}}{\pgfqpoint{4.011666in}{1.990650in}}%
\pgfpathcurveto{\pgfqpoint{4.000616in}{1.990650in}}{\pgfqpoint{3.990016in}{1.986260in}}{\pgfqpoint{3.982203in}{1.978446in}}%
\pgfpathcurveto{\pgfqpoint{3.974389in}{1.970633in}}{\pgfqpoint{3.969999in}{1.960034in}}{\pgfqpoint{3.969999in}{1.948983in}}%
\pgfpathcurveto{\pgfqpoint{3.969999in}{1.937933in}}{\pgfqpoint{3.974389in}{1.927334in}}{\pgfqpoint{3.982203in}{1.919521in}}%
\pgfpathcurveto{\pgfqpoint{3.990016in}{1.911707in}}{\pgfqpoint{4.000616in}{1.907317in}}{\pgfqpoint{4.011666in}{1.907317in}}%
\pgfpathclose%
\pgfusepath{stroke,fill}%
\end{pgfscope}%
\begin{pgfscope}%
\pgfpathrectangle{\pgfqpoint{0.800000in}{0.528000in}}{\pgfqpoint{4.960000in}{3.696000in}}%
\pgfusepath{clip}%
\pgfsetbuttcap%
\pgfsetroundjoin%
\definecolor{currentfill}{rgb}{0.000000,0.000000,0.000000}%
\pgfsetfillcolor{currentfill}%
\pgfsetlinewidth{1.003750pt}%
\definecolor{currentstroke}{rgb}{0.000000,0.000000,0.000000}%
\pgfsetstrokecolor{currentstroke}%
\pgfsetdash{}{0pt}%
\pgfpathmoveto{\pgfqpoint{4.011666in}{1.926735in}}%
\pgfpathcurveto{\pgfqpoint{4.022716in}{1.926735in}}{\pgfqpoint{4.033315in}{1.931125in}}{\pgfqpoint{4.041128in}{1.938939in}}%
\pgfpathcurveto{\pgfqpoint{4.048942in}{1.946752in}}{\pgfqpoint{4.053332in}{1.957351in}}{\pgfqpoint{4.053332in}{1.968401in}}%
\pgfpathcurveto{\pgfqpoint{4.053332in}{1.979451in}}{\pgfqpoint{4.048942in}{1.990051in}}{\pgfqpoint{4.041128in}{1.997864in}}%
\pgfpathcurveto{\pgfqpoint{4.033315in}{2.005678in}}{\pgfqpoint{4.022716in}{2.010068in}}{\pgfqpoint{4.011666in}{2.010068in}}%
\pgfpathcurveto{\pgfqpoint{4.000616in}{2.010068in}}{\pgfqpoint{3.990016in}{2.005678in}}{\pgfqpoint{3.982203in}{1.997864in}}%
\pgfpathcurveto{\pgfqpoint{3.974389in}{1.990051in}}{\pgfqpoint{3.969999in}{1.979451in}}{\pgfqpoint{3.969999in}{1.968401in}}%
\pgfpathcurveto{\pgfqpoint{3.969999in}{1.957351in}}{\pgfqpoint{3.974389in}{1.946752in}}{\pgfqpoint{3.982203in}{1.938939in}}%
\pgfpathcurveto{\pgfqpoint{3.990016in}{1.931125in}}{\pgfqpoint{4.000616in}{1.926735in}}{\pgfqpoint{4.011666in}{1.926735in}}%
\pgfpathclose%
\pgfusepath{stroke,fill}%
\end{pgfscope}%
\begin{pgfscope}%
\pgfpathrectangle{\pgfqpoint{0.800000in}{0.528000in}}{\pgfqpoint{4.960000in}{3.696000in}}%
\pgfusepath{clip}%
\pgfsetbuttcap%
\pgfsetroundjoin%
\definecolor{currentfill}{rgb}{0.000000,0.000000,0.000000}%
\pgfsetfillcolor{currentfill}%
\pgfsetlinewidth{1.003750pt}%
\definecolor{currentstroke}{rgb}{0.000000,0.000000,0.000000}%
\pgfsetstrokecolor{currentstroke}%
\pgfsetdash{}{0pt}%
\pgfpathmoveto{\pgfqpoint{4.011666in}{1.897608in}}%
\pgfpathcurveto{\pgfqpoint{4.022716in}{1.897608in}}{\pgfqpoint{4.033315in}{1.901998in}}{\pgfqpoint{4.041128in}{1.909812in}}%
\pgfpathcurveto{\pgfqpoint{4.048942in}{1.917625in}}{\pgfqpoint{4.053332in}{1.928224in}}{\pgfqpoint{4.053332in}{1.939274in}}%
\pgfpathcurveto{\pgfqpoint{4.053332in}{1.950325in}}{\pgfqpoint{4.048942in}{1.960924in}}{\pgfqpoint{4.041128in}{1.968737in}}%
\pgfpathcurveto{\pgfqpoint{4.033315in}{1.976551in}}{\pgfqpoint{4.022716in}{1.980941in}}{\pgfqpoint{4.011666in}{1.980941in}}%
\pgfpathcurveto{\pgfqpoint{4.000616in}{1.980941in}}{\pgfqpoint{3.990016in}{1.976551in}}{\pgfqpoint{3.982203in}{1.968737in}}%
\pgfpathcurveto{\pgfqpoint{3.974389in}{1.960924in}}{\pgfqpoint{3.969999in}{1.950325in}}{\pgfqpoint{3.969999in}{1.939274in}}%
\pgfpathcurveto{\pgfqpoint{3.969999in}{1.928224in}}{\pgfqpoint{3.974389in}{1.917625in}}{\pgfqpoint{3.982203in}{1.909812in}}%
\pgfpathcurveto{\pgfqpoint{3.990016in}{1.901998in}}{\pgfqpoint{4.000616in}{1.897608in}}{\pgfqpoint{4.011666in}{1.897608in}}%
\pgfpathclose%
\pgfusepath{stroke,fill}%
\end{pgfscope}%
\begin{pgfscope}%
\pgfpathrectangle{\pgfqpoint{0.800000in}{0.528000in}}{\pgfqpoint{4.960000in}{3.696000in}}%
\pgfusepath{clip}%
\pgfsetbuttcap%
\pgfsetroundjoin%
\definecolor{currentfill}{rgb}{0.000000,0.000000,0.000000}%
\pgfsetfillcolor{currentfill}%
\pgfsetlinewidth{1.003750pt}%
\definecolor{currentstroke}{rgb}{0.000000,0.000000,0.000000}%
\pgfsetstrokecolor{currentstroke}%
\pgfsetdash{}{0pt}%
\pgfpathmoveto{\pgfqpoint{4.011666in}{1.577212in}}%
\pgfpathcurveto{\pgfqpoint{4.022716in}{1.577212in}}{\pgfqpoint{4.033315in}{1.581602in}}{\pgfqpoint{4.041128in}{1.589416in}}%
\pgfpathcurveto{\pgfqpoint{4.048942in}{1.597229in}}{\pgfqpoint{4.053332in}{1.607828in}}{\pgfqpoint{4.053332in}{1.618878in}}%
\pgfpathcurveto{\pgfqpoint{4.053332in}{1.629929in}}{\pgfqpoint{4.048942in}{1.640528in}}{\pgfqpoint{4.041128in}{1.648341in}}%
\pgfpathcurveto{\pgfqpoint{4.033315in}{1.656155in}}{\pgfqpoint{4.022716in}{1.660545in}}{\pgfqpoint{4.011666in}{1.660545in}}%
\pgfpathcurveto{\pgfqpoint{4.000616in}{1.660545in}}{\pgfqpoint{3.990016in}{1.656155in}}{\pgfqpoint{3.982203in}{1.648341in}}%
\pgfpathcurveto{\pgfqpoint{3.974389in}{1.640528in}}{\pgfqpoint{3.969999in}{1.629929in}}{\pgfqpoint{3.969999in}{1.618878in}}%
\pgfpathcurveto{\pgfqpoint{3.969999in}{1.607828in}}{\pgfqpoint{3.974389in}{1.597229in}}{\pgfqpoint{3.982203in}{1.589416in}}%
\pgfpathcurveto{\pgfqpoint{3.990016in}{1.581602in}}{\pgfqpoint{4.000616in}{1.577212in}}{\pgfqpoint{4.011666in}{1.577212in}}%
\pgfpathclose%
\pgfusepath{stroke,fill}%
\end{pgfscope}%
\begin{pgfscope}%
\pgfpathrectangle{\pgfqpoint{0.800000in}{0.528000in}}{\pgfqpoint{4.960000in}{3.696000in}}%
\pgfusepath{clip}%
\pgfsetbuttcap%
\pgfsetroundjoin%
\definecolor{currentfill}{rgb}{0.000000,0.000000,0.000000}%
\pgfsetfillcolor{currentfill}%
\pgfsetlinewidth{1.003750pt}%
\definecolor{currentstroke}{rgb}{0.000000,0.000000,0.000000}%
\pgfsetstrokecolor{currentstroke}%
\pgfsetdash{}{0pt}%
\pgfpathmoveto{\pgfqpoint{4.011666in}{1.849063in}}%
\pgfpathcurveto{\pgfqpoint{4.022716in}{1.849063in}}{\pgfqpoint{4.033315in}{1.853453in}}{\pgfqpoint{4.041128in}{1.861267in}}%
\pgfpathcurveto{\pgfqpoint{4.048942in}{1.869080in}}{\pgfqpoint{4.053332in}{1.879679in}}{\pgfqpoint{4.053332in}{1.890730in}}%
\pgfpathcurveto{\pgfqpoint{4.053332in}{1.901780in}}{\pgfqpoint{4.048942in}{1.912379in}}{\pgfqpoint{4.041128in}{1.920192in}}%
\pgfpathcurveto{\pgfqpoint{4.033315in}{1.928006in}}{\pgfqpoint{4.022716in}{1.932396in}}{\pgfqpoint{4.011666in}{1.932396in}}%
\pgfpathcurveto{\pgfqpoint{4.000616in}{1.932396in}}{\pgfqpoint{3.990016in}{1.928006in}}{\pgfqpoint{3.982203in}{1.920192in}}%
\pgfpathcurveto{\pgfqpoint{3.974389in}{1.912379in}}{\pgfqpoint{3.969999in}{1.901780in}}{\pgfqpoint{3.969999in}{1.890730in}}%
\pgfpathcurveto{\pgfqpoint{3.969999in}{1.879679in}}{\pgfqpoint{3.974389in}{1.869080in}}{\pgfqpoint{3.982203in}{1.861267in}}%
\pgfpathcurveto{\pgfqpoint{3.990016in}{1.853453in}}{\pgfqpoint{4.000616in}{1.849063in}}{\pgfqpoint{4.011666in}{1.849063in}}%
\pgfpathclose%
\pgfusepath{stroke,fill}%
\end{pgfscope}%
\begin{pgfscope}%
\pgfpathrectangle{\pgfqpoint{0.800000in}{0.528000in}}{\pgfqpoint{4.960000in}{3.696000in}}%
\pgfusepath{clip}%
\pgfsetbuttcap%
\pgfsetroundjoin%
\definecolor{currentfill}{rgb}{0.000000,0.000000,0.000000}%
\pgfsetfillcolor{currentfill}%
\pgfsetlinewidth{1.003750pt}%
\definecolor{currentstroke}{rgb}{0.000000,0.000000,0.000000}%
\pgfsetstrokecolor{currentstroke}%
\pgfsetdash{}{0pt}%
\pgfpathmoveto{\pgfqpoint{4.011666in}{1.732555in}}%
\pgfpathcurveto{\pgfqpoint{4.022716in}{1.732555in}}{\pgfqpoint{4.033315in}{1.736946in}}{\pgfqpoint{4.041128in}{1.744759in}}%
\pgfpathcurveto{\pgfqpoint{4.048942in}{1.752573in}}{\pgfqpoint{4.053332in}{1.763172in}}{\pgfqpoint{4.053332in}{1.774222in}}%
\pgfpathcurveto{\pgfqpoint{4.053332in}{1.785272in}}{\pgfqpoint{4.048942in}{1.795871in}}{\pgfqpoint{4.041128in}{1.803685in}}%
\pgfpathcurveto{\pgfqpoint{4.033315in}{1.811498in}}{\pgfqpoint{4.022716in}{1.815889in}}{\pgfqpoint{4.011666in}{1.815889in}}%
\pgfpathcurveto{\pgfqpoint{4.000616in}{1.815889in}}{\pgfqpoint{3.990016in}{1.811498in}}{\pgfqpoint{3.982203in}{1.803685in}}%
\pgfpathcurveto{\pgfqpoint{3.974389in}{1.795871in}}{\pgfqpoint{3.969999in}{1.785272in}}{\pgfqpoint{3.969999in}{1.774222in}}%
\pgfpathcurveto{\pgfqpoint{3.969999in}{1.763172in}}{\pgfqpoint{3.974389in}{1.752573in}}{\pgfqpoint{3.982203in}{1.744759in}}%
\pgfpathcurveto{\pgfqpoint{3.990016in}{1.736946in}}{\pgfqpoint{4.000616in}{1.732555in}}{\pgfqpoint{4.011666in}{1.732555in}}%
\pgfpathclose%
\pgfusepath{stroke,fill}%
\end{pgfscope}%
\begin{pgfscope}%
\pgfpathrectangle{\pgfqpoint{0.800000in}{0.528000in}}{\pgfqpoint{4.960000in}{3.696000in}}%
\pgfusepath{clip}%
\pgfsetbuttcap%
\pgfsetroundjoin%
\definecolor{currentfill}{rgb}{0.000000,0.000000,0.000000}%
\pgfsetfillcolor{currentfill}%
\pgfsetlinewidth{1.003750pt}%
\definecolor{currentstroke}{rgb}{0.000000,0.000000,0.000000}%
\pgfsetstrokecolor{currentstroke}%
\pgfsetdash{}{0pt}%
\pgfpathmoveto{\pgfqpoint{4.011666in}{1.645175in}}%
\pgfpathcurveto{\pgfqpoint{4.022716in}{1.645175in}}{\pgfqpoint{4.033315in}{1.649565in}}{\pgfqpoint{4.041128in}{1.657378in}}%
\pgfpathcurveto{\pgfqpoint{4.048942in}{1.665192in}}{\pgfqpoint{4.053332in}{1.675791in}}{\pgfqpoint{4.053332in}{1.686841in}}%
\pgfpathcurveto{\pgfqpoint{4.053332in}{1.697891in}}{\pgfqpoint{4.048942in}{1.708490in}}{\pgfqpoint{4.041128in}{1.716304in}}%
\pgfpathcurveto{\pgfqpoint{4.033315in}{1.724118in}}{\pgfqpoint{4.022716in}{1.728508in}}{\pgfqpoint{4.011666in}{1.728508in}}%
\pgfpathcurveto{\pgfqpoint{4.000616in}{1.728508in}}{\pgfqpoint{3.990016in}{1.724118in}}{\pgfqpoint{3.982203in}{1.716304in}}%
\pgfpathcurveto{\pgfqpoint{3.974389in}{1.708490in}}{\pgfqpoint{3.969999in}{1.697891in}}{\pgfqpoint{3.969999in}{1.686841in}}%
\pgfpathcurveto{\pgfqpoint{3.969999in}{1.675791in}}{\pgfqpoint{3.974389in}{1.665192in}}{\pgfqpoint{3.982203in}{1.657378in}}%
\pgfpathcurveto{\pgfqpoint{3.990016in}{1.649565in}}{\pgfqpoint{4.000616in}{1.645175in}}{\pgfqpoint{4.011666in}{1.645175in}}%
\pgfpathclose%
\pgfusepath{stroke,fill}%
\end{pgfscope}%
\begin{pgfscope}%
\pgfpathrectangle{\pgfqpoint{0.800000in}{0.528000in}}{\pgfqpoint{4.960000in}{3.696000in}}%
\pgfusepath{clip}%
\pgfsetbuttcap%
\pgfsetroundjoin%
\definecolor{currentfill}{rgb}{0.000000,0.000000,0.000000}%
\pgfsetfillcolor{currentfill}%
\pgfsetlinewidth{1.003750pt}%
\definecolor{currentstroke}{rgb}{0.000000,0.000000,0.000000}%
\pgfsetstrokecolor{currentstroke}%
\pgfsetdash{}{0pt}%
\pgfpathmoveto{\pgfqpoint{4.011666in}{1.790809in}}%
\pgfpathcurveto{\pgfqpoint{4.022716in}{1.790809in}}{\pgfqpoint{4.033315in}{1.795199in}}{\pgfqpoint{4.041128in}{1.803013in}}%
\pgfpathcurveto{\pgfqpoint{4.048942in}{1.810827in}}{\pgfqpoint{4.053332in}{1.821426in}}{\pgfqpoint{4.053332in}{1.832476in}}%
\pgfpathcurveto{\pgfqpoint{4.053332in}{1.843526in}}{\pgfqpoint{4.048942in}{1.854125in}}{\pgfqpoint{4.041128in}{1.861939in}}%
\pgfpathcurveto{\pgfqpoint{4.033315in}{1.869752in}}{\pgfqpoint{4.022716in}{1.874142in}}{\pgfqpoint{4.011666in}{1.874142in}}%
\pgfpathcurveto{\pgfqpoint{4.000616in}{1.874142in}}{\pgfqpoint{3.990016in}{1.869752in}}{\pgfqpoint{3.982203in}{1.861939in}}%
\pgfpathcurveto{\pgfqpoint{3.974389in}{1.854125in}}{\pgfqpoint{3.969999in}{1.843526in}}{\pgfqpoint{3.969999in}{1.832476in}}%
\pgfpathcurveto{\pgfqpoint{3.969999in}{1.821426in}}{\pgfqpoint{3.974389in}{1.810827in}}{\pgfqpoint{3.982203in}{1.803013in}}%
\pgfpathcurveto{\pgfqpoint{3.990016in}{1.795199in}}{\pgfqpoint{4.000616in}{1.790809in}}{\pgfqpoint{4.011666in}{1.790809in}}%
\pgfpathclose%
\pgfusepath{stroke,fill}%
\end{pgfscope}%
\begin{pgfscope}%
\pgfpathrectangle{\pgfqpoint{0.800000in}{0.528000in}}{\pgfqpoint{4.960000in}{3.696000in}}%
\pgfusepath{clip}%
\pgfsetbuttcap%
\pgfsetroundjoin%
\definecolor{currentfill}{rgb}{0.000000,0.000000,0.000000}%
\pgfsetfillcolor{currentfill}%
\pgfsetlinewidth{1.003750pt}%
\definecolor{currentstroke}{rgb}{0.000000,0.000000,0.000000}%
\pgfsetstrokecolor{currentstroke}%
\pgfsetdash{}{0pt}%
\pgfpathmoveto{\pgfqpoint{4.011666in}{1.839354in}}%
\pgfpathcurveto{\pgfqpoint{4.022716in}{1.839354in}}{\pgfqpoint{4.033315in}{1.843744in}}{\pgfqpoint{4.041128in}{1.851558in}}%
\pgfpathcurveto{\pgfqpoint{4.048942in}{1.859371in}}{\pgfqpoint{4.053332in}{1.869971in}}{\pgfqpoint{4.053332in}{1.881021in}}%
\pgfpathcurveto{\pgfqpoint{4.053332in}{1.892071in}}{\pgfqpoint{4.048942in}{1.902670in}}{\pgfqpoint{4.041128in}{1.910483in}}%
\pgfpathcurveto{\pgfqpoint{4.033315in}{1.918297in}}{\pgfqpoint{4.022716in}{1.922687in}}{\pgfqpoint{4.011666in}{1.922687in}}%
\pgfpathcurveto{\pgfqpoint{4.000616in}{1.922687in}}{\pgfqpoint{3.990016in}{1.918297in}}{\pgfqpoint{3.982203in}{1.910483in}}%
\pgfpathcurveto{\pgfqpoint{3.974389in}{1.902670in}}{\pgfqpoint{3.969999in}{1.892071in}}{\pgfqpoint{3.969999in}{1.881021in}}%
\pgfpathcurveto{\pgfqpoint{3.969999in}{1.869971in}}{\pgfqpoint{3.974389in}{1.859371in}}{\pgfqpoint{3.982203in}{1.851558in}}%
\pgfpathcurveto{\pgfqpoint{3.990016in}{1.843744in}}{\pgfqpoint{4.000616in}{1.839354in}}{\pgfqpoint{4.011666in}{1.839354in}}%
\pgfpathclose%
\pgfusepath{stroke,fill}%
\end{pgfscope}%
\begin{pgfscope}%
\pgfpathrectangle{\pgfqpoint{0.800000in}{0.528000in}}{\pgfqpoint{4.960000in}{3.696000in}}%
\pgfusepath{clip}%
\pgfsetbuttcap%
\pgfsetroundjoin%
\definecolor{currentfill}{rgb}{0.000000,0.000000,0.000000}%
\pgfsetfillcolor{currentfill}%
\pgfsetlinewidth{1.003750pt}%
\definecolor{currentstroke}{rgb}{0.000000,0.000000,0.000000}%
\pgfsetstrokecolor{currentstroke}%
\pgfsetdash{}{0pt}%
\pgfpathmoveto{\pgfqpoint{4.011666in}{1.616048in}}%
\pgfpathcurveto{\pgfqpoint{4.022716in}{1.616048in}}{\pgfqpoint{4.033315in}{1.620438in}}{\pgfqpoint{4.041128in}{1.628252in}}%
\pgfpathcurveto{\pgfqpoint{4.048942in}{1.636065in}}{\pgfqpoint{4.053332in}{1.646664in}}{\pgfqpoint{4.053332in}{1.657714in}}%
\pgfpathcurveto{\pgfqpoint{4.053332in}{1.668764in}}{\pgfqpoint{4.048942in}{1.679364in}}{\pgfqpoint{4.041128in}{1.687177in}}%
\pgfpathcurveto{\pgfqpoint{4.033315in}{1.694991in}}{\pgfqpoint{4.022716in}{1.699381in}}{\pgfqpoint{4.011666in}{1.699381in}}%
\pgfpathcurveto{\pgfqpoint{4.000616in}{1.699381in}}{\pgfqpoint{3.990016in}{1.694991in}}{\pgfqpoint{3.982203in}{1.687177in}}%
\pgfpathcurveto{\pgfqpoint{3.974389in}{1.679364in}}{\pgfqpoint{3.969999in}{1.668764in}}{\pgfqpoint{3.969999in}{1.657714in}}%
\pgfpathcurveto{\pgfqpoint{3.969999in}{1.646664in}}{\pgfqpoint{3.974389in}{1.636065in}}{\pgfqpoint{3.982203in}{1.628252in}}%
\pgfpathcurveto{\pgfqpoint{3.990016in}{1.620438in}}{\pgfqpoint{4.000616in}{1.616048in}}{\pgfqpoint{4.011666in}{1.616048in}}%
\pgfpathclose%
\pgfusepath{stroke,fill}%
\end{pgfscope}%
\begin{pgfscope}%
\pgfpathrectangle{\pgfqpoint{0.800000in}{0.528000in}}{\pgfqpoint{4.960000in}{3.696000in}}%
\pgfusepath{clip}%
\pgfsetbuttcap%
\pgfsetroundjoin%
\definecolor{currentfill}{rgb}{0.000000,0.000000,0.000000}%
\pgfsetfillcolor{currentfill}%
\pgfsetlinewidth{1.003750pt}%
\definecolor{currentstroke}{rgb}{0.000000,0.000000,0.000000}%
\pgfsetstrokecolor{currentstroke}%
\pgfsetdash{}{0pt}%
\pgfpathmoveto{\pgfqpoint{4.011666in}{1.548085in}}%
\pgfpathcurveto{\pgfqpoint{4.022716in}{1.548085in}}{\pgfqpoint{4.033315in}{1.552475in}}{\pgfqpoint{4.041128in}{1.560289in}}%
\pgfpathcurveto{\pgfqpoint{4.048942in}{1.568102in}}{\pgfqpoint{4.053332in}{1.578701in}}{\pgfqpoint{4.053332in}{1.589752in}}%
\pgfpathcurveto{\pgfqpoint{4.053332in}{1.600802in}}{\pgfqpoint{4.048942in}{1.611401in}}{\pgfqpoint{4.041128in}{1.619214in}}%
\pgfpathcurveto{\pgfqpoint{4.033315in}{1.627028in}}{\pgfqpoint{4.022716in}{1.631418in}}{\pgfqpoint{4.011666in}{1.631418in}}%
\pgfpathcurveto{\pgfqpoint{4.000616in}{1.631418in}}{\pgfqpoint{3.990016in}{1.627028in}}{\pgfqpoint{3.982203in}{1.619214in}}%
\pgfpathcurveto{\pgfqpoint{3.974389in}{1.611401in}}{\pgfqpoint{3.969999in}{1.600802in}}{\pgfqpoint{3.969999in}{1.589752in}}%
\pgfpathcurveto{\pgfqpoint{3.969999in}{1.578701in}}{\pgfqpoint{3.974389in}{1.568102in}}{\pgfqpoint{3.982203in}{1.560289in}}%
\pgfpathcurveto{\pgfqpoint{3.990016in}{1.552475in}}{\pgfqpoint{4.000616in}{1.548085in}}{\pgfqpoint{4.011666in}{1.548085in}}%
\pgfpathclose%
\pgfusepath{stroke,fill}%
\end{pgfscope}%
\begin{pgfscope}%
\pgfpathrectangle{\pgfqpoint{0.800000in}{0.528000in}}{\pgfqpoint{4.960000in}{3.696000in}}%
\pgfusepath{clip}%
\pgfsetbuttcap%
\pgfsetroundjoin%
\definecolor{currentfill}{rgb}{0.000000,0.000000,0.000000}%
\pgfsetfillcolor{currentfill}%
\pgfsetlinewidth{1.003750pt}%
\definecolor{currentstroke}{rgb}{0.000000,0.000000,0.000000}%
\pgfsetstrokecolor{currentstroke}%
\pgfsetdash{}{0pt}%
\pgfpathmoveto{\pgfqpoint{4.011666in}{1.596630in}}%
\pgfpathcurveto{\pgfqpoint{4.022716in}{1.596630in}}{\pgfqpoint{4.033315in}{1.601020in}}{\pgfqpoint{4.041128in}{1.608834in}}%
\pgfpathcurveto{\pgfqpoint{4.048942in}{1.616647in}}{\pgfqpoint{4.053332in}{1.627246in}}{\pgfqpoint{4.053332in}{1.638296in}}%
\pgfpathcurveto{\pgfqpoint{4.053332in}{1.649347in}}{\pgfqpoint{4.048942in}{1.659946in}}{\pgfqpoint{4.041128in}{1.667759in}}%
\pgfpathcurveto{\pgfqpoint{4.033315in}{1.675573in}}{\pgfqpoint{4.022716in}{1.679963in}}{\pgfqpoint{4.011666in}{1.679963in}}%
\pgfpathcurveto{\pgfqpoint{4.000616in}{1.679963in}}{\pgfqpoint{3.990016in}{1.675573in}}{\pgfqpoint{3.982203in}{1.667759in}}%
\pgfpathcurveto{\pgfqpoint{3.974389in}{1.659946in}}{\pgfqpoint{3.969999in}{1.649347in}}{\pgfqpoint{3.969999in}{1.638296in}}%
\pgfpathcurveto{\pgfqpoint{3.969999in}{1.627246in}}{\pgfqpoint{3.974389in}{1.616647in}}{\pgfqpoint{3.982203in}{1.608834in}}%
\pgfpathcurveto{\pgfqpoint{3.990016in}{1.601020in}}{\pgfqpoint{4.000616in}{1.596630in}}{\pgfqpoint{4.011666in}{1.596630in}}%
\pgfpathclose%
\pgfusepath{stroke,fill}%
\end{pgfscope}%
\begin{pgfscope}%
\pgfpathrectangle{\pgfqpoint{0.800000in}{0.528000in}}{\pgfqpoint{4.960000in}{3.696000in}}%
\pgfusepath{clip}%
\pgfsetbuttcap%
\pgfsetroundjoin%
\definecolor{currentfill}{rgb}{0.000000,0.000000,0.000000}%
\pgfsetfillcolor{currentfill}%
\pgfsetlinewidth{1.003750pt}%
\definecolor{currentstroke}{rgb}{0.000000,0.000000,0.000000}%
\pgfsetstrokecolor{currentstroke}%
\pgfsetdash{}{0pt}%
\pgfpathmoveto{\pgfqpoint{4.011666in}{1.761682in}}%
\pgfpathcurveto{\pgfqpoint{4.022716in}{1.761682in}}{\pgfqpoint{4.033315in}{1.766072in}}{\pgfqpoint{4.041128in}{1.773886in}}%
\pgfpathcurveto{\pgfqpoint{4.048942in}{1.781700in}}{\pgfqpoint{4.053332in}{1.792299in}}{\pgfqpoint{4.053332in}{1.803349in}}%
\pgfpathcurveto{\pgfqpoint{4.053332in}{1.814399in}}{\pgfqpoint{4.048942in}{1.824998in}}{\pgfqpoint{4.041128in}{1.832812in}}%
\pgfpathcurveto{\pgfqpoint{4.033315in}{1.840625in}}{\pgfqpoint{4.022716in}{1.845016in}}{\pgfqpoint{4.011666in}{1.845016in}}%
\pgfpathcurveto{\pgfqpoint{4.000616in}{1.845016in}}{\pgfqpoint{3.990016in}{1.840625in}}{\pgfqpoint{3.982203in}{1.832812in}}%
\pgfpathcurveto{\pgfqpoint{3.974389in}{1.824998in}}{\pgfqpoint{3.969999in}{1.814399in}}{\pgfqpoint{3.969999in}{1.803349in}}%
\pgfpathcurveto{\pgfqpoint{3.969999in}{1.792299in}}{\pgfqpoint{3.974389in}{1.781700in}}{\pgfqpoint{3.982203in}{1.773886in}}%
\pgfpathcurveto{\pgfqpoint{3.990016in}{1.766072in}}{\pgfqpoint{4.000616in}{1.761682in}}{\pgfqpoint{4.011666in}{1.761682in}}%
\pgfpathclose%
\pgfusepath{stroke,fill}%
\end{pgfscope}%
\begin{pgfscope}%
\pgfpathrectangle{\pgfqpoint{0.800000in}{0.528000in}}{\pgfqpoint{4.960000in}{3.696000in}}%
\pgfusepath{clip}%
\pgfsetbuttcap%
\pgfsetroundjoin%
\definecolor{currentfill}{rgb}{0.000000,0.000000,0.000000}%
\pgfsetfillcolor{currentfill}%
\pgfsetlinewidth{1.003750pt}%
\definecolor{currentstroke}{rgb}{0.000000,0.000000,0.000000}%
\pgfsetstrokecolor{currentstroke}%
\pgfsetdash{}{0pt}%
\pgfpathmoveto{\pgfqpoint{4.011666in}{1.713137in}}%
\pgfpathcurveto{\pgfqpoint{4.022716in}{1.713137in}}{\pgfqpoint{4.033315in}{1.717528in}}{\pgfqpoint{4.041128in}{1.725341in}}%
\pgfpathcurveto{\pgfqpoint{4.048942in}{1.733155in}}{\pgfqpoint{4.053332in}{1.743754in}}{\pgfqpoint{4.053332in}{1.754804in}}%
\pgfpathcurveto{\pgfqpoint{4.053332in}{1.765854in}}{\pgfqpoint{4.048942in}{1.776453in}}{\pgfqpoint{4.041128in}{1.784267in}}%
\pgfpathcurveto{\pgfqpoint{4.033315in}{1.792080in}}{\pgfqpoint{4.022716in}{1.796471in}}{\pgfqpoint{4.011666in}{1.796471in}}%
\pgfpathcurveto{\pgfqpoint{4.000616in}{1.796471in}}{\pgfqpoint{3.990016in}{1.792080in}}{\pgfqpoint{3.982203in}{1.784267in}}%
\pgfpathcurveto{\pgfqpoint{3.974389in}{1.776453in}}{\pgfqpoint{3.969999in}{1.765854in}}{\pgfqpoint{3.969999in}{1.754804in}}%
\pgfpathcurveto{\pgfqpoint{3.969999in}{1.743754in}}{\pgfqpoint{3.974389in}{1.733155in}}{\pgfqpoint{3.982203in}{1.725341in}}%
\pgfpathcurveto{\pgfqpoint{3.990016in}{1.717528in}}{\pgfqpoint{4.000616in}{1.713137in}}{\pgfqpoint{4.011666in}{1.713137in}}%
\pgfpathclose%
\pgfusepath{stroke,fill}%
\end{pgfscope}%
\begin{pgfscope}%
\pgfpathrectangle{\pgfqpoint{0.800000in}{0.528000in}}{\pgfqpoint{4.960000in}{3.696000in}}%
\pgfusepath{clip}%
\pgfsetbuttcap%
\pgfsetroundjoin%
\definecolor{currentfill}{rgb}{0.000000,0.000000,0.000000}%
\pgfsetfillcolor{currentfill}%
\pgfsetlinewidth{1.003750pt}%
\definecolor{currentstroke}{rgb}{0.000000,0.000000,0.000000}%
\pgfsetstrokecolor{currentstroke}%
\pgfsetdash{}{0pt}%
\pgfpathmoveto{\pgfqpoint{4.011666in}{1.596630in}}%
\pgfpathcurveto{\pgfqpoint{4.022716in}{1.596630in}}{\pgfqpoint{4.033315in}{1.601020in}}{\pgfqpoint{4.041128in}{1.608834in}}%
\pgfpathcurveto{\pgfqpoint{4.048942in}{1.616647in}}{\pgfqpoint{4.053332in}{1.627246in}}{\pgfqpoint{4.053332in}{1.638296in}}%
\pgfpathcurveto{\pgfqpoint{4.053332in}{1.649347in}}{\pgfqpoint{4.048942in}{1.659946in}}{\pgfqpoint{4.041128in}{1.667759in}}%
\pgfpathcurveto{\pgfqpoint{4.033315in}{1.675573in}}{\pgfqpoint{4.022716in}{1.679963in}}{\pgfqpoint{4.011666in}{1.679963in}}%
\pgfpathcurveto{\pgfqpoint{4.000616in}{1.679963in}}{\pgfqpoint{3.990016in}{1.675573in}}{\pgfqpoint{3.982203in}{1.667759in}}%
\pgfpathcurveto{\pgfqpoint{3.974389in}{1.659946in}}{\pgfqpoint{3.969999in}{1.649347in}}{\pgfqpoint{3.969999in}{1.638296in}}%
\pgfpathcurveto{\pgfqpoint{3.969999in}{1.627246in}}{\pgfqpoint{3.974389in}{1.616647in}}{\pgfqpoint{3.982203in}{1.608834in}}%
\pgfpathcurveto{\pgfqpoint{3.990016in}{1.601020in}}{\pgfqpoint{4.000616in}{1.596630in}}{\pgfqpoint{4.011666in}{1.596630in}}%
\pgfpathclose%
\pgfusepath{stroke,fill}%
\end{pgfscope}%
\begin{pgfscope}%
\pgfpathrectangle{\pgfqpoint{0.800000in}{0.528000in}}{\pgfqpoint{4.960000in}{3.696000in}}%
\pgfusepath{clip}%
\pgfsetbuttcap%
\pgfsetroundjoin%
\definecolor{currentfill}{rgb}{0.000000,0.000000,0.000000}%
\pgfsetfillcolor{currentfill}%
\pgfsetlinewidth{1.003750pt}%
\definecolor{currentstroke}{rgb}{0.000000,0.000000,0.000000}%
\pgfsetstrokecolor{currentstroke}%
\pgfsetdash{}{0pt}%
\pgfpathmoveto{\pgfqpoint{4.011666in}{1.625757in}}%
\pgfpathcurveto{\pgfqpoint{4.022716in}{1.625757in}}{\pgfqpoint{4.033315in}{1.630147in}}{\pgfqpoint{4.041128in}{1.637961in}}%
\pgfpathcurveto{\pgfqpoint{4.048942in}{1.645774in}}{\pgfqpoint{4.053332in}{1.656373in}}{\pgfqpoint{4.053332in}{1.667423in}}%
\pgfpathcurveto{\pgfqpoint{4.053332in}{1.678473in}}{\pgfqpoint{4.048942in}{1.689072in}}{\pgfqpoint{4.041128in}{1.696886in}}%
\pgfpathcurveto{\pgfqpoint{4.033315in}{1.704700in}}{\pgfqpoint{4.022716in}{1.709090in}}{\pgfqpoint{4.011666in}{1.709090in}}%
\pgfpathcurveto{\pgfqpoint{4.000616in}{1.709090in}}{\pgfqpoint{3.990016in}{1.704700in}}{\pgfqpoint{3.982203in}{1.696886in}}%
\pgfpathcurveto{\pgfqpoint{3.974389in}{1.689072in}}{\pgfqpoint{3.969999in}{1.678473in}}{\pgfqpoint{3.969999in}{1.667423in}}%
\pgfpathcurveto{\pgfqpoint{3.969999in}{1.656373in}}{\pgfqpoint{3.974389in}{1.645774in}}{\pgfqpoint{3.982203in}{1.637961in}}%
\pgfpathcurveto{\pgfqpoint{3.990016in}{1.630147in}}{\pgfqpoint{4.000616in}{1.625757in}}{\pgfqpoint{4.011666in}{1.625757in}}%
\pgfpathclose%
\pgfusepath{stroke,fill}%
\end{pgfscope}%
\begin{pgfscope}%
\pgfpathrectangle{\pgfqpoint{0.800000in}{0.528000in}}{\pgfqpoint{4.960000in}{3.696000in}}%
\pgfusepath{clip}%
\pgfsetbuttcap%
\pgfsetroundjoin%
\definecolor{currentfill}{rgb}{0.000000,0.000000,0.000000}%
\pgfsetfillcolor{currentfill}%
\pgfsetlinewidth{1.003750pt}%
\definecolor{currentstroke}{rgb}{0.000000,0.000000,0.000000}%
\pgfsetstrokecolor{currentstroke}%
\pgfsetdash{}{0pt}%
\pgfpathmoveto{\pgfqpoint{4.011666in}{1.693719in}}%
\pgfpathcurveto{\pgfqpoint{4.022716in}{1.693719in}}{\pgfqpoint{4.033315in}{1.698110in}}{\pgfqpoint{4.041128in}{1.705923in}}%
\pgfpathcurveto{\pgfqpoint{4.048942in}{1.713737in}}{\pgfqpoint{4.053332in}{1.724336in}}{\pgfqpoint{4.053332in}{1.735386in}}%
\pgfpathcurveto{\pgfqpoint{4.053332in}{1.746436in}}{\pgfqpoint{4.048942in}{1.757035in}}{\pgfqpoint{4.041128in}{1.764849in}}%
\pgfpathcurveto{\pgfqpoint{4.033315in}{1.772663in}}{\pgfqpoint{4.022716in}{1.777053in}}{\pgfqpoint{4.011666in}{1.777053in}}%
\pgfpathcurveto{\pgfqpoint{4.000616in}{1.777053in}}{\pgfqpoint{3.990016in}{1.772663in}}{\pgfqpoint{3.982203in}{1.764849in}}%
\pgfpathcurveto{\pgfqpoint{3.974389in}{1.757035in}}{\pgfqpoint{3.969999in}{1.746436in}}{\pgfqpoint{3.969999in}{1.735386in}}%
\pgfpathcurveto{\pgfqpoint{3.969999in}{1.724336in}}{\pgfqpoint{3.974389in}{1.713737in}}{\pgfqpoint{3.982203in}{1.705923in}}%
\pgfpathcurveto{\pgfqpoint{3.990016in}{1.698110in}}{\pgfqpoint{4.000616in}{1.693719in}}{\pgfqpoint{4.011666in}{1.693719in}}%
\pgfpathclose%
\pgfusepath{stroke,fill}%
\end{pgfscope}%
\begin{pgfscope}%
\pgfpathrectangle{\pgfqpoint{0.800000in}{0.528000in}}{\pgfqpoint{4.960000in}{3.696000in}}%
\pgfusepath{clip}%
\pgfsetbuttcap%
\pgfsetroundjoin%
\definecolor{currentfill}{rgb}{0.000000,0.000000,0.000000}%
\pgfsetfillcolor{currentfill}%
\pgfsetlinewidth{1.003750pt}%
\definecolor{currentstroke}{rgb}{0.000000,0.000000,0.000000}%
\pgfsetstrokecolor{currentstroke}%
\pgfsetdash{}{0pt}%
\pgfpathmoveto{\pgfqpoint{4.011666in}{1.606339in}}%
\pgfpathcurveto{\pgfqpoint{4.022716in}{1.606339in}}{\pgfqpoint{4.033315in}{1.610729in}}{\pgfqpoint{4.041128in}{1.618543in}}%
\pgfpathcurveto{\pgfqpoint{4.048942in}{1.626356in}}{\pgfqpoint{4.053332in}{1.636955in}}{\pgfqpoint{4.053332in}{1.648005in}}%
\pgfpathcurveto{\pgfqpoint{4.053332in}{1.659056in}}{\pgfqpoint{4.048942in}{1.669655in}}{\pgfqpoint{4.041128in}{1.677468in}}%
\pgfpathcurveto{\pgfqpoint{4.033315in}{1.685282in}}{\pgfqpoint{4.022716in}{1.689672in}}{\pgfqpoint{4.011666in}{1.689672in}}%
\pgfpathcurveto{\pgfqpoint{4.000616in}{1.689672in}}{\pgfqpoint{3.990016in}{1.685282in}}{\pgfqpoint{3.982203in}{1.677468in}}%
\pgfpathcurveto{\pgfqpoint{3.974389in}{1.669655in}}{\pgfqpoint{3.969999in}{1.659056in}}{\pgfqpoint{3.969999in}{1.648005in}}%
\pgfpathcurveto{\pgfqpoint{3.969999in}{1.636955in}}{\pgfqpoint{3.974389in}{1.626356in}}{\pgfqpoint{3.982203in}{1.618543in}}%
\pgfpathcurveto{\pgfqpoint{3.990016in}{1.610729in}}{\pgfqpoint{4.000616in}{1.606339in}}{\pgfqpoint{4.011666in}{1.606339in}}%
\pgfpathclose%
\pgfusepath{stroke,fill}%
\end{pgfscope}%
\begin{pgfscope}%
\pgfpathrectangle{\pgfqpoint{0.800000in}{0.528000in}}{\pgfqpoint{4.960000in}{3.696000in}}%
\pgfusepath{clip}%
\pgfsetbuttcap%
\pgfsetroundjoin%
\definecolor{currentfill}{rgb}{0.000000,0.000000,0.000000}%
\pgfsetfillcolor{currentfill}%
\pgfsetlinewidth{1.003750pt}%
\definecolor{currentstroke}{rgb}{0.000000,0.000000,0.000000}%
\pgfsetstrokecolor{currentstroke}%
\pgfsetdash{}{0pt}%
\pgfpathmoveto{\pgfqpoint{4.011666in}{1.751973in}}%
\pgfpathcurveto{\pgfqpoint{4.022716in}{1.751973in}}{\pgfqpoint{4.033315in}{1.756364in}}{\pgfqpoint{4.041128in}{1.764177in}}%
\pgfpathcurveto{\pgfqpoint{4.048942in}{1.771991in}}{\pgfqpoint{4.053332in}{1.782590in}}{\pgfqpoint{4.053332in}{1.793640in}}%
\pgfpathcurveto{\pgfqpoint{4.053332in}{1.804690in}}{\pgfqpoint{4.048942in}{1.815289in}}{\pgfqpoint{4.041128in}{1.823103in}}%
\pgfpathcurveto{\pgfqpoint{4.033315in}{1.830916in}}{\pgfqpoint{4.022716in}{1.835307in}}{\pgfqpoint{4.011666in}{1.835307in}}%
\pgfpathcurveto{\pgfqpoint{4.000616in}{1.835307in}}{\pgfqpoint{3.990016in}{1.830916in}}{\pgfqpoint{3.982203in}{1.823103in}}%
\pgfpathcurveto{\pgfqpoint{3.974389in}{1.815289in}}{\pgfqpoint{3.969999in}{1.804690in}}{\pgfqpoint{3.969999in}{1.793640in}}%
\pgfpathcurveto{\pgfqpoint{3.969999in}{1.782590in}}{\pgfqpoint{3.974389in}{1.771991in}}{\pgfqpoint{3.982203in}{1.764177in}}%
\pgfpathcurveto{\pgfqpoint{3.990016in}{1.756364in}}{\pgfqpoint{4.000616in}{1.751973in}}{\pgfqpoint{4.011666in}{1.751973in}}%
\pgfpathclose%
\pgfusepath{stroke,fill}%
\end{pgfscope}%
\begin{pgfscope}%
\pgfpathrectangle{\pgfqpoint{0.800000in}{0.528000in}}{\pgfqpoint{4.960000in}{3.696000in}}%
\pgfusepath{clip}%
\pgfsetbuttcap%
\pgfsetroundjoin%
\definecolor{currentfill}{rgb}{0.000000,0.000000,0.000000}%
\pgfsetfillcolor{currentfill}%
\pgfsetlinewidth{1.003750pt}%
\definecolor{currentstroke}{rgb}{0.000000,0.000000,0.000000}%
\pgfsetstrokecolor{currentstroke}%
\pgfsetdash{}{0pt}%
\pgfpathmoveto{\pgfqpoint{4.011666in}{1.713137in}}%
\pgfpathcurveto{\pgfqpoint{4.022716in}{1.713137in}}{\pgfqpoint{4.033315in}{1.717528in}}{\pgfqpoint{4.041128in}{1.725341in}}%
\pgfpathcurveto{\pgfqpoint{4.048942in}{1.733155in}}{\pgfqpoint{4.053332in}{1.743754in}}{\pgfqpoint{4.053332in}{1.754804in}}%
\pgfpathcurveto{\pgfqpoint{4.053332in}{1.765854in}}{\pgfqpoint{4.048942in}{1.776453in}}{\pgfqpoint{4.041128in}{1.784267in}}%
\pgfpathcurveto{\pgfqpoint{4.033315in}{1.792080in}}{\pgfqpoint{4.022716in}{1.796471in}}{\pgfqpoint{4.011666in}{1.796471in}}%
\pgfpathcurveto{\pgfqpoint{4.000616in}{1.796471in}}{\pgfqpoint{3.990016in}{1.792080in}}{\pgfqpoint{3.982203in}{1.784267in}}%
\pgfpathcurveto{\pgfqpoint{3.974389in}{1.776453in}}{\pgfqpoint{3.969999in}{1.765854in}}{\pgfqpoint{3.969999in}{1.754804in}}%
\pgfpathcurveto{\pgfqpoint{3.969999in}{1.743754in}}{\pgfqpoint{3.974389in}{1.733155in}}{\pgfqpoint{3.982203in}{1.725341in}}%
\pgfpathcurveto{\pgfqpoint{3.990016in}{1.717528in}}{\pgfqpoint{4.000616in}{1.713137in}}{\pgfqpoint{4.011666in}{1.713137in}}%
\pgfpathclose%
\pgfusepath{stroke,fill}%
\end{pgfscope}%
\begin{pgfscope}%
\pgfpathrectangle{\pgfqpoint{0.800000in}{0.528000in}}{\pgfqpoint{4.960000in}{3.696000in}}%
\pgfusepath{clip}%
\pgfsetbuttcap%
\pgfsetroundjoin%
\definecolor{currentfill}{rgb}{0.000000,0.000000,0.000000}%
\pgfsetfillcolor{currentfill}%
\pgfsetlinewidth{1.003750pt}%
\definecolor{currentstroke}{rgb}{0.000000,0.000000,0.000000}%
\pgfsetstrokecolor{currentstroke}%
\pgfsetdash{}{0pt}%
\pgfpathmoveto{\pgfqpoint{4.011666in}{1.606339in}}%
\pgfpathcurveto{\pgfqpoint{4.022716in}{1.606339in}}{\pgfqpoint{4.033315in}{1.610729in}}{\pgfqpoint{4.041128in}{1.618543in}}%
\pgfpathcurveto{\pgfqpoint{4.048942in}{1.626356in}}{\pgfqpoint{4.053332in}{1.636955in}}{\pgfqpoint{4.053332in}{1.648005in}}%
\pgfpathcurveto{\pgfqpoint{4.053332in}{1.659056in}}{\pgfqpoint{4.048942in}{1.669655in}}{\pgfqpoint{4.041128in}{1.677468in}}%
\pgfpathcurveto{\pgfqpoint{4.033315in}{1.685282in}}{\pgfqpoint{4.022716in}{1.689672in}}{\pgfqpoint{4.011666in}{1.689672in}}%
\pgfpathcurveto{\pgfqpoint{4.000616in}{1.689672in}}{\pgfqpoint{3.990016in}{1.685282in}}{\pgfqpoint{3.982203in}{1.677468in}}%
\pgfpathcurveto{\pgfqpoint{3.974389in}{1.669655in}}{\pgfqpoint{3.969999in}{1.659056in}}{\pgfqpoint{3.969999in}{1.648005in}}%
\pgfpathcurveto{\pgfqpoint{3.969999in}{1.636955in}}{\pgfqpoint{3.974389in}{1.626356in}}{\pgfqpoint{3.982203in}{1.618543in}}%
\pgfpathcurveto{\pgfqpoint{3.990016in}{1.610729in}}{\pgfqpoint{4.000616in}{1.606339in}}{\pgfqpoint{4.011666in}{1.606339in}}%
\pgfpathclose%
\pgfusepath{stroke,fill}%
\end{pgfscope}%
\begin{pgfscope}%
\pgfpathrectangle{\pgfqpoint{0.800000in}{0.528000in}}{\pgfqpoint{4.960000in}{3.696000in}}%
\pgfusepath{clip}%
\pgfsetbuttcap%
\pgfsetroundjoin%
\definecolor{currentfill}{rgb}{0.000000,0.000000,0.000000}%
\pgfsetfillcolor{currentfill}%
\pgfsetlinewidth{1.003750pt}%
\definecolor{currentstroke}{rgb}{0.000000,0.000000,0.000000}%
\pgfsetstrokecolor{currentstroke}%
\pgfsetdash{}{0pt}%
\pgfpathmoveto{\pgfqpoint{4.011666in}{1.819936in}}%
\pgfpathcurveto{\pgfqpoint{4.022716in}{1.819936in}}{\pgfqpoint{4.033315in}{1.824326in}}{\pgfqpoint{4.041128in}{1.832140in}}%
\pgfpathcurveto{\pgfqpoint{4.048942in}{1.839954in}}{\pgfqpoint{4.053332in}{1.850553in}}{\pgfqpoint{4.053332in}{1.861603in}}%
\pgfpathcurveto{\pgfqpoint{4.053332in}{1.872653in}}{\pgfqpoint{4.048942in}{1.883252in}}{\pgfqpoint{4.041128in}{1.891065in}}%
\pgfpathcurveto{\pgfqpoint{4.033315in}{1.898879in}}{\pgfqpoint{4.022716in}{1.903269in}}{\pgfqpoint{4.011666in}{1.903269in}}%
\pgfpathcurveto{\pgfqpoint{4.000616in}{1.903269in}}{\pgfqpoint{3.990016in}{1.898879in}}{\pgfqpoint{3.982203in}{1.891065in}}%
\pgfpathcurveto{\pgfqpoint{3.974389in}{1.883252in}}{\pgfqpoint{3.969999in}{1.872653in}}{\pgfqpoint{3.969999in}{1.861603in}}%
\pgfpathcurveto{\pgfqpoint{3.969999in}{1.850553in}}{\pgfqpoint{3.974389in}{1.839954in}}{\pgfqpoint{3.982203in}{1.832140in}}%
\pgfpathcurveto{\pgfqpoint{3.990016in}{1.824326in}}{\pgfqpoint{4.000616in}{1.819936in}}{\pgfqpoint{4.011666in}{1.819936in}}%
\pgfpathclose%
\pgfusepath{stroke,fill}%
\end{pgfscope}%
\begin{pgfscope}%
\pgfpathrectangle{\pgfqpoint{0.800000in}{0.528000in}}{\pgfqpoint{4.960000in}{3.696000in}}%
\pgfusepath{clip}%
\pgfsetbuttcap%
\pgfsetroundjoin%
\definecolor{currentfill}{rgb}{0.000000,0.000000,0.000000}%
\pgfsetfillcolor{currentfill}%
\pgfsetlinewidth{1.003750pt}%
\definecolor{currentstroke}{rgb}{0.000000,0.000000,0.000000}%
\pgfsetstrokecolor{currentstroke}%
\pgfsetdash{}{0pt}%
\pgfpathmoveto{\pgfqpoint{4.011666in}{1.742264in}}%
\pgfpathcurveto{\pgfqpoint{4.022716in}{1.742264in}}{\pgfqpoint{4.033315in}{1.746655in}}{\pgfqpoint{4.041128in}{1.754468in}}%
\pgfpathcurveto{\pgfqpoint{4.048942in}{1.762282in}}{\pgfqpoint{4.053332in}{1.772881in}}{\pgfqpoint{4.053332in}{1.783931in}}%
\pgfpathcurveto{\pgfqpoint{4.053332in}{1.794981in}}{\pgfqpoint{4.048942in}{1.805580in}}{\pgfqpoint{4.041128in}{1.813394in}}%
\pgfpathcurveto{\pgfqpoint{4.033315in}{1.821207in}}{\pgfqpoint{4.022716in}{1.825598in}}{\pgfqpoint{4.011666in}{1.825598in}}%
\pgfpathcurveto{\pgfqpoint{4.000616in}{1.825598in}}{\pgfqpoint{3.990016in}{1.821207in}}{\pgfqpoint{3.982203in}{1.813394in}}%
\pgfpathcurveto{\pgfqpoint{3.974389in}{1.805580in}}{\pgfqpoint{3.969999in}{1.794981in}}{\pgfqpoint{3.969999in}{1.783931in}}%
\pgfpathcurveto{\pgfqpoint{3.969999in}{1.772881in}}{\pgfqpoint{3.974389in}{1.762282in}}{\pgfqpoint{3.982203in}{1.754468in}}%
\pgfpathcurveto{\pgfqpoint{3.990016in}{1.746655in}}{\pgfqpoint{4.000616in}{1.742264in}}{\pgfqpoint{4.011666in}{1.742264in}}%
\pgfpathclose%
\pgfusepath{stroke,fill}%
\end{pgfscope}%
\begin{pgfscope}%
\pgfpathrectangle{\pgfqpoint{0.800000in}{0.528000in}}{\pgfqpoint{4.960000in}{3.696000in}}%
\pgfusepath{clip}%
\pgfsetbuttcap%
\pgfsetroundjoin%
\definecolor{currentfill}{rgb}{0.000000,0.000000,0.000000}%
\pgfsetfillcolor{currentfill}%
\pgfsetlinewidth{1.003750pt}%
\definecolor{currentstroke}{rgb}{0.000000,0.000000,0.000000}%
\pgfsetstrokecolor{currentstroke}%
\pgfsetdash{}{0pt}%
\pgfpathmoveto{\pgfqpoint{4.011666in}{1.742264in}}%
\pgfpathcurveto{\pgfqpoint{4.022716in}{1.742264in}}{\pgfqpoint{4.033315in}{1.746655in}}{\pgfqpoint{4.041128in}{1.754468in}}%
\pgfpathcurveto{\pgfqpoint{4.048942in}{1.762282in}}{\pgfqpoint{4.053332in}{1.772881in}}{\pgfqpoint{4.053332in}{1.783931in}}%
\pgfpathcurveto{\pgfqpoint{4.053332in}{1.794981in}}{\pgfqpoint{4.048942in}{1.805580in}}{\pgfqpoint{4.041128in}{1.813394in}}%
\pgfpathcurveto{\pgfqpoint{4.033315in}{1.821207in}}{\pgfqpoint{4.022716in}{1.825598in}}{\pgfqpoint{4.011666in}{1.825598in}}%
\pgfpathcurveto{\pgfqpoint{4.000616in}{1.825598in}}{\pgfqpoint{3.990016in}{1.821207in}}{\pgfqpoint{3.982203in}{1.813394in}}%
\pgfpathcurveto{\pgfqpoint{3.974389in}{1.805580in}}{\pgfqpoint{3.969999in}{1.794981in}}{\pgfqpoint{3.969999in}{1.783931in}}%
\pgfpathcurveto{\pgfqpoint{3.969999in}{1.772881in}}{\pgfqpoint{3.974389in}{1.762282in}}{\pgfqpoint{3.982203in}{1.754468in}}%
\pgfpathcurveto{\pgfqpoint{3.990016in}{1.746655in}}{\pgfqpoint{4.000616in}{1.742264in}}{\pgfqpoint{4.011666in}{1.742264in}}%
\pgfpathclose%
\pgfusepath{stroke,fill}%
\end{pgfscope}%
\begin{pgfscope}%
\pgfpathrectangle{\pgfqpoint{0.800000in}{0.528000in}}{\pgfqpoint{4.960000in}{3.696000in}}%
\pgfusepath{clip}%
\pgfsetbuttcap%
\pgfsetroundjoin%
\definecolor{currentfill}{rgb}{0.000000,0.000000,0.000000}%
\pgfsetfillcolor{currentfill}%
\pgfsetlinewidth{1.003750pt}%
\definecolor{currentstroke}{rgb}{0.000000,0.000000,0.000000}%
\pgfsetstrokecolor{currentstroke}%
\pgfsetdash{}{0pt}%
\pgfpathmoveto{\pgfqpoint{4.011666in}{1.800518in}}%
\pgfpathcurveto{\pgfqpoint{4.022716in}{1.800518in}}{\pgfqpoint{4.033315in}{1.804908in}}{\pgfqpoint{4.041128in}{1.812722in}}%
\pgfpathcurveto{\pgfqpoint{4.048942in}{1.820536in}}{\pgfqpoint{4.053332in}{1.831135in}}{\pgfqpoint{4.053332in}{1.842185in}}%
\pgfpathcurveto{\pgfqpoint{4.053332in}{1.853235in}}{\pgfqpoint{4.048942in}{1.863834in}}{\pgfqpoint{4.041128in}{1.871648in}}%
\pgfpathcurveto{\pgfqpoint{4.033315in}{1.879461in}}{\pgfqpoint{4.022716in}{1.883851in}}{\pgfqpoint{4.011666in}{1.883851in}}%
\pgfpathcurveto{\pgfqpoint{4.000616in}{1.883851in}}{\pgfqpoint{3.990016in}{1.879461in}}{\pgfqpoint{3.982203in}{1.871648in}}%
\pgfpathcurveto{\pgfqpoint{3.974389in}{1.863834in}}{\pgfqpoint{3.969999in}{1.853235in}}{\pgfqpoint{3.969999in}{1.842185in}}%
\pgfpathcurveto{\pgfqpoint{3.969999in}{1.831135in}}{\pgfqpoint{3.974389in}{1.820536in}}{\pgfqpoint{3.982203in}{1.812722in}}%
\pgfpathcurveto{\pgfqpoint{3.990016in}{1.804908in}}{\pgfqpoint{4.000616in}{1.800518in}}{\pgfqpoint{4.011666in}{1.800518in}}%
\pgfpathclose%
\pgfusepath{stroke,fill}%
\end{pgfscope}%
\begin{pgfscope}%
\pgfpathrectangle{\pgfqpoint{0.800000in}{0.528000in}}{\pgfqpoint{4.960000in}{3.696000in}}%
\pgfusepath{clip}%
\pgfsetbuttcap%
\pgfsetroundjoin%
\definecolor{currentfill}{rgb}{0.000000,0.000000,0.000000}%
\pgfsetfillcolor{currentfill}%
\pgfsetlinewidth{1.003750pt}%
\definecolor{currentstroke}{rgb}{0.000000,0.000000,0.000000}%
\pgfsetstrokecolor{currentstroke}%
\pgfsetdash{}{0pt}%
\pgfpathmoveto{\pgfqpoint{4.011666in}{1.625757in}}%
\pgfpathcurveto{\pgfqpoint{4.022716in}{1.625757in}}{\pgfqpoint{4.033315in}{1.630147in}}{\pgfqpoint{4.041128in}{1.637961in}}%
\pgfpathcurveto{\pgfqpoint{4.048942in}{1.645774in}}{\pgfqpoint{4.053332in}{1.656373in}}{\pgfqpoint{4.053332in}{1.667423in}}%
\pgfpathcurveto{\pgfqpoint{4.053332in}{1.678473in}}{\pgfqpoint{4.048942in}{1.689072in}}{\pgfqpoint{4.041128in}{1.696886in}}%
\pgfpathcurveto{\pgfqpoint{4.033315in}{1.704700in}}{\pgfqpoint{4.022716in}{1.709090in}}{\pgfqpoint{4.011666in}{1.709090in}}%
\pgfpathcurveto{\pgfqpoint{4.000616in}{1.709090in}}{\pgfqpoint{3.990016in}{1.704700in}}{\pgfqpoint{3.982203in}{1.696886in}}%
\pgfpathcurveto{\pgfqpoint{3.974389in}{1.689072in}}{\pgfqpoint{3.969999in}{1.678473in}}{\pgfqpoint{3.969999in}{1.667423in}}%
\pgfpathcurveto{\pgfqpoint{3.969999in}{1.656373in}}{\pgfqpoint{3.974389in}{1.645774in}}{\pgfqpoint{3.982203in}{1.637961in}}%
\pgfpathcurveto{\pgfqpoint{3.990016in}{1.630147in}}{\pgfqpoint{4.000616in}{1.625757in}}{\pgfqpoint{4.011666in}{1.625757in}}%
\pgfpathclose%
\pgfusepath{stroke,fill}%
\end{pgfscope}%
\begin{pgfscope}%
\pgfpathrectangle{\pgfqpoint{0.800000in}{0.528000in}}{\pgfqpoint{4.960000in}{3.696000in}}%
\pgfusepath{clip}%
\pgfsetbuttcap%
\pgfsetroundjoin%
\definecolor{currentfill}{rgb}{0.000000,0.000000,0.000000}%
\pgfsetfillcolor{currentfill}%
\pgfsetlinewidth{1.003750pt}%
\definecolor{currentstroke}{rgb}{0.000000,0.000000,0.000000}%
\pgfsetstrokecolor{currentstroke}%
\pgfsetdash{}{0pt}%
\pgfpathmoveto{\pgfqpoint{4.011666in}{1.761682in}}%
\pgfpathcurveto{\pgfqpoint{4.022716in}{1.761682in}}{\pgfqpoint{4.033315in}{1.766072in}}{\pgfqpoint{4.041128in}{1.773886in}}%
\pgfpathcurveto{\pgfqpoint{4.048942in}{1.781700in}}{\pgfqpoint{4.053332in}{1.792299in}}{\pgfqpoint{4.053332in}{1.803349in}}%
\pgfpathcurveto{\pgfqpoint{4.053332in}{1.814399in}}{\pgfqpoint{4.048942in}{1.824998in}}{\pgfqpoint{4.041128in}{1.832812in}}%
\pgfpathcurveto{\pgfqpoint{4.033315in}{1.840625in}}{\pgfqpoint{4.022716in}{1.845016in}}{\pgfqpoint{4.011666in}{1.845016in}}%
\pgfpathcurveto{\pgfqpoint{4.000616in}{1.845016in}}{\pgfqpoint{3.990016in}{1.840625in}}{\pgfqpoint{3.982203in}{1.832812in}}%
\pgfpathcurveto{\pgfqpoint{3.974389in}{1.824998in}}{\pgfqpoint{3.969999in}{1.814399in}}{\pgfqpoint{3.969999in}{1.803349in}}%
\pgfpathcurveto{\pgfqpoint{3.969999in}{1.792299in}}{\pgfqpoint{3.974389in}{1.781700in}}{\pgfqpoint{3.982203in}{1.773886in}}%
\pgfpathcurveto{\pgfqpoint{3.990016in}{1.766072in}}{\pgfqpoint{4.000616in}{1.761682in}}{\pgfqpoint{4.011666in}{1.761682in}}%
\pgfpathclose%
\pgfusepath{stroke,fill}%
\end{pgfscope}%
\begin{pgfscope}%
\pgfpathrectangle{\pgfqpoint{0.800000in}{0.528000in}}{\pgfqpoint{4.960000in}{3.696000in}}%
\pgfusepath{clip}%
\pgfsetbuttcap%
\pgfsetroundjoin%
\definecolor{currentfill}{rgb}{0.000000,0.000000,0.000000}%
\pgfsetfillcolor{currentfill}%
\pgfsetlinewidth{1.003750pt}%
\definecolor{currentstroke}{rgb}{0.000000,0.000000,0.000000}%
\pgfsetstrokecolor{currentstroke}%
\pgfsetdash{}{0pt}%
\pgfpathmoveto{\pgfqpoint{4.011666in}{1.577212in}}%
\pgfpathcurveto{\pgfqpoint{4.022716in}{1.577212in}}{\pgfqpoint{4.033315in}{1.581602in}}{\pgfqpoint{4.041128in}{1.589416in}}%
\pgfpathcurveto{\pgfqpoint{4.048942in}{1.597229in}}{\pgfqpoint{4.053332in}{1.607828in}}{\pgfqpoint{4.053332in}{1.618878in}}%
\pgfpathcurveto{\pgfqpoint{4.053332in}{1.629929in}}{\pgfqpoint{4.048942in}{1.640528in}}{\pgfqpoint{4.041128in}{1.648341in}}%
\pgfpathcurveto{\pgfqpoint{4.033315in}{1.656155in}}{\pgfqpoint{4.022716in}{1.660545in}}{\pgfqpoint{4.011666in}{1.660545in}}%
\pgfpathcurveto{\pgfqpoint{4.000616in}{1.660545in}}{\pgfqpoint{3.990016in}{1.656155in}}{\pgfqpoint{3.982203in}{1.648341in}}%
\pgfpathcurveto{\pgfqpoint{3.974389in}{1.640528in}}{\pgfqpoint{3.969999in}{1.629929in}}{\pgfqpoint{3.969999in}{1.618878in}}%
\pgfpathcurveto{\pgfqpoint{3.969999in}{1.607828in}}{\pgfqpoint{3.974389in}{1.597229in}}{\pgfqpoint{3.982203in}{1.589416in}}%
\pgfpathcurveto{\pgfqpoint{3.990016in}{1.581602in}}{\pgfqpoint{4.000616in}{1.577212in}}{\pgfqpoint{4.011666in}{1.577212in}}%
\pgfpathclose%
\pgfusepath{stroke,fill}%
\end{pgfscope}%
\begin{pgfscope}%
\pgfpathrectangle{\pgfqpoint{0.800000in}{0.528000in}}{\pgfqpoint{4.960000in}{3.696000in}}%
\pgfusepath{clip}%
\pgfsetbuttcap%
\pgfsetroundjoin%
\definecolor{currentfill}{rgb}{0.000000,0.000000,0.000000}%
\pgfsetfillcolor{currentfill}%
\pgfsetlinewidth{1.003750pt}%
\definecolor{currentstroke}{rgb}{0.000000,0.000000,0.000000}%
\pgfsetstrokecolor{currentstroke}%
\pgfsetdash{}{0pt}%
\pgfpathmoveto{\pgfqpoint{4.011666in}{1.616048in}}%
\pgfpathcurveto{\pgfqpoint{4.022716in}{1.616048in}}{\pgfqpoint{4.033315in}{1.620438in}}{\pgfqpoint{4.041128in}{1.628252in}}%
\pgfpathcurveto{\pgfqpoint{4.048942in}{1.636065in}}{\pgfqpoint{4.053332in}{1.646664in}}{\pgfqpoint{4.053332in}{1.657714in}}%
\pgfpathcurveto{\pgfqpoint{4.053332in}{1.668764in}}{\pgfqpoint{4.048942in}{1.679364in}}{\pgfqpoint{4.041128in}{1.687177in}}%
\pgfpathcurveto{\pgfqpoint{4.033315in}{1.694991in}}{\pgfqpoint{4.022716in}{1.699381in}}{\pgfqpoint{4.011666in}{1.699381in}}%
\pgfpathcurveto{\pgfqpoint{4.000616in}{1.699381in}}{\pgfqpoint{3.990016in}{1.694991in}}{\pgfqpoint{3.982203in}{1.687177in}}%
\pgfpathcurveto{\pgfqpoint{3.974389in}{1.679364in}}{\pgfqpoint{3.969999in}{1.668764in}}{\pgfqpoint{3.969999in}{1.657714in}}%
\pgfpathcurveto{\pgfqpoint{3.969999in}{1.646664in}}{\pgfqpoint{3.974389in}{1.636065in}}{\pgfqpoint{3.982203in}{1.628252in}}%
\pgfpathcurveto{\pgfqpoint{3.990016in}{1.620438in}}{\pgfqpoint{4.000616in}{1.616048in}}{\pgfqpoint{4.011666in}{1.616048in}}%
\pgfpathclose%
\pgfusepath{stroke,fill}%
\end{pgfscope}%
\begin{pgfscope}%
\pgfpathrectangle{\pgfqpoint{0.800000in}{0.528000in}}{\pgfqpoint{4.960000in}{3.696000in}}%
\pgfusepath{clip}%
\pgfsetbuttcap%
\pgfsetroundjoin%
\definecolor{currentfill}{rgb}{0.000000,0.000000,0.000000}%
\pgfsetfillcolor{currentfill}%
\pgfsetlinewidth{1.003750pt}%
\definecolor{currentstroke}{rgb}{0.000000,0.000000,0.000000}%
\pgfsetstrokecolor{currentstroke}%
\pgfsetdash{}{0pt}%
\pgfpathmoveto{\pgfqpoint{4.011666in}{1.713137in}}%
\pgfpathcurveto{\pgfqpoint{4.022716in}{1.713137in}}{\pgfqpoint{4.033315in}{1.717528in}}{\pgfqpoint{4.041128in}{1.725341in}}%
\pgfpathcurveto{\pgfqpoint{4.048942in}{1.733155in}}{\pgfqpoint{4.053332in}{1.743754in}}{\pgfqpoint{4.053332in}{1.754804in}}%
\pgfpathcurveto{\pgfqpoint{4.053332in}{1.765854in}}{\pgfqpoint{4.048942in}{1.776453in}}{\pgfqpoint{4.041128in}{1.784267in}}%
\pgfpathcurveto{\pgfqpoint{4.033315in}{1.792080in}}{\pgfqpoint{4.022716in}{1.796471in}}{\pgfqpoint{4.011666in}{1.796471in}}%
\pgfpathcurveto{\pgfqpoint{4.000616in}{1.796471in}}{\pgfqpoint{3.990016in}{1.792080in}}{\pgfqpoint{3.982203in}{1.784267in}}%
\pgfpathcurveto{\pgfqpoint{3.974389in}{1.776453in}}{\pgfqpoint{3.969999in}{1.765854in}}{\pgfqpoint{3.969999in}{1.754804in}}%
\pgfpathcurveto{\pgfqpoint{3.969999in}{1.743754in}}{\pgfqpoint{3.974389in}{1.733155in}}{\pgfqpoint{3.982203in}{1.725341in}}%
\pgfpathcurveto{\pgfqpoint{3.990016in}{1.717528in}}{\pgfqpoint{4.000616in}{1.713137in}}{\pgfqpoint{4.011666in}{1.713137in}}%
\pgfpathclose%
\pgfusepath{stroke,fill}%
\end{pgfscope}%
\begin{pgfscope}%
\pgfpathrectangle{\pgfqpoint{0.800000in}{0.528000in}}{\pgfqpoint{4.960000in}{3.696000in}}%
\pgfusepath{clip}%
\pgfsetbuttcap%
\pgfsetroundjoin%
\definecolor{currentfill}{rgb}{0.000000,0.000000,0.000000}%
\pgfsetfillcolor{currentfill}%
\pgfsetlinewidth{1.003750pt}%
\definecolor{currentstroke}{rgb}{0.000000,0.000000,0.000000}%
\pgfsetstrokecolor{currentstroke}%
\pgfsetdash{}{0pt}%
\pgfpathmoveto{\pgfqpoint{4.011666in}{1.693719in}}%
\pgfpathcurveto{\pgfqpoint{4.022716in}{1.693719in}}{\pgfqpoint{4.033315in}{1.698110in}}{\pgfqpoint{4.041128in}{1.705923in}}%
\pgfpathcurveto{\pgfqpoint{4.048942in}{1.713737in}}{\pgfqpoint{4.053332in}{1.724336in}}{\pgfqpoint{4.053332in}{1.735386in}}%
\pgfpathcurveto{\pgfqpoint{4.053332in}{1.746436in}}{\pgfqpoint{4.048942in}{1.757035in}}{\pgfqpoint{4.041128in}{1.764849in}}%
\pgfpathcurveto{\pgfqpoint{4.033315in}{1.772663in}}{\pgfqpoint{4.022716in}{1.777053in}}{\pgfqpoint{4.011666in}{1.777053in}}%
\pgfpathcurveto{\pgfqpoint{4.000616in}{1.777053in}}{\pgfqpoint{3.990016in}{1.772663in}}{\pgfqpoint{3.982203in}{1.764849in}}%
\pgfpathcurveto{\pgfqpoint{3.974389in}{1.757035in}}{\pgfqpoint{3.969999in}{1.746436in}}{\pgfqpoint{3.969999in}{1.735386in}}%
\pgfpathcurveto{\pgfqpoint{3.969999in}{1.724336in}}{\pgfqpoint{3.974389in}{1.713737in}}{\pgfqpoint{3.982203in}{1.705923in}}%
\pgfpathcurveto{\pgfqpoint{3.990016in}{1.698110in}}{\pgfqpoint{4.000616in}{1.693719in}}{\pgfqpoint{4.011666in}{1.693719in}}%
\pgfpathclose%
\pgfusepath{stroke,fill}%
\end{pgfscope}%
\begin{pgfscope}%
\pgfpathrectangle{\pgfqpoint{0.800000in}{0.528000in}}{\pgfqpoint{4.960000in}{3.696000in}}%
\pgfusepath{clip}%
\pgfsetbuttcap%
\pgfsetroundjoin%
\definecolor{currentfill}{rgb}{0.000000,0.000000,0.000000}%
\pgfsetfillcolor{currentfill}%
\pgfsetlinewidth{1.003750pt}%
\definecolor{currentstroke}{rgb}{0.000000,0.000000,0.000000}%
\pgfsetstrokecolor{currentstroke}%
\pgfsetdash{}{0pt}%
\pgfpathmoveto{\pgfqpoint{4.011666in}{1.742264in}}%
\pgfpathcurveto{\pgfqpoint{4.022716in}{1.742264in}}{\pgfqpoint{4.033315in}{1.746655in}}{\pgfqpoint{4.041128in}{1.754468in}}%
\pgfpathcurveto{\pgfqpoint{4.048942in}{1.762282in}}{\pgfqpoint{4.053332in}{1.772881in}}{\pgfqpoint{4.053332in}{1.783931in}}%
\pgfpathcurveto{\pgfqpoint{4.053332in}{1.794981in}}{\pgfqpoint{4.048942in}{1.805580in}}{\pgfqpoint{4.041128in}{1.813394in}}%
\pgfpathcurveto{\pgfqpoint{4.033315in}{1.821207in}}{\pgfqpoint{4.022716in}{1.825598in}}{\pgfqpoint{4.011666in}{1.825598in}}%
\pgfpathcurveto{\pgfqpoint{4.000616in}{1.825598in}}{\pgfqpoint{3.990016in}{1.821207in}}{\pgfqpoint{3.982203in}{1.813394in}}%
\pgfpathcurveto{\pgfqpoint{3.974389in}{1.805580in}}{\pgfqpoint{3.969999in}{1.794981in}}{\pgfqpoint{3.969999in}{1.783931in}}%
\pgfpathcurveto{\pgfqpoint{3.969999in}{1.772881in}}{\pgfqpoint{3.974389in}{1.762282in}}{\pgfqpoint{3.982203in}{1.754468in}}%
\pgfpathcurveto{\pgfqpoint{3.990016in}{1.746655in}}{\pgfqpoint{4.000616in}{1.742264in}}{\pgfqpoint{4.011666in}{1.742264in}}%
\pgfpathclose%
\pgfusepath{stroke,fill}%
\end{pgfscope}%
\begin{pgfscope}%
\pgfpathrectangle{\pgfqpoint{0.800000in}{0.528000in}}{\pgfqpoint{4.960000in}{3.696000in}}%
\pgfusepath{clip}%
\pgfsetbuttcap%
\pgfsetroundjoin%
\definecolor{currentfill}{rgb}{0.000000,0.000000,0.000000}%
\pgfsetfillcolor{currentfill}%
\pgfsetlinewidth{1.003750pt}%
\definecolor{currentstroke}{rgb}{0.000000,0.000000,0.000000}%
\pgfsetstrokecolor{currentstroke}%
\pgfsetdash{}{0pt}%
\pgfpathmoveto{\pgfqpoint{4.011666in}{1.577212in}}%
\pgfpathcurveto{\pgfqpoint{4.022716in}{1.577212in}}{\pgfqpoint{4.033315in}{1.581602in}}{\pgfqpoint{4.041128in}{1.589416in}}%
\pgfpathcurveto{\pgfqpoint{4.048942in}{1.597229in}}{\pgfqpoint{4.053332in}{1.607828in}}{\pgfqpoint{4.053332in}{1.618878in}}%
\pgfpathcurveto{\pgfqpoint{4.053332in}{1.629929in}}{\pgfqpoint{4.048942in}{1.640528in}}{\pgfqpoint{4.041128in}{1.648341in}}%
\pgfpathcurveto{\pgfqpoint{4.033315in}{1.656155in}}{\pgfqpoint{4.022716in}{1.660545in}}{\pgfqpoint{4.011666in}{1.660545in}}%
\pgfpathcurveto{\pgfqpoint{4.000616in}{1.660545in}}{\pgfqpoint{3.990016in}{1.656155in}}{\pgfqpoint{3.982203in}{1.648341in}}%
\pgfpathcurveto{\pgfqpoint{3.974389in}{1.640528in}}{\pgfqpoint{3.969999in}{1.629929in}}{\pgfqpoint{3.969999in}{1.618878in}}%
\pgfpathcurveto{\pgfqpoint{3.969999in}{1.607828in}}{\pgfqpoint{3.974389in}{1.597229in}}{\pgfqpoint{3.982203in}{1.589416in}}%
\pgfpathcurveto{\pgfqpoint{3.990016in}{1.581602in}}{\pgfqpoint{4.000616in}{1.577212in}}{\pgfqpoint{4.011666in}{1.577212in}}%
\pgfpathclose%
\pgfusepath{stroke,fill}%
\end{pgfscope}%
\begin{pgfscope}%
\pgfpathrectangle{\pgfqpoint{0.800000in}{0.528000in}}{\pgfqpoint{4.960000in}{3.696000in}}%
\pgfusepath{clip}%
\pgfsetbuttcap%
\pgfsetroundjoin%
\definecolor{currentfill}{rgb}{0.000000,0.000000,0.000000}%
\pgfsetfillcolor{currentfill}%
\pgfsetlinewidth{1.003750pt}%
\definecolor{currentstroke}{rgb}{0.000000,0.000000,0.000000}%
\pgfsetstrokecolor{currentstroke}%
\pgfsetdash{}{0pt}%
\pgfpathmoveto{\pgfqpoint{4.011666in}{1.713137in}}%
\pgfpathcurveto{\pgfqpoint{4.022716in}{1.713137in}}{\pgfqpoint{4.033315in}{1.717528in}}{\pgfqpoint{4.041128in}{1.725341in}}%
\pgfpathcurveto{\pgfqpoint{4.048942in}{1.733155in}}{\pgfqpoint{4.053332in}{1.743754in}}{\pgfqpoint{4.053332in}{1.754804in}}%
\pgfpathcurveto{\pgfqpoint{4.053332in}{1.765854in}}{\pgfqpoint{4.048942in}{1.776453in}}{\pgfqpoint{4.041128in}{1.784267in}}%
\pgfpathcurveto{\pgfqpoint{4.033315in}{1.792080in}}{\pgfqpoint{4.022716in}{1.796471in}}{\pgfqpoint{4.011666in}{1.796471in}}%
\pgfpathcurveto{\pgfqpoint{4.000616in}{1.796471in}}{\pgfqpoint{3.990016in}{1.792080in}}{\pgfqpoint{3.982203in}{1.784267in}}%
\pgfpathcurveto{\pgfqpoint{3.974389in}{1.776453in}}{\pgfqpoint{3.969999in}{1.765854in}}{\pgfqpoint{3.969999in}{1.754804in}}%
\pgfpathcurveto{\pgfqpoint{3.969999in}{1.743754in}}{\pgfqpoint{3.974389in}{1.733155in}}{\pgfqpoint{3.982203in}{1.725341in}}%
\pgfpathcurveto{\pgfqpoint{3.990016in}{1.717528in}}{\pgfqpoint{4.000616in}{1.713137in}}{\pgfqpoint{4.011666in}{1.713137in}}%
\pgfpathclose%
\pgfusepath{stroke,fill}%
\end{pgfscope}%
\begin{pgfscope}%
\pgfpathrectangle{\pgfqpoint{0.800000in}{0.528000in}}{\pgfqpoint{4.960000in}{3.696000in}}%
\pgfusepath{clip}%
\pgfsetbuttcap%
\pgfsetroundjoin%
\definecolor{currentfill}{rgb}{0.000000,0.000000,0.000000}%
\pgfsetfillcolor{currentfill}%
\pgfsetlinewidth{1.003750pt}%
\definecolor{currentstroke}{rgb}{0.000000,0.000000,0.000000}%
\pgfsetstrokecolor{currentstroke}%
\pgfsetdash{}{0pt}%
\pgfpathmoveto{\pgfqpoint{4.011666in}{1.616048in}}%
\pgfpathcurveto{\pgfqpoint{4.022716in}{1.616048in}}{\pgfqpoint{4.033315in}{1.620438in}}{\pgfqpoint{4.041128in}{1.628252in}}%
\pgfpathcurveto{\pgfqpoint{4.048942in}{1.636065in}}{\pgfqpoint{4.053332in}{1.646664in}}{\pgfqpoint{4.053332in}{1.657714in}}%
\pgfpathcurveto{\pgfqpoint{4.053332in}{1.668764in}}{\pgfqpoint{4.048942in}{1.679364in}}{\pgfqpoint{4.041128in}{1.687177in}}%
\pgfpathcurveto{\pgfqpoint{4.033315in}{1.694991in}}{\pgfqpoint{4.022716in}{1.699381in}}{\pgfqpoint{4.011666in}{1.699381in}}%
\pgfpathcurveto{\pgfqpoint{4.000616in}{1.699381in}}{\pgfqpoint{3.990016in}{1.694991in}}{\pgfqpoint{3.982203in}{1.687177in}}%
\pgfpathcurveto{\pgfqpoint{3.974389in}{1.679364in}}{\pgfqpoint{3.969999in}{1.668764in}}{\pgfqpoint{3.969999in}{1.657714in}}%
\pgfpathcurveto{\pgfqpoint{3.969999in}{1.646664in}}{\pgfqpoint{3.974389in}{1.636065in}}{\pgfqpoint{3.982203in}{1.628252in}}%
\pgfpathcurveto{\pgfqpoint{3.990016in}{1.620438in}}{\pgfqpoint{4.000616in}{1.616048in}}{\pgfqpoint{4.011666in}{1.616048in}}%
\pgfpathclose%
\pgfusepath{stroke,fill}%
\end{pgfscope}%
\begin{pgfscope}%
\pgfpathrectangle{\pgfqpoint{0.800000in}{0.528000in}}{\pgfqpoint{4.960000in}{3.696000in}}%
\pgfusepath{clip}%
\pgfsetbuttcap%
\pgfsetroundjoin%
\definecolor{currentfill}{rgb}{0.000000,0.000000,0.000000}%
\pgfsetfillcolor{currentfill}%
\pgfsetlinewidth{1.003750pt}%
\definecolor{currentstroke}{rgb}{0.000000,0.000000,0.000000}%
\pgfsetstrokecolor{currentstroke}%
\pgfsetdash{}{0pt}%
\pgfpathmoveto{\pgfqpoint{4.011666in}{1.664593in}}%
\pgfpathcurveto{\pgfqpoint{4.022716in}{1.664593in}}{\pgfqpoint{4.033315in}{1.668983in}}{\pgfqpoint{4.041128in}{1.676796in}}%
\pgfpathcurveto{\pgfqpoint{4.048942in}{1.684610in}}{\pgfqpoint{4.053332in}{1.695209in}}{\pgfqpoint{4.053332in}{1.706259in}}%
\pgfpathcurveto{\pgfqpoint{4.053332in}{1.717309in}}{\pgfqpoint{4.048942in}{1.727908in}}{\pgfqpoint{4.041128in}{1.735722in}}%
\pgfpathcurveto{\pgfqpoint{4.033315in}{1.743536in}}{\pgfqpoint{4.022716in}{1.747926in}}{\pgfqpoint{4.011666in}{1.747926in}}%
\pgfpathcurveto{\pgfqpoint{4.000616in}{1.747926in}}{\pgfqpoint{3.990016in}{1.743536in}}{\pgfqpoint{3.982203in}{1.735722in}}%
\pgfpathcurveto{\pgfqpoint{3.974389in}{1.727908in}}{\pgfqpoint{3.969999in}{1.717309in}}{\pgfqpoint{3.969999in}{1.706259in}}%
\pgfpathcurveto{\pgfqpoint{3.969999in}{1.695209in}}{\pgfqpoint{3.974389in}{1.684610in}}{\pgfqpoint{3.982203in}{1.676796in}}%
\pgfpathcurveto{\pgfqpoint{3.990016in}{1.668983in}}{\pgfqpoint{4.000616in}{1.664593in}}{\pgfqpoint{4.011666in}{1.664593in}}%
\pgfpathclose%
\pgfusepath{stroke,fill}%
\end{pgfscope}%
\begin{pgfscope}%
\pgfpathrectangle{\pgfqpoint{0.800000in}{0.528000in}}{\pgfqpoint{4.960000in}{3.696000in}}%
\pgfusepath{clip}%
\pgfsetbuttcap%
\pgfsetroundjoin%
\definecolor{currentfill}{rgb}{0.000000,0.000000,0.000000}%
\pgfsetfillcolor{currentfill}%
\pgfsetlinewidth{1.003750pt}%
\definecolor{currentstroke}{rgb}{0.000000,0.000000,0.000000}%
\pgfsetstrokecolor{currentstroke}%
\pgfsetdash{}{0pt}%
\pgfpathmoveto{\pgfqpoint{4.011666in}{1.742264in}}%
\pgfpathcurveto{\pgfqpoint{4.022716in}{1.742264in}}{\pgfqpoint{4.033315in}{1.746655in}}{\pgfqpoint{4.041128in}{1.754468in}}%
\pgfpathcurveto{\pgfqpoint{4.048942in}{1.762282in}}{\pgfqpoint{4.053332in}{1.772881in}}{\pgfqpoint{4.053332in}{1.783931in}}%
\pgfpathcurveto{\pgfqpoint{4.053332in}{1.794981in}}{\pgfqpoint{4.048942in}{1.805580in}}{\pgfqpoint{4.041128in}{1.813394in}}%
\pgfpathcurveto{\pgfqpoint{4.033315in}{1.821207in}}{\pgfqpoint{4.022716in}{1.825598in}}{\pgfqpoint{4.011666in}{1.825598in}}%
\pgfpathcurveto{\pgfqpoint{4.000616in}{1.825598in}}{\pgfqpoint{3.990016in}{1.821207in}}{\pgfqpoint{3.982203in}{1.813394in}}%
\pgfpathcurveto{\pgfqpoint{3.974389in}{1.805580in}}{\pgfqpoint{3.969999in}{1.794981in}}{\pgfqpoint{3.969999in}{1.783931in}}%
\pgfpathcurveto{\pgfqpoint{3.969999in}{1.772881in}}{\pgfqpoint{3.974389in}{1.762282in}}{\pgfqpoint{3.982203in}{1.754468in}}%
\pgfpathcurveto{\pgfqpoint{3.990016in}{1.746655in}}{\pgfqpoint{4.000616in}{1.742264in}}{\pgfqpoint{4.011666in}{1.742264in}}%
\pgfpathclose%
\pgfusepath{stroke,fill}%
\end{pgfscope}%
\begin{pgfscope}%
\pgfpathrectangle{\pgfqpoint{0.800000in}{0.528000in}}{\pgfqpoint{4.960000in}{3.696000in}}%
\pgfusepath{clip}%
\pgfsetbuttcap%
\pgfsetroundjoin%
\definecolor{currentfill}{rgb}{0.000000,0.000000,0.000000}%
\pgfsetfillcolor{currentfill}%
\pgfsetlinewidth{1.003750pt}%
\definecolor{currentstroke}{rgb}{0.000000,0.000000,0.000000}%
\pgfsetstrokecolor{currentstroke}%
\pgfsetdash{}{0pt}%
\pgfpathmoveto{\pgfqpoint{4.011666in}{1.703428in}}%
\pgfpathcurveto{\pgfqpoint{4.022716in}{1.703428in}}{\pgfqpoint{4.033315in}{1.707819in}}{\pgfqpoint{4.041128in}{1.715632in}}%
\pgfpathcurveto{\pgfqpoint{4.048942in}{1.723446in}}{\pgfqpoint{4.053332in}{1.734045in}}{\pgfqpoint{4.053332in}{1.745095in}}%
\pgfpathcurveto{\pgfqpoint{4.053332in}{1.756145in}}{\pgfqpoint{4.048942in}{1.766744in}}{\pgfqpoint{4.041128in}{1.774558in}}%
\pgfpathcurveto{\pgfqpoint{4.033315in}{1.782371in}}{\pgfqpoint{4.022716in}{1.786762in}}{\pgfqpoint{4.011666in}{1.786762in}}%
\pgfpathcurveto{\pgfqpoint{4.000616in}{1.786762in}}{\pgfqpoint{3.990016in}{1.782371in}}{\pgfqpoint{3.982203in}{1.774558in}}%
\pgfpathcurveto{\pgfqpoint{3.974389in}{1.766744in}}{\pgfqpoint{3.969999in}{1.756145in}}{\pgfqpoint{3.969999in}{1.745095in}}%
\pgfpathcurveto{\pgfqpoint{3.969999in}{1.734045in}}{\pgfqpoint{3.974389in}{1.723446in}}{\pgfqpoint{3.982203in}{1.715632in}}%
\pgfpathcurveto{\pgfqpoint{3.990016in}{1.707819in}}{\pgfqpoint{4.000616in}{1.703428in}}{\pgfqpoint{4.011666in}{1.703428in}}%
\pgfpathclose%
\pgfusepath{stroke,fill}%
\end{pgfscope}%
\begin{pgfscope}%
\pgfpathrectangle{\pgfqpoint{0.800000in}{0.528000in}}{\pgfqpoint{4.960000in}{3.696000in}}%
\pgfusepath{clip}%
\pgfsetbuttcap%
\pgfsetroundjoin%
\definecolor{currentfill}{rgb}{0.000000,0.000000,0.000000}%
\pgfsetfillcolor{currentfill}%
\pgfsetlinewidth{1.003750pt}%
\definecolor{currentstroke}{rgb}{0.000000,0.000000,0.000000}%
\pgfsetstrokecolor{currentstroke}%
\pgfsetdash{}{0pt}%
\pgfpathmoveto{\pgfqpoint{4.011666in}{1.596630in}}%
\pgfpathcurveto{\pgfqpoint{4.022716in}{1.596630in}}{\pgfqpoint{4.033315in}{1.601020in}}{\pgfqpoint{4.041128in}{1.608834in}}%
\pgfpathcurveto{\pgfqpoint{4.048942in}{1.616647in}}{\pgfqpoint{4.053332in}{1.627246in}}{\pgfqpoint{4.053332in}{1.638296in}}%
\pgfpathcurveto{\pgfqpoint{4.053332in}{1.649347in}}{\pgfqpoint{4.048942in}{1.659946in}}{\pgfqpoint{4.041128in}{1.667759in}}%
\pgfpathcurveto{\pgfqpoint{4.033315in}{1.675573in}}{\pgfqpoint{4.022716in}{1.679963in}}{\pgfqpoint{4.011666in}{1.679963in}}%
\pgfpathcurveto{\pgfqpoint{4.000616in}{1.679963in}}{\pgfqpoint{3.990016in}{1.675573in}}{\pgfqpoint{3.982203in}{1.667759in}}%
\pgfpathcurveto{\pgfqpoint{3.974389in}{1.659946in}}{\pgfqpoint{3.969999in}{1.649347in}}{\pgfqpoint{3.969999in}{1.638296in}}%
\pgfpathcurveto{\pgfqpoint{3.969999in}{1.627246in}}{\pgfqpoint{3.974389in}{1.616647in}}{\pgfqpoint{3.982203in}{1.608834in}}%
\pgfpathcurveto{\pgfqpoint{3.990016in}{1.601020in}}{\pgfqpoint{4.000616in}{1.596630in}}{\pgfqpoint{4.011666in}{1.596630in}}%
\pgfpathclose%
\pgfusepath{stroke,fill}%
\end{pgfscope}%
\begin{pgfscope}%
\pgfpathrectangle{\pgfqpoint{0.800000in}{0.528000in}}{\pgfqpoint{4.960000in}{3.696000in}}%
\pgfusepath{clip}%
\pgfsetbuttcap%
\pgfsetroundjoin%
\definecolor{currentfill}{rgb}{0.000000,0.000000,0.000000}%
\pgfsetfillcolor{currentfill}%
\pgfsetlinewidth{1.003750pt}%
\definecolor{currentstroke}{rgb}{0.000000,0.000000,0.000000}%
\pgfsetstrokecolor{currentstroke}%
\pgfsetdash{}{0pt}%
\pgfpathmoveto{\pgfqpoint{4.011666in}{1.984989in}}%
\pgfpathcurveto{\pgfqpoint{4.022716in}{1.984989in}}{\pgfqpoint{4.033315in}{1.989379in}}{\pgfqpoint{4.041128in}{1.997192in}}%
\pgfpathcurveto{\pgfqpoint{4.048942in}{2.005006in}}{\pgfqpoint{4.053332in}{2.015605in}}{\pgfqpoint{4.053332in}{2.026655in}}%
\pgfpathcurveto{\pgfqpoint{4.053332in}{2.037705in}}{\pgfqpoint{4.048942in}{2.048304in}}{\pgfqpoint{4.041128in}{2.056118in}}%
\pgfpathcurveto{\pgfqpoint{4.033315in}{2.063932in}}{\pgfqpoint{4.022716in}{2.068322in}}{\pgfqpoint{4.011666in}{2.068322in}}%
\pgfpathcurveto{\pgfqpoint{4.000616in}{2.068322in}}{\pgfqpoint{3.990016in}{2.063932in}}{\pgfqpoint{3.982203in}{2.056118in}}%
\pgfpathcurveto{\pgfqpoint{3.974389in}{2.048304in}}{\pgfqpoint{3.969999in}{2.037705in}}{\pgfqpoint{3.969999in}{2.026655in}}%
\pgfpathcurveto{\pgfqpoint{3.969999in}{2.015605in}}{\pgfqpoint{3.974389in}{2.005006in}}{\pgfqpoint{3.982203in}{1.997192in}}%
\pgfpathcurveto{\pgfqpoint{3.990016in}{1.989379in}}{\pgfqpoint{4.000616in}{1.984989in}}{\pgfqpoint{4.011666in}{1.984989in}}%
\pgfpathclose%
\pgfusepath{stroke,fill}%
\end{pgfscope}%
\begin{pgfscope}%
\pgfpathrectangle{\pgfqpoint{0.800000in}{0.528000in}}{\pgfqpoint{4.960000in}{3.696000in}}%
\pgfusepath{clip}%
\pgfsetbuttcap%
\pgfsetroundjoin%
\definecolor{currentfill}{rgb}{0.000000,0.000000,0.000000}%
\pgfsetfillcolor{currentfill}%
\pgfsetlinewidth{1.003750pt}%
\definecolor{currentstroke}{rgb}{0.000000,0.000000,0.000000}%
\pgfsetstrokecolor{currentstroke}%
\pgfsetdash{}{0pt}%
\pgfpathmoveto{\pgfqpoint{4.011666in}{1.761682in}}%
\pgfpathcurveto{\pgfqpoint{4.022716in}{1.761682in}}{\pgfqpoint{4.033315in}{1.766072in}}{\pgfqpoint{4.041128in}{1.773886in}}%
\pgfpathcurveto{\pgfqpoint{4.048942in}{1.781700in}}{\pgfqpoint{4.053332in}{1.792299in}}{\pgfqpoint{4.053332in}{1.803349in}}%
\pgfpathcurveto{\pgfqpoint{4.053332in}{1.814399in}}{\pgfqpoint{4.048942in}{1.824998in}}{\pgfqpoint{4.041128in}{1.832812in}}%
\pgfpathcurveto{\pgfqpoint{4.033315in}{1.840625in}}{\pgfqpoint{4.022716in}{1.845016in}}{\pgfqpoint{4.011666in}{1.845016in}}%
\pgfpathcurveto{\pgfqpoint{4.000616in}{1.845016in}}{\pgfqpoint{3.990016in}{1.840625in}}{\pgfqpoint{3.982203in}{1.832812in}}%
\pgfpathcurveto{\pgfqpoint{3.974389in}{1.824998in}}{\pgfqpoint{3.969999in}{1.814399in}}{\pgfqpoint{3.969999in}{1.803349in}}%
\pgfpathcurveto{\pgfqpoint{3.969999in}{1.792299in}}{\pgfqpoint{3.974389in}{1.781700in}}{\pgfqpoint{3.982203in}{1.773886in}}%
\pgfpathcurveto{\pgfqpoint{3.990016in}{1.766072in}}{\pgfqpoint{4.000616in}{1.761682in}}{\pgfqpoint{4.011666in}{1.761682in}}%
\pgfpathclose%
\pgfusepath{stroke,fill}%
\end{pgfscope}%
\begin{pgfscope}%
\pgfpathrectangle{\pgfqpoint{0.800000in}{0.528000in}}{\pgfqpoint{4.960000in}{3.696000in}}%
\pgfusepath{clip}%
\pgfsetbuttcap%
\pgfsetroundjoin%
\definecolor{currentfill}{rgb}{0.000000,0.000000,0.000000}%
\pgfsetfillcolor{currentfill}%
\pgfsetlinewidth{1.003750pt}%
\definecolor{currentstroke}{rgb}{0.000000,0.000000,0.000000}%
\pgfsetstrokecolor{currentstroke}%
\pgfsetdash{}{0pt}%
\pgfpathmoveto{\pgfqpoint{4.011666in}{1.518958in}}%
\pgfpathcurveto{\pgfqpoint{4.022716in}{1.518958in}}{\pgfqpoint{4.033315in}{1.523348in}}{\pgfqpoint{4.041128in}{1.531162in}}%
\pgfpathcurveto{\pgfqpoint{4.048942in}{1.538975in}}{\pgfqpoint{4.053332in}{1.549575in}}{\pgfqpoint{4.053332in}{1.560625in}}%
\pgfpathcurveto{\pgfqpoint{4.053332in}{1.571675in}}{\pgfqpoint{4.048942in}{1.582274in}}{\pgfqpoint{4.041128in}{1.590087in}}%
\pgfpathcurveto{\pgfqpoint{4.033315in}{1.597901in}}{\pgfqpoint{4.022716in}{1.602291in}}{\pgfqpoint{4.011666in}{1.602291in}}%
\pgfpathcurveto{\pgfqpoint{4.000616in}{1.602291in}}{\pgfqpoint{3.990016in}{1.597901in}}{\pgfqpoint{3.982203in}{1.590087in}}%
\pgfpathcurveto{\pgfqpoint{3.974389in}{1.582274in}}{\pgfqpoint{3.969999in}{1.571675in}}{\pgfqpoint{3.969999in}{1.560625in}}%
\pgfpathcurveto{\pgfqpoint{3.969999in}{1.549575in}}{\pgfqpoint{3.974389in}{1.538975in}}{\pgfqpoint{3.982203in}{1.531162in}}%
\pgfpathcurveto{\pgfqpoint{3.990016in}{1.523348in}}{\pgfqpoint{4.000616in}{1.518958in}}{\pgfqpoint{4.011666in}{1.518958in}}%
\pgfpathclose%
\pgfusepath{stroke,fill}%
\end{pgfscope}%
\begin{pgfscope}%
\pgfpathrectangle{\pgfqpoint{0.800000in}{0.528000in}}{\pgfqpoint{4.960000in}{3.696000in}}%
\pgfusepath{clip}%
\pgfsetbuttcap%
\pgfsetroundjoin%
\definecolor{currentfill}{rgb}{0.000000,0.000000,0.000000}%
\pgfsetfillcolor{currentfill}%
\pgfsetlinewidth{1.003750pt}%
\definecolor{currentstroke}{rgb}{0.000000,0.000000,0.000000}%
\pgfsetstrokecolor{currentstroke}%
\pgfsetdash{}{0pt}%
\pgfpathmoveto{\pgfqpoint{4.011666in}{1.878190in}}%
\pgfpathcurveto{\pgfqpoint{4.022716in}{1.878190in}}{\pgfqpoint{4.033315in}{1.882580in}}{\pgfqpoint{4.041128in}{1.890394in}}%
\pgfpathcurveto{\pgfqpoint{4.048942in}{1.898207in}}{\pgfqpoint{4.053332in}{1.908806in}}{\pgfqpoint{4.053332in}{1.919857in}}%
\pgfpathcurveto{\pgfqpoint{4.053332in}{1.930907in}}{\pgfqpoint{4.048942in}{1.941506in}}{\pgfqpoint{4.041128in}{1.949319in}}%
\pgfpathcurveto{\pgfqpoint{4.033315in}{1.957133in}}{\pgfqpoint{4.022716in}{1.961523in}}{\pgfqpoint{4.011666in}{1.961523in}}%
\pgfpathcurveto{\pgfqpoint{4.000616in}{1.961523in}}{\pgfqpoint{3.990016in}{1.957133in}}{\pgfqpoint{3.982203in}{1.949319in}}%
\pgfpathcurveto{\pgfqpoint{3.974389in}{1.941506in}}{\pgfqpoint{3.969999in}{1.930907in}}{\pgfqpoint{3.969999in}{1.919857in}}%
\pgfpathcurveto{\pgfqpoint{3.969999in}{1.908806in}}{\pgfqpoint{3.974389in}{1.898207in}}{\pgfqpoint{3.982203in}{1.890394in}}%
\pgfpathcurveto{\pgfqpoint{3.990016in}{1.882580in}}{\pgfqpoint{4.000616in}{1.878190in}}{\pgfqpoint{4.011666in}{1.878190in}}%
\pgfpathclose%
\pgfusepath{stroke,fill}%
\end{pgfscope}%
\begin{pgfscope}%
\pgfpathrectangle{\pgfqpoint{0.800000in}{0.528000in}}{\pgfqpoint{4.960000in}{3.696000in}}%
\pgfusepath{clip}%
\pgfsetbuttcap%
\pgfsetroundjoin%
\definecolor{currentfill}{rgb}{0.000000,0.000000,0.000000}%
\pgfsetfillcolor{currentfill}%
\pgfsetlinewidth{1.003750pt}%
\definecolor{currentstroke}{rgb}{0.000000,0.000000,0.000000}%
\pgfsetstrokecolor{currentstroke}%
\pgfsetdash{}{0pt}%
\pgfpathmoveto{\pgfqpoint{4.011666in}{1.664593in}}%
\pgfpathcurveto{\pgfqpoint{4.022716in}{1.664593in}}{\pgfqpoint{4.033315in}{1.668983in}}{\pgfqpoint{4.041128in}{1.676796in}}%
\pgfpathcurveto{\pgfqpoint{4.048942in}{1.684610in}}{\pgfqpoint{4.053332in}{1.695209in}}{\pgfqpoint{4.053332in}{1.706259in}}%
\pgfpathcurveto{\pgfqpoint{4.053332in}{1.717309in}}{\pgfqpoint{4.048942in}{1.727908in}}{\pgfqpoint{4.041128in}{1.735722in}}%
\pgfpathcurveto{\pgfqpoint{4.033315in}{1.743536in}}{\pgfqpoint{4.022716in}{1.747926in}}{\pgfqpoint{4.011666in}{1.747926in}}%
\pgfpathcurveto{\pgfqpoint{4.000616in}{1.747926in}}{\pgfqpoint{3.990016in}{1.743536in}}{\pgfqpoint{3.982203in}{1.735722in}}%
\pgfpathcurveto{\pgfqpoint{3.974389in}{1.727908in}}{\pgfqpoint{3.969999in}{1.717309in}}{\pgfqpoint{3.969999in}{1.706259in}}%
\pgfpathcurveto{\pgfqpoint{3.969999in}{1.695209in}}{\pgfqpoint{3.974389in}{1.684610in}}{\pgfqpoint{3.982203in}{1.676796in}}%
\pgfpathcurveto{\pgfqpoint{3.990016in}{1.668983in}}{\pgfqpoint{4.000616in}{1.664593in}}{\pgfqpoint{4.011666in}{1.664593in}}%
\pgfpathclose%
\pgfusepath{stroke,fill}%
\end{pgfscope}%
\begin{pgfscope}%
\pgfpathrectangle{\pgfqpoint{0.800000in}{0.528000in}}{\pgfqpoint{4.960000in}{3.696000in}}%
\pgfusepath{clip}%
\pgfsetbuttcap%
\pgfsetroundjoin%
\definecolor{currentfill}{rgb}{0.000000,0.000000,0.000000}%
\pgfsetfillcolor{currentfill}%
\pgfsetlinewidth{1.003750pt}%
\definecolor{currentstroke}{rgb}{0.000000,0.000000,0.000000}%
\pgfsetstrokecolor{currentstroke}%
\pgfsetdash{}{0pt}%
\pgfpathmoveto{\pgfqpoint{4.011666in}{1.907317in}}%
\pgfpathcurveto{\pgfqpoint{4.022716in}{1.907317in}}{\pgfqpoint{4.033315in}{1.911707in}}{\pgfqpoint{4.041128in}{1.919521in}}%
\pgfpathcurveto{\pgfqpoint{4.048942in}{1.927334in}}{\pgfqpoint{4.053332in}{1.937933in}}{\pgfqpoint{4.053332in}{1.948983in}}%
\pgfpathcurveto{\pgfqpoint{4.053332in}{1.960034in}}{\pgfqpoint{4.048942in}{1.970633in}}{\pgfqpoint{4.041128in}{1.978446in}}%
\pgfpathcurveto{\pgfqpoint{4.033315in}{1.986260in}}{\pgfqpoint{4.022716in}{1.990650in}}{\pgfqpoint{4.011666in}{1.990650in}}%
\pgfpathcurveto{\pgfqpoint{4.000616in}{1.990650in}}{\pgfqpoint{3.990016in}{1.986260in}}{\pgfqpoint{3.982203in}{1.978446in}}%
\pgfpathcurveto{\pgfqpoint{3.974389in}{1.970633in}}{\pgfqpoint{3.969999in}{1.960034in}}{\pgfqpoint{3.969999in}{1.948983in}}%
\pgfpathcurveto{\pgfqpoint{3.969999in}{1.937933in}}{\pgfqpoint{3.974389in}{1.927334in}}{\pgfqpoint{3.982203in}{1.919521in}}%
\pgfpathcurveto{\pgfqpoint{3.990016in}{1.911707in}}{\pgfqpoint{4.000616in}{1.907317in}}{\pgfqpoint{4.011666in}{1.907317in}}%
\pgfpathclose%
\pgfusepath{stroke,fill}%
\end{pgfscope}%
\begin{pgfscope}%
\pgfpathrectangle{\pgfqpoint{0.800000in}{0.528000in}}{\pgfqpoint{4.960000in}{3.696000in}}%
\pgfusepath{clip}%
\pgfsetbuttcap%
\pgfsetroundjoin%
\definecolor{currentfill}{rgb}{0.000000,0.000000,0.000000}%
\pgfsetfillcolor{currentfill}%
\pgfsetlinewidth{1.003750pt}%
\definecolor{currentstroke}{rgb}{0.000000,0.000000,0.000000}%
\pgfsetstrokecolor{currentstroke}%
\pgfsetdash{}{0pt}%
\pgfpathmoveto{\pgfqpoint{4.011666in}{1.897608in}}%
\pgfpathcurveto{\pgfqpoint{4.022716in}{1.897608in}}{\pgfqpoint{4.033315in}{1.901998in}}{\pgfqpoint{4.041128in}{1.909812in}}%
\pgfpathcurveto{\pgfqpoint{4.048942in}{1.917625in}}{\pgfqpoint{4.053332in}{1.928224in}}{\pgfqpoint{4.053332in}{1.939274in}}%
\pgfpathcurveto{\pgfqpoint{4.053332in}{1.950325in}}{\pgfqpoint{4.048942in}{1.960924in}}{\pgfqpoint{4.041128in}{1.968737in}}%
\pgfpathcurveto{\pgfqpoint{4.033315in}{1.976551in}}{\pgfqpoint{4.022716in}{1.980941in}}{\pgfqpoint{4.011666in}{1.980941in}}%
\pgfpathcurveto{\pgfqpoint{4.000616in}{1.980941in}}{\pgfqpoint{3.990016in}{1.976551in}}{\pgfqpoint{3.982203in}{1.968737in}}%
\pgfpathcurveto{\pgfqpoint{3.974389in}{1.960924in}}{\pgfqpoint{3.969999in}{1.950325in}}{\pgfqpoint{3.969999in}{1.939274in}}%
\pgfpathcurveto{\pgfqpoint{3.969999in}{1.928224in}}{\pgfqpoint{3.974389in}{1.917625in}}{\pgfqpoint{3.982203in}{1.909812in}}%
\pgfpathcurveto{\pgfqpoint{3.990016in}{1.901998in}}{\pgfqpoint{4.000616in}{1.897608in}}{\pgfqpoint{4.011666in}{1.897608in}}%
\pgfpathclose%
\pgfusepath{stroke,fill}%
\end{pgfscope}%
\begin{pgfscope}%
\pgfpathrectangle{\pgfqpoint{0.800000in}{0.528000in}}{\pgfqpoint{4.960000in}{3.696000in}}%
\pgfusepath{clip}%
\pgfsetbuttcap%
\pgfsetroundjoin%
\definecolor{currentfill}{rgb}{0.000000,0.000000,0.000000}%
\pgfsetfillcolor{currentfill}%
\pgfsetlinewidth{1.003750pt}%
\definecolor{currentstroke}{rgb}{0.000000,0.000000,0.000000}%
\pgfsetstrokecolor{currentstroke}%
\pgfsetdash{}{0pt}%
\pgfpathmoveto{\pgfqpoint{4.011666in}{1.586921in}}%
\pgfpathcurveto{\pgfqpoint{4.022716in}{1.586921in}}{\pgfqpoint{4.033315in}{1.591311in}}{\pgfqpoint{4.041128in}{1.599125in}}%
\pgfpathcurveto{\pgfqpoint{4.048942in}{1.606938in}}{\pgfqpoint{4.053332in}{1.617537in}}{\pgfqpoint{4.053332in}{1.628587in}}%
\pgfpathcurveto{\pgfqpoint{4.053332in}{1.639638in}}{\pgfqpoint{4.048942in}{1.650237in}}{\pgfqpoint{4.041128in}{1.658050in}}%
\pgfpathcurveto{\pgfqpoint{4.033315in}{1.665864in}}{\pgfqpoint{4.022716in}{1.670254in}}{\pgfqpoint{4.011666in}{1.670254in}}%
\pgfpathcurveto{\pgfqpoint{4.000616in}{1.670254in}}{\pgfqpoint{3.990016in}{1.665864in}}{\pgfqpoint{3.982203in}{1.658050in}}%
\pgfpathcurveto{\pgfqpoint{3.974389in}{1.650237in}}{\pgfqpoint{3.969999in}{1.639638in}}{\pgfqpoint{3.969999in}{1.628587in}}%
\pgfpathcurveto{\pgfqpoint{3.969999in}{1.617537in}}{\pgfqpoint{3.974389in}{1.606938in}}{\pgfqpoint{3.982203in}{1.599125in}}%
\pgfpathcurveto{\pgfqpoint{3.990016in}{1.591311in}}{\pgfqpoint{4.000616in}{1.586921in}}{\pgfqpoint{4.011666in}{1.586921in}}%
\pgfpathclose%
\pgfusepath{stroke,fill}%
\end{pgfscope}%
\begin{pgfscope}%
\pgfpathrectangle{\pgfqpoint{0.800000in}{0.528000in}}{\pgfqpoint{4.960000in}{3.696000in}}%
\pgfusepath{clip}%
\pgfsetbuttcap%
\pgfsetroundjoin%
\definecolor{currentfill}{rgb}{0.000000,0.000000,0.000000}%
\pgfsetfillcolor{currentfill}%
\pgfsetlinewidth{1.003750pt}%
\definecolor{currentstroke}{rgb}{0.000000,0.000000,0.000000}%
\pgfsetstrokecolor{currentstroke}%
\pgfsetdash{}{0pt}%
\pgfpathmoveto{\pgfqpoint{4.011666in}{1.557794in}}%
\pgfpathcurveto{\pgfqpoint{4.022716in}{1.557794in}}{\pgfqpoint{4.033315in}{1.562184in}}{\pgfqpoint{4.041128in}{1.569998in}}%
\pgfpathcurveto{\pgfqpoint{4.048942in}{1.577811in}}{\pgfqpoint{4.053332in}{1.588410in}}{\pgfqpoint{4.053332in}{1.599461in}}%
\pgfpathcurveto{\pgfqpoint{4.053332in}{1.610511in}}{\pgfqpoint{4.048942in}{1.621110in}}{\pgfqpoint{4.041128in}{1.628923in}}%
\pgfpathcurveto{\pgfqpoint{4.033315in}{1.636737in}}{\pgfqpoint{4.022716in}{1.641127in}}{\pgfqpoint{4.011666in}{1.641127in}}%
\pgfpathcurveto{\pgfqpoint{4.000616in}{1.641127in}}{\pgfqpoint{3.990016in}{1.636737in}}{\pgfqpoint{3.982203in}{1.628923in}}%
\pgfpathcurveto{\pgfqpoint{3.974389in}{1.621110in}}{\pgfqpoint{3.969999in}{1.610511in}}{\pgfqpoint{3.969999in}{1.599461in}}%
\pgfpathcurveto{\pgfqpoint{3.969999in}{1.588410in}}{\pgfqpoint{3.974389in}{1.577811in}}{\pgfqpoint{3.982203in}{1.569998in}}%
\pgfpathcurveto{\pgfqpoint{3.990016in}{1.562184in}}{\pgfqpoint{4.000616in}{1.557794in}}{\pgfqpoint{4.011666in}{1.557794in}}%
\pgfpathclose%
\pgfusepath{stroke,fill}%
\end{pgfscope}%
\begin{pgfscope}%
\pgfpathrectangle{\pgfqpoint{0.800000in}{0.528000in}}{\pgfqpoint{4.960000in}{3.696000in}}%
\pgfusepath{clip}%
\pgfsetbuttcap%
\pgfsetroundjoin%
\definecolor{currentfill}{rgb}{0.000000,0.000000,0.000000}%
\pgfsetfillcolor{currentfill}%
\pgfsetlinewidth{1.003750pt}%
\definecolor{currentstroke}{rgb}{0.000000,0.000000,0.000000}%
\pgfsetstrokecolor{currentstroke}%
\pgfsetdash{}{0pt}%
\pgfpathmoveto{\pgfqpoint{4.011666in}{1.868481in}}%
\pgfpathcurveto{\pgfqpoint{4.022716in}{1.868481in}}{\pgfqpoint{4.033315in}{1.872871in}}{\pgfqpoint{4.041128in}{1.880685in}}%
\pgfpathcurveto{\pgfqpoint{4.048942in}{1.888498in}}{\pgfqpoint{4.053332in}{1.899097in}}{\pgfqpoint{4.053332in}{1.910148in}}%
\pgfpathcurveto{\pgfqpoint{4.053332in}{1.921198in}}{\pgfqpoint{4.048942in}{1.931797in}}{\pgfqpoint{4.041128in}{1.939610in}}%
\pgfpathcurveto{\pgfqpoint{4.033315in}{1.947424in}}{\pgfqpoint{4.022716in}{1.951814in}}{\pgfqpoint{4.011666in}{1.951814in}}%
\pgfpathcurveto{\pgfqpoint{4.000616in}{1.951814in}}{\pgfqpoint{3.990016in}{1.947424in}}{\pgfqpoint{3.982203in}{1.939610in}}%
\pgfpathcurveto{\pgfqpoint{3.974389in}{1.931797in}}{\pgfqpoint{3.969999in}{1.921198in}}{\pgfqpoint{3.969999in}{1.910148in}}%
\pgfpathcurveto{\pgfqpoint{3.969999in}{1.899097in}}{\pgfqpoint{3.974389in}{1.888498in}}{\pgfqpoint{3.982203in}{1.880685in}}%
\pgfpathcurveto{\pgfqpoint{3.990016in}{1.872871in}}{\pgfqpoint{4.000616in}{1.868481in}}{\pgfqpoint{4.011666in}{1.868481in}}%
\pgfpathclose%
\pgfusepath{stroke,fill}%
\end{pgfscope}%
\begin{pgfscope}%
\pgfpathrectangle{\pgfqpoint{0.800000in}{0.528000in}}{\pgfqpoint{4.960000in}{3.696000in}}%
\pgfusepath{clip}%
\pgfsetbuttcap%
\pgfsetroundjoin%
\definecolor{currentfill}{rgb}{0.000000,0.000000,0.000000}%
\pgfsetfillcolor{currentfill}%
\pgfsetlinewidth{1.003750pt}%
\definecolor{currentstroke}{rgb}{0.000000,0.000000,0.000000}%
\pgfsetstrokecolor{currentstroke}%
\pgfsetdash{}{0pt}%
\pgfpathmoveto{\pgfqpoint{4.011666in}{2.052951in}}%
\pgfpathcurveto{\pgfqpoint{4.022716in}{2.052951in}}{\pgfqpoint{4.033315in}{2.057342in}}{\pgfqpoint{4.041128in}{2.065155in}}%
\pgfpathcurveto{\pgfqpoint{4.048942in}{2.072969in}}{\pgfqpoint{4.053332in}{2.083568in}}{\pgfqpoint{4.053332in}{2.094618in}}%
\pgfpathcurveto{\pgfqpoint{4.053332in}{2.105668in}}{\pgfqpoint{4.048942in}{2.116267in}}{\pgfqpoint{4.041128in}{2.124081in}}%
\pgfpathcurveto{\pgfqpoint{4.033315in}{2.131894in}}{\pgfqpoint{4.022716in}{2.136285in}}{\pgfqpoint{4.011666in}{2.136285in}}%
\pgfpathcurveto{\pgfqpoint{4.000616in}{2.136285in}}{\pgfqpoint{3.990016in}{2.131894in}}{\pgfqpoint{3.982203in}{2.124081in}}%
\pgfpathcurveto{\pgfqpoint{3.974389in}{2.116267in}}{\pgfqpoint{3.969999in}{2.105668in}}{\pgfqpoint{3.969999in}{2.094618in}}%
\pgfpathcurveto{\pgfqpoint{3.969999in}{2.083568in}}{\pgfqpoint{3.974389in}{2.072969in}}{\pgfqpoint{3.982203in}{2.065155in}}%
\pgfpathcurveto{\pgfqpoint{3.990016in}{2.057342in}}{\pgfqpoint{4.000616in}{2.052951in}}{\pgfqpoint{4.011666in}{2.052951in}}%
\pgfpathclose%
\pgfusepath{stroke,fill}%
\end{pgfscope}%
\begin{pgfscope}%
\pgfpathrectangle{\pgfqpoint{0.800000in}{0.528000in}}{\pgfqpoint{4.960000in}{3.696000in}}%
\pgfusepath{clip}%
\pgfsetbuttcap%
\pgfsetroundjoin%
\definecolor{currentfill}{rgb}{0.000000,0.000000,0.000000}%
\pgfsetfillcolor{currentfill}%
\pgfsetlinewidth{1.003750pt}%
\definecolor{currentstroke}{rgb}{0.000000,0.000000,0.000000}%
\pgfsetstrokecolor{currentstroke}%
\pgfsetdash{}{0pt}%
\pgfpathmoveto{\pgfqpoint{4.011666in}{1.586921in}}%
\pgfpathcurveto{\pgfqpoint{4.022716in}{1.586921in}}{\pgfqpoint{4.033315in}{1.591311in}}{\pgfqpoint{4.041128in}{1.599125in}}%
\pgfpathcurveto{\pgfqpoint{4.048942in}{1.606938in}}{\pgfqpoint{4.053332in}{1.617537in}}{\pgfqpoint{4.053332in}{1.628587in}}%
\pgfpathcurveto{\pgfqpoint{4.053332in}{1.639638in}}{\pgfqpoint{4.048942in}{1.650237in}}{\pgfqpoint{4.041128in}{1.658050in}}%
\pgfpathcurveto{\pgfqpoint{4.033315in}{1.665864in}}{\pgfqpoint{4.022716in}{1.670254in}}{\pgfqpoint{4.011666in}{1.670254in}}%
\pgfpathcurveto{\pgfqpoint{4.000616in}{1.670254in}}{\pgfqpoint{3.990016in}{1.665864in}}{\pgfqpoint{3.982203in}{1.658050in}}%
\pgfpathcurveto{\pgfqpoint{3.974389in}{1.650237in}}{\pgfqpoint{3.969999in}{1.639638in}}{\pgfqpoint{3.969999in}{1.628587in}}%
\pgfpathcurveto{\pgfqpoint{3.969999in}{1.617537in}}{\pgfqpoint{3.974389in}{1.606938in}}{\pgfqpoint{3.982203in}{1.599125in}}%
\pgfpathcurveto{\pgfqpoint{3.990016in}{1.591311in}}{\pgfqpoint{4.000616in}{1.586921in}}{\pgfqpoint{4.011666in}{1.586921in}}%
\pgfpathclose%
\pgfusepath{stroke,fill}%
\end{pgfscope}%
\begin{pgfscope}%
\pgfpathrectangle{\pgfqpoint{0.800000in}{0.528000in}}{\pgfqpoint{4.960000in}{3.696000in}}%
\pgfusepath{clip}%
\pgfsetbuttcap%
\pgfsetroundjoin%
\definecolor{currentfill}{rgb}{0.000000,0.000000,0.000000}%
\pgfsetfillcolor{currentfill}%
\pgfsetlinewidth{1.003750pt}%
\definecolor{currentstroke}{rgb}{0.000000,0.000000,0.000000}%
\pgfsetstrokecolor{currentstroke}%
\pgfsetdash{}{0pt}%
\pgfpathmoveto{\pgfqpoint{4.011666in}{1.548085in}}%
\pgfpathcurveto{\pgfqpoint{4.022716in}{1.548085in}}{\pgfqpoint{4.033315in}{1.552475in}}{\pgfqpoint{4.041128in}{1.560289in}}%
\pgfpathcurveto{\pgfqpoint{4.048942in}{1.568102in}}{\pgfqpoint{4.053332in}{1.578701in}}{\pgfqpoint{4.053332in}{1.589752in}}%
\pgfpathcurveto{\pgfqpoint{4.053332in}{1.600802in}}{\pgfqpoint{4.048942in}{1.611401in}}{\pgfqpoint{4.041128in}{1.619214in}}%
\pgfpathcurveto{\pgfqpoint{4.033315in}{1.627028in}}{\pgfqpoint{4.022716in}{1.631418in}}{\pgfqpoint{4.011666in}{1.631418in}}%
\pgfpathcurveto{\pgfqpoint{4.000616in}{1.631418in}}{\pgfqpoint{3.990016in}{1.627028in}}{\pgfqpoint{3.982203in}{1.619214in}}%
\pgfpathcurveto{\pgfqpoint{3.974389in}{1.611401in}}{\pgfqpoint{3.969999in}{1.600802in}}{\pgfqpoint{3.969999in}{1.589752in}}%
\pgfpathcurveto{\pgfqpoint{3.969999in}{1.578701in}}{\pgfqpoint{3.974389in}{1.568102in}}{\pgfqpoint{3.982203in}{1.560289in}}%
\pgfpathcurveto{\pgfqpoint{3.990016in}{1.552475in}}{\pgfqpoint{4.000616in}{1.548085in}}{\pgfqpoint{4.011666in}{1.548085in}}%
\pgfpathclose%
\pgfusepath{stroke,fill}%
\end{pgfscope}%
\begin{pgfscope}%
\pgfpathrectangle{\pgfqpoint{0.800000in}{0.528000in}}{\pgfqpoint{4.960000in}{3.696000in}}%
\pgfusepath{clip}%
\pgfsetbuttcap%
\pgfsetroundjoin%
\definecolor{currentfill}{rgb}{0.000000,0.000000,0.000000}%
\pgfsetfillcolor{currentfill}%
\pgfsetlinewidth{1.003750pt}%
\definecolor{currentstroke}{rgb}{0.000000,0.000000,0.000000}%
\pgfsetstrokecolor{currentstroke}%
\pgfsetdash{}{0pt}%
\pgfpathmoveto{\pgfqpoint{4.011666in}{2.179168in}}%
\pgfpathcurveto{\pgfqpoint{4.022716in}{2.179168in}}{\pgfqpoint{4.033315in}{2.183558in}}{\pgfqpoint{4.041128in}{2.191372in}}%
\pgfpathcurveto{\pgfqpoint{4.048942in}{2.199185in}}{\pgfqpoint{4.053332in}{2.209784in}}{\pgfqpoint{4.053332in}{2.220835in}}%
\pgfpathcurveto{\pgfqpoint{4.053332in}{2.231885in}}{\pgfqpoint{4.048942in}{2.242484in}}{\pgfqpoint{4.041128in}{2.250297in}}%
\pgfpathcurveto{\pgfqpoint{4.033315in}{2.258111in}}{\pgfqpoint{4.022716in}{2.262501in}}{\pgfqpoint{4.011666in}{2.262501in}}%
\pgfpathcurveto{\pgfqpoint{4.000616in}{2.262501in}}{\pgfqpoint{3.990016in}{2.258111in}}{\pgfqpoint{3.982203in}{2.250297in}}%
\pgfpathcurveto{\pgfqpoint{3.974389in}{2.242484in}}{\pgfqpoint{3.969999in}{2.231885in}}{\pgfqpoint{3.969999in}{2.220835in}}%
\pgfpathcurveto{\pgfqpoint{3.969999in}{2.209784in}}{\pgfqpoint{3.974389in}{2.199185in}}{\pgfqpoint{3.982203in}{2.191372in}}%
\pgfpathcurveto{\pgfqpoint{3.990016in}{2.183558in}}{\pgfqpoint{4.000616in}{2.179168in}}{\pgfqpoint{4.011666in}{2.179168in}}%
\pgfpathclose%
\pgfusepath{stroke,fill}%
\end{pgfscope}%
\begin{pgfscope}%
\pgfpathrectangle{\pgfqpoint{0.800000in}{0.528000in}}{\pgfqpoint{4.960000in}{3.696000in}}%
\pgfusepath{clip}%
\pgfsetbuttcap%
\pgfsetroundjoin%
\definecolor{currentfill}{rgb}{0.000000,0.000000,0.000000}%
\pgfsetfillcolor{currentfill}%
\pgfsetlinewidth{1.003750pt}%
\definecolor{currentstroke}{rgb}{0.000000,0.000000,0.000000}%
\pgfsetstrokecolor{currentstroke}%
\pgfsetdash{}{0pt}%
\pgfpathmoveto{\pgfqpoint{4.011666in}{1.616048in}}%
\pgfpathcurveto{\pgfqpoint{4.022716in}{1.616048in}}{\pgfqpoint{4.033315in}{1.620438in}}{\pgfqpoint{4.041128in}{1.628252in}}%
\pgfpathcurveto{\pgfqpoint{4.048942in}{1.636065in}}{\pgfqpoint{4.053332in}{1.646664in}}{\pgfqpoint{4.053332in}{1.657714in}}%
\pgfpathcurveto{\pgfqpoint{4.053332in}{1.668764in}}{\pgfqpoint{4.048942in}{1.679364in}}{\pgfqpoint{4.041128in}{1.687177in}}%
\pgfpathcurveto{\pgfqpoint{4.033315in}{1.694991in}}{\pgfqpoint{4.022716in}{1.699381in}}{\pgfqpoint{4.011666in}{1.699381in}}%
\pgfpathcurveto{\pgfqpoint{4.000616in}{1.699381in}}{\pgfqpoint{3.990016in}{1.694991in}}{\pgfqpoint{3.982203in}{1.687177in}}%
\pgfpathcurveto{\pgfqpoint{3.974389in}{1.679364in}}{\pgfqpoint{3.969999in}{1.668764in}}{\pgfqpoint{3.969999in}{1.657714in}}%
\pgfpathcurveto{\pgfqpoint{3.969999in}{1.646664in}}{\pgfqpoint{3.974389in}{1.636065in}}{\pgfqpoint{3.982203in}{1.628252in}}%
\pgfpathcurveto{\pgfqpoint{3.990016in}{1.620438in}}{\pgfqpoint{4.000616in}{1.616048in}}{\pgfqpoint{4.011666in}{1.616048in}}%
\pgfpathclose%
\pgfusepath{stroke,fill}%
\end{pgfscope}%
\begin{pgfscope}%
\pgfpathrectangle{\pgfqpoint{0.800000in}{0.528000in}}{\pgfqpoint{4.960000in}{3.696000in}}%
\pgfusepath{clip}%
\pgfsetbuttcap%
\pgfsetroundjoin%
\definecolor{currentfill}{rgb}{0.000000,0.000000,0.000000}%
\pgfsetfillcolor{currentfill}%
\pgfsetlinewidth{1.003750pt}%
\definecolor{currentstroke}{rgb}{0.000000,0.000000,0.000000}%
\pgfsetstrokecolor{currentstroke}%
\pgfsetdash{}{0pt}%
\pgfpathmoveto{\pgfqpoint{4.011666in}{1.781100in}}%
\pgfpathcurveto{\pgfqpoint{4.022716in}{1.781100in}}{\pgfqpoint{4.033315in}{1.785490in}}{\pgfqpoint{4.041128in}{1.793304in}}%
\pgfpathcurveto{\pgfqpoint{4.048942in}{1.801118in}}{\pgfqpoint{4.053332in}{1.811717in}}{\pgfqpoint{4.053332in}{1.822767in}}%
\pgfpathcurveto{\pgfqpoint{4.053332in}{1.833817in}}{\pgfqpoint{4.048942in}{1.844416in}}{\pgfqpoint{4.041128in}{1.852230in}}%
\pgfpathcurveto{\pgfqpoint{4.033315in}{1.860043in}}{\pgfqpoint{4.022716in}{1.864433in}}{\pgfqpoint{4.011666in}{1.864433in}}%
\pgfpathcurveto{\pgfqpoint{4.000616in}{1.864433in}}{\pgfqpoint{3.990016in}{1.860043in}}{\pgfqpoint{3.982203in}{1.852230in}}%
\pgfpathcurveto{\pgfqpoint{3.974389in}{1.844416in}}{\pgfqpoint{3.969999in}{1.833817in}}{\pgfqpoint{3.969999in}{1.822767in}}%
\pgfpathcurveto{\pgfqpoint{3.969999in}{1.811717in}}{\pgfqpoint{3.974389in}{1.801118in}}{\pgfqpoint{3.982203in}{1.793304in}}%
\pgfpathcurveto{\pgfqpoint{3.990016in}{1.785490in}}{\pgfqpoint{4.000616in}{1.781100in}}{\pgfqpoint{4.011666in}{1.781100in}}%
\pgfpathclose%
\pgfusepath{stroke,fill}%
\end{pgfscope}%
\begin{pgfscope}%
\pgfpathrectangle{\pgfqpoint{0.800000in}{0.528000in}}{\pgfqpoint{4.960000in}{3.696000in}}%
\pgfusepath{clip}%
\pgfsetbuttcap%
\pgfsetroundjoin%
\definecolor{currentfill}{rgb}{0.000000,0.000000,0.000000}%
\pgfsetfillcolor{currentfill}%
\pgfsetlinewidth{1.003750pt}%
\definecolor{currentstroke}{rgb}{0.000000,0.000000,0.000000}%
\pgfsetstrokecolor{currentstroke}%
\pgfsetdash{}{0pt}%
\pgfpathmoveto{\pgfqpoint{4.011666in}{1.645175in}}%
\pgfpathcurveto{\pgfqpoint{4.022716in}{1.645175in}}{\pgfqpoint{4.033315in}{1.649565in}}{\pgfqpoint{4.041128in}{1.657378in}}%
\pgfpathcurveto{\pgfqpoint{4.048942in}{1.665192in}}{\pgfqpoint{4.053332in}{1.675791in}}{\pgfqpoint{4.053332in}{1.686841in}}%
\pgfpathcurveto{\pgfqpoint{4.053332in}{1.697891in}}{\pgfqpoint{4.048942in}{1.708490in}}{\pgfqpoint{4.041128in}{1.716304in}}%
\pgfpathcurveto{\pgfqpoint{4.033315in}{1.724118in}}{\pgfqpoint{4.022716in}{1.728508in}}{\pgfqpoint{4.011666in}{1.728508in}}%
\pgfpathcurveto{\pgfqpoint{4.000616in}{1.728508in}}{\pgfqpoint{3.990016in}{1.724118in}}{\pgfqpoint{3.982203in}{1.716304in}}%
\pgfpathcurveto{\pgfqpoint{3.974389in}{1.708490in}}{\pgfqpoint{3.969999in}{1.697891in}}{\pgfqpoint{3.969999in}{1.686841in}}%
\pgfpathcurveto{\pgfqpoint{3.969999in}{1.675791in}}{\pgfqpoint{3.974389in}{1.665192in}}{\pgfqpoint{3.982203in}{1.657378in}}%
\pgfpathcurveto{\pgfqpoint{3.990016in}{1.649565in}}{\pgfqpoint{4.000616in}{1.645175in}}{\pgfqpoint{4.011666in}{1.645175in}}%
\pgfpathclose%
\pgfusepath{stroke,fill}%
\end{pgfscope}%
\begin{pgfscope}%
\pgfpathrectangle{\pgfqpoint{0.800000in}{0.528000in}}{\pgfqpoint{4.960000in}{3.696000in}}%
\pgfusepath{clip}%
\pgfsetbuttcap%
\pgfsetroundjoin%
\definecolor{currentfill}{rgb}{0.000000,0.000000,0.000000}%
\pgfsetfillcolor{currentfill}%
\pgfsetlinewidth{1.003750pt}%
\definecolor{currentstroke}{rgb}{0.000000,0.000000,0.000000}%
\pgfsetstrokecolor{currentstroke}%
\pgfsetdash{}{0pt}%
\pgfpathmoveto{\pgfqpoint{4.011666in}{1.742264in}}%
\pgfpathcurveto{\pgfqpoint{4.022716in}{1.742264in}}{\pgfqpoint{4.033315in}{1.746655in}}{\pgfqpoint{4.041128in}{1.754468in}}%
\pgfpathcurveto{\pgfqpoint{4.048942in}{1.762282in}}{\pgfqpoint{4.053332in}{1.772881in}}{\pgfqpoint{4.053332in}{1.783931in}}%
\pgfpathcurveto{\pgfqpoint{4.053332in}{1.794981in}}{\pgfqpoint{4.048942in}{1.805580in}}{\pgfqpoint{4.041128in}{1.813394in}}%
\pgfpathcurveto{\pgfqpoint{4.033315in}{1.821207in}}{\pgfqpoint{4.022716in}{1.825598in}}{\pgfqpoint{4.011666in}{1.825598in}}%
\pgfpathcurveto{\pgfqpoint{4.000616in}{1.825598in}}{\pgfqpoint{3.990016in}{1.821207in}}{\pgfqpoint{3.982203in}{1.813394in}}%
\pgfpathcurveto{\pgfqpoint{3.974389in}{1.805580in}}{\pgfqpoint{3.969999in}{1.794981in}}{\pgfqpoint{3.969999in}{1.783931in}}%
\pgfpathcurveto{\pgfqpoint{3.969999in}{1.772881in}}{\pgfqpoint{3.974389in}{1.762282in}}{\pgfqpoint{3.982203in}{1.754468in}}%
\pgfpathcurveto{\pgfqpoint{3.990016in}{1.746655in}}{\pgfqpoint{4.000616in}{1.742264in}}{\pgfqpoint{4.011666in}{1.742264in}}%
\pgfpathclose%
\pgfusepath{stroke,fill}%
\end{pgfscope}%
\begin{pgfscope}%
\pgfpathrectangle{\pgfqpoint{0.800000in}{0.528000in}}{\pgfqpoint{4.960000in}{3.696000in}}%
\pgfusepath{clip}%
\pgfsetbuttcap%
\pgfsetroundjoin%
\definecolor{currentfill}{rgb}{0.000000,0.000000,0.000000}%
\pgfsetfillcolor{currentfill}%
\pgfsetlinewidth{1.003750pt}%
\definecolor{currentstroke}{rgb}{0.000000,0.000000,0.000000}%
\pgfsetstrokecolor{currentstroke}%
\pgfsetdash{}{0pt}%
\pgfpathmoveto{\pgfqpoint{4.011666in}{1.742264in}}%
\pgfpathcurveto{\pgfqpoint{4.022716in}{1.742264in}}{\pgfqpoint{4.033315in}{1.746655in}}{\pgfqpoint{4.041128in}{1.754468in}}%
\pgfpathcurveto{\pgfqpoint{4.048942in}{1.762282in}}{\pgfqpoint{4.053332in}{1.772881in}}{\pgfqpoint{4.053332in}{1.783931in}}%
\pgfpathcurveto{\pgfqpoint{4.053332in}{1.794981in}}{\pgfqpoint{4.048942in}{1.805580in}}{\pgfqpoint{4.041128in}{1.813394in}}%
\pgfpathcurveto{\pgfqpoint{4.033315in}{1.821207in}}{\pgfqpoint{4.022716in}{1.825598in}}{\pgfqpoint{4.011666in}{1.825598in}}%
\pgfpathcurveto{\pgfqpoint{4.000616in}{1.825598in}}{\pgfqpoint{3.990016in}{1.821207in}}{\pgfqpoint{3.982203in}{1.813394in}}%
\pgfpathcurveto{\pgfqpoint{3.974389in}{1.805580in}}{\pgfqpoint{3.969999in}{1.794981in}}{\pgfqpoint{3.969999in}{1.783931in}}%
\pgfpathcurveto{\pgfqpoint{3.969999in}{1.772881in}}{\pgfqpoint{3.974389in}{1.762282in}}{\pgfqpoint{3.982203in}{1.754468in}}%
\pgfpathcurveto{\pgfqpoint{3.990016in}{1.746655in}}{\pgfqpoint{4.000616in}{1.742264in}}{\pgfqpoint{4.011666in}{1.742264in}}%
\pgfpathclose%
\pgfusepath{stroke,fill}%
\end{pgfscope}%
\begin{pgfscope}%
\pgfpathrectangle{\pgfqpoint{0.800000in}{0.528000in}}{\pgfqpoint{4.960000in}{3.696000in}}%
\pgfusepath{clip}%
\pgfsetbuttcap%
\pgfsetroundjoin%
\definecolor{currentfill}{rgb}{0.000000,0.000000,0.000000}%
\pgfsetfillcolor{currentfill}%
\pgfsetlinewidth{1.003750pt}%
\definecolor{currentstroke}{rgb}{0.000000,0.000000,0.000000}%
\pgfsetstrokecolor{currentstroke}%
\pgfsetdash{}{0pt}%
\pgfpathmoveto{\pgfqpoint{4.011666in}{1.635466in}}%
\pgfpathcurveto{\pgfqpoint{4.022716in}{1.635466in}}{\pgfqpoint{4.033315in}{1.639856in}}{\pgfqpoint{4.041128in}{1.647670in}}%
\pgfpathcurveto{\pgfqpoint{4.048942in}{1.655483in}}{\pgfqpoint{4.053332in}{1.666082in}}{\pgfqpoint{4.053332in}{1.677132in}}%
\pgfpathcurveto{\pgfqpoint{4.053332in}{1.688182in}}{\pgfqpoint{4.048942in}{1.698781in}}{\pgfqpoint{4.041128in}{1.706595in}}%
\pgfpathcurveto{\pgfqpoint{4.033315in}{1.714409in}}{\pgfqpoint{4.022716in}{1.718799in}}{\pgfqpoint{4.011666in}{1.718799in}}%
\pgfpathcurveto{\pgfqpoint{4.000616in}{1.718799in}}{\pgfqpoint{3.990016in}{1.714409in}}{\pgfqpoint{3.982203in}{1.706595in}}%
\pgfpathcurveto{\pgfqpoint{3.974389in}{1.698781in}}{\pgfqpoint{3.969999in}{1.688182in}}{\pgfqpoint{3.969999in}{1.677132in}}%
\pgfpathcurveto{\pgfqpoint{3.969999in}{1.666082in}}{\pgfqpoint{3.974389in}{1.655483in}}{\pgfqpoint{3.982203in}{1.647670in}}%
\pgfpathcurveto{\pgfqpoint{3.990016in}{1.639856in}}{\pgfqpoint{4.000616in}{1.635466in}}{\pgfqpoint{4.011666in}{1.635466in}}%
\pgfpathclose%
\pgfusepath{stroke,fill}%
\end{pgfscope}%
\begin{pgfscope}%
\pgfpathrectangle{\pgfqpoint{0.800000in}{0.528000in}}{\pgfqpoint{4.960000in}{3.696000in}}%
\pgfusepath{clip}%
\pgfsetbuttcap%
\pgfsetroundjoin%
\definecolor{currentfill}{rgb}{0.000000,0.000000,0.000000}%
\pgfsetfillcolor{currentfill}%
\pgfsetlinewidth{1.003750pt}%
\definecolor{currentstroke}{rgb}{0.000000,0.000000,0.000000}%
\pgfsetstrokecolor{currentstroke}%
\pgfsetdash{}{0pt}%
\pgfpathmoveto{\pgfqpoint{4.011666in}{1.645175in}}%
\pgfpathcurveto{\pgfqpoint{4.022716in}{1.645175in}}{\pgfqpoint{4.033315in}{1.649565in}}{\pgfqpoint{4.041128in}{1.657378in}}%
\pgfpathcurveto{\pgfqpoint{4.048942in}{1.665192in}}{\pgfqpoint{4.053332in}{1.675791in}}{\pgfqpoint{4.053332in}{1.686841in}}%
\pgfpathcurveto{\pgfqpoint{4.053332in}{1.697891in}}{\pgfqpoint{4.048942in}{1.708490in}}{\pgfqpoint{4.041128in}{1.716304in}}%
\pgfpathcurveto{\pgfqpoint{4.033315in}{1.724118in}}{\pgfqpoint{4.022716in}{1.728508in}}{\pgfqpoint{4.011666in}{1.728508in}}%
\pgfpathcurveto{\pgfqpoint{4.000616in}{1.728508in}}{\pgfqpoint{3.990016in}{1.724118in}}{\pgfqpoint{3.982203in}{1.716304in}}%
\pgfpathcurveto{\pgfqpoint{3.974389in}{1.708490in}}{\pgfqpoint{3.969999in}{1.697891in}}{\pgfqpoint{3.969999in}{1.686841in}}%
\pgfpathcurveto{\pgfqpoint{3.969999in}{1.675791in}}{\pgfqpoint{3.974389in}{1.665192in}}{\pgfqpoint{3.982203in}{1.657378in}}%
\pgfpathcurveto{\pgfqpoint{3.990016in}{1.649565in}}{\pgfqpoint{4.000616in}{1.645175in}}{\pgfqpoint{4.011666in}{1.645175in}}%
\pgfpathclose%
\pgfusepath{stroke,fill}%
\end{pgfscope}%
\begin{pgfscope}%
\pgfpathrectangle{\pgfqpoint{0.800000in}{0.528000in}}{\pgfqpoint{4.960000in}{3.696000in}}%
\pgfusepath{clip}%
\pgfsetbuttcap%
\pgfsetroundjoin%
\definecolor{currentfill}{rgb}{0.000000,0.000000,0.000000}%
\pgfsetfillcolor{currentfill}%
\pgfsetlinewidth{1.003750pt}%
\definecolor{currentstroke}{rgb}{0.000000,0.000000,0.000000}%
\pgfsetstrokecolor{currentstroke}%
\pgfsetdash{}{0pt}%
\pgfpathmoveto{\pgfqpoint{4.011666in}{1.713137in}}%
\pgfpathcurveto{\pgfqpoint{4.022716in}{1.713137in}}{\pgfqpoint{4.033315in}{1.717528in}}{\pgfqpoint{4.041128in}{1.725341in}}%
\pgfpathcurveto{\pgfqpoint{4.048942in}{1.733155in}}{\pgfqpoint{4.053332in}{1.743754in}}{\pgfqpoint{4.053332in}{1.754804in}}%
\pgfpathcurveto{\pgfqpoint{4.053332in}{1.765854in}}{\pgfqpoint{4.048942in}{1.776453in}}{\pgfqpoint{4.041128in}{1.784267in}}%
\pgfpathcurveto{\pgfqpoint{4.033315in}{1.792080in}}{\pgfqpoint{4.022716in}{1.796471in}}{\pgfqpoint{4.011666in}{1.796471in}}%
\pgfpathcurveto{\pgfqpoint{4.000616in}{1.796471in}}{\pgfqpoint{3.990016in}{1.792080in}}{\pgfqpoint{3.982203in}{1.784267in}}%
\pgfpathcurveto{\pgfqpoint{3.974389in}{1.776453in}}{\pgfqpoint{3.969999in}{1.765854in}}{\pgfqpoint{3.969999in}{1.754804in}}%
\pgfpathcurveto{\pgfqpoint{3.969999in}{1.743754in}}{\pgfqpoint{3.974389in}{1.733155in}}{\pgfqpoint{3.982203in}{1.725341in}}%
\pgfpathcurveto{\pgfqpoint{3.990016in}{1.717528in}}{\pgfqpoint{4.000616in}{1.713137in}}{\pgfqpoint{4.011666in}{1.713137in}}%
\pgfpathclose%
\pgfusepath{stroke,fill}%
\end{pgfscope}%
\begin{pgfscope}%
\pgfpathrectangle{\pgfqpoint{0.800000in}{0.528000in}}{\pgfqpoint{4.960000in}{3.696000in}}%
\pgfusepath{clip}%
\pgfsetbuttcap%
\pgfsetroundjoin%
\definecolor{currentfill}{rgb}{0.000000,0.000000,0.000000}%
\pgfsetfillcolor{currentfill}%
\pgfsetlinewidth{1.003750pt}%
\definecolor{currentstroke}{rgb}{0.000000,0.000000,0.000000}%
\pgfsetstrokecolor{currentstroke}%
\pgfsetdash{}{0pt}%
\pgfpathmoveto{\pgfqpoint{4.011666in}{1.800518in}}%
\pgfpathcurveto{\pgfqpoint{4.022716in}{1.800518in}}{\pgfqpoint{4.033315in}{1.804908in}}{\pgfqpoint{4.041128in}{1.812722in}}%
\pgfpathcurveto{\pgfqpoint{4.048942in}{1.820536in}}{\pgfqpoint{4.053332in}{1.831135in}}{\pgfqpoint{4.053332in}{1.842185in}}%
\pgfpathcurveto{\pgfqpoint{4.053332in}{1.853235in}}{\pgfqpoint{4.048942in}{1.863834in}}{\pgfqpoint{4.041128in}{1.871648in}}%
\pgfpathcurveto{\pgfqpoint{4.033315in}{1.879461in}}{\pgfqpoint{4.022716in}{1.883851in}}{\pgfqpoint{4.011666in}{1.883851in}}%
\pgfpathcurveto{\pgfqpoint{4.000616in}{1.883851in}}{\pgfqpoint{3.990016in}{1.879461in}}{\pgfqpoint{3.982203in}{1.871648in}}%
\pgfpathcurveto{\pgfqpoint{3.974389in}{1.863834in}}{\pgfqpoint{3.969999in}{1.853235in}}{\pgfqpoint{3.969999in}{1.842185in}}%
\pgfpathcurveto{\pgfqpoint{3.969999in}{1.831135in}}{\pgfqpoint{3.974389in}{1.820536in}}{\pgfqpoint{3.982203in}{1.812722in}}%
\pgfpathcurveto{\pgfqpoint{3.990016in}{1.804908in}}{\pgfqpoint{4.000616in}{1.800518in}}{\pgfqpoint{4.011666in}{1.800518in}}%
\pgfpathclose%
\pgfusepath{stroke,fill}%
\end{pgfscope}%
\begin{pgfscope}%
\pgfpathrectangle{\pgfqpoint{0.800000in}{0.528000in}}{\pgfqpoint{4.960000in}{3.696000in}}%
\pgfusepath{clip}%
\pgfsetbuttcap%
\pgfsetroundjoin%
\definecolor{currentfill}{rgb}{0.000000,0.000000,0.000000}%
\pgfsetfillcolor{currentfill}%
\pgfsetlinewidth{1.003750pt}%
\definecolor{currentstroke}{rgb}{0.000000,0.000000,0.000000}%
\pgfsetstrokecolor{currentstroke}%
\pgfsetdash{}{0pt}%
\pgfpathmoveto{\pgfqpoint{4.011666in}{1.693719in}}%
\pgfpathcurveto{\pgfqpoint{4.022716in}{1.693719in}}{\pgfqpoint{4.033315in}{1.698110in}}{\pgfqpoint{4.041128in}{1.705923in}}%
\pgfpathcurveto{\pgfqpoint{4.048942in}{1.713737in}}{\pgfqpoint{4.053332in}{1.724336in}}{\pgfqpoint{4.053332in}{1.735386in}}%
\pgfpathcurveto{\pgfqpoint{4.053332in}{1.746436in}}{\pgfqpoint{4.048942in}{1.757035in}}{\pgfqpoint{4.041128in}{1.764849in}}%
\pgfpathcurveto{\pgfqpoint{4.033315in}{1.772663in}}{\pgfqpoint{4.022716in}{1.777053in}}{\pgfqpoint{4.011666in}{1.777053in}}%
\pgfpathcurveto{\pgfqpoint{4.000616in}{1.777053in}}{\pgfqpoint{3.990016in}{1.772663in}}{\pgfqpoint{3.982203in}{1.764849in}}%
\pgfpathcurveto{\pgfqpoint{3.974389in}{1.757035in}}{\pgfqpoint{3.969999in}{1.746436in}}{\pgfqpoint{3.969999in}{1.735386in}}%
\pgfpathcurveto{\pgfqpoint{3.969999in}{1.724336in}}{\pgfqpoint{3.974389in}{1.713737in}}{\pgfqpoint{3.982203in}{1.705923in}}%
\pgfpathcurveto{\pgfqpoint{3.990016in}{1.698110in}}{\pgfqpoint{4.000616in}{1.693719in}}{\pgfqpoint{4.011666in}{1.693719in}}%
\pgfpathclose%
\pgfusepath{stroke,fill}%
\end{pgfscope}%
\begin{pgfscope}%
\pgfpathrectangle{\pgfqpoint{0.800000in}{0.528000in}}{\pgfqpoint{4.960000in}{3.696000in}}%
\pgfusepath{clip}%
\pgfsetbuttcap%
\pgfsetroundjoin%
\definecolor{currentfill}{rgb}{0.000000,0.000000,0.000000}%
\pgfsetfillcolor{currentfill}%
\pgfsetlinewidth{1.003750pt}%
\definecolor{currentstroke}{rgb}{0.000000,0.000000,0.000000}%
\pgfsetstrokecolor{currentstroke}%
\pgfsetdash{}{0pt}%
\pgfpathmoveto{\pgfqpoint{4.011666in}{1.771391in}}%
\pgfpathcurveto{\pgfqpoint{4.022716in}{1.771391in}}{\pgfqpoint{4.033315in}{1.775781in}}{\pgfqpoint{4.041128in}{1.783595in}}%
\pgfpathcurveto{\pgfqpoint{4.048942in}{1.791409in}}{\pgfqpoint{4.053332in}{1.802008in}}{\pgfqpoint{4.053332in}{1.813058in}}%
\pgfpathcurveto{\pgfqpoint{4.053332in}{1.824108in}}{\pgfqpoint{4.048942in}{1.834707in}}{\pgfqpoint{4.041128in}{1.842521in}}%
\pgfpathcurveto{\pgfqpoint{4.033315in}{1.850334in}}{\pgfqpoint{4.022716in}{1.854725in}}{\pgfqpoint{4.011666in}{1.854725in}}%
\pgfpathcurveto{\pgfqpoint{4.000616in}{1.854725in}}{\pgfqpoint{3.990016in}{1.850334in}}{\pgfqpoint{3.982203in}{1.842521in}}%
\pgfpathcurveto{\pgfqpoint{3.974389in}{1.834707in}}{\pgfqpoint{3.969999in}{1.824108in}}{\pgfqpoint{3.969999in}{1.813058in}}%
\pgfpathcurveto{\pgfqpoint{3.969999in}{1.802008in}}{\pgfqpoint{3.974389in}{1.791409in}}{\pgfqpoint{3.982203in}{1.783595in}}%
\pgfpathcurveto{\pgfqpoint{3.990016in}{1.775781in}}{\pgfqpoint{4.000616in}{1.771391in}}{\pgfqpoint{4.011666in}{1.771391in}}%
\pgfpathclose%
\pgfusepath{stroke,fill}%
\end{pgfscope}%
\begin{pgfscope}%
\pgfpathrectangle{\pgfqpoint{0.800000in}{0.528000in}}{\pgfqpoint{4.960000in}{3.696000in}}%
\pgfusepath{clip}%
\pgfsetbuttcap%
\pgfsetroundjoin%
\definecolor{currentfill}{rgb}{0.000000,0.000000,0.000000}%
\pgfsetfillcolor{currentfill}%
\pgfsetlinewidth{1.003750pt}%
\definecolor{currentstroke}{rgb}{0.000000,0.000000,0.000000}%
\pgfsetstrokecolor{currentstroke}%
\pgfsetdash{}{0pt}%
\pgfpathmoveto{\pgfqpoint{4.011666in}{1.761682in}}%
\pgfpathcurveto{\pgfqpoint{4.022716in}{1.761682in}}{\pgfqpoint{4.033315in}{1.766072in}}{\pgfqpoint{4.041128in}{1.773886in}}%
\pgfpathcurveto{\pgfqpoint{4.048942in}{1.781700in}}{\pgfqpoint{4.053332in}{1.792299in}}{\pgfqpoint{4.053332in}{1.803349in}}%
\pgfpathcurveto{\pgfqpoint{4.053332in}{1.814399in}}{\pgfqpoint{4.048942in}{1.824998in}}{\pgfqpoint{4.041128in}{1.832812in}}%
\pgfpathcurveto{\pgfqpoint{4.033315in}{1.840625in}}{\pgfqpoint{4.022716in}{1.845016in}}{\pgfqpoint{4.011666in}{1.845016in}}%
\pgfpathcurveto{\pgfqpoint{4.000616in}{1.845016in}}{\pgfqpoint{3.990016in}{1.840625in}}{\pgfqpoint{3.982203in}{1.832812in}}%
\pgfpathcurveto{\pgfqpoint{3.974389in}{1.824998in}}{\pgfqpoint{3.969999in}{1.814399in}}{\pgfqpoint{3.969999in}{1.803349in}}%
\pgfpathcurveto{\pgfqpoint{3.969999in}{1.792299in}}{\pgfqpoint{3.974389in}{1.781700in}}{\pgfqpoint{3.982203in}{1.773886in}}%
\pgfpathcurveto{\pgfqpoint{3.990016in}{1.766072in}}{\pgfqpoint{4.000616in}{1.761682in}}{\pgfqpoint{4.011666in}{1.761682in}}%
\pgfpathclose%
\pgfusepath{stroke,fill}%
\end{pgfscope}%
\begin{pgfscope}%
\pgfpathrectangle{\pgfqpoint{0.800000in}{0.528000in}}{\pgfqpoint{4.960000in}{3.696000in}}%
\pgfusepath{clip}%
\pgfsetbuttcap%
\pgfsetroundjoin%
\definecolor{currentfill}{rgb}{0.000000,0.000000,0.000000}%
\pgfsetfillcolor{currentfill}%
\pgfsetlinewidth{1.003750pt}%
\definecolor{currentstroke}{rgb}{0.000000,0.000000,0.000000}%
\pgfsetstrokecolor{currentstroke}%
\pgfsetdash{}{0pt}%
\pgfpathmoveto{\pgfqpoint{4.011666in}{1.693719in}}%
\pgfpathcurveto{\pgfqpoint{4.022716in}{1.693719in}}{\pgfqpoint{4.033315in}{1.698110in}}{\pgfqpoint{4.041128in}{1.705923in}}%
\pgfpathcurveto{\pgfqpoint{4.048942in}{1.713737in}}{\pgfqpoint{4.053332in}{1.724336in}}{\pgfqpoint{4.053332in}{1.735386in}}%
\pgfpathcurveto{\pgfqpoint{4.053332in}{1.746436in}}{\pgfqpoint{4.048942in}{1.757035in}}{\pgfqpoint{4.041128in}{1.764849in}}%
\pgfpathcurveto{\pgfqpoint{4.033315in}{1.772663in}}{\pgfqpoint{4.022716in}{1.777053in}}{\pgfqpoint{4.011666in}{1.777053in}}%
\pgfpathcurveto{\pgfqpoint{4.000616in}{1.777053in}}{\pgfqpoint{3.990016in}{1.772663in}}{\pgfqpoint{3.982203in}{1.764849in}}%
\pgfpathcurveto{\pgfqpoint{3.974389in}{1.757035in}}{\pgfqpoint{3.969999in}{1.746436in}}{\pgfqpoint{3.969999in}{1.735386in}}%
\pgfpathcurveto{\pgfqpoint{3.969999in}{1.724336in}}{\pgfqpoint{3.974389in}{1.713737in}}{\pgfqpoint{3.982203in}{1.705923in}}%
\pgfpathcurveto{\pgfqpoint{3.990016in}{1.698110in}}{\pgfqpoint{4.000616in}{1.693719in}}{\pgfqpoint{4.011666in}{1.693719in}}%
\pgfpathclose%
\pgfusepath{stroke,fill}%
\end{pgfscope}%
\begin{pgfscope}%
\pgfpathrectangle{\pgfqpoint{0.800000in}{0.528000in}}{\pgfqpoint{4.960000in}{3.696000in}}%
\pgfusepath{clip}%
\pgfsetbuttcap%
\pgfsetroundjoin%
\definecolor{currentfill}{rgb}{0.000000,0.000000,0.000000}%
\pgfsetfillcolor{currentfill}%
\pgfsetlinewidth{1.003750pt}%
\definecolor{currentstroke}{rgb}{0.000000,0.000000,0.000000}%
\pgfsetstrokecolor{currentstroke}%
\pgfsetdash{}{0pt}%
\pgfpathmoveto{\pgfqpoint{4.011666in}{1.577212in}}%
\pgfpathcurveto{\pgfqpoint{4.022716in}{1.577212in}}{\pgfqpoint{4.033315in}{1.581602in}}{\pgfqpoint{4.041128in}{1.589416in}}%
\pgfpathcurveto{\pgfqpoint{4.048942in}{1.597229in}}{\pgfqpoint{4.053332in}{1.607828in}}{\pgfqpoint{4.053332in}{1.618878in}}%
\pgfpathcurveto{\pgfqpoint{4.053332in}{1.629929in}}{\pgfqpoint{4.048942in}{1.640528in}}{\pgfqpoint{4.041128in}{1.648341in}}%
\pgfpathcurveto{\pgfqpoint{4.033315in}{1.656155in}}{\pgfqpoint{4.022716in}{1.660545in}}{\pgfqpoint{4.011666in}{1.660545in}}%
\pgfpathcurveto{\pgfqpoint{4.000616in}{1.660545in}}{\pgfqpoint{3.990016in}{1.656155in}}{\pgfqpoint{3.982203in}{1.648341in}}%
\pgfpathcurveto{\pgfqpoint{3.974389in}{1.640528in}}{\pgfqpoint{3.969999in}{1.629929in}}{\pgfqpoint{3.969999in}{1.618878in}}%
\pgfpathcurveto{\pgfqpoint{3.969999in}{1.607828in}}{\pgfqpoint{3.974389in}{1.597229in}}{\pgfqpoint{3.982203in}{1.589416in}}%
\pgfpathcurveto{\pgfqpoint{3.990016in}{1.581602in}}{\pgfqpoint{4.000616in}{1.577212in}}{\pgfqpoint{4.011666in}{1.577212in}}%
\pgfpathclose%
\pgfusepath{stroke,fill}%
\end{pgfscope}%
\begin{pgfscope}%
\pgfpathrectangle{\pgfqpoint{0.800000in}{0.528000in}}{\pgfqpoint{4.960000in}{3.696000in}}%
\pgfusepath{clip}%
\pgfsetbuttcap%
\pgfsetroundjoin%
\definecolor{currentfill}{rgb}{0.000000,0.000000,0.000000}%
\pgfsetfillcolor{currentfill}%
\pgfsetlinewidth{1.003750pt}%
\definecolor{currentstroke}{rgb}{0.000000,0.000000,0.000000}%
\pgfsetstrokecolor{currentstroke}%
\pgfsetdash{}{0pt}%
\pgfpathmoveto{\pgfqpoint{4.011666in}{1.693719in}}%
\pgfpathcurveto{\pgfqpoint{4.022716in}{1.693719in}}{\pgfqpoint{4.033315in}{1.698110in}}{\pgfqpoint{4.041128in}{1.705923in}}%
\pgfpathcurveto{\pgfqpoint{4.048942in}{1.713737in}}{\pgfqpoint{4.053332in}{1.724336in}}{\pgfqpoint{4.053332in}{1.735386in}}%
\pgfpathcurveto{\pgfqpoint{4.053332in}{1.746436in}}{\pgfqpoint{4.048942in}{1.757035in}}{\pgfqpoint{4.041128in}{1.764849in}}%
\pgfpathcurveto{\pgfqpoint{4.033315in}{1.772663in}}{\pgfqpoint{4.022716in}{1.777053in}}{\pgfqpoint{4.011666in}{1.777053in}}%
\pgfpathcurveto{\pgfqpoint{4.000616in}{1.777053in}}{\pgfqpoint{3.990016in}{1.772663in}}{\pgfqpoint{3.982203in}{1.764849in}}%
\pgfpathcurveto{\pgfqpoint{3.974389in}{1.757035in}}{\pgfqpoint{3.969999in}{1.746436in}}{\pgfqpoint{3.969999in}{1.735386in}}%
\pgfpathcurveto{\pgfqpoint{3.969999in}{1.724336in}}{\pgfqpoint{3.974389in}{1.713737in}}{\pgfqpoint{3.982203in}{1.705923in}}%
\pgfpathcurveto{\pgfqpoint{3.990016in}{1.698110in}}{\pgfqpoint{4.000616in}{1.693719in}}{\pgfqpoint{4.011666in}{1.693719in}}%
\pgfpathclose%
\pgfusepath{stroke,fill}%
\end{pgfscope}%
\begin{pgfscope}%
\pgfpathrectangle{\pgfqpoint{0.800000in}{0.528000in}}{\pgfqpoint{4.960000in}{3.696000in}}%
\pgfusepath{clip}%
\pgfsetbuttcap%
\pgfsetroundjoin%
\definecolor{currentfill}{rgb}{0.000000,0.000000,0.000000}%
\pgfsetfillcolor{currentfill}%
\pgfsetlinewidth{1.003750pt}%
\definecolor{currentstroke}{rgb}{0.000000,0.000000,0.000000}%
\pgfsetstrokecolor{currentstroke}%
\pgfsetdash{}{0pt}%
\pgfpathmoveto{\pgfqpoint{4.011666in}{1.887899in}}%
\pgfpathcurveto{\pgfqpoint{4.022716in}{1.887899in}}{\pgfqpoint{4.033315in}{1.892289in}}{\pgfqpoint{4.041128in}{1.900103in}}%
\pgfpathcurveto{\pgfqpoint{4.048942in}{1.907916in}}{\pgfqpoint{4.053332in}{1.918515in}}{\pgfqpoint{4.053332in}{1.929565in}}%
\pgfpathcurveto{\pgfqpoint{4.053332in}{1.940616in}}{\pgfqpoint{4.048942in}{1.951215in}}{\pgfqpoint{4.041128in}{1.959028in}}%
\pgfpathcurveto{\pgfqpoint{4.033315in}{1.966842in}}{\pgfqpoint{4.022716in}{1.971232in}}{\pgfqpoint{4.011666in}{1.971232in}}%
\pgfpathcurveto{\pgfqpoint{4.000616in}{1.971232in}}{\pgfqpoint{3.990016in}{1.966842in}}{\pgfqpoint{3.982203in}{1.959028in}}%
\pgfpathcurveto{\pgfqpoint{3.974389in}{1.951215in}}{\pgfqpoint{3.969999in}{1.940616in}}{\pgfqpoint{3.969999in}{1.929565in}}%
\pgfpathcurveto{\pgfqpoint{3.969999in}{1.918515in}}{\pgfqpoint{3.974389in}{1.907916in}}{\pgfqpoint{3.982203in}{1.900103in}}%
\pgfpathcurveto{\pgfqpoint{3.990016in}{1.892289in}}{\pgfqpoint{4.000616in}{1.887899in}}{\pgfqpoint{4.011666in}{1.887899in}}%
\pgfpathclose%
\pgfusepath{stroke,fill}%
\end{pgfscope}%
\begin{pgfscope}%
\pgfpathrectangle{\pgfqpoint{0.800000in}{0.528000in}}{\pgfqpoint{4.960000in}{3.696000in}}%
\pgfusepath{clip}%
\pgfsetbuttcap%
\pgfsetroundjoin%
\definecolor{currentfill}{rgb}{0.000000,0.000000,0.000000}%
\pgfsetfillcolor{currentfill}%
\pgfsetlinewidth{1.003750pt}%
\definecolor{currentstroke}{rgb}{0.000000,0.000000,0.000000}%
\pgfsetstrokecolor{currentstroke}%
\pgfsetdash{}{0pt}%
\pgfpathmoveto{\pgfqpoint{4.011666in}{1.800518in}}%
\pgfpathcurveto{\pgfqpoint{4.022716in}{1.800518in}}{\pgfqpoint{4.033315in}{1.804908in}}{\pgfqpoint{4.041128in}{1.812722in}}%
\pgfpathcurveto{\pgfqpoint{4.048942in}{1.820536in}}{\pgfqpoint{4.053332in}{1.831135in}}{\pgfqpoint{4.053332in}{1.842185in}}%
\pgfpathcurveto{\pgfqpoint{4.053332in}{1.853235in}}{\pgfqpoint{4.048942in}{1.863834in}}{\pgfqpoint{4.041128in}{1.871648in}}%
\pgfpathcurveto{\pgfqpoint{4.033315in}{1.879461in}}{\pgfqpoint{4.022716in}{1.883851in}}{\pgfqpoint{4.011666in}{1.883851in}}%
\pgfpathcurveto{\pgfqpoint{4.000616in}{1.883851in}}{\pgfqpoint{3.990016in}{1.879461in}}{\pgfqpoint{3.982203in}{1.871648in}}%
\pgfpathcurveto{\pgfqpoint{3.974389in}{1.863834in}}{\pgfqpoint{3.969999in}{1.853235in}}{\pgfqpoint{3.969999in}{1.842185in}}%
\pgfpathcurveto{\pgfqpoint{3.969999in}{1.831135in}}{\pgfqpoint{3.974389in}{1.820536in}}{\pgfqpoint{3.982203in}{1.812722in}}%
\pgfpathcurveto{\pgfqpoint{3.990016in}{1.804908in}}{\pgfqpoint{4.000616in}{1.800518in}}{\pgfqpoint{4.011666in}{1.800518in}}%
\pgfpathclose%
\pgfusepath{stroke,fill}%
\end{pgfscope}%
\begin{pgfscope}%
\pgfpathrectangle{\pgfqpoint{0.800000in}{0.528000in}}{\pgfqpoint{4.960000in}{3.696000in}}%
\pgfusepath{clip}%
\pgfsetbuttcap%
\pgfsetroundjoin%
\definecolor{currentfill}{rgb}{0.000000,0.000000,0.000000}%
\pgfsetfillcolor{currentfill}%
\pgfsetlinewidth{1.003750pt}%
\definecolor{currentstroke}{rgb}{0.000000,0.000000,0.000000}%
\pgfsetstrokecolor{currentstroke}%
\pgfsetdash{}{0pt}%
\pgfpathmoveto{\pgfqpoint{4.011666in}{1.528667in}}%
\pgfpathcurveto{\pgfqpoint{4.022716in}{1.528667in}}{\pgfqpoint{4.033315in}{1.533057in}}{\pgfqpoint{4.041128in}{1.540871in}}%
\pgfpathcurveto{\pgfqpoint{4.048942in}{1.548684in}}{\pgfqpoint{4.053332in}{1.559283in}}{\pgfqpoint{4.053332in}{1.570334in}}%
\pgfpathcurveto{\pgfqpoint{4.053332in}{1.581384in}}{\pgfqpoint{4.048942in}{1.591983in}}{\pgfqpoint{4.041128in}{1.599796in}}%
\pgfpathcurveto{\pgfqpoint{4.033315in}{1.607610in}}{\pgfqpoint{4.022716in}{1.612000in}}{\pgfqpoint{4.011666in}{1.612000in}}%
\pgfpathcurveto{\pgfqpoint{4.000616in}{1.612000in}}{\pgfqpoint{3.990016in}{1.607610in}}{\pgfqpoint{3.982203in}{1.599796in}}%
\pgfpathcurveto{\pgfqpoint{3.974389in}{1.591983in}}{\pgfqpoint{3.969999in}{1.581384in}}{\pgfqpoint{3.969999in}{1.570334in}}%
\pgfpathcurveto{\pgfqpoint{3.969999in}{1.559283in}}{\pgfqpoint{3.974389in}{1.548684in}}{\pgfqpoint{3.982203in}{1.540871in}}%
\pgfpathcurveto{\pgfqpoint{3.990016in}{1.533057in}}{\pgfqpoint{4.000616in}{1.528667in}}{\pgfqpoint{4.011666in}{1.528667in}}%
\pgfpathclose%
\pgfusepath{stroke,fill}%
\end{pgfscope}%
\begin{pgfscope}%
\pgfpathrectangle{\pgfqpoint{0.800000in}{0.528000in}}{\pgfqpoint{4.960000in}{3.696000in}}%
\pgfusepath{clip}%
\pgfsetbuttcap%
\pgfsetroundjoin%
\definecolor{currentfill}{rgb}{0.000000,0.000000,0.000000}%
\pgfsetfillcolor{currentfill}%
\pgfsetlinewidth{1.003750pt}%
\definecolor{currentstroke}{rgb}{0.000000,0.000000,0.000000}%
\pgfsetstrokecolor{currentstroke}%
\pgfsetdash{}{0pt}%
\pgfpathmoveto{\pgfqpoint{4.011666in}{1.858772in}}%
\pgfpathcurveto{\pgfqpoint{4.022716in}{1.858772in}}{\pgfqpoint{4.033315in}{1.863162in}}{\pgfqpoint{4.041128in}{1.870976in}}%
\pgfpathcurveto{\pgfqpoint{4.048942in}{1.878789in}}{\pgfqpoint{4.053332in}{1.889388in}}{\pgfqpoint{4.053332in}{1.900439in}}%
\pgfpathcurveto{\pgfqpoint{4.053332in}{1.911489in}}{\pgfqpoint{4.048942in}{1.922088in}}{\pgfqpoint{4.041128in}{1.929901in}}%
\pgfpathcurveto{\pgfqpoint{4.033315in}{1.937715in}}{\pgfqpoint{4.022716in}{1.942105in}}{\pgfqpoint{4.011666in}{1.942105in}}%
\pgfpathcurveto{\pgfqpoint{4.000616in}{1.942105in}}{\pgfqpoint{3.990016in}{1.937715in}}{\pgfqpoint{3.982203in}{1.929901in}}%
\pgfpathcurveto{\pgfqpoint{3.974389in}{1.922088in}}{\pgfqpoint{3.969999in}{1.911489in}}{\pgfqpoint{3.969999in}{1.900439in}}%
\pgfpathcurveto{\pgfqpoint{3.969999in}{1.889388in}}{\pgfqpoint{3.974389in}{1.878789in}}{\pgfqpoint{3.982203in}{1.870976in}}%
\pgfpathcurveto{\pgfqpoint{3.990016in}{1.863162in}}{\pgfqpoint{4.000616in}{1.858772in}}{\pgfqpoint{4.011666in}{1.858772in}}%
\pgfpathclose%
\pgfusepath{stroke,fill}%
\end{pgfscope}%
\begin{pgfscope}%
\pgfpathrectangle{\pgfqpoint{0.800000in}{0.528000in}}{\pgfqpoint{4.960000in}{3.696000in}}%
\pgfusepath{clip}%
\pgfsetbuttcap%
\pgfsetroundjoin%
\definecolor{currentfill}{rgb}{0.000000,0.000000,0.000000}%
\pgfsetfillcolor{currentfill}%
\pgfsetlinewidth{1.003750pt}%
\definecolor{currentstroke}{rgb}{0.000000,0.000000,0.000000}%
\pgfsetstrokecolor{currentstroke}%
\pgfsetdash{}{0pt}%
\pgfpathmoveto{\pgfqpoint{4.011666in}{1.878190in}}%
\pgfpathcurveto{\pgfqpoint{4.022716in}{1.878190in}}{\pgfqpoint{4.033315in}{1.882580in}}{\pgfqpoint{4.041128in}{1.890394in}}%
\pgfpathcurveto{\pgfqpoint{4.048942in}{1.898207in}}{\pgfqpoint{4.053332in}{1.908806in}}{\pgfqpoint{4.053332in}{1.919857in}}%
\pgfpathcurveto{\pgfqpoint{4.053332in}{1.930907in}}{\pgfqpoint{4.048942in}{1.941506in}}{\pgfqpoint{4.041128in}{1.949319in}}%
\pgfpathcurveto{\pgfqpoint{4.033315in}{1.957133in}}{\pgfqpoint{4.022716in}{1.961523in}}{\pgfqpoint{4.011666in}{1.961523in}}%
\pgfpathcurveto{\pgfqpoint{4.000616in}{1.961523in}}{\pgfqpoint{3.990016in}{1.957133in}}{\pgfqpoint{3.982203in}{1.949319in}}%
\pgfpathcurveto{\pgfqpoint{3.974389in}{1.941506in}}{\pgfqpoint{3.969999in}{1.930907in}}{\pgfqpoint{3.969999in}{1.919857in}}%
\pgfpathcurveto{\pgfqpoint{3.969999in}{1.908806in}}{\pgfqpoint{3.974389in}{1.898207in}}{\pgfqpoint{3.982203in}{1.890394in}}%
\pgfpathcurveto{\pgfqpoint{3.990016in}{1.882580in}}{\pgfqpoint{4.000616in}{1.878190in}}{\pgfqpoint{4.011666in}{1.878190in}}%
\pgfpathclose%
\pgfusepath{stroke,fill}%
\end{pgfscope}%
\begin{pgfscope}%
\pgfpathrectangle{\pgfqpoint{0.800000in}{0.528000in}}{\pgfqpoint{4.960000in}{3.696000in}}%
\pgfusepath{clip}%
\pgfsetbuttcap%
\pgfsetroundjoin%
\definecolor{currentfill}{rgb}{0.000000,0.000000,0.000000}%
\pgfsetfillcolor{currentfill}%
\pgfsetlinewidth{1.003750pt}%
\definecolor{currentstroke}{rgb}{0.000000,0.000000,0.000000}%
\pgfsetstrokecolor{currentstroke}%
\pgfsetdash{}{0pt}%
\pgfpathmoveto{\pgfqpoint{4.011666in}{1.751973in}}%
\pgfpathcurveto{\pgfqpoint{4.022716in}{1.751973in}}{\pgfqpoint{4.033315in}{1.756364in}}{\pgfqpoint{4.041128in}{1.764177in}}%
\pgfpathcurveto{\pgfqpoint{4.048942in}{1.771991in}}{\pgfqpoint{4.053332in}{1.782590in}}{\pgfqpoint{4.053332in}{1.793640in}}%
\pgfpathcurveto{\pgfqpoint{4.053332in}{1.804690in}}{\pgfqpoint{4.048942in}{1.815289in}}{\pgfqpoint{4.041128in}{1.823103in}}%
\pgfpathcurveto{\pgfqpoint{4.033315in}{1.830916in}}{\pgfqpoint{4.022716in}{1.835307in}}{\pgfqpoint{4.011666in}{1.835307in}}%
\pgfpathcurveto{\pgfqpoint{4.000616in}{1.835307in}}{\pgfqpoint{3.990016in}{1.830916in}}{\pgfqpoint{3.982203in}{1.823103in}}%
\pgfpathcurveto{\pgfqpoint{3.974389in}{1.815289in}}{\pgfqpoint{3.969999in}{1.804690in}}{\pgfqpoint{3.969999in}{1.793640in}}%
\pgfpathcurveto{\pgfqpoint{3.969999in}{1.782590in}}{\pgfqpoint{3.974389in}{1.771991in}}{\pgfqpoint{3.982203in}{1.764177in}}%
\pgfpathcurveto{\pgfqpoint{3.990016in}{1.756364in}}{\pgfqpoint{4.000616in}{1.751973in}}{\pgfqpoint{4.011666in}{1.751973in}}%
\pgfpathclose%
\pgfusepath{stroke,fill}%
\end{pgfscope}%
\begin{pgfscope}%
\pgfpathrectangle{\pgfqpoint{0.800000in}{0.528000in}}{\pgfqpoint{4.960000in}{3.696000in}}%
\pgfusepath{clip}%
\pgfsetbuttcap%
\pgfsetroundjoin%
\definecolor{currentfill}{rgb}{0.000000,0.000000,0.000000}%
\pgfsetfillcolor{currentfill}%
\pgfsetlinewidth{1.003750pt}%
\definecolor{currentstroke}{rgb}{0.000000,0.000000,0.000000}%
\pgfsetstrokecolor{currentstroke}%
\pgfsetdash{}{0pt}%
\pgfpathmoveto{\pgfqpoint{4.011666in}{1.567503in}}%
\pgfpathcurveto{\pgfqpoint{4.022716in}{1.567503in}}{\pgfqpoint{4.033315in}{1.571893in}}{\pgfqpoint{4.041128in}{1.579707in}}%
\pgfpathcurveto{\pgfqpoint{4.048942in}{1.587520in}}{\pgfqpoint{4.053332in}{1.598119in}}{\pgfqpoint{4.053332in}{1.609170in}}%
\pgfpathcurveto{\pgfqpoint{4.053332in}{1.620220in}}{\pgfqpoint{4.048942in}{1.630819in}}{\pgfqpoint{4.041128in}{1.638632in}}%
\pgfpathcurveto{\pgfqpoint{4.033315in}{1.646446in}}{\pgfqpoint{4.022716in}{1.650836in}}{\pgfqpoint{4.011666in}{1.650836in}}%
\pgfpathcurveto{\pgfqpoint{4.000616in}{1.650836in}}{\pgfqpoint{3.990016in}{1.646446in}}{\pgfqpoint{3.982203in}{1.638632in}}%
\pgfpathcurveto{\pgfqpoint{3.974389in}{1.630819in}}{\pgfqpoint{3.969999in}{1.620220in}}{\pgfqpoint{3.969999in}{1.609170in}}%
\pgfpathcurveto{\pgfqpoint{3.969999in}{1.598119in}}{\pgfqpoint{3.974389in}{1.587520in}}{\pgfqpoint{3.982203in}{1.579707in}}%
\pgfpathcurveto{\pgfqpoint{3.990016in}{1.571893in}}{\pgfqpoint{4.000616in}{1.567503in}}{\pgfqpoint{4.011666in}{1.567503in}}%
\pgfpathclose%
\pgfusepath{stroke,fill}%
\end{pgfscope}%
\begin{pgfscope}%
\pgfpathrectangle{\pgfqpoint{0.800000in}{0.528000in}}{\pgfqpoint{4.960000in}{3.696000in}}%
\pgfusepath{clip}%
\pgfsetbuttcap%
\pgfsetroundjoin%
\definecolor{currentfill}{rgb}{0.000000,0.000000,0.000000}%
\pgfsetfillcolor{currentfill}%
\pgfsetlinewidth{1.003750pt}%
\definecolor{currentstroke}{rgb}{0.000000,0.000000,0.000000}%
\pgfsetstrokecolor{currentstroke}%
\pgfsetdash{}{0pt}%
\pgfpathmoveto{\pgfqpoint{4.011666in}{1.703428in}}%
\pgfpathcurveto{\pgfqpoint{4.022716in}{1.703428in}}{\pgfqpoint{4.033315in}{1.707819in}}{\pgfqpoint{4.041128in}{1.715632in}}%
\pgfpathcurveto{\pgfqpoint{4.048942in}{1.723446in}}{\pgfqpoint{4.053332in}{1.734045in}}{\pgfqpoint{4.053332in}{1.745095in}}%
\pgfpathcurveto{\pgfqpoint{4.053332in}{1.756145in}}{\pgfqpoint{4.048942in}{1.766744in}}{\pgfqpoint{4.041128in}{1.774558in}}%
\pgfpathcurveto{\pgfqpoint{4.033315in}{1.782371in}}{\pgfqpoint{4.022716in}{1.786762in}}{\pgfqpoint{4.011666in}{1.786762in}}%
\pgfpathcurveto{\pgfqpoint{4.000616in}{1.786762in}}{\pgfqpoint{3.990016in}{1.782371in}}{\pgfqpoint{3.982203in}{1.774558in}}%
\pgfpathcurveto{\pgfqpoint{3.974389in}{1.766744in}}{\pgfqpoint{3.969999in}{1.756145in}}{\pgfqpoint{3.969999in}{1.745095in}}%
\pgfpathcurveto{\pgfqpoint{3.969999in}{1.734045in}}{\pgfqpoint{3.974389in}{1.723446in}}{\pgfqpoint{3.982203in}{1.715632in}}%
\pgfpathcurveto{\pgfqpoint{3.990016in}{1.707819in}}{\pgfqpoint{4.000616in}{1.703428in}}{\pgfqpoint{4.011666in}{1.703428in}}%
\pgfpathclose%
\pgfusepath{stroke,fill}%
\end{pgfscope}%
\begin{pgfscope}%
\pgfpathrectangle{\pgfqpoint{0.800000in}{0.528000in}}{\pgfqpoint{4.960000in}{3.696000in}}%
\pgfusepath{clip}%
\pgfsetbuttcap%
\pgfsetroundjoin%
\definecolor{currentfill}{rgb}{0.000000,0.000000,0.000000}%
\pgfsetfillcolor{currentfill}%
\pgfsetlinewidth{1.003750pt}%
\definecolor{currentstroke}{rgb}{0.000000,0.000000,0.000000}%
\pgfsetstrokecolor{currentstroke}%
\pgfsetdash{}{0pt}%
\pgfpathmoveto{\pgfqpoint{4.011666in}{1.781100in}}%
\pgfpathcurveto{\pgfqpoint{4.022716in}{1.781100in}}{\pgfqpoint{4.033315in}{1.785490in}}{\pgfqpoint{4.041128in}{1.793304in}}%
\pgfpathcurveto{\pgfqpoint{4.048942in}{1.801118in}}{\pgfqpoint{4.053332in}{1.811717in}}{\pgfqpoint{4.053332in}{1.822767in}}%
\pgfpathcurveto{\pgfqpoint{4.053332in}{1.833817in}}{\pgfqpoint{4.048942in}{1.844416in}}{\pgfqpoint{4.041128in}{1.852230in}}%
\pgfpathcurveto{\pgfqpoint{4.033315in}{1.860043in}}{\pgfqpoint{4.022716in}{1.864433in}}{\pgfqpoint{4.011666in}{1.864433in}}%
\pgfpathcurveto{\pgfqpoint{4.000616in}{1.864433in}}{\pgfqpoint{3.990016in}{1.860043in}}{\pgfqpoint{3.982203in}{1.852230in}}%
\pgfpathcurveto{\pgfqpoint{3.974389in}{1.844416in}}{\pgfqpoint{3.969999in}{1.833817in}}{\pgfqpoint{3.969999in}{1.822767in}}%
\pgfpathcurveto{\pgfqpoint{3.969999in}{1.811717in}}{\pgfqpoint{3.974389in}{1.801118in}}{\pgfqpoint{3.982203in}{1.793304in}}%
\pgfpathcurveto{\pgfqpoint{3.990016in}{1.785490in}}{\pgfqpoint{4.000616in}{1.781100in}}{\pgfqpoint{4.011666in}{1.781100in}}%
\pgfpathclose%
\pgfusepath{stroke,fill}%
\end{pgfscope}%
\begin{pgfscope}%
\pgfpathrectangle{\pgfqpoint{0.800000in}{0.528000in}}{\pgfqpoint{4.960000in}{3.696000in}}%
\pgfusepath{clip}%
\pgfsetbuttcap%
\pgfsetroundjoin%
\definecolor{currentfill}{rgb}{0.000000,0.000000,0.000000}%
\pgfsetfillcolor{currentfill}%
\pgfsetlinewidth{1.003750pt}%
\definecolor{currentstroke}{rgb}{0.000000,0.000000,0.000000}%
\pgfsetstrokecolor{currentstroke}%
\pgfsetdash{}{0pt}%
\pgfpathmoveto{\pgfqpoint{4.011666in}{1.616048in}}%
\pgfpathcurveto{\pgfqpoint{4.022716in}{1.616048in}}{\pgfqpoint{4.033315in}{1.620438in}}{\pgfqpoint{4.041128in}{1.628252in}}%
\pgfpathcurveto{\pgfqpoint{4.048942in}{1.636065in}}{\pgfqpoint{4.053332in}{1.646664in}}{\pgfqpoint{4.053332in}{1.657714in}}%
\pgfpathcurveto{\pgfqpoint{4.053332in}{1.668764in}}{\pgfqpoint{4.048942in}{1.679364in}}{\pgfqpoint{4.041128in}{1.687177in}}%
\pgfpathcurveto{\pgfqpoint{4.033315in}{1.694991in}}{\pgfqpoint{4.022716in}{1.699381in}}{\pgfqpoint{4.011666in}{1.699381in}}%
\pgfpathcurveto{\pgfqpoint{4.000616in}{1.699381in}}{\pgfqpoint{3.990016in}{1.694991in}}{\pgfqpoint{3.982203in}{1.687177in}}%
\pgfpathcurveto{\pgfqpoint{3.974389in}{1.679364in}}{\pgfqpoint{3.969999in}{1.668764in}}{\pgfqpoint{3.969999in}{1.657714in}}%
\pgfpathcurveto{\pgfqpoint{3.969999in}{1.646664in}}{\pgfqpoint{3.974389in}{1.636065in}}{\pgfqpoint{3.982203in}{1.628252in}}%
\pgfpathcurveto{\pgfqpoint{3.990016in}{1.620438in}}{\pgfqpoint{4.000616in}{1.616048in}}{\pgfqpoint{4.011666in}{1.616048in}}%
\pgfpathclose%
\pgfusepath{stroke,fill}%
\end{pgfscope}%
\begin{pgfscope}%
\pgfpathrectangle{\pgfqpoint{0.800000in}{0.528000in}}{\pgfqpoint{4.960000in}{3.696000in}}%
\pgfusepath{clip}%
\pgfsetbuttcap%
\pgfsetroundjoin%
\definecolor{currentfill}{rgb}{0.000000,0.000000,0.000000}%
\pgfsetfillcolor{currentfill}%
\pgfsetlinewidth{1.003750pt}%
\definecolor{currentstroke}{rgb}{0.000000,0.000000,0.000000}%
\pgfsetstrokecolor{currentstroke}%
\pgfsetdash{}{0pt}%
\pgfpathmoveto{\pgfqpoint{4.011666in}{1.917026in}}%
\pgfpathcurveto{\pgfqpoint{4.022716in}{1.917026in}}{\pgfqpoint{4.033315in}{1.921416in}}{\pgfqpoint{4.041128in}{1.929230in}}%
\pgfpathcurveto{\pgfqpoint{4.048942in}{1.937043in}}{\pgfqpoint{4.053332in}{1.947642in}}{\pgfqpoint{4.053332in}{1.958692in}}%
\pgfpathcurveto{\pgfqpoint{4.053332in}{1.969743in}}{\pgfqpoint{4.048942in}{1.980342in}}{\pgfqpoint{4.041128in}{1.988155in}}%
\pgfpathcurveto{\pgfqpoint{4.033315in}{1.995969in}}{\pgfqpoint{4.022716in}{2.000359in}}{\pgfqpoint{4.011666in}{2.000359in}}%
\pgfpathcurveto{\pgfqpoint{4.000616in}{2.000359in}}{\pgfqpoint{3.990016in}{1.995969in}}{\pgfqpoint{3.982203in}{1.988155in}}%
\pgfpathcurveto{\pgfqpoint{3.974389in}{1.980342in}}{\pgfqpoint{3.969999in}{1.969743in}}{\pgfqpoint{3.969999in}{1.958692in}}%
\pgfpathcurveto{\pgfqpoint{3.969999in}{1.947642in}}{\pgfqpoint{3.974389in}{1.937043in}}{\pgfqpoint{3.982203in}{1.929230in}}%
\pgfpathcurveto{\pgfqpoint{3.990016in}{1.921416in}}{\pgfqpoint{4.000616in}{1.917026in}}{\pgfqpoint{4.011666in}{1.917026in}}%
\pgfpathclose%
\pgfusepath{stroke,fill}%
\end{pgfscope}%
\begin{pgfscope}%
\pgfpathrectangle{\pgfqpoint{0.800000in}{0.528000in}}{\pgfqpoint{4.960000in}{3.696000in}}%
\pgfusepath{clip}%
\pgfsetbuttcap%
\pgfsetroundjoin%
\definecolor{currentfill}{rgb}{0.000000,0.000000,0.000000}%
\pgfsetfillcolor{currentfill}%
\pgfsetlinewidth{1.003750pt}%
\definecolor{currentstroke}{rgb}{0.000000,0.000000,0.000000}%
\pgfsetstrokecolor{currentstroke}%
\pgfsetdash{}{0pt}%
\pgfpathmoveto{\pgfqpoint{4.011666in}{1.538376in}}%
\pgfpathcurveto{\pgfqpoint{4.022716in}{1.538376in}}{\pgfqpoint{4.033315in}{1.542766in}}{\pgfqpoint{4.041128in}{1.550580in}}%
\pgfpathcurveto{\pgfqpoint{4.048942in}{1.558393in}}{\pgfqpoint{4.053332in}{1.568992in}}{\pgfqpoint{4.053332in}{1.580043in}}%
\pgfpathcurveto{\pgfqpoint{4.053332in}{1.591093in}}{\pgfqpoint{4.048942in}{1.601692in}}{\pgfqpoint{4.041128in}{1.609505in}}%
\pgfpathcurveto{\pgfqpoint{4.033315in}{1.617319in}}{\pgfqpoint{4.022716in}{1.621709in}}{\pgfqpoint{4.011666in}{1.621709in}}%
\pgfpathcurveto{\pgfqpoint{4.000616in}{1.621709in}}{\pgfqpoint{3.990016in}{1.617319in}}{\pgfqpoint{3.982203in}{1.609505in}}%
\pgfpathcurveto{\pgfqpoint{3.974389in}{1.601692in}}{\pgfqpoint{3.969999in}{1.591093in}}{\pgfqpoint{3.969999in}{1.580043in}}%
\pgfpathcurveto{\pgfqpoint{3.969999in}{1.568992in}}{\pgfqpoint{3.974389in}{1.558393in}}{\pgfqpoint{3.982203in}{1.550580in}}%
\pgfpathcurveto{\pgfqpoint{3.990016in}{1.542766in}}{\pgfqpoint{4.000616in}{1.538376in}}{\pgfqpoint{4.011666in}{1.538376in}}%
\pgfpathclose%
\pgfusepath{stroke,fill}%
\end{pgfscope}%
\begin{pgfscope}%
\pgfpathrectangle{\pgfqpoint{0.800000in}{0.528000in}}{\pgfqpoint{4.960000in}{3.696000in}}%
\pgfusepath{clip}%
\pgfsetbuttcap%
\pgfsetroundjoin%
\definecolor{currentfill}{rgb}{0.000000,0.000000,0.000000}%
\pgfsetfillcolor{currentfill}%
\pgfsetlinewidth{1.003750pt}%
\definecolor{currentstroke}{rgb}{0.000000,0.000000,0.000000}%
\pgfsetstrokecolor{currentstroke}%
\pgfsetdash{}{0pt}%
\pgfpathmoveto{\pgfqpoint{4.011666in}{1.693719in}}%
\pgfpathcurveto{\pgfqpoint{4.022716in}{1.693719in}}{\pgfqpoint{4.033315in}{1.698110in}}{\pgfqpoint{4.041128in}{1.705923in}}%
\pgfpathcurveto{\pgfqpoint{4.048942in}{1.713737in}}{\pgfqpoint{4.053332in}{1.724336in}}{\pgfqpoint{4.053332in}{1.735386in}}%
\pgfpathcurveto{\pgfqpoint{4.053332in}{1.746436in}}{\pgfqpoint{4.048942in}{1.757035in}}{\pgfqpoint{4.041128in}{1.764849in}}%
\pgfpathcurveto{\pgfqpoint{4.033315in}{1.772663in}}{\pgfqpoint{4.022716in}{1.777053in}}{\pgfqpoint{4.011666in}{1.777053in}}%
\pgfpathcurveto{\pgfqpoint{4.000616in}{1.777053in}}{\pgfqpoint{3.990016in}{1.772663in}}{\pgfqpoint{3.982203in}{1.764849in}}%
\pgfpathcurveto{\pgfqpoint{3.974389in}{1.757035in}}{\pgfqpoint{3.969999in}{1.746436in}}{\pgfqpoint{3.969999in}{1.735386in}}%
\pgfpathcurveto{\pgfqpoint{3.969999in}{1.724336in}}{\pgfqpoint{3.974389in}{1.713737in}}{\pgfqpoint{3.982203in}{1.705923in}}%
\pgfpathcurveto{\pgfqpoint{3.990016in}{1.698110in}}{\pgfqpoint{4.000616in}{1.693719in}}{\pgfqpoint{4.011666in}{1.693719in}}%
\pgfpathclose%
\pgfusepath{stroke,fill}%
\end{pgfscope}%
\begin{pgfscope}%
\pgfpathrectangle{\pgfqpoint{0.800000in}{0.528000in}}{\pgfqpoint{4.960000in}{3.696000in}}%
\pgfusepath{clip}%
\pgfsetbuttcap%
\pgfsetroundjoin%
\definecolor{currentfill}{rgb}{0.000000,0.000000,0.000000}%
\pgfsetfillcolor{currentfill}%
\pgfsetlinewidth{1.003750pt}%
\definecolor{currentstroke}{rgb}{0.000000,0.000000,0.000000}%
\pgfsetstrokecolor{currentstroke}%
\pgfsetdash{}{0pt}%
\pgfpathmoveto{\pgfqpoint{4.011666in}{1.703428in}}%
\pgfpathcurveto{\pgfqpoint{4.022716in}{1.703428in}}{\pgfqpoint{4.033315in}{1.707819in}}{\pgfqpoint{4.041128in}{1.715632in}}%
\pgfpathcurveto{\pgfqpoint{4.048942in}{1.723446in}}{\pgfqpoint{4.053332in}{1.734045in}}{\pgfqpoint{4.053332in}{1.745095in}}%
\pgfpathcurveto{\pgfqpoint{4.053332in}{1.756145in}}{\pgfqpoint{4.048942in}{1.766744in}}{\pgfqpoint{4.041128in}{1.774558in}}%
\pgfpathcurveto{\pgfqpoint{4.033315in}{1.782371in}}{\pgfqpoint{4.022716in}{1.786762in}}{\pgfqpoint{4.011666in}{1.786762in}}%
\pgfpathcurveto{\pgfqpoint{4.000616in}{1.786762in}}{\pgfqpoint{3.990016in}{1.782371in}}{\pgfqpoint{3.982203in}{1.774558in}}%
\pgfpathcurveto{\pgfqpoint{3.974389in}{1.766744in}}{\pgfqpoint{3.969999in}{1.756145in}}{\pgfqpoint{3.969999in}{1.745095in}}%
\pgfpathcurveto{\pgfqpoint{3.969999in}{1.734045in}}{\pgfqpoint{3.974389in}{1.723446in}}{\pgfqpoint{3.982203in}{1.715632in}}%
\pgfpathcurveto{\pgfqpoint{3.990016in}{1.707819in}}{\pgfqpoint{4.000616in}{1.703428in}}{\pgfqpoint{4.011666in}{1.703428in}}%
\pgfpathclose%
\pgfusepath{stroke,fill}%
\end{pgfscope}%
\begin{pgfscope}%
\pgfpathrectangle{\pgfqpoint{0.800000in}{0.528000in}}{\pgfqpoint{4.960000in}{3.696000in}}%
\pgfusepath{clip}%
\pgfsetbuttcap%
\pgfsetroundjoin%
\definecolor{currentfill}{rgb}{0.000000,0.000000,0.000000}%
\pgfsetfillcolor{currentfill}%
\pgfsetlinewidth{1.003750pt}%
\definecolor{currentstroke}{rgb}{0.000000,0.000000,0.000000}%
\pgfsetstrokecolor{currentstroke}%
\pgfsetdash{}{0pt}%
\pgfpathmoveto{\pgfqpoint{4.011666in}{1.761682in}}%
\pgfpathcurveto{\pgfqpoint{4.022716in}{1.761682in}}{\pgfqpoint{4.033315in}{1.766072in}}{\pgfqpoint{4.041128in}{1.773886in}}%
\pgfpathcurveto{\pgfqpoint{4.048942in}{1.781700in}}{\pgfqpoint{4.053332in}{1.792299in}}{\pgfqpoint{4.053332in}{1.803349in}}%
\pgfpathcurveto{\pgfqpoint{4.053332in}{1.814399in}}{\pgfqpoint{4.048942in}{1.824998in}}{\pgfqpoint{4.041128in}{1.832812in}}%
\pgfpathcurveto{\pgfqpoint{4.033315in}{1.840625in}}{\pgfqpoint{4.022716in}{1.845016in}}{\pgfqpoint{4.011666in}{1.845016in}}%
\pgfpathcurveto{\pgfqpoint{4.000616in}{1.845016in}}{\pgfqpoint{3.990016in}{1.840625in}}{\pgfqpoint{3.982203in}{1.832812in}}%
\pgfpathcurveto{\pgfqpoint{3.974389in}{1.824998in}}{\pgfqpoint{3.969999in}{1.814399in}}{\pgfqpoint{3.969999in}{1.803349in}}%
\pgfpathcurveto{\pgfqpoint{3.969999in}{1.792299in}}{\pgfqpoint{3.974389in}{1.781700in}}{\pgfqpoint{3.982203in}{1.773886in}}%
\pgfpathcurveto{\pgfqpoint{3.990016in}{1.766072in}}{\pgfqpoint{4.000616in}{1.761682in}}{\pgfqpoint{4.011666in}{1.761682in}}%
\pgfpathclose%
\pgfusepath{stroke,fill}%
\end{pgfscope}%
\begin{pgfscope}%
\pgfpathrectangle{\pgfqpoint{0.800000in}{0.528000in}}{\pgfqpoint{4.960000in}{3.696000in}}%
\pgfusepath{clip}%
\pgfsetbuttcap%
\pgfsetroundjoin%
\definecolor{currentfill}{rgb}{0.000000,0.000000,0.000000}%
\pgfsetfillcolor{currentfill}%
\pgfsetlinewidth{1.003750pt}%
\definecolor{currentstroke}{rgb}{0.000000,0.000000,0.000000}%
\pgfsetstrokecolor{currentstroke}%
\pgfsetdash{}{0pt}%
\pgfpathmoveto{\pgfqpoint{4.011666in}{1.781100in}}%
\pgfpathcurveto{\pgfqpoint{4.022716in}{1.781100in}}{\pgfqpoint{4.033315in}{1.785490in}}{\pgfqpoint{4.041128in}{1.793304in}}%
\pgfpathcurveto{\pgfqpoint{4.048942in}{1.801118in}}{\pgfqpoint{4.053332in}{1.811717in}}{\pgfqpoint{4.053332in}{1.822767in}}%
\pgfpathcurveto{\pgfqpoint{4.053332in}{1.833817in}}{\pgfqpoint{4.048942in}{1.844416in}}{\pgfqpoint{4.041128in}{1.852230in}}%
\pgfpathcurveto{\pgfqpoint{4.033315in}{1.860043in}}{\pgfqpoint{4.022716in}{1.864433in}}{\pgfqpoint{4.011666in}{1.864433in}}%
\pgfpathcurveto{\pgfqpoint{4.000616in}{1.864433in}}{\pgfqpoint{3.990016in}{1.860043in}}{\pgfqpoint{3.982203in}{1.852230in}}%
\pgfpathcurveto{\pgfqpoint{3.974389in}{1.844416in}}{\pgfqpoint{3.969999in}{1.833817in}}{\pgfqpoint{3.969999in}{1.822767in}}%
\pgfpathcurveto{\pgfqpoint{3.969999in}{1.811717in}}{\pgfqpoint{3.974389in}{1.801118in}}{\pgfqpoint{3.982203in}{1.793304in}}%
\pgfpathcurveto{\pgfqpoint{3.990016in}{1.785490in}}{\pgfqpoint{4.000616in}{1.781100in}}{\pgfqpoint{4.011666in}{1.781100in}}%
\pgfpathclose%
\pgfusepath{stroke,fill}%
\end{pgfscope}%
\begin{pgfscope}%
\pgfpathrectangle{\pgfqpoint{0.800000in}{0.528000in}}{\pgfqpoint{4.960000in}{3.696000in}}%
\pgfusepath{clip}%
\pgfsetbuttcap%
\pgfsetroundjoin%
\definecolor{currentfill}{rgb}{0.000000,0.000000,0.000000}%
\pgfsetfillcolor{currentfill}%
\pgfsetlinewidth{1.003750pt}%
\definecolor{currentstroke}{rgb}{0.000000,0.000000,0.000000}%
\pgfsetstrokecolor{currentstroke}%
\pgfsetdash{}{0pt}%
\pgfpathmoveto{\pgfqpoint{4.011666in}{1.917026in}}%
\pgfpathcurveto{\pgfqpoint{4.022716in}{1.917026in}}{\pgfqpoint{4.033315in}{1.921416in}}{\pgfqpoint{4.041128in}{1.929230in}}%
\pgfpathcurveto{\pgfqpoint{4.048942in}{1.937043in}}{\pgfqpoint{4.053332in}{1.947642in}}{\pgfqpoint{4.053332in}{1.958692in}}%
\pgfpathcurveto{\pgfqpoint{4.053332in}{1.969743in}}{\pgfqpoint{4.048942in}{1.980342in}}{\pgfqpoint{4.041128in}{1.988155in}}%
\pgfpathcurveto{\pgfqpoint{4.033315in}{1.995969in}}{\pgfqpoint{4.022716in}{2.000359in}}{\pgfqpoint{4.011666in}{2.000359in}}%
\pgfpathcurveto{\pgfqpoint{4.000616in}{2.000359in}}{\pgfqpoint{3.990016in}{1.995969in}}{\pgfqpoint{3.982203in}{1.988155in}}%
\pgfpathcurveto{\pgfqpoint{3.974389in}{1.980342in}}{\pgfqpoint{3.969999in}{1.969743in}}{\pgfqpoint{3.969999in}{1.958692in}}%
\pgfpathcurveto{\pgfqpoint{3.969999in}{1.947642in}}{\pgfqpoint{3.974389in}{1.937043in}}{\pgfqpoint{3.982203in}{1.929230in}}%
\pgfpathcurveto{\pgfqpoint{3.990016in}{1.921416in}}{\pgfqpoint{4.000616in}{1.917026in}}{\pgfqpoint{4.011666in}{1.917026in}}%
\pgfpathclose%
\pgfusepath{stroke,fill}%
\end{pgfscope}%
\begin{pgfscope}%
\pgfpathrectangle{\pgfqpoint{0.800000in}{0.528000in}}{\pgfqpoint{4.960000in}{3.696000in}}%
\pgfusepath{clip}%
\pgfsetbuttcap%
\pgfsetroundjoin%
\definecolor{currentfill}{rgb}{0.000000,0.000000,0.000000}%
\pgfsetfillcolor{currentfill}%
\pgfsetlinewidth{1.003750pt}%
\definecolor{currentstroke}{rgb}{0.000000,0.000000,0.000000}%
\pgfsetstrokecolor{currentstroke}%
\pgfsetdash{}{0pt}%
\pgfpathmoveto{\pgfqpoint{4.011666in}{1.635466in}}%
\pgfpathcurveto{\pgfqpoint{4.022716in}{1.635466in}}{\pgfqpoint{4.033315in}{1.639856in}}{\pgfqpoint{4.041128in}{1.647670in}}%
\pgfpathcurveto{\pgfqpoint{4.048942in}{1.655483in}}{\pgfqpoint{4.053332in}{1.666082in}}{\pgfqpoint{4.053332in}{1.677132in}}%
\pgfpathcurveto{\pgfqpoint{4.053332in}{1.688182in}}{\pgfqpoint{4.048942in}{1.698781in}}{\pgfqpoint{4.041128in}{1.706595in}}%
\pgfpathcurveto{\pgfqpoint{4.033315in}{1.714409in}}{\pgfqpoint{4.022716in}{1.718799in}}{\pgfqpoint{4.011666in}{1.718799in}}%
\pgfpathcurveto{\pgfqpoint{4.000616in}{1.718799in}}{\pgfqpoint{3.990016in}{1.714409in}}{\pgfqpoint{3.982203in}{1.706595in}}%
\pgfpathcurveto{\pgfqpoint{3.974389in}{1.698781in}}{\pgfqpoint{3.969999in}{1.688182in}}{\pgfqpoint{3.969999in}{1.677132in}}%
\pgfpathcurveto{\pgfqpoint{3.969999in}{1.666082in}}{\pgfqpoint{3.974389in}{1.655483in}}{\pgfqpoint{3.982203in}{1.647670in}}%
\pgfpathcurveto{\pgfqpoint{3.990016in}{1.639856in}}{\pgfqpoint{4.000616in}{1.635466in}}{\pgfqpoint{4.011666in}{1.635466in}}%
\pgfpathclose%
\pgfusepath{stroke,fill}%
\end{pgfscope}%
\begin{pgfscope}%
\pgfpathrectangle{\pgfqpoint{0.800000in}{0.528000in}}{\pgfqpoint{4.960000in}{3.696000in}}%
\pgfusepath{clip}%
\pgfsetbuttcap%
\pgfsetroundjoin%
\definecolor{currentfill}{rgb}{0.000000,0.000000,0.000000}%
\pgfsetfillcolor{currentfill}%
\pgfsetlinewidth{1.003750pt}%
\definecolor{currentstroke}{rgb}{0.000000,0.000000,0.000000}%
\pgfsetstrokecolor{currentstroke}%
\pgfsetdash{}{0pt}%
\pgfpathmoveto{\pgfqpoint{4.011666in}{1.596630in}}%
\pgfpathcurveto{\pgfqpoint{4.022716in}{1.596630in}}{\pgfqpoint{4.033315in}{1.601020in}}{\pgfqpoint{4.041128in}{1.608834in}}%
\pgfpathcurveto{\pgfqpoint{4.048942in}{1.616647in}}{\pgfqpoint{4.053332in}{1.627246in}}{\pgfqpoint{4.053332in}{1.638296in}}%
\pgfpathcurveto{\pgfqpoint{4.053332in}{1.649347in}}{\pgfqpoint{4.048942in}{1.659946in}}{\pgfqpoint{4.041128in}{1.667759in}}%
\pgfpathcurveto{\pgfqpoint{4.033315in}{1.675573in}}{\pgfqpoint{4.022716in}{1.679963in}}{\pgfqpoint{4.011666in}{1.679963in}}%
\pgfpathcurveto{\pgfqpoint{4.000616in}{1.679963in}}{\pgfqpoint{3.990016in}{1.675573in}}{\pgfqpoint{3.982203in}{1.667759in}}%
\pgfpathcurveto{\pgfqpoint{3.974389in}{1.659946in}}{\pgfqpoint{3.969999in}{1.649347in}}{\pgfqpoint{3.969999in}{1.638296in}}%
\pgfpathcurveto{\pgfqpoint{3.969999in}{1.627246in}}{\pgfqpoint{3.974389in}{1.616647in}}{\pgfqpoint{3.982203in}{1.608834in}}%
\pgfpathcurveto{\pgfqpoint{3.990016in}{1.601020in}}{\pgfqpoint{4.000616in}{1.596630in}}{\pgfqpoint{4.011666in}{1.596630in}}%
\pgfpathclose%
\pgfusepath{stroke,fill}%
\end{pgfscope}%
\begin{pgfscope}%
\pgfpathrectangle{\pgfqpoint{0.800000in}{0.528000in}}{\pgfqpoint{4.960000in}{3.696000in}}%
\pgfusepath{clip}%
\pgfsetbuttcap%
\pgfsetroundjoin%
\definecolor{currentfill}{rgb}{0.000000,0.000000,0.000000}%
\pgfsetfillcolor{currentfill}%
\pgfsetlinewidth{1.003750pt}%
\definecolor{currentstroke}{rgb}{0.000000,0.000000,0.000000}%
\pgfsetstrokecolor{currentstroke}%
\pgfsetdash{}{0pt}%
\pgfpathmoveto{\pgfqpoint{4.011666in}{1.713137in}}%
\pgfpathcurveto{\pgfqpoint{4.022716in}{1.713137in}}{\pgfqpoint{4.033315in}{1.717528in}}{\pgfqpoint{4.041128in}{1.725341in}}%
\pgfpathcurveto{\pgfqpoint{4.048942in}{1.733155in}}{\pgfqpoint{4.053332in}{1.743754in}}{\pgfqpoint{4.053332in}{1.754804in}}%
\pgfpathcurveto{\pgfqpoint{4.053332in}{1.765854in}}{\pgfqpoint{4.048942in}{1.776453in}}{\pgfqpoint{4.041128in}{1.784267in}}%
\pgfpathcurveto{\pgfqpoint{4.033315in}{1.792080in}}{\pgfqpoint{4.022716in}{1.796471in}}{\pgfqpoint{4.011666in}{1.796471in}}%
\pgfpathcurveto{\pgfqpoint{4.000616in}{1.796471in}}{\pgfqpoint{3.990016in}{1.792080in}}{\pgfqpoint{3.982203in}{1.784267in}}%
\pgfpathcurveto{\pgfqpoint{3.974389in}{1.776453in}}{\pgfqpoint{3.969999in}{1.765854in}}{\pgfqpoint{3.969999in}{1.754804in}}%
\pgfpathcurveto{\pgfqpoint{3.969999in}{1.743754in}}{\pgfqpoint{3.974389in}{1.733155in}}{\pgfqpoint{3.982203in}{1.725341in}}%
\pgfpathcurveto{\pgfqpoint{3.990016in}{1.717528in}}{\pgfqpoint{4.000616in}{1.713137in}}{\pgfqpoint{4.011666in}{1.713137in}}%
\pgfpathclose%
\pgfusepath{stroke,fill}%
\end{pgfscope}%
\begin{pgfscope}%
\pgfpathrectangle{\pgfqpoint{0.800000in}{0.528000in}}{\pgfqpoint{4.960000in}{3.696000in}}%
\pgfusepath{clip}%
\pgfsetbuttcap%
\pgfsetroundjoin%
\definecolor{currentfill}{rgb}{0.000000,0.000000,0.000000}%
\pgfsetfillcolor{currentfill}%
\pgfsetlinewidth{1.003750pt}%
\definecolor{currentstroke}{rgb}{0.000000,0.000000,0.000000}%
\pgfsetstrokecolor{currentstroke}%
\pgfsetdash{}{0pt}%
\pgfpathmoveto{\pgfqpoint{4.011666in}{1.616048in}}%
\pgfpathcurveto{\pgfqpoint{4.022716in}{1.616048in}}{\pgfqpoint{4.033315in}{1.620438in}}{\pgfqpoint{4.041128in}{1.628252in}}%
\pgfpathcurveto{\pgfqpoint{4.048942in}{1.636065in}}{\pgfqpoint{4.053332in}{1.646664in}}{\pgfqpoint{4.053332in}{1.657714in}}%
\pgfpathcurveto{\pgfqpoint{4.053332in}{1.668764in}}{\pgfqpoint{4.048942in}{1.679364in}}{\pgfqpoint{4.041128in}{1.687177in}}%
\pgfpathcurveto{\pgfqpoint{4.033315in}{1.694991in}}{\pgfqpoint{4.022716in}{1.699381in}}{\pgfqpoint{4.011666in}{1.699381in}}%
\pgfpathcurveto{\pgfqpoint{4.000616in}{1.699381in}}{\pgfqpoint{3.990016in}{1.694991in}}{\pgfqpoint{3.982203in}{1.687177in}}%
\pgfpathcurveto{\pgfqpoint{3.974389in}{1.679364in}}{\pgfqpoint{3.969999in}{1.668764in}}{\pgfqpoint{3.969999in}{1.657714in}}%
\pgfpathcurveto{\pgfqpoint{3.969999in}{1.646664in}}{\pgfqpoint{3.974389in}{1.636065in}}{\pgfqpoint{3.982203in}{1.628252in}}%
\pgfpathcurveto{\pgfqpoint{3.990016in}{1.620438in}}{\pgfqpoint{4.000616in}{1.616048in}}{\pgfqpoint{4.011666in}{1.616048in}}%
\pgfpathclose%
\pgfusepath{stroke,fill}%
\end{pgfscope}%
\begin{pgfscope}%
\pgfpathrectangle{\pgfqpoint{0.800000in}{0.528000in}}{\pgfqpoint{4.960000in}{3.696000in}}%
\pgfusepath{clip}%
\pgfsetbuttcap%
\pgfsetroundjoin%
\definecolor{currentfill}{rgb}{0.000000,0.000000,0.000000}%
\pgfsetfillcolor{currentfill}%
\pgfsetlinewidth{1.003750pt}%
\definecolor{currentstroke}{rgb}{0.000000,0.000000,0.000000}%
\pgfsetstrokecolor{currentstroke}%
\pgfsetdash{}{0pt}%
\pgfpathmoveto{\pgfqpoint{4.011666in}{1.538376in}}%
\pgfpathcurveto{\pgfqpoint{4.022716in}{1.538376in}}{\pgfqpoint{4.033315in}{1.542766in}}{\pgfqpoint{4.041128in}{1.550580in}}%
\pgfpathcurveto{\pgfqpoint{4.048942in}{1.558393in}}{\pgfqpoint{4.053332in}{1.568992in}}{\pgfqpoint{4.053332in}{1.580043in}}%
\pgfpathcurveto{\pgfqpoint{4.053332in}{1.591093in}}{\pgfqpoint{4.048942in}{1.601692in}}{\pgfqpoint{4.041128in}{1.609505in}}%
\pgfpathcurveto{\pgfqpoint{4.033315in}{1.617319in}}{\pgfqpoint{4.022716in}{1.621709in}}{\pgfqpoint{4.011666in}{1.621709in}}%
\pgfpathcurveto{\pgfqpoint{4.000616in}{1.621709in}}{\pgfqpoint{3.990016in}{1.617319in}}{\pgfqpoint{3.982203in}{1.609505in}}%
\pgfpathcurveto{\pgfqpoint{3.974389in}{1.601692in}}{\pgfqpoint{3.969999in}{1.591093in}}{\pgfqpoint{3.969999in}{1.580043in}}%
\pgfpathcurveto{\pgfqpoint{3.969999in}{1.568992in}}{\pgfqpoint{3.974389in}{1.558393in}}{\pgfqpoint{3.982203in}{1.550580in}}%
\pgfpathcurveto{\pgfqpoint{3.990016in}{1.542766in}}{\pgfqpoint{4.000616in}{1.538376in}}{\pgfqpoint{4.011666in}{1.538376in}}%
\pgfpathclose%
\pgfusepath{stroke,fill}%
\end{pgfscope}%
\begin{pgfscope}%
\pgfpathrectangle{\pgfqpoint{0.800000in}{0.528000in}}{\pgfqpoint{4.960000in}{3.696000in}}%
\pgfusepath{clip}%
\pgfsetbuttcap%
\pgfsetroundjoin%
\definecolor{currentfill}{rgb}{0.000000,0.000000,0.000000}%
\pgfsetfillcolor{currentfill}%
\pgfsetlinewidth{1.003750pt}%
\definecolor{currentstroke}{rgb}{0.000000,0.000000,0.000000}%
\pgfsetstrokecolor{currentstroke}%
\pgfsetdash{}{0pt}%
\pgfpathmoveto{\pgfqpoint{5.504545in}{3.227737in}}%
\pgfpathcurveto{\pgfqpoint{5.515596in}{3.227737in}}{\pgfqpoint{5.526195in}{3.232127in}}{\pgfqpoint{5.534008in}{3.239940in}}%
\pgfpathcurveto{\pgfqpoint{5.541822in}{3.247754in}}{\pgfqpoint{5.546212in}{3.258353in}}{\pgfqpoint{5.546212in}{3.269403in}}%
\pgfpathcurveto{\pgfqpoint{5.546212in}{3.280453in}}{\pgfqpoint{5.541822in}{3.291052in}}{\pgfqpoint{5.534008in}{3.298866in}}%
\pgfpathcurveto{\pgfqpoint{5.526195in}{3.306680in}}{\pgfqpoint{5.515596in}{3.311070in}}{\pgfqpoint{5.504545in}{3.311070in}}%
\pgfpathcurveto{\pgfqpoint{5.493495in}{3.311070in}}{\pgfqpoint{5.482896in}{3.306680in}}{\pgfqpoint{5.475083in}{3.298866in}}%
\pgfpathcurveto{\pgfqpoint{5.467269in}{3.291052in}}{\pgfqpoint{5.462879in}{3.280453in}}{\pgfqpoint{5.462879in}{3.269403in}}%
\pgfpathcurveto{\pgfqpoint{5.462879in}{3.258353in}}{\pgfqpoint{5.467269in}{3.247754in}}{\pgfqpoint{5.475083in}{3.239940in}}%
\pgfpathcurveto{\pgfqpoint{5.482896in}{3.232127in}}{\pgfqpoint{5.493495in}{3.227737in}}{\pgfqpoint{5.504545in}{3.227737in}}%
\pgfpathclose%
\pgfusepath{stroke,fill}%
\end{pgfscope}%
\begin{pgfscope}%
\pgfpathrectangle{\pgfqpoint{0.800000in}{0.528000in}}{\pgfqpoint{4.960000in}{3.696000in}}%
\pgfusepath{clip}%
\pgfsetbuttcap%
\pgfsetroundjoin%
\definecolor{currentfill}{rgb}{0.000000,0.000000,0.000000}%
\pgfsetfillcolor{currentfill}%
\pgfsetlinewidth{1.003750pt}%
\definecolor{currentstroke}{rgb}{0.000000,0.000000,0.000000}%
\pgfsetstrokecolor{currentstroke}%
\pgfsetdash{}{0pt}%
\pgfpathmoveto{\pgfqpoint{5.504545in}{3.140356in}}%
\pgfpathcurveto{\pgfqpoint{5.515596in}{3.140356in}}{\pgfqpoint{5.526195in}{3.144746in}}{\pgfqpoint{5.534008in}{3.152560in}}%
\pgfpathcurveto{\pgfqpoint{5.541822in}{3.160373in}}{\pgfqpoint{5.546212in}{3.170972in}}{\pgfqpoint{5.546212in}{3.182022in}}%
\pgfpathcurveto{\pgfqpoint{5.546212in}{3.193073in}}{\pgfqpoint{5.541822in}{3.203672in}}{\pgfqpoint{5.534008in}{3.211485in}}%
\pgfpathcurveto{\pgfqpoint{5.526195in}{3.219299in}}{\pgfqpoint{5.515596in}{3.223689in}}{\pgfqpoint{5.504545in}{3.223689in}}%
\pgfpathcurveto{\pgfqpoint{5.493495in}{3.223689in}}{\pgfqpoint{5.482896in}{3.219299in}}{\pgfqpoint{5.475083in}{3.211485in}}%
\pgfpathcurveto{\pgfqpoint{5.467269in}{3.203672in}}{\pgfqpoint{5.462879in}{3.193073in}}{\pgfqpoint{5.462879in}{3.182022in}}%
\pgfpathcurveto{\pgfqpoint{5.462879in}{3.170972in}}{\pgfqpoint{5.467269in}{3.160373in}}{\pgfqpoint{5.475083in}{3.152560in}}%
\pgfpathcurveto{\pgfqpoint{5.482896in}{3.144746in}}{\pgfqpoint{5.493495in}{3.140356in}}{\pgfqpoint{5.504545in}{3.140356in}}%
\pgfpathclose%
\pgfusepath{stroke,fill}%
\end{pgfscope}%
\begin{pgfscope}%
\pgfpathrectangle{\pgfqpoint{0.800000in}{0.528000in}}{\pgfqpoint{4.960000in}{3.696000in}}%
\pgfusepath{clip}%
\pgfsetbuttcap%
\pgfsetroundjoin%
\definecolor{currentfill}{rgb}{0.000000,0.000000,0.000000}%
\pgfsetfillcolor{currentfill}%
\pgfsetlinewidth{1.003750pt}%
\definecolor{currentstroke}{rgb}{0.000000,0.000000,0.000000}%
\pgfsetstrokecolor{currentstroke}%
\pgfsetdash{}{0pt}%
\pgfpathmoveto{\pgfqpoint{5.504545in}{3.014139in}}%
\pgfpathcurveto{\pgfqpoint{5.515596in}{3.014139in}}{\pgfqpoint{5.526195in}{3.018529in}}{\pgfqpoint{5.534008in}{3.026343in}}%
\pgfpathcurveto{\pgfqpoint{5.541822in}{3.034157in}}{\pgfqpoint{5.546212in}{3.044756in}}{\pgfqpoint{5.546212in}{3.055806in}}%
\pgfpathcurveto{\pgfqpoint{5.546212in}{3.066856in}}{\pgfqpoint{5.541822in}{3.077455in}}{\pgfqpoint{5.534008in}{3.085269in}}%
\pgfpathcurveto{\pgfqpoint{5.526195in}{3.093082in}}{\pgfqpoint{5.515596in}{3.097473in}}{\pgfqpoint{5.504545in}{3.097473in}}%
\pgfpathcurveto{\pgfqpoint{5.493495in}{3.097473in}}{\pgfqpoint{5.482896in}{3.093082in}}{\pgfqpoint{5.475083in}{3.085269in}}%
\pgfpathcurveto{\pgfqpoint{5.467269in}{3.077455in}}{\pgfqpoint{5.462879in}{3.066856in}}{\pgfqpoint{5.462879in}{3.055806in}}%
\pgfpathcurveto{\pgfqpoint{5.462879in}{3.044756in}}{\pgfqpoint{5.467269in}{3.034157in}}{\pgfqpoint{5.475083in}{3.026343in}}%
\pgfpathcurveto{\pgfqpoint{5.482896in}{3.018529in}}{\pgfqpoint{5.493495in}{3.014139in}}{\pgfqpoint{5.504545in}{3.014139in}}%
\pgfpathclose%
\pgfusepath{stroke,fill}%
\end{pgfscope}%
\begin{pgfscope}%
\pgfpathrectangle{\pgfqpoint{0.800000in}{0.528000in}}{\pgfqpoint{4.960000in}{3.696000in}}%
\pgfusepath{clip}%
\pgfsetbuttcap%
\pgfsetroundjoin%
\definecolor{currentfill}{rgb}{0.000000,0.000000,0.000000}%
\pgfsetfillcolor{currentfill}%
\pgfsetlinewidth{1.003750pt}%
\definecolor{currentstroke}{rgb}{0.000000,0.000000,0.000000}%
\pgfsetstrokecolor{currentstroke}%
\pgfsetdash{}{0pt}%
\pgfpathmoveto{\pgfqpoint{5.504545in}{3.150065in}}%
\pgfpathcurveto{\pgfqpoint{5.515596in}{3.150065in}}{\pgfqpoint{5.526195in}{3.154455in}}{\pgfqpoint{5.534008in}{3.162269in}}%
\pgfpathcurveto{\pgfqpoint{5.541822in}{3.170082in}}{\pgfqpoint{5.546212in}{3.180681in}}{\pgfqpoint{5.546212in}{3.191731in}}%
\pgfpathcurveto{\pgfqpoint{5.546212in}{3.202782in}}{\pgfqpoint{5.541822in}{3.213381in}}{\pgfqpoint{5.534008in}{3.221194in}}%
\pgfpathcurveto{\pgfqpoint{5.526195in}{3.229008in}}{\pgfqpoint{5.515596in}{3.233398in}}{\pgfqpoint{5.504545in}{3.233398in}}%
\pgfpathcurveto{\pgfqpoint{5.493495in}{3.233398in}}{\pgfqpoint{5.482896in}{3.229008in}}{\pgfqpoint{5.475083in}{3.221194in}}%
\pgfpathcurveto{\pgfqpoint{5.467269in}{3.213381in}}{\pgfqpoint{5.462879in}{3.202782in}}{\pgfqpoint{5.462879in}{3.191731in}}%
\pgfpathcurveto{\pgfqpoint{5.462879in}{3.180681in}}{\pgfqpoint{5.467269in}{3.170082in}}{\pgfqpoint{5.475083in}{3.162269in}}%
\pgfpathcurveto{\pgfqpoint{5.482896in}{3.154455in}}{\pgfqpoint{5.493495in}{3.150065in}}{\pgfqpoint{5.504545in}{3.150065in}}%
\pgfpathclose%
\pgfusepath{stroke,fill}%
\end{pgfscope}%
\begin{pgfscope}%
\pgfpathrectangle{\pgfqpoint{0.800000in}{0.528000in}}{\pgfqpoint{4.960000in}{3.696000in}}%
\pgfusepath{clip}%
\pgfsetbuttcap%
\pgfsetroundjoin%
\definecolor{currentfill}{rgb}{0.000000,0.000000,0.000000}%
\pgfsetfillcolor{currentfill}%
\pgfsetlinewidth{1.003750pt}%
\definecolor{currentstroke}{rgb}{0.000000,0.000000,0.000000}%
\pgfsetstrokecolor{currentstroke}%
\pgfsetdash{}{0pt}%
\pgfpathmoveto{\pgfqpoint{5.504545in}{3.150065in}}%
\pgfpathcurveto{\pgfqpoint{5.515596in}{3.150065in}}{\pgfqpoint{5.526195in}{3.154455in}}{\pgfqpoint{5.534008in}{3.162269in}}%
\pgfpathcurveto{\pgfqpoint{5.541822in}{3.170082in}}{\pgfqpoint{5.546212in}{3.180681in}}{\pgfqpoint{5.546212in}{3.191731in}}%
\pgfpathcurveto{\pgfqpoint{5.546212in}{3.202782in}}{\pgfqpoint{5.541822in}{3.213381in}}{\pgfqpoint{5.534008in}{3.221194in}}%
\pgfpathcurveto{\pgfqpoint{5.526195in}{3.229008in}}{\pgfqpoint{5.515596in}{3.233398in}}{\pgfqpoint{5.504545in}{3.233398in}}%
\pgfpathcurveto{\pgfqpoint{5.493495in}{3.233398in}}{\pgfqpoint{5.482896in}{3.229008in}}{\pgfqpoint{5.475083in}{3.221194in}}%
\pgfpathcurveto{\pgfqpoint{5.467269in}{3.213381in}}{\pgfqpoint{5.462879in}{3.202782in}}{\pgfqpoint{5.462879in}{3.191731in}}%
\pgfpathcurveto{\pgfqpoint{5.462879in}{3.180681in}}{\pgfqpoint{5.467269in}{3.170082in}}{\pgfqpoint{5.475083in}{3.162269in}}%
\pgfpathcurveto{\pgfqpoint{5.482896in}{3.154455in}}{\pgfqpoint{5.493495in}{3.150065in}}{\pgfqpoint{5.504545in}{3.150065in}}%
\pgfpathclose%
\pgfusepath{stroke,fill}%
\end{pgfscope}%
\begin{pgfscope}%
\pgfpathrectangle{\pgfqpoint{0.800000in}{0.528000in}}{\pgfqpoint{4.960000in}{3.696000in}}%
\pgfusepath{clip}%
\pgfsetbuttcap%
\pgfsetroundjoin%
\definecolor{currentfill}{rgb}{0.000000,0.000000,0.000000}%
\pgfsetfillcolor{currentfill}%
\pgfsetlinewidth{1.003750pt}%
\definecolor{currentstroke}{rgb}{0.000000,0.000000,0.000000}%
\pgfsetstrokecolor{currentstroke}%
\pgfsetdash{}{0pt}%
\pgfpathmoveto{\pgfqpoint{5.504545in}{3.188901in}}%
\pgfpathcurveto{\pgfqpoint{5.515596in}{3.188901in}}{\pgfqpoint{5.526195in}{3.193291in}}{\pgfqpoint{5.534008in}{3.201105in}}%
\pgfpathcurveto{\pgfqpoint{5.541822in}{3.208918in}}{\pgfqpoint{5.546212in}{3.219517in}}{\pgfqpoint{5.546212in}{3.230567in}}%
\pgfpathcurveto{\pgfqpoint{5.546212in}{3.241617in}}{\pgfqpoint{5.541822in}{3.252217in}}{\pgfqpoint{5.534008in}{3.260030in}}%
\pgfpathcurveto{\pgfqpoint{5.526195in}{3.267844in}}{\pgfqpoint{5.515596in}{3.272234in}}{\pgfqpoint{5.504545in}{3.272234in}}%
\pgfpathcurveto{\pgfqpoint{5.493495in}{3.272234in}}{\pgfqpoint{5.482896in}{3.267844in}}{\pgfqpoint{5.475083in}{3.260030in}}%
\pgfpathcurveto{\pgfqpoint{5.467269in}{3.252217in}}{\pgfqpoint{5.462879in}{3.241617in}}{\pgfqpoint{5.462879in}{3.230567in}}%
\pgfpathcurveto{\pgfqpoint{5.462879in}{3.219517in}}{\pgfqpoint{5.467269in}{3.208918in}}{\pgfqpoint{5.475083in}{3.201105in}}%
\pgfpathcurveto{\pgfqpoint{5.482896in}{3.193291in}}{\pgfqpoint{5.493495in}{3.188901in}}{\pgfqpoint{5.504545in}{3.188901in}}%
\pgfpathclose%
\pgfusepath{stroke,fill}%
\end{pgfscope}%
\begin{pgfscope}%
\pgfpathrectangle{\pgfqpoint{0.800000in}{0.528000in}}{\pgfqpoint{4.960000in}{3.696000in}}%
\pgfusepath{clip}%
\pgfsetbuttcap%
\pgfsetroundjoin%
\definecolor{currentfill}{rgb}{0.000000,0.000000,0.000000}%
\pgfsetfillcolor{currentfill}%
\pgfsetlinewidth{1.003750pt}%
\definecolor{currentstroke}{rgb}{0.000000,0.000000,0.000000}%
\pgfsetstrokecolor{currentstroke}%
\pgfsetdash{}{0pt}%
\pgfpathmoveto{\pgfqpoint{5.504545in}{3.519006in}}%
\pgfpathcurveto{\pgfqpoint{5.515596in}{3.519006in}}{\pgfqpoint{5.526195in}{3.523396in}}{\pgfqpoint{5.534008in}{3.531210in}}%
\pgfpathcurveto{\pgfqpoint{5.541822in}{3.539023in}}{\pgfqpoint{5.546212in}{3.549622in}}{\pgfqpoint{5.546212in}{3.560672in}}%
\pgfpathcurveto{\pgfqpoint{5.546212in}{3.571722in}}{\pgfqpoint{5.541822in}{3.582321in}}{\pgfqpoint{5.534008in}{3.590135in}}%
\pgfpathcurveto{\pgfqpoint{5.526195in}{3.597949in}}{\pgfqpoint{5.515596in}{3.602339in}}{\pgfqpoint{5.504545in}{3.602339in}}%
\pgfpathcurveto{\pgfqpoint{5.493495in}{3.602339in}}{\pgfqpoint{5.482896in}{3.597949in}}{\pgfqpoint{5.475083in}{3.590135in}}%
\pgfpathcurveto{\pgfqpoint{5.467269in}{3.582321in}}{\pgfqpoint{5.462879in}{3.571722in}}{\pgfqpoint{5.462879in}{3.560672in}}%
\pgfpathcurveto{\pgfqpoint{5.462879in}{3.549622in}}{\pgfqpoint{5.467269in}{3.539023in}}{\pgfqpoint{5.475083in}{3.531210in}}%
\pgfpathcurveto{\pgfqpoint{5.482896in}{3.523396in}}{\pgfqpoint{5.493495in}{3.519006in}}{\pgfqpoint{5.504545in}{3.519006in}}%
\pgfpathclose%
\pgfusepath{stroke,fill}%
\end{pgfscope}%
\begin{pgfscope}%
\pgfpathrectangle{\pgfqpoint{0.800000in}{0.528000in}}{\pgfqpoint{4.960000in}{3.696000in}}%
\pgfusepath{clip}%
\pgfsetbuttcap%
\pgfsetroundjoin%
\definecolor{currentfill}{rgb}{0.000000,0.000000,0.000000}%
\pgfsetfillcolor{currentfill}%
\pgfsetlinewidth{1.003750pt}%
\definecolor{currentstroke}{rgb}{0.000000,0.000000,0.000000}%
\pgfsetstrokecolor{currentstroke}%
\pgfsetdash{}{0pt}%
\pgfpathmoveto{\pgfqpoint{5.504545in}{3.169483in}}%
\pgfpathcurveto{\pgfqpoint{5.515596in}{3.169483in}}{\pgfqpoint{5.526195in}{3.173873in}}{\pgfqpoint{5.534008in}{3.181687in}}%
\pgfpathcurveto{\pgfqpoint{5.541822in}{3.189500in}}{\pgfqpoint{5.546212in}{3.200099in}}{\pgfqpoint{5.546212in}{3.211149in}}%
\pgfpathcurveto{\pgfqpoint{5.546212in}{3.222200in}}{\pgfqpoint{5.541822in}{3.232799in}}{\pgfqpoint{5.534008in}{3.240612in}}%
\pgfpathcurveto{\pgfqpoint{5.526195in}{3.248426in}}{\pgfqpoint{5.515596in}{3.252816in}}{\pgfqpoint{5.504545in}{3.252816in}}%
\pgfpathcurveto{\pgfqpoint{5.493495in}{3.252816in}}{\pgfqpoint{5.482896in}{3.248426in}}{\pgfqpoint{5.475083in}{3.240612in}}%
\pgfpathcurveto{\pgfqpoint{5.467269in}{3.232799in}}{\pgfqpoint{5.462879in}{3.222200in}}{\pgfqpoint{5.462879in}{3.211149in}}%
\pgfpathcurveto{\pgfqpoint{5.462879in}{3.200099in}}{\pgfqpoint{5.467269in}{3.189500in}}{\pgfqpoint{5.475083in}{3.181687in}}%
\pgfpathcurveto{\pgfqpoint{5.482896in}{3.173873in}}{\pgfqpoint{5.493495in}{3.169483in}}{\pgfqpoint{5.504545in}{3.169483in}}%
\pgfpathclose%
\pgfusepath{stroke,fill}%
\end{pgfscope}%
\begin{pgfscope}%
\pgfpathrectangle{\pgfqpoint{0.800000in}{0.528000in}}{\pgfqpoint{4.960000in}{3.696000in}}%
\pgfusepath{clip}%
\pgfsetbuttcap%
\pgfsetroundjoin%
\definecolor{currentfill}{rgb}{0.000000,0.000000,0.000000}%
\pgfsetfillcolor{currentfill}%
\pgfsetlinewidth{1.003750pt}%
\definecolor{currentstroke}{rgb}{0.000000,0.000000,0.000000}%
\pgfsetstrokecolor{currentstroke}%
\pgfsetdash{}{0pt}%
\pgfpathmoveto{\pgfqpoint{5.504545in}{3.043266in}}%
\pgfpathcurveto{\pgfqpoint{5.515596in}{3.043266in}}{\pgfqpoint{5.526195in}{3.047656in}}{\pgfqpoint{5.534008in}{3.055470in}}%
\pgfpathcurveto{\pgfqpoint{5.541822in}{3.063284in}}{\pgfqpoint{5.546212in}{3.073883in}}{\pgfqpoint{5.546212in}{3.084933in}}%
\pgfpathcurveto{\pgfqpoint{5.546212in}{3.095983in}}{\pgfqpoint{5.541822in}{3.106582in}}{\pgfqpoint{5.534008in}{3.114396in}}%
\pgfpathcurveto{\pgfqpoint{5.526195in}{3.122209in}}{\pgfqpoint{5.515596in}{3.126599in}}{\pgfqpoint{5.504545in}{3.126599in}}%
\pgfpathcurveto{\pgfqpoint{5.493495in}{3.126599in}}{\pgfqpoint{5.482896in}{3.122209in}}{\pgfqpoint{5.475083in}{3.114396in}}%
\pgfpathcurveto{\pgfqpoint{5.467269in}{3.106582in}}{\pgfqpoint{5.462879in}{3.095983in}}{\pgfqpoint{5.462879in}{3.084933in}}%
\pgfpathcurveto{\pgfqpoint{5.462879in}{3.073883in}}{\pgfqpoint{5.467269in}{3.063284in}}{\pgfqpoint{5.475083in}{3.055470in}}%
\pgfpathcurveto{\pgfqpoint{5.482896in}{3.047656in}}{\pgfqpoint{5.493495in}{3.043266in}}{\pgfqpoint{5.504545in}{3.043266in}}%
\pgfpathclose%
\pgfusepath{stroke,fill}%
\end{pgfscope}%
\begin{pgfscope}%
\pgfpathrectangle{\pgfqpoint{0.800000in}{0.528000in}}{\pgfqpoint{4.960000in}{3.696000in}}%
\pgfusepath{clip}%
\pgfsetbuttcap%
\pgfsetroundjoin%
\definecolor{currentfill}{rgb}{0.000000,0.000000,0.000000}%
\pgfsetfillcolor{currentfill}%
\pgfsetlinewidth{1.003750pt}%
\definecolor{currentstroke}{rgb}{0.000000,0.000000,0.000000}%
\pgfsetstrokecolor{currentstroke}%
\pgfsetdash{}{0pt}%
\pgfpathmoveto{\pgfqpoint{5.504545in}{3.227737in}}%
\pgfpathcurveto{\pgfqpoint{5.515596in}{3.227737in}}{\pgfqpoint{5.526195in}{3.232127in}}{\pgfqpoint{5.534008in}{3.239940in}}%
\pgfpathcurveto{\pgfqpoint{5.541822in}{3.247754in}}{\pgfqpoint{5.546212in}{3.258353in}}{\pgfqpoint{5.546212in}{3.269403in}}%
\pgfpathcurveto{\pgfqpoint{5.546212in}{3.280453in}}{\pgfqpoint{5.541822in}{3.291052in}}{\pgfqpoint{5.534008in}{3.298866in}}%
\pgfpathcurveto{\pgfqpoint{5.526195in}{3.306680in}}{\pgfqpoint{5.515596in}{3.311070in}}{\pgfqpoint{5.504545in}{3.311070in}}%
\pgfpathcurveto{\pgfqpoint{5.493495in}{3.311070in}}{\pgfqpoint{5.482896in}{3.306680in}}{\pgfqpoint{5.475083in}{3.298866in}}%
\pgfpathcurveto{\pgfqpoint{5.467269in}{3.291052in}}{\pgfqpoint{5.462879in}{3.280453in}}{\pgfqpoint{5.462879in}{3.269403in}}%
\pgfpathcurveto{\pgfqpoint{5.462879in}{3.258353in}}{\pgfqpoint{5.467269in}{3.247754in}}{\pgfqpoint{5.475083in}{3.239940in}}%
\pgfpathcurveto{\pgfqpoint{5.482896in}{3.232127in}}{\pgfqpoint{5.493495in}{3.227737in}}{\pgfqpoint{5.504545in}{3.227737in}}%
\pgfpathclose%
\pgfusepath{stroke,fill}%
\end{pgfscope}%
\begin{pgfscope}%
\pgfpathrectangle{\pgfqpoint{0.800000in}{0.528000in}}{\pgfqpoint{4.960000in}{3.696000in}}%
\pgfusepath{clip}%
\pgfsetbuttcap%
\pgfsetroundjoin%
\definecolor{currentfill}{rgb}{0.000000,0.000000,0.000000}%
\pgfsetfillcolor{currentfill}%
\pgfsetlinewidth{1.003750pt}%
\definecolor{currentstroke}{rgb}{0.000000,0.000000,0.000000}%
\pgfsetstrokecolor{currentstroke}%
\pgfsetdash{}{0pt}%
\pgfpathmoveto{\pgfqpoint{5.504545in}{3.334535in}}%
\pgfpathcurveto{\pgfqpoint{5.515596in}{3.334535in}}{\pgfqpoint{5.526195in}{3.338925in}}{\pgfqpoint{5.534008in}{3.346739in}}%
\pgfpathcurveto{\pgfqpoint{5.541822in}{3.354553in}}{\pgfqpoint{5.546212in}{3.365152in}}{\pgfqpoint{5.546212in}{3.376202in}}%
\pgfpathcurveto{\pgfqpoint{5.546212in}{3.387252in}}{\pgfqpoint{5.541822in}{3.397851in}}{\pgfqpoint{5.534008in}{3.405665in}}%
\pgfpathcurveto{\pgfqpoint{5.526195in}{3.413478in}}{\pgfqpoint{5.515596in}{3.417869in}}{\pgfqpoint{5.504545in}{3.417869in}}%
\pgfpathcurveto{\pgfqpoint{5.493495in}{3.417869in}}{\pgfqpoint{5.482896in}{3.413478in}}{\pgfqpoint{5.475083in}{3.405665in}}%
\pgfpathcurveto{\pgfqpoint{5.467269in}{3.397851in}}{\pgfqpoint{5.462879in}{3.387252in}}{\pgfqpoint{5.462879in}{3.376202in}}%
\pgfpathcurveto{\pgfqpoint{5.462879in}{3.365152in}}{\pgfqpoint{5.467269in}{3.354553in}}{\pgfqpoint{5.475083in}{3.346739in}}%
\pgfpathcurveto{\pgfqpoint{5.482896in}{3.338925in}}{\pgfqpoint{5.493495in}{3.334535in}}{\pgfqpoint{5.504545in}{3.334535in}}%
\pgfpathclose%
\pgfusepath{stroke,fill}%
\end{pgfscope}%
\begin{pgfscope}%
\pgfpathrectangle{\pgfqpoint{0.800000in}{0.528000in}}{\pgfqpoint{4.960000in}{3.696000in}}%
\pgfusepath{clip}%
\pgfsetbuttcap%
\pgfsetroundjoin%
\definecolor{currentfill}{rgb}{0.000000,0.000000,0.000000}%
\pgfsetfillcolor{currentfill}%
\pgfsetlinewidth{1.003750pt}%
\definecolor{currentstroke}{rgb}{0.000000,0.000000,0.000000}%
\pgfsetstrokecolor{currentstroke}%
\pgfsetdash{}{0pt}%
\pgfpathmoveto{\pgfqpoint{5.504545in}{2.975303in}}%
\pgfpathcurveto{\pgfqpoint{5.515596in}{2.975303in}}{\pgfqpoint{5.526195in}{2.979694in}}{\pgfqpoint{5.534008in}{2.987507in}}%
\pgfpathcurveto{\pgfqpoint{5.541822in}{2.995321in}}{\pgfqpoint{5.546212in}{3.005920in}}{\pgfqpoint{5.546212in}{3.016970in}}%
\pgfpathcurveto{\pgfqpoint{5.546212in}{3.028020in}}{\pgfqpoint{5.541822in}{3.038619in}}{\pgfqpoint{5.534008in}{3.046433in}}%
\pgfpathcurveto{\pgfqpoint{5.526195in}{3.054246in}}{\pgfqpoint{5.515596in}{3.058637in}}{\pgfqpoint{5.504545in}{3.058637in}}%
\pgfpathcurveto{\pgfqpoint{5.493495in}{3.058637in}}{\pgfqpoint{5.482896in}{3.054246in}}{\pgfqpoint{5.475083in}{3.046433in}}%
\pgfpathcurveto{\pgfqpoint{5.467269in}{3.038619in}}{\pgfqpoint{5.462879in}{3.028020in}}{\pgfqpoint{5.462879in}{3.016970in}}%
\pgfpathcurveto{\pgfqpoint{5.462879in}{3.005920in}}{\pgfqpoint{5.467269in}{2.995321in}}{\pgfqpoint{5.475083in}{2.987507in}}%
\pgfpathcurveto{\pgfqpoint{5.482896in}{2.979694in}}{\pgfqpoint{5.493495in}{2.975303in}}{\pgfqpoint{5.504545in}{2.975303in}}%
\pgfpathclose%
\pgfusepath{stroke,fill}%
\end{pgfscope}%
\begin{pgfscope}%
\pgfpathrectangle{\pgfqpoint{0.800000in}{0.528000in}}{\pgfqpoint{4.960000in}{3.696000in}}%
\pgfusepath{clip}%
\pgfsetbuttcap%
\pgfsetroundjoin%
\definecolor{currentfill}{rgb}{0.000000,0.000000,0.000000}%
\pgfsetfillcolor{currentfill}%
\pgfsetlinewidth{1.003750pt}%
\definecolor{currentstroke}{rgb}{0.000000,0.000000,0.000000}%
\pgfsetstrokecolor{currentstroke}%
\pgfsetdash{}{0pt}%
\pgfpathmoveto{\pgfqpoint{5.504545in}{3.402498in}}%
\pgfpathcurveto{\pgfqpoint{5.515596in}{3.402498in}}{\pgfqpoint{5.526195in}{3.406888in}}{\pgfqpoint{5.534008in}{3.414702in}}%
\pgfpathcurveto{\pgfqpoint{5.541822in}{3.422516in}}{\pgfqpoint{5.546212in}{3.433115in}}{\pgfqpoint{5.546212in}{3.444165in}}%
\pgfpathcurveto{\pgfqpoint{5.546212in}{3.455215in}}{\pgfqpoint{5.541822in}{3.465814in}}{\pgfqpoint{5.534008in}{3.473627in}}%
\pgfpathcurveto{\pgfqpoint{5.526195in}{3.481441in}}{\pgfqpoint{5.515596in}{3.485831in}}{\pgfqpoint{5.504545in}{3.485831in}}%
\pgfpathcurveto{\pgfqpoint{5.493495in}{3.485831in}}{\pgfqpoint{5.482896in}{3.481441in}}{\pgfqpoint{5.475083in}{3.473627in}}%
\pgfpathcurveto{\pgfqpoint{5.467269in}{3.465814in}}{\pgfqpoint{5.462879in}{3.455215in}}{\pgfqpoint{5.462879in}{3.444165in}}%
\pgfpathcurveto{\pgfqpoint{5.462879in}{3.433115in}}{\pgfqpoint{5.467269in}{3.422516in}}{\pgfqpoint{5.475083in}{3.414702in}}%
\pgfpathcurveto{\pgfqpoint{5.482896in}{3.406888in}}{\pgfqpoint{5.493495in}{3.402498in}}{\pgfqpoint{5.504545in}{3.402498in}}%
\pgfpathclose%
\pgfusepath{stroke,fill}%
\end{pgfscope}%
\begin{pgfscope}%
\pgfpathrectangle{\pgfqpoint{0.800000in}{0.528000in}}{\pgfqpoint{4.960000in}{3.696000in}}%
\pgfusepath{clip}%
\pgfsetbuttcap%
\pgfsetroundjoin%
\definecolor{currentfill}{rgb}{0.000000,0.000000,0.000000}%
\pgfsetfillcolor{currentfill}%
\pgfsetlinewidth{1.003750pt}%
\definecolor{currentstroke}{rgb}{0.000000,0.000000,0.000000}%
\pgfsetstrokecolor{currentstroke}%
\pgfsetdash{}{0pt}%
\pgfpathmoveto{\pgfqpoint{5.504545in}{3.470461in}}%
\pgfpathcurveto{\pgfqpoint{5.515596in}{3.470461in}}{\pgfqpoint{5.526195in}{3.474851in}}{\pgfqpoint{5.534008in}{3.482665in}}%
\pgfpathcurveto{\pgfqpoint{5.541822in}{3.490478in}}{\pgfqpoint{5.546212in}{3.501077in}}{\pgfqpoint{5.546212in}{3.512127in}}%
\pgfpathcurveto{\pgfqpoint{5.546212in}{3.523178in}}{\pgfqpoint{5.541822in}{3.533777in}}{\pgfqpoint{5.534008in}{3.541590in}}%
\pgfpathcurveto{\pgfqpoint{5.526195in}{3.549404in}}{\pgfqpoint{5.515596in}{3.553794in}}{\pgfqpoint{5.504545in}{3.553794in}}%
\pgfpathcurveto{\pgfqpoint{5.493495in}{3.553794in}}{\pgfqpoint{5.482896in}{3.549404in}}{\pgfqpoint{5.475083in}{3.541590in}}%
\pgfpathcurveto{\pgfqpoint{5.467269in}{3.533777in}}{\pgfqpoint{5.462879in}{3.523178in}}{\pgfqpoint{5.462879in}{3.512127in}}%
\pgfpathcurveto{\pgfqpoint{5.462879in}{3.501077in}}{\pgfqpoint{5.467269in}{3.490478in}}{\pgfqpoint{5.475083in}{3.482665in}}%
\pgfpathcurveto{\pgfqpoint{5.482896in}{3.474851in}}{\pgfqpoint{5.493495in}{3.470461in}}{\pgfqpoint{5.504545in}{3.470461in}}%
\pgfpathclose%
\pgfusepath{stroke,fill}%
\end{pgfscope}%
\begin{pgfscope}%
\pgfpathrectangle{\pgfqpoint{0.800000in}{0.528000in}}{\pgfqpoint{4.960000in}{3.696000in}}%
\pgfusepath{clip}%
\pgfsetbuttcap%
\pgfsetroundjoin%
\definecolor{currentfill}{rgb}{0.000000,0.000000,0.000000}%
\pgfsetfillcolor{currentfill}%
\pgfsetlinewidth{1.003750pt}%
\definecolor{currentstroke}{rgb}{0.000000,0.000000,0.000000}%
\pgfsetstrokecolor{currentstroke}%
\pgfsetdash{}{0pt}%
\pgfpathmoveto{\pgfqpoint{5.504545in}{3.324826in}}%
\pgfpathcurveto{\pgfqpoint{5.515596in}{3.324826in}}{\pgfqpoint{5.526195in}{3.329217in}}{\pgfqpoint{5.534008in}{3.337030in}}%
\pgfpathcurveto{\pgfqpoint{5.541822in}{3.344844in}}{\pgfqpoint{5.546212in}{3.355443in}}{\pgfqpoint{5.546212in}{3.366493in}}%
\pgfpathcurveto{\pgfqpoint{5.546212in}{3.377543in}}{\pgfqpoint{5.541822in}{3.388142in}}{\pgfqpoint{5.534008in}{3.395956in}}%
\pgfpathcurveto{\pgfqpoint{5.526195in}{3.403769in}}{\pgfqpoint{5.515596in}{3.408160in}}{\pgfqpoint{5.504545in}{3.408160in}}%
\pgfpathcurveto{\pgfqpoint{5.493495in}{3.408160in}}{\pgfqpoint{5.482896in}{3.403769in}}{\pgfqpoint{5.475083in}{3.395956in}}%
\pgfpathcurveto{\pgfqpoint{5.467269in}{3.388142in}}{\pgfqpoint{5.462879in}{3.377543in}}{\pgfqpoint{5.462879in}{3.366493in}}%
\pgfpathcurveto{\pgfqpoint{5.462879in}{3.355443in}}{\pgfqpoint{5.467269in}{3.344844in}}{\pgfqpoint{5.475083in}{3.337030in}}%
\pgfpathcurveto{\pgfqpoint{5.482896in}{3.329217in}}{\pgfqpoint{5.493495in}{3.324826in}}{\pgfqpoint{5.504545in}{3.324826in}}%
\pgfpathclose%
\pgfusepath{stroke,fill}%
\end{pgfscope}%
\begin{pgfscope}%
\pgfpathrectangle{\pgfqpoint{0.800000in}{0.528000in}}{\pgfqpoint{4.960000in}{3.696000in}}%
\pgfusepath{clip}%
\pgfsetbuttcap%
\pgfsetroundjoin%
\definecolor{currentfill}{rgb}{0.000000,0.000000,0.000000}%
\pgfsetfillcolor{currentfill}%
\pgfsetlinewidth{1.003750pt}%
\definecolor{currentstroke}{rgb}{0.000000,0.000000,0.000000}%
\pgfsetstrokecolor{currentstroke}%
\pgfsetdash{}{0pt}%
\pgfpathmoveto{\pgfqpoint{5.504545in}{3.761730in}}%
\pgfpathcurveto{\pgfqpoint{5.515596in}{3.761730in}}{\pgfqpoint{5.526195in}{3.766120in}}{\pgfqpoint{5.534008in}{3.773934in}}%
\pgfpathcurveto{\pgfqpoint{5.541822in}{3.781747in}}{\pgfqpoint{5.546212in}{3.792346in}}{\pgfqpoint{5.546212in}{3.803397in}}%
\pgfpathcurveto{\pgfqpoint{5.546212in}{3.814447in}}{\pgfqpoint{5.541822in}{3.825046in}}{\pgfqpoint{5.534008in}{3.832859in}}%
\pgfpathcurveto{\pgfqpoint{5.526195in}{3.840673in}}{\pgfqpoint{5.515596in}{3.845063in}}{\pgfqpoint{5.504545in}{3.845063in}}%
\pgfpathcurveto{\pgfqpoint{5.493495in}{3.845063in}}{\pgfqpoint{5.482896in}{3.840673in}}{\pgfqpoint{5.475083in}{3.832859in}}%
\pgfpathcurveto{\pgfqpoint{5.467269in}{3.825046in}}{\pgfqpoint{5.462879in}{3.814447in}}{\pgfqpoint{5.462879in}{3.803397in}}%
\pgfpathcurveto{\pgfqpoint{5.462879in}{3.792346in}}{\pgfqpoint{5.467269in}{3.781747in}}{\pgfqpoint{5.475083in}{3.773934in}}%
\pgfpathcurveto{\pgfqpoint{5.482896in}{3.766120in}}{\pgfqpoint{5.493495in}{3.761730in}}{\pgfqpoint{5.504545in}{3.761730in}}%
\pgfpathclose%
\pgfusepath{stroke,fill}%
\end{pgfscope}%
\begin{pgfscope}%
\pgfpathrectangle{\pgfqpoint{0.800000in}{0.528000in}}{\pgfqpoint{4.960000in}{3.696000in}}%
\pgfusepath{clip}%
\pgfsetbuttcap%
\pgfsetroundjoin%
\definecolor{currentfill}{rgb}{0.000000,0.000000,0.000000}%
\pgfsetfillcolor{currentfill}%
\pgfsetlinewidth{1.003750pt}%
\definecolor{currentstroke}{rgb}{0.000000,0.000000,0.000000}%
\pgfsetstrokecolor{currentstroke}%
\pgfsetdash{}{0pt}%
\pgfpathmoveto{\pgfqpoint{5.504545in}{3.072393in}}%
\pgfpathcurveto{\pgfqpoint{5.515596in}{3.072393in}}{\pgfqpoint{5.526195in}{3.076783in}}{\pgfqpoint{5.534008in}{3.084597in}}%
\pgfpathcurveto{\pgfqpoint{5.541822in}{3.092411in}}{\pgfqpoint{5.546212in}{3.103010in}}{\pgfqpoint{5.546212in}{3.114060in}}%
\pgfpathcurveto{\pgfqpoint{5.546212in}{3.125110in}}{\pgfqpoint{5.541822in}{3.135709in}}{\pgfqpoint{5.534008in}{3.143522in}}%
\pgfpathcurveto{\pgfqpoint{5.526195in}{3.151336in}}{\pgfqpoint{5.515596in}{3.155726in}}{\pgfqpoint{5.504545in}{3.155726in}}%
\pgfpathcurveto{\pgfqpoint{5.493495in}{3.155726in}}{\pgfqpoint{5.482896in}{3.151336in}}{\pgfqpoint{5.475083in}{3.143522in}}%
\pgfpathcurveto{\pgfqpoint{5.467269in}{3.135709in}}{\pgfqpoint{5.462879in}{3.125110in}}{\pgfqpoint{5.462879in}{3.114060in}}%
\pgfpathcurveto{\pgfqpoint{5.462879in}{3.103010in}}{\pgfqpoint{5.467269in}{3.092411in}}{\pgfqpoint{5.475083in}{3.084597in}}%
\pgfpathcurveto{\pgfqpoint{5.482896in}{3.076783in}}{\pgfqpoint{5.493495in}{3.072393in}}{\pgfqpoint{5.504545in}{3.072393in}}%
\pgfpathclose%
\pgfusepath{stroke,fill}%
\end{pgfscope}%
\begin{pgfscope}%
\pgfpathrectangle{\pgfqpoint{0.800000in}{0.528000in}}{\pgfqpoint{4.960000in}{3.696000in}}%
\pgfusepath{clip}%
\pgfsetbuttcap%
\pgfsetroundjoin%
\definecolor{currentfill}{rgb}{0.000000,0.000000,0.000000}%
\pgfsetfillcolor{currentfill}%
\pgfsetlinewidth{1.003750pt}%
\definecolor{currentstroke}{rgb}{0.000000,0.000000,0.000000}%
\pgfsetstrokecolor{currentstroke}%
\pgfsetdash{}{0pt}%
\pgfpathmoveto{\pgfqpoint{5.504545in}{3.790857in}}%
\pgfpathcurveto{\pgfqpoint{5.515596in}{3.790857in}}{\pgfqpoint{5.526195in}{3.795247in}}{\pgfqpoint{5.534008in}{3.803061in}}%
\pgfpathcurveto{\pgfqpoint{5.541822in}{3.810874in}}{\pgfqpoint{5.546212in}{3.821473in}}{\pgfqpoint{5.546212in}{3.832523in}}%
\pgfpathcurveto{\pgfqpoint{5.546212in}{3.843574in}}{\pgfqpoint{5.541822in}{3.854173in}}{\pgfqpoint{5.534008in}{3.861986in}}%
\pgfpathcurveto{\pgfqpoint{5.526195in}{3.869800in}}{\pgfqpoint{5.515596in}{3.874190in}}{\pgfqpoint{5.504545in}{3.874190in}}%
\pgfpathcurveto{\pgfqpoint{5.493495in}{3.874190in}}{\pgfqpoint{5.482896in}{3.869800in}}{\pgfqpoint{5.475083in}{3.861986in}}%
\pgfpathcurveto{\pgfqpoint{5.467269in}{3.854173in}}{\pgfqpoint{5.462879in}{3.843574in}}{\pgfqpoint{5.462879in}{3.832523in}}%
\pgfpathcurveto{\pgfqpoint{5.462879in}{3.821473in}}{\pgfqpoint{5.467269in}{3.810874in}}{\pgfqpoint{5.475083in}{3.803061in}}%
\pgfpathcurveto{\pgfqpoint{5.482896in}{3.795247in}}{\pgfqpoint{5.493495in}{3.790857in}}{\pgfqpoint{5.504545in}{3.790857in}}%
\pgfpathclose%
\pgfusepath{stroke,fill}%
\end{pgfscope}%
\begin{pgfscope}%
\pgfpathrectangle{\pgfqpoint{0.800000in}{0.528000in}}{\pgfqpoint{4.960000in}{3.696000in}}%
\pgfusepath{clip}%
\pgfsetbuttcap%
\pgfsetroundjoin%
\definecolor{currentfill}{rgb}{0.000000,0.000000,0.000000}%
\pgfsetfillcolor{currentfill}%
\pgfsetlinewidth{1.003750pt}%
\definecolor{currentstroke}{rgb}{0.000000,0.000000,0.000000}%
\pgfsetstrokecolor{currentstroke}%
\pgfsetdash{}{0pt}%
\pgfpathmoveto{\pgfqpoint{5.504545in}{3.276281in}}%
\pgfpathcurveto{\pgfqpoint{5.515596in}{3.276281in}}{\pgfqpoint{5.526195in}{3.280672in}}{\pgfqpoint{5.534008in}{3.288485in}}%
\pgfpathcurveto{\pgfqpoint{5.541822in}{3.296299in}}{\pgfqpoint{5.546212in}{3.306898in}}{\pgfqpoint{5.546212in}{3.317948in}}%
\pgfpathcurveto{\pgfqpoint{5.546212in}{3.328998in}}{\pgfqpoint{5.541822in}{3.339597in}}{\pgfqpoint{5.534008in}{3.347411in}}%
\pgfpathcurveto{\pgfqpoint{5.526195in}{3.355224in}}{\pgfqpoint{5.515596in}{3.359615in}}{\pgfqpoint{5.504545in}{3.359615in}}%
\pgfpathcurveto{\pgfqpoint{5.493495in}{3.359615in}}{\pgfqpoint{5.482896in}{3.355224in}}{\pgfqpoint{5.475083in}{3.347411in}}%
\pgfpathcurveto{\pgfqpoint{5.467269in}{3.339597in}}{\pgfqpoint{5.462879in}{3.328998in}}{\pgfqpoint{5.462879in}{3.317948in}}%
\pgfpathcurveto{\pgfqpoint{5.462879in}{3.306898in}}{\pgfqpoint{5.467269in}{3.296299in}}{\pgfqpoint{5.475083in}{3.288485in}}%
\pgfpathcurveto{\pgfqpoint{5.482896in}{3.280672in}}{\pgfqpoint{5.493495in}{3.276281in}}{\pgfqpoint{5.504545in}{3.276281in}}%
\pgfpathclose%
\pgfusepath{stroke,fill}%
\end{pgfscope}%
\begin{pgfscope}%
\pgfpathrectangle{\pgfqpoint{0.800000in}{0.528000in}}{\pgfqpoint{4.960000in}{3.696000in}}%
\pgfusepath{clip}%
\pgfsetbuttcap%
\pgfsetroundjoin%
\definecolor{currentfill}{rgb}{0.000000,0.000000,0.000000}%
\pgfsetfillcolor{currentfill}%
\pgfsetlinewidth{1.003750pt}%
\definecolor{currentstroke}{rgb}{0.000000,0.000000,0.000000}%
\pgfsetstrokecolor{currentstroke}%
\pgfsetdash{}{0pt}%
\pgfpathmoveto{\pgfqpoint{5.504545in}{3.344244in}}%
\pgfpathcurveto{\pgfqpoint{5.515596in}{3.344244in}}{\pgfqpoint{5.526195in}{3.348634in}}{\pgfqpoint{5.534008in}{3.356448in}}%
\pgfpathcurveto{\pgfqpoint{5.541822in}{3.364262in}}{\pgfqpoint{5.546212in}{3.374861in}}{\pgfqpoint{5.546212in}{3.385911in}}%
\pgfpathcurveto{\pgfqpoint{5.546212in}{3.396961in}}{\pgfqpoint{5.541822in}{3.407560in}}{\pgfqpoint{5.534008in}{3.415374in}}%
\pgfpathcurveto{\pgfqpoint{5.526195in}{3.423187in}}{\pgfqpoint{5.515596in}{3.427578in}}{\pgfqpoint{5.504545in}{3.427578in}}%
\pgfpathcurveto{\pgfqpoint{5.493495in}{3.427578in}}{\pgfqpoint{5.482896in}{3.423187in}}{\pgfqpoint{5.475083in}{3.415374in}}%
\pgfpathcurveto{\pgfqpoint{5.467269in}{3.407560in}}{\pgfqpoint{5.462879in}{3.396961in}}{\pgfqpoint{5.462879in}{3.385911in}}%
\pgfpathcurveto{\pgfqpoint{5.462879in}{3.374861in}}{\pgfqpoint{5.467269in}{3.364262in}}{\pgfqpoint{5.475083in}{3.356448in}}%
\pgfpathcurveto{\pgfqpoint{5.482896in}{3.348634in}}{\pgfqpoint{5.493495in}{3.344244in}}{\pgfqpoint{5.504545in}{3.344244in}}%
\pgfpathclose%
\pgfusepath{stroke,fill}%
\end{pgfscope}%
\begin{pgfscope}%
\pgfpathrectangle{\pgfqpoint{0.800000in}{0.528000in}}{\pgfqpoint{4.960000in}{3.696000in}}%
\pgfusepath{clip}%
\pgfsetbuttcap%
\pgfsetroundjoin%
\definecolor{currentfill}{rgb}{0.000000,0.000000,0.000000}%
\pgfsetfillcolor{currentfill}%
\pgfsetlinewidth{1.003750pt}%
\definecolor{currentstroke}{rgb}{0.000000,0.000000,0.000000}%
\pgfsetstrokecolor{currentstroke}%
\pgfsetdash{}{0pt}%
\pgfpathmoveto{\pgfqpoint{5.504545in}{3.111229in}}%
\pgfpathcurveto{\pgfqpoint{5.515596in}{3.111229in}}{\pgfqpoint{5.526195in}{3.115619in}}{\pgfqpoint{5.534008in}{3.123433in}}%
\pgfpathcurveto{\pgfqpoint{5.541822in}{3.131246in}}{\pgfqpoint{5.546212in}{3.141845in}}{\pgfqpoint{5.546212in}{3.152896in}}%
\pgfpathcurveto{\pgfqpoint{5.546212in}{3.163946in}}{\pgfqpoint{5.541822in}{3.174545in}}{\pgfqpoint{5.534008in}{3.182358in}}%
\pgfpathcurveto{\pgfqpoint{5.526195in}{3.190172in}}{\pgfqpoint{5.515596in}{3.194562in}}{\pgfqpoint{5.504545in}{3.194562in}}%
\pgfpathcurveto{\pgfqpoint{5.493495in}{3.194562in}}{\pgfqpoint{5.482896in}{3.190172in}}{\pgfqpoint{5.475083in}{3.182358in}}%
\pgfpathcurveto{\pgfqpoint{5.467269in}{3.174545in}}{\pgfqpoint{5.462879in}{3.163946in}}{\pgfqpoint{5.462879in}{3.152896in}}%
\pgfpathcurveto{\pgfqpoint{5.462879in}{3.141845in}}{\pgfqpoint{5.467269in}{3.131246in}}{\pgfqpoint{5.475083in}{3.123433in}}%
\pgfpathcurveto{\pgfqpoint{5.482896in}{3.115619in}}{\pgfqpoint{5.493495in}{3.111229in}}{\pgfqpoint{5.504545in}{3.111229in}}%
\pgfpathclose%
\pgfusepath{stroke,fill}%
\end{pgfscope}%
\begin{pgfscope}%
\pgfpathrectangle{\pgfqpoint{0.800000in}{0.528000in}}{\pgfqpoint{4.960000in}{3.696000in}}%
\pgfusepath{clip}%
\pgfsetbuttcap%
\pgfsetroundjoin%
\definecolor{currentfill}{rgb}{0.000000,0.000000,0.000000}%
\pgfsetfillcolor{currentfill}%
\pgfsetlinewidth{1.003750pt}%
\definecolor{currentstroke}{rgb}{0.000000,0.000000,0.000000}%
\pgfsetstrokecolor{currentstroke}%
\pgfsetdash{}{0pt}%
\pgfpathmoveto{\pgfqpoint{5.504545in}{2.994721in}}%
\pgfpathcurveto{\pgfqpoint{5.515596in}{2.994721in}}{\pgfqpoint{5.526195in}{2.999112in}}{\pgfqpoint{5.534008in}{3.006925in}}%
\pgfpathcurveto{\pgfqpoint{5.541822in}{3.014739in}}{\pgfqpoint{5.546212in}{3.025338in}}{\pgfqpoint{5.546212in}{3.036388in}}%
\pgfpathcurveto{\pgfqpoint{5.546212in}{3.047438in}}{\pgfqpoint{5.541822in}{3.058037in}}{\pgfqpoint{5.534008in}{3.065851in}}%
\pgfpathcurveto{\pgfqpoint{5.526195in}{3.073664in}}{\pgfqpoint{5.515596in}{3.078055in}}{\pgfqpoint{5.504545in}{3.078055in}}%
\pgfpathcurveto{\pgfqpoint{5.493495in}{3.078055in}}{\pgfqpoint{5.482896in}{3.073664in}}{\pgfqpoint{5.475083in}{3.065851in}}%
\pgfpathcurveto{\pgfqpoint{5.467269in}{3.058037in}}{\pgfqpoint{5.462879in}{3.047438in}}{\pgfqpoint{5.462879in}{3.036388in}}%
\pgfpathcurveto{\pgfqpoint{5.462879in}{3.025338in}}{\pgfqpoint{5.467269in}{3.014739in}}{\pgfqpoint{5.475083in}{3.006925in}}%
\pgfpathcurveto{\pgfqpoint{5.482896in}{2.999112in}}{\pgfqpoint{5.493495in}{2.994721in}}{\pgfqpoint{5.504545in}{2.994721in}}%
\pgfpathclose%
\pgfusepath{stroke,fill}%
\end{pgfscope}%
\begin{pgfscope}%
\pgfpathrectangle{\pgfqpoint{0.800000in}{0.528000in}}{\pgfqpoint{4.960000in}{3.696000in}}%
\pgfusepath{clip}%
\pgfsetbuttcap%
\pgfsetroundjoin%
\definecolor{currentfill}{rgb}{0.000000,0.000000,0.000000}%
\pgfsetfillcolor{currentfill}%
\pgfsetlinewidth{1.003750pt}%
\definecolor{currentstroke}{rgb}{0.000000,0.000000,0.000000}%
\pgfsetstrokecolor{currentstroke}%
\pgfsetdash{}{0pt}%
\pgfpathmoveto{\pgfqpoint{5.504545in}{3.353953in}}%
\pgfpathcurveto{\pgfqpoint{5.515596in}{3.353953in}}{\pgfqpoint{5.526195in}{3.358343in}}{\pgfqpoint{5.534008in}{3.366157in}}%
\pgfpathcurveto{\pgfqpoint{5.541822in}{3.373971in}}{\pgfqpoint{5.546212in}{3.384570in}}{\pgfqpoint{5.546212in}{3.395620in}}%
\pgfpathcurveto{\pgfqpoint{5.546212in}{3.406670in}}{\pgfqpoint{5.541822in}{3.417269in}}{\pgfqpoint{5.534008in}{3.425083in}}%
\pgfpathcurveto{\pgfqpoint{5.526195in}{3.432896in}}{\pgfqpoint{5.515596in}{3.437286in}}{\pgfqpoint{5.504545in}{3.437286in}}%
\pgfpathcurveto{\pgfqpoint{5.493495in}{3.437286in}}{\pgfqpoint{5.482896in}{3.432896in}}{\pgfqpoint{5.475083in}{3.425083in}}%
\pgfpathcurveto{\pgfqpoint{5.467269in}{3.417269in}}{\pgfqpoint{5.462879in}{3.406670in}}{\pgfqpoint{5.462879in}{3.395620in}}%
\pgfpathcurveto{\pgfqpoint{5.462879in}{3.384570in}}{\pgfqpoint{5.467269in}{3.373971in}}{\pgfqpoint{5.475083in}{3.366157in}}%
\pgfpathcurveto{\pgfqpoint{5.482896in}{3.358343in}}{\pgfqpoint{5.493495in}{3.353953in}}{\pgfqpoint{5.504545in}{3.353953in}}%
\pgfpathclose%
\pgfusepath{stroke,fill}%
\end{pgfscope}%
\begin{pgfscope}%
\pgfpathrectangle{\pgfqpoint{0.800000in}{0.528000in}}{\pgfqpoint{4.960000in}{3.696000in}}%
\pgfusepath{clip}%
\pgfsetbuttcap%
\pgfsetroundjoin%
\definecolor{currentfill}{rgb}{0.000000,0.000000,0.000000}%
\pgfsetfillcolor{currentfill}%
\pgfsetlinewidth{1.003750pt}%
\definecolor{currentstroke}{rgb}{0.000000,0.000000,0.000000}%
\pgfsetstrokecolor{currentstroke}%
\pgfsetdash{}{0pt}%
\pgfpathmoveto{\pgfqpoint{5.504545in}{3.334535in}}%
\pgfpathcurveto{\pgfqpoint{5.515596in}{3.334535in}}{\pgfqpoint{5.526195in}{3.338925in}}{\pgfqpoint{5.534008in}{3.346739in}}%
\pgfpathcurveto{\pgfqpoint{5.541822in}{3.354553in}}{\pgfqpoint{5.546212in}{3.365152in}}{\pgfqpoint{5.546212in}{3.376202in}}%
\pgfpathcurveto{\pgfqpoint{5.546212in}{3.387252in}}{\pgfqpoint{5.541822in}{3.397851in}}{\pgfqpoint{5.534008in}{3.405665in}}%
\pgfpathcurveto{\pgfqpoint{5.526195in}{3.413478in}}{\pgfqpoint{5.515596in}{3.417869in}}{\pgfqpoint{5.504545in}{3.417869in}}%
\pgfpathcurveto{\pgfqpoint{5.493495in}{3.417869in}}{\pgfqpoint{5.482896in}{3.413478in}}{\pgfqpoint{5.475083in}{3.405665in}}%
\pgfpathcurveto{\pgfqpoint{5.467269in}{3.397851in}}{\pgfqpoint{5.462879in}{3.387252in}}{\pgfqpoint{5.462879in}{3.376202in}}%
\pgfpathcurveto{\pgfqpoint{5.462879in}{3.365152in}}{\pgfqpoint{5.467269in}{3.354553in}}{\pgfqpoint{5.475083in}{3.346739in}}%
\pgfpathcurveto{\pgfqpoint{5.482896in}{3.338925in}}{\pgfqpoint{5.493495in}{3.334535in}}{\pgfqpoint{5.504545in}{3.334535in}}%
\pgfpathclose%
\pgfusepath{stroke,fill}%
\end{pgfscope}%
\begin{pgfscope}%
\pgfpathrectangle{\pgfqpoint{0.800000in}{0.528000in}}{\pgfqpoint{4.960000in}{3.696000in}}%
\pgfusepath{clip}%
\pgfsetbuttcap%
\pgfsetroundjoin%
\definecolor{currentfill}{rgb}{0.000000,0.000000,0.000000}%
\pgfsetfillcolor{currentfill}%
\pgfsetlinewidth{1.003750pt}%
\definecolor{currentstroke}{rgb}{0.000000,0.000000,0.000000}%
\pgfsetstrokecolor{currentstroke}%
\pgfsetdash{}{0pt}%
\pgfpathmoveto{\pgfqpoint{5.504545in}{3.091811in}}%
\pgfpathcurveto{\pgfqpoint{5.515596in}{3.091811in}}{\pgfqpoint{5.526195in}{3.096201in}}{\pgfqpoint{5.534008in}{3.104015in}}%
\pgfpathcurveto{\pgfqpoint{5.541822in}{3.111828in}}{\pgfqpoint{5.546212in}{3.122428in}}{\pgfqpoint{5.546212in}{3.133478in}}%
\pgfpathcurveto{\pgfqpoint{5.546212in}{3.144528in}}{\pgfqpoint{5.541822in}{3.155127in}}{\pgfqpoint{5.534008in}{3.162940in}}%
\pgfpathcurveto{\pgfqpoint{5.526195in}{3.170754in}}{\pgfqpoint{5.515596in}{3.175144in}}{\pgfqpoint{5.504545in}{3.175144in}}%
\pgfpathcurveto{\pgfqpoint{5.493495in}{3.175144in}}{\pgfqpoint{5.482896in}{3.170754in}}{\pgfqpoint{5.475083in}{3.162940in}}%
\pgfpathcurveto{\pgfqpoint{5.467269in}{3.155127in}}{\pgfqpoint{5.462879in}{3.144528in}}{\pgfqpoint{5.462879in}{3.133478in}}%
\pgfpathcurveto{\pgfqpoint{5.462879in}{3.122428in}}{\pgfqpoint{5.467269in}{3.111828in}}{\pgfqpoint{5.475083in}{3.104015in}}%
\pgfpathcurveto{\pgfqpoint{5.482896in}{3.096201in}}{\pgfqpoint{5.493495in}{3.091811in}}{\pgfqpoint{5.504545in}{3.091811in}}%
\pgfpathclose%
\pgfusepath{stroke,fill}%
\end{pgfscope}%
\begin{pgfscope}%
\pgfpathrectangle{\pgfqpoint{0.800000in}{0.528000in}}{\pgfqpoint{4.960000in}{3.696000in}}%
\pgfusepath{clip}%
\pgfsetbuttcap%
\pgfsetroundjoin%
\definecolor{currentfill}{rgb}{0.000000,0.000000,0.000000}%
\pgfsetfillcolor{currentfill}%
\pgfsetlinewidth{1.003750pt}%
\definecolor{currentstroke}{rgb}{0.000000,0.000000,0.000000}%
\pgfsetstrokecolor{currentstroke}%
\pgfsetdash{}{0pt}%
\pgfpathmoveto{\pgfqpoint{5.504545in}{2.907341in}}%
\pgfpathcurveto{\pgfqpoint{5.515596in}{2.907341in}}{\pgfqpoint{5.526195in}{2.911731in}}{\pgfqpoint{5.534008in}{2.919544in}}%
\pgfpathcurveto{\pgfqpoint{5.541822in}{2.927358in}}{\pgfqpoint{5.546212in}{2.937957in}}{\pgfqpoint{5.546212in}{2.949007in}}%
\pgfpathcurveto{\pgfqpoint{5.546212in}{2.960057in}}{\pgfqpoint{5.541822in}{2.970656in}}{\pgfqpoint{5.534008in}{2.978470in}}%
\pgfpathcurveto{\pgfqpoint{5.526195in}{2.986284in}}{\pgfqpoint{5.515596in}{2.990674in}}{\pgfqpoint{5.504545in}{2.990674in}}%
\pgfpathcurveto{\pgfqpoint{5.493495in}{2.990674in}}{\pgfqpoint{5.482896in}{2.986284in}}{\pgfqpoint{5.475083in}{2.978470in}}%
\pgfpathcurveto{\pgfqpoint{5.467269in}{2.970656in}}{\pgfqpoint{5.462879in}{2.960057in}}{\pgfqpoint{5.462879in}{2.949007in}}%
\pgfpathcurveto{\pgfqpoint{5.462879in}{2.937957in}}{\pgfqpoint{5.467269in}{2.927358in}}{\pgfqpoint{5.475083in}{2.919544in}}%
\pgfpathcurveto{\pgfqpoint{5.482896in}{2.911731in}}{\pgfqpoint{5.493495in}{2.907341in}}{\pgfqpoint{5.504545in}{2.907341in}}%
\pgfpathclose%
\pgfusepath{stroke,fill}%
\end{pgfscope}%
\begin{pgfscope}%
\pgfpathrectangle{\pgfqpoint{0.800000in}{0.528000in}}{\pgfqpoint{4.960000in}{3.696000in}}%
\pgfusepath{clip}%
\pgfsetbuttcap%
\pgfsetroundjoin%
\definecolor{currentfill}{rgb}{0.000000,0.000000,0.000000}%
\pgfsetfillcolor{currentfill}%
\pgfsetlinewidth{1.003750pt}%
\definecolor{currentstroke}{rgb}{0.000000,0.000000,0.000000}%
\pgfsetstrokecolor{currentstroke}%
\pgfsetdash{}{0pt}%
\pgfpathmoveto{\pgfqpoint{5.504545in}{3.266572in}}%
\pgfpathcurveto{\pgfqpoint{5.515596in}{3.266572in}}{\pgfqpoint{5.526195in}{3.270963in}}{\pgfqpoint{5.534008in}{3.278776in}}%
\pgfpathcurveto{\pgfqpoint{5.541822in}{3.286590in}}{\pgfqpoint{5.546212in}{3.297189in}}{\pgfqpoint{5.546212in}{3.308239in}}%
\pgfpathcurveto{\pgfqpoint{5.546212in}{3.319289in}}{\pgfqpoint{5.541822in}{3.329888in}}{\pgfqpoint{5.534008in}{3.337702in}}%
\pgfpathcurveto{\pgfqpoint{5.526195in}{3.345515in}}{\pgfqpoint{5.515596in}{3.349906in}}{\pgfqpoint{5.504545in}{3.349906in}}%
\pgfpathcurveto{\pgfqpoint{5.493495in}{3.349906in}}{\pgfqpoint{5.482896in}{3.345515in}}{\pgfqpoint{5.475083in}{3.337702in}}%
\pgfpathcurveto{\pgfqpoint{5.467269in}{3.329888in}}{\pgfqpoint{5.462879in}{3.319289in}}{\pgfqpoint{5.462879in}{3.308239in}}%
\pgfpathcurveto{\pgfqpoint{5.462879in}{3.297189in}}{\pgfqpoint{5.467269in}{3.286590in}}{\pgfqpoint{5.475083in}{3.278776in}}%
\pgfpathcurveto{\pgfqpoint{5.482896in}{3.270963in}}{\pgfqpoint{5.493495in}{3.266572in}}{\pgfqpoint{5.504545in}{3.266572in}}%
\pgfpathclose%
\pgfusepath{stroke,fill}%
\end{pgfscope}%
\begin{pgfscope}%
\pgfpathrectangle{\pgfqpoint{0.800000in}{0.528000in}}{\pgfqpoint{4.960000in}{3.696000in}}%
\pgfusepath{clip}%
\pgfsetbuttcap%
\pgfsetroundjoin%
\definecolor{currentfill}{rgb}{0.000000,0.000000,0.000000}%
\pgfsetfillcolor{currentfill}%
\pgfsetlinewidth{1.003750pt}%
\definecolor{currentstroke}{rgb}{0.000000,0.000000,0.000000}%
\pgfsetstrokecolor{currentstroke}%
\pgfsetdash{}{0pt}%
\pgfpathmoveto{\pgfqpoint{5.504545in}{3.014139in}}%
\pgfpathcurveto{\pgfqpoint{5.515596in}{3.014139in}}{\pgfqpoint{5.526195in}{3.018529in}}{\pgfqpoint{5.534008in}{3.026343in}}%
\pgfpathcurveto{\pgfqpoint{5.541822in}{3.034157in}}{\pgfqpoint{5.546212in}{3.044756in}}{\pgfqpoint{5.546212in}{3.055806in}}%
\pgfpathcurveto{\pgfqpoint{5.546212in}{3.066856in}}{\pgfqpoint{5.541822in}{3.077455in}}{\pgfqpoint{5.534008in}{3.085269in}}%
\pgfpathcurveto{\pgfqpoint{5.526195in}{3.093082in}}{\pgfqpoint{5.515596in}{3.097473in}}{\pgfqpoint{5.504545in}{3.097473in}}%
\pgfpathcurveto{\pgfqpoint{5.493495in}{3.097473in}}{\pgfqpoint{5.482896in}{3.093082in}}{\pgfqpoint{5.475083in}{3.085269in}}%
\pgfpathcurveto{\pgfqpoint{5.467269in}{3.077455in}}{\pgfqpoint{5.462879in}{3.066856in}}{\pgfqpoint{5.462879in}{3.055806in}}%
\pgfpathcurveto{\pgfqpoint{5.462879in}{3.044756in}}{\pgfqpoint{5.467269in}{3.034157in}}{\pgfqpoint{5.475083in}{3.026343in}}%
\pgfpathcurveto{\pgfqpoint{5.482896in}{3.018529in}}{\pgfqpoint{5.493495in}{3.014139in}}{\pgfqpoint{5.504545in}{3.014139in}}%
\pgfpathclose%
\pgfusepath{stroke,fill}%
\end{pgfscope}%
\begin{pgfscope}%
\pgfpathrectangle{\pgfqpoint{0.800000in}{0.528000in}}{\pgfqpoint{4.960000in}{3.696000in}}%
\pgfusepath{clip}%
\pgfsetbuttcap%
\pgfsetroundjoin%
\definecolor{currentfill}{rgb}{0.000000,0.000000,0.000000}%
\pgfsetfillcolor{currentfill}%
\pgfsetlinewidth{1.003750pt}%
\definecolor{currentstroke}{rgb}{0.000000,0.000000,0.000000}%
\pgfsetstrokecolor{currentstroke}%
\pgfsetdash{}{0pt}%
\pgfpathmoveto{\pgfqpoint{5.504545in}{3.324826in}}%
\pgfpathcurveto{\pgfqpoint{5.515596in}{3.324826in}}{\pgfqpoint{5.526195in}{3.329217in}}{\pgfqpoint{5.534008in}{3.337030in}}%
\pgfpathcurveto{\pgfqpoint{5.541822in}{3.344844in}}{\pgfqpoint{5.546212in}{3.355443in}}{\pgfqpoint{5.546212in}{3.366493in}}%
\pgfpathcurveto{\pgfqpoint{5.546212in}{3.377543in}}{\pgfqpoint{5.541822in}{3.388142in}}{\pgfqpoint{5.534008in}{3.395956in}}%
\pgfpathcurveto{\pgfqpoint{5.526195in}{3.403769in}}{\pgfqpoint{5.515596in}{3.408160in}}{\pgfqpoint{5.504545in}{3.408160in}}%
\pgfpathcurveto{\pgfqpoint{5.493495in}{3.408160in}}{\pgfqpoint{5.482896in}{3.403769in}}{\pgfqpoint{5.475083in}{3.395956in}}%
\pgfpathcurveto{\pgfqpoint{5.467269in}{3.388142in}}{\pgfqpoint{5.462879in}{3.377543in}}{\pgfqpoint{5.462879in}{3.366493in}}%
\pgfpathcurveto{\pgfqpoint{5.462879in}{3.355443in}}{\pgfqpoint{5.467269in}{3.344844in}}{\pgfqpoint{5.475083in}{3.337030in}}%
\pgfpathcurveto{\pgfqpoint{5.482896in}{3.329217in}}{\pgfqpoint{5.493495in}{3.324826in}}{\pgfqpoint{5.504545in}{3.324826in}}%
\pgfpathclose%
\pgfusepath{stroke,fill}%
\end{pgfscope}%
\begin{pgfscope}%
\pgfpathrectangle{\pgfqpoint{0.800000in}{0.528000in}}{\pgfqpoint{4.960000in}{3.696000in}}%
\pgfusepath{clip}%
\pgfsetbuttcap%
\pgfsetroundjoin%
\definecolor{currentfill}{rgb}{0.000000,0.000000,0.000000}%
\pgfsetfillcolor{currentfill}%
\pgfsetlinewidth{1.003750pt}%
\definecolor{currentstroke}{rgb}{0.000000,0.000000,0.000000}%
\pgfsetstrokecolor{currentstroke}%
\pgfsetdash{}{0pt}%
\pgfpathmoveto{\pgfqpoint{5.504545in}{3.334535in}}%
\pgfpathcurveto{\pgfqpoint{5.515596in}{3.334535in}}{\pgfqpoint{5.526195in}{3.338925in}}{\pgfqpoint{5.534008in}{3.346739in}}%
\pgfpathcurveto{\pgfqpoint{5.541822in}{3.354553in}}{\pgfqpoint{5.546212in}{3.365152in}}{\pgfqpoint{5.546212in}{3.376202in}}%
\pgfpathcurveto{\pgfqpoint{5.546212in}{3.387252in}}{\pgfqpoint{5.541822in}{3.397851in}}{\pgfqpoint{5.534008in}{3.405665in}}%
\pgfpathcurveto{\pgfqpoint{5.526195in}{3.413478in}}{\pgfqpoint{5.515596in}{3.417869in}}{\pgfqpoint{5.504545in}{3.417869in}}%
\pgfpathcurveto{\pgfqpoint{5.493495in}{3.417869in}}{\pgfqpoint{5.482896in}{3.413478in}}{\pgfqpoint{5.475083in}{3.405665in}}%
\pgfpathcurveto{\pgfqpoint{5.467269in}{3.397851in}}{\pgfqpoint{5.462879in}{3.387252in}}{\pgfqpoint{5.462879in}{3.376202in}}%
\pgfpathcurveto{\pgfqpoint{5.462879in}{3.365152in}}{\pgfqpoint{5.467269in}{3.354553in}}{\pgfqpoint{5.475083in}{3.346739in}}%
\pgfpathcurveto{\pgfqpoint{5.482896in}{3.338925in}}{\pgfqpoint{5.493495in}{3.334535in}}{\pgfqpoint{5.504545in}{3.334535in}}%
\pgfpathclose%
\pgfusepath{stroke,fill}%
\end{pgfscope}%
\begin{pgfscope}%
\pgfpathrectangle{\pgfqpoint{0.800000in}{0.528000in}}{\pgfqpoint{4.960000in}{3.696000in}}%
\pgfusepath{clip}%
\pgfsetbuttcap%
\pgfsetroundjoin%
\definecolor{currentfill}{rgb}{0.000000,0.000000,0.000000}%
\pgfsetfillcolor{currentfill}%
\pgfsetlinewidth{1.003750pt}%
\definecolor{currentstroke}{rgb}{0.000000,0.000000,0.000000}%
\pgfsetstrokecolor{currentstroke}%
\pgfsetdash{}{0pt}%
\pgfpathmoveto{\pgfqpoint{5.504545in}{3.072393in}}%
\pgfpathcurveto{\pgfqpoint{5.515596in}{3.072393in}}{\pgfqpoint{5.526195in}{3.076783in}}{\pgfqpoint{5.534008in}{3.084597in}}%
\pgfpathcurveto{\pgfqpoint{5.541822in}{3.092411in}}{\pgfqpoint{5.546212in}{3.103010in}}{\pgfqpoint{5.546212in}{3.114060in}}%
\pgfpathcurveto{\pgfqpoint{5.546212in}{3.125110in}}{\pgfqpoint{5.541822in}{3.135709in}}{\pgfqpoint{5.534008in}{3.143522in}}%
\pgfpathcurveto{\pgfqpoint{5.526195in}{3.151336in}}{\pgfqpoint{5.515596in}{3.155726in}}{\pgfqpoint{5.504545in}{3.155726in}}%
\pgfpathcurveto{\pgfqpoint{5.493495in}{3.155726in}}{\pgfqpoint{5.482896in}{3.151336in}}{\pgfqpoint{5.475083in}{3.143522in}}%
\pgfpathcurveto{\pgfqpoint{5.467269in}{3.135709in}}{\pgfqpoint{5.462879in}{3.125110in}}{\pgfqpoint{5.462879in}{3.114060in}}%
\pgfpathcurveto{\pgfqpoint{5.462879in}{3.103010in}}{\pgfqpoint{5.467269in}{3.092411in}}{\pgfqpoint{5.475083in}{3.084597in}}%
\pgfpathcurveto{\pgfqpoint{5.482896in}{3.076783in}}{\pgfqpoint{5.493495in}{3.072393in}}{\pgfqpoint{5.504545in}{3.072393in}}%
\pgfpathclose%
\pgfusepath{stroke,fill}%
\end{pgfscope}%
\begin{pgfscope}%
\pgfpathrectangle{\pgfqpoint{0.800000in}{0.528000in}}{\pgfqpoint{4.960000in}{3.696000in}}%
\pgfusepath{clip}%
\pgfsetbuttcap%
\pgfsetroundjoin%
\definecolor{currentfill}{rgb}{0.000000,0.000000,0.000000}%
\pgfsetfillcolor{currentfill}%
\pgfsetlinewidth{1.003750pt}%
\definecolor{currentstroke}{rgb}{0.000000,0.000000,0.000000}%
\pgfsetstrokecolor{currentstroke}%
\pgfsetdash{}{0pt}%
\pgfpathmoveto{\pgfqpoint{5.504545in}{3.489879in}}%
\pgfpathcurveto{\pgfqpoint{5.515596in}{3.489879in}}{\pgfqpoint{5.526195in}{3.494269in}}{\pgfqpoint{5.534008in}{3.502083in}}%
\pgfpathcurveto{\pgfqpoint{5.541822in}{3.509896in}}{\pgfqpoint{5.546212in}{3.520495in}}{\pgfqpoint{5.546212in}{3.531545in}}%
\pgfpathcurveto{\pgfqpoint{5.546212in}{3.542596in}}{\pgfqpoint{5.541822in}{3.553195in}}{\pgfqpoint{5.534008in}{3.561008in}}%
\pgfpathcurveto{\pgfqpoint{5.526195in}{3.568822in}}{\pgfqpoint{5.515596in}{3.573212in}}{\pgfqpoint{5.504545in}{3.573212in}}%
\pgfpathcurveto{\pgfqpoint{5.493495in}{3.573212in}}{\pgfqpoint{5.482896in}{3.568822in}}{\pgfqpoint{5.475083in}{3.561008in}}%
\pgfpathcurveto{\pgfqpoint{5.467269in}{3.553195in}}{\pgfqpoint{5.462879in}{3.542596in}}{\pgfqpoint{5.462879in}{3.531545in}}%
\pgfpathcurveto{\pgfqpoint{5.462879in}{3.520495in}}{\pgfqpoint{5.467269in}{3.509896in}}{\pgfqpoint{5.475083in}{3.502083in}}%
\pgfpathcurveto{\pgfqpoint{5.482896in}{3.494269in}}{\pgfqpoint{5.493495in}{3.489879in}}{\pgfqpoint{5.504545in}{3.489879in}}%
\pgfpathclose%
\pgfusepath{stroke,fill}%
\end{pgfscope}%
\begin{pgfscope}%
\pgfpathrectangle{\pgfqpoint{0.800000in}{0.528000in}}{\pgfqpoint{4.960000in}{3.696000in}}%
\pgfusepath{clip}%
\pgfsetbuttcap%
\pgfsetroundjoin%
\definecolor{currentfill}{rgb}{0.000000,0.000000,0.000000}%
\pgfsetfillcolor{currentfill}%
\pgfsetlinewidth{1.003750pt}%
\definecolor{currentstroke}{rgb}{0.000000,0.000000,0.000000}%
\pgfsetstrokecolor{currentstroke}%
\pgfsetdash{}{0pt}%
\pgfpathmoveto{\pgfqpoint{5.504545in}{3.392789in}}%
\pgfpathcurveto{\pgfqpoint{5.515596in}{3.392789in}}{\pgfqpoint{5.526195in}{3.397179in}}{\pgfqpoint{5.534008in}{3.404993in}}%
\pgfpathcurveto{\pgfqpoint{5.541822in}{3.412807in}}{\pgfqpoint{5.546212in}{3.423406in}}{\pgfqpoint{5.546212in}{3.434456in}}%
\pgfpathcurveto{\pgfqpoint{5.546212in}{3.445506in}}{\pgfqpoint{5.541822in}{3.456105in}}{\pgfqpoint{5.534008in}{3.463918in}}%
\pgfpathcurveto{\pgfqpoint{5.526195in}{3.471732in}}{\pgfqpoint{5.515596in}{3.476122in}}{\pgfqpoint{5.504545in}{3.476122in}}%
\pgfpathcurveto{\pgfqpoint{5.493495in}{3.476122in}}{\pgfqpoint{5.482896in}{3.471732in}}{\pgfqpoint{5.475083in}{3.463918in}}%
\pgfpathcurveto{\pgfqpoint{5.467269in}{3.456105in}}{\pgfqpoint{5.462879in}{3.445506in}}{\pgfqpoint{5.462879in}{3.434456in}}%
\pgfpathcurveto{\pgfqpoint{5.462879in}{3.423406in}}{\pgfqpoint{5.467269in}{3.412807in}}{\pgfqpoint{5.475083in}{3.404993in}}%
\pgfpathcurveto{\pgfqpoint{5.482896in}{3.397179in}}{\pgfqpoint{5.493495in}{3.392789in}}{\pgfqpoint{5.504545in}{3.392789in}}%
\pgfpathclose%
\pgfusepath{stroke,fill}%
\end{pgfscope}%
\begin{pgfscope}%
\pgfpathrectangle{\pgfqpoint{0.800000in}{0.528000in}}{\pgfqpoint{4.960000in}{3.696000in}}%
\pgfusepath{clip}%
\pgfsetbuttcap%
\pgfsetroundjoin%
\definecolor{currentfill}{rgb}{0.000000,0.000000,0.000000}%
\pgfsetfillcolor{currentfill}%
\pgfsetlinewidth{1.003750pt}%
\definecolor{currentstroke}{rgb}{0.000000,0.000000,0.000000}%
\pgfsetstrokecolor{currentstroke}%
\pgfsetdash{}{0pt}%
\pgfpathmoveto{\pgfqpoint{5.504545in}{3.538424in}}%
\pgfpathcurveto{\pgfqpoint{5.515596in}{3.538424in}}{\pgfqpoint{5.526195in}{3.542814in}}{\pgfqpoint{5.534008in}{3.550627in}}%
\pgfpathcurveto{\pgfqpoint{5.541822in}{3.558441in}}{\pgfqpoint{5.546212in}{3.569040in}}{\pgfqpoint{5.546212in}{3.580090in}}%
\pgfpathcurveto{\pgfqpoint{5.546212in}{3.591140in}}{\pgfqpoint{5.541822in}{3.601739in}}{\pgfqpoint{5.534008in}{3.609553in}}%
\pgfpathcurveto{\pgfqpoint{5.526195in}{3.617367in}}{\pgfqpoint{5.515596in}{3.621757in}}{\pgfqpoint{5.504545in}{3.621757in}}%
\pgfpathcurveto{\pgfqpoint{5.493495in}{3.621757in}}{\pgfqpoint{5.482896in}{3.617367in}}{\pgfqpoint{5.475083in}{3.609553in}}%
\pgfpathcurveto{\pgfqpoint{5.467269in}{3.601739in}}{\pgfqpoint{5.462879in}{3.591140in}}{\pgfqpoint{5.462879in}{3.580090in}}%
\pgfpathcurveto{\pgfqpoint{5.462879in}{3.569040in}}{\pgfqpoint{5.467269in}{3.558441in}}{\pgfqpoint{5.475083in}{3.550627in}}%
\pgfpathcurveto{\pgfqpoint{5.482896in}{3.542814in}}{\pgfqpoint{5.493495in}{3.538424in}}{\pgfqpoint{5.504545in}{3.538424in}}%
\pgfpathclose%
\pgfusepath{stroke,fill}%
\end{pgfscope}%
\begin{pgfscope}%
\pgfpathrectangle{\pgfqpoint{0.800000in}{0.528000in}}{\pgfqpoint{4.960000in}{3.696000in}}%
\pgfusepath{clip}%
\pgfsetbuttcap%
\pgfsetroundjoin%
\definecolor{currentfill}{rgb}{0.000000,0.000000,0.000000}%
\pgfsetfillcolor{currentfill}%
\pgfsetlinewidth{1.003750pt}%
\definecolor{currentstroke}{rgb}{0.000000,0.000000,0.000000}%
\pgfsetstrokecolor{currentstroke}%
\pgfsetdash{}{0pt}%
\pgfpathmoveto{\pgfqpoint{5.504545in}{3.392789in}}%
\pgfpathcurveto{\pgfqpoint{5.515596in}{3.392789in}}{\pgfqpoint{5.526195in}{3.397179in}}{\pgfqpoint{5.534008in}{3.404993in}}%
\pgfpathcurveto{\pgfqpoint{5.541822in}{3.412807in}}{\pgfqpoint{5.546212in}{3.423406in}}{\pgfqpoint{5.546212in}{3.434456in}}%
\pgfpathcurveto{\pgfqpoint{5.546212in}{3.445506in}}{\pgfqpoint{5.541822in}{3.456105in}}{\pgfqpoint{5.534008in}{3.463918in}}%
\pgfpathcurveto{\pgfqpoint{5.526195in}{3.471732in}}{\pgfqpoint{5.515596in}{3.476122in}}{\pgfqpoint{5.504545in}{3.476122in}}%
\pgfpathcurveto{\pgfqpoint{5.493495in}{3.476122in}}{\pgfqpoint{5.482896in}{3.471732in}}{\pgfqpoint{5.475083in}{3.463918in}}%
\pgfpathcurveto{\pgfqpoint{5.467269in}{3.456105in}}{\pgfqpoint{5.462879in}{3.445506in}}{\pgfqpoint{5.462879in}{3.434456in}}%
\pgfpathcurveto{\pgfqpoint{5.462879in}{3.423406in}}{\pgfqpoint{5.467269in}{3.412807in}}{\pgfqpoint{5.475083in}{3.404993in}}%
\pgfpathcurveto{\pgfqpoint{5.482896in}{3.397179in}}{\pgfqpoint{5.493495in}{3.392789in}}{\pgfqpoint{5.504545in}{3.392789in}}%
\pgfpathclose%
\pgfusepath{stroke,fill}%
\end{pgfscope}%
\begin{pgfscope}%
\pgfpathrectangle{\pgfqpoint{0.800000in}{0.528000in}}{\pgfqpoint{4.960000in}{3.696000in}}%
\pgfusepath{clip}%
\pgfsetbuttcap%
\pgfsetroundjoin%
\definecolor{currentfill}{rgb}{0.000000,0.000000,0.000000}%
\pgfsetfillcolor{currentfill}%
\pgfsetlinewidth{1.003750pt}%
\definecolor{currentstroke}{rgb}{0.000000,0.000000,0.000000}%
\pgfsetstrokecolor{currentstroke}%
\pgfsetdash{}{0pt}%
\pgfpathmoveto{\pgfqpoint{5.504545in}{3.150065in}}%
\pgfpathcurveto{\pgfqpoint{5.515596in}{3.150065in}}{\pgfqpoint{5.526195in}{3.154455in}}{\pgfqpoint{5.534008in}{3.162269in}}%
\pgfpathcurveto{\pgfqpoint{5.541822in}{3.170082in}}{\pgfqpoint{5.546212in}{3.180681in}}{\pgfqpoint{5.546212in}{3.191731in}}%
\pgfpathcurveto{\pgfqpoint{5.546212in}{3.202782in}}{\pgfqpoint{5.541822in}{3.213381in}}{\pgfqpoint{5.534008in}{3.221194in}}%
\pgfpathcurveto{\pgfqpoint{5.526195in}{3.229008in}}{\pgfqpoint{5.515596in}{3.233398in}}{\pgfqpoint{5.504545in}{3.233398in}}%
\pgfpathcurveto{\pgfqpoint{5.493495in}{3.233398in}}{\pgfqpoint{5.482896in}{3.229008in}}{\pgfqpoint{5.475083in}{3.221194in}}%
\pgfpathcurveto{\pgfqpoint{5.467269in}{3.213381in}}{\pgfqpoint{5.462879in}{3.202782in}}{\pgfqpoint{5.462879in}{3.191731in}}%
\pgfpathcurveto{\pgfqpoint{5.462879in}{3.180681in}}{\pgfqpoint{5.467269in}{3.170082in}}{\pgfqpoint{5.475083in}{3.162269in}}%
\pgfpathcurveto{\pgfqpoint{5.482896in}{3.154455in}}{\pgfqpoint{5.493495in}{3.150065in}}{\pgfqpoint{5.504545in}{3.150065in}}%
\pgfpathclose%
\pgfusepath{stroke,fill}%
\end{pgfscope}%
\begin{pgfscope}%
\pgfpathrectangle{\pgfqpoint{0.800000in}{0.528000in}}{\pgfqpoint{4.960000in}{3.696000in}}%
\pgfusepath{clip}%
\pgfsetbuttcap%
\pgfsetroundjoin%
\definecolor{currentfill}{rgb}{0.000000,0.000000,0.000000}%
\pgfsetfillcolor{currentfill}%
\pgfsetlinewidth{1.003750pt}%
\definecolor{currentstroke}{rgb}{0.000000,0.000000,0.000000}%
\pgfsetstrokecolor{currentstroke}%
\pgfsetdash{}{0pt}%
\pgfpathmoveto{\pgfqpoint{5.504545in}{3.324826in}}%
\pgfpathcurveto{\pgfqpoint{5.515596in}{3.324826in}}{\pgfqpoint{5.526195in}{3.329217in}}{\pgfqpoint{5.534008in}{3.337030in}}%
\pgfpathcurveto{\pgfqpoint{5.541822in}{3.344844in}}{\pgfqpoint{5.546212in}{3.355443in}}{\pgfqpoint{5.546212in}{3.366493in}}%
\pgfpathcurveto{\pgfqpoint{5.546212in}{3.377543in}}{\pgfqpoint{5.541822in}{3.388142in}}{\pgfqpoint{5.534008in}{3.395956in}}%
\pgfpathcurveto{\pgfqpoint{5.526195in}{3.403769in}}{\pgfqpoint{5.515596in}{3.408160in}}{\pgfqpoint{5.504545in}{3.408160in}}%
\pgfpathcurveto{\pgfqpoint{5.493495in}{3.408160in}}{\pgfqpoint{5.482896in}{3.403769in}}{\pgfqpoint{5.475083in}{3.395956in}}%
\pgfpathcurveto{\pgfqpoint{5.467269in}{3.388142in}}{\pgfqpoint{5.462879in}{3.377543in}}{\pgfqpoint{5.462879in}{3.366493in}}%
\pgfpathcurveto{\pgfqpoint{5.462879in}{3.355443in}}{\pgfqpoint{5.467269in}{3.344844in}}{\pgfqpoint{5.475083in}{3.337030in}}%
\pgfpathcurveto{\pgfqpoint{5.482896in}{3.329217in}}{\pgfqpoint{5.493495in}{3.324826in}}{\pgfqpoint{5.504545in}{3.324826in}}%
\pgfpathclose%
\pgfusepath{stroke,fill}%
\end{pgfscope}%
\begin{pgfscope}%
\pgfpathrectangle{\pgfqpoint{0.800000in}{0.528000in}}{\pgfqpoint{4.960000in}{3.696000in}}%
\pgfusepath{clip}%
\pgfsetbuttcap%
\pgfsetroundjoin%
\definecolor{currentfill}{rgb}{0.000000,0.000000,0.000000}%
\pgfsetfillcolor{currentfill}%
\pgfsetlinewidth{1.003750pt}%
\definecolor{currentstroke}{rgb}{0.000000,0.000000,0.000000}%
\pgfsetstrokecolor{currentstroke}%
\pgfsetdash{}{0pt}%
\pgfpathmoveto{\pgfqpoint{5.504545in}{2.975303in}}%
\pgfpathcurveto{\pgfqpoint{5.515596in}{2.975303in}}{\pgfqpoint{5.526195in}{2.979694in}}{\pgfqpoint{5.534008in}{2.987507in}}%
\pgfpathcurveto{\pgfqpoint{5.541822in}{2.995321in}}{\pgfqpoint{5.546212in}{3.005920in}}{\pgfqpoint{5.546212in}{3.016970in}}%
\pgfpathcurveto{\pgfqpoint{5.546212in}{3.028020in}}{\pgfqpoint{5.541822in}{3.038619in}}{\pgfqpoint{5.534008in}{3.046433in}}%
\pgfpathcurveto{\pgfqpoint{5.526195in}{3.054246in}}{\pgfqpoint{5.515596in}{3.058637in}}{\pgfqpoint{5.504545in}{3.058637in}}%
\pgfpathcurveto{\pgfqpoint{5.493495in}{3.058637in}}{\pgfqpoint{5.482896in}{3.054246in}}{\pgfqpoint{5.475083in}{3.046433in}}%
\pgfpathcurveto{\pgfqpoint{5.467269in}{3.038619in}}{\pgfqpoint{5.462879in}{3.028020in}}{\pgfqpoint{5.462879in}{3.016970in}}%
\pgfpathcurveto{\pgfqpoint{5.462879in}{3.005920in}}{\pgfqpoint{5.467269in}{2.995321in}}{\pgfqpoint{5.475083in}{2.987507in}}%
\pgfpathcurveto{\pgfqpoint{5.482896in}{2.979694in}}{\pgfqpoint{5.493495in}{2.975303in}}{\pgfqpoint{5.504545in}{2.975303in}}%
\pgfpathclose%
\pgfusepath{stroke,fill}%
\end{pgfscope}%
\begin{pgfscope}%
\pgfpathrectangle{\pgfqpoint{0.800000in}{0.528000in}}{\pgfqpoint{4.960000in}{3.696000in}}%
\pgfusepath{clip}%
\pgfsetbuttcap%
\pgfsetroundjoin%
\definecolor{currentfill}{rgb}{0.000000,0.000000,0.000000}%
\pgfsetfillcolor{currentfill}%
\pgfsetlinewidth{1.003750pt}%
\definecolor{currentstroke}{rgb}{0.000000,0.000000,0.000000}%
\pgfsetstrokecolor{currentstroke}%
\pgfsetdash{}{0pt}%
\pgfpathmoveto{\pgfqpoint{5.504545in}{3.130647in}}%
\pgfpathcurveto{\pgfqpoint{5.515596in}{3.130647in}}{\pgfqpoint{5.526195in}{3.135037in}}{\pgfqpoint{5.534008in}{3.142851in}}%
\pgfpathcurveto{\pgfqpoint{5.541822in}{3.150664in}}{\pgfqpoint{5.546212in}{3.161263in}}{\pgfqpoint{5.546212in}{3.172314in}}%
\pgfpathcurveto{\pgfqpoint{5.546212in}{3.183364in}}{\pgfqpoint{5.541822in}{3.193963in}}{\pgfqpoint{5.534008in}{3.201776in}}%
\pgfpathcurveto{\pgfqpoint{5.526195in}{3.209590in}}{\pgfqpoint{5.515596in}{3.213980in}}{\pgfqpoint{5.504545in}{3.213980in}}%
\pgfpathcurveto{\pgfqpoint{5.493495in}{3.213980in}}{\pgfqpoint{5.482896in}{3.209590in}}{\pgfqpoint{5.475083in}{3.201776in}}%
\pgfpathcurveto{\pgfqpoint{5.467269in}{3.193963in}}{\pgfqpoint{5.462879in}{3.183364in}}{\pgfqpoint{5.462879in}{3.172314in}}%
\pgfpathcurveto{\pgfqpoint{5.462879in}{3.161263in}}{\pgfqpoint{5.467269in}{3.150664in}}{\pgfqpoint{5.475083in}{3.142851in}}%
\pgfpathcurveto{\pgfqpoint{5.482896in}{3.135037in}}{\pgfqpoint{5.493495in}{3.130647in}}{\pgfqpoint{5.504545in}{3.130647in}}%
\pgfpathclose%
\pgfusepath{stroke,fill}%
\end{pgfscope}%
\begin{pgfscope}%
\pgfpathrectangle{\pgfqpoint{0.800000in}{0.528000in}}{\pgfqpoint{4.960000in}{3.696000in}}%
\pgfusepath{clip}%
\pgfsetbuttcap%
\pgfsetroundjoin%
\definecolor{currentfill}{rgb}{0.000000,0.000000,0.000000}%
\pgfsetfillcolor{currentfill}%
\pgfsetlinewidth{1.003750pt}%
\definecolor{currentstroke}{rgb}{0.000000,0.000000,0.000000}%
\pgfsetstrokecolor{currentstroke}%
\pgfsetdash{}{0pt}%
\pgfpathmoveto{\pgfqpoint{5.504545in}{3.179192in}}%
\pgfpathcurveto{\pgfqpoint{5.515596in}{3.179192in}}{\pgfqpoint{5.526195in}{3.183582in}}{\pgfqpoint{5.534008in}{3.191396in}}%
\pgfpathcurveto{\pgfqpoint{5.541822in}{3.199209in}}{\pgfqpoint{5.546212in}{3.209808in}}{\pgfqpoint{5.546212in}{3.220858in}}%
\pgfpathcurveto{\pgfqpoint{5.546212in}{3.231909in}}{\pgfqpoint{5.541822in}{3.242508in}}{\pgfqpoint{5.534008in}{3.250321in}}%
\pgfpathcurveto{\pgfqpoint{5.526195in}{3.258135in}}{\pgfqpoint{5.515596in}{3.262525in}}{\pgfqpoint{5.504545in}{3.262525in}}%
\pgfpathcurveto{\pgfqpoint{5.493495in}{3.262525in}}{\pgfqpoint{5.482896in}{3.258135in}}{\pgfqpoint{5.475083in}{3.250321in}}%
\pgfpathcurveto{\pgfqpoint{5.467269in}{3.242508in}}{\pgfqpoint{5.462879in}{3.231909in}}{\pgfqpoint{5.462879in}{3.220858in}}%
\pgfpathcurveto{\pgfqpoint{5.462879in}{3.209808in}}{\pgfqpoint{5.467269in}{3.199209in}}{\pgfqpoint{5.475083in}{3.191396in}}%
\pgfpathcurveto{\pgfqpoint{5.482896in}{3.183582in}}{\pgfqpoint{5.493495in}{3.179192in}}{\pgfqpoint{5.504545in}{3.179192in}}%
\pgfpathclose%
\pgfusepath{stroke,fill}%
\end{pgfscope}%
\begin{pgfscope}%
\pgfpathrectangle{\pgfqpoint{0.800000in}{0.528000in}}{\pgfqpoint{4.960000in}{3.696000in}}%
\pgfusepath{clip}%
\pgfsetbuttcap%
\pgfsetroundjoin%
\definecolor{currentfill}{rgb}{0.000000,0.000000,0.000000}%
\pgfsetfillcolor{currentfill}%
\pgfsetlinewidth{1.003750pt}%
\definecolor{currentstroke}{rgb}{0.000000,0.000000,0.000000}%
\pgfsetstrokecolor{currentstroke}%
\pgfsetdash{}{0pt}%
\pgfpathmoveto{\pgfqpoint{5.504545in}{3.198610in}}%
\pgfpathcurveto{\pgfqpoint{5.515596in}{3.198610in}}{\pgfqpoint{5.526195in}{3.203000in}}{\pgfqpoint{5.534008in}{3.210814in}}%
\pgfpathcurveto{\pgfqpoint{5.541822in}{3.218627in}}{\pgfqpoint{5.546212in}{3.229226in}}{\pgfqpoint{5.546212in}{3.240276in}}%
\pgfpathcurveto{\pgfqpoint{5.546212in}{3.251326in}}{\pgfqpoint{5.541822in}{3.261925in}}{\pgfqpoint{5.534008in}{3.269739in}}%
\pgfpathcurveto{\pgfqpoint{5.526195in}{3.277553in}}{\pgfqpoint{5.515596in}{3.281943in}}{\pgfqpoint{5.504545in}{3.281943in}}%
\pgfpathcurveto{\pgfqpoint{5.493495in}{3.281943in}}{\pgfqpoint{5.482896in}{3.277553in}}{\pgfqpoint{5.475083in}{3.269739in}}%
\pgfpathcurveto{\pgfqpoint{5.467269in}{3.261925in}}{\pgfqpoint{5.462879in}{3.251326in}}{\pgfqpoint{5.462879in}{3.240276in}}%
\pgfpathcurveto{\pgfqpoint{5.462879in}{3.229226in}}{\pgfqpoint{5.467269in}{3.218627in}}{\pgfqpoint{5.475083in}{3.210814in}}%
\pgfpathcurveto{\pgfqpoint{5.482896in}{3.203000in}}{\pgfqpoint{5.493495in}{3.198610in}}{\pgfqpoint{5.504545in}{3.198610in}}%
\pgfpathclose%
\pgfusepath{stroke,fill}%
\end{pgfscope}%
\begin{pgfscope}%
\pgfpathrectangle{\pgfqpoint{0.800000in}{0.528000in}}{\pgfqpoint{4.960000in}{3.696000in}}%
\pgfusepath{clip}%
\pgfsetbuttcap%
\pgfsetroundjoin%
\definecolor{currentfill}{rgb}{0.000000,0.000000,0.000000}%
\pgfsetfillcolor{currentfill}%
\pgfsetlinewidth{1.003750pt}%
\definecolor{currentstroke}{rgb}{0.000000,0.000000,0.000000}%
\pgfsetstrokecolor{currentstroke}%
\pgfsetdash{}{0pt}%
\pgfpathmoveto{\pgfqpoint{5.504545in}{3.208319in}}%
\pgfpathcurveto{\pgfqpoint{5.515596in}{3.208319in}}{\pgfqpoint{5.526195in}{3.212709in}}{\pgfqpoint{5.534008in}{3.220522in}}%
\pgfpathcurveto{\pgfqpoint{5.541822in}{3.228336in}}{\pgfqpoint{5.546212in}{3.238935in}}{\pgfqpoint{5.546212in}{3.249985in}}%
\pgfpathcurveto{\pgfqpoint{5.546212in}{3.261035in}}{\pgfqpoint{5.541822in}{3.271634in}}{\pgfqpoint{5.534008in}{3.279448in}}%
\pgfpathcurveto{\pgfqpoint{5.526195in}{3.287262in}}{\pgfqpoint{5.515596in}{3.291652in}}{\pgfqpoint{5.504545in}{3.291652in}}%
\pgfpathcurveto{\pgfqpoint{5.493495in}{3.291652in}}{\pgfqpoint{5.482896in}{3.287262in}}{\pgfqpoint{5.475083in}{3.279448in}}%
\pgfpathcurveto{\pgfqpoint{5.467269in}{3.271634in}}{\pgfqpoint{5.462879in}{3.261035in}}{\pgfqpoint{5.462879in}{3.249985in}}%
\pgfpathcurveto{\pgfqpoint{5.462879in}{3.238935in}}{\pgfqpoint{5.467269in}{3.228336in}}{\pgfqpoint{5.475083in}{3.220522in}}%
\pgfpathcurveto{\pgfqpoint{5.482896in}{3.212709in}}{\pgfqpoint{5.493495in}{3.208319in}}{\pgfqpoint{5.504545in}{3.208319in}}%
\pgfpathclose%
\pgfusepath{stroke,fill}%
\end{pgfscope}%
\begin{pgfscope}%
\pgfpathrectangle{\pgfqpoint{0.800000in}{0.528000in}}{\pgfqpoint{4.960000in}{3.696000in}}%
\pgfusepath{clip}%
\pgfsetbuttcap%
\pgfsetroundjoin%
\definecolor{currentfill}{rgb}{0.000000,0.000000,0.000000}%
\pgfsetfillcolor{currentfill}%
\pgfsetlinewidth{1.003750pt}%
\definecolor{currentstroke}{rgb}{0.000000,0.000000,0.000000}%
\pgfsetstrokecolor{currentstroke}%
\pgfsetdash{}{0pt}%
\pgfpathmoveto{\pgfqpoint{5.504545in}{3.208319in}}%
\pgfpathcurveto{\pgfqpoint{5.515596in}{3.208319in}}{\pgfqpoint{5.526195in}{3.212709in}}{\pgfqpoint{5.534008in}{3.220522in}}%
\pgfpathcurveto{\pgfqpoint{5.541822in}{3.228336in}}{\pgfqpoint{5.546212in}{3.238935in}}{\pgfqpoint{5.546212in}{3.249985in}}%
\pgfpathcurveto{\pgfqpoint{5.546212in}{3.261035in}}{\pgfqpoint{5.541822in}{3.271634in}}{\pgfqpoint{5.534008in}{3.279448in}}%
\pgfpathcurveto{\pgfqpoint{5.526195in}{3.287262in}}{\pgfqpoint{5.515596in}{3.291652in}}{\pgfqpoint{5.504545in}{3.291652in}}%
\pgfpathcurveto{\pgfqpoint{5.493495in}{3.291652in}}{\pgfqpoint{5.482896in}{3.287262in}}{\pgfqpoint{5.475083in}{3.279448in}}%
\pgfpathcurveto{\pgfqpoint{5.467269in}{3.271634in}}{\pgfqpoint{5.462879in}{3.261035in}}{\pgfqpoint{5.462879in}{3.249985in}}%
\pgfpathcurveto{\pgfqpoint{5.462879in}{3.238935in}}{\pgfqpoint{5.467269in}{3.228336in}}{\pgfqpoint{5.475083in}{3.220522in}}%
\pgfpathcurveto{\pgfqpoint{5.482896in}{3.212709in}}{\pgfqpoint{5.493495in}{3.208319in}}{\pgfqpoint{5.504545in}{3.208319in}}%
\pgfpathclose%
\pgfusepath{stroke,fill}%
\end{pgfscope}%
\begin{pgfscope}%
\pgfpathrectangle{\pgfqpoint{0.800000in}{0.528000in}}{\pgfqpoint{4.960000in}{3.696000in}}%
\pgfusepath{clip}%
\pgfsetbuttcap%
\pgfsetroundjoin%
\definecolor{currentfill}{rgb}{0.000000,0.000000,0.000000}%
\pgfsetfillcolor{currentfill}%
\pgfsetlinewidth{1.003750pt}%
\definecolor{currentstroke}{rgb}{0.000000,0.000000,0.000000}%
\pgfsetstrokecolor{currentstroke}%
\pgfsetdash{}{0pt}%
\pgfpathmoveto{\pgfqpoint{5.504545in}{3.519006in}}%
\pgfpathcurveto{\pgfqpoint{5.515596in}{3.519006in}}{\pgfqpoint{5.526195in}{3.523396in}}{\pgfqpoint{5.534008in}{3.531210in}}%
\pgfpathcurveto{\pgfqpoint{5.541822in}{3.539023in}}{\pgfqpoint{5.546212in}{3.549622in}}{\pgfqpoint{5.546212in}{3.560672in}}%
\pgfpathcurveto{\pgfqpoint{5.546212in}{3.571722in}}{\pgfqpoint{5.541822in}{3.582321in}}{\pgfqpoint{5.534008in}{3.590135in}}%
\pgfpathcurveto{\pgfqpoint{5.526195in}{3.597949in}}{\pgfqpoint{5.515596in}{3.602339in}}{\pgfqpoint{5.504545in}{3.602339in}}%
\pgfpathcurveto{\pgfqpoint{5.493495in}{3.602339in}}{\pgfqpoint{5.482896in}{3.597949in}}{\pgfqpoint{5.475083in}{3.590135in}}%
\pgfpathcurveto{\pgfqpoint{5.467269in}{3.582321in}}{\pgfqpoint{5.462879in}{3.571722in}}{\pgfqpoint{5.462879in}{3.560672in}}%
\pgfpathcurveto{\pgfqpoint{5.462879in}{3.549622in}}{\pgfqpoint{5.467269in}{3.539023in}}{\pgfqpoint{5.475083in}{3.531210in}}%
\pgfpathcurveto{\pgfqpoint{5.482896in}{3.523396in}}{\pgfqpoint{5.493495in}{3.519006in}}{\pgfqpoint{5.504545in}{3.519006in}}%
\pgfpathclose%
\pgfusepath{stroke,fill}%
\end{pgfscope}%
\begin{pgfscope}%
\pgfpathrectangle{\pgfqpoint{0.800000in}{0.528000in}}{\pgfqpoint{4.960000in}{3.696000in}}%
\pgfusepath{clip}%
\pgfsetbuttcap%
\pgfsetroundjoin%
\definecolor{currentfill}{rgb}{0.000000,0.000000,0.000000}%
\pgfsetfillcolor{currentfill}%
\pgfsetlinewidth{1.003750pt}%
\definecolor{currentstroke}{rgb}{0.000000,0.000000,0.000000}%
\pgfsetstrokecolor{currentstroke}%
\pgfsetdash{}{0pt}%
\pgfpathmoveto{\pgfqpoint{5.504545in}{3.101520in}}%
\pgfpathcurveto{\pgfqpoint{5.515596in}{3.101520in}}{\pgfqpoint{5.526195in}{3.105910in}}{\pgfqpoint{5.534008in}{3.113724in}}%
\pgfpathcurveto{\pgfqpoint{5.541822in}{3.121537in}}{\pgfqpoint{5.546212in}{3.132136in}}{\pgfqpoint{5.546212in}{3.143187in}}%
\pgfpathcurveto{\pgfqpoint{5.546212in}{3.154237in}}{\pgfqpoint{5.541822in}{3.164836in}}{\pgfqpoint{5.534008in}{3.172649in}}%
\pgfpathcurveto{\pgfqpoint{5.526195in}{3.180463in}}{\pgfqpoint{5.515596in}{3.184853in}}{\pgfqpoint{5.504545in}{3.184853in}}%
\pgfpathcurveto{\pgfqpoint{5.493495in}{3.184853in}}{\pgfqpoint{5.482896in}{3.180463in}}{\pgfqpoint{5.475083in}{3.172649in}}%
\pgfpathcurveto{\pgfqpoint{5.467269in}{3.164836in}}{\pgfqpoint{5.462879in}{3.154237in}}{\pgfqpoint{5.462879in}{3.143187in}}%
\pgfpathcurveto{\pgfqpoint{5.462879in}{3.132136in}}{\pgfqpoint{5.467269in}{3.121537in}}{\pgfqpoint{5.475083in}{3.113724in}}%
\pgfpathcurveto{\pgfqpoint{5.482896in}{3.105910in}}{\pgfqpoint{5.493495in}{3.101520in}}{\pgfqpoint{5.504545in}{3.101520in}}%
\pgfpathclose%
\pgfusepath{stroke,fill}%
\end{pgfscope}%
\begin{pgfscope}%
\pgfpathrectangle{\pgfqpoint{0.800000in}{0.528000in}}{\pgfqpoint{4.960000in}{3.696000in}}%
\pgfusepath{clip}%
\pgfsetbuttcap%
\pgfsetroundjoin%
\definecolor{currentfill}{rgb}{0.000000,0.000000,0.000000}%
\pgfsetfillcolor{currentfill}%
\pgfsetlinewidth{1.003750pt}%
\definecolor{currentstroke}{rgb}{0.000000,0.000000,0.000000}%
\pgfsetstrokecolor{currentstroke}%
\pgfsetdash{}{0pt}%
\pgfpathmoveto{\pgfqpoint{5.504545in}{2.985012in}}%
\pgfpathcurveto{\pgfqpoint{5.515596in}{2.985012in}}{\pgfqpoint{5.526195in}{2.989403in}}{\pgfqpoint{5.534008in}{2.997216in}}%
\pgfpathcurveto{\pgfqpoint{5.541822in}{3.005030in}}{\pgfqpoint{5.546212in}{3.015629in}}{\pgfqpoint{5.546212in}{3.026679in}}%
\pgfpathcurveto{\pgfqpoint{5.546212in}{3.037729in}}{\pgfqpoint{5.541822in}{3.048328in}}{\pgfqpoint{5.534008in}{3.056142in}}%
\pgfpathcurveto{\pgfqpoint{5.526195in}{3.063955in}}{\pgfqpoint{5.515596in}{3.068346in}}{\pgfqpoint{5.504545in}{3.068346in}}%
\pgfpathcurveto{\pgfqpoint{5.493495in}{3.068346in}}{\pgfqpoint{5.482896in}{3.063955in}}{\pgfqpoint{5.475083in}{3.056142in}}%
\pgfpathcurveto{\pgfqpoint{5.467269in}{3.048328in}}{\pgfqpoint{5.462879in}{3.037729in}}{\pgfqpoint{5.462879in}{3.026679in}}%
\pgfpathcurveto{\pgfqpoint{5.462879in}{3.015629in}}{\pgfqpoint{5.467269in}{3.005030in}}{\pgfqpoint{5.475083in}{2.997216in}}%
\pgfpathcurveto{\pgfqpoint{5.482896in}{2.989403in}}{\pgfqpoint{5.493495in}{2.985012in}}{\pgfqpoint{5.504545in}{2.985012in}}%
\pgfpathclose%
\pgfusepath{stroke,fill}%
\end{pgfscope}%
\begin{pgfscope}%
\pgfpathrectangle{\pgfqpoint{0.800000in}{0.528000in}}{\pgfqpoint{4.960000in}{3.696000in}}%
\pgfusepath{clip}%
\pgfsetbuttcap%
\pgfsetroundjoin%
\definecolor{currentfill}{rgb}{0.000000,0.000000,0.000000}%
\pgfsetfillcolor{currentfill}%
\pgfsetlinewidth{1.003750pt}%
\definecolor{currentstroke}{rgb}{0.000000,0.000000,0.000000}%
\pgfsetstrokecolor{currentstroke}%
\pgfsetdash{}{0pt}%
\pgfpathmoveto{\pgfqpoint{5.504545in}{3.538424in}}%
\pgfpathcurveto{\pgfqpoint{5.515596in}{3.538424in}}{\pgfqpoint{5.526195in}{3.542814in}}{\pgfqpoint{5.534008in}{3.550627in}}%
\pgfpathcurveto{\pgfqpoint{5.541822in}{3.558441in}}{\pgfqpoint{5.546212in}{3.569040in}}{\pgfqpoint{5.546212in}{3.580090in}}%
\pgfpathcurveto{\pgfqpoint{5.546212in}{3.591140in}}{\pgfqpoint{5.541822in}{3.601739in}}{\pgfqpoint{5.534008in}{3.609553in}}%
\pgfpathcurveto{\pgfqpoint{5.526195in}{3.617367in}}{\pgfqpoint{5.515596in}{3.621757in}}{\pgfqpoint{5.504545in}{3.621757in}}%
\pgfpathcurveto{\pgfqpoint{5.493495in}{3.621757in}}{\pgfqpoint{5.482896in}{3.617367in}}{\pgfqpoint{5.475083in}{3.609553in}}%
\pgfpathcurveto{\pgfqpoint{5.467269in}{3.601739in}}{\pgfqpoint{5.462879in}{3.591140in}}{\pgfqpoint{5.462879in}{3.580090in}}%
\pgfpathcurveto{\pgfqpoint{5.462879in}{3.569040in}}{\pgfqpoint{5.467269in}{3.558441in}}{\pgfqpoint{5.475083in}{3.550627in}}%
\pgfpathcurveto{\pgfqpoint{5.482896in}{3.542814in}}{\pgfqpoint{5.493495in}{3.538424in}}{\pgfqpoint{5.504545in}{3.538424in}}%
\pgfpathclose%
\pgfusepath{stroke,fill}%
\end{pgfscope}%
\begin{pgfscope}%
\pgfpathrectangle{\pgfqpoint{0.800000in}{0.528000in}}{\pgfqpoint{4.960000in}{3.696000in}}%
\pgfusepath{clip}%
\pgfsetbuttcap%
\pgfsetroundjoin%
\definecolor{currentfill}{rgb}{0.000000,0.000000,0.000000}%
\pgfsetfillcolor{currentfill}%
\pgfsetlinewidth{1.003750pt}%
\definecolor{currentstroke}{rgb}{0.000000,0.000000,0.000000}%
\pgfsetstrokecolor{currentstroke}%
\pgfsetdash{}{0pt}%
\pgfpathmoveto{\pgfqpoint{5.504545in}{3.150065in}}%
\pgfpathcurveto{\pgfqpoint{5.515596in}{3.150065in}}{\pgfqpoint{5.526195in}{3.154455in}}{\pgfqpoint{5.534008in}{3.162269in}}%
\pgfpathcurveto{\pgfqpoint{5.541822in}{3.170082in}}{\pgfqpoint{5.546212in}{3.180681in}}{\pgfqpoint{5.546212in}{3.191731in}}%
\pgfpathcurveto{\pgfqpoint{5.546212in}{3.202782in}}{\pgfqpoint{5.541822in}{3.213381in}}{\pgfqpoint{5.534008in}{3.221194in}}%
\pgfpathcurveto{\pgfqpoint{5.526195in}{3.229008in}}{\pgfqpoint{5.515596in}{3.233398in}}{\pgfqpoint{5.504545in}{3.233398in}}%
\pgfpathcurveto{\pgfqpoint{5.493495in}{3.233398in}}{\pgfqpoint{5.482896in}{3.229008in}}{\pgfqpoint{5.475083in}{3.221194in}}%
\pgfpathcurveto{\pgfqpoint{5.467269in}{3.213381in}}{\pgfqpoint{5.462879in}{3.202782in}}{\pgfqpoint{5.462879in}{3.191731in}}%
\pgfpathcurveto{\pgfqpoint{5.462879in}{3.180681in}}{\pgfqpoint{5.467269in}{3.170082in}}{\pgfqpoint{5.475083in}{3.162269in}}%
\pgfpathcurveto{\pgfqpoint{5.482896in}{3.154455in}}{\pgfqpoint{5.493495in}{3.150065in}}{\pgfqpoint{5.504545in}{3.150065in}}%
\pgfpathclose%
\pgfusepath{stroke,fill}%
\end{pgfscope}%
\begin{pgfscope}%
\pgfpathrectangle{\pgfqpoint{0.800000in}{0.528000in}}{\pgfqpoint{4.960000in}{3.696000in}}%
\pgfusepath{clip}%
\pgfsetbuttcap%
\pgfsetroundjoin%
\definecolor{currentfill}{rgb}{0.000000,0.000000,0.000000}%
\pgfsetfillcolor{currentfill}%
\pgfsetlinewidth{1.003750pt}%
\definecolor{currentstroke}{rgb}{0.000000,0.000000,0.000000}%
\pgfsetstrokecolor{currentstroke}%
\pgfsetdash{}{0pt}%
\pgfpathmoveto{\pgfqpoint{5.504545in}{3.072393in}}%
\pgfpathcurveto{\pgfqpoint{5.515596in}{3.072393in}}{\pgfqpoint{5.526195in}{3.076783in}}{\pgfqpoint{5.534008in}{3.084597in}}%
\pgfpathcurveto{\pgfqpoint{5.541822in}{3.092411in}}{\pgfqpoint{5.546212in}{3.103010in}}{\pgfqpoint{5.546212in}{3.114060in}}%
\pgfpathcurveto{\pgfqpoint{5.546212in}{3.125110in}}{\pgfqpoint{5.541822in}{3.135709in}}{\pgfqpoint{5.534008in}{3.143522in}}%
\pgfpathcurveto{\pgfqpoint{5.526195in}{3.151336in}}{\pgfqpoint{5.515596in}{3.155726in}}{\pgfqpoint{5.504545in}{3.155726in}}%
\pgfpathcurveto{\pgfqpoint{5.493495in}{3.155726in}}{\pgfqpoint{5.482896in}{3.151336in}}{\pgfqpoint{5.475083in}{3.143522in}}%
\pgfpathcurveto{\pgfqpoint{5.467269in}{3.135709in}}{\pgfqpoint{5.462879in}{3.125110in}}{\pgfqpoint{5.462879in}{3.114060in}}%
\pgfpathcurveto{\pgfqpoint{5.462879in}{3.103010in}}{\pgfqpoint{5.467269in}{3.092411in}}{\pgfqpoint{5.475083in}{3.084597in}}%
\pgfpathcurveto{\pgfqpoint{5.482896in}{3.076783in}}{\pgfqpoint{5.493495in}{3.072393in}}{\pgfqpoint{5.504545in}{3.072393in}}%
\pgfpathclose%
\pgfusepath{stroke,fill}%
\end{pgfscope}%
\begin{pgfscope}%
\pgfpathrectangle{\pgfqpoint{0.800000in}{0.528000in}}{\pgfqpoint{4.960000in}{3.696000in}}%
\pgfusepath{clip}%
\pgfsetbuttcap%
\pgfsetroundjoin%
\definecolor{currentfill}{rgb}{0.000000,0.000000,0.000000}%
\pgfsetfillcolor{currentfill}%
\pgfsetlinewidth{1.003750pt}%
\definecolor{currentstroke}{rgb}{0.000000,0.000000,0.000000}%
\pgfsetstrokecolor{currentstroke}%
\pgfsetdash{}{0pt}%
\pgfpathmoveto{\pgfqpoint{5.504545in}{3.781148in}}%
\pgfpathcurveto{\pgfqpoint{5.515596in}{3.781148in}}{\pgfqpoint{5.526195in}{3.785538in}}{\pgfqpoint{5.534008in}{3.793352in}}%
\pgfpathcurveto{\pgfqpoint{5.541822in}{3.801165in}}{\pgfqpoint{5.546212in}{3.811764in}}{\pgfqpoint{5.546212in}{3.822814in}}%
\pgfpathcurveto{\pgfqpoint{5.546212in}{3.833865in}}{\pgfqpoint{5.541822in}{3.844464in}}{\pgfqpoint{5.534008in}{3.852277in}}%
\pgfpathcurveto{\pgfqpoint{5.526195in}{3.860091in}}{\pgfqpoint{5.515596in}{3.864481in}}{\pgfqpoint{5.504545in}{3.864481in}}%
\pgfpathcurveto{\pgfqpoint{5.493495in}{3.864481in}}{\pgfqpoint{5.482896in}{3.860091in}}{\pgfqpoint{5.475083in}{3.852277in}}%
\pgfpathcurveto{\pgfqpoint{5.467269in}{3.844464in}}{\pgfqpoint{5.462879in}{3.833865in}}{\pgfqpoint{5.462879in}{3.822814in}}%
\pgfpathcurveto{\pgfqpoint{5.462879in}{3.811764in}}{\pgfqpoint{5.467269in}{3.801165in}}{\pgfqpoint{5.475083in}{3.793352in}}%
\pgfpathcurveto{\pgfqpoint{5.482896in}{3.785538in}}{\pgfqpoint{5.493495in}{3.781148in}}{\pgfqpoint{5.504545in}{3.781148in}}%
\pgfpathclose%
\pgfusepath{stroke,fill}%
\end{pgfscope}%
\begin{pgfscope}%
\pgfpathrectangle{\pgfqpoint{0.800000in}{0.528000in}}{\pgfqpoint{4.960000in}{3.696000in}}%
\pgfusepath{clip}%
\pgfsetbuttcap%
\pgfsetroundjoin%
\definecolor{currentfill}{rgb}{0.000000,0.000000,0.000000}%
\pgfsetfillcolor{currentfill}%
\pgfsetlinewidth{1.003750pt}%
\definecolor{currentstroke}{rgb}{0.000000,0.000000,0.000000}%
\pgfsetstrokecolor{currentstroke}%
\pgfsetdash{}{0pt}%
\pgfpathmoveto{\pgfqpoint{5.504545in}{3.276281in}}%
\pgfpathcurveto{\pgfqpoint{5.515596in}{3.276281in}}{\pgfqpoint{5.526195in}{3.280672in}}{\pgfqpoint{5.534008in}{3.288485in}}%
\pgfpathcurveto{\pgfqpoint{5.541822in}{3.296299in}}{\pgfqpoint{5.546212in}{3.306898in}}{\pgfqpoint{5.546212in}{3.317948in}}%
\pgfpathcurveto{\pgfqpoint{5.546212in}{3.328998in}}{\pgfqpoint{5.541822in}{3.339597in}}{\pgfqpoint{5.534008in}{3.347411in}}%
\pgfpathcurveto{\pgfqpoint{5.526195in}{3.355224in}}{\pgfqpoint{5.515596in}{3.359615in}}{\pgfqpoint{5.504545in}{3.359615in}}%
\pgfpathcurveto{\pgfqpoint{5.493495in}{3.359615in}}{\pgfqpoint{5.482896in}{3.355224in}}{\pgfqpoint{5.475083in}{3.347411in}}%
\pgfpathcurveto{\pgfqpoint{5.467269in}{3.339597in}}{\pgfqpoint{5.462879in}{3.328998in}}{\pgfqpoint{5.462879in}{3.317948in}}%
\pgfpathcurveto{\pgfqpoint{5.462879in}{3.306898in}}{\pgfqpoint{5.467269in}{3.296299in}}{\pgfqpoint{5.475083in}{3.288485in}}%
\pgfpathcurveto{\pgfqpoint{5.482896in}{3.280672in}}{\pgfqpoint{5.493495in}{3.276281in}}{\pgfqpoint{5.504545in}{3.276281in}}%
\pgfpathclose%
\pgfusepath{stroke,fill}%
\end{pgfscope}%
\begin{pgfscope}%
\pgfpathrectangle{\pgfqpoint{0.800000in}{0.528000in}}{\pgfqpoint{4.960000in}{3.696000in}}%
\pgfusepath{clip}%
\pgfsetbuttcap%
\pgfsetroundjoin%
\definecolor{currentfill}{rgb}{0.000000,0.000000,0.000000}%
\pgfsetfillcolor{currentfill}%
\pgfsetlinewidth{1.003750pt}%
\definecolor{currentstroke}{rgb}{0.000000,0.000000,0.000000}%
\pgfsetstrokecolor{currentstroke}%
\pgfsetdash{}{0pt}%
\pgfpathmoveto{\pgfqpoint{5.504545in}{3.101520in}}%
\pgfpathcurveto{\pgfqpoint{5.515596in}{3.101520in}}{\pgfqpoint{5.526195in}{3.105910in}}{\pgfqpoint{5.534008in}{3.113724in}}%
\pgfpathcurveto{\pgfqpoint{5.541822in}{3.121537in}}{\pgfqpoint{5.546212in}{3.132136in}}{\pgfqpoint{5.546212in}{3.143187in}}%
\pgfpathcurveto{\pgfqpoint{5.546212in}{3.154237in}}{\pgfqpoint{5.541822in}{3.164836in}}{\pgfqpoint{5.534008in}{3.172649in}}%
\pgfpathcurveto{\pgfqpoint{5.526195in}{3.180463in}}{\pgfqpoint{5.515596in}{3.184853in}}{\pgfqpoint{5.504545in}{3.184853in}}%
\pgfpathcurveto{\pgfqpoint{5.493495in}{3.184853in}}{\pgfqpoint{5.482896in}{3.180463in}}{\pgfqpoint{5.475083in}{3.172649in}}%
\pgfpathcurveto{\pgfqpoint{5.467269in}{3.164836in}}{\pgfqpoint{5.462879in}{3.154237in}}{\pgfqpoint{5.462879in}{3.143187in}}%
\pgfpathcurveto{\pgfqpoint{5.462879in}{3.132136in}}{\pgfqpoint{5.467269in}{3.121537in}}{\pgfqpoint{5.475083in}{3.113724in}}%
\pgfpathcurveto{\pgfqpoint{5.482896in}{3.105910in}}{\pgfqpoint{5.493495in}{3.101520in}}{\pgfqpoint{5.504545in}{3.101520in}}%
\pgfpathclose%
\pgfusepath{stroke,fill}%
\end{pgfscope}%
\begin{pgfscope}%
\pgfpathrectangle{\pgfqpoint{0.800000in}{0.528000in}}{\pgfqpoint{4.960000in}{3.696000in}}%
\pgfusepath{clip}%
\pgfsetbuttcap%
\pgfsetroundjoin%
\definecolor{currentfill}{rgb}{0.000000,0.000000,0.000000}%
\pgfsetfillcolor{currentfill}%
\pgfsetlinewidth{1.003750pt}%
\definecolor{currentstroke}{rgb}{0.000000,0.000000,0.000000}%
\pgfsetstrokecolor{currentstroke}%
\pgfsetdash{}{0pt}%
\pgfpathmoveto{\pgfqpoint{5.504545in}{3.334535in}}%
\pgfpathcurveto{\pgfqpoint{5.515596in}{3.334535in}}{\pgfqpoint{5.526195in}{3.338925in}}{\pgfqpoint{5.534008in}{3.346739in}}%
\pgfpathcurveto{\pgfqpoint{5.541822in}{3.354553in}}{\pgfqpoint{5.546212in}{3.365152in}}{\pgfqpoint{5.546212in}{3.376202in}}%
\pgfpathcurveto{\pgfqpoint{5.546212in}{3.387252in}}{\pgfqpoint{5.541822in}{3.397851in}}{\pgfqpoint{5.534008in}{3.405665in}}%
\pgfpathcurveto{\pgfqpoint{5.526195in}{3.413478in}}{\pgfqpoint{5.515596in}{3.417869in}}{\pgfqpoint{5.504545in}{3.417869in}}%
\pgfpathcurveto{\pgfqpoint{5.493495in}{3.417869in}}{\pgfqpoint{5.482896in}{3.413478in}}{\pgfqpoint{5.475083in}{3.405665in}}%
\pgfpathcurveto{\pgfqpoint{5.467269in}{3.397851in}}{\pgfqpoint{5.462879in}{3.387252in}}{\pgfqpoint{5.462879in}{3.376202in}}%
\pgfpathcurveto{\pgfqpoint{5.462879in}{3.365152in}}{\pgfqpoint{5.467269in}{3.354553in}}{\pgfqpoint{5.475083in}{3.346739in}}%
\pgfpathcurveto{\pgfqpoint{5.482896in}{3.338925in}}{\pgfqpoint{5.493495in}{3.334535in}}{\pgfqpoint{5.504545in}{3.334535in}}%
\pgfpathclose%
\pgfusepath{stroke,fill}%
\end{pgfscope}%
\begin{pgfscope}%
\pgfpathrectangle{\pgfqpoint{0.800000in}{0.528000in}}{\pgfqpoint{4.960000in}{3.696000in}}%
\pgfusepath{clip}%
\pgfsetbuttcap%
\pgfsetroundjoin%
\definecolor{currentfill}{rgb}{0.000000,0.000000,0.000000}%
\pgfsetfillcolor{currentfill}%
\pgfsetlinewidth{1.003750pt}%
\definecolor{currentstroke}{rgb}{0.000000,0.000000,0.000000}%
\pgfsetstrokecolor{currentstroke}%
\pgfsetdash{}{0pt}%
\pgfpathmoveto{\pgfqpoint{5.504545in}{3.295699in}}%
\pgfpathcurveto{\pgfqpoint{5.515596in}{3.295699in}}{\pgfqpoint{5.526195in}{3.300090in}}{\pgfqpoint{5.534008in}{3.307903in}}%
\pgfpathcurveto{\pgfqpoint{5.541822in}{3.315717in}}{\pgfqpoint{5.546212in}{3.326316in}}{\pgfqpoint{5.546212in}{3.337366in}}%
\pgfpathcurveto{\pgfqpoint{5.546212in}{3.348416in}}{\pgfqpoint{5.541822in}{3.359015in}}{\pgfqpoint{5.534008in}{3.366829in}}%
\pgfpathcurveto{\pgfqpoint{5.526195in}{3.374642in}}{\pgfqpoint{5.515596in}{3.379033in}}{\pgfqpoint{5.504545in}{3.379033in}}%
\pgfpathcurveto{\pgfqpoint{5.493495in}{3.379033in}}{\pgfqpoint{5.482896in}{3.374642in}}{\pgfqpoint{5.475083in}{3.366829in}}%
\pgfpathcurveto{\pgfqpoint{5.467269in}{3.359015in}}{\pgfqpoint{5.462879in}{3.348416in}}{\pgfqpoint{5.462879in}{3.337366in}}%
\pgfpathcurveto{\pgfqpoint{5.462879in}{3.326316in}}{\pgfqpoint{5.467269in}{3.315717in}}{\pgfqpoint{5.475083in}{3.307903in}}%
\pgfpathcurveto{\pgfqpoint{5.482896in}{3.300090in}}{\pgfqpoint{5.493495in}{3.295699in}}{\pgfqpoint{5.504545in}{3.295699in}}%
\pgfpathclose%
\pgfusepath{stroke,fill}%
\end{pgfscope}%
\begin{pgfscope}%
\pgfpathrectangle{\pgfqpoint{0.800000in}{0.528000in}}{\pgfqpoint{4.960000in}{3.696000in}}%
\pgfusepath{clip}%
\pgfsetbuttcap%
\pgfsetroundjoin%
\definecolor{currentfill}{rgb}{0.000000,0.000000,0.000000}%
\pgfsetfillcolor{currentfill}%
\pgfsetlinewidth{1.003750pt}%
\definecolor{currentstroke}{rgb}{0.000000,0.000000,0.000000}%
\pgfsetstrokecolor{currentstroke}%
\pgfsetdash{}{0pt}%
\pgfpathmoveto{\pgfqpoint{5.504545in}{3.907364in}}%
\pgfpathcurveto{\pgfqpoint{5.515596in}{3.907364in}}{\pgfqpoint{5.526195in}{3.911755in}}{\pgfqpoint{5.534008in}{3.919568in}}%
\pgfpathcurveto{\pgfqpoint{5.541822in}{3.927382in}}{\pgfqpoint{5.546212in}{3.937981in}}{\pgfqpoint{5.546212in}{3.949031in}}%
\pgfpathcurveto{\pgfqpoint{5.546212in}{3.960081in}}{\pgfqpoint{5.541822in}{3.970680in}}{\pgfqpoint{5.534008in}{3.978494in}}%
\pgfpathcurveto{\pgfqpoint{5.526195in}{3.986307in}}{\pgfqpoint{5.515596in}{3.990698in}}{\pgfqpoint{5.504545in}{3.990698in}}%
\pgfpathcurveto{\pgfqpoint{5.493495in}{3.990698in}}{\pgfqpoint{5.482896in}{3.986307in}}{\pgfqpoint{5.475083in}{3.978494in}}%
\pgfpathcurveto{\pgfqpoint{5.467269in}{3.970680in}}{\pgfqpoint{5.462879in}{3.960081in}}{\pgfqpoint{5.462879in}{3.949031in}}%
\pgfpathcurveto{\pgfqpoint{5.462879in}{3.937981in}}{\pgfqpoint{5.467269in}{3.927382in}}{\pgfqpoint{5.475083in}{3.919568in}}%
\pgfpathcurveto{\pgfqpoint{5.482896in}{3.911755in}}{\pgfqpoint{5.493495in}{3.907364in}}{\pgfqpoint{5.504545in}{3.907364in}}%
\pgfpathclose%
\pgfusepath{stroke,fill}%
\end{pgfscope}%
\begin{pgfscope}%
\pgfpathrectangle{\pgfqpoint{0.800000in}{0.528000in}}{\pgfqpoint{4.960000in}{3.696000in}}%
\pgfusepath{clip}%
\pgfsetbuttcap%
\pgfsetroundjoin%
\definecolor{currentfill}{rgb}{0.000000,0.000000,0.000000}%
\pgfsetfillcolor{currentfill}%
\pgfsetlinewidth{1.003750pt}%
\definecolor{currentstroke}{rgb}{0.000000,0.000000,0.000000}%
\pgfsetstrokecolor{currentstroke}%
\pgfsetdash{}{0pt}%
\pgfpathmoveto{\pgfqpoint{5.504545in}{3.528715in}}%
\pgfpathcurveto{\pgfqpoint{5.515596in}{3.528715in}}{\pgfqpoint{5.526195in}{3.533105in}}{\pgfqpoint{5.534008in}{3.540918in}}%
\pgfpathcurveto{\pgfqpoint{5.541822in}{3.548732in}}{\pgfqpoint{5.546212in}{3.559331in}}{\pgfqpoint{5.546212in}{3.570381in}}%
\pgfpathcurveto{\pgfqpoint{5.546212in}{3.581431in}}{\pgfqpoint{5.541822in}{3.592030in}}{\pgfqpoint{5.534008in}{3.599844in}}%
\pgfpathcurveto{\pgfqpoint{5.526195in}{3.607658in}}{\pgfqpoint{5.515596in}{3.612048in}}{\pgfqpoint{5.504545in}{3.612048in}}%
\pgfpathcurveto{\pgfqpoint{5.493495in}{3.612048in}}{\pgfqpoint{5.482896in}{3.607658in}}{\pgfqpoint{5.475083in}{3.599844in}}%
\pgfpathcurveto{\pgfqpoint{5.467269in}{3.592030in}}{\pgfqpoint{5.462879in}{3.581431in}}{\pgfqpoint{5.462879in}{3.570381in}}%
\pgfpathcurveto{\pgfqpoint{5.462879in}{3.559331in}}{\pgfqpoint{5.467269in}{3.548732in}}{\pgfqpoint{5.475083in}{3.540918in}}%
\pgfpathcurveto{\pgfqpoint{5.482896in}{3.533105in}}{\pgfqpoint{5.493495in}{3.528715in}}{\pgfqpoint{5.504545in}{3.528715in}}%
\pgfpathclose%
\pgfusepath{stroke,fill}%
\end{pgfscope}%
\begin{pgfscope}%
\pgfpathrectangle{\pgfqpoint{0.800000in}{0.528000in}}{\pgfqpoint{4.960000in}{3.696000in}}%
\pgfusepath{clip}%
\pgfsetbuttcap%
\pgfsetroundjoin%
\definecolor{currentfill}{rgb}{0.000000,0.000000,0.000000}%
\pgfsetfillcolor{currentfill}%
\pgfsetlinewidth{1.003750pt}%
\definecolor{currentstroke}{rgb}{0.000000,0.000000,0.000000}%
\pgfsetstrokecolor{currentstroke}%
\pgfsetdash{}{0pt}%
\pgfpathmoveto{\pgfqpoint{5.504545in}{3.256863in}}%
\pgfpathcurveto{\pgfqpoint{5.515596in}{3.256863in}}{\pgfqpoint{5.526195in}{3.261254in}}{\pgfqpoint{5.534008in}{3.269067in}}%
\pgfpathcurveto{\pgfqpoint{5.541822in}{3.276881in}}{\pgfqpoint{5.546212in}{3.287480in}}{\pgfqpoint{5.546212in}{3.298530in}}%
\pgfpathcurveto{\pgfqpoint{5.546212in}{3.309580in}}{\pgfqpoint{5.541822in}{3.320179in}}{\pgfqpoint{5.534008in}{3.327993in}}%
\pgfpathcurveto{\pgfqpoint{5.526195in}{3.335807in}}{\pgfqpoint{5.515596in}{3.340197in}}{\pgfqpoint{5.504545in}{3.340197in}}%
\pgfpathcurveto{\pgfqpoint{5.493495in}{3.340197in}}{\pgfqpoint{5.482896in}{3.335807in}}{\pgfqpoint{5.475083in}{3.327993in}}%
\pgfpathcurveto{\pgfqpoint{5.467269in}{3.320179in}}{\pgfqpoint{5.462879in}{3.309580in}}{\pgfqpoint{5.462879in}{3.298530in}}%
\pgfpathcurveto{\pgfqpoint{5.462879in}{3.287480in}}{\pgfqpoint{5.467269in}{3.276881in}}{\pgfqpoint{5.475083in}{3.269067in}}%
\pgfpathcurveto{\pgfqpoint{5.482896in}{3.261254in}}{\pgfqpoint{5.493495in}{3.256863in}}{\pgfqpoint{5.504545in}{3.256863in}}%
\pgfpathclose%
\pgfusepath{stroke,fill}%
\end{pgfscope}%
\begin{pgfscope}%
\pgfpathrectangle{\pgfqpoint{0.800000in}{0.528000in}}{\pgfqpoint{4.960000in}{3.696000in}}%
\pgfusepath{clip}%
\pgfsetbuttcap%
\pgfsetroundjoin%
\definecolor{currentfill}{rgb}{0.000000,0.000000,0.000000}%
\pgfsetfillcolor{currentfill}%
\pgfsetlinewidth{1.003750pt}%
\definecolor{currentstroke}{rgb}{0.000000,0.000000,0.000000}%
\pgfsetstrokecolor{currentstroke}%
\pgfsetdash{}{0pt}%
\pgfpathmoveto{\pgfqpoint{5.504545in}{3.101520in}}%
\pgfpathcurveto{\pgfqpoint{5.515596in}{3.101520in}}{\pgfqpoint{5.526195in}{3.105910in}}{\pgfqpoint{5.534008in}{3.113724in}}%
\pgfpathcurveto{\pgfqpoint{5.541822in}{3.121537in}}{\pgfqpoint{5.546212in}{3.132136in}}{\pgfqpoint{5.546212in}{3.143187in}}%
\pgfpathcurveto{\pgfqpoint{5.546212in}{3.154237in}}{\pgfqpoint{5.541822in}{3.164836in}}{\pgfqpoint{5.534008in}{3.172649in}}%
\pgfpathcurveto{\pgfqpoint{5.526195in}{3.180463in}}{\pgfqpoint{5.515596in}{3.184853in}}{\pgfqpoint{5.504545in}{3.184853in}}%
\pgfpathcurveto{\pgfqpoint{5.493495in}{3.184853in}}{\pgfqpoint{5.482896in}{3.180463in}}{\pgfqpoint{5.475083in}{3.172649in}}%
\pgfpathcurveto{\pgfqpoint{5.467269in}{3.164836in}}{\pgfqpoint{5.462879in}{3.154237in}}{\pgfqpoint{5.462879in}{3.143187in}}%
\pgfpathcurveto{\pgfqpoint{5.462879in}{3.132136in}}{\pgfqpoint{5.467269in}{3.121537in}}{\pgfqpoint{5.475083in}{3.113724in}}%
\pgfpathcurveto{\pgfqpoint{5.482896in}{3.105910in}}{\pgfqpoint{5.493495in}{3.101520in}}{\pgfqpoint{5.504545in}{3.101520in}}%
\pgfpathclose%
\pgfusepath{stroke,fill}%
\end{pgfscope}%
\begin{pgfscope}%
\pgfpathrectangle{\pgfqpoint{0.800000in}{0.528000in}}{\pgfqpoint{4.960000in}{3.696000in}}%
\pgfusepath{clip}%
\pgfsetbuttcap%
\pgfsetroundjoin%
\definecolor{currentfill}{rgb}{0.000000,0.000000,0.000000}%
\pgfsetfillcolor{currentfill}%
\pgfsetlinewidth{1.003750pt}%
\definecolor{currentstroke}{rgb}{0.000000,0.000000,0.000000}%
\pgfsetstrokecolor{currentstroke}%
\pgfsetdash{}{0pt}%
\pgfpathmoveto{\pgfqpoint{5.504545in}{3.460752in}}%
\pgfpathcurveto{\pgfqpoint{5.515596in}{3.460752in}}{\pgfqpoint{5.526195in}{3.465142in}}{\pgfqpoint{5.534008in}{3.472956in}}%
\pgfpathcurveto{\pgfqpoint{5.541822in}{3.480769in}}{\pgfqpoint{5.546212in}{3.491368in}}{\pgfqpoint{5.546212in}{3.502418in}}%
\pgfpathcurveto{\pgfqpoint{5.546212in}{3.513469in}}{\pgfqpoint{5.541822in}{3.524068in}}{\pgfqpoint{5.534008in}{3.531881in}}%
\pgfpathcurveto{\pgfqpoint{5.526195in}{3.539695in}}{\pgfqpoint{5.515596in}{3.544085in}}{\pgfqpoint{5.504545in}{3.544085in}}%
\pgfpathcurveto{\pgfqpoint{5.493495in}{3.544085in}}{\pgfqpoint{5.482896in}{3.539695in}}{\pgfqpoint{5.475083in}{3.531881in}}%
\pgfpathcurveto{\pgfqpoint{5.467269in}{3.524068in}}{\pgfqpoint{5.462879in}{3.513469in}}{\pgfqpoint{5.462879in}{3.502418in}}%
\pgfpathcurveto{\pgfqpoint{5.462879in}{3.491368in}}{\pgfqpoint{5.467269in}{3.480769in}}{\pgfqpoint{5.475083in}{3.472956in}}%
\pgfpathcurveto{\pgfqpoint{5.482896in}{3.465142in}}{\pgfqpoint{5.493495in}{3.460752in}}{\pgfqpoint{5.504545in}{3.460752in}}%
\pgfpathclose%
\pgfusepath{stroke,fill}%
\end{pgfscope}%
\begin{pgfscope}%
\pgfpathrectangle{\pgfqpoint{0.800000in}{0.528000in}}{\pgfqpoint{4.960000in}{3.696000in}}%
\pgfusepath{clip}%
\pgfsetbuttcap%
\pgfsetroundjoin%
\definecolor{currentfill}{rgb}{0.000000,0.000000,0.000000}%
\pgfsetfillcolor{currentfill}%
\pgfsetlinewidth{1.003750pt}%
\definecolor{currentstroke}{rgb}{0.000000,0.000000,0.000000}%
\pgfsetstrokecolor{currentstroke}%
\pgfsetdash{}{0pt}%
\pgfpathmoveto{\pgfqpoint{5.504545in}{3.985036in}}%
\pgfpathcurveto{\pgfqpoint{5.515596in}{3.985036in}}{\pgfqpoint{5.526195in}{3.989426in}}{\pgfqpoint{5.534008in}{3.997240in}}%
\pgfpathcurveto{\pgfqpoint{5.541822in}{4.005054in}}{\pgfqpoint{5.546212in}{4.015653in}}{\pgfqpoint{5.546212in}{4.026703in}}%
\pgfpathcurveto{\pgfqpoint{5.546212in}{4.037753in}}{\pgfqpoint{5.541822in}{4.048352in}}{\pgfqpoint{5.534008in}{4.056166in}}%
\pgfpathcurveto{\pgfqpoint{5.526195in}{4.063979in}}{\pgfqpoint{5.515596in}{4.068369in}}{\pgfqpoint{5.504545in}{4.068369in}}%
\pgfpathcurveto{\pgfqpoint{5.493495in}{4.068369in}}{\pgfqpoint{5.482896in}{4.063979in}}{\pgfqpoint{5.475083in}{4.056166in}}%
\pgfpathcurveto{\pgfqpoint{5.467269in}{4.048352in}}{\pgfqpoint{5.462879in}{4.037753in}}{\pgfqpoint{5.462879in}{4.026703in}}%
\pgfpathcurveto{\pgfqpoint{5.462879in}{4.015653in}}{\pgfqpoint{5.467269in}{4.005054in}}{\pgfqpoint{5.475083in}{3.997240in}}%
\pgfpathcurveto{\pgfqpoint{5.482896in}{3.989426in}}{\pgfqpoint{5.493495in}{3.985036in}}{\pgfqpoint{5.504545in}{3.985036in}}%
\pgfpathclose%
\pgfusepath{stroke,fill}%
\end{pgfscope}%
\begin{pgfscope}%
\pgfpathrectangle{\pgfqpoint{0.800000in}{0.528000in}}{\pgfqpoint{4.960000in}{3.696000in}}%
\pgfusepath{clip}%
\pgfsetbuttcap%
\pgfsetroundjoin%
\definecolor{currentfill}{rgb}{0.000000,0.000000,0.000000}%
\pgfsetfillcolor{currentfill}%
\pgfsetlinewidth{1.003750pt}%
\definecolor{currentstroke}{rgb}{0.000000,0.000000,0.000000}%
\pgfsetstrokecolor{currentstroke}%
\pgfsetdash{}{0pt}%
\pgfpathmoveto{\pgfqpoint{5.504545in}{3.227737in}}%
\pgfpathcurveto{\pgfqpoint{5.515596in}{3.227737in}}{\pgfqpoint{5.526195in}{3.232127in}}{\pgfqpoint{5.534008in}{3.239940in}}%
\pgfpathcurveto{\pgfqpoint{5.541822in}{3.247754in}}{\pgfqpoint{5.546212in}{3.258353in}}{\pgfqpoint{5.546212in}{3.269403in}}%
\pgfpathcurveto{\pgfqpoint{5.546212in}{3.280453in}}{\pgfqpoint{5.541822in}{3.291052in}}{\pgfqpoint{5.534008in}{3.298866in}}%
\pgfpathcurveto{\pgfqpoint{5.526195in}{3.306680in}}{\pgfqpoint{5.515596in}{3.311070in}}{\pgfqpoint{5.504545in}{3.311070in}}%
\pgfpathcurveto{\pgfqpoint{5.493495in}{3.311070in}}{\pgfqpoint{5.482896in}{3.306680in}}{\pgfqpoint{5.475083in}{3.298866in}}%
\pgfpathcurveto{\pgfqpoint{5.467269in}{3.291052in}}{\pgfqpoint{5.462879in}{3.280453in}}{\pgfqpoint{5.462879in}{3.269403in}}%
\pgfpathcurveto{\pgfqpoint{5.462879in}{3.258353in}}{\pgfqpoint{5.467269in}{3.247754in}}{\pgfqpoint{5.475083in}{3.239940in}}%
\pgfpathcurveto{\pgfqpoint{5.482896in}{3.232127in}}{\pgfqpoint{5.493495in}{3.227737in}}{\pgfqpoint{5.504545in}{3.227737in}}%
\pgfpathclose%
\pgfusepath{stroke,fill}%
\end{pgfscope}%
\begin{pgfscope}%
\pgfpathrectangle{\pgfqpoint{0.800000in}{0.528000in}}{\pgfqpoint{4.960000in}{3.696000in}}%
\pgfusepath{clip}%
\pgfsetbuttcap%
\pgfsetroundjoin%
\definecolor{currentfill}{rgb}{0.000000,0.000000,0.000000}%
\pgfsetfillcolor{currentfill}%
\pgfsetlinewidth{1.003750pt}%
\definecolor{currentstroke}{rgb}{0.000000,0.000000,0.000000}%
\pgfsetstrokecolor{currentstroke}%
\pgfsetdash{}{0pt}%
\pgfpathmoveto{\pgfqpoint{5.504545in}{3.023848in}}%
\pgfpathcurveto{\pgfqpoint{5.515596in}{3.023848in}}{\pgfqpoint{5.526195in}{3.028238in}}{\pgfqpoint{5.534008in}{3.036052in}}%
\pgfpathcurveto{\pgfqpoint{5.541822in}{3.043866in}}{\pgfqpoint{5.546212in}{3.054465in}}{\pgfqpoint{5.546212in}{3.065515in}}%
\pgfpathcurveto{\pgfqpoint{5.546212in}{3.076565in}}{\pgfqpoint{5.541822in}{3.087164in}}{\pgfqpoint{5.534008in}{3.094978in}}%
\pgfpathcurveto{\pgfqpoint{5.526195in}{3.102791in}}{\pgfqpoint{5.515596in}{3.107182in}}{\pgfqpoint{5.504545in}{3.107182in}}%
\pgfpathcurveto{\pgfqpoint{5.493495in}{3.107182in}}{\pgfqpoint{5.482896in}{3.102791in}}{\pgfqpoint{5.475083in}{3.094978in}}%
\pgfpathcurveto{\pgfqpoint{5.467269in}{3.087164in}}{\pgfqpoint{5.462879in}{3.076565in}}{\pgfqpoint{5.462879in}{3.065515in}}%
\pgfpathcurveto{\pgfqpoint{5.462879in}{3.054465in}}{\pgfqpoint{5.467269in}{3.043866in}}{\pgfqpoint{5.475083in}{3.036052in}}%
\pgfpathcurveto{\pgfqpoint{5.482896in}{3.028238in}}{\pgfqpoint{5.493495in}{3.023848in}}{\pgfqpoint{5.504545in}{3.023848in}}%
\pgfpathclose%
\pgfusepath{stroke,fill}%
\end{pgfscope}%
\begin{pgfscope}%
\pgfpathrectangle{\pgfqpoint{0.800000in}{0.528000in}}{\pgfqpoint{4.960000in}{3.696000in}}%
\pgfusepath{clip}%
\pgfsetbuttcap%
\pgfsetroundjoin%
\definecolor{currentfill}{rgb}{0.000000,0.000000,0.000000}%
\pgfsetfillcolor{currentfill}%
\pgfsetlinewidth{1.003750pt}%
\definecolor{currentstroke}{rgb}{0.000000,0.000000,0.000000}%
\pgfsetstrokecolor{currentstroke}%
\pgfsetdash{}{0pt}%
\pgfpathmoveto{\pgfqpoint{5.504545in}{3.256863in}}%
\pgfpathcurveto{\pgfqpoint{5.515596in}{3.256863in}}{\pgfqpoint{5.526195in}{3.261254in}}{\pgfqpoint{5.534008in}{3.269067in}}%
\pgfpathcurveto{\pgfqpoint{5.541822in}{3.276881in}}{\pgfqpoint{5.546212in}{3.287480in}}{\pgfqpoint{5.546212in}{3.298530in}}%
\pgfpathcurveto{\pgfqpoint{5.546212in}{3.309580in}}{\pgfqpoint{5.541822in}{3.320179in}}{\pgfqpoint{5.534008in}{3.327993in}}%
\pgfpathcurveto{\pgfqpoint{5.526195in}{3.335807in}}{\pgfqpoint{5.515596in}{3.340197in}}{\pgfqpoint{5.504545in}{3.340197in}}%
\pgfpathcurveto{\pgfqpoint{5.493495in}{3.340197in}}{\pgfqpoint{5.482896in}{3.335807in}}{\pgfqpoint{5.475083in}{3.327993in}}%
\pgfpathcurveto{\pgfqpoint{5.467269in}{3.320179in}}{\pgfqpoint{5.462879in}{3.309580in}}{\pgfqpoint{5.462879in}{3.298530in}}%
\pgfpathcurveto{\pgfqpoint{5.462879in}{3.287480in}}{\pgfqpoint{5.467269in}{3.276881in}}{\pgfqpoint{5.475083in}{3.269067in}}%
\pgfpathcurveto{\pgfqpoint{5.482896in}{3.261254in}}{\pgfqpoint{5.493495in}{3.256863in}}{\pgfqpoint{5.504545in}{3.256863in}}%
\pgfpathclose%
\pgfusepath{stroke,fill}%
\end{pgfscope}%
\begin{pgfscope}%
\pgfpathrectangle{\pgfqpoint{0.800000in}{0.528000in}}{\pgfqpoint{4.960000in}{3.696000in}}%
\pgfusepath{clip}%
\pgfsetbuttcap%
\pgfsetroundjoin%
\definecolor{currentfill}{rgb}{0.000000,0.000000,0.000000}%
\pgfsetfillcolor{currentfill}%
\pgfsetlinewidth{1.003750pt}%
\definecolor{currentstroke}{rgb}{0.000000,0.000000,0.000000}%
\pgfsetstrokecolor{currentstroke}%
\pgfsetdash{}{0pt}%
\pgfpathmoveto{\pgfqpoint{5.504545in}{3.179192in}}%
\pgfpathcurveto{\pgfqpoint{5.515596in}{3.179192in}}{\pgfqpoint{5.526195in}{3.183582in}}{\pgfqpoint{5.534008in}{3.191396in}}%
\pgfpathcurveto{\pgfqpoint{5.541822in}{3.199209in}}{\pgfqpoint{5.546212in}{3.209808in}}{\pgfqpoint{5.546212in}{3.220858in}}%
\pgfpathcurveto{\pgfqpoint{5.546212in}{3.231909in}}{\pgfqpoint{5.541822in}{3.242508in}}{\pgfqpoint{5.534008in}{3.250321in}}%
\pgfpathcurveto{\pgfqpoint{5.526195in}{3.258135in}}{\pgfqpoint{5.515596in}{3.262525in}}{\pgfqpoint{5.504545in}{3.262525in}}%
\pgfpathcurveto{\pgfqpoint{5.493495in}{3.262525in}}{\pgfqpoint{5.482896in}{3.258135in}}{\pgfqpoint{5.475083in}{3.250321in}}%
\pgfpathcurveto{\pgfqpoint{5.467269in}{3.242508in}}{\pgfqpoint{5.462879in}{3.231909in}}{\pgfqpoint{5.462879in}{3.220858in}}%
\pgfpathcurveto{\pgfqpoint{5.462879in}{3.209808in}}{\pgfqpoint{5.467269in}{3.199209in}}{\pgfqpoint{5.475083in}{3.191396in}}%
\pgfpathcurveto{\pgfqpoint{5.482896in}{3.183582in}}{\pgfqpoint{5.493495in}{3.179192in}}{\pgfqpoint{5.504545in}{3.179192in}}%
\pgfpathclose%
\pgfusepath{stroke,fill}%
\end{pgfscope}%
\begin{pgfscope}%
\pgfpathrectangle{\pgfqpoint{0.800000in}{0.528000in}}{\pgfqpoint{4.960000in}{3.696000in}}%
\pgfusepath{clip}%
\pgfsetbuttcap%
\pgfsetroundjoin%
\definecolor{currentfill}{rgb}{0.000000,0.000000,0.000000}%
\pgfsetfillcolor{currentfill}%
\pgfsetlinewidth{1.003750pt}%
\definecolor{currentstroke}{rgb}{0.000000,0.000000,0.000000}%
\pgfsetstrokecolor{currentstroke}%
\pgfsetdash{}{0pt}%
\pgfpathmoveto{\pgfqpoint{5.504545in}{3.596677in}}%
\pgfpathcurveto{\pgfqpoint{5.515596in}{3.596677in}}{\pgfqpoint{5.526195in}{3.601068in}}{\pgfqpoint{5.534008in}{3.608881in}}%
\pgfpathcurveto{\pgfqpoint{5.541822in}{3.616695in}}{\pgfqpoint{5.546212in}{3.627294in}}{\pgfqpoint{5.546212in}{3.638344in}}%
\pgfpathcurveto{\pgfqpoint{5.546212in}{3.649394in}}{\pgfqpoint{5.541822in}{3.659993in}}{\pgfqpoint{5.534008in}{3.667807in}}%
\pgfpathcurveto{\pgfqpoint{5.526195in}{3.675620in}}{\pgfqpoint{5.515596in}{3.680011in}}{\pgfqpoint{5.504545in}{3.680011in}}%
\pgfpathcurveto{\pgfqpoint{5.493495in}{3.680011in}}{\pgfqpoint{5.482896in}{3.675620in}}{\pgfqpoint{5.475083in}{3.667807in}}%
\pgfpathcurveto{\pgfqpoint{5.467269in}{3.659993in}}{\pgfqpoint{5.462879in}{3.649394in}}{\pgfqpoint{5.462879in}{3.638344in}}%
\pgfpathcurveto{\pgfqpoint{5.462879in}{3.627294in}}{\pgfqpoint{5.467269in}{3.616695in}}{\pgfqpoint{5.475083in}{3.608881in}}%
\pgfpathcurveto{\pgfqpoint{5.482896in}{3.601068in}}{\pgfqpoint{5.493495in}{3.596677in}}{\pgfqpoint{5.504545in}{3.596677in}}%
\pgfpathclose%
\pgfusepath{stroke,fill}%
\end{pgfscope}%
\begin{pgfscope}%
\pgfpathrectangle{\pgfqpoint{0.800000in}{0.528000in}}{\pgfqpoint{4.960000in}{3.696000in}}%
\pgfusepath{clip}%
\pgfsetbuttcap%
\pgfsetroundjoin%
\definecolor{currentfill}{rgb}{0.000000,0.000000,0.000000}%
\pgfsetfillcolor{currentfill}%
\pgfsetlinewidth{1.003750pt}%
\definecolor{currentstroke}{rgb}{0.000000,0.000000,0.000000}%
\pgfsetstrokecolor{currentstroke}%
\pgfsetdash{}{0pt}%
\pgfpathmoveto{\pgfqpoint{5.504545in}{3.247154in}}%
\pgfpathcurveto{\pgfqpoint{5.515596in}{3.247154in}}{\pgfqpoint{5.526195in}{3.251545in}}{\pgfqpoint{5.534008in}{3.259358in}}%
\pgfpathcurveto{\pgfqpoint{5.541822in}{3.267172in}}{\pgfqpoint{5.546212in}{3.277771in}}{\pgfqpoint{5.546212in}{3.288821in}}%
\pgfpathcurveto{\pgfqpoint{5.546212in}{3.299871in}}{\pgfqpoint{5.541822in}{3.310470in}}{\pgfqpoint{5.534008in}{3.318284in}}%
\pgfpathcurveto{\pgfqpoint{5.526195in}{3.326098in}}{\pgfqpoint{5.515596in}{3.330488in}}{\pgfqpoint{5.504545in}{3.330488in}}%
\pgfpathcurveto{\pgfqpoint{5.493495in}{3.330488in}}{\pgfqpoint{5.482896in}{3.326098in}}{\pgfqpoint{5.475083in}{3.318284in}}%
\pgfpathcurveto{\pgfqpoint{5.467269in}{3.310470in}}{\pgfqpoint{5.462879in}{3.299871in}}{\pgfqpoint{5.462879in}{3.288821in}}%
\pgfpathcurveto{\pgfqpoint{5.462879in}{3.277771in}}{\pgfqpoint{5.467269in}{3.267172in}}{\pgfqpoint{5.475083in}{3.259358in}}%
\pgfpathcurveto{\pgfqpoint{5.482896in}{3.251545in}}{\pgfqpoint{5.493495in}{3.247154in}}{\pgfqpoint{5.504545in}{3.247154in}}%
\pgfpathclose%
\pgfusepath{stroke,fill}%
\end{pgfscope}%
\begin{pgfscope}%
\pgfpathrectangle{\pgfqpoint{0.800000in}{0.528000in}}{\pgfqpoint{4.960000in}{3.696000in}}%
\pgfusepath{clip}%
\pgfsetbuttcap%
\pgfsetroundjoin%
\definecolor{currentfill}{rgb}{0.000000,0.000000,0.000000}%
\pgfsetfillcolor{currentfill}%
\pgfsetlinewidth{1.003750pt}%
\definecolor{currentstroke}{rgb}{0.000000,0.000000,0.000000}%
\pgfsetstrokecolor{currentstroke}%
\pgfsetdash{}{0pt}%
\pgfpathmoveto{\pgfqpoint{5.504545in}{3.285990in}}%
\pgfpathcurveto{\pgfqpoint{5.515596in}{3.285990in}}{\pgfqpoint{5.526195in}{3.290381in}}{\pgfqpoint{5.534008in}{3.298194in}}%
\pgfpathcurveto{\pgfqpoint{5.541822in}{3.306008in}}{\pgfqpoint{5.546212in}{3.316607in}}{\pgfqpoint{5.546212in}{3.327657in}}%
\pgfpathcurveto{\pgfqpoint{5.546212in}{3.338707in}}{\pgfqpoint{5.541822in}{3.349306in}}{\pgfqpoint{5.534008in}{3.357120in}}%
\pgfpathcurveto{\pgfqpoint{5.526195in}{3.364933in}}{\pgfqpoint{5.515596in}{3.369324in}}{\pgfqpoint{5.504545in}{3.369324in}}%
\pgfpathcurveto{\pgfqpoint{5.493495in}{3.369324in}}{\pgfqpoint{5.482896in}{3.364933in}}{\pgfqpoint{5.475083in}{3.357120in}}%
\pgfpathcurveto{\pgfqpoint{5.467269in}{3.349306in}}{\pgfqpoint{5.462879in}{3.338707in}}{\pgfqpoint{5.462879in}{3.327657in}}%
\pgfpathcurveto{\pgfqpoint{5.462879in}{3.316607in}}{\pgfqpoint{5.467269in}{3.306008in}}{\pgfqpoint{5.475083in}{3.298194in}}%
\pgfpathcurveto{\pgfqpoint{5.482896in}{3.290381in}}{\pgfqpoint{5.493495in}{3.285990in}}{\pgfqpoint{5.504545in}{3.285990in}}%
\pgfpathclose%
\pgfusepath{stroke,fill}%
\end{pgfscope}%
\begin{pgfscope}%
\pgfpathrectangle{\pgfqpoint{0.800000in}{0.528000in}}{\pgfqpoint{4.960000in}{3.696000in}}%
\pgfusepath{clip}%
\pgfsetbuttcap%
\pgfsetroundjoin%
\definecolor{currentfill}{rgb}{0.000000,0.000000,0.000000}%
\pgfsetfillcolor{currentfill}%
\pgfsetlinewidth{1.003750pt}%
\definecolor{currentstroke}{rgb}{0.000000,0.000000,0.000000}%
\pgfsetstrokecolor{currentstroke}%
\pgfsetdash{}{0pt}%
\pgfpathmoveto{\pgfqpoint{5.504545in}{3.645222in}}%
\pgfpathcurveto{\pgfqpoint{5.515596in}{3.645222in}}{\pgfqpoint{5.526195in}{3.649612in}}{\pgfqpoint{5.534008in}{3.657426in}}%
\pgfpathcurveto{\pgfqpoint{5.541822in}{3.665240in}}{\pgfqpoint{5.546212in}{3.675839in}}{\pgfqpoint{5.546212in}{3.686889in}}%
\pgfpathcurveto{\pgfqpoint{5.546212in}{3.697939in}}{\pgfqpoint{5.541822in}{3.708538in}}{\pgfqpoint{5.534008in}{3.716352in}}%
\pgfpathcurveto{\pgfqpoint{5.526195in}{3.724165in}}{\pgfqpoint{5.515596in}{3.728556in}}{\pgfqpoint{5.504545in}{3.728556in}}%
\pgfpathcurveto{\pgfqpoint{5.493495in}{3.728556in}}{\pgfqpoint{5.482896in}{3.724165in}}{\pgfqpoint{5.475083in}{3.716352in}}%
\pgfpathcurveto{\pgfqpoint{5.467269in}{3.708538in}}{\pgfqpoint{5.462879in}{3.697939in}}{\pgfqpoint{5.462879in}{3.686889in}}%
\pgfpathcurveto{\pgfqpoint{5.462879in}{3.675839in}}{\pgfqpoint{5.467269in}{3.665240in}}{\pgfqpoint{5.475083in}{3.657426in}}%
\pgfpathcurveto{\pgfqpoint{5.482896in}{3.649612in}}{\pgfqpoint{5.493495in}{3.645222in}}{\pgfqpoint{5.504545in}{3.645222in}}%
\pgfpathclose%
\pgfusepath{stroke,fill}%
\end{pgfscope}%
\begin{pgfscope}%
\pgfpathrectangle{\pgfqpoint{0.800000in}{0.528000in}}{\pgfqpoint{4.960000in}{3.696000in}}%
\pgfusepath{clip}%
\pgfsetbuttcap%
\pgfsetroundjoin%
\definecolor{currentfill}{rgb}{0.000000,0.000000,0.000000}%
\pgfsetfillcolor{currentfill}%
\pgfsetlinewidth{1.003750pt}%
\definecolor{currentstroke}{rgb}{0.000000,0.000000,0.000000}%
\pgfsetstrokecolor{currentstroke}%
\pgfsetdash{}{0pt}%
\pgfpathmoveto{\pgfqpoint{5.504545in}{3.315117in}}%
\pgfpathcurveto{\pgfqpoint{5.515596in}{3.315117in}}{\pgfqpoint{5.526195in}{3.319508in}}{\pgfqpoint{5.534008in}{3.327321in}}%
\pgfpathcurveto{\pgfqpoint{5.541822in}{3.335135in}}{\pgfqpoint{5.546212in}{3.345734in}}{\pgfqpoint{5.546212in}{3.356784in}}%
\pgfpathcurveto{\pgfqpoint{5.546212in}{3.367834in}}{\pgfqpoint{5.541822in}{3.378433in}}{\pgfqpoint{5.534008in}{3.386247in}}%
\pgfpathcurveto{\pgfqpoint{5.526195in}{3.394060in}}{\pgfqpoint{5.515596in}{3.398451in}}{\pgfqpoint{5.504545in}{3.398451in}}%
\pgfpathcurveto{\pgfqpoint{5.493495in}{3.398451in}}{\pgfqpoint{5.482896in}{3.394060in}}{\pgfqpoint{5.475083in}{3.386247in}}%
\pgfpathcurveto{\pgfqpoint{5.467269in}{3.378433in}}{\pgfqpoint{5.462879in}{3.367834in}}{\pgfqpoint{5.462879in}{3.356784in}}%
\pgfpathcurveto{\pgfqpoint{5.462879in}{3.345734in}}{\pgfqpoint{5.467269in}{3.335135in}}{\pgfqpoint{5.475083in}{3.327321in}}%
\pgfpathcurveto{\pgfqpoint{5.482896in}{3.319508in}}{\pgfqpoint{5.493495in}{3.315117in}}{\pgfqpoint{5.504545in}{3.315117in}}%
\pgfpathclose%
\pgfusepath{stroke,fill}%
\end{pgfscope}%
\begin{pgfscope}%
\pgfpathrectangle{\pgfqpoint{0.800000in}{0.528000in}}{\pgfqpoint{4.960000in}{3.696000in}}%
\pgfusepath{clip}%
\pgfsetbuttcap%
\pgfsetroundjoin%
\definecolor{currentfill}{rgb}{0.000000,0.000000,0.000000}%
\pgfsetfillcolor{currentfill}%
\pgfsetlinewidth{1.003750pt}%
\definecolor{currentstroke}{rgb}{0.000000,0.000000,0.000000}%
\pgfsetstrokecolor{currentstroke}%
\pgfsetdash{}{0pt}%
\pgfpathmoveto{\pgfqpoint{5.504545in}{3.946200in}}%
\pgfpathcurveto{\pgfqpoint{5.515596in}{3.946200in}}{\pgfqpoint{5.526195in}{3.950591in}}{\pgfqpoint{5.534008in}{3.958404in}}%
\pgfpathcurveto{\pgfqpoint{5.541822in}{3.966218in}}{\pgfqpoint{5.546212in}{3.976817in}}{\pgfqpoint{5.546212in}{3.987867in}}%
\pgfpathcurveto{\pgfqpoint{5.546212in}{3.998917in}}{\pgfqpoint{5.541822in}{4.009516in}}{\pgfqpoint{5.534008in}{4.017330in}}%
\pgfpathcurveto{\pgfqpoint{5.526195in}{4.025143in}}{\pgfqpoint{5.515596in}{4.029534in}}{\pgfqpoint{5.504545in}{4.029534in}}%
\pgfpathcurveto{\pgfqpoint{5.493495in}{4.029534in}}{\pgfqpoint{5.482896in}{4.025143in}}{\pgfqpoint{5.475083in}{4.017330in}}%
\pgfpathcurveto{\pgfqpoint{5.467269in}{4.009516in}}{\pgfqpoint{5.462879in}{3.998917in}}{\pgfqpoint{5.462879in}{3.987867in}}%
\pgfpathcurveto{\pgfqpoint{5.462879in}{3.976817in}}{\pgfqpoint{5.467269in}{3.966218in}}{\pgfqpoint{5.475083in}{3.958404in}}%
\pgfpathcurveto{\pgfqpoint{5.482896in}{3.950591in}}{\pgfqpoint{5.493495in}{3.946200in}}{\pgfqpoint{5.504545in}{3.946200in}}%
\pgfpathclose%
\pgfusepath{stroke,fill}%
\end{pgfscope}%
\begin{pgfscope}%
\pgfpathrectangle{\pgfqpoint{0.800000in}{0.528000in}}{\pgfqpoint{4.960000in}{3.696000in}}%
\pgfusepath{clip}%
\pgfsetbuttcap%
\pgfsetroundjoin%
\definecolor{currentfill}{rgb}{0.000000,0.000000,0.000000}%
\pgfsetfillcolor{currentfill}%
\pgfsetlinewidth{1.003750pt}%
\definecolor{currentstroke}{rgb}{0.000000,0.000000,0.000000}%
\pgfsetstrokecolor{currentstroke}%
\pgfsetdash{}{0pt}%
\pgfpathmoveto{\pgfqpoint{5.504545in}{3.159774in}}%
\pgfpathcurveto{\pgfqpoint{5.515596in}{3.159774in}}{\pgfqpoint{5.526195in}{3.164164in}}{\pgfqpoint{5.534008in}{3.171978in}}%
\pgfpathcurveto{\pgfqpoint{5.541822in}{3.179791in}}{\pgfqpoint{5.546212in}{3.190390in}}{\pgfqpoint{5.546212in}{3.201440in}}%
\pgfpathcurveto{\pgfqpoint{5.546212in}{3.212491in}}{\pgfqpoint{5.541822in}{3.223090in}}{\pgfqpoint{5.534008in}{3.230903in}}%
\pgfpathcurveto{\pgfqpoint{5.526195in}{3.238717in}}{\pgfqpoint{5.515596in}{3.243107in}}{\pgfqpoint{5.504545in}{3.243107in}}%
\pgfpathcurveto{\pgfqpoint{5.493495in}{3.243107in}}{\pgfqpoint{5.482896in}{3.238717in}}{\pgfqpoint{5.475083in}{3.230903in}}%
\pgfpathcurveto{\pgfqpoint{5.467269in}{3.223090in}}{\pgfqpoint{5.462879in}{3.212491in}}{\pgfqpoint{5.462879in}{3.201440in}}%
\pgfpathcurveto{\pgfqpoint{5.462879in}{3.190390in}}{\pgfqpoint{5.467269in}{3.179791in}}{\pgfqpoint{5.475083in}{3.171978in}}%
\pgfpathcurveto{\pgfqpoint{5.482896in}{3.164164in}}{\pgfqpoint{5.493495in}{3.159774in}}{\pgfqpoint{5.504545in}{3.159774in}}%
\pgfpathclose%
\pgfusepath{stroke,fill}%
\end{pgfscope}%
\begin{pgfscope}%
\pgfpathrectangle{\pgfqpoint{0.800000in}{0.528000in}}{\pgfqpoint{4.960000in}{3.696000in}}%
\pgfusepath{clip}%
\pgfsetbuttcap%
\pgfsetroundjoin%
\definecolor{currentfill}{rgb}{0.000000,0.000000,0.000000}%
\pgfsetfillcolor{currentfill}%
\pgfsetlinewidth{1.003750pt}%
\definecolor{currentstroke}{rgb}{0.000000,0.000000,0.000000}%
\pgfsetstrokecolor{currentstroke}%
\pgfsetdash{}{0pt}%
\pgfpathmoveto{\pgfqpoint{5.504545in}{3.140356in}}%
\pgfpathcurveto{\pgfqpoint{5.515596in}{3.140356in}}{\pgfqpoint{5.526195in}{3.144746in}}{\pgfqpoint{5.534008in}{3.152560in}}%
\pgfpathcurveto{\pgfqpoint{5.541822in}{3.160373in}}{\pgfqpoint{5.546212in}{3.170972in}}{\pgfqpoint{5.546212in}{3.182022in}}%
\pgfpathcurveto{\pgfqpoint{5.546212in}{3.193073in}}{\pgfqpoint{5.541822in}{3.203672in}}{\pgfqpoint{5.534008in}{3.211485in}}%
\pgfpathcurveto{\pgfqpoint{5.526195in}{3.219299in}}{\pgfqpoint{5.515596in}{3.223689in}}{\pgfqpoint{5.504545in}{3.223689in}}%
\pgfpathcurveto{\pgfqpoint{5.493495in}{3.223689in}}{\pgfqpoint{5.482896in}{3.219299in}}{\pgfqpoint{5.475083in}{3.211485in}}%
\pgfpathcurveto{\pgfqpoint{5.467269in}{3.203672in}}{\pgfqpoint{5.462879in}{3.193073in}}{\pgfqpoint{5.462879in}{3.182022in}}%
\pgfpathcurveto{\pgfqpoint{5.462879in}{3.170972in}}{\pgfqpoint{5.467269in}{3.160373in}}{\pgfqpoint{5.475083in}{3.152560in}}%
\pgfpathcurveto{\pgfqpoint{5.482896in}{3.144746in}}{\pgfqpoint{5.493495in}{3.140356in}}{\pgfqpoint{5.504545in}{3.140356in}}%
\pgfpathclose%
\pgfusepath{stroke,fill}%
\end{pgfscope}%
\begin{pgfscope}%
\pgfpathrectangle{\pgfqpoint{0.800000in}{0.528000in}}{\pgfqpoint{4.960000in}{3.696000in}}%
\pgfusepath{clip}%
\pgfsetbuttcap%
\pgfsetroundjoin%
\definecolor{currentfill}{rgb}{0.000000,0.000000,0.000000}%
\pgfsetfillcolor{currentfill}%
\pgfsetlinewidth{1.003750pt}%
\definecolor{currentstroke}{rgb}{0.000000,0.000000,0.000000}%
\pgfsetstrokecolor{currentstroke}%
\pgfsetdash{}{0pt}%
\pgfpathmoveto{\pgfqpoint{5.504545in}{3.383080in}}%
\pgfpathcurveto{\pgfqpoint{5.515596in}{3.383080in}}{\pgfqpoint{5.526195in}{3.387470in}}{\pgfqpoint{5.534008in}{3.395284in}}%
\pgfpathcurveto{\pgfqpoint{5.541822in}{3.403098in}}{\pgfqpoint{5.546212in}{3.413697in}}{\pgfqpoint{5.546212in}{3.424747in}}%
\pgfpathcurveto{\pgfqpoint{5.546212in}{3.435797in}}{\pgfqpoint{5.541822in}{3.446396in}}{\pgfqpoint{5.534008in}{3.454210in}}%
\pgfpathcurveto{\pgfqpoint{5.526195in}{3.462023in}}{\pgfqpoint{5.515596in}{3.466413in}}{\pgfqpoint{5.504545in}{3.466413in}}%
\pgfpathcurveto{\pgfqpoint{5.493495in}{3.466413in}}{\pgfqpoint{5.482896in}{3.462023in}}{\pgfqpoint{5.475083in}{3.454210in}}%
\pgfpathcurveto{\pgfqpoint{5.467269in}{3.446396in}}{\pgfqpoint{5.462879in}{3.435797in}}{\pgfqpoint{5.462879in}{3.424747in}}%
\pgfpathcurveto{\pgfqpoint{5.462879in}{3.413697in}}{\pgfqpoint{5.467269in}{3.403098in}}{\pgfqpoint{5.475083in}{3.395284in}}%
\pgfpathcurveto{\pgfqpoint{5.482896in}{3.387470in}}{\pgfqpoint{5.493495in}{3.383080in}}{\pgfqpoint{5.504545in}{3.383080in}}%
\pgfpathclose%
\pgfusepath{stroke,fill}%
\end{pgfscope}%
\begin{pgfscope}%
\pgfpathrectangle{\pgfqpoint{0.800000in}{0.528000in}}{\pgfqpoint{4.960000in}{3.696000in}}%
\pgfusepath{clip}%
\pgfsetbuttcap%
\pgfsetroundjoin%
\definecolor{currentfill}{rgb}{0.000000,0.000000,0.000000}%
\pgfsetfillcolor{currentfill}%
\pgfsetlinewidth{1.003750pt}%
\definecolor{currentstroke}{rgb}{0.000000,0.000000,0.000000}%
\pgfsetstrokecolor{currentstroke}%
\pgfsetdash{}{0pt}%
\pgfpathmoveto{\pgfqpoint{5.504545in}{3.285990in}}%
\pgfpathcurveto{\pgfqpoint{5.515596in}{3.285990in}}{\pgfqpoint{5.526195in}{3.290381in}}{\pgfqpoint{5.534008in}{3.298194in}}%
\pgfpathcurveto{\pgfqpoint{5.541822in}{3.306008in}}{\pgfqpoint{5.546212in}{3.316607in}}{\pgfqpoint{5.546212in}{3.327657in}}%
\pgfpathcurveto{\pgfqpoint{5.546212in}{3.338707in}}{\pgfqpoint{5.541822in}{3.349306in}}{\pgfqpoint{5.534008in}{3.357120in}}%
\pgfpathcurveto{\pgfqpoint{5.526195in}{3.364933in}}{\pgfqpoint{5.515596in}{3.369324in}}{\pgfqpoint{5.504545in}{3.369324in}}%
\pgfpathcurveto{\pgfqpoint{5.493495in}{3.369324in}}{\pgfqpoint{5.482896in}{3.364933in}}{\pgfqpoint{5.475083in}{3.357120in}}%
\pgfpathcurveto{\pgfqpoint{5.467269in}{3.349306in}}{\pgfqpoint{5.462879in}{3.338707in}}{\pgfqpoint{5.462879in}{3.327657in}}%
\pgfpathcurveto{\pgfqpoint{5.462879in}{3.316607in}}{\pgfqpoint{5.467269in}{3.306008in}}{\pgfqpoint{5.475083in}{3.298194in}}%
\pgfpathcurveto{\pgfqpoint{5.482896in}{3.290381in}}{\pgfqpoint{5.493495in}{3.285990in}}{\pgfqpoint{5.504545in}{3.285990in}}%
\pgfpathclose%
\pgfusepath{stroke,fill}%
\end{pgfscope}%
\begin{pgfscope}%
\pgfpathrectangle{\pgfqpoint{0.800000in}{0.528000in}}{\pgfqpoint{4.960000in}{3.696000in}}%
\pgfusepath{clip}%
\pgfsetbuttcap%
\pgfsetroundjoin%
\definecolor{currentfill}{rgb}{0.000000,0.000000,0.000000}%
\pgfsetfillcolor{currentfill}%
\pgfsetlinewidth{1.003750pt}%
\definecolor{currentstroke}{rgb}{0.000000,0.000000,0.000000}%
\pgfsetstrokecolor{currentstroke}%
\pgfsetdash{}{0pt}%
\pgfpathmoveto{\pgfqpoint{5.504545in}{3.849111in}}%
\pgfpathcurveto{\pgfqpoint{5.515596in}{3.849111in}}{\pgfqpoint{5.526195in}{3.853501in}}{\pgfqpoint{5.534008in}{3.861314in}}%
\pgfpathcurveto{\pgfqpoint{5.541822in}{3.869128in}}{\pgfqpoint{5.546212in}{3.879727in}}{\pgfqpoint{5.546212in}{3.890777in}}%
\pgfpathcurveto{\pgfqpoint{5.546212in}{3.901827in}}{\pgfqpoint{5.541822in}{3.912426in}}{\pgfqpoint{5.534008in}{3.920240in}}%
\pgfpathcurveto{\pgfqpoint{5.526195in}{3.928054in}}{\pgfqpoint{5.515596in}{3.932444in}}{\pgfqpoint{5.504545in}{3.932444in}}%
\pgfpathcurveto{\pgfqpoint{5.493495in}{3.932444in}}{\pgfqpoint{5.482896in}{3.928054in}}{\pgfqpoint{5.475083in}{3.920240in}}%
\pgfpathcurveto{\pgfqpoint{5.467269in}{3.912426in}}{\pgfqpoint{5.462879in}{3.901827in}}{\pgfqpoint{5.462879in}{3.890777in}}%
\pgfpathcurveto{\pgfqpoint{5.462879in}{3.879727in}}{\pgfqpoint{5.467269in}{3.869128in}}{\pgfqpoint{5.475083in}{3.861314in}}%
\pgfpathcurveto{\pgfqpoint{5.482896in}{3.853501in}}{\pgfqpoint{5.493495in}{3.849111in}}{\pgfqpoint{5.504545in}{3.849111in}}%
\pgfpathclose%
\pgfusepath{stroke,fill}%
\end{pgfscope}%
\begin{pgfscope}%
\pgfpathrectangle{\pgfqpoint{0.800000in}{0.528000in}}{\pgfqpoint{4.960000in}{3.696000in}}%
\pgfusepath{clip}%
\pgfsetbuttcap%
\pgfsetroundjoin%
\definecolor{currentfill}{rgb}{0.000000,0.000000,0.000000}%
\pgfsetfillcolor{currentfill}%
\pgfsetlinewidth{1.003750pt}%
\definecolor{currentstroke}{rgb}{0.000000,0.000000,0.000000}%
\pgfsetstrokecolor{currentstroke}%
\pgfsetdash{}{0pt}%
\pgfpathmoveto{\pgfqpoint{5.504545in}{3.159774in}}%
\pgfpathcurveto{\pgfqpoint{5.515596in}{3.159774in}}{\pgfqpoint{5.526195in}{3.164164in}}{\pgfqpoint{5.534008in}{3.171978in}}%
\pgfpathcurveto{\pgfqpoint{5.541822in}{3.179791in}}{\pgfqpoint{5.546212in}{3.190390in}}{\pgfqpoint{5.546212in}{3.201440in}}%
\pgfpathcurveto{\pgfqpoint{5.546212in}{3.212491in}}{\pgfqpoint{5.541822in}{3.223090in}}{\pgfqpoint{5.534008in}{3.230903in}}%
\pgfpathcurveto{\pgfqpoint{5.526195in}{3.238717in}}{\pgfqpoint{5.515596in}{3.243107in}}{\pgfqpoint{5.504545in}{3.243107in}}%
\pgfpathcurveto{\pgfqpoint{5.493495in}{3.243107in}}{\pgfqpoint{5.482896in}{3.238717in}}{\pgfqpoint{5.475083in}{3.230903in}}%
\pgfpathcurveto{\pgfqpoint{5.467269in}{3.223090in}}{\pgfqpoint{5.462879in}{3.212491in}}{\pgfqpoint{5.462879in}{3.201440in}}%
\pgfpathcurveto{\pgfqpoint{5.462879in}{3.190390in}}{\pgfqpoint{5.467269in}{3.179791in}}{\pgfqpoint{5.475083in}{3.171978in}}%
\pgfpathcurveto{\pgfqpoint{5.482896in}{3.164164in}}{\pgfqpoint{5.493495in}{3.159774in}}{\pgfqpoint{5.504545in}{3.159774in}}%
\pgfpathclose%
\pgfusepath{stroke,fill}%
\end{pgfscope}%
\begin{pgfscope}%
\pgfpathrectangle{\pgfqpoint{0.800000in}{0.528000in}}{\pgfqpoint{4.960000in}{3.696000in}}%
\pgfusepath{clip}%
\pgfsetbuttcap%
\pgfsetroundjoin%
\definecolor{currentfill}{rgb}{0.000000,0.000000,0.000000}%
\pgfsetfillcolor{currentfill}%
\pgfsetlinewidth{1.003750pt}%
\definecolor{currentstroke}{rgb}{0.000000,0.000000,0.000000}%
\pgfsetstrokecolor{currentstroke}%
\pgfsetdash{}{0pt}%
\pgfpathmoveto{\pgfqpoint{5.504545in}{3.062684in}}%
\pgfpathcurveto{\pgfqpoint{5.515596in}{3.062684in}}{\pgfqpoint{5.526195in}{3.067074in}}{\pgfqpoint{5.534008in}{3.074888in}}%
\pgfpathcurveto{\pgfqpoint{5.541822in}{3.082702in}}{\pgfqpoint{5.546212in}{3.093301in}}{\pgfqpoint{5.546212in}{3.104351in}}%
\pgfpathcurveto{\pgfqpoint{5.546212in}{3.115401in}}{\pgfqpoint{5.541822in}{3.126000in}}{\pgfqpoint{5.534008in}{3.133814in}}%
\pgfpathcurveto{\pgfqpoint{5.526195in}{3.141627in}}{\pgfqpoint{5.515596in}{3.146017in}}{\pgfqpoint{5.504545in}{3.146017in}}%
\pgfpathcurveto{\pgfqpoint{5.493495in}{3.146017in}}{\pgfqpoint{5.482896in}{3.141627in}}{\pgfqpoint{5.475083in}{3.133814in}}%
\pgfpathcurveto{\pgfqpoint{5.467269in}{3.126000in}}{\pgfqpoint{5.462879in}{3.115401in}}{\pgfqpoint{5.462879in}{3.104351in}}%
\pgfpathcurveto{\pgfqpoint{5.462879in}{3.093301in}}{\pgfqpoint{5.467269in}{3.082702in}}{\pgfqpoint{5.475083in}{3.074888in}}%
\pgfpathcurveto{\pgfqpoint{5.482896in}{3.067074in}}{\pgfqpoint{5.493495in}{3.062684in}}{\pgfqpoint{5.504545in}{3.062684in}}%
\pgfpathclose%
\pgfusepath{stroke,fill}%
\end{pgfscope}%
\begin{pgfscope}%
\pgfpathrectangle{\pgfqpoint{0.800000in}{0.528000in}}{\pgfqpoint{4.960000in}{3.696000in}}%
\pgfusepath{clip}%
\pgfsetbuttcap%
\pgfsetroundjoin%
\definecolor{currentfill}{rgb}{0.000000,0.000000,0.000000}%
\pgfsetfillcolor{currentfill}%
\pgfsetlinewidth{1.003750pt}%
\definecolor{currentstroke}{rgb}{0.000000,0.000000,0.000000}%
\pgfsetstrokecolor{currentstroke}%
\pgfsetdash{}{0pt}%
\pgfpathmoveto{\pgfqpoint{5.504545in}{3.557842in}}%
\pgfpathcurveto{\pgfqpoint{5.515596in}{3.557842in}}{\pgfqpoint{5.526195in}{3.562232in}}{\pgfqpoint{5.534008in}{3.570045in}}%
\pgfpathcurveto{\pgfqpoint{5.541822in}{3.577859in}}{\pgfqpoint{5.546212in}{3.588458in}}{\pgfqpoint{5.546212in}{3.599508in}}%
\pgfpathcurveto{\pgfqpoint{5.546212in}{3.610558in}}{\pgfqpoint{5.541822in}{3.621157in}}{\pgfqpoint{5.534008in}{3.628971in}}%
\pgfpathcurveto{\pgfqpoint{5.526195in}{3.636785in}}{\pgfqpoint{5.515596in}{3.641175in}}{\pgfqpoint{5.504545in}{3.641175in}}%
\pgfpathcurveto{\pgfqpoint{5.493495in}{3.641175in}}{\pgfqpoint{5.482896in}{3.636785in}}{\pgfqpoint{5.475083in}{3.628971in}}%
\pgfpathcurveto{\pgfqpoint{5.467269in}{3.621157in}}{\pgfqpoint{5.462879in}{3.610558in}}{\pgfqpoint{5.462879in}{3.599508in}}%
\pgfpathcurveto{\pgfqpoint{5.462879in}{3.588458in}}{\pgfqpoint{5.467269in}{3.577859in}}{\pgfqpoint{5.475083in}{3.570045in}}%
\pgfpathcurveto{\pgfqpoint{5.482896in}{3.562232in}}{\pgfqpoint{5.493495in}{3.557842in}}{\pgfqpoint{5.504545in}{3.557842in}}%
\pgfpathclose%
\pgfusepath{stroke,fill}%
\end{pgfscope}%
\begin{pgfscope}%
\pgfpathrectangle{\pgfqpoint{0.800000in}{0.528000in}}{\pgfqpoint{4.960000in}{3.696000in}}%
\pgfusepath{clip}%
\pgfsetbuttcap%
\pgfsetroundjoin%
\definecolor{currentfill}{rgb}{0.000000,0.000000,0.000000}%
\pgfsetfillcolor{currentfill}%
\pgfsetlinewidth{1.003750pt}%
\definecolor{currentstroke}{rgb}{0.000000,0.000000,0.000000}%
\pgfsetstrokecolor{currentstroke}%
\pgfsetdash{}{0pt}%
\pgfpathmoveto{\pgfqpoint{5.504545in}{3.072393in}}%
\pgfpathcurveto{\pgfqpoint{5.515596in}{3.072393in}}{\pgfqpoint{5.526195in}{3.076783in}}{\pgfqpoint{5.534008in}{3.084597in}}%
\pgfpathcurveto{\pgfqpoint{5.541822in}{3.092411in}}{\pgfqpoint{5.546212in}{3.103010in}}{\pgfqpoint{5.546212in}{3.114060in}}%
\pgfpathcurveto{\pgfqpoint{5.546212in}{3.125110in}}{\pgfqpoint{5.541822in}{3.135709in}}{\pgfqpoint{5.534008in}{3.143522in}}%
\pgfpathcurveto{\pgfqpoint{5.526195in}{3.151336in}}{\pgfqpoint{5.515596in}{3.155726in}}{\pgfqpoint{5.504545in}{3.155726in}}%
\pgfpathcurveto{\pgfqpoint{5.493495in}{3.155726in}}{\pgfqpoint{5.482896in}{3.151336in}}{\pgfqpoint{5.475083in}{3.143522in}}%
\pgfpathcurveto{\pgfqpoint{5.467269in}{3.135709in}}{\pgfqpoint{5.462879in}{3.125110in}}{\pgfqpoint{5.462879in}{3.114060in}}%
\pgfpathcurveto{\pgfqpoint{5.462879in}{3.103010in}}{\pgfqpoint{5.467269in}{3.092411in}}{\pgfqpoint{5.475083in}{3.084597in}}%
\pgfpathcurveto{\pgfqpoint{5.482896in}{3.076783in}}{\pgfqpoint{5.493495in}{3.072393in}}{\pgfqpoint{5.504545in}{3.072393in}}%
\pgfpathclose%
\pgfusepath{stroke,fill}%
\end{pgfscope}%
\begin{pgfscope}%
\pgfpathrectangle{\pgfqpoint{0.800000in}{0.528000in}}{\pgfqpoint{4.960000in}{3.696000in}}%
\pgfusepath{clip}%
\pgfsetbuttcap%
\pgfsetroundjoin%
\definecolor{currentfill}{rgb}{0.000000,0.000000,0.000000}%
\pgfsetfillcolor{currentfill}%
\pgfsetlinewidth{1.003750pt}%
\definecolor{currentstroke}{rgb}{0.000000,0.000000,0.000000}%
\pgfsetstrokecolor{currentstroke}%
\pgfsetdash{}{0pt}%
\pgfpathmoveto{\pgfqpoint{5.504545in}{3.839402in}}%
\pgfpathcurveto{\pgfqpoint{5.515596in}{3.839402in}}{\pgfqpoint{5.526195in}{3.843792in}}{\pgfqpoint{5.534008in}{3.851605in}}%
\pgfpathcurveto{\pgfqpoint{5.541822in}{3.859419in}}{\pgfqpoint{5.546212in}{3.870018in}}{\pgfqpoint{5.546212in}{3.881068in}}%
\pgfpathcurveto{\pgfqpoint{5.546212in}{3.892118in}}{\pgfqpoint{5.541822in}{3.902717in}}{\pgfqpoint{5.534008in}{3.910531in}}%
\pgfpathcurveto{\pgfqpoint{5.526195in}{3.918345in}}{\pgfqpoint{5.515596in}{3.922735in}}{\pgfqpoint{5.504545in}{3.922735in}}%
\pgfpathcurveto{\pgfqpoint{5.493495in}{3.922735in}}{\pgfqpoint{5.482896in}{3.918345in}}{\pgfqpoint{5.475083in}{3.910531in}}%
\pgfpathcurveto{\pgfqpoint{5.467269in}{3.902717in}}{\pgfqpoint{5.462879in}{3.892118in}}{\pgfqpoint{5.462879in}{3.881068in}}%
\pgfpathcurveto{\pgfqpoint{5.462879in}{3.870018in}}{\pgfqpoint{5.467269in}{3.859419in}}{\pgfqpoint{5.475083in}{3.851605in}}%
\pgfpathcurveto{\pgfqpoint{5.482896in}{3.843792in}}{\pgfqpoint{5.493495in}{3.839402in}}{\pgfqpoint{5.504545in}{3.839402in}}%
\pgfpathclose%
\pgfusepath{stroke,fill}%
\end{pgfscope}%
\begin{pgfscope}%
\pgfpathrectangle{\pgfqpoint{0.800000in}{0.528000in}}{\pgfqpoint{4.960000in}{3.696000in}}%
\pgfusepath{clip}%
\pgfsetbuttcap%
\pgfsetroundjoin%
\definecolor{currentfill}{rgb}{0.000000,0.000000,0.000000}%
\pgfsetfillcolor{currentfill}%
\pgfsetlinewidth{1.003750pt}%
\definecolor{currentstroke}{rgb}{0.000000,0.000000,0.000000}%
\pgfsetstrokecolor{currentstroke}%
\pgfsetdash{}{0pt}%
\pgfpathmoveto{\pgfqpoint{5.504545in}{3.052975in}}%
\pgfpathcurveto{\pgfqpoint{5.515596in}{3.052975in}}{\pgfqpoint{5.526195in}{3.057365in}}{\pgfqpoint{5.534008in}{3.065179in}}%
\pgfpathcurveto{\pgfqpoint{5.541822in}{3.072993in}}{\pgfqpoint{5.546212in}{3.083592in}}{\pgfqpoint{5.546212in}{3.094642in}}%
\pgfpathcurveto{\pgfqpoint{5.546212in}{3.105692in}}{\pgfqpoint{5.541822in}{3.116291in}}{\pgfqpoint{5.534008in}{3.124105in}}%
\pgfpathcurveto{\pgfqpoint{5.526195in}{3.131918in}}{\pgfqpoint{5.515596in}{3.136308in}}{\pgfqpoint{5.504545in}{3.136308in}}%
\pgfpathcurveto{\pgfqpoint{5.493495in}{3.136308in}}{\pgfqpoint{5.482896in}{3.131918in}}{\pgfqpoint{5.475083in}{3.124105in}}%
\pgfpathcurveto{\pgfqpoint{5.467269in}{3.116291in}}{\pgfqpoint{5.462879in}{3.105692in}}{\pgfqpoint{5.462879in}{3.094642in}}%
\pgfpathcurveto{\pgfqpoint{5.462879in}{3.083592in}}{\pgfqpoint{5.467269in}{3.072993in}}{\pgfqpoint{5.475083in}{3.065179in}}%
\pgfpathcurveto{\pgfqpoint{5.482896in}{3.057365in}}{\pgfqpoint{5.493495in}{3.052975in}}{\pgfqpoint{5.504545in}{3.052975in}}%
\pgfpathclose%
\pgfusepath{stroke,fill}%
\end{pgfscope}%
\begin{pgfscope}%
\pgfpathrectangle{\pgfqpoint{0.800000in}{0.528000in}}{\pgfqpoint{4.960000in}{3.696000in}}%
\pgfusepath{clip}%
\pgfsetbuttcap%
\pgfsetroundjoin%
\definecolor{currentfill}{rgb}{0.000000,0.000000,0.000000}%
\pgfsetfillcolor{currentfill}%
\pgfsetlinewidth{1.003750pt}%
\definecolor{currentstroke}{rgb}{0.000000,0.000000,0.000000}%
\pgfsetstrokecolor{currentstroke}%
\pgfsetdash{}{0pt}%
\pgfpathmoveto{\pgfqpoint{5.504545in}{3.451043in}}%
\pgfpathcurveto{\pgfqpoint{5.515596in}{3.451043in}}{\pgfqpoint{5.526195in}{3.455433in}}{\pgfqpoint{5.534008in}{3.463247in}}%
\pgfpathcurveto{\pgfqpoint{5.541822in}{3.471060in}}{\pgfqpoint{5.546212in}{3.481659in}}{\pgfqpoint{5.546212in}{3.492710in}}%
\pgfpathcurveto{\pgfqpoint{5.546212in}{3.503760in}}{\pgfqpoint{5.541822in}{3.514359in}}{\pgfqpoint{5.534008in}{3.522172in}}%
\pgfpathcurveto{\pgfqpoint{5.526195in}{3.529986in}}{\pgfqpoint{5.515596in}{3.534376in}}{\pgfqpoint{5.504545in}{3.534376in}}%
\pgfpathcurveto{\pgfqpoint{5.493495in}{3.534376in}}{\pgfqpoint{5.482896in}{3.529986in}}{\pgfqpoint{5.475083in}{3.522172in}}%
\pgfpathcurveto{\pgfqpoint{5.467269in}{3.514359in}}{\pgfqpoint{5.462879in}{3.503760in}}{\pgfqpoint{5.462879in}{3.492710in}}%
\pgfpathcurveto{\pgfqpoint{5.462879in}{3.481659in}}{\pgfqpoint{5.467269in}{3.471060in}}{\pgfqpoint{5.475083in}{3.463247in}}%
\pgfpathcurveto{\pgfqpoint{5.482896in}{3.455433in}}{\pgfqpoint{5.493495in}{3.451043in}}{\pgfqpoint{5.504545in}{3.451043in}}%
\pgfpathclose%
\pgfusepath{stroke,fill}%
\end{pgfscope}%
\begin{pgfscope}%
\pgfpathrectangle{\pgfqpoint{0.800000in}{0.528000in}}{\pgfqpoint{4.960000in}{3.696000in}}%
\pgfusepath{clip}%
\pgfsetbuttcap%
\pgfsetroundjoin%
\definecolor{currentfill}{rgb}{0.000000,0.000000,0.000000}%
\pgfsetfillcolor{currentfill}%
\pgfsetlinewidth{1.003750pt}%
\definecolor{currentstroke}{rgb}{0.000000,0.000000,0.000000}%
\pgfsetstrokecolor{currentstroke}%
\pgfsetdash{}{0pt}%
\pgfpathmoveto{\pgfqpoint{5.504545in}{3.188901in}}%
\pgfpathcurveto{\pgfqpoint{5.515596in}{3.188901in}}{\pgfqpoint{5.526195in}{3.193291in}}{\pgfqpoint{5.534008in}{3.201105in}}%
\pgfpathcurveto{\pgfqpoint{5.541822in}{3.208918in}}{\pgfqpoint{5.546212in}{3.219517in}}{\pgfqpoint{5.546212in}{3.230567in}}%
\pgfpathcurveto{\pgfqpoint{5.546212in}{3.241617in}}{\pgfqpoint{5.541822in}{3.252217in}}{\pgfqpoint{5.534008in}{3.260030in}}%
\pgfpathcurveto{\pgfqpoint{5.526195in}{3.267844in}}{\pgfqpoint{5.515596in}{3.272234in}}{\pgfqpoint{5.504545in}{3.272234in}}%
\pgfpathcurveto{\pgfqpoint{5.493495in}{3.272234in}}{\pgfqpoint{5.482896in}{3.267844in}}{\pgfqpoint{5.475083in}{3.260030in}}%
\pgfpathcurveto{\pgfqpoint{5.467269in}{3.252217in}}{\pgfqpoint{5.462879in}{3.241617in}}{\pgfqpoint{5.462879in}{3.230567in}}%
\pgfpathcurveto{\pgfqpoint{5.462879in}{3.219517in}}{\pgfqpoint{5.467269in}{3.208918in}}{\pgfqpoint{5.475083in}{3.201105in}}%
\pgfpathcurveto{\pgfqpoint{5.482896in}{3.193291in}}{\pgfqpoint{5.493495in}{3.188901in}}{\pgfqpoint{5.504545in}{3.188901in}}%
\pgfpathclose%
\pgfusepath{stroke,fill}%
\end{pgfscope}%
\begin{pgfscope}%
\pgfpathrectangle{\pgfqpoint{0.800000in}{0.528000in}}{\pgfqpoint{4.960000in}{3.696000in}}%
\pgfusepath{clip}%
\pgfsetbuttcap%
\pgfsetroundjoin%
\definecolor{currentfill}{rgb}{0.000000,0.000000,0.000000}%
\pgfsetfillcolor{currentfill}%
\pgfsetlinewidth{1.003750pt}%
\definecolor{currentstroke}{rgb}{0.000000,0.000000,0.000000}%
\pgfsetstrokecolor{currentstroke}%
\pgfsetdash{}{0pt}%
\pgfpathmoveto{\pgfqpoint{5.504545in}{3.285990in}}%
\pgfpathcurveto{\pgfqpoint{5.515596in}{3.285990in}}{\pgfqpoint{5.526195in}{3.290381in}}{\pgfqpoint{5.534008in}{3.298194in}}%
\pgfpathcurveto{\pgfqpoint{5.541822in}{3.306008in}}{\pgfqpoint{5.546212in}{3.316607in}}{\pgfqpoint{5.546212in}{3.327657in}}%
\pgfpathcurveto{\pgfqpoint{5.546212in}{3.338707in}}{\pgfqpoint{5.541822in}{3.349306in}}{\pgfqpoint{5.534008in}{3.357120in}}%
\pgfpathcurveto{\pgfqpoint{5.526195in}{3.364933in}}{\pgfqpoint{5.515596in}{3.369324in}}{\pgfqpoint{5.504545in}{3.369324in}}%
\pgfpathcurveto{\pgfqpoint{5.493495in}{3.369324in}}{\pgfqpoint{5.482896in}{3.364933in}}{\pgfqpoint{5.475083in}{3.357120in}}%
\pgfpathcurveto{\pgfqpoint{5.467269in}{3.349306in}}{\pgfqpoint{5.462879in}{3.338707in}}{\pgfqpoint{5.462879in}{3.327657in}}%
\pgfpathcurveto{\pgfqpoint{5.462879in}{3.316607in}}{\pgfqpoint{5.467269in}{3.306008in}}{\pgfqpoint{5.475083in}{3.298194in}}%
\pgfpathcurveto{\pgfqpoint{5.482896in}{3.290381in}}{\pgfqpoint{5.493495in}{3.285990in}}{\pgfqpoint{5.504545in}{3.285990in}}%
\pgfpathclose%
\pgfusepath{stroke,fill}%
\end{pgfscope}%
\begin{pgfscope}%
\pgfpathrectangle{\pgfqpoint{0.800000in}{0.528000in}}{\pgfqpoint{4.960000in}{3.696000in}}%
\pgfusepath{clip}%
\pgfsetbuttcap%
\pgfsetroundjoin%
\definecolor{currentfill}{rgb}{0.000000,0.000000,0.000000}%
\pgfsetfillcolor{currentfill}%
\pgfsetlinewidth{1.003750pt}%
\definecolor{currentstroke}{rgb}{0.000000,0.000000,0.000000}%
\pgfsetstrokecolor{currentstroke}%
\pgfsetdash{}{0pt}%
\pgfpathmoveto{\pgfqpoint{5.504545in}{3.227737in}}%
\pgfpathcurveto{\pgfqpoint{5.515596in}{3.227737in}}{\pgfqpoint{5.526195in}{3.232127in}}{\pgfqpoint{5.534008in}{3.239940in}}%
\pgfpathcurveto{\pgfqpoint{5.541822in}{3.247754in}}{\pgfqpoint{5.546212in}{3.258353in}}{\pgfqpoint{5.546212in}{3.269403in}}%
\pgfpathcurveto{\pgfqpoint{5.546212in}{3.280453in}}{\pgfqpoint{5.541822in}{3.291052in}}{\pgfqpoint{5.534008in}{3.298866in}}%
\pgfpathcurveto{\pgfqpoint{5.526195in}{3.306680in}}{\pgfqpoint{5.515596in}{3.311070in}}{\pgfqpoint{5.504545in}{3.311070in}}%
\pgfpathcurveto{\pgfqpoint{5.493495in}{3.311070in}}{\pgfqpoint{5.482896in}{3.306680in}}{\pgfqpoint{5.475083in}{3.298866in}}%
\pgfpathcurveto{\pgfqpoint{5.467269in}{3.291052in}}{\pgfqpoint{5.462879in}{3.280453in}}{\pgfqpoint{5.462879in}{3.269403in}}%
\pgfpathcurveto{\pgfqpoint{5.462879in}{3.258353in}}{\pgfqpoint{5.467269in}{3.247754in}}{\pgfqpoint{5.475083in}{3.239940in}}%
\pgfpathcurveto{\pgfqpoint{5.482896in}{3.232127in}}{\pgfqpoint{5.493495in}{3.227737in}}{\pgfqpoint{5.504545in}{3.227737in}}%
\pgfpathclose%
\pgfusepath{stroke,fill}%
\end{pgfscope}%
\begin{pgfscope}%
\pgfpathrectangle{\pgfqpoint{0.800000in}{0.528000in}}{\pgfqpoint{4.960000in}{3.696000in}}%
\pgfusepath{clip}%
\pgfsetbuttcap%
\pgfsetroundjoin%
\definecolor{currentfill}{rgb}{0.000000,0.000000,0.000000}%
\pgfsetfillcolor{currentfill}%
\pgfsetlinewidth{1.003750pt}%
\definecolor{currentstroke}{rgb}{0.000000,0.000000,0.000000}%
\pgfsetstrokecolor{currentstroke}%
\pgfsetdash{}{0pt}%
\pgfpathmoveto{\pgfqpoint{5.504545in}{3.256863in}}%
\pgfpathcurveto{\pgfqpoint{5.515596in}{3.256863in}}{\pgfqpoint{5.526195in}{3.261254in}}{\pgfqpoint{5.534008in}{3.269067in}}%
\pgfpathcurveto{\pgfqpoint{5.541822in}{3.276881in}}{\pgfqpoint{5.546212in}{3.287480in}}{\pgfqpoint{5.546212in}{3.298530in}}%
\pgfpathcurveto{\pgfqpoint{5.546212in}{3.309580in}}{\pgfqpoint{5.541822in}{3.320179in}}{\pgfqpoint{5.534008in}{3.327993in}}%
\pgfpathcurveto{\pgfqpoint{5.526195in}{3.335807in}}{\pgfqpoint{5.515596in}{3.340197in}}{\pgfqpoint{5.504545in}{3.340197in}}%
\pgfpathcurveto{\pgfqpoint{5.493495in}{3.340197in}}{\pgfqpoint{5.482896in}{3.335807in}}{\pgfqpoint{5.475083in}{3.327993in}}%
\pgfpathcurveto{\pgfqpoint{5.467269in}{3.320179in}}{\pgfqpoint{5.462879in}{3.309580in}}{\pgfqpoint{5.462879in}{3.298530in}}%
\pgfpathcurveto{\pgfqpoint{5.462879in}{3.287480in}}{\pgfqpoint{5.467269in}{3.276881in}}{\pgfqpoint{5.475083in}{3.269067in}}%
\pgfpathcurveto{\pgfqpoint{5.482896in}{3.261254in}}{\pgfqpoint{5.493495in}{3.256863in}}{\pgfqpoint{5.504545in}{3.256863in}}%
\pgfpathclose%
\pgfusepath{stroke,fill}%
\end{pgfscope}%
\begin{pgfscope}%
\pgfpathrectangle{\pgfqpoint{0.800000in}{0.528000in}}{\pgfqpoint{4.960000in}{3.696000in}}%
\pgfusepath{clip}%
\pgfsetbuttcap%
\pgfsetroundjoin%
\definecolor{currentfill}{rgb}{0.000000,0.000000,0.000000}%
\pgfsetfillcolor{currentfill}%
\pgfsetlinewidth{1.003750pt}%
\definecolor{currentstroke}{rgb}{0.000000,0.000000,0.000000}%
\pgfsetstrokecolor{currentstroke}%
\pgfsetdash{}{0pt}%
\pgfpathmoveto{\pgfqpoint{5.504545in}{3.596677in}}%
\pgfpathcurveto{\pgfqpoint{5.515596in}{3.596677in}}{\pgfqpoint{5.526195in}{3.601068in}}{\pgfqpoint{5.534008in}{3.608881in}}%
\pgfpathcurveto{\pgfqpoint{5.541822in}{3.616695in}}{\pgfqpoint{5.546212in}{3.627294in}}{\pgfqpoint{5.546212in}{3.638344in}}%
\pgfpathcurveto{\pgfqpoint{5.546212in}{3.649394in}}{\pgfqpoint{5.541822in}{3.659993in}}{\pgfqpoint{5.534008in}{3.667807in}}%
\pgfpathcurveto{\pgfqpoint{5.526195in}{3.675620in}}{\pgfqpoint{5.515596in}{3.680011in}}{\pgfqpoint{5.504545in}{3.680011in}}%
\pgfpathcurveto{\pgfqpoint{5.493495in}{3.680011in}}{\pgfqpoint{5.482896in}{3.675620in}}{\pgfqpoint{5.475083in}{3.667807in}}%
\pgfpathcurveto{\pgfqpoint{5.467269in}{3.659993in}}{\pgfqpoint{5.462879in}{3.649394in}}{\pgfqpoint{5.462879in}{3.638344in}}%
\pgfpathcurveto{\pgfqpoint{5.462879in}{3.627294in}}{\pgfqpoint{5.467269in}{3.616695in}}{\pgfqpoint{5.475083in}{3.608881in}}%
\pgfpathcurveto{\pgfqpoint{5.482896in}{3.601068in}}{\pgfqpoint{5.493495in}{3.596677in}}{\pgfqpoint{5.504545in}{3.596677in}}%
\pgfpathclose%
\pgfusepath{stroke,fill}%
\end{pgfscope}%
\begin{pgfscope}%
\pgfpathrectangle{\pgfqpoint{0.800000in}{0.528000in}}{\pgfqpoint{4.960000in}{3.696000in}}%
\pgfusepath{clip}%
\pgfsetbuttcap%
\pgfsetroundjoin%
\definecolor{currentfill}{rgb}{0.000000,0.000000,0.000000}%
\pgfsetfillcolor{currentfill}%
\pgfsetlinewidth{1.003750pt}%
\definecolor{currentstroke}{rgb}{0.000000,0.000000,0.000000}%
\pgfsetstrokecolor{currentstroke}%
\pgfsetdash{}{0pt}%
\pgfpathmoveto{\pgfqpoint{5.504545in}{3.120938in}}%
\pgfpathcurveto{\pgfqpoint{5.515596in}{3.120938in}}{\pgfqpoint{5.526195in}{3.125328in}}{\pgfqpoint{5.534008in}{3.133142in}}%
\pgfpathcurveto{\pgfqpoint{5.541822in}{3.140955in}}{\pgfqpoint{5.546212in}{3.151554in}}{\pgfqpoint{5.546212in}{3.162605in}}%
\pgfpathcurveto{\pgfqpoint{5.546212in}{3.173655in}}{\pgfqpoint{5.541822in}{3.184254in}}{\pgfqpoint{5.534008in}{3.192067in}}%
\pgfpathcurveto{\pgfqpoint{5.526195in}{3.199881in}}{\pgfqpoint{5.515596in}{3.204271in}}{\pgfqpoint{5.504545in}{3.204271in}}%
\pgfpathcurveto{\pgfqpoint{5.493495in}{3.204271in}}{\pgfqpoint{5.482896in}{3.199881in}}{\pgfqpoint{5.475083in}{3.192067in}}%
\pgfpathcurveto{\pgfqpoint{5.467269in}{3.184254in}}{\pgfqpoint{5.462879in}{3.173655in}}{\pgfqpoint{5.462879in}{3.162605in}}%
\pgfpathcurveto{\pgfqpoint{5.462879in}{3.151554in}}{\pgfqpoint{5.467269in}{3.140955in}}{\pgfqpoint{5.475083in}{3.133142in}}%
\pgfpathcurveto{\pgfqpoint{5.482896in}{3.125328in}}{\pgfqpoint{5.493495in}{3.120938in}}{\pgfqpoint{5.504545in}{3.120938in}}%
\pgfpathclose%
\pgfusepath{stroke,fill}%
\end{pgfscope}%
\begin{pgfscope}%
\pgfpathrectangle{\pgfqpoint{0.800000in}{0.528000in}}{\pgfqpoint{4.960000in}{3.696000in}}%
\pgfusepath{clip}%
\pgfsetbuttcap%
\pgfsetroundjoin%
\definecolor{currentfill}{rgb}{0.000000,0.000000,0.000000}%
\pgfsetfillcolor{currentfill}%
\pgfsetlinewidth{1.003750pt}%
\definecolor{currentstroke}{rgb}{0.000000,0.000000,0.000000}%
\pgfsetstrokecolor{currentstroke}%
\pgfsetdash{}{0pt}%
\pgfpathmoveto{\pgfqpoint{5.504545in}{3.713185in}}%
\pgfpathcurveto{\pgfqpoint{5.515596in}{3.713185in}}{\pgfqpoint{5.526195in}{3.717575in}}{\pgfqpoint{5.534008in}{3.725389in}}%
\pgfpathcurveto{\pgfqpoint{5.541822in}{3.733203in}}{\pgfqpoint{5.546212in}{3.743802in}}{\pgfqpoint{5.546212in}{3.754852in}}%
\pgfpathcurveto{\pgfqpoint{5.546212in}{3.765902in}}{\pgfqpoint{5.541822in}{3.776501in}}{\pgfqpoint{5.534008in}{3.784314in}}%
\pgfpathcurveto{\pgfqpoint{5.526195in}{3.792128in}}{\pgfqpoint{5.515596in}{3.796518in}}{\pgfqpoint{5.504545in}{3.796518in}}%
\pgfpathcurveto{\pgfqpoint{5.493495in}{3.796518in}}{\pgfqpoint{5.482896in}{3.792128in}}{\pgfqpoint{5.475083in}{3.784314in}}%
\pgfpathcurveto{\pgfqpoint{5.467269in}{3.776501in}}{\pgfqpoint{5.462879in}{3.765902in}}{\pgfqpoint{5.462879in}{3.754852in}}%
\pgfpathcurveto{\pgfqpoint{5.462879in}{3.743802in}}{\pgfqpoint{5.467269in}{3.733203in}}{\pgfqpoint{5.475083in}{3.725389in}}%
\pgfpathcurveto{\pgfqpoint{5.482896in}{3.717575in}}{\pgfqpoint{5.493495in}{3.713185in}}{\pgfqpoint{5.504545in}{3.713185in}}%
\pgfpathclose%
\pgfusepath{stroke,fill}%
\end{pgfscope}%
\begin{pgfscope}%
\pgfpathrectangle{\pgfqpoint{0.800000in}{0.528000in}}{\pgfqpoint{4.960000in}{3.696000in}}%
\pgfusepath{clip}%
\pgfsetbuttcap%
\pgfsetroundjoin%
\definecolor{currentfill}{rgb}{0.000000,0.000000,0.000000}%
\pgfsetfillcolor{currentfill}%
\pgfsetlinewidth{1.003750pt}%
\definecolor{currentstroke}{rgb}{0.000000,0.000000,0.000000}%
\pgfsetstrokecolor{currentstroke}%
\pgfsetdash{}{0pt}%
\pgfpathmoveto{\pgfqpoint{5.504545in}{3.276281in}}%
\pgfpathcurveto{\pgfqpoint{5.515596in}{3.276281in}}{\pgfqpoint{5.526195in}{3.280672in}}{\pgfqpoint{5.534008in}{3.288485in}}%
\pgfpathcurveto{\pgfqpoint{5.541822in}{3.296299in}}{\pgfqpoint{5.546212in}{3.306898in}}{\pgfqpoint{5.546212in}{3.317948in}}%
\pgfpathcurveto{\pgfqpoint{5.546212in}{3.328998in}}{\pgfqpoint{5.541822in}{3.339597in}}{\pgfqpoint{5.534008in}{3.347411in}}%
\pgfpathcurveto{\pgfqpoint{5.526195in}{3.355224in}}{\pgfqpoint{5.515596in}{3.359615in}}{\pgfqpoint{5.504545in}{3.359615in}}%
\pgfpathcurveto{\pgfqpoint{5.493495in}{3.359615in}}{\pgfqpoint{5.482896in}{3.355224in}}{\pgfqpoint{5.475083in}{3.347411in}}%
\pgfpathcurveto{\pgfqpoint{5.467269in}{3.339597in}}{\pgfqpoint{5.462879in}{3.328998in}}{\pgfqpoint{5.462879in}{3.317948in}}%
\pgfpathcurveto{\pgfqpoint{5.462879in}{3.306898in}}{\pgfqpoint{5.467269in}{3.296299in}}{\pgfqpoint{5.475083in}{3.288485in}}%
\pgfpathcurveto{\pgfqpoint{5.482896in}{3.280672in}}{\pgfqpoint{5.493495in}{3.276281in}}{\pgfqpoint{5.504545in}{3.276281in}}%
\pgfpathclose%
\pgfusepath{stroke,fill}%
\end{pgfscope}%
\begin{pgfscope}%
\pgfpathrectangle{\pgfqpoint{0.800000in}{0.528000in}}{\pgfqpoint{4.960000in}{3.696000in}}%
\pgfusepath{clip}%
\pgfsetbuttcap%
\pgfsetroundjoin%
\definecolor{currentfill}{rgb}{0.000000,0.000000,0.000000}%
\pgfsetfillcolor{currentfill}%
\pgfsetlinewidth{1.003750pt}%
\definecolor{currentstroke}{rgb}{0.000000,0.000000,0.000000}%
\pgfsetstrokecolor{currentstroke}%
\pgfsetdash{}{0pt}%
\pgfpathmoveto{\pgfqpoint{5.504545in}{3.606386in}}%
\pgfpathcurveto{\pgfqpoint{5.515596in}{3.606386in}}{\pgfqpoint{5.526195in}{3.610777in}}{\pgfqpoint{5.534008in}{3.618590in}}%
\pgfpathcurveto{\pgfqpoint{5.541822in}{3.626404in}}{\pgfqpoint{5.546212in}{3.637003in}}{\pgfqpoint{5.546212in}{3.648053in}}%
\pgfpathcurveto{\pgfqpoint{5.546212in}{3.659103in}}{\pgfqpoint{5.541822in}{3.669702in}}{\pgfqpoint{5.534008in}{3.677516in}}%
\pgfpathcurveto{\pgfqpoint{5.526195in}{3.685329in}}{\pgfqpoint{5.515596in}{3.689720in}}{\pgfqpoint{5.504545in}{3.689720in}}%
\pgfpathcurveto{\pgfqpoint{5.493495in}{3.689720in}}{\pgfqpoint{5.482896in}{3.685329in}}{\pgfqpoint{5.475083in}{3.677516in}}%
\pgfpathcurveto{\pgfqpoint{5.467269in}{3.669702in}}{\pgfqpoint{5.462879in}{3.659103in}}{\pgfqpoint{5.462879in}{3.648053in}}%
\pgfpathcurveto{\pgfqpoint{5.462879in}{3.637003in}}{\pgfqpoint{5.467269in}{3.626404in}}{\pgfqpoint{5.475083in}{3.618590in}}%
\pgfpathcurveto{\pgfqpoint{5.482896in}{3.610777in}}{\pgfqpoint{5.493495in}{3.606386in}}{\pgfqpoint{5.504545in}{3.606386in}}%
\pgfpathclose%
\pgfusepath{stroke,fill}%
\end{pgfscope}%
\begin{pgfscope}%
\pgfpathrectangle{\pgfqpoint{0.800000in}{0.528000in}}{\pgfqpoint{4.960000in}{3.696000in}}%
\pgfusepath{clip}%
\pgfsetbuttcap%
\pgfsetroundjoin%
\definecolor{currentfill}{rgb}{0.000000,0.000000,0.000000}%
\pgfsetfillcolor{currentfill}%
\pgfsetlinewidth{1.003750pt}%
\definecolor{currentstroke}{rgb}{0.000000,0.000000,0.000000}%
\pgfsetstrokecolor{currentstroke}%
\pgfsetdash{}{0pt}%
\pgfpathmoveto{\pgfqpoint{5.504545in}{3.091811in}}%
\pgfpathcurveto{\pgfqpoint{5.515596in}{3.091811in}}{\pgfqpoint{5.526195in}{3.096201in}}{\pgfqpoint{5.534008in}{3.104015in}}%
\pgfpathcurveto{\pgfqpoint{5.541822in}{3.111828in}}{\pgfqpoint{5.546212in}{3.122428in}}{\pgfqpoint{5.546212in}{3.133478in}}%
\pgfpathcurveto{\pgfqpoint{5.546212in}{3.144528in}}{\pgfqpoint{5.541822in}{3.155127in}}{\pgfqpoint{5.534008in}{3.162940in}}%
\pgfpathcurveto{\pgfqpoint{5.526195in}{3.170754in}}{\pgfqpoint{5.515596in}{3.175144in}}{\pgfqpoint{5.504545in}{3.175144in}}%
\pgfpathcurveto{\pgfqpoint{5.493495in}{3.175144in}}{\pgfqpoint{5.482896in}{3.170754in}}{\pgfqpoint{5.475083in}{3.162940in}}%
\pgfpathcurveto{\pgfqpoint{5.467269in}{3.155127in}}{\pgfqpoint{5.462879in}{3.144528in}}{\pgfqpoint{5.462879in}{3.133478in}}%
\pgfpathcurveto{\pgfqpoint{5.462879in}{3.122428in}}{\pgfqpoint{5.467269in}{3.111828in}}{\pgfqpoint{5.475083in}{3.104015in}}%
\pgfpathcurveto{\pgfqpoint{5.482896in}{3.096201in}}{\pgfqpoint{5.493495in}{3.091811in}}{\pgfqpoint{5.504545in}{3.091811in}}%
\pgfpathclose%
\pgfusepath{stroke,fill}%
\end{pgfscope}%
\begin{pgfscope}%
\pgfpathrectangle{\pgfqpoint{0.800000in}{0.528000in}}{\pgfqpoint{4.960000in}{3.696000in}}%
\pgfusepath{clip}%
\pgfsetbuttcap%
\pgfsetroundjoin%
\definecolor{currentfill}{rgb}{0.000000,0.000000,0.000000}%
\pgfsetfillcolor{currentfill}%
\pgfsetlinewidth{1.003750pt}%
\definecolor{currentstroke}{rgb}{0.000000,0.000000,0.000000}%
\pgfsetstrokecolor{currentstroke}%
\pgfsetdash{}{0pt}%
\pgfpathmoveto{\pgfqpoint{5.504545in}{3.567550in}}%
\pgfpathcurveto{\pgfqpoint{5.515596in}{3.567550in}}{\pgfqpoint{5.526195in}{3.571941in}}{\pgfqpoint{5.534008in}{3.579754in}}%
\pgfpathcurveto{\pgfqpoint{5.541822in}{3.587568in}}{\pgfqpoint{5.546212in}{3.598167in}}{\pgfqpoint{5.546212in}{3.609217in}}%
\pgfpathcurveto{\pgfqpoint{5.546212in}{3.620267in}}{\pgfqpoint{5.541822in}{3.630866in}}{\pgfqpoint{5.534008in}{3.638680in}}%
\pgfpathcurveto{\pgfqpoint{5.526195in}{3.646494in}}{\pgfqpoint{5.515596in}{3.650884in}}{\pgfqpoint{5.504545in}{3.650884in}}%
\pgfpathcurveto{\pgfqpoint{5.493495in}{3.650884in}}{\pgfqpoint{5.482896in}{3.646494in}}{\pgfqpoint{5.475083in}{3.638680in}}%
\pgfpathcurveto{\pgfqpoint{5.467269in}{3.630866in}}{\pgfqpoint{5.462879in}{3.620267in}}{\pgfqpoint{5.462879in}{3.609217in}}%
\pgfpathcurveto{\pgfqpoint{5.462879in}{3.598167in}}{\pgfqpoint{5.467269in}{3.587568in}}{\pgfqpoint{5.475083in}{3.579754in}}%
\pgfpathcurveto{\pgfqpoint{5.482896in}{3.571941in}}{\pgfqpoint{5.493495in}{3.567550in}}{\pgfqpoint{5.504545in}{3.567550in}}%
\pgfpathclose%
\pgfusepath{stroke,fill}%
\end{pgfscope}%
\begin{pgfscope}%
\pgfpathrectangle{\pgfqpoint{0.800000in}{0.528000in}}{\pgfqpoint{4.960000in}{3.696000in}}%
\pgfusepath{clip}%
\pgfsetbuttcap%
\pgfsetroundjoin%
\definecolor{currentfill}{rgb}{0.000000,0.000000,0.000000}%
\pgfsetfillcolor{currentfill}%
\pgfsetlinewidth{1.003750pt}%
\definecolor{currentstroke}{rgb}{0.000000,0.000000,0.000000}%
\pgfsetstrokecolor{currentstroke}%
\pgfsetdash{}{0pt}%
\pgfpathmoveto{\pgfqpoint{5.504545in}{3.091811in}}%
\pgfpathcurveto{\pgfqpoint{5.515596in}{3.091811in}}{\pgfqpoint{5.526195in}{3.096201in}}{\pgfqpoint{5.534008in}{3.104015in}}%
\pgfpathcurveto{\pgfqpoint{5.541822in}{3.111828in}}{\pgfqpoint{5.546212in}{3.122428in}}{\pgfqpoint{5.546212in}{3.133478in}}%
\pgfpathcurveto{\pgfqpoint{5.546212in}{3.144528in}}{\pgfqpoint{5.541822in}{3.155127in}}{\pgfqpoint{5.534008in}{3.162940in}}%
\pgfpathcurveto{\pgfqpoint{5.526195in}{3.170754in}}{\pgfqpoint{5.515596in}{3.175144in}}{\pgfqpoint{5.504545in}{3.175144in}}%
\pgfpathcurveto{\pgfqpoint{5.493495in}{3.175144in}}{\pgfqpoint{5.482896in}{3.170754in}}{\pgfqpoint{5.475083in}{3.162940in}}%
\pgfpathcurveto{\pgfqpoint{5.467269in}{3.155127in}}{\pgfqpoint{5.462879in}{3.144528in}}{\pgfqpoint{5.462879in}{3.133478in}}%
\pgfpathcurveto{\pgfqpoint{5.462879in}{3.122428in}}{\pgfqpoint{5.467269in}{3.111828in}}{\pgfqpoint{5.475083in}{3.104015in}}%
\pgfpathcurveto{\pgfqpoint{5.482896in}{3.096201in}}{\pgfqpoint{5.493495in}{3.091811in}}{\pgfqpoint{5.504545in}{3.091811in}}%
\pgfpathclose%
\pgfusepath{stroke,fill}%
\end{pgfscope}%
\begin{pgfscope}%
\pgfpathrectangle{\pgfqpoint{0.800000in}{0.528000in}}{\pgfqpoint{4.960000in}{3.696000in}}%
\pgfusepath{clip}%
\pgfsetbuttcap%
\pgfsetroundjoin%
\definecolor{currentfill}{rgb}{0.000000,0.000000,0.000000}%
\pgfsetfillcolor{currentfill}%
\pgfsetlinewidth{1.003750pt}%
\definecolor{currentstroke}{rgb}{0.000000,0.000000,0.000000}%
\pgfsetstrokecolor{currentstroke}%
\pgfsetdash{}{0pt}%
\pgfpathmoveto{\pgfqpoint{5.504545in}{3.169483in}}%
\pgfpathcurveto{\pgfqpoint{5.515596in}{3.169483in}}{\pgfqpoint{5.526195in}{3.173873in}}{\pgfqpoint{5.534008in}{3.181687in}}%
\pgfpathcurveto{\pgfqpoint{5.541822in}{3.189500in}}{\pgfqpoint{5.546212in}{3.200099in}}{\pgfqpoint{5.546212in}{3.211149in}}%
\pgfpathcurveto{\pgfqpoint{5.546212in}{3.222200in}}{\pgfqpoint{5.541822in}{3.232799in}}{\pgfqpoint{5.534008in}{3.240612in}}%
\pgfpathcurveto{\pgfqpoint{5.526195in}{3.248426in}}{\pgfqpoint{5.515596in}{3.252816in}}{\pgfqpoint{5.504545in}{3.252816in}}%
\pgfpathcurveto{\pgfqpoint{5.493495in}{3.252816in}}{\pgfqpoint{5.482896in}{3.248426in}}{\pgfqpoint{5.475083in}{3.240612in}}%
\pgfpathcurveto{\pgfqpoint{5.467269in}{3.232799in}}{\pgfqpoint{5.462879in}{3.222200in}}{\pgfqpoint{5.462879in}{3.211149in}}%
\pgfpathcurveto{\pgfqpoint{5.462879in}{3.200099in}}{\pgfqpoint{5.467269in}{3.189500in}}{\pgfqpoint{5.475083in}{3.181687in}}%
\pgfpathcurveto{\pgfqpoint{5.482896in}{3.173873in}}{\pgfqpoint{5.493495in}{3.169483in}}{\pgfqpoint{5.504545in}{3.169483in}}%
\pgfpathclose%
\pgfusepath{stroke,fill}%
\end{pgfscope}%
\begin{pgfscope}%
\pgfpathrectangle{\pgfqpoint{0.800000in}{0.528000in}}{\pgfqpoint{4.960000in}{3.696000in}}%
\pgfusepath{clip}%
\pgfsetbuttcap%
\pgfsetroundjoin%
\definecolor{currentfill}{rgb}{0.000000,0.000000,0.000000}%
\pgfsetfillcolor{currentfill}%
\pgfsetlinewidth{1.003750pt}%
\definecolor{currentstroke}{rgb}{0.000000,0.000000,0.000000}%
\pgfsetstrokecolor{currentstroke}%
\pgfsetdash{}{0pt}%
\pgfpathmoveto{\pgfqpoint{5.504545in}{3.218028in}}%
\pgfpathcurveto{\pgfqpoint{5.515596in}{3.218028in}}{\pgfqpoint{5.526195in}{3.222418in}}{\pgfqpoint{5.534008in}{3.230231in}}%
\pgfpathcurveto{\pgfqpoint{5.541822in}{3.238045in}}{\pgfqpoint{5.546212in}{3.248644in}}{\pgfqpoint{5.546212in}{3.259694in}}%
\pgfpathcurveto{\pgfqpoint{5.546212in}{3.270744in}}{\pgfqpoint{5.541822in}{3.281343in}}{\pgfqpoint{5.534008in}{3.289157in}}%
\pgfpathcurveto{\pgfqpoint{5.526195in}{3.296971in}}{\pgfqpoint{5.515596in}{3.301361in}}{\pgfqpoint{5.504545in}{3.301361in}}%
\pgfpathcurveto{\pgfqpoint{5.493495in}{3.301361in}}{\pgfqpoint{5.482896in}{3.296971in}}{\pgfqpoint{5.475083in}{3.289157in}}%
\pgfpathcurveto{\pgfqpoint{5.467269in}{3.281343in}}{\pgfqpoint{5.462879in}{3.270744in}}{\pgfqpoint{5.462879in}{3.259694in}}%
\pgfpathcurveto{\pgfqpoint{5.462879in}{3.248644in}}{\pgfqpoint{5.467269in}{3.238045in}}{\pgfqpoint{5.475083in}{3.230231in}}%
\pgfpathcurveto{\pgfqpoint{5.482896in}{3.222418in}}{\pgfqpoint{5.493495in}{3.218028in}}{\pgfqpoint{5.504545in}{3.218028in}}%
\pgfpathclose%
\pgfusepath{stroke,fill}%
\end{pgfscope}%
\begin{pgfscope}%
\pgfpathrectangle{\pgfqpoint{0.800000in}{0.528000in}}{\pgfqpoint{4.960000in}{3.696000in}}%
\pgfusepath{clip}%
\pgfsetbuttcap%
\pgfsetroundjoin%
\definecolor{currentfill}{rgb}{0.000000,0.000000,0.000000}%
\pgfsetfillcolor{currentfill}%
\pgfsetlinewidth{1.003750pt}%
\definecolor{currentstroke}{rgb}{0.000000,0.000000,0.000000}%
\pgfsetstrokecolor{currentstroke}%
\pgfsetdash{}{0pt}%
\pgfpathmoveto{\pgfqpoint{5.504545in}{3.052975in}}%
\pgfpathcurveto{\pgfqpoint{5.515596in}{3.052975in}}{\pgfqpoint{5.526195in}{3.057365in}}{\pgfqpoint{5.534008in}{3.065179in}}%
\pgfpathcurveto{\pgfqpoint{5.541822in}{3.072993in}}{\pgfqpoint{5.546212in}{3.083592in}}{\pgfqpoint{5.546212in}{3.094642in}}%
\pgfpathcurveto{\pgfqpoint{5.546212in}{3.105692in}}{\pgfqpoint{5.541822in}{3.116291in}}{\pgfqpoint{5.534008in}{3.124105in}}%
\pgfpathcurveto{\pgfqpoint{5.526195in}{3.131918in}}{\pgfqpoint{5.515596in}{3.136308in}}{\pgfqpoint{5.504545in}{3.136308in}}%
\pgfpathcurveto{\pgfqpoint{5.493495in}{3.136308in}}{\pgfqpoint{5.482896in}{3.131918in}}{\pgfqpoint{5.475083in}{3.124105in}}%
\pgfpathcurveto{\pgfqpoint{5.467269in}{3.116291in}}{\pgfqpoint{5.462879in}{3.105692in}}{\pgfqpoint{5.462879in}{3.094642in}}%
\pgfpathcurveto{\pgfqpoint{5.462879in}{3.083592in}}{\pgfqpoint{5.467269in}{3.072993in}}{\pgfqpoint{5.475083in}{3.065179in}}%
\pgfpathcurveto{\pgfqpoint{5.482896in}{3.057365in}}{\pgfqpoint{5.493495in}{3.052975in}}{\pgfqpoint{5.504545in}{3.052975in}}%
\pgfpathclose%
\pgfusepath{stroke,fill}%
\end{pgfscope}%
\begin{pgfscope}%
\pgfpathrectangle{\pgfqpoint{0.800000in}{0.528000in}}{\pgfqpoint{4.960000in}{3.696000in}}%
\pgfusepath{clip}%
\pgfsetbuttcap%
\pgfsetroundjoin%
\definecolor{currentfill}{rgb}{0.000000,0.000000,0.000000}%
\pgfsetfillcolor{currentfill}%
\pgfsetlinewidth{1.003750pt}%
\definecolor{currentstroke}{rgb}{0.000000,0.000000,0.000000}%
\pgfsetstrokecolor{currentstroke}%
\pgfsetdash{}{0pt}%
\pgfpathmoveto{\pgfqpoint{5.504545in}{3.363662in}}%
\pgfpathcurveto{\pgfqpoint{5.515596in}{3.363662in}}{\pgfqpoint{5.526195in}{3.368052in}}{\pgfqpoint{5.534008in}{3.375866in}}%
\pgfpathcurveto{\pgfqpoint{5.541822in}{3.383680in}}{\pgfqpoint{5.546212in}{3.394279in}}{\pgfqpoint{5.546212in}{3.405329in}}%
\pgfpathcurveto{\pgfqpoint{5.546212in}{3.416379in}}{\pgfqpoint{5.541822in}{3.426978in}}{\pgfqpoint{5.534008in}{3.434792in}}%
\pgfpathcurveto{\pgfqpoint{5.526195in}{3.442605in}}{\pgfqpoint{5.515596in}{3.446995in}}{\pgfqpoint{5.504545in}{3.446995in}}%
\pgfpathcurveto{\pgfqpoint{5.493495in}{3.446995in}}{\pgfqpoint{5.482896in}{3.442605in}}{\pgfqpoint{5.475083in}{3.434792in}}%
\pgfpathcurveto{\pgfqpoint{5.467269in}{3.426978in}}{\pgfqpoint{5.462879in}{3.416379in}}{\pgfqpoint{5.462879in}{3.405329in}}%
\pgfpathcurveto{\pgfqpoint{5.462879in}{3.394279in}}{\pgfqpoint{5.467269in}{3.383680in}}{\pgfqpoint{5.475083in}{3.375866in}}%
\pgfpathcurveto{\pgfqpoint{5.482896in}{3.368052in}}{\pgfqpoint{5.493495in}{3.363662in}}{\pgfqpoint{5.504545in}{3.363662in}}%
\pgfpathclose%
\pgfusepath{stroke,fill}%
\end{pgfscope}%
\begin{pgfscope}%
\pgfpathrectangle{\pgfqpoint{0.800000in}{0.528000in}}{\pgfqpoint{4.960000in}{3.696000in}}%
\pgfusepath{clip}%
\pgfsetbuttcap%
\pgfsetroundjoin%
\definecolor{currentfill}{rgb}{0.000000,0.000000,0.000000}%
\pgfsetfillcolor{currentfill}%
\pgfsetlinewidth{1.003750pt}%
\definecolor{currentstroke}{rgb}{0.000000,0.000000,0.000000}%
\pgfsetstrokecolor{currentstroke}%
\pgfsetdash{}{0pt}%
\pgfpathmoveto{\pgfqpoint{5.504545in}{3.315117in}}%
\pgfpathcurveto{\pgfqpoint{5.515596in}{3.315117in}}{\pgfqpoint{5.526195in}{3.319508in}}{\pgfqpoint{5.534008in}{3.327321in}}%
\pgfpathcurveto{\pgfqpoint{5.541822in}{3.335135in}}{\pgfqpoint{5.546212in}{3.345734in}}{\pgfqpoint{5.546212in}{3.356784in}}%
\pgfpathcurveto{\pgfqpoint{5.546212in}{3.367834in}}{\pgfqpoint{5.541822in}{3.378433in}}{\pgfqpoint{5.534008in}{3.386247in}}%
\pgfpathcurveto{\pgfqpoint{5.526195in}{3.394060in}}{\pgfqpoint{5.515596in}{3.398451in}}{\pgfqpoint{5.504545in}{3.398451in}}%
\pgfpathcurveto{\pgfqpoint{5.493495in}{3.398451in}}{\pgfqpoint{5.482896in}{3.394060in}}{\pgfqpoint{5.475083in}{3.386247in}}%
\pgfpathcurveto{\pgfqpoint{5.467269in}{3.378433in}}{\pgfqpoint{5.462879in}{3.367834in}}{\pgfqpoint{5.462879in}{3.356784in}}%
\pgfpathcurveto{\pgfqpoint{5.462879in}{3.345734in}}{\pgfqpoint{5.467269in}{3.335135in}}{\pgfqpoint{5.475083in}{3.327321in}}%
\pgfpathcurveto{\pgfqpoint{5.482896in}{3.319508in}}{\pgfqpoint{5.493495in}{3.315117in}}{\pgfqpoint{5.504545in}{3.315117in}}%
\pgfpathclose%
\pgfusepath{stroke,fill}%
\end{pgfscope}%
\begin{pgfscope}%
\pgfpathrectangle{\pgfqpoint{0.800000in}{0.528000in}}{\pgfqpoint{4.960000in}{3.696000in}}%
\pgfusepath{clip}%
\pgfsetbuttcap%
\pgfsetroundjoin%
\definecolor{currentfill}{rgb}{0.000000,0.000000,0.000000}%
\pgfsetfillcolor{currentfill}%
\pgfsetlinewidth{1.003750pt}%
\definecolor{currentstroke}{rgb}{0.000000,0.000000,0.000000}%
\pgfsetstrokecolor{currentstroke}%
\pgfsetdash{}{0pt}%
\pgfpathmoveto{\pgfqpoint{5.504545in}{3.043266in}}%
\pgfpathcurveto{\pgfqpoint{5.515596in}{3.043266in}}{\pgfqpoint{5.526195in}{3.047656in}}{\pgfqpoint{5.534008in}{3.055470in}}%
\pgfpathcurveto{\pgfqpoint{5.541822in}{3.063284in}}{\pgfqpoint{5.546212in}{3.073883in}}{\pgfqpoint{5.546212in}{3.084933in}}%
\pgfpathcurveto{\pgfqpoint{5.546212in}{3.095983in}}{\pgfqpoint{5.541822in}{3.106582in}}{\pgfqpoint{5.534008in}{3.114396in}}%
\pgfpathcurveto{\pgfqpoint{5.526195in}{3.122209in}}{\pgfqpoint{5.515596in}{3.126599in}}{\pgfqpoint{5.504545in}{3.126599in}}%
\pgfpathcurveto{\pgfqpoint{5.493495in}{3.126599in}}{\pgfqpoint{5.482896in}{3.122209in}}{\pgfqpoint{5.475083in}{3.114396in}}%
\pgfpathcurveto{\pgfqpoint{5.467269in}{3.106582in}}{\pgfqpoint{5.462879in}{3.095983in}}{\pgfqpoint{5.462879in}{3.084933in}}%
\pgfpathcurveto{\pgfqpoint{5.462879in}{3.073883in}}{\pgfqpoint{5.467269in}{3.063284in}}{\pgfqpoint{5.475083in}{3.055470in}}%
\pgfpathcurveto{\pgfqpoint{5.482896in}{3.047656in}}{\pgfqpoint{5.493495in}{3.043266in}}{\pgfqpoint{5.504545in}{3.043266in}}%
\pgfpathclose%
\pgfusepath{stroke,fill}%
\end{pgfscope}%
\begin{pgfscope}%
\pgfsetbuttcap%
\pgfsetroundjoin%
\definecolor{currentfill}{rgb}{0.000000,0.000000,0.000000}%
\pgfsetfillcolor{currentfill}%
\pgfsetlinewidth{0.803000pt}%
\definecolor{currentstroke}{rgb}{0.000000,0.000000,0.000000}%
\pgfsetstrokecolor{currentstroke}%
\pgfsetdash{}{0pt}%
\pgfsys@defobject{currentmarker}{\pgfqpoint{0.000000in}{-0.048611in}}{\pgfqpoint{0.000000in}{0.000000in}}{%
\pgfpathmoveto{\pgfqpoint{0.000000in}{0.000000in}}%
\pgfpathlineto{\pgfqpoint{0.000000in}{-0.048611in}}%
\pgfusepath{stroke,fill}%
}%
\begin{pgfscope}%
\pgfsys@transformshift{1.025906in}{0.528000in}%
\pgfsys@useobject{currentmarker}{}%
\end{pgfscope}%
\end{pgfscope}%
\begin{pgfscope}%
\definecolor{textcolor}{rgb}{0.000000,0.000000,0.000000}%
\pgfsetstrokecolor{textcolor}%
\pgfsetfillcolor{textcolor}%
\pgftext[x=1.025906in,y=0.430778in,,top]{\color{textcolor}\sffamily\fontsize{10.000000}{12.000000}\selectfont 20}%
\end{pgfscope}%
\begin{pgfscope}%
\pgfsetbuttcap%
\pgfsetroundjoin%
\definecolor{currentfill}{rgb}{0.000000,0.000000,0.000000}%
\pgfsetfillcolor{currentfill}%
\pgfsetlinewidth{0.803000pt}%
\definecolor{currentstroke}{rgb}{0.000000,0.000000,0.000000}%
\pgfsetstrokecolor{currentstroke}%
\pgfsetdash{}{0pt}%
\pgfsys@defobject{currentmarker}{\pgfqpoint{0.000000in}{-0.048611in}}{\pgfqpoint{0.000000in}{0.000000in}}{%
\pgfpathmoveto{\pgfqpoint{0.000000in}{0.000000in}}%
\pgfpathlineto{\pgfqpoint{0.000000in}{-0.048611in}}%
\pgfusepath{stroke,fill}%
}%
\begin{pgfscope}%
\pgfsys@transformshift{2.518786in}{0.528000in}%
\pgfsys@useobject{currentmarker}{}%
\end{pgfscope}%
\end{pgfscope}%
\begin{pgfscope}%
\definecolor{textcolor}{rgb}{0.000000,0.000000,0.000000}%
\pgfsetstrokecolor{textcolor}%
\pgfsetfillcolor{textcolor}%
\pgftext[x=2.518786in,y=0.430778in,,top]{\color{textcolor}\sffamily\fontsize{10.000000}{12.000000}\selectfont 40}%
\end{pgfscope}%
\begin{pgfscope}%
\pgfsetbuttcap%
\pgfsetroundjoin%
\definecolor{currentfill}{rgb}{0.000000,0.000000,0.000000}%
\pgfsetfillcolor{currentfill}%
\pgfsetlinewidth{0.803000pt}%
\definecolor{currentstroke}{rgb}{0.000000,0.000000,0.000000}%
\pgfsetstrokecolor{currentstroke}%
\pgfsetdash{}{0pt}%
\pgfsys@defobject{currentmarker}{\pgfqpoint{0.000000in}{-0.048611in}}{\pgfqpoint{0.000000in}{0.000000in}}{%
\pgfpathmoveto{\pgfqpoint{0.000000in}{0.000000in}}%
\pgfpathlineto{\pgfqpoint{0.000000in}{-0.048611in}}%
\pgfusepath{stroke,fill}%
}%
\begin{pgfscope}%
\pgfsys@transformshift{4.011666in}{0.528000in}%
\pgfsys@useobject{currentmarker}{}%
\end{pgfscope}%
\end{pgfscope}%
\begin{pgfscope}%
\definecolor{textcolor}{rgb}{0.000000,0.000000,0.000000}%
\pgfsetstrokecolor{textcolor}%
\pgfsetfillcolor{textcolor}%
\pgftext[x=4.011666in,y=0.430778in,,top]{\color{textcolor}\sffamily\fontsize{10.000000}{12.000000}\selectfont 60}%
\end{pgfscope}%
\begin{pgfscope}%
\pgfsetbuttcap%
\pgfsetroundjoin%
\definecolor{currentfill}{rgb}{0.000000,0.000000,0.000000}%
\pgfsetfillcolor{currentfill}%
\pgfsetlinewidth{0.803000pt}%
\definecolor{currentstroke}{rgb}{0.000000,0.000000,0.000000}%
\pgfsetstrokecolor{currentstroke}%
\pgfsetdash{}{0pt}%
\pgfsys@defobject{currentmarker}{\pgfqpoint{0.000000in}{-0.048611in}}{\pgfqpoint{0.000000in}{0.000000in}}{%
\pgfpathmoveto{\pgfqpoint{0.000000in}{0.000000in}}%
\pgfpathlineto{\pgfqpoint{0.000000in}{-0.048611in}}%
\pgfusepath{stroke,fill}%
}%
\begin{pgfscope}%
\pgfsys@transformshift{5.504545in}{0.528000in}%
\pgfsys@useobject{currentmarker}{}%
\end{pgfscope}%
\end{pgfscope}%
\begin{pgfscope}%
\definecolor{textcolor}{rgb}{0.000000,0.000000,0.000000}%
\pgfsetstrokecolor{textcolor}%
\pgfsetfillcolor{textcolor}%
\pgftext[x=5.504545in,y=0.430778in,,top]{\color{textcolor}\sffamily\fontsize{10.000000}{12.000000}\selectfont 80}%
\end{pgfscope}%
\begin{pgfscope}%
\definecolor{textcolor}{rgb}{0.000000,0.000000,0.000000}%
\pgfsetstrokecolor{textcolor}%
\pgfsetfillcolor{textcolor}%
\pgftext[x=3.280000in,y=0.240809in,,top]{\color{textcolor}\sffamily\fontsize{10.000000}{12.000000}\selectfont \(\displaystyle k\)}%
\end{pgfscope}%
\begin{pgfscope}%
\pgfsetbuttcap%
\pgfsetroundjoin%
\definecolor{currentfill}{rgb}{0.000000,0.000000,0.000000}%
\pgfsetfillcolor{currentfill}%
\pgfsetlinewidth{0.803000pt}%
\definecolor{currentstroke}{rgb}{0.000000,0.000000,0.000000}%
\pgfsetstrokecolor{currentstroke}%
\pgfsetdash{}{0pt}%
\pgfsys@defobject{currentmarker}{\pgfqpoint{-0.048611in}{0.000000in}}{\pgfqpoint{0.000000in}{0.000000in}}{%
\pgfpathmoveto{\pgfqpoint{0.000000in}{0.000000in}}%
\pgfpathlineto{\pgfqpoint{-0.048611in}{0.000000in}}%
\pgfusepath{stroke,fill}%
}%
\begin{pgfscope}%
\pgfsys@transformshift{0.800000in}{0.910124in}%
\pgfsys@useobject{currentmarker}{}%
\end{pgfscope}%
\end{pgfscope}%
\begin{pgfscope}%
\definecolor{textcolor}{rgb}{0.000000,0.000000,0.000000}%
\pgfsetstrokecolor{textcolor}%
\pgfsetfillcolor{textcolor}%
\pgftext[x=0.526047in,y=0.857362in,left,base]{\color{textcolor}\sffamily\fontsize{10.000000}{12.000000}\selectfont 40}%
\end{pgfscope}%
\begin{pgfscope}%
\pgfsetbuttcap%
\pgfsetroundjoin%
\definecolor{currentfill}{rgb}{0.000000,0.000000,0.000000}%
\pgfsetfillcolor{currentfill}%
\pgfsetlinewidth{0.803000pt}%
\definecolor{currentstroke}{rgb}{0.000000,0.000000,0.000000}%
\pgfsetstrokecolor{currentstroke}%
\pgfsetdash{}{0pt}%
\pgfsys@defobject{currentmarker}{\pgfqpoint{-0.048611in}{0.000000in}}{\pgfqpoint{0.000000in}{0.000000in}}{%
\pgfpathmoveto{\pgfqpoint{0.000000in}{0.000000in}}%
\pgfpathlineto{\pgfqpoint{-0.048611in}{0.000000in}}%
\pgfusepath{stroke,fill}%
}%
\begin{pgfscope}%
\pgfsys@transformshift{0.800000in}{1.298482in}%
\pgfsys@useobject{currentmarker}{}%
\end{pgfscope}%
\end{pgfscope}%
\begin{pgfscope}%
\definecolor{textcolor}{rgb}{0.000000,0.000000,0.000000}%
\pgfsetstrokecolor{textcolor}%
\pgfsetfillcolor{textcolor}%
\pgftext[x=0.526047in,y=1.245721in,left,base]{\color{textcolor}\sffamily\fontsize{10.000000}{12.000000}\selectfont 80}%
\end{pgfscope}%
\begin{pgfscope}%
\pgfsetbuttcap%
\pgfsetroundjoin%
\definecolor{currentfill}{rgb}{0.000000,0.000000,0.000000}%
\pgfsetfillcolor{currentfill}%
\pgfsetlinewidth{0.803000pt}%
\definecolor{currentstroke}{rgb}{0.000000,0.000000,0.000000}%
\pgfsetstrokecolor{currentstroke}%
\pgfsetdash{}{0pt}%
\pgfsys@defobject{currentmarker}{\pgfqpoint{-0.048611in}{0.000000in}}{\pgfqpoint{0.000000in}{0.000000in}}{%
\pgfpathmoveto{\pgfqpoint{0.000000in}{0.000000in}}%
\pgfpathlineto{\pgfqpoint{-0.048611in}{0.000000in}}%
\pgfusepath{stroke,fill}%
}%
\begin{pgfscope}%
\pgfsys@transformshift{0.800000in}{1.686841in}%
\pgfsys@useobject{currentmarker}{}%
\end{pgfscope}%
\end{pgfscope}%
\begin{pgfscope}%
\definecolor{textcolor}{rgb}{0.000000,0.000000,0.000000}%
\pgfsetstrokecolor{textcolor}%
\pgfsetfillcolor{textcolor}%
\pgftext[x=0.437682in,y=1.634080in,left,base]{\color{textcolor}\sffamily\fontsize{10.000000}{12.000000}\selectfont 120}%
\end{pgfscope}%
\begin{pgfscope}%
\pgfsetbuttcap%
\pgfsetroundjoin%
\definecolor{currentfill}{rgb}{0.000000,0.000000,0.000000}%
\pgfsetfillcolor{currentfill}%
\pgfsetlinewidth{0.803000pt}%
\definecolor{currentstroke}{rgb}{0.000000,0.000000,0.000000}%
\pgfsetstrokecolor{currentstroke}%
\pgfsetdash{}{0pt}%
\pgfsys@defobject{currentmarker}{\pgfqpoint{-0.048611in}{0.000000in}}{\pgfqpoint{0.000000in}{0.000000in}}{%
\pgfpathmoveto{\pgfqpoint{0.000000in}{0.000000in}}%
\pgfpathlineto{\pgfqpoint{-0.048611in}{0.000000in}}%
\pgfusepath{stroke,fill}%
}%
\begin{pgfscope}%
\pgfsys@transformshift{0.800000in}{2.075200in}%
\pgfsys@useobject{currentmarker}{}%
\end{pgfscope}%
\end{pgfscope}%
\begin{pgfscope}%
\definecolor{textcolor}{rgb}{0.000000,0.000000,0.000000}%
\pgfsetstrokecolor{textcolor}%
\pgfsetfillcolor{textcolor}%
\pgftext[x=0.437682in,y=2.022438in,left,base]{\color{textcolor}\sffamily\fontsize{10.000000}{12.000000}\selectfont 160}%
\end{pgfscope}%
\begin{pgfscope}%
\pgfsetbuttcap%
\pgfsetroundjoin%
\definecolor{currentfill}{rgb}{0.000000,0.000000,0.000000}%
\pgfsetfillcolor{currentfill}%
\pgfsetlinewidth{0.803000pt}%
\definecolor{currentstroke}{rgb}{0.000000,0.000000,0.000000}%
\pgfsetstrokecolor{currentstroke}%
\pgfsetdash{}{0pt}%
\pgfsys@defobject{currentmarker}{\pgfqpoint{-0.048611in}{0.000000in}}{\pgfqpoint{0.000000in}{0.000000in}}{%
\pgfpathmoveto{\pgfqpoint{0.000000in}{0.000000in}}%
\pgfpathlineto{\pgfqpoint{-0.048611in}{0.000000in}}%
\pgfusepath{stroke,fill}%
}%
\begin{pgfscope}%
\pgfsys@transformshift{0.800000in}{2.463559in}%
\pgfsys@useobject{currentmarker}{}%
\end{pgfscope}%
\end{pgfscope}%
\begin{pgfscope}%
\definecolor{textcolor}{rgb}{0.000000,0.000000,0.000000}%
\pgfsetstrokecolor{textcolor}%
\pgfsetfillcolor{textcolor}%
\pgftext[x=0.437682in,y=2.410797in,left,base]{\color{textcolor}\sffamily\fontsize{10.000000}{12.000000}\selectfont 200}%
\end{pgfscope}%
\begin{pgfscope}%
\pgfsetbuttcap%
\pgfsetroundjoin%
\definecolor{currentfill}{rgb}{0.000000,0.000000,0.000000}%
\pgfsetfillcolor{currentfill}%
\pgfsetlinewidth{0.803000pt}%
\definecolor{currentstroke}{rgb}{0.000000,0.000000,0.000000}%
\pgfsetstrokecolor{currentstroke}%
\pgfsetdash{}{0pt}%
\pgfsys@defobject{currentmarker}{\pgfqpoint{-0.048611in}{0.000000in}}{\pgfqpoint{0.000000in}{0.000000in}}{%
\pgfpathmoveto{\pgfqpoint{0.000000in}{0.000000in}}%
\pgfpathlineto{\pgfqpoint{-0.048611in}{0.000000in}}%
\pgfusepath{stroke,fill}%
}%
\begin{pgfscope}%
\pgfsys@transformshift{0.800000in}{2.851918in}%
\pgfsys@useobject{currentmarker}{}%
\end{pgfscope}%
\end{pgfscope}%
\begin{pgfscope}%
\definecolor{textcolor}{rgb}{0.000000,0.000000,0.000000}%
\pgfsetstrokecolor{textcolor}%
\pgfsetfillcolor{textcolor}%
\pgftext[x=0.437682in,y=2.799156in,left,base]{\color{textcolor}\sffamily\fontsize{10.000000}{12.000000}\selectfont 240}%
\end{pgfscope}%
\begin{pgfscope}%
\pgfsetbuttcap%
\pgfsetroundjoin%
\definecolor{currentfill}{rgb}{0.000000,0.000000,0.000000}%
\pgfsetfillcolor{currentfill}%
\pgfsetlinewidth{0.803000pt}%
\definecolor{currentstroke}{rgb}{0.000000,0.000000,0.000000}%
\pgfsetstrokecolor{currentstroke}%
\pgfsetdash{}{0pt}%
\pgfsys@defobject{currentmarker}{\pgfqpoint{-0.048611in}{0.000000in}}{\pgfqpoint{0.000000in}{0.000000in}}{%
\pgfpathmoveto{\pgfqpoint{0.000000in}{0.000000in}}%
\pgfpathlineto{\pgfqpoint{-0.048611in}{0.000000in}}%
\pgfusepath{stroke,fill}%
}%
\begin{pgfscope}%
\pgfsys@transformshift{0.800000in}{3.240276in}%
\pgfsys@useobject{currentmarker}{}%
\end{pgfscope}%
\end{pgfscope}%
\begin{pgfscope}%
\definecolor{textcolor}{rgb}{0.000000,0.000000,0.000000}%
\pgfsetstrokecolor{textcolor}%
\pgfsetfillcolor{textcolor}%
\pgftext[x=0.437682in,y=3.187515in,left,base]{\color{textcolor}\sffamily\fontsize{10.000000}{12.000000}\selectfont 280}%
\end{pgfscope}%
\begin{pgfscope}%
\pgfsetbuttcap%
\pgfsetroundjoin%
\definecolor{currentfill}{rgb}{0.000000,0.000000,0.000000}%
\pgfsetfillcolor{currentfill}%
\pgfsetlinewidth{0.803000pt}%
\definecolor{currentstroke}{rgb}{0.000000,0.000000,0.000000}%
\pgfsetstrokecolor{currentstroke}%
\pgfsetdash{}{0pt}%
\pgfsys@defobject{currentmarker}{\pgfqpoint{-0.048611in}{0.000000in}}{\pgfqpoint{0.000000in}{0.000000in}}{%
\pgfpathmoveto{\pgfqpoint{0.000000in}{0.000000in}}%
\pgfpathlineto{\pgfqpoint{-0.048611in}{0.000000in}}%
\pgfusepath{stroke,fill}%
}%
\begin{pgfscope}%
\pgfsys@transformshift{0.800000in}{3.628635in}%
\pgfsys@useobject{currentmarker}{}%
\end{pgfscope}%
\end{pgfscope}%
\begin{pgfscope}%
\definecolor{textcolor}{rgb}{0.000000,0.000000,0.000000}%
\pgfsetstrokecolor{textcolor}%
\pgfsetfillcolor{textcolor}%
\pgftext[x=0.437682in,y=3.575874in,left,base]{\color{textcolor}\sffamily\fontsize{10.000000}{12.000000}\selectfont 320}%
\end{pgfscope}%
\begin{pgfscope}%
\pgfsetbuttcap%
\pgfsetroundjoin%
\definecolor{currentfill}{rgb}{0.000000,0.000000,0.000000}%
\pgfsetfillcolor{currentfill}%
\pgfsetlinewidth{0.803000pt}%
\definecolor{currentstroke}{rgb}{0.000000,0.000000,0.000000}%
\pgfsetstrokecolor{currentstroke}%
\pgfsetdash{}{0pt}%
\pgfsys@defobject{currentmarker}{\pgfqpoint{-0.048611in}{0.000000in}}{\pgfqpoint{0.000000in}{0.000000in}}{%
\pgfpathmoveto{\pgfqpoint{0.000000in}{0.000000in}}%
\pgfpathlineto{\pgfqpoint{-0.048611in}{0.000000in}}%
\pgfusepath{stroke,fill}%
}%
\begin{pgfscope}%
\pgfsys@transformshift{0.800000in}{4.016994in}%
\pgfsys@useobject{currentmarker}{}%
\end{pgfscope}%
\end{pgfscope}%
\begin{pgfscope}%
\definecolor{textcolor}{rgb}{0.000000,0.000000,0.000000}%
\pgfsetstrokecolor{textcolor}%
\pgfsetfillcolor{textcolor}%
\pgftext[x=0.437682in,y=3.964232in,left,base]{\color{textcolor}\sffamily\fontsize{10.000000}{12.000000}\selectfont 360}%
\end{pgfscope}%
\begin{pgfscope}%
\definecolor{textcolor}{rgb}{0.000000,0.000000,0.000000}%
\pgfsetstrokecolor{textcolor}%
\pgfsetfillcolor{textcolor}%
\pgftext[x=0.382126in,y=2.376000in,,bottom,rotate=90.000000]{\color{textcolor}\sffamily\fontsize{10.000000}{12.000000}\selectfont Number of GMRES Iterations}%
\end{pgfscope}%
\begin{pgfscope}%
\pgfsetrectcap%
\pgfsetmiterjoin%
\pgfsetlinewidth{0.803000pt}%
\definecolor{currentstroke}{rgb}{0.000000,0.000000,0.000000}%
\pgfsetstrokecolor{currentstroke}%
\pgfsetdash{}{0pt}%
\pgfpathmoveto{\pgfqpoint{0.800000in}{0.528000in}}%
\pgfpathlineto{\pgfqpoint{0.800000in}{4.224000in}}%
\pgfusepath{stroke}%
\end{pgfscope}%
\begin{pgfscope}%
\pgfsetrectcap%
\pgfsetmiterjoin%
\pgfsetlinewidth{0.803000pt}%
\definecolor{currentstroke}{rgb}{0.000000,0.000000,0.000000}%
\pgfsetstrokecolor{currentstroke}%
\pgfsetdash{}{0pt}%
\pgfpathmoveto{\pgfqpoint{5.760000in}{0.528000in}}%
\pgfpathlineto{\pgfqpoint{5.760000in}{4.224000in}}%
\pgfusepath{stroke}%
\end{pgfscope}%
\begin{pgfscope}%
\pgfsetrectcap%
\pgfsetmiterjoin%
\pgfsetlinewidth{0.803000pt}%
\definecolor{currentstroke}{rgb}{0.000000,0.000000,0.000000}%
\pgfsetstrokecolor{currentstroke}%
\pgfsetdash{}{0pt}%
\pgfpathmoveto{\pgfqpoint{0.800000in}{0.528000in}}%
\pgfpathlineto{\pgfqpoint{5.760000in}{0.528000in}}%
\pgfusepath{stroke}%
\end{pgfscope}%
\begin{pgfscope}%
\pgfsetrectcap%
\pgfsetmiterjoin%
\pgfsetlinewidth{0.803000pt}%
\definecolor{currentstroke}{rgb}{0.000000,0.000000,0.000000}%
\pgfsetstrokecolor{currentstroke}%
\pgfsetdash{}{0pt}%
\pgfpathmoveto{\pgfqpoint{0.800000in}{4.224000in}}%
\pgfpathlineto{\pgfqpoint{5.760000in}{4.224000in}}%
\pgfusepath{stroke}%
\end{pgfscope}%
\end{pgfpicture}%
\makeatother%
\endgroup%

\caption{GMRES iteration counts for $\alpha = 0.5$}\label{fig:linfinityn0}
    \end{subfigure}
    
    \begin{subfigure}{\textwidth}
      \centering
%% Creator: Matplotlib, PGF backend
%%
%% To include the figure in your LaTeX document, write
%%   \input{<filename>.pgf}
%%
%% Make sure the required packages are loaded in your preamble
%%   \usepackage{pgf}
%%
%% Figures using additional raster images can only be included by \input if
%% they are in the same directory as the main LaTeX file. For loading figures
%% from other directories you can use the `import` package
%%   \usepackage{import}
%% and then include the figures with
%%   \import{<path to file>}{<filename>.pgf}
%%
%% Matplotlib used the following preamble
%%   \usepackage{fontspec}
%%   \setmainfont{DejaVuSerif.ttf}[Path=/home/owen/progs/firedrake-complex/firedrake/lib/python3.5/site-packages/matplotlib/mpl-data/fonts/ttf/]
%%   \setsansfont{DejaVuSans.ttf}[Path=/home/owen/progs/firedrake-complex/firedrake/lib/python3.5/site-packages/matplotlib/mpl-data/fonts/ttf/]
%%   \setmonofont{DejaVuSansMono.ttf}[Path=/home/owen/progs/firedrake-complex/firedrake/lib/python3.5/site-packages/matplotlib/mpl-data/fonts/ttf/]
%%
\begingroup%
\makeatletter%
\begin{pgfpicture}%
\pgfpathrectangle{\pgfpointorigin}{\pgfqpoint{4.500000in}{2.500000in}}%
\pgfusepath{use as bounding box, clip}%
\begin{pgfscope}%
\pgfsetbuttcap%
\pgfsetmiterjoin%
\definecolor{currentfill}{rgb}{1.000000,1.000000,1.000000}%
\pgfsetfillcolor{currentfill}%
\pgfsetlinewidth{0.000000pt}%
\definecolor{currentstroke}{rgb}{1.000000,1.000000,1.000000}%
\pgfsetstrokecolor{currentstroke}%
\pgfsetdash{}{0pt}%
\pgfpathmoveto{\pgfqpoint{0.000000in}{0.000000in}}%
\pgfpathlineto{\pgfqpoint{4.500000in}{0.000000in}}%
\pgfpathlineto{\pgfqpoint{4.500000in}{2.500000in}}%
\pgfpathlineto{\pgfqpoint{0.000000in}{2.500000in}}%
\pgfpathclose%
\pgfusepath{fill}%
\end{pgfscope}%
\begin{pgfscope}%
\pgfsetbuttcap%
\pgfsetmiterjoin%
\definecolor{currentfill}{rgb}{1.000000,1.000000,1.000000}%
\pgfsetfillcolor{currentfill}%
\pgfsetlinewidth{0.000000pt}%
\definecolor{currentstroke}{rgb}{0.000000,0.000000,0.000000}%
\pgfsetstrokecolor{currentstroke}%
\pgfsetstrokeopacity{0.000000}%
\pgfsetdash{}{0pt}%
\pgfpathmoveto{\pgfqpoint{0.562500in}{0.275000in}}%
\pgfpathlineto{\pgfqpoint{4.050000in}{0.275000in}}%
\pgfpathlineto{\pgfqpoint{4.050000in}{2.200000in}}%
\pgfpathlineto{\pgfqpoint{0.562500in}{2.200000in}}%
\pgfpathclose%
\pgfusepath{fill}%
\end{pgfscope}%
\begin{pgfscope}%
\pgfpathrectangle{\pgfqpoint{0.562500in}{0.275000in}}{\pgfqpoint{3.487500in}{1.925000in}}%
\pgfusepath{clip}%
\pgfsetbuttcap%
\pgfsetroundjoin%
\definecolor{currentfill}{rgb}{0.000000,0.000000,0.000000}%
\pgfsetfillcolor{currentfill}%
\pgfsetlinewidth{1.003750pt}%
\definecolor{currentstroke}{rgb}{0.000000,0.000000,0.000000}%
\pgfsetstrokecolor{currentstroke}%
\pgfsetdash{}{0pt}%
\pgfpathmoveto{\pgfqpoint{0.721249in}{0.356602in}}%
\pgfpathcurveto{\pgfqpoint{0.726774in}{0.356602in}}{\pgfqpoint{0.732073in}{0.358798in}}{\pgfqpoint{0.735980in}{0.362704in}}%
\pgfpathcurveto{\pgfqpoint{0.739887in}{0.366611in}}{\pgfqpoint{0.742082in}{0.371911in}}{\pgfqpoint{0.742082in}{0.377436in}}%
\pgfpathcurveto{\pgfqpoint{0.742082in}{0.382961in}}{\pgfqpoint{0.739887in}{0.388260in}}{\pgfqpoint{0.735980in}{0.392167in}}%
\pgfpathcurveto{\pgfqpoint{0.732073in}{0.396074in}}{\pgfqpoint{0.726774in}{0.398269in}}{\pgfqpoint{0.721249in}{0.398269in}}%
\pgfpathcurveto{\pgfqpoint{0.715724in}{0.398269in}}{\pgfqpoint{0.710424in}{0.396074in}}{\pgfqpoint{0.706518in}{0.392167in}}%
\pgfpathcurveto{\pgfqpoint{0.702611in}{0.388260in}}{\pgfqpoint{0.700416in}{0.382961in}}{\pgfqpoint{0.700416in}{0.377436in}}%
\pgfpathcurveto{\pgfqpoint{0.700416in}{0.371911in}}{\pgfqpoint{0.702611in}{0.366611in}}{\pgfqpoint{0.706518in}{0.362704in}}%
\pgfpathcurveto{\pgfqpoint{0.710424in}{0.358798in}}{\pgfqpoint{0.715724in}{0.356602in}}{\pgfqpoint{0.721249in}{0.356602in}}%
\pgfpathclose%
\pgfusepath{stroke,fill}%
\end{pgfscope}%
\begin{pgfscope}%
\pgfpathrectangle{\pgfqpoint{0.562500in}{0.275000in}}{\pgfqpoint{3.487500in}{1.925000in}}%
\pgfusepath{clip}%
\pgfsetbuttcap%
\pgfsetroundjoin%
\definecolor{currentfill}{rgb}{0.000000,0.000000,0.000000}%
\pgfsetfillcolor{currentfill}%
\pgfsetlinewidth{1.003750pt}%
\definecolor{currentstroke}{rgb}{0.000000,0.000000,0.000000}%
\pgfsetstrokecolor{currentstroke}%
\pgfsetdash{}{0pt}%
\pgfpathmoveto{\pgfqpoint{0.721249in}{0.356602in}}%
\pgfpathcurveto{\pgfqpoint{0.726774in}{0.356602in}}{\pgfqpoint{0.732073in}{0.358798in}}{\pgfqpoint{0.735980in}{0.362704in}}%
\pgfpathcurveto{\pgfqpoint{0.739887in}{0.366611in}}{\pgfqpoint{0.742082in}{0.371911in}}{\pgfqpoint{0.742082in}{0.377436in}}%
\pgfpathcurveto{\pgfqpoint{0.742082in}{0.382961in}}{\pgfqpoint{0.739887in}{0.388260in}}{\pgfqpoint{0.735980in}{0.392167in}}%
\pgfpathcurveto{\pgfqpoint{0.732073in}{0.396074in}}{\pgfqpoint{0.726774in}{0.398269in}}{\pgfqpoint{0.721249in}{0.398269in}}%
\pgfpathcurveto{\pgfqpoint{0.715724in}{0.398269in}}{\pgfqpoint{0.710424in}{0.396074in}}{\pgfqpoint{0.706518in}{0.392167in}}%
\pgfpathcurveto{\pgfqpoint{0.702611in}{0.388260in}}{\pgfqpoint{0.700416in}{0.382961in}}{\pgfqpoint{0.700416in}{0.377436in}}%
\pgfpathcurveto{\pgfqpoint{0.700416in}{0.371911in}}{\pgfqpoint{0.702611in}{0.366611in}}{\pgfqpoint{0.706518in}{0.362704in}}%
\pgfpathcurveto{\pgfqpoint{0.710424in}{0.358798in}}{\pgfqpoint{0.715724in}{0.356602in}}{\pgfqpoint{0.721249in}{0.356602in}}%
\pgfpathclose%
\pgfusepath{stroke,fill}%
\end{pgfscope}%
\begin{pgfscope}%
\pgfpathrectangle{\pgfqpoint{0.562500in}{0.275000in}}{\pgfqpoint{3.487500in}{1.925000in}}%
\pgfusepath{clip}%
\pgfsetbuttcap%
\pgfsetroundjoin%
\definecolor{currentfill}{rgb}{0.000000,0.000000,0.000000}%
\pgfsetfillcolor{currentfill}%
\pgfsetlinewidth{1.003750pt}%
\definecolor{currentstroke}{rgb}{0.000000,0.000000,0.000000}%
\pgfsetstrokecolor{currentstroke}%
\pgfsetdash{}{0pt}%
\pgfpathmoveto{\pgfqpoint{0.721249in}{0.356602in}}%
\pgfpathcurveto{\pgfqpoint{0.726774in}{0.356602in}}{\pgfqpoint{0.732073in}{0.358798in}}{\pgfqpoint{0.735980in}{0.362704in}}%
\pgfpathcurveto{\pgfqpoint{0.739887in}{0.366611in}}{\pgfqpoint{0.742082in}{0.371911in}}{\pgfqpoint{0.742082in}{0.377436in}}%
\pgfpathcurveto{\pgfqpoint{0.742082in}{0.382961in}}{\pgfqpoint{0.739887in}{0.388260in}}{\pgfqpoint{0.735980in}{0.392167in}}%
\pgfpathcurveto{\pgfqpoint{0.732073in}{0.396074in}}{\pgfqpoint{0.726774in}{0.398269in}}{\pgfqpoint{0.721249in}{0.398269in}}%
\pgfpathcurveto{\pgfqpoint{0.715724in}{0.398269in}}{\pgfqpoint{0.710424in}{0.396074in}}{\pgfqpoint{0.706518in}{0.392167in}}%
\pgfpathcurveto{\pgfqpoint{0.702611in}{0.388260in}}{\pgfqpoint{0.700416in}{0.382961in}}{\pgfqpoint{0.700416in}{0.377436in}}%
\pgfpathcurveto{\pgfqpoint{0.700416in}{0.371911in}}{\pgfqpoint{0.702611in}{0.366611in}}{\pgfqpoint{0.706518in}{0.362704in}}%
\pgfpathcurveto{\pgfqpoint{0.710424in}{0.358798in}}{\pgfqpoint{0.715724in}{0.356602in}}{\pgfqpoint{0.721249in}{0.356602in}}%
\pgfpathclose%
\pgfusepath{stroke,fill}%
\end{pgfscope}%
\begin{pgfscope}%
\pgfpathrectangle{\pgfqpoint{0.562500in}{0.275000in}}{\pgfqpoint{3.487500in}{1.925000in}}%
\pgfusepath{clip}%
\pgfsetbuttcap%
\pgfsetroundjoin%
\definecolor{currentfill}{rgb}{0.000000,0.000000,0.000000}%
\pgfsetfillcolor{currentfill}%
\pgfsetlinewidth{1.003750pt}%
\definecolor{currentstroke}{rgb}{0.000000,0.000000,0.000000}%
\pgfsetstrokecolor{currentstroke}%
\pgfsetdash{}{0pt}%
\pgfpathmoveto{\pgfqpoint{0.721249in}{0.356602in}}%
\pgfpathcurveto{\pgfqpoint{0.726774in}{0.356602in}}{\pgfqpoint{0.732073in}{0.358798in}}{\pgfqpoint{0.735980in}{0.362704in}}%
\pgfpathcurveto{\pgfqpoint{0.739887in}{0.366611in}}{\pgfqpoint{0.742082in}{0.371911in}}{\pgfqpoint{0.742082in}{0.377436in}}%
\pgfpathcurveto{\pgfqpoint{0.742082in}{0.382961in}}{\pgfqpoint{0.739887in}{0.388260in}}{\pgfqpoint{0.735980in}{0.392167in}}%
\pgfpathcurveto{\pgfqpoint{0.732073in}{0.396074in}}{\pgfqpoint{0.726774in}{0.398269in}}{\pgfqpoint{0.721249in}{0.398269in}}%
\pgfpathcurveto{\pgfqpoint{0.715724in}{0.398269in}}{\pgfqpoint{0.710424in}{0.396074in}}{\pgfqpoint{0.706518in}{0.392167in}}%
\pgfpathcurveto{\pgfqpoint{0.702611in}{0.388260in}}{\pgfqpoint{0.700416in}{0.382961in}}{\pgfqpoint{0.700416in}{0.377436in}}%
\pgfpathcurveto{\pgfqpoint{0.700416in}{0.371911in}}{\pgfqpoint{0.702611in}{0.366611in}}{\pgfqpoint{0.706518in}{0.362704in}}%
\pgfpathcurveto{\pgfqpoint{0.710424in}{0.358798in}}{\pgfqpoint{0.715724in}{0.356602in}}{\pgfqpoint{0.721249in}{0.356602in}}%
\pgfpathclose%
\pgfusepath{stroke,fill}%
\end{pgfscope}%
\begin{pgfscope}%
\pgfpathrectangle{\pgfqpoint{0.562500in}{0.275000in}}{\pgfqpoint{3.487500in}{1.925000in}}%
\pgfusepath{clip}%
\pgfsetbuttcap%
\pgfsetroundjoin%
\definecolor{currentfill}{rgb}{0.000000,0.000000,0.000000}%
\pgfsetfillcolor{currentfill}%
\pgfsetlinewidth{1.003750pt}%
\definecolor{currentstroke}{rgb}{0.000000,0.000000,0.000000}%
\pgfsetstrokecolor{currentstroke}%
\pgfsetdash{}{0pt}%
\pgfpathmoveto{\pgfqpoint{0.721249in}{0.356602in}}%
\pgfpathcurveto{\pgfqpoint{0.726774in}{0.356602in}}{\pgfqpoint{0.732073in}{0.358798in}}{\pgfqpoint{0.735980in}{0.362704in}}%
\pgfpathcurveto{\pgfqpoint{0.739887in}{0.366611in}}{\pgfqpoint{0.742082in}{0.371911in}}{\pgfqpoint{0.742082in}{0.377436in}}%
\pgfpathcurveto{\pgfqpoint{0.742082in}{0.382961in}}{\pgfqpoint{0.739887in}{0.388260in}}{\pgfqpoint{0.735980in}{0.392167in}}%
\pgfpathcurveto{\pgfqpoint{0.732073in}{0.396074in}}{\pgfqpoint{0.726774in}{0.398269in}}{\pgfqpoint{0.721249in}{0.398269in}}%
\pgfpathcurveto{\pgfqpoint{0.715724in}{0.398269in}}{\pgfqpoint{0.710424in}{0.396074in}}{\pgfqpoint{0.706518in}{0.392167in}}%
\pgfpathcurveto{\pgfqpoint{0.702611in}{0.388260in}}{\pgfqpoint{0.700416in}{0.382961in}}{\pgfqpoint{0.700416in}{0.377436in}}%
\pgfpathcurveto{\pgfqpoint{0.700416in}{0.371911in}}{\pgfqpoint{0.702611in}{0.366611in}}{\pgfqpoint{0.706518in}{0.362704in}}%
\pgfpathcurveto{\pgfqpoint{0.710424in}{0.358798in}}{\pgfqpoint{0.715724in}{0.356602in}}{\pgfqpoint{0.721249in}{0.356602in}}%
\pgfpathclose%
\pgfusepath{stroke,fill}%
\end{pgfscope}%
\begin{pgfscope}%
\pgfpathrectangle{\pgfqpoint{0.562500in}{0.275000in}}{\pgfqpoint{3.487500in}{1.925000in}}%
\pgfusepath{clip}%
\pgfsetbuttcap%
\pgfsetroundjoin%
\definecolor{currentfill}{rgb}{0.000000,0.000000,0.000000}%
\pgfsetfillcolor{currentfill}%
\pgfsetlinewidth{1.003750pt}%
\definecolor{currentstroke}{rgb}{0.000000,0.000000,0.000000}%
\pgfsetstrokecolor{currentstroke}%
\pgfsetdash{}{0pt}%
\pgfpathmoveto{\pgfqpoint{0.721249in}{0.356602in}}%
\pgfpathcurveto{\pgfqpoint{0.726774in}{0.356602in}}{\pgfqpoint{0.732073in}{0.358798in}}{\pgfqpoint{0.735980in}{0.362704in}}%
\pgfpathcurveto{\pgfqpoint{0.739887in}{0.366611in}}{\pgfqpoint{0.742082in}{0.371911in}}{\pgfqpoint{0.742082in}{0.377436in}}%
\pgfpathcurveto{\pgfqpoint{0.742082in}{0.382961in}}{\pgfqpoint{0.739887in}{0.388260in}}{\pgfqpoint{0.735980in}{0.392167in}}%
\pgfpathcurveto{\pgfqpoint{0.732073in}{0.396074in}}{\pgfqpoint{0.726774in}{0.398269in}}{\pgfqpoint{0.721249in}{0.398269in}}%
\pgfpathcurveto{\pgfqpoint{0.715724in}{0.398269in}}{\pgfqpoint{0.710424in}{0.396074in}}{\pgfqpoint{0.706518in}{0.392167in}}%
\pgfpathcurveto{\pgfqpoint{0.702611in}{0.388260in}}{\pgfqpoint{0.700416in}{0.382961in}}{\pgfqpoint{0.700416in}{0.377436in}}%
\pgfpathcurveto{\pgfqpoint{0.700416in}{0.371911in}}{\pgfqpoint{0.702611in}{0.366611in}}{\pgfqpoint{0.706518in}{0.362704in}}%
\pgfpathcurveto{\pgfqpoint{0.710424in}{0.358798in}}{\pgfqpoint{0.715724in}{0.356602in}}{\pgfqpoint{0.721249in}{0.356602in}}%
\pgfpathclose%
\pgfusepath{stroke,fill}%
\end{pgfscope}%
\begin{pgfscope}%
\pgfpathrectangle{\pgfqpoint{0.562500in}{0.275000in}}{\pgfqpoint{3.487500in}{1.925000in}}%
\pgfusepath{clip}%
\pgfsetbuttcap%
\pgfsetroundjoin%
\definecolor{currentfill}{rgb}{0.000000,0.000000,0.000000}%
\pgfsetfillcolor{currentfill}%
\pgfsetlinewidth{1.003750pt}%
\definecolor{currentstroke}{rgb}{0.000000,0.000000,0.000000}%
\pgfsetstrokecolor{currentstroke}%
\pgfsetdash{}{0pt}%
\pgfpathmoveto{\pgfqpoint{0.721249in}{0.356602in}}%
\pgfpathcurveto{\pgfqpoint{0.726774in}{0.356602in}}{\pgfqpoint{0.732073in}{0.358798in}}{\pgfqpoint{0.735980in}{0.362704in}}%
\pgfpathcurveto{\pgfqpoint{0.739887in}{0.366611in}}{\pgfqpoint{0.742082in}{0.371911in}}{\pgfqpoint{0.742082in}{0.377436in}}%
\pgfpathcurveto{\pgfqpoint{0.742082in}{0.382961in}}{\pgfqpoint{0.739887in}{0.388260in}}{\pgfqpoint{0.735980in}{0.392167in}}%
\pgfpathcurveto{\pgfqpoint{0.732073in}{0.396074in}}{\pgfqpoint{0.726774in}{0.398269in}}{\pgfqpoint{0.721249in}{0.398269in}}%
\pgfpathcurveto{\pgfqpoint{0.715724in}{0.398269in}}{\pgfqpoint{0.710424in}{0.396074in}}{\pgfqpoint{0.706518in}{0.392167in}}%
\pgfpathcurveto{\pgfqpoint{0.702611in}{0.388260in}}{\pgfqpoint{0.700416in}{0.382961in}}{\pgfqpoint{0.700416in}{0.377436in}}%
\pgfpathcurveto{\pgfqpoint{0.700416in}{0.371911in}}{\pgfqpoint{0.702611in}{0.366611in}}{\pgfqpoint{0.706518in}{0.362704in}}%
\pgfpathcurveto{\pgfqpoint{0.710424in}{0.358798in}}{\pgfqpoint{0.715724in}{0.356602in}}{\pgfqpoint{0.721249in}{0.356602in}}%
\pgfpathclose%
\pgfusepath{stroke,fill}%
\end{pgfscope}%
\begin{pgfscope}%
\pgfpathrectangle{\pgfqpoint{0.562500in}{0.275000in}}{\pgfqpoint{3.487500in}{1.925000in}}%
\pgfusepath{clip}%
\pgfsetbuttcap%
\pgfsetroundjoin%
\definecolor{currentfill}{rgb}{0.000000,0.000000,0.000000}%
\pgfsetfillcolor{currentfill}%
\pgfsetlinewidth{1.003750pt}%
\definecolor{currentstroke}{rgb}{0.000000,0.000000,0.000000}%
\pgfsetstrokecolor{currentstroke}%
\pgfsetdash{}{0pt}%
\pgfpathmoveto{\pgfqpoint{0.721249in}{0.356602in}}%
\pgfpathcurveto{\pgfqpoint{0.726774in}{0.356602in}}{\pgfqpoint{0.732073in}{0.358798in}}{\pgfqpoint{0.735980in}{0.362704in}}%
\pgfpathcurveto{\pgfqpoint{0.739887in}{0.366611in}}{\pgfqpoint{0.742082in}{0.371911in}}{\pgfqpoint{0.742082in}{0.377436in}}%
\pgfpathcurveto{\pgfqpoint{0.742082in}{0.382961in}}{\pgfqpoint{0.739887in}{0.388260in}}{\pgfqpoint{0.735980in}{0.392167in}}%
\pgfpathcurveto{\pgfqpoint{0.732073in}{0.396074in}}{\pgfqpoint{0.726774in}{0.398269in}}{\pgfqpoint{0.721249in}{0.398269in}}%
\pgfpathcurveto{\pgfqpoint{0.715724in}{0.398269in}}{\pgfqpoint{0.710424in}{0.396074in}}{\pgfqpoint{0.706518in}{0.392167in}}%
\pgfpathcurveto{\pgfqpoint{0.702611in}{0.388260in}}{\pgfqpoint{0.700416in}{0.382961in}}{\pgfqpoint{0.700416in}{0.377436in}}%
\pgfpathcurveto{\pgfqpoint{0.700416in}{0.371911in}}{\pgfqpoint{0.702611in}{0.366611in}}{\pgfqpoint{0.706518in}{0.362704in}}%
\pgfpathcurveto{\pgfqpoint{0.710424in}{0.358798in}}{\pgfqpoint{0.715724in}{0.356602in}}{\pgfqpoint{0.721249in}{0.356602in}}%
\pgfpathclose%
\pgfusepath{stroke,fill}%
\end{pgfscope}%
\begin{pgfscope}%
\pgfpathrectangle{\pgfqpoint{0.562500in}{0.275000in}}{\pgfqpoint{3.487500in}{1.925000in}}%
\pgfusepath{clip}%
\pgfsetbuttcap%
\pgfsetroundjoin%
\definecolor{currentfill}{rgb}{0.000000,0.000000,0.000000}%
\pgfsetfillcolor{currentfill}%
\pgfsetlinewidth{1.003750pt}%
\definecolor{currentstroke}{rgb}{0.000000,0.000000,0.000000}%
\pgfsetstrokecolor{currentstroke}%
\pgfsetdash{}{0pt}%
\pgfpathmoveto{\pgfqpoint{0.721249in}{0.356602in}}%
\pgfpathcurveto{\pgfqpoint{0.726774in}{0.356602in}}{\pgfqpoint{0.732073in}{0.358798in}}{\pgfqpoint{0.735980in}{0.362704in}}%
\pgfpathcurveto{\pgfqpoint{0.739887in}{0.366611in}}{\pgfqpoint{0.742082in}{0.371911in}}{\pgfqpoint{0.742082in}{0.377436in}}%
\pgfpathcurveto{\pgfqpoint{0.742082in}{0.382961in}}{\pgfqpoint{0.739887in}{0.388260in}}{\pgfqpoint{0.735980in}{0.392167in}}%
\pgfpathcurveto{\pgfqpoint{0.732073in}{0.396074in}}{\pgfqpoint{0.726774in}{0.398269in}}{\pgfqpoint{0.721249in}{0.398269in}}%
\pgfpathcurveto{\pgfqpoint{0.715724in}{0.398269in}}{\pgfqpoint{0.710424in}{0.396074in}}{\pgfqpoint{0.706518in}{0.392167in}}%
\pgfpathcurveto{\pgfqpoint{0.702611in}{0.388260in}}{\pgfqpoint{0.700416in}{0.382961in}}{\pgfqpoint{0.700416in}{0.377436in}}%
\pgfpathcurveto{\pgfqpoint{0.700416in}{0.371911in}}{\pgfqpoint{0.702611in}{0.366611in}}{\pgfqpoint{0.706518in}{0.362704in}}%
\pgfpathcurveto{\pgfqpoint{0.710424in}{0.358798in}}{\pgfqpoint{0.715724in}{0.356602in}}{\pgfqpoint{0.721249in}{0.356602in}}%
\pgfpathclose%
\pgfusepath{stroke,fill}%
\end{pgfscope}%
\begin{pgfscope}%
\pgfpathrectangle{\pgfqpoint{0.562500in}{0.275000in}}{\pgfqpoint{3.487500in}{1.925000in}}%
\pgfusepath{clip}%
\pgfsetbuttcap%
\pgfsetroundjoin%
\definecolor{currentfill}{rgb}{0.000000,0.000000,0.000000}%
\pgfsetfillcolor{currentfill}%
\pgfsetlinewidth{1.003750pt}%
\definecolor{currentstroke}{rgb}{0.000000,0.000000,0.000000}%
\pgfsetstrokecolor{currentstroke}%
\pgfsetdash{}{0pt}%
\pgfpathmoveto{\pgfqpoint{0.721249in}{0.356602in}}%
\pgfpathcurveto{\pgfqpoint{0.726774in}{0.356602in}}{\pgfqpoint{0.732073in}{0.358798in}}{\pgfqpoint{0.735980in}{0.362704in}}%
\pgfpathcurveto{\pgfqpoint{0.739887in}{0.366611in}}{\pgfqpoint{0.742082in}{0.371911in}}{\pgfqpoint{0.742082in}{0.377436in}}%
\pgfpathcurveto{\pgfqpoint{0.742082in}{0.382961in}}{\pgfqpoint{0.739887in}{0.388260in}}{\pgfqpoint{0.735980in}{0.392167in}}%
\pgfpathcurveto{\pgfqpoint{0.732073in}{0.396074in}}{\pgfqpoint{0.726774in}{0.398269in}}{\pgfqpoint{0.721249in}{0.398269in}}%
\pgfpathcurveto{\pgfqpoint{0.715724in}{0.398269in}}{\pgfqpoint{0.710424in}{0.396074in}}{\pgfqpoint{0.706518in}{0.392167in}}%
\pgfpathcurveto{\pgfqpoint{0.702611in}{0.388260in}}{\pgfqpoint{0.700416in}{0.382961in}}{\pgfqpoint{0.700416in}{0.377436in}}%
\pgfpathcurveto{\pgfqpoint{0.700416in}{0.371911in}}{\pgfqpoint{0.702611in}{0.366611in}}{\pgfqpoint{0.706518in}{0.362704in}}%
\pgfpathcurveto{\pgfqpoint{0.710424in}{0.358798in}}{\pgfqpoint{0.715724in}{0.356602in}}{\pgfqpoint{0.721249in}{0.356602in}}%
\pgfpathclose%
\pgfusepath{stroke,fill}%
\end{pgfscope}%
\begin{pgfscope}%
\pgfpathrectangle{\pgfqpoint{0.562500in}{0.275000in}}{\pgfqpoint{3.487500in}{1.925000in}}%
\pgfusepath{clip}%
\pgfsetbuttcap%
\pgfsetroundjoin%
\definecolor{currentfill}{rgb}{0.000000,0.000000,0.000000}%
\pgfsetfillcolor{currentfill}%
\pgfsetlinewidth{1.003750pt}%
\definecolor{currentstroke}{rgb}{0.000000,0.000000,0.000000}%
\pgfsetstrokecolor{currentstroke}%
\pgfsetdash{}{0pt}%
\pgfpathmoveto{\pgfqpoint{0.721249in}{1.216635in}}%
\pgfpathcurveto{\pgfqpoint{0.726774in}{1.216635in}}{\pgfqpoint{0.732073in}{1.218830in}}{\pgfqpoint{0.735980in}{1.222736in}}%
\pgfpathcurveto{\pgfqpoint{0.739887in}{1.226643in}}{\pgfqpoint{0.742082in}{1.231943in}}{\pgfqpoint{0.742082in}{1.237468in}}%
\pgfpathcurveto{\pgfqpoint{0.742082in}{1.242993in}}{\pgfqpoint{0.739887in}{1.248292in}}{\pgfqpoint{0.735980in}{1.252199in}}%
\pgfpathcurveto{\pgfqpoint{0.732073in}{1.256106in}}{\pgfqpoint{0.726774in}{1.258301in}}{\pgfqpoint{0.721249in}{1.258301in}}%
\pgfpathcurveto{\pgfqpoint{0.715724in}{1.258301in}}{\pgfqpoint{0.710424in}{1.256106in}}{\pgfqpoint{0.706518in}{1.252199in}}%
\pgfpathcurveto{\pgfqpoint{0.702611in}{1.248292in}}{\pgfqpoint{0.700416in}{1.242993in}}{\pgfqpoint{0.700416in}{1.237468in}}%
\pgfpathcurveto{\pgfqpoint{0.700416in}{1.231943in}}{\pgfqpoint{0.702611in}{1.226643in}}{\pgfqpoint{0.706518in}{1.222736in}}%
\pgfpathcurveto{\pgfqpoint{0.710424in}{1.218830in}}{\pgfqpoint{0.715724in}{1.216635in}}{\pgfqpoint{0.721249in}{1.216635in}}%
\pgfpathclose%
\pgfusepath{stroke,fill}%
\end{pgfscope}%
\begin{pgfscope}%
\pgfpathrectangle{\pgfqpoint{0.562500in}{0.275000in}}{\pgfqpoint{3.487500in}{1.925000in}}%
\pgfusepath{clip}%
\pgfsetbuttcap%
\pgfsetroundjoin%
\definecolor{currentfill}{rgb}{0.000000,0.000000,0.000000}%
\pgfsetfillcolor{currentfill}%
\pgfsetlinewidth{1.003750pt}%
\definecolor{currentstroke}{rgb}{0.000000,0.000000,0.000000}%
\pgfsetstrokecolor{currentstroke}%
\pgfsetdash{}{0pt}%
\pgfpathmoveto{\pgfqpoint{0.721249in}{1.216635in}}%
\pgfpathcurveto{\pgfqpoint{0.726774in}{1.216635in}}{\pgfqpoint{0.732073in}{1.218830in}}{\pgfqpoint{0.735980in}{1.222736in}}%
\pgfpathcurveto{\pgfqpoint{0.739887in}{1.226643in}}{\pgfqpoint{0.742082in}{1.231943in}}{\pgfqpoint{0.742082in}{1.237468in}}%
\pgfpathcurveto{\pgfqpoint{0.742082in}{1.242993in}}{\pgfqpoint{0.739887in}{1.248292in}}{\pgfqpoint{0.735980in}{1.252199in}}%
\pgfpathcurveto{\pgfqpoint{0.732073in}{1.256106in}}{\pgfqpoint{0.726774in}{1.258301in}}{\pgfqpoint{0.721249in}{1.258301in}}%
\pgfpathcurveto{\pgfqpoint{0.715724in}{1.258301in}}{\pgfqpoint{0.710424in}{1.256106in}}{\pgfqpoint{0.706518in}{1.252199in}}%
\pgfpathcurveto{\pgfqpoint{0.702611in}{1.248292in}}{\pgfqpoint{0.700416in}{1.242993in}}{\pgfqpoint{0.700416in}{1.237468in}}%
\pgfpathcurveto{\pgfqpoint{0.700416in}{1.231943in}}{\pgfqpoint{0.702611in}{1.226643in}}{\pgfqpoint{0.706518in}{1.222736in}}%
\pgfpathcurveto{\pgfqpoint{0.710424in}{1.218830in}}{\pgfqpoint{0.715724in}{1.216635in}}{\pgfqpoint{0.721249in}{1.216635in}}%
\pgfpathclose%
\pgfusepath{stroke,fill}%
\end{pgfscope}%
\begin{pgfscope}%
\pgfpathrectangle{\pgfqpoint{0.562500in}{0.275000in}}{\pgfqpoint{3.487500in}{1.925000in}}%
\pgfusepath{clip}%
\pgfsetbuttcap%
\pgfsetroundjoin%
\definecolor{currentfill}{rgb}{0.000000,0.000000,0.000000}%
\pgfsetfillcolor{currentfill}%
\pgfsetlinewidth{1.003750pt}%
\definecolor{currentstroke}{rgb}{0.000000,0.000000,0.000000}%
\pgfsetstrokecolor{currentstroke}%
\pgfsetdash{}{0pt}%
\pgfpathmoveto{\pgfqpoint{0.721249in}{0.356602in}}%
\pgfpathcurveto{\pgfqpoint{0.726774in}{0.356602in}}{\pgfqpoint{0.732073in}{0.358798in}}{\pgfqpoint{0.735980in}{0.362704in}}%
\pgfpathcurveto{\pgfqpoint{0.739887in}{0.366611in}}{\pgfqpoint{0.742082in}{0.371911in}}{\pgfqpoint{0.742082in}{0.377436in}}%
\pgfpathcurveto{\pgfqpoint{0.742082in}{0.382961in}}{\pgfqpoint{0.739887in}{0.388260in}}{\pgfqpoint{0.735980in}{0.392167in}}%
\pgfpathcurveto{\pgfqpoint{0.732073in}{0.396074in}}{\pgfqpoint{0.726774in}{0.398269in}}{\pgfqpoint{0.721249in}{0.398269in}}%
\pgfpathcurveto{\pgfqpoint{0.715724in}{0.398269in}}{\pgfqpoint{0.710424in}{0.396074in}}{\pgfqpoint{0.706518in}{0.392167in}}%
\pgfpathcurveto{\pgfqpoint{0.702611in}{0.388260in}}{\pgfqpoint{0.700416in}{0.382961in}}{\pgfqpoint{0.700416in}{0.377436in}}%
\pgfpathcurveto{\pgfqpoint{0.700416in}{0.371911in}}{\pgfqpoint{0.702611in}{0.366611in}}{\pgfqpoint{0.706518in}{0.362704in}}%
\pgfpathcurveto{\pgfqpoint{0.710424in}{0.358798in}}{\pgfqpoint{0.715724in}{0.356602in}}{\pgfqpoint{0.721249in}{0.356602in}}%
\pgfpathclose%
\pgfusepath{stroke,fill}%
\end{pgfscope}%
\begin{pgfscope}%
\pgfpathrectangle{\pgfqpoint{0.562500in}{0.275000in}}{\pgfqpoint{3.487500in}{1.925000in}}%
\pgfusepath{clip}%
\pgfsetbuttcap%
\pgfsetroundjoin%
\definecolor{currentfill}{rgb}{0.000000,0.000000,0.000000}%
\pgfsetfillcolor{currentfill}%
\pgfsetlinewidth{1.003750pt}%
\definecolor{currentstroke}{rgb}{0.000000,0.000000,0.000000}%
\pgfsetstrokecolor{currentstroke}%
\pgfsetdash{}{0pt}%
\pgfpathmoveto{\pgfqpoint{0.721249in}{0.356602in}}%
\pgfpathcurveto{\pgfqpoint{0.726774in}{0.356602in}}{\pgfqpoint{0.732073in}{0.358798in}}{\pgfqpoint{0.735980in}{0.362704in}}%
\pgfpathcurveto{\pgfqpoint{0.739887in}{0.366611in}}{\pgfqpoint{0.742082in}{0.371911in}}{\pgfqpoint{0.742082in}{0.377436in}}%
\pgfpathcurveto{\pgfqpoint{0.742082in}{0.382961in}}{\pgfqpoint{0.739887in}{0.388260in}}{\pgfqpoint{0.735980in}{0.392167in}}%
\pgfpathcurveto{\pgfqpoint{0.732073in}{0.396074in}}{\pgfqpoint{0.726774in}{0.398269in}}{\pgfqpoint{0.721249in}{0.398269in}}%
\pgfpathcurveto{\pgfqpoint{0.715724in}{0.398269in}}{\pgfqpoint{0.710424in}{0.396074in}}{\pgfqpoint{0.706518in}{0.392167in}}%
\pgfpathcurveto{\pgfqpoint{0.702611in}{0.388260in}}{\pgfqpoint{0.700416in}{0.382961in}}{\pgfqpoint{0.700416in}{0.377436in}}%
\pgfpathcurveto{\pgfqpoint{0.700416in}{0.371911in}}{\pgfqpoint{0.702611in}{0.366611in}}{\pgfqpoint{0.706518in}{0.362704in}}%
\pgfpathcurveto{\pgfqpoint{0.710424in}{0.358798in}}{\pgfqpoint{0.715724in}{0.356602in}}{\pgfqpoint{0.721249in}{0.356602in}}%
\pgfpathclose%
\pgfusepath{stroke,fill}%
\end{pgfscope}%
\begin{pgfscope}%
\pgfpathrectangle{\pgfqpoint{0.562500in}{0.275000in}}{\pgfqpoint{3.487500in}{1.925000in}}%
\pgfusepath{clip}%
\pgfsetbuttcap%
\pgfsetroundjoin%
\definecolor{currentfill}{rgb}{0.000000,0.000000,0.000000}%
\pgfsetfillcolor{currentfill}%
\pgfsetlinewidth{1.003750pt}%
\definecolor{currentstroke}{rgb}{0.000000,0.000000,0.000000}%
\pgfsetstrokecolor{currentstroke}%
\pgfsetdash{}{0pt}%
\pgfpathmoveto{\pgfqpoint{0.721249in}{0.356602in}}%
\pgfpathcurveto{\pgfqpoint{0.726774in}{0.356602in}}{\pgfqpoint{0.732073in}{0.358798in}}{\pgfqpoint{0.735980in}{0.362704in}}%
\pgfpathcurveto{\pgfqpoint{0.739887in}{0.366611in}}{\pgfqpoint{0.742082in}{0.371911in}}{\pgfqpoint{0.742082in}{0.377436in}}%
\pgfpathcurveto{\pgfqpoint{0.742082in}{0.382961in}}{\pgfqpoint{0.739887in}{0.388260in}}{\pgfqpoint{0.735980in}{0.392167in}}%
\pgfpathcurveto{\pgfqpoint{0.732073in}{0.396074in}}{\pgfqpoint{0.726774in}{0.398269in}}{\pgfqpoint{0.721249in}{0.398269in}}%
\pgfpathcurveto{\pgfqpoint{0.715724in}{0.398269in}}{\pgfqpoint{0.710424in}{0.396074in}}{\pgfqpoint{0.706518in}{0.392167in}}%
\pgfpathcurveto{\pgfqpoint{0.702611in}{0.388260in}}{\pgfqpoint{0.700416in}{0.382961in}}{\pgfqpoint{0.700416in}{0.377436in}}%
\pgfpathcurveto{\pgfqpoint{0.700416in}{0.371911in}}{\pgfqpoint{0.702611in}{0.366611in}}{\pgfqpoint{0.706518in}{0.362704in}}%
\pgfpathcurveto{\pgfqpoint{0.710424in}{0.358798in}}{\pgfqpoint{0.715724in}{0.356602in}}{\pgfqpoint{0.721249in}{0.356602in}}%
\pgfpathclose%
\pgfusepath{stroke,fill}%
\end{pgfscope}%
\begin{pgfscope}%
\pgfpathrectangle{\pgfqpoint{0.562500in}{0.275000in}}{\pgfqpoint{3.487500in}{1.925000in}}%
\pgfusepath{clip}%
\pgfsetbuttcap%
\pgfsetroundjoin%
\definecolor{currentfill}{rgb}{0.000000,0.000000,0.000000}%
\pgfsetfillcolor{currentfill}%
\pgfsetlinewidth{1.003750pt}%
\definecolor{currentstroke}{rgb}{0.000000,0.000000,0.000000}%
\pgfsetstrokecolor{currentstroke}%
\pgfsetdash{}{0pt}%
\pgfpathmoveto{\pgfqpoint{0.721249in}{0.356602in}}%
\pgfpathcurveto{\pgfqpoint{0.726774in}{0.356602in}}{\pgfqpoint{0.732073in}{0.358798in}}{\pgfqpoint{0.735980in}{0.362704in}}%
\pgfpathcurveto{\pgfqpoint{0.739887in}{0.366611in}}{\pgfqpoint{0.742082in}{0.371911in}}{\pgfqpoint{0.742082in}{0.377436in}}%
\pgfpathcurveto{\pgfqpoint{0.742082in}{0.382961in}}{\pgfqpoint{0.739887in}{0.388260in}}{\pgfqpoint{0.735980in}{0.392167in}}%
\pgfpathcurveto{\pgfqpoint{0.732073in}{0.396074in}}{\pgfqpoint{0.726774in}{0.398269in}}{\pgfqpoint{0.721249in}{0.398269in}}%
\pgfpathcurveto{\pgfqpoint{0.715724in}{0.398269in}}{\pgfqpoint{0.710424in}{0.396074in}}{\pgfqpoint{0.706518in}{0.392167in}}%
\pgfpathcurveto{\pgfqpoint{0.702611in}{0.388260in}}{\pgfqpoint{0.700416in}{0.382961in}}{\pgfqpoint{0.700416in}{0.377436in}}%
\pgfpathcurveto{\pgfqpoint{0.700416in}{0.371911in}}{\pgfqpoint{0.702611in}{0.366611in}}{\pgfqpoint{0.706518in}{0.362704in}}%
\pgfpathcurveto{\pgfqpoint{0.710424in}{0.358798in}}{\pgfqpoint{0.715724in}{0.356602in}}{\pgfqpoint{0.721249in}{0.356602in}}%
\pgfpathclose%
\pgfusepath{stroke,fill}%
\end{pgfscope}%
\begin{pgfscope}%
\pgfpathrectangle{\pgfqpoint{0.562500in}{0.275000in}}{\pgfqpoint{3.487500in}{1.925000in}}%
\pgfusepath{clip}%
\pgfsetbuttcap%
\pgfsetroundjoin%
\definecolor{currentfill}{rgb}{0.000000,0.000000,0.000000}%
\pgfsetfillcolor{currentfill}%
\pgfsetlinewidth{1.003750pt}%
\definecolor{currentstroke}{rgb}{0.000000,0.000000,0.000000}%
\pgfsetstrokecolor{currentstroke}%
\pgfsetdash{}{0pt}%
\pgfpathmoveto{\pgfqpoint{0.721249in}{0.356602in}}%
\pgfpathcurveto{\pgfqpoint{0.726774in}{0.356602in}}{\pgfqpoint{0.732073in}{0.358798in}}{\pgfqpoint{0.735980in}{0.362704in}}%
\pgfpathcurveto{\pgfqpoint{0.739887in}{0.366611in}}{\pgfqpoint{0.742082in}{0.371911in}}{\pgfqpoint{0.742082in}{0.377436in}}%
\pgfpathcurveto{\pgfqpoint{0.742082in}{0.382961in}}{\pgfqpoint{0.739887in}{0.388260in}}{\pgfqpoint{0.735980in}{0.392167in}}%
\pgfpathcurveto{\pgfqpoint{0.732073in}{0.396074in}}{\pgfqpoint{0.726774in}{0.398269in}}{\pgfqpoint{0.721249in}{0.398269in}}%
\pgfpathcurveto{\pgfqpoint{0.715724in}{0.398269in}}{\pgfqpoint{0.710424in}{0.396074in}}{\pgfqpoint{0.706518in}{0.392167in}}%
\pgfpathcurveto{\pgfqpoint{0.702611in}{0.388260in}}{\pgfqpoint{0.700416in}{0.382961in}}{\pgfqpoint{0.700416in}{0.377436in}}%
\pgfpathcurveto{\pgfqpoint{0.700416in}{0.371911in}}{\pgfqpoint{0.702611in}{0.366611in}}{\pgfqpoint{0.706518in}{0.362704in}}%
\pgfpathcurveto{\pgfqpoint{0.710424in}{0.358798in}}{\pgfqpoint{0.715724in}{0.356602in}}{\pgfqpoint{0.721249in}{0.356602in}}%
\pgfpathclose%
\pgfusepath{stroke,fill}%
\end{pgfscope}%
\begin{pgfscope}%
\pgfpathrectangle{\pgfqpoint{0.562500in}{0.275000in}}{\pgfqpoint{3.487500in}{1.925000in}}%
\pgfusepath{clip}%
\pgfsetbuttcap%
\pgfsetroundjoin%
\definecolor{currentfill}{rgb}{0.000000,0.000000,0.000000}%
\pgfsetfillcolor{currentfill}%
\pgfsetlinewidth{1.003750pt}%
\definecolor{currentstroke}{rgb}{0.000000,0.000000,0.000000}%
\pgfsetstrokecolor{currentstroke}%
\pgfsetdash{}{0pt}%
\pgfpathmoveto{\pgfqpoint{0.721249in}{0.356602in}}%
\pgfpathcurveto{\pgfqpoint{0.726774in}{0.356602in}}{\pgfqpoint{0.732073in}{0.358798in}}{\pgfqpoint{0.735980in}{0.362704in}}%
\pgfpathcurveto{\pgfqpoint{0.739887in}{0.366611in}}{\pgfqpoint{0.742082in}{0.371911in}}{\pgfqpoint{0.742082in}{0.377436in}}%
\pgfpathcurveto{\pgfqpoint{0.742082in}{0.382961in}}{\pgfqpoint{0.739887in}{0.388260in}}{\pgfqpoint{0.735980in}{0.392167in}}%
\pgfpathcurveto{\pgfqpoint{0.732073in}{0.396074in}}{\pgfqpoint{0.726774in}{0.398269in}}{\pgfqpoint{0.721249in}{0.398269in}}%
\pgfpathcurveto{\pgfqpoint{0.715724in}{0.398269in}}{\pgfqpoint{0.710424in}{0.396074in}}{\pgfqpoint{0.706518in}{0.392167in}}%
\pgfpathcurveto{\pgfqpoint{0.702611in}{0.388260in}}{\pgfqpoint{0.700416in}{0.382961in}}{\pgfqpoint{0.700416in}{0.377436in}}%
\pgfpathcurveto{\pgfqpoint{0.700416in}{0.371911in}}{\pgfqpoint{0.702611in}{0.366611in}}{\pgfqpoint{0.706518in}{0.362704in}}%
\pgfpathcurveto{\pgfqpoint{0.710424in}{0.358798in}}{\pgfqpoint{0.715724in}{0.356602in}}{\pgfqpoint{0.721249in}{0.356602in}}%
\pgfpathclose%
\pgfusepath{stroke,fill}%
\end{pgfscope}%
\begin{pgfscope}%
\pgfpathrectangle{\pgfqpoint{0.562500in}{0.275000in}}{\pgfqpoint{3.487500in}{1.925000in}}%
\pgfusepath{clip}%
\pgfsetbuttcap%
\pgfsetroundjoin%
\definecolor{currentfill}{rgb}{0.000000,0.000000,0.000000}%
\pgfsetfillcolor{currentfill}%
\pgfsetlinewidth{1.003750pt}%
\definecolor{currentstroke}{rgb}{0.000000,0.000000,0.000000}%
\pgfsetstrokecolor{currentstroke}%
\pgfsetdash{}{0pt}%
\pgfpathmoveto{\pgfqpoint{0.721249in}{0.356602in}}%
\pgfpathcurveto{\pgfqpoint{0.726774in}{0.356602in}}{\pgfqpoint{0.732073in}{0.358798in}}{\pgfqpoint{0.735980in}{0.362704in}}%
\pgfpathcurveto{\pgfqpoint{0.739887in}{0.366611in}}{\pgfqpoint{0.742082in}{0.371911in}}{\pgfqpoint{0.742082in}{0.377436in}}%
\pgfpathcurveto{\pgfqpoint{0.742082in}{0.382961in}}{\pgfqpoint{0.739887in}{0.388260in}}{\pgfqpoint{0.735980in}{0.392167in}}%
\pgfpathcurveto{\pgfqpoint{0.732073in}{0.396074in}}{\pgfqpoint{0.726774in}{0.398269in}}{\pgfqpoint{0.721249in}{0.398269in}}%
\pgfpathcurveto{\pgfqpoint{0.715724in}{0.398269in}}{\pgfqpoint{0.710424in}{0.396074in}}{\pgfqpoint{0.706518in}{0.392167in}}%
\pgfpathcurveto{\pgfqpoint{0.702611in}{0.388260in}}{\pgfqpoint{0.700416in}{0.382961in}}{\pgfqpoint{0.700416in}{0.377436in}}%
\pgfpathcurveto{\pgfqpoint{0.700416in}{0.371911in}}{\pgfqpoint{0.702611in}{0.366611in}}{\pgfqpoint{0.706518in}{0.362704in}}%
\pgfpathcurveto{\pgfqpoint{0.710424in}{0.358798in}}{\pgfqpoint{0.715724in}{0.356602in}}{\pgfqpoint{0.721249in}{0.356602in}}%
\pgfpathclose%
\pgfusepath{stroke,fill}%
\end{pgfscope}%
\begin{pgfscope}%
\pgfpathrectangle{\pgfqpoint{0.562500in}{0.275000in}}{\pgfqpoint{3.487500in}{1.925000in}}%
\pgfusepath{clip}%
\pgfsetbuttcap%
\pgfsetroundjoin%
\definecolor{currentfill}{rgb}{0.000000,0.000000,0.000000}%
\pgfsetfillcolor{currentfill}%
\pgfsetlinewidth{1.003750pt}%
\definecolor{currentstroke}{rgb}{0.000000,0.000000,0.000000}%
\pgfsetstrokecolor{currentstroke}%
\pgfsetdash{}{0pt}%
\pgfpathmoveto{\pgfqpoint{0.721249in}{0.356602in}}%
\pgfpathcurveto{\pgfqpoint{0.726774in}{0.356602in}}{\pgfqpoint{0.732073in}{0.358798in}}{\pgfqpoint{0.735980in}{0.362704in}}%
\pgfpathcurveto{\pgfqpoint{0.739887in}{0.366611in}}{\pgfqpoint{0.742082in}{0.371911in}}{\pgfqpoint{0.742082in}{0.377436in}}%
\pgfpathcurveto{\pgfqpoint{0.742082in}{0.382961in}}{\pgfqpoint{0.739887in}{0.388260in}}{\pgfqpoint{0.735980in}{0.392167in}}%
\pgfpathcurveto{\pgfqpoint{0.732073in}{0.396074in}}{\pgfqpoint{0.726774in}{0.398269in}}{\pgfqpoint{0.721249in}{0.398269in}}%
\pgfpathcurveto{\pgfqpoint{0.715724in}{0.398269in}}{\pgfqpoint{0.710424in}{0.396074in}}{\pgfqpoint{0.706518in}{0.392167in}}%
\pgfpathcurveto{\pgfqpoint{0.702611in}{0.388260in}}{\pgfqpoint{0.700416in}{0.382961in}}{\pgfqpoint{0.700416in}{0.377436in}}%
\pgfpathcurveto{\pgfqpoint{0.700416in}{0.371911in}}{\pgfqpoint{0.702611in}{0.366611in}}{\pgfqpoint{0.706518in}{0.362704in}}%
\pgfpathcurveto{\pgfqpoint{0.710424in}{0.358798in}}{\pgfqpoint{0.715724in}{0.356602in}}{\pgfqpoint{0.721249in}{0.356602in}}%
\pgfpathclose%
\pgfusepath{stroke,fill}%
\end{pgfscope}%
\begin{pgfscope}%
\pgfpathrectangle{\pgfqpoint{0.562500in}{0.275000in}}{\pgfqpoint{3.487500in}{1.925000in}}%
\pgfusepath{clip}%
\pgfsetbuttcap%
\pgfsetroundjoin%
\definecolor{currentfill}{rgb}{0.000000,0.000000,0.000000}%
\pgfsetfillcolor{currentfill}%
\pgfsetlinewidth{1.003750pt}%
\definecolor{currentstroke}{rgb}{0.000000,0.000000,0.000000}%
\pgfsetstrokecolor{currentstroke}%
\pgfsetdash{}{0pt}%
\pgfpathmoveto{\pgfqpoint{0.721249in}{0.356602in}}%
\pgfpathcurveto{\pgfqpoint{0.726774in}{0.356602in}}{\pgfqpoint{0.732073in}{0.358798in}}{\pgfqpoint{0.735980in}{0.362704in}}%
\pgfpathcurveto{\pgfqpoint{0.739887in}{0.366611in}}{\pgfqpoint{0.742082in}{0.371911in}}{\pgfqpoint{0.742082in}{0.377436in}}%
\pgfpathcurveto{\pgfqpoint{0.742082in}{0.382961in}}{\pgfqpoint{0.739887in}{0.388260in}}{\pgfqpoint{0.735980in}{0.392167in}}%
\pgfpathcurveto{\pgfqpoint{0.732073in}{0.396074in}}{\pgfqpoint{0.726774in}{0.398269in}}{\pgfqpoint{0.721249in}{0.398269in}}%
\pgfpathcurveto{\pgfqpoint{0.715724in}{0.398269in}}{\pgfqpoint{0.710424in}{0.396074in}}{\pgfqpoint{0.706518in}{0.392167in}}%
\pgfpathcurveto{\pgfqpoint{0.702611in}{0.388260in}}{\pgfqpoint{0.700416in}{0.382961in}}{\pgfqpoint{0.700416in}{0.377436in}}%
\pgfpathcurveto{\pgfqpoint{0.700416in}{0.371911in}}{\pgfqpoint{0.702611in}{0.366611in}}{\pgfqpoint{0.706518in}{0.362704in}}%
\pgfpathcurveto{\pgfqpoint{0.710424in}{0.358798in}}{\pgfqpoint{0.715724in}{0.356602in}}{\pgfqpoint{0.721249in}{0.356602in}}%
\pgfpathclose%
\pgfusepath{stroke,fill}%
\end{pgfscope}%
\begin{pgfscope}%
\pgfpathrectangle{\pgfqpoint{0.562500in}{0.275000in}}{\pgfqpoint{3.487500in}{1.925000in}}%
\pgfusepath{clip}%
\pgfsetbuttcap%
\pgfsetroundjoin%
\definecolor{currentfill}{rgb}{0.000000,0.000000,0.000000}%
\pgfsetfillcolor{currentfill}%
\pgfsetlinewidth{1.003750pt}%
\definecolor{currentstroke}{rgb}{0.000000,0.000000,0.000000}%
\pgfsetstrokecolor{currentstroke}%
\pgfsetdash{}{0pt}%
\pgfpathmoveto{\pgfqpoint{0.721249in}{0.356602in}}%
\pgfpathcurveto{\pgfqpoint{0.726774in}{0.356602in}}{\pgfqpoint{0.732073in}{0.358798in}}{\pgfqpoint{0.735980in}{0.362704in}}%
\pgfpathcurveto{\pgfqpoint{0.739887in}{0.366611in}}{\pgfqpoint{0.742082in}{0.371911in}}{\pgfqpoint{0.742082in}{0.377436in}}%
\pgfpathcurveto{\pgfqpoint{0.742082in}{0.382961in}}{\pgfqpoint{0.739887in}{0.388260in}}{\pgfqpoint{0.735980in}{0.392167in}}%
\pgfpathcurveto{\pgfqpoint{0.732073in}{0.396074in}}{\pgfqpoint{0.726774in}{0.398269in}}{\pgfqpoint{0.721249in}{0.398269in}}%
\pgfpathcurveto{\pgfqpoint{0.715724in}{0.398269in}}{\pgfqpoint{0.710424in}{0.396074in}}{\pgfqpoint{0.706518in}{0.392167in}}%
\pgfpathcurveto{\pgfqpoint{0.702611in}{0.388260in}}{\pgfqpoint{0.700416in}{0.382961in}}{\pgfqpoint{0.700416in}{0.377436in}}%
\pgfpathcurveto{\pgfqpoint{0.700416in}{0.371911in}}{\pgfqpoint{0.702611in}{0.366611in}}{\pgfqpoint{0.706518in}{0.362704in}}%
\pgfpathcurveto{\pgfqpoint{0.710424in}{0.358798in}}{\pgfqpoint{0.715724in}{0.356602in}}{\pgfqpoint{0.721249in}{0.356602in}}%
\pgfpathclose%
\pgfusepath{stroke,fill}%
\end{pgfscope}%
\begin{pgfscope}%
\pgfpathrectangle{\pgfqpoint{0.562500in}{0.275000in}}{\pgfqpoint{3.487500in}{1.925000in}}%
\pgfusepath{clip}%
\pgfsetbuttcap%
\pgfsetroundjoin%
\definecolor{currentfill}{rgb}{0.000000,0.000000,0.000000}%
\pgfsetfillcolor{currentfill}%
\pgfsetlinewidth{1.003750pt}%
\definecolor{currentstroke}{rgb}{0.000000,0.000000,0.000000}%
\pgfsetstrokecolor{currentstroke}%
\pgfsetdash{}{0pt}%
\pgfpathmoveto{\pgfqpoint{0.721249in}{0.356602in}}%
\pgfpathcurveto{\pgfqpoint{0.726774in}{0.356602in}}{\pgfqpoint{0.732073in}{0.358798in}}{\pgfqpoint{0.735980in}{0.362704in}}%
\pgfpathcurveto{\pgfqpoint{0.739887in}{0.366611in}}{\pgfqpoint{0.742082in}{0.371911in}}{\pgfqpoint{0.742082in}{0.377436in}}%
\pgfpathcurveto{\pgfqpoint{0.742082in}{0.382961in}}{\pgfqpoint{0.739887in}{0.388260in}}{\pgfqpoint{0.735980in}{0.392167in}}%
\pgfpathcurveto{\pgfqpoint{0.732073in}{0.396074in}}{\pgfqpoint{0.726774in}{0.398269in}}{\pgfqpoint{0.721249in}{0.398269in}}%
\pgfpathcurveto{\pgfqpoint{0.715724in}{0.398269in}}{\pgfqpoint{0.710424in}{0.396074in}}{\pgfqpoint{0.706518in}{0.392167in}}%
\pgfpathcurveto{\pgfqpoint{0.702611in}{0.388260in}}{\pgfqpoint{0.700416in}{0.382961in}}{\pgfqpoint{0.700416in}{0.377436in}}%
\pgfpathcurveto{\pgfqpoint{0.700416in}{0.371911in}}{\pgfqpoint{0.702611in}{0.366611in}}{\pgfqpoint{0.706518in}{0.362704in}}%
\pgfpathcurveto{\pgfqpoint{0.710424in}{0.358798in}}{\pgfqpoint{0.715724in}{0.356602in}}{\pgfqpoint{0.721249in}{0.356602in}}%
\pgfpathclose%
\pgfusepath{stroke,fill}%
\end{pgfscope}%
\begin{pgfscope}%
\pgfpathrectangle{\pgfqpoint{0.562500in}{0.275000in}}{\pgfqpoint{3.487500in}{1.925000in}}%
\pgfusepath{clip}%
\pgfsetbuttcap%
\pgfsetroundjoin%
\definecolor{currentfill}{rgb}{0.000000,0.000000,0.000000}%
\pgfsetfillcolor{currentfill}%
\pgfsetlinewidth{1.003750pt}%
\definecolor{currentstroke}{rgb}{0.000000,0.000000,0.000000}%
\pgfsetstrokecolor{currentstroke}%
\pgfsetdash{}{0pt}%
\pgfpathmoveto{\pgfqpoint{0.721249in}{1.216635in}}%
\pgfpathcurveto{\pgfqpoint{0.726774in}{1.216635in}}{\pgfqpoint{0.732073in}{1.218830in}}{\pgfqpoint{0.735980in}{1.222736in}}%
\pgfpathcurveto{\pgfqpoint{0.739887in}{1.226643in}}{\pgfqpoint{0.742082in}{1.231943in}}{\pgfqpoint{0.742082in}{1.237468in}}%
\pgfpathcurveto{\pgfqpoint{0.742082in}{1.242993in}}{\pgfqpoint{0.739887in}{1.248292in}}{\pgfqpoint{0.735980in}{1.252199in}}%
\pgfpathcurveto{\pgfqpoint{0.732073in}{1.256106in}}{\pgfqpoint{0.726774in}{1.258301in}}{\pgfqpoint{0.721249in}{1.258301in}}%
\pgfpathcurveto{\pgfqpoint{0.715724in}{1.258301in}}{\pgfqpoint{0.710424in}{1.256106in}}{\pgfqpoint{0.706518in}{1.252199in}}%
\pgfpathcurveto{\pgfqpoint{0.702611in}{1.248292in}}{\pgfqpoint{0.700416in}{1.242993in}}{\pgfqpoint{0.700416in}{1.237468in}}%
\pgfpathcurveto{\pgfqpoint{0.700416in}{1.231943in}}{\pgfqpoint{0.702611in}{1.226643in}}{\pgfqpoint{0.706518in}{1.222736in}}%
\pgfpathcurveto{\pgfqpoint{0.710424in}{1.218830in}}{\pgfqpoint{0.715724in}{1.216635in}}{\pgfqpoint{0.721249in}{1.216635in}}%
\pgfpathclose%
\pgfusepath{stroke,fill}%
\end{pgfscope}%
\begin{pgfscope}%
\pgfpathrectangle{\pgfqpoint{0.562500in}{0.275000in}}{\pgfqpoint{3.487500in}{1.925000in}}%
\pgfusepath{clip}%
\pgfsetbuttcap%
\pgfsetroundjoin%
\definecolor{currentfill}{rgb}{0.000000,0.000000,0.000000}%
\pgfsetfillcolor{currentfill}%
\pgfsetlinewidth{1.003750pt}%
\definecolor{currentstroke}{rgb}{0.000000,0.000000,0.000000}%
\pgfsetstrokecolor{currentstroke}%
\pgfsetdash{}{0pt}%
\pgfpathmoveto{\pgfqpoint{0.721249in}{0.356602in}}%
\pgfpathcurveto{\pgfqpoint{0.726774in}{0.356602in}}{\pgfqpoint{0.732073in}{0.358798in}}{\pgfqpoint{0.735980in}{0.362704in}}%
\pgfpathcurveto{\pgfqpoint{0.739887in}{0.366611in}}{\pgfqpoint{0.742082in}{0.371911in}}{\pgfqpoint{0.742082in}{0.377436in}}%
\pgfpathcurveto{\pgfqpoint{0.742082in}{0.382961in}}{\pgfqpoint{0.739887in}{0.388260in}}{\pgfqpoint{0.735980in}{0.392167in}}%
\pgfpathcurveto{\pgfqpoint{0.732073in}{0.396074in}}{\pgfqpoint{0.726774in}{0.398269in}}{\pgfqpoint{0.721249in}{0.398269in}}%
\pgfpathcurveto{\pgfqpoint{0.715724in}{0.398269in}}{\pgfqpoint{0.710424in}{0.396074in}}{\pgfqpoint{0.706518in}{0.392167in}}%
\pgfpathcurveto{\pgfqpoint{0.702611in}{0.388260in}}{\pgfqpoint{0.700416in}{0.382961in}}{\pgfqpoint{0.700416in}{0.377436in}}%
\pgfpathcurveto{\pgfqpoint{0.700416in}{0.371911in}}{\pgfqpoint{0.702611in}{0.366611in}}{\pgfqpoint{0.706518in}{0.362704in}}%
\pgfpathcurveto{\pgfqpoint{0.710424in}{0.358798in}}{\pgfqpoint{0.715724in}{0.356602in}}{\pgfqpoint{0.721249in}{0.356602in}}%
\pgfpathclose%
\pgfusepath{stroke,fill}%
\end{pgfscope}%
\begin{pgfscope}%
\pgfpathrectangle{\pgfqpoint{0.562500in}{0.275000in}}{\pgfqpoint{3.487500in}{1.925000in}}%
\pgfusepath{clip}%
\pgfsetbuttcap%
\pgfsetroundjoin%
\definecolor{currentfill}{rgb}{0.000000,0.000000,0.000000}%
\pgfsetfillcolor{currentfill}%
\pgfsetlinewidth{1.003750pt}%
\definecolor{currentstroke}{rgb}{0.000000,0.000000,0.000000}%
\pgfsetstrokecolor{currentstroke}%
\pgfsetdash{}{0pt}%
\pgfpathmoveto{\pgfqpoint{0.721249in}{0.356602in}}%
\pgfpathcurveto{\pgfqpoint{0.726774in}{0.356602in}}{\pgfqpoint{0.732073in}{0.358798in}}{\pgfqpoint{0.735980in}{0.362704in}}%
\pgfpathcurveto{\pgfqpoint{0.739887in}{0.366611in}}{\pgfqpoint{0.742082in}{0.371911in}}{\pgfqpoint{0.742082in}{0.377436in}}%
\pgfpathcurveto{\pgfqpoint{0.742082in}{0.382961in}}{\pgfqpoint{0.739887in}{0.388260in}}{\pgfqpoint{0.735980in}{0.392167in}}%
\pgfpathcurveto{\pgfqpoint{0.732073in}{0.396074in}}{\pgfqpoint{0.726774in}{0.398269in}}{\pgfqpoint{0.721249in}{0.398269in}}%
\pgfpathcurveto{\pgfqpoint{0.715724in}{0.398269in}}{\pgfqpoint{0.710424in}{0.396074in}}{\pgfqpoint{0.706518in}{0.392167in}}%
\pgfpathcurveto{\pgfqpoint{0.702611in}{0.388260in}}{\pgfqpoint{0.700416in}{0.382961in}}{\pgfqpoint{0.700416in}{0.377436in}}%
\pgfpathcurveto{\pgfqpoint{0.700416in}{0.371911in}}{\pgfqpoint{0.702611in}{0.366611in}}{\pgfqpoint{0.706518in}{0.362704in}}%
\pgfpathcurveto{\pgfqpoint{0.710424in}{0.358798in}}{\pgfqpoint{0.715724in}{0.356602in}}{\pgfqpoint{0.721249in}{0.356602in}}%
\pgfpathclose%
\pgfusepath{stroke,fill}%
\end{pgfscope}%
\begin{pgfscope}%
\pgfpathrectangle{\pgfqpoint{0.562500in}{0.275000in}}{\pgfqpoint{3.487500in}{1.925000in}}%
\pgfusepath{clip}%
\pgfsetbuttcap%
\pgfsetroundjoin%
\definecolor{currentfill}{rgb}{0.000000,0.000000,0.000000}%
\pgfsetfillcolor{currentfill}%
\pgfsetlinewidth{1.003750pt}%
\definecolor{currentstroke}{rgb}{0.000000,0.000000,0.000000}%
\pgfsetstrokecolor{currentstroke}%
\pgfsetdash{}{0pt}%
\pgfpathmoveto{\pgfqpoint{0.721249in}{0.356602in}}%
\pgfpathcurveto{\pgfqpoint{0.726774in}{0.356602in}}{\pgfqpoint{0.732073in}{0.358798in}}{\pgfqpoint{0.735980in}{0.362704in}}%
\pgfpathcurveto{\pgfqpoint{0.739887in}{0.366611in}}{\pgfqpoint{0.742082in}{0.371911in}}{\pgfqpoint{0.742082in}{0.377436in}}%
\pgfpathcurveto{\pgfqpoint{0.742082in}{0.382961in}}{\pgfqpoint{0.739887in}{0.388260in}}{\pgfqpoint{0.735980in}{0.392167in}}%
\pgfpathcurveto{\pgfqpoint{0.732073in}{0.396074in}}{\pgfqpoint{0.726774in}{0.398269in}}{\pgfqpoint{0.721249in}{0.398269in}}%
\pgfpathcurveto{\pgfqpoint{0.715724in}{0.398269in}}{\pgfqpoint{0.710424in}{0.396074in}}{\pgfqpoint{0.706518in}{0.392167in}}%
\pgfpathcurveto{\pgfqpoint{0.702611in}{0.388260in}}{\pgfqpoint{0.700416in}{0.382961in}}{\pgfqpoint{0.700416in}{0.377436in}}%
\pgfpathcurveto{\pgfqpoint{0.700416in}{0.371911in}}{\pgfqpoint{0.702611in}{0.366611in}}{\pgfqpoint{0.706518in}{0.362704in}}%
\pgfpathcurveto{\pgfqpoint{0.710424in}{0.358798in}}{\pgfqpoint{0.715724in}{0.356602in}}{\pgfqpoint{0.721249in}{0.356602in}}%
\pgfpathclose%
\pgfusepath{stroke,fill}%
\end{pgfscope}%
\begin{pgfscope}%
\pgfpathrectangle{\pgfqpoint{0.562500in}{0.275000in}}{\pgfqpoint{3.487500in}{1.925000in}}%
\pgfusepath{clip}%
\pgfsetbuttcap%
\pgfsetroundjoin%
\definecolor{currentfill}{rgb}{0.000000,0.000000,0.000000}%
\pgfsetfillcolor{currentfill}%
\pgfsetlinewidth{1.003750pt}%
\definecolor{currentstroke}{rgb}{0.000000,0.000000,0.000000}%
\pgfsetstrokecolor{currentstroke}%
\pgfsetdash{}{0pt}%
\pgfpathmoveto{\pgfqpoint{0.721249in}{0.356602in}}%
\pgfpathcurveto{\pgfqpoint{0.726774in}{0.356602in}}{\pgfqpoint{0.732073in}{0.358798in}}{\pgfqpoint{0.735980in}{0.362704in}}%
\pgfpathcurveto{\pgfqpoint{0.739887in}{0.366611in}}{\pgfqpoint{0.742082in}{0.371911in}}{\pgfqpoint{0.742082in}{0.377436in}}%
\pgfpathcurveto{\pgfqpoint{0.742082in}{0.382961in}}{\pgfqpoint{0.739887in}{0.388260in}}{\pgfqpoint{0.735980in}{0.392167in}}%
\pgfpathcurveto{\pgfqpoint{0.732073in}{0.396074in}}{\pgfqpoint{0.726774in}{0.398269in}}{\pgfqpoint{0.721249in}{0.398269in}}%
\pgfpathcurveto{\pgfqpoint{0.715724in}{0.398269in}}{\pgfqpoint{0.710424in}{0.396074in}}{\pgfqpoint{0.706518in}{0.392167in}}%
\pgfpathcurveto{\pgfqpoint{0.702611in}{0.388260in}}{\pgfqpoint{0.700416in}{0.382961in}}{\pgfqpoint{0.700416in}{0.377436in}}%
\pgfpathcurveto{\pgfqpoint{0.700416in}{0.371911in}}{\pgfqpoint{0.702611in}{0.366611in}}{\pgfqpoint{0.706518in}{0.362704in}}%
\pgfpathcurveto{\pgfqpoint{0.710424in}{0.358798in}}{\pgfqpoint{0.715724in}{0.356602in}}{\pgfqpoint{0.721249in}{0.356602in}}%
\pgfpathclose%
\pgfusepath{stroke,fill}%
\end{pgfscope}%
\begin{pgfscope}%
\pgfpathrectangle{\pgfqpoint{0.562500in}{0.275000in}}{\pgfqpoint{3.487500in}{1.925000in}}%
\pgfusepath{clip}%
\pgfsetbuttcap%
\pgfsetroundjoin%
\definecolor{currentfill}{rgb}{0.000000,0.000000,0.000000}%
\pgfsetfillcolor{currentfill}%
\pgfsetlinewidth{1.003750pt}%
\definecolor{currentstroke}{rgb}{0.000000,0.000000,0.000000}%
\pgfsetstrokecolor{currentstroke}%
\pgfsetdash{}{0pt}%
\pgfpathmoveto{\pgfqpoint{0.721249in}{1.216635in}}%
\pgfpathcurveto{\pgfqpoint{0.726774in}{1.216635in}}{\pgfqpoint{0.732073in}{1.218830in}}{\pgfqpoint{0.735980in}{1.222736in}}%
\pgfpathcurveto{\pgfqpoint{0.739887in}{1.226643in}}{\pgfqpoint{0.742082in}{1.231943in}}{\pgfqpoint{0.742082in}{1.237468in}}%
\pgfpathcurveto{\pgfqpoint{0.742082in}{1.242993in}}{\pgfqpoint{0.739887in}{1.248292in}}{\pgfqpoint{0.735980in}{1.252199in}}%
\pgfpathcurveto{\pgfqpoint{0.732073in}{1.256106in}}{\pgfqpoint{0.726774in}{1.258301in}}{\pgfqpoint{0.721249in}{1.258301in}}%
\pgfpathcurveto{\pgfqpoint{0.715724in}{1.258301in}}{\pgfqpoint{0.710424in}{1.256106in}}{\pgfqpoint{0.706518in}{1.252199in}}%
\pgfpathcurveto{\pgfqpoint{0.702611in}{1.248292in}}{\pgfqpoint{0.700416in}{1.242993in}}{\pgfqpoint{0.700416in}{1.237468in}}%
\pgfpathcurveto{\pgfqpoint{0.700416in}{1.231943in}}{\pgfqpoint{0.702611in}{1.226643in}}{\pgfqpoint{0.706518in}{1.222736in}}%
\pgfpathcurveto{\pgfqpoint{0.710424in}{1.218830in}}{\pgfqpoint{0.715724in}{1.216635in}}{\pgfqpoint{0.721249in}{1.216635in}}%
\pgfpathclose%
\pgfusepath{stroke,fill}%
\end{pgfscope}%
\begin{pgfscope}%
\pgfpathrectangle{\pgfqpoint{0.562500in}{0.275000in}}{\pgfqpoint{3.487500in}{1.925000in}}%
\pgfusepath{clip}%
\pgfsetbuttcap%
\pgfsetroundjoin%
\definecolor{currentfill}{rgb}{0.000000,0.000000,0.000000}%
\pgfsetfillcolor{currentfill}%
\pgfsetlinewidth{1.003750pt}%
\definecolor{currentstroke}{rgb}{0.000000,0.000000,0.000000}%
\pgfsetstrokecolor{currentstroke}%
\pgfsetdash{}{0pt}%
\pgfpathmoveto{\pgfqpoint{0.721249in}{1.216635in}}%
\pgfpathcurveto{\pgfqpoint{0.726774in}{1.216635in}}{\pgfqpoint{0.732073in}{1.218830in}}{\pgfqpoint{0.735980in}{1.222736in}}%
\pgfpathcurveto{\pgfqpoint{0.739887in}{1.226643in}}{\pgfqpoint{0.742082in}{1.231943in}}{\pgfqpoint{0.742082in}{1.237468in}}%
\pgfpathcurveto{\pgfqpoint{0.742082in}{1.242993in}}{\pgfqpoint{0.739887in}{1.248292in}}{\pgfqpoint{0.735980in}{1.252199in}}%
\pgfpathcurveto{\pgfqpoint{0.732073in}{1.256106in}}{\pgfqpoint{0.726774in}{1.258301in}}{\pgfqpoint{0.721249in}{1.258301in}}%
\pgfpathcurveto{\pgfqpoint{0.715724in}{1.258301in}}{\pgfqpoint{0.710424in}{1.256106in}}{\pgfqpoint{0.706518in}{1.252199in}}%
\pgfpathcurveto{\pgfqpoint{0.702611in}{1.248292in}}{\pgfqpoint{0.700416in}{1.242993in}}{\pgfqpoint{0.700416in}{1.237468in}}%
\pgfpathcurveto{\pgfqpoint{0.700416in}{1.231943in}}{\pgfqpoint{0.702611in}{1.226643in}}{\pgfqpoint{0.706518in}{1.222736in}}%
\pgfpathcurveto{\pgfqpoint{0.710424in}{1.218830in}}{\pgfqpoint{0.715724in}{1.216635in}}{\pgfqpoint{0.721249in}{1.216635in}}%
\pgfpathclose%
\pgfusepath{stroke,fill}%
\end{pgfscope}%
\begin{pgfscope}%
\pgfpathrectangle{\pgfqpoint{0.562500in}{0.275000in}}{\pgfqpoint{3.487500in}{1.925000in}}%
\pgfusepath{clip}%
\pgfsetbuttcap%
\pgfsetroundjoin%
\definecolor{currentfill}{rgb}{0.000000,0.000000,0.000000}%
\pgfsetfillcolor{currentfill}%
\pgfsetlinewidth{1.003750pt}%
\definecolor{currentstroke}{rgb}{0.000000,0.000000,0.000000}%
\pgfsetstrokecolor{currentstroke}%
\pgfsetdash{}{0pt}%
\pgfpathmoveto{\pgfqpoint{0.721249in}{1.216635in}}%
\pgfpathcurveto{\pgfqpoint{0.726774in}{1.216635in}}{\pgfqpoint{0.732073in}{1.218830in}}{\pgfqpoint{0.735980in}{1.222736in}}%
\pgfpathcurveto{\pgfqpoint{0.739887in}{1.226643in}}{\pgfqpoint{0.742082in}{1.231943in}}{\pgfqpoint{0.742082in}{1.237468in}}%
\pgfpathcurveto{\pgfqpoint{0.742082in}{1.242993in}}{\pgfqpoint{0.739887in}{1.248292in}}{\pgfqpoint{0.735980in}{1.252199in}}%
\pgfpathcurveto{\pgfqpoint{0.732073in}{1.256106in}}{\pgfqpoint{0.726774in}{1.258301in}}{\pgfqpoint{0.721249in}{1.258301in}}%
\pgfpathcurveto{\pgfqpoint{0.715724in}{1.258301in}}{\pgfqpoint{0.710424in}{1.256106in}}{\pgfqpoint{0.706518in}{1.252199in}}%
\pgfpathcurveto{\pgfqpoint{0.702611in}{1.248292in}}{\pgfqpoint{0.700416in}{1.242993in}}{\pgfqpoint{0.700416in}{1.237468in}}%
\pgfpathcurveto{\pgfqpoint{0.700416in}{1.231943in}}{\pgfqpoint{0.702611in}{1.226643in}}{\pgfqpoint{0.706518in}{1.222736in}}%
\pgfpathcurveto{\pgfqpoint{0.710424in}{1.218830in}}{\pgfqpoint{0.715724in}{1.216635in}}{\pgfqpoint{0.721249in}{1.216635in}}%
\pgfpathclose%
\pgfusepath{stroke,fill}%
\end{pgfscope}%
\begin{pgfscope}%
\pgfpathrectangle{\pgfqpoint{0.562500in}{0.275000in}}{\pgfqpoint{3.487500in}{1.925000in}}%
\pgfusepath{clip}%
\pgfsetbuttcap%
\pgfsetroundjoin%
\definecolor{currentfill}{rgb}{0.000000,0.000000,0.000000}%
\pgfsetfillcolor{currentfill}%
\pgfsetlinewidth{1.003750pt}%
\definecolor{currentstroke}{rgb}{0.000000,0.000000,0.000000}%
\pgfsetstrokecolor{currentstroke}%
\pgfsetdash{}{0pt}%
\pgfpathmoveto{\pgfqpoint{0.721249in}{1.216635in}}%
\pgfpathcurveto{\pgfqpoint{0.726774in}{1.216635in}}{\pgfqpoint{0.732073in}{1.218830in}}{\pgfqpoint{0.735980in}{1.222736in}}%
\pgfpathcurveto{\pgfqpoint{0.739887in}{1.226643in}}{\pgfqpoint{0.742082in}{1.231943in}}{\pgfqpoint{0.742082in}{1.237468in}}%
\pgfpathcurveto{\pgfqpoint{0.742082in}{1.242993in}}{\pgfqpoint{0.739887in}{1.248292in}}{\pgfqpoint{0.735980in}{1.252199in}}%
\pgfpathcurveto{\pgfqpoint{0.732073in}{1.256106in}}{\pgfqpoint{0.726774in}{1.258301in}}{\pgfqpoint{0.721249in}{1.258301in}}%
\pgfpathcurveto{\pgfqpoint{0.715724in}{1.258301in}}{\pgfqpoint{0.710424in}{1.256106in}}{\pgfqpoint{0.706518in}{1.252199in}}%
\pgfpathcurveto{\pgfqpoint{0.702611in}{1.248292in}}{\pgfqpoint{0.700416in}{1.242993in}}{\pgfqpoint{0.700416in}{1.237468in}}%
\pgfpathcurveto{\pgfqpoint{0.700416in}{1.231943in}}{\pgfqpoint{0.702611in}{1.226643in}}{\pgfqpoint{0.706518in}{1.222736in}}%
\pgfpathcurveto{\pgfqpoint{0.710424in}{1.218830in}}{\pgfqpoint{0.715724in}{1.216635in}}{\pgfqpoint{0.721249in}{1.216635in}}%
\pgfpathclose%
\pgfusepath{stroke,fill}%
\end{pgfscope}%
\begin{pgfscope}%
\pgfpathrectangle{\pgfqpoint{0.562500in}{0.275000in}}{\pgfqpoint{3.487500in}{1.925000in}}%
\pgfusepath{clip}%
\pgfsetbuttcap%
\pgfsetroundjoin%
\definecolor{currentfill}{rgb}{0.000000,0.000000,0.000000}%
\pgfsetfillcolor{currentfill}%
\pgfsetlinewidth{1.003750pt}%
\definecolor{currentstroke}{rgb}{0.000000,0.000000,0.000000}%
\pgfsetstrokecolor{currentstroke}%
\pgfsetdash{}{0pt}%
\pgfpathmoveto{\pgfqpoint{0.721249in}{0.356602in}}%
\pgfpathcurveto{\pgfqpoint{0.726774in}{0.356602in}}{\pgfqpoint{0.732073in}{0.358798in}}{\pgfqpoint{0.735980in}{0.362704in}}%
\pgfpathcurveto{\pgfqpoint{0.739887in}{0.366611in}}{\pgfqpoint{0.742082in}{0.371911in}}{\pgfqpoint{0.742082in}{0.377436in}}%
\pgfpathcurveto{\pgfqpoint{0.742082in}{0.382961in}}{\pgfqpoint{0.739887in}{0.388260in}}{\pgfqpoint{0.735980in}{0.392167in}}%
\pgfpathcurveto{\pgfqpoint{0.732073in}{0.396074in}}{\pgfqpoint{0.726774in}{0.398269in}}{\pgfqpoint{0.721249in}{0.398269in}}%
\pgfpathcurveto{\pgfqpoint{0.715724in}{0.398269in}}{\pgfqpoint{0.710424in}{0.396074in}}{\pgfqpoint{0.706518in}{0.392167in}}%
\pgfpathcurveto{\pgfqpoint{0.702611in}{0.388260in}}{\pgfqpoint{0.700416in}{0.382961in}}{\pgfqpoint{0.700416in}{0.377436in}}%
\pgfpathcurveto{\pgfqpoint{0.700416in}{0.371911in}}{\pgfqpoint{0.702611in}{0.366611in}}{\pgfqpoint{0.706518in}{0.362704in}}%
\pgfpathcurveto{\pgfqpoint{0.710424in}{0.358798in}}{\pgfqpoint{0.715724in}{0.356602in}}{\pgfqpoint{0.721249in}{0.356602in}}%
\pgfpathclose%
\pgfusepath{stroke,fill}%
\end{pgfscope}%
\begin{pgfscope}%
\pgfpathrectangle{\pgfqpoint{0.562500in}{0.275000in}}{\pgfqpoint{3.487500in}{1.925000in}}%
\pgfusepath{clip}%
\pgfsetbuttcap%
\pgfsetroundjoin%
\definecolor{currentfill}{rgb}{0.000000,0.000000,0.000000}%
\pgfsetfillcolor{currentfill}%
\pgfsetlinewidth{1.003750pt}%
\definecolor{currentstroke}{rgb}{0.000000,0.000000,0.000000}%
\pgfsetstrokecolor{currentstroke}%
\pgfsetdash{}{0pt}%
\pgfpathmoveto{\pgfqpoint{0.721249in}{0.356602in}}%
\pgfpathcurveto{\pgfqpoint{0.726774in}{0.356602in}}{\pgfqpoint{0.732073in}{0.358798in}}{\pgfqpoint{0.735980in}{0.362704in}}%
\pgfpathcurveto{\pgfqpoint{0.739887in}{0.366611in}}{\pgfqpoint{0.742082in}{0.371911in}}{\pgfqpoint{0.742082in}{0.377436in}}%
\pgfpathcurveto{\pgfqpoint{0.742082in}{0.382961in}}{\pgfqpoint{0.739887in}{0.388260in}}{\pgfqpoint{0.735980in}{0.392167in}}%
\pgfpathcurveto{\pgfqpoint{0.732073in}{0.396074in}}{\pgfqpoint{0.726774in}{0.398269in}}{\pgfqpoint{0.721249in}{0.398269in}}%
\pgfpathcurveto{\pgfqpoint{0.715724in}{0.398269in}}{\pgfqpoint{0.710424in}{0.396074in}}{\pgfqpoint{0.706518in}{0.392167in}}%
\pgfpathcurveto{\pgfqpoint{0.702611in}{0.388260in}}{\pgfqpoint{0.700416in}{0.382961in}}{\pgfqpoint{0.700416in}{0.377436in}}%
\pgfpathcurveto{\pgfqpoint{0.700416in}{0.371911in}}{\pgfqpoint{0.702611in}{0.366611in}}{\pgfqpoint{0.706518in}{0.362704in}}%
\pgfpathcurveto{\pgfqpoint{0.710424in}{0.358798in}}{\pgfqpoint{0.715724in}{0.356602in}}{\pgfqpoint{0.721249in}{0.356602in}}%
\pgfpathclose%
\pgfusepath{stroke,fill}%
\end{pgfscope}%
\begin{pgfscope}%
\pgfpathrectangle{\pgfqpoint{0.562500in}{0.275000in}}{\pgfqpoint{3.487500in}{1.925000in}}%
\pgfusepath{clip}%
\pgfsetbuttcap%
\pgfsetroundjoin%
\definecolor{currentfill}{rgb}{0.000000,0.000000,0.000000}%
\pgfsetfillcolor{currentfill}%
\pgfsetlinewidth{1.003750pt}%
\definecolor{currentstroke}{rgb}{0.000000,0.000000,0.000000}%
\pgfsetstrokecolor{currentstroke}%
\pgfsetdash{}{0pt}%
\pgfpathmoveto{\pgfqpoint{0.721249in}{0.356602in}}%
\pgfpathcurveto{\pgfqpoint{0.726774in}{0.356602in}}{\pgfqpoint{0.732073in}{0.358798in}}{\pgfqpoint{0.735980in}{0.362704in}}%
\pgfpathcurveto{\pgfqpoint{0.739887in}{0.366611in}}{\pgfqpoint{0.742082in}{0.371911in}}{\pgfqpoint{0.742082in}{0.377436in}}%
\pgfpathcurveto{\pgfqpoint{0.742082in}{0.382961in}}{\pgfqpoint{0.739887in}{0.388260in}}{\pgfqpoint{0.735980in}{0.392167in}}%
\pgfpathcurveto{\pgfqpoint{0.732073in}{0.396074in}}{\pgfqpoint{0.726774in}{0.398269in}}{\pgfqpoint{0.721249in}{0.398269in}}%
\pgfpathcurveto{\pgfqpoint{0.715724in}{0.398269in}}{\pgfqpoint{0.710424in}{0.396074in}}{\pgfqpoint{0.706518in}{0.392167in}}%
\pgfpathcurveto{\pgfqpoint{0.702611in}{0.388260in}}{\pgfqpoint{0.700416in}{0.382961in}}{\pgfqpoint{0.700416in}{0.377436in}}%
\pgfpathcurveto{\pgfqpoint{0.700416in}{0.371911in}}{\pgfqpoint{0.702611in}{0.366611in}}{\pgfqpoint{0.706518in}{0.362704in}}%
\pgfpathcurveto{\pgfqpoint{0.710424in}{0.358798in}}{\pgfqpoint{0.715724in}{0.356602in}}{\pgfqpoint{0.721249in}{0.356602in}}%
\pgfpathclose%
\pgfusepath{stroke,fill}%
\end{pgfscope}%
\begin{pgfscope}%
\pgfpathrectangle{\pgfqpoint{0.562500in}{0.275000in}}{\pgfqpoint{3.487500in}{1.925000in}}%
\pgfusepath{clip}%
\pgfsetbuttcap%
\pgfsetroundjoin%
\definecolor{currentfill}{rgb}{0.000000,0.000000,0.000000}%
\pgfsetfillcolor{currentfill}%
\pgfsetlinewidth{1.003750pt}%
\definecolor{currentstroke}{rgb}{0.000000,0.000000,0.000000}%
\pgfsetstrokecolor{currentstroke}%
\pgfsetdash{}{0pt}%
\pgfpathmoveto{\pgfqpoint{0.721249in}{0.356602in}}%
\pgfpathcurveto{\pgfqpoint{0.726774in}{0.356602in}}{\pgfqpoint{0.732073in}{0.358798in}}{\pgfqpoint{0.735980in}{0.362704in}}%
\pgfpathcurveto{\pgfqpoint{0.739887in}{0.366611in}}{\pgfqpoint{0.742082in}{0.371911in}}{\pgfqpoint{0.742082in}{0.377436in}}%
\pgfpathcurveto{\pgfqpoint{0.742082in}{0.382961in}}{\pgfqpoint{0.739887in}{0.388260in}}{\pgfqpoint{0.735980in}{0.392167in}}%
\pgfpathcurveto{\pgfqpoint{0.732073in}{0.396074in}}{\pgfqpoint{0.726774in}{0.398269in}}{\pgfqpoint{0.721249in}{0.398269in}}%
\pgfpathcurveto{\pgfqpoint{0.715724in}{0.398269in}}{\pgfqpoint{0.710424in}{0.396074in}}{\pgfqpoint{0.706518in}{0.392167in}}%
\pgfpathcurveto{\pgfqpoint{0.702611in}{0.388260in}}{\pgfqpoint{0.700416in}{0.382961in}}{\pgfqpoint{0.700416in}{0.377436in}}%
\pgfpathcurveto{\pgfqpoint{0.700416in}{0.371911in}}{\pgfqpoint{0.702611in}{0.366611in}}{\pgfqpoint{0.706518in}{0.362704in}}%
\pgfpathcurveto{\pgfqpoint{0.710424in}{0.358798in}}{\pgfqpoint{0.715724in}{0.356602in}}{\pgfqpoint{0.721249in}{0.356602in}}%
\pgfpathclose%
\pgfusepath{stroke,fill}%
\end{pgfscope}%
\begin{pgfscope}%
\pgfpathrectangle{\pgfqpoint{0.562500in}{0.275000in}}{\pgfqpoint{3.487500in}{1.925000in}}%
\pgfusepath{clip}%
\pgfsetbuttcap%
\pgfsetroundjoin%
\definecolor{currentfill}{rgb}{0.000000,0.000000,0.000000}%
\pgfsetfillcolor{currentfill}%
\pgfsetlinewidth{1.003750pt}%
\definecolor{currentstroke}{rgb}{0.000000,0.000000,0.000000}%
\pgfsetstrokecolor{currentstroke}%
\pgfsetdash{}{0pt}%
\pgfpathmoveto{\pgfqpoint{0.721249in}{0.356602in}}%
\pgfpathcurveto{\pgfqpoint{0.726774in}{0.356602in}}{\pgfqpoint{0.732073in}{0.358798in}}{\pgfqpoint{0.735980in}{0.362704in}}%
\pgfpathcurveto{\pgfqpoint{0.739887in}{0.366611in}}{\pgfqpoint{0.742082in}{0.371911in}}{\pgfqpoint{0.742082in}{0.377436in}}%
\pgfpathcurveto{\pgfqpoint{0.742082in}{0.382961in}}{\pgfqpoint{0.739887in}{0.388260in}}{\pgfqpoint{0.735980in}{0.392167in}}%
\pgfpathcurveto{\pgfqpoint{0.732073in}{0.396074in}}{\pgfqpoint{0.726774in}{0.398269in}}{\pgfqpoint{0.721249in}{0.398269in}}%
\pgfpathcurveto{\pgfqpoint{0.715724in}{0.398269in}}{\pgfqpoint{0.710424in}{0.396074in}}{\pgfqpoint{0.706518in}{0.392167in}}%
\pgfpathcurveto{\pgfqpoint{0.702611in}{0.388260in}}{\pgfqpoint{0.700416in}{0.382961in}}{\pgfqpoint{0.700416in}{0.377436in}}%
\pgfpathcurveto{\pgfqpoint{0.700416in}{0.371911in}}{\pgfqpoint{0.702611in}{0.366611in}}{\pgfqpoint{0.706518in}{0.362704in}}%
\pgfpathcurveto{\pgfqpoint{0.710424in}{0.358798in}}{\pgfqpoint{0.715724in}{0.356602in}}{\pgfqpoint{0.721249in}{0.356602in}}%
\pgfpathclose%
\pgfusepath{stroke,fill}%
\end{pgfscope}%
\begin{pgfscope}%
\pgfpathrectangle{\pgfqpoint{0.562500in}{0.275000in}}{\pgfqpoint{3.487500in}{1.925000in}}%
\pgfusepath{clip}%
\pgfsetbuttcap%
\pgfsetroundjoin%
\definecolor{currentfill}{rgb}{0.000000,0.000000,0.000000}%
\pgfsetfillcolor{currentfill}%
\pgfsetlinewidth{1.003750pt}%
\definecolor{currentstroke}{rgb}{0.000000,0.000000,0.000000}%
\pgfsetstrokecolor{currentstroke}%
\pgfsetdash{}{0pt}%
\pgfpathmoveto{\pgfqpoint{0.721249in}{0.356602in}}%
\pgfpathcurveto{\pgfqpoint{0.726774in}{0.356602in}}{\pgfqpoint{0.732073in}{0.358798in}}{\pgfqpoint{0.735980in}{0.362704in}}%
\pgfpathcurveto{\pgfqpoint{0.739887in}{0.366611in}}{\pgfqpoint{0.742082in}{0.371911in}}{\pgfqpoint{0.742082in}{0.377436in}}%
\pgfpathcurveto{\pgfqpoint{0.742082in}{0.382961in}}{\pgfqpoint{0.739887in}{0.388260in}}{\pgfqpoint{0.735980in}{0.392167in}}%
\pgfpathcurveto{\pgfqpoint{0.732073in}{0.396074in}}{\pgfqpoint{0.726774in}{0.398269in}}{\pgfqpoint{0.721249in}{0.398269in}}%
\pgfpathcurveto{\pgfqpoint{0.715724in}{0.398269in}}{\pgfqpoint{0.710424in}{0.396074in}}{\pgfqpoint{0.706518in}{0.392167in}}%
\pgfpathcurveto{\pgfqpoint{0.702611in}{0.388260in}}{\pgfqpoint{0.700416in}{0.382961in}}{\pgfqpoint{0.700416in}{0.377436in}}%
\pgfpathcurveto{\pgfqpoint{0.700416in}{0.371911in}}{\pgfqpoint{0.702611in}{0.366611in}}{\pgfqpoint{0.706518in}{0.362704in}}%
\pgfpathcurveto{\pgfqpoint{0.710424in}{0.358798in}}{\pgfqpoint{0.715724in}{0.356602in}}{\pgfqpoint{0.721249in}{0.356602in}}%
\pgfpathclose%
\pgfusepath{stroke,fill}%
\end{pgfscope}%
\begin{pgfscope}%
\pgfpathrectangle{\pgfqpoint{0.562500in}{0.275000in}}{\pgfqpoint{3.487500in}{1.925000in}}%
\pgfusepath{clip}%
\pgfsetbuttcap%
\pgfsetroundjoin%
\definecolor{currentfill}{rgb}{0.000000,0.000000,0.000000}%
\pgfsetfillcolor{currentfill}%
\pgfsetlinewidth{1.003750pt}%
\definecolor{currentstroke}{rgb}{0.000000,0.000000,0.000000}%
\pgfsetstrokecolor{currentstroke}%
\pgfsetdash{}{0pt}%
\pgfpathmoveto{\pgfqpoint{0.721249in}{1.216635in}}%
\pgfpathcurveto{\pgfqpoint{0.726774in}{1.216635in}}{\pgfqpoint{0.732073in}{1.218830in}}{\pgfqpoint{0.735980in}{1.222736in}}%
\pgfpathcurveto{\pgfqpoint{0.739887in}{1.226643in}}{\pgfqpoint{0.742082in}{1.231943in}}{\pgfqpoint{0.742082in}{1.237468in}}%
\pgfpathcurveto{\pgfqpoint{0.742082in}{1.242993in}}{\pgfqpoint{0.739887in}{1.248292in}}{\pgfqpoint{0.735980in}{1.252199in}}%
\pgfpathcurveto{\pgfqpoint{0.732073in}{1.256106in}}{\pgfqpoint{0.726774in}{1.258301in}}{\pgfqpoint{0.721249in}{1.258301in}}%
\pgfpathcurveto{\pgfqpoint{0.715724in}{1.258301in}}{\pgfqpoint{0.710424in}{1.256106in}}{\pgfqpoint{0.706518in}{1.252199in}}%
\pgfpathcurveto{\pgfqpoint{0.702611in}{1.248292in}}{\pgfqpoint{0.700416in}{1.242993in}}{\pgfqpoint{0.700416in}{1.237468in}}%
\pgfpathcurveto{\pgfqpoint{0.700416in}{1.231943in}}{\pgfqpoint{0.702611in}{1.226643in}}{\pgfqpoint{0.706518in}{1.222736in}}%
\pgfpathcurveto{\pgfqpoint{0.710424in}{1.218830in}}{\pgfqpoint{0.715724in}{1.216635in}}{\pgfqpoint{0.721249in}{1.216635in}}%
\pgfpathclose%
\pgfusepath{stroke,fill}%
\end{pgfscope}%
\begin{pgfscope}%
\pgfpathrectangle{\pgfqpoint{0.562500in}{0.275000in}}{\pgfqpoint{3.487500in}{1.925000in}}%
\pgfusepath{clip}%
\pgfsetbuttcap%
\pgfsetroundjoin%
\definecolor{currentfill}{rgb}{0.000000,0.000000,0.000000}%
\pgfsetfillcolor{currentfill}%
\pgfsetlinewidth{1.003750pt}%
\definecolor{currentstroke}{rgb}{0.000000,0.000000,0.000000}%
\pgfsetstrokecolor{currentstroke}%
\pgfsetdash{}{0pt}%
\pgfpathmoveto{\pgfqpoint{0.721249in}{1.216635in}}%
\pgfpathcurveto{\pgfqpoint{0.726774in}{1.216635in}}{\pgfqpoint{0.732073in}{1.218830in}}{\pgfqpoint{0.735980in}{1.222736in}}%
\pgfpathcurveto{\pgfqpoint{0.739887in}{1.226643in}}{\pgfqpoint{0.742082in}{1.231943in}}{\pgfqpoint{0.742082in}{1.237468in}}%
\pgfpathcurveto{\pgfqpoint{0.742082in}{1.242993in}}{\pgfqpoint{0.739887in}{1.248292in}}{\pgfqpoint{0.735980in}{1.252199in}}%
\pgfpathcurveto{\pgfqpoint{0.732073in}{1.256106in}}{\pgfqpoint{0.726774in}{1.258301in}}{\pgfqpoint{0.721249in}{1.258301in}}%
\pgfpathcurveto{\pgfqpoint{0.715724in}{1.258301in}}{\pgfqpoint{0.710424in}{1.256106in}}{\pgfqpoint{0.706518in}{1.252199in}}%
\pgfpathcurveto{\pgfqpoint{0.702611in}{1.248292in}}{\pgfqpoint{0.700416in}{1.242993in}}{\pgfqpoint{0.700416in}{1.237468in}}%
\pgfpathcurveto{\pgfqpoint{0.700416in}{1.231943in}}{\pgfqpoint{0.702611in}{1.226643in}}{\pgfqpoint{0.706518in}{1.222736in}}%
\pgfpathcurveto{\pgfqpoint{0.710424in}{1.218830in}}{\pgfqpoint{0.715724in}{1.216635in}}{\pgfqpoint{0.721249in}{1.216635in}}%
\pgfpathclose%
\pgfusepath{stroke,fill}%
\end{pgfscope}%
\begin{pgfscope}%
\pgfpathrectangle{\pgfqpoint{0.562500in}{0.275000in}}{\pgfqpoint{3.487500in}{1.925000in}}%
\pgfusepath{clip}%
\pgfsetbuttcap%
\pgfsetroundjoin%
\definecolor{currentfill}{rgb}{0.000000,0.000000,0.000000}%
\pgfsetfillcolor{currentfill}%
\pgfsetlinewidth{1.003750pt}%
\definecolor{currentstroke}{rgb}{0.000000,0.000000,0.000000}%
\pgfsetstrokecolor{currentstroke}%
\pgfsetdash{}{0pt}%
\pgfpathmoveto{\pgfqpoint{0.721249in}{0.356602in}}%
\pgfpathcurveto{\pgfqpoint{0.726774in}{0.356602in}}{\pgfqpoint{0.732073in}{0.358798in}}{\pgfqpoint{0.735980in}{0.362704in}}%
\pgfpathcurveto{\pgfqpoint{0.739887in}{0.366611in}}{\pgfqpoint{0.742082in}{0.371911in}}{\pgfqpoint{0.742082in}{0.377436in}}%
\pgfpathcurveto{\pgfqpoint{0.742082in}{0.382961in}}{\pgfqpoint{0.739887in}{0.388260in}}{\pgfqpoint{0.735980in}{0.392167in}}%
\pgfpathcurveto{\pgfqpoint{0.732073in}{0.396074in}}{\pgfqpoint{0.726774in}{0.398269in}}{\pgfqpoint{0.721249in}{0.398269in}}%
\pgfpathcurveto{\pgfqpoint{0.715724in}{0.398269in}}{\pgfqpoint{0.710424in}{0.396074in}}{\pgfqpoint{0.706518in}{0.392167in}}%
\pgfpathcurveto{\pgfqpoint{0.702611in}{0.388260in}}{\pgfqpoint{0.700416in}{0.382961in}}{\pgfqpoint{0.700416in}{0.377436in}}%
\pgfpathcurveto{\pgfqpoint{0.700416in}{0.371911in}}{\pgfqpoint{0.702611in}{0.366611in}}{\pgfqpoint{0.706518in}{0.362704in}}%
\pgfpathcurveto{\pgfqpoint{0.710424in}{0.358798in}}{\pgfqpoint{0.715724in}{0.356602in}}{\pgfqpoint{0.721249in}{0.356602in}}%
\pgfpathclose%
\pgfusepath{stroke,fill}%
\end{pgfscope}%
\begin{pgfscope}%
\pgfpathrectangle{\pgfqpoint{0.562500in}{0.275000in}}{\pgfqpoint{3.487500in}{1.925000in}}%
\pgfusepath{clip}%
\pgfsetbuttcap%
\pgfsetroundjoin%
\definecolor{currentfill}{rgb}{0.000000,0.000000,0.000000}%
\pgfsetfillcolor{currentfill}%
\pgfsetlinewidth{1.003750pt}%
\definecolor{currentstroke}{rgb}{0.000000,0.000000,0.000000}%
\pgfsetstrokecolor{currentstroke}%
\pgfsetdash{}{0pt}%
\pgfpathmoveto{\pgfqpoint{0.721249in}{0.356602in}}%
\pgfpathcurveto{\pgfqpoint{0.726774in}{0.356602in}}{\pgfqpoint{0.732073in}{0.358798in}}{\pgfqpoint{0.735980in}{0.362704in}}%
\pgfpathcurveto{\pgfqpoint{0.739887in}{0.366611in}}{\pgfqpoint{0.742082in}{0.371911in}}{\pgfqpoint{0.742082in}{0.377436in}}%
\pgfpathcurveto{\pgfqpoint{0.742082in}{0.382961in}}{\pgfqpoint{0.739887in}{0.388260in}}{\pgfqpoint{0.735980in}{0.392167in}}%
\pgfpathcurveto{\pgfqpoint{0.732073in}{0.396074in}}{\pgfqpoint{0.726774in}{0.398269in}}{\pgfqpoint{0.721249in}{0.398269in}}%
\pgfpathcurveto{\pgfqpoint{0.715724in}{0.398269in}}{\pgfqpoint{0.710424in}{0.396074in}}{\pgfqpoint{0.706518in}{0.392167in}}%
\pgfpathcurveto{\pgfqpoint{0.702611in}{0.388260in}}{\pgfqpoint{0.700416in}{0.382961in}}{\pgfqpoint{0.700416in}{0.377436in}}%
\pgfpathcurveto{\pgfqpoint{0.700416in}{0.371911in}}{\pgfqpoint{0.702611in}{0.366611in}}{\pgfqpoint{0.706518in}{0.362704in}}%
\pgfpathcurveto{\pgfqpoint{0.710424in}{0.358798in}}{\pgfqpoint{0.715724in}{0.356602in}}{\pgfqpoint{0.721249in}{0.356602in}}%
\pgfpathclose%
\pgfusepath{stroke,fill}%
\end{pgfscope}%
\begin{pgfscope}%
\pgfpathrectangle{\pgfqpoint{0.562500in}{0.275000in}}{\pgfqpoint{3.487500in}{1.925000in}}%
\pgfusepath{clip}%
\pgfsetbuttcap%
\pgfsetroundjoin%
\definecolor{currentfill}{rgb}{0.000000,0.000000,0.000000}%
\pgfsetfillcolor{currentfill}%
\pgfsetlinewidth{1.003750pt}%
\definecolor{currentstroke}{rgb}{0.000000,0.000000,0.000000}%
\pgfsetstrokecolor{currentstroke}%
\pgfsetdash{}{0pt}%
\pgfpathmoveto{\pgfqpoint{0.721249in}{0.356602in}}%
\pgfpathcurveto{\pgfqpoint{0.726774in}{0.356602in}}{\pgfqpoint{0.732073in}{0.358798in}}{\pgfqpoint{0.735980in}{0.362704in}}%
\pgfpathcurveto{\pgfqpoint{0.739887in}{0.366611in}}{\pgfqpoint{0.742082in}{0.371911in}}{\pgfqpoint{0.742082in}{0.377436in}}%
\pgfpathcurveto{\pgfqpoint{0.742082in}{0.382961in}}{\pgfqpoint{0.739887in}{0.388260in}}{\pgfqpoint{0.735980in}{0.392167in}}%
\pgfpathcurveto{\pgfqpoint{0.732073in}{0.396074in}}{\pgfqpoint{0.726774in}{0.398269in}}{\pgfqpoint{0.721249in}{0.398269in}}%
\pgfpathcurveto{\pgfqpoint{0.715724in}{0.398269in}}{\pgfqpoint{0.710424in}{0.396074in}}{\pgfqpoint{0.706518in}{0.392167in}}%
\pgfpathcurveto{\pgfqpoint{0.702611in}{0.388260in}}{\pgfqpoint{0.700416in}{0.382961in}}{\pgfqpoint{0.700416in}{0.377436in}}%
\pgfpathcurveto{\pgfqpoint{0.700416in}{0.371911in}}{\pgfqpoint{0.702611in}{0.366611in}}{\pgfqpoint{0.706518in}{0.362704in}}%
\pgfpathcurveto{\pgfqpoint{0.710424in}{0.358798in}}{\pgfqpoint{0.715724in}{0.356602in}}{\pgfqpoint{0.721249in}{0.356602in}}%
\pgfpathclose%
\pgfusepath{stroke,fill}%
\end{pgfscope}%
\begin{pgfscope}%
\pgfpathrectangle{\pgfqpoint{0.562500in}{0.275000in}}{\pgfqpoint{3.487500in}{1.925000in}}%
\pgfusepath{clip}%
\pgfsetbuttcap%
\pgfsetroundjoin%
\definecolor{currentfill}{rgb}{0.000000,0.000000,0.000000}%
\pgfsetfillcolor{currentfill}%
\pgfsetlinewidth{1.003750pt}%
\definecolor{currentstroke}{rgb}{0.000000,0.000000,0.000000}%
\pgfsetstrokecolor{currentstroke}%
\pgfsetdash{}{0pt}%
\pgfpathmoveto{\pgfqpoint{0.721249in}{0.356602in}}%
\pgfpathcurveto{\pgfqpoint{0.726774in}{0.356602in}}{\pgfqpoint{0.732073in}{0.358798in}}{\pgfqpoint{0.735980in}{0.362704in}}%
\pgfpathcurveto{\pgfqpoint{0.739887in}{0.366611in}}{\pgfqpoint{0.742082in}{0.371911in}}{\pgfqpoint{0.742082in}{0.377436in}}%
\pgfpathcurveto{\pgfqpoint{0.742082in}{0.382961in}}{\pgfqpoint{0.739887in}{0.388260in}}{\pgfqpoint{0.735980in}{0.392167in}}%
\pgfpathcurveto{\pgfqpoint{0.732073in}{0.396074in}}{\pgfqpoint{0.726774in}{0.398269in}}{\pgfqpoint{0.721249in}{0.398269in}}%
\pgfpathcurveto{\pgfqpoint{0.715724in}{0.398269in}}{\pgfqpoint{0.710424in}{0.396074in}}{\pgfqpoint{0.706518in}{0.392167in}}%
\pgfpathcurveto{\pgfqpoint{0.702611in}{0.388260in}}{\pgfqpoint{0.700416in}{0.382961in}}{\pgfqpoint{0.700416in}{0.377436in}}%
\pgfpathcurveto{\pgfqpoint{0.700416in}{0.371911in}}{\pgfqpoint{0.702611in}{0.366611in}}{\pgfqpoint{0.706518in}{0.362704in}}%
\pgfpathcurveto{\pgfqpoint{0.710424in}{0.358798in}}{\pgfqpoint{0.715724in}{0.356602in}}{\pgfqpoint{0.721249in}{0.356602in}}%
\pgfpathclose%
\pgfusepath{stroke,fill}%
\end{pgfscope}%
\begin{pgfscope}%
\pgfpathrectangle{\pgfqpoint{0.562500in}{0.275000in}}{\pgfqpoint{3.487500in}{1.925000in}}%
\pgfusepath{clip}%
\pgfsetbuttcap%
\pgfsetroundjoin%
\definecolor{currentfill}{rgb}{0.000000,0.000000,0.000000}%
\pgfsetfillcolor{currentfill}%
\pgfsetlinewidth{1.003750pt}%
\definecolor{currentstroke}{rgb}{0.000000,0.000000,0.000000}%
\pgfsetstrokecolor{currentstroke}%
\pgfsetdash{}{0pt}%
\pgfpathmoveto{\pgfqpoint{0.721249in}{0.356602in}}%
\pgfpathcurveto{\pgfqpoint{0.726774in}{0.356602in}}{\pgfqpoint{0.732073in}{0.358798in}}{\pgfqpoint{0.735980in}{0.362704in}}%
\pgfpathcurveto{\pgfqpoint{0.739887in}{0.366611in}}{\pgfqpoint{0.742082in}{0.371911in}}{\pgfqpoint{0.742082in}{0.377436in}}%
\pgfpathcurveto{\pgfqpoint{0.742082in}{0.382961in}}{\pgfqpoint{0.739887in}{0.388260in}}{\pgfqpoint{0.735980in}{0.392167in}}%
\pgfpathcurveto{\pgfqpoint{0.732073in}{0.396074in}}{\pgfqpoint{0.726774in}{0.398269in}}{\pgfqpoint{0.721249in}{0.398269in}}%
\pgfpathcurveto{\pgfqpoint{0.715724in}{0.398269in}}{\pgfqpoint{0.710424in}{0.396074in}}{\pgfqpoint{0.706518in}{0.392167in}}%
\pgfpathcurveto{\pgfqpoint{0.702611in}{0.388260in}}{\pgfqpoint{0.700416in}{0.382961in}}{\pgfqpoint{0.700416in}{0.377436in}}%
\pgfpathcurveto{\pgfqpoint{0.700416in}{0.371911in}}{\pgfqpoint{0.702611in}{0.366611in}}{\pgfqpoint{0.706518in}{0.362704in}}%
\pgfpathcurveto{\pgfqpoint{0.710424in}{0.358798in}}{\pgfqpoint{0.715724in}{0.356602in}}{\pgfqpoint{0.721249in}{0.356602in}}%
\pgfpathclose%
\pgfusepath{stroke,fill}%
\end{pgfscope}%
\begin{pgfscope}%
\pgfpathrectangle{\pgfqpoint{0.562500in}{0.275000in}}{\pgfqpoint{3.487500in}{1.925000in}}%
\pgfusepath{clip}%
\pgfsetbuttcap%
\pgfsetroundjoin%
\definecolor{currentfill}{rgb}{0.000000,0.000000,0.000000}%
\pgfsetfillcolor{currentfill}%
\pgfsetlinewidth{1.003750pt}%
\definecolor{currentstroke}{rgb}{0.000000,0.000000,0.000000}%
\pgfsetstrokecolor{currentstroke}%
\pgfsetdash{}{0pt}%
\pgfpathmoveto{\pgfqpoint{0.721249in}{0.356602in}}%
\pgfpathcurveto{\pgfqpoint{0.726774in}{0.356602in}}{\pgfqpoint{0.732073in}{0.358798in}}{\pgfqpoint{0.735980in}{0.362704in}}%
\pgfpathcurveto{\pgfqpoint{0.739887in}{0.366611in}}{\pgfqpoint{0.742082in}{0.371911in}}{\pgfqpoint{0.742082in}{0.377436in}}%
\pgfpathcurveto{\pgfqpoint{0.742082in}{0.382961in}}{\pgfqpoint{0.739887in}{0.388260in}}{\pgfqpoint{0.735980in}{0.392167in}}%
\pgfpathcurveto{\pgfqpoint{0.732073in}{0.396074in}}{\pgfqpoint{0.726774in}{0.398269in}}{\pgfqpoint{0.721249in}{0.398269in}}%
\pgfpathcurveto{\pgfqpoint{0.715724in}{0.398269in}}{\pgfqpoint{0.710424in}{0.396074in}}{\pgfqpoint{0.706518in}{0.392167in}}%
\pgfpathcurveto{\pgfqpoint{0.702611in}{0.388260in}}{\pgfqpoint{0.700416in}{0.382961in}}{\pgfqpoint{0.700416in}{0.377436in}}%
\pgfpathcurveto{\pgfqpoint{0.700416in}{0.371911in}}{\pgfqpoint{0.702611in}{0.366611in}}{\pgfqpoint{0.706518in}{0.362704in}}%
\pgfpathcurveto{\pgfqpoint{0.710424in}{0.358798in}}{\pgfqpoint{0.715724in}{0.356602in}}{\pgfqpoint{0.721249in}{0.356602in}}%
\pgfpathclose%
\pgfusepath{stroke,fill}%
\end{pgfscope}%
\begin{pgfscope}%
\pgfpathrectangle{\pgfqpoint{0.562500in}{0.275000in}}{\pgfqpoint{3.487500in}{1.925000in}}%
\pgfusepath{clip}%
\pgfsetbuttcap%
\pgfsetroundjoin%
\definecolor{currentfill}{rgb}{0.000000,0.000000,0.000000}%
\pgfsetfillcolor{currentfill}%
\pgfsetlinewidth{1.003750pt}%
\definecolor{currentstroke}{rgb}{0.000000,0.000000,0.000000}%
\pgfsetstrokecolor{currentstroke}%
\pgfsetdash{}{0pt}%
\pgfpathmoveto{\pgfqpoint{0.721249in}{1.216635in}}%
\pgfpathcurveto{\pgfqpoint{0.726774in}{1.216635in}}{\pgfqpoint{0.732073in}{1.218830in}}{\pgfqpoint{0.735980in}{1.222736in}}%
\pgfpathcurveto{\pgfqpoint{0.739887in}{1.226643in}}{\pgfqpoint{0.742082in}{1.231943in}}{\pgfqpoint{0.742082in}{1.237468in}}%
\pgfpathcurveto{\pgfqpoint{0.742082in}{1.242993in}}{\pgfqpoint{0.739887in}{1.248292in}}{\pgfqpoint{0.735980in}{1.252199in}}%
\pgfpathcurveto{\pgfqpoint{0.732073in}{1.256106in}}{\pgfqpoint{0.726774in}{1.258301in}}{\pgfqpoint{0.721249in}{1.258301in}}%
\pgfpathcurveto{\pgfqpoint{0.715724in}{1.258301in}}{\pgfqpoint{0.710424in}{1.256106in}}{\pgfqpoint{0.706518in}{1.252199in}}%
\pgfpathcurveto{\pgfqpoint{0.702611in}{1.248292in}}{\pgfqpoint{0.700416in}{1.242993in}}{\pgfqpoint{0.700416in}{1.237468in}}%
\pgfpathcurveto{\pgfqpoint{0.700416in}{1.231943in}}{\pgfqpoint{0.702611in}{1.226643in}}{\pgfqpoint{0.706518in}{1.222736in}}%
\pgfpathcurveto{\pgfqpoint{0.710424in}{1.218830in}}{\pgfqpoint{0.715724in}{1.216635in}}{\pgfqpoint{0.721249in}{1.216635in}}%
\pgfpathclose%
\pgfusepath{stroke,fill}%
\end{pgfscope}%
\begin{pgfscope}%
\pgfpathrectangle{\pgfqpoint{0.562500in}{0.275000in}}{\pgfqpoint{3.487500in}{1.925000in}}%
\pgfusepath{clip}%
\pgfsetbuttcap%
\pgfsetroundjoin%
\definecolor{currentfill}{rgb}{0.000000,0.000000,0.000000}%
\pgfsetfillcolor{currentfill}%
\pgfsetlinewidth{1.003750pt}%
\definecolor{currentstroke}{rgb}{0.000000,0.000000,0.000000}%
\pgfsetstrokecolor{currentstroke}%
\pgfsetdash{}{0pt}%
\pgfpathmoveto{\pgfqpoint{0.721249in}{1.216635in}}%
\pgfpathcurveto{\pgfqpoint{0.726774in}{1.216635in}}{\pgfqpoint{0.732073in}{1.218830in}}{\pgfqpoint{0.735980in}{1.222736in}}%
\pgfpathcurveto{\pgfqpoint{0.739887in}{1.226643in}}{\pgfqpoint{0.742082in}{1.231943in}}{\pgfqpoint{0.742082in}{1.237468in}}%
\pgfpathcurveto{\pgfqpoint{0.742082in}{1.242993in}}{\pgfqpoint{0.739887in}{1.248292in}}{\pgfqpoint{0.735980in}{1.252199in}}%
\pgfpathcurveto{\pgfqpoint{0.732073in}{1.256106in}}{\pgfqpoint{0.726774in}{1.258301in}}{\pgfqpoint{0.721249in}{1.258301in}}%
\pgfpathcurveto{\pgfqpoint{0.715724in}{1.258301in}}{\pgfqpoint{0.710424in}{1.256106in}}{\pgfqpoint{0.706518in}{1.252199in}}%
\pgfpathcurveto{\pgfqpoint{0.702611in}{1.248292in}}{\pgfqpoint{0.700416in}{1.242993in}}{\pgfqpoint{0.700416in}{1.237468in}}%
\pgfpathcurveto{\pgfqpoint{0.700416in}{1.231943in}}{\pgfqpoint{0.702611in}{1.226643in}}{\pgfqpoint{0.706518in}{1.222736in}}%
\pgfpathcurveto{\pgfqpoint{0.710424in}{1.218830in}}{\pgfqpoint{0.715724in}{1.216635in}}{\pgfqpoint{0.721249in}{1.216635in}}%
\pgfpathclose%
\pgfusepath{stroke,fill}%
\end{pgfscope}%
\begin{pgfscope}%
\pgfpathrectangle{\pgfqpoint{0.562500in}{0.275000in}}{\pgfqpoint{3.487500in}{1.925000in}}%
\pgfusepath{clip}%
\pgfsetbuttcap%
\pgfsetroundjoin%
\definecolor{currentfill}{rgb}{0.000000,0.000000,0.000000}%
\pgfsetfillcolor{currentfill}%
\pgfsetlinewidth{1.003750pt}%
\definecolor{currentstroke}{rgb}{0.000000,0.000000,0.000000}%
\pgfsetstrokecolor{currentstroke}%
\pgfsetdash{}{0pt}%
\pgfpathmoveto{\pgfqpoint{0.721249in}{0.356602in}}%
\pgfpathcurveto{\pgfqpoint{0.726774in}{0.356602in}}{\pgfqpoint{0.732073in}{0.358798in}}{\pgfqpoint{0.735980in}{0.362704in}}%
\pgfpathcurveto{\pgfqpoint{0.739887in}{0.366611in}}{\pgfqpoint{0.742082in}{0.371911in}}{\pgfqpoint{0.742082in}{0.377436in}}%
\pgfpathcurveto{\pgfqpoint{0.742082in}{0.382961in}}{\pgfqpoint{0.739887in}{0.388260in}}{\pgfqpoint{0.735980in}{0.392167in}}%
\pgfpathcurveto{\pgfqpoint{0.732073in}{0.396074in}}{\pgfqpoint{0.726774in}{0.398269in}}{\pgfqpoint{0.721249in}{0.398269in}}%
\pgfpathcurveto{\pgfqpoint{0.715724in}{0.398269in}}{\pgfqpoint{0.710424in}{0.396074in}}{\pgfqpoint{0.706518in}{0.392167in}}%
\pgfpathcurveto{\pgfqpoint{0.702611in}{0.388260in}}{\pgfqpoint{0.700416in}{0.382961in}}{\pgfqpoint{0.700416in}{0.377436in}}%
\pgfpathcurveto{\pgfqpoint{0.700416in}{0.371911in}}{\pgfqpoint{0.702611in}{0.366611in}}{\pgfqpoint{0.706518in}{0.362704in}}%
\pgfpathcurveto{\pgfqpoint{0.710424in}{0.358798in}}{\pgfqpoint{0.715724in}{0.356602in}}{\pgfqpoint{0.721249in}{0.356602in}}%
\pgfpathclose%
\pgfusepath{stroke,fill}%
\end{pgfscope}%
\begin{pgfscope}%
\pgfpathrectangle{\pgfqpoint{0.562500in}{0.275000in}}{\pgfqpoint{3.487500in}{1.925000in}}%
\pgfusepath{clip}%
\pgfsetbuttcap%
\pgfsetroundjoin%
\definecolor{currentfill}{rgb}{0.000000,0.000000,0.000000}%
\pgfsetfillcolor{currentfill}%
\pgfsetlinewidth{1.003750pt}%
\definecolor{currentstroke}{rgb}{0.000000,0.000000,0.000000}%
\pgfsetstrokecolor{currentstroke}%
\pgfsetdash{}{0pt}%
\pgfpathmoveto{\pgfqpoint{0.721249in}{0.356602in}}%
\pgfpathcurveto{\pgfqpoint{0.726774in}{0.356602in}}{\pgfqpoint{0.732073in}{0.358798in}}{\pgfqpoint{0.735980in}{0.362704in}}%
\pgfpathcurveto{\pgfqpoint{0.739887in}{0.366611in}}{\pgfqpoint{0.742082in}{0.371911in}}{\pgfqpoint{0.742082in}{0.377436in}}%
\pgfpathcurveto{\pgfqpoint{0.742082in}{0.382961in}}{\pgfqpoint{0.739887in}{0.388260in}}{\pgfqpoint{0.735980in}{0.392167in}}%
\pgfpathcurveto{\pgfqpoint{0.732073in}{0.396074in}}{\pgfqpoint{0.726774in}{0.398269in}}{\pgfqpoint{0.721249in}{0.398269in}}%
\pgfpathcurveto{\pgfqpoint{0.715724in}{0.398269in}}{\pgfqpoint{0.710424in}{0.396074in}}{\pgfqpoint{0.706518in}{0.392167in}}%
\pgfpathcurveto{\pgfqpoint{0.702611in}{0.388260in}}{\pgfqpoint{0.700416in}{0.382961in}}{\pgfqpoint{0.700416in}{0.377436in}}%
\pgfpathcurveto{\pgfqpoint{0.700416in}{0.371911in}}{\pgfqpoint{0.702611in}{0.366611in}}{\pgfqpoint{0.706518in}{0.362704in}}%
\pgfpathcurveto{\pgfqpoint{0.710424in}{0.358798in}}{\pgfqpoint{0.715724in}{0.356602in}}{\pgfqpoint{0.721249in}{0.356602in}}%
\pgfpathclose%
\pgfusepath{stroke,fill}%
\end{pgfscope}%
\begin{pgfscope}%
\pgfpathrectangle{\pgfqpoint{0.562500in}{0.275000in}}{\pgfqpoint{3.487500in}{1.925000in}}%
\pgfusepath{clip}%
\pgfsetbuttcap%
\pgfsetroundjoin%
\definecolor{currentfill}{rgb}{0.000000,0.000000,0.000000}%
\pgfsetfillcolor{currentfill}%
\pgfsetlinewidth{1.003750pt}%
\definecolor{currentstroke}{rgb}{0.000000,0.000000,0.000000}%
\pgfsetstrokecolor{currentstroke}%
\pgfsetdash{}{0pt}%
\pgfpathmoveto{\pgfqpoint{0.721249in}{0.356602in}}%
\pgfpathcurveto{\pgfqpoint{0.726774in}{0.356602in}}{\pgfqpoint{0.732073in}{0.358798in}}{\pgfqpoint{0.735980in}{0.362704in}}%
\pgfpathcurveto{\pgfqpoint{0.739887in}{0.366611in}}{\pgfqpoint{0.742082in}{0.371911in}}{\pgfqpoint{0.742082in}{0.377436in}}%
\pgfpathcurveto{\pgfqpoint{0.742082in}{0.382961in}}{\pgfqpoint{0.739887in}{0.388260in}}{\pgfqpoint{0.735980in}{0.392167in}}%
\pgfpathcurveto{\pgfqpoint{0.732073in}{0.396074in}}{\pgfqpoint{0.726774in}{0.398269in}}{\pgfqpoint{0.721249in}{0.398269in}}%
\pgfpathcurveto{\pgfqpoint{0.715724in}{0.398269in}}{\pgfqpoint{0.710424in}{0.396074in}}{\pgfqpoint{0.706518in}{0.392167in}}%
\pgfpathcurveto{\pgfqpoint{0.702611in}{0.388260in}}{\pgfqpoint{0.700416in}{0.382961in}}{\pgfqpoint{0.700416in}{0.377436in}}%
\pgfpathcurveto{\pgfqpoint{0.700416in}{0.371911in}}{\pgfqpoint{0.702611in}{0.366611in}}{\pgfqpoint{0.706518in}{0.362704in}}%
\pgfpathcurveto{\pgfqpoint{0.710424in}{0.358798in}}{\pgfqpoint{0.715724in}{0.356602in}}{\pgfqpoint{0.721249in}{0.356602in}}%
\pgfpathclose%
\pgfusepath{stroke,fill}%
\end{pgfscope}%
\begin{pgfscope}%
\pgfpathrectangle{\pgfqpoint{0.562500in}{0.275000in}}{\pgfqpoint{3.487500in}{1.925000in}}%
\pgfusepath{clip}%
\pgfsetbuttcap%
\pgfsetroundjoin%
\definecolor{currentfill}{rgb}{0.000000,0.000000,0.000000}%
\pgfsetfillcolor{currentfill}%
\pgfsetlinewidth{1.003750pt}%
\definecolor{currentstroke}{rgb}{0.000000,0.000000,0.000000}%
\pgfsetstrokecolor{currentstroke}%
\pgfsetdash{}{0pt}%
\pgfpathmoveto{\pgfqpoint{0.721249in}{0.356602in}}%
\pgfpathcurveto{\pgfqpoint{0.726774in}{0.356602in}}{\pgfqpoint{0.732073in}{0.358798in}}{\pgfqpoint{0.735980in}{0.362704in}}%
\pgfpathcurveto{\pgfqpoint{0.739887in}{0.366611in}}{\pgfqpoint{0.742082in}{0.371911in}}{\pgfqpoint{0.742082in}{0.377436in}}%
\pgfpathcurveto{\pgfqpoint{0.742082in}{0.382961in}}{\pgfqpoint{0.739887in}{0.388260in}}{\pgfqpoint{0.735980in}{0.392167in}}%
\pgfpathcurveto{\pgfqpoint{0.732073in}{0.396074in}}{\pgfqpoint{0.726774in}{0.398269in}}{\pgfqpoint{0.721249in}{0.398269in}}%
\pgfpathcurveto{\pgfqpoint{0.715724in}{0.398269in}}{\pgfqpoint{0.710424in}{0.396074in}}{\pgfqpoint{0.706518in}{0.392167in}}%
\pgfpathcurveto{\pgfqpoint{0.702611in}{0.388260in}}{\pgfqpoint{0.700416in}{0.382961in}}{\pgfqpoint{0.700416in}{0.377436in}}%
\pgfpathcurveto{\pgfqpoint{0.700416in}{0.371911in}}{\pgfqpoint{0.702611in}{0.366611in}}{\pgfqpoint{0.706518in}{0.362704in}}%
\pgfpathcurveto{\pgfqpoint{0.710424in}{0.358798in}}{\pgfqpoint{0.715724in}{0.356602in}}{\pgfqpoint{0.721249in}{0.356602in}}%
\pgfpathclose%
\pgfusepath{stroke,fill}%
\end{pgfscope}%
\begin{pgfscope}%
\pgfpathrectangle{\pgfqpoint{0.562500in}{0.275000in}}{\pgfqpoint{3.487500in}{1.925000in}}%
\pgfusepath{clip}%
\pgfsetbuttcap%
\pgfsetroundjoin%
\definecolor{currentfill}{rgb}{0.000000,0.000000,0.000000}%
\pgfsetfillcolor{currentfill}%
\pgfsetlinewidth{1.003750pt}%
\definecolor{currentstroke}{rgb}{0.000000,0.000000,0.000000}%
\pgfsetstrokecolor{currentstroke}%
\pgfsetdash{}{0pt}%
\pgfpathmoveto{\pgfqpoint{0.721249in}{1.216635in}}%
\pgfpathcurveto{\pgfqpoint{0.726774in}{1.216635in}}{\pgfqpoint{0.732073in}{1.218830in}}{\pgfqpoint{0.735980in}{1.222736in}}%
\pgfpathcurveto{\pgfqpoint{0.739887in}{1.226643in}}{\pgfqpoint{0.742082in}{1.231943in}}{\pgfqpoint{0.742082in}{1.237468in}}%
\pgfpathcurveto{\pgfqpoint{0.742082in}{1.242993in}}{\pgfqpoint{0.739887in}{1.248292in}}{\pgfqpoint{0.735980in}{1.252199in}}%
\pgfpathcurveto{\pgfqpoint{0.732073in}{1.256106in}}{\pgfqpoint{0.726774in}{1.258301in}}{\pgfqpoint{0.721249in}{1.258301in}}%
\pgfpathcurveto{\pgfqpoint{0.715724in}{1.258301in}}{\pgfqpoint{0.710424in}{1.256106in}}{\pgfqpoint{0.706518in}{1.252199in}}%
\pgfpathcurveto{\pgfqpoint{0.702611in}{1.248292in}}{\pgfqpoint{0.700416in}{1.242993in}}{\pgfqpoint{0.700416in}{1.237468in}}%
\pgfpathcurveto{\pgfqpoint{0.700416in}{1.231943in}}{\pgfqpoint{0.702611in}{1.226643in}}{\pgfqpoint{0.706518in}{1.222736in}}%
\pgfpathcurveto{\pgfqpoint{0.710424in}{1.218830in}}{\pgfqpoint{0.715724in}{1.216635in}}{\pgfqpoint{0.721249in}{1.216635in}}%
\pgfpathclose%
\pgfusepath{stroke,fill}%
\end{pgfscope}%
\begin{pgfscope}%
\pgfpathrectangle{\pgfqpoint{0.562500in}{0.275000in}}{\pgfqpoint{3.487500in}{1.925000in}}%
\pgfusepath{clip}%
\pgfsetbuttcap%
\pgfsetroundjoin%
\definecolor{currentfill}{rgb}{0.000000,0.000000,0.000000}%
\pgfsetfillcolor{currentfill}%
\pgfsetlinewidth{1.003750pt}%
\definecolor{currentstroke}{rgb}{0.000000,0.000000,0.000000}%
\pgfsetstrokecolor{currentstroke}%
\pgfsetdash{}{0pt}%
\pgfpathmoveto{\pgfqpoint{0.721249in}{0.356602in}}%
\pgfpathcurveto{\pgfqpoint{0.726774in}{0.356602in}}{\pgfqpoint{0.732073in}{0.358798in}}{\pgfqpoint{0.735980in}{0.362704in}}%
\pgfpathcurveto{\pgfqpoint{0.739887in}{0.366611in}}{\pgfqpoint{0.742082in}{0.371911in}}{\pgfqpoint{0.742082in}{0.377436in}}%
\pgfpathcurveto{\pgfqpoint{0.742082in}{0.382961in}}{\pgfqpoint{0.739887in}{0.388260in}}{\pgfqpoint{0.735980in}{0.392167in}}%
\pgfpathcurveto{\pgfqpoint{0.732073in}{0.396074in}}{\pgfqpoint{0.726774in}{0.398269in}}{\pgfqpoint{0.721249in}{0.398269in}}%
\pgfpathcurveto{\pgfqpoint{0.715724in}{0.398269in}}{\pgfqpoint{0.710424in}{0.396074in}}{\pgfqpoint{0.706518in}{0.392167in}}%
\pgfpathcurveto{\pgfqpoint{0.702611in}{0.388260in}}{\pgfqpoint{0.700416in}{0.382961in}}{\pgfqpoint{0.700416in}{0.377436in}}%
\pgfpathcurveto{\pgfqpoint{0.700416in}{0.371911in}}{\pgfqpoint{0.702611in}{0.366611in}}{\pgfqpoint{0.706518in}{0.362704in}}%
\pgfpathcurveto{\pgfqpoint{0.710424in}{0.358798in}}{\pgfqpoint{0.715724in}{0.356602in}}{\pgfqpoint{0.721249in}{0.356602in}}%
\pgfpathclose%
\pgfusepath{stroke,fill}%
\end{pgfscope}%
\begin{pgfscope}%
\pgfpathrectangle{\pgfqpoint{0.562500in}{0.275000in}}{\pgfqpoint{3.487500in}{1.925000in}}%
\pgfusepath{clip}%
\pgfsetbuttcap%
\pgfsetroundjoin%
\definecolor{currentfill}{rgb}{0.000000,0.000000,0.000000}%
\pgfsetfillcolor{currentfill}%
\pgfsetlinewidth{1.003750pt}%
\definecolor{currentstroke}{rgb}{0.000000,0.000000,0.000000}%
\pgfsetstrokecolor{currentstroke}%
\pgfsetdash{}{0pt}%
\pgfpathmoveto{\pgfqpoint{0.721249in}{0.356602in}}%
\pgfpathcurveto{\pgfqpoint{0.726774in}{0.356602in}}{\pgfqpoint{0.732073in}{0.358798in}}{\pgfqpoint{0.735980in}{0.362704in}}%
\pgfpathcurveto{\pgfqpoint{0.739887in}{0.366611in}}{\pgfqpoint{0.742082in}{0.371911in}}{\pgfqpoint{0.742082in}{0.377436in}}%
\pgfpathcurveto{\pgfqpoint{0.742082in}{0.382961in}}{\pgfqpoint{0.739887in}{0.388260in}}{\pgfqpoint{0.735980in}{0.392167in}}%
\pgfpathcurveto{\pgfqpoint{0.732073in}{0.396074in}}{\pgfqpoint{0.726774in}{0.398269in}}{\pgfqpoint{0.721249in}{0.398269in}}%
\pgfpathcurveto{\pgfqpoint{0.715724in}{0.398269in}}{\pgfqpoint{0.710424in}{0.396074in}}{\pgfqpoint{0.706518in}{0.392167in}}%
\pgfpathcurveto{\pgfqpoint{0.702611in}{0.388260in}}{\pgfqpoint{0.700416in}{0.382961in}}{\pgfqpoint{0.700416in}{0.377436in}}%
\pgfpathcurveto{\pgfqpoint{0.700416in}{0.371911in}}{\pgfqpoint{0.702611in}{0.366611in}}{\pgfqpoint{0.706518in}{0.362704in}}%
\pgfpathcurveto{\pgfqpoint{0.710424in}{0.358798in}}{\pgfqpoint{0.715724in}{0.356602in}}{\pgfqpoint{0.721249in}{0.356602in}}%
\pgfpathclose%
\pgfusepath{stroke,fill}%
\end{pgfscope}%
\begin{pgfscope}%
\pgfpathrectangle{\pgfqpoint{0.562500in}{0.275000in}}{\pgfqpoint{3.487500in}{1.925000in}}%
\pgfusepath{clip}%
\pgfsetbuttcap%
\pgfsetroundjoin%
\definecolor{currentfill}{rgb}{0.000000,0.000000,0.000000}%
\pgfsetfillcolor{currentfill}%
\pgfsetlinewidth{1.003750pt}%
\definecolor{currentstroke}{rgb}{0.000000,0.000000,0.000000}%
\pgfsetstrokecolor{currentstroke}%
\pgfsetdash{}{0pt}%
\pgfpathmoveto{\pgfqpoint{0.721249in}{0.356602in}}%
\pgfpathcurveto{\pgfqpoint{0.726774in}{0.356602in}}{\pgfqpoint{0.732073in}{0.358798in}}{\pgfqpoint{0.735980in}{0.362704in}}%
\pgfpathcurveto{\pgfqpoint{0.739887in}{0.366611in}}{\pgfqpoint{0.742082in}{0.371911in}}{\pgfqpoint{0.742082in}{0.377436in}}%
\pgfpathcurveto{\pgfqpoint{0.742082in}{0.382961in}}{\pgfqpoint{0.739887in}{0.388260in}}{\pgfqpoint{0.735980in}{0.392167in}}%
\pgfpathcurveto{\pgfqpoint{0.732073in}{0.396074in}}{\pgfqpoint{0.726774in}{0.398269in}}{\pgfqpoint{0.721249in}{0.398269in}}%
\pgfpathcurveto{\pgfqpoint{0.715724in}{0.398269in}}{\pgfqpoint{0.710424in}{0.396074in}}{\pgfqpoint{0.706518in}{0.392167in}}%
\pgfpathcurveto{\pgfqpoint{0.702611in}{0.388260in}}{\pgfqpoint{0.700416in}{0.382961in}}{\pgfqpoint{0.700416in}{0.377436in}}%
\pgfpathcurveto{\pgfqpoint{0.700416in}{0.371911in}}{\pgfqpoint{0.702611in}{0.366611in}}{\pgfqpoint{0.706518in}{0.362704in}}%
\pgfpathcurveto{\pgfqpoint{0.710424in}{0.358798in}}{\pgfqpoint{0.715724in}{0.356602in}}{\pgfqpoint{0.721249in}{0.356602in}}%
\pgfpathclose%
\pgfusepath{stroke,fill}%
\end{pgfscope}%
\begin{pgfscope}%
\pgfpathrectangle{\pgfqpoint{0.562500in}{0.275000in}}{\pgfqpoint{3.487500in}{1.925000in}}%
\pgfusepath{clip}%
\pgfsetbuttcap%
\pgfsetroundjoin%
\definecolor{currentfill}{rgb}{0.000000,0.000000,0.000000}%
\pgfsetfillcolor{currentfill}%
\pgfsetlinewidth{1.003750pt}%
\definecolor{currentstroke}{rgb}{0.000000,0.000000,0.000000}%
\pgfsetstrokecolor{currentstroke}%
\pgfsetdash{}{0pt}%
\pgfpathmoveto{\pgfqpoint{0.721249in}{0.356602in}}%
\pgfpathcurveto{\pgfqpoint{0.726774in}{0.356602in}}{\pgfqpoint{0.732073in}{0.358798in}}{\pgfqpoint{0.735980in}{0.362704in}}%
\pgfpathcurveto{\pgfqpoint{0.739887in}{0.366611in}}{\pgfqpoint{0.742082in}{0.371911in}}{\pgfqpoint{0.742082in}{0.377436in}}%
\pgfpathcurveto{\pgfqpoint{0.742082in}{0.382961in}}{\pgfqpoint{0.739887in}{0.388260in}}{\pgfqpoint{0.735980in}{0.392167in}}%
\pgfpathcurveto{\pgfqpoint{0.732073in}{0.396074in}}{\pgfqpoint{0.726774in}{0.398269in}}{\pgfqpoint{0.721249in}{0.398269in}}%
\pgfpathcurveto{\pgfqpoint{0.715724in}{0.398269in}}{\pgfqpoint{0.710424in}{0.396074in}}{\pgfqpoint{0.706518in}{0.392167in}}%
\pgfpathcurveto{\pgfqpoint{0.702611in}{0.388260in}}{\pgfqpoint{0.700416in}{0.382961in}}{\pgfqpoint{0.700416in}{0.377436in}}%
\pgfpathcurveto{\pgfqpoint{0.700416in}{0.371911in}}{\pgfqpoint{0.702611in}{0.366611in}}{\pgfqpoint{0.706518in}{0.362704in}}%
\pgfpathcurveto{\pgfqpoint{0.710424in}{0.358798in}}{\pgfqpoint{0.715724in}{0.356602in}}{\pgfqpoint{0.721249in}{0.356602in}}%
\pgfpathclose%
\pgfusepath{stroke,fill}%
\end{pgfscope}%
\begin{pgfscope}%
\pgfpathrectangle{\pgfqpoint{0.562500in}{0.275000in}}{\pgfqpoint{3.487500in}{1.925000in}}%
\pgfusepath{clip}%
\pgfsetbuttcap%
\pgfsetroundjoin%
\definecolor{currentfill}{rgb}{0.000000,0.000000,0.000000}%
\pgfsetfillcolor{currentfill}%
\pgfsetlinewidth{1.003750pt}%
\definecolor{currentstroke}{rgb}{0.000000,0.000000,0.000000}%
\pgfsetstrokecolor{currentstroke}%
\pgfsetdash{}{0pt}%
\pgfpathmoveto{\pgfqpoint{0.721249in}{0.356602in}}%
\pgfpathcurveto{\pgfqpoint{0.726774in}{0.356602in}}{\pgfqpoint{0.732073in}{0.358798in}}{\pgfqpoint{0.735980in}{0.362704in}}%
\pgfpathcurveto{\pgfqpoint{0.739887in}{0.366611in}}{\pgfqpoint{0.742082in}{0.371911in}}{\pgfqpoint{0.742082in}{0.377436in}}%
\pgfpathcurveto{\pgfqpoint{0.742082in}{0.382961in}}{\pgfqpoint{0.739887in}{0.388260in}}{\pgfqpoint{0.735980in}{0.392167in}}%
\pgfpathcurveto{\pgfqpoint{0.732073in}{0.396074in}}{\pgfqpoint{0.726774in}{0.398269in}}{\pgfqpoint{0.721249in}{0.398269in}}%
\pgfpathcurveto{\pgfqpoint{0.715724in}{0.398269in}}{\pgfqpoint{0.710424in}{0.396074in}}{\pgfqpoint{0.706518in}{0.392167in}}%
\pgfpathcurveto{\pgfqpoint{0.702611in}{0.388260in}}{\pgfqpoint{0.700416in}{0.382961in}}{\pgfqpoint{0.700416in}{0.377436in}}%
\pgfpathcurveto{\pgfqpoint{0.700416in}{0.371911in}}{\pgfqpoint{0.702611in}{0.366611in}}{\pgfqpoint{0.706518in}{0.362704in}}%
\pgfpathcurveto{\pgfqpoint{0.710424in}{0.358798in}}{\pgfqpoint{0.715724in}{0.356602in}}{\pgfqpoint{0.721249in}{0.356602in}}%
\pgfpathclose%
\pgfusepath{stroke,fill}%
\end{pgfscope}%
\begin{pgfscope}%
\pgfpathrectangle{\pgfqpoint{0.562500in}{0.275000in}}{\pgfqpoint{3.487500in}{1.925000in}}%
\pgfusepath{clip}%
\pgfsetbuttcap%
\pgfsetroundjoin%
\definecolor{currentfill}{rgb}{0.000000,0.000000,0.000000}%
\pgfsetfillcolor{currentfill}%
\pgfsetlinewidth{1.003750pt}%
\definecolor{currentstroke}{rgb}{0.000000,0.000000,0.000000}%
\pgfsetstrokecolor{currentstroke}%
\pgfsetdash{}{0pt}%
\pgfpathmoveto{\pgfqpoint{0.721249in}{0.356602in}}%
\pgfpathcurveto{\pgfqpoint{0.726774in}{0.356602in}}{\pgfqpoint{0.732073in}{0.358798in}}{\pgfqpoint{0.735980in}{0.362704in}}%
\pgfpathcurveto{\pgfqpoint{0.739887in}{0.366611in}}{\pgfqpoint{0.742082in}{0.371911in}}{\pgfqpoint{0.742082in}{0.377436in}}%
\pgfpathcurveto{\pgfqpoint{0.742082in}{0.382961in}}{\pgfqpoint{0.739887in}{0.388260in}}{\pgfqpoint{0.735980in}{0.392167in}}%
\pgfpathcurveto{\pgfqpoint{0.732073in}{0.396074in}}{\pgfqpoint{0.726774in}{0.398269in}}{\pgfqpoint{0.721249in}{0.398269in}}%
\pgfpathcurveto{\pgfqpoint{0.715724in}{0.398269in}}{\pgfqpoint{0.710424in}{0.396074in}}{\pgfqpoint{0.706518in}{0.392167in}}%
\pgfpathcurveto{\pgfqpoint{0.702611in}{0.388260in}}{\pgfqpoint{0.700416in}{0.382961in}}{\pgfqpoint{0.700416in}{0.377436in}}%
\pgfpathcurveto{\pgfqpoint{0.700416in}{0.371911in}}{\pgfqpoint{0.702611in}{0.366611in}}{\pgfqpoint{0.706518in}{0.362704in}}%
\pgfpathcurveto{\pgfqpoint{0.710424in}{0.358798in}}{\pgfqpoint{0.715724in}{0.356602in}}{\pgfqpoint{0.721249in}{0.356602in}}%
\pgfpathclose%
\pgfusepath{stroke,fill}%
\end{pgfscope}%
\begin{pgfscope}%
\pgfpathrectangle{\pgfqpoint{0.562500in}{0.275000in}}{\pgfqpoint{3.487500in}{1.925000in}}%
\pgfusepath{clip}%
\pgfsetbuttcap%
\pgfsetroundjoin%
\definecolor{currentfill}{rgb}{0.000000,0.000000,0.000000}%
\pgfsetfillcolor{currentfill}%
\pgfsetlinewidth{1.003750pt}%
\definecolor{currentstroke}{rgb}{0.000000,0.000000,0.000000}%
\pgfsetstrokecolor{currentstroke}%
\pgfsetdash{}{0pt}%
\pgfpathmoveto{\pgfqpoint{0.721249in}{0.356602in}}%
\pgfpathcurveto{\pgfqpoint{0.726774in}{0.356602in}}{\pgfqpoint{0.732073in}{0.358798in}}{\pgfqpoint{0.735980in}{0.362704in}}%
\pgfpathcurveto{\pgfqpoint{0.739887in}{0.366611in}}{\pgfqpoint{0.742082in}{0.371911in}}{\pgfqpoint{0.742082in}{0.377436in}}%
\pgfpathcurveto{\pgfqpoint{0.742082in}{0.382961in}}{\pgfqpoint{0.739887in}{0.388260in}}{\pgfqpoint{0.735980in}{0.392167in}}%
\pgfpathcurveto{\pgfqpoint{0.732073in}{0.396074in}}{\pgfqpoint{0.726774in}{0.398269in}}{\pgfqpoint{0.721249in}{0.398269in}}%
\pgfpathcurveto{\pgfqpoint{0.715724in}{0.398269in}}{\pgfqpoint{0.710424in}{0.396074in}}{\pgfqpoint{0.706518in}{0.392167in}}%
\pgfpathcurveto{\pgfqpoint{0.702611in}{0.388260in}}{\pgfqpoint{0.700416in}{0.382961in}}{\pgfqpoint{0.700416in}{0.377436in}}%
\pgfpathcurveto{\pgfqpoint{0.700416in}{0.371911in}}{\pgfqpoint{0.702611in}{0.366611in}}{\pgfqpoint{0.706518in}{0.362704in}}%
\pgfpathcurveto{\pgfqpoint{0.710424in}{0.358798in}}{\pgfqpoint{0.715724in}{0.356602in}}{\pgfqpoint{0.721249in}{0.356602in}}%
\pgfpathclose%
\pgfusepath{stroke,fill}%
\end{pgfscope}%
\begin{pgfscope}%
\pgfpathrectangle{\pgfqpoint{0.562500in}{0.275000in}}{\pgfqpoint{3.487500in}{1.925000in}}%
\pgfusepath{clip}%
\pgfsetbuttcap%
\pgfsetroundjoin%
\definecolor{currentfill}{rgb}{0.000000,0.000000,0.000000}%
\pgfsetfillcolor{currentfill}%
\pgfsetlinewidth{1.003750pt}%
\definecolor{currentstroke}{rgb}{0.000000,0.000000,0.000000}%
\pgfsetstrokecolor{currentstroke}%
\pgfsetdash{}{0pt}%
\pgfpathmoveto{\pgfqpoint{0.721249in}{0.356602in}}%
\pgfpathcurveto{\pgfqpoint{0.726774in}{0.356602in}}{\pgfqpoint{0.732073in}{0.358798in}}{\pgfqpoint{0.735980in}{0.362704in}}%
\pgfpathcurveto{\pgfqpoint{0.739887in}{0.366611in}}{\pgfqpoint{0.742082in}{0.371911in}}{\pgfqpoint{0.742082in}{0.377436in}}%
\pgfpathcurveto{\pgfqpoint{0.742082in}{0.382961in}}{\pgfqpoint{0.739887in}{0.388260in}}{\pgfqpoint{0.735980in}{0.392167in}}%
\pgfpathcurveto{\pgfqpoint{0.732073in}{0.396074in}}{\pgfqpoint{0.726774in}{0.398269in}}{\pgfqpoint{0.721249in}{0.398269in}}%
\pgfpathcurveto{\pgfqpoint{0.715724in}{0.398269in}}{\pgfqpoint{0.710424in}{0.396074in}}{\pgfqpoint{0.706518in}{0.392167in}}%
\pgfpathcurveto{\pgfqpoint{0.702611in}{0.388260in}}{\pgfqpoint{0.700416in}{0.382961in}}{\pgfqpoint{0.700416in}{0.377436in}}%
\pgfpathcurveto{\pgfqpoint{0.700416in}{0.371911in}}{\pgfqpoint{0.702611in}{0.366611in}}{\pgfqpoint{0.706518in}{0.362704in}}%
\pgfpathcurveto{\pgfqpoint{0.710424in}{0.358798in}}{\pgfqpoint{0.715724in}{0.356602in}}{\pgfqpoint{0.721249in}{0.356602in}}%
\pgfpathclose%
\pgfusepath{stroke,fill}%
\end{pgfscope}%
\begin{pgfscope}%
\pgfpathrectangle{\pgfqpoint{0.562500in}{0.275000in}}{\pgfqpoint{3.487500in}{1.925000in}}%
\pgfusepath{clip}%
\pgfsetbuttcap%
\pgfsetroundjoin%
\definecolor{currentfill}{rgb}{0.000000,0.000000,0.000000}%
\pgfsetfillcolor{currentfill}%
\pgfsetlinewidth{1.003750pt}%
\definecolor{currentstroke}{rgb}{0.000000,0.000000,0.000000}%
\pgfsetstrokecolor{currentstroke}%
\pgfsetdash{}{0pt}%
\pgfpathmoveto{\pgfqpoint{0.721249in}{0.356602in}}%
\pgfpathcurveto{\pgfqpoint{0.726774in}{0.356602in}}{\pgfqpoint{0.732073in}{0.358798in}}{\pgfqpoint{0.735980in}{0.362704in}}%
\pgfpathcurveto{\pgfqpoint{0.739887in}{0.366611in}}{\pgfqpoint{0.742082in}{0.371911in}}{\pgfqpoint{0.742082in}{0.377436in}}%
\pgfpathcurveto{\pgfqpoint{0.742082in}{0.382961in}}{\pgfqpoint{0.739887in}{0.388260in}}{\pgfqpoint{0.735980in}{0.392167in}}%
\pgfpathcurveto{\pgfqpoint{0.732073in}{0.396074in}}{\pgfqpoint{0.726774in}{0.398269in}}{\pgfqpoint{0.721249in}{0.398269in}}%
\pgfpathcurveto{\pgfqpoint{0.715724in}{0.398269in}}{\pgfqpoint{0.710424in}{0.396074in}}{\pgfqpoint{0.706518in}{0.392167in}}%
\pgfpathcurveto{\pgfqpoint{0.702611in}{0.388260in}}{\pgfqpoint{0.700416in}{0.382961in}}{\pgfqpoint{0.700416in}{0.377436in}}%
\pgfpathcurveto{\pgfqpoint{0.700416in}{0.371911in}}{\pgfqpoint{0.702611in}{0.366611in}}{\pgfqpoint{0.706518in}{0.362704in}}%
\pgfpathcurveto{\pgfqpoint{0.710424in}{0.358798in}}{\pgfqpoint{0.715724in}{0.356602in}}{\pgfqpoint{0.721249in}{0.356602in}}%
\pgfpathclose%
\pgfusepath{stroke,fill}%
\end{pgfscope}%
\begin{pgfscope}%
\pgfpathrectangle{\pgfqpoint{0.562500in}{0.275000in}}{\pgfqpoint{3.487500in}{1.925000in}}%
\pgfusepath{clip}%
\pgfsetbuttcap%
\pgfsetroundjoin%
\definecolor{currentfill}{rgb}{0.000000,0.000000,0.000000}%
\pgfsetfillcolor{currentfill}%
\pgfsetlinewidth{1.003750pt}%
\definecolor{currentstroke}{rgb}{0.000000,0.000000,0.000000}%
\pgfsetstrokecolor{currentstroke}%
\pgfsetdash{}{0pt}%
\pgfpathmoveto{\pgfqpoint{0.721249in}{0.356602in}}%
\pgfpathcurveto{\pgfqpoint{0.726774in}{0.356602in}}{\pgfqpoint{0.732073in}{0.358798in}}{\pgfqpoint{0.735980in}{0.362704in}}%
\pgfpathcurveto{\pgfqpoint{0.739887in}{0.366611in}}{\pgfqpoint{0.742082in}{0.371911in}}{\pgfqpoint{0.742082in}{0.377436in}}%
\pgfpathcurveto{\pgfqpoint{0.742082in}{0.382961in}}{\pgfqpoint{0.739887in}{0.388260in}}{\pgfqpoint{0.735980in}{0.392167in}}%
\pgfpathcurveto{\pgfqpoint{0.732073in}{0.396074in}}{\pgfqpoint{0.726774in}{0.398269in}}{\pgfqpoint{0.721249in}{0.398269in}}%
\pgfpathcurveto{\pgfqpoint{0.715724in}{0.398269in}}{\pgfqpoint{0.710424in}{0.396074in}}{\pgfqpoint{0.706518in}{0.392167in}}%
\pgfpathcurveto{\pgfqpoint{0.702611in}{0.388260in}}{\pgfqpoint{0.700416in}{0.382961in}}{\pgfqpoint{0.700416in}{0.377436in}}%
\pgfpathcurveto{\pgfqpoint{0.700416in}{0.371911in}}{\pgfqpoint{0.702611in}{0.366611in}}{\pgfqpoint{0.706518in}{0.362704in}}%
\pgfpathcurveto{\pgfqpoint{0.710424in}{0.358798in}}{\pgfqpoint{0.715724in}{0.356602in}}{\pgfqpoint{0.721249in}{0.356602in}}%
\pgfpathclose%
\pgfusepath{stroke,fill}%
\end{pgfscope}%
\begin{pgfscope}%
\pgfpathrectangle{\pgfqpoint{0.562500in}{0.275000in}}{\pgfqpoint{3.487500in}{1.925000in}}%
\pgfusepath{clip}%
\pgfsetbuttcap%
\pgfsetroundjoin%
\definecolor{currentfill}{rgb}{0.000000,0.000000,0.000000}%
\pgfsetfillcolor{currentfill}%
\pgfsetlinewidth{1.003750pt}%
\definecolor{currentstroke}{rgb}{0.000000,0.000000,0.000000}%
\pgfsetstrokecolor{currentstroke}%
\pgfsetdash{}{0pt}%
\pgfpathmoveto{\pgfqpoint{0.721249in}{0.356602in}}%
\pgfpathcurveto{\pgfqpoint{0.726774in}{0.356602in}}{\pgfqpoint{0.732073in}{0.358798in}}{\pgfqpoint{0.735980in}{0.362704in}}%
\pgfpathcurveto{\pgfqpoint{0.739887in}{0.366611in}}{\pgfqpoint{0.742082in}{0.371911in}}{\pgfqpoint{0.742082in}{0.377436in}}%
\pgfpathcurveto{\pgfqpoint{0.742082in}{0.382961in}}{\pgfqpoint{0.739887in}{0.388260in}}{\pgfqpoint{0.735980in}{0.392167in}}%
\pgfpathcurveto{\pgfqpoint{0.732073in}{0.396074in}}{\pgfqpoint{0.726774in}{0.398269in}}{\pgfqpoint{0.721249in}{0.398269in}}%
\pgfpathcurveto{\pgfqpoint{0.715724in}{0.398269in}}{\pgfqpoint{0.710424in}{0.396074in}}{\pgfqpoint{0.706518in}{0.392167in}}%
\pgfpathcurveto{\pgfqpoint{0.702611in}{0.388260in}}{\pgfqpoint{0.700416in}{0.382961in}}{\pgfqpoint{0.700416in}{0.377436in}}%
\pgfpathcurveto{\pgfqpoint{0.700416in}{0.371911in}}{\pgfqpoint{0.702611in}{0.366611in}}{\pgfqpoint{0.706518in}{0.362704in}}%
\pgfpathcurveto{\pgfqpoint{0.710424in}{0.358798in}}{\pgfqpoint{0.715724in}{0.356602in}}{\pgfqpoint{0.721249in}{0.356602in}}%
\pgfpathclose%
\pgfusepath{stroke,fill}%
\end{pgfscope}%
\begin{pgfscope}%
\pgfpathrectangle{\pgfqpoint{0.562500in}{0.275000in}}{\pgfqpoint{3.487500in}{1.925000in}}%
\pgfusepath{clip}%
\pgfsetbuttcap%
\pgfsetroundjoin%
\definecolor{currentfill}{rgb}{0.000000,0.000000,0.000000}%
\pgfsetfillcolor{currentfill}%
\pgfsetlinewidth{1.003750pt}%
\definecolor{currentstroke}{rgb}{0.000000,0.000000,0.000000}%
\pgfsetstrokecolor{currentstroke}%
\pgfsetdash{}{0pt}%
\pgfpathmoveto{\pgfqpoint{0.721249in}{0.356602in}}%
\pgfpathcurveto{\pgfqpoint{0.726774in}{0.356602in}}{\pgfqpoint{0.732073in}{0.358798in}}{\pgfqpoint{0.735980in}{0.362704in}}%
\pgfpathcurveto{\pgfqpoint{0.739887in}{0.366611in}}{\pgfqpoint{0.742082in}{0.371911in}}{\pgfqpoint{0.742082in}{0.377436in}}%
\pgfpathcurveto{\pgfqpoint{0.742082in}{0.382961in}}{\pgfqpoint{0.739887in}{0.388260in}}{\pgfqpoint{0.735980in}{0.392167in}}%
\pgfpathcurveto{\pgfqpoint{0.732073in}{0.396074in}}{\pgfqpoint{0.726774in}{0.398269in}}{\pgfqpoint{0.721249in}{0.398269in}}%
\pgfpathcurveto{\pgfqpoint{0.715724in}{0.398269in}}{\pgfqpoint{0.710424in}{0.396074in}}{\pgfqpoint{0.706518in}{0.392167in}}%
\pgfpathcurveto{\pgfqpoint{0.702611in}{0.388260in}}{\pgfqpoint{0.700416in}{0.382961in}}{\pgfqpoint{0.700416in}{0.377436in}}%
\pgfpathcurveto{\pgfqpoint{0.700416in}{0.371911in}}{\pgfqpoint{0.702611in}{0.366611in}}{\pgfqpoint{0.706518in}{0.362704in}}%
\pgfpathcurveto{\pgfqpoint{0.710424in}{0.358798in}}{\pgfqpoint{0.715724in}{0.356602in}}{\pgfqpoint{0.721249in}{0.356602in}}%
\pgfpathclose%
\pgfusepath{stroke,fill}%
\end{pgfscope}%
\begin{pgfscope}%
\pgfpathrectangle{\pgfqpoint{0.562500in}{0.275000in}}{\pgfqpoint{3.487500in}{1.925000in}}%
\pgfusepath{clip}%
\pgfsetbuttcap%
\pgfsetroundjoin%
\definecolor{currentfill}{rgb}{0.000000,0.000000,0.000000}%
\pgfsetfillcolor{currentfill}%
\pgfsetlinewidth{1.003750pt}%
\definecolor{currentstroke}{rgb}{0.000000,0.000000,0.000000}%
\pgfsetstrokecolor{currentstroke}%
\pgfsetdash{}{0pt}%
\pgfpathmoveto{\pgfqpoint{0.721249in}{0.356602in}}%
\pgfpathcurveto{\pgfqpoint{0.726774in}{0.356602in}}{\pgfqpoint{0.732073in}{0.358798in}}{\pgfqpoint{0.735980in}{0.362704in}}%
\pgfpathcurveto{\pgfqpoint{0.739887in}{0.366611in}}{\pgfqpoint{0.742082in}{0.371911in}}{\pgfqpoint{0.742082in}{0.377436in}}%
\pgfpathcurveto{\pgfqpoint{0.742082in}{0.382961in}}{\pgfqpoint{0.739887in}{0.388260in}}{\pgfqpoint{0.735980in}{0.392167in}}%
\pgfpathcurveto{\pgfqpoint{0.732073in}{0.396074in}}{\pgfqpoint{0.726774in}{0.398269in}}{\pgfqpoint{0.721249in}{0.398269in}}%
\pgfpathcurveto{\pgfqpoint{0.715724in}{0.398269in}}{\pgfqpoint{0.710424in}{0.396074in}}{\pgfqpoint{0.706518in}{0.392167in}}%
\pgfpathcurveto{\pgfqpoint{0.702611in}{0.388260in}}{\pgfqpoint{0.700416in}{0.382961in}}{\pgfqpoint{0.700416in}{0.377436in}}%
\pgfpathcurveto{\pgfqpoint{0.700416in}{0.371911in}}{\pgfqpoint{0.702611in}{0.366611in}}{\pgfqpoint{0.706518in}{0.362704in}}%
\pgfpathcurveto{\pgfqpoint{0.710424in}{0.358798in}}{\pgfqpoint{0.715724in}{0.356602in}}{\pgfqpoint{0.721249in}{0.356602in}}%
\pgfpathclose%
\pgfusepath{stroke,fill}%
\end{pgfscope}%
\begin{pgfscope}%
\pgfpathrectangle{\pgfqpoint{0.562500in}{0.275000in}}{\pgfqpoint{3.487500in}{1.925000in}}%
\pgfusepath{clip}%
\pgfsetbuttcap%
\pgfsetroundjoin%
\definecolor{currentfill}{rgb}{0.000000,0.000000,0.000000}%
\pgfsetfillcolor{currentfill}%
\pgfsetlinewidth{1.003750pt}%
\definecolor{currentstroke}{rgb}{0.000000,0.000000,0.000000}%
\pgfsetstrokecolor{currentstroke}%
\pgfsetdash{}{0pt}%
\pgfpathmoveto{\pgfqpoint{0.721249in}{1.216635in}}%
\pgfpathcurveto{\pgfqpoint{0.726774in}{1.216635in}}{\pgfqpoint{0.732073in}{1.218830in}}{\pgfqpoint{0.735980in}{1.222736in}}%
\pgfpathcurveto{\pgfqpoint{0.739887in}{1.226643in}}{\pgfqpoint{0.742082in}{1.231943in}}{\pgfqpoint{0.742082in}{1.237468in}}%
\pgfpathcurveto{\pgfqpoint{0.742082in}{1.242993in}}{\pgfqpoint{0.739887in}{1.248292in}}{\pgfqpoint{0.735980in}{1.252199in}}%
\pgfpathcurveto{\pgfqpoint{0.732073in}{1.256106in}}{\pgfqpoint{0.726774in}{1.258301in}}{\pgfqpoint{0.721249in}{1.258301in}}%
\pgfpathcurveto{\pgfqpoint{0.715724in}{1.258301in}}{\pgfqpoint{0.710424in}{1.256106in}}{\pgfqpoint{0.706518in}{1.252199in}}%
\pgfpathcurveto{\pgfqpoint{0.702611in}{1.248292in}}{\pgfqpoint{0.700416in}{1.242993in}}{\pgfqpoint{0.700416in}{1.237468in}}%
\pgfpathcurveto{\pgfqpoint{0.700416in}{1.231943in}}{\pgfqpoint{0.702611in}{1.226643in}}{\pgfqpoint{0.706518in}{1.222736in}}%
\pgfpathcurveto{\pgfqpoint{0.710424in}{1.218830in}}{\pgfqpoint{0.715724in}{1.216635in}}{\pgfqpoint{0.721249in}{1.216635in}}%
\pgfpathclose%
\pgfusepath{stroke,fill}%
\end{pgfscope}%
\begin{pgfscope}%
\pgfpathrectangle{\pgfqpoint{0.562500in}{0.275000in}}{\pgfqpoint{3.487500in}{1.925000in}}%
\pgfusepath{clip}%
\pgfsetbuttcap%
\pgfsetroundjoin%
\definecolor{currentfill}{rgb}{0.000000,0.000000,0.000000}%
\pgfsetfillcolor{currentfill}%
\pgfsetlinewidth{1.003750pt}%
\definecolor{currentstroke}{rgb}{0.000000,0.000000,0.000000}%
\pgfsetstrokecolor{currentstroke}%
\pgfsetdash{}{0pt}%
\pgfpathmoveto{\pgfqpoint{0.721249in}{0.356602in}}%
\pgfpathcurveto{\pgfqpoint{0.726774in}{0.356602in}}{\pgfqpoint{0.732073in}{0.358798in}}{\pgfqpoint{0.735980in}{0.362704in}}%
\pgfpathcurveto{\pgfqpoint{0.739887in}{0.366611in}}{\pgfqpoint{0.742082in}{0.371911in}}{\pgfqpoint{0.742082in}{0.377436in}}%
\pgfpathcurveto{\pgfqpoint{0.742082in}{0.382961in}}{\pgfqpoint{0.739887in}{0.388260in}}{\pgfqpoint{0.735980in}{0.392167in}}%
\pgfpathcurveto{\pgfqpoint{0.732073in}{0.396074in}}{\pgfqpoint{0.726774in}{0.398269in}}{\pgfqpoint{0.721249in}{0.398269in}}%
\pgfpathcurveto{\pgfqpoint{0.715724in}{0.398269in}}{\pgfqpoint{0.710424in}{0.396074in}}{\pgfqpoint{0.706518in}{0.392167in}}%
\pgfpathcurveto{\pgfqpoint{0.702611in}{0.388260in}}{\pgfqpoint{0.700416in}{0.382961in}}{\pgfqpoint{0.700416in}{0.377436in}}%
\pgfpathcurveto{\pgfqpoint{0.700416in}{0.371911in}}{\pgfqpoint{0.702611in}{0.366611in}}{\pgfqpoint{0.706518in}{0.362704in}}%
\pgfpathcurveto{\pgfqpoint{0.710424in}{0.358798in}}{\pgfqpoint{0.715724in}{0.356602in}}{\pgfqpoint{0.721249in}{0.356602in}}%
\pgfpathclose%
\pgfusepath{stroke,fill}%
\end{pgfscope}%
\begin{pgfscope}%
\pgfpathrectangle{\pgfqpoint{0.562500in}{0.275000in}}{\pgfqpoint{3.487500in}{1.925000in}}%
\pgfusepath{clip}%
\pgfsetbuttcap%
\pgfsetroundjoin%
\definecolor{currentfill}{rgb}{0.000000,0.000000,0.000000}%
\pgfsetfillcolor{currentfill}%
\pgfsetlinewidth{1.003750pt}%
\definecolor{currentstroke}{rgb}{0.000000,0.000000,0.000000}%
\pgfsetstrokecolor{currentstroke}%
\pgfsetdash{}{0pt}%
\pgfpathmoveto{\pgfqpoint{0.721249in}{0.356602in}}%
\pgfpathcurveto{\pgfqpoint{0.726774in}{0.356602in}}{\pgfqpoint{0.732073in}{0.358798in}}{\pgfqpoint{0.735980in}{0.362704in}}%
\pgfpathcurveto{\pgfqpoint{0.739887in}{0.366611in}}{\pgfqpoint{0.742082in}{0.371911in}}{\pgfqpoint{0.742082in}{0.377436in}}%
\pgfpathcurveto{\pgfqpoint{0.742082in}{0.382961in}}{\pgfqpoint{0.739887in}{0.388260in}}{\pgfqpoint{0.735980in}{0.392167in}}%
\pgfpathcurveto{\pgfqpoint{0.732073in}{0.396074in}}{\pgfqpoint{0.726774in}{0.398269in}}{\pgfqpoint{0.721249in}{0.398269in}}%
\pgfpathcurveto{\pgfqpoint{0.715724in}{0.398269in}}{\pgfqpoint{0.710424in}{0.396074in}}{\pgfqpoint{0.706518in}{0.392167in}}%
\pgfpathcurveto{\pgfqpoint{0.702611in}{0.388260in}}{\pgfqpoint{0.700416in}{0.382961in}}{\pgfqpoint{0.700416in}{0.377436in}}%
\pgfpathcurveto{\pgfqpoint{0.700416in}{0.371911in}}{\pgfqpoint{0.702611in}{0.366611in}}{\pgfqpoint{0.706518in}{0.362704in}}%
\pgfpathcurveto{\pgfqpoint{0.710424in}{0.358798in}}{\pgfqpoint{0.715724in}{0.356602in}}{\pgfqpoint{0.721249in}{0.356602in}}%
\pgfpathclose%
\pgfusepath{stroke,fill}%
\end{pgfscope}%
\begin{pgfscope}%
\pgfpathrectangle{\pgfqpoint{0.562500in}{0.275000in}}{\pgfqpoint{3.487500in}{1.925000in}}%
\pgfusepath{clip}%
\pgfsetbuttcap%
\pgfsetroundjoin%
\definecolor{currentfill}{rgb}{0.000000,0.000000,0.000000}%
\pgfsetfillcolor{currentfill}%
\pgfsetlinewidth{1.003750pt}%
\definecolor{currentstroke}{rgb}{0.000000,0.000000,0.000000}%
\pgfsetstrokecolor{currentstroke}%
\pgfsetdash{}{0pt}%
\pgfpathmoveto{\pgfqpoint{0.721249in}{0.356602in}}%
\pgfpathcurveto{\pgfqpoint{0.726774in}{0.356602in}}{\pgfqpoint{0.732073in}{0.358798in}}{\pgfqpoint{0.735980in}{0.362704in}}%
\pgfpathcurveto{\pgfqpoint{0.739887in}{0.366611in}}{\pgfqpoint{0.742082in}{0.371911in}}{\pgfqpoint{0.742082in}{0.377436in}}%
\pgfpathcurveto{\pgfqpoint{0.742082in}{0.382961in}}{\pgfqpoint{0.739887in}{0.388260in}}{\pgfqpoint{0.735980in}{0.392167in}}%
\pgfpathcurveto{\pgfqpoint{0.732073in}{0.396074in}}{\pgfqpoint{0.726774in}{0.398269in}}{\pgfqpoint{0.721249in}{0.398269in}}%
\pgfpathcurveto{\pgfqpoint{0.715724in}{0.398269in}}{\pgfqpoint{0.710424in}{0.396074in}}{\pgfqpoint{0.706518in}{0.392167in}}%
\pgfpathcurveto{\pgfqpoint{0.702611in}{0.388260in}}{\pgfqpoint{0.700416in}{0.382961in}}{\pgfqpoint{0.700416in}{0.377436in}}%
\pgfpathcurveto{\pgfqpoint{0.700416in}{0.371911in}}{\pgfqpoint{0.702611in}{0.366611in}}{\pgfqpoint{0.706518in}{0.362704in}}%
\pgfpathcurveto{\pgfqpoint{0.710424in}{0.358798in}}{\pgfqpoint{0.715724in}{0.356602in}}{\pgfqpoint{0.721249in}{0.356602in}}%
\pgfpathclose%
\pgfusepath{stroke,fill}%
\end{pgfscope}%
\begin{pgfscope}%
\pgfpathrectangle{\pgfqpoint{0.562500in}{0.275000in}}{\pgfqpoint{3.487500in}{1.925000in}}%
\pgfusepath{clip}%
\pgfsetbuttcap%
\pgfsetroundjoin%
\definecolor{currentfill}{rgb}{0.000000,0.000000,0.000000}%
\pgfsetfillcolor{currentfill}%
\pgfsetlinewidth{1.003750pt}%
\definecolor{currentstroke}{rgb}{0.000000,0.000000,0.000000}%
\pgfsetstrokecolor{currentstroke}%
\pgfsetdash{}{0pt}%
\pgfpathmoveto{\pgfqpoint{0.721249in}{1.216635in}}%
\pgfpathcurveto{\pgfqpoint{0.726774in}{1.216635in}}{\pgfqpoint{0.732073in}{1.218830in}}{\pgfqpoint{0.735980in}{1.222736in}}%
\pgfpathcurveto{\pgfqpoint{0.739887in}{1.226643in}}{\pgfqpoint{0.742082in}{1.231943in}}{\pgfqpoint{0.742082in}{1.237468in}}%
\pgfpathcurveto{\pgfqpoint{0.742082in}{1.242993in}}{\pgfqpoint{0.739887in}{1.248292in}}{\pgfqpoint{0.735980in}{1.252199in}}%
\pgfpathcurveto{\pgfqpoint{0.732073in}{1.256106in}}{\pgfqpoint{0.726774in}{1.258301in}}{\pgfqpoint{0.721249in}{1.258301in}}%
\pgfpathcurveto{\pgfqpoint{0.715724in}{1.258301in}}{\pgfqpoint{0.710424in}{1.256106in}}{\pgfqpoint{0.706518in}{1.252199in}}%
\pgfpathcurveto{\pgfqpoint{0.702611in}{1.248292in}}{\pgfqpoint{0.700416in}{1.242993in}}{\pgfqpoint{0.700416in}{1.237468in}}%
\pgfpathcurveto{\pgfqpoint{0.700416in}{1.231943in}}{\pgfqpoint{0.702611in}{1.226643in}}{\pgfqpoint{0.706518in}{1.222736in}}%
\pgfpathcurveto{\pgfqpoint{0.710424in}{1.218830in}}{\pgfqpoint{0.715724in}{1.216635in}}{\pgfqpoint{0.721249in}{1.216635in}}%
\pgfpathclose%
\pgfusepath{stroke,fill}%
\end{pgfscope}%
\begin{pgfscope}%
\pgfpathrectangle{\pgfqpoint{0.562500in}{0.275000in}}{\pgfqpoint{3.487500in}{1.925000in}}%
\pgfusepath{clip}%
\pgfsetbuttcap%
\pgfsetroundjoin%
\definecolor{currentfill}{rgb}{0.000000,0.000000,0.000000}%
\pgfsetfillcolor{currentfill}%
\pgfsetlinewidth{1.003750pt}%
\definecolor{currentstroke}{rgb}{0.000000,0.000000,0.000000}%
\pgfsetstrokecolor{currentstroke}%
\pgfsetdash{}{0pt}%
\pgfpathmoveto{\pgfqpoint{0.721249in}{0.356602in}}%
\pgfpathcurveto{\pgfqpoint{0.726774in}{0.356602in}}{\pgfqpoint{0.732073in}{0.358798in}}{\pgfqpoint{0.735980in}{0.362704in}}%
\pgfpathcurveto{\pgfqpoint{0.739887in}{0.366611in}}{\pgfqpoint{0.742082in}{0.371911in}}{\pgfqpoint{0.742082in}{0.377436in}}%
\pgfpathcurveto{\pgfqpoint{0.742082in}{0.382961in}}{\pgfqpoint{0.739887in}{0.388260in}}{\pgfqpoint{0.735980in}{0.392167in}}%
\pgfpathcurveto{\pgfqpoint{0.732073in}{0.396074in}}{\pgfqpoint{0.726774in}{0.398269in}}{\pgfqpoint{0.721249in}{0.398269in}}%
\pgfpathcurveto{\pgfqpoint{0.715724in}{0.398269in}}{\pgfqpoint{0.710424in}{0.396074in}}{\pgfqpoint{0.706518in}{0.392167in}}%
\pgfpathcurveto{\pgfqpoint{0.702611in}{0.388260in}}{\pgfqpoint{0.700416in}{0.382961in}}{\pgfqpoint{0.700416in}{0.377436in}}%
\pgfpathcurveto{\pgfqpoint{0.700416in}{0.371911in}}{\pgfqpoint{0.702611in}{0.366611in}}{\pgfqpoint{0.706518in}{0.362704in}}%
\pgfpathcurveto{\pgfqpoint{0.710424in}{0.358798in}}{\pgfqpoint{0.715724in}{0.356602in}}{\pgfqpoint{0.721249in}{0.356602in}}%
\pgfpathclose%
\pgfusepath{stroke,fill}%
\end{pgfscope}%
\begin{pgfscope}%
\pgfpathrectangle{\pgfqpoint{0.562500in}{0.275000in}}{\pgfqpoint{3.487500in}{1.925000in}}%
\pgfusepath{clip}%
\pgfsetbuttcap%
\pgfsetroundjoin%
\definecolor{currentfill}{rgb}{0.000000,0.000000,0.000000}%
\pgfsetfillcolor{currentfill}%
\pgfsetlinewidth{1.003750pt}%
\definecolor{currentstroke}{rgb}{0.000000,0.000000,0.000000}%
\pgfsetstrokecolor{currentstroke}%
\pgfsetdash{}{0pt}%
\pgfpathmoveto{\pgfqpoint{0.721249in}{0.356602in}}%
\pgfpathcurveto{\pgfqpoint{0.726774in}{0.356602in}}{\pgfqpoint{0.732073in}{0.358798in}}{\pgfqpoint{0.735980in}{0.362704in}}%
\pgfpathcurveto{\pgfqpoint{0.739887in}{0.366611in}}{\pgfqpoint{0.742082in}{0.371911in}}{\pgfqpoint{0.742082in}{0.377436in}}%
\pgfpathcurveto{\pgfqpoint{0.742082in}{0.382961in}}{\pgfqpoint{0.739887in}{0.388260in}}{\pgfqpoint{0.735980in}{0.392167in}}%
\pgfpathcurveto{\pgfqpoint{0.732073in}{0.396074in}}{\pgfqpoint{0.726774in}{0.398269in}}{\pgfqpoint{0.721249in}{0.398269in}}%
\pgfpathcurveto{\pgfqpoint{0.715724in}{0.398269in}}{\pgfqpoint{0.710424in}{0.396074in}}{\pgfqpoint{0.706518in}{0.392167in}}%
\pgfpathcurveto{\pgfqpoint{0.702611in}{0.388260in}}{\pgfqpoint{0.700416in}{0.382961in}}{\pgfqpoint{0.700416in}{0.377436in}}%
\pgfpathcurveto{\pgfqpoint{0.700416in}{0.371911in}}{\pgfqpoint{0.702611in}{0.366611in}}{\pgfqpoint{0.706518in}{0.362704in}}%
\pgfpathcurveto{\pgfqpoint{0.710424in}{0.358798in}}{\pgfqpoint{0.715724in}{0.356602in}}{\pgfqpoint{0.721249in}{0.356602in}}%
\pgfpathclose%
\pgfusepath{stroke,fill}%
\end{pgfscope}%
\begin{pgfscope}%
\pgfpathrectangle{\pgfqpoint{0.562500in}{0.275000in}}{\pgfqpoint{3.487500in}{1.925000in}}%
\pgfusepath{clip}%
\pgfsetbuttcap%
\pgfsetroundjoin%
\definecolor{currentfill}{rgb}{0.000000,0.000000,0.000000}%
\pgfsetfillcolor{currentfill}%
\pgfsetlinewidth{1.003750pt}%
\definecolor{currentstroke}{rgb}{0.000000,0.000000,0.000000}%
\pgfsetstrokecolor{currentstroke}%
\pgfsetdash{}{0pt}%
\pgfpathmoveto{\pgfqpoint{0.721249in}{0.356602in}}%
\pgfpathcurveto{\pgfqpoint{0.726774in}{0.356602in}}{\pgfqpoint{0.732073in}{0.358798in}}{\pgfqpoint{0.735980in}{0.362704in}}%
\pgfpathcurveto{\pgfqpoint{0.739887in}{0.366611in}}{\pgfqpoint{0.742082in}{0.371911in}}{\pgfqpoint{0.742082in}{0.377436in}}%
\pgfpathcurveto{\pgfqpoint{0.742082in}{0.382961in}}{\pgfqpoint{0.739887in}{0.388260in}}{\pgfqpoint{0.735980in}{0.392167in}}%
\pgfpathcurveto{\pgfqpoint{0.732073in}{0.396074in}}{\pgfqpoint{0.726774in}{0.398269in}}{\pgfqpoint{0.721249in}{0.398269in}}%
\pgfpathcurveto{\pgfqpoint{0.715724in}{0.398269in}}{\pgfqpoint{0.710424in}{0.396074in}}{\pgfqpoint{0.706518in}{0.392167in}}%
\pgfpathcurveto{\pgfqpoint{0.702611in}{0.388260in}}{\pgfqpoint{0.700416in}{0.382961in}}{\pgfqpoint{0.700416in}{0.377436in}}%
\pgfpathcurveto{\pgfqpoint{0.700416in}{0.371911in}}{\pgfqpoint{0.702611in}{0.366611in}}{\pgfqpoint{0.706518in}{0.362704in}}%
\pgfpathcurveto{\pgfqpoint{0.710424in}{0.358798in}}{\pgfqpoint{0.715724in}{0.356602in}}{\pgfqpoint{0.721249in}{0.356602in}}%
\pgfpathclose%
\pgfusepath{stroke,fill}%
\end{pgfscope}%
\begin{pgfscope}%
\pgfpathrectangle{\pgfqpoint{0.562500in}{0.275000in}}{\pgfqpoint{3.487500in}{1.925000in}}%
\pgfusepath{clip}%
\pgfsetbuttcap%
\pgfsetroundjoin%
\definecolor{currentfill}{rgb}{0.000000,0.000000,0.000000}%
\pgfsetfillcolor{currentfill}%
\pgfsetlinewidth{1.003750pt}%
\definecolor{currentstroke}{rgb}{0.000000,0.000000,0.000000}%
\pgfsetstrokecolor{currentstroke}%
\pgfsetdash{}{0pt}%
\pgfpathmoveto{\pgfqpoint{0.721249in}{0.356602in}}%
\pgfpathcurveto{\pgfqpoint{0.726774in}{0.356602in}}{\pgfqpoint{0.732073in}{0.358798in}}{\pgfqpoint{0.735980in}{0.362704in}}%
\pgfpathcurveto{\pgfqpoint{0.739887in}{0.366611in}}{\pgfqpoint{0.742082in}{0.371911in}}{\pgfqpoint{0.742082in}{0.377436in}}%
\pgfpathcurveto{\pgfqpoint{0.742082in}{0.382961in}}{\pgfqpoint{0.739887in}{0.388260in}}{\pgfqpoint{0.735980in}{0.392167in}}%
\pgfpathcurveto{\pgfqpoint{0.732073in}{0.396074in}}{\pgfqpoint{0.726774in}{0.398269in}}{\pgfqpoint{0.721249in}{0.398269in}}%
\pgfpathcurveto{\pgfqpoint{0.715724in}{0.398269in}}{\pgfqpoint{0.710424in}{0.396074in}}{\pgfqpoint{0.706518in}{0.392167in}}%
\pgfpathcurveto{\pgfqpoint{0.702611in}{0.388260in}}{\pgfqpoint{0.700416in}{0.382961in}}{\pgfqpoint{0.700416in}{0.377436in}}%
\pgfpathcurveto{\pgfqpoint{0.700416in}{0.371911in}}{\pgfqpoint{0.702611in}{0.366611in}}{\pgfqpoint{0.706518in}{0.362704in}}%
\pgfpathcurveto{\pgfqpoint{0.710424in}{0.358798in}}{\pgfqpoint{0.715724in}{0.356602in}}{\pgfqpoint{0.721249in}{0.356602in}}%
\pgfpathclose%
\pgfusepath{stroke,fill}%
\end{pgfscope}%
\begin{pgfscope}%
\pgfpathrectangle{\pgfqpoint{0.562500in}{0.275000in}}{\pgfqpoint{3.487500in}{1.925000in}}%
\pgfusepath{clip}%
\pgfsetbuttcap%
\pgfsetroundjoin%
\definecolor{currentfill}{rgb}{0.000000,0.000000,0.000000}%
\pgfsetfillcolor{currentfill}%
\pgfsetlinewidth{1.003750pt}%
\definecolor{currentstroke}{rgb}{0.000000,0.000000,0.000000}%
\pgfsetstrokecolor{currentstroke}%
\pgfsetdash{}{0pt}%
\pgfpathmoveto{\pgfqpoint{0.721249in}{0.356602in}}%
\pgfpathcurveto{\pgfqpoint{0.726774in}{0.356602in}}{\pgfqpoint{0.732073in}{0.358798in}}{\pgfqpoint{0.735980in}{0.362704in}}%
\pgfpathcurveto{\pgfqpoint{0.739887in}{0.366611in}}{\pgfqpoint{0.742082in}{0.371911in}}{\pgfqpoint{0.742082in}{0.377436in}}%
\pgfpathcurveto{\pgfqpoint{0.742082in}{0.382961in}}{\pgfqpoint{0.739887in}{0.388260in}}{\pgfqpoint{0.735980in}{0.392167in}}%
\pgfpathcurveto{\pgfqpoint{0.732073in}{0.396074in}}{\pgfqpoint{0.726774in}{0.398269in}}{\pgfqpoint{0.721249in}{0.398269in}}%
\pgfpathcurveto{\pgfqpoint{0.715724in}{0.398269in}}{\pgfqpoint{0.710424in}{0.396074in}}{\pgfqpoint{0.706518in}{0.392167in}}%
\pgfpathcurveto{\pgfqpoint{0.702611in}{0.388260in}}{\pgfqpoint{0.700416in}{0.382961in}}{\pgfqpoint{0.700416in}{0.377436in}}%
\pgfpathcurveto{\pgfqpoint{0.700416in}{0.371911in}}{\pgfqpoint{0.702611in}{0.366611in}}{\pgfqpoint{0.706518in}{0.362704in}}%
\pgfpathcurveto{\pgfqpoint{0.710424in}{0.358798in}}{\pgfqpoint{0.715724in}{0.356602in}}{\pgfqpoint{0.721249in}{0.356602in}}%
\pgfpathclose%
\pgfusepath{stroke,fill}%
\end{pgfscope}%
\begin{pgfscope}%
\pgfpathrectangle{\pgfqpoint{0.562500in}{0.275000in}}{\pgfqpoint{3.487500in}{1.925000in}}%
\pgfusepath{clip}%
\pgfsetbuttcap%
\pgfsetroundjoin%
\definecolor{currentfill}{rgb}{0.000000,0.000000,0.000000}%
\pgfsetfillcolor{currentfill}%
\pgfsetlinewidth{1.003750pt}%
\definecolor{currentstroke}{rgb}{0.000000,0.000000,0.000000}%
\pgfsetstrokecolor{currentstroke}%
\pgfsetdash{}{0pt}%
\pgfpathmoveto{\pgfqpoint{0.721249in}{0.356602in}}%
\pgfpathcurveto{\pgfqpoint{0.726774in}{0.356602in}}{\pgfqpoint{0.732073in}{0.358798in}}{\pgfqpoint{0.735980in}{0.362704in}}%
\pgfpathcurveto{\pgfqpoint{0.739887in}{0.366611in}}{\pgfqpoint{0.742082in}{0.371911in}}{\pgfqpoint{0.742082in}{0.377436in}}%
\pgfpathcurveto{\pgfqpoint{0.742082in}{0.382961in}}{\pgfqpoint{0.739887in}{0.388260in}}{\pgfqpoint{0.735980in}{0.392167in}}%
\pgfpathcurveto{\pgfqpoint{0.732073in}{0.396074in}}{\pgfqpoint{0.726774in}{0.398269in}}{\pgfqpoint{0.721249in}{0.398269in}}%
\pgfpathcurveto{\pgfqpoint{0.715724in}{0.398269in}}{\pgfqpoint{0.710424in}{0.396074in}}{\pgfqpoint{0.706518in}{0.392167in}}%
\pgfpathcurveto{\pgfqpoint{0.702611in}{0.388260in}}{\pgfqpoint{0.700416in}{0.382961in}}{\pgfqpoint{0.700416in}{0.377436in}}%
\pgfpathcurveto{\pgfqpoint{0.700416in}{0.371911in}}{\pgfqpoint{0.702611in}{0.366611in}}{\pgfqpoint{0.706518in}{0.362704in}}%
\pgfpathcurveto{\pgfqpoint{0.710424in}{0.358798in}}{\pgfqpoint{0.715724in}{0.356602in}}{\pgfqpoint{0.721249in}{0.356602in}}%
\pgfpathclose%
\pgfusepath{stroke,fill}%
\end{pgfscope}%
\begin{pgfscope}%
\pgfpathrectangle{\pgfqpoint{0.562500in}{0.275000in}}{\pgfqpoint{3.487500in}{1.925000in}}%
\pgfusepath{clip}%
\pgfsetbuttcap%
\pgfsetroundjoin%
\definecolor{currentfill}{rgb}{0.000000,0.000000,0.000000}%
\pgfsetfillcolor{currentfill}%
\pgfsetlinewidth{1.003750pt}%
\definecolor{currentstroke}{rgb}{0.000000,0.000000,0.000000}%
\pgfsetstrokecolor{currentstroke}%
\pgfsetdash{}{0pt}%
\pgfpathmoveto{\pgfqpoint{0.721249in}{0.356602in}}%
\pgfpathcurveto{\pgfqpoint{0.726774in}{0.356602in}}{\pgfqpoint{0.732073in}{0.358798in}}{\pgfqpoint{0.735980in}{0.362704in}}%
\pgfpathcurveto{\pgfqpoint{0.739887in}{0.366611in}}{\pgfqpoint{0.742082in}{0.371911in}}{\pgfqpoint{0.742082in}{0.377436in}}%
\pgfpathcurveto{\pgfqpoint{0.742082in}{0.382961in}}{\pgfqpoint{0.739887in}{0.388260in}}{\pgfqpoint{0.735980in}{0.392167in}}%
\pgfpathcurveto{\pgfqpoint{0.732073in}{0.396074in}}{\pgfqpoint{0.726774in}{0.398269in}}{\pgfqpoint{0.721249in}{0.398269in}}%
\pgfpathcurveto{\pgfqpoint{0.715724in}{0.398269in}}{\pgfqpoint{0.710424in}{0.396074in}}{\pgfqpoint{0.706518in}{0.392167in}}%
\pgfpathcurveto{\pgfqpoint{0.702611in}{0.388260in}}{\pgfqpoint{0.700416in}{0.382961in}}{\pgfqpoint{0.700416in}{0.377436in}}%
\pgfpathcurveto{\pgfqpoint{0.700416in}{0.371911in}}{\pgfqpoint{0.702611in}{0.366611in}}{\pgfqpoint{0.706518in}{0.362704in}}%
\pgfpathcurveto{\pgfqpoint{0.710424in}{0.358798in}}{\pgfqpoint{0.715724in}{0.356602in}}{\pgfqpoint{0.721249in}{0.356602in}}%
\pgfpathclose%
\pgfusepath{stroke,fill}%
\end{pgfscope}%
\begin{pgfscope}%
\pgfpathrectangle{\pgfqpoint{0.562500in}{0.275000in}}{\pgfqpoint{3.487500in}{1.925000in}}%
\pgfusepath{clip}%
\pgfsetbuttcap%
\pgfsetroundjoin%
\definecolor{currentfill}{rgb}{0.000000,0.000000,0.000000}%
\pgfsetfillcolor{currentfill}%
\pgfsetlinewidth{1.003750pt}%
\definecolor{currentstroke}{rgb}{0.000000,0.000000,0.000000}%
\pgfsetstrokecolor{currentstroke}%
\pgfsetdash{}{0pt}%
\pgfpathmoveto{\pgfqpoint{0.721249in}{0.356602in}}%
\pgfpathcurveto{\pgfqpoint{0.726774in}{0.356602in}}{\pgfqpoint{0.732073in}{0.358798in}}{\pgfqpoint{0.735980in}{0.362704in}}%
\pgfpathcurveto{\pgfqpoint{0.739887in}{0.366611in}}{\pgfqpoint{0.742082in}{0.371911in}}{\pgfqpoint{0.742082in}{0.377436in}}%
\pgfpathcurveto{\pgfqpoint{0.742082in}{0.382961in}}{\pgfqpoint{0.739887in}{0.388260in}}{\pgfqpoint{0.735980in}{0.392167in}}%
\pgfpathcurveto{\pgfqpoint{0.732073in}{0.396074in}}{\pgfqpoint{0.726774in}{0.398269in}}{\pgfqpoint{0.721249in}{0.398269in}}%
\pgfpathcurveto{\pgfqpoint{0.715724in}{0.398269in}}{\pgfqpoint{0.710424in}{0.396074in}}{\pgfqpoint{0.706518in}{0.392167in}}%
\pgfpathcurveto{\pgfqpoint{0.702611in}{0.388260in}}{\pgfqpoint{0.700416in}{0.382961in}}{\pgfqpoint{0.700416in}{0.377436in}}%
\pgfpathcurveto{\pgfqpoint{0.700416in}{0.371911in}}{\pgfqpoint{0.702611in}{0.366611in}}{\pgfqpoint{0.706518in}{0.362704in}}%
\pgfpathcurveto{\pgfqpoint{0.710424in}{0.358798in}}{\pgfqpoint{0.715724in}{0.356602in}}{\pgfqpoint{0.721249in}{0.356602in}}%
\pgfpathclose%
\pgfusepath{stroke,fill}%
\end{pgfscope}%
\begin{pgfscope}%
\pgfpathrectangle{\pgfqpoint{0.562500in}{0.275000in}}{\pgfqpoint{3.487500in}{1.925000in}}%
\pgfusepath{clip}%
\pgfsetbuttcap%
\pgfsetroundjoin%
\definecolor{currentfill}{rgb}{0.000000,0.000000,0.000000}%
\pgfsetfillcolor{currentfill}%
\pgfsetlinewidth{1.003750pt}%
\definecolor{currentstroke}{rgb}{0.000000,0.000000,0.000000}%
\pgfsetstrokecolor{currentstroke}%
\pgfsetdash{}{0pt}%
\pgfpathmoveto{\pgfqpoint{0.721249in}{0.356602in}}%
\pgfpathcurveto{\pgfqpoint{0.726774in}{0.356602in}}{\pgfqpoint{0.732073in}{0.358798in}}{\pgfqpoint{0.735980in}{0.362704in}}%
\pgfpathcurveto{\pgfqpoint{0.739887in}{0.366611in}}{\pgfqpoint{0.742082in}{0.371911in}}{\pgfqpoint{0.742082in}{0.377436in}}%
\pgfpathcurveto{\pgfqpoint{0.742082in}{0.382961in}}{\pgfqpoint{0.739887in}{0.388260in}}{\pgfqpoint{0.735980in}{0.392167in}}%
\pgfpathcurveto{\pgfqpoint{0.732073in}{0.396074in}}{\pgfqpoint{0.726774in}{0.398269in}}{\pgfqpoint{0.721249in}{0.398269in}}%
\pgfpathcurveto{\pgfqpoint{0.715724in}{0.398269in}}{\pgfqpoint{0.710424in}{0.396074in}}{\pgfqpoint{0.706518in}{0.392167in}}%
\pgfpathcurveto{\pgfqpoint{0.702611in}{0.388260in}}{\pgfqpoint{0.700416in}{0.382961in}}{\pgfqpoint{0.700416in}{0.377436in}}%
\pgfpathcurveto{\pgfqpoint{0.700416in}{0.371911in}}{\pgfqpoint{0.702611in}{0.366611in}}{\pgfqpoint{0.706518in}{0.362704in}}%
\pgfpathcurveto{\pgfqpoint{0.710424in}{0.358798in}}{\pgfqpoint{0.715724in}{0.356602in}}{\pgfqpoint{0.721249in}{0.356602in}}%
\pgfpathclose%
\pgfusepath{stroke,fill}%
\end{pgfscope}%
\begin{pgfscope}%
\pgfpathrectangle{\pgfqpoint{0.562500in}{0.275000in}}{\pgfqpoint{3.487500in}{1.925000in}}%
\pgfusepath{clip}%
\pgfsetbuttcap%
\pgfsetroundjoin%
\definecolor{currentfill}{rgb}{0.000000,0.000000,0.000000}%
\pgfsetfillcolor{currentfill}%
\pgfsetlinewidth{1.003750pt}%
\definecolor{currentstroke}{rgb}{0.000000,0.000000,0.000000}%
\pgfsetstrokecolor{currentstroke}%
\pgfsetdash{}{0pt}%
\pgfpathmoveto{\pgfqpoint{0.721249in}{0.356602in}}%
\pgfpathcurveto{\pgfqpoint{0.726774in}{0.356602in}}{\pgfqpoint{0.732073in}{0.358798in}}{\pgfqpoint{0.735980in}{0.362704in}}%
\pgfpathcurveto{\pgfqpoint{0.739887in}{0.366611in}}{\pgfqpoint{0.742082in}{0.371911in}}{\pgfqpoint{0.742082in}{0.377436in}}%
\pgfpathcurveto{\pgfqpoint{0.742082in}{0.382961in}}{\pgfqpoint{0.739887in}{0.388260in}}{\pgfqpoint{0.735980in}{0.392167in}}%
\pgfpathcurveto{\pgfqpoint{0.732073in}{0.396074in}}{\pgfqpoint{0.726774in}{0.398269in}}{\pgfqpoint{0.721249in}{0.398269in}}%
\pgfpathcurveto{\pgfqpoint{0.715724in}{0.398269in}}{\pgfqpoint{0.710424in}{0.396074in}}{\pgfqpoint{0.706518in}{0.392167in}}%
\pgfpathcurveto{\pgfqpoint{0.702611in}{0.388260in}}{\pgfqpoint{0.700416in}{0.382961in}}{\pgfqpoint{0.700416in}{0.377436in}}%
\pgfpathcurveto{\pgfqpoint{0.700416in}{0.371911in}}{\pgfqpoint{0.702611in}{0.366611in}}{\pgfqpoint{0.706518in}{0.362704in}}%
\pgfpathcurveto{\pgfqpoint{0.710424in}{0.358798in}}{\pgfqpoint{0.715724in}{0.356602in}}{\pgfqpoint{0.721249in}{0.356602in}}%
\pgfpathclose%
\pgfusepath{stroke,fill}%
\end{pgfscope}%
\begin{pgfscope}%
\pgfpathrectangle{\pgfqpoint{0.562500in}{0.275000in}}{\pgfqpoint{3.487500in}{1.925000in}}%
\pgfusepath{clip}%
\pgfsetbuttcap%
\pgfsetroundjoin%
\definecolor{currentfill}{rgb}{0.000000,0.000000,0.000000}%
\pgfsetfillcolor{currentfill}%
\pgfsetlinewidth{1.003750pt}%
\definecolor{currentstroke}{rgb}{0.000000,0.000000,0.000000}%
\pgfsetstrokecolor{currentstroke}%
\pgfsetdash{}{0pt}%
\pgfpathmoveto{\pgfqpoint{0.721249in}{0.356602in}}%
\pgfpathcurveto{\pgfqpoint{0.726774in}{0.356602in}}{\pgfqpoint{0.732073in}{0.358798in}}{\pgfqpoint{0.735980in}{0.362704in}}%
\pgfpathcurveto{\pgfqpoint{0.739887in}{0.366611in}}{\pgfqpoint{0.742082in}{0.371911in}}{\pgfqpoint{0.742082in}{0.377436in}}%
\pgfpathcurveto{\pgfqpoint{0.742082in}{0.382961in}}{\pgfqpoint{0.739887in}{0.388260in}}{\pgfqpoint{0.735980in}{0.392167in}}%
\pgfpathcurveto{\pgfqpoint{0.732073in}{0.396074in}}{\pgfqpoint{0.726774in}{0.398269in}}{\pgfqpoint{0.721249in}{0.398269in}}%
\pgfpathcurveto{\pgfqpoint{0.715724in}{0.398269in}}{\pgfqpoint{0.710424in}{0.396074in}}{\pgfqpoint{0.706518in}{0.392167in}}%
\pgfpathcurveto{\pgfqpoint{0.702611in}{0.388260in}}{\pgfqpoint{0.700416in}{0.382961in}}{\pgfqpoint{0.700416in}{0.377436in}}%
\pgfpathcurveto{\pgfqpoint{0.700416in}{0.371911in}}{\pgfqpoint{0.702611in}{0.366611in}}{\pgfqpoint{0.706518in}{0.362704in}}%
\pgfpathcurveto{\pgfqpoint{0.710424in}{0.358798in}}{\pgfqpoint{0.715724in}{0.356602in}}{\pgfqpoint{0.721249in}{0.356602in}}%
\pgfpathclose%
\pgfusepath{stroke,fill}%
\end{pgfscope}%
\begin{pgfscope}%
\pgfpathrectangle{\pgfqpoint{0.562500in}{0.275000in}}{\pgfqpoint{3.487500in}{1.925000in}}%
\pgfusepath{clip}%
\pgfsetbuttcap%
\pgfsetroundjoin%
\definecolor{currentfill}{rgb}{0.000000,0.000000,0.000000}%
\pgfsetfillcolor{currentfill}%
\pgfsetlinewidth{1.003750pt}%
\definecolor{currentstroke}{rgb}{0.000000,0.000000,0.000000}%
\pgfsetstrokecolor{currentstroke}%
\pgfsetdash{}{0pt}%
\pgfpathmoveto{\pgfqpoint{0.721249in}{1.216635in}}%
\pgfpathcurveto{\pgfqpoint{0.726774in}{1.216635in}}{\pgfqpoint{0.732073in}{1.218830in}}{\pgfqpoint{0.735980in}{1.222736in}}%
\pgfpathcurveto{\pgfqpoint{0.739887in}{1.226643in}}{\pgfqpoint{0.742082in}{1.231943in}}{\pgfqpoint{0.742082in}{1.237468in}}%
\pgfpathcurveto{\pgfqpoint{0.742082in}{1.242993in}}{\pgfqpoint{0.739887in}{1.248292in}}{\pgfqpoint{0.735980in}{1.252199in}}%
\pgfpathcurveto{\pgfqpoint{0.732073in}{1.256106in}}{\pgfqpoint{0.726774in}{1.258301in}}{\pgfqpoint{0.721249in}{1.258301in}}%
\pgfpathcurveto{\pgfqpoint{0.715724in}{1.258301in}}{\pgfqpoint{0.710424in}{1.256106in}}{\pgfqpoint{0.706518in}{1.252199in}}%
\pgfpathcurveto{\pgfqpoint{0.702611in}{1.248292in}}{\pgfqpoint{0.700416in}{1.242993in}}{\pgfqpoint{0.700416in}{1.237468in}}%
\pgfpathcurveto{\pgfqpoint{0.700416in}{1.231943in}}{\pgfqpoint{0.702611in}{1.226643in}}{\pgfqpoint{0.706518in}{1.222736in}}%
\pgfpathcurveto{\pgfqpoint{0.710424in}{1.218830in}}{\pgfqpoint{0.715724in}{1.216635in}}{\pgfqpoint{0.721249in}{1.216635in}}%
\pgfpathclose%
\pgfusepath{stroke,fill}%
\end{pgfscope}%
\begin{pgfscope}%
\pgfpathrectangle{\pgfqpoint{0.562500in}{0.275000in}}{\pgfqpoint{3.487500in}{1.925000in}}%
\pgfusepath{clip}%
\pgfsetbuttcap%
\pgfsetroundjoin%
\definecolor{currentfill}{rgb}{0.000000,0.000000,0.000000}%
\pgfsetfillcolor{currentfill}%
\pgfsetlinewidth{1.003750pt}%
\definecolor{currentstroke}{rgb}{0.000000,0.000000,0.000000}%
\pgfsetstrokecolor{currentstroke}%
\pgfsetdash{}{0pt}%
\pgfpathmoveto{\pgfqpoint{0.721249in}{0.356602in}}%
\pgfpathcurveto{\pgfqpoint{0.726774in}{0.356602in}}{\pgfqpoint{0.732073in}{0.358798in}}{\pgfqpoint{0.735980in}{0.362704in}}%
\pgfpathcurveto{\pgfqpoint{0.739887in}{0.366611in}}{\pgfqpoint{0.742082in}{0.371911in}}{\pgfqpoint{0.742082in}{0.377436in}}%
\pgfpathcurveto{\pgfqpoint{0.742082in}{0.382961in}}{\pgfqpoint{0.739887in}{0.388260in}}{\pgfqpoint{0.735980in}{0.392167in}}%
\pgfpathcurveto{\pgfqpoint{0.732073in}{0.396074in}}{\pgfqpoint{0.726774in}{0.398269in}}{\pgfqpoint{0.721249in}{0.398269in}}%
\pgfpathcurveto{\pgfqpoint{0.715724in}{0.398269in}}{\pgfqpoint{0.710424in}{0.396074in}}{\pgfqpoint{0.706518in}{0.392167in}}%
\pgfpathcurveto{\pgfqpoint{0.702611in}{0.388260in}}{\pgfqpoint{0.700416in}{0.382961in}}{\pgfqpoint{0.700416in}{0.377436in}}%
\pgfpathcurveto{\pgfqpoint{0.700416in}{0.371911in}}{\pgfqpoint{0.702611in}{0.366611in}}{\pgfqpoint{0.706518in}{0.362704in}}%
\pgfpathcurveto{\pgfqpoint{0.710424in}{0.358798in}}{\pgfqpoint{0.715724in}{0.356602in}}{\pgfqpoint{0.721249in}{0.356602in}}%
\pgfpathclose%
\pgfusepath{stroke,fill}%
\end{pgfscope}%
\begin{pgfscope}%
\pgfpathrectangle{\pgfqpoint{0.562500in}{0.275000in}}{\pgfqpoint{3.487500in}{1.925000in}}%
\pgfusepath{clip}%
\pgfsetbuttcap%
\pgfsetroundjoin%
\definecolor{currentfill}{rgb}{0.000000,0.000000,0.000000}%
\pgfsetfillcolor{currentfill}%
\pgfsetlinewidth{1.003750pt}%
\definecolor{currentstroke}{rgb}{0.000000,0.000000,0.000000}%
\pgfsetstrokecolor{currentstroke}%
\pgfsetdash{}{0pt}%
\pgfpathmoveto{\pgfqpoint{0.721249in}{0.356602in}}%
\pgfpathcurveto{\pgfqpoint{0.726774in}{0.356602in}}{\pgfqpoint{0.732073in}{0.358798in}}{\pgfqpoint{0.735980in}{0.362704in}}%
\pgfpathcurveto{\pgfqpoint{0.739887in}{0.366611in}}{\pgfqpoint{0.742082in}{0.371911in}}{\pgfqpoint{0.742082in}{0.377436in}}%
\pgfpathcurveto{\pgfqpoint{0.742082in}{0.382961in}}{\pgfqpoint{0.739887in}{0.388260in}}{\pgfqpoint{0.735980in}{0.392167in}}%
\pgfpathcurveto{\pgfqpoint{0.732073in}{0.396074in}}{\pgfqpoint{0.726774in}{0.398269in}}{\pgfqpoint{0.721249in}{0.398269in}}%
\pgfpathcurveto{\pgfqpoint{0.715724in}{0.398269in}}{\pgfqpoint{0.710424in}{0.396074in}}{\pgfqpoint{0.706518in}{0.392167in}}%
\pgfpathcurveto{\pgfqpoint{0.702611in}{0.388260in}}{\pgfqpoint{0.700416in}{0.382961in}}{\pgfqpoint{0.700416in}{0.377436in}}%
\pgfpathcurveto{\pgfqpoint{0.700416in}{0.371911in}}{\pgfqpoint{0.702611in}{0.366611in}}{\pgfqpoint{0.706518in}{0.362704in}}%
\pgfpathcurveto{\pgfqpoint{0.710424in}{0.358798in}}{\pgfqpoint{0.715724in}{0.356602in}}{\pgfqpoint{0.721249in}{0.356602in}}%
\pgfpathclose%
\pgfusepath{stroke,fill}%
\end{pgfscope}%
\begin{pgfscope}%
\pgfpathrectangle{\pgfqpoint{0.562500in}{0.275000in}}{\pgfqpoint{3.487500in}{1.925000in}}%
\pgfusepath{clip}%
\pgfsetbuttcap%
\pgfsetroundjoin%
\definecolor{currentfill}{rgb}{0.000000,0.000000,0.000000}%
\pgfsetfillcolor{currentfill}%
\pgfsetlinewidth{1.003750pt}%
\definecolor{currentstroke}{rgb}{0.000000,0.000000,0.000000}%
\pgfsetstrokecolor{currentstroke}%
\pgfsetdash{}{0pt}%
\pgfpathmoveto{\pgfqpoint{0.721249in}{0.356602in}}%
\pgfpathcurveto{\pgfqpoint{0.726774in}{0.356602in}}{\pgfqpoint{0.732073in}{0.358798in}}{\pgfqpoint{0.735980in}{0.362704in}}%
\pgfpathcurveto{\pgfqpoint{0.739887in}{0.366611in}}{\pgfqpoint{0.742082in}{0.371911in}}{\pgfqpoint{0.742082in}{0.377436in}}%
\pgfpathcurveto{\pgfqpoint{0.742082in}{0.382961in}}{\pgfqpoint{0.739887in}{0.388260in}}{\pgfqpoint{0.735980in}{0.392167in}}%
\pgfpathcurveto{\pgfqpoint{0.732073in}{0.396074in}}{\pgfqpoint{0.726774in}{0.398269in}}{\pgfqpoint{0.721249in}{0.398269in}}%
\pgfpathcurveto{\pgfqpoint{0.715724in}{0.398269in}}{\pgfqpoint{0.710424in}{0.396074in}}{\pgfqpoint{0.706518in}{0.392167in}}%
\pgfpathcurveto{\pgfqpoint{0.702611in}{0.388260in}}{\pgfqpoint{0.700416in}{0.382961in}}{\pgfqpoint{0.700416in}{0.377436in}}%
\pgfpathcurveto{\pgfqpoint{0.700416in}{0.371911in}}{\pgfqpoint{0.702611in}{0.366611in}}{\pgfqpoint{0.706518in}{0.362704in}}%
\pgfpathcurveto{\pgfqpoint{0.710424in}{0.358798in}}{\pgfqpoint{0.715724in}{0.356602in}}{\pgfqpoint{0.721249in}{0.356602in}}%
\pgfpathclose%
\pgfusepath{stroke,fill}%
\end{pgfscope}%
\begin{pgfscope}%
\pgfpathrectangle{\pgfqpoint{0.562500in}{0.275000in}}{\pgfqpoint{3.487500in}{1.925000in}}%
\pgfusepath{clip}%
\pgfsetbuttcap%
\pgfsetroundjoin%
\definecolor{currentfill}{rgb}{0.000000,0.000000,0.000000}%
\pgfsetfillcolor{currentfill}%
\pgfsetlinewidth{1.003750pt}%
\definecolor{currentstroke}{rgb}{0.000000,0.000000,0.000000}%
\pgfsetstrokecolor{currentstroke}%
\pgfsetdash{}{0pt}%
\pgfpathmoveto{\pgfqpoint{0.721249in}{1.216635in}}%
\pgfpathcurveto{\pgfqpoint{0.726774in}{1.216635in}}{\pgfqpoint{0.732073in}{1.218830in}}{\pgfqpoint{0.735980in}{1.222736in}}%
\pgfpathcurveto{\pgfqpoint{0.739887in}{1.226643in}}{\pgfqpoint{0.742082in}{1.231943in}}{\pgfqpoint{0.742082in}{1.237468in}}%
\pgfpathcurveto{\pgfqpoint{0.742082in}{1.242993in}}{\pgfqpoint{0.739887in}{1.248292in}}{\pgfqpoint{0.735980in}{1.252199in}}%
\pgfpathcurveto{\pgfqpoint{0.732073in}{1.256106in}}{\pgfqpoint{0.726774in}{1.258301in}}{\pgfqpoint{0.721249in}{1.258301in}}%
\pgfpathcurveto{\pgfqpoint{0.715724in}{1.258301in}}{\pgfqpoint{0.710424in}{1.256106in}}{\pgfqpoint{0.706518in}{1.252199in}}%
\pgfpathcurveto{\pgfqpoint{0.702611in}{1.248292in}}{\pgfqpoint{0.700416in}{1.242993in}}{\pgfqpoint{0.700416in}{1.237468in}}%
\pgfpathcurveto{\pgfqpoint{0.700416in}{1.231943in}}{\pgfqpoint{0.702611in}{1.226643in}}{\pgfqpoint{0.706518in}{1.222736in}}%
\pgfpathcurveto{\pgfqpoint{0.710424in}{1.218830in}}{\pgfqpoint{0.715724in}{1.216635in}}{\pgfqpoint{0.721249in}{1.216635in}}%
\pgfpathclose%
\pgfusepath{stroke,fill}%
\end{pgfscope}%
\begin{pgfscope}%
\pgfpathrectangle{\pgfqpoint{0.562500in}{0.275000in}}{\pgfqpoint{3.487500in}{1.925000in}}%
\pgfusepath{clip}%
\pgfsetbuttcap%
\pgfsetroundjoin%
\definecolor{currentfill}{rgb}{0.000000,0.000000,0.000000}%
\pgfsetfillcolor{currentfill}%
\pgfsetlinewidth{1.003750pt}%
\definecolor{currentstroke}{rgb}{0.000000,0.000000,0.000000}%
\pgfsetstrokecolor{currentstroke}%
\pgfsetdash{}{0pt}%
\pgfpathmoveto{\pgfqpoint{0.721249in}{0.356602in}}%
\pgfpathcurveto{\pgfqpoint{0.726774in}{0.356602in}}{\pgfqpoint{0.732073in}{0.358798in}}{\pgfqpoint{0.735980in}{0.362704in}}%
\pgfpathcurveto{\pgfqpoint{0.739887in}{0.366611in}}{\pgfqpoint{0.742082in}{0.371911in}}{\pgfqpoint{0.742082in}{0.377436in}}%
\pgfpathcurveto{\pgfqpoint{0.742082in}{0.382961in}}{\pgfqpoint{0.739887in}{0.388260in}}{\pgfqpoint{0.735980in}{0.392167in}}%
\pgfpathcurveto{\pgfqpoint{0.732073in}{0.396074in}}{\pgfqpoint{0.726774in}{0.398269in}}{\pgfqpoint{0.721249in}{0.398269in}}%
\pgfpathcurveto{\pgfqpoint{0.715724in}{0.398269in}}{\pgfqpoint{0.710424in}{0.396074in}}{\pgfqpoint{0.706518in}{0.392167in}}%
\pgfpathcurveto{\pgfqpoint{0.702611in}{0.388260in}}{\pgfqpoint{0.700416in}{0.382961in}}{\pgfqpoint{0.700416in}{0.377436in}}%
\pgfpathcurveto{\pgfqpoint{0.700416in}{0.371911in}}{\pgfqpoint{0.702611in}{0.366611in}}{\pgfqpoint{0.706518in}{0.362704in}}%
\pgfpathcurveto{\pgfqpoint{0.710424in}{0.358798in}}{\pgfqpoint{0.715724in}{0.356602in}}{\pgfqpoint{0.721249in}{0.356602in}}%
\pgfpathclose%
\pgfusepath{stroke,fill}%
\end{pgfscope}%
\begin{pgfscope}%
\pgfpathrectangle{\pgfqpoint{0.562500in}{0.275000in}}{\pgfqpoint{3.487500in}{1.925000in}}%
\pgfusepath{clip}%
\pgfsetbuttcap%
\pgfsetroundjoin%
\definecolor{currentfill}{rgb}{0.000000,0.000000,0.000000}%
\pgfsetfillcolor{currentfill}%
\pgfsetlinewidth{1.003750pt}%
\definecolor{currentstroke}{rgb}{0.000000,0.000000,0.000000}%
\pgfsetstrokecolor{currentstroke}%
\pgfsetdash{}{0pt}%
\pgfpathmoveto{\pgfqpoint{0.721249in}{1.216635in}}%
\pgfpathcurveto{\pgfqpoint{0.726774in}{1.216635in}}{\pgfqpoint{0.732073in}{1.218830in}}{\pgfqpoint{0.735980in}{1.222736in}}%
\pgfpathcurveto{\pgfqpoint{0.739887in}{1.226643in}}{\pgfqpoint{0.742082in}{1.231943in}}{\pgfqpoint{0.742082in}{1.237468in}}%
\pgfpathcurveto{\pgfqpoint{0.742082in}{1.242993in}}{\pgfqpoint{0.739887in}{1.248292in}}{\pgfqpoint{0.735980in}{1.252199in}}%
\pgfpathcurveto{\pgfqpoint{0.732073in}{1.256106in}}{\pgfqpoint{0.726774in}{1.258301in}}{\pgfqpoint{0.721249in}{1.258301in}}%
\pgfpathcurveto{\pgfqpoint{0.715724in}{1.258301in}}{\pgfqpoint{0.710424in}{1.256106in}}{\pgfqpoint{0.706518in}{1.252199in}}%
\pgfpathcurveto{\pgfqpoint{0.702611in}{1.248292in}}{\pgfqpoint{0.700416in}{1.242993in}}{\pgfqpoint{0.700416in}{1.237468in}}%
\pgfpathcurveto{\pgfqpoint{0.700416in}{1.231943in}}{\pgfqpoint{0.702611in}{1.226643in}}{\pgfqpoint{0.706518in}{1.222736in}}%
\pgfpathcurveto{\pgfqpoint{0.710424in}{1.218830in}}{\pgfqpoint{0.715724in}{1.216635in}}{\pgfqpoint{0.721249in}{1.216635in}}%
\pgfpathclose%
\pgfusepath{stroke,fill}%
\end{pgfscope}%
\begin{pgfscope}%
\pgfpathrectangle{\pgfqpoint{0.562500in}{0.275000in}}{\pgfqpoint{3.487500in}{1.925000in}}%
\pgfusepath{clip}%
\pgfsetbuttcap%
\pgfsetroundjoin%
\definecolor{currentfill}{rgb}{0.000000,0.000000,0.000000}%
\pgfsetfillcolor{currentfill}%
\pgfsetlinewidth{1.003750pt}%
\definecolor{currentstroke}{rgb}{0.000000,0.000000,0.000000}%
\pgfsetstrokecolor{currentstroke}%
\pgfsetdash{}{0pt}%
\pgfpathmoveto{\pgfqpoint{0.721249in}{0.356602in}}%
\pgfpathcurveto{\pgfqpoint{0.726774in}{0.356602in}}{\pgfqpoint{0.732073in}{0.358798in}}{\pgfqpoint{0.735980in}{0.362704in}}%
\pgfpathcurveto{\pgfqpoint{0.739887in}{0.366611in}}{\pgfqpoint{0.742082in}{0.371911in}}{\pgfqpoint{0.742082in}{0.377436in}}%
\pgfpathcurveto{\pgfqpoint{0.742082in}{0.382961in}}{\pgfqpoint{0.739887in}{0.388260in}}{\pgfqpoint{0.735980in}{0.392167in}}%
\pgfpathcurveto{\pgfqpoint{0.732073in}{0.396074in}}{\pgfqpoint{0.726774in}{0.398269in}}{\pgfqpoint{0.721249in}{0.398269in}}%
\pgfpathcurveto{\pgfqpoint{0.715724in}{0.398269in}}{\pgfqpoint{0.710424in}{0.396074in}}{\pgfqpoint{0.706518in}{0.392167in}}%
\pgfpathcurveto{\pgfqpoint{0.702611in}{0.388260in}}{\pgfqpoint{0.700416in}{0.382961in}}{\pgfqpoint{0.700416in}{0.377436in}}%
\pgfpathcurveto{\pgfqpoint{0.700416in}{0.371911in}}{\pgfqpoint{0.702611in}{0.366611in}}{\pgfqpoint{0.706518in}{0.362704in}}%
\pgfpathcurveto{\pgfqpoint{0.710424in}{0.358798in}}{\pgfqpoint{0.715724in}{0.356602in}}{\pgfqpoint{0.721249in}{0.356602in}}%
\pgfpathclose%
\pgfusepath{stroke,fill}%
\end{pgfscope}%
\begin{pgfscope}%
\pgfpathrectangle{\pgfqpoint{0.562500in}{0.275000in}}{\pgfqpoint{3.487500in}{1.925000in}}%
\pgfusepath{clip}%
\pgfsetbuttcap%
\pgfsetroundjoin%
\definecolor{currentfill}{rgb}{0.000000,0.000000,0.000000}%
\pgfsetfillcolor{currentfill}%
\pgfsetlinewidth{1.003750pt}%
\definecolor{currentstroke}{rgb}{0.000000,0.000000,0.000000}%
\pgfsetstrokecolor{currentstroke}%
\pgfsetdash{}{0pt}%
\pgfpathmoveto{\pgfqpoint{0.721249in}{0.356602in}}%
\pgfpathcurveto{\pgfqpoint{0.726774in}{0.356602in}}{\pgfqpoint{0.732073in}{0.358798in}}{\pgfqpoint{0.735980in}{0.362704in}}%
\pgfpathcurveto{\pgfqpoint{0.739887in}{0.366611in}}{\pgfqpoint{0.742082in}{0.371911in}}{\pgfqpoint{0.742082in}{0.377436in}}%
\pgfpathcurveto{\pgfqpoint{0.742082in}{0.382961in}}{\pgfqpoint{0.739887in}{0.388260in}}{\pgfqpoint{0.735980in}{0.392167in}}%
\pgfpathcurveto{\pgfqpoint{0.732073in}{0.396074in}}{\pgfqpoint{0.726774in}{0.398269in}}{\pgfqpoint{0.721249in}{0.398269in}}%
\pgfpathcurveto{\pgfqpoint{0.715724in}{0.398269in}}{\pgfqpoint{0.710424in}{0.396074in}}{\pgfqpoint{0.706518in}{0.392167in}}%
\pgfpathcurveto{\pgfqpoint{0.702611in}{0.388260in}}{\pgfqpoint{0.700416in}{0.382961in}}{\pgfqpoint{0.700416in}{0.377436in}}%
\pgfpathcurveto{\pgfqpoint{0.700416in}{0.371911in}}{\pgfqpoint{0.702611in}{0.366611in}}{\pgfqpoint{0.706518in}{0.362704in}}%
\pgfpathcurveto{\pgfqpoint{0.710424in}{0.358798in}}{\pgfqpoint{0.715724in}{0.356602in}}{\pgfqpoint{0.721249in}{0.356602in}}%
\pgfpathclose%
\pgfusepath{stroke,fill}%
\end{pgfscope}%
\begin{pgfscope}%
\pgfpathrectangle{\pgfqpoint{0.562500in}{0.275000in}}{\pgfqpoint{3.487500in}{1.925000in}}%
\pgfusepath{clip}%
\pgfsetbuttcap%
\pgfsetroundjoin%
\definecolor{currentfill}{rgb}{0.000000,0.000000,0.000000}%
\pgfsetfillcolor{currentfill}%
\pgfsetlinewidth{1.003750pt}%
\definecolor{currentstroke}{rgb}{0.000000,0.000000,0.000000}%
\pgfsetstrokecolor{currentstroke}%
\pgfsetdash{}{0pt}%
\pgfpathmoveto{\pgfqpoint{0.721249in}{0.356602in}}%
\pgfpathcurveto{\pgfqpoint{0.726774in}{0.356602in}}{\pgfqpoint{0.732073in}{0.358798in}}{\pgfqpoint{0.735980in}{0.362704in}}%
\pgfpathcurveto{\pgfqpoint{0.739887in}{0.366611in}}{\pgfqpoint{0.742082in}{0.371911in}}{\pgfqpoint{0.742082in}{0.377436in}}%
\pgfpathcurveto{\pgfqpoint{0.742082in}{0.382961in}}{\pgfqpoint{0.739887in}{0.388260in}}{\pgfqpoint{0.735980in}{0.392167in}}%
\pgfpathcurveto{\pgfqpoint{0.732073in}{0.396074in}}{\pgfqpoint{0.726774in}{0.398269in}}{\pgfqpoint{0.721249in}{0.398269in}}%
\pgfpathcurveto{\pgfqpoint{0.715724in}{0.398269in}}{\pgfqpoint{0.710424in}{0.396074in}}{\pgfqpoint{0.706518in}{0.392167in}}%
\pgfpathcurveto{\pgfqpoint{0.702611in}{0.388260in}}{\pgfqpoint{0.700416in}{0.382961in}}{\pgfqpoint{0.700416in}{0.377436in}}%
\pgfpathcurveto{\pgfqpoint{0.700416in}{0.371911in}}{\pgfqpoint{0.702611in}{0.366611in}}{\pgfqpoint{0.706518in}{0.362704in}}%
\pgfpathcurveto{\pgfqpoint{0.710424in}{0.358798in}}{\pgfqpoint{0.715724in}{0.356602in}}{\pgfqpoint{0.721249in}{0.356602in}}%
\pgfpathclose%
\pgfusepath{stroke,fill}%
\end{pgfscope}%
\begin{pgfscope}%
\pgfpathrectangle{\pgfqpoint{0.562500in}{0.275000in}}{\pgfqpoint{3.487500in}{1.925000in}}%
\pgfusepath{clip}%
\pgfsetbuttcap%
\pgfsetroundjoin%
\definecolor{currentfill}{rgb}{0.000000,0.000000,0.000000}%
\pgfsetfillcolor{currentfill}%
\pgfsetlinewidth{1.003750pt}%
\definecolor{currentstroke}{rgb}{0.000000,0.000000,0.000000}%
\pgfsetstrokecolor{currentstroke}%
\pgfsetdash{}{0pt}%
\pgfpathmoveto{\pgfqpoint{0.721249in}{0.356602in}}%
\pgfpathcurveto{\pgfqpoint{0.726774in}{0.356602in}}{\pgfqpoint{0.732073in}{0.358798in}}{\pgfqpoint{0.735980in}{0.362704in}}%
\pgfpathcurveto{\pgfqpoint{0.739887in}{0.366611in}}{\pgfqpoint{0.742082in}{0.371911in}}{\pgfqpoint{0.742082in}{0.377436in}}%
\pgfpathcurveto{\pgfqpoint{0.742082in}{0.382961in}}{\pgfqpoint{0.739887in}{0.388260in}}{\pgfqpoint{0.735980in}{0.392167in}}%
\pgfpathcurveto{\pgfqpoint{0.732073in}{0.396074in}}{\pgfqpoint{0.726774in}{0.398269in}}{\pgfqpoint{0.721249in}{0.398269in}}%
\pgfpathcurveto{\pgfqpoint{0.715724in}{0.398269in}}{\pgfqpoint{0.710424in}{0.396074in}}{\pgfqpoint{0.706518in}{0.392167in}}%
\pgfpathcurveto{\pgfqpoint{0.702611in}{0.388260in}}{\pgfqpoint{0.700416in}{0.382961in}}{\pgfqpoint{0.700416in}{0.377436in}}%
\pgfpathcurveto{\pgfqpoint{0.700416in}{0.371911in}}{\pgfqpoint{0.702611in}{0.366611in}}{\pgfqpoint{0.706518in}{0.362704in}}%
\pgfpathcurveto{\pgfqpoint{0.710424in}{0.358798in}}{\pgfqpoint{0.715724in}{0.356602in}}{\pgfqpoint{0.721249in}{0.356602in}}%
\pgfpathclose%
\pgfusepath{stroke,fill}%
\end{pgfscope}%
\begin{pgfscope}%
\pgfpathrectangle{\pgfqpoint{0.562500in}{0.275000in}}{\pgfqpoint{3.487500in}{1.925000in}}%
\pgfusepath{clip}%
\pgfsetbuttcap%
\pgfsetroundjoin%
\definecolor{currentfill}{rgb}{0.000000,0.000000,0.000000}%
\pgfsetfillcolor{currentfill}%
\pgfsetlinewidth{1.003750pt}%
\definecolor{currentstroke}{rgb}{0.000000,0.000000,0.000000}%
\pgfsetstrokecolor{currentstroke}%
\pgfsetdash{}{0pt}%
\pgfpathmoveto{\pgfqpoint{0.721249in}{0.356602in}}%
\pgfpathcurveto{\pgfqpoint{0.726774in}{0.356602in}}{\pgfqpoint{0.732073in}{0.358798in}}{\pgfqpoint{0.735980in}{0.362704in}}%
\pgfpathcurveto{\pgfqpoint{0.739887in}{0.366611in}}{\pgfqpoint{0.742082in}{0.371911in}}{\pgfqpoint{0.742082in}{0.377436in}}%
\pgfpathcurveto{\pgfqpoint{0.742082in}{0.382961in}}{\pgfqpoint{0.739887in}{0.388260in}}{\pgfqpoint{0.735980in}{0.392167in}}%
\pgfpathcurveto{\pgfqpoint{0.732073in}{0.396074in}}{\pgfqpoint{0.726774in}{0.398269in}}{\pgfqpoint{0.721249in}{0.398269in}}%
\pgfpathcurveto{\pgfqpoint{0.715724in}{0.398269in}}{\pgfqpoint{0.710424in}{0.396074in}}{\pgfqpoint{0.706518in}{0.392167in}}%
\pgfpathcurveto{\pgfqpoint{0.702611in}{0.388260in}}{\pgfqpoint{0.700416in}{0.382961in}}{\pgfqpoint{0.700416in}{0.377436in}}%
\pgfpathcurveto{\pgfqpoint{0.700416in}{0.371911in}}{\pgfqpoint{0.702611in}{0.366611in}}{\pgfqpoint{0.706518in}{0.362704in}}%
\pgfpathcurveto{\pgfqpoint{0.710424in}{0.358798in}}{\pgfqpoint{0.715724in}{0.356602in}}{\pgfqpoint{0.721249in}{0.356602in}}%
\pgfpathclose%
\pgfusepath{stroke,fill}%
\end{pgfscope}%
\begin{pgfscope}%
\pgfpathrectangle{\pgfqpoint{0.562500in}{0.275000in}}{\pgfqpoint{3.487500in}{1.925000in}}%
\pgfusepath{clip}%
\pgfsetbuttcap%
\pgfsetroundjoin%
\definecolor{currentfill}{rgb}{0.000000,0.000000,0.000000}%
\pgfsetfillcolor{currentfill}%
\pgfsetlinewidth{1.003750pt}%
\definecolor{currentstroke}{rgb}{0.000000,0.000000,0.000000}%
\pgfsetstrokecolor{currentstroke}%
\pgfsetdash{}{0pt}%
\pgfpathmoveto{\pgfqpoint{0.721249in}{0.356602in}}%
\pgfpathcurveto{\pgfqpoint{0.726774in}{0.356602in}}{\pgfqpoint{0.732073in}{0.358798in}}{\pgfqpoint{0.735980in}{0.362704in}}%
\pgfpathcurveto{\pgfqpoint{0.739887in}{0.366611in}}{\pgfqpoint{0.742082in}{0.371911in}}{\pgfqpoint{0.742082in}{0.377436in}}%
\pgfpathcurveto{\pgfqpoint{0.742082in}{0.382961in}}{\pgfqpoint{0.739887in}{0.388260in}}{\pgfqpoint{0.735980in}{0.392167in}}%
\pgfpathcurveto{\pgfqpoint{0.732073in}{0.396074in}}{\pgfqpoint{0.726774in}{0.398269in}}{\pgfqpoint{0.721249in}{0.398269in}}%
\pgfpathcurveto{\pgfqpoint{0.715724in}{0.398269in}}{\pgfqpoint{0.710424in}{0.396074in}}{\pgfqpoint{0.706518in}{0.392167in}}%
\pgfpathcurveto{\pgfqpoint{0.702611in}{0.388260in}}{\pgfqpoint{0.700416in}{0.382961in}}{\pgfqpoint{0.700416in}{0.377436in}}%
\pgfpathcurveto{\pgfqpoint{0.700416in}{0.371911in}}{\pgfqpoint{0.702611in}{0.366611in}}{\pgfqpoint{0.706518in}{0.362704in}}%
\pgfpathcurveto{\pgfqpoint{0.710424in}{0.358798in}}{\pgfqpoint{0.715724in}{0.356602in}}{\pgfqpoint{0.721249in}{0.356602in}}%
\pgfpathclose%
\pgfusepath{stroke,fill}%
\end{pgfscope}%
\begin{pgfscope}%
\pgfpathrectangle{\pgfqpoint{0.562500in}{0.275000in}}{\pgfqpoint{3.487500in}{1.925000in}}%
\pgfusepath{clip}%
\pgfsetbuttcap%
\pgfsetroundjoin%
\definecolor{currentfill}{rgb}{0.000000,0.000000,0.000000}%
\pgfsetfillcolor{currentfill}%
\pgfsetlinewidth{1.003750pt}%
\definecolor{currentstroke}{rgb}{0.000000,0.000000,0.000000}%
\pgfsetstrokecolor{currentstroke}%
\pgfsetdash{}{0pt}%
\pgfpathmoveto{\pgfqpoint{0.721249in}{0.356602in}}%
\pgfpathcurveto{\pgfqpoint{0.726774in}{0.356602in}}{\pgfqpoint{0.732073in}{0.358798in}}{\pgfqpoint{0.735980in}{0.362704in}}%
\pgfpathcurveto{\pgfqpoint{0.739887in}{0.366611in}}{\pgfqpoint{0.742082in}{0.371911in}}{\pgfqpoint{0.742082in}{0.377436in}}%
\pgfpathcurveto{\pgfqpoint{0.742082in}{0.382961in}}{\pgfqpoint{0.739887in}{0.388260in}}{\pgfqpoint{0.735980in}{0.392167in}}%
\pgfpathcurveto{\pgfqpoint{0.732073in}{0.396074in}}{\pgfqpoint{0.726774in}{0.398269in}}{\pgfqpoint{0.721249in}{0.398269in}}%
\pgfpathcurveto{\pgfqpoint{0.715724in}{0.398269in}}{\pgfqpoint{0.710424in}{0.396074in}}{\pgfqpoint{0.706518in}{0.392167in}}%
\pgfpathcurveto{\pgfqpoint{0.702611in}{0.388260in}}{\pgfqpoint{0.700416in}{0.382961in}}{\pgfqpoint{0.700416in}{0.377436in}}%
\pgfpathcurveto{\pgfqpoint{0.700416in}{0.371911in}}{\pgfqpoint{0.702611in}{0.366611in}}{\pgfqpoint{0.706518in}{0.362704in}}%
\pgfpathcurveto{\pgfqpoint{0.710424in}{0.358798in}}{\pgfqpoint{0.715724in}{0.356602in}}{\pgfqpoint{0.721249in}{0.356602in}}%
\pgfpathclose%
\pgfusepath{stroke,fill}%
\end{pgfscope}%
\begin{pgfscope}%
\pgfpathrectangle{\pgfqpoint{0.562500in}{0.275000in}}{\pgfqpoint{3.487500in}{1.925000in}}%
\pgfusepath{clip}%
\pgfsetbuttcap%
\pgfsetroundjoin%
\definecolor{currentfill}{rgb}{0.000000,0.000000,0.000000}%
\pgfsetfillcolor{currentfill}%
\pgfsetlinewidth{1.003750pt}%
\definecolor{currentstroke}{rgb}{0.000000,0.000000,0.000000}%
\pgfsetstrokecolor{currentstroke}%
\pgfsetdash{}{0pt}%
\pgfpathmoveto{\pgfqpoint{0.721249in}{0.356602in}}%
\pgfpathcurveto{\pgfqpoint{0.726774in}{0.356602in}}{\pgfqpoint{0.732073in}{0.358798in}}{\pgfqpoint{0.735980in}{0.362704in}}%
\pgfpathcurveto{\pgfqpoint{0.739887in}{0.366611in}}{\pgfqpoint{0.742082in}{0.371911in}}{\pgfqpoint{0.742082in}{0.377436in}}%
\pgfpathcurveto{\pgfqpoint{0.742082in}{0.382961in}}{\pgfqpoint{0.739887in}{0.388260in}}{\pgfqpoint{0.735980in}{0.392167in}}%
\pgfpathcurveto{\pgfqpoint{0.732073in}{0.396074in}}{\pgfqpoint{0.726774in}{0.398269in}}{\pgfqpoint{0.721249in}{0.398269in}}%
\pgfpathcurveto{\pgfqpoint{0.715724in}{0.398269in}}{\pgfqpoint{0.710424in}{0.396074in}}{\pgfqpoint{0.706518in}{0.392167in}}%
\pgfpathcurveto{\pgfqpoint{0.702611in}{0.388260in}}{\pgfqpoint{0.700416in}{0.382961in}}{\pgfqpoint{0.700416in}{0.377436in}}%
\pgfpathcurveto{\pgfqpoint{0.700416in}{0.371911in}}{\pgfqpoint{0.702611in}{0.366611in}}{\pgfqpoint{0.706518in}{0.362704in}}%
\pgfpathcurveto{\pgfqpoint{0.710424in}{0.358798in}}{\pgfqpoint{0.715724in}{0.356602in}}{\pgfqpoint{0.721249in}{0.356602in}}%
\pgfpathclose%
\pgfusepath{stroke,fill}%
\end{pgfscope}%
\begin{pgfscope}%
\pgfpathrectangle{\pgfqpoint{0.562500in}{0.275000in}}{\pgfqpoint{3.487500in}{1.925000in}}%
\pgfusepath{clip}%
\pgfsetbuttcap%
\pgfsetroundjoin%
\definecolor{currentfill}{rgb}{0.000000,0.000000,0.000000}%
\pgfsetfillcolor{currentfill}%
\pgfsetlinewidth{1.003750pt}%
\definecolor{currentstroke}{rgb}{0.000000,0.000000,0.000000}%
\pgfsetstrokecolor{currentstroke}%
\pgfsetdash{}{0pt}%
\pgfpathmoveto{\pgfqpoint{0.721249in}{0.356602in}}%
\pgfpathcurveto{\pgfqpoint{0.726774in}{0.356602in}}{\pgfqpoint{0.732073in}{0.358798in}}{\pgfqpoint{0.735980in}{0.362704in}}%
\pgfpathcurveto{\pgfqpoint{0.739887in}{0.366611in}}{\pgfqpoint{0.742082in}{0.371911in}}{\pgfqpoint{0.742082in}{0.377436in}}%
\pgfpathcurveto{\pgfqpoint{0.742082in}{0.382961in}}{\pgfqpoint{0.739887in}{0.388260in}}{\pgfqpoint{0.735980in}{0.392167in}}%
\pgfpathcurveto{\pgfqpoint{0.732073in}{0.396074in}}{\pgfqpoint{0.726774in}{0.398269in}}{\pgfqpoint{0.721249in}{0.398269in}}%
\pgfpathcurveto{\pgfqpoint{0.715724in}{0.398269in}}{\pgfqpoint{0.710424in}{0.396074in}}{\pgfqpoint{0.706518in}{0.392167in}}%
\pgfpathcurveto{\pgfqpoint{0.702611in}{0.388260in}}{\pgfqpoint{0.700416in}{0.382961in}}{\pgfqpoint{0.700416in}{0.377436in}}%
\pgfpathcurveto{\pgfqpoint{0.700416in}{0.371911in}}{\pgfqpoint{0.702611in}{0.366611in}}{\pgfqpoint{0.706518in}{0.362704in}}%
\pgfpathcurveto{\pgfqpoint{0.710424in}{0.358798in}}{\pgfqpoint{0.715724in}{0.356602in}}{\pgfqpoint{0.721249in}{0.356602in}}%
\pgfpathclose%
\pgfusepath{stroke,fill}%
\end{pgfscope}%
\begin{pgfscope}%
\pgfpathrectangle{\pgfqpoint{0.562500in}{0.275000in}}{\pgfqpoint{3.487500in}{1.925000in}}%
\pgfusepath{clip}%
\pgfsetbuttcap%
\pgfsetroundjoin%
\definecolor{currentfill}{rgb}{0.000000,0.000000,0.000000}%
\pgfsetfillcolor{currentfill}%
\pgfsetlinewidth{1.003750pt}%
\definecolor{currentstroke}{rgb}{0.000000,0.000000,0.000000}%
\pgfsetstrokecolor{currentstroke}%
\pgfsetdash{}{0pt}%
\pgfpathmoveto{\pgfqpoint{0.721249in}{0.356602in}}%
\pgfpathcurveto{\pgfqpoint{0.726774in}{0.356602in}}{\pgfqpoint{0.732073in}{0.358798in}}{\pgfqpoint{0.735980in}{0.362704in}}%
\pgfpathcurveto{\pgfqpoint{0.739887in}{0.366611in}}{\pgfqpoint{0.742082in}{0.371911in}}{\pgfqpoint{0.742082in}{0.377436in}}%
\pgfpathcurveto{\pgfqpoint{0.742082in}{0.382961in}}{\pgfqpoint{0.739887in}{0.388260in}}{\pgfqpoint{0.735980in}{0.392167in}}%
\pgfpathcurveto{\pgfqpoint{0.732073in}{0.396074in}}{\pgfqpoint{0.726774in}{0.398269in}}{\pgfqpoint{0.721249in}{0.398269in}}%
\pgfpathcurveto{\pgfqpoint{0.715724in}{0.398269in}}{\pgfqpoint{0.710424in}{0.396074in}}{\pgfqpoint{0.706518in}{0.392167in}}%
\pgfpathcurveto{\pgfqpoint{0.702611in}{0.388260in}}{\pgfqpoint{0.700416in}{0.382961in}}{\pgfqpoint{0.700416in}{0.377436in}}%
\pgfpathcurveto{\pgfqpoint{0.700416in}{0.371911in}}{\pgfqpoint{0.702611in}{0.366611in}}{\pgfqpoint{0.706518in}{0.362704in}}%
\pgfpathcurveto{\pgfqpoint{0.710424in}{0.358798in}}{\pgfqpoint{0.715724in}{0.356602in}}{\pgfqpoint{0.721249in}{0.356602in}}%
\pgfpathclose%
\pgfusepath{stroke,fill}%
\end{pgfscope}%
\begin{pgfscope}%
\pgfpathrectangle{\pgfqpoint{0.562500in}{0.275000in}}{\pgfqpoint{3.487500in}{1.925000in}}%
\pgfusepath{clip}%
\pgfsetbuttcap%
\pgfsetroundjoin%
\definecolor{currentfill}{rgb}{0.000000,0.000000,0.000000}%
\pgfsetfillcolor{currentfill}%
\pgfsetlinewidth{1.003750pt}%
\definecolor{currentstroke}{rgb}{0.000000,0.000000,0.000000}%
\pgfsetstrokecolor{currentstroke}%
\pgfsetdash{}{0pt}%
\pgfpathmoveto{\pgfqpoint{0.721249in}{0.356602in}}%
\pgfpathcurveto{\pgfqpoint{0.726774in}{0.356602in}}{\pgfqpoint{0.732073in}{0.358798in}}{\pgfqpoint{0.735980in}{0.362704in}}%
\pgfpathcurveto{\pgfqpoint{0.739887in}{0.366611in}}{\pgfqpoint{0.742082in}{0.371911in}}{\pgfqpoint{0.742082in}{0.377436in}}%
\pgfpathcurveto{\pgfqpoint{0.742082in}{0.382961in}}{\pgfqpoint{0.739887in}{0.388260in}}{\pgfqpoint{0.735980in}{0.392167in}}%
\pgfpathcurveto{\pgfqpoint{0.732073in}{0.396074in}}{\pgfqpoint{0.726774in}{0.398269in}}{\pgfqpoint{0.721249in}{0.398269in}}%
\pgfpathcurveto{\pgfqpoint{0.715724in}{0.398269in}}{\pgfqpoint{0.710424in}{0.396074in}}{\pgfqpoint{0.706518in}{0.392167in}}%
\pgfpathcurveto{\pgfqpoint{0.702611in}{0.388260in}}{\pgfqpoint{0.700416in}{0.382961in}}{\pgfqpoint{0.700416in}{0.377436in}}%
\pgfpathcurveto{\pgfqpoint{0.700416in}{0.371911in}}{\pgfqpoint{0.702611in}{0.366611in}}{\pgfqpoint{0.706518in}{0.362704in}}%
\pgfpathcurveto{\pgfqpoint{0.710424in}{0.358798in}}{\pgfqpoint{0.715724in}{0.356602in}}{\pgfqpoint{0.721249in}{0.356602in}}%
\pgfpathclose%
\pgfusepath{stroke,fill}%
\end{pgfscope}%
\begin{pgfscope}%
\pgfpathrectangle{\pgfqpoint{0.562500in}{0.275000in}}{\pgfqpoint{3.487500in}{1.925000in}}%
\pgfusepath{clip}%
\pgfsetbuttcap%
\pgfsetroundjoin%
\definecolor{currentfill}{rgb}{0.000000,0.000000,0.000000}%
\pgfsetfillcolor{currentfill}%
\pgfsetlinewidth{1.003750pt}%
\definecolor{currentstroke}{rgb}{0.000000,0.000000,0.000000}%
\pgfsetstrokecolor{currentstroke}%
\pgfsetdash{}{0pt}%
\pgfpathmoveto{\pgfqpoint{1.772992in}{1.216635in}}%
\pgfpathcurveto{\pgfqpoint{1.778517in}{1.216635in}}{\pgfqpoint{1.783816in}{1.218830in}}{\pgfqpoint{1.787723in}{1.222736in}}%
\pgfpathcurveto{\pgfqpoint{1.791630in}{1.226643in}}{\pgfqpoint{1.793825in}{1.231943in}}{\pgfqpoint{1.793825in}{1.237468in}}%
\pgfpathcurveto{\pgfqpoint{1.793825in}{1.242993in}}{\pgfqpoint{1.791630in}{1.248292in}}{\pgfqpoint{1.787723in}{1.252199in}}%
\pgfpathcurveto{\pgfqpoint{1.783816in}{1.256106in}}{\pgfqpoint{1.778517in}{1.258301in}}{\pgfqpoint{1.772992in}{1.258301in}}%
\pgfpathcurveto{\pgfqpoint{1.767467in}{1.258301in}}{\pgfqpoint{1.762167in}{1.256106in}}{\pgfqpoint{1.758260in}{1.252199in}}%
\pgfpathcurveto{\pgfqpoint{1.754353in}{1.248292in}}{\pgfqpoint{1.752158in}{1.242993in}}{\pgfqpoint{1.752158in}{1.237468in}}%
\pgfpathcurveto{\pgfqpoint{1.752158in}{1.231943in}}{\pgfqpoint{1.754353in}{1.226643in}}{\pgfqpoint{1.758260in}{1.222736in}}%
\pgfpathcurveto{\pgfqpoint{1.762167in}{1.218830in}}{\pgfqpoint{1.767467in}{1.216635in}}{\pgfqpoint{1.772992in}{1.216635in}}%
\pgfpathclose%
\pgfusepath{stroke,fill}%
\end{pgfscope}%
\begin{pgfscope}%
\pgfpathrectangle{\pgfqpoint{0.562500in}{0.275000in}}{\pgfqpoint{3.487500in}{1.925000in}}%
\pgfusepath{clip}%
\pgfsetbuttcap%
\pgfsetroundjoin%
\definecolor{currentfill}{rgb}{0.000000,0.000000,0.000000}%
\pgfsetfillcolor{currentfill}%
\pgfsetlinewidth{1.003750pt}%
\definecolor{currentstroke}{rgb}{0.000000,0.000000,0.000000}%
\pgfsetstrokecolor{currentstroke}%
\pgfsetdash{}{0pt}%
\pgfpathmoveto{\pgfqpoint{1.772992in}{1.216635in}}%
\pgfpathcurveto{\pgfqpoint{1.778517in}{1.216635in}}{\pgfqpoint{1.783816in}{1.218830in}}{\pgfqpoint{1.787723in}{1.222736in}}%
\pgfpathcurveto{\pgfqpoint{1.791630in}{1.226643in}}{\pgfqpoint{1.793825in}{1.231943in}}{\pgfqpoint{1.793825in}{1.237468in}}%
\pgfpathcurveto{\pgfqpoint{1.793825in}{1.242993in}}{\pgfqpoint{1.791630in}{1.248292in}}{\pgfqpoint{1.787723in}{1.252199in}}%
\pgfpathcurveto{\pgfqpoint{1.783816in}{1.256106in}}{\pgfqpoint{1.778517in}{1.258301in}}{\pgfqpoint{1.772992in}{1.258301in}}%
\pgfpathcurveto{\pgfqpoint{1.767467in}{1.258301in}}{\pgfqpoint{1.762167in}{1.256106in}}{\pgfqpoint{1.758260in}{1.252199in}}%
\pgfpathcurveto{\pgfqpoint{1.754353in}{1.248292in}}{\pgfqpoint{1.752158in}{1.242993in}}{\pgfqpoint{1.752158in}{1.237468in}}%
\pgfpathcurveto{\pgfqpoint{1.752158in}{1.231943in}}{\pgfqpoint{1.754353in}{1.226643in}}{\pgfqpoint{1.758260in}{1.222736in}}%
\pgfpathcurveto{\pgfqpoint{1.762167in}{1.218830in}}{\pgfqpoint{1.767467in}{1.216635in}}{\pgfqpoint{1.772992in}{1.216635in}}%
\pgfpathclose%
\pgfusepath{stroke,fill}%
\end{pgfscope}%
\begin{pgfscope}%
\pgfpathrectangle{\pgfqpoint{0.562500in}{0.275000in}}{\pgfqpoint{3.487500in}{1.925000in}}%
\pgfusepath{clip}%
\pgfsetbuttcap%
\pgfsetroundjoin%
\definecolor{currentfill}{rgb}{0.000000,0.000000,0.000000}%
\pgfsetfillcolor{currentfill}%
\pgfsetlinewidth{1.003750pt}%
\definecolor{currentstroke}{rgb}{0.000000,0.000000,0.000000}%
\pgfsetstrokecolor{currentstroke}%
\pgfsetdash{}{0pt}%
\pgfpathmoveto{\pgfqpoint{1.772992in}{1.216635in}}%
\pgfpathcurveto{\pgfqpoint{1.778517in}{1.216635in}}{\pgfqpoint{1.783816in}{1.218830in}}{\pgfqpoint{1.787723in}{1.222736in}}%
\pgfpathcurveto{\pgfqpoint{1.791630in}{1.226643in}}{\pgfqpoint{1.793825in}{1.231943in}}{\pgfqpoint{1.793825in}{1.237468in}}%
\pgfpathcurveto{\pgfqpoint{1.793825in}{1.242993in}}{\pgfqpoint{1.791630in}{1.248292in}}{\pgfqpoint{1.787723in}{1.252199in}}%
\pgfpathcurveto{\pgfqpoint{1.783816in}{1.256106in}}{\pgfqpoint{1.778517in}{1.258301in}}{\pgfqpoint{1.772992in}{1.258301in}}%
\pgfpathcurveto{\pgfqpoint{1.767467in}{1.258301in}}{\pgfqpoint{1.762167in}{1.256106in}}{\pgfqpoint{1.758260in}{1.252199in}}%
\pgfpathcurveto{\pgfqpoint{1.754353in}{1.248292in}}{\pgfqpoint{1.752158in}{1.242993in}}{\pgfqpoint{1.752158in}{1.237468in}}%
\pgfpathcurveto{\pgfqpoint{1.752158in}{1.231943in}}{\pgfqpoint{1.754353in}{1.226643in}}{\pgfqpoint{1.758260in}{1.222736in}}%
\pgfpathcurveto{\pgfqpoint{1.762167in}{1.218830in}}{\pgfqpoint{1.767467in}{1.216635in}}{\pgfqpoint{1.772992in}{1.216635in}}%
\pgfpathclose%
\pgfusepath{stroke,fill}%
\end{pgfscope}%
\begin{pgfscope}%
\pgfpathrectangle{\pgfqpoint{0.562500in}{0.275000in}}{\pgfqpoint{3.487500in}{1.925000in}}%
\pgfusepath{clip}%
\pgfsetbuttcap%
\pgfsetroundjoin%
\definecolor{currentfill}{rgb}{0.000000,0.000000,0.000000}%
\pgfsetfillcolor{currentfill}%
\pgfsetlinewidth{1.003750pt}%
\definecolor{currentstroke}{rgb}{0.000000,0.000000,0.000000}%
\pgfsetstrokecolor{currentstroke}%
\pgfsetdash{}{0pt}%
\pgfpathmoveto{\pgfqpoint{1.772992in}{1.216635in}}%
\pgfpathcurveto{\pgfqpoint{1.778517in}{1.216635in}}{\pgfqpoint{1.783816in}{1.218830in}}{\pgfqpoint{1.787723in}{1.222736in}}%
\pgfpathcurveto{\pgfqpoint{1.791630in}{1.226643in}}{\pgfqpoint{1.793825in}{1.231943in}}{\pgfqpoint{1.793825in}{1.237468in}}%
\pgfpathcurveto{\pgfqpoint{1.793825in}{1.242993in}}{\pgfqpoint{1.791630in}{1.248292in}}{\pgfqpoint{1.787723in}{1.252199in}}%
\pgfpathcurveto{\pgfqpoint{1.783816in}{1.256106in}}{\pgfqpoint{1.778517in}{1.258301in}}{\pgfqpoint{1.772992in}{1.258301in}}%
\pgfpathcurveto{\pgfqpoint{1.767467in}{1.258301in}}{\pgfqpoint{1.762167in}{1.256106in}}{\pgfqpoint{1.758260in}{1.252199in}}%
\pgfpathcurveto{\pgfqpoint{1.754353in}{1.248292in}}{\pgfqpoint{1.752158in}{1.242993in}}{\pgfqpoint{1.752158in}{1.237468in}}%
\pgfpathcurveto{\pgfqpoint{1.752158in}{1.231943in}}{\pgfqpoint{1.754353in}{1.226643in}}{\pgfqpoint{1.758260in}{1.222736in}}%
\pgfpathcurveto{\pgfqpoint{1.762167in}{1.218830in}}{\pgfqpoint{1.767467in}{1.216635in}}{\pgfqpoint{1.772992in}{1.216635in}}%
\pgfpathclose%
\pgfusepath{stroke,fill}%
\end{pgfscope}%
\begin{pgfscope}%
\pgfpathrectangle{\pgfqpoint{0.562500in}{0.275000in}}{\pgfqpoint{3.487500in}{1.925000in}}%
\pgfusepath{clip}%
\pgfsetbuttcap%
\pgfsetroundjoin%
\definecolor{currentfill}{rgb}{0.000000,0.000000,0.000000}%
\pgfsetfillcolor{currentfill}%
\pgfsetlinewidth{1.003750pt}%
\definecolor{currentstroke}{rgb}{0.000000,0.000000,0.000000}%
\pgfsetstrokecolor{currentstroke}%
\pgfsetdash{}{0pt}%
\pgfpathmoveto{\pgfqpoint{1.772992in}{1.216635in}}%
\pgfpathcurveto{\pgfqpoint{1.778517in}{1.216635in}}{\pgfqpoint{1.783816in}{1.218830in}}{\pgfqpoint{1.787723in}{1.222736in}}%
\pgfpathcurveto{\pgfqpoint{1.791630in}{1.226643in}}{\pgfqpoint{1.793825in}{1.231943in}}{\pgfqpoint{1.793825in}{1.237468in}}%
\pgfpathcurveto{\pgfqpoint{1.793825in}{1.242993in}}{\pgfqpoint{1.791630in}{1.248292in}}{\pgfqpoint{1.787723in}{1.252199in}}%
\pgfpathcurveto{\pgfqpoint{1.783816in}{1.256106in}}{\pgfqpoint{1.778517in}{1.258301in}}{\pgfqpoint{1.772992in}{1.258301in}}%
\pgfpathcurveto{\pgfqpoint{1.767467in}{1.258301in}}{\pgfqpoint{1.762167in}{1.256106in}}{\pgfqpoint{1.758260in}{1.252199in}}%
\pgfpathcurveto{\pgfqpoint{1.754353in}{1.248292in}}{\pgfqpoint{1.752158in}{1.242993in}}{\pgfqpoint{1.752158in}{1.237468in}}%
\pgfpathcurveto{\pgfqpoint{1.752158in}{1.231943in}}{\pgfqpoint{1.754353in}{1.226643in}}{\pgfqpoint{1.758260in}{1.222736in}}%
\pgfpathcurveto{\pgfqpoint{1.762167in}{1.218830in}}{\pgfqpoint{1.767467in}{1.216635in}}{\pgfqpoint{1.772992in}{1.216635in}}%
\pgfpathclose%
\pgfusepath{stroke,fill}%
\end{pgfscope}%
\begin{pgfscope}%
\pgfpathrectangle{\pgfqpoint{0.562500in}{0.275000in}}{\pgfqpoint{3.487500in}{1.925000in}}%
\pgfusepath{clip}%
\pgfsetbuttcap%
\pgfsetroundjoin%
\definecolor{currentfill}{rgb}{0.000000,0.000000,0.000000}%
\pgfsetfillcolor{currentfill}%
\pgfsetlinewidth{1.003750pt}%
\definecolor{currentstroke}{rgb}{0.000000,0.000000,0.000000}%
\pgfsetstrokecolor{currentstroke}%
\pgfsetdash{}{0pt}%
\pgfpathmoveto{\pgfqpoint{1.772992in}{1.216635in}}%
\pgfpathcurveto{\pgfqpoint{1.778517in}{1.216635in}}{\pgfqpoint{1.783816in}{1.218830in}}{\pgfqpoint{1.787723in}{1.222736in}}%
\pgfpathcurveto{\pgfqpoint{1.791630in}{1.226643in}}{\pgfqpoint{1.793825in}{1.231943in}}{\pgfqpoint{1.793825in}{1.237468in}}%
\pgfpathcurveto{\pgfqpoint{1.793825in}{1.242993in}}{\pgfqpoint{1.791630in}{1.248292in}}{\pgfqpoint{1.787723in}{1.252199in}}%
\pgfpathcurveto{\pgfqpoint{1.783816in}{1.256106in}}{\pgfqpoint{1.778517in}{1.258301in}}{\pgfqpoint{1.772992in}{1.258301in}}%
\pgfpathcurveto{\pgfqpoint{1.767467in}{1.258301in}}{\pgfqpoint{1.762167in}{1.256106in}}{\pgfqpoint{1.758260in}{1.252199in}}%
\pgfpathcurveto{\pgfqpoint{1.754353in}{1.248292in}}{\pgfqpoint{1.752158in}{1.242993in}}{\pgfqpoint{1.752158in}{1.237468in}}%
\pgfpathcurveto{\pgfqpoint{1.752158in}{1.231943in}}{\pgfqpoint{1.754353in}{1.226643in}}{\pgfqpoint{1.758260in}{1.222736in}}%
\pgfpathcurveto{\pgfqpoint{1.762167in}{1.218830in}}{\pgfqpoint{1.767467in}{1.216635in}}{\pgfqpoint{1.772992in}{1.216635in}}%
\pgfpathclose%
\pgfusepath{stroke,fill}%
\end{pgfscope}%
\begin{pgfscope}%
\pgfpathrectangle{\pgfqpoint{0.562500in}{0.275000in}}{\pgfqpoint{3.487500in}{1.925000in}}%
\pgfusepath{clip}%
\pgfsetbuttcap%
\pgfsetroundjoin%
\definecolor{currentfill}{rgb}{0.000000,0.000000,0.000000}%
\pgfsetfillcolor{currentfill}%
\pgfsetlinewidth{1.003750pt}%
\definecolor{currentstroke}{rgb}{0.000000,0.000000,0.000000}%
\pgfsetstrokecolor{currentstroke}%
\pgfsetdash{}{0pt}%
\pgfpathmoveto{\pgfqpoint{1.772992in}{1.216635in}}%
\pgfpathcurveto{\pgfqpoint{1.778517in}{1.216635in}}{\pgfqpoint{1.783816in}{1.218830in}}{\pgfqpoint{1.787723in}{1.222736in}}%
\pgfpathcurveto{\pgfqpoint{1.791630in}{1.226643in}}{\pgfqpoint{1.793825in}{1.231943in}}{\pgfqpoint{1.793825in}{1.237468in}}%
\pgfpathcurveto{\pgfqpoint{1.793825in}{1.242993in}}{\pgfqpoint{1.791630in}{1.248292in}}{\pgfqpoint{1.787723in}{1.252199in}}%
\pgfpathcurveto{\pgfqpoint{1.783816in}{1.256106in}}{\pgfqpoint{1.778517in}{1.258301in}}{\pgfqpoint{1.772992in}{1.258301in}}%
\pgfpathcurveto{\pgfqpoint{1.767467in}{1.258301in}}{\pgfqpoint{1.762167in}{1.256106in}}{\pgfqpoint{1.758260in}{1.252199in}}%
\pgfpathcurveto{\pgfqpoint{1.754353in}{1.248292in}}{\pgfqpoint{1.752158in}{1.242993in}}{\pgfqpoint{1.752158in}{1.237468in}}%
\pgfpathcurveto{\pgfqpoint{1.752158in}{1.231943in}}{\pgfqpoint{1.754353in}{1.226643in}}{\pgfqpoint{1.758260in}{1.222736in}}%
\pgfpathcurveto{\pgfqpoint{1.762167in}{1.218830in}}{\pgfqpoint{1.767467in}{1.216635in}}{\pgfqpoint{1.772992in}{1.216635in}}%
\pgfpathclose%
\pgfusepath{stroke,fill}%
\end{pgfscope}%
\begin{pgfscope}%
\pgfpathrectangle{\pgfqpoint{0.562500in}{0.275000in}}{\pgfqpoint{3.487500in}{1.925000in}}%
\pgfusepath{clip}%
\pgfsetbuttcap%
\pgfsetroundjoin%
\definecolor{currentfill}{rgb}{0.000000,0.000000,0.000000}%
\pgfsetfillcolor{currentfill}%
\pgfsetlinewidth{1.003750pt}%
\definecolor{currentstroke}{rgb}{0.000000,0.000000,0.000000}%
\pgfsetstrokecolor{currentstroke}%
\pgfsetdash{}{0pt}%
\pgfpathmoveto{\pgfqpoint{1.772992in}{1.216635in}}%
\pgfpathcurveto{\pgfqpoint{1.778517in}{1.216635in}}{\pgfqpoint{1.783816in}{1.218830in}}{\pgfqpoint{1.787723in}{1.222736in}}%
\pgfpathcurveto{\pgfqpoint{1.791630in}{1.226643in}}{\pgfqpoint{1.793825in}{1.231943in}}{\pgfqpoint{1.793825in}{1.237468in}}%
\pgfpathcurveto{\pgfqpoint{1.793825in}{1.242993in}}{\pgfqpoint{1.791630in}{1.248292in}}{\pgfqpoint{1.787723in}{1.252199in}}%
\pgfpathcurveto{\pgfqpoint{1.783816in}{1.256106in}}{\pgfqpoint{1.778517in}{1.258301in}}{\pgfqpoint{1.772992in}{1.258301in}}%
\pgfpathcurveto{\pgfqpoint{1.767467in}{1.258301in}}{\pgfqpoint{1.762167in}{1.256106in}}{\pgfqpoint{1.758260in}{1.252199in}}%
\pgfpathcurveto{\pgfqpoint{1.754353in}{1.248292in}}{\pgfqpoint{1.752158in}{1.242993in}}{\pgfqpoint{1.752158in}{1.237468in}}%
\pgfpathcurveto{\pgfqpoint{1.752158in}{1.231943in}}{\pgfqpoint{1.754353in}{1.226643in}}{\pgfqpoint{1.758260in}{1.222736in}}%
\pgfpathcurveto{\pgfqpoint{1.762167in}{1.218830in}}{\pgfqpoint{1.767467in}{1.216635in}}{\pgfqpoint{1.772992in}{1.216635in}}%
\pgfpathclose%
\pgfusepath{stroke,fill}%
\end{pgfscope}%
\begin{pgfscope}%
\pgfpathrectangle{\pgfqpoint{0.562500in}{0.275000in}}{\pgfqpoint{3.487500in}{1.925000in}}%
\pgfusepath{clip}%
\pgfsetbuttcap%
\pgfsetroundjoin%
\definecolor{currentfill}{rgb}{0.000000,0.000000,0.000000}%
\pgfsetfillcolor{currentfill}%
\pgfsetlinewidth{1.003750pt}%
\definecolor{currentstroke}{rgb}{0.000000,0.000000,0.000000}%
\pgfsetstrokecolor{currentstroke}%
\pgfsetdash{}{0pt}%
\pgfpathmoveto{\pgfqpoint{1.772992in}{1.216635in}}%
\pgfpathcurveto{\pgfqpoint{1.778517in}{1.216635in}}{\pgfqpoint{1.783816in}{1.218830in}}{\pgfqpoint{1.787723in}{1.222736in}}%
\pgfpathcurveto{\pgfqpoint{1.791630in}{1.226643in}}{\pgfqpoint{1.793825in}{1.231943in}}{\pgfqpoint{1.793825in}{1.237468in}}%
\pgfpathcurveto{\pgfqpoint{1.793825in}{1.242993in}}{\pgfqpoint{1.791630in}{1.248292in}}{\pgfqpoint{1.787723in}{1.252199in}}%
\pgfpathcurveto{\pgfqpoint{1.783816in}{1.256106in}}{\pgfqpoint{1.778517in}{1.258301in}}{\pgfqpoint{1.772992in}{1.258301in}}%
\pgfpathcurveto{\pgfqpoint{1.767467in}{1.258301in}}{\pgfqpoint{1.762167in}{1.256106in}}{\pgfqpoint{1.758260in}{1.252199in}}%
\pgfpathcurveto{\pgfqpoint{1.754353in}{1.248292in}}{\pgfqpoint{1.752158in}{1.242993in}}{\pgfqpoint{1.752158in}{1.237468in}}%
\pgfpathcurveto{\pgfqpoint{1.752158in}{1.231943in}}{\pgfqpoint{1.754353in}{1.226643in}}{\pgfqpoint{1.758260in}{1.222736in}}%
\pgfpathcurveto{\pgfqpoint{1.762167in}{1.218830in}}{\pgfqpoint{1.767467in}{1.216635in}}{\pgfqpoint{1.772992in}{1.216635in}}%
\pgfpathclose%
\pgfusepath{stroke,fill}%
\end{pgfscope}%
\begin{pgfscope}%
\pgfpathrectangle{\pgfqpoint{0.562500in}{0.275000in}}{\pgfqpoint{3.487500in}{1.925000in}}%
\pgfusepath{clip}%
\pgfsetbuttcap%
\pgfsetroundjoin%
\definecolor{currentfill}{rgb}{0.000000,0.000000,0.000000}%
\pgfsetfillcolor{currentfill}%
\pgfsetlinewidth{1.003750pt}%
\definecolor{currentstroke}{rgb}{0.000000,0.000000,0.000000}%
\pgfsetstrokecolor{currentstroke}%
\pgfsetdash{}{0pt}%
\pgfpathmoveto{\pgfqpoint{1.772992in}{1.216635in}}%
\pgfpathcurveto{\pgfqpoint{1.778517in}{1.216635in}}{\pgfqpoint{1.783816in}{1.218830in}}{\pgfqpoint{1.787723in}{1.222736in}}%
\pgfpathcurveto{\pgfqpoint{1.791630in}{1.226643in}}{\pgfqpoint{1.793825in}{1.231943in}}{\pgfqpoint{1.793825in}{1.237468in}}%
\pgfpathcurveto{\pgfqpoint{1.793825in}{1.242993in}}{\pgfqpoint{1.791630in}{1.248292in}}{\pgfqpoint{1.787723in}{1.252199in}}%
\pgfpathcurveto{\pgfqpoint{1.783816in}{1.256106in}}{\pgfqpoint{1.778517in}{1.258301in}}{\pgfqpoint{1.772992in}{1.258301in}}%
\pgfpathcurveto{\pgfqpoint{1.767467in}{1.258301in}}{\pgfqpoint{1.762167in}{1.256106in}}{\pgfqpoint{1.758260in}{1.252199in}}%
\pgfpathcurveto{\pgfqpoint{1.754353in}{1.248292in}}{\pgfqpoint{1.752158in}{1.242993in}}{\pgfqpoint{1.752158in}{1.237468in}}%
\pgfpathcurveto{\pgfqpoint{1.752158in}{1.231943in}}{\pgfqpoint{1.754353in}{1.226643in}}{\pgfqpoint{1.758260in}{1.222736in}}%
\pgfpathcurveto{\pgfqpoint{1.762167in}{1.218830in}}{\pgfqpoint{1.767467in}{1.216635in}}{\pgfqpoint{1.772992in}{1.216635in}}%
\pgfpathclose%
\pgfusepath{stroke,fill}%
\end{pgfscope}%
\begin{pgfscope}%
\pgfpathrectangle{\pgfqpoint{0.562500in}{0.275000in}}{\pgfqpoint{3.487500in}{1.925000in}}%
\pgfusepath{clip}%
\pgfsetbuttcap%
\pgfsetroundjoin%
\definecolor{currentfill}{rgb}{0.000000,0.000000,0.000000}%
\pgfsetfillcolor{currentfill}%
\pgfsetlinewidth{1.003750pt}%
\definecolor{currentstroke}{rgb}{0.000000,0.000000,0.000000}%
\pgfsetstrokecolor{currentstroke}%
\pgfsetdash{}{0pt}%
\pgfpathmoveto{\pgfqpoint{1.772992in}{2.076667in}}%
\pgfpathcurveto{\pgfqpoint{1.778517in}{2.076667in}}{\pgfqpoint{1.783816in}{2.078862in}}{\pgfqpoint{1.787723in}{2.082769in}}%
\pgfpathcurveto{\pgfqpoint{1.791630in}{2.086675in}}{\pgfqpoint{1.793825in}{2.091975in}}{\pgfqpoint{1.793825in}{2.097500in}}%
\pgfpathcurveto{\pgfqpoint{1.793825in}{2.103025in}}{\pgfqpoint{1.791630in}{2.108325in}}{\pgfqpoint{1.787723in}{2.112231in}}%
\pgfpathcurveto{\pgfqpoint{1.783816in}{2.116138in}}{\pgfqpoint{1.778517in}{2.118333in}}{\pgfqpoint{1.772992in}{2.118333in}}%
\pgfpathcurveto{\pgfqpoint{1.767467in}{2.118333in}}{\pgfqpoint{1.762167in}{2.116138in}}{\pgfqpoint{1.758260in}{2.112231in}}%
\pgfpathcurveto{\pgfqpoint{1.754353in}{2.108325in}}{\pgfqpoint{1.752158in}{2.103025in}}{\pgfqpoint{1.752158in}{2.097500in}}%
\pgfpathcurveto{\pgfqpoint{1.752158in}{2.091975in}}{\pgfqpoint{1.754353in}{2.086675in}}{\pgfqpoint{1.758260in}{2.082769in}}%
\pgfpathcurveto{\pgfqpoint{1.762167in}{2.078862in}}{\pgfqpoint{1.767467in}{2.076667in}}{\pgfqpoint{1.772992in}{2.076667in}}%
\pgfpathclose%
\pgfusepath{stroke,fill}%
\end{pgfscope}%
\begin{pgfscope}%
\pgfpathrectangle{\pgfqpoint{0.562500in}{0.275000in}}{\pgfqpoint{3.487500in}{1.925000in}}%
\pgfusepath{clip}%
\pgfsetbuttcap%
\pgfsetroundjoin%
\definecolor{currentfill}{rgb}{0.000000,0.000000,0.000000}%
\pgfsetfillcolor{currentfill}%
\pgfsetlinewidth{1.003750pt}%
\definecolor{currentstroke}{rgb}{0.000000,0.000000,0.000000}%
\pgfsetstrokecolor{currentstroke}%
\pgfsetdash{}{0pt}%
\pgfpathmoveto{\pgfqpoint{1.772992in}{1.216635in}}%
\pgfpathcurveto{\pgfqpoint{1.778517in}{1.216635in}}{\pgfqpoint{1.783816in}{1.218830in}}{\pgfqpoint{1.787723in}{1.222736in}}%
\pgfpathcurveto{\pgfqpoint{1.791630in}{1.226643in}}{\pgfqpoint{1.793825in}{1.231943in}}{\pgfqpoint{1.793825in}{1.237468in}}%
\pgfpathcurveto{\pgfqpoint{1.793825in}{1.242993in}}{\pgfqpoint{1.791630in}{1.248292in}}{\pgfqpoint{1.787723in}{1.252199in}}%
\pgfpathcurveto{\pgfqpoint{1.783816in}{1.256106in}}{\pgfqpoint{1.778517in}{1.258301in}}{\pgfqpoint{1.772992in}{1.258301in}}%
\pgfpathcurveto{\pgfqpoint{1.767467in}{1.258301in}}{\pgfqpoint{1.762167in}{1.256106in}}{\pgfqpoint{1.758260in}{1.252199in}}%
\pgfpathcurveto{\pgfqpoint{1.754353in}{1.248292in}}{\pgfqpoint{1.752158in}{1.242993in}}{\pgfqpoint{1.752158in}{1.237468in}}%
\pgfpathcurveto{\pgfqpoint{1.752158in}{1.231943in}}{\pgfqpoint{1.754353in}{1.226643in}}{\pgfqpoint{1.758260in}{1.222736in}}%
\pgfpathcurveto{\pgfqpoint{1.762167in}{1.218830in}}{\pgfqpoint{1.767467in}{1.216635in}}{\pgfqpoint{1.772992in}{1.216635in}}%
\pgfpathclose%
\pgfusepath{stroke,fill}%
\end{pgfscope}%
\begin{pgfscope}%
\pgfpathrectangle{\pgfqpoint{0.562500in}{0.275000in}}{\pgfqpoint{3.487500in}{1.925000in}}%
\pgfusepath{clip}%
\pgfsetbuttcap%
\pgfsetroundjoin%
\definecolor{currentfill}{rgb}{0.000000,0.000000,0.000000}%
\pgfsetfillcolor{currentfill}%
\pgfsetlinewidth{1.003750pt}%
\definecolor{currentstroke}{rgb}{0.000000,0.000000,0.000000}%
\pgfsetstrokecolor{currentstroke}%
\pgfsetdash{}{0pt}%
\pgfpathmoveto{\pgfqpoint{1.772992in}{1.216635in}}%
\pgfpathcurveto{\pgfqpoint{1.778517in}{1.216635in}}{\pgfqpoint{1.783816in}{1.218830in}}{\pgfqpoint{1.787723in}{1.222736in}}%
\pgfpathcurveto{\pgfqpoint{1.791630in}{1.226643in}}{\pgfqpoint{1.793825in}{1.231943in}}{\pgfqpoint{1.793825in}{1.237468in}}%
\pgfpathcurveto{\pgfqpoint{1.793825in}{1.242993in}}{\pgfqpoint{1.791630in}{1.248292in}}{\pgfqpoint{1.787723in}{1.252199in}}%
\pgfpathcurveto{\pgfqpoint{1.783816in}{1.256106in}}{\pgfqpoint{1.778517in}{1.258301in}}{\pgfqpoint{1.772992in}{1.258301in}}%
\pgfpathcurveto{\pgfqpoint{1.767467in}{1.258301in}}{\pgfqpoint{1.762167in}{1.256106in}}{\pgfqpoint{1.758260in}{1.252199in}}%
\pgfpathcurveto{\pgfqpoint{1.754353in}{1.248292in}}{\pgfqpoint{1.752158in}{1.242993in}}{\pgfqpoint{1.752158in}{1.237468in}}%
\pgfpathcurveto{\pgfqpoint{1.752158in}{1.231943in}}{\pgfqpoint{1.754353in}{1.226643in}}{\pgfqpoint{1.758260in}{1.222736in}}%
\pgfpathcurveto{\pgfqpoint{1.762167in}{1.218830in}}{\pgfqpoint{1.767467in}{1.216635in}}{\pgfqpoint{1.772992in}{1.216635in}}%
\pgfpathclose%
\pgfusepath{stroke,fill}%
\end{pgfscope}%
\begin{pgfscope}%
\pgfpathrectangle{\pgfqpoint{0.562500in}{0.275000in}}{\pgfqpoint{3.487500in}{1.925000in}}%
\pgfusepath{clip}%
\pgfsetbuttcap%
\pgfsetroundjoin%
\definecolor{currentfill}{rgb}{0.000000,0.000000,0.000000}%
\pgfsetfillcolor{currentfill}%
\pgfsetlinewidth{1.003750pt}%
\definecolor{currentstroke}{rgb}{0.000000,0.000000,0.000000}%
\pgfsetstrokecolor{currentstroke}%
\pgfsetdash{}{0pt}%
\pgfpathmoveto{\pgfqpoint{1.772992in}{1.216635in}}%
\pgfpathcurveto{\pgfqpoint{1.778517in}{1.216635in}}{\pgfqpoint{1.783816in}{1.218830in}}{\pgfqpoint{1.787723in}{1.222736in}}%
\pgfpathcurveto{\pgfqpoint{1.791630in}{1.226643in}}{\pgfqpoint{1.793825in}{1.231943in}}{\pgfqpoint{1.793825in}{1.237468in}}%
\pgfpathcurveto{\pgfqpoint{1.793825in}{1.242993in}}{\pgfqpoint{1.791630in}{1.248292in}}{\pgfqpoint{1.787723in}{1.252199in}}%
\pgfpathcurveto{\pgfqpoint{1.783816in}{1.256106in}}{\pgfqpoint{1.778517in}{1.258301in}}{\pgfqpoint{1.772992in}{1.258301in}}%
\pgfpathcurveto{\pgfqpoint{1.767467in}{1.258301in}}{\pgfqpoint{1.762167in}{1.256106in}}{\pgfqpoint{1.758260in}{1.252199in}}%
\pgfpathcurveto{\pgfqpoint{1.754353in}{1.248292in}}{\pgfqpoint{1.752158in}{1.242993in}}{\pgfqpoint{1.752158in}{1.237468in}}%
\pgfpathcurveto{\pgfqpoint{1.752158in}{1.231943in}}{\pgfqpoint{1.754353in}{1.226643in}}{\pgfqpoint{1.758260in}{1.222736in}}%
\pgfpathcurveto{\pgfqpoint{1.762167in}{1.218830in}}{\pgfqpoint{1.767467in}{1.216635in}}{\pgfqpoint{1.772992in}{1.216635in}}%
\pgfpathclose%
\pgfusepath{stroke,fill}%
\end{pgfscope}%
\begin{pgfscope}%
\pgfpathrectangle{\pgfqpoint{0.562500in}{0.275000in}}{\pgfqpoint{3.487500in}{1.925000in}}%
\pgfusepath{clip}%
\pgfsetbuttcap%
\pgfsetroundjoin%
\definecolor{currentfill}{rgb}{0.000000,0.000000,0.000000}%
\pgfsetfillcolor{currentfill}%
\pgfsetlinewidth{1.003750pt}%
\definecolor{currentstroke}{rgb}{0.000000,0.000000,0.000000}%
\pgfsetstrokecolor{currentstroke}%
\pgfsetdash{}{0pt}%
\pgfpathmoveto{\pgfqpoint{1.772992in}{1.216635in}}%
\pgfpathcurveto{\pgfqpoint{1.778517in}{1.216635in}}{\pgfqpoint{1.783816in}{1.218830in}}{\pgfqpoint{1.787723in}{1.222736in}}%
\pgfpathcurveto{\pgfqpoint{1.791630in}{1.226643in}}{\pgfqpoint{1.793825in}{1.231943in}}{\pgfqpoint{1.793825in}{1.237468in}}%
\pgfpathcurveto{\pgfqpoint{1.793825in}{1.242993in}}{\pgfqpoint{1.791630in}{1.248292in}}{\pgfqpoint{1.787723in}{1.252199in}}%
\pgfpathcurveto{\pgfqpoint{1.783816in}{1.256106in}}{\pgfqpoint{1.778517in}{1.258301in}}{\pgfqpoint{1.772992in}{1.258301in}}%
\pgfpathcurveto{\pgfqpoint{1.767467in}{1.258301in}}{\pgfqpoint{1.762167in}{1.256106in}}{\pgfqpoint{1.758260in}{1.252199in}}%
\pgfpathcurveto{\pgfqpoint{1.754353in}{1.248292in}}{\pgfqpoint{1.752158in}{1.242993in}}{\pgfqpoint{1.752158in}{1.237468in}}%
\pgfpathcurveto{\pgfqpoint{1.752158in}{1.231943in}}{\pgfqpoint{1.754353in}{1.226643in}}{\pgfqpoint{1.758260in}{1.222736in}}%
\pgfpathcurveto{\pgfqpoint{1.762167in}{1.218830in}}{\pgfqpoint{1.767467in}{1.216635in}}{\pgfqpoint{1.772992in}{1.216635in}}%
\pgfpathclose%
\pgfusepath{stroke,fill}%
\end{pgfscope}%
\begin{pgfscope}%
\pgfpathrectangle{\pgfqpoint{0.562500in}{0.275000in}}{\pgfqpoint{3.487500in}{1.925000in}}%
\pgfusepath{clip}%
\pgfsetbuttcap%
\pgfsetroundjoin%
\definecolor{currentfill}{rgb}{0.000000,0.000000,0.000000}%
\pgfsetfillcolor{currentfill}%
\pgfsetlinewidth{1.003750pt}%
\definecolor{currentstroke}{rgb}{0.000000,0.000000,0.000000}%
\pgfsetstrokecolor{currentstroke}%
\pgfsetdash{}{0pt}%
\pgfpathmoveto{\pgfqpoint{1.772992in}{1.216635in}}%
\pgfpathcurveto{\pgfqpoint{1.778517in}{1.216635in}}{\pgfqpoint{1.783816in}{1.218830in}}{\pgfqpoint{1.787723in}{1.222736in}}%
\pgfpathcurveto{\pgfqpoint{1.791630in}{1.226643in}}{\pgfqpoint{1.793825in}{1.231943in}}{\pgfqpoint{1.793825in}{1.237468in}}%
\pgfpathcurveto{\pgfqpoint{1.793825in}{1.242993in}}{\pgfqpoint{1.791630in}{1.248292in}}{\pgfqpoint{1.787723in}{1.252199in}}%
\pgfpathcurveto{\pgfqpoint{1.783816in}{1.256106in}}{\pgfqpoint{1.778517in}{1.258301in}}{\pgfqpoint{1.772992in}{1.258301in}}%
\pgfpathcurveto{\pgfqpoint{1.767467in}{1.258301in}}{\pgfqpoint{1.762167in}{1.256106in}}{\pgfqpoint{1.758260in}{1.252199in}}%
\pgfpathcurveto{\pgfqpoint{1.754353in}{1.248292in}}{\pgfqpoint{1.752158in}{1.242993in}}{\pgfqpoint{1.752158in}{1.237468in}}%
\pgfpathcurveto{\pgfqpoint{1.752158in}{1.231943in}}{\pgfqpoint{1.754353in}{1.226643in}}{\pgfqpoint{1.758260in}{1.222736in}}%
\pgfpathcurveto{\pgfqpoint{1.762167in}{1.218830in}}{\pgfqpoint{1.767467in}{1.216635in}}{\pgfqpoint{1.772992in}{1.216635in}}%
\pgfpathclose%
\pgfusepath{stroke,fill}%
\end{pgfscope}%
\begin{pgfscope}%
\pgfpathrectangle{\pgfqpoint{0.562500in}{0.275000in}}{\pgfqpoint{3.487500in}{1.925000in}}%
\pgfusepath{clip}%
\pgfsetbuttcap%
\pgfsetroundjoin%
\definecolor{currentfill}{rgb}{0.000000,0.000000,0.000000}%
\pgfsetfillcolor{currentfill}%
\pgfsetlinewidth{1.003750pt}%
\definecolor{currentstroke}{rgb}{0.000000,0.000000,0.000000}%
\pgfsetstrokecolor{currentstroke}%
\pgfsetdash{}{0pt}%
\pgfpathmoveto{\pgfqpoint{1.772992in}{1.216635in}}%
\pgfpathcurveto{\pgfqpoint{1.778517in}{1.216635in}}{\pgfqpoint{1.783816in}{1.218830in}}{\pgfqpoint{1.787723in}{1.222736in}}%
\pgfpathcurveto{\pgfqpoint{1.791630in}{1.226643in}}{\pgfqpoint{1.793825in}{1.231943in}}{\pgfqpoint{1.793825in}{1.237468in}}%
\pgfpathcurveto{\pgfqpoint{1.793825in}{1.242993in}}{\pgfqpoint{1.791630in}{1.248292in}}{\pgfqpoint{1.787723in}{1.252199in}}%
\pgfpathcurveto{\pgfqpoint{1.783816in}{1.256106in}}{\pgfqpoint{1.778517in}{1.258301in}}{\pgfqpoint{1.772992in}{1.258301in}}%
\pgfpathcurveto{\pgfqpoint{1.767467in}{1.258301in}}{\pgfqpoint{1.762167in}{1.256106in}}{\pgfqpoint{1.758260in}{1.252199in}}%
\pgfpathcurveto{\pgfqpoint{1.754353in}{1.248292in}}{\pgfqpoint{1.752158in}{1.242993in}}{\pgfqpoint{1.752158in}{1.237468in}}%
\pgfpathcurveto{\pgfqpoint{1.752158in}{1.231943in}}{\pgfqpoint{1.754353in}{1.226643in}}{\pgfqpoint{1.758260in}{1.222736in}}%
\pgfpathcurveto{\pgfqpoint{1.762167in}{1.218830in}}{\pgfqpoint{1.767467in}{1.216635in}}{\pgfqpoint{1.772992in}{1.216635in}}%
\pgfpathclose%
\pgfusepath{stroke,fill}%
\end{pgfscope}%
\begin{pgfscope}%
\pgfpathrectangle{\pgfqpoint{0.562500in}{0.275000in}}{\pgfqpoint{3.487500in}{1.925000in}}%
\pgfusepath{clip}%
\pgfsetbuttcap%
\pgfsetroundjoin%
\definecolor{currentfill}{rgb}{0.000000,0.000000,0.000000}%
\pgfsetfillcolor{currentfill}%
\pgfsetlinewidth{1.003750pt}%
\definecolor{currentstroke}{rgb}{0.000000,0.000000,0.000000}%
\pgfsetstrokecolor{currentstroke}%
\pgfsetdash{}{0pt}%
\pgfpathmoveto{\pgfqpoint{1.772992in}{1.216635in}}%
\pgfpathcurveto{\pgfqpoint{1.778517in}{1.216635in}}{\pgfqpoint{1.783816in}{1.218830in}}{\pgfqpoint{1.787723in}{1.222736in}}%
\pgfpathcurveto{\pgfqpoint{1.791630in}{1.226643in}}{\pgfqpoint{1.793825in}{1.231943in}}{\pgfqpoint{1.793825in}{1.237468in}}%
\pgfpathcurveto{\pgfqpoint{1.793825in}{1.242993in}}{\pgfqpoint{1.791630in}{1.248292in}}{\pgfqpoint{1.787723in}{1.252199in}}%
\pgfpathcurveto{\pgfqpoint{1.783816in}{1.256106in}}{\pgfqpoint{1.778517in}{1.258301in}}{\pgfqpoint{1.772992in}{1.258301in}}%
\pgfpathcurveto{\pgfqpoint{1.767467in}{1.258301in}}{\pgfqpoint{1.762167in}{1.256106in}}{\pgfqpoint{1.758260in}{1.252199in}}%
\pgfpathcurveto{\pgfqpoint{1.754353in}{1.248292in}}{\pgfqpoint{1.752158in}{1.242993in}}{\pgfqpoint{1.752158in}{1.237468in}}%
\pgfpathcurveto{\pgfqpoint{1.752158in}{1.231943in}}{\pgfqpoint{1.754353in}{1.226643in}}{\pgfqpoint{1.758260in}{1.222736in}}%
\pgfpathcurveto{\pgfqpoint{1.762167in}{1.218830in}}{\pgfqpoint{1.767467in}{1.216635in}}{\pgfqpoint{1.772992in}{1.216635in}}%
\pgfpathclose%
\pgfusepath{stroke,fill}%
\end{pgfscope}%
\begin{pgfscope}%
\pgfpathrectangle{\pgfqpoint{0.562500in}{0.275000in}}{\pgfqpoint{3.487500in}{1.925000in}}%
\pgfusepath{clip}%
\pgfsetbuttcap%
\pgfsetroundjoin%
\definecolor{currentfill}{rgb}{0.000000,0.000000,0.000000}%
\pgfsetfillcolor{currentfill}%
\pgfsetlinewidth{1.003750pt}%
\definecolor{currentstroke}{rgb}{0.000000,0.000000,0.000000}%
\pgfsetstrokecolor{currentstroke}%
\pgfsetdash{}{0pt}%
\pgfpathmoveto{\pgfqpoint{1.772992in}{1.216635in}}%
\pgfpathcurveto{\pgfqpoint{1.778517in}{1.216635in}}{\pgfqpoint{1.783816in}{1.218830in}}{\pgfqpoint{1.787723in}{1.222736in}}%
\pgfpathcurveto{\pgfqpoint{1.791630in}{1.226643in}}{\pgfqpoint{1.793825in}{1.231943in}}{\pgfqpoint{1.793825in}{1.237468in}}%
\pgfpathcurveto{\pgfqpoint{1.793825in}{1.242993in}}{\pgfqpoint{1.791630in}{1.248292in}}{\pgfqpoint{1.787723in}{1.252199in}}%
\pgfpathcurveto{\pgfqpoint{1.783816in}{1.256106in}}{\pgfqpoint{1.778517in}{1.258301in}}{\pgfqpoint{1.772992in}{1.258301in}}%
\pgfpathcurveto{\pgfqpoint{1.767467in}{1.258301in}}{\pgfqpoint{1.762167in}{1.256106in}}{\pgfqpoint{1.758260in}{1.252199in}}%
\pgfpathcurveto{\pgfqpoint{1.754353in}{1.248292in}}{\pgfqpoint{1.752158in}{1.242993in}}{\pgfqpoint{1.752158in}{1.237468in}}%
\pgfpathcurveto{\pgfqpoint{1.752158in}{1.231943in}}{\pgfqpoint{1.754353in}{1.226643in}}{\pgfqpoint{1.758260in}{1.222736in}}%
\pgfpathcurveto{\pgfqpoint{1.762167in}{1.218830in}}{\pgfqpoint{1.767467in}{1.216635in}}{\pgfqpoint{1.772992in}{1.216635in}}%
\pgfpathclose%
\pgfusepath{stroke,fill}%
\end{pgfscope}%
\begin{pgfscope}%
\pgfpathrectangle{\pgfqpoint{0.562500in}{0.275000in}}{\pgfqpoint{3.487500in}{1.925000in}}%
\pgfusepath{clip}%
\pgfsetbuttcap%
\pgfsetroundjoin%
\definecolor{currentfill}{rgb}{0.000000,0.000000,0.000000}%
\pgfsetfillcolor{currentfill}%
\pgfsetlinewidth{1.003750pt}%
\definecolor{currentstroke}{rgb}{0.000000,0.000000,0.000000}%
\pgfsetstrokecolor{currentstroke}%
\pgfsetdash{}{0pt}%
\pgfpathmoveto{\pgfqpoint{1.772992in}{1.216635in}}%
\pgfpathcurveto{\pgfqpoint{1.778517in}{1.216635in}}{\pgfqpoint{1.783816in}{1.218830in}}{\pgfqpoint{1.787723in}{1.222736in}}%
\pgfpathcurveto{\pgfqpoint{1.791630in}{1.226643in}}{\pgfqpoint{1.793825in}{1.231943in}}{\pgfqpoint{1.793825in}{1.237468in}}%
\pgfpathcurveto{\pgfqpoint{1.793825in}{1.242993in}}{\pgfqpoint{1.791630in}{1.248292in}}{\pgfqpoint{1.787723in}{1.252199in}}%
\pgfpathcurveto{\pgfqpoint{1.783816in}{1.256106in}}{\pgfqpoint{1.778517in}{1.258301in}}{\pgfqpoint{1.772992in}{1.258301in}}%
\pgfpathcurveto{\pgfqpoint{1.767467in}{1.258301in}}{\pgfqpoint{1.762167in}{1.256106in}}{\pgfqpoint{1.758260in}{1.252199in}}%
\pgfpathcurveto{\pgfqpoint{1.754353in}{1.248292in}}{\pgfqpoint{1.752158in}{1.242993in}}{\pgfqpoint{1.752158in}{1.237468in}}%
\pgfpathcurveto{\pgfqpoint{1.752158in}{1.231943in}}{\pgfqpoint{1.754353in}{1.226643in}}{\pgfqpoint{1.758260in}{1.222736in}}%
\pgfpathcurveto{\pgfqpoint{1.762167in}{1.218830in}}{\pgfqpoint{1.767467in}{1.216635in}}{\pgfqpoint{1.772992in}{1.216635in}}%
\pgfpathclose%
\pgfusepath{stroke,fill}%
\end{pgfscope}%
\begin{pgfscope}%
\pgfpathrectangle{\pgfqpoint{0.562500in}{0.275000in}}{\pgfqpoint{3.487500in}{1.925000in}}%
\pgfusepath{clip}%
\pgfsetbuttcap%
\pgfsetroundjoin%
\definecolor{currentfill}{rgb}{0.000000,0.000000,0.000000}%
\pgfsetfillcolor{currentfill}%
\pgfsetlinewidth{1.003750pt}%
\definecolor{currentstroke}{rgb}{0.000000,0.000000,0.000000}%
\pgfsetstrokecolor{currentstroke}%
\pgfsetdash{}{0pt}%
\pgfpathmoveto{\pgfqpoint{1.772992in}{1.216635in}}%
\pgfpathcurveto{\pgfqpoint{1.778517in}{1.216635in}}{\pgfqpoint{1.783816in}{1.218830in}}{\pgfqpoint{1.787723in}{1.222736in}}%
\pgfpathcurveto{\pgfqpoint{1.791630in}{1.226643in}}{\pgfqpoint{1.793825in}{1.231943in}}{\pgfqpoint{1.793825in}{1.237468in}}%
\pgfpathcurveto{\pgfqpoint{1.793825in}{1.242993in}}{\pgfqpoint{1.791630in}{1.248292in}}{\pgfqpoint{1.787723in}{1.252199in}}%
\pgfpathcurveto{\pgfqpoint{1.783816in}{1.256106in}}{\pgfqpoint{1.778517in}{1.258301in}}{\pgfqpoint{1.772992in}{1.258301in}}%
\pgfpathcurveto{\pgfqpoint{1.767467in}{1.258301in}}{\pgfqpoint{1.762167in}{1.256106in}}{\pgfqpoint{1.758260in}{1.252199in}}%
\pgfpathcurveto{\pgfqpoint{1.754353in}{1.248292in}}{\pgfqpoint{1.752158in}{1.242993in}}{\pgfqpoint{1.752158in}{1.237468in}}%
\pgfpathcurveto{\pgfqpoint{1.752158in}{1.231943in}}{\pgfqpoint{1.754353in}{1.226643in}}{\pgfqpoint{1.758260in}{1.222736in}}%
\pgfpathcurveto{\pgfqpoint{1.762167in}{1.218830in}}{\pgfqpoint{1.767467in}{1.216635in}}{\pgfqpoint{1.772992in}{1.216635in}}%
\pgfpathclose%
\pgfusepath{stroke,fill}%
\end{pgfscope}%
\begin{pgfscope}%
\pgfpathrectangle{\pgfqpoint{0.562500in}{0.275000in}}{\pgfqpoint{3.487500in}{1.925000in}}%
\pgfusepath{clip}%
\pgfsetbuttcap%
\pgfsetroundjoin%
\definecolor{currentfill}{rgb}{0.000000,0.000000,0.000000}%
\pgfsetfillcolor{currentfill}%
\pgfsetlinewidth{1.003750pt}%
\definecolor{currentstroke}{rgb}{0.000000,0.000000,0.000000}%
\pgfsetstrokecolor{currentstroke}%
\pgfsetdash{}{0pt}%
\pgfpathmoveto{\pgfqpoint{1.772992in}{1.216635in}}%
\pgfpathcurveto{\pgfqpoint{1.778517in}{1.216635in}}{\pgfqpoint{1.783816in}{1.218830in}}{\pgfqpoint{1.787723in}{1.222736in}}%
\pgfpathcurveto{\pgfqpoint{1.791630in}{1.226643in}}{\pgfqpoint{1.793825in}{1.231943in}}{\pgfqpoint{1.793825in}{1.237468in}}%
\pgfpathcurveto{\pgfqpoint{1.793825in}{1.242993in}}{\pgfqpoint{1.791630in}{1.248292in}}{\pgfqpoint{1.787723in}{1.252199in}}%
\pgfpathcurveto{\pgfqpoint{1.783816in}{1.256106in}}{\pgfqpoint{1.778517in}{1.258301in}}{\pgfqpoint{1.772992in}{1.258301in}}%
\pgfpathcurveto{\pgfqpoint{1.767467in}{1.258301in}}{\pgfqpoint{1.762167in}{1.256106in}}{\pgfqpoint{1.758260in}{1.252199in}}%
\pgfpathcurveto{\pgfqpoint{1.754353in}{1.248292in}}{\pgfqpoint{1.752158in}{1.242993in}}{\pgfqpoint{1.752158in}{1.237468in}}%
\pgfpathcurveto{\pgfqpoint{1.752158in}{1.231943in}}{\pgfqpoint{1.754353in}{1.226643in}}{\pgfqpoint{1.758260in}{1.222736in}}%
\pgfpathcurveto{\pgfqpoint{1.762167in}{1.218830in}}{\pgfqpoint{1.767467in}{1.216635in}}{\pgfqpoint{1.772992in}{1.216635in}}%
\pgfpathclose%
\pgfusepath{stroke,fill}%
\end{pgfscope}%
\begin{pgfscope}%
\pgfpathrectangle{\pgfqpoint{0.562500in}{0.275000in}}{\pgfqpoint{3.487500in}{1.925000in}}%
\pgfusepath{clip}%
\pgfsetbuttcap%
\pgfsetroundjoin%
\definecolor{currentfill}{rgb}{0.000000,0.000000,0.000000}%
\pgfsetfillcolor{currentfill}%
\pgfsetlinewidth{1.003750pt}%
\definecolor{currentstroke}{rgb}{0.000000,0.000000,0.000000}%
\pgfsetstrokecolor{currentstroke}%
\pgfsetdash{}{0pt}%
\pgfpathmoveto{\pgfqpoint{1.772992in}{1.216635in}}%
\pgfpathcurveto{\pgfqpoint{1.778517in}{1.216635in}}{\pgfqpoint{1.783816in}{1.218830in}}{\pgfqpoint{1.787723in}{1.222736in}}%
\pgfpathcurveto{\pgfqpoint{1.791630in}{1.226643in}}{\pgfqpoint{1.793825in}{1.231943in}}{\pgfqpoint{1.793825in}{1.237468in}}%
\pgfpathcurveto{\pgfqpoint{1.793825in}{1.242993in}}{\pgfqpoint{1.791630in}{1.248292in}}{\pgfqpoint{1.787723in}{1.252199in}}%
\pgfpathcurveto{\pgfqpoint{1.783816in}{1.256106in}}{\pgfqpoint{1.778517in}{1.258301in}}{\pgfqpoint{1.772992in}{1.258301in}}%
\pgfpathcurveto{\pgfqpoint{1.767467in}{1.258301in}}{\pgfqpoint{1.762167in}{1.256106in}}{\pgfqpoint{1.758260in}{1.252199in}}%
\pgfpathcurveto{\pgfqpoint{1.754353in}{1.248292in}}{\pgfqpoint{1.752158in}{1.242993in}}{\pgfqpoint{1.752158in}{1.237468in}}%
\pgfpathcurveto{\pgfqpoint{1.752158in}{1.231943in}}{\pgfqpoint{1.754353in}{1.226643in}}{\pgfqpoint{1.758260in}{1.222736in}}%
\pgfpathcurveto{\pgfqpoint{1.762167in}{1.218830in}}{\pgfqpoint{1.767467in}{1.216635in}}{\pgfqpoint{1.772992in}{1.216635in}}%
\pgfpathclose%
\pgfusepath{stroke,fill}%
\end{pgfscope}%
\begin{pgfscope}%
\pgfpathrectangle{\pgfqpoint{0.562500in}{0.275000in}}{\pgfqpoint{3.487500in}{1.925000in}}%
\pgfusepath{clip}%
\pgfsetbuttcap%
\pgfsetroundjoin%
\definecolor{currentfill}{rgb}{0.000000,0.000000,0.000000}%
\pgfsetfillcolor{currentfill}%
\pgfsetlinewidth{1.003750pt}%
\definecolor{currentstroke}{rgb}{0.000000,0.000000,0.000000}%
\pgfsetstrokecolor{currentstroke}%
\pgfsetdash{}{0pt}%
\pgfpathmoveto{\pgfqpoint{1.772992in}{1.216635in}}%
\pgfpathcurveto{\pgfqpoint{1.778517in}{1.216635in}}{\pgfqpoint{1.783816in}{1.218830in}}{\pgfqpoint{1.787723in}{1.222736in}}%
\pgfpathcurveto{\pgfqpoint{1.791630in}{1.226643in}}{\pgfqpoint{1.793825in}{1.231943in}}{\pgfqpoint{1.793825in}{1.237468in}}%
\pgfpathcurveto{\pgfqpoint{1.793825in}{1.242993in}}{\pgfqpoint{1.791630in}{1.248292in}}{\pgfqpoint{1.787723in}{1.252199in}}%
\pgfpathcurveto{\pgfqpoint{1.783816in}{1.256106in}}{\pgfqpoint{1.778517in}{1.258301in}}{\pgfqpoint{1.772992in}{1.258301in}}%
\pgfpathcurveto{\pgfqpoint{1.767467in}{1.258301in}}{\pgfqpoint{1.762167in}{1.256106in}}{\pgfqpoint{1.758260in}{1.252199in}}%
\pgfpathcurveto{\pgfqpoint{1.754353in}{1.248292in}}{\pgfqpoint{1.752158in}{1.242993in}}{\pgfqpoint{1.752158in}{1.237468in}}%
\pgfpathcurveto{\pgfqpoint{1.752158in}{1.231943in}}{\pgfqpoint{1.754353in}{1.226643in}}{\pgfqpoint{1.758260in}{1.222736in}}%
\pgfpathcurveto{\pgfqpoint{1.762167in}{1.218830in}}{\pgfqpoint{1.767467in}{1.216635in}}{\pgfqpoint{1.772992in}{1.216635in}}%
\pgfpathclose%
\pgfusepath{stroke,fill}%
\end{pgfscope}%
\begin{pgfscope}%
\pgfpathrectangle{\pgfqpoint{0.562500in}{0.275000in}}{\pgfqpoint{3.487500in}{1.925000in}}%
\pgfusepath{clip}%
\pgfsetbuttcap%
\pgfsetroundjoin%
\definecolor{currentfill}{rgb}{0.000000,0.000000,0.000000}%
\pgfsetfillcolor{currentfill}%
\pgfsetlinewidth{1.003750pt}%
\definecolor{currentstroke}{rgb}{0.000000,0.000000,0.000000}%
\pgfsetstrokecolor{currentstroke}%
\pgfsetdash{}{0pt}%
\pgfpathmoveto{\pgfqpoint{1.772992in}{1.216635in}}%
\pgfpathcurveto{\pgfqpoint{1.778517in}{1.216635in}}{\pgfqpoint{1.783816in}{1.218830in}}{\pgfqpoint{1.787723in}{1.222736in}}%
\pgfpathcurveto{\pgfqpoint{1.791630in}{1.226643in}}{\pgfqpoint{1.793825in}{1.231943in}}{\pgfqpoint{1.793825in}{1.237468in}}%
\pgfpathcurveto{\pgfqpoint{1.793825in}{1.242993in}}{\pgfqpoint{1.791630in}{1.248292in}}{\pgfqpoint{1.787723in}{1.252199in}}%
\pgfpathcurveto{\pgfqpoint{1.783816in}{1.256106in}}{\pgfqpoint{1.778517in}{1.258301in}}{\pgfqpoint{1.772992in}{1.258301in}}%
\pgfpathcurveto{\pgfqpoint{1.767467in}{1.258301in}}{\pgfqpoint{1.762167in}{1.256106in}}{\pgfqpoint{1.758260in}{1.252199in}}%
\pgfpathcurveto{\pgfqpoint{1.754353in}{1.248292in}}{\pgfqpoint{1.752158in}{1.242993in}}{\pgfqpoint{1.752158in}{1.237468in}}%
\pgfpathcurveto{\pgfqpoint{1.752158in}{1.231943in}}{\pgfqpoint{1.754353in}{1.226643in}}{\pgfqpoint{1.758260in}{1.222736in}}%
\pgfpathcurveto{\pgfqpoint{1.762167in}{1.218830in}}{\pgfqpoint{1.767467in}{1.216635in}}{\pgfqpoint{1.772992in}{1.216635in}}%
\pgfpathclose%
\pgfusepath{stroke,fill}%
\end{pgfscope}%
\begin{pgfscope}%
\pgfpathrectangle{\pgfqpoint{0.562500in}{0.275000in}}{\pgfqpoint{3.487500in}{1.925000in}}%
\pgfusepath{clip}%
\pgfsetbuttcap%
\pgfsetroundjoin%
\definecolor{currentfill}{rgb}{0.000000,0.000000,0.000000}%
\pgfsetfillcolor{currentfill}%
\pgfsetlinewidth{1.003750pt}%
\definecolor{currentstroke}{rgb}{0.000000,0.000000,0.000000}%
\pgfsetstrokecolor{currentstroke}%
\pgfsetdash{}{0pt}%
\pgfpathmoveto{\pgfqpoint{1.772992in}{1.216635in}}%
\pgfpathcurveto{\pgfqpoint{1.778517in}{1.216635in}}{\pgfqpoint{1.783816in}{1.218830in}}{\pgfqpoint{1.787723in}{1.222736in}}%
\pgfpathcurveto{\pgfqpoint{1.791630in}{1.226643in}}{\pgfqpoint{1.793825in}{1.231943in}}{\pgfqpoint{1.793825in}{1.237468in}}%
\pgfpathcurveto{\pgfqpoint{1.793825in}{1.242993in}}{\pgfqpoint{1.791630in}{1.248292in}}{\pgfqpoint{1.787723in}{1.252199in}}%
\pgfpathcurveto{\pgfqpoint{1.783816in}{1.256106in}}{\pgfqpoint{1.778517in}{1.258301in}}{\pgfqpoint{1.772992in}{1.258301in}}%
\pgfpathcurveto{\pgfqpoint{1.767467in}{1.258301in}}{\pgfqpoint{1.762167in}{1.256106in}}{\pgfqpoint{1.758260in}{1.252199in}}%
\pgfpathcurveto{\pgfqpoint{1.754353in}{1.248292in}}{\pgfqpoint{1.752158in}{1.242993in}}{\pgfqpoint{1.752158in}{1.237468in}}%
\pgfpathcurveto{\pgfqpoint{1.752158in}{1.231943in}}{\pgfqpoint{1.754353in}{1.226643in}}{\pgfqpoint{1.758260in}{1.222736in}}%
\pgfpathcurveto{\pgfqpoint{1.762167in}{1.218830in}}{\pgfqpoint{1.767467in}{1.216635in}}{\pgfqpoint{1.772992in}{1.216635in}}%
\pgfpathclose%
\pgfusepath{stroke,fill}%
\end{pgfscope}%
\begin{pgfscope}%
\pgfpathrectangle{\pgfqpoint{0.562500in}{0.275000in}}{\pgfqpoint{3.487500in}{1.925000in}}%
\pgfusepath{clip}%
\pgfsetbuttcap%
\pgfsetroundjoin%
\definecolor{currentfill}{rgb}{0.000000,0.000000,0.000000}%
\pgfsetfillcolor{currentfill}%
\pgfsetlinewidth{1.003750pt}%
\definecolor{currentstroke}{rgb}{0.000000,0.000000,0.000000}%
\pgfsetstrokecolor{currentstroke}%
\pgfsetdash{}{0pt}%
\pgfpathmoveto{\pgfqpoint{1.772992in}{1.216635in}}%
\pgfpathcurveto{\pgfqpoint{1.778517in}{1.216635in}}{\pgfqpoint{1.783816in}{1.218830in}}{\pgfqpoint{1.787723in}{1.222736in}}%
\pgfpathcurveto{\pgfqpoint{1.791630in}{1.226643in}}{\pgfqpoint{1.793825in}{1.231943in}}{\pgfqpoint{1.793825in}{1.237468in}}%
\pgfpathcurveto{\pgfqpoint{1.793825in}{1.242993in}}{\pgfqpoint{1.791630in}{1.248292in}}{\pgfqpoint{1.787723in}{1.252199in}}%
\pgfpathcurveto{\pgfqpoint{1.783816in}{1.256106in}}{\pgfqpoint{1.778517in}{1.258301in}}{\pgfqpoint{1.772992in}{1.258301in}}%
\pgfpathcurveto{\pgfqpoint{1.767467in}{1.258301in}}{\pgfqpoint{1.762167in}{1.256106in}}{\pgfqpoint{1.758260in}{1.252199in}}%
\pgfpathcurveto{\pgfqpoint{1.754353in}{1.248292in}}{\pgfqpoint{1.752158in}{1.242993in}}{\pgfqpoint{1.752158in}{1.237468in}}%
\pgfpathcurveto{\pgfqpoint{1.752158in}{1.231943in}}{\pgfqpoint{1.754353in}{1.226643in}}{\pgfqpoint{1.758260in}{1.222736in}}%
\pgfpathcurveto{\pgfqpoint{1.762167in}{1.218830in}}{\pgfqpoint{1.767467in}{1.216635in}}{\pgfqpoint{1.772992in}{1.216635in}}%
\pgfpathclose%
\pgfusepath{stroke,fill}%
\end{pgfscope}%
\begin{pgfscope}%
\pgfpathrectangle{\pgfqpoint{0.562500in}{0.275000in}}{\pgfqpoint{3.487500in}{1.925000in}}%
\pgfusepath{clip}%
\pgfsetbuttcap%
\pgfsetroundjoin%
\definecolor{currentfill}{rgb}{0.000000,0.000000,0.000000}%
\pgfsetfillcolor{currentfill}%
\pgfsetlinewidth{1.003750pt}%
\definecolor{currentstroke}{rgb}{0.000000,0.000000,0.000000}%
\pgfsetstrokecolor{currentstroke}%
\pgfsetdash{}{0pt}%
\pgfpathmoveto{\pgfqpoint{1.772992in}{1.216635in}}%
\pgfpathcurveto{\pgfqpoint{1.778517in}{1.216635in}}{\pgfqpoint{1.783816in}{1.218830in}}{\pgfqpoint{1.787723in}{1.222736in}}%
\pgfpathcurveto{\pgfqpoint{1.791630in}{1.226643in}}{\pgfqpoint{1.793825in}{1.231943in}}{\pgfqpoint{1.793825in}{1.237468in}}%
\pgfpathcurveto{\pgfqpoint{1.793825in}{1.242993in}}{\pgfqpoint{1.791630in}{1.248292in}}{\pgfqpoint{1.787723in}{1.252199in}}%
\pgfpathcurveto{\pgfqpoint{1.783816in}{1.256106in}}{\pgfqpoint{1.778517in}{1.258301in}}{\pgfqpoint{1.772992in}{1.258301in}}%
\pgfpathcurveto{\pgfqpoint{1.767467in}{1.258301in}}{\pgfqpoint{1.762167in}{1.256106in}}{\pgfqpoint{1.758260in}{1.252199in}}%
\pgfpathcurveto{\pgfqpoint{1.754353in}{1.248292in}}{\pgfqpoint{1.752158in}{1.242993in}}{\pgfqpoint{1.752158in}{1.237468in}}%
\pgfpathcurveto{\pgfqpoint{1.752158in}{1.231943in}}{\pgfqpoint{1.754353in}{1.226643in}}{\pgfqpoint{1.758260in}{1.222736in}}%
\pgfpathcurveto{\pgfqpoint{1.762167in}{1.218830in}}{\pgfqpoint{1.767467in}{1.216635in}}{\pgfqpoint{1.772992in}{1.216635in}}%
\pgfpathclose%
\pgfusepath{stroke,fill}%
\end{pgfscope}%
\begin{pgfscope}%
\pgfpathrectangle{\pgfqpoint{0.562500in}{0.275000in}}{\pgfqpoint{3.487500in}{1.925000in}}%
\pgfusepath{clip}%
\pgfsetbuttcap%
\pgfsetroundjoin%
\definecolor{currentfill}{rgb}{0.000000,0.000000,0.000000}%
\pgfsetfillcolor{currentfill}%
\pgfsetlinewidth{1.003750pt}%
\definecolor{currentstroke}{rgb}{0.000000,0.000000,0.000000}%
\pgfsetstrokecolor{currentstroke}%
\pgfsetdash{}{0pt}%
\pgfpathmoveto{\pgfqpoint{1.772992in}{1.216635in}}%
\pgfpathcurveto{\pgfqpoint{1.778517in}{1.216635in}}{\pgfqpoint{1.783816in}{1.218830in}}{\pgfqpoint{1.787723in}{1.222736in}}%
\pgfpathcurveto{\pgfqpoint{1.791630in}{1.226643in}}{\pgfqpoint{1.793825in}{1.231943in}}{\pgfqpoint{1.793825in}{1.237468in}}%
\pgfpathcurveto{\pgfqpoint{1.793825in}{1.242993in}}{\pgfqpoint{1.791630in}{1.248292in}}{\pgfqpoint{1.787723in}{1.252199in}}%
\pgfpathcurveto{\pgfqpoint{1.783816in}{1.256106in}}{\pgfqpoint{1.778517in}{1.258301in}}{\pgfqpoint{1.772992in}{1.258301in}}%
\pgfpathcurveto{\pgfqpoint{1.767467in}{1.258301in}}{\pgfqpoint{1.762167in}{1.256106in}}{\pgfqpoint{1.758260in}{1.252199in}}%
\pgfpathcurveto{\pgfqpoint{1.754353in}{1.248292in}}{\pgfqpoint{1.752158in}{1.242993in}}{\pgfqpoint{1.752158in}{1.237468in}}%
\pgfpathcurveto{\pgfqpoint{1.752158in}{1.231943in}}{\pgfqpoint{1.754353in}{1.226643in}}{\pgfqpoint{1.758260in}{1.222736in}}%
\pgfpathcurveto{\pgfqpoint{1.762167in}{1.218830in}}{\pgfqpoint{1.767467in}{1.216635in}}{\pgfqpoint{1.772992in}{1.216635in}}%
\pgfpathclose%
\pgfusepath{stroke,fill}%
\end{pgfscope}%
\begin{pgfscope}%
\pgfpathrectangle{\pgfqpoint{0.562500in}{0.275000in}}{\pgfqpoint{3.487500in}{1.925000in}}%
\pgfusepath{clip}%
\pgfsetbuttcap%
\pgfsetroundjoin%
\definecolor{currentfill}{rgb}{0.000000,0.000000,0.000000}%
\pgfsetfillcolor{currentfill}%
\pgfsetlinewidth{1.003750pt}%
\definecolor{currentstroke}{rgb}{0.000000,0.000000,0.000000}%
\pgfsetstrokecolor{currentstroke}%
\pgfsetdash{}{0pt}%
\pgfpathmoveto{\pgfqpoint{1.772992in}{1.216635in}}%
\pgfpathcurveto{\pgfqpoint{1.778517in}{1.216635in}}{\pgfqpoint{1.783816in}{1.218830in}}{\pgfqpoint{1.787723in}{1.222736in}}%
\pgfpathcurveto{\pgfqpoint{1.791630in}{1.226643in}}{\pgfqpoint{1.793825in}{1.231943in}}{\pgfqpoint{1.793825in}{1.237468in}}%
\pgfpathcurveto{\pgfqpoint{1.793825in}{1.242993in}}{\pgfqpoint{1.791630in}{1.248292in}}{\pgfqpoint{1.787723in}{1.252199in}}%
\pgfpathcurveto{\pgfqpoint{1.783816in}{1.256106in}}{\pgfqpoint{1.778517in}{1.258301in}}{\pgfqpoint{1.772992in}{1.258301in}}%
\pgfpathcurveto{\pgfqpoint{1.767467in}{1.258301in}}{\pgfqpoint{1.762167in}{1.256106in}}{\pgfqpoint{1.758260in}{1.252199in}}%
\pgfpathcurveto{\pgfqpoint{1.754353in}{1.248292in}}{\pgfqpoint{1.752158in}{1.242993in}}{\pgfqpoint{1.752158in}{1.237468in}}%
\pgfpathcurveto{\pgfqpoint{1.752158in}{1.231943in}}{\pgfqpoint{1.754353in}{1.226643in}}{\pgfqpoint{1.758260in}{1.222736in}}%
\pgfpathcurveto{\pgfqpoint{1.762167in}{1.218830in}}{\pgfqpoint{1.767467in}{1.216635in}}{\pgfqpoint{1.772992in}{1.216635in}}%
\pgfpathclose%
\pgfusepath{stroke,fill}%
\end{pgfscope}%
\begin{pgfscope}%
\pgfpathrectangle{\pgfqpoint{0.562500in}{0.275000in}}{\pgfqpoint{3.487500in}{1.925000in}}%
\pgfusepath{clip}%
\pgfsetbuttcap%
\pgfsetroundjoin%
\definecolor{currentfill}{rgb}{0.000000,0.000000,0.000000}%
\pgfsetfillcolor{currentfill}%
\pgfsetlinewidth{1.003750pt}%
\definecolor{currentstroke}{rgb}{0.000000,0.000000,0.000000}%
\pgfsetstrokecolor{currentstroke}%
\pgfsetdash{}{0pt}%
\pgfpathmoveto{\pgfqpoint{1.772992in}{1.216635in}}%
\pgfpathcurveto{\pgfqpoint{1.778517in}{1.216635in}}{\pgfqpoint{1.783816in}{1.218830in}}{\pgfqpoint{1.787723in}{1.222736in}}%
\pgfpathcurveto{\pgfqpoint{1.791630in}{1.226643in}}{\pgfqpoint{1.793825in}{1.231943in}}{\pgfqpoint{1.793825in}{1.237468in}}%
\pgfpathcurveto{\pgfqpoint{1.793825in}{1.242993in}}{\pgfqpoint{1.791630in}{1.248292in}}{\pgfqpoint{1.787723in}{1.252199in}}%
\pgfpathcurveto{\pgfqpoint{1.783816in}{1.256106in}}{\pgfqpoint{1.778517in}{1.258301in}}{\pgfqpoint{1.772992in}{1.258301in}}%
\pgfpathcurveto{\pgfqpoint{1.767467in}{1.258301in}}{\pgfqpoint{1.762167in}{1.256106in}}{\pgfqpoint{1.758260in}{1.252199in}}%
\pgfpathcurveto{\pgfqpoint{1.754353in}{1.248292in}}{\pgfqpoint{1.752158in}{1.242993in}}{\pgfqpoint{1.752158in}{1.237468in}}%
\pgfpathcurveto{\pgfqpoint{1.752158in}{1.231943in}}{\pgfqpoint{1.754353in}{1.226643in}}{\pgfqpoint{1.758260in}{1.222736in}}%
\pgfpathcurveto{\pgfqpoint{1.762167in}{1.218830in}}{\pgfqpoint{1.767467in}{1.216635in}}{\pgfqpoint{1.772992in}{1.216635in}}%
\pgfpathclose%
\pgfusepath{stroke,fill}%
\end{pgfscope}%
\begin{pgfscope}%
\pgfpathrectangle{\pgfqpoint{0.562500in}{0.275000in}}{\pgfqpoint{3.487500in}{1.925000in}}%
\pgfusepath{clip}%
\pgfsetbuttcap%
\pgfsetroundjoin%
\definecolor{currentfill}{rgb}{0.000000,0.000000,0.000000}%
\pgfsetfillcolor{currentfill}%
\pgfsetlinewidth{1.003750pt}%
\definecolor{currentstroke}{rgb}{0.000000,0.000000,0.000000}%
\pgfsetstrokecolor{currentstroke}%
\pgfsetdash{}{0pt}%
\pgfpathmoveto{\pgfqpoint{1.772992in}{1.216635in}}%
\pgfpathcurveto{\pgfqpoint{1.778517in}{1.216635in}}{\pgfqpoint{1.783816in}{1.218830in}}{\pgfqpoint{1.787723in}{1.222736in}}%
\pgfpathcurveto{\pgfqpoint{1.791630in}{1.226643in}}{\pgfqpoint{1.793825in}{1.231943in}}{\pgfqpoint{1.793825in}{1.237468in}}%
\pgfpathcurveto{\pgfqpoint{1.793825in}{1.242993in}}{\pgfqpoint{1.791630in}{1.248292in}}{\pgfqpoint{1.787723in}{1.252199in}}%
\pgfpathcurveto{\pgfqpoint{1.783816in}{1.256106in}}{\pgfqpoint{1.778517in}{1.258301in}}{\pgfqpoint{1.772992in}{1.258301in}}%
\pgfpathcurveto{\pgfqpoint{1.767467in}{1.258301in}}{\pgfqpoint{1.762167in}{1.256106in}}{\pgfqpoint{1.758260in}{1.252199in}}%
\pgfpathcurveto{\pgfqpoint{1.754353in}{1.248292in}}{\pgfqpoint{1.752158in}{1.242993in}}{\pgfqpoint{1.752158in}{1.237468in}}%
\pgfpathcurveto{\pgfqpoint{1.752158in}{1.231943in}}{\pgfqpoint{1.754353in}{1.226643in}}{\pgfqpoint{1.758260in}{1.222736in}}%
\pgfpathcurveto{\pgfqpoint{1.762167in}{1.218830in}}{\pgfqpoint{1.767467in}{1.216635in}}{\pgfqpoint{1.772992in}{1.216635in}}%
\pgfpathclose%
\pgfusepath{stroke,fill}%
\end{pgfscope}%
\begin{pgfscope}%
\pgfpathrectangle{\pgfqpoint{0.562500in}{0.275000in}}{\pgfqpoint{3.487500in}{1.925000in}}%
\pgfusepath{clip}%
\pgfsetbuttcap%
\pgfsetroundjoin%
\definecolor{currentfill}{rgb}{0.000000,0.000000,0.000000}%
\pgfsetfillcolor{currentfill}%
\pgfsetlinewidth{1.003750pt}%
\definecolor{currentstroke}{rgb}{0.000000,0.000000,0.000000}%
\pgfsetstrokecolor{currentstroke}%
\pgfsetdash{}{0pt}%
\pgfpathmoveto{\pgfqpoint{1.772992in}{1.216635in}}%
\pgfpathcurveto{\pgfqpoint{1.778517in}{1.216635in}}{\pgfqpoint{1.783816in}{1.218830in}}{\pgfqpoint{1.787723in}{1.222736in}}%
\pgfpathcurveto{\pgfqpoint{1.791630in}{1.226643in}}{\pgfqpoint{1.793825in}{1.231943in}}{\pgfqpoint{1.793825in}{1.237468in}}%
\pgfpathcurveto{\pgfqpoint{1.793825in}{1.242993in}}{\pgfqpoint{1.791630in}{1.248292in}}{\pgfqpoint{1.787723in}{1.252199in}}%
\pgfpathcurveto{\pgfqpoint{1.783816in}{1.256106in}}{\pgfqpoint{1.778517in}{1.258301in}}{\pgfqpoint{1.772992in}{1.258301in}}%
\pgfpathcurveto{\pgfqpoint{1.767467in}{1.258301in}}{\pgfqpoint{1.762167in}{1.256106in}}{\pgfqpoint{1.758260in}{1.252199in}}%
\pgfpathcurveto{\pgfqpoint{1.754353in}{1.248292in}}{\pgfqpoint{1.752158in}{1.242993in}}{\pgfqpoint{1.752158in}{1.237468in}}%
\pgfpathcurveto{\pgfqpoint{1.752158in}{1.231943in}}{\pgfqpoint{1.754353in}{1.226643in}}{\pgfqpoint{1.758260in}{1.222736in}}%
\pgfpathcurveto{\pgfqpoint{1.762167in}{1.218830in}}{\pgfqpoint{1.767467in}{1.216635in}}{\pgfqpoint{1.772992in}{1.216635in}}%
\pgfpathclose%
\pgfusepath{stroke,fill}%
\end{pgfscope}%
\begin{pgfscope}%
\pgfpathrectangle{\pgfqpoint{0.562500in}{0.275000in}}{\pgfqpoint{3.487500in}{1.925000in}}%
\pgfusepath{clip}%
\pgfsetbuttcap%
\pgfsetroundjoin%
\definecolor{currentfill}{rgb}{0.000000,0.000000,0.000000}%
\pgfsetfillcolor{currentfill}%
\pgfsetlinewidth{1.003750pt}%
\definecolor{currentstroke}{rgb}{0.000000,0.000000,0.000000}%
\pgfsetstrokecolor{currentstroke}%
\pgfsetdash{}{0pt}%
\pgfpathmoveto{\pgfqpoint{1.772992in}{1.216635in}}%
\pgfpathcurveto{\pgfqpoint{1.778517in}{1.216635in}}{\pgfqpoint{1.783816in}{1.218830in}}{\pgfqpoint{1.787723in}{1.222736in}}%
\pgfpathcurveto{\pgfqpoint{1.791630in}{1.226643in}}{\pgfqpoint{1.793825in}{1.231943in}}{\pgfqpoint{1.793825in}{1.237468in}}%
\pgfpathcurveto{\pgfqpoint{1.793825in}{1.242993in}}{\pgfqpoint{1.791630in}{1.248292in}}{\pgfqpoint{1.787723in}{1.252199in}}%
\pgfpathcurveto{\pgfqpoint{1.783816in}{1.256106in}}{\pgfqpoint{1.778517in}{1.258301in}}{\pgfqpoint{1.772992in}{1.258301in}}%
\pgfpathcurveto{\pgfqpoint{1.767467in}{1.258301in}}{\pgfqpoint{1.762167in}{1.256106in}}{\pgfqpoint{1.758260in}{1.252199in}}%
\pgfpathcurveto{\pgfqpoint{1.754353in}{1.248292in}}{\pgfqpoint{1.752158in}{1.242993in}}{\pgfqpoint{1.752158in}{1.237468in}}%
\pgfpathcurveto{\pgfqpoint{1.752158in}{1.231943in}}{\pgfqpoint{1.754353in}{1.226643in}}{\pgfqpoint{1.758260in}{1.222736in}}%
\pgfpathcurveto{\pgfqpoint{1.762167in}{1.218830in}}{\pgfqpoint{1.767467in}{1.216635in}}{\pgfqpoint{1.772992in}{1.216635in}}%
\pgfpathclose%
\pgfusepath{stroke,fill}%
\end{pgfscope}%
\begin{pgfscope}%
\pgfpathrectangle{\pgfqpoint{0.562500in}{0.275000in}}{\pgfqpoint{3.487500in}{1.925000in}}%
\pgfusepath{clip}%
\pgfsetbuttcap%
\pgfsetroundjoin%
\definecolor{currentfill}{rgb}{0.000000,0.000000,0.000000}%
\pgfsetfillcolor{currentfill}%
\pgfsetlinewidth{1.003750pt}%
\definecolor{currentstroke}{rgb}{0.000000,0.000000,0.000000}%
\pgfsetstrokecolor{currentstroke}%
\pgfsetdash{}{0pt}%
\pgfpathmoveto{\pgfqpoint{1.772992in}{1.216635in}}%
\pgfpathcurveto{\pgfqpoint{1.778517in}{1.216635in}}{\pgfqpoint{1.783816in}{1.218830in}}{\pgfqpoint{1.787723in}{1.222736in}}%
\pgfpathcurveto{\pgfqpoint{1.791630in}{1.226643in}}{\pgfqpoint{1.793825in}{1.231943in}}{\pgfqpoint{1.793825in}{1.237468in}}%
\pgfpathcurveto{\pgfqpoint{1.793825in}{1.242993in}}{\pgfqpoint{1.791630in}{1.248292in}}{\pgfqpoint{1.787723in}{1.252199in}}%
\pgfpathcurveto{\pgfqpoint{1.783816in}{1.256106in}}{\pgfqpoint{1.778517in}{1.258301in}}{\pgfqpoint{1.772992in}{1.258301in}}%
\pgfpathcurveto{\pgfqpoint{1.767467in}{1.258301in}}{\pgfqpoint{1.762167in}{1.256106in}}{\pgfqpoint{1.758260in}{1.252199in}}%
\pgfpathcurveto{\pgfqpoint{1.754353in}{1.248292in}}{\pgfqpoint{1.752158in}{1.242993in}}{\pgfqpoint{1.752158in}{1.237468in}}%
\pgfpathcurveto{\pgfqpoint{1.752158in}{1.231943in}}{\pgfqpoint{1.754353in}{1.226643in}}{\pgfqpoint{1.758260in}{1.222736in}}%
\pgfpathcurveto{\pgfqpoint{1.762167in}{1.218830in}}{\pgfqpoint{1.767467in}{1.216635in}}{\pgfqpoint{1.772992in}{1.216635in}}%
\pgfpathclose%
\pgfusepath{stroke,fill}%
\end{pgfscope}%
\begin{pgfscope}%
\pgfpathrectangle{\pgfqpoint{0.562500in}{0.275000in}}{\pgfqpoint{3.487500in}{1.925000in}}%
\pgfusepath{clip}%
\pgfsetbuttcap%
\pgfsetroundjoin%
\definecolor{currentfill}{rgb}{0.000000,0.000000,0.000000}%
\pgfsetfillcolor{currentfill}%
\pgfsetlinewidth{1.003750pt}%
\definecolor{currentstroke}{rgb}{0.000000,0.000000,0.000000}%
\pgfsetstrokecolor{currentstroke}%
\pgfsetdash{}{0pt}%
\pgfpathmoveto{\pgfqpoint{1.772992in}{1.216635in}}%
\pgfpathcurveto{\pgfqpoint{1.778517in}{1.216635in}}{\pgfqpoint{1.783816in}{1.218830in}}{\pgfqpoint{1.787723in}{1.222736in}}%
\pgfpathcurveto{\pgfqpoint{1.791630in}{1.226643in}}{\pgfqpoint{1.793825in}{1.231943in}}{\pgfqpoint{1.793825in}{1.237468in}}%
\pgfpathcurveto{\pgfqpoint{1.793825in}{1.242993in}}{\pgfqpoint{1.791630in}{1.248292in}}{\pgfqpoint{1.787723in}{1.252199in}}%
\pgfpathcurveto{\pgfqpoint{1.783816in}{1.256106in}}{\pgfqpoint{1.778517in}{1.258301in}}{\pgfqpoint{1.772992in}{1.258301in}}%
\pgfpathcurveto{\pgfqpoint{1.767467in}{1.258301in}}{\pgfqpoint{1.762167in}{1.256106in}}{\pgfqpoint{1.758260in}{1.252199in}}%
\pgfpathcurveto{\pgfqpoint{1.754353in}{1.248292in}}{\pgfqpoint{1.752158in}{1.242993in}}{\pgfqpoint{1.752158in}{1.237468in}}%
\pgfpathcurveto{\pgfqpoint{1.752158in}{1.231943in}}{\pgfqpoint{1.754353in}{1.226643in}}{\pgfqpoint{1.758260in}{1.222736in}}%
\pgfpathcurveto{\pgfqpoint{1.762167in}{1.218830in}}{\pgfqpoint{1.767467in}{1.216635in}}{\pgfqpoint{1.772992in}{1.216635in}}%
\pgfpathclose%
\pgfusepath{stroke,fill}%
\end{pgfscope}%
\begin{pgfscope}%
\pgfpathrectangle{\pgfqpoint{0.562500in}{0.275000in}}{\pgfqpoint{3.487500in}{1.925000in}}%
\pgfusepath{clip}%
\pgfsetbuttcap%
\pgfsetroundjoin%
\definecolor{currentfill}{rgb}{0.000000,0.000000,0.000000}%
\pgfsetfillcolor{currentfill}%
\pgfsetlinewidth{1.003750pt}%
\definecolor{currentstroke}{rgb}{0.000000,0.000000,0.000000}%
\pgfsetstrokecolor{currentstroke}%
\pgfsetdash{}{0pt}%
\pgfpathmoveto{\pgfqpoint{1.772992in}{1.216635in}}%
\pgfpathcurveto{\pgfqpoint{1.778517in}{1.216635in}}{\pgfqpoint{1.783816in}{1.218830in}}{\pgfqpoint{1.787723in}{1.222736in}}%
\pgfpathcurveto{\pgfqpoint{1.791630in}{1.226643in}}{\pgfqpoint{1.793825in}{1.231943in}}{\pgfqpoint{1.793825in}{1.237468in}}%
\pgfpathcurveto{\pgfqpoint{1.793825in}{1.242993in}}{\pgfqpoint{1.791630in}{1.248292in}}{\pgfqpoint{1.787723in}{1.252199in}}%
\pgfpathcurveto{\pgfqpoint{1.783816in}{1.256106in}}{\pgfqpoint{1.778517in}{1.258301in}}{\pgfqpoint{1.772992in}{1.258301in}}%
\pgfpathcurveto{\pgfqpoint{1.767467in}{1.258301in}}{\pgfqpoint{1.762167in}{1.256106in}}{\pgfqpoint{1.758260in}{1.252199in}}%
\pgfpathcurveto{\pgfqpoint{1.754353in}{1.248292in}}{\pgfqpoint{1.752158in}{1.242993in}}{\pgfqpoint{1.752158in}{1.237468in}}%
\pgfpathcurveto{\pgfqpoint{1.752158in}{1.231943in}}{\pgfqpoint{1.754353in}{1.226643in}}{\pgfqpoint{1.758260in}{1.222736in}}%
\pgfpathcurveto{\pgfqpoint{1.762167in}{1.218830in}}{\pgfqpoint{1.767467in}{1.216635in}}{\pgfqpoint{1.772992in}{1.216635in}}%
\pgfpathclose%
\pgfusepath{stroke,fill}%
\end{pgfscope}%
\begin{pgfscope}%
\pgfpathrectangle{\pgfqpoint{0.562500in}{0.275000in}}{\pgfqpoint{3.487500in}{1.925000in}}%
\pgfusepath{clip}%
\pgfsetbuttcap%
\pgfsetroundjoin%
\definecolor{currentfill}{rgb}{0.000000,0.000000,0.000000}%
\pgfsetfillcolor{currentfill}%
\pgfsetlinewidth{1.003750pt}%
\definecolor{currentstroke}{rgb}{0.000000,0.000000,0.000000}%
\pgfsetstrokecolor{currentstroke}%
\pgfsetdash{}{0pt}%
\pgfpathmoveto{\pgfqpoint{1.772992in}{1.216635in}}%
\pgfpathcurveto{\pgfqpoint{1.778517in}{1.216635in}}{\pgfqpoint{1.783816in}{1.218830in}}{\pgfqpoint{1.787723in}{1.222736in}}%
\pgfpathcurveto{\pgfqpoint{1.791630in}{1.226643in}}{\pgfqpoint{1.793825in}{1.231943in}}{\pgfqpoint{1.793825in}{1.237468in}}%
\pgfpathcurveto{\pgfqpoint{1.793825in}{1.242993in}}{\pgfqpoint{1.791630in}{1.248292in}}{\pgfqpoint{1.787723in}{1.252199in}}%
\pgfpathcurveto{\pgfqpoint{1.783816in}{1.256106in}}{\pgfqpoint{1.778517in}{1.258301in}}{\pgfqpoint{1.772992in}{1.258301in}}%
\pgfpathcurveto{\pgfqpoint{1.767467in}{1.258301in}}{\pgfqpoint{1.762167in}{1.256106in}}{\pgfqpoint{1.758260in}{1.252199in}}%
\pgfpathcurveto{\pgfqpoint{1.754353in}{1.248292in}}{\pgfqpoint{1.752158in}{1.242993in}}{\pgfqpoint{1.752158in}{1.237468in}}%
\pgfpathcurveto{\pgfqpoint{1.752158in}{1.231943in}}{\pgfqpoint{1.754353in}{1.226643in}}{\pgfqpoint{1.758260in}{1.222736in}}%
\pgfpathcurveto{\pgfqpoint{1.762167in}{1.218830in}}{\pgfqpoint{1.767467in}{1.216635in}}{\pgfqpoint{1.772992in}{1.216635in}}%
\pgfpathclose%
\pgfusepath{stroke,fill}%
\end{pgfscope}%
\begin{pgfscope}%
\pgfpathrectangle{\pgfqpoint{0.562500in}{0.275000in}}{\pgfqpoint{3.487500in}{1.925000in}}%
\pgfusepath{clip}%
\pgfsetbuttcap%
\pgfsetroundjoin%
\definecolor{currentfill}{rgb}{0.000000,0.000000,0.000000}%
\pgfsetfillcolor{currentfill}%
\pgfsetlinewidth{1.003750pt}%
\definecolor{currentstroke}{rgb}{0.000000,0.000000,0.000000}%
\pgfsetstrokecolor{currentstroke}%
\pgfsetdash{}{0pt}%
\pgfpathmoveto{\pgfqpoint{1.772992in}{1.216635in}}%
\pgfpathcurveto{\pgfqpoint{1.778517in}{1.216635in}}{\pgfqpoint{1.783816in}{1.218830in}}{\pgfqpoint{1.787723in}{1.222736in}}%
\pgfpathcurveto{\pgfqpoint{1.791630in}{1.226643in}}{\pgfqpoint{1.793825in}{1.231943in}}{\pgfqpoint{1.793825in}{1.237468in}}%
\pgfpathcurveto{\pgfqpoint{1.793825in}{1.242993in}}{\pgfqpoint{1.791630in}{1.248292in}}{\pgfqpoint{1.787723in}{1.252199in}}%
\pgfpathcurveto{\pgfqpoint{1.783816in}{1.256106in}}{\pgfqpoint{1.778517in}{1.258301in}}{\pgfqpoint{1.772992in}{1.258301in}}%
\pgfpathcurveto{\pgfqpoint{1.767467in}{1.258301in}}{\pgfqpoint{1.762167in}{1.256106in}}{\pgfqpoint{1.758260in}{1.252199in}}%
\pgfpathcurveto{\pgfqpoint{1.754353in}{1.248292in}}{\pgfqpoint{1.752158in}{1.242993in}}{\pgfqpoint{1.752158in}{1.237468in}}%
\pgfpathcurveto{\pgfqpoint{1.752158in}{1.231943in}}{\pgfqpoint{1.754353in}{1.226643in}}{\pgfqpoint{1.758260in}{1.222736in}}%
\pgfpathcurveto{\pgfqpoint{1.762167in}{1.218830in}}{\pgfqpoint{1.767467in}{1.216635in}}{\pgfqpoint{1.772992in}{1.216635in}}%
\pgfpathclose%
\pgfusepath{stroke,fill}%
\end{pgfscope}%
\begin{pgfscope}%
\pgfpathrectangle{\pgfqpoint{0.562500in}{0.275000in}}{\pgfqpoint{3.487500in}{1.925000in}}%
\pgfusepath{clip}%
\pgfsetbuttcap%
\pgfsetroundjoin%
\definecolor{currentfill}{rgb}{0.000000,0.000000,0.000000}%
\pgfsetfillcolor{currentfill}%
\pgfsetlinewidth{1.003750pt}%
\definecolor{currentstroke}{rgb}{0.000000,0.000000,0.000000}%
\pgfsetstrokecolor{currentstroke}%
\pgfsetdash{}{0pt}%
\pgfpathmoveto{\pgfqpoint{1.772992in}{1.216635in}}%
\pgfpathcurveto{\pgfqpoint{1.778517in}{1.216635in}}{\pgfqpoint{1.783816in}{1.218830in}}{\pgfqpoint{1.787723in}{1.222736in}}%
\pgfpathcurveto{\pgfqpoint{1.791630in}{1.226643in}}{\pgfqpoint{1.793825in}{1.231943in}}{\pgfqpoint{1.793825in}{1.237468in}}%
\pgfpathcurveto{\pgfqpoint{1.793825in}{1.242993in}}{\pgfqpoint{1.791630in}{1.248292in}}{\pgfqpoint{1.787723in}{1.252199in}}%
\pgfpathcurveto{\pgfqpoint{1.783816in}{1.256106in}}{\pgfqpoint{1.778517in}{1.258301in}}{\pgfqpoint{1.772992in}{1.258301in}}%
\pgfpathcurveto{\pgfqpoint{1.767467in}{1.258301in}}{\pgfqpoint{1.762167in}{1.256106in}}{\pgfqpoint{1.758260in}{1.252199in}}%
\pgfpathcurveto{\pgfqpoint{1.754353in}{1.248292in}}{\pgfqpoint{1.752158in}{1.242993in}}{\pgfqpoint{1.752158in}{1.237468in}}%
\pgfpathcurveto{\pgfqpoint{1.752158in}{1.231943in}}{\pgfqpoint{1.754353in}{1.226643in}}{\pgfqpoint{1.758260in}{1.222736in}}%
\pgfpathcurveto{\pgfqpoint{1.762167in}{1.218830in}}{\pgfqpoint{1.767467in}{1.216635in}}{\pgfqpoint{1.772992in}{1.216635in}}%
\pgfpathclose%
\pgfusepath{stroke,fill}%
\end{pgfscope}%
\begin{pgfscope}%
\pgfpathrectangle{\pgfqpoint{0.562500in}{0.275000in}}{\pgfqpoint{3.487500in}{1.925000in}}%
\pgfusepath{clip}%
\pgfsetbuttcap%
\pgfsetroundjoin%
\definecolor{currentfill}{rgb}{0.000000,0.000000,0.000000}%
\pgfsetfillcolor{currentfill}%
\pgfsetlinewidth{1.003750pt}%
\definecolor{currentstroke}{rgb}{0.000000,0.000000,0.000000}%
\pgfsetstrokecolor{currentstroke}%
\pgfsetdash{}{0pt}%
\pgfpathmoveto{\pgfqpoint{1.772992in}{1.216635in}}%
\pgfpathcurveto{\pgfqpoint{1.778517in}{1.216635in}}{\pgfqpoint{1.783816in}{1.218830in}}{\pgfqpoint{1.787723in}{1.222736in}}%
\pgfpathcurveto{\pgfqpoint{1.791630in}{1.226643in}}{\pgfqpoint{1.793825in}{1.231943in}}{\pgfqpoint{1.793825in}{1.237468in}}%
\pgfpathcurveto{\pgfqpoint{1.793825in}{1.242993in}}{\pgfqpoint{1.791630in}{1.248292in}}{\pgfqpoint{1.787723in}{1.252199in}}%
\pgfpathcurveto{\pgfqpoint{1.783816in}{1.256106in}}{\pgfqpoint{1.778517in}{1.258301in}}{\pgfqpoint{1.772992in}{1.258301in}}%
\pgfpathcurveto{\pgfqpoint{1.767467in}{1.258301in}}{\pgfqpoint{1.762167in}{1.256106in}}{\pgfqpoint{1.758260in}{1.252199in}}%
\pgfpathcurveto{\pgfqpoint{1.754353in}{1.248292in}}{\pgfqpoint{1.752158in}{1.242993in}}{\pgfqpoint{1.752158in}{1.237468in}}%
\pgfpathcurveto{\pgfqpoint{1.752158in}{1.231943in}}{\pgfqpoint{1.754353in}{1.226643in}}{\pgfqpoint{1.758260in}{1.222736in}}%
\pgfpathcurveto{\pgfqpoint{1.762167in}{1.218830in}}{\pgfqpoint{1.767467in}{1.216635in}}{\pgfqpoint{1.772992in}{1.216635in}}%
\pgfpathclose%
\pgfusepath{stroke,fill}%
\end{pgfscope}%
\begin{pgfscope}%
\pgfpathrectangle{\pgfqpoint{0.562500in}{0.275000in}}{\pgfqpoint{3.487500in}{1.925000in}}%
\pgfusepath{clip}%
\pgfsetbuttcap%
\pgfsetroundjoin%
\definecolor{currentfill}{rgb}{0.000000,0.000000,0.000000}%
\pgfsetfillcolor{currentfill}%
\pgfsetlinewidth{1.003750pt}%
\definecolor{currentstroke}{rgb}{0.000000,0.000000,0.000000}%
\pgfsetstrokecolor{currentstroke}%
\pgfsetdash{}{0pt}%
\pgfpathmoveto{\pgfqpoint{1.772992in}{1.216635in}}%
\pgfpathcurveto{\pgfqpoint{1.778517in}{1.216635in}}{\pgfqpoint{1.783816in}{1.218830in}}{\pgfqpoint{1.787723in}{1.222736in}}%
\pgfpathcurveto{\pgfqpoint{1.791630in}{1.226643in}}{\pgfqpoint{1.793825in}{1.231943in}}{\pgfqpoint{1.793825in}{1.237468in}}%
\pgfpathcurveto{\pgfqpoint{1.793825in}{1.242993in}}{\pgfqpoint{1.791630in}{1.248292in}}{\pgfqpoint{1.787723in}{1.252199in}}%
\pgfpathcurveto{\pgfqpoint{1.783816in}{1.256106in}}{\pgfqpoint{1.778517in}{1.258301in}}{\pgfqpoint{1.772992in}{1.258301in}}%
\pgfpathcurveto{\pgfqpoint{1.767467in}{1.258301in}}{\pgfqpoint{1.762167in}{1.256106in}}{\pgfqpoint{1.758260in}{1.252199in}}%
\pgfpathcurveto{\pgfqpoint{1.754353in}{1.248292in}}{\pgfqpoint{1.752158in}{1.242993in}}{\pgfqpoint{1.752158in}{1.237468in}}%
\pgfpathcurveto{\pgfqpoint{1.752158in}{1.231943in}}{\pgfqpoint{1.754353in}{1.226643in}}{\pgfqpoint{1.758260in}{1.222736in}}%
\pgfpathcurveto{\pgfqpoint{1.762167in}{1.218830in}}{\pgfqpoint{1.767467in}{1.216635in}}{\pgfqpoint{1.772992in}{1.216635in}}%
\pgfpathclose%
\pgfusepath{stroke,fill}%
\end{pgfscope}%
\begin{pgfscope}%
\pgfpathrectangle{\pgfqpoint{0.562500in}{0.275000in}}{\pgfqpoint{3.487500in}{1.925000in}}%
\pgfusepath{clip}%
\pgfsetbuttcap%
\pgfsetroundjoin%
\definecolor{currentfill}{rgb}{0.000000,0.000000,0.000000}%
\pgfsetfillcolor{currentfill}%
\pgfsetlinewidth{1.003750pt}%
\definecolor{currentstroke}{rgb}{0.000000,0.000000,0.000000}%
\pgfsetstrokecolor{currentstroke}%
\pgfsetdash{}{0pt}%
\pgfpathmoveto{\pgfqpoint{1.772992in}{1.216635in}}%
\pgfpathcurveto{\pgfqpoint{1.778517in}{1.216635in}}{\pgfqpoint{1.783816in}{1.218830in}}{\pgfqpoint{1.787723in}{1.222736in}}%
\pgfpathcurveto{\pgfqpoint{1.791630in}{1.226643in}}{\pgfqpoint{1.793825in}{1.231943in}}{\pgfqpoint{1.793825in}{1.237468in}}%
\pgfpathcurveto{\pgfqpoint{1.793825in}{1.242993in}}{\pgfqpoint{1.791630in}{1.248292in}}{\pgfqpoint{1.787723in}{1.252199in}}%
\pgfpathcurveto{\pgfqpoint{1.783816in}{1.256106in}}{\pgfqpoint{1.778517in}{1.258301in}}{\pgfqpoint{1.772992in}{1.258301in}}%
\pgfpathcurveto{\pgfqpoint{1.767467in}{1.258301in}}{\pgfqpoint{1.762167in}{1.256106in}}{\pgfqpoint{1.758260in}{1.252199in}}%
\pgfpathcurveto{\pgfqpoint{1.754353in}{1.248292in}}{\pgfqpoint{1.752158in}{1.242993in}}{\pgfqpoint{1.752158in}{1.237468in}}%
\pgfpathcurveto{\pgfqpoint{1.752158in}{1.231943in}}{\pgfqpoint{1.754353in}{1.226643in}}{\pgfqpoint{1.758260in}{1.222736in}}%
\pgfpathcurveto{\pgfqpoint{1.762167in}{1.218830in}}{\pgfqpoint{1.767467in}{1.216635in}}{\pgfqpoint{1.772992in}{1.216635in}}%
\pgfpathclose%
\pgfusepath{stroke,fill}%
\end{pgfscope}%
\begin{pgfscope}%
\pgfpathrectangle{\pgfqpoint{0.562500in}{0.275000in}}{\pgfqpoint{3.487500in}{1.925000in}}%
\pgfusepath{clip}%
\pgfsetbuttcap%
\pgfsetroundjoin%
\definecolor{currentfill}{rgb}{0.000000,0.000000,0.000000}%
\pgfsetfillcolor{currentfill}%
\pgfsetlinewidth{1.003750pt}%
\definecolor{currentstroke}{rgb}{0.000000,0.000000,0.000000}%
\pgfsetstrokecolor{currentstroke}%
\pgfsetdash{}{0pt}%
\pgfpathmoveto{\pgfqpoint{1.772992in}{1.216635in}}%
\pgfpathcurveto{\pgfqpoint{1.778517in}{1.216635in}}{\pgfqpoint{1.783816in}{1.218830in}}{\pgfqpoint{1.787723in}{1.222736in}}%
\pgfpathcurveto{\pgfqpoint{1.791630in}{1.226643in}}{\pgfqpoint{1.793825in}{1.231943in}}{\pgfqpoint{1.793825in}{1.237468in}}%
\pgfpathcurveto{\pgfqpoint{1.793825in}{1.242993in}}{\pgfqpoint{1.791630in}{1.248292in}}{\pgfqpoint{1.787723in}{1.252199in}}%
\pgfpathcurveto{\pgfqpoint{1.783816in}{1.256106in}}{\pgfqpoint{1.778517in}{1.258301in}}{\pgfqpoint{1.772992in}{1.258301in}}%
\pgfpathcurveto{\pgfqpoint{1.767467in}{1.258301in}}{\pgfqpoint{1.762167in}{1.256106in}}{\pgfqpoint{1.758260in}{1.252199in}}%
\pgfpathcurveto{\pgfqpoint{1.754353in}{1.248292in}}{\pgfqpoint{1.752158in}{1.242993in}}{\pgfqpoint{1.752158in}{1.237468in}}%
\pgfpathcurveto{\pgfqpoint{1.752158in}{1.231943in}}{\pgfqpoint{1.754353in}{1.226643in}}{\pgfqpoint{1.758260in}{1.222736in}}%
\pgfpathcurveto{\pgfqpoint{1.762167in}{1.218830in}}{\pgfqpoint{1.767467in}{1.216635in}}{\pgfqpoint{1.772992in}{1.216635in}}%
\pgfpathclose%
\pgfusepath{stroke,fill}%
\end{pgfscope}%
\begin{pgfscope}%
\pgfpathrectangle{\pgfqpoint{0.562500in}{0.275000in}}{\pgfqpoint{3.487500in}{1.925000in}}%
\pgfusepath{clip}%
\pgfsetbuttcap%
\pgfsetroundjoin%
\definecolor{currentfill}{rgb}{0.000000,0.000000,0.000000}%
\pgfsetfillcolor{currentfill}%
\pgfsetlinewidth{1.003750pt}%
\definecolor{currentstroke}{rgb}{0.000000,0.000000,0.000000}%
\pgfsetstrokecolor{currentstroke}%
\pgfsetdash{}{0pt}%
\pgfpathmoveto{\pgfqpoint{1.772992in}{1.216635in}}%
\pgfpathcurveto{\pgfqpoint{1.778517in}{1.216635in}}{\pgfqpoint{1.783816in}{1.218830in}}{\pgfqpoint{1.787723in}{1.222736in}}%
\pgfpathcurveto{\pgfqpoint{1.791630in}{1.226643in}}{\pgfqpoint{1.793825in}{1.231943in}}{\pgfqpoint{1.793825in}{1.237468in}}%
\pgfpathcurveto{\pgfqpoint{1.793825in}{1.242993in}}{\pgfqpoint{1.791630in}{1.248292in}}{\pgfqpoint{1.787723in}{1.252199in}}%
\pgfpathcurveto{\pgfqpoint{1.783816in}{1.256106in}}{\pgfqpoint{1.778517in}{1.258301in}}{\pgfqpoint{1.772992in}{1.258301in}}%
\pgfpathcurveto{\pgfqpoint{1.767467in}{1.258301in}}{\pgfqpoint{1.762167in}{1.256106in}}{\pgfqpoint{1.758260in}{1.252199in}}%
\pgfpathcurveto{\pgfqpoint{1.754353in}{1.248292in}}{\pgfqpoint{1.752158in}{1.242993in}}{\pgfqpoint{1.752158in}{1.237468in}}%
\pgfpathcurveto{\pgfqpoint{1.752158in}{1.231943in}}{\pgfqpoint{1.754353in}{1.226643in}}{\pgfqpoint{1.758260in}{1.222736in}}%
\pgfpathcurveto{\pgfqpoint{1.762167in}{1.218830in}}{\pgfqpoint{1.767467in}{1.216635in}}{\pgfqpoint{1.772992in}{1.216635in}}%
\pgfpathclose%
\pgfusepath{stroke,fill}%
\end{pgfscope}%
\begin{pgfscope}%
\pgfpathrectangle{\pgfqpoint{0.562500in}{0.275000in}}{\pgfqpoint{3.487500in}{1.925000in}}%
\pgfusepath{clip}%
\pgfsetbuttcap%
\pgfsetroundjoin%
\definecolor{currentfill}{rgb}{0.000000,0.000000,0.000000}%
\pgfsetfillcolor{currentfill}%
\pgfsetlinewidth{1.003750pt}%
\definecolor{currentstroke}{rgb}{0.000000,0.000000,0.000000}%
\pgfsetstrokecolor{currentstroke}%
\pgfsetdash{}{0pt}%
\pgfpathmoveto{\pgfqpoint{1.772992in}{1.216635in}}%
\pgfpathcurveto{\pgfqpoint{1.778517in}{1.216635in}}{\pgfqpoint{1.783816in}{1.218830in}}{\pgfqpoint{1.787723in}{1.222736in}}%
\pgfpathcurveto{\pgfqpoint{1.791630in}{1.226643in}}{\pgfqpoint{1.793825in}{1.231943in}}{\pgfqpoint{1.793825in}{1.237468in}}%
\pgfpathcurveto{\pgfqpoint{1.793825in}{1.242993in}}{\pgfqpoint{1.791630in}{1.248292in}}{\pgfqpoint{1.787723in}{1.252199in}}%
\pgfpathcurveto{\pgfqpoint{1.783816in}{1.256106in}}{\pgfqpoint{1.778517in}{1.258301in}}{\pgfqpoint{1.772992in}{1.258301in}}%
\pgfpathcurveto{\pgfqpoint{1.767467in}{1.258301in}}{\pgfqpoint{1.762167in}{1.256106in}}{\pgfqpoint{1.758260in}{1.252199in}}%
\pgfpathcurveto{\pgfqpoint{1.754353in}{1.248292in}}{\pgfqpoint{1.752158in}{1.242993in}}{\pgfqpoint{1.752158in}{1.237468in}}%
\pgfpathcurveto{\pgfqpoint{1.752158in}{1.231943in}}{\pgfqpoint{1.754353in}{1.226643in}}{\pgfqpoint{1.758260in}{1.222736in}}%
\pgfpathcurveto{\pgfqpoint{1.762167in}{1.218830in}}{\pgfqpoint{1.767467in}{1.216635in}}{\pgfqpoint{1.772992in}{1.216635in}}%
\pgfpathclose%
\pgfusepath{stroke,fill}%
\end{pgfscope}%
\begin{pgfscope}%
\pgfpathrectangle{\pgfqpoint{0.562500in}{0.275000in}}{\pgfqpoint{3.487500in}{1.925000in}}%
\pgfusepath{clip}%
\pgfsetbuttcap%
\pgfsetroundjoin%
\definecolor{currentfill}{rgb}{0.000000,0.000000,0.000000}%
\pgfsetfillcolor{currentfill}%
\pgfsetlinewidth{1.003750pt}%
\definecolor{currentstroke}{rgb}{0.000000,0.000000,0.000000}%
\pgfsetstrokecolor{currentstroke}%
\pgfsetdash{}{0pt}%
\pgfpathmoveto{\pgfqpoint{1.772992in}{1.216635in}}%
\pgfpathcurveto{\pgfqpoint{1.778517in}{1.216635in}}{\pgfqpoint{1.783816in}{1.218830in}}{\pgfqpoint{1.787723in}{1.222736in}}%
\pgfpathcurveto{\pgfqpoint{1.791630in}{1.226643in}}{\pgfqpoint{1.793825in}{1.231943in}}{\pgfqpoint{1.793825in}{1.237468in}}%
\pgfpathcurveto{\pgfqpoint{1.793825in}{1.242993in}}{\pgfqpoint{1.791630in}{1.248292in}}{\pgfqpoint{1.787723in}{1.252199in}}%
\pgfpathcurveto{\pgfqpoint{1.783816in}{1.256106in}}{\pgfqpoint{1.778517in}{1.258301in}}{\pgfqpoint{1.772992in}{1.258301in}}%
\pgfpathcurveto{\pgfqpoint{1.767467in}{1.258301in}}{\pgfqpoint{1.762167in}{1.256106in}}{\pgfqpoint{1.758260in}{1.252199in}}%
\pgfpathcurveto{\pgfqpoint{1.754353in}{1.248292in}}{\pgfqpoint{1.752158in}{1.242993in}}{\pgfqpoint{1.752158in}{1.237468in}}%
\pgfpathcurveto{\pgfqpoint{1.752158in}{1.231943in}}{\pgfqpoint{1.754353in}{1.226643in}}{\pgfqpoint{1.758260in}{1.222736in}}%
\pgfpathcurveto{\pgfqpoint{1.762167in}{1.218830in}}{\pgfqpoint{1.767467in}{1.216635in}}{\pgfqpoint{1.772992in}{1.216635in}}%
\pgfpathclose%
\pgfusepath{stroke,fill}%
\end{pgfscope}%
\begin{pgfscope}%
\pgfpathrectangle{\pgfqpoint{0.562500in}{0.275000in}}{\pgfqpoint{3.487500in}{1.925000in}}%
\pgfusepath{clip}%
\pgfsetbuttcap%
\pgfsetroundjoin%
\definecolor{currentfill}{rgb}{0.000000,0.000000,0.000000}%
\pgfsetfillcolor{currentfill}%
\pgfsetlinewidth{1.003750pt}%
\definecolor{currentstroke}{rgb}{0.000000,0.000000,0.000000}%
\pgfsetstrokecolor{currentstroke}%
\pgfsetdash{}{0pt}%
\pgfpathmoveto{\pgfqpoint{1.772992in}{2.076667in}}%
\pgfpathcurveto{\pgfqpoint{1.778517in}{2.076667in}}{\pgfqpoint{1.783816in}{2.078862in}}{\pgfqpoint{1.787723in}{2.082769in}}%
\pgfpathcurveto{\pgfqpoint{1.791630in}{2.086675in}}{\pgfqpoint{1.793825in}{2.091975in}}{\pgfqpoint{1.793825in}{2.097500in}}%
\pgfpathcurveto{\pgfqpoint{1.793825in}{2.103025in}}{\pgfqpoint{1.791630in}{2.108325in}}{\pgfqpoint{1.787723in}{2.112231in}}%
\pgfpathcurveto{\pgfqpoint{1.783816in}{2.116138in}}{\pgfqpoint{1.778517in}{2.118333in}}{\pgfqpoint{1.772992in}{2.118333in}}%
\pgfpathcurveto{\pgfqpoint{1.767467in}{2.118333in}}{\pgfqpoint{1.762167in}{2.116138in}}{\pgfqpoint{1.758260in}{2.112231in}}%
\pgfpathcurveto{\pgfqpoint{1.754353in}{2.108325in}}{\pgfqpoint{1.752158in}{2.103025in}}{\pgfqpoint{1.752158in}{2.097500in}}%
\pgfpathcurveto{\pgfqpoint{1.752158in}{2.091975in}}{\pgfqpoint{1.754353in}{2.086675in}}{\pgfqpoint{1.758260in}{2.082769in}}%
\pgfpathcurveto{\pgfqpoint{1.762167in}{2.078862in}}{\pgfqpoint{1.767467in}{2.076667in}}{\pgfqpoint{1.772992in}{2.076667in}}%
\pgfpathclose%
\pgfusepath{stroke,fill}%
\end{pgfscope}%
\begin{pgfscope}%
\pgfpathrectangle{\pgfqpoint{0.562500in}{0.275000in}}{\pgfqpoint{3.487500in}{1.925000in}}%
\pgfusepath{clip}%
\pgfsetbuttcap%
\pgfsetroundjoin%
\definecolor{currentfill}{rgb}{0.000000,0.000000,0.000000}%
\pgfsetfillcolor{currentfill}%
\pgfsetlinewidth{1.003750pt}%
\definecolor{currentstroke}{rgb}{0.000000,0.000000,0.000000}%
\pgfsetstrokecolor{currentstroke}%
\pgfsetdash{}{0pt}%
\pgfpathmoveto{\pgfqpoint{1.772992in}{1.216635in}}%
\pgfpathcurveto{\pgfqpoint{1.778517in}{1.216635in}}{\pgfqpoint{1.783816in}{1.218830in}}{\pgfqpoint{1.787723in}{1.222736in}}%
\pgfpathcurveto{\pgfqpoint{1.791630in}{1.226643in}}{\pgfqpoint{1.793825in}{1.231943in}}{\pgfqpoint{1.793825in}{1.237468in}}%
\pgfpathcurveto{\pgfqpoint{1.793825in}{1.242993in}}{\pgfqpoint{1.791630in}{1.248292in}}{\pgfqpoint{1.787723in}{1.252199in}}%
\pgfpathcurveto{\pgfqpoint{1.783816in}{1.256106in}}{\pgfqpoint{1.778517in}{1.258301in}}{\pgfqpoint{1.772992in}{1.258301in}}%
\pgfpathcurveto{\pgfqpoint{1.767467in}{1.258301in}}{\pgfqpoint{1.762167in}{1.256106in}}{\pgfqpoint{1.758260in}{1.252199in}}%
\pgfpathcurveto{\pgfqpoint{1.754353in}{1.248292in}}{\pgfqpoint{1.752158in}{1.242993in}}{\pgfqpoint{1.752158in}{1.237468in}}%
\pgfpathcurveto{\pgfqpoint{1.752158in}{1.231943in}}{\pgfqpoint{1.754353in}{1.226643in}}{\pgfqpoint{1.758260in}{1.222736in}}%
\pgfpathcurveto{\pgfqpoint{1.762167in}{1.218830in}}{\pgfqpoint{1.767467in}{1.216635in}}{\pgfqpoint{1.772992in}{1.216635in}}%
\pgfpathclose%
\pgfusepath{stroke,fill}%
\end{pgfscope}%
\begin{pgfscope}%
\pgfpathrectangle{\pgfqpoint{0.562500in}{0.275000in}}{\pgfqpoint{3.487500in}{1.925000in}}%
\pgfusepath{clip}%
\pgfsetbuttcap%
\pgfsetroundjoin%
\definecolor{currentfill}{rgb}{0.000000,0.000000,0.000000}%
\pgfsetfillcolor{currentfill}%
\pgfsetlinewidth{1.003750pt}%
\definecolor{currentstroke}{rgb}{0.000000,0.000000,0.000000}%
\pgfsetstrokecolor{currentstroke}%
\pgfsetdash{}{0pt}%
\pgfpathmoveto{\pgfqpoint{1.772992in}{1.216635in}}%
\pgfpathcurveto{\pgfqpoint{1.778517in}{1.216635in}}{\pgfqpoint{1.783816in}{1.218830in}}{\pgfqpoint{1.787723in}{1.222736in}}%
\pgfpathcurveto{\pgfqpoint{1.791630in}{1.226643in}}{\pgfqpoint{1.793825in}{1.231943in}}{\pgfqpoint{1.793825in}{1.237468in}}%
\pgfpathcurveto{\pgfqpoint{1.793825in}{1.242993in}}{\pgfqpoint{1.791630in}{1.248292in}}{\pgfqpoint{1.787723in}{1.252199in}}%
\pgfpathcurveto{\pgfqpoint{1.783816in}{1.256106in}}{\pgfqpoint{1.778517in}{1.258301in}}{\pgfqpoint{1.772992in}{1.258301in}}%
\pgfpathcurveto{\pgfqpoint{1.767467in}{1.258301in}}{\pgfqpoint{1.762167in}{1.256106in}}{\pgfqpoint{1.758260in}{1.252199in}}%
\pgfpathcurveto{\pgfqpoint{1.754353in}{1.248292in}}{\pgfqpoint{1.752158in}{1.242993in}}{\pgfqpoint{1.752158in}{1.237468in}}%
\pgfpathcurveto{\pgfqpoint{1.752158in}{1.231943in}}{\pgfqpoint{1.754353in}{1.226643in}}{\pgfqpoint{1.758260in}{1.222736in}}%
\pgfpathcurveto{\pgfqpoint{1.762167in}{1.218830in}}{\pgfqpoint{1.767467in}{1.216635in}}{\pgfqpoint{1.772992in}{1.216635in}}%
\pgfpathclose%
\pgfusepath{stroke,fill}%
\end{pgfscope}%
\begin{pgfscope}%
\pgfpathrectangle{\pgfqpoint{0.562500in}{0.275000in}}{\pgfqpoint{3.487500in}{1.925000in}}%
\pgfusepath{clip}%
\pgfsetbuttcap%
\pgfsetroundjoin%
\definecolor{currentfill}{rgb}{0.000000,0.000000,0.000000}%
\pgfsetfillcolor{currentfill}%
\pgfsetlinewidth{1.003750pt}%
\definecolor{currentstroke}{rgb}{0.000000,0.000000,0.000000}%
\pgfsetstrokecolor{currentstroke}%
\pgfsetdash{}{0pt}%
\pgfpathmoveto{\pgfqpoint{1.772992in}{1.216635in}}%
\pgfpathcurveto{\pgfqpoint{1.778517in}{1.216635in}}{\pgfqpoint{1.783816in}{1.218830in}}{\pgfqpoint{1.787723in}{1.222736in}}%
\pgfpathcurveto{\pgfqpoint{1.791630in}{1.226643in}}{\pgfqpoint{1.793825in}{1.231943in}}{\pgfqpoint{1.793825in}{1.237468in}}%
\pgfpathcurveto{\pgfqpoint{1.793825in}{1.242993in}}{\pgfqpoint{1.791630in}{1.248292in}}{\pgfqpoint{1.787723in}{1.252199in}}%
\pgfpathcurveto{\pgfqpoint{1.783816in}{1.256106in}}{\pgfqpoint{1.778517in}{1.258301in}}{\pgfqpoint{1.772992in}{1.258301in}}%
\pgfpathcurveto{\pgfqpoint{1.767467in}{1.258301in}}{\pgfqpoint{1.762167in}{1.256106in}}{\pgfqpoint{1.758260in}{1.252199in}}%
\pgfpathcurveto{\pgfqpoint{1.754353in}{1.248292in}}{\pgfqpoint{1.752158in}{1.242993in}}{\pgfqpoint{1.752158in}{1.237468in}}%
\pgfpathcurveto{\pgfqpoint{1.752158in}{1.231943in}}{\pgfqpoint{1.754353in}{1.226643in}}{\pgfqpoint{1.758260in}{1.222736in}}%
\pgfpathcurveto{\pgfqpoint{1.762167in}{1.218830in}}{\pgfqpoint{1.767467in}{1.216635in}}{\pgfqpoint{1.772992in}{1.216635in}}%
\pgfpathclose%
\pgfusepath{stroke,fill}%
\end{pgfscope}%
\begin{pgfscope}%
\pgfpathrectangle{\pgfqpoint{0.562500in}{0.275000in}}{\pgfqpoint{3.487500in}{1.925000in}}%
\pgfusepath{clip}%
\pgfsetbuttcap%
\pgfsetroundjoin%
\definecolor{currentfill}{rgb}{0.000000,0.000000,0.000000}%
\pgfsetfillcolor{currentfill}%
\pgfsetlinewidth{1.003750pt}%
\definecolor{currentstroke}{rgb}{0.000000,0.000000,0.000000}%
\pgfsetstrokecolor{currentstroke}%
\pgfsetdash{}{0pt}%
\pgfpathmoveto{\pgfqpoint{1.772992in}{1.216635in}}%
\pgfpathcurveto{\pgfqpoint{1.778517in}{1.216635in}}{\pgfqpoint{1.783816in}{1.218830in}}{\pgfqpoint{1.787723in}{1.222736in}}%
\pgfpathcurveto{\pgfqpoint{1.791630in}{1.226643in}}{\pgfqpoint{1.793825in}{1.231943in}}{\pgfqpoint{1.793825in}{1.237468in}}%
\pgfpathcurveto{\pgfqpoint{1.793825in}{1.242993in}}{\pgfqpoint{1.791630in}{1.248292in}}{\pgfqpoint{1.787723in}{1.252199in}}%
\pgfpathcurveto{\pgfqpoint{1.783816in}{1.256106in}}{\pgfqpoint{1.778517in}{1.258301in}}{\pgfqpoint{1.772992in}{1.258301in}}%
\pgfpathcurveto{\pgfqpoint{1.767467in}{1.258301in}}{\pgfqpoint{1.762167in}{1.256106in}}{\pgfqpoint{1.758260in}{1.252199in}}%
\pgfpathcurveto{\pgfqpoint{1.754353in}{1.248292in}}{\pgfqpoint{1.752158in}{1.242993in}}{\pgfqpoint{1.752158in}{1.237468in}}%
\pgfpathcurveto{\pgfqpoint{1.752158in}{1.231943in}}{\pgfqpoint{1.754353in}{1.226643in}}{\pgfqpoint{1.758260in}{1.222736in}}%
\pgfpathcurveto{\pgfqpoint{1.762167in}{1.218830in}}{\pgfqpoint{1.767467in}{1.216635in}}{\pgfqpoint{1.772992in}{1.216635in}}%
\pgfpathclose%
\pgfusepath{stroke,fill}%
\end{pgfscope}%
\begin{pgfscope}%
\pgfpathrectangle{\pgfqpoint{0.562500in}{0.275000in}}{\pgfqpoint{3.487500in}{1.925000in}}%
\pgfusepath{clip}%
\pgfsetbuttcap%
\pgfsetroundjoin%
\definecolor{currentfill}{rgb}{0.000000,0.000000,0.000000}%
\pgfsetfillcolor{currentfill}%
\pgfsetlinewidth{1.003750pt}%
\definecolor{currentstroke}{rgb}{0.000000,0.000000,0.000000}%
\pgfsetstrokecolor{currentstroke}%
\pgfsetdash{}{0pt}%
\pgfpathmoveto{\pgfqpoint{1.772992in}{2.076667in}}%
\pgfpathcurveto{\pgfqpoint{1.778517in}{2.076667in}}{\pgfqpoint{1.783816in}{2.078862in}}{\pgfqpoint{1.787723in}{2.082769in}}%
\pgfpathcurveto{\pgfqpoint{1.791630in}{2.086675in}}{\pgfqpoint{1.793825in}{2.091975in}}{\pgfqpoint{1.793825in}{2.097500in}}%
\pgfpathcurveto{\pgfqpoint{1.793825in}{2.103025in}}{\pgfqpoint{1.791630in}{2.108325in}}{\pgfqpoint{1.787723in}{2.112231in}}%
\pgfpathcurveto{\pgfqpoint{1.783816in}{2.116138in}}{\pgfqpoint{1.778517in}{2.118333in}}{\pgfqpoint{1.772992in}{2.118333in}}%
\pgfpathcurveto{\pgfqpoint{1.767467in}{2.118333in}}{\pgfqpoint{1.762167in}{2.116138in}}{\pgfqpoint{1.758260in}{2.112231in}}%
\pgfpathcurveto{\pgfqpoint{1.754353in}{2.108325in}}{\pgfqpoint{1.752158in}{2.103025in}}{\pgfqpoint{1.752158in}{2.097500in}}%
\pgfpathcurveto{\pgfqpoint{1.752158in}{2.091975in}}{\pgfqpoint{1.754353in}{2.086675in}}{\pgfqpoint{1.758260in}{2.082769in}}%
\pgfpathcurveto{\pgfqpoint{1.762167in}{2.078862in}}{\pgfqpoint{1.767467in}{2.076667in}}{\pgfqpoint{1.772992in}{2.076667in}}%
\pgfpathclose%
\pgfusepath{stroke,fill}%
\end{pgfscope}%
\begin{pgfscope}%
\pgfpathrectangle{\pgfqpoint{0.562500in}{0.275000in}}{\pgfqpoint{3.487500in}{1.925000in}}%
\pgfusepath{clip}%
\pgfsetbuttcap%
\pgfsetroundjoin%
\definecolor{currentfill}{rgb}{0.000000,0.000000,0.000000}%
\pgfsetfillcolor{currentfill}%
\pgfsetlinewidth{1.003750pt}%
\definecolor{currentstroke}{rgb}{0.000000,0.000000,0.000000}%
\pgfsetstrokecolor{currentstroke}%
\pgfsetdash{}{0pt}%
\pgfpathmoveto{\pgfqpoint{1.772992in}{1.216635in}}%
\pgfpathcurveto{\pgfqpoint{1.778517in}{1.216635in}}{\pgfqpoint{1.783816in}{1.218830in}}{\pgfqpoint{1.787723in}{1.222736in}}%
\pgfpathcurveto{\pgfqpoint{1.791630in}{1.226643in}}{\pgfqpoint{1.793825in}{1.231943in}}{\pgfqpoint{1.793825in}{1.237468in}}%
\pgfpathcurveto{\pgfqpoint{1.793825in}{1.242993in}}{\pgfqpoint{1.791630in}{1.248292in}}{\pgfqpoint{1.787723in}{1.252199in}}%
\pgfpathcurveto{\pgfqpoint{1.783816in}{1.256106in}}{\pgfqpoint{1.778517in}{1.258301in}}{\pgfqpoint{1.772992in}{1.258301in}}%
\pgfpathcurveto{\pgfqpoint{1.767467in}{1.258301in}}{\pgfqpoint{1.762167in}{1.256106in}}{\pgfqpoint{1.758260in}{1.252199in}}%
\pgfpathcurveto{\pgfqpoint{1.754353in}{1.248292in}}{\pgfqpoint{1.752158in}{1.242993in}}{\pgfqpoint{1.752158in}{1.237468in}}%
\pgfpathcurveto{\pgfqpoint{1.752158in}{1.231943in}}{\pgfqpoint{1.754353in}{1.226643in}}{\pgfqpoint{1.758260in}{1.222736in}}%
\pgfpathcurveto{\pgfqpoint{1.762167in}{1.218830in}}{\pgfqpoint{1.767467in}{1.216635in}}{\pgfqpoint{1.772992in}{1.216635in}}%
\pgfpathclose%
\pgfusepath{stroke,fill}%
\end{pgfscope}%
\begin{pgfscope}%
\pgfpathrectangle{\pgfqpoint{0.562500in}{0.275000in}}{\pgfqpoint{3.487500in}{1.925000in}}%
\pgfusepath{clip}%
\pgfsetbuttcap%
\pgfsetroundjoin%
\definecolor{currentfill}{rgb}{0.000000,0.000000,0.000000}%
\pgfsetfillcolor{currentfill}%
\pgfsetlinewidth{1.003750pt}%
\definecolor{currentstroke}{rgb}{0.000000,0.000000,0.000000}%
\pgfsetstrokecolor{currentstroke}%
\pgfsetdash{}{0pt}%
\pgfpathmoveto{\pgfqpoint{1.772992in}{1.216635in}}%
\pgfpathcurveto{\pgfqpoint{1.778517in}{1.216635in}}{\pgfqpoint{1.783816in}{1.218830in}}{\pgfqpoint{1.787723in}{1.222736in}}%
\pgfpathcurveto{\pgfqpoint{1.791630in}{1.226643in}}{\pgfqpoint{1.793825in}{1.231943in}}{\pgfqpoint{1.793825in}{1.237468in}}%
\pgfpathcurveto{\pgfqpoint{1.793825in}{1.242993in}}{\pgfqpoint{1.791630in}{1.248292in}}{\pgfqpoint{1.787723in}{1.252199in}}%
\pgfpathcurveto{\pgfqpoint{1.783816in}{1.256106in}}{\pgfqpoint{1.778517in}{1.258301in}}{\pgfqpoint{1.772992in}{1.258301in}}%
\pgfpathcurveto{\pgfqpoint{1.767467in}{1.258301in}}{\pgfqpoint{1.762167in}{1.256106in}}{\pgfqpoint{1.758260in}{1.252199in}}%
\pgfpathcurveto{\pgfqpoint{1.754353in}{1.248292in}}{\pgfqpoint{1.752158in}{1.242993in}}{\pgfqpoint{1.752158in}{1.237468in}}%
\pgfpathcurveto{\pgfqpoint{1.752158in}{1.231943in}}{\pgfqpoint{1.754353in}{1.226643in}}{\pgfqpoint{1.758260in}{1.222736in}}%
\pgfpathcurveto{\pgfqpoint{1.762167in}{1.218830in}}{\pgfqpoint{1.767467in}{1.216635in}}{\pgfqpoint{1.772992in}{1.216635in}}%
\pgfpathclose%
\pgfusepath{stroke,fill}%
\end{pgfscope}%
\begin{pgfscope}%
\pgfpathrectangle{\pgfqpoint{0.562500in}{0.275000in}}{\pgfqpoint{3.487500in}{1.925000in}}%
\pgfusepath{clip}%
\pgfsetbuttcap%
\pgfsetroundjoin%
\definecolor{currentfill}{rgb}{0.000000,0.000000,0.000000}%
\pgfsetfillcolor{currentfill}%
\pgfsetlinewidth{1.003750pt}%
\definecolor{currentstroke}{rgb}{0.000000,0.000000,0.000000}%
\pgfsetstrokecolor{currentstroke}%
\pgfsetdash{}{0pt}%
\pgfpathmoveto{\pgfqpoint{1.772992in}{1.216635in}}%
\pgfpathcurveto{\pgfqpoint{1.778517in}{1.216635in}}{\pgfqpoint{1.783816in}{1.218830in}}{\pgfqpoint{1.787723in}{1.222736in}}%
\pgfpathcurveto{\pgfqpoint{1.791630in}{1.226643in}}{\pgfqpoint{1.793825in}{1.231943in}}{\pgfqpoint{1.793825in}{1.237468in}}%
\pgfpathcurveto{\pgfqpoint{1.793825in}{1.242993in}}{\pgfqpoint{1.791630in}{1.248292in}}{\pgfqpoint{1.787723in}{1.252199in}}%
\pgfpathcurveto{\pgfqpoint{1.783816in}{1.256106in}}{\pgfqpoint{1.778517in}{1.258301in}}{\pgfqpoint{1.772992in}{1.258301in}}%
\pgfpathcurveto{\pgfqpoint{1.767467in}{1.258301in}}{\pgfqpoint{1.762167in}{1.256106in}}{\pgfqpoint{1.758260in}{1.252199in}}%
\pgfpathcurveto{\pgfqpoint{1.754353in}{1.248292in}}{\pgfqpoint{1.752158in}{1.242993in}}{\pgfqpoint{1.752158in}{1.237468in}}%
\pgfpathcurveto{\pgfqpoint{1.752158in}{1.231943in}}{\pgfqpoint{1.754353in}{1.226643in}}{\pgfqpoint{1.758260in}{1.222736in}}%
\pgfpathcurveto{\pgfqpoint{1.762167in}{1.218830in}}{\pgfqpoint{1.767467in}{1.216635in}}{\pgfqpoint{1.772992in}{1.216635in}}%
\pgfpathclose%
\pgfusepath{stroke,fill}%
\end{pgfscope}%
\begin{pgfscope}%
\pgfpathrectangle{\pgfqpoint{0.562500in}{0.275000in}}{\pgfqpoint{3.487500in}{1.925000in}}%
\pgfusepath{clip}%
\pgfsetbuttcap%
\pgfsetroundjoin%
\definecolor{currentfill}{rgb}{0.000000,0.000000,0.000000}%
\pgfsetfillcolor{currentfill}%
\pgfsetlinewidth{1.003750pt}%
\definecolor{currentstroke}{rgb}{0.000000,0.000000,0.000000}%
\pgfsetstrokecolor{currentstroke}%
\pgfsetdash{}{0pt}%
\pgfpathmoveto{\pgfqpoint{1.772992in}{1.216635in}}%
\pgfpathcurveto{\pgfqpoint{1.778517in}{1.216635in}}{\pgfqpoint{1.783816in}{1.218830in}}{\pgfqpoint{1.787723in}{1.222736in}}%
\pgfpathcurveto{\pgfqpoint{1.791630in}{1.226643in}}{\pgfqpoint{1.793825in}{1.231943in}}{\pgfqpoint{1.793825in}{1.237468in}}%
\pgfpathcurveto{\pgfqpoint{1.793825in}{1.242993in}}{\pgfqpoint{1.791630in}{1.248292in}}{\pgfqpoint{1.787723in}{1.252199in}}%
\pgfpathcurveto{\pgfqpoint{1.783816in}{1.256106in}}{\pgfqpoint{1.778517in}{1.258301in}}{\pgfqpoint{1.772992in}{1.258301in}}%
\pgfpathcurveto{\pgfqpoint{1.767467in}{1.258301in}}{\pgfqpoint{1.762167in}{1.256106in}}{\pgfqpoint{1.758260in}{1.252199in}}%
\pgfpathcurveto{\pgfqpoint{1.754353in}{1.248292in}}{\pgfqpoint{1.752158in}{1.242993in}}{\pgfqpoint{1.752158in}{1.237468in}}%
\pgfpathcurveto{\pgfqpoint{1.752158in}{1.231943in}}{\pgfqpoint{1.754353in}{1.226643in}}{\pgfqpoint{1.758260in}{1.222736in}}%
\pgfpathcurveto{\pgfqpoint{1.762167in}{1.218830in}}{\pgfqpoint{1.767467in}{1.216635in}}{\pgfqpoint{1.772992in}{1.216635in}}%
\pgfpathclose%
\pgfusepath{stroke,fill}%
\end{pgfscope}%
\begin{pgfscope}%
\pgfpathrectangle{\pgfqpoint{0.562500in}{0.275000in}}{\pgfqpoint{3.487500in}{1.925000in}}%
\pgfusepath{clip}%
\pgfsetbuttcap%
\pgfsetroundjoin%
\definecolor{currentfill}{rgb}{0.000000,0.000000,0.000000}%
\pgfsetfillcolor{currentfill}%
\pgfsetlinewidth{1.003750pt}%
\definecolor{currentstroke}{rgb}{0.000000,0.000000,0.000000}%
\pgfsetstrokecolor{currentstroke}%
\pgfsetdash{}{0pt}%
\pgfpathmoveto{\pgfqpoint{1.772992in}{0.356602in}}%
\pgfpathcurveto{\pgfqpoint{1.778517in}{0.356602in}}{\pgfqpoint{1.783816in}{0.358798in}}{\pgfqpoint{1.787723in}{0.362704in}}%
\pgfpathcurveto{\pgfqpoint{1.791630in}{0.366611in}}{\pgfqpoint{1.793825in}{0.371911in}}{\pgfqpoint{1.793825in}{0.377436in}}%
\pgfpathcurveto{\pgfqpoint{1.793825in}{0.382961in}}{\pgfqpoint{1.791630in}{0.388260in}}{\pgfqpoint{1.787723in}{0.392167in}}%
\pgfpathcurveto{\pgfqpoint{1.783816in}{0.396074in}}{\pgfqpoint{1.778517in}{0.398269in}}{\pgfqpoint{1.772992in}{0.398269in}}%
\pgfpathcurveto{\pgfqpoint{1.767467in}{0.398269in}}{\pgfqpoint{1.762167in}{0.396074in}}{\pgfqpoint{1.758260in}{0.392167in}}%
\pgfpathcurveto{\pgfqpoint{1.754353in}{0.388260in}}{\pgfqpoint{1.752158in}{0.382961in}}{\pgfqpoint{1.752158in}{0.377436in}}%
\pgfpathcurveto{\pgfqpoint{1.752158in}{0.371911in}}{\pgfqpoint{1.754353in}{0.366611in}}{\pgfqpoint{1.758260in}{0.362704in}}%
\pgfpathcurveto{\pgfqpoint{1.762167in}{0.358798in}}{\pgfqpoint{1.767467in}{0.356602in}}{\pgfqpoint{1.772992in}{0.356602in}}%
\pgfpathclose%
\pgfusepath{stroke,fill}%
\end{pgfscope}%
\begin{pgfscope}%
\pgfpathrectangle{\pgfqpoint{0.562500in}{0.275000in}}{\pgfqpoint{3.487500in}{1.925000in}}%
\pgfusepath{clip}%
\pgfsetbuttcap%
\pgfsetroundjoin%
\definecolor{currentfill}{rgb}{0.000000,0.000000,0.000000}%
\pgfsetfillcolor{currentfill}%
\pgfsetlinewidth{1.003750pt}%
\definecolor{currentstroke}{rgb}{0.000000,0.000000,0.000000}%
\pgfsetstrokecolor{currentstroke}%
\pgfsetdash{}{0pt}%
\pgfpathmoveto{\pgfqpoint{1.772992in}{2.076667in}}%
\pgfpathcurveto{\pgfqpoint{1.778517in}{2.076667in}}{\pgfqpoint{1.783816in}{2.078862in}}{\pgfqpoint{1.787723in}{2.082769in}}%
\pgfpathcurveto{\pgfqpoint{1.791630in}{2.086675in}}{\pgfqpoint{1.793825in}{2.091975in}}{\pgfqpoint{1.793825in}{2.097500in}}%
\pgfpathcurveto{\pgfqpoint{1.793825in}{2.103025in}}{\pgfqpoint{1.791630in}{2.108325in}}{\pgfqpoint{1.787723in}{2.112231in}}%
\pgfpathcurveto{\pgfqpoint{1.783816in}{2.116138in}}{\pgfqpoint{1.778517in}{2.118333in}}{\pgfqpoint{1.772992in}{2.118333in}}%
\pgfpathcurveto{\pgfqpoint{1.767467in}{2.118333in}}{\pgfqpoint{1.762167in}{2.116138in}}{\pgfqpoint{1.758260in}{2.112231in}}%
\pgfpathcurveto{\pgfqpoint{1.754353in}{2.108325in}}{\pgfqpoint{1.752158in}{2.103025in}}{\pgfqpoint{1.752158in}{2.097500in}}%
\pgfpathcurveto{\pgfqpoint{1.752158in}{2.091975in}}{\pgfqpoint{1.754353in}{2.086675in}}{\pgfqpoint{1.758260in}{2.082769in}}%
\pgfpathcurveto{\pgfqpoint{1.762167in}{2.078862in}}{\pgfqpoint{1.767467in}{2.076667in}}{\pgfqpoint{1.772992in}{2.076667in}}%
\pgfpathclose%
\pgfusepath{stroke,fill}%
\end{pgfscope}%
\begin{pgfscope}%
\pgfpathrectangle{\pgfqpoint{0.562500in}{0.275000in}}{\pgfqpoint{3.487500in}{1.925000in}}%
\pgfusepath{clip}%
\pgfsetbuttcap%
\pgfsetroundjoin%
\definecolor{currentfill}{rgb}{0.000000,0.000000,0.000000}%
\pgfsetfillcolor{currentfill}%
\pgfsetlinewidth{1.003750pt}%
\definecolor{currentstroke}{rgb}{0.000000,0.000000,0.000000}%
\pgfsetstrokecolor{currentstroke}%
\pgfsetdash{}{0pt}%
\pgfpathmoveto{\pgfqpoint{1.772992in}{1.216635in}}%
\pgfpathcurveto{\pgfqpoint{1.778517in}{1.216635in}}{\pgfqpoint{1.783816in}{1.218830in}}{\pgfqpoint{1.787723in}{1.222736in}}%
\pgfpathcurveto{\pgfqpoint{1.791630in}{1.226643in}}{\pgfqpoint{1.793825in}{1.231943in}}{\pgfqpoint{1.793825in}{1.237468in}}%
\pgfpathcurveto{\pgfqpoint{1.793825in}{1.242993in}}{\pgfqpoint{1.791630in}{1.248292in}}{\pgfqpoint{1.787723in}{1.252199in}}%
\pgfpathcurveto{\pgfqpoint{1.783816in}{1.256106in}}{\pgfqpoint{1.778517in}{1.258301in}}{\pgfqpoint{1.772992in}{1.258301in}}%
\pgfpathcurveto{\pgfqpoint{1.767467in}{1.258301in}}{\pgfqpoint{1.762167in}{1.256106in}}{\pgfqpoint{1.758260in}{1.252199in}}%
\pgfpathcurveto{\pgfqpoint{1.754353in}{1.248292in}}{\pgfqpoint{1.752158in}{1.242993in}}{\pgfqpoint{1.752158in}{1.237468in}}%
\pgfpathcurveto{\pgfqpoint{1.752158in}{1.231943in}}{\pgfqpoint{1.754353in}{1.226643in}}{\pgfqpoint{1.758260in}{1.222736in}}%
\pgfpathcurveto{\pgfqpoint{1.762167in}{1.218830in}}{\pgfqpoint{1.767467in}{1.216635in}}{\pgfqpoint{1.772992in}{1.216635in}}%
\pgfpathclose%
\pgfusepath{stroke,fill}%
\end{pgfscope}%
\begin{pgfscope}%
\pgfpathrectangle{\pgfqpoint{0.562500in}{0.275000in}}{\pgfqpoint{3.487500in}{1.925000in}}%
\pgfusepath{clip}%
\pgfsetbuttcap%
\pgfsetroundjoin%
\definecolor{currentfill}{rgb}{0.000000,0.000000,0.000000}%
\pgfsetfillcolor{currentfill}%
\pgfsetlinewidth{1.003750pt}%
\definecolor{currentstroke}{rgb}{0.000000,0.000000,0.000000}%
\pgfsetstrokecolor{currentstroke}%
\pgfsetdash{}{0pt}%
\pgfpathmoveto{\pgfqpoint{1.772992in}{1.216635in}}%
\pgfpathcurveto{\pgfqpoint{1.778517in}{1.216635in}}{\pgfqpoint{1.783816in}{1.218830in}}{\pgfqpoint{1.787723in}{1.222736in}}%
\pgfpathcurveto{\pgfqpoint{1.791630in}{1.226643in}}{\pgfqpoint{1.793825in}{1.231943in}}{\pgfqpoint{1.793825in}{1.237468in}}%
\pgfpathcurveto{\pgfqpoint{1.793825in}{1.242993in}}{\pgfqpoint{1.791630in}{1.248292in}}{\pgfqpoint{1.787723in}{1.252199in}}%
\pgfpathcurveto{\pgfqpoint{1.783816in}{1.256106in}}{\pgfqpoint{1.778517in}{1.258301in}}{\pgfqpoint{1.772992in}{1.258301in}}%
\pgfpathcurveto{\pgfqpoint{1.767467in}{1.258301in}}{\pgfqpoint{1.762167in}{1.256106in}}{\pgfqpoint{1.758260in}{1.252199in}}%
\pgfpathcurveto{\pgfqpoint{1.754353in}{1.248292in}}{\pgfqpoint{1.752158in}{1.242993in}}{\pgfqpoint{1.752158in}{1.237468in}}%
\pgfpathcurveto{\pgfqpoint{1.752158in}{1.231943in}}{\pgfqpoint{1.754353in}{1.226643in}}{\pgfqpoint{1.758260in}{1.222736in}}%
\pgfpathcurveto{\pgfqpoint{1.762167in}{1.218830in}}{\pgfqpoint{1.767467in}{1.216635in}}{\pgfqpoint{1.772992in}{1.216635in}}%
\pgfpathclose%
\pgfusepath{stroke,fill}%
\end{pgfscope}%
\begin{pgfscope}%
\pgfpathrectangle{\pgfqpoint{0.562500in}{0.275000in}}{\pgfqpoint{3.487500in}{1.925000in}}%
\pgfusepath{clip}%
\pgfsetbuttcap%
\pgfsetroundjoin%
\definecolor{currentfill}{rgb}{0.000000,0.000000,0.000000}%
\pgfsetfillcolor{currentfill}%
\pgfsetlinewidth{1.003750pt}%
\definecolor{currentstroke}{rgb}{0.000000,0.000000,0.000000}%
\pgfsetstrokecolor{currentstroke}%
\pgfsetdash{}{0pt}%
\pgfpathmoveto{\pgfqpoint{1.772992in}{1.216635in}}%
\pgfpathcurveto{\pgfqpoint{1.778517in}{1.216635in}}{\pgfqpoint{1.783816in}{1.218830in}}{\pgfqpoint{1.787723in}{1.222736in}}%
\pgfpathcurveto{\pgfqpoint{1.791630in}{1.226643in}}{\pgfqpoint{1.793825in}{1.231943in}}{\pgfqpoint{1.793825in}{1.237468in}}%
\pgfpathcurveto{\pgfqpoint{1.793825in}{1.242993in}}{\pgfqpoint{1.791630in}{1.248292in}}{\pgfqpoint{1.787723in}{1.252199in}}%
\pgfpathcurveto{\pgfqpoint{1.783816in}{1.256106in}}{\pgfqpoint{1.778517in}{1.258301in}}{\pgfqpoint{1.772992in}{1.258301in}}%
\pgfpathcurveto{\pgfqpoint{1.767467in}{1.258301in}}{\pgfqpoint{1.762167in}{1.256106in}}{\pgfqpoint{1.758260in}{1.252199in}}%
\pgfpathcurveto{\pgfqpoint{1.754353in}{1.248292in}}{\pgfqpoint{1.752158in}{1.242993in}}{\pgfqpoint{1.752158in}{1.237468in}}%
\pgfpathcurveto{\pgfqpoint{1.752158in}{1.231943in}}{\pgfqpoint{1.754353in}{1.226643in}}{\pgfqpoint{1.758260in}{1.222736in}}%
\pgfpathcurveto{\pgfqpoint{1.762167in}{1.218830in}}{\pgfqpoint{1.767467in}{1.216635in}}{\pgfqpoint{1.772992in}{1.216635in}}%
\pgfpathclose%
\pgfusepath{stroke,fill}%
\end{pgfscope}%
\begin{pgfscope}%
\pgfpathrectangle{\pgfqpoint{0.562500in}{0.275000in}}{\pgfqpoint{3.487500in}{1.925000in}}%
\pgfusepath{clip}%
\pgfsetbuttcap%
\pgfsetroundjoin%
\definecolor{currentfill}{rgb}{0.000000,0.000000,0.000000}%
\pgfsetfillcolor{currentfill}%
\pgfsetlinewidth{1.003750pt}%
\definecolor{currentstroke}{rgb}{0.000000,0.000000,0.000000}%
\pgfsetstrokecolor{currentstroke}%
\pgfsetdash{}{0pt}%
\pgfpathmoveto{\pgfqpoint{1.772992in}{1.216635in}}%
\pgfpathcurveto{\pgfqpoint{1.778517in}{1.216635in}}{\pgfqpoint{1.783816in}{1.218830in}}{\pgfqpoint{1.787723in}{1.222736in}}%
\pgfpathcurveto{\pgfqpoint{1.791630in}{1.226643in}}{\pgfqpoint{1.793825in}{1.231943in}}{\pgfqpoint{1.793825in}{1.237468in}}%
\pgfpathcurveto{\pgfqpoint{1.793825in}{1.242993in}}{\pgfqpoint{1.791630in}{1.248292in}}{\pgfqpoint{1.787723in}{1.252199in}}%
\pgfpathcurveto{\pgfqpoint{1.783816in}{1.256106in}}{\pgfqpoint{1.778517in}{1.258301in}}{\pgfqpoint{1.772992in}{1.258301in}}%
\pgfpathcurveto{\pgfqpoint{1.767467in}{1.258301in}}{\pgfqpoint{1.762167in}{1.256106in}}{\pgfqpoint{1.758260in}{1.252199in}}%
\pgfpathcurveto{\pgfqpoint{1.754353in}{1.248292in}}{\pgfqpoint{1.752158in}{1.242993in}}{\pgfqpoint{1.752158in}{1.237468in}}%
\pgfpathcurveto{\pgfqpoint{1.752158in}{1.231943in}}{\pgfqpoint{1.754353in}{1.226643in}}{\pgfqpoint{1.758260in}{1.222736in}}%
\pgfpathcurveto{\pgfqpoint{1.762167in}{1.218830in}}{\pgfqpoint{1.767467in}{1.216635in}}{\pgfqpoint{1.772992in}{1.216635in}}%
\pgfpathclose%
\pgfusepath{stroke,fill}%
\end{pgfscope}%
\begin{pgfscope}%
\pgfpathrectangle{\pgfqpoint{0.562500in}{0.275000in}}{\pgfqpoint{3.487500in}{1.925000in}}%
\pgfusepath{clip}%
\pgfsetbuttcap%
\pgfsetroundjoin%
\definecolor{currentfill}{rgb}{0.000000,0.000000,0.000000}%
\pgfsetfillcolor{currentfill}%
\pgfsetlinewidth{1.003750pt}%
\definecolor{currentstroke}{rgb}{0.000000,0.000000,0.000000}%
\pgfsetstrokecolor{currentstroke}%
\pgfsetdash{}{0pt}%
\pgfpathmoveto{\pgfqpoint{1.772992in}{1.216635in}}%
\pgfpathcurveto{\pgfqpoint{1.778517in}{1.216635in}}{\pgfqpoint{1.783816in}{1.218830in}}{\pgfqpoint{1.787723in}{1.222736in}}%
\pgfpathcurveto{\pgfqpoint{1.791630in}{1.226643in}}{\pgfqpoint{1.793825in}{1.231943in}}{\pgfqpoint{1.793825in}{1.237468in}}%
\pgfpathcurveto{\pgfqpoint{1.793825in}{1.242993in}}{\pgfqpoint{1.791630in}{1.248292in}}{\pgfqpoint{1.787723in}{1.252199in}}%
\pgfpathcurveto{\pgfqpoint{1.783816in}{1.256106in}}{\pgfqpoint{1.778517in}{1.258301in}}{\pgfqpoint{1.772992in}{1.258301in}}%
\pgfpathcurveto{\pgfqpoint{1.767467in}{1.258301in}}{\pgfqpoint{1.762167in}{1.256106in}}{\pgfqpoint{1.758260in}{1.252199in}}%
\pgfpathcurveto{\pgfqpoint{1.754353in}{1.248292in}}{\pgfqpoint{1.752158in}{1.242993in}}{\pgfqpoint{1.752158in}{1.237468in}}%
\pgfpathcurveto{\pgfqpoint{1.752158in}{1.231943in}}{\pgfqpoint{1.754353in}{1.226643in}}{\pgfqpoint{1.758260in}{1.222736in}}%
\pgfpathcurveto{\pgfqpoint{1.762167in}{1.218830in}}{\pgfqpoint{1.767467in}{1.216635in}}{\pgfqpoint{1.772992in}{1.216635in}}%
\pgfpathclose%
\pgfusepath{stroke,fill}%
\end{pgfscope}%
\begin{pgfscope}%
\pgfpathrectangle{\pgfqpoint{0.562500in}{0.275000in}}{\pgfqpoint{3.487500in}{1.925000in}}%
\pgfusepath{clip}%
\pgfsetbuttcap%
\pgfsetroundjoin%
\definecolor{currentfill}{rgb}{0.000000,0.000000,0.000000}%
\pgfsetfillcolor{currentfill}%
\pgfsetlinewidth{1.003750pt}%
\definecolor{currentstroke}{rgb}{0.000000,0.000000,0.000000}%
\pgfsetstrokecolor{currentstroke}%
\pgfsetdash{}{0pt}%
\pgfpathmoveto{\pgfqpoint{1.772992in}{1.216635in}}%
\pgfpathcurveto{\pgfqpoint{1.778517in}{1.216635in}}{\pgfqpoint{1.783816in}{1.218830in}}{\pgfqpoint{1.787723in}{1.222736in}}%
\pgfpathcurveto{\pgfqpoint{1.791630in}{1.226643in}}{\pgfqpoint{1.793825in}{1.231943in}}{\pgfqpoint{1.793825in}{1.237468in}}%
\pgfpathcurveto{\pgfqpoint{1.793825in}{1.242993in}}{\pgfqpoint{1.791630in}{1.248292in}}{\pgfqpoint{1.787723in}{1.252199in}}%
\pgfpathcurveto{\pgfqpoint{1.783816in}{1.256106in}}{\pgfqpoint{1.778517in}{1.258301in}}{\pgfqpoint{1.772992in}{1.258301in}}%
\pgfpathcurveto{\pgfqpoint{1.767467in}{1.258301in}}{\pgfqpoint{1.762167in}{1.256106in}}{\pgfqpoint{1.758260in}{1.252199in}}%
\pgfpathcurveto{\pgfqpoint{1.754353in}{1.248292in}}{\pgfqpoint{1.752158in}{1.242993in}}{\pgfqpoint{1.752158in}{1.237468in}}%
\pgfpathcurveto{\pgfqpoint{1.752158in}{1.231943in}}{\pgfqpoint{1.754353in}{1.226643in}}{\pgfqpoint{1.758260in}{1.222736in}}%
\pgfpathcurveto{\pgfqpoint{1.762167in}{1.218830in}}{\pgfqpoint{1.767467in}{1.216635in}}{\pgfqpoint{1.772992in}{1.216635in}}%
\pgfpathclose%
\pgfusepath{stroke,fill}%
\end{pgfscope}%
\begin{pgfscope}%
\pgfpathrectangle{\pgfqpoint{0.562500in}{0.275000in}}{\pgfqpoint{3.487500in}{1.925000in}}%
\pgfusepath{clip}%
\pgfsetbuttcap%
\pgfsetroundjoin%
\definecolor{currentfill}{rgb}{0.000000,0.000000,0.000000}%
\pgfsetfillcolor{currentfill}%
\pgfsetlinewidth{1.003750pt}%
\definecolor{currentstroke}{rgb}{0.000000,0.000000,0.000000}%
\pgfsetstrokecolor{currentstroke}%
\pgfsetdash{}{0pt}%
\pgfpathmoveto{\pgfqpoint{1.772992in}{1.216635in}}%
\pgfpathcurveto{\pgfqpoint{1.778517in}{1.216635in}}{\pgfqpoint{1.783816in}{1.218830in}}{\pgfqpoint{1.787723in}{1.222736in}}%
\pgfpathcurveto{\pgfqpoint{1.791630in}{1.226643in}}{\pgfqpoint{1.793825in}{1.231943in}}{\pgfqpoint{1.793825in}{1.237468in}}%
\pgfpathcurveto{\pgfqpoint{1.793825in}{1.242993in}}{\pgfqpoint{1.791630in}{1.248292in}}{\pgfqpoint{1.787723in}{1.252199in}}%
\pgfpathcurveto{\pgfqpoint{1.783816in}{1.256106in}}{\pgfqpoint{1.778517in}{1.258301in}}{\pgfqpoint{1.772992in}{1.258301in}}%
\pgfpathcurveto{\pgfqpoint{1.767467in}{1.258301in}}{\pgfqpoint{1.762167in}{1.256106in}}{\pgfqpoint{1.758260in}{1.252199in}}%
\pgfpathcurveto{\pgfqpoint{1.754353in}{1.248292in}}{\pgfqpoint{1.752158in}{1.242993in}}{\pgfqpoint{1.752158in}{1.237468in}}%
\pgfpathcurveto{\pgfqpoint{1.752158in}{1.231943in}}{\pgfqpoint{1.754353in}{1.226643in}}{\pgfqpoint{1.758260in}{1.222736in}}%
\pgfpathcurveto{\pgfqpoint{1.762167in}{1.218830in}}{\pgfqpoint{1.767467in}{1.216635in}}{\pgfqpoint{1.772992in}{1.216635in}}%
\pgfpathclose%
\pgfusepath{stroke,fill}%
\end{pgfscope}%
\begin{pgfscope}%
\pgfpathrectangle{\pgfqpoint{0.562500in}{0.275000in}}{\pgfqpoint{3.487500in}{1.925000in}}%
\pgfusepath{clip}%
\pgfsetbuttcap%
\pgfsetroundjoin%
\definecolor{currentfill}{rgb}{0.000000,0.000000,0.000000}%
\pgfsetfillcolor{currentfill}%
\pgfsetlinewidth{1.003750pt}%
\definecolor{currentstroke}{rgb}{0.000000,0.000000,0.000000}%
\pgfsetstrokecolor{currentstroke}%
\pgfsetdash{}{0pt}%
\pgfpathmoveto{\pgfqpoint{1.772992in}{1.216635in}}%
\pgfpathcurveto{\pgfqpoint{1.778517in}{1.216635in}}{\pgfqpoint{1.783816in}{1.218830in}}{\pgfqpoint{1.787723in}{1.222736in}}%
\pgfpathcurveto{\pgfqpoint{1.791630in}{1.226643in}}{\pgfqpoint{1.793825in}{1.231943in}}{\pgfqpoint{1.793825in}{1.237468in}}%
\pgfpathcurveto{\pgfqpoint{1.793825in}{1.242993in}}{\pgfqpoint{1.791630in}{1.248292in}}{\pgfqpoint{1.787723in}{1.252199in}}%
\pgfpathcurveto{\pgfqpoint{1.783816in}{1.256106in}}{\pgfqpoint{1.778517in}{1.258301in}}{\pgfqpoint{1.772992in}{1.258301in}}%
\pgfpathcurveto{\pgfqpoint{1.767467in}{1.258301in}}{\pgfqpoint{1.762167in}{1.256106in}}{\pgfqpoint{1.758260in}{1.252199in}}%
\pgfpathcurveto{\pgfqpoint{1.754353in}{1.248292in}}{\pgfqpoint{1.752158in}{1.242993in}}{\pgfqpoint{1.752158in}{1.237468in}}%
\pgfpathcurveto{\pgfqpoint{1.752158in}{1.231943in}}{\pgfqpoint{1.754353in}{1.226643in}}{\pgfqpoint{1.758260in}{1.222736in}}%
\pgfpathcurveto{\pgfqpoint{1.762167in}{1.218830in}}{\pgfqpoint{1.767467in}{1.216635in}}{\pgfqpoint{1.772992in}{1.216635in}}%
\pgfpathclose%
\pgfusepath{stroke,fill}%
\end{pgfscope}%
\begin{pgfscope}%
\pgfpathrectangle{\pgfqpoint{0.562500in}{0.275000in}}{\pgfqpoint{3.487500in}{1.925000in}}%
\pgfusepath{clip}%
\pgfsetbuttcap%
\pgfsetroundjoin%
\definecolor{currentfill}{rgb}{0.000000,0.000000,0.000000}%
\pgfsetfillcolor{currentfill}%
\pgfsetlinewidth{1.003750pt}%
\definecolor{currentstroke}{rgb}{0.000000,0.000000,0.000000}%
\pgfsetstrokecolor{currentstroke}%
\pgfsetdash{}{0pt}%
\pgfpathmoveto{\pgfqpoint{1.772992in}{1.216635in}}%
\pgfpathcurveto{\pgfqpoint{1.778517in}{1.216635in}}{\pgfqpoint{1.783816in}{1.218830in}}{\pgfqpoint{1.787723in}{1.222736in}}%
\pgfpathcurveto{\pgfqpoint{1.791630in}{1.226643in}}{\pgfqpoint{1.793825in}{1.231943in}}{\pgfqpoint{1.793825in}{1.237468in}}%
\pgfpathcurveto{\pgfqpoint{1.793825in}{1.242993in}}{\pgfqpoint{1.791630in}{1.248292in}}{\pgfqpoint{1.787723in}{1.252199in}}%
\pgfpathcurveto{\pgfqpoint{1.783816in}{1.256106in}}{\pgfqpoint{1.778517in}{1.258301in}}{\pgfqpoint{1.772992in}{1.258301in}}%
\pgfpathcurveto{\pgfqpoint{1.767467in}{1.258301in}}{\pgfqpoint{1.762167in}{1.256106in}}{\pgfqpoint{1.758260in}{1.252199in}}%
\pgfpathcurveto{\pgfqpoint{1.754353in}{1.248292in}}{\pgfqpoint{1.752158in}{1.242993in}}{\pgfqpoint{1.752158in}{1.237468in}}%
\pgfpathcurveto{\pgfqpoint{1.752158in}{1.231943in}}{\pgfqpoint{1.754353in}{1.226643in}}{\pgfqpoint{1.758260in}{1.222736in}}%
\pgfpathcurveto{\pgfqpoint{1.762167in}{1.218830in}}{\pgfqpoint{1.767467in}{1.216635in}}{\pgfqpoint{1.772992in}{1.216635in}}%
\pgfpathclose%
\pgfusepath{stroke,fill}%
\end{pgfscope}%
\begin{pgfscope}%
\pgfpathrectangle{\pgfqpoint{0.562500in}{0.275000in}}{\pgfqpoint{3.487500in}{1.925000in}}%
\pgfusepath{clip}%
\pgfsetbuttcap%
\pgfsetroundjoin%
\definecolor{currentfill}{rgb}{0.000000,0.000000,0.000000}%
\pgfsetfillcolor{currentfill}%
\pgfsetlinewidth{1.003750pt}%
\definecolor{currentstroke}{rgb}{0.000000,0.000000,0.000000}%
\pgfsetstrokecolor{currentstroke}%
\pgfsetdash{}{0pt}%
\pgfpathmoveto{\pgfqpoint{1.772992in}{1.216635in}}%
\pgfpathcurveto{\pgfqpoint{1.778517in}{1.216635in}}{\pgfqpoint{1.783816in}{1.218830in}}{\pgfqpoint{1.787723in}{1.222736in}}%
\pgfpathcurveto{\pgfqpoint{1.791630in}{1.226643in}}{\pgfqpoint{1.793825in}{1.231943in}}{\pgfqpoint{1.793825in}{1.237468in}}%
\pgfpathcurveto{\pgfqpoint{1.793825in}{1.242993in}}{\pgfqpoint{1.791630in}{1.248292in}}{\pgfqpoint{1.787723in}{1.252199in}}%
\pgfpathcurveto{\pgfqpoint{1.783816in}{1.256106in}}{\pgfqpoint{1.778517in}{1.258301in}}{\pgfqpoint{1.772992in}{1.258301in}}%
\pgfpathcurveto{\pgfqpoint{1.767467in}{1.258301in}}{\pgfqpoint{1.762167in}{1.256106in}}{\pgfqpoint{1.758260in}{1.252199in}}%
\pgfpathcurveto{\pgfqpoint{1.754353in}{1.248292in}}{\pgfqpoint{1.752158in}{1.242993in}}{\pgfqpoint{1.752158in}{1.237468in}}%
\pgfpathcurveto{\pgfqpoint{1.752158in}{1.231943in}}{\pgfqpoint{1.754353in}{1.226643in}}{\pgfqpoint{1.758260in}{1.222736in}}%
\pgfpathcurveto{\pgfqpoint{1.762167in}{1.218830in}}{\pgfqpoint{1.767467in}{1.216635in}}{\pgfqpoint{1.772992in}{1.216635in}}%
\pgfpathclose%
\pgfusepath{stroke,fill}%
\end{pgfscope}%
\begin{pgfscope}%
\pgfpathrectangle{\pgfqpoint{0.562500in}{0.275000in}}{\pgfqpoint{3.487500in}{1.925000in}}%
\pgfusepath{clip}%
\pgfsetbuttcap%
\pgfsetroundjoin%
\definecolor{currentfill}{rgb}{0.000000,0.000000,0.000000}%
\pgfsetfillcolor{currentfill}%
\pgfsetlinewidth{1.003750pt}%
\definecolor{currentstroke}{rgb}{0.000000,0.000000,0.000000}%
\pgfsetstrokecolor{currentstroke}%
\pgfsetdash{}{0pt}%
\pgfpathmoveto{\pgfqpoint{1.772992in}{1.216635in}}%
\pgfpathcurveto{\pgfqpoint{1.778517in}{1.216635in}}{\pgfqpoint{1.783816in}{1.218830in}}{\pgfqpoint{1.787723in}{1.222736in}}%
\pgfpathcurveto{\pgfqpoint{1.791630in}{1.226643in}}{\pgfqpoint{1.793825in}{1.231943in}}{\pgfqpoint{1.793825in}{1.237468in}}%
\pgfpathcurveto{\pgfqpoint{1.793825in}{1.242993in}}{\pgfqpoint{1.791630in}{1.248292in}}{\pgfqpoint{1.787723in}{1.252199in}}%
\pgfpathcurveto{\pgfqpoint{1.783816in}{1.256106in}}{\pgfqpoint{1.778517in}{1.258301in}}{\pgfqpoint{1.772992in}{1.258301in}}%
\pgfpathcurveto{\pgfqpoint{1.767467in}{1.258301in}}{\pgfqpoint{1.762167in}{1.256106in}}{\pgfqpoint{1.758260in}{1.252199in}}%
\pgfpathcurveto{\pgfqpoint{1.754353in}{1.248292in}}{\pgfqpoint{1.752158in}{1.242993in}}{\pgfqpoint{1.752158in}{1.237468in}}%
\pgfpathcurveto{\pgfqpoint{1.752158in}{1.231943in}}{\pgfqpoint{1.754353in}{1.226643in}}{\pgfqpoint{1.758260in}{1.222736in}}%
\pgfpathcurveto{\pgfqpoint{1.762167in}{1.218830in}}{\pgfqpoint{1.767467in}{1.216635in}}{\pgfqpoint{1.772992in}{1.216635in}}%
\pgfpathclose%
\pgfusepath{stroke,fill}%
\end{pgfscope}%
\begin{pgfscope}%
\pgfpathrectangle{\pgfqpoint{0.562500in}{0.275000in}}{\pgfqpoint{3.487500in}{1.925000in}}%
\pgfusepath{clip}%
\pgfsetbuttcap%
\pgfsetroundjoin%
\definecolor{currentfill}{rgb}{0.000000,0.000000,0.000000}%
\pgfsetfillcolor{currentfill}%
\pgfsetlinewidth{1.003750pt}%
\definecolor{currentstroke}{rgb}{0.000000,0.000000,0.000000}%
\pgfsetstrokecolor{currentstroke}%
\pgfsetdash{}{0pt}%
\pgfpathmoveto{\pgfqpoint{1.772992in}{2.076667in}}%
\pgfpathcurveto{\pgfqpoint{1.778517in}{2.076667in}}{\pgfqpoint{1.783816in}{2.078862in}}{\pgfqpoint{1.787723in}{2.082769in}}%
\pgfpathcurveto{\pgfqpoint{1.791630in}{2.086675in}}{\pgfqpoint{1.793825in}{2.091975in}}{\pgfqpoint{1.793825in}{2.097500in}}%
\pgfpathcurveto{\pgfqpoint{1.793825in}{2.103025in}}{\pgfqpoint{1.791630in}{2.108325in}}{\pgfqpoint{1.787723in}{2.112231in}}%
\pgfpathcurveto{\pgfqpoint{1.783816in}{2.116138in}}{\pgfqpoint{1.778517in}{2.118333in}}{\pgfqpoint{1.772992in}{2.118333in}}%
\pgfpathcurveto{\pgfqpoint{1.767467in}{2.118333in}}{\pgfqpoint{1.762167in}{2.116138in}}{\pgfqpoint{1.758260in}{2.112231in}}%
\pgfpathcurveto{\pgfqpoint{1.754353in}{2.108325in}}{\pgfqpoint{1.752158in}{2.103025in}}{\pgfqpoint{1.752158in}{2.097500in}}%
\pgfpathcurveto{\pgfqpoint{1.752158in}{2.091975in}}{\pgfqpoint{1.754353in}{2.086675in}}{\pgfqpoint{1.758260in}{2.082769in}}%
\pgfpathcurveto{\pgfqpoint{1.762167in}{2.078862in}}{\pgfqpoint{1.767467in}{2.076667in}}{\pgfqpoint{1.772992in}{2.076667in}}%
\pgfpathclose%
\pgfusepath{stroke,fill}%
\end{pgfscope}%
\begin{pgfscope}%
\pgfpathrectangle{\pgfqpoint{0.562500in}{0.275000in}}{\pgfqpoint{3.487500in}{1.925000in}}%
\pgfusepath{clip}%
\pgfsetbuttcap%
\pgfsetroundjoin%
\definecolor{currentfill}{rgb}{0.000000,0.000000,0.000000}%
\pgfsetfillcolor{currentfill}%
\pgfsetlinewidth{1.003750pt}%
\definecolor{currentstroke}{rgb}{0.000000,0.000000,0.000000}%
\pgfsetstrokecolor{currentstroke}%
\pgfsetdash{}{0pt}%
\pgfpathmoveto{\pgfqpoint{1.772992in}{1.216635in}}%
\pgfpathcurveto{\pgfqpoint{1.778517in}{1.216635in}}{\pgfqpoint{1.783816in}{1.218830in}}{\pgfqpoint{1.787723in}{1.222736in}}%
\pgfpathcurveto{\pgfqpoint{1.791630in}{1.226643in}}{\pgfqpoint{1.793825in}{1.231943in}}{\pgfqpoint{1.793825in}{1.237468in}}%
\pgfpathcurveto{\pgfqpoint{1.793825in}{1.242993in}}{\pgfqpoint{1.791630in}{1.248292in}}{\pgfqpoint{1.787723in}{1.252199in}}%
\pgfpathcurveto{\pgfqpoint{1.783816in}{1.256106in}}{\pgfqpoint{1.778517in}{1.258301in}}{\pgfqpoint{1.772992in}{1.258301in}}%
\pgfpathcurveto{\pgfqpoint{1.767467in}{1.258301in}}{\pgfqpoint{1.762167in}{1.256106in}}{\pgfqpoint{1.758260in}{1.252199in}}%
\pgfpathcurveto{\pgfqpoint{1.754353in}{1.248292in}}{\pgfqpoint{1.752158in}{1.242993in}}{\pgfqpoint{1.752158in}{1.237468in}}%
\pgfpathcurveto{\pgfqpoint{1.752158in}{1.231943in}}{\pgfqpoint{1.754353in}{1.226643in}}{\pgfqpoint{1.758260in}{1.222736in}}%
\pgfpathcurveto{\pgfqpoint{1.762167in}{1.218830in}}{\pgfqpoint{1.767467in}{1.216635in}}{\pgfqpoint{1.772992in}{1.216635in}}%
\pgfpathclose%
\pgfusepath{stroke,fill}%
\end{pgfscope}%
\begin{pgfscope}%
\pgfpathrectangle{\pgfqpoint{0.562500in}{0.275000in}}{\pgfqpoint{3.487500in}{1.925000in}}%
\pgfusepath{clip}%
\pgfsetbuttcap%
\pgfsetroundjoin%
\definecolor{currentfill}{rgb}{0.000000,0.000000,0.000000}%
\pgfsetfillcolor{currentfill}%
\pgfsetlinewidth{1.003750pt}%
\definecolor{currentstroke}{rgb}{0.000000,0.000000,0.000000}%
\pgfsetstrokecolor{currentstroke}%
\pgfsetdash{}{0pt}%
\pgfpathmoveto{\pgfqpoint{1.772992in}{1.216635in}}%
\pgfpathcurveto{\pgfqpoint{1.778517in}{1.216635in}}{\pgfqpoint{1.783816in}{1.218830in}}{\pgfqpoint{1.787723in}{1.222736in}}%
\pgfpathcurveto{\pgfqpoint{1.791630in}{1.226643in}}{\pgfqpoint{1.793825in}{1.231943in}}{\pgfqpoint{1.793825in}{1.237468in}}%
\pgfpathcurveto{\pgfqpoint{1.793825in}{1.242993in}}{\pgfqpoint{1.791630in}{1.248292in}}{\pgfqpoint{1.787723in}{1.252199in}}%
\pgfpathcurveto{\pgfqpoint{1.783816in}{1.256106in}}{\pgfqpoint{1.778517in}{1.258301in}}{\pgfqpoint{1.772992in}{1.258301in}}%
\pgfpathcurveto{\pgfqpoint{1.767467in}{1.258301in}}{\pgfqpoint{1.762167in}{1.256106in}}{\pgfqpoint{1.758260in}{1.252199in}}%
\pgfpathcurveto{\pgfqpoint{1.754353in}{1.248292in}}{\pgfqpoint{1.752158in}{1.242993in}}{\pgfqpoint{1.752158in}{1.237468in}}%
\pgfpathcurveto{\pgfqpoint{1.752158in}{1.231943in}}{\pgfqpoint{1.754353in}{1.226643in}}{\pgfqpoint{1.758260in}{1.222736in}}%
\pgfpathcurveto{\pgfqpoint{1.762167in}{1.218830in}}{\pgfqpoint{1.767467in}{1.216635in}}{\pgfqpoint{1.772992in}{1.216635in}}%
\pgfpathclose%
\pgfusepath{stroke,fill}%
\end{pgfscope}%
\begin{pgfscope}%
\pgfpathrectangle{\pgfqpoint{0.562500in}{0.275000in}}{\pgfqpoint{3.487500in}{1.925000in}}%
\pgfusepath{clip}%
\pgfsetbuttcap%
\pgfsetroundjoin%
\definecolor{currentfill}{rgb}{0.000000,0.000000,0.000000}%
\pgfsetfillcolor{currentfill}%
\pgfsetlinewidth{1.003750pt}%
\definecolor{currentstroke}{rgb}{0.000000,0.000000,0.000000}%
\pgfsetstrokecolor{currentstroke}%
\pgfsetdash{}{0pt}%
\pgfpathmoveto{\pgfqpoint{1.772992in}{1.216635in}}%
\pgfpathcurveto{\pgfqpoint{1.778517in}{1.216635in}}{\pgfqpoint{1.783816in}{1.218830in}}{\pgfqpoint{1.787723in}{1.222736in}}%
\pgfpathcurveto{\pgfqpoint{1.791630in}{1.226643in}}{\pgfqpoint{1.793825in}{1.231943in}}{\pgfqpoint{1.793825in}{1.237468in}}%
\pgfpathcurveto{\pgfqpoint{1.793825in}{1.242993in}}{\pgfqpoint{1.791630in}{1.248292in}}{\pgfqpoint{1.787723in}{1.252199in}}%
\pgfpathcurveto{\pgfqpoint{1.783816in}{1.256106in}}{\pgfqpoint{1.778517in}{1.258301in}}{\pgfqpoint{1.772992in}{1.258301in}}%
\pgfpathcurveto{\pgfqpoint{1.767467in}{1.258301in}}{\pgfqpoint{1.762167in}{1.256106in}}{\pgfqpoint{1.758260in}{1.252199in}}%
\pgfpathcurveto{\pgfqpoint{1.754353in}{1.248292in}}{\pgfqpoint{1.752158in}{1.242993in}}{\pgfqpoint{1.752158in}{1.237468in}}%
\pgfpathcurveto{\pgfqpoint{1.752158in}{1.231943in}}{\pgfqpoint{1.754353in}{1.226643in}}{\pgfqpoint{1.758260in}{1.222736in}}%
\pgfpathcurveto{\pgfqpoint{1.762167in}{1.218830in}}{\pgfqpoint{1.767467in}{1.216635in}}{\pgfqpoint{1.772992in}{1.216635in}}%
\pgfpathclose%
\pgfusepath{stroke,fill}%
\end{pgfscope}%
\begin{pgfscope}%
\pgfpathrectangle{\pgfqpoint{0.562500in}{0.275000in}}{\pgfqpoint{3.487500in}{1.925000in}}%
\pgfusepath{clip}%
\pgfsetbuttcap%
\pgfsetroundjoin%
\definecolor{currentfill}{rgb}{0.000000,0.000000,0.000000}%
\pgfsetfillcolor{currentfill}%
\pgfsetlinewidth{1.003750pt}%
\definecolor{currentstroke}{rgb}{0.000000,0.000000,0.000000}%
\pgfsetstrokecolor{currentstroke}%
\pgfsetdash{}{0pt}%
\pgfpathmoveto{\pgfqpoint{1.772992in}{1.216635in}}%
\pgfpathcurveto{\pgfqpoint{1.778517in}{1.216635in}}{\pgfqpoint{1.783816in}{1.218830in}}{\pgfqpoint{1.787723in}{1.222736in}}%
\pgfpathcurveto{\pgfqpoint{1.791630in}{1.226643in}}{\pgfqpoint{1.793825in}{1.231943in}}{\pgfqpoint{1.793825in}{1.237468in}}%
\pgfpathcurveto{\pgfqpoint{1.793825in}{1.242993in}}{\pgfqpoint{1.791630in}{1.248292in}}{\pgfqpoint{1.787723in}{1.252199in}}%
\pgfpathcurveto{\pgfqpoint{1.783816in}{1.256106in}}{\pgfqpoint{1.778517in}{1.258301in}}{\pgfqpoint{1.772992in}{1.258301in}}%
\pgfpathcurveto{\pgfqpoint{1.767467in}{1.258301in}}{\pgfqpoint{1.762167in}{1.256106in}}{\pgfqpoint{1.758260in}{1.252199in}}%
\pgfpathcurveto{\pgfqpoint{1.754353in}{1.248292in}}{\pgfqpoint{1.752158in}{1.242993in}}{\pgfqpoint{1.752158in}{1.237468in}}%
\pgfpathcurveto{\pgfqpoint{1.752158in}{1.231943in}}{\pgfqpoint{1.754353in}{1.226643in}}{\pgfqpoint{1.758260in}{1.222736in}}%
\pgfpathcurveto{\pgfqpoint{1.762167in}{1.218830in}}{\pgfqpoint{1.767467in}{1.216635in}}{\pgfqpoint{1.772992in}{1.216635in}}%
\pgfpathclose%
\pgfusepath{stroke,fill}%
\end{pgfscope}%
\begin{pgfscope}%
\pgfpathrectangle{\pgfqpoint{0.562500in}{0.275000in}}{\pgfqpoint{3.487500in}{1.925000in}}%
\pgfusepath{clip}%
\pgfsetbuttcap%
\pgfsetroundjoin%
\definecolor{currentfill}{rgb}{0.000000,0.000000,0.000000}%
\pgfsetfillcolor{currentfill}%
\pgfsetlinewidth{1.003750pt}%
\definecolor{currentstroke}{rgb}{0.000000,0.000000,0.000000}%
\pgfsetstrokecolor{currentstroke}%
\pgfsetdash{}{0pt}%
\pgfpathmoveto{\pgfqpoint{1.772992in}{1.216635in}}%
\pgfpathcurveto{\pgfqpoint{1.778517in}{1.216635in}}{\pgfqpoint{1.783816in}{1.218830in}}{\pgfqpoint{1.787723in}{1.222736in}}%
\pgfpathcurveto{\pgfqpoint{1.791630in}{1.226643in}}{\pgfqpoint{1.793825in}{1.231943in}}{\pgfqpoint{1.793825in}{1.237468in}}%
\pgfpathcurveto{\pgfqpoint{1.793825in}{1.242993in}}{\pgfqpoint{1.791630in}{1.248292in}}{\pgfqpoint{1.787723in}{1.252199in}}%
\pgfpathcurveto{\pgfqpoint{1.783816in}{1.256106in}}{\pgfqpoint{1.778517in}{1.258301in}}{\pgfqpoint{1.772992in}{1.258301in}}%
\pgfpathcurveto{\pgfqpoint{1.767467in}{1.258301in}}{\pgfqpoint{1.762167in}{1.256106in}}{\pgfqpoint{1.758260in}{1.252199in}}%
\pgfpathcurveto{\pgfqpoint{1.754353in}{1.248292in}}{\pgfqpoint{1.752158in}{1.242993in}}{\pgfqpoint{1.752158in}{1.237468in}}%
\pgfpathcurveto{\pgfqpoint{1.752158in}{1.231943in}}{\pgfqpoint{1.754353in}{1.226643in}}{\pgfqpoint{1.758260in}{1.222736in}}%
\pgfpathcurveto{\pgfqpoint{1.762167in}{1.218830in}}{\pgfqpoint{1.767467in}{1.216635in}}{\pgfqpoint{1.772992in}{1.216635in}}%
\pgfpathclose%
\pgfusepath{stroke,fill}%
\end{pgfscope}%
\begin{pgfscope}%
\pgfpathrectangle{\pgfqpoint{0.562500in}{0.275000in}}{\pgfqpoint{3.487500in}{1.925000in}}%
\pgfusepath{clip}%
\pgfsetbuttcap%
\pgfsetroundjoin%
\definecolor{currentfill}{rgb}{0.000000,0.000000,0.000000}%
\pgfsetfillcolor{currentfill}%
\pgfsetlinewidth{1.003750pt}%
\definecolor{currentstroke}{rgb}{0.000000,0.000000,0.000000}%
\pgfsetstrokecolor{currentstroke}%
\pgfsetdash{}{0pt}%
\pgfpathmoveto{\pgfqpoint{1.772992in}{2.076667in}}%
\pgfpathcurveto{\pgfqpoint{1.778517in}{2.076667in}}{\pgfqpoint{1.783816in}{2.078862in}}{\pgfqpoint{1.787723in}{2.082769in}}%
\pgfpathcurveto{\pgfqpoint{1.791630in}{2.086675in}}{\pgfqpoint{1.793825in}{2.091975in}}{\pgfqpoint{1.793825in}{2.097500in}}%
\pgfpathcurveto{\pgfqpoint{1.793825in}{2.103025in}}{\pgfqpoint{1.791630in}{2.108325in}}{\pgfqpoint{1.787723in}{2.112231in}}%
\pgfpathcurveto{\pgfqpoint{1.783816in}{2.116138in}}{\pgfqpoint{1.778517in}{2.118333in}}{\pgfqpoint{1.772992in}{2.118333in}}%
\pgfpathcurveto{\pgfqpoint{1.767467in}{2.118333in}}{\pgfqpoint{1.762167in}{2.116138in}}{\pgfqpoint{1.758260in}{2.112231in}}%
\pgfpathcurveto{\pgfqpoint{1.754353in}{2.108325in}}{\pgfqpoint{1.752158in}{2.103025in}}{\pgfqpoint{1.752158in}{2.097500in}}%
\pgfpathcurveto{\pgfqpoint{1.752158in}{2.091975in}}{\pgfqpoint{1.754353in}{2.086675in}}{\pgfqpoint{1.758260in}{2.082769in}}%
\pgfpathcurveto{\pgfqpoint{1.762167in}{2.078862in}}{\pgfqpoint{1.767467in}{2.076667in}}{\pgfqpoint{1.772992in}{2.076667in}}%
\pgfpathclose%
\pgfusepath{stroke,fill}%
\end{pgfscope}%
\begin{pgfscope}%
\pgfpathrectangle{\pgfqpoint{0.562500in}{0.275000in}}{\pgfqpoint{3.487500in}{1.925000in}}%
\pgfusepath{clip}%
\pgfsetbuttcap%
\pgfsetroundjoin%
\definecolor{currentfill}{rgb}{0.000000,0.000000,0.000000}%
\pgfsetfillcolor{currentfill}%
\pgfsetlinewidth{1.003750pt}%
\definecolor{currentstroke}{rgb}{0.000000,0.000000,0.000000}%
\pgfsetstrokecolor{currentstroke}%
\pgfsetdash{}{0pt}%
\pgfpathmoveto{\pgfqpoint{1.772992in}{1.216635in}}%
\pgfpathcurveto{\pgfqpoint{1.778517in}{1.216635in}}{\pgfqpoint{1.783816in}{1.218830in}}{\pgfqpoint{1.787723in}{1.222736in}}%
\pgfpathcurveto{\pgfqpoint{1.791630in}{1.226643in}}{\pgfqpoint{1.793825in}{1.231943in}}{\pgfqpoint{1.793825in}{1.237468in}}%
\pgfpathcurveto{\pgfqpoint{1.793825in}{1.242993in}}{\pgfqpoint{1.791630in}{1.248292in}}{\pgfqpoint{1.787723in}{1.252199in}}%
\pgfpathcurveto{\pgfqpoint{1.783816in}{1.256106in}}{\pgfqpoint{1.778517in}{1.258301in}}{\pgfqpoint{1.772992in}{1.258301in}}%
\pgfpathcurveto{\pgfqpoint{1.767467in}{1.258301in}}{\pgfqpoint{1.762167in}{1.256106in}}{\pgfqpoint{1.758260in}{1.252199in}}%
\pgfpathcurveto{\pgfqpoint{1.754353in}{1.248292in}}{\pgfqpoint{1.752158in}{1.242993in}}{\pgfqpoint{1.752158in}{1.237468in}}%
\pgfpathcurveto{\pgfqpoint{1.752158in}{1.231943in}}{\pgfqpoint{1.754353in}{1.226643in}}{\pgfqpoint{1.758260in}{1.222736in}}%
\pgfpathcurveto{\pgfqpoint{1.762167in}{1.218830in}}{\pgfqpoint{1.767467in}{1.216635in}}{\pgfqpoint{1.772992in}{1.216635in}}%
\pgfpathclose%
\pgfusepath{stroke,fill}%
\end{pgfscope}%
\begin{pgfscope}%
\pgfpathrectangle{\pgfqpoint{0.562500in}{0.275000in}}{\pgfqpoint{3.487500in}{1.925000in}}%
\pgfusepath{clip}%
\pgfsetbuttcap%
\pgfsetroundjoin%
\definecolor{currentfill}{rgb}{0.000000,0.000000,0.000000}%
\pgfsetfillcolor{currentfill}%
\pgfsetlinewidth{1.003750pt}%
\definecolor{currentstroke}{rgb}{0.000000,0.000000,0.000000}%
\pgfsetstrokecolor{currentstroke}%
\pgfsetdash{}{0pt}%
\pgfpathmoveto{\pgfqpoint{1.772992in}{1.216635in}}%
\pgfpathcurveto{\pgfqpoint{1.778517in}{1.216635in}}{\pgfqpoint{1.783816in}{1.218830in}}{\pgfqpoint{1.787723in}{1.222736in}}%
\pgfpathcurveto{\pgfqpoint{1.791630in}{1.226643in}}{\pgfqpoint{1.793825in}{1.231943in}}{\pgfqpoint{1.793825in}{1.237468in}}%
\pgfpathcurveto{\pgfqpoint{1.793825in}{1.242993in}}{\pgfqpoint{1.791630in}{1.248292in}}{\pgfqpoint{1.787723in}{1.252199in}}%
\pgfpathcurveto{\pgfqpoint{1.783816in}{1.256106in}}{\pgfqpoint{1.778517in}{1.258301in}}{\pgfqpoint{1.772992in}{1.258301in}}%
\pgfpathcurveto{\pgfqpoint{1.767467in}{1.258301in}}{\pgfqpoint{1.762167in}{1.256106in}}{\pgfqpoint{1.758260in}{1.252199in}}%
\pgfpathcurveto{\pgfqpoint{1.754353in}{1.248292in}}{\pgfqpoint{1.752158in}{1.242993in}}{\pgfqpoint{1.752158in}{1.237468in}}%
\pgfpathcurveto{\pgfqpoint{1.752158in}{1.231943in}}{\pgfqpoint{1.754353in}{1.226643in}}{\pgfqpoint{1.758260in}{1.222736in}}%
\pgfpathcurveto{\pgfqpoint{1.762167in}{1.218830in}}{\pgfqpoint{1.767467in}{1.216635in}}{\pgfqpoint{1.772992in}{1.216635in}}%
\pgfpathclose%
\pgfusepath{stroke,fill}%
\end{pgfscope}%
\begin{pgfscope}%
\pgfpathrectangle{\pgfqpoint{0.562500in}{0.275000in}}{\pgfqpoint{3.487500in}{1.925000in}}%
\pgfusepath{clip}%
\pgfsetbuttcap%
\pgfsetroundjoin%
\definecolor{currentfill}{rgb}{0.000000,0.000000,0.000000}%
\pgfsetfillcolor{currentfill}%
\pgfsetlinewidth{1.003750pt}%
\definecolor{currentstroke}{rgb}{0.000000,0.000000,0.000000}%
\pgfsetstrokecolor{currentstroke}%
\pgfsetdash{}{0pt}%
\pgfpathmoveto{\pgfqpoint{1.772992in}{1.216635in}}%
\pgfpathcurveto{\pgfqpoint{1.778517in}{1.216635in}}{\pgfqpoint{1.783816in}{1.218830in}}{\pgfqpoint{1.787723in}{1.222736in}}%
\pgfpathcurveto{\pgfqpoint{1.791630in}{1.226643in}}{\pgfqpoint{1.793825in}{1.231943in}}{\pgfqpoint{1.793825in}{1.237468in}}%
\pgfpathcurveto{\pgfqpoint{1.793825in}{1.242993in}}{\pgfqpoint{1.791630in}{1.248292in}}{\pgfqpoint{1.787723in}{1.252199in}}%
\pgfpathcurveto{\pgfqpoint{1.783816in}{1.256106in}}{\pgfqpoint{1.778517in}{1.258301in}}{\pgfqpoint{1.772992in}{1.258301in}}%
\pgfpathcurveto{\pgfqpoint{1.767467in}{1.258301in}}{\pgfqpoint{1.762167in}{1.256106in}}{\pgfqpoint{1.758260in}{1.252199in}}%
\pgfpathcurveto{\pgfqpoint{1.754353in}{1.248292in}}{\pgfqpoint{1.752158in}{1.242993in}}{\pgfqpoint{1.752158in}{1.237468in}}%
\pgfpathcurveto{\pgfqpoint{1.752158in}{1.231943in}}{\pgfqpoint{1.754353in}{1.226643in}}{\pgfqpoint{1.758260in}{1.222736in}}%
\pgfpathcurveto{\pgfqpoint{1.762167in}{1.218830in}}{\pgfqpoint{1.767467in}{1.216635in}}{\pgfqpoint{1.772992in}{1.216635in}}%
\pgfpathclose%
\pgfusepath{stroke,fill}%
\end{pgfscope}%
\begin{pgfscope}%
\pgfpathrectangle{\pgfqpoint{0.562500in}{0.275000in}}{\pgfqpoint{3.487500in}{1.925000in}}%
\pgfusepath{clip}%
\pgfsetbuttcap%
\pgfsetroundjoin%
\definecolor{currentfill}{rgb}{0.000000,0.000000,0.000000}%
\pgfsetfillcolor{currentfill}%
\pgfsetlinewidth{1.003750pt}%
\definecolor{currentstroke}{rgb}{0.000000,0.000000,0.000000}%
\pgfsetstrokecolor{currentstroke}%
\pgfsetdash{}{0pt}%
\pgfpathmoveto{\pgfqpoint{1.772992in}{1.216635in}}%
\pgfpathcurveto{\pgfqpoint{1.778517in}{1.216635in}}{\pgfqpoint{1.783816in}{1.218830in}}{\pgfqpoint{1.787723in}{1.222736in}}%
\pgfpathcurveto{\pgfqpoint{1.791630in}{1.226643in}}{\pgfqpoint{1.793825in}{1.231943in}}{\pgfqpoint{1.793825in}{1.237468in}}%
\pgfpathcurveto{\pgfqpoint{1.793825in}{1.242993in}}{\pgfqpoint{1.791630in}{1.248292in}}{\pgfqpoint{1.787723in}{1.252199in}}%
\pgfpathcurveto{\pgfqpoint{1.783816in}{1.256106in}}{\pgfqpoint{1.778517in}{1.258301in}}{\pgfqpoint{1.772992in}{1.258301in}}%
\pgfpathcurveto{\pgfqpoint{1.767467in}{1.258301in}}{\pgfqpoint{1.762167in}{1.256106in}}{\pgfqpoint{1.758260in}{1.252199in}}%
\pgfpathcurveto{\pgfqpoint{1.754353in}{1.248292in}}{\pgfqpoint{1.752158in}{1.242993in}}{\pgfqpoint{1.752158in}{1.237468in}}%
\pgfpathcurveto{\pgfqpoint{1.752158in}{1.231943in}}{\pgfqpoint{1.754353in}{1.226643in}}{\pgfqpoint{1.758260in}{1.222736in}}%
\pgfpathcurveto{\pgfqpoint{1.762167in}{1.218830in}}{\pgfqpoint{1.767467in}{1.216635in}}{\pgfqpoint{1.772992in}{1.216635in}}%
\pgfpathclose%
\pgfusepath{stroke,fill}%
\end{pgfscope}%
\begin{pgfscope}%
\pgfpathrectangle{\pgfqpoint{0.562500in}{0.275000in}}{\pgfqpoint{3.487500in}{1.925000in}}%
\pgfusepath{clip}%
\pgfsetbuttcap%
\pgfsetroundjoin%
\definecolor{currentfill}{rgb}{0.000000,0.000000,0.000000}%
\pgfsetfillcolor{currentfill}%
\pgfsetlinewidth{1.003750pt}%
\definecolor{currentstroke}{rgb}{0.000000,0.000000,0.000000}%
\pgfsetstrokecolor{currentstroke}%
\pgfsetdash{}{0pt}%
\pgfpathmoveto{\pgfqpoint{1.772992in}{1.216635in}}%
\pgfpathcurveto{\pgfqpoint{1.778517in}{1.216635in}}{\pgfqpoint{1.783816in}{1.218830in}}{\pgfqpoint{1.787723in}{1.222736in}}%
\pgfpathcurveto{\pgfqpoint{1.791630in}{1.226643in}}{\pgfqpoint{1.793825in}{1.231943in}}{\pgfqpoint{1.793825in}{1.237468in}}%
\pgfpathcurveto{\pgfqpoint{1.793825in}{1.242993in}}{\pgfqpoint{1.791630in}{1.248292in}}{\pgfqpoint{1.787723in}{1.252199in}}%
\pgfpathcurveto{\pgfqpoint{1.783816in}{1.256106in}}{\pgfqpoint{1.778517in}{1.258301in}}{\pgfqpoint{1.772992in}{1.258301in}}%
\pgfpathcurveto{\pgfqpoint{1.767467in}{1.258301in}}{\pgfqpoint{1.762167in}{1.256106in}}{\pgfqpoint{1.758260in}{1.252199in}}%
\pgfpathcurveto{\pgfqpoint{1.754353in}{1.248292in}}{\pgfqpoint{1.752158in}{1.242993in}}{\pgfqpoint{1.752158in}{1.237468in}}%
\pgfpathcurveto{\pgfqpoint{1.752158in}{1.231943in}}{\pgfqpoint{1.754353in}{1.226643in}}{\pgfqpoint{1.758260in}{1.222736in}}%
\pgfpathcurveto{\pgfqpoint{1.762167in}{1.218830in}}{\pgfqpoint{1.767467in}{1.216635in}}{\pgfqpoint{1.772992in}{1.216635in}}%
\pgfpathclose%
\pgfusepath{stroke,fill}%
\end{pgfscope}%
\begin{pgfscope}%
\pgfpathrectangle{\pgfqpoint{0.562500in}{0.275000in}}{\pgfqpoint{3.487500in}{1.925000in}}%
\pgfusepath{clip}%
\pgfsetbuttcap%
\pgfsetroundjoin%
\definecolor{currentfill}{rgb}{0.000000,0.000000,0.000000}%
\pgfsetfillcolor{currentfill}%
\pgfsetlinewidth{1.003750pt}%
\definecolor{currentstroke}{rgb}{0.000000,0.000000,0.000000}%
\pgfsetstrokecolor{currentstroke}%
\pgfsetdash{}{0pt}%
\pgfpathmoveto{\pgfqpoint{1.772992in}{2.076667in}}%
\pgfpathcurveto{\pgfqpoint{1.778517in}{2.076667in}}{\pgfqpoint{1.783816in}{2.078862in}}{\pgfqpoint{1.787723in}{2.082769in}}%
\pgfpathcurveto{\pgfqpoint{1.791630in}{2.086675in}}{\pgfqpoint{1.793825in}{2.091975in}}{\pgfqpoint{1.793825in}{2.097500in}}%
\pgfpathcurveto{\pgfqpoint{1.793825in}{2.103025in}}{\pgfqpoint{1.791630in}{2.108325in}}{\pgfqpoint{1.787723in}{2.112231in}}%
\pgfpathcurveto{\pgfqpoint{1.783816in}{2.116138in}}{\pgfqpoint{1.778517in}{2.118333in}}{\pgfqpoint{1.772992in}{2.118333in}}%
\pgfpathcurveto{\pgfqpoint{1.767467in}{2.118333in}}{\pgfqpoint{1.762167in}{2.116138in}}{\pgfqpoint{1.758260in}{2.112231in}}%
\pgfpathcurveto{\pgfqpoint{1.754353in}{2.108325in}}{\pgfqpoint{1.752158in}{2.103025in}}{\pgfqpoint{1.752158in}{2.097500in}}%
\pgfpathcurveto{\pgfqpoint{1.752158in}{2.091975in}}{\pgfqpoint{1.754353in}{2.086675in}}{\pgfqpoint{1.758260in}{2.082769in}}%
\pgfpathcurveto{\pgfqpoint{1.762167in}{2.078862in}}{\pgfqpoint{1.767467in}{2.076667in}}{\pgfqpoint{1.772992in}{2.076667in}}%
\pgfpathclose%
\pgfusepath{stroke,fill}%
\end{pgfscope}%
\begin{pgfscope}%
\pgfpathrectangle{\pgfqpoint{0.562500in}{0.275000in}}{\pgfqpoint{3.487500in}{1.925000in}}%
\pgfusepath{clip}%
\pgfsetbuttcap%
\pgfsetroundjoin%
\definecolor{currentfill}{rgb}{0.000000,0.000000,0.000000}%
\pgfsetfillcolor{currentfill}%
\pgfsetlinewidth{1.003750pt}%
\definecolor{currentstroke}{rgb}{0.000000,0.000000,0.000000}%
\pgfsetstrokecolor{currentstroke}%
\pgfsetdash{}{0pt}%
\pgfpathmoveto{\pgfqpoint{1.772992in}{1.216635in}}%
\pgfpathcurveto{\pgfqpoint{1.778517in}{1.216635in}}{\pgfqpoint{1.783816in}{1.218830in}}{\pgfqpoint{1.787723in}{1.222736in}}%
\pgfpathcurveto{\pgfqpoint{1.791630in}{1.226643in}}{\pgfqpoint{1.793825in}{1.231943in}}{\pgfqpoint{1.793825in}{1.237468in}}%
\pgfpathcurveto{\pgfqpoint{1.793825in}{1.242993in}}{\pgfqpoint{1.791630in}{1.248292in}}{\pgfqpoint{1.787723in}{1.252199in}}%
\pgfpathcurveto{\pgfqpoint{1.783816in}{1.256106in}}{\pgfqpoint{1.778517in}{1.258301in}}{\pgfqpoint{1.772992in}{1.258301in}}%
\pgfpathcurveto{\pgfqpoint{1.767467in}{1.258301in}}{\pgfqpoint{1.762167in}{1.256106in}}{\pgfqpoint{1.758260in}{1.252199in}}%
\pgfpathcurveto{\pgfqpoint{1.754353in}{1.248292in}}{\pgfqpoint{1.752158in}{1.242993in}}{\pgfqpoint{1.752158in}{1.237468in}}%
\pgfpathcurveto{\pgfqpoint{1.752158in}{1.231943in}}{\pgfqpoint{1.754353in}{1.226643in}}{\pgfqpoint{1.758260in}{1.222736in}}%
\pgfpathcurveto{\pgfqpoint{1.762167in}{1.218830in}}{\pgfqpoint{1.767467in}{1.216635in}}{\pgfqpoint{1.772992in}{1.216635in}}%
\pgfpathclose%
\pgfusepath{stroke,fill}%
\end{pgfscope}%
\begin{pgfscope}%
\pgfpathrectangle{\pgfqpoint{0.562500in}{0.275000in}}{\pgfqpoint{3.487500in}{1.925000in}}%
\pgfusepath{clip}%
\pgfsetbuttcap%
\pgfsetroundjoin%
\definecolor{currentfill}{rgb}{0.000000,0.000000,0.000000}%
\pgfsetfillcolor{currentfill}%
\pgfsetlinewidth{1.003750pt}%
\definecolor{currentstroke}{rgb}{0.000000,0.000000,0.000000}%
\pgfsetstrokecolor{currentstroke}%
\pgfsetdash{}{0pt}%
\pgfpathmoveto{\pgfqpoint{1.772992in}{1.216635in}}%
\pgfpathcurveto{\pgfqpoint{1.778517in}{1.216635in}}{\pgfqpoint{1.783816in}{1.218830in}}{\pgfqpoint{1.787723in}{1.222736in}}%
\pgfpathcurveto{\pgfqpoint{1.791630in}{1.226643in}}{\pgfqpoint{1.793825in}{1.231943in}}{\pgfqpoint{1.793825in}{1.237468in}}%
\pgfpathcurveto{\pgfqpoint{1.793825in}{1.242993in}}{\pgfqpoint{1.791630in}{1.248292in}}{\pgfqpoint{1.787723in}{1.252199in}}%
\pgfpathcurveto{\pgfqpoint{1.783816in}{1.256106in}}{\pgfqpoint{1.778517in}{1.258301in}}{\pgfqpoint{1.772992in}{1.258301in}}%
\pgfpathcurveto{\pgfqpoint{1.767467in}{1.258301in}}{\pgfqpoint{1.762167in}{1.256106in}}{\pgfqpoint{1.758260in}{1.252199in}}%
\pgfpathcurveto{\pgfqpoint{1.754353in}{1.248292in}}{\pgfqpoint{1.752158in}{1.242993in}}{\pgfqpoint{1.752158in}{1.237468in}}%
\pgfpathcurveto{\pgfqpoint{1.752158in}{1.231943in}}{\pgfqpoint{1.754353in}{1.226643in}}{\pgfqpoint{1.758260in}{1.222736in}}%
\pgfpathcurveto{\pgfqpoint{1.762167in}{1.218830in}}{\pgfqpoint{1.767467in}{1.216635in}}{\pgfqpoint{1.772992in}{1.216635in}}%
\pgfpathclose%
\pgfusepath{stroke,fill}%
\end{pgfscope}%
\begin{pgfscope}%
\pgfpathrectangle{\pgfqpoint{0.562500in}{0.275000in}}{\pgfqpoint{3.487500in}{1.925000in}}%
\pgfusepath{clip}%
\pgfsetbuttcap%
\pgfsetroundjoin%
\definecolor{currentfill}{rgb}{0.000000,0.000000,0.000000}%
\pgfsetfillcolor{currentfill}%
\pgfsetlinewidth{1.003750pt}%
\definecolor{currentstroke}{rgb}{0.000000,0.000000,0.000000}%
\pgfsetstrokecolor{currentstroke}%
\pgfsetdash{}{0pt}%
\pgfpathmoveto{\pgfqpoint{1.772992in}{1.216635in}}%
\pgfpathcurveto{\pgfqpoint{1.778517in}{1.216635in}}{\pgfqpoint{1.783816in}{1.218830in}}{\pgfqpoint{1.787723in}{1.222736in}}%
\pgfpathcurveto{\pgfqpoint{1.791630in}{1.226643in}}{\pgfqpoint{1.793825in}{1.231943in}}{\pgfqpoint{1.793825in}{1.237468in}}%
\pgfpathcurveto{\pgfqpoint{1.793825in}{1.242993in}}{\pgfqpoint{1.791630in}{1.248292in}}{\pgfqpoint{1.787723in}{1.252199in}}%
\pgfpathcurveto{\pgfqpoint{1.783816in}{1.256106in}}{\pgfqpoint{1.778517in}{1.258301in}}{\pgfqpoint{1.772992in}{1.258301in}}%
\pgfpathcurveto{\pgfqpoint{1.767467in}{1.258301in}}{\pgfqpoint{1.762167in}{1.256106in}}{\pgfqpoint{1.758260in}{1.252199in}}%
\pgfpathcurveto{\pgfqpoint{1.754353in}{1.248292in}}{\pgfqpoint{1.752158in}{1.242993in}}{\pgfqpoint{1.752158in}{1.237468in}}%
\pgfpathcurveto{\pgfqpoint{1.752158in}{1.231943in}}{\pgfqpoint{1.754353in}{1.226643in}}{\pgfqpoint{1.758260in}{1.222736in}}%
\pgfpathcurveto{\pgfqpoint{1.762167in}{1.218830in}}{\pgfqpoint{1.767467in}{1.216635in}}{\pgfqpoint{1.772992in}{1.216635in}}%
\pgfpathclose%
\pgfusepath{stroke,fill}%
\end{pgfscope}%
\begin{pgfscope}%
\pgfpathrectangle{\pgfqpoint{0.562500in}{0.275000in}}{\pgfqpoint{3.487500in}{1.925000in}}%
\pgfusepath{clip}%
\pgfsetbuttcap%
\pgfsetroundjoin%
\definecolor{currentfill}{rgb}{0.000000,0.000000,0.000000}%
\pgfsetfillcolor{currentfill}%
\pgfsetlinewidth{1.003750pt}%
\definecolor{currentstroke}{rgb}{0.000000,0.000000,0.000000}%
\pgfsetstrokecolor{currentstroke}%
\pgfsetdash{}{0pt}%
\pgfpathmoveto{\pgfqpoint{1.772992in}{1.216635in}}%
\pgfpathcurveto{\pgfqpoint{1.778517in}{1.216635in}}{\pgfqpoint{1.783816in}{1.218830in}}{\pgfqpoint{1.787723in}{1.222736in}}%
\pgfpathcurveto{\pgfqpoint{1.791630in}{1.226643in}}{\pgfqpoint{1.793825in}{1.231943in}}{\pgfqpoint{1.793825in}{1.237468in}}%
\pgfpathcurveto{\pgfqpoint{1.793825in}{1.242993in}}{\pgfqpoint{1.791630in}{1.248292in}}{\pgfqpoint{1.787723in}{1.252199in}}%
\pgfpathcurveto{\pgfqpoint{1.783816in}{1.256106in}}{\pgfqpoint{1.778517in}{1.258301in}}{\pgfqpoint{1.772992in}{1.258301in}}%
\pgfpathcurveto{\pgfqpoint{1.767467in}{1.258301in}}{\pgfqpoint{1.762167in}{1.256106in}}{\pgfqpoint{1.758260in}{1.252199in}}%
\pgfpathcurveto{\pgfqpoint{1.754353in}{1.248292in}}{\pgfqpoint{1.752158in}{1.242993in}}{\pgfqpoint{1.752158in}{1.237468in}}%
\pgfpathcurveto{\pgfqpoint{1.752158in}{1.231943in}}{\pgfqpoint{1.754353in}{1.226643in}}{\pgfqpoint{1.758260in}{1.222736in}}%
\pgfpathcurveto{\pgfqpoint{1.762167in}{1.218830in}}{\pgfqpoint{1.767467in}{1.216635in}}{\pgfqpoint{1.772992in}{1.216635in}}%
\pgfpathclose%
\pgfusepath{stroke,fill}%
\end{pgfscope}%
\begin{pgfscope}%
\pgfpathrectangle{\pgfqpoint{0.562500in}{0.275000in}}{\pgfqpoint{3.487500in}{1.925000in}}%
\pgfusepath{clip}%
\pgfsetbuttcap%
\pgfsetroundjoin%
\definecolor{currentfill}{rgb}{0.000000,0.000000,0.000000}%
\pgfsetfillcolor{currentfill}%
\pgfsetlinewidth{1.003750pt}%
\definecolor{currentstroke}{rgb}{0.000000,0.000000,0.000000}%
\pgfsetstrokecolor{currentstroke}%
\pgfsetdash{}{0pt}%
\pgfpathmoveto{\pgfqpoint{1.772992in}{1.216635in}}%
\pgfpathcurveto{\pgfqpoint{1.778517in}{1.216635in}}{\pgfqpoint{1.783816in}{1.218830in}}{\pgfqpoint{1.787723in}{1.222736in}}%
\pgfpathcurveto{\pgfqpoint{1.791630in}{1.226643in}}{\pgfqpoint{1.793825in}{1.231943in}}{\pgfqpoint{1.793825in}{1.237468in}}%
\pgfpathcurveto{\pgfqpoint{1.793825in}{1.242993in}}{\pgfqpoint{1.791630in}{1.248292in}}{\pgfqpoint{1.787723in}{1.252199in}}%
\pgfpathcurveto{\pgfqpoint{1.783816in}{1.256106in}}{\pgfqpoint{1.778517in}{1.258301in}}{\pgfqpoint{1.772992in}{1.258301in}}%
\pgfpathcurveto{\pgfqpoint{1.767467in}{1.258301in}}{\pgfqpoint{1.762167in}{1.256106in}}{\pgfqpoint{1.758260in}{1.252199in}}%
\pgfpathcurveto{\pgfqpoint{1.754353in}{1.248292in}}{\pgfqpoint{1.752158in}{1.242993in}}{\pgfqpoint{1.752158in}{1.237468in}}%
\pgfpathcurveto{\pgfqpoint{1.752158in}{1.231943in}}{\pgfqpoint{1.754353in}{1.226643in}}{\pgfqpoint{1.758260in}{1.222736in}}%
\pgfpathcurveto{\pgfqpoint{1.762167in}{1.218830in}}{\pgfqpoint{1.767467in}{1.216635in}}{\pgfqpoint{1.772992in}{1.216635in}}%
\pgfpathclose%
\pgfusepath{stroke,fill}%
\end{pgfscope}%
\begin{pgfscope}%
\pgfpathrectangle{\pgfqpoint{0.562500in}{0.275000in}}{\pgfqpoint{3.487500in}{1.925000in}}%
\pgfusepath{clip}%
\pgfsetbuttcap%
\pgfsetroundjoin%
\definecolor{currentfill}{rgb}{0.000000,0.000000,0.000000}%
\pgfsetfillcolor{currentfill}%
\pgfsetlinewidth{1.003750pt}%
\definecolor{currentstroke}{rgb}{0.000000,0.000000,0.000000}%
\pgfsetstrokecolor{currentstroke}%
\pgfsetdash{}{0pt}%
\pgfpathmoveto{\pgfqpoint{1.772992in}{1.216635in}}%
\pgfpathcurveto{\pgfqpoint{1.778517in}{1.216635in}}{\pgfqpoint{1.783816in}{1.218830in}}{\pgfqpoint{1.787723in}{1.222736in}}%
\pgfpathcurveto{\pgfqpoint{1.791630in}{1.226643in}}{\pgfqpoint{1.793825in}{1.231943in}}{\pgfqpoint{1.793825in}{1.237468in}}%
\pgfpathcurveto{\pgfqpoint{1.793825in}{1.242993in}}{\pgfqpoint{1.791630in}{1.248292in}}{\pgfqpoint{1.787723in}{1.252199in}}%
\pgfpathcurveto{\pgfqpoint{1.783816in}{1.256106in}}{\pgfqpoint{1.778517in}{1.258301in}}{\pgfqpoint{1.772992in}{1.258301in}}%
\pgfpathcurveto{\pgfqpoint{1.767467in}{1.258301in}}{\pgfqpoint{1.762167in}{1.256106in}}{\pgfqpoint{1.758260in}{1.252199in}}%
\pgfpathcurveto{\pgfqpoint{1.754353in}{1.248292in}}{\pgfqpoint{1.752158in}{1.242993in}}{\pgfqpoint{1.752158in}{1.237468in}}%
\pgfpathcurveto{\pgfqpoint{1.752158in}{1.231943in}}{\pgfqpoint{1.754353in}{1.226643in}}{\pgfqpoint{1.758260in}{1.222736in}}%
\pgfpathcurveto{\pgfqpoint{1.762167in}{1.218830in}}{\pgfqpoint{1.767467in}{1.216635in}}{\pgfqpoint{1.772992in}{1.216635in}}%
\pgfpathclose%
\pgfusepath{stroke,fill}%
\end{pgfscope}%
\begin{pgfscope}%
\pgfpathrectangle{\pgfqpoint{0.562500in}{0.275000in}}{\pgfqpoint{3.487500in}{1.925000in}}%
\pgfusepath{clip}%
\pgfsetbuttcap%
\pgfsetroundjoin%
\definecolor{currentfill}{rgb}{0.000000,0.000000,0.000000}%
\pgfsetfillcolor{currentfill}%
\pgfsetlinewidth{1.003750pt}%
\definecolor{currentstroke}{rgb}{0.000000,0.000000,0.000000}%
\pgfsetstrokecolor{currentstroke}%
\pgfsetdash{}{0pt}%
\pgfpathmoveto{\pgfqpoint{1.772992in}{1.216635in}}%
\pgfpathcurveto{\pgfqpoint{1.778517in}{1.216635in}}{\pgfqpoint{1.783816in}{1.218830in}}{\pgfqpoint{1.787723in}{1.222736in}}%
\pgfpathcurveto{\pgfqpoint{1.791630in}{1.226643in}}{\pgfqpoint{1.793825in}{1.231943in}}{\pgfqpoint{1.793825in}{1.237468in}}%
\pgfpathcurveto{\pgfqpoint{1.793825in}{1.242993in}}{\pgfqpoint{1.791630in}{1.248292in}}{\pgfqpoint{1.787723in}{1.252199in}}%
\pgfpathcurveto{\pgfqpoint{1.783816in}{1.256106in}}{\pgfqpoint{1.778517in}{1.258301in}}{\pgfqpoint{1.772992in}{1.258301in}}%
\pgfpathcurveto{\pgfqpoint{1.767467in}{1.258301in}}{\pgfqpoint{1.762167in}{1.256106in}}{\pgfqpoint{1.758260in}{1.252199in}}%
\pgfpathcurveto{\pgfqpoint{1.754353in}{1.248292in}}{\pgfqpoint{1.752158in}{1.242993in}}{\pgfqpoint{1.752158in}{1.237468in}}%
\pgfpathcurveto{\pgfqpoint{1.752158in}{1.231943in}}{\pgfqpoint{1.754353in}{1.226643in}}{\pgfqpoint{1.758260in}{1.222736in}}%
\pgfpathcurveto{\pgfqpoint{1.762167in}{1.218830in}}{\pgfqpoint{1.767467in}{1.216635in}}{\pgfqpoint{1.772992in}{1.216635in}}%
\pgfpathclose%
\pgfusepath{stroke,fill}%
\end{pgfscope}%
\begin{pgfscope}%
\pgfpathrectangle{\pgfqpoint{0.562500in}{0.275000in}}{\pgfqpoint{3.487500in}{1.925000in}}%
\pgfusepath{clip}%
\pgfsetbuttcap%
\pgfsetroundjoin%
\definecolor{currentfill}{rgb}{0.000000,0.000000,0.000000}%
\pgfsetfillcolor{currentfill}%
\pgfsetlinewidth{1.003750pt}%
\definecolor{currentstroke}{rgb}{0.000000,0.000000,0.000000}%
\pgfsetstrokecolor{currentstroke}%
\pgfsetdash{}{0pt}%
\pgfpathmoveto{\pgfqpoint{1.772992in}{1.216635in}}%
\pgfpathcurveto{\pgfqpoint{1.778517in}{1.216635in}}{\pgfqpoint{1.783816in}{1.218830in}}{\pgfqpoint{1.787723in}{1.222736in}}%
\pgfpathcurveto{\pgfqpoint{1.791630in}{1.226643in}}{\pgfqpoint{1.793825in}{1.231943in}}{\pgfqpoint{1.793825in}{1.237468in}}%
\pgfpathcurveto{\pgfqpoint{1.793825in}{1.242993in}}{\pgfqpoint{1.791630in}{1.248292in}}{\pgfqpoint{1.787723in}{1.252199in}}%
\pgfpathcurveto{\pgfqpoint{1.783816in}{1.256106in}}{\pgfqpoint{1.778517in}{1.258301in}}{\pgfqpoint{1.772992in}{1.258301in}}%
\pgfpathcurveto{\pgfqpoint{1.767467in}{1.258301in}}{\pgfqpoint{1.762167in}{1.256106in}}{\pgfqpoint{1.758260in}{1.252199in}}%
\pgfpathcurveto{\pgfqpoint{1.754353in}{1.248292in}}{\pgfqpoint{1.752158in}{1.242993in}}{\pgfqpoint{1.752158in}{1.237468in}}%
\pgfpathcurveto{\pgfqpoint{1.752158in}{1.231943in}}{\pgfqpoint{1.754353in}{1.226643in}}{\pgfqpoint{1.758260in}{1.222736in}}%
\pgfpathcurveto{\pgfqpoint{1.762167in}{1.218830in}}{\pgfqpoint{1.767467in}{1.216635in}}{\pgfqpoint{1.772992in}{1.216635in}}%
\pgfpathclose%
\pgfusepath{stroke,fill}%
\end{pgfscope}%
\begin{pgfscope}%
\pgfpathrectangle{\pgfqpoint{0.562500in}{0.275000in}}{\pgfqpoint{3.487500in}{1.925000in}}%
\pgfusepath{clip}%
\pgfsetbuttcap%
\pgfsetroundjoin%
\definecolor{currentfill}{rgb}{0.000000,0.000000,0.000000}%
\pgfsetfillcolor{currentfill}%
\pgfsetlinewidth{1.003750pt}%
\definecolor{currentstroke}{rgb}{0.000000,0.000000,0.000000}%
\pgfsetstrokecolor{currentstroke}%
\pgfsetdash{}{0pt}%
\pgfpathmoveto{\pgfqpoint{1.772992in}{1.216635in}}%
\pgfpathcurveto{\pgfqpoint{1.778517in}{1.216635in}}{\pgfqpoint{1.783816in}{1.218830in}}{\pgfqpoint{1.787723in}{1.222736in}}%
\pgfpathcurveto{\pgfqpoint{1.791630in}{1.226643in}}{\pgfqpoint{1.793825in}{1.231943in}}{\pgfqpoint{1.793825in}{1.237468in}}%
\pgfpathcurveto{\pgfqpoint{1.793825in}{1.242993in}}{\pgfqpoint{1.791630in}{1.248292in}}{\pgfqpoint{1.787723in}{1.252199in}}%
\pgfpathcurveto{\pgfqpoint{1.783816in}{1.256106in}}{\pgfqpoint{1.778517in}{1.258301in}}{\pgfqpoint{1.772992in}{1.258301in}}%
\pgfpathcurveto{\pgfqpoint{1.767467in}{1.258301in}}{\pgfqpoint{1.762167in}{1.256106in}}{\pgfqpoint{1.758260in}{1.252199in}}%
\pgfpathcurveto{\pgfqpoint{1.754353in}{1.248292in}}{\pgfqpoint{1.752158in}{1.242993in}}{\pgfqpoint{1.752158in}{1.237468in}}%
\pgfpathcurveto{\pgfqpoint{1.752158in}{1.231943in}}{\pgfqpoint{1.754353in}{1.226643in}}{\pgfqpoint{1.758260in}{1.222736in}}%
\pgfpathcurveto{\pgfqpoint{1.762167in}{1.218830in}}{\pgfqpoint{1.767467in}{1.216635in}}{\pgfqpoint{1.772992in}{1.216635in}}%
\pgfpathclose%
\pgfusepath{stroke,fill}%
\end{pgfscope}%
\begin{pgfscope}%
\pgfpathrectangle{\pgfqpoint{0.562500in}{0.275000in}}{\pgfqpoint{3.487500in}{1.925000in}}%
\pgfusepath{clip}%
\pgfsetbuttcap%
\pgfsetroundjoin%
\definecolor{currentfill}{rgb}{0.000000,0.000000,0.000000}%
\pgfsetfillcolor{currentfill}%
\pgfsetlinewidth{1.003750pt}%
\definecolor{currentstroke}{rgb}{0.000000,0.000000,0.000000}%
\pgfsetstrokecolor{currentstroke}%
\pgfsetdash{}{0pt}%
\pgfpathmoveto{\pgfqpoint{1.772992in}{1.216635in}}%
\pgfpathcurveto{\pgfqpoint{1.778517in}{1.216635in}}{\pgfqpoint{1.783816in}{1.218830in}}{\pgfqpoint{1.787723in}{1.222736in}}%
\pgfpathcurveto{\pgfqpoint{1.791630in}{1.226643in}}{\pgfqpoint{1.793825in}{1.231943in}}{\pgfqpoint{1.793825in}{1.237468in}}%
\pgfpathcurveto{\pgfqpoint{1.793825in}{1.242993in}}{\pgfqpoint{1.791630in}{1.248292in}}{\pgfqpoint{1.787723in}{1.252199in}}%
\pgfpathcurveto{\pgfqpoint{1.783816in}{1.256106in}}{\pgfqpoint{1.778517in}{1.258301in}}{\pgfqpoint{1.772992in}{1.258301in}}%
\pgfpathcurveto{\pgfqpoint{1.767467in}{1.258301in}}{\pgfqpoint{1.762167in}{1.256106in}}{\pgfqpoint{1.758260in}{1.252199in}}%
\pgfpathcurveto{\pgfqpoint{1.754353in}{1.248292in}}{\pgfqpoint{1.752158in}{1.242993in}}{\pgfqpoint{1.752158in}{1.237468in}}%
\pgfpathcurveto{\pgfqpoint{1.752158in}{1.231943in}}{\pgfqpoint{1.754353in}{1.226643in}}{\pgfqpoint{1.758260in}{1.222736in}}%
\pgfpathcurveto{\pgfqpoint{1.762167in}{1.218830in}}{\pgfqpoint{1.767467in}{1.216635in}}{\pgfqpoint{1.772992in}{1.216635in}}%
\pgfpathclose%
\pgfusepath{stroke,fill}%
\end{pgfscope}%
\begin{pgfscope}%
\pgfpathrectangle{\pgfqpoint{0.562500in}{0.275000in}}{\pgfqpoint{3.487500in}{1.925000in}}%
\pgfusepath{clip}%
\pgfsetbuttcap%
\pgfsetroundjoin%
\definecolor{currentfill}{rgb}{0.000000,0.000000,0.000000}%
\pgfsetfillcolor{currentfill}%
\pgfsetlinewidth{1.003750pt}%
\definecolor{currentstroke}{rgb}{0.000000,0.000000,0.000000}%
\pgfsetstrokecolor{currentstroke}%
\pgfsetdash{}{0pt}%
\pgfpathmoveto{\pgfqpoint{1.772992in}{1.216635in}}%
\pgfpathcurveto{\pgfqpoint{1.778517in}{1.216635in}}{\pgfqpoint{1.783816in}{1.218830in}}{\pgfqpoint{1.787723in}{1.222736in}}%
\pgfpathcurveto{\pgfqpoint{1.791630in}{1.226643in}}{\pgfqpoint{1.793825in}{1.231943in}}{\pgfqpoint{1.793825in}{1.237468in}}%
\pgfpathcurveto{\pgfqpoint{1.793825in}{1.242993in}}{\pgfqpoint{1.791630in}{1.248292in}}{\pgfqpoint{1.787723in}{1.252199in}}%
\pgfpathcurveto{\pgfqpoint{1.783816in}{1.256106in}}{\pgfqpoint{1.778517in}{1.258301in}}{\pgfqpoint{1.772992in}{1.258301in}}%
\pgfpathcurveto{\pgfqpoint{1.767467in}{1.258301in}}{\pgfqpoint{1.762167in}{1.256106in}}{\pgfqpoint{1.758260in}{1.252199in}}%
\pgfpathcurveto{\pgfqpoint{1.754353in}{1.248292in}}{\pgfqpoint{1.752158in}{1.242993in}}{\pgfqpoint{1.752158in}{1.237468in}}%
\pgfpathcurveto{\pgfqpoint{1.752158in}{1.231943in}}{\pgfqpoint{1.754353in}{1.226643in}}{\pgfqpoint{1.758260in}{1.222736in}}%
\pgfpathcurveto{\pgfqpoint{1.762167in}{1.218830in}}{\pgfqpoint{1.767467in}{1.216635in}}{\pgfqpoint{1.772992in}{1.216635in}}%
\pgfpathclose%
\pgfusepath{stroke,fill}%
\end{pgfscope}%
\begin{pgfscope}%
\pgfpathrectangle{\pgfqpoint{0.562500in}{0.275000in}}{\pgfqpoint{3.487500in}{1.925000in}}%
\pgfusepath{clip}%
\pgfsetbuttcap%
\pgfsetroundjoin%
\definecolor{currentfill}{rgb}{0.000000,0.000000,0.000000}%
\pgfsetfillcolor{currentfill}%
\pgfsetlinewidth{1.003750pt}%
\definecolor{currentstroke}{rgb}{0.000000,0.000000,0.000000}%
\pgfsetstrokecolor{currentstroke}%
\pgfsetdash{}{0pt}%
\pgfpathmoveto{\pgfqpoint{1.772992in}{1.216635in}}%
\pgfpathcurveto{\pgfqpoint{1.778517in}{1.216635in}}{\pgfqpoint{1.783816in}{1.218830in}}{\pgfqpoint{1.787723in}{1.222736in}}%
\pgfpathcurveto{\pgfqpoint{1.791630in}{1.226643in}}{\pgfqpoint{1.793825in}{1.231943in}}{\pgfqpoint{1.793825in}{1.237468in}}%
\pgfpathcurveto{\pgfqpoint{1.793825in}{1.242993in}}{\pgfqpoint{1.791630in}{1.248292in}}{\pgfqpoint{1.787723in}{1.252199in}}%
\pgfpathcurveto{\pgfqpoint{1.783816in}{1.256106in}}{\pgfqpoint{1.778517in}{1.258301in}}{\pgfqpoint{1.772992in}{1.258301in}}%
\pgfpathcurveto{\pgfqpoint{1.767467in}{1.258301in}}{\pgfqpoint{1.762167in}{1.256106in}}{\pgfqpoint{1.758260in}{1.252199in}}%
\pgfpathcurveto{\pgfqpoint{1.754353in}{1.248292in}}{\pgfqpoint{1.752158in}{1.242993in}}{\pgfqpoint{1.752158in}{1.237468in}}%
\pgfpathcurveto{\pgfqpoint{1.752158in}{1.231943in}}{\pgfqpoint{1.754353in}{1.226643in}}{\pgfqpoint{1.758260in}{1.222736in}}%
\pgfpathcurveto{\pgfqpoint{1.762167in}{1.218830in}}{\pgfqpoint{1.767467in}{1.216635in}}{\pgfqpoint{1.772992in}{1.216635in}}%
\pgfpathclose%
\pgfusepath{stroke,fill}%
\end{pgfscope}%
\begin{pgfscope}%
\pgfpathrectangle{\pgfqpoint{0.562500in}{0.275000in}}{\pgfqpoint{3.487500in}{1.925000in}}%
\pgfusepath{clip}%
\pgfsetbuttcap%
\pgfsetroundjoin%
\definecolor{currentfill}{rgb}{0.000000,0.000000,0.000000}%
\pgfsetfillcolor{currentfill}%
\pgfsetlinewidth{1.003750pt}%
\definecolor{currentstroke}{rgb}{0.000000,0.000000,0.000000}%
\pgfsetstrokecolor{currentstroke}%
\pgfsetdash{}{0pt}%
\pgfpathmoveto{\pgfqpoint{1.772992in}{1.216635in}}%
\pgfpathcurveto{\pgfqpoint{1.778517in}{1.216635in}}{\pgfqpoint{1.783816in}{1.218830in}}{\pgfqpoint{1.787723in}{1.222736in}}%
\pgfpathcurveto{\pgfqpoint{1.791630in}{1.226643in}}{\pgfqpoint{1.793825in}{1.231943in}}{\pgfqpoint{1.793825in}{1.237468in}}%
\pgfpathcurveto{\pgfqpoint{1.793825in}{1.242993in}}{\pgfqpoint{1.791630in}{1.248292in}}{\pgfqpoint{1.787723in}{1.252199in}}%
\pgfpathcurveto{\pgfqpoint{1.783816in}{1.256106in}}{\pgfqpoint{1.778517in}{1.258301in}}{\pgfqpoint{1.772992in}{1.258301in}}%
\pgfpathcurveto{\pgfqpoint{1.767467in}{1.258301in}}{\pgfqpoint{1.762167in}{1.256106in}}{\pgfqpoint{1.758260in}{1.252199in}}%
\pgfpathcurveto{\pgfqpoint{1.754353in}{1.248292in}}{\pgfqpoint{1.752158in}{1.242993in}}{\pgfqpoint{1.752158in}{1.237468in}}%
\pgfpathcurveto{\pgfqpoint{1.752158in}{1.231943in}}{\pgfqpoint{1.754353in}{1.226643in}}{\pgfqpoint{1.758260in}{1.222736in}}%
\pgfpathcurveto{\pgfqpoint{1.762167in}{1.218830in}}{\pgfqpoint{1.767467in}{1.216635in}}{\pgfqpoint{1.772992in}{1.216635in}}%
\pgfpathclose%
\pgfusepath{stroke,fill}%
\end{pgfscope}%
\begin{pgfscope}%
\pgfpathrectangle{\pgfqpoint{0.562500in}{0.275000in}}{\pgfqpoint{3.487500in}{1.925000in}}%
\pgfusepath{clip}%
\pgfsetbuttcap%
\pgfsetroundjoin%
\definecolor{currentfill}{rgb}{0.000000,0.000000,0.000000}%
\pgfsetfillcolor{currentfill}%
\pgfsetlinewidth{1.003750pt}%
\definecolor{currentstroke}{rgb}{0.000000,0.000000,0.000000}%
\pgfsetstrokecolor{currentstroke}%
\pgfsetdash{}{0pt}%
\pgfpathmoveto{\pgfqpoint{1.772992in}{1.216635in}}%
\pgfpathcurveto{\pgfqpoint{1.778517in}{1.216635in}}{\pgfqpoint{1.783816in}{1.218830in}}{\pgfqpoint{1.787723in}{1.222736in}}%
\pgfpathcurveto{\pgfqpoint{1.791630in}{1.226643in}}{\pgfqpoint{1.793825in}{1.231943in}}{\pgfqpoint{1.793825in}{1.237468in}}%
\pgfpathcurveto{\pgfqpoint{1.793825in}{1.242993in}}{\pgfqpoint{1.791630in}{1.248292in}}{\pgfqpoint{1.787723in}{1.252199in}}%
\pgfpathcurveto{\pgfqpoint{1.783816in}{1.256106in}}{\pgfqpoint{1.778517in}{1.258301in}}{\pgfqpoint{1.772992in}{1.258301in}}%
\pgfpathcurveto{\pgfqpoint{1.767467in}{1.258301in}}{\pgfqpoint{1.762167in}{1.256106in}}{\pgfqpoint{1.758260in}{1.252199in}}%
\pgfpathcurveto{\pgfqpoint{1.754353in}{1.248292in}}{\pgfqpoint{1.752158in}{1.242993in}}{\pgfqpoint{1.752158in}{1.237468in}}%
\pgfpathcurveto{\pgfqpoint{1.752158in}{1.231943in}}{\pgfqpoint{1.754353in}{1.226643in}}{\pgfqpoint{1.758260in}{1.222736in}}%
\pgfpathcurveto{\pgfqpoint{1.762167in}{1.218830in}}{\pgfqpoint{1.767467in}{1.216635in}}{\pgfqpoint{1.772992in}{1.216635in}}%
\pgfpathclose%
\pgfusepath{stroke,fill}%
\end{pgfscope}%
\begin{pgfscope}%
\pgfpathrectangle{\pgfqpoint{0.562500in}{0.275000in}}{\pgfqpoint{3.487500in}{1.925000in}}%
\pgfusepath{clip}%
\pgfsetbuttcap%
\pgfsetroundjoin%
\definecolor{currentfill}{rgb}{0.000000,0.000000,0.000000}%
\pgfsetfillcolor{currentfill}%
\pgfsetlinewidth{1.003750pt}%
\definecolor{currentstroke}{rgb}{0.000000,0.000000,0.000000}%
\pgfsetstrokecolor{currentstroke}%
\pgfsetdash{}{0pt}%
\pgfpathmoveto{\pgfqpoint{1.772992in}{1.216635in}}%
\pgfpathcurveto{\pgfqpoint{1.778517in}{1.216635in}}{\pgfqpoint{1.783816in}{1.218830in}}{\pgfqpoint{1.787723in}{1.222736in}}%
\pgfpathcurveto{\pgfqpoint{1.791630in}{1.226643in}}{\pgfqpoint{1.793825in}{1.231943in}}{\pgfqpoint{1.793825in}{1.237468in}}%
\pgfpathcurveto{\pgfqpoint{1.793825in}{1.242993in}}{\pgfqpoint{1.791630in}{1.248292in}}{\pgfqpoint{1.787723in}{1.252199in}}%
\pgfpathcurveto{\pgfqpoint{1.783816in}{1.256106in}}{\pgfqpoint{1.778517in}{1.258301in}}{\pgfqpoint{1.772992in}{1.258301in}}%
\pgfpathcurveto{\pgfqpoint{1.767467in}{1.258301in}}{\pgfqpoint{1.762167in}{1.256106in}}{\pgfqpoint{1.758260in}{1.252199in}}%
\pgfpathcurveto{\pgfqpoint{1.754353in}{1.248292in}}{\pgfqpoint{1.752158in}{1.242993in}}{\pgfqpoint{1.752158in}{1.237468in}}%
\pgfpathcurveto{\pgfqpoint{1.752158in}{1.231943in}}{\pgfqpoint{1.754353in}{1.226643in}}{\pgfqpoint{1.758260in}{1.222736in}}%
\pgfpathcurveto{\pgfqpoint{1.762167in}{1.218830in}}{\pgfqpoint{1.767467in}{1.216635in}}{\pgfqpoint{1.772992in}{1.216635in}}%
\pgfpathclose%
\pgfusepath{stroke,fill}%
\end{pgfscope}%
\begin{pgfscope}%
\pgfpathrectangle{\pgfqpoint{0.562500in}{0.275000in}}{\pgfqpoint{3.487500in}{1.925000in}}%
\pgfusepath{clip}%
\pgfsetbuttcap%
\pgfsetroundjoin%
\definecolor{currentfill}{rgb}{0.000000,0.000000,0.000000}%
\pgfsetfillcolor{currentfill}%
\pgfsetlinewidth{1.003750pt}%
\definecolor{currentstroke}{rgb}{0.000000,0.000000,0.000000}%
\pgfsetstrokecolor{currentstroke}%
\pgfsetdash{}{0pt}%
\pgfpathmoveto{\pgfqpoint{1.772992in}{1.216635in}}%
\pgfpathcurveto{\pgfqpoint{1.778517in}{1.216635in}}{\pgfqpoint{1.783816in}{1.218830in}}{\pgfqpoint{1.787723in}{1.222736in}}%
\pgfpathcurveto{\pgfqpoint{1.791630in}{1.226643in}}{\pgfqpoint{1.793825in}{1.231943in}}{\pgfqpoint{1.793825in}{1.237468in}}%
\pgfpathcurveto{\pgfqpoint{1.793825in}{1.242993in}}{\pgfqpoint{1.791630in}{1.248292in}}{\pgfqpoint{1.787723in}{1.252199in}}%
\pgfpathcurveto{\pgfqpoint{1.783816in}{1.256106in}}{\pgfqpoint{1.778517in}{1.258301in}}{\pgfqpoint{1.772992in}{1.258301in}}%
\pgfpathcurveto{\pgfqpoint{1.767467in}{1.258301in}}{\pgfqpoint{1.762167in}{1.256106in}}{\pgfqpoint{1.758260in}{1.252199in}}%
\pgfpathcurveto{\pgfqpoint{1.754353in}{1.248292in}}{\pgfqpoint{1.752158in}{1.242993in}}{\pgfqpoint{1.752158in}{1.237468in}}%
\pgfpathcurveto{\pgfqpoint{1.752158in}{1.231943in}}{\pgfqpoint{1.754353in}{1.226643in}}{\pgfqpoint{1.758260in}{1.222736in}}%
\pgfpathcurveto{\pgfqpoint{1.762167in}{1.218830in}}{\pgfqpoint{1.767467in}{1.216635in}}{\pgfqpoint{1.772992in}{1.216635in}}%
\pgfpathclose%
\pgfusepath{stroke,fill}%
\end{pgfscope}%
\begin{pgfscope}%
\pgfpathrectangle{\pgfqpoint{0.562500in}{0.275000in}}{\pgfqpoint{3.487500in}{1.925000in}}%
\pgfusepath{clip}%
\pgfsetbuttcap%
\pgfsetroundjoin%
\definecolor{currentfill}{rgb}{0.000000,0.000000,0.000000}%
\pgfsetfillcolor{currentfill}%
\pgfsetlinewidth{1.003750pt}%
\definecolor{currentstroke}{rgb}{0.000000,0.000000,0.000000}%
\pgfsetstrokecolor{currentstroke}%
\pgfsetdash{}{0pt}%
\pgfpathmoveto{\pgfqpoint{1.772992in}{1.216635in}}%
\pgfpathcurveto{\pgfqpoint{1.778517in}{1.216635in}}{\pgfqpoint{1.783816in}{1.218830in}}{\pgfqpoint{1.787723in}{1.222736in}}%
\pgfpathcurveto{\pgfqpoint{1.791630in}{1.226643in}}{\pgfqpoint{1.793825in}{1.231943in}}{\pgfqpoint{1.793825in}{1.237468in}}%
\pgfpathcurveto{\pgfqpoint{1.793825in}{1.242993in}}{\pgfqpoint{1.791630in}{1.248292in}}{\pgfqpoint{1.787723in}{1.252199in}}%
\pgfpathcurveto{\pgfqpoint{1.783816in}{1.256106in}}{\pgfqpoint{1.778517in}{1.258301in}}{\pgfqpoint{1.772992in}{1.258301in}}%
\pgfpathcurveto{\pgfqpoint{1.767467in}{1.258301in}}{\pgfqpoint{1.762167in}{1.256106in}}{\pgfqpoint{1.758260in}{1.252199in}}%
\pgfpathcurveto{\pgfqpoint{1.754353in}{1.248292in}}{\pgfqpoint{1.752158in}{1.242993in}}{\pgfqpoint{1.752158in}{1.237468in}}%
\pgfpathcurveto{\pgfqpoint{1.752158in}{1.231943in}}{\pgfqpoint{1.754353in}{1.226643in}}{\pgfqpoint{1.758260in}{1.222736in}}%
\pgfpathcurveto{\pgfqpoint{1.762167in}{1.218830in}}{\pgfqpoint{1.767467in}{1.216635in}}{\pgfqpoint{1.772992in}{1.216635in}}%
\pgfpathclose%
\pgfusepath{stroke,fill}%
\end{pgfscope}%
\begin{pgfscope}%
\pgfpathrectangle{\pgfqpoint{0.562500in}{0.275000in}}{\pgfqpoint{3.487500in}{1.925000in}}%
\pgfusepath{clip}%
\pgfsetbuttcap%
\pgfsetroundjoin%
\definecolor{currentfill}{rgb}{0.000000,0.000000,0.000000}%
\pgfsetfillcolor{currentfill}%
\pgfsetlinewidth{1.003750pt}%
\definecolor{currentstroke}{rgb}{0.000000,0.000000,0.000000}%
\pgfsetstrokecolor{currentstroke}%
\pgfsetdash{}{0pt}%
\pgfpathmoveto{\pgfqpoint{2.824734in}{1.216635in}}%
\pgfpathcurveto{\pgfqpoint{2.830260in}{1.216635in}}{\pgfqpoint{2.835559in}{1.218830in}}{\pgfqpoint{2.839466in}{1.222736in}}%
\pgfpathcurveto{\pgfqpoint{2.843373in}{1.226643in}}{\pgfqpoint{2.845568in}{1.231943in}}{\pgfqpoint{2.845568in}{1.237468in}}%
\pgfpathcurveto{\pgfqpoint{2.845568in}{1.242993in}}{\pgfqpoint{2.843373in}{1.248292in}}{\pgfqpoint{2.839466in}{1.252199in}}%
\pgfpathcurveto{\pgfqpoint{2.835559in}{1.256106in}}{\pgfqpoint{2.830260in}{1.258301in}}{\pgfqpoint{2.824734in}{1.258301in}}%
\pgfpathcurveto{\pgfqpoint{2.819209in}{1.258301in}}{\pgfqpoint{2.813910in}{1.256106in}}{\pgfqpoint{2.810003in}{1.252199in}}%
\pgfpathcurveto{\pgfqpoint{2.806096in}{1.248292in}}{\pgfqpoint{2.803901in}{1.242993in}}{\pgfqpoint{2.803901in}{1.237468in}}%
\pgfpathcurveto{\pgfqpoint{2.803901in}{1.231943in}}{\pgfqpoint{2.806096in}{1.226643in}}{\pgfqpoint{2.810003in}{1.222736in}}%
\pgfpathcurveto{\pgfqpoint{2.813910in}{1.218830in}}{\pgfqpoint{2.819209in}{1.216635in}}{\pgfqpoint{2.824734in}{1.216635in}}%
\pgfpathclose%
\pgfusepath{stroke,fill}%
\end{pgfscope}%
\begin{pgfscope}%
\pgfpathrectangle{\pgfqpoint{0.562500in}{0.275000in}}{\pgfqpoint{3.487500in}{1.925000in}}%
\pgfusepath{clip}%
\pgfsetbuttcap%
\pgfsetroundjoin%
\definecolor{currentfill}{rgb}{0.000000,0.000000,0.000000}%
\pgfsetfillcolor{currentfill}%
\pgfsetlinewidth{1.003750pt}%
\definecolor{currentstroke}{rgb}{0.000000,0.000000,0.000000}%
\pgfsetstrokecolor{currentstroke}%
\pgfsetdash{}{0pt}%
\pgfpathmoveto{\pgfqpoint{2.824734in}{1.216635in}}%
\pgfpathcurveto{\pgfqpoint{2.830260in}{1.216635in}}{\pgfqpoint{2.835559in}{1.218830in}}{\pgfqpoint{2.839466in}{1.222736in}}%
\pgfpathcurveto{\pgfqpoint{2.843373in}{1.226643in}}{\pgfqpoint{2.845568in}{1.231943in}}{\pgfqpoint{2.845568in}{1.237468in}}%
\pgfpathcurveto{\pgfqpoint{2.845568in}{1.242993in}}{\pgfqpoint{2.843373in}{1.248292in}}{\pgfqpoint{2.839466in}{1.252199in}}%
\pgfpathcurveto{\pgfqpoint{2.835559in}{1.256106in}}{\pgfqpoint{2.830260in}{1.258301in}}{\pgfqpoint{2.824734in}{1.258301in}}%
\pgfpathcurveto{\pgfqpoint{2.819209in}{1.258301in}}{\pgfqpoint{2.813910in}{1.256106in}}{\pgfqpoint{2.810003in}{1.252199in}}%
\pgfpathcurveto{\pgfqpoint{2.806096in}{1.248292in}}{\pgfqpoint{2.803901in}{1.242993in}}{\pgfqpoint{2.803901in}{1.237468in}}%
\pgfpathcurveto{\pgfqpoint{2.803901in}{1.231943in}}{\pgfqpoint{2.806096in}{1.226643in}}{\pgfqpoint{2.810003in}{1.222736in}}%
\pgfpathcurveto{\pgfqpoint{2.813910in}{1.218830in}}{\pgfqpoint{2.819209in}{1.216635in}}{\pgfqpoint{2.824734in}{1.216635in}}%
\pgfpathclose%
\pgfusepath{stroke,fill}%
\end{pgfscope}%
\begin{pgfscope}%
\pgfpathrectangle{\pgfqpoint{0.562500in}{0.275000in}}{\pgfqpoint{3.487500in}{1.925000in}}%
\pgfusepath{clip}%
\pgfsetbuttcap%
\pgfsetroundjoin%
\definecolor{currentfill}{rgb}{0.000000,0.000000,0.000000}%
\pgfsetfillcolor{currentfill}%
\pgfsetlinewidth{1.003750pt}%
\definecolor{currentstroke}{rgb}{0.000000,0.000000,0.000000}%
\pgfsetstrokecolor{currentstroke}%
\pgfsetdash{}{0pt}%
\pgfpathmoveto{\pgfqpoint{2.824734in}{1.216635in}}%
\pgfpathcurveto{\pgfqpoint{2.830260in}{1.216635in}}{\pgfqpoint{2.835559in}{1.218830in}}{\pgfqpoint{2.839466in}{1.222736in}}%
\pgfpathcurveto{\pgfqpoint{2.843373in}{1.226643in}}{\pgfqpoint{2.845568in}{1.231943in}}{\pgfqpoint{2.845568in}{1.237468in}}%
\pgfpathcurveto{\pgfqpoint{2.845568in}{1.242993in}}{\pgfqpoint{2.843373in}{1.248292in}}{\pgfqpoint{2.839466in}{1.252199in}}%
\pgfpathcurveto{\pgfqpoint{2.835559in}{1.256106in}}{\pgfqpoint{2.830260in}{1.258301in}}{\pgfqpoint{2.824734in}{1.258301in}}%
\pgfpathcurveto{\pgfqpoint{2.819209in}{1.258301in}}{\pgfqpoint{2.813910in}{1.256106in}}{\pgfqpoint{2.810003in}{1.252199in}}%
\pgfpathcurveto{\pgfqpoint{2.806096in}{1.248292in}}{\pgfqpoint{2.803901in}{1.242993in}}{\pgfqpoint{2.803901in}{1.237468in}}%
\pgfpathcurveto{\pgfqpoint{2.803901in}{1.231943in}}{\pgfqpoint{2.806096in}{1.226643in}}{\pgfqpoint{2.810003in}{1.222736in}}%
\pgfpathcurveto{\pgfqpoint{2.813910in}{1.218830in}}{\pgfqpoint{2.819209in}{1.216635in}}{\pgfqpoint{2.824734in}{1.216635in}}%
\pgfpathclose%
\pgfusepath{stroke,fill}%
\end{pgfscope}%
\begin{pgfscope}%
\pgfpathrectangle{\pgfqpoint{0.562500in}{0.275000in}}{\pgfqpoint{3.487500in}{1.925000in}}%
\pgfusepath{clip}%
\pgfsetbuttcap%
\pgfsetroundjoin%
\definecolor{currentfill}{rgb}{0.000000,0.000000,0.000000}%
\pgfsetfillcolor{currentfill}%
\pgfsetlinewidth{1.003750pt}%
\definecolor{currentstroke}{rgb}{0.000000,0.000000,0.000000}%
\pgfsetstrokecolor{currentstroke}%
\pgfsetdash{}{0pt}%
\pgfpathmoveto{\pgfqpoint{2.824734in}{1.216635in}}%
\pgfpathcurveto{\pgfqpoint{2.830260in}{1.216635in}}{\pgfqpoint{2.835559in}{1.218830in}}{\pgfqpoint{2.839466in}{1.222736in}}%
\pgfpathcurveto{\pgfqpoint{2.843373in}{1.226643in}}{\pgfqpoint{2.845568in}{1.231943in}}{\pgfqpoint{2.845568in}{1.237468in}}%
\pgfpathcurveto{\pgfqpoint{2.845568in}{1.242993in}}{\pgfqpoint{2.843373in}{1.248292in}}{\pgfqpoint{2.839466in}{1.252199in}}%
\pgfpathcurveto{\pgfqpoint{2.835559in}{1.256106in}}{\pgfqpoint{2.830260in}{1.258301in}}{\pgfqpoint{2.824734in}{1.258301in}}%
\pgfpathcurveto{\pgfqpoint{2.819209in}{1.258301in}}{\pgfqpoint{2.813910in}{1.256106in}}{\pgfqpoint{2.810003in}{1.252199in}}%
\pgfpathcurveto{\pgfqpoint{2.806096in}{1.248292in}}{\pgfqpoint{2.803901in}{1.242993in}}{\pgfqpoint{2.803901in}{1.237468in}}%
\pgfpathcurveto{\pgfqpoint{2.803901in}{1.231943in}}{\pgfqpoint{2.806096in}{1.226643in}}{\pgfqpoint{2.810003in}{1.222736in}}%
\pgfpathcurveto{\pgfqpoint{2.813910in}{1.218830in}}{\pgfqpoint{2.819209in}{1.216635in}}{\pgfqpoint{2.824734in}{1.216635in}}%
\pgfpathclose%
\pgfusepath{stroke,fill}%
\end{pgfscope}%
\begin{pgfscope}%
\pgfpathrectangle{\pgfqpoint{0.562500in}{0.275000in}}{\pgfqpoint{3.487500in}{1.925000in}}%
\pgfusepath{clip}%
\pgfsetbuttcap%
\pgfsetroundjoin%
\definecolor{currentfill}{rgb}{0.000000,0.000000,0.000000}%
\pgfsetfillcolor{currentfill}%
\pgfsetlinewidth{1.003750pt}%
\definecolor{currentstroke}{rgb}{0.000000,0.000000,0.000000}%
\pgfsetstrokecolor{currentstroke}%
\pgfsetdash{}{0pt}%
\pgfpathmoveto{\pgfqpoint{2.824734in}{1.216635in}}%
\pgfpathcurveto{\pgfqpoint{2.830260in}{1.216635in}}{\pgfqpoint{2.835559in}{1.218830in}}{\pgfqpoint{2.839466in}{1.222736in}}%
\pgfpathcurveto{\pgfqpoint{2.843373in}{1.226643in}}{\pgfqpoint{2.845568in}{1.231943in}}{\pgfqpoint{2.845568in}{1.237468in}}%
\pgfpathcurveto{\pgfqpoint{2.845568in}{1.242993in}}{\pgfqpoint{2.843373in}{1.248292in}}{\pgfqpoint{2.839466in}{1.252199in}}%
\pgfpathcurveto{\pgfqpoint{2.835559in}{1.256106in}}{\pgfqpoint{2.830260in}{1.258301in}}{\pgfqpoint{2.824734in}{1.258301in}}%
\pgfpathcurveto{\pgfqpoint{2.819209in}{1.258301in}}{\pgfqpoint{2.813910in}{1.256106in}}{\pgfqpoint{2.810003in}{1.252199in}}%
\pgfpathcurveto{\pgfqpoint{2.806096in}{1.248292in}}{\pgfqpoint{2.803901in}{1.242993in}}{\pgfqpoint{2.803901in}{1.237468in}}%
\pgfpathcurveto{\pgfqpoint{2.803901in}{1.231943in}}{\pgfqpoint{2.806096in}{1.226643in}}{\pgfqpoint{2.810003in}{1.222736in}}%
\pgfpathcurveto{\pgfqpoint{2.813910in}{1.218830in}}{\pgfqpoint{2.819209in}{1.216635in}}{\pgfqpoint{2.824734in}{1.216635in}}%
\pgfpathclose%
\pgfusepath{stroke,fill}%
\end{pgfscope}%
\begin{pgfscope}%
\pgfpathrectangle{\pgfqpoint{0.562500in}{0.275000in}}{\pgfqpoint{3.487500in}{1.925000in}}%
\pgfusepath{clip}%
\pgfsetbuttcap%
\pgfsetroundjoin%
\definecolor{currentfill}{rgb}{0.000000,0.000000,0.000000}%
\pgfsetfillcolor{currentfill}%
\pgfsetlinewidth{1.003750pt}%
\definecolor{currentstroke}{rgb}{0.000000,0.000000,0.000000}%
\pgfsetstrokecolor{currentstroke}%
\pgfsetdash{}{0pt}%
\pgfpathmoveto{\pgfqpoint{2.824734in}{1.216635in}}%
\pgfpathcurveto{\pgfqpoint{2.830260in}{1.216635in}}{\pgfqpoint{2.835559in}{1.218830in}}{\pgfqpoint{2.839466in}{1.222736in}}%
\pgfpathcurveto{\pgfqpoint{2.843373in}{1.226643in}}{\pgfqpoint{2.845568in}{1.231943in}}{\pgfqpoint{2.845568in}{1.237468in}}%
\pgfpathcurveto{\pgfqpoint{2.845568in}{1.242993in}}{\pgfqpoint{2.843373in}{1.248292in}}{\pgfqpoint{2.839466in}{1.252199in}}%
\pgfpathcurveto{\pgfqpoint{2.835559in}{1.256106in}}{\pgfqpoint{2.830260in}{1.258301in}}{\pgfqpoint{2.824734in}{1.258301in}}%
\pgfpathcurveto{\pgfqpoint{2.819209in}{1.258301in}}{\pgfqpoint{2.813910in}{1.256106in}}{\pgfqpoint{2.810003in}{1.252199in}}%
\pgfpathcurveto{\pgfqpoint{2.806096in}{1.248292in}}{\pgfqpoint{2.803901in}{1.242993in}}{\pgfqpoint{2.803901in}{1.237468in}}%
\pgfpathcurveto{\pgfqpoint{2.803901in}{1.231943in}}{\pgfqpoint{2.806096in}{1.226643in}}{\pgfqpoint{2.810003in}{1.222736in}}%
\pgfpathcurveto{\pgfqpoint{2.813910in}{1.218830in}}{\pgfqpoint{2.819209in}{1.216635in}}{\pgfqpoint{2.824734in}{1.216635in}}%
\pgfpathclose%
\pgfusepath{stroke,fill}%
\end{pgfscope}%
\begin{pgfscope}%
\pgfpathrectangle{\pgfqpoint{0.562500in}{0.275000in}}{\pgfqpoint{3.487500in}{1.925000in}}%
\pgfusepath{clip}%
\pgfsetbuttcap%
\pgfsetroundjoin%
\definecolor{currentfill}{rgb}{0.000000,0.000000,0.000000}%
\pgfsetfillcolor{currentfill}%
\pgfsetlinewidth{1.003750pt}%
\definecolor{currentstroke}{rgb}{0.000000,0.000000,0.000000}%
\pgfsetstrokecolor{currentstroke}%
\pgfsetdash{}{0pt}%
\pgfpathmoveto{\pgfqpoint{2.824734in}{1.216635in}}%
\pgfpathcurveto{\pgfqpoint{2.830260in}{1.216635in}}{\pgfqpoint{2.835559in}{1.218830in}}{\pgfqpoint{2.839466in}{1.222736in}}%
\pgfpathcurveto{\pgfqpoint{2.843373in}{1.226643in}}{\pgfqpoint{2.845568in}{1.231943in}}{\pgfqpoint{2.845568in}{1.237468in}}%
\pgfpathcurveto{\pgfqpoint{2.845568in}{1.242993in}}{\pgfqpoint{2.843373in}{1.248292in}}{\pgfqpoint{2.839466in}{1.252199in}}%
\pgfpathcurveto{\pgfqpoint{2.835559in}{1.256106in}}{\pgfqpoint{2.830260in}{1.258301in}}{\pgfqpoint{2.824734in}{1.258301in}}%
\pgfpathcurveto{\pgfqpoint{2.819209in}{1.258301in}}{\pgfqpoint{2.813910in}{1.256106in}}{\pgfqpoint{2.810003in}{1.252199in}}%
\pgfpathcurveto{\pgfqpoint{2.806096in}{1.248292in}}{\pgfqpoint{2.803901in}{1.242993in}}{\pgfqpoint{2.803901in}{1.237468in}}%
\pgfpathcurveto{\pgfqpoint{2.803901in}{1.231943in}}{\pgfqpoint{2.806096in}{1.226643in}}{\pgfqpoint{2.810003in}{1.222736in}}%
\pgfpathcurveto{\pgfqpoint{2.813910in}{1.218830in}}{\pgfqpoint{2.819209in}{1.216635in}}{\pgfqpoint{2.824734in}{1.216635in}}%
\pgfpathclose%
\pgfusepath{stroke,fill}%
\end{pgfscope}%
\begin{pgfscope}%
\pgfpathrectangle{\pgfqpoint{0.562500in}{0.275000in}}{\pgfqpoint{3.487500in}{1.925000in}}%
\pgfusepath{clip}%
\pgfsetbuttcap%
\pgfsetroundjoin%
\definecolor{currentfill}{rgb}{0.000000,0.000000,0.000000}%
\pgfsetfillcolor{currentfill}%
\pgfsetlinewidth{1.003750pt}%
\definecolor{currentstroke}{rgb}{0.000000,0.000000,0.000000}%
\pgfsetstrokecolor{currentstroke}%
\pgfsetdash{}{0pt}%
\pgfpathmoveto{\pgfqpoint{2.824734in}{1.216635in}}%
\pgfpathcurveto{\pgfqpoint{2.830260in}{1.216635in}}{\pgfqpoint{2.835559in}{1.218830in}}{\pgfqpoint{2.839466in}{1.222736in}}%
\pgfpathcurveto{\pgfqpoint{2.843373in}{1.226643in}}{\pgfqpoint{2.845568in}{1.231943in}}{\pgfqpoint{2.845568in}{1.237468in}}%
\pgfpathcurveto{\pgfqpoint{2.845568in}{1.242993in}}{\pgfqpoint{2.843373in}{1.248292in}}{\pgfqpoint{2.839466in}{1.252199in}}%
\pgfpathcurveto{\pgfqpoint{2.835559in}{1.256106in}}{\pgfqpoint{2.830260in}{1.258301in}}{\pgfqpoint{2.824734in}{1.258301in}}%
\pgfpathcurveto{\pgfqpoint{2.819209in}{1.258301in}}{\pgfqpoint{2.813910in}{1.256106in}}{\pgfqpoint{2.810003in}{1.252199in}}%
\pgfpathcurveto{\pgfqpoint{2.806096in}{1.248292in}}{\pgfqpoint{2.803901in}{1.242993in}}{\pgfqpoint{2.803901in}{1.237468in}}%
\pgfpathcurveto{\pgfqpoint{2.803901in}{1.231943in}}{\pgfqpoint{2.806096in}{1.226643in}}{\pgfqpoint{2.810003in}{1.222736in}}%
\pgfpathcurveto{\pgfqpoint{2.813910in}{1.218830in}}{\pgfqpoint{2.819209in}{1.216635in}}{\pgfqpoint{2.824734in}{1.216635in}}%
\pgfpathclose%
\pgfusepath{stroke,fill}%
\end{pgfscope}%
\begin{pgfscope}%
\pgfpathrectangle{\pgfqpoint{0.562500in}{0.275000in}}{\pgfqpoint{3.487500in}{1.925000in}}%
\pgfusepath{clip}%
\pgfsetbuttcap%
\pgfsetroundjoin%
\definecolor{currentfill}{rgb}{0.000000,0.000000,0.000000}%
\pgfsetfillcolor{currentfill}%
\pgfsetlinewidth{1.003750pt}%
\definecolor{currentstroke}{rgb}{0.000000,0.000000,0.000000}%
\pgfsetstrokecolor{currentstroke}%
\pgfsetdash{}{0pt}%
\pgfpathmoveto{\pgfqpoint{2.824734in}{1.216635in}}%
\pgfpathcurveto{\pgfqpoint{2.830260in}{1.216635in}}{\pgfqpoint{2.835559in}{1.218830in}}{\pgfqpoint{2.839466in}{1.222736in}}%
\pgfpathcurveto{\pgfqpoint{2.843373in}{1.226643in}}{\pgfqpoint{2.845568in}{1.231943in}}{\pgfqpoint{2.845568in}{1.237468in}}%
\pgfpathcurveto{\pgfqpoint{2.845568in}{1.242993in}}{\pgfqpoint{2.843373in}{1.248292in}}{\pgfqpoint{2.839466in}{1.252199in}}%
\pgfpathcurveto{\pgfqpoint{2.835559in}{1.256106in}}{\pgfqpoint{2.830260in}{1.258301in}}{\pgfqpoint{2.824734in}{1.258301in}}%
\pgfpathcurveto{\pgfqpoint{2.819209in}{1.258301in}}{\pgfqpoint{2.813910in}{1.256106in}}{\pgfqpoint{2.810003in}{1.252199in}}%
\pgfpathcurveto{\pgfqpoint{2.806096in}{1.248292in}}{\pgfqpoint{2.803901in}{1.242993in}}{\pgfqpoint{2.803901in}{1.237468in}}%
\pgfpathcurveto{\pgfqpoint{2.803901in}{1.231943in}}{\pgfqpoint{2.806096in}{1.226643in}}{\pgfqpoint{2.810003in}{1.222736in}}%
\pgfpathcurveto{\pgfqpoint{2.813910in}{1.218830in}}{\pgfqpoint{2.819209in}{1.216635in}}{\pgfqpoint{2.824734in}{1.216635in}}%
\pgfpathclose%
\pgfusepath{stroke,fill}%
\end{pgfscope}%
\begin{pgfscope}%
\pgfpathrectangle{\pgfqpoint{0.562500in}{0.275000in}}{\pgfqpoint{3.487500in}{1.925000in}}%
\pgfusepath{clip}%
\pgfsetbuttcap%
\pgfsetroundjoin%
\definecolor{currentfill}{rgb}{0.000000,0.000000,0.000000}%
\pgfsetfillcolor{currentfill}%
\pgfsetlinewidth{1.003750pt}%
\definecolor{currentstroke}{rgb}{0.000000,0.000000,0.000000}%
\pgfsetstrokecolor{currentstroke}%
\pgfsetdash{}{0pt}%
\pgfpathmoveto{\pgfqpoint{2.824734in}{1.216635in}}%
\pgfpathcurveto{\pgfqpoint{2.830260in}{1.216635in}}{\pgfqpoint{2.835559in}{1.218830in}}{\pgfqpoint{2.839466in}{1.222736in}}%
\pgfpathcurveto{\pgfqpoint{2.843373in}{1.226643in}}{\pgfqpoint{2.845568in}{1.231943in}}{\pgfqpoint{2.845568in}{1.237468in}}%
\pgfpathcurveto{\pgfqpoint{2.845568in}{1.242993in}}{\pgfqpoint{2.843373in}{1.248292in}}{\pgfqpoint{2.839466in}{1.252199in}}%
\pgfpathcurveto{\pgfqpoint{2.835559in}{1.256106in}}{\pgfqpoint{2.830260in}{1.258301in}}{\pgfqpoint{2.824734in}{1.258301in}}%
\pgfpathcurveto{\pgfqpoint{2.819209in}{1.258301in}}{\pgfqpoint{2.813910in}{1.256106in}}{\pgfqpoint{2.810003in}{1.252199in}}%
\pgfpathcurveto{\pgfqpoint{2.806096in}{1.248292in}}{\pgfqpoint{2.803901in}{1.242993in}}{\pgfqpoint{2.803901in}{1.237468in}}%
\pgfpathcurveto{\pgfqpoint{2.803901in}{1.231943in}}{\pgfqpoint{2.806096in}{1.226643in}}{\pgfqpoint{2.810003in}{1.222736in}}%
\pgfpathcurveto{\pgfqpoint{2.813910in}{1.218830in}}{\pgfqpoint{2.819209in}{1.216635in}}{\pgfqpoint{2.824734in}{1.216635in}}%
\pgfpathclose%
\pgfusepath{stroke,fill}%
\end{pgfscope}%
\begin{pgfscope}%
\pgfpathrectangle{\pgfqpoint{0.562500in}{0.275000in}}{\pgfqpoint{3.487500in}{1.925000in}}%
\pgfusepath{clip}%
\pgfsetbuttcap%
\pgfsetroundjoin%
\definecolor{currentfill}{rgb}{0.000000,0.000000,0.000000}%
\pgfsetfillcolor{currentfill}%
\pgfsetlinewidth{1.003750pt}%
\definecolor{currentstroke}{rgb}{0.000000,0.000000,0.000000}%
\pgfsetstrokecolor{currentstroke}%
\pgfsetdash{}{0pt}%
\pgfpathmoveto{\pgfqpoint{2.824734in}{1.216635in}}%
\pgfpathcurveto{\pgfqpoint{2.830260in}{1.216635in}}{\pgfqpoint{2.835559in}{1.218830in}}{\pgfqpoint{2.839466in}{1.222736in}}%
\pgfpathcurveto{\pgfqpoint{2.843373in}{1.226643in}}{\pgfqpoint{2.845568in}{1.231943in}}{\pgfqpoint{2.845568in}{1.237468in}}%
\pgfpathcurveto{\pgfqpoint{2.845568in}{1.242993in}}{\pgfqpoint{2.843373in}{1.248292in}}{\pgfqpoint{2.839466in}{1.252199in}}%
\pgfpathcurveto{\pgfqpoint{2.835559in}{1.256106in}}{\pgfqpoint{2.830260in}{1.258301in}}{\pgfqpoint{2.824734in}{1.258301in}}%
\pgfpathcurveto{\pgfqpoint{2.819209in}{1.258301in}}{\pgfqpoint{2.813910in}{1.256106in}}{\pgfqpoint{2.810003in}{1.252199in}}%
\pgfpathcurveto{\pgfqpoint{2.806096in}{1.248292in}}{\pgfqpoint{2.803901in}{1.242993in}}{\pgfqpoint{2.803901in}{1.237468in}}%
\pgfpathcurveto{\pgfqpoint{2.803901in}{1.231943in}}{\pgfqpoint{2.806096in}{1.226643in}}{\pgfqpoint{2.810003in}{1.222736in}}%
\pgfpathcurveto{\pgfqpoint{2.813910in}{1.218830in}}{\pgfqpoint{2.819209in}{1.216635in}}{\pgfqpoint{2.824734in}{1.216635in}}%
\pgfpathclose%
\pgfusepath{stroke,fill}%
\end{pgfscope}%
\begin{pgfscope}%
\pgfpathrectangle{\pgfqpoint{0.562500in}{0.275000in}}{\pgfqpoint{3.487500in}{1.925000in}}%
\pgfusepath{clip}%
\pgfsetbuttcap%
\pgfsetroundjoin%
\definecolor{currentfill}{rgb}{0.000000,0.000000,0.000000}%
\pgfsetfillcolor{currentfill}%
\pgfsetlinewidth{1.003750pt}%
\definecolor{currentstroke}{rgb}{0.000000,0.000000,0.000000}%
\pgfsetstrokecolor{currentstroke}%
\pgfsetdash{}{0pt}%
\pgfpathmoveto{\pgfqpoint{2.824734in}{2.076667in}}%
\pgfpathcurveto{\pgfqpoint{2.830260in}{2.076667in}}{\pgfqpoint{2.835559in}{2.078862in}}{\pgfqpoint{2.839466in}{2.082769in}}%
\pgfpathcurveto{\pgfqpoint{2.843373in}{2.086675in}}{\pgfqpoint{2.845568in}{2.091975in}}{\pgfqpoint{2.845568in}{2.097500in}}%
\pgfpathcurveto{\pgfqpoint{2.845568in}{2.103025in}}{\pgfqpoint{2.843373in}{2.108325in}}{\pgfqpoint{2.839466in}{2.112231in}}%
\pgfpathcurveto{\pgfqpoint{2.835559in}{2.116138in}}{\pgfqpoint{2.830260in}{2.118333in}}{\pgfqpoint{2.824734in}{2.118333in}}%
\pgfpathcurveto{\pgfqpoint{2.819209in}{2.118333in}}{\pgfqpoint{2.813910in}{2.116138in}}{\pgfqpoint{2.810003in}{2.112231in}}%
\pgfpathcurveto{\pgfqpoint{2.806096in}{2.108325in}}{\pgfqpoint{2.803901in}{2.103025in}}{\pgfqpoint{2.803901in}{2.097500in}}%
\pgfpathcurveto{\pgfqpoint{2.803901in}{2.091975in}}{\pgfqpoint{2.806096in}{2.086675in}}{\pgfqpoint{2.810003in}{2.082769in}}%
\pgfpathcurveto{\pgfqpoint{2.813910in}{2.078862in}}{\pgfqpoint{2.819209in}{2.076667in}}{\pgfqpoint{2.824734in}{2.076667in}}%
\pgfpathclose%
\pgfusepath{stroke,fill}%
\end{pgfscope}%
\begin{pgfscope}%
\pgfpathrectangle{\pgfqpoint{0.562500in}{0.275000in}}{\pgfqpoint{3.487500in}{1.925000in}}%
\pgfusepath{clip}%
\pgfsetbuttcap%
\pgfsetroundjoin%
\definecolor{currentfill}{rgb}{0.000000,0.000000,0.000000}%
\pgfsetfillcolor{currentfill}%
\pgfsetlinewidth{1.003750pt}%
\definecolor{currentstroke}{rgb}{0.000000,0.000000,0.000000}%
\pgfsetstrokecolor{currentstroke}%
\pgfsetdash{}{0pt}%
\pgfpathmoveto{\pgfqpoint{2.824734in}{2.076667in}}%
\pgfpathcurveto{\pgfqpoint{2.830260in}{2.076667in}}{\pgfqpoint{2.835559in}{2.078862in}}{\pgfqpoint{2.839466in}{2.082769in}}%
\pgfpathcurveto{\pgfqpoint{2.843373in}{2.086675in}}{\pgfqpoint{2.845568in}{2.091975in}}{\pgfqpoint{2.845568in}{2.097500in}}%
\pgfpathcurveto{\pgfqpoint{2.845568in}{2.103025in}}{\pgfqpoint{2.843373in}{2.108325in}}{\pgfqpoint{2.839466in}{2.112231in}}%
\pgfpathcurveto{\pgfqpoint{2.835559in}{2.116138in}}{\pgfqpoint{2.830260in}{2.118333in}}{\pgfqpoint{2.824734in}{2.118333in}}%
\pgfpathcurveto{\pgfqpoint{2.819209in}{2.118333in}}{\pgfqpoint{2.813910in}{2.116138in}}{\pgfqpoint{2.810003in}{2.112231in}}%
\pgfpathcurveto{\pgfqpoint{2.806096in}{2.108325in}}{\pgfqpoint{2.803901in}{2.103025in}}{\pgfqpoint{2.803901in}{2.097500in}}%
\pgfpathcurveto{\pgfqpoint{2.803901in}{2.091975in}}{\pgfqpoint{2.806096in}{2.086675in}}{\pgfqpoint{2.810003in}{2.082769in}}%
\pgfpathcurveto{\pgfqpoint{2.813910in}{2.078862in}}{\pgfqpoint{2.819209in}{2.076667in}}{\pgfqpoint{2.824734in}{2.076667in}}%
\pgfpathclose%
\pgfusepath{stroke,fill}%
\end{pgfscope}%
\begin{pgfscope}%
\pgfpathrectangle{\pgfqpoint{0.562500in}{0.275000in}}{\pgfqpoint{3.487500in}{1.925000in}}%
\pgfusepath{clip}%
\pgfsetbuttcap%
\pgfsetroundjoin%
\definecolor{currentfill}{rgb}{0.000000,0.000000,0.000000}%
\pgfsetfillcolor{currentfill}%
\pgfsetlinewidth{1.003750pt}%
\definecolor{currentstroke}{rgb}{0.000000,0.000000,0.000000}%
\pgfsetstrokecolor{currentstroke}%
\pgfsetdash{}{0pt}%
\pgfpathmoveto{\pgfqpoint{2.824734in}{1.216635in}}%
\pgfpathcurveto{\pgfqpoint{2.830260in}{1.216635in}}{\pgfqpoint{2.835559in}{1.218830in}}{\pgfqpoint{2.839466in}{1.222736in}}%
\pgfpathcurveto{\pgfqpoint{2.843373in}{1.226643in}}{\pgfqpoint{2.845568in}{1.231943in}}{\pgfqpoint{2.845568in}{1.237468in}}%
\pgfpathcurveto{\pgfqpoint{2.845568in}{1.242993in}}{\pgfqpoint{2.843373in}{1.248292in}}{\pgfqpoint{2.839466in}{1.252199in}}%
\pgfpathcurveto{\pgfqpoint{2.835559in}{1.256106in}}{\pgfqpoint{2.830260in}{1.258301in}}{\pgfqpoint{2.824734in}{1.258301in}}%
\pgfpathcurveto{\pgfqpoint{2.819209in}{1.258301in}}{\pgfqpoint{2.813910in}{1.256106in}}{\pgfqpoint{2.810003in}{1.252199in}}%
\pgfpathcurveto{\pgfqpoint{2.806096in}{1.248292in}}{\pgfqpoint{2.803901in}{1.242993in}}{\pgfqpoint{2.803901in}{1.237468in}}%
\pgfpathcurveto{\pgfqpoint{2.803901in}{1.231943in}}{\pgfqpoint{2.806096in}{1.226643in}}{\pgfqpoint{2.810003in}{1.222736in}}%
\pgfpathcurveto{\pgfqpoint{2.813910in}{1.218830in}}{\pgfqpoint{2.819209in}{1.216635in}}{\pgfqpoint{2.824734in}{1.216635in}}%
\pgfpathclose%
\pgfusepath{stroke,fill}%
\end{pgfscope}%
\begin{pgfscope}%
\pgfpathrectangle{\pgfqpoint{0.562500in}{0.275000in}}{\pgfqpoint{3.487500in}{1.925000in}}%
\pgfusepath{clip}%
\pgfsetbuttcap%
\pgfsetroundjoin%
\definecolor{currentfill}{rgb}{0.000000,0.000000,0.000000}%
\pgfsetfillcolor{currentfill}%
\pgfsetlinewidth{1.003750pt}%
\definecolor{currentstroke}{rgb}{0.000000,0.000000,0.000000}%
\pgfsetstrokecolor{currentstroke}%
\pgfsetdash{}{0pt}%
\pgfpathmoveto{\pgfqpoint{2.824734in}{2.076667in}}%
\pgfpathcurveto{\pgfqpoint{2.830260in}{2.076667in}}{\pgfqpoint{2.835559in}{2.078862in}}{\pgfqpoint{2.839466in}{2.082769in}}%
\pgfpathcurveto{\pgfqpoint{2.843373in}{2.086675in}}{\pgfqpoint{2.845568in}{2.091975in}}{\pgfqpoint{2.845568in}{2.097500in}}%
\pgfpathcurveto{\pgfqpoint{2.845568in}{2.103025in}}{\pgfqpoint{2.843373in}{2.108325in}}{\pgfqpoint{2.839466in}{2.112231in}}%
\pgfpathcurveto{\pgfqpoint{2.835559in}{2.116138in}}{\pgfqpoint{2.830260in}{2.118333in}}{\pgfqpoint{2.824734in}{2.118333in}}%
\pgfpathcurveto{\pgfqpoint{2.819209in}{2.118333in}}{\pgfqpoint{2.813910in}{2.116138in}}{\pgfqpoint{2.810003in}{2.112231in}}%
\pgfpathcurveto{\pgfqpoint{2.806096in}{2.108325in}}{\pgfqpoint{2.803901in}{2.103025in}}{\pgfqpoint{2.803901in}{2.097500in}}%
\pgfpathcurveto{\pgfqpoint{2.803901in}{2.091975in}}{\pgfqpoint{2.806096in}{2.086675in}}{\pgfqpoint{2.810003in}{2.082769in}}%
\pgfpathcurveto{\pgfqpoint{2.813910in}{2.078862in}}{\pgfqpoint{2.819209in}{2.076667in}}{\pgfqpoint{2.824734in}{2.076667in}}%
\pgfpathclose%
\pgfusepath{stroke,fill}%
\end{pgfscope}%
\begin{pgfscope}%
\pgfpathrectangle{\pgfqpoint{0.562500in}{0.275000in}}{\pgfqpoint{3.487500in}{1.925000in}}%
\pgfusepath{clip}%
\pgfsetbuttcap%
\pgfsetroundjoin%
\definecolor{currentfill}{rgb}{0.000000,0.000000,0.000000}%
\pgfsetfillcolor{currentfill}%
\pgfsetlinewidth{1.003750pt}%
\definecolor{currentstroke}{rgb}{0.000000,0.000000,0.000000}%
\pgfsetstrokecolor{currentstroke}%
\pgfsetdash{}{0pt}%
\pgfpathmoveto{\pgfqpoint{2.824734in}{1.216635in}}%
\pgfpathcurveto{\pgfqpoint{2.830260in}{1.216635in}}{\pgfqpoint{2.835559in}{1.218830in}}{\pgfqpoint{2.839466in}{1.222736in}}%
\pgfpathcurveto{\pgfqpoint{2.843373in}{1.226643in}}{\pgfqpoint{2.845568in}{1.231943in}}{\pgfqpoint{2.845568in}{1.237468in}}%
\pgfpathcurveto{\pgfqpoint{2.845568in}{1.242993in}}{\pgfqpoint{2.843373in}{1.248292in}}{\pgfqpoint{2.839466in}{1.252199in}}%
\pgfpathcurveto{\pgfqpoint{2.835559in}{1.256106in}}{\pgfqpoint{2.830260in}{1.258301in}}{\pgfqpoint{2.824734in}{1.258301in}}%
\pgfpathcurveto{\pgfqpoint{2.819209in}{1.258301in}}{\pgfqpoint{2.813910in}{1.256106in}}{\pgfqpoint{2.810003in}{1.252199in}}%
\pgfpathcurveto{\pgfqpoint{2.806096in}{1.248292in}}{\pgfqpoint{2.803901in}{1.242993in}}{\pgfqpoint{2.803901in}{1.237468in}}%
\pgfpathcurveto{\pgfqpoint{2.803901in}{1.231943in}}{\pgfqpoint{2.806096in}{1.226643in}}{\pgfqpoint{2.810003in}{1.222736in}}%
\pgfpathcurveto{\pgfqpoint{2.813910in}{1.218830in}}{\pgfqpoint{2.819209in}{1.216635in}}{\pgfqpoint{2.824734in}{1.216635in}}%
\pgfpathclose%
\pgfusepath{stroke,fill}%
\end{pgfscope}%
\begin{pgfscope}%
\pgfpathrectangle{\pgfqpoint{0.562500in}{0.275000in}}{\pgfqpoint{3.487500in}{1.925000in}}%
\pgfusepath{clip}%
\pgfsetbuttcap%
\pgfsetroundjoin%
\definecolor{currentfill}{rgb}{0.000000,0.000000,0.000000}%
\pgfsetfillcolor{currentfill}%
\pgfsetlinewidth{1.003750pt}%
\definecolor{currentstroke}{rgb}{0.000000,0.000000,0.000000}%
\pgfsetstrokecolor{currentstroke}%
\pgfsetdash{}{0pt}%
\pgfpathmoveto{\pgfqpoint{2.824734in}{1.216635in}}%
\pgfpathcurveto{\pgfqpoint{2.830260in}{1.216635in}}{\pgfqpoint{2.835559in}{1.218830in}}{\pgfqpoint{2.839466in}{1.222736in}}%
\pgfpathcurveto{\pgfqpoint{2.843373in}{1.226643in}}{\pgfqpoint{2.845568in}{1.231943in}}{\pgfqpoint{2.845568in}{1.237468in}}%
\pgfpathcurveto{\pgfqpoint{2.845568in}{1.242993in}}{\pgfqpoint{2.843373in}{1.248292in}}{\pgfqpoint{2.839466in}{1.252199in}}%
\pgfpathcurveto{\pgfqpoint{2.835559in}{1.256106in}}{\pgfqpoint{2.830260in}{1.258301in}}{\pgfqpoint{2.824734in}{1.258301in}}%
\pgfpathcurveto{\pgfqpoint{2.819209in}{1.258301in}}{\pgfqpoint{2.813910in}{1.256106in}}{\pgfqpoint{2.810003in}{1.252199in}}%
\pgfpathcurveto{\pgfqpoint{2.806096in}{1.248292in}}{\pgfqpoint{2.803901in}{1.242993in}}{\pgfqpoint{2.803901in}{1.237468in}}%
\pgfpathcurveto{\pgfqpoint{2.803901in}{1.231943in}}{\pgfqpoint{2.806096in}{1.226643in}}{\pgfqpoint{2.810003in}{1.222736in}}%
\pgfpathcurveto{\pgfqpoint{2.813910in}{1.218830in}}{\pgfqpoint{2.819209in}{1.216635in}}{\pgfqpoint{2.824734in}{1.216635in}}%
\pgfpathclose%
\pgfusepath{stroke,fill}%
\end{pgfscope}%
\begin{pgfscope}%
\pgfpathrectangle{\pgfqpoint{0.562500in}{0.275000in}}{\pgfqpoint{3.487500in}{1.925000in}}%
\pgfusepath{clip}%
\pgfsetbuttcap%
\pgfsetroundjoin%
\definecolor{currentfill}{rgb}{0.000000,0.000000,0.000000}%
\pgfsetfillcolor{currentfill}%
\pgfsetlinewidth{1.003750pt}%
\definecolor{currentstroke}{rgb}{0.000000,0.000000,0.000000}%
\pgfsetstrokecolor{currentstroke}%
\pgfsetdash{}{0pt}%
\pgfpathmoveto{\pgfqpoint{2.824734in}{1.216635in}}%
\pgfpathcurveto{\pgfqpoint{2.830260in}{1.216635in}}{\pgfqpoint{2.835559in}{1.218830in}}{\pgfqpoint{2.839466in}{1.222736in}}%
\pgfpathcurveto{\pgfqpoint{2.843373in}{1.226643in}}{\pgfqpoint{2.845568in}{1.231943in}}{\pgfqpoint{2.845568in}{1.237468in}}%
\pgfpathcurveto{\pgfqpoint{2.845568in}{1.242993in}}{\pgfqpoint{2.843373in}{1.248292in}}{\pgfqpoint{2.839466in}{1.252199in}}%
\pgfpathcurveto{\pgfqpoint{2.835559in}{1.256106in}}{\pgfqpoint{2.830260in}{1.258301in}}{\pgfqpoint{2.824734in}{1.258301in}}%
\pgfpathcurveto{\pgfqpoint{2.819209in}{1.258301in}}{\pgfqpoint{2.813910in}{1.256106in}}{\pgfqpoint{2.810003in}{1.252199in}}%
\pgfpathcurveto{\pgfqpoint{2.806096in}{1.248292in}}{\pgfqpoint{2.803901in}{1.242993in}}{\pgfqpoint{2.803901in}{1.237468in}}%
\pgfpathcurveto{\pgfqpoint{2.803901in}{1.231943in}}{\pgfqpoint{2.806096in}{1.226643in}}{\pgfqpoint{2.810003in}{1.222736in}}%
\pgfpathcurveto{\pgfqpoint{2.813910in}{1.218830in}}{\pgfqpoint{2.819209in}{1.216635in}}{\pgfqpoint{2.824734in}{1.216635in}}%
\pgfpathclose%
\pgfusepath{stroke,fill}%
\end{pgfscope}%
\begin{pgfscope}%
\pgfpathrectangle{\pgfqpoint{0.562500in}{0.275000in}}{\pgfqpoint{3.487500in}{1.925000in}}%
\pgfusepath{clip}%
\pgfsetbuttcap%
\pgfsetroundjoin%
\definecolor{currentfill}{rgb}{0.000000,0.000000,0.000000}%
\pgfsetfillcolor{currentfill}%
\pgfsetlinewidth{1.003750pt}%
\definecolor{currentstroke}{rgb}{0.000000,0.000000,0.000000}%
\pgfsetstrokecolor{currentstroke}%
\pgfsetdash{}{0pt}%
\pgfpathmoveto{\pgfqpoint{2.824734in}{1.216635in}}%
\pgfpathcurveto{\pgfqpoint{2.830260in}{1.216635in}}{\pgfqpoint{2.835559in}{1.218830in}}{\pgfqpoint{2.839466in}{1.222736in}}%
\pgfpathcurveto{\pgfqpoint{2.843373in}{1.226643in}}{\pgfqpoint{2.845568in}{1.231943in}}{\pgfqpoint{2.845568in}{1.237468in}}%
\pgfpathcurveto{\pgfqpoint{2.845568in}{1.242993in}}{\pgfqpoint{2.843373in}{1.248292in}}{\pgfqpoint{2.839466in}{1.252199in}}%
\pgfpathcurveto{\pgfqpoint{2.835559in}{1.256106in}}{\pgfqpoint{2.830260in}{1.258301in}}{\pgfqpoint{2.824734in}{1.258301in}}%
\pgfpathcurveto{\pgfqpoint{2.819209in}{1.258301in}}{\pgfqpoint{2.813910in}{1.256106in}}{\pgfqpoint{2.810003in}{1.252199in}}%
\pgfpathcurveto{\pgfqpoint{2.806096in}{1.248292in}}{\pgfqpoint{2.803901in}{1.242993in}}{\pgfqpoint{2.803901in}{1.237468in}}%
\pgfpathcurveto{\pgfqpoint{2.803901in}{1.231943in}}{\pgfqpoint{2.806096in}{1.226643in}}{\pgfqpoint{2.810003in}{1.222736in}}%
\pgfpathcurveto{\pgfqpoint{2.813910in}{1.218830in}}{\pgfqpoint{2.819209in}{1.216635in}}{\pgfqpoint{2.824734in}{1.216635in}}%
\pgfpathclose%
\pgfusepath{stroke,fill}%
\end{pgfscope}%
\begin{pgfscope}%
\pgfpathrectangle{\pgfqpoint{0.562500in}{0.275000in}}{\pgfqpoint{3.487500in}{1.925000in}}%
\pgfusepath{clip}%
\pgfsetbuttcap%
\pgfsetroundjoin%
\definecolor{currentfill}{rgb}{0.000000,0.000000,0.000000}%
\pgfsetfillcolor{currentfill}%
\pgfsetlinewidth{1.003750pt}%
\definecolor{currentstroke}{rgb}{0.000000,0.000000,0.000000}%
\pgfsetstrokecolor{currentstroke}%
\pgfsetdash{}{0pt}%
\pgfpathmoveto{\pgfqpoint{2.824734in}{1.216635in}}%
\pgfpathcurveto{\pgfqpoint{2.830260in}{1.216635in}}{\pgfqpoint{2.835559in}{1.218830in}}{\pgfqpoint{2.839466in}{1.222736in}}%
\pgfpathcurveto{\pgfqpoint{2.843373in}{1.226643in}}{\pgfqpoint{2.845568in}{1.231943in}}{\pgfqpoint{2.845568in}{1.237468in}}%
\pgfpathcurveto{\pgfqpoint{2.845568in}{1.242993in}}{\pgfqpoint{2.843373in}{1.248292in}}{\pgfqpoint{2.839466in}{1.252199in}}%
\pgfpathcurveto{\pgfqpoint{2.835559in}{1.256106in}}{\pgfqpoint{2.830260in}{1.258301in}}{\pgfqpoint{2.824734in}{1.258301in}}%
\pgfpathcurveto{\pgfqpoint{2.819209in}{1.258301in}}{\pgfqpoint{2.813910in}{1.256106in}}{\pgfqpoint{2.810003in}{1.252199in}}%
\pgfpathcurveto{\pgfqpoint{2.806096in}{1.248292in}}{\pgfqpoint{2.803901in}{1.242993in}}{\pgfqpoint{2.803901in}{1.237468in}}%
\pgfpathcurveto{\pgfqpoint{2.803901in}{1.231943in}}{\pgfqpoint{2.806096in}{1.226643in}}{\pgfqpoint{2.810003in}{1.222736in}}%
\pgfpathcurveto{\pgfqpoint{2.813910in}{1.218830in}}{\pgfqpoint{2.819209in}{1.216635in}}{\pgfqpoint{2.824734in}{1.216635in}}%
\pgfpathclose%
\pgfusepath{stroke,fill}%
\end{pgfscope}%
\begin{pgfscope}%
\pgfpathrectangle{\pgfqpoint{0.562500in}{0.275000in}}{\pgfqpoint{3.487500in}{1.925000in}}%
\pgfusepath{clip}%
\pgfsetbuttcap%
\pgfsetroundjoin%
\definecolor{currentfill}{rgb}{0.000000,0.000000,0.000000}%
\pgfsetfillcolor{currentfill}%
\pgfsetlinewidth{1.003750pt}%
\definecolor{currentstroke}{rgb}{0.000000,0.000000,0.000000}%
\pgfsetstrokecolor{currentstroke}%
\pgfsetdash{}{0pt}%
\pgfpathmoveto{\pgfqpoint{2.824734in}{1.216635in}}%
\pgfpathcurveto{\pgfqpoint{2.830260in}{1.216635in}}{\pgfqpoint{2.835559in}{1.218830in}}{\pgfqpoint{2.839466in}{1.222736in}}%
\pgfpathcurveto{\pgfqpoint{2.843373in}{1.226643in}}{\pgfqpoint{2.845568in}{1.231943in}}{\pgfqpoint{2.845568in}{1.237468in}}%
\pgfpathcurveto{\pgfqpoint{2.845568in}{1.242993in}}{\pgfqpoint{2.843373in}{1.248292in}}{\pgfqpoint{2.839466in}{1.252199in}}%
\pgfpathcurveto{\pgfqpoint{2.835559in}{1.256106in}}{\pgfqpoint{2.830260in}{1.258301in}}{\pgfqpoint{2.824734in}{1.258301in}}%
\pgfpathcurveto{\pgfqpoint{2.819209in}{1.258301in}}{\pgfqpoint{2.813910in}{1.256106in}}{\pgfqpoint{2.810003in}{1.252199in}}%
\pgfpathcurveto{\pgfqpoint{2.806096in}{1.248292in}}{\pgfqpoint{2.803901in}{1.242993in}}{\pgfqpoint{2.803901in}{1.237468in}}%
\pgfpathcurveto{\pgfqpoint{2.803901in}{1.231943in}}{\pgfqpoint{2.806096in}{1.226643in}}{\pgfqpoint{2.810003in}{1.222736in}}%
\pgfpathcurveto{\pgfqpoint{2.813910in}{1.218830in}}{\pgfqpoint{2.819209in}{1.216635in}}{\pgfqpoint{2.824734in}{1.216635in}}%
\pgfpathclose%
\pgfusepath{stroke,fill}%
\end{pgfscope}%
\begin{pgfscope}%
\pgfpathrectangle{\pgfqpoint{0.562500in}{0.275000in}}{\pgfqpoint{3.487500in}{1.925000in}}%
\pgfusepath{clip}%
\pgfsetbuttcap%
\pgfsetroundjoin%
\definecolor{currentfill}{rgb}{0.000000,0.000000,0.000000}%
\pgfsetfillcolor{currentfill}%
\pgfsetlinewidth{1.003750pt}%
\definecolor{currentstroke}{rgb}{0.000000,0.000000,0.000000}%
\pgfsetstrokecolor{currentstroke}%
\pgfsetdash{}{0pt}%
\pgfpathmoveto{\pgfqpoint{2.824734in}{1.216635in}}%
\pgfpathcurveto{\pgfqpoint{2.830260in}{1.216635in}}{\pgfqpoint{2.835559in}{1.218830in}}{\pgfqpoint{2.839466in}{1.222736in}}%
\pgfpathcurveto{\pgfqpoint{2.843373in}{1.226643in}}{\pgfqpoint{2.845568in}{1.231943in}}{\pgfqpoint{2.845568in}{1.237468in}}%
\pgfpathcurveto{\pgfqpoint{2.845568in}{1.242993in}}{\pgfqpoint{2.843373in}{1.248292in}}{\pgfqpoint{2.839466in}{1.252199in}}%
\pgfpathcurveto{\pgfqpoint{2.835559in}{1.256106in}}{\pgfqpoint{2.830260in}{1.258301in}}{\pgfqpoint{2.824734in}{1.258301in}}%
\pgfpathcurveto{\pgfqpoint{2.819209in}{1.258301in}}{\pgfqpoint{2.813910in}{1.256106in}}{\pgfqpoint{2.810003in}{1.252199in}}%
\pgfpathcurveto{\pgfqpoint{2.806096in}{1.248292in}}{\pgfqpoint{2.803901in}{1.242993in}}{\pgfqpoint{2.803901in}{1.237468in}}%
\pgfpathcurveto{\pgfqpoint{2.803901in}{1.231943in}}{\pgfqpoint{2.806096in}{1.226643in}}{\pgfqpoint{2.810003in}{1.222736in}}%
\pgfpathcurveto{\pgfqpoint{2.813910in}{1.218830in}}{\pgfqpoint{2.819209in}{1.216635in}}{\pgfqpoint{2.824734in}{1.216635in}}%
\pgfpathclose%
\pgfusepath{stroke,fill}%
\end{pgfscope}%
\begin{pgfscope}%
\pgfpathrectangle{\pgfqpoint{0.562500in}{0.275000in}}{\pgfqpoint{3.487500in}{1.925000in}}%
\pgfusepath{clip}%
\pgfsetbuttcap%
\pgfsetroundjoin%
\definecolor{currentfill}{rgb}{0.000000,0.000000,0.000000}%
\pgfsetfillcolor{currentfill}%
\pgfsetlinewidth{1.003750pt}%
\definecolor{currentstroke}{rgb}{0.000000,0.000000,0.000000}%
\pgfsetstrokecolor{currentstroke}%
\pgfsetdash{}{0pt}%
\pgfpathmoveto{\pgfqpoint{2.824734in}{1.216635in}}%
\pgfpathcurveto{\pgfqpoint{2.830260in}{1.216635in}}{\pgfqpoint{2.835559in}{1.218830in}}{\pgfqpoint{2.839466in}{1.222736in}}%
\pgfpathcurveto{\pgfqpoint{2.843373in}{1.226643in}}{\pgfqpoint{2.845568in}{1.231943in}}{\pgfqpoint{2.845568in}{1.237468in}}%
\pgfpathcurveto{\pgfqpoint{2.845568in}{1.242993in}}{\pgfqpoint{2.843373in}{1.248292in}}{\pgfqpoint{2.839466in}{1.252199in}}%
\pgfpathcurveto{\pgfqpoint{2.835559in}{1.256106in}}{\pgfqpoint{2.830260in}{1.258301in}}{\pgfqpoint{2.824734in}{1.258301in}}%
\pgfpathcurveto{\pgfqpoint{2.819209in}{1.258301in}}{\pgfqpoint{2.813910in}{1.256106in}}{\pgfqpoint{2.810003in}{1.252199in}}%
\pgfpathcurveto{\pgfqpoint{2.806096in}{1.248292in}}{\pgfqpoint{2.803901in}{1.242993in}}{\pgfqpoint{2.803901in}{1.237468in}}%
\pgfpathcurveto{\pgfqpoint{2.803901in}{1.231943in}}{\pgfqpoint{2.806096in}{1.226643in}}{\pgfqpoint{2.810003in}{1.222736in}}%
\pgfpathcurveto{\pgfqpoint{2.813910in}{1.218830in}}{\pgfqpoint{2.819209in}{1.216635in}}{\pgfqpoint{2.824734in}{1.216635in}}%
\pgfpathclose%
\pgfusepath{stroke,fill}%
\end{pgfscope}%
\begin{pgfscope}%
\pgfpathrectangle{\pgfqpoint{0.562500in}{0.275000in}}{\pgfqpoint{3.487500in}{1.925000in}}%
\pgfusepath{clip}%
\pgfsetbuttcap%
\pgfsetroundjoin%
\definecolor{currentfill}{rgb}{0.000000,0.000000,0.000000}%
\pgfsetfillcolor{currentfill}%
\pgfsetlinewidth{1.003750pt}%
\definecolor{currentstroke}{rgb}{0.000000,0.000000,0.000000}%
\pgfsetstrokecolor{currentstroke}%
\pgfsetdash{}{0pt}%
\pgfpathmoveto{\pgfqpoint{2.824734in}{1.216635in}}%
\pgfpathcurveto{\pgfqpoint{2.830260in}{1.216635in}}{\pgfqpoint{2.835559in}{1.218830in}}{\pgfqpoint{2.839466in}{1.222736in}}%
\pgfpathcurveto{\pgfqpoint{2.843373in}{1.226643in}}{\pgfqpoint{2.845568in}{1.231943in}}{\pgfqpoint{2.845568in}{1.237468in}}%
\pgfpathcurveto{\pgfqpoint{2.845568in}{1.242993in}}{\pgfqpoint{2.843373in}{1.248292in}}{\pgfqpoint{2.839466in}{1.252199in}}%
\pgfpathcurveto{\pgfqpoint{2.835559in}{1.256106in}}{\pgfqpoint{2.830260in}{1.258301in}}{\pgfqpoint{2.824734in}{1.258301in}}%
\pgfpathcurveto{\pgfqpoint{2.819209in}{1.258301in}}{\pgfqpoint{2.813910in}{1.256106in}}{\pgfqpoint{2.810003in}{1.252199in}}%
\pgfpathcurveto{\pgfqpoint{2.806096in}{1.248292in}}{\pgfqpoint{2.803901in}{1.242993in}}{\pgfqpoint{2.803901in}{1.237468in}}%
\pgfpathcurveto{\pgfqpoint{2.803901in}{1.231943in}}{\pgfqpoint{2.806096in}{1.226643in}}{\pgfqpoint{2.810003in}{1.222736in}}%
\pgfpathcurveto{\pgfqpoint{2.813910in}{1.218830in}}{\pgfqpoint{2.819209in}{1.216635in}}{\pgfqpoint{2.824734in}{1.216635in}}%
\pgfpathclose%
\pgfusepath{stroke,fill}%
\end{pgfscope}%
\begin{pgfscope}%
\pgfpathrectangle{\pgfqpoint{0.562500in}{0.275000in}}{\pgfqpoint{3.487500in}{1.925000in}}%
\pgfusepath{clip}%
\pgfsetbuttcap%
\pgfsetroundjoin%
\definecolor{currentfill}{rgb}{0.000000,0.000000,0.000000}%
\pgfsetfillcolor{currentfill}%
\pgfsetlinewidth{1.003750pt}%
\definecolor{currentstroke}{rgb}{0.000000,0.000000,0.000000}%
\pgfsetstrokecolor{currentstroke}%
\pgfsetdash{}{0pt}%
\pgfpathmoveto{\pgfqpoint{2.824734in}{1.216635in}}%
\pgfpathcurveto{\pgfqpoint{2.830260in}{1.216635in}}{\pgfqpoint{2.835559in}{1.218830in}}{\pgfqpoint{2.839466in}{1.222736in}}%
\pgfpathcurveto{\pgfqpoint{2.843373in}{1.226643in}}{\pgfqpoint{2.845568in}{1.231943in}}{\pgfqpoint{2.845568in}{1.237468in}}%
\pgfpathcurveto{\pgfqpoint{2.845568in}{1.242993in}}{\pgfqpoint{2.843373in}{1.248292in}}{\pgfqpoint{2.839466in}{1.252199in}}%
\pgfpathcurveto{\pgfqpoint{2.835559in}{1.256106in}}{\pgfqpoint{2.830260in}{1.258301in}}{\pgfqpoint{2.824734in}{1.258301in}}%
\pgfpathcurveto{\pgfqpoint{2.819209in}{1.258301in}}{\pgfqpoint{2.813910in}{1.256106in}}{\pgfqpoint{2.810003in}{1.252199in}}%
\pgfpathcurveto{\pgfqpoint{2.806096in}{1.248292in}}{\pgfqpoint{2.803901in}{1.242993in}}{\pgfqpoint{2.803901in}{1.237468in}}%
\pgfpathcurveto{\pgfqpoint{2.803901in}{1.231943in}}{\pgfqpoint{2.806096in}{1.226643in}}{\pgfqpoint{2.810003in}{1.222736in}}%
\pgfpathcurveto{\pgfqpoint{2.813910in}{1.218830in}}{\pgfqpoint{2.819209in}{1.216635in}}{\pgfqpoint{2.824734in}{1.216635in}}%
\pgfpathclose%
\pgfusepath{stroke,fill}%
\end{pgfscope}%
\begin{pgfscope}%
\pgfpathrectangle{\pgfqpoint{0.562500in}{0.275000in}}{\pgfqpoint{3.487500in}{1.925000in}}%
\pgfusepath{clip}%
\pgfsetbuttcap%
\pgfsetroundjoin%
\definecolor{currentfill}{rgb}{0.000000,0.000000,0.000000}%
\pgfsetfillcolor{currentfill}%
\pgfsetlinewidth{1.003750pt}%
\definecolor{currentstroke}{rgb}{0.000000,0.000000,0.000000}%
\pgfsetstrokecolor{currentstroke}%
\pgfsetdash{}{0pt}%
\pgfpathmoveto{\pgfqpoint{2.824734in}{1.216635in}}%
\pgfpathcurveto{\pgfqpoint{2.830260in}{1.216635in}}{\pgfqpoint{2.835559in}{1.218830in}}{\pgfqpoint{2.839466in}{1.222736in}}%
\pgfpathcurveto{\pgfqpoint{2.843373in}{1.226643in}}{\pgfqpoint{2.845568in}{1.231943in}}{\pgfqpoint{2.845568in}{1.237468in}}%
\pgfpathcurveto{\pgfqpoint{2.845568in}{1.242993in}}{\pgfqpoint{2.843373in}{1.248292in}}{\pgfqpoint{2.839466in}{1.252199in}}%
\pgfpathcurveto{\pgfqpoint{2.835559in}{1.256106in}}{\pgfqpoint{2.830260in}{1.258301in}}{\pgfqpoint{2.824734in}{1.258301in}}%
\pgfpathcurveto{\pgfqpoint{2.819209in}{1.258301in}}{\pgfqpoint{2.813910in}{1.256106in}}{\pgfqpoint{2.810003in}{1.252199in}}%
\pgfpathcurveto{\pgfqpoint{2.806096in}{1.248292in}}{\pgfqpoint{2.803901in}{1.242993in}}{\pgfqpoint{2.803901in}{1.237468in}}%
\pgfpathcurveto{\pgfqpoint{2.803901in}{1.231943in}}{\pgfqpoint{2.806096in}{1.226643in}}{\pgfqpoint{2.810003in}{1.222736in}}%
\pgfpathcurveto{\pgfqpoint{2.813910in}{1.218830in}}{\pgfqpoint{2.819209in}{1.216635in}}{\pgfqpoint{2.824734in}{1.216635in}}%
\pgfpathclose%
\pgfusepath{stroke,fill}%
\end{pgfscope}%
\begin{pgfscope}%
\pgfpathrectangle{\pgfqpoint{0.562500in}{0.275000in}}{\pgfqpoint{3.487500in}{1.925000in}}%
\pgfusepath{clip}%
\pgfsetbuttcap%
\pgfsetroundjoin%
\definecolor{currentfill}{rgb}{0.000000,0.000000,0.000000}%
\pgfsetfillcolor{currentfill}%
\pgfsetlinewidth{1.003750pt}%
\definecolor{currentstroke}{rgb}{0.000000,0.000000,0.000000}%
\pgfsetstrokecolor{currentstroke}%
\pgfsetdash{}{0pt}%
\pgfpathmoveto{\pgfqpoint{2.824734in}{1.216635in}}%
\pgfpathcurveto{\pgfqpoint{2.830260in}{1.216635in}}{\pgfqpoint{2.835559in}{1.218830in}}{\pgfqpoint{2.839466in}{1.222736in}}%
\pgfpathcurveto{\pgfqpoint{2.843373in}{1.226643in}}{\pgfqpoint{2.845568in}{1.231943in}}{\pgfqpoint{2.845568in}{1.237468in}}%
\pgfpathcurveto{\pgfqpoint{2.845568in}{1.242993in}}{\pgfqpoint{2.843373in}{1.248292in}}{\pgfqpoint{2.839466in}{1.252199in}}%
\pgfpathcurveto{\pgfqpoint{2.835559in}{1.256106in}}{\pgfqpoint{2.830260in}{1.258301in}}{\pgfqpoint{2.824734in}{1.258301in}}%
\pgfpathcurveto{\pgfqpoint{2.819209in}{1.258301in}}{\pgfqpoint{2.813910in}{1.256106in}}{\pgfqpoint{2.810003in}{1.252199in}}%
\pgfpathcurveto{\pgfqpoint{2.806096in}{1.248292in}}{\pgfqpoint{2.803901in}{1.242993in}}{\pgfqpoint{2.803901in}{1.237468in}}%
\pgfpathcurveto{\pgfqpoint{2.803901in}{1.231943in}}{\pgfqpoint{2.806096in}{1.226643in}}{\pgfqpoint{2.810003in}{1.222736in}}%
\pgfpathcurveto{\pgfqpoint{2.813910in}{1.218830in}}{\pgfqpoint{2.819209in}{1.216635in}}{\pgfqpoint{2.824734in}{1.216635in}}%
\pgfpathclose%
\pgfusepath{stroke,fill}%
\end{pgfscope}%
\begin{pgfscope}%
\pgfpathrectangle{\pgfqpoint{0.562500in}{0.275000in}}{\pgfqpoint{3.487500in}{1.925000in}}%
\pgfusepath{clip}%
\pgfsetbuttcap%
\pgfsetroundjoin%
\definecolor{currentfill}{rgb}{0.000000,0.000000,0.000000}%
\pgfsetfillcolor{currentfill}%
\pgfsetlinewidth{1.003750pt}%
\definecolor{currentstroke}{rgb}{0.000000,0.000000,0.000000}%
\pgfsetstrokecolor{currentstroke}%
\pgfsetdash{}{0pt}%
\pgfpathmoveto{\pgfqpoint{2.824734in}{1.216635in}}%
\pgfpathcurveto{\pgfqpoint{2.830260in}{1.216635in}}{\pgfqpoint{2.835559in}{1.218830in}}{\pgfqpoint{2.839466in}{1.222736in}}%
\pgfpathcurveto{\pgfqpoint{2.843373in}{1.226643in}}{\pgfqpoint{2.845568in}{1.231943in}}{\pgfqpoint{2.845568in}{1.237468in}}%
\pgfpathcurveto{\pgfqpoint{2.845568in}{1.242993in}}{\pgfqpoint{2.843373in}{1.248292in}}{\pgfqpoint{2.839466in}{1.252199in}}%
\pgfpathcurveto{\pgfqpoint{2.835559in}{1.256106in}}{\pgfqpoint{2.830260in}{1.258301in}}{\pgfqpoint{2.824734in}{1.258301in}}%
\pgfpathcurveto{\pgfqpoint{2.819209in}{1.258301in}}{\pgfqpoint{2.813910in}{1.256106in}}{\pgfqpoint{2.810003in}{1.252199in}}%
\pgfpathcurveto{\pgfqpoint{2.806096in}{1.248292in}}{\pgfqpoint{2.803901in}{1.242993in}}{\pgfqpoint{2.803901in}{1.237468in}}%
\pgfpathcurveto{\pgfqpoint{2.803901in}{1.231943in}}{\pgfqpoint{2.806096in}{1.226643in}}{\pgfqpoint{2.810003in}{1.222736in}}%
\pgfpathcurveto{\pgfqpoint{2.813910in}{1.218830in}}{\pgfqpoint{2.819209in}{1.216635in}}{\pgfqpoint{2.824734in}{1.216635in}}%
\pgfpathclose%
\pgfusepath{stroke,fill}%
\end{pgfscope}%
\begin{pgfscope}%
\pgfpathrectangle{\pgfqpoint{0.562500in}{0.275000in}}{\pgfqpoint{3.487500in}{1.925000in}}%
\pgfusepath{clip}%
\pgfsetbuttcap%
\pgfsetroundjoin%
\definecolor{currentfill}{rgb}{0.000000,0.000000,0.000000}%
\pgfsetfillcolor{currentfill}%
\pgfsetlinewidth{1.003750pt}%
\definecolor{currentstroke}{rgb}{0.000000,0.000000,0.000000}%
\pgfsetstrokecolor{currentstroke}%
\pgfsetdash{}{0pt}%
\pgfpathmoveto{\pgfqpoint{2.824734in}{1.216635in}}%
\pgfpathcurveto{\pgfqpoint{2.830260in}{1.216635in}}{\pgfqpoint{2.835559in}{1.218830in}}{\pgfqpoint{2.839466in}{1.222736in}}%
\pgfpathcurveto{\pgfqpoint{2.843373in}{1.226643in}}{\pgfqpoint{2.845568in}{1.231943in}}{\pgfqpoint{2.845568in}{1.237468in}}%
\pgfpathcurveto{\pgfqpoint{2.845568in}{1.242993in}}{\pgfqpoint{2.843373in}{1.248292in}}{\pgfqpoint{2.839466in}{1.252199in}}%
\pgfpathcurveto{\pgfqpoint{2.835559in}{1.256106in}}{\pgfqpoint{2.830260in}{1.258301in}}{\pgfqpoint{2.824734in}{1.258301in}}%
\pgfpathcurveto{\pgfqpoint{2.819209in}{1.258301in}}{\pgfqpoint{2.813910in}{1.256106in}}{\pgfqpoint{2.810003in}{1.252199in}}%
\pgfpathcurveto{\pgfqpoint{2.806096in}{1.248292in}}{\pgfqpoint{2.803901in}{1.242993in}}{\pgfqpoint{2.803901in}{1.237468in}}%
\pgfpathcurveto{\pgfqpoint{2.803901in}{1.231943in}}{\pgfqpoint{2.806096in}{1.226643in}}{\pgfqpoint{2.810003in}{1.222736in}}%
\pgfpathcurveto{\pgfqpoint{2.813910in}{1.218830in}}{\pgfqpoint{2.819209in}{1.216635in}}{\pgfqpoint{2.824734in}{1.216635in}}%
\pgfpathclose%
\pgfusepath{stroke,fill}%
\end{pgfscope}%
\begin{pgfscope}%
\pgfpathrectangle{\pgfqpoint{0.562500in}{0.275000in}}{\pgfqpoint{3.487500in}{1.925000in}}%
\pgfusepath{clip}%
\pgfsetbuttcap%
\pgfsetroundjoin%
\definecolor{currentfill}{rgb}{0.000000,0.000000,0.000000}%
\pgfsetfillcolor{currentfill}%
\pgfsetlinewidth{1.003750pt}%
\definecolor{currentstroke}{rgb}{0.000000,0.000000,0.000000}%
\pgfsetstrokecolor{currentstroke}%
\pgfsetdash{}{0pt}%
\pgfpathmoveto{\pgfqpoint{2.824734in}{1.216635in}}%
\pgfpathcurveto{\pgfqpoint{2.830260in}{1.216635in}}{\pgfqpoint{2.835559in}{1.218830in}}{\pgfqpoint{2.839466in}{1.222736in}}%
\pgfpathcurveto{\pgfqpoint{2.843373in}{1.226643in}}{\pgfqpoint{2.845568in}{1.231943in}}{\pgfqpoint{2.845568in}{1.237468in}}%
\pgfpathcurveto{\pgfqpoint{2.845568in}{1.242993in}}{\pgfqpoint{2.843373in}{1.248292in}}{\pgfqpoint{2.839466in}{1.252199in}}%
\pgfpathcurveto{\pgfqpoint{2.835559in}{1.256106in}}{\pgfqpoint{2.830260in}{1.258301in}}{\pgfqpoint{2.824734in}{1.258301in}}%
\pgfpathcurveto{\pgfqpoint{2.819209in}{1.258301in}}{\pgfqpoint{2.813910in}{1.256106in}}{\pgfqpoint{2.810003in}{1.252199in}}%
\pgfpathcurveto{\pgfqpoint{2.806096in}{1.248292in}}{\pgfqpoint{2.803901in}{1.242993in}}{\pgfqpoint{2.803901in}{1.237468in}}%
\pgfpathcurveto{\pgfqpoint{2.803901in}{1.231943in}}{\pgfqpoint{2.806096in}{1.226643in}}{\pgfqpoint{2.810003in}{1.222736in}}%
\pgfpathcurveto{\pgfqpoint{2.813910in}{1.218830in}}{\pgfqpoint{2.819209in}{1.216635in}}{\pgfqpoint{2.824734in}{1.216635in}}%
\pgfpathclose%
\pgfusepath{stroke,fill}%
\end{pgfscope}%
\begin{pgfscope}%
\pgfpathrectangle{\pgfqpoint{0.562500in}{0.275000in}}{\pgfqpoint{3.487500in}{1.925000in}}%
\pgfusepath{clip}%
\pgfsetbuttcap%
\pgfsetroundjoin%
\definecolor{currentfill}{rgb}{0.000000,0.000000,0.000000}%
\pgfsetfillcolor{currentfill}%
\pgfsetlinewidth{1.003750pt}%
\definecolor{currentstroke}{rgb}{0.000000,0.000000,0.000000}%
\pgfsetstrokecolor{currentstroke}%
\pgfsetdash{}{0pt}%
\pgfpathmoveto{\pgfqpoint{2.824734in}{1.216635in}}%
\pgfpathcurveto{\pgfqpoint{2.830260in}{1.216635in}}{\pgfqpoint{2.835559in}{1.218830in}}{\pgfqpoint{2.839466in}{1.222736in}}%
\pgfpathcurveto{\pgfqpoint{2.843373in}{1.226643in}}{\pgfqpoint{2.845568in}{1.231943in}}{\pgfqpoint{2.845568in}{1.237468in}}%
\pgfpathcurveto{\pgfqpoint{2.845568in}{1.242993in}}{\pgfqpoint{2.843373in}{1.248292in}}{\pgfqpoint{2.839466in}{1.252199in}}%
\pgfpathcurveto{\pgfqpoint{2.835559in}{1.256106in}}{\pgfqpoint{2.830260in}{1.258301in}}{\pgfqpoint{2.824734in}{1.258301in}}%
\pgfpathcurveto{\pgfqpoint{2.819209in}{1.258301in}}{\pgfqpoint{2.813910in}{1.256106in}}{\pgfqpoint{2.810003in}{1.252199in}}%
\pgfpathcurveto{\pgfqpoint{2.806096in}{1.248292in}}{\pgfqpoint{2.803901in}{1.242993in}}{\pgfqpoint{2.803901in}{1.237468in}}%
\pgfpathcurveto{\pgfqpoint{2.803901in}{1.231943in}}{\pgfqpoint{2.806096in}{1.226643in}}{\pgfqpoint{2.810003in}{1.222736in}}%
\pgfpathcurveto{\pgfqpoint{2.813910in}{1.218830in}}{\pgfqpoint{2.819209in}{1.216635in}}{\pgfqpoint{2.824734in}{1.216635in}}%
\pgfpathclose%
\pgfusepath{stroke,fill}%
\end{pgfscope}%
\begin{pgfscope}%
\pgfpathrectangle{\pgfqpoint{0.562500in}{0.275000in}}{\pgfqpoint{3.487500in}{1.925000in}}%
\pgfusepath{clip}%
\pgfsetbuttcap%
\pgfsetroundjoin%
\definecolor{currentfill}{rgb}{0.000000,0.000000,0.000000}%
\pgfsetfillcolor{currentfill}%
\pgfsetlinewidth{1.003750pt}%
\definecolor{currentstroke}{rgb}{0.000000,0.000000,0.000000}%
\pgfsetstrokecolor{currentstroke}%
\pgfsetdash{}{0pt}%
\pgfpathmoveto{\pgfqpoint{2.824734in}{1.216635in}}%
\pgfpathcurveto{\pgfqpoint{2.830260in}{1.216635in}}{\pgfqpoint{2.835559in}{1.218830in}}{\pgfqpoint{2.839466in}{1.222736in}}%
\pgfpathcurveto{\pgfqpoint{2.843373in}{1.226643in}}{\pgfqpoint{2.845568in}{1.231943in}}{\pgfqpoint{2.845568in}{1.237468in}}%
\pgfpathcurveto{\pgfqpoint{2.845568in}{1.242993in}}{\pgfqpoint{2.843373in}{1.248292in}}{\pgfqpoint{2.839466in}{1.252199in}}%
\pgfpathcurveto{\pgfqpoint{2.835559in}{1.256106in}}{\pgfqpoint{2.830260in}{1.258301in}}{\pgfqpoint{2.824734in}{1.258301in}}%
\pgfpathcurveto{\pgfqpoint{2.819209in}{1.258301in}}{\pgfqpoint{2.813910in}{1.256106in}}{\pgfqpoint{2.810003in}{1.252199in}}%
\pgfpathcurveto{\pgfqpoint{2.806096in}{1.248292in}}{\pgfqpoint{2.803901in}{1.242993in}}{\pgfqpoint{2.803901in}{1.237468in}}%
\pgfpathcurveto{\pgfqpoint{2.803901in}{1.231943in}}{\pgfqpoint{2.806096in}{1.226643in}}{\pgfqpoint{2.810003in}{1.222736in}}%
\pgfpathcurveto{\pgfqpoint{2.813910in}{1.218830in}}{\pgfqpoint{2.819209in}{1.216635in}}{\pgfqpoint{2.824734in}{1.216635in}}%
\pgfpathclose%
\pgfusepath{stroke,fill}%
\end{pgfscope}%
\begin{pgfscope}%
\pgfpathrectangle{\pgfqpoint{0.562500in}{0.275000in}}{\pgfqpoint{3.487500in}{1.925000in}}%
\pgfusepath{clip}%
\pgfsetbuttcap%
\pgfsetroundjoin%
\definecolor{currentfill}{rgb}{0.000000,0.000000,0.000000}%
\pgfsetfillcolor{currentfill}%
\pgfsetlinewidth{1.003750pt}%
\definecolor{currentstroke}{rgb}{0.000000,0.000000,0.000000}%
\pgfsetstrokecolor{currentstroke}%
\pgfsetdash{}{0pt}%
\pgfpathmoveto{\pgfqpoint{2.824734in}{1.216635in}}%
\pgfpathcurveto{\pgfqpoint{2.830260in}{1.216635in}}{\pgfqpoint{2.835559in}{1.218830in}}{\pgfqpoint{2.839466in}{1.222736in}}%
\pgfpathcurveto{\pgfqpoint{2.843373in}{1.226643in}}{\pgfqpoint{2.845568in}{1.231943in}}{\pgfqpoint{2.845568in}{1.237468in}}%
\pgfpathcurveto{\pgfqpoint{2.845568in}{1.242993in}}{\pgfqpoint{2.843373in}{1.248292in}}{\pgfqpoint{2.839466in}{1.252199in}}%
\pgfpathcurveto{\pgfqpoint{2.835559in}{1.256106in}}{\pgfqpoint{2.830260in}{1.258301in}}{\pgfqpoint{2.824734in}{1.258301in}}%
\pgfpathcurveto{\pgfqpoint{2.819209in}{1.258301in}}{\pgfqpoint{2.813910in}{1.256106in}}{\pgfqpoint{2.810003in}{1.252199in}}%
\pgfpathcurveto{\pgfqpoint{2.806096in}{1.248292in}}{\pgfqpoint{2.803901in}{1.242993in}}{\pgfqpoint{2.803901in}{1.237468in}}%
\pgfpathcurveto{\pgfqpoint{2.803901in}{1.231943in}}{\pgfqpoint{2.806096in}{1.226643in}}{\pgfqpoint{2.810003in}{1.222736in}}%
\pgfpathcurveto{\pgfqpoint{2.813910in}{1.218830in}}{\pgfqpoint{2.819209in}{1.216635in}}{\pgfqpoint{2.824734in}{1.216635in}}%
\pgfpathclose%
\pgfusepath{stroke,fill}%
\end{pgfscope}%
\begin{pgfscope}%
\pgfpathrectangle{\pgfqpoint{0.562500in}{0.275000in}}{\pgfqpoint{3.487500in}{1.925000in}}%
\pgfusepath{clip}%
\pgfsetbuttcap%
\pgfsetroundjoin%
\definecolor{currentfill}{rgb}{0.000000,0.000000,0.000000}%
\pgfsetfillcolor{currentfill}%
\pgfsetlinewidth{1.003750pt}%
\definecolor{currentstroke}{rgb}{0.000000,0.000000,0.000000}%
\pgfsetstrokecolor{currentstroke}%
\pgfsetdash{}{0pt}%
\pgfpathmoveto{\pgfqpoint{2.824734in}{2.076667in}}%
\pgfpathcurveto{\pgfqpoint{2.830260in}{2.076667in}}{\pgfqpoint{2.835559in}{2.078862in}}{\pgfqpoint{2.839466in}{2.082769in}}%
\pgfpathcurveto{\pgfqpoint{2.843373in}{2.086675in}}{\pgfqpoint{2.845568in}{2.091975in}}{\pgfqpoint{2.845568in}{2.097500in}}%
\pgfpathcurveto{\pgfqpoint{2.845568in}{2.103025in}}{\pgfqpoint{2.843373in}{2.108325in}}{\pgfqpoint{2.839466in}{2.112231in}}%
\pgfpathcurveto{\pgfqpoint{2.835559in}{2.116138in}}{\pgfqpoint{2.830260in}{2.118333in}}{\pgfqpoint{2.824734in}{2.118333in}}%
\pgfpathcurveto{\pgfqpoint{2.819209in}{2.118333in}}{\pgfqpoint{2.813910in}{2.116138in}}{\pgfqpoint{2.810003in}{2.112231in}}%
\pgfpathcurveto{\pgfqpoint{2.806096in}{2.108325in}}{\pgfqpoint{2.803901in}{2.103025in}}{\pgfqpoint{2.803901in}{2.097500in}}%
\pgfpathcurveto{\pgfqpoint{2.803901in}{2.091975in}}{\pgfqpoint{2.806096in}{2.086675in}}{\pgfqpoint{2.810003in}{2.082769in}}%
\pgfpathcurveto{\pgfqpoint{2.813910in}{2.078862in}}{\pgfqpoint{2.819209in}{2.076667in}}{\pgfqpoint{2.824734in}{2.076667in}}%
\pgfpathclose%
\pgfusepath{stroke,fill}%
\end{pgfscope}%
\begin{pgfscope}%
\pgfpathrectangle{\pgfqpoint{0.562500in}{0.275000in}}{\pgfqpoint{3.487500in}{1.925000in}}%
\pgfusepath{clip}%
\pgfsetbuttcap%
\pgfsetroundjoin%
\definecolor{currentfill}{rgb}{0.000000,0.000000,0.000000}%
\pgfsetfillcolor{currentfill}%
\pgfsetlinewidth{1.003750pt}%
\definecolor{currentstroke}{rgb}{0.000000,0.000000,0.000000}%
\pgfsetstrokecolor{currentstroke}%
\pgfsetdash{}{0pt}%
\pgfpathmoveto{\pgfqpoint{2.824734in}{1.216635in}}%
\pgfpathcurveto{\pgfqpoint{2.830260in}{1.216635in}}{\pgfqpoint{2.835559in}{1.218830in}}{\pgfqpoint{2.839466in}{1.222736in}}%
\pgfpathcurveto{\pgfqpoint{2.843373in}{1.226643in}}{\pgfqpoint{2.845568in}{1.231943in}}{\pgfqpoint{2.845568in}{1.237468in}}%
\pgfpathcurveto{\pgfqpoint{2.845568in}{1.242993in}}{\pgfqpoint{2.843373in}{1.248292in}}{\pgfqpoint{2.839466in}{1.252199in}}%
\pgfpathcurveto{\pgfqpoint{2.835559in}{1.256106in}}{\pgfqpoint{2.830260in}{1.258301in}}{\pgfqpoint{2.824734in}{1.258301in}}%
\pgfpathcurveto{\pgfqpoint{2.819209in}{1.258301in}}{\pgfqpoint{2.813910in}{1.256106in}}{\pgfqpoint{2.810003in}{1.252199in}}%
\pgfpathcurveto{\pgfqpoint{2.806096in}{1.248292in}}{\pgfqpoint{2.803901in}{1.242993in}}{\pgfqpoint{2.803901in}{1.237468in}}%
\pgfpathcurveto{\pgfqpoint{2.803901in}{1.231943in}}{\pgfqpoint{2.806096in}{1.226643in}}{\pgfqpoint{2.810003in}{1.222736in}}%
\pgfpathcurveto{\pgfqpoint{2.813910in}{1.218830in}}{\pgfqpoint{2.819209in}{1.216635in}}{\pgfqpoint{2.824734in}{1.216635in}}%
\pgfpathclose%
\pgfusepath{stroke,fill}%
\end{pgfscope}%
\begin{pgfscope}%
\pgfpathrectangle{\pgfqpoint{0.562500in}{0.275000in}}{\pgfqpoint{3.487500in}{1.925000in}}%
\pgfusepath{clip}%
\pgfsetbuttcap%
\pgfsetroundjoin%
\definecolor{currentfill}{rgb}{0.000000,0.000000,0.000000}%
\pgfsetfillcolor{currentfill}%
\pgfsetlinewidth{1.003750pt}%
\definecolor{currentstroke}{rgb}{0.000000,0.000000,0.000000}%
\pgfsetstrokecolor{currentstroke}%
\pgfsetdash{}{0pt}%
\pgfpathmoveto{\pgfqpoint{2.824734in}{1.216635in}}%
\pgfpathcurveto{\pgfqpoint{2.830260in}{1.216635in}}{\pgfqpoint{2.835559in}{1.218830in}}{\pgfqpoint{2.839466in}{1.222736in}}%
\pgfpathcurveto{\pgfqpoint{2.843373in}{1.226643in}}{\pgfqpoint{2.845568in}{1.231943in}}{\pgfqpoint{2.845568in}{1.237468in}}%
\pgfpathcurveto{\pgfqpoint{2.845568in}{1.242993in}}{\pgfqpoint{2.843373in}{1.248292in}}{\pgfqpoint{2.839466in}{1.252199in}}%
\pgfpathcurveto{\pgfqpoint{2.835559in}{1.256106in}}{\pgfqpoint{2.830260in}{1.258301in}}{\pgfqpoint{2.824734in}{1.258301in}}%
\pgfpathcurveto{\pgfqpoint{2.819209in}{1.258301in}}{\pgfqpoint{2.813910in}{1.256106in}}{\pgfqpoint{2.810003in}{1.252199in}}%
\pgfpathcurveto{\pgfqpoint{2.806096in}{1.248292in}}{\pgfqpoint{2.803901in}{1.242993in}}{\pgfqpoint{2.803901in}{1.237468in}}%
\pgfpathcurveto{\pgfqpoint{2.803901in}{1.231943in}}{\pgfqpoint{2.806096in}{1.226643in}}{\pgfqpoint{2.810003in}{1.222736in}}%
\pgfpathcurveto{\pgfqpoint{2.813910in}{1.218830in}}{\pgfqpoint{2.819209in}{1.216635in}}{\pgfqpoint{2.824734in}{1.216635in}}%
\pgfpathclose%
\pgfusepath{stroke,fill}%
\end{pgfscope}%
\begin{pgfscope}%
\pgfpathrectangle{\pgfqpoint{0.562500in}{0.275000in}}{\pgfqpoint{3.487500in}{1.925000in}}%
\pgfusepath{clip}%
\pgfsetbuttcap%
\pgfsetroundjoin%
\definecolor{currentfill}{rgb}{0.000000,0.000000,0.000000}%
\pgfsetfillcolor{currentfill}%
\pgfsetlinewidth{1.003750pt}%
\definecolor{currentstroke}{rgb}{0.000000,0.000000,0.000000}%
\pgfsetstrokecolor{currentstroke}%
\pgfsetdash{}{0pt}%
\pgfpathmoveto{\pgfqpoint{2.824734in}{2.076667in}}%
\pgfpathcurveto{\pgfqpoint{2.830260in}{2.076667in}}{\pgfqpoint{2.835559in}{2.078862in}}{\pgfqpoint{2.839466in}{2.082769in}}%
\pgfpathcurveto{\pgfqpoint{2.843373in}{2.086675in}}{\pgfqpoint{2.845568in}{2.091975in}}{\pgfqpoint{2.845568in}{2.097500in}}%
\pgfpathcurveto{\pgfqpoint{2.845568in}{2.103025in}}{\pgfqpoint{2.843373in}{2.108325in}}{\pgfqpoint{2.839466in}{2.112231in}}%
\pgfpathcurveto{\pgfqpoint{2.835559in}{2.116138in}}{\pgfqpoint{2.830260in}{2.118333in}}{\pgfqpoint{2.824734in}{2.118333in}}%
\pgfpathcurveto{\pgfqpoint{2.819209in}{2.118333in}}{\pgfqpoint{2.813910in}{2.116138in}}{\pgfqpoint{2.810003in}{2.112231in}}%
\pgfpathcurveto{\pgfqpoint{2.806096in}{2.108325in}}{\pgfqpoint{2.803901in}{2.103025in}}{\pgfqpoint{2.803901in}{2.097500in}}%
\pgfpathcurveto{\pgfqpoint{2.803901in}{2.091975in}}{\pgfqpoint{2.806096in}{2.086675in}}{\pgfqpoint{2.810003in}{2.082769in}}%
\pgfpathcurveto{\pgfqpoint{2.813910in}{2.078862in}}{\pgfqpoint{2.819209in}{2.076667in}}{\pgfqpoint{2.824734in}{2.076667in}}%
\pgfpathclose%
\pgfusepath{stroke,fill}%
\end{pgfscope}%
\begin{pgfscope}%
\pgfpathrectangle{\pgfqpoint{0.562500in}{0.275000in}}{\pgfqpoint{3.487500in}{1.925000in}}%
\pgfusepath{clip}%
\pgfsetbuttcap%
\pgfsetroundjoin%
\definecolor{currentfill}{rgb}{0.000000,0.000000,0.000000}%
\pgfsetfillcolor{currentfill}%
\pgfsetlinewidth{1.003750pt}%
\definecolor{currentstroke}{rgb}{0.000000,0.000000,0.000000}%
\pgfsetstrokecolor{currentstroke}%
\pgfsetdash{}{0pt}%
\pgfpathmoveto{\pgfqpoint{2.824734in}{1.216635in}}%
\pgfpathcurveto{\pgfqpoint{2.830260in}{1.216635in}}{\pgfqpoint{2.835559in}{1.218830in}}{\pgfqpoint{2.839466in}{1.222736in}}%
\pgfpathcurveto{\pgfqpoint{2.843373in}{1.226643in}}{\pgfqpoint{2.845568in}{1.231943in}}{\pgfqpoint{2.845568in}{1.237468in}}%
\pgfpathcurveto{\pgfqpoint{2.845568in}{1.242993in}}{\pgfqpoint{2.843373in}{1.248292in}}{\pgfqpoint{2.839466in}{1.252199in}}%
\pgfpathcurveto{\pgfqpoint{2.835559in}{1.256106in}}{\pgfqpoint{2.830260in}{1.258301in}}{\pgfqpoint{2.824734in}{1.258301in}}%
\pgfpathcurveto{\pgfqpoint{2.819209in}{1.258301in}}{\pgfqpoint{2.813910in}{1.256106in}}{\pgfqpoint{2.810003in}{1.252199in}}%
\pgfpathcurveto{\pgfqpoint{2.806096in}{1.248292in}}{\pgfqpoint{2.803901in}{1.242993in}}{\pgfqpoint{2.803901in}{1.237468in}}%
\pgfpathcurveto{\pgfqpoint{2.803901in}{1.231943in}}{\pgfqpoint{2.806096in}{1.226643in}}{\pgfqpoint{2.810003in}{1.222736in}}%
\pgfpathcurveto{\pgfqpoint{2.813910in}{1.218830in}}{\pgfqpoint{2.819209in}{1.216635in}}{\pgfqpoint{2.824734in}{1.216635in}}%
\pgfpathclose%
\pgfusepath{stroke,fill}%
\end{pgfscope}%
\begin{pgfscope}%
\pgfpathrectangle{\pgfqpoint{0.562500in}{0.275000in}}{\pgfqpoint{3.487500in}{1.925000in}}%
\pgfusepath{clip}%
\pgfsetbuttcap%
\pgfsetroundjoin%
\definecolor{currentfill}{rgb}{0.000000,0.000000,0.000000}%
\pgfsetfillcolor{currentfill}%
\pgfsetlinewidth{1.003750pt}%
\definecolor{currentstroke}{rgb}{0.000000,0.000000,0.000000}%
\pgfsetstrokecolor{currentstroke}%
\pgfsetdash{}{0pt}%
\pgfpathmoveto{\pgfqpoint{2.824734in}{1.216635in}}%
\pgfpathcurveto{\pgfqpoint{2.830260in}{1.216635in}}{\pgfqpoint{2.835559in}{1.218830in}}{\pgfqpoint{2.839466in}{1.222736in}}%
\pgfpathcurveto{\pgfqpoint{2.843373in}{1.226643in}}{\pgfqpoint{2.845568in}{1.231943in}}{\pgfqpoint{2.845568in}{1.237468in}}%
\pgfpathcurveto{\pgfqpoint{2.845568in}{1.242993in}}{\pgfqpoint{2.843373in}{1.248292in}}{\pgfqpoint{2.839466in}{1.252199in}}%
\pgfpathcurveto{\pgfqpoint{2.835559in}{1.256106in}}{\pgfqpoint{2.830260in}{1.258301in}}{\pgfqpoint{2.824734in}{1.258301in}}%
\pgfpathcurveto{\pgfqpoint{2.819209in}{1.258301in}}{\pgfqpoint{2.813910in}{1.256106in}}{\pgfqpoint{2.810003in}{1.252199in}}%
\pgfpathcurveto{\pgfqpoint{2.806096in}{1.248292in}}{\pgfqpoint{2.803901in}{1.242993in}}{\pgfqpoint{2.803901in}{1.237468in}}%
\pgfpathcurveto{\pgfqpoint{2.803901in}{1.231943in}}{\pgfqpoint{2.806096in}{1.226643in}}{\pgfqpoint{2.810003in}{1.222736in}}%
\pgfpathcurveto{\pgfqpoint{2.813910in}{1.218830in}}{\pgfqpoint{2.819209in}{1.216635in}}{\pgfqpoint{2.824734in}{1.216635in}}%
\pgfpathclose%
\pgfusepath{stroke,fill}%
\end{pgfscope}%
\begin{pgfscope}%
\pgfpathrectangle{\pgfqpoint{0.562500in}{0.275000in}}{\pgfqpoint{3.487500in}{1.925000in}}%
\pgfusepath{clip}%
\pgfsetbuttcap%
\pgfsetroundjoin%
\definecolor{currentfill}{rgb}{0.000000,0.000000,0.000000}%
\pgfsetfillcolor{currentfill}%
\pgfsetlinewidth{1.003750pt}%
\definecolor{currentstroke}{rgb}{0.000000,0.000000,0.000000}%
\pgfsetstrokecolor{currentstroke}%
\pgfsetdash{}{0pt}%
\pgfpathmoveto{\pgfqpoint{2.824734in}{1.216635in}}%
\pgfpathcurveto{\pgfqpoint{2.830260in}{1.216635in}}{\pgfqpoint{2.835559in}{1.218830in}}{\pgfqpoint{2.839466in}{1.222736in}}%
\pgfpathcurveto{\pgfqpoint{2.843373in}{1.226643in}}{\pgfqpoint{2.845568in}{1.231943in}}{\pgfqpoint{2.845568in}{1.237468in}}%
\pgfpathcurveto{\pgfqpoint{2.845568in}{1.242993in}}{\pgfqpoint{2.843373in}{1.248292in}}{\pgfqpoint{2.839466in}{1.252199in}}%
\pgfpathcurveto{\pgfqpoint{2.835559in}{1.256106in}}{\pgfqpoint{2.830260in}{1.258301in}}{\pgfqpoint{2.824734in}{1.258301in}}%
\pgfpathcurveto{\pgfqpoint{2.819209in}{1.258301in}}{\pgfqpoint{2.813910in}{1.256106in}}{\pgfqpoint{2.810003in}{1.252199in}}%
\pgfpathcurveto{\pgfqpoint{2.806096in}{1.248292in}}{\pgfqpoint{2.803901in}{1.242993in}}{\pgfqpoint{2.803901in}{1.237468in}}%
\pgfpathcurveto{\pgfqpoint{2.803901in}{1.231943in}}{\pgfqpoint{2.806096in}{1.226643in}}{\pgfqpoint{2.810003in}{1.222736in}}%
\pgfpathcurveto{\pgfqpoint{2.813910in}{1.218830in}}{\pgfqpoint{2.819209in}{1.216635in}}{\pgfqpoint{2.824734in}{1.216635in}}%
\pgfpathclose%
\pgfusepath{stroke,fill}%
\end{pgfscope}%
\begin{pgfscope}%
\pgfpathrectangle{\pgfqpoint{0.562500in}{0.275000in}}{\pgfqpoint{3.487500in}{1.925000in}}%
\pgfusepath{clip}%
\pgfsetbuttcap%
\pgfsetroundjoin%
\definecolor{currentfill}{rgb}{0.000000,0.000000,0.000000}%
\pgfsetfillcolor{currentfill}%
\pgfsetlinewidth{1.003750pt}%
\definecolor{currentstroke}{rgb}{0.000000,0.000000,0.000000}%
\pgfsetstrokecolor{currentstroke}%
\pgfsetdash{}{0pt}%
\pgfpathmoveto{\pgfqpoint{2.824734in}{1.216635in}}%
\pgfpathcurveto{\pgfqpoint{2.830260in}{1.216635in}}{\pgfqpoint{2.835559in}{1.218830in}}{\pgfqpoint{2.839466in}{1.222736in}}%
\pgfpathcurveto{\pgfqpoint{2.843373in}{1.226643in}}{\pgfqpoint{2.845568in}{1.231943in}}{\pgfqpoint{2.845568in}{1.237468in}}%
\pgfpathcurveto{\pgfqpoint{2.845568in}{1.242993in}}{\pgfqpoint{2.843373in}{1.248292in}}{\pgfqpoint{2.839466in}{1.252199in}}%
\pgfpathcurveto{\pgfqpoint{2.835559in}{1.256106in}}{\pgfqpoint{2.830260in}{1.258301in}}{\pgfqpoint{2.824734in}{1.258301in}}%
\pgfpathcurveto{\pgfqpoint{2.819209in}{1.258301in}}{\pgfqpoint{2.813910in}{1.256106in}}{\pgfqpoint{2.810003in}{1.252199in}}%
\pgfpathcurveto{\pgfqpoint{2.806096in}{1.248292in}}{\pgfqpoint{2.803901in}{1.242993in}}{\pgfqpoint{2.803901in}{1.237468in}}%
\pgfpathcurveto{\pgfqpoint{2.803901in}{1.231943in}}{\pgfqpoint{2.806096in}{1.226643in}}{\pgfqpoint{2.810003in}{1.222736in}}%
\pgfpathcurveto{\pgfqpoint{2.813910in}{1.218830in}}{\pgfqpoint{2.819209in}{1.216635in}}{\pgfqpoint{2.824734in}{1.216635in}}%
\pgfpathclose%
\pgfusepath{stroke,fill}%
\end{pgfscope}%
\begin{pgfscope}%
\pgfpathrectangle{\pgfqpoint{0.562500in}{0.275000in}}{\pgfqpoint{3.487500in}{1.925000in}}%
\pgfusepath{clip}%
\pgfsetbuttcap%
\pgfsetroundjoin%
\definecolor{currentfill}{rgb}{0.000000,0.000000,0.000000}%
\pgfsetfillcolor{currentfill}%
\pgfsetlinewidth{1.003750pt}%
\definecolor{currentstroke}{rgb}{0.000000,0.000000,0.000000}%
\pgfsetstrokecolor{currentstroke}%
\pgfsetdash{}{0pt}%
\pgfpathmoveto{\pgfqpoint{2.824734in}{1.216635in}}%
\pgfpathcurveto{\pgfqpoint{2.830260in}{1.216635in}}{\pgfqpoint{2.835559in}{1.218830in}}{\pgfqpoint{2.839466in}{1.222736in}}%
\pgfpathcurveto{\pgfqpoint{2.843373in}{1.226643in}}{\pgfqpoint{2.845568in}{1.231943in}}{\pgfqpoint{2.845568in}{1.237468in}}%
\pgfpathcurveto{\pgfqpoint{2.845568in}{1.242993in}}{\pgfqpoint{2.843373in}{1.248292in}}{\pgfqpoint{2.839466in}{1.252199in}}%
\pgfpathcurveto{\pgfqpoint{2.835559in}{1.256106in}}{\pgfqpoint{2.830260in}{1.258301in}}{\pgfqpoint{2.824734in}{1.258301in}}%
\pgfpathcurveto{\pgfqpoint{2.819209in}{1.258301in}}{\pgfqpoint{2.813910in}{1.256106in}}{\pgfqpoint{2.810003in}{1.252199in}}%
\pgfpathcurveto{\pgfqpoint{2.806096in}{1.248292in}}{\pgfqpoint{2.803901in}{1.242993in}}{\pgfqpoint{2.803901in}{1.237468in}}%
\pgfpathcurveto{\pgfqpoint{2.803901in}{1.231943in}}{\pgfqpoint{2.806096in}{1.226643in}}{\pgfqpoint{2.810003in}{1.222736in}}%
\pgfpathcurveto{\pgfqpoint{2.813910in}{1.218830in}}{\pgfqpoint{2.819209in}{1.216635in}}{\pgfqpoint{2.824734in}{1.216635in}}%
\pgfpathclose%
\pgfusepath{stroke,fill}%
\end{pgfscope}%
\begin{pgfscope}%
\pgfpathrectangle{\pgfqpoint{0.562500in}{0.275000in}}{\pgfqpoint{3.487500in}{1.925000in}}%
\pgfusepath{clip}%
\pgfsetbuttcap%
\pgfsetroundjoin%
\definecolor{currentfill}{rgb}{0.000000,0.000000,0.000000}%
\pgfsetfillcolor{currentfill}%
\pgfsetlinewidth{1.003750pt}%
\definecolor{currentstroke}{rgb}{0.000000,0.000000,0.000000}%
\pgfsetstrokecolor{currentstroke}%
\pgfsetdash{}{0pt}%
\pgfpathmoveto{\pgfqpoint{2.824734in}{1.216635in}}%
\pgfpathcurveto{\pgfqpoint{2.830260in}{1.216635in}}{\pgfqpoint{2.835559in}{1.218830in}}{\pgfqpoint{2.839466in}{1.222736in}}%
\pgfpathcurveto{\pgfqpoint{2.843373in}{1.226643in}}{\pgfqpoint{2.845568in}{1.231943in}}{\pgfqpoint{2.845568in}{1.237468in}}%
\pgfpathcurveto{\pgfqpoint{2.845568in}{1.242993in}}{\pgfqpoint{2.843373in}{1.248292in}}{\pgfqpoint{2.839466in}{1.252199in}}%
\pgfpathcurveto{\pgfqpoint{2.835559in}{1.256106in}}{\pgfqpoint{2.830260in}{1.258301in}}{\pgfqpoint{2.824734in}{1.258301in}}%
\pgfpathcurveto{\pgfqpoint{2.819209in}{1.258301in}}{\pgfqpoint{2.813910in}{1.256106in}}{\pgfqpoint{2.810003in}{1.252199in}}%
\pgfpathcurveto{\pgfqpoint{2.806096in}{1.248292in}}{\pgfqpoint{2.803901in}{1.242993in}}{\pgfqpoint{2.803901in}{1.237468in}}%
\pgfpathcurveto{\pgfqpoint{2.803901in}{1.231943in}}{\pgfqpoint{2.806096in}{1.226643in}}{\pgfqpoint{2.810003in}{1.222736in}}%
\pgfpathcurveto{\pgfqpoint{2.813910in}{1.218830in}}{\pgfqpoint{2.819209in}{1.216635in}}{\pgfqpoint{2.824734in}{1.216635in}}%
\pgfpathclose%
\pgfusepath{stroke,fill}%
\end{pgfscope}%
\begin{pgfscope}%
\pgfpathrectangle{\pgfqpoint{0.562500in}{0.275000in}}{\pgfqpoint{3.487500in}{1.925000in}}%
\pgfusepath{clip}%
\pgfsetbuttcap%
\pgfsetroundjoin%
\definecolor{currentfill}{rgb}{0.000000,0.000000,0.000000}%
\pgfsetfillcolor{currentfill}%
\pgfsetlinewidth{1.003750pt}%
\definecolor{currentstroke}{rgb}{0.000000,0.000000,0.000000}%
\pgfsetstrokecolor{currentstroke}%
\pgfsetdash{}{0pt}%
\pgfpathmoveto{\pgfqpoint{2.824734in}{1.216635in}}%
\pgfpathcurveto{\pgfqpoint{2.830260in}{1.216635in}}{\pgfqpoint{2.835559in}{1.218830in}}{\pgfqpoint{2.839466in}{1.222736in}}%
\pgfpathcurveto{\pgfqpoint{2.843373in}{1.226643in}}{\pgfqpoint{2.845568in}{1.231943in}}{\pgfqpoint{2.845568in}{1.237468in}}%
\pgfpathcurveto{\pgfqpoint{2.845568in}{1.242993in}}{\pgfqpoint{2.843373in}{1.248292in}}{\pgfqpoint{2.839466in}{1.252199in}}%
\pgfpathcurveto{\pgfqpoint{2.835559in}{1.256106in}}{\pgfqpoint{2.830260in}{1.258301in}}{\pgfqpoint{2.824734in}{1.258301in}}%
\pgfpathcurveto{\pgfqpoint{2.819209in}{1.258301in}}{\pgfqpoint{2.813910in}{1.256106in}}{\pgfqpoint{2.810003in}{1.252199in}}%
\pgfpathcurveto{\pgfqpoint{2.806096in}{1.248292in}}{\pgfqpoint{2.803901in}{1.242993in}}{\pgfqpoint{2.803901in}{1.237468in}}%
\pgfpathcurveto{\pgfqpoint{2.803901in}{1.231943in}}{\pgfqpoint{2.806096in}{1.226643in}}{\pgfqpoint{2.810003in}{1.222736in}}%
\pgfpathcurveto{\pgfqpoint{2.813910in}{1.218830in}}{\pgfqpoint{2.819209in}{1.216635in}}{\pgfqpoint{2.824734in}{1.216635in}}%
\pgfpathclose%
\pgfusepath{stroke,fill}%
\end{pgfscope}%
\begin{pgfscope}%
\pgfpathrectangle{\pgfqpoint{0.562500in}{0.275000in}}{\pgfqpoint{3.487500in}{1.925000in}}%
\pgfusepath{clip}%
\pgfsetbuttcap%
\pgfsetroundjoin%
\definecolor{currentfill}{rgb}{0.000000,0.000000,0.000000}%
\pgfsetfillcolor{currentfill}%
\pgfsetlinewidth{1.003750pt}%
\definecolor{currentstroke}{rgb}{0.000000,0.000000,0.000000}%
\pgfsetstrokecolor{currentstroke}%
\pgfsetdash{}{0pt}%
\pgfpathmoveto{\pgfqpoint{2.824734in}{1.216635in}}%
\pgfpathcurveto{\pgfqpoint{2.830260in}{1.216635in}}{\pgfqpoint{2.835559in}{1.218830in}}{\pgfqpoint{2.839466in}{1.222736in}}%
\pgfpathcurveto{\pgfqpoint{2.843373in}{1.226643in}}{\pgfqpoint{2.845568in}{1.231943in}}{\pgfqpoint{2.845568in}{1.237468in}}%
\pgfpathcurveto{\pgfqpoint{2.845568in}{1.242993in}}{\pgfqpoint{2.843373in}{1.248292in}}{\pgfqpoint{2.839466in}{1.252199in}}%
\pgfpathcurveto{\pgfqpoint{2.835559in}{1.256106in}}{\pgfqpoint{2.830260in}{1.258301in}}{\pgfqpoint{2.824734in}{1.258301in}}%
\pgfpathcurveto{\pgfqpoint{2.819209in}{1.258301in}}{\pgfqpoint{2.813910in}{1.256106in}}{\pgfqpoint{2.810003in}{1.252199in}}%
\pgfpathcurveto{\pgfqpoint{2.806096in}{1.248292in}}{\pgfqpoint{2.803901in}{1.242993in}}{\pgfqpoint{2.803901in}{1.237468in}}%
\pgfpathcurveto{\pgfqpoint{2.803901in}{1.231943in}}{\pgfqpoint{2.806096in}{1.226643in}}{\pgfqpoint{2.810003in}{1.222736in}}%
\pgfpathcurveto{\pgfqpoint{2.813910in}{1.218830in}}{\pgfqpoint{2.819209in}{1.216635in}}{\pgfqpoint{2.824734in}{1.216635in}}%
\pgfpathclose%
\pgfusepath{stroke,fill}%
\end{pgfscope}%
\begin{pgfscope}%
\pgfpathrectangle{\pgfqpoint{0.562500in}{0.275000in}}{\pgfqpoint{3.487500in}{1.925000in}}%
\pgfusepath{clip}%
\pgfsetbuttcap%
\pgfsetroundjoin%
\definecolor{currentfill}{rgb}{0.000000,0.000000,0.000000}%
\pgfsetfillcolor{currentfill}%
\pgfsetlinewidth{1.003750pt}%
\definecolor{currentstroke}{rgb}{0.000000,0.000000,0.000000}%
\pgfsetstrokecolor{currentstroke}%
\pgfsetdash{}{0pt}%
\pgfpathmoveto{\pgfqpoint{2.824734in}{1.216635in}}%
\pgfpathcurveto{\pgfqpoint{2.830260in}{1.216635in}}{\pgfqpoint{2.835559in}{1.218830in}}{\pgfqpoint{2.839466in}{1.222736in}}%
\pgfpathcurveto{\pgfqpoint{2.843373in}{1.226643in}}{\pgfqpoint{2.845568in}{1.231943in}}{\pgfqpoint{2.845568in}{1.237468in}}%
\pgfpathcurveto{\pgfqpoint{2.845568in}{1.242993in}}{\pgfqpoint{2.843373in}{1.248292in}}{\pgfqpoint{2.839466in}{1.252199in}}%
\pgfpathcurveto{\pgfqpoint{2.835559in}{1.256106in}}{\pgfqpoint{2.830260in}{1.258301in}}{\pgfqpoint{2.824734in}{1.258301in}}%
\pgfpathcurveto{\pgfqpoint{2.819209in}{1.258301in}}{\pgfqpoint{2.813910in}{1.256106in}}{\pgfqpoint{2.810003in}{1.252199in}}%
\pgfpathcurveto{\pgfqpoint{2.806096in}{1.248292in}}{\pgfqpoint{2.803901in}{1.242993in}}{\pgfqpoint{2.803901in}{1.237468in}}%
\pgfpathcurveto{\pgfqpoint{2.803901in}{1.231943in}}{\pgfqpoint{2.806096in}{1.226643in}}{\pgfqpoint{2.810003in}{1.222736in}}%
\pgfpathcurveto{\pgfqpoint{2.813910in}{1.218830in}}{\pgfqpoint{2.819209in}{1.216635in}}{\pgfqpoint{2.824734in}{1.216635in}}%
\pgfpathclose%
\pgfusepath{stroke,fill}%
\end{pgfscope}%
\begin{pgfscope}%
\pgfpathrectangle{\pgfqpoint{0.562500in}{0.275000in}}{\pgfqpoint{3.487500in}{1.925000in}}%
\pgfusepath{clip}%
\pgfsetbuttcap%
\pgfsetroundjoin%
\definecolor{currentfill}{rgb}{0.000000,0.000000,0.000000}%
\pgfsetfillcolor{currentfill}%
\pgfsetlinewidth{1.003750pt}%
\definecolor{currentstroke}{rgb}{0.000000,0.000000,0.000000}%
\pgfsetstrokecolor{currentstroke}%
\pgfsetdash{}{0pt}%
\pgfpathmoveto{\pgfqpoint{2.824734in}{1.216635in}}%
\pgfpathcurveto{\pgfqpoint{2.830260in}{1.216635in}}{\pgfqpoint{2.835559in}{1.218830in}}{\pgfqpoint{2.839466in}{1.222736in}}%
\pgfpathcurveto{\pgfqpoint{2.843373in}{1.226643in}}{\pgfqpoint{2.845568in}{1.231943in}}{\pgfqpoint{2.845568in}{1.237468in}}%
\pgfpathcurveto{\pgfqpoint{2.845568in}{1.242993in}}{\pgfqpoint{2.843373in}{1.248292in}}{\pgfqpoint{2.839466in}{1.252199in}}%
\pgfpathcurveto{\pgfqpoint{2.835559in}{1.256106in}}{\pgfqpoint{2.830260in}{1.258301in}}{\pgfqpoint{2.824734in}{1.258301in}}%
\pgfpathcurveto{\pgfqpoint{2.819209in}{1.258301in}}{\pgfqpoint{2.813910in}{1.256106in}}{\pgfqpoint{2.810003in}{1.252199in}}%
\pgfpathcurveto{\pgfqpoint{2.806096in}{1.248292in}}{\pgfqpoint{2.803901in}{1.242993in}}{\pgfqpoint{2.803901in}{1.237468in}}%
\pgfpathcurveto{\pgfqpoint{2.803901in}{1.231943in}}{\pgfqpoint{2.806096in}{1.226643in}}{\pgfqpoint{2.810003in}{1.222736in}}%
\pgfpathcurveto{\pgfqpoint{2.813910in}{1.218830in}}{\pgfqpoint{2.819209in}{1.216635in}}{\pgfqpoint{2.824734in}{1.216635in}}%
\pgfpathclose%
\pgfusepath{stroke,fill}%
\end{pgfscope}%
\begin{pgfscope}%
\pgfpathrectangle{\pgfqpoint{0.562500in}{0.275000in}}{\pgfqpoint{3.487500in}{1.925000in}}%
\pgfusepath{clip}%
\pgfsetbuttcap%
\pgfsetroundjoin%
\definecolor{currentfill}{rgb}{0.000000,0.000000,0.000000}%
\pgfsetfillcolor{currentfill}%
\pgfsetlinewidth{1.003750pt}%
\definecolor{currentstroke}{rgb}{0.000000,0.000000,0.000000}%
\pgfsetstrokecolor{currentstroke}%
\pgfsetdash{}{0pt}%
\pgfpathmoveto{\pgfqpoint{2.824734in}{1.216635in}}%
\pgfpathcurveto{\pgfqpoint{2.830260in}{1.216635in}}{\pgfqpoint{2.835559in}{1.218830in}}{\pgfqpoint{2.839466in}{1.222736in}}%
\pgfpathcurveto{\pgfqpoint{2.843373in}{1.226643in}}{\pgfqpoint{2.845568in}{1.231943in}}{\pgfqpoint{2.845568in}{1.237468in}}%
\pgfpathcurveto{\pgfqpoint{2.845568in}{1.242993in}}{\pgfqpoint{2.843373in}{1.248292in}}{\pgfqpoint{2.839466in}{1.252199in}}%
\pgfpathcurveto{\pgfqpoint{2.835559in}{1.256106in}}{\pgfqpoint{2.830260in}{1.258301in}}{\pgfqpoint{2.824734in}{1.258301in}}%
\pgfpathcurveto{\pgfqpoint{2.819209in}{1.258301in}}{\pgfqpoint{2.813910in}{1.256106in}}{\pgfqpoint{2.810003in}{1.252199in}}%
\pgfpathcurveto{\pgfqpoint{2.806096in}{1.248292in}}{\pgfqpoint{2.803901in}{1.242993in}}{\pgfqpoint{2.803901in}{1.237468in}}%
\pgfpathcurveto{\pgfqpoint{2.803901in}{1.231943in}}{\pgfqpoint{2.806096in}{1.226643in}}{\pgfqpoint{2.810003in}{1.222736in}}%
\pgfpathcurveto{\pgfqpoint{2.813910in}{1.218830in}}{\pgfqpoint{2.819209in}{1.216635in}}{\pgfqpoint{2.824734in}{1.216635in}}%
\pgfpathclose%
\pgfusepath{stroke,fill}%
\end{pgfscope}%
\begin{pgfscope}%
\pgfpathrectangle{\pgfqpoint{0.562500in}{0.275000in}}{\pgfqpoint{3.487500in}{1.925000in}}%
\pgfusepath{clip}%
\pgfsetbuttcap%
\pgfsetroundjoin%
\definecolor{currentfill}{rgb}{0.000000,0.000000,0.000000}%
\pgfsetfillcolor{currentfill}%
\pgfsetlinewidth{1.003750pt}%
\definecolor{currentstroke}{rgb}{0.000000,0.000000,0.000000}%
\pgfsetstrokecolor{currentstroke}%
\pgfsetdash{}{0pt}%
\pgfpathmoveto{\pgfqpoint{2.824734in}{1.216635in}}%
\pgfpathcurveto{\pgfqpoint{2.830260in}{1.216635in}}{\pgfqpoint{2.835559in}{1.218830in}}{\pgfqpoint{2.839466in}{1.222736in}}%
\pgfpathcurveto{\pgfqpoint{2.843373in}{1.226643in}}{\pgfqpoint{2.845568in}{1.231943in}}{\pgfqpoint{2.845568in}{1.237468in}}%
\pgfpathcurveto{\pgfqpoint{2.845568in}{1.242993in}}{\pgfqpoint{2.843373in}{1.248292in}}{\pgfqpoint{2.839466in}{1.252199in}}%
\pgfpathcurveto{\pgfqpoint{2.835559in}{1.256106in}}{\pgfqpoint{2.830260in}{1.258301in}}{\pgfqpoint{2.824734in}{1.258301in}}%
\pgfpathcurveto{\pgfqpoint{2.819209in}{1.258301in}}{\pgfqpoint{2.813910in}{1.256106in}}{\pgfqpoint{2.810003in}{1.252199in}}%
\pgfpathcurveto{\pgfqpoint{2.806096in}{1.248292in}}{\pgfqpoint{2.803901in}{1.242993in}}{\pgfqpoint{2.803901in}{1.237468in}}%
\pgfpathcurveto{\pgfqpoint{2.803901in}{1.231943in}}{\pgfqpoint{2.806096in}{1.226643in}}{\pgfqpoint{2.810003in}{1.222736in}}%
\pgfpathcurveto{\pgfqpoint{2.813910in}{1.218830in}}{\pgfqpoint{2.819209in}{1.216635in}}{\pgfqpoint{2.824734in}{1.216635in}}%
\pgfpathclose%
\pgfusepath{stroke,fill}%
\end{pgfscope}%
\begin{pgfscope}%
\pgfpathrectangle{\pgfqpoint{0.562500in}{0.275000in}}{\pgfqpoint{3.487500in}{1.925000in}}%
\pgfusepath{clip}%
\pgfsetbuttcap%
\pgfsetroundjoin%
\definecolor{currentfill}{rgb}{0.000000,0.000000,0.000000}%
\pgfsetfillcolor{currentfill}%
\pgfsetlinewidth{1.003750pt}%
\definecolor{currentstroke}{rgb}{0.000000,0.000000,0.000000}%
\pgfsetstrokecolor{currentstroke}%
\pgfsetdash{}{0pt}%
\pgfpathmoveto{\pgfqpoint{2.824734in}{2.076667in}}%
\pgfpathcurveto{\pgfqpoint{2.830260in}{2.076667in}}{\pgfqpoint{2.835559in}{2.078862in}}{\pgfqpoint{2.839466in}{2.082769in}}%
\pgfpathcurveto{\pgfqpoint{2.843373in}{2.086675in}}{\pgfqpoint{2.845568in}{2.091975in}}{\pgfqpoint{2.845568in}{2.097500in}}%
\pgfpathcurveto{\pgfqpoint{2.845568in}{2.103025in}}{\pgfqpoint{2.843373in}{2.108325in}}{\pgfqpoint{2.839466in}{2.112231in}}%
\pgfpathcurveto{\pgfqpoint{2.835559in}{2.116138in}}{\pgfqpoint{2.830260in}{2.118333in}}{\pgfqpoint{2.824734in}{2.118333in}}%
\pgfpathcurveto{\pgfqpoint{2.819209in}{2.118333in}}{\pgfqpoint{2.813910in}{2.116138in}}{\pgfqpoint{2.810003in}{2.112231in}}%
\pgfpathcurveto{\pgfqpoint{2.806096in}{2.108325in}}{\pgfqpoint{2.803901in}{2.103025in}}{\pgfqpoint{2.803901in}{2.097500in}}%
\pgfpathcurveto{\pgfqpoint{2.803901in}{2.091975in}}{\pgfqpoint{2.806096in}{2.086675in}}{\pgfqpoint{2.810003in}{2.082769in}}%
\pgfpathcurveto{\pgfqpoint{2.813910in}{2.078862in}}{\pgfqpoint{2.819209in}{2.076667in}}{\pgfqpoint{2.824734in}{2.076667in}}%
\pgfpathclose%
\pgfusepath{stroke,fill}%
\end{pgfscope}%
\begin{pgfscope}%
\pgfpathrectangle{\pgfqpoint{0.562500in}{0.275000in}}{\pgfqpoint{3.487500in}{1.925000in}}%
\pgfusepath{clip}%
\pgfsetbuttcap%
\pgfsetroundjoin%
\definecolor{currentfill}{rgb}{0.000000,0.000000,0.000000}%
\pgfsetfillcolor{currentfill}%
\pgfsetlinewidth{1.003750pt}%
\definecolor{currentstroke}{rgb}{0.000000,0.000000,0.000000}%
\pgfsetstrokecolor{currentstroke}%
\pgfsetdash{}{0pt}%
\pgfpathmoveto{\pgfqpoint{2.824734in}{1.216635in}}%
\pgfpathcurveto{\pgfqpoint{2.830260in}{1.216635in}}{\pgfqpoint{2.835559in}{1.218830in}}{\pgfqpoint{2.839466in}{1.222736in}}%
\pgfpathcurveto{\pgfqpoint{2.843373in}{1.226643in}}{\pgfqpoint{2.845568in}{1.231943in}}{\pgfqpoint{2.845568in}{1.237468in}}%
\pgfpathcurveto{\pgfqpoint{2.845568in}{1.242993in}}{\pgfqpoint{2.843373in}{1.248292in}}{\pgfqpoint{2.839466in}{1.252199in}}%
\pgfpathcurveto{\pgfqpoint{2.835559in}{1.256106in}}{\pgfqpoint{2.830260in}{1.258301in}}{\pgfqpoint{2.824734in}{1.258301in}}%
\pgfpathcurveto{\pgfqpoint{2.819209in}{1.258301in}}{\pgfqpoint{2.813910in}{1.256106in}}{\pgfqpoint{2.810003in}{1.252199in}}%
\pgfpathcurveto{\pgfqpoint{2.806096in}{1.248292in}}{\pgfqpoint{2.803901in}{1.242993in}}{\pgfqpoint{2.803901in}{1.237468in}}%
\pgfpathcurveto{\pgfqpoint{2.803901in}{1.231943in}}{\pgfqpoint{2.806096in}{1.226643in}}{\pgfqpoint{2.810003in}{1.222736in}}%
\pgfpathcurveto{\pgfqpoint{2.813910in}{1.218830in}}{\pgfqpoint{2.819209in}{1.216635in}}{\pgfqpoint{2.824734in}{1.216635in}}%
\pgfpathclose%
\pgfusepath{stroke,fill}%
\end{pgfscope}%
\begin{pgfscope}%
\pgfpathrectangle{\pgfqpoint{0.562500in}{0.275000in}}{\pgfqpoint{3.487500in}{1.925000in}}%
\pgfusepath{clip}%
\pgfsetbuttcap%
\pgfsetroundjoin%
\definecolor{currentfill}{rgb}{0.000000,0.000000,0.000000}%
\pgfsetfillcolor{currentfill}%
\pgfsetlinewidth{1.003750pt}%
\definecolor{currentstroke}{rgb}{0.000000,0.000000,0.000000}%
\pgfsetstrokecolor{currentstroke}%
\pgfsetdash{}{0pt}%
\pgfpathmoveto{\pgfqpoint{2.824734in}{1.216635in}}%
\pgfpathcurveto{\pgfqpoint{2.830260in}{1.216635in}}{\pgfqpoint{2.835559in}{1.218830in}}{\pgfqpoint{2.839466in}{1.222736in}}%
\pgfpathcurveto{\pgfqpoint{2.843373in}{1.226643in}}{\pgfqpoint{2.845568in}{1.231943in}}{\pgfqpoint{2.845568in}{1.237468in}}%
\pgfpathcurveto{\pgfqpoint{2.845568in}{1.242993in}}{\pgfqpoint{2.843373in}{1.248292in}}{\pgfqpoint{2.839466in}{1.252199in}}%
\pgfpathcurveto{\pgfqpoint{2.835559in}{1.256106in}}{\pgfqpoint{2.830260in}{1.258301in}}{\pgfqpoint{2.824734in}{1.258301in}}%
\pgfpathcurveto{\pgfqpoint{2.819209in}{1.258301in}}{\pgfqpoint{2.813910in}{1.256106in}}{\pgfqpoint{2.810003in}{1.252199in}}%
\pgfpathcurveto{\pgfqpoint{2.806096in}{1.248292in}}{\pgfqpoint{2.803901in}{1.242993in}}{\pgfqpoint{2.803901in}{1.237468in}}%
\pgfpathcurveto{\pgfqpoint{2.803901in}{1.231943in}}{\pgfqpoint{2.806096in}{1.226643in}}{\pgfqpoint{2.810003in}{1.222736in}}%
\pgfpathcurveto{\pgfqpoint{2.813910in}{1.218830in}}{\pgfqpoint{2.819209in}{1.216635in}}{\pgfqpoint{2.824734in}{1.216635in}}%
\pgfpathclose%
\pgfusepath{stroke,fill}%
\end{pgfscope}%
\begin{pgfscope}%
\pgfpathrectangle{\pgfqpoint{0.562500in}{0.275000in}}{\pgfqpoint{3.487500in}{1.925000in}}%
\pgfusepath{clip}%
\pgfsetbuttcap%
\pgfsetroundjoin%
\definecolor{currentfill}{rgb}{0.000000,0.000000,0.000000}%
\pgfsetfillcolor{currentfill}%
\pgfsetlinewidth{1.003750pt}%
\definecolor{currentstroke}{rgb}{0.000000,0.000000,0.000000}%
\pgfsetstrokecolor{currentstroke}%
\pgfsetdash{}{0pt}%
\pgfpathmoveto{\pgfqpoint{2.824734in}{2.076667in}}%
\pgfpathcurveto{\pgfqpoint{2.830260in}{2.076667in}}{\pgfqpoint{2.835559in}{2.078862in}}{\pgfqpoint{2.839466in}{2.082769in}}%
\pgfpathcurveto{\pgfqpoint{2.843373in}{2.086675in}}{\pgfqpoint{2.845568in}{2.091975in}}{\pgfqpoint{2.845568in}{2.097500in}}%
\pgfpathcurveto{\pgfqpoint{2.845568in}{2.103025in}}{\pgfqpoint{2.843373in}{2.108325in}}{\pgfqpoint{2.839466in}{2.112231in}}%
\pgfpathcurveto{\pgfqpoint{2.835559in}{2.116138in}}{\pgfqpoint{2.830260in}{2.118333in}}{\pgfqpoint{2.824734in}{2.118333in}}%
\pgfpathcurveto{\pgfqpoint{2.819209in}{2.118333in}}{\pgfqpoint{2.813910in}{2.116138in}}{\pgfqpoint{2.810003in}{2.112231in}}%
\pgfpathcurveto{\pgfqpoint{2.806096in}{2.108325in}}{\pgfqpoint{2.803901in}{2.103025in}}{\pgfqpoint{2.803901in}{2.097500in}}%
\pgfpathcurveto{\pgfqpoint{2.803901in}{2.091975in}}{\pgfqpoint{2.806096in}{2.086675in}}{\pgfqpoint{2.810003in}{2.082769in}}%
\pgfpathcurveto{\pgfqpoint{2.813910in}{2.078862in}}{\pgfqpoint{2.819209in}{2.076667in}}{\pgfqpoint{2.824734in}{2.076667in}}%
\pgfpathclose%
\pgfusepath{stroke,fill}%
\end{pgfscope}%
\begin{pgfscope}%
\pgfpathrectangle{\pgfqpoint{0.562500in}{0.275000in}}{\pgfqpoint{3.487500in}{1.925000in}}%
\pgfusepath{clip}%
\pgfsetbuttcap%
\pgfsetroundjoin%
\definecolor{currentfill}{rgb}{0.000000,0.000000,0.000000}%
\pgfsetfillcolor{currentfill}%
\pgfsetlinewidth{1.003750pt}%
\definecolor{currentstroke}{rgb}{0.000000,0.000000,0.000000}%
\pgfsetstrokecolor{currentstroke}%
\pgfsetdash{}{0pt}%
\pgfpathmoveto{\pgfqpoint{2.824734in}{1.216635in}}%
\pgfpathcurveto{\pgfqpoint{2.830260in}{1.216635in}}{\pgfqpoint{2.835559in}{1.218830in}}{\pgfqpoint{2.839466in}{1.222736in}}%
\pgfpathcurveto{\pgfqpoint{2.843373in}{1.226643in}}{\pgfqpoint{2.845568in}{1.231943in}}{\pgfqpoint{2.845568in}{1.237468in}}%
\pgfpathcurveto{\pgfqpoint{2.845568in}{1.242993in}}{\pgfqpoint{2.843373in}{1.248292in}}{\pgfqpoint{2.839466in}{1.252199in}}%
\pgfpathcurveto{\pgfqpoint{2.835559in}{1.256106in}}{\pgfqpoint{2.830260in}{1.258301in}}{\pgfqpoint{2.824734in}{1.258301in}}%
\pgfpathcurveto{\pgfqpoint{2.819209in}{1.258301in}}{\pgfqpoint{2.813910in}{1.256106in}}{\pgfqpoint{2.810003in}{1.252199in}}%
\pgfpathcurveto{\pgfqpoint{2.806096in}{1.248292in}}{\pgfqpoint{2.803901in}{1.242993in}}{\pgfqpoint{2.803901in}{1.237468in}}%
\pgfpathcurveto{\pgfqpoint{2.803901in}{1.231943in}}{\pgfqpoint{2.806096in}{1.226643in}}{\pgfqpoint{2.810003in}{1.222736in}}%
\pgfpathcurveto{\pgfqpoint{2.813910in}{1.218830in}}{\pgfqpoint{2.819209in}{1.216635in}}{\pgfqpoint{2.824734in}{1.216635in}}%
\pgfpathclose%
\pgfusepath{stroke,fill}%
\end{pgfscope}%
\begin{pgfscope}%
\pgfpathrectangle{\pgfqpoint{0.562500in}{0.275000in}}{\pgfqpoint{3.487500in}{1.925000in}}%
\pgfusepath{clip}%
\pgfsetbuttcap%
\pgfsetroundjoin%
\definecolor{currentfill}{rgb}{0.000000,0.000000,0.000000}%
\pgfsetfillcolor{currentfill}%
\pgfsetlinewidth{1.003750pt}%
\definecolor{currentstroke}{rgb}{0.000000,0.000000,0.000000}%
\pgfsetstrokecolor{currentstroke}%
\pgfsetdash{}{0pt}%
\pgfpathmoveto{\pgfqpoint{2.824734in}{1.216635in}}%
\pgfpathcurveto{\pgfqpoint{2.830260in}{1.216635in}}{\pgfqpoint{2.835559in}{1.218830in}}{\pgfqpoint{2.839466in}{1.222736in}}%
\pgfpathcurveto{\pgfqpoint{2.843373in}{1.226643in}}{\pgfqpoint{2.845568in}{1.231943in}}{\pgfqpoint{2.845568in}{1.237468in}}%
\pgfpathcurveto{\pgfqpoint{2.845568in}{1.242993in}}{\pgfqpoint{2.843373in}{1.248292in}}{\pgfqpoint{2.839466in}{1.252199in}}%
\pgfpathcurveto{\pgfqpoint{2.835559in}{1.256106in}}{\pgfqpoint{2.830260in}{1.258301in}}{\pgfqpoint{2.824734in}{1.258301in}}%
\pgfpathcurveto{\pgfqpoint{2.819209in}{1.258301in}}{\pgfqpoint{2.813910in}{1.256106in}}{\pgfqpoint{2.810003in}{1.252199in}}%
\pgfpathcurveto{\pgfqpoint{2.806096in}{1.248292in}}{\pgfqpoint{2.803901in}{1.242993in}}{\pgfqpoint{2.803901in}{1.237468in}}%
\pgfpathcurveto{\pgfqpoint{2.803901in}{1.231943in}}{\pgfqpoint{2.806096in}{1.226643in}}{\pgfqpoint{2.810003in}{1.222736in}}%
\pgfpathcurveto{\pgfqpoint{2.813910in}{1.218830in}}{\pgfqpoint{2.819209in}{1.216635in}}{\pgfqpoint{2.824734in}{1.216635in}}%
\pgfpathclose%
\pgfusepath{stroke,fill}%
\end{pgfscope}%
\begin{pgfscope}%
\pgfpathrectangle{\pgfqpoint{0.562500in}{0.275000in}}{\pgfqpoint{3.487500in}{1.925000in}}%
\pgfusepath{clip}%
\pgfsetbuttcap%
\pgfsetroundjoin%
\definecolor{currentfill}{rgb}{0.000000,0.000000,0.000000}%
\pgfsetfillcolor{currentfill}%
\pgfsetlinewidth{1.003750pt}%
\definecolor{currentstroke}{rgb}{0.000000,0.000000,0.000000}%
\pgfsetstrokecolor{currentstroke}%
\pgfsetdash{}{0pt}%
\pgfpathmoveto{\pgfqpoint{2.824734in}{2.076667in}}%
\pgfpathcurveto{\pgfqpoint{2.830260in}{2.076667in}}{\pgfqpoint{2.835559in}{2.078862in}}{\pgfqpoint{2.839466in}{2.082769in}}%
\pgfpathcurveto{\pgfqpoint{2.843373in}{2.086675in}}{\pgfqpoint{2.845568in}{2.091975in}}{\pgfqpoint{2.845568in}{2.097500in}}%
\pgfpathcurveto{\pgfqpoint{2.845568in}{2.103025in}}{\pgfqpoint{2.843373in}{2.108325in}}{\pgfqpoint{2.839466in}{2.112231in}}%
\pgfpathcurveto{\pgfqpoint{2.835559in}{2.116138in}}{\pgfqpoint{2.830260in}{2.118333in}}{\pgfqpoint{2.824734in}{2.118333in}}%
\pgfpathcurveto{\pgfqpoint{2.819209in}{2.118333in}}{\pgfqpoint{2.813910in}{2.116138in}}{\pgfqpoint{2.810003in}{2.112231in}}%
\pgfpathcurveto{\pgfqpoint{2.806096in}{2.108325in}}{\pgfqpoint{2.803901in}{2.103025in}}{\pgfqpoint{2.803901in}{2.097500in}}%
\pgfpathcurveto{\pgfqpoint{2.803901in}{2.091975in}}{\pgfqpoint{2.806096in}{2.086675in}}{\pgfqpoint{2.810003in}{2.082769in}}%
\pgfpathcurveto{\pgfqpoint{2.813910in}{2.078862in}}{\pgfqpoint{2.819209in}{2.076667in}}{\pgfqpoint{2.824734in}{2.076667in}}%
\pgfpathclose%
\pgfusepath{stroke,fill}%
\end{pgfscope}%
\begin{pgfscope}%
\pgfpathrectangle{\pgfqpoint{0.562500in}{0.275000in}}{\pgfqpoint{3.487500in}{1.925000in}}%
\pgfusepath{clip}%
\pgfsetbuttcap%
\pgfsetroundjoin%
\definecolor{currentfill}{rgb}{0.000000,0.000000,0.000000}%
\pgfsetfillcolor{currentfill}%
\pgfsetlinewidth{1.003750pt}%
\definecolor{currentstroke}{rgb}{0.000000,0.000000,0.000000}%
\pgfsetstrokecolor{currentstroke}%
\pgfsetdash{}{0pt}%
\pgfpathmoveto{\pgfqpoint{2.824734in}{1.216635in}}%
\pgfpathcurveto{\pgfqpoint{2.830260in}{1.216635in}}{\pgfqpoint{2.835559in}{1.218830in}}{\pgfqpoint{2.839466in}{1.222736in}}%
\pgfpathcurveto{\pgfqpoint{2.843373in}{1.226643in}}{\pgfqpoint{2.845568in}{1.231943in}}{\pgfqpoint{2.845568in}{1.237468in}}%
\pgfpathcurveto{\pgfqpoint{2.845568in}{1.242993in}}{\pgfqpoint{2.843373in}{1.248292in}}{\pgfqpoint{2.839466in}{1.252199in}}%
\pgfpathcurveto{\pgfqpoint{2.835559in}{1.256106in}}{\pgfqpoint{2.830260in}{1.258301in}}{\pgfqpoint{2.824734in}{1.258301in}}%
\pgfpathcurveto{\pgfqpoint{2.819209in}{1.258301in}}{\pgfqpoint{2.813910in}{1.256106in}}{\pgfqpoint{2.810003in}{1.252199in}}%
\pgfpathcurveto{\pgfqpoint{2.806096in}{1.248292in}}{\pgfqpoint{2.803901in}{1.242993in}}{\pgfqpoint{2.803901in}{1.237468in}}%
\pgfpathcurveto{\pgfqpoint{2.803901in}{1.231943in}}{\pgfqpoint{2.806096in}{1.226643in}}{\pgfqpoint{2.810003in}{1.222736in}}%
\pgfpathcurveto{\pgfqpoint{2.813910in}{1.218830in}}{\pgfqpoint{2.819209in}{1.216635in}}{\pgfqpoint{2.824734in}{1.216635in}}%
\pgfpathclose%
\pgfusepath{stroke,fill}%
\end{pgfscope}%
\begin{pgfscope}%
\pgfpathrectangle{\pgfqpoint{0.562500in}{0.275000in}}{\pgfqpoint{3.487500in}{1.925000in}}%
\pgfusepath{clip}%
\pgfsetbuttcap%
\pgfsetroundjoin%
\definecolor{currentfill}{rgb}{0.000000,0.000000,0.000000}%
\pgfsetfillcolor{currentfill}%
\pgfsetlinewidth{1.003750pt}%
\definecolor{currentstroke}{rgb}{0.000000,0.000000,0.000000}%
\pgfsetstrokecolor{currentstroke}%
\pgfsetdash{}{0pt}%
\pgfpathmoveto{\pgfqpoint{2.824734in}{1.216635in}}%
\pgfpathcurveto{\pgfqpoint{2.830260in}{1.216635in}}{\pgfqpoint{2.835559in}{1.218830in}}{\pgfqpoint{2.839466in}{1.222736in}}%
\pgfpathcurveto{\pgfqpoint{2.843373in}{1.226643in}}{\pgfqpoint{2.845568in}{1.231943in}}{\pgfqpoint{2.845568in}{1.237468in}}%
\pgfpathcurveto{\pgfqpoint{2.845568in}{1.242993in}}{\pgfqpoint{2.843373in}{1.248292in}}{\pgfqpoint{2.839466in}{1.252199in}}%
\pgfpathcurveto{\pgfqpoint{2.835559in}{1.256106in}}{\pgfqpoint{2.830260in}{1.258301in}}{\pgfqpoint{2.824734in}{1.258301in}}%
\pgfpathcurveto{\pgfqpoint{2.819209in}{1.258301in}}{\pgfqpoint{2.813910in}{1.256106in}}{\pgfqpoint{2.810003in}{1.252199in}}%
\pgfpathcurveto{\pgfqpoint{2.806096in}{1.248292in}}{\pgfqpoint{2.803901in}{1.242993in}}{\pgfqpoint{2.803901in}{1.237468in}}%
\pgfpathcurveto{\pgfqpoint{2.803901in}{1.231943in}}{\pgfqpoint{2.806096in}{1.226643in}}{\pgfqpoint{2.810003in}{1.222736in}}%
\pgfpathcurveto{\pgfqpoint{2.813910in}{1.218830in}}{\pgfqpoint{2.819209in}{1.216635in}}{\pgfqpoint{2.824734in}{1.216635in}}%
\pgfpathclose%
\pgfusepath{stroke,fill}%
\end{pgfscope}%
\begin{pgfscope}%
\pgfpathrectangle{\pgfqpoint{0.562500in}{0.275000in}}{\pgfqpoint{3.487500in}{1.925000in}}%
\pgfusepath{clip}%
\pgfsetbuttcap%
\pgfsetroundjoin%
\definecolor{currentfill}{rgb}{0.000000,0.000000,0.000000}%
\pgfsetfillcolor{currentfill}%
\pgfsetlinewidth{1.003750pt}%
\definecolor{currentstroke}{rgb}{0.000000,0.000000,0.000000}%
\pgfsetstrokecolor{currentstroke}%
\pgfsetdash{}{0pt}%
\pgfpathmoveto{\pgfqpoint{2.824734in}{2.076667in}}%
\pgfpathcurveto{\pgfqpoint{2.830260in}{2.076667in}}{\pgfqpoint{2.835559in}{2.078862in}}{\pgfqpoint{2.839466in}{2.082769in}}%
\pgfpathcurveto{\pgfqpoint{2.843373in}{2.086675in}}{\pgfqpoint{2.845568in}{2.091975in}}{\pgfqpoint{2.845568in}{2.097500in}}%
\pgfpathcurveto{\pgfqpoint{2.845568in}{2.103025in}}{\pgfqpoint{2.843373in}{2.108325in}}{\pgfqpoint{2.839466in}{2.112231in}}%
\pgfpathcurveto{\pgfqpoint{2.835559in}{2.116138in}}{\pgfqpoint{2.830260in}{2.118333in}}{\pgfqpoint{2.824734in}{2.118333in}}%
\pgfpathcurveto{\pgfqpoint{2.819209in}{2.118333in}}{\pgfqpoint{2.813910in}{2.116138in}}{\pgfqpoint{2.810003in}{2.112231in}}%
\pgfpathcurveto{\pgfqpoint{2.806096in}{2.108325in}}{\pgfqpoint{2.803901in}{2.103025in}}{\pgfqpoint{2.803901in}{2.097500in}}%
\pgfpathcurveto{\pgfqpoint{2.803901in}{2.091975in}}{\pgfqpoint{2.806096in}{2.086675in}}{\pgfqpoint{2.810003in}{2.082769in}}%
\pgfpathcurveto{\pgfqpoint{2.813910in}{2.078862in}}{\pgfqpoint{2.819209in}{2.076667in}}{\pgfqpoint{2.824734in}{2.076667in}}%
\pgfpathclose%
\pgfusepath{stroke,fill}%
\end{pgfscope}%
\begin{pgfscope}%
\pgfpathrectangle{\pgfqpoint{0.562500in}{0.275000in}}{\pgfqpoint{3.487500in}{1.925000in}}%
\pgfusepath{clip}%
\pgfsetbuttcap%
\pgfsetroundjoin%
\definecolor{currentfill}{rgb}{0.000000,0.000000,0.000000}%
\pgfsetfillcolor{currentfill}%
\pgfsetlinewidth{1.003750pt}%
\definecolor{currentstroke}{rgb}{0.000000,0.000000,0.000000}%
\pgfsetstrokecolor{currentstroke}%
\pgfsetdash{}{0pt}%
\pgfpathmoveto{\pgfqpoint{2.824734in}{1.216635in}}%
\pgfpathcurveto{\pgfqpoint{2.830260in}{1.216635in}}{\pgfqpoint{2.835559in}{1.218830in}}{\pgfqpoint{2.839466in}{1.222736in}}%
\pgfpathcurveto{\pgfqpoint{2.843373in}{1.226643in}}{\pgfqpoint{2.845568in}{1.231943in}}{\pgfqpoint{2.845568in}{1.237468in}}%
\pgfpathcurveto{\pgfqpoint{2.845568in}{1.242993in}}{\pgfqpoint{2.843373in}{1.248292in}}{\pgfqpoint{2.839466in}{1.252199in}}%
\pgfpathcurveto{\pgfqpoint{2.835559in}{1.256106in}}{\pgfqpoint{2.830260in}{1.258301in}}{\pgfqpoint{2.824734in}{1.258301in}}%
\pgfpathcurveto{\pgfqpoint{2.819209in}{1.258301in}}{\pgfqpoint{2.813910in}{1.256106in}}{\pgfqpoint{2.810003in}{1.252199in}}%
\pgfpathcurveto{\pgfqpoint{2.806096in}{1.248292in}}{\pgfqpoint{2.803901in}{1.242993in}}{\pgfqpoint{2.803901in}{1.237468in}}%
\pgfpathcurveto{\pgfqpoint{2.803901in}{1.231943in}}{\pgfqpoint{2.806096in}{1.226643in}}{\pgfqpoint{2.810003in}{1.222736in}}%
\pgfpathcurveto{\pgfqpoint{2.813910in}{1.218830in}}{\pgfqpoint{2.819209in}{1.216635in}}{\pgfqpoint{2.824734in}{1.216635in}}%
\pgfpathclose%
\pgfusepath{stroke,fill}%
\end{pgfscope}%
\begin{pgfscope}%
\pgfpathrectangle{\pgfqpoint{0.562500in}{0.275000in}}{\pgfqpoint{3.487500in}{1.925000in}}%
\pgfusepath{clip}%
\pgfsetbuttcap%
\pgfsetroundjoin%
\definecolor{currentfill}{rgb}{0.000000,0.000000,0.000000}%
\pgfsetfillcolor{currentfill}%
\pgfsetlinewidth{1.003750pt}%
\definecolor{currentstroke}{rgb}{0.000000,0.000000,0.000000}%
\pgfsetstrokecolor{currentstroke}%
\pgfsetdash{}{0pt}%
\pgfpathmoveto{\pgfqpoint{2.824734in}{1.216635in}}%
\pgfpathcurveto{\pgfqpoint{2.830260in}{1.216635in}}{\pgfqpoint{2.835559in}{1.218830in}}{\pgfqpoint{2.839466in}{1.222736in}}%
\pgfpathcurveto{\pgfqpoint{2.843373in}{1.226643in}}{\pgfqpoint{2.845568in}{1.231943in}}{\pgfqpoint{2.845568in}{1.237468in}}%
\pgfpathcurveto{\pgfqpoint{2.845568in}{1.242993in}}{\pgfqpoint{2.843373in}{1.248292in}}{\pgfqpoint{2.839466in}{1.252199in}}%
\pgfpathcurveto{\pgfqpoint{2.835559in}{1.256106in}}{\pgfqpoint{2.830260in}{1.258301in}}{\pgfqpoint{2.824734in}{1.258301in}}%
\pgfpathcurveto{\pgfqpoint{2.819209in}{1.258301in}}{\pgfqpoint{2.813910in}{1.256106in}}{\pgfqpoint{2.810003in}{1.252199in}}%
\pgfpathcurveto{\pgfqpoint{2.806096in}{1.248292in}}{\pgfqpoint{2.803901in}{1.242993in}}{\pgfqpoint{2.803901in}{1.237468in}}%
\pgfpathcurveto{\pgfqpoint{2.803901in}{1.231943in}}{\pgfqpoint{2.806096in}{1.226643in}}{\pgfqpoint{2.810003in}{1.222736in}}%
\pgfpathcurveto{\pgfqpoint{2.813910in}{1.218830in}}{\pgfqpoint{2.819209in}{1.216635in}}{\pgfqpoint{2.824734in}{1.216635in}}%
\pgfpathclose%
\pgfusepath{stroke,fill}%
\end{pgfscope}%
\begin{pgfscope}%
\pgfpathrectangle{\pgfqpoint{0.562500in}{0.275000in}}{\pgfqpoint{3.487500in}{1.925000in}}%
\pgfusepath{clip}%
\pgfsetbuttcap%
\pgfsetroundjoin%
\definecolor{currentfill}{rgb}{0.000000,0.000000,0.000000}%
\pgfsetfillcolor{currentfill}%
\pgfsetlinewidth{1.003750pt}%
\definecolor{currentstroke}{rgb}{0.000000,0.000000,0.000000}%
\pgfsetstrokecolor{currentstroke}%
\pgfsetdash{}{0pt}%
\pgfpathmoveto{\pgfqpoint{2.824734in}{1.216635in}}%
\pgfpathcurveto{\pgfqpoint{2.830260in}{1.216635in}}{\pgfqpoint{2.835559in}{1.218830in}}{\pgfqpoint{2.839466in}{1.222736in}}%
\pgfpathcurveto{\pgfqpoint{2.843373in}{1.226643in}}{\pgfqpoint{2.845568in}{1.231943in}}{\pgfqpoint{2.845568in}{1.237468in}}%
\pgfpathcurveto{\pgfqpoint{2.845568in}{1.242993in}}{\pgfqpoint{2.843373in}{1.248292in}}{\pgfqpoint{2.839466in}{1.252199in}}%
\pgfpathcurveto{\pgfqpoint{2.835559in}{1.256106in}}{\pgfqpoint{2.830260in}{1.258301in}}{\pgfqpoint{2.824734in}{1.258301in}}%
\pgfpathcurveto{\pgfqpoint{2.819209in}{1.258301in}}{\pgfqpoint{2.813910in}{1.256106in}}{\pgfqpoint{2.810003in}{1.252199in}}%
\pgfpathcurveto{\pgfqpoint{2.806096in}{1.248292in}}{\pgfqpoint{2.803901in}{1.242993in}}{\pgfqpoint{2.803901in}{1.237468in}}%
\pgfpathcurveto{\pgfqpoint{2.803901in}{1.231943in}}{\pgfqpoint{2.806096in}{1.226643in}}{\pgfqpoint{2.810003in}{1.222736in}}%
\pgfpathcurveto{\pgfqpoint{2.813910in}{1.218830in}}{\pgfqpoint{2.819209in}{1.216635in}}{\pgfqpoint{2.824734in}{1.216635in}}%
\pgfpathclose%
\pgfusepath{stroke,fill}%
\end{pgfscope}%
\begin{pgfscope}%
\pgfpathrectangle{\pgfqpoint{0.562500in}{0.275000in}}{\pgfqpoint{3.487500in}{1.925000in}}%
\pgfusepath{clip}%
\pgfsetbuttcap%
\pgfsetroundjoin%
\definecolor{currentfill}{rgb}{0.000000,0.000000,0.000000}%
\pgfsetfillcolor{currentfill}%
\pgfsetlinewidth{1.003750pt}%
\definecolor{currentstroke}{rgb}{0.000000,0.000000,0.000000}%
\pgfsetstrokecolor{currentstroke}%
\pgfsetdash{}{0pt}%
\pgfpathmoveto{\pgfqpoint{2.824734in}{2.076667in}}%
\pgfpathcurveto{\pgfqpoint{2.830260in}{2.076667in}}{\pgfqpoint{2.835559in}{2.078862in}}{\pgfqpoint{2.839466in}{2.082769in}}%
\pgfpathcurveto{\pgfqpoint{2.843373in}{2.086675in}}{\pgfqpoint{2.845568in}{2.091975in}}{\pgfqpoint{2.845568in}{2.097500in}}%
\pgfpathcurveto{\pgfqpoint{2.845568in}{2.103025in}}{\pgfqpoint{2.843373in}{2.108325in}}{\pgfqpoint{2.839466in}{2.112231in}}%
\pgfpathcurveto{\pgfqpoint{2.835559in}{2.116138in}}{\pgfqpoint{2.830260in}{2.118333in}}{\pgfqpoint{2.824734in}{2.118333in}}%
\pgfpathcurveto{\pgfqpoint{2.819209in}{2.118333in}}{\pgfqpoint{2.813910in}{2.116138in}}{\pgfqpoint{2.810003in}{2.112231in}}%
\pgfpathcurveto{\pgfqpoint{2.806096in}{2.108325in}}{\pgfqpoint{2.803901in}{2.103025in}}{\pgfqpoint{2.803901in}{2.097500in}}%
\pgfpathcurveto{\pgfqpoint{2.803901in}{2.091975in}}{\pgfqpoint{2.806096in}{2.086675in}}{\pgfqpoint{2.810003in}{2.082769in}}%
\pgfpathcurveto{\pgfqpoint{2.813910in}{2.078862in}}{\pgfqpoint{2.819209in}{2.076667in}}{\pgfqpoint{2.824734in}{2.076667in}}%
\pgfpathclose%
\pgfusepath{stroke,fill}%
\end{pgfscope}%
\begin{pgfscope}%
\pgfpathrectangle{\pgfqpoint{0.562500in}{0.275000in}}{\pgfqpoint{3.487500in}{1.925000in}}%
\pgfusepath{clip}%
\pgfsetbuttcap%
\pgfsetroundjoin%
\definecolor{currentfill}{rgb}{0.000000,0.000000,0.000000}%
\pgfsetfillcolor{currentfill}%
\pgfsetlinewidth{1.003750pt}%
\definecolor{currentstroke}{rgb}{0.000000,0.000000,0.000000}%
\pgfsetstrokecolor{currentstroke}%
\pgfsetdash{}{0pt}%
\pgfpathmoveto{\pgfqpoint{2.824734in}{1.216635in}}%
\pgfpathcurveto{\pgfqpoint{2.830260in}{1.216635in}}{\pgfqpoint{2.835559in}{1.218830in}}{\pgfqpoint{2.839466in}{1.222736in}}%
\pgfpathcurveto{\pgfqpoint{2.843373in}{1.226643in}}{\pgfqpoint{2.845568in}{1.231943in}}{\pgfqpoint{2.845568in}{1.237468in}}%
\pgfpathcurveto{\pgfqpoint{2.845568in}{1.242993in}}{\pgfqpoint{2.843373in}{1.248292in}}{\pgfqpoint{2.839466in}{1.252199in}}%
\pgfpathcurveto{\pgfqpoint{2.835559in}{1.256106in}}{\pgfqpoint{2.830260in}{1.258301in}}{\pgfqpoint{2.824734in}{1.258301in}}%
\pgfpathcurveto{\pgfqpoint{2.819209in}{1.258301in}}{\pgfqpoint{2.813910in}{1.256106in}}{\pgfqpoint{2.810003in}{1.252199in}}%
\pgfpathcurveto{\pgfqpoint{2.806096in}{1.248292in}}{\pgfqpoint{2.803901in}{1.242993in}}{\pgfqpoint{2.803901in}{1.237468in}}%
\pgfpathcurveto{\pgfqpoint{2.803901in}{1.231943in}}{\pgfqpoint{2.806096in}{1.226643in}}{\pgfqpoint{2.810003in}{1.222736in}}%
\pgfpathcurveto{\pgfqpoint{2.813910in}{1.218830in}}{\pgfqpoint{2.819209in}{1.216635in}}{\pgfqpoint{2.824734in}{1.216635in}}%
\pgfpathclose%
\pgfusepath{stroke,fill}%
\end{pgfscope}%
\begin{pgfscope}%
\pgfpathrectangle{\pgfqpoint{0.562500in}{0.275000in}}{\pgfqpoint{3.487500in}{1.925000in}}%
\pgfusepath{clip}%
\pgfsetbuttcap%
\pgfsetroundjoin%
\definecolor{currentfill}{rgb}{0.000000,0.000000,0.000000}%
\pgfsetfillcolor{currentfill}%
\pgfsetlinewidth{1.003750pt}%
\definecolor{currentstroke}{rgb}{0.000000,0.000000,0.000000}%
\pgfsetstrokecolor{currentstroke}%
\pgfsetdash{}{0pt}%
\pgfpathmoveto{\pgfqpoint{2.824734in}{1.216635in}}%
\pgfpathcurveto{\pgfqpoint{2.830260in}{1.216635in}}{\pgfqpoint{2.835559in}{1.218830in}}{\pgfqpoint{2.839466in}{1.222736in}}%
\pgfpathcurveto{\pgfqpoint{2.843373in}{1.226643in}}{\pgfqpoint{2.845568in}{1.231943in}}{\pgfqpoint{2.845568in}{1.237468in}}%
\pgfpathcurveto{\pgfqpoint{2.845568in}{1.242993in}}{\pgfqpoint{2.843373in}{1.248292in}}{\pgfqpoint{2.839466in}{1.252199in}}%
\pgfpathcurveto{\pgfqpoint{2.835559in}{1.256106in}}{\pgfqpoint{2.830260in}{1.258301in}}{\pgfqpoint{2.824734in}{1.258301in}}%
\pgfpathcurveto{\pgfqpoint{2.819209in}{1.258301in}}{\pgfqpoint{2.813910in}{1.256106in}}{\pgfqpoint{2.810003in}{1.252199in}}%
\pgfpathcurveto{\pgfqpoint{2.806096in}{1.248292in}}{\pgfqpoint{2.803901in}{1.242993in}}{\pgfqpoint{2.803901in}{1.237468in}}%
\pgfpathcurveto{\pgfqpoint{2.803901in}{1.231943in}}{\pgfqpoint{2.806096in}{1.226643in}}{\pgfqpoint{2.810003in}{1.222736in}}%
\pgfpathcurveto{\pgfqpoint{2.813910in}{1.218830in}}{\pgfqpoint{2.819209in}{1.216635in}}{\pgfqpoint{2.824734in}{1.216635in}}%
\pgfpathclose%
\pgfusepath{stroke,fill}%
\end{pgfscope}%
\begin{pgfscope}%
\pgfpathrectangle{\pgfqpoint{0.562500in}{0.275000in}}{\pgfqpoint{3.487500in}{1.925000in}}%
\pgfusepath{clip}%
\pgfsetbuttcap%
\pgfsetroundjoin%
\definecolor{currentfill}{rgb}{0.000000,0.000000,0.000000}%
\pgfsetfillcolor{currentfill}%
\pgfsetlinewidth{1.003750pt}%
\definecolor{currentstroke}{rgb}{0.000000,0.000000,0.000000}%
\pgfsetstrokecolor{currentstroke}%
\pgfsetdash{}{0pt}%
\pgfpathmoveto{\pgfqpoint{2.824734in}{1.216635in}}%
\pgfpathcurveto{\pgfqpoint{2.830260in}{1.216635in}}{\pgfqpoint{2.835559in}{1.218830in}}{\pgfqpoint{2.839466in}{1.222736in}}%
\pgfpathcurveto{\pgfqpoint{2.843373in}{1.226643in}}{\pgfqpoint{2.845568in}{1.231943in}}{\pgfqpoint{2.845568in}{1.237468in}}%
\pgfpathcurveto{\pgfqpoint{2.845568in}{1.242993in}}{\pgfqpoint{2.843373in}{1.248292in}}{\pgfqpoint{2.839466in}{1.252199in}}%
\pgfpathcurveto{\pgfqpoint{2.835559in}{1.256106in}}{\pgfqpoint{2.830260in}{1.258301in}}{\pgfqpoint{2.824734in}{1.258301in}}%
\pgfpathcurveto{\pgfqpoint{2.819209in}{1.258301in}}{\pgfqpoint{2.813910in}{1.256106in}}{\pgfqpoint{2.810003in}{1.252199in}}%
\pgfpathcurveto{\pgfqpoint{2.806096in}{1.248292in}}{\pgfqpoint{2.803901in}{1.242993in}}{\pgfqpoint{2.803901in}{1.237468in}}%
\pgfpathcurveto{\pgfqpoint{2.803901in}{1.231943in}}{\pgfqpoint{2.806096in}{1.226643in}}{\pgfqpoint{2.810003in}{1.222736in}}%
\pgfpathcurveto{\pgfqpoint{2.813910in}{1.218830in}}{\pgfqpoint{2.819209in}{1.216635in}}{\pgfqpoint{2.824734in}{1.216635in}}%
\pgfpathclose%
\pgfusepath{stroke,fill}%
\end{pgfscope}%
\begin{pgfscope}%
\pgfpathrectangle{\pgfqpoint{0.562500in}{0.275000in}}{\pgfqpoint{3.487500in}{1.925000in}}%
\pgfusepath{clip}%
\pgfsetbuttcap%
\pgfsetroundjoin%
\definecolor{currentfill}{rgb}{0.000000,0.000000,0.000000}%
\pgfsetfillcolor{currentfill}%
\pgfsetlinewidth{1.003750pt}%
\definecolor{currentstroke}{rgb}{0.000000,0.000000,0.000000}%
\pgfsetstrokecolor{currentstroke}%
\pgfsetdash{}{0pt}%
\pgfpathmoveto{\pgfqpoint{2.824734in}{1.216635in}}%
\pgfpathcurveto{\pgfqpoint{2.830260in}{1.216635in}}{\pgfqpoint{2.835559in}{1.218830in}}{\pgfqpoint{2.839466in}{1.222736in}}%
\pgfpathcurveto{\pgfqpoint{2.843373in}{1.226643in}}{\pgfqpoint{2.845568in}{1.231943in}}{\pgfqpoint{2.845568in}{1.237468in}}%
\pgfpathcurveto{\pgfqpoint{2.845568in}{1.242993in}}{\pgfqpoint{2.843373in}{1.248292in}}{\pgfqpoint{2.839466in}{1.252199in}}%
\pgfpathcurveto{\pgfqpoint{2.835559in}{1.256106in}}{\pgfqpoint{2.830260in}{1.258301in}}{\pgfqpoint{2.824734in}{1.258301in}}%
\pgfpathcurveto{\pgfqpoint{2.819209in}{1.258301in}}{\pgfqpoint{2.813910in}{1.256106in}}{\pgfqpoint{2.810003in}{1.252199in}}%
\pgfpathcurveto{\pgfqpoint{2.806096in}{1.248292in}}{\pgfqpoint{2.803901in}{1.242993in}}{\pgfqpoint{2.803901in}{1.237468in}}%
\pgfpathcurveto{\pgfqpoint{2.803901in}{1.231943in}}{\pgfqpoint{2.806096in}{1.226643in}}{\pgfqpoint{2.810003in}{1.222736in}}%
\pgfpathcurveto{\pgfqpoint{2.813910in}{1.218830in}}{\pgfqpoint{2.819209in}{1.216635in}}{\pgfqpoint{2.824734in}{1.216635in}}%
\pgfpathclose%
\pgfusepath{stroke,fill}%
\end{pgfscope}%
\begin{pgfscope}%
\pgfpathrectangle{\pgfqpoint{0.562500in}{0.275000in}}{\pgfqpoint{3.487500in}{1.925000in}}%
\pgfusepath{clip}%
\pgfsetbuttcap%
\pgfsetroundjoin%
\definecolor{currentfill}{rgb}{0.000000,0.000000,0.000000}%
\pgfsetfillcolor{currentfill}%
\pgfsetlinewidth{1.003750pt}%
\definecolor{currentstroke}{rgb}{0.000000,0.000000,0.000000}%
\pgfsetstrokecolor{currentstroke}%
\pgfsetdash{}{0pt}%
\pgfpathmoveto{\pgfqpoint{2.824734in}{1.216635in}}%
\pgfpathcurveto{\pgfqpoint{2.830260in}{1.216635in}}{\pgfqpoint{2.835559in}{1.218830in}}{\pgfqpoint{2.839466in}{1.222736in}}%
\pgfpathcurveto{\pgfqpoint{2.843373in}{1.226643in}}{\pgfqpoint{2.845568in}{1.231943in}}{\pgfqpoint{2.845568in}{1.237468in}}%
\pgfpathcurveto{\pgfqpoint{2.845568in}{1.242993in}}{\pgfqpoint{2.843373in}{1.248292in}}{\pgfqpoint{2.839466in}{1.252199in}}%
\pgfpathcurveto{\pgfqpoint{2.835559in}{1.256106in}}{\pgfqpoint{2.830260in}{1.258301in}}{\pgfqpoint{2.824734in}{1.258301in}}%
\pgfpathcurveto{\pgfqpoint{2.819209in}{1.258301in}}{\pgfqpoint{2.813910in}{1.256106in}}{\pgfqpoint{2.810003in}{1.252199in}}%
\pgfpathcurveto{\pgfqpoint{2.806096in}{1.248292in}}{\pgfqpoint{2.803901in}{1.242993in}}{\pgfqpoint{2.803901in}{1.237468in}}%
\pgfpathcurveto{\pgfqpoint{2.803901in}{1.231943in}}{\pgfqpoint{2.806096in}{1.226643in}}{\pgfqpoint{2.810003in}{1.222736in}}%
\pgfpathcurveto{\pgfqpoint{2.813910in}{1.218830in}}{\pgfqpoint{2.819209in}{1.216635in}}{\pgfqpoint{2.824734in}{1.216635in}}%
\pgfpathclose%
\pgfusepath{stroke,fill}%
\end{pgfscope}%
\begin{pgfscope}%
\pgfpathrectangle{\pgfqpoint{0.562500in}{0.275000in}}{\pgfqpoint{3.487500in}{1.925000in}}%
\pgfusepath{clip}%
\pgfsetbuttcap%
\pgfsetroundjoin%
\definecolor{currentfill}{rgb}{0.000000,0.000000,0.000000}%
\pgfsetfillcolor{currentfill}%
\pgfsetlinewidth{1.003750pt}%
\definecolor{currentstroke}{rgb}{0.000000,0.000000,0.000000}%
\pgfsetstrokecolor{currentstroke}%
\pgfsetdash{}{0pt}%
\pgfpathmoveto{\pgfqpoint{2.824734in}{1.216635in}}%
\pgfpathcurveto{\pgfqpoint{2.830260in}{1.216635in}}{\pgfqpoint{2.835559in}{1.218830in}}{\pgfqpoint{2.839466in}{1.222736in}}%
\pgfpathcurveto{\pgfqpoint{2.843373in}{1.226643in}}{\pgfqpoint{2.845568in}{1.231943in}}{\pgfqpoint{2.845568in}{1.237468in}}%
\pgfpathcurveto{\pgfqpoint{2.845568in}{1.242993in}}{\pgfqpoint{2.843373in}{1.248292in}}{\pgfqpoint{2.839466in}{1.252199in}}%
\pgfpathcurveto{\pgfqpoint{2.835559in}{1.256106in}}{\pgfqpoint{2.830260in}{1.258301in}}{\pgfqpoint{2.824734in}{1.258301in}}%
\pgfpathcurveto{\pgfqpoint{2.819209in}{1.258301in}}{\pgfqpoint{2.813910in}{1.256106in}}{\pgfqpoint{2.810003in}{1.252199in}}%
\pgfpathcurveto{\pgfqpoint{2.806096in}{1.248292in}}{\pgfqpoint{2.803901in}{1.242993in}}{\pgfqpoint{2.803901in}{1.237468in}}%
\pgfpathcurveto{\pgfqpoint{2.803901in}{1.231943in}}{\pgfqpoint{2.806096in}{1.226643in}}{\pgfqpoint{2.810003in}{1.222736in}}%
\pgfpathcurveto{\pgfqpoint{2.813910in}{1.218830in}}{\pgfqpoint{2.819209in}{1.216635in}}{\pgfqpoint{2.824734in}{1.216635in}}%
\pgfpathclose%
\pgfusepath{stroke,fill}%
\end{pgfscope}%
\begin{pgfscope}%
\pgfpathrectangle{\pgfqpoint{0.562500in}{0.275000in}}{\pgfqpoint{3.487500in}{1.925000in}}%
\pgfusepath{clip}%
\pgfsetbuttcap%
\pgfsetroundjoin%
\definecolor{currentfill}{rgb}{0.000000,0.000000,0.000000}%
\pgfsetfillcolor{currentfill}%
\pgfsetlinewidth{1.003750pt}%
\definecolor{currentstroke}{rgb}{0.000000,0.000000,0.000000}%
\pgfsetstrokecolor{currentstroke}%
\pgfsetdash{}{0pt}%
\pgfpathmoveto{\pgfqpoint{2.824734in}{1.216635in}}%
\pgfpathcurveto{\pgfqpoint{2.830260in}{1.216635in}}{\pgfqpoint{2.835559in}{1.218830in}}{\pgfqpoint{2.839466in}{1.222736in}}%
\pgfpathcurveto{\pgfqpoint{2.843373in}{1.226643in}}{\pgfqpoint{2.845568in}{1.231943in}}{\pgfqpoint{2.845568in}{1.237468in}}%
\pgfpathcurveto{\pgfqpoint{2.845568in}{1.242993in}}{\pgfqpoint{2.843373in}{1.248292in}}{\pgfqpoint{2.839466in}{1.252199in}}%
\pgfpathcurveto{\pgfqpoint{2.835559in}{1.256106in}}{\pgfqpoint{2.830260in}{1.258301in}}{\pgfqpoint{2.824734in}{1.258301in}}%
\pgfpathcurveto{\pgfqpoint{2.819209in}{1.258301in}}{\pgfqpoint{2.813910in}{1.256106in}}{\pgfqpoint{2.810003in}{1.252199in}}%
\pgfpathcurveto{\pgfqpoint{2.806096in}{1.248292in}}{\pgfqpoint{2.803901in}{1.242993in}}{\pgfqpoint{2.803901in}{1.237468in}}%
\pgfpathcurveto{\pgfqpoint{2.803901in}{1.231943in}}{\pgfqpoint{2.806096in}{1.226643in}}{\pgfqpoint{2.810003in}{1.222736in}}%
\pgfpathcurveto{\pgfqpoint{2.813910in}{1.218830in}}{\pgfqpoint{2.819209in}{1.216635in}}{\pgfqpoint{2.824734in}{1.216635in}}%
\pgfpathclose%
\pgfusepath{stroke,fill}%
\end{pgfscope}%
\begin{pgfscope}%
\pgfpathrectangle{\pgfqpoint{0.562500in}{0.275000in}}{\pgfqpoint{3.487500in}{1.925000in}}%
\pgfusepath{clip}%
\pgfsetbuttcap%
\pgfsetroundjoin%
\definecolor{currentfill}{rgb}{0.000000,0.000000,0.000000}%
\pgfsetfillcolor{currentfill}%
\pgfsetlinewidth{1.003750pt}%
\definecolor{currentstroke}{rgb}{0.000000,0.000000,0.000000}%
\pgfsetstrokecolor{currentstroke}%
\pgfsetdash{}{0pt}%
\pgfpathmoveto{\pgfqpoint{2.824734in}{2.076667in}}%
\pgfpathcurveto{\pgfqpoint{2.830260in}{2.076667in}}{\pgfqpoint{2.835559in}{2.078862in}}{\pgfqpoint{2.839466in}{2.082769in}}%
\pgfpathcurveto{\pgfqpoint{2.843373in}{2.086675in}}{\pgfqpoint{2.845568in}{2.091975in}}{\pgfqpoint{2.845568in}{2.097500in}}%
\pgfpathcurveto{\pgfqpoint{2.845568in}{2.103025in}}{\pgfqpoint{2.843373in}{2.108325in}}{\pgfqpoint{2.839466in}{2.112231in}}%
\pgfpathcurveto{\pgfqpoint{2.835559in}{2.116138in}}{\pgfqpoint{2.830260in}{2.118333in}}{\pgfqpoint{2.824734in}{2.118333in}}%
\pgfpathcurveto{\pgfqpoint{2.819209in}{2.118333in}}{\pgfqpoint{2.813910in}{2.116138in}}{\pgfqpoint{2.810003in}{2.112231in}}%
\pgfpathcurveto{\pgfqpoint{2.806096in}{2.108325in}}{\pgfqpoint{2.803901in}{2.103025in}}{\pgfqpoint{2.803901in}{2.097500in}}%
\pgfpathcurveto{\pgfqpoint{2.803901in}{2.091975in}}{\pgfqpoint{2.806096in}{2.086675in}}{\pgfqpoint{2.810003in}{2.082769in}}%
\pgfpathcurveto{\pgfqpoint{2.813910in}{2.078862in}}{\pgfqpoint{2.819209in}{2.076667in}}{\pgfqpoint{2.824734in}{2.076667in}}%
\pgfpathclose%
\pgfusepath{stroke,fill}%
\end{pgfscope}%
\begin{pgfscope}%
\pgfpathrectangle{\pgfqpoint{0.562500in}{0.275000in}}{\pgfqpoint{3.487500in}{1.925000in}}%
\pgfusepath{clip}%
\pgfsetbuttcap%
\pgfsetroundjoin%
\definecolor{currentfill}{rgb}{0.000000,0.000000,0.000000}%
\pgfsetfillcolor{currentfill}%
\pgfsetlinewidth{1.003750pt}%
\definecolor{currentstroke}{rgb}{0.000000,0.000000,0.000000}%
\pgfsetstrokecolor{currentstroke}%
\pgfsetdash{}{0pt}%
\pgfpathmoveto{\pgfqpoint{2.824734in}{1.216635in}}%
\pgfpathcurveto{\pgfqpoint{2.830260in}{1.216635in}}{\pgfqpoint{2.835559in}{1.218830in}}{\pgfqpoint{2.839466in}{1.222736in}}%
\pgfpathcurveto{\pgfqpoint{2.843373in}{1.226643in}}{\pgfqpoint{2.845568in}{1.231943in}}{\pgfqpoint{2.845568in}{1.237468in}}%
\pgfpathcurveto{\pgfqpoint{2.845568in}{1.242993in}}{\pgfqpoint{2.843373in}{1.248292in}}{\pgfqpoint{2.839466in}{1.252199in}}%
\pgfpathcurveto{\pgfqpoint{2.835559in}{1.256106in}}{\pgfqpoint{2.830260in}{1.258301in}}{\pgfqpoint{2.824734in}{1.258301in}}%
\pgfpathcurveto{\pgfqpoint{2.819209in}{1.258301in}}{\pgfqpoint{2.813910in}{1.256106in}}{\pgfqpoint{2.810003in}{1.252199in}}%
\pgfpathcurveto{\pgfqpoint{2.806096in}{1.248292in}}{\pgfqpoint{2.803901in}{1.242993in}}{\pgfqpoint{2.803901in}{1.237468in}}%
\pgfpathcurveto{\pgfqpoint{2.803901in}{1.231943in}}{\pgfqpoint{2.806096in}{1.226643in}}{\pgfqpoint{2.810003in}{1.222736in}}%
\pgfpathcurveto{\pgfqpoint{2.813910in}{1.218830in}}{\pgfqpoint{2.819209in}{1.216635in}}{\pgfqpoint{2.824734in}{1.216635in}}%
\pgfpathclose%
\pgfusepath{stroke,fill}%
\end{pgfscope}%
\begin{pgfscope}%
\pgfpathrectangle{\pgfqpoint{0.562500in}{0.275000in}}{\pgfqpoint{3.487500in}{1.925000in}}%
\pgfusepath{clip}%
\pgfsetbuttcap%
\pgfsetroundjoin%
\definecolor{currentfill}{rgb}{0.000000,0.000000,0.000000}%
\pgfsetfillcolor{currentfill}%
\pgfsetlinewidth{1.003750pt}%
\definecolor{currentstroke}{rgb}{0.000000,0.000000,0.000000}%
\pgfsetstrokecolor{currentstroke}%
\pgfsetdash{}{0pt}%
\pgfpathmoveto{\pgfqpoint{2.824734in}{1.216635in}}%
\pgfpathcurveto{\pgfqpoint{2.830260in}{1.216635in}}{\pgfqpoint{2.835559in}{1.218830in}}{\pgfqpoint{2.839466in}{1.222736in}}%
\pgfpathcurveto{\pgfqpoint{2.843373in}{1.226643in}}{\pgfqpoint{2.845568in}{1.231943in}}{\pgfqpoint{2.845568in}{1.237468in}}%
\pgfpathcurveto{\pgfqpoint{2.845568in}{1.242993in}}{\pgfqpoint{2.843373in}{1.248292in}}{\pgfqpoint{2.839466in}{1.252199in}}%
\pgfpathcurveto{\pgfqpoint{2.835559in}{1.256106in}}{\pgfqpoint{2.830260in}{1.258301in}}{\pgfqpoint{2.824734in}{1.258301in}}%
\pgfpathcurveto{\pgfqpoint{2.819209in}{1.258301in}}{\pgfqpoint{2.813910in}{1.256106in}}{\pgfqpoint{2.810003in}{1.252199in}}%
\pgfpathcurveto{\pgfqpoint{2.806096in}{1.248292in}}{\pgfqpoint{2.803901in}{1.242993in}}{\pgfqpoint{2.803901in}{1.237468in}}%
\pgfpathcurveto{\pgfqpoint{2.803901in}{1.231943in}}{\pgfqpoint{2.806096in}{1.226643in}}{\pgfqpoint{2.810003in}{1.222736in}}%
\pgfpathcurveto{\pgfqpoint{2.813910in}{1.218830in}}{\pgfqpoint{2.819209in}{1.216635in}}{\pgfqpoint{2.824734in}{1.216635in}}%
\pgfpathclose%
\pgfusepath{stroke,fill}%
\end{pgfscope}%
\begin{pgfscope}%
\pgfpathrectangle{\pgfqpoint{0.562500in}{0.275000in}}{\pgfqpoint{3.487500in}{1.925000in}}%
\pgfusepath{clip}%
\pgfsetbuttcap%
\pgfsetroundjoin%
\definecolor{currentfill}{rgb}{0.000000,0.000000,0.000000}%
\pgfsetfillcolor{currentfill}%
\pgfsetlinewidth{1.003750pt}%
\definecolor{currentstroke}{rgb}{0.000000,0.000000,0.000000}%
\pgfsetstrokecolor{currentstroke}%
\pgfsetdash{}{0pt}%
\pgfpathmoveto{\pgfqpoint{2.824734in}{1.216635in}}%
\pgfpathcurveto{\pgfqpoint{2.830260in}{1.216635in}}{\pgfqpoint{2.835559in}{1.218830in}}{\pgfqpoint{2.839466in}{1.222736in}}%
\pgfpathcurveto{\pgfqpoint{2.843373in}{1.226643in}}{\pgfqpoint{2.845568in}{1.231943in}}{\pgfqpoint{2.845568in}{1.237468in}}%
\pgfpathcurveto{\pgfqpoint{2.845568in}{1.242993in}}{\pgfqpoint{2.843373in}{1.248292in}}{\pgfqpoint{2.839466in}{1.252199in}}%
\pgfpathcurveto{\pgfqpoint{2.835559in}{1.256106in}}{\pgfqpoint{2.830260in}{1.258301in}}{\pgfqpoint{2.824734in}{1.258301in}}%
\pgfpathcurveto{\pgfqpoint{2.819209in}{1.258301in}}{\pgfqpoint{2.813910in}{1.256106in}}{\pgfqpoint{2.810003in}{1.252199in}}%
\pgfpathcurveto{\pgfqpoint{2.806096in}{1.248292in}}{\pgfqpoint{2.803901in}{1.242993in}}{\pgfqpoint{2.803901in}{1.237468in}}%
\pgfpathcurveto{\pgfqpoint{2.803901in}{1.231943in}}{\pgfqpoint{2.806096in}{1.226643in}}{\pgfqpoint{2.810003in}{1.222736in}}%
\pgfpathcurveto{\pgfqpoint{2.813910in}{1.218830in}}{\pgfqpoint{2.819209in}{1.216635in}}{\pgfqpoint{2.824734in}{1.216635in}}%
\pgfpathclose%
\pgfusepath{stroke,fill}%
\end{pgfscope}%
\begin{pgfscope}%
\pgfpathrectangle{\pgfqpoint{0.562500in}{0.275000in}}{\pgfqpoint{3.487500in}{1.925000in}}%
\pgfusepath{clip}%
\pgfsetbuttcap%
\pgfsetroundjoin%
\definecolor{currentfill}{rgb}{0.000000,0.000000,0.000000}%
\pgfsetfillcolor{currentfill}%
\pgfsetlinewidth{1.003750pt}%
\definecolor{currentstroke}{rgb}{0.000000,0.000000,0.000000}%
\pgfsetstrokecolor{currentstroke}%
\pgfsetdash{}{0pt}%
\pgfpathmoveto{\pgfqpoint{2.824734in}{1.216635in}}%
\pgfpathcurveto{\pgfqpoint{2.830260in}{1.216635in}}{\pgfqpoint{2.835559in}{1.218830in}}{\pgfqpoint{2.839466in}{1.222736in}}%
\pgfpathcurveto{\pgfqpoint{2.843373in}{1.226643in}}{\pgfqpoint{2.845568in}{1.231943in}}{\pgfqpoint{2.845568in}{1.237468in}}%
\pgfpathcurveto{\pgfqpoint{2.845568in}{1.242993in}}{\pgfqpoint{2.843373in}{1.248292in}}{\pgfqpoint{2.839466in}{1.252199in}}%
\pgfpathcurveto{\pgfqpoint{2.835559in}{1.256106in}}{\pgfqpoint{2.830260in}{1.258301in}}{\pgfqpoint{2.824734in}{1.258301in}}%
\pgfpathcurveto{\pgfqpoint{2.819209in}{1.258301in}}{\pgfqpoint{2.813910in}{1.256106in}}{\pgfqpoint{2.810003in}{1.252199in}}%
\pgfpathcurveto{\pgfqpoint{2.806096in}{1.248292in}}{\pgfqpoint{2.803901in}{1.242993in}}{\pgfqpoint{2.803901in}{1.237468in}}%
\pgfpathcurveto{\pgfqpoint{2.803901in}{1.231943in}}{\pgfqpoint{2.806096in}{1.226643in}}{\pgfqpoint{2.810003in}{1.222736in}}%
\pgfpathcurveto{\pgfqpoint{2.813910in}{1.218830in}}{\pgfqpoint{2.819209in}{1.216635in}}{\pgfqpoint{2.824734in}{1.216635in}}%
\pgfpathclose%
\pgfusepath{stroke,fill}%
\end{pgfscope}%
\begin{pgfscope}%
\pgfpathrectangle{\pgfqpoint{0.562500in}{0.275000in}}{\pgfqpoint{3.487500in}{1.925000in}}%
\pgfusepath{clip}%
\pgfsetbuttcap%
\pgfsetroundjoin%
\definecolor{currentfill}{rgb}{0.000000,0.000000,0.000000}%
\pgfsetfillcolor{currentfill}%
\pgfsetlinewidth{1.003750pt}%
\definecolor{currentstroke}{rgb}{0.000000,0.000000,0.000000}%
\pgfsetstrokecolor{currentstroke}%
\pgfsetdash{}{0pt}%
\pgfpathmoveto{\pgfqpoint{2.824734in}{1.216635in}}%
\pgfpathcurveto{\pgfqpoint{2.830260in}{1.216635in}}{\pgfqpoint{2.835559in}{1.218830in}}{\pgfqpoint{2.839466in}{1.222736in}}%
\pgfpathcurveto{\pgfqpoint{2.843373in}{1.226643in}}{\pgfqpoint{2.845568in}{1.231943in}}{\pgfqpoint{2.845568in}{1.237468in}}%
\pgfpathcurveto{\pgfqpoint{2.845568in}{1.242993in}}{\pgfqpoint{2.843373in}{1.248292in}}{\pgfqpoint{2.839466in}{1.252199in}}%
\pgfpathcurveto{\pgfqpoint{2.835559in}{1.256106in}}{\pgfqpoint{2.830260in}{1.258301in}}{\pgfqpoint{2.824734in}{1.258301in}}%
\pgfpathcurveto{\pgfqpoint{2.819209in}{1.258301in}}{\pgfqpoint{2.813910in}{1.256106in}}{\pgfqpoint{2.810003in}{1.252199in}}%
\pgfpathcurveto{\pgfqpoint{2.806096in}{1.248292in}}{\pgfqpoint{2.803901in}{1.242993in}}{\pgfqpoint{2.803901in}{1.237468in}}%
\pgfpathcurveto{\pgfqpoint{2.803901in}{1.231943in}}{\pgfqpoint{2.806096in}{1.226643in}}{\pgfqpoint{2.810003in}{1.222736in}}%
\pgfpathcurveto{\pgfqpoint{2.813910in}{1.218830in}}{\pgfqpoint{2.819209in}{1.216635in}}{\pgfqpoint{2.824734in}{1.216635in}}%
\pgfpathclose%
\pgfusepath{stroke,fill}%
\end{pgfscope}%
\begin{pgfscope}%
\pgfpathrectangle{\pgfqpoint{0.562500in}{0.275000in}}{\pgfqpoint{3.487500in}{1.925000in}}%
\pgfusepath{clip}%
\pgfsetbuttcap%
\pgfsetroundjoin%
\definecolor{currentfill}{rgb}{0.000000,0.000000,0.000000}%
\pgfsetfillcolor{currentfill}%
\pgfsetlinewidth{1.003750pt}%
\definecolor{currentstroke}{rgb}{0.000000,0.000000,0.000000}%
\pgfsetstrokecolor{currentstroke}%
\pgfsetdash{}{0pt}%
\pgfpathmoveto{\pgfqpoint{2.824734in}{1.216635in}}%
\pgfpathcurveto{\pgfqpoint{2.830260in}{1.216635in}}{\pgfqpoint{2.835559in}{1.218830in}}{\pgfqpoint{2.839466in}{1.222736in}}%
\pgfpathcurveto{\pgfqpoint{2.843373in}{1.226643in}}{\pgfqpoint{2.845568in}{1.231943in}}{\pgfqpoint{2.845568in}{1.237468in}}%
\pgfpathcurveto{\pgfqpoint{2.845568in}{1.242993in}}{\pgfqpoint{2.843373in}{1.248292in}}{\pgfqpoint{2.839466in}{1.252199in}}%
\pgfpathcurveto{\pgfqpoint{2.835559in}{1.256106in}}{\pgfqpoint{2.830260in}{1.258301in}}{\pgfqpoint{2.824734in}{1.258301in}}%
\pgfpathcurveto{\pgfqpoint{2.819209in}{1.258301in}}{\pgfqpoint{2.813910in}{1.256106in}}{\pgfqpoint{2.810003in}{1.252199in}}%
\pgfpathcurveto{\pgfqpoint{2.806096in}{1.248292in}}{\pgfqpoint{2.803901in}{1.242993in}}{\pgfqpoint{2.803901in}{1.237468in}}%
\pgfpathcurveto{\pgfqpoint{2.803901in}{1.231943in}}{\pgfqpoint{2.806096in}{1.226643in}}{\pgfqpoint{2.810003in}{1.222736in}}%
\pgfpathcurveto{\pgfqpoint{2.813910in}{1.218830in}}{\pgfqpoint{2.819209in}{1.216635in}}{\pgfqpoint{2.824734in}{1.216635in}}%
\pgfpathclose%
\pgfusepath{stroke,fill}%
\end{pgfscope}%
\begin{pgfscope}%
\pgfpathrectangle{\pgfqpoint{0.562500in}{0.275000in}}{\pgfqpoint{3.487500in}{1.925000in}}%
\pgfusepath{clip}%
\pgfsetbuttcap%
\pgfsetroundjoin%
\definecolor{currentfill}{rgb}{0.000000,0.000000,0.000000}%
\pgfsetfillcolor{currentfill}%
\pgfsetlinewidth{1.003750pt}%
\definecolor{currentstroke}{rgb}{0.000000,0.000000,0.000000}%
\pgfsetstrokecolor{currentstroke}%
\pgfsetdash{}{0pt}%
\pgfpathmoveto{\pgfqpoint{2.824734in}{1.216635in}}%
\pgfpathcurveto{\pgfqpoint{2.830260in}{1.216635in}}{\pgfqpoint{2.835559in}{1.218830in}}{\pgfqpoint{2.839466in}{1.222736in}}%
\pgfpathcurveto{\pgfqpoint{2.843373in}{1.226643in}}{\pgfqpoint{2.845568in}{1.231943in}}{\pgfqpoint{2.845568in}{1.237468in}}%
\pgfpathcurveto{\pgfqpoint{2.845568in}{1.242993in}}{\pgfqpoint{2.843373in}{1.248292in}}{\pgfqpoint{2.839466in}{1.252199in}}%
\pgfpathcurveto{\pgfqpoint{2.835559in}{1.256106in}}{\pgfqpoint{2.830260in}{1.258301in}}{\pgfqpoint{2.824734in}{1.258301in}}%
\pgfpathcurveto{\pgfqpoint{2.819209in}{1.258301in}}{\pgfqpoint{2.813910in}{1.256106in}}{\pgfqpoint{2.810003in}{1.252199in}}%
\pgfpathcurveto{\pgfqpoint{2.806096in}{1.248292in}}{\pgfqpoint{2.803901in}{1.242993in}}{\pgfqpoint{2.803901in}{1.237468in}}%
\pgfpathcurveto{\pgfqpoint{2.803901in}{1.231943in}}{\pgfqpoint{2.806096in}{1.226643in}}{\pgfqpoint{2.810003in}{1.222736in}}%
\pgfpathcurveto{\pgfqpoint{2.813910in}{1.218830in}}{\pgfqpoint{2.819209in}{1.216635in}}{\pgfqpoint{2.824734in}{1.216635in}}%
\pgfpathclose%
\pgfusepath{stroke,fill}%
\end{pgfscope}%
\begin{pgfscope}%
\pgfpathrectangle{\pgfqpoint{0.562500in}{0.275000in}}{\pgfqpoint{3.487500in}{1.925000in}}%
\pgfusepath{clip}%
\pgfsetbuttcap%
\pgfsetroundjoin%
\definecolor{currentfill}{rgb}{0.000000,0.000000,0.000000}%
\pgfsetfillcolor{currentfill}%
\pgfsetlinewidth{1.003750pt}%
\definecolor{currentstroke}{rgb}{0.000000,0.000000,0.000000}%
\pgfsetstrokecolor{currentstroke}%
\pgfsetdash{}{0pt}%
\pgfpathmoveto{\pgfqpoint{2.824734in}{2.076667in}}%
\pgfpathcurveto{\pgfqpoint{2.830260in}{2.076667in}}{\pgfqpoint{2.835559in}{2.078862in}}{\pgfqpoint{2.839466in}{2.082769in}}%
\pgfpathcurveto{\pgfqpoint{2.843373in}{2.086675in}}{\pgfqpoint{2.845568in}{2.091975in}}{\pgfqpoint{2.845568in}{2.097500in}}%
\pgfpathcurveto{\pgfqpoint{2.845568in}{2.103025in}}{\pgfqpoint{2.843373in}{2.108325in}}{\pgfqpoint{2.839466in}{2.112231in}}%
\pgfpathcurveto{\pgfqpoint{2.835559in}{2.116138in}}{\pgfqpoint{2.830260in}{2.118333in}}{\pgfqpoint{2.824734in}{2.118333in}}%
\pgfpathcurveto{\pgfqpoint{2.819209in}{2.118333in}}{\pgfqpoint{2.813910in}{2.116138in}}{\pgfqpoint{2.810003in}{2.112231in}}%
\pgfpathcurveto{\pgfqpoint{2.806096in}{2.108325in}}{\pgfqpoint{2.803901in}{2.103025in}}{\pgfqpoint{2.803901in}{2.097500in}}%
\pgfpathcurveto{\pgfqpoint{2.803901in}{2.091975in}}{\pgfqpoint{2.806096in}{2.086675in}}{\pgfqpoint{2.810003in}{2.082769in}}%
\pgfpathcurveto{\pgfqpoint{2.813910in}{2.078862in}}{\pgfqpoint{2.819209in}{2.076667in}}{\pgfqpoint{2.824734in}{2.076667in}}%
\pgfpathclose%
\pgfusepath{stroke,fill}%
\end{pgfscope}%
\begin{pgfscope}%
\pgfpathrectangle{\pgfqpoint{0.562500in}{0.275000in}}{\pgfqpoint{3.487500in}{1.925000in}}%
\pgfusepath{clip}%
\pgfsetbuttcap%
\pgfsetroundjoin%
\definecolor{currentfill}{rgb}{0.000000,0.000000,0.000000}%
\pgfsetfillcolor{currentfill}%
\pgfsetlinewidth{1.003750pt}%
\definecolor{currentstroke}{rgb}{0.000000,0.000000,0.000000}%
\pgfsetstrokecolor{currentstroke}%
\pgfsetdash{}{0pt}%
\pgfpathmoveto{\pgfqpoint{2.824734in}{1.216635in}}%
\pgfpathcurveto{\pgfqpoint{2.830260in}{1.216635in}}{\pgfqpoint{2.835559in}{1.218830in}}{\pgfqpoint{2.839466in}{1.222736in}}%
\pgfpathcurveto{\pgfqpoint{2.843373in}{1.226643in}}{\pgfqpoint{2.845568in}{1.231943in}}{\pgfqpoint{2.845568in}{1.237468in}}%
\pgfpathcurveto{\pgfqpoint{2.845568in}{1.242993in}}{\pgfqpoint{2.843373in}{1.248292in}}{\pgfqpoint{2.839466in}{1.252199in}}%
\pgfpathcurveto{\pgfqpoint{2.835559in}{1.256106in}}{\pgfqpoint{2.830260in}{1.258301in}}{\pgfqpoint{2.824734in}{1.258301in}}%
\pgfpathcurveto{\pgfqpoint{2.819209in}{1.258301in}}{\pgfqpoint{2.813910in}{1.256106in}}{\pgfqpoint{2.810003in}{1.252199in}}%
\pgfpathcurveto{\pgfqpoint{2.806096in}{1.248292in}}{\pgfqpoint{2.803901in}{1.242993in}}{\pgfqpoint{2.803901in}{1.237468in}}%
\pgfpathcurveto{\pgfqpoint{2.803901in}{1.231943in}}{\pgfqpoint{2.806096in}{1.226643in}}{\pgfqpoint{2.810003in}{1.222736in}}%
\pgfpathcurveto{\pgfqpoint{2.813910in}{1.218830in}}{\pgfqpoint{2.819209in}{1.216635in}}{\pgfqpoint{2.824734in}{1.216635in}}%
\pgfpathclose%
\pgfusepath{stroke,fill}%
\end{pgfscope}%
\begin{pgfscope}%
\pgfpathrectangle{\pgfqpoint{0.562500in}{0.275000in}}{\pgfqpoint{3.487500in}{1.925000in}}%
\pgfusepath{clip}%
\pgfsetbuttcap%
\pgfsetroundjoin%
\definecolor{currentfill}{rgb}{0.000000,0.000000,0.000000}%
\pgfsetfillcolor{currentfill}%
\pgfsetlinewidth{1.003750pt}%
\definecolor{currentstroke}{rgb}{0.000000,0.000000,0.000000}%
\pgfsetstrokecolor{currentstroke}%
\pgfsetdash{}{0pt}%
\pgfpathmoveto{\pgfqpoint{2.824734in}{1.216635in}}%
\pgfpathcurveto{\pgfqpoint{2.830260in}{1.216635in}}{\pgfqpoint{2.835559in}{1.218830in}}{\pgfqpoint{2.839466in}{1.222736in}}%
\pgfpathcurveto{\pgfqpoint{2.843373in}{1.226643in}}{\pgfqpoint{2.845568in}{1.231943in}}{\pgfqpoint{2.845568in}{1.237468in}}%
\pgfpathcurveto{\pgfqpoint{2.845568in}{1.242993in}}{\pgfqpoint{2.843373in}{1.248292in}}{\pgfqpoint{2.839466in}{1.252199in}}%
\pgfpathcurveto{\pgfqpoint{2.835559in}{1.256106in}}{\pgfqpoint{2.830260in}{1.258301in}}{\pgfqpoint{2.824734in}{1.258301in}}%
\pgfpathcurveto{\pgfqpoint{2.819209in}{1.258301in}}{\pgfqpoint{2.813910in}{1.256106in}}{\pgfqpoint{2.810003in}{1.252199in}}%
\pgfpathcurveto{\pgfqpoint{2.806096in}{1.248292in}}{\pgfqpoint{2.803901in}{1.242993in}}{\pgfqpoint{2.803901in}{1.237468in}}%
\pgfpathcurveto{\pgfqpoint{2.803901in}{1.231943in}}{\pgfqpoint{2.806096in}{1.226643in}}{\pgfqpoint{2.810003in}{1.222736in}}%
\pgfpathcurveto{\pgfqpoint{2.813910in}{1.218830in}}{\pgfqpoint{2.819209in}{1.216635in}}{\pgfqpoint{2.824734in}{1.216635in}}%
\pgfpathclose%
\pgfusepath{stroke,fill}%
\end{pgfscope}%
\begin{pgfscope}%
\pgfpathrectangle{\pgfqpoint{0.562500in}{0.275000in}}{\pgfqpoint{3.487500in}{1.925000in}}%
\pgfusepath{clip}%
\pgfsetbuttcap%
\pgfsetroundjoin%
\definecolor{currentfill}{rgb}{0.000000,0.000000,0.000000}%
\pgfsetfillcolor{currentfill}%
\pgfsetlinewidth{1.003750pt}%
\definecolor{currentstroke}{rgb}{0.000000,0.000000,0.000000}%
\pgfsetstrokecolor{currentstroke}%
\pgfsetdash{}{0pt}%
\pgfpathmoveto{\pgfqpoint{2.824734in}{1.216635in}}%
\pgfpathcurveto{\pgfqpoint{2.830260in}{1.216635in}}{\pgfqpoint{2.835559in}{1.218830in}}{\pgfqpoint{2.839466in}{1.222736in}}%
\pgfpathcurveto{\pgfqpoint{2.843373in}{1.226643in}}{\pgfqpoint{2.845568in}{1.231943in}}{\pgfqpoint{2.845568in}{1.237468in}}%
\pgfpathcurveto{\pgfqpoint{2.845568in}{1.242993in}}{\pgfqpoint{2.843373in}{1.248292in}}{\pgfqpoint{2.839466in}{1.252199in}}%
\pgfpathcurveto{\pgfqpoint{2.835559in}{1.256106in}}{\pgfqpoint{2.830260in}{1.258301in}}{\pgfqpoint{2.824734in}{1.258301in}}%
\pgfpathcurveto{\pgfqpoint{2.819209in}{1.258301in}}{\pgfqpoint{2.813910in}{1.256106in}}{\pgfqpoint{2.810003in}{1.252199in}}%
\pgfpathcurveto{\pgfqpoint{2.806096in}{1.248292in}}{\pgfqpoint{2.803901in}{1.242993in}}{\pgfqpoint{2.803901in}{1.237468in}}%
\pgfpathcurveto{\pgfqpoint{2.803901in}{1.231943in}}{\pgfqpoint{2.806096in}{1.226643in}}{\pgfqpoint{2.810003in}{1.222736in}}%
\pgfpathcurveto{\pgfqpoint{2.813910in}{1.218830in}}{\pgfqpoint{2.819209in}{1.216635in}}{\pgfqpoint{2.824734in}{1.216635in}}%
\pgfpathclose%
\pgfusepath{stroke,fill}%
\end{pgfscope}%
\begin{pgfscope}%
\pgfpathrectangle{\pgfqpoint{0.562500in}{0.275000in}}{\pgfqpoint{3.487500in}{1.925000in}}%
\pgfusepath{clip}%
\pgfsetbuttcap%
\pgfsetroundjoin%
\definecolor{currentfill}{rgb}{0.000000,0.000000,0.000000}%
\pgfsetfillcolor{currentfill}%
\pgfsetlinewidth{1.003750pt}%
\definecolor{currentstroke}{rgb}{0.000000,0.000000,0.000000}%
\pgfsetstrokecolor{currentstroke}%
\pgfsetdash{}{0pt}%
\pgfpathmoveto{\pgfqpoint{2.824734in}{2.076667in}}%
\pgfpathcurveto{\pgfqpoint{2.830260in}{2.076667in}}{\pgfqpoint{2.835559in}{2.078862in}}{\pgfqpoint{2.839466in}{2.082769in}}%
\pgfpathcurveto{\pgfqpoint{2.843373in}{2.086675in}}{\pgfqpoint{2.845568in}{2.091975in}}{\pgfqpoint{2.845568in}{2.097500in}}%
\pgfpathcurveto{\pgfqpoint{2.845568in}{2.103025in}}{\pgfqpoint{2.843373in}{2.108325in}}{\pgfqpoint{2.839466in}{2.112231in}}%
\pgfpathcurveto{\pgfqpoint{2.835559in}{2.116138in}}{\pgfqpoint{2.830260in}{2.118333in}}{\pgfqpoint{2.824734in}{2.118333in}}%
\pgfpathcurveto{\pgfqpoint{2.819209in}{2.118333in}}{\pgfqpoint{2.813910in}{2.116138in}}{\pgfqpoint{2.810003in}{2.112231in}}%
\pgfpathcurveto{\pgfqpoint{2.806096in}{2.108325in}}{\pgfqpoint{2.803901in}{2.103025in}}{\pgfqpoint{2.803901in}{2.097500in}}%
\pgfpathcurveto{\pgfqpoint{2.803901in}{2.091975in}}{\pgfqpoint{2.806096in}{2.086675in}}{\pgfqpoint{2.810003in}{2.082769in}}%
\pgfpathcurveto{\pgfqpoint{2.813910in}{2.078862in}}{\pgfqpoint{2.819209in}{2.076667in}}{\pgfqpoint{2.824734in}{2.076667in}}%
\pgfpathclose%
\pgfusepath{stroke,fill}%
\end{pgfscope}%
\begin{pgfscope}%
\pgfpathrectangle{\pgfqpoint{0.562500in}{0.275000in}}{\pgfqpoint{3.487500in}{1.925000in}}%
\pgfusepath{clip}%
\pgfsetbuttcap%
\pgfsetroundjoin%
\definecolor{currentfill}{rgb}{0.000000,0.000000,0.000000}%
\pgfsetfillcolor{currentfill}%
\pgfsetlinewidth{1.003750pt}%
\definecolor{currentstroke}{rgb}{0.000000,0.000000,0.000000}%
\pgfsetstrokecolor{currentstroke}%
\pgfsetdash{}{0pt}%
\pgfpathmoveto{\pgfqpoint{2.824734in}{1.216635in}}%
\pgfpathcurveto{\pgfqpoint{2.830260in}{1.216635in}}{\pgfqpoint{2.835559in}{1.218830in}}{\pgfqpoint{2.839466in}{1.222736in}}%
\pgfpathcurveto{\pgfqpoint{2.843373in}{1.226643in}}{\pgfqpoint{2.845568in}{1.231943in}}{\pgfqpoint{2.845568in}{1.237468in}}%
\pgfpathcurveto{\pgfqpoint{2.845568in}{1.242993in}}{\pgfqpoint{2.843373in}{1.248292in}}{\pgfqpoint{2.839466in}{1.252199in}}%
\pgfpathcurveto{\pgfqpoint{2.835559in}{1.256106in}}{\pgfqpoint{2.830260in}{1.258301in}}{\pgfqpoint{2.824734in}{1.258301in}}%
\pgfpathcurveto{\pgfqpoint{2.819209in}{1.258301in}}{\pgfqpoint{2.813910in}{1.256106in}}{\pgfqpoint{2.810003in}{1.252199in}}%
\pgfpathcurveto{\pgfqpoint{2.806096in}{1.248292in}}{\pgfqpoint{2.803901in}{1.242993in}}{\pgfqpoint{2.803901in}{1.237468in}}%
\pgfpathcurveto{\pgfqpoint{2.803901in}{1.231943in}}{\pgfqpoint{2.806096in}{1.226643in}}{\pgfqpoint{2.810003in}{1.222736in}}%
\pgfpathcurveto{\pgfqpoint{2.813910in}{1.218830in}}{\pgfqpoint{2.819209in}{1.216635in}}{\pgfqpoint{2.824734in}{1.216635in}}%
\pgfpathclose%
\pgfusepath{stroke,fill}%
\end{pgfscope}%
\begin{pgfscope}%
\pgfpathrectangle{\pgfqpoint{0.562500in}{0.275000in}}{\pgfqpoint{3.487500in}{1.925000in}}%
\pgfusepath{clip}%
\pgfsetbuttcap%
\pgfsetroundjoin%
\definecolor{currentfill}{rgb}{0.000000,0.000000,0.000000}%
\pgfsetfillcolor{currentfill}%
\pgfsetlinewidth{1.003750pt}%
\definecolor{currentstroke}{rgb}{0.000000,0.000000,0.000000}%
\pgfsetstrokecolor{currentstroke}%
\pgfsetdash{}{0pt}%
\pgfpathmoveto{\pgfqpoint{2.824734in}{1.216635in}}%
\pgfpathcurveto{\pgfqpoint{2.830260in}{1.216635in}}{\pgfqpoint{2.835559in}{1.218830in}}{\pgfqpoint{2.839466in}{1.222736in}}%
\pgfpathcurveto{\pgfqpoint{2.843373in}{1.226643in}}{\pgfqpoint{2.845568in}{1.231943in}}{\pgfqpoint{2.845568in}{1.237468in}}%
\pgfpathcurveto{\pgfqpoint{2.845568in}{1.242993in}}{\pgfqpoint{2.843373in}{1.248292in}}{\pgfqpoint{2.839466in}{1.252199in}}%
\pgfpathcurveto{\pgfqpoint{2.835559in}{1.256106in}}{\pgfqpoint{2.830260in}{1.258301in}}{\pgfqpoint{2.824734in}{1.258301in}}%
\pgfpathcurveto{\pgfqpoint{2.819209in}{1.258301in}}{\pgfqpoint{2.813910in}{1.256106in}}{\pgfqpoint{2.810003in}{1.252199in}}%
\pgfpathcurveto{\pgfqpoint{2.806096in}{1.248292in}}{\pgfqpoint{2.803901in}{1.242993in}}{\pgfqpoint{2.803901in}{1.237468in}}%
\pgfpathcurveto{\pgfqpoint{2.803901in}{1.231943in}}{\pgfqpoint{2.806096in}{1.226643in}}{\pgfqpoint{2.810003in}{1.222736in}}%
\pgfpathcurveto{\pgfqpoint{2.813910in}{1.218830in}}{\pgfqpoint{2.819209in}{1.216635in}}{\pgfqpoint{2.824734in}{1.216635in}}%
\pgfpathclose%
\pgfusepath{stroke,fill}%
\end{pgfscope}%
\begin{pgfscope}%
\pgfpathrectangle{\pgfqpoint{0.562500in}{0.275000in}}{\pgfqpoint{3.487500in}{1.925000in}}%
\pgfusepath{clip}%
\pgfsetbuttcap%
\pgfsetroundjoin%
\definecolor{currentfill}{rgb}{0.000000,0.000000,0.000000}%
\pgfsetfillcolor{currentfill}%
\pgfsetlinewidth{1.003750pt}%
\definecolor{currentstroke}{rgb}{0.000000,0.000000,0.000000}%
\pgfsetstrokecolor{currentstroke}%
\pgfsetdash{}{0pt}%
\pgfpathmoveto{\pgfqpoint{2.824734in}{1.216635in}}%
\pgfpathcurveto{\pgfqpoint{2.830260in}{1.216635in}}{\pgfqpoint{2.835559in}{1.218830in}}{\pgfqpoint{2.839466in}{1.222736in}}%
\pgfpathcurveto{\pgfqpoint{2.843373in}{1.226643in}}{\pgfqpoint{2.845568in}{1.231943in}}{\pgfqpoint{2.845568in}{1.237468in}}%
\pgfpathcurveto{\pgfqpoint{2.845568in}{1.242993in}}{\pgfqpoint{2.843373in}{1.248292in}}{\pgfqpoint{2.839466in}{1.252199in}}%
\pgfpathcurveto{\pgfqpoint{2.835559in}{1.256106in}}{\pgfqpoint{2.830260in}{1.258301in}}{\pgfqpoint{2.824734in}{1.258301in}}%
\pgfpathcurveto{\pgfqpoint{2.819209in}{1.258301in}}{\pgfqpoint{2.813910in}{1.256106in}}{\pgfqpoint{2.810003in}{1.252199in}}%
\pgfpathcurveto{\pgfqpoint{2.806096in}{1.248292in}}{\pgfqpoint{2.803901in}{1.242993in}}{\pgfqpoint{2.803901in}{1.237468in}}%
\pgfpathcurveto{\pgfqpoint{2.803901in}{1.231943in}}{\pgfqpoint{2.806096in}{1.226643in}}{\pgfqpoint{2.810003in}{1.222736in}}%
\pgfpathcurveto{\pgfqpoint{2.813910in}{1.218830in}}{\pgfqpoint{2.819209in}{1.216635in}}{\pgfqpoint{2.824734in}{1.216635in}}%
\pgfpathclose%
\pgfusepath{stroke,fill}%
\end{pgfscope}%
\begin{pgfscope}%
\pgfpathrectangle{\pgfqpoint{0.562500in}{0.275000in}}{\pgfqpoint{3.487500in}{1.925000in}}%
\pgfusepath{clip}%
\pgfsetbuttcap%
\pgfsetroundjoin%
\definecolor{currentfill}{rgb}{0.000000,0.000000,0.000000}%
\pgfsetfillcolor{currentfill}%
\pgfsetlinewidth{1.003750pt}%
\definecolor{currentstroke}{rgb}{0.000000,0.000000,0.000000}%
\pgfsetstrokecolor{currentstroke}%
\pgfsetdash{}{0pt}%
\pgfpathmoveto{\pgfqpoint{2.824734in}{1.216635in}}%
\pgfpathcurveto{\pgfqpoint{2.830260in}{1.216635in}}{\pgfqpoint{2.835559in}{1.218830in}}{\pgfqpoint{2.839466in}{1.222736in}}%
\pgfpathcurveto{\pgfqpoint{2.843373in}{1.226643in}}{\pgfqpoint{2.845568in}{1.231943in}}{\pgfqpoint{2.845568in}{1.237468in}}%
\pgfpathcurveto{\pgfqpoint{2.845568in}{1.242993in}}{\pgfqpoint{2.843373in}{1.248292in}}{\pgfqpoint{2.839466in}{1.252199in}}%
\pgfpathcurveto{\pgfqpoint{2.835559in}{1.256106in}}{\pgfqpoint{2.830260in}{1.258301in}}{\pgfqpoint{2.824734in}{1.258301in}}%
\pgfpathcurveto{\pgfqpoint{2.819209in}{1.258301in}}{\pgfqpoint{2.813910in}{1.256106in}}{\pgfqpoint{2.810003in}{1.252199in}}%
\pgfpathcurveto{\pgfqpoint{2.806096in}{1.248292in}}{\pgfqpoint{2.803901in}{1.242993in}}{\pgfqpoint{2.803901in}{1.237468in}}%
\pgfpathcurveto{\pgfqpoint{2.803901in}{1.231943in}}{\pgfqpoint{2.806096in}{1.226643in}}{\pgfqpoint{2.810003in}{1.222736in}}%
\pgfpathcurveto{\pgfqpoint{2.813910in}{1.218830in}}{\pgfqpoint{2.819209in}{1.216635in}}{\pgfqpoint{2.824734in}{1.216635in}}%
\pgfpathclose%
\pgfusepath{stroke,fill}%
\end{pgfscope}%
\begin{pgfscope}%
\pgfpathrectangle{\pgfqpoint{0.562500in}{0.275000in}}{\pgfqpoint{3.487500in}{1.925000in}}%
\pgfusepath{clip}%
\pgfsetbuttcap%
\pgfsetroundjoin%
\definecolor{currentfill}{rgb}{0.000000,0.000000,0.000000}%
\pgfsetfillcolor{currentfill}%
\pgfsetlinewidth{1.003750pt}%
\definecolor{currentstroke}{rgb}{0.000000,0.000000,0.000000}%
\pgfsetstrokecolor{currentstroke}%
\pgfsetdash{}{0pt}%
\pgfpathmoveto{\pgfqpoint{2.824734in}{1.216635in}}%
\pgfpathcurveto{\pgfqpoint{2.830260in}{1.216635in}}{\pgfqpoint{2.835559in}{1.218830in}}{\pgfqpoint{2.839466in}{1.222736in}}%
\pgfpathcurveto{\pgfqpoint{2.843373in}{1.226643in}}{\pgfqpoint{2.845568in}{1.231943in}}{\pgfqpoint{2.845568in}{1.237468in}}%
\pgfpathcurveto{\pgfqpoint{2.845568in}{1.242993in}}{\pgfqpoint{2.843373in}{1.248292in}}{\pgfqpoint{2.839466in}{1.252199in}}%
\pgfpathcurveto{\pgfqpoint{2.835559in}{1.256106in}}{\pgfqpoint{2.830260in}{1.258301in}}{\pgfqpoint{2.824734in}{1.258301in}}%
\pgfpathcurveto{\pgfqpoint{2.819209in}{1.258301in}}{\pgfqpoint{2.813910in}{1.256106in}}{\pgfqpoint{2.810003in}{1.252199in}}%
\pgfpathcurveto{\pgfqpoint{2.806096in}{1.248292in}}{\pgfqpoint{2.803901in}{1.242993in}}{\pgfqpoint{2.803901in}{1.237468in}}%
\pgfpathcurveto{\pgfqpoint{2.803901in}{1.231943in}}{\pgfqpoint{2.806096in}{1.226643in}}{\pgfqpoint{2.810003in}{1.222736in}}%
\pgfpathcurveto{\pgfqpoint{2.813910in}{1.218830in}}{\pgfqpoint{2.819209in}{1.216635in}}{\pgfqpoint{2.824734in}{1.216635in}}%
\pgfpathclose%
\pgfusepath{stroke,fill}%
\end{pgfscope}%
\begin{pgfscope}%
\pgfpathrectangle{\pgfqpoint{0.562500in}{0.275000in}}{\pgfqpoint{3.487500in}{1.925000in}}%
\pgfusepath{clip}%
\pgfsetbuttcap%
\pgfsetroundjoin%
\definecolor{currentfill}{rgb}{0.000000,0.000000,0.000000}%
\pgfsetfillcolor{currentfill}%
\pgfsetlinewidth{1.003750pt}%
\definecolor{currentstroke}{rgb}{0.000000,0.000000,0.000000}%
\pgfsetstrokecolor{currentstroke}%
\pgfsetdash{}{0pt}%
\pgfpathmoveto{\pgfqpoint{2.824734in}{2.076667in}}%
\pgfpathcurveto{\pgfqpoint{2.830260in}{2.076667in}}{\pgfqpoint{2.835559in}{2.078862in}}{\pgfqpoint{2.839466in}{2.082769in}}%
\pgfpathcurveto{\pgfqpoint{2.843373in}{2.086675in}}{\pgfqpoint{2.845568in}{2.091975in}}{\pgfqpoint{2.845568in}{2.097500in}}%
\pgfpathcurveto{\pgfqpoint{2.845568in}{2.103025in}}{\pgfqpoint{2.843373in}{2.108325in}}{\pgfqpoint{2.839466in}{2.112231in}}%
\pgfpathcurveto{\pgfqpoint{2.835559in}{2.116138in}}{\pgfqpoint{2.830260in}{2.118333in}}{\pgfqpoint{2.824734in}{2.118333in}}%
\pgfpathcurveto{\pgfqpoint{2.819209in}{2.118333in}}{\pgfqpoint{2.813910in}{2.116138in}}{\pgfqpoint{2.810003in}{2.112231in}}%
\pgfpathcurveto{\pgfqpoint{2.806096in}{2.108325in}}{\pgfqpoint{2.803901in}{2.103025in}}{\pgfqpoint{2.803901in}{2.097500in}}%
\pgfpathcurveto{\pgfqpoint{2.803901in}{2.091975in}}{\pgfqpoint{2.806096in}{2.086675in}}{\pgfqpoint{2.810003in}{2.082769in}}%
\pgfpathcurveto{\pgfqpoint{2.813910in}{2.078862in}}{\pgfqpoint{2.819209in}{2.076667in}}{\pgfqpoint{2.824734in}{2.076667in}}%
\pgfpathclose%
\pgfusepath{stroke,fill}%
\end{pgfscope}%
\begin{pgfscope}%
\pgfpathrectangle{\pgfqpoint{0.562500in}{0.275000in}}{\pgfqpoint{3.487500in}{1.925000in}}%
\pgfusepath{clip}%
\pgfsetbuttcap%
\pgfsetroundjoin%
\definecolor{currentfill}{rgb}{0.000000,0.000000,0.000000}%
\pgfsetfillcolor{currentfill}%
\pgfsetlinewidth{1.003750pt}%
\definecolor{currentstroke}{rgb}{0.000000,0.000000,0.000000}%
\pgfsetstrokecolor{currentstroke}%
\pgfsetdash{}{0pt}%
\pgfpathmoveto{\pgfqpoint{2.824734in}{1.216635in}}%
\pgfpathcurveto{\pgfqpoint{2.830260in}{1.216635in}}{\pgfqpoint{2.835559in}{1.218830in}}{\pgfqpoint{2.839466in}{1.222736in}}%
\pgfpathcurveto{\pgfqpoint{2.843373in}{1.226643in}}{\pgfqpoint{2.845568in}{1.231943in}}{\pgfqpoint{2.845568in}{1.237468in}}%
\pgfpathcurveto{\pgfqpoint{2.845568in}{1.242993in}}{\pgfqpoint{2.843373in}{1.248292in}}{\pgfqpoint{2.839466in}{1.252199in}}%
\pgfpathcurveto{\pgfqpoint{2.835559in}{1.256106in}}{\pgfqpoint{2.830260in}{1.258301in}}{\pgfqpoint{2.824734in}{1.258301in}}%
\pgfpathcurveto{\pgfqpoint{2.819209in}{1.258301in}}{\pgfqpoint{2.813910in}{1.256106in}}{\pgfqpoint{2.810003in}{1.252199in}}%
\pgfpathcurveto{\pgfqpoint{2.806096in}{1.248292in}}{\pgfqpoint{2.803901in}{1.242993in}}{\pgfqpoint{2.803901in}{1.237468in}}%
\pgfpathcurveto{\pgfqpoint{2.803901in}{1.231943in}}{\pgfqpoint{2.806096in}{1.226643in}}{\pgfqpoint{2.810003in}{1.222736in}}%
\pgfpathcurveto{\pgfqpoint{2.813910in}{1.218830in}}{\pgfqpoint{2.819209in}{1.216635in}}{\pgfqpoint{2.824734in}{1.216635in}}%
\pgfpathclose%
\pgfusepath{stroke,fill}%
\end{pgfscope}%
\begin{pgfscope}%
\pgfpathrectangle{\pgfqpoint{0.562500in}{0.275000in}}{\pgfqpoint{3.487500in}{1.925000in}}%
\pgfusepath{clip}%
\pgfsetbuttcap%
\pgfsetroundjoin%
\definecolor{currentfill}{rgb}{0.000000,0.000000,0.000000}%
\pgfsetfillcolor{currentfill}%
\pgfsetlinewidth{1.003750pt}%
\definecolor{currentstroke}{rgb}{0.000000,0.000000,0.000000}%
\pgfsetstrokecolor{currentstroke}%
\pgfsetdash{}{0pt}%
\pgfpathmoveto{\pgfqpoint{2.824734in}{1.216635in}}%
\pgfpathcurveto{\pgfqpoint{2.830260in}{1.216635in}}{\pgfqpoint{2.835559in}{1.218830in}}{\pgfqpoint{2.839466in}{1.222736in}}%
\pgfpathcurveto{\pgfqpoint{2.843373in}{1.226643in}}{\pgfqpoint{2.845568in}{1.231943in}}{\pgfqpoint{2.845568in}{1.237468in}}%
\pgfpathcurveto{\pgfqpoint{2.845568in}{1.242993in}}{\pgfqpoint{2.843373in}{1.248292in}}{\pgfqpoint{2.839466in}{1.252199in}}%
\pgfpathcurveto{\pgfqpoint{2.835559in}{1.256106in}}{\pgfqpoint{2.830260in}{1.258301in}}{\pgfqpoint{2.824734in}{1.258301in}}%
\pgfpathcurveto{\pgfqpoint{2.819209in}{1.258301in}}{\pgfqpoint{2.813910in}{1.256106in}}{\pgfqpoint{2.810003in}{1.252199in}}%
\pgfpathcurveto{\pgfqpoint{2.806096in}{1.248292in}}{\pgfqpoint{2.803901in}{1.242993in}}{\pgfqpoint{2.803901in}{1.237468in}}%
\pgfpathcurveto{\pgfqpoint{2.803901in}{1.231943in}}{\pgfqpoint{2.806096in}{1.226643in}}{\pgfqpoint{2.810003in}{1.222736in}}%
\pgfpathcurveto{\pgfqpoint{2.813910in}{1.218830in}}{\pgfqpoint{2.819209in}{1.216635in}}{\pgfqpoint{2.824734in}{1.216635in}}%
\pgfpathclose%
\pgfusepath{stroke,fill}%
\end{pgfscope}%
\begin{pgfscope}%
\pgfpathrectangle{\pgfqpoint{0.562500in}{0.275000in}}{\pgfqpoint{3.487500in}{1.925000in}}%
\pgfusepath{clip}%
\pgfsetbuttcap%
\pgfsetroundjoin%
\definecolor{currentfill}{rgb}{0.000000,0.000000,0.000000}%
\pgfsetfillcolor{currentfill}%
\pgfsetlinewidth{1.003750pt}%
\definecolor{currentstroke}{rgb}{0.000000,0.000000,0.000000}%
\pgfsetstrokecolor{currentstroke}%
\pgfsetdash{}{0pt}%
\pgfpathmoveto{\pgfqpoint{2.824734in}{1.216635in}}%
\pgfpathcurveto{\pgfqpoint{2.830260in}{1.216635in}}{\pgfqpoint{2.835559in}{1.218830in}}{\pgfqpoint{2.839466in}{1.222736in}}%
\pgfpathcurveto{\pgfqpoint{2.843373in}{1.226643in}}{\pgfqpoint{2.845568in}{1.231943in}}{\pgfqpoint{2.845568in}{1.237468in}}%
\pgfpathcurveto{\pgfqpoint{2.845568in}{1.242993in}}{\pgfqpoint{2.843373in}{1.248292in}}{\pgfqpoint{2.839466in}{1.252199in}}%
\pgfpathcurveto{\pgfqpoint{2.835559in}{1.256106in}}{\pgfqpoint{2.830260in}{1.258301in}}{\pgfqpoint{2.824734in}{1.258301in}}%
\pgfpathcurveto{\pgfqpoint{2.819209in}{1.258301in}}{\pgfqpoint{2.813910in}{1.256106in}}{\pgfqpoint{2.810003in}{1.252199in}}%
\pgfpathcurveto{\pgfqpoint{2.806096in}{1.248292in}}{\pgfqpoint{2.803901in}{1.242993in}}{\pgfqpoint{2.803901in}{1.237468in}}%
\pgfpathcurveto{\pgfqpoint{2.803901in}{1.231943in}}{\pgfqpoint{2.806096in}{1.226643in}}{\pgfqpoint{2.810003in}{1.222736in}}%
\pgfpathcurveto{\pgfqpoint{2.813910in}{1.218830in}}{\pgfqpoint{2.819209in}{1.216635in}}{\pgfqpoint{2.824734in}{1.216635in}}%
\pgfpathclose%
\pgfusepath{stroke,fill}%
\end{pgfscope}%
\begin{pgfscope}%
\pgfpathrectangle{\pgfqpoint{0.562500in}{0.275000in}}{\pgfqpoint{3.487500in}{1.925000in}}%
\pgfusepath{clip}%
\pgfsetbuttcap%
\pgfsetroundjoin%
\definecolor{currentfill}{rgb}{0.000000,0.000000,0.000000}%
\pgfsetfillcolor{currentfill}%
\pgfsetlinewidth{1.003750pt}%
\definecolor{currentstroke}{rgb}{0.000000,0.000000,0.000000}%
\pgfsetstrokecolor{currentstroke}%
\pgfsetdash{}{0pt}%
\pgfpathmoveto{\pgfqpoint{2.824734in}{1.216635in}}%
\pgfpathcurveto{\pgfqpoint{2.830260in}{1.216635in}}{\pgfqpoint{2.835559in}{1.218830in}}{\pgfqpoint{2.839466in}{1.222736in}}%
\pgfpathcurveto{\pgfqpoint{2.843373in}{1.226643in}}{\pgfqpoint{2.845568in}{1.231943in}}{\pgfqpoint{2.845568in}{1.237468in}}%
\pgfpathcurveto{\pgfqpoint{2.845568in}{1.242993in}}{\pgfqpoint{2.843373in}{1.248292in}}{\pgfqpoint{2.839466in}{1.252199in}}%
\pgfpathcurveto{\pgfqpoint{2.835559in}{1.256106in}}{\pgfqpoint{2.830260in}{1.258301in}}{\pgfqpoint{2.824734in}{1.258301in}}%
\pgfpathcurveto{\pgfqpoint{2.819209in}{1.258301in}}{\pgfqpoint{2.813910in}{1.256106in}}{\pgfqpoint{2.810003in}{1.252199in}}%
\pgfpathcurveto{\pgfqpoint{2.806096in}{1.248292in}}{\pgfqpoint{2.803901in}{1.242993in}}{\pgfqpoint{2.803901in}{1.237468in}}%
\pgfpathcurveto{\pgfqpoint{2.803901in}{1.231943in}}{\pgfqpoint{2.806096in}{1.226643in}}{\pgfqpoint{2.810003in}{1.222736in}}%
\pgfpathcurveto{\pgfqpoint{2.813910in}{1.218830in}}{\pgfqpoint{2.819209in}{1.216635in}}{\pgfqpoint{2.824734in}{1.216635in}}%
\pgfpathclose%
\pgfusepath{stroke,fill}%
\end{pgfscope}%
\begin{pgfscope}%
\pgfpathrectangle{\pgfqpoint{0.562500in}{0.275000in}}{\pgfqpoint{3.487500in}{1.925000in}}%
\pgfusepath{clip}%
\pgfsetbuttcap%
\pgfsetroundjoin%
\definecolor{currentfill}{rgb}{0.000000,0.000000,0.000000}%
\pgfsetfillcolor{currentfill}%
\pgfsetlinewidth{1.003750pt}%
\definecolor{currentstroke}{rgb}{0.000000,0.000000,0.000000}%
\pgfsetstrokecolor{currentstroke}%
\pgfsetdash{}{0pt}%
\pgfpathmoveto{\pgfqpoint{2.824734in}{2.076667in}}%
\pgfpathcurveto{\pgfqpoint{2.830260in}{2.076667in}}{\pgfqpoint{2.835559in}{2.078862in}}{\pgfqpoint{2.839466in}{2.082769in}}%
\pgfpathcurveto{\pgfqpoint{2.843373in}{2.086675in}}{\pgfqpoint{2.845568in}{2.091975in}}{\pgfqpoint{2.845568in}{2.097500in}}%
\pgfpathcurveto{\pgfqpoint{2.845568in}{2.103025in}}{\pgfqpoint{2.843373in}{2.108325in}}{\pgfqpoint{2.839466in}{2.112231in}}%
\pgfpathcurveto{\pgfqpoint{2.835559in}{2.116138in}}{\pgfqpoint{2.830260in}{2.118333in}}{\pgfqpoint{2.824734in}{2.118333in}}%
\pgfpathcurveto{\pgfqpoint{2.819209in}{2.118333in}}{\pgfqpoint{2.813910in}{2.116138in}}{\pgfqpoint{2.810003in}{2.112231in}}%
\pgfpathcurveto{\pgfqpoint{2.806096in}{2.108325in}}{\pgfqpoint{2.803901in}{2.103025in}}{\pgfqpoint{2.803901in}{2.097500in}}%
\pgfpathcurveto{\pgfqpoint{2.803901in}{2.091975in}}{\pgfqpoint{2.806096in}{2.086675in}}{\pgfqpoint{2.810003in}{2.082769in}}%
\pgfpathcurveto{\pgfqpoint{2.813910in}{2.078862in}}{\pgfqpoint{2.819209in}{2.076667in}}{\pgfqpoint{2.824734in}{2.076667in}}%
\pgfpathclose%
\pgfusepath{stroke,fill}%
\end{pgfscope}%
\begin{pgfscope}%
\pgfpathrectangle{\pgfqpoint{0.562500in}{0.275000in}}{\pgfqpoint{3.487500in}{1.925000in}}%
\pgfusepath{clip}%
\pgfsetbuttcap%
\pgfsetroundjoin%
\definecolor{currentfill}{rgb}{0.000000,0.000000,0.000000}%
\pgfsetfillcolor{currentfill}%
\pgfsetlinewidth{1.003750pt}%
\definecolor{currentstroke}{rgb}{0.000000,0.000000,0.000000}%
\pgfsetstrokecolor{currentstroke}%
\pgfsetdash{}{0pt}%
\pgfpathmoveto{\pgfqpoint{2.824734in}{1.216635in}}%
\pgfpathcurveto{\pgfqpoint{2.830260in}{1.216635in}}{\pgfqpoint{2.835559in}{1.218830in}}{\pgfqpoint{2.839466in}{1.222736in}}%
\pgfpathcurveto{\pgfqpoint{2.843373in}{1.226643in}}{\pgfqpoint{2.845568in}{1.231943in}}{\pgfqpoint{2.845568in}{1.237468in}}%
\pgfpathcurveto{\pgfqpoint{2.845568in}{1.242993in}}{\pgfqpoint{2.843373in}{1.248292in}}{\pgfqpoint{2.839466in}{1.252199in}}%
\pgfpathcurveto{\pgfqpoint{2.835559in}{1.256106in}}{\pgfqpoint{2.830260in}{1.258301in}}{\pgfqpoint{2.824734in}{1.258301in}}%
\pgfpathcurveto{\pgfqpoint{2.819209in}{1.258301in}}{\pgfqpoint{2.813910in}{1.256106in}}{\pgfqpoint{2.810003in}{1.252199in}}%
\pgfpathcurveto{\pgfqpoint{2.806096in}{1.248292in}}{\pgfqpoint{2.803901in}{1.242993in}}{\pgfqpoint{2.803901in}{1.237468in}}%
\pgfpathcurveto{\pgfqpoint{2.803901in}{1.231943in}}{\pgfqpoint{2.806096in}{1.226643in}}{\pgfqpoint{2.810003in}{1.222736in}}%
\pgfpathcurveto{\pgfqpoint{2.813910in}{1.218830in}}{\pgfqpoint{2.819209in}{1.216635in}}{\pgfqpoint{2.824734in}{1.216635in}}%
\pgfpathclose%
\pgfusepath{stroke,fill}%
\end{pgfscope}%
\begin{pgfscope}%
\pgfpathrectangle{\pgfqpoint{0.562500in}{0.275000in}}{\pgfqpoint{3.487500in}{1.925000in}}%
\pgfusepath{clip}%
\pgfsetbuttcap%
\pgfsetroundjoin%
\definecolor{currentfill}{rgb}{0.000000,0.000000,0.000000}%
\pgfsetfillcolor{currentfill}%
\pgfsetlinewidth{1.003750pt}%
\definecolor{currentstroke}{rgb}{0.000000,0.000000,0.000000}%
\pgfsetstrokecolor{currentstroke}%
\pgfsetdash{}{0pt}%
\pgfpathmoveto{\pgfqpoint{2.824734in}{1.216635in}}%
\pgfpathcurveto{\pgfqpoint{2.830260in}{1.216635in}}{\pgfqpoint{2.835559in}{1.218830in}}{\pgfqpoint{2.839466in}{1.222736in}}%
\pgfpathcurveto{\pgfqpoint{2.843373in}{1.226643in}}{\pgfqpoint{2.845568in}{1.231943in}}{\pgfqpoint{2.845568in}{1.237468in}}%
\pgfpathcurveto{\pgfqpoint{2.845568in}{1.242993in}}{\pgfqpoint{2.843373in}{1.248292in}}{\pgfqpoint{2.839466in}{1.252199in}}%
\pgfpathcurveto{\pgfqpoint{2.835559in}{1.256106in}}{\pgfqpoint{2.830260in}{1.258301in}}{\pgfqpoint{2.824734in}{1.258301in}}%
\pgfpathcurveto{\pgfqpoint{2.819209in}{1.258301in}}{\pgfqpoint{2.813910in}{1.256106in}}{\pgfqpoint{2.810003in}{1.252199in}}%
\pgfpathcurveto{\pgfqpoint{2.806096in}{1.248292in}}{\pgfqpoint{2.803901in}{1.242993in}}{\pgfqpoint{2.803901in}{1.237468in}}%
\pgfpathcurveto{\pgfqpoint{2.803901in}{1.231943in}}{\pgfqpoint{2.806096in}{1.226643in}}{\pgfqpoint{2.810003in}{1.222736in}}%
\pgfpathcurveto{\pgfqpoint{2.813910in}{1.218830in}}{\pgfqpoint{2.819209in}{1.216635in}}{\pgfqpoint{2.824734in}{1.216635in}}%
\pgfpathclose%
\pgfusepath{stroke,fill}%
\end{pgfscope}%
\begin{pgfscope}%
\pgfpathrectangle{\pgfqpoint{0.562500in}{0.275000in}}{\pgfqpoint{3.487500in}{1.925000in}}%
\pgfusepath{clip}%
\pgfsetbuttcap%
\pgfsetroundjoin%
\definecolor{currentfill}{rgb}{0.000000,0.000000,0.000000}%
\pgfsetfillcolor{currentfill}%
\pgfsetlinewidth{1.003750pt}%
\definecolor{currentstroke}{rgb}{0.000000,0.000000,0.000000}%
\pgfsetstrokecolor{currentstroke}%
\pgfsetdash{}{0pt}%
\pgfpathmoveto{\pgfqpoint{2.824734in}{1.216635in}}%
\pgfpathcurveto{\pgfqpoint{2.830260in}{1.216635in}}{\pgfqpoint{2.835559in}{1.218830in}}{\pgfqpoint{2.839466in}{1.222736in}}%
\pgfpathcurveto{\pgfqpoint{2.843373in}{1.226643in}}{\pgfqpoint{2.845568in}{1.231943in}}{\pgfqpoint{2.845568in}{1.237468in}}%
\pgfpathcurveto{\pgfqpoint{2.845568in}{1.242993in}}{\pgfqpoint{2.843373in}{1.248292in}}{\pgfqpoint{2.839466in}{1.252199in}}%
\pgfpathcurveto{\pgfqpoint{2.835559in}{1.256106in}}{\pgfqpoint{2.830260in}{1.258301in}}{\pgfqpoint{2.824734in}{1.258301in}}%
\pgfpathcurveto{\pgfqpoint{2.819209in}{1.258301in}}{\pgfqpoint{2.813910in}{1.256106in}}{\pgfqpoint{2.810003in}{1.252199in}}%
\pgfpathcurveto{\pgfqpoint{2.806096in}{1.248292in}}{\pgfqpoint{2.803901in}{1.242993in}}{\pgfqpoint{2.803901in}{1.237468in}}%
\pgfpathcurveto{\pgfqpoint{2.803901in}{1.231943in}}{\pgfqpoint{2.806096in}{1.226643in}}{\pgfqpoint{2.810003in}{1.222736in}}%
\pgfpathcurveto{\pgfqpoint{2.813910in}{1.218830in}}{\pgfqpoint{2.819209in}{1.216635in}}{\pgfqpoint{2.824734in}{1.216635in}}%
\pgfpathclose%
\pgfusepath{stroke,fill}%
\end{pgfscope}%
\begin{pgfscope}%
\pgfpathrectangle{\pgfqpoint{0.562500in}{0.275000in}}{\pgfqpoint{3.487500in}{1.925000in}}%
\pgfusepath{clip}%
\pgfsetbuttcap%
\pgfsetroundjoin%
\definecolor{currentfill}{rgb}{0.000000,0.000000,0.000000}%
\pgfsetfillcolor{currentfill}%
\pgfsetlinewidth{1.003750pt}%
\definecolor{currentstroke}{rgb}{0.000000,0.000000,0.000000}%
\pgfsetstrokecolor{currentstroke}%
\pgfsetdash{}{0pt}%
\pgfpathmoveto{\pgfqpoint{2.824734in}{1.216635in}}%
\pgfpathcurveto{\pgfqpoint{2.830260in}{1.216635in}}{\pgfqpoint{2.835559in}{1.218830in}}{\pgfqpoint{2.839466in}{1.222736in}}%
\pgfpathcurveto{\pgfqpoint{2.843373in}{1.226643in}}{\pgfqpoint{2.845568in}{1.231943in}}{\pgfqpoint{2.845568in}{1.237468in}}%
\pgfpathcurveto{\pgfqpoint{2.845568in}{1.242993in}}{\pgfqpoint{2.843373in}{1.248292in}}{\pgfqpoint{2.839466in}{1.252199in}}%
\pgfpathcurveto{\pgfqpoint{2.835559in}{1.256106in}}{\pgfqpoint{2.830260in}{1.258301in}}{\pgfqpoint{2.824734in}{1.258301in}}%
\pgfpathcurveto{\pgfqpoint{2.819209in}{1.258301in}}{\pgfqpoint{2.813910in}{1.256106in}}{\pgfqpoint{2.810003in}{1.252199in}}%
\pgfpathcurveto{\pgfqpoint{2.806096in}{1.248292in}}{\pgfqpoint{2.803901in}{1.242993in}}{\pgfqpoint{2.803901in}{1.237468in}}%
\pgfpathcurveto{\pgfqpoint{2.803901in}{1.231943in}}{\pgfqpoint{2.806096in}{1.226643in}}{\pgfqpoint{2.810003in}{1.222736in}}%
\pgfpathcurveto{\pgfqpoint{2.813910in}{1.218830in}}{\pgfqpoint{2.819209in}{1.216635in}}{\pgfqpoint{2.824734in}{1.216635in}}%
\pgfpathclose%
\pgfusepath{stroke,fill}%
\end{pgfscope}%
\begin{pgfscope}%
\pgfpathrectangle{\pgfqpoint{0.562500in}{0.275000in}}{\pgfqpoint{3.487500in}{1.925000in}}%
\pgfusepath{clip}%
\pgfsetbuttcap%
\pgfsetroundjoin%
\definecolor{currentfill}{rgb}{0.000000,0.000000,0.000000}%
\pgfsetfillcolor{currentfill}%
\pgfsetlinewidth{1.003750pt}%
\definecolor{currentstroke}{rgb}{0.000000,0.000000,0.000000}%
\pgfsetstrokecolor{currentstroke}%
\pgfsetdash{}{0pt}%
\pgfpathmoveto{\pgfqpoint{2.824734in}{1.216635in}}%
\pgfpathcurveto{\pgfqpoint{2.830260in}{1.216635in}}{\pgfqpoint{2.835559in}{1.218830in}}{\pgfqpoint{2.839466in}{1.222736in}}%
\pgfpathcurveto{\pgfqpoint{2.843373in}{1.226643in}}{\pgfqpoint{2.845568in}{1.231943in}}{\pgfqpoint{2.845568in}{1.237468in}}%
\pgfpathcurveto{\pgfqpoint{2.845568in}{1.242993in}}{\pgfqpoint{2.843373in}{1.248292in}}{\pgfqpoint{2.839466in}{1.252199in}}%
\pgfpathcurveto{\pgfqpoint{2.835559in}{1.256106in}}{\pgfqpoint{2.830260in}{1.258301in}}{\pgfqpoint{2.824734in}{1.258301in}}%
\pgfpathcurveto{\pgfqpoint{2.819209in}{1.258301in}}{\pgfqpoint{2.813910in}{1.256106in}}{\pgfqpoint{2.810003in}{1.252199in}}%
\pgfpathcurveto{\pgfqpoint{2.806096in}{1.248292in}}{\pgfqpoint{2.803901in}{1.242993in}}{\pgfqpoint{2.803901in}{1.237468in}}%
\pgfpathcurveto{\pgfqpoint{2.803901in}{1.231943in}}{\pgfqpoint{2.806096in}{1.226643in}}{\pgfqpoint{2.810003in}{1.222736in}}%
\pgfpathcurveto{\pgfqpoint{2.813910in}{1.218830in}}{\pgfqpoint{2.819209in}{1.216635in}}{\pgfqpoint{2.824734in}{1.216635in}}%
\pgfpathclose%
\pgfusepath{stroke,fill}%
\end{pgfscope}%
\begin{pgfscope}%
\pgfpathrectangle{\pgfqpoint{0.562500in}{0.275000in}}{\pgfqpoint{3.487500in}{1.925000in}}%
\pgfusepath{clip}%
\pgfsetbuttcap%
\pgfsetroundjoin%
\definecolor{currentfill}{rgb}{0.000000,0.000000,0.000000}%
\pgfsetfillcolor{currentfill}%
\pgfsetlinewidth{1.003750pt}%
\definecolor{currentstroke}{rgb}{0.000000,0.000000,0.000000}%
\pgfsetstrokecolor{currentstroke}%
\pgfsetdash{}{0pt}%
\pgfpathmoveto{\pgfqpoint{2.824734in}{1.216635in}}%
\pgfpathcurveto{\pgfqpoint{2.830260in}{1.216635in}}{\pgfqpoint{2.835559in}{1.218830in}}{\pgfqpoint{2.839466in}{1.222736in}}%
\pgfpathcurveto{\pgfqpoint{2.843373in}{1.226643in}}{\pgfqpoint{2.845568in}{1.231943in}}{\pgfqpoint{2.845568in}{1.237468in}}%
\pgfpathcurveto{\pgfqpoint{2.845568in}{1.242993in}}{\pgfqpoint{2.843373in}{1.248292in}}{\pgfqpoint{2.839466in}{1.252199in}}%
\pgfpathcurveto{\pgfqpoint{2.835559in}{1.256106in}}{\pgfqpoint{2.830260in}{1.258301in}}{\pgfqpoint{2.824734in}{1.258301in}}%
\pgfpathcurveto{\pgfqpoint{2.819209in}{1.258301in}}{\pgfqpoint{2.813910in}{1.256106in}}{\pgfqpoint{2.810003in}{1.252199in}}%
\pgfpathcurveto{\pgfqpoint{2.806096in}{1.248292in}}{\pgfqpoint{2.803901in}{1.242993in}}{\pgfqpoint{2.803901in}{1.237468in}}%
\pgfpathcurveto{\pgfqpoint{2.803901in}{1.231943in}}{\pgfqpoint{2.806096in}{1.226643in}}{\pgfqpoint{2.810003in}{1.222736in}}%
\pgfpathcurveto{\pgfqpoint{2.813910in}{1.218830in}}{\pgfqpoint{2.819209in}{1.216635in}}{\pgfqpoint{2.824734in}{1.216635in}}%
\pgfpathclose%
\pgfusepath{stroke,fill}%
\end{pgfscope}%
\begin{pgfscope}%
\pgfpathrectangle{\pgfqpoint{0.562500in}{0.275000in}}{\pgfqpoint{3.487500in}{1.925000in}}%
\pgfusepath{clip}%
\pgfsetbuttcap%
\pgfsetroundjoin%
\definecolor{currentfill}{rgb}{0.000000,0.000000,0.000000}%
\pgfsetfillcolor{currentfill}%
\pgfsetlinewidth{1.003750pt}%
\definecolor{currentstroke}{rgb}{0.000000,0.000000,0.000000}%
\pgfsetstrokecolor{currentstroke}%
\pgfsetdash{}{0pt}%
\pgfpathmoveto{\pgfqpoint{3.876477in}{1.216635in}}%
\pgfpathcurveto{\pgfqpoint{3.882002in}{1.216635in}}{\pgfqpoint{3.887302in}{1.218830in}}{\pgfqpoint{3.891209in}{1.222736in}}%
\pgfpathcurveto{\pgfqpoint{3.895115in}{1.226643in}}{\pgfqpoint{3.897311in}{1.231943in}}{\pgfqpoint{3.897311in}{1.237468in}}%
\pgfpathcurveto{\pgfqpoint{3.897311in}{1.242993in}}{\pgfqpoint{3.895115in}{1.248292in}}{\pgfqpoint{3.891209in}{1.252199in}}%
\pgfpathcurveto{\pgfqpoint{3.887302in}{1.256106in}}{\pgfqpoint{3.882002in}{1.258301in}}{\pgfqpoint{3.876477in}{1.258301in}}%
\pgfpathcurveto{\pgfqpoint{3.870952in}{1.258301in}}{\pgfqpoint{3.865653in}{1.256106in}}{\pgfqpoint{3.861746in}{1.252199in}}%
\pgfpathcurveto{\pgfqpoint{3.857839in}{1.248292in}}{\pgfqpoint{3.855644in}{1.242993in}}{\pgfqpoint{3.855644in}{1.237468in}}%
\pgfpathcurveto{\pgfqpoint{3.855644in}{1.231943in}}{\pgfqpoint{3.857839in}{1.226643in}}{\pgfqpoint{3.861746in}{1.222736in}}%
\pgfpathcurveto{\pgfqpoint{3.865653in}{1.218830in}}{\pgfqpoint{3.870952in}{1.216635in}}{\pgfqpoint{3.876477in}{1.216635in}}%
\pgfpathclose%
\pgfusepath{stroke,fill}%
\end{pgfscope}%
\begin{pgfscope}%
\pgfpathrectangle{\pgfqpoint{0.562500in}{0.275000in}}{\pgfqpoint{3.487500in}{1.925000in}}%
\pgfusepath{clip}%
\pgfsetbuttcap%
\pgfsetroundjoin%
\definecolor{currentfill}{rgb}{0.000000,0.000000,0.000000}%
\pgfsetfillcolor{currentfill}%
\pgfsetlinewidth{1.003750pt}%
\definecolor{currentstroke}{rgb}{0.000000,0.000000,0.000000}%
\pgfsetstrokecolor{currentstroke}%
\pgfsetdash{}{0pt}%
\pgfpathmoveto{\pgfqpoint{3.876477in}{2.076667in}}%
\pgfpathcurveto{\pgfqpoint{3.882002in}{2.076667in}}{\pgfqpoint{3.887302in}{2.078862in}}{\pgfqpoint{3.891209in}{2.082769in}}%
\pgfpathcurveto{\pgfqpoint{3.895115in}{2.086675in}}{\pgfqpoint{3.897311in}{2.091975in}}{\pgfqpoint{3.897311in}{2.097500in}}%
\pgfpathcurveto{\pgfqpoint{3.897311in}{2.103025in}}{\pgfqpoint{3.895115in}{2.108325in}}{\pgfqpoint{3.891209in}{2.112231in}}%
\pgfpathcurveto{\pgfqpoint{3.887302in}{2.116138in}}{\pgfqpoint{3.882002in}{2.118333in}}{\pgfqpoint{3.876477in}{2.118333in}}%
\pgfpathcurveto{\pgfqpoint{3.870952in}{2.118333in}}{\pgfqpoint{3.865653in}{2.116138in}}{\pgfqpoint{3.861746in}{2.112231in}}%
\pgfpathcurveto{\pgfqpoint{3.857839in}{2.108325in}}{\pgfqpoint{3.855644in}{2.103025in}}{\pgfqpoint{3.855644in}{2.097500in}}%
\pgfpathcurveto{\pgfqpoint{3.855644in}{2.091975in}}{\pgfqpoint{3.857839in}{2.086675in}}{\pgfqpoint{3.861746in}{2.082769in}}%
\pgfpathcurveto{\pgfqpoint{3.865653in}{2.078862in}}{\pgfqpoint{3.870952in}{2.076667in}}{\pgfqpoint{3.876477in}{2.076667in}}%
\pgfpathclose%
\pgfusepath{stroke,fill}%
\end{pgfscope}%
\begin{pgfscope}%
\pgfpathrectangle{\pgfqpoint{0.562500in}{0.275000in}}{\pgfqpoint{3.487500in}{1.925000in}}%
\pgfusepath{clip}%
\pgfsetbuttcap%
\pgfsetroundjoin%
\definecolor{currentfill}{rgb}{0.000000,0.000000,0.000000}%
\pgfsetfillcolor{currentfill}%
\pgfsetlinewidth{1.003750pt}%
\definecolor{currentstroke}{rgb}{0.000000,0.000000,0.000000}%
\pgfsetstrokecolor{currentstroke}%
\pgfsetdash{}{0pt}%
\pgfpathmoveto{\pgfqpoint{3.876477in}{1.216635in}}%
\pgfpathcurveto{\pgfqpoint{3.882002in}{1.216635in}}{\pgfqpoint{3.887302in}{1.218830in}}{\pgfqpoint{3.891209in}{1.222736in}}%
\pgfpathcurveto{\pgfqpoint{3.895115in}{1.226643in}}{\pgfqpoint{3.897311in}{1.231943in}}{\pgfqpoint{3.897311in}{1.237468in}}%
\pgfpathcurveto{\pgfqpoint{3.897311in}{1.242993in}}{\pgfqpoint{3.895115in}{1.248292in}}{\pgfqpoint{3.891209in}{1.252199in}}%
\pgfpathcurveto{\pgfqpoint{3.887302in}{1.256106in}}{\pgfqpoint{3.882002in}{1.258301in}}{\pgfqpoint{3.876477in}{1.258301in}}%
\pgfpathcurveto{\pgfqpoint{3.870952in}{1.258301in}}{\pgfqpoint{3.865653in}{1.256106in}}{\pgfqpoint{3.861746in}{1.252199in}}%
\pgfpathcurveto{\pgfqpoint{3.857839in}{1.248292in}}{\pgfqpoint{3.855644in}{1.242993in}}{\pgfqpoint{3.855644in}{1.237468in}}%
\pgfpathcurveto{\pgfqpoint{3.855644in}{1.231943in}}{\pgfqpoint{3.857839in}{1.226643in}}{\pgfqpoint{3.861746in}{1.222736in}}%
\pgfpathcurveto{\pgfqpoint{3.865653in}{1.218830in}}{\pgfqpoint{3.870952in}{1.216635in}}{\pgfqpoint{3.876477in}{1.216635in}}%
\pgfpathclose%
\pgfusepath{stroke,fill}%
\end{pgfscope}%
\begin{pgfscope}%
\pgfpathrectangle{\pgfqpoint{0.562500in}{0.275000in}}{\pgfqpoint{3.487500in}{1.925000in}}%
\pgfusepath{clip}%
\pgfsetbuttcap%
\pgfsetroundjoin%
\definecolor{currentfill}{rgb}{0.000000,0.000000,0.000000}%
\pgfsetfillcolor{currentfill}%
\pgfsetlinewidth{1.003750pt}%
\definecolor{currentstroke}{rgb}{0.000000,0.000000,0.000000}%
\pgfsetstrokecolor{currentstroke}%
\pgfsetdash{}{0pt}%
\pgfpathmoveto{\pgfqpoint{3.876477in}{2.076667in}}%
\pgfpathcurveto{\pgfqpoint{3.882002in}{2.076667in}}{\pgfqpoint{3.887302in}{2.078862in}}{\pgfqpoint{3.891209in}{2.082769in}}%
\pgfpathcurveto{\pgfqpoint{3.895115in}{2.086675in}}{\pgfqpoint{3.897311in}{2.091975in}}{\pgfqpoint{3.897311in}{2.097500in}}%
\pgfpathcurveto{\pgfqpoint{3.897311in}{2.103025in}}{\pgfqpoint{3.895115in}{2.108325in}}{\pgfqpoint{3.891209in}{2.112231in}}%
\pgfpathcurveto{\pgfqpoint{3.887302in}{2.116138in}}{\pgfqpoint{3.882002in}{2.118333in}}{\pgfqpoint{3.876477in}{2.118333in}}%
\pgfpathcurveto{\pgfqpoint{3.870952in}{2.118333in}}{\pgfqpoint{3.865653in}{2.116138in}}{\pgfqpoint{3.861746in}{2.112231in}}%
\pgfpathcurveto{\pgfqpoint{3.857839in}{2.108325in}}{\pgfqpoint{3.855644in}{2.103025in}}{\pgfqpoint{3.855644in}{2.097500in}}%
\pgfpathcurveto{\pgfqpoint{3.855644in}{2.091975in}}{\pgfqpoint{3.857839in}{2.086675in}}{\pgfqpoint{3.861746in}{2.082769in}}%
\pgfpathcurveto{\pgfqpoint{3.865653in}{2.078862in}}{\pgfqpoint{3.870952in}{2.076667in}}{\pgfqpoint{3.876477in}{2.076667in}}%
\pgfpathclose%
\pgfusepath{stroke,fill}%
\end{pgfscope}%
\begin{pgfscope}%
\pgfpathrectangle{\pgfqpoint{0.562500in}{0.275000in}}{\pgfqpoint{3.487500in}{1.925000in}}%
\pgfusepath{clip}%
\pgfsetbuttcap%
\pgfsetroundjoin%
\definecolor{currentfill}{rgb}{0.000000,0.000000,0.000000}%
\pgfsetfillcolor{currentfill}%
\pgfsetlinewidth{1.003750pt}%
\definecolor{currentstroke}{rgb}{0.000000,0.000000,0.000000}%
\pgfsetstrokecolor{currentstroke}%
\pgfsetdash{}{0pt}%
\pgfpathmoveto{\pgfqpoint{3.876477in}{1.216635in}}%
\pgfpathcurveto{\pgfqpoint{3.882002in}{1.216635in}}{\pgfqpoint{3.887302in}{1.218830in}}{\pgfqpoint{3.891209in}{1.222736in}}%
\pgfpathcurveto{\pgfqpoint{3.895115in}{1.226643in}}{\pgfqpoint{3.897311in}{1.231943in}}{\pgfqpoint{3.897311in}{1.237468in}}%
\pgfpathcurveto{\pgfqpoint{3.897311in}{1.242993in}}{\pgfqpoint{3.895115in}{1.248292in}}{\pgfqpoint{3.891209in}{1.252199in}}%
\pgfpathcurveto{\pgfqpoint{3.887302in}{1.256106in}}{\pgfqpoint{3.882002in}{1.258301in}}{\pgfqpoint{3.876477in}{1.258301in}}%
\pgfpathcurveto{\pgfqpoint{3.870952in}{1.258301in}}{\pgfqpoint{3.865653in}{1.256106in}}{\pgfqpoint{3.861746in}{1.252199in}}%
\pgfpathcurveto{\pgfqpoint{3.857839in}{1.248292in}}{\pgfqpoint{3.855644in}{1.242993in}}{\pgfqpoint{3.855644in}{1.237468in}}%
\pgfpathcurveto{\pgfqpoint{3.855644in}{1.231943in}}{\pgfqpoint{3.857839in}{1.226643in}}{\pgfqpoint{3.861746in}{1.222736in}}%
\pgfpathcurveto{\pgfqpoint{3.865653in}{1.218830in}}{\pgfqpoint{3.870952in}{1.216635in}}{\pgfqpoint{3.876477in}{1.216635in}}%
\pgfpathclose%
\pgfusepath{stroke,fill}%
\end{pgfscope}%
\begin{pgfscope}%
\pgfpathrectangle{\pgfqpoint{0.562500in}{0.275000in}}{\pgfqpoint{3.487500in}{1.925000in}}%
\pgfusepath{clip}%
\pgfsetbuttcap%
\pgfsetroundjoin%
\definecolor{currentfill}{rgb}{0.000000,0.000000,0.000000}%
\pgfsetfillcolor{currentfill}%
\pgfsetlinewidth{1.003750pt}%
\definecolor{currentstroke}{rgb}{0.000000,0.000000,0.000000}%
\pgfsetstrokecolor{currentstroke}%
\pgfsetdash{}{0pt}%
\pgfpathmoveto{\pgfqpoint{3.876477in}{1.216635in}}%
\pgfpathcurveto{\pgfqpoint{3.882002in}{1.216635in}}{\pgfqpoint{3.887302in}{1.218830in}}{\pgfqpoint{3.891209in}{1.222736in}}%
\pgfpathcurveto{\pgfqpoint{3.895115in}{1.226643in}}{\pgfqpoint{3.897311in}{1.231943in}}{\pgfqpoint{3.897311in}{1.237468in}}%
\pgfpathcurveto{\pgfqpoint{3.897311in}{1.242993in}}{\pgfqpoint{3.895115in}{1.248292in}}{\pgfqpoint{3.891209in}{1.252199in}}%
\pgfpathcurveto{\pgfqpoint{3.887302in}{1.256106in}}{\pgfqpoint{3.882002in}{1.258301in}}{\pgfqpoint{3.876477in}{1.258301in}}%
\pgfpathcurveto{\pgfqpoint{3.870952in}{1.258301in}}{\pgfqpoint{3.865653in}{1.256106in}}{\pgfqpoint{3.861746in}{1.252199in}}%
\pgfpathcurveto{\pgfqpoint{3.857839in}{1.248292in}}{\pgfqpoint{3.855644in}{1.242993in}}{\pgfqpoint{3.855644in}{1.237468in}}%
\pgfpathcurveto{\pgfqpoint{3.855644in}{1.231943in}}{\pgfqpoint{3.857839in}{1.226643in}}{\pgfqpoint{3.861746in}{1.222736in}}%
\pgfpathcurveto{\pgfqpoint{3.865653in}{1.218830in}}{\pgfqpoint{3.870952in}{1.216635in}}{\pgfqpoint{3.876477in}{1.216635in}}%
\pgfpathclose%
\pgfusepath{stroke,fill}%
\end{pgfscope}%
\begin{pgfscope}%
\pgfpathrectangle{\pgfqpoint{0.562500in}{0.275000in}}{\pgfqpoint{3.487500in}{1.925000in}}%
\pgfusepath{clip}%
\pgfsetbuttcap%
\pgfsetroundjoin%
\definecolor{currentfill}{rgb}{0.000000,0.000000,0.000000}%
\pgfsetfillcolor{currentfill}%
\pgfsetlinewidth{1.003750pt}%
\definecolor{currentstroke}{rgb}{0.000000,0.000000,0.000000}%
\pgfsetstrokecolor{currentstroke}%
\pgfsetdash{}{0pt}%
\pgfpathmoveto{\pgfqpoint{3.876477in}{2.076667in}}%
\pgfpathcurveto{\pgfqpoint{3.882002in}{2.076667in}}{\pgfqpoint{3.887302in}{2.078862in}}{\pgfqpoint{3.891209in}{2.082769in}}%
\pgfpathcurveto{\pgfqpoint{3.895115in}{2.086675in}}{\pgfqpoint{3.897311in}{2.091975in}}{\pgfqpoint{3.897311in}{2.097500in}}%
\pgfpathcurveto{\pgfqpoint{3.897311in}{2.103025in}}{\pgfqpoint{3.895115in}{2.108325in}}{\pgfqpoint{3.891209in}{2.112231in}}%
\pgfpathcurveto{\pgfqpoint{3.887302in}{2.116138in}}{\pgfqpoint{3.882002in}{2.118333in}}{\pgfqpoint{3.876477in}{2.118333in}}%
\pgfpathcurveto{\pgfqpoint{3.870952in}{2.118333in}}{\pgfqpoint{3.865653in}{2.116138in}}{\pgfqpoint{3.861746in}{2.112231in}}%
\pgfpathcurveto{\pgfqpoint{3.857839in}{2.108325in}}{\pgfqpoint{3.855644in}{2.103025in}}{\pgfqpoint{3.855644in}{2.097500in}}%
\pgfpathcurveto{\pgfqpoint{3.855644in}{2.091975in}}{\pgfqpoint{3.857839in}{2.086675in}}{\pgfqpoint{3.861746in}{2.082769in}}%
\pgfpathcurveto{\pgfqpoint{3.865653in}{2.078862in}}{\pgfqpoint{3.870952in}{2.076667in}}{\pgfqpoint{3.876477in}{2.076667in}}%
\pgfpathclose%
\pgfusepath{stroke,fill}%
\end{pgfscope}%
\begin{pgfscope}%
\pgfpathrectangle{\pgfqpoint{0.562500in}{0.275000in}}{\pgfqpoint{3.487500in}{1.925000in}}%
\pgfusepath{clip}%
\pgfsetbuttcap%
\pgfsetroundjoin%
\definecolor{currentfill}{rgb}{0.000000,0.000000,0.000000}%
\pgfsetfillcolor{currentfill}%
\pgfsetlinewidth{1.003750pt}%
\definecolor{currentstroke}{rgb}{0.000000,0.000000,0.000000}%
\pgfsetstrokecolor{currentstroke}%
\pgfsetdash{}{0pt}%
\pgfpathmoveto{\pgfqpoint{3.876477in}{2.076667in}}%
\pgfpathcurveto{\pgfqpoint{3.882002in}{2.076667in}}{\pgfqpoint{3.887302in}{2.078862in}}{\pgfqpoint{3.891209in}{2.082769in}}%
\pgfpathcurveto{\pgfqpoint{3.895115in}{2.086675in}}{\pgfqpoint{3.897311in}{2.091975in}}{\pgfqpoint{3.897311in}{2.097500in}}%
\pgfpathcurveto{\pgfqpoint{3.897311in}{2.103025in}}{\pgfqpoint{3.895115in}{2.108325in}}{\pgfqpoint{3.891209in}{2.112231in}}%
\pgfpathcurveto{\pgfqpoint{3.887302in}{2.116138in}}{\pgfqpoint{3.882002in}{2.118333in}}{\pgfqpoint{3.876477in}{2.118333in}}%
\pgfpathcurveto{\pgfqpoint{3.870952in}{2.118333in}}{\pgfqpoint{3.865653in}{2.116138in}}{\pgfqpoint{3.861746in}{2.112231in}}%
\pgfpathcurveto{\pgfqpoint{3.857839in}{2.108325in}}{\pgfqpoint{3.855644in}{2.103025in}}{\pgfqpoint{3.855644in}{2.097500in}}%
\pgfpathcurveto{\pgfqpoint{3.855644in}{2.091975in}}{\pgfqpoint{3.857839in}{2.086675in}}{\pgfqpoint{3.861746in}{2.082769in}}%
\pgfpathcurveto{\pgfqpoint{3.865653in}{2.078862in}}{\pgfqpoint{3.870952in}{2.076667in}}{\pgfqpoint{3.876477in}{2.076667in}}%
\pgfpathclose%
\pgfusepath{stroke,fill}%
\end{pgfscope}%
\begin{pgfscope}%
\pgfpathrectangle{\pgfqpoint{0.562500in}{0.275000in}}{\pgfqpoint{3.487500in}{1.925000in}}%
\pgfusepath{clip}%
\pgfsetbuttcap%
\pgfsetroundjoin%
\definecolor{currentfill}{rgb}{0.000000,0.000000,0.000000}%
\pgfsetfillcolor{currentfill}%
\pgfsetlinewidth{1.003750pt}%
\definecolor{currentstroke}{rgb}{0.000000,0.000000,0.000000}%
\pgfsetstrokecolor{currentstroke}%
\pgfsetdash{}{0pt}%
\pgfpathmoveto{\pgfqpoint{3.876477in}{1.216635in}}%
\pgfpathcurveto{\pgfqpoint{3.882002in}{1.216635in}}{\pgfqpoint{3.887302in}{1.218830in}}{\pgfqpoint{3.891209in}{1.222736in}}%
\pgfpathcurveto{\pgfqpoint{3.895115in}{1.226643in}}{\pgfqpoint{3.897311in}{1.231943in}}{\pgfqpoint{3.897311in}{1.237468in}}%
\pgfpathcurveto{\pgfqpoint{3.897311in}{1.242993in}}{\pgfqpoint{3.895115in}{1.248292in}}{\pgfqpoint{3.891209in}{1.252199in}}%
\pgfpathcurveto{\pgfqpoint{3.887302in}{1.256106in}}{\pgfqpoint{3.882002in}{1.258301in}}{\pgfqpoint{3.876477in}{1.258301in}}%
\pgfpathcurveto{\pgfqpoint{3.870952in}{1.258301in}}{\pgfqpoint{3.865653in}{1.256106in}}{\pgfqpoint{3.861746in}{1.252199in}}%
\pgfpathcurveto{\pgfqpoint{3.857839in}{1.248292in}}{\pgfqpoint{3.855644in}{1.242993in}}{\pgfqpoint{3.855644in}{1.237468in}}%
\pgfpathcurveto{\pgfqpoint{3.855644in}{1.231943in}}{\pgfqpoint{3.857839in}{1.226643in}}{\pgfqpoint{3.861746in}{1.222736in}}%
\pgfpathcurveto{\pgfqpoint{3.865653in}{1.218830in}}{\pgfqpoint{3.870952in}{1.216635in}}{\pgfqpoint{3.876477in}{1.216635in}}%
\pgfpathclose%
\pgfusepath{stroke,fill}%
\end{pgfscope}%
\begin{pgfscope}%
\pgfpathrectangle{\pgfqpoint{0.562500in}{0.275000in}}{\pgfqpoint{3.487500in}{1.925000in}}%
\pgfusepath{clip}%
\pgfsetbuttcap%
\pgfsetroundjoin%
\definecolor{currentfill}{rgb}{0.000000,0.000000,0.000000}%
\pgfsetfillcolor{currentfill}%
\pgfsetlinewidth{1.003750pt}%
\definecolor{currentstroke}{rgb}{0.000000,0.000000,0.000000}%
\pgfsetstrokecolor{currentstroke}%
\pgfsetdash{}{0pt}%
\pgfpathmoveto{\pgfqpoint{3.876477in}{1.216635in}}%
\pgfpathcurveto{\pgfqpoint{3.882002in}{1.216635in}}{\pgfqpoint{3.887302in}{1.218830in}}{\pgfqpoint{3.891209in}{1.222736in}}%
\pgfpathcurveto{\pgfqpoint{3.895115in}{1.226643in}}{\pgfqpoint{3.897311in}{1.231943in}}{\pgfqpoint{3.897311in}{1.237468in}}%
\pgfpathcurveto{\pgfqpoint{3.897311in}{1.242993in}}{\pgfqpoint{3.895115in}{1.248292in}}{\pgfqpoint{3.891209in}{1.252199in}}%
\pgfpathcurveto{\pgfqpoint{3.887302in}{1.256106in}}{\pgfqpoint{3.882002in}{1.258301in}}{\pgfqpoint{3.876477in}{1.258301in}}%
\pgfpathcurveto{\pgfqpoint{3.870952in}{1.258301in}}{\pgfqpoint{3.865653in}{1.256106in}}{\pgfqpoint{3.861746in}{1.252199in}}%
\pgfpathcurveto{\pgfqpoint{3.857839in}{1.248292in}}{\pgfqpoint{3.855644in}{1.242993in}}{\pgfqpoint{3.855644in}{1.237468in}}%
\pgfpathcurveto{\pgfqpoint{3.855644in}{1.231943in}}{\pgfqpoint{3.857839in}{1.226643in}}{\pgfqpoint{3.861746in}{1.222736in}}%
\pgfpathcurveto{\pgfqpoint{3.865653in}{1.218830in}}{\pgfqpoint{3.870952in}{1.216635in}}{\pgfqpoint{3.876477in}{1.216635in}}%
\pgfpathclose%
\pgfusepath{stroke,fill}%
\end{pgfscope}%
\begin{pgfscope}%
\pgfpathrectangle{\pgfqpoint{0.562500in}{0.275000in}}{\pgfqpoint{3.487500in}{1.925000in}}%
\pgfusepath{clip}%
\pgfsetbuttcap%
\pgfsetroundjoin%
\definecolor{currentfill}{rgb}{0.000000,0.000000,0.000000}%
\pgfsetfillcolor{currentfill}%
\pgfsetlinewidth{1.003750pt}%
\definecolor{currentstroke}{rgb}{0.000000,0.000000,0.000000}%
\pgfsetstrokecolor{currentstroke}%
\pgfsetdash{}{0pt}%
\pgfpathmoveto{\pgfqpoint{3.876477in}{2.076667in}}%
\pgfpathcurveto{\pgfqpoint{3.882002in}{2.076667in}}{\pgfqpoint{3.887302in}{2.078862in}}{\pgfqpoint{3.891209in}{2.082769in}}%
\pgfpathcurveto{\pgfqpoint{3.895115in}{2.086675in}}{\pgfqpoint{3.897311in}{2.091975in}}{\pgfqpoint{3.897311in}{2.097500in}}%
\pgfpathcurveto{\pgfqpoint{3.897311in}{2.103025in}}{\pgfqpoint{3.895115in}{2.108325in}}{\pgfqpoint{3.891209in}{2.112231in}}%
\pgfpathcurveto{\pgfqpoint{3.887302in}{2.116138in}}{\pgfqpoint{3.882002in}{2.118333in}}{\pgfqpoint{3.876477in}{2.118333in}}%
\pgfpathcurveto{\pgfqpoint{3.870952in}{2.118333in}}{\pgfqpoint{3.865653in}{2.116138in}}{\pgfqpoint{3.861746in}{2.112231in}}%
\pgfpathcurveto{\pgfqpoint{3.857839in}{2.108325in}}{\pgfqpoint{3.855644in}{2.103025in}}{\pgfqpoint{3.855644in}{2.097500in}}%
\pgfpathcurveto{\pgfqpoint{3.855644in}{2.091975in}}{\pgfqpoint{3.857839in}{2.086675in}}{\pgfqpoint{3.861746in}{2.082769in}}%
\pgfpathcurveto{\pgfqpoint{3.865653in}{2.078862in}}{\pgfqpoint{3.870952in}{2.076667in}}{\pgfqpoint{3.876477in}{2.076667in}}%
\pgfpathclose%
\pgfusepath{stroke,fill}%
\end{pgfscope}%
\begin{pgfscope}%
\pgfpathrectangle{\pgfqpoint{0.562500in}{0.275000in}}{\pgfqpoint{3.487500in}{1.925000in}}%
\pgfusepath{clip}%
\pgfsetbuttcap%
\pgfsetroundjoin%
\definecolor{currentfill}{rgb}{0.000000,0.000000,0.000000}%
\pgfsetfillcolor{currentfill}%
\pgfsetlinewidth{1.003750pt}%
\definecolor{currentstroke}{rgb}{0.000000,0.000000,0.000000}%
\pgfsetstrokecolor{currentstroke}%
\pgfsetdash{}{0pt}%
\pgfpathmoveto{\pgfqpoint{3.876477in}{1.216635in}}%
\pgfpathcurveto{\pgfqpoint{3.882002in}{1.216635in}}{\pgfqpoint{3.887302in}{1.218830in}}{\pgfqpoint{3.891209in}{1.222736in}}%
\pgfpathcurveto{\pgfqpoint{3.895115in}{1.226643in}}{\pgfqpoint{3.897311in}{1.231943in}}{\pgfqpoint{3.897311in}{1.237468in}}%
\pgfpathcurveto{\pgfqpoint{3.897311in}{1.242993in}}{\pgfqpoint{3.895115in}{1.248292in}}{\pgfqpoint{3.891209in}{1.252199in}}%
\pgfpathcurveto{\pgfqpoint{3.887302in}{1.256106in}}{\pgfqpoint{3.882002in}{1.258301in}}{\pgfqpoint{3.876477in}{1.258301in}}%
\pgfpathcurveto{\pgfqpoint{3.870952in}{1.258301in}}{\pgfqpoint{3.865653in}{1.256106in}}{\pgfqpoint{3.861746in}{1.252199in}}%
\pgfpathcurveto{\pgfqpoint{3.857839in}{1.248292in}}{\pgfqpoint{3.855644in}{1.242993in}}{\pgfqpoint{3.855644in}{1.237468in}}%
\pgfpathcurveto{\pgfqpoint{3.855644in}{1.231943in}}{\pgfqpoint{3.857839in}{1.226643in}}{\pgfqpoint{3.861746in}{1.222736in}}%
\pgfpathcurveto{\pgfqpoint{3.865653in}{1.218830in}}{\pgfqpoint{3.870952in}{1.216635in}}{\pgfqpoint{3.876477in}{1.216635in}}%
\pgfpathclose%
\pgfusepath{stroke,fill}%
\end{pgfscope}%
\begin{pgfscope}%
\pgfpathrectangle{\pgfqpoint{0.562500in}{0.275000in}}{\pgfqpoint{3.487500in}{1.925000in}}%
\pgfusepath{clip}%
\pgfsetbuttcap%
\pgfsetroundjoin%
\definecolor{currentfill}{rgb}{0.000000,0.000000,0.000000}%
\pgfsetfillcolor{currentfill}%
\pgfsetlinewidth{1.003750pt}%
\definecolor{currentstroke}{rgb}{0.000000,0.000000,0.000000}%
\pgfsetstrokecolor{currentstroke}%
\pgfsetdash{}{0pt}%
\pgfpathmoveto{\pgfqpoint{3.876477in}{1.216635in}}%
\pgfpathcurveto{\pgfqpoint{3.882002in}{1.216635in}}{\pgfqpoint{3.887302in}{1.218830in}}{\pgfqpoint{3.891209in}{1.222736in}}%
\pgfpathcurveto{\pgfqpoint{3.895115in}{1.226643in}}{\pgfqpoint{3.897311in}{1.231943in}}{\pgfqpoint{3.897311in}{1.237468in}}%
\pgfpathcurveto{\pgfqpoint{3.897311in}{1.242993in}}{\pgfqpoint{3.895115in}{1.248292in}}{\pgfqpoint{3.891209in}{1.252199in}}%
\pgfpathcurveto{\pgfqpoint{3.887302in}{1.256106in}}{\pgfqpoint{3.882002in}{1.258301in}}{\pgfqpoint{3.876477in}{1.258301in}}%
\pgfpathcurveto{\pgfqpoint{3.870952in}{1.258301in}}{\pgfqpoint{3.865653in}{1.256106in}}{\pgfqpoint{3.861746in}{1.252199in}}%
\pgfpathcurveto{\pgfqpoint{3.857839in}{1.248292in}}{\pgfqpoint{3.855644in}{1.242993in}}{\pgfqpoint{3.855644in}{1.237468in}}%
\pgfpathcurveto{\pgfqpoint{3.855644in}{1.231943in}}{\pgfqpoint{3.857839in}{1.226643in}}{\pgfqpoint{3.861746in}{1.222736in}}%
\pgfpathcurveto{\pgfqpoint{3.865653in}{1.218830in}}{\pgfqpoint{3.870952in}{1.216635in}}{\pgfqpoint{3.876477in}{1.216635in}}%
\pgfpathclose%
\pgfusepath{stroke,fill}%
\end{pgfscope}%
\begin{pgfscope}%
\pgfpathrectangle{\pgfqpoint{0.562500in}{0.275000in}}{\pgfqpoint{3.487500in}{1.925000in}}%
\pgfusepath{clip}%
\pgfsetbuttcap%
\pgfsetroundjoin%
\definecolor{currentfill}{rgb}{0.000000,0.000000,0.000000}%
\pgfsetfillcolor{currentfill}%
\pgfsetlinewidth{1.003750pt}%
\definecolor{currentstroke}{rgb}{0.000000,0.000000,0.000000}%
\pgfsetstrokecolor{currentstroke}%
\pgfsetdash{}{0pt}%
\pgfpathmoveto{\pgfqpoint{3.876477in}{1.216635in}}%
\pgfpathcurveto{\pgfqpoint{3.882002in}{1.216635in}}{\pgfqpoint{3.887302in}{1.218830in}}{\pgfqpoint{3.891209in}{1.222736in}}%
\pgfpathcurveto{\pgfqpoint{3.895115in}{1.226643in}}{\pgfqpoint{3.897311in}{1.231943in}}{\pgfqpoint{3.897311in}{1.237468in}}%
\pgfpathcurveto{\pgfqpoint{3.897311in}{1.242993in}}{\pgfqpoint{3.895115in}{1.248292in}}{\pgfqpoint{3.891209in}{1.252199in}}%
\pgfpathcurveto{\pgfqpoint{3.887302in}{1.256106in}}{\pgfqpoint{3.882002in}{1.258301in}}{\pgfqpoint{3.876477in}{1.258301in}}%
\pgfpathcurveto{\pgfqpoint{3.870952in}{1.258301in}}{\pgfqpoint{3.865653in}{1.256106in}}{\pgfqpoint{3.861746in}{1.252199in}}%
\pgfpathcurveto{\pgfqpoint{3.857839in}{1.248292in}}{\pgfqpoint{3.855644in}{1.242993in}}{\pgfqpoint{3.855644in}{1.237468in}}%
\pgfpathcurveto{\pgfqpoint{3.855644in}{1.231943in}}{\pgfqpoint{3.857839in}{1.226643in}}{\pgfqpoint{3.861746in}{1.222736in}}%
\pgfpathcurveto{\pgfqpoint{3.865653in}{1.218830in}}{\pgfqpoint{3.870952in}{1.216635in}}{\pgfqpoint{3.876477in}{1.216635in}}%
\pgfpathclose%
\pgfusepath{stroke,fill}%
\end{pgfscope}%
\begin{pgfscope}%
\pgfpathrectangle{\pgfqpoint{0.562500in}{0.275000in}}{\pgfqpoint{3.487500in}{1.925000in}}%
\pgfusepath{clip}%
\pgfsetbuttcap%
\pgfsetroundjoin%
\definecolor{currentfill}{rgb}{0.000000,0.000000,0.000000}%
\pgfsetfillcolor{currentfill}%
\pgfsetlinewidth{1.003750pt}%
\definecolor{currentstroke}{rgb}{0.000000,0.000000,0.000000}%
\pgfsetstrokecolor{currentstroke}%
\pgfsetdash{}{0pt}%
\pgfpathmoveto{\pgfqpoint{3.876477in}{2.076667in}}%
\pgfpathcurveto{\pgfqpoint{3.882002in}{2.076667in}}{\pgfqpoint{3.887302in}{2.078862in}}{\pgfqpoint{3.891209in}{2.082769in}}%
\pgfpathcurveto{\pgfqpoint{3.895115in}{2.086675in}}{\pgfqpoint{3.897311in}{2.091975in}}{\pgfqpoint{3.897311in}{2.097500in}}%
\pgfpathcurveto{\pgfqpoint{3.897311in}{2.103025in}}{\pgfqpoint{3.895115in}{2.108325in}}{\pgfqpoint{3.891209in}{2.112231in}}%
\pgfpathcurveto{\pgfqpoint{3.887302in}{2.116138in}}{\pgfqpoint{3.882002in}{2.118333in}}{\pgfqpoint{3.876477in}{2.118333in}}%
\pgfpathcurveto{\pgfqpoint{3.870952in}{2.118333in}}{\pgfqpoint{3.865653in}{2.116138in}}{\pgfqpoint{3.861746in}{2.112231in}}%
\pgfpathcurveto{\pgfqpoint{3.857839in}{2.108325in}}{\pgfqpoint{3.855644in}{2.103025in}}{\pgfqpoint{3.855644in}{2.097500in}}%
\pgfpathcurveto{\pgfqpoint{3.855644in}{2.091975in}}{\pgfqpoint{3.857839in}{2.086675in}}{\pgfqpoint{3.861746in}{2.082769in}}%
\pgfpathcurveto{\pgfqpoint{3.865653in}{2.078862in}}{\pgfqpoint{3.870952in}{2.076667in}}{\pgfqpoint{3.876477in}{2.076667in}}%
\pgfpathclose%
\pgfusepath{stroke,fill}%
\end{pgfscope}%
\begin{pgfscope}%
\pgfpathrectangle{\pgfqpoint{0.562500in}{0.275000in}}{\pgfqpoint{3.487500in}{1.925000in}}%
\pgfusepath{clip}%
\pgfsetbuttcap%
\pgfsetroundjoin%
\definecolor{currentfill}{rgb}{0.000000,0.000000,0.000000}%
\pgfsetfillcolor{currentfill}%
\pgfsetlinewidth{1.003750pt}%
\definecolor{currentstroke}{rgb}{0.000000,0.000000,0.000000}%
\pgfsetstrokecolor{currentstroke}%
\pgfsetdash{}{0pt}%
\pgfpathmoveto{\pgfqpoint{3.876477in}{2.076667in}}%
\pgfpathcurveto{\pgfqpoint{3.882002in}{2.076667in}}{\pgfqpoint{3.887302in}{2.078862in}}{\pgfqpoint{3.891209in}{2.082769in}}%
\pgfpathcurveto{\pgfqpoint{3.895115in}{2.086675in}}{\pgfqpoint{3.897311in}{2.091975in}}{\pgfqpoint{3.897311in}{2.097500in}}%
\pgfpathcurveto{\pgfqpoint{3.897311in}{2.103025in}}{\pgfqpoint{3.895115in}{2.108325in}}{\pgfqpoint{3.891209in}{2.112231in}}%
\pgfpathcurveto{\pgfqpoint{3.887302in}{2.116138in}}{\pgfqpoint{3.882002in}{2.118333in}}{\pgfqpoint{3.876477in}{2.118333in}}%
\pgfpathcurveto{\pgfqpoint{3.870952in}{2.118333in}}{\pgfqpoint{3.865653in}{2.116138in}}{\pgfqpoint{3.861746in}{2.112231in}}%
\pgfpathcurveto{\pgfqpoint{3.857839in}{2.108325in}}{\pgfqpoint{3.855644in}{2.103025in}}{\pgfqpoint{3.855644in}{2.097500in}}%
\pgfpathcurveto{\pgfqpoint{3.855644in}{2.091975in}}{\pgfqpoint{3.857839in}{2.086675in}}{\pgfqpoint{3.861746in}{2.082769in}}%
\pgfpathcurveto{\pgfqpoint{3.865653in}{2.078862in}}{\pgfqpoint{3.870952in}{2.076667in}}{\pgfqpoint{3.876477in}{2.076667in}}%
\pgfpathclose%
\pgfusepath{stroke,fill}%
\end{pgfscope}%
\begin{pgfscope}%
\pgfpathrectangle{\pgfqpoint{0.562500in}{0.275000in}}{\pgfqpoint{3.487500in}{1.925000in}}%
\pgfusepath{clip}%
\pgfsetbuttcap%
\pgfsetroundjoin%
\definecolor{currentfill}{rgb}{0.000000,0.000000,0.000000}%
\pgfsetfillcolor{currentfill}%
\pgfsetlinewidth{1.003750pt}%
\definecolor{currentstroke}{rgb}{0.000000,0.000000,0.000000}%
\pgfsetstrokecolor{currentstroke}%
\pgfsetdash{}{0pt}%
\pgfpathmoveto{\pgfqpoint{3.876477in}{1.216635in}}%
\pgfpathcurveto{\pgfqpoint{3.882002in}{1.216635in}}{\pgfqpoint{3.887302in}{1.218830in}}{\pgfqpoint{3.891209in}{1.222736in}}%
\pgfpathcurveto{\pgfqpoint{3.895115in}{1.226643in}}{\pgfqpoint{3.897311in}{1.231943in}}{\pgfqpoint{3.897311in}{1.237468in}}%
\pgfpathcurveto{\pgfqpoint{3.897311in}{1.242993in}}{\pgfqpoint{3.895115in}{1.248292in}}{\pgfqpoint{3.891209in}{1.252199in}}%
\pgfpathcurveto{\pgfqpoint{3.887302in}{1.256106in}}{\pgfqpoint{3.882002in}{1.258301in}}{\pgfqpoint{3.876477in}{1.258301in}}%
\pgfpathcurveto{\pgfqpoint{3.870952in}{1.258301in}}{\pgfqpoint{3.865653in}{1.256106in}}{\pgfqpoint{3.861746in}{1.252199in}}%
\pgfpathcurveto{\pgfqpoint{3.857839in}{1.248292in}}{\pgfqpoint{3.855644in}{1.242993in}}{\pgfqpoint{3.855644in}{1.237468in}}%
\pgfpathcurveto{\pgfqpoint{3.855644in}{1.231943in}}{\pgfqpoint{3.857839in}{1.226643in}}{\pgfqpoint{3.861746in}{1.222736in}}%
\pgfpathcurveto{\pgfqpoint{3.865653in}{1.218830in}}{\pgfqpoint{3.870952in}{1.216635in}}{\pgfqpoint{3.876477in}{1.216635in}}%
\pgfpathclose%
\pgfusepath{stroke,fill}%
\end{pgfscope}%
\begin{pgfscope}%
\pgfpathrectangle{\pgfqpoint{0.562500in}{0.275000in}}{\pgfqpoint{3.487500in}{1.925000in}}%
\pgfusepath{clip}%
\pgfsetbuttcap%
\pgfsetroundjoin%
\definecolor{currentfill}{rgb}{0.000000,0.000000,0.000000}%
\pgfsetfillcolor{currentfill}%
\pgfsetlinewidth{1.003750pt}%
\definecolor{currentstroke}{rgb}{0.000000,0.000000,0.000000}%
\pgfsetstrokecolor{currentstroke}%
\pgfsetdash{}{0pt}%
\pgfpathmoveto{\pgfqpoint{3.876477in}{2.076667in}}%
\pgfpathcurveto{\pgfqpoint{3.882002in}{2.076667in}}{\pgfqpoint{3.887302in}{2.078862in}}{\pgfqpoint{3.891209in}{2.082769in}}%
\pgfpathcurveto{\pgfqpoint{3.895115in}{2.086675in}}{\pgfqpoint{3.897311in}{2.091975in}}{\pgfqpoint{3.897311in}{2.097500in}}%
\pgfpathcurveto{\pgfqpoint{3.897311in}{2.103025in}}{\pgfqpoint{3.895115in}{2.108325in}}{\pgfqpoint{3.891209in}{2.112231in}}%
\pgfpathcurveto{\pgfqpoint{3.887302in}{2.116138in}}{\pgfqpoint{3.882002in}{2.118333in}}{\pgfqpoint{3.876477in}{2.118333in}}%
\pgfpathcurveto{\pgfqpoint{3.870952in}{2.118333in}}{\pgfqpoint{3.865653in}{2.116138in}}{\pgfqpoint{3.861746in}{2.112231in}}%
\pgfpathcurveto{\pgfqpoint{3.857839in}{2.108325in}}{\pgfqpoint{3.855644in}{2.103025in}}{\pgfqpoint{3.855644in}{2.097500in}}%
\pgfpathcurveto{\pgfqpoint{3.855644in}{2.091975in}}{\pgfqpoint{3.857839in}{2.086675in}}{\pgfqpoint{3.861746in}{2.082769in}}%
\pgfpathcurveto{\pgfqpoint{3.865653in}{2.078862in}}{\pgfqpoint{3.870952in}{2.076667in}}{\pgfqpoint{3.876477in}{2.076667in}}%
\pgfpathclose%
\pgfusepath{stroke,fill}%
\end{pgfscope}%
\begin{pgfscope}%
\pgfpathrectangle{\pgfqpoint{0.562500in}{0.275000in}}{\pgfqpoint{3.487500in}{1.925000in}}%
\pgfusepath{clip}%
\pgfsetbuttcap%
\pgfsetroundjoin%
\definecolor{currentfill}{rgb}{0.000000,0.000000,0.000000}%
\pgfsetfillcolor{currentfill}%
\pgfsetlinewidth{1.003750pt}%
\definecolor{currentstroke}{rgb}{0.000000,0.000000,0.000000}%
\pgfsetstrokecolor{currentstroke}%
\pgfsetdash{}{0pt}%
\pgfpathmoveto{\pgfqpoint{3.876477in}{2.076667in}}%
\pgfpathcurveto{\pgfqpoint{3.882002in}{2.076667in}}{\pgfqpoint{3.887302in}{2.078862in}}{\pgfqpoint{3.891209in}{2.082769in}}%
\pgfpathcurveto{\pgfqpoint{3.895115in}{2.086675in}}{\pgfqpoint{3.897311in}{2.091975in}}{\pgfqpoint{3.897311in}{2.097500in}}%
\pgfpathcurveto{\pgfqpoint{3.897311in}{2.103025in}}{\pgfqpoint{3.895115in}{2.108325in}}{\pgfqpoint{3.891209in}{2.112231in}}%
\pgfpathcurveto{\pgfqpoint{3.887302in}{2.116138in}}{\pgfqpoint{3.882002in}{2.118333in}}{\pgfqpoint{3.876477in}{2.118333in}}%
\pgfpathcurveto{\pgfqpoint{3.870952in}{2.118333in}}{\pgfqpoint{3.865653in}{2.116138in}}{\pgfqpoint{3.861746in}{2.112231in}}%
\pgfpathcurveto{\pgfqpoint{3.857839in}{2.108325in}}{\pgfqpoint{3.855644in}{2.103025in}}{\pgfqpoint{3.855644in}{2.097500in}}%
\pgfpathcurveto{\pgfqpoint{3.855644in}{2.091975in}}{\pgfqpoint{3.857839in}{2.086675in}}{\pgfqpoint{3.861746in}{2.082769in}}%
\pgfpathcurveto{\pgfqpoint{3.865653in}{2.078862in}}{\pgfqpoint{3.870952in}{2.076667in}}{\pgfqpoint{3.876477in}{2.076667in}}%
\pgfpathclose%
\pgfusepath{stroke,fill}%
\end{pgfscope}%
\begin{pgfscope}%
\pgfpathrectangle{\pgfqpoint{0.562500in}{0.275000in}}{\pgfqpoint{3.487500in}{1.925000in}}%
\pgfusepath{clip}%
\pgfsetbuttcap%
\pgfsetroundjoin%
\definecolor{currentfill}{rgb}{0.000000,0.000000,0.000000}%
\pgfsetfillcolor{currentfill}%
\pgfsetlinewidth{1.003750pt}%
\definecolor{currentstroke}{rgb}{0.000000,0.000000,0.000000}%
\pgfsetstrokecolor{currentstroke}%
\pgfsetdash{}{0pt}%
\pgfpathmoveto{\pgfqpoint{3.876477in}{2.076667in}}%
\pgfpathcurveto{\pgfqpoint{3.882002in}{2.076667in}}{\pgfqpoint{3.887302in}{2.078862in}}{\pgfqpoint{3.891209in}{2.082769in}}%
\pgfpathcurveto{\pgfqpoint{3.895115in}{2.086675in}}{\pgfqpoint{3.897311in}{2.091975in}}{\pgfqpoint{3.897311in}{2.097500in}}%
\pgfpathcurveto{\pgfqpoint{3.897311in}{2.103025in}}{\pgfqpoint{3.895115in}{2.108325in}}{\pgfqpoint{3.891209in}{2.112231in}}%
\pgfpathcurveto{\pgfqpoint{3.887302in}{2.116138in}}{\pgfqpoint{3.882002in}{2.118333in}}{\pgfqpoint{3.876477in}{2.118333in}}%
\pgfpathcurveto{\pgfqpoint{3.870952in}{2.118333in}}{\pgfqpoint{3.865653in}{2.116138in}}{\pgfqpoint{3.861746in}{2.112231in}}%
\pgfpathcurveto{\pgfqpoint{3.857839in}{2.108325in}}{\pgfqpoint{3.855644in}{2.103025in}}{\pgfqpoint{3.855644in}{2.097500in}}%
\pgfpathcurveto{\pgfqpoint{3.855644in}{2.091975in}}{\pgfqpoint{3.857839in}{2.086675in}}{\pgfqpoint{3.861746in}{2.082769in}}%
\pgfpathcurveto{\pgfqpoint{3.865653in}{2.078862in}}{\pgfqpoint{3.870952in}{2.076667in}}{\pgfqpoint{3.876477in}{2.076667in}}%
\pgfpathclose%
\pgfusepath{stroke,fill}%
\end{pgfscope}%
\begin{pgfscope}%
\pgfpathrectangle{\pgfqpoint{0.562500in}{0.275000in}}{\pgfqpoint{3.487500in}{1.925000in}}%
\pgfusepath{clip}%
\pgfsetbuttcap%
\pgfsetroundjoin%
\definecolor{currentfill}{rgb}{0.000000,0.000000,0.000000}%
\pgfsetfillcolor{currentfill}%
\pgfsetlinewidth{1.003750pt}%
\definecolor{currentstroke}{rgb}{0.000000,0.000000,0.000000}%
\pgfsetstrokecolor{currentstroke}%
\pgfsetdash{}{0pt}%
\pgfpathmoveto{\pgfqpoint{3.876477in}{1.216635in}}%
\pgfpathcurveto{\pgfqpoint{3.882002in}{1.216635in}}{\pgfqpoint{3.887302in}{1.218830in}}{\pgfqpoint{3.891209in}{1.222736in}}%
\pgfpathcurveto{\pgfqpoint{3.895115in}{1.226643in}}{\pgfqpoint{3.897311in}{1.231943in}}{\pgfqpoint{3.897311in}{1.237468in}}%
\pgfpathcurveto{\pgfqpoint{3.897311in}{1.242993in}}{\pgfqpoint{3.895115in}{1.248292in}}{\pgfqpoint{3.891209in}{1.252199in}}%
\pgfpathcurveto{\pgfqpoint{3.887302in}{1.256106in}}{\pgfqpoint{3.882002in}{1.258301in}}{\pgfqpoint{3.876477in}{1.258301in}}%
\pgfpathcurveto{\pgfqpoint{3.870952in}{1.258301in}}{\pgfqpoint{3.865653in}{1.256106in}}{\pgfqpoint{3.861746in}{1.252199in}}%
\pgfpathcurveto{\pgfqpoint{3.857839in}{1.248292in}}{\pgfqpoint{3.855644in}{1.242993in}}{\pgfqpoint{3.855644in}{1.237468in}}%
\pgfpathcurveto{\pgfqpoint{3.855644in}{1.231943in}}{\pgfqpoint{3.857839in}{1.226643in}}{\pgfqpoint{3.861746in}{1.222736in}}%
\pgfpathcurveto{\pgfqpoint{3.865653in}{1.218830in}}{\pgfqpoint{3.870952in}{1.216635in}}{\pgfqpoint{3.876477in}{1.216635in}}%
\pgfpathclose%
\pgfusepath{stroke,fill}%
\end{pgfscope}%
\begin{pgfscope}%
\pgfpathrectangle{\pgfqpoint{0.562500in}{0.275000in}}{\pgfqpoint{3.487500in}{1.925000in}}%
\pgfusepath{clip}%
\pgfsetbuttcap%
\pgfsetroundjoin%
\definecolor{currentfill}{rgb}{0.000000,0.000000,0.000000}%
\pgfsetfillcolor{currentfill}%
\pgfsetlinewidth{1.003750pt}%
\definecolor{currentstroke}{rgb}{0.000000,0.000000,0.000000}%
\pgfsetstrokecolor{currentstroke}%
\pgfsetdash{}{0pt}%
\pgfpathmoveto{\pgfqpoint{3.876477in}{1.216635in}}%
\pgfpathcurveto{\pgfqpoint{3.882002in}{1.216635in}}{\pgfqpoint{3.887302in}{1.218830in}}{\pgfqpoint{3.891209in}{1.222736in}}%
\pgfpathcurveto{\pgfqpoint{3.895115in}{1.226643in}}{\pgfqpoint{3.897311in}{1.231943in}}{\pgfqpoint{3.897311in}{1.237468in}}%
\pgfpathcurveto{\pgfqpoint{3.897311in}{1.242993in}}{\pgfqpoint{3.895115in}{1.248292in}}{\pgfqpoint{3.891209in}{1.252199in}}%
\pgfpathcurveto{\pgfqpoint{3.887302in}{1.256106in}}{\pgfqpoint{3.882002in}{1.258301in}}{\pgfqpoint{3.876477in}{1.258301in}}%
\pgfpathcurveto{\pgfqpoint{3.870952in}{1.258301in}}{\pgfqpoint{3.865653in}{1.256106in}}{\pgfqpoint{3.861746in}{1.252199in}}%
\pgfpathcurveto{\pgfqpoint{3.857839in}{1.248292in}}{\pgfqpoint{3.855644in}{1.242993in}}{\pgfqpoint{3.855644in}{1.237468in}}%
\pgfpathcurveto{\pgfqpoint{3.855644in}{1.231943in}}{\pgfqpoint{3.857839in}{1.226643in}}{\pgfqpoint{3.861746in}{1.222736in}}%
\pgfpathcurveto{\pgfqpoint{3.865653in}{1.218830in}}{\pgfqpoint{3.870952in}{1.216635in}}{\pgfqpoint{3.876477in}{1.216635in}}%
\pgfpathclose%
\pgfusepath{stroke,fill}%
\end{pgfscope}%
\begin{pgfscope}%
\pgfpathrectangle{\pgfqpoint{0.562500in}{0.275000in}}{\pgfqpoint{3.487500in}{1.925000in}}%
\pgfusepath{clip}%
\pgfsetbuttcap%
\pgfsetroundjoin%
\definecolor{currentfill}{rgb}{0.000000,0.000000,0.000000}%
\pgfsetfillcolor{currentfill}%
\pgfsetlinewidth{1.003750pt}%
\definecolor{currentstroke}{rgb}{0.000000,0.000000,0.000000}%
\pgfsetstrokecolor{currentstroke}%
\pgfsetdash{}{0pt}%
\pgfpathmoveto{\pgfqpoint{3.876477in}{2.076667in}}%
\pgfpathcurveto{\pgfqpoint{3.882002in}{2.076667in}}{\pgfqpoint{3.887302in}{2.078862in}}{\pgfqpoint{3.891209in}{2.082769in}}%
\pgfpathcurveto{\pgfqpoint{3.895115in}{2.086675in}}{\pgfqpoint{3.897311in}{2.091975in}}{\pgfqpoint{3.897311in}{2.097500in}}%
\pgfpathcurveto{\pgfqpoint{3.897311in}{2.103025in}}{\pgfqpoint{3.895115in}{2.108325in}}{\pgfqpoint{3.891209in}{2.112231in}}%
\pgfpathcurveto{\pgfqpoint{3.887302in}{2.116138in}}{\pgfqpoint{3.882002in}{2.118333in}}{\pgfqpoint{3.876477in}{2.118333in}}%
\pgfpathcurveto{\pgfqpoint{3.870952in}{2.118333in}}{\pgfqpoint{3.865653in}{2.116138in}}{\pgfqpoint{3.861746in}{2.112231in}}%
\pgfpathcurveto{\pgfqpoint{3.857839in}{2.108325in}}{\pgfqpoint{3.855644in}{2.103025in}}{\pgfqpoint{3.855644in}{2.097500in}}%
\pgfpathcurveto{\pgfqpoint{3.855644in}{2.091975in}}{\pgfqpoint{3.857839in}{2.086675in}}{\pgfqpoint{3.861746in}{2.082769in}}%
\pgfpathcurveto{\pgfqpoint{3.865653in}{2.078862in}}{\pgfqpoint{3.870952in}{2.076667in}}{\pgfqpoint{3.876477in}{2.076667in}}%
\pgfpathclose%
\pgfusepath{stroke,fill}%
\end{pgfscope}%
\begin{pgfscope}%
\pgfpathrectangle{\pgfqpoint{0.562500in}{0.275000in}}{\pgfqpoint{3.487500in}{1.925000in}}%
\pgfusepath{clip}%
\pgfsetbuttcap%
\pgfsetroundjoin%
\definecolor{currentfill}{rgb}{0.000000,0.000000,0.000000}%
\pgfsetfillcolor{currentfill}%
\pgfsetlinewidth{1.003750pt}%
\definecolor{currentstroke}{rgb}{0.000000,0.000000,0.000000}%
\pgfsetstrokecolor{currentstroke}%
\pgfsetdash{}{0pt}%
\pgfpathmoveto{\pgfqpoint{3.876477in}{2.076667in}}%
\pgfpathcurveto{\pgfqpoint{3.882002in}{2.076667in}}{\pgfqpoint{3.887302in}{2.078862in}}{\pgfqpoint{3.891209in}{2.082769in}}%
\pgfpathcurveto{\pgfqpoint{3.895115in}{2.086675in}}{\pgfqpoint{3.897311in}{2.091975in}}{\pgfqpoint{3.897311in}{2.097500in}}%
\pgfpathcurveto{\pgfqpoint{3.897311in}{2.103025in}}{\pgfqpoint{3.895115in}{2.108325in}}{\pgfqpoint{3.891209in}{2.112231in}}%
\pgfpathcurveto{\pgfqpoint{3.887302in}{2.116138in}}{\pgfqpoint{3.882002in}{2.118333in}}{\pgfqpoint{3.876477in}{2.118333in}}%
\pgfpathcurveto{\pgfqpoint{3.870952in}{2.118333in}}{\pgfqpoint{3.865653in}{2.116138in}}{\pgfqpoint{3.861746in}{2.112231in}}%
\pgfpathcurveto{\pgfqpoint{3.857839in}{2.108325in}}{\pgfqpoint{3.855644in}{2.103025in}}{\pgfqpoint{3.855644in}{2.097500in}}%
\pgfpathcurveto{\pgfqpoint{3.855644in}{2.091975in}}{\pgfqpoint{3.857839in}{2.086675in}}{\pgfqpoint{3.861746in}{2.082769in}}%
\pgfpathcurveto{\pgfqpoint{3.865653in}{2.078862in}}{\pgfqpoint{3.870952in}{2.076667in}}{\pgfqpoint{3.876477in}{2.076667in}}%
\pgfpathclose%
\pgfusepath{stroke,fill}%
\end{pgfscope}%
\begin{pgfscope}%
\pgfpathrectangle{\pgfqpoint{0.562500in}{0.275000in}}{\pgfqpoint{3.487500in}{1.925000in}}%
\pgfusepath{clip}%
\pgfsetbuttcap%
\pgfsetroundjoin%
\definecolor{currentfill}{rgb}{0.000000,0.000000,0.000000}%
\pgfsetfillcolor{currentfill}%
\pgfsetlinewidth{1.003750pt}%
\definecolor{currentstroke}{rgb}{0.000000,0.000000,0.000000}%
\pgfsetstrokecolor{currentstroke}%
\pgfsetdash{}{0pt}%
\pgfpathmoveto{\pgfqpoint{3.876477in}{1.216635in}}%
\pgfpathcurveto{\pgfqpoint{3.882002in}{1.216635in}}{\pgfqpoint{3.887302in}{1.218830in}}{\pgfqpoint{3.891209in}{1.222736in}}%
\pgfpathcurveto{\pgfqpoint{3.895115in}{1.226643in}}{\pgfqpoint{3.897311in}{1.231943in}}{\pgfqpoint{3.897311in}{1.237468in}}%
\pgfpathcurveto{\pgfqpoint{3.897311in}{1.242993in}}{\pgfqpoint{3.895115in}{1.248292in}}{\pgfqpoint{3.891209in}{1.252199in}}%
\pgfpathcurveto{\pgfqpoint{3.887302in}{1.256106in}}{\pgfqpoint{3.882002in}{1.258301in}}{\pgfqpoint{3.876477in}{1.258301in}}%
\pgfpathcurveto{\pgfqpoint{3.870952in}{1.258301in}}{\pgfqpoint{3.865653in}{1.256106in}}{\pgfqpoint{3.861746in}{1.252199in}}%
\pgfpathcurveto{\pgfqpoint{3.857839in}{1.248292in}}{\pgfqpoint{3.855644in}{1.242993in}}{\pgfqpoint{3.855644in}{1.237468in}}%
\pgfpathcurveto{\pgfqpoint{3.855644in}{1.231943in}}{\pgfqpoint{3.857839in}{1.226643in}}{\pgfqpoint{3.861746in}{1.222736in}}%
\pgfpathcurveto{\pgfqpoint{3.865653in}{1.218830in}}{\pgfqpoint{3.870952in}{1.216635in}}{\pgfqpoint{3.876477in}{1.216635in}}%
\pgfpathclose%
\pgfusepath{stroke,fill}%
\end{pgfscope}%
\begin{pgfscope}%
\pgfpathrectangle{\pgfqpoint{0.562500in}{0.275000in}}{\pgfqpoint{3.487500in}{1.925000in}}%
\pgfusepath{clip}%
\pgfsetbuttcap%
\pgfsetroundjoin%
\definecolor{currentfill}{rgb}{0.000000,0.000000,0.000000}%
\pgfsetfillcolor{currentfill}%
\pgfsetlinewidth{1.003750pt}%
\definecolor{currentstroke}{rgb}{0.000000,0.000000,0.000000}%
\pgfsetstrokecolor{currentstroke}%
\pgfsetdash{}{0pt}%
\pgfpathmoveto{\pgfqpoint{3.876477in}{1.216635in}}%
\pgfpathcurveto{\pgfqpoint{3.882002in}{1.216635in}}{\pgfqpoint{3.887302in}{1.218830in}}{\pgfqpoint{3.891209in}{1.222736in}}%
\pgfpathcurveto{\pgfqpoint{3.895115in}{1.226643in}}{\pgfqpoint{3.897311in}{1.231943in}}{\pgfqpoint{3.897311in}{1.237468in}}%
\pgfpathcurveto{\pgfqpoint{3.897311in}{1.242993in}}{\pgfqpoint{3.895115in}{1.248292in}}{\pgfqpoint{3.891209in}{1.252199in}}%
\pgfpathcurveto{\pgfqpoint{3.887302in}{1.256106in}}{\pgfqpoint{3.882002in}{1.258301in}}{\pgfqpoint{3.876477in}{1.258301in}}%
\pgfpathcurveto{\pgfqpoint{3.870952in}{1.258301in}}{\pgfqpoint{3.865653in}{1.256106in}}{\pgfqpoint{3.861746in}{1.252199in}}%
\pgfpathcurveto{\pgfqpoint{3.857839in}{1.248292in}}{\pgfqpoint{3.855644in}{1.242993in}}{\pgfqpoint{3.855644in}{1.237468in}}%
\pgfpathcurveto{\pgfqpoint{3.855644in}{1.231943in}}{\pgfqpoint{3.857839in}{1.226643in}}{\pgfqpoint{3.861746in}{1.222736in}}%
\pgfpathcurveto{\pgfqpoint{3.865653in}{1.218830in}}{\pgfqpoint{3.870952in}{1.216635in}}{\pgfqpoint{3.876477in}{1.216635in}}%
\pgfpathclose%
\pgfusepath{stroke,fill}%
\end{pgfscope}%
\begin{pgfscope}%
\pgfpathrectangle{\pgfqpoint{0.562500in}{0.275000in}}{\pgfqpoint{3.487500in}{1.925000in}}%
\pgfusepath{clip}%
\pgfsetbuttcap%
\pgfsetroundjoin%
\definecolor{currentfill}{rgb}{0.000000,0.000000,0.000000}%
\pgfsetfillcolor{currentfill}%
\pgfsetlinewidth{1.003750pt}%
\definecolor{currentstroke}{rgb}{0.000000,0.000000,0.000000}%
\pgfsetstrokecolor{currentstroke}%
\pgfsetdash{}{0pt}%
\pgfpathmoveto{\pgfqpoint{3.876477in}{2.076667in}}%
\pgfpathcurveto{\pgfqpoint{3.882002in}{2.076667in}}{\pgfqpoint{3.887302in}{2.078862in}}{\pgfqpoint{3.891209in}{2.082769in}}%
\pgfpathcurveto{\pgfqpoint{3.895115in}{2.086675in}}{\pgfqpoint{3.897311in}{2.091975in}}{\pgfqpoint{3.897311in}{2.097500in}}%
\pgfpathcurveto{\pgfqpoint{3.897311in}{2.103025in}}{\pgfqpoint{3.895115in}{2.108325in}}{\pgfqpoint{3.891209in}{2.112231in}}%
\pgfpathcurveto{\pgfqpoint{3.887302in}{2.116138in}}{\pgfqpoint{3.882002in}{2.118333in}}{\pgfqpoint{3.876477in}{2.118333in}}%
\pgfpathcurveto{\pgfqpoint{3.870952in}{2.118333in}}{\pgfqpoint{3.865653in}{2.116138in}}{\pgfqpoint{3.861746in}{2.112231in}}%
\pgfpathcurveto{\pgfqpoint{3.857839in}{2.108325in}}{\pgfqpoint{3.855644in}{2.103025in}}{\pgfqpoint{3.855644in}{2.097500in}}%
\pgfpathcurveto{\pgfqpoint{3.855644in}{2.091975in}}{\pgfqpoint{3.857839in}{2.086675in}}{\pgfqpoint{3.861746in}{2.082769in}}%
\pgfpathcurveto{\pgfqpoint{3.865653in}{2.078862in}}{\pgfqpoint{3.870952in}{2.076667in}}{\pgfqpoint{3.876477in}{2.076667in}}%
\pgfpathclose%
\pgfusepath{stroke,fill}%
\end{pgfscope}%
\begin{pgfscope}%
\pgfpathrectangle{\pgfqpoint{0.562500in}{0.275000in}}{\pgfqpoint{3.487500in}{1.925000in}}%
\pgfusepath{clip}%
\pgfsetbuttcap%
\pgfsetroundjoin%
\definecolor{currentfill}{rgb}{0.000000,0.000000,0.000000}%
\pgfsetfillcolor{currentfill}%
\pgfsetlinewidth{1.003750pt}%
\definecolor{currentstroke}{rgb}{0.000000,0.000000,0.000000}%
\pgfsetstrokecolor{currentstroke}%
\pgfsetdash{}{0pt}%
\pgfpathmoveto{\pgfqpoint{3.876477in}{1.216635in}}%
\pgfpathcurveto{\pgfqpoint{3.882002in}{1.216635in}}{\pgfqpoint{3.887302in}{1.218830in}}{\pgfqpoint{3.891209in}{1.222736in}}%
\pgfpathcurveto{\pgfqpoint{3.895115in}{1.226643in}}{\pgfqpoint{3.897311in}{1.231943in}}{\pgfqpoint{3.897311in}{1.237468in}}%
\pgfpathcurveto{\pgfqpoint{3.897311in}{1.242993in}}{\pgfqpoint{3.895115in}{1.248292in}}{\pgfqpoint{3.891209in}{1.252199in}}%
\pgfpathcurveto{\pgfqpoint{3.887302in}{1.256106in}}{\pgfqpoint{3.882002in}{1.258301in}}{\pgfqpoint{3.876477in}{1.258301in}}%
\pgfpathcurveto{\pgfqpoint{3.870952in}{1.258301in}}{\pgfqpoint{3.865653in}{1.256106in}}{\pgfqpoint{3.861746in}{1.252199in}}%
\pgfpathcurveto{\pgfqpoint{3.857839in}{1.248292in}}{\pgfqpoint{3.855644in}{1.242993in}}{\pgfqpoint{3.855644in}{1.237468in}}%
\pgfpathcurveto{\pgfqpoint{3.855644in}{1.231943in}}{\pgfqpoint{3.857839in}{1.226643in}}{\pgfqpoint{3.861746in}{1.222736in}}%
\pgfpathcurveto{\pgfqpoint{3.865653in}{1.218830in}}{\pgfqpoint{3.870952in}{1.216635in}}{\pgfqpoint{3.876477in}{1.216635in}}%
\pgfpathclose%
\pgfusepath{stroke,fill}%
\end{pgfscope}%
\begin{pgfscope}%
\pgfpathrectangle{\pgfqpoint{0.562500in}{0.275000in}}{\pgfqpoint{3.487500in}{1.925000in}}%
\pgfusepath{clip}%
\pgfsetbuttcap%
\pgfsetroundjoin%
\definecolor{currentfill}{rgb}{0.000000,0.000000,0.000000}%
\pgfsetfillcolor{currentfill}%
\pgfsetlinewidth{1.003750pt}%
\definecolor{currentstroke}{rgb}{0.000000,0.000000,0.000000}%
\pgfsetstrokecolor{currentstroke}%
\pgfsetdash{}{0pt}%
\pgfpathmoveto{\pgfqpoint{3.876477in}{2.076667in}}%
\pgfpathcurveto{\pgfqpoint{3.882002in}{2.076667in}}{\pgfqpoint{3.887302in}{2.078862in}}{\pgfqpoint{3.891209in}{2.082769in}}%
\pgfpathcurveto{\pgfqpoint{3.895115in}{2.086675in}}{\pgfqpoint{3.897311in}{2.091975in}}{\pgfqpoint{3.897311in}{2.097500in}}%
\pgfpathcurveto{\pgfqpoint{3.897311in}{2.103025in}}{\pgfqpoint{3.895115in}{2.108325in}}{\pgfqpoint{3.891209in}{2.112231in}}%
\pgfpathcurveto{\pgfqpoint{3.887302in}{2.116138in}}{\pgfqpoint{3.882002in}{2.118333in}}{\pgfqpoint{3.876477in}{2.118333in}}%
\pgfpathcurveto{\pgfqpoint{3.870952in}{2.118333in}}{\pgfqpoint{3.865653in}{2.116138in}}{\pgfqpoint{3.861746in}{2.112231in}}%
\pgfpathcurveto{\pgfqpoint{3.857839in}{2.108325in}}{\pgfqpoint{3.855644in}{2.103025in}}{\pgfqpoint{3.855644in}{2.097500in}}%
\pgfpathcurveto{\pgfqpoint{3.855644in}{2.091975in}}{\pgfqpoint{3.857839in}{2.086675in}}{\pgfqpoint{3.861746in}{2.082769in}}%
\pgfpathcurveto{\pgfqpoint{3.865653in}{2.078862in}}{\pgfqpoint{3.870952in}{2.076667in}}{\pgfqpoint{3.876477in}{2.076667in}}%
\pgfpathclose%
\pgfusepath{stroke,fill}%
\end{pgfscope}%
\begin{pgfscope}%
\pgfpathrectangle{\pgfqpoint{0.562500in}{0.275000in}}{\pgfqpoint{3.487500in}{1.925000in}}%
\pgfusepath{clip}%
\pgfsetbuttcap%
\pgfsetroundjoin%
\definecolor{currentfill}{rgb}{0.000000,0.000000,0.000000}%
\pgfsetfillcolor{currentfill}%
\pgfsetlinewidth{1.003750pt}%
\definecolor{currentstroke}{rgb}{0.000000,0.000000,0.000000}%
\pgfsetstrokecolor{currentstroke}%
\pgfsetdash{}{0pt}%
\pgfpathmoveto{\pgfqpoint{3.876477in}{1.216635in}}%
\pgfpathcurveto{\pgfqpoint{3.882002in}{1.216635in}}{\pgfqpoint{3.887302in}{1.218830in}}{\pgfqpoint{3.891209in}{1.222736in}}%
\pgfpathcurveto{\pgfqpoint{3.895115in}{1.226643in}}{\pgfqpoint{3.897311in}{1.231943in}}{\pgfqpoint{3.897311in}{1.237468in}}%
\pgfpathcurveto{\pgfqpoint{3.897311in}{1.242993in}}{\pgfqpoint{3.895115in}{1.248292in}}{\pgfqpoint{3.891209in}{1.252199in}}%
\pgfpathcurveto{\pgfqpoint{3.887302in}{1.256106in}}{\pgfqpoint{3.882002in}{1.258301in}}{\pgfqpoint{3.876477in}{1.258301in}}%
\pgfpathcurveto{\pgfqpoint{3.870952in}{1.258301in}}{\pgfqpoint{3.865653in}{1.256106in}}{\pgfqpoint{3.861746in}{1.252199in}}%
\pgfpathcurveto{\pgfqpoint{3.857839in}{1.248292in}}{\pgfqpoint{3.855644in}{1.242993in}}{\pgfqpoint{3.855644in}{1.237468in}}%
\pgfpathcurveto{\pgfqpoint{3.855644in}{1.231943in}}{\pgfqpoint{3.857839in}{1.226643in}}{\pgfqpoint{3.861746in}{1.222736in}}%
\pgfpathcurveto{\pgfqpoint{3.865653in}{1.218830in}}{\pgfqpoint{3.870952in}{1.216635in}}{\pgfqpoint{3.876477in}{1.216635in}}%
\pgfpathclose%
\pgfusepath{stroke,fill}%
\end{pgfscope}%
\begin{pgfscope}%
\pgfpathrectangle{\pgfqpoint{0.562500in}{0.275000in}}{\pgfqpoint{3.487500in}{1.925000in}}%
\pgfusepath{clip}%
\pgfsetbuttcap%
\pgfsetroundjoin%
\definecolor{currentfill}{rgb}{0.000000,0.000000,0.000000}%
\pgfsetfillcolor{currentfill}%
\pgfsetlinewidth{1.003750pt}%
\definecolor{currentstroke}{rgb}{0.000000,0.000000,0.000000}%
\pgfsetstrokecolor{currentstroke}%
\pgfsetdash{}{0pt}%
\pgfpathmoveto{\pgfqpoint{3.876477in}{1.216635in}}%
\pgfpathcurveto{\pgfqpoint{3.882002in}{1.216635in}}{\pgfqpoint{3.887302in}{1.218830in}}{\pgfqpoint{3.891209in}{1.222736in}}%
\pgfpathcurveto{\pgfqpoint{3.895115in}{1.226643in}}{\pgfqpoint{3.897311in}{1.231943in}}{\pgfqpoint{3.897311in}{1.237468in}}%
\pgfpathcurveto{\pgfqpoint{3.897311in}{1.242993in}}{\pgfqpoint{3.895115in}{1.248292in}}{\pgfqpoint{3.891209in}{1.252199in}}%
\pgfpathcurveto{\pgfqpoint{3.887302in}{1.256106in}}{\pgfqpoint{3.882002in}{1.258301in}}{\pgfqpoint{3.876477in}{1.258301in}}%
\pgfpathcurveto{\pgfqpoint{3.870952in}{1.258301in}}{\pgfqpoint{3.865653in}{1.256106in}}{\pgfqpoint{3.861746in}{1.252199in}}%
\pgfpathcurveto{\pgfqpoint{3.857839in}{1.248292in}}{\pgfqpoint{3.855644in}{1.242993in}}{\pgfqpoint{3.855644in}{1.237468in}}%
\pgfpathcurveto{\pgfqpoint{3.855644in}{1.231943in}}{\pgfqpoint{3.857839in}{1.226643in}}{\pgfqpoint{3.861746in}{1.222736in}}%
\pgfpathcurveto{\pgfqpoint{3.865653in}{1.218830in}}{\pgfqpoint{3.870952in}{1.216635in}}{\pgfqpoint{3.876477in}{1.216635in}}%
\pgfpathclose%
\pgfusepath{stroke,fill}%
\end{pgfscope}%
\begin{pgfscope}%
\pgfpathrectangle{\pgfqpoint{0.562500in}{0.275000in}}{\pgfqpoint{3.487500in}{1.925000in}}%
\pgfusepath{clip}%
\pgfsetbuttcap%
\pgfsetroundjoin%
\definecolor{currentfill}{rgb}{0.000000,0.000000,0.000000}%
\pgfsetfillcolor{currentfill}%
\pgfsetlinewidth{1.003750pt}%
\definecolor{currentstroke}{rgb}{0.000000,0.000000,0.000000}%
\pgfsetstrokecolor{currentstroke}%
\pgfsetdash{}{0pt}%
\pgfpathmoveto{\pgfqpoint{3.876477in}{2.076667in}}%
\pgfpathcurveto{\pgfqpoint{3.882002in}{2.076667in}}{\pgfqpoint{3.887302in}{2.078862in}}{\pgfqpoint{3.891209in}{2.082769in}}%
\pgfpathcurveto{\pgfqpoint{3.895115in}{2.086675in}}{\pgfqpoint{3.897311in}{2.091975in}}{\pgfqpoint{3.897311in}{2.097500in}}%
\pgfpathcurveto{\pgfqpoint{3.897311in}{2.103025in}}{\pgfqpoint{3.895115in}{2.108325in}}{\pgfqpoint{3.891209in}{2.112231in}}%
\pgfpathcurveto{\pgfqpoint{3.887302in}{2.116138in}}{\pgfqpoint{3.882002in}{2.118333in}}{\pgfqpoint{3.876477in}{2.118333in}}%
\pgfpathcurveto{\pgfqpoint{3.870952in}{2.118333in}}{\pgfqpoint{3.865653in}{2.116138in}}{\pgfqpoint{3.861746in}{2.112231in}}%
\pgfpathcurveto{\pgfqpoint{3.857839in}{2.108325in}}{\pgfqpoint{3.855644in}{2.103025in}}{\pgfqpoint{3.855644in}{2.097500in}}%
\pgfpathcurveto{\pgfqpoint{3.855644in}{2.091975in}}{\pgfqpoint{3.857839in}{2.086675in}}{\pgfqpoint{3.861746in}{2.082769in}}%
\pgfpathcurveto{\pgfqpoint{3.865653in}{2.078862in}}{\pgfqpoint{3.870952in}{2.076667in}}{\pgfqpoint{3.876477in}{2.076667in}}%
\pgfpathclose%
\pgfusepath{stroke,fill}%
\end{pgfscope}%
\begin{pgfscope}%
\pgfpathrectangle{\pgfqpoint{0.562500in}{0.275000in}}{\pgfqpoint{3.487500in}{1.925000in}}%
\pgfusepath{clip}%
\pgfsetbuttcap%
\pgfsetroundjoin%
\definecolor{currentfill}{rgb}{0.000000,0.000000,0.000000}%
\pgfsetfillcolor{currentfill}%
\pgfsetlinewidth{1.003750pt}%
\definecolor{currentstroke}{rgb}{0.000000,0.000000,0.000000}%
\pgfsetstrokecolor{currentstroke}%
\pgfsetdash{}{0pt}%
\pgfpathmoveto{\pgfqpoint{3.876477in}{1.216635in}}%
\pgfpathcurveto{\pgfqpoint{3.882002in}{1.216635in}}{\pgfqpoint{3.887302in}{1.218830in}}{\pgfqpoint{3.891209in}{1.222736in}}%
\pgfpathcurveto{\pgfqpoint{3.895115in}{1.226643in}}{\pgfqpoint{3.897311in}{1.231943in}}{\pgfqpoint{3.897311in}{1.237468in}}%
\pgfpathcurveto{\pgfqpoint{3.897311in}{1.242993in}}{\pgfqpoint{3.895115in}{1.248292in}}{\pgfqpoint{3.891209in}{1.252199in}}%
\pgfpathcurveto{\pgfqpoint{3.887302in}{1.256106in}}{\pgfqpoint{3.882002in}{1.258301in}}{\pgfqpoint{3.876477in}{1.258301in}}%
\pgfpathcurveto{\pgfqpoint{3.870952in}{1.258301in}}{\pgfqpoint{3.865653in}{1.256106in}}{\pgfqpoint{3.861746in}{1.252199in}}%
\pgfpathcurveto{\pgfqpoint{3.857839in}{1.248292in}}{\pgfqpoint{3.855644in}{1.242993in}}{\pgfqpoint{3.855644in}{1.237468in}}%
\pgfpathcurveto{\pgfqpoint{3.855644in}{1.231943in}}{\pgfqpoint{3.857839in}{1.226643in}}{\pgfqpoint{3.861746in}{1.222736in}}%
\pgfpathcurveto{\pgfqpoint{3.865653in}{1.218830in}}{\pgfqpoint{3.870952in}{1.216635in}}{\pgfqpoint{3.876477in}{1.216635in}}%
\pgfpathclose%
\pgfusepath{stroke,fill}%
\end{pgfscope}%
\begin{pgfscope}%
\pgfpathrectangle{\pgfqpoint{0.562500in}{0.275000in}}{\pgfqpoint{3.487500in}{1.925000in}}%
\pgfusepath{clip}%
\pgfsetbuttcap%
\pgfsetroundjoin%
\definecolor{currentfill}{rgb}{0.000000,0.000000,0.000000}%
\pgfsetfillcolor{currentfill}%
\pgfsetlinewidth{1.003750pt}%
\definecolor{currentstroke}{rgb}{0.000000,0.000000,0.000000}%
\pgfsetstrokecolor{currentstroke}%
\pgfsetdash{}{0pt}%
\pgfpathmoveto{\pgfqpoint{3.876477in}{2.076667in}}%
\pgfpathcurveto{\pgfqpoint{3.882002in}{2.076667in}}{\pgfqpoint{3.887302in}{2.078862in}}{\pgfqpoint{3.891209in}{2.082769in}}%
\pgfpathcurveto{\pgfqpoint{3.895115in}{2.086675in}}{\pgfqpoint{3.897311in}{2.091975in}}{\pgfqpoint{3.897311in}{2.097500in}}%
\pgfpathcurveto{\pgfqpoint{3.897311in}{2.103025in}}{\pgfqpoint{3.895115in}{2.108325in}}{\pgfqpoint{3.891209in}{2.112231in}}%
\pgfpathcurveto{\pgfqpoint{3.887302in}{2.116138in}}{\pgfqpoint{3.882002in}{2.118333in}}{\pgfqpoint{3.876477in}{2.118333in}}%
\pgfpathcurveto{\pgfqpoint{3.870952in}{2.118333in}}{\pgfqpoint{3.865653in}{2.116138in}}{\pgfqpoint{3.861746in}{2.112231in}}%
\pgfpathcurveto{\pgfqpoint{3.857839in}{2.108325in}}{\pgfqpoint{3.855644in}{2.103025in}}{\pgfqpoint{3.855644in}{2.097500in}}%
\pgfpathcurveto{\pgfqpoint{3.855644in}{2.091975in}}{\pgfqpoint{3.857839in}{2.086675in}}{\pgfqpoint{3.861746in}{2.082769in}}%
\pgfpathcurveto{\pgfqpoint{3.865653in}{2.078862in}}{\pgfqpoint{3.870952in}{2.076667in}}{\pgfqpoint{3.876477in}{2.076667in}}%
\pgfpathclose%
\pgfusepath{stroke,fill}%
\end{pgfscope}%
\begin{pgfscope}%
\pgfpathrectangle{\pgfqpoint{0.562500in}{0.275000in}}{\pgfqpoint{3.487500in}{1.925000in}}%
\pgfusepath{clip}%
\pgfsetbuttcap%
\pgfsetroundjoin%
\definecolor{currentfill}{rgb}{0.000000,0.000000,0.000000}%
\pgfsetfillcolor{currentfill}%
\pgfsetlinewidth{1.003750pt}%
\definecolor{currentstroke}{rgb}{0.000000,0.000000,0.000000}%
\pgfsetstrokecolor{currentstroke}%
\pgfsetdash{}{0pt}%
\pgfpathmoveto{\pgfqpoint{3.876477in}{2.076667in}}%
\pgfpathcurveto{\pgfqpoint{3.882002in}{2.076667in}}{\pgfqpoint{3.887302in}{2.078862in}}{\pgfqpoint{3.891209in}{2.082769in}}%
\pgfpathcurveto{\pgfqpoint{3.895115in}{2.086675in}}{\pgfqpoint{3.897311in}{2.091975in}}{\pgfqpoint{3.897311in}{2.097500in}}%
\pgfpathcurveto{\pgfqpoint{3.897311in}{2.103025in}}{\pgfqpoint{3.895115in}{2.108325in}}{\pgfqpoint{3.891209in}{2.112231in}}%
\pgfpathcurveto{\pgfqpoint{3.887302in}{2.116138in}}{\pgfqpoint{3.882002in}{2.118333in}}{\pgfqpoint{3.876477in}{2.118333in}}%
\pgfpathcurveto{\pgfqpoint{3.870952in}{2.118333in}}{\pgfqpoint{3.865653in}{2.116138in}}{\pgfqpoint{3.861746in}{2.112231in}}%
\pgfpathcurveto{\pgfqpoint{3.857839in}{2.108325in}}{\pgfqpoint{3.855644in}{2.103025in}}{\pgfqpoint{3.855644in}{2.097500in}}%
\pgfpathcurveto{\pgfqpoint{3.855644in}{2.091975in}}{\pgfqpoint{3.857839in}{2.086675in}}{\pgfqpoint{3.861746in}{2.082769in}}%
\pgfpathcurveto{\pgfqpoint{3.865653in}{2.078862in}}{\pgfqpoint{3.870952in}{2.076667in}}{\pgfqpoint{3.876477in}{2.076667in}}%
\pgfpathclose%
\pgfusepath{stroke,fill}%
\end{pgfscope}%
\begin{pgfscope}%
\pgfpathrectangle{\pgfqpoint{0.562500in}{0.275000in}}{\pgfqpoint{3.487500in}{1.925000in}}%
\pgfusepath{clip}%
\pgfsetbuttcap%
\pgfsetroundjoin%
\definecolor{currentfill}{rgb}{0.000000,0.000000,0.000000}%
\pgfsetfillcolor{currentfill}%
\pgfsetlinewidth{1.003750pt}%
\definecolor{currentstroke}{rgb}{0.000000,0.000000,0.000000}%
\pgfsetstrokecolor{currentstroke}%
\pgfsetdash{}{0pt}%
\pgfpathmoveto{\pgfqpoint{3.876477in}{1.216635in}}%
\pgfpathcurveto{\pgfqpoint{3.882002in}{1.216635in}}{\pgfqpoint{3.887302in}{1.218830in}}{\pgfqpoint{3.891209in}{1.222736in}}%
\pgfpathcurveto{\pgfqpoint{3.895115in}{1.226643in}}{\pgfqpoint{3.897311in}{1.231943in}}{\pgfqpoint{3.897311in}{1.237468in}}%
\pgfpathcurveto{\pgfqpoint{3.897311in}{1.242993in}}{\pgfqpoint{3.895115in}{1.248292in}}{\pgfqpoint{3.891209in}{1.252199in}}%
\pgfpathcurveto{\pgfqpoint{3.887302in}{1.256106in}}{\pgfqpoint{3.882002in}{1.258301in}}{\pgfqpoint{3.876477in}{1.258301in}}%
\pgfpathcurveto{\pgfqpoint{3.870952in}{1.258301in}}{\pgfqpoint{3.865653in}{1.256106in}}{\pgfqpoint{3.861746in}{1.252199in}}%
\pgfpathcurveto{\pgfqpoint{3.857839in}{1.248292in}}{\pgfqpoint{3.855644in}{1.242993in}}{\pgfqpoint{3.855644in}{1.237468in}}%
\pgfpathcurveto{\pgfqpoint{3.855644in}{1.231943in}}{\pgfqpoint{3.857839in}{1.226643in}}{\pgfqpoint{3.861746in}{1.222736in}}%
\pgfpathcurveto{\pgfqpoint{3.865653in}{1.218830in}}{\pgfqpoint{3.870952in}{1.216635in}}{\pgfqpoint{3.876477in}{1.216635in}}%
\pgfpathclose%
\pgfusepath{stroke,fill}%
\end{pgfscope}%
\begin{pgfscope}%
\pgfpathrectangle{\pgfqpoint{0.562500in}{0.275000in}}{\pgfqpoint{3.487500in}{1.925000in}}%
\pgfusepath{clip}%
\pgfsetbuttcap%
\pgfsetroundjoin%
\definecolor{currentfill}{rgb}{0.000000,0.000000,0.000000}%
\pgfsetfillcolor{currentfill}%
\pgfsetlinewidth{1.003750pt}%
\definecolor{currentstroke}{rgb}{0.000000,0.000000,0.000000}%
\pgfsetstrokecolor{currentstroke}%
\pgfsetdash{}{0pt}%
\pgfpathmoveto{\pgfqpoint{3.876477in}{1.216635in}}%
\pgfpathcurveto{\pgfqpoint{3.882002in}{1.216635in}}{\pgfqpoint{3.887302in}{1.218830in}}{\pgfqpoint{3.891209in}{1.222736in}}%
\pgfpathcurveto{\pgfqpoint{3.895115in}{1.226643in}}{\pgfqpoint{3.897311in}{1.231943in}}{\pgfqpoint{3.897311in}{1.237468in}}%
\pgfpathcurveto{\pgfqpoint{3.897311in}{1.242993in}}{\pgfqpoint{3.895115in}{1.248292in}}{\pgfqpoint{3.891209in}{1.252199in}}%
\pgfpathcurveto{\pgfqpoint{3.887302in}{1.256106in}}{\pgfqpoint{3.882002in}{1.258301in}}{\pgfqpoint{3.876477in}{1.258301in}}%
\pgfpathcurveto{\pgfqpoint{3.870952in}{1.258301in}}{\pgfqpoint{3.865653in}{1.256106in}}{\pgfqpoint{3.861746in}{1.252199in}}%
\pgfpathcurveto{\pgfqpoint{3.857839in}{1.248292in}}{\pgfqpoint{3.855644in}{1.242993in}}{\pgfqpoint{3.855644in}{1.237468in}}%
\pgfpathcurveto{\pgfqpoint{3.855644in}{1.231943in}}{\pgfqpoint{3.857839in}{1.226643in}}{\pgfqpoint{3.861746in}{1.222736in}}%
\pgfpathcurveto{\pgfqpoint{3.865653in}{1.218830in}}{\pgfqpoint{3.870952in}{1.216635in}}{\pgfqpoint{3.876477in}{1.216635in}}%
\pgfpathclose%
\pgfusepath{stroke,fill}%
\end{pgfscope}%
\begin{pgfscope}%
\pgfpathrectangle{\pgfqpoint{0.562500in}{0.275000in}}{\pgfqpoint{3.487500in}{1.925000in}}%
\pgfusepath{clip}%
\pgfsetbuttcap%
\pgfsetroundjoin%
\definecolor{currentfill}{rgb}{0.000000,0.000000,0.000000}%
\pgfsetfillcolor{currentfill}%
\pgfsetlinewidth{1.003750pt}%
\definecolor{currentstroke}{rgb}{0.000000,0.000000,0.000000}%
\pgfsetstrokecolor{currentstroke}%
\pgfsetdash{}{0pt}%
\pgfpathmoveto{\pgfqpoint{3.876477in}{1.216635in}}%
\pgfpathcurveto{\pgfqpoint{3.882002in}{1.216635in}}{\pgfqpoint{3.887302in}{1.218830in}}{\pgfqpoint{3.891209in}{1.222736in}}%
\pgfpathcurveto{\pgfqpoint{3.895115in}{1.226643in}}{\pgfqpoint{3.897311in}{1.231943in}}{\pgfqpoint{3.897311in}{1.237468in}}%
\pgfpathcurveto{\pgfqpoint{3.897311in}{1.242993in}}{\pgfqpoint{3.895115in}{1.248292in}}{\pgfqpoint{3.891209in}{1.252199in}}%
\pgfpathcurveto{\pgfqpoint{3.887302in}{1.256106in}}{\pgfqpoint{3.882002in}{1.258301in}}{\pgfqpoint{3.876477in}{1.258301in}}%
\pgfpathcurveto{\pgfqpoint{3.870952in}{1.258301in}}{\pgfqpoint{3.865653in}{1.256106in}}{\pgfqpoint{3.861746in}{1.252199in}}%
\pgfpathcurveto{\pgfqpoint{3.857839in}{1.248292in}}{\pgfqpoint{3.855644in}{1.242993in}}{\pgfqpoint{3.855644in}{1.237468in}}%
\pgfpathcurveto{\pgfqpoint{3.855644in}{1.231943in}}{\pgfqpoint{3.857839in}{1.226643in}}{\pgfqpoint{3.861746in}{1.222736in}}%
\pgfpathcurveto{\pgfqpoint{3.865653in}{1.218830in}}{\pgfqpoint{3.870952in}{1.216635in}}{\pgfqpoint{3.876477in}{1.216635in}}%
\pgfpathclose%
\pgfusepath{stroke,fill}%
\end{pgfscope}%
\begin{pgfscope}%
\pgfpathrectangle{\pgfqpoint{0.562500in}{0.275000in}}{\pgfqpoint{3.487500in}{1.925000in}}%
\pgfusepath{clip}%
\pgfsetbuttcap%
\pgfsetroundjoin%
\definecolor{currentfill}{rgb}{0.000000,0.000000,0.000000}%
\pgfsetfillcolor{currentfill}%
\pgfsetlinewidth{1.003750pt}%
\definecolor{currentstroke}{rgb}{0.000000,0.000000,0.000000}%
\pgfsetstrokecolor{currentstroke}%
\pgfsetdash{}{0pt}%
\pgfpathmoveto{\pgfqpoint{3.876477in}{1.216635in}}%
\pgfpathcurveto{\pgfqpoint{3.882002in}{1.216635in}}{\pgfqpoint{3.887302in}{1.218830in}}{\pgfqpoint{3.891209in}{1.222736in}}%
\pgfpathcurveto{\pgfqpoint{3.895115in}{1.226643in}}{\pgfqpoint{3.897311in}{1.231943in}}{\pgfqpoint{3.897311in}{1.237468in}}%
\pgfpathcurveto{\pgfqpoint{3.897311in}{1.242993in}}{\pgfqpoint{3.895115in}{1.248292in}}{\pgfqpoint{3.891209in}{1.252199in}}%
\pgfpathcurveto{\pgfqpoint{3.887302in}{1.256106in}}{\pgfqpoint{3.882002in}{1.258301in}}{\pgfqpoint{3.876477in}{1.258301in}}%
\pgfpathcurveto{\pgfqpoint{3.870952in}{1.258301in}}{\pgfqpoint{3.865653in}{1.256106in}}{\pgfqpoint{3.861746in}{1.252199in}}%
\pgfpathcurveto{\pgfqpoint{3.857839in}{1.248292in}}{\pgfqpoint{3.855644in}{1.242993in}}{\pgfqpoint{3.855644in}{1.237468in}}%
\pgfpathcurveto{\pgfqpoint{3.855644in}{1.231943in}}{\pgfqpoint{3.857839in}{1.226643in}}{\pgfqpoint{3.861746in}{1.222736in}}%
\pgfpathcurveto{\pgfqpoint{3.865653in}{1.218830in}}{\pgfqpoint{3.870952in}{1.216635in}}{\pgfqpoint{3.876477in}{1.216635in}}%
\pgfpathclose%
\pgfusepath{stroke,fill}%
\end{pgfscope}%
\begin{pgfscope}%
\pgfpathrectangle{\pgfqpoint{0.562500in}{0.275000in}}{\pgfqpoint{3.487500in}{1.925000in}}%
\pgfusepath{clip}%
\pgfsetbuttcap%
\pgfsetroundjoin%
\definecolor{currentfill}{rgb}{0.000000,0.000000,0.000000}%
\pgfsetfillcolor{currentfill}%
\pgfsetlinewidth{1.003750pt}%
\definecolor{currentstroke}{rgb}{0.000000,0.000000,0.000000}%
\pgfsetstrokecolor{currentstroke}%
\pgfsetdash{}{0pt}%
\pgfpathmoveto{\pgfqpoint{3.876477in}{2.076667in}}%
\pgfpathcurveto{\pgfqpoint{3.882002in}{2.076667in}}{\pgfqpoint{3.887302in}{2.078862in}}{\pgfqpoint{3.891209in}{2.082769in}}%
\pgfpathcurveto{\pgfqpoint{3.895115in}{2.086675in}}{\pgfqpoint{3.897311in}{2.091975in}}{\pgfqpoint{3.897311in}{2.097500in}}%
\pgfpathcurveto{\pgfqpoint{3.897311in}{2.103025in}}{\pgfqpoint{3.895115in}{2.108325in}}{\pgfqpoint{3.891209in}{2.112231in}}%
\pgfpathcurveto{\pgfqpoint{3.887302in}{2.116138in}}{\pgfqpoint{3.882002in}{2.118333in}}{\pgfqpoint{3.876477in}{2.118333in}}%
\pgfpathcurveto{\pgfqpoint{3.870952in}{2.118333in}}{\pgfqpoint{3.865653in}{2.116138in}}{\pgfqpoint{3.861746in}{2.112231in}}%
\pgfpathcurveto{\pgfqpoint{3.857839in}{2.108325in}}{\pgfqpoint{3.855644in}{2.103025in}}{\pgfqpoint{3.855644in}{2.097500in}}%
\pgfpathcurveto{\pgfqpoint{3.855644in}{2.091975in}}{\pgfqpoint{3.857839in}{2.086675in}}{\pgfqpoint{3.861746in}{2.082769in}}%
\pgfpathcurveto{\pgfqpoint{3.865653in}{2.078862in}}{\pgfqpoint{3.870952in}{2.076667in}}{\pgfqpoint{3.876477in}{2.076667in}}%
\pgfpathclose%
\pgfusepath{stroke,fill}%
\end{pgfscope}%
\begin{pgfscope}%
\pgfpathrectangle{\pgfqpoint{0.562500in}{0.275000in}}{\pgfqpoint{3.487500in}{1.925000in}}%
\pgfusepath{clip}%
\pgfsetbuttcap%
\pgfsetroundjoin%
\definecolor{currentfill}{rgb}{0.000000,0.000000,0.000000}%
\pgfsetfillcolor{currentfill}%
\pgfsetlinewidth{1.003750pt}%
\definecolor{currentstroke}{rgb}{0.000000,0.000000,0.000000}%
\pgfsetstrokecolor{currentstroke}%
\pgfsetdash{}{0pt}%
\pgfpathmoveto{\pgfqpoint{3.876477in}{2.076667in}}%
\pgfpathcurveto{\pgfqpoint{3.882002in}{2.076667in}}{\pgfqpoint{3.887302in}{2.078862in}}{\pgfqpoint{3.891209in}{2.082769in}}%
\pgfpathcurveto{\pgfqpoint{3.895115in}{2.086675in}}{\pgfqpoint{3.897311in}{2.091975in}}{\pgfqpoint{3.897311in}{2.097500in}}%
\pgfpathcurveto{\pgfqpoint{3.897311in}{2.103025in}}{\pgfqpoint{3.895115in}{2.108325in}}{\pgfqpoint{3.891209in}{2.112231in}}%
\pgfpathcurveto{\pgfqpoint{3.887302in}{2.116138in}}{\pgfqpoint{3.882002in}{2.118333in}}{\pgfqpoint{3.876477in}{2.118333in}}%
\pgfpathcurveto{\pgfqpoint{3.870952in}{2.118333in}}{\pgfqpoint{3.865653in}{2.116138in}}{\pgfqpoint{3.861746in}{2.112231in}}%
\pgfpathcurveto{\pgfqpoint{3.857839in}{2.108325in}}{\pgfqpoint{3.855644in}{2.103025in}}{\pgfqpoint{3.855644in}{2.097500in}}%
\pgfpathcurveto{\pgfqpoint{3.855644in}{2.091975in}}{\pgfqpoint{3.857839in}{2.086675in}}{\pgfqpoint{3.861746in}{2.082769in}}%
\pgfpathcurveto{\pgfqpoint{3.865653in}{2.078862in}}{\pgfqpoint{3.870952in}{2.076667in}}{\pgfqpoint{3.876477in}{2.076667in}}%
\pgfpathclose%
\pgfusepath{stroke,fill}%
\end{pgfscope}%
\begin{pgfscope}%
\pgfpathrectangle{\pgfqpoint{0.562500in}{0.275000in}}{\pgfqpoint{3.487500in}{1.925000in}}%
\pgfusepath{clip}%
\pgfsetbuttcap%
\pgfsetroundjoin%
\definecolor{currentfill}{rgb}{0.000000,0.000000,0.000000}%
\pgfsetfillcolor{currentfill}%
\pgfsetlinewidth{1.003750pt}%
\definecolor{currentstroke}{rgb}{0.000000,0.000000,0.000000}%
\pgfsetstrokecolor{currentstroke}%
\pgfsetdash{}{0pt}%
\pgfpathmoveto{\pgfqpoint{3.876477in}{2.076667in}}%
\pgfpathcurveto{\pgfqpoint{3.882002in}{2.076667in}}{\pgfqpoint{3.887302in}{2.078862in}}{\pgfqpoint{3.891209in}{2.082769in}}%
\pgfpathcurveto{\pgfqpoint{3.895115in}{2.086675in}}{\pgfqpoint{3.897311in}{2.091975in}}{\pgfqpoint{3.897311in}{2.097500in}}%
\pgfpathcurveto{\pgfqpoint{3.897311in}{2.103025in}}{\pgfqpoint{3.895115in}{2.108325in}}{\pgfqpoint{3.891209in}{2.112231in}}%
\pgfpathcurveto{\pgfqpoint{3.887302in}{2.116138in}}{\pgfqpoint{3.882002in}{2.118333in}}{\pgfqpoint{3.876477in}{2.118333in}}%
\pgfpathcurveto{\pgfqpoint{3.870952in}{2.118333in}}{\pgfqpoint{3.865653in}{2.116138in}}{\pgfqpoint{3.861746in}{2.112231in}}%
\pgfpathcurveto{\pgfqpoint{3.857839in}{2.108325in}}{\pgfqpoint{3.855644in}{2.103025in}}{\pgfqpoint{3.855644in}{2.097500in}}%
\pgfpathcurveto{\pgfqpoint{3.855644in}{2.091975in}}{\pgfqpoint{3.857839in}{2.086675in}}{\pgfqpoint{3.861746in}{2.082769in}}%
\pgfpathcurveto{\pgfqpoint{3.865653in}{2.078862in}}{\pgfqpoint{3.870952in}{2.076667in}}{\pgfqpoint{3.876477in}{2.076667in}}%
\pgfpathclose%
\pgfusepath{stroke,fill}%
\end{pgfscope}%
\begin{pgfscope}%
\pgfpathrectangle{\pgfqpoint{0.562500in}{0.275000in}}{\pgfqpoint{3.487500in}{1.925000in}}%
\pgfusepath{clip}%
\pgfsetbuttcap%
\pgfsetroundjoin%
\definecolor{currentfill}{rgb}{0.000000,0.000000,0.000000}%
\pgfsetfillcolor{currentfill}%
\pgfsetlinewidth{1.003750pt}%
\definecolor{currentstroke}{rgb}{0.000000,0.000000,0.000000}%
\pgfsetstrokecolor{currentstroke}%
\pgfsetdash{}{0pt}%
\pgfpathmoveto{\pgfqpoint{3.876477in}{2.076667in}}%
\pgfpathcurveto{\pgfqpoint{3.882002in}{2.076667in}}{\pgfqpoint{3.887302in}{2.078862in}}{\pgfqpoint{3.891209in}{2.082769in}}%
\pgfpathcurveto{\pgfqpoint{3.895115in}{2.086675in}}{\pgfqpoint{3.897311in}{2.091975in}}{\pgfqpoint{3.897311in}{2.097500in}}%
\pgfpathcurveto{\pgfqpoint{3.897311in}{2.103025in}}{\pgfqpoint{3.895115in}{2.108325in}}{\pgfqpoint{3.891209in}{2.112231in}}%
\pgfpathcurveto{\pgfqpoint{3.887302in}{2.116138in}}{\pgfqpoint{3.882002in}{2.118333in}}{\pgfqpoint{3.876477in}{2.118333in}}%
\pgfpathcurveto{\pgfqpoint{3.870952in}{2.118333in}}{\pgfqpoint{3.865653in}{2.116138in}}{\pgfqpoint{3.861746in}{2.112231in}}%
\pgfpathcurveto{\pgfqpoint{3.857839in}{2.108325in}}{\pgfqpoint{3.855644in}{2.103025in}}{\pgfqpoint{3.855644in}{2.097500in}}%
\pgfpathcurveto{\pgfqpoint{3.855644in}{2.091975in}}{\pgfqpoint{3.857839in}{2.086675in}}{\pgfqpoint{3.861746in}{2.082769in}}%
\pgfpathcurveto{\pgfqpoint{3.865653in}{2.078862in}}{\pgfqpoint{3.870952in}{2.076667in}}{\pgfqpoint{3.876477in}{2.076667in}}%
\pgfpathclose%
\pgfusepath{stroke,fill}%
\end{pgfscope}%
\begin{pgfscope}%
\pgfpathrectangle{\pgfqpoint{0.562500in}{0.275000in}}{\pgfqpoint{3.487500in}{1.925000in}}%
\pgfusepath{clip}%
\pgfsetbuttcap%
\pgfsetroundjoin%
\definecolor{currentfill}{rgb}{0.000000,0.000000,0.000000}%
\pgfsetfillcolor{currentfill}%
\pgfsetlinewidth{1.003750pt}%
\definecolor{currentstroke}{rgb}{0.000000,0.000000,0.000000}%
\pgfsetstrokecolor{currentstroke}%
\pgfsetdash{}{0pt}%
\pgfpathmoveto{\pgfqpoint{3.876477in}{2.076667in}}%
\pgfpathcurveto{\pgfqpoint{3.882002in}{2.076667in}}{\pgfqpoint{3.887302in}{2.078862in}}{\pgfqpoint{3.891209in}{2.082769in}}%
\pgfpathcurveto{\pgfqpoint{3.895115in}{2.086675in}}{\pgfqpoint{3.897311in}{2.091975in}}{\pgfqpoint{3.897311in}{2.097500in}}%
\pgfpathcurveto{\pgfqpoint{3.897311in}{2.103025in}}{\pgfqpoint{3.895115in}{2.108325in}}{\pgfqpoint{3.891209in}{2.112231in}}%
\pgfpathcurveto{\pgfqpoint{3.887302in}{2.116138in}}{\pgfqpoint{3.882002in}{2.118333in}}{\pgfqpoint{3.876477in}{2.118333in}}%
\pgfpathcurveto{\pgfqpoint{3.870952in}{2.118333in}}{\pgfqpoint{3.865653in}{2.116138in}}{\pgfqpoint{3.861746in}{2.112231in}}%
\pgfpathcurveto{\pgfqpoint{3.857839in}{2.108325in}}{\pgfqpoint{3.855644in}{2.103025in}}{\pgfqpoint{3.855644in}{2.097500in}}%
\pgfpathcurveto{\pgfqpoint{3.855644in}{2.091975in}}{\pgfqpoint{3.857839in}{2.086675in}}{\pgfqpoint{3.861746in}{2.082769in}}%
\pgfpathcurveto{\pgfqpoint{3.865653in}{2.078862in}}{\pgfqpoint{3.870952in}{2.076667in}}{\pgfqpoint{3.876477in}{2.076667in}}%
\pgfpathclose%
\pgfusepath{stroke,fill}%
\end{pgfscope}%
\begin{pgfscope}%
\pgfpathrectangle{\pgfqpoint{0.562500in}{0.275000in}}{\pgfqpoint{3.487500in}{1.925000in}}%
\pgfusepath{clip}%
\pgfsetbuttcap%
\pgfsetroundjoin%
\definecolor{currentfill}{rgb}{0.000000,0.000000,0.000000}%
\pgfsetfillcolor{currentfill}%
\pgfsetlinewidth{1.003750pt}%
\definecolor{currentstroke}{rgb}{0.000000,0.000000,0.000000}%
\pgfsetstrokecolor{currentstroke}%
\pgfsetdash{}{0pt}%
\pgfpathmoveto{\pgfqpoint{3.876477in}{2.076667in}}%
\pgfpathcurveto{\pgfqpoint{3.882002in}{2.076667in}}{\pgfqpoint{3.887302in}{2.078862in}}{\pgfqpoint{3.891209in}{2.082769in}}%
\pgfpathcurveto{\pgfqpoint{3.895115in}{2.086675in}}{\pgfqpoint{3.897311in}{2.091975in}}{\pgfqpoint{3.897311in}{2.097500in}}%
\pgfpathcurveto{\pgfqpoint{3.897311in}{2.103025in}}{\pgfqpoint{3.895115in}{2.108325in}}{\pgfqpoint{3.891209in}{2.112231in}}%
\pgfpathcurveto{\pgfqpoint{3.887302in}{2.116138in}}{\pgfqpoint{3.882002in}{2.118333in}}{\pgfqpoint{3.876477in}{2.118333in}}%
\pgfpathcurveto{\pgfqpoint{3.870952in}{2.118333in}}{\pgfqpoint{3.865653in}{2.116138in}}{\pgfqpoint{3.861746in}{2.112231in}}%
\pgfpathcurveto{\pgfqpoint{3.857839in}{2.108325in}}{\pgfqpoint{3.855644in}{2.103025in}}{\pgfqpoint{3.855644in}{2.097500in}}%
\pgfpathcurveto{\pgfqpoint{3.855644in}{2.091975in}}{\pgfqpoint{3.857839in}{2.086675in}}{\pgfqpoint{3.861746in}{2.082769in}}%
\pgfpathcurveto{\pgfqpoint{3.865653in}{2.078862in}}{\pgfqpoint{3.870952in}{2.076667in}}{\pgfqpoint{3.876477in}{2.076667in}}%
\pgfpathclose%
\pgfusepath{stroke,fill}%
\end{pgfscope}%
\begin{pgfscope}%
\pgfpathrectangle{\pgfqpoint{0.562500in}{0.275000in}}{\pgfqpoint{3.487500in}{1.925000in}}%
\pgfusepath{clip}%
\pgfsetbuttcap%
\pgfsetroundjoin%
\definecolor{currentfill}{rgb}{0.000000,0.000000,0.000000}%
\pgfsetfillcolor{currentfill}%
\pgfsetlinewidth{1.003750pt}%
\definecolor{currentstroke}{rgb}{0.000000,0.000000,0.000000}%
\pgfsetstrokecolor{currentstroke}%
\pgfsetdash{}{0pt}%
\pgfpathmoveto{\pgfqpoint{3.876477in}{1.216635in}}%
\pgfpathcurveto{\pgfqpoint{3.882002in}{1.216635in}}{\pgfqpoint{3.887302in}{1.218830in}}{\pgfqpoint{3.891209in}{1.222736in}}%
\pgfpathcurveto{\pgfqpoint{3.895115in}{1.226643in}}{\pgfqpoint{3.897311in}{1.231943in}}{\pgfqpoint{3.897311in}{1.237468in}}%
\pgfpathcurveto{\pgfqpoint{3.897311in}{1.242993in}}{\pgfqpoint{3.895115in}{1.248292in}}{\pgfqpoint{3.891209in}{1.252199in}}%
\pgfpathcurveto{\pgfqpoint{3.887302in}{1.256106in}}{\pgfqpoint{3.882002in}{1.258301in}}{\pgfqpoint{3.876477in}{1.258301in}}%
\pgfpathcurveto{\pgfqpoint{3.870952in}{1.258301in}}{\pgfqpoint{3.865653in}{1.256106in}}{\pgfqpoint{3.861746in}{1.252199in}}%
\pgfpathcurveto{\pgfqpoint{3.857839in}{1.248292in}}{\pgfqpoint{3.855644in}{1.242993in}}{\pgfqpoint{3.855644in}{1.237468in}}%
\pgfpathcurveto{\pgfqpoint{3.855644in}{1.231943in}}{\pgfqpoint{3.857839in}{1.226643in}}{\pgfqpoint{3.861746in}{1.222736in}}%
\pgfpathcurveto{\pgfqpoint{3.865653in}{1.218830in}}{\pgfqpoint{3.870952in}{1.216635in}}{\pgfqpoint{3.876477in}{1.216635in}}%
\pgfpathclose%
\pgfusepath{stroke,fill}%
\end{pgfscope}%
\begin{pgfscope}%
\pgfpathrectangle{\pgfqpoint{0.562500in}{0.275000in}}{\pgfqpoint{3.487500in}{1.925000in}}%
\pgfusepath{clip}%
\pgfsetbuttcap%
\pgfsetroundjoin%
\definecolor{currentfill}{rgb}{0.000000,0.000000,0.000000}%
\pgfsetfillcolor{currentfill}%
\pgfsetlinewidth{1.003750pt}%
\definecolor{currentstroke}{rgb}{0.000000,0.000000,0.000000}%
\pgfsetstrokecolor{currentstroke}%
\pgfsetdash{}{0pt}%
\pgfpathmoveto{\pgfqpoint{3.876477in}{2.076667in}}%
\pgfpathcurveto{\pgfqpoint{3.882002in}{2.076667in}}{\pgfqpoint{3.887302in}{2.078862in}}{\pgfqpoint{3.891209in}{2.082769in}}%
\pgfpathcurveto{\pgfqpoint{3.895115in}{2.086675in}}{\pgfqpoint{3.897311in}{2.091975in}}{\pgfqpoint{3.897311in}{2.097500in}}%
\pgfpathcurveto{\pgfqpoint{3.897311in}{2.103025in}}{\pgfqpoint{3.895115in}{2.108325in}}{\pgfqpoint{3.891209in}{2.112231in}}%
\pgfpathcurveto{\pgfqpoint{3.887302in}{2.116138in}}{\pgfqpoint{3.882002in}{2.118333in}}{\pgfqpoint{3.876477in}{2.118333in}}%
\pgfpathcurveto{\pgfqpoint{3.870952in}{2.118333in}}{\pgfqpoint{3.865653in}{2.116138in}}{\pgfqpoint{3.861746in}{2.112231in}}%
\pgfpathcurveto{\pgfqpoint{3.857839in}{2.108325in}}{\pgfqpoint{3.855644in}{2.103025in}}{\pgfqpoint{3.855644in}{2.097500in}}%
\pgfpathcurveto{\pgfqpoint{3.855644in}{2.091975in}}{\pgfqpoint{3.857839in}{2.086675in}}{\pgfqpoint{3.861746in}{2.082769in}}%
\pgfpathcurveto{\pgfqpoint{3.865653in}{2.078862in}}{\pgfqpoint{3.870952in}{2.076667in}}{\pgfqpoint{3.876477in}{2.076667in}}%
\pgfpathclose%
\pgfusepath{stroke,fill}%
\end{pgfscope}%
\begin{pgfscope}%
\pgfpathrectangle{\pgfqpoint{0.562500in}{0.275000in}}{\pgfqpoint{3.487500in}{1.925000in}}%
\pgfusepath{clip}%
\pgfsetbuttcap%
\pgfsetroundjoin%
\definecolor{currentfill}{rgb}{0.000000,0.000000,0.000000}%
\pgfsetfillcolor{currentfill}%
\pgfsetlinewidth{1.003750pt}%
\definecolor{currentstroke}{rgb}{0.000000,0.000000,0.000000}%
\pgfsetstrokecolor{currentstroke}%
\pgfsetdash{}{0pt}%
\pgfpathmoveto{\pgfqpoint{3.876477in}{1.216635in}}%
\pgfpathcurveto{\pgfqpoint{3.882002in}{1.216635in}}{\pgfqpoint{3.887302in}{1.218830in}}{\pgfqpoint{3.891209in}{1.222736in}}%
\pgfpathcurveto{\pgfqpoint{3.895115in}{1.226643in}}{\pgfqpoint{3.897311in}{1.231943in}}{\pgfqpoint{3.897311in}{1.237468in}}%
\pgfpathcurveto{\pgfqpoint{3.897311in}{1.242993in}}{\pgfqpoint{3.895115in}{1.248292in}}{\pgfqpoint{3.891209in}{1.252199in}}%
\pgfpathcurveto{\pgfqpoint{3.887302in}{1.256106in}}{\pgfqpoint{3.882002in}{1.258301in}}{\pgfqpoint{3.876477in}{1.258301in}}%
\pgfpathcurveto{\pgfqpoint{3.870952in}{1.258301in}}{\pgfqpoint{3.865653in}{1.256106in}}{\pgfqpoint{3.861746in}{1.252199in}}%
\pgfpathcurveto{\pgfqpoint{3.857839in}{1.248292in}}{\pgfqpoint{3.855644in}{1.242993in}}{\pgfqpoint{3.855644in}{1.237468in}}%
\pgfpathcurveto{\pgfqpoint{3.855644in}{1.231943in}}{\pgfqpoint{3.857839in}{1.226643in}}{\pgfqpoint{3.861746in}{1.222736in}}%
\pgfpathcurveto{\pgfqpoint{3.865653in}{1.218830in}}{\pgfqpoint{3.870952in}{1.216635in}}{\pgfqpoint{3.876477in}{1.216635in}}%
\pgfpathclose%
\pgfusepath{stroke,fill}%
\end{pgfscope}%
\begin{pgfscope}%
\pgfpathrectangle{\pgfqpoint{0.562500in}{0.275000in}}{\pgfqpoint{3.487500in}{1.925000in}}%
\pgfusepath{clip}%
\pgfsetbuttcap%
\pgfsetroundjoin%
\definecolor{currentfill}{rgb}{0.000000,0.000000,0.000000}%
\pgfsetfillcolor{currentfill}%
\pgfsetlinewidth{1.003750pt}%
\definecolor{currentstroke}{rgb}{0.000000,0.000000,0.000000}%
\pgfsetstrokecolor{currentstroke}%
\pgfsetdash{}{0pt}%
\pgfpathmoveto{\pgfqpoint{3.876477in}{1.216635in}}%
\pgfpathcurveto{\pgfqpoint{3.882002in}{1.216635in}}{\pgfqpoint{3.887302in}{1.218830in}}{\pgfqpoint{3.891209in}{1.222736in}}%
\pgfpathcurveto{\pgfqpoint{3.895115in}{1.226643in}}{\pgfqpoint{3.897311in}{1.231943in}}{\pgfqpoint{3.897311in}{1.237468in}}%
\pgfpathcurveto{\pgfqpoint{3.897311in}{1.242993in}}{\pgfqpoint{3.895115in}{1.248292in}}{\pgfqpoint{3.891209in}{1.252199in}}%
\pgfpathcurveto{\pgfqpoint{3.887302in}{1.256106in}}{\pgfqpoint{3.882002in}{1.258301in}}{\pgfqpoint{3.876477in}{1.258301in}}%
\pgfpathcurveto{\pgfqpoint{3.870952in}{1.258301in}}{\pgfqpoint{3.865653in}{1.256106in}}{\pgfqpoint{3.861746in}{1.252199in}}%
\pgfpathcurveto{\pgfqpoint{3.857839in}{1.248292in}}{\pgfqpoint{3.855644in}{1.242993in}}{\pgfqpoint{3.855644in}{1.237468in}}%
\pgfpathcurveto{\pgfqpoint{3.855644in}{1.231943in}}{\pgfqpoint{3.857839in}{1.226643in}}{\pgfqpoint{3.861746in}{1.222736in}}%
\pgfpathcurveto{\pgfqpoint{3.865653in}{1.218830in}}{\pgfqpoint{3.870952in}{1.216635in}}{\pgfqpoint{3.876477in}{1.216635in}}%
\pgfpathclose%
\pgfusepath{stroke,fill}%
\end{pgfscope}%
\begin{pgfscope}%
\pgfpathrectangle{\pgfqpoint{0.562500in}{0.275000in}}{\pgfqpoint{3.487500in}{1.925000in}}%
\pgfusepath{clip}%
\pgfsetbuttcap%
\pgfsetroundjoin%
\definecolor{currentfill}{rgb}{0.000000,0.000000,0.000000}%
\pgfsetfillcolor{currentfill}%
\pgfsetlinewidth{1.003750pt}%
\definecolor{currentstroke}{rgb}{0.000000,0.000000,0.000000}%
\pgfsetstrokecolor{currentstroke}%
\pgfsetdash{}{0pt}%
\pgfpathmoveto{\pgfqpoint{3.876477in}{2.076667in}}%
\pgfpathcurveto{\pgfqpoint{3.882002in}{2.076667in}}{\pgfqpoint{3.887302in}{2.078862in}}{\pgfqpoint{3.891209in}{2.082769in}}%
\pgfpathcurveto{\pgfqpoint{3.895115in}{2.086675in}}{\pgfqpoint{3.897311in}{2.091975in}}{\pgfqpoint{3.897311in}{2.097500in}}%
\pgfpathcurveto{\pgfqpoint{3.897311in}{2.103025in}}{\pgfqpoint{3.895115in}{2.108325in}}{\pgfqpoint{3.891209in}{2.112231in}}%
\pgfpathcurveto{\pgfqpoint{3.887302in}{2.116138in}}{\pgfqpoint{3.882002in}{2.118333in}}{\pgfqpoint{3.876477in}{2.118333in}}%
\pgfpathcurveto{\pgfqpoint{3.870952in}{2.118333in}}{\pgfqpoint{3.865653in}{2.116138in}}{\pgfqpoint{3.861746in}{2.112231in}}%
\pgfpathcurveto{\pgfqpoint{3.857839in}{2.108325in}}{\pgfqpoint{3.855644in}{2.103025in}}{\pgfqpoint{3.855644in}{2.097500in}}%
\pgfpathcurveto{\pgfqpoint{3.855644in}{2.091975in}}{\pgfqpoint{3.857839in}{2.086675in}}{\pgfqpoint{3.861746in}{2.082769in}}%
\pgfpathcurveto{\pgfqpoint{3.865653in}{2.078862in}}{\pgfqpoint{3.870952in}{2.076667in}}{\pgfqpoint{3.876477in}{2.076667in}}%
\pgfpathclose%
\pgfusepath{stroke,fill}%
\end{pgfscope}%
\begin{pgfscope}%
\pgfpathrectangle{\pgfqpoint{0.562500in}{0.275000in}}{\pgfqpoint{3.487500in}{1.925000in}}%
\pgfusepath{clip}%
\pgfsetbuttcap%
\pgfsetroundjoin%
\definecolor{currentfill}{rgb}{0.000000,0.000000,0.000000}%
\pgfsetfillcolor{currentfill}%
\pgfsetlinewidth{1.003750pt}%
\definecolor{currentstroke}{rgb}{0.000000,0.000000,0.000000}%
\pgfsetstrokecolor{currentstroke}%
\pgfsetdash{}{0pt}%
\pgfpathmoveto{\pgfqpoint{3.876477in}{2.076667in}}%
\pgfpathcurveto{\pgfqpoint{3.882002in}{2.076667in}}{\pgfqpoint{3.887302in}{2.078862in}}{\pgfqpoint{3.891209in}{2.082769in}}%
\pgfpathcurveto{\pgfqpoint{3.895115in}{2.086675in}}{\pgfqpoint{3.897311in}{2.091975in}}{\pgfqpoint{3.897311in}{2.097500in}}%
\pgfpathcurveto{\pgfqpoint{3.897311in}{2.103025in}}{\pgfqpoint{3.895115in}{2.108325in}}{\pgfqpoint{3.891209in}{2.112231in}}%
\pgfpathcurveto{\pgfqpoint{3.887302in}{2.116138in}}{\pgfqpoint{3.882002in}{2.118333in}}{\pgfqpoint{3.876477in}{2.118333in}}%
\pgfpathcurveto{\pgfqpoint{3.870952in}{2.118333in}}{\pgfqpoint{3.865653in}{2.116138in}}{\pgfqpoint{3.861746in}{2.112231in}}%
\pgfpathcurveto{\pgfqpoint{3.857839in}{2.108325in}}{\pgfqpoint{3.855644in}{2.103025in}}{\pgfqpoint{3.855644in}{2.097500in}}%
\pgfpathcurveto{\pgfqpoint{3.855644in}{2.091975in}}{\pgfqpoint{3.857839in}{2.086675in}}{\pgfqpoint{3.861746in}{2.082769in}}%
\pgfpathcurveto{\pgfqpoint{3.865653in}{2.078862in}}{\pgfqpoint{3.870952in}{2.076667in}}{\pgfqpoint{3.876477in}{2.076667in}}%
\pgfpathclose%
\pgfusepath{stroke,fill}%
\end{pgfscope}%
\begin{pgfscope}%
\pgfpathrectangle{\pgfqpoint{0.562500in}{0.275000in}}{\pgfqpoint{3.487500in}{1.925000in}}%
\pgfusepath{clip}%
\pgfsetbuttcap%
\pgfsetroundjoin%
\definecolor{currentfill}{rgb}{0.000000,0.000000,0.000000}%
\pgfsetfillcolor{currentfill}%
\pgfsetlinewidth{1.003750pt}%
\definecolor{currentstroke}{rgb}{0.000000,0.000000,0.000000}%
\pgfsetstrokecolor{currentstroke}%
\pgfsetdash{}{0pt}%
\pgfpathmoveto{\pgfqpoint{3.876477in}{1.216635in}}%
\pgfpathcurveto{\pgfqpoint{3.882002in}{1.216635in}}{\pgfqpoint{3.887302in}{1.218830in}}{\pgfqpoint{3.891209in}{1.222736in}}%
\pgfpathcurveto{\pgfqpoint{3.895115in}{1.226643in}}{\pgfqpoint{3.897311in}{1.231943in}}{\pgfqpoint{3.897311in}{1.237468in}}%
\pgfpathcurveto{\pgfqpoint{3.897311in}{1.242993in}}{\pgfqpoint{3.895115in}{1.248292in}}{\pgfqpoint{3.891209in}{1.252199in}}%
\pgfpathcurveto{\pgfqpoint{3.887302in}{1.256106in}}{\pgfqpoint{3.882002in}{1.258301in}}{\pgfqpoint{3.876477in}{1.258301in}}%
\pgfpathcurveto{\pgfqpoint{3.870952in}{1.258301in}}{\pgfqpoint{3.865653in}{1.256106in}}{\pgfqpoint{3.861746in}{1.252199in}}%
\pgfpathcurveto{\pgfqpoint{3.857839in}{1.248292in}}{\pgfqpoint{3.855644in}{1.242993in}}{\pgfqpoint{3.855644in}{1.237468in}}%
\pgfpathcurveto{\pgfqpoint{3.855644in}{1.231943in}}{\pgfqpoint{3.857839in}{1.226643in}}{\pgfqpoint{3.861746in}{1.222736in}}%
\pgfpathcurveto{\pgfqpoint{3.865653in}{1.218830in}}{\pgfqpoint{3.870952in}{1.216635in}}{\pgfqpoint{3.876477in}{1.216635in}}%
\pgfpathclose%
\pgfusepath{stroke,fill}%
\end{pgfscope}%
\begin{pgfscope}%
\pgfpathrectangle{\pgfqpoint{0.562500in}{0.275000in}}{\pgfqpoint{3.487500in}{1.925000in}}%
\pgfusepath{clip}%
\pgfsetbuttcap%
\pgfsetroundjoin%
\definecolor{currentfill}{rgb}{0.000000,0.000000,0.000000}%
\pgfsetfillcolor{currentfill}%
\pgfsetlinewidth{1.003750pt}%
\definecolor{currentstroke}{rgb}{0.000000,0.000000,0.000000}%
\pgfsetstrokecolor{currentstroke}%
\pgfsetdash{}{0pt}%
\pgfpathmoveto{\pgfqpoint{3.876477in}{2.076667in}}%
\pgfpathcurveto{\pgfqpoint{3.882002in}{2.076667in}}{\pgfqpoint{3.887302in}{2.078862in}}{\pgfqpoint{3.891209in}{2.082769in}}%
\pgfpathcurveto{\pgfqpoint{3.895115in}{2.086675in}}{\pgfqpoint{3.897311in}{2.091975in}}{\pgfqpoint{3.897311in}{2.097500in}}%
\pgfpathcurveto{\pgfqpoint{3.897311in}{2.103025in}}{\pgfqpoint{3.895115in}{2.108325in}}{\pgfqpoint{3.891209in}{2.112231in}}%
\pgfpathcurveto{\pgfqpoint{3.887302in}{2.116138in}}{\pgfqpoint{3.882002in}{2.118333in}}{\pgfqpoint{3.876477in}{2.118333in}}%
\pgfpathcurveto{\pgfqpoint{3.870952in}{2.118333in}}{\pgfqpoint{3.865653in}{2.116138in}}{\pgfqpoint{3.861746in}{2.112231in}}%
\pgfpathcurveto{\pgfqpoint{3.857839in}{2.108325in}}{\pgfqpoint{3.855644in}{2.103025in}}{\pgfqpoint{3.855644in}{2.097500in}}%
\pgfpathcurveto{\pgfqpoint{3.855644in}{2.091975in}}{\pgfqpoint{3.857839in}{2.086675in}}{\pgfqpoint{3.861746in}{2.082769in}}%
\pgfpathcurveto{\pgfqpoint{3.865653in}{2.078862in}}{\pgfqpoint{3.870952in}{2.076667in}}{\pgfqpoint{3.876477in}{2.076667in}}%
\pgfpathclose%
\pgfusepath{stroke,fill}%
\end{pgfscope}%
\begin{pgfscope}%
\pgfpathrectangle{\pgfqpoint{0.562500in}{0.275000in}}{\pgfqpoint{3.487500in}{1.925000in}}%
\pgfusepath{clip}%
\pgfsetbuttcap%
\pgfsetroundjoin%
\definecolor{currentfill}{rgb}{0.000000,0.000000,0.000000}%
\pgfsetfillcolor{currentfill}%
\pgfsetlinewidth{1.003750pt}%
\definecolor{currentstroke}{rgb}{0.000000,0.000000,0.000000}%
\pgfsetstrokecolor{currentstroke}%
\pgfsetdash{}{0pt}%
\pgfpathmoveto{\pgfqpoint{3.876477in}{2.076667in}}%
\pgfpathcurveto{\pgfqpoint{3.882002in}{2.076667in}}{\pgfqpoint{3.887302in}{2.078862in}}{\pgfqpoint{3.891209in}{2.082769in}}%
\pgfpathcurveto{\pgfqpoint{3.895115in}{2.086675in}}{\pgfqpoint{3.897311in}{2.091975in}}{\pgfqpoint{3.897311in}{2.097500in}}%
\pgfpathcurveto{\pgfqpoint{3.897311in}{2.103025in}}{\pgfqpoint{3.895115in}{2.108325in}}{\pgfqpoint{3.891209in}{2.112231in}}%
\pgfpathcurveto{\pgfqpoint{3.887302in}{2.116138in}}{\pgfqpoint{3.882002in}{2.118333in}}{\pgfqpoint{3.876477in}{2.118333in}}%
\pgfpathcurveto{\pgfqpoint{3.870952in}{2.118333in}}{\pgfqpoint{3.865653in}{2.116138in}}{\pgfqpoint{3.861746in}{2.112231in}}%
\pgfpathcurveto{\pgfqpoint{3.857839in}{2.108325in}}{\pgfqpoint{3.855644in}{2.103025in}}{\pgfqpoint{3.855644in}{2.097500in}}%
\pgfpathcurveto{\pgfqpoint{3.855644in}{2.091975in}}{\pgfqpoint{3.857839in}{2.086675in}}{\pgfqpoint{3.861746in}{2.082769in}}%
\pgfpathcurveto{\pgfqpoint{3.865653in}{2.078862in}}{\pgfqpoint{3.870952in}{2.076667in}}{\pgfqpoint{3.876477in}{2.076667in}}%
\pgfpathclose%
\pgfusepath{stroke,fill}%
\end{pgfscope}%
\begin{pgfscope}%
\pgfpathrectangle{\pgfqpoint{0.562500in}{0.275000in}}{\pgfqpoint{3.487500in}{1.925000in}}%
\pgfusepath{clip}%
\pgfsetbuttcap%
\pgfsetroundjoin%
\definecolor{currentfill}{rgb}{0.000000,0.000000,0.000000}%
\pgfsetfillcolor{currentfill}%
\pgfsetlinewidth{1.003750pt}%
\definecolor{currentstroke}{rgb}{0.000000,0.000000,0.000000}%
\pgfsetstrokecolor{currentstroke}%
\pgfsetdash{}{0pt}%
\pgfpathmoveto{\pgfqpoint{3.876477in}{2.076667in}}%
\pgfpathcurveto{\pgfqpoint{3.882002in}{2.076667in}}{\pgfqpoint{3.887302in}{2.078862in}}{\pgfqpoint{3.891209in}{2.082769in}}%
\pgfpathcurveto{\pgfqpoint{3.895115in}{2.086675in}}{\pgfqpoint{3.897311in}{2.091975in}}{\pgfqpoint{3.897311in}{2.097500in}}%
\pgfpathcurveto{\pgfqpoint{3.897311in}{2.103025in}}{\pgfqpoint{3.895115in}{2.108325in}}{\pgfqpoint{3.891209in}{2.112231in}}%
\pgfpathcurveto{\pgfqpoint{3.887302in}{2.116138in}}{\pgfqpoint{3.882002in}{2.118333in}}{\pgfqpoint{3.876477in}{2.118333in}}%
\pgfpathcurveto{\pgfqpoint{3.870952in}{2.118333in}}{\pgfqpoint{3.865653in}{2.116138in}}{\pgfqpoint{3.861746in}{2.112231in}}%
\pgfpathcurveto{\pgfqpoint{3.857839in}{2.108325in}}{\pgfqpoint{3.855644in}{2.103025in}}{\pgfqpoint{3.855644in}{2.097500in}}%
\pgfpathcurveto{\pgfqpoint{3.855644in}{2.091975in}}{\pgfqpoint{3.857839in}{2.086675in}}{\pgfqpoint{3.861746in}{2.082769in}}%
\pgfpathcurveto{\pgfqpoint{3.865653in}{2.078862in}}{\pgfqpoint{3.870952in}{2.076667in}}{\pgfqpoint{3.876477in}{2.076667in}}%
\pgfpathclose%
\pgfusepath{stroke,fill}%
\end{pgfscope}%
\begin{pgfscope}%
\pgfpathrectangle{\pgfqpoint{0.562500in}{0.275000in}}{\pgfqpoint{3.487500in}{1.925000in}}%
\pgfusepath{clip}%
\pgfsetbuttcap%
\pgfsetroundjoin%
\definecolor{currentfill}{rgb}{0.000000,0.000000,0.000000}%
\pgfsetfillcolor{currentfill}%
\pgfsetlinewidth{1.003750pt}%
\definecolor{currentstroke}{rgb}{0.000000,0.000000,0.000000}%
\pgfsetstrokecolor{currentstroke}%
\pgfsetdash{}{0pt}%
\pgfpathmoveto{\pgfqpoint{3.876477in}{2.076667in}}%
\pgfpathcurveto{\pgfqpoint{3.882002in}{2.076667in}}{\pgfqpoint{3.887302in}{2.078862in}}{\pgfqpoint{3.891209in}{2.082769in}}%
\pgfpathcurveto{\pgfqpoint{3.895115in}{2.086675in}}{\pgfqpoint{3.897311in}{2.091975in}}{\pgfqpoint{3.897311in}{2.097500in}}%
\pgfpathcurveto{\pgfqpoint{3.897311in}{2.103025in}}{\pgfqpoint{3.895115in}{2.108325in}}{\pgfqpoint{3.891209in}{2.112231in}}%
\pgfpathcurveto{\pgfqpoint{3.887302in}{2.116138in}}{\pgfqpoint{3.882002in}{2.118333in}}{\pgfqpoint{3.876477in}{2.118333in}}%
\pgfpathcurveto{\pgfqpoint{3.870952in}{2.118333in}}{\pgfqpoint{3.865653in}{2.116138in}}{\pgfqpoint{3.861746in}{2.112231in}}%
\pgfpathcurveto{\pgfqpoint{3.857839in}{2.108325in}}{\pgfqpoint{3.855644in}{2.103025in}}{\pgfqpoint{3.855644in}{2.097500in}}%
\pgfpathcurveto{\pgfqpoint{3.855644in}{2.091975in}}{\pgfqpoint{3.857839in}{2.086675in}}{\pgfqpoint{3.861746in}{2.082769in}}%
\pgfpathcurveto{\pgfqpoint{3.865653in}{2.078862in}}{\pgfqpoint{3.870952in}{2.076667in}}{\pgfqpoint{3.876477in}{2.076667in}}%
\pgfpathclose%
\pgfusepath{stroke,fill}%
\end{pgfscope}%
\begin{pgfscope}%
\pgfpathrectangle{\pgfqpoint{0.562500in}{0.275000in}}{\pgfqpoint{3.487500in}{1.925000in}}%
\pgfusepath{clip}%
\pgfsetbuttcap%
\pgfsetroundjoin%
\definecolor{currentfill}{rgb}{0.000000,0.000000,0.000000}%
\pgfsetfillcolor{currentfill}%
\pgfsetlinewidth{1.003750pt}%
\definecolor{currentstroke}{rgb}{0.000000,0.000000,0.000000}%
\pgfsetstrokecolor{currentstroke}%
\pgfsetdash{}{0pt}%
\pgfpathmoveto{\pgfqpoint{3.876477in}{2.076667in}}%
\pgfpathcurveto{\pgfqpoint{3.882002in}{2.076667in}}{\pgfqpoint{3.887302in}{2.078862in}}{\pgfqpoint{3.891209in}{2.082769in}}%
\pgfpathcurveto{\pgfqpoint{3.895115in}{2.086675in}}{\pgfqpoint{3.897311in}{2.091975in}}{\pgfqpoint{3.897311in}{2.097500in}}%
\pgfpathcurveto{\pgfqpoint{3.897311in}{2.103025in}}{\pgfqpoint{3.895115in}{2.108325in}}{\pgfqpoint{3.891209in}{2.112231in}}%
\pgfpathcurveto{\pgfqpoint{3.887302in}{2.116138in}}{\pgfqpoint{3.882002in}{2.118333in}}{\pgfqpoint{3.876477in}{2.118333in}}%
\pgfpathcurveto{\pgfqpoint{3.870952in}{2.118333in}}{\pgfqpoint{3.865653in}{2.116138in}}{\pgfqpoint{3.861746in}{2.112231in}}%
\pgfpathcurveto{\pgfqpoint{3.857839in}{2.108325in}}{\pgfqpoint{3.855644in}{2.103025in}}{\pgfqpoint{3.855644in}{2.097500in}}%
\pgfpathcurveto{\pgfqpoint{3.855644in}{2.091975in}}{\pgfqpoint{3.857839in}{2.086675in}}{\pgfqpoint{3.861746in}{2.082769in}}%
\pgfpathcurveto{\pgfqpoint{3.865653in}{2.078862in}}{\pgfqpoint{3.870952in}{2.076667in}}{\pgfqpoint{3.876477in}{2.076667in}}%
\pgfpathclose%
\pgfusepath{stroke,fill}%
\end{pgfscope}%
\begin{pgfscope}%
\pgfpathrectangle{\pgfqpoint{0.562500in}{0.275000in}}{\pgfqpoint{3.487500in}{1.925000in}}%
\pgfusepath{clip}%
\pgfsetbuttcap%
\pgfsetroundjoin%
\definecolor{currentfill}{rgb}{0.000000,0.000000,0.000000}%
\pgfsetfillcolor{currentfill}%
\pgfsetlinewidth{1.003750pt}%
\definecolor{currentstroke}{rgb}{0.000000,0.000000,0.000000}%
\pgfsetstrokecolor{currentstroke}%
\pgfsetdash{}{0pt}%
\pgfpathmoveto{\pgfqpoint{3.876477in}{1.216635in}}%
\pgfpathcurveto{\pgfqpoint{3.882002in}{1.216635in}}{\pgfqpoint{3.887302in}{1.218830in}}{\pgfqpoint{3.891209in}{1.222736in}}%
\pgfpathcurveto{\pgfqpoint{3.895115in}{1.226643in}}{\pgfqpoint{3.897311in}{1.231943in}}{\pgfqpoint{3.897311in}{1.237468in}}%
\pgfpathcurveto{\pgfqpoint{3.897311in}{1.242993in}}{\pgfqpoint{3.895115in}{1.248292in}}{\pgfqpoint{3.891209in}{1.252199in}}%
\pgfpathcurveto{\pgfqpoint{3.887302in}{1.256106in}}{\pgfqpoint{3.882002in}{1.258301in}}{\pgfqpoint{3.876477in}{1.258301in}}%
\pgfpathcurveto{\pgfqpoint{3.870952in}{1.258301in}}{\pgfqpoint{3.865653in}{1.256106in}}{\pgfqpoint{3.861746in}{1.252199in}}%
\pgfpathcurveto{\pgfqpoint{3.857839in}{1.248292in}}{\pgfqpoint{3.855644in}{1.242993in}}{\pgfqpoint{3.855644in}{1.237468in}}%
\pgfpathcurveto{\pgfqpoint{3.855644in}{1.231943in}}{\pgfqpoint{3.857839in}{1.226643in}}{\pgfqpoint{3.861746in}{1.222736in}}%
\pgfpathcurveto{\pgfqpoint{3.865653in}{1.218830in}}{\pgfqpoint{3.870952in}{1.216635in}}{\pgfqpoint{3.876477in}{1.216635in}}%
\pgfpathclose%
\pgfusepath{stroke,fill}%
\end{pgfscope}%
\begin{pgfscope}%
\pgfpathrectangle{\pgfqpoint{0.562500in}{0.275000in}}{\pgfqpoint{3.487500in}{1.925000in}}%
\pgfusepath{clip}%
\pgfsetbuttcap%
\pgfsetroundjoin%
\definecolor{currentfill}{rgb}{0.000000,0.000000,0.000000}%
\pgfsetfillcolor{currentfill}%
\pgfsetlinewidth{1.003750pt}%
\definecolor{currentstroke}{rgb}{0.000000,0.000000,0.000000}%
\pgfsetstrokecolor{currentstroke}%
\pgfsetdash{}{0pt}%
\pgfpathmoveto{\pgfqpoint{3.876477in}{2.076667in}}%
\pgfpathcurveto{\pgfqpoint{3.882002in}{2.076667in}}{\pgfqpoint{3.887302in}{2.078862in}}{\pgfqpoint{3.891209in}{2.082769in}}%
\pgfpathcurveto{\pgfqpoint{3.895115in}{2.086675in}}{\pgfqpoint{3.897311in}{2.091975in}}{\pgfqpoint{3.897311in}{2.097500in}}%
\pgfpathcurveto{\pgfqpoint{3.897311in}{2.103025in}}{\pgfqpoint{3.895115in}{2.108325in}}{\pgfqpoint{3.891209in}{2.112231in}}%
\pgfpathcurveto{\pgfqpoint{3.887302in}{2.116138in}}{\pgfqpoint{3.882002in}{2.118333in}}{\pgfqpoint{3.876477in}{2.118333in}}%
\pgfpathcurveto{\pgfqpoint{3.870952in}{2.118333in}}{\pgfqpoint{3.865653in}{2.116138in}}{\pgfqpoint{3.861746in}{2.112231in}}%
\pgfpathcurveto{\pgfqpoint{3.857839in}{2.108325in}}{\pgfqpoint{3.855644in}{2.103025in}}{\pgfqpoint{3.855644in}{2.097500in}}%
\pgfpathcurveto{\pgfqpoint{3.855644in}{2.091975in}}{\pgfqpoint{3.857839in}{2.086675in}}{\pgfqpoint{3.861746in}{2.082769in}}%
\pgfpathcurveto{\pgfqpoint{3.865653in}{2.078862in}}{\pgfqpoint{3.870952in}{2.076667in}}{\pgfqpoint{3.876477in}{2.076667in}}%
\pgfpathclose%
\pgfusepath{stroke,fill}%
\end{pgfscope}%
\begin{pgfscope}%
\pgfpathrectangle{\pgfqpoint{0.562500in}{0.275000in}}{\pgfqpoint{3.487500in}{1.925000in}}%
\pgfusepath{clip}%
\pgfsetbuttcap%
\pgfsetroundjoin%
\definecolor{currentfill}{rgb}{0.000000,0.000000,0.000000}%
\pgfsetfillcolor{currentfill}%
\pgfsetlinewidth{1.003750pt}%
\definecolor{currentstroke}{rgb}{0.000000,0.000000,0.000000}%
\pgfsetstrokecolor{currentstroke}%
\pgfsetdash{}{0pt}%
\pgfpathmoveto{\pgfqpoint{3.876477in}{2.076667in}}%
\pgfpathcurveto{\pgfqpoint{3.882002in}{2.076667in}}{\pgfqpoint{3.887302in}{2.078862in}}{\pgfqpoint{3.891209in}{2.082769in}}%
\pgfpathcurveto{\pgfqpoint{3.895115in}{2.086675in}}{\pgfqpoint{3.897311in}{2.091975in}}{\pgfqpoint{3.897311in}{2.097500in}}%
\pgfpathcurveto{\pgfqpoint{3.897311in}{2.103025in}}{\pgfqpoint{3.895115in}{2.108325in}}{\pgfqpoint{3.891209in}{2.112231in}}%
\pgfpathcurveto{\pgfqpoint{3.887302in}{2.116138in}}{\pgfqpoint{3.882002in}{2.118333in}}{\pgfqpoint{3.876477in}{2.118333in}}%
\pgfpathcurveto{\pgfqpoint{3.870952in}{2.118333in}}{\pgfqpoint{3.865653in}{2.116138in}}{\pgfqpoint{3.861746in}{2.112231in}}%
\pgfpathcurveto{\pgfqpoint{3.857839in}{2.108325in}}{\pgfqpoint{3.855644in}{2.103025in}}{\pgfqpoint{3.855644in}{2.097500in}}%
\pgfpathcurveto{\pgfqpoint{3.855644in}{2.091975in}}{\pgfqpoint{3.857839in}{2.086675in}}{\pgfqpoint{3.861746in}{2.082769in}}%
\pgfpathcurveto{\pgfqpoint{3.865653in}{2.078862in}}{\pgfqpoint{3.870952in}{2.076667in}}{\pgfqpoint{3.876477in}{2.076667in}}%
\pgfpathclose%
\pgfusepath{stroke,fill}%
\end{pgfscope}%
\begin{pgfscope}%
\pgfpathrectangle{\pgfqpoint{0.562500in}{0.275000in}}{\pgfqpoint{3.487500in}{1.925000in}}%
\pgfusepath{clip}%
\pgfsetbuttcap%
\pgfsetroundjoin%
\definecolor{currentfill}{rgb}{0.000000,0.000000,0.000000}%
\pgfsetfillcolor{currentfill}%
\pgfsetlinewidth{1.003750pt}%
\definecolor{currentstroke}{rgb}{0.000000,0.000000,0.000000}%
\pgfsetstrokecolor{currentstroke}%
\pgfsetdash{}{0pt}%
\pgfpathmoveto{\pgfqpoint{3.876477in}{1.216635in}}%
\pgfpathcurveto{\pgfqpoint{3.882002in}{1.216635in}}{\pgfqpoint{3.887302in}{1.218830in}}{\pgfqpoint{3.891209in}{1.222736in}}%
\pgfpathcurveto{\pgfqpoint{3.895115in}{1.226643in}}{\pgfqpoint{3.897311in}{1.231943in}}{\pgfqpoint{3.897311in}{1.237468in}}%
\pgfpathcurveto{\pgfqpoint{3.897311in}{1.242993in}}{\pgfqpoint{3.895115in}{1.248292in}}{\pgfqpoint{3.891209in}{1.252199in}}%
\pgfpathcurveto{\pgfqpoint{3.887302in}{1.256106in}}{\pgfqpoint{3.882002in}{1.258301in}}{\pgfqpoint{3.876477in}{1.258301in}}%
\pgfpathcurveto{\pgfqpoint{3.870952in}{1.258301in}}{\pgfqpoint{3.865653in}{1.256106in}}{\pgfqpoint{3.861746in}{1.252199in}}%
\pgfpathcurveto{\pgfqpoint{3.857839in}{1.248292in}}{\pgfqpoint{3.855644in}{1.242993in}}{\pgfqpoint{3.855644in}{1.237468in}}%
\pgfpathcurveto{\pgfqpoint{3.855644in}{1.231943in}}{\pgfqpoint{3.857839in}{1.226643in}}{\pgfqpoint{3.861746in}{1.222736in}}%
\pgfpathcurveto{\pgfqpoint{3.865653in}{1.218830in}}{\pgfqpoint{3.870952in}{1.216635in}}{\pgfqpoint{3.876477in}{1.216635in}}%
\pgfpathclose%
\pgfusepath{stroke,fill}%
\end{pgfscope}%
\begin{pgfscope}%
\pgfpathrectangle{\pgfqpoint{0.562500in}{0.275000in}}{\pgfqpoint{3.487500in}{1.925000in}}%
\pgfusepath{clip}%
\pgfsetbuttcap%
\pgfsetroundjoin%
\definecolor{currentfill}{rgb}{0.000000,0.000000,0.000000}%
\pgfsetfillcolor{currentfill}%
\pgfsetlinewidth{1.003750pt}%
\definecolor{currentstroke}{rgb}{0.000000,0.000000,0.000000}%
\pgfsetstrokecolor{currentstroke}%
\pgfsetdash{}{0pt}%
\pgfpathmoveto{\pgfqpoint{3.876477in}{1.216635in}}%
\pgfpathcurveto{\pgfqpoint{3.882002in}{1.216635in}}{\pgfqpoint{3.887302in}{1.218830in}}{\pgfqpoint{3.891209in}{1.222736in}}%
\pgfpathcurveto{\pgfqpoint{3.895115in}{1.226643in}}{\pgfqpoint{3.897311in}{1.231943in}}{\pgfqpoint{3.897311in}{1.237468in}}%
\pgfpathcurveto{\pgfqpoint{3.897311in}{1.242993in}}{\pgfqpoint{3.895115in}{1.248292in}}{\pgfqpoint{3.891209in}{1.252199in}}%
\pgfpathcurveto{\pgfqpoint{3.887302in}{1.256106in}}{\pgfqpoint{3.882002in}{1.258301in}}{\pgfqpoint{3.876477in}{1.258301in}}%
\pgfpathcurveto{\pgfqpoint{3.870952in}{1.258301in}}{\pgfqpoint{3.865653in}{1.256106in}}{\pgfqpoint{3.861746in}{1.252199in}}%
\pgfpathcurveto{\pgfqpoint{3.857839in}{1.248292in}}{\pgfqpoint{3.855644in}{1.242993in}}{\pgfqpoint{3.855644in}{1.237468in}}%
\pgfpathcurveto{\pgfqpoint{3.855644in}{1.231943in}}{\pgfqpoint{3.857839in}{1.226643in}}{\pgfqpoint{3.861746in}{1.222736in}}%
\pgfpathcurveto{\pgfqpoint{3.865653in}{1.218830in}}{\pgfqpoint{3.870952in}{1.216635in}}{\pgfqpoint{3.876477in}{1.216635in}}%
\pgfpathclose%
\pgfusepath{stroke,fill}%
\end{pgfscope}%
\begin{pgfscope}%
\pgfpathrectangle{\pgfqpoint{0.562500in}{0.275000in}}{\pgfqpoint{3.487500in}{1.925000in}}%
\pgfusepath{clip}%
\pgfsetbuttcap%
\pgfsetroundjoin%
\definecolor{currentfill}{rgb}{0.000000,0.000000,0.000000}%
\pgfsetfillcolor{currentfill}%
\pgfsetlinewidth{1.003750pt}%
\definecolor{currentstroke}{rgb}{0.000000,0.000000,0.000000}%
\pgfsetstrokecolor{currentstroke}%
\pgfsetdash{}{0pt}%
\pgfpathmoveto{\pgfqpoint{3.876477in}{1.216635in}}%
\pgfpathcurveto{\pgfqpoint{3.882002in}{1.216635in}}{\pgfqpoint{3.887302in}{1.218830in}}{\pgfqpoint{3.891209in}{1.222736in}}%
\pgfpathcurveto{\pgfqpoint{3.895115in}{1.226643in}}{\pgfqpoint{3.897311in}{1.231943in}}{\pgfqpoint{3.897311in}{1.237468in}}%
\pgfpathcurveto{\pgfqpoint{3.897311in}{1.242993in}}{\pgfqpoint{3.895115in}{1.248292in}}{\pgfqpoint{3.891209in}{1.252199in}}%
\pgfpathcurveto{\pgfqpoint{3.887302in}{1.256106in}}{\pgfqpoint{3.882002in}{1.258301in}}{\pgfqpoint{3.876477in}{1.258301in}}%
\pgfpathcurveto{\pgfqpoint{3.870952in}{1.258301in}}{\pgfqpoint{3.865653in}{1.256106in}}{\pgfqpoint{3.861746in}{1.252199in}}%
\pgfpathcurveto{\pgfqpoint{3.857839in}{1.248292in}}{\pgfqpoint{3.855644in}{1.242993in}}{\pgfqpoint{3.855644in}{1.237468in}}%
\pgfpathcurveto{\pgfqpoint{3.855644in}{1.231943in}}{\pgfqpoint{3.857839in}{1.226643in}}{\pgfqpoint{3.861746in}{1.222736in}}%
\pgfpathcurveto{\pgfqpoint{3.865653in}{1.218830in}}{\pgfqpoint{3.870952in}{1.216635in}}{\pgfqpoint{3.876477in}{1.216635in}}%
\pgfpathclose%
\pgfusepath{stroke,fill}%
\end{pgfscope}%
\begin{pgfscope}%
\pgfpathrectangle{\pgfqpoint{0.562500in}{0.275000in}}{\pgfqpoint{3.487500in}{1.925000in}}%
\pgfusepath{clip}%
\pgfsetbuttcap%
\pgfsetroundjoin%
\definecolor{currentfill}{rgb}{0.000000,0.000000,0.000000}%
\pgfsetfillcolor{currentfill}%
\pgfsetlinewidth{1.003750pt}%
\definecolor{currentstroke}{rgb}{0.000000,0.000000,0.000000}%
\pgfsetstrokecolor{currentstroke}%
\pgfsetdash{}{0pt}%
\pgfpathmoveto{\pgfqpoint{3.876477in}{1.216635in}}%
\pgfpathcurveto{\pgfqpoint{3.882002in}{1.216635in}}{\pgfqpoint{3.887302in}{1.218830in}}{\pgfqpoint{3.891209in}{1.222736in}}%
\pgfpathcurveto{\pgfqpoint{3.895115in}{1.226643in}}{\pgfqpoint{3.897311in}{1.231943in}}{\pgfqpoint{3.897311in}{1.237468in}}%
\pgfpathcurveto{\pgfqpoint{3.897311in}{1.242993in}}{\pgfqpoint{3.895115in}{1.248292in}}{\pgfqpoint{3.891209in}{1.252199in}}%
\pgfpathcurveto{\pgfqpoint{3.887302in}{1.256106in}}{\pgfqpoint{3.882002in}{1.258301in}}{\pgfqpoint{3.876477in}{1.258301in}}%
\pgfpathcurveto{\pgfqpoint{3.870952in}{1.258301in}}{\pgfqpoint{3.865653in}{1.256106in}}{\pgfqpoint{3.861746in}{1.252199in}}%
\pgfpathcurveto{\pgfqpoint{3.857839in}{1.248292in}}{\pgfqpoint{3.855644in}{1.242993in}}{\pgfqpoint{3.855644in}{1.237468in}}%
\pgfpathcurveto{\pgfqpoint{3.855644in}{1.231943in}}{\pgfqpoint{3.857839in}{1.226643in}}{\pgfqpoint{3.861746in}{1.222736in}}%
\pgfpathcurveto{\pgfqpoint{3.865653in}{1.218830in}}{\pgfqpoint{3.870952in}{1.216635in}}{\pgfqpoint{3.876477in}{1.216635in}}%
\pgfpathclose%
\pgfusepath{stroke,fill}%
\end{pgfscope}%
\begin{pgfscope}%
\pgfpathrectangle{\pgfqpoint{0.562500in}{0.275000in}}{\pgfqpoint{3.487500in}{1.925000in}}%
\pgfusepath{clip}%
\pgfsetbuttcap%
\pgfsetroundjoin%
\definecolor{currentfill}{rgb}{0.000000,0.000000,0.000000}%
\pgfsetfillcolor{currentfill}%
\pgfsetlinewidth{1.003750pt}%
\definecolor{currentstroke}{rgb}{0.000000,0.000000,0.000000}%
\pgfsetstrokecolor{currentstroke}%
\pgfsetdash{}{0pt}%
\pgfpathmoveto{\pgfqpoint{3.876477in}{2.076667in}}%
\pgfpathcurveto{\pgfqpoint{3.882002in}{2.076667in}}{\pgfqpoint{3.887302in}{2.078862in}}{\pgfqpoint{3.891209in}{2.082769in}}%
\pgfpathcurveto{\pgfqpoint{3.895115in}{2.086675in}}{\pgfqpoint{3.897311in}{2.091975in}}{\pgfqpoint{3.897311in}{2.097500in}}%
\pgfpathcurveto{\pgfqpoint{3.897311in}{2.103025in}}{\pgfqpoint{3.895115in}{2.108325in}}{\pgfqpoint{3.891209in}{2.112231in}}%
\pgfpathcurveto{\pgfqpoint{3.887302in}{2.116138in}}{\pgfqpoint{3.882002in}{2.118333in}}{\pgfqpoint{3.876477in}{2.118333in}}%
\pgfpathcurveto{\pgfqpoint{3.870952in}{2.118333in}}{\pgfqpoint{3.865653in}{2.116138in}}{\pgfqpoint{3.861746in}{2.112231in}}%
\pgfpathcurveto{\pgfqpoint{3.857839in}{2.108325in}}{\pgfqpoint{3.855644in}{2.103025in}}{\pgfqpoint{3.855644in}{2.097500in}}%
\pgfpathcurveto{\pgfqpoint{3.855644in}{2.091975in}}{\pgfqpoint{3.857839in}{2.086675in}}{\pgfqpoint{3.861746in}{2.082769in}}%
\pgfpathcurveto{\pgfqpoint{3.865653in}{2.078862in}}{\pgfqpoint{3.870952in}{2.076667in}}{\pgfqpoint{3.876477in}{2.076667in}}%
\pgfpathclose%
\pgfusepath{stroke,fill}%
\end{pgfscope}%
\begin{pgfscope}%
\pgfpathrectangle{\pgfqpoint{0.562500in}{0.275000in}}{\pgfqpoint{3.487500in}{1.925000in}}%
\pgfusepath{clip}%
\pgfsetbuttcap%
\pgfsetroundjoin%
\definecolor{currentfill}{rgb}{0.000000,0.000000,0.000000}%
\pgfsetfillcolor{currentfill}%
\pgfsetlinewidth{1.003750pt}%
\definecolor{currentstroke}{rgb}{0.000000,0.000000,0.000000}%
\pgfsetstrokecolor{currentstroke}%
\pgfsetdash{}{0pt}%
\pgfpathmoveto{\pgfqpoint{3.876477in}{2.076667in}}%
\pgfpathcurveto{\pgfqpoint{3.882002in}{2.076667in}}{\pgfqpoint{3.887302in}{2.078862in}}{\pgfqpoint{3.891209in}{2.082769in}}%
\pgfpathcurveto{\pgfqpoint{3.895115in}{2.086675in}}{\pgfqpoint{3.897311in}{2.091975in}}{\pgfqpoint{3.897311in}{2.097500in}}%
\pgfpathcurveto{\pgfqpoint{3.897311in}{2.103025in}}{\pgfqpoint{3.895115in}{2.108325in}}{\pgfqpoint{3.891209in}{2.112231in}}%
\pgfpathcurveto{\pgfqpoint{3.887302in}{2.116138in}}{\pgfqpoint{3.882002in}{2.118333in}}{\pgfqpoint{3.876477in}{2.118333in}}%
\pgfpathcurveto{\pgfqpoint{3.870952in}{2.118333in}}{\pgfqpoint{3.865653in}{2.116138in}}{\pgfqpoint{3.861746in}{2.112231in}}%
\pgfpathcurveto{\pgfqpoint{3.857839in}{2.108325in}}{\pgfqpoint{3.855644in}{2.103025in}}{\pgfqpoint{3.855644in}{2.097500in}}%
\pgfpathcurveto{\pgfqpoint{3.855644in}{2.091975in}}{\pgfqpoint{3.857839in}{2.086675in}}{\pgfqpoint{3.861746in}{2.082769in}}%
\pgfpathcurveto{\pgfqpoint{3.865653in}{2.078862in}}{\pgfqpoint{3.870952in}{2.076667in}}{\pgfqpoint{3.876477in}{2.076667in}}%
\pgfpathclose%
\pgfusepath{stroke,fill}%
\end{pgfscope}%
\begin{pgfscope}%
\pgfpathrectangle{\pgfqpoint{0.562500in}{0.275000in}}{\pgfqpoint{3.487500in}{1.925000in}}%
\pgfusepath{clip}%
\pgfsetbuttcap%
\pgfsetroundjoin%
\definecolor{currentfill}{rgb}{0.000000,0.000000,0.000000}%
\pgfsetfillcolor{currentfill}%
\pgfsetlinewidth{1.003750pt}%
\definecolor{currentstroke}{rgb}{0.000000,0.000000,0.000000}%
\pgfsetstrokecolor{currentstroke}%
\pgfsetdash{}{0pt}%
\pgfpathmoveto{\pgfqpoint{3.876477in}{1.216635in}}%
\pgfpathcurveto{\pgfqpoint{3.882002in}{1.216635in}}{\pgfqpoint{3.887302in}{1.218830in}}{\pgfqpoint{3.891209in}{1.222736in}}%
\pgfpathcurveto{\pgfqpoint{3.895115in}{1.226643in}}{\pgfqpoint{3.897311in}{1.231943in}}{\pgfqpoint{3.897311in}{1.237468in}}%
\pgfpathcurveto{\pgfqpoint{3.897311in}{1.242993in}}{\pgfqpoint{3.895115in}{1.248292in}}{\pgfqpoint{3.891209in}{1.252199in}}%
\pgfpathcurveto{\pgfqpoint{3.887302in}{1.256106in}}{\pgfqpoint{3.882002in}{1.258301in}}{\pgfqpoint{3.876477in}{1.258301in}}%
\pgfpathcurveto{\pgfqpoint{3.870952in}{1.258301in}}{\pgfqpoint{3.865653in}{1.256106in}}{\pgfqpoint{3.861746in}{1.252199in}}%
\pgfpathcurveto{\pgfqpoint{3.857839in}{1.248292in}}{\pgfqpoint{3.855644in}{1.242993in}}{\pgfqpoint{3.855644in}{1.237468in}}%
\pgfpathcurveto{\pgfqpoint{3.855644in}{1.231943in}}{\pgfqpoint{3.857839in}{1.226643in}}{\pgfqpoint{3.861746in}{1.222736in}}%
\pgfpathcurveto{\pgfqpoint{3.865653in}{1.218830in}}{\pgfqpoint{3.870952in}{1.216635in}}{\pgfqpoint{3.876477in}{1.216635in}}%
\pgfpathclose%
\pgfusepath{stroke,fill}%
\end{pgfscope}%
\begin{pgfscope}%
\pgfpathrectangle{\pgfqpoint{0.562500in}{0.275000in}}{\pgfqpoint{3.487500in}{1.925000in}}%
\pgfusepath{clip}%
\pgfsetbuttcap%
\pgfsetroundjoin%
\definecolor{currentfill}{rgb}{0.000000,0.000000,0.000000}%
\pgfsetfillcolor{currentfill}%
\pgfsetlinewidth{1.003750pt}%
\definecolor{currentstroke}{rgb}{0.000000,0.000000,0.000000}%
\pgfsetstrokecolor{currentstroke}%
\pgfsetdash{}{0pt}%
\pgfpathmoveto{\pgfqpoint{3.876477in}{2.076667in}}%
\pgfpathcurveto{\pgfqpoint{3.882002in}{2.076667in}}{\pgfqpoint{3.887302in}{2.078862in}}{\pgfqpoint{3.891209in}{2.082769in}}%
\pgfpathcurveto{\pgfqpoint{3.895115in}{2.086675in}}{\pgfqpoint{3.897311in}{2.091975in}}{\pgfqpoint{3.897311in}{2.097500in}}%
\pgfpathcurveto{\pgfqpoint{3.897311in}{2.103025in}}{\pgfqpoint{3.895115in}{2.108325in}}{\pgfqpoint{3.891209in}{2.112231in}}%
\pgfpathcurveto{\pgfqpoint{3.887302in}{2.116138in}}{\pgfqpoint{3.882002in}{2.118333in}}{\pgfqpoint{3.876477in}{2.118333in}}%
\pgfpathcurveto{\pgfqpoint{3.870952in}{2.118333in}}{\pgfqpoint{3.865653in}{2.116138in}}{\pgfqpoint{3.861746in}{2.112231in}}%
\pgfpathcurveto{\pgfqpoint{3.857839in}{2.108325in}}{\pgfqpoint{3.855644in}{2.103025in}}{\pgfqpoint{3.855644in}{2.097500in}}%
\pgfpathcurveto{\pgfqpoint{3.855644in}{2.091975in}}{\pgfqpoint{3.857839in}{2.086675in}}{\pgfqpoint{3.861746in}{2.082769in}}%
\pgfpathcurveto{\pgfqpoint{3.865653in}{2.078862in}}{\pgfqpoint{3.870952in}{2.076667in}}{\pgfqpoint{3.876477in}{2.076667in}}%
\pgfpathclose%
\pgfusepath{stroke,fill}%
\end{pgfscope}%
\begin{pgfscope}%
\pgfpathrectangle{\pgfqpoint{0.562500in}{0.275000in}}{\pgfqpoint{3.487500in}{1.925000in}}%
\pgfusepath{clip}%
\pgfsetbuttcap%
\pgfsetroundjoin%
\definecolor{currentfill}{rgb}{0.000000,0.000000,0.000000}%
\pgfsetfillcolor{currentfill}%
\pgfsetlinewidth{1.003750pt}%
\definecolor{currentstroke}{rgb}{0.000000,0.000000,0.000000}%
\pgfsetstrokecolor{currentstroke}%
\pgfsetdash{}{0pt}%
\pgfpathmoveto{\pgfqpoint{3.876477in}{2.076667in}}%
\pgfpathcurveto{\pgfqpoint{3.882002in}{2.076667in}}{\pgfqpoint{3.887302in}{2.078862in}}{\pgfqpoint{3.891209in}{2.082769in}}%
\pgfpathcurveto{\pgfqpoint{3.895115in}{2.086675in}}{\pgfqpoint{3.897311in}{2.091975in}}{\pgfqpoint{3.897311in}{2.097500in}}%
\pgfpathcurveto{\pgfqpoint{3.897311in}{2.103025in}}{\pgfqpoint{3.895115in}{2.108325in}}{\pgfqpoint{3.891209in}{2.112231in}}%
\pgfpathcurveto{\pgfqpoint{3.887302in}{2.116138in}}{\pgfqpoint{3.882002in}{2.118333in}}{\pgfqpoint{3.876477in}{2.118333in}}%
\pgfpathcurveto{\pgfqpoint{3.870952in}{2.118333in}}{\pgfqpoint{3.865653in}{2.116138in}}{\pgfqpoint{3.861746in}{2.112231in}}%
\pgfpathcurveto{\pgfqpoint{3.857839in}{2.108325in}}{\pgfqpoint{3.855644in}{2.103025in}}{\pgfqpoint{3.855644in}{2.097500in}}%
\pgfpathcurveto{\pgfqpoint{3.855644in}{2.091975in}}{\pgfqpoint{3.857839in}{2.086675in}}{\pgfqpoint{3.861746in}{2.082769in}}%
\pgfpathcurveto{\pgfqpoint{3.865653in}{2.078862in}}{\pgfqpoint{3.870952in}{2.076667in}}{\pgfqpoint{3.876477in}{2.076667in}}%
\pgfpathclose%
\pgfusepath{stroke,fill}%
\end{pgfscope}%
\begin{pgfscope}%
\pgfpathrectangle{\pgfqpoint{0.562500in}{0.275000in}}{\pgfqpoint{3.487500in}{1.925000in}}%
\pgfusepath{clip}%
\pgfsetbuttcap%
\pgfsetroundjoin%
\definecolor{currentfill}{rgb}{0.000000,0.000000,0.000000}%
\pgfsetfillcolor{currentfill}%
\pgfsetlinewidth{1.003750pt}%
\definecolor{currentstroke}{rgb}{0.000000,0.000000,0.000000}%
\pgfsetstrokecolor{currentstroke}%
\pgfsetdash{}{0pt}%
\pgfpathmoveto{\pgfqpoint{3.876477in}{2.076667in}}%
\pgfpathcurveto{\pgfqpoint{3.882002in}{2.076667in}}{\pgfqpoint{3.887302in}{2.078862in}}{\pgfqpoint{3.891209in}{2.082769in}}%
\pgfpathcurveto{\pgfqpoint{3.895115in}{2.086675in}}{\pgfqpoint{3.897311in}{2.091975in}}{\pgfqpoint{3.897311in}{2.097500in}}%
\pgfpathcurveto{\pgfqpoint{3.897311in}{2.103025in}}{\pgfqpoint{3.895115in}{2.108325in}}{\pgfqpoint{3.891209in}{2.112231in}}%
\pgfpathcurveto{\pgfqpoint{3.887302in}{2.116138in}}{\pgfqpoint{3.882002in}{2.118333in}}{\pgfqpoint{3.876477in}{2.118333in}}%
\pgfpathcurveto{\pgfqpoint{3.870952in}{2.118333in}}{\pgfqpoint{3.865653in}{2.116138in}}{\pgfqpoint{3.861746in}{2.112231in}}%
\pgfpathcurveto{\pgfqpoint{3.857839in}{2.108325in}}{\pgfqpoint{3.855644in}{2.103025in}}{\pgfqpoint{3.855644in}{2.097500in}}%
\pgfpathcurveto{\pgfqpoint{3.855644in}{2.091975in}}{\pgfqpoint{3.857839in}{2.086675in}}{\pgfqpoint{3.861746in}{2.082769in}}%
\pgfpathcurveto{\pgfqpoint{3.865653in}{2.078862in}}{\pgfqpoint{3.870952in}{2.076667in}}{\pgfqpoint{3.876477in}{2.076667in}}%
\pgfpathclose%
\pgfusepath{stroke,fill}%
\end{pgfscope}%
\begin{pgfscope}%
\pgfpathrectangle{\pgfqpoint{0.562500in}{0.275000in}}{\pgfqpoint{3.487500in}{1.925000in}}%
\pgfusepath{clip}%
\pgfsetbuttcap%
\pgfsetroundjoin%
\definecolor{currentfill}{rgb}{0.000000,0.000000,0.000000}%
\pgfsetfillcolor{currentfill}%
\pgfsetlinewidth{1.003750pt}%
\definecolor{currentstroke}{rgb}{0.000000,0.000000,0.000000}%
\pgfsetstrokecolor{currentstroke}%
\pgfsetdash{}{0pt}%
\pgfpathmoveto{\pgfqpoint{3.876477in}{1.216635in}}%
\pgfpathcurveto{\pgfqpoint{3.882002in}{1.216635in}}{\pgfqpoint{3.887302in}{1.218830in}}{\pgfqpoint{3.891209in}{1.222736in}}%
\pgfpathcurveto{\pgfqpoint{3.895115in}{1.226643in}}{\pgfqpoint{3.897311in}{1.231943in}}{\pgfqpoint{3.897311in}{1.237468in}}%
\pgfpathcurveto{\pgfqpoint{3.897311in}{1.242993in}}{\pgfqpoint{3.895115in}{1.248292in}}{\pgfqpoint{3.891209in}{1.252199in}}%
\pgfpathcurveto{\pgfqpoint{3.887302in}{1.256106in}}{\pgfqpoint{3.882002in}{1.258301in}}{\pgfqpoint{3.876477in}{1.258301in}}%
\pgfpathcurveto{\pgfqpoint{3.870952in}{1.258301in}}{\pgfqpoint{3.865653in}{1.256106in}}{\pgfqpoint{3.861746in}{1.252199in}}%
\pgfpathcurveto{\pgfqpoint{3.857839in}{1.248292in}}{\pgfqpoint{3.855644in}{1.242993in}}{\pgfqpoint{3.855644in}{1.237468in}}%
\pgfpathcurveto{\pgfqpoint{3.855644in}{1.231943in}}{\pgfqpoint{3.857839in}{1.226643in}}{\pgfqpoint{3.861746in}{1.222736in}}%
\pgfpathcurveto{\pgfqpoint{3.865653in}{1.218830in}}{\pgfqpoint{3.870952in}{1.216635in}}{\pgfqpoint{3.876477in}{1.216635in}}%
\pgfpathclose%
\pgfusepath{stroke,fill}%
\end{pgfscope}%
\begin{pgfscope}%
\pgfpathrectangle{\pgfqpoint{0.562500in}{0.275000in}}{\pgfqpoint{3.487500in}{1.925000in}}%
\pgfusepath{clip}%
\pgfsetbuttcap%
\pgfsetroundjoin%
\definecolor{currentfill}{rgb}{0.000000,0.000000,0.000000}%
\pgfsetfillcolor{currentfill}%
\pgfsetlinewidth{1.003750pt}%
\definecolor{currentstroke}{rgb}{0.000000,0.000000,0.000000}%
\pgfsetstrokecolor{currentstroke}%
\pgfsetdash{}{0pt}%
\pgfpathmoveto{\pgfqpoint{3.876477in}{1.216635in}}%
\pgfpathcurveto{\pgfqpoint{3.882002in}{1.216635in}}{\pgfqpoint{3.887302in}{1.218830in}}{\pgfqpoint{3.891209in}{1.222736in}}%
\pgfpathcurveto{\pgfqpoint{3.895115in}{1.226643in}}{\pgfqpoint{3.897311in}{1.231943in}}{\pgfqpoint{3.897311in}{1.237468in}}%
\pgfpathcurveto{\pgfqpoint{3.897311in}{1.242993in}}{\pgfqpoint{3.895115in}{1.248292in}}{\pgfqpoint{3.891209in}{1.252199in}}%
\pgfpathcurveto{\pgfqpoint{3.887302in}{1.256106in}}{\pgfqpoint{3.882002in}{1.258301in}}{\pgfqpoint{3.876477in}{1.258301in}}%
\pgfpathcurveto{\pgfqpoint{3.870952in}{1.258301in}}{\pgfqpoint{3.865653in}{1.256106in}}{\pgfqpoint{3.861746in}{1.252199in}}%
\pgfpathcurveto{\pgfqpoint{3.857839in}{1.248292in}}{\pgfqpoint{3.855644in}{1.242993in}}{\pgfqpoint{3.855644in}{1.237468in}}%
\pgfpathcurveto{\pgfqpoint{3.855644in}{1.231943in}}{\pgfqpoint{3.857839in}{1.226643in}}{\pgfqpoint{3.861746in}{1.222736in}}%
\pgfpathcurveto{\pgfqpoint{3.865653in}{1.218830in}}{\pgfqpoint{3.870952in}{1.216635in}}{\pgfqpoint{3.876477in}{1.216635in}}%
\pgfpathclose%
\pgfusepath{stroke,fill}%
\end{pgfscope}%
\begin{pgfscope}%
\pgfpathrectangle{\pgfqpoint{0.562500in}{0.275000in}}{\pgfqpoint{3.487500in}{1.925000in}}%
\pgfusepath{clip}%
\pgfsetbuttcap%
\pgfsetroundjoin%
\definecolor{currentfill}{rgb}{0.000000,0.000000,0.000000}%
\pgfsetfillcolor{currentfill}%
\pgfsetlinewidth{1.003750pt}%
\definecolor{currentstroke}{rgb}{0.000000,0.000000,0.000000}%
\pgfsetstrokecolor{currentstroke}%
\pgfsetdash{}{0pt}%
\pgfpathmoveto{\pgfqpoint{3.876477in}{2.076667in}}%
\pgfpathcurveto{\pgfqpoint{3.882002in}{2.076667in}}{\pgfqpoint{3.887302in}{2.078862in}}{\pgfqpoint{3.891209in}{2.082769in}}%
\pgfpathcurveto{\pgfqpoint{3.895115in}{2.086675in}}{\pgfqpoint{3.897311in}{2.091975in}}{\pgfqpoint{3.897311in}{2.097500in}}%
\pgfpathcurveto{\pgfqpoint{3.897311in}{2.103025in}}{\pgfqpoint{3.895115in}{2.108325in}}{\pgfqpoint{3.891209in}{2.112231in}}%
\pgfpathcurveto{\pgfqpoint{3.887302in}{2.116138in}}{\pgfqpoint{3.882002in}{2.118333in}}{\pgfqpoint{3.876477in}{2.118333in}}%
\pgfpathcurveto{\pgfqpoint{3.870952in}{2.118333in}}{\pgfqpoint{3.865653in}{2.116138in}}{\pgfqpoint{3.861746in}{2.112231in}}%
\pgfpathcurveto{\pgfqpoint{3.857839in}{2.108325in}}{\pgfqpoint{3.855644in}{2.103025in}}{\pgfqpoint{3.855644in}{2.097500in}}%
\pgfpathcurveto{\pgfqpoint{3.855644in}{2.091975in}}{\pgfqpoint{3.857839in}{2.086675in}}{\pgfqpoint{3.861746in}{2.082769in}}%
\pgfpathcurveto{\pgfqpoint{3.865653in}{2.078862in}}{\pgfqpoint{3.870952in}{2.076667in}}{\pgfqpoint{3.876477in}{2.076667in}}%
\pgfpathclose%
\pgfusepath{stroke,fill}%
\end{pgfscope}%
\begin{pgfscope}%
\pgfpathrectangle{\pgfqpoint{0.562500in}{0.275000in}}{\pgfqpoint{3.487500in}{1.925000in}}%
\pgfusepath{clip}%
\pgfsetbuttcap%
\pgfsetroundjoin%
\definecolor{currentfill}{rgb}{0.000000,0.000000,0.000000}%
\pgfsetfillcolor{currentfill}%
\pgfsetlinewidth{1.003750pt}%
\definecolor{currentstroke}{rgb}{0.000000,0.000000,0.000000}%
\pgfsetstrokecolor{currentstroke}%
\pgfsetdash{}{0pt}%
\pgfpathmoveto{\pgfqpoint{3.876477in}{1.216635in}}%
\pgfpathcurveto{\pgfqpoint{3.882002in}{1.216635in}}{\pgfqpoint{3.887302in}{1.218830in}}{\pgfqpoint{3.891209in}{1.222736in}}%
\pgfpathcurveto{\pgfqpoint{3.895115in}{1.226643in}}{\pgfqpoint{3.897311in}{1.231943in}}{\pgfqpoint{3.897311in}{1.237468in}}%
\pgfpathcurveto{\pgfqpoint{3.897311in}{1.242993in}}{\pgfqpoint{3.895115in}{1.248292in}}{\pgfqpoint{3.891209in}{1.252199in}}%
\pgfpathcurveto{\pgfqpoint{3.887302in}{1.256106in}}{\pgfqpoint{3.882002in}{1.258301in}}{\pgfqpoint{3.876477in}{1.258301in}}%
\pgfpathcurveto{\pgfqpoint{3.870952in}{1.258301in}}{\pgfqpoint{3.865653in}{1.256106in}}{\pgfqpoint{3.861746in}{1.252199in}}%
\pgfpathcurveto{\pgfqpoint{3.857839in}{1.248292in}}{\pgfqpoint{3.855644in}{1.242993in}}{\pgfqpoint{3.855644in}{1.237468in}}%
\pgfpathcurveto{\pgfqpoint{3.855644in}{1.231943in}}{\pgfqpoint{3.857839in}{1.226643in}}{\pgfqpoint{3.861746in}{1.222736in}}%
\pgfpathcurveto{\pgfqpoint{3.865653in}{1.218830in}}{\pgfqpoint{3.870952in}{1.216635in}}{\pgfqpoint{3.876477in}{1.216635in}}%
\pgfpathclose%
\pgfusepath{stroke,fill}%
\end{pgfscope}%
\begin{pgfscope}%
\pgfpathrectangle{\pgfqpoint{0.562500in}{0.275000in}}{\pgfqpoint{3.487500in}{1.925000in}}%
\pgfusepath{clip}%
\pgfsetbuttcap%
\pgfsetroundjoin%
\definecolor{currentfill}{rgb}{0.000000,0.000000,0.000000}%
\pgfsetfillcolor{currentfill}%
\pgfsetlinewidth{1.003750pt}%
\definecolor{currentstroke}{rgb}{0.000000,0.000000,0.000000}%
\pgfsetstrokecolor{currentstroke}%
\pgfsetdash{}{0pt}%
\pgfpathmoveto{\pgfqpoint{3.876477in}{2.076667in}}%
\pgfpathcurveto{\pgfqpoint{3.882002in}{2.076667in}}{\pgfqpoint{3.887302in}{2.078862in}}{\pgfqpoint{3.891209in}{2.082769in}}%
\pgfpathcurveto{\pgfqpoint{3.895115in}{2.086675in}}{\pgfqpoint{3.897311in}{2.091975in}}{\pgfqpoint{3.897311in}{2.097500in}}%
\pgfpathcurveto{\pgfqpoint{3.897311in}{2.103025in}}{\pgfqpoint{3.895115in}{2.108325in}}{\pgfqpoint{3.891209in}{2.112231in}}%
\pgfpathcurveto{\pgfqpoint{3.887302in}{2.116138in}}{\pgfqpoint{3.882002in}{2.118333in}}{\pgfqpoint{3.876477in}{2.118333in}}%
\pgfpathcurveto{\pgfqpoint{3.870952in}{2.118333in}}{\pgfqpoint{3.865653in}{2.116138in}}{\pgfqpoint{3.861746in}{2.112231in}}%
\pgfpathcurveto{\pgfqpoint{3.857839in}{2.108325in}}{\pgfqpoint{3.855644in}{2.103025in}}{\pgfqpoint{3.855644in}{2.097500in}}%
\pgfpathcurveto{\pgfqpoint{3.855644in}{2.091975in}}{\pgfqpoint{3.857839in}{2.086675in}}{\pgfqpoint{3.861746in}{2.082769in}}%
\pgfpathcurveto{\pgfqpoint{3.865653in}{2.078862in}}{\pgfqpoint{3.870952in}{2.076667in}}{\pgfqpoint{3.876477in}{2.076667in}}%
\pgfpathclose%
\pgfusepath{stroke,fill}%
\end{pgfscope}%
\begin{pgfscope}%
\pgfpathrectangle{\pgfqpoint{0.562500in}{0.275000in}}{\pgfqpoint{3.487500in}{1.925000in}}%
\pgfusepath{clip}%
\pgfsetbuttcap%
\pgfsetroundjoin%
\definecolor{currentfill}{rgb}{0.000000,0.000000,0.000000}%
\pgfsetfillcolor{currentfill}%
\pgfsetlinewidth{1.003750pt}%
\definecolor{currentstroke}{rgb}{0.000000,0.000000,0.000000}%
\pgfsetstrokecolor{currentstroke}%
\pgfsetdash{}{0pt}%
\pgfpathmoveto{\pgfqpoint{3.876477in}{1.216635in}}%
\pgfpathcurveto{\pgfqpoint{3.882002in}{1.216635in}}{\pgfqpoint{3.887302in}{1.218830in}}{\pgfqpoint{3.891209in}{1.222736in}}%
\pgfpathcurveto{\pgfqpoint{3.895115in}{1.226643in}}{\pgfqpoint{3.897311in}{1.231943in}}{\pgfqpoint{3.897311in}{1.237468in}}%
\pgfpathcurveto{\pgfqpoint{3.897311in}{1.242993in}}{\pgfqpoint{3.895115in}{1.248292in}}{\pgfqpoint{3.891209in}{1.252199in}}%
\pgfpathcurveto{\pgfqpoint{3.887302in}{1.256106in}}{\pgfqpoint{3.882002in}{1.258301in}}{\pgfqpoint{3.876477in}{1.258301in}}%
\pgfpathcurveto{\pgfqpoint{3.870952in}{1.258301in}}{\pgfqpoint{3.865653in}{1.256106in}}{\pgfqpoint{3.861746in}{1.252199in}}%
\pgfpathcurveto{\pgfqpoint{3.857839in}{1.248292in}}{\pgfqpoint{3.855644in}{1.242993in}}{\pgfqpoint{3.855644in}{1.237468in}}%
\pgfpathcurveto{\pgfqpoint{3.855644in}{1.231943in}}{\pgfqpoint{3.857839in}{1.226643in}}{\pgfqpoint{3.861746in}{1.222736in}}%
\pgfpathcurveto{\pgfqpoint{3.865653in}{1.218830in}}{\pgfqpoint{3.870952in}{1.216635in}}{\pgfqpoint{3.876477in}{1.216635in}}%
\pgfpathclose%
\pgfusepath{stroke,fill}%
\end{pgfscope}%
\begin{pgfscope}%
\pgfpathrectangle{\pgfqpoint{0.562500in}{0.275000in}}{\pgfqpoint{3.487500in}{1.925000in}}%
\pgfusepath{clip}%
\pgfsetbuttcap%
\pgfsetroundjoin%
\definecolor{currentfill}{rgb}{0.000000,0.000000,0.000000}%
\pgfsetfillcolor{currentfill}%
\pgfsetlinewidth{1.003750pt}%
\definecolor{currentstroke}{rgb}{0.000000,0.000000,0.000000}%
\pgfsetstrokecolor{currentstroke}%
\pgfsetdash{}{0pt}%
\pgfpathmoveto{\pgfqpoint{3.876477in}{1.216635in}}%
\pgfpathcurveto{\pgfqpoint{3.882002in}{1.216635in}}{\pgfqpoint{3.887302in}{1.218830in}}{\pgfqpoint{3.891209in}{1.222736in}}%
\pgfpathcurveto{\pgfqpoint{3.895115in}{1.226643in}}{\pgfqpoint{3.897311in}{1.231943in}}{\pgfqpoint{3.897311in}{1.237468in}}%
\pgfpathcurveto{\pgfqpoint{3.897311in}{1.242993in}}{\pgfqpoint{3.895115in}{1.248292in}}{\pgfqpoint{3.891209in}{1.252199in}}%
\pgfpathcurveto{\pgfqpoint{3.887302in}{1.256106in}}{\pgfqpoint{3.882002in}{1.258301in}}{\pgfqpoint{3.876477in}{1.258301in}}%
\pgfpathcurveto{\pgfqpoint{3.870952in}{1.258301in}}{\pgfqpoint{3.865653in}{1.256106in}}{\pgfqpoint{3.861746in}{1.252199in}}%
\pgfpathcurveto{\pgfqpoint{3.857839in}{1.248292in}}{\pgfqpoint{3.855644in}{1.242993in}}{\pgfqpoint{3.855644in}{1.237468in}}%
\pgfpathcurveto{\pgfqpoint{3.855644in}{1.231943in}}{\pgfqpoint{3.857839in}{1.226643in}}{\pgfqpoint{3.861746in}{1.222736in}}%
\pgfpathcurveto{\pgfqpoint{3.865653in}{1.218830in}}{\pgfqpoint{3.870952in}{1.216635in}}{\pgfqpoint{3.876477in}{1.216635in}}%
\pgfpathclose%
\pgfusepath{stroke,fill}%
\end{pgfscope}%
\begin{pgfscope}%
\pgfpathrectangle{\pgfqpoint{0.562500in}{0.275000in}}{\pgfqpoint{3.487500in}{1.925000in}}%
\pgfusepath{clip}%
\pgfsetbuttcap%
\pgfsetroundjoin%
\definecolor{currentfill}{rgb}{0.000000,0.000000,0.000000}%
\pgfsetfillcolor{currentfill}%
\pgfsetlinewidth{1.003750pt}%
\definecolor{currentstroke}{rgb}{0.000000,0.000000,0.000000}%
\pgfsetstrokecolor{currentstroke}%
\pgfsetdash{}{0pt}%
\pgfpathmoveto{\pgfqpoint{3.876477in}{2.076667in}}%
\pgfpathcurveto{\pgfqpoint{3.882002in}{2.076667in}}{\pgfqpoint{3.887302in}{2.078862in}}{\pgfqpoint{3.891209in}{2.082769in}}%
\pgfpathcurveto{\pgfqpoint{3.895115in}{2.086675in}}{\pgfqpoint{3.897311in}{2.091975in}}{\pgfqpoint{3.897311in}{2.097500in}}%
\pgfpathcurveto{\pgfqpoint{3.897311in}{2.103025in}}{\pgfqpoint{3.895115in}{2.108325in}}{\pgfqpoint{3.891209in}{2.112231in}}%
\pgfpathcurveto{\pgfqpoint{3.887302in}{2.116138in}}{\pgfqpoint{3.882002in}{2.118333in}}{\pgfqpoint{3.876477in}{2.118333in}}%
\pgfpathcurveto{\pgfqpoint{3.870952in}{2.118333in}}{\pgfqpoint{3.865653in}{2.116138in}}{\pgfqpoint{3.861746in}{2.112231in}}%
\pgfpathcurveto{\pgfqpoint{3.857839in}{2.108325in}}{\pgfqpoint{3.855644in}{2.103025in}}{\pgfqpoint{3.855644in}{2.097500in}}%
\pgfpathcurveto{\pgfqpoint{3.855644in}{2.091975in}}{\pgfqpoint{3.857839in}{2.086675in}}{\pgfqpoint{3.861746in}{2.082769in}}%
\pgfpathcurveto{\pgfqpoint{3.865653in}{2.078862in}}{\pgfqpoint{3.870952in}{2.076667in}}{\pgfqpoint{3.876477in}{2.076667in}}%
\pgfpathclose%
\pgfusepath{stroke,fill}%
\end{pgfscope}%
\begin{pgfscope}%
\pgfpathrectangle{\pgfqpoint{0.562500in}{0.275000in}}{\pgfqpoint{3.487500in}{1.925000in}}%
\pgfusepath{clip}%
\pgfsetbuttcap%
\pgfsetroundjoin%
\definecolor{currentfill}{rgb}{0.000000,0.000000,0.000000}%
\pgfsetfillcolor{currentfill}%
\pgfsetlinewidth{1.003750pt}%
\definecolor{currentstroke}{rgb}{0.000000,0.000000,0.000000}%
\pgfsetstrokecolor{currentstroke}%
\pgfsetdash{}{0pt}%
\pgfpathmoveto{\pgfqpoint{3.876477in}{1.216635in}}%
\pgfpathcurveto{\pgfqpoint{3.882002in}{1.216635in}}{\pgfqpoint{3.887302in}{1.218830in}}{\pgfqpoint{3.891209in}{1.222736in}}%
\pgfpathcurveto{\pgfqpoint{3.895115in}{1.226643in}}{\pgfqpoint{3.897311in}{1.231943in}}{\pgfqpoint{3.897311in}{1.237468in}}%
\pgfpathcurveto{\pgfqpoint{3.897311in}{1.242993in}}{\pgfqpoint{3.895115in}{1.248292in}}{\pgfqpoint{3.891209in}{1.252199in}}%
\pgfpathcurveto{\pgfqpoint{3.887302in}{1.256106in}}{\pgfqpoint{3.882002in}{1.258301in}}{\pgfqpoint{3.876477in}{1.258301in}}%
\pgfpathcurveto{\pgfqpoint{3.870952in}{1.258301in}}{\pgfqpoint{3.865653in}{1.256106in}}{\pgfqpoint{3.861746in}{1.252199in}}%
\pgfpathcurveto{\pgfqpoint{3.857839in}{1.248292in}}{\pgfqpoint{3.855644in}{1.242993in}}{\pgfqpoint{3.855644in}{1.237468in}}%
\pgfpathcurveto{\pgfqpoint{3.855644in}{1.231943in}}{\pgfqpoint{3.857839in}{1.226643in}}{\pgfqpoint{3.861746in}{1.222736in}}%
\pgfpathcurveto{\pgfqpoint{3.865653in}{1.218830in}}{\pgfqpoint{3.870952in}{1.216635in}}{\pgfqpoint{3.876477in}{1.216635in}}%
\pgfpathclose%
\pgfusepath{stroke,fill}%
\end{pgfscope}%
\begin{pgfscope}%
\pgfpathrectangle{\pgfqpoint{0.562500in}{0.275000in}}{\pgfqpoint{3.487500in}{1.925000in}}%
\pgfusepath{clip}%
\pgfsetbuttcap%
\pgfsetroundjoin%
\definecolor{currentfill}{rgb}{0.000000,0.000000,0.000000}%
\pgfsetfillcolor{currentfill}%
\pgfsetlinewidth{1.003750pt}%
\definecolor{currentstroke}{rgb}{0.000000,0.000000,0.000000}%
\pgfsetstrokecolor{currentstroke}%
\pgfsetdash{}{0pt}%
\pgfpathmoveto{\pgfqpoint{3.876477in}{2.076667in}}%
\pgfpathcurveto{\pgfqpoint{3.882002in}{2.076667in}}{\pgfqpoint{3.887302in}{2.078862in}}{\pgfqpoint{3.891209in}{2.082769in}}%
\pgfpathcurveto{\pgfqpoint{3.895115in}{2.086675in}}{\pgfqpoint{3.897311in}{2.091975in}}{\pgfqpoint{3.897311in}{2.097500in}}%
\pgfpathcurveto{\pgfqpoint{3.897311in}{2.103025in}}{\pgfqpoint{3.895115in}{2.108325in}}{\pgfqpoint{3.891209in}{2.112231in}}%
\pgfpathcurveto{\pgfqpoint{3.887302in}{2.116138in}}{\pgfqpoint{3.882002in}{2.118333in}}{\pgfqpoint{3.876477in}{2.118333in}}%
\pgfpathcurveto{\pgfqpoint{3.870952in}{2.118333in}}{\pgfqpoint{3.865653in}{2.116138in}}{\pgfqpoint{3.861746in}{2.112231in}}%
\pgfpathcurveto{\pgfqpoint{3.857839in}{2.108325in}}{\pgfqpoint{3.855644in}{2.103025in}}{\pgfqpoint{3.855644in}{2.097500in}}%
\pgfpathcurveto{\pgfqpoint{3.855644in}{2.091975in}}{\pgfqpoint{3.857839in}{2.086675in}}{\pgfqpoint{3.861746in}{2.082769in}}%
\pgfpathcurveto{\pgfqpoint{3.865653in}{2.078862in}}{\pgfqpoint{3.870952in}{2.076667in}}{\pgfqpoint{3.876477in}{2.076667in}}%
\pgfpathclose%
\pgfusepath{stroke,fill}%
\end{pgfscope}%
\begin{pgfscope}%
\pgfpathrectangle{\pgfqpoint{0.562500in}{0.275000in}}{\pgfqpoint{3.487500in}{1.925000in}}%
\pgfusepath{clip}%
\pgfsetbuttcap%
\pgfsetroundjoin%
\definecolor{currentfill}{rgb}{0.000000,0.000000,0.000000}%
\pgfsetfillcolor{currentfill}%
\pgfsetlinewidth{1.003750pt}%
\definecolor{currentstroke}{rgb}{0.000000,0.000000,0.000000}%
\pgfsetstrokecolor{currentstroke}%
\pgfsetdash{}{0pt}%
\pgfpathmoveto{\pgfqpoint{3.876477in}{1.216635in}}%
\pgfpathcurveto{\pgfqpoint{3.882002in}{1.216635in}}{\pgfqpoint{3.887302in}{1.218830in}}{\pgfqpoint{3.891209in}{1.222736in}}%
\pgfpathcurveto{\pgfqpoint{3.895115in}{1.226643in}}{\pgfqpoint{3.897311in}{1.231943in}}{\pgfqpoint{3.897311in}{1.237468in}}%
\pgfpathcurveto{\pgfqpoint{3.897311in}{1.242993in}}{\pgfqpoint{3.895115in}{1.248292in}}{\pgfqpoint{3.891209in}{1.252199in}}%
\pgfpathcurveto{\pgfqpoint{3.887302in}{1.256106in}}{\pgfqpoint{3.882002in}{1.258301in}}{\pgfqpoint{3.876477in}{1.258301in}}%
\pgfpathcurveto{\pgfqpoint{3.870952in}{1.258301in}}{\pgfqpoint{3.865653in}{1.256106in}}{\pgfqpoint{3.861746in}{1.252199in}}%
\pgfpathcurveto{\pgfqpoint{3.857839in}{1.248292in}}{\pgfqpoint{3.855644in}{1.242993in}}{\pgfqpoint{3.855644in}{1.237468in}}%
\pgfpathcurveto{\pgfqpoint{3.855644in}{1.231943in}}{\pgfqpoint{3.857839in}{1.226643in}}{\pgfqpoint{3.861746in}{1.222736in}}%
\pgfpathcurveto{\pgfqpoint{3.865653in}{1.218830in}}{\pgfqpoint{3.870952in}{1.216635in}}{\pgfqpoint{3.876477in}{1.216635in}}%
\pgfpathclose%
\pgfusepath{stroke,fill}%
\end{pgfscope}%
\begin{pgfscope}%
\pgfpathrectangle{\pgfqpoint{0.562500in}{0.275000in}}{\pgfqpoint{3.487500in}{1.925000in}}%
\pgfusepath{clip}%
\pgfsetbuttcap%
\pgfsetroundjoin%
\definecolor{currentfill}{rgb}{0.000000,0.000000,0.000000}%
\pgfsetfillcolor{currentfill}%
\pgfsetlinewidth{1.003750pt}%
\definecolor{currentstroke}{rgb}{0.000000,0.000000,0.000000}%
\pgfsetstrokecolor{currentstroke}%
\pgfsetdash{}{0pt}%
\pgfpathmoveto{\pgfqpoint{3.876477in}{1.216635in}}%
\pgfpathcurveto{\pgfqpoint{3.882002in}{1.216635in}}{\pgfqpoint{3.887302in}{1.218830in}}{\pgfqpoint{3.891209in}{1.222736in}}%
\pgfpathcurveto{\pgfqpoint{3.895115in}{1.226643in}}{\pgfqpoint{3.897311in}{1.231943in}}{\pgfqpoint{3.897311in}{1.237468in}}%
\pgfpathcurveto{\pgfqpoint{3.897311in}{1.242993in}}{\pgfqpoint{3.895115in}{1.248292in}}{\pgfqpoint{3.891209in}{1.252199in}}%
\pgfpathcurveto{\pgfqpoint{3.887302in}{1.256106in}}{\pgfqpoint{3.882002in}{1.258301in}}{\pgfqpoint{3.876477in}{1.258301in}}%
\pgfpathcurveto{\pgfqpoint{3.870952in}{1.258301in}}{\pgfqpoint{3.865653in}{1.256106in}}{\pgfqpoint{3.861746in}{1.252199in}}%
\pgfpathcurveto{\pgfqpoint{3.857839in}{1.248292in}}{\pgfqpoint{3.855644in}{1.242993in}}{\pgfqpoint{3.855644in}{1.237468in}}%
\pgfpathcurveto{\pgfqpoint{3.855644in}{1.231943in}}{\pgfqpoint{3.857839in}{1.226643in}}{\pgfqpoint{3.861746in}{1.222736in}}%
\pgfpathcurveto{\pgfqpoint{3.865653in}{1.218830in}}{\pgfqpoint{3.870952in}{1.216635in}}{\pgfqpoint{3.876477in}{1.216635in}}%
\pgfpathclose%
\pgfusepath{stroke,fill}%
\end{pgfscope}%
\begin{pgfscope}%
\pgfpathrectangle{\pgfqpoint{0.562500in}{0.275000in}}{\pgfqpoint{3.487500in}{1.925000in}}%
\pgfusepath{clip}%
\pgfsetbuttcap%
\pgfsetroundjoin%
\definecolor{currentfill}{rgb}{0.000000,0.000000,0.000000}%
\pgfsetfillcolor{currentfill}%
\pgfsetlinewidth{1.003750pt}%
\definecolor{currentstroke}{rgb}{0.000000,0.000000,0.000000}%
\pgfsetstrokecolor{currentstroke}%
\pgfsetdash{}{0pt}%
\pgfpathmoveto{\pgfqpoint{3.876477in}{1.216635in}}%
\pgfpathcurveto{\pgfqpoint{3.882002in}{1.216635in}}{\pgfqpoint{3.887302in}{1.218830in}}{\pgfqpoint{3.891209in}{1.222736in}}%
\pgfpathcurveto{\pgfqpoint{3.895115in}{1.226643in}}{\pgfqpoint{3.897311in}{1.231943in}}{\pgfqpoint{3.897311in}{1.237468in}}%
\pgfpathcurveto{\pgfqpoint{3.897311in}{1.242993in}}{\pgfqpoint{3.895115in}{1.248292in}}{\pgfqpoint{3.891209in}{1.252199in}}%
\pgfpathcurveto{\pgfqpoint{3.887302in}{1.256106in}}{\pgfqpoint{3.882002in}{1.258301in}}{\pgfqpoint{3.876477in}{1.258301in}}%
\pgfpathcurveto{\pgfqpoint{3.870952in}{1.258301in}}{\pgfqpoint{3.865653in}{1.256106in}}{\pgfqpoint{3.861746in}{1.252199in}}%
\pgfpathcurveto{\pgfqpoint{3.857839in}{1.248292in}}{\pgfqpoint{3.855644in}{1.242993in}}{\pgfqpoint{3.855644in}{1.237468in}}%
\pgfpathcurveto{\pgfqpoint{3.855644in}{1.231943in}}{\pgfqpoint{3.857839in}{1.226643in}}{\pgfqpoint{3.861746in}{1.222736in}}%
\pgfpathcurveto{\pgfqpoint{3.865653in}{1.218830in}}{\pgfqpoint{3.870952in}{1.216635in}}{\pgfqpoint{3.876477in}{1.216635in}}%
\pgfpathclose%
\pgfusepath{stroke,fill}%
\end{pgfscope}%
\begin{pgfscope}%
\pgfpathrectangle{\pgfqpoint{0.562500in}{0.275000in}}{\pgfqpoint{3.487500in}{1.925000in}}%
\pgfusepath{clip}%
\pgfsetbuttcap%
\pgfsetroundjoin%
\definecolor{currentfill}{rgb}{0.000000,0.000000,0.000000}%
\pgfsetfillcolor{currentfill}%
\pgfsetlinewidth{1.003750pt}%
\definecolor{currentstroke}{rgb}{0.000000,0.000000,0.000000}%
\pgfsetstrokecolor{currentstroke}%
\pgfsetdash{}{0pt}%
\pgfpathmoveto{\pgfqpoint{3.876477in}{2.076667in}}%
\pgfpathcurveto{\pgfqpoint{3.882002in}{2.076667in}}{\pgfqpoint{3.887302in}{2.078862in}}{\pgfqpoint{3.891209in}{2.082769in}}%
\pgfpathcurveto{\pgfqpoint{3.895115in}{2.086675in}}{\pgfqpoint{3.897311in}{2.091975in}}{\pgfqpoint{3.897311in}{2.097500in}}%
\pgfpathcurveto{\pgfqpoint{3.897311in}{2.103025in}}{\pgfqpoint{3.895115in}{2.108325in}}{\pgfqpoint{3.891209in}{2.112231in}}%
\pgfpathcurveto{\pgfqpoint{3.887302in}{2.116138in}}{\pgfqpoint{3.882002in}{2.118333in}}{\pgfqpoint{3.876477in}{2.118333in}}%
\pgfpathcurveto{\pgfqpoint{3.870952in}{2.118333in}}{\pgfqpoint{3.865653in}{2.116138in}}{\pgfqpoint{3.861746in}{2.112231in}}%
\pgfpathcurveto{\pgfqpoint{3.857839in}{2.108325in}}{\pgfqpoint{3.855644in}{2.103025in}}{\pgfqpoint{3.855644in}{2.097500in}}%
\pgfpathcurveto{\pgfqpoint{3.855644in}{2.091975in}}{\pgfqpoint{3.857839in}{2.086675in}}{\pgfqpoint{3.861746in}{2.082769in}}%
\pgfpathcurveto{\pgfqpoint{3.865653in}{2.078862in}}{\pgfqpoint{3.870952in}{2.076667in}}{\pgfqpoint{3.876477in}{2.076667in}}%
\pgfpathclose%
\pgfusepath{stroke,fill}%
\end{pgfscope}%
\begin{pgfscope}%
\pgfpathrectangle{\pgfqpoint{0.562500in}{0.275000in}}{\pgfqpoint{3.487500in}{1.925000in}}%
\pgfusepath{clip}%
\pgfsetbuttcap%
\pgfsetroundjoin%
\definecolor{currentfill}{rgb}{0.000000,0.000000,0.000000}%
\pgfsetfillcolor{currentfill}%
\pgfsetlinewidth{1.003750pt}%
\definecolor{currentstroke}{rgb}{0.000000,0.000000,0.000000}%
\pgfsetstrokecolor{currentstroke}%
\pgfsetdash{}{0pt}%
\pgfpathmoveto{\pgfqpoint{3.876477in}{2.076667in}}%
\pgfpathcurveto{\pgfqpoint{3.882002in}{2.076667in}}{\pgfqpoint{3.887302in}{2.078862in}}{\pgfqpoint{3.891209in}{2.082769in}}%
\pgfpathcurveto{\pgfqpoint{3.895115in}{2.086675in}}{\pgfqpoint{3.897311in}{2.091975in}}{\pgfqpoint{3.897311in}{2.097500in}}%
\pgfpathcurveto{\pgfqpoint{3.897311in}{2.103025in}}{\pgfqpoint{3.895115in}{2.108325in}}{\pgfqpoint{3.891209in}{2.112231in}}%
\pgfpathcurveto{\pgfqpoint{3.887302in}{2.116138in}}{\pgfqpoint{3.882002in}{2.118333in}}{\pgfqpoint{3.876477in}{2.118333in}}%
\pgfpathcurveto{\pgfqpoint{3.870952in}{2.118333in}}{\pgfqpoint{3.865653in}{2.116138in}}{\pgfqpoint{3.861746in}{2.112231in}}%
\pgfpathcurveto{\pgfqpoint{3.857839in}{2.108325in}}{\pgfqpoint{3.855644in}{2.103025in}}{\pgfqpoint{3.855644in}{2.097500in}}%
\pgfpathcurveto{\pgfqpoint{3.855644in}{2.091975in}}{\pgfqpoint{3.857839in}{2.086675in}}{\pgfqpoint{3.861746in}{2.082769in}}%
\pgfpathcurveto{\pgfqpoint{3.865653in}{2.078862in}}{\pgfqpoint{3.870952in}{2.076667in}}{\pgfqpoint{3.876477in}{2.076667in}}%
\pgfpathclose%
\pgfusepath{stroke,fill}%
\end{pgfscope}%
\begin{pgfscope}%
\pgfpathrectangle{\pgfqpoint{0.562500in}{0.275000in}}{\pgfqpoint{3.487500in}{1.925000in}}%
\pgfusepath{clip}%
\pgfsetbuttcap%
\pgfsetroundjoin%
\definecolor{currentfill}{rgb}{0.000000,0.000000,0.000000}%
\pgfsetfillcolor{currentfill}%
\pgfsetlinewidth{1.003750pt}%
\definecolor{currentstroke}{rgb}{0.000000,0.000000,0.000000}%
\pgfsetstrokecolor{currentstroke}%
\pgfsetdash{}{0pt}%
\pgfpathmoveto{\pgfqpoint{3.876477in}{1.216635in}}%
\pgfpathcurveto{\pgfqpoint{3.882002in}{1.216635in}}{\pgfqpoint{3.887302in}{1.218830in}}{\pgfqpoint{3.891209in}{1.222736in}}%
\pgfpathcurveto{\pgfqpoint{3.895115in}{1.226643in}}{\pgfqpoint{3.897311in}{1.231943in}}{\pgfqpoint{3.897311in}{1.237468in}}%
\pgfpathcurveto{\pgfqpoint{3.897311in}{1.242993in}}{\pgfqpoint{3.895115in}{1.248292in}}{\pgfqpoint{3.891209in}{1.252199in}}%
\pgfpathcurveto{\pgfqpoint{3.887302in}{1.256106in}}{\pgfqpoint{3.882002in}{1.258301in}}{\pgfqpoint{3.876477in}{1.258301in}}%
\pgfpathcurveto{\pgfqpoint{3.870952in}{1.258301in}}{\pgfqpoint{3.865653in}{1.256106in}}{\pgfqpoint{3.861746in}{1.252199in}}%
\pgfpathcurveto{\pgfqpoint{3.857839in}{1.248292in}}{\pgfqpoint{3.855644in}{1.242993in}}{\pgfqpoint{3.855644in}{1.237468in}}%
\pgfpathcurveto{\pgfqpoint{3.855644in}{1.231943in}}{\pgfqpoint{3.857839in}{1.226643in}}{\pgfqpoint{3.861746in}{1.222736in}}%
\pgfpathcurveto{\pgfqpoint{3.865653in}{1.218830in}}{\pgfqpoint{3.870952in}{1.216635in}}{\pgfqpoint{3.876477in}{1.216635in}}%
\pgfpathclose%
\pgfusepath{stroke,fill}%
\end{pgfscope}%
\begin{pgfscope}%
\pgfpathrectangle{\pgfqpoint{0.562500in}{0.275000in}}{\pgfqpoint{3.487500in}{1.925000in}}%
\pgfusepath{clip}%
\pgfsetbuttcap%
\pgfsetroundjoin%
\definecolor{currentfill}{rgb}{0.000000,0.000000,0.000000}%
\pgfsetfillcolor{currentfill}%
\pgfsetlinewidth{1.003750pt}%
\definecolor{currentstroke}{rgb}{0.000000,0.000000,0.000000}%
\pgfsetstrokecolor{currentstroke}%
\pgfsetdash{}{0pt}%
\pgfpathmoveto{\pgfqpoint{3.876477in}{2.076667in}}%
\pgfpathcurveto{\pgfqpoint{3.882002in}{2.076667in}}{\pgfqpoint{3.887302in}{2.078862in}}{\pgfqpoint{3.891209in}{2.082769in}}%
\pgfpathcurveto{\pgfqpoint{3.895115in}{2.086675in}}{\pgfqpoint{3.897311in}{2.091975in}}{\pgfqpoint{3.897311in}{2.097500in}}%
\pgfpathcurveto{\pgfqpoint{3.897311in}{2.103025in}}{\pgfqpoint{3.895115in}{2.108325in}}{\pgfqpoint{3.891209in}{2.112231in}}%
\pgfpathcurveto{\pgfqpoint{3.887302in}{2.116138in}}{\pgfqpoint{3.882002in}{2.118333in}}{\pgfqpoint{3.876477in}{2.118333in}}%
\pgfpathcurveto{\pgfqpoint{3.870952in}{2.118333in}}{\pgfqpoint{3.865653in}{2.116138in}}{\pgfqpoint{3.861746in}{2.112231in}}%
\pgfpathcurveto{\pgfqpoint{3.857839in}{2.108325in}}{\pgfqpoint{3.855644in}{2.103025in}}{\pgfqpoint{3.855644in}{2.097500in}}%
\pgfpathcurveto{\pgfqpoint{3.855644in}{2.091975in}}{\pgfqpoint{3.857839in}{2.086675in}}{\pgfqpoint{3.861746in}{2.082769in}}%
\pgfpathcurveto{\pgfqpoint{3.865653in}{2.078862in}}{\pgfqpoint{3.870952in}{2.076667in}}{\pgfqpoint{3.876477in}{2.076667in}}%
\pgfpathclose%
\pgfusepath{stroke,fill}%
\end{pgfscope}%
\begin{pgfscope}%
\pgfpathrectangle{\pgfqpoint{0.562500in}{0.275000in}}{\pgfqpoint{3.487500in}{1.925000in}}%
\pgfusepath{clip}%
\pgfsetbuttcap%
\pgfsetroundjoin%
\definecolor{currentfill}{rgb}{0.000000,0.000000,0.000000}%
\pgfsetfillcolor{currentfill}%
\pgfsetlinewidth{1.003750pt}%
\definecolor{currentstroke}{rgb}{0.000000,0.000000,0.000000}%
\pgfsetstrokecolor{currentstroke}%
\pgfsetdash{}{0pt}%
\pgfpathmoveto{\pgfqpoint{3.876477in}{1.216635in}}%
\pgfpathcurveto{\pgfqpoint{3.882002in}{1.216635in}}{\pgfqpoint{3.887302in}{1.218830in}}{\pgfqpoint{3.891209in}{1.222736in}}%
\pgfpathcurveto{\pgfqpoint{3.895115in}{1.226643in}}{\pgfqpoint{3.897311in}{1.231943in}}{\pgfqpoint{3.897311in}{1.237468in}}%
\pgfpathcurveto{\pgfqpoint{3.897311in}{1.242993in}}{\pgfqpoint{3.895115in}{1.248292in}}{\pgfqpoint{3.891209in}{1.252199in}}%
\pgfpathcurveto{\pgfqpoint{3.887302in}{1.256106in}}{\pgfqpoint{3.882002in}{1.258301in}}{\pgfqpoint{3.876477in}{1.258301in}}%
\pgfpathcurveto{\pgfqpoint{3.870952in}{1.258301in}}{\pgfqpoint{3.865653in}{1.256106in}}{\pgfqpoint{3.861746in}{1.252199in}}%
\pgfpathcurveto{\pgfqpoint{3.857839in}{1.248292in}}{\pgfqpoint{3.855644in}{1.242993in}}{\pgfqpoint{3.855644in}{1.237468in}}%
\pgfpathcurveto{\pgfqpoint{3.855644in}{1.231943in}}{\pgfqpoint{3.857839in}{1.226643in}}{\pgfqpoint{3.861746in}{1.222736in}}%
\pgfpathcurveto{\pgfqpoint{3.865653in}{1.218830in}}{\pgfqpoint{3.870952in}{1.216635in}}{\pgfqpoint{3.876477in}{1.216635in}}%
\pgfpathclose%
\pgfusepath{stroke,fill}%
\end{pgfscope}%
\begin{pgfscope}%
\pgfpathrectangle{\pgfqpoint{0.562500in}{0.275000in}}{\pgfqpoint{3.487500in}{1.925000in}}%
\pgfusepath{clip}%
\pgfsetbuttcap%
\pgfsetroundjoin%
\definecolor{currentfill}{rgb}{0.000000,0.000000,0.000000}%
\pgfsetfillcolor{currentfill}%
\pgfsetlinewidth{1.003750pt}%
\definecolor{currentstroke}{rgb}{0.000000,0.000000,0.000000}%
\pgfsetstrokecolor{currentstroke}%
\pgfsetdash{}{0pt}%
\pgfpathmoveto{\pgfqpoint{3.876477in}{2.076667in}}%
\pgfpathcurveto{\pgfqpoint{3.882002in}{2.076667in}}{\pgfqpoint{3.887302in}{2.078862in}}{\pgfqpoint{3.891209in}{2.082769in}}%
\pgfpathcurveto{\pgfqpoint{3.895115in}{2.086675in}}{\pgfqpoint{3.897311in}{2.091975in}}{\pgfqpoint{3.897311in}{2.097500in}}%
\pgfpathcurveto{\pgfqpoint{3.897311in}{2.103025in}}{\pgfqpoint{3.895115in}{2.108325in}}{\pgfqpoint{3.891209in}{2.112231in}}%
\pgfpathcurveto{\pgfqpoint{3.887302in}{2.116138in}}{\pgfqpoint{3.882002in}{2.118333in}}{\pgfqpoint{3.876477in}{2.118333in}}%
\pgfpathcurveto{\pgfqpoint{3.870952in}{2.118333in}}{\pgfqpoint{3.865653in}{2.116138in}}{\pgfqpoint{3.861746in}{2.112231in}}%
\pgfpathcurveto{\pgfqpoint{3.857839in}{2.108325in}}{\pgfqpoint{3.855644in}{2.103025in}}{\pgfqpoint{3.855644in}{2.097500in}}%
\pgfpathcurveto{\pgfqpoint{3.855644in}{2.091975in}}{\pgfqpoint{3.857839in}{2.086675in}}{\pgfqpoint{3.861746in}{2.082769in}}%
\pgfpathcurveto{\pgfqpoint{3.865653in}{2.078862in}}{\pgfqpoint{3.870952in}{2.076667in}}{\pgfqpoint{3.876477in}{2.076667in}}%
\pgfpathclose%
\pgfusepath{stroke,fill}%
\end{pgfscope}%
\begin{pgfscope}%
\pgfpathrectangle{\pgfqpoint{0.562500in}{0.275000in}}{\pgfqpoint{3.487500in}{1.925000in}}%
\pgfusepath{clip}%
\pgfsetbuttcap%
\pgfsetroundjoin%
\definecolor{currentfill}{rgb}{0.000000,0.000000,0.000000}%
\pgfsetfillcolor{currentfill}%
\pgfsetlinewidth{1.003750pt}%
\definecolor{currentstroke}{rgb}{0.000000,0.000000,0.000000}%
\pgfsetstrokecolor{currentstroke}%
\pgfsetdash{}{0pt}%
\pgfpathmoveto{\pgfqpoint{3.876477in}{1.216635in}}%
\pgfpathcurveto{\pgfqpoint{3.882002in}{1.216635in}}{\pgfqpoint{3.887302in}{1.218830in}}{\pgfqpoint{3.891209in}{1.222736in}}%
\pgfpathcurveto{\pgfqpoint{3.895115in}{1.226643in}}{\pgfqpoint{3.897311in}{1.231943in}}{\pgfqpoint{3.897311in}{1.237468in}}%
\pgfpathcurveto{\pgfqpoint{3.897311in}{1.242993in}}{\pgfqpoint{3.895115in}{1.248292in}}{\pgfqpoint{3.891209in}{1.252199in}}%
\pgfpathcurveto{\pgfqpoint{3.887302in}{1.256106in}}{\pgfqpoint{3.882002in}{1.258301in}}{\pgfqpoint{3.876477in}{1.258301in}}%
\pgfpathcurveto{\pgfqpoint{3.870952in}{1.258301in}}{\pgfqpoint{3.865653in}{1.256106in}}{\pgfqpoint{3.861746in}{1.252199in}}%
\pgfpathcurveto{\pgfqpoint{3.857839in}{1.248292in}}{\pgfqpoint{3.855644in}{1.242993in}}{\pgfqpoint{3.855644in}{1.237468in}}%
\pgfpathcurveto{\pgfqpoint{3.855644in}{1.231943in}}{\pgfqpoint{3.857839in}{1.226643in}}{\pgfqpoint{3.861746in}{1.222736in}}%
\pgfpathcurveto{\pgfqpoint{3.865653in}{1.218830in}}{\pgfqpoint{3.870952in}{1.216635in}}{\pgfqpoint{3.876477in}{1.216635in}}%
\pgfpathclose%
\pgfusepath{stroke,fill}%
\end{pgfscope}%
\begin{pgfscope}%
\pgfpathrectangle{\pgfqpoint{0.562500in}{0.275000in}}{\pgfqpoint{3.487500in}{1.925000in}}%
\pgfusepath{clip}%
\pgfsetbuttcap%
\pgfsetroundjoin%
\definecolor{currentfill}{rgb}{0.000000,0.000000,0.000000}%
\pgfsetfillcolor{currentfill}%
\pgfsetlinewidth{1.003750pt}%
\definecolor{currentstroke}{rgb}{0.000000,0.000000,0.000000}%
\pgfsetstrokecolor{currentstroke}%
\pgfsetdash{}{0pt}%
\pgfpathmoveto{\pgfqpoint{3.876477in}{2.076667in}}%
\pgfpathcurveto{\pgfqpoint{3.882002in}{2.076667in}}{\pgfqpoint{3.887302in}{2.078862in}}{\pgfqpoint{3.891209in}{2.082769in}}%
\pgfpathcurveto{\pgfqpoint{3.895115in}{2.086675in}}{\pgfqpoint{3.897311in}{2.091975in}}{\pgfqpoint{3.897311in}{2.097500in}}%
\pgfpathcurveto{\pgfqpoint{3.897311in}{2.103025in}}{\pgfqpoint{3.895115in}{2.108325in}}{\pgfqpoint{3.891209in}{2.112231in}}%
\pgfpathcurveto{\pgfqpoint{3.887302in}{2.116138in}}{\pgfqpoint{3.882002in}{2.118333in}}{\pgfqpoint{3.876477in}{2.118333in}}%
\pgfpathcurveto{\pgfqpoint{3.870952in}{2.118333in}}{\pgfqpoint{3.865653in}{2.116138in}}{\pgfqpoint{3.861746in}{2.112231in}}%
\pgfpathcurveto{\pgfqpoint{3.857839in}{2.108325in}}{\pgfqpoint{3.855644in}{2.103025in}}{\pgfqpoint{3.855644in}{2.097500in}}%
\pgfpathcurveto{\pgfqpoint{3.855644in}{2.091975in}}{\pgfqpoint{3.857839in}{2.086675in}}{\pgfqpoint{3.861746in}{2.082769in}}%
\pgfpathcurveto{\pgfqpoint{3.865653in}{2.078862in}}{\pgfqpoint{3.870952in}{2.076667in}}{\pgfqpoint{3.876477in}{2.076667in}}%
\pgfpathclose%
\pgfusepath{stroke,fill}%
\end{pgfscope}%
\begin{pgfscope}%
\pgfpathrectangle{\pgfqpoint{0.562500in}{0.275000in}}{\pgfqpoint{3.487500in}{1.925000in}}%
\pgfusepath{clip}%
\pgfsetbuttcap%
\pgfsetroundjoin%
\definecolor{currentfill}{rgb}{0.000000,0.000000,0.000000}%
\pgfsetfillcolor{currentfill}%
\pgfsetlinewidth{1.003750pt}%
\definecolor{currentstroke}{rgb}{0.000000,0.000000,0.000000}%
\pgfsetstrokecolor{currentstroke}%
\pgfsetdash{}{0pt}%
\pgfpathmoveto{\pgfqpoint{3.876477in}{1.216635in}}%
\pgfpathcurveto{\pgfqpoint{3.882002in}{1.216635in}}{\pgfqpoint{3.887302in}{1.218830in}}{\pgfqpoint{3.891209in}{1.222736in}}%
\pgfpathcurveto{\pgfqpoint{3.895115in}{1.226643in}}{\pgfqpoint{3.897311in}{1.231943in}}{\pgfqpoint{3.897311in}{1.237468in}}%
\pgfpathcurveto{\pgfqpoint{3.897311in}{1.242993in}}{\pgfqpoint{3.895115in}{1.248292in}}{\pgfqpoint{3.891209in}{1.252199in}}%
\pgfpathcurveto{\pgfqpoint{3.887302in}{1.256106in}}{\pgfqpoint{3.882002in}{1.258301in}}{\pgfqpoint{3.876477in}{1.258301in}}%
\pgfpathcurveto{\pgfqpoint{3.870952in}{1.258301in}}{\pgfqpoint{3.865653in}{1.256106in}}{\pgfqpoint{3.861746in}{1.252199in}}%
\pgfpathcurveto{\pgfqpoint{3.857839in}{1.248292in}}{\pgfqpoint{3.855644in}{1.242993in}}{\pgfqpoint{3.855644in}{1.237468in}}%
\pgfpathcurveto{\pgfqpoint{3.855644in}{1.231943in}}{\pgfqpoint{3.857839in}{1.226643in}}{\pgfqpoint{3.861746in}{1.222736in}}%
\pgfpathcurveto{\pgfqpoint{3.865653in}{1.218830in}}{\pgfqpoint{3.870952in}{1.216635in}}{\pgfqpoint{3.876477in}{1.216635in}}%
\pgfpathclose%
\pgfusepath{stroke,fill}%
\end{pgfscope}%
\begin{pgfscope}%
\pgfpathrectangle{\pgfqpoint{0.562500in}{0.275000in}}{\pgfqpoint{3.487500in}{1.925000in}}%
\pgfusepath{clip}%
\pgfsetbuttcap%
\pgfsetroundjoin%
\definecolor{currentfill}{rgb}{0.000000,0.000000,0.000000}%
\pgfsetfillcolor{currentfill}%
\pgfsetlinewidth{1.003750pt}%
\definecolor{currentstroke}{rgb}{0.000000,0.000000,0.000000}%
\pgfsetstrokecolor{currentstroke}%
\pgfsetdash{}{0pt}%
\pgfpathmoveto{\pgfqpoint{3.876477in}{2.076667in}}%
\pgfpathcurveto{\pgfqpoint{3.882002in}{2.076667in}}{\pgfqpoint{3.887302in}{2.078862in}}{\pgfqpoint{3.891209in}{2.082769in}}%
\pgfpathcurveto{\pgfqpoint{3.895115in}{2.086675in}}{\pgfqpoint{3.897311in}{2.091975in}}{\pgfqpoint{3.897311in}{2.097500in}}%
\pgfpathcurveto{\pgfqpoint{3.897311in}{2.103025in}}{\pgfqpoint{3.895115in}{2.108325in}}{\pgfqpoint{3.891209in}{2.112231in}}%
\pgfpathcurveto{\pgfqpoint{3.887302in}{2.116138in}}{\pgfqpoint{3.882002in}{2.118333in}}{\pgfqpoint{3.876477in}{2.118333in}}%
\pgfpathcurveto{\pgfqpoint{3.870952in}{2.118333in}}{\pgfqpoint{3.865653in}{2.116138in}}{\pgfqpoint{3.861746in}{2.112231in}}%
\pgfpathcurveto{\pgfqpoint{3.857839in}{2.108325in}}{\pgfqpoint{3.855644in}{2.103025in}}{\pgfqpoint{3.855644in}{2.097500in}}%
\pgfpathcurveto{\pgfqpoint{3.855644in}{2.091975in}}{\pgfqpoint{3.857839in}{2.086675in}}{\pgfqpoint{3.861746in}{2.082769in}}%
\pgfpathcurveto{\pgfqpoint{3.865653in}{2.078862in}}{\pgfqpoint{3.870952in}{2.076667in}}{\pgfqpoint{3.876477in}{2.076667in}}%
\pgfpathclose%
\pgfusepath{stroke,fill}%
\end{pgfscope}%
\begin{pgfscope}%
\pgfpathrectangle{\pgfqpoint{0.562500in}{0.275000in}}{\pgfqpoint{3.487500in}{1.925000in}}%
\pgfusepath{clip}%
\pgfsetbuttcap%
\pgfsetroundjoin%
\definecolor{currentfill}{rgb}{0.000000,0.000000,0.000000}%
\pgfsetfillcolor{currentfill}%
\pgfsetlinewidth{1.003750pt}%
\definecolor{currentstroke}{rgb}{0.000000,0.000000,0.000000}%
\pgfsetstrokecolor{currentstroke}%
\pgfsetdash{}{0pt}%
\pgfpathmoveto{\pgfqpoint{3.876477in}{1.216635in}}%
\pgfpathcurveto{\pgfqpoint{3.882002in}{1.216635in}}{\pgfqpoint{3.887302in}{1.218830in}}{\pgfqpoint{3.891209in}{1.222736in}}%
\pgfpathcurveto{\pgfqpoint{3.895115in}{1.226643in}}{\pgfqpoint{3.897311in}{1.231943in}}{\pgfqpoint{3.897311in}{1.237468in}}%
\pgfpathcurveto{\pgfqpoint{3.897311in}{1.242993in}}{\pgfqpoint{3.895115in}{1.248292in}}{\pgfqpoint{3.891209in}{1.252199in}}%
\pgfpathcurveto{\pgfqpoint{3.887302in}{1.256106in}}{\pgfqpoint{3.882002in}{1.258301in}}{\pgfqpoint{3.876477in}{1.258301in}}%
\pgfpathcurveto{\pgfqpoint{3.870952in}{1.258301in}}{\pgfqpoint{3.865653in}{1.256106in}}{\pgfqpoint{3.861746in}{1.252199in}}%
\pgfpathcurveto{\pgfqpoint{3.857839in}{1.248292in}}{\pgfqpoint{3.855644in}{1.242993in}}{\pgfqpoint{3.855644in}{1.237468in}}%
\pgfpathcurveto{\pgfqpoint{3.855644in}{1.231943in}}{\pgfqpoint{3.857839in}{1.226643in}}{\pgfqpoint{3.861746in}{1.222736in}}%
\pgfpathcurveto{\pgfqpoint{3.865653in}{1.218830in}}{\pgfqpoint{3.870952in}{1.216635in}}{\pgfqpoint{3.876477in}{1.216635in}}%
\pgfpathclose%
\pgfusepath{stroke,fill}%
\end{pgfscope}%
\begin{pgfscope}%
\pgfpathrectangle{\pgfqpoint{0.562500in}{0.275000in}}{\pgfqpoint{3.487500in}{1.925000in}}%
\pgfusepath{clip}%
\pgfsetbuttcap%
\pgfsetroundjoin%
\definecolor{currentfill}{rgb}{0.000000,0.000000,0.000000}%
\pgfsetfillcolor{currentfill}%
\pgfsetlinewidth{1.003750pt}%
\definecolor{currentstroke}{rgb}{0.000000,0.000000,0.000000}%
\pgfsetstrokecolor{currentstroke}%
\pgfsetdash{}{0pt}%
\pgfpathmoveto{\pgfqpoint{3.876477in}{2.076667in}}%
\pgfpathcurveto{\pgfqpoint{3.882002in}{2.076667in}}{\pgfqpoint{3.887302in}{2.078862in}}{\pgfqpoint{3.891209in}{2.082769in}}%
\pgfpathcurveto{\pgfqpoint{3.895115in}{2.086675in}}{\pgfqpoint{3.897311in}{2.091975in}}{\pgfqpoint{3.897311in}{2.097500in}}%
\pgfpathcurveto{\pgfqpoint{3.897311in}{2.103025in}}{\pgfqpoint{3.895115in}{2.108325in}}{\pgfqpoint{3.891209in}{2.112231in}}%
\pgfpathcurveto{\pgfqpoint{3.887302in}{2.116138in}}{\pgfqpoint{3.882002in}{2.118333in}}{\pgfqpoint{3.876477in}{2.118333in}}%
\pgfpathcurveto{\pgfqpoint{3.870952in}{2.118333in}}{\pgfqpoint{3.865653in}{2.116138in}}{\pgfqpoint{3.861746in}{2.112231in}}%
\pgfpathcurveto{\pgfqpoint{3.857839in}{2.108325in}}{\pgfqpoint{3.855644in}{2.103025in}}{\pgfqpoint{3.855644in}{2.097500in}}%
\pgfpathcurveto{\pgfqpoint{3.855644in}{2.091975in}}{\pgfqpoint{3.857839in}{2.086675in}}{\pgfqpoint{3.861746in}{2.082769in}}%
\pgfpathcurveto{\pgfqpoint{3.865653in}{2.078862in}}{\pgfqpoint{3.870952in}{2.076667in}}{\pgfqpoint{3.876477in}{2.076667in}}%
\pgfpathclose%
\pgfusepath{stroke,fill}%
\end{pgfscope}%
\begin{pgfscope}%
\pgfpathrectangle{\pgfqpoint{0.562500in}{0.275000in}}{\pgfqpoint{3.487500in}{1.925000in}}%
\pgfusepath{clip}%
\pgfsetbuttcap%
\pgfsetroundjoin%
\definecolor{currentfill}{rgb}{0.000000,0.000000,0.000000}%
\pgfsetfillcolor{currentfill}%
\pgfsetlinewidth{1.003750pt}%
\definecolor{currentstroke}{rgb}{0.000000,0.000000,0.000000}%
\pgfsetstrokecolor{currentstroke}%
\pgfsetdash{}{0pt}%
\pgfpathmoveto{\pgfqpoint{3.876477in}{2.076667in}}%
\pgfpathcurveto{\pgfqpoint{3.882002in}{2.076667in}}{\pgfqpoint{3.887302in}{2.078862in}}{\pgfqpoint{3.891209in}{2.082769in}}%
\pgfpathcurveto{\pgfqpoint{3.895115in}{2.086675in}}{\pgfqpoint{3.897311in}{2.091975in}}{\pgfqpoint{3.897311in}{2.097500in}}%
\pgfpathcurveto{\pgfqpoint{3.897311in}{2.103025in}}{\pgfqpoint{3.895115in}{2.108325in}}{\pgfqpoint{3.891209in}{2.112231in}}%
\pgfpathcurveto{\pgfqpoint{3.887302in}{2.116138in}}{\pgfqpoint{3.882002in}{2.118333in}}{\pgfqpoint{3.876477in}{2.118333in}}%
\pgfpathcurveto{\pgfqpoint{3.870952in}{2.118333in}}{\pgfqpoint{3.865653in}{2.116138in}}{\pgfqpoint{3.861746in}{2.112231in}}%
\pgfpathcurveto{\pgfqpoint{3.857839in}{2.108325in}}{\pgfqpoint{3.855644in}{2.103025in}}{\pgfqpoint{3.855644in}{2.097500in}}%
\pgfpathcurveto{\pgfqpoint{3.855644in}{2.091975in}}{\pgfqpoint{3.857839in}{2.086675in}}{\pgfqpoint{3.861746in}{2.082769in}}%
\pgfpathcurveto{\pgfqpoint{3.865653in}{2.078862in}}{\pgfqpoint{3.870952in}{2.076667in}}{\pgfqpoint{3.876477in}{2.076667in}}%
\pgfpathclose%
\pgfusepath{stroke,fill}%
\end{pgfscope}%
\begin{pgfscope}%
\pgfpathrectangle{\pgfqpoint{0.562500in}{0.275000in}}{\pgfqpoint{3.487500in}{1.925000in}}%
\pgfusepath{clip}%
\pgfsetbuttcap%
\pgfsetroundjoin%
\definecolor{currentfill}{rgb}{0.000000,0.000000,0.000000}%
\pgfsetfillcolor{currentfill}%
\pgfsetlinewidth{1.003750pt}%
\definecolor{currentstroke}{rgb}{0.000000,0.000000,0.000000}%
\pgfsetstrokecolor{currentstroke}%
\pgfsetdash{}{0pt}%
\pgfpathmoveto{\pgfqpoint{3.876477in}{2.076667in}}%
\pgfpathcurveto{\pgfqpoint{3.882002in}{2.076667in}}{\pgfqpoint{3.887302in}{2.078862in}}{\pgfqpoint{3.891209in}{2.082769in}}%
\pgfpathcurveto{\pgfqpoint{3.895115in}{2.086675in}}{\pgfqpoint{3.897311in}{2.091975in}}{\pgfqpoint{3.897311in}{2.097500in}}%
\pgfpathcurveto{\pgfqpoint{3.897311in}{2.103025in}}{\pgfqpoint{3.895115in}{2.108325in}}{\pgfqpoint{3.891209in}{2.112231in}}%
\pgfpathcurveto{\pgfqpoint{3.887302in}{2.116138in}}{\pgfqpoint{3.882002in}{2.118333in}}{\pgfqpoint{3.876477in}{2.118333in}}%
\pgfpathcurveto{\pgfqpoint{3.870952in}{2.118333in}}{\pgfqpoint{3.865653in}{2.116138in}}{\pgfqpoint{3.861746in}{2.112231in}}%
\pgfpathcurveto{\pgfqpoint{3.857839in}{2.108325in}}{\pgfqpoint{3.855644in}{2.103025in}}{\pgfqpoint{3.855644in}{2.097500in}}%
\pgfpathcurveto{\pgfqpoint{3.855644in}{2.091975in}}{\pgfqpoint{3.857839in}{2.086675in}}{\pgfqpoint{3.861746in}{2.082769in}}%
\pgfpathcurveto{\pgfqpoint{3.865653in}{2.078862in}}{\pgfqpoint{3.870952in}{2.076667in}}{\pgfqpoint{3.876477in}{2.076667in}}%
\pgfpathclose%
\pgfusepath{stroke,fill}%
\end{pgfscope}%
\begin{pgfscope}%
\pgfpathrectangle{\pgfqpoint{0.562500in}{0.275000in}}{\pgfqpoint{3.487500in}{1.925000in}}%
\pgfusepath{clip}%
\pgfsetbuttcap%
\pgfsetroundjoin%
\definecolor{currentfill}{rgb}{0.000000,0.000000,0.000000}%
\pgfsetfillcolor{currentfill}%
\pgfsetlinewidth{1.003750pt}%
\definecolor{currentstroke}{rgb}{0.000000,0.000000,0.000000}%
\pgfsetstrokecolor{currentstroke}%
\pgfsetdash{}{0pt}%
\pgfpathmoveto{\pgfqpoint{3.876477in}{2.076667in}}%
\pgfpathcurveto{\pgfqpoint{3.882002in}{2.076667in}}{\pgfqpoint{3.887302in}{2.078862in}}{\pgfqpoint{3.891209in}{2.082769in}}%
\pgfpathcurveto{\pgfqpoint{3.895115in}{2.086675in}}{\pgfqpoint{3.897311in}{2.091975in}}{\pgfqpoint{3.897311in}{2.097500in}}%
\pgfpathcurveto{\pgfqpoint{3.897311in}{2.103025in}}{\pgfqpoint{3.895115in}{2.108325in}}{\pgfqpoint{3.891209in}{2.112231in}}%
\pgfpathcurveto{\pgfqpoint{3.887302in}{2.116138in}}{\pgfqpoint{3.882002in}{2.118333in}}{\pgfqpoint{3.876477in}{2.118333in}}%
\pgfpathcurveto{\pgfqpoint{3.870952in}{2.118333in}}{\pgfqpoint{3.865653in}{2.116138in}}{\pgfqpoint{3.861746in}{2.112231in}}%
\pgfpathcurveto{\pgfqpoint{3.857839in}{2.108325in}}{\pgfqpoint{3.855644in}{2.103025in}}{\pgfqpoint{3.855644in}{2.097500in}}%
\pgfpathcurveto{\pgfqpoint{3.855644in}{2.091975in}}{\pgfqpoint{3.857839in}{2.086675in}}{\pgfqpoint{3.861746in}{2.082769in}}%
\pgfpathcurveto{\pgfqpoint{3.865653in}{2.078862in}}{\pgfqpoint{3.870952in}{2.076667in}}{\pgfqpoint{3.876477in}{2.076667in}}%
\pgfpathclose%
\pgfusepath{stroke,fill}%
\end{pgfscope}%
\begin{pgfscope}%
\pgfpathrectangle{\pgfqpoint{0.562500in}{0.275000in}}{\pgfqpoint{3.487500in}{1.925000in}}%
\pgfusepath{clip}%
\pgfsetbuttcap%
\pgfsetroundjoin%
\definecolor{currentfill}{rgb}{0.000000,0.000000,0.000000}%
\pgfsetfillcolor{currentfill}%
\pgfsetlinewidth{1.003750pt}%
\definecolor{currentstroke}{rgb}{0.000000,0.000000,0.000000}%
\pgfsetstrokecolor{currentstroke}%
\pgfsetdash{}{0pt}%
\pgfpathmoveto{\pgfqpoint{3.876477in}{1.216635in}}%
\pgfpathcurveto{\pgfqpoint{3.882002in}{1.216635in}}{\pgfqpoint{3.887302in}{1.218830in}}{\pgfqpoint{3.891209in}{1.222736in}}%
\pgfpathcurveto{\pgfqpoint{3.895115in}{1.226643in}}{\pgfqpoint{3.897311in}{1.231943in}}{\pgfqpoint{3.897311in}{1.237468in}}%
\pgfpathcurveto{\pgfqpoint{3.897311in}{1.242993in}}{\pgfqpoint{3.895115in}{1.248292in}}{\pgfqpoint{3.891209in}{1.252199in}}%
\pgfpathcurveto{\pgfqpoint{3.887302in}{1.256106in}}{\pgfqpoint{3.882002in}{1.258301in}}{\pgfqpoint{3.876477in}{1.258301in}}%
\pgfpathcurveto{\pgfqpoint{3.870952in}{1.258301in}}{\pgfqpoint{3.865653in}{1.256106in}}{\pgfqpoint{3.861746in}{1.252199in}}%
\pgfpathcurveto{\pgfqpoint{3.857839in}{1.248292in}}{\pgfqpoint{3.855644in}{1.242993in}}{\pgfqpoint{3.855644in}{1.237468in}}%
\pgfpathcurveto{\pgfqpoint{3.855644in}{1.231943in}}{\pgfqpoint{3.857839in}{1.226643in}}{\pgfqpoint{3.861746in}{1.222736in}}%
\pgfpathcurveto{\pgfqpoint{3.865653in}{1.218830in}}{\pgfqpoint{3.870952in}{1.216635in}}{\pgfqpoint{3.876477in}{1.216635in}}%
\pgfpathclose%
\pgfusepath{stroke,fill}%
\end{pgfscope}%
\begin{pgfscope}%
\pgfpathrectangle{\pgfqpoint{0.562500in}{0.275000in}}{\pgfqpoint{3.487500in}{1.925000in}}%
\pgfusepath{clip}%
\pgfsetbuttcap%
\pgfsetroundjoin%
\definecolor{currentfill}{rgb}{0.000000,0.000000,0.000000}%
\pgfsetfillcolor{currentfill}%
\pgfsetlinewidth{1.003750pt}%
\definecolor{currentstroke}{rgb}{0.000000,0.000000,0.000000}%
\pgfsetstrokecolor{currentstroke}%
\pgfsetdash{}{0pt}%
\pgfpathmoveto{\pgfqpoint{3.876477in}{1.216635in}}%
\pgfpathcurveto{\pgfqpoint{3.882002in}{1.216635in}}{\pgfqpoint{3.887302in}{1.218830in}}{\pgfqpoint{3.891209in}{1.222736in}}%
\pgfpathcurveto{\pgfqpoint{3.895115in}{1.226643in}}{\pgfqpoint{3.897311in}{1.231943in}}{\pgfqpoint{3.897311in}{1.237468in}}%
\pgfpathcurveto{\pgfqpoint{3.897311in}{1.242993in}}{\pgfqpoint{3.895115in}{1.248292in}}{\pgfqpoint{3.891209in}{1.252199in}}%
\pgfpathcurveto{\pgfqpoint{3.887302in}{1.256106in}}{\pgfqpoint{3.882002in}{1.258301in}}{\pgfqpoint{3.876477in}{1.258301in}}%
\pgfpathcurveto{\pgfqpoint{3.870952in}{1.258301in}}{\pgfqpoint{3.865653in}{1.256106in}}{\pgfqpoint{3.861746in}{1.252199in}}%
\pgfpathcurveto{\pgfqpoint{3.857839in}{1.248292in}}{\pgfqpoint{3.855644in}{1.242993in}}{\pgfqpoint{3.855644in}{1.237468in}}%
\pgfpathcurveto{\pgfqpoint{3.855644in}{1.231943in}}{\pgfqpoint{3.857839in}{1.226643in}}{\pgfqpoint{3.861746in}{1.222736in}}%
\pgfpathcurveto{\pgfqpoint{3.865653in}{1.218830in}}{\pgfqpoint{3.870952in}{1.216635in}}{\pgfqpoint{3.876477in}{1.216635in}}%
\pgfpathclose%
\pgfusepath{stroke,fill}%
\end{pgfscope}%
\begin{pgfscope}%
\pgfsetbuttcap%
\pgfsetroundjoin%
\definecolor{currentfill}{rgb}{0.000000,0.000000,0.000000}%
\pgfsetfillcolor{currentfill}%
\pgfsetlinewidth{0.803000pt}%
\definecolor{currentstroke}{rgb}{0.000000,0.000000,0.000000}%
\pgfsetstrokecolor{currentstroke}%
\pgfsetdash{}{0pt}%
\pgfsys@defobject{currentmarker}{\pgfqpoint{0.000000in}{-0.048611in}}{\pgfqpoint{0.000000in}{0.000000in}}{%
\pgfpathmoveto{\pgfqpoint{0.000000in}{0.000000in}}%
\pgfpathlineto{\pgfqpoint{0.000000in}{-0.048611in}}%
\pgfusepath{stroke,fill}%
}%
\begin{pgfscope}%
\pgfsys@transformshift{0.721249in}{0.275000in}%
\pgfsys@useobject{currentmarker}{}%
\end{pgfscope}%
\end{pgfscope}%
\begin{pgfscope}%
\definecolor{textcolor}{rgb}{0.000000,0.000000,0.000000}%
\pgfsetstrokecolor{textcolor}%
\pgfsetfillcolor{textcolor}%
\pgftext[x=0.721249in,y=0.177778in,,top]{\color{textcolor}\sffamily\fontsize{10.000000}{12.000000}\selectfont 20}%
\end{pgfscope}%
\begin{pgfscope}%
\pgfsetbuttcap%
\pgfsetroundjoin%
\definecolor{currentfill}{rgb}{0.000000,0.000000,0.000000}%
\pgfsetfillcolor{currentfill}%
\pgfsetlinewidth{0.803000pt}%
\definecolor{currentstroke}{rgb}{0.000000,0.000000,0.000000}%
\pgfsetstrokecolor{currentstroke}%
\pgfsetdash{}{0pt}%
\pgfsys@defobject{currentmarker}{\pgfqpoint{0.000000in}{-0.048611in}}{\pgfqpoint{0.000000in}{0.000000in}}{%
\pgfpathmoveto{\pgfqpoint{0.000000in}{0.000000in}}%
\pgfpathlineto{\pgfqpoint{0.000000in}{-0.048611in}}%
\pgfusepath{stroke,fill}%
}%
\begin{pgfscope}%
\pgfsys@transformshift{1.772992in}{0.275000in}%
\pgfsys@useobject{currentmarker}{}%
\end{pgfscope}%
\end{pgfscope}%
\begin{pgfscope}%
\definecolor{textcolor}{rgb}{0.000000,0.000000,0.000000}%
\pgfsetstrokecolor{textcolor}%
\pgfsetfillcolor{textcolor}%
\pgftext[x=1.772992in,y=0.177778in,,top]{\color{textcolor}\sffamily\fontsize{10.000000}{12.000000}\selectfont 40}%
\end{pgfscope}%
\begin{pgfscope}%
\pgfsetbuttcap%
\pgfsetroundjoin%
\definecolor{currentfill}{rgb}{0.000000,0.000000,0.000000}%
\pgfsetfillcolor{currentfill}%
\pgfsetlinewidth{0.803000pt}%
\definecolor{currentstroke}{rgb}{0.000000,0.000000,0.000000}%
\pgfsetstrokecolor{currentstroke}%
\pgfsetdash{}{0pt}%
\pgfsys@defobject{currentmarker}{\pgfqpoint{0.000000in}{-0.048611in}}{\pgfqpoint{0.000000in}{0.000000in}}{%
\pgfpathmoveto{\pgfqpoint{0.000000in}{0.000000in}}%
\pgfpathlineto{\pgfqpoint{0.000000in}{-0.048611in}}%
\pgfusepath{stroke,fill}%
}%
\begin{pgfscope}%
\pgfsys@transformshift{2.824734in}{0.275000in}%
\pgfsys@useobject{currentmarker}{}%
\end{pgfscope}%
\end{pgfscope}%
\begin{pgfscope}%
\definecolor{textcolor}{rgb}{0.000000,0.000000,0.000000}%
\pgfsetstrokecolor{textcolor}%
\pgfsetfillcolor{textcolor}%
\pgftext[x=2.824734in,y=0.177778in,,top]{\color{textcolor}\sffamily\fontsize{10.000000}{12.000000}\selectfont 60}%
\end{pgfscope}%
\begin{pgfscope}%
\pgfsetbuttcap%
\pgfsetroundjoin%
\definecolor{currentfill}{rgb}{0.000000,0.000000,0.000000}%
\pgfsetfillcolor{currentfill}%
\pgfsetlinewidth{0.803000pt}%
\definecolor{currentstroke}{rgb}{0.000000,0.000000,0.000000}%
\pgfsetstrokecolor{currentstroke}%
\pgfsetdash{}{0pt}%
\pgfsys@defobject{currentmarker}{\pgfqpoint{0.000000in}{-0.048611in}}{\pgfqpoint{0.000000in}{0.000000in}}{%
\pgfpathmoveto{\pgfqpoint{0.000000in}{0.000000in}}%
\pgfpathlineto{\pgfqpoint{0.000000in}{-0.048611in}}%
\pgfusepath{stroke,fill}%
}%
\begin{pgfscope}%
\pgfsys@transformshift{3.876477in}{0.275000in}%
\pgfsys@useobject{currentmarker}{}%
\end{pgfscope}%
\end{pgfscope}%
\begin{pgfscope}%
\definecolor{textcolor}{rgb}{0.000000,0.000000,0.000000}%
\pgfsetstrokecolor{textcolor}%
\pgfsetfillcolor{textcolor}%
\pgftext[x=3.876477in,y=0.177778in,,top]{\color{textcolor}\sffamily\fontsize{10.000000}{12.000000}\selectfont 80}%
\end{pgfscope}%
\begin{pgfscope}%
\definecolor{textcolor}{rgb}{0.000000,0.000000,0.000000}%
\pgfsetstrokecolor{textcolor}%
\pgfsetfillcolor{textcolor}%
\pgftext[x=2.306250in,y=-0.012191in,,top]{\color{textcolor}\sffamily\fontsize{10.000000}{12.000000}\selectfont \(\displaystyle k\)}%
\end{pgfscope}%
\begin{pgfscope}%
\pgfsetbuttcap%
\pgfsetroundjoin%
\definecolor{currentfill}{rgb}{0.000000,0.000000,0.000000}%
\pgfsetfillcolor{currentfill}%
\pgfsetlinewidth{0.803000pt}%
\definecolor{currentstroke}{rgb}{0.000000,0.000000,0.000000}%
\pgfsetstrokecolor{currentstroke}%
\pgfsetdash{}{0pt}%
\pgfsys@defobject{currentmarker}{\pgfqpoint{-0.048611in}{0.000000in}}{\pgfqpoint{0.000000in}{0.000000in}}{%
\pgfpathmoveto{\pgfqpoint{0.000000in}{0.000000in}}%
\pgfpathlineto{\pgfqpoint{-0.048611in}{0.000000in}}%
\pgfusepath{stroke,fill}%
}%
\begin{pgfscope}%
\pgfsys@transformshift{0.562500in}{0.377436in}%
\pgfsys@useobject{currentmarker}{}%
\end{pgfscope}%
\end{pgfscope}%
\begin{pgfscope}%
\definecolor{textcolor}{rgb}{0.000000,0.000000,0.000000}%
\pgfsetstrokecolor{textcolor}%
\pgfsetfillcolor{textcolor}%
\pgftext[x=0.376912in,y=0.324674in,left,base]{\color{textcolor}\sffamily\fontsize{10.000000}{12.000000}\selectfont 8}%
\end{pgfscope}%
\begin{pgfscope}%
\pgfsetbuttcap%
\pgfsetroundjoin%
\definecolor{currentfill}{rgb}{0.000000,0.000000,0.000000}%
\pgfsetfillcolor{currentfill}%
\pgfsetlinewidth{0.803000pt}%
\definecolor{currentstroke}{rgb}{0.000000,0.000000,0.000000}%
\pgfsetstrokecolor{currentstroke}%
\pgfsetdash{}{0pt}%
\pgfsys@defobject{currentmarker}{\pgfqpoint{-0.048611in}{0.000000in}}{\pgfqpoint{0.000000in}{0.000000in}}{%
\pgfpathmoveto{\pgfqpoint{0.000000in}{0.000000in}}%
\pgfpathlineto{\pgfqpoint{-0.048611in}{0.000000in}}%
\pgfusepath{stroke,fill}%
}%
\begin{pgfscope}%
\pgfsys@transformshift{0.562500in}{1.237468in}%
\pgfsys@useobject{currentmarker}{}%
\end{pgfscope}%
\end{pgfscope}%
\begin{pgfscope}%
\definecolor{textcolor}{rgb}{0.000000,0.000000,0.000000}%
\pgfsetstrokecolor{textcolor}%
\pgfsetfillcolor{textcolor}%
\pgftext[x=0.376912in,y=1.184706in,left,base]{\color{textcolor}\sffamily\fontsize{10.000000}{12.000000}\selectfont 9}%
\end{pgfscope}%
\begin{pgfscope}%
\pgfsetbuttcap%
\pgfsetroundjoin%
\definecolor{currentfill}{rgb}{0.000000,0.000000,0.000000}%
\pgfsetfillcolor{currentfill}%
\pgfsetlinewidth{0.803000pt}%
\definecolor{currentstroke}{rgb}{0.000000,0.000000,0.000000}%
\pgfsetstrokecolor{currentstroke}%
\pgfsetdash{}{0pt}%
\pgfsys@defobject{currentmarker}{\pgfqpoint{-0.048611in}{0.000000in}}{\pgfqpoint{0.000000in}{0.000000in}}{%
\pgfpathmoveto{\pgfqpoint{0.000000in}{0.000000in}}%
\pgfpathlineto{\pgfqpoint{-0.048611in}{0.000000in}}%
\pgfusepath{stroke,fill}%
}%
\begin{pgfscope}%
\pgfsys@transformshift{0.562500in}{2.097500in}%
\pgfsys@useobject{currentmarker}{}%
\end{pgfscope}%
\end{pgfscope}%
\begin{pgfscope}%
\definecolor{textcolor}{rgb}{0.000000,0.000000,0.000000}%
\pgfsetstrokecolor{textcolor}%
\pgfsetfillcolor{textcolor}%
\pgftext[x=0.288547in,y=2.044738in,left,base]{\color{textcolor}\sffamily\fontsize{10.000000}{12.000000}\selectfont 10}%
\end{pgfscope}%
\begin{pgfscope}%
\definecolor{textcolor}{rgb}{0.000000,0.000000,0.000000}%
\pgfsetstrokecolor{textcolor}%
\pgfsetfillcolor{textcolor}%
\pgftext[x=0.232992in,y=1.237500in,,bottom,rotate=90.000000]{\color{textcolor}\sffamily\fontsize{10.000000}{12.000000}\selectfont Number of GMRES Iterations}%
\end{pgfscope}%
\begin{pgfscope}%
\pgfsetrectcap%
\pgfsetmiterjoin%
\pgfsetlinewidth{0.803000pt}%
\definecolor{currentstroke}{rgb}{0.000000,0.000000,0.000000}%
\pgfsetstrokecolor{currentstroke}%
\pgfsetdash{}{0pt}%
\pgfpathmoveto{\pgfqpoint{0.562500in}{0.275000in}}%
\pgfpathlineto{\pgfqpoint{0.562500in}{2.200000in}}%
\pgfusepath{stroke}%
\end{pgfscope}%
\begin{pgfscope}%
\pgfsetrectcap%
\pgfsetmiterjoin%
\pgfsetlinewidth{0.803000pt}%
\definecolor{currentstroke}{rgb}{0.000000,0.000000,0.000000}%
\pgfsetstrokecolor{currentstroke}%
\pgfsetdash{}{0pt}%
\pgfpathmoveto{\pgfqpoint{4.050000in}{0.275000in}}%
\pgfpathlineto{\pgfqpoint{4.050000in}{2.200000in}}%
\pgfusepath{stroke}%
\end{pgfscope}%
\begin{pgfscope}%
\pgfsetrectcap%
\pgfsetmiterjoin%
\pgfsetlinewidth{0.803000pt}%
\definecolor{currentstroke}{rgb}{0.000000,0.000000,0.000000}%
\pgfsetstrokecolor{currentstroke}%
\pgfsetdash{}{0pt}%
\pgfpathmoveto{\pgfqpoint{0.562500in}{0.275000in}}%
\pgfpathlineto{\pgfqpoint{4.050000in}{0.275000in}}%
\pgfusepath{stroke}%
\end{pgfscope}%
\begin{pgfscope}%
\pgfsetrectcap%
\pgfsetmiterjoin%
\pgfsetlinewidth{0.803000pt}%
\definecolor{currentstroke}{rgb}{0.000000,0.000000,0.000000}%
\pgfsetstrokecolor{currentstroke}%
\pgfsetdash{}{0pt}%
\pgfpathmoveto{\pgfqpoint{0.562500in}{2.200000in}}%
\pgfpathlineto{\pgfqpoint{4.050000in}{2.200000in}}%
\pgfusepath{stroke}%
\end{pgfscope}%
\end{pgfpicture}%
\makeatother%
\endgroup%

  \caption{GMRES iteration counts for $\alpha = 0.5/k^{1/2}$}\label{fig:linfinityn1}
\end{subfigure}

    \begin{subfigure}{\textwidth}
      \centering
\input{nbpc-linfinity-n-2.pgf}
  \caption{GMRES iteration counts for $\alpha = 0.5/k$}\label{fig:linfinityn2}
\end{subfigure}
\caption{GMRES iteration counts for $\AmatoI\Amatt$ where $\Aso=\Ast=1$ and $\NLiDRRR{\nso-\nst} = \alpha$ as described in \cref{sec:num}.}
\end{figure}
\optodo{Need to figure out how to get computer modern consistently in axis labels. Some bookmarks saved that may help.}
  


%Say that these are for the TEDP defined in \cref{def:TEDP}.

\section{Definitions and conditions}\label{sec:3}

We now state the necessary technical definitions to prove \cref{cor:1,cor:1a} above.

\subsection{The variational problem and the Galerkin method}\label{sec:vpGm}
As this \lcnamecref{chap:nbpc} concerns finite-element discretisations of the Helmholtz equation, we will work with the variational formulation of \cref{prob:edp}, \cref{prob:vedp} above.

\bre[The EDP with data in $(\HokDR)'$]
In \cref{prob:edp} we defined the EDP with the antilinear functional $\LE$ arising from a function $f\in \LtDR$. In the rest of the \lcnamecref{chap:nbpc}, 
%\item In the rest of the paper, we usually consider the EDP with data given by $F$ defined in \cref{eq:EDPvar}, but sometimes 
we sometimes consider the EDP with general $\LE\in \HozDDRp$ and we indicate when this is the case.
In this latter situation, we define the dual norm by
\beq\label{eq:dualnorm}
\NHokDRp{L}= \sup_{v\in \HozDDR} \frac{\abs{\LE(v)}}{\NHokDR{v}}.
\eeq
%where $\|\cdot\|_{\HokDR}$ is defined by \cref{eq:1knorm}.
\ere

For the remainder of this \lcnamecref{chap:nbpc}, we let $(\Vhp)_{h>0}$ be a family of finite-dimensional, nested subspaces of $\HozDDR$, whose union is dense in $\HozDDR$. More specifically, we let $\Vhp$ consist of piecewise-polynomials on a simplicial mesh $\cTh$ with mesh-size $h$
%\ednote{Euan says: have problem that want to allow $C^{1,1}$ $\Dm$, so that statements later about $H^2$ regularity are covered, but easiest to define triangulation and hence subspaces on Lipschitz domains -- Euan to discuss with Ivan}
and fixed polynomial degree $p$. (Note that the dimension $N$ of $\Vhp$ then satisfies $N\sim h^{-d}$.) As in \cref{rem:crimes} above, we ignore any variational crimes resulting from this discretisation.

We now define the analogue of the finite-element approximation \cref{prob:fevtedp} for \cref{prob:vedp} with general data $L \in \HokDRp.$
\bprob[Finite-element approximation of \cref{prob:vedp} with general data]\label{prob:fevedpgen}
We say that $\uh \in \Vhp$ is the \defn{finite-element approximation of $u$} (the solution to \cref{prob:vedp} with general right-hand side $L \in \HokDRp$) if
\beq\label{eq:galerkin}
\aE(\uh,\vh) = \LE(\vh) \tforall \vh \in \Vhp.
\eeq
\eprob
Observe that implicit in our use of $\aE$ in \cref{eq:galerkin} is the fact that we are realising the Dirichlet-to-Neumann map $\TR$ exactly on $\GR.$
%Definition of Galerkin method

We now define the matrices associated with our finite-element discretisation. Let $\{\phi_i, i= 1, \ldots, N\}$ be a basis for $\Vhp$ with each $\phi_i$ \emph{real-valued}.
Let 
\beq\label{eq:matrixSjdef}
\big(\Smat_{A}\big)_{ij}:= \int_\Omega \big(A \nabla \phi_j)\cdot\nabla \phi_i, \quad
\big(\Mmat_{n}\big)_{ij}:= \int_\Omega n\,\phi_i\, \phi_j,
\quad\tand\quad
\big(\Nmat\big)_{ij}:= \int_{\GR} T_R (\gamma\phi_j) \,\gamma \phi_i
\eeq
be the stiffness, domain-mass, and boundary-mass matrices, respectively. Note that both $\Smat_A$ and $\Mmat_n$ are \emph{real-valued}, but $\Nmat$ is \emph{complex-valued} (because the DtN operator $T_R$ is complex-valued).
Let
\beq\label{eq:matrixAdef}
\Amat := \Smat_{A} - k^2 \Mmat_{n} - \Nmat,
\eeq
and let $u_h:= \sum_j u_j \phi_j$. Then \cref{eq:galerkin} implies that
\beqs
\Amat \bu = \bff,
\eeqs
where $(\bff)_i := \FE(\phi_i)$.
Similar to above we let 
\beq\label{eq:matrixAjdef}
\Amatj := \Smat_{A^{(j)}} - k^2 \Mmat_{n^{(j)}} - \Nmat.
\eeq
Finally, let 
\beq\label{eq:Dk2}
\Dmat_k:= \Smat_I + k^2 \Mmat_1;
\eeq
so that \cref{eq:Dk} holds.
%then the weighted norm $\|\cdot\|_{\Dmat_k}$ is given by 
%\beq\label{eq:Dk3}
%\N{\bv}_{\Dmat_k}^2:=   \N{v_h}^2_{\HokDR}=\big( \Dmat_k \bv,\bv\big)_2,
%\eeq
%for
%$v_h =\sum_i v_i \phi_i$.

We recall the definition of quasi-uniformity (c.f. \cite[Definition 4.4.13]{BrSc:08}):

\bde[Quasi-uniform]\label{def:quasiuniform}
Let $\set{\cTh}_{h>0}$ be a set of triangulations of $\DR$ indexed by their mesh size $h.$ For each $T \in \cTh,$ let $\BT$ denote the largest ball contained in $T$.
If there exists $\rho > 0$ such that 
\beqs
\min\set{\diam \BT \st T \in \cT} \geq \rho h,
\eeqs
then $\set{\cTh}_{h>0}$ is said to be \defn{quasi-uniform}.
\ede

\ble[Norm equivalences of FE functions]\label{lem:normequiv}
If $\set{{\cTh}_{h>0}}$ is quasi-uniform with a nodal basis, then
there exist $m_\pm$ and $s_\pm$, independent of $h$ and $p$, such that
\beq\label{eq:normequiv1}
m_- h^{d/2} \N{\bv}_2 \leq \N{v_h}_{\LtDR} \leq m_+ h^{d/2} \N{\bv}_2,
\eeq
and
\beq\label{eq:normequiv2}
s_- h^{d/2} \N{\bv}_2 \leq \N{\nabla v_h}_{\LtDR} \leq s_+ h^{d/2-1} \N{\bv}_2,
\eeq
for all finite-element functions $v_h =\sum_i v_i \phi_i \in \Vhp$.
\ele

Written in terms of the matrices $\Mmat_1$ and $\Smat_I$ defined in \cref{eq:matrixSjdef}, the bounds \cref{eq:normequiv1} and \cref{eq:normequiv2} are, respectively, the familiar bounds
\beqs
(\Mmat_1 \bv,\bv)_2 \sim h^d \N{\bv}^2_2 \quad\tand\quad h^{d}\N{\bv}^2_2 \lesssim (\Smat_I \bv,\bv)_2 \lesssim h^{d-2} \N{\bv}^2_2.
\eeqs

For a proof of \cref{lem:normequiv}, see\footnote{In \cite[Chapter V, Lemma 2.5] the assumption is made that the meshes underlying $\Vhp$ are \emph{uniform}. However, the definition of uniformity in \cite[Chapter 2, Definition 5.1(4)]{Br:07} is the same as the more standard definition of quasi-uniformity in \cref{def:quasiuniform}.} \cite[Chapter V, Lemma 2.5]{Br:07}.\ednote{Euan---in \cite[Chapter V, Lemma 2.5]{Br:07} this result is written slightly differently (with scaled basis functions and a scaled 2-norm) but it's equivalent to our result. I presume I don't need to put anything 'translating' their result to ours? (Can download the 3rd edition from the CUP website if you use your Bath institutional login.)}

%% \bpf[Sketch proof of \cref{lem:normequiv}]\opntodo{can omit this if can find a good reference. One possibility . Need to check basis scaling business.}
%% The inequalities in \cref{eq:normequiv1} follow from writing $\|v_h\|_{\LtDR}$ as a sum of integrals over elements of the mesh, and then mapping to the reference element \ednote{Euan to discuss with Ivan}.
%% %\beqs
%% %\N{v_h}^2_{L^2(\O
%% %\eeqs
%% Then, \cref{eq:normequiv2} follows from \cref{eq:normequiv1} and the inequalities
%% \beqs
%% \N{v_h}_{L^2(\DR)}\lesssim \N{\nabla v_h}_{L^2(\DR)}\lesssim \frac{1}{h} \N{v_h}_{L^2(\DR)},
%% \eeqs
%% the first of which follows from the Poincar\'e inequality, since $v_h \in \HozDDR$
%% (see, e.g., \cite[Proposition 5.3.4]{BrSc:00}), the second of which follows from a standard inverse estimate (see, e.g., \cite[Theorem 4.5.11]{BrSc:00}).
%% \epf


Finally, we need the concept of the \emph{adjoint} sesquilinear form to $a(\cdot,\cdot)$.
\begin{definition}[The adjoint sesquilinear form $a^*(\cdot,\cdot)$]\label{def:adjoint}
Let $\Dm$, $n$, and $A$, be as in the definition of the EDP (\cref{prob:edp}). The adjoint sesquilinear form, $a^*(\cdot,\cdot)$, to $a(\cdot,\cdot)$ defined in \cref{eq:aedp} is given by
\beq\label{eq:EDPadjoint}
a^*(\vo,\vt) := \int_{\DR} 
\Big((A \grad \vo)\cdot\grad \vtb
 - k^2 n \vo\vtb\Big) - \big\langle \gamma \vo,T_R(\gamma \vt)\big\rangle_{\GR}.
\eeq
\end{definition}

\noi It is then straightforward to check that
\beq\label{eq:A*}
\Amat^* := \Smat_A -k^2 \Mmat_n - \Nmat^*
\eeq
(where $^*$ denotes conjugate transpose) is the Galerkin matrix for the sesquilinear form $a^*(\cdot,\cdot)$; i.e.~$(\Amat^*)_{ij} = a^*(\phi_j, \phi_i)$.

\ble[Link between variational problems involving $a(\cdot,\cdot)$ and $a^*(\cdot,\cdot)$]\label{lem:adjoint}
Given $F\in (\HozDDR)'$, if $u$ is the solution to the variational problem
\beq\label{eq:adjoint1}
a^*(u,v)= F(v) \quad\tfa v\in \HozDDR,
\eeq
then $\overline{u}$ satisfies
\beq\label{eq:adjoint2}
a(\overline{u},v)= \overline{F(\overline{v})} \quad\tfa v\in \HozDDR.
\eeq
\ele\optodo{Consider putting something like this in Chap 2.}

For the proof of \cref{lem:adjoint} we need the following property of the DtN map $T_R$:
\beq\label{eq:DtN}
\big\langle T_R\psi, \overline{\phi} \big\rangle_\Gamma = \big\langle T_R \phi, \overline{\psi}\big\rangle_\Gamma \quad\tfa \phi,\psi \in H^{1/2}(\GR).
\eeq
This property follows from the fact that, if $\uo$ and $\ut$ are solutions of the homogeneous Helmholtz equation $\Delta u +k^2 u=0$ in $\RRd\setminus \overline{\BR}$, both satisfying the Sommerfeld radiation condition \cref{eq:src}, then
\beqs
\int_{\GR} (\gamma \uo)\, \dn \ut = \int_{\GR} (\gamma \ut)\,\dn \uo;
\eeqs
see, e.g., \cite[Lemma 6.13]{Sp:15}.

\bpf[Proof of \cref{lem:adjoint}]
From \cref{eq:adjoint1} we have that 
\beqs
\overline{a^*(u,\overline{v})}= \overline{F(\overline{v})} \quad\tfa v\in \HozDDR.
\eeqs
Using the definition of $a^*(\cdot,\cdot)$ and the property \cref{eq:DtN} in the left-hand side of this last equation, we find \cref{eq:adjoint2}.
\epf



\section{Proofs of the main results}\label{sec:proofs}

The main part of the proofs of \cref{cor:1,cor:1a} is the following \lcnamecref{thm:1}.

\begin{theorem}[Main ingredient of the answer to \cref{it:nbpcq1}]\label{thm:1}
Assume that $\Dm$, $\Aso$, and $\nso$ satisfy \cref{cond:1nbpc}, and assume that $h$ and $p$ satisfy \cref{cond:2}. 
Let the $k$- and $h$-independent constants $\mpm$ and $\spm$ be given as in \cref{lem:normequiv}.
Then, given $\kz>0$, there exist $\Co, \Ct>0$, independent of $h$ and $k$ (but dependent on $\Dm, \Aso, \nso$, $p$, and $\kz$) such that
\begin{align}\nonumber
&\max\Big\{
\NDmatk{\Imat - (\Amat^{(1)})^{-1}\Amat^{(2)}}, 
\N{\Imat -\Amat^{(2)} (\Amat^{(1)})^{-1}}_{(\Dmat_k)^{-1}}
\Big\}\\
&\hspace{3cm} 
\leq C_1 \,k \,
\NLiDRRRdtd{\Aso-\Ast} + C_2 \, k \, \NLiDRRR{\nso-\nst}
\label{eq:main1}
\end{align}
and 
\begin{align}\nonumber
&\max\Big\{
\N{\Imat - (\Amat^{(1)})^{-1}\Amat^{(2)}}_2, 
\N{\Imat -\Amat^{(2)} (\Amat^{(1)})^{-1}}_2
\Big\}\\
&\hspace{0cm} 
\leq C_1 \,\left(\frac{s_+}{m_-}\right) \,\frac{1}{h} \,
\NLiDRRRdtd{\Aso-\Ast} + C_2 \, \left(\frac{m_+}{m_-} \right)k \, \NLiDRRR{\nso-\nst}
\label{eq:main1a}
\end{align}
for all $k\geq k_0$. 
\end{theorem}



\subsection{Proof of \cref{thm:1}} 

The following two lemmas are the heart of the proof of \cref{thm:1}.

\ble[Bounds on $(\Amato)^{-1} \Mmat_{n}$]\label{lem:keylemma1}
Assume that \cref{cond:1nbpc} holds, and assume that Part (i) of \cref{cond:2} holds. Then, for $n\in \LiDRRR$,
\beq\label{eq:keybound1}
\max\Big\{\big\| (\Amato)^{-1} \Mmat_{n} \big\|_{\Dmat_k}, \,\,
\big\|  \Mmat_{n}(\Amato)^{-1} \big\|_{(\Dmat_k)^{-1}}
\Big\}\leq 
C_2
%\frac{m_+}{m_-} \left[ C_{\rm FEM1}^{(1)} + C_{\rm bound}^{(1)}\right] 
\frac{\N{n}_{L^\infty(\DR)}}{k}
\eeq
and 
\beq\label{eq:keybound1a}
\max\Big\{\big\| (\Amato)^{-1} \Mmat_{n} \big\|_2, \,\,
\big\|  \Mmat_{n}(\Amato)^{-1} \big\|_2 
\Big\}\leq 
C_2 
%\frac{m_+}{m_-} \left[ C_{\rm FEM1}^{(1)} + C_{\rm bound}^{(1)}\right] 
\left(\frac{m_+}{m_-}\right) \frac{\N{n}_{L^\infty(\DR)}}{k}
\eeq
for all $k\geq k_0$,
where
\beq\label{eq:C2}
C_2:=%\frac{m_+}{m_-} 
%\left[ 
C_{\rm FEM1}^{(1)} + C_{\rm bound}^{(1)}.%\right].
\eeq
\ele

\ble[Bounds on $(\Amato)^{-1} \Smat_A$]\label{lem:keylemma2}
Assume that \cref{cond:1nbpc} holds, and assume that Part (ii) of \cref{cond:2} holds. Then, for $A\in L^\infty(\DR,\RR^{d\times d})$,
\beq\label{eq:keybound2}
\max\Big\{\big\| (\Amato)^{-1} \Smat_A \big\|_{(\Dmat_k)^{-1}}, \,\,
\big\| \Smat_A (\Amato)^{-1} \big\|_{\Dmat_k}\Big\} \leq C_1\, k\N{A}_{L^\infty(\DR)}
\eeq
and
\beq\label{eq:keybound2a}
\max\Big\{\big\| (\Amato)^{-1} \Smat_A \big\|_2, \,\,
\big\| \Smat_A (\Amato)^{-1} \big\|_2\Big\} \leq C_1\,\left(\frac{s_+}{m_-}\right) \frac{1}{h}\N{A}_{L^\infty(\DR)}
\eeq
%\begin{align}\nonumber
%&\max\Big\{\big\| (\Amato)^{-1} \Smat_A \big\|_2, \,\,
%\big\| \Smat_A (\Amato)^{-1} \big\|_2\Big\}\nonumber \\
%&\hspace{2cm}
% \leq \frac{s_+}{s_-} \left[ C_{\rm FEM2}^{(1)} + 
% \frac{1}{\min\big\{\Asomin,\nsomin\big\}}\left( \frac{1}{k_0} + 2 C^{(1)}_{\rm bound}\nsomax  \right) \right]k\N{A}_{L^\infty(\DR)}\label{eq:keybound2}
%% + C_{\rm bound}^{(1)}\right) \frac{\N{n}_{L^\infty(\DR)}}{k}.
%\end{align}
for all $k\geq k_0$, where
\beq\label{eq:C1nbpc}
C_1:=%\frac{s_+}{s_-} 
\left[ C_{\rm FEM2}^{(1)} + 
 \frac{1}{\min\big\{\Asomin,\nsomin\big\}}\left( \frac{1}{k_0} + 2 C^{(1)}_{\rm bound}\nsomax  \right) \right]
\eeq
\ele

\bpf[Proof of \cref{thm:1} using \cref{lem:keylemma1,lem:keylemma2}]
Using the definition of the matrices $\Amatj, \SmatA$, and $\Mmatn$ in \cref{eq:matrixAjdef} and \cref{eq:matrixSjdef}, we have
\begin{align}\nonumber
\Imat - (\Amato)^{-1}\Amatt = (\Amato)^{-1}\big(\Amato-\Amatt\big) &=  (\Amato)^{-1}\left( \Smat_{A^{(1)}} - \Smat_{A^{(2)}} - k^2 \big(\Mmat_{n^{(1)}}-\Mmat_{n^{(2)}}\big)\right)\\
&= (\Amato)^{-1}\left( \Smat_{A^{(1)}-A^{(2)}} - k^2 \Mmat_{n^{(1)}-n^{(2)}}\right),\label{eq:idea1}
\end{align}
and similarly 
\beq\label{eq:idea2}
\Imat -\Amatt  (\Amato)^{-1}= \left( \Smat_{A^{(1)}-A^{(2)}} - k^2 \Mmat_{n^{(1)}-n^{(2)}}\right)(\Amato)^{-1}.
\eeq
The bounds  \cref{eq:main1} on $\|\Imat - (\Amato)^{-1}\Amatt\|_2$ and  $\|\Imat - \Amatt(\Amato)^{-1}\|_2$ then follow from using the bounds \cref{eq:keybound1} and \cref{eq:keybound2} in \cref{eq:idea1} and \cref{eq:idea2}.
%
%, and $C_1$, $C_2$ in \cref{eq:main1} are given explicitly by
%\beq\label{eq:C1nbpc}
%C_1:=%\frac{s_+}{s_-} 
%\left[ C_{\rm FEM2}^{(1)} + 
% \frac{1}{\min\big\{\Asomin,\nsomin\big\}}\left( \frac{1}{k_0} + 2 C^{(1)}_{\rm bound}\nsomax  \right) \right] \,\,\tand\,\,
%\quad C_2:=  %+ \frac{m_+}{m_-} 
% \left[ C_{\rm FEM1}^{(1)} + C_{\rm bound}^{(1)}\right].
%\eeq
\epf

\

\bpf[Proof of \cref{lem:keylemma1}]
We first concentrate on proving \cref{eq:keybound1}.
Given $\bff \in \CC^N$ and $n\in \LiDRRR$, we create a variational problem whose Galerkin discretisation leads to the equation $\Amato \tbu = \Mmat_n\,\bff$.
Indeed, let $\widetilde{f} := \sum_j f_j \phi_j\in \HozDDR$. Define $\widetilde{u}$ to be the solution of the variational problem 
\beq\label{eq:411}
a^{(1)}(\widetilde{u},v)= (n\widetilde{f},v)_{L^2(\Omega)} \quad\text{ for all } v\in H^1(\Omega),
\eeq
and let $\tu_h$ be the solution of the finite-element approximation of \cref{eq:411}, i.e.,
\beq\label{eq:41}
a^{(1)}(\tu_h,v_h)= (n\widetilde{f},v_h)_{L^2(\Omega)} \quad\text{ for all } v_h\in \Vhp,
\eeq
and let $\tbu$ be the vector of nodal values of $\tu_h$. The definition of $\widetilde{f}$ then implies that \cref{eq:41} is equivalent to the linear system $\Amato \tbu = \Mmat_{n}\,\bff$, and so to obtain a bound on $\|(\Amato)^{-1}\Mmat_n\|_{\Dmat_k}$ we need to bound $\|\tbu\|_{\Dmat_k}$ in terms of $\|\bff\|_{\Dmat_k}$. (Recall $\bff \in \CCN$ was arbitrary.) Because of the definition 
of $\|\cdot\|_{\Dmat_k}$ in \cref{eq:Dk}, this is bound equivalent to bounding $\|\tu_h\|_{\HokDR}$ in terms of $\|\widetilde{f}\|_{\HokDR}$.

%First observe that the bound \cref{eq:bound3} from Part (i) of \cref{cond:2} holds for the solution of the variational problem
%\beqs%\label{eq:411}
%a^{(1)}(u,v)= (n\phi_j,v)_{L^2(\Omega)} \quad\text{ for all } v\in H^1(\Omega),
%\eeqs
%and hence, by linearity, it also holds for the solution $\widetilde{u}$ of the variational problem \cref{eq:411}.

Using %the bounds in \cref{eq:normequiv1}, 
the triangle inequality and the bounds \cref{eq:bound3} and \cref{eq:bound1} from \cref{cond:2,cond:1nbpc} respectively, we find
%Note that the hypotheses imply that the bound on the solution operator 
%\cref{eq:bound_unif} holds (by \cref{cor:uniform}), and also that if $h k\sqrt{|k^2-\eps|} \leq C_1$ then quasi-optimality \cref{eq:qoeps_lemma} holds (by \cref{lem:qo}).
%Starting with \cref{eq:equiv} we then have 
\begin{align}
%m_- h^{d/2}k \N{\tbu}_2 \leq k\N{\tu_h}_{\LtDR}\leq  
\N{\tu_h}_{\HokDR} \leq
\N{\tu-\tu_h}_{\HokDR} + \N{\tu}_{\HokDR} \label{eq:mainevent1}
& \leq C^{(1)}_{\rm FEM1}\NLtDR{n\ftilde} + C^{(1)}_{\rm bound}\NLtDR{n\ftilde} \\ 
& \leq \mleft(C^{(1)}_{\rm FEM1} + C^{(1)}_{\rm bound}\mright)\NLiDRRR{n}\NLtDR{\ftilde} \label{eq:mainevent1a} \\
& \leq\big(C^{(1)}_{\rm FEM1}+  C^{(1)}_{\rm bound}\big)\NLiDRRR{n}\frac{\big\|\widetilde{f}\big\|_{\HokDR}}{k};\nonumber
%& \leq\big(C^{(1)}_{\rm FEM1}+  C^{(1)}_{\rm bound}\big)\N{n}_{L^\infty(\DR)} m_+ h^{d/2} \N{\bff}_2,
\end{align}
the bound on $\|(\Amato)^{-1}\Mmat_n\|_{\Dmat_k}$ in \cref{eq:keybound1} then follows from the definition of $\|\cdot\|_{\Dmat_k}$ in \cref{eq:Dk} and the definition of $C_2$ \cref{eq:C2}.

To prove the bound on $\|\Mmat_n(\Amato)^{-1}\|_{(\Dmat_k)^{-1}}$ in \cref{eq:keybound1}, first observe that the definitions of $\|\cdot\|_{\Dmat_k}$ and $\|\cdot\|_{(\Dmat_k)^{-1}}$ in \cref{eq:Dk} imply that, for any matrix $\Cmat \in \CCNtN$ and for any $\bv\in \CC^N$,
\beq\label{eq:A380-0}
\frac{
\big\|\matrixC \bv \big\|_{(\Dmat_k)^{-1}}
}{
\big\|\bv\|_{(\Dmat_k)^{-1}}
} = 
\frac{
\big\|\matrixC^* \bw \big\|_{\Dmat_k}
}{
\big\|\bw\|_{\Dmat_k}
}
\eeq
where $\bw := (\Dmat_k)^{1/2}\bv$, and where $\matrixC^*$ is the conjugate transpose of $\matrixC$ (i.e.~the adjoint with respect to $(\cdot,\cdot)_2$).
Therefore, since $\Mmat_n$ is a real, symmetric matrix,
\beqs
\frac{
\big\|\Mmat_n (\Amato)^{-1}\bv\big\|_{(\Dmat_k)^{-1}}
}{
\N{\bv}_{(\Dmat_k)^{-1}}
}
=
\frac{\NDk{\mleft(\AmatoI\Mmatn\mright)^* \bw}}{\NDk{\bw}}
= 
\frac{
\big\|((\Amato)^*)^{-1}\Mmat_n\bw\big\|_{\Dmat_k}
}{
\N{\bw}_{\Dmat_k}
},
 \eeqs
 so that 
\beq\label{eq:A380} 
 \big\|\Mmat_n (\Amato)^{-1}\big\|_{(\Dmat_k)^{-1}}=\big\|((\Amato)^*)^{-1}\Mmat_n\big\|_{\Dmat_k}.
 \eeq 
Recall from the text below \cref{eq:A*} that $(\Amato)^*$ is the Galerkin matrix corresponding to the variational problem \cref{eq:adjoint1} -- the adjoint problem. \cref{lem:adjoint} implies that if the EDP %with coefficients $A^{(1)}$ and $n^{(1)}$ 
satisfies \cref{cond:1nbpc,cond:2}, then so does the adjoint problem. Therefore, the argument above leading to the bound on $\|(\Amato)^{-1}\Mmat_n\|_{\Dmat_k}$ under \cref{cond:1nbpc} and Part (i) of \cref{cond:2} proves the same bound on $\|((\Amato)^*)^{-1}\Mmat_n\|_{\Dmat_k}$, and then, using \cref{eq:A380}, also on $\big\|\Mmat_n(\Amato)^{-1}\big\|_{(\Dmat_k)^{-1}}$.

To prove the bound on  $\|(\Amato)^{-1}\Mmat_n\|_{2}$ in \cref{eq:keybound1a}, we use the bounds 
\beqs
m_- h^{d/2} k \N{\tbu}_2 \leq k \N{\widetilde{u}_h}_{\LtDR} \leq \N{\widetilde{u}_h}_{\HokDR}
\,\tand\,
\big\|\widetilde{f}\big\|_{\LtDR} \leq m_+ h^{d/2}\N{\bff}_2,
\eeqs
on either side of the inequality \cref{eq:mainevent1}, with these bounds coming from \cref{eq:normequiv1}. The proof of the bound on 
$\|\Mmat_n((\Amato)^*)^{-1}\|_{2}$ in \cref{eq:keybound1a} follows in a similar way to above, using the fact that 
$\|\Mmat_n (\Amato)^{-1}\|_2=\|((\Amato)^*)^{-1}\Mmat_n\|_2$ (compare to \cref{eq:A380}).
%, namely the variational problem \cref{eq:EDPvar} with the operator $T_R$ in $a^{(1)}(\cdot,\cdot)$ replaced by $\overline{T_R}$ (corresponding to the $-\ri k$ in the radiation condition \cref{eq:src} being changed to $+\ri k$).
%
%Now, if $u$ is the solution of the adjoint problem with data $F(v)$, then $\overline{u}$ is the solution of the original problem with data $\overline{F(\overline{v})}$; 
%
%in particular if $F(v)$ is as in \cref{eq:EDPa}, then the $L^2$ data of the adjoint problem is just $\overline{f}$. Therefore, if the EDP satisfies \cref{cond:1nbpc,cond:2}, then so does its adjoint, and
% the bound in \cref{eq:keybound1} on $\|(\Amato)^{-1}\Mmat_n\|_{2}$ also holds for $\|((\Amato)^*)^{-1}\Mmat_n\|_{2}$.
\epf

The proof of \cref{lem:keylemma2} uses the following \lcnamecref{lem:H1}, which one can prove sing the G\aa rding inequality \cref{eq:gardingbrief}; see \cite[Lemma 5.1]{GrPeSp:19}.

\ble[Bound for data in $\HozDDRs$]\label{lem:H1}
%With the sesquilinear form $a(\cdot,\cdot)$ defined by \cref{eq:EDPa} with $A=\Aso$ and $n=\nso$, 
Given $\widetilde{F}\in \HozDDRs$, let $\widetilde{u}$ be the solution of the variational problem
\beqs
\text{ find } \,\,\widetilde{u} \in H^1_{0,D}(\DR) \,\,\tst \,\,
a^{(1)}(\widetilde{u},v)=\widetilde{F}(v) \,\, \tfa v\in H^1_{0,D}(\DR).
\eeqs
If \cref{cond:1nbpc} holds, then $\widetilde{u}$ exists, is unique, and satisfies the bound
\beq\label{eq:bound2}
\N{\widetilde{u}}_{\HokDR} \leq \frac{1}{\min\{\Asomin,\nsomin\}}\left( 1 + 2 C^{(1)}_{\rm bound}\nsomax  k\right) \big\|\widetilde{F}\big\|_{(\HokDR)'}
\eeq
for all $k\geq k_0$.
\ele
Observe that, similar to \cref{rem:yesitis}, \cref{eq:bound2} is a $k$-independent bound, due to the norm $\NHokDRs{F}$ on the right-hand side.


\bpf[Proof of \cref{lem:keylemma2}]
In a similar way to the proof of \cref{lem:keylemma1}, given $\bff \in \CC^N$ and a symmetric $A\in L^\infty(\DR, \RR^{d\times d})$, let $\widetilde{f} := \sum_j f_j \phi_j$ and observe that $\widetilde{f} \in \HozDDR$. Define $\widetilde{u}$ to be the solution of the variational problem 
\beq\label{eq:411a}
a^{(1)}(\widetilde{u},v)= \widetilde{F}(v) \quad\text{ for all } v\in H^1(\Omega),
\quad\text{ where } \quad
 \widetilde{F}(v) :=(A\nabla\widetilde{f},\nabla v)_{L^2(\Omega)}.
\eeq
Observe that the definition of the norms $\|\cdot\|_{(\HokDR)'}$ \cref{eq:dualnorm} and $\|\cdot\|_{\HokDR}$ \cref{eq:weightednorm} and the Cauchy-Schwarz inequality imply that
\begin{align}
\big\| \widetilde{F}\big\|_{(\HokDR)'}&\leq \big\|A\nabla \widetilde{f}\big\|_{\LtDR}\nonumber\\
&\leq \NLiDRRRdtd{A} \big\|\nabla \widetilde{f}\big\|_{\LtDR}\label{eq:Fbounda}\\
&\leq \NLiDRRRdtd{A} \big\| \widetilde{f}\big\|_{\HokDR}.\label{eq:Fbound}
\end{align}
Let $\tu_h$ be the solution of the finite element approximation of \cref{eq:411a}, i.e.,
\beq\label{eq:41a}
a^{(1)}(\tu_h,v_h)= \widetilde{F}(v_h) \quad\text{ for all } v_h\in \Vhp,
\eeq
and let $\tbu$ be the vector of nodal values of $\tu_h$. The definition of $\widetilde{f}$ then implies that \cref{eq:41a} is equivalent to $\Amato \tbu = \Smat_A\,\bff$. 

Similar to the proof of \cref{lem:keylemma1},
using the triangle inequality, the bound \cref{eq:bound4} from \cref{cond:2}, the bound \cref{eq:bound2} from \cref{lem:H1}, the bound \cref{eq:Fbound}, and the definition of $C_1$ \cref{eq:C1nbpc},
we find
%Note that the hypotheses imply that the bound on the solution operator 
%\cref{eq:bound_unif} holds (by \cref{cor:uniform}), and also that if $h k\sqrt{|k^2-\eps|} \leq C_1$ then quasi-optimality \cref{eq:qoeps_lemma} holds (by \cref{lem:qo}).
%Starting with \cref{eq:equiv} we then have 
\begin{align}\nonumber 
%s_- h^{(d-2)/2} \N{\tbu}_2 &\leq \N{\nabla \tu_h}_{\LtDR}\leq  
\N{\tu_h}_{\HokDR} &\leq
\N{\tu-\tu_h}_{\HokDR} + \N{\tu}_{\HokDR},\nonumber \\ \nonumber
& \leq \left[ C^{(1)}_{\rm FEM2} k + 
\frac{1}{\min\{\Asomin,\nsomin\}}\left( 1 + 2 C^{(1)}_{\rm bound}\nsomax k  \right) 
\right]\big\|\widetilde{F}\big\|_{(\HokDR)'},\\
&\leq C_1 \, k\, 
%\left[C^{(1)}_{\rm FEM2} k + \frac{1}{\min\{\Asomin,\nsomin\}}\left( 1 + 2 C^{(1)}_{\rm bound}\nsomaxk  \right) \right]
\NLiDRRRdtd{A} \big\|\nabla\widetilde{f}\big\|_{\LtDR},\label{eq:mainevent2}\\
&\leq C_1 \, k\, 
%\left[C^{(1)}_{\rm FEM2} k + \frac{1}{\min\{\Asomin,\nsomin\}}\left( 1 + 2 C^{(1)}_{\rm bound}\nsomaxk  \right) \right]
\NLiDRRRdtd{A} \big\|\widetilde{f}\big\|_{\HokDR},\nonumber
%&\leq \left[ C^{(1)}_{\rm FEM2} k + 
%\frac{1}{\min\{\Asomin,\nsomin\}}\left( 1 + 2 C^{(1)}_{\rm bound}\nsomaxk  \right) 
%\right]\big\|A\big\|_{L^\infty(\DR)}s_+ h^{(d-2)/2} \N{\bff}_2,
\end{align}
and the bound on $\|(\Amato)^{-1}\Smat_A\|_{\Dmat_k}$ in \cref{eq:keybound2} follows.

The bound on $\|\Smat_A(\Amato)^{-1}\|_{(\Dmat_k)^{-1}}$ follows in a similar way to how we obtained the 
bound on  $\|(\Amato)^{-1}\Mmat_n\|_{(\Dmat_k)^{-1}}$ from the bound on $\|\Mmat_n(\Amato)^{-1}\|_{\Dmat_k}$ in Part (i). Indeed, 
\cref{eq:A380-0} and the fact that $\Smat_A$ is a real, symmetric matrix imply that 
\beq\label{eq:A380-2} 
 \big\|\Smat_A (\Amato)^{-1}\big\|_{(\Dmat_k)^{-1}}=\big\|\big((\Amato)^*\big)^{-1}\Smat_A\big\|_{\Dmat_k}
 \eeq 
%since 
%\beqs
%\big\|\Smat_A(\Amato)^{-1}\big\|_{2}=\big\|(\Smat_A(\Amato)^{-1})^*\big\|_{2}=\big\|((\Amato)^*)^{-1}\Smat_A\big\|_{2},
%\eeqs
(c.f. \cref{eq:A380}),
and then the arguments in the proof of part (i) imply that 
the bound in \cref{eq:keybound2} on $\|(\Amato)^{-1}\Smat_A\|_{\Dmat_k}$ also holds for $\|((\Amato)^*)^{-1}\Smat_A\|_{\Dmat_k}$.

To prove the bound on  $\|(\Amato)^{-1}\Smat_A\|_{2}$ in \cref{eq:keybound2a}, we use the bounds 
\beqs
m_- h^{d/2} k \N{\tbu}_2 \leq k \N{\widetilde{u}_h}_{\LtDR} \leq \N{\widetilde{u}_h}_{\HokDR}
\,\tand\,
\big\|\nabla \widetilde{f}\big\|_{\LtDR} \leq s_+ h^{d/2-1}\N{\bff}_2,
\eeqs
on either side of the inequality \cref{eq:mainevent2}, with these bounds coming from \cref{eq:normequiv1}.and \cref{eq:normequiv2} respectively. The proof of the bound on 
$\|\Smat_A((\Amato)^*)^{-1}\|_{2}$ in \cref{eq:keybound2a} follows in a similar way to above, using \cref{eq:Fbound}.
\epf

\bre[Link to the results of \cite{GaGrSp:15}]
A result analogous to the Euclidean-norm bounds in \cref{thm:1} was proved in \cite{GaGrSp:15} for the case that $\Aso= \Ast= I$, $\nst= 1$, and $\nso = 1 + \ri\eps/k^2$, with the `absorption parameter' or `shift' $\eps$ satisfying $0<\eps\lesssim k^2$. The motivation for proving this result was that the so-called `shifted Laplacian preconditioning' of the Helmholtz equation is based on preconditioning (with these choices of parameters) $\Amatt$ with an approximation of $\Amato$. Similar to \cref{cor:1}, bounds on $\|\Imat -  (\Amato)^{-1}\Amatt \|_2$ and 
$\|\Imat - \Amatt  (\Amato)^{-1}\|_2$
 then give upper bounds on large the `shift' $\eps$ can be for GMRES to converge in a $k$-independent number of iterations in the case when the action of $(\Amato)^{-1}$ is computed exactly.

%\cite[Lemma 4.1]{GaGrSp:15}
The main differences between \cite{GaGrSp:15} and the present paper are that (i)  \cite{GaGrSp:15} focused on the TEDP, not the EDP,
(ii) \cite{GaGrSp:15} focused on the particular case that $\Dm$ is star-shaped with respect to a ball, finding a $k$- and $\eps$-explicit expression for $C^{(1)}_{\rm bound}$ in this case using Morawetz identities, whereas we assume the existence of $\Cboundo,$
(iii) \cite{GaGrSp:15} required a bound on 
$(\Amato)^{-1}\Mmat_{n}$, analogous to the bounds in \cref{lem:keylemma1} along with one on $(\Amato)^{-1}\Nmat$ (in the case that $T_R$ is approximated by $\ri k$), but \emph{not} on 
$(\Amato)^{-1}\Smat_{A}$, and (iv) \cite{GaGrSp:15} only proved bounds in the $\|\cdot\|_2$ norm.
%The result of \cref{thm:1}
\ere

%\bre[Analogue of \cref{thm:1} in a weighted norm]\label{rem:weight1}
%The PDE analysis of the Helmholtz equation naturally takes place in the weighted $H^1$ norm $\|\cdot\|_{\HokDR}$ defined by \cref{eq:1knorm}. The discrete analogue of this norm is the norm $\|\cdot\|_{\Dmat_k}$ defined by 
%\beq\label{eq:Dk}
%\N{\bv}_{\Dmat_k}^2:= \big( (\Smat_I + k^2 \Mmat_1)\bv,\bv\big)_2 = \N{v_h}^2_{\HokDR}
%\eeq
%for
%$v_h =\sum_i v_i \phi_i$. 
%This norm is used, e.g., in recent results about the convergence of domain-decomposition methods %in this norm are proved 
%for the Helmholtz equation \cite{GrSpVa:17}, \cite{GrSpZo:18}, and for the time-harmonic Maxwell equations \cite{BoDoGrSpTo:19}. 
%
%Inspecting the proof of \cref{lem:keylemma}, we see that the bounds \cref{eq:keybound1} and \cref{eq:keybound2} hold with the $\|\cdot\|_2$ norm replaced by the $\|\cdot\|_{\Dmat_k}$ norm and without the terms involving $m_\pm$ and $s_\pm$ on the right-hand side. \cref{thm:1} 
%%(and also \cref{cor:1}) 
%therefore also holds with the $\|\cdot\|_2$ norm replaced by the $\|\cdot\|_{\Dmat_k}$ norm and the constant $C_1$ modified appropriately.
%\ere

\subsection{Proof of \cref{cor:1,cor:1a}}

We first give the set-up for weighted GMRES.
Consider the abstract  linear system 
% \begin{equation*}
$\matrixC \bx = \bd$
%\end{equation*}
in $\mathbb{C}^N$, where $\matrixC \in \CC^{N\times N}$ is invertible.   
Given an initial guess $\bx^0$, we introduce the residual $\br^0 := \bd- C \bx^0$ and 
the usual Krylov spaces:  
\beqs  
\cK^m(C, \br^0) := \mathrm{span}\big\{\matrixC^j \br^0 : j = 0, \ldots, m-1\big\}.
\eeqs
Let $(\cdot , \cdot )_{\Dmat}$ denote the inner product on $\CC^n$ 
induced by some Hermitian positive-definite matrix $\Dmat$, i.e.~
%\begin{equation*}
$(\bv,\bw)_{\Dmat} := (\Dmat \bv, \bw)_2,$
%\end{equation*} 
with induced norm $\Vert \cdot \Vert_\Dmat$. For $m \geq 1$, define   $\bx^m$  to be  the unique element of $\cK^m$ satisfying  the  
 minimal residual  property: 
$$ \ \Vert \br^m \Vert_\Dmat := \Vert \bd - \matrixC \bx^m \Vert_\Dmat \ = \ \min_{\bx \in \cK^m(C, \br^0)} \Vert {\bd} - {\matrixC} {\bx} \Vert_\Dmat. $$
When $\Dmat = \Imat$ this is just the usual GMRES algorithm, but for  more general  $\Dmat$ it 
is the weighted GMRES method \cite{Es:98} in which case  
its implementation requires the application of the weighted Arnoldi process \cite{GuPe:14}.
Let 
\beq\label{eq:fov}
W_\Dmat(\matrixC):= \Big\{ (\matrixC \bx, \bx)_{\Dmat} : \bx \in \CCN, \|\bx\|_\Dmat=1\Big\};
\eeq
$W_\Dmat(\matrixC)$ is called the \emph{numerical range} or \emph{field of values} of $\matrixC$ (in the $(\cdot,\cdot)_\Dmat$ inner product).

%Recall the so-called ``Elman estimate" for GMRES

\begin{theorem}[Elman estimate for weighted GMRES]\label{thm:GMRES1_intro} 
Let $\matrixC$ be a matrix with $0\notin W(\matrixC)$. Let $\beta\in[0,\pi/2)$ be defined such that
\beq\label{eq:cosbeta}
\cos \beta := \frac{\mathrm{dist}\big(0, W(\matrixC)\big)}{\N{\matrixC}_{2}}.
\eeq
If the matrix equation $\matrixC \bx = \by$ is solved using weighted GMRES then, 
for $m\in \mathbb{N}$, the GMRES residual $\br_m$ %:= \matrixC \bx_m - \by$ 
satisfies
\beq\label{eq:Elman}
\frac{\N{\bfr_m}_{\Dmat}}{\N{\bfr_0}_{\Dmat}} \leq \sin^m \beta. %, \quad \text{ where}\quad 
\eeq
\end{theorem}
The bound \cref{eq:Elman} with $\Dmat=\Imat$ was originally proved in \cite{El:82} (see also \cite[Theorem 3.3]{EiElSc:83}) and appears in the form above in \cite[Equation 1.2]{BeGoTy:06}. The bound \cref{eq:Elman} (for arbitrary Hermitian positive-definite $\Dmat$) was stated (without proof) in \cite{CaWi:92} and proved in \cite[Theorem 5.1]{GrSpVa:17}. % (see also \cite[Remark 5.2]{GrSpVa:17}). 



\cref{thm:GMRES1_intro} has the following corollary, and the proofs of \cref{cor:1,cor:1a} follow from combining this with \cref{thm:1}.

\begin{corollary}
\label{cor:GMRES_intro} 
If $\|\Imat - \matrixC \|_\Dmat \leq \alpha < 1$, then, with $\beta$ defined as in \cref{eq:cosbeta},
\beqs
\cos \beta \geq \frac{1-\alpha}{1+\alpha}\eeqs
and
\beq\label{eq:gmressin}
\sin \beta \leq \frac{2 \sqrt{\alpha}}{(1+\alpha)^2}.
\eeq
\end{corollary}

\bpf[Proof of \cref{cor:1}]
This follows from \cref{thm:1} by applying \cref{cor:GMRES_intro} first with $\matrixC= (\Amato)^{-1} \Amatt$, $\Dmat=\Dmat_k$, and $\alpha=1/2$, and then with $\matrixC= \Amatt(\Amato)^{-1} $, $\Dmat=(\Dmat_k)^{-1}$, and $\alpha=1/2$.
\epf

\

\bpf[Proof of \cref{cor:1a}]
This follows from \cref{thm:1} by applying \cref{cor:GMRES_intro} first with $\matrixC= (\Amato)^{-1} \Amatt$, $\Dmat=\Imat$, and $\alpha=1/2$, and then with $\matrixC= \Amatt(\Amato)^{-1} $, $\Dmat=\Imat$, and $\alpha=1/2$.
\epf


\bre[The improvement of the Elman estimate \cref{eq:Elman} in \cite{BeGoTy:06}]
A stronger result than \cref{eq:Elman} is given for standard (unweighted) GMRES in \cite[Theorem 2.1]{BeGoTy:06}, and then converted to a result about weighted GMRES in \cite[Theorem 5.3]{BoDoGrSpTo:19}; indeed, the convergence factor $\sin \beta$ is replaced by a function of $\beta$ strictly less than $\sin\beta$ for $\beta\in (0,\pi/2)$. Using this stronger result, however, does not improve the $k$-dependence of \cref{cor:1}.
\ere


%\section{Proof of }\label{sec:proofPDE}

\subsection{Proofs of \cref{thm:2}, and \cref{lem:sharp}}

\bpf[Proof of \cref{thm:2}]
%We first prove the upper bound \cref{eq:PDEbound}.
By \cref{prob:edp}, $u^{(1)}$ and $u^{(2)}$ satisfy $a^{(1)}(u^{(1)}, v) = F(v)$ and 
$a^{(2)}(u^{(2)}, v) = F(v)$ respectively. Subtracting these equations, we find that $u^{(1)}- u^{(2)}$ satisfies the variational problem
\beq\label{eq:vp1}
a^{(1)}(u^{(1)}-u^{(2)},v) = \widetilde{F}(v) \quad\tfa v\in H^1_{0,D}(\DR)
\eeq
where
\beqs
 \widetilde{F}(v):= \int_{\DR} \left((\Ast-\Aso) \nabla u^{(2)}\right) \cdot\overline{\nabla v} + k^2 (\nso-\nst) u^{(2)}\overline{v}.
\eeqs
Now, by the Cauchy-Schwarz inequality and the definition of the norm $\|\cdot\|_{\HokDR}$ \cref{eq:weightednorm}, we have that
\begin{align*}
| \widetilde{F}(v)| &\leq \big\|\Aso-\Ast\big\|_{L^\infty(\DR)} \big\|\nabla u^{(2)}\big\|_{L^2(\DR)}
\N{\nabla v}_{L^2(\DR)} 
\\& \hspace{5cm}+ k^2 
\big\|\nso-\nst\big\|_{L^\infty(\DR)} \big\| u^{(2)}\big\|_{L^2(\DR)}
\N{v}_{L^2(\DR)}\\
&\leq\max\Big\{\big\|\Aso-\Ast\big\|_{L^\infty(\DR)}\,,\, \big\|\nso-\nst\big\|_{L^\infty(\DR)}\Big\}
\big\| u^{(2)}\big\|_{\HokDR} \N{v}_{\HokDR}.
\end{align*}
and thus, by the definition of the norm $\|\cdot\|_{(\HokDR)'}$ \cref{eq:dualnorm},
\beqs
\big\|\widetilde{F}\big\|_{(\HokDR)'}\leq \max\Big\{\big\|\Aso-\Ast\big\|_{L^\infty(\DR)}\,,\, \big\|\nso-\nst\big\|_{L^\infty(\DR)}\Big\}
\big\| u^{(2)}\big\|_{\HokDR}.
\eeqs
Since \cref{cond:1nbpc} holds, we can then apply the result of \cref{lem:H1}, i.e.~the bound \cref{eq:bound2}, to the solution of the variational problem \cref{eq:vp1}  to find that 
\begin{align*}
\frac{\big\| u^{(1)} - u^{(1)}\big\|_{\HokDR}}
{\big\| u^{(2)}\big\|_{\HokDR}, 
}
 \leq 
\,&\frac{1}{\min\big\{\Asomin,\nsomin\big\}}\left( 1 + 2 C^{(1)}_{\rm bound}\nsomax  k\right)
\\
&\quad\times \left(\max\Big\{\big\|\Aso-\Ast\big\|_{L^\infty(\DR)}\,,\, \big\|\nso-\nst\big\|_{L^\infty(\DR)}\Big\}\right),
\end{align*}
and then the result \cref{eq:PDEbound} follows with 
\beq\label{eq:C3}
C_3:= \frac{1}{\min\big\{\Asomin,\nsomin\big\}}\left( \frac{1}{k_0} + 2 C^{(1)}_{\rm bound}\nsomax  \right).
\eeq
\epf

\bpf[Proof of \cref{lem:sharp}]
We actually prove the stronger result that given any function $c(k)$ such that $c(k)>0$ for all $k>0$, there exist 
$f, \nso, \nst$ (with $\nso\not= \nst$), such that, firstly,
\beqs
\big\|\nso-\nst\big\|_{L^\infty(\DR)} \sim c(k)
\eeqs
and, secondly,
the corresponding solutions $u^{(1)}$ and $u^{(2)}$ of the exterior Dirichlet problem with $\Aso = \Ast= I$ exist, are unique, and satisfy \cref{eq:sharp1}. 

The heart of the proof is the equation
\beq\label{eq:obs1}
(\Delta + k^2) \big(\re^{\ri k r}\chi(r)\big) =  \re^{\ri k r}\left(\ri k \frac{d-1}{r} \chi(r) + 2 \ri k \diff{\chi}{r}(r) + \Delta \chi(r)\right)=: -\widetilde{f}(r),
\eeq
where $\chi(r)$ is chosen to have $\supp \chi \subset \DR$.
This equation proves the sharpness of the nontrapping resolvent estimate \cref{eq:bound1}, since both the $L^2(\DR)$ norm of $\widetilde{f}$ and the $\HokDR$ norm of $\re^{\ri kr}\chi(r)$ are proportional to $k$, and hence each to other (see, e.g., \cite[Lemma 3.10]{ChMo:08},  \cite[Lemma 4.12]{Sp:14}).

The overall idea of the proof is to set things up so that $(u^{(1)}-u^{(2)})(\bx) = \re^{\ri k r}\chi(r)$, the rationale being that \cref{eq:obs1} proves the sharpness of \cref{eq:bound1}, and \cref{eq:bound1} and its corollary \cref{eq:bound2} (applied to $u^{(1)}-u^{(2)}$) are the main ingredients in the proof of \cref{thm:2}.
% to prove the sharpness of \cref{thm:2} (at least when $\Aj:=I$, $j=1,2,$), 

Observe that, when $\Aj:=I$, $j=1,2,$ and $\nso:=1$, the variational problem \cref{eq:vp1} implies that 
\beq\label{eq:obs2}
\Delta \big( u^{(1)} - u^{(2)}\big) + k^2 \big( u^{(1)} - u^{(2)}\big) = -k^2 \big(1-\nst\big)u^{(2)}.
\eeq
Let $\nst:= 1 + c(k)\widetilde{\chi}$ with $\widetilde{\chi}= \widetilde{\chi}(r)$, $\widetilde{\chi}\in C^{\infty}(\DR)$, $\widetilde{\chi}\not = 1$ (so that $\nst\not = \nso$), and 
 $\supp \, \widetilde{\chi} \subset\DR$. 
%observe then that $\nst(\bx) >1$ for all $\bx \in \DR$ and thus, in particular, %for some function $c(k)>0$ for all $k>0$ 
%$\nst\not = \nso$. 
As above, let $\chi=\chi(r)$ with $\chi \in C^{\infty}(\DR)$ and
$\supp \,\chi$ a compact subset of $\DR$. Then with $\widetilde{f}$ defined in \cref{eq:obs1}, our goal is to let 
\beq\label{eq:obs3}
u^{(2)}(\bx):= -\frac{1}{k^2 c(k)}\frac{\widetilde{f}(r)}{\widetilde{\chi}(r)} \quad\tand\quad  f(\bx):= -\big(\Delta +k^2 \nst(\bx)\big) u^{(2)}(\bx);
\eeq
however, since $\widetilde{\chi}(r)$ has compact support, we need to tie both the support of $\widetilde{\chi}$ and how fast $\widetilde{\chi}$ vanishes in a neighbourhood of its support to the definition of $\chi$ for both $u^{(2)}$ and $f$ to be well defined.

Setting aside for the moment this need to synchronise the definitions of $\chi$ and $\widetilde{\chi}$, since $\supp \,\widetilde{f}$ is a compact subset of $\DR$, so is 
$\supp \,u^{(2)}$, and so $u^{(2)}$ is then the solution of the exterior Dirichlet problem (in the sense of \cref{prob:edp}) with data $f$ defined in \cref{eq:obs3} and coefficient $\nst:=1 + c(k)\widetilde{\chi}$.
Finally, define $u^{(1)}$ to be the solution of the exterior Dirichlet problem with $f$ defined in \cref{eq:obs3}. The whole point of these definitions is that, combined with \cref{eq:obs1} and \cref{eq:obs2} and the uniqueness of the solution of the exterior Dirichlet problem, they imply that 
\beq\label{eq:obs4}
u^{(1)}(\bx)- u^{(2)}(\bx) = \re^{\ri k x_1}\chi(r),
\eeq
and from this we therefore have that
\beqs
\big\|u^{(1)}-u^{(2)}\big\|_{L^2(\DR)} \sim 1
\quad \tand \quad
\big\|u^{(1)}-u^{(2)}\big\|_{\HokDR} \sim k.
\eeqs
Furthermore, the definitions of $u^{(2)}$ \cref{eq:obs3} and $\widetilde{f}$ \cref{eq:obs1} imply that
\beqs
\big\| u^{(2)}\big\|_{L^2(\DR)} \sim \frac{1}{k\, c(k)} \quad\tand \quad 
\big\| u^{(2)}\big\|_{\HokDR} \sim \frac{1}{c(k)},
\eeqs 
and, since $\|\nso- \nst\|_{L^\infty(\DR)} = c(k)$, \cref{eq:sharp1} holds.

Therefore, to complete the proof, we only need to show that there exists a choice of $\chi$ and $\widetilde{\chi}$ for which $u^{(2)}$ and $f$ defined by \cref{eq:obs3} are 
in $H^{1}(\DR)$ and $\LtDR$ respectively (in fact, we prove that they are in $W^{1,\infty}(\DR)$ and $L^\infty(\DR)$ respectively).
%well-defined. 
Since $\chi$ and $\widetilde{\chi} \in C^\infty(\DR)$, the only issue is what happens at the boundary of the support of $\widetilde{\chi}$, where $u^{(2)}$ has the potential to be singular.
Since $\overline{\Omega_-} \subset \BR$, there exist $0<R_1<R_2<R$ such that $\overline{\Dm} \subset B_{R_2}\setminus B_{R_1} \subset \BR$. Let $\supp \chi = B_{R_2}\setminus B_{R_1}$ and let $\chi$ vanish to order $m$ at $r= R_1$ and $r=R_2$; i.e.~$\chi(r) \sim (r-R_1)^m$ as $r \rightarrow (R_1)^+$ and 
$\chi(r) \sim (R_2-r)^m$ as $r \rightarrow (R_2)^-$. The definition of $\widetilde{f}$ \cref{eq:obs3} then implies that $\widetilde{f}$ vanishes to order $m-2$. Let $\widetilde{\chi}(r)$ vanish to order $n$ at $r= R_1$ and $r=R_2$. 
We now claim that if $m >n+4$, then $u^{(2)}\in W^{1,\infty}(\DR)$ and $f$ $\in L^\infty(\DR)$. Indeed,  
straightforward calculation from \cref{eq:obs3} shows that  $u^{(2)}(r) \sim (r-R_1)^{m-n-2}$, $\nabla u^{(2)}(r) \sim (r-R_1)^{m-n-3}$, and $\Delta u^{(2)}(r) \sim (r-R_1)^{m-n-4}$ as $r \rightarrow (R_1)^+$, with analogous behaviour at $r=R_2$.
%vanishes to order $m-n-2$ and $\Delta u^{(2)}$ vanishes to order $m-n-4$ at $r= R_1$ and $r=R_2$. 
The assumption 
$m >n+4$ therefore implies that $u^{(2)}$, $\nabla u ^{(2)}$, and $\Delta u^{(2)}$ vanish (and hence are finite) at $r=R_1$ and $r=R_2$.
\epf

\bre[Why doesn't \cref{lem:sharp} cover the case $\Aso\neq  \Ast$?]
When $\nj:=1$, $j=1,2,$, $\Aso:=I$, and $\Ast:= I + c(k)\widetilde{\chi}$, the variational problem \cref{eq:vp1} implies that 
\beqs%\label{eq:obs2}
\Delta \big( u^{(1)} - u^{(2)}\big) + k^2 \big( u^{(1)} - u^{(2)}\big) = c(k)\nabla\cdot \big(\widetilde{\chi}\nabla u^{(2)}\big).
\eeqs
It is now much harder than in \cref{eq:obs2} to set things up so that $ u^{(1)}(\bx) - u^{(2)}(\bx)=\re^{\ri kr}\chi(r)$ (so that one can then use \cref{eq:obs1}).
\ere

%\section{Proof of \cref{lem:2}}

\section{Extension of the results to weaker norms and probabilistic convergence results}
\subsection{Analogues of \cref{thm:1,cor:1,cor:1a} in weaker norms}\label{sec:weaknorm}
Recall from \cref{sec:num,sec:main} that GMRES applied to $\AmatoI \Amatt$ converges in a $k$-independent number of iterations if $\NLiDRRR{\nso-\nst} \sim 1/k$ (with an analagous result for $\Aso-\Ast$). This result show that $1/k$ is a sharp threshold when we consider the \emph{magnitude}  of the difference between $\nso$ and $\nst$. However, this result does not say anything about the \emph{spatial} variability between $\nso$ and $\nst$. For example if $\nso$ and $\nst$ (defined on the unit square) are given by
\beq\label{eq:noweak}
\nso(\bx) =
\begin{dcases}
  1 &\tif \bx_1 \leq \half\\
  2  &\tif \bx_1 > \half
  \end{dcases}
\eeq
and
\beq\label{eq:ntweak}
\nst(\bx) =
\begin{dcases}
  1 &\tif \bx_1 \leq \half+\alpha\\
  2  &\tif \bx_1 > \half+\alpha
  \end{dcases}
\eeq
for some $0 < \alpha < 1/2,$ then $\NLiDRRR{\nso-\nst} = 1$ for all $\alpha$, but one would expect that for small $\alpha$ the corresponding solutions of \cref{prob:edp} satisfy $\uso \approx \ust,$ and one might expect that GMRES applied to $\AmatoI\Amatt$ would converge in a $k$-independent number of iterations. Therefore, in this \lcnamecref{sec:weaknorm} we seek to obtain analogues of \cref{thm:1,cor:1,cor:1a} with the difference in $\nso-\nst$ and $\Aso-\Ast$ measured in weaker norms than the $L^\infty$ norm.

The (realistic) best-case result we could obtain would be that GMRES applied to $\AmatoI\Amatt$ converges in a $k$-independent number of iterations if $\NLoDRRR{\nso-\nst} \lesssim 1/k$. This result is `best' in the sense that it depends optimally on $k$; recall the discussion in \cref{rem:physical1k} that the length scale $1/k$ is the length scale governing the behaviour of Helmholtz problems. It is also `best' with regards to the norm used to measure $\nso-\nst$. When we measure $\nso-\nst$ in the $L^\infty$ norm, we are able to control the \emph{magnitude} of $\nso-\nst$, but not the spatial variability; if $\nso-\nst \neq 0$ only on a set of small (but nonzero) measure, where $\nso-\nst=1,$ then $\NLiDRRR{\nso-\nst} = 1$, regardless of the measure of the set. In contrast, the $L^1$ norm allows us to control both the magnitude of $\nso-\nst$ and the measure of the sets on which it is nonzero.

We will give numerical results indicating that this best-case result is sharp (our results actually indicate that we obtain $k$-independent convergence is $\NLqDRRR{\nso-\nst}\sim 1/k$ for any $q>0$). We will also provide theory results that are, to our knowledge, the best one can prove, although they are sub-optimal in both $q$ and the dependence on $k.$


\subsubsection{Theory in weaker norms}\label{sec:weakertheory}
The reason that the terms $\NLiDRRRdtd{\Aso-\Ast}$ and $\NLiDRRR{\nso-\nst}$ appear in \cref{thm:1} is that the terms $\NLiDRRR{n}$ and $\NLiDRRRdtd{A}$ appear in \cref{lem:keylemma1,lem:keylemma2}, respectively. These terms appear because in \cref{eq:mainevent1a,eq:Fbounda} we use the bounds
\beq\label{eq:keynbound}
\NLtDR{n\ftilde} \leq \NLiDRRR{n}\NLtDR{\ftilde}
\eeq
and
\beq\label{eq:keyAbound}
\NLtDR{A \grad \ftilde} \leq \NLiDRRRdtd{A}\NLtDR{\grad \ftilde}
\eeq
respectively, for some function $\ftilde,$ and these bounds are carried through the rest of the proof.

However, we observe that we have the following generalisation of H\"older's inequality: If $\ptilde,\qtilde > 2$ such that $1/2 = 1/\ptilde + /\qtilde,$ then
\beq\label{eq:genholder}
\NLtDR{\vo\vt} \leq \NLptildeDR{\vo}\NLqtildeDR{\vt}.
\eeq

If we instead use \cref{eq:genholder} to bound \cref{eq:keynbound,eq:keyAbound} we instead obtain
\beq\label{eq:keynbound2}
\NLtDR{n\ftilde} \leq \NLqtildeDR{n}\NLptildeDR{\ftilde}
\eeq
and
\beq\label{eq:keyAbound2}
\NLtDR{A\grad\ftilde} \leq \NLqtildeDR{A}\NLptildeDR{\ftilde}.
\eeq

As $\ftilde \in \Vhp$, we can apply an inverse inequality to bound $\NLptildeDR{\ftilde}$ by $\NLtDR{\ftilde}$. The required inverse inequality is (see \cite[Theorem 4.5.11 and Remark 4.5.20]{BrSc:08}
\beq\label{eq:inverseptilde}
\NLptildeDR{\ftilde} \leq \Cinvptilde h^{d\mleft(\frac1{\ptilde} - \half\mright)} \NLtDR{\ftilde}.
\eeq
If we then apply \cref{eq:inverseptilde} to \cref{eq:keynbound2,eq:keyAbound2} we obtain
\beq\label{eq:keynboundfinal}
\NLtDR{n\ftilde} \leq \Cinvptilde \NLqtildeDR{n} h^{d\mleft(\frac1{\ptilde} - \half\mright)} \NLtDR{\ftilde}
\eeq
and
\beq\label{eq:keyAboundfinal}
\NLtDR{A\grad\ftilde} \leq \Cinvptilde \NLqtildeDR{A} h^{d\mleft(\frac1{\ptilde} - \half\mright)} \NLtDR{\grad\ftilde}.
\eeq

Replacing \cref{eq:mainevent1a,eq:Fbounda} with \cref{eq:keynboundfinal,eq:keyAboundfinal}, and proceeding as in the proofs of \cref{lem:keylemma1,lem:keylemma2}, we obtain \cref{lem:keylemma1a,lem:keylemma2a} below, the analogues of \cref{lem:keylemma1,lem:keylemma2}.

\ble[Alternative bounds on $(\Amato)^{-1} \Mmat_{n}$]\label{lem:keylemma1a}
Under the assumptions of \cref{lem:keylemma1}, for $n\in \LiDRRR$ and for any $\ptilde,\qtilde > 2$ such that $1/\ptilde + 1/\qtilde = \half$,
\beq\label{eq:keybound12}
\max\set{\NDk{\AmatoI \Mmatn},\NDkI{\Mmatn\AmatoI}} \leq \Cttilde h^{d\mleft(\frac1{\ptilde}-\half\mright)} \frac{\NLqtildeDR{n}}k
\eeq
and 
\beq\label{eq:keybound1a2}
\max\set{\Nt{\AmatoI \Mmatn},\Nt{\Mmatn\AmatoI}} \leq \Cttilde\mleft(\frac{\mplus}{\mminus}\mright) h^{d\mleft(\frac1{\ptilde}-\half\mright)} \frac{\NLqtildeDR{n}}k
\eeq
for all $k\geq \kz$,
where
\beq\label{eq:C2tilde}
\Cttilde\de%\frac{m_+}{m_-} 
%\left[ 
\Cinvptilde\Ct,
\eeq
where $\Ct$ is defined by \cref{eq:C2}.
\ele

\ble[Alternative bounds on $(\Amato)^{-1} \Smat_A$]\label{lem:keylemma2a}
Under the assumptions of \cref{lem:keylemma2}, for $A\in L^\infty(\DR,\RR^{d\times d})$ and for any $\ptilde,\qtilde > 2$ such that $1/\ptilde + 1/\qtilde = \half$,
\beq\label{eq:keybound22}
\max\set{\NDk{\AmatoI \SmatA},\NDkI{\SmatA\AmatoI}} \leq \Cotilde h^{d\mleft(\frac1{\ptilde}-\half\mright)}k \NLqtildeDR{A}
\eeq
and
\beq\label{eq:keybound2a2}
\max\set{\Nt{\AmatoI \SmatA},\Nt{\SmatA\AmatoI}} \leq \Cotilde\mleft(\frac{\splus}{\mminus}\mright) h^{d\mleft(\frac1{\ptilde}-\half\mright)-1} \NLqtildeDR{A}
\eeq
%\begin{align}\nonumber
%&\max\Big\{\big\| (\Amato)^{-1} \Smat_A \big\|_2, \,\,
%\big\| \Smat_A (\Amato)^{-1} \big\|_2\Big\}\nonumber \\
%&\hspace{2cm}
% \leq \frac{s_+}{s_-} \left[ C_{\rm FEM2}^{(1)} + 
% \frac{1}{\min\big\{\Asomin,\nsomin\big\}}\left( \frac{1}{k_0} + 2 C^{(1)}_{\rm bound}\nsomax  \right) \right]k\N{A}_{L^\infty(\DR)}\label{eq:keybound2}
%% + C_{\rm bound}^{(1)}\right) \frac{\N{n}_{L^\infty(\DR)}}{k}.
%\end{align}
for all $k\geq k_0$, where
\beq\label{eq:C1tildenbpc}
\Cotilde \de \Cinvptilde\Co,
\eeq
where $\Co$ is given by \cref{eq:C1nbpc}.
\ele


\begin{theorem}[Alternative main ingredient to answer to \cref{it:nbpcq1}]\label{thm:1alt}
Let the assumptions of \cref{thm:1} hold.   Then, given $\kz>0$ and $\ptilde,\qtilde >2$ such that $1/\ptilde + 1/\qtilde = 1/2$, there exist $\Cotilde, \Cttilde>0$, independent of $h$ and $k$ (but dependent on $\Dm, \Aso, \nso$, $p$, $\ptilde$, and $\kz$) such that
\begin{align}\nonumber
&\max\set{\NDk{\Imat - \AmatoI\Amatt},\NDkI{\Imat -\Amatt\AmatoI}}\\
&\hspace{3cm} 
\leq \Cotilde kh^{d\mleft(\frac1{\ptilde}-\half\mright)} \NLqtildeDR{\Aso-\Ast} + \Cttilde  kh^{d\mleft(\frac1{\ptilde}-\half\mright)}  \NLqtildeDR{\nso-\nst}
\label{eq:main1alt}
\end{align}
and 
\begin{align}\nonumber
&\max\set{\Nt{\Imat - \AmatoI\Amatt}, \Nt{\Imat -\Amatt\AmatoI}}\\
&\hspace{0cm}
\leq \Cotilde \mleft(\frac{\splus}{\mminus}\mright) h^{d\mleft(\frac1{\ptilde}-\half\mright)-1}\NLqtildeDR{\Aso-\Ast} + \Cttilde \mleft(\frac{\mplus}{\mminus}\mright) kh^{d\mleft(\frac1{\ptilde}-\half\mright)}\NLqtildeDR{\nso-\nst}
\label{eq:main1aalt}
\end{align}
for all $k\geq k_0$. 
\end{theorem}

\begin{corollary}[Alternative answer to \cref{it:nbpcq1}: $k$-independent weighted GMRES iterations]\label{cor:1alt}
Let the assumptions of \cref{cor:1a} hold.  Given $k_0>0$ and $\qtilde >2$, let $\Cotilde$ and $\Cttilde$ be as in \cref{thm:1alt}. Then if 
% there exists $C_2>0$, independent of $h$ and $k$ (but dependent on $\Dm, \Aso, \nso$, $p$, and $k_0$) and given explicitly in \cref{eq:C2} below,
% such that if 
\beq\label{eq:condalt}
\Cotilde kh^{-\frac{d}{\qtilde}} \NLqtildeDR{\Aso-\Ast} +\Cttilde  kh^{-\frac{d}{\qtilde}} \NLqtildeDR{\nso-\nst}
\leq \frac{1}{2}
\eeq
for all $k\geq k_0$, then \emph{both} weighted GMRES working in $\|\cdot\|_{\Dmat_k}$ (and the associated inner product) applied to \cref{eq:pcsystem1} \emph{and} weighted GMRES working in $\|\cdot\|_{(\Dmat_k)^{-1}}$ (and the associated inner product) applied to \cref{eq:pcsystem2}  converge in a $k$-independent number of iterations for all $k\geq k_0$.
\end{corollary}

\begin{corollary}[Alternative answer to \cref{it:nbpcq1}: $k$-independent (unweighted) GMRES iterations]\label{cor:1aalt}
Let the assumptions of \cref{cor:1a} hold.  Given $k_0>0$, and $\qtilde >2$, let $\Cotilde$ and $\Cttilde$ be as in \cref{thm:1alt}. Then if 
% there exists $C_2>0$, independent of $h$ and $k$ (but dependent on $\Dm, \Aso, \nso$, $p$, and $k_0$) and given explicitly in \cref{eq:C2} below,
% such that if 
\beq\label{eq:condaalt}
\Cotilde \mleft(\frac{\splus}{\mminus}\mright) h^{-\frac{d}{\qtilde}-1} \NLqtildeDR{\Aso-\Ast} + \Cttilde \mleft(\frac{\mplus}{\mminus}\mright) kh^{-\frac{d}{\qtilde}} \NLqtildeDR{\nso-\nst} \leq \half
\eeq
for all $k\geq k_0$, then standard GMRES (working in the Euclidean norm and inner product) applied to either of the equations \cref{eq:pcsystem1} or \cref{eq:pcsystem2}
%\beqs
%(\Amat^{(1)})^{-1}\Amat^{(2)}\bu = \bff\quad\text{ or } \quad\Amat^{(2)}(\Amat^{(1)})^{-1}\bv = \bff
%\eeqs
 converges in a $k$-independent number of iterations for all $k\geq k_0$.
\end{corollary}

Observe that in \cref{cor:1alt,cor:1aalt} there is a trade-off between the norm that one uses to measure $\no-\nt$ and the restriction on the magnitude of this norm. E.g., the condition on $\no-\nt$ in both \cref{cor:1alt,cor:1aalt} can be summarised as
\beq\label{eq:altsufficientlysmall}
\NLqtildeDRRR{\no-\nt} k h^{-\frac{d}{\qtilde}} \text{ is sufficiently small}.
\eeq
Observe that as $\qtilde \downarrow 2,$ we measure $\no-\nt$ in a weaker norm, but the condition \cref{eq:altsufficientlysmall} becomes more restrictive; the power of $h$ increases. Conversely, as $\qtilde \uparrow \infty,$ we measure $\no-\nt$ in a stronger norm, but the condition \cref{eq:altsufficientlysmall} becomes less restrictive; the power of $h$ decreases. (Also observe that in the $\qtilde=\infty$ limit we recover the condition \cref{eq:sufficientlysmall} we previously proved for $\NLiDRRR{\no-\nt}.$

However, the numerics below suggest that in certain cases, a sufficient condition for nearby preconditioning to be effective is
\beq\label{eq:experimentalsufficientlysmall}
\NLqtildeDRRR{\no-\nt} k \text{ is sufficiently small},
\eeq
for \emph{any} $\qtilde \geq 1$, and moreover \cref{eq:experimentalsufficientlysmall} may be sharp in its $k$-dependence. (This requirement would fit with our previous observation about $1/k$ being the length scale below which perturbations cannot be seen---see \cref{rem:physical1k} above.) However, we do not saythat \cref{eq:experimentalsufficientlysmall} is sufficient for all cases; recall that for transmission problems, very small perturbations in $n$ can lead to very different behaviour in the solution $u$ if $k$ is a quasi-resonance for $\no$ or $\nt$; see the discussion at the end of \cref{sec:wpdisc} above.


\subsubsection{Numerics in weaker norms}\label{sec:weakernumerics}
For our computations, we use the computational setup as in \cref{app:compsetup}, with $f$ and $\gI$ corresponding to a plane wave passing through homogeneous media. We let $\Aso=\Ast=I,$ and we define $\nso$ and $\nst$ by \cref{eq:noweak,eq:ntweak}. For $\alpha = 0.2,0.2/k^{0.1},0.2/k^{0.2},\ldots,0.2/k$ and for $k=10,20,\ldots,100$ we use GMRES to solve $\AmatoI\Amatt = \AmatoI \bff$ (for $\bff$ given by the Helmholtz problem), and we record the number of GMRES iterations taken to achieve convergence.

Our results in \cref{fig:l1low,fig:l1med,fig:l1high} indicate the following conclusions for $\NLqDRR{\nso-\nst} \sim 0.1/k^{\beta}$ :
\bit
\item For $\beta \in (0,0.6)$ there is clear growth of the number of GMRES iterations with $k$,
\item For $\beta = 1$ there is clear boundedness of the number of GMRES iterations with $k$, and
  \item for $\beta \in (0.7,0.9)$ it is unclear if the number of GMRES iterations grows with $k.$
\eit

If we compare our numerical results with the theory results in \cref{cor:aalt}, we see that the theory (with $h \sim k^{-3/2}$ and $d=2$) predicts that the number of iterations will remain bounded if $\NLqDRR{\nso-\nst} k^{1+3/q}$ is sufficiently small, for any $q \in (2,\infty).$ Our computed results indicate that this result is not sharp. The computed results indicate that if $\NLqDRR{\nso-\nst} \sim k^{-1}$ for any $q \geq 1,$ then the number of GMRES iterations is bounded as $k$ increases. Observe that the `best case' $1/k$ condition is only predicted by the theory in the $q\rightarrow \infty$ limit.

\begin{figure}
%% Creator: Matplotlib, PGF backend
%%
%% To include the figure in your LaTeX document, write
%%   \input{<filename>.pgf}
%%
%% Make sure the required packages are loaded in your preamble
%%   \usepackage{pgf}
%%
%% Figures using additional raster images can only be included by \input if
%% they are in the same directory as the main LaTeX file. For loading figures
%% from other directories you can use the `import` package
%%   \usepackage{import}
%% and then include the figures with
%%   \import{<path to file>}{<filename>.pgf}
%%
%% Matplotlib used the following preamble
%%   \usepackage{fontspec}
%%   \setmainfont{DejaVuSerif.ttf}[Path=/home/owen/progs/firedrake-complex/firedrake/lib/python3.5/site-packages/matplotlib/mpl-data/fonts/ttf/]
%%   \setsansfont{DejaVuSans.ttf}[Path=/home/owen/progs/firedrake-complex/firedrake/lib/python3.5/site-packages/matplotlib/mpl-data/fonts/ttf/]
%%   \setmonofont{DejaVuSansMono.ttf}[Path=/home/owen/progs/firedrake-complex/firedrake/lib/python3.5/site-packages/matplotlib/mpl-data/fonts/ttf/]
%%
\begingroup%
\makeatletter%
\begin{pgfpicture}%
\pgfpathrectangle{\pgfpointorigin}{\pgfqpoint{6.400000in}{4.800000in}}%
\pgfusepath{use as bounding box, clip}%
\begin{pgfscope}%
\pgfsetbuttcap%
\pgfsetmiterjoin%
\definecolor{currentfill}{rgb}{1.000000,1.000000,1.000000}%
\pgfsetfillcolor{currentfill}%
\pgfsetlinewidth{0.000000pt}%
\definecolor{currentstroke}{rgb}{1.000000,1.000000,1.000000}%
\pgfsetstrokecolor{currentstroke}%
\pgfsetdash{}{0pt}%
\pgfpathmoveto{\pgfqpoint{0.000000in}{0.000000in}}%
\pgfpathlineto{\pgfqpoint{6.400000in}{0.000000in}}%
\pgfpathlineto{\pgfqpoint{6.400000in}{4.800000in}}%
\pgfpathlineto{\pgfqpoint{0.000000in}{4.800000in}}%
\pgfpathclose%
\pgfusepath{fill}%
\end{pgfscope}%
\begin{pgfscope}%
\pgfsetbuttcap%
\pgfsetmiterjoin%
\definecolor{currentfill}{rgb}{1.000000,1.000000,1.000000}%
\pgfsetfillcolor{currentfill}%
\pgfsetlinewidth{0.000000pt}%
\definecolor{currentstroke}{rgb}{0.000000,0.000000,0.000000}%
\pgfsetstrokecolor{currentstroke}%
\pgfsetstrokeopacity{0.000000}%
\pgfsetdash{}{0pt}%
\pgfpathmoveto{\pgfqpoint{0.800000in}{0.528000in}}%
\pgfpathlineto{\pgfqpoint{5.760000in}{0.528000in}}%
\pgfpathlineto{\pgfqpoint{5.760000in}{4.224000in}}%
\pgfpathlineto{\pgfqpoint{0.800000in}{4.224000in}}%
\pgfpathclose%
\pgfusepath{fill}%
\end{pgfscope}%
\begin{pgfscope}%
\pgfsetbuttcap%
\pgfsetroundjoin%
\definecolor{currentfill}{rgb}{0.000000,0.000000,0.000000}%
\pgfsetfillcolor{currentfill}%
\pgfsetlinewidth{0.803000pt}%
\definecolor{currentstroke}{rgb}{0.000000,0.000000,0.000000}%
\pgfsetstrokecolor{currentstroke}%
\pgfsetdash{}{0pt}%
\pgfsys@defobject{currentmarker}{\pgfqpoint{0.000000in}{-0.048611in}}{\pgfqpoint{0.000000in}{0.000000in}}{%
\pgfpathmoveto{\pgfqpoint{0.000000in}{0.000000in}}%
\pgfpathlineto{\pgfqpoint{0.000000in}{-0.048611in}}%
\pgfusepath{stroke,fill}%
}%
\begin{pgfscope}%
\pgfsys@transformshift{1.250909in}{0.528000in}%
\pgfsys@useobject{currentmarker}{}%
\end{pgfscope}%
\end{pgfscope}%
\begin{pgfscope}%
\definecolor{textcolor}{rgb}{0.000000,0.000000,0.000000}%
\pgfsetstrokecolor{textcolor}%
\pgfsetfillcolor{textcolor}%
\pgftext[x=1.250909in,y=0.430778in,,top]{\color{textcolor}\sffamily\fontsize{10.000000}{12.000000}\selectfont 10}%
\end{pgfscope}%
\begin{pgfscope}%
\pgfsetbuttcap%
\pgfsetroundjoin%
\definecolor{currentfill}{rgb}{0.000000,0.000000,0.000000}%
\pgfsetfillcolor{currentfill}%
\pgfsetlinewidth{0.803000pt}%
\definecolor{currentstroke}{rgb}{0.000000,0.000000,0.000000}%
\pgfsetstrokecolor{currentstroke}%
\pgfsetdash{}{0pt}%
\pgfsys@defobject{currentmarker}{\pgfqpoint{0.000000in}{-0.048611in}}{\pgfqpoint{0.000000in}{0.000000in}}{%
\pgfpathmoveto{\pgfqpoint{0.000000in}{0.000000in}}%
\pgfpathlineto{\pgfqpoint{0.000000in}{-0.048611in}}%
\pgfusepath{stroke,fill}%
}%
\begin{pgfscope}%
\pgfsys@transformshift{1.701818in}{0.528000in}%
\pgfsys@useobject{currentmarker}{}%
\end{pgfscope}%
\end{pgfscope}%
\begin{pgfscope}%
\definecolor{textcolor}{rgb}{0.000000,0.000000,0.000000}%
\pgfsetstrokecolor{textcolor}%
\pgfsetfillcolor{textcolor}%
\pgftext[x=1.701818in,y=0.430778in,,top]{\color{textcolor}\sffamily\fontsize{10.000000}{12.000000}\selectfont 20}%
\end{pgfscope}%
\begin{pgfscope}%
\pgfsetbuttcap%
\pgfsetroundjoin%
\definecolor{currentfill}{rgb}{0.000000,0.000000,0.000000}%
\pgfsetfillcolor{currentfill}%
\pgfsetlinewidth{0.803000pt}%
\definecolor{currentstroke}{rgb}{0.000000,0.000000,0.000000}%
\pgfsetstrokecolor{currentstroke}%
\pgfsetdash{}{0pt}%
\pgfsys@defobject{currentmarker}{\pgfqpoint{0.000000in}{-0.048611in}}{\pgfqpoint{0.000000in}{0.000000in}}{%
\pgfpathmoveto{\pgfqpoint{0.000000in}{0.000000in}}%
\pgfpathlineto{\pgfqpoint{0.000000in}{-0.048611in}}%
\pgfusepath{stroke,fill}%
}%
\begin{pgfscope}%
\pgfsys@transformshift{2.152727in}{0.528000in}%
\pgfsys@useobject{currentmarker}{}%
\end{pgfscope}%
\end{pgfscope}%
\begin{pgfscope}%
\definecolor{textcolor}{rgb}{0.000000,0.000000,0.000000}%
\pgfsetstrokecolor{textcolor}%
\pgfsetfillcolor{textcolor}%
\pgftext[x=2.152727in,y=0.430778in,,top]{\color{textcolor}\sffamily\fontsize{10.000000}{12.000000}\selectfont 30}%
\end{pgfscope}%
\begin{pgfscope}%
\pgfsetbuttcap%
\pgfsetroundjoin%
\definecolor{currentfill}{rgb}{0.000000,0.000000,0.000000}%
\pgfsetfillcolor{currentfill}%
\pgfsetlinewidth{0.803000pt}%
\definecolor{currentstroke}{rgb}{0.000000,0.000000,0.000000}%
\pgfsetstrokecolor{currentstroke}%
\pgfsetdash{}{0pt}%
\pgfsys@defobject{currentmarker}{\pgfqpoint{0.000000in}{-0.048611in}}{\pgfqpoint{0.000000in}{0.000000in}}{%
\pgfpathmoveto{\pgfqpoint{0.000000in}{0.000000in}}%
\pgfpathlineto{\pgfqpoint{0.000000in}{-0.048611in}}%
\pgfusepath{stroke,fill}%
}%
\begin{pgfscope}%
\pgfsys@transformshift{2.603636in}{0.528000in}%
\pgfsys@useobject{currentmarker}{}%
\end{pgfscope}%
\end{pgfscope}%
\begin{pgfscope}%
\definecolor{textcolor}{rgb}{0.000000,0.000000,0.000000}%
\pgfsetstrokecolor{textcolor}%
\pgfsetfillcolor{textcolor}%
\pgftext[x=2.603636in,y=0.430778in,,top]{\color{textcolor}\sffamily\fontsize{10.000000}{12.000000}\selectfont 40}%
\end{pgfscope}%
\begin{pgfscope}%
\pgfsetbuttcap%
\pgfsetroundjoin%
\definecolor{currentfill}{rgb}{0.000000,0.000000,0.000000}%
\pgfsetfillcolor{currentfill}%
\pgfsetlinewidth{0.803000pt}%
\definecolor{currentstroke}{rgb}{0.000000,0.000000,0.000000}%
\pgfsetstrokecolor{currentstroke}%
\pgfsetdash{}{0pt}%
\pgfsys@defobject{currentmarker}{\pgfqpoint{0.000000in}{-0.048611in}}{\pgfqpoint{0.000000in}{0.000000in}}{%
\pgfpathmoveto{\pgfqpoint{0.000000in}{0.000000in}}%
\pgfpathlineto{\pgfqpoint{0.000000in}{-0.048611in}}%
\pgfusepath{stroke,fill}%
}%
\begin{pgfscope}%
\pgfsys@transformshift{3.054545in}{0.528000in}%
\pgfsys@useobject{currentmarker}{}%
\end{pgfscope}%
\end{pgfscope}%
\begin{pgfscope}%
\definecolor{textcolor}{rgb}{0.000000,0.000000,0.000000}%
\pgfsetstrokecolor{textcolor}%
\pgfsetfillcolor{textcolor}%
\pgftext[x=3.054545in,y=0.430778in,,top]{\color{textcolor}\sffamily\fontsize{10.000000}{12.000000}\selectfont 50}%
\end{pgfscope}%
\begin{pgfscope}%
\pgfsetbuttcap%
\pgfsetroundjoin%
\definecolor{currentfill}{rgb}{0.000000,0.000000,0.000000}%
\pgfsetfillcolor{currentfill}%
\pgfsetlinewidth{0.803000pt}%
\definecolor{currentstroke}{rgb}{0.000000,0.000000,0.000000}%
\pgfsetstrokecolor{currentstroke}%
\pgfsetdash{}{0pt}%
\pgfsys@defobject{currentmarker}{\pgfqpoint{0.000000in}{-0.048611in}}{\pgfqpoint{0.000000in}{0.000000in}}{%
\pgfpathmoveto{\pgfqpoint{0.000000in}{0.000000in}}%
\pgfpathlineto{\pgfqpoint{0.000000in}{-0.048611in}}%
\pgfusepath{stroke,fill}%
}%
\begin{pgfscope}%
\pgfsys@transformshift{3.505455in}{0.528000in}%
\pgfsys@useobject{currentmarker}{}%
\end{pgfscope}%
\end{pgfscope}%
\begin{pgfscope}%
\definecolor{textcolor}{rgb}{0.000000,0.000000,0.000000}%
\pgfsetstrokecolor{textcolor}%
\pgfsetfillcolor{textcolor}%
\pgftext[x=3.505455in,y=0.430778in,,top]{\color{textcolor}\sffamily\fontsize{10.000000}{12.000000}\selectfont 60}%
\end{pgfscope}%
\begin{pgfscope}%
\pgfsetbuttcap%
\pgfsetroundjoin%
\definecolor{currentfill}{rgb}{0.000000,0.000000,0.000000}%
\pgfsetfillcolor{currentfill}%
\pgfsetlinewidth{0.803000pt}%
\definecolor{currentstroke}{rgb}{0.000000,0.000000,0.000000}%
\pgfsetstrokecolor{currentstroke}%
\pgfsetdash{}{0pt}%
\pgfsys@defobject{currentmarker}{\pgfqpoint{0.000000in}{-0.048611in}}{\pgfqpoint{0.000000in}{0.000000in}}{%
\pgfpathmoveto{\pgfqpoint{0.000000in}{0.000000in}}%
\pgfpathlineto{\pgfqpoint{0.000000in}{-0.048611in}}%
\pgfusepath{stroke,fill}%
}%
\begin{pgfscope}%
\pgfsys@transformshift{3.956364in}{0.528000in}%
\pgfsys@useobject{currentmarker}{}%
\end{pgfscope}%
\end{pgfscope}%
\begin{pgfscope}%
\definecolor{textcolor}{rgb}{0.000000,0.000000,0.000000}%
\pgfsetstrokecolor{textcolor}%
\pgfsetfillcolor{textcolor}%
\pgftext[x=3.956364in,y=0.430778in,,top]{\color{textcolor}\sffamily\fontsize{10.000000}{12.000000}\selectfont 70}%
\end{pgfscope}%
\begin{pgfscope}%
\pgfsetbuttcap%
\pgfsetroundjoin%
\definecolor{currentfill}{rgb}{0.000000,0.000000,0.000000}%
\pgfsetfillcolor{currentfill}%
\pgfsetlinewidth{0.803000pt}%
\definecolor{currentstroke}{rgb}{0.000000,0.000000,0.000000}%
\pgfsetstrokecolor{currentstroke}%
\pgfsetdash{}{0pt}%
\pgfsys@defobject{currentmarker}{\pgfqpoint{0.000000in}{-0.048611in}}{\pgfqpoint{0.000000in}{0.000000in}}{%
\pgfpathmoveto{\pgfqpoint{0.000000in}{0.000000in}}%
\pgfpathlineto{\pgfqpoint{0.000000in}{-0.048611in}}%
\pgfusepath{stroke,fill}%
}%
\begin{pgfscope}%
\pgfsys@transformshift{4.407273in}{0.528000in}%
\pgfsys@useobject{currentmarker}{}%
\end{pgfscope}%
\end{pgfscope}%
\begin{pgfscope}%
\definecolor{textcolor}{rgb}{0.000000,0.000000,0.000000}%
\pgfsetstrokecolor{textcolor}%
\pgfsetfillcolor{textcolor}%
\pgftext[x=4.407273in,y=0.430778in,,top]{\color{textcolor}\sffamily\fontsize{10.000000}{12.000000}\selectfont 80}%
\end{pgfscope}%
\begin{pgfscope}%
\pgfsetbuttcap%
\pgfsetroundjoin%
\definecolor{currentfill}{rgb}{0.000000,0.000000,0.000000}%
\pgfsetfillcolor{currentfill}%
\pgfsetlinewidth{0.803000pt}%
\definecolor{currentstroke}{rgb}{0.000000,0.000000,0.000000}%
\pgfsetstrokecolor{currentstroke}%
\pgfsetdash{}{0pt}%
\pgfsys@defobject{currentmarker}{\pgfqpoint{0.000000in}{-0.048611in}}{\pgfqpoint{0.000000in}{0.000000in}}{%
\pgfpathmoveto{\pgfqpoint{0.000000in}{0.000000in}}%
\pgfpathlineto{\pgfqpoint{0.000000in}{-0.048611in}}%
\pgfusepath{stroke,fill}%
}%
\begin{pgfscope}%
\pgfsys@transformshift{4.858182in}{0.528000in}%
\pgfsys@useobject{currentmarker}{}%
\end{pgfscope}%
\end{pgfscope}%
\begin{pgfscope}%
\definecolor{textcolor}{rgb}{0.000000,0.000000,0.000000}%
\pgfsetstrokecolor{textcolor}%
\pgfsetfillcolor{textcolor}%
\pgftext[x=4.858182in,y=0.430778in,,top]{\color{textcolor}\sffamily\fontsize{10.000000}{12.000000}\selectfont 90}%
\end{pgfscope}%
\begin{pgfscope}%
\pgfsetbuttcap%
\pgfsetroundjoin%
\definecolor{currentfill}{rgb}{0.000000,0.000000,0.000000}%
\pgfsetfillcolor{currentfill}%
\pgfsetlinewidth{0.803000pt}%
\definecolor{currentstroke}{rgb}{0.000000,0.000000,0.000000}%
\pgfsetstrokecolor{currentstroke}%
\pgfsetdash{}{0pt}%
\pgfsys@defobject{currentmarker}{\pgfqpoint{0.000000in}{-0.048611in}}{\pgfqpoint{0.000000in}{0.000000in}}{%
\pgfpathmoveto{\pgfqpoint{0.000000in}{0.000000in}}%
\pgfpathlineto{\pgfqpoint{0.000000in}{-0.048611in}}%
\pgfusepath{stroke,fill}%
}%
\begin{pgfscope}%
\pgfsys@transformshift{5.309091in}{0.528000in}%
\pgfsys@useobject{currentmarker}{}%
\end{pgfscope}%
\end{pgfscope}%
\begin{pgfscope}%
\definecolor{textcolor}{rgb}{0.000000,0.000000,0.000000}%
\pgfsetstrokecolor{textcolor}%
\pgfsetfillcolor{textcolor}%
\pgftext[x=5.309091in,y=0.430778in,,top]{\color{textcolor}\sffamily\fontsize{10.000000}{12.000000}\selectfont 100}%
\end{pgfscope}%
\begin{pgfscope}%
\definecolor{textcolor}{rgb}{0.000000,0.000000,0.000000}%
\pgfsetstrokecolor{textcolor}%
\pgfsetfillcolor{textcolor}%
\pgftext[x=3.280000in,y=0.240809in,,top]{\color{textcolor}\sffamily\fontsize{10.000000}{12.000000}\selectfont \(\displaystyle k\)}%
\end{pgfscope}%
\begin{pgfscope}%
\pgfsetbuttcap%
\pgfsetroundjoin%
\definecolor{currentfill}{rgb}{0.000000,0.000000,0.000000}%
\pgfsetfillcolor{currentfill}%
\pgfsetlinewidth{0.803000pt}%
\definecolor{currentstroke}{rgb}{0.000000,0.000000,0.000000}%
\pgfsetstrokecolor{currentstroke}%
\pgfsetdash{}{0pt}%
\pgfsys@defobject{currentmarker}{\pgfqpoint{-0.048611in}{0.000000in}}{\pgfqpoint{0.000000in}{0.000000in}}{%
\pgfpathmoveto{\pgfqpoint{0.000000in}{0.000000in}}%
\pgfpathlineto{\pgfqpoint{-0.048611in}{0.000000in}}%
\pgfusepath{stroke,fill}%
}%
\begin{pgfscope}%
\pgfsys@transformshift{0.800000in}{0.678442in}%
\pgfsys@useobject{currentmarker}{}%
\end{pgfscope}%
\end{pgfscope}%
\begin{pgfscope}%
\definecolor{textcolor}{rgb}{0.000000,0.000000,0.000000}%
\pgfsetstrokecolor{textcolor}%
\pgfsetfillcolor{textcolor}%
\pgftext[x=0.614413in,y=0.625680in,left,base]{\color{textcolor}\sffamily\fontsize{10.000000}{12.000000}\selectfont 0}%
\end{pgfscope}%
\begin{pgfscope}%
\pgfsetbuttcap%
\pgfsetroundjoin%
\definecolor{currentfill}{rgb}{0.000000,0.000000,0.000000}%
\pgfsetfillcolor{currentfill}%
\pgfsetlinewidth{0.803000pt}%
\definecolor{currentstroke}{rgb}{0.000000,0.000000,0.000000}%
\pgfsetstrokecolor{currentstroke}%
\pgfsetdash{}{0pt}%
\pgfsys@defobject{currentmarker}{\pgfqpoint{-0.048611in}{0.000000in}}{\pgfqpoint{0.000000in}{0.000000in}}{%
\pgfpathmoveto{\pgfqpoint{0.000000in}{0.000000in}}%
\pgfpathlineto{\pgfqpoint{-0.048611in}{0.000000in}}%
\pgfusepath{stroke,fill}%
}%
\begin{pgfscope}%
\pgfsys@transformshift{0.800000in}{1.077492in}%
\pgfsys@useobject{currentmarker}{}%
\end{pgfscope}%
\end{pgfscope}%
\begin{pgfscope}%
\definecolor{textcolor}{rgb}{0.000000,0.000000,0.000000}%
\pgfsetstrokecolor{textcolor}%
\pgfsetfillcolor{textcolor}%
\pgftext[x=0.437682in,y=1.024730in,left,base]{\color{textcolor}\sffamily\fontsize{10.000000}{12.000000}\selectfont 250}%
\end{pgfscope}%
\begin{pgfscope}%
\pgfsetbuttcap%
\pgfsetroundjoin%
\definecolor{currentfill}{rgb}{0.000000,0.000000,0.000000}%
\pgfsetfillcolor{currentfill}%
\pgfsetlinewidth{0.803000pt}%
\definecolor{currentstroke}{rgb}{0.000000,0.000000,0.000000}%
\pgfsetstrokecolor{currentstroke}%
\pgfsetdash{}{0pt}%
\pgfsys@defobject{currentmarker}{\pgfqpoint{-0.048611in}{0.000000in}}{\pgfqpoint{0.000000in}{0.000000in}}{%
\pgfpathmoveto{\pgfqpoint{0.000000in}{0.000000in}}%
\pgfpathlineto{\pgfqpoint{-0.048611in}{0.000000in}}%
\pgfusepath{stroke,fill}%
}%
\begin{pgfscope}%
\pgfsys@transformshift{0.800000in}{1.476542in}%
\pgfsys@useobject{currentmarker}{}%
\end{pgfscope}%
\end{pgfscope}%
\begin{pgfscope}%
\definecolor{textcolor}{rgb}{0.000000,0.000000,0.000000}%
\pgfsetstrokecolor{textcolor}%
\pgfsetfillcolor{textcolor}%
\pgftext[x=0.437682in,y=1.423780in,left,base]{\color{textcolor}\sffamily\fontsize{10.000000}{12.000000}\selectfont 500}%
\end{pgfscope}%
\begin{pgfscope}%
\pgfsetbuttcap%
\pgfsetroundjoin%
\definecolor{currentfill}{rgb}{0.000000,0.000000,0.000000}%
\pgfsetfillcolor{currentfill}%
\pgfsetlinewidth{0.803000pt}%
\definecolor{currentstroke}{rgb}{0.000000,0.000000,0.000000}%
\pgfsetstrokecolor{currentstroke}%
\pgfsetdash{}{0pt}%
\pgfsys@defobject{currentmarker}{\pgfqpoint{-0.048611in}{0.000000in}}{\pgfqpoint{0.000000in}{0.000000in}}{%
\pgfpathmoveto{\pgfqpoint{0.000000in}{0.000000in}}%
\pgfpathlineto{\pgfqpoint{-0.048611in}{0.000000in}}%
\pgfusepath{stroke,fill}%
}%
\begin{pgfscope}%
\pgfsys@transformshift{0.800000in}{1.875591in}%
\pgfsys@useobject{currentmarker}{}%
\end{pgfscope}%
\end{pgfscope}%
\begin{pgfscope}%
\definecolor{textcolor}{rgb}{0.000000,0.000000,0.000000}%
\pgfsetstrokecolor{textcolor}%
\pgfsetfillcolor{textcolor}%
\pgftext[x=0.437682in,y=1.822830in,left,base]{\color{textcolor}\sffamily\fontsize{10.000000}{12.000000}\selectfont 750}%
\end{pgfscope}%
\begin{pgfscope}%
\pgfsetbuttcap%
\pgfsetroundjoin%
\definecolor{currentfill}{rgb}{0.000000,0.000000,0.000000}%
\pgfsetfillcolor{currentfill}%
\pgfsetlinewidth{0.803000pt}%
\definecolor{currentstroke}{rgb}{0.000000,0.000000,0.000000}%
\pgfsetstrokecolor{currentstroke}%
\pgfsetdash{}{0pt}%
\pgfsys@defobject{currentmarker}{\pgfqpoint{-0.048611in}{0.000000in}}{\pgfqpoint{0.000000in}{0.000000in}}{%
\pgfpathmoveto{\pgfqpoint{0.000000in}{0.000000in}}%
\pgfpathlineto{\pgfqpoint{-0.048611in}{0.000000in}}%
\pgfusepath{stroke,fill}%
}%
\begin{pgfscope}%
\pgfsys@transformshift{0.800000in}{2.274641in}%
\pgfsys@useobject{currentmarker}{}%
\end{pgfscope}%
\end{pgfscope}%
\begin{pgfscope}%
\definecolor{textcolor}{rgb}{0.000000,0.000000,0.000000}%
\pgfsetstrokecolor{textcolor}%
\pgfsetfillcolor{textcolor}%
\pgftext[x=0.349316in,y=2.221880in,left,base]{\color{textcolor}\sffamily\fontsize{10.000000}{12.000000}\selectfont 1000}%
\end{pgfscope}%
\begin{pgfscope}%
\pgfsetbuttcap%
\pgfsetroundjoin%
\definecolor{currentfill}{rgb}{0.000000,0.000000,0.000000}%
\pgfsetfillcolor{currentfill}%
\pgfsetlinewidth{0.803000pt}%
\definecolor{currentstroke}{rgb}{0.000000,0.000000,0.000000}%
\pgfsetstrokecolor{currentstroke}%
\pgfsetdash{}{0pt}%
\pgfsys@defobject{currentmarker}{\pgfqpoint{-0.048611in}{0.000000in}}{\pgfqpoint{0.000000in}{0.000000in}}{%
\pgfpathmoveto{\pgfqpoint{0.000000in}{0.000000in}}%
\pgfpathlineto{\pgfqpoint{-0.048611in}{0.000000in}}%
\pgfusepath{stroke,fill}%
}%
\begin{pgfscope}%
\pgfsys@transformshift{0.800000in}{2.673691in}%
\pgfsys@useobject{currentmarker}{}%
\end{pgfscope}%
\end{pgfscope}%
\begin{pgfscope}%
\definecolor{textcolor}{rgb}{0.000000,0.000000,0.000000}%
\pgfsetstrokecolor{textcolor}%
\pgfsetfillcolor{textcolor}%
\pgftext[x=0.349316in,y=2.620930in,left,base]{\color{textcolor}\sffamily\fontsize{10.000000}{12.000000}\selectfont 1250}%
\end{pgfscope}%
\begin{pgfscope}%
\pgfsetbuttcap%
\pgfsetroundjoin%
\definecolor{currentfill}{rgb}{0.000000,0.000000,0.000000}%
\pgfsetfillcolor{currentfill}%
\pgfsetlinewidth{0.803000pt}%
\definecolor{currentstroke}{rgb}{0.000000,0.000000,0.000000}%
\pgfsetstrokecolor{currentstroke}%
\pgfsetdash{}{0pt}%
\pgfsys@defobject{currentmarker}{\pgfqpoint{-0.048611in}{0.000000in}}{\pgfqpoint{0.000000in}{0.000000in}}{%
\pgfpathmoveto{\pgfqpoint{0.000000in}{0.000000in}}%
\pgfpathlineto{\pgfqpoint{-0.048611in}{0.000000in}}%
\pgfusepath{stroke,fill}%
}%
\begin{pgfscope}%
\pgfsys@transformshift{0.800000in}{3.072741in}%
\pgfsys@useobject{currentmarker}{}%
\end{pgfscope}%
\end{pgfscope}%
\begin{pgfscope}%
\definecolor{textcolor}{rgb}{0.000000,0.000000,0.000000}%
\pgfsetstrokecolor{textcolor}%
\pgfsetfillcolor{textcolor}%
\pgftext[x=0.349316in,y=3.019980in,left,base]{\color{textcolor}\sffamily\fontsize{10.000000}{12.000000}\selectfont 1500}%
\end{pgfscope}%
\begin{pgfscope}%
\pgfsetbuttcap%
\pgfsetroundjoin%
\definecolor{currentfill}{rgb}{0.000000,0.000000,0.000000}%
\pgfsetfillcolor{currentfill}%
\pgfsetlinewidth{0.803000pt}%
\definecolor{currentstroke}{rgb}{0.000000,0.000000,0.000000}%
\pgfsetstrokecolor{currentstroke}%
\pgfsetdash{}{0pt}%
\pgfsys@defobject{currentmarker}{\pgfqpoint{-0.048611in}{0.000000in}}{\pgfqpoint{0.000000in}{0.000000in}}{%
\pgfpathmoveto{\pgfqpoint{0.000000in}{0.000000in}}%
\pgfpathlineto{\pgfqpoint{-0.048611in}{0.000000in}}%
\pgfusepath{stroke,fill}%
}%
\begin{pgfscope}%
\pgfsys@transformshift{0.800000in}{3.471791in}%
\pgfsys@useobject{currentmarker}{}%
\end{pgfscope}%
\end{pgfscope}%
\begin{pgfscope}%
\definecolor{textcolor}{rgb}{0.000000,0.000000,0.000000}%
\pgfsetstrokecolor{textcolor}%
\pgfsetfillcolor{textcolor}%
\pgftext[x=0.349316in,y=3.419029in,left,base]{\color{textcolor}\sffamily\fontsize{10.000000}{12.000000}\selectfont 1750}%
\end{pgfscope}%
\begin{pgfscope}%
\pgfsetbuttcap%
\pgfsetroundjoin%
\definecolor{currentfill}{rgb}{0.000000,0.000000,0.000000}%
\pgfsetfillcolor{currentfill}%
\pgfsetlinewidth{0.803000pt}%
\definecolor{currentstroke}{rgb}{0.000000,0.000000,0.000000}%
\pgfsetstrokecolor{currentstroke}%
\pgfsetdash{}{0pt}%
\pgfsys@defobject{currentmarker}{\pgfqpoint{-0.048611in}{0.000000in}}{\pgfqpoint{0.000000in}{0.000000in}}{%
\pgfpathmoveto{\pgfqpoint{0.000000in}{0.000000in}}%
\pgfpathlineto{\pgfqpoint{-0.048611in}{0.000000in}}%
\pgfusepath{stroke,fill}%
}%
\begin{pgfscope}%
\pgfsys@transformshift{0.800000in}{3.870841in}%
\pgfsys@useobject{currentmarker}{}%
\end{pgfscope}%
\end{pgfscope}%
\begin{pgfscope}%
\definecolor{textcolor}{rgb}{0.000000,0.000000,0.000000}%
\pgfsetstrokecolor{textcolor}%
\pgfsetfillcolor{textcolor}%
\pgftext[x=0.349316in,y=3.818079in,left,base]{\color{textcolor}\sffamily\fontsize{10.000000}{12.000000}\selectfont 2000}%
\end{pgfscope}%
\begin{pgfscope}%
\definecolor{textcolor}{rgb}{0.000000,0.000000,0.000000}%
\pgfsetstrokecolor{textcolor}%
\pgfsetfillcolor{textcolor}%
\pgftext[x=0.293761in,y=2.376000in,,bottom,rotate=90.000000]{\color{textcolor}\sffamily\fontsize{10.000000}{12.000000}\selectfont Number of GMRES iterations}%
\end{pgfscope}%
\begin{pgfscope}%
\pgfpathrectangle{\pgfqpoint{0.800000in}{0.528000in}}{\pgfqpoint{4.960000in}{3.696000in}}%
\pgfusepath{clip}%
\pgfsetbuttcap%
\pgfsetroundjoin%
\pgfsetlinewidth{1.505625pt}%
\definecolor{currentstroke}{rgb}{0.000000,0.000000,0.000000}%
\pgfsetstrokecolor{currentstroke}%
\pgfsetdash{{5.550000pt}{2.400000pt}}{0.000000pt}%
\pgfpathmoveto{\pgfqpoint{1.250909in}{0.700789in}}%
\pgfpathlineto{\pgfqpoint{1.701818in}{0.742290in}}%
\pgfpathlineto{\pgfqpoint{2.152727in}{0.868390in}}%
\pgfpathlineto{\pgfqpoint{2.603636in}{1.090261in}}%
\pgfpathlineto{\pgfqpoint{3.054545in}{1.360019in}}%
\pgfpathlineto{\pgfqpoint{3.505455in}{1.679259in}}%
\pgfpathlineto{\pgfqpoint{3.956364in}{2.178869in}}%
\pgfpathlineto{\pgfqpoint{4.407273in}{2.712000in}}%
\pgfpathlineto{\pgfqpoint{4.858182in}{3.384000in}}%
\pgfpathlineto{\pgfqpoint{5.309091in}{4.056000in}}%
\pgfusepath{stroke}%
\end{pgfscope}%
\begin{pgfscope}%
\pgfpathrectangle{\pgfqpoint{0.800000in}{0.528000in}}{\pgfqpoint{4.960000in}{3.696000in}}%
\pgfusepath{clip}%
\pgfsetbuttcap%
\pgfsetroundjoin%
\definecolor{currentfill}{rgb}{0.000000,0.000000,0.000000}%
\pgfsetfillcolor{currentfill}%
\pgfsetlinewidth{1.003750pt}%
\definecolor{currentstroke}{rgb}{0.000000,0.000000,0.000000}%
\pgfsetstrokecolor{currentstroke}%
\pgfsetdash{}{0pt}%
\pgfsys@defobject{currentmarker}{\pgfqpoint{-0.041667in}{-0.041667in}}{\pgfqpoint{0.041667in}{0.041667in}}{%
\pgfpathmoveto{\pgfqpoint{0.000000in}{-0.041667in}}%
\pgfpathcurveto{\pgfqpoint{0.011050in}{-0.041667in}}{\pgfqpoint{0.021649in}{-0.037276in}}{\pgfqpoint{0.029463in}{-0.029463in}}%
\pgfpathcurveto{\pgfqpoint{0.037276in}{-0.021649in}}{\pgfqpoint{0.041667in}{-0.011050in}}{\pgfqpoint{0.041667in}{0.000000in}}%
\pgfpathcurveto{\pgfqpoint{0.041667in}{0.011050in}}{\pgfqpoint{0.037276in}{0.021649in}}{\pgfqpoint{0.029463in}{0.029463in}}%
\pgfpathcurveto{\pgfqpoint{0.021649in}{0.037276in}}{\pgfqpoint{0.011050in}{0.041667in}}{\pgfqpoint{0.000000in}{0.041667in}}%
\pgfpathcurveto{\pgfqpoint{-0.011050in}{0.041667in}}{\pgfqpoint{-0.021649in}{0.037276in}}{\pgfqpoint{-0.029463in}{0.029463in}}%
\pgfpathcurveto{\pgfqpoint{-0.037276in}{0.021649in}}{\pgfqpoint{-0.041667in}{0.011050in}}{\pgfqpoint{-0.041667in}{0.000000in}}%
\pgfpathcurveto{\pgfqpoint{-0.041667in}{-0.011050in}}{\pgfqpoint{-0.037276in}{-0.021649in}}{\pgfqpoint{-0.029463in}{-0.029463in}}%
\pgfpathcurveto{\pgfqpoint{-0.021649in}{-0.037276in}}{\pgfqpoint{-0.011050in}{-0.041667in}}{\pgfqpoint{0.000000in}{-0.041667in}}%
\pgfpathclose%
\pgfusepath{stroke,fill}%
}%
\begin{pgfscope}%
\pgfsys@transformshift{1.250909in}{0.700789in}%
\pgfsys@useobject{currentmarker}{}%
\end{pgfscope}%
\begin{pgfscope}%
\pgfsys@transformshift{1.701818in}{0.742290in}%
\pgfsys@useobject{currentmarker}{}%
\end{pgfscope}%
\begin{pgfscope}%
\pgfsys@transformshift{2.152727in}{0.868390in}%
\pgfsys@useobject{currentmarker}{}%
\end{pgfscope}%
\begin{pgfscope}%
\pgfsys@transformshift{2.603636in}{1.090261in}%
\pgfsys@useobject{currentmarker}{}%
\end{pgfscope}%
\begin{pgfscope}%
\pgfsys@transformshift{3.054545in}{1.360019in}%
\pgfsys@useobject{currentmarker}{}%
\end{pgfscope}%
\begin{pgfscope}%
\pgfsys@transformshift{3.505455in}{1.679259in}%
\pgfsys@useobject{currentmarker}{}%
\end{pgfscope}%
\begin{pgfscope}%
\pgfsys@transformshift{3.956364in}{2.178869in}%
\pgfsys@useobject{currentmarker}{}%
\end{pgfscope}%
\begin{pgfscope}%
\pgfsys@transformshift{4.407273in}{2.712000in}%
\pgfsys@useobject{currentmarker}{}%
\end{pgfscope}%
\begin{pgfscope}%
\pgfsys@transformshift{4.858182in}{3.384000in}%
\pgfsys@useobject{currentmarker}{}%
\end{pgfscope}%
\begin{pgfscope}%
\pgfsys@transformshift{5.309091in}{4.056000in}%
\pgfsys@useobject{currentmarker}{}%
\end{pgfscope}%
\end{pgfscope}%
\begin{pgfscope}%
\pgfpathrectangle{\pgfqpoint{0.800000in}{0.528000in}}{\pgfqpoint{4.960000in}{3.696000in}}%
\pgfusepath{clip}%
\pgfsetbuttcap%
\pgfsetroundjoin%
\pgfsetlinewidth{1.505625pt}%
\definecolor{currentstroke}{rgb}{0.000000,0.000000,0.000000}%
\pgfsetstrokecolor{currentstroke}%
\pgfsetdash{{5.550000pt}{2.400000pt}}{0.000000pt}%
\pgfpathmoveto{\pgfqpoint{1.250909in}{0.699192in}}%
\pgfpathlineto{\pgfqpoint{1.701818in}{0.721539in}}%
\pgfpathlineto{\pgfqpoint{2.152727in}{0.790176in}}%
\pgfpathlineto{\pgfqpoint{2.603636in}{0.913083in}}%
\pgfpathlineto{\pgfqpoint{3.054545in}{1.096646in}}%
\pgfpathlineto{\pgfqpoint{3.505455in}{1.307344in}}%
\pgfpathlineto{\pgfqpoint{3.956364in}{1.620200in}}%
\pgfpathlineto{\pgfqpoint{4.407273in}{1.995306in}}%
\pgfpathlineto{\pgfqpoint{4.858182in}{2.478955in}}%
\pgfpathlineto{\pgfqpoint{5.309091in}{2.901948in}}%
\pgfusepath{stroke}%
\end{pgfscope}%
\begin{pgfscope}%
\pgfpathrectangle{\pgfqpoint{0.800000in}{0.528000in}}{\pgfqpoint{4.960000in}{3.696000in}}%
\pgfusepath{clip}%
\pgfsetbuttcap%
\pgfsetmiterjoin%
\definecolor{currentfill}{rgb}{0.000000,0.000000,0.000000}%
\pgfsetfillcolor{currentfill}%
\pgfsetlinewidth{1.003750pt}%
\definecolor{currentstroke}{rgb}{0.000000,0.000000,0.000000}%
\pgfsetstrokecolor{currentstroke}%
\pgfsetdash{}{0pt}%
\pgfsys@defobject{currentmarker}{\pgfqpoint{-0.041667in}{-0.041667in}}{\pgfqpoint{0.041667in}{0.041667in}}{%
\pgfpathmoveto{\pgfqpoint{-0.000000in}{-0.041667in}}%
\pgfpathlineto{\pgfqpoint{0.041667in}{0.041667in}}%
\pgfpathlineto{\pgfqpoint{-0.041667in}{0.041667in}}%
\pgfpathclose%
\pgfusepath{stroke,fill}%
}%
\begin{pgfscope}%
\pgfsys@transformshift{1.250909in}{0.699192in}%
\pgfsys@useobject{currentmarker}{}%
\end{pgfscope}%
\begin{pgfscope}%
\pgfsys@transformshift{1.701818in}{0.721539in}%
\pgfsys@useobject{currentmarker}{}%
\end{pgfscope}%
\begin{pgfscope}%
\pgfsys@transformshift{2.152727in}{0.790176in}%
\pgfsys@useobject{currentmarker}{}%
\end{pgfscope}%
\begin{pgfscope}%
\pgfsys@transformshift{2.603636in}{0.913083in}%
\pgfsys@useobject{currentmarker}{}%
\end{pgfscope}%
\begin{pgfscope}%
\pgfsys@transformshift{3.054545in}{1.096646in}%
\pgfsys@useobject{currentmarker}{}%
\end{pgfscope}%
\begin{pgfscope}%
\pgfsys@transformshift{3.505455in}{1.307344in}%
\pgfsys@useobject{currentmarker}{}%
\end{pgfscope}%
\begin{pgfscope}%
\pgfsys@transformshift{3.956364in}{1.620200in}%
\pgfsys@useobject{currentmarker}{}%
\end{pgfscope}%
\begin{pgfscope}%
\pgfsys@transformshift{4.407273in}{1.995306in}%
\pgfsys@useobject{currentmarker}{}%
\end{pgfscope}%
\begin{pgfscope}%
\pgfsys@transformshift{4.858182in}{2.478955in}%
\pgfsys@useobject{currentmarker}{}%
\end{pgfscope}%
\begin{pgfscope}%
\pgfsys@transformshift{5.309091in}{2.901948in}%
\pgfsys@useobject{currentmarker}{}%
\end{pgfscope}%
\end{pgfscope}%
\begin{pgfscope}%
\pgfpathrectangle{\pgfqpoint{0.800000in}{0.528000in}}{\pgfqpoint{4.960000in}{3.696000in}}%
\pgfusepath{clip}%
\pgfsetbuttcap%
\pgfsetroundjoin%
\pgfsetlinewidth{1.505625pt}%
\definecolor{currentstroke}{rgb}{0.000000,0.000000,0.000000}%
\pgfsetstrokecolor{currentstroke}%
\pgfsetdash{{5.550000pt}{2.400000pt}}{0.000000pt}%
\pgfpathmoveto{\pgfqpoint{1.250909in}{0.697596in}}%
\pgfpathlineto{\pgfqpoint{1.701818in}{0.713558in}}%
\pgfpathlineto{\pgfqpoint{2.152727in}{0.742290in}}%
\pgfpathlineto{\pgfqpoint{2.603636in}{0.801349in}}%
\pgfpathlineto{\pgfqpoint{3.054545in}{0.892333in}}%
\pgfpathlineto{\pgfqpoint{3.505455in}{0.996086in}}%
\pgfpathlineto{\pgfqpoint{3.956364in}{1.144532in}}%
\pgfpathlineto{\pgfqpoint{4.407273in}{1.347249in}}%
\pgfpathlineto{\pgfqpoint{4.858182in}{1.557948in}}%
\pgfpathlineto{\pgfqpoint{5.309091in}{1.837283in}}%
\pgfusepath{stroke}%
\end{pgfscope}%
\begin{pgfscope}%
\pgfpathrectangle{\pgfqpoint{0.800000in}{0.528000in}}{\pgfqpoint{4.960000in}{3.696000in}}%
\pgfusepath{clip}%
\pgfsetbuttcap%
\pgfsetmiterjoin%
\definecolor{currentfill}{rgb}{0.000000,0.000000,0.000000}%
\pgfsetfillcolor{currentfill}%
\pgfsetlinewidth{1.003750pt}%
\definecolor{currentstroke}{rgb}{0.000000,0.000000,0.000000}%
\pgfsetstrokecolor{currentstroke}%
\pgfsetdash{}{0pt}%
\pgfsys@defobject{currentmarker}{\pgfqpoint{-0.041667in}{-0.041667in}}{\pgfqpoint{0.041667in}{0.041667in}}{%
\pgfpathmoveto{\pgfqpoint{-0.041667in}{-0.041667in}}%
\pgfpathlineto{\pgfqpoint{0.041667in}{-0.041667in}}%
\pgfpathlineto{\pgfqpoint{0.041667in}{0.041667in}}%
\pgfpathlineto{\pgfqpoint{-0.041667in}{0.041667in}}%
\pgfpathclose%
\pgfusepath{stroke,fill}%
}%
\begin{pgfscope}%
\pgfsys@transformshift{1.250909in}{0.697596in}%
\pgfsys@useobject{currentmarker}{}%
\end{pgfscope}%
\begin{pgfscope}%
\pgfsys@transformshift{1.701818in}{0.713558in}%
\pgfsys@useobject{currentmarker}{}%
\end{pgfscope}%
\begin{pgfscope}%
\pgfsys@transformshift{2.152727in}{0.742290in}%
\pgfsys@useobject{currentmarker}{}%
\end{pgfscope}%
\begin{pgfscope}%
\pgfsys@transformshift{2.603636in}{0.801349in}%
\pgfsys@useobject{currentmarker}{}%
\end{pgfscope}%
\begin{pgfscope}%
\pgfsys@transformshift{3.054545in}{0.892333in}%
\pgfsys@useobject{currentmarker}{}%
\end{pgfscope}%
\begin{pgfscope}%
\pgfsys@transformshift{3.505455in}{0.996086in}%
\pgfsys@useobject{currentmarker}{}%
\end{pgfscope}%
\begin{pgfscope}%
\pgfsys@transformshift{3.956364in}{1.144532in}%
\pgfsys@useobject{currentmarker}{}%
\end{pgfscope}%
\begin{pgfscope}%
\pgfsys@transformshift{4.407273in}{1.347249in}%
\pgfsys@useobject{currentmarker}{}%
\end{pgfscope}%
\begin{pgfscope}%
\pgfsys@transformshift{4.858182in}{1.557948in}%
\pgfsys@useobject{currentmarker}{}%
\end{pgfscope}%
\begin{pgfscope}%
\pgfsys@transformshift{5.309091in}{1.837283in}%
\pgfsys@useobject{currentmarker}{}%
\end{pgfscope}%
\end{pgfscope}%
\begin{pgfscope}%
\pgfpathrectangle{\pgfqpoint{0.800000in}{0.528000in}}{\pgfqpoint{4.960000in}{3.696000in}}%
\pgfusepath{clip}%
\pgfsetbuttcap%
\pgfsetroundjoin%
\pgfsetlinewidth{1.505625pt}%
\definecolor{currentstroke}{rgb}{0.000000,0.000000,0.000000}%
\pgfsetstrokecolor{currentstroke}%
\pgfsetdash{{5.550000pt}{2.400000pt}}{0.000000pt}%
\pgfpathmoveto{\pgfqpoint{1.250909in}{0.696000in}}%
\pgfpathlineto{\pgfqpoint{1.701818in}{0.707173in}}%
\pgfpathlineto{\pgfqpoint{2.152727in}{0.718347in}}%
\pgfpathlineto{\pgfqpoint{2.603636in}{0.742290in}}%
\pgfpathlineto{\pgfqpoint{3.054545in}{0.771021in}}%
\pgfpathlineto{\pgfqpoint{3.505455in}{0.815715in}}%
\pgfpathlineto{\pgfqpoint{3.956364in}{0.868390in}}%
\pgfpathlineto{\pgfqpoint{4.407273in}{0.938622in}}%
\pgfpathlineto{\pgfqpoint{4.858182in}{1.012048in}}%
\pgfpathlineto{\pgfqpoint{5.309091in}{1.109416in}}%
\pgfusepath{stroke}%
\end{pgfscope}%
\begin{pgfscope}%
\pgfpathrectangle{\pgfqpoint{0.800000in}{0.528000in}}{\pgfqpoint{4.960000in}{3.696000in}}%
\pgfusepath{clip}%
\pgfsetbuttcap%
\pgfsetmiterjoin%
\definecolor{currentfill}{rgb}{0.000000,0.000000,0.000000}%
\pgfsetfillcolor{currentfill}%
\pgfsetlinewidth{1.003750pt}%
\definecolor{currentstroke}{rgb}{0.000000,0.000000,0.000000}%
\pgfsetstrokecolor{currentstroke}%
\pgfsetdash{}{0pt}%
\pgfsys@defobject{currentmarker}{\pgfqpoint{-0.035355in}{-0.058926in}}{\pgfqpoint{0.035355in}{0.058926in}}{%
\pgfpathmoveto{\pgfqpoint{-0.000000in}{-0.058926in}}%
\pgfpathlineto{\pgfqpoint{0.035355in}{0.000000in}}%
\pgfpathlineto{\pgfqpoint{0.000000in}{0.058926in}}%
\pgfpathlineto{\pgfqpoint{-0.035355in}{0.000000in}}%
\pgfpathclose%
\pgfusepath{stroke,fill}%
}%
\begin{pgfscope}%
\pgfsys@transformshift{1.250909in}{0.696000in}%
\pgfsys@useobject{currentmarker}{}%
\end{pgfscope}%
\begin{pgfscope}%
\pgfsys@transformshift{1.701818in}{0.707173in}%
\pgfsys@useobject{currentmarker}{}%
\end{pgfscope}%
\begin{pgfscope}%
\pgfsys@transformshift{2.152727in}{0.718347in}%
\pgfsys@useobject{currentmarker}{}%
\end{pgfscope}%
\begin{pgfscope}%
\pgfsys@transformshift{2.603636in}{0.742290in}%
\pgfsys@useobject{currentmarker}{}%
\end{pgfscope}%
\begin{pgfscope}%
\pgfsys@transformshift{3.054545in}{0.771021in}%
\pgfsys@useobject{currentmarker}{}%
\end{pgfscope}%
\begin{pgfscope}%
\pgfsys@transformshift{3.505455in}{0.815715in}%
\pgfsys@useobject{currentmarker}{}%
\end{pgfscope}%
\begin{pgfscope}%
\pgfsys@transformshift{3.956364in}{0.868390in}%
\pgfsys@useobject{currentmarker}{}%
\end{pgfscope}%
\begin{pgfscope}%
\pgfsys@transformshift{4.407273in}{0.938622in}%
\pgfsys@useobject{currentmarker}{}%
\end{pgfscope}%
\begin{pgfscope}%
\pgfsys@transformshift{4.858182in}{1.012048in}%
\pgfsys@useobject{currentmarker}{}%
\end{pgfscope}%
\begin{pgfscope}%
\pgfsys@transformshift{5.309091in}{1.109416in}%
\pgfsys@useobject{currentmarker}{}%
\end{pgfscope}%
\end{pgfscope}%
\begin{pgfscope}%
\pgfsetrectcap%
\pgfsetmiterjoin%
\pgfsetlinewidth{0.803000pt}%
\definecolor{currentstroke}{rgb}{0.000000,0.000000,0.000000}%
\pgfsetstrokecolor{currentstroke}%
\pgfsetdash{}{0pt}%
\pgfpathmoveto{\pgfqpoint{0.800000in}{0.528000in}}%
\pgfpathlineto{\pgfqpoint{0.800000in}{4.224000in}}%
\pgfusepath{stroke}%
\end{pgfscope}%
\begin{pgfscope}%
\pgfsetrectcap%
\pgfsetmiterjoin%
\pgfsetlinewidth{0.803000pt}%
\definecolor{currentstroke}{rgb}{0.000000,0.000000,0.000000}%
\pgfsetstrokecolor{currentstroke}%
\pgfsetdash{}{0pt}%
\pgfpathmoveto{\pgfqpoint{5.760000in}{0.528000in}}%
\pgfpathlineto{\pgfqpoint{5.760000in}{4.224000in}}%
\pgfusepath{stroke}%
\end{pgfscope}%
\begin{pgfscope}%
\pgfsetrectcap%
\pgfsetmiterjoin%
\pgfsetlinewidth{0.803000pt}%
\definecolor{currentstroke}{rgb}{0.000000,0.000000,0.000000}%
\pgfsetstrokecolor{currentstroke}%
\pgfsetdash{}{0pt}%
\pgfpathmoveto{\pgfqpoint{0.800000in}{0.528000in}}%
\pgfpathlineto{\pgfqpoint{5.760000in}{0.528000in}}%
\pgfusepath{stroke}%
\end{pgfscope}%
\begin{pgfscope}%
\pgfsetrectcap%
\pgfsetmiterjoin%
\pgfsetlinewidth{0.803000pt}%
\definecolor{currentstroke}{rgb}{0.000000,0.000000,0.000000}%
\pgfsetstrokecolor{currentstroke}%
\pgfsetdash{}{0pt}%
\pgfpathmoveto{\pgfqpoint{0.800000in}{4.224000in}}%
\pgfpathlineto{\pgfqpoint{5.760000in}{4.224000in}}%
\pgfusepath{stroke}%
\end{pgfscope}%
\begin{pgfscope}%
\pgfsetbuttcap%
\pgfsetmiterjoin%
\definecolor{currentfill}{rgb}{1.000000,1.000000,1.000000}%
\pgfsetfillcolor{currentfill}%
\pgfsetfillopacity{0.800000}%
\pgfsetlinewidth{1.003750pt}%
\definecolor{currentstroke}{rgb}{0.800000,0.800000,0.800000}%
\pgfsetstrokecolor{currentstroke}%
\pgfsetstrokeopacity{0.800000}%
\pgfsetdash{}{0pt}%
\pgfpathmoveto{\pgfqpoint{0.897222in}{3.236530in}}%
\pgfpathlineto{\pgfqpoint{2.090461in}{3.236530in}}%
\pgfpathquadraticcurveto{\pgfqpoint{2.118239in}{3.236530in}}{\pgfqpoint{2.118239in}{3.264308in}}%
\pgfpathlineto{\pgfqpoint{2.118239in}{4.126778in}}%
\pgfpathquadraticcurveto{\pgfqpoint{2.118239in}{4.154556in}}{\pgfqpoint{2.090461in}{4.154556in}}%
\pgfpathlineto{\pgfqpoint{0.897222in}{4.154556in}}%
\pgfpathquadraticcurveto{\pgfqpoint{0.869444in}{4.154556in}}{\pgfqpoint{0.869444in}{4.126778in}}%
\pgfpathlineto{\pgfqpoint{0.869444in}{3.264308in}}%
\pgfpathquadraticcurveto{\pgfqpoint{0.869444in}{3.236530in}}{\pgfqpoint{0.897222in}{3.236530in}}%
\pgfpathclose%
\pgfusepath{stroke,fill}%
\end{pgfscope}%
\begin{pgfscope}%
\pgfsetbuttcap%
\pgfsetroundjoin%
\pgfsetlinewidth{1.505625pt}%
\definecolor{currentstroke}{rgb}{0.000000,0.000000,0.000000}%
\pgfsetstrokecolor{currentstroke}%
\pgfsetdash{{5.550000pt}{2.400000pt}}{0.000000pt}%
\pgfpathmoveto{\pgfqpoint{0.925000in}{4.042088in}}%
\pgfpathlineto{\pgfqpoint{1.202778in}{4.042088in}}%
\pgfusepath{stroke}%
\end{pgfscope}%
\begin{pgfscope}%
\pgfsetbuttcap%
\pgfsetroundjoin%
\definecolor{currentfill}{rgb}{0.000000,0.000000,0.000000}%
\pgfsetfillcolor{currentfill}%
\pgfsetlinewidth{1.003750pt}%
\definecolor{currentstroke}{rgb}{0.000000,0.000000,0.000000}%
\pgfsetstrokecolor{currentstroke}%
\pgfsetdash{}{0pt}%
\pgfsys@defobject{currentmarker}{\pgfqpoint{-0.041667in}{-0.041667in}}{\pgfqpoint{0.041667in}{0.041667in}}{%
\pgfpathmoveto{\pgfqpoint{0.000000in}{-0.041667in}}%
\pgfpathcurveto{\pgfqpoint{0.011050in}{-0.041667in}}{\pgfqpoint{0.021649in}{-0.037276in}}{\pgfqpoint{0.029463in}{-0.029463in}}%
\pgfpathcurveto{\pgfqpoint{0.037276in}{-0.021649in}}{\pgfqpoint{0.041667in}{-0.011050in}}{\pgfqpoint{0.041667in}{0.000000in}}%
\pgfpathcurveto{\pgfqpoint{0.041667in}{0.011050in}}{\pgfqpoint{0.037276in}{0.021649in}}{\pgfqpoint{0.029463in}{0.029463in}}%
\pgfpathcurveto{\pgfqpoint{0.021649in}{0.037276in}}{\pgfqpoint{0.011050in}{0.041667in}}{\pgfqpoint{0.000000in}{0.041667in}}%
\pgfpathcurveto{\pgfqpoint{-0.011050in}{0.041667in}}{\pgfqpoint{-0.021649in}{0.037276in}}{\pgfqpoint{-0.029463in}{0.029463in}}%
\pgfpathcurveto{\pgfqpoint{-0.037276in}{0.021649in}}{\pgfqpoint{-0.041667in}{0.011050in}}{\pgfqpoint{-0.041667in}{0.000000in}}%
\pgfpathcurveto{\pgfqpoint{-0.041667in}{-0.011050in}}{\pgfqpoint{-0.037276in}{-0.021649in}}{\pgfqpoint{-0.029463in}{-0.029463in}}%
\pgfpathcurveto{\pgfqpoint{-0.021649in}{-0.037276in}}{\pgfqpoint{-0.011050in}{-0.041667in}}{\pgfqpoint{0.000000in}{-0.041667in}}%
\pgfpathclose%
\pgfusepath{stroke,fill}%
}%
\begin{pgfscope}%
\pgfsys@transformshift{1.063889in}{4.042088in}%
\pgfsys@useobject{currentmarker}{}%
\end{pgfscope}%
\end{pgfscope}%
\begin{pgfscope}%
\definecolor{textcolor}{rgb}{0.000000,0.000000,0.000000}%
\pgfsetstrokecolor{textcolor}%
\pgfsetfillcolor{textcolor}%
\pgftext[x=1.313889in,y=3.993477in,left,base]{\color{textcolor}\sffamily\fontsize{10.000000}{12.000000}\selectfont \(\displaystyle \alpha = 0.2\)}%
\end{pgfscope}%
\begin{pgfscope}%
\pgfsetbuttcap%
\pgfsetroundjoin%
\pgfsetlinewidth{1.505625pt}%
\definecolor{currentstroke}{rgb}{0.000000,0.000000,0.000000}%
\pgfsetstrokecolor{currentstroke}%
\pgfsetdash{{5.550000pt}{2.400000pt}}{0.000000pt}%
\pgfpathmoveto{\pgfqpoint{0.925000in}{3.823753in}}%
\pgfpathlineto{\pgfqpoint{1.202778in}{3.823753in}}%
\pgfusepath{stroke}%
\end{pgfscope}%
\begin{pgfscope}%
\pgfsetbuttcap%
\pgfsetmiterjoin%
\definecolor{currentfill}{rgb}{0.000000,0.000000,0.000000}%
\pgfsetfillcolor{currentfill}%
\pgfsetlinewidth{1.003750pt}%
\definecolor{currentstroke}{rgb}{0.000000,0.000000,0.000000}%
\pgfsetstrokecolor{currentstroke}%
\pgfsetdash{}{0pt}%
\pgfsys@defobject{currentmarker}{\pgfqpoint{-0.041667in}{-0.041667in}}{\pgfqpoint{0.041667in}{0.041667in}}{%
\pgfpathmoveto{\pgfqpoint{-0.000000in}{-0.041667in}}%
\pgfpathlineto{\pgfqpoint{0.041667in}{0.041667in}}%
\pgfpathlineto{\pgfqpoint{-0.041667in}{0.041667in}}%
\pgfpathclose%
\pgfusepath{stroke,fill}%
}%
\begin{pgfscope}%
\pgfsys@transformshift{1.063889in}{3.823753in}%
\pgfsys@useobject{currentmarker}{}%
\end{pgfscope}%
\end{pgfscope}%
\begin{pgfscope}%
\definecolor{textcolor}{rgb}{0.000000,0.000000,0.000000}%
\pgfsetstrokecolor{textcolor}%
\pgfsetfillcolor{textcolor}%
\pgftext[x=1.313889in,y=3.775142in,left,base]{\color{textcolor}\sffamily\fontsize{10.000000}{12.000000}\selectfont \(\displaystyle \alpha = 0.2/k^{0.1}\)}%
\end{pgfscope}%
\begin{pgfscope}%
\pgfsetbuttcap%
\pgfsetroundjoin%
\pgfsetlinewidth{1.505625pt}%
\definecolor{currentstroke}{rgb}{0.000000,0.000000,0.000000}%
\pgfsetstrokecolor{currentstroke}%
\pgfsetdash{{5.550000pt}{2.400000pt}}{0.000000pt}%
\pgfpathmoveto{\pgfqpoint{0.925000in}{3.599586in}}%
\pgfpathlineto{\pgfqpoint{1.202778in}{3.599586in}}%
\pgfusepath{stroke}%
\end{pgfscope}%
\begin{pgfscope}%
\pgfsetbuttcap%
\pgfsetmiterjoin%
\definecolor{currentfill}{rgb}{0.000000,0.000000,0.000000}%
\pgfsetfillcolor{currentfill}%
\pgfsetlinewidth{1.003750pt}%
\definecolor{currentstroke}{rgb}{0.000000,0.000000,0.000000}%
\pgfsetstrokecolor{currentstroke}%
\pgfsetdash{}{0pt}%
\pgfsys@defobject{currentmarker}{\pgfqpoint{-0.041667in}{-0.041667in}}{\pgfqpoint{0.041667in}{0.041667in}}{%
\pgfpathmoveto{\pgfqpoint{-0.041667in}{-0.041667in}}%
\pgfpathlineto{\pgfqpoint{0.041667in}{-0.041667in}}%
\pgfpathlineto{\pgfqpoint{0.041667in}{0.041667in}}%
\pgfpathlineto{\pgfqpoint{-0.041667in}{0.041667in}}%
\pgfpathclose%
\pgfusepath{stroke,fill}%
}%
\begin{pgfscope}%
\pgfsys@transformshift{1.063889in}{3.599586in}%
\pgfsys@useobject{currentmarker}{}%
\end{pgfscope}%
\end{pgfscope}%
\begin{pgfscope}%
\definecolor{textcolor}{rgb}{0.000000,0.000000,0.000000}%
\pgfsetstrokecolor{textcolor}%
\pgfsetfillcolor{textcolor}%
\pgftext[x=1.313889in,y=3.550975in,left,base]{\color{textcolor}\sffamily\fontsize{10.000000}{12.000000}\selectfont \(\displaystyle \alpha = 0.2/k^{0.2}\)}%
\end{pgfscope}%
\begin{pgfscope}%
\pgfsetbuttcap%
\pgfsetroundjoin%
\pgfsetlinewidth{1.505625pt}%
\definecolor{currentstroke}{rgb}{0.000000,0.000000,0.000000}%
\pgfsetstrokecolor{currentstroke}%
\pgfsetdash{{5.550000pt}{2.400000pt}}{0.000000pt}%
\pgfpathmoveto{\pgfqpoint{0.925000in}{3.375419in}}%
\pgfpathlineto{\pgfqpoint{1.202778in}{3.375419in}}%
\pgfusepath{stroke}%
\end{pgfscope}%
\begin{pgfscope}%
\pgfsetbuttcap%
\pgfsetmiterjoin%
\definecolor{currentfill}{rgb}{0.000000,0.000000,0.000000}%
\pgfsetfillcolor{currentfill}%
\pgfsetlinewidth{1.003750pt}%
\definecolor{currentstroke}{rgb}{0.000000,0.000000,0.000000}%
\pgfsetstrokecolor{currentstroke}%
\pgfsetdash{}{0pt}%
\pgfsys@defobject{currentmarker}{\pgfqpoint{-0.035355in}{-0.058926in}}{\pgfqpoint{0.035355in}{0.058926in}}{%
\pgfpathmoveto{\pgfqpoint{-0.000000in}{-0.058926in}}%
\pgfpathlineto{\pgfqpoint{0.035355in}{0.000000in}}%
\pgfpathlineto{\pgfqpoint{0.000000in}{0.058926in}}%
\pgfpathlineto{\pgfqpoint{-0.035355in}{0.000000in}}%
\pgfpathclose%
\pgfusepath{stroke,fill}%
}%
\begin{pgfscope}%
\pgfsys@transformshift{1.063889in}{3.375419in}%
\pgfsys@useobject{currentmarker}{}%
\end{pgfscope}%
\end{pgfscope}%
\begin{pgfscope}%
\definecolor{textcolor}{rgb}{0.000000,0.000000,0.000000}%
\pgfsetstrokecolor{textcolor}%
\pgfsetfillcolor{textcolor}%
\pgftext[x=1.313889in,y=3.326808in,left,base]{\color{textcolor}\sffamily\fontsize{10.000000}{12.000000}\selectfont \(\displaystyle \alpha = 0.2/k^{0.3}\)}%
\end{pgfscope}%
\end{pgfpicture}%
\makeatother%
\endgroup%

  \caption{GMRES iteration counts for $\AmatoI\Amatt$ given by \cref{eq:noweak,eq:ntweak}, where $\alpha = 0.2/k^\beta,$ for $\beta = 0,0.1,0.2,0.3.$}\label{fig:l1low}
\end{figure}

\begin{figure}
  %% Creator: Matplotlib, PGF backend
%%
%% To include the figure in your LaTeX document, write
%%   \input{<filename>.pgf}
%%
%% Make sure the required packages are loaded in your preamble
%%   \usepackage{pgf}
%%
%% Figures using additional raster images can only be included by \input if
%% they are in the same directory as the main LaTeX file. For loading figures
%% from other directories you can use the `import` package
%%   \usepackage{import}
%% and then include the figures with
%%   \import{<path to file>}{<filename>.pgf}
%%
%% Matplotlib used the following preamble
%%   \usepackage{fontspec}
%%   \setmainfont{DejaVuSerif.ttf}[Path=/home/owen/progs/firedrake-complex/firedrake/lib/python3.5/site-packages/matplotlib/mpl-data/fonts/ttf/]
%%   \setsansfont{DejaVuSans.ttf}[Path=/home/owen/progs/firedrake-complex/firedrake/lib/python3.5/site-packages/matplotlib/mpl-data/fonts/ttf/]
%%   \setmonofont{DejaVuSansMono.ttf}[Path=/home/owen/progs/firedrake-complex/firedrake/lib/python3.5/site-packages/matplotlib/mpl-data/fonts/ttf/]
%%
\begingroup%
\makeatletter%
\begin{pgfpicture}%
\pgfpathrectangle{\pgfpointorigin}{\pgfqpoint{6.400000in}{4.800000in}}%
\pgfusepath{use as bounding box, clip}%
\begin{pgfscope}%
\pgfsetbuttcap%
\pgfsetmiterjoin%
\definecolor{currentfill}{rgb}{1.000000,1.000000,1.000000}%
\pgfsetfillcolor{currentfill}%
\pgfsetlinewidth{0.000000pt}%
\definecolor{currentstroke}{rgb}{1.000000,1.000000,1.000000}%
\pgfsetstrokecolor{currentstroke}%
\pgfsetdash{}{0pt}%
\pgfpathmoveto{\pgfqpoint{0.000000in}{0.000000in}}%
\pgfpathlineto{\pgfqpoint{6.400000in}{0.000000in}}%
\pgfpathlineto{\pgfqpoint{6.400000in}{4.800000in}}%
\pgfpathlineto{\pgfqpoint{0.000000in}{4.800000in}}%
\pgfpathclose%
\pgfusepath{fill}%
\end{pgfscope}%
\begin{pgfscope}%
\pgfsetbuttcap%
\pgfsetmiterjoin%
\definecolor{currentfill}{rgb}{1.000000,1.000000,1.000000}%
\pgfsetfillcolor{currentfill}%
\pgfsetlinewidth{0.000000pt}%
\definecolor{currentstroke}{rgb}{0.000000,0.000000,0.000000}%
\pgfsetstrokecolor{currentstroke}%
\pgfsetstrokeopacity{0.000000}%
\pgfsetdash{}{0pt}%
\pgfpathmoveto{\pgfqpoint{0.800000in}{0.528000in}}%
\pgfpathlineto{\pgfqpoint{5.760000in}{0.528000in}}%
\pgfpathlineto{\pgfqpoint{5.760000in}{4.224000in}}%
\pgfpathlineto{\pgfqpoint{0.800000in}{4.224000in}}%
\pgfpathclose%
\pgfusepath{fill}%
\end{pgfscope}%
\begin{pgfscope}%
\pgfsetbuttcap%
\pgfsetroundjoin%
\definecolor{currentfill}{rgb}{0.000000,0.000000,0.000000}%
\pgfsetfillcolor{currentfill}%
\pgfsetlinewidth{0.803000pt}%
\definecolor{currentstroke}{rgb}{0.000000,0.000000,0.000000}%
\pgfsetstrokecolor{currentstroke}%
\pgfsetdash{}{0pt}%
\pgfsys@defobject{currentmarker}{\pgfqpoint{0.000000in}{-0.048611in}}{\pgfqpoint{0.000000in}{0.000000in}}{%
\pgfpathmoveto{\pgfqpoint{0.000000in}{0.000000in}}%
\pgfpathlineto{\pgfqpoint{0.000000in}{-0.048611in}}%
\pgfusepath{stroke,fill}%
}%
\begin{pgfscope}%
\pgfsys@transformshift{1.250909in}{0.528000in}%
\pgfsys@useobject{currentmarker}{}%
\end{pgfscope}%
\end{pgfscope}%
\begin{pgfscope}%
\definecolor{textcolor}{rgb}{0.000000,0.000000,0.000000}%
\pgfsetstrokecolor{textcolor}%
\pgfsetfillcolor{textcolor}%
\pgftext[x=1.250909in,y=0.430778in,,top]{\color{textcolor}\sffamily\fontsize{10.000000}{12.000000}\selectfont 10}%
\end{pgfscope}%
\begin{pgfscope}%
\pgfsetbuttcap%
\pgfsetroundjoin%
\definecolor{currentfill}{rgb}{0.000000,0.000000,0.000000}%
\pgfsetfillcolor{currentfill}%
\pgfsetlinewidth{0.803000pt}%
\definecolor{currentstroke}{rgb}{0.000000,0.000000,0.000000}%
\pgfsetstrokecolor{currentstroke}%
\pgfsetdash{}{0pt}%
\pgfsys@defobject{currentmarker}{\pgfqpoint{0.000000in}{-0.048611in}}{\pgfqpoint{0.000000in}{0.000000in}}{%
\pgfpathmoveto{\pgfqpoint{0.000000in}{0.000000in}}%
\pgfpathlineto{\pgfqpoint{0.000000in}{-0.048611in}}%
\pgfusepath{stroke,fill}%
}%
\begin{pgfscope}%
\pgfsys@transformshift{1.701818in}{0.528000in}%
\pgfsys@useobject{currentmarker}{}%
\end{pgfscope}%
\end{pgfscope}%
\begin{pgfscope}%
\definecolor{textcolor}{rgb}{0.000000,0.000000,0.000000}%
\pgfsetstrokecolor{textcolor}%
\pgfsetfillcolor{textcolor}%
\pgftext[x=1.701818in,y=0.430778in,,top]{\color{textcolor}\sffamily\fontsize{10.000000}{12.000000}\selectfont 20}%
\end{pgfscope}%
\begin{pgfscope}%
\pgfsetbuttcap%
\pgfsetroundjoin%
\definecolor{currentfill}{rgb}{0.000000,0.000000,0.000000}%
\pgfsetfillcolor{currentfill}%
\pgfsetlinewidth{0.803000pt}%
\definecolor{currentstroke}{rgb}{0.000000,0.000000,0.000000}%
\pgfsetstrokecolor{currentstroke}%
\pgfsetdash{}{0pt}%
\pgfsys@defobject{currentmarker}{\pgfqpoint{0.000000in}{-0.048611in}}{\pgfqpoint{0.000000in}{0.000000in}}{%
\pgfpathmoveto{\pgfqpoint{0.000000in}{0.000000in}}%
\pgfpathlineto{\pgfqpoint{0.000000in}{-0.048611in}}%
\pgfusepath{stroke,fill}%
}%
\begin{pgfscope}%
\pgfsys@transformshift{2.152727in}{0.528000in}%
\pgfsys@useobject{currentmarker}{}%
\end{pgfscope}%
\end{pgfscope}%
\begin{pgfscope}%
\definecolor{textcolor}{rgb}{0.000000,0.000000,0.000000}%
\pgfsetstrokecolor{textcolor}%
\pgfsetfillcolor{textcolor}%
\pgftext[x=2.152727in,y=0.430778in,,top]{\color{textcolor}\sffamily\fontsize{10.000000}{12.000000}\selectfont 30}%
\end{pgfscope}%
\begin{pgfscope}%
\pgfsetbuttcap%
\pgfsetroundjoin%
\definecolor{currentfill}{rgb}{0.000000,0.000000,0.000000}%
\pgfsetfillcolor{currentfill}%
\pgfsetlinewidth{0.803000pt}%
\definecolor{currentstroke}{rgb}{0.000000,0.000000,0.000000}%
\pgfsetstrokecolor{currentstroke}%
\pgfsetdash{}{0pt}%
\pgfsys@defobject{currentmarker}{\pgfqpoint{0.000000in}{-0.048611in}}{\pgfqpoint{0.000000in}{0.000000in}}{%
\pgfpathmoveto{\pgfqpoint{0.000000in}{0.000000in}}%
\pgfpathlineto{\pgfqpoint{0.000000in}{-0.048611in}}%
\pgfusepath{stroke,fill}%
}%
\begin{pgfscope}%
\pgfsys@transformshift{2.603636in}{0.528000in}%
\pgfsys@useobject{currentmarker}{}%
\end{pgfscope}%
\end{pgfscope}%
\begin{pgfscope}%
\definecolor{textcolor}{rgb}{0.000000,0.000000,0.000000}%
\pgfsetstrokecolor{textcolor}%
\pgfsetfillcolor{textcolor}%
\pgftext[x=2.603636in,y=0.430778in,,top]{\color{textcolor}\sffamily\fontsize{10.000000}{12.000000}\selectfont 40}%
\end{pgfscope}%
\begin{pgfscope}%
\pgfsetbuttcap%
\pgfsetroundjoin%
\definecolor{currentfill}{rgb}{0.000000,0.000000,0.000000}%
\pgfsetfillcolor{currentfill}%
\pgfsetlinewidth{0.803000pt}%
\definecolor{currentstroke}{rgb}{0.000000,0.000000,0.000000}%
\pgfsetstrokecolor{currentstroke}%
\pgfsetdash{}{0pt}%
\pgfsys@defobject{currentmarker}{\pgfqpoint{0.000000in}{-0.048611in}}{\pgfqpoint{0.000000in}{0.000000in}}{%
\pgfpathmoveto{\pgfqpoint{0.000000in}{0.000000in}}%
\pgfpathlineto{\pgfqpoint{0.000000in}{-0.048611in}}%
\pgfusepath{stroke,fill}%
}%
\begin{pgfscope}%
\pgfsys@transformshift{3.054545in}{0.528000in}%
\pgfsys@useobject{currentmarker}{}%
\end{pgfscope}%
\end{pgfscope}%
\begin{pgfscope}%
\definecolor{textcolor}{rgb}{0.000000,0.000000,0.000000}%
\pgfsetstrokecolor{textcolor}%
\pgfsetfillcolor{textcolor}%
\pgftext[x=3.054545in,y=0.430778in,,top]{\color{textcolor}\sffamily\fontsize{10.000000}{12.000000}\selectfont 50}%
\end{pgfscope}%
\begin{pgfscope}%
\pgfsetbuttcap%
\pgfsetroundjoin%
\definecolor{currentfill}{rgb}{0.000000,0.000000,0.000000}%
\pgfsetfillcolor{currentfill}%
\pgfsetlinewidth{0.803000pt}%
\definecolor{currentstroke}{rgb}{0.000000,0.000000,0.000000}%
\pgfsetstrokecolor{currentstroke}%
\pgfsetdash{}{0pt}%
\pgfsys@defobject{currentmarker}{\pgfqpoint{0.000000in}{-0.048611in}}{\pgfqpoint{0.000000in}{0.000000in}}{%
\pgfpathmoveto{\pgfqpoint{0.000000in}{0.000000in}}%
\pgfpathlineto{\pgfqpoint{0.000000in}{-0.048611in}}%
\pgfusepath{stroke,fill}%
}%
\begin{pgfscope}%
\pgfsys@transformshift{3.505455in}{0.528000in}%
\pgfsys@useobject{currentmarker}{}%
\end{pgfscope}%
\end{pgfscope}%
\begin{pgfscope}%
\definecolor{textcolor}{rgb}{0.000000,0.000000,0.000000}%
\pgfsetstrokecolor{textcolor}%
\pgfsetfillcolor{textcolor}%
\pgftext[x=3.505455in,y=0.430778in,,top]{\color{textcolor}\sffamily\fontsize{10.000000}{12.000000}\selectfont 60}%
\end{pgfscope}%
\begin{pgfscope}%
\pgfsetbuttcap%
\pgfsetroundjoin%
\definecolor{currentfill}{rgb}{0.000000,0.000000,0.000000}%
\pgfsetfillcolor{currentfill}%
\pgfsetlinewidth{0.803000pt}%
\definecolor{currentstroke}{rgb}{0.000000,0.000000,0.000000}%
\pgfsetstrokecolor{currentstroke}%
\pgfsetdash{}{0pt}%
\pgfsys@defobject{currentmarker}{\pgfqpoint{0.000000in}{-0.048611in}}{\pgfqpoint{0.000000in}{0.000000in}}{%
\pgfpathmoveto{\pgfqpoint{0.000000in}{0.000000in}}%
\pgfpathlineto{\pgfqpoint{0.000000in}{-0.048611in}}%
\pgfusepath{stroke,fill}%
}%
\begin{pgfscope}%
\pgfsys@transformshift{3.956364in}{0.528000in}%
\pgfsys@useobject{currentmarker}{}%
\end{pgfscope}%
\end{pgfscope}%
\begin{pgfscope}%
\definecolor{textcolor}{rgb}{0.000000,0.000000,0.000000}%
\pgfsetstrokecolor{textcolor}%
\pgfsetfillcolor{textcolor}%
\pgftext[x=3.956364in,y=0.430778in,,top]{\color{textcolor}\sffamily\fontsize{10.000000}{12.000000}\selectfont 70}%
\end{pgfscope}%
\begin{pgfscope}%
\pgfsetbuttcap%
\pgfsetroundjoin%
\definecolor{currentfill}{rgb}{0.000000,0.000000,0.000000}%
\pgfsetfillcolor{currentfill}%
\pgfsetlinewidth{0.803000pt}%
\definecolor{currentstroke}{rgb}{0.000000,0.000000,0.000000}%
\pgfsetstrokecolor{currentstroke}%
\pgfsetdash{}{0pt}%
\pgfsys@defobject{currentmarker}{\pgfqpoint{0.000000in}{-0.048611in}}{\pgfqpoint{0.000000in}{0.000000in}}{%
\pgfpathmoveto{\pgfqpoint{0.000000in}{0.000000in}}%
\pgfpathlineto{\pgfqpoint{0.000000in}{-0.048611in}}%
\pgfusepath{stroke,fill}%
}%
\begin{pgfscope}%
\pgfsys@transformshift{4.407273in}{0.528000in}%
\pgfsys@useobject{currentmarker}{}%
\end{pgfscope}%
\end{pgfscope}%
\begin{pgfscope}%
\definecolor{textcolor}{rgb}{0.000000,0.000000,0.000000}%
\pgfsetstrokecolor{textcolor}%
\pgfsetfillcolor{textcolor}%
\pgftext[x=4.407273in,y=0.430778in,,top]{\color{textcolor}\sffamily\fontsize{10.000000}{12.000000}\selectfont 80}%
\end{pgfscope}%
\begin{pgfscope}%
\pgfsetbuttcap%
\pgfsetroundjoin%
\definecolor{currentfill}{rgb}{0.000000,0.000000,0.000000}%
\pgfsetfillcolor{currentfill}%
\pgfsetlinewidth{0.803000pt}%
\definecolor{currentstroke}{rgb}{0.000000,0.000000,0.000000}%
\pgfsetstrokecolor{currentstroke}%
\pgfsetdash{}{0pt}%
\pgfsys@defobject{currentmarker}{\pgfqpoint{0.000000in}{-0.048611in}}{\pgfqpoint{0.000000in}{0.000000in}}{%
\pgfpathmoveto{\pgfqpoint{0.000000in}{0.000000in}}%
\pgfpathlineto{\pgfqpoint{0.000000in}{-0.048611in}}%
\pgfusepath{stroke,fill}%
}%
\begin{pgfscope}%
\pgfsys@transformshift{4.858182in}{0.528000in}%
\pgfsys@useobject{currentmarker}{}%
\end{pgfscope}%
\end{pgfscope}%
\begin{pgfscope}%
\definecolor{textcolor}{rgb}{0.000000,0.000000,0.000000}%
\pgfsetstrokecolor{textcolor}%
\pgfsetfillcolor{textcolor}%
\pgftext[x=4.858182in,y=0.430778in,,top]{\color{textcolor}\sffamily\fontsize{10.000000}{12.000000}\selectfont 90}%
\end{pgfscope}%
\begin{pgfscope}%
\pgfsetbuttcap%
\pgfsetroundjoin%
\definecolor{currentfill}{rgb}{0.000000,0.000000,0.000000}%
\pgfsetfillcolor{currentfill}%
\pgfsetlinewidth{0.803000pt}%
\definecolor{currentstroke}{rgb}{0.000000,0.000000,0.000000}%
\pgfsetstrokecolor{currentstroke}%
\pgfsetdash{}{0pt}%
\pgfsys@defobject{currentmarker}{\pgfqpoint{0.000000in}{-0.048611in}}{\pgfqpoint{0.000000in}{0.000000in}}{%
\pgfpathmoveto{\pgfqpoint{0.000000in}{0.000000in}}%
\pgfpathlineto{\pgfqpoint{0.000000in}{-0.048611in}}%
\pgfusepath{stroke,fill}%
}%
\begin{pgfscope}%
\pgfsys@transformshift{5.309091in}{0.528000in}%
\pgfsys@useobject{currentmarker}{}%
\end{pgfscope}%
\end{pgfscope}%
\begin{pgfscope}%
\definecolor{textcolor}{rgb}{0.000000,0.000000,0.000000}%
\pgfsetstrokecolor{textcolor}%
\pgfsetfillcolor{textcolor}%
\pgftext[x=5.309091in,y=0.430778in,,top]{\color{textcolor}\sffamily\fontsize{10.000000}{12.000000}\selectfont 100}%
\end{pgfscope}%
\begin{pgfscope}%
\definecolor{textcolor}{rgb}{0.000000,0.000000,0.000000}%
\pgfsetstrokecolor{textcolor}%
\pgfsetfillcolor{textcolor}%
\pgftext[x=3.280000in,y=0.240809in,,top]{\color{textcolor}\sffamily\fontsize{10.000000}{12.000000}\selectfont \(\displaystyle k\)}%
\end{pgfscope}%
\begin{pgfscope}%
\pgfsetbuttcap%
\pgfsetroundjoin%
\definecolor{currentfill}{rgb}{0.000000,0.000000,0.000000}%
\pgfsetfillcolor{currentfill}%
\pgfsetlinewidth{0.803000pt}%
\definecolor{currentstroke}{rgb}{0.000000,0.000000,0.000000}%
\pgfsetstrokecolor{currentstroke}%
\pgfsetdash{}{0pt}%
\pgfsys@defobject{currentmarker}{\pgfqpoint{-0.048611in}{0.000000in}}{\pgfqpoint{0.000000in}{0.000000in}}{%
\pgfpathmoveto{\pgfqpoint{0.000000in}{0.000000in}}%
\pgfpathlineto{\pgfqpoint{-0.048611in}{0.000000in}}%
\pgfusepath{stroke,fill}%
}%
\begin{pgfscope}%
\pgfsys@transformshift{0.800000in}{0.770667in}%
\pgfsys@useobject{currentmarker}{}%
\end{pgfscope}%
\end{pgfscope}%
\begin{pgfscope}%
\definecolor{textcolor}{rgb}{0.000000,0.000000,0.000000}%
\pgfsetstrokecolor{textcolor}%
\pgfsetfillcolor{textcolor}%
\pgftext[x=0.526047in,y=0.717905in,left,base]{\color{textcolor}\sffamily\fontsize{10.000000}{12.000000}\selectfont 10}%
\end{pgfscope}%
\begin{pgfscope}%
\pgfsetbuttcap%
\pgfsetroundjoin%
\definecolor{currentfill}{rgb}{0.000000,0.000000,0.000000}%
\pgfsetfillcolor{currentfill}%
\pgfsetlinewidth{0.803000pt}%
\definecolor{currentstroke}{rgb}{0.000000,0.000000,0.000000}%
\pgfsetstrokecolor{currentstroke}%
\pgfsetdash{}{0pt}%
\pgfsys@defobject{currentmarker}{\pgfqpoint{-0.048611in}{0.000000in}}{\pgfqpoint{0.000000in}{0.000000in}}{%
\pgfpathmoveto{\pgfqpoint{0.000000in}{0.000000in}}%
\pgfpathlineto{\pgfqpoint{-0.048611in}{0.000000in}}%
\pgfusepath{stroke,fill}%
}%
\begin{pgfscope}%
\pgfsys@transformshift{0.800000in}{1.144000in}%
\pgfsys@useobject{currentmarker}{}%
\end{pgfscope}%
\end{pgfscope}%
\begin{pgfscope}%
\definecolor{textcolor}{rgb}{0.000000,0.000000,0.000000}%
\pgfsetstrokecolor{textcolor}%
\pgfsetfillcolor{textcolor}%
\pgftext[x=0.526047in,y=1.091238in,left,base]{\color{textcolor}\sffamily\fontsize{10.000000}{12.000000}\selectfont 20}%
\end{pgfscope}%
\begin{pgfscope}%
\pgfsetbuttcap%
\pgfsetroundjoin%
\definecolor{currentfill}{rgb}{0.000000,0.000000,0.000000}%
\pgfsetfillcolor{currentfill}%
\pgfsetlinewidth{0.803000pt}%
\definecolor{currentstroke}{rgb}{0.000000,0.000000,0.000000}%
\pgfsetstrokecolor{currentstroke}%
\pgfsetdash{}{0pt}%
\pgfsys@defobject{currentmarker}{\pgfqpoint{-0.048611in}{0.000000in}}{\pgfqpoint{0.000000in}{0.000000in}}{%
\pgfpathmoveto{\pgfqpoint{0.000000in}{0.000000in}}%
\pgfpathlineto{\pgfqpoint{-0.048611in}{0.000000in}}%
\pgfusepath{stroke,fill}%
}%
\begin{pgfscope}%
\pgfsys@transformshift{0.800000in}{1.517333in}%
\pgfsys@useobject{currentmarker}{}%
\end{pgfscope}%
\end{pgfscope}%
\begin{pgfscope}%
\definecolor{textcolor}{rgb}{0.000000,0.000000,0.000000}%
\pgfsetstrokecolor{textcolor}%
\pgfsetfillcolor{textcolor}%
\pgftext[x=0.526047in,y=1.464572in,left,base]{\color{textcolor}\sffamily\fontsize{10.000000}{12.000000}\selectfont 30}%
\end{pgfscope}%
\begin{pgfscope}%
\pgfsetbuttcap%
\pgfsetroundjoin%
\definecolor{currentfill}{rgb}{0.000000,0.000000,0.000000}%
\pgfsetfillcolor{currentfill}%
\pgfsetlinewidth{0.803000pt}%
\definecolor{currentstroke}{rgb}{0.000000,0.000000,0.000000}%
\pgfsetstrokecolor{currentstroke}%
\pgfsetdash{}{0pt}%
\pgfsys@defobject{currentmarker}{\pgfqpoint{-0.048611in}{0.000000in}}{\pgfqpoint{0.000000in}{0.000000in}}{%
\pgfpathmoveto{\pgfqpoint{0.000000in}{0.000000in}}%
\pgfpathlineto{\pgfqpoint{-0.048611in}{0.000000in}}%
\pgfusepath{stroke,fill}%
}%
\begin{pgfscope}%
\pgfsys@transformshift{0.800000in}{1.890667in}%
\pgfsys@useobject{currentmarker}{}%
\end{pgfscope}%
\end{pgfscope}%
\begin{pgfscope}%
\definecolor{textcolor}{rgb}{0.000000,0.000000,0.000000}%
\pgfsetstrokecolor{textcolor}%
\pgfsetfillcolor{textcolor}%
\pgftext[x=0.526047in,y=1.837905in,left,base]{\color{textcolor}\sffamily\fontsize{10.000000}{12.000000}\selectfont 40}%
\end{pgfscope}%
\begin{pgfscope}%
\pgfsetbuttcap%
\pgfsetroundjoin%
\definecolor{currentfill}{rgb}{0.000000,0.000000,0.000000}%
\pgfsetfillcolor{currentfill}%
\pgfsetlinewidth{0.803000pt}%
\definecolor{currentstroke}{rgb}{0.000000,0.000000,0.000000}%
\pgfsetstrokecolor{currentstroke}%
\pgfsetdash{}{0pt}%
\pgfsys@defobject{currentmarker}{\pgfqpoint{-0.048611in}{0.000000in}}{\pgfqpoint{0.000000in}{0.000000in}}{%
\pgfpathmoveto{\pgfqpoint{0.000000in}{0.000000in}}%
\pgfpathlineto{\pgfqpoint{-0.048611in}{0.000000in}}%
\pgfusepath{stroke,fill}%
}%
\begin{pgfscope}%
\pgfsys@transformshift{0.800000in}{2.264000in}%
\pgfsys@useobject{currentmarker}{}%
\end{pgfscope}%
\end{pgfscope}%
\begin{pgfscope}%
\definecolor{textcolor}{rgb}{0.000000,0.000000,0.000000}%
\pgfsetstrokecolor{textcolor}%
\pgfsetfillcolor{textcolor}%
\pgftext[x=0.526047in,y=2.211238in,left,base]{\color{textcolor}\sffamily\fontsize{10.000000}{12.000000}\selectfont 50}%
\end{pgfscope}%
\begin{pgfscope}%
\pgfsetbuttcap%
\pgfsetroundjoin%
\definecolor{currentfill}{rgb}{0.000000,0.000000,0.000000}%
\pgfsetfillcolor{currentfill}%
\pgfsetlinewidth{0.803000pt}%
\definecolor{currentstroke}{rgb}{0.000000,0.000000,0.000000}%
\pgfsetstrokecolor{currentstroke}%
\pgfsetdash{}{0pt}%
\pgfsys@defobject{currentmarker}{\pgfqpoint{-0.048611in}{0.000000in}}{\pgfqpoint{0.000000in}{0.000000in}}{%
\pgfpathmoveto{\pgfqpoint{0.000000in}{0.000000in}}%
\pgfpathlineto{\pgfqpoint{-0.048611in}{0.000000in}}%
\pgfusepath{stroke,fill}%
}%
\begin{pgfscope}%
\pgfsys@transformshift{0.800000in}{2.637333in}%
\pgfsys@useobject{currentmarker}{}%
\end{pgfscope}%
\end{pgfscope}%
\begin{pgfscope}%
\definecolor{textcolor}{rgb}{0.000000,0.000000,0.000000}%
\pgfsetstrokecolor{textcolor}%
\pgfsetfillcolor{textcolor}%
\pgftext[x=0.526047in,y=2.584572in,left,base]{\color{textcolor}\sffamily\fontsize{10.000000}{12.000000}\selectfont 60}%
\end{pgfscope}%
\begin{pgfscope}%
\pgfsetbuttcap%
\pgfsetroundjoin%
\definecolor{currentfill}{rgb}{0.000000,0.000000,0.000000}%
\pgfsetfillcolor{currentfill}%
\pgfsetlinewidth{0.803000pt}%
\definecolor{currentstroke}{rgb}{0.000000,0.000000,0.000000}%
\pgfsetstrokecolor{currentstroke}%
\pgfsetdash{}{0pt}%
\pgfsys@defobject{currentmarker}{\pgfqpoint{-0.048611in}{0.000000in}}{\pgfqpoint{0.000000in}{0.000000in}}{%
\pgfpathmoveto{\pgfqpoint{0.000000in}{0.000000in}}%
\pgfpathlineto{\pgfqpoint{-0.048611in}{0.000000in}}%
\pgfusepath{stroke,fill}%
}%
\begin{pgfscope}%
\pgfsys@transformshift{0.800000in}{3.010667in}%
\pgfsys@useobject{currentmarker}{}%
\end{pgfscope}%
\end{pgfscope}%
\begin{pgfscope}%
\definecolor{textcolor}{rgb}{0.000000,0.000000,0.000000}%
\pgfsetstrokecolor{textcolor}%
\pgfsetfillcolor{textcolor}%
\pgftext[x=0.526047in,y=2.957905in,left,base]{\color{textcolor}\sffamily\fontsize{10.000000}{12.000000}\selectfont 70}%
\end{pgfscope}%
\begin{pgfscope}%
\pgfsetbuttcap%
\pgfsetroundjoin%
\definecolor{currentfill}{rgb}{0.000000,0.000000,0.000000}%
\pgfsetfillcolor{currentfill}%
\pgfsetlinewidth{0.803000pt}%
\definecolor{currentstroke}{rgb}{0.000000,0.000000,0.000000}%
\pgfsetstrokecolor{currentstroke}%
\pgfsetdash{}{0pt}%
\pgfsys@defobject{currentmarker}{\pgfqpoint{-0.048611in}{0.000000in}}{\pgfqpoint{0.000000in}{0.000000in}}{%
\pgfpathmoveto{\pgfqpoint{0.000000in}{0.000000in}}%
\pgfpathlineto{\pgfqpoint{-0.048611in}{0.000000in}}%
\pgfusepath{stroke,fill}%
}%
\begin{pgfscope}%
\pgfsys@transformshift{0.800000in}{3.384000in}%
\pgfsys@useobject{currentmarker}{}%
\end{pgfscope}%
\end{pgfscope}%
\begin{pgfscope}%
\definecolor{textcolor}{rgb}{0.000000,0.000000,0.000000}%
\pgfsetstrokecolor{textcolor}%
\pgfsetfillcolor{textcolor}%
\pgftext[x=0.526047in,y=3.331238in,left,base]{\color{textcolor}\sffamily\fontsize{10.000000}{12.000000}\selectfont 80}%
\end{pgfscope}%
\begin{pgfscope}%
\pgfsetbuttcap%
\pgfsetroundjoin%
\definecolor{currentfill}{rgb}{0.000000,0.000000,0.000000}%
\pgfsetfillcolor{currentfill}%
\pgfsetlinewidth{0.803000pt}%
\definecolor{currentstroke}{rgb}{0.000000,0.000000,0.000000}%
\pgfsetstrokecolor{currentstroke}%
\pgfsetdash{}{0pt}%
\pgfsys@defobject{currentmarker}{\pgfqpoint{-0.048611in}{0.000000in}}{\pgfqpoint{0.000000in}{0.000000in}}{%
\pgfpathmoveto{\pgfqpoint{0.000000in}{0.000000in}}%
\pgfpathlineto{\pgfqpoint{-0.048611in}{0.000000in}}%
\pgfusepath{stroke,fill}%
}%
\begin{pgfscope}%
\pgfsys@transformshift{0.800000in}{3.757333in}%
\pgfsys@useobject{currentmarker}{}%
\end{pgfscope}%
\end{pgfscope}%
\begin{pgfscope}%
\definecolor{textcolor}{rgb}{0.000000,0.000000,0.000000}%
\pgfsetstrokecolor{textcolor}%
\pgfsetfillcolor{textcolor}%
\pgftext[x=0.526047in,y=3.704572in,left,base]{\color{textcolor}\sffamily\fontsize{10.000000}{12.000000}\selectfont 90}%
\end{pgfscope}%
\begin{pgfscope}%
\pgfsetbuttcap%
\pgfsetroundjoin%
\definecolor{currentfill}{rgb}{0.000000,0.000000,0.000000}%
\pgfsetfillcolor{currentfill}%
\pgfsetlinewidth{0.803000pt}%
\definecolor{currentstroke}{rgb}{0.000000,0.000000,0.000000}%
\pgfsetstrokecolor{currentstroke}%
\pgfsetdash{}{0pt}%
\pgfsys@defobject{currentmarker}{\pgfqpoint{-0.048611in}{0.000000in}}{\pgfqpoint{0.000000in}{0.000000in}}{%
\pgfpathmoveto{\pgfqpoint{0.000000in}{0.000000in}}%
\pgfpathlineto{\pgfqpoint{-0.048611in}{0.000000in}}%
\pgfusepath{stroke,fill}%
}%
\begin{pgfscope}%
\pgfsys@transformshift{0.800000in}{4.130667in}%
\pgfsys@useobject{currentmarker}{}%
\end{pgfscope}%
\end{pgfscope}%
\begin{pgfscope}%
\definecolor{textcolor}{rgb}{0.000000,0.000000,0.000000}%
\pgfsetstrokecolor{textcolor}%
\pgfsetfillcolor{textcolor}%
\pgftext[x=0.437682in,y=4.077905in,left,base]{\color{textcolor}\sffamily\fontsize{10.000000}{12.000000}\selectfont 100}%
\end{pgfscope}%
\begin{pgfscope}%
\definecolor{textcolor}{rgb}{0.000000,0.000000,0.000000}%
\pgfsetstrokecolor{textcolor}%
\pgfsetfillcolor{textcolor}%
\pgftext[x=0.382126in,y=2.376000in,,bottom,rotate=90.000000]{\color{textcolor}\sffamily\fontsize{10.000000}{12.000000}\selectfont Number of GMRES iterations}%
\end{pgfscope}%
\begin{pgfscope}%
\pgfpathrectangle{\pgfqpoint{0.800000in}{0.528000in}}{\pgfqpoint{4.960000in}{3.696000in}}%
\pgfusepath{clip}%
\pgfsetbuttcap%
\pgfsetroundjoin%
\pgfsetlinewidth{1.505625pt}%
\definecolor{currentstroke}{rgb}{0.000000,0.000000,0.000000}%
\pgfsetstrokecolor{currentstroke}%
\pgfsetdash{{5.550000pt}{2.400000pt}}{0.000000pt}%
\pgfpathmoveto{\pgfqpoint{1.250909in}{0.770667in}}%
\pgfpathlineto{\pgfqpoint{1.701818in}{0.957333in}}%
\pgfpathlineto{\pgfqpoint{2.152727in}{1.144000in}}%
\pgfpathlineto{\pgfqpoint{2.603636in}{1.330667in}}%
\pgfpathlineto{\pgfqpoint{3.054545in}{1.517333in}}%
\pgfpathlineto{\pgfqpoint{3.505455in}{1.965333in}}%
\pgfpathlineto{\pgfqpoint{3.956364in}{2.376000in}}%
\pgfpathlineto{\pgfqpoint{4.407273in}{2.786667in}}%
\pgfpathlineto{\pgfqpoint{4.858182in}{3.421333in}}%
\pgfpathlineto{\pgfqpoint{5.309091in}{4.056000in}}%
\pgfusepath{stroke}%
\end{pgfscope}%
\begin{pgfscope}%
\pgfpathrectangle{\pgfqpoint{0.800000in}{0.528000in}}{\pgfqpoint{4.960000in}{3.696000in}}%
\pgfusepath{clip}%
\pgfsetbuttcap%
\pgfsetroundjoin%
\definecolor{currentfill}{rgb}{0.000000,0.000000,0.000000}%
\pgfsetfillcolor{currentfill}%
\pgfsetlinewidth{1.003750pt}%
\definecolor{currentstroke}{rgb}{0.000000,0.000000,0.000000}%
\pgfsetstrokecolor{currentstroke}%
\pgfsetdash{}{0pt}%
\pgfsys@defobject{currentmarker}{\pgfqpoint{-0.041667in}{-0.041667in}}{\pgfqpoint{0.041667in}{0.041667in}}{%
\pgfpathmoveto{\pgfqpoint{0.000000in}{-0.041667in}}%
\pgfpathcurveto{\pgfqpoint{0.011050in}{-0.041667in}}{\pgfqpoint{0.021649in}{-0.037276in}}{\pgfqpoint{0.029463in}{-0.029463in}}%
\pgfpathcurveto{\pgfqpoint{0.037276in}{-0.021649in}}{\pgfqpoint{0.041667in}{-0.011050in}}{\pgfqpoint{0.041667in}{0.000000in}}%
\pgfpathcurveto{\pgfqpoint{0.041667in}{0.011050in}}{\pgfqpoint{0.037276in}{0.021649in}}{\pgfqpoint{0.029463in}{0.029463in}}%
\pgfpathcurveto{\pgfqpoint{0.021649in}{0.037276in}}{\pgfqpoint{0.011050in}{0.041667in}}{\pgfqpoint{0.000000in}{0.041667in}}%
\pgfpathcurveto{\pgfqpoint{-0.011050in}{0.041667in}}{\pgfqpoint{-0.021649in}{0.037276in}}{\pgfqpoint{-0.029463in}{0.029463in}}%
\pgfpathcurveto{\pgfqpoint{-0.037276in}{0.021649in}}{\pgfqpoint{-0.041667in}{0.011050in}}{\pgfqpoint{-0.041667in}{0.000000in}}%
\pgfpathcurveto{\pgfqpoint{-0.041667in}{-0.011050in}}{\pgfqpoint{-0.037276in}{-0.021649in}}{\pgfqpoint{-0.029463in}{-0.029463in}}%
\pgfpathcurveto{\pgfqpoint{-0.021649in}{-0.037276in}}{\pgfqpoint{-0.011050in}{-0.041667in}}{\pgfqpoint{0.000000in}{-0.041667in}}%
\pgfpathclose%
\pgfusepath{stroke,fill}%
}%
\begin{pgfscope}%
\pgfsys@transformshift{1.250909in}{0.770667in}%
\pgfsys@useobject{currentmarker}{}%
\end{pgfscope}%
\begin{pgfscope}%
\pgfsys@transformshift{1.701818in}{0.957333in}%
\pgfsys@useobject{currentmarker}{}%
\end{pgfscope}%
\begin{pgfscope}%
\pgfsys@transformshift{2.152727in}{1.144000in}%
\pgfsys@useobject{currentmarker}{}%
\end{pgfscope}%
\begin{pgfscope}%
\pgfsys@transformshift{2.603636in}{1.330667in}%
\pgfsys@useobject{currentmarker}{}%
\end{pgfscope}%
\begin{pgfscope}%
\pgfsys@transformshift{3.054545in}{1.517333in}%
\pgfsys@useobject{currentmarker}{}%
\end{pgfscope}%
\begin{pgfscope}%
\pgfsys@transformshift{3.505455in}{1.965333in}%
\pgfsys@useobject{currentmarker}{}%
\end{pgfscope}%
\begin{pgfscope}%
\pgfsys@transformshift{3.956364in}{2.376000in}%
\pgfsys@useobject{currentmarker}{}%
\end{pgfscope}%
\begin{pgfscope}%
\pgfsys@transformshift{4.407273in}{2.786667in}%
\pgfsys@useobject{currentmarker}{}%
\end{pgfscope}%
\begin{pgfscope}%
\pgfsys@transformshift{4.858182in}{3.421333in}%
\pgfsys@useobject{currentmarker}{}%
\end{pgfscope}%
\begin{pgfscope}%
\pgfsys@transformshift{5.309091in}{4.056000in}%
\pgfsys@useobject{currentmarker}{}%
\end{pgfscope}%
\end{pgfscope}%
\begin{pgfscope}%
\pgfpathrectangle{\pgfqpoint{0.800000in}{0.528000in}}{\pgfqpoint{4.960000in}{3.696000in}}%
\pgfusepath{clip}%
\pgfsetbuttcap%
\pgfsetroundjoin%
\pgfsetlinewidth{1.505625pt}%
\definecolor{currentstroke}{rgb}{0.000000,0.000000,0.000000}%
\pgfsetstrokecolor{currentstroke}%
\pgfsetdash{{5.550000pt}{2.400000pt}}{0.000000pt}%
\pgfpathmoveto{\pgfqpoint{1.250909in}{0.770667in}}%
\pgfpathlineto{\pgfqpoint{1.701818in}{0.882667in}}%
\pgfpathlineto{\pgfqpoint{2.152727in}{0.994667in}}%
\pgfpathlineto{\pgfqpoint{2.603636in}{1.106667in}}%
\pgfpathlineto{\pgfqpoint{3.054545in}{1.218667in}}%
\pgfpathlineto{\pgfqpoint{3.505455in}{1.330667in}}%
\pgfpathlineto{\pgfqpoint{3.956364in}{1.442667in}}%
\pgfpathlineto{\pgfqpoint{4.407273in}{1.554667in}}%
\pgfpathlineto{\pgfqpoint{4.858182in}{1.778667in}}%
\pgfpathlineto{\pgfqpoint{5.309091in}{1.928000in}}%
\pgfusepath{stroke}%
\end{pgfscope}%
\begin{pgfscope}%
\pgfpathrectangle{\pgfqpoint{0.800000in}{0.528000in}}{\pgfqpoint{4.960000in}{3.696000in}}%
\pgfusepath{clip}%
\pgfsetbuttcap%
\pgfsetmiterjoin%
\definecolor{currentfill}{rgb}{0.000000,0.000000,0.000000}%
\pgfsetfillcolor{currentfill}%
\pgfsetlinewidth{1.003750pt}%
\definecolor{currentstroke}{rgb}{0.000000,0.000000,0.000000}%
\pgfsetstrokecolor{currentstroke}%
\pgfsetdash{}{0pt}%
\pgfsys@defobject{currentmarker}{\pgfqpoint{-0.041667in}{-0.041667in}}{\pgfqpoint{0.041667in}{0.041667in}}{%
\pgfpathmoveto{\pgfqpoint{-0.000000in}{-0.041667in}}%
\pgfpathlineto{\pgfqpoint{0.041667in}{0.041667in}}%
\pgfpathlineto{\pgfqpoint{-0.041667in}{0.041667in}}%
\pgfpathclose%
\pgfusepath{stroke,fill}%
}%
\begin{pgfscope}%
\pgfsys@transformshift{1.250909in}{0.770667in}%
\pgfsys@useobject{currentmarker}{}%
\end{pgfscope}%
\begin{pgfscope}%
\pgfsys@transformshift{1.701818in}{0.882667in}%
\pgfsys@useobject{currentmarker}{}%
\end{pgfscope}%
\begin{pgfscope}%
\pgfsys@transformshift{2.152727in}{0.994667in}%
\pgfsys@useobject{currentmarker}{}%
\end{pgfscope}%
\begin{pgfscope}%
\pgfsys@transformshift{2.603636in}{1.106667in}%
\pgfsys@useobject{currentmarker}{}%
\end{pgfscope}%
\begin{pgfscope}%
\pgfsys@transformshift{3.054545in}{1.218667in}%
\pgfsys@useobject{currentmarker}{}%
\end{pgfscope}%
\begin{pgfscope}%
\pgfsys@transformshift{3.505455in}{1.330667in}%
\pgfsys@useobject{currentmarker}{}%
\end{pgfscope}%
\begin{pgfscope}%
\pgfsys@transformshift{3.956364in}{1.442667in}%
\pgfsys@useobject{currentmarker}{}%
\end{pgfscope}%
\begin{pgfscope}%
\pgfsys@transformshift{4.407273in}{1.554667in}%
\pgfsys@useobject{currentmarker}{}%
\end{pgfscope}%
\begin{pgfscope}%
\pgfsys@transformshift{4.858182in}{1.778667in}%
\pgfsys@useobject{currentmarker}{}%
\end{pgfscope}%
\begin{pgfscope}%
\pgfsys@transformshift{5.309091in}{1.928000in}%
\pgfsys@useobject{currentmarker}{}%
\end{pgfscope}%
\end{pgfscope}%
\begin{pgfscope}%
\pgfpathrectangle{\pgfqpoint{0.800000in}{0.528000in}}{\pgfqpoint{4.960000in}{3.696000in}}%
\pgfusepath{clip}%
\pgfsetbuttcap%
\pgfsetroundjoin%
\pgfsetlinewidth{1.505625pt}%
\definecolor{currentstroke}{rgb}{0.000000,0.000000,0.000000}%
\pgfsetstrokecolor{currentstroke}%
\pgfsetdash{{5.550000pt}{2.400000pt}}{0.000000pt}%
\pgfpathmoveto{\pgfqpoint{1.250909in}{0.733333in}}%
\pgfpathlineto{\pgfqpoint{1.701818in}{0.808000in}}%
\pgfpathlineto{\pgfqpoint{2.152727in}{0.882667in}}%
\pgfpathlineto{\pgfqpoint{2.603636in}{0.920000in}}%
\pgfpathlineto{\pgfqpoint{3.054545in}{0.994667in}}%
\pgfpathlineto{\pgfqpoint{3.505455in}{1.032000in}}%
\pgfpathlineto{\pgfqpoint{3.956364in}{1.106667in}}%
\pgfpathlineto{\pgfqpoint{4.407273in}{1.106667in}}%
\pgfpathlineto{\pgfqpoint{4.858182in}{1.181333in}}%
\pgfpathlineto{\pgfqpoint{5.309091in}{1.218667in}}%
\pgfusepath{stroke}%
\end{pgfscope}%
\begin{pgfscope}%
\pgfpathrectangle{\pgfqpoint{0.800000in}{0.528000in}}{\pgfqpoint{4.960000in}{3.696000in}}%
\pgfusepath{clip}%
\pgfsetbuttcap%
\pgfsetmiterjoin%
\definecolor{currentfill}{rgb}{0.000000,0.000000,0.000000}%
\pgfsetfillcolor{currentfill}%
\pgfsetlinewidth{1.003750pt}%
\definecolor{currentstroke}{rgb}{0.000000,0.000000,0.000000}%
\pgfsetstrokecolor{currentstroke}%
\pgfsetdash{}{0pt}%
\pgfsys@defobject{currentmarker}{\pgfqpoint{-0.041667in}{-0.041667in}}{\pgfqpoint{0.041667in}{0.041667in}}{%
\pgfpathmoveto{\pgfqpoint{-0.041667in}{-0.041667in}}%
\pgfpathlineto{\pgfqpoint{0.041667in}{-0.041667in}}%
\pgfpathlineto{\pgfqpoint{0.041667in}{0.041667in}}%
\pgfpathlineto{\pgfqpoint{-0.041667in}{0.041667in}}%
\pgfpathclose%
\pgfusepath{stroke,fill}%
}%
\begin{pgfscope}%
\pgfsys@transformshift{1.250909in}{0.733333in}%
\pgfsys@useobject{currentmarker}{}%
\end{pgfscope}%
\begin{pgfscope}%
\pgfsys@transformshift{1.701818in}{0.808000in}%
\pgfsys@useobject{currentmarker}{}%
\end{pgfscope}%
\begin{pgfscope}%
\pgfsys@transformshift{2.152727in}{0.882667in}%
\pgfsys@useobject{currentmarker}{}%
\end{pgfscope}%
\begin{pgfscope}%
\pgfsys@transformshift{2.603636in}{0.920000in}%
\pgfsys@useobject{currentmarker}{}%
\end{pgfscope}%
\begin{pgfscope}%
\pgfsys@transformshift{3.054545in}{0.994667in}%
\pgfsys@useobject{currentmarker}{}%
\end{pgfscope}%
\begin{pgfscope}%
\pgfsys@transformshift{3.505455in}{1.032000in}%
\pgfsys@useobject{currentmarker}{}%
\end{pgfscope}%
\begin{pgfscope}%
\pgfsys@transformshift{3.956364in}{1.106667in}%
\pgfsys@useobject{currentmarker}{}%
\end{pgfscope}%
\begin{pgfscope}%
\pgfsys@transformshift{4.407273in}{1.106667in}%
\pgfsys@useobject{currentmarker}{}%
\end{pgfscope}%
\begin{pgfscope}%
\pgfsys@transformshift{4.858182in}{1.181333in}%
\pgfsys@useobject{currentmarker}{}%
\end{pgfscope}%
\begin{pgfscope}%
\pgfsys@transformshift{5.309091in}{1.218667in}%
\pgfsys@useobject{currentmarker}{}%
\end{pgfscope}%
\end{pgfscope}%
\begin{pgfscope}%
\pgfpathrectangle{\pgfqpoint{0.800000in}{0.528000in}}{\pgfqpoint{4.960000in}{3.696000in}}%
\pgfusepath{clip}%
\pgfsetbuttcap%
\pgfsetroundjoin%
\pgfsetlinewidth{1.505625pt}%
\definecolor{currentstroke}{rgb}{0.000000,0.000000,0.000000}%
\pgfsetstrokecolor{currentstroke}%
\pgfsetdash{{5.550000pt}{2.400000pt}}{0.000000pt}%
\pgfpathmoveto{\pgfqpoint{1.250909in}{0.696000in}}%
\pgfpathlineto{\pgfqpoint{1.701818in}{0.733333in}}%
\pgfpathlineto{\pgfqpoint{2.152727in}{0.770667in}}%
\pgfpathlineto{\pgfqpoint{2.603636in}{0.808000in}}%
\pgfpathlineto{\pgfqpoint{3.054545in}{0.845333in}}%
\pgfpathlineto{\pgfqpoint{3.505455in}{0.882667in}}%
\pgfpathlineto{\pgfqpoint{3.956364in}{0.882667in}}%
\pgfpathlineto{\pgfqpoint{4.407273in}{0.920000in}}%
\pgfpathlineto{\pgfqpoint{4.858182in}{0.920000in}}%
\pgfpathlineto{\pgfqpoint{5.309091in}{0.920000in}}%
\pgfusepath{stroke}%
\end{pgfscope}%
\begin{pgfscope}%
\pgfpathrectangle{\pgfqpoint{0.800000in}{0.528000in}}{\pgfqpoint{4.960000in}{3.696000in}}%
\pgfusepath{clip}%
\pgfsetbuttcap%
\pgfsetmiterjoin%
\definecolor{currentfill}{rgb}{0.000000,0.000000,0.000000}%
\pgfsetfillcolor{currentfill}%
\pgfsetlinewidth{1.003750pt}%
\definecolor{currentstroke}{rgb}{0.000000,0.000000,0.000000}%
\pgfsetstrokecolor{currentstroke}%
\pgfsetdash{}{0pt}%
\pgfsys@defobject{currentmarker}{\pgfqpoint{-0.035355in}{-0.058926in}}{\pgfqpoint{0.035355in}{0.058926in}}{%
\pgfpathmoveto{\pgfqpoint{-0.000000in}{-0.058926in}}%
\pgfpathlineto{\pgfqpoint{0.035355in}{0.000000in}}%
\pgfpathlineto{\pgfqpoint{0.000000in}{0.058926in}}%
\pgfpathlineto{\pgfqpoint{-0.035355in}{0.000000in}}%
\pgfpathclose%
\pgfusepath{stroke,fill}%
}%
\begin{pgfscope}%
\pgfsys@transformshift{1.250909in}{0.696000in}%
\pgfsys@useobject{currentmarker}{}%
\end{pgfscope}%
\begin{pgfscope}%
\pgfsys@transformshift{1.701818in}{0.733333in}%
\pgfsys@useobject{currentmarker}{}%
\end{pgfscope}%
\begin{pgfscope}%
\pgfsys@transformshift{2.152727in}{0.770667in}%
\pgfsys@useobject{currentmarker}{}%
\end{pgfscope}%
\begin{pgfscope}%
\pgfsys@transformshift{2.603636in}{0.808000in}%
\pgfsys@useobject{currentmarker}{}%
\end{pgfscope}%
\begin{pgfscope}%
\pgfsys@transformshift{3.054545in}{0.845333in}%
\pgfsys@useobject{currentmarker}{}%
\end{pgfscope}%
\begin{pgfscope}%
\pgfsys@transformshift{3.505455in}{0.882667in}%
\pgfsys@useobject{currentmarker}{}%
\end{pgfscope}%
\begin{pgfscope}%
\pgfsys@transformshift{3.956364in}{0.882667in}%
\pgfsys@useobject{currentmarker}{}%
\end{pgfscope}%
\begin{pgfscope}%
\pgfsys@transformshift{4.407273in}{0.920000in}%
\pgfsys@useobject{currentmarker}{}%
\end{pgfscope}%
\begin{pgfscope}%
\pgfsys@transformshift{4.858182in}{0.920000in}%
\pgfsys@useobject{currentmarker}{}%
\end{pgfscope}%
\begin{pgfscope}%
\pgfsys@transformshift{5.309091in}{0.920000in}%
\pgfsys@useobject{currentmarker}{}%
\end{pgfscope}%
\end{pgfscope}%
\begin{pgfscope}%
\pgfsetrectcap%
\pgfsetmiterjoin%
\pgfsetlinewidth{0.803000pt}%
\definecolor{currentstroke}{rgb}{0.000000,0.000000,0.000000}%
\pgfsetstrokecolor{currentstroke}%
\pgfsetdash{}{0pt}%
\pgfpathmoveto{\pgfqpoint{0.800000in}{0.528000in}}%
\pgfpathlineto{\pgfqpoint{0.800000in}{4.224000in}}%
\pgfusepath{stroke}%
\end{pgfscope}%
\begin{pgfscope}%
\pgfsetrectcap%
\pgfsetmiterjoin%
\pgfsetlinewidth{0.803000pt}%
\definecolor{currentstroke}{rgb}{0.000000,0.000000,0.000000}%
\pgfsetstrokecolor{currentstroke}%
\pgfsetdash{}{0pt}%
\pgfpathmoveto{\pgfqpoint{5.760000in}{0.528000in}}%
\pgfpathlineto{\pgfqpoint{5.760000in}{4.224000in}}%
\pgfusepath{stroke}%
\end{pgfscope}%
\begin{pgfscope}%
\pgfsetrectcap%
\pgfsetmiterjoin%
\pgfsetlinewidth{0.803000pt}%
\definecolor{currentstroke}{rgb}{0.000000,0.000000,0.000000}%
\pgfsetstrokecolor{currentstroke}%
\pgfsetdash{}{0pt}%
\pgfpathmoveto{\pgfqpoint{0.800000in}{0.528000in}}%
\pgfpathlineto{\pgfqpoint{5.760000in}{0.528000in}}%
\pgfusepath{stroke}%
\end{pgfscope}%
\begin{pgfscope}%
\pgfsetrectcap%
\pgfsetmiterjoin%
\pgfsetlinewidth{0.803000pt}%
\definecolor{currentstroke}{rgb}{0.000000,0.000000,0.000000}%
\pgfsetstrokecolor{currentstroke}%
\pgfsetdash{}{0pt}%
\pgfpathmoveto{\pgfqpoint{0.800000in}{4.224000in}}%
\pgfpathlineto{\pgfqpoint{5.760000in}{4.224000in}}%
\pgfusepath{stroke}%
\end{pgfscope}%
\begin{pgfscope}%
\pgfsetbuttcap%
\pgfsetmiterjoin%
\definecolor{currentfill}{rgb}{1.000000,1.000000,1.000000}%
\pgfsetfillcolor{currentfill}%
\pgfsetfillopacity{0.800000}%
\pgfsetlinewidth{1.003750pt}%
\definecolor{currentstroke}{rgb}{0.800000,0.800000,0.800000}%
\pgfsetstrokecolor{currentstroke}%
\pgfsetstrokeopacity{0.800000}%
\pgfsetdash{}{0pt}%
\pgfpathmoveto{\pgfqpoint{0.897222in}{3.297460in}}%
\pgfpathlineto{\pgfqpoint{1.790209in}{3.297460in}}%
\pgfpathquadraticcurveto{\pgfqpoint{1.817987in}{3.297460in}}{\pgfqpoint{1.817987in}{3.325238in}}%
\pgfpathlineto{\pgfqpoint{1.817987in}{4.126778in}}%
\pgfpathquadraticcurveto{\pgfqpoint{1.817987in}{4.154556in}}{\pgfqpoint{1.790209in}{4.154556in}}%
\pgfpathlineto{\pgfqpoint{0.897222in}{4.154556in}}%
\pgfpathquadraticcurveto{\pgfqpoint{0.869444in}{4.154556in}}{\pgfqpoint{0.869444in}{4.126778in}}%
\pgfpathlineto{\pgfqpoint{0.869444in}{3.325238in}}%
\pgfpathquadraticcurveto{\pgfqpoint{0.869444in}{3.297460in}}{\pgfqpoint{0.897222in}{3.297460in}}%
\pgfpathclose%
\pgfusepath{stroke,fill}%
\end{pgfscope}%
\begin{pgfscope}%
\pgfsetbuttcap%
\pgfsetroundjoin%
\pgfsetlinewidth{1.505625pt}%
\definecolor{currentstroke}{rgb}{0.000000,0.000000,0.000000}%
\pgfsetstrokecolor{currentstroke}%
\pgfsetdash{{5.550000pt}{2.400000pt}}{0.000000pt}%
\pgfpathmoveto{\pgfqpoint{0.925000in}{4.042088in}}%
\pgfpathlineto{\pgfqpoint{1.202778in}{4.042088in}}%
\pgfusepath{stroke}%
\end{pgfscope}%
\begin{pgfscope}%
\pgfsetbuttcap%
\pgfsetroundjoin%
\definecolor{currentfill}{rgb}{0.000000,0.000000,0.000000}%
\pgfsetfillcolor{currentfill}%
\pgfsetlinewidth{1.003750pt}%
\definecolor{currentstroke}{rgb}{0.000000,0.000000,0.000000}%
\pgfsetstrokecolor{currentstroke}%
\pgfsetdash{}{0pt}%
\pgfsys@defobject{currentmarker}{\pgfqpoint{-0.041667in}{-0.041667in}}{\pgfqpoint{0.041667in}{0.041667in}}{%
\pgfpathmoveto{\pgfqpoint{0.000000in}{-0.041667in}}%
\pgfpathcurveto{\pgfqpoint{0.011050in}{-0.041667in}}{\pgfqpoint{0.021649in}{-0.037276in}}{\pgfqpoint{0.029463in}{-0.029463in}}%
\pgfpathcurveto{\pgfqpoint{0.037276in}{-0.021649in}}{\pgfqpoint{0.041667in}{-0.011050in}}{\pgfqpoint{0.041667in}{0.000000in}}%
\pgfpathcurveto{\pgfqpoint{0.041667in}{0.011050in}}{\pgfqpoint{0.037276in}{0.021649in}}{\pgfqpoint{0.029463in}{0.029463in}}%
\pgfpathcurveto{\pgfqpoint{0.021649in}{0.037276in}}{\pgfqpoint{0.011050in}{0.041667in}}{\pgfqpoint{0.000000in}{0.041667in}}%
\pgfpathcurveto{\pgfqpoint{-0.011050in}{0.041667in}}{\pgfqpoint{-0.021649in}{0.037276in}}{\pgfqpoint{-0.029463in}{0.029463in}}%
\pgfpathcurveto{\pgfqpoint{-0.037276in}{0.021649in}}{\pgfqpoint{-0.041667in}{0.011050in}}{\pgfqpoint{-0.041667in}{0.000000in}}%
\pgfpathcurveto{\pgfqpoint{-0.041667in}{-0.011050in}}{\pgfqpoint{-0.037276in}{-0.021649in}}{\pgfqpoint{-0.029463in}{-0.029463in}}%
\pgfpathcurveto{\pgfqpoint{-0.021649in}{-0.037276in}}{\pgfqpoint{-0.011050in}{-0.041667in}}{\pgfqpoint{0.000000in}{-0.041667in}}%
\pgfpathclose%
\pgfusepath{stroke,fill}%
}%
\begin{pgfscope}%
\pgfsys@transformshift{1.063889in}{4.042088in}%
\pgfsys@useobject{currentmarker}{}%
\end{pgfscope}%
\end{pgfscope}%
\begin{pgfscope}%
\definecolor{textcolor}{rgb}{0.000000,0.000000,0.000000}%
\pgfsetstrokecolor{textcolor}%
\pgfsetfillcolor{textcolor}%
\pgftext[x=1.313889in,y=3.993477in,left,base]{\color{textcolor}\sffamily\fontsize{10.000000}{12.000000}\selectfont \(\displaystyle \beta = 0.4\)}%
\end{pgfscope}%
\begin{pgfscope}%
\pgfsetbuttcap%
\pgfsetroundjoin%
\pgfsetlinewidth{1.505625pt}%
\definecolor{currentstroke}{rgb}{0.000000,0.000000,0.000000}%
\pgfsetstrokecolor{currentstroke}%
\pgfsetdash{{5.550000pt}{2.400000pt}}{0.000000pt}%
\pgfpathmoveto{\pgfqpoint{0.925000in}{3.838231in}}%
\pgfpathlineto{\pgfqpoint{1.202778in}{3.838231in}}%
\pgfusepath{stroke}%
\end{pgfscope}%
\begin{pgfscope}%
\pgfsetbuttcap%
\pgfsetmiterjoin%
\definecolor{currentfill}{rgb}{0.000000,0.000000,0.000000}%
\pgfsetfillcolor{currentfill}%
\pgfsetlinewidth{1.003750pt}%
\definecolor{currentstroke}{rgb}{0.000000,0.000000,0.000000}%
\pgfsetstrokecolor{currentstroke}%
\pgfsetdash{}{0pt}%
\pgfsys@defobject{currentmarker}{\pgfqpoint{-0.041667in}{-0.041667in}}{\pgfqpoint{0.041667in}{0.041667in}}{%
\pgfpathmoveto{\pgfqpoint{-0.000000in}{-0.041667in}}%
\pgfpathlineto{\pgfqpoint{0.041667in}{0.041667in}}%
\pgfpathlineto{\pgfqpoint{-0.041667in}{0.041667in}}%
\pgfpathclose%
\pgfusepath{stroke,fill}%
}%
\begin{pgfscope}%
\pgfsys@transformshift{1.063889in}{3.838231in}%
\pgfsys@useobject{currentmarker}{}%
\end{pgfscope}%
\end{pgfscope}%
\begin{pgfscope}%
\definecolor{textcolor}{rgb}{0.000000,0.000000,0.000000}%
\pgfsetstrokecolor{textcolor}%
\pgfsetfillcolor{textcolor}%
\pgftext[x=1.313889in,y=3.789620in,left,base]{\color{textcolor}\sffamily\fontsize{10.000000}{12.000000}\selectfont \(\displaystyle \beta = 0.5\)}%
\end{pgfscope}%
\begin{pgfscope}%
\pgfsetbuttcap%
\pgfsetroundjoin%
\pgfsetlinewidth{1.505625pt}%
\definecolor{currentstroke}{rgb}{0.000000,0.000000,0.000000}%
\pgfsetstrokecolor{currentstroke}%
\pgfsetdash{{5.550000pt}{2.400000pt}}{0.000000pt}%
\pgfpathmoveto{\pgfqpoint{0.925000in}{3.634374in}}%
\pgfpathlineto{\pgfqpoint{1.202778in}{3.634374in}}%
\pgfusepath{stroke}%
\end{pgfscope}%
\begin{pgfscope}%
\pgfsetbuttcap%
\pgfsetmiterjoin%
\definecolor{currentfill}{rgb}{0.000000,0.000000,0.000000}%
\pgfsetfillcolor{currentfill}%
\pgfsetlinewidth{1.003750pt}%
\definecolor{currentstroke}{rgb}{0.000000,0.000000,0.000000}%
\pgfsetstrokecolor{currentstroke}%
\pgfsetdash{}{0pt}%
\pgfsys@defobject{currentmarker}{\pgfqpoint{-0.041667in}{-0.041667in}}{\pgfqpoint{0.041667in}{0.041667in}}{%
\pgfpathmoveto{\pgfqpoint{-0.041667in}{-0.041667in}}%
\pgfpathlineto{\pgfqpoint{0.041667in}{-0.041667in}}%
\pgfpathlineto{\pgfqpoint{0.041667in}{0.041667in}}%
\pgfpathlineto{\pgfqpoint{-0.041667in}{0.041667in}}%
\pgfpathclose%
\pgfusepath{stroke,fill}%
}%
\begin{pgfscope}%
\pgfsys@transformshift{1.063889in}{3.634374in}%
\pgfsys@useobject{currentmarker}{}%
\end{pgfscope}%
\end{pgfscope}%
\begin{pgfscope}%
\definecolor{textcolor}{rgb}{0.000000,0.000000,0.000000}%
\pgfsetstrokecolor{textcolor}%
\pgfsetfillcolor{textcolor}%
\pgftext[x=1.313889in,y=3.585762in,left,base]{\color{textcolor}\sffamily\fontsize{10.000000}{12.000000}\selectfont \(\displaystyle \beta = 0.6\)}%
\end{pgfscope}%
\begin{pgfscope}%
\pgfsetbuttcap%
\pgfsetroundjoin%
\pgfsetlinewidth{1.505625pt}%
\definecolor{currentstroke}{rgb}{0.000000,0.000000,0.000000}%
\pgfsetstrokecolor{currentstroke}%
\pgfsetdash{{5.550000pt}{2.400000pt}}{0.000000pt}%
\pgfpathmoveto{\pgfqpoint{0.925000in}{3.430516in}}%
\pgfpathlineto{\pgfqpoint{1.202778in}{3.430516in}}%
\pgfusepath{stroke}%
\end{pgfscope}%
\begin{pgfscope}%
\pgfsetbuttcap%
\pgfsetmiterjoin%
\definecolor{currentfill}{rgb}{0.000000,0.000000,0.000000}%
\pgfsetfillcolor{currentfill}%
\pgfsetlinewidth{1.003750pt}%
\definecolor{currentstroke}{rgb}{0.000000,0.000000,0.000000}%
\pgfsetstrokecolor{currentstroke}%
\pgfsetdash{}{0pt}%
\pgfsys@defobject{currentmarker}{\pgfqpoint{-0.035355in}{-0.058926in}}{\pgfqpoint{0.035355in}{0.058926in}}{%
\pgfpathmoveto{\pgfqpoint{-0.000000in}{-0.058926in}}%
\pgfpathlineto{\pgfqpoint{0.035355in}{0.000000in}}%
\pgfpathlineto{\pgfqpoint{0.000000in}{0.058926in}}%
\pgfpathlineto{\pgfqpoint{-0.035355in}{0.000000in}}%
\pgfpathclose%
\pgfusepath{stroke,fill}%
}%
\begin{pgfscope}%
\pgfsys@transformshift{1.063889in}{3.430516in}%
\pgfsys@useobject{currentmarker}{}%
\end{pgfscope}%
\end{pgfscope}%
\begin{pgfscope}%
\definecolor{textcolor}{rgb}{0.000000,0.000000,0.000000}%
\pgfsetstrokecolor{textcolor}%
\pgfsetfillcolor{textcolor}%
\pgftext[x=1.313889in,y=3.381905in,left,base]{\color{textcolor}\sffamily\fontsize{10.000000}{12.000000}\selectfont \(\displaystyle \beta = 0.7\)}%
\end{pgfscope}%
\end{pgfpicture}%
\makeatother%
\endgroup%

    \caption{GMRES iteration counts for $\AmatoI\Amatt$ given by \cref{eq:noweak,eq:ntweak}, where $\alpha = 0.2/k^\beta,$ for $\beta = 0.4,0.5,0.6,0.7.$}\label{fig:l1med}
\end{figure}
    
    \begin{figure}
    %% Creator: Matplotlib, PGF backend
%%
%% To include the figure in your LaTeX document, write
%%   \input{<filename>.pgf}
%%
%% Make sure the required packages are loaded in your preamble
%%   \usepackage{pgf}
%%
%% Figures using additional raster images can only be included by \input if
%% they are in the same directory as the main LaTeX file. For loading figures
%% from other directories you can use the `import` package
%%   \usepackage{import}
%% and then include the figures with
%%   \import{<path to file>}{<filename>.pgf}
%%
%% Matplotlib used the following preamble
%%   \usepackage{fontspec}
%%   \setmainfont{DejaVuSerif.ttf}[Path=/home/owen/progs/firedrake-complex/firedrake/lib/python3.5/site-packages/matplotlib/mpl-data/fonts/ttf/]
%%   \setsansfont{DejaVuSans.ttf}[Path=/home/owen/progs/firedrake-complex/firedrake/lib/python3.5/site-packages/matplotlib/mpl-data/fonts/ttf/]
%%   \setmonofont{DejaVuSansMono.ttf}[Path=/home/owen/progs/firedrake-complex/firedrake/lib/python3.5/site-packages/matplotlib/mpl-data/fonts/ttf/]
%%
\begingroup%
\makeatletter%
\begin{pgfpicture}%
\pgfpathrectangle{\pgfpointorigin}{\pgfqpoint{5.500000in}{5.500000in}}%
\pgfusepath{use as bounding box, clip}%
\begin{pgfscope}%
\pgfsetbuttcap%
\pgfsetmiterjoin%
\definecolor{currentfill}{rgb}{1.000000,1.000000,1.000000}%
\pgfsetfillcolor{currentfill}%
\pgfsetlinewidth{0.000000pt}%
\definecolor{currentstroke}{rgb}{1.000000,1.000000,1.000000}%
\pgfsetstrokecolor{currentstroke}%
\pgfsetdash{}{0pt}%
\pgfpathmoveto{\pgfqpoint{0.000000in}{0.000000in}}%
\pgfpathlineto{\pgfqpoint{5.500000in}{0.000000in}}%
\pgfpathlineto{\pgfqpoint{5.500000in}{5.500000in}}%
\pgfpathlineto{\pgfqpoint{0.000000in}{5.500000in}}%
\pgfpathclose%
\pgfusepath{fill}%
\end{pgfscope}%
\begin{pgfscope}%
\pgfsetbuttcap%
\pgfsetmiterjoin%
\definecolor{currentfill}{rgb}{1.000000,1.000000,1.000000}%
\pgfsetfillcolor{currentfill}%
\pgfsetlinewidth{0.000000pt}%
\definecolor{currentstroke}{rgb}{0.000000,0.000000,0.000000}%
\pgfsetstrokecolor{currentstroke}%
\pgfsetstrokeopacity{0.000000}%
\pgfsetdash{}{0pt}%
\pgfpathmoveto{\pgfqpoint{0.687500in}{0.605000in}}%
\pgfpathlineto{\pgfqpoint{4.950000in}{0.605000in}}%
\pgfpathlineto{\pgfqpoint{4.950000in}{4.840000in}}%
\pgfpathlineto{\pgfqpoint{0.687500in}{4.840000in}}%
\pgfpathclose%
\pgfusepath{fill}%
\end{pgfscope}%
\begin{pgfscope}%
\pgfsetbuttcap%
\pgfsetroundjoin%
\definecolor{currentfill}{rgb}{0.000000,0.000000,0.000000}%
\pgfsetfillcolor{currentfill}%
\pgfsetlinewidth{0.803000pt}%
\definecolor{currentstroke}{rgb}{0.000000,0.000000,0.000000}%
\pgfsetstrokecolor{currentstroke}%
\pgfsetdash{}{0pt}%
\pgfsys@defobject{currentmarker}{\pgfqpoint{0.000000in}{-0.048611in}}{\pgfqpoint{0.000000in}{0.000000in}}{%
\pgfpathmoveto{\pgfqpoint{0.000000in}{0.000000in}}%
\pgfpathlineto{\pgfqpoint{0.000000in}{-0.048611in}}%
\pgfusepath{stroke,fill}%
}%
\begin{pgfscope}%
\pgfsys@transformshift{1.075000in}{0.605000in}%
\pgfsys@useobject{currentmarker}{}%
\end{pgfscope}%
\end{pgfscope}%
\begin{pgfscope}%
\definecolor{textcolor}{rgb}{0.000000,0.000000,0.000000}%
\pgfsetstrokecolor{textcolor}%
\pgfsetfillcolor{textcolor}%
\pgftext[x=1.075000in,y=0.507778in,,top]{\color{textcolor}\sffamily\fontsize{10.000000}{12.000000}\selectfont \(\displaystyle 10\)}%
\end{pgfscope}%
\begin{pgfscope}%
\pgfsetbuttcap%
\pgfsetroundjoin%
\definecolor{currentfill}{rgb}{0.000000,0.000000,0.000000}%
\pgfsetfillcolor{currentfill}%
\pgfsetlinewidth{0.803000pt}%
\definecolor{currentstroke}{rgb}{0.000000,0.000000,0.000000}%
\pgfsetstrokecolor{currentstroke}%
\pgfsetdash{}{0pt}%
\pgfsys@defobject{currentmarker}{\pgfqpoint{0.000000in}{-0.048611in}}{\pgfqpoint{0.000000in}{0.000000in}}{%
\pgfpathmoveto{\pgfqpoint{0.000000in}{0.000000in}}%
\pgfpathlineto{\pgfqpoint{0.000000in}{-0.048611in}}%
\pgfusepath{stroke,fill}%
}%
\begin{pgfscope}%
\pgfsys@transformshift{1.462500in}{0.605000in}%
\pgfsys@useobject{currentmarker}{}%
\end{pgfscope}%
\end{pgfscope}%
\begin{pgfscope}%
\definecolor{textcolor}{rgb}{0.000000,0.000000,0.000000}%
\pgfsetstrokecolor{textcolor}%
\pgfsetfillcolor{textcolor}%
\pgftext[x=1.462500in,y=0.507778in,,top]{\color{textcolor}\sffamily\fontsize{10.000000}{12.000000}\selectfont \(\displaystyle 20\)}%
\end{pgfscope}%
\begin{pgfscope}%
\pgfsetbuttcap%
\pgfsetroundjoin%
\definecolor{currentfill}{rgb}{0.000000,0.000000,0.000000}%
\pgfsetfillcolor{currentfill}%
\pgfsetlinewidth{0.803000pt}%
\definecolor{currentstroke}{rgb}{0.000000,0.000000,0.000000}%
\pgfsetstrokecolor{currentstroke}%
\pgfsetdash{}{0pt}%
\pgfsys@defobject{currentmarker}{\pgfqpoint{0.000000in}{-0.048611in}}{\pgfqpoint{0.000000in}{0.000000in}}{%
\pgfpathmoveto{\pgfqpoint{0.000000in}{0.000000in}}%
\pgfpathlineto{\pgfqpoint{0.000000in}{-0.048611in}}%
\pgfusepath{stroke,fill}%
}%
\begin{pgfscope}%
\pgfsys@transformshift{1.850000in}{0.605000in}%
\pgfsys@useobject{currentmarker}{}%
\end{pgfscope}%
\end{pgfscope}%
\begin{pgfscope}%
\definecolor{textcolor}{rgb}{0.000000,0.000000,0.000000}%
\pgfsetstrokecolor{textcolor}%
\pgfsetfillcolor{textcolor}%
\pgftext[x=1.850000in,y=0.507778in,,top]{\color{textcolor}\sffamily\fontsize{10.000000}{12.000000}\selectfont \(\displaystyle 30\)}%
\end{pgfscope}%
\begin{pgfscope}%
\pgfsetbuttcap%
\pgfsetroundjoin%
\definecolor{currentfill}{rgb}{0.000000,0.000000,0.000000}%
\pgfsetfillcolor{currentfill}%
\pgfsetlinewidth{0.803000pt}%
\definecolor{currentstroke}{rgb}{0.000000,0.000000,0.000000}%
\pgfsetstrokecolor{currentstroke}%
\pgfsetdash{}{0pt}%
\pgfsys@defobject{currentmarker}{\pgfqpoint{0.000000in}{-0.048611in}}{\pgfqpoint{0.000000in}{0.000000in}}{%
\pgfpathmoveto{\pgfqpoint{0.000000in}{0.000000in}}%
\pgfpathlineto{\pgfqpoint{0.000000in}{-0.048611in}}%
\pgfusepath{stroke,fill}%
}%
\begin{pgfscope}%
\pgfsys@transformshift{2.237500in}{0.605000in}%
\pgfsys@useobject{currentmarker}{}%
\end{pgfscope}%
\end{pgfscope}%
\begin{pgfscope}%
\definecolor{textcolor}{rgb}{0.000000,0.000000,0.000000}%
\pgfsetstrokecolor{textcolor}%
\pgfsetfillcolor{textcolor}%
\pgftext[x=2.237500in,y=0.507778in,,top]{\color{textcolor}\sffamily\fontsize{10.000000}{12.000000}\selectfont \(\displaystyle 40\)}%
\end{pgfscope}%
\begin{pgfscope}%
\pgfsetbuttcap%
\pgfsetroundjoin%
\definecolor{currentfill}{rgb}{0.000000,0.000000,0.000000}%
\pgfsetfillcolor{currentfill}%
\pgfsetlinewidth{0.803000pt}%
\definecolor{currentstroke}{rgb}{0.000000,0.000000,0.000000}%
\pgfsetstrokecolor{currentstroke}%
\pgfsetdash{}{0pt}%
\pgfsys@defobject{currentmarker}{\pgfqpoint{0.000000in}{-0.048611in}}{\pgfqpoint{0.000000in}{0.000000in}}{%
\pgfpathmoveto{\pgfqpoint{0.000000in}{0.000000in}}%
\pgfpathlineto{\pgfqpoint{0.000000in}{-0.048611in}}%
\pgfusepath{stroke,fill}%
}%
\begin{pgfscope}%
\pgfsys@transformshift{2.625000in}{0.605000in}%
\pgfsys@useobject{currentmarker}{}%
\end{pgfscope}%
\end{pgfscope}%
\begin{pgfscope}%
\definecolor{textcolor}{rgb}{0.000000,0.000000,0.000000}%
\pgfsetstrokecolor{textcolor}%
\pgfsetfillcolor{textcolor}%
\pgftext[x=2.625000in,y=0.507778in,,top]{\color{textcolor}\sffamily\fontsize{10.000000}{12.000000}\selectfont \(\displaystyle 50\)}%
\end{pgfscope}%
\begin{pgfscope}%
\pgfsetbuttcap%
\pgfsetroundjoin%
\definecolor{currentfill}{rgb}{0.000000,0.000000,0.000000}%
\pgfsetfillcolor{currentfill}%
\pgfsetlinewidth{0.803000pt}%
\definecolor{currentstroke}{rgb}{0.000000,0.000000,0.000000}%
\pgfsetstrokecolor{currentstroke}%
\pgfsetdash{}{0pt}%
\pgfsys@defobject{currentmarker}{\pgfqpoint{0.000000in}{-0.048611in}}{\pgfqpoint{0.000000in}{0.000000in}}{%
\pgfpathmoveto{\pgfqpoint{0.000000in}{0.000000in}}%
\pgfpathlineto{\pgfqpoint{0.000000in}{-0.048611in}}%
\pgfusepath{stroke,fill}%
}%
\begin{pgfscope}%
\pgfsys@transformshift{3.012500in}{0.605000in}%
\pgfsys@useobject{currentmarker}{}%
\end{pgfscope}%
\end{pgfscope}%
\begin{pgfscope}%
\definecolor{textcolor}{rgb}{0.000000,0.000000,0.000000}%
\pgfsetstrokecolor{textcolor}%
\pgfsetfillcolor{textcolor}%
\pgftext[x=3.012500in,y=0.507778in,,top]{\color{textcolor}\sffamily\fontsize{10.000000}{12.000000}\selectfont \(\displaystyle 60\)}%
\end{pgfscope}%
\begin{pgfscope}%
\pgfsetbuttcap%
\pgfsetroundjoin%
\definecolor{currentfill}{rgb}{0.000000,0.000000,0.000000}%
\pgfsetfillcolor{currentfill}%
\pgfsetlinewidth{0.803000pt}%
\definecolor{currentstroke}{rgb}{0.000000,0.000000,0.000000}%
\pgfsetstrokecolor{currentstroke}%
\pgfsetdash{}{0pt}%
\pgfsys@defobject{currentmarker}{\pgfqpoint{0.000000in}{-0.048611in}}{\pgfqpoint{0.000000in}{0.000000in}}{%
\pgfpathmoveto{\pgfqpoint{0.000000in}{0.000000in}}%
\pgfpathlineto{\pgfqpoint{0.000000in}{-0.048611in}}%
\pgfusepath{stroke,fill}%
}%
\begin{pgfscope}%
\pgfsys@transformshift{3.400000in}{0.605000in}%
\pgfsys@useobject{currentmarker}{}%
\end{pgfscope}%
\end{pgfscope}%
\begin{pgfscope}%
\definecolor{textcolor}{rgb}{0.000000,0.000000,0.000000}%
\pgfsetstrokecolor{textcolor}%
\pgfsetfillcolor{textcolor}%
\pgftext[x=3.400000in,y=0.507778in,,top]{\color{textcolor}\sffamily\fontsize{10.000000}{12.000000}\selectfont \(\displaystyle 70\)}%
\end{pgfscope}%
\begin{pgfscope}%
\pgfsetbuttcap%
\pgfsetroundjoin%
\definecolor{currentfill}{rgb}{0.000000,0.000000,0.000000}%
\pgfsetfillcolor{currentfill}%
\pgfsetlinewidth{0.803000pt}%
\definecolor{currentstroke}{rgb}{0.000000,0.000000,0.000000}%
\pgfsetstrokecolor{currentstroke}%
\pgfsetdash{}{0pt}%
\pgfsys@defobject{currentmarker}{\pgfqpoint{0.000000in}{-0.048611in}}{\pgfqpoint{0.000000in}{0.000000in}}{%
\pgfpathmoveto{\pgfqpoint{0.000000in}{0.000000in}}%
\pgfpathlineto{\pgfqpoint{0.000000in}{-0.048611in}}%
\pgfusepath{stroke,fill}%
}%
\begin{pgfscope}%
\pgfsys@transformshift{3.787500in}{0.605000in}%
\pgfsys@useobject{currentmarker}{}%
\end{pgfscope}%
\end{pgfscope}%
\begin{pgfscope}%
\definecolor{textcolor}{rgb}{0.000000,0.000000,0.000000}%
\pgfsetstrokecolor{textcolor}%
\pgfsetfillcolor{textcolor}%
\pgftext[x=3.787500in,y=0.507778in,,top]{\color{textcolor}\sffamily\fontsize{10.000000}{12.000000}\selectfont \(\displaystyle 80\)}%
\end{pgfscope}%
\begin{pgfscope}%
\pgfsetbuttcap%
\pgfsetroundjoin%
\definecolor{currentfill}{rgb}{0.000000,0.000000,0.000000}%
\pgfsetfillcolor{currentfill}%
\pgfsetlinewidth{0.803000pt}%
\definecolor{currentstroke}{rgb}{0.000000,0.000000,0.000000}%
\pgfsetstrokecolor{currentstroke}%
\pgfsetdash{}{0pt}%
\pgfsys@defobject{currentmarker}{\pgfqpoint{0.000000in}{-0.048611in}}{\pgfqpoint{0.000000in}{0.000000in}}{%
\pgfpathmoveto{\pgfqpoint{0.000000in}{0.000000in}}%
\pgfpathlineto{\pgfqpoint{0.000000in}{-0.048611in}}%
\pgfusepath{stroke,fill}%
}%
\begin{pgfscope}%
\pgfsys@transformshift{4.175000in}{0.605000in}%
\pgfsys@useobject{currentmarker}{}%
\end{pgfscope}%
\end{pgfscope}%
\begin{pgfscope}%
\definecolor{textcolor}{rgb}{0.000000,0.000000,0.000000}%
\pgfsetstrokecolor{textcolor}%
\pgfsetfillcolor{textcolor}%
\pgftext[x=4.175000in,y=0.507778in,,top]{\color{textcolor}\sffamily\fontsize{10.000000}{12.000000}\selectfont \(\displaystyle 90\)}%
\end{pgfscope}%
\begin{pgfscope}%
\pgfsetbuttcap%
\pgfsetroundjoin%
\definecolor{currentfill}{rgb}{0.000000,0.000000,0.000000}%
\pgfsetfillcolor{currentfill}%
\pgfsetlinewidth{0.803000pt}%
\definecolor{currentstroke}{rgb}{0.000000,0.000000,0.000000}%
\pgfsetstrokecolor{currentstroke}%
\pgfsetdash{}{0pt}%
\pgfsys@defobject{currentmarker}{\pgfqpoint{0.000000in}{-0.048611in}}{\pgfqpoint{0.000000in}{0.000000in}}{%
\pgfpathmoveto{\pgfqpoint{0.000000in}{0.000000in}}%
\pgfpathlineto{\pgfqpoint{0.000000in}{-0.048611in}}%
\pgfusepath{stroke,fill}%
}%
\begin{pgfscope}%
\pgfsys@transformshift{4.562500in}{0.605000in}%
\pgfsys@useobject{currentmarker}{}%
\end{pgfscope}%
\end{pgfscope}%
\begin{pgfscope}%
\definecolor{textcolor}{rgb}{0.000000,0.000000,0.000000}%
\pgfsetstrokecolor{textcolor}%
\pgfsetfillcolor{textcolor}%
\pgftext[x=4.562500in,y=0.507778in,,top]{\color{textcolor}\sffamily\fontsize{10.000000}{12.000000}\selectfont \(\displaystyle 100\)}%
\end{pgfscope}%
\begin{pgfscope}%
\definecolor{textcolor}{rgb}{0.000000,0.000000,0.000000}%
\pgfsetstrokecolor{textcolor}%
\pgfsetfillcolor{textcolor}%
\pgftext[x=2.818750in,y=0.317809in,,top]{\color{textcolor}\sffamily\fontsize{10.000000}{12.000000}\selectfont \(\displaystyle k\)}%
\end{pgfscope}%
\begin{pgfscope}%
\pgfsetbuttcap%
\pgfsetroundjoin%
\definecolor{currentfill}{rgb}{0.000000,0.000000,0.000000}%
\pgfsetfillcolor{currentfill}%
\pgfsetlinewidth{0.803000pt}%
\definecolor{currentstroke}{rgb}{0.000000,0.000000,0.000000}%
\pgfsetstrokecolor{currentstroke}%
\pgfsetdash{}{0pt}%
\pgfsys@defobject{currentmarker}{\pgfqpoint{-0.048611in}{0.000000in}}{\pgfqpoint{0.000000in}{0.000000in}}{%
\pgfpathmoveto{\pgfqpoint{0.000000in}{0.000000in}}%
\pgfpathlineto{\pgfqpoint{-0.048611in}{0.000000in}}%
\pgfusepath{stroke,fill}%
}%
\begin{pgfscope}%
\pgfsys@transformshift{0.687500in}{0.797500in}%
\pgfsys@useobject{currentmarker}{}%
\end{pgfscope}%
\end{pgfscope}%
\begin{pgfscope}%
\definecolor{textcolor}{rgb}{0.000000,0.000000,0.000000}%
\pgfsetstrokecolor{textcolor}%
\pgfsetfillcolor{textcolor}%
\pgftext[x=0.520833in,y=0.744738in,left,base]{\color{textcolor}\sffamily\fontsize{10.000000}{12.000000}\selectfont \(\displaystyle 6\)}%
\end{pgfscope}%
\begin{pgfscope}%
\pgfsetbuttcap%
\pgfsetroundjoin%
\definecolor{currentfill}{rgb}{0.000000,0.000000,0.000000}%
\pgfsetfillcolor{currentfill}%
\pgfsetlinewidth{0.803000pt}%
\definecolor{currentstroke}{rgb}{0.000000,0.000000,0.000000}%
\pgfsetstrokecolor{currentstroke}%
\pgfsetdash{}{0pt}%
\pgfsys@defobject{currentmarker}{\pgfqpoint{-0.048611in}{0.000000in}}{\pgfqpoint{0.000000in}{0.000000in}}{%
\pgfpathmoveto{\pgfqpoint{0.000000in}{0.000000in}}%
\pgfpathlineto{\pgfqpoint{-0.048611in}{0.000000in}}%
\pgfusepath{stroke,fill}%
}%
\begin{pgfscope}%
\pgfsys@transformshift{0.687500in}{1.760000in}%
\pgfsys@useobject{currentmarker}{}%
\end{pgfscope}%
\end{pgfscope}%
\begin{pgfscope}%
\definecolor{textcolor}{rgb}{0.000000,0.000000,0.000000}%
\pgfsetstrokecolor{textcolor}%
\pgfsetfillcolor{textcolor}%
\pgftext[x=0.520833in,y=1.707238in,left,base]{\color{textcolor}\sffamily\fontsize{10.000000}{12.000000}\selectfont \(\displaystyle 7\)}%
\end{pgfscope}%
\begin{pgfscope}%
\pgfsetbuttcap%
\pgfsetroundjoin%
\definecolor{currentfill}{rgb}{0.000000,0.000000,0.000000}%
\pgfsetfillcolor{currentfill}%
\pgfsetlinewidth{0.803000pt}%
\definecolor{currentstroke}{rgb}{0.000000,0.000000,0.000000}%
\pgfsetstrokecolor{currentstroke}%
\pgfsetdash{}{0pt}%
\pgfsys@defobject{currentmarker}{\pgfqpoint{-0.048611in}{0.000000in}}{\pgfqpoint{0.000000in}{0.000000in}}{%
\pgfpathmoveto{\pgfqpoint{0.000000in}{0.000000in}}%
\pgfpathlineto{\pgfqpoint{-0.048611in}{0.000000in}}%
\pgfusepath{stroke,fill}%
}%
\begin{pgfscope}%
\pgfsys@transformshift{0.687500in}{2.722500in}%
\pgfsys@useobject{currentmarker}{}%
\end{pgfscope}%
\end{pgfscope}%
\begin{pgfscope}%
\definecolor{textcolor}{rgb}{0.000000,0.000000,0.000000}%
\pgfsetstrokecolor{textcolor}%
\pgfsetfillcolor{textcolor}%
\pgftext[x=0.520833in,y=2.669738in,left,base]{\color{textcolor}\sffamily\fontsize{10.000000}{12.000000}\selectfont \(\displaystyle 8\)}%
\end{pgfscope}%
\begin{pgfscope}%
\pgfsetbuttcap%
\pgfsetroundjoin%
\definecolor{currentfill}{rgb}{0.000000,0.000000,0.000000}%
\pgfsetfillcolor{currentfill}%
\pgfsetlinewidth{0.803000pt}%
\definecolor{currentstroke}{rgb}{0.000000,0.000000,0.000000}%
\pgfsetstrokecolor{currentstroke}%
\pgfsetdash{}{0pt}%
\pgfsys@defobject{currentmarker}{\pgfqpoint{-0.048611in}{0.000000in}}{\pgfqpoint{0.000000in}{0.000000in}}{%
\pgfpathmoveto{\pgfqpoint{0.000000in}{0.000000in}}%
\pgfpathlineto{\pgfqpoint{-0.048611in}{0.000000in}}%
\pgfusepath{stroke,fill}%
}%
\begin{pgfscope}%
\pgfsys@transformshift{0.687500in}{3.685000in}%
\pgfsys@useobject{currentmarker}{}%
\end{pgfscope}%
\end{pgfscope}%
\begin{pgfscope}%
\definecolor{textcolor}{rgb}{0.000000,0.000000,0.000000}%
\pgfsetstrokecolor{textcolor}%
\pgfsetfillcolor{textcolor}%
\pgftext[x=0.520833in,y=3.632238in,left,base]{\color{textcolor}\sffamily\fontsize{10.000000}{12.000000}\selectfont \(\displaystyle 9\)}%
\end{pgfscope}%
\begin{pgfscope}%
\pgfsetbuttcap%
\pgfsetroundjoin%
\definecolor{currentfill}{rgb}{0.000000,0.000000,0.000000}%
\pgfsetfillcolor{currentfill}%
\pgfsetlinewidth{0.803000pt}%
\definecolor{currentstroke}{rgb}{0.000000,0.000000,0.000000}%
\pgfsetstrokecolor{currentstroke}%
\pgfsetdash{}{0pt}%
\pgfsys@defobject{currentmarker}{\pgfqpoint{-0.048611in}{0.000000in}}{\pgfqpoint{0.000000in}{0.000000in}}{%
\pgfpathmoveto{\pgfqpoint{0.000000in}{0.000000in}}%
\pgfpathlineto{\pgfqpoint{-0.048611in}{0.000000in}}%
\pgfusepath{stroke,fill}%
}%
\begin{pgfscope}%
\pgfsys@transformshift{0.687500in}{4.647500in}%
\pgfsys@useobject{currentmarker}{}%
\end{pgfscope}%
\end{pgfscope}%
\begin{pgfscope}%
\definecolor{textcolor}{rgb}{0.000000,0.000000,0.000000}%
\pgfsetstrokecolor{textcolor}%
\pgfsetfillcolor{textcolor}%
\pgftext[x=0.451388in,y=4.594738in,left,base]{\color{textcolor}\sffamily\fontsize{10.000000}{12.000000}\selectfont \(\displaystyle 10\)}%
\end{pgfscope}%
\begin{pgfscope}%
\definecolor{textcolor}{rgb}{0.000000,0.000000,0.000000}%
\pgfsetstrokecolor{textcolor}%
\pgfsetfillcolor{textcolor}%
\pgftext[x=0.395833in,y=2.722500in,,bottom,rotate=90.000000]{\color{textcolor}\sffamily\fontsize{10.000000}{12.000000}\selectfont Number of GMRES iterations}%
\end{pgfscope}%
\begin{pgfscope}%
\pgfpathrectangle{\pgfqpoint{0.687500in}{0.605000in}}{\pgfqpoint{4.262500in}{4.235000in}}%
\pgfusepath{clip}%
\pgfsetbuttcap%
\pgfsetroundjoin%
\pgfsetlinewidth{1.505625pt}%
\definecolor{currentstroke}{rgb}{0.843137,0.000000,0.000000}%
\pgfsetstrokecolor{currentstroke}%
\pgfsetdash{{5.550000pt}{2.400000pt}}{0.000000pt}%
\pgfpathmoveto{\pgfqpoint{1.075000in}{2.722500in}}%
\pgfpathlineto{\pgfqpoint{1.462500in}{2.722500in}}%
\pgfpathlineto{\pgfqpoint{1.850000in}{3.685000in}}%
\pgfpathlineto{\pgfqpoint{2.237500in}{3.685000in}}%
\pgfpathlineto{\pgfqpoint{2.625000in}{4.647500in}}%
\pgfpathlineto{\pgfqpoint{3.012500in}{4.647500in}}%
\pgfpathlineto{\pgfqpoint{3.400000in}{4.647500in}}%
\pgfpathlineto{\pgfqpoint{3.787500in}{4.647500in}}%
\pgfpathlineto{\pgfqpoint{4.175000in}{4.647500in}}%
\pgfpathlineto{\pgfqpoint{4.562500in}{4.647500in}}%
\pgfusepath{stroke}%
\end{pgfscope}%
\begin{pgfscope}%
\pgfpathrectangle{\pgfqpoint{0.687500in}{0.605000in}}{\pgfqpoint{4.262500in}{4.235000in}}%
\pgfusepath{clip}%
\pgfsetbuttcap%
\pgfsetroundjoin%
\definecolor{currentfill}{rgb}{0.843137,0.000000,0.000000}%
\pgfsetfillcolor{currentfill}%
\pgfsetlinewidth{1.003750pt}%
\definecolor{currentstroke}{rgb}{0.843137,0.000000,0.000000}%
\pgfsetstrokecolor{currentstroke}%
\pgfsetdash{}{0pt}%
\pgfsys@defobject{currentmarker}{\pgfqpoint{-0.041667in}{-0.041667in}}{\pgfqpoint{0.041667in}{0.041667in}}{%
\pgfpathmoveto{\pgfqpoint{0.000000in}{-0.041667in}}%
\pgfpathcurveto{\pgfqpoint{0.011050in}{-0.041667in}}{\pgfqpoint{0.021649in}{-0.037276in}}{\pgfqpoint{0.029463in}{-0.029463in}}%
\pgfpathcurveto{\pgfqpoint{0.037276in}{-0.021649in}}{\pgfqpoint{0.041667in}{-0.011050in}}{\pgfqpoint{0.041667in}{0.000000in}}%
\pgfpathcurveto{\pgfqpoint{0.041667in}{0.011050in}}{\pgfqpoint{0.037276in}{0.021649in}}{\pgfqpoint{0.029463in}{0.029463in}}%
\pgfpathcurveto{\pgfqpoint{0.021649in}{0.037276in}}{\pgfqpoint{0.011050in}{0.041667in}}{\pgfqpoint{0.000000in}{0.041667in}}%
\pgfpathcurveto{\pgfqpoint{-0.011050in}{0.041667in}}{\pgfqpoint{-0.021649in}{0.037276in}}{\pgfqpoint{-0.029463in}{0.029463in}}%
\pgfpathcurveto{\pgfqpoint{-0.037276in}{0.021649in}}{\pgfqpoint{-0.041667in}{0.011050in}}{\pgfqpoint{-0.041667in}{0.000000in}}%
\pgfpathcurveto{\pgfqpoint{-0.041667in}{-0.011050in}}{\pgfqpoint{-0.037276in}{-0.021649in}}{\pgfqpoint{-0.029463in}{-0.029463in}}%
\pgfpathcurveto{\pgfqpoint{-0.021649in}{-0.037276in}}{\pgfqpoint{-0.011050in}{-0.041667in}}{\pgfqpoint{0.000000in}{-0.041667in}}%
\pgfpathclose%
\pgfusepath{stroke,fill}%
}%
\begin{pgfscope}%
\pgfsys@transformshift{1.075000in}{2.722500in}%
\pgfsys@useobject{currentmarker}{}%
\end{pgfscope}%
\begin{pgfscope}%
\pgfsys@transformshift{1.462500in}{2.722500in}%
\pgfsys@useobject{currentmarker}{}%
\end{pgfscope}%
\begin{pgfscope}%
\pgfsys@transformshift{1.850000in}{3.685000in}%
\pgfsys@useobject{currentmarker}{}%
\end{pgfscope}%
\begin{pgfscope}%
\pgfsys@transformshift{2.237500in}{3.685000in}%
\pgfsys@useobject{currentmarker}{}%
\end{pgfscope}%
\begin{pgfscope}%
\pgfsys@transformshift{2.625000in}{4.647500in}%
\pgfsys@useobject{currentmarker}{}%
\end{pgfscope}%
\begin{pgfscope}%
\pgfsys@transformshift{3.012500in}{4.647500in}%
\pgfsys@useobject{currentmarker}{}%
\end{pgfscope}%
\begin{pgfscope}%
\pgfsys@transformshift{3.400000in}{4.647500in}%
\pgfsys@useobject{currentmarker}{}%
\end{pgfscope}%
\begin{pgfscope}%
\pgfsys@transformshift{3.787500in}{4.647500in}%
\pgfsys@useobject{currentmarker}{}%
\end{pgfscope}%
\begin{pgfscope}%
\pgfsys@transformshift{4.175000in}{4.647500in}%
\pgfsys@useobject{currentmarker}{}%
\end{pgfscope}%
\begin{pgfscope}%
\pgfsys@transformshift{4.562500in}{4.647500in}%
\pgfsys@useobject{currentmarker}{}%
\end{pgfscope}%
\end{pgfscope}%
\begin{pgfscope}%
\pgfpathrectangle{\pgfqpoint{0.687500in}{0.605000in}}{\pgfqpoint{4.262500in}{4.235000in}}%
\pgfusepath{clip}%
\pgfsetbuttcap%
\pgfsetroundjoin%
\pgfsetlinewidth{1.505625pt}%
\definecolor{currentstroke}{rgb}{0.549020,0.235294,1.000000}%
\pgfsetstrokecolor{currentstroke}%
\pgfsetdash{{5.550000pt}{2.400000pt}}{0.000000pt}%
\pgfpathmoveto{\pgfqpoint{1.075000in}{1.760000in}}%
\pgfpathlineto{\pgfqpoint{1.462500in}{1.760000in}}%
\pgfpathlineto{\pgfqpoint{1.850000in}{2.722500in}}%
\pgfpathlineto{\pgfqpoint{2.237500in}{2.722500in}}%
\pgfpathlineto{\pgfqpoint{2.625000in}{2.722500in}}%
\pgfpathlineto{\pgfqpoint{3.012500in}{2.722500in}}%
\pgfpathlineto{\pgfqpoint{3.400000in}{2.722500in}}%
\pgfpathlineto{\pgfqpoint{3.787500in}{2.722500in}}%
\pgfpathlineto{\pgfqpoint{4.175000in}{2.722500in}}%
\pgfpathlineto{\pgfqpoint{4.562500in}{2.722500in}}%
\pgfusepath{stroke}%
\end{pgfscope}%
\begin{pgfscope}%
\pgfpathrectangle{\pgfqpoint{0.687500in}{0.605000in}}{\pgfqpoint{4.262500in}{4.235000in}}%
\pgfusepath{clip}%
\pgfsetbuttcap%
\pgfsetmiterjoin%
\definecolor{currentfill}{rgb}{0.549020,0.235294,1.000000}%
\pgfsetfillcolor{currentfill}%
\pgfsetlinewidth{1.003750pt}%
\definecolor{currentstroke}{rgb}{0.549020,0.235294,1.000000}%
\pgfsetstrokecolor{currentstroke}%
\pgfsetdash{}{0pt}%
\pgfsys@defobject{currentmarker}{\pgfqpoint{-0.041667in}{-0.041667in}}{\pgfqpoint{0.041667in}{0.041667in}}{%
\pgfpathmoveto{\pgfqpoint{0.000000in}{0.041667in}}%
\pgfpathlineto{\pgfqpoint{-0.041667in}{-0.041667in}}%
\pgfpathlineto{\pgfqpoint{0.041667in}{-0.041667in}}%
\pgfpathclose%
\pgfusepath{stroke,fill}%
}%
\begin{pgfscope}%
\pgfsys@transformshift{1.075000in}{1.760000in}%
\pgfsys@useobject{currentmarker}{}%
\end{pgfscope}%
\begin{pgfscope}%
\pgfsys@transformshift{1.462500in}{1.760000in}%
\pgfsys@useobject{currentmarker}{}%
\end{pgfscope}%
\begin{pgfscope}%
\pgfsys@transformshift{1.850000in}{2.722500in}%
\pgfsys@useobject{currentmarker}{}%
\end{pgfscope}%
\begin{pgfscope}%
\pgfsys@transformshift{2.237500in}{2.722500in}%
\pgfsys@useobject{currentmarker}{}%
\end{pgfscope}%
\begin{pgfscope}%
\pgfsys@transformshift{2.625000in}{2.722500in}%
\pgfsys@useobject{currentmarker}{}%
\end{pgfscope}%
\begin{pgfscope}%
\pgfsys@transformshift{3.012500in}{2.722500in}%
\pgfsys@useobject{currentmarker}{}%
\end{pgfscope}%
\begin{pgfscope}%
\pgfsys@transformshift{3.400000in}{2.722500in}%
\pgfsys@useobject{currentmarker}{}%
\end{pgfscope}%
\begin{pgfscope}%
\pgfsys@transformshift{3.787500in}{2.722500in}%
\pgfsys@useobject{currentmarker}{}%
\end{pgfscope}%
\begin{pgfscope}%
\pgfsys@transformshift{4.175000in}{2.722500in}%
\pgfsys@useobject{currentmarker}{}%
\end{pgfscope}%
\begin{pgfscope}%
\pgfsys@transformshift{4.562500in}{2.722500in}%
\pgfsys@useobject{currentmarker}{}%
\end{pgfscope}%
\end{pgfscope}%
\begin{pgfscope}%
\pgfpathrectangle{\pgfqpoint{0.687500in}{0.605000in}}{\pgfqpoint{4.262500in}{4.235000in}}%
\pgfusepath{clip}%
\pgfsetbuttcap%
\pgfsetroundjoin%
\pgfsetlinewidth{1.505625pt}%
\definecolor{currentstroke}{rgb}{0.007843,0.533333,0.000000}%
\pgfsetstrokecolor{currentstroke}%
\pgfsetdash{{5.550000pt}{2.400000pt}}{0.000000pt}%
\pgfpathmoveto{\pgfqpoint{1.075000in}{1.760000in}}%
\pgfpathlineto{\pgfqpoint{1.462500in}{0.797500in}}%
\pgfpathlineto{\pgfqpoint{1.850000in}{1.760000in}}%
\pgfpathlineto{\pgfqpoint{2.237500in}{1.760000in}}%
\pgfpathlineto{\pgfqpoint{2.625000in}{1.760000in}}%
\pgfpathlineto{\pgfqpoint{3.012500in}{1.760000in}}%
\pgfpathlineto{\pgfqpoint{3.400000in}{1.760000in}}%
\pgfpathlineto{\pgfqpoint{3.787500in}{1.760000in}}%
\pgfpathlineto{\pgfqpoint{4.175000in}{1.760000in}}%
\pgfpathlineto{\pgfqpoint{4.562500in}{1.760000in}}%
\pgfusepath{stroke}%
\end{pgfscope}%
\begin{pgfscope}%
\pgfpathrectangle{\pgfqpoint{0.687500in}{0.605000in}}{\pgfqpoint{4.262500in}{4.235000in}}%
\pgfusepath{clip}%
\pgfsetbuttcap%
\pgfsetmiterjoin%
\definecolor{currentfill}{rgb}{0.007843,0.533333,0.000000}%
\pgfsetfillcolor{currentfill}%
\pgfsetlinewidth{1.003750pt}%
\definecolor{currentstroke}{rgb}{0.007843,0.533333,0.000000}%
\pgfsetstrokecolor{currentstroke}%
\pgfsetdash{}{0pt}%
\pgfsys@defobject{currentmarker}{\pgfqpoint{-0.041667in}{-0.041667in}}{\pgfqpoint{0.041667in}{0.041667in}}{%
\pgfpathmoveto{\pgfqpoint{-0.000000in}{-0.041667in}}%
\pgfpathlineto{\pgfqpoint{0.041667in}{0.041667in}}%
\pgfpathlineto{\pgfqpoint{-0.041667in}{0.041667in}}%
\pgfpathclose%
\pgfusepath{stroke,fill}%
}%
\begin{pgfscope}%
\pgfsys@transformshift{1.075000in}{1.760000in}%
\pgfsys@useobject{currentmarker}{}%
\end{pgfscope}%
\begin{pgfscope}%
\pgfsys@transformshift{1.462500in}{0.797500in}%
\pgfsys@useobject{currentmarker}{}%
\end{pgfscope}%
\begin{pgfscope}%
\pgfsys@transformshift{1.850000in}{1.760000in}%
\pgfsys@useobject{currentmarker}{}%
\end{pgfscope}%
\begin{pgfscope}%
\pgfsys@transformshift{2.237500in}{1.760000in}%
\pgfsys@useobject{currentmarker}{}%
\end{pgfscope}%
\begin{pgfscope}%
\pgfsys@transformshift{2.625000in}{1.760000in}%
\pgfsys@useobject{currentmarker}{}%
\end{pgfscope}%
\begin{pgfscope}%
\pgfsys@transformshift{3.012500in}{1.760000in}%
\pgfsys@useobject{currentmarker}{}%
\end{pgfscope}%
\begin{pgfscope}%
\pgfsys@transformshift{3.400000in}{1.760000in}%
\pgfsys@useobject{currentmarker}{}%
\end{pgfscope}%
\begin{pgfscope}%
\pgfsys@transformshift{3.787500in}{1.760000in}%
\pgfsys@useobject{currentmarker}{}%
\end{pgfscope}%
\begin{pgfscope}%
\pgfsys@transformshift{4.175000in}{1.760000in}%
\pgfsys@useobject{currentmarker}{}%
\end{pgfscope}%
\begin{pgfscope}%
\pgfsys@transformshift{4.562500in}{1.760000in}%
\pgfsys@useobject{currentmarker}{}%
\end{pgfscope}%
\end{pgfscope}%
\begin{pgfscope}%
\pgfsetrectcap%
\pgfsetmiterjoin%
\pgfsetlinewidth{0.803000pt}%
\definecolor{currentstroke}{rgb}{0.000000,0.000000,0.000000}%
\pgfsetstrokecolor{currentstroke}%
\pgfsetdash{}{0pt}%
\pgfpathmoveto{\pgfqpoint{0.687500in}{0.605000in}}%
\pgfpathlineto{\pgfqpoint{0.687500in}{4.840000in}}%
\pgfusepath{stroke}%
\end{pgfscope}%
\begin{pgfscope}%
\pgfsetrectcap%
\pgfsetmiterjoin%
\pgfsetlinewidth{0.000000pt}%
\definecolor{currentstroke}{rgb}{0.000000,0.000000,0.000000}%
\pgfsetstrokecolor{currentstroke}%
\pgfsetstrokeopacity{0.000000}%
\pgfsetdash{}{0pt}%
\pgfpathmoveto{\pgfqpoint{4.950000in}{0.605000in}}%
\pgfpathlineto{\pgfqpoint{4.950000in}{4.840000in}}%
\pgfusepath{}%
\end{pgfscope}%
\begin{pgfscope}%
\pgfsetrectcap%
\pgfsetmiterjoin%
\pgfsetlinewidth{0.803000pt}%
\definecolor{currentstroke}{rgb}{0.000000,0.000000,0.000000}%
\pgfsetstrokecolor{currentstroke}%
\pgfsetdash{}{0pt}%
\pgfpathmoveto{\pgfqpoint{0.687500in}{0.605000in}}%
\pgfpathlineto{\pgfqpoint{4.950000in}{0.605000in}}%
\pgfusepath{stroke}%
\end{pgfscope}%
\begin{pgfscope}%
\pgfsetrectcap%
\pgfsetmiterjoin%
\pgfsetlinewidth{0.000000pt}%
\definecolor{currentstroke}{rgb}{0.000000,0.000000,0.000000}%
\pgfsetstrokecolor{currentstroke}%
\pgfsetstrokeopacity{0.000000}%
\pgfsetdash{}{0pt}%
\pgfpathmoveto{\pgfqpoint{0.687500in}{4.840000in}}%
\pgfpathlineto{\pgfqpoint{4.950000in}{4.840000in}}%
\pgfusepath{}%
\end{pgfscope}%
\begin{pgfscope}%
\pgfsetbuttcap%
\pgfsetmiterjoin%
\definecolor{currentfill}{rgb}{1.000000,1.000000,1.000000}%
\pgfsetfillcolor{currentfill}%
\pgfsetfillopacity{0.800000}%
\pgfsetlinewidth{1.003750pt}%
\definecolor{currentstroke}{rgb}{0.800000,0.800000,0.800000}%
\pgfsetstrokecolor{currentstroke}%
\pgfsetstrokeopacity{0.800000}%
\pgfsetdash{}{0pt}%
\pgfpathmoveto{\pgfqpoint{0.784722in}{4.117317in}}%
\pgfpathlineto{\pgfqpoint{1.677709in}{4.117317in}}%
\pgfpathquadraticcurveto{\pgfqpoint{1.705487in}{4.117317in}}{\pgfqpoint{1.705487in}{4.145095in}}%
\pgfpathlineto{\pgfqpoint{1.705487in}{4.742778in}}%
\pgfpathquadraticcurveto{\pgfqpoint{1.705487in}{4.770556in}}{\pgfqpoint{1.677709in}{4.770556in}}%
\pgfpathlineto{\pgfqpoint{0.784722in}{4.770556in}}%
\pgfpathquadraticcurveto{\pgfqpoint{0.756944in}{4.770556in}}{\pgfqpoint{0.756944in}{4.742778in}}%
\pgfpathlineto{\pgfqpoint{0.756944in}{4.145095in}}%
\pgfpathquadraticcurveto{\pgfqpoint{0.756944in}{4.117317in}}{\pgfqpoint{0.784722in}{4.117317in}}%
\pgfpathclose%
\pgfusepath{stroke,fill}%
\end{pgfscope}%
\begin{pgfscope}%
\pgfsetbuttcap%
\pgfsetroundjoin%
\pgfsetlinewidth{1.505625pt}%
\definecolor{currentstroke}{rgb}{0.843137,0.000000,0.000000}%
\pgfsetstrokecolor{currentstroke}%
\pgfsetdash{{5.550000pt}{2.400000pt}}{0.000000pt}%
\pgfpathmoveto{\pgfqpoint{0.812500in}{4.658088in}}%
\pgfpathlineto{\pgfqpoint{1.090278in}{4.658088in}}%
\pgfusepath{stroke}%
\end{pgfscope}%
\begin{pgfscope}%
\pgfsetbuttcap%
\pgfsetroundjoin%
\definecolor{currentfill}{rgb}{0.843137,0.000000,0.000000}%
\pgfsetfillcolor{currentfill}%
\pgfsetlinewidth{1.003750pt}%
\definecolor{currentstroke}{rgb}{0.843137,0.000000,0.000000}%
\pgfsetstrokecolor{currentstroke}%
\pgfsetdash{}{0pt}%
\pgfsys@defobject{currentmarker}{\pgfqpoint{-0.041667in}{-0.041667in}}{\pgfqpoint{0.041667in}{0.041667in}}{%
\pgfpathmoveto{\pgfqpoint{0.000000in}{-0.041667in}}%
\pgfpathcurveto{\pgfqpoint{0.011050in}{-0.041667in}}{\pgfqpoint{0.021649in}{-0.037276in}}{\pgfqpoint{0.029463in}{-0.029463in}}%
\pgfpathcurveto{\pgfqpoint{0.037276in}{-0.021649in}}{\pgfqpoint{0.041667in}{-0.011050in}}{\pgfqpoint{0.041667in}{0.000000in}}%
\pgfpathcurveto{\pgfqpoint{0.041667in}{0.011050in}}{\pgfqpoint{0.037276in}{0.021649in}}{\pgfqpoint{0.029463in}{0.029463in}}%
\pgfpathcurveto{\pgfqpoint{0.021649in}{0.037276in}}{\pgfqpoint{0.011050in}{0.041667in}}{\pgfqpoint{0.000000in}{0.041667in}}%
\pgfpathcurveto{\pgfqpoint{-0.011050in}{0.041667in}}{\pgfqpoint{-0.021649in}{0.037276in}}{\pgfqpoint{-0.029463in}{0.029463in}}%
\pgfpathcurveto{\pgfqpoint{-0.037276in}{0.021649in}}{\pgfqpoint{-0.041667in}{0.011050in}}{\pgfqpoint{-0.041667in}{0.000000in}}%
\pgfpathcurveto{\pgfqpoint{-0.041667in}{-0.011050in}}{\pgfqpoint{-0.037276in}{-0.021649in}}{\pgfqpoint{-0.029463in}{-0.029463in}}%
\pgfpathcurveto{\pgfqpoint{-0.021649in}{-0.037276in}}{\pgfqpoint{-0.011050in}{-0.041667in}}{\pgfqpoint{0.000000in}{-0.041667in}}%
\pgfpathclose%
\pgfusepath{stroke,fill}%
}%
\begin{pgfscope}%
\pgfsys@transformshift{0.951389in}{4.658088in}%
\pgfsys@useobject{currentmarker}{}%
\end{pgfscope}%
\end{pgfscope}%
\begin{pgfscope}%
\definecolor{textcolor}{rgb}{0.000000,0.000000,0.000000}%
\pgfsetstrokecolor{textcolor}%
\pgfsetfillcolor{textcolor}%
\pgftext[x=1.201389in,y=4.609477in,left,base]{\color{textcolor}\sffamily\fontsize{10.000000}{12.000000}\selectfont \(\displaystyle \beta = 0.8\)}%
\end{pgfscope}%
\begin{pgfscope}%
\pgfsetbuttcap%
\pgfsetroundjoin%
\pgfsetlinewidth{1.505625pt}%
\definecolor{currentstroke}{rgb}{0.549020,0.235294,1.000000}%
\pgfsetstrokecolor{currentstroke}%
\pgfsetdash{{5.550000pt}{2.400000pt}}{0.000000pt}%
\pgfpathmoveto{\pgfqpoint{0.812500in}{4.454231in}}%
\pgfpathlineto{\pgfqpoint{1.090278in}{4.454231in}}%
\pgfusepath{stroke}%
\end{pgfscope}%
\begin{pgfscope}%
\pgfsetbuttcap%
\pgfsetmiterjoin%
\definecolor{currentfill}{rgb}{0.549020,0.235294,1.000000}%
\pgfsetfillcolor{currentfill}%
\pgfsetlinewidth{1.003750pt}%
\definecolor{currentstroke}{rgb}{0.549020,0.235294,1.000000}%
\pgfsetstrokecolor{currentstroke}%
\pgfsetdash{}{0pt}%
\pgfsys@defobject{currentmarker}{\pgfqpoint{-0.041667in}{-0.041667in}}{\pgfqpoint{0.041667in}{0.041667in}}{%
\pgfpathmoveto{\pgfqpoint{0.000000in}{0.041667in}}%
\pgfpathlineto{\pgfqpoint{-0.041667in}{-0.041667in}}%
\pgfpathlineto{\pgfqpoint{0.041667in}{-0.041667in}}%
\pgfpathclose%
\pgfusepath{stroke,fill}%
}%
\begin{pgfscope}%
\pgfsys@transformshift{0.951389in}{4.454231in}%
\pgfsys@useobject{currentmarker}{}%
\end{pgfscope}%
\end{pgfscope}%
\begin{pgfscope}%
\definecolor{textcolor}{rgb}{0.000000,0.000000,0.000000}%
\pgfsetstrokecolor{textcolor}%
\pgfsetfillcolor{textcolor}%
\pgftext[x=1.201389in,y=4.405620in,left,base]{\color{textcolor}\sffamily\fontsize{10.000000}{12.000000}\selectfont \(\displaystyle \beta = 0.9\)}%
\end{pgfscope}%
\begin{pgfscope}%
\pgfsetbuttcap%
\pgfsetroundjoin%
\pgfsetlinewidth{1.505625pt}%
\definecolor{currentstroke}{rgb}{0.007843,0.533333,0.000000}%
\pgfsetstrokecolor{currentstroke}%
\pgfsetdash{{5.550000pt}{2.400000pt}}{0.000000pt}%
\pgfpathmoveto{\pgfqpoint{0.812500in}{4.250374in}}%
\pgfpathlineto{\pgfqpoint{1.090278in}{4.250374in}}%
\pgfusepath{stroke}%
\end{pgfscope}%
\begin{pgfscope}%
\pgfsetbuttcap%
\pgfsetmiterjoin%
\definecolor{currentfill}{rgb}{0.007843,0.533333,0.000000}%
\pgfsetfillcolor{currentfill}%
\pgfsetlinewidth{1.003750pt}%
\definecolor{currentstroke}{rgb}{0.007843,0.533333,0.000000}%
\pgfsetstrokecolor{currentstroke}%
\pgfsetdash{}{0pt}%
\pgfsys@defobject{currentmarker}{\pgfqpoint{-0.041667in}{-0.041667in}}{\pgfqpoint{0.041667in}{0.041667in}}{%
\pgfpathmoveto{\pgfqpoint{-0.000000in}{-0.041667in}}%
\pgfpathlineto{\pgfqpoint{0.041667in}{0.041667in}}%
\pgfpathlineto{\pgfqpoint{-0.041667in}{0.041667in}}%
\pgfpathclose%
\pgfusepath{stroke,fill}%
}%
\begin{pgfscope}%
\pgfsys@transformshift{0.951389in}{4.250374in}%
\pgfsys@useobject{currentmarker}{}%
\end{pgfscope}%
\end{pgfscope}%
\begin{pgfscope}%
\definecolor{textcolor}{rgb}{0.000000,0.000000,0.000000}%
\pgfsetstrokecolor{textcolor}%
\pgfsetfillcolor{textcolor}%
\pgftext[x=1.201389in,y=4.201763in,left,base]{\color{textcolor}\sffamily\fontsize{10.000000}{12.000000}\selectfont \(\displaystyle \beta = 1\)}%
\end{pgfscope}%
\end{pgfpicture}%
\makeatother%
\endgroup%

      \caption{GMRES iteration counts for $\AmatoI\Amatt$ given by \cref{eq:noweak,eq:ntweak}, where $\alpha = 0.2/k^\beta,$ for $\beta = 0.8,0.9,1.$}\label{fig:l1high}
\end{figure}
      
\begin{table}
  \centering
  \begin{tabular}{Sc Sc Sc Sc Sc Sc Sc Sc Sc ScSc }
\toprule

$\eps$\textbackslash$k$ &  10.0  &  20.0  &  30.0  &  40.0  &  50.0  &  60.0  &  70.0  &  80.0  &  90.0  &  100.0 \\

\midrule

0.0 &     14 &     40 &    119 &    258 &    427 &    627 &    940 &   1274 &   1695 &   2116 \\

0.1 &     13 &     27 &     70 &    147 &    262 &    394 &    590 &    825 &   1128 &   1393 \\

0.2 &     12 &     22 &     40 &     77 &    134 &    199 &    292 &    419 &    551 &    726 \\

0.3 &     11 &     18 &     25 &     40 &     58 &     86 &    119 &    163 &    209 &    270 \\

0.4 &     10 &     15 &     20 &     25 &     30 &     42 &     53 &     64 &     81 &     98 \\

0.5 &     10 &     13 &     16 &     19 &     22 &     25 &     28 &     31 &     37 &     41 \\

0.6 &      9 &     11 &     13 &     14 &     16 &     17 &     19 &     19 &     21 &     22 \\

0.7 &      8 &      9 &     10 &     11 &     12 &     13 &     13 &     14 &     14 &     14 \\

0.8 &      8 &      8 &      9 &      9 &     10 &     10 &     10 &     10 &     10 &     10 \\

0.9 &      7 &      7 &      8 &      8 &      8 &      8 &      8 &      8 &      8 &      8 \\

1.0 &      7 &      6 &      7 &      7 &      7 &      7 &      7 &      7 &      7 &      7 \\

\bottomrule

\end{tabular}


  \caption{GMRES iteration counts for $\AmatoI\Amatt$ given by \cref{eq:noweak,eq:ntweak}, where $\alpha = 0.2/k^\beta.$}\label{tab:l1}
  \end{table}

\section{Applying nearby preconditioning to stochastic problems}\label{sec:nbpcstochastic}

We now apply nearby preconditioning to \emph{stochastic} Helmholtz problems, i.e., the Helmholtz problems defined and described in \cref{chap:stochastic}. We have two applications. In \cref{sec:probnbpc} we develop a probabalistic analogue of \cref{cor:1} above, and in \cref{sec:qmcnbpc} we apply nearby preconditioning to a Quasi-Monte-Carlo (QMC) method, showing that we have significant computational gains by using nearby preconditioning.

Throughout this \cref{sec:nbpcstochastic} we consider \cref{prob:msedp} from \cref{chap:stochastic} but with $A=I$. I.e., for simplicity's sake we only consider the case of random $n$, although everything we sya could be easily extended to random $A$. To maintain consistent notation with the rest of this \cref{chap:nbpc} we will use a superscript ${}^{(2)}$ to refer to the stochastic problem (e.g., the random coefficient will be $\nst(\omega)$, the solution will be $\ust(\omega)$, the matrices arising from the finite-element discretiation will be $\Amatt(\omega),$ etc.. We also let $\nso \in \LiDRRR$ define a \emph{deterministic} Helmholtz problem, as at the beginning of this \cref{sec:nbpc}; we will use this deterministic Helmholtz problem to precondition the (discretisations of) the realisations of the stochastic Helmholtz problem. I.e., we will consider the performance of GMRES applied to
\beq\label{eq:stopc}
\AmatoI\Amatt(\omega)\bu = \AmatoI \bff.
\eeq
For simplicity's sake, in all that follows we will measure $\no-\nt$ in the $L^{\infty}$ norm, or though one could use any of the weaker norms discussed above, and obtain analogous results.

\subsection{Probabilistic analogue of \cref{cor:1a}}
If we apply \cref{cor:1a} to the above set-up we can straightforwardly conclude the following \cref{cor:stonbpcas}

\bco[Almost-sure nearby preconditioning]\label{cor:stonbpcas}
Let the assumptions of \cref{cor:1a} hold, and let $\nt$ satisfy the assumptions at the start of \cref{sec:hh-results}. Then the number of GMRES iterations needed to solve \cref{eq:stopc} is bounded independently of $k$ almost surely if
\beq\label{eq:nbpcas}
\NLiDRRR{\nso-\nst(\omega)} \leq \frac1{2\Ct k}
\eeq
almost surely.
\eco
The numerical results in \cref{sec:num} above can be seen as confirming this result.

\bre[\Cref{cor:stonbpcas} is not ideal]\label{rem:notideal}
\Cref{cor:stonbpcas} is not ideal for two main reasons:
\ben
\item\label[itemreason]{it:notideal1} The condition \cref{eq:nbpcas} must hold almost surely, and
  \item\label[itemreason]{it:notideal2} \Cref{cor:stonbpcas} does not give any explicit information on how the number of GMRES iterations depends on $\NLiDRRR{\nso-\nst(\omega)}.$
    \een
    \Cref{it:notideal1} is not ideal because in many physically realistic problems $\NLiDRRR{\no-\nt(\omega)}$ may be unbounded (e.g., if $\nt$ is a lognormal random field) or even if bounded may not satisfy the condition \cref{eq:nbpcas} almost surely. \Cref{it:notideal2} is not ideal because it means one cannot infer information about the distribution of the number of GMRES iterations from the distribution of$\NLiDRRR{\nso-\nst(\omega)}.$
    \ere
    In order to correct the deficiencies described in \cref{rem:notideal} we will first prove a bound on the number of GMRES iterations depending explicitly on $\NLiDRRR{\nso-\nst(\omega)}$, and then use this bound to prove a probabalistic estimate on the number of GMRES iterations.

The following \lcnamecref{lem:probgmres1} is a \emph{deterministic} result, i.e., it does not require $\nt$ to be a random field. Therefore for this \lcnamecref{lem:probgmres1} only, we assume $\nst$ is as given at the beginning of this \lcnamecref{chap:nbpc}.
\ble[Maximum number of GMRES iterations]\label{lem:probgmres1}
Let $0 < \eps < 1,$ and let $\dofs$ denote the number of degrees of freedom, i.e. the size of the matrices $\Amato$ and $\Amatt$.

Let $\GMRES{\eps}{\nso}{\nst}$ denote the number of iterations required for GMRES in the unweighted norm $\Nt{\cdot}$ with $\Nt{\brz} = 1,$ applied to
\beqs
\AmatoI\Amatt  \bu = \AmatoI \bff
\eeqs
to converge to within a tolerance $\eps,$ i.e., to achieve
\beqs
\frac{\Nt{\bxm}}{\Nt{\bff}} < \eps.
\eeqs

Then there exists a function $\Gfnname:\RRp\rightarrow [0,\dofs]$ such that
\beqs
\GMRES{\eps}{\nso}{\nst} \leq \Gfn{\nso-\nst}.
\eeqs

Moreover, $\Gfnname$ is given by
\beq\label{eq:gdef}
\Gfn{\NLiDRRR{\nso-\nst}} =
\begin{dcases}
\min\set{N,\frac{\log{\eps}}{\log\mleft(\frac{2\alpha^{1/2}}{\mleft(1+\alpha\mright)^2}\mright)}+1} & \tif \alpha < 1\\
N & \tif \alpha \geq 1,
\end{dcases}
\eeq

where $\alpha = \Ct k \NLiDRRR{\nso-\nst},$ where $\Ct$ is given by \cref{eq:C2}.
\ele

See \cref{fig:G} for some examples of the function $\Gfnname$.

The proof of \cref{lem:probgmres1} uses the following corollary \cite[Corollary 3]{SaSc:86} of \cite[Proposition 2]{SaSc:86} on the `lucky breakdown' of GMRES.
\bco[Guaranteed convergence of GMRES]\label{cor:gmresguaranteed}
For an $N \times N$ problem GMRES converges in at most $N$ iterations.
\eco

\bpf[Proof of \cref{lem:probgmres1}]
For $\alpha \geq 1$, the result is immediate from \cref{cor:gmresguaranteed}. For $\alpha < 1,$ if we insert \cref{eq:gmressin} into \cref{eq:Elman} (with $\Dmat=\Imat,$ so $\NDmat{\cdot} = \Nt{\cdot}$), we obtain, for $m \in \NN$
\beq\label{eq:gmressub}
\frac{\Nt{\brm}}{\Nt{\brz}} \leq \mleft(\frac{2\sqrt{\alpha}}{\mleft(1+\alpha\mright)^2}\mright)^m.
\eeq
To obtain a lower bound on the number of iterations needed to obtain the solution to within a tolerance $\eps,$ we set the right-hand side of \cref{eq:gmressub} to be less than $\eps$ and solve for $m$ to obtain that the GMRES residual is less than $\eps$ (recall we assume $\Nt{\brz} = 1$) if
\beq\label{eq:mlower}
m \geq \frac{\log{\eps}}{\log\mleft(\frac{2\alpha^{1/2}}{\mleft(1+\alpha\mright)^2}\mright)}.
\eeq
Hence, if $m$ is the smallest integer satisfying \cref{eq:mlower}, then
\beq\label{eq:gmressub2}
m  \leq\frac{\log{\eps}}{\log\mleft(\frac{2\alpha^{1/2}}{\mleft(1+\alpha\mright)^2}\mright)}+1.
\eeq
The result for $\alpha < 1$ therefore follows from \cref{eq:gmressub2,cor:gmresguaranteed}.
\epf

\bre[Why not use the ceiling in \cref{eq:gmressub2}?]
One could replace the bound \cref{eq:gmressub2} on $m$ by the equality
\beqs
m  =\ceil{\frac{\log{\eps}}{\log\mleft(\frac{2\alpha^{1/2}}{\mleft(1+\alpha\mright)^2}\mright)}}.
\eeqs
However, the change in the definition of $\Gfnname$ \cref{eq:gdef} would mean $\Gfnname$ would no longer be continuous. As we must use numberical methods to calculate probabilities associated with $\Gfnname$ (see \cref{thm:probgmres,rem:gcomputable} below), it is much simpler if $\Gfnname$ is continuous, and so we use \cref{eq:gmressub2}.
\ere

\bre[Why the dependence on $\alpha$ in \cref{lem:probgmres1}?]
The reason that \cref{eq:gdef} has two cases depending on $\alpha = \Ct k \NLiDRRR{\nso-\nst}$ is because the \lcnamecref{cor:GMRES_intro} only holds if $\alpha < 1$. Therefore if $\alpha \geq 1$ the only result available for us to apply is \cref{cor:gmresguaranteed}, stating that GMRES converges in at most $N$ iterations.
\ere

\begin{figure}
  \centering
  %% Creator: Matplotlib, PGF backend
%%
%% To include the figure in your LaTeX document, write
%%   \input{<filename>.pgf}
%%
%% Make sure the required packages are loaded in your preamble
%%   \usepackage{pgf}
%%
%% Figures using additional raster images can only be included by \input if
%% they are in the same directory as the main LaTeX file. For loading figures
%% from other directories you can use the `import` package
%%   \usepackage{import}
%% and then include the figures with
%%   \import{<path to file>}{<filename>.pgf}
%%
%% Matplotlib used the following preamble
%%   \usepackage{amssymb}
%%   \usepackage{mleftright}
%%   \usepackage{fontspec}
%%   \setmainfont{DejaVuSerif.ttf}[Path=/home/owen/progs/firedrake-complex/firedrake/lib/python3.5/site-packages/matplotlib/mpl-data/fonts/ttf/]
%%   \setsansfont{DejaVuSans.ttf}[Path=/home/owen/progs/firedrake-complex/firedrake/lib/python3.5/site-packages/matplotlib/mpl-data/fonts/ttf/]
%%   \setmonofont{DejaVuSansMono.ttf}[Path=/home/owen/progs/firedrake-complex/firedrake/lib/python3.5/site-packages/matplotlib/mpl-data/fonts/ttf/]
%%
\begingroup%
\makeatletter%
\begin{pgfpicture}%
\pgfpathrectangle{\pgfpointorigin}{\pgfqpoint{6.400000in}{4.800000in}}%
\pgfusepath{use as bounding box, clip}%
\begin{pgfscope}%
\pgfsetbuttcap%
\pgfsetmiterjoin%
\definecolor{currentfill}{rgb}{1.000000,1.000000,1.000000}%
\pgfsetfillcolor{currentfill}%
\pgfsetlinewidth{0.000000pt}%
\definecolor{currentstroke}{rgb}{1.000000,1.000000,1.000000}%
\pgfsetstrokecolor{currentstroke}%
\pgfsetdash{}{0pt}%
\pgfpathmoveto{\pgfqpoint{0.000000in}{0.000000in}}%
\pgfpathlineto{\pgfqpoint{6.400000in}{0.000000in}}%
\pgfpathlineto{\pgfqpoint{6.400000in}{4.800000in}}%
\pgfpathlineto{\pgfqpoint{0.000000in}{4.800000in}}%
\pgfpathclose%
\pgfusepath{fill}%
\end{pgfscope}%
\begin{pgfscope}%
\pgfsetbuttcap%
\pgfsetmiterjoin%
\definecolor{currentfill}{rgb}{1.000000,1.000000,1.000000}%
\pgfsetfillcolor{currentfill}%
\pgfsetlinewidth{0.000000pt}%
\definecolor{currentstroke}{rgb}{0.000000,0.000000,0.000000}%
\pgfsetstrokecolor{currentstroke}%
\pgfsetstrokeopacity{0.000000}%
\pgfsetdash{}{0pt}%
\pgfpathmoveto{\pgfqpoint{0.800000in}{0.528000in}}%
\pgfpathlineto{\pgfqpoint{5.760000in}{0.528000in}}%
\pgfpathlineto{\pgfqpoint{5.760000in}{4.224000in}}%
\pgfpathlineto{\pgfqpoint{0.800000in}{4.224000in}}%
\pgfpathclose%
\pgfusepath{fill}%
\end{pgfscope}%
\begin{pgfscope}%
\pgfsetbuttcap%
\pgfsetroundjoin%
\definecolor{currentfill}{rgb}{0.000000,0.000000,0.000000}%
\pgfsetfillcolor{currentfill}%
\pgfsetlinewidth{0.803000pt}%
\definecolor{currentstroke}{rgb}{0.000000,0.000000,0.000000}%
\pgfsetstrokecolor{currentstroke}%
\pgfsetdash{}{0pt}%
\pgfsys@defobject{currentmarker}{\pgfqpoint{0.000000in}{-0.048611in}}{\pgfqpoint{0.000000in}{0.000000in}}{%
\pgfpathmoveto{\pgfqpoint{0.000000in}{0.000000in}}%
\pgfpathlineto{\pgfqpoint{0.000000in}{-0.048611in}}%
\pgfusepath{stroke,fill}%
}%
\begin{pgfscope}%
\pgfsys@transformshift{0.979908in}{0.528000in}%
\pgfsys@useobject{currentmarker}{}%
\end{pgfscope}%
\end{pgfscope}%
\begin{pgfscope}%
\definecolor{textcolor}{rgb}{0.000000,0.000000,0.000000}%
\pgfsetstrokecolor{textcolor}%
\pgfsetfillcolor{textcolor}%
\pgftext[x=0.979908in,y=0.430778in,,top]{\color{textcolor}\sffamily\fontsize{10.000000}{12.000000}\selectfont 0.0}%
\end{pgfscope}%
\begin{pgfscope}%
\pgfsetbuttcap%
\pgfsetroundjoin%
\definecolor{currentfill}{rgb}{0.000000,0.000000,0.000000}%
\pgfsetfillcolor{currentfill}%
\pgfsetlinewidth{0.803000pt}%
\definecolor{currentstroke}{rgb}{0.000000,0.000000,0.000000}%
\pgfsetstrokecolor{currentstroke}%
\pgfsetdash{}{0pt}%
\pgfsys@defobject{currentmarker}{\pgfqpoint{0.000000in}{-0.048611in}}{\pgfqpoint{0.000000in}{0.000000in}}{%
\pgfpathmoveto{\pgfqpoint{0.000000in}{0.000000in}}%
\pgfpathlineto{\pgfqpoint{0.000000in}{-0.048611in}}%
\pgfusepath{stroke,fill}%
}%
\begin{pgfscope}%
\pgfsys@transformshift{1.890836in}{0.528000in}%
\pgfsys@useobject{currentmarker}{}%
\end{pgfscope}%
\end{pgfscope}%
\begin{pgfscope}%
\definecolor{textcolor}{rgb}{0.000000,0.000000,0.000000}%
\pgfsetstrokecolor{textcolor}%
\pgfsetfillcolor{textcolor}%
\pgftext[x=1.890836in,y=0.430778in,,top]{\color{textcolor}\sffamily\fontsize{10.000000}{12.000000}\selectfont 0.2}%
\end{pgfscope}%
\begin{pgfscope}%
\pgfsetbuttcap%
\pgfsetroundjoin%
\definecolor{currentfill}{rgb}{0.000000,0.000000,0.000000}%
\pgfsetfillcolor{currentfill}%
\pgfsetlinewidth{0.803000pt}%
\definecolor{currentstroke}{rgb}{0.000000,0.000000,0.000000}%
\pgfsetstrokecolor{currentstroke}%
\pgfsetdash{}{0pt}%
\pgfsys@defobject{currentmarker}{\pgfqpoint{0.000000in}{-0.048611in}}{\pgfqpoint{0.000000in}{0.000000in}}{%
\pgfpathmoveto{\pgfqpoint{0.000000in}{0.000000in}}%
\pgfpathlineto{\pgfqpoint{0.000000in}{-0.048611in}}%
\pgfusepath{stroke,fill}%
}%
\begin{pgfscope}%
\pgfsys@transformshift{2.801763in}{0.528000in}%
\pgfsys@useobject{currentmarker}{}%
\end{pgfscope}%
\end{pgfscope}%
\begin{pgfscope}%
\definecolor{textcolor}{rgb}{0.000000,0.000000,0.000000}%
\pgfsetstrokecolor{textcolor}%
\pgfsetfillcolor{textcolor}%
\pgftext[x=2.801763in,y=0.430778in,,top]{\color{textcolor}\sffamily\fontsize{10.000000}{12.000000}\selectfont 0.4}%
\end{pgfscope}%
\begin{pgfscope}%
\pgfsetbuttcap%
\pgfsetroundjoin%
\definecolor{currentfill}{rgb}{0.000000,0.000000,0.000000}%
\pgfsetfillcolor{currentfill}%
\pgfsetlinewidth{0.803000pt}%
\definecolor{currentstroke}{rgb}{0.000000,0.000000,0.000000}%
\pgfsetstrokecolor{currentstroke}%
\pgfsetdash{}{0pt}%
\pgfsys@defobject{currentmarker}{\pgfqpoint{0.000000in}{-0.048611in}}{\pgfqpoint{0.000000in}{0.000000in}}{%
\pgfpathmoveto{\pgfqpoint{0.000000in}{0.000000in}}%
\pgfpathlineto{\pgfqpoint{0.000000in}{-0.048611in}}%
\pgfusepath{stroke,fill}%
}%
\begin{pgfscope}%
\pgfsys@transformshift{3.712691in}{0.528000in}%
\pgfsys@useobject{currentmarker}{}%
\end{pgfscope}%
\end{pgfscope}%
\begin{pgfscope}%
\definecolor{textcolor}{rgb}{0.000000,0.000000,0.000000}%
\pgfsetstrokecolor{textcolor}%
\pgfsetfillcolor{textcolor}%
\pgftext[x=3.712691in,y=0.430778in,,top]{\color{textcolor}\sffamily\fontsize{10.000000}{12.000000}\selectfont 0.6}%
\end{pgfscope}%
\begin{pgfscope}%
\pgfsetbuttcap%
\pgfsetroundjoin%
\definecolor{currentfill}{rgb}{0.000000,0.000000,0.000000}%
\pgfsetfillcolor{currentfill}%
\pgfsetlinewidth{0.803000pt}%
\definecolor{currentstroke}{rgb}{0.000000,0.000000,0.000000}%
\pgfsetstrokecolor{currentstroke}%
\pgfsetdash{}{0pt}%
\pgfsys@defobject{currentmarker}{\pgfqpoint{0.000000in}{-0.048611in}}{\pgfqpoint{0.000000in}{0.000000in}}{%
\pgfpathmoveto{\pgfqpoint{0.000000in}{0.000000in}}%
\pgfpathlineto{\pgfqpoint{0.000000in}{-0.048611in}}%
\pgfusepath{stroke,fill}%
}%
\begin{pgfscope}%
\pgfsys@transformshift{4.623618in}{0.528000in}%
\pgfsys@useobject{currentmarker}{}%
\end{pgfscope}%
\end{pgfscope}%
\begin{pgfscope}%
\definecolor{textcolor}{rgb}{0.000000,0.000000,0.000000}%
\pgfsetstrokecolor{textcolor}%
\pgfsetfillcolor{textcolor}%
\pgftext[x=4.623618in,y=0.430778in,,top]{\color{textcolor}\sffamily\fontsize{10.000000}{12.000000}\selectfont 0.8}%
\end{pgfscope}%
\begin{pgfscope}%
\pgfsetbuttcap%
\pgfsetroundjoin%
\definecolor{currentfill}{rgb}{0.000000,0.000000,0.000000}%
\pgfsetfillcolor{currentfill}%
\pgfsetlinewidth{0.803000pt}%
\definecolor{currentstroke}{rgb}{0.000000,0.000000,0.000000}%
\pgfsetstrokecolor{currentstroke}%
\pgfsetdash{}{0pt}%
\pgfsys@defobject{currentmarker}{\pgfqpoint{0.000000in}{-0.048611in}}{\pgfqpoint{0.000000in}{0.000000in}}{%
\pgfpathmoveto{\pgfqpoint{0.000000in}{0.000000in}}%
\pgfpathlineto{\pgfqpoint{0.000000in}{-0.048611in}}%
\pgfusepath{stroke,fill}%
}%
\begin{pgfscope}%
\pgfsys@transformshift{5.534545in}{0.528000in}%
\pgfsys@useobject{currentmarker}{}%
\end{pgfscope}%
\end{pgfscope}%
\begin{pgfscope}%
\definecolor{textcolor}{rgb}{0.000000,0.000000,0.000000}%
\pgfsetstrokecolor{textcolor}%
\pgfsetfillcolor{textcolor}%
\pgftext[x=5.534545in,y=0.430778in,,top]{\color{textcolor}\sffamily\fontsize{10.000000}{12.000000}\selectfont 1.0}%
\end{pgfscope}%
\begin{pgfscope}%
\definecolor{textcolor}{rgb}{0.000000,0.000000,0.000000}%
\pgfsetstrokecolor{textcolor}%
\pgfsetfillcolor{textcolor}%
\pgftext[x=3.280000in,y=0.240809in,,top]{\color{textcolor}\sffamily\fontsize{10.000000}{12.000000}\selectfont \(\displaystyle \|n_{1} - n_{2}\|_{L^{\infty}\mleft(D_{R},\mathbb{R}\mright)}\)}%
\end{pgfscope}%
\begin{pgfscope}%
\pgfsetbuttcap%
\pgfsetroundjoin%
\definecolor{currentfill}{rgb}{0.000000,0.000000,0.000000}%
\pgfsetfillcolor{currentfill}%
\pgfsetlinewidth{0.803000pt}%
\definecolor{currentstroke}{rgb}{0.000000,0.000000,0.000000}%
\pgfsetstrokecolor{currentstroke}%
\pgfsetdash{}{0pt}%
\pgfsys@defobject{currentmarker}{\pgfqpoint{-0.048611in}{0.000000in}}{\pgfqpoint{0.000000in}{0.000000in}}{%
\pgfpathmoveto{\pgfqpoint{0.000000in}{0.000000in}}%
\pgfpathlineto{\pgfqpoint{-0.048611in}{0.000000in}}%
\pgfusepath{stroke,fill}%
}%
\begin{pgfscope}%
\pgfsys@transformshift{0.800000in}{0.700721in}%
\pgfsys@useobject{currentmarker}{}%
\end{pgfscope}%
\end{pgfscope}%
\begin{pgfscope}%
\definecolor{textcolor}{rgb}{0.000000,0.000000,0.000000}%
\pgfsetstrokecolor{textcolor}%
\pgfsetfillcolor{textcolor}%
\pgftext[x=0.501581in,y=0.647960in,left,base]{\color{textcolor}\sffamily\fontsize{10.000000}{12.000000}\selectfont \(\displaystyle {10^{1}}\)}%
\end{pgfscope}%
\begin{pgfscope}%
\pgfsetbuttcap%
\pgfsetroundjoin%
\definecolor{currentfill}{rgb}{0.000000,0.000000,0.000000}%
\pgfsetfillcolor{currentfill}%
\pgfsetlinewidth{0.803000pt}%
\definecolor{currentstroke}{rgb}{0.000000,0.000000,0.000000}%
\pgfsetstrokecolor{currentstroke}%
\pgfsetdash{}{0pt}%
\pgfsys@defobject{currentmarker}{\pgfqpoint{-0.048611in}{0.000000in}}{\pgfqpoint{0.000000in}{0.000000in}}{%
\pgfpathmoveto{\pgfqpoint{0.000000in}{0.000000in}}%
\pgfpathlineto{\pgfqpoint{-0.048611in}{0.000000in}}%
\pgfusepath{stroke,fill}%
}%
\begin{pgfscope}%
\pgfsys@transformshift{0.800000in}{1.371777in}%
\pgfsys@useobject{currentmarker}{}%
\end{pgfscope}%
\end{pgfscope}%
\begin{pgfscope}%
\definecolor{textcolor}{rgb}{0.000000,0.000000,0.000000}%
\pgfsetstrokecolor{textcolor}%
\pgfsetfillcolor{textcolor}%
\pgftext[x=0.501581in,y=1.319015in,left,base]{\color{textcolor}\sffamily\fontsize{10.000000}{12.000000}\selectfont \(\displaystyle {10^{2}}\)}%
\end{pgfscope}%
\begin{pgfscope}%
\pgfsetbuttcap%
\pgfsetroundjoin%
\definecolor{currentfill}{rgb}{0.000000,0.000000,0.000000}%
\pgfsetfillcolor{currentfill}%
\pgfsetlinewidth{0.803000pt}%
\definecolor{currentstroke}{rgb}{0.000000,0.000000,0.000000}%
\pgfsetstrokecolor{currentstroke}%
\pgfsetdash{}{0pt}%
\pgfsys@defobject{currentmarker}{\pgfqpoint{-0.048611in}{0.000000in}}{\pgfqpoint{0.000000in}{0.000000in}}{%
\pgfpathmoveto{\pgfqpoint{0.000000in}{0.000000in}}%
\pgfpathlineto{\pgfqpoint{-0.048611in}{0.000000in}}%
\pgfusepath{stroke,fill}%
}%
\begin{pgfscope}%
\pgfsys@transformshift{0.800000in}{2.042833in}%
\pgfsys@useobject{currentmarker}{}%
\end{pgfscope}%
\end{pgfscope}%
\begin{pgfscope}%
\definecolor{textcolor}{rgb}{0.000000,0.000000,0.000000}%
\pgfsetstrokecolor{textcolor}%
\pgfsetfillcolor{textcolor}%
\pgftext[x=0.501581in,y=1.990071in,left,base]{\color{textcolor}\sffamily\fontsize{10.000000}{12.000000}\selectfont \(\displaystyle {10^{3}}\)}%
\end{pgfscope}%
\begin{pgfscope}%
\pgfsetbuttcap%
\pgfsetroundjoin%
\definecolor{currentfill}{rgb}{0.000000,0.000000,0.000000}%
\pgfsetfillcolor{currentfill}%
\pgfsetlinewidth{0.803000pt}%
\definecolor{currentstroke}{rgb}{0.000000,0.000000,0.000000}%
\pgfsetstrokecolor{currentstroke}%
\pgfsetdash{}{0pt}%
\pgfsys@defobject{currentmarker}{\pgfqpoint{-0.048611in}{0.000000in}}{\pgfqpoint{0.000000in}{0.000000in}}{%
\pgfpathmoveto{\pgfqpoint{0.000000in}{0.000000in}}%
\pgfpathlineto{\pgfqpoint{-0.048611in}{0.000000in}}%
\pgfusepath{stroke,fill}%
}%
\begin{pgfscope}%
\pgfsys@transformshift{0.800000in}{2.713889in}%
\pgfsys@useobject{currentmarker}{}%
\end{pgfscope}%
\end{pgfscope}%
\begin{pgfscope}%
\definecolor{textcolor}{rgb}{0.000000,0.000000,0.000000}%
\pgfsetstrokecolor{textcolor}%
\pgfsetfillcolor{textcolor}%
\pgftext[x=0.501581in,y=2.661127in,left,base]{\color{textcolor}\sffamily\fontsize{10.000000}{12.000000}\selectfont \(\displaystyle {10^{4}}\)}%
\end{pgfscope}%
\begin{pgfscope}%
\pgfsetbuttcap%
\pgfsetroundjoin%
\definecolor{currentfill}{rgb}{0.000000,0.000000,0.000000}%
\pgfsetfillcolor{currentfill}%
\pgfsetlinewidth{0.803000pt}%
\definecolor{currentstroke}{rgb}{0.000000,0.000000,0.000000}%
\pgfsetstrokecolor{currentstroke}%
\pgfsetdash{}{0pt}%
\pgfsys@defobject{currentmarker}{\pgfqpoint{-0.048611in}{0.000000in}}{\pgfqpoint{0.000000in}{0.000000in}}{%
\pgfpathmoveto{\pgfqpoint{0.000000in}{0.000000in}}%
\pgfpathlineto{\pgfqpoint{-0.048611in}{0.000000in}}%
\pgfusepath{stroke,fill}%
}%
\begin{pgfscope}%
\pgfsys@transformshift{0.800000in}{3.384944in}%
\pgfsys@useobject{currentmarker}{}%
\end{pgfscope}%
\end{pgfscope}%
\begin{pgfscope}%
\definecolor{textcolor}{rgb}{0.000000,0.000000,0.000000}%
\pgfsetstrokecolor{textcolor}%
\pgfsetfillcolor{textcolor}%
\pgftext[x=0.501581in,y=3.332183in,left,base]{\color{textcolor}\sffamily\fontsize{10.000000}{12.000000}\selectfont \(\displaystyle {10^{5}}\)}%
\end{pgfscope}%
\begin{pgfscope}%
\pgfsetbuttcap%
\pgfsetroundjoin%
\definecolor{currentfill}{rgb}{0.000000,0.000000,0.000000}%
\pgfsetfillcolor{currentfill}%
\pgfsetlinewidth{0.803000pt}%
\definecolor{currentstroke}{rgb}{0.000000,0.000000,0.000000}%
\pgfsetstrokecolor{currentstroke}%
\pgfsetdash{}{0pt}%
\pgfsys@defobject{currentmarker}{\pgfqpoint{-0.048611in}{0.000000in}}{\pgfqpoint{0.000000in}{0.000000in}}{%
\pgfpathmoveto{\pgfqpoint{0.000000in}{0.000000in}}%
\pgfpathlineto{\pgfqpoint{-0.048611in}{0.000000in}}%
\pgfusepath{stroke,fill}%
}%
\begin{pgfscope}%
\pgfsys@transformshift{0.800000in}{4.056000in}%
\pgfsys@useobject{currentmarker}{}%
\end{pgfscope}%
\end{pgfscope}%
\begin{pgfscope}%
\definecolor{textcolor}{rgb}{0.000000,0.000000,0.000000}%
\pgfsetstrokecolor{textcolor}%
\pgfsetfillcolor{textcolor}%
\pgftext[x=0.501581in,y=4.003238in,left,base]{\color{textcolor}\sffamily\fontsize{10.000000}{12.000000}\selectfont \(\displaystyle {10^{6}}\)}%
\end{pgfscope}%
\begin{pgfscope}%
\pgfsetbuttcap%
\pgfsetroundjoin%
\definecolor{currentfill}{rgb}{0.000000,0.000000,0.000000}%
\pgfsetfillcolor{currentfill}%
\pgfsetlinewidth{0.602250pt}%
\definecolor{currentstroke}{rgb}{0.000000,0.000000,0.000000}%
\pgfsetstrokecolor{currentstroke}%
\pgfsetdash{}{0pt}%
\pgfsys@defobject{currentmarker}{\pgfqpoint{-0.027778in}{0.000000in}}{\pgfqpoint{0.000000in}{0.000000in}}{%
\pgfpathmoveto{\pgfqpoint{0.000000in}{0.000000in}}%
\pgfpathlineto{\pgfqpoint{-0.027778in}{0.000000in}}%
\pgfusepath{stroke,fill}%
}%
\begin{pgfscope}%
\pgfsys@transformshift{0.800000in}{0.551848in}%
\pgfsys@useobject{currentmarker}{}%
\end{pgfscope}%
\end{pgfscope}%
\begin{pgfscope}%
\pgfsetbuttcap%
\pgfsetroundjoin%
\definecolor{currentfill}{rgb}{0.000000,0.000000,0.000000}%
\pgfsetfillcolor{currentfill}%
\pgfsetlinewidth{0.602250pt}%
\definecolor{currentstroke}{rgb}{0.000000,0.000000,0.000000}%
\pgfsetstrokecolor{currentstroke}%
\pgfsetdash{}{0pt}%
\pgfsys@defobject{currentmarker}{\pgfqpoint{-0.027778in}{0.000000in}}{\pgfqpoint{0.000000in}{0.000000in}}{%
\pgfpathmoveto{\pgfqpoint{0.000000in}{0.000000in}}%
\pgfpathlineto{\pgfqpoint{-0.027778in}{0.000000in}}%
\pgfusepath{stroke,fill}%
}%
\begin{pgfscope}%
\pgfsys@transformshift{0.800000in}{0.596773in}%
\pgfsys@useobject{currentmarker}{}%
\end{pgfscope}%
\end{pgfscope}%
\begin{pgfscope}%
\pgfsetbuttcap%
\pgfsetroundjoin%
\definecolor{currentfill}{rgb}{0.000000,0.000000,0.000000}%
\pgfsetfillcolor{currentfill}%
\pgfsetlinewidth{0.602250pt}%
\definecolor{currentstroke}{rgb}{0.000000,0.000000,0.000000}%
\pgfsetstrokecolor{currentstroke}%
\pgfsetdash{}{0pt}%
\pgfsys@defobject{currentmarker}{\pgfqpoint{-0.027778in}{0.000000in}}{\pgfqpoint{0.000000in}{0.000000in}}{%
\pgfpathmoveto{\pgfqpoint{0.000000in}{0.000000in}}%
\pgfpathlineto{\pgfqpoint{-0.027778in}{0.000000in}}%
\pgfusepath{stroke,fill}%
}%
\begin{pgfscope}%
\pgfsys@transformshift{0.800000in}{0.635689in}%
\pgfsys@useobject{currentmarker}{}%
\end{pgfscope}%
\end{pgfscope}%
\begin{pgfscope}%
\pgfsetbuttcap%
\pgfsetroundjoin%
\definecolor{currentfill}{rgb}{0.000000,0.000000,0.000000}%
\pgfsetfillcolor{currentfill}%
\pgfsetlinewidth{0.602250pt}%
\definecolor{currentstroke}{rgb}{0.000000,0.000000,0.000000}%
\pgfsetstrokecolor{currentstroke}%
\pgfsetdash{}{0pt}%
\pgfsys@defobject{currentmarker}{\pgfqpoint{-0.027778in}{0.000000in}}{\pgfqpoint{0.000000in}{0.000000in}}{%
\pgfpathmoveto{\pgfqpoint{0.000000in}{0.000000in}}%
\pgfpathlineto{\pgfqpoint{-0.027778in}{0.000000in}}%
\pgfusepath{stroke,fill}%
}%
\begin{pgfscope}%
\pgfsys@transformshift{0.800000in}{0.670015in}%
\pgfsys@useobject{currentmarker}{}%
\end{pgfscope}%
\end{pgfscope}%
\begin{pgfscope}%
\pgfsetbuttcap%
\pgfsetroundjoin%
\definecolor{currentfill}{rgb}{0.000000,0.000000,0.000000}%
\pgfsetfillcolor{currentfill}%
\pgfsetlinewidth{0.602250pt}%
\definecolor{currentstroke}{rgb}{0.000000,0.000000,0.000000}%
\pgfsetstrokecolor{currentstroke}%
\pgfsetdash{}{0pt}%
\pgfsys@defobject{currentmarker}{\pgfqpoint{-0.027778in}{0.000000in}}{\pgfqpoint{0.000000in}{0.000000in}}{%
\pgfpathmoveto{\pgfqpoint{0.000000in}{0.000000in}}%
\pgfpathlineto{\pgfqpoint{-0.027778in}{0.000000in}}%
\pgfusepath{stroke,fill}%
}%
\begin{pgfscope}%
\pgfsys@transformshift{0.800000in}{0.902729in}%
\pgfsys@useobject{currentmarker}{}%
\end{pgfscope}%
\end{pgfscope}%
\begin{pgfscope}%
\pgfsetbuttcap%
\pgfsetroundjoin%
\definecolor{currentfill}{rgb}{0.000000,0.000000,0.000000}%
\pgfsetfillcolor{currentfill}%
\pgfsetlinewidth{0.602250pt}%
\definecolor{currentstroke}{rgb}{0.000000,0.000000,0.000000}%
\pgfsetstrokecolor{currentstroke}%
\pgfsetdash{}{0pt}%
\pgfsys@defobject{currentmarker}{\pgfqpoint{-0.027778in}{0.000000in}}{\pgfqpoint{0.000000in}{0.000000in}}{%
\pgfpathmoveto{\pgfqpoint{0.000000in}{0.000000in}}%
\pgfpathlineto{\pgfqpoint{-0.027778in}{0.000000in}}%
\pgfusepath{stroke,fill}%
}%
\begin{pgfscope}%
\pgfsys@transformshift{0.800000in}{1.020896in}%
\pgfsys@useobject{currentmarker}{}%
\end{pgfscope}%
\end{pgfscope}%
\begin{pgfscope}%
\pgfsetbuttcap%
\pgfsetroundjoin%
\definecolor{currentfill}{rgb}{0.000000,0.000000,0.000000}%
\pgfsetfillcolor{currentfill}%
\pgfsetlinewidth{0.602250pt}%
\definecolor{currentstroke}{rgb}{0.000000,0.000000,0.000000}%
\pgfsetstrokecolor{currentstroke}%
\pgfsetdash{}{0pt}%
\pgfsys@defobject{currentmarker}{\pgfqpoint{-0.027778in}{0.000000in}}{\pgfqpoint{0.000000in}{0.000000in}}{%
\pgfpathmoveto{\pgfqpoint{0.000000in}{0.000000in}}%
\pgfpathlineto{\pgfqpoint{-0.027778in}{0.000000in}}%
\pgfusepath{stroke,fill}%
}%
\begin{pgfscope}%
\pgfsys@transformshift{0.800000in}{1.104737in}%
\pgfsys@useobject{currentmarker}{}%
\end{pgfscope}%
\end{pgfscope}%
\begin{pgfscope}%
\pgfsetbuttcap%
\pgfsetroundjoin%
\definecolor{currentfill}{rgb}{0.000000,0.000000,0.000000}%
\pgfsetfillcolor{currentfill}%
\pgfsetlinewidth{0.602250pt}%
\definecolor{currentstroke}{rgb}{0.000000,0.000000,0.000000}%
\pgfsetstrokecolor{currentstroke}%
\pgfsetdash{}{0pt}%
\pgfsys@defobject{currentmarker}{\pgfqpoint{-0.027778in}{0.000000in}}{\pgfqpoint{0.000000in}{0.000000in}}{%
\pgfpathmoveto{\pgfqpoint{0.000000in}{0.000000in}}%
\pgfpathlineto{\pgfqpoint{-0.027778in}{0.000000in}}%
\pgfusepath{stroke,fill}%
}%
\begin{pgfscope}%
\pgfsys@transformshift{0.800000in}{1.169769in}%
\pgfsys@useobject{currentmarker}{}%
\end{pgfscope}%
\end{pgfscope}%
\begin{pgfscope}%
\pgfsetbuttcap%
\pgfsetroundjoin%
\definecolor{currentfill}{rgb}{0.000000,0.000000,0.000000}%
\pgfsetfillcolor{currentfill}%
\pgfsetlinewidth{0.602250pt}%
\definecolor{currentstroke}{rgb}{0.000000,0.000000,0.000000}%
\pgfsetstrokecolor{currentstroke}%
\pgfsetdash{}{0pt}%
\pgfsys@defobject{currentmarker}{\pgfqpoint{-0.027778in}{0.000000in}}{\pgfqpoint{0.000000in}{0.000000in}}{%
\pgfpathmoveto{\pgfqpoint{0.000000in}{0.000000in}}%
\pgfpathlineto{\pgfqpoint{-0.027778in}{0.000000in}}%
\pgfusepath{stroke,fill}%
}%
\begin{pgfscope}%
\pgfsys@transformshift{0.800000in}{1.222904in}%
\pgfsys@useobject{currentmarker}{}%
\end{pgfscope}%
\end{pgfscope}%
\begin{pgfscope}%
\pgfsetbuttcap%
\pgfsetroundjoin%
\definecolor{currentfill}{rgb}{0.000000,0.000000,0.000000}%
\pgfsetfillcolor{currentfill}%
\pgfsetlinewidth{0.602250pt}%
\definecolor{currentstroke}{rgb}{0.000000,0.000000,0.000000}%
\pgfsetstrokecolor{currentstroke}%
\pgfsetdash{}{0pt}%
\pgfsys@defobject{currentmarker}{\pgfqpoint{-0.027778in}{0.000000in}}{\pgfqpoint{0.000000in}{0.000000in}}{%
\pgfpathmoveto{\pgfqpoint{0.000000in}{0.000000in}}%
\pgfpathlineto{\pgfqpoint{-0.027778in}{0.000000in}}%
\pgfusepath{stroke,fill}%
}%
\begin{pgfscope}%
\pgfsys@transformshift{0.800000in}{1.267829in}%
\pgfsys@useobject{currentmarker}{}%
\end{pgfscope}%
\end{pgfscope}%
\begin{pgfscope}%
\pgfsetbuttcap%
\pgfsetroundjoin%
\definecolor{currentfill}{rgb}{0.000000,0.000000,0.000000}%
\pgfsetfillcolor{currentfill}%
\pgfsetlinewidth{0.602250pt}%
\definecolor{currentstroke}{rgb}{0.000000,0.000000,0.000000}%
\pgfsetstrokecolor{currentstroke}%
\pgfsetdash{}{0pt}%
\pgfsys@defobject{currentmarker}{\pgfqpoint{-0.027778in}{0.000000in}}{\pgfqpoint{0.000000in}{0.000000in}}{%
\pgfpathmoveto{\pgfqpoint{0.000000in}{0.000000in}}%
\pgfpathlineto{\pgfqpoint{-0.027778in}{0.000000in}}%
\pgfusepath{stroke,fill}%
}%
\begin{pgfscope}%
\pgfsys@transformshift{0.800000in}{1.306745in}%
\pgfsys@useobject{currentmarker}{}%
\end{pgfscope}%
\end{pgfscope}%
\begin{pgfscope}%
\pgfsetbuttcap%
\pgfsetroundjoin%
\definecolor{currentfill}{rgb}{0.000000,0.000000,0.000000}%
\pgfsetfillcolor{currentfill}%
\pgfsetlinewidth{0.602250pt}%
\definecolor{currentstroke}{rgb}{0.000000,0.000000,0.000000}%
\pgfsetstrokecolor{currentstroke}%
\pgfsetdash{}{0pt}%
\pgfsys@defobject{currentmarker}{\pgfqpoint{-0.027778in}{0.000000in}}{\pgfqpoint{0.000000in}{0.000000in}}{%
\pgfpathmoveto{\pgfqpoint{0.000000in}{0.000000in}}%
\pgfpathlineto{\pgfqpoint{-0.027778in}{0.000000in}}%
\pgfusepath{stroke,fill}%
}%
\begin{pgfscope}%
\pgfsys@transformshift{0.800000in}{1.341071in}%
\pgfsys@useobject{currentmarker}{}%
\end{pgfscope}%
\end{pgfscope}%
\begin{pgfscope}%
\pgfsetbuttcap%
\pgfsetroundjoin%
\definecolor{currentfill}{rgb}{0.000000,0.000000,0.000000}%
\pgfsetfillcolor{currentfill}%
\pgfsetlinewidth{0.602250pt}%
\definecolor{currentstroke}{rgb}{0.000000,0.000000,0.000000}%
\pgfsetstrokecolor{currentstroke}%
\pgfsetdash{}{0pt}%
\pgfsys@defobject{currentmarker}{\pgfqpoint{-0.027778in}{0.000000in}}{\pgfqpoint{0.000000in}{0.000000in}}{%
\pgfpathmoveto{\pgfqpoint{0.000000in}{0.000000in}}%
\pgfpathlineto{\pgfqpoint{-0.027778in}{0.000000in}}%
\pgfusepath{stroke,fill}%
}%
\begin{pgfscope}%
\pgfsys@transformshift{0.800000in}{1.573785in}%
\pgfsys@useobject{currentmarker}{}%
\end{pgfscope}%
\end{pgfscope}%
\begin{pgfscope}%
\pgfsetbuttcap%
\pgfsetroundjoin%
\definecolor{currentfill}{rgb}{0.000000,0.000000,0.000000}%
\pgfsetfillcolor{currentfill}%
\pgfsetlinewidth{0.602250pt}%
\definecolor{currentstroke}{rgb}{0.000000,0.000000,0.000000}%
\pgfsetstrokecolor{currentstroke}%
\pgfsetdash{}{0pt}%
\pgfsys@defobject{currentmarker}{\pgfqpoint{-0.027778in}{0.000000in}}{\pgfqpoint{0.000000in}{0.000000in}}{%
\pgfpathmoveto{\pgfqpoint{0.000000in}{0.000000in}}%
\pgfpathlineto{\pgfqpoint{-0.027778in}{0.000000in}}%
\pgfusepath{stroke,fill}%
}%
\begin{pgfscope}%
\pgfsys@transformshift{0.800000in}{1.691952in}%
\pgfsys@useobject{currentmarker}{}%
\end{pgfscope}%
\end{pgfscope}%
\begin{pgfscope}%
\pgfsetbuttcap%
\pgfsetroundjoin%
\definecolor{currentfill}{rgb}{0.000000,0.000000,0.000000}%
\pgfsetfillcolor{currentfill}%
\pgfsetlinewidth{0.602250pt}%
\definecolor{currentstroke}{rgb}{0.000000,0.000000,0.000000}%
\pgfsetstrokecolor{currentstroke}%
\pgfsetdash{}{0pt}%
\pgfsys@defobject{currentmarker}{\pgfqpoint{-0.027778in}{0.000000in}}{\pgfqpoint{0.000000in}{0.000000in}}{%
\pgfpathmoveto{\pgfqpoint{0.000000in}{0.000000in}}%
\pgfpathlineto{\pgfqpoint{-0.027778in}{0.000000in}}%
\pgfusepath{stroke,fill}%
}%
\begin{pgfscope}%
\pgfsys@transformshift{0.800000in}{1.775793in}%
\pgfsys@useobject{currentmarker}{}%
\end{pgfscope}%
\end{pgfscope}%
\begin{pgfscope}%
\pgfsetbuttcap%
\pgfsetroundjoin%
\definecolor{currentfill}{rgb}{0.000000,0.000000,0.000000}%
\pgfsetfillcolor{currentfill}%
\pgfsetlinewidth{0.602250pt}%
\definecolor{currentstroke}{rgb}{0.000000,0.000000,0.000000}%
\pgfsetstrokecolor{currentstroke}%
\pgfsetdash{}{0pt}%
\pgfsys@defobject{currentmarker}{\pgfqpoint{-0.027778in}{0.000000in}}{\pgfqpoint{0.000000in}{0.000000in}}{%
\pgfpathmoveto{\pgfqpoint{0.000000in}{0.000000in}}%
\pgfpathlineto{\pgfqpoint{-0.027778in}{0.000000in}}%
\pgfusepath{stroke,fill}%
}%
\begin{pgfscope}%
\pgfsys@transformshift{0.800000in}{1.840825in}%
\pgfsys@useobject{currentmarker}{}%
\end{pgfscope}%
\end{pgfscope}%
\begin{pgfscope}%
\pgfsetbuttcap%
\pgfsetroundjoin%
\definecolor{currentfill}{rgb}{0.000000,0.000000,0.000000}%
\pgfsetfillcolor{currentfill}%
\pgfsetlinewidth{0.602250pt}%
\definecolor{currentstroke}{rgb}{0.000000,0.000000,0.000000}%
\pgfsetstrokecolor{currentstroke}%
\pgfsetdash{}{0pt}%
\pgfsys@defobject{currentmarker}{\pgfqpoint{-0.027778in}{0.000000in}}{\pgfqpoint{0.000000in}{0.000000in}}{%
\pgfpathmoveto{\pgfqpoint{0.000000in}{0.000000in}}%
\pgfpathlineto{\pgfqpoint{-0.027778in}{0.000000in}}%
\pgfusepath{stroke,fill}%
}%
\begin{pgfscope}%
\pgfsys@transformshift{0.800000in}{1.893960in}%
\pgfsys@useobject{currentmarker}{}%
\end{pgfscope}%
\end{pgfscope}%
\begin{pgfscope}%
\pgfsetbuttcap%
\pgfsetroundjoin%
\definecolor{currentfill}{rgb}{0.000000,0.000000,0.000000}%
\pgfsetfillcolor{currentfill}%
\pgfsetlinewidth{0.602250pt}%
\definecolor{currentstroke}{rgb}{0.000000,0.000000,0.000000}%
\pgfsetstrokecolor{currentstroke}%
\pgfsetdash{}{0pt}%
\pgfsys@defobject{currentmarker}{\pgfqpoint{-0.027778in}{0.000000in}}{\pgfqpoint{0.000000in}{0.000000in}}{%
\pgfpathmoveto{\pgfqpoint{0.000000in}{0.000000in}}%
\pgfpathlineto{\pgfqpoint{-0.027778in}{0.000000in}}%
\pgfusepath{stroke,fill}%
}%
\begin{pgfscope}%
\pgfsys@transformshift{0.800000in}{1.938885in}%
\pgfsys@useobject{currentmarker}{}%
\end{pgfscope}%
\end{pgfscope}%
\begin{pgfscope}%
\pgfsetbuttcap%
\pgfsetroundjoin%
\definecolor{currentfill}{rgb}{0.000000,0.000000,0.000000}%
\pgfsetfillcolor{currentfill}%
\pgfsetlinewidth{0.602250pt}%
\definecolor{currentstroke}{rgb}{0.000000,0.000000,0.000000}%
\pgfsetstrokecolor{currentstroke}%
\pgfsetdash{}{0pt}%
\pgfsys@defobject{currentmarker}{\pgfqpoint{-0.027778in}{0.000000in}}{\pgfqpoint{0.000000in}{0.000000in}}{%
\pgfpathmoveto{\pgfqpoint{0.000000in}{0.000000in}}%
\pgfpathlineto{\pgfqpoint{-0.027778in}{0.000000in}}%
\pgfusepath{stroke,fill}%
}%
\begin{pgfscope}%
\pgfsys@transformshift{0.800000in}{1.977801in}%
\pgfsys@useobject{currentmarker}{}%
\end{pgfscope}%
\end{pgfscope}%
\begin{pgfscope}%
\pgfsetbuttcap%
\pgfsetroundjoin%
\definecolor{currentfill}{rgb}{0.000000,0.000000,0.000000}%
\pgfsetfillcolor{currentfill}%
\pgfsetlinewidth{0.602250pt}%
\definecolor{currentstroke}{rgb}{0.000000,0.000000,0.000000}%
\pgfsetstrokecolor{currentstroke}%
\pgfsetdash{}{0pt}%
\pgfsys@defobject{currentmarker}{\pgfqpoint{-0.027778in}{0.000000in}}{\pgfqpoint{0.000000in}{0.000000in}}{%
\pgfpathmoveto{\pgfqpoint{0.000000in}{0.000000in}}%
\pgfpathlineto{\pgfqpoint{-0.027778in}{0.000000in}}%
\pgfusepath{stroke,fill}%
}%
\begin{pgfscope}%
\pgfsys@transformshift{0.800000in}{2.012127in}%
\pgfsys@useobject{currentmarker}{}%
\end{pgfscope}%
\end{pgfscope}%
\begin{pgfscope}%
\pgfsetbuttcap%
\pgfsetroundjoin%
\definecolor{currentfill}{rgb}{0.000000,0.000000,0.000000}%
\pgfsetfillcolor{currentfill}%
\pgfsetlinewidth{0.602250pt}%
\definecolor{currentstroke}{rgb}{0.000000,0.000000,0.000000}%
\pgfsetstrokecolor{currentstroke}%
\pgfsetdash{}{0pt}%
\pgfsys@defobject{currentmarker}{\pgfqpoint{-0.027778in}{0.000000in}}{\pgfqpoint{0.000000in}{0.000000in}}{%
\pgfpathmoveto{\pgfqpoint{0.000000in}{0.000000in}}%
\pgfpathlineto{\pgfqpoint{-0.027778in}{0.000000in}}%
\pgfusepath{stroke,fill}%
}%
\begin{pgfscope}%
\pgfsys@transformshift{0.800000in}{2.244841in}%
\pgfsys@useobject{currentmarker}{}%
\end{pgfscope}%
\end{pgfscope}%
\begin{pgfscope}%
\pgfsetbuttcap%
\pgfsetroundjoin%
\definecolor{currentfill}{rgb}{0.000000,0.000000,0.000000}%
\pgfsetfillcolor{currentfill}%
\pgfsetlinewidth{0.602250pt}%
\definecolor{currentstroke}{rgb}{0.000000,0.000000,0.000000}%
\pgfsetstrokecolor{currentstroke}%
\pgfsetdash{}{0pt}%
\pgfsys@defobject{currentmarker}{\pgfqpoint{-0.027778in}{0.000000in}}{\pgfqpoint{0.000000in}{0.000000in}}{%
\pgfpathmoveto{\pgfqpoint{0.000000in}{0.000000in}}%
\pgfpathlineto{\pgfqpoint{-0.027778in}{0.000000in}}%
\pgfusepath{stroke,fill}%
}%
\begin{pgfscope}%
\pgfsys@transformshift{0.800000in}{2.363008in}%
\pgfsys@useobject{currentmarker}{}%
\end{pgfscope}%
\end{pgfscope}%
\begin{pgfscope}%
\pgfsetbuttcap%
\pgfsetroundjoin%
\definecolor{currentfill}{rgb}{0.000000,0.000000,0.000000}%
\pgfsetfillcolor{currentfill}%
\pgfsetlinewidth{0.602250pt}%
\definecolor{currentstroke}{rgb}{0.000000,0.000000,0.000000}%
\pgfsetstrokecolor{currentstroke}%
\pgfsetdash{}{0pt}%
\pgfsys@defobject{currentmarker}{\pgfqpoint{-0.027778in}{0.000000in}}{\pgfqpoint{0.000000in}{0.000000in}}{%
\pgfpathmoveto{\pgfqpoint{0.000000in}{0.000000in}}%
\pgfpathlineto{\pgfqpoint{-0.027778in}{0.000000in}}%
\pgfusepath{stroke,fill}%
}%
\begin{pgfscope}%
\pgfsys@transformshift{0.800000in}{2.446849in}%
\pgfsys@useobject{currentmarker}{}%
\end{pgfscope}%
\end{pgfscope}%
\begin{pgfscope}%
\pgfsetbuttcap%
\pgfsetroundjoin%
\definecolor{currentfill}{rgb}{0.000000,0.000000,0.000000}%
\pgfsetfillcolor{currentfill}%
\pgfsetlinewidth{0.602250pt}%
\definecolor{currentstroke}{rgb}{0.000000,0.000000,0.000000}%
\pgfsetstrokecolor{currentstroke}%
\pgfsetdash{}{0pt}%
\pgfsys@defobject{currentmarker}{\pgfqpoint{-0.027778in}{0.000000in}}{\pgfqpoint{0.000000in}{0.000000in}}{%
\pgfpathmoveto{\pgfqpoint{0.000000in}{0.000000in}}%
\pgfpathlineto{\pgfqpoint{-0.027778in}{0.000000in}}%
\pgfusepath{stroke,fill}%
}%
\begin{pgfscope}%
\pgfsys@transformshift{0.800000in}{2.511881in}%
\pgfsys@useobject{currentmarker}{}%
\end{pgfscope}%
\end{pgfscope}%
\begin{pgfscope}%
\pgfsetbuttcap%
\pgfsetroundjoin%
\definecolor{currentfill}{rgb}{0.000000,0.000000,0.000000}%
\pgfsetfillcolor{currentfill}%
\pgfsetlinewidth{0.602250pt}%
\definecolor{currentstroke}{rgb}{0.000000,0.000000,0.000000}%
\pgfsetstrokecolor{currentstroke}%
\pgfsetdash{}{0pt}%
\pgfsys@defobject{currentmarker}{\pgfqpoint{-0.027778in}{0.000000in}}{\pgfqpoint{0.000000in}{0.000000in}}{%
\pgfpathmoveto{\pgfqpoint{0.000000in}{0.000000in}}%
\pgfpathlineto{\pgfqpoint{-0.027778in}{0.000000in}}%
\pgfusepath{stroke,fill}%
}%
\begin{pgfscope}%
\pgfsys@transformshift{0.800000in}{2.565016in}%
\pgfsys@useobject{currentmarker}{}%
\end{pgfscope}%
\end{pgfscope}%
\begin{pgfscope}%
\pgfsetbuttcap%
\pgfsetroundjoin%
\definecolor{currentfill}{rgb}{0.000000,0.000000,0.000000}%
\pgfsetfillcolor{currentfill}%
\pgfsetlinewidth{0.602250pt}%
\definecolor{currentstroke}{rgb}{0.000000,0.000000,0.000000}%
\pgfsetstrokecolor{currentstroke}%
\pgfsetdash{}{0pt}%
\pgfsys@defobject{currentmarker}{\pgfqpoint{-0.027778in}{0.000000in}}{\pgfqpoint{0.000000in}{0.000000in}}{%
\pgfpathmoveto{\pgfqpoint{0.000000in}{0.000000in}}%
\pgfpathlineto{\pgfqpoint{-0.027778in}{0.000000in}}%
\pgfusepath{stroke,fill}%
}%
\begin{pgfscope}%
\pgfsys@transformshift{0.800000in}{2.609941in}%
\pgfsys@useobject{currentmarker}{}%
\end{pgfscope}%
\end{pgfscope}%
\begin{pgfscope}%
\pgfsetbuttcap%
\pgfsetroundjoin%
\definecolor{currentfill}{rgb}{0.000000,0.000000,0.000000}%
\pgfsetfillcolor{currentfill}%
\pgfsetlinewidth{0.602250pt}%
\definecolor{currentstroke}{rgb}{0.000000,0.000000,0.000000}%
\pgfsetstrokecolor{currentstroke}%
\pgfsetdash{}{0pt}%
\pgfsys@defobject{currentmarker}{\pgfqpoint{-0.027778in}{0.000000in}}{\pgfqpoint{0.000000in}{0.000000in}}{%
\pgfpathmoveto{\pgfqpoint{0.000000in}{0.000000in}}%
\pgfpathlineto{\pgfqpoint{-0.027778in}{0.000000in}}%
\pgfusepath{stroke,fill}%
}%
\begin{pgfscope}%
\pgfsys@transformshift{0.800000in}{2.648856in}%
\pgfsys@useobject{currentmarker}{}%
\end{pgfscope}%
\end{pgfscope}%
\begin{pgfscope}%
\pgfsetbuttcap%
\pgfsetroundjoin%
\definecolor{currentfill}{rgb}{0.000000,0.000000,0.000000}%
\pgfsetfillcolor{currentfill}%
\pgfsetlinewidth{0.602250pt}%
\definecolor{currentstroke}{rgb}{0.000000,0.000000,0.000000}%
\pgfsetstrokecolor{currentstroke}%
\pgfsetdash{}{0pt}%
\pgfsys@defobject{currentmarker}{\pgfqpoint{-0.027778in}{0.000000in}}{\pgfqpoint{0.000000in}{0.000000in}}{%
\pgfpathmoveto{\pgfqpoint{0.000000in}{0.000000in}}%
\pgfpathlineto{\pgfqpoint{-0.027778in}{0.000000in}}%
\pgfusepath{stroke,fill}%
}%
\begin{pgfscope}%
\pgfsys@transformshift{0.800000in}{2.683183in}%
\pgfsys@useobject{currentmarker}{}%
\end{pgfscope}%
\end{pgfscope}%
\begin{pgfscope}%
\pgfsetbuttcap%
\pgfsetroundjoin%
\definecolor{currentfill}{rgb}{0.000000,0.000000,0.000000}%
\pgfsetfillcolor{currentfill}%
\pgfsetlinewidth{0.602250pt}%
\definecolor{currentstroke}{rgb}{0.000000,0.000000,0.000000}%
\pgfsetstrokecolor{currentstroke}%
\pgfsetdash{}{0pt}%
\pgfsys@defobject{currentmarker}{\pgfqpoint{-0.027778in}{0.000000in}}{\pgfqpoint{0.000000in}{0.000000in}}{%
\pgfpathmoveto{\pgfqpoint{0.000000in}{0.000000in}}%
\pgfpathlineto{\pgfqpoint{-0.027778in}{0.000000in}}%
\pgfusepath{stroke,fill}%
}%
\begin{pgfscope}%
\pgfsys@transformshift{0.800000in}{2.915896in}%
\pgfsys@useobject{currentmarker}{}%
\end{pgfscope}%
\end{pgfscope}%
\begin{pgfscope}%
\pgfsetbuttcap%
\pgfsetroundjoin%
\definecolor{currentfill}{rgb}{0.000000,0.000000,0.000000}%
\pgfsetfillcolor{currentfill}%
\pgfsetlinewidth{0.602250pt}%
\definecolor{currentstroke}{rgb}{0.000000,0.000000,0.000000}%
\pgfsetstrokecolor{currentstroke}%
\pgfsetdash{}{0pt}%
\pgfsys@defobject{currentmarker}{\pgfqpoint{-0.027778in}{0.000000in}}{\pgfqpoint{0.000000in}{0.000000in}}{%
\pgfpathmoveto{\pgfqpoint{0.000000in}{0.000000in}}%
\pgfpathlineto{\pgfqpoint{-0.027778in}{0.000000in}}%
\pgfusepath{stroke,fill}%
}%
\begin{pgfscope}%
\pgfsys@transformshift{0.800000in}{3.034063in}%
\pgfsys@useobject{currentmarker}{}%
\end{pgfscope}%
\end{pgfscope}%
\begin{pgfscope}%
\pgfsetbuttcap%
\pgfsetroundjoin%
\definecolor{currentfill}{rgb}{0.000000,0.000000,0.000000}%
\pgfsetfillcolor{currentfill}%
\pgfsetlinewidth{0.602250pt}%
\definecolor{currentstroke}{rgb}{0.000000,0.000000,0.000000}%
\pgfsetstrokecolor{currentstroke}%
\pgfsetdash{}{0pt}%
\pgfsys@defobject{currentmarker}{\pgfqpoint{-0.027778in}{0.000000in}}{\pgfqpoint{0.000000in}{0.000000in}}{%
\pgfpathmoveto{\pgfqpoint{0.000000in}{0.000000in}}%
\pgfpathlineto{\pgfqpoint{-0.027778in}{0.000000in}}%
\pgfusepath{stroke,fill}%
}%
\begin{pgfscope}%
\pgfsys@transformshift{0.800000in}{3.117904in}%
\pgfsys@useobject{currentmarker}{}%
\end{pgfscope}%
\end{pgfscope}%
\begin{pgfscope}%
\pgfsetbuttcap%
\pgfsetroundjoin%
\definecolor{currentfill}{rgb}{0.000000,0.000000,0.000000}%
\pgfsetfillcolor{currentfill}%
\pgfsetlinewidth{0.602250pt}%
\definecolor{currentstroke}{rgb}{0.000000,0.000000,0.000000}%
\pgfsetstrokecolor{currentstroke}%
\pgfsetdash{}{0pt}%
\pgfsys@defobject{currentmarker}{\pgfqpoint{-0.027778in}{0.000000in}}{\pgfqpoint{0.000000in}{0.000000in}}{%
\pgfpathmoveto{\pgfqpoint{0.000000in}{0.000000in}}%
\pgfpathlineto{\pgfqpoint{-0.027778in}{0.000000in}}%
\pgfusepath{stroke,fill}%
}%
\begin{pgfscope}%
\pgfsys@transformshift{0.800000in}{3.182936in}%
\pgfsys@useobject{currentmarker}{}%
\end{pgfscope}%
\end{pgfscope}%
\begin{pgfscope}%
\pgfsetbuttcap%
\pgfsetroundjoin%
\definecolor{currentfill}{rgb}{0.000000,0.000000,0.000000}%
\pgfsetfillcolor{currentfill}%
\pgfsetlinewidth{0.602250pt}%
\definecolor{currentstroke}{rgb}{0.000000,0.000000,0.000000}%
\pgfsetstrokecolor{currentstroke}%
\pgfsetdash{}{0pt}%
\pgfsys@defobject{currentmarker}{\pgfqpoint{-0.027778in}{0.000000in}}{\pgfqpoint{0.000000in}{0.000000in}}{%
\pgfpathmoveto{\pgfqpoint{0.000000in}{0.000000in}}%
\pgfpathlineto{\pgfqpoint{-0.027778in}{0.000000in}}%
\pgfusepath{stroke,fill}%
}%
\begin{pgfscope}%
\pgfsys@transformshift{0.800000in}{3.236071in}%
\pgfsys@useobject{currentmarker}{}%
\end{pgfscope}%
\end{pgfscope}%
\begin{pgfscope}%
\pgfsetbuttcap%
\pgfsetroundjoin%
\definecolor{currentfill}{rgb}{0.000000,0.000000,0.000000}%
\pgfsetfillcolor{currentfill}%
\pgfsetlinewidth{0.602250pt}%
\definecolor{currentstroke}{rgb}{0.000000,0.000000,0.000000}%
\pgfsetstrokecolor{currentstroke}%
\pgfsetdash{}{0pt}%
\pgfsys@defobject{currentmarker}{\pgfqpoint{-0.027778in}{0.000000in}}{\pgfqpoint{0.000000in}{0.000000in}}{%
\pgfpathmoveto{\pgfqpoint{0.000000in}{0.000000in}}%
\pgfpathlineto{\pgfqpoint{-0.027778in}{0.000000in}}%
\pgfusepath{stroke,fill}%
}%
\begin{pgfscope}%
\pgfsys@transformshift{0.800000in}{3.280996in}%
\pgfsys@useobject{currentmarker}{}%
\end{pgfscope}%
\end{pgfscope}%
\begin{pgfscope}%
\pgfsetbuttcap%
\pgfsetroundjoin%
\definecolor{currentfill}{rgb}{0.000000,0.000000,0.000000}%
\pgfsetfillcolor{currentfill}%
\pgfsetlinewidth{0.602250pt}%
\definecolor{currentstroke}{rgb}{0.000000,0.000000,0.000000}%
\pgfsetstrokecolor{currentstroke}%
\pgfsetdash{}{0pt}%
\pgfsys@defobject{currentmarker}{\pgfqpoint{-0.027778in}{0.000000in}}{\pgfqpoint{0.000000in}{0.000000in}}{%
\pgfpathmoveto{\pgfqpoint{0.000000in}{0.000000in}}%
\pgfpathlineto{\pgfqpoint{-0.027778in}{0.000000in}}%
\pgfusepath{stroke,fill}%
}%
\begin{pgfscope}%
\pgfsys@transformshift{0.800000in}{3.319912in}%
\pgfsys@useobject{currentmarker}{}%
\end{pgfscope}%
\end{pgfscope}%
\begin{pgfscope}%
\pgfsetbuttcap%
\pgfsetroundjoin%
\definecolor{currentfill}{rgb}{0.000000,0.000000,0.000000}%
\pgfsetfillcolor{currentfill}%
\pgfsetlinewidth{0.602250pt}%
\definecolor{currentstroke}{rgb}{0.000000,0.000000,0.000000}%
\pgfsetstrokecolor{currentstroke}%
\pgfsetdash{}{0pt}%
\pgfsys@defobject{currentmarker}{\pgfqpoint{-0.027778in}{0.000000in}}{\pgfqpoint{0.000000in}{0.000000in}}{%
\pgfpathmoveto{\pgfqpoint{0.000000in}{0.000000in}}%
\pgfpathlineto{\pgfqpoint{-0.027778in}{0.000000in}}%
\pgfusepath{stroke,fill}%
}%
\begin{pgfscope}%
\pgfsys@transformshift{0.800000in}{3.354238in}%
\pgfsys@useobject{currentmarker}{}%
\end{pgfscope}%
\end{pgfscope}%
\begin{pgfscope}%
\pgfsetbuttcap%
\pgfsetroundjoin%
\definecolor{currentfill}{rgb}{0.000000,0.000000,0.000000}%
\pgfsetfillcolor{currentfill}%
\pgfsetlinewidth{0.602250pt}%
\definecolor{currentstroke}{rgb}{0.000000,0.000000,0.000000}%
\pgfsetstrokecolor{currentstroke}%
\pgfsetdash{}{0pt}%
\pgfsys@defobject{currentmarker}{\pgfqpoint{-0.027778in}{0.000000in}}{\pgfqpoint{0.000000in}{0.000000in}}{%
\pgfpathmoveto{\pgfqpoint{0.000000in}{0.000000in}}%
\pgfpathlineto{\pgfqpoint{-0.027778in}{0.000000in}}%
\pgfusepath{stroke,fill}%
}%
\begin{pgfscope}%
\pgfsys@transformshift{0.800000in}{3.586952in}%
\pgfsys@useobject{currentmarker}{}%
\end{pgfscope}%
\end{pgfscope}%
\begin{pgfscope}%
\pgfsetbuttcap%
\pgfsetroundjoin%
\definecolor{currentfill}{rgb}{0.000000,0.000000,0.000000}%
\pgfsetfillcolor{currentfill}%
\pgfsetlinewidth{0.602250pt}%
\definecolor{currentstroke}{rgb}{0.000000,0.000000,0.000000}%
\pgfsetstrokecolor{currentstroke}%
\pgfsetdash{}{0pt}%
\pgfsys@defobject{currentmarker}{\pgfqpoint{-0.027778in}{0.000000in}}{\pgfqpoint{0.000000in}{0.000000in}}{%
\pgfpathmoveto{\pgfqpoint{0.000000in}{0.000000in}}%
\pgfpathlineto{\pgfqpoint{-0.027778in}{0.000000in}}%
\pgfusepath{stroke,fill}%
}%
\begin{pgfscope}%
\pgfsys@transformshift{0.800000in}{3.705119in}%
\pgfsys@useobject{currentmarker}{}%
\end{pgfscope}%
\end{pgfscope}%
\begin{pgfscope}%
\pgfsetbuttcap%
\pgfsetroundjoin%
\definecolor{currentfill}{rgb}{0.000000,0.000000,0.000000}%
\pgfsetfillcolor{currentfill}%
\pgfsetlinewidth{0.602250pt}%
\definecolor{currentstroke}{rgb}{0.000000,0.000000,0.000000}%
\pgfsetstrokecolor{currentstroke}%
\pgfsetdash{}{0pt}%
\pgfsys@defobject{currentmarker}{\pgfqpoint{-0.027778in}{0.000000in}}{\pgfqpoint{0.000000in}{0.000000in}}{%
\pgfpathmoveto{\pgfqpoint{0.000000in}{0.000000in}}%
\pgfpathlineto{\pgfqpoint{-0.027778in}{0.000000in}}%
\pgfusepath{stroke,fill}%
}%
\begin{pgfscope}%
\pgfsys@transformshift{0.800000in}{3.788960in}%
\pgfsys@useobject{currentmarker}{}%
\end{pgfscope}%
\end{pgfscope}%
\begin{pgfscope}%
\pgfsetbuttcap%
\pgfsetroundjoin%
\definecolor{currentfill}{rgb}{0.000000,0.000000,0.000000}%
\pgfsetfillcolor{currentfill}%
\pgfsetlinewidth{0.602250pt}%
\definecolor{currentstroke}{rgb}{0.000000,0.000000,0.000000}%
\pgfsetstrokecolor{currentstroke}%
\pgfsetdash{}{0pt}%
\pgfsys@defobject{currentmarker}{\pgfqpoint{-0.027778in}{0.000000in}}{\pgfqpoint{0.000000in}{0.000000in}}{%
\pgfpathmoveto{\pgfqpoint{0.000000in}{0.000000in}}%
\pgfpathlineto{\pgfqpoint{-0.027778in}{0.000000in}}%
\pgfusepath{stroke,fill}%
}%
\begin{pgfscope}%
\pgfsys@transformshift{0.800000in}{3.853992in}%
\pgfsys@useobject{currentmarker}{}%
\end{pgfscope}%
\end{pgfscope}%
\begin{pgfscope}%
\pgfsetbuttcap%
\pgfsetroundjoin%
\definecolor{currentfill}{rgb}{0.000000,0.000000,0.000000}%
\pgfsetfillcolor{currentfill}%
\pgfsetlinewidth{0.602250pt}%
\definecolor{currentstroke}{rgb}{0.000000,0.000000,0.000000}%
\pgfsetstrokecolor{currentstroke}%
\pgfsetdash{}{0pt}%
\pgfsys@defobject{currentmarker}{\pgfqpoint{-0.027778in}{0.000000in}}{\pgfqpoint{0.000000in}{0.000000in}}{%
\pgfpathmoveto{\pgfqpoint{0.000000in}{0.000000in}}%
\pgfpathlineto{\pgfqpoint{-0.027778in}{0.000000in}}%
\pgfusepath{stroke,fill}%
}%
\begin{pgfscope}%
\pgfsys@transformshift{0.800000in}{3.907127in}%
\pgfsys@useobject{currentmarker}{}%
\end{pgfscope}%
\end{pgfscope}%
\begin{pgfscope}%
\pgfsetbuttcap%
\pgfsetroundjoin%
\definecolor{currentfill}{rgb}{0.000000,0.000000,0.000000}%
\pgfsetfillcolor{currentfill}%
\pgfsetlinewidth{0.602250pt}%
\definecolor{currentstroke}{rgb}{0.000000,0.000000,0.000000}%
\pgfsetstrokecolor{currentstroke}%
\pgfsetdash{}{0pt}%
\pgfsys@defobject{currentmarker}{\pgfqpoint{-0.027778in}{0.000000in}}{\pgfqpoint{0.000000in}{0.000000in}}{%
\pgfpathmoveto{\pgfqpoint{0.000000in}{0.000000in}}%
\pgfpathlineto{\pgfqpoint{-0.027778in}{0.000000in}}%
\pgfusepath{stroke,fill}%
}%
\begin{pgfscope}%
\pgfsys@transformshift{0.800000in}{3.952052in}%
\pgfsys@useobject{currentmarker}{}%
\end{pgfscope}%
\end{pgfscope}%
\begin{pgfscope}%
\pgfsetbuttcap%
\pgfsetroundjoin%
\definecolor{currentfill}{rgb}{0.000000,0.000000,0.000000}%
\pgfsetfillcolor{currentfill}%
\pgfsetlinewidth{0.602250pt}%
\definecolor{currentstroke}{rgb}{0.000000,0.000000,0.000000}%
\pgfsetstrokecolor{currentstroke}%
\pgfsetdash{}{0pt}%
\pgfsys@defobject{currentmarker}{\pgfqpoint{-0.027778in}{0.000000in}}{\pgfqpoint{0.000000in}{0.000000in}}{%
\pgfpathmoveto{\pgfqpoint{0.000000in}{0.000000in}}%
\pgfpathlineto{\pgfqpoint{-0.027778in}{0.000000in}}%
\pgfusepath{stroke,fill}%
}%
\begin{pgfscope}%
\pgfsys@transformshift{0.800000in}{3.990968in}%
\pgfsys@useobject{currentmarker}{}%
\end{pgfscope}%
\end{pgfscope}%
\begin{pgfscope}%
\pgfsetbuttcap%
\pgfsetroundjoin%
\definecolor{currentfill}{rgb}{0.000000,0.000000,0.000000}%
\pgfsetfillcolor{currentfill}%
\pgfsetlinewidth{0.602250pt}%
\definecolor{currentstroke}{rgb}{0.000000,0.000000,0.000000}%
\pgfsetstrokecolor{currentstroke}%
\pgfsetdash{}{0pt}%
\pgfsys@defobject{currentmarker}{\pgfqpoint{-0.027778in}{0.000000in}}{\pgfqpoint{0.000000in}{0.000000in}}{%
\pgfpathmoveto{\pgfqpoint{0.000000in}{0.000000in}}%
\pgfpathlineto{\pgfqpoint{-0.027778in}{0.000000in}}%
\pgfusepath{stroke,fill}%
}%
\begin{pgfscope}%
\pgfsys@transformshift{0.800000in}{4.025294in}%
\pgfsys@useobject{currentmarker}{}%
\end{pgfscope}%
\end{pgfscope}%
\begin{pgfscope}%
\definecolor{textcolor}{rgb}{0.000000,0.000000,0.000000}%
\pgfsetstrokecolor{textcolor}%
\pgfsetfillcolor{textcolor}%
\pgftext[x=0.446026in,y=2.376000in,,bottom,rotate=90.000000]{\color{textcolor}\sffamily\fontsize{10.000000}{12.000000}\selectfont \(\displaystyle G_{\varepsilon}\)}%
\end{pgfscope}%
\begin{pgfscope}%
\pgfpathrectangle{\pgfqpoint{0.800000in}{0.528000in}}{\pgfqpoint{4.960000in}{3.696000in}}%
\pgfusepath{clip}%
\pgfsetbuttcap%
\pgfsetroundjoin%
\pgfsetlinewidth{1.505625pt}%
\definecolor{currentstroke}{rgb}{0.000000,0.000000,0.000000}%
\pgfsetstrokecolor{currentstroke}%
\pgfsetdash{{5.550000pt}{2.400000pt}}{0.000000pt}%
\pgfpathmoveto{\pgfqpoint{1.025455in}{0.696000in}}%
\pgfpathlineto{\pgfqpoint{1.039434in}{0.721839in}}%
\pgfpathlineto{\pgfqpoint{1.055218in}{0.746349in}}%
\pgfpathlineto{\pgfqpoint{1.073256in}{0.770271in}}%
\pgfpathlineto{\pgfqpoint{1.093549in}{0.793492in}}%
\pgfpathlineto{\pgfqpoint{1.116096in}{0.815921in}}%
\pgfpathlineto{\pgfqpoint{1.140899in}{0.837469in}}%
\pgfpathlineto{\pgfqpoint{1.167956in}{0.858045in}}%
\pgfpathlineto{\pgfqpoint{1.197268in}{0.877550in}}%
\pgfpathlineto{\pgfqpoint{1.228835in}{0.895884in}}%
\pgfpathlineto{\pgfqpoint{1.262656in}{0.912941in}}%
\pgfpathlineto{\pgfqpoint{1.298733in}{0.928619in}}%
\pgfpathlineto{\pgfqpoint{1.337064in}{0.942823in}}%
\pgfpathlineto{\pgfqpoint{1.377650in}{0.955467in}}%
\pgfpathlineto{\pgfqpoint{1.420490in}{0.966482in}}%
\pgfpathlineto{\pgfqpoint{1.466488in}{0.975986in}}%
\pgfpathlineto{\pgfqpoint{1.515642in}{0.983833in}}%
\pgfpathlineto{\pgfqpoint{1.568403in}{0.989970in}}%
\pgfpathlineto{\pgfqpoint{1.625675in}{0.994350in}}%
\pgfpathlineto{\pgfqpoint{1.688357in}{0.996862in}}%
\pgfpathlineto{\pgfqpoint{1.757804in}{0.997363in}}%
\pgfpathlineto{\pgfqpoint{1.835819in}{0.995649in}}%
\pgfpathlineto{\pgfqpoint{1.925559in}{0.991399in}}%
\pgfpathlineto{\pgfqpoint{2.032886in}{0.984011in}}%
\pgfpathlineto{\pgfqpoint{2.168623in}{0.972329in}}%
\pgfpathlineto{\pgfqpoint{2.366592in}{0.952842in}}%
\pgfpathlineto{\pgfqpoint{3.026789in}{0.886470in}}%
\pgfpathlineto{\pgfqpoint{3.256776in}{0.865673in}}%
\pgfpathlineto{\pgfqpoint{3.258129in}{2.648856in}}%
\pgfpathlineto{\pgfqpoint{5.534545in}{2.648856in}}%
\pgfpathlineto{\pgfqpoint{5.534545in}{2.648856in}}%
\pgfusepath{stroke}%
\end{pgfscope}%
\begin{pgfscope}%
\pgfpathrectangle{\pgfqpoint{0.800000in}{0.528000in}}{\pgfqpoint{4.960000in}{3.696000in}}%
\pgfusepath{clip}%
\pgfsetbuttcap%
\pgfsetroundjoin%
\pgfsetlinewidth{1.505625pt}%
\definecolor{currentstroke}{rgb}{0.000000,0.000000,0.000000}%
\pgfsetstrokecolor{currentstroke}%
\pgfsetdash{{9.600000pt}{2.400000pt}{1.500000pt}{2.400000pt}}{0.000000pt}%
\pgfpathmoveto{\pgfqpoint{1.025455in}{0.767471in}}%
\pgfpathlineto{\pgfqpoint{1.038532in}{0.797294in}}%
\pgfpathlineto{\pgfqpoint{1.053414in}{0.825680in}}%
\pgfpathlineto{\pgfqpoint{1.069648in}{0.851821in}}%
\pgfpathlineto{\pgfqpoint{1.086784in}{0.875288in}}%
\pgfpathlineto{\pgfqpoint{1.105273in}{0.896857in}}%
\pgfpathlineto{\pgfqpoint{1.124664in}{0.916067in}}%
\pgfpathlineto{\pgfqpoint{1.144957in}{0.933044in}}%
\pgfpathlineto{\pgfqpoint{1.166152in}{0.947871in}}%
\pgfpathlineto{\pgfqpoint{1.188249in}{0.960613in}}%
\pgfpathlineto{\pgfqpoint{1.211248in}{0.971328in}}%
\pgfpathlineto{\pgfqpoint{1.235599in}{0.980223in}}%
\pgfpathlineto{\pgfqpoint{1.261304in}{0.987254in}}%
\pgfpathlineto{\pgfqpoint{1.288812in}{0.992484in}}%
\pgfpathlineto{\pgfqpoint{1.319026in}{0.995921in}}%
\pgfpathlineto{\pgfqpoint{1.351945in}{0.997388in}}%
\pgfpathlineto{\pgfqpoint{1.388924in}{0.996747in}}%
\pgfpathlineto{\pgfqpoint{1.430862in}{0.993733in}}%
\pgfpathlineto{\pgfqpoint{1.480016in}{0.987903in}}%
\pgfpathlineto{\pgfqpoint{1.540444in}{0.978402in}}%
\pgfpathlineto{\pgfqpoint{1.621616in}{0.963238in}}%
\pgfpathlineto{\pgfqpoint{1.774940in}{0.931834in}}%
\pgfpathlineto{\pgfqpoint{1.960734in}{0.894560in}}%
\pgfpathlineto{\pgfqpoint{2.102333in}{0.868476in}}%
\pgfpathlineto{\pgfqpoint{2.118117in}{0.865712in}}%
\pgfpathlineto{\pgfqpoint{2.119469in}{3.254880in}}%
\pgfpathlineto{\pgfqpoint{5.534545in}{3.254880in}}%
\pgfpathlineto{\pgfqpoint{5.534545in}{3.254880in}}%
\pgfusepath{stroke}%
\end{pgfscope}%
\begin{pgfscope}%
\pgfpathrectangle{\pgfqpoint{0.800000in}{0.528000in}}{\pgfqpoint{4.960000in}{3.696000in}}%
\pgfusepath{clip}%
\pgfsetbuttcap%
\pgfsetroundjoin%
\pgfsetlinewidth{1.505625pt}%
\definecolor{currentstroke}{rgb}{0.000000,0.000000,0.000000}%
\pgfsetstrokecolor{currentstroke}%
\pgfsetdash{{1.500000pt}{2.475000pt}}{0.000000pt}%
\pgfpathmoveto{\pgfqpoint{1.025455in}{0.883856in}}%
\pgfpathlineto{\pgfqpoint{1.036277in}{0.912517in}}%
\pgfpathlineto{\pgfqpoint{1.047551in}{0.936099in}}%
\pgfpathlineto{\pgfqpoint{1.058825in}{0.954564in}}%
\pgfpathlineto{\pgfqpoint{1.070099in}{0.968824in}}%
\pgfpathlineto{\pgfqpoint{1.081373in}{0.979569in}}%
\pgfpathlineto{\pgfqpoint{1.092647in}{0.987358in}}%
\pgfpathlineto{\pgfqpoint{1.104372in}{0.992819in}}%
\pgfpathlineto{\pgfqpoint{1.116998in}{0.996190in}}%
\pgfpathlineto{\pgfqpoint{1.130527in}{0.997429in}}%
\pgfpathlineto{\pgfqpoint{1.145408in}{0.996517in}}%
\pgfpathlineto{\pgfqpoint{1.162545in}{0.993176in}}%
\pgfpathlineto{\pgfqpoint{1.182838in}{0.986884in}}%
\pgfpathlineto{\pgfqpoint{1.208542in}{0.976466in}}%
\pgfpathlineto{\pgfqpoint{1.244618in}{0.959242in}}%
\pgfpathlineto{\pgfqpoint{1.420039in}{0.872394in}}%
\pgfpathlineto{\pgfqpoint{1.435372in}{0.865634in}}%
\pgfpathlineto{\pgfqpoint{1.436725in}{4.056000in}}%
\pgfpathlineto{\pgfqpoint{5.534545in}{4.056000in}}%
\pgfpathlineto{\pgfqpoint{5.534545in}{4.056000in}}%
\pgfusepath{stroke}%
\end{pgfscope}%
\begin{pgfscope}%
\pgfsetrectcap%
\pgfsetmiterjoin%
\pgfsetlinewidth{0.803000pt}%
\definecolor{currentstroke}{rgb}{0.000000,0.000000,0.000000}%
\pgfsetstrokecolor{currentstroke}%
\pgfsetdash{}{0pt}%
\pgfpathmoveto{\pgfqpoint{0.800000in}{0.528000in}}%
\pgfpathlineto{\pgfqpoint{0.800000in}{4.224000in}}%
\pgfusepath{stroke}%
\end{pgfscope}%
\begin{pgfscope}%
\pgfsetrectcap%
\pgfsetmiterjoin%
\pgfsetlinewidth{0.803000pt}%
\definecolor{currentstroke}{rgb}{0.000000,0.000000,0.000000}%
\pgfsetstrokecolor{currentstroke}%
\pgfsetdash{}{0pt}%
\pgfpathmoveto{\pgfqpoint{5.760000in}{0.528000in}}%
\pgfpathlineto{\pgfqpoint{5.760000in}{4.224000in}}%
\pgfusepath{stroke}%
\end{pgfscope}%
\begin{pgfscope}%
\pgfsetrectcap%
\pgfsetmiterjoin%
\pgfsetlinewidth{0.803000pt}%
\definecolor{currentstroke}{rgb}{0.000000,0.000000,0.000000}%
\pgfsetstrokecolor{currentstroke}%
\pgfsetdash{}{0pt}%
\pgfpathmoveto{\pgfqpoint{0.800000in}{0.528000in}}%
\pgfpathlineto{\pgfqpoint{5.760000in}{0.528000in}}%
\pgfusepath{stroke}%
\end{pgfscope}%
\begin{pgfscope}%
\pgfsetrectcap%
\pgfsetmiterjoin%
\pgfsetlinewidth{0.803000pt}%
\definecolor{currentstroke}{rgb}{0.000000,0.000000,0.000000}%
\pgfsetstrokecolor{currentstroke}%
\pgfsetdash{}{0pt}%
\pgfpathmoveto{\pgfqpoint{0.800000in}{4.224000in}}%
\pgfpathlineto{\pgfqpoint{5.760000in}{4.224000in}}%
\pgfusepath{stroke}%
\end{pgfscope}%
\begin{pgfscope}%
\pgfsetbuttcap%
\pgfsetmiterjoin%
\definecolor{currentfill}{rgb}{1.000000,1.000000,1.000000}%
\pgfsetfillcolor{currentfill}%
\pgfsetfillopacity{0.800000}%
\pgfsetlinewidth{1.003750pt}%
\definecolor{currentstroke}{rgb}{0.800000,0.800000,0.800000}%
\pgfsetstrokecolor{currentstroke}%
\pgfsetstrokeopacity{0.800000}%
\pgfsetdash{}{0pt}%
\pgfpathmoveto{\pgfqpoint{4.748137in}{0.597444in}}%
\pgfpathlineto{\pgfqpoint{5.662778in}{0.597444in}}%
\pgfpathquadraticcurveto{\pgfqpoint{5.690556in}{0.597444in}}{\pgfqpoint{5.690556in}{0.625222in}}%
\pgfpathlineto{\pgfqpoint{5.690556in}{1.222905in}}%
\pgfpathquadraticcurveto{\pgfqpoint{5.690556in}{1.250683in}}{\pgfqpoint{5.662778in}{1.250683in}}%
\pgfpathlineto{\pgfqpoint{4.748137in}{1.250683in}}%
\pgfpathquadraticcurveto{\pgfqpoint{4.720360in}{1.250683in}}{\pgfqpoint{4.720360in}{1.222905in}}%
\pgfpathlineto{\pgfqpoint{4.720360in}{0.625222in}}%
\pgfpathquadraticcurveto{\pgfqpoint{4.720360in}{0.597444in}}{\pgfqpoint{4.748137in}{0.597444in}}%
\pgfpathclose%
\pgfusepath{stroke,fill}%
\end{pgfscope}%
\begin{pgfscope}%
\pgfsetbuttcap%
\pgfsetroundjoin%
\pgfsetlinewidth{1.505625pt}%
\definecolor{currentstroke}{rgb}{0.000000,0.000000,0.000000}%
\pgfsetstrokecolor{currentstroke}%
\pgfsetdash{{5.550000pt}{2.400000pt}}{0.000000pt}%
\pgfpathmoveto{\pgfqpoint{4.775915in}{1.138215in}}%
\pgfpathlineto{\pgfqpoint{5.053693in}{1.138215in}}%
\pgfusepath{stroke}%
\end{pgfscope}%
\begin{pgfscope}%
\definecolor{textcolor}{rgb}{0.000000,0.000000,0.000000}%
\pgfsetstrokecolor{textcolor}%
\pgfsetfillcolor{textcolor}%
\pgftext[x=5.164804in,y=1.089604in,left,base]{\color{textcolor}\sffamily\fontsize{10.000000}{12.000000}\selectfont \(\displaystyle k = 20\)}%
\end{pgfscope}%
\begin{pgfscope}%
\pgfsetbuttcap%
\pgfsetroundjoin%
\pgfsetlinewidth{1.505625pt}%
\definecolor{currentstroke}{rgb}{0.000000,0.000000,0.000000}%
\pgfsetstrokecolor{currentstroke}%
\pgfsetdash{{9.600000pt}{2.400000pt}{1.500000pt}{2.400000pt}}{0.000000pt}%
\pgfpathmoveto{\pgfqpoint{4.775915in}{0.934358in}}%
\pgfpathlineto{\pgfqpoint{5.053693in}{0.934358in}}%
\pgfusepath{stroke}%
\end{pgfscope}%
\begin{pgfscope}%
\definecolor{textcolor}{rgb}{0.000000,0.000000,0.000000}%
\pgfsetstrokecolor{textcolor}%
\pgfsetfillcolor{textcolor}%
\pgftext[x=5.164804in,y=0.885747in,left,base]{\color{textcolor}\sffamily\fontsize{10.000000}{12.000000}\selectfont \(\displaystyle k = 40\)}%
\end{pgfscope}%
\begin{pgfscope}%
\pgfsetbuttcap%
\pgfsetroundjoin%
\pgfsetlinewidth{1.505625pt}%
\definecolor{currentstroke}{rgb}{0.000000,0.000000,0.000000}%
\pgfsetstrokecolor{currentstroke}%
\pgfsetdash{{1.500000pt}{2.475000pt}}{0.000000pt}%
\pgfpathmoveto{\pgfqpoint{4.775915in}{0.730501in}}%
\pgfpathlineto{\pgfqpoint{5.053693in}{0.730501in}}%
\pgfusepath{stroke}%
\end{pgfscope}%
\begin{pgfscope}%
\definecolor{textcolor}{rgb}{0.000000,0.000000,0.000000}%
\pgfsetstrokecolor{textcolor}%
\pgfsetfillcolor{textcolor}%
\pgftext[x=5.164804in,y=0.681890in,left,base]{\color{textcolor}\sffamily\fontsize{10.000000}{12.000000}\selectfont \(\displaystyle k = 100\)}%
\end{pgfscope}%
\end{pgfpicture}%
\makeatother%
\endgroup%

  \caption{Plot of $\Gfnname$ for $\NLiDRRR{\no-\nt} \in \mleft(0.01,1.0\mright)$, for $k=20,40,100$ $\Ct = 0.1,$ $N=\ceil{k^3},$ and $\eps = 10^{-5}$.\label{fig:G}}
    \end{figure}


\Cref{lem:probgmres1} gives us the relationship between the (bound on the) number of GMRES iterations and $\NLiDRRR{\nso-\nst(\omega)}.$ We can therefore infer probabalistic properties of the number of GMRES iterations from the probability distribution of $\NLiDRRR{\nso-\nst(\omega)}.$ (For probabalistic notation, we refer the reader to \cref{chap:stochastic}.)

\bth[Probabilistic GMRES convergence]\label{thm:probgmres}
Let $\nso \in \LiDRRR$ be fixed, and let $\nst:\Omega\rightarrow\LiDRRR$ be a random field. Let $\eps$ and $\dofs$ be as in \cref{lem:probgmres1}, and let $\Aso = \Ast = I.$ Fix $ R \in \NN.$ Then
\beq\label{eq:GMRESprob}
\PP\mleft(\Gfn{\NLiDRRR{\nso-\nst}} \leq R\mright)\leq\PP\mleft(\GMRES{\eps}{\nso}{\nst} \leq R\mright) .
\eeq
\enth

\bpf[Proof of \cref{thm:probgmres}]
By \cref{lem:probgmres1} we have the implication: if $\Gfn{\nso-\nst(\omega)} \leq R$, then $\GMRES{\eps}{\nso}{\nst(\omega)} \leq R.$ Therefore we have the set inclusion
\beqs
\set{\omega \in \Omega \st \Gfn{\nso-\nst(\omega)} \leq R} \subseteq \set{\omega \in \Omega \st \GMRES{\eps}{\nso}{\nst(\omega)} \leq R}.
\eeqs
The result immediately follows.
\epf

\bre[$\GMRES{\eps}{\no}{\nt}$ is a random variable]
All of the operations used in constructing the vectors $\bxm$ in the GMRES algorithm are measurable functions of $\bxmmo$ and $\AmatoI\Amatt$ (see, e.g., \cite[Algorithms 11.4.2 and 5.1.3]{GoVa:13}), therefore $\mleft(\bxm\mright)_{m=1}^N$ is a sequence of random variables, a stochastic process (see, e.g., \cite[Definition 2.1.4]{Ok:13}). The stopping criterion $\Nt{\bxm} < \eps$ is an \emph{exit time} from the set $\mleft[\eps,\infty\mright)$, and therefore, because we assume $\OFP$ is a complete probability space, it follows from, e.g.,  \cite[Example 7.2.2]{Ok:13} that $\GMRES{\eps}{\no}{\nt}$ is a stopping time. See \cite[Definition 7.2.1]{Ok:13} for the probabalistic definition of a stopping time. Informally, a random time is a stopping time, if one can determine whether it has occured based only on past knowledge. For example, the \emph{first} time a stochastic process exits a set is a stopping time, the \emph{last} time a stochastic process exits a set is not. Because $\GMRES{\eps}{\no}{\nt}$ is a stopping time, it is measurable with respect to the associated filtration (see, e.g., \cite[Definition 3.2.2]{Ok:13}), and so is measurable with respect to $\cF$. I.e., $\GMRES{\eps}{\no}{\nt}$ is a random variable.
\ere

\bre[The expression \cref{eq:GMRESprob} is computable]\label{rem:computable}
Because the function $\Gfnname$ is not invertible (as is clear from \cref{fig:G} above), one cannot write the left-hand side of \cref{eq:GMRESprob} as
\beqs
\PP\mleft(\NLiDRRR{\nso-\nst} \leq \GfnnameI\mleft(\mleft[0,R\mright]\mright)\mright).
\eeqs
However, one can still compute the set
\beqs
\GfnnameI\mleft(\mleft[0,R\mright]\mright) = \set{\alpha \st \Gfn{\alpha} \in \mleft[0,R\mright]}
\eeqs
(where $\GfnnameI$ here denotes the pullback), and therefore one can compute the probabilities in \cref{eq:GMRESprob}. The main effort in computing $\GfnI{\mleft[0,R\mright]}$ is finding if there are any values of $\alpha < 1$ such that $\Gfn{alpha} = R$. However, these values can be computed numerically using standard root-finding software.
\ere

\bre[\Cref{thm:probgmres} is pessimistic]\label{rem:pessimistic}
Observe that we expect the bound in \cref{thm:probgmres} to be pessimistic, i.e., we expect that \cref{eq:GMRESprob} is not sharp. We especially expect \cref{eq:GMRESprob} is not sharp for large values of $R:$

One can show via elementary calculus that (assuming that for $\alpha < 1$ the expression involving $\alpha$ in \cref{eq:gdef} is always $\leq N$) for $\alpha < 1$ $\Gfnname$ achieves its maximum when $\alpha = 1/3$. Also observe that $\Gfnname$ over the range $\alpha < 1$ only depends on $k$ through the dependence of $\alpha$ on $k.$ Therefore, the maximum of $\Gfnname$ over $\alpha \in (0,1)$ is independent of $k.$ Let $\Gfnmaxlo$ denote the value of this maximum. Then, for any $R \in \mleft(\Gfnmaxlo,N\mright)$, the estimate $\PP\mleft(\Gfn{\nso-\nst} \leq R\mright)$ is equal to $\PP\mleft(\NLiDRRR{\nso-\nst} \leq 1/\mleft(\Ct k\mright)\mright),$ i.e., the lower-bound estimate is \emph{independent} of $R$. This is almost certainly not sharp - we would expect $\PP\mleft(\GMRES{\eps}{\nso}{\nst} \leq R\mright)$ to increase with $R$. However, because the only rigorous result we have available if $\alpha \geq 1$ is \cref{cor:gmresguaranteed}, we cannot prove a better bound.
\ere

\subsubsection{Qualitative predictions from \cref{thm:probgmres}}\label{sec:qualgmres}

Notwithstanding the comments in \cref{rem:pessimistic}, we wil show in this \lcnamecref{sec:qualgmres} that \cref{thm:probgmres} gives \emph{qualitatively} correct predictions. To illustrate this correctness, we take $\Ao=\At=I,$ $\no =1$, and $\nt = \no + \eta,$ where $\eta$ is an $\Exp{\sigma}$ random variable, where $\sigma$ is the shape parameter, that we will vary. By construction, $\NLiDRRR{\no-\nt} = \eta$. Recall that the standard deviation of $\eta$ is $\sigma$, and therefore one might expect (based on \cref{cor:1,cor:1a}) that if $\sigma \sim 1/k,$ the number of GMRES iterations may be controllable (in some sense) as $k \rightarrow \infty.$

In our numerical experiments we use the computational setup described in \cref{app:compsetup}, with $f=1$ and $\gI=0.$ we consider three cases:
\ben
\item\label{it:sigma1} $\displaystyle \sigma  = 1,$
\item\label{it:sigma2} $\displaystyle \sigma  = \frac{1}k,$ and
  \item\label{it:sigma3} $\displaystyle \sigma  = \frac{1}{k^2}.$
    \een

    For each of these cases we calculate
    \beq\label{eq:gmresprob}
    \PP\mleft(\GMRES{\eps}{\no}{\nt} \leq 12\mright),
    \eeq
     and compare these experimental results to the lower bounds given by \cref{thm:probgmres}.

    Qualitatively, from \cref{fig:prob-theory-plot-0.0,fig:prob-theory-plot-1.0,fig:prob-theory-plot-2.0} we expect that in \cref{it:sigma1} the probability \cref{eq:gmresprob} \emph{decreases} as $k$ increases, in \cref{it:sigma2} the probability \cref{eq:gmresprob} \emph{is constant} as $k$ increases, and in \cref{it:sigma3} the probability \cref{eq:gmresprob} \emph{increases} as $k$ increases. This qualitative behaviour is what we see experimentally in \cref{fig:prob-plot-0.0,fig:prob-plot-1.0,fig:prob-plot-2.0}, although the values for the probabilities are higher, as is expected, because \cref{thm:probgmres} is a lower bound.

\begin{figure}[h]
    \centering
%% Creator: Matplotlib, PGF backend
%%
%% To include the figure in your LaTeX document, write
%%   \input{<filename>.pgf}
%%
%% Make sure the required packages are loaded in your preamble
%%   \usepackage{pgf}
%%
%% Figures using additional raster images can only be included by \input if
%% they are in the same directory as the main LaTeX file. For loading figures
%% from other directories you can use the `import` package
%%   \usepackage{import}
%% and then include the figures with
%%   \import{<path to file>}{<filename>.pgf}
%%
%% Matplotlib used the following preamble
%%   \usepackage{fontspec}
%%   \setmainfont{DejaVuSerif.ttf}[Path=/home/owen/progs/firedrake-complex/firedrake/lib/python3.5/site-packages/matplotlib/mpl-data/fonts/ttf/]
%%   \setsansfont{DejaVuSans.ttf}[Path=/home/owen/progs/firedrake-complex/firedrake/lib/python3.5/site-packages/matplotlib/mpl-data/fonts/ttf/]
%%   \setmonofont{DejaVuSansMono.ttf}[Path=/home/owen/progs/firedrake-complex/firedrake/lib/python3.5/site-packages/matplotlib/mpl-data/fonts/ttf/]
%%
\begingroup%
\makeatletter%
\begin{pgfpicture}%
\pgfpathrectangle{\pgfpointorigin}{\pgfqpoint{6.000000in}{2.500000in}}%
\pgfusepath{use as bounding box, clip}%
\begin{pgfscope}%
\pgfsetbuttcap%
\pgfsetmiterjoin%
\definecolor{currentfill}{rgb}{1.000000,1.000000,1.000000}%
\pgfsetfillcolor{currentfill}%
\pgfsetlinewidth{0.000000pt}%
\definecolor{currentstroke}{rgb}{1.000000,1.000000,1.000000}%
\pgfsetstrokecolor{currentstroke}%
\pgfsetdash{}{0pt}%
\pgfpathmoveto{\pgfqpoint{0.000000in}{0.000000in}}%
\pgfpathlineto{\pgfqpoint{6.000000in}{0.000000in}}%
\pgfpathlineto{\pgfqpoint{6.000000in}{2.500000in}}%
\pgfpathlineto{\pgfqpoint{0.000000in}{2.500000in}}%
\pgfpathclose%
\pgfusepath{fill}%
\end{pgfscope}%
\begin{pgfscope}%
\pgfsetbuttcap%
\pgfsetmiterjoin%
\definecolor{currentfill}{rgb}{1.000000,1.000000,1.000000}%
\pgfsetfillcolor{currentfill}%
\pgfsetlinewidth{0.000000pt}%
\definecolor{currentstroke}{rgb}{0.000000,0.000000,0.000000}%
\pgfsetstrokecolor{currentstroke}%
\pgfsetstrokeopacity{0.000000}%
\pgfsetdash{}{0pt}%
\pgfpathmoveto{\pgfqpoint{0.750000in}{0.275000in}}%
\pgfpathlineto{\pgfqpoint{5.400000in}{0.275000in}}%
\pgfpathlineto{\pgfqpoint{5.400000in}{2.200000in}}%
\pgfpathlineto{\pgfqpoint{0.750000in}{2.200000in}}%
\pgfpathclose%
\pgfusepath{fill}%
\end{pgfscope}%
\begin{pgfscope}%
\pgfsetbuttcap%
\pgfsetroundjoin%
\definecolor{currentfill}{rgb}{0.000000,0.000000,0.000000}%
\pgfsetfillcolor{currentfill}%
\pgfsetlinewidth{0.803000pt}%
\definecolor{currentstroke}{rgb}{0.000000,0.000000,0.000000}%
\pgfsetstrokecolor{currentstroke}%
\pgfsetdash{}{0pt}%
\pgfsys@defobject{currentmarker}{\pgfqpoint{0.000000in}{-0.048611in}}{\pgfqpoint{0.000000in}{0.000000in}}{%
\pgfpathmoveto{\pgfqpoint{0.000000in}{0.000000in}}%
\pgfpathlineto{\pgfqpoint{0.000000in}{-0.048611in}}%
\pgfusepath{stroke,fill}%
}%
\begin{pgfscope}%
\pgfsys@transformshift{0.961364in}{0.275000in}%
\pgfsys@useobject{currentmarker}{}%
\end{pgfscope}%
\end{pgfscope}%
\begin{pgfscope}%
\definecolor{textcolor}{rgb}{0.000000,0.000000,0.000000}%
\pgfsetstrokecolor{textcolor}%
\pgfsetfillcolor{textcolor}%
\pgftext[x=0.961364in,y=0.177778in,,top]{\color{textcolor}\sffamily\fontsize{10.000000}{12.000000}\selectfont \(\displaystyle 10\)}%
\end{pgfscope}%
\begin{pgfscope}%
\pgfsetbuttcap%
\pgfsetroundjoin%
\definecolor{currentfill}{rgb}{0.000000,0.000000,0.000000}%
\pgfsetfillcolor{currentfill}%
\pgfsetlinewidth{0.803000pt}%
\definecolor{currentstroke}{rgb}{0.000000,0.000000,0.000000}%
\pgfsetstrokecolor{currentstroke}%
\pgfsetdash{}{0pt}%
\pgfsys@defobject{currentmarker}{\pgfqpoint{0.000000in}{-0.048611in}}{\pgfqpoint{0.000000in}{0.000000in}}{%
\pgfpathmoveto{\pgfqpoint{0.000000in}{0.000000in}}%
\pgfpathlineto{\pgfqpoint{0.000000in}{-0.048611in}}%
\pgfusepath{stroke,fill}%
}%
\begin{pgfscope}%
\pgfsys@transformshift{1.665909in}{0.275000in}%
\pgfsys@useobject{currentmarker}{}%
\end{pgfscope}%
\end{pgfscope}%
\begin{pgfscope}%
\definecolor{textcolor}{rgb}{0.000000,0.000000,0.000000}%
\pgfsetstrokecolor{textcolor}%
\pgfsetfillcolor{textcolor}%
\pgftext[x=1.665909in,y=0.177778in,,top]{\color{textcolor}\sffamily\fontsize{10.000000}{12.000000}\selectfont \(\displaystyle 15\)}%
\end{pgfscope}%
\begin{pgfscope}%
\pgfsetbuttcap%
\pgfsetroundjoin%
\definecolor{currentfill}{rgb}{0.000000,0.000000,0.000000}%
\pgfsetfillcolor{currentfill}%
\pgfsetlinewidth{0.803000pt}%
\definecolor{currentstroke}{rgb}{0.000000,0.000000,0.000000}%
\pgfsetstrokecolor{currentstroke}%
\pgfsetdash{}{0pt}%
\pgfsys@defobject{currentmarker}{\pgfqpoint{0.000000in}{-0.048611in}}{\pgfqpoint{0.000000in}{0.000000in}}{%
\pgfpathmoveto{\pgfqpoint{0.000000in}{0.000000in}}%
\pgfpathlineto{\pgfqpoint{0.000000in}{-0.048611in}}%
\pgfusepath{stroke,fill}%
}%
\begin{pgfscope}%
\pgfsys@transformshift{2.370455in}{0.275000in}%
\pgfsys@useobject{currentmarker}{}%
\end{pgfscope}%
\end{pgfscope}%
\begin{pgfscope}%
\definecolor{textcolor}{rgb}{0.000000,0.000000,0.000000}%
\pgfsetstrokecolor{textcolor}%
\pgfsetfillcolor{textcolor}%
\pgftext[x=2.370455in,y=0.177778in,,top]{\color{textcolor}\sffamily\fontsize{10.000000}{12.000000}\selectfont \(\displaystyle 20\)}%
\end{pgfscope}%
\begin{pgfscope}%
\pgfsetbuttcap%
\pgfsetroundjoin%
\definecolor{currentfill}{rgb}{0.000000,0.000000,0.000000}%
\pgfsetfillcolor{currentfill}%
\pgfsetlinewidth{0.803000pt}%
\definecolor{currentstroke}{rgb}{0.000000,0.000000,0.000000}%
\pgfsetstrokecolor{currentstroke}%
\pgfsetdash{}{0pt}%
\pgfsys@defobject{currentmarker}{\pgfqpoint{0.000000in}{-0.048611in}}{\pgfqpoint{0.000000in}{0.000000in}}{%
\pgfpathmoveto{\pgfqpoint{0.000000in}{0.000000in}}%
\pgfpathlineto{\pgfqpoint{0.000000in}{-0.048611in}}%
\pgfusepath{stroke,fill}%
}%
\begin{pgfscope}%
\pgfsys@transformshift{3.075000in}{0.275000in}%
\pgfsys@useobject{currentmarker}{}%
\end{pgfscope}%
\end{pgfscope}%
\begin{pgfscope}%
\definecolor{textcolor}{rgb}{0.000000,0.000000,0.000000}%
\pgfsetstrokecolor{textcolor}%
\pgfsetfillcolor{textcolor}%
\pgftext[x=3.075000in,y=0.177778in,,top]{\color{textcolor}\sffamily\fontsize{10.000000}{12.000000}\selectfont \(\displaystyle 25\)}%
\end{pgfscope}%
\begin{pgfscope}%
\pgfsetbuttcap%
\pgfsetroundjoin%
\definecolor{currentfill}{rgb}{0.000000,0.000000,0.000000}%
\pgfsetfillcolor{currentfill}%
\pgfsetlinewidth{0.803000pt}%
\definecolor{currentstroke}{rgb}{0.000000,0.000000,0.000000}%
\pgfsetstrokecolor{currentstroke}%
\pgfsetdash{}{0pt}%
\pgfsys@defobject{currentmarker}{\pgfqpoint{0.000000in}{-0.048611in}}{\pgfqpoint{0.000000in}{0.000000in}}{%
\pgfpathmoveto{\pgfqpoint{0.000000in}{0.000000in}}%
\pgfpathlineto{\pgfqpoint{0.000000in}{-0.048611in}}%
\pgfusepath{stroke,fill}%
}%
\begin{pgfscope}%
\pgfsys@transformshift{3.779545in}{0.275000in}%
\pgfsys@useobject{currentmarker}{}%
\end{pgfscope}%
\end{pgfscope}%
\begin{pgfscope}%
\definecolor{textcolor}{rgb}{0.000000,0.000000,0.000000}%
\pgfsetstrokecolor{textcolor}%
\pgfsetfillcolor{textcolor}%
\pgftext[x=3.779545in,y=0.177778in,,top]{\color{textcolor}\sffamily\fontsize{10.000000}{12.000000}\selectfont \(\displaystyle 30\)}%
\end{pgfscope}%
\begin{pgfscope}%
\pgfsetbuttcap%
\pgfsetroundjoin%
\definecolor{currentfill}{rgb}{0.000000,0.000000,0.000000}%
\pgfsetfillcolor{currentfill}%
\pgfsetlinewidth{0.803000pt}%
\definecolor{currentstroke}{rgb}{0.000000,0.000000,0.000000}%
\pgfsetstrokecolor{currentstroke}%
\pgfsetdash{}{0pt}%
\pgfsys@defobject{currentmarker}{\pgfqpoint{0.000000in}{-0.048611in}}{\pgfqpoint{0.000000in}{0.000000in}}{%
\pgfpathmoveto{\pgfqpoint{0.000000in}{0.000000in}}%
\pgfpathlineto{\pgfqpoint{0.000000in}{-0.048611in}}%
\pgfusepath{stroke,fill}%
}%
\begin{pgfscope}%
\pgfsys@transformshift{4.484091in}{0.275000in}%
\pgfsys@useobject{currentmarker}{}%
\end{pgfscope}%
\end{pgfscope}%
\begin{pgfscope}%
\definecolor{textcolor}{rgb}{0.000000,0.000000,0.000000}%
\pgfsetstrokecolor{textcolor}%
\pgfsetfillcolor{textcolor}%
\pgftext[x=4.484091in,y=0.177778in,,top]{\color{textcolor}\sffamily\fontsize{10.000000}{12.000000}\selectfont \(\displaystyle 35\)}%
\end{pgfscope}%
\begin{pgfscope}%
\pgfsetbuttcap%
\pgfsetroundjoin%
\definecolor{currentfill}{rgb}{0.000000,0.000000,0.000000}%
\pgfsetfillcolor{currentfill}%
\pgfsetlinewidth{0.803000pt}%
\definecolor{currentstroke}{rgb}{0.000000,0.000000,0.000000}%
\pgfsetstrokecolor{currentstroke}%
\pgfsetdash{}{0pt}%
\pgfsys@defobject{currentmarker}{\pgfqpoint{0.000000in}{-0.048611in}}{\pgfqpoint{0.000000in}{0.000000in}}{%
\pgfpathmoveto{\pgfqpoint{0.000000in}{0.000000in}}%
\pgfpathlineto{\pgfqpoint{0.000000in}{-0.048611in}}%
\pgfusepath{stroke,fill}%
}%
\begin{pgfscope}%
\pgfsys@transformshift{5.188636in}{0.275000in}%
\pgfsys@useobject{currentmarker}{}%
\end{pgfscope}%
\end{pgfscope}%
\begin{pgfscope}%
\definecolor{textcolor}{rgb}{0.000000,0.000000,0.000000}%
\pgfsetstrokecolor{textcolor}%
\pgfsetfillcolor{textcolor}%
\pgftext[x=5.188636in,y=0.177778in,,top]{\color{textcolor}\sffamily\fontsize{10.000000}{12.000000}\selectfont \(\displaystyle 40\)}%
\end{pgfscope}%
\begin{pgfscope}%
\definecolor{textcolor}{rgb}{0.000000,0.000000,0.000000}%
\pgfsetstrokecolor{textcolor}%
\pgfsetfillcolor{textcolor}%
\pgftext[x=3.075000in,y=-0.012191in,,top]{\color{textcolor}\sffamily\fontsize{10.000000}{12.000000}\selectfont \(\displaystyle k\)}%
\end{pgfscope}%
\begin{pgfscope}%
\pgfsetbuttcap%
\pgfsetroundjoin%
\definecolor{currentfill}{rgb}{0.000000,0.000000,0.000000}%
\pgfsetfillcolor{currentfill}%
\pgfsetlinewidth{0.803000pt}%
\definecolor{currentstroke}{rgb}{0.000000,0.000000,0.000000}%
\pgfsetstrokecolor{currentstroke}%
\pgfsetdash{}{0pt}%
\pgfsys@defobject{currentmarker}{\pgfqpoint{-0.048611in}{0.000000in}}{\pgfqpoint{0.000000in}{0.000000in}}{%
\pgfpathmoveto{\pgfqpoint{0.000000in}{0.000000in}}%
\pgfpathlineto{\pgfqpoint{-0.048611in}{0.000000in}}%
\pgfusepath{stroke,fill}%
}%
\begin{pgfscope}%
\pgfsys@transformshift{0.750000in}{0.442128in}%
\pgfsys@useobject{currentmarker}{}%
\end{pgfscope}%
\end{pgfscope}%
\begin{pgfscope}%
\definecolor{textcolor}{rgb}{0.000000,0.000000,0.000000}%
\pgfsetstrokecolor{textcolor}%
\pgfsetfillcolor{textcolor}%
\pgftext[x=0.405863in,y=0.389366in,left,base]{\color{textcolor}\sffamily\fontsize{10.000000}{12.000000}\selectfont \(\displaystyle 0.01\)}%
\end{pgfscope}%
\begin{pgfscope}%
\pgfsetbuttcap%
\pgfsetroundjoin%
\definecolor{currentfill}{rgb}{0.000000,0.000000,0.000000}%
\pgfsetfillcolor{currentfill}%
\pgfsetlinewidth{0.803000pt}%
\definecolor{currentstroke}{rgb}{0.000000,0.000000,0.000000}%
\pgfsetstrokecolor{currentstroke}%
\pgfsetdash{}{0pt}%
\pgfsys@defobject{currentmarker}{\pgfqpoint{-0.048611in}{0.000000in}}{\pgfqpoint{0.000000in}{0.000000in}}{%
\pgfpathmoveto{\pgfqpoint{0.000000in}{0.000000in}}%
\pgfpathlineto{\pgfqpoint{-0.048611in}{0.000000in}}%
\pgfusepath{stroke,fill}%
}%
\begin{pgfscope}%
\pgfsys@transformshift{0.750000in}{1.115498in}%
\pgfsys@useobject{currentmarker}{}%
\end{pgfscope}%
\end{pgfscope}%
\begin{pgfscope}%
\definecolor{textcolor}{rgb}{0.000000,0.000000,0.000000}%
\pgfsetstrokecolor{textcolor}%
\pgfsetfillcolor{textcolor}%
\pgftext[x=0.405863in,y=1.062737in,left,base]{\color{textcolor}\sffamily\fontsize{10.000000}{12.000000}\selectfont \(\displaystyle 0.02\)}%
\end{pgfscope}%
\begin{pgfscope}%
\pgfsetbuttcap%
\pgfsetroundjoin%
\definecolor{currentfill}{rgb}{0.000000,0.000000,0.000000}%
\pgfsetfillcolor{currentfill}%
\pgfsetlinewidth{0.803000pt}%
\definecolor{currentstroke}{rgb}{0.000000,0.000000,0.000000}%
\pgfsetstrokecolor{currentstroke}%
\pgfsetdash{}{0pt}%
\pgfsys@defobject{currentmarker}{\pgfqpoint{-0.048611in}{0.000000in}}{\pgfqpoint{0.000000in}{0.000000in}}{%
\pgfpathmoveto{\pgfqpoint{0.000000in}{0.000000in}}%
\pgfpathlineto{\pgfqpoint{-0.048611in}{0.000000in}}%
\pgfusepath{stroke,fill}%
}%
\begin{pgfscope}%
\pgfsys@transformshift{0.750000in}{1.788869in}%
\pgfsys@useobject{currentmarker}{}%
\end{pgfscope}%
\end{pgfscope}%
\begin{pgfscope}%
\definecolor{textcolor}{rgb}{0.000000,0.000000,0.000000}%
\pgfsetstrokecolor{textcolor}%
\pgfsetfillcolor{textcolor}%
\pgftext[x=0.405863in,y=1.736107in,left,base]{\color{textcolor}\sffamily\fontsize{10.000000}{12.000000}\selectfont \(\displaystyle 0.03\)}%
\end{pgfscope}%
\begin{pgfscope}%
\definecolor{textcolor}{rgb}{0.000000,0.000000,0.000000}%
\pgfsetstrokecolor{textcolor}%
\pgfsetfillcolor{textcolor}%
\pgftext[x=0.165901in,y=0.325194in,left,base,rotate=90.000000]{\color{textcolor}\sffamily\fontsize{10.000000}{12.000000}\selectfont Probability that number of}%
\end{pgfscope}%
\begin{pgfscope}%
\definecolor{textcolor}{rgb}{0.000000,0.000000,0.000000}%
\pgfsetstrokecolor{textcolor}%
\pgfsetfillcolor{textcolor}%
\pgftext[x=0.321418in,y=0.160467in,left,base,rotate=90.000000]{\color{textcolor}\sffamily\fontsize{10.000000}{12.000000}\selectfont GMRES iterations is at most 12}%
\end{pgfscope}%
\begin{pgfscope}%
\pgfpathrectangle{\pgfqpoint{0.750000in}{0.275000in}}{\pgfqpoint{4.650000in}{1.925000in}}%
\pgfusepath{clip}%
\pgfsetbuttcap%
\pgfsetroundjoin%
\definecolor{currentfill}{rgb}{0.000000,0.000000,0.000000}%
\pgfsetfillcolor{currentfill}%
\pgfsetlinewidth{1.003750pt}%
\definecolor{currentstroke}{rgb}{0.000000,0.000000,0.000000}%
\pgfsetstrokecolor{currentstroke}%
\pgfsetdash{}{0pt}%
\pgfsys@defobject{currentmarker}{\pgfqpoint{-0.020833in}{-0.020833in}}{\pgfqpoint{0.020833in}{0.020833in}}{%
\pgfpathmoveto{\pgfqpoint{0.000000in}{-0.020833in}}%
\pgfpathcurveto{\pgfqpoint{0.005525in}{-0.020833in}}{\pgfqpoint{0.010825in}{-0.018638in}}{\pgfqpoint{0.014731in}{-0.014731in}}%
\pgfpathcurveto{\pgfqpoint{0.018638in}{-0.010825in}}{\pgfqpoint{0.020833in}{-0.005525in}}{\pgfqpoint{0.020833in}{0.000000in}}%
\pgfpathcurveto{\pgfqpoint{0.020833in}{0.005525in}}{\pgfqpoint{0.018638in}{0.010825in}}{\pgfqpoint{0.014731in}{0.014731in}}%
\pgfpathcurveto{\pgfqpoint{0.010825in}{0.018638in}}{\pgfqpoint{0.005525in}{0.020833in}}{\pgfqpoint{0.000000in}{0.020833in}}%
\pgfpathcurveto{\pgfqpoint{-0.005525in}{0.020833in}}{\pgfqpoint{-0.010825in}{0.018638in}}{\pgfqpoint{-0.014731in}{0.014731in}}%
\pgfpathcurveto{\pgfqpoint{-0.018638in}{0.010825in}}{\pgfqpoint{-0.020833in}{0.005525in}}{\pgfqpoint{-0.020833in}{0.000000in}}%
\pgfpathcurveto{\pgfqpoint{-0.020833in}{-0.005525in}}{\pgfqpoint{-0.018638in}{-0.010825in}}{\pgfqpoint{-0.014731in}{-0.014731in}}%
\pgfpathcurveto{\pgfqpoint{-0.010825in}{-0.018638in}}{\pgfqpoint{-0.005525in}{-0.020833in}}{\pgfqpoint{0.000000in}{-0.020833in}}%
\pgfpathclose%
\pgfusepath{stroke,fill}%
}%
\begin{pgfscope}%
\pgfsys@transformshift{0.961364in}{2.112500in}%
\pgfsys@useobject{currentmarker}{}%
\end{pgfscope}%
\begin{pgfscope}%
\pgfsys@transformshift{1.003636in}{2.045403in}%
\pgfsys@useobject{currentmarker}{}%
\end{pgfscope}%
\begin{pgfscope}%
\pgfsys@transformshift{1.045909in}{1.982041in}%
\pgfsys@useobject{currentmarker}{}%
\end{pgfscope}%
\begin{pgfscope}%
\pgfsys@transformshift{1.088182in}{1.922109in}%
\pgfsys@useobject{currentmarker}{}%
\end{pgfscope}%
\begin{pgfscope}%
\pgfsys@transformshift{1.130455in}{1.865338in}%
\pgfsys@useobject{currentmarker}{}%
\end{pgfscope}%
\begin{pgfscope}%
\pgfsys@transformshift{1.172727in}{1.811483in}%
\pgfsys@useobject{currentmarker}{}%
\end{pgfscope}%
\begin{pgfscope}%
\pgfsys@transformshift{1.215000in}{1.760325in}%
\pgfsys@useobject{currentmarker}{}%
\end{pgfscope}%
\begin{pgfscope}%
\pgfsys@transformshift{1.257273in}{1.711666in}%
\pgfsys@useobject{currentmarker}{}%
\end{pgfscope}%
\begin{pgfscope}%
\pgfsys@transformshift{1.299545in}{1.665329in}%
\pgfsys@useobject{currentmarker}{}%
\end{pgfscope}%
\begin{pgfscope}%
\pgfsys@transformshift{1.341818in}{1.621150in}%
\pgfsys@useobject{currentmarker}{}%
\end{pgfscope}%
\begin{pgfscope}%
\pgfsys@transformshift{1.384091in}{1.578982in}%
\pgfsys@useobject{currentmarker}{}%
\end{pgfscope}%
\begin{pgfscope}%
\pgfsys@transformshift{1.426364in}{1.538692in}%
\pgfsys@useobject{currentmarker}{}%
\end{pgfscope}%
\begin{pgfscope}%
\pgfsys@transformshift{1.468636in}{1.500155in}%
\pgfsys@useobject{currentmarker}{}%
\end{pgfscope}%
\begin{pgfscope}%
\pgfsys@transformshift{1.510909in}{1.463261in}%
\pgfsys@useobject{currentmarker}{}%
\end{pgfscope}%
\begin{pgfscope}%
\pgfsys@transformshift{1.553182in}{1.427907in}%
\pgfsys@useobject{currentmarker}{}%
\end{pgfscope}%
\begin{pgfscope}%
\pgfsys@transformshift{1.595455in}{1.393997in}%
\pgfsys@useobject{currentmarker}{}%
\end{pgfscope}%
\begin{pgfscope}%
\pgfsys@transformshift{1.637727in}{1.361446in}%
\pgfsys@useobject{currentmarker}{}%
\end{pgfscope}%
\begin{pgfscope}%
\pgfsys@transformshift{1.680000in}{1.330173in}%
\pgfsys@useobject{currentmarker}{}%
\end{pgfscope}%
\begin{pgfscope}%
\pgfsys@transformshift{1.722273in}{1.300104in}%
\pgfsys@useobject{currentmarker}{}%
\end{pgfscope}%
\begin{pgfscope}%
\pgfsys@transformshift{1.764545in}{1.271172in}%
\pgfsys@useobject{currentmarker}{}%
\end{pgfscope}%
\begin{pgfscope}%
\pgfsys@transformshift{1.806818in}{1.243312in}%
\pgfsys@useobject{currentmarker}{}%
\end{pgfscope}%
\begin{pgfscope}%
\pgfsys@transformshift{1.849091in}{1.216467in}%
\pgfsys@useobject{currentmarker}{}%
\end{pgfscope}%
\begin{pgfscope}%
\pgfsys@transformshift{1.891364in}{1.190582in}%
\pgfsys@useobject{currentmarker}{}%
\end{pgfscope}%
\begin{pgfscope}%
\pgfsys@transformshift{1.933636in}{1.165606in}%
\pgfsys@useobject{currentmarker}{}%
\end{pgfscope}%
\begin{pgfscope}%
\pgfsys@transformshift{1.975909in}{1.141492in}%
\pgfsys@useobject{currentmarker}{}%
\end{pgfscope}%
\begin{pgfscope}%
\pgfsys@transformshift{2.018182in}{1.118197in}%
\pgfsys@useobject{currentmarker}{}%
\end{pgfscope}%
\begin{pgfscope}%
\pgfsys@transformshift{2.060455in}{1.095679in}%
\pgfsys@useobject{currentmarker}{}%
\end{pgfscope}%
\begin{pgfscope}%
\pgfsys@transformshift{2.102727in}{1.073901in}%
\pgfsys@useobject{currentmarker}{}%
\end{pgfscope}%
\begin{pgfscope}%
\pgfsys@transformshift{2.145000in}{1.052825in}%
\pgfsys@useobject{currentmarker}{}%
\end{pgfscope}%
\begin{pgfscope}%
\pgfsys@transformshift{2.187273in}{1.032420in}%
\pgfsys@useobject{currentmarker}{}%
\end{pgfscope}%
\begin{pgfscope}%
\pgfsys@transformshift{2.229545in}{1.012653in}%
\pgfsys@useobject{currentmarker}{}%
\end{pgfscope}%
\begin{pgfscope}%
\pgfsys@transformshift{2.271818in}{0.993494in}%
\pgfsys@useobject{currentmarker}{}%
\end{pgfscope}%
\begin{pgfscope}%
\pgfsys@transformshift{2.314091in}{0.974917in}%
\pgfsys@useobject{currentmarker}{}%
\end{pgfscope}%
\begin{pgfscope}%
\pgfsys@transformshift{2.356364in}{0.956895in}%
\pgfsys@useobject{currentmarker}{}%
\end{pgfscope}%
\begin{pgfscope}%
\pgfsys@transformshift{2.398636in}{0.939404in}%
\pgfsys@useobject{currentmarker}{}%
\end{pgfscope}%
\begin{pgfscope}%
\pgfsys@transformshift{2.440909in}{0.922420in}%
\pgfsys@useobject{currentmarker}{}%
\end{pgfscope}%
\begin{pgfscope}%
\pgfsys@transformshift{2.483182in}{0.905922in}%
\pgfsys@useobject{currentmarker}{}%
\end{pgfscope}%
\begin{pgfscope}%
\pgfsys@transformshift{2.525455in}{0.889889in}%
\pgfsys@useobject{currentmarker}{}%
\end{pgfscope}%
\begin{pgfscope}%
\pgfsys@transformshift{2.567727in}{0.874302in}%
\pgfsys@useobject{currentmarker}{}%
\end{pgfscope}%
\begin{pgfscope}%
\pgfsys@transformshift{2.610000in}{0.859143in}%
\pgfsys@useobject{currentmarker}{}%
\end{pgfscope}%
\begin{pgfscope}%
\pgfsys@transformshift{2.652273in}{0.844393in}%
\pgfsys@useobject{currentmarker}{}%
\end{pgfscope}%
\begin{pgfscope}%
\pgfsys@transformshift{2.694545in}{0.830037in}%
\pgfsys@useobject{currentmarker}{}%
\end{pgfscope}%
\begin{pgfscope}%
\pgfsys@transformshift{2.736818in}{0.816060in}%
\pgfsys@useobject{currentmarker}{}%
\end{pgfscope}%
\begin{pgfscope}%
\pgfsys@transformshift{2.779091in}{0.802446in}%
\pgfsys@useobject{currentmarker}{}%
\end{pgfscope}%
\begin{pgfscope}%
\pgfsys@transformshift{2.821364in}{0.789181in}%
\pgfsys@useobject{currentmarker}{}%
\end{pgfscope}%
\begin{pgfscope}%
\pgfsys@transformshift{2.863636in}{0.776252in}%
\pgfsys@useobject{currentmarker}{}%
\end{pgfscope}%
\begin{pgfscope}%
\pgfsys@transformshift{2.905909in}{0.763647in}%
\pgfsys@useobject{currentmarker}{}%
\end{pgfscope}%
\begin{pgfscope}%
\pgfsys@transformshift{2.948182in}{0.751353in}%
\pgfsys@useobject{currentmarker}{}%
\end{pgfscope}%
\begin{pgfscope}%
\pgfsys@transformshift{2.990455in}{0.739359in}%
\pgfsys@useobject{currentmarker}{}%
\end{pgfscope}%
\begin{pgfscope}%
\pgfsys@transformshift{3.032727in}{0.727655in}%
\pgfsys@useobject{currentmarker}{}%
\end{pgfscope}%
\begin{pgfscope}%
\pgfsys@transformshift{3.075000in}{0.716230in}%
\pgfsys@useobject{currentmarker}{}%
\end{pgfscope}%
\begin{pgfscope}%
\pgfsys@transformshift{3.117273in}{0.705073in}%
\pgfsys@useobject{currentmarker}{}%
\end{pgfscope}%
\begin{pgfscope}%
\pgfsys@transformshift{3.159545in}{0.694177in}%
\pgfsys@useobject{currentmarker}{}%
\end{pgfscope}%
\begin{pgfscope}%
\pgfsys@transformshift{3.201818in}{0.683531in}%
\pgfsys@useobject{currentmarker}{}%
\end{pgfscope}%
\begin{pgfscope}%
\pgfsys@transformshift{3.244091in}{0.673127in}%
\pgfsys@useobject{currentmarker}{}%
\end{pgfscope}%
\begin{pgfscope}%
\pgfsys@transformshift{3.286364in}{0.662957in}%
\pgfsys@useobject{currentmarker}{}%
\end{pgfscope}%
\begin{pgfscope}%
\pgfsys@transformshift{3.328636in}{0.653013in}%
\pgfsys@useobject{currentmarker}{}%
\end{pgfscope}%
\begin{pgfscope}%
\pgfsys@transformshift{3.370909in}{0.643288in}%
\pgfsys@useobject{currentmarker}{}%
\end{pgfscope}%
\begin{pgfscope}%
\pgfsys@transformshift{3.413182in}{0.633775in}%
\pgfsys@useobject{currentmarker}{}%
\end{pgfscope}%
\begin{pgfscope}%
\pgfsys@transformshift{3.455455in}{0.624466in}%
\pgfsys@useobject{currentmarker}{}%
\end{pgfscope}%
\begin{pgfscope}%
\pgfsys@transformshift{3.497727in}{0.615356in}%
\pgfsys@useobject{currentmarker}{}%
\end{pgfscope}%
\begin{pgfscope}%
\pgfsys@transformshift{3.540000in}{0.606437in}%
\pgfsys@useobject{currentmarker}{}%
\end{pgfscope}%
\begin{pgfscope}%
\pgfsys@transformshift{3.582273in}{0.597705in}%
\pgfsys@useobject{currentmarker}{}%
\end{pgfscope}%
\begin{pgfscope}%
\pgfsys@transformshift{3.624545in}{0.589152in}%
\pgfsys@useobject{currentmarker}{}%
\end{pgfscope}%
\begin{pgfscope}%
\pgfsys@transformshift{3.666818in}{0.580775in}%
\pgfsys@useobject{currentmarker}{}%
\end{pgfscope}%
\begin{pgfscope}%
\pgfsys@transformshift{3.709091in}{0.572566in}%
\pgfsys@useobject{currentmarker}{}%
\end{pgfscope}%
\begin{pgfscope}%
\pgfsys@transformshift{3.751364in}{0.564522in}%
\pgfsys@useobject{currentmarker}{}%
\end{pgfscope}%
\begin{pgfscope}%
\pgfsys@transformshift{3.793636in}{0.556638in}%
\pgfsys@useobject{currentmarker}{}%
\end{pgfscope}%
\begin{pgfscope}%
\pgfsys@transformshift{3.835909in}{0.548908in}%
\pgfsys@useobject{currentmarker}{}%
\end{pgfscope}%
\begin{pgfscope}%
\pgfsys@transformshift{3.878182in}{0.541328in}%
\pgfsys@useobject{currentmarker}{}%
\end{pgfscope}%
\begin{pgfscope}%
\pgfsys@transformshift{3.920455in}{0.533894in}%
\pgfsys@useobject{currentmarker}{}%
\end{pgfscope}%
\begin{pgfscope}%
\pgfsys@transformshift{3.962727in}{0.526602in}%
\pgfsys@useobject{currentmarker}{}%
\end{pgfscope}%
\begin{pgfscope}%
\pgfsys@transformshift{4.005000in}{0.519448in}%
\pgfsys@useobject{currentmarker}{}%
\end{pgfscope}%
\begin{pgfscope}%
\pgfsys@transformshift{4.047273in}{0.512427in}%
\pgfsys@useobject{currentmarker}{}%
\end{pgfscope}%
\begin{pgfscope}%
\pgfsys@transformshift{4.089545in}{0.505536in}%
\pgfsys@useobject{currentmarker}{}%
\end{pgfscope}%
\begin{pgfscope}%
\pgfsys@transformshift{4.131818in}{0.498772in}%
\pgfsys@useobject{currentmarker}{}%
\end{pgfscope}%
\begin{pgfscope}%
\pgfsys@transformshift{4.174091in}{0.492131in}%
\pgfsys@useobject{currentmarker}{}%
\end{pgfscope}%
\begin{pgfscope}%
\pgfsys@transformshift{4.216364in}{0.485610in}%
\pgfsys@useobject{currentmarker}{}%
\end{pgfscope}%
\begin{pgfscope}%
\pgfsys@transformshift{4.258636in}{0.479205in}%
\pgfsys@useobject{currentmarker}{}%
\end{pgfscope}%
\begin{pgfscope}%
\pgfsys@transformshift{4.300909in}{0.472914in}%
\pgfsys@useobject{currentmarker}{}%
\end{pgfscope}%
\begin{pgfscope}%
\pgfsys@transformshift{4.343182in}{0.466733in}%
\pgfsys@useobject{currentmarker}{}%
\end{pgfscope}%
\begin{pgfscope}%
\pgfsys@transformshift{4.385455in}{0.460660in}%
\pgfsys@useobject{currentmarker}{}%
\end{pgfscope}%
\begin{pgfscope}%
\pgfsys@transformshift{4.427727in}{0.454692in}%
\pgfsys@useobject{currentmarker}{}%
\end{pgfscope}%
\begin{pgfscope}%
\pgfsys@transformshift{4.470000in}{0.448825in}%
\pgfsys@useobject{currentmarker}{}%
\end{pgfscope}%
\begin{pgfscope}%
\pgfsys@transformshift{4.512273in}{0.443058in}%
\pgfsys@useobject{currentmarker}{}%
\end{pgfscope}%
\begin{pgfscope}%
\pgfsys@transformshift{4.554545in}{0.437388in}%
\pgfsys@useobject{currentmarker}{}%
\end{pgfscope}%
\begin{pgfscope}%
\pgfsys@transformshift{4.596818in}{0.431813in}%
\pgfsys@useobject{currentmarker}{}%
\end{pgfscope}%
\begin{pgfscope}%
\pgfsys@transformshift{4.639091in}{0.426330in}%
\pgfsys@useobject{currentmarker}{}%
\end{pgfscope}%
\begin{pgfscope}%
\pgfsys@transformshift{4.681364in}{0.420937in}%
\pgfsys@useobject{currentmarker}{}%
\end{pgfscope}%
\begin{pgfscope}%
\pgfsys@transformshift{4.723636in}{0.415631in}%
\pgfsys@useobject{currentmarker}{}%
\end{pgfscope}%
\begin{pgfscope}%
\pgfsys@transformshift{4.765909in}{0.410411in}%
\pgfsys@useobject{currentmarker}{}%
\end{pgfscope}%
\begin{pgfscope}%
\pgfsys@transformshift{4.808182in}{0.405275in}%
\pgfsys@useobject{currentmarker}{}%
\end{pgfscope}%
\begin{pgfscope}%
\pgfsys@transformshift{4.850455in}{0.400220in}%
\pgfsys@useobject{currentmarker}{}%
\end{pgfscope}%
\begin{pgfscope}%
\pgfsys@transformshift{4.892727in}{0.395245in}%
\pgfsys@useobject{currentmarker}{}%
\end{pgfscope}%
\begin{pgfscope}%
\pgfsys@transformshift{4.935000in}{0.390348in}%
\pgfsys@useobject{currentmarker}{}%
\end{pgfscope}%
\begin{pgfscope}%
\pgfsys@transformshift{4.977273in}{0.385527in}%
\pgfsys@useobject{currentmarker}{}%
\end{pgfscope}%
\begin{pgfscope}%
\pgfsys@transformshift{5.019545in}{0.380779in}%
\pgfsys@useobject{currentmarker}{}%
\end{pgfscope}%
\begin{pgfscope}%
\pgfsys@transformshift{5.061818in}{0.376105in}%
\pgfsys@useobject{currentmarker}{}%
\end{pgfscope}%
\begin{pgfscope}%
\pgfsys@transformshift{5.104091in}{0.371501in}%
\pgfsys@useobject{currentmarker}{}%
\end{pgfscope}%
\begin{pgfscope}%
\pgfsys@transformshift{5.146364in}{0.366967in}%
\pgfsys@useobject{currentmarker}{}%
\end{pgfscope}%
\begin{pgfscope}%
\pgfsys@transformshift{5.188636in}{0.362500in}%
\pgfsys@useobject{currentmarker}{}%
\end{pgfscope}%
\end{pgfscope}%
\begin{pgfscope}%
\pgfsetrectcap%
\pgfsetmiterjoin%
\pgfsetlinewidth{0.803000pt}%
\definecolor{currentstroke}{rgb}{0.000000,0.000000,0.000000}%
\pgfsetstrokecolor{currentstroke}%
\pgfsetdash{}{0pt}%
\pgfpathmoveto{\pgfqpoint{0.750000in}{0.275000in}}%
\pgfpathlineto{\pgfqpoint{0.750000in}{2.200000in}}%
\pgfusepath{stroke}%
\end{pgfscope}%
\begin{pgfscope}%
\pgfsetrectcap%
\pgfsetmiterjoin%
\pgfsetlinewidth{0.803000pt}%
\definecolor{currentstroke}{rgb}{0.000000,0.000000,0.000000}%
\pgfsetstrokecolor{currentstroke}%
\pgfsetdash{}{0pt}%
\pgfpathmoveto{\pgfqpoint{5.400000in}{0.275000in}}%
\pgfpathlineto{\pgfqpoint{5.400000in}{2.200000in}}%
\pgfusepath{stroke}%
\end{pgfscope}%
\begin{pgfscope}%
\pgfsetrectcap%
\pgfsetmiterjoin%
\pgfsetlinewidth{0.803000pt}%
\definecolor{currentstroke}{rgb}{0.000000,0.000000,0.000000}%
\pgfsetstrokecolor{currentstroke}%
\pgfsetdash{}{0pt}%
\pgfpathmoveto{\pgfqpoint{0.750000in}{0.275000in}}%
\pgfpathlineto{\pgfqpoint{5.400000in}{0.275000in}}%
\pgfusepath{stroke}%
\end{pgfscope}%
\begin{pgfscope}%
\pgfsetrectcap%
\pgfsetmiterjoin%
\pgfsetlinewidth{0.803000pt}%
\definecolor{currentstroke}{rgb}{0.000000,0.000000,0.000000}%
\pgfsetstrokecolor{currentstroke}%
\pgfsetdash{}{0pt}%
\pgfpathmoveto{\pgfqpoint{0.750000in}{2.200000in}}%
\pgfpathlineto{\pgfqpoint{5.400000in}{2.200000in}}%
\pgfusepath{stroke}%
\end{pgfscope}%
\end{pgfpicture}%
\makeatother%
\endgroup%

\caption{The lower bound in \cref{eq:GMRESprob} for $R=12$, $\eps = 10^{-5}$, $N = \ceil{k^{3}}$, and $\Ct=0.1,$ with $\NLiDRR{\no-\nt} \sim \Exp{\sigma}$ with $\sigma = 1.$\label{fig:prob-theory-plot-0.0}}
\end{figure}

\begin{figure}[h]
    \centering
%% Creator: Matplotlib, PGF backend
%%
%% To include the figure in your LaTeX document, write
%%   \input{<filename>.pgf}
%%
%% Make sure the required packages are loaded in your preamble
%%   \usepackage{pgf}
%%
%% Figures using additional raster images can only be included by \input if
%% they are in the same directory as the main LaTeX file. For loading figures
%% from other directories you can use the `import` package
%%   \usepackage{import}
%% and then include the figures with
%%   \import{<path to file>}{<filename>.pgf}
%%
%% Matplotlib used the following preamble
%%   \usepackage{fontspec}
%%   \setmainfont{DejaVuSerif.ttf}[Path=/home/owen/progs/firedrake-complex/firedrake/lib/python3.5/site-packages/matplotlib/mpl-data/fonts/ttf/]
%%   \setsansfont{DejaVuSans.ttf}[Path=/home/owen/progs/firedrake-complex/firedrake/lib/python3.5/site-packages/matplotlib/mpl-data/fonts/ttf/]
%%   \setmonofont{DejaVuSansMono.ttf}[Path=/home/owen/progs/firedrake-complex/firedrake/lib/python3.5/site-packages/matplotlib/mpl-data/fonts/ttf/]
%%
\begingroup%
\makeatletter%
\begin{pgfpicture}%
\pgfpathrectangle{\pgfpointorigin}{\pgfqpoint{6.400000in}{4.800000in}}%
\pgfusepath{use as bounding box, clip}%
\begin{pgfscope}%
\pgfsetbuttcap%
\pgfsetmiterjoin%
\definecolor{currentfill}{rgb}{1.000000,1.000000,1.000000}%
\pgfsetfillcolor{currentfill}%
\pgfsetlinewidth{0.000000pt}%
\definecolor{currentstroke}{rgb}{1.000000,1.000000,1.000000}%
\pgfsetstrokecolor{currentstroke}%
\pgfsetdash{}{0pt}%
\pgfpathmoveto{\pgfqpoint{0.000000in}{0.000000in}}%
\pgfpathlineto{\pgfqpoint{6.400000in}{0.000000in}}%
\pgfpathlineto{\pgfqpoint{6.400000in}{4.800000in}}%
\pgfpathlineto{\pgfqpoint{0.000000in}{4.800000in}}%
\pgfpathclose%
\pgfusepath{fill}%
\end{pgfscope}%
\begin{pgfscope}%
\pgfsetbuttcap%
\pgfsetmiterjoin%
\definecolor{currentfill}{rgb}{1.000000,1.000000,1.000000}%
\pgfsetfillcolor{currentfill}%
\pgfsetlinewidth{0.000000pt}%
\definecolor{currentstroke}{rgb}{0.000000,0.000000,0.000000}%
\pgfsetstrokecolor{currentstroke}%
\pgfsetstrokeopacity{0.000000}%
\pgfsetdash{}{0pt}%
\pgfpathmoveto{\pgfqpoint{0.800000in}{0.528000in}}%
\pgfpathlineto{\pgfqpoint{5.760000in}{0.528000in}}%
\pgfpathlineto{\pgfqpoint{5.760000in}{4.224000in}}%
\pgfpathlineto{\pgfqpoint{0.800000in}{4.224000in}}%
\pgfpathclose%
\pgfusepath{fill}%
\end{pgfscope}%
\begin{pgfscope}%
\pgfsetbuttcap%
\pgfsetroundjoin%
\definecolor{currentfill}{rgb}{0.000000,0.000000,0.000000}%
\pgfsetfillcolor{currentfill}%
\pgfsetlinewidth{0.803000pt}%
\definecolor{currentstroke}{rgb}{0.000000,0.000000,0.000000}%
\pgfsetstrokecolor{currentstroke}%
\pgfsetdash{}{0pt}%
\pgfsys@defobject{currentmarker}{\pgfqpoint{0.000000in}{-0.048611in}}{\pgfqpoint{0.000000in}{0.000000in}}{%
\pgfpathmoveto{\pgfqpoint{0.000000in}{0.000000in}}%
\pgfpathlineto{\pgfqpoint{0.000000in}{-0.048611in}}%
\pgfusepath{stroke,fill}%
}%
\begin{pgfscope}%
\pgfsys@transformshift{1.025455in}{0.528000in}%
\pgfsys@useobject{currentmarker}{}%
\end{pgfscope}%
\end{pgfscope}%
\begin{pgfscope}%
\definecolor{textcolor}{rgb}{0.000000,0.000000,0.000000}%
\pgfsetstrokecolor{textcolor}%
\pgfsetfillcolor{textcolor}%
\pgftext[x=1.025455in,y=0.430778in,,top]{\color{textcolor}\sffamily\fontsize{10.000000}{12.000000}\selectfont 10}%
\end{pgfscope}%
\begin{pgfscope}%
\pgfsetbuttcap%
\pgfsetroundjoin%
\definecolor{currentfill}{rgb}{0.000000,0.000000,0.000000}%
\pgfsetfillcolor{currentfill}%
\pgfsetlinewidth{0.803000pt}%
\definecolor{currentstroke}{rgb}{0.000000,0.000000,0.000000}%
\pgfsetstrokecolor{currentstroke}%
\pgfsetdash{}{0pt}%
\pgfsys@defobject{currentmarker}{\pgfqpoint{0.000000in}{-0.048611in}}{\pgfqpoint{0.000000in}{0.000000in}}{%
\pgfpathmoveto{\pgfqpoint{0.000000in}{0.000000in}}%
\pgfpathlineto{\pgfqpoint{0.000000in}{-0.048611in}}%
\pgfusepath{stroke,fill}%
}%
\begin{pgfscope}%
\pgfsys@transformshift{1.776970in}{0.528000in}%
\pgfsys@useobject{currentmarker}{}%
\end{pgfscope}%
\end{pgfscope}%
\begin{pgfscope}%
\definecolor{textcolor}{rgb}{0.000000,0.000000,0.000000}%
\pgfsetstrokecolor{textcolor}%
\pgfsetfillcolor{textcolor}%
\pgftext[x=1.776970in,y=0.430778in,,top]{\color{textcolor}\sffamily\fontsize{10.000000}{12.000000}\selectfont 15}%
\end{pgfscope}%
\begin{pgfscope}%
\pgfsetbuttcap%
\pgfsetroundjoin%
\definecolor{currentfill}{rgb}{0.000000,0.000000,0.000000}%
\pgfsetfillcolor{currentfill}%
\pgfsetlinewidth{0.803000pt}%
\definecolor{currentstroke}{rgb}{0.000000,0.000000,0.000000}%
\pgfsetstrokecolor{currentstroke}%
\pgfsetdash{}{0pt}%
\pgfsys@defobject{currentmarker}{\pgfqpoint{0.000000in}{-0.048611in}}{\pgfqpoint{0.000000in}{0.000000in}}{%
\pgfpathmoveto{\pgfqpoint{0.000000in}{0.000000in}}%
\pgfpathlineto{\pgfqpoint{0.000000in}{-0.048611in}}%
\pgfusepath{stroke,fill}%
}%
\begin{pgfscope}%
\pgfsys@transformshift{2.528485in}{0.528000in}%
\pgfsys@useobject{currentmarker}{}%
\end{pgfscope}%
\end{pgfscope}%
\begin{pgfscope}%
\definecolor{textcolor}{rgb}{0.000000,0.000000,0.000000}%
\pgfsetstrokecolor{textcolor}%
\pgfsetfillcolor{textcolor}%
\pgftext[x=2.528485in,y=0.430778in,,top]{\color{textcolor}\sffamily\fontsize{10.000000}{12.000000}\selectfont 20}%
\end{pgfscope}%
\begin{pgfscope}%
\pgfsetbuttcap%
\pgfsetroundjoin%
\definecolor{currentfill}{rgb}{0.000000,0.000000,0.000000}%
\pgfsetfillcolor{currentfill}%
\pgfsetlinewidth{0.803000pt}%
\definecolor{currentstroke}{rgb}{0.000000,0.000000,0.000000}%
\pgfsetstrokecolor{currentstroke}%
\pgfsetdash{}{0pt}%
\pgfsys@defobject{currentmarker}{\pgfqpoint{0.000000in}{-0.048611in}}{\pgfqpoint{0.000000in}{0.000000in}}{%
\pgfpathmoveto{\pgfqpoint{0.000000in}{0.000000in}}%
\pgfpathlineto{\pgfqpoint{0.000000in}{-0.048611in}}%
\pgfusepath{stroke,fill}%
}%
\begin{pgfscope}%
\pgfsys@transformshift{3.280000in}{0.528000in}%
\pgfsys@useobject{currentmarker}{}%
\end{pgfscope}%
\end{pgfscope}%
\begin{pgfscope}%
\definecolor{textcolor}{rgb}{0.000000,0.000000,0.000000}%
\pgfsetstrokecolor{textcolor}%
\pgfsetfillcolor{textcolor}%
\pgftext[x=3.280000in,y=0.430778in,,top]{\color{textcolor}\sffamily\fontsize{10.000000}{12.000000}\selectfont 25}%
\end{pgfscope}%
\begin{pgfscope}%
\pgfsetbuttcap%
\pgfsetroundjoin%
\definecolor{currentfill}{rgb}{0.000000,0.000000,0.000000}%
\pgfsetfillcolor{currentfill}%
\pgfsetlinewidth{0.803000pt}%
\definecolor{currentstroke}{rgb}{0.000000,0.000000,0.000000}%
\pgfsetstrokecolor{currentstroke}%
\pgfsetdash{}{0pt}%
\pgfsys@defobject{currentmarker}{\pgfqpoint{0.000000in}{-0.048611in}}{\pgfqpoint{0.000000in}{0.000000in}}{%
\pgfpathmoveto{\pgfqpoint{0.000000in}{0.000000in}}%
\pgfpathlineto{\pgfqpoint{0.000000in}{-0.048611in}}%
\pgfusepath{stroke,fill}%
}%
\begin{pgfscope}%
\pgfsys@transformshift{4.031515in}{0.528000in}%
\pgfsys@useobject{currentmarker}{}%
\end{pgfscope}%
\end{pgfscope}%
\begin{pgfscope}%
\definecolor{textcolor}{rgb}{0.000000,0.000000,0.000000}%
\pgfsetstrokecolor{textcolor}%
\pgfsetfillcolor{textcolor}%
\pgftext[x=4.031515in,y=0.430778in,,top]{\color{textcolor}\sffamily\fontsize{10.000000}{12.000000}\selectfont 30}%
\end{pgfscope}%
\begin{pgfscope}%
\pgfsetbuttcap%
\pgfsetroundjoin%
\definecolor{currentfill}{rgb}{0.000000,0.000000,0.000000}%
\pgfsetfillcolor{currentfill}%
\pgfsetlinewidth{0.803000pt}%
\definecolor{currentstroke}{rgb}{0.000000,0.000000,0.000000}%
\pgfsetstrokecolor{currentstroke}%
\pgfsetdash{}{0pt}%
\pgfsys@defobject{currentmarker}{\pgfqpoint{0.000000in}{-0.048611in}}{\pgfqpoint{0.000000in}{0.000000in}}{%
\pgfpathmoveto{\pgfqpoint{0.000000in}{0.000000in}}%
\pgfpathlineto{\pgfqpoint{0.000000in}{-0.048611in}}%
\pgfusepath{stroke,fill}%
}%
\begin{pgfscope}%
\pgfsys@transformshift{4.783030in}{0.528000in}%
\pgfsys@useobject{currentmarker}{}%
\end{pgfscope}%
\end{pgfscope}%
\begin{pgfscope}%
\definecolor{textcolor}{rgb}{0.000000,0.000000,0.000000}%
\pgfsetstrokecolor{textcolor}%
\pgfsetfillcolor{textcolor}%
\pgftext[x=4.783030in,y=0.430778in,,top]{\color{textcolor}\sffamily\fontsize{10.000000}{12.000000}\selectfont 35}%
\end{pgfscope}%
\begin{pgfscope}%
\pgfsetbuttcap%
\pgfsetroundjoin%
\definecolor{currentfill}{rgb}{0.000000,0.000000,0.000000}%
\pgfsetfillcolor{currentfill}%
\pgfsetlinewidth{0.803000pt}%
\definecolor{currentstroke}{rgb}{0.000000,0.000000,0.000000}%
\pgfsetstrokecolor{currentstroke}%
\pgfsetdash{}{0pt}%
\pgfsys@defobject{currentmarker}{\pgfqpoint{0.000000in}{-0.048611in}}{\pgfqpoint{0.000000in}{0.000000in}}{%
\pgfpathmoveto{\pgfqpoint{0.000000in}{0.000000in}}%
\pgfpathlineto{\pgfqpoint{0.000000in}{-0.048611in}}%
\pgfusepath{stroke,fill}%
}%
\begin{pgfscope}%
\pgfsys@transformshift{5.534545in}{0.528000in}%
\pgfsys@useobject{currentmarker}{}%
\end{pgfscope}%
\end{pgfscope}%
\begin{pgfscope}%
\definecolor{textcolor}{rgb}{0.000000,0.000000,0.000000}%
\pgfsetstrokecolor{textcolor}%
\pgfsetfillcolor{textcolor}%
\pgftext[x=5.534545in,y=0.430778in,,top]{\color{textcolor}\sffamily\fontsize{10.000000}{12.000000}\selectfont 40}%
\end{pgfscope}%
\begin{pgfscope}%
\definecolor{textcolor}{rgb}{0.000000,0.000000,0.000000}%
\pgfsetstrokecolor{textcolor}%
\pgfsetfillcolor{textcolor}%
\pgftext[x=3.280000in,y=0.240809in,,top]{\color{textcolor}\sffamily\fontsize{10.000000}{12.000000}\selectfont \(\displaystyle k\)}%
\end{pgfscope}%
\begin{pgfscope}%
\pgfsetbuttcap%
\pgfsetroundjoin%
\definecolor{currentfill}{rgb}{0.000000,0.000000,0.000000}%
\pgfsetfillcolor{currentfill}%
\pgfsetlinewidth{0.803000pt}%
\definecolor{currentstroke}{rgb}{0.000000,0.000000,0.000000}%
\pgfsetstrokecolor{currentstroke}%
\pgfsetdash{}{0pt}%
\pgfsys@defobject{currentmarker}{\pgfqpoint{-0.048611in}{0.000000in}}{\pgfqpoint{0.000000in}{0.000000in}}{%
\pgfpathmoveto{\pgfqpoint{0.000000in}{0.000000in}}%
\pgfpathlineto{\pgfqpoint{-0.048611in}{0.000000in}}%
\pgfusepath{stroke,fill}%
}%
\begin{pgfscope}%
\pgfsys@transformshift{0.800000in}{1.059932in}%
\pgfsys@useobject{currentmarker}{}%
\end{pgfscope}%
\end{pgfscope}%
\begin{pgfscope}%
\definecolor{textcolor}{rgb}{0.000000,0.000000,0.000000}%
\pgfsetstrokecolor{textcolor}%
\pgfsetfillcolor{textcolor}%
\pgftext[x=0.481898in,y=1.007170in,left,base]{\color{textcolor}\sffamily\fontsize{10.000000}{12.000000}\selectfont 6.5}%
\end{pgfscope}%
\begin{pgfscope}%
\pgfsetbuttcap%
\pgfsetroundjoin%
\definecolor{currentfill}{rgb}{0.000000,0.000000,0.000000}%
\pgfsetfillcolor{currentfill}%
\pgfsetlinewidth{0.803000pt}%
\definecolor{currentstroke}{rgb}{0.000000,0.000000,0.000000}%
\pgfsetstrokecolor{currentstroke}%
\pgfsetdash{}{0pt}%
\pgfsys@defobject{currentmarker}{\pgfqpoint{-0.048611in}{0.000000in}}{\pgfqpoint{0.000000in}{0.000000in}}{%
\pgfpathmoveto{\pgfqpoint{0.000000in}{0.000000in}}%
\pgfpathlineto{\pgfqpoint{-0.048611in}{0.000000in}}%
\pgfusepath{stroke,fill}%
}%
\begin{pgfscope}%
\pgfsys@transformshift{0.800000in}{2.047100in}%
\pgfsys@useobject{currentmarker}{}%
\end{pgfscope}%
\end{pgfscope}%
\begin{pgfscope}%
\definecolor{textcolor}{rgb}{0.000000,0.000000,0.000000}%
\pgfsetstrokecolor{textcolor}%
\pgfsetfillcolor{textcolor}%
\pgftext[x=0.481898in,y=1.994338in,left,base]{\color{textcolor}\sffamily\fontsize{10.000000}{12.000000}\selectfont 7.0}%
\end{pgfscope}%
\begin{pgfscope}%
\pgfsetbuttcap%
\pgfsetroundjoin%
\definecolor{currentfill}{rgb}{0.000000,0.000000,0.000000}%
\pgfsetfillcolor{currentfill}%
\pgfsetlinewidth{0.803000pt}%
\definecolor{currentstroke}{rgb}{0.000000,0.000000,0.000000}%
\pgfsetstrokecolor{currentstroke}%
\pgfsetdash{}{0pt}%
\pgfsys@defobject{currentmarker}{\pgfqpoint{-0.048611in}{0.000000in}}{\pgfqpoint{0.000000in}{0.000000in}}{%
\pgfpathmoveto{\pgfqpoint{0.000000in}{0.000000in}}%
\pgfpathlineto{\pgfqpoint{-0.048611in}{0.000000in}}%
\pgfusepath{stroke,fill}%
}%
\begin{pgfscope}%
\pgfsys@transformshift{0.800000in}{3.034268in}%
\pgfsys@useobject{currentmarker}{}%
\end{pgfscope}%
\end{pgfscope}%
\begin{pgfscope}%
\definecolor{textcolor}{rgb}{0.000000,0.000000,0.000000}%
\pgfsetstrokecolor{textcolor}%
\pgfsetfillcolor{textcolor}%
\pgftext[x=0.481898in,y=2.981506in,left,base]{\color{textcolor}\sffamily\fontsize{10.000000}{12.000000}\selectfont 7.5}%
\end{pgfscope}%
\begin{pgfscope}%
\pgfsetbuttcap%
\pgfsetroundjoin%
\definecolor{currentfill}{rgb}{0.000000,0.000000,0.000000}%
\pgfsetfillcolor{currentfill}%
\pgfsetlinewidth{0.803000pt}%
\definecolor{currentstroke}{rgb}{0.000000,0.000000,0.000000}%
\pgfsetstrokecolor{currentstroke}%
\pgfsetdash{}{0pt}%
\pgfsys@defobject{currentmarker}{\pgfqpoint{-0.048611in}{0.000000in}}{\pgfqpoint{0.000000in}{0.000000in}}{%
\pgfpathmoveto{\pgfqpoint{0.000000in}{0.000000in}}%
\pgfpathlineto{\pgfqpoint{-0.048611in}{0.000000in}}%
\pgfusepath{stroke,fill}%
}%
\begin{pgfscope}%
\pgfsys@transformshift{0.800000in}{4.021436in}%
\pgfsys@useobject{currentmarker}{}%
\end{pgfscope}%
\end{pgfscope}%
\begin{pgfscope}%
\definecolor{textcolor}{rgb}{0.000000,0.000000,0.000000}%
\pgfsetstrokecolor{textcolor}%
\pgfsetfillcolor{textcolor}%
\pgftext[x=0.481898in,y=3.968674in,left,base]{\color{textcolor}\sffamily\fontsize{10.000000}{12.000000}\selectfont 8.0}%
\end{pgfscope}%
\begin{pgfscope}%
\definecolor{textcolor}{rgb}{0.000000,0.000000,0.000000}%
\pgfsetstrokecolor{textcolor}%
\pgfsetfillcolor{textcolor}%
\pgftext[x=0.426343in,y=2.376000in,,bottom,rotate=90.000000]{\color{textcolor}\sffamily\fontsize{10.000000}{12.000000}\selectfont Probability Number of GMRES iterations is at most 12}%
\end{pgfscope}%
\begin{pgfscope}%
\definecolor{textcolor}{rgb}{0.000000,0.000000,0.000000}%
\pgfsetstrokecolor{textcolor}%
\pgfsetfillcolor{textcolor}%
\pgftext[x=0.800000in,y=4.265667in,left,base]{\color{textcolor}\sffamily\fontsize{10.000000}{12.000000}\selectfont 1e−11+2.983096487e−1}%
\end{pgfscope}%
\begin{pgfscope}%
\pgfpathrectangle{\pgfqpoint{0.800000in}{0.528000in}}{\pgfqpoint{4.960000in}{3.696000in}}%
\pgfusepath{clip}%
\pgfsetbuttcap%
\pgfsetroundjoin%
\definecolor{currentfill}{rgb}{0.121569,0.466667,0.705882}%
\pgfsetfillcolor{currentfill}%
\pgfsetlinewidth{1.003750pt}%
\definecolor{currentstroke}{rgb}{0.121569,0.466667,0.705882}%
\pgfsetstrokecolor{currentstroke}%
\pgfsetdash{}{0pt}%
\pgfsys@defobject{currentmarker}{\pgfqpoint{-0.020833in}{-0.020833in}}{\pgfqpoint{0.020833in}{0.020833in}}{%
\pgfpathmoveto{\pgfqpoint{0.000000in}{-0.020833in}}%
\pgfpathcurveto{\pgfqpoint{0.005525in}{-0.020833in}}{\pgfqpoint{0.010825in}{-0.018638in}}{\pgfqpoint{0.014731in}{-0.014731in}}%
\pgfpathcurveto{\pgfqpoint{0.018638in}{-0.010825in}}{\pgfqpoint{0.020833in}{-0.005525in}}{\pgfqpoint{0.020833in}{0.000000in}}%
\pgfpathcurveto{\pgfqpoint{0.020833in}{0.005525in}}{\pgfqpoint{0.018638in}{0.010825in}}{\pgfqpoint{0.014731in}{0.014731in}}%
\pgfpathcurveto{\pgfqpoint{0.010825in}{0.018638in}}{\pgfqpoint{0.005525in}{0.020833in}}{\pgfqpoint{0.000000in}{0.020833in}}%
\pgfpathcurveto{\pgfqpoint{-0.005525in}{0.020833in}}{\pgfqpoint{-0.010825in}{0.018638in}}{\pgfqpoint{-0.014731in}{0.014731in}}%
\pgfpathcurveto{\pgfqpoint{-0.018638in}{0.010825in}}{\pgfqpoint{-0.020833in}{0.005525in}}{\pgfqpoint{-0.020833in}{0.000000in}}%
\pgfpathcurveto{\pgfqpoint{-0.020833in}{-0.005525in}}{\pgfqpoint{-0.018638in}{-0.010825in}}{\pgfqpoint{-0.014731in}{-0.014731in}}%
\pgfpathcurveto{\pgfqpoint{-0.010825in}{-0.018638in}}{\pgfqpoint{-0.005525in}{-0.020833in}}{\pgfqpoint{0.000000in}{-0.020833in}}%
\pgfpathclose%
\pgfusepath{stroke,fill}%
}%
\begin{pgfscope}%
\pgfsys@transformshift{1.025455in}{2.375996in}%
\pgfsys@useobject{currentmarker}{}%
\end{pgfscope}%
\begin{pgfscope}%
\pgfsys@transformshift{1.070545in}{2.375986in}%
\pgfsys@useobject{currentmarker}{}%
\end{pgfscope}%
\begin{pgfscope}%
\pgfsys@transformshift{1.115636in}{2.375996in}%
\pgfsys@useobject{currentmarker}{}%
\end{pgfscope}%
\begin{pgfscope}%
\pgfsys@transformshift{1.160727in}{2.376006in}%
\pgfsys@useobject{currentmarker}{}%
\end{pgfscope}%
\begin{pgfscope}%
\pgfsys@transformshift{1.205818in}{2.375986in}%
\pgfsys@useobject{currentmarker}{}%
\end{pgfscope}%
\begin{pgfscope}%
\pgfsys@transformshift{1.250909in}{2.375967in}%
\pgfsys@useobject{currentmarker}{}%
\end{pgfscope}%
\begin{pgfscope}%
\pgfsys@transformshift{1.296000in}{2.375996in}%
\pgfsys@useobject{currentmarker}{}%
\end{pgfscope}%
\begin{pgfscope}%
\pgfsys@transformshift{1.341091in}{2.375986in}%
\pgfsys@useobject{currentmarker}{}%
\end{pgfscope}%
\begin{pgfscope}%
\pgfsys@transformshift{1.386182in}{4.055967in}%
\pgfsys@useobject{currentmarker}{}%
\end{pgfscope}%
\begin{pgfscope}%
\pgfsys@transformshift{1.431273in}{4.055967in}%
\pgfsys@useobject{currentmarker}{}%
\end{pgfscope}%
\begin{pgfscope}%
\pgfsys@transformshift{1.476364in}{4.056006in}%
\pgfsys@useobject{currentmarker}{}%
\end{pgfscope}%
\begin{pgfscope}%
\pgfsys@transformshift{1.521455in}{4.055996in}%
\pgfsys@useobject{currentmarker}{}%
\end{pgfscope}%
\begin{pgfscope}%
\pgfsys@transformshift{1.566545in}{4.055996in}%
\pgfsys@useobject{currentmarker}{}%
\end{pgfscope}%
\begin{pgfscope}%
\pgfsys@transformshift{1.611636in}{4.055977in}%
\pgfsys@useobject{currentmarker}{}%
\end{pgfscope}%
\begin{pgfscope}%
\pgfsys@transformshift{1.656727in}{4.055977in}%
\pgfsys@useobject{currentmarker}{}%
\end{pgfscope}%
\begin{pgfscope}%
\pgfsys@transformshift{1.701818in}{4.056006in}%
\pgfsys@useobject{currentmarker}{}%
\end{pgfscope}%
\begin{pgfscope}%
\pgfsys@transformshift{1.746909in}{4.056006in}%
\pgfsys@useobject{currentmarker}{}%
\end{pgfscope}%
\begin{pgfscope}%
\pgfsys@transformshift{1.792000in}{4.055967in}%
\pgfsys@useobject{currentmarker}{}%
\end{pgfscope}%
\begin{pgfscope}%
\pgfsys@transformshift{1.837091in}{4.055996in}%
\pgfsys@useobject{currentmarker}{}%
\end{pgfscope}%
\begin{pgfscope}%
\pgfsys@transformshift{1.882182in}{4.055977in}%
\pgfsys@useobject{currentmarker}{}%
\end{pgfscope}%
\begin{pgfscope}%
\pgfsys@transformshift{1.927273in}{4.055996in}%
\pgfsys@useobject{currentmarker}{}%
\end{pgfscope}%
\begin{pgfscope}%
\pgfsys@transformshift{1.972364in}{4.055967in}%
\pgfsys@useobject{currentmarker}{}%
\end{pgfscope}%
\begin{pgfscope}%
\pgfsys@transformshift{2.017455in}{4.055977in}%
\pgfsys@useobject{currentmarker}{}%
\end{pgfscope}%
\begin{pgfscope}%
\pgfsys@transformshift{2.062545in}{4.055977in}%
\pgfsys@useobject{currentmarker}{}%
\end{pgfscope}%
\begin{pgfscope}%
\pgfsys@transformshift{2.107636in}{4.055996in}%
\pgfsys@useobject{currentmarker}{}%
\end{pgfscope}%
\begin{pgfscope}%
\pgfsys@transformshift{2.152727in}{4.055977in}%
\pgfsys@useobject{currentmarker}{}%
\end{pgfscope}%
\begin{pgfscope}%
\pgfsys@transformshift{2.197818in}{4.055977in}%
\pgfsys@useobject{currentmarker}{}%
\end{pgfscope}%
\begin{pgfscope}%
\pgfsys@transformshift{2.242909in}{4.055967in}%
\pgfsys@useobject{currentmarker}{}%
\end{pgfscope}%
\begin{pgfscope}%
\pgfsys@transformshift{2.288000in}{4.055996in}%
\pgfsys@useobject{currentmarker}{}%
\end{pgfscope}%
\begin{pgfscope}%
\pgfsys@transformshift{2.333091in}{4.056006in}%
\pgfsys@useobject{currentmarker}{}%
\end{pgfscope}%
\begin{pgfscope}%
\pgfsys@transformshift{2.378182in}{4.055967in}%
\pgfsys@useobject{currentmarker}{}%
\end{pgfscope}%
\begin{pgfscope}%
\pgfsys@transformshift{2.423273in}{4.055977in}%
\pgfsys@useobject{currentmarker}{}%
\end{pgfscope}%
\begin{pgfscope}%
\pgfsys@transformshift{2.468364in}{4.055957in}%
\pgfsys@useobject{currentmarker}{}%
\end{pgfscope}%
\begin{pgfscope}%
\pgfsys@transformshift{2.513455in}{4.055977in}%
\pgfsys@useobject{currentmarker}{}%
\end{pgfscope}%
\begin{pgfscope}%
\pgfsys@transformshift{2.558545in}{4.056006in}%
\pgfsys@useobject{currentmarker}{}%
\end{pgfscope}%
\begin{pgfscope}%
\pgfsys@transformshift{2.603636in}{4.055996in}%
\pgfsys@useobject{currentmarker}{}%
\end{pgfscope}%
\begin{pgfscope}%
\pgfsys@transformshift{2.648727in}{4.055977in}%
\pgfsys@useobject{currentmarker}{}%
\end{pgfscope}%
\begin{pgfscope}%
\pgfsys@transformshift{2.693818in}{4.055977in}%
\pgfsys@useobject{currentmarker}{}%
\end{pgfscope}%
\begin{pgfscope}%
\pgfsys@transformshift{2.738909in}{4.055957in}%
\pgfsys@useobject{currentmarker}{}%
\end{pgfscope}%
\begin{pgfscope}%
\pgfsys@transformshift{2.784000in}{4.055996in}%
\pgfsys@useobject{currentmarker}{}%
\end{pgfscope}%
\begin{pgfscope}%
\pgfsys@transformshift{2.829091in}{4.055967in}%
\pgfsys@useobject{currentmarker}{}%
\end{pgfscope}%
\begin{pgfscope}%
\pgfsys@transformshift{2.874182in}{4.055977in}%
\pgfsys@useobject{currentmarker}{}%
\end{pgfscope}%
\begin{pgfscope}%
\pgfsys@transformshift{2.919273in}{4.055977in}%
\pgfsys@useobject{currentmarker}{}%
\end{pgfscope}%
\begin{pgfscope}%
\pgfsys@transformshift{2.964364in}{4.056006in}%
\pgfsys@useobject{currentmarker}{}%
\end{pgfscope}%
\begin{pgfscope}%
\pgfsys@transformshift{3.009455in}{4.055996in}%
\pgfsys@useobject{currentmarker}{}%
\end{pgfscope}%
\begin{pgfscope}%
\pgfsys@transformshift{3.054545in}{4.055996in}%
\pgfsys@useobject{currentmarker}{}%
\end{pgfscope}%
\begin{pgfscope}%
\pgfsys@transformshift{3.099636in}{4.055996in}%
\pgfsys@useobject{currentmarker}{}%
\end{pgfscope}%
\begin{pgfscope}%
\pgfsys@transformshift{3.144727in}{4.055996in}%
\pgfsys@useobject{currentmarker}{}%
\end{pgfscope}%
\begin{pgfscope}%
\pgfsys@transformshift{3.189818in}{0.696016in}%
\pgfsys@useobject{currentmarker}{}%
\end{pgfscope}%
\begin{pgfscope}%
\pgfsys@transformshift{3.234909in}{0.696025in}%
\pgfsys@useobject{currentmarker}{}%
\end{pgfscope}%
\begin{pgfscope}%
\pgfsys@transformshift{3.280000in}{0.696016in}%
\pgfsys@useobject{currentmarker}{}%
\end{pgfscope}%
\begin{pgfscope}%
\pgfsys@transformshift{3.325091in}{0.696025in}%
\pgfsys@useobject{currentmarker}{}%
\end{pgfscope}%
\begin{pgfscope}%
\pgfsys@transformshift{3.370182in}{0.696025in}%
\pgfsys@useobject{currentmarker}{}%
\end{pgfscope}%
\begin{pgfscope}%
\pgfsys@transformshift{3.415273in}{0.696025in}%
\pgfsys@useobject{currentmarker}{}%
\end{pgfscope}%
\begin{pgfscope}%
\pgfsys@transformshift{3.460364in}{0.696016in}%
\pgfsys@useobject{currentmarker}{}%
\end{pgfscope}%
\begin{pgfscope}%
\pgfsys@transformshift{3.505455in}{0.696016in}%
\pgfsys@useobject{currentmarker}{}%
\end{pgfscope}%
\begin{pgfscope}%
\pgfsys@transformshift{3.550545in}{0.696025in}%
\pgfsys@useobject{currentmarker}{}%
\end{pgfscope}%
\begin{pgfscope}%
\pgfsys@transformshift{3.595636in}{0.696025in}%
\pgfsys@useobject{currentmarker}{}%
\end{pgfscope}%
\begin{pgfscope}%
\pgfsys@transformshift{3.640727in}{0.696025in}%
\pgfsys@useobject{currentmarker}{}%
\end{pgfscope}%
\begin{pgfscope}%
\pgfsys@transformshift{3.685818in}{0.696025in}%
\pgfsys@useobject{currentmarker}{}%
\end{pgfscope}%
\begin{pgfscope}%
\pgfsys@transformshift{3.730909in}{0.696016in}%
\pgfsys@useobject{currentmarker}{}%
\end{pgfscope}%
\begin{pgfscope}%
\pgfsys@transformshift{3.776000in}{0.696016in}%
\pgfsys@useobject{currentmarker}{}%
\end{pgfscope}%
\begin{pgfscope}%
\pgfsys@transformshift{3.821091in}{0.696016in}%
\pgfsys@useobject{currentmarker}{}%
\end{pgfscope}%
\begin{pgfscope}%
\pgfsys@transformshift{3.866182in}{0.696025in}%
\pgfsys@useobject{currentmarker}{}%
\end{pgfscope}%
\begin{pgfscope}%
\pgfsys@transformshift{3.911273in}{0.696025in}%
\pgfsys@useobject{currentmarker}{}%
\end{pgfscope}%
\begin{pgfscope}%
\pgfsys@transformshift{3.956364in}{0.696016in}%
\pgfsys@useobject{currentmarker}{}%
\end{pgfscope}%
\begin{pgfscope}%
\pgfsys@transformshift{4.001455in}{0.696025in}%
\pgfsys@useobject{currentmarker}{}%
\end{pgfscope}%
\begin{pgfscope}%
\pgfsys@transformshift{4.046545in}{0.696025in}%
\pgfsys@useobject{currentmarker}{}%
\end{pgfscope}%
\begin{pgfscope}%
\pgfsys@transformshift{4.091636in}{0.696025in}%
\pgfsys@useobject{currentmarker}{}%
\end{pgfscope}%
\begin{pgfscope}%
\pgfsys@transformshift{4.136727in}{0.696025in}%
\pgfsys@useobject{currentmarker}{}%
\end{pgfscope}%
\begin{pgfscope}%
\pgfsys@transformshift{4.181818in}{0.696025in}%
\pgfsys@useobject{currentmarker}{}%
\end{pgfscope}%
\begin{pgfscope}%
\pgfsys@transformshift{4.226909in}{0.696025in}%
\pgfsys@useobject{currentmarker}{}%
\end{pgfscope}%
\begin{pgfscope}%
\pgfsys@transformshift{4.272000in}{0.696016in}%
\pgfsys@useobject{currentmarker}{}%
\end{pgfscope}%
\begin{pgfscope}%
\pgfsys@transformshift{4.317091in}{0.696025in}%
\pgfsys@useobject{currentmarker}{}%
\end{pgfscope}%
\begin{pgfscope}%
\pgfsys@transformshift{4.362182in}{0.696016in}%
\pgfsys@useobject{currentmarker}{}%
\end{pgfscope}%
\begin{pgfscope}%
\pgfsys@transformshift{4.407273in}{0.696025in}%
\pgfsys@useobject{currentmarker}{}%
\end{pgfscope}%
\begin{pgfscope}%
\pgfsys@transformshift{4.452364in}{0.696025in}%
\pgfsys@useobject{currentmarker}{}%
\end{pgfscope}%
\begin{pgfscope}%
\pgfsys@transformshift{4.497455in}{0.696025in}%
\pgfsys@useobject{currentmarker}{}%
\end{pgfscope}%
\begin{pgfscope}%
\pgfsys@transformshift{4.542545in}{0.696025in}%
\pgfsys@useobject{currentmarker}{}%
\end{pgfscope}%
\begin{pgfscope}%
\pgfsys@transformshift{4.587636in}{0.696016in}%
\pgfsys@useobject{currentmarker}{}%
\end{pgfscope}%
\begin{pgfscope}%
\pgfsys@transformshift{4.632727in}{0.696025in}%
\pgfsys@useobject{currentmarker}{}%
\end{pgfscope}%
\begin{pgfscope}%
\pgfsys@transformshift{4.677818in}{0.696016in}%
\pgfsys@useobject{currentmarker}{}%
\end{pgfscope}%
\begin{pgfscope}%
\pgfsys@transformshift{4.722909in}{0.696016in}%
\pgfsys@useobject{currentmarker}{}%
\end{pgfscope}%
\begin{pgfscope}%
\pgfsys@transformshift{4.768000in}{0.696016in}%
\pgfsys@useobject{currentmarker}{}%
\end{pgfscope}%
\begin{pgfscope}%
\pgfsys@transformshift{4.813091in}{0.696016in}%
\pgfsys@useobject{currentmarker}{}%
\end{pgfscope}%
\begin{pgfscope}%
\pgfsys@transformshift{4.858182in}{0.696016in}%
\pgfsys@useobject{currentmarker}{}%
\end{pgfscope}%
\begin{pgfscope}%
\pgfsys@transformshift{4.903273in}{0.696016in}%
\pgfsys@useobject{currentmarker}{}%
\end{pgfscope}%
\begin{pgfscope}%
\pgfsys@transformshift{4.948364in}{0.696016in}%
\pgfsys@useobject{currentmarker}{}%
\end{pgfscope}%
\begin{pgfscope}%
\pgfsys@transformshift{4.993455in}{0.696016in}%
\pgfsys@useobject{currentmarker}{}%
\end{pgfscope}%
\begin{pgfscope}%
\pgfsys@transformshift{5.038545in}{0.696025in}%
\pgfsys@useobject{currentmarker}{}%
\end{pgfscope}%
\begin{pgfscope}%
\pgfsys@transformshift{5.083636in}{0.696025in}%
\pgfsys@useobject{currentmarker}{}%
\end{pgfscope}%
\begin{pgfscope}%
\pgfsys@transformshift{5.128727in}{0.696025in}%
\pgfsys@useobject{currentmarker}{}%
\end{pgfscope}%
\begin{pgfscope}%
\pgfsys@transformshift{5.173818in}{0.696025in}%
\pgfsys@useobject{currentmarker}{}%
\end{pgfscope}%
\begin{pgfscope}%
\pgfsys@transformshift{5.218909in}{0.696016in}%
\pgfsys@useobject{currentmarker}{}%
\end{pgfscope}%
\begin{pgfscope}%
\pgfsys@transformshift{5.264000in}{0.696025in}%
\pgfsys@useobject{currentmarker}{}%
\end{pgfscope}%
\begin{pgfscope}%
\pgfsys@transformshift{5.309091in}{0.696006in}%
\pgfsys@useobject{currentmarker}{}%
\end{pgfscope}%
\begin{pgfscope}%
\pgfsys@transformshift{5.354182in}{0.696016in}%
\pgfsys@useobject{currentmarker}{}%
\end{pgfscope}%
\begin{pgfscope}%
\pgfsys@transformshift{5.399273in}{0.696025in}%
\pgfsys@useobject{currentmarker}{}%
\end{pgfscope}%
\begin{pgfscope}%
\pgfsys@transformshift{5.444364in}{0.696016in}%
\pgfsys@useobject{currentmarker}{}%
\end{pgfscope}%
\begin{pgfscope}%
\pgfsys@transformshift{5.489455in}{0.696016in}%
\pgfsys@useobject{currentmarker}{}%
\end{pgfscope}%
\begin{pgfscope}%
\pgfsys@transformshift{5.534545in}{0.696025in}%
\pgfsys@useobject{currentmarker}{}%
\end{pgfscope}%
\end{pgfscope}%
\begin{pgfscope}%
\pgfsetrectcap%
\pgfsetmiterjoin%
\pgfsetlinewidth{0.803000pt}%
\definecolor{currentstroke}{rgb}{0.000000,0.000000,0.000000}%
\pgfsetstrokecolor{currentstroke}%
\pgfsetdash{}{0pt}%
\pgfpathmoveto{\pgfqpoint{0.800000in}{0.528008in}}%
\pgfpathlineto{\pgfqpoint{0.800000in}{4.224004in}}%
\pgfusepath{stroke}%
\end{pgfscope}%
\begin{pgfscope}%
\pgfsetrectcap%
\pgfsetmiterjoin%
\pgfsetlinewidth{0.803000pt}%
\definecolor{currentstroke}{rgb}{0.000000,0.000000,0.000000}%
\pgfsetstrokecolor{currentstroke}%
\pgfsetdash{}{0pt}%
\pgfpathmoveto{\pgfqpoint{5.760000in}{0.528008in}}%
\pgfpathlineto{\pgfqpoint{5.760000in}{4.224004in}}%
\pgfusepath{stroke}%
\end{pgfscope}%
\begin{pgfscope}%
\pgfsetrectcap%
\pgfsetmiterjoin%
\pgfsetlinewidth{0.803000pt}%
\definecolor{currentstroke}{rgb}{0.000000,0.000000,0.000000}%
\pgfsetstrokecolor{currentstroke}%
\pgfsetdash{}{0pt}%
\pgfpathmoveto{\pgfqpoint{0.800000in}{0.528000in}}%
\pgfpathlineto{\pgfqpoint{5.760000in}{0.528000in}}%
\pgfusepath{stroke}%
\end{pgfscope}%
\begin{pgfscope}%
\pgfsetrectcap%
\pgfsetmiterjoin%
\pgfsetlinewidth{0.803000pt}%
\definecolor{currentstroke}{rgb}{0.000000,0.000000,0.000000}%
\pgfsetstrokecolor{currentstroke}%
\pgfsetdash{}{0pt}%
\pgfpathmoveto{\pgfqpoint{0.800000in}{4.224000in}}%
\pgfpathlineto{\pgfqpoint{5.760000in}{4.224000in}}%
\pgfusepath{stroke}%
\end{pgfscope}%
\end{pgfpicture}%
\makeatother%
\endgroup%

\caption{The lower bound in \cref{eq:GMRESprob} for $R=12$, $\eps = 10^{-5}$, $N = \ceil{k^{3}}$, and $\Ct=0.1,$ with $\NLiDRR{\no-\nt} \sim \Exp{\sigma}$ with $\sigma = 1/k.$\label{fig:prob-theory-plot-1.0}}
\end{figure}

\begin{figure}[h]
    \centering
    %% Creator: Matplotlib, PGF backend
%%
%% To include the figure in your LaTeX document, write
%%   \input{<filename>.pgf}
%%
%% Make sure the required packages are loaded in your preamble
%%   \usepackage{pgf}
%%
%% Figures using additional raster images can only be included by \input if
%% they are in the same directory as the main LaTeX file. For loading figures
%% from other directories you can use the `import` package
%%   \usepackage{import}
%% and then include the figures with
%%   \import{<path to file>}{<filename>.pgf}
%%
%% Matplotlib used the following preamble
%%   \usepackage{fontspec}
%%   \setmainfont{DejaVuSerif.ttf}[Path=/home/owen/progs/firedrake-complex/firedrake/lib/python3.5/site-packages/matplotlib/mpl-data/fonts/ttf/]
%%   \setsansfont{DejaVuSans.ttf}[Path=/home/owen/progs/firedrake-complex/firedrake/lib/python3.5/site-packages/matplotlib/mpl-data/fonts/ttf/]
%%   \setmonofont{DejaVuSansMono.ttf}[Path=/home/owen/progs/firedrake-complex/firedrake/lib/python3.5/site-packages/matplotlib/mpl-data/fonts/ttf/]
%%
\begingroup%
\makeatletter%
\begin{pgfpicture}%
\pgfpathrectangle{\pgfpointorigin}{\pgfqpoint{6.000000in}{2.500000in}}%
\pgfusepath{use as bounding box, clip}%
\begin{pgfscope}%
\pgfsetbuttcap%
\pgfsetmiterjoin%
\definecolor{currentfill}{rgb}{1.000000,1.000000,1.000000}%
\pgfsetfillcolor{currentfill}%
\pgfsetlinewidth{0.000000pt}%
\definecolor{currentstroke}{rgb}{1.000000,1.000000,1.000000}%
\pgfsetstrokecolor{currentstroke}%
\pgfsetdash{}{0pt}%
\pgfpathmoveto{\pgfqpoint{0.000000in}{0.000000in}}%
\pgfpathlineto{\pgfqpoint{6.000000in}{0.000000in}}%
\pgfpathlineto{\pgfqpoint{6.000000in}{2.500000in}}%
\pgfpathlineto{\pgfqpoint{0.000000in}{2.500000in}}%
\pgfpathclose%
\pgfusepath{fill}%
\end{pgfscope}%
\begin{pgfscope}%
\pgfsetbuttcap%
\pgfsetmiterjoin%
\definecolor{currentfill}{rgb}{1.000000,1.000000,1.000000}%
\pgfsetfillcolor{currentfill}%
\pgfsetlinewidth{0.000000pt}%
\definecolor{currentstroke}{rgb}{0.000000,0.000000,0.000000}%
\pgfsetstrokecolor{currentstroke}%
\pgfsetstrokeopacity{0.000000}%
\pgfsetdash{}{0pt}%
\pgfpathmoveto{\pgfqpoint{0.750000in}{0.275000in}}%
\pgfpathlineto{\pgfqpoint{5.400000in}{0.275000in}}%
\pgfpathlineto{\pgfqpoint{5.400000in}{2.200000in}}%
\pgfpathlineto{\pgfqpoint{0.750000in}{2.200000in}}%
\pgfpathclose%
\pgfusepath{fill}%
\end{pgfscope}%
\begin{pgfscope}%
\pgfsetbuttcap%
\pgfsetroundjoin%
\definecolor{currentfill}{rgb}{0.000000,0.000000,0.000000}%
\pgfsetfillcolor{currentfill}%
\pgfsetlinewidth{0.803000pt}%
\definecolor{currentstroke}{rgb}{0.000000,0.000000,0.000000}%
\pgfsetstrokecolor{currentstroke}%
\pgfsetdash{}{0pt}%
\pgfsys@defobject{currentmarker}{\pgfqpoint{0.000000in}{-0.048611in}}{\pgfqpoint{0.000000in}{0.000000in}}{%
\pgfpathmoveto{\pgfqpoint{0.000000in}{0.000000in}}%
\pgfpathlineto{\pgfqpoint{0.000000in}{-0.048611in}}%
\pgfusepath{stroke,fill}%
}%
\begin{pgfscope}%
\pgfsys@transformshift{0.961364in}{0.275000in}%
\pgfsys@useobject{currentmarker}{}%
\end{pgfscope}%
\end{pgfscope}%
\begin{pgfscope}%
\definecolor{textcolor}{rgb}{0.000000,0.000000,0.000000}%
\pgfsetstrokecolor{textcolor}%
\pgfsetfillcolor{textcolor}%
\pgftext[x=0.961364in,y=0.177778in,,top]{\color{textcolor}\sffamily\fontsize{10.000000}{12.000000}\selectfont \(\displaystyle 10\)}%
\end{pgfscope}%
\begin{pgfscope}%
\pgfsetbuttcap%
\pgfsetroundjoin%
\definecolor{currentfill}{rgb}{0.000000,0.000000,0.000000}%
\pgfsetfillcolor{currentfill}%
\pgfsetlinewidth{0.803000pt}%
\definecolor{currentstroke}{rgb}{0.000000,0.000000,0.000000}%
\pgfsetstrokecolor{currentstroke}%
\pgfsetdash{}{0pt}%
\pgfsys@defobject{currentmarker}{\pgfqpoint{0.000000in}{-0.048611in}}{\pgfqpoint{0.000000in}{0.000000in}}{%
\pgfpathmoveto{\pgfqpoint{0.000000in}{0.000000in}}%
\pgfpathlineto{\pgfqpoint{0.000000in}{-0.048611in}}%
\pgfusepath{stroke,fill}%
}%
\begin{pgfscope}%
\pgfsys@transformshift{1.665909in}{0.275000in}%
\pgfsys@useobject{currentmarker}{}%
\end{pgfscope}%
\end{pgfscope}%
\begin{pgfscope}%
\definecolor{textcolor}{rgb}{0.000000,0.000000,0.000000}%
\pgfsetstrokecolor{textcolor}%
\pgfsetfillcolor{textcolor}%
\pgftext[x=1.665909in,y=0.177778in,,top]{\color{textcolor}\sffamily\fontsize{10.000000}{12.000000}\selectfont \(\displaystyle 15\)}%
\end{pgfscope}%
\begin{pgfscope}%
\pgfsetbuttcap%
\pgfsetroundjoin%
\definecolor{currentfill}{rgb}{0.000000,0.000000,0.000000}%
\pgfsetfillcolor{currentfill}%
\pgfsetlinewidth{0.803000pt}%
\definecolor{currentstroke}{rgb}{0.000000,0.000000,0.000000}%
\pgfsetstrokecolor{currentstroke}%
\pgfsetdash{}{0pt}%
\pgfsys@defobject{currentmarker}{\pgfqpoint{0.000000in}{-0.048611in}}{\pgfqpoint{0.000000in}{0.000000in}}{%
\pgfpathmoveto{\pgfqpoint{0.000000in}{0.000000in}}%
\pgfpathlineto{\pgfqpoint{0.000000in}{-0.048611in}}%
\pgfusepath{stroke,fill}%
}%
\begin{pgfscope}%
\pgfsys@transformshift{2.370455in}{0.275000in}%
\pgfsys@useobject{currentmarker}{}%
\end{pgfscope}%
\end{pgfscope}%
\begin{pgfscope}%
\definecolor{textcolor}{rgb}{0.000000,0.000000,0.000000}%
\pgfsetstrokecolor{textcolor}%
\pgfsetfillcolor{textcolor}%
\pgftext[x=2.370455in,y=0.177778in,,top]{\color{textcolor}\sffamily\fontsize{10.000000}{12.000000}\selectfont \(\displaystyle 20\)}%
\end{pgfscope}%
\begin{pgfscope}%
\pgfsetbuttcap%
\pgfsetroundjoin%
\definecolor{currentfill}{rgb}{0.000000,0.000000,0.000000}%
\pgfsetfillcolor{currentfill}%
\pgfsetlinewidth{0.803000pt}%
\definecolor{currentstroke}{rgb}{0.000000,0.000000,0.000000}%
\pgfsetstrokecolor{currentstroke}%
\pgfsetdash{}{0pt}%
\pgfsys@defobject{currentmarker}{\pgfqpoint{0.000000in}{-0.048611in}}{\pgfqpoint{0.000000in}{0.000000in}}{%
\pgfpathmoveto{\pgfqpoint{0.000000in}{0.000000in}}%
\pgfpathlineto{\pgfqpoint{0.000000in}{-0.048611in}}%
\pgfusepath{stroke,fill}%
}%
\begin{pgfscope}%
\pgfsys@transformshift{3.075000in}{0.275000in}%
\pgfsys@useobject{currentmarker}{}%
\end{pgfscope}%
\end{pgfscope}%
\begin{pgfscope}%
\definecolor{textcolor}{rgb}{0.000000,0.000000,0.000000}%
\pgfsetstrokecolor{textcolor}%
\pgfsetfillcolor{textcolor}%
\pgftext[x=3.075000in,y=0.177778in,,top]{\color{textcolor}\sffamily\fontsize{10.000000}{12.000000}\selectfont \(\displaystyle 25\)}%
\end{pgfscope}%
\begin{pgfscope}%
\pgfsetbuttcap%
\pgfsetroundjoin%
\definecolor{currentfill}{rgb}{0.000000,0.000000,0.000000}%
\pgfsetfillcolor{currentfill}%
\pgfsetlinewidth{0.803000pt}%
\definecolor{currentstroke}{rgb}{0.000000,0.000000,0.000000}%
\pgfsetstrokecolor{currentstroke}%
\pgfsetdash{}{0pt}%
\pgfsys@defobject{currentmarker}{\pgfqpoint{0.000000in}{-0.048611in}}{\pgfqpoint{0.000000in}{0.000000in}}{%
\pgfpathmoveto{\pgfqpoint{0.000000in}{0.000000in}}%
\pgfpathlineto{\pgfqpoint{0.000000in}{-0.048611in}}%
\pgfusepath{stroke,fill}%
}%
\begin{pgfscope}%
\pgfsys@transformshift{3.779545in}{0.275000in}%
\pgfsys@useobject{currentmarker}{}%
\end{pgfscope}%
\end{pgfscope}%
\begin{pgfscope}%
\definecolor{textcolor}{rgb}{0.000000,0.000000,0.000000}%
\pgfsetstrokecolor{textcolor}%
\pgfsetfillcolor{textcolor}%
\pgftext[x=3.779545in,y=0.177778in,,top]{\color{textcolor}\sffamily\fontsize{10.000000}{12.000000}\selectfont \(\displaystyle 30\)}%
\end{pgfscope}%
\begin{pgfscope}%
\pgfsetbuttcap%
\pgfsetroundjoin%
\definecolor{currentfill}{rgb}{0.000000,0.000000,0.000000}%
\pgfsetfillcolor{currentfill}%
\pgfsetlinewidth{0.803000pt}%
\definecolor{currentstroke}{rgb}{0.000000,0.000000,0.000000}%
\pgfsetstrokecolor{currentstroke}%
\pgfsetdash{}{0pt}%
\pgfsys@defobject{currentmarker}{\pgfqpoint{0.000000in}{-0.048611in}}{\pgfqpoint{0.000000in}{0.000000in}}{%
\pgfpathmoveto{\pgfqpoint{0.000000in}{0.000000in}}%
\pgfpathlineto{\pgfqpoint{0.000000in}{-0.048611in}}%
\pgfusepath{stroke,fill}%
}%
\begin{pgfscope}%
\pgfsys@transformshift{4.484091in}{0.275000in}%
\pgfsys@useobject{currentmarker}{}%
\end{pgfscope}%
\end{pgfscope}%
\begin{pgfscope}%
\definecolor{textcolor}{rgb}{0.000000,0.000000,0.000000}%
\pgfsetstrokecolor{textcolor}%
\pgfsetfillcolor{textcolor}%
\pgftext[x=4.484091in,y=0.177778in,,top]{\color{textcolor}\sffamily\fontsize{10.000000}{12.000000}\selectfont \(\displaystyle 35\)}%
\end{pgfscope}%
\begin{pgfscope}%
\pgfsetbuttcap%
\pgfsetroundjoin%
\definecolor{currentfill}{rgb}{0.000000,0.000000,0.000000}%
\pgfsetfillcolor{currentfill}%
\pgfsetlinewidth{0.803000pt}%
\definecolor{currentstroke}{rgb}{0.000000,0.000000,0.000000}%
\pgfsetstrokecolor{currentstroke}%
\pgfsetdash{}{0pt}%
\pgfsys@defobject{currentmarker}{\pgfqpoint{0.000000in}{-0.048611in}}{\pgfqpoint{0.000000in}{0.000000in}}{%
\pgfpathmoveto{\pgfqpoint{0.000000in}{0.000000in}}%
\pgfpathlineto{\pgfqpoint{0.000000in}{-0.048611in}}%
\pgfusepath{stroke,fill}%
}%
\begin{pgfscope}%
\pgfsys@transformshift{5.188636in}{0.275000in}%
\pgfsys@useobject{currentmarker}{}%
\end{pgfscope}%
\end{pgfscope}%
\begin{pgfscope}%
\definecolor{textcolor}{rgb}{0.000000,0.000000,0.000000}%
\pgfsetstrokecolor{textcolor}%
\pgfsetfillcolor{textcolor}%
\pgftext[x=5.188636in,y=0.177778in,,top]{\color{textcolor}\sffamily\fontsize{10.000000}{12.000000}\selectfont \(\displaystyle 40\)}%
\end{pgfscope}%
\begin{pgfscope}%
\definecolor{textcolor}{rgb}{0.000000,0.000000,0.000000}%
\pgfsetstrokecolor{textcolor}%
\pgfsetfillcolor{textcolor}%
\pgftext[x=3.075000in,y=-0.012191in,,top]{\color{textcolor}\sffamily\fontsize{10.000000}{12.000000}\selectfont \(\displaystyle k\)}%
\end{pgfscope}%
\begin{pgfscope}%
\pgfsetbuttcap%
\pgfsetroundjoin%
\definecolor{currentfill}{rgb}{0.000000,0.000000,0.000000}%
\pgfsetfillcolor{currentfill}%
\pgfsetlinewidth{0.803000pt}%
\definecolor{currentstroke}{rgb}{0.000000,0.000000,0.000000}%
\pgfsetstrokecolor{currentstroke}%
\pgfsetdash{}{0pt}%
\pgfsys@defobject{currentmarker}{\pgfqpoint{-0.048611in}{0.000000in}}{\pgfqpoint{0.000000in}{0.000000in}}{%
\pgfpathmoveto{\pgfqpoint{0.000000in}{0.000000in}}%
\pgfpathlineto{\pgfqpoint{-0.048611in}{0.000000in}}%
\pgfusepath{stroke,fill}%
}%
\begin{pgfscope}%
\pgfsys@transformshift{0.750000in}{0.298219in}%
\pgfsys@useobject{currentmarker}{}%
\end{pgfscope}%
\end{pgfscope}%
\begin{pgfscope}%
\definecolor{textcolor}{rgb}{0.000000,0.000000,0.000000}%
\pgfsetstrokecolor{textcolor}%
\pgfsetfillcolor{textcolor}%
\pgftext[x=0.405863in,y=0.245457in,left,base]{\color{textcolor}\sffamily\fontsize{10.000000}{12.000000}\selectfont \(\displaystyle 0.97\)}%
\end{pgfscope}%
\begin{pgfscope}%
\pgfsetbuttcap%
\pgfsetroundjoin%
\definecolor{currentfill}{rgb}{0.000000,0.000000,0.000000}%
\pgfsetfillcolor{currentfill}%
\pgfsetlinewidth{0.803000pt}%
\definecolor{currentstroke}{rgb}{0.000000,0.000000,0.000000}%
\pgfsetstrokecolor{currentstroke}%
\pgfsetdash{}{0pt}%
\pgfsys@defobject{currentmarker}{\pgfqpoint{-0.048611in}{0.000000in}}{\pgfqpoint{0.000000in}{0.000000in}}{%
\pgfpathmoveto{\pgfqpoint{0.000000in}{0.000000in}}%
\pgfpathlineto{\pgfqpoint{-0.048611in}{0.000000in}}%
\pgfusepath{stroke,fill}%
}%
\begin{pgfscope}%
\pgfsys@transformshift{0.750000in}{0.902993in}%
\pgfsys@useobject{currentmarker}{}%
\end{pgfscope}%
\end{pgfscope}%
\begin{pgfscope}%
\definecolor{textcolor}{rgb}{0.000000,0.000000,0.000000}%
\pgfsetstrokecolor{textcolor}%
\pgfsetfillcolor{textcolor}%
\pgftext[x=0.405863in,y=0.850232in,left,base]{\color{textcolor}\sffamily\fontsize{10.000000}{12.000000}\selectfont \(\displaystyle 0.98\)}%
\end{pgfscope}%
\begin{pgfscope}%
\pgfsetbuttcap%
\pgfsetroundjoin%
\definecolor{currentfill}{rgb}{0.000000,0.000000,0.000000}%
\pgfsetfillcolor{currentfill}%
\pgfsetlinewidth{0.803000pt}%
\definecolor{currentstroke}{rgb}{0.000000,0.000000,0.000000}%
\pgfsetstrokecolor{currentstroke}%
\pgfsetdash{}{0pt}%
\pgfsys@defobject{currentmarker}{\pgfqpoint{-0.048611in}{0.000000in}}{\pgfqpoint{0.000000in}{0.000000in}}{%
\pgfpathmoveto{\pgfqpoint{0.000000in}{0.000000in}}%
\pgfpathlineto{\pgfqpoint{-0.048611in}{0.000000in}}%
\pgfusepath{stroke,fill}%
}%
\begin{pgfscope}%
\pgfsys@transformshift{0.750000in}{1.507768in}%
\pgfsys@useobject{currentmarker}{}%
\end{pgfscope}%
\end{pgfscope}%
\begin{pgfscope}%
\definecolor{textcolor}{rgb}{0.000000,0.000000,0.000000}%
\pgfsetstrokecolor{textcolor}%
\pgfsetfillcolor{textcolor}%
\pgftext[x=0.405863in,y=1.455006in,left,base]{\color{textcolor}\sffamily\fontsize{10.000000}{12.000000}\selectfont \(\displaystyle 0.99\)}%
\end{pgfscope}%
\begin{pgfscope}%
\pgfsetbuttcap%
\pgfsetroundjoin%
\definecolor{currentfill}{rgb}{0.000000,0.000000,0.000000}%
\pgfsetfillcolor{currentfill}%
\pgfsetlinewidth{0.803000pt}%
\definecolor{currentstroke}{rgb}{0.000000,0.000000,0.000000}%
\pgfsetstrokecolor{currentstroke}%
\pgfsetdash{}{0pt}%
\pgfsys@defobject{currentmarker}{\pgfqpoint{-0.048611in}{0.000000in}}{\pgfqpoint{0.000000in}{0.000000in}}{%
\pgfpathmoveto{\pgfqpoint{0.000000in}{0.000000in}}%
\pgfpathlineto{\pgfqpoint{-0.048611in}{0.000000in}}%
\pgfusepath{stroke,fill}%
}%
\begin{pgfscope}%
\pgfsys@transformshift{0.750000in}{2.112542in}%
\pgfsys@useobject{currentmarker}{}%
\end{pgfscope}%
\end{pgfscope}%
\begin{pgfscope}%
\definecolor{textcolor}{rgb}{0.000000,0.000000,0.000000}%
\pgfsetstrokecolor{textcolor}%
\pgfsetfillcolor{textcolor}%
\pgftext[x=0.405863in,y=2.059781in,left,base]{\color{textcolor}\sffamily\fontsize{10.000000}{12.000000}\selectfont \(\displaystyle 1.00\)}%
\end{pgfscope}%
\begin{pgfscope}%
\definecolor{textcolor}{rgb}{0.000000,0.000000,0.000000}%
\pgfsetstrokecolor{textcolor}%
\pgfsetfillcolor{textcolor}%
\pgftext[x=0.165901in,y=0.325194in,left,base,rotate=90.000000]{\color{textcolor}\sffamily\fontsize{10.000000}{12.000000}\selectfont Probability that number of}%
\end{pgfscope}%
\begin{pgfscope}%
\definecolor{textcolor}{rgb}{0.000000,0.000000,0.000000}%
\pgfsetstrokecolor{textcolor}%
\pgfsetfillcolor{textcolor}%
\pgftext[x=0.321418in,y=0.160467in,left,base,rotate=90.000000]{\color{textcolor}\sffamily\fontsize{10.000000}{12.000000}\selectfont GMRES iterations is at most 12}%
\end{pgfscope}%
\begin{pgfscope}%
\pgfpathrectangle{\pgfqpoint{0.750000in}{0.275000in}}{\pgfqpoint{4.650000in}{1.925000in}}%
\pgfusepath{clip}%
\pgfsetbuttcap%
\pgfsetroundjoin%
\definecolor{currentfill}{rgb}{0.000000,0.000000,0.000000}%
\pgfsetfillcolor{currentfill}%
\pgfsetlinewidth{1.003750pt}%
\definecolor{currentstroke}{rgb}{0.000000,0.000000,0.000000}%
\pgfsetstrokecolor{currentstroke}%
\pgfsetdash{}{0pt}%
\pgfsys@defobject{currentmarker}{\pgfqpoint{-0.020833in}{-0.020833in}}{\pgfqpoint{0.020833in}{0.020833in}}{%
\pgfpathmoveto{\pgfqpoint{0.000000in}{-0.020833in}}%
\pgfpathcurveto{\pgfqpoint{0.005525in}{-0.020833in}}{\pgfqpoint{0.010825in}{-0.018638in}}{\pgfqpoint{0.014731in}{-0.014731in}}%
\pgfpathcurveto{\pgfqpoint{0.018638in}{-0.010825in}}{\pgfqpoint{0.020833in}{-0.005525in}}{\pgfqpoint{0.020833in}{0.000000in}}%
\pgfpathcurveto{\pgfqpoint{0.020833in}{0.005525in}}{\pgfqpoint{0.018638in}{0.010825in}}{\pgfqpoint{0.014731in}{0.014731in}}%
\pgfpathcurveto{\pgfqpoint{0.010825in}{0.018638in}}{\pgfqpoint{0.005525in}{0.020833in}}{\pgfqpoint{0.000000in}{0.020833in}}%
\pgfpathcurveto{\pgfqpoint{-0.005525in}{0.020833in}}{\pgfqpoint{-0.010825in}{0.018638in}}{\pgfqpoint{-0.014731in}{0.014731in}}%
\pgfpathcurveto{\pgfqpoint{-0.018638in}{0.010825in}}{\pgfqpoint{-0.020833in}{0.005525in}}{\pgfqpoint{-0.020833in}{0.000000in}}%
\pgfpathcurveto{\pgfqpoint{-0.020833in}{-0.005525in}}{\pgfqpoint{-0.018638in}{-0.010825in}}{\pgfqpoint{-0.014731in}{-0.014731in}}%
\pgfpathcurveto{\pgfqpoint{-0.010825in}{-0.018638in}}{\pgfqpoint{-0.005525in}{-0.020833in}}{\pgfqpoint{0.000000in}{-0.020833in}}%
\pgfpathclose%
\pgfusepath{stroke,fill}%
}%
\begin{pgfscope}%
\pgfsys@transformshift{0.961364in}{0.362500in}%
\pgfsys@useobject{currentmarker}{}%
\end{pgfscope}%
\begin{pgfscope}%
\pgfsys@transformshift{1.003636in}{0.538950in}%
\pgfsys@useobject{currentmarker}{}%
\end{pgfscope}%
\begin{pgfscope}%
\pgfsys@transformshift{1.045909in}{0.697609in}%
\pgfsys@useobject{currentmarker}{}%
\end{pgfscope}%
\begin{pgfscope}%
\pgfsys@transformshift{1.088182in}{0.840272in}%
\pgfsys@useobject{currentmarker}{}%
\end{pgfscope}%
\begin{pgfscope}%
\pgfsys@transformshift{1.130455in}{0.968550in}%
\pgfsys@useobject{currentmarker}{}%
\end{pgfscope}%
\begin{pgfscope}%
\pgfsys@transformshift{1.172727in}{1.083894in}%
\pgfsys@useobject{currentmarker}{}%
\end{pgfscope}%
\begin{pgfscope}%
\pgfsys@transformshift{1.215000in}{1.187609in}%
\pgfsys@useobject{currentmarker}{}%
\end{pgfscope}%
\begin{pgfscope}%
\pgfsys@transformshift{1.257273in}{1.280866in}%
\pgfsys@useobject{currentmarker}{}%
\end{pgfscope}%
\begin{pgfscope}%
\pgfsys@transformshift{1.299545in}{1.364721in}%
\pgfsys@useobject{currentmarker}{}%
\end{pgfscope}%
\begin{pgfscope}%
\pgfsys@transformshift{1.341818in}{1.440121in}%
\pgfsys@useobject{currentmarker}{}%
\end{pgfscope}%
\begin{pgfscope}%
\pgfsys@transformshift{1.384091in}{1.507919in}%
\pgfsys@useobject{currentmarker}{}%
\end{pgfscope}%
\begin{pgfscope}%
\pgfsys@transformshift{1.426364in}{1.568881in}%
\pgfsys@useobject{currentmarker}{}%
\end{pgfscope}%
\begin{pgfscope}%
\pgfsys@transformshift{1.468636in}{1.623696in}%
\pgfsys@useobject{currentmarker}{}%
\end{pgfscope}%
\begin{pgfscope}%
\pgfsys@transformshift{1.510909in}{1.672985in}%
\pgfsys@useobject{currentmarker}{}%
\end{pgfscope}%
\begin{pgfscope}%
\pgfsys@transformshift{1.553182in}{1.717303in}%
\pgfsys@useobject{currentmarker}{}%
\end{pgfscope}%
\begin{pgfscope}%
\pgfsys@transformshift{1.595455in}{1.757154in}%
\pgfsys@useobject{currentmarker}{}%
\end{pgfscope}%
\begin{pgfscope}%
\pgfsys@transformshift{1.637727in}{1.792986in}%
\pgfsys@useobject{currentmarker}{}%
\end{pgfscope}%
\begin{pgfscope}%
\pgfsys@transformshift{1.680000in}{1.825206in}%
\pgfsys@useobject{currentmarker}{}%
\end{pgfscope}%
\begin{pgfscope}%
\pgfsys@transformshift{1.722273in}{1.854177in}%
\pgfsys@useobject{currentmarker}{}%
\end{pgfscope}%
\begin{pgfscope}%
\pgfsys@transformshift{1.764545in}{1.880227in}%
\pgfsys@useobject{currentmarker}{}%
\end{pgfscope}%
\begin{pgfscope}%
\pgfsys@transformshift{1.806818in}{1.903651in}%
\pgfsys@useobject{currentmarker}{}%
\end{pgfscope}%
\begin{pgfscope}%
\pgfsys@transformshift{1.849091in}{1.924712in}%
\pgfsys@useobject{currentmarker}{}%
\end{pgfscope}%
\begin{pgfscope}%
\pgfsys@transformshift{1.891364in}{1.943650in}%
\pgfsys@useobject{currentmarker}{}%
\end{pgfscope}%
\begin{pgfscope}%
\pgfsys@transformshift{1.933636in}{1.960679in}%
\pgfsys@useobject{currentmarker}{}%
\end{pgfscope}%
\begin{pgfscope}%
\pgfsys@transformshift{1.975909in}{1.975991in}%
\pgfsys@useobject{currentmarker}{}%
\end{pgfscope}%
\begin{pgfscope}%
\pgfsys@transformshift{2.018182in}{1.989759in}%
\pgfsys@useobject{currentmarker}{}%
\end{pgfscope}%
\begin{pgfscope}%
\pgfsys@transformshift{2.060455in}{2.002139in}%
\pgfsys@useobject{currentmarker}{}%
\end{pgfscope}%
\begin{pgfscope}%
\pgfsys@transformshift{2.102727in}{2.013270in}%
\pgfsys@useobject{currentmarker}{}%
\end{pgfscope}%
\begin{pgfscope}%
\pgfsys@transformshift{2.145000in}{2.023280in}%
\pgfsys@useobject{currentmarker}{}%
\end{pgfscope}%
\begin{pgfscope}%
\pgfsys@transformshift{2.187273in}{2.032280in}%
\pgfsys@useobject{currentmarker}{}%
\end{pgfscope}%
\begin{pgfscope}%
\pgfsys@transformshift{2.229545in}{2.040372in}%
\pgfsys@useobject{currentmarker}{}%
\end{pgfscope}%
\begin{pgfscope}%
\pgfsys@transformshift{2.271818in}{2.047649in}%
\pgfsys@useobject{currentmarker}{}%
\end{pgfscope}%
\begin{pgfscope}%
\pgfsys@transformshift{2.314091in}{2.054192in}%
\pgfsys@useobject{currentmarker}{}%
\end{pgfscope}%
\begin{pgfscope}%
\pgfsys@transformshift{2.356364in}{2.060075in}%
\pgfsys@useobject{currentmarker}{}%
\end{pgfscope}%
\begin{pgfscope}%
\pgfsys@transformshift{2.398636in}{2.065365in}%
\pgfsys@useobject{currentmarker}{}%
\end{pgfscope}%
\begin{pgfscope}%
\pgfsys@transformshift{2.440909in}{2.070122in}%
\pgfsys@useobject{currentmarker}{}%
\end{pgfscope}%
\begin{pgfscope}%
\pgfsys@transformshift{2.483182in}{2.074399in}%
\pgfsys@useobject{currentmarker}{}%
\end{pgfscope}%
\begin{pgfscope}%
\pgfsys@transformshift{2.525455in}{2.078245in}%
\pgfsys@useobject{currentmarker}{}%
\end{pgfscope}%
\begin{pgfscope}%
\pgfsys@transformshift{2.567727in}{2.081703in}%
\pgfsys@useobject{currentmarker}{}%
\end{pgfscope}%
\begin{pgfscope}%
\pgfsys@transformshift{2.610000in}{2.084812in}%
\pgfsys@useobject{currentmarker}{}%
\end{pgfscope}%
\begin{pgfscope}%
\pgfsys@transformshift{2.652273in}{2.087608in}%
\pgfsys@useobject{currentmarker}{}%
\end{pgfscope}%
\begin{pgfscope}%
\pgfsys@transformshift{2.694545in}{2.090122in}%
\pgfsys@useobject{currentmarker}{}%
\end{pgfscope}%
\begin{pgfscope}%
\pgfsys@transformshift{2.736818in}{2.092383in}%
\pgfsys@useobject{currentmarker}{}%
\end{pgfscope}%
\begin{pgfscope}%
\pgfsys@transformshift{2.779091in}{2.094415in}%
\pgfsys@useobject{currentmarker}{}%
\end{pgfscope}%
\begin{pgfscope}%
\pgfsys@transformshift{2.821364in}{2.096243in}%
\pgfsys@useobject{currentmarker}{}%
\end{pgfscope}%
\begin{pgfscope}%
\pgfsys@transformshift{2.863636in}{2.097886in}%
\pgfsys@useobject{currentmarker}{}%
\end{pgfscope}%
\begin{pgfscope}%
\pgfsys@transformshift{2.905909in}{2.099364in}%
\pgfsys@useobject{currentmarker}{}%
\end{pgfscope}%
\begin{pgfscope}%
\pgfsys@transformshift{2.948182in}{2.100693in}%
\pgfsys@useobject{currentmarker}{}%
\end{pgfscope}%
\begin{pgfscope}%
\pgfsys@transformshift{2.990455in}{2.101888in}%
\pgfsys@useobject{currentmarker}{}%
\end{pgfscope}%
\begin{pgfscope}%
\pgfsys@transformshift{3.032727in}{2.102962in}%
\pgfsys@useobject{currentmarker}{}%
\end{pgfscope}%
\begin{pgfscope}%
\pgfsys@transformshift{3.075000in}{2.103928in}%
\pgfsys@useobject{currentmarker}{}%
\end{pgfscope}%
\begin{pgfscope}%
\pgfsys@transformshift{3.117273in}{2.104796in}%
\pgfsys@useobject{currentmarker}{}%
\end{pgfscope}%
\begin{pgfscope}%
\pgfsys@transformshift{3.159545in}{2.105577in}%
\pgfsys@useobject{currentmarker}{}%
\end{pgfscope}%
\begin{pgfscope}%
\pgfsys@transformshift{3.201818in}{2.106280in}%
\pgfsys@useobject{currentmarker}{}%
\end{pgfscope}%
\begin{pgfscope}%
\pgfsys@transformshift{3.244091in}{2.106911in}%
\pgfsys@useobject{currentmarker}{}%
\end{pgfscope}%
\begin{pgfscope}%
\pgfsys@transformshift{3.286364in}{2.107479in}%
\pgfsys@useobject{currentmarker}{}%
\end{pgfscope}%
\begin{pgfscope}%
\pgfsys@transformshift{3.328636in}{2.107989in}%
\pgfsys@useobject{currentmarker}{}%
\end{pgfscope}%
\begin{pgfscope}%
\pgfsys@transformshift{3.370909in}{2.108449in}%
\pgfsys@useobject{currentmarker}{}%
\end{pgfscope}%
\begin{pgfscope}%
\pgfsys@transformshift{3.413182in}{2.108861in}%
\pgfsys@useobject{currentmarker}{}%
\end{pgfscope}%
\begin{pgfscope}%
\pgfsys@transformshift{3.455455in}{2.109232in}%
\pgfsys@useobject{currentmarker}{}%
\end{pgfscope}%
\begin{pgfscope}%
\pgfsys@transformshift{3.497727in}{2.109566in}%
\pgfsys@useobject{currentmarker}{}%
\end{pgfscope}%
\begin{pgfscope}%
\pgfsys@transformshift{3.540000in}{2.109866in}%
\pgfsys@useobject{currentmarker}{}%
\end{pgfscope}%
\begin{pgfscope}%
\pgfsys@transformshift{3.582273in}{2.110136in}%
\pgfsys@useobject{currentmarker}{}%
\end{pgfscope}%
\begin{pgfscope}%
\pgfsys@transformshift{3.624545in}{2.110379in}%
\pgfsys@useobject{currentmarker}{}%
\end{pgfscope}%
\begin{pgfscope}%
\pgfsys@transformshift{3.666818in}{2.110597in}%
\pgfsys@useobject{currentmarker}{}%
\end{pgfscope}%
\begin{pgfscope}%
\pgfsys@transformshift{3.709091in}{2.110793in}%
\pgfsys@useobject{currentmarker}{}%
\end{pgfscope}%
\begin{pgfscope}%
\pgfsys@transformshift{3.751364in}{2.110969in}%
\pgfsys@useobject{currentmarker}{}%
\end{pgfscope}%
\begin{pgfscope}%
\pgfsys@transformshift{3.793636in}{2.111128in}%
\pgfsys@useobject{currentmarker}{}%
\end{pgfscope}%
\begin{pgfscope}%
\pgfsys@transformshift{3.835909in}{2.111271in}%
\pgfsys@useobject{currentmarker}{}%
\end{pgfscope}%
\begin{pgfscope}%
\pgfsys@transformshift{3.878182in}{2.111399in}%
\pgfsys@useobject{currentmarker}{}%
\end{pgfscope}%
\begin{pgfscope}%
\pgfsys@transformshift{3.920455in}{2.111514in}%
\pgfsys@useobject{currentmarker}{}%
\end{pgfscope}%
\begin{pgfscope}%
\pgfsys@transformshift{3.962727in}{2.111618in}%
\pgfsys@useobject{currentmarker}{}%
\end{pgfscope}%
\begin{pgfscope}%
\pgfsys@transformshift{4.005000in}{2.111711in}%
\pgfsys@useobject{currentmarker}{}%
\end{pgfscope}%
\begin{pgfscope}%
\pgfsys@transformshift{4.047273in}{2.111795in}%
\pgfsys@useobject{currentmarker}{}%
\end{pgfscope}%
\begin{pgfscope}%
\pgfsys@transformshift{4.089545in}{2.111870in}%
\pgfsys@useobject{currentmarker}{}%
\end{pgfscope}%
\begin{pgfscope}%
\pgfsys@transformshift{4.131818in}{2.111938in}%
\pgfsys@useobject{currentmarker}{}%
\end{pgfscope}%
\begin{pgfscope}%
\pgfsys@transformshift{4.174091in}{2.111999in}%
\pgfsys@useobject{currentmarker}{}%
\end{pgfscope}%
\begin{pgfscope}%
\pgfsys@transformshift{4.216364in}{2.112054in}%
\pgfsys@useobject{currentmarker}{}%
\end{pgfscope}%
\begin{pgfscope}%
\pgfsys@transformshift{4.258636in}{2.112103in}%
\pgfsys@useobject{currentmarker}{}%
\end{pgfscope}%
\begin{pgfscope}%
\pgfsys@transformshift{4.300909in}{2.112147in}%
\pgfsys@useobject{currentmarker}{}%
\end{pgfscope}%
\begin{pgfscope}%
\pgfsys@transformshift{4.343182in}{2.112187in}%
\pgfsys@useobject{currentmarker}{}%
\end{pgfscope}%
\begin{pgfscope}%
\pgfsys@transformshift{4.385455in}{2.112223in}%
\pgfsys@useobject{currentmarker}{}%
\end{pgfscope}%
\begin{pgfscope}%
\pgfsys@transformshift{4.427727in}{2.112255in}%
\pgfsys@useobject{currentmarker}{}%
\end{pgfscope}%
\begin{pgfscope}%
\pgfsys@transformshift{4.470000in}{2.112284in}%
\pgfsys@useobject{currentmarker}{}%
\end{pgfscope}%
\begin{pgfscope}%
\pgfsys@transformshift{4.512273in}{2.112310in}%
\pgfsys@useobject{currentmarker}{}%
\end{pgfscope}%
\begin{pgfscope}%
\pgfsys@transformshift{4.554545in}{2.112334in}%
\pgfsys@useobject{currentmarker}{}%
\end{pgfscope}%
\begin{pgfscope}%
\pgfsys@transformshift{4.596818in}{2.112355in}%
\pgfsys@useobject{currentmarker}{}%
\end{pgfscope}%
\begin{pgfscope}%
\pgfsys@transformshift{4.639091in}{2.112374in}%
\pgfsys@useobject{currentmarker}{}%
\end{pgfscope}%
\begin{pgfscope}%
\pgfsys@transformshift{4.681364in}{2.112391in}%
\pgfsys@useobject{currentmarker}{}%
\end{pgfscope}%
\begin{pgfscope}%
\pgfsys@transformshift{4.723636in}{2.112406in}%
\pgfsys@useobject{currentmarker}{}%
\end{pgfscope}%
\begin{pgfscope}%
\pgfsys@transformshift{4.765909in}{2.112420in}%
\pgfsys@useobject{currentmarker}{}%
\end{pgfscope}%
\begin{pgfscope}%
\pgfsys@transformshift{4.808182in}{2.112432in}%
\pgfsys@useobject{currentmarker}{}%
\end{pgfscope}%
\begin{pgfscope}%
\pgfsys@transformshift{4.850455in}{2.112443in}%
\pgfsys@useobject{currentmarker}{}%
\end{pgfscope}%
\begin{pgfscope}%
\pgfsys@transformshift{4.892727in}{2.112453in}%
\pgfsys@useobject{currentmarker}{}%
\end{pgfscope}%
\begin{pgfscope}%
\pgfsys@transformshift{4.935000in}{2.112462in}%
\pgfsys@useobject{currentmarker}{}%
\end{pgfscope}%
\begin{pgfscope}%
\pgfsys@transformshift{4.977273in}{2.112470in}%
\pgfsys@useobject{currentmarker}{}%
\end{pgfscope}%
\begin{pgfscope}%
\pgfsys@transformshift{5.019545in}{2.112478in}%
\pgfsys@useobject{currentmarker}{}%
\end{pgfscope}%
\begin{pgfscope}%
\pgfsys@transformshift{5.061818in}{2.112484in}%
\pgfsys@useobject{currentmarker}{}%
\end{pgfscope}%
\begin{pgfscope}%
\pgfsys@transformshift{5.104091in}{2.112490in}%
\pgfsys@useobject{currentmarker}{}%
\end{pgfscope}%
\begin{pgfscope}%
\pgfsys@transformshift{5.146364in}{2.112495in}%
\pgfsys@useobject{currentmarker}{}%
\end{pgfscope}%
\begin{pgfscope}%
\pgfsys@transformshift{5.188636in}{2.112500in}%
\pgfsys@useobject{currentmarker}{}%
\end{pgfscope}%
\end{pgfscope}%
\begin{pgfscope}%
\pgfsetrectcap%
\pgfsetmiterjoin%
\pgfsetlinewidth{0.803000pt}%
\definecolor{currentstroke}{rgb}{0.000000,0.000000,0.000000}%
\pgfsetstrokecolor{currentstroke}%
\pgfsetdash{}{0pt}%
\pgfpathmoveto{\pgfqpoint{0.750000in}{0.275000in}}%
\pgfpathlineto{\pgfqpoint{0.750000in}{2.200000in}}%
\pgfusepath{stroke}%
\end{pgfscope}%
\begin{pgfscope}%
\pgfsetrectcap%
\pgfsetmiterjoin%
\pgfsetlinewidth{0.803000pt}%
\definecolor{currentstroke}{rgb}{0.000000,0.000000,0.000000}%
\pgfsetstrokecolor{currentstroke}%
\pgfsetdash{}{0pt}%
\pgfpathmoveto{\pgfqpoint{5.400000in}{0.275000in}}%
\pgfpathlineto{\pgfqpoint{5.400000in}{2.200000in}}%
\pgfusepath{stroke}%
\end{pgfscope}%
\begin{pgfscope}%
\pgfsetrectcap%
\pgfsetmiterjoin%
\pgfsetlinewidth{0.803000pt}%
\definecolor{currentstroke}{rgb}{0.000000,0.000000,0.000000}%
\pgfsetstrokecolor{currentstroke}%
\pgfsetdash{}{0pt}%
\pgfpathmoveto{\pgfqpoint{0.750000in}{0.275000in}}%
\pgfpathlineto{\pgfqpoint{5.400000in}{0.275000in}}%
\pgfusepath{stroke}%
\end{pgfscope}%
\begin{pgfscope}%
\pgfsetrectcap%
\pgfsetmiterjoin%
\pgfsetlinewidth{0.803000pt}%
\definecolor{currentstroke}{rgb}{0.000000,0.000000,0.000000}%
\pgfsetstrokecolor{currentstroke}%
\pgfsetdash{}{0pt}%
\pgfpathmoveto{\pgfqpoint{0.750000in}{2.200000in}}%
\pgfpathlineto{\pgfqpoint{5.400000in}{2.200000in}}%
\pgfusepath{stroke}%
\end{pgfscope}%
\end{pgfpicture}%
\makeatother%
\endgroup%

    \caption{The lower bound in \cref{eq:GMRESprob} for $R=12$, $\eps = 10^{-5}$, $N = \ceil{k^{3}}$, and $\Ct=0.1,$ with $\NLiDRR{\no-\nt} \sim \Exp{\sigma}$ with $\sigma = 1/k^2.$\label{fig:prob-theory-plot-2.0}}
\end{figure}
\begin{figure}[h]
    \centering
%% Creator: Matplotlib, PGF backend
%%
%% To include the figure in your LaTeX document, write
%%   \input{<filename>.pgf}
%%
%% Make sure the required packages are loaded in your preamble
%%   \usepackage{pgf}
%%
%% Figures using additional raster images can only be included by \input if
%% they are in the same directory as the main LaTeX file. For loading figures
%% from other directories you can use the `import` package
%%   \usepackage{import}
%% and then include the figures with
%%   \import{<path to file>}{<filename>.pgf}
%%
%% Matplotlib used the following preamble
%%   \usepackage{mleftright}
%%   \usepackage{fontspec}
%%   \setmainfont{DejaVuSerif.ttf}[Path=/home/owen/progs/firedrake-complex/firedrake/lib/python3.5/site-packages/matplotlib/mpl-data/fonts/ttf/]
%%   \setsansfont{DejaVuSans.ttf}[Path=/home/owen/progs/firedrake-complex/firedrake/lib/python3.5/site-packages/matplotlib/mpl-data/fonts/ttf/]
%%   \setmonofont{DejaVuSansMono.ttf}[Path=/home/owen/progs/firedrake-complex/firedrake/lib/python3.5/site-packages/matplotlib/mpl-data/fonts/ttf/]
%%
\begingroup%
\makeatletter%
\begin{pgfpicture}%
\pgfpathrectangle{\pgfpointorigin}{\pgfqpoint{5.570218in}{2.581603in}}%
\pgfusepath{use as bounding box, clip}%
\begin{pgfscope}%
\pgfsetbuttcap%
\pgfsetmiterjoin%
\definecolor{currentfill}{rgb}{1.000000,1.000000,1.000000}%
\pgfsetfillcolor{currentfill}%
\pgfsetlinewidth{0.000000pt}%
\definecolor{currentstroke}{rgb}{1.000000,1.000000,1.000000}%
\pgfsetstrokecolor{currentstroke}%
\pgfsetdash{}{0pt}%
\pgfpathmoveto{\pgfqpoint{0.000000in}{0.000000in}}%
\pgfpathlineto{\pgfqpoint{5.570218in}{0.000000in}}%
\pgfpathlineto{\pgfqpoint{5.570218in}{2.581603in}}%
\pgfpathlineto{\pgfqpoint{0.000000in}{2.581603in}}%
\pgfpathclose%
\pgfusepath{fill}%
\end{pgfscope}%
\begin{pgfscope}%
\pgfsetbuttcap%
\pgfsetmiterjoin%
\definecolor{currentfill}{rgb}{1.000000,1.000000,1.000000}%
\pgfsetfillcolor{currentfill}%
\pgfsetlinewidth{0.000000pt}%
\definecolor{currentstroke}{rgb}{0.000000,0.000000,0.000000}%
\pgfsetstrokecolor{currentstroke}%
\pgfsetstrokeopacity{0.000000}%
\pgfsetdash{}{0pt}%
\pgfpathmoveto{\pgfqpoint{0.785218in}{0.521603in}}%
\pgfpathlineto{\pgfqpoint{5.435218in}{0.521603in}}%
\pgfpathlineto{\pgfqpoint{5.435218in}{2.446603in}}%
\pgfpathlineto{\pgfqpoint{0.785218in}{2.446603in}}%
\pgfpathclose%
\pgfusepath{fill}%
\end{pgfscope}%
\begin{pgfscope}%
\pgfsetbuttcap%
\pgfsetroundjoin%
\definecolor{currentfill}{rgb}{0.000000,0.000000,0.000000}%
\pgfsetfillcolor{currentfill}%
\pgfsetlinewidth{0.803000pt}%
\definecolor{currentstroke}{rgb}{0.000000,0.000000,0.000000}%
\pgfsetstrokecolor{currentstroke}%
\pgfsetdash{}{0pt}%
\pgfsys@defobject{currentmarker}{\pgfqpoint{0.000000in}{-0.048611in}}{\pgfqpoint{0.000000in}{0.000000in}}{%
\pgfpathmoveto{\pgfqpoint{0.000000in}{0.000000in}}%
\pgfpathlineto{\pgfqpoint{0.000000in}{-0.048611in}}%
\pgfusepath{stroke,fill}%
}%
\begin{pgfscope}%
\pgfsys@transformshift{0.996582in}{0.521603in}%
\pgfsys@useobject{currentmarker}{}%
\end{pgfscope}%
\end{pgfscope}%
\begin{pgfscope}%
\definecolor{textcolor}{rgb}{0.000000,0.000000,0.000000}%
\pgfsetstrokecolor{textcolor}%
\pgfsetfillcolor{textcolor}%
\pgftext[x=0.996582in,y=0.424381in,,top]{\color{textcolor}\sffamily\fontsize{10.000000}{12.000000}\selectfont \(\displaystyle 10\)}%
\end{pgfscope}%
\begin{pgfscope}%
\pgfsetbuttcap%
\pgfsetroundjoin%
\definecolor{currentfill}{rgb}{0.000000,0.000000,0.000000}%
\pgfsetfillcolor{currentfill}%
\pgfsetlinewidth{0.803000pt}%
\definecolor{currentstroke}{rgb}{0.000000,0.000000,0.000000}%
\pgfsetstrokecolor{currentstroke}%
\pgfsetdash{}{0pt}%
\pgfsys@defobject{currentmarker}{\pgfqpoint{0.000000in}{-0.048611in}}{\pgfqpoint{0.000000in}{0.000000in}}{%
\pgfpathmoveto{\pgfqpoint{0.000000in}{0.000000in}}%
\pgfpathlineto{\pgfqpoint{0.000000in}{-0.048611in}}%
\pgfusepath{stroke,fill}%
}%
\begin{pgfscope}%
\pgfsys@transformshift{2.405673in}{0.521603in}%
\pgfsys@useobject{currentmarker}{}%
\end{pgfscope}%
\end{pgfscope}%
\begin{pgfscope}%
\definecolor{textcolor}{rgb}{0.000000,0.000000,0.000000}%
\pgfsetstrokecolor{textcolor}%
\pgfsetfillcolor{textcolor}%
\pgftext[x=2.405673in,y=0.424381in,,top]{\color{textcolor}\sffamily\fontsize{10.000000}{12.000000}\selectfont \(\displaystyle 20\)}%
\end{pgfscope}%
\begin{pgfscope}%
\pgfsetbuttcap%
\pgfsetroundjoin%
\definecolor{currentfill}{rgb}{0.000000,0.000000,0.000000}%
\pgfsetfillcolor{currentfill}%
\pgfsetlinewidth{0.803000pt}%
\definecolor{currentstroke}{rgb}{0.000000,0.000000,0.000000}%
\pgfsetstrokecolor{currentstroke}%
\pgfsetdash{}{0pt}%
\pgfsys@defobject{currentmarker}{\pgfqpoint{0.000000in}{-0.048611in}}{\pgfqpoint{0.000000in}{0.000000in}}{%
\pgfpathmoveto{\pgfqpoint{0.000000in}{0.000000in}}%
\pgfpathlineto{\pgfqpoint{0.000000in}{-0.048611in}}%
\pgfusepath{stroke,fill}%
}%
\begin{pgfscope}%
\pgfsys@transformshift{3.814763in}{0.521603in}%
\pgfsys@useobject{currentmarker}{}%
\end{pgfscope}%
\end{pgfscope}%
\begin{pgfscope}%
\definecolor{textcolor}{rgb}{0.000000,0.000000,0.000000}%
\pgfsetstrokecolor{textcolor}%
\pgfsetfillcolor{textcolor}%
\pgftext[x=3.814763in,y=0.424381in,,top]{\color{textcolor}\sffamily\fontsize{10.000000}{12.000000}\selectfont \(\displaystyle 30\)}%
\end{pgfscope}%
\begin{pgfscope}%
\pgfsetbuttcap%
\pgfsetroundjoin%
\definecolor{currentfill}{rgb}{0.000000,0.000000,0.000000}%
\pgfsetfillcolor{currentfill}%
\pgfsetlinewidth{0.803000pt}%
\definecolor{currentstroke}{rgb}{0.000000,0.000000,0.000000}%
\pgfsetstrokecolor{currentstroke}%
\pgfsetdash{}{0pt}%
\pgfsys@defobject{currentmarker}{\pgfqpoint{0.000000in}{-0.048611in}}{\pgfqpoint{0.000000in}{0.000000in}}{%
\pgfpathmoveto{\pgfqpoint{0.000000in}{0.000000in}}%
\pgfpathlineto{\pgfqpoint{0.000000in}{-0.048611in}}%
\pgfusepath{stroke,fill}%
}%
\begin{pgfscope}%
\pgfsys@transformshift{5.223854in}{0.521603in}%
\pgfsys@useobject{currentmarker}{}%
\end{pgfscope}%
\end{pgfscope}%
\begin{pgfscope}%
\definecolor{textcolor}{rgb}{0.000000,0.000000,0.000000}%
\pgfsetstrokecolor{textcolor}%
\pgfsetfillcolor{textcolor}%
\pgftext[x=5.223854in,y=0.424381in,,top]{\color{textcolor}\sffamily\fontsize{10.000000}{12.000000}\selectfont \(\displaystyle 40\)}%
\end{pgfscope}%
\begin{pgfscope}%
\definecolor{textcolor}{rgb}{0.000000,0.000000,0.000000}%
\pgfsetstrokecolor{textcolor}%
\pgfsetfillcolor{textcolor}%
\pgftext[x=3.110218in,y=0.234413in,,top]{\color{textcolor}\sffamily\fontsize{10.000000}{12.000000}\selectfont \(\displaystyle k\)}%
\end{pgfscope}%
\begin{pgfscope}%
\pgfsetbuttcap%
\pgfsetroundjoin%
\definecolor{currentfill}{rgb}{0.000000,0.000000,0.000000}%
\pgfsetfillcolor{currentfill}%
\pgfsetlinewidth{0.803000pt}%
\definecolor{currentstroke}{rgb}{0.000000,0.000000,0.000000}%
\pgfsetstrokecolor{currentstroke}%
\pgfsetdash{}{0pt}%
\pgfsys@defobject{currentmarker}{\pgfqpoint{-0.048611in}{0.000000in}}{\pgfqpoint{0.000000in}{0.000000in}}{%
\pgfpathmoveto{\pgfqpoint{0.000000in}{0.000000in}}%
\pgfpathlineto{\pgfqpoint{-0.048611in}{0.000000in}}%
\pgfusepath{stroke,fill}%
}%
\begin{pgfscope}%
\pgfsys@transformshift{0.785218in}{0.609103in}%
\pgfsys@useobject{currentmarker}{}%
\end{pgfscope}%
\end{pgfscope}%
\begin{pgfscope}%
\definecolor{textcolor}{rgb}{0.000000,0.000000,0.000000}%
\pgfsetstrokecolor{textcolor}%
\pgfsetfillcolor{textcolor}%
\pgftext[x=0.510526in,y=0.556342in,left,base]{\color{textcolor}\sffamily\fontsize{10.000000}{12.000000}\selectfont \(\displaystyle 0.0\)}%
\end{pgfscope}%
\begin{pgfscope}%
\pgfsetbuttcap%
\pgfsetroundjoin%
\definecolor{currentfill}{rgb}{0.000000,0.000000,0.000000}%
\pgfsetfillcolor{currentfill}%
\pgfsetlinewidth{0.803000pt}%
\definecolor{currentstroke}{rgb}{0.000000,0.000000,0.000000}%
\pgfsetstrokecolor{currentstroke}%
\pgfsetdash{}{0pt}%
\pgfsys@defobject{currentmarker}{\pgfqpoint{-0.048611in}{0.000000in}}{\pgfqpoint{0.000000in}{0.000000in}}{%
\pgfpathmoveto{\pgfqpoint{0.000000in}{0.000000in}}%
\pgfpathlineto{\pgfqpoint{-0.048611in}{0.000000in}}%
\pgfusepath{stroke,fill}%
}%
\begin{pgfscope}%
\pgfsys@transformshift{0.785218in}{1.035933in}%
\pgfsys@useobject{currentmarker}{}%
\end{pgfscope}%
\end{pgfscope}%
\begin{pgfscope}%
\definecolor{textcolor}{rgb}{0.000000,0.000000,0.000000}%
\pgfsetstrokecolor{textcolor}%
\pgfsetfillcolor{textcolor}%
\pgftext[x=0.510526in,y=0.983171in,left,base]{\color{textcolor}\sffamily\fontsize{10.000000}{12.000000}\selectfont \(\displaystyle 0.1\)}%
\end{pgfscope}%
\begin{pgfscope}%
\pgfsetbuttcap%
\pgfsetroundjoin%
\definecolor{currentfill}{rgb}{0.000000,0.000000,0.000000}%
\pgfsetfillcolor{currentfill}%
\pgfsetlinewidth{0.803000pt}%
\definecolor{currentstroke}{rgb}{0.000000,0.000000,0.000000}%
\pgfsetstrokecolor{currentstroke}%
\pgfsetdash{}{0pt}%
\pgfsys@defobject{currentmarker}{\pgfqpoint{-0.048611in}{0.000000in}}{\pgfqpoint{0.000000in}{0.000000in}}{%
\pgfpathmoveto{\pgfqpoint{0.000000in}{0.000000in}}%
\pgfpathlineto{\pgfqpoint{-0.048611in}{0.000000in}}%
\pgfusepath{stroke,fill}%
}%
\begin{pgfscope}%
\pgfsys@transformshift{0.785218in}{1.462762in}%
\pgfsys@useobject{currentmarker}{}%
\end{pgfscope}%
\end{pgfscope}%
\begin{pgfscope}%
\definecolor{textcolor}{rgb}{0.000000,0.000000,0.000000}%
\pgfsetstrokecolor{textcolor}%
\pgfsetfillcolor{textcolor}%
\pgftext[x=0.510526in,y=1.410000in,left,base]{\color{textcolor}\sffamily\fontsize{10.000000}{12.000000}\selectfont \(\displaystyle 0.2\)}%
\end{pgfscope}%
\begin{pgfscope}%
\pgfsetbuttcap%
\pgfsetroundjoin%
\definecolor{currentfill}{rgb}{0.000000,0.000000,0.000000}%
\pgfsetfillcolor{currentfill}%
\pgfsetlinewidth{0.803000pt}%
\definecolor{currentstroke}{rgb}{0.000000,0.000000,0.000000}%
\pgfsetstrokecolor{currentstroke}%
\pgfsetdash{}{0pt}%
\pgfsys@defobject{currentmarker}{\pgfqpoint{-0.048611in}{0.000000in}}{\pgfqpoint{0.000000in}{0.000000in}}{%
\pgfpathmoveto{\pgfqpoint{0.000000in}{0.000000in}}%
\pgfpathlineto{\pgfqpoint{-0.048611in}{0.000000in}}%
\pgfusepath{stroke,fill}%
}%
\begin{pgfscope}%
\pgfsys@transformshift{0.785218in}{1.889591in}%
\pgfsys@useobject{currentmarker}{}%
\end{pgfscope}%
\end{pgfscope}%
\begin{pgfscope}%
\definecolor{textcolor}{rgb}{0.000000,0.000000,0.000000}%
\pgfsetstrokecolor{textcolor}%
\pgfsetfillcolor{textcolor}%
\pgftext[x=0.510526in,y=1.836830in,left,base]{\color{textcolor}\sffamily\fontsize{10.000000}{12.000000}\selectfont \(\displaystyle 0.3\)}%
\end{pgfscope}%
\begin{pgfscope}%
\pgfsetbuttcap%
\pgfsetroundjoin%
\definecolor{currentfill}{rgb}{0.000000,0.000000,0.000000}%
\pgfsetfillcolor{currentfill}%
\pgfsetlinewidth{0.803000pt}%
\definecolor{currentstroke}{rgb}{0.000000,0.000000,0.000000}%
\pgfsetstrokecolor{currentstroke}%
\pgfsetdash{}{0pt}%
\pgfsys@defobject{currentmarker}{\pgfqpoint{-0.048611in}{0.000000in}}{\pgfqpoint{0.000000in}{0.000000in}}{%
\pgfpathmoveto{\pgfqpoint{0.000000in}{0.000000in}}%
\pgfpathlineto{\pgfqpoint{-0.048611in}{0.000000in}}%
\pgfusepath{stroke,fill}%
}%
\begin{pgfscope}%
\pgfsys@transformshift{0.785218in}{2.316420in}%
\pgfsys@useobject{currentmarker}{}%
\end{pgfscope}%
\end{pgfscope}%
\begin{pgfscope}%
\definecolor{textcolor}{rgb}{0.000000,0.000000,0.000000}%
\pgfsetstrokecolor{textcolor}%
\pgfsetfillcolor{textcolor}%
\pgftext[x=0.510526in,y=2.263659in,left,base]{\color{textcolor}\sffamily\fontsize{10.000000}{12.000000}\selectfont \(\displaystyle 0.4\)}%
\end{pgfscope}%
\begin{pgfscope}%
\definecolor{textcolor}{rgb}{0.000000,0.000000,0.000000}%
\pgfsetstrokecolor{textcolor}%
\pgfsetfillcolor{textcolor}%
\pgftext[x=0.144134in,y=0.605638in,left,base,rotate=90.000000]{\color{textcolor}\sffamily\fontsize{10.000000}{12.000000}\selectfont Empirical probability that}%
\end{pgfscope}%
\begin{pgfscope}%
\definecolor{textcolor}{rgb}{0.000000,0.000000,0.000000}%
\pgfsetstrokecolor{textcolor}%
\pgfsetfillcolor{textcolor}%
\pgftext[x=0.364691in,y=0.670455in,left,base,rotate=90.000000]{\color{textcolor}\sffamily\fontsize{10.000000}{12.000000}\selectfont \(\displaystyle \mathrm{GMRES}\mleft(\epsilon, n^{(1)} n^{(2)}\mright) \leq 12\)}%
\end{pgfscope}%
\begin{pgfscope}%
\pgfpathrectangle{\pgfqpoint{0.785218in}{0.521603in}}{\pgfqpoint{4.650000in}{1.925000in}}%
\pgfusepath{clip}%
\pgfsetbuttcap%
\pgfsetroundjoin%
\pgfsetlinewidth{1.505625pt}%
\definecolor{currentstroke}{rgb}{0.000000,0.000000,0.000000}%
\pgfsetstrokecolor{currentstroke}%
\pgfsetdash{{5.550000pt}{2.400000pt}}{0.000000pt}%
\pgfpathmoveto{\pgfqpoint{0.996582in}{2.359103in}}%
\pgfpathlineto{\pgfqpoint{2.405673in}{0.732884in}}%
\pgfpathlineto{\pgfqpoint{3.814763in}{0.609103in}}%
\pgfpathlineto{\pgfqpoint{5.223854in}{0.609103in}}%
\pgfusepath{stroke}%
\end{pgfscope}%
\begin{pgfscope}%
\pgfpathrectangle{\pgfqpoint{0.785218in}{0.521603in}}{\pgfqpoint{4.650000in}{1.925000in}}%
\pgfusepath{clip}%
\pgfsetbuttcap%
\pgfsetroundjoin%
\definecolor{currentfill}{rgb}{0.000000,0.000000,0.000000}%
\pgfsetfillcolor{currentfill}%
\pgfsetlinewidth{1.003750pt}%
\definecolor{currentstroke}{rgb}{0.000000,0.000000,0.000000}%
\pgfsetstrokecolor{currentstroke}%
\pgfsetdash{}{0pt}%
\pgfsys@defobject{currentmarker}{\pgfqpoint{-0.041667in}{-0.041667in}}{\pgfqpoint{0.041667in}{0.041667in}}{%
\pgfpathmoveto{\pgfqpoint{0.000000in}{-0.041667in}}%
\pgfpathcurveto{\pgfqpoint{0.011050in}{-0.041667in}}{\pgfqpoint{0.021649in}{-0.037276in}}{\pgfqpoint{0.029463in}{-0.029463in}}%
\pgfpathcurveto{\pgfqpoint{0.037276in}{-0.021649in}}{\pgfqpoint{0.041667in}{-0.011050in}}{\pgfqpoint{0.041667in}{0.000000in}}%
\pgfpathcurveto{\pgfqpoint{0.041667in}{0.011050in}}{\pgfqpoint{0.037276in}{0.021649in}}{\pgfqpoint{0.029463in}{0.029463in}}%
\pgfpathcurveto{\pgfqpoint{0.021649in}{0.037276in}}{\pgfqpoint{0.011050in}{0.041667in}}{\pgfqpoint{0.000000in}{0.041667in}}%
\pgfpathcurveto{\pgfqpoint{-0.011050in}{0.041667in}}{\pgfqpoint{-0.021649in}{0.037276in}}{\pgfqpoint{-0.029463in}{0.029463in}}%
\pgfpathcurveto{\pgfqpoint{-0.037276in}{0.021649in}}{\pgfqpoint{-0.041667in}{0.011050in}}{\pgfqpoint{-0.041667in}{0.000000in}}%
\pgfpathcurveto{\pgfqpoint{-0.041667in}{-0.011050in}}{\pgfqpoint{-0.037276in}{-0.021649in}}{\pgfqpoint{-0.029463in}{-0.029463in}}%
\pgfpathcurveto{\pgfqpoint{-0.021649in}{-0.037276in}}{\pgfqpoint{-0.011050in}{-0.041667in}}{\pgfqpoint{0.000000in}{-0.041667in}}%
\pgfpathclose%
\pgfusepath{stroke,fill}%
}%
\begin{pgfscope}%
\pgfsys@transformshift{0.996582in}{2.359103in}%
\pgfsys@useobject{currentmarker}{}%
\end{pgfscope}%
\begin{pgfscope}%
\pgfsys@transformshift{2.405673in}{0.732884in}%
\pgfsys@useobject{currentmarker}{}%
\end{pgfscope}%
\begin{pgfscope}%
\pgfsys@transformshift{3.814763in}{0.609103in}%
\pgfsys@useobject{currentmarker}{}%
\end{pgfscope}%
\begin{pgfscope}%
\pgfsys@transformshift{5.223854in}{0.609103in}%
\pgfsys@useobject{currentmarker}{}%
\end{pgfscope}%
\end{pgfscope}%
\begin{pgfscope}%
\pgfsetrectcap%
\pgfsetmiterjoin%
\pgfsetlinewidth{0.803000pt}%
\definecolor{currentstroke}{rgb}{0.000000,0.000000,0.000000}%
\pgfsetstrokecolor{currentstroke}%
\pgfsetdash{}{0pt}%
\pgfpathmoveto{\pgfqpoint{0.785218in}{0.521603in}}%
\pgfpathlineto{\pgfqpoint{0.785218in}{2.446603in}}%
\pgfusepath{stroke}%
\end{pgfscope}%
\begin{pgfscope}%
\pgfsetrectcap%
\pgfsetmiterjoin%
\pgfsetlinewidth{0.000000pt}%
\definecolor{currentstroke}{rgb}{0.000000,0.000000,0.000000}%
\pgfsetstrokecolor{currentstroke}%
\pgfsetstrokeopacity{0.000000}%
\pgfsetdash{}{0pt}%
\pgfpathmoveto{\pgfqpoint{5.435218in}{0.521603in}}%
\pgfpathlineto{\pgfqpoint{5.435218in}{2.446603in}}%
\pgfusepath{}%
\end{pgfscope}%
\begin{pgfscope}%
\pgfsetrectcap%
\pgfsetmiterjoin%
\pgfsetlinewidth{0.803000pt}%
\definecolor{currentstroke}{rgb}{0.000000,0.000000,0.000000}%
\pgfsetstrokecolor{currentstroke}%
\pgfsetdash{}{0pt}%
\pgfpathmoveto{\pgfqpoint{0.785218in}{0.521603in}}%
\pgfpathlineto{\pgfqpoint{5.435218in}{0.521603in}}%
\pgfusepath{stroke}%
\end{pgfscope}%
\begin{pgfscope}%
\pgfsetrectcap%
\pgfsetmiterjoin%
\pgfsetlinewidth{0.000000pt}%
\definecolor{currentstroke}{rgb}{0.000000,0.000000,0.000000}%
\pgfsetstrokecolor{currentstroke}%
\pgfsetstrokeopacity{0.000000}%
\pgfsetdash{}{0pt}%
\pgfpathmoveto{\pgfqpoint{0.785218in}{2.446603in}}%
\pgfpathlineto{\pgfqpoint{5.435218in}{2.446603in}}%
\pgfusepath{}%
\end{pgfscope}%
\end{pgfpicture}%
\makeatother%
\endgroup%

\caption{Empricial probability (calculated from 1000 realisations) that $\GMRES{\eps}{\no}{\nt}\leq 12$ for $k = 10, 20, 30, 40,$ where $R=12$, $\eps = 10^{-5}$, $N = \ceil{k^{3}}$, and $\Ct=0.1,$ with $\NLiDRR{\no-\nt} \sim \Exp{\sigma}$ with $\sigma = 1.$\label{fig:prob-plot-0.0}}
\end{figure}

\begin{figure}[h]
    \centering
%% Creator: Matplotlib, PGF backend
%%
%% To include the figure in your LaTeX document, write
%%   \input{<filename>.pgf}
%%
%% Make sure the required packages are loaded in your preamble
%%   \usepackage{pgf}
%%
%% Figures using additional raster images can only be included by \input if
%% they are in the same directory as the main LaTeX file. For loading figures
%% from other directories you can use the `import` package
%%   \usepackage{import}
%% and then include the figures with
%%   \import{<path to file>}{<filename>.pgf}
%%
%% Matplotlib used the following preamble
%%   \usepackage{fontspec}
%%   \setmainfont{DejaVuSerif.ttf}[Path=/home/owen/progs/firedrake-complex/firedrake/lib/python3.5/site-packages/matplotlib/mpl-data/fonts/ttf/]
%%   \setsansfont{DejaVuSans.ttf}[Path=/home/owen/progs/firedrake-complex/firedrake/lib/python3.5/site-packages/matplotlib/mpl-data/fonts/ttf/]
%%   \setmonofont{DejaVuSansMono.ttf}[Path=/home/owen/progs/firedrake-complex/firedrake/lib/python3.5/site-packages/matplotlib/mpl-data/fonts/ttf/]
%%
\begingroup%
\makeatletter%
\begin{pgfpicture}%
\pgfpathrectangle{\pgfpointorigin}{\pgfqpoint{6.400000in}{4.800000in}}%
\pgfusepath{use as bounding box, clip}%
\begin{pgfscope}%
\pgfsetbuttcap%
\pgfsetmiterjoin%
\definecolor{currentfill}{rgb}{1.000000,1.000000,1.000000}%
\pgfsetfillcolor{currentfill}%
\pgfsetlinewidth{0.000000pt}%
\definecolor{currentstroke}{rgb}{1.000000,1.000000,1.000000}%
\pgfsetstrokecolor{currentstroke}%
\pgfsetdash{}{0pt}%
\pgfpathmoveto{\pgfqpoint{0.000000in}{0.000000in}}%
\pgfpathlineto{\pgfqpoint{6.400000in}{0.000000in}}%
\pgfpathlineto{\pgfqpoint{6.400000in}{4.800000in}}%
\pgfpathlineto{\pgfqpoint{0.000000in}{4.800000in}}%
\pgfpathclose%
\pgfusepath{fill}%
\end{pgfscope}%
\begin{pgfscope}%
\pgfsetbuttcap%
\pgfsetmiterjoin%
\definecolor{currentfill}{rgb}{1.000000,1.000000,1.000000}%
\pgfsetfillcolor{currentfill}%
\pgfsetlinewidth{0.000000pt}%
\definecolor{currentstroke}{rgb}{0.000000,0.000000,0.000000}%
\pgfsetstrokecolor{currentstroke}%
\pgfsetstrokeopacity{0.000000}%
\pgfsetdash{}{0pt}%
\pgfpathmoveto{\pgfqpoint{0.800000in}{0.528000in}}%
\pgfpathlineto{\pgfqpoint{5.760000in}{0.528000in}}%
\pgfpathlineto{\pgfqpoint{5.760000in}{4.224000in}}%
\pgfpathlineto{\pgfqpoint{0.800000in}{4.224000in}}%
\pgfpathclose%
\pgfusepath{fill}%
\end{pgfscope}%
\begin{pgfscope}%
\pgfsetbuttcap%
\pgfsetroundjoin%
\definecolor{currentfill}{rgb}{0.000000,0.000000,0.000000}%
\pgfsetfillcolor{currentfill}%
\pgfsetlinewidth{0.803000pt}%
\definecolor{currentstroke}{rgb}{0.000000,0.000000,0.000000}%
\pgfsetstrokecolor{currentstroke}%
\pgfsetdash{}{0pt}%
\pgfsys@defobject{currentmarker}{\pgfqpoint{0.000000in}{-0.048611in}}{\pgfqpoint{0.000000in}{0.000000in}}{%
\pgfpathmoveto{\pgfqpoint{0.000000in}{0.000000in}}%
\pgfpathlineto{\pgfqpoint{0.000000in}{-0.048611in}}%
\pgfusepath{stroke,fill}%
}%
\begin{pgfscope}%
\pgfsys@transformshift{1.025455in}{0.528000in}%
\pgfsys@useobject{currentmarker}{}%
\end{pgfscope}%
\end{pgfscope}%
\begin{pgfscope}%
\definecolor{textcolor}{rgb}{0.000000,0.000000,0.000000}%
\pgfsetstrokecolor{textcolor}%
\pgfsetfillcolor{textcolor}%
\pgftext[x=1.025455in,y=0.430778in,,top]{\color{textcolor}\sffamily\fontsize{10.000000}{12.000000}\selectfont 10}%
\end{pgfscope}%
\begin{pgfscope}%
\pgfsetbuttcap%
\pgfsetroundjoin%
\definecolor{currentfill}{rgb}{0.000000,0.000000,0.000000}%
\pgfsetfillcolor{currentfill}%
\pgfsetlinewidth{0.803000pt}%
\definecolor{currentstroke}{rgb}{0.000000,0.000000,0.000000}%
\pgfsetstrokecolor{currentstroke}%
\pgfsetdash{}{0pt}%
\pgfsys@defobject{currentmarker}{\pgfqpoint{0.000000in}{-0.048611in}}{\pgfqpoint{0.000000in}{0.000000in}}{%
\pgfpathmoveto{\pgfqpoint{0.000000in}{0.000000in}}%
\pgfpathlineto{\pgfqpoint{0.000000in}{-0.048611in}}%
\pgfusepath{stroke,fill}%
}%
\begin{pgfscope}%
\pgfsys@transformshift{2.528485in}{0.528000in}%
\pgfsys@useobject{currentmarker}{}%
\end{pgfscope}%
\end{pgfscope}%
\begin{pgfscope}%
\definecolor{textcolor}{rgb}{0.000000,0.000000,0.000000}%
\pgfsetstrokecolor{textcolor}%
\pgfsetfillcolor{textcolor}%
\pgftext[x=2.528485in,y=0.430778in,,top]{\color{textcolor}\sffamily\fontsize{10.000000}{12.000000}\selectfont 20}%
\end{pgfscope}%
\begin{pgfscope}%
\pgfsetbuttcap%
\pgfsetroundjoin%
\definecolor{currentfill}{rgb}{0.000000,0.000000,0.000000}%
\pgfsetfillcolor{currentfill}%
\pgfsetlinewidth{0.803000pt}%
\definecolor{currentstroke}{rgb}{0.000000,0.000000,0.000000}%
\pgfsetstrokecolor{currentstroke}%
\pgfsetdash{}{0pt}%
\pgfsys@defobject{currentmarker}{\pgfqpoint{0.000000in}{-0.048611in}}{\pgfqpoint{0.000000in}{0.000000in}}{%
\pgfpathmoveto{\pgfqpoint{0.000000in}{0.000000in}}%
\pgfpathlineto{\pgfqpoint{0.000000in}{-0.048611in}}%
\pgfusepath{stroke,fill}%
}%
\begin{pgfscope}%
\pgfsys@transformshift{4.031515in}{0.528000in}%
\pgfsys@useobject{currentmarker}{}%
\end{pgfscope}%
\end{pgfscope}%
\begin{pgfscope}%
\definecolor{textcolor}{rgb}{0.000000,0.000000,0.000000}%
\pgfsetstrokecolor{textcolor}%
\pgfsetfillcolor{textcolor}%
\pgftext[x=4.031515in,y=0.430778in,,top]{\color{textcolor}\sffamily\fontsize{10.000000}{12.000000}\selectfont 30}%
\end{pgfscope}%
\begin{pgfscope}%
\pgfsetbuttcap%
\pgfsetroundjoin%
\definecolor{currentfill}{rgb}{0.000000,0.000000,0.000000}%
\pgfsetfillcolor{currentfill}%
\pgfsetlinewidth{0.803000pt}%
\definecolor{currentstroke}{rgb}{0.000000,0.000000,0.000000}%
\pgfsetstrokecolor{currentstroke}%
\pgfsetdash{}{0pt}%
\pgfsys@defobject{currentmarker}{\pgfqpoint{0.000000in}{-0.048611in}}{\pgfqpoint{0.000000in}{0.000000in}}{%
\pgfpathmoveto{\pgfqpoint{0.000000in}{0.000000in}}%
\pgfpathlineto{\pgfqpoint{0.000000in}{-0.048611in}}%
\pgfusepath{stroke,fill}%
}%
\begin{pgfscope}%
\pgfsys@transformshift{5.534545in}{0.528000in}%
\pgfsys@useobject{currentmarker}{}%
\end{pgfscope}%
\end{pgfscope}%
\begin{pgfscope}%
\definecolor{textcolor}{rgb}{0.000000,0.000000,0.000000}%
\pgfsetstrokecolor{textcolor}%
\pgfsetfillcolor{textcolor}%
\pgftext[x=5.534545in,y=0.430778in,,top]{\color{textcolor}\sffamily\fontsize{10.000000}{12.000000}\selectfont 40}%
\end{pgfscope}%
\begin{pgfscope}%
\definecolor{textcolor}{rgb}{0.000000,0.000000,0.000000}%
\pgfsetstrokecolor{textcolor}%
\pgfsetfillcolor{textcolor}%
\pgftext[x=3.280000in,y=0.240809in,,top]{\color{textcolor}\sffamily\fontsize{10.000000}{12.000000}\selectfont \(\displaystyle k\)}%
\end{pgfscope}%
\begin{pgfscope}%
\pgfsetbuttcap%
\pgfsetroundjoin%
\definecolor{currentfill}{rgb}{0.000000,0.000000,0.000000}%
\pgfsetfillcolor{currentfill}%
\pgfsetlinewidth{0.803000pt}%
\definecolor{currentstroke}{rgb}{0.000000,0.000000,0.000000}%
\pgfsetstrokecolor{currentstroke}%
\pgfsetdash{}{0pt}%
\pgfsys@defobject{currentmarker}{\pgfqpoint{-0.048611in}{0.000000in}}{\pgfqpoint{0.000000in}{0.000000in}}{%
\pgfpathmoveto{\pgfqpoint{0.000000in}{0.000000in}}%
\pgfpathlineto{\pgfqpoint{-0.048611in}{0.000000in}}%
\pgfusepath{stroke,fill}%
}%
\begin{pgfscope}%
\pgfsys@transformshift{0.800000in}{0.696000in}%
\pgfsys@useobject{currentmarker}{}%
\end{pgfscope}%
\end{pgfscope}%
\begin{pgfscope}%
\definecolor{textcolor}{rgb}{0.000000,0.000000,0.000000}%
\pgfsetstrokecolor{textcolor}%
\pgfsetfillcolor{textcolor}%
\pgftext[x=0.216802in,y=0.643238in,left,base]{\color{textcolor}\sffamily\fontsize{10.000000}{12.000000}\selectfont 0.9920}%
\end{pgfscope}%
\begin{pgfscope}%
\pgfsetbuttcap%
\pgfsetroundjoin%
\definecolor{currentfill}{rgb}{0.000000,0.000000,0.000000}%
\pgfsetfillcolor{currentfill}%
\pgfsetlinewidth{0.803000pt}%
\definecolor{currentstroke}{rgb}{0.000000,0.000000,0.000000}%
\pgfsetstrokecolor{currentstroke}%
\pgfsetdash{}{0pt}%
\pgfsys@defobject{currentmarker}{\pgfqpoint{-0.048611in}{0.000000in}}{\pgfqpoint{0.000000in}{0.000000in}}{%
\pgfpathmoveto{\pgfqpoint{0.000000in}{0.000000in}}%
\pgfpathlineto{\pgfqpoint{-0.048611in}{0.000000in}}%
\pgfusepath{stroke,fill}%
}%
\begin{pgfscope}%
\pgfsys@transformshift{0.800000in}{1.080000in}%
\pgfsys@useobject{currentmarker}{}%
\end{pgfscope}%
\end{pgfscope}%
\begin{pgfscope}%
\definecolor{textcolor}{rgb}{0.000000,0.000000,0.000000}%
\pgfsetstrokecolor{textcolor}%
\pgfsetfillcolor{textcolor}%
\pgftext[x=0.216802in,y=1.027238in,left,base]{\color{textcolor}\sffamily\fontsize{10.000000}{12.000000}\selectfont 0.9928}%
\end{pgfscope}%
\begin{pgfscope}%
\pgfsetbuttcap%
\pgfsetroundjoin%
\definecolor{currentfill}{rgb}{0.000000,0.000000,0.000000}%
\pgfsetfillcolor{currentfill}%
\pgfsetlinewidth{0.803000pt}%
\definecolor{currentstroke}{rgb}{0.000000,0.000000,0.000000}%
\pgfsetstrokecolor{currentstroke}%
\pgfsetdash{}{0pt}%
\pgfsys@defobject{currentmarker}{\pgfqpoint{-0.048611in}{0.000000in}}{\pgfqpoint{0.000000in}{0.000000in}}{%
\pgfpathmoveto{\pgfqpoint{0.000000in}{0.000000in}}%
\pgfpathlineto{\pgfqpoint{-0.048611in}{0.000000in}}%
\pgfusepath{stroke,fill}%
}%
\begin{pgfscope}%
\pgfsys@transformshift{0.800000in}{1.464000in}%
\pgfsys@useobject{currentmarker}{}%
\end{pgfscope}%
\end{pgfscope}%
\begin{pgfscope}%
\definecolor{textcolor}{rgb}{0.000000,0.000000,0.000000}%
\pgfsetstrokecolor{textcolor}%
\pgfsetfillcolor{textcolor}%
\pgftext[x=0.216802in,y=1.411238in,left,base]{\color{textcolor}\sffamily\fontsize{10.000000}{12.000000}\selectfont 0.9936}%
\end{pgfscope}%
\begin{pgfscope}%
\pgfsetbuttcap%
\pgfsetroundjoin%
\definecolor{currentfill}{rgb}{0.000000,0.000000,0.000000}%
\pgfsetfillcolor{currentfill}%
\pgfsetlinewidth{0.803000pt}%
\definecolor{currentstroke}{rgb}{0.000000,0.000000,0.000000}%
\pgfsetstrokecolor{currentstroke}%
\pgfsetdash{}{0pt}%
\pgfsys@defobject{currentmarker}{\pgfqpoint{-0.048611in}{0.000000in}}{\pgfqpoint{0.000000in}{0.000000in}}{%
\pgfpathmoveto{\pgfqpoint{0.000000in}{0.000000in}}%
\pgfpathlineto{\pgfqpoint{-0.048611in}{0.000000in}}%
\pgfusepath{stroke,fill}%
}%
\begin{pgfscope}%
\pgfsys@transformshift{0.800000in}{1.848000in}%
\pgfsys@useobject{currentmarker}{}%
\end{pgfscope}%
\end{pgfscope}%
\begin{pgfscope}%
\definecolor{textcolor}{rgb}{0.000000,0.000000,0.000000}%
\pgfsetstrokecolor{textcolor}%
\pgfsetfillcolor{textcolor}%
\pgftext[x=0.216802in,y=1.795238in,left,base]{\color{textcolor}\sffamily\fontsize{10.000000}{12.000000}\selectfont 0.9944}%
\end{pgfscope}%
\begin{pgfscope}%
\pgfsetbuttcap%
\pgfsetroundjoin%
\definecolor{currentfill}{rgb}{0.000000,0.000000,0.000000}%
\pgfsetfillcolor{currentfill}%
\pgfsetlinewidth{0.803000pt}%
\definecolor{currentstroke}{rgb}{0.000000,0.000000,0.000000}%
\pgfsetstrokecolor{currentstroke}%
\pgfsetdash{}{0pt}%
\pgfsys@defobject{currentmarker}{\pgfqpoint{-0.048611in}{0.000000in}}{\pgfqpoint{0.000000in}{0.000000in}}{%
\pgfpathmoveto{\pgfqpoint{0.000000in}{0.000000in}}%
\pgfpathlineto{\pgfqpoint{-0.048611in}{0.000000in}}%
\pgfusepath{stroke,fill}%
}%
\begin{pgfscope}%
\pgfsys@transformshift{0.800000in}{2.232000in}%
\pgfsys@useobject{currentmarker}{}%
\end{pgfscope}%
\end{pgfscope}%
\begin{pgfscope}%
\definecolor{textcolor}{rgb}{0.000000,0.000000,0.000000}%
\pgfsetstrokecolor{textcolor}%
\pgfsetfillcolor{textcolor}%
\pgftext[x=0.216802in,y=2.179238in,left,base]{\color{textcolor}\sffamily\fontsize{10.000000}{12.000000}\selectfont 0.9952}%
\end{pgfscope}%
\begin{pgfscope}%
\pgfsetbuttcap%
\pgfsetroundjoin%
\definecolor{currentfill}{rgb}{0.000000,0.000000,0.000000}%
\pgfsetfillcolor{currentfill}%
\pgfsetlinewidth{0.803000pt}%
\definecolor{currentstroke}{rgb}{0.000000,0.000000,0.000000}%
\pgfsetstrokecolor{currentstroke}%
\pgfsetdash{}{0pt}%
\pgfsys@defobject{currentmarker}{\pgfqpoint{-0.048611in}{0.000000in}}{\pgfqpoint{0.000000in}{0.000000in}}{%
\pgfpathmoveto{\pgfqpoint{0.000000in}{0.000000in}}%
\pgfpathlineto{\pgfqpoint{-0.048611in}{0.000000in}}%
\pgfusepath{stroke,fill}%
}%
\begin{pgfscope}%
\pgfsys@transformshift{0.800000in}{2.616000in}%
\pgfsys@useobject{currentmarker}{}%
\end{pgfscope}%
\end{pgfscope}%
\begin{pgfscope}%
\definecolor{textcolor}{rgb}{0.000000,0.000000,0.000000}%
\pgfsetstrokecolor{textcolor}%
\pgfsetfillcolor{textcolor}%
\pgftext[x=0.216802in,y=2.563238in,left,base]{\color{textcolor}\sffamily\fontsize{10.000000}{12.000000}\selectfont 0.9960}%
\end{pgfscope}%
\begin{pgfscope}%
\pgfsetbuttcap%
\pgfsetroundjoin%
\definecolor{currentfill}{rgb}{0.000000,0.000000,0.000000}%
\pgfsetfillcolor{currentfill}%
\pgfsetlinewidth{0.803000pt}%
\definecolor{currentstroke}{rgb}{0.000000,0.000000,0.000000}%
\pgfsetstrokecolor{currentstroke}%
\pgfsetdash{}{0pt}%
\pgfsys@defobject{currentmarker}{\pgfqpoint{-0.048611in}{0.000000in}}{\pgfqpoint{0.000000in}{0.000000in}}{%
\pgfpathmoveto{\pgfqpoint{0.000000in}{0.000000in}}%
\pgfpathlineto{\pgfqpoint{-0.048611in}{0.000000in}}%
\pgfusepath{stroke,fill}%
}%
\begin{pgfscope}%
\pgfsys@transformshift{0.800000in}{3.000000in}%
\pgfsys@useobject{currentmarker}{}%
\end{pgfscope}%
\end{pgfscope}%
\begin{pgfscope}%
\definecolor{textcolor}{rgb}{0.000000,0.000000,0.000000}%
\pgfsetstrokecolor{textcolor}%
\pgfsetfillcolor{textcolor}%
\pgftext[x=0.216802in,y=2.947238in,left,base]{\color{textcolor}\sffamily\fontsize{10.000000}{12.000000}\selectfont 0.9968}%
\end{pgfscope}%
\begin{pgfscope}%
\pgfsetbuttcap%
\pgfsetroundjoin%
\definecolor{currentfill}{rgb}{0.000000,0.000000,0.000000}%
\pgfsetfillcolor{currentfill}%
\pgfsetlinewidth{0.803000pt}%
\definecolor{currentstroke}{rgb}{0.000000,0.000000,0.000000}%
\pgfsetstrokecolor{currentstroke}%
\pgfsetdash{}{0pt}%
\pgfsys@defobject{currentmarker}{\pgfqpoint{-0.048611in}{0.000000in}}{\pgfqpoint{0.000000in}{0.000000in}}{%
\pgfpathmoveto{\pgfqpoint{0.000000in}{0.000000in}}%
\pgfpathlineto{\pgfqpoint{-0.048611in}{0.000000in}}%
\pgfusepath{stroke,fill}%
}%
\begin{pgfscope}%
\pgfsys@transformshift{0.800000in}{3.384000in}%
\pgfsys@useobject{currentmarker}{}%
\end{pgfscope}%
\end{pgfscope}%
\begin{pgfscope}%
\definecolor{textcolor}{rgb}{0.000000,0.000000,0.000000}%
\pgfsetstrokecolor{textcolor}%
\pgfsetfillcolor{textcolor}%
\pgftext[x=0.216802in,y=3.331238in,left,base]{\color{textcolor}\sffamily\fontsize{10.000000}{12.000000}\selectfont 0.9976}%
\end{pgfscope}%
\begin{pgfscope}%
\pgfsetbuttcap%
\pgfsetroundjoin%
\definecolor{currentfill}{rgb}{0.000000,0.000000,0.000000}%
\pgfsetfillcolor{currentfill}%
\pgfsetlinewidth{0.803000pt}%
\definecolor{currentstroke}{rgb}{0.000000,0.000000,0.000000}%
\pgfsetstrokecolor{currentstroke}%
\pgfsetdash{}{0pt}%
\pgfsys@defobject{currentmarker}{\pgfqpoint{-0.048611in}{0.000000in}}{\pgfqpoint{0.000000in}{0.000000in}}{%
\pgfpathmoveto{\pgfqpoint{0.000000in}{0.000000in}}%
\pgfpathlineto{\pgfqpoint{-0.048611in}{0.000000in}}%
\pgfusepath{stroke,fill}%
}%
\begin{pgfscope}%
\pgfsys@transformshift{0.800000in}{3.768000in}%
\pgfsys@useobject{currentmarker}{}%
\end{pgfscope}%
\end{pgfscope}%
\begin{pgfscope}%
\definecolor{textcolor}{rgb}{0.000000,0.000000,0.000000}%
\pgfsetstrokecolor{textcolor}%
\pgfsetfillcolor{textcolor}%
\pgftext[x=0.216802in,y=3.715238in,left,base]{\color{textcolor}\sffamily\fontsize{10.000000}{12.000000}\selectfont 0.9984}%
\end{pgfscope}%
\begin{pgfscope}%
\pgfsetbuttcap%
\pgfsetroundjoin%
\definecolor{currentfill}{rgb}{0.000000,0.000000,0.000000}%
\pgfsetfillcolor{currentfill}%
\pgfsetlinewidth{0.803000pt}%
\definecolor{currentstroke}{rgb}{0.000000,0.000000,0.000000}%
\pgfsetstrokecolor{currentstroke}%
\pgfsetdash{}{0pt}%
\pgfsys@defobject{currentmarker}{\pgfqpoint{-0.048611in}{0.000000in}}{\pgfqpoint{0.000000in}{0.000000in}}{%
\pgfpathmoveto{\pgfqpoint{0.000000in}{0.000000in}}%
\pgfpathlineto{\pgfqpoint{-0.048611in}{0.000000in}}%
\pgfusepath{stroke,fill}%
}%
\begin{pgfscope}%
\pgfsys@transformshift{0.800000in}{4.152000in}%
\pgfsys@useobject{currentmarker}{}%
\end{pgfscope}%
\end{pgfscope}%
\begin{pgfscope}%
\definecolor{textcolor}{rgb}{0.000000,0.000000,0.000000}%
\pgfsetstrokecolor{textcolor}%
\pgfsetfillcolor{textcolor}%
\pgftext[x=0.216802in,y=4.099238in,left,base]{\color{textcolor}\sffamily\fontsize{10.000000}{12.000000}\selectfont 0.9992}%
\end{pgfscope}%
\begin{pgfscope}%
\definecolor{textcolor}{rgb}{0.000000,0.000000,0.000000}%
\pgfsetstrokecolor{textcolor}%
\pgfsetfillcolor{textcolor}%
\pgftext[x=0.161247in,y=2.376000in,,bottom,rotate=90.000000]{\color{textcolor}\sffamily\fontsize{10.000000}{12.000000}\selectfont Number of GMRES iterations}%
\end{pgfscope}%
\begin{pgfscope}%
\pgfpathrectangle{\pgfqpoint{0.800000in}{0.528000in}}{\pgfqpoint{4.960000in}{3.696000in}}%
\pgfusepath{clip}%
\pgfsetbuttcap%
\pgfsetroundjoin%
\definecolor{currentfill}{rgb}{0.000000,0.000000,0.000000}%
\pgfsetfillcolor{currentfill}%
\pgfsetlinewidth{1.003750pt}%
\definecolor{currentstroke}{rgb}{0.000000,0.000000,0.000000}%
\pgfsetstrokecolor{currentstroke}%
\pgfsetdash{}{0pt}%
\pgfsys@defobject{currentmarker}{\pgfqpoint{-0.041667in}{-0.041667in}}{\pgfqpoint{0.041667in}{0.041667in}}{%
\pgfpathmoveto{\pgfqpoint{0.000000in}{-0.041667in}}%
\pgfpathcurveto{\pgfqpoint{0.011050in}{-0.041667in}}{\pgfqpoint{0.021649in}{-0.037276in}}{\pgfqpoint{0.029463in}{-0.029463in}}%
\pgfpathcurveto{\pgfqpoint{0.037276in}{-0.021649in}}{\pgfqpoint{0.041667in}{-0.011050in}}{\pgfqpoint{0.041667in}{0.000000in}}%
\pgfpathcurveto{\pgfqpoint{0.041667in}{0.011050in}}{\pgfqpoint{0.037276in}{0.021649in}}{\pgfqpoint{0.029463in}{0.029463in}}%
\pgfpathcurveto{\pgfqpoint{0.021649in}{0.037276in}}{\pgfqpoint{0.011050in}{0.041667in}}{\pgfqpoint{0.000000in}{0.041667in}}%
\pgfpathcurveto{\pgfqpoint{-0.011050in}{0.041667in}}{\pgfqpoint{-0.021649in}{0.037276in}}{\pgfqpoint{-0.029463in}{0.029463in}}%
\pgfpathcurveto{\pgfqpoint{-0.037276in}{0.021649in}}{\pgfqpoint{-0.041667in}{0.011050in}}{\pgfqpoint{-0.041667in}{0.000000in}}%
\pgfpathcurveto{\pgfqpoint{-0.041667in}{-0.011050in}}{\pgfqpoint{-0.037276in}{-0.021649in}}{\pgfqpoint{-0.029463in}{-0.029463in}}%
\pgfpathcurveto{\pgfqpoint{-0.021649in}{-0.037276in}}{\pgfqpoint{-0.011050in}{-0.041667in}}{\pgfqpoint{0.000000in}{-0.041667in}}%
\pgfpathclose%
\pgfusepath{stroke,fill}%
}%
\begin{pgfscope}%
\pgfsys@transformshift{1.025455in}{4.056000in}%
\pgfsys@useobject{currentmarker}{}%
\end{pgfscope}%
\end{pgfscope}%
\begin{pgfscope}%
\pgfpathrectangle{\pgfqpoint{0.800000in}{0.528000in}}{\pgfqpoint{4.960000in}{3.696000in}}%
\pgfusepath{clip}%
\pgfsetbuttcap%
\pgfsetroundjoin%
\definecolor{currentfill}{rgb}{0.000000,0.000000,0.000000}%
\pgfsetfillcolor{currentfill}%
\pgfsetlinewidth{1.003750pt}%
\definecolor{currentstroke}{rgb}{0.000000,0.000000,0.000000}%
\pgfsetstrokecolor{currentstroke}%
\pgfsetdash{}{0pt}%
\pgfsys@defobject{currentmarker}{\pgfqpoint{-0.041667in}{-0.041667in}}{\pgfqpoint{0.041667in}{0.041667in}}{%
\pgfpathmoveto{\pgfqpoint{0.000000in}{-0.041667in}}%
\pgfpathcurveto{\pgfqpoint{0.011050in}{-0.041667in}}{\pgfqpoint{0.021649in}{-0.037276in}}{\pgfqpoint{0.029463in}{-0.029463in}}%
\pgfpathcurveto{\pgfqpoint{0.037276in}{-0.021649in}}{\pgfqpoint{0.041667in}{-0.011050in}}{\pgfqpoint{0.041667in}{0.000000in}}%
\pgfpathcurveto{\pgfqpoint{0.041667in}{0.011050in}}{\pgfqpoint{0.037276in}{0.021649in}}{\pgfqpoint{0.029463in}{0.029463in}}%
\pgfpathcurveto{\pgfqpoint{0.021649in}{0.037276in}}{\pgfqpoint{0.011050in}{0.041667in}}{\pgfqpoint{0.000000in}{0.041667in}}%
\pgfpathcurveto{\pgfqpoint{-0.011050in}{0.041667in}}{\pgfqpoint{-0.021649in}{0.037276in}}{\pgfqpoint{-0.029463in}{0.029463in}}%
\pgfpathcurveto{\pgfqpoint{-0.037276in}{0.021649in}}{\pgfqpoint{-0.041667in}{0.011050in}}{\pgfqpoint{-0.041667in}{0.000000in}}%
\pgfpathcurveto{\pgfqpoint{-0.041667in}{-0.011050in}}{\pgfqpoint{-0.037276in}{-0.021649in}}{\pgfqpoint{-0.029463in}{-0.029463in}}%
\pgfpathcurveto{\pgfqpoint{-0.021649in}{-0.037276in}}{\pgfqpoint{-0.011050in}{-0.041667in}}{\pgfqpoint{0.000000in}{-0.041667in}}%
\pgfpathclose%
\pgfusepath{stroke,fill}%
}%
\begin{pgfscope}%
\pgfsys@transformshift{2.528485in}{0.696000in}%
\pgfsys@useobject{currentmarker}{}%
\end{pgfscope}%
\end{pgfscope}%
\begin{pgfscope}%
\pgfpathrectangle{\pgfqpoint{0.800000in}{0.528000in}}{\pgfqpoint{4.960000in}{3.696000in}}%
\pgfusepath{clip}%
\pgfsetbuttcap%
\pgfsetroundjoin%
\definecolor{currentfill}{rgb}{0.000000,0.000000,0.000000}%
\pgfsetfillcolor{currentfill}%
\pgfsetlinewidth{1.003750pt}%
\definecolor{currentstroke}{rgb}{0.000000,0.000000,0.000000}%
\pgfsetstrokecolor{currentstroke}%
\pgfsetdash{}{0pt}%
\pgfsys@defobject{currentmarker}{\pgfqpoint{-0.041667in}{-0.041667in}}{\pgfqpoint{0.041667in}{0.041667in}}{%
\pgfpathmoveto{\pgfqpoint{0.000000in}{-0.041667in}}%
\pgfpathcurveto{\pgfqpoint{0.011050in}{-0.041667in}}{\pgfqpoint{0.021649in}{-0.037276in}}{\pgfqpoint{0.029463in}{-0.029463in}}%
\pgfpathcurveto{\pgfqpoint{0.037276in}{-0.021649in}}{\pgfqpoint{0.041667in}{-0.011050in}}{\pgfqpoint{0.041667in}{0.000000in}}%
\pgfpathcurveto{\pgfqpoint{0.041667in}{0.011050in}}{\pgfqpoint{0.037276in}{0.021649in}}{\pgfqpoint{0.029463in}{0.029463in}}%
\pgfpathcurveto{\pgfqpoint{0.021649in}{0.037276in}}{\pgfqpoint{0.011050in}{0.041667in}}{\pgfqpoint{0.000000in}{0.041667in}}%
\pgfpathcurveto{\pgfqpoint{-0.011050in}{0.041667in}}{\pgfqpoint{-0.021649in}{0.037276in}}{\pgfqpoint{-0.029463in}{0.029463in}}%
\pgfpathcurveto{\pgfqpoint{-0.037276in}{0.021649in}}{\pgfqpoint{-0.041667in}{0.011050in}}{\pgfqpoint{-0.041667in}{0.000000in}}%
\pgfpathcurveto{\pgfqpoint{-0.041667in}{-0.011050in}}{\pgfqpoint{-0.037276in}{-0.021649in}}{\pgfqpoint{-0.029463in}{-0.029463in}}%
\pgfpathcurveto{\pgfqpoint{-0.021649in}{-0.037276in}}{\pgfqpoint{-0.011050in}{-0.041667in}}{\pgfqpoint{0.000000in}{-0.041667in}}%
\pgfpathclose%
\pgfusepath{stroke,fill}%
}%
\begin{pgfscope}%
\pgfsys@transformshift{4.031515in}{0.696000in}%
\pgfsys@useobject{currentmarker}{}%
\end{pgfscope}%
\end{pgfscope}%
\begin{pgfscope}%
\pgfpathrectangle{\pgfqpoint{0.800000in}{0.528000in}}{\pgfqpoint{4.960000in}{3.696000in}}%
\pgfusepath{clip}%
\pgfsetbuttcap%
\pgfsetroundjoin%
\definecolor{currentfill}{rgb}{0.000000,0.000000,0.000000}%
\pgfsetfillcolor{currentfill}%
\pgfsetlinewidth{1.003750pt}%
\definecolor{currentstroke}{rgb}{0.000000,0.000000,0.000000}%
\pgfsetstrokecolor{currentstroke}%
\pgfsetdash{}{0pt}%
\pgfsys@defobject{currentmarker}{\pgfqpoint{-0.041667in}{-0.041667in}}{\pgfqpoint{0.041667in}{0.041667in}}{%
\pgfpathmoveto{\pgfqpoint{0.000000in}{-0.041667in}}%
\pgfpathcurveto{\pgfqpoint{0.011050in}{-0.041667in}}{\pgfqpoint{0.021649in}{-0.037276in}}{\pgfqpoint{0.029463in}{-0.029463in}}%
\pgfpathcurveto{\pgfqpoint{0.037276in}{-0.021649in}}{\pgfqpoint{0.041667in}{-0.011050in}}{\pgfqpoint{0.041667in}{0.000000in}}%
\pgfpathcurveto{\pgfqpoint{0.041667in}{0.011050in}}{\pgfqpoint{0.037276in}{0.021649in}}{\pgfqpoint{0.029463in}{0.029463in}}%
\pgfpathcurveto{\pgfqpoint{0.021649in}{0.037276in}}{\pgfqpoint{0.011050in}{0.041667in}}{\pgfqpoint{0.000000in}{0.041667in}}%
\pgfpathcurveto{\pgfqpoint{-0.011050in}{0.041667in}}{\pgfqpoint{-0.021649in}{0.037276in}}{\pgfqpoint{-0.029463in}{0.029463in}}%
\pgfpathcurveto{\pgfqpoint{-0.037276in}{0.021649in}}{\pgfqpoint{-0.041667in}{0.011050in}}{\pgfqpoint{-0.041667in}{0.000000in}}%
\pgfpathcurveto{\pgfqpoint{-0.041667in}{-0.011050in}}{\pgfqpoint{-0.037276in}{-0.021649in}}{\pgfqpoint{-0.029463in}{-0.029463in}}%
\pgfpathcurveto{\pgfqpoint{-0.021649in}{-0.037276in}}{\pgfqpoint{-0.011050in}{-0.041667in}}{\pgfqpoint{0.000000in}{-0.041667in}}%
\pgfpathclose%
\pgfusepath{stroke,fill}%
}%
\begin{pgfscope}%
\pgfsys@transformshift{5.534545in}{1.656000in}%
\pgfsys@useobject{currentmarker}{}%
\end{pgfscope}%
\end{pgfscope}%
\begin{pgfscope}%
\pgfsetrectcap%
\pgfsetmiterjoin%
\pgfsetlinewidth{0.803000pt}%
\definecolor{currentstroke}{rgb}{0.000000,0.000000,0.000000}%
\pgfsetstrokecolor{currentstroke}%
\pgfsetdash{}{0pt}%
\pgfpathmoveto{\pgfqpoint{0.800000in}{0.528000in}}%
\pgfpathlineto{\pgfqpoint{0.800000in}{4.224000in}}%
\pgfusepath{stroke}%
\end{pgfscope}%
\begin{pgfscope}%
\pgfsetrectcap%
\pgfsetmiterjoin%
\pgfsetlinewidth{0.803000pt}%
\definecolor{currentstroke}{rgb}{0.000000,0.000000,0.000000}%
\pgfsetstrokecolor{currentstroke}%
\pgfsetdash{}{0pt}%
\pgfpathmoveto{\pgfqpoint{5.760000in}{0.528000in}}%
\pgfpathlineto{\pgfqpoint{5.760000in}{4.224000in}}%
\pgfusepath{stroke}%
\end{pgfscope}%
\begin{pgfscope}%
\pgfsetrectcap%
\pgfsetmiterjoin%
\pgfsetlinewidth{0.803000pt}%
\definecolor{currentstroke}{rgb}{0.000000,0.000000,0.000000}%
\pgfsetstrokecolor{currentstroke}%
\pgfsetdash{}{0pt}%
\pgfpathmoveto{\pgfqpoint{0.800000in}{0.528000in}}%
\pgfpathlineto{\pgfqpoint{5.760000in}{0.528000in}}%
\pgfusepath{stroke}%
\end{pgfscope}%
\begin{pgfscope}%
\pgfsetrectcap%
\pgfsetmiterjoin%
\pgfsetlinewidth{0.803000pt}%
\definecolor{currentstroke}{rgb}{0.000000,0.000000,0.000000}%
\pgfsetstrokecolor{currentstroke}%
\pgfsetdash{}{0pt}%
\pgfpathmoveto{\pgfqpoint{0.800000in}{4.224000in}}%
\pgfpathlineto{\pgfqpoint{5.760000in}{4.224000in}}%
\pgfusepath{stroke}%
\end{pgfscope}%
\end{pgfpicture}%
\makeatother%
\endgroup%

\caption{Empricial probability (calculated from 1000 realisations) that $\GMRES{\eps}{\no}{\nt}\leq 12$ for $k = 10, 20, 30, 40,$ where $R=12$, $\eps = 10^{-5}$, $N = \ceil{k^{3}}$, and $\Ct=0.1,$ with $\NLiDRR{\no-\nt} \sim \Exp{\sigma}$ with $\sigma = 1/k$\label{fig:prob-plot-1.0}}
\end{figure}

\begin{figure}[h]
    \centering
%% Creator: Matplotlib, PGF backend
%%
%% To include the figure in your LaTeX document, write
%%   \input{<filename>.pgf}
%%
%% Make sure the required packages are loaded in your preamble
%%   \usepackage{pgf}
%%
%% Figures using additional raster images can only be included by \input if
%% they are in the same directory as the main LaTeX file. For loading figures
%% from other directories you can use the `import` package
%%   \usepackage{import}
%% and then include the figures with
%%   \import{<path to file>}{<filename>.pgf}
%%
%% Matplotlib used the following preamble
%%   \usepackage{mleftright}
%%   \usepackage{fontspec}
%%   \setmainfont{DejaVuSerif.ttf}[Path=/home/owen/progs/firedrake-complex/firedrake/lib/python3.5/site-packages/matplotlib/mpl-data/fonts/ttf/]
%%   \setsansfont{DejaVuSans.ttf}[Path=/home/owen/progs/firedrake-complex/firedrake/lib/python3.5/site-packages/matplotlib/mpl-data/fonts/ttf/]
%%   \setmonofont{DejaVuSansMono.ttf}[Path=/home/owen/progs/firedrake-complex/firedrake/lib/python3.5/site-packages/matplotlib/mpl-data/fonts/ttf/]
%%
\begingroup%
\makeatletter%
\begin{pgfpicture}%
\pgfpathrectangle{\pgfpointorigin}{\pgfqpoint{5.462193in}{2.581603in}}%
\pgfusepath{use as bounding box, clip}%
\begin{pgfscope}%
\pgfsetbuttcap%
\pgfsetmiterjoin%
\definecolor{currentfill}{rgb}{1.000000,1.000000,1.000000}%
\pgfsetfillcolor{currentfill}%
\pgfsetlinewidth{0.000000pt}%
\definecolor{currentstroke}{rgb}{1.000000,1.000000,1.000000}%
\pgfsetstrokecolor{currentstroke}%
\pgfsetdash{}{0pt}%
\pgfpathmoveto{\pgfqpoint{-0.000000in}{0.000000in}}%
\pgfpathlineto{\pgfqpoint{5.462193in}{0.000000in}}%
\pgfpathlineto{\pgfqpoint{5.462193in}{2.581603in}}%
\pgfpathlineto{\pgfqpoint{-0.000000in}{2.581603in}}%
\pgfpathclose%
\pgfusepath{fill}%
\end{pgfscope}%
\begin{pgfscope}%
\pgfsetbuttcap%
\pgfsetmiterjoin%
\definecolor{currentfill}{rgb}{1.000000,1.000000,1.000000}%
\pgfsetfillcolor{currentfill}%
\pgfsetlinewidth{0.000000pt}%
\definecolor{currentstroke}{rgb}{0.000000,0.000000,0.000000}%
\pgfsetstrokecolor{currentstroke}%
\pgfsetstrokeopacity{0.000000}%
\pgfsetdash{}{0pt}%
\pgfpathmoveto{\pgfqpoint{0.677193in}{0.521603in}}%
\pgfpathlineto{\pgfqpoint{5.327193in}{0.521603in}}%
\pgfpathlineto{\pgfqpoint{5.327193in}{2.446603in}}%
\pgfpathlineto{\pgfqpoint{0.677193in}{2.446603in}}%
\pgfpathclose%
\pgfusepath{fill}%
\end{pgfscope}%
\begin{pgfscope}%
\pgfsetbuttcap%
\pgfsetroundjoin%
\definecolor{currentfill}{rgb}{0.000000,0.000000,0.000000}%
\pgfsetfillcolor{currentfill}%
\pgfsetlinewidth{0.803000pt}%
\definecolor{currentstroke}{rgb}{0.000000,0.000000,0.000000}%
\pgfsetstrokecolor{currentstroke}%
\pgfsetdash{}{0pt}%
\pgfsys@defobject{currentmarker}{\pgfqpoint{0.000000in}{-0.048611in}}{\pgfqpoint{0.000000in}{0.000000in}}{%
\pgfpathmoveto{\pgfqpoint{0.000000in}{0.000000in}}%
\pgfpathlineto{\pgfqpoint{0.000000in}{-0.048611in}}%
\pgfusepath{stroke,fill}%
}%
\begin{pgfscope}%
\pgfsys@transformshift{0.888557in}{0.521603in}%
\pgfsys@useobject{currentmarker}{}%
\end{pgfscope}%
\end{pgfscope}%
\begin{pgfscope}%
\definecolor{textcolor}{rgb}{0.000000,0.000000,0.000000}%
\pgfsetstrokecolor{textcolor}%
\pgfsetfillcolor{textcolor}%
\pgftext[x=0.888557in,y=0.424381in,,top]{\color{textcolor}\sffamily\fontsize{10.000000}{12.000000}\selectfont \(\displaystyle 10\)}%
\end{pgfscope}%
\begin{pgfscope}%
\pgfsetbuttcap%
\pgfsetroundjoin%
\definecolor{currentfill}{rgb}{0.000000,0.000000,0.000000}%
\pgfsetfillcolor{currentfill}%
\pgfsetlinewidth{0.803000pt}%
\definecolor{currentstroke}{rgb}{0.000000,0.000000,0.000000}%
\pgfsetstrokecolor{currentstroke}%
\pgfsetdash{}{0pt}%
\pgfsys@defobject{currentmarker}{\pgfqpoint{0.000000in}{-0.048611in}}{\pgfqpoint{0.000000in}{0.000000in}}{%
\pgfpathmoveto{\pgfqpoint{0.000000in}{0.000000in}}%
\pgfpathlineto{\pgfqpoint{0.000000in}{-0.048611in}}%
\pgfusepath{stroke,fill}%
}%
\begin{pgfscope}%
\pgfsys@transformshift{2.297648in}{0.521603in}%
\pgfsys@useobject{currentmarker}{}%
\end{pgfscope}%
\end{pgfscope}%
\begin{pgfscope}%
\definecolor{textcolor}{rgb}{0.000000,0.000000,0.000000}%
\pgfsetstrokecolor{textcolor}%
\pgfsetfillcolor{textcolor}%
\pgftext[x=2.297648in,y=0.424381in,,top]{\color{textcolor}\sffamily\fontsize{10.000000}{12.000000}\selectfont \(\displaystyle 20\)}%
\end{pgfscope}%
\begin{pgfscope}%
\pgfsetbuttcap%
\pgfsetroundjoin%
\definecolor{currentfill}{rgb}{0.000000,0.000000,0.000000}%
\pgfsetfillcolor{currentfill}%
\pgfsetlinewidth{0.803000pt}%
\definecolor{currentstroke}{rgb}{0.000000,0.000000,0.000000}%
\pgfsetstrokecolor{currentstroke}%
\pgfsetdash{}{0pt}%
\pgfsys@defobject{currentmarker}{\pgfqpoint{0.000000in}{-0.048611in}}{\pgfqpoint{0.000000in}{0.000000in}}{%
\pgfpathmoveto{\pgfqpoint{0.000000in}{0.000000in}}%
\pgfpathlineto{\pgfqpoint{0.000000in}{-0.048611in}}%
\pgfusepath{stroke,fill}%
}%
\begin{pgfscope}%
\pgfsys@transformshift{3.706738in}{0.521603in}%
\pgfsys@useobject{currentmarker}{}%
\end{pgfscope}%
\end{pgfscope}%
\begin{pgfscope}%
\definecolor{textcolor}{rgb}{0.000000,0.000000,0.000000}%
\pgfsetstrokecolor{textcolor}%
\pgfsetfillcolor{textcolor}%
\pgftext[x=3.706738in,y=0.424381in,,top]{\color{textcolor}\sffamily\fontsize{10.000000}{12.000000}\selectfont \(\displaystyle 30\)}%
\end{pgfscope}%
\begin{pgfscope}%
\pgfsetbuttcap%
\pgfsetroundjoin%
\definecolor{currentfill}{rgb}{0.000000,0.000000,0.000000}%
\pgfsetfillcolor{currentfill}%
\pgfsetlinewidth{0.803000pt}%
\definecolor{currentstroke}{rgb}{0.000000,0.000000,0.000000}%
\pgfsetstrokecolor{currentstroke}%
\pgfsetdash{}{0pt}%
\pgfsys@defobject{currentmarker}{\pgfqpoint{0.000000in}{-0.048611in}}{\pgfqpoint{0.000000in}{0.000000in}}{%
\pgfpathmoveto{\pgfqpoint{0.000000in}{0.000000in}}%
\pgfpathlineto{\pgfqpoint{0.000000in}{-0.048611in}}%
\pgfusepath{stroke,fill}%
}%
\begin{pgfscope}%
\pgfsys@transformshift{5.115829in}{0.521603in}%
\pgfsys@useobject{currentmarker}{}%
\end{pgfscope}%
\end{pgfscope}%
\begin{pgfscope}%
\definecolor{textcolor}{rgb}{0.000000,0.000000,0.000000}%
\pgfsetstrokecolor{textcolor}%
\pgfsetfillcolor{textcolor}%
\pgftext[x=5.115829in,y=0.424381in,,top]{\color{textcolor}\sffamily\fontsize{10.000000}{12.000000}\selectfont \(\displaystyle 40\)}%
\end{pgfscope}%
\begin{pgfscope}%
\definecolor{textcolor}{rgb}{0.000000,0.000000,0.000000}%
\pgfsetstrokecolor{textcolor}%
\pgfsetfillcolor{textcolor}%
\pgftext[x=3.002193in,y=0.234413in,,top]{\color{textcolor}\sffamily\fontsize{10.000000}{12.000000}\selectfont \(\displaystyle k\)}%
\end{pgfscope}%
\begin{pgfscope}%
\pgfsetbuttcap%
\pgfsetroundjoin%
\definecolor{currentfill}{rgb}{0.000000,0.000000,0.000000}%
\pgfsetfillcolor{currentfill}%
\pgfsetlinewidth{0.803000pt}%
\definecolor{currentstroke}{rgb}{0.000000,0.000000,0.000000}%
\pgfsetstrokecolor{currentstroke}%
\pgfsetdash{}{0pt}%
\pgfsys@defobject{currentmarker}{\pgfqpoint{-0.048611in}{0.000000in}}{\pgfqpoint{0.000000in}{0.000000in}}{%
\pgfpathmoveto{\pgfqpoint{0.000000in}{0.000000in}}%
\pgfpathlineto{\pgfqpoint{-0.048611in}{0.000000in}}%
\pgfusepath{stroke,fill}%
}%
\begin{pgfscope}%
\pgfsys@transformshift{0.677193in}{1.484103in}%
\pgfsys@useobject{currentmarker}{}%
\end{pgfscope}%
\end{pgfscope}%
\begin{pgfscope}%
\definecolor{textcolor}{rgb}{0.000000,0.000000,0.000000}%
\pgfsetstrokecolor{textcolor}%
\pgfsetfillcolor{textcolor}%
\pgftext[x=0.510526in,y=1.431342in,left,base]{\color{textcolor}\sffamily\fontsize{10.000000}{12.000000}\selectfont \(\displaystyle 1\)}%
\end{pgfscope}%
\begin{pgfscope}%
\definecolor{textcolor}{rgb}{0.000000,0.000000,0.000000}%
\pgfsetstrokecolor{textcolor}%
\pgfsetfillcolor{textcolor}%
\pgftext[x=0.144134in,y=0.605638in,left,base,rotate=90.000000]{\color{textcolor}\sffamily\fontsize{10.000000}{12.000000}\selectfont Empirical probability that}%
\end{pgfscope}%
\begin{pgfscope}%
\definecolor{textcolor}{rgb}{0.000000,0.000000,0.000000}%
\pgfsetstrokecolor{textcolor}%
\pgfsetfillcolor{textcolor}%
\pgftext[x=0.364691in,y=0.670455in,left,base,rotate=90.000000]{\color{textcolor}\sffamily\fontsize{10.000000}{12.000000}\selectfont \(\displaystyle \mathrm{GMRES}\mleft(\epsilon, n^{(1)} n^{(2)}\mright) \leq 12\)}%
\end{pgfscope}%
\begin{pgfscope}%
\pgfpathrectangle{\pgfqpoint{0.677193in}{0.521603in}}{\pgfqpoint{4.650000in}{1.925000in}}%
\pgfusepath{clip}%
\pgfsetbuttcap%
\pgfsetroundjoin%
\pgfsetlinewidth{1.505625pt}%
\definecolor{currentstroke}{rgb}{0.000000,0.000000,0.000000}%
\pgfsetstrokecolor{currentstroke}%
\pgfsetdash{{5.550000pt}{2.400000pt}}{0.000000pt}%
\pgfpathmoveto{\pgfqpoint{0.888557in}{1.484103in}}%
\pgfpathlineto{\pgfqpoint{2.297648in}{1.484103in}}%
\pgfpathlineto{\pgfqpoint{3.706738in}{1.484103in}}%
\pgfpathlineto{\pgfqpoint{5.115829in}{1.484103in}}%
\pgfusepath{stroke}%
\end{pgfscope}%
\begin{pgfscope}%
\pgfpathrectangle{\pgfqpoint{0.677193in}{0.521603in}}{\pgfqpoint{4.650000in}{1.925000in}}%
\pgfusepath{clip}%
\pgfsetbuttcap%
\pgfsetroundjoin%
\definecolor{currentfill}{rgb}{0.000000,0.000000,0.000000}%
\pgfsetfillcolor{currentfill}%
\pgfsetlinewidth{1.003750pt}%
\definecolor{currentstroke}{rgb}{0.000000,0.000000,0.000000}%
\pgfsetstrokecolor{currentstroke}%
\pgfsetdash{}{0pt}%
\pgfsys@defobject{currentmarker}{\pgfqpoint{-0.041667in}{-0.041667in}}{\pgfqpoint{0.041667in}{0.041667in}}{%
\pgfpathmoveto{\pgfqpoint{0.000000in}{-0.041667in}}%
\pgfpathcurveto{\pgfqpoint{0.011050in}{-0.041667in}}{\pgfqpoint{0.021649in}{-0.037276in}}{\pgfqpoint{0.029463in}{-0.029463in}}%
\pgfpathcurveto{\pgfqpoint{0.037276in}{-0.021649in}}{\pgfqpoint{0.041667in}{-0.011050in}}{\pgfqpoint{0.041667in}{0.000000in}}%
\pgfpathcurveto{\pgfqpoint{0.041667in}{0.011050in}}{\pgfqpoint{0.037276in}{0.021649in}}{\pgfqpoint{0.029463in}{0.029463in}}%
\pgfpathcurveto{\pgfqpoint{0.021649in}{0.037276in}}{\pgfqpoint{0.011050in}{0.041667in}}{\pgfqpoint{0.000000in}{0.041667in}}%
\pgfpathcurveto{\pgfqpoint{-0.011050in}{0.041667in}}{\pgfqpoint{-0.021649in}{0.037276in}}{\pgfqpoint{-0.029463in}{0.029463in}}%
\pgfpathcurveto{\pgfqpoint{-0.037276in}{0.021649in}}{\pgfqpoint{-0.041667in}{0.011050in}}{\pgfqpoint{-0.041667in}{0.000000in}}%
\pgfpathcurveto{\pgfqpoint{-0.041667in}{-0.011050in}}{\pgfqpoint{-0.037276in}{-0.021649in}}{\pgfqpoint{-0.029463in}{-0.029463in}}%
\pgfpathcurveto{\pgfqpoint{-0.021649in}{-0.037276in}}{\pgfqpoint{-0.011050in}{-0.041667in}}{\pgfqpoint{0.000000in}{-0.041667in}}%
\pgfpathclose%
\pgfusepath{stroke,fill}%
}%
\begin{pgfscope}%
\pgfsys@transformshift{0.888557in}{1.484103in}%
\pgfsys@useobject{currentmarker}{}%
\end{pgfscope}%
\begin{pgfscope}%
\pgfsys@transformshift{2.297648in}{1.484103in}%
\pgfsys@useobject{currentmarker}{}%
\end{pgfscope}%
\begin{pgfscope}%
\pgfsys@transformshift{3.706738in}{1.484103in}%
\pgfsys@useobject{currentmarker}{}%
\end{pgfscope}%
\begin{pgfscope}%
\pgfsys@transformshift{5.115829in}{1.484103in}%
\pgfsys@useobject{currentmarker}{}%
\end{pgfscope}%
\end{pgfscope}%
\begin{pgfscope}%
\pgfsetrectcap%
\pgfsetmiterjoin%
\pgfsetlinewidth{0.803000pt}%
\definecolor{currentstroke}{rgb}{0.000000,0.000000,0.000000}%
\pgfsetstrokecolor{currentstroke}%
\pgfsetdash{}{0pt}%
\pgfpathmoveto{\pgfqpoint{0.677193in}{0.521603in}}%
\pgfpathlineto{\pgfqpoint{0.677193in}{2.446603in}}%
\pgfusepath{stroke}%
\end{pgfscope}%
\begin{pgfscope}%
\pgfsetrectcap%
\pgfsetmiterjoin%
\pgfsetlinewidth{0.000000pt}%
\definecolor{currentstroke}{rgb}{0.000000,0.000000,0.000000}%
\pgfsetstrokecolor{currentstroke}%
\pgfsetstrokeopacity{0.000000}%
\pgfsetdash{}{0pt}%
\pgfpathmoveto{\pgfqpoint{5.327193in}{0.521603in}}%
\pgfpathlineto{\pgfqpoint{5.327193in}{2.446603in}}%
\pgfusepath{}%
\end{pgfscope}%
\begin{pgfscope}%
\pgfsetrectcap%
\pgfsetmiterjoin%
\pgfsetlinewidth{0.803000pt}%
\definecolor{currentstroke}{rgb}{0.000000,0.000000,0.000000}%
\pgfsetstrokecolor{currentstroke}%
\pgfsetdash{}{0pt}%
\pgfpathmoveto{\pgfqpoint{0.677193in}{0.521603in}}%
\pgfpathlineto{\pgfqpoint{5.327193in}{0.521603in}}%
\pgfusepath{stroke}%
\end{pgfscope}%
\begin{pgfscope}%
\pgfsetrectcap%
\pgfsetmiterjoin%
\pgfsetlinewidth{0.000000pt}%
\definecolor{currentstroke}{rgb}{0.000000,0.000000,0.000000}%
\pgfsetstrokecolor{currentstroke}%
\pgfsetstrokeopacity{0.000000}%
\pgfsetdash{}{0pt}%
\pgfpathmoveto{\pgfqpoint{0.677193in}{2.446603in}}%
\pgfpathlineto{\pgfqpoint{5.327193in}{2.446603in}}%
\pgfusepath{}%
\end{pgfscope}%
\end{pgfpicture}%
\makeatother%
\endgroup%

\caption{Empricial probability (calculated from 1000 realisations) that $\GMRES{\eps}{\no}{\nt}\leq 12$ for $k = 10, 20, 30, 40,$ where $R=12$, $\eps = 10^{-5}$, $N = \ceil{k^{3}}$, and $\Ct=0.1,$ with $\NLiDRR{\no-\nt} \sim \Exp{\sigma}$ with $\sigma = 1.$\label{fig:prob-plot-2.0}}
\end{figure}


\subsection{Application to Quasi-Monte-Carlo methods for the Helmholtz equation}\label{sec:nbpcqmc}

We now apply nearby preconditioning to a Quasi-Monte-Carlo (QMC) method for the Helmholtz equation. We begin with a brief description of QMC methods, before detailing two ways we apply nearby preconditioning to these methods. Finally, we give computational results for applying nearby preconditioning to QMC methods for the Helmholtz equation.

\subsubsection{Brief description of QMC}

QMC methods (or rules) are high-dimensional integration rules, designed to yield superior rates of convergence (with respect to the number of integration points) compared to Monte-Carlo methods. Suppose one wants to approximate $\EXP{Q},$ where $Q$ is some random variable (later in this \lcnamecref{sec:nbpcqmc}, $Q$ will be a function of a stochastic PDE). By definition, the expectation is
\beq\label{eq:qmcexpdef}
\EXP{Q} = \int_\Omega Q(\omega) \ddPPomega.
\eeq

If we now suppose $Q$ depends on the sample space via a finite set of random variables $\Uo,\ldots,\UJ$, then we can rewrite \cref{eq:qmcexpdef} as
\beq\label{eq:qmcexp2}
\EXP{Q} = \int_\Omega Q\mleft((\Uo(\omega),\ldots,\UJ(\omega)\mright) \ddPPomega.
\eeq
If, for example, the $\Uj$ are all Uniform random variables on $\mleft[-1/2,1/2\mright]$, then \cref{eq:qmcexp2} can be rewritten as
\beq\label{eq:qmcexp3}
\EXP{Q} = \int_{\cube{J}} Q\mleft(\by\mright) \dd\lambda,
\eeq
where $\by \in \cube{J}$ and $\lambda$ denotes Lebesgue measure.

Any integration rule, or method for approximating $\EXP{Q}$, can then be seen as a method for approximating the $J$-dimensional integral on the right-hand side of \cref{eq:qmcexp3}. Sampling-based rules will choose points $\byo,\ldots,\byNpoints \in \cube{J}$ and use the approximation
\beqs
\EXP{Q} \approx \frac1{\Npoints}\sum_{l=1}^{\Npoints} Q\mleft(\byl\mright).
\eeqs
Monte-Carlo and Quasi-Monte-Carlo rules are methods for choosing the points $\byl$. In a Monte-Carlo rule the points are chosen at random in accordance with the associated probability distribution. For example, in this case, the points are chosen according to the Uniform distribution on $\cube{J}$. Observe that Monte-Carlo rules do not need the dependence of $Q$ on $\omega$ to take the form prescribed above, they apply to any random variable.

Quasi-Monte-Carlo rules, in contrast to Monte-Carlo rules, do require the dependence on $\omega$ to take the form prescribed above, as QMC rules are high-dimensional integration rules (where we are integrating across the high-dimensional cube $\cube{J}).$ In QMC rules the points $\byl$ are not chosen completely at random, unlike Monte-Carlo rules.

The advantages of QMC rules is that they can exhibit higher rates of convergence compared to Monte-Carlo rules; Monte Carlo rules typically converge with rate $1/\Npoints^{1/2}$ (see\ednote{I'll sketch this in the MLMC chapter, and add in a reference once it's done.}), whereas QMC rules can converge with rates up to $1/\Npoints$  or with even higher rates for higher-order QMC rules, see, e.g., \cite[Penultimate paragraph of Section 1.2]{KuNu:16}.

In applying QMC rules to stochastic PDEs, we assume that the random coefficient is dependent on fintely many (or countably many) random variables, as in \cref{eq:qmcexp2} above, and we then use QMC rules to estimate expectations of quantities of interest for the stochastic PDE. Seee the next \lcnamecref{sec:nbpcqmcnum} for this setup worked out in more detail for the Helmholtz equation. Applying QMC rules to stochastic PDEs is a vibrant and active research area. For recent overviews of this field, see \cite{KuNu:16} (and the associated tutorial \cite{KuNU:18a}) and \cite{KuNu:18b}. We observe that there is currently no rigorous study of how QMC methods behave for the Helmholtz equation, although we undertstand such work is underway in \cite{GaKuSl}.

The key idea in applying nearby preconditioning to QMC methods for the Helmholtz equation is that we do not just choose one realisation $\no$ of the coefficient for which to calculate the preconditioner, rather, we choose as many as a are necessary to ensure the number of GMRES iterations across all samples remains bounded. The key questions the algorithms in this \lcnamecref{sec:nbpcqmc} answer are therefore: How many preconditioners should one calculate, and for which realisations of the coefficient should they be calculated?

\subsubsection{Methods for applying nearby preconditioning to QMC}\label{sec:nbpcqmcnum}
We now detail two methods for using nearby preconditioning to speed up QMC methods for the Helmholtz equation. To apply these methods, we use the following model problem: We consider the Interior Impedance Problem in 2-d with $f=1$ and $\gI=0$, $A = I$ and $n$ given by
\beq\label{eq:artificialkl}
n(\omega,\bx) = 1 + \sum_{j=1}^{10} \Uj(omega) \sqrt{\lambdaj} \psij(\bx),
\eeq
where
\beqs
\sqrt{\lambdaj} = j^{-2}
\eeqs
and
\beqs
\psij(\bx) = \cos\mleft(\frac{j\pi}4 x\mright)\cos\mleft(\frac{\mleft(j+1\mright)\pi}4 y\mright).
\eeqs
Observe that $\NLiDRR{\psij}=1$ for all $j,$ and $\sqrt{\lambdaj} \rightarrow 0$ as $j \rightarrow \infty.$ This expansion is based on the random-field expansion in \cite[Section 5.1]{GiGrKuScSl:19}. Expansions similar to \cref{eq:artificalkl} are often decribed as `artificial Karhunen--Lo\`eve expansions' due to their similarity with the Karhunen--Loe\`eve expansion of a random field, where the $\Uj$ are independent random variables, and the $\lambdaj$ and $\psij$ are the eigenvalues and eigenvectors of the covariance operator, see, e.g., \cite[Section 7.4]{LoPoSh:14}. In view of the fact that we will be using QMC methods to approximate $\EXP{Q(u)}$ (for some quantities of interest $Q$) we will sometimes instead write $n(\by)$ for $\by \in \cube{10}$, by which we mean
\beqs
n(\by) = 1 + 1 + \sum_{j=1}^{10} \by_{j} \sqrt{\lambdaj} \psij.
\eeqs
There is no a priori reason that one must have such an affine dependence of the random field on the randomness in order to apply nearby preconditioning to QMC methods (one could, for example, take $n$ to be a lognormal random field). However, affine dependence will allow us to easily define the so-called `paralleisable' nearby-preconditioning-QMC algorithm below.

We stress that the results in this \lcnamecref{sec:nbpcqmcnum} are strictly numerical; there is no current theory to support these calculations. In particular, we show in \cref{sec:nbpcqmcnumerics} below that it appears that for the QMC error for Helmholtz problems to remain bounded as $k$ increases, one must increase the number of QMC points with $k.$ We again remark that there is currently no theoretical justification for this behaviour.


\paragraph{Terminology} Before we describe the nearby-preconditioning-QMC algorithms in detail we establish two pieces of terminology that will be of use in describing these algorithms. Firstly, we will use the word `point' to refer to a point in the parameter space $\cube{J}$, and use phrases such as `calculate a preconditioner at the point $\by$' as shorthand for `calculate the preconditioner corresponding to the finite-element discretisation of the Helmholtz IIP (as described above) with coefficient $n(\by)$'.

We we alse use the words `nearby' and `nearest' (when referring to QMC points) to mean: nearest in the metric
\beqs
\dQMC(\byo,\byt) = \NLiDRRR{n(\byo)-n(\byt)}.
\eeqs
We use this metric to describe the geometry of our QMC points as our results in \cref{sec:intronbpc} above indicate that it is the $L^\infty$-norm of the different in the coefficients that dictates the effectiveness of nearby preconditioning. Therefore, when considering which QMC points will yield preconditioners suitable for use with other QMC points, this metric is a natural metric to use.


\paragraph{A sequential algorithm} We first describe a straightforward algorithm that uses nearby preconditioning to speed up QMC calculation for the Helmholtz equation. We call this a `sequential' algorithm because, unlike the `parallel' algorithm we describe below, it is intrinsically sequential and cannot be parallelised. I.e., finite-element solves for different realisations of the random field $n$ cannot be treated in parallel. One can, of course, use parallelisation for inidividual finite-element solves.

An overview of the algorithm is:
\ben
\item Choose a point $\by$ for which to calculate a preconditioner
\item\label{it:nearest} Find the nearest non-computed point and attempt a preconditioned GMRES solve at that point.
    \item If GMRES does not take too long to converge, return to \cref{it:nearest}.
\item If GMRES takes too long to converge, recalculate the preconditioner at the current point, and return to \cref{it:nearest}.
  \een
  The algorithm is written in more formal pseudocode in \cref{alg:seq}.
\begin{algorithm}[h]
\DontPrintSemicolon
\SetKwInOut{Input}{input}\SetKwInOut{Output}{output}
\SetKwFunction{Nearest}{nearest}

\Input{$\maxGMRES$,$\SQMC$}
\BlankLine
Choose starting point $\bystart$\;
$\bypre \defined \bystart$\;
$\Sremaining \defined \SQMC\setminus\set{\bypre}$\;
Calculate preconditioner $\Pmat$ at $\bypre$\;
$\bycurrent \defined$ \Nearest{$\bypre,\Sremaining$}\;
\While{$\Sremaining \neq \emptyset$}{
\lIf{GMRES applied to $\Pmat \Amat(\bycurrent) = \Pmat \bb$ converges in fewer than $\maxGMRES$ iterations}{
$\Sremaining \defined \Sremaining\setminus\set{\bycurrent}$\;
$\bycurrent \defined$ \Nearest{$\bypre,\Sremaining$}\;
}
\Else{
$\bypre \defined \bycurrent$\;
Calculate preconditioner $\Pmat$ at $\bypre$\;
}
}
\caption{Algorithm to perform all solves in a QMC method using nearby preconditioning\label{alg:seq}}
\end{algorithm}
\paragraph{Parallelisable algorithm} The main disadvantage of the `sequential' algorithm described above is that the points at which preconditioners are calculated are identified as the algorithm progresses. Therefore the algorithm cannot be parallelised by sending different collections of QMC points to different processors (as one would need to know which preconditioner to use for each point at the start of the algorithm). Therefore, we now suggest an alternative algorithm that allows one to specify the number of preconditioning points before the algorithm commences, and then calculates which points to use as preconditioning points, and then performs the all of the QMC solves, potentially in parallel if required. The most complicated part of the algorithm is deciding at which points to calculate the preconditioners, and so we describe this part of the algorithm in more detail here. A more formal description of the algorithm is available in \cref{alg:par}.

Suppose we are given a set $\set{\byo,\ldots,\byNQMC}$ of QMC points and a number $\Npretarget$; the target number of preconditioners to compute. The aim of this algorithm is to select (approximately) $\Npretarget$ QMC points that are (approximately) equally spaced with respect to the $\dQMC$ metric defined above. If such a goal is acheived, then one expects that the preconditioning points are best located to minimise the total number of GMRES iterations across all solves across all of the QMC points.

The algorithm contains two key ideas:
\ben
  \item Use a surrogate metric in place of $\dQMC$, and
\item Locate the preconditioning points according to a tensor-product rule.
  \een
  We now describe each of these two ideas in turn.

  Whilst the metric $\dQMC$ is the metric related to the performance of nearby preconditioning (as described in \cref{sec:intronbpc} above), in practice $\dQMC$ is difficult to work with; it is not obvious what geometry it induces on $\cube{J}$, nor how to theoretically work with it (we ignore the issue of \emph{computing} $\dQMC$ here). Therefore, we work in an alternative, although related metric
  \beqs
\dapprox(\byo,\by) = \sum_{j=1}^{J} \sqrt{\lambdaj} \abs{{\byo}_{j} - {\byt}_{j}}.
\eeqs
Observe that $\dapprox$ is a weighted $L^1$ metric, with the weights corresponding to the terms in \cref{eq:artificalkl}. Recall that $\sqrt{\lambdaj} \rightarrow 0$ as $j \rightarrow \infty$; therefore the higher dimensions contribute less to the value of $\dapprox$ (or, informally, points are `closer' in higher dimensions, or higher dimensions are `smaller' than lower dimensions). However, it is easy to compute with $\dapprox,$ and it is obvious that it enables one to think of $\cube{J}$ as the high-dimensional rectangle $\mleft[0,\sqrt{\lambdao}\mright]\times\cdots\times\mleft[0,\sqrt{\lambdaJ}\mright]$ equipped with the standard $L^1$ metric.

To understand why we use locate the preconditioning points using a tensor-product rule, we first decribe the heuristic we use. Let us assume we want to cover $\cube{J}$ with balls of radius $r$ (where these balls are measured in the $\dapprox$ metric). Therefore, given the centres of two of these balls $\bcone$ and $\bct$ (if we suppose these are the centres of `adjacent' balls), then we will have
\beq\label{eq:centres2r}
\dapprox(\bcone,\bct) = 2r.
\eeq
The question now arises of how we choose $\bcone$ and $\bct$ so that \cref{eq:centres2r} holds. We observe that, by the definition of $\dapprox$, if we choose $\bcone$ and $\bct$ such that
\beqs
\sqrt{\lambdaj}\abs{{\bcone}_{j}-{\bct}_j} = \frac{2r}{J},
\eeqs
then we will have \cref{eq:centres2r} by construction, because
\beqs
\dapprox(\bcone,\bct) = \sum_{j=1}^J \frac{2r}J = 2r.
\eeqs
Therefore, in dimension $j$ we choose the centres of the balls to be spaced
\beqs
\min\set{\frac{2r}{J\sqrt{\lambdaj}},1}
\eeqs
apart (where we include the minimum so that, for high dimensions, we include at least one centre). That is, in dimension $j$, we take
\beqs
\Nj \de \max\set{1,\frac{J\sqrt{\lambdaj}}{2r}}
\eeqs
equally spaced points $\centresj = \set{c_{j,1},\ldots,c_{j,\Nj}},$ and then we form the centres $\bcone,\ldots,\bcNpre$ by taking tensor products of the points in $\centreso,\ldots,\centresJ,$ giving a total of $\Npre = \No \times \cdots \times \NJ$ preconditioning points.

However, there are three immediate objections to the above approach:
\ben
\item The above procedure assumes we know the radius of the balls we wish to construct, and then returns the total number of preconditioning points, and their locations, but we only know in advance the ideal total number of preconditioning points.
\item There is no guarantee that the numbers of points $\Nj$ calculated above are integers.
  \item There is no guarantee the preconditioning points given by the above procedure are QMC points.
    \een
    These questions are all completely valid, and so we slightly modify the above procedure to deal with them.

    Recall that we assume that we are given a target number of preconditioners $\Npretarget$. The above procedure (amongst other things) defines a map $\Npreideal:\RRp \rightarrow \RRp$ given by $r \mapsto \Npre.$ Therefore we can calculate numerically the value $\rideal$ such that $\Npreideal(\rideal) = \Npretarget.$ (In our computations, we do this via interval bisection.)

    Once we know $\rideal,$ we can calculate the numbers of centres in each dimension $\Npreidealo,\ldots,\NpreidealJ$ as above (recalling that the $\Npreidealj$ are not necessarily integers). We then obtain integers $\Npreactualj = \round{\Npreidealj}$, where $\round{\cdot}$ denotes rounding to the nearest integer. (Recall $\Npreidealj \geq 1$ for all $j$ by construction, so $\Npreactualj$ will be a positive integer for all $j.$)

We then take $\Npreactualj$ centres in each dimension, as described above. We then obtain a total of $\Npreactual = \Npreactualo \times \cdots \times \NpreactualJ$ peconditioning points.

These points may not be QMC points. We could simply calculate the preconditioners at these non-QMC points. However we instead calculate the preconditioners at QMC points, and so we simply replace each calculated centre with its nearest QMC point.

    This algorithm is summarised more formally in \cref{alg:par}.
    
    

%% Define
%% \beqs
%% \Npreidealj(r) = \max\set{\frac{J \sqrt{\lambdaj}}{2r},1}.
%% \eeqs
%% The `ideal' total number of QMC points is
%% \beqs
%% \Npreideal(r)=\prod_{j=1}^J  \Npreidealj(r)
%% \eeqs

%% Want to calculate the number of preconditioners $\Npre$, the set
%% \beqs
%% \Spre=\set{\ypreo,\ldots,\ypreNpre}
%% \eeqs
%% of QMC points at which to calculate the preconditioner and the map
%% \beqs
%% \nearestpre:\SQMC\rightarrow\Spre
%% \eeqs
%% taking each QMC point to its nearest (in the induced spatial $L^\infty$ norm) preconditioner, where $\SQMC$ is the set of QMC points.

\begin{algorithm}[h]
\DontPrintSemicolon
\SetKwInOut{Input}{input}\SetKwInOut{Output}{output}
\SetKwFunction{Round}{round}

\Input{$\Npretarget \in \NN$}
\Output{$\Spre$, $\nearestpre$}
\BlankLine
Solve (numerically) $\Npreideal(\rideal) = \Npretarget$ for $\rideal$\;
\For{j $= 1$ \KwTo $J$}{
Calculate $\Npreactualj =$ \Round{$\Npreidealj(\rideal)$}\;
Define $\Sprej$ to be set of $\Npreactualj$ equally spaced points in $\mleft[-1/2,1/2\mright]$\;
}
Define $\displaystyle\Npre = \prod_{j=1}^J \Npreactualj$\;
Define $\Spre$ by taking all possible tensor products of points in $\Sprej$\;
\For{l $=1$ \KwTo $\NQMC$}{
Calculate $\nearestpre\mleft(\by^{(l)}\mright)$ by brute force\;
}
\caption{Algorithm to determine $\Spre$ and $\nearestpre$\label{alg:par}}
\end{algorithm}

\paragraph{Advantages and disadvantages of each method} The advantages of the sequential algorithm are:
\bit
\item Its simplicity; the algorithm is simple and intuitive to describe, and
\item Its lack of heuristics - one only needs to specify the maximum number of GMRES iterations; this could be determined, for example by the memory constraints of the machine one is using.
    \eit
    However, the disadvantages of the sequential algorithm re:
    \bit
  \item The algorithm is inherently serial; one must see whether a given solve converges in the required number of GMRES iterations before performing subsequent solves. (In principle one could parallelise the algorithm by splitting the QMC points up onto different groups of processors, and then using the sequential algorithm on each group of processors. However, there is no guarantee one would split the QMC points up in a way that groups nearby points, which could lead to a substantial increase in computational work.)
    \item There is no guarantee that this method for exploring the sample space and choosing the preconditioning points will yield an optimal collection of preconditioning points (optimal in the sense of the minimal number of preconditioning points needed).
    \eit

    The advantages and disadvantages of the parallel algorithm are, by and large, the reverse of those for the sequential algorithm. The advantages of the parallel algorithm are:
    \bit
  \item The algorithm is fully parallelisable; once the preconditioning points and the map from QMC points to preconditioning points have been calculated, one can send different linear solves to different groups of processors as one chooses. (Although note that, unless one sends all of the QMC points corresponding to a single preconditioner to the \emph{same} group of processors, one may need to calculate the same preconditioner multiple times, on different groups of processes\footnote{In our code, we split up the points with respect to the order they are generated by the QMC code. This was purely to make the code simpler.} However, the decrease in computational time gained from parallelisation should more than offset this increase in computational effort.)
    \item The preconditioning points should fill the parameter space `well'. Given the points are chosen a priori to be well spaced according to the $\dapprox$ metric, one expects they will be a close to optimal collection (in the sense described above).
      \eit
      The disadvantages of the parallel algorithm are:
      \bit
    \item One needs a heuristic for how many preconditioning points to choose, as this is not given by the algorithm. (In our numerical experiments below, we obtain this heuristic by using the sequential algorithm for low $k$, and then extrapolating the proportion of preconditioning points used to larger values of $k.$
      \item The number of preconditioning points generated is not exactly $\Npretarget$ due to rounding the `ideal' number of centres in each dimension to the nearest integer. However, we expect the number of generated points will be close $\Npretarget$.
      \eit


\subsubsection{Numerical Experiments}\label{sec:nbpcqmcnumerics}
We now describe numerical experiments demonstrating the effectiveness of nearby preconditioning for tackling Helmholtz problems.

As described above, before we perform our numerical experiments, we need to determine:
\bit
\item How the number of QMC points should scale with $k$, and
  \item How many preconditioners we should choose.
    \eit
    We tackle each of these in turn, before using the parallelisable algorithm detail above to apply nearby preconditioning to QMC methods for the Helmholtz equation. Throughout this \lcnamecref{sec:nbpcqmcnumerics} we use the model problem detailed above.

    To determine how the number of QMC points should scale with $k$, we first estimate the QMC error for increasing $k.$ To estimate the QMC error we use a randomly shifted QMC rule; our exposition here follows \cite[Section 4.2]{GrKuNuScSl:11}.

    Suppose our QMC points are $\byo,\ldots,\byNQMC$, and the resulting QMC rule is
    \beqs
\QMC{\NQMC}{Q} = \frac1{\NQMC}\sum_{l=1}^{\NQMC} Q\mleft(u\mleft(\byl\mright)\mright).
\eeqs
For a `shift' $\shift \in \cube{J}$ we define the shifted QMC rule
\beqs
\QMCshift{\NQMC}{Q}{\shift} = \frac1{\NQMC}\sum_{l=1}^{\NQMC} Q\mleft(u\mleft(\byl\oplus\shift\mright)\mright),
\eeqs
where $\by \oplus \shift$ denotes $\by + \shift$ `wrapped around' onto the hypercube $\cube{J}$. (Formally $\by \oplus \shift = \fracoperator{\mleft(\by + \bhalf\mright)+\shift} - \bhalf,$ where $\fracoperator{\cdot}$ denotes the fractional part.)

We then define the randomly shifted QMC rule (with multiple randomly chosen shifts $\shifto,\ldots,\shiftNshifts$)
\beqs
\QMCrandshift{Q}{\Nshifts} = \frac1{\Nshifts}\sum_{s=1}^{\Nshifts} \QMCshift{Q}{\shifts} = \frac1{\NQMC\Nshifts}\sum_{s=1}^{\Nshifts}\sum_{l=1}^{\NQMC} Q\mleft(u\mleft(\byl\oplus \shifts\mright)\mright).
\eeqs

Having defined the randomly shifted QMC rule, one can the use the standard statistical estimator of the standard deviation of the error in $\QMCrandshift{Q}{\Nshifts}$ \cite[Equation (4.6)]{GrKuNuScSl:11}
\beqs
\QMCerror{\NQMC}{\Nshifts} = \mleft(\frac1{\Nshifts\mleft(\Nshifts-1\mright)}\sum_{s=1}^{\Nshifts} \mleft(\QMCshift{Q}{\shifts} - \QMCrandshift{Q}{\Nshifts}\mright)^2\mright)^{\half}.
\eeqs
(See \cref{app:complexerror} for proof that $\QMCerror{\NQMC}{\Nshifts}^2$ is an unbiased estimator of the variance of $\QMCrandshift{Q}{\Nshifts}$; recall that it does \emph{not} follow that $\QMCerror{\NQMC}{\Nshifts}$ is an \emph{unbiased} estimator of the standard deviation of $\QMCrandshift{Q}{\Nshifts}$.)

We estimated $\QMCerror{\NQMC}{\Nshifts}$ for the setup described in \cref{app:compsetup}with $2048$ QMC points and $\Nshifts=20$ (i.e., 40,960 PDE solves in total) for $k = 10,20,30,40,50,60$. We set $h = 0.002$ for all of the computations (as $0.002 \approx 60^{-3/2}$) to avoid having to consider the effect of finite-element error. The quantities of interest we considered were:
\bit
\item The integral of the solution over the whole domain,
\item The value of the solution at the origin,
\item The value of the solution at the top-right corner of the domain, and
\item The $x$-component of the gradient of the solution at the top-right corner of the domain.
  \eit
  Observe that these QOIs require increasing regularity of the solution (the integral is defined for functions in $\LoD$, point evaluation for functions in $\Hfn{}{3/2 + \eps}{D}$ and point evaluation of the gradient for functions in $\Hfn{}{5/2+\eps}{D}$ (in 3-d - the corresponding function spaces are $\Hfn{}{1+\eps}{D}$ and $\Hfn{}{2+\eps}{D}$ in 2-d) for any $\eps > 0.$), and so computing for this range of quantities of interest will give a good insight into the behaviour of QMC applied to the Helmholtz equation for a wide range of QoIs\footnote{We can evaluate point values of $\uh$ because $\uh$ is continuous, and we evaluate $\grad\uh((1,1))$ as the value of $\grad \uh$ on the upper-rightmost mesh element; such an evaluation is possible due the structure of our mesh; see \cref{fig:grid}.}.
%%     That is, we randomly choose $\shifto,\ldots,\shiftNshifts$ points in $\cube{J}$ (the `shifts')rause the standard error estimator
%%     \beqs
%%     \mleft(\frac1{\nu\mleft(\nu-1\mright)} \sum_{s=1}^{\Nshifts} 
%%     \eeqs$h = 0.002$ (relation to $k=60$ - maximum?)

In \cref{fig:integralCalpha,fig:originCalpha,fig:toprightCalpha,fig:gradienttoprightCalpha} we plot how $C$ and $\alpha$ depend on $k$, for the plots of the QMC error with increasing $\NQMC$ for each value of $k,$ see \cref{app:hhqmcconv}.

\begin{figure}[h]
    \centering
  \begin{subfigure}{\textwidth}
%% Creator: Matplotlib, PGF backend
%%
%% To include the figure in your LaTeX document, write
%%   \input{<filename>.pgf}
%%
%% Make sure the required packages are loaded in your preamble
%%   \usepackage{pgf}
%%
%% Figures using additional raster images can only be included by \input if
%% they are in the same directory as the main LaTeX file. For loading figures
%% from other directories you can use the `import` package
%%   \usepackage{import}
%% and then include the figures with
%%   \import{<path to file>}{<filename>.pgf}
%%
%% Matplotlib used the following preamble
%%   \usepackage{fontspec}
%%   \setmainfont{DejaVuSerif.ttf}[Path=/home/owen/progs/firedrake-complex/firedrake/lib/python3.5/site-packages/matplotlib/mpl-data/fonts/ttf/]
%%   \setsansfont{DejaVuSans.ttf}[Path=/home/owen/progs/firedrake-complex/firedrake/lib/python3.5/site-packages/matplotlib/mpl-data/fonts/ttf/]
%%   \setmonofont{DejaVuSansMono.ttf}[Path=/home/owen/progs/firedrake-complex/firedrake/lib/python3.5/site-packages/matplotlib/mpl-data/fonts/ttf/]
%%
\begingroup%
\makeatletter%
\begin{pgfpicture}%
\pgfpathrectangle{\pgfpointorigin}{\pgfqpoint{5.000000in}{4.000000in}}%
\pgfusepath{use as bounding box, clip}%
\begin{pgfscope}%
\pgfsetbuttcap%
\pgfsetmiterjoin%
\definecolor{currentfill}{rgb}{1.000000,1.000000,1.000000}%
\pgfsetfillcolor{currentfill}%
\pgfsetlinewidth{0.000000pt}%
\definecolor{currentstroke}{rgb}{1.000000,1.000000,1.000000}%
\pgfsetstrokecolor{currentstroke}%
\pgfsetdash{}{0pt}%
\pgfpathmoveto{\pgfqpoint{0.000000in}{0.000000in}}%
\pgfpathlineto{\pgfqpoint{5.000000in}{0.000000in}}%
\pgfpathlineto{\pgfqpoint{5.000000in}{4.000000in}}%
\pgfpathlineto{\pgfqpoint{0.000000in}{4.000000in}}%
\pgfpathclose%
\pgfusepath{fill}%
\end{pgfscope}%
\begin{pgfscope}%
\pgfsetbuttcap%
\pgfsetmiterjoin%
\definecolor{currentfill}{rgb}{1.000000,1.000000,1.000000}%
\pgfsetfillcolor{currentfill}%
\pgfsetlinewidth{0.000000pt}%
\definecolor{currentstroke}{rgb}{0.000000,0.000000,0.000000}%
\pgfsetstrokecolor{currentstroke}%
\pgfsetstrokeopacity{0.000000}%
\pgfsetdash{}{0pt}%
\pgfpathmoveto{\pgfqpoint{0.625000in}{0.440000in}}%
\pgfpathlineto{\pgfqpoint{4.500000in}{0.440000in}}%
\pgfpathlineto{\pgfqpoint{4.500000in}{3.520000in}}%
\pgfpathlineto{\pgfqpoint{0.625000in}{3.520000in}}%
\pgfpathclose%
\pgfusepath{fill}%
\end{pgfscope}%
\begin{pgfscope}%
\pgfsetbuttcap%
\pgfsetroundjoin%
\definecolor{currentfill}{rgb}{0.000000,0.000000,0.000000}%
\pgfsetfillcolor{currentfill}%
\pgfsetlinewidth{0.803000pt}%
\definecolor{currentstroke}{rgb}{0.000000,0.000000,0.000000}%
\pgfsetstrokecolor{currentstroke}%
\pgfsetdash{}{0pt}%
\pgfsys@defobject{currentmarker}{\pgfqpoint{0.000000in}{-0.048611in}}{\pgfqpoint{0.000000in}{0.000000in}}{%
\pgfpathmoveto{\pgfqpoint{0.000000in}{0.000000in}}%
\pgfpathlineto{\pgfqpoint{0.000000in}{-0.048611in}}%
\pgfusepath{stroke,fill}%
}%
\begin{pgfscope}%
\pgfsys@transformshift{0.801136in}{0.440000in}%
\pgfsys@useobject{currentmarker}{}%
\end{pgfscope}%
\end{pgfscope}%
\begin{pgfscope}%
\definecolor{textcolor}{rgb}{0.000000,0.000000,0.000000}%
\pgfsetstrokecolor{textcolor}%
\pgfsetfillcolor{textcolor}%
\pgftext[x=0.801136in,y=0.342778in,,top]{\color{textcolor}\sffamily\fontsize{10.000000}{12.000000}\selectfont \(\displaystyle {10^{1}}\)}%
\end{pgfscope}%
\begin{pgfscope}%
\pgfsetbuttcap%
\pgfsetroundjoin%
\definecolor{currentfill}{rgb}{0.000000,0.000000,0.000000}%
\pgfsetfillcolor{currentfill}%
\pgfsetlinewidth{0.602250pt}%
\definecolor{currentstroke}{rgb}{0.000000,0.000000,0.000000}%
\pgfsetstrokecolor{currentstroke}%
\pgfsetdash{}{0pt}%
\pgfsys@defobject{currentmarker}{\pgfqpoint{0.000000in}{-0.027778in}}{\pgfqpoint{0.000000in}{0.000000in}}{%
\pgfpathmoveto{\pgfqpoint{0.000000in}{0.000000in}}%
\pgfpathlineto{\pgfqpoint{0.000000in}{-0.027778in}}%
\pgfusepath{stroke,fill}%
}%
\begin{pgfscope}%
\pgfsys@transformshift{2.163913in}{0.440000in}%
\pgfsys@useobject{currentmarker}{}%
\end{pgfscope}%
\end{pgfscope}%
\begin{pgfscope}%
\definecolor{textcolor}{rgb}{0.000000,0.000000,0.000000}%
\pgfsetstrokecolor{textcolor}%
\pgfsetfillcolor{textcolor}%
\pgftext[x=2.163913in,y=0.365000in,,top]{\color{textcolor}\sffamily\fontsize{10.000000}{12.000000}\selectfont \(\displaystyle {2\times10^{1}}\)}%
\end{pgfscope}%
\begin{pgfscope}%
\pgfsetbuttcap%
\pgfsetroundjoin%
\definecolor{currentfill}{rgb}{0.000000,0.000000,0.000000}%
\pgfsetfillcolor{currentfill}%
\pgfsetlinewidth{0.602250pt}%
\definecolor{currentstroke}{rgb}{0.000000,0.000000,0.000000}%
\pgfsetstrokecolor{currentstroke}%
\pgfsetdash{}{0pt}%
\pgfsys@defobject{currentmarker}{\pgfqpoint{0.000000in}{-0.027778in}}{\pgfqpoint{0.000000in}{0.000000in}}{%
\pgfpathmoveto{\pgfqpoint{0.000000in}{0.000000in}}%
\pgfpathlineto{\pgfqpoint{0.000000in}{-0.027778in}}%
\pgfusepath{stroke,fill}%
}%
\begin{pgfscope}%
\pgfsys@transformshift{2.961087in}{0.440000in}%
\pgfsys@useobject{currentmarker}{}%
\end{pgfscope}%
\end{pgfscope}%
\begin{pgfscope}%
\definecolor{textcolor}{rgb}{0.000000,0.000000,0.000000}%
\pgfsetstrokecolor{textcolor}%
\pgfsetfillcolor{textcolor}%
\pgftext[x=2.961087in,y=0.365000in,,top]{\color{textcolor}\sffamily\fontsize{10.000000}{12.000000}\selectfont \(\displaystyle {3\times10^{1}}\)}%
\end{pgfscope}%
\begin{pgfscope}%
\pgfsetbuttcap%
\pgfsetroundjoin%
\definecolor{currentfill}{rgb}{0.000000,0.000000,0.000000}%
\pgfsetfillcolor{currentfill}%
\pgfsetlinewidth{0.602250pt}%
\definecolor{currentstroke}{rgb}{0.000000,0.000000,0.000000}%
\pgfsetstrokecolor{currentstroke}%
\pgfsetdash{}{0pt}%
\pgfsys@defobject{currentmarker}{\pgfqpoint{0.000000in}{-0.027778in}}{\pgfqpoint{0.000000in}{0.000000in}}{%
\pgfpathmoveto{\pgfqpoint{0.000000in}{0.000000in}}%
\pgfpathlineto{\pgfqpoint{0.000000in}{-0.027778in}}%
\pgfusepath{stroke,fill}%
}%
\begin{pgfscope}%
\pgfsys@transformshift{3.526690in}{0.440000in}%
\pgfsys@useobject{currentmarker}{}%
\end{pgfscope}%
\end{pgfscope}%
\begin{pgfscope}%
\definecolor{textcolor}{rgb}{0.000000,0.000000,0.000000}%
\pgfsetstrokecolor{textcolor}%
\pgfsetfillcolor{textcolor}%
\pgftext[x=3.526690in,y=0.365000in,,top]{\color{textcolor}\sffamily\fontsize{10.000000}{12.000000}\selectfont \(\displaystyle {4\times10^{1}}\)}%
\end{pgfscope}%
\begin{pgfscope}%
\pgfsetbuttcap%
\pgfsetroundjoin%
\definecolor{currentfill}{rgb}{0.000000,0.000000,0.000000}%
\pgfsetfillcolor{currentfill}%
\pgfsetlinewidth{0.602250pt}%
\definecolor{currentstroke}{rgb}{0.000000,0.000000,0.000000}%
\pgfsetstrokecolor{currentstroke}%
\pgfsetdash{}{0pt}%
\pgfsys@defobject{currentmarker}{\pgfqpoint{0.000000in}{-0.027778in}}{\pgfqpoint{0.000000in}{0.000000in}}{%
\pgfpathmoveto{\pgfqpoint{0.000000in}{0.000000in}}%
\pgfpathlineto{\pgfqpoint{0.000000in}{-0.027778in}}%
\pgfusepath{stroke,fill}%
}%
\begin{pgfscope}%
\pgfsys@transformshift{3.965406in}{0.440000in}%
\pgfsys@useobject{currentmarker}{}%
\end{pgfscope}%
\end{pgfscope}%
\begin{pgfscope}%
\pgfsetbuttcap%
\pgfsetroundjoin%
\definecolor{currentfill}{rgb}{0.000000,0.000000,0.000000}%
\pgfsetfillcolor{currentfill}%
\pgfsetlinewidth{0.602250pt}%
\definecolor{currentstroke}{rgb}{0.000000,0.000000,0.000000}%
\pgfsetstrokecolor{currentstroke}%
\pgfsetdash{}{0pt}%
\pgfsys@defobject{currentmarker}{\pgfqpoint{0.000000in}{-0.027778in}}{\pgfqpoint{0.000000in}{0.000000in}}{%
\pgfpathmoveto{\pgfqpoint{0.000000in}{0.000000in}}%
\pgfpathlineto{\pgfqpoint{0.000000in}{-0.027778in}}%
\pgfusepath{stroke,fill}%
}%
\begin{pgfscope}%
\pgfsys@transformshift{4.323864in}{0.440000in}%
\pgfsys@useobject{currentmarker}{}%
\end{pgfscope}%
\end{pgfscope}%
\begin{pgfscope}%
\definecolor{textcolor}{rgb}{0.000000,0.000000,0.000000}%
\pgfsetstrokecolor{textcolor}%
\pgfsetfillcolor{textcolor}%
\pgftext[x=4.323864in,y=0.365000in,,top]{\color{textcolor}\sffamily\fontsize{10.000000}{12.000000}\selectfont \(\displaystyle {6\times10^{1}}\)}%
\end{pgfscope}%
\begin{pgfscope}%
\definecolor{textcolor}{rgb}{0.000000,0.000000,0.000000}%
\pgfsetstrokecolor{textcolor}%
\pgfsetfillcolor{textcolor}%
\pgftext[x=2.562500in,y=0.152809in,,top]{\color{textcolor}\sffamily\fontsize{10.000000}{12.000000}\selectfont k}%
\end{pgfscope}%
\begin{pgfscope}%
\pgfsetbuttcap%
\pgfsetroundjoin%
\definecolor{currentfill}{rgb}{0.000000,0.000000,0.000000}%
\pgfsetfillcolor{currentfill}%
\pgfsetlinewidth{0.803000pt}%
\definecolor{currentstroke}{rgb}{0.000000,0.000000,0.000000}%
\pgfsetstrokecolor{currentstroke}%
\pgfsetdash{}{0pt}%
\pgfsys@defobject{currentmarker}{\pgfqpoint{-0.048611in}{0.000000in}}{\pgfqpoint{0.000000in}{0.000000in}}{%
\pgfpathmoveto{\pgfqpoint{0.000000in}{0.000000in}}%
\pgfpathlineto{\pgfqpoint{-0.048611in}{0.000000in}}%
\pgfusepath{stroke,fill}%
}%
\begin{pgfscope}%
\pgfsys@transformshift{0.625000in}{2.395872in}%
\pgfsys@useobject{currentmarker}{}%
\end{pgfscope}%
\end{pgfscope}%
\begin{pgfscope}%
\definecolor{textcolor}{rgb}{0.000000,0.000000,0.000000}%
\pgfsetstrokecolor{textcolor}%
\pgfsetfillcolor{textcolor}%
\pgftext[x=0.239775in,y=2.343110in,left,base]{\color{textcolor}\sffamily\fontsize{10.000000}{12.000000}\selectfont \(\displaystyle {10^{-3}}\)}%
\end{pgfscope}%
\begin{pgfscope}%
\pgfsetbuttcap%
\pgfsetroundjoin%
\definecolor{currentfill}{rgb}{0.000000,0.000000,0.000000}%
\pgfsetfillcolor{currentfill}%
\pgfsetlinewidth{0.602250pt}%
\definecolor{currentstroke}{rgb}{0.000000,0.000000,0.000000}%
\pgfsetstrokecolor{currentstroke}%
\pgfsetdash{}{0pt}%
\pgfsys@defobject{currentmarker}{\pgfqpoint{-0.027778in}{0.000000in}}{\pgfqpoint{0.000000in}{0.000000in}}{%
\pgfpathmoveto{\pgfqpoint{0.000000in}{0.000000in}}%
\pgfpathlineto{\pgfqpoint{-0.027778in}{0.000000in}}%
\pgfusepath{stroke,fill}%
}%
\begin{pgfscope}%
\pgfsys@transformshift{0.625000in}{0.951144in}%
\pgfsys@useobject{currentmarker}{}%
\end{pgfscope}%
\end{pgfscope}%
\begin{pgfscope}%
\pgfsetbuttcap%
\pgfsetroundjoin%
\definecolor{currentfill}{rgb}{0.000000,0.000000,0.000000}%
\pgfsetfillcolor{currentfill}%
\pgfsetlinewidth{0.602250pt}%
\definecolor{currentstroke}{rgb}{0.000000,0.000000,0.000000}%
\pgfsetstrokecolor{currentstroke}%
\pgfsetdash{}{0pt}%
\pgfsys@defobject{currentmarker}{\pgfqpoint{-0.027778in}{0.000000in}}{\pgfqpoint{0.000000in}{0.000000in}}{%
\pgfpathmoveto{\pgfqpoint{0.000000in}{0.000000in}}%
\pgfpathlineto{\pgfqpoint{-0.027778in}{0.000000in}}%
\pgfusepath{stroke,fill}%
}%
\begin{pgfscope}%
\pgfsys@transformshift{0.625000in}{1.315114in}%
\pgfsys@useobject{currentmarker}{}%
\end{pgfscope}%
\end{pgfscope}%
\begin{pgfscope}%
\pgfsetbuttcap%
\pgfsetroundjoin%
\definecolor{currentfill}{rgb}{0.000000,0.000000,0.000000}%
\pgfsetfillcolor{currentfill}%
\pgfsetlinewidth{0.602250pt}%
\definecolor{currentstroke}{rgb}{0.000000,0.000000,0.000000}%
\pgfsetstrokecolor{currentstroke}%
\pgfsetdash{}{0pt}%
\pgfsys@defobject{currentmarker}{\pgfqpoint{-0.027778in}{0.000000in}}{\pgfqpoint{0.000000in}{0.000000in}}{%
\pgfpathmoveto{\pgfqpoint{0.000000in}{0.000000in}}%
\pgfpathlineto{\pgfqpoint{-0.027778in}{0.000000in}}%
\pgfusepath{stroke,fill}%
}%
\begin{pgfscope}%
\pgfsys@transformshift{0.625000in}{1.573354in}%
\pgfsys@useobject{currentmarker}{}%
\end{pgfscope}%
\end{pgfscope}%
\begin{pgfscope}%
\pgfsetbuttcap%
\pgfsetroundjoin%
\definecolor{currentfill}{rgb}{0.000000,0.000000,0.000000}%
\pgfsetfillcolor{currentfill}%
\pgfsetlinewidth{0.602250pt}%
\definecolor{currentstroke}{rgb}{0.000000,0.000000,0.000000}%
\pgfsetstrokecolor{currentstroke}%
\pgfsetdash{}{0pt}%
\pgfsys@defobject{currentmarker}{\pgfqpoint{-0.027778in}{0.000000in}}{\pgfqpoint{0.000000in}{0.000000in}}{%
\pgfpathmoveto{\pgfqpoint{0.000000in}{0.000000in}}%
\pgfpathlineto{\pgfqpoint{-0.027778in}{0.000000in}}%
\pgfusepath{stroke,fill}%
}%
\begin{pgfscope}%
\pgfsys@transformshift{0.625000in}{1.773661in}%
\pgfsys@useobject{currentmarker}{}%
\end{pgfscope}%
\end{pgfscope}%
\begin{pgfscope}%
\pgfsetbuttcap%
\pgfsetroundjoin%
\definecolor{currentfill}{rgb}{0.000000,0.000000,0.000000}%
\pgfsetfillcolor{currentfill}%
\pgfsetlinewidth{0.602250pt}%
\definecolor{currentstroke}{rgb}{0.000000,0.000000,0.000000}%
\pgfsetstrokecolor{currentstroke}%
\pgfsetdash{}{0pt}%
\pgfsys@defobject{currentmarker}{\pgfqpoint{-0.027778in}{0.000000in}}{\pgfqpoint{0.000000in}{0.000000in}}{%
\pgfpathmoveto{\pgfqpoint{0.000000in}{0.000000in}}%
\pgfpathlineto{\pgfqpoint{-0.027778in}{0.000000in}}%
\pgfusepath{stroke,fill}%
}%
\begin{pgfscope}%
\pgfsys@transformshift{0.625000in}{1.937324in}%
\pgfsys@useobject{currentmarker}{}%
\end{pgfscope}%
\end{pgfscope}%
\begin{pgfscope}%
\pgfsetbuttcap%
\pgfsetroundjoin%
\definecolor{currentfill}{rgb}{0.000000,0.000000,0.000000}%
\pgfsetfillcolor{currentfill}%
\pgfsetlinewidth{0.602250pt}%
\definecolor{currentstroke}{rgb}{0.000000,0.000000,0.000000}%
\pgfsetstrokecolor{currentstroke}%
\pgfsetdash{}{0pt}%
\pgfsys@defobject{currentmarker}{\pgfqpoint{-0.027778in}{0.000000in}}{\pgfqpoint{0.000000in}{0.000000in}}{%
\pgfpathmoveto{\pgfqpoint{0.000000in}{0.000000in}}%
\pgfpathlineto{\pgfqpoint{-0.027778in}{0.000000in}}%
\pgfusepath{stroke,fill}%
}%
\begin{pgfscope}%
\pgfsys@transformshift{0.625000in}{2.075699in}%
\pgfsys@useobject{currentmarker}{}%
\end{pgfscope}%
\end{pgfscope}%
\begin{pgfscope}%
\pgfsetbuttcap%
\pgfsetroundjoin%
\definecolor{currentfill}{rgb}{0.000000,0.000000,0.000000}%
\pgfsetfillcolor{currentfill}%
\pgfsetlinewidth{0.602250pt}%
\definecolor{currentstroke}{rgb}{0.000000,0.000000,0.000000}%
\pgfsetstrokecolor{currentstroke}%
\pgfsetdash{}{0pt}%
\pgfsys@defobject{currentmarker}{\pgfqpoint{-0.027778in}{0.000000in}}{\pgfqpoint{0.000000in}{0.000000in}}{%
\pgfpathmoveto{\pgfqpoint{0.000000in}{0.000000in}}%
\pgfpathlineto{\pgfqpoint{-0.027778in}{0.000000in}}%
\pgfusepath{stroke,fill}%
}%
\begin{pgfscope}%
\pgfsys@transformshift{0.625000in}{2.195565in}%
\pgfsys@useobject{currentmarker}{}%
\end{pgfscope}%
\end{pgfscope}%
\begin{pgfscope}%
\pgfsetbuttcap%
\pgfsetroundjoin%
\definecolor{currentfill}{rgb}{0.000000,0.000000,0.000000}%
\pgfsetfillcolor{currentfill}%
\pgfsetlinewidth{0.602250pt}%
\definecolor{currentstroke}{rgb}{0.000000,0.000000,0.000000}%
\pgfsetstrokecolor{currentstroke}%
\pgfsetdash{}{0pt}%
\pgfsys@defobject{currentmarker}{\pgfqpoint{-0.027778in}{0.000000in}}{\pgfqpoint{0.000000in}{0.000000in}}{%
\pgfpathmoveto{\pgfqpoint{0.000000in}{0.000000in}}%
\pgfpathlineto{\pgfqpoint{-0.027778in}{0.000000in}}%
\pgfusepath{stroke,fill}%
}%
\begin{pgfscope}%
\pgfsys@transformshift{0.625000in}{2.301294in}%
\pgfsys@useobject{currentmarker}{}%
\end{pgfscope}%
\end{pgfscope}%
\begin{pgfscope}%
\pgfsetbuttcap%
\pgfsetroundjoin%
\definecolor{currentfill}{rgb}{0.000000,0.000000,0.000000}%
\pgfsetfillcolor{currentfill}%
\pgfsetlinewidth{0.602250pt}%
\definecolor{currentstroke}{rgb}{0.000000,0.000000,0.000000}%
\pgfsetstrokecolor{currentstroke}%
\pgfsetdash{}{0pt}%
\pgfsys@defobject{currentmarker}{\pgfqpoint{-0.027778in}{0.000000in}}{\pgfqpoint{0.000000in}{0.000000in}}{%
\pgfpathmoveto{\pgfqpoint{0.000000in}{0.000000in}}%
\pgfpathlineto{\pgfqpoint{-0.027778in}{0.000000in}}%
\pgfusepath{stroke,fill}%
}%
\begin{pgfscope}%
\pgfsys@transformshift{0.625000in}{3.018082in}%
\pgfsys@useobject{currentmarker}{}%
\end{pgfscope}%
\end{pgfscope}%
\begin{pgfscope}%
\pgfsetbuttcap%
\pgfsetroundjoin%
\definecolor{currentfill}{rgb}{0.000000,0.000000,0.000000}%
\pgfsetfillcolor{currentfill}%
\pgfsetlinewidth{0.602250pt}%
\definecolor{currentstroke}{rgb}{0.000000,0.000000,0.000000}%
\pgfsetstrokecolor{currentstroke}%
\pgfsetdash{}{0pt}%
\pgfsys@defobject{currentmarker}{\pgfqpoint{-0.027778in}{0.000000in}}{\pgfqpoint{0.000000in}{0.000000in}}{%
\pgfpathmoveto{\pgfqpoint{0.000000in}{0.000000in}}%
\pgfpathlineto{\pgfqpoint{-0.027778in}{0.000000in}}%
\pgfusepath{stroke,fill}%
}%
\begin{pgfscope}%
\pgfsys@transformshift{0.625000in}{3.382052in}%
\pgfsys@useobject{currentmarker}{}%
\end{pgfscope}%
\end{pgfscope}%
\begin{pgfscope}%
\definecolor{textcolor}{rgb}{0.000000,0.000000,0.000000}%
\pgfsetstrokecolor{textcolor}%
\pgfsetfillcolor{textcolor}%
\pgftext[x=0.184220in,y=1.980000in,,bottom,rotate=90.000000]{\color{textcolor}\sffamily\fontsize{10.000000}{12.000000}\selectfont C}%
\end{pgfscope}%
\begin{pgfscope}%
\pgfpathrectangle{\pgfqpoint{0.625000in}{0.440000in}}{\pgfqpoint{3.875000in}{3.080000in}}%
\pgfusepath{clip}%
\pgfsetbuttcap%
\pgfsetroundjoin%
\definecolor{currentfill}{rgb}{0.000000,0.000000,0.000000}%
\pgfsetfillcolor{currentfill}%
\pgfsetlinewidth{1.003750pt}%
\definecolor{currentstroke}{rgb}{0.000000,0.000000,0.000000}%
\pgfsetstrokecolor{currentstroke}%
\pgfsetdash{}{0pt}%
\pgfsys@defobject{currentmarker}{\pgfqpoint{-0.041667in}{-0.041667in}}{\pgfqpoint{0.041667in}{0.041667in}}{%
\pgfpathmoveto{\pgfqpoint{0.000000in}{-0.041667in}}%
\pgfpathcurveto{\pgfqpoint{0.011050in}{-0.041667in}}{\pgfqpoint{0.021649in}{-0.037276in}}{\pgfqpoint{0.029463in}{-0.029463in}}%
\pgfpathcurveto{\pgfqpoint{0.037276in}{-0.021649in}}{\pgfqpoint{0.041667in}{-0.011050in}}{\pgfqpoint{0.041667in}{0.000000in}}%
\pgfpathcurveto{\pgfqpoint{0.041667in}{0.011050in}}{\pgfqpoint{0.037276in}{0.021649in}}{\pgfqpoint{0.029463in}{0.029463in}}%
\pgfpathcurveto{\pgfqpoint{0.021649in}{0.037276in}}{\pgfqpoint{0.011050in}{0.041667in}}{\pgfqpoint{0.000000in}{0.041667in}}%
\pgfpathcurveto{\pgfqpoint{-0.011050in}{0.041667in}}{\pgfqpoint{-0.021649in}{0.037276in}}{\pgfqpoint{-0.029463in}{0.029463in}}%
\pgfpathcurveto{\pgfqpoint{-0.037276in}{0.021649in}}{\pgfqpoint{-0.041667in}{0.011050in}}{\pgfqpoint{-0.041667in}{0.000000in}}%
\pgfpathcurveto{\pgfqpoint{-0.041667in}{-0.011050in}}{\pgfqpoint{-0.037276in}{-0.021649in}}{\pgfqpoint{-0.029463in}{-0.029463in}}%
\pgfpathcurveto{\pgfqpoint{-0.021649in}{-0.037276in}}{\pgfqpoint{-0.011050in}{-0.041667in}}{\pgfqpoint{0.000000in}{-0.041667in}}%
\pgfpathclose%
\pgfusepath{stroke,fill}%
}%
\begin{pgfscope}%
\pgfsys@transformshift{0.801136in}{3.380000in}%
\pgfsys@useobject{currentmarker}{}%
\end{pgfscope}%
\begin{pgfscope}%
\pgfsys@transformshift{2.163913in}{2.654291in}%
\pgfsys@useobject{currentmarker}{}%
\end{pgfscope}%
\begin{pgfscope}%
\pgfsys@transformshift{2.961087in}{2.293438in}%
\pgfsys@useobject{currentmarker}{}%
\end{pgfscope}%
\begin{pgfscope}%
\pgfsys@transformshift{3.526690in}{0.957210in}%
\pgfsys@useobject{currentmarker}{}%
\end{pgfscope}%
\begin{pgfscope}%
\pgfsys@transformshift{3.965406in}{0.901992in}%
\pgfsys@useobject{currentmarker}{}%
\end{pgfscope}%
\begin{pgfscope}%
\pgfsys@transformshift{4.323864in}{0.580000in}%
\pgfsys@useobject{currentmarker}{}%
\end{pgfscope}%
\end{pgfscope}%
\begin{pgfscope}%
\pgfsetrectcap%
\pgfsetmiterjoin%
\pgfsetlinewidth{0.803000pt}%
\definecolor{currentstroke}{rgb}{0.000000,0.000000,0.000000}%
\pgfsetstrokecolor{currentstroke}%
\pgfsetdash{}{0pt}%
\pgfpathmoveto{\pgfqpoint{0.625000in}{0.440000in}}%
\pgfpathlineto{\pgfqpoint{0.625000in}{3.520000in}}%
\pgfusepath{stroke}%
\end{pgfscope}%
\begin{pgfscope}%
\pgfsetrectcap%
\pgfsetmiterjoin%
\pgfsetlinewidth{0.803000pt}%
\definecolor{currentstroke}{rgb}{0.000000,0.000000,0.000000}%
\pgfsetstrokecolor{currentstroke}%
\pgfsetdash{}{0pt}%
\pgfpathmoveto{\pgfqpoint{4.500000in}{0.440000in}}%
\pgfpathlineto{\pgfqpoint{4.500000in}{3.520000in}}%
\pgfusepath{stroke}%
\end{pgfscope}%
\begin{pgfscope}%
\pgfsetrectcap%
\pgfsetmiterjoin%
\pgfsetlinewidth{0.803000pt}%
\definecolor{currentstroke}{rgb}{0.000000,0.000000,0.000000}%
\pgfsetstrokecolor{currentstroke}%
\pgfsetdash{}{0pt}%
\pgfpathmoveto{\pgfqpoint{0.625000in}{0.440000in}}%
\pgfpathlineto{\pgfqpoint{4.500000in}{0.440000in}}%
\pgfusepath{stroke}%
\end{pgfscope}%
\begin{pgfscope}%
\pgfsetrectcap%
\pgfsetmiterjoin%
\pgfsetlinewidth{0.803000pt}%
\definecolor{currentstroke}{rgb}{0.000000,0.000000,0.000000}%
\pgfsetstrokecolor{currentstroke}%
\pgfsetdash{}{0pt}%
\pgfpathmoveto{\pgfqpoint{0.625000in}{3.520000in}}%
\pgfpathlineto{\pgfqpoint{4.500000in}{3.520000in}}%
\pgfusepath{stroke}%
\end{pgfscope}%
\end{pgfpicture}%
\makeatother%
\endgroup%

  \end{subfigure}
    \begin{subfigure}{\textwidth}
%% Creator: Matplotlib, PGF backend
%%
%% To include the figure in your LaTeX document, write
%%   \input{<filename>.pgf}
%%
%% Make sure the required packages are loaded in your preamble
%%   \usepackage{pgf}
%%
%% Figures using additional raster images can only be included by \input if
%% they are in the same directory as the main LaTeX file. For loading figures
%% from other directories you can use the `import` package
%%   \usepackage{import}
%% and then include the figures with
%%   \import{<path to file>}{<filename>.pgf}
%%
%% Matplotlib used the following preamble
%%   \usepackage{fontspec}
%%   \setmainfont{DejaVuSerif.ttf}[Path=/home/owen/progs/firedrake-complex/firedrake/lib/python3.5/site-packages/matplotlib/mpl-data/fonts/ttf/]
%%   \setsansfont{DejaVuSans.ttf}[Path=/home/owen/progs/firedrake-complex/firedrake/lib/python3.5/site-packages/matplotlib/mpl-data/fonts/ttf/]
%%   \setmonofont{DejaVuSansMono.ttf}[Path=/home/owen/progs/firedrake-complex/firedrake/lib/python3.5/site-packages/matplotlib/mpl-data/fonts/ttf/]
%%
\begingroup%
\makeatletter%
\begin{pgfpicture}%
\pgfpathrectangle{\pgfpointorigin}{\pgfqpoint{5.000000in}{4.000000in}}%
\pgfusepath{use as bounding box, clip}%
\begin{pgfscope}%
\pgfsetbuttcap%
\pgfsetmiterjoin%
\definecolor{currentfill}{rgb}{1.000000,1.000000,1.000000}%
\pgfsetfillcolor{currentfill}%
\pgfsetlinewidth{0.000000pt}%
\definecolor{currentstroke}{rgb}{1.000000,1.000000,1.000000}%
\pgfsetstrokecolor{currentstroke}%
\pgfsetdash{}{0pt}%
\pgfpathmoveto{\pgfqpoint{0.000000in}{0.000000in}}%
\pgfpathlineto{\pgfqpoint{5.000000in}{0.000000in}}%
\pgfpathlineto{\pgfqpoint{5.000000in}{4.000000in}}%
\pgfpathlineto{\pgfqpoint{0.000000in}{4.000000in}}%
\pgfpathclose%
\pgfusepath{fill}%
\end{pgfscope}%
\begin{pgfscope}%
\pgfsetbuttcap%
\pgfsetmiterjoin%
\definecolor{currentfill}{rgb}{1.000000,1.000000,1.000000}%
\pgfsetfillcolor{currentfill}%
\pgfsetlinewidth{0.000000pt}%
\definecolor{currentstroke}{rgb}{0.000000,0.000000,0.000000}%
\pgfsetstrokecolor{currentstroke}%
\pgfsetstrokeopacity{0.000000}%
\pgfsetdash{}{0pt}%
\pgfpathmoveto{\pgfqpoint{0.625000in}{0.440000in}}%
\pgfpathlineto{\pgfqpoint{4.500000in}{0.440000in}}%
\pgfpathlineto{\pgfqpoint{4.500000in}{3.520000in}}%
\pgfpathlineto{\pgfqpoint{0.625000in}{3.520000in}}%
\pgfpathclose%
\pgfusepath{fill}%
\end{pgfscope}%
\begin{pgfscope}%
\pgfsetbuttcap%
\pgfsetroundjoin%
\definecolor{currentfill}{rgb}{0.000000,0.000000,0.000000}%
\pgfsetfillcolor{currentfill}%
\pgfsetlinewidth{0.803000pt}%
\definecolor{currentstroke}{rgb}{0.000000,0.000000,0.000000}%
\pgfsetstrokecolor{currentstroke}%
\pgfsetdash{}{0pt}%
\pgfsys@defobject{currentmarker}{\pgfqpoint{0.000000in}{-0.048611in}}{\pgfqpoint{0.000000in}{0.000000in}}{%
\pgfpathmoveto{\pgfqpoint{0.000000in}{0.000000in}}%
\pgfpathlineto{\pgfqpoint{0.000000in}{-0.048611in}}%
\pgfusepath{stroke,fill}%
}%
\begin{pgfscope}%
\pgfsys@transformshift{2.172304in}{0.440000in}%
\pgfsys@useobject{currentmarker}{}%
\end{pgfscope}%
\end{pgfscope}%
\begin{pgfscope}%
\definecolor{textcolor}{rgb}{0.000000,0.000000,0.000000}%
\pgfsetstrokecolor{textcolor}%
\pgfsetfillcolor{textcolor}%
\pgftext[x=2.172304in,y=0.342778in,,top]{\color{textcolor}\sffamily\fontsize{10.000000}{12.000000}\selectfont \(\displaystyle {2.718281828459045^{3}}\)}%
\end{pgfscope}%
\begin{pgfscope}%
\pgfsetbuttcap%
\pgfsetroundjoin%
\definecolor{currentfill}{rgb}{0.000000,0.000000,0.000000}%
\pgfsetfillcolor{currentfill}%
\pgfsetlinewidth{0.803000pt}%
\definecolor{currentstroke}{rgb}{0.000000,0.000000,0.000000}%
\pgfsetstrokecolor{currentstroke}%
\pgfsetdash{}{0pt}%
\pgfsys@defobject{currentmarker}{\pgfqpoint{0.000000in}{-0.048611in}}{\pgfqpoint{0.000000in}{0.000000in}}{%
\pgfpathmoveto{\pgfqpoint{0.000000in}{0.000000in}}%
\pgfpathlineto{\pgfqpoint{0.000000in}{-0.048611in}}%
\pgfusepath{stroke,fill}%
}%
\begin{pgfscope}%
\pgfsys@transformshift{4.138375in}{0.440000in}%
\pgfsys@useobject{currentmarker}{}%
\end{pgfscope}%
\end{pgfscope}%
\begin{pgfscope}%
\definecolor{textcolor}{rgb}{0.000000,0.000000,0.000000}%
\pgfsetstrokecolor{textcolor}%
\pgfsetfillcolor{textcolor}%
\pgftext[x=4.138375in,y=0.342778in,,top]{\color{textcolor}\sffamily\fontsize{10.000000}{12.000000}\selectfont \(\displaystyle {2.718281828459045^{4}}\)}%
\end{pgfscope}%
\begin{pgfscope}%
\definecolor{textcolor}{rgb}{0.000000,0.000000,0.000000}%
\pgfsetstrokecolor{textcolor}%
\pgfsetfillcolor{textcolor}%
\pgftext[x=2.562500in,y=0.152809in,,top]{\color{textcolor}\sffamily\fontsize{10.000000}{12.000000}\selectfont \(\displaystyle k\)}%
\end{pgfscope}%
\begin{pgfscope}%
\pgfsetbuttcap%
\pgfsetroundjoin%
\definecolor{currentfill}{rgb}{0.000000,0.000000,0.000000}%
\pgfsetfillcolor{currentfill}%
\pgfsetlinewidth{0.803000pt}%
\definecolor{currentstroke}{rgb}{0.000000,0.000000,0.000000}%
\pgfsetstrokecolor{currentstroke}%
\pgfsetdash{}{0pt}%
\pgfsys@defobject{currentmarker}{\pgfqpoint{-0.048611in}{0.000000in}}{\pgfqpoint{0.000000in}{0.000000in}}{%
\pgfpathmoveto{\pgfqpoint{0.000000in}{0.000000in}}%
\pgfpathlineto{\pgfqpoint{-0.048611in}{0.000000in}}%
\pgfusepath{stroke,fill}%
}%
\begin{pgfscope}%
\pgfsys@transformshift{0.625000in}{0.567382in}%
\pgfsys@useobject{currentmarker}{}%
\end{pgfscope}%
\end{pgfscope}%
\begin{pgfscope}%
\definecolor{textcolor}{rgb}{0.000000,0.000000,0.000000}%
\pgfsetstrokecolor{textcolor}%
\pgfsetfillcolor{textcolor}%
\pgftext[x=0.218533in,y=0.514621in,left,base]{\color{textcolor}\sffamily\fontsize{10.000000}{12.000000}\selectfont 0.65}%
\end{pgfscope}%
\begin{pgfscope}%
\pgfsetbuttcap%
\pgfsetroundjoin%
\definecolor{currentfill}{rgb}{0.000000,0.000000,0.000000}%
\pgfsetfillcolor{currentfill}%
\pgfsetlinewidth{0.803000pt}%
\definecolor{currentstroke}{rgb}{0.000000,0.000000,0.000000}%
\pgfsetstrokecolor{currentstroke}%
\pgfsetdash{}{0pt}%
\pgfsys@defobject{currentmarker}{\pgfqpoint{-0.048611in}{0.000000in}}{\pgfqpoint{0.000000in}{0.000000in}}{%
\pgfpathmoveto{\pgfqpoint{0.000000in}{0.000000in}}%
\pgfpathlineto{\pgfqpoint{-0.048611in}{0.000000in}}%
\pgfusepath{stroke,fill}%
}%
\begin{pgfscope}%
\pgfsys@transformshift{0.625000in}{0.985227in}%
\pgfsys@useobject{currentmarker}{}%
\end{pgfscope}%
\end{pgfscope}%
\begin{pgfscope}%
\definecolor{textcolor}{rgb}{0.000000,0.000000,0.000000}%
\pgfsetstrokecolor{textcolor}%
\pgfsetfillcolor{textcolor}%
\pgftext[x=0.218533in,y=0.932465in,left,base]{\color{textcolor}\sffamily\fontsize{10.000000}{12.000000}\selectfont 0.70}%
\end{pgfscope}%
\begin{pgfscope}%
\pgfsetbuttcap%
\pgfsetroundjoin%
\definecolor{currentfill}{rgb}{0.000000,0.000000,0.000000}%
\pgfsetfillcolor{currentfill}%
\pgfsetlinewidth{0.803000pt}%
\definecolor{currentstroke}{rgb}{0.000000,0.000000,0.000000}%
\pgfsetstrokecolor{currentstroke}%
\pgfsetdash{}{0pt}%
\pgfsys@defobject{currentmarker}{\pgfqpoint{-0.048611in}{0.000000in}}{\pgfqpoint{0.000000in}{0.000000in}}{%
\pgfpathmoveto{\pgfqpoint{0.000000in}{0.000000in}}%
\pgfpathlineto{\pgfqpoint{-0.048611in}{0.000000in}}%
\pgfusepath{stroke,fill}%
}%
\begin{pgfscope}%
\pgfsys@transformshift{0.625000in}{1.403071in}%
\pgfsys@useobject{currentmarker}{}%
\end{pgfscope}%
\end{pgfscope}%
\begin{pgfscope}%
\definecolor{textcolor}{rgb}{0.000000,0.000000,0.000000}%
\pgfsetstrokecolor{textcolor}%
\pgfsetfillcolor{textcolor}%
\pgftext[x=0.218533in,y=1.350310in,left,base]{\color{textcolor}\sffamily\fontsize{10.000000}{12.000000}\selectfont 0.75}%
\end{pgfscope}%
\begin{pgfscope}%
\pgfsetbuttcap%
\pgfsetroundjoin%
\definecolor{currentfill}{rgb}{0.000000,0.000000,0.000000}%
\pgfsetfillcolor{currentfill}%
\pgfsetlinewidth{0.803000pt}%
\definecolor{currentstroke}{rgb}{0.000000,0.000000,0.000000}%
\pgfsetstrokecolor{currentstroke}%
\pgfsetdash{}{0pt}%
\pgfsys@defobject{currentmarker}{\pgfqpoint{-0.048611in}{0.000000in}}{\pgfqpoint{0.000000in}{0.000000in}}{%
\pgfpathmoveto{\pgfqpoint{0.000000in}{0.000000in}}%
\pgfpathlineto{\pgfqpoint{-0.048611in}{0.000000in}}%
\pgfusepath{stroke,fill}%
}%
\begin{pgfscope}%
\pgfsys@transformshift{0.625000in}{1.820916in}%
\pgfsys@useobject{currentmarker}{}%
\end{pgfscope}%
\end{pgfscope}%
\begin{pgfscope}%
\definecolor{textcolor}{rgb}{0.000000,0.000000,0.000000}%
\pgfsetstrokecolor{textcolor}%
\pgfsetfillcolor{textcolor}%
\pgftext[x=0.218533in,y=1.768154in,left,base]{\color{textcolor}\sffamily\fontsize{10.000000}{12.000000}\selectfont 0.80}%
\end{pgfscope}%
\begin{pgfscope}%
\pgfsetbuttcap%
\pgfsetroundjoin%
\definecolor{currentfill}{rgb}{0.000000,0.000000,0.000000}%
\pgfsetfillcolor{currentfill}%
\pgfsetlinewidth{0.803000pt}%
\definecolor{currentstroke}{rgb}{0.000000,0.000000,0.000000}%
\pgfsetstrokecolor{currentstroke}%
\pgfsetdash{}{0pt}%
\pgfsys@defobject{currentmarker}{\pgfqpoint{-0.048611in}{0.000000in}}{\pgfqpoint{0.000000in}{0.000000in}}{%
\pgfpathmoveto{\pgfqpoint{0.000000in}{0.000000in}}%
\pgfpathlineto{\pgfqpoint{-0.048611in}{0.000000in}}%
\pgfusepath{stroke,fill}%
}%
\begin{pgfscope}%
\pgfsys@transformshift{0.625000in}{2.238760in}%
\pgfsys@useobject{currentmarker}{}%
\end{pgfscope}%
\end{pgfscope}%
\begin{pgfscope}%
\definecolor{textcolor}{rgb}{0.000000,0.000000,0.000000}%
\pgfsetstrokecolor{textcolor}%
\pgfsetfillcolor{textcolor}%
\pgftext[x=0.218533in,y=2.185999in,left,base]{\color{textcolor}\sffamily\fontsize{10.000000}{12.000000}\selectfont 0.85}%
\end{pgfscope}%
\begin{pgfscope}%
\pgfsetbuttcap%
\pgfsetroundjoin%
\definecolor{currentfill}{rgb}{0.000000,0.000000,0.000000}%
\pgfsetfillcolor{currentfill}%
\pgfsetlinewidth{0.803000pt}%
\definecolor{currentstroke}{rgb}{0.000000,0.000000,0.000000}%
\pgfsetstrokecolor{currentstroke}%
\pgfsetdash{}{0pt}%
\pgfsys@defobject{currentmarker}{\pgfqpoint{-0.048611in}{0.000000in}}{\pgfqpoint{0.000000in}{0.000000in}}{%
\pgfpathmoveto{\pgfqpoint{0.000000in}{0.000000in}}%
\pgfpathlineto{\pgfqpoint{-0.048611in}{0.000000in}}%
\pgfusepath{stroke,fill}%
}%
\begin{pgfscope}%
\pgfsys@transformshift{0.625000in}{2.656605in}%
\pgfsys@useobject{currentmarker}{}%
\end{pgfscope}%
\end{pgfscope}%
\begin{pgfscope}%
\definecolor{textcolor}{rgb}{0.000000,0.000000,0.000000}%
\pgfsetstrokecolor{textcolor}%
\pgfsetfillcolor{textcolor}%
\pgftext[x=0.218533in,y=2.603843in,left,base]{\color{textcolor}\sffamily\fontsize{10.000000}{12.000000}\selectfont 0.90}%
\end{pgfscope}%
\begin{pgfscope}%
\pgfsetbuttcap%
\pgfsetroundjoin%
\definecolor{currentfill}{rgb}{0.000000,0.000000,0.000000}%
\pgfsetfillcolor{currentfill}%
\pgfsetlinewidth{0.803000pt}%
\definecolor{currentstroke}{rgb}{0.000000,0.000000,0.000000}%
\pgfsetstrokecolor{currentstroke}%
\pgfsetdash{}{0pt}%
\pgfsys@defobject{currentmarker}{\pgfqpoint{-0.048611in}{0.000000in}}{\pgfqpoint{0.000000in}{0.000000in}}{%
\pgfpathmoveto{\pgfqpoint{0.000000in}{0.000000in}}%
\pgfpathlineto{\pgfqpoint{-0.048611in}{0.000000in}}%
\pgfusepath{stroke,fill}%
}%
\begin{pgfscope}%
\pgfsys@transformshift{0.625000in}{3.074449in}%
\pgfsys@useobject{currentmarker}{}%
\end{pgfscope}%
\end{pgfscope}%
\begin{pgfscope}%
\definecolor{textcolor}{rgb}{0.000000,0.000000,0.000000}%
\pgfsetstrokecolor{textcolor}%
\pgfsetfillcolor{textcolor}%
\pgftext[x=0.218533in,y=3.021687in,left,base]{\color{textcolor}\sffamily\fontsize{10.000000}{12.000000}\selectfont 0.95}%
\end{pgfscope}%
\begin{pgfscope}%
\pgfsetbuttcap%
\pgfsetroundjoin%
\definecolor{currentfill}{rgb}{0.000000,0.000000,0.000000}%
\pgfsetfillcolor{currentfill}%
\pgfsetlinewidth{0.803000pt}%
\definecolor{currentstroke}{rgb}{0.000000,0.000000,0.000000}%
\pgfsetstrokecolor{currentstroke}%
\pgfsetdash{}{0pt}%
\pgfsys@defobject{currentmarker}{\pgfqpoint{-0.048611in}{0.000000in}}{\pgfqpoint{0.000000in}{0.000000in}}{%
\pgfpathmoveto{\pgfqpoint{0.000000in}{0.000000in}}%
\pgfpathlineto{\pgfqpoint{-0.048611in}{0.000000in}}%
\pgfusepath{stroke,fill}%
}%
\begin{pgfscope}%
\pgfsys@transformshift{0.625000in}{3.492293in}%
\pgfsys@useobject{currentmarker}{}%
\end{pgfscope}%
\end{pgfscope}%
\begin{pgfscope}%
\definecolor{textcolor}{rgb}{0.000000,0.000000,0.000000}%
\pgfsetstrokecolor{textcolor}%
\pgfsetfillcolor{textcolor}%
\pgftext[x=0.218533in,y=3.439532in,left,base]{\color{textcolor}\sffamily\fontsize{10.000000}{12.000000}\selectfont 1.00}%
\end{pgfscope}%
\begin{pgfscope}%
\definecolor{textcolor}{rgb}{0.000000,0.000000,0.000000}%
\pgfsetstrokecolor{textcolor}%
\pgfsetfillcolor{textcolor}%
\pgftext[x=0.162977in,y=1.980000in,,bottom,rotate=90.000000]{\color{textcolor}\sffamily\fontsize{10.000000}{12.000000}\selectfont \(\displaystyle \alpha\)}%
\end{pgfscope}%
\begin{pgfscope}%
\pgfpathrectangle{\pgfqpoint{0.625000in}{0.440000in}}{\pgfqpoint{3.875000in}{3.080000in}}%
\pgfusepath{clip}%
\pgfsetbuttcap%
\pgfsetroundjoin%
\pgfsetlinewidth{1.505625pt}%
\definecolor{currentstroke}{rgb}{0.000000,0.000000,0.000000}%
\pgfsetstrokecolor{currentstroke}%
\pgfsetdash{{5.550000pt}{2.400000pt}}{0.000000pt}%
\pgfpathmoveto{\pgfqpoint{0.801136in}{3.287929in}}%
\pgfpathlineto{\pgfqpoint{2.163913in}{2.521471in}}%
\pgfpathlineto{\pgfqpoint{2.961087in}{2.073122in}}%
\pgfpathlineto{\pgfqpoint{3.526690in}{1.755013in}}%
\pgfpathlineto{\pgfqpoint{3.965406in}{1.508269in}}%
\pgfpathlineto{\pgfqpoint{4.323864in}{1.306664in}}%
\pgfusepath{stroke}%
\end{pgfscope}%
\begin{pgfscope}%
\pgfpathrectangle{\pgfqpoint{0.625000in}{0.440000in}}{\pgfqpoint{3.875000in}{3.080000in}}%
\pgfusepath{clip}%
\pgfsetbuttcap%
\pgfsetroundjoin%
\definecolor{currentfill}{rgb}{0.000000,0.000000,0.000000}%
\pgfsetfillcolor{currentfill}%
\pgfsetlinewidth{1.003750pt}%
\definecolor{currentstroke}{rgb}{0.000000,0.000000,0.000000}%
\pgfsetstrokecolor{currentstroke}%
\pgfsetdash{}{0pt}%
\pgfsys@defobject{currentmarker}{\pgfqpoint{-0.041667in}{-0.041667in}}{\pgfqpoint{0.041667in}{0.041667in}}{%
\pgfpathmoveto{\pgfqpoint{0.000000in}{-0.041667in}}%
\pgfpathcurveto{\pgfqpoint{0.011050in}{-0.041667in}}{\pgfqpoint{0.021649in}{-0.037276in}}{\pgfqpoint{0.029463in}{-0.029463in}}%
\pgfpathcurveto{\pgfqpoint{0.037276in}{-0.021649in}}{\pgfqpoint{0.041667in}{-0.011050in}}{\pgfqpoint{0.041667in}{0.000000in}}%
\pgfpathcurveto{\pgfqpoint{0.041667in}{0.011050in}}{\pgfqpoint{0.037276in}{0.021649in}}{\pgfqpoint{0.029463in}{0.029463in}}%
\pgfpathcurveto{\pgfqpoint{0.021649in}{0.037276in}}{\pgfqpoint{0.011050in}{0.041667in}}{\pgfqpoint{0.000000in}{0.041667in}}%
\pgfpathcurveto{\pgfqpoint{-0.011050in}{0.041667in}}{\pgfqpoint{-0.021649in}{0.037276in}}{\pgfqpoint{-0.029463in}{0.029463in}}%
\pgfpathcurveto{\pgfqpoint{-0.037276in}{0.021649in}}{\pgfqpoint{-0.041667in}{0.011050in}}{\pgfqpoint{-0.041667in}{0.000000in}}%
\pgfpathcurveto{\pgfqpoint{-0.041667in}{-0.011050in}}{\pgfqpoint{-0.037276in}{-0.021649in}}{\pgfqpoint{-0.029463in}{-0.029463in}}%
\pgfpathcurveto{\pgfqpoint{-0.021649in}{-0.037276in}}{\pgfqpoint{-0.011050in}{-0.041667in}}{\pgfqpoint{0.000000in}{-0.041667in}}%
\pgfpathclose%
\pgfusepath{stroke,fill}%
}%
\begin{pgfscope}%
\pgfsys@transformshift{0.801136in}{2.658440in}%
\pgfsys@useobject{currentmarker}{}%
\end{pgfscope}%
\begin{pgfscope}%
\pgfsys@transformshift{2.163913in}{2.797698in}%
\pgfsys@useobject{currentmarker}{}%
\end{pgfscope}%
\begin{pgfscope}%
\pgfsys@transformshift{2.961087in}{3.380000in}%
\pgfsys@useobject{currentmarker}{}%
\end{pgfscope}%
\begin{pgfscope}%
\pgfsys@transformshift{3.526690in}{1.575520in}%
\pgfsys@useobject{currentmarker}{}%
\end{pgfscope}%
\begin{pgfscope}%
\pgfsys@transformshift{3.965406in}{1.460810in}%
\pgfsys@useobject{currentmarker}{}%
\end{pgfscope}%
\begin{pgfscope}%
\pgfsys@transformshift{4.323864in}{0.580000in}%
\pgfsys@useobject{currentmarker}{}%
\end{pgfscope}%
\end{pgfscope}%
\begin{pgfscope}%
\pgfsetrectcap%
\pgfsetmiterjoin%
\pgfsetlinewidth{0.803000pt}%
\definecolor{currentstroke}{rgb}{0.000000,0.000000,0.000000}%
\pgfsetstrokecolor{currentstroke}%
\pgfsetdash{}{0pt}%
\pgfpathmoveto{\pgfqpoint{0.625000in}{0.440000in}}%
\pgfpathlineto{\pgfqpoint{0.625000in}{3.520000in}}%
\pgfusepath{stroke}%
\end{pgfscope}%
\begin{pgfscope}%
\pgfsetrectcap%
\pgfsetmiterjoin%
\pgfsetlinewidth{0.803000pt}%
\definecolor{currentstroke}{rgb}{0.000000,0.000000,0.000000}%
\pgfsetstrokecolor{currentstroke}%
\pgfsetdash{}{0pt}%
\pgfpathmoveto{\pgfqpoint{4.500000in}{0.440000in}}%
\pgfpathlineto{\pgfqpoint{4.500000in}{3.520000in}}%
\pgfusepath{stroke}%
\end{pgfscope}%
\begin{pgfscope}%
\pgfsetrectcap%
\pgfsetmiterjoin%
\pgfsetlinewidth{0.803000pt}%
\definecolor{currentstroke}{rgb}{0.000000,0.000000,0.000000}%
\pgfsetstrokecolor{currentstroke}%
\pgfsetdash{}{0pt}%
\pgfpathmoveto{\pgfqpoint{0.625000in}{0.440000in}}%
\pgfpathlineto{\pgfqpoint{4.500000in}{0.440000in}}%
\pgfusepath{stroke}%
\end{pgfscope}%
\begin{pgfscope}%
\pgfsetrectcap%
\pgfsetmiterjoin%
\pgfsetlinewidth{0.803000pt}%
\definecolor{currentstroke}{rgb}{0.000000,0.000000,0.000000}%
\pgfsetstrokecolor{currentstroke}%
\pgfsetdash{}{0pt}%
\pgfpathmoveto{\pgfqpoint{0.625000in}{3.520000in}}%
\pgfpathlineto{\pgfqpoint{4.500000in}{3.520000in}}%
\pgfusepath{stroke}%
\end{pgfscope}%
\begin{pgfscope}%
\pgfsetbuttcap%
\pgfsetmiterjoin%
\definecolor{currentfill}{rgb}{1.000000,1.000000,1.000000}%
\pgfsetfillcolor{currentfill}%
\pgfsetfillopacity{0.800000}%
\pgfsetlinewidth{1.003750pt}%
\definecolor{currentstroke}{rgb}{0.800000,0.800000,0.800000}%
\pgfsetstrokecolor{currentstroke}%
\pgfsetstrokeopacity{0.800000}%
\pgfsetdash{}{0pt}%
\pgfpathmoveto{\pgfqpoint{0.722222in}{0.509444in}}%
\pgfpathlineto{\pgfqpoint{2.807368in}{0.509444in}}%
\pgfpathquadraticcurveto{\pgfqpoint{2.835146in}{0.509444in}}{\pgfqpoint{2.835146in}{0.537222in}}%
\pgfpathlineto{\pgfqpoint{2.835146in}{0.733023in}}%
\pgfpathquadraticcurveto{\pgfqpoint{2.835146in}{0.760801in}}{\pgfqpoint{2.807368in}{0.760801in}}%
\pgfpathlineto{\pgfqpoint{0.722222in}{0.760801in}}%
\pgfpathquadraticcurveto{\pgfqpoint{0.694444in}{0.760801in}}{\pgfqpoint{0.694444in}{0.733023in}}%
\pgfpathlineto{\pgfqpoint{0.694444in}{0.537222in}}%
\pgfpathquadraticcurveto{\pgfqpoint{0.694444in}{0.509444in}}{\pgfqpoint{0.722222in}{0.509444in}}%
\pgfpathclose%
\pgfusepath{stroke,fill}%
\end{pgfscope}%
\begin{pgfscope}%
\pgfsetbuttcap%
\pgfsetroundjoin%
\pgfsetlinewidth{1.505625pt}%
\definecolor{currentstroke}{rgb}{0.000000,0.000000,0.000000}%
\pgfsetstrokecolor{currentstroke}%
\pgfsetdash{{5.550000pt}{2.400000pt}}{0.000000pt}%
\pgfpathmoveto{\pgfqpoint{0.750000in}{0.648333in}}%
\pgfpathlineto{\pgfqpoint{1.027778in}{0.648333in}}%
\pgfusepath{stroke}%
\end{pgfscope}%
\begin{pgfscope}%
\definecolor{textcolor}{rgb}{0.000000,0.000000,0.000000}%
\pgfsetstrokecolor{textcolor}%
\pgfsetfillcolor{textcolor}%
\pgftext[x=1.138889in,y=0.599722in,left,base]{\color{textcolor}\sffamily\fontsize{10.000000}{12.000000}\selectfont \(\displaystyle \alpha = 1.2802 - 0.1323\log(k)\)}%
\end{pgfscope}%
\end{pgfpicture}%
\makeatother%
\endgroup%

    \end{subfigure}
\caption{Plots of the dependence of $C$ and $\alpha$ on $k$ in \cref{eq:qmcerrorform} for $Q(u) = \int_D u$. Observe the $x$-axis is on a $\log_{10}$ scale, but $\log$ is the natural logarithm. \label{fig:integralCalpha}}
\end{figure}

\begin{figure}[h]
    \centering
  \begin{subfigure}{\textwidth}
%% Creator: Matplotlib, PGF backend
%%
%% To include the figure in your LaTeX document, write
%%   \input{<filename>.pgf}
%%
%% Make sure the required packages are loaded in your preamble
%%   \usepackage{pgf}
%%
%% Figures using additional raster images can only be included by \input if
%% they are in the same directory as the main LaTeX file. For loading figures
%% from other directories you can use the `import` package
%%   \usepackage{import}
%% and then include the figures with
%%   \import{<path to file>}{<filename>.pgf}
%%
%% Matplotlib used the following preamble
%%   \usepackage{fontspec}
%%   \setmainfont{DejaVuSerif.ttf}[Path=/home/owen/progs/firedrake-complex/firedrake/lib/python3.5/site-packages/matplotlib/mpl-data/fonts/ttf/]
%%   \setsansfont{DejaVuSans.ttf}[Path=/home/owen/progs/firedrake-complex/firedrake/lib/python3.5/site-packages/matplotlib/mpl-data/fonts/ttf/]
%%   \setmonofont{DejaVuSansMono.ttf}[Path=/home/owen/progs/firedrake-complex/firedrake/lib/python3.5/site-packages/matplotlib/mpl-data/fonts/ttf/]
%%
\begingroup%
\makeatletter%
\begin{pgfpicture}%
\pgfpathrectangle{\pgfpointorigin}{\pgfqpoint{5.000000in}{4.000000in}}%
\pgfusepath{use as bounding box, clip}%
\begin{pgfscope}%
\pgfsetbuttcap%
\pgfsetmiterjoin%
\definecolor{currentfill}{rgb}{1.000000,1.000000,1.000000}%
\pgfsetfillcolor{currentfill}%
\pgfsetlinewidth{0.000000pt}%
\definecolor{currentstroke}{rgb}{1.000000,1.000000,1.000000}%
\pgfsetstrokecolor{currentstroke}%
\pgfsetdash{}{0pt}%
\pgfpathmoveto{\pgfqpoint{0.000000in}{0.000000in}}%
\pgfpathlineto{\pgfqpoint{5.000000in}{0.000000in}}%
\pgfpathlineto{\pgfqpoint{5.000000in}{4.000000in}}%
\pgfpathlineto{\pgfqpoint{0.000000in}{4.000000in}}%
\pgfpathclose%
\pgfusepath{fill}%
\end{pgfscope}%
\begin{pgfscope}%
\pgfsetbuttcap%
\pgfsetmiterjoin%
\definecolor{currentfill}{rgb}{1.000000,1.000000,1.000000}%
\pgfsetfillcolor{currentfill}%
\pgfsetlinewidth{0.000000pt}%
\definecolor{currentstroke}{rgb}{0.000000,0.000000,0.000000}%
\pgfsetstrokecolor{currentstroke}%
\pgfsetstrokeopacity{0.000000}%
\pgfsetdash{}{0pt}%
\pgfpathmoveto{\pgfqpoint{0.827140in}{0.582778in}}%
\pgfpathlineto{\pgfqpoint{4.810222in}{0.582778in}}%
\pgfpathlineto{\pgfqpoint{4.810222in}{3.808710in}}%
\pgfpathlineto{\pgfqpoint{0.827140in}{3.808710in}}%
\pgfpathclose%
\pgfusepath{fill}%
\end{pgfscope}%
\begin{pgfscope}%
\pgfsetbuttcap%
\pgfsetroundjoin%
\definecolor{currentfill}{rgb}{0.000000,0.000000,0.000000}%
\pgfsetfillcolor{currentfill}%
\pgfsetlinewidth{0.803000pt}%
\definecolor{currentstroke}{rgb}{0.000000,0.000000,0.000000}%
\pgfsetstrokecolor{currentstroke}%
\pgfsetdash{}{0pt}%
\pgfsys@defobject{currentmarker}{\pgfqpoint{0.000000in}{-0.048611in}}{\pgfqpoint{0.000000in}{0.000000in}}{%
\pgfpathmoveto{\pgfqpoint{0.000000in}{0.000000in}}%
\pgfpathlineto{\pgfqpoint{0.000000in}{-0.048611in}}%
\pgfusepath{stroke,fill}%
}%
\begin{pgfscope}%
\pgfsys@transformshift{1.008189in}{0.582778in}%
\pgfsys@useobject{currentmarker}{}%
\end{pgfscope}%
\end{pgfscope}%
\begin{pgfscope}%
\definecolor{textcolor}{rgb}{0.000000,0.000000,0.000000}%
\pgfsetstrokecolor{textcolor}%
\pgfsetfillcolor{textcolor}%
\pgftext[x=1.008189in,y=0.485556in,,top]{\color{textcolor}\sffamily\fontsize{10.000000}{12.000000}\selectfont \(\displaystyle 10^{1}\)}%
\end{pgfscope}%
\begin{pgfscope}%
\pgfsetbuttcap%
\pgfsetroundjoin%
\definecolor{currentfill}{rgb}{0.000000,0.000000,0.000000}%
\pgfsetfillcolor{currentfill}%
\pgfsetlinewidth{0.602250pt}%
\definecolor{currentstroke}{rgb}{0.000000,0.000000,0.000000}%
\pgfsetstrokecolor{currentstroke}%
\pgfsetdash{}{0pt}%
\pgfsys@defobject{currentmarker}{\pgfqpoint{0.000000in}{-0.027778in}}{\pgfqpoint{0.000000in}{0.000000in}}{%
\pgfpathmoveto{\pgfqpoint{0.000000in}{0.000000in}}%
\pgfpathlineto{\pgfqpoint{0.000000in}{-0.027778in}}%
\pgfusepath{stroke,fill}%
}%
\begin{pgfscope}%
\pgfsys@transformshift{2.408977in}{0.582778in}%
\pgfsys@useobject{currentmarker}{}%
\end{pgfscope}%
\end{pgfscope}%
\begin{pgfscope}%
\definecolor{textcolor}{rgb}{0.000000,0.000000,0.000000}%
\pgfsetstrokecolor{textcolor}%
\pgfsetfillcolor{textcolor}%
\pgftext[x=2.408977in,y=0.507778in,,top]{\color{textcolor}\sffamily\fontsize{10.000000}{12.000000}\selectfont \(\displaystyle 2\times10^{1}\)}%
\end{pgfscope}%
\begin{pgfscope}%
\pgfsetbuttcap%
\pgfsetroundjoin%
\definecolor{currentfill}{rgb}{0.000000,0.000000,0.000000}%
\pgfsetfillcolor{currentfill}%
\pgfsetlinewidth{0.602250pt}%
\definecolor{currentstroke}{rgb}{0.000000,0.000000,0.000000}%
\pgfsetstrokecolor{currentstroke}%
\pgfsetdash{}{0pt}%
\pgfsys@defobject{currentmarker}{\pgfqpoint{0.000000in}{-0.027778in}}{\pgfqpoint{0.000000in}{0.000000in}}{%
\pgfpathmoveto{\pgfqpoint{0.000000in}{0.000000in}}%
\pgfpathlineto{\pgfqpoint{0.000000in}{-0.027778in}}%
\pgfusepath{stroke,fill}%
}%
\begin{pgfscope}%
\pgfsys@transformshift{3.228385in}{0.582778in}%
\pgfsys@useobject{currentmarker}{}%
\end{pgfscope}%
\end{pgfscope}%
\begin{pgfscope}%
\definecolor{textcolor}{rgb}{0.000000,0.000000,0.000000}%
\pgfsetstrokecolor{textcolor}%
\pgfsetfillcolor{textcolor}%
\pgftext[x=3.228385in,y=0.507778in,,top]{\color{textcolor}\sffamily\fontsize{10.000000}{12.000000}\selectfont \(\displaystyle 3\times10^{1}\)}%
\end{pgfscope}%
\begin{pgfscope}%
\pgfsetbuttcap%
\pgfsetroundjoin%
\definecolor{currentfill}{rgb}{0.000000,0.000000,0.000000}%
\pgfsetfillcolor{currentfill}%
\pgfsetlinewidth{0.602250pt}%
\definecolor{currentstroke}{rgb}{0.000000,0.000000,0.000000}%
\pgfsetstrokecolor{currentstroke}%
\pgfsetdash{}{0pt}%
\pgfsys@defobject{currentmarker}{\pgfqpoint{0.000000in}{-0.027778in}}{\pgfqpoint{0.000000in}{0.000000in}}{%
\pgfpathmoveto{\pgfqpoint{0.000000in}{0.000000in}}%
\pgfpathlineto{\pgfqpoint{0.000000in}{-0.027778in}}%
\pgfusepath{stroke,fill}%
}%
\begin{pgfscope}%
\pgfsys@transformshift{3.809764in}{0.582778in}%
\pgfsys@useobject{currentmarker}{}%
\end{pgfscope}%
\end{pgfscope}%
\begin{pgfscope}%
\definecolor{textcolor}{rgb}{0.000000,0.000000,0.000000}%
\pgfsetstrokecolor{textcolor}%
\pgfsetfillcolor{textcolor}%
\pgftext[x=3.809764in,y=0.507778in,,top]{\color{textcolor}\sffamily\fontsize{10.000000}{12.000000}\selectfont \(\displaystyle 4\times10^{1}\)}%
\end{pgfscope}%
\begin{pgfscope}%
\pgfsetbuttcap%
\pgfsetroundjoin%
\definecolor{currentfill}{rgb}{0.000000,0.000000,0.000000}%
\pgfsetfillcolor{currentfill}%
\pgfsetlinewidth{0.602250pt}%
\definecolor{currentstroke}{rgb}{0.000000,0.000000,0.000000}%
\pgfsetstrokecolor{currentstroke}%
\pgfsetdash{}{0pt}%
\pgfsys@defobject{currentmarker}{\pgfqpoint{0.000000in}{-0.027778in}}{\pgfqpoint{0.000000in}{0.000000in}}{%
\pgfpathmoveto{\pgfqpoint{0.000000in}{0.000000in}}%
\pgfpathlineto{\pgfqpoint{0.000000in}{-0.027778in}}%
\pgfusepath{stroke,fill}%
}%
\begin{pgfscope}%
\pgfsys@transformshift{4.260717in}{0.582778in}%
\pgfsys@useobject{currentmarker}{}%
\end{pgfscope}%
\end{pgfscope}%
\begin{pgfscope}%
\pgfsetbuttcap%
\pgfsetroundjoin%
\definecolor{currentfill}{rgb}{0.000000,0.000000,0.000000}%
\pgfsetfillcolor{currentfill}%
\pgfsetlinewidth{0.602250pt}%
\definecolor{currentstroke}{rgb}{0.000000,0.000000,0.000000}%
\pgfsetstrokecolor{currentstroke}%
\pgfsetdash{}{0pt}%
\pgfsys@defobject{currentmarker}{\pgfqpoint{0.000000in}{-0.027778in}}{\pgfqpoint{0.000000in}{0.000000in}}{%
\pgfpathmoveto{\pgfqpoint{0.000000in}{0.000000in}}%
\pgfpathlineto{\pgfqpoint{0.000000in}{-0.027778in}}%
\pgfusepath{stroke,fill}%
}%
\begin{pgfscope}%
\pgfsys@transformshift{4.629172in}{0.582778in}%
\pgfsys@useobject{currentmarker}{}%
\end{pgfscope}%
\end{pgfscope}%
\begin{pgfscope}%
\definecolor{textcolor}{rgb}{0.000000,0.000000,0.000000}%
\pgfsetstrokecolor{textcolor}%
\pgfsetfillcolor{textcolor}%
\pgftext[x=4.629172in,y=0.507778in,,top]{\color{textcolor}\sffamily\fontsize{10.000000}{12.000000}\selectfont \(\displaystyle 6\times10^{1}\)}%
\end{pgfscope}%
\begin{pgfscope}%
\definecolor{textcolor}{rgb}{0.000000,0.000000,0.000000}%
\pgfsetstrokecolor{textcolor}%
\pgfsetfillcolor{textcolor}%
\pgftext[x=2.818681in,y=0.295587in,,top]{\color{textcolor}\sffamily\fontsize{10.000000}{12.000000}\selectfont \(\displaystyle k\)}%
\end{pgfscope}%
\begin{pgfscope}%
\pgfsetbuttcap%
\pgfsetroundjoin%
\definecolor{currentfill}{rgb}{0.000000,0.000000,0.000000}%
\pgfsetfillcolor{currentfill}%
\pgfsetlinewidth{0.803000pt}%
\definecolor{currentstroke}{rgb}{0.000000,0.000000,0.000000}%
\pgfsetstrokecolor{currentstroke}%
\pgfsetdash{}{0pt}%
\pgfsys@defobject{currentmarker}{\pgfqpoint{-0.048611in}{0.000000in}}{\pgfqpoint{0.000000in}{0.000000in}}{%
\pgfpathmoveto{\pgfqpoint{0.000000in}{0.000000in}}%
\pgfpathlineto{\pgfqpoint{-0.048611in}{0.000000in}}%
\pgfusepath{stroke,fill}%
}%
\begin{pgfscope}%
\pgfsys@transformshift{0.827140in}{1.023971in}%
\pgfsys@useobject{currentmarker}{}%
\end{pgfscope}%
\end{pgfscope}%
\begin{pgfscope}%
\definecolor{textcolor}{rgb}{0.000000,0.000000,0.000000}%
\pgfsetstrokecolor{textcolor}%
\pgfsetfillcolor{textcolor}%
\pgftext[x=0.344114in,y=0.971210in,left,base]{\color{textcolor}\sffamily\fontsize{10.000000}{12.000000}\selectfont \(\displaystyle 0.0080\)}%
\end{pgfscope}%
\begin{pgfscope}%
\pgfsetbuttcap%
\pgfsetroundjoin%
\definecolor{currentfill}{rgb}{0.000000,0.000000,0.000000}%
\pgfsetfillcolor{currentfill}%
\pgfsetlinewidth{0.803000pt}%
\definecolor{currentstroke}{rgb}{0.000000,0.000000,0.000000}%
\pgfsetstrokecolor{currentstroke}%
\pgfsetdash{}{0pt}%
\pgfsys@defobject{currentmarker}{\pgfqpoint{-0.048611in}{0.000000in}}{\pgfqpoint{0.000000in}{0.000000in}}{%
\pgfpathmoveto{\pgfqpoint{0.000000in}{0.000000in}}%
\pgfpathlineto{\pgfqpoint{-0.048611in}{0.000000in}}%
\pgfusepath{stroke,fill}%
}%
\begin{pgfscope}%
\pgfsys@transformshift{0.827140in}{1.716566in}%
\pgfsys@useobject{currentmarker}{}%
\end{pgfscope}%
\end{pgfscope}%
\begin{pgfscope}%
\definecolor{textcolor}{rgb}{0.000000,0.000000,0.000000}%
\pgfsetstrokecolor{textcolor}%
\pgfsetfillcolor{textcolor}%
\pgftext[x=0.344114in,y=1.663804in,left,base]{\color{textcolor}\sffamily\fontsize{10.000000}{12.000000}\selectfont \(\displaystyle 0.0085\)}%
\end{pgfscope}%
\begin{pgfscope}%
\pgfsetbuttcap%
\pgfsetroundjoin%
\definecolor{currentfill}{rgb}{0.000000,0.000000,0.000000}%
\pgfsetfillcolor{currentfill}%
\pgfsetlinewidth{0.803000pt}%
\definecolor{currentstroke}{rgb}{0.000000,0.000000,0.000000}%
\pgfsetstrokecolor{currentstroke}%
\pgfsetdash{}{0pt}%
\pgfsys@defobject{currentmarker}{\pgfqpoint{-0.048611in}{0.000000in}}{\pgfqpoint{0.000000in}{0.000000in}}{%
\pgfpathmoveto{\pgfqpoint{0.000000in}{0.000000in}}%
\pgfpathlineto{\pgfqpoint{-0.048611in}{0.000000in}}%
\pgfusepath{stroke,fill}%
}%
\begin{pgfscope}%
\pgfsys@transformshift{0.827140in}{2.409161in}%
\pgfsys@useobject{currentmarker}{}%
\end{pgfscope}%
\end{pgfscope}%
\begin{pgfscope}%
\definecolor{textcolor}{rgb}{0.000000,0.000000,0.000000}%
\pgfsetstrokecolor{textcolor}%
\pgfsetfillcolor{textcolor}%
\pgftext[x=0.344114in,y=2.356399in,left,base]{\color{textcolor}\sffamily\fontsize{10.000000}{12.000000}\selectfont \(\displaystyle 0.0090\)}%
\end{pgfscope}%
\begin{pgfscope}%
\pgfsetbuttcap%
\pgfsetroundjoin%
\definecolor{currentfill}{rgb}{0.000000,0.000000,0.000000}%
\pgfsetfillcolor{currentfill}%
\pgfsetlinewidth{0.803000pt}%
\definecolor{currentstroke}{rgb}{0.000000,0.000000,0.000000}%
\pgfsetstrokecolor{currentstroke}%
\pgfsetdash{}{0pt}%
\pgfsys@defobject{currentmarker}{\pgfqpoint{-0.048611in}{0.000000in}}{\pgfqpoint{0.000000in}{0.000000in}}{%
\pgfpathmoveto{\pgfqpoint{0.000000in}{0.000000in}}%
\pgfpathlineto{\pgfqpoint{-0.048611in}{0.000000in}}%
\pgfusepath{stroke,fill}%
}%
\begin{pgfscope}%
\pgfsys@transformshift{0.827140in}{3.101756in}%
\pgfsys@useobject{currentmarker}{}%
\end{pgfscope}%
\end{pgfscope}%
\begin{pgfscope}%
\definecolor{textcolor}{rgb}{0.000000,0.000000,0.000000}%
\pgfsetstrokecolor{textcolor}%
\pgfsetfillcolor{textcolor}%
\pgftext[x=0.344114in,y=3.048994in,left,base]{\color{textcolor}\sffamily\fontsize{10.000000}{12.000000}\selectfont \(\displaystyle 0.0095\)}%
\end{pgfscope}%
\begin{pgfscope}%
\pgfsetbuttcap%
\pgfsetroundjoin%
\definecolor{currentfill}{rgb}{0.000000,0.000000,0.000000}%
\pgfsetfillcolor{currentfill}%
\pgfsetlinewidth{0.803000pt}%
\definecolor{currentstroke}{rgb}{0.000000,0.000000,0.000000}%
\pgfsetstrokecolor{currentstroke}%
\pgfsetdash{}{0pt}%
\pgfsys@defobject{currentmarker}{\pgfqpoint{-0.048611in}{0.000000in}}{\pgfqpoint{0.000000in}{0.000000in}}{%
\pgfpathmoveto{\pgfqpoint{0.000000in}{0.000000in}}%
\pgfpathlineto{\pgfqpoint{-0.048611in}{0.000000in}}%
\pgfusepath{stroke,fill}%
}%
\begin{pgfscope}%
\pgfsys@transformshift{0.827140in}{3.794350in}%
\pgfsys@useobject{currentmarker}{}%
\end{pgfscope}%
\end{pgfscope}%
\begin{pgfscope}%
\definecolor{textcolor}{rgb}{0.000000,0.000000,0.000000}%
\pgfsetstrokecolor{textcolor}%
\pgfsetfillcolor{textcolor}%
\pgftext[x=0.344114in,y=3.741589in,left,base]{\color{textcolor}\sffamily\fontsize{10.000000}{12.000000}\selectfont \(\displaystyle 0.0100\)}%
\end{pgfscope}%
\begin{pgfscope}%
\definecolor{textcolor}{rgb}{0.000000,0.000000,0.000000}%
\pgfsetstrokecolor{textcolor}%
\pgfsetfillcolor{textcolor}%
\pgftext[x=0.288559in,y=2.195744in,,bottom,rotate=90.000000]{\color{textcolor}\sffamily\fontsize{10.000000}{12.000000}\selectfont \(\displaystyle C\)}%
\end{pgfscope}%
\begin{pgfscope}%
\pgfpathrectangle{\pgfqpoint{0.827140in}{0.582778in}}{\pgfqpoint{3.983082in}{3.225932in}}%
\pgfusepath{clip}%
\pgfsetbuttcap%
\pgfsetroundjoin%
\definecolor{currentfill}{rgb}{0.000000,0.000000,0.000000}%
\pgfsetfillcolor{currentfill}%
\pgfsetlinewidth{1.003750pt}%
\definecolor{currentstroke}{rgb}{0.000000,0.000000,0.000000}%
\pgfsetstrokecolor{currentstroke}%
\pgfsetdash{}{0pt}%
\pgfsys@defobject{currentmarker}{\pgfqpoint{-0.041667in}{-0.041667in}}{\pgfqpoint{0.041667in}{0.041667in}}{%
\pgfpathmoveto{\pgfqpoint{0.000000in}{-0.041667in}}%
\pgfpathcurveto{\pgfqpoint{0.011050in}{-0.041667in}}{\pgfqpoint{0.021649in}{-0.037276in}}{\pgfqpoint{0.029463in}{-0.029463in}}%
\pgfpathcurveto{\pgfqpoint{0.037276in}{-0.021649in}}{\pgfqpoint{0.041667in}{-0.011050in}}{\pgfqpoint{0.041667in}{0.000000in}}%
\pgfpathcurveto{\pgfqpoint{0.041667in}{0.011050in}}{\pgfqpoint{0.037276in}{0.021649in}}{\pgfqpoint{0.029463in}{0.029463in}}%
\pgfpathcurveto{\pgfqpoint{0.021649in}{0.037276in}}{\pgfqpoint{0.011050in}{0.041667in}}{\pgfqpoint{0.000000in}{0.041667in}}%
\pgfpathcurveto{\pgfqpoint{-0.011050in}{0.041667in}}{\pgfqpoint{-0.021649in}{0.037276in}}{\pgfqpoint{-0.029463in}{0.029463in}}%
\pgfpathcurveto{\pgfqpoint{-0.037276in}{0.021649in}}{\pgfqpoint{-0.041667in}{0.011050in}}{\pgfqpoint{-0.041667in}{0.000000in}}%
\pgfpathcurveto{\pgfqpoint{-0.041667in}{-0.011050in}}{\pgfqpoint{-0.037276in}{-0.021649in}}{\pgfqpoint{-0.029463in}{-0.029463in}}%
\pgfpathcurveto{\pgfqpoint{-0.021649in}{-0.037276in}}{\pgfqpoint{-0.011050in}{-0.041667in}}{\pgfqpoint{0.000000in}{-0.041667in}}%
\pgfpathclose%
\pgfusepath{stroke,fill}%
}%
\begin{pgfscope}%
\pgfsys@transformshift{1.008189in}{0.784599in}%
\pgfsys@useobject{currentmarker}{}%
\end{pgfscope}%
\begin{pgfscope}%
\pgfsys@transformshift{2.408977in}{1.228252in}%
\pgfsys@useobject{currentmarker}{}%
\end{pgfscope}%
\begin{pgfscope}%
\pgfsys@transformshift{3.228385in}{0.729411in}%
\pgfsys@useobject{currentmarker}{}%
\end{pgfscope}%
\begin{pgfscope}%
\pgfsys@transformshift{3.809764in}{3.662077in}%
\pgfsys@useobject{currentmarker}{}%
\end{pgfscope}%
\begin{pgfscope}%
\pgfsys@transformshift{4.260717in}{2.077043in}%
\pgfsys@useobject{currentmarker}{}%
\end{pgfscope}%
\begin{pgfscope}%
\pgfsys@transformshift{4.629172in}{1.993444in}%
\pgfsys@useobject{currentmarker}{}%
\end{pgfscope}%
\end{pgfscope}%
\begin{pgfscope}%
\pgfsetrectcap%
\pgfsetmiterjoin%
\pgfsetlinewidth{0.803000pt}%
\definecolor{currentstroke}{rgb}{0.000000,0.000000,0.000000}%
\pgfsetstrokecolor{currentstroke}%
\pgfsetdash{}{0pt}%
\pgfpathmoveto{\pgfqpoint{0.827140in}{0.582778in}}%
\pgfpathlineto{\pgfqpoint{0.827140in}{3.808710in}}%
\pgfusepath{stroke}%
\end{pgfscope}%
\begin{pgfscope}%
\pgfsetrectcap%
\pgfsetmiterjoin%
\pgfsetlinewidth{0.000000pt}%
\definecolor{currentstroke}{rgb}{0.000000,0.000000,0.000000}%
\pgfsetstrokecolor{currentstroke}%
\pgfsetstrokeopacity{0.000000}%
\pgfsetdash{}{0pt}%
\pgfpathmoveto{\pgfqpoint{4.810222in}{0.582778in}}%
\pgfpathlineto{\pgfqpoint{4.810222in}{3.808710in}}%
\pgfusepath{}%
\end{pgfscope}%
\begin{pgfscope}%
\pgfsetrectcap%
\pgfsetmiterjoin%
\pgfsetlinewidth{0.803000pt}%
\definecolor{currentstroke}{rgb}{0.000000,0.000000,0.000000}%
\pgfsetstrokecolor{currentstroke}%
\pgfsetdash{}{0pt}%
\pgfpathmoveto{\pgfqpoint{0.827140in}{0.582778in}}%
\pgfpathlineto{\pgfqpoint{4.810222in}{0.582778in}}%
\pgfusepath{stroke}%
\end{pgfscope}%
\begin{pgfscope}%
\pgfsetrectcap%
\pgfsetmiterjoin%
\pgfsetlinewidth{0.000000pt}%
\definecolor{currentstroke}{rgb}{0.000000,0.000000,0.000000}%
\pgfsetstrokecolor{currentstroke}%
\pgfsetstrokeopacity{0.000000}%
\pgfsetdash{}{0pt}%
\pgfpathmoveto{\pgfqpoint{0.827140in}{3.808710in}}%
\pgfpathlineto{\pgfqpoint{4.810222in}{3.808710in}}%
\pgfusepath{}%
\end{pgfscope}%
\end{pgfpicture}%
\makeatother%
\endgroup%

  \end{subfigure}
    \begin{subfigure}{\textwidth}
%% Creator: Matplotlib, PGF backend
%%
%% To include the figure in your LaTeX document, write
%%   \input{<filename>.pgf}
%%
%% Make sure the required packages are loaded in your preamble
%%   \usepackage{pgf}
%%
%% Figures using additional raster images can only be included by \input if
%% they are in the same directory as the main LaTeX file. For loading figures
%% from other directories you can use the `import` package
%%   \usepackage{import}
%% and then include the figures with
%%   \import{<path to file>}{<filename>.pgf}
%%
%% Matplotlib used the following preamble
%%   \usepackage{fontspec}
%%   \setmainfont{DejaVuSerif.ttf}[Path=/home/owen/progs/firedrake-complex/firedrake/lib/python3.5/site-packages/matplotlib/mpl-data/fonts/ttf/]
%%   \setsansfont{DejaVuSans.ttf}[Path=/home/owen/progs/firedrake-complex/firedrake/lib/python3.5/site-packages/matplotlib/mpl-data/fonts/ttf/]
%%   \setmonofont{DejaVuSansMono.ttf}[Path=/home/owen/progs/firedrake-complex/firedrake/lib/python3.5/site-packages/matplotlib/mpl-data/fonts/ttf/]
%%
\begingroup%
\makeatletter%
\begin{pgfpicture}%
\pgfpathrectangle{\pgfpointorigin}{\pgfqpoint{3.000000in}{3.000000in}}%
\pgfusepath{use as bounding box, clip}%
\begin{pgfscope}%
\pgfsetbuttcap%
\pgfsetmiterjoin%
\definecolor{currentfill}{rgb}{1.000000,1.000000,1.000000}%
\pgfsetfillcolor{currentfill}%
\pgfsetlinewidth{0.000000pt}%
\definecolor{currentstroke}{rgb}{1.000000,1.000000,1.000000}%
\pgfsetstrokecolor{currentstroke}%
\pgfsetdash{}{0pt}%
\pgfpathmoveto{\pgfqpoint{0.000000in}{0.000000in}}%
\pgfpathlineto{\pgfqpoint{3.000000in}{0.000000in}}%
\pgfpathlineto{\pgfqpoint{3.000000in}{3.000000in}}%
\pgfpathlineto{\pgfqpoint{0.000000in}{3.000000in}}%
\pgfpathclose%
\pgfusepath{fill}%
\end{pgfscope}%
\begin{pgfscope}%
\pgfsetbuttcap%
\pgfsetmiterjoin%
\definecolor{currentfill}{rgb}{1.000000,1.000000,1.000000}%
\pgfsetfillcolor{currentfill}%
\pgfsetlinewidth{0.000000pt}%
\definecolor{currentstroke}{rgb}{0.000000,0.000000,0.000000}%
\pgfsetstrokecolor{currentstroke}%
\pgfsetstrokeopacity{0.000000}%
\pgfsetdash{}{0pt}%
\pgfpathmoveto{\pgfqpoint{0.375000in}{0.330000in}}%
\pgfpathlineto{\pgfqpoint{2.700000in}{0.330000in}}%
\pgfpathlineto{\pgfqpoint{2.700000in}{2.640000in}}%
\pgfpathlineto{\pgfqpoint{0.375000in}{2.640000in}}%
\pgfpathclose%
\pgfusepath{fill}%
\end{pgfscope}%
\begin{pgfscope}%
\pgfsetbuttcap%
\pgfsetroundjoin%
\definecolor{currentfill}{rgb}{0.000000,0.000000,0.000000}%
\pgfsetfillcolor{currentfill}%
\pgfsetlinewidth{0.803000pt}%
\definecolor{currentstroke}{rgb}{0.000000,0.000000,0.000000}%
\pgfsetstrokecolor{currentstroke}%
\pgfsetdash{}{0pt}%
\pgfsys@defobject{currentmarker}{\pgfqpoint{0.000000in}{-0.048611in}}{\pgfqpoint{0.000000in}{0.000000in}}{%
\pgfpathmoveto{\pgfqpoint{0.000000in}{0.000000in}}%
\pgfpathlineto{\pgfqpoint{0.000000in}{-0.048611in}}%
\pgfusepath{stroke,fill}%
}%
\begin{pgfscope}%
\pgfsys@transformshift{0.480682in}{0.330000in}%
\pgfsys@useobject{currentmarker}{}%
\end{pgfscope}%
\end{pgfscope}%
\begin{pgfscope}%
\definecolor{textcolor}{rgb}{0.000000,0.000000,0.000000}%
\pgfsetstrokecolor{textcolor}%
\pgfsetfillcolor{textcolor}%
\pgftext[x=0.480682in,y=0.232778in,,top]{\color{textcolor}\sffamily\fontsize{10.000000}{12.000000}\selectfont \(\displaystyle {10^{1}}\)}%
\end{pgfscope}%
\begin{pgfscope}%
\pgfsetbuttcap%
\pgfsetroundjoin%
\definecolor{currentfill}{rgb}{0.000000,0.000000,0.000000}%
\pgfsetfillcolor{currentfill}%
\pgfsetlinewidth{0.602250pt}%
\definecolor{currentstroke}{rgb}{0.000000,0.000000,0.000000}%
\pgfsetstrokecolor{currentstroke}%
\pgfsetdash{}{0pt}%
\pgfsys@defobject{currentmarker}{\pgfqpoint{0.000000in}{-0.027778in}}{\pgfqpoint{0.000000in}{0.000000in}}{%
\pgfpathmoveto{\pgfqpoint{0.000000in}{0.000000in}}%
\pgfpathlineto{\pgfqpoint{0.000000in}{-0.027778in}}%
\pgfusepath{stroke,fill}%
}%
\begin{pgfscope}%
\pgfsys@transformshift{1.298348in}{0.330000in}%
\pgfsys@useobject{currentmarker}{}%
\end{pgfscope}%
\end{pgfscope}%
\begin{pgfscope}%
\definecolor{textcolor}{rgb}{0.000000,0.000000,0.000000}%
\pgfsetstrokecolor{textcolor}%
\pgfsetfillcolor{textcolor}%
\pgftext[x=1.298348in,y=0.255000in,,top]{\color{textcolor}\sffamily\fontsize{10.000000}{12.000000}\selectfont \(\displaystyle {2\times10^{1}}\)}%
\end{pgfscope}%
\begin{pgfscope}%
\pgfsetbuttcap%
\pgfsetroundjoin%
\definecolor{currentfill}{rgb}{0.000000,0.000000,0.000000}%
\pgfsetfillcolor{currentfill}%
\pgfsetlinewidth{0.602250pt}%
\definecolor{currentstroke}{rgb}{0.000000,0.000000,0.000000}%
\pgfsetstrokecolor{currentstroke}%
\pgfsetdash{}{0pt}%
\pgfsys@defobject{currentmarker}{\pgfqpoint{0.000000in}{-0.027778in}}{\pgfqpoint{0.000000in}{0.000000in}}{%
\pgfpathmoveto{\pgfqpoint{0.000000in}{0.000000in}}%
\pgfpathlineto{\pgfqpoint{0.000000in}{-0.027778in}}%
\pgfusepath{stroke,fill}%
}%
\begin{pgfscope}%
\pgfsys@transformshift{1.776652in}{0.330000in}%
\pgfsys@useobject{currentmarker}{}%
\end{pgfscope}%
\end{pgfscope}%
\begin{pgfscope}%
\definecolor{textcolor}{rgb}{0.000000,0.000000,0.000000}%
\pgfsetstrokecolor{textcolor}%
\pgfsetfillcolor{textcolor}%
\pgftext[x=1.776652in,y=0.255000in,,top]{\color{textcolor}\sffamily\fontsize{10.000000}{12.000000}\selectfont \(\displaystyle {3\times10^{1}}\)}%
\end{pgfscope}%
\begin{pgfscope}%
\pgfsetbuttcap%
\pgfsetroundjoin%
\definecolor{currentfill}{rgb}{0.000000,0.000000,0.000000}%
\pgfsetfillcolor{currentfill}%
\pgfsetlinewidth{0.602250pt}%
\definecolor{currentstroke}{rgb}{0.000000,0.000000,0.000000}%
\pgfsetstrokecolor{currentstroke}%
\pgfsetdash{}{0pt}%
\pgfsys@defobject{currentmarker}{\pgfqpoint{0.000000in}{-0.027778in}}{\pgfqpoint{0.000000in}{0.000000in}}{%
\pgfpathmoveto{\pgfqpoint{0.000000in}{0.000000in}}%
\pgfpathlineto{\pgfqpoint{0.000000in}{-0.027778in}}%
\pgfusepath{stroke,fill}%
}%
\begin{pgfscope}%
\pgfsys@transformshift{2.116014in}{0.330000in}%
\pgfsys@useobject{currentmarker}{}%
\end{pgfscope}%
\end{pgfscope}%
\begin{pgfscope}%
\definecolor{textcolor}{rgb}{0.000000,0.000000,0.000000}%
\pgfsetstrokecolor{textcolor}%
\pgfsetfillcolor{textcolor}%
\pgftext[x=2.116014in,y=0.255000in,,top]{\color{textcolor}\sffamily\fontsize{10.000000}{12.000000}\selectfont \(\displaystyle {4\times10^{1}}\)}%
\end{pgfscope}%
\begin{pgfscope}%
\pgfsetbuttcap%
\pgfsetroundjoin%
\definecolor{currentfill}{rgb}{0.000000,0.000000,0.000000}%
\pgfsetfillcolor{currentfill}%
\pgfsetlinewidth{0.602250pt}%
\definecolor{currentstroke}{rgb}{0.000000,0.000000,0.000000}%
\pgfsetstrokecolor{currentstroke}%
\pgfsetdash{}{0pt}%
\pgfsys@defobject{currentmarker}{\pgfqpoint{0.000000in}{-0.027778in}}{\pgfqpoint{0.000000in}{0.000000in}}{%
\pgfpathmoveto{\pgfqpoint{0.000000in}{0.000000in}}%
\pgfpathlineto{\pgfqpoint{0.000000in}{-0.027778in}}%
\pgfusepath{stroke,fill}%
}%
\begin{pgfscope}%
\pgfsys@transformshift{2.379244in}{0.330000in}%
\pgfsys@useobject{currentmarker}{}%
\end{pgfscope}%
\end{pgfscope}%
\begin{pgfscope}%
\pgfsetbuttcap%
\pgfsetroundjoin%
\definecolor{currentfill}{rgb}{0.000000,0.000000,0.000000}%
\pgfsetfillcolor{currentfill}%
\pgfsetlinewidth{0.602250pt}%
\definecolor{currentstroke}{rgb}{0.000000,0.000000,0.000000}%
\pgfsetstrokecolor{currentstroke}%
\pgfsetdash{}{0pt}%
\pgfsys@defobject{currentmarker}{\pgfqpoint{0.000000in}{-0.027778in}}{\pgfqpoint{0.000000in}{0.000000in}}{%
\pgfpathmoveto{\pgfqpoint{0.000000in}{0.000000in}}%
\pgfpathlineto{\pgfqpoint{0.000000in}{-0.027778in}}%
\pgfusepath{stroke,fill}%
}%
\begin{pgfscope}%
\pgfsys@transformshift{2.594318in}{0.330000in}%
\pgfsys@useobject{currentmarker}{}%
\end{pgfscope}%
\end{pgfscope}%
\begin{pgfscope}%
\definecolor{textcolor}{rgb}{0.000000,0.000000,0.000000}%
\pgfsetstrokecolor{textcolor}%
\pgfsetfillcolor{textcolor}%
\pgftext[x=2.594318in,y=0.255000in,,top]{\color{textcolor}\sffamily\fontsize{10.000000}{12.000000}\selectfont \(\displaystyle {6\times10^{1}}\)}%
\end{pgfscope}%
\begin{pgfscope}%
\definecolor{textcolor}{rgb}{0.000000,0.000000,0.000000}%
\pgfsetstrokecolor{textcolor}%
\pgfsetfillcolor{textcolor}%
\pgftext[x=1.537500in,y=0.042809in,,top]{\color{textcolor}\sffamily\fontsize{10.000000}{12.000000}\selectfont \(\displaystyle k\)}%
\end{pgfscope}%
\begin{pgfscope}%
\pgfsetbuttcap%
\pgfsetroundjoin%
\definecolor{currentfill}{rgb}{0.000000,0.000000,0.000000}%
\pgfsetfillcolor{currentfill}%
\pgfsetlinewidth{0.803000pt}%
\definecolor{currentstroke}{rgb}{0.000000,0.000000,0.000000}%
\pgfsetstrokecolor{currentstroke}%
\pgfsetdash{}{0pt}%
\pgfsys@defobject{currentmarker}{\pgfqpoint{-0.048611in}{0.000000in}}{\pgfqpoint{0.000000in}{0.000000in}}{%
\pgfpathmoveto{\pgfqpoint{0.000000in}{0.000000in}}%
\pgfpathlineto{\pgfqpoint{-0.048611in}{0.000000in}}%
\pgfusepath{stroke,fill}%
}%
\begin{pgfscope}%
\pgfsys@transformshift{0.375000in}{0.619643in}%
\pgfsys@useobject{currentmarker}{}%
\end{pgfscope}%
\end{pgfscope}%
\begin{pgfscope}%
\definecolor{textcolor}{rgb}{0.000000,0.000000,0.000000}%
\pgfsetstrokecolor{textcolor}%
\pgfsetfillcolor{textcolor}%
\pgftext[x=-0.031467in,y=0.566881in,left,base]{\color{textcolor}\sffamily\fontsize{10.000000}{12.000000}\selectfont 0.70}%
\end{pgfscope}%
\begin{pgfscope}%
\pgfsetbuttcap%
\pgfsetroundjoin%
\definecolor{currentfill}{rgb}{0.000000,0.000000,0.000000}%
\pgfsetfillcolor{currentfill}%
\pgfsetlinewidth{0.803000pt}%
\definecolor{currentstroke}{rgb}{0.000000,0.000000,0.000000}%
\pgfsetstrokecolor{currentstroke}%
\pgfsetdash{}{0pt}%
\pgfsys@defobject{currentmarker}{\pgfqpoint{-0.048611in}{0.000000in}}{\pgfqpoint{0.000000in}{0.000000in}}{%
\pgfpathmoveto{\pgfqpoint{0.000000in}{0.000000in}}%
\pgfpathlineto{\pgfqpoint{-0.048611in}{0.000000in}}%
\pgfusepath{stroke,fill}%
}%
\begin{pgfscope}%
\pgfsys@transformshift{0.375000in}{0.970999in}%
\pgfsys@useobject{currentmarker}{}%
\end{pgfscope}%
\end{pgfscope}%
\begin{pgfscope}%
\definecolor{textcolor}{rgb}{0.000000,0.000000,0.000000}%
\pgfsetstrokecolor{textcolor}%
\pgfsetfillcolor{textcolor}%
\pgftext[x=-0.031467in,y=0.918238in,left,base]{\color{textcolor}\sffamily\fontsize{10.000000}{12.000000}\selectfont 0.75}%
\end{pgfscope}%
\begin{pgfscope}%
\pgfsetbuttcap%
\pgfsetroundjoin%
\definecolor{currentfill}{rgb}{0.000000,0.000000,0.000000}%
\pgfsetfillcolor{currentfill}%
\pgfsetlinewidth{0.803000pt}%
\definecolor{currentstroke}{rgb}{0.000000,0.000000,0.000000}%
\pgfsetstrokecolor{currentstroke}%
\pgfsetdash{}{0pt}%
\pgfsys@defobject{currentmarker}{\pgfqpoint{-0.048611in}{0.000000in}}{\pgfqpoint{0.000000in}{0.000000in}}{%
\pgfpathmoveto{\pgfqpoint{0.000000in}{0.000000in}}%
\pgfpathlineto{\pgfqpoint{-0.048611in}{0.000000in}}%
\pgfusepath{stroke,fill}%
}%
\begin{pgfscope}%
\pgfsys@transformshift{0.375000in}{1.322356in}%
\pgfsys@useobject{currentmarker}{}%
\end{pgfscope}%
\end{pgfscope}%
\begin{pgfscope}%
\definecolor{textcolor}{rgb}{0.000000,0.000000,0.000000}%
\pgfsetstrokecolor{textcolor}%
\pgfsetfillcolor{textcolor}%
\pgftext[x=-0.031467in,y=1.269594in,left,base]{\color{textcolor}\sffamily\fontsize{10.000000}{12.000000}\selectfont 0.80}%
\end{pgfscope}%
\begin{pgfscope}%
\pgfsetbuttcap%
\pgfsetroundjoin%
\definecolor{currentfill}{rgb}{0.000000,0.000000,0.000000}%
\pgfsetfillcolor{currentfill}%
\pgfsetlinewidth{0.803000pt}%
\definecolor{currentstroke}{rgb}{0.000000,0.000000,0.000000}%
\pgfsetstrokecolor{currentstroke}%
\pgfsetdash{}{0pt}%
\pgfsys@defobject{currentmarker}{\pgfqpoint{-0.048611in}{0.000000in}}{\pgfqpoint{0.000000in}{0.000000in}}{%
\pgfpathmoveto{\pgfqpoint{0.000000in}{0.000000in}}%
\pgfpathlineto{\pgfqpoint{-0.048611in}{0.000000in}}%
\pgfusepath{stroke,fill}%
}%
\begin{pgfscope}%
\pgfsys@transformshift{0.375000in}{1.673713in}%
\pgfsys@useobject{currentmarker}{}%
\end{pgfscope}%
\end{pgfscope}%
\begin{pgfscope}%
\definecolor{textcolor}{rgb}{0.000000,0.000000,0.000000}%
\pgfsetstrokecolor{textcolor}%
\pgfsetfillcolor{textcolor}%
\pgftext[x=-0.031467in,y=1.620951in,left,base]{\color{textcolor}\sffamily\fontsize{10.000000}{12.000000}\selectfont 0.85}%
\end{pgfscope}%
\begin{pgfscope}%
\pgfsetbuttcap%
\pgfsetroundjoin%
\definecolor{currentfill}{rgb}{0.000000,0.000000,0.000000}%
\pgfsetfillcolor{currentfill}%
\pgfsetlinewidth{0.803000pt}%
\definecolor{currentstroke}{rgb}{0.000000,0.000000,0.000000}%
\pgfsetstrokecolor{currentstroke}%
\pgfsetdash{}{0pt}%
\pgfsys@defobject{currentmarker}{\pgfqpoint{-0.048611in}{0.000000in}}{\pgfqpoint{0.000000in}{0.000000in}}{%
\pgfpathmoveto{\pgfqpoint{0.000000in}{0.000000in}}%
\pgfpathlineto{\pgfqpoint{-0.048611in}{0.000000in}}%
\pgfusepath{stroke,fill}%
}%
\begin{pgfscope}%
\pgfsys@transformshift{0.375000in}{2.025069in}%
\pgfsys@useobject{currentmarker}{}%
\end{pgfscope}%
\end{pgfscope}%
\begin{pgfscope}%
\definecolor{textcolor}{rgb}{0.000000,0.000000,0.000000}%
\pgfsetstrokecolor{textcolor}%
\pgfsetfillcolor{textcolor}%
\pgftext[x=-0.031467in,y=1.972308in,left,base]{\color{textcolor}\sffamily\fontsize{10.000000}{12.000000}\selectfont 0.90}%
\end{pgfscope}%
\begin{pgfscope}%
\pgfsetbuttcap%
\pgfsetroundjoin%
\definecolor{currentfill}{rgb}{0.000000,0.000000,0.000000}%
\pgfsetfillcolor{currentfill}%
\pgfsetlinewidth{0.803000pt}%
\definecolor{currentstroke}{rgb}{0.000000,0.000000,0.000000}%
\pgfsetstrokecolor{currentstroke}%
\pgfsetdash{}{0pt}%
\pgfsys@defobject{currentmarker}{\pgfqpoint{-0.048611in}{0.000000in}}{\pgfqpoint{0.000000in}{0.000000in}}{%
\pgfpathmoveto{\pgfqpoint{0.000000in}{0.000000in}}%
\pgfpathlineto{\pgfqpoint{-0.048611in}{0.000000in}}%
\pgfusepath{stroke,fill}%
}%
\begin{pgfscope}%
\pgfsys@transformshift{0.375000in}{2.376426in}%
\pgfsys@useobject{currentmarker}{}%
\end{pgfscope}%
\end{pgfscope}%
\begin{pgfscope}%
\definecolor{textcolor}{rgb}{0.000000,0.000000,0.000000}%
\pgfsetstrokecolor{textcolor}%
\pgfsetfillcolor{textcolor}%
\pgftext[x=-0.031467in,y=2.323664in,left,base]{\color{textcolor}\sffamily\fontsize{10.000000}{12.000000}\selectfont 0.95}%
\end{pgfscope}%
\begin{pgfscope}%
\definecolor{textcolor}{rgb}{0.000000,0.000000,0.000000}%
\pgfsetstrokecolor{textcolor}%
\pgfsetfillcolor{textcolor}%
\pgftext[x=-0.087023in,y=1.485000in,,bottom,rotate=90.000000]{\color{textcolor}\sffamily\fontsize{10.000000}{12.000000}\selectfont \(\displaystyle \alpha\)}%
\end{pgfscope}%
\begin{pgfscope}%
\pgfpathrectangle{\pgfqpoint{0.375000in}{0.330000in}}{\pgfqpoint{2.325000in}{2.310000in}}%
\pgfusepath{clip}%
\pgfsetbuttcap%
\pgfsetroundjoin%
\pgfsetlinewidth{1.505625pt}%
\definecolor{currentstroke}{rgb}{0.000000,0.000000,0.000000}%
\pgfsetstrokecolor{currentstroke}%
\pgfsetdash{{5.550000pt}{2.400000pt}}{0.000000pt}%
\pgfpathmoveto{\pgfqpoint{0.480682in}{2.535000in}}%
\pgfpathlineto{\pgfqpoint{1.298348in}{1.722609in}}%
\pgfpathlineto{\pgfqpoint{1.776652in}{1.247391in}}%
\pgfpathlineto{\pgfqpoint{2.116014in}{0.910218in}}%
\pgfpathlineto{\pgfqpoint{2.379244in}{0.648687in}}%
\pgfpathlineto{\pgfqpoint{2.594318in}{0.435000in}}%
\pgfusepath{stroke}%
\end{pgfscope}%
\begin{pgfscope}%
\pgfpathrectangle{\pgfqpoint{0.375000in}{0.330000in}}{\pgfqpoint{2.325000in}{2.310000in}}%
\pgfusepath{clip}%
\pgfsetbuttcap%
\pgfsetroundjoin%
\definecolor{currentfill}{rgb}{0.000000,0.000000,0.000000}%
\pgfsetfillcolor{currentfill}%
\pgfsetlinewidth{1.003750pt}%
\definecolor{currentstroke}{rgb}{0.000000,0.000000,0.000000}%
\pgfsetstrokecolor{currentstroke}%
\pgfsetdash{}{0pt}%
\pgfsys@defobject{currentmarker}{\pgfqpoint{-0.041667in}{-0.041667in}}{\pgfqpoint{0.041667in}{0.041667in}}{%
\pgfpathmoveto{\pgfqpoint{0.000000in}{-0.041667in}}%
\pgfpathcurveto{\pgfqpoint{0.011050in}{-0.041667in}}{\pgfqpoint{0.021649in}{-0.037276in}}{\pgfqpoint{0.029463in}{-0.029463in}}%
\pgfpathcurveto{\pgfqpoint{0.037276in}{-0.021649in}}{\pgfqpoint{0.041667in}{-0.011050in}}{\pgfqpoint{0.041667in}{0.000000in}}%
\pgfpathcurveto{\pgfqpoint{0.041667in}{0.011050in}}{\pgfqpoint{0.037276in}{0.021649in}}{\pgfqpoint{0.029463in}{0.029463in}}%
\pgfpathcurveto{\pgfqpoint{0.021649in}{0.037276in}}{\pgfqpoint{0.011050in}{0.041667in}}{\pgfqpoint{0.000000in}{0.041667in}}%
\pgfpathcurveto{\pgfqpoint{-0.011050in}{0.041667in}}{\pgfqpoint{-0.021649in}{0.037276in}}{\pgfqpoint{-0.029463in}{0.029463in}}%
\pgfpathcurveto{\pgfqpoint{-0.037276in}{0.021649in}}{\pgfqpoint{-0.041667in}{0.011050in}}{\pgfqpoint{-0.041667in}{0.000000in}}%
\pgfpathcurveto{\pgfqpoint{-0.041667in}{-0.011050in}}{\pgfqpoint{-0.037276in}{-0.021649in}}{\pgfqpoint{-0.029463in}{-0.029463in}}%
\pgfpathcurveto{\pgfqpoint{-0.021649in}{-0.037276in}}{\pgfqpoint{-0.011050in}{-0.041667in}}{\pgfqpoint{0.000000in}{-0.041667in}}%
\pgfpathclose%
\pgfusepath{stroke,fill}%
}%
\begin{pgfscope}%
\pgfsys@transformshift{0.480682in}{2.512308in}%
\pgfsys@useobject{currentmarker}{}%
\end{pgfscope}%
\begin{pgfscope}%
\pgfsys@transformshift{1.298348in}{1.821112in}%
\pgfsys@useobject{currentmarker}{}%
\end{pgfscope}%
\begin{pgfscope}%
\pgfsys@transformshift{1.776652in}{1.142383in}%
\pgfsys@useobject{currentmarker}{}%
\end{pgfscope}%
\begin{pgfscope}%
\pgfsys@transformshift{2.116014in}{0.929580in}%
\pgfsys@useobject{currentmarker}{}%
\end{pgfscope}%
\begin{pgfscope}%
\pgfsys@transformshift{2.379244in}{0.634304in}%
\pgfsys@useobject{currentmarker}{}%
\end{pgfscope}%
\begin{pgfscope}%
\pgfsys@transformshift{2.594318in}{0.459218in}%
\pgfsys@useobject{currentmarker}{}%
\end{pgfscope}%
\end{pgfscope}%
\begin{pgfscope}%
\pgfsetrectcap%
\pgfsetmiterjoin%
\pgfsetlinewidth{0.803000pt}%
\definecolor{currentstroke}{rgb}{0.000000,0.000000,0.000000}%
\pgfsetstrokecolor{currentstroke}%
\pgfsetdash{}{0pt}%
\pgfpathmoveto{\pgfqpoint{0.375000in}{0.330000in}}%
\pgfpathlineto{\pgfqpoint{0.375000in}{2.640000in}}%
\pgfusepath{stroke}%
\end{pgfscope}%
\begin{pgfscope}%
\pgfsetrectcap%
\pgfsetmiterjoin%
\pgfsetlinewidth{0.803000pt}%
\definecolor{currentstroke}{rgb}{0.000000,0.000000,0.000000}%
\pgfsetstrokecolor{currentstroke}%
\pgfsetdash{}{0pt}%
\pgfpathmoveto{\pgfqpoint{2.700000in}{0.330000in}}%
\pgfpathlineto{\pgfqpoint{2.700000in}{2.640000in}}%
\pgfusepath{stroke}%
\end{pgfscope}%
\begin{pgfscope}%
\pgfsetrectcap%
\pgfsetmiterjoin%
\pgfsetlinewidth{0.803000pt}%
\definecolor{currentstroke}{rgb}{0.000000,0.000000,0.000000}%
\pgfsetstrokecolor{currentstroke}%
\pgfsetdash{}{0pt}%
\pgfpathmoveto{\pgfqpoint{0.375000in}{0.330000in}}%
\pgfpathlineto{\pgfqpoint{2.700000in}{0.330000in}}%
\pgfusepath{stroke}%
\end{pgfscope}%
\begin{pgfscope}%
\pgfsetrectcap%
\pgfsetmiterjoin%
\pgfsetlinewidth{0.803000pt}%
\definecolor{currentstroke}{rgb}{0.000000,0.000000,0.000000}%
\pgfsetstrokecolor{currentstroke}%
\pgfsetdash{}{0pt}%
\pgfpathmoveto{\pgfqpoint{0.375000in}{2.640000in}}%
\pgfpathlineto{\pgfqpoint{2.700000in}{2.640000in}}%
\pgfusepath{stroke}%
\end{pgfscope}%
\begin{pgfscope}%
\pgfsetbuttcap%
\pgfsetmiterjoin%
\definecolor{currentfill}{rgb}{1.000000,1.000000,1.000000}%
\pgfsetfillcolor{currentfill}%
\pgfsetfillopacity{0.800000}%
\pgfsetlinewidth{1.003750pt}%
\definecolor{currentstroke}{rgb}{0.800000,0.800000,0.800000}%
\pgfsetstrokecolor{currentstroke}%
\pgfsetstrokeopacity{0.800000}%
\pgfsetdash{}{0pt}%
\pgfpathmoveto{\pgfqpoint{0.517632in}{2.319199in}}%
\pgfpathlineto{\pgfqpoint{2.602778in}{2.319199in}}%
\pgfpathquadraticcurveto{\pgfqpoint{2.630556in}{2.319199in}}{\pgfqpoint{2.630556in}{2.346977in}}%
\pgfpathlineto{\pgfqpoint{2.630556in}{2.542778in}}%
\pgfpathquadraticcurveto{\pgfqpoint{2.630556in}{2.570556in}}{\pgfqpoint{2.602778in}{2.570556in}}%
\pgfpathlineto{\pgfqpoint{0.517632in}{2.570556in}}%
\pgfpathquadraticcurveto{\pgfqpoint{0.489854in}{2.570556in}}{\pgfqpoint{0.489854in}{2.542778in}}%
\pgfpathlineto{\pgfqpoint{0.489854in}{2.346977in}}%
\pgfpathquadraticcurveto{\pgfqpoint{0.489854in}{2.319199in}}{\pgfqpoint{0.517632in}{2.319199in}}%
\pgfpathclose%
\pgfusepath{stroke,fill}%
\end{pgfscope}%
\begin{pgfscope}%
\pgfsetbuttcap%
\pgfsetroundjoin%
\pgfsetlinewidth{1.505625pt}%
\definecolor{currentstroke}{rgb}{0.000000,0.000000,0.000000}%
\pgfsetstrokecolor{currentstroke}%
\pgfsetdash{{5.550000pt}{2.400000pt}}{0.000000pt}%
\pgfpathmoveto{\pgfqpoint{0.545410in}{2.458088in}}%
\pgfpathlineto{\pgfqpoint{0.823187in}{2.458088in}}%
\pgfusepath{stroke}%
\end{pgfscope}%
\begin{pgfscope}%
\definecolor{textcolor}{rgb}{0.000000,0.000000,0.000000}%
\pgfsetstrokecolor{textcolor}%
\pgfsetfillcolor{textcolor}%
\pgftext[x=0.934298in,y=2.409477in,left,base]{\color{textcolor}\sffamily\fontsize{10.000000}{12.000000}\selectfont \(\displaystyle \alpha = 1.3566 - 0.3840\log(k)\)}%
\end{pgfscope}%
\end{pgfpicture}%
\makeatother%
\endgroup%

    \end{subfigure}
\caption{Plots of the dependence of $C$ and $\alpha$ on $k$ in \cref{eq:qmcerrorform} for $Q(u) =  u(\bzero)$. Observe the $x$-axis is on a $\log_{10}$ scale, but $\log$ is the natural logarithm. \label{fig:originCalpha}}
\end{figure}

\begin{figure}[h]
    \centering
  \begin{subfigure}{\textwidth}
%% Creator: Matplotlib, PGF backend
%%
%% To include the figure in your LaTeX document, write
%%   \input{<filename>.pgf}
%%
%% Make sure the required packages are loaded in your preamble
%%   \usepackage{pgf}
%%
%% Figures using additional raster images can only be included by \input if
%% they are in the same directory as the main LaTeX file. For loading figures
%% from other directories you can use the `import` package
%%   \usepackage{import}
%% and then include the figures with
%%   \import{<path to file>}{<filename>.pgf}
%%
%% Matplotlib used the following preamble
%%   \usepackage{fontspec}
%%   \setmainfont{DejaVuSerif.ttf}[Path=/home/owen/progs/firedrake-complex/firedrake/lib/python3.5/site-packages/matplotlib/mpl-data/fonts/ttf/]
%%   \setsansfont{DejaVuSans.ttf}[Path=/home/owen/progs/firedrake-complex/firedrake/lib/python3.5/site-packages/matplotlib/mpl-data/fonts/ttf/]
%%   \setmonofont{DejaVuSansMono.ttf}[Path=/home/owen/progs/firedrake-complex/firedrake/lib/python3.5/site-packages/matplotlib/mpl-data/fonts/ttf/]
%%
\begingroup%
\makeatletter%
\begin{pgfpicture}%
\pgfpathrectangle{\pgfpointorigin}{\pgfqpoint{5.000000in}{4.000000in}}%
\pgfusepath{use as bounding box, clip}%
\begin{pgfscope}%
\pgfsetbuttcap%
\pgfsetmiterjoin%
\definecolor{currentfill}{rgb}{1.000000,1.000000,1.000000}%
\pgfsetfillcolor{currentfill}%
\pgfsetlinewidth{0.000000pt}%
\definecolor{currentstroke}{rgb}{1.000000,1.000000,1.000000}%
\pgfsetstrokecolor{currentstroke}%
\pgfsetdash{}{0pt}%
\pgfpathmoveto{\pgfqpoint{0.000000in}{0.000000in}}%
\pgfpathlineto{\pgfqpoint{5.000000in}{0.000000in}}%
\pgfpathlineto{\pgfqpoint{5.000000in}{4.000000in}}%
\pgfpathlineto{\pgfqpoint{0.000000in}{4.000000in}}%
\pgfpathclose%
\pgfusepath{fill}%
\end{pgfscope}%
\begin{pgfscope}%
\pgfsetbuttcap%
\pgfsetmiterjoin%
\definecolor{currentfill}{rgb}{1.000000,1.000000,1.000000}%
\pgfsetfillcolor{currentfill}%
\pgfsetlinewidth{0.000000pt}%
\definecolor{currentstroke}{rgb}{0.000000,0.000000,0.000000}%
\pgfsetstrokecolor{currentstroke}%
\pgfsetstrokeopacity{0.000000}%
\pgfsetdash{}{0pt}%
\pgfpathmoveto{\pgfqpoint{0.629421in}{0.617778in}}%
\pgfpathlineto{\pgfqpoint{4.793685in}{0.617778in}}%
\pgfpathlineto{\pgfqpoint{4.793685in}{3.800000in}}%
\pgfpathlineto{\pgfqpoint{0.629421in}{3.800000in}}%
\pgfpathclose%
\pgfusepath{fill}%
\end{pgfscope}%
\begin{pgfscope}%
\pgfsetbuttcap%
\pgfsetroundjoin%
\definecolor{currentfill}{rgb}{0.000000,0.000000,0.000000}%
\pgfsetfillcolor{currentfill}%
\pgfsetlinewidth{0.803000pt}%
\definecolor{currentstroke}{rgb}{0.000000,0.000000,0.000000}%
\pgfsetstrokecolor{currentstroke}%
\pgfsetdash{}{0pt}%
\pgfsys@defobject{currentmarker}{\pgfqpoint{0.000000in}{-0.048611in}}{\pgfqpoint{0.000000in}{0.000000in}}{%
\pgfpathmoveto{\pgfqpoint{0.000000in}{0.000000in}}%
\pgfpathlineto{\pgfqpoint{0.000000in}{-0.048611in}}%
\pgfusepath{stroke,fill}%
}%
\begin{pgfscope}%
\pgfsys@transformshift{0.818706in}{0.617778in}%
\pgfsys@useobject{currentmarker}{}%
\end{pgfscope}%
\end{pgfscope}%
\begin{pgfscope}%
\definecolor{textcolor}{rgb}{0.000000,0.000000,0.000000}%
\pgfsetstrokecolor{textcolor}%
\pgfsetfillcolor{textcolor}%
\pgftext[x=0.818706in,y=0.520556in,,top]{\color{textcolor}\sffamily\fontsize{11.000000}{13.200000}\selectfont \(\displaystyle 10^{1}\)}%
\end{pgfscope}%
\begin{pgfscope}%
\pgfsetbuttcap%
\pgfsetroundjoin%
\definecolor{currentfill}{rgb}{0.000000,0.000000,0.000000}%
\pgfsetfillcolor{currentfill}%
\pgfsetlinewidth{0.602250pt}%
\definecolor{currentstroke}{rgb}{0.000000,0.000000,0.000000}%
\pgfsetstrokecolor{currentstroke}%
\pgfsetdash{}{0pt}%
\pgfsys@defobject{currentmarker}{\pgfqpoint{0.000000in}{-0.027778in}}{\pgfqpoint{0.000000in}{0.000000in}}{%
\pgfpathmoveto{\pgfqpoint{0.000000in}{0.000000in}}%
\pgfpathlineto{\pgfqpoint{0.000000in}{-0.027778in}}%
\pgfusepath{stroke,fill}%
}%
\begin{pgfscope}%
\pgfsys@transformshift{2.283212in}{0.617778in}%
\pgfsys@useobject{currentmarker}{}%
\end{pgfscope}%
\end{pgfscope}%
\begin{pgfscope}%
\definecolor{textcolor}{rgb}{0.000000,0.000000,0.000000}%
\pgfsetstrokecolor{textcolor}%
\pgfsetfillcolor{textcolor}%
\pgftext[x=2.283212in,y=0.542778in,,top]{\color{textcolor}\sffamily\fontsize{11.000000}{13.200000}\selectfont \(\displaystyle 2\times10^{1}\)}%
\end{pgfscope}%
\begin{pgfscope}%
\pgfsetbuttcap%
\pgfsetroundjoin%
\definecolor{currentfill}{rgb}{0.000000,0.000000,0.000000}%
\pgfsetfillcolor{currentfill}%
\pgfsetlinewidth{0.602250pt}%
\definecolor{currentstroke}{rgb}{0.000000,0.000000,0.000000}%
\pgfsetstrokecolor{currentstroke}%
\pgfsetdash{}{0pt}%
\pgfsys@defobject{currentmarker}{\pgfqpoint{0.000000in}{-0.027778in}}{\pgfqpoint{0.000000in}{0.000000in}}{%
\pgfpathmoveto{\pgfqpoint{0.000000in}{0.000000in}}%
\pgfpathlineto{\pgfqpoint{0.000000in}{-0.027778in}}%
\pgfusepath{stroke,fill}%
}%
\begin{pgfscope}%
\pgfsys@transformshift{3.139894in}{0.617778in}%
\pgfsys@useobject{currentmarker}{}%
\end{pgfscope}%
\end{pgfscope}%
\begin{pgfscope}%
\definecolor{textcolor}{rgb}{0.000000,0.000000,0.000000}%
\pgfsetstrokecolor{textcolor}%
\pgfsetfillcolor{textcolor}%
\pgftext[x=3.139894in,y=0.542778in,,top]{\color{textcolor}\sffamily\fontsize{11.000000}{13.200000}\selectfont \(\displaystyle 3\times10^{1}\)}%
\end{pgfscope}%
\begin{pgfscope}%
\pgfsetbuttcap%
\pgfsetroundjoin%
\definecolor{currentfill}{rgb}{0.000000,0.000000,0.000000}%
\pgfsetfillcolor{currentfill}%
\pgfsetlinewidth{0.602250pt}%
\definecolor{currentstroke}{rgb}{0.000000,0.000000,0.000000}%
\pgfsetstrokecolor{currentstroke}%
\pgfsetdash{}{0pt}%
\pgfsys@defobject{currentmarker}{\pgfqpoint{0.000000in}{-0.027778in}}{\pgfqpoint{0.000000in}{0.000000in}}{%
\pgfpathmoveto{\pgfqpoint{0.000000in}{0.000000in}}%
\pgfpathlineto{\pgfqpoint{0.000000in}{-0.027778in}}%
\pgfusepath{stroke,fill}%
}%
\begin{pgfscope}%
\pgfsys@transformshift{3.747719in}{0.617778in}%
\pgfsys@useobject{currentmarker}{}%
\end{pgfscope}%
\end{pgfscope}%
\begin{pgfscope}%
\definecolor{textcolor}{rgb}{0.000000,0.000000,0.000000}%
\pgfsetstrokecolor{textcolor}%
\pgfsetfillcolor{textcolor}%
\pgftext[x=3.747719in,y=0.542778in,,top]{\color{textcolor}\sffamily\fontsize{11.000000}{13.200000}\selectfont \(\displaystyle 4\times10^{1}\)}%
\end{pgfscope}%
\begin{pgfscope}%
\pgfsetbuttcap%
\pgfsetroundjoin%
\definecolor{currentfill}{rgb}{0.000000,0.000000,0.000000}%
\pgfsetfillcolor{currentfill}%
\pgfsetlinewidth{0.602250pt}%
\definecolor{currentstroke}{rgb}{0.000000,0.000000,0.000000}%
\pgfsetstrokecolor{currentstroke}%
\pgfsetdash{}{0pt}%
\pgfsys@defobject{currentmarker}{\pgfqpoint{0.000000in}{-0.027778in}}{\pgfqpoint{0.000000in}{0.000000in}}{%
\pgfpathmoveto{\pgfqpoint{0.000000in}{0.000000in}}%
\pgfpathlineto{\pgfqpoint{0.000000in}{-0.027778in}}%
\pgfusepath{stroke,fill}%
}%
\begin{pgfscope}%
\pgfsys@transformshift{4.219184in}{0.617778in}%
\pgfsys@useobject{currentmarker}{}%
\end{pgfscope}%
\end{pgfscope}%
\begin{pgfscope}%
\pgfsetbuttcap%
\pgfsetroundjoin%
\definecolor{currentfill}{rgb}{0.000000,0.000000,0.000000}%
\pgfsetfillcolor{currentfill}%
\pgfsetlinewidth{0.602250pt}%
\definecolor{currentstroke}{rgb}{0.000000,0.000000,0.000000}%
\pgfsetstrokecolor{currentstroke}%
\pgfsetdash{}{0pt}%
\pgfsys@defobject{currentmarker}{\pgfqpoint{0.000000in}{-0.027778in}}{\pgfqpoint{0.000000in}{0.000000in}}{%
\pgfpathmoveto{\pgfqpoint{0.000000in}{0.000000in}}%
\pgfpathlineto{\pgfqpoint{0.000000in}{-0.027778in}}%
\pgfusepath{stroke,fill}%
}%
\begin{pgfscope}%
\pgfsys@transformshift{4.604400in}{0.617778in}%
\pgfsys@useobject{currentmarker}{}%
\end{pgfscope}%
\end{pgfscope}%
\begin{pgfscope}%
\definecolor{textcolor}{rgb}{0.000000,0.000000,0.000000}%
\pgfsetstrokecolor{textcolor}%
\pgfsetfillcolor{textcolor}%
\pgftext[x=4.604400in,y=0.542778in,,top]{\color{textcolor}\sffamily\fontsize{11.000000}{13.200000}\selectfont \(\displaystyle 6\times10^{1}\)}%
\end{pgfscope}%
\begin{pgfscope}%
\definecolor{textcolor}{rgb}{0.000000,0.000000,0.000000}%
\pgfsetstrokecolor{textcolor}%
\pgfsetfillcolor{textcolor}%
\pgftext[x=2.711553in,y=0.317146in,,top]{\color{textcolor}\sffamily\fontsize{11.000000}{13.200000}\selectfont \(\displaystyle k\)}%
\end{pgfscope}%
\begin{pgfscope}%
\pgfsetbuttcap%
\pgfsetroundjoin%
\definecolor{currentfill}{rgb}{0.000000,0.000000,0.000000}%
\pgfsetfillcolor{currentfill}%
\pgfsetlinewidth{0.803000pt}%
\definecolor{currentstroke}{rgb}{0.000000,0.000000,0.000000}%
\pgfsetstrokecolor{currentstroke}%
\pgfsetdash{}{0pt}%
\pgfsys@defobject{currentmarker}{\pgfqpoint{-0.048611in}{0.000000in}}{\pgfqpoint{0.000000in}{0.000000in}}{%
\pgfpathmoveto{\pgfqpoint{0.000000in}{0.000000in}}%
\pgfpathlineto{\pgfqpoint{-0.048611in}{0.000000in}}%
\pgfusepath{stroke,fill}%
}%
\begin{pgfscope}%
\pgfsys@transformshift{0.629421in}{0.964204in}%
\pgfsys@useobject{currentmarker}{}%
\end{pgfscope}%
\end{pgfscope}%
\begin{pgfscope}%
\definecolor{textcolor}{rgb}{0.000000,0.000000,0.000000}%
\pgfsetstrokecolor{textcolor}%
\pgfsetfillcolor{textcolor}%
\pgftext[x=0.261828in,y=0.906167in,left,base]{\color{textcolor}\sffamily\fontsize{11.000000}{13.200000}\selectfont \(\displaystyle 0.10\)}%
\end{pgfscope}%
\begin{pgfscope}%
\pgfsetbuttcap%
\pgfsetroundjoin%
\definecolor{currentfill}{rgb}{0.000000,0.000000,0.000000}%
\pgfsetfillcolor{currentfill}%
\pgfsetlinewidth{0.803000pt}%
\definecolor{currentstroke}{rgb}{0.000000,0.000000,0.000000}%
\pgfsetstrokecolor{currentstroke}%
\pgfsetdash{}{0pt}%
\pgfsys@defobject{currentmarker}{\pgfqpoint{-0.048611in}{0.000000in}}{\pgfqpoint{0.000000in}{0.000000in}}{%
\pgfpathmoveto{\pgfqpoint{0.000000in}{0.000000in}}%
\pgfpathlineto{\pgfqpoint{-0.048611in}{0.000000in}}%
\pgfusepath{stroke,fill}%
}%
\begin{pgfscope}%
\pgfsys@transformshift{0.629421in}{1.573444in}%
\pgfsys@useobject{currentmarker}{}%
\end{pgfscope}%
\end{pgfscope}%
\begin{pgfscope}%
\definecolor{textcolor}{rgb}{0.000000,0.000000,0.000000}%
\pgfsetstrokecolor{textcolor}%
\pgfsetfillcolor{textcolor}%
\pgftext[x=0.261828in,y=1.515406in,left,base]{\color{textcolor}\sffamily\fontsize{11.000000}{13.200000}\selectfont \(\displaystyle 0.12\)}%
\end{pgfscope}%
\begin{pgfscope}%
\pgfsetbuttcap%
\pgfsetroundjoin%
\definecolor{currentfill}{rgb}{0.000000,0.000000,0.000000}%
\pgfsetfillcolor{currentfill}%
\pgfsetlinewidth{0.803000pt}%
\definecolor{currentstroke}{rgb}{0.000000,0.000000,0.000000}%
\pgfsetstrokecolor{currentstroke}%
\pgfsetdash{}{0pt}%
\pgfsys@defobject{currentmarker}{\pgfqpoint{-0.048611in}{0.000000in}}{\pgfqpoint{0.000000in}{0.000000in}}{%
\pgfpathmoveto{\pgfqpoint{0.000000in}{0.000000in}}%
\pgfpathlineto{\pgfqpoint{-0.048611in}{0.000000in}}%
\pgfusepath{stroke,fill}%
}%
\begin{pgfscope}%
\pgfsys@transformshift{0.629421in}{2.182683in}%
\pgfsys@useobject{currentmarker}{}%
\end{pgfscope}%
\end{pgfscope}%
\begin{pgfscope}%
\definecolor{textcolor}{rgb}{0.000000,0.000000,0.000000}%
\pgfsetstrokecolor{textcolor}%
\pgfsetfillcolor{textcolor}%
\pgftext[x=0.261828in,y=2.124646in,left,base]{\color{textcolor}\sffamily\fontsize{11.000000}{13.200000}\selectfont \(\displaystyle 0.14\)}%
\end{pgfscope}%
\begin{pgfscope}%
\pgfsetbuttcap%
\pgfsetroundjoin%
\definecolor{currentfill}{rgb}{0.000000,0.000000,0.000000}%
\pgfsetfillcolor{currentfill}%
\pgfsetlinewidth{0.803000pt}%
\definecolor{currentstroke}{rgb}{0.000000,0.000000,0.000000}%
\pgfsetstrokecolor{currentstroke}%
\pgfsetdash{}{0pt}%
\pgfsys@defobject{currentmarker}{\pgfqpoint{-0.048611in}{0.000000in}}{\pgfqpoint{0.000000in}{0.000000in}}{%
\pgfpathmoveto{\pgfqpoint{0.000000in}{0.000000in}}%
\pgfpathlineto{\pgfqpoint{-0.048611in}{0.000000in}}%
\pgfusepath{stroke,fill}%
}%
\begin{pgfscope}%
\pgfsys@transformshift{0.629421in}{2.791923in}%
\pgfsys@useobject{currentmarker}{}%
\end{pgfscope}%
\end{pgfscope}%
\begin{pgfscope}%
\definecolor{textcolor}{rgb}{0.000000,0.000000,0.000000}%
\pgfsetstrokecolor{textcolor}%
\pgfsetfillcolor{textcolor}%
\pgftext[x=0.261828in,y=2.733885in,left,base]{\color{textcolor}\sffamily\fontsize{11.000000}{13.200000}\selectfont \(\displaystyle 0.16\)}%
\end{pgfscope}%
\begin{pgfscope}%
\pgfsetbuttcap%
\pgfsetroundjoin%
\definecolor{currentfill}{rgb}{0.000000,0.000000,0.000000}%
\pgfsetfillcolor{currentfill}%
\pgfsetlinewidth{0.803000pt}%
\definecolor{currentstroke}{rgb}{0.000000,0.000000,0.000000}%
\pgfsetstrokecolor{currentstroke}%
\pgfsetdash{}{0pt}%
\pgfsys@defobject{currentmarker}{\pgfqpoint{-0.048611in}{0.000000in}}{\pgfqpoint{0.000000in}{0.000000in}}{%
\pgfpathmoveto{\pgfqpoint{0.000000in}{0.000000in}}%
\pgfpathlineto{\pgfqpoint{-0.048611in}{0.000000in}}%
\pgfusepath{stroke,fill}%
}%
\begin{pgfscope}%
\pgfsys@transformshift{0.629421in}{3.401163in}%
\pgfsys@useobject{currentmarker}{}%
\end{pgfscope}%
\end{pgfscope}%
\begin{pgfscope}%
\definecolor{textcolor}{rgb}{0.000000,0.000000,0.000000}%
\pgfsetstrokecolor{textcolor}%
\pgfsetfillcolor{textcolor}%
\pgftext[x=0.261828in,y=3.343125in,left,base]{\color{textcolor}\sffamily\fontsize{11.000000}{13.200000}\selectfont \(\displaystyle 0.18\)}%
\end{pgfscope}%
\begin{pgfscope}%
\definecolor{textcolor}{rgb}{0.000000,0.000000,0.000000}%
\pgfsetstrokecolor{textcolor}%
\pgfsetfillcolor{textcolor}%
\pgftext[x=0.206273in,y=2.208889in,,bottom]{\color{textcolor}\sffamily\fontsize{11.000000}{13.200000}\selectfont \(\displaystyle C\)}%
\end{pgfscope}%
\begin{pgfscope}%
\pgfpathrectangle{\pgfqpoint{0.629421in}{0.617778in}}{\pgfqpoint{4.164264in}{3.182222in}}%
\pgfusepath{clip}%
\pgfsetbuttcap%
\pgfsetroundjoin%
\definecolor{currentfill}{rgb}{0.000000,0.000000,0.000000}%
\pgfsetfillcolor{currentfill}%
\pgfsetlinewidth{1.003750pt}%
\definecolor{currentstroke}{rgb}{0.000000,0.000000,0.000000}%
\pgfsetstrokecolor{currentstroke}%
\pgfsetdash{}{0pt}%
\pgfsys@defobject{currentmarker}{\pgfqpoint{-0.041667in}{-0.041667in}}{\pgfqpoint{0.041667in}{0.041667in}}{%
\pgfpathmoveto{\pgfqpoint{0.000000in}{-0.041667in}}%
\pgfpathcurveto{\pgfqpoint{0.011050in}{-0.041667in}}{\pgfqpoint{0.021649in}{-0.037276in}}{\pgfqpoint{0.029463in}{-0.029463in}}%
\pgfpathcurveto{\pgfqpoint{0.037276in}{-0.021649in}}{\pgfqpoint{0.041667in}{-0.011050in}}{\pgfqpoint{0.041667in}{0.000000in}}%
\pgfpathcurveto{\pgfqpoint{0.041667in}{0.011050in}}{\pgfqpoint{0.037276in}{0.021649in}}{\pgfqpoint{0.029463in}{0.029463in}}%
\pgfpathcurveto{\pgfqpoint{0.021649in}{0.037276in}}{\pgfqpoint{0.011050in}{0.041667in}}{\pgfqpoint{0.000000in}{0.041667in}}%
\pgfpathcurveto{\pgfqpoint{-0.011050in}{0.041667in}}{\pgfqpoint{-0.021649in}{0.037276in}}{\pgfqpoint{-0.029463in}{0.029463in}}%
\pgfpathcurveto{\pgfqpoint{-0.037276in}{0.021649in}}{\pgfqpoint{-0.041667in}{0.011050in}}{\pgfqpoint{-0.041667in}{0.000000in}}%
\pgfpathcurveto{\pgfqpoint{-0.041667in}{-0.011050in}}{\pgfqpoint{-0.037276in}{-0.021649in}}{\pgfqpoint{-0.029463in}{-0.029463in}}%
\pgfpathcurveto{\pgfqpoint{-0.021649in}{-0.037276in}}{\pgfqpoint{-0.011050in}{-0.041667in}}{\pgfqpoint{0.000000in}{-0.041667in}}%
\pgfpathclose%
\pgfusepath{stroke,fill}%
}%
\begin{pgfscope}%
\pgfsys@transformshift{0.818706in}{0.762424in}%
\pgfsys@useobject{currentmarker}{}%
\end{pgfscope}%
\begin{pgfscope}%
\pgfsys@transformshift{2.283212in}{1.979947in}%
\pgfsys@useobject{currentmarker}{}%
\end{pgfscope}%
\begin{pgfscope}%
\pgfsys@transformshift{3.139894in}{3.432012in}%
\pgfsys@useobject{currentmarker}{}%
\end{pgfscope}%
\begin{pgfscope}%
\pgfsys@transformshift{3.747719in}{2.830476in}%
\pgfsys@useobject{currentmarker}{}%
\end{pgfscope}%
\begin{pgfscope}%
\pgfsys@transformshift{4.219184in}{3.655354in}%
\pgfsys@useobject{currentmarker}{}%
\end{pgfscope}%
\begin{pgfscope}%
\pgfsys@transformshift{4.604400in}{2.294962in}%
\pgfsys@useobject{currentmarker}{}%
\end{pgfscope}%
\end{pgfscope}%
\begin{pgfscope}%
\pgfsetrectcap%
\pgfsetmiterjoin%
\pgfsetlinewidth{0.803000pt}%
\definecolor{currentstroke}{rgb}{0.000000,0.000000,0.000000}%
\pgfsetstrokecolor{currentstroke}%
\pgfsetdash{}{0pt}%
\pgfpathmoveto{\pgfqpoint{0.629421in}{0.617778in}}%
\pgfpathlineto{\pgfqpoint{0.629421in}{3.800000in}}%
\pgfusepath{stroke}%
\end{pgfscope}%
\begin{pgfscope}%
\pgfsetrectcap%
\pgfsetmiterjoin%
\pgfsetlinewidth{0.000000pt}%
\definecolor{currentstroke}{rgb}{0.000000,0.000000,0.000000}%
\pgfsetstrokecolor{currentstroke}%
\pgfsetstrokeopacity{0.000000}%
\pgfsetdash{}{0pt}%
\pgfpathmoveto{\pgfqpoint{4.793685in}{0.617778in}}%
\pgfpathlineto{\pgfqpoint{4.793685in}{3.800000in}}%
\pgfusepath{}%
\end{pgfscope}%
\begin{pgfscope}%
\pgfsetrectcap%
\pgfsetmiterjoin%
\pgfsetlinewidth{0.803000pt}%
\definecolor{currentstroke}{rgb}{0.000000,0.000000,0.000000}%
\pgfsetstrokecolor{currentstroke}%
\pgfsetdash{}{0pt}%
\pgfpathmoveto{\pgfqpoint{0.629421in}{0.617778in}}%
\pgfpathlineto{\pgfqpoint{4.793685in}{0.617778in}}%
\pgfusepath{stroke}%
\end{pgfscope}%
\begin{pgfscope}%
\pgfsetrectcap%
\pgfsetmiterjoin%
\pgfsetlinewidth{0.000000pt}%
\definecolor{currentstroke}{rgb}{0.000000,0.000000,0.000000}%
\pgfsetstrokecolor{currentstroke}%
\pgfsetstrokeopacity{0.000000}%
\pgfsetdash{}{0pt}%
\pgfpathmoveto{\pgfqpoint{0.629421in}{3.800000in}}%
\pgfpathlineto{\pgfqpoint{4.793685in}{3.800000in}}%
\pgfusepath{}%
\end{pgfscope}%
\end{pgfpicture}%
\makeatother%
\endgroup%

  \end{subfigure}
    \begin{subfigure}{\textwidth}
%% Creator: Matplotlib, PGF backend
%%
%% To include the figure in your LaTeX document, write
%%   \input{<filename>.pgf}
%%
%% Make sure the required packages are loaded in your preamble
%%   \usepackage{pgf}
%%
%% Figures using additional raster images can only be included by \input if
%% they are in the same directory as the main LaTeX file. For loading figures
%% from other directories you can use the `import` package
%%   \usepackage{import}
%% and then include the figures with
%%   \import{<path to file>}{<filename>.pgf}
%%
%% Matplotlib used the following preamble
%%   \usepackage{fontspec}
%%   \setmainfont{DejaVuSerif.ttf}[Path=/home/owen/progs/firedrake-complex/firedrake/lib/python3.5/site-packages/matplotlib/mpl-data/fonts/ttf/]
%%   \setsansfont{DejaVuSans.ttf}[Path=/home/owen/progs/firedrake-complex/firedrake/lib/python3.5/site-packages/matplotlib/mpl-data/fonts/ttf/]
%%   \setmonofont{DejaVuSansMono.ttf}[Path=/home/owen/progs/firedrake-complex/firedrake/lib/python3.5/site-packages/matplotlib/mpl-data/fonts/ttf/]
%%
\begingroup%
\makeatletter%
\begin{pgfpicture}%
\pgfpathrectangle{\pgfpointorigin}{\pgfqpoint{3.000000in}{3.000000in}}%
\pgfusepath{use as bounding box, clip}%
\begin{pgfscope}%
\pgfsetbuttcap%
\pgfsetmiterjoin%
\definecolor{currentfill}{rgb}{1.000000,1.000000,1.000000}%
\pgfsetfillcolor{currentfill}%
\pgfsetlinewidth{0.000000pt}%
\definecolor{currentstroke}{rgb}{1.000000,1.000000,1.000000}%
\pgfsetstrokecolor{currentstroke}%
\pgfsetdash{}{0pt}%
\pgfpathmoveto{\pgfqpoint{0.000000in}{0.000000in}}%
\pgfpathlineto{\pgfqpoint{3.000000in}{0.000000in}}%
\pgfpathlineto{\pgfqpoint{3.000000in}{3.000000in}}%
\pgfpathlineto{\pgfqpoint{0.000000in}{3.000000in}}%
\pgfpathclose%
\pgfusepath{fill}%
\end{pgfscope}%
\begin{pgfscope}%
\pgfsetbuttcap%
\pgfsetmiterjoin%
\definecolor{currentfill}{rgb}{1.000000,1.000000,1.000000}%
\pgfsetfillcolor{currentfill}%
\pgfsetlinewidth{0.000000pt}%
\definecolor{currentstroke}{rgb}{0.000000,0.000000,0.000000}%
\pgfsetstrokecolor{currentstroke}%
\pgfsetstrokeopacity{0.000000}%
\pgfsetdash{}{0pt}%
\pgfpathmoveto{\pgfqpoint{0.375000in}{0.330000in}}%
\pgfpathlineto{\pgfqpoint{2.700000in}{0.330000in}}%
\pgfpathlineto{\pgfqpoint{2.700000in}{2.640000in}}%
\pgfpathlineto{\pgfqpoint{0.375000in}{2.640000in}}%
\pgfpathclose%
\pgfusepath{fill}%
\end{pgfscope}%
\begin{pgfscope}%
\pgfsetbuttcap%
\pgfsetroundjoin%
\definecolor{currentfill}{rgb}{0.000000,0.000000,0.000000}%
\pgfsetfillcolor{currentfill}%
\pgfsetlinewidth{0.803000pt}%
\definecolor{currentstroke}{rgb}{0.000000,0.000000,0.000000}%
\pgfsetstrokecolor{currentstroke}%
\pgfsetdash{}{0pt}%
\pgfsys@defobject{currentmarker}{\pgfqpoint{0.000000in}{-0.048611in}}{\pgfqpoint{0.000000in}{0.000000in}}{%
\pgfpathmoveto{\pgfqpoint{0.000000in}{0.000000in}}%
\pgfpathlineto{\pgfqpoint{0.000000in}{-0.048611in}}%
\pgfusepath{stroke,fill}%
}%
\begin{pgfscope}%
\pgfsys@transformshift{0.480682in}{0.330000in}%
\pgfsys@useobject{currentmarker}{}%
\end{pgfscope}%
\end{pgfscope}%
\begin{pgfscope}%
\definecolor{textcolor}{rgb}{0.000000,0.000000,0.000000}%
\pgfsetstrokecolor{textcolor}%
\pgfsetfillcolor{textcolor}%
\pgftext[x=0.480682in,y=0.232778in,,top]{\color{textcolor}\sffamily\fontsize{10.000000}{12.000000}\selectfont \(\displaystyle {10^{1}}\)}%
\end{pgfscope}%
\begin{pgfscope}%
\pgfsetbuttcap%
\pgfsetroundjoin%
\definecolor{currentfill}{rgb}{0.000000,0.000000,0.000000}%
\pgfsetfillcolor{currentfill}%
\pgfsetlinewidth{0.602250pt}%
\definecolor{currentstroke}{rgb}{0.000000,0.000000,0.000000}%
\pgfsetstrokecolor{currentstroke}%
\pgfsetdash{}{0pt}%
\pgfsys@defobject{currentmarker}{\pgfqpoint{0.000000in}{-0.027778in}}{\pgfqpoint{0.000000in}{0.000000in}}{%
\pgfpathmoveto{\pgfqpoint{0.000000in}{0.000000in}}%
\pgfpathlineto{\pgfqpoint{0.000000in}{-0.027778in}}%
\pgfusepath{stroke,fill}%
}%
\begin{pgfscope}%
\pgfsys@transformshift{1.298348in}{0.330000in}%
\pgfsys@useobject{currentmarker}{}%
\end{pgfscope}%
\end{pgfscope}%
\begin{pgfscope}%
\definecolor{textcolor}{rgb}{0.000000,0.000000,0.000000}%
\pgfsetstrokecolor{textcolor}%
\pgfsetfillcolor{textcolor}%
\pgftext[x=1.298348in,y=0.255000in,,top]{\color{textcolor}\sffamily\fontsize{10.000000}{12.000000}\selectfont \(\displaystyle {2\times10^{1}}\)}%
\end{pgfscope}%
\begin{pgfscope}%
\pgfsetbuttcap%
\pgfsetroundjoin%
\definecolor{currentfill}{rgb}{0.000000,0.000000,0.000000}%
\pgfsetfillcolor{currentfill}%
\pgfsetlinewidth{0.602250pt}%
\definecolor{currentstroke}{rgb}{0.000000,0.000000,0.000000}%
\pgfsetstrokecolor{currentstroke}%
\pgfsetdash{}{0pt}%
\pgfsys@defobject{currentmarker}{\pgfqpoint{0.000000in}{-0.027778in}}{\pgfqpoint{0.000000in}{0.000000in}}{%
\pgfpathmoveto{\pgfqpoint{0.000000in}{0.000000in}}%
\pgfpathlineto{\pgfqpoint{0.000000in}{-0.027778in}}%
\pgfusepath{stroke,fill}%
}%
\begin{pgfscope}%
\pgfsys@transformshift{1.776652in}{0.330000in}%
\pgfsys@useobject{currentmarker}{}%
\end{pgfscope}%
\end{pgfscope}%
\begin{pgfscope}%
\definecolor{textcolor}{rgb}{0.000000,0.000000,0.000000}%
\pgfsetstrokecolor{textcolor}%
\pgfsetfillcolor{textcolor}%
\pgftext[x=1.776652in,y=0.255000in,,top]{\color{textcolor}\sffamily\fontsize{10.000000}{12.000000}\selectfont \(\displaystyle {3\times10^{1}}\)}%
\end{pgfscope}%
\begin{pgfscope}%
\pgfsetbuttcap%
\pgfsetroundjoin%
\definecolor{currentfill}{rgb}{0.000000,0.000000,0.000000}%
\pgfsetfillcolor{currentfill}%
\pgfsetlinewidth{0.602250pt}%
\definecolor{currentstroke}{rgb}{0.000000,0.000000,0.000000}%
\pgfsetstrokecolor{currentstroke}%
\pgfsetdash{}{0pt}%
\pgfsys@defobject{currentmarker}{\pgfqpoint{0.000000in}{-0.027778in}}{\pgfqpoint{0.000000in}{0.000000in}}{%
\pgfpathmoveto{\pgfqpoint{0.000000in}{0.000000in}}%
\pgfpathlineto{\pgfqpoint{0.000000in}{-0.027778in}}%
\pgfusepath{stroke,fill}%
}%
\begin{pgfscope}%
\pgfsys@transformshift{2.116014in}{0.330000in}%
\pgfsys@useobject{currentmarker}{}%
\end{pgfscope}%
\end{pgfscope}%
\begin{pgfscope}%
\definecolor{textcolor}{rgb}{0.000000,0.000000,0.000000}%
\pgfsetstrokecolor{textcolor}%
\pgfsetfillcolor{textcolor}%
\pgftext[x=2.116014in,y=0.255000in,,top]{\color{textcolor}\sffamily\fontsize{10.000000}{12.000000}\selectfont \(\displaystyle {4\times10^{1}}\)}%
\end{pgfscope}%
\begin{pgfscope}%
\pgfsetbuttcap%
\pgfsetroundjoin%
\definecolor{currentfill}{rgb}{0.000000,0.000000,0.000000}%
\pgfsetfillcolor{currentfill}%
\pgfsetlinewidth{0.602250pt}%
\definecolor{currentstroke}{rgb}{0.000000,0.000000,0.000000}%
\pgfsetstrokecolor{currentstroke}%
\pgfsetdash{}{0pt}%
\pgfsys@defobject{currentmarker}{\pgfqpoint{0.000000in}{-0.027778in}}{\pgfqpoint{0.000000in}{0.000000in}}{%
\pgfpathmoveto{\pgfqpoint{0.000000in}{0.000000in}}%
\pgfpathlineto{\pgfqpoint{0.000000in}{-0.027778in}}%
\pgfusepath{stroke,fill}%
}%
\begin{pgfscope}%
\pgfsys@transformshift{2.379244in}{0.330000in}%
\pgfsys@useobject{currentmarker}{}%
\end{pgfscope}%
\end{pgfscope}%
\begin{pgfscope}%
\pgfsetbuttcap%
\pgfsetroundjoin%
\definecolor{currentfill}{rgb}{0.000000,0.000000,0.000000}%
\pgfsetfillcolor{currentfill}%
\pgfsetlinewidth{0.602250pt}%
\definecolor{currentstroke}{rgb}{0.000000,0.000000,0.000000}%
\pgfsetstrokecolor{currentstroke}%
\pgfsetdash{}{0pt}%
\pgfsys@defobject{currentmarker}{\pgfqpoint{0.000000in}{-0.027778in}}{\pgfqpoint{0.000000in}{0.000000in}}{%
\pgfpathmoveto{\pgfqpoint{0.000000in}{0.000000in}}%
\pgfpathlineto{\pgfqpoint{0.000000in}{-0.027778in}}%
\pgfusepath{stroke,fill}%
}%
\begin{pgfscope}%
\pgfsys@transformshift{2.594318in}{0.330000in}%
\pgfsys@useobject{currentmarker}{}%
\end{pgfscope}%
\end{pgfscope}%
\begin{pgfscope}%
\definecolor{textcolor}{rgb}{0.000000,0.000000,0.000000}%
\pgfsetstrokecolor{textcolor}%
\pgfsetfillcolor{textcolor}%
\pgftext[x=2.594318in,y=0.255000in,,top]{\color{textcolor}\sffamily\fontsize{10.000000}{12.000000}\selectfont \(\displaystyle {6\times10^{1}}\)}%
\end{pgfscope}%
\begin{pgfscope}%
\definecolor{textcolor}{rgb}{0.000000,0.000000,0.000000}%
\pgfsetstrokecolor{textcolor}%
\pgfsetfillcolor{textcolor}%
\pgftext[x=1.537500in,y=0.042809in,,top]{\color{textcolor}\sffamily\fontsize{10.000000}{12.000000}\selectfont \(\displaystyle k\)}%
\end{pgfscope}%
\begin{pgfscope}%
\pgfsetbuttcap%
\pgfsetroundjoin%
\definecolor{currentfill}{rgb}{0.000000,0.000000,0.000000}%
\pgfsetfillcolor{currentfill}%
\pgfsetlinewidth{0.803000pt}%
\definecolor{currentstroke}{rgb}{0.000000,0.000000,0.000000}%
\pgfsetstrokecolor{currentstroke}%
\pgfsetdash{}{0pt}%
\pgfsys@defobject{currentmarker}{\pgfqpoint{-0.048611in}{0.000000in}}{\pgfqpoint{0.000000in}{0.000000in}}{%
\pgfpathmoveto{\pgfqpoint{0.000000in}{0.000000in}}%
\pgfpathlineto{\pgfqpoint{-0.048611in}{0.000000in}}%
\pgfusepath{stroke,fill}%
}%
\begin{pgfscope}%
\pgfsys@transformshift{0.375000in}{0.385906in}%
\pgfsys@useobject{currentmarker}{}%
\end{pgfscope}%
\end{pgfscope}%
\begin{pgfscope}%
\definecolor{textcolor}{rgb}{0.000000,0.000000,0.000000}%
\pgfsetstrokecolor{textcolor}%
\pgfsetfillcolor{textcolor}%
\pgftext[x=-0.031467in,y=0.333145in,left,base]{\color{textcolor}\sffamily\fontsize{10.000000}{12.000000}\selectfont 0.65}%
\end{pgfscope}%
\begin{pgfscope}%
\pgfsetbuttcap%
\pgfsetroundjoin%
\definecolor{currentfill}{rgb}{0.000000,0.000000,0.000000}%
\pgfsetfillcolor{currentfill}%
\pgfsetlinewidth{0.803000pt}%
\definecolor{currentstroke}{rgb}{0.000000,0.000000,0.000000}%
\pgfsetstrokecolor{currentstroke}%
\pgfsetdash{}{0pt}%
\pgfsys@defobject{currentmarker}{\pgfqpoint{-0.048611in}{0.000000in}}{\pgfqpoint{0.000000in}{0.000000in}}{%
\pgfpathmoveto{\pgfqpoint{0.000000in}{0.000000in}}%
\pgfpathlineto{\pgfqpoint{-0.048611in}{0.000000in}}%
\pgfusepath{stroke,fill}%
}%
\begin{pgfscope}%
\pgfsys@transformshift{0.375000in}{0.675658in}%
\pgfsys@useobject{currentmarker}{}%
\end{pgfscope}%
\end{pgfscope}%
\begin{pgfscope}%
\definecolor{textcolor}{rgb}{0.000000,0.000000,0.000000}%
\pgfsetstrokecolor{textcolor}%
\pgfsetfillcolor{textcolor}%
\pgftext[x=-0.031467in,y=0.622897in,left,base]{\color{textcolor}\sffamily\fontsize{10.000000}{12.000000}\selectfont 0.70}%
\end{pgfscope}%
\begin{pgfscope}%
\pgfsetbuttcap%
\pgfsetroundjoin%
\definecolor{currentfill}{rgb}{0.000000,0.000000,0.000000}%
\pgfsetfillcolor{currentfill}%
\pgfsetlinewidth{0.803000pt}%
\definecolor{currentstroke}{rgb}{0.000000,0.000000,0.000000}%
\pgfsetstrokecolor{currentstroke}%
\pgfsetdash{}{0pt}%
\pgfsys@defobject{currentmarker}{\pgfqpoint{-0.048611in}{0.000000in}}{\pgfqpoint{0.000000in}{0.000000in}}{%
\pgfpathmoveto{\pgfqpoint{0.000000in}{0.000000in}}%
\pgfpathlineto{\pgfqpoint{-0.048611in}{0.000000in}}%
\pgfusepath{stroke,fill}%
}%
\begin{pgfscope}%
\pgfsys@transformshift{0.375000in}{0.965410in}%
\pgfsys@useobject{currentmarker}{}%
\end{pgfscope}%
\end{pgfscope}%
\begin{pgfscope}%
\definecolor{textcolor}{rgb}{0.000000,0.000000,0.000000}%
\pgfsetstrokecolor{textcolor}%
\pgfsetfillcolor{textcolor}%
\pgftext[x=-0.031467in,y=0.912648in,left,base]{\color{textcolor}\sffamily\fontsize{10.000000}{12.000000}\selectfont 0.75}%
\end{pgfscope}%
\begin{pgfscope}%
\pgfsetbuttcap%
\pgfsetroundjoin%
\definecolor{currentfill}{rgb}{0.000000,0.000000,0.000000}%
\pgfsetfillcolor{currentfill}%
\pgfsetlinewidth{0.803000pt}%
\definecolor{currentstroke}{rgb}{0.000000,0.000000,0.000000}%
\pgfsetstrokecolor{currentstroke}%
\pgfsetdash{}{0pt}%
\pgfsys@defobject{currentmarker}{\pgfqpoint{-0.048611in}{0.000000in}}{\pgfqpoint{0.000000in}{0.000000in}}{%
\pgfpathmoveto{\pgfqpoint{0.000000in}{0.000000in}}%
\pgfpathlineto{\pgfqpoint{-0.048611in}{0.000000in}}%
\pgfusepath{stroke,fill}%
}%
\begin{pgfscope}%
\pgfsys@transformshift{0.375000in}{1.255162in}%
\pgfsys@useobject{currentmarker}{}%
\end{pgfscope}%
\end{pgfscope}%
\begin{pgfscope}%
\definecolor{textcolor}{rgb}{0.000000,0.000000,0.000000}%
\pgfsetstrokecolor{textcolor}%
\pgfsetfillcolor{textcolor}%
\pgftext[x=-0.031467in,y=1.202400in,left,base]{\color{textcolor}\sffamily\fontsize{10.000000}{12.000000}\selectfont 0.80}%
\end{pgfscope}%
\begin{pgfscope}%
\pgfsetbuttcap%
\pgfsetroundjoin%
\definecolor{currentfill}{rgb}{0.000000,0.000000,0.000000}%
\pgfsetfillcolor{currentfill}%
\pgfsetlinewidth{0.803000pt}%
\definecolor{currentstroke}{rgb}{0.000000,0.000000,0.000000}%
\pgfsetstrokecolor{currentstroke}%
\pgfsetdash{}{0pt}%
\pgfsys@defobject{currentmarker}{\pgfqpoint{-0.048611in}{0.000000in}}{\pgfqpoint{0.000000in}{0.000000in}}{%
\pgfpathmoveto{\pgfqpoint{0.000000in}{0.000000in}}%
\pgfpathlineto{\pgfqpoint{-0.048611in}{0.000000in}}%
\pgfusepath{stroke,fill}%
}%
\begin{pgfscope}%
\pgfsys@transformshift{0.375000in}{1.544913in}%
\pgfsys@useobject{currentmarker}{}%
\end{pgfscope}%
\end{pgfscope}%
\begin{pgfscope}%
\definecolor{textcolor}{rgb}{0.000000,0.000000,0.000000}%
\pgfsetstrokecolor{textcolor}%
\pgfsetfillcolor{textcolor}%
\pgftext[x=-0.031467in,y=1.492152in,left,base]{\color{textcolor}\sffamily\fontsize{10.000000}{12.000000}\selectfont 0.85}%
\end{pgfscope}%
\begin{pgfscope}%
\pgfsetbuttcap%
\pgfsetroundjoin%
\definecolor{currentfill}{rgb}{0.000000,0.000000,0.000000}%
\pgfsetfillcolor{currentfill}%
\pgfsetlinewidth{0.803000pt}%
\definecolor{currentstroke}{rgb}{0.000000,0.000000,0.000000}%
\pgfsetstrokecolor{currentstroke}%
\pgfsetdash{}{0pt}%
\pgfsys@defobject{currentmarker}{\pgfqpoint{-0.048611in}{0.000000in}}{\pgfqpoint{0.000000in}{0.000000in}}{%
\pgfpathmoveto{\pgfqpoint{0.000000in}{0.000000in}}%
\pgfpathlineto{\pgfqpoint{-0.048611in}{0.000000in}}%
\pgfusepath{stroke,fill}%
}%
\begin{pgfscope}%
\pgfsys@transformshift{0.375000in}{1.834665in}%
\pgfsys@useobject{currentmarker}{}%
\end{pgfscope}%
\end{pgfscope}%
\begin{pgfscope}%
\definecolor{textcolor}{rgb}{0.000000,0.000000,0.000000}%
\pgfsetstrokecolor{textcolor}%
\pgfsetfillcolor{textcolor}%
\pgftext[x=-0.031467in,y=1.781904in,left,base]{\color{textcolor}\sffamily\fontsize{10.000000}{12.000000}\selectfont 0.90}%
\end{pgfscope}%
\begin{pgfscope}%
\pgfsetbuttcap%
\pgfsetroundjoin%
\definecolor{currentfill}{rgb}{0.000000,0.000000,0.000000}%
\pgfsetfillcolor{currentfill}%
\pgfsetlinewidth{0.803000pt}%
\definecolor{currentstroke}{rgb}{0.000000,0.000000,0.000000}%
\pgfsetstrokecolor{currentstroke}%
\pgfsetdash{}{0pt}%
\pgfsys@defobject{currentmarker}{\pgfqpoint{-0.048611in}{0.000000in}}{\pgfqpoint{0.000000in}{0.000000in}}{%
\pgfpathmoveto{\pgfqpoint{0.000000in}{0.000000in}}%
\pgfpathlineto{\pgfqpoint{-0.048611in}{0.000000in}}%
\pgfusepath{stroke,fill}%
}%
\begin{pgfscope}%
\pgfsys@transformshift{0.375000in}{2.124417in}%
\pgfsys@useobject{currentmarker}{}%
\end{pgfscope}%
\end{pgfscope}%
\begin{pgfscope}%
\definecolor{textcolor}{rgb}{0.000000,0.000000,0.000000}%
\pgfsetstrokecolor{textcolor}%
\pgfsetfillcolor{textcolor}%
\pgftext[x=-0.031467in,y=2.071655in,left,base]{\color{textcolor}\sffamily\fontsize{10.000000}{12.000000}\selectfont 0.95}%
\end{pgfscope}%
\begin{pgfscope}%
\pgfsetbuttcap%
\pgfsetroundjoin%
\definecolor{currentfill}{rgb}{0.000000,0.000000,0.000000}%
\pgfsetfillcolor{currentfill}%
\pgfsetlinewidth{0.803000pt}%
\definecolor{currentstroke}{rgb}{0.000000,0.000000,0.000000}%
\pgfsetstrokecolor{currentstroke}%
\pgfsetdash{}{0pt}%
\pgfsys@defobject{currentmarker}{\pgfqpoint{-0.048611in}{0.000000in}}{\pgfqpoint{0.000000in}{0.000000in}}{%
\pgfpathmoveto{\pgfqpoint{0.000000in}{0.000000in}}%
\pgfpathlineto{\pgfqpoint{-0.048611in}{0.000000in}}%
\pgfusepath{stroke,fill}%
}%
\begin{pgfscope}%
\pgfsys@transformshift{0.375000in}{2.414169in}%
\pgfsys@useobject{currentmarker}{}%
\end{pgfscope}%
\end{pgfscope}%
\begin{pgfscope}%
\definecolor{textcolor}{rgb}{0.000000,0.000000,0.000000}%
\pgfsetstrokecolor{textcolor}%
\pgfsetfillcolor{textcolor}%
\pgftext[x=-0.031467in,y=2.361407in,left,base]{\color{textcolor}\sffamily\fontsize{10.000000}{12.000000}\selectfont 1.00}%
\end{pgfscope}%
\begin{pgfscope}%
\definecolor{textcolor}{rgb}{0.000000,0.000000,0.000000}%
\pgfsetstrokecolor{textcolor}%
\pgfsetfillcolor{textcolor}%
\pgftext[x=-0.087023in,y=1.485000in,,bottom,rotate=90.000000]{\color{textcolor}\sffamily\fontsize{10.000000}{12.000000}\selectfont \(\displaystyle \alpha\)}%
\end{pgfscope}%
\begin{pgfscope}%
\pgfpathrectangle{\pgfqpoint{0.375000in}{0.330000in}}{\pgfqpoint{2.325000in}{2.310000in}}%
\pgfusepath{clip}%
\pgfsetbuttcap%
\pgfsetroundjoin%
\pgfsetlinewidth{1.505625pt}%
\definecolor{currentstroke}{rgb}{0.000000,0.000000,0.000000}%
\pgfsetstrokecolor{currentstroke}%
\pgfsetdash{{5.550000pt}{2.400000pt}}{0.000000pt}%
\pgfpathmoveto{\pgfqpoint{0.480682in}{2.535000in}}%
\pgfpathlineto{\pgfqpoint{1.298348in}{1.813714in}}%
\pgfpathlineto{\pgfqpoint{1.776652in}{1.391789in}}%
\pgfpathlineto{\pgfqpoint{2.116014in}{1.092429in}}%
\pgfpathlineto{\pgfqpoint{2.379244in}{0.860227in}}%
\pgfpathlineto{\pgfqpoint{2.594318in}{0.670504in}}%
\pgfusepath{stroke}%
\end{pgfscope}%
\begin{pgfscope}%
\pgfpathrectangle{\pgfqpoint{0.375000in}{0.330000in}}{\pgfqpoint{2.325000in}{2.310000in}}%
\pgfusepath{clip}%
\pgfsetbuttcap%
\pgfsetroundjoin%
\definecolor{currentfill}{rgb}{0.000000,0.000000,0.000000}%
\pgfsetfillcolor{currentfill}%
\pgfsetlinewidth{1.003750pt}%
\definecolor{currentstroke}{rgb}{0.000000,0.000000,0.000000}%
\pgfsetstrokecolor{currentstroke}%
\pgfsetdash{}{0pt}%
\pgfsys@defobject{currentmarker}{\pgfqpoint{-0.041667in}{-0.041667in}}{\pgfqpoint{0.041667in}{0.041667in}}{%
\pgfpathmoveto{\pgfqpoint{0.000000in}{-0.041667in}}%
\pgfpathcurveto{\pgfqpoint{0.011050in}{-0.041667in}}{\pgfqpoint{0.021649in}{-0.037276in}}{\pgfqpoint{0.029463in}{-0.029463in}}%
\pgfpathcurveto{\pgfqpoint{0.037276in}{-0.021649in}}{\pgfqpoint{0.041667in}{-0.011050in}}{\pgfqpoint{0.041667in}{0.000000in}}%
\pgfpathcurveto{\pgfqpoint{0.041667in}{0.011050in}}{\pgfqpoint{0.037276in}{0.021649in}}{\pgfqpoint{0.029463in}{0.029463in}}%
\pgfpathcurveto{\pgfqpoint{0.021649in}{0.037276in}}{\pgfqpoint{0.011050in}{0.041667in}}{\pgfqpoint{0.000000in}{0.041667in}}%
\pgfpathcurveto{\pgfqpoint{-0.011050in}{0.041667in}}{\pgfqpoint{-0.021649in}{0.037276in}}{\pgfqpoint{-0.029463in}{0.029463in}}%
\pgfpathcurveto{\pgfqpoint{-0.037276in}{0.021649in}}{\pgfqpoint{-0.041667in}{0.011050in}}{\pgfqpoint{-0.041667in}{0.000000in}}%
\pgfpathcurveto{\pgfqpoint{-0.041667in}{-0.011050in}}{\pgfqpoint{-0.037276in}{-0.021649in}}{\pgfqpoint{-0.029463in}{-0.029463in}}%
\pgfpathcurveto{\pgfqpoint{-0.021649in}{-0.037276in}}{\pgfqpoint{-0.011050in}{-0.041667in}}{\pgfqpoint{0.000000in}{-0.041667in}}%
\pgfpathclose%
\pgfusepath{stroke,fill}%
}%
\begin{pgfscope}%
\pgfsys@transformshift{0.480682in}{2.496116in}%
\pgfsys@useobject{currentmarker}{}%
\end{pgfscope}%
\begin{pgfscope}%
\pgfsys@transformshift{1.298348in}{1.765386in}%
\pgfsys@useobject{currentmarker}{}%
\end{pgfscope}%
\begin{pgfscope}%
\pgfsys@transformshift{1.776652in}{1.612099in}%
\pgfsys@useobject{currentmarker}{}%
\end{pgfscope}%
\begin{pgfscope}%
\pgfsys@transformshift{2.116014in}{0.874574in}%
\pgfsys@useobject{currentmarker}{}%
\end{pgfscope}%
\begin{pgfscope}%
\pgfsys@transformshift{2.379244in}{1.180488in}%
\pgfsys@useobject{currentmarker}{}%
\end{pgfscope}%
\begin{pgfscope}%
\pgfsys@transformshift{2.594318in}{0.435000in}%
\pgfsys@useobject{currentmarker}{}%
\end{pgfscope}%
\end{pgfscope}%
\begin{pgfscope}%
\pgfsetrectcap%
\pgfsetmiterjoin%
\pgfsetlinewidth{0.803000pt}%
\definecolor{currentstroke}{rgb}{0.000000,0.000000,0.000000}%
\pgfsetstrokecolor{currentstroke}%
\pgfsetdash{}{0pt}%
\pgfpathmoveto{\pgfqpoint{0.375000in}{0.330000in}}%
\pgfpathlineto{\pgfqpoint{0.375000in}{2.640000in}}%
\pgfusepath{stroke}%
\end{pgfscope}%
\begin{pgfscope}%
\pgfsetrectcap%
\pgfsetmiterjoin%
\pgfsetlinewidth{0.803000pt}%
\definecolor{currentstroke}{rgb}{0.000000,0.000000,0.000000}%
\pgfsetstrokecolor{currentstroke}%
\pgfsetdash{}{0pt}%
\pgfpathmoveto{\pgfqpoint{2.700000in}{0.330000in}}%
\pgfpathlineto{\pgfqpoint{2.700000in}{2.640000in}}%
\pgfusepath{stroke}%
\end{pgfscope}%
\begin{pgfscope}%
\pgfsetrectcap%
\pgfsetmiterjoin%
\pgfsetlinewidth{0.803000pt}%
\definecolor{currentstroke}{rgb}{0.000000,0.000000,0.000000}%
\pgfsetstrokecolor{currentstroke}%
\pgfsetdash{}{0pt}%
\pgfpathmoveto{\pgfqpoint{0.375000in}{0.330000in}}%
\pgfpathlineto{\pgfqpoint{2.700000in}{0.330000in}}%
\pgfusepath{stroke}%
\end{pgfscope}%
\begin{pgfscope}%
\pgfsetrectcap%
\pgfsetmiterjoin%
\pgfsetlinewidth{0.803000pt}%
\definecolor{currentstroke}{rgb}{0.000000,0.000000,0.000000}%
\pgfsetstrokecolor{currentstroke}%
\pgfsetdash{}{0pt}%
\pgfpathmoveto{\pgfqpoint{0.375000in}{2.640000in}}%
\pgfpathlineto{\pgfqpoint{2.700000in}{2.640000in}}%
\pgfusepath{stroke}%
\end{pgfscope}%
\begin{pgfscope}%
\pgfsetbuttcap%
\pgfsetmiterjoin%
\definecolor{currentfill}{rgb}{1.000000,1.000000,1.000000}%
\pgfsetfillcolor{currentfill}%
\pgfsetfillopacity{0.800000}%
\pgfsetlinewidth{1.003750pt}%
\definecolor{currentstroke}{rgb}{0.800000,0.800000,0.800000}%
\pgfsetstrokecolor{currentstroke}%
\pgfsetstrokeopacity{0.800000}%
\pgfsetdash{}{0pt}%
\pgfpathmoveto{\pgfqpoint{0.472222in}{0.399444in}}%
\pgfpathlineto{\pgfqpoint{2.557368in}{0.399444in}}%
\pgfpathquadraticcurveto{\pgfqpoint{2.585146in}{0.399444in}}{\pgfqpoint{2.585146in}{0.427222in}}%
\pgfpathlineto{\pgfqpoint{2.585146in}{0.623023in}}%
\pgfpathquadraticcurveto{\pgfqpoint{2.585146in}{0.650801in}}{\pgfqpoint{2.557368in}{0.650801in}}%
\pgfpathlineto{\pgfqpoint{0.472222in}{0.650801in}}%
\pgfpathquadraticcurveto{\pgfqpoint{0.444444in}{0.650801in}}{\pgfqpoint{0.444444in}{0.623023in}}%
\pgfpathlineto{\pgfqpoint{0.444444in}{0.427222in}}%
\pgfpathquadraticcurveto{\pgfqpoint{0.444444in}{0.399444in}}{\pgfqpoint{0.472222in}{0.399444in}}%
\pgfpathclose%
\pgfusepath{stroke,fill}%
\end{pgfscope}%
\begin{pgfscope}%
\pgfsetbuttcap%
\pgfsetroundjoin%
\pgfsetlinewidth{1.505625pt}%
\definecolor{currentstroke}{rgb}{0.000000,0.000000,0.000000}%
\pgfsetstrokecolor{currentstroke}%
\pgfsetdash{{5.550000pt}{2.400000pt}}{0.000000pt}%
\pgfpathmoveto{\pgfqpoint{0.500000in}{0.538333in}}%
\pgfpathlineto{\pgfqpoint{0.777778in}{0.538333in}}%
\pgfusepath{stroke}%
\end{pgfscope}%
\begin{pgfscope}%
\definecolor{textcolor}{rgb}{0.000000,0.000000,0.000000}%
\pgfsetstrokecolor{textcolor}%
\pgfsetfillcolor{textcolor}%
\pgftext[x=0.888889in,y=0.489722in,left,base]{\color{textcolor}\sffamily\fontsize{10.000000}{12.000000}\selectfont \(\displaystyle \alpha = 1.4343 - 0.4134\log(k)\)}%
\end{pgfscope}%
\end{pgfpicture}%
\makeatother%
\endgroup%

    \end{subfigure}
\caption{Plots of the dependence of $C$ and $\alpha$ on $k$ in \cref{eq:qmcerrorform} for $Q(u) = u((1,1))$. Observe the $x$-axis is on a $\log_{10}$ scale, but $\log$ is the natural logarithm.  \label{fig:toprightCalpha}}
\end{figure}

\begin{figure}[h]
    \centering
  \begin{subfigure}{\textwidth}
%% Creator: Matplotlib, PGF backend
%%
%% To include the figure in your LaTeX document, write
%%   \input{<filename>.pgf}
%%
%% Make sure the required packages are loaded in your preamble
%%   \usepackage{pgf}
%%
%% Figures using additional raster images can only be included by \input if
%% they are in the same directory as the main LaTeX file. For loading figures
%% from other directories you can use the `import` package
%%   \usepackage{import}
%% and then include the figures with
%%   \import{<path to file>}{<filename>.pgf}
%%
%% Matplotlib used the following preamble
%%   \usepackage{fontspec}
%%   \setmainfont{DejaVuSerif.ttf}[Path=/home/owen/progs/firedrake-complex/firedrake/lib/python3.5/site-packages/matplotlib/mpl-data/fonts/ttf/]
%%   \setsansfont{DejaVuSans.ttf}[Path=/home/owen/progs/firedrake-complex/firedrake/lib/python3.5/site-packages/matplotlib/mpl-data/fonts/ttf/]
%%   \setmonofont{DejaVuSansMono.ttf}[Path=/home/owen/progs/firedrake-complex/firedrake/lib/python3.5/site-packages/matplotlib/mpl-data/fonts/ttf/]
%%
\begingroup%
\makeatletter%
\begin{pgfpicture}%
\pgfpathrectangle{\pgfpointorigin}{\pgfqpoint{5.000000in}{4.000000in}}%
\pgfusepath{use as bounding box, clip}%
\begin{pgfscope}%
\pgfsetbuttcap%
\pgfsetmiterjoin%
\definecolor{currentfill}{rgb}{1.000000,1.000000,1.000000}%
\pgfsetfillcolor{currentfill}%
\pgfsetlinewidth{0.000000pt}%
\definecolor{currentstroke}{rgb}{1.000000,1.000000,1.000000}%
\pgfsetstrokecolor{currentstroke}%
\pgfsetdash{}{0pt}%
\pgfpathmoveto{\pgfqpoint{0.000000in}{0.000000in}}%
\pgfpathlineto{\pgfqpoint{5.000000in}{0.000000in}}%
\pgfpathlineto{\pgfqpoint{5.000000in}{4.000000in}}%
\pgfpathlineto{\pgfqpoint{0.000000in}{4.000000in}}%
\pgfpathclose%
\pgfusepath{fill}%
\end{pgfscope}%
\begin{pgfscope}%
\pgfsetbuttcap%
\pgfsetmiterjoin%
\definecolor{currentfill}{rgb}{1.000000,1.000000,1.000000}%
\pgfsetfillcolor{currentfill}%
\pgfsetlinewidth{0.000000pt}%
\definecolor{currentstroke}{rgb}{0.000000,0.000000,0.000000}%
\pgfsetstrokecolor{currentstroke}%
\pgfsetstrokeopacity{0.000000}%
\pgfsetdash{}{0pt}%
\pgfpathmoveto{\pgfqpoint{0.511963in}{0.582778in}}%
\pgfpathlineto{\pgfqpoint{4.810222in}{0.582778in}}%
\pgfpathlineto{\pgfqpoint{4.810222in}{3.815000in}}%
\pgfpathlineto{\pgfqpoint{0.511963in}{3.815000in}}%
\pgfpathclose%
\pgfusepath{fill}%
\end{pgfscope}%
\begin{pgfscope}%
\pgfsetbuttcap%
\pgfsetroundjoin%
\definecolor{currentfill}{rgb}{0.000000,0.000000,0.000000}%
\pgfsetfillcolor{currentfill}%
\pgfsetlinewidth{0.803000pt}%
\definecolor{currentstroke}{rgb}{0.000000,0.000000,0.000000}%
\pgfsetstrokecolor{currentstroke}%
\pgfsetdash{}{0pt}%
\pgfsys@defobject{currentmarker}{\pgfqpoint{0.000000in}{-0.048611in}}{\pgfqpoint{0.000000in}{0.000000in}}{%
\pgfpathmoveto{\pgfqpoint{0.000000in}{0.000000in}}%
\pgfpathlineto{\pgfqpoint{0.000000in}{-0.048611in}}%
\pgfusepath{stroke,fill}%
}%
\begin{pgfscope}%
\pgfsys@transformshift{0.707338in}{0.582778in}%
\pgfsys@useobject{currentmarker}{}%
\end{pgfscope}%
\end{pgfscope}%
\begin{pgfscope}%
\definecolor{textcolor}{rgb}{0.000000,0.000000,0.000000}%
\pgfsetstrokecolor{textcolor}%
\pgfsetfillcolor{textcolor}%
\pgftext[x=0.707338in,y=0.485556in,,top]{\color{textcolor}\sffamily\fontsize{10.000000}{12.000000}\selectfont \(\displaystyle 10^{1}\)}%
\end{pgfscope}%
\begin{pgfscope}%
\pgfsetbuttcap%
\pgfsetroundjoin%
\definecolor{currentfill}{rgb}{0.000000,0.000000,0.000000}%
\pgfsetfillcolor{currentfill}%
\pgfsetlinewidth{0.602250pt}%
\definecolor{currentstroke}{rgb}{0.000000,0.000000,0.000000}%
\pgfsetstrokecolor{currentstroke}%
\pgfsetdash{}{0pt}%
\pgfsys@defobject{currentmarker}{\pgfqpoint{0.000000in}{-0.027778in}}{\pgfqpoint{0.000000in}{0.000000in}}{%
\pgfpathmoveto{\pgfqpoint{0.000000in}{0.000000in}}%
\pgfpathlineto{\pgfqpoint{0.000000in}{-0.027778in}}%
\pgfusepath{stroke,fill}%
}%
\begin{pgfscope}%
\pgfsys@transformshift{2.218969in}{0.582778in}%
\pgfsys@useobject{currentmarker}{}%
\end{pgfscope}%
\end{pgfscope}%
\begin{pgfscope}%
\definecolor{textcolor}{rgb}{0.000000,0.000000,0.000000}%
\pgfsetstrokecolor{textcolor}%
\pgfsetfillcolor{textcolor}%
\pgftext[x=2.218969in,y=0.507778in,,top]{\color{textcolor}\sffamily\fontsize{10.000000}{12.000000}\selectfont \(\displaystyle 2\times10^{1}\)}%
\end{pgfscope}%
\begin{pgfscope}%
\pgfsetbuttcap%
\pgfsetroundjoin%
\definecolor{currentfill}{rgb}{0.000000,0.000000,0.000000}%
\pgfsetfillcolor{currentfill}%
\pgfsetlinewidth{0.602250pt}%
\definecolor{currentstroke}{rgb}{0.000000,0.000000,0.000000}%
\pgfsetstrokecolor{currentstroke}%
\pgfsetdash{}{0pt}%
\pgfsys@defobject{currentmarker}{\pgfqpoint{0.000000in}{-0.027778in}}{\pgfqpoint{0.000000in}{0.000000in}}{%
\pgfpathmoveto{\pgfqpoint{0.000000in}{0.000000in}}%
\pgfpathlineto{\pgfqpoint{0.000000in}{-0.027778in}}%
\pgfusepath{stroke,fill}%
}%
\begin{pgfscope}%
\pgfsys@transformshift{3.103216in}{0.582778in}%
\pgfsys@useobject{currentmarker}{}%
\end{pgfscope}%
\end{pgfscope}%
\begin{pgfscope}%
\definecolor{textcolor}{rgb}{0.000000,0.000000,0.000000}%
\pgfsetstrokecolor{textcolor}%
\pgfsetfillcolor{textcolor}%
\pgftext[x=3.103216in,y=0.507778in,,top]{\color{textcolor}\sffamily\fontsize{10.000000}{12.000000}\selectfont \(\displaystyle 3\times10^{1}\)}%
\end{pgfscope}%
\begin{pgfscope}%
\pgfsetbuttcap%
\pgfsetroundjoin%
\definecolor{currentfill}{rgb}{0.000000,0.000000,0.000000}%
\pgfsetfillcolor{currentfill}%
\pgfsetlinewidth{0.602250pt}%
\definecolor{currentstroke}{rgb}{0.000000,0.000000,0.000000}%
\pgfsetstrokecolor{currentstroke}%
\pgfsetdash{}{0pt}%
\pgfsys@defobject{currentmarker}{\pgfqpoint{0.000000in}{-0.027778in}}{\pgfqpoint{0.000000in}{0.000000in}}{%
\pgfpathmoveto{\pgfqpoint{0.000000in}{0.000000in}}%
\pgfpathlineto{\pgfqpoint{0.000000in}{-0.027778in}}%
\pgfusepath{stroke,fill}%
}%
\begin{pgfscope}%
\pgfsys@transformshift{3.730599in}{0.582778in}%
\pgfsys@useobject{currentmarker}{}%
\end{pgfscope}%
\end{pgfscope}%
\begin{pgfscope}%
\definecolor{textcolor}{rgb}{0.000000,0.000000,0.000000}%
\pgfsetstrokecolor{textcolor}%
\pgfsetfillcolor{textcolor}%
\pgftext[x=3.730599in,y=0.507778in,,top]{\color{textcolor}\sffamily\fontsize{10.000000}{12.000000}\selectfont \(\displaystyle 4\times10^{1}\)}%
\end{pgfscope}%
\begin{pgfscope}%
\pgfsetbuttcap%
\pgfsetroundjoin%
\definecolor{currentfill}{rgb}{0.000000,0.000000,0.000000}%
\pgfsetfillcolor{currentfill}%
\pgfsetlinewidth{0.602250pt}%
\definecolor{currentstroke}{rgb}{0.000000,0.000000,0.000000}%
\pgfsetstrokecolor{currentstroke}%
\pgfsetdash{}{0pt}%
\pgfsys@defobject{currentmarker}{\pgfqpoint{0.000000in}{-0.027778in}}{\pgfqpoint{0.000000in}{0.000000in}}{%
\pgfpathmoveto{\pgfqpoint{0.000000in}{0.000000in}}%
\pgfpathlineto{\pgfqpoint{0.000000in}{-0.027778in}}%
\pgfusepath{stroke,fill}%
}%
\begin{pgfscope}%
\pgfsys@transformshift{4.217235in}{0.582778in}%
\pgfsys@useobject{currentmarker}{}%
\end{pgfscope}%
\end{pgfscope}%
\begin{pgfscope}%
\pgfsetbuttcap%
\pgfsetroundjoin%
\definecolor{currentfill}{rgb}{0.000000,0.000000,0.000000}%
\pgfsetfillcolor{currentfill}%
\pgfsetlinewidth{0.602250pt}%
\definecolor{currentstroke}{rgb}{0.000000,0.000000,0.000000}%
\pgfsetstrokecolor{currentstroke}%
\pgfsetdash{}{0pt}%
\pgfsys@defobject{currentmarker}{\pgfqpoint{0.000000in}{-0.027778in}}{\pgfqpoint{0.000000in}{0.000000in}}{%
\pgfpathmoveto{\pgfqpoint{0.000000in}{0.000000in}}%
\pgfpathlineto{\pgfqpoint{0.000000in}{-0.027778in}}%
\pgfusepath{stroke,fill}%
}%
\begin{pgfscope}%
\pgfsys@transformshift{4.614846in}{0.582778in}%
\pgfsys@useobject{currentmarker}{}%
\end{pgfscope}%
\end{pgfscope}%
\begin{pgfscope}%
\definecolor{textcolor}{rgb}{0.000000,0.000000,0.000000}%
\pgfsetstrokecolor{textcolor}%
\pgfsetfillcolor{textcolor}%
\pgftext[x=4.614846in,y=0.507778in,,top]{\color{textcolor}\sffamily\fontsize{10.000000}{12.000000}\selectfont \(\displaystyle 6\times10^{1}\)}%
\end{pgfscope}%
\begin{pgfscope}%
\definecolor{textcolor}{rgb}{0.000000,0.000000,0.000000}%
\pgfsetstrokecolor{textcolor}%
\pgfsetfillcolor{textcolor}%
\pgftext[x=2.661092in,y=0.295587in,,top]{\color{textcolor}\sffamily\fontsize{10.000000}{12.000000}\selectfont \(\displaystyle k\)}%
\end{pgfscope}%
\begin{pgfscope}%
\pgfsetbuttcap%
\pgfsetroundjoin%
\definecolor{currentfill}{rgb}{0.000000,0.000000,0.000000}%
\pgfsetfillcolor{currentfill}%
\pgfsetlinewidth{0.803000pt}%
\definecolor{currentstroke}{rgb}{0.000000,0.000000,0.000000}%
\pgfsetstrokecolor{currentstroke}%
\pgfsetdash{}{0pt}%
\pgfsys@defobject{currentmarker}{\pgfqpoint{-0.048611in}{0.000000in}}{\pgfqpoint{0.000000in}{0.000000in}}{%
\pgfpathmoveto{\pgfqpoint{0.000000in}{0.000000in}}%
\pgfpathlineto{\pgfqpoint{-0.048611in}{0.000000in}}%
\pgfusepath{stroke,fill}%
}%
\begin{pgfscope}%
\pgfsys@transformshift{0.511963in}{1.100383in}%
\pgfsys@useobject{currentmarker}{}%
\end{pgfscope}%
\end{pgfscope}%
\begin{pgfscope}%
\definecolor{textcolor}{rgb}{0.000000,0.000000,0.000000}%
\pgfsetstrokecolor{textcolor}%
\pgfsetfillcolor{textcolor}%
\pgftext[x=0.345296in,y=1.047621in,left,base]{\color{textcolor}\sffamily\fontsize{10.000000}{12.000000}\selectfont \(\displaystyle 2\)}%
\end{pgfscope}%
\begin{pgfscope}%
\pgfsetbuttcap%
\pgfsetroundjoin%
\definecolor{currentfill}{rgb}{0.000000,0.000000,0.000000}%
\pgfsetfillcolor{currentfill}%
\pgfsetlinewidth{0.803000pt}%
\definecolor{currentstroke}{rgb}{0.000000,0.000000,0.000000}%
\pgfsetstrokecolor{currentstroke}%
\pgfsetdash{}{0pt}%
\pgfsys@defobject{currentmarker}{\pgfqpoint{-0.048611in}{0.000000in}}{\pgfqpoint{0.000000in}{0.000000in}}{%
\pgfpathmoveto{\pgfqpoint{0.000000in}{0.000000in}}%
\pgfpathlineto{\pgfqpoint{-0.048611in}{0.000000in}}%
\pgfusepath{stroke,fill}%
}%
\begin{pgfscope}%
\pgfsys@transformshift{0.511963in}{1.795343in}%
\pgfsys@useobject{currentmarker}{}%
\end{pgfscope}%
\end{pgfscope}%
\begin{pgfscope}%
\definecolor{textcolor}{rgb}{0.000000,0.000000,0.000000}%
\pgfsetstrokecolor{textcolor}%
\pgfsetfillcolor{textcolor}%
\pgftext[x=0.345296in,y=1.742582in,left,base]{\color{textcolor}\sffamily\fontsize{10.000000}{12.000000}\selectfont \(\displaystyle 4\)}%
\end{pgfscope}%
\begin{pgfscope}%
\pgfsetbuttcap%
\pgfsetroundjoin%
\definecolor{currentfill}{rgb}{0.000000,0.000000,0.000000}%
\pgfsetfillcolor{currentfill}%
\pgfsetlinewidth{0.803000pt}%
\definecolor{currentstroke}{rgb}{0.000000,0.000000,0.000000}%
\pgfsetstrokecolor{currentstroke}%
\pgfsetdash{}{0pt}%
\pgfsys@defobject{currentmarker}{\pgfqpoint{-0.048611in}{0.000000in}}{\pgfqpoint{0.000000in}{0.000000in}}{%
\pgfpathmoveto{\pgfqpoint{0.000000in}{0.000000in}}%
\pgfpathlineto{\pgfqpoint{-0.048611in}{0.000000in}}%
\pgfusepath{stroke,fill}%
}%
\begin{pgfscope}%
\pgfsys@transformshift{0.511963in}{2.490303in}%
\pgfsys@useobject{currentmarker}{}%
\end{pgfscope}%
\end{pgfscope}%
\begin{pgfscope}%
\definecolor{textcolor}{rgb}{0.000000,0.000000,0.000000}%
\pgfsetstrokecolor{textcolor}%
\pgfsetfillcolor{textcolor}%
\pgftext[x=0.345296in,y=2.437542in,left,base]{\color{textcolor}\sffamily\fontsize{10.000000}{12.000000}\selectfont \(\displaystyle 6\)}%
\end{pgfscope}%
\begin{pgfscope}%
\pgfsetbuttcap%
\pgfsetroundjoin%
\definecolor{currentfill}{rgb}{0.000000,0.000000,0.000000}%
\pgfsetfillcolor{currentfill}%
\pgfsetlinewidth{0.803000pt}%
\definecolor{currentstroke}{rgb}{0.000000,0.000000,0.000000}%
\pgfsetstrokecolor{currentstroke}%
\pgfsetdash{}{0pt}%
\pgfsys@defobject{currentmarker}{\pgfqpoint{-0.048611in}{0.000000in}}{\pgfqpoint{0.000000in}{0.000000in}}{%
\pgfpathmoveto{\pgfqpoint{0.000000in}{0.000000in}}%
\pgfpathlineto{\pgfqpoint{-0.048611in}{0.000000in}}%
\pgfusepath{stroke,fill}%
}%
\begin{pgfscope}%
\pgfsys@transformshift{0.511963in}{3.185263in}%
\pgfsys@useobject{currentmarker}{}%
\end{pgfscope}%
\end{pgfscope}%
\begin{pgfscope}%
\definecolor{textcolor}{rgb}{0.000000,0.000000,0.000000}%
\pgfsetstrokecolor{textcolor}%
\pgfsetfillcolor{textcolor}%
\pgftext[x=0.345296in,y=3.132502in,left,base]{\color{textcolor}\sffamily\fontsize{10.000000}{12.000000}\selectfont \(\displaystyle 8\)}%
\end{pgfscope}%
\begin{pgfscope}%
\definecolor{textcolor}{rgb}{0.000000,0.000000,0.000000}%
\pgfsetstrokecolor{textcolor}%
\pgfsetfillcolor{textcolor}%
\pgftext[x=0.289740in,y=2.198889in,,bottom,rotate=90.000000]{\color{textcolor}\sffamily\fontsize{10.000000}{12.000000}\selectfont \(\displaystyle C\)}%
\end{pgfscope}%
\begin{pgfscope}%
\pgfpathrectangle{\pgfqpoint{0.511963in}{0.582778in}}{\pgfqpoint{4.298259in}{3.232222in}}%
\pgfusepath{clip}%
\pgfsetbuttcap%
\pgfsetroundjoin%
\definecolor{currentfill}{rgb}{0.000000,0.000000,0.000000}%
\pgfsetfillcolor{currentfill}%
\pgfsetlinewidth{1.003750pt}%
\definecolor{currentstroke}{rgb}{0.000000,0.000000,0.000000}%
\pgfsetstrokecolor{currentstroke}%
\pgfsetdash{}{0pt}%
\pgfsys@defobject{currentmarker}{\pgfqpoint{-0.041667in}{-0.041667in}}{\pgfqpoint{0.041667in}{0.041667in}}{%
\pgfpathmoveto{\pgfqpoint{0.000000in}{-0.041667in}}%
\pgfpathcurveto{\pgfqpoint{0.011050in}{-0.041667in}}{\pgfqpoint{0.021649in}{-0.037276in}}{\pgfqpoint{0.029463in}{-0.029463in}}%
\pgfpathcurveto{\pgfqpoint{0.037276in}{-0.021649in}}{\pgfqpoint{0.041667in}{-0.011050in}}{\pgfqpoint{0.041667in}{0.000000in}}%
\pgfpathcurveto{\pgfqpoint{0.041667in}{0.011050in}}{\pgfqpoint{0.037276in}{0.021649in}}{\pgfqpoint{0.029463in}{0.029463in}}%
\pgfpathcurveto{\pgfqpoint{0.021649in}{0.037276in}}{\pgfqpoint{0.011050in}{0.041667in}}{\pgfqpoint{0.000000in}{0.041667in}}%
\pgfpathcurveto{\pgfqpoint{-0.011050in}{0.041667in}}{\pgfqpoint{-0.021649in}{0.037276in}}{\pgfqpoint{-0.029463in}{0.029463in}}%
\pgfpathcurveto{\pgfqpoint{-0.037276in}{0.021649in}}{\pgfqpoint{-0.041667in}{0.011050in}}{\pgfqpoint{-0.041667in}{0.000000in}}%
\pgfpathcurveto{\pgfqpoint{-0.041667in}{-0.011050in}}{\pgfqpoint{-0.037276in}{-0.021649in}}{\pgfqpoint{-0.029463in}{-0.029463in}}%
\pgfpathcurveto{\pgfqpoint{-0.021649in}{-0.037276in}}{\pgfqpoint{-0.011050in}{-0.041667in}}{\pgfqpoint{0.000000in}{-0.041667in}}%
\pgfpathclose%
\pgfusepath{stroke,fill}%
}%
\begin{pgfscope}%
\pgfsys@transformshift{0.707338in}{0.729697in}%
\pgfsys@useobject{currentmarker}{}%
\end{pgfscope}%
\begin{pgfscope}%
\pgfsys@transformshift{2.218969in}{1.330412in}%
\pgfsys@useobject{currentmarker}{}%
\end{pgfscope}%
\begin{pgfscope}%
\pgfsys@transformshift{3.103216in}{2.289398in}%
\pgfsys@useobject{currentmarker}{}%
\end{pgfscope}%
\begin{pgfscope}%
\pgfsys@transformshift{3.730599in}{2.640983in}%
\pgfsys@useobject{currentmarker}{}%
\end{pgfscope}%
\begin{pgfscope}%
\pgfsys@transformshift{4.217235in}{3.668081in}%
\pgfsys@useobject{currentmarker}{}%
\end{pgfscope}%
\begin{pgfscope}%
\pgfsys@transformshift{4.614846in}{3.387773in}%
\pgfsys@useobject{currentmarker}{}%
\end{pgfscope}%
\end{pgfscope}%
\begin{pgfscope}%
\pgfsetrectcap%
\pgfsetmiterjoin%
\pgfsetlinewidth{0.803000pt}%
\definecolor{currentstroke}{rgb}{0.000000,0.000000,0.000000}%
\pgfsetstrokecolor{currentstroke}%
\pgfsetdash{}{0pt}%
\pgfpathmoveto{\pgfqpoint{0.511963in}{0.582778in}}%
\pgfpathlineto{\pgfqpoint{0.511963in}{3.815000in}}%
\pgfusepath{stroke}%
\end{pgfscope}%
\begin{pgfscope}%
\pgfsetrectcap%
\pgfsetmiterjoin%
\pgfsetlinewidth{0.803000pt}%
\definecolor{currentstroke}{rgb}{0.000000,0.000000,0.000000}%
\pgfsetstrokecolor{currentstroke}%
\pgfsetdash{}{0pt}%
\pgfpathmoveto{\pgfqpoint{4.810222in}{0.582778in}}%
\pgfpathlineto{\pgfqpoint{4.810222in}{3.815000in}}%
\pgfusepath{stroke}%
\end{pgfscope}%
\begin{pgfscope}%
\pgfsetrectcap%
\pgfsetmiterjoin%
\pgfsetlinewidth{0.803000pt}%
\definecolor{currentstroke}{rgb}{0.000000,0.000000,0.000000}%
\pgfsetstrokecolor{currentstroke}%
\pgfsetdash{}{0pt}%
\pgfpathmoveto{\pgfqpoint{0.511963in}{0.582778in}}%
\pgfpathlineto{\pgfqpoint{4.810222in}{0.582778in}}%
\pgfusepath{stroke}%
\end{pgfscope}%
\begin{pgfscope}%
\pgfsetrectcap%
\pgfsetmiterjoin%
\pgfsetlinewidth{0.803000pt}%
\definecolor{currentstroke}{rgb}{0.000000,0.000000,0.000000}%
\pgfsetstrokecolor{currentstroke}%
\pgfsetdash{}{0pt}%
\pgfpathmoveto{\pgfqpoint{0.511963in}{3.815000in}}%
\pgfpathlineto{\pgfqpoint{4.810222in}{3.815000in}}%
\pgfusepath{stroke}%
\end{pgfscope}%
\end{pgfpicture}%
\makeatother%
\endgroup%

  \end{subfigure}
    \begin{subfigure}{\textwidth}
\input{gradient_top_right-alpha-plot.pgf}
    \end{subfigure}
\caption{Plots of the dependence of $C$ and $\alpha$ on $k$ in \cref{eq:qmcerrorform} for $Q(u) = \gradu((1,1))$. Observe the $x$-axis is on a $\log_{10}$ scale, but $\log$ is the natural logarithm.  \label{fig:gradienttoprightCalpha}}
\end{figure}


\begin{table}[h]
  \centering
  \begin{tabular}{Sc Sc Sc Sc Sc}
\toprule
{} & $Q = \int_D u$ & $Q = u(\bzero)$ & $Q = u((1,1))$ & $Q = \gradu((1,1))$ \\
\midrule
$\alphaz$ &           1.34 &            1.38 &           1.51 &                1.51 \\
$\alphao$ &           0.16 &            0.19 &           0.21 &                0.21 \\
\bottomrule
\end{tabular}

  \caption{The quantities $\alphaz$ and $\alphao$ for differents QoIs, where the QMC error $Err \approx C \NQMC^{\alphaz - \alphao\log(k)}$.}\label{tab:qmcalpha}
  \end{table}

We use these computational results to determine how $\NQMC$ should increase with $k$, in order to keep the QMC error bounded. Based on QMC theory results for the stationary diffusion equation, e.g., \cite[Equation 4.2]{GrKuNuScSl:11}, we make the assumption that the QMC error (with one shift) satisfies
\beq\label{eq:qmcerrorform}
\QMCerror{Q}{1} = C \NQMC^{-\alpha},
\eeq
for some $C, \alpha > 0.$ Based on this assumption, \cref{fig:integralCalpha,fig:originCalpha,fig:toprightCalpha,fig:gradienttoprightCalpha} plot $C$ and $\alpha$ for increasing $k$. (In \cref{app:hhqmcconv}, we plot the QMC error for increasing $\NQMC$ for each $k \in \set{10,20,30,40,50,60}$ and for each QoI---these plots allow us to determine the values of $C$ and $\alpha$ for each value of $k.$) For the QoIs that are point evaluations (\cref{fig:originCalpha,fig:toprightCalpha}), $C$ appears to be constant; we assume $C$ is constant in all of the following calculations.

Based on the evidence in \cref{fig:integralCalpha,fig:originCalpha,fig:toprightCalpha,fig:gradienttoprightCalpha}, we conjecture
\beq\label{eq:alphaform}
\alpha(k) = \alphaz - \alphao\log(k),
\eeq
for some constants $\alphaz,\alphao > 0.$ (Throughout this \lcnamecref{sec:nbpcqmcnumerics}, $\log$ denotes the natural logarithm.) We fitted $\alphaz$ and $\alphao$, and have plotted the resulting line of best fit on \cref{fig:integralCalpha,fig:originCalpha,fig:toprightCalpha,fig:gradienttoprightCalpha}. (Observe that the conjectured form \cref{eq:alphaform} cannot hold for $k$ very large, as then $\alpha(k)$ would be negative. Nevertheless, for the range of $k$ we consider in these numerical experiments, the form \cref{eq:alphaform} seems to give a good fit with the data.)

Having understood how the QMC error increases with $k$ for fixed $\NQMC$, we now use this knowledge to determine how one should increase $\NQMC$ with $k$ in order to keep the QMC error bounded. Recalling that we assume $C$ in \cref{eq:qmcerrorform} is constant, if we take
\beq\label{eq:Nform}
\NQMC(k) = \exp\mleft(\Ctilde \alpha(k)^{-1}\mright),
\eeq
for some constant $\Ctilde > 0$, then substituting \cref{eq:Nform} into \cref{eq:qmcerrorform}, we see that the QMC error should remain bounded, with
\beqs
\QMCerror{Q}{1} = C \exp\mleft(-\Ctilde\mright).
\eeqs

In our numerical experiments with increasing $\NQMC(k)$ below, we choose $\Ctilde$ so that $\NQMC(10) = 2048,$ as in our numerical experiments to determine the behaviour of the QMC error, we used $\NQMC = 2048$ (with 20 shifts). Also, for ease of computational set-up, in our numerical experiments we take the number of QMC points to be a power of 2, chosen so that $\NQMCactual(k) = 2^{M(k)},$ where
\beqs
M(k) = \round{\logtwo\mleft(\NQMC(k)\mright)}.
\eeqs

Based on the results for the QoIs in \cref{tab:qmcalpha} (excluding the results for the QoI being the integral of the solution, as this seems to display slightly different convergence characteristics), in our numerical experiments we take $\alpha(k) = 1.4 - 0.18 \log(k).$ The resulting values of $\NQMC$ are summarised in \cref{tab:nqmc}.

\begin{table}[h]
  \centering
  \begin{tabular}{Sc Sc Sc }

\toprule

$k$ & $\NQMC$ & $\NQMCactual$\\
\midrule

10 &    2048 &          2048 \\

20 &    6184 &          8192 \\

30 &   13885 &         16384 \\

40 &   27164 &         32768 \\

50 &   48971 &         65536 \\

60 &   83577 &         65536 \\

\bottomrule

\end{tabular}


  \caption{The ideal and actual number of QMC points $\NQMC$, chosen so that the QMC error is empirically bounded for all $k$.}\label{tab:qmcpoints}
  \end{table}

Now that we understand how the number of QMC points should scale with $k$ in order to keep the QMC error bounded, we apply nearby preconditioning to QMC (with the number of points scaling as above) and observe how the computational work of this nearby-preconditioning-QMC (NP-QMC) algorithm scales with $k.$\ednote{Should I do the QMC calculations with multiple shifts, but this point scaling, again, and check that the estimate of the QMC error is roughly constant? It could take a while, and be rather expensive....}

As outlined above, we combine our sequential- and parallel-NPQMC algorithms:
\bit
\item We first use the sequential algorithm for low wave numbers (fixing the maximum number of GMRES iterations) and observe how the number of preconditioners (as a proportion of the number of QMC points) changes with $k$.
  \item We then use the parallel algorithm (with the above proportion of preconditioners) for higher values of $k.$
    \eit
    We remark that, in principle, one could use the sequential algorithm for all values of $k$, however, this would take an incredibly long time--- we see below that for $k=60$ we must perform $2^{20}$ Helmholtz solves; if we performed these solves sequentially, and each solve took 5 seconds, this computation would take on the order of 60 days to complete.

    The results for the sequential algorithm are summarised in \cref{tab:seq}, for $k$ up to 30. These results show that nearby preconditioning is very effective, with the number of preconditioners remaining (approximately) constant as a percentage of the total number of solves. This is the best result one could realistically hope for; in this case the decreasing (like $1/k$) radius over which nearby preconditioning is effective is offset by the growing number of QMC points, meaning the number of preconditioners is constant as a fraction of the total number of solves.

    Based on these sequential results, we then used the parallel algorithm with a target proportion of preconditioners of 0.2\%. (Although recall from our discussion above that the actual proportion of preconditioners used can vary due to rounding in the algorithm.) The results of these computations are summarised in \cref{tab:par}. We observe that the fraction of preconditioners is approximate 0.2\%, but the maximum (and average) number of GMRES iterations appears to grow slowly with $k.$ This may be because the placement of the preconditioning points is not optimal with respect to the $\dQMC$ metric; we conjecture that oversampling the number of preconditioners needed (for example, taking a proportion of 0.5\%) may result in a bounded number of GMRES iterations\ednote{Both---Should I run these computations?} Nevertheless, we see that nearby preconditioning gives considerable speedup, drastically reducing the number of preconditinoers that must be calculated.

    \begin{table}
  \centering
  \begin{tabular}{Sc Sc Sc Sc Sc Sc}
\toprule

$k$ & \# LU factorisations & \makecell{Total \#\\linear systems} & \makecell{\# LU factorisations$/$\\\# linear systems}(\%) & \makecell{Average \#\\GMRES iterations} & \makecell{Max. \#\\GMRES iterations}\\
\midrule

10 &                    5 &                                2048 &                                               0.24 &                                    7.13 &                                   10 \\

20 &                   39 &                                8192 &                                               0.48 &                                    7.26 &                                   10 \\

30 &                  127 &                               16384 &                                               0.78 &                                    7.47 &                                   10 \\

\bottomrule

\end{tabular}


  \caption{Results applying our sequential nearby-preconditioning-Quasi-Monte-Carlo algorithm, with the maximum number of GMRES iterations $=10$.}\label{tab:nbpcqmcseq}
\end{table}

\begin{table}
  \centering
  \begin{tabular}{Sc Sc Sc Sc Sc Sc}
\toprule

$k$ & \# LU factorisations & \makecell{Total \#\\linear systems} & \makecell{\# LU factorisations$/$\\\# linear systems}(\%) & \makecell{Average \#\\GMRES iterations} & \makecell{Max. \#\\GMRES iterations}\\
\midrule

10 &                    4 &                                2048 &                                               0.20 &                                    6.46 &                                   10 \\

30 &                  199 &                              131072 &                                               0.15 &                                    6.48 &                                   12 \\

40 &                  410 &                              262144 &                                               0.16 &                                    6.88 &                                   14 \\

50 &                 1118 &                              524288 &                                               0.21 &                                    6.97 &                                   14 \\

60 &                 1475 &                             1048576 &                                               0.14 &                                    7.35 &                                   16 \\

\bottomrule

\end{tabular}


  \caption{Results applying our parallel nearby-preconditioning-Quasi-Monte-Carlo algorithm with the target proportion of preconditioners as $0.2$\%.}\label{tab:nbpcqmcspar}
  \end{table}


    In conclusion, we see that nearby preconditioning gives very significant speedup when applied to a QMC model problem. We therefore expect that this technique will give significant speed up when applied to other, more realistic problems.
    

    
\section{Extension of the results to the truncated exterior Dirichlet problem}\label{sec:TEDP}

%\subsection{Definition of the TEDP and analogues of the results in \cref{sec:3}}

\paragraph{The impedance boundary $\Gamma_I$.} By comparing \cref{eq:src,eq:ibc}, we see that, in the case $g_I=0$, the TEDP approximates the DtN operator $T_R$ by $\ri k$. Indeed, by using Green's first identity and the definition of the normal derivative (see, e.g., \cite[Lemma 4.3]{Mc:00}), show that the boundary condition on $\Gamma_I$ imposed in the variational problem \cref{prob:vtedp} is 
%In this BVP, the DtN operator $T_R$ Sommerfeld radiation condition 
\beq\label{eq:imp}
\dudnu - \ri k\gamma u = g_I \ton \Gamma_I.
\eeq
where $\nu$ is the unit outward-pointing normal vector to $\Omega$ on $\Gamma_I$.

\paragraph{Existence and uniqueness of a solution to the TEDP.} The sesquilinear form $\aT(\cdot,\cdot)$ defined in \cref{eq:aT} satisfies the G\aa rding inequality \cref{eq:gardingbrief}, and existence and uniqueness of a solution to the TEDP follow under the same condition on $A$ (piecewise-Lipschitz) as for the EDP, as discussed in \cref{sec:vpGm}; in the case of Lipschitz scalar $A$, these unique-continuation arguments are summarised in \cite[\S2]{GrSa:18}.

\paragraph{Finite-element/Galerkin solution.}
The Galerkin matrix is defined exactly as in \cref{eq:matrixAdef}, except that 
\beq\label{eq:NTEDP}
\big(\Nmat\big)_{ij}:= \ri k\int_{\Gamma_I}  (\gamma\phi_i) \,\gamma \phi_j.
\eeq

\paragraph{The adjoint sesquilinear form.} For the TEDP, the adjoint sesquilinear form is given by 
\beq\label{eq:TEDPadjoint}
a^*(u,v) := \int_{\DR} 
\Big((A \grad u)\cdot\grad \vb
 - k^2 n u\vb\Big) +\ri k\int_{\Gamma_I} \gamma u\, \overline{\gamma v};
\eeq
then \cref{eq:A*} holds (with $\Nmat$ now given by \cref{eq:NTEDP}, and the analogue of \cref{lem:adjoint} follows in a straightforward way.


\paragraph{The analogues of \cref{cond:1nbpc,cond:2}.}
The statement of the TEDP analogues of \cref{cond:1nbpc,cond:2} are the same as for the EDP, apart from the following.
\ben
\item
$\supp \,f$ need not be a subset of $\widetilde{\Omega}$ (i.e.~the support of $f$ can go up to the impedance boundary $\Gamma_I$), and
\item the assumption $g_I= 0$ needs to be added to \cref{cond:1nbpc} and Part (i) of \cref{cond:2}.
\een
 Note that, since $a(\cdot,\cdot)$ for the TEDP satisfies the same G\aa rding inequality \cref{eq:gardingbrief} as the $a(\cdot,\cdot)$ for the EDP, \cref{lem:H1} holds for the TEDP under the TEDP-analogue of \cref{cond:1nbpc}.

\paragraph{The main results \cref{thm:1,cor:1}.}
Since \cref{cond:1nbpc,cond:2} are essentially unchanged from the EDP case, \cref{lem:keylemma1,lem:keylemma2} hold for the TEDP, and thus so do \cref{thm:1,cor:1,cor:1a}.

\paragraph{The PDE results \cref{thm:2} and \cref{lem:sharp}.}

The PDE bound \cref{thm:2} relies only on \cref{lem:H1}, which, as stated above, also holds for the TEDP. Therefore \cref{thm:2} holds for the TEDP under the TEDP-analogue of \cref{cond:1nbpc} described above. The construction in \cref{lem:sharp} to show sharpness of the bound in \cref{thm:1} (at least when $\Aso= \Ast= I$) also holds for the TEDP; this is because one can choose the supports of $\chi$ and $\widetilde{\chi}$ to be contained inside $\widetilde{\Omega}$, and then $u^{(1)}$ and $u^{(2)}$ defined in \cref{lem:sharp} satisfy the impedance boundary condition \cref{eq:imp} on $\Gamma_I$.

%% \paragraph{When the TEDP-analogue of \cref{cond:1nbpc} holds.}

%% In \cref{sec:cond1hold} we discussed 4 situations (Cases 1-4) where \cref{cond:1nbpc} is proved to hold for the EDP. We now discuss the TEDP-analogues of these.
%% %Cases 1, 3, and 4 (there is no proof yet for the TEDP-analogue of Case 2).

%% \emph{Cases 1 and 2: $\Aso$, $\nso$, and $\Gamma_I$  are $C^\infty$.} 
%% With the rays defined as in the EDP case (by the Melrose--Sj{\"o}strand generalized bicharacteristic flow 
%% \cite[\S24.3]{Ho:85}), the TEDP-analogue of nontrapping for the EDP is the assumption that 
%% every ray eventually hits the boundary at a \emph{non-diffractive point} (defined in \cite[Page 1037]{BaLeRa:92}). Note that, in the case $\Dm=\emptyset$ $\Aso= I$, and $\nso=1$, every ray eventually hits the boundary at a non-diffractive point by \cite[Lemma 5.3]{BaSpWu:16}.
%% Under the additional assumption that $\nso= 1$, \cref{cond:1nbpc} follows from the results of \cite{BaLeRa:92} by combining \cite[Theorem 1.8]{BaSpWu:16} and \cite[Remark 5.6]{BaSpWu:16}, but $C^{(1)}_{\rm bound}$ is not given explicitly.

%% \emph{Case 3: $\Dm$ is starshaped with respect to the origin, $\Aso$ and $\nso$ are Lipschitz and satisfy radial monotonicity-like conditions.}
%% When $\Gamma_I$ is also starshaped with respect to the origin and $A$ and $n$ satisfy \cref{eq:A1nbpc} and \cref{eq:n1nbpc} respectively (with $\Dp$ replaced by $\Omega$), 
%% \cite[Theorem A.6(i)]{GrPeSp:19} proves that
%% \cref{cond:1nbpc} holds, with an explicit expression for $C^{(1)}_{\rm bound}$. Analogous results when (a) $2\Aso - (\bx\cdot\nabla)\Aso \geq \mu_1$ and $\nso= 1$,
%% and  (b) $\Aso= I$ and  $2\nso + \bx \cdot \nabla \nso \geq \mu_2$, 
%% are contained in \cite[Theorem A.6(ii)]{GrPeSp:19} and \cite[Theorem A.6(iii)]{GrPeSp:19} respectively.
%% When $A$ is scalar, these results were also proved in \cite[Theorem 1]{BrGaPe:17} and, when $\Aso= I$ and $\Dm=\emptyset$, also in \cite[Theorem 3.2]{GrSa:18}.

%% \emph{Case 4: %\item[Case 4:]
%%  $\Aso$ and $\nso$ are allowed to be discontinuous.}
%% %\een
%% \cref{cond:1nbpc} is proved in \cite{CaVo:10} (without an explicit expression for $C^{(1)}_{\rm bound}$) when $\Dm$ is $C^\infty$ and nontrapping, $\Gamma_I$ is $C^\infty$, $\Aso= I $, and $\nso$ is a piecewise-constant, monotonically non-decreasing function, jumping on interfaces that are $C^\infty$ with strictly positive curvature.
%% Recall from \cref{cond:1nbpc} that \cite[Theorem 2.7]{GrPeSp:19} proves that \cref{cond:1nbpc} holds for the EDP (with an explicit expression for $C^{(1)}_{\rm bound}$) when $\Dm$ is starshaped with respect to the origin, $A$ and $n$ are $L^\infty$, with $A$ monotonically \emph{non-increasing} in the radial direction, and $n$ monotonically \emph{non-decreasing}. This proof can be extended to the TEDP, with the additional assumption that $\Gamma_I$ is star-shaped with respect to the origin; see the discussion in \cite[Section A.2]{GrPeSp:19}.

%\cref{cond:1nbpc} is proved, with an explicit expression for $C^{(1)}_{\rm bound}$, when 

%\newpage
%
%\section*{Questions for Th\'eo}
%
%\ben
%\item At the place marked A on the scanned pages, you seem to use the inequality 
%\beq\label{eq:Theo1}
%\vert\vert\vert \xi - \cP_h \xi\vert\vert\vert \lesssim h^\alpha \N{u_\phi- \cP_h u_\phi}_{0,\Omega}.
%\eeq
%\een
%
%\newpag

\section*{Owen to do list}
\ben
\item Varying  $\|\Aso-\Ast\|_{L^\infty}$ and $\|\nso-\nst\|_{L^\infty}$ in standard GMRES.
\item Computations where $\|\Aso-\Ast\|_{L^\infty}$ and $\|\nso-\nst\|_{L^\infty}$ are sometimes large; is having the standard deviations of these $\sim 1/k$ good enough for $k$-independent GMRES iterations?
\item ***on backburner*** Checking under what conditions (if any) Part (ii) \cref{cond:2} holds by running the following experiment:
%\item Exciting experiments for random $n$ that you told us about last week.
%\item In the weighted norm, the condition on $A$ is ``$k \|\Aso-\Ast\|_{L^\infty}$ sufficiently small" but in the Euclidean norm the best we have so far is ``$h^{-1} \|\Aso-\Ast\|_{L^\infty}$ sufficiently small". You indicated before that experiments seemed to indicate that ``$k \|\Aso-\Ast\|_{L^\infty}$ sufficiently small" seemed correct for the Euclidean norm too. The next time we meet, can you show me these results please?
%\item Please run the following numerical experiment.
\bit
\item TEDP with $\Omega$ a square/rectangle.
\item $\Aso$ being at least Lipschitz (but smooth is fine). To keep things simple, just take scalar- (as opposed to matrix-) valued $\Aso$ and don't worry about making it nontrapping.
\item Smoothness of $\nso$ doesn't really matter, just take smooth in the first instance for simplicity (and also don't worry about nontrapping).
\item $\Vhp$ piecewise linear.
\item Linear system $\Amato \bu = \Smat_{A} \balpha$ for some arbitrary complex-valued vector $\balpha$ and some arbitrary $A\in L^\infty$. (I claim this corresponds to the problem described in Part (ii) of \cref{cond:2},  but please check this!)
\item For each $\Aso, \nso, \balpha$, solve linear system for increasing values of $k$, first with $h\sim k^{-2}$, and then with $h\sim k^{-3/2}$.
\item Goal: see if the bound \cref{eq:bound4} holds, using 
\beqs
\N{\sum_j \alpha_j (A\nabla \phi_j)}_{\LtDR} \quad \text{ as a proxy for } \quad \N{F}_{(\HokDR)'}.
\eeqs
\eit
%\item Varying  $\|\Aso-\Ast\|_{L^\infty}$ and $\|\nso-\nst\|_{L^\infty}$ in \emph{weighted} GMRES.
\een

