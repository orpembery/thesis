\section{Introduction and Motivation from UQ}\label{sec:intronbpc}

\subsection{Motivation from uncertainty quantification of the Helmholtz equation} 
Consider the stochastic Helmholtz equation 
\beq\label{eq:nbpchh}
\nabla\cdot\big(A(\omega,\bx) \nabla u(\omega,\bx) \big) + k^2 n(\omega,\bx) u(\omega,\bx) =-f(\bx), \quad \bx\in\Dp,
\eeq
as defined in \cref{chap:stochastic}. To calculate quantities of interest of the solution $u(\omega,\cdot)$, one must solve many deterministic Helmholtz problems, with each one corresponding to different realisations of the coefficients $A(\omega,\cdot)$ and $n(\omega,\cdot)$.
Solving all these deterministic problems is a very computationally-intensive task because linear systems arising from discretisations of the Helmholtz equation are notoriously difficult to solve; see the discussion in \cref{sec:numsolve} above.

Moreover, when using preconditioned iterative methods, as described in \cref{sec:numsolve} above, the cost of \emph{constructing} preconditioners for the linear systems is also very expensive. For example, if our preconditioner is given by an exact $LU$ factorisation, then (in 3-d) the cost of calculating the preconditioner is $\cO\mleft(N^2\mright)$ and the cost of applying the preconditioner is $\cO\mleft(N^{4/3}\mright)$ (where $N$ is the number of unknowns), see, e.g., \cite[Section 4]{GaZh:19}.

Observe that the cost of applying the preconditioner is significantly less than the cost of constructing the preconditioner. Therefore, if one calculates a preconditioner for one realisation of \cref{eq:nbpchh} and then re-uses it for many `nearby' (in some sense) realisations of \cref{eq:nbpchh}, then one could obtain considerable computational savings. This idea of `reusing' preconditioners for `nearby' linear systems is the `nearby preconditioning' strategy proposed in this \lcnamecref{chap:nbpc}.

If one does not use an exact $LU$-factorisation as the preconditioner, one may still obtain considerable savings by reusing preconditioners. Even if one uses a modern Helmholtz preconditioner such as a sweeping preconditioner or a domain-decompostion preconditioner (see the recent review article \cite{GaZh:19} for an overview of many types of Helmholtz preconditioners) one still performs $LU$-factorisations (or direct solves) on subdomains of $\Dp$ (see, e.g., \cite[Section 2]{GaZh:19}). Therefore resuing preconditioners, and thereby calculating fewer $LU$ factorisations may still lead to considerable savings.

In this \lcnamecref{chap:nbpc}, for simplicity, we restrict our attention to the case where the preconditioner is an exact $LU$ decomposition. In this case, the only error that arises in the nearby preconditioning process is due to applying a preconditioner constructed for one realisation of the Helmholtz equation to a different, nearby, realisation. Therefore  we do not need to consider the error arising from the preconditioning process itself.% The answer to \cref{it:nbpcq1} above makes $k$-explicit the extent to which the $LU$ factorisation for one realisation of \cref{eq:nbpchh} can be used as an effective preconditioner for Galerkin matrices arising from different realisations of $A$ and $n$.

\subsection{Statement of the problem}\label{sec:problem}
Let $\Aj, \nj$, $j=1,2$ satisfy the properties of $A$ and $n$ in \cref{prob:vedp}, with $\uj$ the corresponding solution and $\Dm,$ $f$, etc. as in \cref{prob:vedp}. Let $\Amatj$, $j=1,2,$ be the Galerkin matrices corresponding to $h$-finite-element discretisations of \cref{prob:vedp} (see \cref{eq:matrixAjdef} below for a precise definition of $\Amatj$). The results we prove in this \lcnamecref{chap:nbpc} also hold for the TEDP, \cref{prob:vtedp}, see \cref{sec:TEDP} below for the minor changes one must make in this case.
 
This \lcnamecref{chap:nbpc} answers the following question:

% Inspired by the usage of enumitem here: https://tex.stackexchange.com/a/58714
\ben[label=Q1., ref=Q1]
\item\label[itemblank]{it:nbpcq1} How small must $\NLi{\Aso - \Ast}$ and 
$\NLi{\nso - \nst}$ be (in some norm, in terms of $k$-dependence) for GMRES 
applied to $(\Amat^{(1)})^{-1}\Amat^{(2)}$ to converge in a $k$-independent number of iterations
%$ to be a good preconditioner for $\Amat^{(2)}$
 for arbitrarily large $k$? 
\een


\subsection{Statement of the main results}\label{sec:main}
Our main results about \cref{it:nbpcq1} are \cref{cor:1,cor:1a} below. \Cref{cor:1a} gives results in the \emph{Euclidean} norm on matrices, denoted by $\|\cdot\|_2$ (induced by the Euclidean norm on vectors), whereas \Cref{cor:1} gives results in the \emph{weighted} norms $\NDmatk{\cdot}$ and $\NDmatkI{\cdot}$. These weighted norms are induced by the corresponding vector norms
\beq\label{eq:Dk}
\NDmatk{\bv}^2\de \big( \Dmatk \bv, \bv\big)_2 = %\big( (\Smat_I + k^2 \Mmat_1)\bv,\bv\big)_2 
\N{v_h}^2_{\HokDR}
\quad \tand
\quad \NDmatkI{\bv}^2\de \big( \Dmatk^{-1} \bv, \bv\big)_2 %= %\big( (\Smat_I + k^2 \Mmat_1)\bv,\bv\big)_2 
%\N{v_h}^2_{\HokDR}
\eeq
where $\Dmatk$ is given in terms of familiar finite-element stiffness- and mass-matrices by \cref{eq:Dk2} below and $\vh =\sum_i \vi \phii$, where the $\phii$ are the finite-element basis functions.

As described in \cref{sec:wpdisc}, the PDE analysis of the Helmholtz equation naturally takes place in the norm $\NHokDR{\cdot}$, and \cref{eq:Dk} shows that the norm $\NDmatk{\cdot}$ is
the discrete analogue of the norm $\NHokDR{\cdot}$. %See \cref{eq:Dk2} and \cref{eq:Dk3} below for this norm expressed in terms of the Euc
The norms $\NDmatk{\cdot}$ and $\NDmatkI{\cdot}$ recently appeared in results about the convergence of domain-decomposition methods %in this norm are proved 
for the Helmholtz equation \cite{GrSpVa:17}, \cite{GrSpZo:18}, and for the time-harmonic Maxwell equations \cite{BoDoGrSpTo:19}. 

\Cref{cor:1,cor:1a} are proved under the \Cref{cond:1nbpc,cond:2} below. These \lcnamecrefs{cond:1nbpc} can be informally stated as 
\bit
\item the obstacle $\Dm$ and the coefficients $\Aso$ and $\nso$ are such that $u^{(1)}$ exists and the problem is \emph{nontrapping} (in the sense described in  \cref{sec:wpdisc} above), and
\item the meshsize $h$ and polynomial degree $p$ in the finite-element method are chosen to depend on $k$ to ensure that the 
finite-element solution to the problem with coefficients $\Aso$ and $\nso$ exists, is unique, and 
%Galerkin method (with coefficients $\Aso$ and $\no$) 
is uniformly accurate as $k\tendi$. 
\eit 

\begin{condition}[Nontrapping bound on $u^{(1)}$]\label{cond:1nbpc}
$\Aso, \nso,$ and $\Dm$ are such that, given $f\in L^2(\DR)$ with $\supp \, f \subset \BR$, 
the solution of \cref{prob:vedp} %(\cref{prob:edp}) 
$u^{(1)}$ exists, is unique, and, given $k_0>0$, $u^{(1)}$ satisfies the bound 
\beq\label{eq:bound1}
\big\|u^{(1)}\big\|_{\HokDR} \leq C^{(1)}_{\rm bound} \N{f}_{L^2(\Dp)} \quad \tfa k\geq k_0,
\eeq
where $C^{(1)}_{\rm bound}$ is independent of $k$, but dependent on $\Aso, \nso, \Dm, R$, and $k_0$.
\end{condition}

\begin{condition}[$k$-independent accuracy of the FE solution for $a^{(1)}(\cdot,\cdot)$]
\label{cond:2}

\

(i) Given $k_0>0$, $h$ and $p$ are such that, given $f\in \LtDR$ with $\supp\, f \subset \BR$, the solution $u_h$ of the Galerkin method \cref{eq:galerkin} with $a(\cdot,\cdot)=a^{(1)}(\cdot,\cdot)$ (and with $\FE(v)$ defined in \cref{eq:Ledp}) exists and is unique for all $k\geq k_0$. % (so that the matrix $\Amato$ is invertible). 
Furthermore, if $f= n\sum_j \alpha_j\phi_j$ for some $\alpha_j \in \CC$ and  $n\in \LiDRRR$  (i.e.~$f$ is an arbitrary element of $\Vhp$ multiplied by $n$), then
\beq\label{eq:bound3}
\N{u-u_h}_{\HokDR} \leq C^{(1)}_{\rm FEM1} \N{f}_{\LtDR} \quad\tfa k\geq k_0, 
\eeq
where $C^{(1)}_{\rm FEM1}$  is independent of $k$ and $h$, but dependent on $\Aso, \nso, \Dm, R, k_0$, and $p$.

(ii) Given $k_0>0$, $h$ and $p$ are such that, given $\LE\in (\HozDDR)'$, the solution $u_h$ of the Galerkin method \cref{eq:galerkin} 
with $a(\cdot,\cdot)=a^{(1)}(\cdot,\cdot)$
exists, and is unique for all $k\geq k_0$.
%(so that the matrix $\Amato$ is invertible). 
Furthermore, if $\LE(v)= (A\nabla \widetilde{f},\nabla v)_{\LtDR}$, where $A\in L^\infty(\DR, \RR^{d\times d})$, $A$ is symmetric, and $\widetilde{f} \de \sum_j \alpha_j \phi_j$ with $\alpha_j\in \CC$
 (i.e.~$\widetilde{f}$ is an arbitrary element of $\Vhp$), then
\beq\label{eq:bound4}
\N{u-u_h}_{\HokDR} \leq C^{(1)}_{\rm FEM2}\,k\, \N{\LE}_{(\HokDR)'} \quad\tfa k\geq k_0, 
\eeq
where $C^{(1)}_{\rm FEM2}$  is independent of $k$ and $h$, but dependent on $\Aso, \nso, \Dm, R, k_0$, and $p$.
\end{condition}

\bre[\Cref{eq:bound4} is $k$-independent]\label{rem:yesitis}
Note that \cref{eq:bound4} is a $k$-independent bound on $\NHokDR{\utilde},$ despite the fact that a factor $k$ appears on the right-hand side. The factor $k$ appears because the weighted norm $\NHokDR{\cdot}$ is used in the definition of $\NHokDRs{\cdot},$ and therefore $\NHokDRs{\Ltilde} \sim \NHoDRs{\Ltilde}/k.$
\ere




\bth[Answer to \cref{it:nbpcq1}: $k$-independent weighted GMRES iterations]\label{cor:1}
Assume that $\Dm$, $\Aso$, and $\nso$ satisfy \cref{cond:1nbpc}, and $h$ and $p$ satisfy \cref{cond:2}. Given $k_0>0$, there exist constants $\Co$ and $\Ct$  independent of $h$ and $k$ (but dependent on $\Dm, \Aso, \nso, p$, and $\kz$) such that if 
% there exists $C_2>0$, independent of $h$ and $k$ (but dependent on $\Dm, \Aso, \nso$, $p$, and $k_0$) and given explicitly in \cref{eq:C2} below,
% such that if 
\beq\label{eq:cond}
C_1 \,k \,\NLiDRRRdtd{\Aso-\Ast} +C_2 \, k\, \NLiDRRR{\nso-\nst}
\leq \frac{1}{2}
\eeq
for all $k\geq k_0$, then \emph{both} weighted GMRES working in $\|\cdot\|_{\Dmat_k}$ (and the associated inner product) applied to 
\beq\label{eq:pcsystem1}
(\Amat^{(1)})^{-1}\Amat^{(2)}\bu = \bff
\eeq
\emph{and} weighted GMRES working in $\|\cdot\|_{(\Dmat_k)^{-1}}$ (and the associated inner product) applied to 
\beq\label{eq:pcsystem2}
\Amat^{(2)}(\Amat^{(1)})^{-1}\bv = \bff
\eeq
 converge in a $k$-independent number of iterations for all $k\geq k_0$.
\enth

\bth[Answer to \cref{it:nbpcq1}: $k$-independent (unweighted) GMRES iterations]\label{cor:1a}
Assume that $\Dm$, $\Aso$, and $\nso$ satisfy \cref{cond:1nbpc}, and $h$ and $p$ satisfy \cref{cond:2}. Given $k_0>0$,
let $C_1$ and $C_2$ be as in \cref{cor:1}, and let $s_{\pm}$ and $m_{\pm}$ be as in \cref{lem:normequiv} below (note that all these constants are independent of $k$, $h$, and $p$). Then if 
% there exists $C_2>0$, independent of $h$ and $k$ (but dependent on $\Dm, \Aso, \nso$, $p$, and $k_0$) and given explicitly in \cref{eq:C2} below,
% such that if 
\beq\label{eq:conda}
 C_1 \,\left(\frac{s_+}{m_-}\right) \,\frac{1}{h} \,
\NLiDRRRdtd{\Aso-\Ast} + C_2 \, \left(\frac{m_+}{m_-} \right)k \, \NLiDRRR{\nso-\nst}
\leq \frac{1}{2}
\eeq
for all $k\geq k_0$, then standard GMRES (working in the Euclidean norm and inner product) applied to either of the equations \cref{eq:pcsystem1} or \cref{eq:pcsystem2}
%\beqs
%(\Amat^{(1)})^{-1}\Amat^{(2)}\bu = \bff\quad\text{ or } \quad\Amat^{(2)}(\Amat^{(1)})^{-1}\bv = \bff
%\eeqs
 converges in a $k$-independent number of iterations for all $k\geq k_0$.
\enth

Two notes regarding \cref{cor:1,cor:1a}: (i) the constants $C_1$ and $C_2$ are expressed explicitly in \cref{eq:C1nbpc} and \cref{eq:C2} below in terms of constants appearing in \cref{cond:1nbpc,cond:2}, and (ii) the $L^\infty(\DR)$ norm on a matrix-valued functions appearing on the right-hand sides of \cref{eq:main1} and \cref{eq:main1a} is defined by
\beqs
\NLiDRRRdtd{A}\de \esssup_{\bx\in\DR}\N{A(\bx)}_2.
\eeqs


The factor $1/2$ on the right-hand sides of \cref{eq:cond} and \cref{eq:conda} can be replaced by any number $<1$ and the result still holds, although the number of GMRES iterations is then larger -- but still independent of $k$.
%In \cref{sec:proofFEM}, the constant $C_2$ is expressed explicitly in terms of $C_1$.

The proofs of \cref{cor:1,cor:1a} are in \cref{sec:mainproofs} below.

\bre
When $h\sim  k^{-1}$, the bounds \cref{eq:main1} and \cref{eq:main1a} (and hence also \cref{eq:cond} and \cref{eq:conda}) are identical in their $k$-dependence; however, when $h\ll k^{-1}$ (as one needs to take to overcome the pollution effect, as discussed in \cref{sec:helmfedisc}) the bound \cref{eq:main1a} for standard GMRES is more pessimistic than the bound \cref{eq:main1} for weighted GMRES.
\ere


\paragraph{How sharp are \cref{cor:1,cor:1a} in their $k$-dependence?}
Numerical experiments in \cref{sec:num} indicate that the condition \cref{eq:cond} is sharp, i.e., that the $k$ in \cref{eq:cond} cannot be replaced by $k^\alpha$ for $\alpha<1$. This indicated sharpness of \cref{eq:cond} is also supported by the PDE-result \cref{thm:2} below. Indeed, \cref{thm:2} % and \cref{lem:1} 
 shows that the condition
\beq\label{eq:sufficientlysmall}
k\,
\NLiDRRRdtd{\Aso-\Ast} \quad\text{ and } \quad k\,\NLiDRRR{\nso-\nst}
%\Big) 
\quad\text{ both sufficiently small}
\eeq
is not only an answer to \cref{it:nbpcq1} (about finite-element discretisations), but is also the natural answer to the analogue of \cref{it:nbpcq1} at the level of PDEs, namely 
\ben[label=Q2., ref=Q2]
\item\label[itemblank]{it:nbpcq2}
How small must $\NLiDRRRdtd{\Aso - \Ast}$ and 
$\NLiDRRR{\nso - \nst}$ be (in terms of $k$-dependence) for the relative error in approximating 
%$u^{(1)}$ to be a good approximation to 
$u^{(2)}$ by $u^{(1)}$ to be bounded independently of $k$ for arbitrarily-large $k$? 
\een
\cref{lem:sharp} then shows that the condition ``$k\NLiDRRR{\nso - \nst}$ sufficiently small" is the \emph{provably-sharp} answer to \cref{it:nbpcq2} when $\Aso= \Ast= I$.

%Before stating these PDE results, we define the weighted $H^1$ norm
%\beq\label{eq:1knorm}
%\N{v}^2_{\HokDR} \de \N{\grad v}^2_{L^2(\DR)} + k^2 \N{v}^2_{L^2(\DR)} \quad \tfor v \in H^1_{0,D}(\DR),
%\eeq
%where the space $H^1_{0,D}(\DR)$, defined by \cref{eq:spaceEDP} below, is the natural space containing the solution of the exterior Dirichlet problem. 
To state these PDE results, we use the notation for $a,b>0$ that $a\lesssim b$ when $a\leq C b$ for some $C>0$, independent of $k$, and $a\sim b$ if $a\lesssim b$ and $b\lesssim a$.


%The sharpness of \cref{eq:cond} and \cref{eq:main1} is also supported by the answer to the analogue of \cref{it:nbpcq1} at the level of PDEs. Indeed, the following \cref{thm:2} is the analogue of \cref{thm:1} 

\begin{theorem}[Answer to \cref{it:nbpcq2} (the PDE analogue of \cref{it:nbpcq1})]\label{thm:2}
%Given $f\in L^2(\DR)$ such that $\supp \, f \subset \BR$, 
Let $\Dm$, $\Aso$, and $\nso$ satisfy \cref{cond:1nbpc}, and let $\Dm$, $\Ast$, and $\nst$ be such that $u^{(2)}$ exists
for any $f\in L^2(\DR)$ such that $\supp \, f \subset \BR$. 
Then, given $k_0>0$, there exists $C_3>0$, independent of $k$ and given explicitly in terms of $\Dm$, $\Aso$, and $\nso$ in \cref{eq:C3} below, such that
\beq\label{eq:PDEbound}
\frac{\big\|u^{(1)}-u^{(2)}\big\|_{\HokDR}
}{
\N{u^{(2)}}_{\HokDR}
}\leq C_3 \,k\, \max\set{\NLiDRRRdtd{\Aso-\Ast}\,,\, \NLiDRRR{\nso-\nst}}
\eeq
for all $k\geq k_0$. 
\end{theorem}

\ble[Sharpness of the bound \cref{eq:PDEbound} when $\Aso = \Ast= I$]\label{lem:sharp}
There exist $f, \,\nso$, and $\nst$ (with $\nso\neq \nst$) such that 
the corresponding solutions $u^{(1)}$ and $u^{(2)}$ of \cref{prob:edp} with $\Aso = \Ast= I$ exist, are unique, and satisfy
\beq\label{eq:sharp1}
\frac{\N{u^{(1)}-u^{(2)}}_{\HokDR}
}{
\N{u^{(2)}}_{\HokDR}
}
\sim 
\frac{\N{u^{(1)}-u^{(2)}}_{L^2(\DR)}
}{
\N{u^{(2)}}_{L^2(\DR)}
}\sim k \NLiDRRR{\nso-\nst}.
\eeq
%\noi (ii) There exist $f, \Aso, \Ast$, (with $\Aso\not\equiv \Ast$), such that 
%the corresponding solutions $u^{(1)}$ and $u^{(2)}$ of the exterior Dirichlet problem with $\nso \equiv \nst\equiv 1$ exist, are unique, and satisfy
%%There exist $f\in L^2(\DR), \Aj \in C^{0,1}(\DR)$, $j=1,2$ (with $\Aso\not\equiv \Ast$), such that the corresponding solutions $u^{(1)}$ and $u^{(2)}$ of the exterior Dirichlet problem with $\nso \equiv \nst\equiv 1$ satisfy
%\beq\label{eq:sharp2}
%\frac{\N{u^{(1)}-u^{(2)}}_{\HokDR}
%}{
%\N{u^{(2)}}_{\HokDR}
%}
%\sim 
%\frac{\N{u^{(1)}-u^{(2)}}_{L^2(\DR)}
%}{
%\N{u^{(2)}}_{L^2(\DR)}
%}\sim k \big\|\Aso-\Ast\big\|_{L^\infty(\DR)}.
%\eeq
\ele

\bre[Physical interpretation for $k$-dependence]\label{rem:physical1k}
It is unsurprising that the condition \cref{eq:sufficientlysmall} is a sufficient condition to answer both \cref{it:nbpcq1} and \cref{it:nbpcq2}. Recall that $1/k$ is proportional to the wavelength $2\pi/k$ of the wave $u$ (at least when $A=I$ and $n=1$). As the wavelength is the natural length scale associated with the wave $u$, it is unsurprising that perturbations of size (up to) $1/k$ give bounded relative difference (in \cref{it:nbpcq2}) and bounded GMRES iterations for the nearby-preconditioned linear system (in \cref{it:nbpcq1}). In some sense, perturbations of size up to $1/k$ are `unseen' by the PDE or numerical method. Perturbations of order $1/k$ being `unseen' by the PDE can also be seen in bounds proved for $u$ where $n = \no + \eta,$ with $\no$ nontrapping and $\NLiDRRR{\eta} \lesssim 1/k,$ see \cref{rem:kdep} above.
\ere


\section{Numerical experiments}\label{sec:num}
\subsection{Investigating the sharpness of \cref{thm:1} and \cref{cor:1}}

\Cref{thm:1,cor:1} were stated for the exterior Dirichlet problem, but, as highlighted in \cref{sec:problem} and shown in \cref{sec:TEDP}, they hold also for the truncated exterior Dirichlet problem (TEDP) \cref{prob:vtedp}. The numerical experiments in this section seek to verify the analogues of \cref{thm:1} and \cref{cor:1} for the TEDP, and investigate their sharpness. More specifically, the experiments seek to verify whether the condition \cref{eq:cond} is:
\ben
\item sufficient, and
\item necessary
  \een
  for \emph{standard} GMRES applied to \cref{eq:pcsystem1} to converge in a number of iterations that is independent of $k.$

Based on the PDE results \cref{thm:2,lem:sharp} above, we expect that the condition \cref{eq:sufficientlysmall} is a necessary and sufficient condition for standard GMRES applied to \cref{eq:pcsystem1} to converge in a $k$-independent number of iterations, even though we can only prove this is a sufficient condition for \emph{weighted} GMRES. We expect this because \cref{eq:sufficientlysmall} is a sufficient condition for \cref{it:nbpcq2}, the PDE analogue of \cref{it:nbpcq1}.

To verify this expected behaviour, we perform numerical experiments for the setup described in \cref{app:compsetup} with $\Aso = I$ and $\nso = 1$. We define $f$ and $\gI$ to correspond to a plane wave incident from the bottom left passing through a homogeneous medium given by coefficients $\Aso$ and $\nso$. We perform experiments for $A$ and $n$ separately, i.e., first we perform experiments with $\Ast=I$ and $\nst$ varying, and then we perform experiments with $\Ast$ varying and $\nst=1.$

We define $\Ast$ and $\nst$ to be piecewise constant on a $10\times10$ square grid, with their values on each square chosen independently at random from a $\Unif\mleft(1-\alpha,1+\alpha\mright)$ distribution, with $\alpha \in (0,1)$ chosen as described below. For $\Ast,$ we also impose the restriction that $\Ast$ is (piecewise) positive-definite almost surely. We solve the linear systems \cref{eq:pcsystem1} for $k = 20,40,60,80,100$ using standard GMRES and record the number of GMRES iterations taken to achieve a (relative) tolerance of $10^{-5}$ (relative to $\Nt{\bfb}$).

We perform experiments for three different functional forms of $\alpha$  taking $\alpha = 0.5,\, 0.5/k^{1/2},$ and $ 0.5/k.$ We expect that when $\alpha = 0.5$ or $\alpha = 0.5/k^{1/2},$ the number of GMRES iterations required for convergence will increase as $k$ increases, whereas we expect that when $\alpha = 0.5/k$ the number of GMRES iterations required for convergence will remain bounded as $k$ increases, even though this behaviour has only been proven for $\NLiDRRRdtd{\Aso-\Ast}$ for weighted GMRES.


  For both $\NLiDRRRdtd{\Aso-\Ast}$ and $\NLiDRRR{\nso-\nst}$ we see the expected behaviour in all three cases. When $\alpha = 0.5$ we see growth in the number of GMRES iterations needed to achieve convergence, and when $\alpha = 0.5/k,$ we see that the number of GMRES iterations is bounded as $k$ increases. When $\alpha = 0.5/k^{1/2}$ we see slower growth as $k$ increases, (at least for $\Aso-\Ast$) but growth nonetheless.

  The behaviour for $\alpha = 0.5$ indicates that the effectiveness of this `nearby preconditioning' strategy deteriorates rapdily as $k$ increases when $\NLiDRRRdtd{\Aso-\Ast}$ and $\NLiDRRR{\nso-\nst}$ are independent of $k$. The behaviour for $\alpha = 0.5/k$ verifies that the condition \cref{eq:sufficientlysmall} is sufficient to achieve bounded GMRES iterations in this case, and the behaviour for $\alpha = 0.5/k^{1/2}$ indicates that this condition is sharp; reducing the magnitude of the difference betwen the Helmholtz problems as $k$ increases, but not at the rate $1/k$ is insufficient for the nearby preconditioning strategy to achieve $k$-independent GMRES iterations.
  \begin{figure}
    \centering
    \begin{subfigure}{\textwidth}
      \centering
%% Creator: Matplotlib, PGF backend
%%
%% To include the figure in your LaTeX document, write
%%   \input{<filename>.pgf}
%%
%% Make sure the required packages are loaded in your preamble
%%   \usepackage{pgf}
%%
%% Figures using additional raster images can only be included by \input if
%% they are in the same directory as the main LaTeX file. For loading figures
%% from other directories you can use the `import` package
%%   \usepackage{import}
%% and then include the figures with
%%   \import{<path to file>}{<filename>.pgf}
%%
%% Matplotlib used the following preamble
%%   \usepackage{fontspec}
%%   \setmainfont{DejaVuSerif.ttf}[Path=/home/owen/progs/firedrake-complex/firedrake/lib/python3.5/site-packages/matplotlib/mpl-data/fonts/ttf/]
%%   \setsansfont{DejaVuSans.ttf}[Path=/home/owen/progs/firedrake-complex/firedrake/lib/python3.5/site-packages/matplotlib/mpl-data/fonts/ttf/]
%%   \setmonofont{DejaVuSansMono.ttf}[Path=/home/owen/progs/firedrake-complex/firedrake/lib/python3.5/site-packages/matplotlib/mpl-data/fonts/ttf/]
%%
\begingroup%
\makeatletter%
\begin{pgfpicture}%
\pgfpathrectangle{\pgfpointorigin}{\pgfqpoint{6.400000in}{4.800000in}}%
\pgfusepath{use as bounding box, clip}%
\begin{pgfscope}%
\pgfsetbuttcap%
\pgfsetmiterjoin%
\definecolor{currentfill}{rgb}{1.000000,1.000000,1.000000}%
\pgfsetfillcolor{currentfill}%
\pgfsetlinewidth{0.000000pt}%
\definecolor{currentstroke}{rgb}{1.000000,1.000000,1.000000}%
\pgfsetstrokecolor{currentstroke}%
\pgfsetdash{}{0pt}%
\pgfpathmoveto{\pgfqpoint{0.000000in}{0.000000in}}%
\pgfpathlineto{\pgfqpoint{6.400000in}{0.000000in}}%
\pgfpathlineto{\pgfqpoint{6.400000in}{4.800000in}}%
\pgfpathlineto{\pgfqpoint{0.000000in}{4.800000in}}%
\pgfpathclose%
\pgfusepath{fill}%
\end{pgfscope}%
\begin{pgfscope}%
\pgfsetbuttcap%
\pgfsetmiterjoin%
\definecolor{currentfill}{rgb}{1.000000,1.000000,1.000000}%
\pgfsetfillcolor{currentfill}%
\pgfsetlinewidth{0.000000pt}%
\definecolor{currentstroke}{rgb}{0.000000,0.000000,0.000000}%
\pgfsetstrokecolor{currentstroke}%
\pgfsetstrokeopacity{0.000000}%
\pgfsetdash{}{0pt}%
\pgfpathmoveto{\pgfqpoint{0.800000in}{0.528000in}}%
\pgfpathlineto{\pgfqpoint{5.760000in}{0.528000in}}%
\pgfpathlineto{\pgfqpoint{5.760000in}{4.224000in}}%
\pgfpathlineto{\pgfqpoint{0.800000in}{4.224000in}}%
\pgfpathclose%
\pgfusepath{fill}%
\end{pgfscope}%
\begin{pgfscope}%
\pgfpathrectangle{\pgfqpoint{0.800000in}{0.528000in}}{\pgfqpoint{4.960000in}{3.696000in}}%
\pgfusepath{clip}%
\pgfsetbuttcap%
\pgfsetroundjoin%
\definecolor{currentfill}{rgb}{0.000000,0.000000,0.000000}%
\pgfsetfillcolor{currentfill}%
\pgfsetlinewidth{1.003750pt}%
\definecolor{currentstroke}{rgb}{0.000000,0.000000,0.000000}%
\pgfsetstrokecolor{currentstroke}%
\pgfsetdash{}{0pt}%
\pgfpathmoveto{\pgfqpoint{1.025906in}{0.698302in}}%
\pgfpathcurveto{\pgfqpoint{1.036956in}{0.698302in}}{\pgfqpoint{1.047555in}{0.702692in}}{\pgfqpoint{1.055369in}{0.710506in}}%
\pgfpathcurveto{\pgfqpoint{1.063182in}{0.718319in}}{\pgfqpoint{1.067573in}{0.728918in}}{\pgfqpoint{1.067573in}{0.739969in}}%
\pgfpathcurveto{\pgfqpoint{1.067573in}{0.751019in}}{\pgfqpoint{1.063182in}{0.761618in}}{\pgfqpoint{1.055369in}{0.769431in}}%
\pgfpathcurveto{\pgfqpoint{1.047555in}{0.777245in}}{\pgfqpoint{1.036956in}{0.781635in}}{\pgfqpoint{1.025906in}{0.781635in}}%
\pgfpathcurveto{\pgfqpoint{1.014856in}{0.781635in}}{\pgfqpoint{1.004257in}{0.777245in}}{\pgfqpoint{0.996443in}{0.769431in}}%
\pgfpathcurveto{\pgfqpoint{0.988630in}{0.761618in}}{\pgfqpoint{0.984239in}{0.751019in}}{\pgfqpoint{0.984239in}{0.739969in}}%
\pgfpathcurveto{\pgfqpoint{0.984239in}{0.728918in}}{\pgfqpoint{0.988630in}{0.718319in}}{\pgfqpoint{0.996443in}{0.710506in}}%
\pgfpathcurveto{\pgfqpoint{1.004257in}{0.702692in}}{\pgfqpoint{1.014856in}{0.698302in}}{\pgfqpoint{1.025906in}{0.698302in}}%
\pgfpathclose%
\pgfusepath{stroke,fill}%
\end{pgfscope}%
\begin{pgfscope}%
\pgfpathrectangle{\pgfqpoint{0.800000in}{0.528000in}}{\pgfqpoint{4.960000in}{3.696000in}}%
\pgfusepath{clip}%
\pgfsetbuttcap%
\pgfsetroundjoin%
\definecolor{currentfill}{rgb}{0.000000,0.000000,0.000000}%
\pgfsetfillcolor{currentfill}%
\pgfsetlinewidth{1.003750pt}%
\definecolor{currentstroke}{rgb}{0.000000,0.000000,0.000000}%
\pgfsetstrokecolor{currentstroke}%
\pgfsetdash{}{0pt}%
\pgfpathmoveto{\pgfqpoint{1.025906in}{0.698302in}}%
\pgfpathcurveto{\pgfqpoint{1.036956in}{0.698302in}}{\pgfqpoint{1.047555in}{0.702692in}}{\pgfqpoint{1.055369in}{0.710506in}}%
\pgfpathcurveto{\pgfqpoint{1.063182in}{0.718319in}}{\pgfqpoint{1.067573in}{0.728918in}}{\pgfqpoint{1.067573in}{0.739969in}}%
\pgfpathcurveto{\pgfqpoint{1.067573in}{0.751019in}}{\pgfqpoint{1.063182in}{0.761618in}}{\pgfqpoint{1.055369in}{0.769431in}}%
\pgfpathcurveto{\pgfqpoint{1.047555in}{0.777245in}}{\pgfqpoint{1.036956in}{0.781635in}}{\pgfqpoint{1.025906in}{0.781635in}}%
\pgfpathcurveto{\pgfqpoint{1.014856in}{0.781635in}}{\pgfqpoint{1.004257in}{0.777245in}}{\pgfqpoint{0.996443in}{0.769431in}}%
\pgfpathcurveto{\pgfqpoint{0.988630in}{0.761618in}}{\pgfqpoint{0.984239in}{0.751019in}}{\pgfqpoint{0.984239in}{0.739969in}}%
\pgfpathcurveto{\pgfqpoint{0.984239in}{0.728918in}}{\pgfqpoint{0.988630in}{0.718319in}}{\pgfqpoint{0.996443in}{0.710506in}}%
\pgfpathcurveto{\pgfqpoint{1.004257in}{0.702692in}}{\pgfqpoint{1.014856in}{0.698302in}}{\pgfqpoint{1.025906in}{0.698302in}}%
\pgfpathclose%
\pgfusepath{stroke,fill}%
\end{pgfscope}%
\begin{pgfscope}%
\pgfpathrectangle{\pgfqpoint{0.800000in}{0.528000in}}{\pgfqpoint{4.960000in}{3.696000in}}%
\pgfusepath{clip}%
\pgfsetbuttcap%
\pgfsetroundjoin%
\definecolor{currentfill}{rgb}{0.000000,0.000000,0.000000}%
\pgfsetfillcolor{currentfill}%
\pgfsetlinewidth{1.003750pt}%
\definecolor{currentstroke}{rgb}{0.000000,0.000000,0.000000}%
\pgfsetstrokecolor{currentstroke}%
\pgfsetdash{}{0pt}%
\pgfpathmoveto{\pgfqpoint{1.025906in}{0.676804in}}%
\pgfpathcurveto{\pgfqpoint{1.036956in}{0.676804in}}{\pgfqpoint{1.047555in}{0.681195in}}{\pgfqpoint{1.055369in}{0.689008in}}%
\pgfpathcurveto{\pgfqpoint{1.063182in}{0.696822in}}{\pgfqpoint{1.067573in}{0.707421in}}{\pgfqpoint{1.067573in}{0.718471in}}%
\pgfpathcurveto{\pgfqpoint{1.067573in}{0.729521in}}{\pgfqpoint{1.063182in}{0.740120in}}{\pgfqpoint{1.055369in}{0.747934in}}%
\pgfpathcurveto{\pgfqpoint{1.047555in}{0.755748in}}{\pgfqpoint{1.036956in}{0.760138in}}{\pgfqpoint{1.025906in}{0.760138in}}%
\pgfpathcurveto{\pgfqpoint{1.014856in}{0.760138in}}{\pgfqpoint{1.004257in}{0.755748in}}{\pgfqpoint{0.996443in}{0.747934in}}%
\pgfpathcurveto{\pgfqpoint{0.988630in}{0.740120in}}{\pgfqpoint{0.984239in}{0.729521in}}{\pgfqpoint{0.984239in}{0.718471in}}%
\pgfpathcurveto{\pgfqpoint{0.984239in}{0.707421in}}{\pgfqpoint{0.988630in}{0.696822in}}{\pgfqpoint{0.996443in}{0.689008in}}%
\pgfpathcurveto{\pgfqpoint{1.004257in}{0.681195in}}{\pgfqpoint{1.014856in}{0.676804in}}{\pgfqpoint{1.025906in}{0.676804in}}%
\pgfpathclose%
\pgfusepath{stroke,fill}%
\end{pgfscope}%
\begin{pgfscope}%
\pgfpathrectangle{\pgfqpoint{0.800000in}{0.528000in}}{\pgfqpoint{4.960000in}{3.696000in}}%
\pgfusepath{clip}%
\pgfsetbuttcap%
\pgfsetroundjoin%
\definecolor{currentfill}{rgb}{0.000000,0.000000,0.000000}%
\pgfsetfillcolor{currentfill}%
\pgfsetlinewidth{1.003750pt}%
\definecolor{currentstroke}{rgb}{0.000000,0.000000,0.000000}%
\pgfsetstrokecolor{currentstroke}%
\pgfsetdash{}{0pt}%
\pgfpathmoveto{\pgfqpoint{1.025906in}{0.655307in}}%
\pgfpathcurveto{\pgfqpoint{1.036956in}{0.655307in}}{\pgfqpoint{1.047555in}{0.659697in}}{\pgfqpoint{1.055369in}{0.667511in}}%
\pgfpathcurveto{\pgfqpoint{1.063182in}{0.675324in}}{\pgfqpoint{1.067573in}{0.685924in}}{\pgfqpoint{1.067573in}{0.696974in}}%
\pgfpathcurveto{\pgfqpoint{1.067573in}{0.708024in}}{\pgfqpoint{1.063182in}{0.718623in}}{\pgfqpoint{1.055369in}{0.726436in}}%
\pgfpathcurveto{\pgfqpoint{1.047555in}{0.734250in}}{\pgfqpoint{1.036956in}{0.738640in}}{\pgfqpoint{1.025906in}{0.738640in}}%
\pgfpathcurveto{\pgfqpoint{1.014856in}{0.738640in}}{\pgfqpoint{1.004257in}{0.734250in}}{\pgfqpoint{0.996443in}{0.726436in}}%
\pgfpathcurveto{\pgfqpoint{0.988630in}{0.718623in}}{\pgfqpoint{0.984239in}{0.708024in}}{\pgfqpoint{0.984239in}{0.696974in}}%
\pgfpathcurveto{\pgfqpoint{0.984239in}{0.685924in}}{\pgfqpoint{0.988630in}{0.675324in}}{\pgfqpoint{0.996443in}{0.667511in}}%
\pgfpathcurveto{\pgfqpoint{1.004257in}{0.659697in}}{\pgfqpoint{1.014856in}{0.655307in}}{\pgfqpoint{1.025906in}{0.655307in}}%
\pgfpathclose%
\pgfusepath{stroke,fill}%
\end{pgfscope}%
\begin{pgfscope}%
\pgfpathrectangle{\pgfqpoint{0.800000in}{0.528000in}}{\pgfqpoint{4.960000in}{3.696000in}}%
\pgfusepath{clip}%
\pgfsetbuttcap%
\pgfsetroundjoin%
\definecolor{currentfill}{rgb}{0.000000,0.000000,0.000000}%
\pgfsetfillcolor{currentfill}%
\pgfsetlinewidth{1.003750pt}%
\definecolor{currentstroke}{rgb}{0.000000,0.000000,0.000000}%
\pgfsetstrokecolor{currentstroke}%
\pgfsetdash{}{0pt}%
\pgfpathmoveto{\pgfqpoint{1.025906in}{0.698302in}}%
\pgfpathcurveto{\pgfqpoint{1.036956in}{0.698302in}}{\pgfqpoint{1.047555in}{0.702692in}}{\pgfqpoint{1.055369in}{0.710506in}}%
\pgfpathcurveto{\pgfqpoint{1.063182in}{0.718319in}}{\pgfqpoint{1.067573in}{0.728918in}}{\pgfqpoint{1.067573in}{0.739969in}}%
\pgfpathcurveto{\pgfqpoint{1.067573in}{0.751019in}}{\pgfqpoint{1.063182in}{0.761618in}}{\pgfqpoint{1.055369in}{0.769431in}}%
\pgfpathcurveto{\pgfqpoint{1.047555in}{0.777245in}}{\pgfqpoint{1.036956in}{0.781635in}}{\pgfqpoint{1.025906in}{0.781635in}}%
\pgfpathcurveto{\pgfqpoint{1.014856in}{0.781635in}}{\pgfqpoint{1.004257in}{0.777245in}}{\pgfqpoint{0.996443in}{0.769431in}}%
\pgfpathcurveto{\pgfqpoint{0.988630in}{0.761618in}}{\pgfqpoint{0.984239in}{0.751019in}}{\pgfqpoint{0.984239in}{0.739969in}}%
\pgfpathcurveto{\pgfqpoint{0.984239in}{0.728918in}}{\pgfqpoint{0.988630in}{0.718319in}}{\pgfqpoint{0.996443in}{0.710506in}}%
\pgfpathcurveto{\pgfqpoint{1.004257in}{0.702692in}}{\pgfqpoint{1.014856in}{0.698302in}}{\pgfqpoint{1.025906in}{0.698302in}}%
\pgfpathclose%
\pgfusepath{stroke,fill}%
\end{pgfscope}%
\begin{pgfscope}%
\pgfpathrectangle{\pgfqpoint{0.800000in}{0.528000in}}{\pgfqpoint{4.960000in}{3.696000in}}%
\pgfusepath{clip}%
\pgfsetbuttcap%
\pgfsetroundjoin%
\definecolor{currentfill}{rgb}{0.000000,0.000000,0.000000}%
\pgfsetfillcolor{currentfill}%
\pgfsetlinewidth{1.003750pt}%
\definecolor{currentstroke}{rgb}{0.000000,0.000000,0.000000}%
\pgfsetstrokecolor{currentstroke}%
\pgfsetdash{}{0pt}%
\pgfpathmoveto{\pgfqpoint{1.025906in}{0.719799in}}%
\pgfpathcurveto{\pgfqpoint{1.036956in}{0.719799in}}{\pgfqpoint{1.047555in}{0.724190in}}{\pgfqpoint{1.055369in}{0.732003in}}%
\pgfpathcurveto{\pgfqpoint{1.063182in}{0.739817in}}{\pgfqpoint{1.067573in}{0.750416in}}{\pgfqpoint{1.067573in}{0.761466in}}%
\pgfpathcurveto{\pgfqpoint{1.067573in}{0.772516in}}{\pgfqpoint{1.063182in}{0.783115in}}{\pgfqpoint{1.055369in}{0.790929in}}%
\pgfpathcurveto{\pgfqpoint{1.047555in}{0.798743in}}{\pgfqpoint{1.036956in}{0.803133in}}{\pgfqpoint{1.025906in}{0.803133in}}%
\pgfpathcurveto{\pgfqpoint{1.014856in}{0.803133in}}{\pgfqpoint{1.004257in}{0.798743in}}{\pgfqpoint{0.996443in}{0.790929in}}%
\pgfpathcurveto{\pgfqpoint{0.988630in}{0.783115in}}{\pgfqpoint{0.984239in}{0.772516in}}{\pgfqpoint{0.984239in}{0.761466in}}%
\pgfpathcurveto{\pgfqpoint{0.984239in}{0.750416in}}{\pgfqpoint{0.988630in}{0.739817in}}{\pgfqpoint{0.996443in}{0.732003in}}%
\pgfpathcurveto{\pgfqpoint{1.004257in}{0.724190in}}{\pgfqpoint{1.014856in}{0.719799in}}{\pgfqpoint{1.025906in}{0.719799in}}%
\pgfpathclose%
\pgfusepath{stroke,fill}%
\end{pgfscope}%
\begin{pgfscope}%
\pgfpathrectangle{\pgfqpoint{0.800000in}{0.528000in}}{\pgfqpoint{4.960000in}{3.696000in}}%
\pgfusepath{clip}%
\pgfsetbuttcap%
\pgfsetroundjoin%
\definecolor{currentfill}{rgb}{0.000000,0.000000,0.000000}%
\pgfsetfillcolor{currentfill}%
\pgfsetlinewidth{1.003750pt}%
\definecolor{currentstroke}{rgb}{0.000000,0.000000,0.000000}%
\pgfsetstrokecolor{currentstroke}%
\pgfsetdash{}{0pt}%
\pgfpathmoveto{\pgfqpoint{1.025906in}{0.676804in}}%
\pgfpathcurveto{\pgfqpoint{1.036956in}{0.676804in}}{\pgfqpoint{1.047555in}{0.681195in}}{\pgfqpoint{1.055369in}{0.689008in}}%
\pgfpathcurveto{\pgfqpoint{1.063182in}{0.696822in}}{\pgfqpoint{1.067573in}{0.707421in}}{\pgfqpoint{1.067573in}{0.718471in}}%
\pgfpathcurveto{\pgfqpoint{1.067573in}{0.729521in}}{\pgfqpoint{1.063182in}{0.740120in}}{\pgfqpoint{1.055369in}{0.747934in}}%
\pgfpathcurveto{\pgfqpoint{1.047555in}{0.755748in}}{\pgfqpoint{1.036956in}{0.760138in}}{\pgfqpoint{1.025906in}{0.760138in}}%
\pgfpathcurveto{\pgfqpoint{1.014856in}{0.760138in}}{\pgfqpoint{1.004257in}{0.755748in}}{\pgfqpoint{0.996443in}{0.747934in}}%
\pgfpathcurveto{\pgfqpoint{0.988630in}{0.740120in}}{\pgfqpoint{0.984239in}{0.729521in}}{\pgfqpoint{0.984239in}{0.718471in}}%
\pgfpathcurveto{\pgfqpoint{0.984239in}{0.707421in}}{\pgfqpoint{0.988630in}{0.696822in}}{\pgfqpoint{0.996443in}{0.689008in}}%
\pgfpathcurveto{\pgfqpoint{1.004257in}{0.681195in}}{\pgfqpoint{1.014856in}{0.676804in}}{\pgfqpoint{1.025906in}{0.676804in}}%
\pgfpathclose%
\pgfusepath{stroke,fill}%
\end{pgfscope}%
\begin{pgfscope}%
\pgfpathrectangle{\pgfqpoint{0.800000in}{0.528000in}}{\pgfqpoint{4.960000in}{3.696000in}}%
\pgfusepath{clip}%
\pgfsetbuttcap%
\pgfsetroundjoin%
\definecolor{currentfill}{rgb}{0.000000,0.000000,0.000000}%
\pgfsetfillcolor{currentfill}%
\pgfsetlinewidth{1.003750pt}%
\definecolor{currentstroke}{rgb}{0.000000,0.000000,0.000000}%
\pgfsetstrokecolor{currentstroke}%
\pgfsetdash{}{0pt}%
\pgfpathmoveto{\pgfqpoint{1.025906in}{0.676804in}}%
\pgfpathcurveto{\pgfqpoint{1.036956in}{0.676804in}}{\pgfqpoint{1.047555in}{0.681195in}}{\pgfqpoint{1.055369in}{0.689008in}}%
\pgfpathcurveto{\pgfqpoint{1.063182in}{0.696822in}}{\pgfqpoint{1.067573in}{0.707421in}}{\pgfqpoint{1.067573in}{0.718471in}}%
\pgfpathcurveto{\pgfqpoint{1.067573in}{0.729521in}}{\pgfqpoint{1.063182in}{0.740120in}}{\pgfqpoint{1.055369in}{0.747934in}}%
\pgfpathcurveto{\pgfqpoint{1.047555in}{0.755748in}}{\pgfqpoint{1.036956in}{0.760138in}}{\pgfqpoint{1.025906in}{0.760138in}}%
\pgfpathcurveto{\pgfqpoint{1.014856in}{0.760138in}}{\pgfqpoint{1.004257in}{0.755748in}}{\pgfqpoint{0.996443in}{0.747934in}}%
\pgfpathcurveto{\pgfqpoint{0.988630in}{0.740120in}}{\pgfqpoint{0.984239in}{0.729521in}}{\pgfqpoint{0.984239in}{0.718471in}}%
\pgfpathcurveto{\pgfqpoint{0.984239in}{0.707421in}}{\pgfqpoint{0.988630in}{0.696822in}}{\pgfqpoint{0.996443in}{0.689008in}}%
\pgfpathcurveto{\pgfqpoint{1.004257in}{0.681195in}}{\pgfqpoint{1.014856in}{0.676804in}}{\pgfqpoint{1.025906in}{0.676804in}}%
\pgfpathclose%
\pgfusepath{stroke,fill}%
\end{pgfscope}%
\begin{pgfscope}%
\pgfpathrectangle{\pgfqpoint{0.800000in}{0.528000in}}{\pgfqpoint{4.960000in}{3.696000in}}%
\pgfusepath{clip}%
\pgfsetbuttcap%
\pgfsetroundjoin%
\definecolor{currentfill}{rgb}{0.000000,0.000000,0.000000}%
\pgfsetfillcolor{currentfill}%
\pgfsetlinewidth{1.003750pt}%
\definecolor{currentstroke}{rgb}{0.000000,0.000000,0.000000}%
\pgfsetstrokecolor{currentstroke}%
\pgfsetdash{}{0pt}%
\pgfpathmoveto{\pgfqpoint{1.025906in}{0.655307in}}%
\pgfpathcurveto{\pgfqpoint{1.036956in}{0.655307in}}{\pgfqpoint{1.047555in}{0.659697in}}{\pgfqpoint{1.055369in}{0.667511in}}%
\pgfpathcurveto{\pgfqpoint{1.063182in}{0.675324in}}{\pgfqpoint{1.067573in}{0.685924in}}{\pgfqpoint{1.067573in}{0.696974in}}%
\pgfpathcurveto{\pgfqpoint{1.067573in}{0.708024in}}{\pgfqpoint{1.063182in}{0.718623in}}{\pgfqpoint{1.055369in}{0.726436in}}%
\pgfpathcurveto{\pgfqpoint{1.047555in}{0.734250in}}{\pgfqpoint{1.036956in}{0.738640in}}{\pgfqpoint{1.025906in}{0.738640in}}%
\pgfpathcurveto{\pgfqpoint{1.014856in}{0.738640in}}{\pgfqpoint{1.004257in}{0.734250in}}{\pgfqpoint{0.996443in}{0.726436in}}%
\pgfpathcurveto{\pgfqpoint{0.988630in}{0.718623in}}{\pgfqpoint{0.984239in}{0.708024in}}{\pgfqpoint{0.984239in}{0.696974in}}%
\pgfpathcurveto{\pgfqpoint{0.984239in}{0.685924in}}{\pgfqpoint{0.988630in}{0.675324in}}{\pgfqpoint{0.996443in}{0.667511in}}%
\pgfpathcurveto{\pgfqpoint{1.004257in}{0.659697in}}{\pgfqpoint{1.014856in}{0.655307in}}{\pgfqpoint{1.025906in}{0.655307in}}%
\pgfpathclose%
\pgfusepath{stroke,fill}%
\end{pgfscope}%
\begin{pgfscope}%
\pgfpathrectangle{\pgfqpoint{0.800000in}{0.528000in}}{\pgfqpoint{4.960000in}{3.696000in}}%
\pgfusepath{clip}%
\pgfsetbuttcap%
\pgfsetroundjoin%
\definecolor{currentfill}{rgb}{0.000000,0.000000,0.000000}%
\pgfsetfillcolor{currentfill}%
\pgfsetlinewidth{1.003750pt}%
\definecolor{currentstroke}{rgb}{0.000000,0.000000,0.000000}%
\pgfsetstrokecolor{currentstroke}%
\pgfsetdash{}{0pt}%
\pgfpathmoveto{\pgfqpoint{1.025906in}{0.698302in}}%
\pgfpathcurveto{\pgfqpoint{1.036956in}{0.698302in}}{\pgfqpoint{1.047555in}{0.702692in}}{\pgfqpoint{1.055369in}{0.710506in}}%
\pgfpathcurveto{\pgfqpoint{1.063182in}{0.718319in}}{\pgfqpoint{1.067573in}{0.728918in}}{\pgfqpoint{1.067573in}{0.739969in}}%
\pgfpathcurveto{\pgfqpoint{1.067573in}{0.751019in}}{\pgfqpoint{1.063182in}{0.761618in}}{\pgfqpoint{1.055369in}{0.769431in}}%
\pgfpathcurveto{\pgfqpoint{1.047555in}{0.777245in}}{\pgfqpoint{1.036956in}{0.781635in}}{\pgfqpoint{1.025906in}{0.781635in}}%
\pgfpathcurveto{\pgfqpoint{1.014856in}{0.781635in}}{\pgfqpoint{1.004257in}{0.777245in}}{\pgfqpoint{0.996443in}{0.769431in}}%
\pgfpathcurveto{\pgfqpoint{0.988630in}{0.761618in}}{\pgfqpoint{0.984239in}{0.751019in}}{\pgfqpoint{0.984239in}{0.739969in}}%
\pgfpathcurveto{\pgfqpoint{0.984239in}{0.728918in}}{\pgfqpoint{0.988630in}{0.718319in}}{\pgfqpoint{0.996443in}{0.710506in}}%
\pgfpathcurveto{\pgfqpoint{1.004257in}{0.702692in}}{\pgfqpoint{1.014856in}{0.698302in}}{\pgfqpoint{1.025906in}{0.698302in}}%
\pgfpathclose%
\pgfusepath{stroke,fill}%
\end{pgfscope}%
\begin{pgfscope}%
\pgfpathrectangle{\pgfqpoint{0.800000in}{0.528000in}}{\pgfqpoint{4.960000in}{3.696000in}}%
\pgfusepath{clip}%
\pgfsetbuttcap%
\pgfsetroundjoin%
\definecolor{currentfill}{rgb}{0.000000,0.000000,0.000000}%
\pgfsetfillcolor{currentfill}%
\pgfsetlinewidth{1.003750pt}%
\definecolor{currentstroke}{rgb}{0.000000,0.000000,0.000000}%
\pgfsetstrokecolor{currentstroke}%
\pgfsetdash{}{0pt}%
\pgfpathmoveto{\pgfqpoint{1.025906in}{0.719799in}}%
\pgfpathcurveto{\pgfqpoint{1.036956in}{0.719799in}}{\pgfqpoint{1.047555in}{0.724190in}}{\pgfqpoint{1.055369in}{0.732003in}}%
\pgfpathcurveto{\pgfqpoint{1.063182in}{0.739817in}}{\pgfqpoint{1.067573in}{0.750416in}}{\pgfqpoint{1.067573in}{0.761466in}}%
\pgfpathcurveto{\pgfqpoint{1.067573in}{0.772516in}}{\pgfqpoint{1.063182in}{0.783115in}}{\pgfqpoint{1.055369in}{0.790929in}}%
\pgfpathcurveto{\pgfqpoint{1.047555in}{0.798743in}}{\pgfqpoint{1.036956in}{0.803133in}}{\pgfqpoint{1.025906in}{0.803133in}}%
\pgfpathcurveto{\pgfqpoint{1.014856in}{0.803133in}}{\pgfqpoint{1.004257in}{0.798743in}}{\pgfqpoint{0.996443in}{0.790929in}}%
\pgfpathcurveto{\pgfqpoint{0.988630in}{0.783115in}}{\pgfqpoint{0.984239in}{0.772516in}}{\pgfqpoint{0.984239in}{0.761466in}}%
\pgfpathcurveto{\pgfqpoint{0.984239in}{0.750416in}}{\pgfqpoint{0.988630in}{0.739817in}}{\pgfqpoint{0.996443in}{0.732003in}}%
\pgfpathcurveto{\pgfqpoint{1.004257in}{0.724190in}}{\pgfqpoint{1.014856in}{0.719799in}}{\pgfqpoint{1.025906in}{0.719799in}}%
\pgfpathclose%
\pgfusepath{stroke,fill}%
\end{pgfscope}%
\begin{pgfscope}%
\pgfpathrectangle{\pgfqpoint{0.800000in}{0.528000in}}{\pgfqpoint{4.960000in}{3.696000in}}%
\pgfusepath{clip}%
\pgfsetbuttcap%
\pgfsetroundjoin%
\definecolor{currentfill}{rgb}{0.000000,0.000000,0.000000}%
\pgfsetfillcolor{currentfill}%
\pgfsetlinewidth{1.003750pt}%
\definecolor{currentstroke}{rgb}{0.000000,0.000000,0.000000}%
\pgfsetstrokecolor{currentstroke}%
\pgfsetdash{}{0pt}%
\pgfpathmoveto{\pgfqpoint{1.025906in}{0.698302in}}%
\pgfpathcurveto{\pgfqpoint{1.036956in}{0.698302in}}{\pgfqpoint{1.047555in}{0.702692in}}{\pgfqpoint{1.055369in}{0.710506in}}%
\pgfpathcurveto{\pgfqpoint{1.063182in}{0.718319in}}{\pgfqpoint{1.067573in}{0.728918in}}{\pgfqpoint{1.067573in}{0.739969in}}%
\pgfpathcurveto{\pgfqpoint{1.067573in}{0.751019in}}{\pgfqpoint{1.063182in}{0.761618in}}{\pgfqpoint{1.055369in}{0.769431in}}%
\pgfpathcurveto{\pgfqpoint{1.047555in}{0.777245in}}{\pgfqpoint{1.036956in}{0.781635in}}{\pgfqpoint{1.025906in}{0.781635in}}%
\pgfpathcurveto{\pgfqpoint{1.014856in}{0.781635in}}{\pgfqpoint{1.004257in}{0.777245in}}{\pgfqpoint{0.996443in}{0.769431in}}%
\pgfpathcurveto{\pgfqpoint{0.988630in}{0.761618in}}{\pgfqpoint{0.984239in}{0.751019in}}{\pgfqpoint{0.984239in}{0.739969in}}%
\pgfpathcurveto{\pgfqpoint{0.984239in}{0.728918in}}{\pgfqpoint{0.988630in}{0.718319in}}{\pgfqpoint{0.996443in}{0.710506in}}%
\pgfpathcurveto{\pgfqpoint{1.004257in}{0.702692in}}{\pgfqpoint{1.014856in}{0.698302in}}{\pgfqpoint{1.025906in}{0.698302in}}%
\pgfpathclose%
\pgfusepath{stroke,fill}%
\end{pgfscope}%
\begin{pgfscope}%
\pgfpathrectangle{\pgfqpoint{0.800000in}{0.528000in}}{\pgfqpoint{4.960000in}{3.696000in}}%
\pgfusepath{clip}%
\pgfsetbuttcap%
\pgfsetroundjoin%
\definecolor{currentfill}{rgb}{0.000000,0.000000,0.000000}%
\pgfsetfillcolor{currentfill}%
\pgfsetlinewidth{1.003750pt}%
\definecolor{currentstroke}{rgb}{0.000000,0.000000,0.000000}%
\pgfsetstrokecolor{currentstroke}%
\pgfsetdash{}{0pt}%
\pgfpathmoveto{\pgfqpoint{1.025906in}{0.719799in}}%
\pgfpathcurveto{\pgfqpoint{1.036956in}{0.719799in}}{\pgfqpoint{1.047555in}{0.724190in}}{\pgfqpoint{1.055369in}{0.732003in}}%
\pgfpathcurveto{\pgfqpoint{1.063182in}{0.739817in}}{\pgfqpoint{1.067573in}{0.750416in}}{\pgfqpoint{1.067573in}{0.761466in}}%
\pgfpathcurveto{\pgfqpoint{1.067573in}{0.772516in}}{\pgfqpoint{1.063182in}{0.783115in}}{\pgfqpoint{1.055369in}{0.790929in}}%
\pgfpathcurveto{\pgfqpoint{1.047555in}{0.798743in}}{\pgfqpoint{1.036956in}{0.803133in}}{\pgfqpoint{1.025906in}{0.803133in}}%
\pgfpathcurveto{\pgfqpoint{1.014856in}{0.803133in}}{\pgfqpoint{1.004257in}{0.798743in}}{\pgfqpoint{0.996443in}{0.790929in}}%
\pgfpathcurveto{\pgfqpoint{0.988630in}{0.783115in}}{\pgfqpoint{0.984239in}{0.772516in}}{\pgfqpoint{0.984239in}{0.761466in}}%
\pgfpathcurveto{\pgfqpoint{0.984239in}{0.750416in}}{\pgfqpoint{0.988630in}{0.739817in}}{\pgfqpoint{0.996443in}{0.732003in}}%
\pgfpathcurveto{\pgfqpoint{1.004257in}{0.724190in}}{\pgfqpoint{1.014856in}{0.719799in}}{\pgfqpoint{1.025906in}{0.719799in}}%
\pgfpathclose%
\pgfusepath{stroke,fill}%
\end{pgfscope}%
\begin{pgfscope}%
\pgfpathrectangle{\pgfqpoint{0.800000in}{0.528000in}}{\pgfqpoint{4.960000in}{3.696000in}}%
\pgfusepath{clip}%
\pgfsetbuttcap%
\pgfsetroundjoin%
\definecolor{currentfill}{rgb}{0.000000,0.000000,0.000000}%
\pgfsetfillcolor{currentfill}%
\pgfsetlinewidth{1.003750pt}%
\definecolor{currentstroke}{rgb}{0.000000,0.000000,0.000000}%
\pgfsetstrokecolor{currentstroke}%
\pgfsetdash{}{0pt}%
\pgfpathmoveto{\pgfqpoint{1.025906in}{0.719799in}}%
\pgfpathcurveto{\pgfqpoint{1.036956in}{0.719799in}}{\pgfqpoint{1.047555in}{0.724190in}}{\pgfqpoint{1.055369in}{0.732003in}}%
\pgfpathcurveto{\pgfqpoint{1.063182in}{0.739817in}}{\pgfqpoint{1.067573in}{0.750416in}}{\pgfqpoint{1.067573in}{0.761466in}}%
\pgfpathcurveto{\pgfqpoint{1.067573in}{0.772516in}}{\pgfqpoint{1.063182in}{0.783115in}}{\pgfqpoint{1.055369in}{0.790929in}}%
\pgfpathcurveto{\pgfqpoint{1.047555in}{0.798743in}}{\pgfqpoint{1.036956in}{0.803133in}}{\pgfqpoint{1.025906in}{0.803133in}}%
\pgfpathcurveto{\pgfqpoint{1.014856in}{0.803133in}}{\pgfqpoint{1.004257in}{0.798743in}}{\pgfqpoint{0.996443in}{0.790929in}}%
\pgfpathcurveto{\pgfqpoint{0.988630in}{0.783115in}}{\pgfqpoint{0.984239in}{0.772516in}}{\pgfqpoint{0.984239in}{0.761466in}}%
\pgfpathcurveto{\pgfqpoint{0.984239in}{0.750416in}}{\pgfqpoint{0.988630in}{0.739817in}}{\pgfqpoint{0.996443in}{0.732003in}}%
\pgfpathcurveto{\pgfqpoint{1.004257in}{0.724190in}}{\pgfqpoint{1.014856in}{0.719799in}}{\pgfqpoint{1.025906in}{0.719799in}}%
\pgfpathclose%
\pgfusepath{stroke,fill}%
\end{pgfscope}%
\begin{pgfscope}%
\pgfpathrectangle{\pgfqpoint{0.800000in}{0.528000in}}{\pgfqpoint{4.960000in}{3.696000in}}%
\pgfusepath{clip}%
\pgfsetbuttcap%
\pgfsetroundjoin%
\definecolor{currentfill}{rgb}{0.000000,0.000000,0.000000}%
\pgfsetfillcolor{currentfill}%
\pgfsetlinewidth{1.003750pt}%
\definecolor{currentstroke}{rgb}{0.000000,0.000000,0.000000}%
\pgfsetstrokecolor{currentstroke}%
\pgfsetdash{}{0pt}%
\pgfpathmoveto{\pgfqpoint{1.025906in}{0.698302in}}%
\pgfpathcurveto{\pgfqpoint{1.036956in}{0.698302in}}{\pgfqpoint{1.047555in}{0.702692in}}{\pgfqpoint{1.055369in}{0.710506in}}%
\pgfpathcurveto{\pgfqpoint{1.063182in}{0.718319in}}{\pgfqpoint{1.067573in}{0.728918in}}{\pgfqpoint{1.067573in}{0.739969in}}%
\pgfpathcurveto{\pgfqpoint{1.067573in}{0.751019in}}{\pgfqpoint{1.063182in}{0.761618in}}{\pgfqpoint{1.055369in}{0.769431in}}%
\pgfpathcurveto{\pgfqpoint{1.047555in}{0.777245in}}{\pgfqpoint{1.036956in}{0.781635in}}{\pgfqpoint{1.025906in}{0.781635in}}%
\pgfpathcurveto{\pgfqpoint{1.014856in}{0.781635in}}{\pgfqpoint{1.004257in}{0.777245in}}{\pgfqpoint{0.996443in}{0.769431in}}%
\pgfpathcurveto{\pgfqpoint{0.988630in}{0.761618in}}{\pgfqpoint{0.984239in}{0.751019in}}{\pgfqpoint{0.984239in}{0.739969in}}%
\pgfpathcurveto{\pgfqpoint{0.984239in}{0.728918in}}{\pgfqpoint{0.988630in}{0.718319in}}{\pgfqpoint{0.996443in}{0.710506in}}%
\pgfpathcurveto{\pgfqpoint{1.004257in}{0.702692in}}{\pgfqpoint{1.014856in}{0.698302in}}{\pgfqpoint{1.025906in}{0.698302in}}%
\pgfpathclose%
\pgfusepath{stroke,fill}%
\end{pgfscope}%
\begin{pgfscope}%
\pgfpathrectangle{\pgfqpoint{0.800000in}{0.528000in}}{\pgfqpoint{4.960000in}{3.696000in}}%
\pgfusepath{clip}%
\pgfsetbuttcap%
\pgfsetroundjoin%
\definecolor{currentfill}{rgb}{0.000000,0.000000,0.000000}%
\pgfsetfillcolor{currentfill}%
\pgfsetlinewidth{1.003750pt}%
\definecolor{currentstroke}{rgb}{0.000000,0.000000,0.000000}%
\pgfsetstrokecolor{currentstroke}%
\pgfsetdash{}{0pt}%
\pgfpathmoveto{\pgfqpoint{1.025906in}{0.698302in}}%
\pgfpathcurveto{\pgfqpoint{1.036956in}{0.698302in}}{\pgfqpoint{1.047555in}{0.702692in}}{\pgfqpoint{1.055369in}{0.710506in}}%
\pgfpathcurveto{\pgfqpoint{1.063182in}{0.718319in}}{\pgfqpoint{1.067573in}{0.728918in}}{\pgfqpoint{1.067573in}{0.739969in}}%
\pgfpathcurveto{\pgfqpoint{1.067573in}{0.751019in}}{\pgfqpoint{1.063182in}{0.761618in}}{\pgfqpoint{1.055369in}{0.769431in}}%
\pgfpathcurveto{\pgfqpoint{1.047555in}{0.777245in}}{\pgfqpoint{1.036956in}{0.781635in}}{\pgfqpoint{1.025906in}{0.781635in}}%
\pgfpathcurveto{\pgfqpoint{1.014856in}{0.781635in}}{\pgfqpoint{1.004257in}{0.777245in}}{\pgfqpoint{0.996443in}{0.769431in}}%
\pgfpathcurveto{\pgfqpoint{0.988630in}{0.761618in}}{\pgfqpoint{0.984239in}{0.751019in}}{\pgfqpoint{0.984239in}{0.739969in}}%
\pgfpathcurveto{\pgfqpoint{0.984239in}{0.728918in}}{\pgfqpoint{0.988630in}{0.718319in}}{\pgfqpoint{0.996443in}{0.710506in}}%
\pgfpathcurveto{\pgfqpoint{1.004257in}{0.702692in}}{\pgfqpoint{1.014856in}{0.698302in}}{\pgfqpoint{1.025906in}{0.698302in}}%
\pgfpathclose%
\pgfusepath{stroke,fill}%
\end{pgfscope}%
\begin{pgfscope}%
\pgfpathrectangle{\pgfqpoint{0.800000in}{0.528000in}}{\pgfqpoint{4.960000in}{3.696000in}}%
\pgfusepath{clip}%
\pgfsetbuttcap%
\pgfsetroundjoin%
\definecolor{currentfill}{rgb}{0.000000,0.000000,0.000000}%
\pgfsetfillcolor{currentfill}%
\pgfsetlinewidth{1.003750pt}%
\definecolor{currentstroke}{rgb}{0.000000,0.000000,0.000000}%
\pgfsetstrokecolor{currentstroke}%
\pgfsetdash{}{0pt}%
\pgfpathmoveto{\pgfqpoint{1.025906in}{0.698302in}}%
\pgfpathcurveto{\pgfqpoint{1.036956in}{0.698302in}}{\pgfqpoint{1.047555in}{0.702692in}}{\pgfqpoint{1.055369in}{0.710506in}}%
\pgfpathcurveto{\pgfqpoint{1.063182in}{0.718319in}}{\pgfqpoint{1.067573in}{0.728918in}}{\pgfqpoint{1.067573in}{0.739969in}}%
\pgfpathcurveto{\pgfqpoint{1.067573in}{0.751019in}}{\pgfqpoint{1.063182in}{0.761618in}}{\pgfqpoint{1.055369in}{0.769431in}}%
\pgfpathcurveto{\pgfqpoint{1.047555in}{0.777245in}}{\pgfqpoint{1.036956in}{0.781635in}}{\pgfqpoint{1.025906in}{0.781635in}}%
\pgfpathcurveto{\pgfqpoint{1.014856in}{0.781635in}}{\pgfqpoint{1.004257in}{0.777245in}}{\pgfqpoint{0.996443in}{0.769431in}}%
\pgfpathcurveto{\pgfqpoint{0.988630in}{0.761618in}}{\pgfqpoint{0.984239in}{0.751019in}}{\pgfqpoint{0.984239in}{0.739969in}}%
\pgfpathcurveto{\pgfqpoint{0.984239in}{0.728918in}}{\pgfqpoint{0.988630in}{0.718319in}}{\pgfqpoint{0.996443in}{0.710506in}}%
\pgfpathcurveto{\pgfqpoint{1.004257in}{0.702692in}}{\pgfqpoint{1.014856in}{0.698302in}}{\pgfqpoint{1.025906in}{0.698302in}}%
\pgfpathclose%
\pgfusepath{stroke,fill}%
\end{pgfscope}%
\begin{pgfscope}%
\pgfpathrectangle{\pgfqpoint{0.800000in}{0.528000in}}{\pgfqpoint{4.960000in}{3.696000in}}%
\pgfusepath{clip}%
\pgfsetbuttcap%
\pgfsetroundjoin%
\definecolor{currentfill}{rgb}{0.000000,0.000000,0.000000}%
\pgfsetfillcolor{currentfill}%
\pgfsetlinewidth{1.003750pt}%
\definecolor{currentstroke}{rgb}{0.000000,0.000000,0.000000}%
\pgfsetstrokecolor{currentstroke}%
\pgfsetdash{}{0pt}%
\pgfpathmoveto{\pgfqpoint{1.025906in}{0.676804in}}%
\pgfpathcurveto{\pgfqpoint{1.036956in}{0.676804in}}{\pgfqpoint{1.047555in}{0.681195in}}{\pgfqpoint{1.055369in}{0.689008in}}%
\pgfpathcurveto{\pgfqpoint{1.063182in}{0.696822in}}{\pgfqpoint{1.067573in}{0.707421in}}{\pgfqpoint{1.067573in}{0.718471in}}%
\pgfpathcurveto{\pgfqpoint{1.067573in}{0.729521in}}{\pgfqpoint{1.063182in}{0.740120in}}{\pgfqpoint{1.055369in}{0.747934in}}%
\pgfpathcurveto{\pgfqpoint{1.047555in}{0.755748in}}{\pgfqpoint{1.036956in}{0.760138in}}{\pgfqpoint{1.025906in}{0.760138in}}%
\pgfpathcurveto{\pgfqpoint{1.014856in}{0.760138in}}{\pgfqpoint{1.004257in}{0.755748in}}{\pgfqpoint{0.996443in}{0.747934in}}%
\pgfpathcurveto{\pgfqpoint{0.988630in}{0.740120in}}{\pgfqpoint{0.984239in}{0.729521in}}{\pgfqpoint{0.984239in}{0.718471in}}%
\pgfpathcurveto{\pgfqpoint{0.984239in}{0.707421in}}{\pgfqpoint{0.988630in}{0.696822in}}{\pgfqpoint{0.996443in}{0.689008in}}%
\pgfpathcurveto{\pgfqpoint{1.004257in}{0.681195in}}{\pgfqpoint{1.014856in}{0.676804in}}{\pgfqpoint{1.025906in}{0.676804in}}%
\pgfpathclose%
\pgfusepath{stroke,fill}%
\end{pgfscope}%
\begin{pgfscope}%
\pgfpathrectangle{\pgfqpoint{0.800000in}{0.528000in}}{\pgfqpoint{4.960000in}{3.696000in}}%
\pgfusepath{clip}%
\pgfsetbuttcap%
\pgfsetroundjoin%
\definecolor{currentfill}{rgb}{0.000000,0.000000,0.000000}%
\pgfsetfillcolor{currentfill}%
\pgfsetlinewidth{1.003750pt}%
\definecolor{currentstroke}{rgb}{0.000000,0.000000,0.000000}%
\pgfsetstrokecolor{currentstroke}%
\pgfsetdash{}{0pt}%
\pgfpathmoveto{\pgfqpoint{1.025906in}{0.655307in}}%
\pgfpathcurveto{\pgfqpoint{1.036956in}{0.655307in}}{\pgfqpoint{1.047555in}{0.659697in}}{\pgfqpoint{1.055369in}{0.667511in}}%
\pgfpathcurveto{\pgfqpoint{1.063182in}{0.675324in}}{\pgfqpoint{1.067573in}{0.685924in}}{\pgfqpoint{1.067573in}{0.696974in}}%
\pgfpathcurveto{\pgfqpoint{1.067573in}{0.708024in}}{\pgfqpoint{1.063182in}{0.718623in}}{\pgfqpoint{1.055369in}{0.726436in}}%
\pgfpathcurveto{\pgfqpoint{1.047555in}{0.734250in}}{\pgfqpoint{1.036956in}{0.738640in}}{\pgfqpoint{1.025906in}{0.738640in}}%
\pgfpathcurveto{\pgfqpoint{1.014856in}{0.738640in}}{\pgfqpoint{1.004257in}{0.734250in}}{\pgfqpoint{0.996443in}{0.726436in}}%
\pgfpathcurveto{\pgfqpoint{0.988630in}{0.718623in}}{\pgfqpoint{0.984239in}{0.708024in}}{\pgfqpoint{0.984239in}{0.696974in}}%
\pgfpathcurveto{\pgfqpoint{0.984239in}{0.685924in}}{\pgfqpoint{0.988630in}{0.675324in}}{\pgfqpoint{0.996443in}{0.667511in}}%
\pgfpathcurveto{\pgfqpoint{1.004257in}{0.659697in}}{\pgfqpoint{1.014856in}{0.655307in}}{\pgfqpoint{1.025906in}{0.655307in}}%
\pgfpathclose%
\pgfusepath{stroke,fill}%
\end{pgfscope}%
\begin{pgfscope}%
\pgfpathrectangle{\pgfqpoint{0.800000in}{0.528000in}}{\pgfqpoint{4.960000in}{3.696000in}}%
\pgfusepath{clip}%
\pgfsetbuttcap%
\pgfsetroundjoin%
\definecolor{currentfill}{rgb}{0.000000,0.000000,0.000000}%
\pgfsetfillcolor{currentfill}%
\pgfsetlinewidth{1.003750pt}%
\definecolor{currentstroke}{rgb}{0.000000,0.000000,0.000000}%
\pgfsetstrokecolor{currentstroke}%
\pgfsetdash{}{0pt}%
\pgfpathmoveto{\pgfqpoint{1.025906in}{0.676804in}}%
\pgfpathcurveto{\pgfqpoint{1.036956in}{0.676804in}}{\pgfqpoint{1.047555in}{0.681195in}}{\pgfqpoint{1.055369in}{0.689008in}}%
\pgfpathcurveto{\pgfqpoint{1.063182in}{0.696822in}}{\pgfqpoint{1.067573in}{0.707421in}}{\pgfqpoint{1.067573in}{0.718471in}}%
\pgfpathcurveto{\pgfqpoint{1.067573in}{0.729521in}}{\pgfqpoint{1.063182in}{0.740120in}}{\pgfqpoint{1.055369in}{0.747934in}}%
\pgfpathcurveto{\pgfqpoint{1.047555in}{0.755748in}}{\pgfqpoint{1.036956in}{0.760138in}}{\pgfqpoint{1.025906in}{0.760138in}}%
\pgfpathcurveto{\pgfqpoint{1.014856in}{0.760138in}}{\pgfqpoint{1.004257in}{0.755748in}}{\pgfqpoint{0.996443in}{0.747934in}}%
\pgfpathcurveto{\pgfqpoint{0.988630in}{0.740120in}}{\pgfqpoint{0.984239in}{0.729521in}}{\pgfqpoint{0.984239in}{0.718471in}}%
\pgfpathcurveto{\pgfqpoint{0.984239in}{0.707421in}}{\pgfqpoint{0.988630in}{0.696822in}}{\pgfqpoint{0.996443in}{0.689008in}}%
\pgfpathcurveto{\pgfqpoint{1.004257in}{0.681195in}}{\pgfqpoint{1.014856in}{0.676804in}}{\pgfqpoint{1.025906in}{0.676804in}}%
\pgfpathclose%
\pgfusepath{stroke,fill}%
\end{pgfscope}%
\begin{pgfscope}%
\pgfpathrectangle{\pgfqpoint{0.800000in}{0.528000in}}{\pgfqpoint{4.960000in}{3.696000in}}%
\pgfusepath{clip}%
\pgfsetbuttcap%
\pgfsetroundjoin%
\definecolor{currentfill}{rgb}{0.000000,0.000000,0.000000}%
\pgfsetfillcolor{currentfill}%
\pgfsetlinewidth{1.003750pt}%
\definecolor{currentstroke}{rgb}{0.000000,0.000000,0.000000}%
\pgfsetstrokecolor{currentstroke}%
\pgfsetdash{}{0pt}%
\pgfpathmoveto{\pgfqpoint{1.025906in}{0.655307in}}%
\pgfpathcurveto{\pgfqpoint{1.036956in}{0.655307in}}{\pgfqpoint{1.047555in}{0.659697in}}{\pgfqpoint{1.055369in}{0.667511in}}%
\pgfpathcurveto{\pgfqpoint{1.063182in}{0.675324in}}{\pgfqpoint{1.067573in}{0.685924in}}{\pgfqpoint{1.067573in}{0.696974in}}%
\pgfpathcurveto{\pgfqpoint{1.067573in}{0.708024in}}{\pgfqpoint{1.063182in}{0.718623in}}{\pgfqpoint{1.055369in}{0.726436in}}%
\pgfpathcurveto{\pgfqpoint{1.047555in}{0.734250in}}{\pgfqpoint{1.036956in}{0.738640in}}{\pgfqpoint{1.025906in}{0.738640in}}%
\pgfpathcurveto{\pgfqpoint{1.014856in}{0.738640in}}{\pgfqpoint{1.004257in}{0.734250in}}{\pgfqpoint{0.996443in}{0.726436in}}%
\pgfpathcurveto{\pgfqpoint{0.988630in}{0.718623in}}{\pgfqpoint{0.984239in}{0.708024in}}{\pgfqpoint{0.984239in}{0.696974in}}%
\pgfpathcurveto{\pgfqpoint{0.984239in}{0.685924in}}{\pgfqpoint{0.988630in}{0.675324in}}{\pgfqpoint{0.996443in}{0.667511in}}%
\pgfpathcurveto{\pgfqpoint{1.004257in}{0.659697in}}{\pgfqpoint{1.014856in}{0.655307in}}{\pgfqpoint{1.025906in}{0.655307in}}%
\pgfpathclose%
\pgfusepath{stroke,fill}%
\end{pgfscope}%
\begin{pgfscope}%
\pgfpathrectangle{\pgfqpoint{0.800000in}{0.528000in}}{\pgfqpoint{4.960000in}{3.696000in}}%
\pgfusepath{clip}%
\pgfsetbuttcap%
\pgfsetroundjoin%
\definecolor{currentfill}{rgb}{0.000000,0.000000,0.000000}%
\pgfsetfillcolor{currentfill}%
\pgfsetlinewidth{1.003750pt}%
\definecolor{currentstroke}{rgb}{0.000000,0.000000,0.000000}%
\pgfsetstrokecolor{currentstroke}%
\pgfsetdash{}{0pt}%
\pgfpathmoveto{\pgfqpoint{1.025906in}{0.676804in}}%
\pgfpathcurveto{\pgfqpoint{1.036956in}{0.676804in}}{\pgfqpoint{1.047555in}{0.681195in}}{\pgfqpoint{1.055369in}{0.689008in}}%
\pgfpathcurveto{\pgfqpoint{1.063182in}{0.696822in}}{\pgfqpoint{1.067573in}{0.707421in}}{\pgfqpoint{1.067573in}{0.718471in}}%
\pgfpathcurveto{\pgfqpoint{1.067573in}{0.729521in}}{\pgfqpoint{1.063182in}{0.740120in}}{\pgfqpoint{1.055369in}{0.747934in}}%
\pgfpathcurveto{\pgfqpoint{1.047555in}{0.755748in}}{\pgfqpoint{1.036956in}{0.760138in}}{\pgfqpoint{1.025906in}{0.760138in}}%
\pgfpathcurveto{\pgfqpoint{1.014856in}{0.760138in}}{\pgfqpoint{1.004257in}{0.755748in}}{\pgfqpoint{0.996443in}{0.747934in}}%
\pgfpathcurveto{\pgfqpoint{0.988630in}{0.740120in}}{\pgfqpoint{0.984239in}{0.729521in}}{\pgfqpoint{0.984239in}{0.718471in}}%
\pgfpathcurveto{\pgfqpoint{0.984239in}{0.707421in}}{\pgfqpoint{0.988630in}{0.696822in}}{\pgfqpoint{0.996443in}{0.689008in}}%
\pgfpathcurveto{\pgfqpoint{1.004257in}{0.681195in}}{\pgfqpoint{1.014856in}{0.676804in}}{\pgfqpoint{1.025906in}{0.676804in}}%
\pgfpathclose%
\pgfusepath{stroke,fill}%
\end{pgfscope}%
\begin{pgfscope}%
\pgfpathrectangle{\pgfqpoint{0.800000in}{0.528000in}}{\pgfqpoint{4.960000in}{3.696000in}}%
\pgfusepath{clip}%
\pgfsetbuttcap%
\pgfsetroundjoin%
\definecolor{currentfill}{rgb}{0.000000,0.000000,0.000000}%
\pgfsetfillcolor{currentfill}%
\pgfsetlinewidth{1.003750pt}%
\definecolor{currentstroke}{rgb}{0.000000,0.000000,0.000000}%
\pgfsetstrokecolor{currentstroke}%
\pgfsetdash{}{0pt}%
\pgfpathmoveto{\pgfqpoint{1.025906in}{0.719799in}}%
\pgfpathcurveto{\pgfqpoint{1.036956in}{0.719799in}}{\pgfqpoint{1.047555in}{0.724190in}}{\pgfqpoint{1.055369in}{0.732003in}}%
\pgfpathcurveto{\pgfqpoint{1.063182in}{0.739817in}}{\pgfqpoint{1.067573in}{0.750416in}}{\pgfqpoint{1.067573in}{0.761466in}}%
\pgfpathcurveto{\pgfqpoint{1.067573in}{0.772516in}}{\pgfqpoint{1.063182in}{0.783115in}}{\pgfqpoint{1.055369in}{0.790929in}}%
\pgfpathcurveto{\pgfqpoint{1.047555in}{0.798743in}}{\pgfqpoint{1.036956in}{0.803133in}}{\pgfqpoint{1.025906in}{0.803133in}}%
\pgfpathcurveto{\pgfqpoint{1.014856in}{0.803133in}}{\pgfqpoint{1.004257in}{0.798743in}}{\pgfqpoint{0.996443in}{0.790929in}}%
\pgfpathcurveto{\pgfqpoint{0.988630in}{0.783115in}}{\pgfqpoint{0.984239in}{0.772516in}}{\pgfqpoint{0.984239in}{0.761466in}}%
\pgfpathcurveto{\pgfqpoint{0.984239in}{0.750416in}}{\pgfqpoint{0.988630in}{0.739817in}}{\pgfqpoint{0.996443in}{0.732003in}}%
\pgfpathcurveto{\pgfqpoint{1.004257in}{0.724190in}}{\pgfqpoint{1.014856in}{0.719799in}}{\pgfqpoint{1.025906in}{0.719799in}}%
\pgfpathclose%
\pgfusepath{stroke,fill}%
\end{pgfscope}%
\begin{pgfscope}%
\pgfpathrectangle{\pgfqpoint{0.800000in}{0.528000in}}{\pgfqpoint{4.960000in}{3.696000in}}%
\pgfusepath{clip}%
\pgfsetbuttcap%
\pgfsetroundjoin%
\definecolor{currentfill}{rgb}{0.000000,0.000000,0.000000}%
\pgfsetfillcolor{currentfill}%
\pgfsetlinewidth{1.003750pt}%
\definecolor{currentstroke}{rgb}{0.000000,0.000000,0.000000}%
\pgfsetstrokecolor{currentstroke}%
\pgfsetdash{}{0pt}%
\pgfpathmoveto{\pgfqpoint{1.025906in}{0.719799in}}%
\pgfpathcurveto{\pgfqpoint{1.036956in}{0.719799in}}{\pgfqpoint{1.047555in}{0.724190in}}{\pgfqpoint{1.055369in}{0.732003in}}%
\pgfpathcurveto{\pgfqpoint{1.063182in}{0.739817in}}{\pgfqpoint{1.067573in}{0.750416in}}{\pgfqpoint{1.067573in}{0.761466in}}%
\pgfpathcurveto{\pgfqpoint{1.067573in}{0.772516in}}{\pgfqpoint{1.063182in}{0.783115in}}{\pgfqpoint{1.055369in}{0.790929in}}%
\pgfpathcurveto{\pgfqpoint{1.047555in}{0.798743in}}{\pgfqpoint{1.036956in}{0.803133in}}{\pgfqpoint{1.025906in}{0.803133in}}%
\pgfpathcurveto{\pgfqpoint{1.014856in}{0.803133in}}{\pgfqpoint{1.004257in}{0.798743in}}{\pgfqpoint{0.996443in}{0.790929in}}%
\pgfpathcurveto{\pgfqpoint{0.988630in}{0.783115in}}{\pgfqpoint{0.984239in}{0.772516in}}{\pgfqpoint{0.984239in}{0.761466in}}%
\pgfpathcurveto{\pgfqpoint{0.984239in}{0.750416in}}{\pgfqpoint{0.988630in}{0.739817in}}{\pgfqpoint{0.996443in}{0.732003in}}%
\pgfpathcurveto{\pgfqpoint{1.004257in}{0.724190in}}{\pgfqpoint{1.014856in}{0.719799in}}{\pgfqpoint{1.025906in}{0.719799in}}%
\pgfpathclose%
\pgfusepath{stroke,fill}%
\end{pgfscope}%
\begin{pgfscope}%
\pgfpathrectangle{\pgfqpoint{0.800000in}{0.528000in}}{\pgfqpoint{4.960000in}{3.696000in}}%
\pgfusepath{clip}%
\pgfsetbuttcap%
\pgfsetroundjoin%
\definecolor{currentfill}{rgb}{0.000000,0.000000,0.000000}%
\pgfsetfillcolor{currentfill}%
\pgfsetlinewidth{1.003750pt}%
\definecolor{currentstroke}{rgb}{0.000000,0.000000,0.000000}%
\pgfsetstrokecolor{currentstroke}%
\pgfsetdash{}{0pt}%
\pgfpathmoveto{\pgfqpoint{1.025906in}{0.676804in}}%
\pgfpathcurveto{\pgfqpoint{1.036956in}{0.676804in}}{\pgfqpoint{1.047555in}{0.681195in}}{\pgfqpoint{1.055369in}{0.689008in}}%
\pgfpathcurveto{\pgfqpoint{1.063182in}{0.696822in}}{\pgfqpoint{1.067573in}{0.707421in}}{\pgfqpoint{1.067573in}{0.718471in}}%
\pgfpathcurveto{\pgfqpoint{1.067573in}{0.729521in}}{\pgfqpoint{1.063182in}{0.740120in}}{\pgfqpoint{1.055369in}{0.747934in}}%
\pgfpathcurveto{\pgfqpoint{1.047555in}{0.755748in}}{\pgfqpoint{1.036956in}{0.760138in}}{\pgfqpoint{1.025906in}{0.760138in}}%
\pgfpathcurveto{\pgfqpoint{1.014856in}{0.760138in}}{\pgfqpoint{1.004257in}{0.755748in}}{\pgfqpoint{0.996443in}{0.747934in}}%
\pgfpathcurveto{\pgfqpoint{0.988630in}{0.740120in}}{\pgfqpoint{0.984239in}{0.729521in}}{\pgfqpoint{0.984239in}{0.718471in}}%
\pgfpathcurveto{\pgfqpoint{0.984239in}{0.707421in}}{\pgfqpoint{0.988630in}{0.696822in}}{\pgfqpoint{0.996443in}{0.689008in}}%
\pgfpathcurveto{\pgfqpoint{1.004257in}{0.681195in}}{\pgfqpoint{1.014856in}{0.676804in}}{\pgfqpoint{1.025906in}{0.676804in}}%
\pgfpathclose%
\pgfusepath{stroke,fill}%
\end{pgfscope}%
\begin{pgfscope}%
\pgfpathrectangle{\pgfqpoint{0.800000in}{0.528000in}}{\pgfqpoint{4.960000in}{3.696000in}}%
\pgfusepath{clip}%
\pgfsetbuttcap%
\pgfsetroundjoin%
\definecolor{currentfill}{rgb}{0.000000,0.000000,0.000000}%
\pgfsetfillcolor{currentfill}%
\pgfsetlinewidth{1.003750pt}%
\definecolor{currentstroke}{rgb}{0.000000,0.000000,0.000000}%
\pgfsetstrokecolor{currentstroke}%
\pgfsetdash{}{0pt}%
\pgfpathmoveto{\pgfqpoint{1.025906in}{0.719799in}}%
\pgfpathcurveto{\pgfqpoint{1.036956in}{0.719799in}}{\pgfqpoint{1.047555in}{0.724190in}}{\pgfqpoint{1.055369in}{0.732003in}}%
\pgfpathcurveto{\pgfqpoint{1.063182in}{0.739817in}}{\pgfqpoint{1.067573in}{0.750416in}}{\pgfqpoint{1.067573in}{0.761466in}}%
\pgfpathcurveto{\pgfqpoint{1.067573in}{0.772516in}}{\pgfqpoint{1.063182in}{0.783115in}}{\pgfqpoint{1.055369in}{0.790929in}}%
\pgfpathcurveto{\pgfqpoint{1.047555in}{0.798743in}}{\pgfqpoint{1.036956in}{0.803133in}}{\pgfqpoint{1.025906in}{0.803133in}}%
\pgfpathcurveto{\pgfqpoint{1.014856in}{0.803133in}}{\pgfqpoint{1.004257in}{0.798743in}}{\pgfqpoint{0.996443in}{0.790929in}}%
\pgfpathcurveto{\pgfqpoint{0.988630in}{0.783115in}}{\pgfqpoint{0.984239in}{0.772516in}}{\pgfqpoint{0.984239in}{0.761466in}}%
\pgfpathcurveto{\pgfqpoint{0.984239in}{0.750416in}}{\pgfqpoint{0.988630in}{0.739817in}}{\pgfqpoint{0.996443in}{0.732003in}}%
\pgfpathcurveto{\pgfqpoint{1.004257in}{0.724190in}}{\pgfqpoint{1.014856in}{0.719799in}}{\pgfqpoint{1.025906in}{0.719799in}}%
\pgfpathclose%
\pgfusepath{stroke,fill}%
\end{pgfscope}%
\begin{pgfscope}%
\pgfpathrectangle{\pgfqpoint{0.800000in}{0.528000in}}{\pgfqpoint{4.960000in}{3.696000in}}%
\pgfusepath{clip}%
\pgfsetbuttcap%
\pgfsetroundjoin%
\definecolor{currentfill}{rgb}{0.000000,0.000000,0.000000}%
\pgfsetfillcolor{currentfill}%
\pgfsetlinewidth{1.003750pt}%
\definecolor{currentstroke}{rgb}{0.000000,0.000000,0.000000}%
\pgfsetstrokecolor{currentstroke}%
\pgfsetdash{}{0pt}%
\pgfpathmoveto{\pgfqpoint{1.025906in}{0.719799in}}%
\pgfpathcurveto{\pgfqpoint{1.036956in}{0.719799in}}{\pgfqpoint{1.047555in}{0.724190in}}{\pgfqpoint{1.055369in}{0.732003in}}%
\pgfpathcurveto{\pgfqpoint{1.063182in}{0.739817in}}{\pgfqpoint{1.067573in}{0.750416in}}{\pgfqpoint{1.067573in}{0.761466in}}%
\pgfpathcurveto{\pgfqpoint{1.067573in}{0.772516in}}{\pgfqpoint{1.063182in}{0.783115in}}{\pgfqpoint{1.055369in}{0.790929in}}%
\pgfpathcurveto{\pgfqpoint{1.047555in}{0.798743in}}{\pgfqpoint{1.036956in}{0.803133in}}{\pgfqpoint{1.025906in}{0.803133in}}%
\pgfpathcurveto{\pgfqpoint{1.014856in}{0.803133in}}{\pgfqpoint{1.004257in}{0.798743in}}{\pgfqpoint{0.996443in}{0.790929in}}%
\pgfpathcurveto{\pgfqpoint{0.988630in}{0.783115in}}{\pgfqpoint{0.984239in}{0.772516in}}{\pgfqpoint{0.984239in}{0.761466in}}%
\pgfpathcurveto{\pgfqpoint{0.984239in}{0.750416in}}{\pgfqpoint{0.988630in}{0.739817in}}{\pgfqpoint{0.996443in}{0.732003in}}%
\pgfpathcurveto{\pgfqpoint{1.004257in}{0.724190in}}{\pgfqpoint{1.014856in}{0.719799in}}{\pgfqpoint{1.025906in}{0.719799in}}%
\pgfpathclose%
\pgfusepath{stroke,fill}%
\end{pgfscope}%
\begin{pgfscope}%
\pgfpathrectangle{\pgfqpoint{0.800000in}{0.528000in}}{\pgfqpoint{4.960000in}{3.696000in}}%
\pgfusepath{clip}%
\pgfsetbuttcap%
\pgfsetroundjoin%
\definecolor{currentfill}{rgb}{0.000000,0.000000,0.000000}%
\pgfsetfillcolor{currentfill}%
\pgfsetlinewidth{1.003750pt}%
\definecolor{currentstroke}{rgb}{0.000000,0.000000,0.000000}%
\pgfsetstrokecolor{currentstroke}%
\pgfsetdash{}{0pt}%
\pgfpathmoveto{\pgfqpoint{1.025906in}{0.698302in}}%
\pgfpathcurveto{\pgfqpoint{1.036956in}{0.698302in}}{\pgfqpoint{1.047555in}{0.702692in}}{\pgfqpoint{1.055369in}{0.710506in}}%
\pgfpathcurveto{\pgfqpoint{1.063182in}{0.718319in}}{\pgfqpoint{1.067573in}{0.728918in}}{\pgfqpoint{1.067573in}{0.739969in}}%
\pgfpathcurveto{\pgfqpoint{1.067573in}{0.751019in}}{\pgfqpoint{1.063182in}{0.761618in}}{\pgfqpoint{1.055369in}{0.769431in}}%
\pgfpathcurveto{\pgfqpoint{1.047555in}{0.777245in}}{\pgfqpoint{1.036956in}{0.781635in}}{\pgfqpoint{1.025906in}{0.781635in}}%
\pgfpathcurveto{\pgfqpoint{1.014856in}{0.781635in}}{\pgfqpoint{1.004257in}{0.777245in}}{\pgfqpoint{0.996443in}{0.769431in}}%
\pgfpathcurveto{\pgfqpoint{0.988630in}{0.761618in}}{\pgfqpoint{0.984239in}{0.751019in}}{\pgfqpoint{0.984239in}{0.739969in}}%
\pgfpathcurveto{\pgfqpoint{0.984239in}{0.728918in}}{\pgfqpoint{0.988630in}{0.718319in}}{\pgfqpoint{0.996443in}{0.710506in}}%
\pgfpathcurveto{\pgfqpoint{1.004257in}{0.702692in}}{\pgfqpoint{1.014856in}{0.698302in}}{\pgfqpoint{1.025906in}{0.698302in}}%
\pgfpathclose%
\pgfusepath{stroke,fill}%
\end{pgfscope}%
\begin{pgfscope}%
\pgfpathrectangle{\pgfqpoint{0.800000in}{0.528000in}}{\pgfqpoint{4.960000in}{3.696000in}}%
\pgfusepath{clip}%
\pgfsetbuttcap%
\pgfsetroundjoin%
\definecolor{currentfill}{rgb}{0.000000,0.000000,0.000000}%
\pgfsetfillcolor{currentfill}%
\pgfsetlinewidth{1.003750pt}%
\definecolor{currentstroke}{rgb}{0.000000,0.000000,0.000000}%
\pgfsetstrokecolor{currentstroke}%
\pgfsetdash{}{0pt}%
\pgfpathmoveto{\pgfqpoint{1.025906in}{0.676804in}}%
\pgfpathcurveto{\pgfqpoint{1.036956in}{0.676804in}}{\pgfqpoint{1.047555in}{0.681195in}}{\pgfqpoint{1.055369in}{0.689008in}}%
\pgfpathcurveto{\pgfqpoint{1.063182in}{0.696822in}}{\pgfqpoint{1.067573in}{0.707421in}}{\pgfqpoint{1.067573in}{0.718471in}}%
\pgfpathcurveto{\pgfqpoint{1.067573in}{0.729521in}}{\pgfqpoint{1.063182in}{0.740120in}}{\pgfqpoint{1.055369in}{0.747934in}}%
\pgfpathcurveto{\pgfqpoint{1.047555in}{0.755748in}}{\pgfqpoint{1.036956in}{0.760138in}}{\pgfqpoint{1.025906in}{0.760138in}}%
\pgfpathcurveto{\pgfqpoint{1.014856in}{0.760138in}}{\pgfqpoint{1.004257in}{0.755748in}}{\pgfqpoint{0.996443in}{0.747934in}}%
\pgfpathcurveto{\pgfqpoint{0.988630in}{0.740120in}}{\pgfqpoint{0.984239in}{0.729521in}}{\pgfqpoint{0.984239in}{0.718471in}}%
\pgfpathcurveto{\pgfqpoint{0.984239in}{0.707421in}}{\pgfqpoint{0.988630in}{0.696822in}}{\pgfqpoint{0.996443in}{0.689008in}}%
\pgfpathcurveto{\pgfqpoint{1.004257in}{0.681195in}}{\pgfqpoint{1.014856in}{0.676804in}}{\pgfqpoint{1.025906in}{0.676804in}}%
\pgfpathclose%
\pgfusepath{stroke,fill}%
\end{pgfscope}%
\begin{pgfscope}%
\pgfpathrectangle{\pgfqpoint{0.800000in}{0.528000in}}{\pgfqpoint{4.960000in}{3.696000in}}%
\pgfusepath{clip}%
\pgfsetbuttcap%
\pgfsetroundjoin%
\definecolor{currentfill}{rgb}{0.000000,0.000000,0.000000}%
\pgfsetfillcolor{currentfill}%
\pgfsetlinewidth{1.003750pt}%
\definecolor{currentstroke}{rgb}{0.000000,0.000000,0.000000}%
\pgfsetstrokecolor{currentstroke}%
\pgfsetdash{}{0pt}%
\pgfpathmoveto{\pgfqpoint{1.025906in}{0.676804in}}%
\pgfpathcurveto{\pgfqpoint{1.036956in}{0.676804in}}{\pgfqpoint{1.047555in}{0.681195in}}{\pgfqpoint{1.055369in}{0.689008in}}%
\pgfpathcurveto{\pgfqpoint{1.063182in}{0.696822in}}{\pgfqpoint{1.067573in}{0.707421in}}{\pgfqpoint{1.067573in}{0.718471in}}%
\pgfpathcurveto{\pgfqpoint{1.067573in}{0.729521in}}{\pgfqpoint{1.063182in}{0.740120in}}{\pgfqpoint{1.055369in}{0.747934in}}%
\pgfpathcurveto{\pgfqpoint{1.047555in}{0.755748in}}{\pgfqpoint{1.036956in}{0.760138in}}{\pgfqpoint{1.025906in}{0.760138in}}%
\pgfpathcurveto{\pgfqpoint{1.014856in}{0.760138in}}{\pgfqpoint{1.004257in}{0.755748in}}{\pgfqpoint{0.996443in}{0.747934in}}%
\pgfpathcurveto{\pgfqpoint{0.988630in}{0.740120in}}{\pgfqpoint{0.984239in}{0.729521in}}{\pgfqpoint{0.984239in}{0.718471in}}%
\pgfpathcurveto{\pgfqpoint{0.984239in}{0.707421in}}{\pgfqpoint{0.988630in}{0.696822in}}{\pgfqpoint{0.996443in}{0.689008in}}%
\pgfpathcurveto{\pgfqpoint{1.004257in}{0.681195in}}{\pgfqpoint{1.014856in}{0.676804in}}{\pgfqpoint{1.025906in}{0.676804in}}%
\pgfpathclose%
\pgfusepath{stroke,fill}%
\end{pgfscope}%
\begin{pgfscope}%
\pgfpathrectangle{\pgfqpoint{0.800000in}{0.528000in}}{\pgfqpoint{4.960000in}{3.696000in}}%
\pgfusepath{clip}%
\pgfsetbuttcap%
\pgfsetroundjoin%
\definecolor{currentfill}{rgb}{0.000000,0.000000,0.000000}%
\pgfsetfillcolor{currentfill}%
\pgfsetlinewidth{1.003750pt}%
\definecolor{currentstroke}{rgb}{0.000000,0.000000,0.000000}%
\pgfsetstrokecolor{currentstroke}%
\pgfsetdash{}{0pt}%
\pgfpathmoveto{\pgfqpoint{1.025906in}{0.676804in}}%
\pgfpathcurveto{\pgfqpoint{1.036956in}{0.676804in}}{\pgfqpoint{1.047555in}{0.681195in}}{\pgfqpoint{1.055369in}{0.689008in}}%
\pgfpathcurveto{\pgfqpoint{1.063182in}{0.696822in}}{\pgfqpoint{1.067573in}{0.707421in}}{\pgfqpoint{1.067573in}{0.718471in}}%
\pgfpathcurveto{\pgfqpoint{1.067573in}{0.729521in}}{\pgfqpoint{1.063182in}{0.740120in}}{\pgfqpoint{1.055369in}{0.747934in}}%
\pgfpathcurveto{\pgfqpoint{1.047555in}{0.755748in}}{\pgfqpoint{1.036956in}{0.760138in}}{\pgfqpoint{1.025906in}{0.760138in}}%
\pgfpathcurveto{\pgfqpoint{1.014856in}{0.760138in}}{\pgfqpoint{1.004257in}{0.755748in}}{\pgfqpoint{0.996443in}{0.747934in}}%
\pgfpathcurveto{\pgfqpoint{0.988630in}{0.740120in}}{\pgfqpoint{0.984239in}{0.729521in}}{\pgfqpoint{0.984239in}{0.718471in}}%
\pgfpathcurveto{\pgfqpoint{0.984239in}{0.707421in}}{\pgfqpoint{0.988630in}{0.696822in}}{\pgfqpoint{0.996443in}{0.689008in}}%
\pgfpathcurveto{\pgfqpoint{1.004257in}{0.681195in}}{\pgfqpoint{1.014856in}{0.676804in}}{\pgfqpoint{1.025906in}{0.676804in}}%
\pgfpathclose%
\pgfusepath{stroke,fill}%
\end{pgfscope}%
\begin{pgfscope}%
\pgfpathrectangle{\pgfqpoint{0.800000in}{0.528000in}}{\pgfqpoint{4.960000in}{3.696000in}}%
\pgfusepath{clip}%
\pgfsetbuttcap%
\pgfsetroundjoin%
\definecolor{currentfill}{rgb}{0.000000,0.000000,0.000000}%
\pgfsetfillcolor{currentfill}%
\pgfsetlinewidth{1.003750pt}%
\definecolor{currentstroke}{rgb}{0.000000,0.000000,0.000000}%
\pgfsetstrokecolor{currentstroke}%
\pgfsetdash{}{0pt}%
\pgfpathmoveto{\pgfqpoint{1.025906in}{0.698302in}}%
\pgfpathcurveto{\pgfqpoint{1.036956in}{0.698302in}}{\pgfqpoint{1.047555in}{0.702692in}}{\pgfqpoint{1.055369in}{0.710506in}}%
\pgfpathcurveto{\pgfqpoint{1.063182in}{0.718319in}}{\pgfqpoint{1.067573in}{0.728918in}}{\pgfqpoint{1.067573in}{0.739969in}}%
\pgfpathcurveto{\pgfqpoint{1.067573in}{0.751019in}}{\pgfqpoint{1.063182in}{0.761618in}}{\pgfqpoint{1.055369in}{0.769431in}}%
\pgfpathcurveto{\pgfqpoint{1.047555in}{0.777245in}}{\pgfqpoint{1.036956in}{0.781635in}}{\pgfqpoint{1.025906in}{0.781635in}}%
\pgfpathcurveto{\pgfqpoint{1.014856in}{0.781635in}}{\pgfqpoint{1.004257in}{0.777245in}}{\pgfqpoint{0.996443in}{0.769431in}}%
\pgfpathcurveto{\pgfqpoint{0.988630in}{0.761618in}}{\pgfqpoint{0.984239in}{0.751019in}}{\pgfqpoint{0.984239in}{0.739969in}}%
\pgfpathcurveto{\pgfqpoint{0.984239in}{0.728918in}}{\pgfqpoint{0.988630in}{0.718319in}}{\pgfqpoint{0.996443in}{0.710506in}}%
\pgfpathcurveto{\pgfqpoint{1.004257in}{0.702692in}}{\pgfqpoint{1.014856in}{0.698302in}}{\pgfqpoint{1.025906in}{0.698302in}}%
\pgfpathclose%
\pgfusepath{stroke,fill}%
\end{pgfscope}%
\begin{pgfscope}%
\pgfpathrectangle{\pgfqpoint{0.800000in}{0.528000in}}{\pgfqpoint{4.960000in}{3.696000in}}%
\pgfusepath{clip}%
\pgfsetbuttcap%
\pgfsetroundjoin%
\definecolor{currentfill}{rgb}{0.000000,0.000000,0.000000}%
\pgfsetfillcolor{currentfill}%
\pgfsetlinewidth{1.003750pt}%
\definecolor{currentstroke}{rgb}{0.000000,0.000000,0.000000}%
\pgfsetstrokecolor{currentstroke}%
\pgfsetdash{}{0pt}%
\pgfpathmoveto{\pgfqpoint{1.025906in}{0.698302in}}%
\pgfpathcurveto{\pgfqpoint{1.036956in}{0.698302in}}{\pgfqpoint{1.047555in}{0.702692in}}{\pgfqpoint{1.055369in}{0.710506in}}%
\pgfpathcurveto{\pgfqpoint{1.063182in}{0.718319in}}{\pgfqpoint{1.067573in}{0.728918in}}{\pgfqpoint{1.067573in}{0.739969in}}%
\pgfpathcurveto{\pgfqpoint{1.067573in}{0.751019in}}{\pgfqpoint{1.063182in}{0.761618in}}{\pgfqpoint{1.055369in}{0.769431in}}%
\pgfpathcurveto{\pgfqpoint{1.047555in}{0.777245in}}{\pgfqpoint{1.036956in}{0.781635in}}{\pgfqpoint{1.025906in}{0.781635in}}%
\pgfpathcurveto{\pgfqpoint{1.014856in}{0.781635in}}{\pgfqpoint{1.004257in}{0.777245in}}{\pgfqpoint{0.996443in}{0.769431in}}%
\pgfpathcurveto{\pgfqpoint{0.988630in}{0.761618in}}{\pgfqpoint{0.984239in}{0.751019in}}{\pgfqpoint{0.984239in}{0.739969in}}%
\pgfpathcurveto{\pgfqpoint{0.984239in}{0.728918in}}{\pgfqpoint{0.988630in}{0.718319in}}{\pgfqpoint{0.996443in}{0.710506in}}%
\pgfpathcurveto{\pgfqpoint{1.004257in}{0.702692in}}{\pgfqpoint{1.014856in}{0.698302in}}{\pgfqpoint{1.025906in}{0.698302in}}%
\pgfpathclose%
\pgfusepath{stroke,fill}%
\end{pgfscope}%
\begin{pgfscope}%
\pgfpathrectangle{\pgfqpoint{0.800000in}{0.528000in}}{\pgfqpoint{4.960000in}{3.696000in}}%
\pgfusepath{clip}%
\pgfsetbuttcap%
\pgfsetroundjoin%
\definecolor{currentfill}{rgb}{0.000000,0.000000,0.000000}%
\pgfsetfillcolor{currentfill}%
\pgfsetlinewidth{1.003750pt}%
\definecolor{currentstroke}{rgb}{0.000000,0.000000,0.000000}%
\pgfsetstrokecolor{currentstroke}%
\pgfsetdash{}{0pt}%
\pgfpathmoveto{\pgfqpoint{1.025906in}{0.719799in}}%
\pgfpathcurveto{\pgfqpoint{1.036956in}{0.719799in}}{\pgfqpoint{1.047555in}{0.724190in}}{\pgfqpoint{1.055369in}{0.732003in}}%
\pgfpathcurveto{\pgfqpoint{1.063182in}{0.739817in}}{\pgfqpoint{1.067573in}{0.750416in}}{\pgfqpoint{1.067573in}{0.761466in}}%
\pgfpathcurveto{\pgfqpoint{1.067573in}{0.772516in}}{\pgfqpoint{1.063182in}{0.783115in}}{\pgfqpoint{1.055369in}{0.790929in}}%
\pgfpathcurveto{\pgfqpoint{1.047555in}{0.798743in}}{\pgfqpoint{1.036956in}{0.803133in}}{\pgfqpoint{1.025906in}{0.803133in}}%
\pgfpathcurveto{\pgfqpoint{1.014856in}{0.803133in}}{\pgfqpoint{1.004257in}{0.798743in}}{\pgfqpoint{0.996443in}{0.790929in}}%
\pgfpathcurveto{\pgfqpoint{0.988630in}{0.783115in}}{\pgfqpoint{0.984239in}{0.772516in}}{\pgfqpoint{0.984239in}{0.761466in}}%
\pgfpathcurveto{\pgfqpoint{0.984239in}{0.750416in}}{\pgfqpoint{0.988630in}{0.739817in}}{\pgfqpoint{0.996443in}{0.732003in}}%
\pgfpathcurveto{\pgfqpoint{1.004257in}{0.724190in}}{\pgfqpoint{1.014856in}{0.719799in}}{\pgfqpoint{1.025906in}{0.719799in}}%
\pgfpathclose%
\pgfusepath{stroke,fill}%
\end{pgfscope}%
\begin{pgfscope}%
\pgfpathrectangle{\pgfqpoint{0.800000in}{0.528000in}}{\pgfqpoint{4.960000in}{3.696000in}}%
\pgfusepath{clip}%
\pgfsetbuttcap%
\pgfsetroundjoin%
\definecolor{currentfill}{rgb}{0.000000,0.000000,0.000000}%
\pgfsetfillcolor{currentfill}%
\pgfsetlinewidth{1.003750pt}%
\definecolor{currentstroke}{rgb}{0.000000,0.000000,0.000000}%
\pgfsetstrokecolor{currentstroke}%
\pgfsetdash{}{0pt}%
\pgfpathmoveto{\pgfqpoint{1.025906in}{0.698302in}}%
\pgfpathcurveto{\pgfqpoint{1.036956in}{0.698302in}}{\pgfqpoint{1.047555in}{0.702692in}}{\pgfqpoint{1.055369in}{0.710506in}}%
\pgfpathcurveto{\pgfqpoint{1.063182in}{0.718319in}}{\pgfqpoint{1.067573in}{0.728918in}}{\pgfqpoint{1.067573in}{0.739969in}}%
\pgfpathcurveto{\pgfqpoint{1.067573in}{0.751019in}}{\pgfqpoint{1.063182in}{0.761618in}}{\pgfqpoint{1.055369in}{0.769431in}}%
\pgfpathcurveto{\pgfqpoint{1.047555in}{0.777245in}}{\pgfqpoint{1.036956in}{0.781635in}}{\pgfqpoint{1.025906in}{0.781635in}}%
\pgfpathcurveto{\pgfqpoint{1.014856in}{0.781635in}}{\pgfqpoint{1.004257in}{0.777245in}}{\pgfqpoint{0.996443in}{0.769431in}}%
\pgfpathcurveto{\pgfqpoint{0.988630in}{0.761618in}}{\pgfqpoint{0.984239in}{0.751019in}}{\pgfqpoint{0.984239in}{0.739969in}}%
\pgfpathcurveto{\pgfqpoint{0.984239in}{0.728918in}}{\pgfqpoint{0.988630in}{0.718319in}}{\pgfqpoint{0.996443in}{0.710506in}}%
\pgfpathcurveto{\pgfqpoint{1.004257in}{0.702692in}}{\pgfqpoint{1.014856in}{0.698302in}}{\pgfqpoint{1.025906in}{0.698302in}}%
\pgfpathclose%
\pgfusepath{stroke,fill}%
\end{pgfscope}%
\begin{pgfscope}%
\pgfpathrectangle{\pgfqpoint{0.800000in}{0.528000in}}{\pgfqpoint{4.960000in}{3.696000in}}%
\pgfusepath{clip}%
\pgfsetbuttcap%
\pgfsetroundjoin%
\definecolor{currentfill}{rgb}{0.000000,0.000000,0.000000}%
\pgfsetfillcolor{currentfill}%
\pgfsetlinewidth{1.003750pt}%
\definecolor{currentstroke}{rgb}{0.000000,0.000000,0.000000}%
\pgfsetstrokecolor{currentstroke}%
\pgfsetdash{}{0pt}%
\pgfpathmoveto{\pgfqpoint{1.025906in}{0.719799in}}%
\pgfpathcurveto{\pgfqpoint{1.036956in}{0.719799in}}{\pgfqpoint{1.047555in}{0.724190in}}{\pgfqpoint{1.055369in}{0.732003in}}%
\pgfpathcurveto{\pgfqpoint{1.063182in}{0.739817in}}{\pgfqpoint{1.067573in}{0.750416in}}{\pgfqpoint{1.067573in}{0.761466in}}%
\pgfpathcurveto{\pgfqpoint{1.067573in}{0.772516in}}{\pgfqpoint{1.063182in}{0.783115in}}{\pgfqpoint{1.055369in}{0.790929in}}%
\pgfpathcurveto{\pgfqpoint{1.047555in}{0.798743in}}{\pgfqpoint{1.036956in}{0.803133in}}{\pgfqpoint{1.025906in}{0.803133in}}%
\pgfpathcurveto{\pgfqpoint{1.014856in}{0.803133in}}{\pgfqpoint{1.004257in}{0.798743in}}{\pgfqpoint{0.996443in}{0.790929in}}%
\pgfpathcurveto{\pgfqpoint{0.988630in}{0.783115in}}{\pgfqpoint{0.984239in}{0.772516in}}{\pgfqpoint{0.984239in}{0.761466in}}%
\pgfpathcurveto{\pgfqpoint{0.984239in}{0.750416in}}{\pgfqpoint{0.988630in}{0.739817in}}{\pgfqpoint{0.996443in}{0.732003in}}%
\pgfpathcurveto{\pgfqpoint{1.004257in}{0.724190in}}{\pgfqpoint{1.014856in}{0.719799in}}{\pgfqpoint{1.025906in}{0.719799in}}%
\pgfpathclose%
\pgfusepath{stroke,fill}%
\end{pgfscope}%
\begin{pgfscope}%
\pgfpathrectangle{\pgfqpoint{0.800000in}{0.528000in}}{\pgfqpoint{4.960000in}{3.696000in}}%
\pgfusepath{clip}%
\pgfsetbuttcap%
\pgfsetroundjoin%
\definecolor{currentfill}{rgb}{0.000000,0.000000,0.000000}%
\pgfsetfillcolor{currentfill}%
\pgfsetlinewidth{1.003750pt}%
\definecolor{currentstroke}{rgb}{0.000000,0.000000,0.000000}%
\pgfsetstrokecolor{currentstroke}%
\pgfsetdash{}{0pt}%
\pgfpathmoveto{\pgfqpoint{1.025906in}{0.676804in}}%
\pgfpathcurveto{\pgfqpoint{1.036956in}{0.676804in}}{\pgfqpoint{1.047555in}{0.681195in}}{\pgfqpoint{1.055369in}{0.689008in}}%
\pgfpathcurveto{\pgfqpoint{1.063182in}{0.696822in}}{\pgfqpoint{1.067573in}{0.707421in}}{\pgfqpoint{1.067573in}{0.718471in}}%
\pgfpathcurveto{\pgfqpoint{1.067573in}{0.729521in}}{\pgfqpoint{1.063182in}{0.740120in}}{\pgfqpoint{1.055369in}{0.747934in}}%
\pgfpathcurveto{\pgfqpoint{1.047555in}{0.755748in}}{\pgfqpoint{1.036956in}{0.760138in}}{\pgfqpoint{1.025906in}{0.760138in}}%
\pgfpathcurveto{\pgfqpoint{1.014856in}{0.760138in}}{\pgfqpoint{1.004257in}{0.755748in}}{\pgfqpoint{0.996443in}{0.747934in}}%
\pgfpathcurveto{\pgfqpoint{0.988630in}{0.740120in}}{\pgfqpoint{0.984239in}{0.729521in}}{\pgfqpoint{0.984239in}{0.718471in}}%
\pgfpathcurveto{\pgfqpoint{0.984239in}{0.707421in}}{\pgfqpoint{0.988630in}{0.696822in}}{\pgfqpoint{0.996443in}{0.689008in}}%
\pgfpathcurveto{\pgfqpoint{1.004257in}{0.681195in}}{\pgfqpoint{1.014856in}{0.676804in}}{\pgfqpoint{1.025906in}{0.676804in}}%
\pgfpathclose%
\pgfusepath{stroke,fill}%
\end{pgfscope}%
\begin{pgfscope}%
\pgfpathrectangle{\pgfqpoint{0.800000in}{0.528000in}}{\pgfqpoint{4.960000in}{3.696000in}}%
\pgfusepath{clip}%
\pgfsetbuttcap%
\pgfsetroundjoin%
\definecolor{currentfill}{rgb}{0.000000,0.000000,0.000000}%
\pgfsetfillcolor{currentfill}%
\pgfsetlinewidth{1.003750pt}%
\definecolor{currentstroke}{rgb}{0.000000,0.000000,0.000000}%
\pgfsetstrokecolor{currentstroke}%
\pgfsetdash{}{0pt}%
\pgfpathmoveto{\pgfqpoint{1.025906in}{0.719799in}}%
\pgfpathcurveto{\pgfqpoint{1.036956in}{0.719799in}}{\pgfqpoint{1.047555in}{0.724190in}}{\pgfqpoint{1.055369in}{0.732003in}}%
\pgfpathcurveto{\pgfqpoint{1.063182in}{0.739817in}}{\pgfqpoint{1.067573in}{0.750416in}}{\pgfqpoint{1.067573in}{0.761466in}}%
\pgfpathcurveto{\pgfqpoint{1.067573in}{0.772516in}}{\pgfqpoint{1.063182in}{0.783115in}}{\pgfqpoint{1.055369in}{0.790929in}}%
\pgfpathcurveto{\pgfqpoint{1.047555in}{0.798743in}}{\pgfqpoint{1.036956in}{0.803133in}}{\pgfqpoint{1.025906in}{0.803133in}}%
\pgfpathcurveto{\pgfqpoint{1.014856in}{0.803133in}}{\pgfqpoint{1.004257in}{0.798743in}}{\pgfqpoint{0.996443in}{0.790929in}}%
\pgfpathcurveto{\pgfqpoint{0.988630in}{0.783115in}}{\pgfqpoint{0.984239in}{0.772516in}}{\pgfqpoint{0.984239in}{0.761466in}}%
\pgfpathcurveto{\pgfqpoint{0.984239in}{0.750416in}}{\pgfqpoint{0.988630in}{0.739817in}}{\pgfqpoint{0.996443in}{0.732003in}}%
\pgfpathcurveto{\pgfqpoint{1.004257in}{0.724190in}}{\pgfqpoint{1.014856in}{0.719799in}}{\pgfqpoint{1.025906in}{0.719799in}}%
\pgfpathclose%
\pgfusepath{stroke,fill}%
\end{pgfscope}%
\begin{pgfscope}%
\pgfpathrectangle{\pgfqpoint{0.800000in}{0.528000in}}{\pgfqpoint{4.960000in}{3.696000in}}%
\pgfusepath{clip}%
\pgfsetbuttcap%
\pgfsetroundjoin%
\definecolor{currentfill}{rgb}{0.000000,0.000000,0.000000}%
\pgfsetfillcolor{currentfill}%
\pgfsetlinewidth{1.003750pt}%
\definecolor{currentstroke}{rgb}{0.000000,0.000000,0.000000}%
\pgfsetstrokecolor{currentstroke}%
\pgfsetdash{}{0pt}%
\pgfpathmoveto{\pgfqpoint{1.025906in}{0.719799in}}%
\pgfpathcurveto{\pgfqpoint{1.036956in}{0.719799in}}{\pgfqpoint{1.047555in}{0.724190in}}{\pgfqpoint{1.055369in}{0.732003in}}%
\pgfpathcurveto{\pgfqpoint{1.063182in}{0.739817in}}{\pgfqpoint{1.067573in}{0.750416in}}{\pgfqpoint{1.067573in}{0.761466in}}%
\pgfpathcurveto{\pgfqpoint{1.067573in}{0.772516in}}{\pgfqpoint{1.063182in}{0.783115in}}{\pgfqpoint{1.055369in}{0.790929in}}%
\pgfpathcurveto{\pgfqpoint{1.047555in}{0.798743in}}{\pgfqpoint{1.036956in}{0.803133in}}{\pgfqpoint{1.025906in}{0.803133in}}%
\pgfpathcurveto{\pgfqpoint{1.014856in}{0.803133in}}{\pgfqpoint{1.004257in}{0.798743in}}{\pgfqpoint{0.996443in}{0.790929in}}%
\pgfpathcurveto{\pgfqpoint{0.988630in}{0.783115in}}{\pgfqpoint{0.984239in}{0.772516in}}{\pgfqpoint{0.984239in}{0.761466in}}%
\pgfpathcurveto{\pgfqpoint{0.984239in}{0.750416in}}{\pgfqpoint{0.988630in}{0.739817in}}{\pgfqpoint{0.996443in}{0.732003in}}%
\pgfpathcurveto{\pgfqpoint{1.004257in}{0.724190in}}{\pgfqpoint{1.014856in}{0.719799in}}{\pgfqpoint{1.025906in}{0.719799in}}%
\pgfpathclose%
\pgfusepath{stroke,fill}%
\end{pgfscope}%
\begin{pgfscope}%
\pgfpathrectangle{\pgfqpoint{0.800000in}{0.528000in}}{\pgfqpoint{4.960000in}{3.696000in}}%
\pgfusepath{clip}%
\pgfsetbuttcap%
\pgfsetroundjoin%
\definecolor{currentfill}{rgb}{0.000000,0.000000,0.000000}%
\pgfsetfillcolor{currentfill}%
\pgfsetlinewidth{1.003750pt}%
\definecolor{currentstroke}{rgb}{0.000000,0.000000,0.000000}%
\pgfsetstrokecolor{currentstroke}%
\pgfsetdash{}{0pt}%
\pgfpathmoveto{\pgfqpoint{1.025906in}{0.719799in}}%
\pgfpathcurveto{\pgfqpoint{1.036956in}{0.719799in}}{\pgfqpoint{1.047555in}{0.724190in}}{\pgfqpoint{1.055369in}{0.732003in}}%
\pgfpathcurveto{\pgfqpoint{1.063182in}{0.739817in}}{\pgfqpoint{1.067573in}{0.750416in}}{\pgfqpoint{1.067573in}{0.761466in}}%
\pgfpathcurveto{\pgfqpoint{1.067573in}{0.772516in}}{\pgfqpoint{1.063182in}{0.783115in}}{\pgfqpoint{1.055369in}{0.790929in}}%
\pgfpathcurveto{\pgfqpoint{1.047555in}{0.798743in}}{\pgfqpoint{1.036956in}{0.803133in}}{\pgfqpoint{1.025906in}{0.803133in}}%
\pgfpathcurveto{\pgfqpoint{1.014856in}{0.803133in}}{\pgfqpoint{1.004257in}{0.798743in}}{\pgfqpoint{0.996443in}{0.790929in}}%
\pgfpathcurveto{\pgfqpoint{0.988630in}{0.783115in}}{\pgfqpoint{0.984239in}{0.772516in}}{\pgfqpoint{0.984239in}{0.761466in}}%
\pgfpathcurveto{\pgfqpoint{0.984239in}{0.750416in}}{\pgfqpoint{0.988630in}{0.739817in}}{\pgfqpoint{0.996443in}{0.732003in}}%
\pgfpathcurveto{\pgfqpoint{1.004257in}{0.724190in}}{\pgfqpoint{1.014856in}{0.719799in}}{\pgfqpoint{1.025906in}{0.719799in}}%
\pgfpathclose%
\pgfusepath{stroke,fill}%
\end{pgfscope}%
\begin{pgfscope}%
\pgfpathrectangle{\pgfqpoint{0.800000in}{0.528000in}}{\pgfqpoint{4.960000in}{3.696000in}}%
\pgfusepath{clip}%
\pgfsetbuttcap%
\pgfsetroundjoin%
\definecolor{currentfill}{rgb}{0.000000,0.000000,0.000000}%
\pgfsetfillcolor{currentfill}%
\pgfsetlinewidth{1.003750pt}%
\definecolor{currentstroke}{rgb}{0.000000,0.000000,0.000000}%
\pgfsetstrokecolor{currentstroke}%
\pgfsetdash{}{0pt}%
\pgfpathmoveto{\pgfqpoint{1.025906in}{0.698302in}}%
\pgfpathcurveto{\pgfqpoint{1.036956in}{0.698302in}}{\pgfqpoint{1.047555in}{0.702692in}}{\pgfqpoint{1.055369in}{0.710506in}}%
\pgfpathcurveto{\pgfqpoint{1.063182in}{0.718319in}}{\pgfqpoint{1.067573in}{0.728918in}}{\pgfqpoint{1.067573in}{0.739969in}}%
\pgfpathcurveto{\pgfqpoint{1.067573in}{0.751019in}}{\pgfqpoint{1.063182in}{0.761618in}}{\pgfqpoint{1.055369in}{0.769431in}}%
\pgfpathcurveto{\pgfqpoint{1.047555in}{0.777245in}}{\pgfqpoint{1.036956in}{0.781635in}}{\pgfqpoint{1.025906in}{0.781635in}}%
\pgfpathcurveto{\pgfqpoint{1.014856in}{0.781635in}}{\pgfqpoint{1.004257in}{0.777245in}}{\pgfqpoint{0.996443in}{0.769431in}}%
\pgfpathcurveto{\pgfqpoint{0.988630in}{0.761618in}}{\pgfqpoint{0.984239in}{0.751019in}}{\pgfqpoint{0.984239in}{0.739969in}}%
\pgfpathcurveto{\pgfqpoint{0.984239in}{0.728918in}}{\pgfqpoint{0.988630in}{0.718319in}}{\pgfqpoint{0.996443in}{0.710506in}}%
\pgfpathcurveto{\pgfqpoint{1.004257in}{0.702692in}}{\pgfqpoint{1.014856in}{0.698302in}}{\pgfqpoint{1.025906in}{0.698302in}}%
\pgfpathclose%
\pgfusepath{stroke,fill}%
\end{pgfscope}%
\begin{pgfscope}%
\pgfpathrectangle{\pgfqpoint{0.800000in}{0.528000in}}{\pgfqpoint{4.960000in}{3.696000in}}%
\pgfusepath{clip}%
\pgfsetbuttcap%
\pgfsetroundjoin%
\definecolor{currentfill}{rgb}{0.000000,0.000000,0.000000}%
\pgfsetfillcolor{currentfill}%
\pgfsetlinewidth{1.003750pt}%
\definecolor{currentstroke}{rgb}{0.000000,0.000000,0.000000}%
\pgfsetstrokecolor{currentstroke}%
\pgfsetdash{}{0pt}%
\pgfpathmoveto{\pgfqpoint{1.025906in}{0.719799in}}%
\pgfpathcurveto{\pgfqpoint{1.036956in}{0.719799in}}{\pgfqpoint{1.047555in}{0.724190in}}{\pgfqpoint{1.055369in}{0.732003in}}%
\pgfpathcurveto{\pgfqpoint{1.063182in}{0.739817in}}{\pgfqpoint{1.067573in}{0.750416in}}{\pgfqpoint{1.067573in}{0.761466in}}%
\pgfpathcurveto{\pgfqpoint{1.067573in}{0.772516in}}{\pgfqpoint{1.063182in}{0.783115in}}{\pgfqpoint{1.055369in}{0.790929in}}%
\pgfpathcurveto{\pgfqpoint{1.047555in}{0.798743in}}{\pgfqpoint{1.036956in}{0.803133in}}{\pgfqpoint{1.025906in}{0.803133in}}%
\pgfpathcurveto{\pgfqpoint{1.014856in}{0.803133in}}{\pgfqpoint{1.004257in}{0.798743in}}{\pgfqpoint{0.996443in}{0.790929in}}%
\pgfpathcurveto{\pgfqpoint{0.988630in}{0.783115in}}{\pgfqpoint{0.984239in}{0.772516in}}{\pgfqpoint{0.984239in}{0.761466in}}%
\pgfpathcurveto{\pgfqpoint{0.984239in}{0.750416in}}{\pgfqpoint{0.988630in}{0.739817in}}{\pgfqpoint{0.996443in}{0.732003in}}%
\pgfpathcurveto{\pgfqpoint{1.004257in}{0.724190in}}{\pgfqpoint{1.014856in}{0.719799in}}{\pgfqpoint{1.025906in}{0.719799in}}%
\pgfpathclose%
\pgfusepath{stroke,fill}%
\end{pgfscope}%
\begin{pgfscope}%
\pgfpathrectangle{\pgfqpoint{0.800000in}{0.528000in}}{\pgfqpoint{4.960000in}{3.696000in}}%
\pgfusepath{clip}%
\pgfsetbuttcap%
\pgfsetroundjoin%
\definecolor{currentfill}{rgb}{0.000000,0.000000,0.000000}%
\pgfsetfillcolor{currentfill}%
\pgfsetlinewidth{1.003750pt}%
\definecolor{currentstroke}{rgb}{0.000000,0.000000,0.000000}%
\pgfsetstrokecolor{currentstroke}%
\pgfsetdash{}{0pt}%
\pgfpathmoveto{\pgfqpoint{1.025906in}{0.676804in}}%
\pgfpathcurveto{\pgfqpoint{1.036956in}{0.676804in}}{\pgfqpoint{1.047555in}{0.681195in}}{\pgfqpoint{1.055369in}{0.689008in}}%
\pgfpathcurveto{\pgfqpoint{1.063182in}{0.696822in}}{\pgfqpoint{1.067573in}{0.707421in}}{\pgfqpoint{1.067573in}{0.718471in}}%
\pgfpathcurveto{\pgfqpoint{1.067573in}{0.729521in}}{\pgfqpoint{1.063182in}{0.740120in}}{\pgfqpoint{1.055369in}{0.747934in}}%
\pgfpathcurveto{\pgfqpoint{1.047555in}{0.755748in}}{\pgfqpoint{1.036956in}{0.760138in}}{\pgfqpoint{1.025906in}{0.760138in}}%
\pgfpathcurveto{\pgfqpoint{1.014856in}{0.760138in}}{\pgfqpoint{1.004257in}{0.755748in}}{\pgfqpoint{0.996443in}{0.747934in}}%
\pgfpathcurveto{\pgfqpoint{0.988630in}{0.740120in}}{\pgfqpoint{0.984239in}{0.729521in}}{\pgfqpoint{0.984239in}{0.718471in}}%
\pgfpathcurveto{\pgfqpoint{0.984239in}{0.707421in}}{\pgfqpoint{0.988630in}{0.696822in}}{\pgfqpoint{0.996443in}{0.689008in}}%
\pgfpathcurveto{\pgfqpoint{1.004257in}{0.681195in}}{\pgfqpoint{1.014856in}{0.676804in}}{\pgfqpoint{1.025906in}{0.676804in}}%
\pgfpathclose%
\pgfusepath{stroke,fill}%
\end{pgfscope}%
\begin{pgfscope}%
\pgfpathrectangle{\pgfqpoint{0.800000in}{0.528000in}}{\pgfqpoint{4.960000in}{3.696000in}}%
\pgfusepath{clip}%
\pgfsetbuttcap%
\pgfsetroundjoin%
\definecolor{currentfill}{rgb}{0.000000,0.000000,0.000000}%
\pgfsetfillcolor{currentfill}%
\pgfsetlinewidth{1.003750pt}%
\definecolor{currentstroke}{rgb}{0.000000,0.000000,0.000000}%
\pgfsetstrokecolor{currentstroke}%
\pgfsetdash{}{0pt}%
\pgfpathmoveto{\pgfqpoint{1.025906in}{0.676804in}}%
\pgfpathcurveto{\pgfqpoint{1.036956in}{0.676804in}}{\pgfqpoint{1.047555in}{0.681195in}}{\pgfqpoint{1.055369in}{0.689008in}}%
\pgfpathcurveto{\pgfqpoint{1.063182in}{0.696822in}}{\pgfqpoint{1.067573in}{0.707421in}}{\pgfqpoint{1.067573in}{0.718471in}}%
\pgfpathcurveto{\pgfqpoint{1.067573in}{0.729521in}}{\pgfqpoint{1.063182in}{0.740120in}}{\pgfqpoint{1.055369in}{0.747934in}}%
\pgfpathcurveto{\pgfqpoint{1.047555in}{0.755748in}}{\pgfqpoint{1.036956in}{0.760138in}}{\pgfqpoint{1.025906in}{0.760138in}}%
\pgfpathcurveto{\pgfqpoint{1.014856in}{0.760138in}}{\pgfqpoint{1.004257in}{0.755748in}}{\pgfqpoint{0.996443in}{0.747934in}}%
\pgfpathcurveto{\pgfqpoint{0.988630in}{0.740120in}}{\pgfqpoint{0.984239in}{0.729521in}}{\pgfqpoint{0.984239in}{0.718471in}}%
\pgfpathcurveto{\pgfqpoint{0.984239in}{0.707421in}}{\pgfqpoint{0.988630in}{0.696822in}}{\pgfqpoint{0.996443in}{0.689008in}}%
\pgfpathcurveto{\pgfqpoint{1.004257in}{0.681195in}}{\pgfqpoint{1.014856in}{0.676804in}}{\pgfqpoint{1.025906in}{0.676804in}}%
\pgfpathclose%
\pgfusepath{stroke,fill}%
\end{pgfscope}%
\begin{pgfscope}%
\pgfpathrectangle{\pgfqpoint{0.800000in}{0.528000in}}{\pgfqpoint{4.960000in}{3.696000in}}%
\pgfusepath{clip}%
\pgfsetbuttcap%
\pgfsetroundjoin%
\definecolor{currentfill}{rgb}{0.000000,0.000000,0.000000}%
\pgfsetfillcolor{currentfill}%
\pgfsetlinewidth{1.003750pt}%
\definecolor{currentstroke}{rgb}{0.000000,0.000000,0.000000}%
\pgfsetstrokecolor{currentstroke}%
\pgfsetdash{}{0pt}%
\pgfpathmoveto{\pgfqpoint{1.025906in}{0.676804in}}%
\pgfpathcurveto{\pgfqpoint{1.036956in}{0.676804in}}{\pgfqpoint{1.047555in}{0.681195in}}{\pgfqpoint{1.055369in}{0.689008in}}%
\pgfpathcurveto{\pgfqpoint{1.063182in}{0.696822in}}{\pgfqpoint{1.067573in}{0.707421in}}{\pgfqpoint{1.067573in}{0.718471in}}%
\pgfpathcurveto{\pgfqpoint{1.067573in}{0.729521in}}{\pgfqpoint{1.063182in}{0.740120in}}{\pgfqpoint{1.055369in}{0.747934in}}%
\pgfpathcurveto{\pgfqpoint{1.047555in}{0.755748in}}{\pgfqpoint{1.036956in}{0.760138in}}{\pgfqpoint{1.025906in}{0.760138in}}%
\pgfpathcurveto{\pgfqpoint{1.014856in}{0.760138in}}{\pgfqpoint{1.004257in}{0.755748in}}{\pgfqpoint{0.996443in}{0.747934in}}%
\pgfpathcurveto{\pgfqpoint{0.988630in}{0.740120in}}{\pgfqpoint{0.984239in}{0.729521in}}{\pgfqpoint{0.984239in}{0.718471in}}%
\pgfpathcurveto{\pgfqpoint{0.984239in}{0.707421in}}{\pgfqpoint{0.988630in}{0.696822in}}{\pgfqpoint{0.996443in}{0.689008in}}%
\pgfpathcurveto{\pgfqpoint{1.004257in}{0.681195in}}{\pgfqpoint{1.014856in}{0.676804in}}{\pgfqpoint{1.025906in}{0.676804in}}%
\pgfpathclose%
\pgfusepath{stroke,fill}%
\end{pgfscope}%
\begin{pgfscope}%
\pgfpathrectangle{\pgfqpoint{0.800000in}{0.528000in}}{\pgfqpoint{4.960000in}{3.696000in}}%
\pgfusepath{clip}%
\pgfsetbuttcap%
\pgfsetroundjoin%
\definecolor{currentfill}{rgb}{0.000000,0.000000,0.000000}%
\pgfsetfillcolor{currentfill}%
\pgfsetlinewidth{1.003750pt}%
\definecolor{currentstroke}{rgb}{0.000000,0.000000,0.000000}%
\pgfsetstrokecolor{currentstroke}%
\pgfsetdash{}{0pt}%
\pgfpathmoveto{\pgfqpoint{1.025906in}{0.676804in}}%
\pgfpathcurveto{\pgfqpoint{1.036956in}{0.676804in}}{\pgfqpoint{1.047555in}{0.681195in}}{\pgfqpoint{1.055369in}{0.689008in}}%
\pgfpathcurveto{\pgfqpoint{1.063182in}{0.696822in}}{\pgfqpoint{1.067573in}{0.707421in}}{\pgfqpoint{1.067573in}{0.718471in}}%
\pgfpathcurveto{\pgfqpoint{1.067573in}{0.729521in}}{\pgfqpoint{1.063182in}{0.740120in}}{\pgfqpoint{1.055369in}{0.747934in}}%
\pgfpathcurveto{\pgfqpoint{1.047555in}{0.755748in}}{\pgfqpoint{1.036956in}{0.760138in}}{\pgfqpoint{1.025906in}{0.760138in}}%
\pgfpathcurveto{\pgfqpoint{1.014856in}{0.760138in}}{\pgfqpoint{1.004257in}{0.755748in}}{\pgfqpoint{0.996443in}{0.747934in}}%
\pgfpathcurveto{\pgfqpoint{0.988630in}{0.740120in}}{\pgfqpoint{0.984239in}{0.729521in}}{\pgfqpoint{0.984239in}{0.718471in}}%
\pgfpathcurveto{\pgfqpoint{0.984239in}{0.707421in}}{\pgfqpoint{0.988630in}{0.696822in}}{\pgfqpoint{0.996443in}{0.689008in}}%
\pgfpathcurveto{\pgfqpoint{1.004257in}{0.681195in}}{\pgfqpoint{1.014856in}{0.676804in}}{\pgfqpoint{1.025906in}{0.676804in}}%
\pgfpathclose%
\pgfusepath{stroke,fill}%
\end{pgfscope}%
\begin{pgfscope}%
\pgfpathrectangle{\pgfqpoint{0.800000in}{0.528000in}}{\pgfqpoint{4.960000in}{3.696000in}}%
\pgfusepath{clip}%
\pgfsetbuttcap%
\pgfsetroundjoin%
\definecolor{currentfill}{rgb}{0.000000,0.000000,0.000000}%
\pgfsetfillcolor{currentfill}%
\pgfsetlinewidth{1.003750pt}%
\definecolor{currentstroke}{rgb}{0.000000,0.000000,0.000000}%
\pgfsetstrokecolor{currentstroke}%
\pgfsetdash{}{0pt}%
\pgfpathmoveto{\pgfqpoint{1.025906in}{0.698302in}}%
\pgfpathcurveto{\pgfqpoint{1.036956in}{0.698302in}}{\pgfqpoint{1.047555in}{0.702692in}}{\pgfqpoint{1.055369in}{0.710506in}}%
\pgfpathcurveto{\pgfqpoint{1.063182in}{0.718319in}}{\pgfqpoint{1.067573in}{0.728918in}}{\pgfqpoint{1.067573in}{0.739969in}}%
\pgfpathcurveto{\pgfqpoint{1.067573in}{0.751019in}}{\pgfqpoint{1.063182in}{0.761618in}}{\pgfqpoint{1.055369in}{0.769431in}}%
\pgfpathcurveto{\pgfqpoint{1.047555in}{0.777245in}}{\pgfqpoint{1.036956in}{0.781635in}}{\pgfqpoint{1.025906in}{0.781635in}}%
\pgfpathcurveto{\pgfqpoint{1.014856in}{0.781635in}}{\pgfqpoint{1.004257in}{0.777245in}}{\pgfqpoint{0.996443in}{0.769431in}}%
\pgfpathcurveto{\pgfqpoint{0.988630in}{0.761618in}}{\pgfqpoint{0.984239in}{0.751019in}}{\pgfqpoint{0.984239in}{0.739969in}}%
\pgfpathcurveto{\pgfqpoint{0.984239in}{0.728918in}}{\pgfqpoint{0.988630in}{0.718319in}}{\pgfqpoint{0.996443in}{0.710506in}}%
\pgfpathcurveto{\pgfqpoint{1.004257in}{0.702692in}}{\pgfqpoint{1.014856in}{0.698302in}}{\pgfqpoint{1.025906in}{0.698302in}}%
\pgfpathclose%
\pgfusepath{stroke,fill}%
\end{pgfscope}%
\begin{pgfscope}%
\pgfpathrectangle{\pgfqpoint{0.800000in}{0.528000in}}{\pgfqpoint{4.960000in}{3.696000in}}%
\pgfusepath{clip}%
\pgfsetbuttcap%
\pgfsetroundjoin%
\definecolor{currentfill}{rgb}{0.000000,0.000000,0.000000}%
\pgfsetfillcolor{currentfill}%
\pgfsetlinewidth{1.003750pt}%
\definecolor{currentstroke}{rgb}{0.000000,0.000000,0.000000}%
\pgfsetstrokecolor{currentstroke}%
\pgfsetdash{}{0pt}%
\pgfpathmoveto{\pgfqpoint{1.025906in}{0.698302in}}%
\pgfpathcurveto{\pgfqpoint{1.036956in}{0.698302in}}{\pgfqpoint{1.047555in}{0.702692in}}{\pgfqpoint{1.055369in}{0.710506in}}%
\pgfpathcurveto{\pgfqpoint{1.063182in}{0.718319in}}{\pgfqpoint{1.067573in}{0.728918in}}{\pgfqpoint{1.067573in}{0.739969in}}%
\pgfpathcurveto{\pgfqpoint{1.067573in}{0.751019in}}{\pgfqpoint{1.063182in}{0.761618in}}{\pgfqpoint{1.055369in}{0.769431in}}%
\pgfpathcurveto{\pgfqpoint{1.047555in}{0.777245in}}{\pgfqpoint{1.036956in}{0.781635in}}{\pgfqpoint{1.025906in}{0.781635in}}%
\pgfpathcurveto{\pgfqpoint{1.014856in}{0.781635in}}{\pgfqpoint{1.004257in}{0.777245in}}{\pgfqpoint{0.996443in}{0.769431in}}%
\pgfpathcurveto{\pgfqpoint{0.988630in}{0.761618in}}{\pgfqpoint{0.984239in}{0.751019in}}{\pgfqpoint{0.984239in}{0.739969in}}%
\pgfpathcurveto{\pgfqpoint{0.984239in}{0.728918in}}{\pgfqpoint{0.988630in}{0.718319in}}{\pgfqpoint{0.996443in}{0.710506in}}%
\pgfpathcurveto{\pgfqpoint{1.004257in}{0.702692in}}{\pgfqpoint{1.014856in}{0.698302in}}{\pgfqpoint{1.025906in}{0.698302in}}%
\pgfpathclose%
\pgfusepath{stroke,fill}%
\end{pgfscope}%
\begin{pgfscope}%
\pgfpathrectangle{\pgfqpoint{0.800000in}{0.528000in}}{\pgfqpoint{4.960000in}{3.696000in}}%
\pgfusepath{clip}%
\pgfsetbuttcap%
\pgfsetroundjoin%
\definecolor{currentfill}{rgb}{0.000000,0.000000,0.000000}%
\pgfsetfillcolor{currentfill}%
\pgfsetlinewidth{1.003750pt}%
\definecolor{currentstroke}{rgb}{0.000000,0.000000,0.000000}%
\pgfsetstrokecolor{currentstroke}%
\pgfsetdash{}{0pt}%
\pgfpathmoveto{\pgfqpoint{1.025906in}{0.719799in}}%
\pgfpathcurveto{\pgfqpoint{1.036956in}{0.719799in}}{\pgfqpoint{1.047555in}{0.724190in}}{\pgfqpoint{1.055369in}{0.732003in}}%
\pgfpathcurveto{\pgfqpoint{1.063182in}{0.739817in}}{\pgfqpoint{1.067573in}{0.750416in}}{\pgfqpoint{1.067573in}{0.761466in}}%
\pgfpathcurveto{\pgfqpoint{1.067573in}{0.772516in}}{\pgfqpoint{1.063182in}{0.783115in}}{\pgfqpoint{1.055369in}{0.790929in}}%
\pgfpathcurveto{\pgfqpoint{1.047555in}{0.798743in}}{\pgfqpoint{1.036956in}{0.803133in}}{\pgfqpoint{1.025906in}{0.803133in}}%
\pgfpathcurveto{\pgfqpoint{1.014856in}{0.803133in}}{\pgfqpoint{1.004257in}{0.798743in}}{\pgfqpoint{0.996443in}{0.790929in}}%
\pgfpathcurveto{\pgfqpoint{0.988630in}{0.783115in}}{\pgfqpoint{0.984239in}{0.772516in}}{\pgfqpoint{0.984239in}{0.761466in}}%
\pgfpathcurveto{\pgfqpoint{0.984239in}{0.750416in}}{\pgfqpoint{0.988630in}{0.739817in}}{\pgfqpoint{0.996443in}{0.732003in}}%
\pgfpathcurveto{\pgfqpoint{1.004257in}{0.724190in}}{\pgfqpoint{1.014856in}{0.719799in}}{\pgfqpoint{1.025906in}{0.719799in}}%
\pgfpathclose%
\pgfusepath{stroke,fill}%
\end{pgfscope}%
\begin{pgfscope}%
\pgfpathrectangle{\pgfqpoint{0.800000in}{0.528000in}}{\pgfqpoint{4.960000in}{3.696000in}}%
\pgfusepath{clip}%
\pgfsetbuttcap%
\pgfsetroundjoin%
\definecolor{currentfill}{rgb}{0.000000,0.000000,0.000000}%
\pgfsetfillcolor{currentfill}%
\pgfsetlinewidth{1.003750pt}%
\definecolor{currentstroke}{rgb}{0.000000,0.000000,0.000000}%
\pgfsetstrokecolor{currentstroke}%
\pgfsetdash{}{0pt}%
\pgfpathmoveto{\pgfqpoint{1.025906in}{0.676804in}}%
\pgfpathcurveto{\pgfqpoint{1.036956in}{0.676804in}}{\pgfqpoint{1.047555in}{0.681195in}}{\pgfqpoint{1.055369in}{0.689008in}}%
\pgfpathcurveto{\pgfqpoint{1.063182in}{0.696822in}}{\pgfqpoint{1.067573in}{0.707421in}}{\pgfqpoint{1.067573in}{0.718471in}}%
\pgfpathcurveto{\pgfqpoint{1.067573in}{0.729521in}}{\pgfqpoint{1.063182in}{0.740120in}}{\pgfqpoint{1.055369in}{0.747934in}}%
\pgfpathcurveto{\pgfqpoint{1.047555in}{0.755748in}}{\pgfqpoint{1.036956in}{0.760138in}}{\pgfqpoint{1.025906in}{0.760138in}}%
\pgfpathcurveto{\pgfqpoint{1.014856in}{0.760138in}}{\pgfqpoint{1.004257in}{0.755748in}}{\pgfqpoint{0.996443in}{0.747934in}}%
\pgfpathcurveto{\pgfqpoint{0.988630in}{0.740120in}}{\pgfqpoint{0.984239in}{0.729521in}}{\pgfqpoint{0.984239in}{0.718471in}}%
\pgfpathcurveto{\pgfqpoint{0.984239in}{0.707421in}}{\pgfqpoint{0.988630in}{0.696822in}}{\pgfqpoint{0.996443in}{0.689008in}}%
\pgfpathcurveto{\pgfqpoint{1.004257in}{0.681195in}}{\pgfqpoint{1.014856in}{0.676804in}}{\pgfqpoint{1.025906in}{0.676804in}}%
\pgfpathclose%
\pgfusepath{stroke,fill}%
\end{pgfscope}%
\begin{pgfscope}%
\pgfpathrectangle{\pgfqpoint{0.800000in}{0.528000in}}{\pgfqpoint{4.960000in}{3.696000in}}%
\pgfusepath{clip}%
\pgfsetbuttcap%
\pgfsetroundjoin%
\definecolor{currentfill}{rgb}{0.000000,0.000000,0.000000}%
\pgfsetfillcolor{currentfill}%
\pgfsetlinewidth{1.003750pt}%
\definecolor{currentstroke}{rgb}{0.000000,0.000000,0.000000}%
\pgfsetstrokecolor{currentstroke}%
\pgfsetdash{}{0pt}%
\pgfpathmoveto{\pgfqpoint{1.025906in}{0.762794in}}%
\pgfpathcurveto{\pgfqpoint{1.036956in}{0.762794in}}{\pgfqpoint{1.047555in}{0.767185in}}{\pgfqpoint{1.055369in}{0.774998in}}%
\pgfpathcurveto{\pgfqpoint{1.063182in}{0.782812in}}{\pgfqpoint{1.067573in}{0.793411in}}{\pgfqpoint{1.067573in}{0.804461in}}%
\pgfpathcurveto{\pgfqpoint{1.067573in}{0.815511in}}{\pgfqpoint{1.063182in}{0.826110in}}{\pgfqpoint{1.055369in}{0.833924in}}%
\pgfpathcurveto{\pgfqpoint{1.047555in}{0.841738in}}{\pgfqpoint{1.036956in}{0.846128in}}{\pgfqpoint{1.025906in}{0.846128in}}%
\pgfpathcurveto{\pgfqpoint{1.014856in}{0.846128in}}{\pgfqpoint{1.004257in}{0.841738in}}{\pgfqpoint{0.996443in}{0.833924in}}%
\pgfpathcurveto{\pgfqpoint{0.988630in}{0.826110in}}{\pgfqpoint{0.984239in}{0.815511in}}{\pgfqpoint{0.984239in}{0.804461in}}%
\pgfpathcurveto{\pgfqpoint{0.984239in}{0.793411in}}{\pgfqpoint{0.988630in}{0.782812in}}{\pgfqpoint{0.996443in}{0.774998in}}%
\pgfpathcurveto{\pgfqpoint{1.004257in}{0.767185in}}{\pgfqpoint{1.014856in}{0.762794in}}{\pgfqpoint{1.025906in}{0.762794in}}%
\pgfpathclose%
\pgfusepath{stroke,fill}%
\end{pgfscope}%
\begin{pgfscope}%
\pgfpathrectangle{\pgfqpoint{0.800000in}{0.528000in}}{\pgfqpoint{4.960000in}{3.696000in}}%
\pgfusepath{clip}%
\pgfsetbuttcap%
\pgfsetroundjoin%
\definecolor{currentfill}{rgb}{0.000000,0.000000,0.000000}%
\pgfsetfillcolor{currentfill}%
\pgfsetlinewidth{1.003750pt}%
\definecolor{currentstroke}{rgb}{0.000000,0.000000,0.000000}%
\pgfsetstrokecolor{currentstroke}%
\pgfsetdash{}{0pt}%
\pgfpathmoveto{\pgfqpoint{1.025906in}{0.698302in}}%
\pgfpathcurveto{\pgfqpoint{1.036956in}{0.698302in}}{\pgfqpoint{1.047555in}{0.702692in}}{\pgfqpoint{1.055369in}{0.710506in}}%
\pgfpathcurveto{\pgfqpoint{1.063182in}{0.718319in}}{\pgfqpoint{1.067573in}{0.728918in}}{\pgfqpoint{1.067573in}{0.739969in}}%
\pgfpathcurveto{\pgfqpoint{1.067573in}{0.751019in}}{\pgfqpoint{1.063182in}{0.761618in}}{\pgfqpoint{1.055369in}{0.769431in}}%
\pgfpathcurveto{\pgfqpoint{1.047555in}{0.777245in}}{\pgfqpoint{1.036956in}{0.781635in}}{\pgfqpoint{1.025906in}{0.781635in}}%
\pgfpathcurveto{\pgfqpoint{1.014856in}{0.781635in}}{\pgfqpoint{1.004257in}{0.777245in}}{\pgfqpoint{0.996443in}{0.769431in}}%
\pgfpathcurveto{\pgfqpoint{0.988630in}{0.761618in}}{\pgfqpoint{0.984239in}{0.751019in}}{\pgfqpoint{0.984239in}{0.739969in}}%
\pgfpathcurveto{\pgfqpoint{0.984239in}{0.728918in}}{\pgfqpoint{0.988630in}{0.718319in}}{\pgfqpoint{0.996443in}{0.710506in}}%
\pgfpathcurveto{\pgfqpoint{1.004257in}{0.702692in}}{\pgfqpoint{1.014856in}{0.698302in}}{\pgfqpoint{1.025906in}{0.698302in}}%
\pgfpathclose%
\pgfusepath{stroke,fill}%
\end{pgfscope}%
\begin{pgfscope}%
\pgfpathrectangle{\pgfqpoint{0.800000in}{0.528000in}}{\pgfqpoint{4.960000in}{3.696000in}}%
\pgfusepath{clip}%
\pgfsetbuttcap%
\pgfsetroundjoin%
\definecolor{currentfill}{rgb}{0.000000,0.000000,0.000000}%
\pgfsetfillcolor{currentfill}%
\pgfsetlinewidth{1.003750pt}%
\definecolor{currentstroke}{rgb}{0.000000,0.000000,0.000000}%
\pgfsetstrokecolor{currentstroke}%
\pgfsetdash{}{0pt}%
\pgfpathmoveto{\pgfqpoint{1.025906in}{0.698302in}}%
\pgfpathcurveto{\pgfqpoint{1.036956in}{0.698302in}}{\pgfqpoint{1.047555in}{0.702692in}}{\pgfqpoint{1.055369in}{0.710506in}}%
\pgfpathcurveto{\pgfqpoint{1.063182in}{0.718319in}}{\pgfqpoint{1.067573in}{0.728918in}}{\pgfqpoint{1.067573in}{0.739969in}}%
\pgfpathcurveto{\pgfqpoint{1.067573in}{0.751019in}}{\pgfqpoint{1.063182in}{0.761618in}}{\pgfqpoint{1.055369in}{0.769431in}}%
\pgfpathcurveto{\pgfqpoint{1.047555in}{0.777245in}}{\pgfqpoint{1.036956in}{0.781635in}}{\pgfqpoint{1.025906in}{0.781635in}}%
\pgfpathcurveto{\pgfqpoint{1.014856in}{0.781635in}}{\pgfqpoint{1.004257in}{0.777245in}}{\pgfqpoint{0.996443in}{0.769431in}}%
\pgfpathcurveto{\pgfqpoint{0.988630in}{0.761618in}}{\pgfqpoint{0.984239in}{0.751019in}}{\pgfqpoint{0.984239in}{0.739969in}}%
\pgfpathcurveto{\pgfqpoint{0.984239in}{0.728918in}}{\pgfqpoint{0.988630in}{0.718319in}}{\pgfqpoint{0.996443in}{0.710506in}}%
\pgfpathcurveto{\pgfqpoint{1.004257in}{0.702692in}}{\pgfqpoint{1.014856in}{0.698302in}}{\pgfqpoint{1.025906in}{0.698302in}}%
\pgfpathclose%
\pgfusepath{stroke,fill}%
\end{pgfscope}%
\begin{pgfscope}%
\pgfpathrectangle{\pgfqpoint{0.800000in}{0.528000in}}{\pgfqpoint{4.960000in}{3.696000in}}%
\pgfusepath{clip}%
\pgfsetbuttcap%
\pgfsetroundjoin%
\definecolor{currentfill}{rgb}{0.000000,0.000000,0.000000}%
\pgfsetfillcolor{currentfill}%
\pgfsetlinewidth{1.003750pt}%
\definecolor{currentstroke}{rgb}{0.000000,0.000000,0.000000}%
\pgfsetstrokecolor{currentstroke}%
\pgfsetdash{}{0pt}%
\pgfpathmoveto{\pgfqpoint{1.025906in}{0.676804in}}%
\pgfpathcurveto{\pgfqpoint{1.036956in}{0.676804in}}{\pgfqpoint{1.047555in}{0.681195in}}{\pgfqpoint{1.055369in}{0.689008in}}%
\pgfpathcurveto{\pgfqpoint{1.063182in}{0.696822in}}{\pgfqpoint{1.067573in}{0.707421in}}{\pgfqpoint{1.067573in}{0.718471in}}%
\pgfpathcurveto{\pgfqpoint{1.067573in}{0.729521in}}{\pgfqpoint{1.063182in}{0.740120in}}{\pgfqpoint{1.055369in}{0.747934in}}%
\pgfpathcurveto{\pgfqpoint{1.047555in}{0.755748in}}{\pgfqpoint{1.036956in}{0.760138in}}{\pgfqpoint{1.025906in}{0.760138in}}%
\pgfpathcurveto{\pgfqpoint{1.014856in}{0.760138in}}{\pgfqpoint{1.004257in}{0.755748in}}{\pgfqpoint{0.996443in}{0.747934in}}%
\pgfpathcurveto{\pgfqpoint{0.988630in}{0.740120in}}{\pgfqpoint{0.984239in}{0.729521in}}{\pgfqpoint{0.984239in}{0.718471in}}%
\pgfpathcurveto{\pgfqpoint{0.984239in}{0.707421in}}{\pgfqpoint{0.988630in}{0.696822in}}{\pgfqpoint{0.996443in}{0.689008in}}%
\pgfpathcurveto{\pgfqpoint{1.004257in}{0.681195in}}{\pgfqpoint{1.014856in}{0.676804in}}{\pgfqpoint{1.025906in}{0.676804in}}%
\pgfpathclose%
\pgfusepath{stroke,fill}%
\end{pgfscope}%
\begin{pgfscope}%
\pgfpathrectangle{\pgfqpoint{0.800000in}{0.528000in}}{\pgfqpoint{4.960000in}{3.696000in}}%
\pgfusepath{clip}%
\pgfsetbuttcap%
\pgfsetroundjoin%
\definecolor{currentfill}{rgb}{0.000000,0.000000,0.000000}%
\pgfsetfillcolor{currentfill}%
\pgfsetlinewidth{1.003750pt}%
\definecolor{currentstroke}{rgb}{0.000000,0.000000,0.000000}%
\pgfsetstrokecolor{currentstroke}%
\pgfsetdash{}{0pt}%
\pgfpathmoveto{\pgfqpoint{1.025906in}{0.719799in}}%
\pgfpathcurveto{\pgfqpoint{1.036956in}{0.719799in}}{\pgfqpoint{1.047555in}{0.724190in}}{\pgfqpoint{1.055369in}{0.732003in}}%
\pgfpathcurveto{\pgfqpoint{1.063182in}{0.739817in}}{\pgfqpoint{1.067573in}{0.750416in}}{\pgfqpoint{1.067573in}{0.761466in}}%
\pgfpathcurveto{\pgfqpoint{1.067573in}{0.772516in}}{\pgfqpoint{1.063182in}{0.783115in}}{\pgfqpoint{1.055369in}{0.790929in}}%
\pgfpathcurveto{\pgfqpoint{1.047555in}{0.798743in}}{\pgfqpoint{1.036956in}{0.803133in}}{\pgfqpoint{1.025906in}{0.803133in}}%
\pgfpathcurveto{\pgfqpoint{1.014856in}{0.803133in}}{\pgfqpoint{1.004257in}{0.798743in}}{\pgfqpoint{0.996443in}{0.790929in}}%
\pgfpathcurveto{\pgfqpoint{0.988630in}{0.783115in}}{\pgfqpoint{0.984239in}{0.772516in}}{\pgfqpoint{0.984239in}{0.761466in}}%
\pgfpathcurveto{\pgfqpoint{0.984239in}{0.750416in}}{\pgfqpoint{0.988630in}{0.739817in}}{\pgfqpoint{0.996443in}{0.732003in}}%
\pgfpathcurveto{\pgfqpoint{1.004257in}{0.724190in}}{\pgfqpoint{1.014856in}{0.719799in}}{\pgfqpoint{1.025906in}{0.719799in}}%
\pgfpathclose%
\pgfusepath{stroke,fill}%
\end{pgfscope}%
\begin{pgfscope}%
\pgfpathrectangle{\pgfqpoint{0.800000in}{0.528000in}}{\pgfqpoint{4.960000in}{3.696000in}}%
\pgfusepath{clip}%
\pgfsetbuttcap%
\pgfsetroundjoin%
\definecolor{currentfill}{rgb}{0.000000,0.000000,0.000000}%
\pgfsetfillcolor{currentfill}%
\pgfsetlinewidth{1.003750pt}%
\definecolor{currentstroke}{rgb}{0.000000,0.000000,0.000000}%
\pgfsetstrokecolor{currentstroke}%
\pgfsetdash{}{0pt}%
\pgfpathmoveto{\pgfqpoint{1.025906in}{0.719799in}}%
\pgfpathcurveto{\pgfqpoint{1.036956in}{0.719799in}}{\pgfqpoint{1.047555in}{0.724190in}}{\pgfqpoint{1.055369in}{0.732003in}}%
\pgfpathcurveto{\pgfqpoint{1.063182in}{0.739817in}}{\pgfqpoint{1.067573in}{0.750416in}}{\pgfqpoint{1.067573in}{0.761466in}}%
\pgfpathcurveto{\pgfqpoint{1.067573in}{0.772516in}}{\pgfqpoint{1.063182in}{0.783115in}}{\pgfqpoint{1.055369in}{0.790929in}}%
\pgfpathcurveto{\pgfqpoint{1.047555in}{0.798743in}}{\pgfqpoint{1.036956in}{0.803133in}}{\pgfqpoint{1.025906in}{0.803133in}}%
\pgfpathcurveto{\pgfqpoint{1.014856in}{0.803133in}}{\pgfqpoint{1.004257in}{0.798743in}}{\pgfqpoint{0.996443in}{0.790929in}}%
\pgfpathcurveto{\pgfqpoint{0.988630in}{0.783115in}}{\pgfqpoint{0.984239in}{0.772516in}}{\pgfqpoint{0.984239in}{0.761466in}}%
\pgfpathcurveto{\pgfqpoint{0.984239in}{0.750416in}}{\pgfqpoint{0.988630in}{0.739817in}}{\pgfqpoint{0.996443in}{0.732003in}}%
\pgfpathcurveto{\pgfqpoint{1.004257in}{0.724190in}}{\pgfqpoint{1.014856in}{0.719799in}}{\pgfqpoint{1.025906in}{0.719799in}}%
\pgfpathclose%
\pgfusepath{stroke,fill}%
\end{pgfscope}%
\begin{pgfscope}%
\pgfpathrectangle{\pgfqpoint{0.800000in}{0.528000in}}{\pgfqpoint{4.960000in}{3.696000in}}%
\pgfusepath{clip}%
\pgfsetbuttcap%
\pgfsetroundjoin%
\definecolor{currentfill}{rgb}{0.000000,0.000000,0.000000}%
\pgfsetfillcolor{currentfill}%
\pgfsetlinewidth{1.003750pt}%
\definecolor{currentstroke}{rgb}{0.000000,0.000000,0.000000}%
\pgfsetstrokecolor{currentstroke}%
\pgfsetdash{}{0pt}%
\pgfpathmoveto{\pgfqpoint{1.025906in}{0.698302in}}%
\pgfpathcurveto{\pgfqpoint{1.036956in}{0.698302in}}{\pgfqpoint{1.047555in}{0.702692in}}{\pgfqpoint{1.055369in}{0.710506in}}%
\pgfpathcurveto{\pgfqpoint{1.063182in}{0.718319in}}{\pgfqpoint{1.067573in}{0.728918in}}{\pgfqpoint{1.067573in}{0.739969in}}%
\pgfpathcurveto{\pgfqpoint{1.067573in}{0.751019in}}{\pgfqpoint{1.063182in}{0.761618in}}{\pgfqpoint{1.055369in}{0.769431in}}%
\pgfpathcurveto{\pgfqpoint{1.047555in}{0.777245in}}{\pgfqpoint{1.036956in}{0.781635in}}{\pgfqpoint{1.025906in}{0.781635in}}%
\pgfpathcurveto{\pgfqpoint{1.014856in}{0.781635in}}{\pgfqpoint{1.004257in}{0.777245in}}{\pgfqpoint{0.996443in}{0.769431in}}%
\pgfpathcurveto{\pgfqpoint{0.988630in}{0.761618in}}{\pgfqpoint{0.984239in}{0.751019in}}{\pgfqpoint{0.984239in}{0.739969in}}%
\pgfpathcurveto{\pgfqpoint{0.984239in}{0.728918in}}{\pgfqpoint{0.988630in}{0.718319in}}{\pgfqpoint{0.996443in}{0.710506in}}%
\pgfpathcurveto{\pgfqpoint{1.004257in}{0.702692in}}{\pgfqpoint{1.014856in}{0.698302in}}{\pgfqpoint{1.025906in}{0.698302in}}%
\pgfpathclose%
\pgfusepath{stroke,fill}%
\end{pgfscope}%
\begin{pgfscope}%
\pgfpathrectangle{\pgfqpoint{0.800000in}{0.528000in}}{\pgfqpoint{4.960000in}{3.696000in}}%
\pgfusepath{clip}%
\pgfsetbuttcap%
\pgfsetroundjoin%
\definecolor{currentfill}{rgb}{0.000000,0.000000,0.000000}%
\pgfsetfillcolor{currentfill}%
\pgfsetlinewidth{1.003750pt}%
\definecolor{currentstroke}{rgb}{0.000000,0.000000,0.000000}%
\pgfsetstrokecolor{currentstroke}%
\pgfsetdash{}{0pt}%
\pgfpathmoveto{\pgfqpoint{1.025906in}{0.719799in}}%
\pgfpathcurveto{\pgfqpoint{1.036956in}{0.719799in}}{\pgfqpoint{1.047555in}{0.724190in}}{\pgfqpoint{1.055369in}{0.732003in}}%
\pgfpathcurveto{\pgfqpoint{1.063182in}{0.739817in}}{\pgfqpoint{1.067573in}{0.750416in}}{\pgfqpoint{1.067573in}{0.761466in}}%
\pgfpathcurveto{\pgfqpoint{1.067573in}{0.772516in}}{\pgfqpoint{1.063182in}{0.783115in}}{\pgfqpoint{1.055369in}{0.790929in}}%
\pgfpathcurveto{\pgfqpoint{1.047555in}{0.798743in}}{\pgfqpoint{1.036956in}{0.803133in}}{\pgfqpoint{1.025906in}{0.803133in}}%
\pgfpathcurveto{\pgfqpoint{1.014856in}{0.803133in}}{\pgfqpoint{1.004257in}{0.798743in}}{\pgfqpoint{0.996443in}{0.790929in}}%
\pgfpathcurveto{\pgfqpoint{0.988630in}{0.783115in}}{\pgfqpoint{0.984239in}{0.772516in}}{\pgfqpoint{0.984239in}{0.761466in}}%
\pgfpathcurveto{\pgfqpoint{0.984239in}{0.750416in}}{\pgfqpoint{0.988630in}{0.739817in}}{\pgfqpoint{0.996443in}{0.732003in}}%
\pgfpathcurveto{\pgfqpoint{1.004257in}{0.724190in}}{\pgfqpoint{1.014856in}{0.719799in}}{\pgfqpoint{1.025906in}{0.719799in}}%
\pgfpathclose%
\pgfusepath{stroke,fill}%
\end{pgfscope}%
\begin{pgfscope}%
\pgfpathrectangle{\pgfqpoint{0.800000in}{0.528000in}}{\pgfqpoint{4.960000in}{3.696000in}}%
\pgfusepath{clip}%
\pgfsetbuttcap%
\pgfsetroundjoin%
\definecolor{currentfill}{rgb}{0.000000,0.000000,0.000000}%
\pgfsetfillcolor{currentfill}%
\pgfsetlinewidth{1.003750pt}%
\definecolor{currentstroke}{rgb}{0.000000,0.000000,0.000000}%
\pgfsetstrokecolor{currentstroke}%
\pgfsetdash{}{0pt}%
\pgfpathmoveto{\pgfqpoint{1.025906in}{0.676804in}}%
\pgfpathcurveto{\pgfqpoint{1.036956in}{0.676804in}}{\pgfqpoint{1.047555in}{0.681195in}}{\pgfqpoint{1.055369in}{0.689008in}}%
\pgfpathcurveto{\pgfqpoint{1.063182in}{0.696822in}}{\pgfqpoint{1.067573in}{0.707421in}}{\pgfqpoint{1.067573in}{0.718471in}}%
\pgfpathcurveto{\pgfqpoint{1.067573in}{0.729521in}}{\pgfqpoint{1.063182in}{0.740120in}}{\pgfqpoint{1.055369in}{0.747934in}}%
\pgfpathcurveto{\pgfqpoint{1.047555in}{0.755748in}}{\pgfqpoint{1.036956in}{0.760138in}}{\pgfqpoint{1.025906in}{0.760138in}}%
\pgfpathcurveto{\pgfqpoint{1.014856in}{0.760138in}}{\pgfqpoint{1.004257in}{0.755748in}}{\pgfqpoint{0.996443in}{0.747934in}}%
\pgfpathcurveto{\pgfqpoint{0.988630in}{0.740120in}}{\pgfqpoint{0.984239in}{0.729521in}}{\pgfqpoint{0.984239in}{0.718471in}}%
\pgfpathcurveto{\pgfqpoint{0.984239in}{0.707421in}}{\pgfqpoint{0.988630in}{0.696822in}}{\pgfqpoint{0.996443in}{0.689008in}}%
\pgfpathcurveto{\pgfqpoint{1.004257in}{0.681195in}}{\pgfqpoint{1.014856in}{0.676804in}}{\pgfqpoint{1.025906in}{0.676804in}}%
\pgfpathclose%
\pgfusepath{stroke,fill}%
\end{pgfscope}%
\begin{pgfscope}%
\pgfpathrectangle{\pgfqpoint{0.800000in}{0.528000in}}{\pgfqpoint{4.960000in}{3.696000in}}%
\pgfusepath{clip}%
\pgfsetbuttcap%
\pgfsetroundjoin%
\definecolor{currentfill}{rgb}{0.000000,0.000000,0.000000}%
\pgfsetfillcolor{currentfill}%
\pgfsetlinewidth{1.003750pt}%
\definecolor{currentstroke}{rgb}{0.000000,0.000000,0.000000}%
\pgfsetstrokecolor{currentstroke}%
\pgfsetdash{}{0pt}%
\pgfpathmoveto{\pgfqpoint{1.025906in}{0.698302in}}%
\pgfpathcurveto{\pgfqpoint{1.036956in}{0.698302in}}{\pgfqpoint{1.047555in}{0.702692in}}{\pgfqpoint{1.055369in}{0.710506in}}%
\pgfpathcurveto{\pgfqpoint{1.063182in}{0.718319in}}{\pgfqpoint{1.067573in}{0.728918in}}{\pgfqpoint{1.067573in}{0.739969in}}%
\pgfpathcurveto{\pgfqpoint{1.067573in}{0.751019in}}{\pgfqpoint{1.063182in}{0.761618in}}{\pgfqpoint{1.055369in}{0.769431in}}%
\pgfpathcurveto{\pgfqpoint{1.047555in}{0.777245in}}{\pgfqpoint{1.036956in}{0.781635in}}{\pgfqpoint{1.025906in}{0.781635in}}%
\pgfpathcurveto{\pgfqpoint{1.014856in}{0.781635in}}{\pgfqpoint{1.004257in}{0.777245in}}{\pgfqpoint{0.996443in}{0.769431in}}%
\pgfpathcurveto{\pgfqpoint{0.988630in}{0.761618in}}{\pgfqpoint{0.984239in}{0.751019in}}{\pgfqpoint{0.984239in}{0.739969in}}%
\pgfpathcurveto{\pgfqpoint{0.984239in}{0.728918in}}{\pgfqpoint{0.988630in}{0.718319in}}{\pgfqpoint{0.996443in}{0.710506in}}%
\pgfpathcurveto{\pgfqpoint{1.004257in}{0.702692in}}{\pgfqpoint{1.014856in}{0.698302in}}{\pgfqpoint{1.025906in}{0.698302in}}%
\pgfpathclose%
\pgfusepath{stroke,fill}%
\end{pgfscope}%
\begin{pgfscope}%
\pgfpathrectangle{\pgfqpoint{0.800000in}{0.528000in}}{\pgfqpoint{4.960000in}{3.696000in}}%
\pgfusepath{clip}%
\pgfsetbuttcap%
\pgfsetroundjoin%
\definecolor{currentfill}{rgb}{0.000000,0.000000,0.000000}%
\pgfsetfillcolor{currentfill}%
\pgfsetlinewidth{1.003750pt}%
\definecolor{currentstroke}{rgb}{0.000000,0.000000,0.000000}%
\pgfsetstrokecolor{currentstroke}%
\pgfsetdash{}{0pt}%
\pgfpathmoveto{\pgfqpoint{1.025906in}{0.698302in}}%
\pgfpathcurveto{\pgfqpoint{1.036956in}{0.698302in}}{\pgfqpoint{1.047555in}{0.702692in}}{\pgfqpoint{1.055369in}{0.710506in}}%
\pgfpathcurveto{\pgfqpoint{1.063182in}{0.718319in}}{\pgfqpoint{1.067573in}{0.728918in}}{\pgfqpoint{1.067573in}{0.739969in}}%
\pgfpathcurveto{\pgfqpoint{1.067573in}{0.751019in}}{\pgfqpoint{1.063182in}{0.761618in}}{\pgfqpoint{1.055369in}{0.769431in}}%
\pgfpathcurveto{\pgfqpoint{1.047555in}{0.777245in}}{\pgfqpoint{1.036956in}{0.781635in}}{\pgfqpoint{1.025906in}{0.781635in}}%
\pgfpathcurveto{\pgfqpoint{1.014856in}{0.781635in}}{\pgfqpoint{1.004257in}{0.777245in}}{\pgfqpoint{0.996443in}{0.769431in}}%
\pgfpathcurveto{\pgfqpoint{0.988630in}{0.761618in}}{\pgfqpoint{0.984239in}{0.751019in}}{\pgfqpoint{0.984239in}{0.739969in}}%
\pgfpathcurveto{\pgfqpoint{0.984239in}{0.728918in}}{\pgfqpoint{0.988630in}{0.718319in}}{\pgfqpoint{0.996443in}{0.710506in}}%
\pgfpathcurveto{\pgfqpoint{1.004257in}{0.702692in}}{\pgfqpoint{1.014856in}{0.698302in}}{\pgfqpoint{1.025906in}{0.698302in}}%
\pgfpathclose%
\pgfusepath{stroke,fill}%
\end{pgfscope}%
\begin{pgfscope}%
\pgfpathrectangle{\pgfqpoint{0.800000in}{0.528000in}}{\pgfqpoint{4.960000in}{3.696000in}}%
\pgfusepath{clip}%
\pgfsetbuttcap%
\pgfsetroundjoin%
\definecolor{currentfill}{rgb}{0.000000,0.000000,0.000000}%
\pgfsetfillcolor{currentfill}%
\pgfsetlinewidth{1.003750pt}%
\definecolor{currentstroke}{rgb}{0.000000,0.000000,0.000000}%
\pgfsetstrokecolor{currentstroke}%
\pgfsetdash{}{0pt}%
\pgfpathmoveto{\pgfqpoint{1.025906in}{0.698302in}}%
\pgfpathcurveto{\pgfqpoint{1.036956in}{0.698302in}}{\pgfqpoint{1.047555in}{0.702692in}}{\pgfqpoint{1.055369in}{0.710506in}}%
\pgfpathcurveto{\pgfqpoint{1.063182in}{0.718319in}}{\pgfqpoint{1.067573in}{0.728918in}}{\pgfqpoint{1.067573in}{0.739969in}}%
\pgfpathcurveto{\pgfqpoint{1.067573in}{0.751019in}}{\pgfqpoint{1.063182in}{0.761618in}}{\pgfqpoint{1.055369in}{0.769431in}}%
\pgfpathcurveto{\pgfqpoint{1.047555in}{0.777245in}}{\pgfqpoint{1.036956in}{0.781635in}}{\pgfqpoint{1.025906in}{0.781635in}}%
\pgfpathcurveto{\pgfqpoint{1.014856in}{0.781635in}}{\pgfqpoint{1.004257in}{0.777245in}}{\pgfqpoint{0.996443in}{0.769431in}}%
\pgfpathcurveto{\pgfqpoint{0.988630in}{0.761618in}}{\pgfqpoint{0.984239in}{0.751019in}}{\pgfqpoint{0.984239in}{0.739969in}}%
\pgfpathcurveto{\pgfqpoint{0.984239in}{0.728918in}}{\pgfqpoint{0.988630in}{0.718319in}}{\pgfqpoint{0.996443in}{0.710506in}}%
\pgfpathcurveto{\pgfqpoint{1.004257in}{0.702692in}}{\pgfqpoint{1.014856in}{0.698302in}}{\pgfqpoint{1.025906in}{0.698302in}}%
\pgfpathclose%
\pgfusepath{stroke,fill}%
\end{pgfscope}%
\begin{pgfscope}%
\pgfpathrectangle{\pgfqpoint{0.800000in}{0.528000in}}{\pgfqpoint{4.960000in}{3.696000in}}%
\pgfusepath{clip}%
\pgfsetbuttcap%
\pgfsetroundjoin%
\definecolor{currentfill}{rgb}{0.000000,0.000000,0.000000}%
\pgfsetfillcolor{currentfill}%
\pgfsetlinewidth{1.003750pt}%
\definecolor{currentstroke}{rgb}{0.000000,0.000000,0.000000}%
\pgfsetstrokecolor{currentstroke}%
\pgfsetdash{}{0pt}%
\pgfpathmoveto{\pgfqpoint{1.025906in}{0.719799in}}%
\pgfpathcurveto{\pgfqpoint{1.036956in}{0.719799in}}{\pgfqpoint{1.047555in}{0.724190in}}{\pgfqpoint{1.055369in}{0.732003in}}%
\pgfpathcurveto{\pgfqpoint{1.063182in}{0.739817in}}{\pgfqpoint{1.067573in}{0.750416in}}{\pgfqpoint{1.067573in}{0.761466in}}%
\pgfpathcurveto{\pgfqpoint{1.067573in}{0.772516in}}{\pgfqpoint{1.063182in}{0.783115in}}{\pgfqpoint{1.055369in}{0.790929in}}%
\pgfpathcurveto{\pgfqpoint{1.047555in}{0.798743in}}{\pgfqpoint{1.036956in}{0.803133in}}{\pgfqpoint{1.025906in}{0.803133in}}%
\pgfpathcurveto{\pgfqpoint{1.014856in}{0.803133in}}{\pgfqpoint{1.004257in}{0.798743in}}{\pgfqpoint{0.996443in}{0.790929in}}%
\pgfpathcurveto{\pgfqpoint{0.988630in}{0.783115in}}{\pgfqpoint{0.984239in}{0.772516in}}{\pgfqpoint{0.984239in}{0.761466in}}%
\pgfpathcurveto{\pgfqpoint{0.984239in}{0.750416in}}{\pgfqpoint{0.988630in}{0.739817in}}{\pgfqpoint{0.996443in}{0.732003in}}%
\pgfpathcurveto{\pgfqpoint{1.004257in}{0.724190in}}{\pgfqpoint{1.014856in}{0.719799in}}{\pgfqpoint{1.025906in}{0.719799in}}%
\pgfpathclose%
\pgfusepath{stroke,fill}%
\end{pgfscope}%
\begin{pgfscope}%
\pgfpathrectangle{\pgfqpoint{0.800000in}{0.528000in}}{\pgfqpoint{4.960000in}{3.696000in}}%
\pgfusepath{clip}%
\pgfsetbuttcap%
\pgfsetroundjoin%
\definecolor{currentfill}{rgb}{0.000000,0.000000,0.000000}%
\pgfsetfillcolor{currentfill}%
\pgfsetlinewidth{1.003750pt}%
\definecolor{currentstroke}{rgb}{0.000000,0.000000,0.000000}%
\pgfsetstrokecolor{currentstroke}%
\pgfsetdash{}{0pt}%
\pgfpathmoveto{\pgfqpoint{1.025906in}{0.698302in}}%
\pgfpathcurveto{\pgfqpoint{1.036956in}{0.698302in}}{\pgfqpoint{1.047555in}{0.702692in}}{\pgfqpoint{1.055369in}{0.710506in}}%
\pgfpathcurveto{\pgfqpoint{1.063182in}{0.718319in}}{\pgfqpoint{1.067573in}{0.728918in}}{\pgfqpoint{1.067573in}{0.739969in}}%
\pgfpathcurveto{\pgfqpoint{1.067573in}{0.751019in}}{\pgfqpoint{1.063182in}{0.761618in}}{\pgfqpoint{1.055369in}{0.769431in}}%
\pgfpathcurveto{\pgfqpoint{1.047555in}{0.777245in}}{\pgfqpoint{1.036956in}{0.781635in}}{\pgfqpoint{1.025906in}{0.781635in}}%
\pgfpathcurveto{\pgfqpoint{1.014856in}{0.781635in}}{\pgfqpoint{1.004257in}{0.777245in}}{\pgfqpoint{0.996443in}{0.769431in}}%
\pgfpathcurveto{\pgfqpoint{0.988630in}{0.761618in}}{\pgfqpoint{0.984239in}{0.751019in}}{\pgfqpoint{0.984239in}{0.739969in}}%
\pgfpathcurveto{\pgfqpoint{0.984239in}{0.728918in}}{\pgfqpoint{0.988630in}{0.718319in}}{\pgfqpoint{0.996443in}{0.710506in}}%
\pgfpathcurveto{\pgfqpoint{1.004257in}{0.702692in}}{\pgfqpoint{1.014856in}{0.698302in}}{\pgfqpoint{1.025906in}{0.698302in}}%
\pgfpathclose%
\pgfusepath{stroke,fill}%
\end{pgfscope}%
\begin{pgfscope}%
\pgfpathrectangle{\pgfqpoint{0.800000in}{0.528000in}}{\pgfqpoint{4.960000in}{3.696000in}}%
\pgfusepath{clip}%
\pgfsetbuttcap%
\pgfsetroundjoin%
\definecolor{currentfill}{rgb}{0.000000,0.000000,0.000000}%
\pgfsetfillcolor{currentfill}%
\pgfsetlinewidth{1.003750pt}%
\definecolor{currentstroke}{rgb}{0.000000,0.000000,0.000000}%
\pgfsetstrokecolor{currentstroke}%
\pgfsetdash{}{0pt}%
\pgfpathmoveto{\pgfqpoint{1.025906in}{0.698302in}}%
\pgfpathcurveto{\pgfqpoint{1.036956in}{0.698302in}}{\pgfqpoint{1.047555in}{0.702692in}}{\pgfqpoint{1.055369in}{0.710506in}}%
\pgfpathcurveto{\pgfqpoint{1.063182in}{0.718319in}}{\pgfqpoint{1.067573in}{0.728918in}}{\pgfqpoint{1.067573in}{0.739969in}}%
\pgfpathcurveto{\pgfqpoint{1.067573in}{0.751019in}}{\pgfqpoint{1.063182in}{0.761618in}}{\pgfqpoint{1.055369in}{0.769431in}}%
\pgfpathcurveto{\pgfqpoint{1.047555in}{0.777245in}}{\pgfqpoint{1.036956in}{0.781635in}}{\pgfqpoint{1.025906in}{0.781635in}}%
\pgfpathcurveto{\pgfqpoint{1.014856in}{0.781635in}}{\pgfqpoint{1.004257in}{0.777245in}}{\pgfqpoint{0.996443in}{0.769431in}}%
\pgfpathcurveto{\pgfqpoint{0.988630in}{0.761618in}}{\pgfqpoint{0.984239in}{0.751019in}}{\pgfqpoint{0.984239in}{0.739969in}}%
\pgfpathcurveto{\pgfqpoint{0.984239in}{0.728918in}}{\pgfqpoint{0.988630in}{0.718319in}}{\pgfqpoint{0.996443in}{0.710506in}}%
\pgfpathcurveto{\pgfqpoint{1.004257in}{0.702692in}}{\pgfqpoint{1.014856in}{0.698302in}}{\pgfqpoint{1.025906in}{0.698302in}}%
\pgfpathclose%
\pgfusepath{stroke,fill}%
\end{pgfscope}%
\begin{pgfscope}%
\pgfpathrectangle{\pgfqpoint{0.800000in}{0.528000in}}{\pgfqpoint{4.960000in}{3.696000in}}%
\pgfusepath{clip}%
\pgfsetbuttcap%
\pgfsetroundjoin%
\definecolor{currentfill}{rgb}{0.000000,0.000000,0.000000}%
\pgfsetfillcolor{currentfill}%
\pgfsetlinewidth{1.003750pt}%
\definecolor{currentstroke}{rgb}{0.000000,0.000000,0.000000}%
\pgfsetstrokecolor{currentstroke}%
\pgfsetdash{}{0pt}%
\pgfpathmoveto{\pgfqpoint{1.025906in}{0.676804in}}%
\pgfpathcurveto{\pgfqpoint{1.036956in}{0.676804in}}{\pgfqpoint{1.047555in}{0.681195in}}{\pgfqpoint{1.055369in}{0.689008in}}%
\pgfpathcurveto{\pgfqpoint{1.063182in}{0.696822in}}{\pgfqpoint{1.067573in}{0.707421in}}{\pgfqpoint{1.067573in}{0.718471in}}%
\pgfpathcurveto{\pgfqpoint{1.067573in}{0.729521in}}{\pgfqpoint{1.063182in}{0.740120in}}{\pgfqpoint{1.055369in}{0.747934in}}%
\pgfpathcurveto{\pgfqpoint{1.047555in}{0.755748in}}{\pgfqpoint{1.036956in}{0.760138in}}{\pgfqpoint{1.025906in}{0.760138in}}%
\pgfpathcurveto{\pgfqpoint{1.014856in}{0.760138in}}{\pgfqpoint{1.004257in}{0.755748in}}{\pgfqpoint{0.996443in}{0.747934in}}%
\pgfpathcurveto{\pgfqpoint{0.988630in}{0.740120in}}{\pgfqpoint{0.984239in}{0.729521in}}{\pgfqpoint{0.984239in}{0.718471in}}%
\pgfpathcurveto{\pgfqpoint{0.984239in}{0.707421in}}{\pgfqpoint{0.988630in}{0.696822in}}{\pgfqpoint{0.996443in}{0.689008in}}%
\pgfpathcurveto{\pgfqpoint{1.004257in}{0.681195in}}{\pgfqpoint{1.014856in}{0.676804in}}{\pgfqpoint{1.025906in}{0.676804in}}%
\pgfpathclose%
\pgfusepath{stroke,fill}%
\end{pgfscope}%
\begin{pgfscope}%
\pgfpathrectangle{\pgfqpoint{0.800000in}{0.528000in}}{\pgfqpoint{4.960000in}{3.696000in}}%
\pgfusepath{clip}%
\pgfsetbuttcap%
\pgfsetroundjoin%
\definecolor{currentfill}{rgb}{0.000000,0.000000,0.000000}%
\pgfsetfillcolor{currentfill}%
\pgfsetlinewidth{1.003750pt}%
\definecolor{currentstroke}{rgb}{0.000000,0.000000,0.000000}%
\pgfsetstrokecolor{currentstroke}%
\pgfsetdash{}{0pt}%
\pgfpathmoveto{\pgfqpoint{1.025906in}{0.719799in}}%
\pgfpathcurveto{\pgfqpoint{1.036956in}{0.719799in}}{\pgfqpoint{1.047555in}{0.724190in}}{\pgfqpoint{1.055369in}{0.732003in}}%
\pgfpathcurveto{\pgfqpoint{1.063182in}{0.739817in}}{\pgfqpoint{1.067573in}{0.750416in}}{\pgfqpoint{1.067573in}{0.761466in}}%
\pgfpathcurveto{\pgfqpoint{1.067573in}{0.772516in}}{\pgfqpoint{1.063182in}{0.783115in}}{\pgfqpoint{1.055369in}{0.790929in}}%
\pgfpathcurveto{\pgfqpoint{1.047555in}{0.798743in}}{\pgfqpoint{1.036956in}{0.803133in}}{\pgfqpoint{1.025906in}{0.803133in}}%
\pgfpathcurveto{\pgfqpoint{1.014856in}{0.803133in}}{\pgfqpoint{1.004257in}{0.798743in}}{\pgfqpoint{0.996443in}{0.790929in}}%
\pgfpathcurveto{\pgfqpoint{0.988630in}{0.783115in}}{\pgfqpoint{0.984239in}{0.772516in}}{\pgfqpoint{0.984239in}{0.761466in}}%
\pgfpathcurveto{\pgfqpoint{0.984239in}{0.750416in}}{\pgfqpoint{0.988630in}{0.739817in}}{\pgfqpoint{0.996443in}{0.732003in}}%
\pgfpathcurveto{\pgfqpoint{1.004257in}{0.724190in}}{\pgfqpoint{1.014856in}{0.719799in}}{\pgfqpoint{1.025906in}{0.719799in}}%
\pgfpathclose%
\pgfusepath{stroke,fill}%
\end{pgfscope}%
\begin{pgfscope}%
\pgfpathrectangle{\pgfqpoint{0.800000in}{0.528000in}}{\pgfqpoint{4.960000in}{3.696000in}}%
\pgfusepath{clip}%
\pgfsetbuttcap%
\pgfsetroundjoin%
\definecolor{currentfill}{rgb}{0.000000,0.000000,0.000000}%
\pgfsetfillcolor{currentfill}%
\pgfsetlinewidth{1.003750pt}%
\definecolor{currentstroke}{rgb}{0.000000,0.000000,0.000000}%
\pgfsetstrokecolor{currentstroke}%
\pgfsetdash{}{0pt}%
\pgfpathmoveto{\pgfqpoint{1.025906in}{0.676804in}}%
\pgfpathcurveto{\pgfqpoint{1.036956in}{0.676804in}}{\pgfqpoint{1.047555in}{0.681195in}}{\pgfqpoint{1.055369in}{0.689008in}}%
\pgfpathcurveto{\pgfqpoint{1.063182in}{0.696822in}}{\pgfqpoint{1.067573in}{0.707421in}}{\pgfqpoint{1.067573in}{0.718471in}}%
\pgfpathcurveto{\pgfqpoint{1.067573in}{0.729521in}}{\pgfqpoint{1.063182in}{0.740120in}}{\pgfqpoint{1.055369in}{0.747934in}}%
\pgfpathcurveto{\pgfqpoint{1.047555in}{0.755748in}}{\pgfqpoint{1.036956in}{0.760138in}}{\pgfqpoint{1.025906in}{0.760138in}}%
\pgfpathcurveto{\pgfqpoint{1.014856in}{0.760138in}}{\pgfqpoint{1.004257in}{0.755748in}}{\pgfqpoint{0.996443in}{0.747934in}}%
\pgfpathcurveto{\pgfqpoint{0.988630in}{0.740120in}}{\pgfqpoint{0.984239in}{0.729521in}}{\pgfqpoint{0.984239in}{0.718471in}}%
\pgfpathcurveto{\pgfqpoint{0.984239in}{0.707421in}}{\pgfqpoint{0.988630in}{0.696822in}}{\pgfqpoint{0.996443in}{0.689008in}}%
\pgfpathcurveto{\pgfqpoint{1.004257in}{0.681195in}}{\pgfqpoint{1.014856in}{0.676804in}}{\pgfqpoint{1.025906in}{0.676804in}}%
\pgfpathclose%
\pgfusepath{stroke,fill}%
\end{pgfscope}%
\begin{pgfscope}%
\pgfpathrectangle{\pgfqpoint{0.800000in}{0.528000in}}{\pgfqpoint{4.960000in}{3.696000in}}%
\pgfusepath{clip}%
\pgfsetbuttcap%
\pgfsetroundjoin%
\definecolor{currentfill}{rgb}{0.000000,0.000000,0.000000}%
\pgfsetfillcolor{currentfill}%
\pgfsetlinewidth{1.003750pt}%
\definecolor{currentstroke}{rgb}{0.000000,0.000000,0.000000}%
\pgfsetstrokecolor{currentstroke}%
\pgfsetdash{}{0pt}%
\pgfpathmoveto{\pgfqpoint{1.025906in}{0.698302in}}%
\pgfpathcurveto{\pgfqpoint{1.036956in}{0.698302in}}{\pgfqpoint{1.047555in}{0.702692in}}{\pgfqpoint{1.055369in}{0.710506in}}%
\pgfpathcurveto{\pgfqpoint{1.063182in}{0.718319in}}{\pgfqpoint{1.067573in}{0.728918in}}{\pgfqpoint{1.067573in}{0.739969in}}%
\pgfpathcurveto{\pgfqpoint{1.067573in}{0.751019in}}{\pgfqpoint{1.063182in}{0.761618in}}{\pgfqpoint{1.055369in}{0.769431in}}%
\pgfpathcurveto{\pgfqpoint{1.047555in}{0.777245in}}{\pgfqpoint{1.036956in}{0.781635in}}{\pgfqpoint{1.025906in}{0.781635in}}%
\pgfpathcurveto{\pgfqpoint{1.014856in}{0.781635in}}{\pgfqpoint{1.004257in}{0.777245in}}{\pgfqpoint{0.996443in}{0.769431in}}%
\pgfpathcurveto{\pgfqpoint{0.988630in}{0.761618in}}{\pgfqpoint{0.984239in}{0.751019in}}{\pgfqpoint{0.984239in}{0.739969in}}%
\pgfpathcurveto{\pgfqpoint{0.984239in}{0.728918in}}{\pgfqpoint{0.988630in}{0.718319in}}{\pgfqpoint{0.996443in}{0.710506in}}%
\pgfpathcurveto{\pgfqpoint{1.004257in}{0.702692in}}{\pgfqpoint{1.014856in}{0.698302in}}{\pgfqpoint{1.025906in}{0.698302in}}%
\pgfpathclose%
\pgfusepath{stroke,fill}%
\end{pgfscope}%
\begin{pgfscope}%
\pgfpathrectangle{\pgfqpoint{0.800000in}{0.528000in}}{\pgfqpoint{4.960000in}{3.696000in}}%
\pgfusepath{clip}%
\pgfsetbuttcap%
\pgfsetroundjoin%
\definecolor{currentfill}{rgb}{0.000000,0.000000,0.000000}%
\pgfsetfillcolor{currentfill}%
\pgfsetlinewidth{1.003750pt}%
\definecolor{currentstroke}{rgb}{0.000000,0.000000,0.000000}%
\pgfsetstrokecolor{currentstroke}%
\pgfsetdash{}{0pt}%
\pgfpathmoveto{\pgfqpoint{1.025906in}{0.698302in}}%
\pgfpathcurveto{\pgfqpoint{1.036956in}{0.698302in}}{\pgfqpoint{1.047555in}{0.702692in}}{\pgfqpoint{1.055369in}{0.710506in}}%
\pgfpathcurveto{\pgfqpoint{1.063182in}{0.718319in}}{\pgfqpoint{1.067573in}{0.728918in}}{\pgfqpoint{1.067573in}{0.739969in}}%
\pgfpathcurveto{\pgfqpoint{1.067573in}{0.751019in}}{\pgfqpoint{1.063182in}{0.761618in}}{\pgfqpoint{1.055369in}{0.769431in}}%
\pgfpathcurveto{\pgfqpoint{1.047555in}{0.777245in}}{\pgfqpoint{1.036956in}{0.781635in}}{\pgfqpoint{1.025906in}{0.781635in}}%
\pgfpathcurveto{\pgfqpoint{1.014856in}{0.781635in}}{\pgfqpoint{1.004257in}{0.777245in}}{\pgfqpoint{0.996443in}{0.769431in}}%
\pgfpathcurveto{\pgfqpoint{0.988630in}{0.761618in}}{\pgfqpoint{0.984239in}{0.751019in}}{\pgfqpoint{0.984239in}{0.739969in}}%
\pgfpathcurveto{\pgfqpoint{0.984239in}{0.728918in}}{\pgfqpoint{0.988630in}{0.718319in}}{\pgfqpoint{0.996443in}{0.710506in}}%
\pgfpathcurveto{\pgfqpoint{1.004257in}{0.702692in}}{\pgfqpoint{1.014856in}{0.698302in}}{\pgfqpoint{1.025906in}{0.698302in}}%
\pgfpathclose%
\pgfusepath{stroke,fill}%
\end{pgfscope}%
\begin{pgfscope}%
\pgfpathrectangle{\pgfqpoint{0.800000in}{0.528000in}}{\pgfqpoint{4.960000in}{3.696000in}}%
\pgfusepath{clip}%
\pgfsetbuttcap%
\pgfsetroundjoin%
\definecolor{currentfill}{rgb}{0.000000,0.000000,0.000000}%
\pgfsetfillcolor{currentfill}%
\pgfsetlinewidth{1.003750pt}%
\definecolor{currentstroke}{rgb}{0.000000,0.000000,0.000000}%
\pgfsetstrokecolor{currentstroke}%
\pgfsetdash{}{0pt}%
\pgfpathmoveto{\pgfqpoint{1.025906in}{0.676804in}}%
\pgfpathcurveto{\pgfqpoint{1.036956in}{0.676804in}}{\pgfqpoint{1.047555in}{0.681195in}}{\pgfqpoint{1.055369in}{0.689008in}}%
\pgfpathcurveto{\pgfqpoint{1.063182in}{0.696822in}}{\pgfqpoint{1.067573in}{0.707421in}}{\pgfqpoint{1.067573in}{0.718471in}}%
\pgfpathcurveto{\pgfqpoint{1.067573in}{0.729521in}}{\pgfqpoint{1.063182in}{0.740120in}}{\pgfqpoint{1.055369in}{0.747934in}}%
\pgfpathcurveto{\pgfqpoint{1.047555in}{0.755748in}}{\pgfqpoint{1.036956in}{0.760138in}}{\pgfqpoint{1.025906in}{0.760138in}}%
\pgfpathcurveto{\pgfqpoint{1.014856in}{0.760138in}}{\pgfqpoint{1.004257in}{0.755748in}}{\pgfqpoint{0.996443in}{0.747934in}}%
\pgfpathcurveto{\pgfqpoint{0.988630in}{0.740120in}}{\pgfqpoint{0.984239in}{0.729521in}}{\pgfqpoint{0.984239in}{0.718471in}}%
\pgfpathcurveto{\pgfqpoint{0.984239in}{0.707421in}}{\pgfqpoint{0.988630in}{0.696822in}}{\pgfqpoint{0.996443in}{0.689008in}}%
\pgfpathcurveto{\pgfqpoint{1.004257in}{0.681195in}}{\pgfqpoint{1.014856in}{0.676804in}}{\pgfqpoint{1.025906in}{0.676804in}}%
\pgfpathclose%
\pgfusepath{stroke,fill}%
\end{pgfscope}%
\begin{pgfscope}%
\pgfpathrectangle{\pgfqpoint{0.800000in}{0.528000in}}{\pgfqpoint{4.960000in}{3.696000in}}%
\pgfusepath{clip}%
\pgfsetbuttcap%
\pgfsetroundjoin%
\definecolor{currentfill}{rgb}{0.000000,0.000000,0.000000}%
\pgfsetfillcolor{currentfill}%
\pgfsetlinewidth{1.003750pt}%
\definecolor{currentstroke}{rgb}{0.000000,0.000000,0.000000}%
\pgfsetstrokecolor{currentstroke}%
\pgfsetdash{}{0pt}%
\pgfpathmoveto{\pgfqpoint{1.025906in}{0.719799in}}%
\pgfpathcurveto{\pgfqpoint{1.036956in}{0.719799in}}{\pgfqpoint{1.047555in}{0.724190in}}{\pgfqpoint{1.055369in}{0.732003in}}%
\pgfpathcurveto{\pgfqpoint{1.063182in}{0.739817in}}{\pgfqpoint{1.067573in}{0.750416in}}{\pgfqpoint{1.067573in}{0.761466in}}%
\pgfpathcurveto{\pgfqpoint{1.067573in}{0.772516in}}{\pgfqpoint{1.063182in}{0.783115in}}{\pgfqpoint{1.055369in}{0.790929in}}%
\pgfpathcurveto{\pgfqpoint{1.047555in}{0.798743in}}{\pgfqpoint{1.036956in}{0.803133in}}{\pgfqpoint{1.025906in}{0.803133in}}%
\pgfpathcurveto{\pgfqpoint{1.014856in}{0.803133in}}{\pgfqpoint{1.004257in}{0.798743in}}{\pgfqpoint{0.996443in}{0.790929in}}%
\pgfpathcurveto{\pgfqpoint{0.988630in}{0.783115in}}{\pgfqpoint{0.984239in}{0.772516in}}{\pgfqpoint{0.984239in}{0.761466in}}%
\pgfpathcurveto{\pgfqpoint{0.984239in}{0.750416in}}{\pgfqpoint{0.988630in}{0.739817in}}{\pgfqpoint{0.996443in}{0.732003in}}%
\pgfpathcurveto{\pgfqpoint{1.004257in}{0.724190in}}{\pgfqpoint{1.014856in}{0.719799in}}{\pgfqpoint{1.025906in}{0.719799in}}%
\pgfpathclose%
\pgfusepath{stroke,fill}%
\end{pgfscope}%
\begin{pgfscope}%
\pgfpathrectangle{\pgfqpoint{0.800000in}{0.528000in}}{\pgfqpoint{4.960000in}{3.696000in}}%
\pgfusepath{clip}%
\pgfsetbuttcap%
\pgfsetroundjoin%
\definecolor{currentfill}{rgb}{0.000000,0.000000,0.000000}%
\pgfsetfillcolor{currentfill}%
\pgfsetlinewidth{1.003750pt}%
\definecolor{currentstroke}{rgb}{0.000000,0.000000,0.000000}%
\pgfsetstrokecolor{currentstroke}%
\pgfsetdash{}{0pt}%
\pgfpathmoveto{\pgfqpoint{1.025906in}{0.719799in}}%
\pgfpathcurveto{\pgfqpoint{1.036956in}{0.719799in}}{\pgfqpoint{1.047555in}{0.724190in}}{\pgfqpoint{1.055369in}{0.732003in}}%
\pgfpathcurveto{\pgfqpoint{1.063182in}{0.739817in}}{\pgfqpoint{1.067573in}{0.750416in}}{\pgfqpoint{1.067573in}{0.761466in}}%
\pgfpathcurveto{\pgfqpoint{1.067573in}{0.772516in}}{\pgfqpoint{1.063182in}{0.783115in}}{\pgfqpoint{1.055369in}{0.790929in}}%
\pgfpathcurveto{\pgfqpoint{1.047555in}{0.798743in}}{\pgfqpoint{1.036956in}{0.803133in}}{\pgfqpoint{1.025906in}{0.803133in}}%
\pgfpathcurveto{\pgfqpoint{1.014856in}{0.803133in}}{\pgfqpoint{1.004257in}{0.798743in}}{\pgfqpoint{0.996443in}{0.790929in}}%
\pgfpathcurveto{\pgfqpoint{0.988630in}{0.783115in}}{\pgfqpoint{0.984239in}{0.772516in}}{\pgfqpoint{0.984239in}{0.761466in}}%
\pgfpathcurveto{\pgfqpoint{0.984239in}{0.750416in}}{\pgfqpoint{0.988630in}{0.739817in}}{\pgfqpoint{0.996443in}{0.732003in}}%
\pgfpathcurveto{\pgfqpoint{1.004257in}{0.724190in}}{\pgfqpoint{1.014856in}{0.719799in}}{\pgfqpoint{1.025906in}{0.719799in}}%
\pgfpathclose%
\pgfusepath{stroke,fill}%
\end{pgfscope}%
\begin{pgfscope}%
\pgfpathrectangle{\pgfqpoint{0.800000in}{0.528000in}}{\pgfqpoint{4.960000in}{3.696000in}}%
\pgfusepath{clip}%
\pgfsetbuttcap%
\pgfsetroundjoin%
\definecolor{currentfill}{rgb}{0.000000,0.000000,0.000000}%
\pgfsetfillcolor{currentfill}%
\pgfsetlinewidth{1.003750pt}%
\definecolor{currentstroke}{rgb}{0.000000,0.000000,0.000000}%
\pgfsetstrokecolor{currentstroke}%
\pgfsetdash{}{0pt}%
\pgfpathmoveto{\pgfqpoint{1.025906in}{0.698302in}}%
\pgfpathcurveto{\pgfqpoint{1.036956in}{0.698302in}}{\pgfqpoint{1.047555in}{0.702692in}}{\pgfqpoint{1.055369in}{0.710506in}}%
\pgfpathcurveto{\pgfqpoint{1.063182in}{0.718319in}}{\pgfqpoint{1.067573in}{0.728918in}}{\pgfqpoint{1.067573in}{0.739969in}}%
\pgfpathcurveto{\pgfqpoint{1.067573in}{0.751019in}}{\pgfqpoint{1.063182in}{0.761618in}}{\pgfqpoint{1.055369in}{0.769431in}}%
\pgfpathcurveto{\pgfqpoint{1.047555in}{0.777245in}}{\pgfqpoint{1.036956in}{0.781635in}}{\pgfqpoint{1.025906in}{0.781635in}}%
\pgfpathcurveto{\pgfqpoint{1.014856in}{0.781635in}}{\pgfqpoint{1.004257in}{0.777245in}}{\pgfqpoint{0.996443in}{0.769431in}}%
\pgfpathcurveto{\pgfqpoint{0.988630in}{0.761618in}}{\pgfqpoint{0.984239in}{0.751019in}}{\pgfqpoint{0.984239in}{0.739969in}}%
\pgfpathcurveto{\pgfqpoint{0.984239in}{0.728918in}}{\pgfqpoint{0.988630in}{0.718319in}}{\pgfqpoint{0.996443in}{0.710506in}}%
\pgfpathcurveto{\pgfqpoint{1.004257in}{0.702692in}}{\pgfqpoint{1.014856in}{0.698302in}}{\pgfqpoint{1.025906in}{0.698302in}}%
\pgfpathclose%
\pgfusepath{stroke,fill}%
\end{pgfscope}%
\begin{pgfscope}%
\pgfpathrectangle{\pgfqpoint{0.800000in}{0.528000in}}{\pgfqpoint{4.960000in}{3.696000in}}%
\pgfusepath{clip}%
\pgfsetbuttcap%
\pgfsetroundjoin%
\definecolor{currentfill}{rgb}{0.000000,0.000000,0.000000}%
\pgfsetfillcolor{currentfill}%
\pgfsetlinewidth{1.003750pt}%
\definecolor{currentstroke}{rgb}{0.000000,0.000000,0.000000}%
\pgfsetstrokecolor{currentstroke}%
\pgfsetdash{}{0pt}%
\pgfpathmoveto{\pgfqpoint{1.025906in}{0.676804in}}%
\pgfpathcurveto{\pgfqpoint{1.036956in}{0.676804in}}{\pgfqpoint{1.047555in}{0.681195in}}{\pgfqpoint{1.055369in}{0.689008in}}%
\pgfpathcurveto{\pgfqpoint{1.063182in}{0.696822in}}{\pgfqpoint{1.067573in}{0.707421in}}{\pgfqpoint{1.067573in}{0.718471in}}%
\pgfpathcurveto{\pgfqpoint{1.067573in}{0.729521in}}{\pgfqpoint{1.063182in}{0.740120in}}{\pgfqpoint{1.055369in}{0.747934in}}%
\pgfpathcurveto{\pgfqpoint{1.047555in}{0.755748in}}{\pgfqpoint{1.036956in}{0.760138in}}{\pgfqpoint{1.025906in}{0.760138in}}%
\pgfpathcurveto{\pgfqpoint{1.014856in}{0.760138in}}{\pgfqpoint{1.004257in}{0.755748in}}{\pgfqpoint{0.996443in}{0.747934in}}%
\pgfpathcurveto{\pgfqpoint{0.988630in}{0.740120in}}{\pgfqpoint{0.984239in}{0.729521in}}{\pgfqpoint{0.984239in}{0.718471in}}%
\pgfpathcurveto{\pgfqpoint{0.984239in}{0.707421in}}{\pgfqpoint{0.988630in}{0.696822in}}{\pgfqpoint{0.996443in}{0.689008in}}%
\pgfpathcurveto{\pgfqpoint{1.004257in}{0.681195in}}{\pgfqpoint{1.014856in}{0.676804in}}{\pgfqpoint{1.025906in}{0.676804in}}%
\pgfpathclose%
\pgfusepath{stroke,fill}%
\end{pgfscope}%
\begin{pgfscope}%
\pgfpathrectangle{\pgfqpoint{0.800000in}{0.528000in}}{\pgfqpoint{4.960000in}{3.696000in}}%
\pgfusepath{clip}%
\pgfsetbuttcap%
\pgfsetroundjoin%
\definecolor{currentfill}{rgb}{0.000000,0.000000,0.000000}%
\pgfsetfillcolor{currentfill}%
\pgfsetlinewidth{1.003750pt}%
\definecolor{currentstroke}{rgb}{0.000000,0.000000,0.000000}%
\pgfsetstrokecolor{currentstroke}%
\pgfsetdash{}{0pt}%
\pgfpathmoveto{\pgfqpoint{1.025906in}{0.698302in}}%
\pgfpathcurveto{\pgfqpoint{1.036956in}{0.698302in}}{\pgfqpoint{1.047555in}{0.702692in}}{\pgfqpoint{1.055369in}{0.710506in}}%
\pgfpathcurveto{\pgfqpoint{1.063182in}{0.718319in}}{\pgfqpoint{1.067573in}{0.728918in}}{\pgfqpoint{1.067573in}{0.739969in}}%
\pgfpathcurveto{\pgfqpoint{1.067573in}{0.751019in}}{\pgfqpoint{1.063182in}{0.761618in}}{\pgfqpoint{1.055369in}{0.769431in}}%
\pgfpathcurveto{\pgfqpoint{1.047555in}{0.777245in}}{\pgfqpoint{1.036956in}{0.781635in}}{\pgfqpoint{1.025906in}{0.781635in}}%
\pgfpathcurveto{\pgfqpoint{1.014856in}{0.781635in}}{\pgfqpoint{1.004257in}{0.777245in}}{\pgfqpoint{0.996443in}{0.769431in}}%
\pgfpathcurveto{\pgfqpoint{0.988630in}{0.761618in}}{\pgfqpoint{0.984239in}{0.751019in}}{\pgfqpoint{0.984239in}{0.739969in}}%
\pgfpathcurveto{\pgfqpoint{0.984239in}{0.728918in}}{\pgfqpoint{0.988630in}{0.718319in}}{\pgfqpoint{0.996443in}{0.710506in}}%
\pgfpathcurveto{\pgfqpoint{1.004257in}{0.702692in}}{\pgfqpoint{1.014856in}{0.698302in}}{\pgfqpoint{1.025906in}{0.698302in}}%
\pgfpathclose%
\pgfusepath{stroke,fill}%
\end{pgfscope}%
\begin{pgfscope}%
\pgfpathrectangle{\pgfqpoint{0.800000in}{0.528000in}}{\pgfqpoint{4.960000in}{3.696000in}}%
\pgfusepath{clip}%
\pgfsetbuttcap%
\pgfsetroundjoin%
\definecolor{currentfill}{rgb}{0.000000,0.000000,0.000000}%
\pgfsetfillcolor{currentfill}%
\pgfsetlinewidth{1.003750pt}%
\definecolor{currentstroke}{rgb}{0.000000,0.000000,0.000000}%
\pgfsetstrokecolor{currentstroke}%
\pgfsetdash{}{0pt}%
\pgfpathmoveto{\pgfqpoint{1.025906in}{0.698302in}}%
\pgfpathcurveto{\pgfqpoint{1.036956in}{0.698302in}}{\pgfqpoint{1.047555in}{0.702692in}}{\pgfqpoint{1.055369in}{0.710506in}}%
\pgfpathcurveto{\pgfqpoint{1.063182in}{0.718319in}}{\pgfqpoint{1.067573in}{0.728918in}}{\pgfqpoint{1.067573in}{0.739969in}}%
\pgfpathcurveto{\pgfqpoint{1.067573in}{0.751019in}}{\pgfqpoint{1.063182in}{0.761618in}}{\pgfqpoint{1.055369in}{0.769431in}}%
\pgfpathcurveto{\pgfqpoint{1.047555in}{0.777245in}}{\pgfqpoint{1.036956in}{0.781635in}}{\pgfqpoint{1.025906in}{0.781635in}}%
\pgfpathcurveto{\pgfqpoint{1.014856in}{0.781635in}}{\pgfqpoint{1.004257in}{0.777245in}}{\pgfqpoint{0.996443in}{0.769431in}}%
\pgfpathcurveto{\pgfqpoint{0.988630in}{0.761618in}}{\pgfqpoint{0.984239in}{0.751019in}}{\pgfqpoint{0.984239in}{0.739969in}}%
\pgfpathcurveto{\pgfqpoint{0.984239in}{0.728918in}}{\pgfqpoint{0.988630in}{0.718319in}}{\pgfqpoint{0.996443in}{0.710506in}}%
\pgfpathcurveto{\pgfqpoint{1.004257in}{0.702692in}}{\pgfqpoint{1.014856in}{0.698302in}}{\pgfqpoint{1.025906in}{0.698302in}}%
\pgfpathclose%
\pgfusepath{stroke,fill}%
\end{pgfscope}%
\begin{pgfscope}%
\pgfpathrectangle{\pgfqpoint{0.800000in}{0.528000in}}{\pgfqpoint{4.960000in}{3.696000in}}%
\pgfusepath{clip}%
\pgfsetbuttcap%
\pgfsetroundjoin%
\definecolor{currentfill}{rgb}{0.000000,0.000000,0.000000}%
\pgfsetfillcolor{currentfill}%
\pgfsetlinewidth{1.003750pt}%
\definecolor{currentstroke}{rgb}{0.000000,0.000000,0.000000}%
\pgfsetstrokecolor{currentstroke}%
\pgfsetdash{}{0pt}%
\pgfpathmoveto{\pgfqpoint{1.025906in}{0.719799in}}%
\pgfpathcurveto{\pgfqpoint{1.036956in}{0.719799in}}{\pgfqpoint{1.047555in}{0.724190in}}{\pgfqpoint{1.055369in}{0.732003in}}%
\pgfpathcurveto{\pgfqpoint{1.063182in}{0.739817in}}{\pgfqpoint{1.067573in}{0.750416in}}{\pgfqpoint{1.067573in}{0.761466in}}%
\pgfpathcurveto{\pgfqpoint{1.067573in}{0.772516in}}{\pgfqpoint{1.063182in}{0.783115in}}{\pgfqpoint{1.055369in}{0.790929in}}%
\pgfpathcurveto{\pgfqpoint{1.047555in}{0.798743in}}{\pgfqpoint{1.036956in}{0.803133in}}{\pgfqpoint{1.025906in}{0.803133in}}%
\pgfpathcurveto{\pgfqpoint{1.014856in}{0.803133in}}{\pgfqpoint{1.004257in}{0.798743in}}{\pgfqpoint{0.996443in}{0.790929in}}%
\pgfpathcurveto{\pgfqpoint{0.988630in}{0.783115in}}{\pgfqpoint{0.984239in}{0.772516in}}{\pgfqpoint{0.984239in}{0.761466in}}%
\pgfpathcurveto{\pgfqpoint{0.984239in}{0.750416in}}{\pgfqpoint{0.988630in}{0.739817in}}{\pgfqpoint{0.996443in}{0.732003in}}%
\pgfpathcurveto{\pgfqpoint{1.004257in}{0.724190in}}{\pgfqpoint{1.014856in}{0.719799in}}{\pgfqpoint{1.025906in}{0.719799in}}%
\pgfpathclose%
\pgfusepath{stroke,fill}%
\end{pgfscope}%
\begin{pgfscope}%
\pgfpathrectangle{\pgfqpoint{0.800000in}{0.528000in}}{\pgfqpoint{4.960000in}{3.696000in}}%
\pgfusepath{clip}%
\pgfsetbuttcap%
\pgfsetroundjoin%
\definecolor{currentfill}{rgb}{0.000000,0.000000,0.000000}%
\pgfsetfillcolor{currentfill}%
\pgfsetlinewidth{1.003750pt}%
\definecolor{currentstroke}{rgb}{0.000000,0.000000,0.000000}%
\pgfsetstrokecolor{currentstroke}%
\pgfsetdash{}{0pt}%
\pgfpathmoveto{\pgfqpoint{1.025906in}{0.698302in}}%
\pgfpathcurveto{\pgfqpoint{1.036956in}{0.698302in}}{\pgfqpoint{1.047555in}{0.702692in}}{\pgfqpoint{1.055369in}{0.710506in}}%
\pgfpathcurveto{\pgfqpoint{1.063182in}{0.718319in}}{\pgfqpoint{1.067573in}{0.728918in}}{\pgfqpoint{1.067573in}{0.739969in}}%
\pgfpathcurveto{\pgfqpoint{1.067573in}{0.751019in}}{\pgfqpoint{1.063182in}{0.761618in}}{\pgfqpoint{1.055369in}{0.769431in}}%
\pgfpathcurveto{\pgfqpoint{1.047555in}{0.777245in}}{\pgfqpoint{1.036956in}{0.781635in}}{\pgfqpoint{1.025906in}{0.781635in}}%
\pgfpathcurveto{\pgfqpoint{1.014856in}{0.781635in}}{\pgfqpoint{1.004257in}{0.777245in}}{\pgfqpoint{0.996443in}{0.769431in}}%
\pgfpathcurveto{\pgfqpoint{0.988630in}{0.761618in}}{\pgfqpoint{0.984239in}{0.751019in}}{\pgfqpoint{0.984239in}{0.739969in}}%
\pgfpathcurveto{\pgfqpoint{0.984239in}{0.728918in}}{\pgfqpoint{0.988630in}{0.718319in}}{\pgfqpoint{0.996443in}{0.710506in}}%
\pgfpathcurveto{\pgfqpoint{1.004257in}{0.702692in}}{\pgfqpoint{1.014856in}{0.698302in}}{\pgfqpoint{1.025906in}{0.698302in}}%
\pgfpathclose%
\pgfusepath{stroke,fill}%
\end{pgfscope}%
\begin{pgfscope}%
\pgfpathrectangle{\pgfqpoint{0.800000in}{0.528000in}}{\pgfqpoint{4.960000in}{3.696000in}}%
\pgfusepath{clip}%
\pgfsetbuttcap%
\pgfsetroundjoin%
\definecolor{currentfill}{rgb}{0.000000,0.000000,0.000000}%
\pgfsetfillcolor{currentfill}%
\pgfsetlinewidth{1.003750pt}%
\definecolor{currentstroke}{rgb}{0.000000,0.000000,0.000000}%
\pgfsetstrokecolor{currentstroke}%
\pgfsetdash{}{0pt}%
\pgfpathmoveto{\pgfqpoint{1.025906in}{0.655307in}}%
\pgfpathcurveto{\pgfqpoint{1.036956in}{0.655307in}}{\pgfqpoint{1.047555in}{0.659697in}}{\pgfqpoint{1.055369in}{0.667511in}}%
\pgfpathcurveto{\pgfqpoint{1.063182in}{0.675324in}}{\pgfqpoint{1.067573in}{0.685924in}}{\pgfqpoint{1.067573in}{0.696974in}}%
\pgfpathcurveto{\pgfqpoint{1.067573in}{0.708024in}}{\pgfqpoint{1.063182in}{0.718623in}}{\pgfqpoint{1.055369in}{0.726436in}}%
\pgfpathcurveto{\pgfqpoint{1.047555in}{0.734250in}}{\pgfqpoint{1.036956in}{0.738640in}}{\pgfqpoint{1.025906in}{0.738640in}}%
\pgfpathcurveto{\pgfqpoint{1.014856in}{0.738640in}}{\pgfqpoint{1.004257in}{0.734250in}}{\pgfqpoint{0.996443in}{0.726436in}}%
\pgfpathcurveto{\pgfqpoint{0.988630in}{0.718623in}}{\pgfqpoint{0.984239in}{0.708024in}}{\pgfqpoint{0.984239in}{0.696974in}}%
\pgfpathcurveto{\pgfqpoint{0.984239in}{0.685924in}}{\pgfqpoint{0.988630in}{0.675324in}}{\pgfqpoint{0.996443in}{0.667511in}}%
\pgfpathcurveto{\pgfqpoint{1.004257in}{0.659697in}}{\pgfqpoint{1.014856in}{0.655307in}}{\pgfqpoint{1.025906in}{0.655307in}}%
\pgfpathclose%
\pgfusepath{stroke,fill}%
\end{pgfscope}%
\begin{pgfscope}%
\pgfpathrectangle{\pgfqpoint{0.800000in}{0.528000in}}{\pgfqpoint{4.960000in}{3.696000in}}%
\pgfusepath{clip}%
\pgfsetbuttcap%
\pgfsetroundjoin%
\definecolor{currentfill}{rgb}{0.000000,0.000000,0.000000}%
\pgfsetfillcolor{currentfill}%
\pgfsetlinewidth{1.003750pt}%
\definecolor{currentstroke}{rgb}{0.000000,0.000000,0.000000}%
\pgfsetstrokecolor{currentstroke}%
\pgfsetdash{}{0pt}%
\pgfpathmoveto{\pgfqpoint{1.025906in}{0.676804in}}%
\pgfpathcurveto{\pgfqpoint{1.036956in}{0.676804in}}{\pgfqpoint{1.047555in}{0.681195in}}{\pgfqpoint{1.055369in}{0.689008in}}%
\pgfpathcurveto{\pgfqpoint{1.063182in}{0.696822in}}{\pgfqpoint{1.067573in}{0.707421in}}{\pgfqpoint{1.067573in}{0.718471in}}%
\pgfpathcurveto{\pgfqpoint{1.067573in}{0.729521in}}{\pgfqpoint{1.063182in}{0.740120in}}{\pgfqpoint{1.055369in}{0.747934in}}%
\pgfpathcurveto{\pgfqpoint{1.047555in}{0.755748in}}{\pgfqpoint{1.036956in}{0.760138in}}{\pgfqpoint{1.025906in}{0.760138in}}%
\pgfpathcurveto{\pgfqpoint{1.014856in}{0.760138in}}{\pgfqpoint{1.004257in}{0.755748in}}{\pgfqpoint{0.996443in}{0.747934in}}%
\pgfpathcurveto{\pgfqpoint{0.988630in}{0.740120in}}{\pgfqpoint{0.984239in}{0.729521in}}{\pgfqpoint{0.984239in}{0.718471in}}%
\pgfpathcurveto{\pgfqpoint{0.984239in}{0.707421in}}{\pgfqpoint{0.988630in}{0.696822in}}{\pgfqpoint{0.996443in}{0.689008in}}%
\pgfpathcurveto{\pgfqpoint{1.004257in}{0.681195in}}{\pgfqpoint{1.014856in}{0.676804in}}{\pgfqpoint{1.025906in}{0.676804in}}%
\pgfpathclose%
\pgfusepath{stroke,fill}%
\end{pgfscope}%
\begin{pgfscope}%
\pgfpathrectangle{\pgfqpoint{0.800000in}{0.528000in}}{\pgfqpoint{4.960000in}{3.696000in}}%
\pgfusepath{clip}%
\pgfsetbuttcap%
\pgfsetroundjoin%
\definecolor{currentfill}{rgb}{0.000000,0.000000,0.000000}%
\pgfsetfillcolor{currentfill}%
\pgfsetlinewidth{1.003750pt}%
\definecolor{currentstroke}{rgb}{0.000000,0.000000,0.000000}%
\pgfsetstrokecolor{currentstroke}%
\pgfsetdash{}{0pt}%
\pgfpathmoveto{\pgfqpoint{1.025906in}{0.698302in}}%
\pgfpathcurveto{\pgfqpoint{1.036956in}{0.698302in}}{\pgfqpoint{1.047555in}{0.702692in}}{\pgfqpoint{1.055369in}{0.710506in}}%
\pgfpathcurveto{\pgfqpoint{1.063182in}{0.718319in}}{\pgfqpoint{1.067573in}{0.728918in}}{\pgfqpoint{1.067573in}{0.739969in}}%
\pgfpathcurveto{\pgfqpoint{1.067573in}{0.751019in}}{\pgfqpoint{1.063182in}{0.761618in}}{\pgfqpoint{1.055369in}{0.769431in}}%
\pgfpathcurveto{\pgfqpoint{1.047555in}{0.777245in}}{\pgfqpoint{1.036956in}{0.781635in}}{\pgfqpoint{1.025906in}{0.781635in}}%
\pgfpathcurveto{\pgfqpoint{1.014856in}{0.781635in}}{\pgfqpoint{1.004257in}{0.777245in}}{\pgfqpoint{0.996443in}{0.769431in}}%
\pgfpathcurveto{\pgfqpoint{0.988630in}{0.761618in}}{\pgfqpoint{0.984239in}{0.751019in}}{\pgfqpoint{0.984239in}{0.739969in}}%
\pgfpathcurveto{\pgfqpoint{0.984239in}{0.728918in}}{\pgfqpoint{0.988630in}{0.718319in}}{\pgfqpoint{0.996443in}{0.710506in}}%
\pgfpathcurveto{\pgfqpoint{1.004257in}{0.702692in}}{\pgfqpoint{1.014856in}{0.698302in}}{\pgfqpoint{1.025906in}{0.698302in}}%
\pgfpathclose%
\pgfusepath{stroke,fill}%
\end{pgfscope}%
\begin{pgfscope}%
\pgfpathrectangle{\pgfqpoint{0.800000in}{0.528000in}}{\pgfqpoint{4.960000in}{3.696000in}}%
\pgfusepath{clip}%
\pgfsetbuttcap%
\pgfsetroundjoin%
\definecolor{currentfill}{rgb}{0.000000,0.000000,0.000000}%
\pgfsetfillcolor{currentfill}%
\pgfsetlinewidth{1.003750pt}%
\definecolor{currentstroke}{rgb}{0.000000,0.000000,0.000000}%
\pgfsetstrokecolor{currentstroke}%
\pgfsetdash{}{0pt}%
\pgfpathmoveto{\pgfqpoint{1.025906in}{0.698302in}}%
\pgfpathcurveto{\pgfqpoint{1.036956in}{0.698302in}}{\pgfqpoint{1.047555in}{0.702692in}}{\pgfqpoint{1.055369in}{0.710506in}}%
\pgfpathcurveto{\pgfqpoint{1.063182in}{0.718319in}}{\pgfqpoint{1.067573in}{0.728918in}}{\pgfqpoint{1.067573in}{0.739969in}}%
\pgfpathcurveto{\pgfqpoint{1.067573in}{0.751019in}}{\pgfqpoint{1.063182in}{0.761618in}}{\pgfqpoint{1.055369in}{0.769431in}}%
\pgfpathcurveto{\pgfqpoint{1.047555in}{0.777245in}}{\pgfqpoint{1.036956in}{0.781635in}}{\pgfqpoint{1.025906in}{0.781635in}}%
\pgfpathcurveto{\pgfqpoint{1.014856in}{0.781635in}}{\pgfqpoint{1.004257in}{0.777245in}}{\pgfqpoint{0.996443in}{0.769431in}}%
\pgfpathcurveto{\pgfqpoint{0.988630in}{0.761618in}}{\pgfqpoint{0.984239in}{0.751019in}}{\pgfqpoint{0.984239in}{0.739969in}}%
\pgfpathcurveto{\pgfqpoint{0.984239in}{0.728918in}}{\pgfqpoint{0.988630in}{0.718319in}}{\pgfqpoint{0.996443in}{0.710506in}}%
\pgfpathcurveto{\pgfqpoint{1.004257in}{0.702692in}}{\pgfqpoint{1.014856in}{0.698302in}}{\pgfqpoint{1.025906in}{0.698302in}}%
\pgfpathclose%
\pgfusepath{stroke,fill}%
\end{pgfscope}%
\begin{pgfscope}%
\pgfpathrectangle{\pgfqpoint{0.800000in}{0.528000in}}{\pgfqpoint{4.960000in}{3.696000in}}%
\pgfusepath{clip}%
\pgfsetbuttcap%
\pgfsetroundjoin%
\definecolor{currentfill}{rgb}{0.000000,0.000000,0.000000}%
\pgfsetfillcolor{currentfill}%
\pgfsetlinewidth{1.003750pt}%
\definecolor{currentstroke}{rgb}{0.000000,0.000000,0.000000}%
\pgfsetstrokecolor{currentstroke}%
\pgfsetdash{}{0pt}%
\pgfpathmoveto{\pgfqpoint{1.025906in}{0.719799in}}%
\pgfpathcurveto{\pgfqpoint{1.036956in}{0.719799in}}{\pgfqpoint{1.047555in}{0.724190in}}{\pgfqpoint{1.055369in}{0.732003in}}%
\pgfpathcurveto{\pgfqpoint{1.063182in}{0.739817in}}{\pgfqpoint{1.067573in}{0.750416in}}{\pgfqpoint{1.067573in}{0.761466in}}%
\pgfpathcurveto{\pgfqpoint{1.067573in}{0.772516in}}{\pgfqpoint{1.063182in}{0.783115in}}{\pgfqpoint{1.055369in}{0.790929in}}%
\pgfpathcurveto{\pgfqpoint{1.047555in}{0.798743in}}{\pgfqpoint{1.036956in}{0.803133in}}{\pgfqpoint{1.025906in}{0.803133in}}%
\pgfpathcurveto{\pgfqpoint{1.014856in}{0.803133in}}{\pgfqpoint{1.004257in}{0.798743in}}{\pgfqpoint{0.996443in}{0.790929in}}%
\pgfpathcurveto{\pgfqpoint{0.988630in}{0.783115in}}{\pgfqpoint{0.984239in}{0.772516in}}{\pgfqpoint{0.984239in}{0.761466in}}%
\pgfpathcurveto{\pgfqpoint{0.984239in}{0.750416in}}{\pgfqpoint{0.988630in}{0.739817in}}{\pgfqpoint{0.996443in}{0.732003in}}%
\pgfpathcurveto{\pgfqpoint{1.004257in}{0.724190in}}{\pgfqpoint{1.014856in}{0.719799in}}{\pgfqpoint{1.025906in}{0.719799in}}%
\pgfpathclose%
\pgfusepath{stroke,fill}%
\end{pgfscope}%
\begin{pgfscope}%
\pgfpathrectangle{\pgfqpoint{0.800000in}{0.528000in}}{\pgfqpoint{4.960000in}{3.696000in}}%
\pgfusepath{clip}%
\pgfsetbuttcap%
\pgfsetroundjoin%
\definecolor{currentfill}{rgb}{0.000000,0.000000,0.000000}%
\pgfsetfillcolor{currentfill}%
\pgfsetlinewidth{1.003750pt}%
\definecolor{currentstroke}{rgb}{0.000000,0.000000,0.000000}%
\pgfsetstrokecolor{currentstroke}%
\pgfsetdash{}{0pt}%
\pgfpathmoveto{\pgfqpoint{1.025906in}{0.698302in}}%
\pgfpathcurveto{\pgfqpoint{1.036956in}{0.698302in}}{\pgfqpoint{1.047555in}{0.702692in}}{\pgfqpoint{1.055369in}{0.710506in}}%
\pgfpathcurveto{\pgfqpoint{1.063182in}{0.718319in}}{\pgfqpoint{1.067573in}{0.728918in}}{\pgfqpoint{1.067573in}{0.739969in}}%
\pgfpathcurveto{\pgfqpoint{1.067573in}{0.751019in}}{\pgfqpoint{1.063182in}{0.761618in}}{\pgfqpoint{1.055369in}{0.769431in}}%
\pgfpathcurveto{\pgfqpoint{1.047555in}{0.777245in}}{\pgfqpoint{1.036956in}{0.781635in}}{\pgfqpoint{1.025906in}{0.781635in}}%
\pgfpathcurveto{\pgfqpoint{1.014856in}{0.781635in}}{\pgfqpoint{1.004257in}{0.777245in}}{\pgfqpoint{0.996443in}{0.769431in}}%
\pgfpathcurveto{\pgfqpoint{0.988630in}{0.761618in}}{\pgfqpoint{0.984239in}{0.751019in}}{\pgfqpoint{0.984239in}{0.739969in}}%
\pgfpathcurveto{\pgfqpoint{0.984239in}{0.728918in}}{\pgfqpoint{0.988630in}{0.718319in}}{\pgfqpoint{0.996443in}{0.710506in}}%
\pgfpathcurveto{\pgfqpoint{1.004257in}{0.702692in}}{\pgfqpoint{1.014856in}{0.698302in}}{\pgfqpoint{1.025906in}{0.698302in}}%
\pgfpathclose%
\pgfusepath{stroke,fill}%
\end{pgfscope}%
\begin{pgfscope}%
\pgfpathrectangle{\pgfqpoint{0.800000in}{0.528000in}}{\pgfqpoint{4.960000in}{3.696000in}}%
\pgfusepath{clip}%
\pgfsetbuttcap%
\pgfsetroundjoin%
\definecolor{currentfill}{rgb}{0.000000,0.000000,0.000000}%
\pgfsetfillcolor{currentfill}%
\pgfsetlinewidth{1.003750pt}%
\definecolor{currentstroke}{rgb}{0.000000,0.000000,0.000000}%
\pgfsetstrokecolor{currentstroke}%
\pgfsetdash{}{0pt}%
\pgfpathmoveto{\pgfqpoint{1.025906in}{0.719799in}}%
\pgfpathcurveto{\pgfqpoint{1.036956in}{0.719799in}}{\pgfqpoint{1.047555in}{0.724190in}}{\pgfqpoint{1.055369in}{0.732003in}}%
\pgfpathcurveto{\pgfqpoint{1.063182in}{0.739817in}}{\pgfqpoint{1.067573in}{0.750416in}}{\pgfqpoint{1.067573in}{0.761466in}}%
\pgfpathcurveto{\pgfqpoint{1.067573in}{0.772516in}}{\pgfqpoint{1.063182in}{0.783115in}}{\pgfqpoint{1.055369in}{0.790929in}}%
\pgfpathcurveto{\pgfqpoint{1.047555in}{0.798743in}}{\pgfqpoint{1.036956in}{0.803133in}}{\pgfqpoint{1.025906in}{0.803133in}}%
\pgfpathcurveto{\pgfqpoint{1.014856in}{0.803133in}}{\pgfqpoint{1.004257in}{0.798743in}}{\pgfqpoint{0.996443in}{0.790929in}}%
\pgfpathcurveto{\pgfqpoint{0.988630in}{0.783115in}}{\pgfqpoint{0.984239in}{0.772516in}}{\pgfqpoint{0.984239in}{0.761466in}}%
\pgfpathcurveto{\pgfqpoint{0.984239in}{0.750416in}}{\pgfqpoint{0.988630in}{0.739817in}}{\pgfqpoint{0.996443in}{0.732003in}}%
\pgfpathcurveto{\pgfqpoint{1.004257in}{0.724190in}}{\pgfqpoint{1.014856in}{0.719799in}}{\pgfqpoint{1.025906in}{0.719799in}}%
\pgfpathclose%
\pgfusepath{stroke,fill}%
\end{pgfscope}%
\begin{pgfscope}%
\pgfpathrectangle{\pgfqpoint{0.800000in}{0.528000in}}{\pgfqpoint{4.960000in}{3.696000in}}%
\pgfusepath{clip}%
\pgfsetbuttcap%
\pgfsetroundjoin%
\definecolor{currentfill}{rgb}{0.000000,0.000000,0.000000}%
\pgfsetfillcolor{currentfill}%
\pgfsetlinewidth{1.003750pt}%
\definecolor{currentstroke}{rgb}{0.000000,0.000000,0.000000}%
\pgfsetstrokecolor{currentstroke}%
\pgfsetdash{}{0pt}%
\pgfpathmoveto{\pgfqpoint{1.025906in}{0.698302in}}%
\pgfpathcurveto{\pgfqpoint{1.036956in}{0.698302in}}{\pgfqpoint{1.047555in}{0.702692in}}{\pgfqpoint{1.055369in}{0.710506in}}%
\pgfpathcurveto{\pgfqpoint{1.063182in}{0.718319in}}{\pgfqpoint{1.067573in}{0.728918in}}{\pgfqpoint{1.067573in}{0.739969in}}%
\pgfpathcurveto{\pgfqpoint{1.067573in}{0.751019in}}{\pgfqpoint{1.063182in}{0.761618in}}{\pgfqpoint{1.055369in}{0.769431in}}%
\pgfpathcurveto{\pgfqpoint{1.047555in}{0.777245in}}{\pgfqpoint{1.036956in}{0.781635in}}{\pgfqpoint{1.025906in}{0.781635in}}%
\pgfpathcurveto{\pgfqpoint{1.014856in}{0.781635in}}{\pgfqpoint{1.004257in}{0.777245in}}{\pgfqpoint{0.996443in}{0.769431in}}%
\pgfpathcurveto{\pgfqpoint{0.988630in}{0.761618in}}{\pgfqpoint{0.984239in}{0.751019in}}{\pgfqpoint{0.984239in}{0.739969in}}%
\pgfpathcurveto{\pgfqpoint{0.984239in}{0.728918in}}{\pgfqpoint{0.988630in}{0.718319in}}{\pgfqpoint{0.996443in}{0.710506in}}%
\pgfpathcurveto{\pgfqpoint{1.004257in}{0.702692in}}{\pgfqpoint{1.014856in}{0.698302in}}{\pgfqpoint{1.025906in}{0.698302in}}%
\pgfpathclose%
\pgfusepath{stroke,fill}%
\end{pgfscope}%
\begin{pgfscope}%
\pgfpathrectangle{\pgfqpoint{0.800000in}{0.528000in}}{\pgfqpoint{4.960000in}{3.696000in}}%
\pgfusepath{clip}%
\pgfsetbuttcap%
\pgfsetroundjoin%
\definecolor{currentfill}{rgb}{0.000000,0.000000,0.000000}%
\pgfsetfillcolor{currentfill}%
\pgfsetlinewidth{1.003750pt}%
\definecolor{currentstroke}{rgb}{0.000000,0.000000,0.000000}%
\pgfsetstrokecolor{currentstroke}%
\pgfsetdash{}{0pt}%
\pgfpathmoveto{\pgfqpoint{1.025906in}{0.698302in}}%
\pgfpathcurveto{\pgfqpoint{1.036956in}{0.698302in}}{\pgfqpoint{1.047555in}{0.702692in}}{\pgfqpoint{1.055369in}{0.710506in}}%
\pgfpathcurveto{\pgfqpoint{1.063182in}{0.718319in}}{\pgfqpoint{1.067573in}{0.728918in}}{\pgfqpoint{1.067573in}{0.739969in}}%
\pgfpathcurveto{\pgfqpoint{1.067573in}{0.751019in}}{\pgfqpoint{1.063182in}{0.761618in}}{\pgfqpoint{1.055369in}{0.769431in}}%
\pgfpathcurveto{\pgfqpoint{1.047555in}{0.777245in}}{\pgfqpoint{1.036956in}{0.781635in}}{\pgfqpoint{1.025906in}{0.781635in}}%
\pgfpathcurveto{\pgfqpoint{1.014856in}{0.781635in}}{\pgfqpoint{1.004257in}{0.777245in}}{\pgfqpoint{0.996443in}{0.769431in}}%
\pgfpathcurveto{\pgfqpoint{0.988630in}{0.761618in}}{\pgfqpoint{0.984239in}{0.751019in}}{\pgfqpoint{0.984239in}{0.739969in}}%
\pgfpathcurveto{\pgfqpoint{0.984239in}{0.728918in}}{\pgfqpoint{0.988630in}{0.718319in}}{\pgfqpoint{0.996443in}{0.710506in}}%
\pgfpathcurveto{\pgfqpoint{1.004257in}{0.702692in}}{\pgfqpoint{1.014856in}{0.698302in}}{\pgfqpoint{1.025906in}{0.698302in}}%
\pgfpathclose%
\pgfusepath{stroke,fill}%
\end{pgfscope}%
\begin{pgfscope}%
\pgfpathrectangle{\pgfqpoint{0.800000in}{0.528000in}}{\pgfqpoint{4.960000in}{3.696000in}}%
\pgfusepath{clip}%
\pgfsetbuttcap%
\pgfsetroundjoin%
\definecolor{currentfill}{rgb}{0.000000,0.000000,0.000000}%
\pgfsetfillcolor{currentfill}%
\pgfsetlinewidth{1.003750pt}%
\definecolor{currentstroke}{rgb}{0.000000,0.000000,0.000000}%
\pgfsetstrokecolor{currentstroke}%
\pgfsetdash{}{0pt}%
\pgfpathmoveto{\pgfqpoint{1.025906in}{0.719799in}}%
\pgfpathcurveto{\pgfqpoint{1.036956in}{0.719799in}}{\pgfqpoint{1.047555in}{0.724190in}}{\pgfqpoint{1.055369in}{0.732003in}}%
\pgfpathcurveto{\pgfqpoint{1.063182in}{0.739817in}}{\pgfqpoint{1.067573in}{0.750416in}}{\pgfqpoint{1.067573in}{0.761466in}}%
\pgfpathcurveto{\pgfqpoint{1.067573in}{0.772516in}}{\pgfqpoint{1.063182in}{0.783115in}}{\pgfqpoint{1.055369in}{0.790929in}}%
\pgfpathcurveto{\pgfqpoint{1.047555in}{0.798743in}}{\pgfqpoint{1.036956in}{0.803133in}}{\pgfqpoint{1.025906in}{0.803133in}}%
\pgfpathcurveto{\pgfqpoint{1.014856in}{0.803133in}}{\pgfqpoint{1.004257in}{0.798743in}}{\pgfqpoint{0.996443in}{0.790929in}}%
\pgfpathcurveto{\pgfqpoint{0.988630in}{0.783115in}}{\pgfqpoint{0.984239in}{0.772516in}}{\pgfqpoint{0.984239in}{0.761466in}}%
\pgfpathcurveto{\pgfqpoint{0.984239in}{0.750416in}}{\pgfqpoint{0.988630in}{0.739817in}}{\pgfqpoint{0.996443in}{0.732003in}}%
\pgfpathcurveto{\pgfqpoint{1.004257in}{0.724190in}}{\pgfqpoint{1.014856in}{0.719799in}}{\pgfqpoint{1.025906in}{0.719799in}}%
\pgfpathclose%
\pgfusepath{stroke,fill}%
\end{pgfscope}%
\begin{pgfscope}%
\pgfpathrectangle{\pgfqpoint{0.800000in}{0.528000in}}{\pgfqpoint{4.960000in}{3.696000in}}%
\pgfusepath{clip}%
\pgfsetbuttcap%
\pgfsetroundjoin%
\definecolor{currentfill}{rgb}{0.000000,0.000000,0.000000}%
\pgfsetfillcolor{currentfill}%
\pgfsetlinewidth{1.003750pt}%
\definecolor{currentstroke}{rgb}{0.000000,0.000000,0.000000}%
\pgfsetstrokecolor{currentstroke}%
\pgfsetdash{}{0pt}%
\pgfpathmoveto{\pgfqpoint{1.025906in}{0.698302in}}%
\pgfpathcurveto{\pgfqpoint{1.036956in}{0.698302in}}{\pgfqpoint{1.047555in}{0.702692in}}{\pgfqpoint{1.055369in}{0.710506in}}%
\pgfpathcurveto{\pgfqpoint{1.063182in}{0.718319in}}{\pgfqpoint{1.067573in}{0.728918in}}{\pgfqpoint{1.067573in}{0.739969in}}%
\pgfpathcurveto{\pgfqpoint{1.067573in}{0.751019in}}{\pgfqpoint{1.063182in}{0.761618in}}{\pgfqpoint{1.055369in}{0.769431in}}%
\pgfpathcurveto{\pgfqpoint{1.047555in}{0.777245in}}{\pgfqpoint{1.036956in}{0.781635in}}{\pgfqpoint{1.025906in}{0.781635in}}%
\pgfpathcurveto{\pgfqpoint{1.014856in}{0.781635in}}{\pgfqpoint{1.004257in}{0.777245in}}{\pgfqpoint{0.996443in}{0.769431in}}%
\pgfpathcurveto{\pgfqpoint{0.988630in}{0.761618in}}{\pgfqpoint{0.984239in}{0.751019in}}{\pgfqpoint{0.984239in}{0.739969in}}%
\pgfpathcurveto{\pgfqpoint{0.984239in}{0.728918in}}{\pgfqpoint{0.988630in}{0.718319in}}{\pgfqpoint{0.996443in}{0.710506in}}%
\pgfpathcurveto{\pgfqpoint{1.004257in}{0.702692in}}{\pgfqpoint{1.014856in}{0.698302in}}{\pgfqpoint{1.025906in}{0.698302in}}%
\pgfpathclose%
\pgfusepath{stroke,fill}%
\end{pgfscope}%
\begin{pgfscope}%
\pgfpathrectangle{\pgfqpoint{0.800000in}{0.528000in}}{\pgfqpoint{4.960000in}{3.696000in}}%
\pgfusepath{clip}%
\pgfsetbuttcap%
\pgfsetroundjoin%
\definecolor{currentfill}{rgb}{0.000000,0.000000,0.000000}%
\pgfsetfillcolor{currentfill}%
\pgfsetlinewidth{1.003750pt}%
\definecolor{currentstroke}{rgb}{0.000000,0.000000,0.000000}%
\pgfsetstrokecolor{currentstroke}%
\pgfsetdash{}{0pt}%
\pgfpathmoveto{\pgfqpoint{1.025906in}{0.676804in}}%
\pgfpathcurveto{\pgfqpoint{1.036956in}{0.676804in}}{\pgfqpoint{1.047555in}{0.681195in}}{\pgfqpoint{1.055369in}{0.689008in}}%
\pgfpathcurveto{\pgfqpoint{1.063182in}{0.696822in}}{\pgfqpoint{1.067573in}{0.707421in}}{\pgfqpoint{1.067573in}{0.718471in}}%
\pgfpathcurveto{\pgfqpoint{1.067573in}{0.729521in}}{\pgfqpoint{1.063182in}{0.740120in}}{\pgfqpoint{1.055369in}{0.747934in}}%
\pgfpathcurveto{\pgfqpoint{1.047555in}{0.755748in}}{\pgfqpoint{1.036956in}{0.760138in}}{\pgfqpoint{1.025906in}{0.760138in}}%
\pgfpathcurveto{\pgfqpoint{1.014856in}{0.760138in}}{\pgfqpoint{1.004257in}{0.755748in}}{\pgfqpoint{0.996443in}{0.747934in}}%
\pgfpathcurveto{\pgfqpoint{0.988630in}{0.740120in}}{\pgfqpoint{0.984239in}{0.729521in}}{\pgfqpoint{0.984239in}{0.718471in}}%
\pgfpathcurveto{\pgfqpoint{0.984239in}{0.707421in}}{\pgfqpoint{0.988630in}{0.696822in}}{\pgfqpoint{0.996443in}{0.689008in}}%
\pgfpathcurveto{\pgfqpoint{1.004257in}{0.681195in}}{\pgfqpoint{1.014856in}{0.676804in}}{\pgfqpoint{1.025906in}{0.676804in}}%
\pgfpathclose%
\pgfusepath{stroke,fill}%
\end{pgfscope}%
\begin{pgfscope}%
\pgfpathrectangle{\pgfqpoint{0.800000in}{0.528000in}}{\pgfqpoint{4.960000in}{3.696000in}}%
\pgfusepath{clip}%
\pgfsetbuttcap%
\pgfsetroundjoin%
\definecolor{currentfill}{rgb}{0.000000,0.000000,0.000000}%
\pgfsetfillcolor{currentfill}%
\pgfsetlinewidth{1.003750pt}%
\definecolor{currentstroke}{rgb}{0.000000,0.000000,0.000000}%
\pgfsetstrokecolor{currentstroke}%
\pgfsetdash{}{0pt}%
\pgfpathmoveto{\pgfqpoint{1.025906in}{0.698302in}}%
\pgfpathcurveto{\pgfqpoint{1.036956in}{0.698302in}}{\pgfqpoint{1.047555in}{0.702692in}}{\pgfqpoint{1.055369in}{0.710506in}}%
\pgfpathcurveto{\pgfqpoint{1.063182in}{0.718319in}}{\pgfqpoint{1.067573in}{0.728918in}}{\pgfqpoint{1.067573in}{0.739969in}}%
\pgfpathcurveto{\pgfqpoint{1.067573in}{0.751019in}}{\pgfqpoint{1.063182in}{0.761618in}}{\pgfqpoint{1.055369in}{0.769431in}}%
\pgfpathcurveto{\pgfqpoint{1.047555in}{0.777245in}}{\pgfqpoint{1.036956in}{0.781635in}}{\pgfqpoint{1.025906in}{0.781635in}}%
\pgfpathcurveto{\pgfqpoint{1.014856in}{0.781635in}}{\pgfqpoint{1.004257in}{0.777245in}}{\pgfqpoint{0.996443in}{0.769431in}}%
\pgfpathcurveto{\pgfqpoint{0.988630in}{0.761618in}}{\pgfqpoint{0.984239in}{0.751019in}}{\pgfqpoint{0.984239in}{0.739969in}}%
\pgfpathcurveto{\pgfqpoint{0.984239in}{0.728918in}}{\pgfqpoint{0.988630in}{0.718319in}}{\pgfqpoint{0.996443in}{0.710506in}}%
\pgfpathcurveto{\pgfqpoint{1.004257in}{0.702692in}}{\pgfqpoint{1.014856in}{0.698302in}}{\pgfqpoint{1.025906in}{0.698302in}}%
\pgfpathclose%
\pgfusepath{stroke,fill}%
\end{pgfscope}%
\begin{pgfscope}%
\pgfpathrectangle{\pgfqpoint{0.800000in}{0.528000in}}{\pgfqpoint{4.960000in}{3.696000in}}%
\pgfusepath{clip}%
\pgfsetbuttcap%
\pgfsetroundjoin%
\definecolor{currentfill}{rgb}{0.000000,0.000000,0.000000}%
\pgfsetfillcolor{currentfill}%
\pgfsetlinewidth{1.003750pt}%
\definecolor{currentstroke}{rgb}{0.000000,0.000000,0.000000}%
\pgfsetstrokecolor{currentstroke}%
\pgfsetdash{}{0pt}%
\pgfpathmoveto{\pgfqpoint{1.025906in}{0.698302in}}%
\pgfpathcurveto{\pgfqpoint{1.036956in}{0.698302in}}{\pgfqpoint{1.047555in}{0.702692in}}{\pgfqpoint{1.055369in}{0.710506in}}%
\pgfpathcurveto{\pgfqpoint{1.063182in}{0.718319in}}{\pgfqpoint{1.067573in}{0.728918in}}{\pgfqpoint{1.067573in}{0.739969in}}%
\pgfpathcurveto{\pgfqpoint{1.067573in}{0.751019in}}{\pgfqpoint{1.063182in}{0.761618in}}{\pgfqpoint{1.055369in}{0.769431in}}%
\pgfpathcurveto{\pgfqpoint{1.047555in}{0.777245in}}{\pgfqpoint{1.036956in}{0.781635in}}{\pgfqpoint{1.025906in}{0.781635in}}%
\pgfpathcurveto{\pgfqpoint{1.014856in}{0.781635in}}{\pgfqpoint{1.004257in}{0.777245in}}{\pgfqpoint{0.996443in}{0.769431in}}%
\pgfpathcurveto{\pgfqpoint{0.988630in}{0.761618in}}{\pgfqpoint{0.984239in}{0.751019in}}{\pgfqpoint{0.984239in}{0.739969in}}%
\pgfpathcurveto{\pgfqpoint{0.984239in}{0.728918in}}{\pgfqpoint{0.988630in}{0.718319in}}{\pgfqpoint{0.996443in}{0.710506in}}%
\pgfpathcurveto{\pgfqpoint{1.004257in}{0.702692in}}{\pgfqpoint{1.014856in}{0.698302in}}{\pgfqpoint{1.025906in}{0.698302in}}%
\pgfpathclose%
\pgfusepath{stroke,fill}%
\end{pgfscope}%
\begin{pgfscope}%
\pgfpathrectangle{\pgfqpoint{0.800000in}{0.528000in}}{\pgfqpoint{4.960000in}{3.696000in}}%
\pgfusepath{clip}%
\pgfsetbuttcap%
\pgfsetroundjoin%
\definecolor{currentfill}{rgb}{0.000000,0.000000,0.000000}%
\pgfsetfillcolor{currentfill}%
\pgfsetlinewidth{1.003750pt}%
\definecolor{currentstroke}{rgb}{0.000000,0.000000,0.000000}%
\pgfsetstrokecolor{currentstroke}%
\pgfsetdash{}{0pt}%
\pgfpathmoveto{\pgfqpoint{1.025906in}{0.698302in}}%
\pgfpathcurveto{\pgfqpoint{1.036956in}{0.698302in}}{\pgfqpoint{1.047555in}{0.702692in}}{\pgfqpoint{1.055369in}{0.710506in}}%
\pgfpathcurveto{\pgfqpoint{1.063182in}{0.718319in}}{\pgfqpoint{1.067573in}{0.728918in}}{\pgfqpoint{1.067573in}{0.739969in}}%
\pgfpathcurveto{\pgfqpoint{1.067573in}{0.751019in}}{\pgfqpoint{1.063182in}{0.761618in}}{\pgfqpoint{1.055369in}{0.769431in}}%
\pgfpathcurveto{\pgfqpoint{1.047555in}{0.777245in}}{\pgfqpoint{1.036956in}{0.781635in}}{\pgfqpoint{1.025906in}{0.781635in}}%
\pgfpathcurveto{\pgfqpoint{1.014856in}{0.781635in}}{\pgfqpoint{1.004257in}{0.777245in}}{\pgfqpoint{0.996443in}{0.769431in}}%
\pgfpathcurveto{\pgfqpoint{0.988630in}{0.761618in}}{\pgfqpoint{0.984239in}{0.751019in}}{\pgfqpoint{0.984239in}{0.739969in}}%
\pgfpathcurveto{\pgfqpoint{0.984239in}{0.728918in}}{\pgfqpoint{0.988630in}{0.718319in}}{\pgfqpoint{0.996443in}{0.710506in}}%
\pgfpathcurveto{\pgfqpoint{1.004257in}{0.702692in}}{\pgfqpoint{1.014856in}{0.698302in}}{\pgfqpoint{1.025906in}{0.698302in}}%
\pgfpathclose%
\pgfusepath{stroke,fill}%
\end{pgfscope}%
\begin{pgfscope}%
\pgfpathrectangle{\pgfqpoint{0.800000in}{0.528000in}}{\pgfqpoint{4.960000in}{3.696000in}}%
\pgfusepath{clip}%
\pgfsetbuttcap%
\pgfsetroundjoin%
\definecolor{currentfill}{rgb}{0.000000,0.000000,0.000000}%
\pgfsetfillcolor{currentfill}%
\pgfsetlinewidth{1.003750pt}%
\definecolor{currentstroke}{rgb}{0.000000,0.000000,0.000000}%
\pgfsetstrokecolor{currentstroke}%
\pgfsetdash{}{0pt}%
\pgfpathmoveto{\pgfqpoint{1.025906in}{0.655307in}}%
\pgfpathcurveto{\pgfqpoint{1.036956in}{0.655307in}}{\pgfqpoint{1.047555in}{0.659697in}}{\pgfqpoint{1.055369in}{0.667511in}}%
\pgfpathcurveto{\pgfqpoint{1.063182in}{0.675324in}}{\pgfqpoint{1.067573in}{0.685924in}}{\pgfqpoint{1.067573in}{0.696974in}}%
\pgfpathcurveto{\pgfqpoint{1.067573in}{0.708024in}}{\pgfqpoint{1.063182in}{0.718623in}}{\pgfqpoint{1.055369in}{0.726436in}}%
\pgfpathcurveto{\pgfqpoint{1.047555in}{0.734250in}}{\pgfqpoint{1.036956in}{0.738640in}}{\pgfqpoint{1.025906in}{0.738640in}}%
\pgfpathcurveto{\pgfqpoint{1.014856in}{0.738640in}}{\pgfqpoint{1.004257in}{0.734250in}}{\pgfqpoint{0.996443in}{0.726436in}}%
\pgfpathcurveto{\pgfqpoint{0.988630in}{0.718623in}}{\pgfqpoint{0.984239in}{0.708024in}}{\pgfqpoint{0.984239in}{0.696974in}}%
\pgfpathcurveto{\pgfqpoint{0.984239in}{0.685924in}}{\pgfqpoint{0.988630in}{0.675324in}}{\pgfqpoint{0.996443in}{0.667511in}}%
\pgfpathcurveto{\pgfqpoint{1.004257in}{0.659697in}}{\pgfqpoint{1.014856in}{0.655307in}}{\pgfqpoint{1.025906in}{0.655307in}}%
\pgfpathclose%
\pgfusepath{stroke,fill}%
\end{pgfscope}%
\begin{pgfscope}%
\pgfpathrectangle{\pgfqpoint{0.800000in}{0.528000in}}{\pgfqpoint{4.960000in}{3.696000in}}%
\pgfusepath{clip}%
\pgfsetbuttcap%
\pgfsetroundjoin%
\definecolor{currentfill}{rgb}{0.000000,0.000000,0.000000}%
\pgfsetfillcolor{currentfill}%
\pgfsetlinewidth{1.003750pt}%
\definecolor{currentstroke}{rgb}{0.000000,0.000000,0.000000}%
\pgfsetstrokecolor{currentstroke}%
\pgfsetdash{}{0pt}%
\pgfpathmoveto{\pgfqpoint{1.025906in}{0.719799in}}%
\pgfpathcurveto{\pgfqpoint{1.036956in}{0.719799in}}{\pgfqpoint{1.047555in}{0.724190in}}{\pgfqpoint{1.055369in}{0.732003in}}%
\pgfpathcurveto{\pgfqpoint{1.063182in}{0.739817in}}{\pgfqpoint{1.067573in}{0.750416in}}{\pgfqpoint{1.067573in}{0.761466in}}%
\pgfpathcurveto{\pgfqpoint{1.067573in}{0.772516in}}{\pgfqpoint{1.063182in}{0.783115in}}{\pgfqpoint{1.055369in}{0.790929in}}%
\pgfpathcurveto{\pgfqpoint{1.047555in}{0.798743in}}{\pgfqpoint{1.036956in}{0.803133in}}{\pgfqpoint{1.025906in}{0.803133in}}%
\pgfpathcurveto{\pgfqpoint{1.014856in}{0.803133in}}{\pgfqpoint{1.004257in}{0.798743in}}{\pgfqpoint{0.996443in}{0.790929in}}%
\pgfpathcurveto{\pgfqpoint{0.988630in}{0.783115in}}{\pgfqpoint{0.984239in}{0.772516in}}{\pgfqpoint{0.984239in}{0.761466in}}%
\pgfpathcurveto{\pgfqpoint{0.984239in}{0.750416in}}{\pgfqpoint{0.988630in}{0.739817in}}{\pgfqpoint{0.996443in}{0.732003in}}%
\pgfpathcurveto{\pgfqpoint{1.004257in}{0.724190in}}{\pgfqpoint{1.014856in}{0.719799in}}{\pgfqpoint{1.025906in}{0.719799in}}%
\pgfpathclose%
\pgfusepath{stroke,fill}%
\end{pgfscope}%
\begin{pgfscope}%
\pgfpathrectangle{\pgfqpoint{0.800000in}{0.528000in}}{\pgfqpoint{4.960000in}{3.696000in}}%
\pgfusepath{clip}%
\pgfsetbuttcap%
\pgfsetroundjoin%
\definecolor{currentfill}{rgb}{0.000000,0.000000,0.000000}%
\pgfsetfillcolor{currentfill}%
\pgfsetlinewidth{1.003750pt}%
\definecolor{currentstroke}{rgb}{0.000000,0.000000,0.000000}%
\pgfsetstrokecolor{currentstroke}%
\pgfsetdash{}{0pt}%
\pgfpathmoveto{\pgfqpoint{1.025906in}{0.655307in}}%
\pgfpathcurveto{\pgfqpoint{1.036956in}{0.655307in}}{\pgfqpoint{1.047555in}{0.659697in}}{\pgfqpoint{1.055369in}{0.667511in}}%
\pgfpathcurveto{\pgfqpoint{1.063182in}{0.675324in}}{\pgfqpoint{1.067573in}{0.685924in}}{\pgfqpoint{1.067573in}{0.696974in}}%
\pgfpathcurveto{\pgfqpoint{1.067573in}{0.708024in}}{\pgfqpoint{1.063182in}{0.718623in}}{\pgfqpoint{1.055369in}{0.726436in}}%
\pgfpathcurveto{\pgfqpoint{1.047555in}{0.734250in}}{\pgfqpoint{1.036956in}{0.738640in}}{\pgfqpoint{1.025906in}{0.738640in}}%
\pgfpathcurveto{\pgfqpoint{1.014856in}{0.738640in}}{\pgfqpoint{1.004257in}{0.734250in}}{\pgfqpoint{0.996443in}{0.726436in}}%
\pgfpathcurveto{\pgfqpoint{0.988630in}{0.718623in}}{\pgfqpoint{0.984239in}{0.708024in}}{\pgfqpoint{0.984239in}{0.696974in}}%
\pgfpathcurveto{\pgfqpoint{0.984239in}{0.685924in}}{\pgfqpoint{0.988630in}{0.675324in}}{\pgfqpoint{0.996443in}{0.667511in}}%
\pgfpathcurveto{\pgfqpoint{1.004257in}{0.659697in}}{\pgfqpoint{1.014856in}{0.655307in}}{\pgfqpoint{1.025906in}{0.655307in}}%
\pgfpathclose%
\pgfusepath{stroke,fill}%
\end{pgfscope}%
\begin{pgfscope}%
\pgfpathrectangle{\pgfqpoint{0.800000in}{0.528000in}}{\pgfqpoint{4.960000in}{3.696000in}}%
\pgfusepath{clip}%
\pgfsetbuttcap%
\pgfsetroundjoin%
\definecolor{currentfill}{rgb}{0.000000,0.000000,0.000000}%
\pgfsetfillcolor{currentfill}%
\pgfsetlinewidth{1.003750pt}%
\definecolor{currentstroke}{rgb}{0.000000,0.000000,0.000000}%
\pgfsetstrokecolor{currentstroke}%
\pgfsetdash{}{0pt}%
\pgfpathmoveto{\pgfqpoint{1.025906in}{0.676804in}}%
\pgfpathcurveto{\pgfqpoint{1.036956in}{0.676804in}}{\pgfqpoint{1.047555in}{0.681195in}}{\pgfqpoint{1.055369in}{0.689008in}}%
\pgfpathcurveto{\pgfqpoint{1.063182in}{0.696822in}}{\pgfqpoint{1.067573in}{0.707421in}}{\pgfqpoint{1.067573in}{0.718471in}}%
\pgfpathcurveto{\pgfqpoint{1.067573in}{0.729521in}}{\pgfqpoint{1.063182in}{0.740120in}}{\pgfqpoint{1.055369in}{0.747934in}}%
\pgfpathcurveto{\pgfqpoint{1.047555in}{0.755748in}}{\pgfqpoint{1.036956in}{0.760138in}}{\pgfqpoint{1.025906in}{0.760138in}}%
\pgfpathcurveto{\pgfqpoint{1.014856in}{0.760138in}}{\pgfqpoint{1.004257in}{0.755748in}}{\pgfqpoint{0.996443in}{0.747934in}}%
\pgfpathcurveto{\pgfqpoint{0.988630in}{0.740120in}}{\pgfqpoint{0.984239in}{0.729521in}}{\pgfqpoint{0.984239in}{0.718471in}}%
\pgfpathcurveto{\pgfqpoint{0.984239in}{0.707421in}}{\pgfqpoint{0.988630in}{0.696822in}}{\pgfqpoint{0.996443in}{0.689008in}}%
\pgfpathcurveto{\pgfqpoint{1.004257in}{0.681195in}}{\pgfqpoint{1.014856in}{0.676804in}}{\pgfqpoint{1.025906in}{0.676804in}}%
\pgfpathclose%
\pgfusepath{stroke,fill}%
\end{pgfscope}%
\begin{pgfscope}%
\pgfpathrectangle{\pgfqpoint{0.800000in}{0.528000in}}{\pgfqpoint{4.960000in}{3.696000in}}%
\pgfusepath{clip}%
\pgfsetbuttcap%
\pgfsetroundjoin%
\definecolor{currentfill}{rgb}{0.000000,0.000000,0.000000}%
\pgfsetfillcolor{currentfill}%
\pgfsetlinewidth{1.003750pt}%
\definecolor{currentstroke}{rgb}{0.000000,0.000000,0.000000}%
\pgfsetstrokecolor{currentstroke}%
\pgfsetdash{}{0pt}%
\pgfpathmoveto{\pgfqpoint{1.025906in}{0.698302in}}%
\pgfpathcurveto{\pgfqpoint{1.036956in}{0.698302in}}{\pgfqpoint{1.047555in}{0.702692in}}{\pgfqpoint{1.055369in}{0.710506in}}%
\pgfpathcurveto{\pgfqpoint{1.063182in}{0.718319in}}{\pgfqpoint{1.067573in}{0.728918in}}{\pgfqpoint{1.067573in}{0.739969in}}%
\pgfpathcurveto{\pgfqpoint{1.067573in}{0.751019in}}{\pgfqpoint{1.063182in}{0.761618in}}{\pgfqpoint{1.055369in}{0.769431in}}%
\pgfpathcurveto{\pgfqpoint{1.047555in}{0.777245in}}{\pgfqpoint{1.036956in}{0.781635in}}{\pgfqpoint{1.025906in}{0.781635in}}%
\pgfpathcurveto{\pgfqpoint{1.014856in}{0.781635in}}{\pgfqpoint{1.004257in}{0.777245in}}{\pgfqpoint{0.996443in}{0.769431in}}%
\pgfpathcurveto{\pgfqpoint{0.988630in}{0.761618in}}{\pgfqpoint{0.984239in}{0.751019in}}{\pgfqpoint{0.984239in}{0.739969in}}%
\pgfpathcurveto{\pgfqpoint{0.984239in}{0.728918in}}{\pgfqpoint{0.988630in}{0.718319in}}{\pgfqpoint{0.996443in}{0.710506in}}%
\pgfpathcurveto{\pgfqpoint{1.004257in}{0.702692in}}{\pgfqpoint{1.014856in}{0.698302in}}{\pgfqpoint{1.025906in}{0.698302in}}%
\pgfpathclose%
\pgfusepath{stroke,fill}%
\end{pgfscope}%
\begin{pgfscope}%
\pgfpathrectangle{\pgfqpoint{0.800000in}{0.528000in}}{\pgfqpoint{4.960000in}{3.696000in}}%
\pgfusepath{clip}%
\pgfsetbuttcap%
\pgfsetroundjoin%
\definecolor{currentfill}{rgb}{0.000000,0.000000,0.000000}%
\pgfsetfillcolor{currentfill}%
\pgfsetlinewidth{1.003750pt}%
\definecolor{currentstroke}{rgb}{0.000000,0.000000,0.000000}%
\pgfsetstrokecolor{currentstroke}%
\pgfsetdash{}{0pt}%
\pgfpathmoveto{\pgfqpoint{1.025906in}{0.676804in}}%
\pgfpathcurveto{\pgfqpoint{1.036956in}{0.676804in}}{\pgfqpoint{1.047555in}{0.681195in}}{\pgfqpoint{1.055369in}{0.689008in}}%
\pgfpathcurveto{\pgfqpoint{1.063182in}{0.696822in}}{\pgfqpoint{1.067573in}{0.707421in}}{\pgfqpoint{1.067573in}{0.718471in}}%
\pgfpathcurveto{\pgfqpoint{1.067573in}{0.729521in}}{\pgfqpoint{1.063182in}{0.740120in}}{\pgfqpoint{1.055369in}{0.747934in}}%
\pgfpathcurveto{\pgfqpoint{1.047555in}{0.755748in}}{\pgfqpoint{1.036956in}{0.760138in}}{\pgfqpoint{1.025906in}{0.760138in}}%
\pgfpathcurveto{\pgfqpoint{1.014856in}{0.760138in}}{\pgfqpoint{1.004257in}{0.755748in}}{\pgfqpoint{0.996443in}{0.747934in}}%
\pgfpathcurveto{\pgfqpoint{0.988630in}{0.740120in}}{\pgfqpoint{0.984239in}{0.729521in}}{\pgfqpoint{0.984239in}{0.718471in}}%
\pgfpathcurveto{\pgfqpoint{0.984239in}{0.707421in}}{\pgfqpoint{0.988630in}{0.696822in}}{\pgfqpoint{0.996443in}{0.689008in}}%
\pgfpathcurveto{\pgfqpoint{1.004257in}{0.681195in}}{\pgfqpoint{1.014856in}{0.676804in}}{\pgfqpoint{1.025906in}{0.676804in}}%
\pgfpathclose%
\pgfusepath{stroke,fill}%
\end{pgfscope}%
\begin{pgfscope}%
\pgfpathrectangle{\pgfqpoint{0.800000in}{0.528000in}}{\pgfqpoint{4.960000in}{3.696000in}}%
\pgfusepath{clip}%
\pgfsetbuttcap%
\pgfsetroundjoin%
\definecolor{currentfill}{rgb}{0.000000,0.000000,0.000000}%
\pgfsetfillcolor{currentfill}%
\pgfsetlinewidth{1.003750pt}%
\definecolor{currentstroke}{rgb}{0.000000,0.000000,0.000000}%
\pgfsetstrokecolor{currentstroke}%
\pgfsetdash{}{0pt}%
\pgfpathmoveto{\pgfqpoint{2.518786in}{1.171247in}}%
\pgfpathcurveto{\pgfqpoint{2.529836in}{1.171247in}}{\pgfqpoint{2.540435in}{1.175637in}}{\pgfqpoint{2.548249in}{1.183451in}}%
\pgfpathcurveto{\pgfqpoint{2.556062in}{1.191264in}}{\pgfqpoint{2.560452in}{1.201863in}}{\pgfqpoint{2.560452in}{1.212913in}}%
\pgfpathcurveto{\pgfqpoint{2.560452in}{1.223964in}}{\pgfqpoint{2.556062in}{1.234563in}}{\pgfqpoint{2.548249in}{1.242376in}}%
\pgfpathcurveto{\pgfqpoint{2.540435in}{1.250190in}}{\pgfqpoint{2.529836in}{1.254580in}}{\pgfqpoint{2.518786in}{1.254580in}}%
\pgfpathcurveto{\pgfqpoint{2.507736in}{1.254580in}}{\pgfqpoint{2.497137in}{1.250190in}}{\pgfqpoint{2.489323in}{1.242376in}}%
\pgfpathcurveto{\pgfqpoint{2.481509in}{1.234563in}}{\pgfqpoint{2.477119in}{1.223964in}}{\pgfqpoint{2.477119in}{1.212913in}}%
\pgfpathcurveto{\pgfqpoint{2.477119in}{1.201863in}}{\pgfqpoint{2.481509in}{1.191264in}}{\pgfqpoint{2.489323in}{1.183451in}}%
\pgfpathcurveto{\pgfqpoint{2.497137in}{1.175637in}}{\pgfqpoint{2.507736in}{1.171247in}}{\pgfqpoint{2.518786in}{1.171247in}}%
\pgfpathclose%
\pgfusepath{stroke,fill}%
\end{pgfscope}%
\begin{pgfscope}%
\pgfpathrectangle{\pgfqpoint{0.800000in}{0.528000in}}{\pgfqpoint{4.960000in}{3.696000in}}%
\pgfusepath{clip}%
\pgfsetbuttcap%
\pgfsetroundjoin%
\definecolor{currentfill}{rgb}{0.000000,0.000000,0.000000}%
\pgfsetfillcolor{currentfill}%
\pgfsetlinewidth{1.003750pt}%
\definecolor{currentstroke}{rgb}{0.000000,0.000000,0.000000}%
\pgfsetstrokecolor{currentstroke}%
\pgfsetdash{}{0pt}%
\pgfpathmoveto{\pgfqpoint{2.518786in}{1.214242in}}%
\pgfpathcurveto{\pgfqpoint{2.529836in}{1.214242in}}{\pgfqpoint{2.540435in}{1.218632in}}{\pgfqpoint{2.548249in}{1.226446in}}%
\pgfpathcurveto{\pgfqpoint{2.556062in}{1.234259in}}{\pgfqpoint{2.560452in}{1.244858in}}{\pgfqpoint{2.560452in}{1.255908in}}%
\pgfpathcurveto{\pgfqpoint{2.560452in}{1.266959in}}{\pgfqpoint{2.556062in}{1.277558in}}{\pgfqpoint{2.548249in}{1.285371in}}%
\pgfpathcurveto{\pgfqpoint{2.540435in}{1.293185in}}{\pgfqpoint{2.529836in}{1.297575in}}{\pgfqpoint{2.518786in}{1.297575in}}%
\pgfpathcurveto{\pgfqpoint{2.507736in}{1.297575in}}{\pgfqpoint{2.497137in}{1.293185in}}{\pgfqpoint{2.489323in}{1.285371in}}%
\pgfpathcurveto{\pgfqpoint{2.481509in}{1.277558in}}{\pgfqpoint{2.477119in}{1.266959in}}{\pgfqpoint{2.477119in}{1.255908in}}%
\pgfpathcurveto{\pgfqpoint{2.477119in}{1.244858in}}{\pgfqpoint{2.481509in}{1.234259in}}{\pgfqpoint{2.489323in}{1.226446in}}%
\pgfpathcurveto{\pgfqpoint{2.497137in}{1.218632in}}{\pgfqpoint{2.507736in}{1.214242in}}{\pgfqpoint{2.518786in}{1.214242in}}%
\pgfpathclose%
\pgfusepath{stroke,fill}%
\end{pgfscope}%
\begin{pgfscope}%
\pgfpathrectangle{\pgfqpoint{0.800000in}{0.528000in}}{\pgfqpoint{4.960000in}{3.696000in}}%
\pgfusepath{clip}%
\pgfsetbuttcap%
\pgfsetroundjoin%
\definecolor{currentfill}{rgb}{0.000000,0.000000,0.000000}%
\pgfsetfillcolor{currentfill}%
\pgfsetlinewidth{1.003750pt}%
\definecolor{currentstroke}{rgb}{0.000000,0.000000,0.000000}%
\pgfsetstrokecolor{currentstroke}%
\pgfsetdash{}{0pt}%
\pgfpathmoveto{\pgfqpoint{2.518786in}{1.106754in}}%
\pgfpathcurveto{\pgfqpoint{2.529836in}{1.106754in}}{\pgfqpoint{2.540435in}{1.111145in}}{\pgfqpoint{2.548249in}{1.118958in}}%
\pgfpathcurveto{\pgfqpoint{2.556062in}{1.126772in}}{\pgfqpoint{2.560452in}{1.137371in}}{\pgfqpoint{2.560452in}{1.148421in}}%
\pgfpathcurveto{\pgfqpoint{2.560452in}{1.159471in}}{\pgfqpoint{2.556062in}{1.170070in}}{\pgfqpoint{2.548249in}{1.177884in}}%
\pgfpathcurveto{\pgfqpoint{2.540435in}{1.185697in}}{\pgfqpoint{2.529836in}{1.190088in}}{\pgfqpoint{2.518786in}{1.190088in}}%
\pgfpathcurveto{\pgfqpoint{2.507736in}{1.190088in}}{\pgfqpoint{2.497137in}{1.185697in}}{\pgfqpoint{2.489323in}{1.177884in}}%
\pgfpathcurveto{\pgfqpoint{2.481509in}{1.170070in}}{\pgfqpoint{2.477119in}{1.159471in}}{\pgfqpoint{2.477119in}{1.148421in}}%
\pgfpathcurveto{\pgfqpoint{2.477119in}{1.137371in}}{\pgfqpoint{2.481509in}{1.126772in}}{\pgfqpoint{2.489323in}{1.118958in}}%
\pgfpathcurveto{\pgfqpoint{2.497137in}{1.111145in}}{\pgfqpoint{2.507736in}{1.106754in}}{\pgfqpoint{2.518786in}{1.106754in}}%
\pgfpathclose%
\pgfusepath{stroke,fill}%
\end{pgfscope}%
\begin{pgfscope}%
\pgfpathrectangle{\pgfqpoint{0.800000in}{0.528000in}}{\pgfqpoint{4.960000in}{3.696000in}}%
\pgfusepath{clip}%
\pgfsetbuttcap%
\pgfsetroundjoin%
\definecolor{currentfill}{rgb}{0.000000,0.000000,0.000000}%
\pgfsetfillcolor{currentfill}%
\pgfsetlinewidth{1.003750pt}%
\definecolor{currentstroke}{rgb}{0.000000,0.000000,0.000000}%
\pgfsetstrokecolor{currentstroke}%
\pgfsetdash{}{0pt}%
\pgfpathmoveto{\pgfqpoint{2.518786in}{1.085257in}}%
\pgfpathcurveto{\pgfqpoint{2.529836in}{1.085257in}}{\pgfqpoint{2.540435in}{1.089647in}}{\pgfqpoint{2.548249in}{1.097461in}}%
\pgfpathcurveto{\pgfqpoint{2.556062in}{1.105274in}}{\pgfqpoint{2.560452in}{1.115873in}}{\pgfqpoint{2.560452in}{1.126923in}}%
\pgfpathcurveto{\pgfqpoint{2.560452in}{1.137974in}}{\pgfqpoint{2.556062in}{1.148573in}}{\pgfqpoint{2.548249in}{1.156386in}}%
\pgfpathcurveto{\pgfqpoint{2.540435in}{1.164200in}}{\pgfqpoint{2.529836in}{1.168590in}}{\pgfqpoint{2.518786in}{1.168590in}}%
\pgfpathcurveto{\pgfqpoint{2.507736in}{1.168590in}}{\pgfqpoint{2.497137in}{1.164200in}}{\pgfqpoint{2.489323in}{1.156386in}}%
\pgfpathcurveto{\pgfqpoint{2.481509in}{1.148573in}}{\pgfqpoint{2.477119in}{1.137974in}}{\pgfqpoint{2.477119in}{1.126923in}}%
\pgfpathcurveto{\pgfqpoint{2.477119in}{1.115873in}}{\pgfqpoint{2.481509in}{1.105274in}}{\pgfqpoint{2.489323in}{1.097461in}}%
\pgfpathcurveto{\pgfqpoint{2.497137in}{1.089647in}}{\pgfqpoint{2.507736in}{1.085257in}}{\pgfqpoint{2.518786in}{1.085257in}}%
\pgfpathclose%
\pgfusepath{stroke,fill}%
\end{pgfscope}%
\begin{pgfscope}%
\pgfpathrectangle{\pgfqpoint{0.800000in}{0.528000in}}{\pgfqpoint{4.960000in}{3.696000in}}%
\pgfusepath{clip}%
\pgfsetbuttcap%
\pgfsetroundjoin%
\definecolor{currentfill}{rgb}{0.000000,0.000000,0.000000}%
\pgfsetfillcolor{currentfill}%
\pgfsetlinewidth{1.003750pt}%
\definecolor{currentstroke}{rgb}{0.000000,0.000000,0.000000}%
\pgfsetstrokecolor{currentstroke}%
\pgfsetdash{}{0pt}%
\pgfpathmoveto{\pgfqpoint{2.518786in}{1.192744in}}%
\pgfpathcurveto{\pgfqpoint{2.529836in}{1.192744in}}{\pgfqpoint{2.540435in}{1.197135in}}{\pgfqpoint{2.548249in}{1.204948in}}%
\pgfpathcurveto{\pgfqpoint{2.556062in}{1.212762in}}{\pgfqpoint{2.560452in}{1.223361in}}{\pgfqpoint{2.560452in}{1.234411in}}%
\pgfpathcurveto{\pgfqpoint{2.560452in}{1.245461in}}{\pgfqpoint{2.556062in}{1.256060in}}{\pgfqpoint{2.548249in}{1.263874in}}%
\pgfpathcurveto{\pgfqpoint{2.540435in}{1.271687in}}{\pgfqpoint{2.529836in}{1.276078in}}{\pgfqpoint{2.518786in}{1.276078in}}%
\pgfpathcurveto{\pgfqpoint{2.507736in}{1.276078in}}{\pgfqpoint{2.497137in}{1.271687in}}{\pgfqpoint{2.489323in}{1.263874in}}%
\pgfpathcurveto{\pgfqpoint{2.481509in}{1.256060in}}{\pgfqpoint{2.477119in}{1.245461in}}{\pgfqpoint{2.477119in}{1.234411in}}%
\pgfpathcurveto{\pgfqpoint{2.477119in}{1.223361in}}{\pgfqpoint{2.481509in}{1.212762in}}{\pgfqpoint{2.489323in}{1.204948in}}%
\pgfpathcurveto{\pgfqpoint{2.497137in}{1.197135in}}{\pgfqpoint{2.507736in}{1.192744in}}{\pgfqpoint{2.518786in}{1.192744in}}%
\pgfpathclose%
\pgfusepath{stroke,fill}%
\end{pgfscope}%
\begin{pgfscope}%
\pgfpathrectangle{\pgfqpoint{0.800000in}{0.528000in}}{\pgfqpoint{4.960000in}{3.696000in}}%
\pgfusepath{clip}%
\pgfsetbuttcap%
\pgfsetroundjoin%
\definecolor{currentfill}{rgb}{0.000000,0.000000,0.000000}%
\pgfsetfillcolor{currentfill}%
\pgfsetlinewidth{1.003750pt}%
\definecolor{currentstroke}{rgb}{0.000000,0.000000,0.000000}%
\pgfsetstrokecolor{currentstroke}%
\pgfsetdash{}{0pt}%
\pgfpathmoveto{\pgfqpoint{2.518786in}{1.235739in}}%
\pgfpathcurveto{\pgfqpoint{2.529836in}{1.235739in}}{\pgfqpoint{2.540435in}{1.240130in}}{\pgfqpoint{2.548249in}{1.247943in}}%
\pgfpathcurveto{\pgfqpoint{2.556062in}{1.255757in}}{\pgfqpoint{2.560452in}{1.266356in}}{\pgfqpoint{2.560452in}{1.277406in}}%
\pgfpathcurveto{\pgfqpoint{2.560452in}{1.288456in}}{\pgfqpoint{2.556062in}{1.299055in}}{\pgfqpoint{2.548249in}{1.306869in}}%
\pgfpathcurveto{\pgfqpoint{2.540435in}{1.314682in}}{\pgfqpoint{2.529836in}{1.319073in}}{\pgfqpoint{2.518786in}{1.319073in}}%
\pgfpathcurveto{\pgfqpoint{2.507736in}{1.319073in}}{\pgfqpoint{2.497137in}{1.314682in}}{\pgfqpoint{2.489323in}{1.306869in}}%
\pgfpathcurveto{\pgfqpoint{2.481509in}{1.299055in}}{\pgfqpoint{2.477119in}{1.288456in}}{\pgfqpoint{2.477119in}{1.277406in}}%
\pgfpathcurveto{\pgfqpoint{2.477119in}{1.266356in}}{\pgfqpoint{2.481509in}{1.255757in}}{\pgfqpoint{2.489323in}{1.247943in}}%
\pgfpathcurveto{\pgfqpoint{2.497137in}{1.240130in}}{\pgfqpoint{2.507736in}{1.235739in}}{\pgfqpoint{2.518786in}{1.235739in}}%
\pgfpathclose%
\pgfusepath{stroke,fill}%
\end{pgfscope}%
\begin{pgfscope}%
\pgfpathrectangle{\pgfqpoint{0.800000in}{0.528000in}}{\pgfqpoint{4.960000in}{3.696000in}}%
\pgfusepath{clip}%
\pgfsetbuttcap%
\pgfsetroundjoin%
\definecolor{currentfill}{rgb}{0.000000,0.000000,0.000000}%
\pgfsetfillcolor{currentfill}%
\pgfsetlinewidth{1.003750pt}%
\definecolor{currentstroke}{rgb}{0.000000,0.000000,0.000000}%
\pgfsetstrokecolor{currentstroke}%
\pgfsetdash{}{0pt}%
\pgfpathmoveto{\pgfqpoint{2.518786in}{1.063759in}}%
\pgfpathcurveto{\pgfqpoint{2.529836in}{1.063759in}}{\pgfqpoint{2.540435in}{1.068150in}}{\pgfqpoint{2.548249in}{1.075963in}}%
\pgfpathcurveto{\pgfqpoint{2.556062in}{1.083777in}}{\pgfqpoint{2.560452in}{1.094376in}}{\pgfqpoint{2.560452in}{1.105426in}}%
\pgfpathcurveto{\pgfqpoint{2.560452in}{1.116476in}}{\pgfqpoint{2.556062in}{1.127075in}}{\pgfqpoint{2.548249in}{1.134889in}}%
\pgfpathcurveto{\pgfqpoint{2.540435in}{1.142702in}}{\pgfqpoint{2.529836in}{1.147093in}}{\pgfqpoint{2.518786in}{1.147093in}}%
\pgfpathcurveto{\pgfqpoint{2.507736in}{1.147093in}}{\pgfqpoint{2.497137in}{1.142702in}}{\pgfqpoint{2.489323in}{1.134889in}}%
\pgfpathcurveto{\pgfqpoint{2.481509in}{1.127075in}}{\pgfqpoint{2.477119in}{1.116476in}}{\pgfqpoint{2.477119in}{1.105426in}}%
\pgfpathcurveto{\pgfqpoint{2.477119in}{1.094376in}}{\pgfqpoint{2.481509in}{1.083777in}}{\pgfqpoint{2.489323in}{1.075963in}}%
\pgfpathcurveto{\pgfqpoint{2.497137in}{1.068150in}}{\pgfqpoint{2.507736in}{1.063759in}}{\pgfqpoint{2.518786in}{1.063759in}}%
\pgfpathclose%
\pgfusepath{stroke,fill}%
\end{pgfscope}%
\begin{pgfscope}%
\pgfpathrectangle{\pgfqpoint{0.800000in}{0.528000in}}{\pgfqpoint{4.960000in}{3.696000in}}%
\pgfusepath{clip}%
\pgfsetbuttcap%
\pgfsetroundjoin%
\definecolor{currentfill}{rgb}{0.000000,0.000000,0.000000}%
\pgfsetfillcolor{currentfill}%
\pgfsetlinewidth{1.003750pt}%
\definecolor{currentstroke}{rgb}{0.000000,0.000000,0.000000}%
\pgfsetstrokecolor{currentstroke}%
\pgfsetdash{}{0pt}%
\pgfpathmoveto{\pgfqpoint{2.518786in}{1.042262in}}%
\pgfpathcurveto{\pgfqpoint{2.529836in}{1.042262in}}{\pgfqpoint{2.540435in}{1.046652in}}{\pgfqpoint{2.548249in}{1.054466in}}%
\pgfpathcurveto{\pgfqpoint{2.556062in}{1.062279in}}{\pgfqpoint{2.560452in}{1.072878in}}{\pgfqpoint{2.560452in}{1.083928in}}%
\pgfpathcurveto{\pgfqpoint{2.560452in}{1.094979in}}{\pgfqpoint{2.556062in}{1.105578in}}{\pgfqpoint{2.548249in}{1.113391in}}%
\pgfpathcurveto{\pgfqpoint{2.540435in}{1.121205in}}{\pgfqpoint{2.529836in}{1.125595in}}{\pgfqpoint{2.518786in}{1.125595in}}%
\pgfpathcurveto{\pgfqpoint{2.507736in}{1.125595in}}{\pgfqpoint{2.497137in}{1.121205in}}{\pgfqpoint{2.489323in}{1.113391in}}%
\pgfpathcurveto{\pgfqpoint{2.481509in}{1.105578in}}{\pgfqpoint{2.477119in}{1.094979in}}{\pgfqpoint{2.477119in}{1.083928in}}%
\pgfpathcurveto{\pgfqpoint{2.477119in}{1.072878in}}{\pgfqpoint{2.481509in}{1.062279in}}{\pgfqpoint{2.489323in}{1.054466in}}%
\pgfpathcurveto{\pgfqpoint{2.497137in}{1.046652in}}{\pgfqpoint{2.507736in}{1.042262in}}{\pgfqpoint{2.518786in}{1.042262in}}%
\pgfpathclose%
\pgfusepath{stroke,fill}%
\end{pgfscope}%
\begin{pgfscope}%
\pgfpathrectangle{\pgfqpoint{0.800000in}{0.528000in}}{\pgfqpoint{4.960000in}{3.696000in}}%
\pgfusepath{clip}%
\pgfsetbuttcap%
\pgfsetroundjoin%
\definecolor{currentfill}{rgb}{0.000000,0.000000,0.000000}%
\pgfsetfillcolor{currentfill}%
\pgfsetlinewidth{1.003750pt}%
\definecolor{currentstroke}{rgb}{0.000000,0.000000,0.000000}%
\pgfsetstrokecolor{currentstroke}%
\pgfsetdash{}{0pt}%
\pgfpathmoveto{\pgfqpoint{2.518786in}{1.106754in}}%
\pgfpathcurveto{\pgfqpoint{2.529836in}{1.106754in}}{\pgfqpoint{2.540435in}{1.111145in}}{\pgfqpoint{2.548249in}{1.118958in}}%
\pgfpathcurveto{\pgfqpoint{2.556062in}{1.126772in}}{\pgfqpoint{2.560452in}{1.137371in}}{\pgfqpoint{2.560452in}{1.148421in}}%
\pgfpathcurveto{\pgfqpoint{2.560452in}{1.159471in}}{\pgfqpoint{2.556062in}{1.170070in}}{\pgfqpoint{2.548249in}{1.177884in}}%
\pgfpathcurveto{\pgfqpoint{2.540435in}{1.185697in}}{\pgfqpoint{2.529836in}{1.190088in}}{\pgfqpoint{2.518786in}{1.190088in}}%
\pgfpathcurveto{\pgfqpoint{2.507736in}{1.190088in}}{\pgfqpoint{2.497137in}{1.185697in}}{\pgfqpoint{2.489323in}{1.177884in}}%
\pgfpathcurveto{\pgfqpoint{2.481509in}{1.170070in}}{\pgfqpoint{2.477119in}{1.159471in}}{\pgfqpoint{2.477119in}{1.148421in}}%
\pgfpathcurveto{\pgfqpoint{2.477119in}{1.137371in}}{\pgfqpoint{2.481509in}{1.126772in}}{\pgfqpoint{2.489323in}{1.118958in}}%
\pgfpathcurveto{\pgfqpoint{2.497137in}{1.111145in}}{\pgfqpoint{2.507736in}{1.106754in}}{\pgfqpoint{2.518786in}{1.106754in}}%
\pgfpathclose%
\pgfusepath{stroke,fill}%
\end{pgfscope}%
\begin{pgfscope}%
\pgfpathrectangle{\pgfqpoint{0.800000in}{0.528000in}}{\pgfqpoint{4.960000in}{3.696000in}}%
\pgfusepath{clip}%
\pgfsetbuttcap%
\pgfsetroundjoin%
\definecolor{currentfill}{rgb}{0.000000,0.000000,0.000000}%
\pgfsetfillcolor{currentfill}%
\pgfsetlinewidth{1.003750pt}%
\definecolor{currentstroke}{rgb}{0.000000,0.000000,0.000000}%
\pgfsetstrokecolor{currentstroke}%
\pgfsetdash{}{0pt}%
\pgfpathmoveto{\pgfqpoint{2.518786in}{1.149749in}}%
\pgfpathcurveto{\pgfqpoint{2.529836in}{1.149749in}}{\pgfqpoint{2.540435in}{1.154140in}}{\pgfqpoint{2.548249in}{1.161953in}}%
\pgfpathcurveto{\pgfqpoint{2.556062in}{1.169767in}}{\pgfqpoint{2.560452in}{1.180366in}}{\pgfqpoint{2.560452in}{1.191416in}}%
\pgfpathcurveto{\pgfqpoint{2.560452in}{1.202466in}}{\pgfqpoint{2.556062in}{1.213065in}}{\pgfqpoint{2.548249in}{1.220879in}}%
\pgfpathcurveto{\pgfqpoint{2.540435in}{1.228692in}}{\pgfqpoint{2.529836in}{1.233083in}}{\pgfqpoint{2.518786in}{1.233083in}}%
\pgfpathcurveto{\pgfqpoint{2.507736in}{1.233083in}}{\pgfqpoint{2.497137in}{1.228692in}}{\pgfqpoint{2.489323in}{1.220879in}}%
\pgfpathcurveto{\pgfqpoint{2.481509in}{1.213065in}}{\pgfqpoint{2.477119in}{1.202466in}}{\pgfqpoint{2.477119in}{1.191416in}}%
\pgfpathcurveto{\pgfqpoint{2.477119in}{1.180366in}}{\pgfqpoint{2.481509in}{1.169767in}}{\pgfqpoint{2.489323in}{1.161953in}}%
\pgfpathcurveto{\pgfqpoint{2.497137in}{1.154140in}}{\pgfqpoint{2.507736in}{1.149749in}}{\pgfqpoint{2.518786in}{1.149749in}}%
\pgfpathclose%
\pgfusepath{stroke,fill}%
\end{pgfscope}%
\begin{pgfscope}%
\pgfpathrectangle{\pgfqpoint{0.800000in}{0.528000in}}{\pgfqpoint{4.960000in}{3.696000in}}%
\pgfusepath{clip}%
\pgfsetbuttcap%
\pgfsetroundjoin%
\definecolor{currentfill}{rgb}{0.000000,0.000000,0.000000}%
\pgfsetfillcolor{currentfill}%
\pgfsetlinewidth{1.003750pt}%
\definecolor{currentstroke}{rgb}{0.000000,0.000000,0.000000}%
\pgfsetstrokecolor{currentstroke}%
\pgfsetdash{}{0pt}%
\pgfpathmoveto{\pgfqpoint{2.518786in}{1.214242in}}%
\pgfpathcurveto{\pgfqpoint{2.529836in}{1.214242in}}{\pgfqpoint{2.540435in}{1.218632in}}{\pgfqpoint{2.548249in}{1.226446in}}%
\pgfpathcurveto{\pgfqpoint{2.556062in}{1.234259in}}{\pgfqpoint{2.560452in}{1.244858in}}{\pgfqpoint{2.560452in}{1.255908in}}%
\pgfpathcurveto{\pgfqpoint{2.560452in}{1.266959in}}{\pgfqpoint{2.556062in}{1.277558in}}{\pgfqpoint{2.548249in}{1.285371in}}%
\pgfpathcurveto{\pgfqpoint{2.540435in}{1.293185in}}{\pgfqpoint{2.529836in}{1.297575in}}{\pgfqpoint{2.518786in}{1.297575in}}%
\pgfpathcurveto{\pgfqpoint{2.507736in}{1.297575in}}{\pgfqpoint{2.497137in}{1.293185in}}{\pgfqpoint{2.489323in}{1.285371in}}%
\pgfpathcurveto{\pgfqpoint{2.481509in}{1.277558in}}{\pgfqpoint{2.477119in}{1.266959in}}{\pgfqpoint{2.477119in}{1.255908in}}%
\pgfpathcurveto{\pgfqpoint{2.477119in}{1.244858in}}{\pgfqpoint{2.481509in}{1.234259in}}{\pgfqpoint{2.489323in}{1.226446in}}%
\pgfpathcurveto{\pgfqpoint{2.497137in}{1.218632in}}{\pgfqpoint{2.507736in}{1.214242in}}{\pgfqpoint{2.518786in}{1.214242in}}%
\pgfpathclose%
\pgfusepath{stroke,fill}%
\end{pgfscope}%
\begin{pgfscope}%
\pgfpathrectangle{\pgfqpoint{0.800000in}{0.528000in}}{\pgfqpoint{4.960000in}{3.696000in}}%
\pgfusepath{clip}%
\pgfsetbuttcap%
\pgfsetroundjoin%
\definecolor{currentfill}{rgb}{0.000000,0.000000,0.000000}%
\pgfsetfillcolor{currentfill}%
\pgfsetlinewidth{1.003750pt}%
\definecolor{currentstroke}{rgb}{0.000000,0.000000,0.000000}%
\pgfsetstrokecolor{currentstroke}%
\pgfsetdash{}{0pt}%
\pgfpathmoveto{\pgfqpoint{2.518786in}{1.192744in}}%
\pgfpathcurveto{\pgfqpoint{2.529836in}{1.192744in}}{\pgfqpoint{2.540435in}{1.197135in}}{\pgfqpoint{2.548249in}{1.204948in}}%
\pgfpathcurveto{\pgfqpoint{2.556062in}{1.212762in}}{\pgfqpoint{2.560452in}{1.223361in}}{\pgfqpoint{2.560452in}{1.234411in}}%
\pgfpathcurveto{\pgfqpoint{2.560452in}{1.245461in}}{\pgfqpoint{2.556062in}{1.256060in}}{\pgfqpoint{2.548249in}{1.263874in}}%
\pgfpathcurveto{\pgfqpoint{2.540435in}{1.271687in}}{\pgfqpoint{2.529836in}{1.276078in}}{\pgfqpoint{2.518786in}{1.276078in}}%
\pgfpathcurveto{\pgfqpoint{2.507736in}{1.276078in}}{\pgfqpoint{2.497137in}{1.271687in}}{\pgfqpoint{2.489323in}{1.263874in}}%
\pgfpathcurveto{\pgfqpoint{2.481509in}{1.256060in}}{\pgfqpoint{2.477119in}{1.245461in}}{\pgfqpoint{2.477119in}{1.234411in}}%
\pgfpathcurveto{\pgfqpoint{2.477119in}{1.223361in}}{\pgfqpoint{2.481509in}{1.212762in}}{\pgfqpoint{2.489323in}{1.204948in}}%
\pgfpathcurveto{\pgfqpoint{2.497137in}{1.197135in}}{\pgfqpoint{2.507736in}{1.192744in}}{\pgfqpoint{2.518786in}{1.192744in}}%
\pgfpathclose%
\pgfusepath{stroke,fill}%
\end{pgfscope}%
\begin{pgfscope}%
\pgfpathrectangle{\pgfqpoint{0.800000in}{0.528000in}}{\pgfqpoint{4.960000in}{3.696000in}}%
\pgfusepath{clip}%
\pgfsetbuttcap%
\pgfsetroundjoin%
\definecolor{currentfill}{rgb}{0.000000,0.000000,0.000000}%
\pgfsetfillcolor{currentfill}%
\pgfsetlinewidth{1.003750pt}%
\definecolor{currentstroke}{rgb}{0.000000,0.000000,0.000000}%
\pgfsetstrokecolor{currentstroke}%
\pgfsetdash{}{0pt}%
\pgfpathmoveto{\pgfqpoint{2.518786in}{1.149749in}}%
\pgfpathcurveto{\pgfqpoint{2.529836in}{1.149749in}}{\pgfqpoint{2.540435in}{1.154140in}}{\pgfqpoint{2.548249in}{1.161953in}}%
\pgfpathcurveto{\pgfqpoint{2.556062in}{1.169767in}}{\pgfqpoint{2.560452in}{1.180366in}}{\pgfqpoint{2.560452in}{1.191416in}}%
\pgfpathcurveto{\pgfqpoint{2.560452in}{1.202466in}}{\pgfqpoint{2.556062in}{1.213065in}}{\pgfqpoint{2.548249in}{1.220879in}}%
\pgfpathcurveto{\pgfqpoint{2.540435in}{1.228692in}}{\pgfqpoint{2.529836in}{1.233083in}}{\pgfqpoint{2.518786in}{1.233083in}}%
\pgfpathcurveto{\pgfqpoint{2.507736in}{1.233083in}}{\pgfqpoint{2.497137in}{1.228692in}}{\pgfqpoint{2.489323in}{1.220879in}}%
\pgfpathcurveto{\pgfqpoint{2.481509in}{1.213065in}}{\pgfqpoint{2.477119in}{1.202466in}}{\pgfqpoint{2.477119in}{1.191416in}}%
\pgfpathcurveto{\pgfqpoint{2.477119in}{1.180366in}}{\pgfqpoint{2.481509in}{1.169767in}}{\pgfqpoint{2.489323in}{1.161953in}}%
\pgfpathcurveto{\pgfqpoint{2.497137in}{1.154140in}}{\pgfqpoint{2.507736in}{1.149749in}}{\pgfqpoint{2.518786in}{1.149749in}}%
\pgfpathclose%
\pgfusepath{stroke,fill}%
\end{pgfscope}%
\begin{pgfscope}%
\pgfpathrectangle{\pgfqpoint{0.800000in}{0.528000in}}{\pgfqpoint{4.960000in}{3.696000in}}%
\pgfusepath{clip}%
\pgfsetbuttcap%
\pgfsetroundjoin%
\definecolor{currentfill}{rgb}{0.000000,0.000000,0.000000}%
\pgfsetfillcolor{currentfill}%
\pgfsetlinewidth{1.003750pt}%
\definecolor{currentstroke}{rgb}{0.000000,0.000000,0.000000}%
\pgfsetstrokecolor{currentstroke}%
\pgfsetdash{}{0pt}%
\pgfpathmoveto{\pgfqpoint{2.518786in}{1.128252in}}%
\pgfpathcurveto{\pgfqpoint{2.529836in}{1.128252in}}{\pgfqpoint{2.540435in}{1.132642in}}{\pgfqpoint{2.548249in}{1.140456in}}%
\pgfpathcurveto{\pgfqpoint{2.556062in}{1.148269in}}{\pgfqpoint{2.560452in}{1.158868in}}{\pgfqpoint{2.560452in}{1.169918in}}%
\pgfpathcurveto{\pgfqpoint{2.560452in}{1.180969in}}{\pgfqpoint{2.556062in}{1.191568in}}{\pgfqpoint{2.548249in}{1.199381in}}%
\pgfpathcurveto{\pgfqpoint{2.540435in}{1.207195in}}{\pgfqpoint{2.529836in}{1.211585in}}{\pgfqpoint{2.518786in}{1.211585in}}%
\pgfpathcurveto{\pgfqpoint{2.507736in}{1.211585in}}{\pgfqpoint{2.497137in}{1.207195in}}{\pgfqpoint{2.489323in}{1.199381in}}%
\pgfpathcurveto{\pgfqpoint{2.481509in}{1.191568in}}{\pgfqpoint{2.477119in}{1.180969in}}{\pgfqpoint{2.477119in}{1.169918in}}%
\pgfpathcurveto{\pgfqpoint{2.477119in}{1.158868in}}{\pgfqpoint{2.481509in}{1.148269in}}{\pgfqpoint{2.489323in}{1.140456in}}%
\pgfpathcurveto{\pgfqpoint{2.497137in}{1.132642in}}{\pgfqpoint{2.507736in}{1.128252in}}{\pgfqpoint{2.518786in}{1.128252in}}%
\pgfpathclose%
\pgfusepath{stroke,fill}%
\end{pgfscope}%
\begin{pgfscope}%
\pgfpathrectangle{\pgfqpoint{0.800000in}{0.528000in}}{\pgfqpoint{4.960000in}{3.696000in}}%
\pgfusepath{clip}%
\pgfsetbuttcap%
\pgfsetroundjoin%
\definecolor{currentfill}{rgb}{0.000000,0.000000,0.000000}%
\pgfsetfillcolor{currentfill}%
\pgfsetlinewidth{1.003750pt}%
\definecolor{currentstroke}{rgb}{0.000000,0.000000,0.000000}%
\pgfsetstrokecolor{currentstroke}%
\pgfsetdash{}{0pt}%
\pgfpathmoveto{\pgfqpoint{2.518786in}{1.106754in}}%
\pgfpathcurveto{\pgfqpoint{2.529836in}{1.106754in}}{\pgfqpoint{2.540435in}{1.111145in}}{\pgfqpoint{2.548249in}{1.118958in}}%
\pgfpathcurveto{\pgfqpoint{2.556062in}{1.126772in}}{\pgfqpoint{2.560452in}{1.137371in}}{\pgfqpoint{2.560452in}{1.148421in}}%
\pgfpathcurveto{\pgfqpoint{2.560452in}{1.159471in}}{\pgfqpoint{2.556062in}{1.170070in}}{\pgfqpoint{2.548249in}{1.177884in}}%
\pgfpathcurveto{\pgfqpoint{2.540435in}{1.185697in}}{\pgfqpoint{2.529836in}{1.190088in}}{\pgfqpoint{2.518786in}{1.190088in}}%
\pgfpathcurveto{\pgfqpoint{2.507736in}{1.190088in}}{\pgfqpoint{2.497137in}{1.185697in}}{\pgfqpoint{2.489323in}{1.177884in}}%
\pgfpathcurveto{\pgfqpoint{2.481509in}{1.170070in}}{\pgfqpoint{2.477119in}{1.159471in}}{\pgfqpoint{2.477119in}{1.148421in}}%
\pgfpathcurveto{\pgfqpoint{2.477119in}{1.137371in}}{\pgfqpoint{2.481509in}{1.126772in}}{\pgfqpoint{2.489323in}{1.118958in}}%
\pgfpathcurveto{\pgfqpoint{2.497137in}{1.111145in}}{\pgfqpoint{2.507736in}{1.106754in}}{\pgfqpoint{2.518786in}{1.106754in}}%
\pgfpathclose%
\pgfusepath{stroke,fill}%
\end{pgfscope}%
\begin{pgfscope}%
\pgfpathrectangle{\pgfqpoint{0.800000in}{0.528000in}}{\pgfqpoint{4.960000in}{3.696000in}}%
\pgfusepath{clip}%
\pgfsetbuttcap%
\pgfsetroundjoin%
\definecolor{currentfill}{rgb}{0.000000,0.000000,0.000000}%
\pgfsetfillcolor{currentfill}%
\pgfsetlinewidth{1.003750pt}%
\definecolor{currentstroke}{rgb}{0.000000,0.000000,0.000000}%
\pgfsetstrokecolor{currentstroke}%
\pgfsetdash{}{0pt}%
\pgfpathmoveto{\pgfqpoint{2.518786in}{1.128252in}}%
\pgfpathcurveto{\pgfqpoint{2.529836in}{1.128252in}}{\pgfqpoint{2.540435in}{1.132642in}}{\pgfqpoint{2.548249in}{1.140456in}}%
\pgfpathcurveto{\pgfqpoint{2.556062in}{1.148269in}}{\pgfqpoint{2.560452in}{1.158868in}}{\pgfqpoint{2.560452in}{1.169918in}}%
\pgfpathcurveto{\pgfqpoint{2.560452in}{1.180969in}}{\pgfqpoint{2.556062in}{1.191568in}}{\pgfqpoint{2.548249in}{1.199381in}}%
\pgfpathcurveto{\pgfqpoint{2.540435in}{1.207195in}}{\pgfqpoint{2.529836in}{1.211585in}}{\pgfqpoint{2.518786in}{1.211585in}}%
\pgfpathcurveto{\pgfqpoint{2.507736in}{1.211585in}}{\pgfqpoint{2.497137in}{1.207195in}}{\pgfqpoint{2.489323in}{1.199381in}}%
\pgfpathcurveto{\pgfqpoint{2.481509in}{1.191568in}}{\pgfqpoint{2.477119in}{1.180969in}}{\pgfqpoint{2.477119in}{1.169918in}}%
\pgfpathcurveto{\pgfqpoint{2.477119in}{1.158868in}}{\pgfqpoint{2.481509in}{1.148269in}}{\pgfqpoint{2.489323in}{1.140456in}}%
\pgfpathcurveto{\pgfqpoint{2.497137in}{1.132642in}}{\pgfqpoint{2.507736in}{1.128252in}}{\pgfqpoint{2.518786in}{1.128252in}}%
\pgfpathclose%
\pgfusepath{stroke,fill}%
\end{pgfscope}%
\begin{pgfscope}%
\pgfpathrectangle{\pgfqpoint{0.800000in}{0.528000in}}{\pgfqpoint{4.960000in}{3.696000in}}%
\pgfusepath{clip}%
\pgfsetbuttcap%
\pgfsetroundjoin%
\definecolor{currentfill}{rgb}{0.000000,0.000000,0.000000}%
\pgfsetfillcolor{currentfill}%
\pgfsetlinewidth{1.003750pt}%
\definecolor{currentstroke}{rgb}{0.000000,0.000000,0.000000}%
\pgfsetstrokecolor{currentstroke}%
\pgfsetdash{}{0pt}%
\pgfpathmoveto{\pgfqpoint{2.518786in}{1.128252in}}%
\pgfpathcurveto{\pgfqpoint{2.529836in}{1.128252in}}{\pgfqpoint{2.540435in}{1.132642in}}{\pgfqpoint{2.548249in}{1.140456in}}%
\pgfpathcurveto{\pgfqpoint{2.556062in}{1.148269in}}{\pgfqpoint{2.560452in}{1.158868in}}{\pgfqpoint{2.560452in}{1.169918in}}%
\pgfpathcurveto{\pgfqpoint{2.560452in}{1.180969in}}{\pgfqpoint{2.556062in}{1.191568in}}{\pgfqpoint{2.548249in}{1.199381in}}%
\pgfpathcurveto{\pgfqpoint{2.540435in}{1.207195in}}{\pgfqpoint{2.529836in}{1.211585in}}{\pgfqpoint{2.518786in}{1.211585in}}%
\pgfpathcurveto{\pgfqpoint{2.507736in}{1.211585in}}{\pgfqpoint{2.497137in}{1.207195in}}{\pgfqpoint{2.489323in}{1.199381in}}%
\pgfpathcurveto{\pgfqpoint{2.481509in}{1.191568in}}{\pgfqpoint{2.477119in}{1.180969in}}{\pgfqpoint{2.477119in}{1.169918in}}%
\pgfpathcurveto{\pgfqpoint{2.477119in}{1.158868in}}{\pgfqpoint{2.481509in}{1.148269in}}{\pgfqpoint{2.489323in}{1.140456in}}%
\pgfpathcurveto{\pgfqpoint{2.497137in}{1.132642in}}{\pgfqpoint{2.507736in}{1.128252in}}{\pgfqpoint{2.518786in}{1.128252in}}%
\pgfpathclose%
\pgfusepath{stroke,fill}%
\end{pgfscope}%
\begin{pgfscope}%
\pgfpathrectangle{\pgfqpoint{0.800000in}{0.528000in}}{\pgfqpoint{4.960000in}{3.696000in}}%
\pgfusepath{clip}%
\pgfsetbuttcap%
\pgfsetroundjoin%
\definecolor{currentfill}{rgb}{0.000000,0.000000,0.000000}%
\pgfsetfillcolor{currentfill}%
\pgfsetlinewidth{1.003750pt}%
\definecolor{currentstroke}{rgb}{0.000000,0.000000,0.000000}%
\pgfsetstrokecolor{currentstroke}%
\pgfsetdash{}{0pt}%
\pgfpathmoveto{\pgfqpoint{2.518786in}{1.085257in}}%
\pgfpathcurveto{\pgfqpoint{2.529836in}{1.085257in}}{\pgfqpoint{2.540435in}{1.089647in}}{\pgfqpoint{2.548249in}{1.097461in}}%
\pgfpathcurveto{\pgfqpoint{2.556062in}{1.105274in}}{\pgfqpoint{2.560452in}{1.115873in}}{\pgfqpoint{2.560452in}{1.126923in}}%
\pgfpathcurveto{\pgfqpoint{2.560452in}{1.137974in}}{\pgfqpoint{2.556062in}{1.148573in}}{\pgfqpoint{2.548249in}{1.156386in}}%
\pgfpathcurveto{\pgfqpoint{2.540435in}{1.164200in}}{\pgfqpoint{2.529836in}{1.168590in}}{\pgfqpoint{2.518786in}{1.168590in}}%
\pgfpathcurveto{\pgfqpoint{2.507736in}{1.168590in}}{\pgfqpoint{2.497137in}{1.164200in}}{\pgfqpoint{2.489323in}{1.156386in}}%
\pgfpathcurveto{\pgfqpoint{2.481509in}{1.148573in}}{\pgfqpoint{2.477119in}{1.137974in}}{\pgfqpoint{2.477119in}{1.126923in}}%
\pgfpathcurveto{\pgfqpoint{2.477119in}{1.115873in}}{\pgfqpoint{2.481509in}{1.105274in}}{\pgfqpoint{2.489323in}{1.097461in}}%
\pgfpathcurveto{\pgfqpoint{2.497137in}{1.089647in}}{\pgfqpoint{2.507736in}{1.085257in}}{\pgfqpoint{2.518786in}{1.085257in}}%
\pgfpathclose%
\pgfusepath{stroke,fill}%
\end{pgfscope}%
\begin{pgfscope}%
\pgfpathrectangle{\pgfqpoint{0.800000in}{0.528000in}}{\pgfqpoint{4.960000in}{3.696000in}}%
\pgfusepath{clip}%
\pgfsetbuttcap%
\pgfsetroundjoin%
\definecolor{currentfill}{rgb}{0.000000,0.000000,0.000000}%
\pgfsetfillcolor{currentfill}%
\pgfsetlinewidth{1.003750pt}%
\definecolor{currentstroke}{rgb}{0.000000,0.000000,0.000000}%
\pgfsetstrokecolor{currentstroke}%
\pgfsetdash{}{0pt}%
\pgfpathmoveto{\pgfqpoint{2.518786in}{1.085257in}}%
\pgfpathcurveto{\pgfqpoint{2.529836in}{1.085257in}}{\pgfqpoint{2.540435in}{1.089647in}}{\pgfqpoint{2.548249in}{1.097461in}}%
\pgfpathcurveto{\pgfqpoint{2.556062in}{1.105274in}}{\pgfqpoint{2.560452in}{1.115873in}}{\pgfqpoint{2.560452in}{1.126923in}}%
\pgfpathcurveto{\pgfqpoint{2.560452in}{1.137974in}}{\pgfqpoint{2.556062in}{1.148573in}}{\pgfqpoint{2.548249in}{1.156386in}}%
\pgfpathcurveto{\pgfqpoint{2.540435in}{1.164200in}}{\pgfqpoint{2.529836in}{1.168590in}}{\pgfqpoint{2.518786in}{1.168590in}}%
\pgfpathcurveto{\pgfqpoint{2.507736in}{1.168590in}}{\pgfqpoint{2.497137in}{1.164200in}}{\pgfqpoint{2.489323in}{1.156386in}}%
\pgfpathcurveto{\pgfqpoint{2.481509in}{1.148573in}}{\pgfqpoint{2.477119in}{1.137974in}}{\pgfqpoint{2.477119in}{1.126923in}}%
\pgfpathcurveto{\pgfqpoint{2.477119in}{1.115873in}}{\pgfqpoint{2.481509in}{1.105274in}}{\pgfqpoint{2.489323in}{1.097461in}}%
\pgfpathcurveto{\pgfqpoint{2.497137in}{1.089647in}}{\pgfqpoint{2.507736in}{1.085257in}}{\pgfqpoint{2.518786in}{1.085257in}}%
\pgfpathclose%
\pgfusepath{stroke,fill}%
\end{pgfscope}%
\begin{pgfscope}%
\pgfpathrectangle{\pgfqpoint{0.800000in}{0.528000in}}{\pgfqpoint{4.960000in}{3.696000in}}%
\pgfusepath{clip}%
\pgfsetbuttcap%
\pgfsetroundjoin%
\definecolor{currentfill}{rgb}{0.000000,0.000000,0.000000}%
\pgfsetfillcolor{currentfill}%
\pgfsetlinewidth{1.003750pt}%
\definecolor{currentstroke}{rgb}{0.000000,0.000000,0.000000}%
\pgfsetstrokecolor{currentstroke}%
\pgfsetdash{}{0pt}%
\pgfpathmoveto{\pgfqpoint{2.518786in}{1.149749in}}%
\pgfpathcurveto{\pgfqpoint{2.529836in}{1.149749in}}{\pgfqpoint{2.540435in}{1.154140in}}{\pgfqpoint{2.548249in}{1.161953in}}%
\pgfpathcurveto{\pgfqpoint{2.556062in}{1.169767in}}{\pgfqpoint{2.560452in}{1.180366in}}{\pgfqpoint{2.560452in}{1.191416in}}%
\pgfpathcurveto{\pgfqpoint{2.560452in}{1.202466in}}{\pgfqpoint{2.556062in}{1.213065in}}{\pgfqpoint{2.548249in}{1.220879in}}%
\pgfpathcurveto{\pgfqpoint{2.540435in}{1.228692in}}{\pgfqpoint{2.529836in}{1.233083in}}{\pgfqpoint{2.518786in}{1.233083in}}%
\pgfpathcurveto{\pgfqpoint{2.507736in}{1.233083in}}{\pgfqpoint{2.497137in}{1.228692in}}{\pgfqpoint{2.489323in}{1.220879in}}%
\pgfpathcurveto{\pgfqpoint{2.481509in}{1.213065in}}{\pgfqpoint{2.477119in}{1.202466in}}{\pgfqpoint{2.477119in}{1.191416in}}%
\pgfpathcurveto{\pgfqpoint{2.477119in}{1.180366in}}{\pgfqpoint{2.481509in}{1.169767in}}{\pgfqpoint{2.489323in}{1.161953in}}%
\pgfpathcurveto{\pgfqpoint{2.497137in}{1.154140in}}{\pgfqpoint{2.507736in}{1.149749in}}{\pgfqpoint{2.518786in}{1.149749in}}%
\pgfpathclose%
\pgfusepath{stroke,fill}%
\end{pgfscope}%
\begin{pgfscope}%
\pgfpathrectangle{\pgfqpoint{0.800000in}{0.528000in}}{\pgfqpoint{4.960000in}{3.696000in}}%
\pgfusepath{clip}%
\pgfsetbuttcap%
\pgfsetroundjoin%
\definecolor{currentfill}{rgb}{0.000000,0.000000,0.000000}%
\pgfsetfillcolor{currentfill}%
\pgfsetlinewidth{1.003750pt}%
\definecolor{currentstroke}{rgb}{0.000000,0.000000,0.000000}%
\pgfsetstrokecolor{currentstroke}%
\pgfsetdash{}{0pt}%
\pgfpathmoveto{\pgfqpoint{2.518786in}{1.106754in}}%
\pgfpathcurveto{\pgfqpoint{2.529836in}{1.106754in}}{\pgfqpoint{2.540435in}{1.111145in}}{\pgfqpoint{2.548249in}{1.118958in}}%
\pgfpathcurveto{\pgfqpoint{2.556062in}{1.126772in}}{\pgfqpoint{2.560452in}{1.137371in}}{\pgfqpoint{2.560452in}{1.148421in}}%
\pgfpathcurveto{\pgfqpoint{2.560452in}{1.159471in}}{\pgfqpoint{2.556062in}{1.170070in}}{\pgfqpoint{2.548249in}{1.177884in}}%
\pgfpathcurveto{\pgfqpoint{2.540435in}{1.185697in}}{\pgfqpoint{2.529836in}{1.190088in}}{\pgfqpoint{2.518786in}{1.190088in}}%
\pgfpathcurveto{\pgfqpoint{2.507736in}{1.190088in}}{\pgfqpoint{2.497137in}{1.185697in}}{\pgfqpoint{2.489323in}{1.177884in}}%
\pgfpathcurveto{\pgfqpoint{2.481509in}{1.170070in}}{\pgfqpoint{2.477119in}{1.159471in}}{\pgfqpoint{2.477119in}{1.148421in}}%
\pgfpathcurveto{\pgfqpoint{2.477119in}{1.137371in}}{\pgfqpoint{2.481509in}{1.126772in}}{\pgfqpoint{2.489323in}{1.118958in}}%
\pgfpathcurveto{\pgfqpoint{2.497137in}{1.111145in}}{\pgfqpoint{2.507736in}{1.106754in}}{\pgfqpoint{2.518786in}{1.106754in}}%
\pgfpathclose%
\pgfusepath{stroke,fill}%
\end{pgfscope}%
\begin{pgfscope}%
\pgfpathrectangle{\pgfqpoint{0.800000in}{0.528000in}}{\pgfqpoint{4.960000in}{3.696000in}}%
\pgfusepath{clip}%
\pgfsetbuttcap%
\pgfsetroundjoin%
\definecolor{currentfill}{rgb}{0.000000,0.000000,0.000000}%
\pgfsetfillcolor{currentfill}%
\pgfsetlinewidth{1.003750pt}%
\definecolor{currentstroke}{rgb}{0.000000,0.000000,0.000000}%
\pgfsetstrokecolor{currentstroke}%
\pgfsetdash{}{0pt}%
\pgfpathmoveto{\pgfqpoint{2.518786in}{1.106754in}}%
\pgfpathcurveto{\pgfqpoint{2.529836in}{1.106754in}}{\pgfqpoint{2.540435in}{1.111145in}}{\pgfqpoint{2.548249in}{1.118958in}}%
\pgfpathcurveto{\pgfqpoint{2.556062in}{1.126772in}}{\pgfqpoint{2.560452in}{1.137371in}}{\pgfqpoint{2.560452in}{1.148421in}}%
\pgfpathcurveto{\pgfqpoint{2.560452in}{1.159471in}}{\pgfqpoint{2.556062in}{1.170070in}}{\pgfqpoint{2.548249in}{1.177884in}}%
\pgfpathcurveto{\pgfqpoint{2.540435in}{1.185697in}}{\pgfqpoint{2.529836in}{1.190088in}}{\pgfqpoint{2.518786in}{1.190088in}}%
\pgfpathcurveto{\pgfqpoint{2.507736in}{1.190088in}}{\pgfqpoint{2.497137in}{1.185697in}}{\pgfqpoint{2.489323in}{1.177884in}}%
\pgfpathcurveto{\pgfqpoint{2.481509in}{1.170070in}}{\pgfqpoint{2.477119in}{1.159471in}}{\pgfqpoint{2.477119in}{1.148421in}}%
\pgfpathcurveto{\pgfqpoint{2.477119in}{1.137371in}}{\pgfqpoint{2.481509in}{1.126772in}}{\pgfqpoint{2.489323in}{1.118958in}}%
\pgfpathcurveto{\pgfqpoint{2.497137in}{1.111145in}}{\pgfqpoint{2.507736in}{1.106754in}}{\pgfqpoint{2.518786in}{1.106754in}}%
\pgfpathclose%
\pgfusepath{stroke,fill}%
\end{pgfscope}%
\begin{pgfscope}%
\pgfpathrectangle{\pgfqpoint{0.800000in}{0.528000in}}{\pgfqpoint{4.960000in}{3.696000in}}%
\pgfusepath{clip}%
\pgfsetbuttcap%
\pgfsetroundjoin%
\definecolor{currentfill}{rgb}{0.000000,0.000000,0.000000}%
\pgfsetfillcolor{currentfill}%
\pgfsetlinewidth{1.003750pt}%
\definecolor{currentstroke}{rgb}{0.000000,0.000000,0.000000}%
\pgfsetstrokecolor{currentstroke}%
\pgfsetdash{}{0pt}%
\pgfpathmoveto{\pgfqpoint{2.518786in}{1.171247in}}%
\pgfpathcurveto{\pgfqpoint{2.529836in}{1.171247in}}{\pgfqpoint{2.540435in}{1.175637in}}{\pgfqpoint{2.548249in}{1.183451in}}%
\pgfpathcurveto{\pgfqpoint{2.556062in}{1.191264in}}{\pgfqpoint{2.560452in}{1.201863in}}{\pgfqpoint{2.560452in}{1.212913in}}%
\pgfpathcurveto{\pgfqpoint{2.560452in}{1.223964in}}{\pgfqpoint{2.556062in}{1.234563in}}{\pgfqpoint{2.548249in}{1.242376in}}%
\pgfpathcurveto{\pgfqpoint{2.540435in}{1.250190in}}{\pgfqpoint{2.529836in}{1.254580in}}{\pgfqpoint{2.518786in}{1.254580in}}%
\pgfpathcurveto{\pgfqpoint{2.507736in}{1.254580in}}{\pgfqpoint{2.497137in}{1.250190in}}{\pgfqpoint{2.489323in}{1.242376in}}%
\pgfpathcurveto{\pgfqpoint{2.481509in}{1.234563in}}{\pgfqpoint{2.477119in}{1.223964in}}{\pgfqpoint{2.477119in}{1.212913in}}%
\pgfpathcurveto{\pgfqpoint{2.477119in}{1.201863in}}{\pgfqpoint{2.481509in}{1.191264in}}{\pgfqpoint{2.489323in}{1.183451in}}%
\pgfpathcurveto{\pgfqpoint{2.497137in}{1.175637in}}{\pgfqpoint{2.507736in}{1.171247in}}{\pgfqpoint{2.518786in}{1.171247in}}%
\pgfpathclose%
\pgfusepath{stroke,fill}%
\end{pgfscope}%
\begin{pgfscope}%
\pgfpathrectangle{\pgfqpoint{0.800000in}{0.528000in}}{\pgfqpoint{4.960000in}{3.696000in}}%
\pgfusepath{clip}%
\pgfsetbuttcap%
\pgfsetroundjoin%
\definecolor{currentfill}{rgb}{0.000000,0.000000,0.000000}%
\pgfsetfillcolor{currentfill}%
\pgfsetlinewidth{1.003750pt}%
\definecolor{currentstroke}{rgb}{0.000000,0.000000,0.000000}%
\pgfsetstrokecolor{currentstroke}%
\pgfsetdash{}{0pt}%
\pgfpathmoveto{\pgfqpoint{2.518786in}{1.171247in}}%
\pgfpathcurveto{\pgfqpoint{2.529836in}{1.171247in}}{\pgfqpoint{2.540435in}{1.175637in}}{\pgfqpoint{2.548249in}{1.183451in}}%
\pgfpathcurveto{\pgfqpoint{2.556062in}{1.191264in}}{\pgfqpoint{2.560452in}{1.201863in}}{\pgfqpoint{2.560452in}{1.212913in}}%
\pgfpathcurveto{\pgfqpoint{2.560452in}{1.223964in}}{\pgfqpoint{2.556062in}{1.234563in}}{\pgfqpoint{2.548249in}{1.242376in}}%
\pgfpathcurveto{\pgfqpoint{2.540435in}{1.250190in}}{\pgfqpoint{2.529836in}{1.254580in}}{\pgfqpoint{2.518786in}{1.254580in}}%
\pgfpathcurveto{\pgfqpoint{2.507736in}{1.254580in}}{\pgfqpoint{2.497137in}{1.250190in}}{\pgfqpoint{2.489323in}{1.242376in}}%
\pgfpathcurveto{\pgfqpoint{2.481509in}{1.234563in}}{\pgfqpoint{2.477119in}{1.223964in}}{\pgfqpoint{2.477119in}{1.212913in}}%
\pgfpathcurveto{\pgfqpoint{2.477119in}{1.201863in}}{\pgfqpoint{2.481509in}{1.191264in}}{\pgfqpoint{2.489323in}{1.183451in}}%
\pgfpathcurveto{\pgfqpoint{2.497137in}{1.175637in}}{\pgfqpoint{2.507736in}{1.171247in}}{\pgfqpoint{2.518786in}{1.171247in}}%
\pgfpathclose%
\pgfusepath{stroke,fill}%
\end{pgfscope}%
\begin{pgfscope}%
\pgfpathrectangle{\pgfqpoint{0.800000in}{0.528000in}}{\pgfqpoint{4.960000in}{3.696000in}}%
\pgfusepath{clip}%
\pgfsetbuttcap%
\pgfsetroundjoin%
\definecolor{currentfill}{rgb}{0.000000,0.000000,0.000000}%
\pgfsetfillcolor{currentfill}%
\pgfsetlinewidth{1.003750pt}%
\definecolor{currentstroke}{rgb}{0.000000,0.000000,0.000000}%
\pgfsetstrokecolor{currentstroke}%
\pgfsetdash{}{0pt}%
\pgfpathmoveto{\pgfqpoint{2.518786in}{1.063759in}}%
\pgfpathcurveto{\pgfqpoint{2.529836in}{1.063759in}}{\pgfqpoint{2.540435in}{1.068150in}}{\pgfqpoint{2.548249in}{1.075963in}}%
\pgfpathcurveto{\pgfqpoint{2.556062in}{1.083777in}}{\pgfqpoint{2.560452in}{1.094376in}}{\pgfqpoint{2.560452in}{1.105426in}}%
\pgfpathcurveto{\pgfqpoint{2.560452in}{1.116476in}}{\pgfqpoint{2.556062in}{1.127075in}}{\pgfqpoint{2.548249in}{1.134889in}}%
\pgfpathcurveto{\pgfqpoint{2.540435in}{1.142702in}}{\pgfqpoint{2.529836in}{1.147093in}}{\pgfqpoint{2.518786in}{1.147093in}}%
\pgfpathcurveto{\pgfqpoint{2.507736in}{1.147093in}}{\pgfqpoint{2.497137in}{1.142702in}}{\pgfqpoint{2.489323in}{1.134889in}}%
\pgfpathcurveto{\pgfqpoint{2.481509in}{1.127075in}}{\pgfqpoint{2.477119in}{1.116476in}}{\pgfqpoint{2.477119in}{1.105426in}}%
\pgfpathcurveto{\pgfqpoint{2.477119in}{1.094376in}}{\pgfqpoint{2.481509in}{1.083777in}}{\pgfqpoint{2.489323in}{1.075963in}}%
\pgfpathcurveto{\pgfqpoint{2.497137in}{1.068150in}}{\pgfqpoint{2.507736in}{1.063759in}}{\pgfqpoint{2.518786in}{1.063759in}}%
\pgfpathclose%
\pgfusepath{stroke,fill}%
\end{pgfscope}%
\begin{pgfscope}%
\pgfpathrectangle{\pgfqpoint{0.800000in}{0.528000in}}{\pgfqpoint{4.960000in}{3.696000in}}%
\pgfusepath{clip}%
\pgfsetbuttcap%
\pgfsetroundjoin%
\definecolor{currentfill}{rgb}{0.000000,0.000000,0.000000}%
\pgfsetfillcolor{currentfill}%
\pgfsetlinewidth{1.003750pt}%
\definecolor{currentstroke}{rgb}{0.000000,0.000000,0.000000}%
\pgfsetstrokecolor{currentstroke}%
\pgfsetdash{}{0pt}%
\pgfpathmoveto{\pgfqpoint{2.518786in}{1.192744in}}%
\pgfpathcurveto{\pgfqpoint{2.529836in}{1.192744in}}{\pgfqpoint{2.540435in}{1.197135in}}{\pgfqpoint{2.548249in}{1.204948in}}%
\pgfpathcurveto{\pgfqpoint{2.556062in}{1.212762in}}{\pgfqpoint{2.560452in}{1.223361in}}{\pgfqpoint{2.560452in}{1.234411in}}%
\pgfpathcurveto{\pgfqpoint{2.560452in}{1.245461in}}{\pgfqpoint{2.556062in}{1.256060in}}{\pgfqpoint{2.548249in}{1.263874in}}%
\pgfpathcurveto{\pgfqpoint{2.540435in}{1.271687in}}{\pgfqpoint{2.529836in}{1.276078in}}{\pgfqpoint{2.518786in}{1.276078in}}%
\pgfpathcurveto{\pgfqpoint{2.507736in}{1.276078in}}{\pgfqpoint{2.497137in}{1.271687in}}{\pgfqpoint{2.489323in}{1.263874in}}%
\pgfpathcurveto{\pgfqpoint{2.481509in}{1.256060in}}{\pgfqpoint{2.477119in}{1.245461in}}{\pgfqpoint{2.477119in}{1.234411in}}%
\pgfpathcurveto{\pgfqpoint{2.477119in}{1.223361in}}{\pgfqpoint{2.481509in}{1.212762in}}{\pgfqpoint{2.489323in}{1.204948in}}%
\pgfpathcurveto{\pgfqpoint{2.497137in}{1.197135in}}{\pgfqpoint{2.507736in}{1.192744in}}{\pgfqpoint{2.518786in}{1.192744in}}%
\pgfpathclose%
\pgfusepath{stroke,fill}%
\end{pgfscope}%
\begin{pgfscope}%
\pgfpathrectangle{\pgfqpoint{0.800000in}{0.528000in}}{\pgfqpoint{4.960000in}{3.696000in}}%
\pgfusepath{clip}%
\pgfsetbuttcap%
\pgfsetroundjoin%
\definecolor{currentfill}{rgb}{0.000000,0.000000,0.000000}%
\pgfsetfillcolor{currentfill}%
\pgfsetlinewidth{1.003750pt}%
\definecolor{currentstroke}{rgb}{0.000000,0.000000,0.000000}%
\pgfsetstrokecolor{currentstroke}%
\pgfsetdash{}{0pt}%
\pgfpathmoveto{\pgfqpoint{2.518786in}{1.278734in}}%
\pgfpathcurveto{\pgfqpoint{2.529836in}{1.278734in}}{\pgfqpoint{2.540435in}{1.283125in}}{\pgfqpoint{2.548249in}{1.290938in}}%
\pgfpathcurveto{\pgfqpoint{2.556062in}{1.298752in}}{\pgfqpoint{2.560452in}{1.309351in}}{\pgfqpoint{2.560452in}{1.320401in}}%
\pgfpathcurveto{\pgfqpoint{2.560452in}{1.331451in}}{\pgfqpoint{2.556062in}{1.342050in}}{\pgfqpoint{2.548249in}{1.349864in}}%
\pgfpathcurveto{\pgfqpoint{2.540435in}{1.357677in}}{\pgfqpoint{2.529836in}{1.362068in}}{\pgfqpoint{2.518786in}{1.362068in}}%
\pgfpathcurveto{\pgfqpoint{2.507736in}{1.362068in}}{\pgfqpoint{2.497137in}{1.357677in}}{\pgfqpoint{2.489323in}{1.349864in}}%
\pgfpathcurveto{\pgfqpoint{2.481509in}{1.342050in}}{\pgfqpoint{2.477119in}{1.331451in}}{\pgfqpoint{2.477119in}{1.320401in}}%
\pgfpathcurveto{\pgfqpoint{2.477119in}{1.309351in}}{\pgfqpoint{2.481509in}{1.298752in}}{\pgfqpoint{2.489323in}{1.290938in}}%
\pgfpathcurveto{\pgfqpoint{2.497137in}{1.283125in}}{\pgfqpoint{2.507736in}{1.278734in}}{\pgfqpoint{2.518786in}{1.278734in}}%
\pgfpathclose%
\pgfusepath{stroke,fill}%
\end{pgfscope}%
\begin{pgfscope}%
\pgfpathrectangle{\pgfqpoint{0.800000in}{0.528000in}}{\pgfqpoint{4.960000in}{3.696000in}}%
\pgfusepath{clip}%
\pgfsetbuttcap%
\pgfsetroundjoin%
\definecolor{currentfill}{rgb}{0.000000,0.000000,0.000000}%
\pgfsetfillcolor{currentfill}%
\pgfsetlinewidth{1.003750pt}%
\definecolor{currentstroke}{rgb}{0.000000,0.000000,0.000000}%
\pgfsetstrokecolor{currentstroke}%
\pgfsetdash{}{0pt}%
\pgfpathmoveto{\pgfqpoint{2.518786in}{1.149749in}}%
\pgfpathcurveto{\pgfqpoint{2.529836in}{1.149749in}}{\pgfqpoint{2.540435in}{1.154140in}}{\pgfqpoint{2.548249in}{1.161953in}}%
\pgfpathcurveto{\pgfqpoint{2.556062in}{1.169767in}}{\pgfqpoint{2.560452in}{1.180366in}}{\pgfqpoint{2.560452in}{1.191416in}}%
\pgfpathcurveto{\pgfqpoint{2.560452in}{1.202466in}}{\pgfqpoint{2.556062in}{1.213065in}}{\pgfqpoint{2.548249in}{1.220879in}}%
\pgfpathcurveto{\pgfqpoint{2.540435in}{1.228692in}}{\pgfqpoint{2.529836in}{1.233083in}}{\pgfqpoint{2.518786in}{1.233083in}}%
\pgfpathcurveto{\pgfqpoint{2.507736in}{1.233083in}}{\pgfqpoint{2.497137in}{1.228692in}}{\pgfqpoint{2.489323in}{1.220879in}}%
\pgfpathcurveto{\pgfqpoint{2.481509in}{1.213065in}}{\pgfqpoint{2.477119in}{1.202466in}}{\pgfqpoint{2.477119in}{1.191416in}}%
\pgfpathcurveto{\pgfqpoint{2.477119in}{1.180366in}}{\pgfqpoint{2.481509in}{1.169767in}}{\pgfqpoint{2.489323in}{1.161953in}}%
\pgfpathcurveto{\pgfqpoint{2.497137in}{1.154140in}}{\pgfqpoint{2.507736in}{1.149749in}}{\pgfqpoint{2.518786in}{1.149749in}}%
\pgfpathclose%
\pgfusepath{stroke,fill}%
\end{pgfscope}%
\begin{pgfscope}%
\pgfpathrectangle{\pgfqpoint{0.800000in}{0.528000in}}{\pgfqpoint{4.960000in}{3.696000in}}%
\pgfusepath{clip}%
\pgfsetbuttcap%
\pgfsetroundjoin%
\definecolor{currentfill}{rgb}{0.000000,0.000000,0.000000}%
\pgfsetfillcolor{currentfill}%
\pgfsetlinewidth{1.003750pt}%
\definecolor{currentstroke}{rgb}{0.000000,0.000000,0.000000}%
\pgfsetstrokecolor{currentstroke}%
\pgfsetdash{}{0pt}%
\pgfpathmoveto{\pgfqpoint{2.518786in}{1.106754in}}%
\pgfpathcurveto{\pgfqpoint{2.529836in}{1.106754in}}{\pgfqpoint{2.540435in}{1.111145in}}{\pgfqpoint{2.548249in}{1.118958in}}%
\pgfpathcurveto{\pgfqpoint{2.556062in}{1.126772in}}{\pgfqpoint{2.560452in}{1.137371in}}{\pgfqpoint{2.560452in}{1.148421in}}%
\pgfpathcurveto{\pgfqpoint{2.560452in}{1.159471in}}{\pgfqpoint{2.556062in}{1.170070in}}{\pgfqpoint{2.548249in}{1.177884in}}%
\pgfpathcurveto{\pgfqpoint{2.540435in}{1.185697in}}{\pgfqpoint{2.529836in}{1.190088in}}{\pgfqpoint{2.518786in}{1.190088in}}%
\pgfpathcurveto{\pgfqpoint{2.507736in}{1.190088in}}{\pgfqpoint{2.497137in}{1.185697in}}{\pgfqpoint{2.489323in}{1.177884in}}%
\pgfpathcurveto{\pgfqpoint{2.481509in}{1.170070in}}{\pgfqpoint{2.477119in}{1.159471in}}{\pgfqpoint{2.477119in}{1.148421in}}%
\pgfpathcurveto{\pgfqpoint{2.477119in}{1.137371in}}{\pgfqpoint{2.481509in}{1.126772in}}{\pgfqpoint{2.489323in}{1.118958in}}%
\pgfpathcurveto{\pgfqpoint{2.497137in}{1.111145in}}{\pgfqpoint{2.507736in}{1.106754in}}{\pgfqpoint{2.518786in}{1.106754in}}%
\pgfpathclose%
\pgfusepath{stroke,fill}%
\end{pgfscope}%
\begin{pgfscope}%
\pgfpathrectangle{\pgfqpoint{0.800000in}{0.528000in}}{\pgfqpoint{4.960000in}{3.696000in}}%
\pgfusepath{clip}%
\pgfsetbuttcap%
\pgfsetroundjoin%
\definecolor{currentfill}{rgb}{0.000000,0.000000,0.000000}%
\pgfsetfillcolor{currentfill}%
\pgfsetlinewidth{1.003750pt}%
\definecolor{currentstroke}{rgb}{0.000000,0.000000,0.000000}%
\pgfsetstrokecolor{currentstroke}%
\pgfsetdash{}{0pt}%
\pgfpathmoveto{\pgfqpoint{2.518786in}{1.171247in}}%
\pgfpathcurveto{\pgfqpoint{2.529836in}{1.171247in}}{\pgfqpoint{2.540435in}{1.175637in}}{\pgfqpoint{2.548249in}{1.183451in}}%
\pgfpathcurveto{\pgfqpoint{2.556062in}{1.191264in}}{\pgfqpoint{2.560452in}{1.201863in}}{\pgfqpoint{2.560452in}{1.212913in}}%
\pgfpathcurveto{\pgfqpoint{2.560452in}{1.223964in}}{\pgfqpoint{2.556062in}{1.234563in}}{\pgfqpoint{2.548249in}{1.242376in}}%
\pgfpathcurveto{\pgfqpoint{2.540435in}{1.250190in}}{\pgfqpoint{2.529836in}{1.254580in}}{\pgfqpoint{2.518786in}{1.254580in}}%
\pgfpathcurveto{\pgfqpoint{2.507736in}{1.254580in}}{\pgfqpoint{2.497137in}{1.250190in}}{\pgfqpoint{2.489323in}{1.242376in}}%
\pgfpathcurveto{\pgfqpoint{2.481509in}{1.234563in}}{\pgfqpoint{2.477119in}{1.223964in}}{\pgfqpoint{2.477119in}{1.212913in}}%
\pgfpathcurveto{\pgfqpoint{2.477119in}{1.201863in}}{\pgfqpoint{2.481509in}{1.191264in}}{\pgfqpoint{2.489323in}{1.183451in}}%
\pgfpathcurveto{\pgfqpoint{2.497137in}{1.175637in}}{\pgfqpoint{2.507736in}{1.171247in}}{\pgfqpoint{2.518786in}{1.171247in}}%
\pgfpathclose%
\pgfusepath{stroke,fill}%
\end{pgfscope}%
\begin{pgfscope}%
\pgfpathrectangle{\pgfqpoint{0.800000in}{0.528000in}}{\pgfqpoint{4.960000in}{3.696000in}}%
\pgfusepath{clip}%
\pgfsetbuttcap%
\pgfsetroundjoin%
\definecolor{currentfill}{rgb}{0.000000,0.000000,0.000000}%
\pgfsetfillcolor{currentfill}%
\pgfsetlinewidth{1.003750pt}%
\definecolor{currentstroke}{rgb}{0.000000,0.000000,0.000000}%
\pgfsetstrokecolor{currentstroke}%
\pgfsetdash{}{0pt}%
\pgfpathmoveto{\pgfqpoint{2.518786in}{1.128252in}}%
\pgfpathcurveto{\pgfqpoint{2.529836in}{1.128252in}}{\pgfqpoint{2.540435in}{1.132642in}}{\pgfqpoint{2.548249in}{1.140456in}}%
\pgfpathcurveto{\pgfqpoint{2.556062in}{1.148269in}}{\pgfqpoint{2.560452in}{1.158868in}}{\pgfqpoint{2.560452in}{1.169918in}}%
\pgfpathcurveto{\pgfqpoint{2.560452in}{1.180969in}}{\pgfqpoint{2.556062in}{1.191568in}}{\pgfqpoint{2.548249in}{1.199381in}}%
\pgfpathcurveto{\pgfqpoint{2.540435in}{1.207195in}}{\pgfqpoint{2.529836in}{1.211585in}}{\pgfqpoint{2.518786in}{1.211585in}}%
\pgfpathcurveto{\pgfqpoint{2.507736in}{1.211585in}}{\pgfqpoint{2.497137in}{1.207195in}}{\pgfqpoint{2.489323in}{1.199381in}}%
\pgfpathcurveto{\pgfqpoint{2.481509in}{1.191568in}}{\pgfqpoint{2.477119in}{1.180969in}}{\pgfqpoint{2.477119in}{1.169918in}}%
\pgfpathcurveto{\pgfqpoint{2.477119in}{1.158868in}}{\pgfqpoint{2.481509in}{1.148269in}}{\pgfqpoint{2.489323in}{1.140456in}}%
\pgfpathcurveto{\pgfqpoint{2.497137in}{1.132642in}}{\pgfqpoint{2.507736in}{1.128252in}}{\pgfqpoint{2.518786in}{1.128252in}}%
\pgfpathclose%
\pgfusepath{stroke,fill}%
\end{pgfscope}%
\begin{pgfscope}%
\pgfpathrectangle{\pgfqpoint{0.800000in}{0.528000in}}{\pgfqpoint{4.960000in}{3.696000in}}%
\pgfusepath{clip}%
\pgfsetbuttcap%
\pgfsetroundjoin%
\definecolor{currentfill}{rgb}{0.000000,0.000000,0.000000}%
\pgfsetfillcolor{currentfill}%
\pgfsetlinewidth{1.003750pt}%
\definecolor{currentstroke}{rgb}{0.000000,0.000000,0.000000}%
\pgfsetstrokecolor{currentstroke}%
\pgfsetdash{}{0pt}%
\pgfpathmoveto{\pgfqpoint{2.518786in}{1.171247in}}%
\pgfpathcurveto{\pgfqpoint{2.529836in}{1.171247in}}{\pgfqpoint{2.540435in}{1.175637in}}{\pgfqpoint{2.548249in}{1.183451in}}%
\pgfpathcurveto{\pgfqpoint{2.556062in}{1.191264in}}{\pgfqpoint{2.560452in}{1.201863in}}{\pgfqpoint{2.560452in}{1.212913in}}%
\pgfpathcurveto{\pgfqpoint{2.560452in}{1.223964in}}{\pgfqpoint{2.556062in}{1.234563in}}{\pgfqpoint{2.548249in}{1.242376in}}%
\pgfpathcurveto{\pgfqpoint{2.540435in}{1.250190in}}{\pgfqpoint{2.529836in}{1.254580in}}{\pgfqpoint{2.518786in}{1.254580in}}%
\pgfpathcurveto{\pgfqpoint{2.507736in}{1.254580in}}{\pgfqpoint{2.497137in}{1.250190in}}{\pgfqpoint{2.489323in}{1.242376in}}%
\pgfpathcurveto{\pgfqpoint{2.481509in}{1.234563in}}{\pgfqpoint{2.477119in}{1.223964in}}{\pgfqpoint{2.477119in}{1.212913in}}%
\pgfpathcurveto{\pgfqpoint{2.477119in}{1.201863in}}{\pgfqpoint{2.481509in}{1.191264in}}{\pgfqpoint{2.489323in}{1.183451in}}%
\pgfpathcurveto{\pgfqpoint{2.497137in}{1.175637in}}{\pgfqpoint{2.507736in}{1.171247in}}{\pgfqpoint{2.518786in}{1.171247in}}%
\pgfpathclose%
\pgfusepath{stroke,fill}%
\end{pgfscope}%
\begin{pgfscope}%
\pgfpathrectangle{\pgfqpoint{0.800000in}{0.528000in}}{\pgfqpoint{4.960000in}{3.696000in}}%
\pgfusepath{clip}%
\pgfsetbuttcap%
\pgfsetroundjoin%
\definecolor{currentfill}{rgb}{0.000000,0.000000,0.000000}%
\pgfsetfillcolor{currentfill}%
\pgfsetlinewidth{1.003750pt}%
\definecolor{currentstroke}{rgb}{0.000000,0.000000,0.000000}%
\pgfsetstrokecolor{currentstroke}%
\pgfsetdash{}{0pt}%
\pgfpathmoveto{\pgfqpoint{2.518786in}{1.171247in}}%
\pgfpathcurveto{\pgfqpoint{2.529836in}{1.171247in}}{\pgfqpoint{2.540435in}{1.175637in}}{\pgfqpoint{2.548249in}{1.183451in}}%
\pgfpathcurveto{\pgfqpoint{2.556062in}{1.191264in}}{\pgfqpoint{2.560452in}{1.201863in}}{\pgfqpoint{2.560452in}{1.212913in}}%
\pgfpathcurveto{\pgfqpoint{2.560452in}{1.223964in}}{\pgfqpoint{2.556062in}{1.234563in}}{\pgfqpoint{2.548249in}{1.242376in}}%
\pgfpathcurveto{\pgfqpoint{2.540435in}{1.250190in}}{\pgfqpoint{2.529836in}{1.254580in}}{\pgfqpoint{2.518786in}{1.254580in}}%
\pgfpathcurveto{\pgfqpoint{2.507736in}{1.254580in}}{\pgfqpoint{2.497137in}{1.250190in}}{\pgfqpoint{2.489323in}{1.242376in}}%
\pgfpathcurveto{\pgfqpoint{2.481509in}{1.234563in}}{\pgfqpoint{2.477119in}{1.223964in}}{\pgfqpoint{2.477119in}{1.212913in}}%
\pgfpathcurveto{\pgfqpoint{2.477119in}{1.201863in}}{\pgfqpoint{2.481509in}{1.191264in}}{\pgfqpoint{2.489323in}{1.183451in}}%
\pgfpathcurveto{\pgfqpoint{2.497137in}{1.175637in}}{\pgfqpoint{2.507736in}{1.171247in}}{\pgfqpoint{2.518786in}{1.171247in}}%
\pgfpathclose%
\pgfusepath{stroke,fill}%
\end{pgfscope}%
\begin{pgfscope}%
\pgfpathrectangle{\pgfqpoint{0.800000in}{0.528000in}}{\pgfqpoint{4.960000in}{3.696000in}}%
\pgfusepath{clip}%
\pgfsetbuttcap%
\pgfsetroundjoin%
\definecolor{currentfill}{rgb}{0.000000,0.000000,0.000000}%
\pgfsetfillcolor{currentfill}%
\pgfsetlinewidth{1.003750pt}%
\definecolor{currentstroke}{rgb}{0.000000,0.000000,0.000000}%
\pgfsetstrokecolor{currentstroke}%
\pgfsetdash{}{0pt}%
\pgfpathmoveto{\pgfqpoint{2.518786in}{1.171247in}}%
\pgfpathcurveto{\pgfqpoint{2.529836in}{1.171247in}}{\pgfqpoint{2.540435in}{1.175637in}}{\pgfqpoint{2.548249in}{1.183451in}}%
\pgfpathcurveto{\pgfqpoint{2.556062in}{1.191264in}}{\pgfqpoint{2.560452in}{1.201863in}}{\pgfqpoint{2.560452in}{1.212913in}}%
\pgfpathcurveto{\pgfqpoint{2.560452in}{1.223964in}}{\pgfqpoint{2.556062in}{1.234563in}}{\pgfqpoint{2.548249in}{1.242376in}}%
\pgfpathcurveto{\pgfqpoint{2.540435in}{1.250190in}}{\pgfqpoint{2.529836in}{1.254580in}}{\pgfqpoint{2.518786in}{1.254580in}}%
\pgfpathcurveto{\pgfqpoint{2.507736in}{1.254580in}}{\pgfqpoint{2.497137in}{1.250190in}}{\pgfqpoint{2.489323in}{1.242376in}}%
\pgfpathcurveto{\pgfqpoint{2.481509in}{1.234563in}}{\pgfqpoint{2.477119in}{1.223964in}}{\pgfqpoint{2.477119in}{1.212913in}}%
\pgfpathcurveto{\pgfqpoint{2.477119in}{1.201863in}}{\pgfqpoint{2.481509in}{1.191264in}}{\pgfqpoint{2.489323in}{1.183451in}}%
\pgfpathcurveto{\pgfqpoint{2.497137in}{1.175637in}}{\pgfqpoint{2.507736in}{1.171247in}}{\pgfqpoint{2.518786in}{1.171247in}}%
\pgfpathclose%
\pgfusepath{stroke,fill}%
\end{pgfscope}%
\begin{pgfscope}%
\pgfpathrectangle{\pgfqpoint{0.800000in}{0.528000in}}{\pgfqpoint{4.960000in}{3.696000in}}%
\pgfusepath{clip}%
\pgfsetbuttcap%
\pgfsetroundjoin%
\definecolor{currentfill}{rgb}{0.000000,0.000000,0.000000}%
\pgfsetfillcolor{currentfill}%
\pgfsetlinewidth{1.003750pt}%
\definecolor{currentstroke}{rgb}{0.000000,0.000000,0.000000}%
\pgfsetstrokecolor{currentstroke}%
\pgfsetdash{}{0pt}%
\pgfpathmoveto{\pgfqpoint{2.518786in}{1.128252in}}%
\pgfpathcurveto{\pgfqpoint{2.529836in}{1.128252in}}{\pgfqpoint{2.540435in}{1.132642in}}{\pgfqpoint{2.548249in}{1.140456in}}%
\pgfpathcurveto{\pgfqpoint{2.556062in}{1.148269in}}{\pgfqpoint{2.560452in}{1.158868in}}{\pgfqpoint{2.560452in}{1.169918in}}%
\pgfpathcurveto{\pgfqpoint{2.560452in}{1.180969in}}{\pgfqpoint{2.556062in}{1.191568in}}{\pgfqpoint{2.548249in}{1.199381in}}%
\pgfpathcurveto{\pgfqpoint{2.540435in}{1.207195in}}{\pgfqpoint{2.529836in}{1.211585in}}{\pgfqpoint{2.518786in}{1.211585in}}%
\pgfpathcurveto{\pgfqpoint{2.507736in}{1.211585in}}{\pgfqpoint{2.497137in}{1.207195in}}{\pgfqpoint{2.489323in}{1.199381in}}%
\pgfpathcurveto{\pgfqpoint{2.481509in}{1.191568in}}{\pgfqpoint{2.477119in}{1.180969in}}{\pgfqpoint{2.477119in}{1.169918in}}%
\pgfpathcurveto{\pgfqpoint{2.477119in}{1.158868in}}{\pgfqpoint{2.481509in}{1.148269in}}{\pgfqpoint{2.489323in}{1.140456in}}%
\pgfpathcurveto{\pgfqpoint{2.497137in}{1.132642in}}{\pgfqpoint{2.507736in}{1.128252in}}{\pgfqpoint{2.518786in}{1.128252in}}%
\pgfpathclose%
\pgfusepath{stroke,fill}%
\end{pgfscope}%
\begin{pgfscope}%
\pgfpathrectangle{\pgfqpoint{0.800000in}{0.528000in}}{\pgfqpoint{4.960000in}{3.696000in}}%
\pgfusepath{clip}%
\pgfsetbuttcap%
\pgfsetroundjoin%
\definecolor{currentfill}{rgb}{0.000000,0.000000,0.000000}%
\pgfsetfillcolor{currentfill}%
\pgfsetlinewidth{1.003750pt}%
\definecolor{currentstroke}{rgb}{0.000000,0.000000,0.000000}%
\pgfsetstrokecolor{currentstroke}%
\pgfsetdash{}{0pt}%
\pgfpathmoveto{\pgfqpoint{2.518786in}{1.128252in}}%
\pgfpathcurveto{\pgfqpoint{2.529836in}{1.128252in}}{\pgfqpoint{2.540435in}{1.132642in}}{\pgfqpoint{2.548249in}{1.140456in}}%
\pgfpathcurveto{\pgfqpoint{2.556062in}{1.148269in}}{\pgfqpoint{2.560452in}{1.158868in}}{\pgfqpoint{2.560452in}{1.169918in}}%
\pgfpathcurveto{\pgfqpoint{2.560452in}{1.180969in}}{\pgfqpoint{2.556062in}{1.191568in}}{\pgfqpoint{2.548249in}{1.199381in}}%
\pgfpathcurveto{\pgfqpoint{2.540435in}{1.207195in}}{\pgfqpoint{2.529836in}{1.211585in}}{\pgfqpoint{2.518786in}{1.211585in}}%
\pgfpathcurveto{\pgfqpoint{2.507736in}{1.211585in}}{\pgfqpoint{2.497137in}{1.207195in}}{\pgfqpoint{2.489323in}{1.199381in}}%
\pgfpathcurveto{\pgfqpoint{2.481509in}{1.191568in}}{\pgfqpoint{2.477119in}{1.180969in}}{\pgfqpoint{2.477119in}{1.169918in}}%
\pgfpathcurveto{\pgfqpoint{2.477119in}{1.158868in}}{\pgfqpoint{2.481509in}{1.148269in}}{\pgfqpoint{2.489323in}{1.140456in}}%
\pgfpathcurveto{\pgfqpoint{2.497137in}{1.132642in}}{\pgfqpoint{2.507736in}{1.128252in}}{\pgfqpoint{2.518786in}{1.128252in}}%
\pgfpathclose%
\pgfusepath{stroke,fill}%
\end{pgfscope}%
\begin{pgfscope}%
\pgfpathrectangle{\pgfqpoint{0.800000in}{0.528000in}}{\pgfqpoint{4.960000in}{3.696000in}}%
\pgfusepath{clip}%
\pgfsetbuttcap%
\pgfsetroundjoin%
\definecolor{currentfill}{rgb}{0.000000,0.000000,0.000000}%
\pgfsetfillcolor{currentfill}%
\pgfsetlinewidth{1.003750pt}%
\definecolor{currentstroke}{rgb}{0.000000,0.000000,0.000000}%
\pgfsetstrokecolor{currentstroke}%
\pgfsetdash{}{0pt}%
\pgfpathmoveto{\pgfqpoint{2.518786in}{1.149749in}}%
\pgfpathcurveto{\pgfqpoint{2.529836in}{1.149749in}}{\pgfqpoint{2.540435in}{1.154140in}}{\pgfqpoint{2.548249in}{1.161953in}}%
\pgfpathcurveto{\pgfqpoint{2.556062in}{1.169767in}}{\pgfqpoint{2.560452in}{1.180366in}}{\pgfqpoint{2.560452in}{1.191416in}}%
\pgfpathcurveto{\pgfqpoint{2.560452in}{1.202466in}}{\pgfqpoint{2.556062in}{1.213065in}}{\pgfqpoint{2.548249in}{1.220879in}}%
\pgfpathcurveto{\pgfqpoint{2.540435in}{1.228692in}}{\pgfqpoint{2.529836in}{1.233083in}}{\pgfqpoint{2.518786in}{1.233083in}}%
\pgfpathcurveto{\pgfqpoint{2.507736in}{1.233083in}}{\pgfqpoint{2.497137in}{1.228692in}}{\pgfqpoint{2.489323in}{1.220879in}}%
\pgfpathcurveto{\pgfqpoint{2.481509in}{1.213065in}}{\pgfqpoint{2.477119in}{1.202466in}}{\pgfqpoint{2.477119in}{1.191416in}}%
\pgfpathcurveto{\pgfqpoint{2.477119in}{1.180366in}}{\pgfqpoint{2.481509in}{1.169767in}}{\pgfqpoint{2.489323in}{1.161953in}}%
\pgfpathcurveto{\pgfqpoint{2.497137in}{1.154140in}}{\pgfqpoint{2.507736in}{1.149749in}}{\pgfqpoint{2.518786in}{1.149749in}}%
\pgfpathclose%
\pgfusepath{stroke,fill}%
\end{pgfscope}%
\begin{pgfscope}%
\pgfpathrectangle{\pgfqpoint{0.800000in}{0.528000in}}{\pgfqpoint{4.960000in}{3.696000in}}%
\pgfusepath{clip}%
\pgfsetbuttcap%
\pgfsetroundjoin%
\definecolor{currentfill}{rgb}{0.000000,0.000000,0.000000}%
\pgfsetfillcolor{currentfill}%
\pgfsetlinewidth{1.003750pt}%
\definecolor{currentstroke}{rgb}{0.000000,0.000000,0.000000}%
\pgfsetstrokecolor{currentstroke}%
\pgfsetdash{}{0pt}%
\pgfpathmoveto{\pgfqpoint{2.518786in}{1.149749in}}%
\pgfpathcurveto{\pgfqpoint{2.529836in}{1.149749in}}{\pgfqpoint{2.540435in}{1.154140in}}{\pgfqpoint{2.548249in}{1.161953in}}%
\pgfpathcurveto{\pgfqpoint{2.556062in}{1.169767in}}{\pgfqpoint{2.560452in}{1.180366in}}{\pgfqpoint{2.560452in}{1.191416in}}%
\pgfpathcurveto{\pgfqpoint{2.560452in}{1.202466in}}{\pgfqpoint{2.556062in}{1.213065in}}{\pgfqpoint{2.548249in}{1.220879in}}%
\pgfpathcurveto{\pgfqpoint{2.540435in}{1.228692in}}{\pgfqpoint{2.529836in}{1.233083in}}{\pgfqpoint{2.518786in}{1.233083in}}%
\pgfpathcurveto{\pgfqpoint{2.507736in}{1.233083in}}{\pgfqpoint{2.497137in}{1.228692in}}{\pgfqpoint{2.489323in}{1.220879in}}%
\pgfpathcurveto{\pgfqpoint{2.481509in}{1.213065in}}{\pgfqpoint{2.477119in}{1.202466in}}{\pgfqpoint{2.477119in}{1.191416in}}%
\pgfpathcurveto{\pgfqpoint{2.477119in}{1.180366in}}{\pgfqpoint{2.481509in}{1.169767in}}{\pgfqpoint{2.489323in}{1.161953in}}%
\pgfpathcurveto{\pgfqpoint{2.497137in}{1.154140in}}{\pgfqpoint{2.507736in}{1.149749in}}{\pgfqpoint{2.518786in}{1.149749in}}%
\pgfpathclose%
\pgfusepath{stroke,fill}%
\end{pgfscope}%
\begin{pgfscope}%
\pgfpathrectangle{\pgfqpoint{0.800000in}{0.528000in}}{\pgfqpoint{4.960000in}{3.696000in}}%
\pgfusepath{clip}%
\pgfsetbuttcap%
\pgfsetroundjoin%
\definecolor{currentfill}{rgb}{0.000000,0.000000,0.000000}%
\pgfsetfillcolor{currentfill}%
\pgfsetlinewidth{1.003750pt}%
\definecolor{currentstroke}{rgb}{0.000000,0.000000,0.000000}%
\pgfsetstrokecolor{currentstroke}%
\pgfsetdash{}{0pt}%
\pgfpathmoveto{\pgfqpoint{2.518786in}{1.257237in}}%
\pgfpathcurveto{\pgfqpoint{2.529836in}{1.257237in}}{\pgfqpoint{2.540435in}{1.261627in}}{\pgfqpoint{2.548249in}{1.269441in}}%
\pgfpathcurveto{\pgfqpoint{2.556062in}{1.277254in}}{\pgfqpoint{2.560452in}{1.287853in}}{\pgfqpoint{2.560452in}{1.298903in}}%
\pgfpathcurveto{\pgfqpoint{2.560452in}{1.309954in}}{\pgfqpoint{2.556062in}{1.320553in}}{\pgfqpoint{2.548249in}{1.328366in}}%
\pgfpathcurveto{\pgfqpoint{2.540435in}{1.336180in}}{\pgfqpoint{2.529836in}{1.340570in}}{\pgfqpoint{2.518786in}{1.340570in}}%
\pgfpathcurveto{\pgfqpoint{2.507736in}{1.340570in}}{\pgfqpoint{2.497137in}{1.336180in}}{\pgfqpoint{2.489323in}{1.328366in}}%
\pgfpathcurveto{\pgfqpoint{2.481509in}{1.320553in}}{\pgfqpoint{2.477119in}{1.309954in}}{\pgfqpoint{2.477119in}{1.298903in}}%
\pgfpathcurveto{\pgfqpoint{2.477119in}{1.287853in}}{\pgfqpoint{2.481509in}{1.277254in}}{\pgfqpoint{2.489323in}{1.269441in}}%
\pgfpathcurveto{\pgfqpoint{2.497137in}{1.261627in}}{\pgfqpoint{2.507736in}{1.257237in}}{\pgfqpoint{2.518786in}{1.257237in}}%
\pgfpathclose%
\pgfusepath{stroke,fill}%
\end{pgfscope}%
\begin{pgfscope}%
\pgfpathrectangle{\pgfqpoint{0.800000in}{0.528000in}}{\pgfqpoint{4.960000in}{3.696000in}}%
\pgfusepath{clip}%
\pgfsetbuttcap%
\pgfsetroundjoin%
\definecolor{currentfill}{rgb}{0.000000,0.000000,0.000000}%
\pgfsetfillcolor{currentfill}%
\pgfsetlinewidth{1.003750pt}%
\definecolor{currentstroke}{rgb}{0.000000,0.000000,0.000000}%
\pgfsetstrokecolor{currentstroke}%
\pgfsetdash{}{0pt}%
\pgfpathmoveto{\pgfqpoint{2.518786in}{1.214242in}}%
\pgfpathcurveto{\pgfqpoint{2.529836in}{1.214242in}}{\pgfqpoint{2.540435in}{1.218632in}}{\pgfqpoint{2.548249in}{1.226446in}}%
\pgfpathcurveto{\pgfqpoint{2.556062in}{1.234259in}}{\pgfqpoint{2.560452in}{1.244858in}}{\pgfqpoint{2.560452in}{1.255908in}}%
\pgfpathcurveto{\pgfqpoint{2.560452in}{1.266959in}}{\pgfqpoint{2.556062in}{1.277558in}}{\pgfqpoint{2.548249in}{1.285371in}}%
\pgfpathcurveto{\pgfqpoint{2.540435in}{1.293185in}}{\pgfqpoint{2.529836in}{1.297575in}}{\pgfqpoint{2.518786in}{1.297575in}}%
\pgfpathcurveto{\pgfqpoint{2.507736in}{1.297575in}}{\pgfqpoint{2.497137in}{1.293185in}}{\pgfqpoint{2.489323in}{1.285371in}}%
\pgfpathcurveto{\pgfqpoint{2.481509in}{1.277558in}}{\pgfqpoint{2.477119in}{1.266959in}}{\pgfqpoint{2.477119in}{1.255908in}}%
\pgfpathcurveto{\pgfqpoint{2.477119in}{1.244858in}}{\pgfqpoint{2.481509in}{1.234259in}}{\pgfqpoint{2.489323in}{1.226446in}}%
\pgfpathcurveto{\pgfqpoint{2.497137in}{1.218632in}}{\pgfqpoint{2.507736in}{1.214242in}}{\pgfqpoint{2.518786in}{1.214242in}}%
\pgfpathclose%
\pgfusepath{stroke,fill}%
\end{pgfscope}%
\begin{pgfscope}%
\pgfpathrectangle{\pgfqpoint{0.800000in}{0.528000in}}{\pgfqpoint{4.960000in}{3.696000in}}%
\pgfusepath{clip}%
\pgfsetbuttcap%
\pgfsetroundjoin%
\definecolor{currentfill}{rgb}{0.000000,0.000000,0.000000}%
\pgfsetfillcolor{currentfill}%
\pgfsetlinewidth{1.003750pt}%
\definecolor{currentstroke}{rgb}{0.000000,0.000000,0.000000}%
\pgfsetstrokecolor{currentstroke}%
\pgfsetdash{}{0pt}%
\pgfpathmoveto{\pgfqpoint{2.518786in}{1.192744in}}%
\pgfpathcurveto{\pgfqpoint{2.529836in}{1.192744in}}{\pgfqpoint{2.540435in}{1.197135in}}{\pgfqpoint{2.548249in}{1.204948in}}%
\pgfpathcurveto{\pgfqpoint{2.556062in}{1.212762in}}{\pgfqpoint{2.560452in}{1.223361in}}{\pgfqpoint{2.560452in}{1.234411in}}%
\pgfpathcurveto{\pgfqpoint{2.560452in}{1.245461in}}{\pgfqpoint{2.556062in}{1.256060in}}{\pgfqpoint{2.548249in}{1.263874in}}%
\pgfpathcurveto{\pgfqpoint{2.540435in}{1.271687in}}{\pgfqpoint{2.529836in}{1.276078in}}{\pgfqpoint{2.518786in}{1.276078in}}%
\pgfpathcurveto{\pgfqpoint{2.507736in}{1.276078in}}{\pgfqpoint{2.497137in}{1.271687in}}{\pgfqpoint{2.489323in}{1.263874in}}%
\pgfpathcurveto{\pgfqpoint{2.481509in}{1.256060in}}{\pgfqpoint{2.477119in}{1.245461in}}{\pgfqpoint{2.477119in}{1.234411in}}%
\pgfpathcurveto{\pgfqpoint{2.477119in}{1.223361in}}{\pgfqpoint{2.481509in}{1.212762in}}{\pgfqpoint{2.489323in}{1.204948in}}%
\pgfpathcurveto{\pgfqpoint{2.497137in}{1.197135in}}{\pgfqpoint{2.507736in}{1.192744in}}{\pgfqpoint{2.518786in}{1.192744in}}%
\pgfpathclose%
\pgfusepath{stroke,fill}%
\end{pgfscope}%
\begin{pgfscope}%
\pgfpathrectangle{\pgfqpoint{0.800000in}{0.528000in}}{\pgfqpoint{4.960000in}{3.696000in}}%
\pgfusepath{clip}%
\pgfsetbuttcap%
\pgfsetroundjoin%
\definecolor{currentfill}{rgb}{0.000000,0.000000,0.000000}%
\pgfsetfillcolor{currentfill}%
\pgfsetlinewidth{1.003750pt}%
\definecolor{currentstroke}{rgb}{0.000000,0.000000,0.000000}%
\pgfsetstrokecolor{currentstroke}%
\pgfsetdash{}{0pt}%
\pgfpathmoveto{\pgfqpoint{2.518786in}{1.192744in}}%
\pgfpathcurveto{\pgfqpoint{2.529836in}{1.192744in}}{\pgfqpoint{2.540435in}{1.197135in}}{\pgfqpoint{2.548249in}{1.204948in}}%
\pgfpathcurveto{\pgfqpoint{2.556062in}{1.212762in}}{\pgfqpoint{2.560452in}{1.223361in}}{\pgfqpoint{2.560452in}{1.234411in}}%
\pgfpathcurveto{\pgfqpoint{2.560452in}{1.245461in}}{\pgfqpoint{2.556062in}{1.256060in}}{\pgfqpoint{2.548249in}{1.263874in}}%
\pgfpathcurveto{\pgfqpoint{2.540435in}{1.271687in}}{\pgfqpoint{2.529836in}{1.276078in}}{\pgfqpoint{2.518786in}{1.276078in}}%
\pgfpathcurveto{\pgfqpoint{2.507736in}{1.276078in}}{\pgfqpoint{2.497137in}{1.271687in}}{\pgfqpoint{2.489323in}{1.263874in}}%
\pgfpathcurveto{\pgfqpoint{2.481509in}{1.256060in}}{\pgfqpoint{2.477119in}{1.245461in}}{\pgfqpoint{2.477119in}{1.234411in}}%
\pgfpathcurveto{\pgfqpoint{2.477119in}{1.223361in}}{\pgfqpoint{2.481509in}{1.212762in}}{\pgfqpoint{2.489323in}{1.204948in}}%
\pgfpathcurveto{\pgfqpoint{2.497137in}{1.197135in}}{\pgfqpoint{2.507736in}{1.192744in}}{\pgfqpoint{2.518786in}{1.192744in}}%
\pgfpathclose%
\pgfusepath{stroke,fill}%
\end{pgfscope}%
\begin{pgfscope}%
\pgfpathrectangle{\pgfqpoint{0.800000in}{0.528000in}}{\pgfqpoint{4.960000in}{3.696000in}}%
\pgfusepath{clip}%
\pgfsetbuttcap%
\pgfsetroundjoin%
\definecolor{currentfill}{rgb}{0.000000,0.000000,0.000000}%
\pgfsetfillcolor{currentfill}%
\pgfsetlinewidth{1.003750pt}%
\definecolor{currentstroke}{rgb}{0.000000,0.000000,0.000000}%
\pgfsetstrokecolor{currentstroke}%
\pgfsetdash{}{0pt}%
\pgfpathmoveto{\pgfqpoint{2.518786in}{1.085257in}}%
\pgfpathcurveto{\pgfqpoint{2.529836in}{1.085257in}}{\pgfqpoint{2.540435in}{1.089647in}}{\pgfqpoint{2.548249in}{1.097461in}}%
\pgfpathcurveto{\pgfqpoint{2.556062in}{1.105274in}}{\pgfqpoint{2.560452in}{1.115873in}}{\pgfqpoint{2.560452in}{1.126923in}}%
\pgfpathcurveto{\pgfqpoint{2.560452in}{1.137974in}}{\pgfqpoint{2.556062in}{1.148573in}}{\pgfqpoint{2.548249in}{1.156386in}}%
\pgfpathcurveto{\pgfqpoint{2.540435in}{1.164200in}}{\pgfqpoint{2.529836in}{1.168590in}}{\pgfqpoint{2.518786in}{1.168590in}}%
\pgfpathcurveto{\pgfqpoint{2.507736in}{1.168590in}}{\pgfqpoint{2.497137in}{1.164200in}}{\pgfqpoint{2.489323in}{1.156386in}}%
\pgfpathcurveto{\pgfqpoint{2.481509in}{1.148573in}}{\pgfqpoint{2.477119in}{1.137974in}}{\pgfqpoint{2.477119in}{1.126923in}}%
\pgfpathcurveto{\pgfqpoint{2.477119in}{1.115873in}}{\pgfqpoint{2.481509in}{1.105274in}}{\pgfqpoint{2.489323in}{1.097461in}}%
\pgfpathcurveto{\pgfqpoint{2.497137in}{1.089647in}}{\pgfqpoint{2.507736in}{1.085257in}}{\pgfqpoint{2.518786in}{1.085257in}}%
\pgfpathclose%
\pgfusepath{stroke,fill}%
\end{pgfscope}%
\begin{pgfscope}%
\pgfpathrectangle{\pgfqpoint{0.800000in}{0.528000in}}{\pgfqpoint{4.960000in}{3.696000in}}%
\pgfusepath{clip}%
\pgfsetbuttcap%
\pgfsetroundjoin%
\definecolor{currentfill}{rgb}{0.000000,0.000000,0.000000}%
\pgfsetfillcolor{currentfill}%
\pgfsetlinewidth{1.003750pt}%
\definecolor{currentstroke}{rgb}{0.000000,0.000000,0.000000}%
\pgfsetstrokecolor{currentstroke}%
\pgfsetdash{}{0pt}%
\pgfpathmoveto{\pgfqpoint{2.518786in}{1.149749in}}%
\pgfpathcurveto{\pgfqpoint{2.529836in}{1.149749in}}{\pgfqpoint{2.540435in}{1.154140in}}{\pgfqpoint{2.548249in}{1.161953in}}%
\pgfpathcurveto{\pgfqpoint{2.556062in}{1.169767in}}{\pgfqpoint{2.560452in}{1.180366in}}{\pgfqpoint{2.560452in}{1.191416in}}%
\pgfpathcurveto{\pgfqpoint{2.560452in}{1.202466in}}{\pgfqpoint{2.556062in}{1.213065in}}{\pgfqpoint{2.548249in}{1.220879in}}%
\pgfpathcurveto{\pgfqpoint{2.540435in}{1.228692in}}{\pgfqpoint{2.529836in}{1.233083in}}{\pgfqpoint{2.518786in}{1.233083in}}%
\pgfpathcurveto{\pgfqpoint{2.507736in}{1.233083in}}{\pgfqpoint{2.497137in}{1.228692in}}{\pgfqpoint{2.489323in}{1.220879in}}%
\pgfpathcurveto{\pgfqpoint{2.481509in}{1.213065in}}{\pgfqpoint{2.477119in}{1.202466in}}{\pgfqpoint{2.477119in}{1.191416in}}%
\pgfpathcurveto{\pgfqpoint{2.477119in}{1.180366in}}{\pgfqpoint{2.481509in}{1.169767in}}{\pgfqpoint{2.489323in}{1.161953in}}%
\pgfpathcurveto{\pgfqpoint{2.497137in}{1.154140in}}{\pgfqpoint{2.507736in}{1.149749in}}{\pgfqpoint{2.518786in}{1.149749in}}%
\pgfpathclose%
\pgfusepath{stroke,fill}%
\end{pgfscope}%
\begin{pgfscope}%
\pgfpathrectangle{\pgfqpoint{0.800000in}{0.528000in}}{\pgfqpoint{4.960000in}{3.696000in}}%
\pgfusepath{clip}%
\pgfsetbuttcap%
\pgfsetroundjoin%
\definecolor{currentfill}{rgb}{0.000000,0.000000,0.000000}%
\pgfsetfillcolor{currentfill}%
\pgfsetlinewidth{1.003750pt}%
\definecolor{currentstroke}{rgb}{0.000000,0.000000,0.000000}%
\pgfsetstrokecolor{currentstroke}%
\pgfsetdash{}{0pt}%
\pgfpathmoveto{\pgfqpoint{2.518786in}{1.106754in}}%
\pgfpathcurveto{\pgfqpoint{2.529836in}{1.106754in}}{\pgfqpoint{2.540435in}{1.111145in}}{\pgfqpoint{2.548249in}{1.118958in}}%
\pgfpathcurveto{\pgfqpoint{2.556062in}{1.126772in}}{\pgfqpoint{2.560452in}{1.137371in}}{\pgfqpoint{2.560452in}{1.148421in}}%
\pgfpathcurveto{\pgfqpoint{2.560452in}{1.159471in}}{\pgfqpoint{2.556062in}{1.170070in}}{\pgfqpoint{2.548249in}{1.177884in}}%
\pgfpathcurveto{\pgfqpoint{2.540435in}{1.185697in}}{\pgfqpoint{2.529836in}{1.190088in}}{\pgfqpoint{2.518786in}{1.190088in}}%
\pgfpathcurveto{\pgfqpoint{2.507736in}{1.190088in}}{\pgfqpoint{2.497137in}{1.185697in}}{\pgfqpoint{2.489323in}{1.177884in}}%
\pgfpathcurveto{\pgfqpoint{2.481509in}{1.170070in}}{\pgfqpoint{2.477119in}{1.159471in}}{\pgfqpoint{2.477119in}{1.148421in}}%
\pgfpathcurveto{\pgfqpoint{2.477119in}{1.137371in}}{\pgfqpoint{2.481509in}{1.126772in}}{\pgfqpoint{2.489323in}{1.118958in}}%
\pgfpathcurveto{\pgfqpoint{2.497137in}{1.111145in}}{\pgfqpoint{2.507736in}{1.106754in}}{\pgfqpoint{2.518786in}{1.106754in}}%
\pgfpathclose%
\pgfusepath{stroke,fill}%
\end{pgfscope}%
\begin{pgfscope}%
\pgfpathrectangle{\pgfqpoint{0.800000in}{0.528000in}}{\pgfqpoint{4.960000in}{3.696000in}}%
\pgfusepath{clip}%
\pgfsetbuttcap%
\pgfsetroundjoin%
\definecolor{currentfill}{rgb}{0.000000,0.000000,0.000000}%
\pgfsetfillcolor{currentfill}%
\pgfsetlinewidth{1.003750pt}%
\definecolor{currentstroke}{rgb}{0.000000,0.000000,0.000000}%
\pgfsetstrokecolor{currentstroke}%
\pgfsetdash{}{0pt}%
\pgfpathmoveto{\pgfqpoint{2.518786in}{1.128252in}}%
\pgfpathcurveto{\pgfqpoint{2.529836in}{1.128252in}}{\pgfqpoint{2.540435in}{1.132642in}}{\pgfqpoint{2.548249in}{1.140456in}}%
\pgfpathcurveto{\pgfqpoint{2.556062in}{1.148269in}}{\pgfqpoint{2.560452in}{1.158868in}}{\pgfqpoint{2.560452in}{1.169918in}}%
\pgfpathcurveto{\pgfqpoint{2.560452in}{1.180969in}}{\pgfqpoint{2.556062in}{1.191568in}}{\pgfqpoint{2.548249in}{1.199381in}}%
\pgfpathcurveto{\pgfqpoint{2.540435in}{1.207195in}}{\pgfqpoint{2.529836in}{1.211585in}}{\pgfqpoint{2.518786in}{1.211585in}}%
\pgfpathcurveto{\pgfqpoint{2.507736in}{1.211585in}}{\pgfqpoint{2.497137in}{1.207195in}}{\pgfqpoint{2.489323in}{1.199381in}}%
\pgfpathcurveto{\pgfqpoint{2.481509in}{1.191568in}}{\pgfqpoint{2.477119in}{1.180969in}}{\pgfqpoint{2.477119in}{1.169918in}}%
\pgfpathcurveto{\pgfqpoint{2.477119in}{1.158868in}}{\pgfqpoint{2.481509in}{1.148269in}}{\pgfqpoint{2.489323in}{1.140456in}}%
\pgfpathcurveto{\pgfqpoint{2.497137in}{1.132642in}}{\pgfqpoint{2.507736in}{1.128252in}}{\pgfqpoint{2.518786in}{1.128252in}}%
\pgfpathclose%
\pgfusepath{stroke,fill}%
\end{pgfscope}%
\begin{pgfscope}%
\pgfpathrectangle{\pgfqpoint{0.800000in}{0.528000in}}{\pgfqpoint{4.960000in}{3.696000in}}%
\pgfusepath{clip}%
\pgfsetbuttcap%
\pgfsetroundjoin%
\definecolor{currentfill}{rgb}{0.000000,0.000000,0.000000}%
\pgfsetfillcolor{currentfill}%
\pgfsetlinewidth{1.003750pt}%
\definecolor{currentstroke}{rgb}{0.000000,0.000000,0.000000}%
\pgfsetstrokecolor{currentstroke}%
\pgfsetdash{}{0pt}%
\pgfpathmoveto{\pgfqpoint{2.518786in}{1.171247in}}%
\pgfpathcurveto{\pgfqpoint{2.529836in}{1.171247in}}{\pgfqpoint{2.540435in}{1.175637in}}{\pgfqpoint{2.548249in}{1.183451in}}%
\pgfpathcurveto{\pgfqpoint{2.556062in}{1.191264in}}{\pgfqpoint{2.560452in}{1.201863in}}{\pgfqpoint{2.560452in}{1.212913in}}%
\pgfpathcurveto{\pgfqpoint{2.560452in}{1.223964in}}{\pgfqpoint{2.556062in}{1.234563in}}{\pgfqpoint{2.548249in}{1.242376in}}%
\pgfpathcurveto{\pgfqpoint{2.540435in}{1.250190in}}{\pgfqpoint{2.529836in}{1.254580in}}{\pgfqpoint{2.518786in}{1.254580in}}%
\pgfpathcurveto{\pgfqpoint{2.507736in}{1.254580in}}{\pgfqpoint{2.497137in}{1.250190in}}{\pgfqpoint{2.489323in}{1.242376in}}%
\pgfpathcurveto{\pgfqpoint{2.481509in}{1.234563in}}{\pgfqpoint{2.477119in}{1.223964in}}{\pgfqpoint{2.477119in}{1.212913in}}%
\pgfpathcurveto{\pgfqpoint{2.477119in}{1.201863in}}{\pgfqpoint{2.481509in}{1.191264in}}{\pgfqpoint{2.489323in}{1.183451in}}%
\pgfpathcurveto{\pgfqpoint{2.497137in}{1.175637in}}{\pgfqpoint{2.507736in}{1.171247in}}{\pgfqpoint{2.518786in}{1.171247in}}%
\pgfpathclose%
\pgfusepath{stroke,fill}%
\end{pgfscope}%
\begin{pgfscope}%
\pgfpathrectangle{\pgfqpoint{0.800000in}{0.528000in}}{\pgfqpoint{4.960000in}{3.696000in}}%
\pgfusepath{clip}%
\pgfsetbuttcap%
\pgfsetroundjoin%
\definecolor{currentfill}{rgb}{0.000000,0.000000,0.000000}%
\pgfsetfillcolor{currentfill}%
\pgfsetlinewidth{1.003750pt}%
\definecolor{currentstroke}{rgb}{0.000000,0.000000,0.000000}%
\pgfsetstrokecolor{currentstroke}%
\pgfsetdash{}{0pt}%
\pgfpathmoveto{\pgfqpoint{2.518786in}{1.128252in}}%
\pgfpathcurveto{\pgfqpoint{2.529836in}{1.128252in}}{\pgfqpoint{2.540435in}{1.132642in}}{\pgfqpoint{2.548249in}{1.140456in}}%
\pgfpathcurveto{\pgfqpoint{2.556062in}{1.148269in}}{\pgfqpoint{2.560452in}{1.158868in}}{\pgfqpoint{2.560452in}{1.169918in}}%
\pgfpathcurveto{\pgfqpoint{2.560452in}{1.180969in}}{\pgfqpoint{2.556062in}{1.191568in}}{\pgfqpoint{2.548249in}{1.199381in}}%
\pgfpathcurveto{\pgfqpoint{2.540435in}{1.207195in}}{\pgfqpoint{2.529836in}{1.211585in}}{\pgfqpoint{2.518786in}{1.211585in}}%
\pgfpathcurveto{\pgfqpoint{2.507736in}{1.211585in}}{\pgfqpoint{2.497137in}{1.207195in}}{\pgfqpoint{2.489323in}{1.199381in}}%
\pgfpathcurveto{\pgfqpoint{2.481509in}{1.191568in}}{\pgfqpoint{2.477119in}{1.180969in}}{\pgfqpoint{2.477119in}{1.169918in}}%
\pgfpathcurveto{\pgfqpoint{2.477119in}{1.158868in}}{\pgfqpoint{2.481509in}{1.148269in}}{\pgfqpoint{2.489323in}{1.140456in}}%
\pgfpathcurveto{\pgfqpoint{2.497137in}{1.132642in}}{\pgfqpoint{2.507736in}{1.128252in}}{\pgfqpoint{2.518786in}{1.128252in}}%
\pgfpathclose%
\pgfusepath{stroke,fill}%
\end{pgfscope}%
\begin{pgfscope}%
\pgfpathrectangle{\pgfqpoint{0.800000in}{0.528000in}}{\pgfqpoint{4.960000in}{3.696000in}}%
\pgfusepath{clip}%
\pgfsetbuttcap%
\pgfsetroundjoin%
\definecolor{currentfill}{rgb}{0.000000,0.000000,0.000000}%
\pgfsetfillcolor{currentfill}%
\pgfsetlinewidth{1.003750pt}%
\definecolor{currentstroke}{rgb}{0.000000,0.000000,0.000000}%
\pgfsetstrokecolor{currentstroke}%
\pgfsetdash{}{0pt}%
\pgfpathmoveto{\pgfqpoint{2.518786in}{1.192744in}}%
\pgfpathcurveto{\pgfqpoint{2.529836in}{1.192744in}}{\pgfqpoint{2.540435in}{1.197135in}}{\pgfqpoint{2.548249in}{1.204948in}}%
\pgfpathcurveto{\pgfqpoint{2.556062in}{1.212762in}}{\pgfqpoint{2.560452in}{1.223361in}}{\pgfqpoint{2.560452in}{1.234411in}}%
\pgfpathcurveto{\pgfqpoint{2.560452in}{1.245461in}}{\pgfqpoint{2.556062in}{1.256060in}}{\pgfqpoint{2.548249in}{1.263874in}}%
\pgfpathcurveto{\pgfqpoint{2.540435in}{1.271687in}}{\pgfqpoint{2.529836in}{1.276078in}}{\pgfqpoint{2.518786in}{1.276078in}}%
\pgfpathcurveto{\pgfqpoint{2.507736in}{1.276078in}}{\pgfqpoint{2.497137in}{1.271687in}}{\pgfqpoint{2.489323in}{1.263874in}}%
\pgfpathcurveto{\pgfqpoint{2.481509in}{1.256060in}}{\pgfqpoint{2.477119in}{1.245461in}}{\pgfqpoint{2.477119in}{1.234411in}}%
\pgfpathcurveto{\pgfqpoint{2.477119in}{1.223361in}}{\pgfqpoint{2.481509in}{1.212762in}}{\pgfqpoint{2.489323in}{1.204948in}}%
\pgfpathcurveto{\pgfqpoint{2.497137in}{1.197135in}}{\pgfqpoint{2.507736in}{1.192744in}}{\pgfqpoint{2.518786in}{1.192744in}}%
\pgfpathclose%
\pgfusepath{stroke,fill}%
\end{pgfscope}%
\begin{pgfscope}%
\pgfpathrectangle{\pgfqpoint{0.800000in}{0.528000in}}{\pgfqpoint{4.960000in}{3.696000in}}%
\pgfusepath{clip}%
\pgfsetbuttcap%
\pgfsetroundjoin%
\definecolor{currentfill}{rgb}{0.000000,0.000000,0.000000}%
\pgfsetfillcolor{currentfill}%
\pgfsetlinewidth{1.003750pt}%
\definecolor{currentstroke}{rgb}{0.000000,0.000000,0.000000}%
\pgfsetstrokecolor{currentstroke}%
\pgfsetdash{}{0pt}%
\pgfpathmoveto{\pgfqpoint{2.518786in}{1.106754in}}%
\pgfpathcurveto{\pgfqpoint{2.529836in}{1.106754in}}{\pgfqpoint{2.540435in}{1.111145in}}{\pgfqpoint{2.548249in}{1.118958in}}%
\pgfpathcurveto{\pgfqpoint{2.556062in}{1.126772in}}{\pgfqpoint{2.560452in}{1.137371in}}{\pgfqpoint{2.560452in}{1.148421in}}%
\pgfpathcurveto{\pgfqpoint{2.560452in}{1.159471in}}{\pgfqpoint{2.556062in}{1.170070in}}{\pgfqpoint{2.548249in}{1.177884in}}%
\pgfpathcurveto{\pgfqpoint{2.540435in}{1.185697in}}{\pgfqpoint{2.529836in}{1.190088in}}{\pgfqpoint{2.518786in}{1.190088in}}%
\pgfpathcurveto{\pgfqpoint{2.507736in}{1.190088in}}{\pgfqpoint{2.497137in}{1.185697in}}{\pgfqpoint{2.489323in}{1.177884in}}%
\pgfpathcurveto{\pgfqpoint{2.481509in}{1.170070in}}{\pgfqpoint{2.477119in}{1.159471in}}{\pgfqpoint{2.477119in}{1.148421in}}%
\pgfpathcurveto{\pgfqpoint{2.477119in}{1.137371in}}{\pgfqpoint{2.481509in}{1.126772in}}{\pgfqpoint{2.489323in}{1.118958in}}%
\pgfpathcurveto{\pgfqpoint{2.497137in}{1.111145in}}{\pgfqpoint{2.507736in}{1.106754in}}{\pgfqpoint{2.518786in}{1.106754in}}%
\pgfpathclose%
\pgfusepath{stroke,fill}%
\end{pgfscope}%
\begin{pgfscope}%
\pgfpathrectangle{\pgfqpoint{0.800000in}{0.528000in}}{\pgfqpoint{4.960000in}{3.696000in}}%
\pgfusepath{clip}%
\pgfsetbuttcap%
\pgfsetroundjoin%
\definecolor{currentfill}{rgb}{0.000000,0.000000,0.000000}%
\pgfsetfillcolor{currentfill}%
\pgfsetlinewidth{1.003750pt}%
\definecolor{currentstroke}{rgb}{0.000000,0.000000,0.000000}%
\pgfsetstrokecolor{currentstroke}%
\pgfsetdash{}{0pt}%
\pgfpathmoveto{\pgfqpoint{2.518786in}{1.278734in}}%
\pgfpathcurveto{\pgfqpoint{2.529836in}{1.278734in}}{\pgfqpoint{2.540435in}{1.283125in}}{\pgfqpoint{2.548249in}{1.290938in}}%
\pgfpathcurveto{\pgfqpoint{2.556062in}{1.298752in}}{\pgfqpoint{2.560452in}{1.309351in}}{\pgfqpoint{2.560452in}{1.320401in}}%
\pgfpathcurveto{\pgfqpoint{2.560452in}{1.331451in}}{\pgfqpoint{2.556062in}{1.342050in}}{\pgfqpoint{2.548249in}{1.349864in}}%
\pgfpathcurveto{\pgfqpoint{2.540435in}{1.357677in}}{\pgfqpoint{2.529836in}{1.362068in}}{\pgfqpoint{2.518786in}{1.362068in}}%
\pgfpathcurveto{\pgfqpoint{2.507736in}{1.362068in}}{\pgfqpoint{2.497137in}{1.357677in}}{\pgfqpoint{2.489323in}{1.349864in}}%
\pgfpathcurveto{\pgfqpoint{2.481509in}{1.342050in}}{\pgfqpoint{2.477119in}{1.331451in}}{\pgfqpoint{2.477119in}{1.320401in}}%
\pgfpathcurveto{\pgfqpoint{2.477119in}{1.309351in}}{\pgfqpoint{2.481509in}{1.298752in}}{\pgfqpoint{2.489323in}{1.290938in}}%
\pgfpathcurveto{\pgfqpoint{2.497137in}{1.283125in}}{\pgfqpoint{2.507736in}{1.278734in}}{\pgfqpoint{2.518786in}{1.278734in}}%
\pgfpathclose%
\pgfusepath{stroke,fill}%
\end{pgfscope}%
\begin{pgfscope}%
\pgfpathrectangle{\pgfqpoint{0.800000in}{0.528000in}}{\pgfqpoint{4.960000in}{3.696000in}}%
\pgfusepath{clip}%
\pgfsetbuttcap%
\pgfsetroundjoin%
\definecolor{currentfill}{rgb}{0.000000,0.000000,0.000000}%
\pgfsetfillcolor{currentfill}%
\pgfsetlinewidth{1.003750pt}%
\definecolor{currentstroke}{rgb}{0.000000,0.000000,0.000000}%
\pgfsetstrokecolor{currentstroke}%
\pgfsetdash{}{0pt}%
\pgfpathmoveto{\pgfqpoint{2.518786in}{1.106754in}}%
\pgfpathcurveto{\pgfqpoint{2.529836in}{1.106754in}}{\pgfqpoint{2.540435in}{1.111145in}}{\pgfqpoint{2.548249in}{1.118958in}}%
\pgfpathcurveto{\pgfqpoint{2.556062in}{1.126772in}}{\pgfqpoint{2.560452in}{1.137371in}}{\pgfqpoint{2.560452in}{1.148421in}}%
\pgfpathcurveto{\pgfqpoint{2.560452in}{1.159471in}}{\pgfqpoint{2.556062in}{1.170070in}}{\pgfqpoint{2.548249in}{1.177884in}}%
\pgfpathcurveto{\pgfqpoint{2.540435in}{1.185697in}}{\pgfqpoint{2.529836in}{1.190088in}}{\pgfqpoint{2.518786in}{1.190088in}}%
\pgfpathcurveto{\pgfqpoint{2.507736in}{1.190088in}}{\pgfqpoint{2.497137in}{1.185697in}}{\pgfqpoint{2.489323in}{1.177884in}}%
\pgfpathcurveto{\pgfqpoint{2.481509in}{1.170070in}}{\pgfqpoint{2.477119in}{1.159471in}}{\pgfqpoint{2.477119in}{1.148421in}}%
\pgfpathcurveto{\pgfqpoint{2.477119in}{1.137371in}}{\pgfqpoint{2.481509in}{1.126772in}}{\pgfqpoint{2.489323in}{1.118958in}}%
\pgfpathcurveto{\pgfqpoint{2.497137in}{1.111145in}}{\pgfqpoint{2.507736in}{1.106754in}}{\pgfqpoint{2.518786in}{1.106754in}}%
\pgfpathclose%
\pgfusepath{stroke,fill}%
\end{pgfscope}%
\begin{pgfscope}%
\pgfpathrectangle{\pgfqpoint{0.800000in}{0.528000in}}{\pgfqpoint{4.960000in}{3.696000in}}%
\pgfusepath{clip}%
\pgfsetbuttcap%
\pgfsetroundjoin%
\definecolor{currentfill}{rgb}{0.000000,0.000000,0.000000}%
\pgfsetfillcolor{currentfill}%
\pgfsetlinewidth{1.003750pt}%
\definecolor{currentstroke}{rgb}{0.000000,0.000000,0.000000}%
\pgfsetstrokecolor{currentstroke}%
\pgfsetdash{}{0pt}%
\pgfpathmoveto{\pgfqpoint{2.518786in}{1.171247in}}%
\pgfpathcurveto{\pgfqpoint{2.529836in}{1.171247in}}{\pgfqpoint{2.540435in}{1.175637in}}{\pgfqpoint{2.548249in}{1.183451in}}%
\pgfpathcurveto{\pgfqpoint{2.556062in}{1.191264in}}{\pgfqpoint{2.560452in}{1.201863in}}{\pgfqpoint{2.560452in}{1.212913in}}%
\pgfpathcurveto{\pgfqpoint{2.560452in}{1.223964in}}{\pgfqpoint{2.556062in}{1.234563in}}{\pgfqpoint{2.548249in}{1.242376in}}%
\pgfpathcurveto{\pgfqpoint{2.540435in}{1.250190in}}{\pgfqpoint{2.529836in}{1.254580in}}{\pgfqpoint{2.518786in}{1.254580in}}%
\pgfpathcurveto{\pgfqpoint{2.507736in}{1.254580in}}{\pgfqpoint{2.497137in}{1.250190in}}{\pgfqpoint{2.489323in}{1.242376in}}%
\pgfpathcurveto{\pgfqpoint{2.481509in}{1.234563in}}{\pgfqpoint{2.477119in}{1.223964in}}{\pgfqpoint{2.477119in}{1.212913in}}%
\pgfpathcurveto{\pgfqpoint{2.477119in}{1.201863in}}{\pgfqpoint{2.481509in}{1.191264in}}{\pgfqpoint{2.489323in}{1.183451in}}%
\pgfpathcurveto{\pgfqpoint{2.497137in}{1.175637in}}{\pgfqpoint{2.507736in}{1.171247in}}{\pgfqpoint{2.518786in}{1.171247in}}%
\pgfpathclose%
\pgfusepath{stroke,fill}%
\end{pgfscope}%
\begin{pgfscope}%
\pgfpathrectangle{\pgfqpoint{0.800000in}{0.528000in}}{\pgfqpoint{4.960000in}{3.696000in}}%
\pgfusepath{clip}%
\pgfsetbuttcap%
\pgfsetroundjoin%
\definecolor{currentfill}{rgb}{0.000000,0.000000,0.000000}%
\pgfsetfillcolor{currentfill}%
\pgfsetlinewidth{1.003750pt}%
\definecolor{currentstroke}{rgb}{0.000000,0.000000,0.000000}%
\pgfsetstrokecolor{currentstroke}%
\pgfsetdash{}{0pt}%
\pgfpathmoveto{\pgfqpoint{2.518786in}{1.149749in}}%
\pgfpathcurveto{\pgfqpoint{2.529836in}{1.149749in}}{\pgfqpoint{2.540435in}{1.154140in}}{\pgfqpoint{2.548249in}{1.161953in}}%
\pgfpathcurveto{\pgfqpoint{2.556062in}{1.169767in}}{\pgfqpoint{2.560452in}{1.180366in}}{\pgfqpoint{2.560452in}{1.191416in}}%
\pgfpathcurveto{\pgfqpoint{2.560452in}{1.202466in}}{\pgfqpoint{2.556062in}{1.213065in}}{\pgfqpoint{2.548249in}{1.220879in}}%
\pgfpathcurveto{\pgfqpoint{2.540435in}{1.228692in}}{\pgfqpoint{2.529836in}{1.233083in}}{\pgfqpoint{2.518786in}{1.233083in}}%
\pgfpathcurveto{\pgfqpoint{2.507736in}{1.233083in}}{\pgfqpoint{2.497137in}{1.228692in}}{\pgfqpoint{2.489323in}{1.220879in}}%
\pgfpathcurveto{\pgfqpoint{2.481509in}{1.213065in}}{\pgfqpoint{2.477119in}{1.202466in}}{\pgfqpoint{2.477119in}{1.191416in}}%
\pgfpathcurveto{\pgfqpoint{2.477119in}{1.180366in}}{\pgfqpoint{2.481509in}{1.169767in}}{\pgfqpoint{2.489323in}{1.161953in}}%
\pgfpathcurveto{\pgfqpoint{2.497137in}{1.154140in}}{\pgfqpoint{2.507736in}{1.149749in}}{\pgfqpoint{2.518786in}{1.149749in}}%
\pgfpathclose%
\pgfusepath{stroke,fill}%
\end{pgfscope}%
\begin{pgfscope}%
\pgfpathrectangle{\pgfqpoint{0.800000in}{0.528000in}}{\pgfqpoint{4.960000in}{3.696000in}}%
\pgfusepath{clip}%
\pgfsetbuttcap%
\pgfsetroundjoin%
\definecolor{currentfill}{rgb}{0.000000,0.000000,0.000000}%
\pgfsetfillcolor{currentfill}%
\pgfsetlinewidth{1.003750pt}%
\definecolor{currentstroke}{rgb}{0.000000,0.000000,0.000000}%
\pgfsetstrokecolor{currentstroke}%
\pgfsetdash{}{0pt}%
\pgfpathmoveto{\pgfqpoint{2.518786in}{1.257237in}}%
\pgfpathcurveto{\pgfqpoint{2.529836in}{1.257237in}}{\pgfqpoint{2.540435in}{1.261627in}}{\pgfqpoint{2.548249in}{1.269441in}}%
\pgfpathcurveto{\pgfqpoint{2.556062in}{1.277254in}}{\pgfqpoint{2.560452in}{1.287853in}}{\pgfqpoint{2.560452in}{1.298903in}}%
\pgfpathcurveto{\pgfqpoint{2.560452in}{1.309954in}}{\pgfqpoint{2.556062in}{1.320553in}}{\pgfqpoint{2.548249in}{1.328366in}}%
\pgfpathcurveto{\pgfqpoint{2.540435in}{1.336180in}}{\pgfqpoint{2.529836in}{1.340570in}}{\pgfqpoint{2.518786in}{1.340570in}}%
\pgfpathcurveto{\pgfqpoint{2.507736in}{1.340570in}}{\pgfqpoint{2.497137in}{1.336180in}}{\pgfqpoint{2.489323in}{1.328366in}}%
\pgfpathcurveto{\pgfqpoint{2.481509in}{1.320553in}}{\pgfqpoint{2.477119in}{1.309954in}}{\pgfqpoint{2.477119in}{1.298903in}}%
\pgfpathcurveto{\pgfqpoint{2.477119in}{1.287853in}}{\pgfqpoint{2.481509in}{1.277254in}}{\pgfqpoint{2.489323in}{1.269441in}}%
\pgfpathcurveto{\pgfqpoint{2.497137in}{1.261627in}}{\pgfqpoint{2.507736in}{1.257237in}}{\pgfqpoint{2.518786in}{1.257237in}}%
\pgfpathclose%
\pgfusepath{stroke,fill}%
\end{pgfscope}%
\begin{pgfscope}%
\pgfpathrectangle{\pgfqpoint{0.800000in}{0.528000in}}{\pgfqpoint{4.960000in}{3.696000in}}%
\pgfusepath{clip}%
\pgfsetbuttcap%
\pgfsetroundjoin%
\definecolor{currentfill}{rgb}{0.000000,0.000000,0.000000}%
\pgfsetfillcolor{currentfill}%
\pgfsetlinewidth{1.003750pt}%
\definecolor{currentstroke}{rgb}{0.000000,0.000000,0.000000}%
\pgfsetstrokecolor{currentstroke}%
\pgfsetdash{}{0pt}%
\pgfpathmoveto{\pgfqpoint{2.518786in}{1.214242in}}%
\pgfpathcurveto{\pgfqpoint{2.529836in}{1.214242in}}{\pgfqpoint{2.540435in}{1.218632in}}{\pgfqpoint{2.548249in}{1.226446in}}%
\pgfpathcurveto{\pgfqpoint{2.556062in}{1.234259in}}{\pgfqpoint{2.560452in}{1.244858in}}{\pgfqpoint{2.560452in}{1.255908in}}%
\pgfpathcurveto{\pgfqpoint{2.560452in}{1.266959in}}{\pgfqpoint{2.556062in}{1.277558in}}{\pgfqpoint{2.548249in}{1.285371in}}%
\pgfpathcurveto{\pgfqpoint{2.540435in}{1.293185in}}{\pgfqpoint{2.529836in}{1.297575in}}{\pgfqpoint{2.518786in}{1.297575in}}%
\pgfpathcurveto{\pgfqpoint{2.507736in}{1.297575in}}{\pgfqpoint{2.497137in}{1.293185in}}{\pgfqpoint{2.489323in}{1.285371in}}%
\pgfpathcurveto{\pgfqpoint{2.481509in}{1.277558in}}{\pgfqpoint{2.477119in}{1.266959in}}{\pgfqpoint{2.477119in}{1.255908in}}%
\pgfpathcurveto{\pgfqpoint{2.477119in}{1.244858in}}{\pgfqpoint{2.481509in}{1.234259in}}{\pgfqpoint{2.489323in}{1.226446in}}%
\pgfpathcurveto{\pgfqpoint{2.497137in}{1.218632in}}{\pgfqpoint{2.507736in}{1.214242in}}{\pgfqpoint{2.518786in}{1.214242in}}%
\pgfpathclose%
\pgfusepath{stroke,fill}%
\end{pgfscope}%
\begin{pgfscope}%
\pgfpathrectangle{\pgfqpoint{0.800000in}{0.528000in}}{\pgfqpoint{4.960000in}{3.696000in}}%
\pgfusepath{clip}%
\pgfsetbuttcap%
\pgfsetroundjoin%
\definecolor{currentfill}{rgb}{0.000000,0.000000,0.000000}%
\pgfsetfillcolor{currentfill}%
\pgfsetlinewidth{1.003750pt}%
\definecolor{currentstroke}{rgb}{0.000000,0.000000,0.000000}%
\pgfsetstrokecolor{currentstroke}%
\pgfsetdash{}{0pt}%
\pgfpathmoveto{\pgfqpoint{2.518786in}{1.149749in}}%
\pgfpathcurveto{\pgfqpoint{2.529836in}{1.149749in}}{\pgfqpoint{2.540435in}{1.154140in}}{\pgfqpoint{2.548249in}{1.161953in}}%
\pgfpathcurveto{\pgfqpoint{2.556062in}{1.169767in}}{\pgfqpoint{2.560452in}{1.180366in}}{\pgfqpoint{2.560452in}{1.191416in}}%
\pgfpathcurveto{\pgfqpoint{2.560452in}{1.202466in}}{\pgfqpoint{2.556062in}{1.213065in}}{\pgfqpoint{2.548249in}{1.220879in}}%
\pgfpathcurveto{\pgfqpoint{2.540435in}{1.228692in}}{\pgfqpoint{2.529836in}{1.233083in}}{\pgfqpoint{2.518786in}{1.233083in}}%
\pgfpathcurveto{\pgfqpoint{2.507736in}{1.233083in}}{\pgfqpoint{2.497137in}{1.228692in}}{\pgfqpoint{2.489323in}{1.220879in}}%
\pgfpathcurveto{\pgfqpoint{2.481509in}{1.213065in}}{\pgfqpoint{2.477119in}{1.202466in}}{\pgfqpoint{2.477119in}{1.191416in}}%
\pgfpathcurveto{\pgfqpoint{2.477119in}{1.180366in}}{\pgfqpoint{2.481509in}{1.169767in}}{\pgfqpoint{2.489323in}{1.161953in}}%
\pgfpathcurveto{\pgfqpoint{2.497137in}{1.154140in}}{\pgfqpoint{2.507736in}{1.149749in}}{\pgfqpoint{2.518786in}{1.149749in}}%
\pgfpathclose%
\pgfusepath{stroke,fill}%
\end{pgfscope}%
\begin{pgfscope}%
\pgfpathrectangle{\pgfqpoint{0.800000in}{0.528000in}}{\pgfqpoint{4.960000in}{3.696000in}}%
\pgfusepath{clip}%
\pgfsetbuttcap%
\pgfsetroundjoin%
\definecolor{currentfill}{rgb}{0.000000,0.000000,0.000000}%
\pgfsetfillcolor{currentfill}%
\pgfsetlinewidth{1.003750pt}%
\definecolor{currentstroke}{rgb}{0.000000,0.000000,0.000000}%
\pgfsetstrokecolor{currentstroke}%
\pgfsetdash{}{0pt}%
\pgfpathmoveto{\pgfqpoint{2.518786in}{1.128252in}}%
\pgfpathcurveto{\pgfqpoint{2.529836in}{1.128252in}}{\pgfqpoint{2.540435in}{1.132642in}}{\pgfqpoint{2.548249in}{1.140456in}}%
\pgfpathcurveto{\pgfqpoint{2.556062in}{1.148269in}}{\pgfqpoint{2.560452in}{1.158868in}}{\pgfqpoint{2.560452in}{1.169918in}}%
\pgfpathcurveto{\pgfqpoint{2.560452in}{1.180969in}}{\pgfqpoint{2.556062in}{1.191568in}}{\pgfqpoint{2.548249in}{1.199381in}}%
\pgfpathcurveto{\pgfqpoint{2.540435in}{1.207195in}}{\pgfqpoint{2.529836in}{1.211585in}}{\pgfqpoint{2.518786in}{1.211585in}}%
\pgfpathcurveto{\pgfqpoint{2.507736in}{1.211585in}}{\pgfqpoint{2.497137in}{1.207195in}}{\pgfqpoint{2.489323in}{1.199381in}}%
\pgfpathcurveto{\pgfqpoint{2.481509in}{1.191568in}}{\pgfqpoint{2.477119in}{1.180969in}}{\pgfqpoint{2.477119in}{1.169918in}}%
\pgfpathcurveto{\pgfqpoint{2.477119in}{1.158868in}}{\pgfqpoint{2.481509in}{1.148269in}}{\pgfqpoint{2.489323in}{1.140456in}}%
\pgfpathcurveto{\pgfqpoint{2.497137in}{1.132642in}}{\pgfqpoint{2.507736in}{1.128252in}}{\pgfqpoint{2.518786in}{1.128252in}}%
\pgfpathclose%
\pgfusepath{stroke,fill}%
\end{pgfscope}%
\begin{pgfscope}%
\pgfpathrectangle{\pgfqpoint{0.800000in}{0.528000in}}{\pgfqpoint{4.960000in}{3.696000in}}%
\pgfusepath{clip}%
\pgfsetbuttcap%
\pgfsetroundjoin%
\definecolor{currentfill}{rgb}{0.000000,0.000000,0.000000}%
\pgfsetfillcolor{currentfill}%
\pgfsetlinewidth{1.003750pt}%
\definecolor{currentstroke}{rgb}{0.000000,0.000000,0.000000}%
\pgfsetstrokecolor{currentstroke}%
\pgfsetdash{}{0pt}%
\pgfpathmoveto{\pgfqpoint{2.518786in}{1.063759in}}%
\pgfpathcurveto{\pgfqpoint{2.529836in}{1.063759in}}{\pgfqpoint{2.540435in}{1.068150in}}{\pgfqpoint{2.548249in}{1.075963in}}%
\pgfpathcurveto{\pgfqpoint{2.556062in}{1.083777in}}{\pgfqpoint{2.560452in}{1.094376in}}{\pgfqpoint{2.560452in}{1.105426in}}%
\pgfpathcurveto{\pgfqpoint{2.560452in}{1.116476in}}{\pgfqpoint{2.556062in}{1.127075in}}{\pgfqpoint{2.548249in}{1.134889in}}%
\pgfpathcurveto{\pgfqpoint{2.540435in}{1.142702in}}{\pgfqpoint{2.529836in}{1.147093in}}{\pgfqpoint{2.518786in}{1.147093in}}%
\pgfpathcurveto{\pgfqpoint{2.507736in}{1.147093in}}{\pgfqpoint{2.497137in}{1.142702in}}{\pgfqpoint{2.489323in}{1.134889in}}%
\pgfpathcurveto{\pgfqpoint{2.481509in}{1.127075in}}{\pgfqpoint{2.477119in}{1.116476in}}{\pgfqpoint{2.477119in}{1.105426in}}%
\pgfpathcurveto{\pgfqpoint{2.477119in}{1.094376in}}{\pgfqpoint{2.481509in}{1.083777in}}{\pgfqpoint{2.489323in}{1.075963in}}%
\pgfpathcurveto{\pgfqpoint{2.497137in}{1.068150in}}{\pgfqpoint{2.507736in}{1.063759in}}{\pgfqpoint{2.518786in}{1.063759in}}%
\pgfpathclose%
\pgfusepath{stroke,fill}%
\end{pgfscope}%
\begin{pgfscope}%
\pgfpathrectangle{\pgfqpoint{0.800000in}{0.528000in}}{\pgfqpoint{4.960000in}{3.696000in}}%
\pgfusepath{clip}%
\pgfsetbuttcap%
\pgfsetroundjoin%
\definecolor{currentfill}{rgb}{0.000000,0.000000,0.000000}%
\pgfsetfillcolor{currentfill}%
\pgfsetlinewidth{1.003750pt}%
\definecolor{currentstroke}{rgb}{0.000000,0.000000,0.000000}%
\pgfsetstrokecolor{currentstroke}%
\pgfsetdash{}{0pt}%
\pgfpathmoveto{\pgfqpoint{2.518786in}{1.128252in}}%
\pgfpathcurveto{\pgfqpoint{2.529836in}{1.128252in}}{\pgfqpoint{2.540435in}{1.132642in}}{\pgfqpoint{2.548249in}{1.140456in}}%
\pgfpathcurveto{\pgfqpoint{2.556062in}{1.148269in}}{\pgfqpoint{2.560452in}{1.158868in}}{\pgfqpoint{2.560452in}{1.169918in}}%
\pgfpathcurveto{\pgfqpoint{2.560452in}{1.180969in}}{\pgfqpoint{2.556062in}{1.191568in}}{\pgfqpoint{2.548249in}{1.199381in}}%
\pgfpathcurveto{\pgfqpoint{2.540435in}{1.207195in}}{\pgfqpoint{2.529836in}{1.211585in}}{\pgfqpoint{2.518786in}{1.211585in}}%
\pgfpathcurveto{\pgfqpoint{2.507736in}{1.211585in}}{\pgfqpoint{2.497137in}{1.207195in}}{\pgfqpoint{2.489323in}{1.199381in}}%
\pgfpathcurveto{\pgfqpoint{2.481509in}{1.191568in}}{\pgfqpoint{2.477119in}{1.180969in}}{\pgfqpoint{2.477119in}{1.169918in}}%
\pgfpathcurveto{\pgfqpoint{2.477119in}{1.158868in}}{\pgfqpoint{2.481509in}{1.148269in}}{\pgfqpoint{2.489323in}{1.140456in}}%
\pgfpathcurveto{\pgfqpoint{2.497137in}{1.132642in}}{\pgfqpoint{2.507736in}{1.128252in}}{\pgfqpoint{2.518786in}{1.128252in}}%
\pgfpathclose%
\pgfusepath{stroke,fill}%
\end{pgfscope}%
\begin{pgfscope}%
\pgfpathrectangle{\pgfqpoint{0.800000in}{0.528000in}}{\pgfqpoint{4.960000in}{3.696000in}}%
\pgfusepath{clip}%
\pgfsetbuttcap%
\pgfsetroundjoin%
\definecolor{currentfill}{rgb}{0.000000,0.000000,0.000000}%
\pgfsetfillcolor{currentfill}%
\pgfsetlinewidth{1.003750pt}%
\definecolor{currentstroke}{rgb}{0.000000,0.000000,0.000000}%
\pgfsetstrokecolor{currentstroke}%
\pgfsetdash{}{0pt}%
\pgfpathmoveto{\pgfqpoint{2.518786in}{1.106754in}}%
\pgfpathcurveto{\pgfqpoint{2.529836in}{1.106754in}}{\pgfqpoint{2.540435in}{1.111145in}}{\pgfqpoint{2.548249in}{1.118958in}}%
\pgfpathcurveto{\pgfqpoint{2.556062in}{1.126772in}}{\pgfqpoint{2.560452in}{1.137371in}}{\pgfqpoint{2.560452in}{1.148421in}}%
\pgfpathcurveto{\pgfqpoint{2.560452in}{1.159471in}}{\pgfqpoint{2.556062in}{1.170070in}}{\pgfqpoint{2.548249in}{1.177884in}}%
\pgfpathcurveto{\pgfqpoint{2.540435in}{1.185697in}}{\pgfqpoint{2.529836in}{1.190088in}}{\pgfqpoint{2.518786in}{1.190088in}}%
\pgfpathcurveto{\pgfqpoint{2.507736in}{1.190088in}}{\pgfqpoint{2.497137in}{1.185697in}}{\pgfqpoint{2.489323in}{1.177884in}}%
\pgfpathcurveto{\pgfqpoint{2.481509in}{1.170070in}}{\pgfqpoint{2.477119in}{1.159471in}}{\pgfqpoint{2.477119in}{1.148421in}}%
\pgfpathcurveto{\pgfqpoint{2.477119in}{1.137371in}}{\pgfqpoint{2.481509in}{1.126772in}}{\pgfqpoint{2.489323in}{1.118958in}}%
\pgfpathcurveto{\pgfqpoint{2.497137in}{1.111145in}}{\pgfqpoint{2.507736in}{1.106754in}}{\pgfqpoint{2.518786in}{1.106754in}}%
\pgfpathclose%
\pgfusepath{stroke,fill}%
\end{pgfscope}%
\begin{pgfscope}%
\pgfpathrectangle{\pgfqpoint{0.800000in}{0.528000in}}{\pgfqpoint{4.960000in}{3.696000in}}%
\pgfusepath{clip}%
\pgfsetbuttcap%
\pgfsetroundjoin%
\definecolor{currentfill}{rgb}{0.000000,0.000000,0.000000}%
\pgfsetfillcolor{currentfill}%
\pgfsetlinewidth{1.003750pt}%
\definecolor{currentstroke}{rgb}{0.000000,0.000000,0.000000}%
\pgfsetstrokecolor{currentstroke}%
\pgfsetdash{}{0pt}%
\pgfpathmoveto{\pgfqpoint{2.518786in}{1.128252in}}%
\pgfpathcurveto{\pgfqpoint{2.529836in}{1.128252in}}{\pgfqpoint{2.540435in}{1.132642in}}{\pgfqpoint{2.548249in}{1.140456in}}%
\pgfpathcurveto{\pgfqpoint{2.556062in}{1.148269in}}{\pgfqpoint{2.560452in}{1.158868in}}{\pgfqpoint{2.560452in}{1.169918in}}%
\pgfpathcurveto{\pgfqpoint{2.560452in}{1.180969in}}{\pgfqpoint{2.556062in}{1.191568in}}{\pgfqpoint{2.548249in}{1.199381in}}%
\pgfpathcurveto{\pgfqpoint{2.540435in}{1.207195in}}{\pgfqpoint{2.529836in}{1.211585in}}{\pgfqpoint{2.518786in}{1.211585in}}%
\pgfpathcurveto{\pgfqpoint{2.507736in}{1.211585in}}{\pgfqpoint{2.497137in}{1.207195in}}{\pgfqpoint{2.489323in}{1.199381in}}%
\pgfpathcurveto{\pgfqpoint{2.481509in}{1.191568in}}{\pgfqpoint{2.477119in}{1.180969in}}{\pgfqpoint{2.477119in}{1.169918in}}%
\pgfpathcurveto{\pgfqpoint{2.477119in}{1.158868in}}{\pgfqpoint{2.481509in}{1.148269in}}{\pgfqpoint{2.489323in}{1.140456in}}%
\pgfpathcurveto{\pgfqpoint{2.497137in}{1.132642in}}{\pgfqpoint{2.507736in}{1.128252in}}{\pgfqpoint{2.518786in}{1.128252in}}%
\pgfpathclose%
\pgfusepath{stroke,fill}%
\end{pgfscope}%
\begin{pgfscope}%
\pgfpathrectangle{\pgfqpoint{0.800000in}{0.528000in}}{\pgfqpoint{4.960000in}{3.696000in}}%
\pgfusepath{clip}%
\pgfsetbuttcap%
\pgfsetroundjoin%
\definecolor{currentfill}{rgb}{0.000000,0.000000,0.000000}%
\pgfsetfillcolor{currentfill}%
\pgfsetlinewidth{1.003750pt}%
\definecolor{currentstroke}{rgb}{0.000000,0.000000,0.000000}%
\pgfsetstrokecolor{currentstroke}%
\pgfsetdash{}{0pt}%
\pgfpathmoveto{\pgfqpoint{2.518786in}{1.214242in}}%
\pgfpathcurveto{\pgfqpoint{2.529836in}{1.214242in}}{\pgfqpoint{2.540435in}{1.218632in}}{\pgfqpoint{2.548249in}{1.226446in}}%
\pgfpathcurveto{\pgfqpoint{2.556062in}{1.234259in}}{\pgfqpoint{2.560452in}{1.244858in}}{\pgfqpoint{2.560452in}{1.255908in}}%
\pgfpathcurveto{\pgfqpoint{2.560452in}{1.266959in}}{\pgfqpoint{2.556062in}{1.277558in}}{\pgfqpoint{2.548249in}{1.285371in}}%
\pgfpathcurveto{\pgfqpoint{2.540435in}{1.293185in}}{\pgfqpoint{2.529836in}{1.297575in}}{\pgfqpoint{2.518786in}{1.297575in}}%
\pgfpathcurveto{\pgfqpoint{2.507736in}{1.297575in}}{\pgfqpoint{2.497137in}{1.293185in}}{\pgfqpoint{2.489323in}{1.285371in}}%
\pgfpathcurveto{\pgfqpoint{2.481509in}{1.277558in}}{\pgfqpoint{2.477119in}{1.266959in}}{\pgfqpoint{2.477119in}{1.255908in}}%
\pgfpathcurveto{\pgfqpoint{2.477119in}{1.244858in}}{\pgfqpoint{2.481509in}{1.234259in}}{\pgfqpoint{2.489323in}{1.226446in}}%
\pgfpathcurveto{\pgfqpoint{2.497137in}{1.218632in}}{\pgfqpoint{2.507736in}{1.214242in}}{\pgfqpoint{2.518786in}{1.214242in}}%
\pgfpathclose%
\pgfusepath{stroke,fill}%
\end{pgfscope}%
\begin{pgfscope}%
\pgfpathrectangle{\pgfqpoint{0.800000in}{0.528000in}}{\pgfqpoint{4.960000in}{3.696000in}}%
\pgfusepath{clip}%
\pgfsetbuttcap%
\pgfsetroundjoin%
\definecolor{currentfill}{rgb}{0.000000,0.000000,0.000000}%
\pgfsetfillcolor{currentfill}%
\pgfsetlinewidth{1.003750pt}%
\definecolor{currentstroke}{rgb}{0.000000,0.000000,0.000000}%
\pgfsetstrokecolor{currentstroke}%
\pgfsetdash{}{0pt}%
\pgfpathmoveto{\pgfqpoint{2.518786in}{1.128252in}}%
\pgfpathcurveto{\pgfqpoint{2.529836in}{1.128252in}}{\pgfqpoint{2.540435in}{1.132642in}}{\pgfqpoint{2.548249in}{1.140456in}}%
\pgfpathcurveto{\pgfqpoint{2.556062in}{1.148269in}}{\pgfqpoint{2.560452in}{1.158868in}}{\pgfqpoint{2.560452in}{1.169918in}}%
\pgfpathcurveto{\pgfqpoint{2.560452in}{1.180969in}}{\pgfqpoint{2.556062in}{1.191568in}}{\pgfqpoint{2.548249in}{1.199381in}}%
\pgfpathcurveto{\pgfqpoint{2.540435in}{1.207195in}}{\pgfqpoint{2.529836in}{1.211585in}}{\pgfqpoint{2.518786in}{1.211585in}}%
\pgfpathcurveto{\pgfqpoint{2.507736in}{1.211585in}}{\pgfqpoint{2.497137in}{1.207195in}}{\pgfqpoint{2.489323in}{1.199381in}}%
\pgfpathcurveto{\pgfqpoint{2.481509in}{1.191568in}}{\pgfqpoint{2.477119in}{1.180969in}}{\pgfqpoint{2.477119in}{1.169918in}}%
\pgfpathcurveto{\pgfqpoint{2.477119in}{1.158868in}}{\pgfqpoint{2.481509in}{1.148269in}}{\pgfqpoint{2.489323in}{1.140456in}}%
\pgfpathcurveto{\pgfqpoint{2.497137in}{1.132642in}}{\pgfqpoint{2.507736in}{1.128252in}}{\pgfqpoint{2.518786in}{1.128252in}}%
\pgfpathclose%
\pgfusepath{stroke,fill}%
\end{pgfscope}%
\begin{pgfscope}%
\pgfpathrectangle{\pgfqpoint{0.800000in}{0.528000in}}{\pgfqpoint{4.960000in}{3.696000in}}%
\pgfusepath{clip}%
\pgfsetbuttcap%
\pgfsetroundjoin%
\definecolor{currentfill}{rgb}{0.000000,0.000000,0.000000}%
\pgfsetfillcolor{currentfill}%
\pgfsetlinewidth{1.003750pt}%
\definecolor{currentstroke}{rgb}{0.000000,0.000000,0.000000}%
\pgfsetstrokecolor{currentstroke}%
\pgfsetdash{}{0pt}%
\pgfpathmoveto{\pgfqpoint{2.518786in}{1.214242in}}%
\pgfpathcurveto{\pgfqpoint{2.529836in}{1.214242in}}{\pgfqpoint{2.540435in}{1.218632in}}{\pgfqpoint{2.548249in}{1.226446in}}%
\pgfpathcurveto{\pgfqpoint{2.556062in}{1.234259in}}{\pgfqpoint{2.560452in}{1.244858in}}{\pgfqpoint{2.560452in}{1.255908in}}%
\pgfpathcurveto{\pgfqpoint{2.560452in}{1.266959in}}{\pgfqpoint{2.556062in}{1.277558in}}{\pgfqpoint{2.548249in}{1.285371in}}%
\pgfpathcurveto{\pgfqpoint{2.540435in}{1.293185in}}{\pgfqpoint{2.529836in}{1.297575in}}{\pgfqpoint{2.518786in}{1.297575in}}%
\pgfpathcurveto{\pgfqpoint{2.507736in}{1.297575in}}{\pgfqpoint{2.497137in}{1.293185in}}{\pgfqpoint{2.489323in}{1.285371in}}%
\pgfpathcurveto{\pgfqpoint{2.481509in}{1.277558in}}{\pgfqpoint{2.477119in}{1.266959in}}{\pgfqpoint{2.477119in}{1.255908in}}%
\pgfpathcurveto{\pgfqpoint{2.477119in}{1.244858in}}{\pgfqpoint{2.481509in}{1.234259in}}{\pgfqpoint{2.489323in}{1.226446in}}%
\pgfpathcurveto{\pgfqpoint{2.497137in}{1.218632in}}{\pgfqpoint{2.507736in}{1.214242in}}{\pgfqpoint{2.518786in}{1.214242in}}%
\pgfpathclose%
\pgfusepath{stroke,fill}%
\end{pgfscope}%
\begin{pgfscope}%
\pgfpathrectangle{\pgfqpoint{0.800000in}{0.528000in}}{\pgfqpoint{4.960000in}{3.696000in}}%
\pgfusepath{clip}%
\pgfsetbuttcap%
\pgfsetroundjoin%
\definecolor{currentfill}{rgb}{0.000000,0.000000,0.000000}%
\pgfsetfillcolor{currentfill}%
\pgfsetlinewidth{1.003750pt}%
\definecolor{currentstroke}{rgb}{0.000000,0.000000,0.000000}%
\pgfsetstrokecolor{currentstroke}%
\pgfsetdash{}{0pt}%
\pgfpathmoveto{\pgfqpoint{2.518786in}{1.106754in}}%
\pgfpathcurveto{\pgfqpoint{2.529836in}{1.106754in}}{\pgfqpoint{2.540435in}{1.111145in}}{\pgfqpoint{2.548249in}{1.118958in}}%
\pgfpathcurveto{\pgfqpoint{2.556062in}{1.126772in}}{\pgfqpoint{2.560452in}{1.137371in}}{\pgfqpoint{2.560452in}{1.148421in}}%
\pgfpathcurveto{\pgfqpoint{2.560452in}{1.159471in}}{\pgfqpoint{2.556062in}{1.170070in}}{\pgfqpoint{2.548249in}{1.177884in}}%
\pgfpathcurveto{\pgfqpoint{2.540435in}{1.185697in}}{\pgfqpoint{2.529836in}{1.190088in}}{\pgfqpoint{2.518786in}{1.190088in}}%
\pgfpathcurveto{\pgfqpoint{2.507736in}{1.190088in}}{\pgfqpoint{2.497137in}{1.185697in}}{\pgfqpoint{2.489323in}{1.177884in}}%
\pgfpathcurveto{\pgfqpoint{2.481509in}{1.170070in}}{\pgfqpoint{2.477119in}{1.159471in}}{\pgfqpoint{2.477119in}{1.148421in}}%
\pgfpathcurveto{\pgfqpoint{2.477119in}{1.137371in}}{\pgfqpoint{2.481509in}{1.126772in}}{\pgfqpoint{2.489323in}{1.118958in}}%
\pgfpathcurveto{\pgfqpoint{2.497137in}{1.111145in}}{\pgfqpoint{2.507736in}{1.106754in}}{\pgfqpoint{2.518786in}{1.106754in}}%
\pgfpathclose%
\pgfusepath{stroke,fill}%
\end{pgfscope}%
\begin{pgfscope}%
\pgfpathrectangle{\pgfqpoint{0.800000in}{0.528000in}}{\pgfqpoint{4.960000in}{3.696000in}}%
\pgfusepath{clip}%
\pgfsetbuttcap%
\pgfsetroundjoin%
\definecolor{currentfill}{rgb}{0.000000,0.000000,0.000000}%
\pgfsetfillcolor{currentfill}%
\pgfsetlinewidth{1.003750pt}%
\definecolor{currentstroke}{rgb}{0.000000,0.000000,0.000000}%
\pgfsetstrokecolor{currentstroke}%
\pgfsetdash{}{0pt}%
\pgfpathmoveto{\pgfqpoint{2.518786in}{1.214242in}}%
\pgfpathcurveto{\pgfqpoint{2.529836in}{1.214242in}}{\pgfqpoint{2.540435in}{1.218632in}}{\pgfqpoint{2.548249in}{1.226446in}}%
\pgfpathcurveto{\pgfqpoint{2.556062in}{1.234259in}}{\pgfqpoint{2.560452in}{1.244858in}}{\pgfqpoint{2.560452in}{1.255908in}}%
\pgfpathcurveto{\pgfqpoint{2.560452in}{1.266959in}}{\pgfqpoint{2.556062in}{1.277558in}}{\pgfqpoint{2.548249in}{1.285371in}}%
\pgfpathcurveto{\pgfqpoint{2.540435in}{1.293185in}}{\pgfqpoint{2.529836in}{1.297575in}}{\pgfqpoint{2.518786in}{1.297575in}}%
\pgfpathcurveto{\pgfqpoint{2.507736in}{1.297575in}}{\pgfqpoint{2.497137in}{1.293185in}}{\pgfqpoint{2.489323in}{1.285371in}}%
\pgfpathcurveto{\pgfqpoint{2.481509in}{1.277558in}}{\pgfqpoint{2.477119in}{1.266959in}}{\pgfqpoint{2.477119in}{1.255908in}}%
\pgfpathcurveto{\pgfqpoint{2.477119in}{1.244858in}}{\pgfqpoint{2.481509in}{1.234259in}}{\pgfqpoint{2.489323in}{1.226446in}}%
\pgfpathcurveto{\pgfqpoint{2.497137in}{1.218632in}}{\pgfqpoint{2.507736in}{1.214242in}}{\pgfqpoint{2.518786in}{1.214242in}}%
\pgfpathclose%
\pgfusepath{stroke,fill}%
\end{pgfscope}%
\begin{pgfscope}%
\pgfpathrectangle{\pgfqpoint{0.800000in}{0.528000in}}{\pgfqpoint{4.960000in}{3.696000in}}%
\pgfusepath{clip}%
\pgfsetbuttcap%
\pgfsetroundjoin%
\definecolor{currentfill}{rgb}{0.000000,0.000000,0.000000}%
\pgfsetfillcolor{currentfill}%
\pgfsetlinewidth{1.003750pt}%
\definecolor{currentstroke}{rgb}{0.000000,0.000000,0.000000}%
\pgfsetstrokecolor{currentstroke}%
\pgfsetdash{}{0pt}%
\pgfpathmoveto{\pgfqpoint{2.518786in}{1.085257in}}%
\pgfpathcurveto{\pgfqpoint{2.529836in}{1.085257in}}{\pgfqpoint{2.540435in}{1.089647in}}{\pgfqpoint{2.548249in}{1.097461in}}%
\pgfpathcurveto{\pgfqpoint{2.556062in}{1.105274in}}{\pgfqpoint{2.560452in}{1.115873in}}{\pgfqpoint{2.560452in}{1.126923in}}%
\pgfpathcurveto{\pgfqpoint{2.560452in}{1.137974in}}{\pgfqpoint{2.556062in}{1.148573in}}{\pgfqpoint{2.548249in}{1.156386in}}%
\pgfpathcurveto{\pgfqpoint{2.540435in}{1.164200in}}{\pgfqpoint{2.529836in}{1.168590in}}{\pgfqpoint{2.518786in}{1.168590in}}%
\pgfpathcurveto{\pgfqpoint{2.507736in}{1.168590in}}{\pgfqpoint{2.497137in}{1.164200in}}{\pgfqpoint{2.489323in}{1.156386in}}%
\pgfpathcurveto{\pgfqpoint{2.481509in}{1.148573in}}{\pgfqpoint{2.477119in}{1.137974in}}{\pgfqpoint{2.477119in}{1.126923in}}%
\pgfpathcurveto{\pgfqpoint{2.477119in}{1.115873in}}{\pgfqpoint{2.481509in}{1.105274in}}{\pgfqpoint{2.489323in}{1.097461in}}%
\pgfpathcurveto{\pgfqpoint{2.497137in}{1.089647in}}{\pgfqpoint{2.507736in}{1.085257in}}{\pgfqpoint{2.518786in}{1.085257in}}%
\pgfpathclose%
\pgfusepath{stroke,fill}%
\end{pgfscope}%
\begin{pgfscope}%
\pgfpathrectangle{\pgfqpoint{0.800000in}{0.528000in}}{\pgfqpoint{4.960000in}{3.696000in}}%
\pgfusepath{clip}%
\pgfsetbuttcap%
\pgfsetroundjoin%
\definecolor{currentfill}{rgb}{0.000000,0.000000,0.000000}%
\pgfsetfillcolor{currentfill}%
\pgfsetlinewidth{1.003750pt}%
\definecolor{currentstroke}{rgb}{0.000000,0.000000,0.000000}%
\pgfsetstrokecolor{currentstroke}%
\pgfsetdash{}{0pt}%
\pgfpathmoveto{\pgfqpoint{2.518786in}{1.192744in}}%
\pgfpathcurveto{\pgfqpoint{2.529836in}{1.192744in}}{\pgfqpoint{2.540435in}{1.197135in}}{\pgfqpoint{2.548249in}{1.204948in}}%
\pgfpathcurveto{\pgfqpoint{2.556062in}{1.212762in}}{\pgfqpoint{2.560452in}{1.223361in}}{\pgfqpoint{2.560452in}{1.234411in}}%
\pgfpathcurveto{\pgfqpoint{2.560452in}{1.245461in}}{\pgfqpoint{2.556062in}{1.256060in}}{\pgfqpoint{2.548249in}{1.263874in}}%
\pgfpathcurveto{\pgfqpoint{2.540435in}{1.271687in}}{\pgfqpoint{2.529836in}{1.276078in}}{\pgfqpoint{2.518786in}{1.276078in}}%
\pgfpathcurveto{\pgfqpoint{2.507736in}{1.276078in}}{\pgfqpoint{2.497137in}{1.271687in}}{\pgfqpoint{2.489323in}{1.263874in}}%
\pgfpathcurveto{\pgfqpoint{2.481509in}{1.256060in}}{\pgfqpoint{2.477119in}{1.245461in}}{\pgfqpoint{2.477119in}{1.234411in}}%
\pgfpathcurveto{\pgfqpoint{2.477119in}{1.223361in}}{\pgfqpoint{2.481509in}{1.212762in}}{\pgfqpoint{2.489323in}{1.204948in}}%
\pgfpathcurveto{\pgfqpoint{2.497137in}{1.197135in}}{\pgfqpoint{2.507736in}{1.192744in}}{\pgfqpoint{2.518786in}{1.192744in}}%
\pgfpathclose%
\pgfusepath{stroke,fill}%
\end{pgfscope}%
\begin{pgfscope}%
\pgfpathrectangle{\pgfqpoint{0.800000in}{0.528000in}}{\pgfqpoint{4.960000in}{3.696000in}}%
\pgfusepath{clip}%
\pgfsetbuttcap%
\pgfsetroundjoin%
\definecolor{currentfill}{rgb}{0.000000,0.000000,0.000000}%
\pgfsetfillcolor{currentfill}%
\pgfsetlinewidth{1.003750pt}%
\definecolor{currentstroke}{rgb}{0.000000,0.000000,0.000000}%
\pgfsetstrokecolor{currentstroke}%
\pgfsetdash{}{0pt}%
\pgfpathmoveto{\pgfqpoint{2.518786in}{1.214242in}}%
\pgfpathcurveto{\pgfqpoint{2.529836in}{1.214242in}}{\pgfqpoint{2.540435in}{1.218632in}}{\pgfqpoint{2.548249in}{1.226446in}}%
\pgfpathcurveto{\pgfqpoint{2.556062in}{1.234259in}}{\pgfqpoint{2.560452in}{1.244858in}}{\pgfqpoint{2.560452in}{1.255908in}}%
\pgfpathcurveto{\pgfqpoint{2.560452in}{1.266959in}}{\pgfqpoint{2.556062in}{1.277558in}}{\pgfqpoint{2.548249in}{1.285371in}}%
\pgfpathcurveto{\pgfqpoint{2.540435in}{1.293185in}}{\pgfqpoint{2.529836in}{1.297575in}}{\pgfqpoint{2.518786in}{1.297575in}}%
\pgfpathcurveto{\pgfqpoint{2.507736in}{1.297575in}}{\pgfqpoint{2.497137in}{1.293185in}}{\pgfqpoint{2.489323in}{1.285371in}}%
\pgfpathcurveto{\pgfqpoint{2.481509in}{1.277558in}}{\pgfqpoint{2.477119in}{1.266959in}}{\pgfqpoint{2.477119in}{1.255908in}}%
\pgfpathcurveto{\pgfqpoint{2.477119in}{1.244858in}}{\pgfqpoint{2.481509in}{1.234259in}}{\pgfqpoint{2.489323in}{1.226446in}}%
\pgfpathcurveto{\pgfqpoint{2.497137in}{1.218632in}}{\pgfqpoint{2.507736in}{1.214242in}}{\pgfqpoint{2.518786in}{1.214242in}}%
\pgfpathclose%
\pgfusepath{stroke,fill}%
\end{pgfscope}%
\begin{pgfscope}%
\pgfpathrectangle{\pgfqpoint{0.800000in}{0.528000in}}{\pgfqpoint{4.960000in}{3.696000in}}%
\pgfusepath{clip}%
\pgfsetbuttcap%
\pgfsetroundjoin%
\definecolor{currentfill}{rgb}{0.000000,0.000000,0.000000}%
\pgfsetfillcolor{currentfill}%
\pgfsetlinewidth{1.003750pt}%
\definecolor{currentstroke}{rgb}{0.000000,0.000000,0.000000}%
\pgfsetstrokecolor{currentstroke}%
\pgfsetdash{}{0pt}%
\pgfpathmoveto{\pgfqpoint{2.518786in}{1.149749in}}%
\pgfpathcurveto{\pgfqpoint{2.529836in}{1.149749in}}{\pgfqpoint{2.540435in}{1.154140in}}{\pgfqpoint{2.548249in}{1.161953in}}%
\pgfpathcurveto{\pgfqpoint{2.556062in}{1.169767in}}{\pgfqpoint{2.560452in}{1.180366in}}{\pgfqpoint{2.560452in}{1.191416in}}%
\pgfpathcurveto{\pgfqpoint{2.560452in}{1.202466in}}{\pgfqpoint{2.556062in}{1.213065in}}{\pgfqpoint{2.548249in}{1.220879in}}%
\pgfpathcurveto{\pgfqpoint{2.540435in}{1.228692in}}{\pgfqpoint{2.529836in}{1.233083in}}{\pgfqpoint{2.518786in}{1.233083in}}%
\pgfpathcurveto{\pgfqpoint{2.507736in}{1.233083in}}{\pgfqpoint{2.497137in}{1.228692in}}{\pgfqpoint{2.489323in}{1.220879in}}%
\pgfpathcurveto{\pgfqpoint{2.481509in}{1.213065in}}{\pgfqpoint{2.477119in}{1.202466in}}{\pgfqpoint{2.477119in}{1.191416in}}%
\pgfpathcurveto{\pgfqpoint{2.477119in}{1.180366in}}{\pgfqpoint{2.481509in}{1.169767in}}{\pgfqpoint{2.489323in}{1.161953in}}%
\pgfpathcurveto{\pgfqpoint{2.497137in}{1.154140in}}{\pgfqpoint{2.507736in}{1.149749in}}{\pgfqpoint{2.518786in}{1.149749in}}%
\pgfpathclose%
\pgfusepath{stroke,fill}%
\end{pgfscope}%
\begin{pgfscope}%
\pgfpathrectangle{\pgfqpoint{0.800000in}{0.528000in}}{\pgfqpoint{4.960000in}{3.696000in}}%
\pgfusepath{clip}%
\pgfsetbuttcap%
\pgfsetroundjoin%
\definecolor{currentfill}{rgb}{0.000000,0.000000,0.000000}%
\pgfsetfillcolor{currentfill}%
\pgfsetlinewidth{1.003750pt}%
\definecolor{currentstroke}{rgb}{0.000000,0.000000,0.000000}%
\pgfsetstrokecolor{currentstroke}%
\pgfsetdash{}{0pt}%
\pgfpathmoveto{\pgfqpoint{2.518786in}{1.257237in}}%
\pgfpathcurveto{\pgfqpoint{2.529836in}{1.257237in}}{\pgfqpoint{2.540435in}{1.261627in}}{\pgfqpoint{2.548249in}{1.269441in}}%
\pgfpathcurveto{\pgfqpoint{2.556062in}{1.277254in}}{\pgfqpoint{2.560452in}{1.287853in}}{\pgfqpoint{2.560452in}{1.298903in}}%
\pgfpathcurveto{\pgfqpoint{2.560452in}{1.309954in}}{\pgfqpoint{2.556062in}{1.320553in}}{\pgfqpoint{2.548249in}{1.328366in}}%
\pgfpathcurveto{\pgfqpoint{2.540435in}{1.336180in}}{\pgfqpoint{2.529836in}{1.340570in}}{\pgfqpoint{2.518786in}{1.340570in}}%
\pgfpathcurveto{\pgfqpoint{2.507736in}{1.340570in}}{\pgfqpoint{2.497137in}{1.336180in}}{\pgfqpoint{2.489323in}{1.328366in}}%
\pgfpathcurveto{\pgfqpoint{2.481509in}{1.320553in}}{\pgfqpoint{2.477119in}{1.309954in}}{\pgfqpoint{2.477119in}{1.298903in}}%
\pgfpathcurveto{\pgfqpoint{2.477119in}{1.287853in}}{\pgfqpoint{2.481509in}{1.277254in}}{\pgfqpoint{2.489323in}{1.269441in}}%
\pgfpathcurveto{\pgfqpoint{2.497137in}{1.261627in}}{\pgfqpoint{2.507736in}{1.257237in}}{\pgfqpoint{2.518786in}{1.257237in}}%
\pgfpathclose%
\pgfusepath{stroke,fill}%
\end{pgfscope}%
\begin{pgfscope}%
\pgfpathrectangle{\pgfqpoint{0.800000in}{0.528000in}}{\pgfqpoint{4.960000in}{3.696000in}}%
\pgfusepath{clip}%
\pgfsetbuttcap%
\pgfsetroundjoin%
\definecolor{currentfill}{rgb}{0.000000,0.000000,0.000000}%
\pgfsetfillcolor{currentfill}%
\pgfsetlinewidth{1.003750pt}%
\definecolor{currentstroke}{rgb}{0.000000,0.000000,0.000000}%
\pgfsetstrokecolor{currentstroke}%
\pgfsetdash{}{0pt}%
\pgfpathmoveto{\pgfqpoint{2.518786in}{1.300232in}}%
\pgfpathcurveto{\pgfqpoint{2.529836in}{1.300232in}}{\pgfqpoint{2.540435in}{1.304622in}}{\pgfqpoint{2.548249in}{1.312436in}}%
\pgfpathcurveto{\pgfqpoint{2.556062in}{1.320249in}}{\pgfqpoint{2.560452in}{1.330848in}}{\pgfqpoint{2.560452in}{1.341898in}}%
\pgfpathcurveto{\pgfqpoint{2.560452in}{1.352949in}}{\pgfqpoint{2.556062in}{1.363548in}}{\pgfqpoint{2.548249in}{1.371361in}}%
\pgfpathcurveto{\pgfqpoint{2.540435in}{1.379175in}}{\pgfqpoint{2.529836in}{1.383565in}}{\pgfqpoint{2.518786in}{1.383565in}}%
\pgfpathcurveto{\pgfqpoint{2.507736in}{1.383565in}}{\pgfqpoint{2.497137in}{1.379175in}}{\pgfqpoint{2.489323in}{1.371361in}}%
\pgfpathcurveto{\pgfqpoint{2.481509in}{1.363548in}}{\pgfqpoint{2.477119in}{1.352949in}}{\pgfqpoint{2.477119in}{1.341898in}}%
\pgfpathcurveto{\pgfqpoint{2.477119in}{1.330848in}}{\pgfqpoint{2.481509in}{1.320249in}}{\pgfqpoint{2.489323in}{1.312436in}}%
\pgfpathcurveto{\pgfqpoint{2.497137in}{1.304622in}}{\pgfqpoint{2.507736in}{1.300232in}}{\pgfqpoint{2.518786in}{1.300232in}}%
\pgfpathclose%
\pgfusepath{stroke,fill}%
\end{pgfscope}%
\begin{pgfscope}%
\pgfpathrectangle{\pgfqpoint{0.800000in}{0.528000in}}{\pgfqpoint{4.960000in}{3.696000in}}%
\pgfusepath{clip}%
\pgfsetbuttcap%
\pgfsetroundjoin%
\definecolor{currentfill}{rgb}{0.000000,0.000000,0.000000}%
\pgfsetfillcolor{currentfill}%
\pgfsetlinewidth{1.003750pt}%
\definecolor{currentstroke}{rgb}{0.000000,0.000000,0.000000}%
\pgfsetstrokecolor{currentstroke}%
\pgfsetdash{}{0pt}%
\pgfpathmoveto{\pgfqpoint{2.518786in}{1.192744in}}%
\pgfpathcurveto{\pgfqpoint{2.529836in}{1.192744in}}{\pgfqpoint{2.540435in}{1.197135in}}{\pgfqpoint{2.548249in}{1.204948in}}%
\pgfpathcurveto{\pgfqpoint{2.556062in}{1.212762in}}{\pgfqpoint{2.560452in}{1.223361in}}{\pgfqpoint{2.560452in}{1.234411in}}%
\pgfpathcurveto{\pgfqpoint{2.560452in}{1.245461in}}{\pgfqpoint{2.556062in}{1.256060in}}{\pgfqpoint{2.548249in}{1.263874in}}%
\pgfpathcurveto{\pgfqpoint{2.540435in}{1.271687in}}{\pgfqpoint{2.529836in}{1.276078in}}{\pgfqpoint{2.518786in}{1.276078in}}%
\pgfpathcurveto{\pgfqpoint{2.507736in}{1.276078in}}{\pgfqpoint{2.497137in}{1.271687in}}{\pgfqpoint{2.489323in}{1.263874in}}%
\pgfpathcurveto{\pgfqpoint{2.481509in}{1.256060in}}{\pgfqpoint{2.477119in}{1.245461in}}{\pgfqpoint{2.477119in}{1.234411in}}%
\pgfpathcurveto{\pgfqpoint{2.477119in}{1.223361in}}{\pgfqpoint{2.481509in}{1.212762in}}{\pgfqpoint{2.489323in}{1.204948in}}%
\pgfpathcurveto{\pgfqpoint{2.497137in}{1.197135in}}{\pgfqpoint{2.507736in}{1.192744in}}{\pgfqpoint{2.518786in}{1.192744in}}%
\pgfpathclose%
\pgfusepath{stroke,fill}%
\end{pgfscope}%
\begin{pgfscope}%
\pgfpathrectangle{\pgfqpoint{0.800000in}{0.528000in}}{\pgfqpoint{4.960000in}{3.696000in}}%
\pgfusepath{clip}%
\pgfsetbuttcap%
\pgfsetroundjoin%
\definecolor{currentfill}{rgb}{0.000000,0.000000,0.000000}%
\pgfsetfillcolor{currentfill}%
\pgfsetlinewidth{1.003750pt}%
\definecolor{currentstroke}{rgb}{0.000000,0.000000,0.000000}%
\pgfsetstrokecolor{currentstroke}%
\pgfsetdash{}{0pt}%
\pgfpathmoveto{\pgfqpoint{2.518786in}{1.128252in}}%
\pgfpathcurveto{\pgfqpoint{2.529836in}{1.128252in}}{\pgfqpoint{2.540435in}{1.132642in}}{\pgfqpoint{2.548249in}{1.140456in}}%
\pgfpathcurveto{\pgfqpoint{2.556062in}{1.148269in}}{\pgfqpoint{2.560452in}{1.158868in}}{\pgfqpoint{2.560452in}{1.169918in}}%
\pgfpathcurveto{\pgfqpoint{2.560452in}{1.180969in}}{\pgfqpoint{2.556062in}{1.191568in}}{\pgfqpoint{2.548249in}{1.199381in}}%
\pgfpathcurveto{\pgfqpoint{2.540435in}{1.207195in}}{\pgfqpoint{2.529836in}{1.211585in}}{\pgfqpoint{2.518786in}{1.211585in}}%
\pgfpathcurveto{\pgfqpoint{2.507736in}{1.211585in}}{\pgfqpoint{2.497137in}{1.207195in}}{\pgfqpoint{2.489323in}{1.199381in}}%
\pgfpathcurveto{\pgfqpoint{2.481509in}{1.191568in}}{\pgfqpoint{2.477119in}{1.180969in}}{\pgfqpoint{2.477119in}{1.169918in}}%
\pgfpathcurveto{\pgfqpoint{2.477119in}{1.158868in}}{\pgfqpoint{2.481509in}{1.148269in}}{\pgfqpoint{2.489323in}{1.140456in}}%
\pgfpathcurveto{\pgfqpoint{2.497137in}{1.132642in}}{\pgfqpoint{2.507736in}{1.128252in}}{\pgfqpoint{2.518786in}{1.128252in}}%
\pgfpathclose%
\pgfusepath{stroke,fill}%
\end{pgfscope}%
\begin{pgfscope}%
\pgfpathrectangle{\pgfqpoint{0.800000in}{0.528000in}}{\pgfqpoint{4.960000in}{3.696000in}}%
\pgfusepath{clip}%
\pgfsetbuttcap%
\pgfsetroundjoin%
\definecolor{currentfill}{rgb}{0.000000,0.000000,0.000000}%
\pgfsetfillcolor{currentfill}%
\pgfsetlinewidth{1.003750pt}%
\definecolor{currentstroke}{rgb}{0.000000,0.000000,0.000000}%
\pgfsetstrokecolor{currentstroke}%
\pgfsetdash{}{0pt}%
\pgfpathmoveto{\pgfqpoint{2.518786in}{1.128252in}}%
\pgfpathcurveto{\pgfqpoint{2.529836in}{1.128252in}}{\pgfqpoint{2.540435in}{1.132642in}}{\pgfqpoint{2.548249in}{1.140456in}}%
\pgfpathcurveto{\pgfqpoint{2.556062in}{1.148269in}}{\pgfqpoint{2.560452in}{1.158868in}}{\pgfqpoint{2.560452in}{1.169918in}}%
\pgfpathcurveto{\pgfqpoint{2.560452in}{1.180969in}}{\pgfqpoint{2.556062in}{1.191568in}}{\pgfqpoint{2.548249in}{1.199381in}}%
\pgfpathcurveto{\pgfqpoint{2.540435in}{1.207195in}}{\pgfqpoint{2.529836in}{1.211585in}}{\pgfqpoint{2.518786in}{1.211585in}}%
\pgfpathcurveto{\pgfqpoint{2.507736in}{1.211585in}}{\pgfqpoint{2.497137in}{1.207195in}}{\pgfqpoint{2.489323in}{1.199381in}}%
\pgfpathcurveto{\pgfqpoint{2.481509in}{1.191568in}}{\pgfqpoint{2.477119in}{1.180969in}}{\pgfqpoint{2.477119in}{1.169918in}}%
\pgfpathcurveto{\pgfqpoint{2.477119in}{1.158868in}}{\pgfqpoint{2.481509in}{1.148269in}}{\pgfqpoint{2.489323in}{1.140456in}}%
\pgfpathcurveto{\pgfqpoint{2.497137in}{1.132642in}}{\pgfqpoint{2.507736in}{1.128252in}}{\pgfqpoint{2.518786in}{1.128252in}}%
\pgfpathclose%
\pgfusepath{stroke,fill}%
\end{pgfscope}%
\begin{pgfscope}%
\pgfpathrectangle{\pgfqpoint{0.800000in}{0.528000in}}{\pgfqpoint{4.960000in}{3.696000in}}%
\pgfusepath{clip}%
\pgfsetbuttcap%
\pgfsetroundjoin%
\definecolor{currentfill}{rgb}{0.000000,0.000000,0.000000}%
\pgfsetfillcolor{currentfill}%
\pgfsetlinewidth{1.003750pt}%
\definecolor{currentstroke}{rgb}{0.000000,0.000000,0.000000}%
\pgfsetstrokecolor{currentstroke}%
\pgfsetdash{}{0pt}%
\pgfpathmoveto{\pgfqpoint{2.518786in}{1.106754in}}%
\pgfpathcurveto{\pgfqpoint{2.529836in}{1.106754in}}{\pgfqpoint{2.540435in}{1.111145in}}{\pgfqpoint{2.548249in}{1.118958in}}%
\pgfpathcurveto{\pgfqpoint{2.556062in}{1.126772in}}{\pgfqpoint{2.560452in}{1.137371in}}{\pgfqpoint{2.560452in}{1.148421in}}%
\pgfpathcurveto{\pgfqpoint{2.560452in}{1.159471in}}{\pgfqpoint{2.556062in}{1.170070in}}{\pgfqpoint{2.548249in}{1.177884in}}%
\pgfpathcurveto{\pgfqpoint{2.540435in}{1.185697in}}{\pgfqpoint{2.529836in}{1.190088in}}{\pgfqpoint{2.518786in}{1.190088in}}%
\pgfpathcurveto{\pgfqpoint{2.507736in}{1.190088in}}{\pgfqpoint{2.497137in}{1.185697in}}{\pgfqpoint{2.489323in}{1.177884in}}%
\pgfpathcurveto{\pgfqpoint{2.481509in}{1.170070in}}{\pgfqpoint{2.477119in}{1.159471in}}{\pgfqpoint{2.477119in}{1.148421in}}%
\pgfpathcurveto{\pgfqpoint{2.477119in}{1.137371in}}{\pgfqpoint{2.481509in}{1.126772in}}{\pgfqpoint{2.489323in}{1.118958in}}%
\pgfpathcurveto{\pgfqpoint{2.497137in}{1.111145in}}{\pgfqpoint{2.507736in}{1.106754in}}{\pgfqpoint{2.518786in}{1.106754in}}%
\pgfpathclose%
\pgfusepath{stroke,fill}%
\end{pgfscope}%
\begin{pgfscope}%
\pgfpathrectangle{\pgfqpoint{0.800000in}{0.528000in}}{\pgfqpoint{4.960000in}{3.696000in}}%
\pgfusepath{clip}%
\pgfsetbuttcap%
\pgfsetroundjoin%
\definecolor{currentfill}{rgb}{0.000000,0.000000,0.000000}%
\pgfsetfillcolor{currentfill}%
\pgfsetlinewidth{1.003750pt}%
\definecolor{currentstroke}{rgb}{0.000000,0.000000,0.000000}%
\pgfsetstrokecolor{currentstroke}%
\pgfsetdash{}{0pt}%
\pgfpathmoveto{\pgfqpoint{2.518786in}{1.214242in}}%
\pgfpathcurveto{\pgfqpoint{2.529836in}{1.214242in}}{\pgfqpoint{2.540435in}{1.218632in}}{\pgfqpoint{2.548249in}{1.226446in}}%
\pgfpathcurveto{\pgfqpoint{2.556062in}{1.234259in}}{\pgfqpoint{2.560452in}{1.244858in}}{\pgfqpoint{2.560452in}{1.255908in}}%
\pgfpathcurveto{\pgfqpoint{2.560452in}{1.266959in}}{\pgfqpoint{2.556062in}{1.277558in}}{\pgfqpoint{2.548249in}{1.285371in}}%
\pgfpathcurveto{\pgfqpoint{2.540435in}{1.293185in}}{\pgfqpoint{2.529836in}{1.297575in}}{\pgfqpoint{2.518786in}{1.297575in}}%
\pgfpathcurveto{\pgfqpoint{2.507736in}{1.297575in}}{\pgfqpoint{2.497137in}{1.293185in}}{\pgfqpoint{2.489323in}{1.285371in}}%
\pgfpathcurveto{\pgfqpoint{2.481509in}{1.277558in}}{\pgfqpoint{2.477119in}{1.266959in}}{\pgfqpoint{2.477119in}{1.255908in}}%
\pgfpathcurveto{\pgfqpoint{2.477119in}{1.244858in}}{\pgfqpoint{2.481509in}{1.234259in}}{\pgfqpoint{2.489323in}{1.226446in}}%
\pgfpathcurveto{\pgfqpoint{2.497137in}{1.218632in}}{\pgfqpoint{2.507736in}{1.214242in}}{\pgfqpoint{2.518786in}{1.214242in}}%
\pgfpathclose%
\pgfusepath{stroke,fill}%
\end{pgfscope}%
\begin{pgfscope}%
\pgfpathrectangle{\pgfqpoint{0.800000in}{0.528000in}}{\pgfqpoint{4.960000in}{3.696000in}}%
\pgfusepath{clip}%
\pgfsetbuttcap%
\pgfsetroundjoin%
\definecolor{currentfill}{rgb}{0.000000,0.000000,0.000000}%
\pgfsetfillcolor{currentfill}%
\pgfsetlinewidth{1.003750pt}%
\definecolor{currentstroke}{rgb}{0.000000,0.000000,0.000000}%
\pgfsetstrokecolor{currentstroke}%
\pgfsetdash{}{0pt}%
\pgfpathmoveto{\pgfqpoint{2.518786in}{1.171247in}}%
\pgfpathcurveto{\pgfqpoint{2.529836in}{1.171247in}}{\pgfqpoint{2.540435in}{1.175637in}}{\pgfqpoint{2.548249in}{1.183451in}}%
\pgfpathcurveto{\pgfqpoint{2.556062in}{1.191264in}}{\pgfqpoint{2.560452in}{1.201863in}}{\pgfqpoint{2.560452in}{1.212913in}}%
\pgfpathcurveto{\pgfqpoint{2.560452in}{1.223964in}}{\pgfqpoint{2.556062in}{1.234563in}}{\pgfqpoint{2.548249in}{1.242376in}}%
\pgfpathcurveto{\pgfqpoint{2.540435in}{1.250190in}}{\pgfqpoint{2.529836in}{1.254580in}}{\pgfqpoint{2.518786in}{1.254580in}}%
\pgfpathcurveto{\pgfqpoint{2.507736in}{1.254580in}}{\pgfqpoint{2.497137in}{1.250190in}}{\pgfqpoint{2.489323in}{1.242376in}}%
\pgfpathcurveto{\pgfqpoint{2.481509in}{1.234563in}}{\pgfqpoint{2.477119in}{1.223964in}}{\pgfqpoint{2.477119in}{1.212913in}}%
\pgfpathcurveto{\pgfqpoint{2.477119in}{1.201863in}}{\pgfqpoint{2.481509in}{1.191264in}}{\pgfqpoint{2.489323in}{1.183451in}}%
\pgfpathcurveto{\pgfqpoint{2.497137in}{1.175637in}}{\pgfqpoint{2.507736in}{1.171247in}}{\pgfqpoint{2.518786in}{1.171247in}}%
\pgfpathclose%
\pgfusepath{stroke,fill}%
\end{pgfscope}%
\begin{pgfscope}%
\pgfpathrectangle{\pgfqpoint{0.800000in}{0.528000in}}{\pgfqpoint{4.960000in}{3.696000in}}%
\pgfusepath{clip}%
\pgfsetbuttcap%
\pgfsetroundjoin%
\definecolor{currentfill}{rgb}{0.000000,0.000000,0.000000}%
\pgfsetfillcolor{currentfill}%
\pgfsetlinewidth{1.003750pt}%
\definecolor{currentstroke}{rgb}{0.000000,0.000000,0.000000}%
\pgfsetstrokecolor{currentstroke}%
\pgfsetdash{}{0pt}%
\pgfpathmoveto{\pgfqpoint{2.518786in}{1.085257in}}%
\pgfpathcurveto{\pgfqpoint{2.529836in}{1.085257in}}{\pgfqpoint{2.540435in}{1.089647in}}{\pgfqpoint{2.548249in}{1.097461in}}%
\pgfpathcurveto{\pgfqpoint{2.556062in}{1.105274in}}{\pgfqpoint{2.560452in}{1.115873in}}{\pgfqpoint{2.560452in}{1.126923in}}%
\pgfpathcurveto{\pgfqpoint{2.560452in}{1.137974in}}{\pgfqpoint{2.556062in}{1.148573in}}{\pgfqpoint{2.548249in}{1.156386in}}%
\pgfpathcurveto{\pgfqpoint{2.540435in}{1.164200in}}{\pgfqpoint{2.529836in}{1.168590in}}{\pgfqpoint{2.518786in}{1.168590in}}%
\pgfpathcurveto{\pgfqpoint{2.507736in}{1.168590in}}{\pgfqpoint{2.497137in}{1.164200in}}{\pgfqpoint{2.489323in}{1.156386in}}%
\pgfpathcurveto{\pgfqpoint{2.481509in}{1.148573in}}{\pgfqpoint{2.477119in}{1.137974in}}{\pgfqpoint{2.477119in}{1.126923in}}%
\pgfpathcurveto{\pgfqpoint{2.477119in}{1.115873in}}{\pgfqpoint{2.481509in}{1.105274in}}{\pgfqpoint{2.489323in}{1.097461in}}%
\pgfpathcurveto{\pgfqpoint{2.497137in}{1.089647in}}{\pgfqpoint{2.507736in}{1.085257in}}{\pgfqpoint{2.518786in}{1.085257in}}%
\pgfpathclose%
\pgfusepath{stroke,fill}%
\end{pgfscope}%
\begin{pgfscope}%
\pgfpathrectangle{\pgfqpoint{0.800000in}{0.528000in}}{\pgfqpoint{4.960000in}{3.696000in}}%
\pgfusepath{clip}%
\pgfsetbuttcap%
\pgfsetroundjoin%
\definecolor{currentfill}{rgb}{0.000000,0.000000,0.000000}%
\pgfsetfillcolor{currentfill}%
\pgfsetlinewidth{1.003750pt}%
\definecolor{currentstroke}{rgb}{0.000000,0.000000,0.000000}%
\pgfsetstrokecolor{currentstroke}%
\pgfsetdash{}{0pt}%
\pgfpathmoveto{\pgfqpoint{2.518786in}{1.020764in}}%
\pgfpathcurveto{\pgfqpoint{2.529836in}{1.020764in}}{\pgfqpoint{2.540435in}{1.025155in}}{\pgfqpoint{2.548249in}{1.032968in}}%
\pgfpathcurveto{\pgfqpoint{2.556062in}{1.040782in}}{\pgfqpoint{2.560452in}{1.051381in}}{\pgfqpoint{2.560452in}{1.062431in}}%
\pgfpathcurveto{\pgfqpoint{2.560452in}{1.073481in}}{\pgfqpoint{2.556062in}{1.084080in}}{\pgfqpoint{2.548249in}{1.091894in}}%
\pgfpathcurveto{\pgfqpoint{2.540435in}{1.099707in}}{\pgfqpoint{2.529836in}{1.104098in}}{\pgfqpoint{2.518786in}{1.104098in}}%
\pgfpathcurveto{\pgfqpoint{2.507736in}{1.104098in}}{\pgfqpoint{2.497137in}{1.099707in}}{\pgfqpoint{2.489323in}{1.091894in}}%
\pgfpathcurveto{\pgfqpoint{2.481509in}{1.084080in}}{\pgfqpoint{2.477119in}{1.073481in}}{\pgfqpoint{2.477119in}{1.062431in}}%
\pgfpathcurveto{\pgfqpoint{2.477119in}{1.051381in}}{\pgfqpoint{2.481509in}{1.040782in}}{\pgfqpoint{2.489323in}{1.032968in}}%
\pgfpathcurveto{\pgfqpoint{2.497137in}{1.025155in}}{\pgfqpoint{2.507736in}{1.020764in}}{\pgfqpoint{2.518786in}{1.020764in}}%
\pgfpathclose%
\pgfusepath{stroke,fill}%
\end{pgfscope}%
\begin{pgfscope}%
\pgfpathrectangle{\pgfqpoint{0.800000in}{0.528000in}}{\pgfqpoint{4.960000in}{3.696000in}}%
\pgfusepath{clip}%
\pgfsetbuttcap%
\pgfsetroundjoin%
\definecolor{currentfill}{rgb}{0.000000,0.000000,0.000000}%
\pgfsetfillcolor{currentfill}%
\pgfsetlinewidth{1.003750pt}%
\definecolor{currentstroke}{rgb}{0.000000,0.000000,0.000000}%
\pgfsetstrokecolor{currentstroke}%
\pgfsetdash{}{0pt}%
\pgfpathmoveto{\pgfqpoint{2.518786in}{1.149749in}}%
\pgfpathcurveto{\pgfqpoint{2.529836in}{1.149749in}}{\pgfqpoint{2.540435in}{1.154140in}}{\pgfqpoint{2.548249in}{1.161953in}}%
\pgfpathcurveto{\pgfqpoint{2.556062in}{1.169767in}}{\pgfqpoint{2.560452in}{1.180366in}}{\pgfqpoint{2.560452in}{1.191416in}}%
\pgfpathcurveto{\pgfqpoint{2.560452in}{1.202466in}}{\pgfqpoint{2.556062in}{1.213065in}}{\pgfqpoint{2.548249in}{1.220879in}}%
\pgfpathcurveto{\pgfqpoint{2.540435in}{1.228692in}}{\pgfqpoint{2.529836in}{1.233083in}}{\pgfqpoint{2.518786in}{1.233083in}}%
\pgfpathcurveto{\pgfqpoint{2.507736in}{1.233083in}}{\pgfqpoint{2.497137in}{1.228692in}}{\pgfqpoint{2.489323in}{1.220879in}}%
\pgfpathcurveto{\pgfqpoint{2.481509in}{1.213065in}}{\pgfqpoint{2.477119in}{1.202466in}}{\pgfqpoint{2.477119in}{1.191416in}}%
\pgfpathcurveto{\pgfqpoint{2.477119in}{1.180366in}}{\pgfqpoint{2.481509in}{1.169767in}}{\pgfqpoint{2.489323in}{1.161953in}}%
\pgfpathcurveto{\pgfqpoint{2.497137in}{1.154140in}}{\pgfqpoint{2.507736in}{1.149749in}}{\pgfqpoint{2.518786in}{1.149749in}}%
\pgfpathclose%
\pgfusepath{stroke,fill}%
\end{pgfscope}%
\begin{pgfscope}%
\pgfpathrectangle{\pgfqpoint{0.800000in}{0.528000in}}{\pgfqpoint{4.960000in}{3.696000in}}%
\pgfusepath{clip}%
\pgfsetbuttcap%
\pgfsetroundjoin%
\definecolor{currentfill}{rgb}{0.000000,0.000000,0.000000}%
\pgfsetfillcolor{currentfill}%
\pgfsetlinewidth{1.003750pt}%
\definecolor{currentstroke}{rgb}{0.000000,0.000000,0.000000}%
\pgfsetstrokecolor{currentstroke}%
\pgfsetdash{}{0pt}%
\pgfpathmoveto{\pgfqpoint{2.518786in}{1.128252in}}%
\pgfpathcurveto{\pgfqpoint{2.529836in}{1.128252in}}{\pgfqpoint{2.540435in}{1.132642in}}{\pgfqpoint{2.548249in}{1.140456in}}%
\pgfpathcurveto{\pgfqpoint{2.556062in}{1.148269in}}{\pgfqpoint{2.560452in}{1.158868in}}{\pgfqpoint{2.560452in}{1.169918in}}%
\pgfpathcurveto{\pgfqpoint{2.560452in}{1.180969in}}{\pgfqpoint{2.556062in}{1.191568in}}{\pgfqpoint{2.548249in}{1.199381in}}%
\pgfpathcurveto{\pgfqpoint{2.540435in}{1.207195in}}{\pgfqpoint{2.529836in}{1.211585in}}{\pgfqpoint{2.518786in}{1.211585in}}%
\pgfpathcurveto{\pgfqpoint{2.507736in}{1.211585in}}{\pgfqpoint{2.497137in}{1.207195in}}{\pgfqpoint{2.489323in}{1.199381in}}%
\pgfpathcurveto{\pgfqpoint{2.481509in}{1.191568in}}{\pgfqpoint{2.477119in}{1.180969in}}{\pgfqpoint{2.477119in}{1.169918in}}%
\pgfpathcurveto{\pgfqpoint{2.477119in}{1.158868in}}{\pgfqpoint{2.481509in}{1.148269in}}{\pgfqpoint{2.489323in}{1.140456in}}%
\pgfpathcurveto{\pgfqpoint{2.497137in}{1.132642in}}{\pgfqpoint{2.507736in}{1.128252in}}{\pgfqpoint{2.518786in}{1.128252in}}%
\pgfpathclose%
\pgfusepath{stroke,fill}%
\end{pgfscope}%
\begin{pgfscope}%
\pgfpathrectangle{\pgfqpoint{0.800000in}{0.528000in}}{\pgfqpoint{4.960000in}{3.696000in}}%
\pgfusepath{clip}%
\pgfsetbuttcap%
\pgfsetroundjoin%
\definecolor{currentfill}{rgb}{0.000000,0.000000,0.000000}%
\pgfsetfillcolor{currentfill}%
\pgfsetlinewidth{1.003750pt}%
\definecolor{currentstroke}{rgb}{0.000000,0.000000,0.000000}%
\pgfsetstrokecolor{currentstroke}%
\pgfsetdash{}{0pt}%
\pgfpathmoveto{\pgfqpoint{2.518786in}{1.192744in}}%
\pgfpathcurveto{\pgfqpoint{2.529836in}{1.192744in}}{\pgfqpoint{2.540435in}{1.197135in}}{\pgfqpoint{2.548249in}{1.204948in}}%
\pgfpathcurveto{\pgfqpoint{2.556062in}{1.212762in}}{\pgfqpoint{2.560452in}{1.223361in}}{\pgfqpoint{2.560452in}{1.234411in}}%
\pgfpathcurveto{\pgfqpoint{2.560452in}{1.245461in}}{\pgfqpoint{2.556062in}{1.256060in}}{\pgfqpoint{2.548249in}{1.263874in}}%
\pgfpathcurveto{\pgfqpoint{2.540435in}{1.271687in}}{\pgfqpoint{2.529836in}{1.276078in}}{\pgfqpoint{2.518786in}{1.276078in}}%
\pgfpathcurveto{\pgfqpoint{2.507736in}{1.276078in}}{\pgfqpoint{2.497137in}{1.271687in}}{\pgfqpoint{2.489323in}{1.263874in}}%
\pgfpathcurveto{\pgfqpoint{2.481509in}{1.256060in}}{\pgfqpoint{2.477119in}{1.245461in}}{\pgfqpoint{2.477119in}{1.234411in}}%
\pgfpathcurveto{\pgfqpoint{2.477119in}{1.223361in}}{\pgfqpoint{2.481509in}{1.212762in}}{\pgfqpoint{2.489323in}{1.204948in}}%
\pgfpathcurveto{\pgfqpoint{2.497137in}{1.197135in}}{\pgfqpoint{2.507736in}{1.192744in}}{\pgfqpoint{2.518786in}{1.192744in}}%
\pgfpathclose%
\pgfusepath{stroke,fill}%
\end{pgfscope}%
\begin{pgfscope}%
\pgfpathrectangle{\pgfqpoint{0.800000in}{0.528000in}}{\pgfqpoint{4.960000in}{3.696000in}}%
\pgfusepath{clip}%
\pgfsetbuttcap%
\pgfsetroundjoin%
\definecolor{currentfill}{rgb}{0.000000,0.000000,0.000000}%
\pgfsetfillcolor{currentfill}%
\pgfsetlinewidth{1.003750pt}%
\definecolor{currentstroke}{rgb}{0.000000,0.000000,0.000000}%
\pgfsetstrokecolor{currentstroke}%
\pgfsetdash{}{0pt}%
\pgfpathmoveto{\pgfqpoint{2.518786in}{1.085257in}}%
\pgfpathcurveto{\pgfqpoint{2.529836in}{1.085257in}}{\pgfqpoint{2.540435in}{1.089647in}}{\pgfqpoint{2.548249in}{1.097461in}}%
\pgfpathcurveto{\pgfqpoint{2.556062in}{1.105274in}}{\pgfqpoint{2.560452in}{1.115873in}}{\pgfqpoint{2.560452in}{1.126923in}}%
\pgfpathcurveto{\pgfqpoint{2.560452in}{1.137974in}}{\pgfqpoint{2.556062in}{1.148573in}}{\pgfqpoint{2.548249in}{1.156386in}}%
\pgfpathcurveto{\pgfqpoint{2.540435in}{1.164200in}}{\pgfqpoint{2.529836in}{1.168590in}}{\pgfqpoint{2.518786in}{1.168590in}}%
\pgfpathcurveto{\pgfqpoint{2.507736in}{1.168590in}}{\pgfqpoint{2.497137in}{1.164200in}}{\pgfqpoint{2.489323in}{1.156386in}}%
\pgfpathcurveto{\pgfqpoint{2.481509in}{1.148573in}}{\pgfqpoint{2.477119in}{1.137974in}}{\pgfqpoint{2.477119in}{1.126923in}}%
\pgfpathcurveto{\pgfqpoint{2.477119in}{1.115873in}}{\pgfqpoint{2.481509in}{1.105274in}}{\pgfqpoint{2.489323in}{1.097461in}}%
\pgfpathcurveto{\pgfqpoint{2.497137in}{1.089647in}}{\pgfqpoint{2.507736in}{1.085257in}}{\pgfqpoint{2.518786in}{1.085257in}}%
\pgfpathclose%
\pgfusepath{stroke,fill}%
\end{pgfscope}%
\begin{pgfscope}%
\pgfpathrectangle{\pgfqpoint{0.800000in}{0.528000in}}{\pgfqpoint{4.960000in}{3.696000in}}%
\pgfusepath{clip}%
\pgfsetbuttcap%
\pgfsetroundjoin%
\definecolor{currentfill}{rgb}{0.000000,0.000000,0.000000}%
\pgfsetfillcolor{currentfill}%
\pgfsetlinewidth{1.003750pt}%
\definecolor{currentstroke}{rgb}{0.000000,0.000000,0.000000}%
\pgfsetstrokecolor{currentstroke}%
\pgfsetdash{}{0pt}%
\pgfpathmoveto{\pgfqpoint{2.518786in}{1.171247in}}%
\pgfpathcurveto{\pgfqpoint{2.529836in}{1.171247in}}{\pgfqpoint{2.540435in}{1.175637in}}{\pgfqpoint{2.548249in}{1.183451in}}%
\pgfpathcurveto{\pgfqpoint{2.556062in}{1.191264in}}{\pgfqpoint{2.560452in}{1.201863in}}{\pgfqpoint{2.560452in}{1.212913in}}%
\pgfpathcurveto{\pgfqpoint{2.560452in}{1.223964in}}{\pgfqpoint{2.556062in}{1.234563in}}{\pgfqpoint{2.548249in}{1.242376in}}%
\pgfpathcurveto{\pgfqpoint{2.540435in}{1.250190in}}{\pgfqpoint{2.529836in}{1.254580in}}{\pgfqpoint{2.518786in}{1.254580in}}%
\pgfpathcurveto{\pgfqpoint{2.507736in}{1.254580in}}{\pgfqpoint{2.497137in}{1.250190in}}{\pgfqpoint{2.489323in}{1.242376in}}%
\pgfpathcurveto{\pgfqpoint{2.481509in}{1.234563in}}{\pgfqpoint{2.477119in}{1.223964in}}{\pgfqpoint{2.477119in}{1.212913in}}%
\pgfpathcurveto{\pgfqpoint{2.477119in}{1.201863in}}{\pgfqpoint{2.481509in}{1.191264in}}{\pgfqpoint{2.489323in}{1.183451in}}%
\pgfpathcurveto{\pgfqpoint{2.497137in}{1.175637in}}{\pgfqpoint{2.507736in}{1.171247in}}{\pgfqpoint{2.518786in}{1.171247in}}%
\pgfpathclose%
\pgfusepath{stroke,fill}%
\end{pgfscope}%
\begin{pgfscope}%
\pgfpathrectangle{\pgfqpoint{0.800000in}{0.528000in}}{\pgfqpoint{4.960000in}{3.696000in}}%
\pgfusepath{clip}%
\pgfsetbuttcap%
\pgfsetroundjoin%
\definecolor{currentfill}{rgb}{0.000000,0.000000,0.000000}%
\pgfsetfillcolor{currentfill}%
\pgfsetlinewidth{1.003750pt}%
\definecolor{currentstroke}{rgb}{0.000000,0.000000,0.000000}%
\pgfsetstrokecolor{currentstroke}%
\pgfsetdash{}{0pt}%
\pgfpathmoveto{\pgfqpoint{2.518786in}{1.128252in}}%
\pgfpathcurveto{\pgfqpoint{2.529836in}{1.128252in}}{\pgfqpoint{2.540435in}{1.132642in}}{\pgfqpoint{2.548249in}{1.140456in}}%
\pgfpathcurveto{\pgfqpoint{2.556062in}{1.148269in}}{\pgfqpoint{2.560452in}{1.158868in}}{\pgfqpoint{2.560452in}{1.169918in}}%
\pgfpathcurveto{\pgfqpoint{2.560452in}{1.180969in}}{\pgfqpoint{2.556062in}{1.191568in}}{\pgfqpoint{2.548249in}{1.199381in}}%
\pgfpathcurveto{\pgfqpoint{2.540435in}{1.207195in}}{\pgfqpoint{2.529836in}{1.211585in}}{\pgfqpoint{2.518786in}{1.211585in}}%
\pgfpathcurveto{\pgfqpoint{2.507736in}{1.211585in}}{\pgfqpoint{2.497137in}{1.207195in}}{\pgfqpoint{2.489323in}{1.199381in}}%
\pgfpathcurveto{\pgfqpoint{2.481509in}{1.191568in}}{\pgfqpoint{2.477119in}{1.180969in}}{\pgfqpoint{2.477119in}{1.169918in}}%
\pgfpathcurveto{\pgfqpoint{2.477119in}{1.158868in}}{\pgfqpoint{2.481509in}{1.148269in}}{\pgfqpoint{2.489323in}{1.140456in}}%
\pgfpathcurveto{\pgfqpoint{2.497137in}{1.132642in}}{\pgfqpoint{2.507736in}{1.128252in}}{\pgfqpoint{2.518786in}{1.128252in}}%
\pgfpathclose%
\pgfusepath{stroke,fill}%
\end{pgfscope}%
\begin{pgfscope}%
\pgfpathrectangle{\pgfqpoint{0.800000in}{0.528000in}}{\pgfqpoint{4.960000in}{3.696000in}}%
\pgfusepath{clip}%
\pgfsetbuttcap%
\pgfsetroundjoin%
\definecolor{currentfill}{rgb}{0.000000,0.000000,0.000000}%
\pgfsetfillcolor{currentfill}%
\pgfsetlinewidth{1.003750pt}%
\definecolor{currentstroke}{rgb}{0.000000,0.000000,0.000000}%
\pgfsetstrokecolor{currentstroke}%
\pgfsetdash{}{0pt}%
\pgfpathmoveto{\pgfqpoint{2.518786in}{1.192744in}}%
\pgfpathcurveto{\pgfqpoint{2.529836in}{1.192744in}}{\pgfqpoint{2.540435in}{1.197135in}}{\pgfqpoint{2.548249in}{1.204948in}}%
\pgfpathcurveto{\pgfqpoint{2.556062in}{1.212762in}}{\pgfqpoint{2.560452in}{1.223361in}}{\pgfqpoint{2.560452in}{1.234411in}}%
\pgfpathcurveto{\pgfqpoint{2.560452in}{1.245461in}}{\pgfqpoint{2.556062in}{1.256060in}}{\pgfqpoint{2.548249in}{1.263874in}}%
\pgfpathcurveto{\pgfqpoint{2.540435in}{1.271687in}}{\pgfqpoint{2.529836in}{1.276078in}}{\pgfqpoint{2.518786in}{1.276078in}}%
\pgfpathcurveto{\pgfqpoint{2.507736in}{1.276078in}}{\pgfqpoint{2.497137in}{1.271687in}}{\pgfqpoint{2.489323in}{1.263874in}}%
\pgfpathcurveto{\pgfqpoint{2.481509in}{1.256060in}}{\pgfqpoint{2.477119in}{1.245461in}}{\pgfqpoint{2.477119in}{1.234411in}}%
\pgfpathcurveto{\pgfqpoint{2.477119in}{1.223361in}}{\pgfqpoint{2.481509in}{1.212762in}}{\pgfqpoint{2.489323in}{1.204948in}}%
\pgfpathcurveto{\pgfqpoint{2.497137in}{1.197135in}}{\pgfqpoint{2.507736in}{1.192744in}}{\pgfqpoint{2.518786in}{1.192744in}}%
\pgfpathclose%
\pgfusepath{stroke,fill}%
\end{pgfscope}%
\begin{pgfscope}%
\pgfpathrectangle{\pgfqpoint{0.800000in}{0.528000in}}{\pgfqpoint{4.960000in}{3.696000in}}%
\pgfusepath{clip}%
\pgfsetbuttcap%
\pgfsetroundjoin%
\definecolor{currentfill}{rgb}{0.000000,0.000000,0.000000}%
\pgfsetfillcolor{currentfill}%
\pgfsetlinewidth{1.003750pt}%
\definecolor{currentstroke}{rgb}{0.000000,0.000000,0.000000}%
\pgfsetstrokecolor{currentstroke}%
\pgfsetdash{}{0pt}%
\pgfpathmoveto{\pgfqpoint{2.518786in}{1.149749in}}%
\pgfpathcurveto{\pgfqpoint{2.529836in}{1.149749in}}{\pgfqpoint{2.540435in}{1.154140in}}{\pgfqpoint{2.548249in}{1.161953in}}%
\pgfpathcurveto{\pgfqpoint{2.556062in}{1.169767in}}{\pgfqpoint{2.560452in}{1.180366in}}{\pgfqpoint{2.560452in}{1.191416in}}%
\pgfpathcurveto{\pgfqpoint{2.560452in}{1.202466in}}{\pgfqpoint{2.556062in}{1.213065in}}{\pgfqpoint{2.548249in}{1.220879in}}%
\pgfpathcurveto{\pgfqpoint{2.540435in}{1.228692in}}{\pgfqpoint{2.529836in}{1.233083in}}{\pgfqpoint{2.518786in}{1.233083in}}%
\pgfpathcurveto{\pgfqpoint{2.507736in}{1.233083in}}{\pgfqpoint{2.497137in}{1.228692in}}{\pgfqpoint{2.489323in}{1.220879in}}%
\pgfpathcurveto{\pgfqpoint{2.481509in}{1.213065in}}{\pgfqpoint{2.477119in}{1.202466in}}{\pgfqpoint{2.477119in}{1.191416in}}%
\pgfpathcurveto{\pgfqpoint{2.477119in}{1.180366in}}{\pgfqpoint{2.481509in}{1.169767in}}{\pgfqpoint{2.489323in}{1.161953in}}%
\pgfpathcurveto{\pgfqpoint{2.497137in}{1.154140in}}{\pgfqpoint{2.507736in}{1.149749in}}{\pgfqpoint{2.518786in}{1.149749in}}%
\pgfpathclose%
\pgfusepath{stroke,fill}%
\end{pgfscope}%
\begin{pgfscope}%
\pgfpathrectangle{\pgfqpoint{0.800000in}{0.528000in}}{\pgfqpoint{4.960000in}{3.696000in}}%
\pgfusepath{clip}%
\pgfsetbuttcap%
\pgfsetroundjoin%
\definecolor{currentfill}{rgb}{0.000000,0.000000,0.000000}%
\pgfsetfillcolor{currentfill}%
\pgfsetlinewidth{1.003750pt}%
\definecolor{currentstroke}{rgb}{0.000000,0.000000,0.000000}%
\pgfsetstrokecolor{currentstroke}%
\pgfsetdash{}{0pt}%
\pgfpathmoveto{\pgfqpoint{2.518786in}{1.128252in}}%
\pgfpathcurveto{\pgfqpoint{2.529836in}{1.128252in}}{\pgfqpoint{2.540435in}{1.132642in}}{\pgfqpoint{2.548249in}{1.140456in}}%
\pgfpathcurveto{\pgfqpoint{2.556062in}{1.148269in}}{\pgfqpoint{2.560452in}{1.158868in}}{\pgfqpoint{2.560452in}{1.169918in}}%
\pgfpathcurveto{\pgfqpoint{2.560452in}{1.180969in}}{\pgfqpoint{2.556062in}{1.191568in}}{\pgfqpoint{2.548249in}{1.199381in}}%
\pgfpathcurveto{\pgfqpoint{2.540435in}{1.207195in}}{\pgfqpoint{2.529836in}{1.211585in}}{\pgfqpoint{2.518786in}{1.211585in}}%
\pgfpathcurveto{\pgfqpoint{2.507736in}{1.211585in}}{\pgfqpoint{2.497137in}{1.207195in}}{\pgfqpoint{2.489323in}{1.199381in}}%
\pgfpathcurveto{\pgfqpoint{2.481509in}{1.191568in}}{\pgfqpoint{2.477119in}{1.180969in}}{\pgfqpoint{2.477119in}{1.169918in}}%
\pgfpathcurveto{\pgfqpoint{2.477119in}{1.158868in}}{\pgfqpoint{2.481509in}{1.148269in}}{\pgfqpoint{2.489323in}{1.140456in}}%
\pgfpathcurveto{\pgfqpoint{2.497137in}{1.132642in}}{\pgfqpoint{2.507736in}{1.128252in}}{\pgfqpoint{2.518786in}{1.128252in}}%
\pgfpathclose%
\pgfusepath{stroke,fill}%
\end{pgfscope}%
\begin{pgfscope}%
\pgfpathrectangle{\pgfqpoint{0.800000in}{0.528000in}}{\pgfqpoint{4.960000in}{3.696000in}}%
\pgfusepath{clip}%
\pgfsetbuttcap%
\pgfsetroundjoin%
\definecolor{currentfill}{rgb}{0.000000,0.000000,0.000000}%
\pgfsetfillcolor{currentfill}%
\pgfsetlinewidth{1.003750pt}%
\definecolor{currentstroke}{rgb}{0.000000,0.000000,0.000000}%
\pgfsetstrokecolor{currentstroke}%
\pgfsetdash{}{0pt}%
\pgfpathmoveto{\pgfqpoint{2.518786in}{1.106754in}}%
\pgfpathcurveto{\pgfqpoint{2.529836in}{1.106754in}}{\pgfqpoint{2.540435in}{1.111145in}}{\pgfqpoint{2.548249in}{1.118958in}}%
\pgfpathcurveto{\pgfqpoint{2.556062in}{1.126772in}}{\pgfqpoint{2.560452in}{1.137371in}}{\pgfqpoint{2.560452in}{1.148421in}}%
\pgfpathcurveto{\pgfqpoint{2.560452in}{1.159471in}}{\pgfqpoint{2.556062in}{1.170070in}}{\pgfqpoint{2.548249in}{1.177884in}}%
\pgfpathcurveto{\pgfqpoint{2.540435in}{1.185697in}}{\pgfqpoint{2.529836in}{1.190088in}}{\pgfqpoint{2.518786in}{1.190088in}}%
\pgfpathcurveto{\pgfqpoint{2.507736in}{1.190088in}}{\pgfqpoint{2.497137in}{1.185697in}}{\pgfqpoint{2.489323in}{1.177884in}}%
\pgfpathcurveto{\pgfqpoint{2.481509in}{1.170070in}}{\pgfqpoint{2.477119in}{1.159471in}}{\pgfqpoint{2.477119in}{1.148421in}}%
\pgfpathcurveto{\pgfqpoint{2.477119in}{1.137371in}}{\pgfqpoint{2.481509in}{1.126772in}}{\pgfqpoint{2.489323in}{1.118958in}}%
\pgfpathcurveto{\pgfqpoint{2.497137in}{1.111145in}}{\pgfqpoint{2.507736in}{1.106754in}}{\pgfqpoint{2.518786in}{1.106754in}}%
\pgfpathclose%
\pgfusepath{stroke,fill}%
\end{pgfscope}%
\begin{pgfscope}%
\pgfpathrectangle{\pgfqpoint{0.800000in}{0.528000in}}{\pgfqpoint{4.960000in}{3.696000in}}%
\pgfusepath{clip}%
\pgfsetbuttcap%
\pgfsetroundjoin%
\definecolor{currentfill}{rgb}{0.000000,0.000000,0.000000}%
\pgfsetfillcolor{currentfill}%
\pgfsetlinewidth{1.003750pt}%
\definecolor{currentstroke}{rgb}{0.000000,0.000000,0.000000}%
\pgfsetstrokecolor{currentstroke}%
\pgfsetdash{}{0pt}%
\pgfpathmoveto{\pgfqpoint{2.518786in}{1.235739in}}%
\pgfpathcurveto{\pgfqpoint{2.529836in}{1.235739in}}{\pgfqpoint{2.540435in}{1.240130in}}{\pgfqpoint{2.548249in}{1.247943in}}%
\pgfpathcurveto{\pgfqpoint{2.556062in}{1.255757in}}{\pgfqpoint{2.560452in}{1.266356in}}{\pgfqpoint{2.560452in}{1.277406in}}%
\pgfpathcurveto{\pgfqpoint{2.560452in}{1.288456in}}{\pgfqpoint{2.556062in}{1.299055in}}{\pgfqpoint{2.548249in}{1.306869in}}%
\pgfpathcurveto{\pgfqpoint{2.540435in}{1.314682in}}{\pgfqpoint{2.529836in}{1.319073in}}{\pgfqpoint{2.518786in}{1.319073in}}%
\pgfpathcurveto{\pgfqpoint{2.507736in}{1.319073in}}{\pgfqpoint{2.497137in}{1.314682in}}{\pgfqpoint{2.489323in}{1.306869in}}%
\pgfpathcurveto{\pgfqpoint{2.481509in}{1.299055in}}{\pgfqpoint{2.477119in}{1.288456in}}{\pgfqpoint{2.477119in}{1.277406in}}%
\pgfpathcurveto{\pgfqpoint{2.477119in}{1.266356in}}{\pgfqpoint{2.481509in}{1.255757in}}{\pgfqpoint{2.489323in}{1.247943in}}%
\pgfpathcurveto{\pgfqpoint{2.497137in}{1.240130in}}{\pgfqpoint{2.507736in}{1.235739in}}{\pgfqpoint{2.518786in}{1.235739in}}%
\pgfpathclose%
\pgfusepath{stroke,fill}%
\end{pgfscope}%
\begin{pgfscope}%
\pgfpathrectangle{\pgfqpoint{0.800000in}{0.528000in}}{\pgfqpoint{4.960000in}{3.696000in}}%
\pgfusepath{clip}%
\pgfsetbuttcap%
\pgfsetroundjoin%
\definecolor{currentfill}{rgb}{0.000000,0.000000,0.000000}%
\pgfsetfillcolor{currentfill}%
\pgfsetlinewidth{1.003750pt}%
\definecolor{currentstroke}{rgb}{0.000000,0.000000,0.000000}%
\pgfsetstrokecolor{currentstroke}%
\pgfsetdash{}{0pt}%
\pgfpathmoveto{\pgfqpoint{2.518786in}{1.149749in}}%
\pgfpathcurveto{\pgfqpoint{2.529836in}{1.149749in}}{\pgfqpoint{2.540435in}{1.154140in}}{\pgfqpoint{2.548249in}{1.161953in}}%
\pgfpathcurveto{\pgfqpoint{2.556062in}{1.169767in}}{\pgfqpoint{2.560452in}{1.180366in}}{\pgfqpoint{2.560452in}{1.191416in}}%
\pgfpathcurveto{\pgfqpoint{2.560452in}{1.202466in}}{\pgfqpoint{2.556062in}{1.213065in}}{\pgfqpoint{2.548249in}{1.220879in}}%
\pgfpathcurveto{\pgfqpoint{2.540435in}{1.228692in}}{\pgfqpoint{2.529836in}{1.233083in}}{\pgfqpoint{2.518786in}{1.233083in}}%
\pgfpathcurveto{\pgfqpoint{2.507736in}{1.233083in}}{\pgfqpoint{2.497137in}{1.228692in}}{\pgfqpoint{2.489323in}{1.220879in}}%
\pgfpathcurveto{\pgfqpoint{2.481509in}{1.213065in}}{\pgfqpoint{2.477119in}{1.202466in}}{\pgfqpoint{2.477119in}{1.191416in}}%
\pgfpathcurveto{\pgfqpoint{2.477119in}{1.180366in}}{\pgfqpoint{2.481509in}{1.169767in}}{\pgfqpoint{2.489323in}{1.161953in}}%
\pgfpathcurveto{\pgfqpoint{2.497137in}{1.154140in}}{\pgfqpoint{2.507736in}{1.149749in}}{\pgfqpoint{2.518786in}{1.149749in}}%
\pgfpathclose%
\pgfusepath{stroke,fill}%
\end{pgfscope}%
\begin{pgfscope}%
\pgfpathrectangle{\pgfqpoint{0.800000in}{0.528000in}}{\pgfqpoint{4.960000in}{3.696000in}}%
\pgfusepath{clip}%
\pgfsetbuttcap%
\pgfsetroundjoin%
\definecolor{currentfill}{rgb}{0.000000,0.000000,0.000000}%
\pgfsetfillcolor{currentfill}%
\pgfsetlinewidth{1.003750pt}%
\definecolor{currentstroke}{rgb}{0.000000,0.000000,0.000000}%
\pgfsetstrokecolor{currentstroke}%
\pgfsetdash{}{0pt}%
\pgfpathmoveto{\pgfqpoint{2.518786in}{1.214242in}}%
\pgfpathcurveto{\pgfqpoint{2.529836in}{1.214242in}}{\pgfqpoint{2.540435in}{1.218632in}}{\pgfqpoint{2.548249in}{1.226446in}}%
\pgfpathcurveto{\pgfqpoint{2.556062in}{1.234259in}}{\pgfqpoint{2.560452in}{1.244858in}}{\pgfqpoint{2.560452in}{1.255908in}}%
\pgfpathcurveto{\pgfqpoint{2.560452in}{1.266959in}}{\pgfqpoint{2.556062in}{1.277558in}}{\pgfqpoint{2.548249in}{1.285371in}}%
\pgfpathcurveto{\pgfqpoint{2.540435in}{1.293185in}}{\pgfqpoint{2.529836in}{1.297575in}}{\pgfqpoint{2.518786in}{1.297575in}}%
\pgfpathcurveto{\pgfqpoint{2.507736in}{1.297575in}}{\pgfqpoint{2.497137in}{1.293185in}}{\pgfqpoint{2.489323in}{1.285371in}}%
\pgfpathcurveto{\pgfqpoint{2.481509in}{1.277558in}}{\pgfqpoint{2.477119in}{1.266959in}}{\pgfqpoint{2.477119in}{1.255908in}}%
\pgfpathcurveto{\pgfqpoint{2.477119in}{1.244858in}}{\pgfqpoint{2.481509in}{1.234259in}}{\pgfqpoint{2.489323in}{1.226446in}}%
\pgfpathcurveto{\pgfqpoint{2.497137in}{1.218632in}}{\pgfqpoint{2.507736in}{1.214242in}}{\pgfqpoint{2.518786in}{1.214242in}}%
\pgfpathclose%
\pgfusepath{stroke,fill}%
\end{pgfscope}%
\begin{pgfscope}%
\pgfpathrectangle{\pgfqpoint{0.800000in}{0.528000in}}{\pgfqpoint{4.960000in}{3.696000in}}%
\pgfusepath{clip}%
\pgfsetbuttcap%
\pgfsetroundjoin%
\definecolor{currentfill}{rgb}{0.000000,0.000000,0.000000}%
\pgfsetfillcolor{currentfill}%
\pgfsetlinewidth{1.003750pt}%
\definecolor{currentstroke}{rgb}{0.000000,0.000000,0.000000}%
\pgfsetstrokecolor{currentstroke}%
\pgfsetdash{}{0pt}%
\pgfpathmoveto{\pgfqpoint{2.518786in}{1.106754in}}%
\pgfpathcurveto{\pgfqpoint{2.529836in}{1.106754in}}{\pgfqpoint{2.540435in}{1.111145in}}{\pgfqpoint{2.548249in}{1.118958in}}%
\pgfpathcurveto{\pgfqpoint{2.556062in}{1.126772in}}{\pgfqpoint{2.560452in}{1.137371in}}{\pgfqpoint{2.560452in}{1.148421in}}%
\pgfpathcurveto{\pgfqpoint{2.560452in}{1.159471in}}{\pgfqpoint{2.556062in}{1.170070in}}{\pgfqpoint{2.548249in}{1.177884in}}%
\pgfpathcurveto{\pgfqpoint{2.540435in}{1.185697in}}{\pgfqpoint{2.529836in}{1.190088in}}{\pgfqpoint{2.518786in}{1.190088in}}%
\pgfpathcurveto{\pgfqpoint{2.507736in}{1.190088in}}{\pgfqpoint{2.497137in}{1.185697in}}{\pgfqpoint{2.489323in}{1.177884in}}%
\pgfpathcurveto{\pgfqpoint{2.481509in}{1.170070in}}{\pgfqpoint{2.477119in}{1.159471in}}{\pgfqpoint{2.477119in}{1.148421in}}%
\pgfpathcurveto{\pgfqpoint{2.477119in}{1.137371in}}{\pgfqpoint{2.481509in}{1.126772in}}{\pgfqpoint{2.489323in}{1.118958in}}%
\pgfpathcurveto{\pgfqpoint{2.497137in}{1.111145in}}{\pgfqpoint{2.507736in}{1.106754in}}{\pgfqpoint{2.518786in}{1.106754in}}%
\pgfpathclose%
\pgfusepath{stroke,fill}%
\end{pgfscope}%
\begin{pgfscope}%
\pgfpathrectangle{\pgfqpoint{0.800000in}{0.528000in}}{\pgfqpoint{4.960000in}{3.696000in}}%
\pgfusepath{clip}%
\pgfsetbuttcap%
\pgfsetroundjoin%
\definecolor{currentfill}{rgb}{0.000000,0.000000,0.000000}%
\pgfsetfillcolor{currentfill}%
\pgfsetlinewidth{1.003750pt}%
\definecolor{currentstroke}{rgb}{0.000000,0.000000,0.000000}%
\pgfsetstrokecolor{currentstroke}%
\pgfsetdash{}{0pt}%
\pgfpathmoveto{\pgfqpoint{2.518786in}{1.171247in}}%
\pgfpathcurveto{\pgfqpoint{2.529836in}{1.171247in}}{\pgfqpoint{2.540435in}{1.175637in}}{\pgfqpoint{2.548249in}{1.183451in}}%
\pgfpathcurveto{\pgfqpoint{2.556062in}{1.191264in}}{\pgfqpoint{2.560452in}{1.201863in}}{\pgfqpoint{2.560452in}{1.212913in}}%
\pgfpathcurveto{\pgfqpoint{2.560452in}{1.223964in}}{\pgfqpoint{2.556062in}{1.234563in}}{\pgfqpoint{2.548249in}{1.242376in}}%
\pgfpathcurveto{\pgfqpoint{2.540435in}{1.250190in}}{\pgfqpoint{2.529836in}{1.254580in}}{\pgfqpoint{2.518786in}{1.254580in}}%
\pgfpathcurveto{\pgfqpoint{2.507736in}{1.254580in}}{\pgfqpoint{2.497137in}{1.250190in}}{\pgfqpoint{2.489323in}{1.242376in}}%
\pgfpathcurveto{\pgfqpoint{2.481509in}{1.234563in}}{\pgfqpoint{2.477119in}{1.223964in}}{\pgfqpoint{2.477119in}{1.212913in}}%
\pgfpathcurveto{\pgfqpoint{2.477119in}{1.201863in}}{\pgfqpoint{2.481509in}{1.191264in}}{\pgfqpoint{2.489323in}{1.183451in}}%
\pgfpathcurveto{\pgfqpoint{2.497137in}{1.175637in}}{\pgfqpoint{2.507736in}{1.171247in}}{\pgfqpoint{2.518786in}{1.171247in}}%
\pgfpathclose%
\pgfusepath{stroke,fill}%
\end{pgfscope}%
\begin{pgfscope}%
\pgfpathrectangle{\pgfqpoint{0.800000in}{0.528000in}}{\pgfqpoint{4.960000in}{3.696000in}}%
\pgfusepath{clip}%
\pgfsetbuttcap%
\pgfsetroundjoin%
\definecolor{currentfill}{rgb}{0.000000,0.000000,0.000000}%
\pgfsetfillcolor{currentfill}%
\pgfsetlinewidth{1.003750pt}%
\definecolor{currentstroke}{rgb}{0.000000,0.000000,0.000000}%
\pgfsetstrokecolor{currentstroke}%
\pgfsetdash{}{0pt}%
\pgfpathmoveto{\pgfqpoint{2.518786in}{1.042262in}}%
\pgfpathcurveto{\pgfqpoint{2.529836in}{1.042262in}}{\pgfqpoint{2.540435in}{1.046652in}}{\pgfqpoint{2.548249in}{1.054466in}}%
\pgfpathcurveto{\pgfqpoint{2.556062in}{1.062279in}}{\pgfqpoint{2.560452in}{1.072878in}}{\pgfqpoint{2.560452in}{1.083928in}}%
\pgfpathcurveto{\pgfqpoint{2.560452in}{1.094979in}}{\pgfqpoint{2.556062in}{1.105578in}}{\pgfqpoint{2.548249in}{1.113391in}}%
\pgfpathcurveto{\pgfqpoint{2.540435in}{1.121205in}}{\pgfqpoint{2.529836in}{1.125595in}}{\pgfqpoint{2.518786in}{1.125595in}}%
\pgfpathcurveto{\pgfqpoint{2.507736in}{1.125595in}}{\pgfqpoint{2.497137in}{1.121205in}}{\pgfqpoint{2.489323in}{1.113391in}}%
\pgfpathcurveto{\pgfqpoint{2.481509in}{1.105578in}}{\pgfqpoint{2.477119in}{1.094979in}}{\pgfqpoint{2.477119in}{1.083928in}}%
\pgfpathcurveto{\pgfqpoint{2.477119in}{1.072878in}}{\pgfqpoint{2.481509in}{1.062279in}}{\pgfqpoint{2.489323in}{1.054466in}}%
\pgfpathcurveto{\pgfqpoint{2.497137in}{1.046652in}}{\pgfqpoint{2.507736in}{1.042262in}}{\pgfqpoint{2.518786in}{1.042262in}}%
\pgfpathclose%
\pgfusepath{stroke,fill}%
\end{pgfscope}%
\begin{pgfscope}%
\pgfpathrectangle{\pgfqpoint{0.800000in}{0.528000in}}{\pgfqpoint{4.960000in}{3.696000in}}%
\pgfusepath{clip}%
\pgfsetbuttcap%
\pgfsetroundjoin%
\definecolor{currentfill}{rgb}{0.000000,0.000000,0.000000}%
\pgfsetfillcolor{currentfill}%
\pgfsetlinewidth{1.003750pt}%
\definecolor{currentstroke}{rgb}{0.000000,0.000000,0.000000}%
\pgfsetstrokecolor{currentstroke}%
\pgfsetdash{}{0pt}%
\pgfpathmoveto{\pgfqpoint{2.518786in}{1.063759in}}%
\pgfpathcurveto{\pgfqpoint{2.529836in}{1.063759in}}{\pgfqpoint{2.540435in}{1.068150in}}{\pgfqpoint{2.548249in}{1.075963in}}%
\pgfpathcurveto{\pgfqpoint{2.556062in}{1.083777in}}{\pgfqpoint{2.560452in}{1.094376in}}{\pgfqpoint{2.560452in}{1.105426in}}%
\pgfpathcurveto{\pgfqpoint{2.560452in}{1.116476in}}{\pgfqpoint{2.556062in}{1.127075in}}{\pgfqpoint{2.548249in}{1.134889in}}%
\pgfpathcurveto{\pgfqpoint{2.540435in}{1.142702in}}{\pgfqpoint{2.529836in}{1.147093in}}{\pgfqpoint{2.518786in}{1.147093in}}%
\pgfpathcurveto{\pgfqpoint{2.507736in}{1.147093in}}{\pgfqpoint{2.497137in}{1.142702in}}{\pgfqpoint{2.489323in}{1.134889in}}%
\pgfpathcurveto{\pgfqpoint{2.481509in}{1.127075in}}{\pgfqpoint{2.477119in}{1.116476in}}{\pgfqpoint{2.477119in}{1.105426in}}%
\pgfpathcurveto{\pgfqpoint{2.477119in}{1.094376in}}{\pgfqpoint{2.481509in}{1.083777in}}{\pgfqpoint{2.489323in}{1.075963in}}%
\pgfpathcurveto{\pgfqpoint{2.497137in}{1.068150in}}{\pgfqpoint{2.507736in}{1.063759in}}{\pgfqpoint{2.518786in}{1.063759in}}%
\pgfpathclose%
\pgfusepath{stroke,fill}%
\end{pgfscope}%
\begin{pgfscope}%
\pgfpathrectangle{\pgfqpoint{0.800000in}{0.528000in}}{\pgfqpoint{4.960000in}{3.696000in}}%
\pgfusepath{clip}%
\pgfsetbuttcap%
\pgfsetroundjoin%
\definecolor{currentfill}{rgb}{0.000000,0.000000,0.000000}%
\pgfsetfillcolor{currentfill}%
\pgfsetlinewidth{1.003750pt}%
\definecolor{currentstroke}{rgb}{0.000000,0.000000,0.000000}%
\pgfsetstrokecolor{currentstroke}%
\pgfsetdash{}{0pt}%
\pgfpathmoveto{\pgfqpoint{2.518786in}{1.214242in}}%
\pgfpathcurveto{\pgfqpoint{2.529836in}{1.214242in}}{\pgfqpoint{2.540435in}{1.218632in}}{\pgfqpoint{2.548249in}{1.226446in}}%
\pgfpathcurveto{\pgfqpoint{2.556062in}{1.234259in}}{\pgfqpoint{2.560452in}{1.244858in}}{\pgfqpoint{2.560452in}{1.255908in}}%
\pgfpathcurveto{\pgfqpoint{2.560452in}{1.266959in}}{\pgfqpoint{2.556062in}{1.277558in}}{\pgfqpoint{2.548249in}{1.285371in}}%
\pgfpathcurveto{\pgfqpoint{2.540435in}{1.293185in}}{\pgfqpoint{2.529836in}{1.297575in}}{\pgfqpoint{2.518786in}{1.297575in}}%
\pgfpathcurveto{\pgfqpoint{2.507736in}{1.297575in}}{\pgfqpoint{2.497137in}{1.293185in}}{\pgfqpoint{2.489323in}{1.285371in}}%
\pgfpathcurveto{\pgfqpoint{2.481509in}{1.277558in}}{\pgfqpoint{2.477119in}{1.266959in}}{\pgfqpoint{2.477119in}{1.255908in}}%
\pgfpathcurveto{\pgfqpoint{2.477119in}{1.244858in}}{\pgfqpoint{2.481509in}{1.234259in}}{\pgfqpoint{2.489323in}{1.226446in}}%
\pgfpathcurveto{\pgfqpoint{2.497137in}{1.218632in}}{\pgfqpoint{2.507736in}{1.214242in}}{\pgfqpoint{2.518786in}{1.214242in}}%
\pgfpathclose%
\pgfusepath{stroke,fill}%
\end{pgfscope}%
\begin{pgfscope}%
\pgfpathrectangle{\pgfqpoint{0.800000in}{0.528000in}}{\pgfqpoint{4.960000in}{3.696000in}}%
\pgfusepath{clip}%
\pgfsetbuttcap%
\pgfsetroundjoin%
\definecolor{currentfill}{rgb}{0.000000,0.000000,0.000000}%
\pgfsetfillcolor{currentfill}%
\pgfsetlinewidth{1.003750pt}%
\definecolor{currentstroke}{rgb}{0.000000,0.000000,0.000000}%
\pgfsetstrokecolor{currentstroke}%
\pgfsetdash{}{0pt}%
\pgfpathmoveto{\pgfqpoint{2.518786in}{1.128252in}}%
\pgfpathcurveto{\pgfqpoint{2.529836in}{1.128252in}}{\pgfqpoint{2.540435in}{1.132642in}}{\pgfqpoint{2.548249in}{1.140456in}}%
\pgfpathcurveto{\pgfqpoint{2.556062in}{1.148269in}}{\pgfqpoint{2.560452in}{1.158868in}}{\pgfqpoint{2.560452in}{1.169918in}}%
\pgfpathcurveto{\pgfqpoint{2.560452in}{1.180969in}}{\pgfqpoint{2.556062in}{1.191568in}}{\pgfqpoint{2.548249in}{1.199381in}}%
\pgfpathcurveto{\pgfqpoint{2.540435in}{1.207195in}}{\pgfqpoint{2.529836in}{1.211585in}}{\pgfqpoint{2.518786in}{1.211585in}}%
\pgfpathcurveto{\pgfqpoint{2.507736in}{1.211585in}}{\pgfqpoint{2.497137in}{1.207195in}}{\pgfqpoint{2.489323in}{1.199381in}}%
\pgfpathcurveto{\pgfqpoint{2.481509in}{1.191568in}}{\pgfqpoint{2.477119in}{1.180969in}}{\pgfqpoint{2.477119in}{1.169918in}}%
\pgfpathcurveto{\pgfqpoint{2.477119in}{1.158868in}}{\pgfqpoint{2.481509in}{1.148269in}}{\pgfqpoint{2.489323in}{1.140456in}}%
\pgfpathcurveto{\pgfqpoint{2.497137in}{1.132642in}}{\pgfqpoint{2.507736in}{1.128252in}}{\pgfqpoint{2.518786in}{1.128252in}}%
\pgfpathclose%
\pgfusepath{stroke,fill}%
\end{pgfscope}%
\begin{pgfscope}%
\pgfpathrectangle{\pgfqpoint{0.800000in}{0.528000in}}{\pgfqpoint{4.960000in}{3.696000in}}%
\pgfusepath{clip}%
\pgfsetbuttcap%
\pgfsetroundjoin%
\definecolor{currentfill}{rgb}{0.000000,0.000000,0.000000}%
\pgfsetfillcolor{currentfill}%
\pgfsetlinewidth{1.003750pt}%
\definecolor{currentstroke}{rgb}{0.000000,0.000000,0.000000}%
\pgfsetstrokecolor{currentstroke}%
\pgfsetdash{}{0pt}%
\pgfpathmoveto{\pgfqpoint{4.011666in}{1.923659in}}%
\pgfpathcurveto{\pgfqpoint{4.022716in}{1.923659in}}{\pgfqpoint{4.033315in}{1.928049in}}{\pgfqpoint{4.041128in}{1.935863in}}%
\pgfpathcurveto{\pgfqpoint{4.048942in}{1.943677in}}{\pgfqpoint{4.053332in}{1.954276in}}{\pgfqpoint{4.053332in}{1.965326in}}%
\pgfpathcurveto{\pgfqpoint{4.053332in}{1.976376in}}{\pgfqpoint{4.048942in}{1.986975in}}{\pgfqpoint{4.041128in}{1.994788in}}%
\pgfpathcurveto{\pgfqpoint{4.033315in}{2.002602in}}{\pgfqpoint{4.022716in}{2.006992in}}{\pgfqpoint{4.011666in}{2.006992in}}%
\pgfpathcurveto{\pgfqpoint{4.000616in}{2.006992in}}{\pgfqpoint{3.990016in}{2.002602in}}{\pgfqpoint{3.982203in}{1.994788in}}%
\pgfpathcurveto{\pgfqpoint{3.974389in}{1.986975in}}{\pgfqpoint{3.969999in}{1.976376in}}{\pgfqpoint{3.969999in}{1.965326in}}%
\pgfpathcurveto{\pgfqpoint{3.969999in}{1.954276in}}{\pgfqpoint{3.974389in}{1.943677in}}{\pgfqpoint{3.982203in}{1.935863in}}%
\pgfpathcurveto{\pgfqpoint{3.990016in}{1.928049in}}{\pgfqpoint{4.000616in}{1.923659in}}{\pgfqpoint{4.011666in}{1.923659in}}%
\pgfpathclose%
\pgfusepath{stroke,fill}%
\end{pgfscope}%
\begin{pgfscope}%
\pgfpathrectangle{\pgfqpoint{0.800000in}{0.528000in}}{\pgfqpoint{4.960000in}{3.696000in}}%
\pgfusepath{clip}%
\pgfsetbuttcap%
\pgfsetroundjoin%
\definecolor{currentfill}{rgb}{0.000000,0.000000,0.000000}%
\pgfsetfillcolor{currentfill}%
\pgfsetlinewidth{1.003750pt}%
\definecolor{currentstroke}{rgb}{0.000000,0.000000,0.000000}%
\pgfsetstrokecolor{currentstroke}%
\pgfsetdash{}{0pt}%
\pgfpathmoveto{\pgfqpoint{4.011666in}{1.794674in}}%
\pgfpathcurveto{\pgfqpoint{4.022716in}{1.794674in}}{\pgfqpoint{4.033315in}{1.799064in}}{\pgfqpoint{4.041128in}{1.806878in}}%
\pgfpathcurveto{\pgfqpoint{4.048942in}{1.814692in}}{\pgfqpoint{4.053332in}{1.825291in}}{\pgfqpoint{4.053332in}{1.836341in}}%
\pgfpathcurveto{\pgfqpoint{4.053332in}{1.847391in}}{\pgfqpoint{4.048942in}{1.857990in}}{\pgfqpoint{4.041128in}{1.865804in}}%
\pgfpathcurveto{\pgfqpoint{4.033315in}{1.873617in}}{\pgfqpoint{4.022716in}{1.878007in}}{\pgfqpoint{4.011666in}{1.878007in}}%
\pgfpathcurveto{\pgfqpoint{4.000616in}{1.878007in}}{\pgfqpoint{3.990016in}{1.873617in}}{\pgfqpoint{3.982203in}{1.865804in}}%
\pgfpathcurveto{\pgfqpoint{3.974389in}{1.857990in}}{\pgfqpoint{3.969999in}{1.847391in}}{\pgfqpoint{3.969999in}{1.836341in}}%
\pgfpathcurveto{\pgfqpoint{3.969999in}{1.825291in}}{\pgfqpoint{3.974389in}{1.814692in}}{\pgfqpoint{3.982203in}{1.806878in}}%
\pgfpathcurveto{\pgfqpoint{3.990016in}{1.799064in}}{\pgfqpoint{4.000616in}{1.794674in}}{\pgfqpoint{4.011666in}{1.794674in}}%
\pgfpathclose%
\pgfusepath{stroke,fill}%
\end{pgfscope}%
\begin{pgfscope}%
\pgfpathrectangle{\pgfqpoint{0.800000in}{0.528000in}}{\pgfqpoint{4.960000in}{3.696000in}}%
\pgfusepath{clip}%
\pgfsetbuttcap%
\pgfsetroundjoin%
\definecolor{currentfill}{rgb}{0.000000,0.000000,0.000000}%
\pgfsetfillcolor{currentfill}%
\pgfsetlinewidth{1.003750pt}%
\definecolor{currentstroke}{rgb}{0.000000,0.000000,0.000000}%
\pgfsetstrokecolor{currentstroke}%
\pgfsetdash{}{0pt}%
\pgfpathmoveto{\pgfqpoint{4.011666in}{1.859167in}}%
\pgfpathcurveto{\pgfqpoint{4.022716in}{1.859167in}}{\pgfqpoint{4.033315in}{1.863557in}}{\pgfqpoint{4.041128in}{1.871370in}}%
\pgfpathcurveto{\pgfqpoint{4.048942in}{1.879184in}}{\pgfqpoint{4.053332in}{1.889783in}}{\pgfqpoint{4.053332in}{1.900833in}}%
\pgfpathcurveto{\pgfqpoint{4.053332in}{1.911883in}}{\pgfqpoint{4.048942in}{1.922482in}}{\pgfqpoint{4.041128in}{1.930296in}}%
\pgfpathcurveto{\pgfqpoint{4.033315in}{1.938110in}}{\pgfqpoint{4.022716in}{1.942500in}}{\pgfqpoint{4.011666in}{1.942500in}}%
\pgfpathcurveto{\pgfqpoint{4.000616in}{1.942500in}}{\pgfqpoint{3.990016in}{1.938110in}}{\pgfqpoint{3.982203in}{1.930296in}}%
\pgfpathcurveto{\pgfqpoint{3.974389in}{1.922482in}}{\pgfqpoint{3.969999in}{1.911883in}}{\pgfqpoint{3.969999in}{1.900833in}}%
\pgfpathcurveto{\pgfqpoint{3.969999in}{1.889783in}}{\pgfqpoint{3.974389in}{1.879184in}}{\pgfqpoint{3.982203in}{1.871370in}}%
\pgfpathcurveto{\pgfqpoint{3.990016in}{1.863557in}}{\pgfqpoint{4.000616in}{1.859167in}}{\pgfqpoint{4.011666in}{1.859167in}}%
\pgfpathclose%
\pgfusepath{stroke,fill}%
\end{pgfscope}%
\begin{pgfscope}%
\pgfpathrectangle{\pgfqpoint{0.800000in}{0.528000in}}{\pgfqpoint{4.960000in}{3.696000in}}%
\pgfusepath{clip}%
\pgfsetbuttcap%
\pgfsetroundjoin%
\definecolor{currentfill}{rgb}{0.000000,0.000000,0.000000}%
\pgfsetfillcolor{currentfill}%
\pgfsetlinewidth{1.003750pt}%
\definecolor{currentstroke}{rgb}{0.000000,0.000000,0.000000}%
\pgfsetstrokecolor{currentstroke}%
\pgfsetdash{}{0pt}%
\pgfpathmoveto{\pgfqpoint{4.011666in}{1.816172in}}%
\pgfpathcurveto{\pgfqpoint{4.022716in}{1.816172in}}{\pgfqpoint{4.033315in}{1.820562in}}{\pgfqpoint{4.041128in}{1.828375in}}%
\pgfpathcurveto{\pgfqpoint{4.048942in}{1.836189in}}{\pgfqpoint{4.053332in}{1.846788in}}{\pgfqpoint{4.053332in}{1.857838in}}%
\pgfpathcurveto{\pgfqpoint{4.053332in}{1.868888in}}{\pgfqpoint{4.048942in}{1.879487in}}{\pgfqpoint{4.041128in}{1.887301in}}%
\pgfpathcurveto{\pgfqpoint{4.033315in}{1.895115in}}{\pgfqpoint{4.022716in}{1.899505in}}{\pgfqpoint{4.011666in}{1.899505in}}%
\pgfpathcurveto{\pgfqpoint{4.000616in}{1.899505in}}{\pgfqpoint{3.990016in}{1.895115in}}{\pgfqpoint{3.982203in}{1.887301in}}%
\pgfpathcurveto{\pgfqpoint{3.974389in}{1.879487in}}{\pgfqpoint{3.969999in}{1.868888in}}{\pgfqpoint{3.969999in}{1.857838in}}%
\pgfpathcurveto{\pgfqpoint{3.969999in}{1.846788in}}{\pgfqpoint{3.974389in}{1.836189in}}{\pgfqpoint{3.982203in}{1.828375in}}%
\pgfpathcurveto{\pgfqpoint{3.990016in}{1.820562in}}{\pgfqpoint{4.000616in}{1.816172in}}{\pgfqpoint{4.011666in}{1.816172in}}%
\pgfpathclose%
\pgfusepath{stroke,fill}%
\end{pgfscope}%
\begin{pgfscope}%
\pgfpathrectangle{\pgfqpoint{0.800000in}{0.528000in}}{\pgfqpoint{4.960000in}{3.696000in}}%
\pgfusepath{clip}%
\pgfsetbuttcap%
\pgfsetroundjoin%
\definecolor{currentfill}{rgb}{0.000000,0.000000,0.000000}%
\pgfsetfillcolor{currentfill}%
\pgfsetlinewidth{1.003750pt}%
\definecolor{currentstroke}{rgb}{0.000000,0.000000,0.000000}%
\pgfsetstrokecolor{currentstroke}%
\pgfsetdash{}{0pt}%
\pgfpathmoveto{\pgfqpoint{4.011666in}{1.945157in}}%
\pgfpathcurveto{\pgfqpoint{4.022716in}{1.945157in}}{\pgfqpoint{4.033315in}{1.949547in}}{\pgfqpoint{4.041128in}{1.957360in}}%
\pgfpathcurveto{\pgfqpoint{4.048942in}{1.965174in}}{\pgfqpoint{4.053332in}{1.975773in}}{\pgfqpoint{4.053332in}{1.986823in}}%
\pgfpathcurveto{\pgfqpoint{4.053332in}{1.997873in}}{\pgfqpoint{4.048942in}{2.008472in}}{\pgfqpoint{4.041128in}{2.016286in}}%
\pgfpathcurveto{\pgfqpoint{4.033315in}{2.024100in}}{\pgfqpoint{4.022716in}{2.028490in}}{\pgfqpoint{4.011666in}{2.028490in}}%
\pgfpathcurveto{\pgfqpoint{4.000616in}{2.028490in}}{\pgfqpoint{3.990016in}{2.024100in}}{\pgfqpoint{3.982203in}{2.016286in}}%
\pgfpathcurveto{\pgfqpoint{3.974389in}{2.008472in}}{\pgfqpoint{3.969999in}{1.997873in}}{\pgfqpoint{3.969999in}{1.986823in}}%
\pgfpathcurveto{\pgfqpoint{3.969999in}{1.975773in}}{\pgfqpoint{3.974389in}{1.965174in}}{\pgfqpoint{3.982203in}{1.957360in}}%
\pgfpathcurveto{\pgfqpoint{3.990016in}{1.949547in}}{\pgfqpoint{4.000616in}{1.945157in}}{\pgfqpoint{4.011666in}{1.945157in}}%
\pgfpathclose%
\pgfusepath{stroke,fill}%
\end{pgfscope}%
\begin{pgfscope}%
\pgfpathrectangle{\pgfqpoint{0.800000in}{0.528000in}}{\pgfqpoint{4.960000in}{3.696000in}}%
\pgfusepath{clip}%
\pgfsetbuttcap%
\pgfsetroundjoin%
\definecolor{currentfill}{rgb}{0.000000,0.000000,0.000000}%
\pgfsetfillcolor{currentfill}%
\pgfsetlinewidth{1.003750pt}%
\definecolor{currentstroke}{rgb}{0.000000,0.000000,0.000000}%
\pgfsetstrokecolor{currentstroke}%
\pgfsetdash{}{0pt}%
\pgfpathmoveto{\pgfqpoint{4.011666in}{1.880664in}}%
\pgfpathcurveto{\pgfqpoint{4.022716in}{1.880664in}}{\pgfqpoint{4.033315in}{1.885054in}}{\pgfqpoint{4.041128in}{1.892868in}}%
\pgfpathcurveto{\pgfqpoint{4.048942in}{1.900682in}}{\pgfqpoint{4.053332in}{1.911281in}}{\pgfqpoint{4.053332in}{1.922331in}}%
\pgfpathcurveto{\pgfqpoint{4.053332in}{1.933381in}}{\pgfqpoint{4.048942in}{1.943980in}}{\pgfqpoint{4.041128in}{1.951793in}}%
\pgfpathcurveto{\pgfqpoint{4.033315in}{1.959607in}}{\pgfqpoint{4.022716in}{1.963997in}}{\pgfqpoint{4.011666in}{1.963997in}}%
\pgfpathcurveto{\pgfqpoint{4.000616in}{1.963997in}}{\pgfqpoint{3.990016in}{1.959607in}}{\pgfqpoint{3.982203in}{1.951793in}}%
\pgfpathcurveto{\pgfqpoint{3.974389in}{1.943980in}}{\pgfqpoint{3.969999in}{1.933381in}}{\pgfqpoint{3.969999in}{1.922331in}}%
\pgfpathcurveto{\pgfqpoint{3.969999in}{1.911281in}}{\pgfqpoint{3.974389in}{1.900682in}}{\pgfqpoint{3.982203in}{1.892868in}}%
\pgfpathcurveto{\pgfqpoint{3.990016in}{1.885054in}}{\pgfqpoint{4.000616in}{1.880664in}}{\pgfqpoint{4.011666in}{1.880664in}}%
\pgfpathclose%
\pgfusepath{stroke,fill}%
\end{pgfscope}%
\begin{pgfscope}%
\pgfpathrectangle{\pgfqpoint{0.800000in}{0.528000in}}{\pgfqpoint{4.960000in}{3.696000in}}%
\pgfusepath{clip}%
\pgfsetbuttcap%
\pgfsetroundjoin%
\definecolor{currentfill}{rgb}{0.000000,0.000000,0.000000}%
\pgfsetfillcolor{currentfill}%
\pgfsetlinewidth{1.003750pt}%
\definecolor{currentstroke}{rgb}{0.000000,0.000000,0.000000}%
\pgfsetstrokecolor{currentstroke}%
\pgfsetdash{}{0pt}%
\pgfpathmoveto{\pgfqpoint{4.011666in}{1.859167in}}%
\pgfpathcurveto{\pgfqpoint{4.022716in}{1.859167in}}{\pgfqpoint{4.033315in}{1.863557in}}{\pgfqpoint{4.041128in}{1.871370in}}%
\pgfpathcurveto{\pgfqpoint{4.048942in}{1.879184in}}{\pgfqpoint{4.053332in}{1.889783in}}{\pgfqpoint{4.053332in}{1.900833in}}%
\pgfpathcurveto{\pgfqpoint{4.053332in}{1.911883in}}{\pgfqpoint{4.048942in}{1.922482in}}{\pgfqpoint{4.041128in}{1.930296in}}%
\pgfpathcurveto{\pgfqpoint{4.033315in}{1.938110in}}{\pgfqpoint{4.022716in}{1.942500in}}{\pgfqpoint{4.011666in}{1.942500in}}%
\pgfpathcurveto{\pgfqpoint{4.000616in}{1.942500in}}{\pgfqpoint{3.990016in}{1.938110in}}{\pgfqpoint{3.982203in}{1.930296in}}%
\pgfpathcurveto{\pgfqpoint{3.974389in}{1.922482in}}{\pgfqpoint{3.969999in}{1.911883in}}{\pgfqpoint{3.969999in}{1.900833in}}%
\pgfpathcurveto{\pgfqpoint{3.969999in}{1.889783in}}{\pgfqpoint{3.974389in}{1.879184in}}{\pgfqpoint{3.982203in}{1.871370in}}%
\pgfpathcurveto{\pgfqpoint{3.990016in}{1.863557in}}{\pgfqpoint{4.000616in}{1.859167in}}{\pgfqpoint{4.011666in}{1.859167in}}%
\pgfpathclose%
\pgfusepath{stroke,fill}%
\end{pgfscope}%
\begin{pgfscope}%
\pgfpathrectangle{\pgfqpoint{0.800000in}{0.528000in}}{\pgfqpoint{4.960000in}{3.696000in}}%
\pgfusepath{clip}%
\pgfsetbuttcap%
\pgfsetroundjoin%
\definecolor{currentfill}{rgb}{0.000000,0.000000,0.000000}%
\pgfsetfillcolor{currentfill}%
\pgfsetlinewidth{1.003750pt}%
\definecolor{currentstroke}{rgb}{0.000000,0.000000,0.000000}%
\pgfsetstrokecolor{currentstroke}%
\pgfsetdash{}{0pt}%
\pgfpathmoveto{\pgfqpoint{4.011666in}{1.794674in}}%
\pgfpathcurveto{\pgfqpoint{4.022716in}{1.794674in}}{\pgfqpoint{4.033315in}{1.799064in}}{\pgfqpoint{4.041128in}{1.806878in}}%
\pgfpathcurveto{\pgfqpoint{4.048942in}{1.814692in}}{\pgfqpoint{4.053332in}{1.825291in}}{\pgfqpoint{4.053332in}{1.836341in}}%
\pgfpathcurveto{\pgfqpoint{4.053332in}{1.847391in}}{\pgfqpoint{4.048942in}{1.857990in}}{\pgfqpoint{4.041128in}{1.865804in}}%
\pgfpathcurveto{\pgfqpoint{4.033315in}{1.873617in}}{\pgfqpoint{4.022716in}{1.878007in}}{\pgfqpoint{4.011666in}{1.878007in}}%
\pgfpathcurveto{\pgfqpoint{4.000616in}{1.878007in}}{\pgfqpoint{3.990016in}{1.873617in}}{\pgfqpoint{3.982203in}{1.865804in}}%
\pgfpathcurveto{\pgfqpoint{3.974389in}{1.857990in}}{\pgfqpoint{3.969999in}{1.847391in}}{\pgfqpoint{3.969999in}{1.836341in}}%
\pgfpathcurveto{\pgfqpoint{3.969999in}{1.825291in}}{\pgfqpoint{3.974389in}{1.814692in}}{\pgfqpoint{3.982203in}{1.806878in}}%
\pgfpathcurveto{\pgfqpoint{3.990016in}{1.799064in}}{\pgfqpoint{4.000616in}{1.794674in}}{\pgfqpoint{4.011666in}{1.794674in}}%
\pgfpathclose%
\pgfusepath{stroke,fill}%
\end{pgfscope}%
\begin{pgfscope}%
\pgfpathrectangle{\pgfqpoint{0.800000in}{0.528000in}}{\pgfqpoint{4.960000in}{3.696000in}}%
\pgfusepath{clip}%
\pgfsetbuttcap%
\pgfsetroundjoin%
\definecolor{currentfill}{rgb}{0.000000,0.000000,0.000000}%
\pgfsetfillcolor{currentfill}%
\pgfsetlinewidth{1.003750pt}%
\definecolor{currentstroke}{rgb}{0.000000,0.000000,0.000000}%
\pgfsetstrokecolor{currentstroke}%
\pgfsetdash{}{0pt}%
\pgfpathmoveto{\pgfqpoint{4.011666in}{1.923659in}}%
\pgfpathcurveto{\pgfqpoint{4.022716in}{1.923659in}}{\pgfqpoint{4.033315in}{1.928049in}}{\pgfqpoint{4.041128in}{1.935863in}}%
\pgfpathcurveto{\pgfqpoint{4.048942in}{1.943677in}}{\pgfqpoint{4.053332in}{1.954276in}}{\pgfqpoint{4.053332in}{1.965326in}}%
\pgfpathcurveto{\pgfqpoint{4.053332in}{1.976376in}}{\pgfqpoint{4.048942in}{1.986975in}}{\pgfqpoint{4.041128in}{1.994788in}}%
\pgfpathcurveto{\pgfqpoint{4.033315in}{2.002602in}}{\pgfqpoint{4.022716in}{2.006992in}}{\pgfqpoint{4.011666in}{2.006992in}}%
\pgfpathcurveto{\pgfqpoint{4.000616in}{2.006992in}}{\pgfqpoint{3.990016in}{2.002602in}}{\pgfqpoint{3.982203in}{1.994788in}}%
\pgfpathcurveto{\pgfqpoint{3.974389in}{1.986975in}}{\pgfqpoint{3.969999in}{1.976376in}}{\pgfqpoint{3.969999in}{1.965326in}}%
\pgfpathcurveto{\pgfqpoint{3.969999in}{1.954276in}}{\pgfqpoint{3.974389in}{1.943677in}}{\pgfqpoint{3.982203in}{1.935863in}}%
\pgfpathcurveto{\pgfqpoint{3.990016in}{1.928049in}}{\pgfqpoint{4.000616in}{1.923659in}}{\pgfqpoint{4.011666in}{1.923659in}}%
\pgfpathclose%
\pgfusepath{stroke,fill}%
\end{pgfscope}%
\begin{pgfscope}%
\pgfpathrectangle{\pgfqpoint{0.800000in}{0.528000in}}{\pgfqpoint{4.960000in}{3.696000in}}%
\pgfusepath{clip}%
\pgfsetbuttcap%
\pgfsetroundjoin%
\definecolor{currentfill}{rgb}{0.000000,0.000000,0.000000}%
\pgfsetfillcolor{currentfill}%
\pgfsetlinewidth{1.003750pt}%
\definecolor{currentstroke}{rgb}{0.000000,0.000000,0.000000}%
\pgfsetstrokecolor{currentstroke}%
\pgfsetdash{}{0pt}%
\pgfpathmoveto{\pgfqpoint{4.011666in}{1.837669in}}%
\pgfpathcurveto{\pgfqpoint{4.022716in}{1.837669in}}{\pgfqpoint{4.033315in}{1.842059in}}{\pgfqpoint{4.041128in}{1.849873in}}%
\pgfpathcurveto{\pgfqpoint{4.048942in}{1.857687in}}{\pgfqpoint{4.053332in}{1.868286in}}{\pgfqpoint{4.053332in}{1.879336in}}%
\pgfpathcurveto{\pgfqpoint{4.053332in}{1.890386in}}{\pgfqpoint{4.048942in}{1.900985in}}{\pgfqpoint{4.041128in}{1.908798in}}%
\pgfpathcurveto{\pgfqpoint{4.033315in}{1.916612in}}{\pgfqpoint{4.022716in}{1.921002in}}{\pgfqpoint{4.011666in}{1.921002in}}%
\pgfpathcurveto{\pgfqpoint{4.000616in}{1.921002in}}{\pgfqpoint{3.990016in}{1.916612in}}{\pgfqpoint{3.982203in}{1.908798in}}%
\pgfpathcurveto{\pgfqpoint{3.974389in}{1.900985in}}{\pgfqpoint{3.969999in}{1.890386in}}{\pgfqpoint{3.969999in}{1.879336in}}%
\pgfpathcurveto{\pgfqpoint{3.969999in}{1.868286in}}{\pgfqpoint{3.974389in}{1.857687in}}{\pgfqpoint{3.982203in}{1.849873in}}%
\pgfpathcurveto{\pgfqpoint{3.990016in}{1.842059in}}{\pgfqpoint{4.000616in}{1.837669in}}{\pgfqpoint{4.011666in}{1.837669in}}%
\pgfpathclose%
\pgfusepath{stroke,fill}%
\end{pgfscope}%
\begin{pgfscope}%
\pgfpathrectangle{\pgfqpoint{0.800000in}{0.528000in}}{\pgfqpoint{4.960000in}{3.696000in}}%
\pgfusepath{clip}%
\pgfsetbuttcap%
\pgfsetroundjoin%
\definecolor{currentfill}{rgb}{0.000000,0.000000,0.000000}%
\pgfsetfillcolor{currentfill}%
\pgfsetlinewidth{1.003750pt}%
\definecolor{currentstroke}{rgb}{0.000000,0.000000,0.000000}%
\pgfsetstrokecolor{currentstroke}%
\pgfsetdash{}{0pt}%
\pgfpathmoveto{\pgfqpoint{4.011666in}{1.880664in}}%
\pgfpathcurveto{\pgfqpoint{4.022716in}{1.880664in}}{\pgfqpoint{4.033315in}{1.885054in}}{\pgfqpoint{4.041128in}{1.892868in}}%
\pgfpathcurveto{\pgfqpoint{4.048942in}{1.900682in}}{\pgfqpoint{4.053332in}{1.911281in}}{\pgfqpoint{4.053332in}{1.922331in}}%
\pgfpathcurveto{\pgfqpoint{4.053332in}{1.933381in}}{\pgfqpoint{4.048942in}{1.943980in}}{\pgfqpoint{4.041128in}{1.951793in}}%
\pgfpathcurveto{\pgfqpoint{4.033315in}{1.959607in}}{\pgfqpoint{4.022716in}{1.963997in}}{\pgfqpoint{4.011666in}{1.963997in}}%
\pgfpathcurveto{\pgfqpoint{4.000616in}{1.963997in}}{\pgfqpoint{3.990016in}{1.959607in}}{\pgfqpoint{3.982203in}{1.951793in}}%
\pgfpathcurveto{\pgfqpoint{3.974389in}{1.943980in}}{\pgfqpoint{3.969999in}{1.933381in}}{\pgfqpoint{3.969999in}{1.922331in}}%
\pgfpathcurveto{\pgfqpoint{3.969999in}{1.911281in}}{\pgfqpoint{3.974389in}{1.900682in}}{\pgfqpoint{3.982203in}{1.892868in}}%
\pgfpathcurveto{\pgfqpoint{3.990016in}{1.885054in}}{\pgfqpoint{4.000616in}{1.880664in}}{\pgfqpoint{4.011666in}{1.880664in}}%
\pgfpathclose%
\pgfusepath{stroke,fill}%
\end{pgfscope}%
\begin{pgfscope}%
\pgfpathrectangle{\pgfqpoint{0.800000in}{0.528000in}}{\pgfqpoint{4.960000in}{3.696000in}}%
\pgfusepath{clip}%
\pgfsetbuttcap%
\pgfsetroundjoin%
\definecolor{currentfill}{rgb}{0.000000,0.000000,0.000000}%
\pgfsetfillcolor{currentfill}%
\pgfsetlinewidth{1.003750pt}%
\definecolor{currentstroke}{rgb}{0.000000,0.000000,0.000000}%
\pgfsetstrokecolor{currentstroke}%
\pgfsetdash{}{0pt}%
\pgfpathmoveto{\pgfqpoint{4.011666in}{1.859167in}}%
\pgfpathcurveto{\pgfqpoint{4.022716in}{1.859167in}}{\pgfqpoint{4.033315in}{1.863557in}}{\pgfqpoint{4.041128in}{1.871370in}}%
\pgfpathcurveto{\pgfqpoint{4.048942in}{1.879184in}}{\pgfqpoint{4.053332in}{1.889783in}}{\pgfqpoint{4.053332in}{1.900833in}}%
\pgfpathcurveto{\pgfqpoint{4.053332in}{1.911883in}}{\pgfqpoint{4.048942in}{1.922482in}}{\pgfqpoint{4.041128in}{1.930296in}}%
\pgfpathcurveto{\pgfqpoint{4.033315in}{1.938110in}}{\pgfqpoint{4.022716in}{1.942500in}}{\pgfqpoint{4.011666in}{1.942500in}}%
\pgfpathcurveto{\pgfqpoint{4.000616in}{1.942500in}}{\pgfqpoint{3.990016in}{1.938110in}}{\pgfqpoint{3.982203in}{1.930296in}}%
\pgfpathcurveto{\pgfqpoint{3.974389in}{1.922482in}}{\pgfqpoint{3.969999in}{1.911883in}}{\pgfqpoint{3.969999in}{1.900833in}}%
\pgfpathcurveto{\pgfqpoint{3.969999in}{1.889783in}}{\pgfqpoint{3.974389in}{1.879184in}}{\pgfqpoint{3.982203in}{1.871370in}}%
\pgfpathcurveto{\pgfqpoint{3.990016in}{1.863557in}}{\pgfqpoint{4.000616in}{1.859167in}}{\pgfqpoint{4.011666in}{1.859167in}}%
\pgfpathclose%
\pgfusepath{stroke,fill}%
\end{pgfscope}%
\begin{pgfscope}%
\pgfpathrectangle{\pgfqpoint{0.800000in}{0.528000in}}{\pgfqpoint{4.960000in}{3.696000in}}%
\pgfusepath{clip}%
\pgfsetbuttcap%
\pgfsetroundjoin%
\definecolor{currentfill}{rgb}{0.000000,0.000000,0.000000}%
\pgfsetfillcolor{currentfill}%
\pgfsetlinewidth{1.003750pt}%
\definecolor{currentstroke}{rgb}{0.000000,0.000000,0.000000}%
\pgfsetstrokecolor{currentstroke}%
\pgfsetdash{}{0pt}%
\pgfpathmoveto{\pgfqpoint{4.011666in}{1.794674in}}%
\pgfpathcurveto{\pgfqpoint{4.022716in}{1.794674in}}{\pgfqpoint{4.033315in}{1.799064in}}{\pgfqpoint{4.041128in}{1.806878in}}%
\pgfpathcurveto{\pgfqpoint{4.048942in}{1.814692in}}{\pgfqpoint{4.053332in}{1.825291in}}{\pgfqpoint{4.053332in}{1.836341in}}%
\pgfpathcurveto{\pgfqpoint{4.053332in}{1.847391in}}{\pgfqpoint{4.048942in}{1.857990in}}{\pgfqpoint{4.041128in}{1.865804in}}%
\pgfpathcurveto{\pgfqpoint{4.033315in}{1.873617in}}{\pgfqpoint{4.022716in}{1.878007in}}{\pgfqpoint{4.011666in}{1.878007in}}%
\pgfpathcurveto{\pgfqpoint{4.000616in}{1.878007in}}{\pgfqpoint{3.990016in}{1.873617in}}{\pgfqpoint{3.982203in}{1.865804in}}%
\pgfpathcurveto{\pgfqpoint{3.974389in}{1.857990in}}{\pgfqpoint{3.969999in}{1.847391in}}{\pgfqpoint{3.969999in}{1.836341in}}%
\pgfpathcurveto{\pgfqpoint{3.969999in}{1.825291in}}{\pgfqpoint{3.974389in}{1.814692in}}{\pgfqpoint{3.982203in}{1.806878in}}%
\pgfpathcurveto{\pgfqpoint{3.990016in}{1.799064in}}{\pgfqpoint{4.000616in}{1.794674in}}{\pgfqpoint{4.011666in}{1.794674in}}%
\pgfpathclose%
\pgfusepath{stroke,fill}%
\end{pgfscope}%
\begin{pgfscope}%
\pgfpathrectangle{\pgfqpoint{0.800000in}{0.528000in}}{\pgfqpoint{4.960000in}{3.696000in}}%
\pgfusepath{clip}%
\pgfsetbuttcap%
\pgfsetroundjoin%
\definecolor{currentfill}{rgb}{0.000000,0.000000,0.000000}%
\pgfsetfillcolor{currentfill}%
\pgfsetlinewidth{1.003750pt}%
\definecolor{currentstroke}{rgb}{0.000000,0.000000,0.000000}%
\pgfsetstrokecolor{currentstroke}%
\pgfsetdash{}{0pt}%
\pgfpathmoveto{\pgfqpoint{4.011666in}{1.794674in}}%
\pgfpathcurveto{\pgfqpoint{4.022716in}{1.794674in}}{\pgfqpoint{4.033315in}{1.799064in}}{\pgfqpoint{4.041128in}{1.806878in}}%
\pgfpathcurveto{\pgfqpoint{4.048942in}{1.814692in}}{\pgfqpoint{4.053332in}{1.825291in}}{\pgfqpoint{4.053332in}{1.836341in}}%
\pgfpathcurveto{\pgfqpoint{4.053332in}{1.847391in}}{\pgfqpoint{4.048942in}{1.857990in}}{\pgfqpoint{4.041128in}{1.865804in}}%
\pgfpathcurveto{\pgfqpoint{4.033315in}{1.873617in}}{\pgfqpoint{4.022716in}{1.878007in}}{\pgfqpoint{4.011666in}{1.878007in}}%
\pgfpathcurveto{\pgfqpoint{4.000616in}{1.878007in}}{\pgfqpoint{3.990016in}{1.873617in}}{\pgfqpoint{3.982203in}{1.865804in}}%
\pgfpathcurveto{\pgfqpoint{3.974389in}{1.857990in}}{\pgfqpoint{3.969999in}{1.847391in}}{\pgfqpoint{3.969999in}{1.836341in}}%
\pgfpathcurveto{\pgfqpoint{3.969999in}{1.825291in}}{\pgfqpoint{3.974389in}{1.814692in}}{\pgfqpoint{3.982203in}{1.806878in}}%
\pgfpathcurveto{\pgfqpoint{3.990016in}{1.799064in}}{\pgfqpoint{4.000616in}{1.794674in}}{\pgfqpoint{4.011666in}{1.794674in}}%
\pgfpathclose%
\pgfusepath{stroke,fill}%
\end{pgfscope}%
\begin{pgfscope}%
\pgfpathrectangle{\pgfqpoint{0.800000in}{0.528000in}}{\pgfqpoint{4.960000in}{3.696000in}}%
\pgfusepath{clip}%
\pgfsetbuttcap%
\pgfsetroundjoin%
\definecolor{currentfill}{rgb}{0.000000,0.000000,0.000000}%
\pgfsetfillcolor{currentfill}%
\pgfsetlinewidth{1.003750pt}%
\definecolor{currentstroke}{rgb}{0.000000,0.000000,0.000000}%
\pgfsetstrokecolor{currentstroke}%
\pgfsetdash{}{0pt}%
\pgfpathmoveto{\pgfqpoint{4.011666in}{1.945157in}}%
\pgfpathcurveto{\pgfqpoint{4.022716in}{1.945157in}}{\pgfqpoint{4.033315in}{1.949547in}}{\pgfqpoint{4.041128in}{1.957360in}}%
\pgfpathcurveto{\pgfqpoint{4.048942in}{1.965174in}}{\pgfqpoint{4.053332in}{1.975773in}}{\pgfqpoint{4.053332in}{1.986823in}}%
\pgfpathcurveto{\pgfqpoint{4.053332in}{1.997873in}}{\pgfqpoint{4.048942in}{2.008472in}}{\pgfqpoint{4.041128in}{2.016286in}}%
\pgfpathcurveto{\pgfqpoint{4.033315in}{2.024100in}}{\pgfqpoint{4.022716in}{2.028490in}}{\pgfqpoint{4.011666in}{2.028490in}}%
\pgfpathcurveto{\pgfqpoint{4.000616in}{2.028490in}}{\pgfqpoint{3.990016in}{2.024100in}}{\pgfqpoint{3.982203in}{2.016286in}}%
\pgfpathcurveto{\pgfqpoint{3.974389in}{2.008472in}}{\pgfqpoint{3.969999in}{1.997873in}}{\pgfqpoint{3.969999in}{1.986823in}}%
\pgfpathcurveto{\pgfqpoint{3.969999in}{1.975773in}}{\pgfqpoint{3.974389in}{1.965174in}}{\pgfqpoint{3.982203in}{1.957360in}}%
\pgfpathcurveto{\pgfqpoint{3.990016in}{1.949547in}}{\pgfqpoint{4.000616in}{1.945157in}}{\pgfqpoint{4.011666in}{1.945157in}}%
\pgfpathclose%
\pgfusepath{stroke,fill}%
\end{pgfscope}%
\begin{pgfscope}%
\pgfpathrectangle{\pgfqpoint{0.800000in}{0.528000in}}{\pgfqpoint{4.960000in}{3.696000in}}%
\pgfusepath{clip}%
\pgfsetbuttcap%
\pgfsetroundjoin%
\definecolor{currentfill}{rgb}{0.000000,0.000000,0.000000}%
\pgfsetfillcolor{currentfill}%
\pgfsetlinewidth{1.003750pt}%
\definecolor{currentstroke}{rgb}{0.000000,0.000000,0.000000}%
\pgfsetstrokecolor{currentstroke}%
\pgfsetdash{}{0pt}%
\pgfpathmoveto{\pgfqpoint{4.011666in}{1.880664in}}%
\pgfpathcurveto{\pgfqpoint{4.022716in}{1.880664in}}{\pgfqpoint{4.033315in}{1.885054in}}{\pgfqpoint{4.041128in}{1.892868in}}%
\pgfpathcurveto{\pgfqpoint{4.048942in}{1.900682in}}{\pgfqpoint{4.053332in}{1.911281in}}{\pgfqpoint{4.053332in}{1.922331in}}%
\pgfpathcurveto{\pgfqpoint{4.053332in}{1.933381in}}{\pgfqpoint{4.048942in}{1.943980in}}{\pgfqpoint{4.041128in}{1.951793in}}%
\pgfpathcurveto{\pgfqpoint{4.033315in}{1.959607in}}{\pgfqpoint{4.022716in}{1.963997in}}{\pgfqpoint{4.011666in}{1.963997in}}%
\pgfpathcurveto{\pgfqpoint{4.000616in}{1.963997in}}{\pgfqpoint{3.990016in}{1.959607in}}{\pgfqpoint{3.982203in}{1.951793in}}%
\pgfpathcurveto{\pgfqpoint{3.974389in}{1.943980in}}{\pgfqpoint{3.969999in}{1.933381in}}{\pgfqpoint{3.969999in}{1.922331in}}%
\pgfpathcurveto{\pgfqpoint{3.969999in}{1.911281in}}{\pgfqpoint{3.974389in}{1.900682in}}{\pgfqpoint{3.982203in}{1.892868in}}%
\pgfpathcurveto{\pgfqpoint{3.990016in}{1.885054in}}{\pgfqpoint{4.000616in}{1.880664in}}{\pgfqpoint{4.011666in}{1.880664in}}%
\pgfpathclose%
\pgfusepath{stroke,fill}%
\end{pgfscope}%
\begin{pgfscope}%
\pgfpathrectangle{\pgfqpoint{0.800000in}{0.528000in}}{\pgfqpoint{4.960000in}{3.696000in}}%
\pgfusepath{clip}%
\pgfsetbuttcap%
\pgfsetroundjoin%
\definecolor{currentfill}{rgb}{0.000000,0.000000,0.000000}%
\pgfsetfillcolor{currentfill}%
\pgfsetlinewidth{1.003750pt}%
\definecolor{currentstroke}{rgb}{0.000000,0.000000,0.000000}%
\pgfsetstrokecolor{currentstroke}%
\pgfsetdash{}{0pt}%
\pgfpathmoveto{\pgfqpoint{4.011666in}{1.880664in}}%
\pgfpathcurveto{\pgfqpoint{4.022716in}{1.880664in}}{\pgfqpoint{4.033315in}{1.885054in}}{\pgfqpoint{4.041128in}{1.892868in}}%
\pgfpathcurveto{\pgfqpoint{4.048942in}{1.900682in}}{\pgfqpoint{4.053332in}{1.911281in}}{\pgfqpoint{4.053332in}{1.922331in}}%
\pgfpathcurveto{\pgfqpoint{4.053332in}{1.933381in}}{\pgfqpoint{4.048942in}{1.943980in}}{\pgfqpoint{4.041128in}{1.951793in}}%
\pgfpathcurveto{\pgfqpoint{4.033315in}{1.959607in}}{\pgfqpoint{4.022716in}{1.963997in}}{\pgfqpoint{4.011666in}{1.963997in}}%
\pgfpathcurveto{\pgfqpoint{4.000616in}{1.963997in}}{\pgfqpoint{3.990016in}{1.959607in}}{\pgfqpoint{3.982203in}{1.951793in}}%
\pgfpathcurveto{\pgfqpoint{3.974389in}{1.943980in}}{\pgfqpoint{3.969999in}{1.933381in}}{\pgfqpoint{3.969999in}{1.922331in}}%
\pgfpathcurveto{\pgfqpoint{3.969999in}{1.911281in}}{\pgfqpoint{3.974389in}{1.900682in}}{\pgfqpoint{3.982203in}{1.892868in}}%
\pgfpathcurveto{\pgfqpoint{3.990016in}{1.885054in}}{\pgfqpoint{4.000616in}{1.880664in}}{\pgfqpoint{4.011666in}{1.880664in}}%
\pgfpathclose%
\pgfusepath{stroke,fill}%
\end{pgfscope}%
\begin{pgfscope}%
\pgfpathrectangle{\pgfqpoint{0.800000in}{0.528000in}}{\pgfqpoint{4.960000in}{3.696000in}}%
\pgfusepath{clip}%
\pgfsetbuttcap%
\pgfsetroundjoin%
\definecolor{currentfill}{rgb}{0.000000,0.000000,0.000000}%
\pgfsetfillcolor{currentfill}%
\pgfsetlinewidth{1.003750pt}%
\definecolor{currentstroke}{rgb}{0.000000,0.000000,0.000000}%
\pgfsetstrokecolor{currentstroke}%
\pgfsetdash{}{0pt}%
\pgfpathmoveto{\pgfqpoint{4.011666in}{1.773177in}}%
\pgfpathcurveto{\pgfqpoint{4.022716in}{1.773177in}}{\pgfqpoint{4.033315in}{1.777567in}}{\pgfqpoint{4.041128in}{1.785380in}}%
\pgfpathcurveto{\pgfqpoint{4.048942in}{1.793194in}}{\pgfqpoint{4.053332in}{1.803793in}}{\pgfqpoint{4.053332in}{1.814843in}}%
\pgfpathcurveto{\pgfqpoint{4.053332in}{1.825893in}}{\pgfqpoint{4.048942in}{1.836492in}}{\pgfqpoint{4.041128in}{1.844306in}}%
\pgfpathcurveto{\pgfqpoint{4.033315in}{1.852120in}}{\pgfqpoint{4.022716in}{1.856510in}}{\pgfqpoint{4.011666in}{1.856510in}}%
\pgfpathcurveto{\pgfqpoint{4.000616in}{1.856510in}}{\pgfqpoint{3.990016in}{1.852120in}}{\pgfqpoint{3.982203in}{1.844306in}}%
\pgfpathcurveto{\pgfqpoint{3.974389in}{1.836492in}}{\pgfqpoint{3.969999in}{1.825893in}}{\pgfqpoint{3.969999in}{1.814843in}}%
\pgfpathcurveto{\pgfqpoint{3.969999in}{1.803793in}}{\pgfqpoint{3.974389in}{1.793194in}}{\pgfqpoint{3.982203in}{1.785380in}}%
\pgfpathcurveto{\pgfqpoint{3.990016in}{1.777567in}}{\pgfqpoint{4.000616in}{1.773177in}}{\pgfqpoint{4.011666in}{1.773177in}}%
\pgfpathclose%
\pgfusepath{stroke,fill}%
\end{pgfscope}%
\begin{pgfscope}%
\pgfpathrectangle{\pgfqpoint{0.800000in}{0.528000in}}{\pgfqpoint{4.960000in}{3.696000in}}%
\pgfusepath{clip}%
\pgfsetbuttcap%
\pgfsetroundjoin%
\definecolor{currentfill}{rgb}{0.000000,0.000000,0.000000}%
\pgfsetfillcolor{currentfill}%
\pgfsetlinewidth{1.003750pt}%
\definecolor{currentstroke}{rgb}{0.000000,0.000000,0.000000}%
\pgfsetstrokecolor{currentstroke}%
\pgfsetdash{}{0pt}%
\pgfpathmoveto{\pgfqpoint{4.011666in}{1.794674in}}%
\pgfpathcurveto{\pgfqpoint{4.022716in}{1.794674in}}{\pgfqpoint{4.033315in}{1.799064in}}{\pgfqpoint{4.041128in}{1.806878in}}%
\pgfpathcurveto{\pgfqpoint{4.048942in}{1.814692in}}{\pgfqpoint{4.053332in}{1.825291in}}{\pgfqpoint{4.053332in}{1.836341in}}%
\pgfpathcurveto{\pgfqpoint{4.053332in}{1.847391in}}{\pgfqpoint{4.048942in}{1.857990in}}{\pgfqpoint{4.041128in}{1.865804in}}%
\pgfpathcurveto{\pgfqpoint{4.033315in}{1.873617in}}{\pgfqpoint{4.022716in}{1.878007in}}{\pgfqpoint{4.011666in}{1.878007in}}%
\pgfpathcurveto{\pgfqpoint{4.000616in}{1.878007in}}{\pgfqpoint{3.990016in}{1.873617in}}{\pgfqpoint{3.982203in}{1.865804in}}%
\pgfpathcurveto{\pgfqpoint{3.974389in}{1.857990in}}{\pgfqpoint{3.969999in}{1.847391in}}{\pgfqpoint{3.969999in}{1.836341in}}%
\pgfpathcurveto{\pgfqpoint{3.969999in}{1.825291in}}{\pgfqpoint{3.974389in}{1.814692in}}{\pgfqpoint{3.982203in}{1.806878in}}%
\pgfpathcurveto{\pgfqpoint{3.990016in}{1.799064in}}{\pgfqpoint{4.000616in}{1.794674in}}{\pgfqpoint{4.011666in}{1.794674in}}%
\pgfpathclose%
\pgfusepath{stroke,fill}%
\end{pgfscope}%
\begin{pgfscope}%
\pgfpathrectangle{\pgfqpoint{0.800000in}{0.528000in}}{\pgfqpoint{4.960000in}{3.696000in}}%
\pgfusepath{clip}%
\pgfsetbuttcap%
\pgfsetroundjoin%
\definecolor{currentfill}{rgb}{0.000000,0.000000,0.000000}%
\pgfsetfillcolor{currentfill}%
\pgfsetlinewidth{1.003750pt}%
\definecolor{currentstroke}{rgb}{0.000000,0.000000,0.000000}%
\pgfsetstrokecolor{currentstroke}%
\pgfsetdash{}{0pt}%
\pgfpathmoveto{\pgfqpoint{4.011666in}{1.880664in}}%
\pgfpathcurveto{\pgfqpoint{4.022716in}{1.880664in}}{\pgfqpoint{4.033315in}{1.885054in}}{\pgfqpoint{4.041128in}{1.892868in}}%
\pgfpathcurveto{\pgfqpoint{4.048942in}{1.900682in}}{\pgfqpoint{4.053332in}{1.911281in}}{\pgfqpoint{4.053332in}{1.922331in}}%
\pgfpathcurveto{\pgfqpoint{4.053332in}{1.933381in}}{\pgfqpoint{4.048942in}{1.943980in}}{\pgfqpoint{4.041128in}{1.951793in}}%
\pgfpathcurveto{\pgfqpoint{4.033315in}{1.959607in}}{\pgfqpoint{4.022716in}{1.963997in}}{\pgfqpoint{4.011666in}{1.963997in}}%
\pgfpathcurveto{\pgfqpoint{4.000616in}{1.963997in}}{\pgfqpoint{3.990016in}{1.959607in}}{\pgfqpoint{3.982203in}{1.951793in}}%
\pgfpathcurveto{\pgfqpoint{3.974389in}{1.943980in}}{\pgfqpoint{3.969999in}{1.933381in}}{\pgfqpoint{3.969999in}{1.922331in}}%
\pgfpathcurveto{\pgfqpoint{3.969999in}{1.911281in}}{\pgfqpoint{3.974389in}{1.900682in}}{\pgfqpoint{3.982203in}{1.892868in}}%
\pgfpathcurveto{\pgfqpoint{3.990016in}{1.885054in}}{\pgfqpoint{4.000616in}{1.880664in}}{\pgfqpoint{4.011666in}{1.880664in}}%
\pgfpathclose%
\pgfusepath{stroke,fill}%
\end{pgfscope}%
\begin{pgfscope}%
\pgfpathrectangle{\pgfqpoint{0.800000in}{0.528000in}}{\pgfqpoint{4.960000in}{3.696000in}}%
\pgfusepath{clip}%
\pgfsetbuttcap%
\pgfsetroundjoin%
\definecolor{currentfill}{rgb}{0.000000,0.000000,0.000000}%
\pgfsetfillcolor{currentfill}%
\pgfsetlinewidth{1.003750pt}%
\definecolor{currentstroke}{rgb}{0.000000,0.000000,0.000000}%
\pgfsetstrokecolor{currentstroke}%
\pgfsetdash{}{0pt}%
\pgfpathmoveto{\pgfqpoint{4.011666in}{1.773177in}}%
\pgfpathcurveto{\pgfqpoint{4.022716in}{1.773177in}}{\pgfqpoint{4.033315in}{1.777567in}}{\pgfqpoint{4.041128in}{1.785380in}}%
\pgfpathcurveto{\pgfqpoint{4.048942in}{1.793194in}}{\pgfqpoint{4.053332in}{1.803793in}}{\pgfqpoint{4.053332in}{1.814843in}}%
\pgfpathcurveto{\pgfqpoint{4.053332in}{1.825893in}}{\pgfqpoint{4.048942in}{1.836492in}}{\pgfqpoint{4.041128in}{1.844306in}}%
\pgfpathcurveto{\pgfqpoint{4.033315in}{1.852120in}}{\pgfqpoint{4.022716in}{1.856510in}}{\pgfqpoint{4.011666in}{1.856510in}}%
\pgfpathcurveto{\pgfqpoint{4.000616in}{1.856510in}}{\pgfqpoint{3.990016in}{1.852120in}}{\pgfqpoint{3.982203in}{1.844306in}}%
\pgfpathcurveto{\pgfqpoint{3.974389in}{1.836492in}}{\pgfqpoint{3.969999in}{1.825893in}}{\pgfqpoint{3.969999in}{1.814843in}}%
\pgfpathcurveto{\pgfqpoint{3.969999in}{1.803793in}}{\pgfqpoint{3.974389in}{1.793194in}}{\pgfqpoint{3.982203in}{1.785380in}}%
\pgfpathcurveto{\pgfqpoint{3.990016in}{1.777567in}}{\pgfqpoint{4.000616in}{1.773177in}}{\pgfqpoint{4.011666in}{1.773177in}}%
\pgfpathclose%
\pgfusepath{stroke,fill}%
\end{pgfscope}%
\begin{pgfscope}%
\pgfpathrectangle{\pgfqpoint{0.800000in}{0.528000in}}{\pgfqpoint{4.960000in}{3.696000in}}%
\pgfusepath{clip}%
\pgfsetbuttcap%
\pgfsetroundjoin%
\definecolor{currentfill}{rgb}{0.000000,0.000000,0.000000}%
\pgfsetfillcolor{currentfill}%
\pgfsetlinewidth{1.003750pt}%
\definecolor{currentstroke}{rgb}{0.000000,0.000000,0.000000}%
\pgfsetstrokecolor{currentstroke}%
\pgfsetdash{}{0pt}%
\pgfpathmoveto{\pgfqpoint{4.011666in}{1.816172in}}%
\pgfpathcurveto{\pgfqpoint{4.022716in}{1.816172in}}{\pgfqpoint{4.033315in}{1.820562in}}{\pgfqpoint{4.041128in}{1.828375in}}%
\pgfpathcurveto{\pgfqpoint{4.048942in}{1.836189in}}{\pgfqpoint{4.053332in}{1.846788in}}{\pgfqpoint{4.053332in}{1.857838in}}%
\pgfpathcurveto{\pgfqpoint{4.053332in}{1.868888in}}{\pgfqpoint{4.048942in}{1.879487in}}{\pgfqpoint{4.041128in}{1.887301in}}%
\pgfpathcurveto{\pgfqpoint{4.033315in}{1.895115in}}{\pgfqpoint{4.022716in}{1.899505in}}{\pgfqpoint{4.011666in}{1.899505in}}%
\pgfpathcurveto{\pgfqpoint{4.000616in}{1.899505in}}{\pgfqpoint{3.990016in}{1.895115in}}{\pgfqpoint{3.982203in}{1.887301in}}%
\pgfpathcurveto{\pgfqpoint{3.974389in}{1.879487in}}{\pgfqpoint{3.969999in}{1.868888in}}{\pgfqpoint{3.969999in}{1.857838in}}%
\pgfpathcurveto{\pgfqpoint{3.969999in}{1.846788in}}{\pgfqpoint{3.974389in}{1.836189in}}{\pgfqpoint{3.982203in}{1.828375in}}%
\pgfpathcurveto{\pgfqpoint{3.990016in}{1.820562in}}{\pgfqpoint{4.000616in}{1.816172in}}{\pgfqpoint{4.011666in}{1.816172in}}%
\pgfpathclose%
\pgfusepath{stroke,fill}%
\end{pgfscope}%
\begin{pgfscope}%
\pgfpathrectangle{\pgfqpoint{0.800000in}{0.528000in}}{\pgfqpoint{4.960000in}{3.696000in}}%
\pgfusepath{clip}%
\pgfsetbuttcap%
\pgfsetroundjoin%
\definecolor{currentfill}{rgb}{0.000000,0.000000,0.000000}%
\pgfsetfillcolor{currentfill}%
\pgfsetlinewidth{1.003750pt}%
\definecolor{currentstroke}{rgb}{0.000000,0.000000,0.000000}%
\pgfsetstrokecolor{currentstroke}%
\pgfsetdash{}{0pt}%
\pgfpathmoveto{\pgfqpoint{4.011666in}{1.988151in}}%
\pgfpathcurveto{\pgfqpoint{4.022716in}{1.988151in}}{\pgfqpoint{4.033315in}{1.992542in}}{\pgfqpoint{4.041128in}{2.000355in}}%
\pgfpathcurveto{\pgfqpoint{4.048942in}{2.008169in}}{\pgfqpoint{4.053332in}{2.018768in}}{\pgfqpoint{4.053332in}{2.029818in}}%
\pgfpathcurveto{\pgfqpoint{4.053332in}{2.040868in}}{\pgfqpoint{4.048942in}{2.051467in}}{\pgfqpoint{4.041128in}{2.059281in}}%
\pgfpathcurveto{\pgfqpoint{4.033315in}{2.067095in}}{\pgfqpoint{4.022716in}{2.071485in}}{\pgfqpoint{4.011666in}{2.071485in}}%
\pgfpathcurveto{\pgfqpoint{4.000616in}{2.071485in}}{\pgfqpoint{3.990016in}{2.067095in}}{\pgfqpoint{3.982203in}{2.059281in}}%
\pgfpathcurveto{\pgfqpoint{3.974389in}{2.051467in}}{\pgfqpoint{3.969999in}{2.040868in}}{\pgfqpoint{3.969999in}{2.029818in}}%
\pgfpathcurveto{\pgfqpoint{3.969999in}{2.018768in}}{\pgfqpoint{3.974389in}{2.008169in}}{\pgfqpoint{3.982203in}{2.000355in}}%
\pgfpathcurveto{\pgfqpoint{3.990016in}{1.992542in}}{\pgfqpoint{4.000616in}{1.988151in}}{\pgfqpoint{4.011666in}{1.988151in}}%
\pgfpathclose%
\pgfusepath{stroke,fill}%
\end{pgfscope}%
\begin{pgfscope}%
\pgfpathrectangle{\pgfqpoint{0.800000in}{0.528000in}}{\pgfqpoint{4.960000in}{3.696000in}}%
\pgfusepath{clip}%
\pgfsetbuttcap%
\pgfsetroundjoin%
\definecolor{currentfill}{rgb}{0.000000,0.000000,0.000000}%
\pgfsetfillcolor{currentfill}%
\pgfsetlinewidth{1.003750pt}%
\definecolor{currentstroke}{rgb}{0.000000,0.000000,0.000000}%
\pgfsetstrokecolor{currentstroke}%
\pgfsetdash{}{0pt}%
\pgfpathmoveto{\pgfqpoint{4.011666in}{2.052644in}}%
\pgfpathcurveto{\pgfqpoint{4.022716in}{2.052644in}}{\pgfqpoint{4.033315in}{2.057034in}}{\pgfqpoint{4.041128in}{2.064848in}}%
\pgfpathcurveto{\pgfqpoint{4.048942in}{2.072661in}}{\pgfqpoint{4.053332in}{2.083261in}}{\pgfqpoint{4.053332in}{2.094311in}}%
\pgfpathcurveto{\pgfqpoint{4.053332in}{2.105361in}}{\pgfqpoint{4.048942in}{2.115960in}}{\pgfqpoint{4.041128in}{2.123773in}}%
\pgfpathcurveto{\pgfqpoint{4.033315in}{2.131587in}}{\pgfqpoint{4.022716in}{2.135977in}}{\pgfqpoint{4.011666in}{2.135977in}}%
\pgfpathcurveto{\pgfqpoint{4.000616in}{2.135977in}}{\pgfqpoint{3.990016in}{2.131587in}}{\pgfqpoint{3.982203in}{2.123773in}}%
\pgfpathcurveto{\pgfqpoint{3.974389in}{2.115960in}}{\pgfqpoint{3.969999in}{2.105361in}}{\pgfqpoint{3.969999in}{2.094311in}}%
\pgfpathcurveto{\pgfqpoint{3.969999in}{2.083261in}}{\pgfqpoint{3.974389in}{2.072661in}}{\pgfqpoint{3.982203in}{2.064848in}}%
\pgfpathcurveto{\pgfqpoint{3.990016in}{2.057034in}}{\pgfqpoint{4.000616in}{2.052644in}}{\pgfqpoint{4.011666in}{2.052644in}}%
\pgfpathclose%
\pgfusepath{stroke,fill}%
\end{pgfscope}%
\begin{pgfscope}%
\pgfpathrectangle{\pgfqpoint{0.800000in}{0.528000in}}{\pgfqpoint{4.960000in}{3.696000in}}%
\pgfusepath{clip}%
\pgfsetbuttcap%
\pgfsetroundjoin%
\definecolor{currentfill}{rgb}{0.000000,0.000000,0.000000}%
\pgfsetfillcolor{currentfill}%
\pgfsetlinewidth{1.003750pt}%
\definecolor{currentstroke}{rgb}{0.000000,0.000000,0.000000}%
\pgfsetstrokecolor{currentstroke}%
\pgfsetdash{}{0pt}%
\pgfpathmoveto{\pgfqpoint{4.011666in}{1.837669in}}%
\pgfpathcurveto{\pgfqpoint{4.022716in}{1.837669in}}{\pgfqpoint{4.033315in}{1.842059in}}{\pgfqpoint{4.041128in}{1.849873in}}%
\pgfpathcurveto{\pgfqpoint{4.048942in}{1.857687in}}{\pgfqpoint{4.053332in}{1.868286in}}{\pgfqpoint{4.053332in}{1.879336in}}%
\pgfpathcurveto{\pgfqpoint{4.053332in}{1.890386in}}{\pgfqpoint{4.048942in}{1.900985in}}{\pgfqpoint{4.041128in}{1.908798in}}%
\pgfpathcurveto{\pgfqpoint{4.033315in}{1.916612in}}{\pgfqpoint{4.022716in}{1.921002in}}{\pgfqpoint{4.011666in}{1.921002in}}%
\pgfpathcurveto{\pgfqpoint{4.000616in}{1.921002in}}{\pgfqpoint{3.990016in}{1.916612in}}{\pgfqpoint{3.982203in}{1.908798in}}%
\pgfpathcurveto{\pgfqpoint{3.974389in}{1.900985in}}{\pgfqpoint{3.969999in}{1.890386in}}{\pgfqpoint{3.969999in}{1.879336in}}%
\pgfpathcurveto{\pgfqpoint{3.969999in}{1.868286in}}{\pgfqpoint{3.974389in}{1.857687in}}{\pgfqpoint{3.982203in}{1.849873in}}%
\pgfpathcurveto{\pgfqpoint{3.990016in}{1.842059in}}{\pgfqpoint{4.000616in}{1.837669in}}{\pgfqpoint{4.011666in}{1.837669in}}%
\pgfpathclose%
\pgfusepath{stroke,fill}%
\end{pgfscope}%
\begin{pgfscope}%
\pgfpathrectangle{\pgfqpoint{0.800000in}{0.528000in}}{\pgfqpoint{4.960000in}{3.696000in}}%
\pgfusepath{clip}%
\pgfsetbuttcap%
\pgfsetroundjoin%
\definecolor{currentfill}{rgb}{0.000000,0.000000,0.000000}%
\pgfsetfillcolor{currentfill}%
\pgfsetlinewidth{1.003750pt}%
\definecolor{currentstroke}{rgb}{0.000000,0.000000,0.000000}%
\pgfsetstrokecolor{currentstroke}%
\pgfsetdash{}{0pt}%
\pgfpathmoveto{\pgfqpoint{4.011666in}{1.880664in}}%
\pgfpathcurveto{\pgfqpoint{4.022716in}{1.880664in}}{\pgfqpoint{4.033315in}{1.885054in}}{\pgfqpoint{4.041128in}{1.892868in}}%
\pgfpathcurveto{\pgfqpoint{4.048942in}{1.900682in}}{\pgfqpoint{4.053332in}{1.911281in}}{\pgfqpoint{4.053332in}{1.922331in}}%
\pgfpathcurveto{\pgfqpoint{4.053332in}{1.933381in}}{\pgfqpoint{4.048942in}{1.943980in}}{\pgfqpoint{4.041128in}{1.951793in}}%
\pgfpathcurveto{\pgfqpoint{4.033315in}{1.959607in}}{\pgfqpoint{4.022716in}{1.963997in}}{\pgfqpoint{4.011666in}{1.963997in}}%
\pgfpathcurveto{\pgfqpoint{4.000616in}{1.963997in}}{\pgfqpoint{3.990016in}{1.959607in}}{\pgfqpoint{3.982203in}{1.951793in}}%
\pgfpathcurveto{\pgfqpoint{3.974389in}{1.943980in}}{\pgfqpoint{3.969999in}{1.933381in}}{\pgfqpoint{3.969999in}{1.922331in}}%
\pgfpathcurveto{\pgfqpoint{3.969999in}{1.911281in}}{\pgfqpoint{3.974389in}{1.900682in}}{\pgfqpoint{3.982203in}{1.892868in}}%
\pgfpathcurveto{\pgfqpoint{3.990016in}{1.885054in}}{\pgfqpoint{4.000616in}{1.880664in}}{\pgfqpoint{4.011666in}{1.880664in}}%
\pgfpathclose%
\pgfusepath{stroke,fill}%
\end{pgfscope}%
\begin{pgfscope}%
\pgfpathrectangle{\pgfqpoint{0.800000in}{0.528000in}}{\pgfqpoint{4.960000in}{3.696000in}}%
\pgfusepath{clip}%
\pgfsetbuttcap%
\pgfsetroundjoin%
\definecolor{currentfill}{rgb}{0.000000,0.000000,0.000000}%
\pgfsetfillcolor{currentfill}%
\pgfsetlinewidth{1.003750pt}%
\definecolor{currentstroke}{rgb}{0.000000,0.000000,0.000000}%
\pgfsetstrokecolor{currentstroke}%
\pgfsetdash{}{0pt}%
\pgfpathmoveto{\pgfqpoint{4.011666in}{1.988151in}}%
\pgfpathcurveto{\pgfqpoint{4.022716in}{1.988151in}}{\pgfqpoint{4.033315in}{1.992542in}}{\pgfqpoint{4.041128in}{2.000355in}}%
\pgfpathcurveto{\pgfqpoint{4.048942in}{2.008169in}}{\pgfqpoint{4.053332in}{2.018768in}}{\pgfqpoint{4.053332in}{2.029818in}}%
\pgfpathcurveto{\pgfqpoint{4.053332in}{2.040868in}}{\pgfqpoint{4.048942in}{2.051467in}}{\pgfqpoint{4.041128in}{2.059281in}}%
\pgfpathcurveto{\pgfqpoint{4.033315in}{2.067095in}}{\pgfqpoint{4.022716in}{2.071485in}}{\pgfqpoint{4.011666in}{2.071485in}}%
\pgfpathcurveto{\pgfqpoint{4.000616in}{2.071485in}}{\pgfqpoint{3.990016in}{2.067095in}}{\pgfqpoint{3.982203in}{2.059281in}}%
\pgfpathcurveto{\pgfqpoint{3.974389in}{2.051467in}}{\pgfqpoint{3.969999in}{2.040868in}}{\pgfqpoint{3.969999in}{2.029818in}}%
\pgfpathcurveto{\pgfqpoint{3.969999in}{2.018768in}}{\pgfqpoint{3.974389in}{2.008169in}}{\pgfqpoint{3.982203in}{2.000355in}}%
\pgfpathcurveto{\pgfqpoint{3.990016in}{1.992542in}}{\pgfqpoint{4.000616in}{1.988151in}}{\pgfqpoint{4.011666in}{1.988151in}}%
\pgfpathclose%
\pgfusepath{stroke,fill}%
\end{pgfscope}%
\begin{pgfscope}%
\pgfpathrectangle{\pgfqpoint{0.800000in}{0.528000in}}{\pgfqpoint{4.960000in}{3.696000in}}%
\pgfusepath{clip}%
\pgfsetbuttcap%
\pgfsetroundjoin%
\definecolor{currentfill}{rgb}{0.000000,0.000000,0.000000}%
\pgfsetfillcolor{currentfill}%
\pgfsetlinewidth{1.003750pt}%
\definecolor{currentstroke}{rgb}{0.000000,0.000000,0.000000}%
\pgfsetstrokecolor{currentstroke}%
\pgfsetdash{}{0pt}%
\pgfpathmoveto{\pgfqpoint{4.011666in}{1.923659in}}%
\pgfpathcurveto{\pgfqpoint{4.022716in}{1.923659in}}{\pgfqpoint{4.033315in}{1.928049in}}{\pgfqpoint{4.041128in}{1.935863in}}%
\pgfpathcurveto{\pgfqpoint{4.048942in}{1.943677in}}{\pgfqpoint{4.053332in}{1.954276in}}{\pgfqpoint{4.053332in}{1.965326in}}%
\pgfpathcurveto{\pgfqpoint{4.053332in}{1.976376in}}{\pgfqpoint{4.048942in}{1.986975in}}{\pgfqpoint{4.041128in}{1.994788in}}%
\pgfpathcurveto{\pgfqpoint{4.033315in}{2.002602in}}{\pgfqpoint{4.022716in}{2.006992in}}{\pgfqpoint{4.011666in}{2.006992in}}%
\pgfpathcurveto{\pgfqpoint{4.000616in}{2.006992in}}{\pgfqpoint{3.990016in}{2.002602in}}{\pgfqpoint{3.982203in}{1.994788in}}%
\pgfpathcurveto{\pgfqpoint{3.974389in}{1.986975in}}{\pgfqpoint{3.969999in}{1.976376in}}{\pgfqpoint{3.969999in}{1.965326in}}%
\pgfpathcurveto{\pgfqpoint{3.969999in}{1.954276in}}{\pgfqpoint{3.974389in}{1.943677in}}{\pgfqpoint{3.982203in}{1.935863in}}%
\pgfpathcurveto{\pgfqpoint{3.990016in}{1.928049in}}{\pgfqpoint{4.000616in}{1.923659in}}{\pgfqpoint{4.011666in}{1.923659in}}%
\pgfpathclose%
\pgfusepath{stroke,fill}%
\end{pgfscope}%
\begin{pgfscope}%
\pgfpathrectangle{\pgfqpoint{0.800000in}{0.528000in}}{\pgfqpoint{4.960000in}{3.696000in}}%
\pgfusepath{clip}%
\pgfsetbuttcap%
\pgfsetroundjoin%
\definecolor{currentfill}{rgb}{0.000000,0.000000,0.000000}%
\pgfsetfillcolor{currentfill}%
\pgfsetlinewidth{1.003750pt}%
\definecolor{currentstroke}{rgb}{0.000000,0.000000,0.000000}%
\pgfsetstrokecolor{currentstroke}%
\pgfsetdash{}{0pt}%
\pgfpathmoveto{\pgfqpoint{4.011666in}{1.816172in}}%
\pgfpathcurveto{\pgfqpoint{4.022716in}{1.816172in}}{\pgfqpoint{4.033315in}{1.820562in}}{\pgfqpoint{4.041128in}{1.828375in}}%
\pgfpathcurveto{\pgfqpoint{4.048942in}{1.836189in}}{\pgfqpoint{4.053332in}{1.846788in}}{\pgfqpoint{4.053332in}{1.857838in}}%
\pgfpathcurveto{\pgfqpoint{4.053332in}{1.868888in}}{\pgfqpoint{4.048942in}{1.879487in}}{\pgfqpoint{4.041128in}{1.887301in}}%
\pgfpathcurveto{\pgfqpoint{4.033315in}{1.895115in}}{\pgfqpoint{4.022716in}{1.899505in}}{\pgfqpoint{4.011666in}{1.899505in}}%
\pgfpathcurveto{\pgfqpoint{4.000616in}{1.899505in}}{\pgfqpoint{3.990016in}{1.895115in}}{\pgfqpoint{3.982203in}{1.887301in}}%
\pgfpathcurveto{\pgfqpoint{3.974389in}{1.879487in}}{\pgfqpoint{3.969999in}{1.868888in}}{\pgfqpoint{3.969999in}{1.857838in}}%
\pgfpathcurveto{\pgfqpoint{3.969999in}{1.846788in}}{\pgfqpoint{3.974389in}{1.836189in}}{\pgfqpoint{3.982203in}{1.828375in}}%
\pgfpathcurveto{\pgfqpoint{3.990016in}{1.820562in}}{\pgfqpoint{4.000616in}{1.816172in}}{\pgfqpoint{4.011666in}{1.816172in}}%
\pgfpathclose%
\pgfusepath{stroke,fill}%
\end{pgfscope}%
\begin{pgfscope}%
\pgfpathrectangle{\pgfqpoint{0.800000in}{0.528000in}}{\pgfqpoint{4.960000in}{3.696000in}}%
\pgfusepath{clip}%
\pgfsetbuttcap%
\pgfsetroundjoin%
\definecolor{currentfill}{rgb}{0.000000,0.000000,0.000000}%
\pgfsetfillcolor{currentfill}%
\pgfsetlinewidth{1.003750pt}%
\definecolor{currentstroke}{rgb}{0.000000,0.000000,0.000000}%
\pgfsetstrokecolor{currentstroke}%
\pgfsetdash{}{0pt}%
\pgfpathmoveto{\pgfqpoint{4.011666in}{2.052644in}}%
\pgfpathcurveto{\pgfqpoint{4.022716in}{2.052644in}}{\pgfqpoint{4.033315in}{2.057034in}}{\pgfqpoint{4.041128in}{2.064848in}}%
\pgfpathcurveto{\pgfqpoint{4.048942in}{2.072661in}}{\pgfqpoint{4.053332in}{2.083261in}}{\pgfqpoint{4.053332in}{2.094311in}}%
\pgfpathcurveto{\pgfqpoint{4.053332in}{2.105361in}}{\pgfqpoint{4.048942in}{2.115960in}}{\pgfqpoint{4.041128in}{2.123773in}}%
\pgfpathcurveto{\pgfqpoint{4.033315in}{2.131587in}}{\pgfqpoint{4.022716in}{2.135977in}}{\pgfqpoint{4.011666in}{2.135977in}}%
\pgfpathcurveto{\pgfqpoint{4.000616in}{2.135977in}}{\pgfqpoint{3.990016in}{2.131587in}}{\pgfqpoint{3.982203in}{2.123773in}}%
\pgfpathcurveto{\pgfqpoint{3.974389in}{2.115960in}}{\pgfqpoint{3.969999in}{2.105361in}}{\pgfqpoint{3.969999in}{2.094311in}}%
\pgfpathcurveto{\pgfqpoint{3.969999in}{2.083261in}}{\pgfqpoint{3.974389in}{2.072661in}}{\pgfqpoint{3.982203in}{2.064848in}}%
\pgfpathcurveto{\pgfqpoint{3.990016in}{2.057034in}}{\pgfqpoint{4.000616in}{2.052644in}}{\pgfqpoint{4.011666in}{2.052644in}}%
\pgfpathclose%
\pgfusepath{stroke,fill}%
\end{pgfscope}%
\begin{pgfscope}%
\pgfpathrectangle{\pgfqpoint{0.800000in}{0.528000in}}{\pgfqpoint{4.960000in}{3.696000in}}%
\pgfusepath{clip}%
\pgfsetbuttcap%
\pgfsetroundjoin%
\definecolor{currentfill}{rgb}{0.000000,0.000000,0.000000}%
\pgfsetfillcolor{currentfill}%
\pgfsetlinewidth{1.003750pt}%
\definecolor{currentstroke}{rgb}{0.000000,0.000000,0.000000}%
\pgfsetstrokecolor{currentstroke}%
\pgfsetdash{}{0pt}%
\pgfpathmoveto{\pgfqpoint{4.011666in}{1.859167in}}%
\pgfpathcurveto{\pgfqpoint{4.022716in}{1.859167in}}{\pgfqpoint{4.033315in}{1.863557in}}{\pgfqpoint{4.041128in}{1.871370in}}%
\pgfpathcurveto{\pgfqpoint{4.048942in}{1.879184in}}{\pgfqpoint{4.053332in}{1.889783in}}{\pgfqpoint{4.053332in}{1.900833in}}%
\pgfpathcurveto{\pgfqpoint{4.053332in}{1.911883in}}{\pgfqpoint{4.048942in}{1.922482in}}{\pgfqpoint{4.041128in}{1.930296in}}%
\pgfpathcurveto{\pgfqpoint{4.033315in}{1.938110in}}{\pgfqpoint{4.022716in}{1.942500in}}{\pgfqpoint{4.011666in}{1.942500in}}%
\pgfpathcurveto{\pgfqpoint{4.000616in}{1.942500in}}{\pgfqpoint{3.990016in}{1.938110in}}{\pgfqpoint{3.982203in}{1.930296in}}%
\pgfpathcurveto{\pgfqpoint{3.974389in}{1.922482in}}{\pgfqpoint{3.969999in}{1.911883in}}{\pgfqpoint{3.969999in}{1.900833in}}%
\pgfpathcurveto{\pgfqpoint{3.969999in}{1.889783in}}{\pgfqpoint{3.974389in}{1.879184in}}{\pgfqpoint{3.982203in}{1.871370in}}%
\pgfpathcurveto{\pgfqpoint{3.990016in}{1.863557in}}{\pgfqpoint{4.000616in}{1.859167in}}{\pgfqpoint{4.011666in}{1.859167in}}%
\pgfpathclose%
\pgfusepath{stroke,fill}%
\end{pgfscope}%
\begin{pgfscope}%
\pgfpathrectangle{\pgfqpoint{0.800000in}{0.528000in}}{\pgfqpoint{4.960000in}{3.696000in}}%
\pgfusepath{clip}%
\pgfsetbuttcap%
\pgfsetroundjoin%
\definecolor{currentfill}{rgb}{0.000000,0.000000,0.000000}%
\pgfsetfillcolor{currentfill}%
\pgfsetlinewidth{1.003750pt}%
\definecolor{currentstroke}{rgb}{0.000000,0.000000,0.000000}%
\pgfsetstrokecolor{currentstroke}%
\pgfsetdash{}{0pt}%
\pgfpathmoveto{\pgfqpoint{4.011666in}{1.816172in}}%
\pgfpathcurveto{\pgfqpoint{4.022716in}{1.816172in}}{\pgfqpoint{4.033315in}{1.820562in}}{\pgfqpoint{4.041128in}{1.828375in}}%
\pgfpathcurveto{\pgfqpoint{4.048942in}{1.836189in}}{\pgfqpoint{4.053332in}{1.846788in}}{\pgfqpoint{4.053332in}{1.857838in}}%
\pgfpathcurveto{\pgfqpoint{4.053332in}{1.868888in}}{\pgfqpoint{4.048942in}{1.879487in}}{\pgfqpoint{4.041128in}{1.887301in}}%
\pgfpathcurveto{\pgfqpoint{4.033315in}{1.895115in}}{\pgfqpoint{4.022716in}{1.899505in}}{\pgfqpoint{4.011666in}{1.899505in}}%
\pgfpathcurveto{\pgfqpoint{4.000616in}{1.899505in}}{\pgfqpoint{3.990016in}{1.895115in}}{\pgfqpoint{3.982203in}{1.887301in}}%
\pgfpathcurveto{\pgfqpoint{3.974389in}{1.879487in}}{\pgfqpoint{3.969999in}{1.868888in}}{\pgfqpoint{3.969999in}{1.857838in}}%
\pgfpathcurveto{\pgfqpoint{3.969999in}{1.846788in}}{\pgfqpoint{3.974389in}{1.836189in}}{\pgfqpoint{3.982203in}{1.828375in}}%
\pgfpathcurveto{\pgfqpoint{3.990016in}{1.820562in}}{\pgfqpoint{4.000616in}{1.816172in}}{\pgfqpoint{4.011666in}{1.816172in}}%
\pgfpathclose%
\pgfusepath{stroke,fill}%
\end{pgfscope}%
\begin{pgfscope}%
\pgfpathrectangle{\pgfqpoint{0.800000in}{0.528000in}}{\pgfqpoint{4.960000in}{3.696000in}}%
\pgfusepath{clip}%
\pgfsetbuttcap%
\pgfsetroundjoin%
\definecolor{currentfill}{rgb}{0.000000,0.000000,0.000000}%
\pgfsetfillcolor{currentfill}%
\pgfsetlinewidth{1.003750pt}%
\definecolor{currentstroke}{rgb}{0.000000,0.000000,0.000000}%
\pgfsetstrokecolor{currentstroke}%
\pgfsetdash{}{0pt}%
\pgfpathmoveto{\pgfqpoint{4.011666in}{1.945157in}}%
\pgfpathcurveto{\pgfqpoint{4.022716in}{1.945157in}}{\pgfqpoint{4.033315in}{1.949547in}}{\pgfqpoint{4.041128in}{1.957360in}}%
\pgfpathcurveto{\pgfqpoint{4.048942in}{1.965174in}}{\pgfqpoint{4.053332in}{1.975773in}}{\pgfqpoint{4.053332in}{1.986823in}}%
\pgfpathcurveto{\pgfqpoint{4.053332in}{1.997873in}}{\pgfqpoint{4.048942in}{2.008472in}}{\pgfqpoint{4.041128in}{2.016286in}}%
\pgfpathcurveto{\pgfqpoint{4.033315in}{2.024100in}}{\pgfqpoint{4.022716in}{2.028490in}}{\pgfqpoint{4.011666in}{2.028490in}}%
\pgfpathcurveto{\pgfqpoint{4.000616in}{2.028490in}}{\pgfqpoint{3.990016in}{2.024100in}}{\pgfqpoint{3.982203in}{2.016286in}}%
\pgfpathcurveto{\pgfqpoint{3.974389in}{2.008472in}}{\pgfqpoint{3.969999in}{1.997873in}}{\pgfqpoint{3.969999in}{1.986823in}}%
\pgfpathcurveto{\pgfqpoint{3.969999in}{1.975773in}}{\pgfqpoint{3.974389in}{1.965174in}}{\pgfqpoint{3.982203in}{1.957360in}}%
\pgfpathcurveto{\pgfqpoint{3.990016in}{1.949547in}}{\pgfqpoint{4.000616in}{1.945157in}}{\pgfqpoint{4.011666in}{1.945157in}}%
\pgfpathclose%
\pgfusepath{stroke,fill}%
\end{pgfscope}%
\begin{pgfscope}%
\pgfpathrectangle{\pgfqpoint{0.800000in}{0.528000in}}{\pgfqpoint{4.960000in}{3.696000in}}%
\pgfusepath{clip}%
\pgfsetbuttcap%
\pgfsetroundjoin%
\definecolor{currentfill}{rgb}{0.000000,0.000000,0.000000}%
\pgfsetfillcolor{currentfill}%
\pgfsetlinewidth{1.003750pt}%
\definecolor{currentstroke}{rgb}{0.000000,0.000000,0.000000}%
\pgfsetstrokecolor{currentstroke}%
\pgfsetdash{}{0pt}%
\pgfpathmoveto{\pgfqpoint{4.011666in}{1.837669in}}%
\pgfpathcurveto{\pgfqpoint{4.022716in}{1.837669in}}{\pgfqpoint{4.033315in}{1.842059in}}{\pgfqpoint{4.041128in}{1.849873in}}%
\pgfpathcurveto{\pgfqpoint{4.048942in}{1.857687in}}{\pgfqpoint{4.053332in}{1.868286in}}{\pgfqpoint{4.053332in}{1.879336in}}%
\pgfpathcurveto{\pgfqpoint{4.053332in}{1.890386in}}{\pgfqpoint{4.048942in}{1.900985in}}{\pgfqpoint{4.041128in}{1.908798in}}%
\pgfpathcurveto{\pgfqpoint{4.033315in}{1.916612in}}{\pgfqpoint{4.022716in}{1.921002in}}{\pgfqpoint{4.011666in}{1.921002in}}%
\pgfpathcurveto{\pgfqpoint{4.000616in}{1.921002in}}{\pgfqpoint{3.990016in}{1.916612in}}{\pgfqpoint{3.982203in}{1.908798in}}%
\pgfpathcurveto{\pgfqpoint{3.974389in}{1.900985in}}{\pgfqpoint{3.969999in}{1.890386in}}{\pgfqpoint{3.969999in}{1.879336in}}%
\pgfpathcurveto{\pgfqpoint{3.969999in}{1.868286in}}{\pgfqpoint{3.974389in}{1.857687in}}{\pgfqpoint{3.982203in}{1.849873in}}%
\pgfpathcurveto{\pgfqpoint{3.990016in}{1.842059in}}{\pgfqpoint{4.000616in}{1.837669in}}{\pgfqpoint{4.011666in}{1.837669in}}%
\pgfpathclose%
\pgfusepath{stroke,fill}%
\end{pgfscope}%
\begin{pgfscope}%
\pgfpathrectangle{\pgfqpoint{0.800000in}{0.528000in}}{\pgfqpoint{4.960000in}{3.696000in}}%
\pgfusepath{clip}%
\pgfsetbuttcap%
\pgfsetroundjoin%
\definecolor{currentfill}{rgb}{0.000000,0.000000,0.000000}%
\pgfsetfillcolor{currentfill}%
\pgfsetlinewidth{1.003750pt}%
\definecolor{currentstroke}{rgb}{0.000000,0.000000,0.000000}%
\pgfsetstrokecolor{currentstroke}%
\pgfsetdash{}{0pt}%
\pgfpathmoveto{\pgfqpoint{4.011666in}{1.816172in}}%
\pgfpathcurveto{\pgfqpoint{4.022716in}{1.816172in}}{\pgfqpoint{4.033315in}{1.820562in}}{\pgfqpoint{4.041128in}{1.828375in}}%
\pgfpathcurveto{\pgfqpoint{4.048942in}{1.836189in}}{\pgfqpoint{4.053332in}{1.846788in}}{\pgfqpoint{4.053332in}{1.857838in}}%
\pgfpathcurveto{\pgfqpoint{4.053332in}{1.868888in}}{\pgfqpoint{4.048942in}{1.879487in}}{\pgfqpoint{4.041128in}{1.887301in}}%
\pgfpathcurveto{\pgfqpoint{4.033315in}{1.895115in}}{\pgfqpoint{4.022716in}{1.899505in}}{\pgfqpoint{4.011666in}{1.899505in}}%
\pgfpathcurveto{\pgfqpoint{4.000616in}{1.899505in}}{\pgfqpoint{3.990016in}{1.895115in}}{\pgfqpoint{3.982203in}{1.887301in}}%
\pgfpathcurveto{\pgfqpoint{3.974389in}{1.879487in}}{\pgfqpoint{3.969999in}{1.868888in}}{\pgfqpoint{3.969999in}{1.857838in}}%
\pgfpathcurveto{\pgfqpoint{3.969999in}{1.846788in}}{\pgfqpoint{3.974389in}{1.836189in}}{\pgfqpoint{3.982203in}{1.828375in}}%
\pgfpathcurveto{\pgfqpoint{3.990016in}{1.820562in}}{\pgfqpoint{4.000616in}{1.816172in}}{\pgfqpoint{4.011666in}{1.816172in}}%
\pgfpathclose%
\pgfusepath{stroke,fill}%
\end{pgfscope}%
\begin{pgfscope}%
\pgfpathrectangle{\pgfqpoint{0.800000in}{0.528000in}}{\pgfqpoint{4.960000in}{3.696000in}}%
\pgfusepath{clip}%
\pgfsetbuttcap%
\pgfsetroundjoin%
\definecolor{currentfill}{rgb}{0.000000,0.000000,0.000000}%
\pgfsetfillcolor{currentfill}%
\pgfsetlinewidth{1.003750pt}%
\definecolor{currentstroke}{rgb}{0.000000,0.000000,0.000000}%
\pgfsetstrokecolor{currentstroke}%
\pgfsetdash{}{0pt}%
\pgfpathmoveto{\pgfqpoint{4.011666in}{1.859167in}}%
\pgfpathcurveto{\pgfqpoint{4.022716in}{1.859167in}}{\pgfqpoint{4.033315in}{1.863557in}}{\pgfqpoint{4.041128in}{1.871370in}}%
\pgfpathcurveto{\pgfqpoint{4.048942in}{1.879184in}}{\pgfqpoint{4.053332in}{1.889783in}}{\pgfqpoint{4.053332in}{1.900833in}}%
\pgfpathcurveto{\pgfqpoint{4.053332in}{1.911883in}}{\pgfqpoint{4.048942in}{1.922482in}}{\pgfqpoint{4.041128in}{1.930296in}}%
\pgfpathcurveto{\pgfqpoint{4.033315in}{1.938110in}}{\pgfqpoint{4.022716in}{1.942500in}}{\pgfqpoint{4.011666in}{1.942500in}}%
\pgfpathcurveto{\pgfqpoint{4.000616in}{1.942500in}}{\pgfqpoint{3.990016in}{1.938110in}}{\pgfqpoint{3.982203in}{1.930296in}}%
\pgfpathcurveto{\pgfqpoint{3.974389in}{1.922482in}}{\pgfqpoint{3.969999in}{1.911883in}}{\pgfqpoint{3.969999in}{1.900833in}}%
\pgfpathcurveto{\pgfqpoint{3.969999in}{1.889783in}}{\pgfqpoint{3.974389in}{1.879184in}}{\pgfqpoint{3.982203in}{1.871370in}}%
\pgfpathcurveto{\pgfqpoint{3.990016in}{1.863557in}}{\pgfqpoint{4.000616in}{1.859167in}}{\pgfqpoint{4.011666in}{1.859167in}}%
\pgfpathclose%
\pgfusepath{stroke,fill}%
\end{pgfscope}%
\begin{pgfscope}%
\pgfpathrectangle{\pgfqpoint{0.800000in}{0.528000in}}{\pgfqpoint{4.960000in}{3.696000in}}%
\pgfusepath{clip}%
\pgfsetbuttcap%
\pgfsetroundjoin%
\definecolor{currentfill}{rgb}{0.000000,0.000000,0.000000}%
\pgfsetfillcolor{currentfill}%
\pgfsetlinewidth{1.003750pt}%
\definecolor{currentstroke}{rgb}{0.000000,0.000000,0.000000}%
\pgfsetstrokecolor{currentstroke}%
\pgfsetdash{}{0pt}%
\pgfpathmoveto{\pgfqpoint{4.011666in}{1.837669in}}%
\pgfpathcurveto{\pgfqpoint{4.022716in}{1.837669in}}{\pgfqpoint{4.033315in}{1.842059in}}{\pgfqpoint{4.041128in}{1.849873in}}%
\pgfpathcurveto{\pgfqpoint{4.048942in}{1.857687in}}{\pgfqpoint{4.053332in}{1.868286in}}{\pgfqpoint{4.053332in}{1.879336in}}%
\pgfpathcurveto{\pgfqpoint{4.053332in}{1.890386in}}{\pgfqpoint{4.048942in}{1.900985in}}{\pgfqpoint{4.041128in}{1.908798in}}%
\pgfpathcurveto{\pgfqpoint{4.033315in}{1.916612in}}{\pgfqpoint{4.022716in}{1.921002in}}{\pgfqpoint{4.011666in}{1.921002in}}%
\pgfpathcurveto{\pgfqpoint{4.000616in}{1.921002in}}{\pgfqpoint{3.990016in}{1.916612in}}{\pgfqpoint{3.982203in}{1.908798in}}%
\pgfpathcurveto{\pgfqpoint{3.974389in}{1.900985in}}{\pgfqpoint{3.969999in}{1.890386in}}{\pgfqpoint{3.969999in}{1.879336in}}%
\pgfpathcurveto{\pgfqpoint{3.969999in}{1.868286in}}{\pgfqpoint{3.974389in}{1.857687in}}{\pgfqpoint{3.982203in}{1.849873in}}%
\pgfpathcurveto{\pgfqpoint{3.990016in}{1.842059in}}{\pgfqpoint{4.000616in}{1.837669in}}{\pgfqpoint{4.011666in}{1.837669in}}%
\pgfpathclose%
\pgfusepath{stroke,fill}%
\end{pgfscope}%
\begin{pgfscope}%
\pgfpathrectangle{\pgfqpoint{0.800000in}{0.528000in}}{\pgfqpoint{4.960000in}{3.696000in}}%
\pgfusepath{clip}%
\pgfsetbuttcap%
\pgfsetroundjoin%
\definecolor{currentfill}{rgb}{0.000000,0.000000,0.000000}%
\pgfsetfillcolor{currentfill}%
\pgfsetlinewidth{1.003750pt}%
\definecolor{currentstroke}{rgb}{0.000000,0.000000,0.000000}%
\pgfsetstrokecolor{currentstroke}%
\pgfsetdash{}{0pt}%
\pgfpathmoveto{\pgfqpoint{4.011666in}{2.009649in}}%
\pgfpathcurveto{\pgfqpoint{4.022716in}{2.009649in}}{\pgfqpoint{4.033315in}{2.014039in}}{\pgfqpoint{4.041128in}{2.021853in}}%
\pgfpathcurveto{\pgfqpoint{4.048942in}{2.029666in}}{\pgfqpoint{4.053332in}{2.040266in}}{\pgfqpoint{4.053332in}{2.051316in}}%
\pgfpathcurveto{\pgfqpoint{4.053332in}{2.062366in}}{\pgfqpoint{4.048942in}{2.072965in}}{\pgfqpoint{4.041128in}{2.080778in}}%
\pgfpathcurveto{\pgfqpoint{4.033315in}{2.088592in}}{\pgfqpoint{4.022716in}{2.092982in}}{\pgfqpoint{4.011666in}{2.092982in}}%
\pgfpathcurveto{\pgfqpoint{4.000616in}{2.092982in}}{\pgfqpoint{3.990016in}{2.088592in}}{\pgfqpoint{3.982203in}{2.080778in}}%
\pgfpathcurveto{\pgfqpoint{3.974389in}{2.072965in}}{\pgfqpoint{3.969999in}{2.062366in}}{\pgfqpoint{3.969999in}{2.051316in}}%
\pgfpathcurveto{\pgfqpoint{3.969999in}{2.040266in}}{\pgfqpoint{3.974389in}{2.029666in}}{\pgfqpoint{3.982203in}{2.021853in}}%
\pgfpathcurveto{\pgfqpoint{3.990016in}{2.014039in}}{\pgfqpoint{4.000616in}{2.009649in}}{\pgfqpoint{4.011666in}{2.009649in}}%
\pgfpathclose%
\pgfusepath{stroke,fill}%
\end{pgfscope}%
\begin{pgfscope}%
\pgfpathrectangle{\pgfqpoint{0.800000in}{0.528000in}}{\pgfqpoint{4.960000in}{3.696000in}}%
\pgfusepath{clip}%
\pgfsetbuttcap%
\pgfsetroundjoin%
\definecolor{currentfill}{rgb}{0.000000,0.000000,0.000000}%
\pgfsetfillcolor{currentfill}%
\pgfsetlinewidth{1.003750pt}%
\definecolor{currentstroke}{rgb}{0.000000,0.000000,0.000000}%
\pgfsetstrokecolor{currentstroke}%
\pgfsetdash{}{0pt}%
\pgfpathmoveto{\pgfqpoint{4.011666in}{1.837669in}}%
\pgfpathcurveto{\pgfqpoint{4.022716in}{1.837669in}}{\pgfqpoint{4.033315in}{1.842059in}}{\pgfqpoint{4.041128in}{1.849873in}}%
\pgfpathcurveto{\pgfqpoint{4.048942in}{1.857687in}}{\pgfqpoint{4.053332in}{1.868286in}}{\pgfqpoint{4.053332in}{1.879336in}}%
\pgfpathcurveto{\pgfqpoint{4.053332in}{1.890386in}}{\pgfqpoint{4.048942in}{1.900985in}}{\pgfqpoint{4.041128in}{1.908798in}}%
\pgfpathcurveto{\pgfqpoint{4.033315in}{1.916612in}}{\pgfqpoint{4.022716in}{1.921002in}}{\pgfqpoint{4.011666in}{1.921002in}}%
\pgfpathcurveto{\pgfqpoint{4.000616in}{1.921002in}}{\pgfqpoint{3.990016in}{1.916612in}}{\pgfqpoint{3.982203in}{1.908798in}}%
\pgfpathcurveto{\pgfqpoint{3.974389in}{1.900985in}}{\pgfqpoint{3.969999in}{1.890386in}}{\pgfqpoint{3.969999in}{1.879336in}}%
\pgfpathcurveto{\pgfqpoint{3.969999in}{1.868286in}}{\pgfqpoint{3.974389in}{1.857687in}}{\pgfqpoint{3.982203in}{1.849873in}}%
\pgfpathcurveto{\pgfqpoint{3.990016in}{1.842059in}}{\pgfqpoint{4.000616in}{1.837669in}}{\pgfqpoint{4.011666in}{1.837669in}}%
\pgfpathclose%
\pgfusepath{stroke,fill}%
\end{pgfscope}%
\begin{pgfscope}%
\pgfpathrectangle{\pgfqpoint{0.800000in}{0.528000in}}{\pgfqpoint{4.960000in}{3.696000in}}%
\pgfusepath{clip}%
\pgfsetbuttcap%
\pgfsetroundjoin%
\definecolor{currentfill}{rgb}{0.000000,0.000000,0.000000}%
\pgfsetfillcolor{currentfill}%
\pgfsetlinewidth{1.003750pt}%
\definecolor{currentstroke}{rgb}{0.000000,0.000000,0.000000}%
\pgfsetstrokecolor{currentstroke}%
\pgfsetdash{}{0pt}%
\pgfpathmoveto{\pgfqpoint{4.011666in}{2.031146in}}%
\pgfpathcurveto{\pgfqpoint{4.022716in}{2.031146in}}{\pgfqpoint{4.033315in}{2.035537in}}{\pgfqpoint{4.041128in}{2.043350in}}%
\pgfpathcurveto{\pgfqpoint{4.048942in}{2.051164in}}{\pgfqpoint{4.053332in}{2.061763in}}{\pgfqpoint{4.053332in}{2.072813in}}%
\pgfpathcurveto{\pgfqpoint{4.053332in}{2.083863in}}{\pgfqpoint{4.048942in}{2.094462in}}{\pgfqpoint{4.041128in}{2.102276in}}%
\pgfpathcurveto{\pgfqpoint{4.033315in}{2.110090in}}{\pgfqpoint{4.022716in}{2.114480in}}{\pgfqpoint{4.011666in}{2.114480in}}%
\pgfpathcurveto{\pgfqpoint{4.000616in}{2.114480in}}{\pgfqpoint{3.990016in}{2.110090in}}{\pgfqpoint{3.982203in}{2.102276in}}%
\pgfpathcurveto{\pgfqpoint{3.974389in}{2.094462in}}{\pgfqpoint{3.969999in}{2.083863in}}{\pgfqpoint{3.969999in}{2.072813in}}%
\pgfpathcurveto{\pgfqpoint{3.969999in}{2.061763in}}{\pgfqpoint{3.974389in}{2.051164in}}{\pgfqpoint{3.982203in}{2.043350in}}%
\pgfpathcurveto{\pgfqpoint{3.990016in}{2.035537in}}{\pgfqpoint{4.000616in}{2.031146in}}{\pgfqpoint{4.011666in}{2.031146in}}%
\pgfpathclose%
\pgfusepath{stroke,fill}%
\end{pgfscope}%
\begin{pgfscope}%
\pgfpathrectangle{\pgfqpoint{0.800000in}{0.528000in}}{\pgfqpoint{4.960000in}{3.696000in}}%
\pgfusepath{clip}%
\pgfsetbuttcap%
\pgfsetroundjoin%
\definecolor{currentfill}{rgb}{0.000000,0.000000,0.000000}%
\pgfsetfillcolor{currentfill}%
\pgfsetlinewidth{1.003750pt}%
\definecolor{currentstroke}{rgb}{0.000000,0.000000,0.000000}%
\pgfsetstrokecolor{currentstroke}%
\pgfsetdash{}{0pt}%
\pgfpathmoveto{\pgfqpoint{4.011666in}{1.988151in}}%
\pgfpathcurveto{\pgfqpoint{4.022716in}{1.988151in}}{\pgfqpoint{4.033315in}{1.992542in}}{\pgfqpoint{4.041128in}{2.000355in}}%
\pgfpathcurveto{\pgfqpoint{4.048942in}{2.008169in}}{\pgfqpoint{4.053332in}{2.018768in}}{\pgfqpoint{4.053332in}{2.029818in}}%
\pgfpathcurveto{\pgfqpoint{4.053332in}{2.040868in}}{\pgfqpoint{4.048942in}{2.051467in}}{\pgfqpoint{4.041128in}{2.059281in}}%
\pgfpathcurveto{\pgfqpoint{4.033315in}{2.067095in}}{\pgfqpoint{4.022716in}{2.071485in}}{\pgfqpoint{4.011666in}{2.071485in}}%
\pgfpathcurveto{\pgfqpoint{4.000616in}{2.071485in}}{\pgfqpoint{3.990016in}{2.067095in}}{\pgfqpoint{3.982203in}{2.059281in}}%
\pgfpathcurveto{\pgfqpoint{3.974389in}{2.051467in}}{\pgfqpoint{3.969999in}{2.040868in}}{\pgfqpoint{3.969999in}{2.029818in}}%
\pgfpathcurveto{\pgfqpoint{3.969999in}{2.018768in}}{\pgfqpoint{3.974389in}{2.008169in}}{\pgfqpoint{3.982203in}{2.000355in}}%
\pgfpathcurveto{\pgfqpoint{3.990016in}{1.992542in}}{\pgfqpoint{4.000616in}{1.988151in}}{\pgfqpoint{4.011666in}{1.988151in}}%
\pgfpathclose%
\pgfusepath{stroke,fill}%
\end{pgfscope}%
\begin{pgfscope}%
\pgfpathrectangle{\pgfqpoint{0.800000in}{0.528000in}}{\pgfqpoint{4.960000in}{3.696000in}}%
\pgfusepath{clip}%
\pgfsetbuttcap%
\pgfsetroundjoin%
\definecolor{currentfill}{rgb}{0.000000,0.000000,0.000000}%
\pgfsetfillcolor{currentfill}%
\pgfsetlinewidth{1.003750pt}%
\definecolor{currentstroke}{rgb}{0.000000,0.000000,0.000000}%
\pgfsetstrokecolor{currentstroke}%
\pgfsetdash{}{0pt}%
\pgfpathmoveto{\pgfqpoint{4.011666in}{1.902162in}}%
\pgfpathcurveto{\pgfqpoint{4.022716in}{1.902162in}}{\pgfqpoint{4.033315in}{1.906552in}}{\pgfqpoint{4.041128in}{1.914365in}}%
\pgfpathcurveto{\pgfqpoint{4.048942in}{1.922179in}}{\pgfqpoint{4.053332in}{1.932778in}}{\pgfqpoint{4.053332in}{1.943828in}}%
\pgfpathcurveto{\pgfqpoint{4.053332in}{1.954878in}}{\pgfqpoint{4.048942in}{1.965477in}}{\pgfqpoint{4.041128in}{1.973291in}}%
\pgfpathcurveto{\pgfqpoint{4.033315in}{1.981105in}}{\pgfqpoint{4.022716in}{1.985495in}}{\pgfqpoint{4.011666in}{1.985495in}}%
\pgfpathcurveto{\pgfqpoint{4.000616in}{1.985495in}}{\pgfqpoint{3.990016in}{1.981105in}}{\pgfqpoint{3.982203in}{1.973291in}}%
\pgfpathcurveto{\pgfqpoint{3.974389in}{1.965477in}}{\pgfqpoint{3.969999in}{1.954878in}}{\pgfqpoint{3.969999in}{1.943828in}}%
\pgfpathcurveto{\pgfqpoint{3.969999in}{1.932778in}}{\pgfqpoint{3.974389in}{1.922179in}}{\pgfqpoint{3.982203in}{1.914365in}}%
\pgfpathcurveto{\pgfqpoint{3.990016in}{1.906552in}}{\pgfqpoint{4.000616in}{1.902162in}}{\pgfqpoint{4.011666in}{1.902162in}}%
\pgfpathclose%
\pgfusepath{stroke,fill}%
\end{pgfscope}%
\begin{pgfscope}%
\pgfpathrectangle{\pgfqpoint{0.800000in}{0.528000in}}{\pgfqpoint{4.960000in}{3.696000in}}%
\pgfusepath{clip}%
\pgfsetbuttcap%
\pgfsetroundjoin%
\definecolor{currentfill}{rgb}{0.000000,0.000000,0.000000}%
\pgfsetfillcolor{currentfill}%
\pgfsetlinewidth{1.003750pt}%
\definecolor{currentstroke}{rgb}{0.000000,0.000000,0.000000}%
\pgfsetstrokecolor{currentstroke}%
\pgfsetdash{}{0pt}%
\pgfpathmoveto{\pgfqpoint{4.011666in}{1.773177in}}%
\pgfpathcurveto{\pgfqpoint{4.022716in}{1.773177in}}{\pgfqpoint{4.033315in}{1.777567in}}{\pgfqpoint{4.041128in}{1.785380in}}%
\pgfpathcurveto{\pgfqpoint{4.048942in}{1.793194in}}{\pgfqpoint{4.053332in}{1.803793in}}{\pgfqpoint{4.053332in}{1.814843in}}%
\pgfpathcurveto{\pgfqpoint{4.053332in}{1.825893in}}{\pgfqpoint{4.048942in}{1.836492in}}{\pgfqpoint{4.041128in}{1.844306in}}%
\pgfpathcurveto{\pgfqpoint{4.033315in}{1.852120in}}{\pgfqpoint{4.022716in}{1.856510in}}{\pgfqpoint{4.011666in}{1.856510in}}%
\pgfpathcurveto{\pgfqpoint{4.000616in}{1.856510in}}{\pgfqpoint{3.990016in}{1.852120in}}{\pgfqpoint{3.982203in}{1.844306in}}%
\pgfpathcurveto{\pgfqpoint{3.974389in}{1.836492in}}{\pgfqpoint{3.969999in}{1.825893in}}{\pgfqpoint{3.969999in}{1.814843in}}%
\pgfpathcurveto{\pgfqpoint{3.969999in}{1.803793in}}{\pgfqpoint{3.974389in}{1.793194in}}{\pgfqpoint{3.982203in}{1.785380in}}%
\pgfpathcurveto{\pgfqpoint{3.990016in}{1.777567in}}{\pgfqpoint{4.000616in}{1.773177in}}{\pgfqpoint{4.011666in}{1.773177in}}%
\pgfpathclose%
\pgfusepath{stroke,fill}%
\end{pgfscope}%
\begin{pgfscope}%
\pgfpathrectangle{\pgfqpoint{0.800000in}{0.528000in}}{\pgfqpoint{4.960000in}{3.696000in}}%
\pgfusepath{clip}%
\pgfsetbuttcap%
\pgfsetroundjoin%
\definecolor{currentfill}{rgb}{0.000000,0.000000,0.000000}%
\pgfsetfillcolor{currentfill}%
\pgfsetlinewidth{1.003750pt}%
\definecolor{currentstroke}{rgb}{0.000000,0.000000,0.000000}%
\pgfsetstrokecolor{currentstroke}%
\pgfsetdash{}{0pt}%
\pgfpathmoveto{\pgfqpoint{4.011666in}{1.837669in}}%
\pgfpathcurveto{\pgfqpoint{4.022716in}{1.837669in}}{\pgfqpoint{4.033315in}{1.842059in}}{\pgfqpoint{4.041128in}{1.849873in}}%
\pgfpathcurveto{\pgfqpoint{4.048942in}{1.857687in}}{\pgfqpoint{4.053332in}{1.868286in}}{\pgfqpoint{4.053332in}{1.879336in}}%
\pgfpathcurveto{\pgfqpoint{4.053332in}{1.890386in}}{\pgfqpoint{4.048942in}{1.900985in}}{\pgfqpoint{4.041128in}{1.908798in}}%
\pgfpathcurveto{\pgfqpoint{4.033315in}{1.916612in}}{\pgfqpoint{4.022716in}{1.921002in}}{\pgfqpoint{4.011666in}{1.921002in}}%
\pgfpathcurveto{\pgfqpoint{4.000616in}{1.921002in}}{\pgfqpoint{3.990016in}{1.916612in}}{\pgfqpoint{3.982203in}{1.908798in}}%
\pgfpathcurveto{\pgfqpoint{3.974389in}{1.900985in}}{\pgfqpoint{3.969999in}{1.890386in}}{\pgfqpoint{3.969999in}{1.879336in}}%
\pgfpathcurveto{\pgfqpoint{3.969999in}{1.868286in}}{\pgfqpoint{3.974389in}{1.857687in}}{\pgfqpoint{3.982203in}{1.849873in}}%
\pgfpathcurveto{\pgfqpoint{3.990016in}{1.842059in}}{\pgfqpoint{4.000616in}{1.837669in}}{\pgfqpoint{4.011666in}{1.837669in}}%
\pgfpathclose%
\pgfusepath{stroke,fill}%
\end{pgfscope}%
\begin{pgfscope}%
\pgfpathrectangle{\pgfqpoint{0.800000in}{0.528000in}}{\pgfqpoint{4.960000in}{3.696000in}}%
\pgfusepath{clip}%
\pgfsetbuttcap%
\pgfsetroundjoin%
\definecolor{currentfill}{rgb}{0.000000,0.000000,0.000000}%
\pgfsetfillcolor{currentfill}%
\pgfsetlinewidth{1.003750pt}%
\definecolor{currentstroke}{rgb}{0.000000,0.000000,0.000000}%
\pgfsetstrokecolor{currentstroke}%
\pgfsetdash{}{0pt}%
\pgfpathmoveto{\pgfqpoint{4.011666in}{1.902162in}}%
\pgfpathcurveto{\pgfqpoint{4.022716in}{1.902162in}}{\pgfqpoint{4.033315in}{1.906552in}}{\pgfqpoint{4.041128in}{1.914365in}}%
\pgfpathcurveto{\pgfqpoint{4.048942in}{1.922179in}}{\pgfqpoint{4.053332in}{1.932778in}}{\pgfqpoint{4.053332in}{1.943828in}}%
\pgfpathcurveto{\pgfqpoint{4.053332in}{1.954878in}}{\pgfqpoint{4.048942in}{1.965477in}}{\pgfqpoint{4.041128in}{1.973291in}}%
\pgfpathcurveto{\pgfqpoint{4.033315in}{1.981105in}}{\pgfqpoint{4.022716in}{1.985495in}}{\pgfqpoint{4.011666in}{1.985495in}}%
\pgfpathcurveto{\pgfqpoint{4.000616in}{1.985495in}}{\pgfqpoint{3.990016in}{1.981105in}}{\pgfqpoint{3.982203in}{1.973291in}}%
\pgfpathcurveto{\pgfqpoint{3.974389in}{1.965477in}}{\pgfqpoint{3.969999in}{1.954878in}}{\pgfqpoint{3.969999in}{1.943828in}}%
\pgfpathcurveto{\pgfqpoint{3.969999in}{1.932778in}}{\pgfqpoint{3.974389in}{1.922179in}}{\pgfqpoint{3.982203in}{1.914365in}}%
\pgfpathcurveto{\pgfqpoint{3.990016in}{1.906552in}}{\pgfqpoint{4.000616in}{1.902162in}}{\pgfqpoint{4.011666in}{1.902162in}}%
\pgfpathclose%
\pgfusepath{stroke,fill}%
\end{pgfscope}%
\begin{pgfscope}%
\pgfpathrectangle{\pgfqpoint{0.800000in}{0.528000in}}{\pgfqpoint{4.960000in}{3.696000in}}%
\pgfusepath{clip}%
\pgfsetbuttcap%
\pgfsetroundjoin%
\definecolor{currentfill}{rgb}{0.000000,0.000000,0.000000}%
\pgfsetfillcolor{currentfill}%
\pgfsetlinewidth{1.003750pt}%
\definecolor{currentstroke}{rgb}{0.000000,0.000000,0.000000}%
\pgfsetstrokecolor{currentstroke}%
\pgfsetdash{}{0pt}%
\pgfpathmoveto{\pgfqpoint{4.011666in}{1.923659in}}%
\pgfpathcurveto{\pgfqpoint{4.022716in}{1.923659in}}{\pgfqpoint{4.033315in}{1.928049in}}{\pgfqpoint{4.041128in}{1.935863in}}%
\pgfpathcurveto{\pgfqpoint{4.048942in}{1.943677in}}{\pgfqpoint{4.053332in}{1.954276in}}{\pgfqpoint{4.053332in}{1.965326in}}%
\pgfpathcurveto{\pgfqpoint{4.053332in}{1.976376in}}{\pgfqpoint{4.048942in}{1.986975in}}{\pgfqpoint{4.041128in}{1.994788in}}%
\pgfpathcurveto{\pgfqpoint{4.033315in}{2.002602in}}{\pgfqpoint{4.022716in}{2.006992in}}{\pgfqpoint{4.011666in}{2.006992in}}%
\pgfpathcurveto{\pgfqpoint{4.000616in}{2.006992in}}{\pgfqpoint{3.990016in}{2.002602in}}{\pgfqpoint{3.982203in}{1.994788in}}%
\pgfpathcurveto{\pgfqpoint{3.974389in}{1.986975in}}{\pgfqpoint{3.969999in}{1.976376in}}{\pgfqpoint{3.969999in}{1.965326in}}%
\pgfpathcurveto{\pgfqpoint{3.969999in}{1.954276in}}{\pgfqpoint{3.974389in}{1.943677in}}{\pgfqpoint{3.982203in}{1.935863in}}%
\pgfpathcurveto{\pgfqpoint{3.990016in}{1.928049in}}{\pgfqpoint{4.000616in}{1.923659in}}{\pgfqpoint{4.011666in}{1.923659in}}%
\pgfpathclose%
\pgfusepath{stroke,fill}%
\end{pgfscope}%
\begin{pgfscope}%
\pgfpathrectangle{\pgfqpoint{0.800000in}{0.528000in}}{\pgfqpoint{4.960000in}{3.696000in}}%
\pgfusepath{clip}%
\pgfsetbuttcap%
\pgfsetroundjoin%
\definecolor{currentfill}{rgb}{0.000000,0.000000,0.000000}%
\pgfsetfillcolor{currentfill}%
\pgfsetlinewidth{1.003750pt}%
\definecolor{currentstroke}{rgb}{0.000000,0.000000,0.000000}%
\pgfsetstrokecolor{currentstroke}%
\pgfsetdash{}{0pt}%
\pgfpathmoveto{\pgfqpoint{4.011666in}{1.988151in}}%
\pgfpathcurveto{\pgfqpoint{4.022716in}{1.988151in}}{\pgfqpoint{4.033315in}{1.992542in}}{\pgfqpoint{4.041128in}{2.000355in}}%
\pgfpathcurveto{\pgfqpoint{4.048942in}{2.008169in}}{\pgfqpoint{4.053332in}{2.018768in}}{\pgfqpoint{4.053332in}{2.029818in}}%
\pgfpathcurveto{\pgfqpoint{4.053332in}{2.040868in}}{\pgfqpoint{4.048942in}{2.051467in}}{\pgfqpoint{4.041128in}{2.059281in}}%
\pgfpathcurveto{\pgfqpoint{4.033315in}{2.067095in}}{\pgfqpoint{4.022716in}{2.071485in}}{\pgfqpoint{4.011666in}{2.071485in}}%
\pgfpathcurveto{\pgfqpoint{4.000616in}{2.071485in}}{\pgfqpoint{3.990016in}{2.067095in}}{\pgfqpoint{3.982203in}{2.059281in}}%
\pgfpathcurveto{\pgfqpoint{3.974389in}{2.051467in}}{\pgfqpoint{3.969999in}{2.040868in}}{\pgfqpoint{3.969999in}{2.029818in}}%
\pgfpathcurveto{\pgfqpoint{3.969999in}{2.018768in}}{\pgfqpoint{3.974389in}{2.008169in}}{\pgfqpoint{3.982203in}{2.000355in}}%
\pgfpathcurveto{\pgfqpoint{3.990016in}{1.992542in}}{\pgfqpoint{4.000616in}{1.988151in}}{\pgfqpoint{4.011666in}{1.988151in}}%
\pgfpathclose%
\pgfusepath{stroke,fill}%
\end{pgfscope}%
\begin{pgfscope}%
\pgfpathrectangle{\pgfqpoint{0.800000in}{0.528000in}}{\pgfqpoint{4.960000in}{3.696000in}}%
\pgfusepath{clip}%
\pgfsetbuttcap%
\pgfsetroundjoin%
\definecolor{currentfill}{rgb}{0.000000,0.000000,0.000000}%
\pgfsetfillcolor{currentfill}%
\pgfsetlinewidth{1.003750pt}%
\definecolor{currentstroke}{rgb}{0.000000,0.000000,0.000000}%
\pgfsetstrokecolor{currentstroke}%
\pgfsetdash{}{0pt}%
\pgfpathmoveto{\pgfqpoint{4.011666in}{1.945157in}}%
\pgfpathcurveto{\pgfqpoint{4.022716in}{1.945157in}}{\pgfqpoint{4.033315in}{1.949547in}}{\pgfqpoint{4.041128in}{1.957360in}}%
\pgfpathcurveto{\pgfqpoint{4.048942in}{1.965174in}}{\pgfqpoint{4.053332in}{1.975773in}}{\pgfqpoint{4.053332in}{1.986823in}}%
\pgfpathcurveto{\pgfqpoint{4.053332in}{1.997873in}}{\pgfqpoint{4.048942in}{2.008472in}}{\pgfqpoint{4.041128in}{2.016286in}}%
\pgfpathcurveto{\pgfqpoint{4.033315in}{2.024100in}}{\pgfqpoint{4.022716in}{2.028490in}}{\pgfqpoint{4.011666in}{2.028490in}}%
\pgfpathcurveto{\pgfqpoint{4.000616in}{2.028490in}}{\pgfqpoint{3.990016in}{2.024100in}}{\pgfqpoint{3.982203in}{2.016286in}}%
\pgfpathcurveto{\pgfqpoint{3.974389in}{2.008472in}}{\pgfqpoint{3.969999in}{1.997873in}}{\pgfqpoint{3.969999in}{1.986823in}}%
\pgfpathcurveto{\pgfqpoint{3.969999in}{1.975773in}}{\pgfqpoint{3.974389in}{1.965174in}}{\pgfqpoint{3.982203in}{1.957360in}}%
\pgfpathcurveto{\pgfqpoint{3.990016in}{1.949547in}}{\pgfqpoint{4.000616in}{1.945157in}}{\pgfqpoint{4.011666in}{1.945157in}}%
\pgfpathclose%
\pgfusepath{stroke,fill}%
\end{pgfscope}%
\begin{pgfscope}%
\pgfpathrectangle{\pgfqpoint{0.800000in}{0.528000in}}{\pgfqpoint{4.960000in}{3.696000in}}%
\pgfusepath{clip}%
\pgfsetbuttcap%
\pgfsetroundjoin%
\definecolor{currentfill}{rgb}{0.000000,0.000000,0.000000}%
\pgfsetfillcolor{currentfill}%
\pgfsetlinewidth{1.003750pt}%
\definecolor{currentstroke}{rgb}{0.000000,0.000000,0.000000}%
\pgfsetstrokecolor{currentstroke}%
\pgfsetdash{}{0pt}%
\pgfpathmoveto{\pgfqpoint{4.011666in}{1.880664in}}%
\pgfpathcurveto{\pgfqpoint{4.022716in}{1.880664in}}{\pgfqpoint{4.033315in}{1.885054in}}{\pgfqpoint{4.041128in}{1.892868in}}%
\pgfpathcurveto{\pgfqpoint{4.048942in}{1.900682in}}{\pgfqpoint{4.053332in}{1.911281in}}{\pgfqpoint{4.053332in}{1.922331in}}%
\pgfpathcurveto{\pgfqpoint{4.053332in}{1.933381in}}{\pgfqpoint{4.048942in}{1.943980in}}{\pgfqpoint{4.041128in}{1.951793in}}%
\pgfpathcurveto{\pgfqpoint{4.033315in}{1.959607in}}{\pgfqpoint{4.022716in}{1.963997in}}{\pgfqpoint{4.011666in}{1.963997in}}%
\pgfpathcurveto{\pgfqpoint{4.000616in}{1.963997in}}{\pgfqpoint{3.990016in}{1.959607in}}{\pgfqpoint{3.982203in}{1.951793in}}%
\pgfpathcurveto{\pgfqpoint{3.974389in}{1.943980in}}{\pgfqpoint{3.969999in}{1.933381in}}{\pgfqpoint{3.969999in}{1.922331in}}%
\pgfpathcurveto{\pgfqpoint{3.969999in}{1.911281in}}{\pgfqpoint{3.974389in}{1.900682in}}{\pgfqpoint{3.982203in}{1.892868in}}%
\pgfpathcurveto{\pgfqpoint{3.990016in}{1.885054in}}{\pgfqpoint{4.000616in}{1.880664in}}{\pgfqpoint{4.011666in}{1.880664in}}%
\pgfpathclose%
\pgfusepath{stroke,fill}%
\end{pgfscope}%
\begin{pgfscope}%
\pgfpathrectangle{\pgfqpoint{0.800000in}{0.528000in}}{\pgfqpoint{4.960000in}{3.696000in}}%
\pgfusepath{clip}%
\pgfsetbuttcap%
\pgfsetroundjoin%
\definecolor{currentfill}{rgb}{0.000000,0.000000,0.000000}%
\pgfsetfillcolor{currentfill}%
\pgfsetlinewidth{1.003750pt}%
\definecolor{currentstroke}{rgb}{0.000000,0.000000,0.000000}%
\pgfsetstrokecolor{currentstroke}%
\pgfsetdash{}{0pt}%
\pgfpathmoveto{\pgfqpoint{4.011666in}{1.923659in}}%
\pgfpathcurveto{\pgfqpoint{4.022716in}{1.923659in}}{\pgfqpoint{4.033315in}{1.928049in}}{\pgfqpoint{4.041128in}{1.935863in}}%
\pgfpathcurveto{\pgfqpoint{4.048942in}{1.943677in}}{\pgfqpoint{4.053332in}{1.954276in}}{\pgfqpoint{4.053332in}{1.965326in}}%
\pgfpathcurveto{\pgfqpoint{4.053332in}{1.976376in}}{\pgfqpoint{4.048942in}{1.986975in}}{\pgfqpoint{4.041128in}{1.994788in}}%
\pgfpathcurveto{\pgfqpoint{4.033315in}{2.002602in}}{\pgfqpoint{4.022716in}{2.006992in}}{\pgfqpoint{4.011666in}{2.006992in}}%
\pgfpathcurveto{\pgfqpoint{4.000616in}{2.006992in}}{\pgfqpoint{3.990016in}{2.002602in}}{\pgfqpoint{3.982203in}{1.994788in}}%
\pgfpathcurveto{\pgfqpoint{3.974389in}{1.986975in}}{\pgfqpoint{3.969999in}{1.976376in}}{\pgfqpoint{3.969999in}{1.965326in}}%
\pgfpathcurveto{\pgfqpoint{3.969999in}{1.954276in}}{\pgfqpoint{3.974389in}{1.943677in}}{\pgfqpoint{3.982203in}{1.935863in}}%
\pgfpathcurveto{\pgfqpoint{3.990016in}{1.928049in}}{\pgfqpoint{4.000616in}{1.923659in}}{\pgfqpoint{4.011666in}{1.923659in}}%
\pgfpathclose%
\pgfusepath{stroke,fill}%
\end{pgfscope}%
\begin{pgfscope}%
\pgfpathrectangle{\pgfqpoint{0.800000in}{0.528000in}}{\pgfqpoint{4.960000in}{3.696000in}}%
\pgfusepath{clip}%
\pgfsetbuttcap%
\pgfsetroundjoin%
\definecolor{currentfill}{rgb}{0.000000,0.000000,0.000000}%
\pgfsetfillcolor{currentfill}%
\pgfsetlinewidth{1.003750pt}%
\definecolor{currentstroke}{rgb}{0.000000,0.000000,0.000000}%
\pgfsetstrokecolor{currentstroke}%
\pgfsetdash{}{0pt}%
\pgfpathmoveto{\pgfqpoint{4.011666in}{2.031146in}}%
\pgfpathcurveto{\pgfqpoint{4.022716in}{2.031146in}}{\pgfqpoint{4.033315in}{2.035537in}}{\pgfqpoint{4.041128in}{2.043350in}}%
\pgfpathcurveto{\pgfqpoint{4.048942in}{2.051164in}}{\pgfqpoint{4.053332in}{2.061763in}}{\pgfqpoint{4.053332in}{2.072813in}}%
\pgfpathcurveto{\pgfqpoint{4.053332in}{2.083863in}}{\pgfqpoint{4.048942in}{2.094462in}}{\pgfqpoint{4.041128in}{2.102276in}}%
\pgfpathcurveto{\pgfqpoint{4.033315in}{2.110090in}}{\pgfqpoint{4.022716in}{2.114480in}}{\pgfqpoint{4.011666in}{2.114480in}}%
\pgfpathcurveto{\pgfqpoint{4.000616in}{2.114480in}}{\pgfqpoint{3.990016in}{2.110090in}}{\pgfqpoint{3.982203in}{2.102276in}}%
\pgfpathcurveto{\pgfqpoint{3.974389in}{2.094462in}}{\pgfqpoint{3.969999in}{2.083863in}}{\pgfqpoint{3.969999in}{2.072813in}}%
\pgfpathcurveto{\pgfqpoint{3.969999in}{2.061763in}}{\pgfqpoint{3.974389in}{2.051164in}}{\pgfqpoint{3.982203in}{2.043350in}}%
\pgfpathcurveto{\pgfqpoint{3.990016in}{2.035537in}}{\pgfqpoint{4.000616in}{2.031146in}}{\pgfqpoint{4.011666in}{2.031146in}}%
\pgfpathclose%
\pgfusepath{stroke,fill}%
\end{pgfscope}%
\begin{pgfscope}%
\pgfpathrectangle{\pgfqpoint{0.800000in}{0.528000in}}{\pgfqpoint{4.960000in}{3.696000in}}%
\pgfusepath{clip}%
\pgfsetbuttcap%
\pgfsetroundjoin%
\definecolor{currentfill}{rgb}{0.000000,0.000000,0.000000}%
\pgfsetfillcolor{currentfill}%
\pgfsetlinewidth{1.003750pt}%
\definecolor{currentstroke}{rgb}{0.000000,0.000000,0.000000}%
\pgfsetstrokecolor{currentstroke}%
\pgfsetdash{}{0pt}%
\pgfpathmoveto{\pgfqpoint{4.011666in}{1.902162in}}%
\pgfpathcurveto{\pgfqpoint{4.022716in}{1.902162in}}{\pgfqpoint{4.033315in}{1.906552in}}{\pgfqpoint{4.041128in}{1.914365in}}%
\pgfpathcurveto{\pgfqpoint{4.048942in}{1.922179in}}{\pgfqpoint{4.053332in}{1.932778in}}{\pgfqpoint{4.053332in}{1.943828in}}%
\pgfpathcurveto{\pgfqpoint{4.053332in}{1.954878in}}{\pgfqpoint{4.048942in}{1.965477in}}{\pgfqpoint{4.041128in}{1.973291in}}%
\pgfpathcurveto{\pgfqpoint{4.033315in}{1.981105in}}{\pgfqpoint{4.022716in}{1.985495in}}{\pgfqpoint{4.011666in}{1.985495in}}%
\pgfpathcurveto{\pgfqpoint{4.000616in}{1.985495in}}{\pgfqpoint{3.990016in}{1.981105in}}{\pgfqpoint{3.982203in}{1.973291in}}%
\pgfpathcurveto{\pgfqpoint{3.974389in}{1.965477in}}{\pgfqpoint{3.969999in}{1.954878in}}{\pgfqpoint{3.969999in}{1.943828in}}%
\pgfpathcurveto{\pgfqpoint{3.969999in}{1.932778in}}{\pgfqpoint{3.974389in}{1.922179in}}{\pgfqpoint{3.982203in}{1.914365in}}%
\pgfpathcurveto{\pgfqpoint{3.990016in}{1.906552in}}{\pgfqpoint{4.000616in}{1.902162in}}{\pgfqpoint{4.011666in}{1.902162in}}%
\pgfpathclose%
\pgfusepath{stroke,fill}%
\end{pgfscope}%
\begin{pgfscope}%
\pgfpathrectangle{\pgfqpoint{0.800000in}{0.528000in}}{\pgfqpoint{4.960000in}{3.696000in}}%
\pgfusepath{clip}%
\pgfsetbuttcap%
\pgfsetroundjoin%
\definecolor{currentfill}{rgb}{0.000000,0.000000,0.000000}%
\pgfsetfillcolor{currentfill}%
\pgfsetlinewidth{1.003750pt}%
\definecolor{currentstroke}{rgb}{0.000000,0.000000,0.000000}%
\pgfsetstrokecolor{currentstroke}%
\pgfsetdash{}{0pt}%
\pgfpathmoveto{\pgfqpoint{4.011666in}{1.859167in}}%
\pgfpathcurveto{\pgfqpoint{4.022716in}{1.859167in}}{\pgfqpoint{4.033315in}{1.863557in}}{\pgfqpoint{4.041128in}{1.871370in}}%
\pgfpathcurveto{\pgfqpoint{4.048942in}{1.879184in}}{\pgfqpoint{4.053332in}{1.889783in}}{\pgfqpoint{4.053332in}{1.900833in}}%
\pgfpathcurveto{\pgfqpoint{4.053332in}{1.911883in}}{\pgfqpoint{4.048942in}{1.922482in}}{\pgfqpoint{4.041128in}{1.930296in}}%
\pgfpathcurveto{\pgfqpoint{4.033315in}{1.938110in}}{\pgfqpoint{4.022716in}{1.942500in}}{\pgfqpoint{4.011666in}{1.942500in}}%
\pgfpathcurveto{\pgfqpoint{4.000616in}{1.942500in}}{\pgfqpoint{3.990016in}{1.938110in}}{\pgfqpoint{3.982203in}{1.930296in}}%
\pgfpathcurveto{\pgfqpoint{3.974389in}{1.922482in}}{\pgfqpoint{3.969999in}{1.911883in}}{\pgfqpoint{3.969999in}{1.900833in}}%
\pgfpathcurveto{\pgfqpoint{3.969999in}{1.889783in}}{\pgfqpoint{3.974389in}{1.879184in}}{\pgfqpoint{3.982203in}{1.871370in}}%
\pgfpathcurveto{\pgfqpoint{3.990016in}{1.863557in}}{\pgfqpoint{4.000616in}{1.859167in}}{\pgfqpoint{4.011666in}{1.859167in}}%
\pgfpathclose%
\pgfusepath{stroke,fill}%
\end{pgfscope}%
\begin{pgfscope}%
\pgfpathrectangle{\pgfqpoint{0.800000in}{0.528000in}}{\pgfqpoint{4.960000in}{3.696000in}}%
\pgfusepath{clip}%
\pgfsetbuttcap%
\pgfsetroundjoin%
\definecolor{currentfill}{rgb}{0.000000,0.000000,0.000000}%
\pgfsetfillcolor{currentfill}%
\pgfsetlinewidth{1.003750pt}%
\definecolor{currentstroke}{rgb}{0.000000,0.000000,0.000000}%
\pgfsetstrokecolor{currentstroke}%
\pgfsetdash{}{0pt}%
\pgfpathmoveto{\pgfqpoint{4.011666in}{1.945157in}}%
\pgfpathcurveto{\pgfqpoint{4.022716in}{1.945157in}}{\pgfqpoint{4.033315in}{1.949547in}}{\pgfqpoint{4.041128in}{1.957360in}}%
\pgfpathcurveto{\pgfqpoint{4.048942in}{1.965174in}}{\pgfqpoint{4.053332in}{1.975773in}}{\pgfqpoint{4.053332in}{1.986823in}}%
\pgfpathcurveto{\pgfqpoint{4.053332in}{1.997873in}}{\pgfqpoint{4.048942in}{2.008472in}}{\pgfqpoint{4.041128in}{2.016286in}}%
\pgfpathcurveto{\pgfqpoint{4.033315in}{2.024100in}}{\pgfqpoint{4.022716in}{2.028490in}}{\pgfqpoint{4.011666in}{2.028490in}}%
\pgfpathcurveto{\pgfqpoint{4.000616in}{2.028490in}}{\pgfqpoint{3.990016in}{2.024100in}}{\pgfqpoint{3.982203in}{2.016286in}}%
\pgfpathcurveto{\pgfqpoint{3.974389in}{2.008472in}}{\pgfqpoint{3.969999in}{1.997873in}}{\pgfqpoint{3.969999in}{1.986823in}}%
\pgfpathcurveto{\pgfqpoint{3.969999in}{1.975773in}}{\pgfqpoint{3.974389in}{1.965174in}}{\pgfqpoint{3.982203in}{1.957360in}}%
\pgfpathcurveto{\pgfqpoint{3.990016in}{1.949547in}}{\pgfqpoint{4.000616in}{1.945157in}}{\pgfqpoint{4.011666in}{1.945157in}}%
\pgfpathclose%
\pgfusepath{stroke,fill}%
\end{pgfscope}%
\begin{pgfscope}%
\pgfpathrectangle{\pgfqpoint{0.800000in}{0.528000in}}{\pgfqpoint{4.960000in}{3.696000in}}%
\pgfusepath{clip}%
\pgfsetbuttcap%
\pgfsetroundjoin%
\definecolor{currentfill}{rgb}{0.000000,0.000000,0.000000}%
\pgfsetfillcolor{currentfill}%
\pgfsetlinewidth{1.003750pt}%
\definecolor{currentstroke}{rgb}{0.000000,0.000000,0.000000}%
\pgfsetstrokecolor{currentstroke}%
\pgfsetdash{}{0pt}%
\pgfpathmoveto{\pgfqpoint{4.011666in}{1.902162in}}%
\pgfpathcurveto{\pgfqpoint{4.022716in}{1.902162in}}{\pgfqpoint{4.033315in}{1.906552in}}{\pgfqpoint{4.041128in}{1.914365in}}%
\pgfpathcurveto{\pgfqpoint{4.048942in}{1.922179in}}{\pgfqpoint{4.053332in}{1.932778in}}{\pgfqpoint{4.053332in}{1.943828in}}%
\pgfpathcurveto{\pgfqpoint{4.053332in}{1.954878in}}{\pgfqpoint{4.048942in}{1.965477in}}{\pgfqpoint{4.041128in}{1.973291in}}%
\pgfpathcurveto{\pgfqpoint{4.033315in}{1.981105in}}{\pgfqpoint{4.022716in}{1.985495in}}{\pgfqpoint{4.011666in}{1.985495in}}%
\pgfpathcurveto{\pgfqpoint{4.000616in}{1.985495in}}{\pgfqpoint{3.990016in}{1.981105in}}{\pgfqpoint{3.982203in}{1.973291in}}%
\pgfpathcurveto{\pgfqpoint{3.974389in}{1.965477in}}{\pgfqpoint{3.969999in}{1.954878in}}{\pgfqpoint{3.969999in}{1.943828in}}%
\pgfpathcurveto{\pgfqpoint{3.969999in}{1.932778in}}{\pgfqpoint{3.974389in}{1.922179in}}{\pgfqpoint{3.982203in}{1.914365in}}%
\pgfpathcurveto{\pgfqpoint{3.990016in}{1.906552in}}{\pgfqpoint{4.000616in}{1.902162in}}{\pgfqpoint{4.011666in}{1.902162in}}%
\pgfpathclose%
\pgfusepath{stroke,fill}%
\end{pgfscope}%
\begin{pgfscope}%
\pgfpathrectangle{\pgfqpoint{0.800000in}{0.528000in}}{\pgfqpoint{4.960000in}{3.696000in}}%
\pgfusepath{clip}%
\pgfsetbuttcap%
\pgfsetroundjoin%
\definecolor{currentfill}{rgb}{0.000000,0.000000,0.000000}%
\pgfsetfillcolor{currentfill}%
\pgfsetlinewidth{1.003750pt}%
\definecolor{currentstroke}{rgb}{0.000000,0.000000,0.000000}%
\pgfsetstrokecolor{currentstroke}%
\pgfsetdash{}{0pt}%
\pgfpathmoveto{\pgfqpoint{4.011666in}{1.923659in}}%
\pgfpathcurveto{\pgfqpoint{4.022716in}{1.923659in}}{\pgfqpoint{4.033315in}{1.928049in}}{\pgfqpoint{4.041128in}{1.935863in}}%
\pgfpathcurveto{\pgfqpoint{4.048942in}{1.943677in}}{\pgfqpoint{4.053332in}{1.954276in}}{\pgfqpoint{4.053332in}{1.965326in}}%
\pgfpathcurveto{\pgfqpoint{4.053332in}{1.976376in}}{\pgfqpoint{4.048942in}{1.986975in}}{\pgfqpoint{4.041128in}{1.994788in}}%
\pgfpathcurveto{\pgfqpoint{4.033315in}{2.002602in}}{\pgfqpoint{4.022716in}{2.006992in}}{\pgfqpoint{4.011666in}{2.006992in}}%
\pgfpathcurveto{\pgfqpoint{4.000616in}{2.006992in}}{\pgfqpoint{3.990016in}{2.002602in}}{\pgfqpoint{3.982203in}{1.994788in}}%
\pgfpathcurveto{\pgfqpoint{3.974389in}{1.986975in}}{\pgfqpoint{3.969999in}{1.976376in}}{\pgfqpoint{3.969999in}{1.965326in}}%
\pgfpathcurveto{\pgfqpoint{3.969999in}{1.954276in}}{\pgfqpoint{3.974389in}{1.943677in}}{\pgfqpoint{3.982203in}{1.935863in}}%
\pgfpathcurveto{\pgfqpoint{3.990016in}{1.928049in}}{\pgfqpoint{4.000616in}{1.923659in}}{\pgfqpoint{4.011666in}{1.923659in}}%
\pgfpathclose%
\pgfusepath{stroke,fill}%
\end{pgfscope}%
\begin{pgfscope}%
\pgfpathrectangle{\pgfqpoint{0.800000in}{0.528000in}}{\pgfqpoint{4.960000in}{3.696000in}}%
\pgfusepath{clip}%
\pgfsetbuttcap%
\pgfsetroundjoin%
\definecolor{currentfill}{rgb}{0.000000,0.000000,0.000000}%
\pgfsetfillcolor{currentfill}%
\pgfsetlinewidth{1.003750pt}%
\definecolor{currentstroke}{rgb}{0.000000,0.000000,0.000000}%
\pgfsetstrokecolor{currentstroke}%
\pgfsetdash{}{0pt}%
\pgfpathmoveto{\pgfqpoint{4.011666in}{1.837669in}}%
\pgfpathcurveto{\pgfqpoint{4.022716in}{1.837669in}}{\pgfqpoint{4.033315in}{1.842059in}}{\pgfqpoint{4.041128in}{1.849873in}}%
\pgfpathcurveto{\pgfqpoint{4.048942in}{1.857687in}}{\pgfqpoint{4.053332in}{1.868286in}}{\pgfqpoint{4.053332in}{1.879336in}}%
\pgfpathcurveto{\pgfqpoint{4.053332in}{1.890386in}}{\pgfqpoint{4.048942in}{1.900985in}}{\pgfqpoint{4.041128in}{1.908798in}}%
\pgfpathcurveto{\pgfqpoint{4.033315in}{1.916612in}}{\pgfqpoint{4.022716in}{1.921002in}}{\pgfqpoint{4.011666in}{1.921002in}}%
\pgfpathcurveto{\pgfqpoint{4.000616in}{1.921002in}}{\pgfqpoint{3.990016in}{1.916612in}}{\pgfqpoint{3.982203in}{1.908798in}}%
\pgfpathcurveto{\pgfqpoint{3.974389in}{1.900985in}}{\pgfqpoint{3.969999in}{1.890386in}}{\pgfqpoint{3.969999in}{1.879336in}}%
\pgfpathcurveto{\pgfqpoint{3.969999in}{1.868286in}}{\pgfqpoint{3.974389in}{1.857687in}}{\pgfqpoint{3.982203in}{1.849873in}}%
\pgfpathcurveto{\pgfqpoint{3.990016in}{1.842059in}}{\pgfqpoint{4.000616in}{1.837669in}}{\pgfqpoint{4.011666in}{1.837669in}}%
\pgfpathclose%
\pgfusepath{stroke,fill}%
\end{pgfscope}%
\begin{pgfscope}%
\pgfpathrectangle{\pgfqpoint{0.800000in}{0.528000in}}{\pgfqpoint{4.960000in}{3.696000in}}%
\pgfusepath{clip}%
\pgfsetbuttcap%
\pgfsetroundjoin%
\definecolor{currentfill}{rgb}{0.000000,0.000000,0.000000}%
\pgfsetfillcolor{currentfill}%
\pgfsetlinewidth{1.003750pt}%
\definecolor{currentstroke}{rgb}{0.000000,0.000000,0.000000}%
\pgfsetstrokecolor{currentstroke}%
\pgfsetdash{}{0pt}%
\pgfpathmoveto{\pgfqpoint{4.011666in}{1.773177in}}%
\pgfpathcurveto{\pgfqpoint{4.022716in}{1.773177in}}{\pgfqpoint{4.033315in}{1.777567in}}{\pgfqpoint{4.041128in}{1.785380in}}%
\pgfpathcurveto{\pgfqpoint{4.048942in}{1.793194in}}{\pgfqpoint{4.053332in}{1.803793in}}{\pgfqpoint{4.053332in}{1.814843in}}%
\pgfpathcurveto{\pgfqpoint{4.053332in}{1.825893in}}{\pgfqpoint{4.048942in}{1.836492in}}{\pgfqpoint{4.041128in}{1.844306in}}%
\pgfpathcurveto{\pgfqpoint{4.033315in}{1.852120in}}{\pgfqpoint{4.022716in}{1.856510in}}{\pgfqpoint{4.011666in}{1.856510in}}%
\pgfpathcurveto{\pgfqpoint{4.000616in}{1.856510in}}{\pgfqpoint{3.990016in}{1.852120in}}{\pgfqpoint{3.982203in}{1.844306in}}%
\pgfpathcurveto{\pgfqpoint{3.974389in}{1.836492in}}{\pgfqpoint{3.969999in}{1.825893in}}{\pgfqpoint{3.969999in}{1.814843in}}%
\pgfpathcurveto{\pgfqpoint{3.969999in}{1.803793in}}{\pgfqpoint{3.974389in}{1.793194in}}{\pgfqpoint{3.982203in}{1.785380in}}%
\pgfpathcurveto{\pgfqpoint{3.990016in}{1.777567in}}{\pgfqpoint{4.000616in}{1.773177in}}{\pgfqpoint{4.011666in}{1.773177in}}%
\pgfpathclose%
\pgfusepath{stroke,fill}%
\end{pgfscope}%
\begin{pgfscope}%
\pgfpathrectangle{\pgfqpoint{0.800000in}{0.528000in}}{\pgfqpoint{4.960000in}{3.696000in}}%
\pgfusepath{clip}%
\pgfsetbuttcap%
\pgfsetroundjoin%
\definecolor{currentfill}{rgb}{0.000000,0.000000,0.000000}%
\pgfsetfillcolor{currentfill}%
\pgfsetlinewidth{1.003750pt}%
\definecolor{currentstroke}{rgb}{0.000000,0.000000,0.000000}%
\pgfsetstrokecolor{currentstroke}%
\pgfsetdash{}{0pt}%
\pgfpathmoveto{\pgfqpoint{4.011666in}{1.859167in}}%
\pgfpathcurveto{\pgfqpoint{4.022716in}{1.859167in}}{\pgfqpoint{4.033315in}{1.863557in}}{\pgfqpoint{4.041128in}{1.871370in}}%
\pgfpathcurveto{\pgfqpoint{4.048942in}{1.879184in}}{\pgfqpoint{4.053332in}{1.889783in}}{\pgfqpoint{4.053332in}{1.900833in}}%
\pgfpathcurveto{\pgfqpoint{4.053332in}{1.911883in}}{\pgfqpoint{4.048942in}{1.922482in}}{\pgfqpoint{4.041128in}{1.930296in}}%
\pgfpathcurveto{\pgfqpoint{4.033315in}{1.938110in}}{\pgfqpoint{4.022716in}{1.942500in}}{\pgfqpoint{4.011666in}{1.942500in}}%
\pgfpathcurveto{\pgfqpoint{4.000616in}{1.942500in}}{\pgfqpoint{3.990016in}{1.938110in}}{\pgfqpoint{3.982203in}{1.930296in}}%
\pgfpathcurveto{\pgfqpoint{3.974389in}{1.922482in}}{\pgfqpoint{3.969999in}{1.911883in}}{\pgfqpoint{3.969999in}{1.900833in}}%
\pgfpathcurveto{\pgfqpoint{3.969999in}{1.889783in}}{\pgfqpoint{3.974389in}{1.879184in}}{\pgfqpoint{3.982203in}{1.871370in}}%
\pgfpathcurveto{\pgfqpoint{3.990016in}{1.863557in}}{\pgfqpoint{4.000616in}{1.859167in}}{\pgfqpoint{4.011666in}{1.859167in}}%
\pgfpathclose%
\pgfusepath{stroke,fill}%
\end{pgfscope}%
\begin{pgfscope}%
\pgfpathrectangle{\pgfqpoint{0.800000in}{0.528000in}}{\pgfqpoint{4.960000in}{3.696000in}}%
\pgfusepath{clip}%
\pgfsetbuttcap%
\pgfsetroundjoin%
\definecolor{currentfill}{rgb}{0.000000,0.000000,0.000000}%
\pgfsetfillcolor{currentfill}%
\pgfsetlinewidth{1.003750pt}%
\definecolor{currentstroke}{rgb}{0.000000,0.000000,0.000000}%
\pgfsetstrokecolor{currentstroke}%
\pgfsetdash{}{0pt}%
\pgfpathmoveto{\pgfqpoint{4.011666in}{1.923659in}}%
\pgfpathcurveto{\pgfqpoint{4.022716in}{1.923659in}}{\pgfqpoint{4.033315in}{1.928049in}}{\pgfqpoint{4.041128in}{1.935863in}}%
\pgfpathcurveto{\pgfqpoint{4.048942in}{1.943677in}}{\pgfqpoint{4.053332in}{1.954276in}}{\pgfqpoint{4.053332in}{1.965326in}}%
\pgfpathcurveto{\pgfqpoint{4.053332in}{1.976376in}}{\pgfqpoint{4.048942in}{1.986975in}}{\pgfqpoint{4.041128in}{1.994788in}}%
\pgfpathcurveto{\pgfqpoint{4.033315in}{2.002602in}}{\pgfqpoint{4.022716in}{2.006992in}}{\pgfqpoint{4.011666in}{2.006992in}}%
\pgfpathcurveto{\pgfqpoint{4.000616in}{2.006992in}}{\pgfqpoint{3.990016in}{2.002602in}}{\pgfqpoint{3.982203in}{1.994788in}}%
\pgfpathcurveto{\pgfqpoint{3.974389in}{1.986975in}}{\pgfqpoint{3.969999in}{1.976376in}}{\pgfqpoint{3.969999in}{1.965326in}}%
\pgfpathcurveto{\pgfqpoint{3.969999in}{1.954276in}}{\pgfqpoint{3.974389in}{1.943677in}}{\pgfqpoint{3.982203in}{1.935863in}}%
\pgfpathcurveto{\pgfqpoint{3.990016in}{1.928049in}}{\pgfqpoint{4.000616in}{1.923659in}}{\pgfqpoint{4.011666in}{1.923659in}}%
\pgfpathclose%
\pgfusepath{stroke,fill}%
\end{pgfscope}%
\begin{pgfscope}%
\pgfpathrectangle{\pgfqpoint{0.800000in}{0.528000in}}{\pgfqpoint{4.960000in}{3.696000in}}%
\pgfusepath{clip}%
\pgfsetbuttcap%
\pgfsetroundjoin%
\definecolor{currentfill}{rgb}{0.000000,0.000000,0.000000}%
\pgfsetfillcolor{currentfill}%
\pgfsetlinewidth{1.003750pt}%
\definecolor{currentstroke}{rgb}{0.000000,0.000000,0.000000}%
\pgfsetstrokecolor{currentstroke}%
\pgfsetdash{}{0pt}%
\pgfpathmoveto{\pgfqpoint{4.011666in}{1.859167in}}%
\pgfpathcurveto{\pgfqpoint{4.022716in}{1.859167in}}{\pgfqpoint{4.033315in}{1.863557in}}{\pgfqpoint{4.041128in}{1.871370in}}%
\pgfpathcurveto{\pgfqpoint{4.048942in}{1.879184in}}{\pgfqpoint{4.053332in}{1.889783in}}{\pgfqpoint{4.053332in}{1.900833in}}%
\pgfpathcurveto{\pgfqpoint{4.053332in}{1.911883in}}{\pgfqpoint{4.048942in}{1.922482in}}{\pgfqpoint{4.041128in}{1.930296in}}%
\pgfpathcurveto{\pgfqpoint{4.033315in}{1.938110in}}{\pgfqpoint{4.022716in}{1.942500in}}{\pgfqpoint{4.011666in}{1.942500in}}%
\pgfpathcurveto{\pgfqpoint{4.000616in}{1.942500in}}{\pgfqpoint{3.990016in}{1.938110in}}{\pgfqpoint{3.982203in}{1.930296in}}%
\pgfpathcurveto{\pgfqpoint{3.974389in}{1.922482in}}{\pgfqpoint{3.969999in}{1.911883in}}{\pgfqpoint{3.969999in}{1.900833in}}%
\pgfpathcurveto{\pgfqpoint{3.969999in}{1.889783in}}{\pgfqpoint{3.974389in}{1.879184in}}{\pgfqpoint{3.982203in}{1.871370in}}%
\pgfpathcurveto{\pgfqpoint{3.990016in}{1.863557in}}{\pgfqpoint{4.000616in}{1.859167in}}{\pgfqpoint{4.011666in}{1.859167in}}%
\pgfpathclose%
\pgfusepath{stroke,fill}%
\end{pgfscope}%
\begin{pgfscope}%
\pgfpathrectangle{\pgfqpoint{0.800000in}{0.528000in}}{\pgfqpoint{4.960000in}{3.696000in}}%
\pgfusepath{clip}%
\pgfsetbuttcap%
\pgfsetroundjoin%
\definecolor{currentfill}{rgb}{0.000000,0.000000,0.000000}%
\pgfsetfillcolor{currentfill}%
\pgfsetlinewidth{1.003750pt}%
\definecolor{currentstroke}{rgb}{0.000000,0.000000,0.000000}%
\pgfsetstrokecolor{currentstroke}%
\pgfsetdash{}{0pt}%
\pgfpathmoveto{\pgfqpoint{4.011666in}{1.966654in}}%
\pgfpathcurveto{\pgfqpoint{4.022716in}{1.966654in}}{\pgfqpoint{4.033315in}{1.971044in}}{\pgfqpoint{4.041128in}{1.978858in}}%
\pgfpathcurveto{\pgfqpoint{4.048942in}{1.986672in}}{\pgfqpoint{4.053332in}{1.997271in}}{\pgfqpoint{4.053332in}{2.008321in}}%
\pgfpathcurveto{\pgfqpoint{4.053332in}{2.019371in}}{\pgfqpoint{4.048942in}{2.029970in}}{\pgfqpoint{4.041128in}{2.037783in}}%
\pgfpathcurveto{\pgfqpoint{4.033315in}{2.045597in}}{\pgfqpoint{4.022716in}{2.049987in}}{\pgfqpoint{4.011666in}{2.049987in}}%
\pgfpathcurveto{\pgfqpoint{4.000616in}{2.049987in}}{\pgfqpoint{3.990016in}{2.045597in}}{\pgfqpoint{3.982203in}{2.037783in}}%
\pgfpathcurveto{\pgfqpoint{3.974389in}{2.029970in}}{\pgfqpoint{3.969999in}{2.019371in}}{\pgfqpoint{3.969999in}{2.008321in}}%
\pgfpathcurveto{\pgfqpoint{3.969999in}{1.997271in}}{\pgfqpoint{3.974389in}{1.986672in}}{\pgfqpoint{3.982203in}{1.978858in}}%
\pgfpathcurveto{\pgfqpoint{3.990016in}{1.971044in}}{\pgfqpoint{4.000616in}{1.966654in}}{\pgfqpoint{4.011666in}{1.966654in}}%
\pgfpathclose%
\pgfusepath{stroke,fill}%
\end{pgfscope}%
\begin{pgfscope}%
\pgfpathrectangle{\pgfqpoint{0.800000in}{0.528000in}}{\pgfqpoint{4.960000in}{3.696000in}}%
\pgfusepath{clip}%
\pgfsetbuttcap%
\pgfsetroundjoin%
\definecolor{currentfill}{rgb}{0.000000,0.000000,0.000000}%
\pgfsetfillcolor{currentfill}%
\pgfsetlinewidth{1.003750pt}%
\definecolor{currentstroke}{rgb}{0.000000,0.000000,0.000000}%
\pgfsetstrokecolor{currentstroke}%
\pgfsetdash{}{0pt}%
\pgfpathmoveto{\pgfqpoint{4.011666in}{2.074141in}}%
\pgfpathcurveto{\pgfqpoint{4.022716in}{2.074141in}}{\pgfqpoint{4.033315in}{2.078532in}}{\pgfqpoint{4.041128in}{2.086345in}}%
\pgfpathcurveto{\pgfqpoint{4.048942in}{2.094159in}}{\pgfqpoint{4.053332in}{2.104758in}}{\pgfqpoint{4.053332in}{2.115808in}}%
\pgfpathcurveto{\pgfqpoint{4.053332in}{2.126858in}}{\pgfqpoint{4.048942in}{2.137457in}}{\pgfqpoint{4.041128in}{2.145271in}}%
\pgfpathcurveto{\pgfqpoint{4.033315in}{2.153085in}}{\pgfqpoint{4.022716in}{2.157475in}}{\pgfqpoint{4.011666in}{2.157475in}}%
\pgfpathcurveto{\pgfqpoint{4.000616in}{2.157475in}}{\pgfqpoint{3.990016in}{2.153085in}}{\pgfqpoint{3.982203in}{2.145271in}}%
\pgfpathcurveto{\pgfqpoint{3.974389in}{2.137457in}}{\pgfqpoint{3.969999in}{2.126858in}}{\pgfqpoint{3.969999in}{2.115808in}}%
\pgfpathcurveto{\pgfqpoint{3.969999in}{2.104758in}}{\pgfqpoint{3.974389in}{2.094159in}}{\pgfqpoint{3.982203in}{2.086345in}}%
\pgfpathcurveto{\pgfqpoint{3.990016in}{2.078532in}}{\pgfqpoint{4.000616in}{2.074141in}}{\pgfqpoint{4.011666in}{2.074141in}}%
\pgfpathclose%
\pgfusepath{stroke,fill}%
\end{pgfscope}%
\begin{pgfscope}%
\pgfpathrectangle{\pgfqpoint{0.800000in}{0.528000in}}{\pgfqpoint{4.960000in}{3.696000in}}%
\pgfusepath{clip}%
\pgfsetbuttcap%
\pgfsetroundjoin%
\definecolor{currentfill}{rgb}{0.000000,0.000000,0.000000}%
\pgfsetfillcolor{currentfill}%
\pgfsetlinewidth{1.003750pt}%
\definecolor{currentstroke}{rgb}{0.000000,0.000000,0.000000}%
\pgfsetstrokecolor{currentstroke}%
\pgfsetdash{}{0pt}%
\pgfpathmoveto{\pgfqpoint{4.011666in}{1.880664in}}%
\pgfpathcurveto{\pgfqpoint{4.022716in}{1.880664in}}{\pgfqpoint{4.033315in}{1.885054in}}{\pgfqpoint{4.041128in}{1.892868in}}%
\pgfpathcurveto{\pgfqpoint{4.048942in}{1.900682in}}{\pgfqpoint{4.053332in}{1.911281in}}{\pgfqpoint{4.053332in}{1.922331in}}%
\pgfpathcurveto{\pgfqpoint{4.053332in}{1.933381in}}{\pgfqpoint{4.048942in}{1.943980in}}{\pgfqpoint{4.041128in}{1.951793in}}%
\pgfpathcurveto{\pgfqpoint{4.033315in}{1.959607in}}{\pgfqpoint{4.022716in}{1.963997in}}{\pgfqpoint{4.011666in}{1.963997in}}%
\pgfpathcurveto{\pgfqpoint{4.000616in}{1.963997in}}{\pgfqpoint{3.990016in}{1.959607in}}{\pgfqpoint{3.982203in}{1.951793in}}%
\pgfpathcurveto{\pgfqpoint{3.974389in}{1.943980in}}{\pgfqpoint{3.969999in}{1.933381in}}{\pgfqpoint{3.969999in}{1.922331in}}%
\pgfpathcurveto{\pgfqpoint{3.969999in}{1.911281in}}{\pgfqpoint{3.974389in}{1.900682in}}{\pgfqpoint{3.982203in}{1.892868in}}%
\pgfpathcurveto{\pgfqpoint{3.990016in}{1.885054in}}{\pgfqpoint{4.000616in}{1.880664in}}{\pgfqpoint{4.011666in}{1.880664in}}%
\pgfpathclose%
\pgfusepath{stroke,fill}%
\end{pgfscope}%
\begin{pgfscope}%
\pgfpathrectangle{\pgfqpoint{0.800000in}{0.528000in}}{\pgfqpoint{4.960000in}{3.696000in}}%
\pgfusepath{clip}%
\pgfsetbuttcap%
\pgfsetroundjoin%
\definecolor{currentfill}{rgb}{0.000000,0.000000,0.000000}%
\pgfsetfillcolor{currentfill}%
\pgfsetlinewidth{1.003750pt}%
\definecolor{currentstroke}{rgb}{0.000000,0.000000,0.000000}%
\pgfsetstrokecolor{currentstroke}%
\pgfsetdash{}{0pt}%
\pgfpathmoveto{\pgfqpoint{4.011666in}{1.816172in}}%
\pgfpathcurveto{\pgfqpoint{4.022716in}{1.816172in}}{\pgfqpoint{4.033315in}{1.820562in}}{\pgfqpoint{4.041128in}{1.828375in}}%
\pgfpathcurveto{\pgfqpoint{4.048942in}{1.836189in}}{\pgfqpoint{4.053332in}{1.846788in}}{\pgfqpoint{4.053332in}{1.857838in}}%
\pgfpathcurveto{\pgfqpoint{4.053332in}{1.868888in}}{\pgfqpoint{4.048942in}{1.879487in}}{\pgfqpoint{4.041128in}{1.887301in}}%
\pgfpathcurveto{\pgfqpoint{4.033315in}{1.895115in}}{\pgfqpoint{4.022716in}{1.899505in}}{\pgfqpoint{4.011666in}{1.899505in}}%
\pgfpathcurveto{\pgfqpoint{4.000616in}{1.899505in}}{\pgfqpoint{3.990016in}{1.895115in}}{\pgfqpoint{3.982203in}{1.887301in}}%
\pgfpathcurveto{\pgfqpoint{3.974389in}{1.879487in}}{\pgfqpoint{3.969999in}{1.868888in}}{\pgfqpoint{3.969999in}{1.857838in}}%
\pgfpathcurveto{\pgfqpoint{3.969999in}{1.846788in}}{\pgfqpoint{3.974389in}{1.836189in}}{\pgfqpoint{3.982203in}{1.828375in}}%
\pgfpathcurveto{\pgfqpoint{3.990016in}{1.820562in}}{\pgfqpoint{4.000616in}{1.816172in}}{\pgfqpoint{4.011666in}{1.816172in}}%
\pgfpathclose%
\pgfusepath{stroke,fill}%
\end{pgfscope}%
\begin{pgfscope}%
\pgfpathrectangle{\pgfqpoint{0.800000in}{0.528000in}}{\pgfqpoint{4.960000in}{3.696000in}}%
\pgfusepath{clip}%
\pgfsetbuttcap%
\pgfsetroundjoin%
\definecolor{currentfill}{rgb}{0.000000,0.000000,0.000000}%
\pgfsetfillcolor{currentfill}%
\pgfsetlinewidth{1.003750pt}%
\definecolor{currentstroke}{rgb}{0.000000,0.000000,0.000000}%
\pgfsetstrokecolor{currentstroke}%
\pgfsetdash{}{0pt}%
\pgfpathmoveto{\pgfqpoint{4.011666in}{1.730182in}}%
\pgfpathcurveto{\pgfqpoint{4.022716in}{1.730182in}}{\pgfqpoint{4.033315in}{1.734572in}}{\pgfqpoint{4.041128in}{1.742385in}}%
\pgfpathcurveto{\pgfqpoint{4.048942in}{1.750199in}}{\pgfqpoint{4.053332in}{1.760798in}}{\pgfqpoint{4.053332in}{1.771848in}}%
\pgfpathcurveto{\pgfqpoint{4.053332in}{1.782898in}}{\pgfqpoint{4.048942in}{1.793497in}}{\pgfqpoint{4.041128in}{1.801311in}}%
\pgfpathcurveto{\pgfqpoint{4.033315in}{1.809125in}}{\pgfqpoint{4.022716in}{1.813515in}}{\pgfqpoint{4.011666in}{1.813515in}}%
\pgfpathcurveto{\pgfqpoint{4.000616in}{1.813515in}}{\pgfqpoint{3.990016in}{1.809125in}}{\pgfqpoint{3.982203in}{1.801311in}}%
\pgfpathcurveto{\pgfqpoint{3.974389in}{1.793497in}}{\pgfqpoint{3.969999in}{1.782898in}}{\pgfqpoint{3.969999in}{1.771848in}}%
\pgfpathcurveto{\pgfqpoint{3.969999in}{1.760798in}}{\pgfqpoint{3.974389in}{1.750199in}}{\pgfqpoint{3.982203in}{1.742385in}}%
\pgfpathcurveto{\pgfqpoint{3.990016in}{1.734572in}}{\pgfqpoint{4.000616in}{1.730182in}}{\pgfqpoint{4.011666in}{1.730182in}}%
\pgfpathclose%
\pgfusepath{stroke,fill}%
\end{pgfscope}%
\begin{pgfscope}%
\pgfpathrectangle{\pgfqpoint{0.800000in}{0.528000in}}{\pgfqpoint{4.960000in}{3.696000in}}%
\pgfusepath{clip}%
\pgfsetbuttcap%
\pgfsetroundjoin%
\definecolor{currentfill}{rgb}{0.000000,0.000000,0.000000}%
\pgfsetfillcolor{currentfill}%
\pgfsetlinewidth{1.003750pt}%
\definecolor{currentstroke}{rgb}{0.000000,0.000000,0.000000}%
\pgfsetstrokecolor{currentstroke}%
\pgfsetdash{}{0pt}%
\pgfpathmoveto{\pgfqpoint{4.011666in}{2.052644in}}%
\pgfpathcurveto{\pgfqpoint{4.022716in}{2.052644in}}{\pgfqpoint{4.033315in}{2.057034in}}{\pgfqpoint{4.041128in}{2.064848in}}%
\pgfpathcurveto{\pgfqpoint{4.048942in}{2.072661in}}{\pgfqpoint{4.053332in}{2.083261in}}{\pgfqpoint{4.053332in}{2.094311in}}%
\pgfpathcurveto{\pgfqpoint{4.053332in}{2.105361in}}{\pgfqpoint{4.048942in}{2.115960in}}{\pgfqpoint{4.041128in}{2.123773in}}%
\pgfpathcurveto{\pgfqpoint{4.033315in}{2.131587in}}{\pgfqpoint{4.022716in}{2.135977in}}{\pgfqpoint{4.011666in}{2.135977in}}%
\pgfpathcurveto{\pgfqpoint{4.000616in}{2.135977in}}{\pgfqpoint{3.990016in}{2.131587in}}{\pgfqpoint{3.982203in}{2.123773in}}%
\pgfpathcurveto{\pgfqpoint{3.974389in}{2.115960in}}{\pgfqpoint{3.969999in}{2.105361in}}{\pgfqpoint{3.969999in}{2.094311in}}%
\pgfpathcurveto{\pgfqpoint{3.969999in}{2.083261in}}{\pgfqpoint{3.974389in}{2.072661in}}{\pgfqpoint{3.982203in}{2.064848in}}%
\pgfpathcurveto{\pgfqpoint{3.990016in}{2.057034in}}{\pgfqpoint{4.000616in}{2.052644in}}{\pgfqpoint{4.011666in}{2.052644in}}%
\pgfpathclose%
\pgfusepath{stroke,fill}%
\end{pgfscope}%
\begin{pgfscope}%
\pgfpathrectangle{\pgfqpoint{0.800000in}{0.528000in}}{\pgfqpoint{4.960000in}{3.696000in}}%
\pgfusepath{clip}%
\pgfsetbuttcap%
\pgfsetroundjoin%
\definecolor{currentfill}{rgb}{0.000000,0.000000,0.000000}%
\pgfsetfillcolor{currentfill}%
\pgfsetlinewidth{1.003750pt}%
\definecolor{currentstroke}{rgb}{0.000000,0.000000,0.000000}%
\pgfsetstrokecolor{currentstroke}%
\pgfsetdash{}{0pt}%
\pgfpathmoveto{\pgfqpoint{4.011666in}{1.794674in}}%
\pgfpathcurveto{\pgfqpoint{4.022716in}{1.794674in}}{\pgfqpoint{4.033315in}{1.799064in}}{\pgfqpoint{4.041128in}{1.806878in}}%
\pgfpathcurveto{\pgfqpoint{4.048942in}{1.814692in}}{\pgfqpoint{4.053332in}{1.825291in}}{\pgfqpoint{4.053332in}{1.836341in}}%
\pgfpathcurveto{\pgfqpoint{4.053332in}{1.847391in}}{\pgfqpoint{4.048942in}{1.857990in}}{\pgfqpoint{4.041128in}{1.865804in}}%
\pgfpathcurveto{\pgfqpoint{4.033315in}{1.873617in}}{\pgfqpoint{4.022716in}{1.878007in}}{\pgfqpoint{4.011666in}{1.878007in}}%
\pgfpathcurveto{\pgfqpoint{4.000616in}{1.878007in}}{\pgfqpoint{3.990016in}{1.873617in}}{\pgfqpoint{3.982203in}{1.865804in}}%
\pgfpathcurveto{\pgfqpoint{3.974389in}{1.857990in}}{\pgfqpoint{3.969999in}{1.847391in}}{\pgfqpoint{3.969999in}{1.836341in}}%
\pgfpathcurveto{\pgfqpoint{3.969999in}{1.825291in}}{\pgfqpoint{3.974389in}{1.814692in}}{\pgfqpoint{3.982203in}{1.806878in}}%
\pgfpathcurveto{\pgfqpoint{3.990016in}{1.799064in}}{\pgfqpoint{4.000616in}{1.794674in}}{\pgfqpoint{4.011666in}{1.794674in}}%
\pgfpathclose%
\pgfusepath{stroke,fill}%
\end{pgfscope}%
\begin{pgfscope}%
\pgfpathrectangle{\pgfqpoint{0.800000in}{0.528000in}}{\pgfqpoint{4.960000in}{3.696000in}}%
\pgfusepath{clip}%
\pgfsetbuttcap%
\pgfsetroundjoin%
\definecolor{currentfill}{rgb}{0.000000,0.000000,0.000000}%
\pgfsetfillcolor{currentfill}%
\pgfsetlinewidth{1.003750pt}%
\definecolor{currentstroke}{rgb}{0.000000,0.000000,0.000000}%
\pgfsetstrokecolor{currentstroke}%
\pgfsetdash{}{0pt}%
\pgfpathmoveto{\pgfqpoint{4.011666in}{2.052644in}}%
\pgfpathcurveto{\pgfqpoint{4.022716in}{2.052644in}}{\pgfqpoint{4.033315in}{2.057034in}}{\pgfqpoint{4.041128in}{2.064848in}}%
\pgfpathcurveto{\pgfqpoint{4.048942in}{2.072661in}}{\pgfqpoint{4.053332in}{2.083261in}}{\pgfqpoint{4.053332in}{2.094311in}}%
\pgfpathcurveto{\pgfqpoint{4.053332in}{2.105361in}}{\pgfqpoint{4.048942in}{2.115960in}}{\pgfqpoint{4.041128in}{2.123773in}}%
\pgfpathcurveto{\pgfqpoint{4.033315in}{2.131587in}}{\pgfqpoint{4.022716in}{2.135977in}}{\pgfqpoint{4.011666in}{2.135977in}}%
\pgfpathcurveto{\pgfqpoint{4.000616in}{2.135977in}}{\pgfqpoint{3.990016in}{2.131587in}}{\pgfqpoint{3.982203in}{2.123773in}}%
\pgfpathcurveto{\pgfqpoint{3.974389in}{2.115960in}}{\pgfqpoint{3.969999in}{2.105361in}}{\pgfqpoint{3.969999in}{2.094311in}}%
\pgfpathcurveto{\pgfqpoint{3.969999in}{2.083261in}}{\pgfqpoint{3.974389in}{2.072661in}}{\pgfqpoint{3.982203in}{2.064848in}}%
\pgfpathcurveto{\pgfqpoint{3.990016in}{2.057034in}}{\pgfqpoint{4.000616in}{2.052644in}}{\pgfqpoint{4.011666in}{2.052644in}}%
\pgfpathclose%
\pgfusepath{stroke,fill}%
\end{pgfscope}%
\begin{pgfscope}%
\pgfpathrectangle{\pgfqpoint{0.800000in}{0.528000in}}{\pgfqpoint{4.960000in}{3.696000in}}%
\pgfusepath{clip}%
\pgfsetbuttcap%
\pgfsetroundjoin%
\definecolor{currentfill}{rgb}{0.000000,0.000000,0.000000}%
\pgfsetfillcolor{currentfill}%
\pgfsetlinewidth{1.003750pt}%
\definecolor{currentstroke}{rgb}{0.000000,0.000000,0.000000}%
\pgfsetstrokecolor{currentstroke}%
\pgfsetdash{}{0pt}%
\pgfpathmoveto{\pgfqpoint{4.011666in}{2.031146in}}%
\pgfpathcurveto{\pgfqpoint{4.022716in}{2.031146in}}{\pgfqpoint{4.033315in}{2.035537in}}{\pgfqpoint{4.041128in}{2.043350in}}%
\pgfpathcurveto{\pgfqpoint{4.048942in}{2.051164in}}{\pgfqpoint{4.053332in}{2.061763in}}{\pgfqpoint{4.053332in}{2.072813in}}%
\pgfpathcurveto{\pgfqpoint{4.053332in}{2.083863in}}{\pgfqpoint{4.048942in}{2.094462in}}{\pgfqpoint{4.041128in}{2.102276in}}%
\pgfpathcurveto{\pgfqpoint{4.033315in}{2.110090in}}{\pgfqpoint{4.022716in}{2.114480in}}{\pgfqpoint{4.011666in}{2.114480in}}%
\pgfpathcurveto{\pgfqpoint{4.000616in}{2.114480in}}{\pgfqpoint{3.990016in}{2.110090in}}{\pgfqpoint{3.982203in}{2.102276in}}%
\pgfpathcurveto{\pgfqpoint{3.974389in}{2.094462in}}{\pgfqpoint{3.969999in}{2.083863in}}{\pgfqpoint{3.969999in}{2.072813in}}%
\pgfpathcurveto{\pgfqpoint{3.969999in}{2.061763in}}{\pgfqpoint{3.974389in}{2.051164in}}{\pgfqpoint{3.982203in}{2.043350in}}%
\pgfpathcurveto{\pgfqpoint{3.990016in}{2.035537in}}{\pgfqpoint{4.000616in}{2.031146in}}{\pgfqpoint{4.011666in}{2.031146in}}%
\pgfpathclose%
\pgfusepath{stroke,fill}%
\end{pgfscope}%
\begin{pgfscope}%
\pgfpathrectangle{\pgfqpoint{0.800000in}{0.528000in}}{\pgfqpoint{4.960000in}{3.696000in}}%
\pgfusepath{clip}%
\pgfsetbuttcap%
\pgfsetroundjoin%
\definecolor{currentfill}{rgb}{0.000000,0.000000,0.000000}%
\pgfsetfillcolor{currentfill}%
\pgfsetlinewidth{1.003750pt}%
\definecolor{currentstroke}{rgb}{0.000000,0.000000,0.000000}%
\pgfsetstrokecolor{currentstroke}%
\pgfsetdash{}{0pt}%
\pgfpathmoveto{\pgfqpoint{4.011666in}{1.816172in}}%
\pgfpathcurveto{\pgfqpoint{4.022716in}{1.816172in}}{\pgfqpoint{4.033315in}{1.820562in}}{\pgfqpoint{4.041128in}{1.828375in}}%
\pgfpathcurveto{\pgfqpoint{4.048942in}{1.836189in}}{\pgfqpoint{4.053332in}{1.846788in}}{\pgfqpoint{4.053332in}{1.857838in}}%
\pgfpathcurveto{\pgfqpoint{4.053332in}{1.868888in}}{\pgfqpoint{4.048942in}{1.879487in}}{\pgfqpoint{4.041128in}{1.887301in}}%
\pgfpathcurveto{\pgfqpoint{4.033315in}{1.895115in}}{\pgfqpoint{4.022716in}{1.899505in}}{\pgfqpoint{4.011666in}{1.899505in}}%
\pgfpathcurveto{\pgfqpoint{4.000616in}{1.899505in}}{\pgfqpoint{3.990016in}{1.895115in}}{\pgfqpoint{3.982203in}{1.887301in}}%
\pgfpathcurveto{\pgfqpoint{3.974389in}{1.879487in}}{\pgfqpoint{3.969999in}{1.868888in}}{\pgfqpoint{3.969999in}{1.857838in}}%
\pgfpathcurveto{\pgfqpoint{3.969999in}{1.846788in}}{\pgfqpoint{3.974389in}{1.836189in}}{\pgfqpoint{3.982203in}{1.828375in}}%
\pgfpathcurveto{\pgfqpoint{3.990016in}{1.820562in}}{\pgfqpoint{4.000616in}{1.816172in}}{\pgfqpoint{4.011666in}{1.816172in}}%
\pgfpathclose%
\pgfusepath{stroke,fill}%
\end{pgfscope}%
\begin{pgfscope}%
\pgfpathrectangle{\pgfqpoint{0.800000in}{0.528000in}}{\pgfqpoint{4.960000in}{3.696000in}}%
\pgfusepath{clip}%
\pgfsetbuttcap%
\pgfsetroundjoin%
\definecolor{currentfill}{rgb}{0.000000,0.000000,0.000000}%
\pgfsetfillcolor{currentfill}%
\pgfsetlinewidth{1.003750pt}%
\definecolor{currentstroke}{rgb}{0.000000,0.000000,0.000000}%
\pgfsetstrokecolor{currentstroke}%
\pgfsetdash{}{0pt}%
\pgfpathmoveto{\pgfqpoint{4.011666in}{1.794674in}}%
\pgfpathcurveto{\pgfqpoint{4.022716in}{1.794674in}}{\pgfqpoint{4.033315in}{1.799064in}}{\pgfqpoint{4.041128in}{1.806878in}}%
\pgfpathcurveto{\pgfqpoint{4.048942in}{1.814692in}}{\pgfqpoint{4.053332in}{1.825291in}}{\pgfqpoint{4.053332in}{1.836341in}}%
\pgfpathcurveto{\pgfqpoint{4.053332in}{1.847391in}}{\pgfqpoint{4.048942in}{1.857990in}}{\pgfqpoint{4.041128in}{1.865804in}}%
\pgfpathcurveto{\pgfqpoint{4.033315in}{1.873617in}}{\pgfqpoint{4.022716in}{1.878007in}}{\pgfqpoint{4.011666in}{1.878007in}}%
\pgfpathcurveto{\pgfqpoint{4.000616in}{1.878007in}}{\pgfqpoint{3.990016in}{1.873617in}}{\pgfqpoint{3.982203in}{1.865804in}}%
\pgfpathcurveto{\pgfqpoint{3.974389in}{1.857990in}}{\pgfqpoint{3.969999in}{1.847391in}}{\pgfqpoint{3.969999in}{1.836341in}}%
\pgfpathcurveto{\pgfqpoint{3.969999in}{1.825291in}}{\pgfqpoint{3.974389in}{1.814692in}}{\pgfqpoint{3.982203in}{1.806878in}}%
\pgfpathcurveto{\pgfqpoint{3.990016in}{1.799064in}}{\pgfqpoint{4.000616in}{1.794674in}}{\pgfqpoint{4.011666in}{1.794674in}}%
\pgfpathclose%
\pgfusepath{stroke,fill}%
\end{pgfscope}%
\begin{pgfscope}%
\pgfpathrectangle{\pgfqpoint{0.800000in}{0.528000in}}{\pgfqpoint{4.960000in}{3.696000in}}%
\pgfusepath{clip}%
\pgfsetbuttcap%
\pgfsetroundjoin%
\definecolor{currentfill}{rgb}{0.000000,0.000000,0.000000}%
\pgfsetfillcolor{currentfill}%
\pgfsetlinewidth{1.003750pt}%
\definecolor{currentstroke}{rgb}{0.000000,0.000000,0.000000}%
\pgfsetstrokecolor{currentstroke}%
\pgfsetdash{}{0pt}%
\pgfpathmoveto{\pgfqpoint{4.011666in}{2.375106in}}%
\pgfpathcurveto{\pgfqpoint{4.022716in}{2.375106in}}{\pgfqpoint{4.033315in}{2.379497in}}{\pgfqpoint{4.041128in}{2.387310in}}%
\pgfpathcurveto{\pgfqpoint{4.048942in}{2.395124in}}{\pgfqpoint{4.053332in}{2.405723in}}{\pgfqpoint{4.053332in}{2.416773in}}%
\pgfpathcurveto{\pgfqpoint{4.053332in}{2.427823in}}{\pgfqpoint{4.048942in}{2.438422in}}{\pgfqpoint{4.041128in}{2.446236in}}%
\pgfpathcurveto{\pgfqpoint{4.033315in}{2.454049in}}{\pgfqpoint{4.022716in}{2.458440in}}{\pgfqpoint{4.011666in}{2.458440in}}%
\pgfpathcurveto{\pgfqpoint{4.000616in}{2.458440in}}{\pgfqpoint{3.990016in}{2.454049in}}{\pgfqpoint{3.982203in}{2.446236in}}%
\pgfpathcurveto{\pgfqpoint{3.974389in}{2.438422in}}{\pgfqpoint{3.969999in}{2.427823in}}{\pgfqpoint{3.969999in}{2.416773in}}%
\pgfpathcurveto{\pgfqpoint{3.969999in}{2.405723in}}{\pgfqpoint{3.974389in}{2.395124in}}{\pgfqpoint{3.982203in}{2.387310in}}%
\pgfpathcurveto{\pgfqpoint{3.990016in}{2.379497in}}{\pgfqpoint{4.000616in}{2.375106in}}{\pgfqpoint{4.011666in}{2.375106in}}%
\pgfpathclose%
\pgfusepath{stroke,fill}%
\end{pgfscope}%
\begin{pgfscope}%
\pgfpathrectangle{\pgfqpoint{0.800000in}{0.528000in}}{\pgfqpoint{4.960000in}{3.696000in}}%
\pgfusepath{clip}%
\pgfsetbuttcap%
\pgfsetroundjoin%
\definecolor{currentfill}{rgb}{0.000000,0.000000,0.000000}%
\pgfsetfillcolor{currentfill}%
\pgfsetlinewidth{1.003750pt}%
\definecolor{currentstroke}{rgb}{0.000000,0.000000,0.000000}%
\pgfsetstrokecolor{currentstroke}%
\pgfsetdash{}{0pt}%
\pgfpathmoveto{\pgfqpoint{4.011666in}{1.880664in}}%
\pgfpathcurveto{\pgfqpoint{4.022716in}{1.880664in}}{\pgfqpoint{4.033315in}{1.885054in}}{\pgfqpoint{4.041128in}{1.892868in}}%
\pgfpathcurveto{\pgfqpoint{4.048942in}{1.900682in}}{\pgfqpoint{4.053332in}{1.911281in}}{\pgfqpoint{4.053332in}{1.922331in}}%
\pgfpathcurveto{\pgfqpoint{4.053332in}{1.933381in}}{\pgfqpoint{4.048942in}{1.943980in}}{\pgfqpoint{4.041128in}{1.951793in}}%
\pgfpathcurveto{\pgfqpoint{4.033315in}{1.959607in}}{\pgfqpoint{4.022716in}{1.963997in}}{\pgfqpoint{4.011666in}{1.963997in}}%
\pgfpathcurveto{\pgfqpoint{4.000616in}{1.963997in}}{\pgfqpoint{3.990016in}{1.959607in}}{\pgfqpoint{3.982203in}{1.951793in}}%
\pgfpathcurveto{\pgfqpoint{3.974389in}{1.943980in}}{\pgfqpoint{3.969999in}{1.933381in}}{\pgfqpoint{3.969999in}{1.922331in}}%
\pgfpathcurveto{\pgfqpoint{3.969999in}{1.911281in}}{\pgfqpoint{3.974389in}{1.900682in}}{\pgfqpoint{3.982203in}{1.892868in}}%
\pgfpathcurveto{\pgfqpoint{3.990016in}{1.885054in}}{\pgfqpoint{4.000616in}{1.880664in}}{\pgfqpoint{4.011666in}{1.880664in}}%
\pgfpathclose%
\pgfusepath{stroke,fill}%
\end{pgfscope}%
\begin{pgfscope}%
\pgfpathrectangle{\pgfqpoint{0.800000in}{0.528000in}}{\pgfqpoint{4.960000in}{3.696000in}}%
\pgfusepath{clip}%
\pgfsetbuttcap%
\pgfsetroundjoin%
\definecolor{currentfill}{rgb}{0.000000,0.000000,0.000000}%
\pgfsetfillcolor{currentfill}%
\pgfsetlinewidth{1.003750pt}%
\definecolor{currentstroke}{rgb}{0.000000,0.000000,0.000000}%
\pgfsetstrokecolor{currentstroke}%
\pgfsetdash{}{0pt}%
\pgfpathmoveto{\pgfqpoint{4.011666in}{1.794674in}}%
\pgfpathcurveto{\pgfqpoint{4.022716in}{1.794674in}}{\pgfqpoint{4.033315in}{1.799064in}}{\pgfqpoint{4.041128in}{1.806878in}}%
\pgfpathcurveto{\pgfqpoint{4.048942in}{1.814692in}}{\pgfqpoint{4.053332in}{1.825291in}}{\pgfqpoint{4.053332in}{1.836341in}}%
\pgfpathcurveto{\pgfqpoint{4.053332in}{1.847391in}}{\pgfqpoint{4.048942in}{1.857990in}}{\pgfqpoint{4.041128in}{1.865804in}}%
\pgfpathcurveto{\pgfqpoint{4.033315in}{1.873617in}}{\pgfqpoint{4.022716in}{1.878007in}}{\pgfqpoint{4.011666in}{1.878007in}}%
\pgfpathcurveto{\pgfqpoint{4.000616in}{1.878007in}}{\pgfqpoint{3.990016in}{1.873617in}}{\pgfqpoint{3.982203in}{1.865804in}}%
\pgfpathcurveto{\pgfqpoint{3.974389in}{1.857990in}}{\pgfqpoint{3.969999in}{1.847391in}}{\pgfqpoint{3.969999in}{1.836341in}}%
\pgfpathcurveto{\pgfqpoint{3.969999in}{1.825291in}}{\pgfqpoint{3.974389in}{1.814692in}}{\pgfqpoint{3.982203in}{1.806878in}}%
\pgfpathcurveto{\pgfqpoint{3.990016in}{1.799064in}}{\pgfqpoint{4.000616in}{1.794674in}}{\pgfqpoint{4.011666in}{1.794674in}}%
\pgfpathclose%
\pgfusepath{stroke,fill}%
\end{pgfscope}%
\begin{pgfscope}%
\pgfpathrectangle{\pgfqpoint{0.800000in}{0.528000in}}{\pgfqpoint{4.960000in}{3.696000in}}%
\pgfusepath{clip}%
\pgfsetbuttcap%
\pgfsetroundjoin%
\definecolor{currentfill}{rgb}{0.000000,0.000000,0.000000}%
\pgfsetfillcolor{currentfill}%
\pgfsetlinewidth{1.003750pt}%
\definecolor{currentstroke}{rgb}{0.000000,0.000000,0.000000}%
\pgfsetstrokecolor{currentstroke}%
\pgfsetdash{}{0pt}%
\pgfpathmoveto{\pgfqpoint{4.011666in}{1.837669in}}%
\pgfpathcurveto{\pgfqpoint{4.022716in}{1.837669in}}{\pgfqpoint{4.033315in}{1.842059in}}{\pgfqpoint{4.041128in}{1.849873in}}%
\pgfpathcurveto{\pgfqpoint{4.048942in}{1.857687in}}{\pgfqpoint{4.053332in}{1.868286in}}{\pgfqpoint{4.053332in}{1.879336in}}%
\pgfpathcurveto{\pgfqpoint{4.053332in}{1.890386in}}{\pgfqpoint{4.048942in}{1.900985in}}{\pgfqpoint{4.041128in}{1.908798in}}%
\pgfpathcurveto{\pgfqpoint{4.033315in}{1.916612in}}{\pgfqpoint{4.022716in}{1.921002in}}{\pgfqpoint{4.011666in}{1.921002in}}%
\pgfpathcurveto{\pgfqpoint{4.000616in}{1.921002in}}{\pgfqpoint{3.990016in}{1.916612in}}{\pgfqpoint{3.982203in}{1.908798in}}%
\pgfpathcurveto{\pgfqpoint{3.974389in}{1.900985in}}{\pgfqpoint{3.969999in}{1.890386in}}{\pgfqpoint{3.969999in}{1.879336in}}%
\pgfpathcurveto{\pgfqpoint{3.969999in}{1.868286in}}{\pgfqpoint{3.974389in}{1.857687in}}{\pgfqpoint{3.982203in}{1.849873in}}%
\pgfpathcurveto{\pgfqpoint{3.990016in}{1.842059in}}{\pgfqpoint{4.000616in}{1.837669in}}{\pgfqpoint{4.011666in}{1.837669in}}%
\pgfpathclose%
\pgfusepath{stroke,fill}%
\end{pgfscope}%
\begin{pgfscope}%
\pgfpathrectangle{\pgfqpoint{0.800000in}{0.528000in}}{\pgfqpoint{4.960000in}{3.696000in}}%
\pgfusepath{clip}%
\pgfsetbuttcap%
\pgfsetroundjoin%
\definecolor{currentfill}{rgb}{0.000000,0.000000,0.000000}%
\pgfsetfillcolor{currentfill}%
\pgfsetlinewidth{1.003750pt}%
\definecolor{currentstroke}{rgb}{0.000000,0.000000,0.000000}%
\pgfsetstrokecolor{currentstroke}%
\pgfsetdash{}{0pt}%
\pgfpathmoveto{\pgfqpoint{4.011666in}{1.794674in}}%
\pgfpathcurveto{\pgfqpoint{4.022716in}{1.794674in}}{\pgfqpoint{4.033315in}{1.799064in}}{\pgfqpoint{4.041128in}{1.806878in}}%
\pgfpathcurveto{\pgfqpoint{4.048942in}{1.814692in}}{\pgfqpoint{4.053332in}{1.825291in}}{\pgfqpoint{4.053332in}{1.836341in}}%
\pgfpathcurveto{\pgfqpoint{4.053332in}{1.847391in}}{\pgfqpoint{4.048942in}{1.857990in}}{\pgfqpoint{4.041128in}{1.865804in}}%
\pgfpathcurveto{\pgfqpoint{4.033315in}{1.873617in}}{\pgfqpoint{4.022716in}{1.878007in}}{\pgfqpoint{4.011666in}{1.878007in}}%
\pgfpathcurveto{\pgfqpoint{4.000616in}{1.878007in}}{\pgfqpoint{3.990016in}{1.873617in}}{\pgfqpoint{3.982203in}{1.865804in}}%
\pgfpathcurveto{\pgfqpoint{3.974389in}{1.857990in}}{\pgfqpoint{3.969999in}{1.847391in}}{\pgfqpoint{3.969999in}{1.836341in}}%
\pgfpathcurveto{\pgfqpoint{3.969999in}{1.825291in}}{\pgfqpoint{3.974389in}{1.814692in}}{\pgfqpoint{3.982203in}{1.806878in}}%
\pgfpathcurveto{\pgfqpoint{3.990016in}{1.799064in}}{\pgfqpoint{4.000616in}{1.794674in}}{\pgfqpoint{4.011666in}{1.794674in}}%
\pgfpathclose%
\pgfusepath{stroke,fill}%
\end{pgfscope}%
\begin{pgfscope}%
\pgfpathrectangle{\pgfqpoint{0.800000in}{0.528000in}}{\pgfqpoint{4.960000in}{3.696000in}}%
\pgfusepath{clip}%
\pgfsetbuttcap%
\pgfsetroundjoin%
\definecolor{currentfill}{rgb}{0.000000,0.000000,0.000000}%
\pgfsetfillcolor{currentfill}%
\pgfsetlinewidth{1.003750pt}%
\definecolor{currentstroke}{rgb}{0.000000,0.000000,0.000000}%
\pgfsetstrokecolor{currentstroke}%
\pgfsetdash{}{0pt}%
\pgfpathmoveto{\pgfqpoint{4.011666in}{2.052644in}}%
\pgfpathcurveto{\pgfqpoint{4.022716in}{2.052644in}}{\pgfqpoint{4.033315in}{2.057034in}}{\pgfqpoint{4.041128in}{2.064848in}}%
\pgfpathcurveto{\pgfqpoint{4.048942in}{2.072661in}}{\pgfqpoint{4.053332in}{2.083261in}}{\pgfqpoint{4.053332in}{2.094311in}}%
\pgfpathcurveto{\pgfqpoint{4.053332in}{2.105361in}}{\pgfqpoint{4.048942in}{2.115960in}}{\pgfqpoint{4.041128in}{2.123773in}}%
\pgfpathcurveto{\pgfqpoint{4.033315in}{2.131587in}}{\pgfqpoint{4.022716in}{2.135977in}}{\pgfqpoint{4.011666in}{2.135977in}}%
\pgfpathcurveto{\pgfqpoint{4.000616in}{2.135977in}}{\pgfqpoint{3.990016in}{2.131587in}}{\pgfqpoint{3.982203in}{2.123773in}}%
\pgfpathcurveto{\pgfqpoint{3.974389in}{2.115960in}}{\pgfqpoint{3.969999in}{2.105361in}}{\pgfqpoint{3.969999in}{2.094311in}}%
\pgfpathcurveto{\pgfqpoint{3.969999in}{2.083261in}}{\pgfqpoint{3.974389in}{2.072661in}}{\pgfqpoint{3.982203in}{2.064848in}}%
\pgfpathcurveto{\pgfqpoint{3.990016in}{2.057034in}}{\pgfqpoint{4.000616in}{2.052644in}}{\pgfqpoint{4.011666in}{2.052644in}}%
\pgfpathclose%
\pgfusepath{stroke,fill}%
\end{pgfscope}%
\begin{pgfscope}%
\pgfpathrectangle{\pgfqpoint{0.800000in}{0.528000in}}{\pgfqpoint{4.960000in}{3.696000in}}%
\pgfusepath{clip}%
\pgfsetbuttcap%
\pgfsetroundjoin%
\definecolor{currentfill}{rgb}{0.000000,0.000000,0.000000}%
\pgfsetfillcolor{currentfill}%
\pgfsetlinewidth{1.003750pt}%
\definecolor{currentstroke}{rgb}{0.000000,0.000000,0.000000}%
\pgfsetstrokecolor{currentstroke}%
\pgfsetdash{}{0pt}%
\pgfpathmoveto{\pgfqpoint{4.011666in}{1.859167in}}%
\pgfpathcurveto{\pgfqpoint{4.022716in}{1.859167in}}{\pgfqpoint{4.033315in}{1.863557in}}{\pgfqpoint{4.041128in}{1.871370in}}%
\pgfpathcurveto{\pgfqpoint{4.048942in}{1.879184in}}{\pgfqpoint{4.053332in}{1.889783in}}{\pgfqpoint{4.053332in}{1.900833in}}%
\pgfpathcurveto{\pgfqpoint{4.053332in}{1.911883in}}{\pgfqpoint{4.048942in}{1.922482in}}{\pgfqpoint{4.041128in}{1.930296in}}%
\pgfpathcurveto{\pgfqpoint{4.033315in}{1.938110in}}{\pgfqpoint{4.022716in}{1.942500in}}{\pgfqpoint{4.011666in}{1.942500in}}%
\pgfpathcurveto{\pgfqpoint{4.000616in}{1.942500in}}{\pgfqpoint{3.990016in}{1.938110in}}{\pgfqpoint{3.982203in}{1.930296in}}%
\pgfpathcurveto{\pgfqpoint{3.974389in}{1.922482in}}{\pgfqpoint{3.969999in}{1.911883in}}{\pgfqpoint{3.969999in}{1.900833in}}%
\pgfpathcurveto{\pgfqpoint{3.969999in}{1.889783in}}{\pgfqpoint{3.974389in}{1.879184in}}{\pgfqpoint{3.982203in}{1.871370in}}%
\pgfpathcurveto{\pgfqpoint{3.990016in}{1.863557in}}{\pgfqpoint{4.000616in}{1.859167in}}{\pgfqpoint{4.011666in}{1.859167in}}%
\pgfpathclose%
\pgfusepath{stroke,fill}%
\end{pgfscope}%
\begin{pgfscope}%
\pgfpathrectangle{\pgfqpoint{0.800000in}{0.528000in}}{\pgfqpoint{4.960000in}{3.696000in}}%
\pgfusepath{clip}%
\pgfsetbuttcap%
\pgfsetroundjoin%
\definecolor{currentfill}{rgb}{0.000000,0.000000,0.000000}%
\pgfsetfillcolor{currentfill}%
\pgfsetlinewidth{1.003750pt}%
\definecolor{currentstroke}{rgb}{0.000000,0.000000,0.000000}%
\pgfsetstrokecolor{currentstroke}%
\pgfsetdash{}{0pt}%
\pgfpathmoveto{\pgfqpoint{4.011666in}{1.859167in}}%
\pgfpathcurveto{\pgfqpoint{4.022716in}{1.859167in}}{\pgfqpoint{4.033315in}{1.863557in}}{\pgfqpoint{4.041128in}{1.871370in}}%
\pgfpathcurveto{\pgfqpoint{4.048942in}{1.879184in}}{\pgfqpoint{4.053332in}{1.889783in}}{\pgfqpoint{4.053332in}{1.900833in}}%
\pgfpathcurveto{\pgfqpoint{4.053332in}{1.911883in}}{\pgfqpoint{4.048942in}{1.922482in}}{\pgfqpoint{4.041128in}{1.930296in}}%
\pgfpathcurveto{\pgfqpoint{4.033315in}{1.938110in}}{\pgfqpoint{4.022716in}{1.942500in}}{\pgfqpoint{4.011666in}{1.942500in}}%
\pgfpathcurveto{\pgfqpoint{4.000616in}{1.942500in}}{\pgfqpoint{3.990016in}{1.938110in}}{\pgfqpoint{3.982203in}{1.930296in}}%
\pgfpathcurveto{\pgfqpoint{3.974389in}{1.922482in}}{\pgfqpoint{3.969999in}{1.911883in}}{\pgfqpoint{3.969999in}{1.900833in}}%
\pgfpathcurveto{\pgfqpoint{3.969999in}{1.889783in}}{\pgfqpoint{3.974389in}{1.879184in}}{\pgfqpoint{3.982203in}{1.871370in}}%
\pgfpathcurveto{\pgfqpoint{3.990016in}{1.863557in}}{\pgfqpoint{4.000616in}{1.859167in}}{\pgfqpoint{4.011666in}{1.859167in}}%
\pgfpathclose%
\pgfusepath{stroke,fill}%
\end{pgfscope}%
\begin{pgfscope}%
\pgfpathrectangle{\pgfqpoint{0.800000in}{0.528000in}}{\pgfqpoint{4.960000in}{3.696000in}}%
\pgfusepath{clip}%
\pgfsetbuttcap%
\pgfsetroundjoin%
\definecolor{currentfill}{rgb}{0.000000,0.000000,0.000000}%
\pgfsetfillcolor{currentfill}%
\pgfsetlinewidth{1.003750pt}%
\definecolor{currentstroke}{rgb}{0.000000,0.000000,0.000000}%
\pgfsetstrokecolor{currentstroke}%
\pgfsetdash{}{0pt}%
\pgfpathmoveto{\pgfqpoint{4.011666in}{1.816172in}}%
\pgfpathcurveto{\pgfqpoint{4.022716in}{1.816172in}}{\pgfqpoint{4.033315in}{1.820562in}}{\pgfqpoint{4.041128in}{1.828375in}}%
\pgfpathcurveto{\pgfqpoint{4.048942in}{1.836189in}}{\pgfqpoint{4.053332in}{1.846788in}}{\pgfqpoint{4.053332in}{1.857838in}}%
\pgfpathcurveto{\pgfqpoint{4.053332in}{1.868888in}}{\pgfqpoint{4.048942in}{1.879487in}}{\pgfqpoint{4.041128in}{1.887301in}}%
\pgfpathcurveto{\pgfqpoint{4.033315in}{1.895115in}}{\pgfqpoint{4.022716in}{1.899505in}}{\pgfqpoint{4.011666in}{1.899505in}}%
\pgfpathcurveto{\pgfqpoint{4.000616in}{1.899505in}}{\pgfqpoint{3.990016in}{1.895115in}}{\pgfqpoint{3.982203in}{1.887301in}}%
\pgfpathcurveto{\pgfqpoint{3.974389in}{1.879487in}}{\pgfqpoint{3.969999in}{1.868888in}}{\pgfqpoint{3.969999in}{1.857838in}}%
\pgfpathcurveto{\pgfqpoint{3.969999in}{1.846788in}}{\pgfqpoint{3.974389in}{1.836189in}}{\pgfqpoint{3.982203in}{1.828375in}}%
\pgfpathcurveto{\pgfqpoint{3.990016in}{1.820562in}}{\pgfqpoint{4.000616in}{1.816172in}}{\pgfqpoint{4.011666in}{1.816172in}}%
\pgfpathclose%
\pgfusepath{stroke,fill}%
\end{pgfscope}%
\begin{pgfscope}%
\pgfpathrectangle{\pgfqpoint{0.800000in}{0.528000in}}{\pgfqpoint{4.960000in}{3.696000in}}%
\pgfusepath{clip}%
\pgfsetbuttcap%
\pgfsetroundjoin%
\definecolor{currentfill}{rgb}{0.000000,0.000000,0.000000}%
\pgfsetfillcolor{currentfill}%
\pgfsetlinewidth{1.003750pt}%
\definecolor{currentstroke}{rgb}{0.000000,0.000000,0.000000}%
\pgfsetstrokecolor{currentstroke}%
\pgfsetdash{}{0pt}%
\pgfpathmoveto{\pgfqpoint{4.011666in}{1.794674in}}%
\pgfpathcurveto{\pgfqpoint{4.022716in}{1.794674in}}{\pgfqpoint{4.033315in}{1.799064in}}{\pgfqpoint{4.041128in}{1.806878in}}%
\pgfpathcurveto{\pgfqpoint{4.048942in}{1.814692in}}{\pgfqpoint{4.053332in}{1.825291in}}{\pgfqpoint{4.053332in}{1.836341in}}%
\pgfpathcurveto{\pgfqpoint{4.053332in}{1.847391in}}{\pgfqpoint{4.048942in}{1.857990in}}{\pgfqpoint{4.041128in}{1.865804in}}%
\pgfpathcurveto{\pgfqpoint{4.033315in}{1.873617in}}{\pgfqpoint{4.022716in}{1.878007in}}{\pgfqpoint{4.011666in}{1.878007in}}%
\pgfpathcurveto{\pgfqpoint{4.000616in}{1.878007in}}{\pgfqpoint{3.990016in}{1.873617in}}{\pgfqpoint{3.982203in}{1.865804in}}%
\pgfpathcurveto{\pgfqpoint{3.974389in}{1.857990in}}{\pgfqpoint{3.969999in}{1.847391in}}{\pgfqpoint{3.969999in}{1.836341in}}%
\pgfpathcurveto{\pgfqpoint{3.969999in}{1.825291in}}{\pgfqpoint{3.974389in}{1.814692in}}{\pgfqpoint{3.982203in}{1.806878in}}%
\pgfpathcurveto{\pgfqpoint{3.990016in}{1.799064in}}{\pgfqpoint{4.000616in}{1.794674in}}{\pgfqpoint{4.011666in}{1.794674in}}%
\pgfpathclose%
\pgfusepath{stroke,fill}%
\end{pgfscope}%
\begin{pgfscope}%
\pgfpathrectangle{\pgfqpoint{0.800000in}{0.528000in}}{\pgfqpoint{4.960000in}{3.696000in}}%
\pgfusepath{clip}%
\pgfsetbuttcap%
\pgfsetroundjoin%
\definecolor{currentfill}{rgb}{0.000000,0.000000,0.000000}%
\pgfsetfillcolor{currentfill}%
\pgfsetlinewidth{1.003750pt}%
\definecolor{currentstroke}{rgb}{0.000000,0.000000,0.000000}%
\pgfsetstrokecolor{currentstroke}%
\pgfsetdash{}{0pt}%
\pgfpathmoveto{\pgfqpoint{4.011666in}{1.730182in}}%
\pgfpathcurveto{\pgfqpoint{4.022716in}{1.730182in}}{\pgfqpoint{4.033315in}{1.734572in}}{\pgfqpoint{4.041128in}{1.742385in}}%
\pgfpathcurveto{\pgfqpoint{4.048942in}{1.750199in}}{\pgfqpoint{4.053332in}{1.760798in}}{\pgfqpoint{4.053332in}{1.771848in}}%
\pgfpathcurveto{\pgfqpoint{4.053332in}{1.782898in}}{\pgfqpoint{4.048942in}{1.793497in}}{\pgfqpoint{4.041128in}{1.801311in}}%
\pgfpathcurveto{\pgfqpoint{4.033315in}{1.809125in}}{\pgfqpoint{4.022716in}{1.813515in}}{\pgfqpoint{4.011666in}{1.813515in}}%
\pgfpathcurveto{\pgfqpoint{4.000616in}{1.813515in}}{\pgfqpoint{3.990016in}{1.809125in}}{\pgfqpoint{3.982203in}{1.801311in}}%
\pgfpathcurveto{\pgfqpoint{3.974389in}{1.793497in}}{\pgfqpoint{3.969999in}{1.782898in}}{\pgfqpoint{3.969999in}{1.771848in}}%
\pgfpathcurveto{\pgfqpoint{3.969999in}{1.760798in}}{\pgfqpoint{3.974389in}{1.750199in}}{\pgfqpoint{3.982203in}{1.742385in}}%
\pgfpathcurveto{\pgfqpoint{3.990016in}{1.734572in}}{\pgfqpoint{4.000616in}{1.730182in}}{\pgfqpoint{4.011666in}{1.730182in}}%
\pgfpathclose%
\pgfusepath{stroke,fill}%
\end{pgfscope}%
\begin{pgfscope}%
\pgfpathrectangle{\pgfqpoint{0.800000in}{0.528000in}}{\pgfqpoint{4.960000in}{3.696000in}}%
\pgfusepath{clip}%
\pgfsetbuttcap%
\pgfsetroundjoin%
\definecolor{currentfill}{rgb}{0.000000,0.000000,0.000000}%
\pgfsetfillcolor{currentfill}%
\pgfsetlinewidth{1.003750pt}%
\definecolor{currentstroke}{rgb}{0.000000,0.000000,0.000000}%
\pgfsetstrokecolor{currentstroke}%
\pgfsetdash{}{0pt}%
\pgfpathmoveto{\pgfqpoint{4.011666in}{1.945157in}}%
\pgfpathcurveto{\pgfqpoint{4.022716in}{1.945157in}}{\pgfqpoint{4.033315in}{1.949547in}}{\pgfqpoint{4.041128in}{1.957360in}}%
\pgfpathcurveto{\pgfqpoint{4.048942in}{1.965174in}}{\pgfqpoint{4.053332in}{1.975773in}}{\pgfqpoint{4.053332in}{1.986823in}}%
\pgfpathcurveto{\pgfqpoint{4.053332in}{1.997873in}}{\pgfqpoint{4.048942in}{2.008472in}}{\pgfqpoint{4.041128in}{2.016286in}}%
\pgfpathcurveto{\pgfqpoint{4.033315in}{2.024100in}}{\pgfqpoint{4.022716in}{2.028490in}}{\pgfqpoint{4.011666in}{2.028490in}}%
\pgfpathcurveto{\pgfqpoint{4.000616in}{2.028490in}}{\pgfqpoint{3.990016in}{2.024100in}}{\pgfqpoint{3.982203in}{2.016286in}}%
\pgfpathcurveto{\pgfqpoint{3.974389in}{2.008472in}}{\pgfqpoint{3.969999in}{1.997873in}}{\pgfqpoint{3.969999in}{1.986823in}}%
\pgfpathcurveto{\pgfqpoint{3.969999in}{1.975773in}}{\pgfqpoint{3.974389in}{1.965174in}}{\pgfqpoint{3.982203in}{1.957360in}}%
\pgfpathcurveto{\pgfqpoint{3.990016in}{1.949547in}}{\pgfqpoint{4.000616in}{1.945157in}}{\pgfqpoint{4.011666in}{1.945157in}}%
\pgfpathclose%
\pgfusepath{stroke,fill}%
\end{pgfscope}%
\begin{pgfscope}%
\pgfpathrectangle{\pgfqpoint{0.800000in}{0.528000in}}{\pgfqpoint{4.960000in}{3.696000in}}%
\pgfusepath{clip}%
\pgfsetbuttcap%
\pgfsetroundjoin%
\definecolor{currentfill}{rgb}{0.000000,0.000000,0.000000}%
\pgfsetfillcolor{currentfill}%
\pgfsetlinewidth{1.003750pt}%
\definecolor{currentstroke}{rgb}{0.000000,0.000000,0.000000}%
\pgfsetstrokecolor{currentstroke}%
\pgfsetdash{}{0pt}%
\pgfpathmoveto{\pgfqpoint{4.011666in}{1.794674in}}%
\pgfpathcurveto{\pgfqpoint{4.022716in}{1.794674in}}{\pgfqpoint{4.033315in}{1.799064in}}{\pgfqpoint{4.041128in}{1.806878in}}%
\pgfpathcurveto{\pgfqpoint{4.048942in}{1.814692in}}{\pgfqpoint{4.053332in}{1.825291in}}{\pgfqpoint{4.053332in}{1.836341in}}%
\pgfpathcurveto{\pgfqpoint{4.053332in}{1.847391in}}{\pgfqpoint{4.048942in}{1.857990in}}{\pgfqpoint{4.041128in}{1.865804in}}%
\pgfpathcurveto{\pgfqpoint{4.033315in}{1.873617in}}{\pgfqpoint{4.022716in}{1.878007in}}{\pgfqpoint{4.011666in}{1.878007in}}%
\pgfpathcurveto{\pgfqpoint{4.000616in}{1.878007in}}{\pgfqpoint{3.990016in}{1.873617in}}{\pgfqpoint{3.982203in}{1.865804in}}%
\pgfpathcurveto{\pgfqpoint{3.974389in}{1.857990in}}{\pgfqpoint{3.969999in}{1.847391in}}{\pgfqpoint{3.969999in}{1.836341in}}%
\pgfpathcurveto{\pgfqpoint{3.969999in}{1.825291in}}{\pgfqpoint{3.974389in}{1.814692in}}{\pgfqpoint{3.982203in}{1.806878in}}%
\pgfpathcurveto{\pgfqpoint{3.990016in}{1.799064in}}{\pgfqpoint{4.000616in}{1.794674in}}{\pgfqpoint{4.011666in}{1.794674in}}%
\pgfpathclose%
\pgfusepath{stroke,fill}%
\end{pgfscope}%
\begin{pgfscope}%
\pgfpathrectangle{\pgfqpoint{0.800000in}{0.528000in}}{\pgfqpoint{4.960000in}{3.696000in}}%
\pgfusepath{clip}%
\pgfsetbuttcap%
\pgfsetroundjoin%
\definecolor{currentfill}{rgb}{0.000000,0.000000,0.000000}%
\pgfsetfillcolor{currentfill}%
\pgfsetlinewidth{1.003750pt}%
\definecolor{currentstroke}{rgb}{0.000000,0.000000,0.000000}%
\pgfsetstrokecolor{currentstroke}%
\pgfsetdash{}{0pt}%
\pgfpathmoveto{\pgfqpoint{4.011666in}{2.009649in}}%
\pgfpathcurveto{\pgfqpoint{4.022716in}{2.009649in}}{\pgfqpoint{4.033315in}{2.014039in}}{\pgfqpoint{4.041128in}{2.021853in}}%
\pgfpathcurveto{\pgfqpoint{4.048942in}{2.029666in}}{\pgfqpoint{4.053332in}{2.040266in}}{\pgfqpoint{4.053332in}{2.051316in}}%
\pgfpathcurveto{\pgfqpoint{4.053332in}{2.062366in}}{\pgfqpoint{4.048942in}{2.072965in}}{\pgfqpoint{4.041128in}{2.080778in}}%
\pgfpathcurveto{\pgfqpoint{4.033315in}{2.088592in}}{\pgfqpoint{4.022716in}{2.092982in}}{\pgfqpoint{4.011666in}{2.092982in}}%
\pgfpathcurveto{\pgfqpoint{4.000616in}{2.092982in}}{\pgfqpoint{3.990016in}{2.088592in}}{\pgfqpoint{3.982203in}{2.080778in}}%
\pgfpathcurveto{\pgfqpoint{3.974389in}{2.072965in}}{\pgfqpoint{3.969999in}{2.062366in}}{\pgfqpoint{3.969999in}{2.051316in}}%
\pgfpathcurveto{\pgfqpoint{3.969999in}{2.040266in}}{\pgfqpoint{3.974389in}{2.029666in}}{\pgfqpoint{3.982203in}{2.021853in}}%
\pgfpathcurveto{\pgfqpoint{3.990016in}{2.014039in}}{\pgfqpoint{4.000616in}{2.009649in}}{\pgfqpoint{4.011666in}{2.009649in}}%
\pgfpathclose%
\pgfusepath{stroke,fill}%
\end{pgfscope}%
\begin{pgfscope}%
\pgfpathrectangle{\pgfqpoint{0.800000in}{0.528000in}}{\pgfqpoint{4.960000in}{3.696000in}}%
\pgfusepath{clip}%
\pgfsetbuttcap%
\pgfsetroundjoin%
\definecolor{currentfill}{rgb}{0.000000,0.000000,0.000000}%
\pgfsetfillcolor{currentfill}%
\pgfsetlinewidth{1.003750pt}%
\definecolor{currentstroke}{rgb}{0.000000,0.000000,0.000000}%
\pgfsetstrokecolor{currentstroke}%
\pgfsetdash{}{0pt}%
\pgfpathmoveto{\pgfqpoint{4.011666in}{1.816172in}}%
\pgfpathcurveto{\pgfqpoint{4.022716in}{1.816172in}}{\pgfqpoint{4.033315in}{1.820562in}}{\pgfqpoint{4.041128in}{1.828375in}}%
\pgfpathcurveto{\pgfqpoint{4.048942in}{1.836189in}}{\pgfqpoint{4.053332in}{1.846788in}}{\pgfqpoint{4.053332in}{1.857838in}}%
\pgfpathcurveto{\pgfqpoint{4.053332in}{1.868888in}}{\pgfqpoint{4.048942in}{1.879487in}}{\pgfqpoint{4.041128in}{1.887301in}}%
\pgfpathcurveto{\pgfqpoint{4.033315in}{1.895115in}}{\pgfqpoint{4.022716in}{1.899505in}}{\pgfqpoint{4.011666in}{1.899505in}}%
\pgfpathcurveto{\pgfqpoint{4.000616in}{1.899505in}}{\pgfqpoint{3.990016in}{1.895115in}}{\pgfqpoint{3.982203in}{1.887301in}}%
\pgfpathcurveto{\pgfqpoint{3.974389in}{1.879487in}}{\pgfqpoint{3.969999in}{1.868888in}}{\pgfqpoint{3.969999in}{1.857838in}}%
\pgfpathcurveto{\pgfqpoint{3.969999in}{1.846788in}}{\pgfqpoint{3.974389in}{1.836189in}}{\pgfqpoint{3.982203in}{1.828375in}}%
\pgfpathcurveto{\pgfqpoint{3.990016in}{1.820562in}}{\pgfqpoint{4.000616in}{1.816172in}}{\pgfqpoint{4.011666in}{1.816172in}}%
\pgfpathclose%
\pgfusepath{stroke,fill}%
\end{pgfscope}%
\begin{pgfscope}%
\pgfpathrectangle{\pgfqpoint{0.800000in}{0.528000in}}{\pgfqpoint{4.960000in}{3.696000in}}%
\pgfusepath{clip}%
\pgfsetbuttcap%
\pgfsetroundjoin%
\definecolor{currentfill}{rgb}{0.000000,0.000000,0.000000}%
\pgfsetfillcolor{currentfill}%
\pgfsetlinewidth{1.003750pt}%
\definecolor{currentstroke}{rgb}{0.000000,0.000000,0.000000}%
\pgfsetstrokecolor{currentstroke}%
\pgfsetdash{}{0pt}%
\pgfpathmoveto{\pgfqpoint{4.011666in}{1.945157in}}%
\pgfpathcurveto{\pgfqpoint{4.022716in}{1.945157in}}{\pgfqpoint{4.033315in}{1.949547in}}{\pgfqpoint{4.041128in}{1.957360in}}%
\pgfpathcurveto{\pgfqpoint{4.048942in}{1.965174in}}{\pgfqpoint{4.053332in}{1.975773in}}{\pgfqpoint{4.053332in}{1.986823in}}%
\pgfpathcurveto{\pgfqpoint{4.053332in}{1.997873in}}{\pgfqpoint{4.048942in}{2.008472in}}{\pgfqpoint{4.041128in}{2.016286in}}%
\pgfpathcurveto{\pgfqpoint{4.033315in}{2.024100in}}{\pgfqpoint{4.022716in}{2.028490in}}{\pgfqpoint{4.011666in}{2.028490in}}%
\pgfpathcurveto{\pgfqpoint{4.000616in}{2.028490in}}{\pgfqpoint{3.990016in}{2.024100in}}{\pgfqpoint{3.982203in}{2.016286in}}%
\pgfpathcurveto{\pgfqpoint{3.974389in}{2.008472in}}{\pgfqpoint{3.969999in}{1.997873in}}{\pgfqpoint{3.969999in}{1.986823in}}%
\pgfpathcurveto{\pgfqpoint{3.969999in}{1.975773in}}{\pgfqpoint{3.974389in}{1.965174in}}{\pgfqpoint{3.982203in}{1.957360in}}%
\pgfpathcurveto{\pgfqpoint{3.990016in}{1.949547in}}{\pgfqpoint{4.000616in}{1.945157in}}{\pgfqpoint{4.011666in}{1.945157in}}%
\pgfpathclose%
\pgfusepath{stroke,fill}%
\end{pgfscope}%
\begin{pgfscope}%
\pgfpathrectangle{\pgfqpoint{0.800000in}{0.528000in}}{\pgfqpoint{4.960000in}{3.696000in}}%
\pgfusepath{clip}%
\pgfsetbuttcap%
\pgfsetroundjoin%
\definecolor{currentfill}{rgb}{0.000000,0.000000,0.000000}%
\pgfsetfillcolor{currentfill}%
\pgfsetlinewidth{1.003750pt}%
\definecolor{currentstroke}{rgb}{0.000000,0.000000,0.000000}%
\pgfsetstrokecolor{currentstroke}%
\pgfsetdash{}{0pt}%
\pgfpathmoveto{\pgfqpoint{4.011666in}{1.859167in}}%
\pgfpathcurveto{\pgfqpoint{4.022716in}{1.859167in}}{\pgfqpoint{4.033315in}{1.863557in}}{\pgfqpoint{4.041128in}{1.871370in}}%
\pgfpathcurveto{\pgfqpoint{4.048942in}{1.879184in}}{\pgfqpoint{4.053332in}{1.889783in}}{\pgfqpoint{4.053332in}{1.900833in}}%
\pgfpathcurveto{\pgfqpoint{4.053332in}{1.911883in}}{\pgfqpoint{4.048942in}{1.922482in}}{\pgfqpoint{4.041128in}{1.930296in}}%
\pgfpathcurveto{\pgfqpoint{4.033315in}{1.938110in}}{\pgfqpoint{4.022716in}{1.942500in}}{\pgfqpoint{4.011666in}{1.942500in}}%
\pgfpathcurveto{\pgfqpoint{4.000616in}{1.942500in}}{\pgfqpoint{3.990016in}{1.938110in}}{\pgfqpoint{3.982203in}{1.930296in}}%
\pgfpathcurveto{\pgfqpoint{3.974389in}{1.922482in}}{\pgfqpoint{3.969999in}{1.911883in}}{\pgfqpoint{3.969999in}{1.900833in}}%
\pgfpathcurveto{\pgfqpoint{3.969999in}{1.889783in}}{\pgfqpoint{3.974389in}{1.879184in}}{\pgfqpoint{3.982203in}{1.871370in}}%
\pgfpathcurveto{\pgfqpoint{3.990016in}{1.863557in}}{\pgfqpoint{4.000616in}{1.859167in}}{\pgfqpoint{4.011666in}{1.859167in}}%
\pgfpathclose%
\pgfusepath{stroke,fill}%
\end{pgfscope}%
\begin{pgfscope}%
\pgfpathrectangle{\pgfqpoint{0.800000in}{0.528000in}}{\pgfqpoint{4.960000in}{3.696000in}}%
\pgfusepath{clip}%
\pgfsetbuttcap%
\pgfsetroundjoin%
\definecolor{currentfill}{rgb}{0.000000,0.000000,0.000000}%
\pgfsetfillcolor{currentfill}%
\pgfsetlinewidth{1.003750pt}%
\definecolor{currentstroke}{rgb}{0.000000,0.000000,0.000000}%
\pgfsetstrokecolor{currentstroke}%
\pgfsetdash{}{0pt}%
\pgfpathmoveto{\pgfqpoint{4.011666in}{1.816172in}}%
\pgfpathcurveto{\pgfqpoint{4.022716in}{1.816172in}}{\pgfqpoint{4.033315in}{1.820562in}}{\pgfqpoint{4.041128in}{1.828375in}}%
\pgfpathcurveto{\pgfqpoint{4.048942in}{1.836189in}}{\pgfqpoint{4.053332in}{1.846788in}}{\pgfqpoint{4.053332in}{1.857838in}}%
\pgfpathcurveto{\pgfqpoint{4.053332in}{1.868888in}}{\pgfqpoint{4.048942in}{1.879487in}}{\pgfqpoint{4.041128in}{1.887301in}}%
\pgfpathcurveto{\pgfqpoint{4.033315in}{1.895115in}}{\pgfqpoint{4.022716in}{1.899505in}}{\pgfqpoint{4.011666in}{1.899505in}}%
\pgfpathcurveto{\pgfqpoint{4.000616in}{1.899505in}}{\pgfqpoint{3.990016in}{1.895115in}}{\pgfqpoint{3.982203in}{1.887301in}}%
\pgfpathcurveto{\pgfqpoint{3.974389in}{1.879487in}}{\pgfqpoint{3.969999in}{1.868888in}}{\pgfqpoint{3.969999in}{1.857838in}}%
\pgfpathcurveto{\pgfqpoint{3.969999in}{1.846788in}}{\pgfqpoint{3.974389in}{1.836189in}}{\pgfqpoint{3.982203in}{1.828375in}}%
\pgfpathcurveto{\pgfqpoint{3.990016in}{1.820562in}}{\pgfqpoint{4.000616in}{1.816172in}}{\pgfqpoint{4.011666in}{1.816172in}}%
\pgfpathclose%
\pgfusepath{stroke,fill}%
\end{pgfscope}%
\begin{pgfscope}%
\pgfpathrectangle{\pgfqpoint{0.800000in}{0.528000in}}{\pgfqpoint{4.960000in}{3.696000in}}%
\pgfusepath{clip}%
\pgfsetbuttcap%
\pgfsetroundjoin%
\definecolor{currentfill}{rgb}{0.000000,0.000000,0.000000}%
\pgfsetfillcolor{currentfill}%
\pgfsetlinewidth{1.003750pt}%
\definecolor{currentstroke}{rgb}{0.000000,0.000000,0.000000}%
\pgfsetstrokecolor{currentstroke}%
\pgfsetdash{}{0pt}%
\pgfpathmoveto{\pgfqpoint{4.011666in}{1.966654in}}%
\pgfpathcurveto{\pgfqpoint{4.022716in}{1.966654in}}{\pgfqpoint{4.033315in}{1.971044in}}{\pgfqpoint{4.041128in}{1.978858in}}%
\pgfpathcurveto{\pgfqpoint{4.048942in}{1.986672in}}{\pgfqpoint{4.053332in}{1.997271in}}{\pgfqpoint{4.053332in}{2.008321in}}%
\pgfpathcurveto{\pgfqpoint{4.053332in}{2.019371in}}{\pgfqpoint{4.048942in}{2.029970in}}{\pgfqpoint{4.041128in}{2.037783in}}%
\pgfpathcurveto{\pgfqpoint{4.033315in}{2.045597in}}{\pgfqpoint{4.022716in}{2.049987in}}{\pgfqpoint{4.011666in}{2.049987in}}%
\pgfpathcurveto{\pgfqpoint{4.000616in}{2.049987in}}{\pgfqpoint{3.990016in}{2.045597in}}{\pgfqpoint{3.982203in}{2.037783in}}%
\pgfpathcurveto{\pgfqpoint{3.974389in}{2.029970in}}{\pgfqpoint{3.969999in}{2.019371in}}{\pgfqpoint{3.969999in}{2.008321in}}%
\pgfpathcurveto{\pgfqpoint{3.969999in}{1.997271in}}{\pgfqpoint{3.974389in}{1.986672in}}{\pgfqpoint{3.982203in}{1.978858in}}%
\pgfpathcurveto{\pgfqpoint{3.990016in}{1.971044in}}{\pgfqpoint{4.000616in}{1.966654in}}{\pgfqpoint{4.011666in}{1.966654in}}%
\pgfpathclose%
\pgfusepath{stroke,fill}%
\end{pgfscope}%
\begin{pgfscope}%
\pgfpathrectangle{\pgfqpoint{0.800000in}{0.528000in}}{\pgfqpoint{4.960000in}{3.696000in}}%
\pgfusepath{clip}%
\pgfsetbuttcap%
\pgfsetroundjoin%
\definecolor{currentfill}{rgb}{0.000000,0.000000,0.000000}%
\pgfsetfillcolor{currentfill}%
\pgfsetlinewidth{1.003750pt}%
\definecolor{currentstroke}{rgb}{0.000000,0.000000,0.000000}%
\pgfsetstrokecolor{currentstroke}%
\pgfsetdash{}{0pt}%
\pgfpathmoveto{\pgfqpoint{4.011666in}{2.052644in}}%
\pgfpathcurveto{\pgfqpoint{4.022716in}{2.052644in}}{\pgfqpoint{4.033315in}{2.057034in}}{\pgfqpoint{4.041128in}{2.064848in}}%
\pgfpathcurveto{\pgfqpoint{4.048942in}{2.072661in}}{\pgfqpoint{4.053332in}{2.083261in}}{\pgfqpoint{4.053332in}{2.094311in}}%
\pgfpathcurveto{\pgfqpoint{4.053332in}{2.105361in}}{\pgfqpoint{4.048942in}{2.115960in}}{\pgfqpoint{4.041128in}{2.123773in}}%
\pgfpathcurveto{\pgfqpoint{4.033315in}{2.131587in}}{\pgfqpoint{4.022716in}{2.135977in}}{\pgfqpoint{4.011666in}{2.135977in}}%
\pgfpathcurveto{\pgfqpoint{4.000616in}{2.135977in}}{\pgfqpoint{3.990016in}{2.131587in}}{\pgfqpoint{3.982203in}{2.123773in}}%
\pgfpathcurveto{\pgfqpoint{3.974389in}{2.115960in}}{\pgfqpoint{3.969999in}{2.105361in}}{\pgfqpoint{3.969999in}{2.094311in}}%
\pgfpathcurveto{\pgfqpoint{3.969999in}{2.083261in}}{\pgfqpoint{3.974389in}{2.072661in}}{\pgfqpoint{3.982203in}{2.064848in}}%
\pgfpathcurveto{\pgfqpoint{3.990016in}{2.057034in}}{\pgfqpoint{4.000616in}{2.052644in}}{\pgfqpoint{4.011666in}{2.052644in}}%
\pgfpathclose%
\pgfusepath{stroke,fill}%
\end{pgfscope}%
\begin{pgfscope}%
\pgfpathrectangle{\pgfqpoint{0.800000in}{0.528000in}}{\pgfqpoint{4.960000in}{3.696000in}}%
\pgfusepath{clip}%
\pgfsetbuttcap%
\pgfsetroundjoin%
\definecolor{currentfill}{rgb}{0.000000,0.000000,0.000000}%
\pgfsetfillcolor{currentfill}%
\pgfsetlinewidth{1.003750pt}%
\definecolor{currentstroke}{rgb}{0.000000,0.000000,0.000000}%
\pgfsetstrokecolor{currentstroke}%
\pgfsetdash{}{0pt}%
\pgfpathmoveto{\pgfqpoint{4.011666in}{1.880664in}}%
\pgfpathcurveto{\pgfqpoint{4.022716in}{1.880664in}}{\pgfqpoint{4.033315in}{1.885054in}}{\pgfqpoint{4.041128in}{1.892868in}}%
\pgfpathcurveto{\pgfqpoint{4.048942in}{1.900682in}}{\pgfqpoint{4.053332in}{1.911281in}}{\pgfqpoint{4.053332in}{1.922331in}}%
\pgfpathcurveto{\pgfqpoint{4.053332in}{1.933381in}}{\pgfqpoint{4.048942in}{1.943980in}}{\pgfqpoint{4.041128in}{1.951793in}}%
\pgfpathcurveto{\pgfqpoint{4.033315in}{1.959607in}}{\pgfqpoint{4.022716in}{1.963997in}}{\pgfqpoint{4.011666in}{1.963997in}}%
\pgfpathcurveto{\pgfqpoint{4.000616in}{1.963997in}}{\pgfqpoint{3.990016in}{1.959607in}}{\pgfqpoint{3.982203in}{1.951793in}}%
\pgfpathcurveto{\pgfqpoint{3.974389in}{1.943980in}}{\pgfqpoint{3.969999in}{1.933381in}}{\pgfqpoint{3.969999in}{1.922331in}}%
\pgfpathcurveto{\pgfqpoint{3.969999in}{1.911281in}}{\pgfqpoint{3.974389in}{1.900682in}}{\pgfqpoint{3.982203in}{1.892868in}}%
\pgfpathcurveto{\pgfqpoint{3.990016in}{1.885054in}}{\pgfqpoint{4.000616in}{1.880664in}}{\pgfqpoint{4.011666in}{1.880664in}}%
\pgfpathclose%
\pgfusepath{stroke,fill}%
\end{pgfscope}%
\begin{pgfscope}%
\pgfpathrectangle{\pgfqpoint{0.800000in}{0.528000in}}{\pgfqpoint{4.960000in}{3.696000in}}%
\pgfusepath{clip}%
\pgfsetbuttcap%
\pgfsetroundjoin%
\definecolor{currentfill}{rgb}{0.000000,0.000000,0.000000}%
\pgfsetfillcolor{currentfill}%
\pgfsetlinewidth{1.003750pt}%
\definecolor{currentstroke}{rgb}{0.000000,0.000000,0.000000}%
\pgfsetstrokecolor{currentstroke}%
\pgfsetdash{}{0pt}%
\pgfpathmoveto{\pgfqpoint{4.011666in}{1.902162in}}%
\pgfpathcurveto{\pgfqpoint{4.022716in}{1.902162in}}{\pgfqpoint{4.033315in}{1.906552in}}{\pgfqpoint{4.041128in}{1.914365in}}%
\pgfpathcurveto{\pgfqpoint{4.048942in}{1.922179in}}{\pgfqpoint{4.053332in}{1.932778in}}{\pgfqpoint{4.053332in}{1.943828in}}%
\pgfpathcurveto{\pgfqpoint{4.053332in}{1.954878in}}{\pgfqpoint{4.048942in}{1.965477in}}{\pgfqpoint{4.041128in}{1.973291in}}%
\pgfpathcurveto{\pgfqpoint{4.033315in}{1.981105in}}{\pgfqpoint{4.022716in}{1.985495in}}{\pgfqpoint{4.011666in}{1.985495in}}%
\pgfpathcurveto{\pgfqpoint{4.000616in}{1.985495in}}{\pgfqpoint{3.990016in}{1.981105in}}{\pgfqpoint{3.982203in}{1.973291in}}%
\pgfpathcurveto{\pgfqpoint{3.974389in}{1.965477in}}{\pgfqpoint{3.969999in}{1.954878in}}{\pgfqpoint{3.969999in}{1.943828in}}%
\pgfpathcurveto{\pgfqpoint{3.969999in}{1.932778in}}{\pgfqpoint{3.974389in}{1.922179in}}{\pgfqpoint{3.982203in}{1.914365in}}%
\pgfpathcurveto{\pgfqpoint{3.990016in}{1.906552in}}{\pgfqpoint{4.000616in}{1.902162in}}{\pgfqpoint{4.011666in}{1.902162in}}%
\pgfpathclose%
\pgfusepath{stroke,fill}%
\end{pgfscope}%
\begin{pgfscope}%
\pgfpathrectangle{\pgfqpoint{0.800000in}{0.528000in}}{\pgfqpoint{4.960000in}{3.696000in}}%
\pgfusepath{clip}%
\pgfsetbuttcap%
\pgfsetroundjoin%
\definecolor{currentfill}{rgb}{0.000000,0.000000,0.000000}%
\pgfsetfillcolor{currentfill}%
\pgfsetlinewidth{1.003750pt}%
\definecolor{currentstroke}{rgb}{0.000000,0.000000,0.000000}%
\pgfsetstrokecolor{currentstroke}%
\pgfsetdash{}{0pt}%
\pgfpathmoveto{\pgfqpoint{4.011666in}{1.945157in}}%
\pgfpathcurveto{\pgfqpoint{4.022716in}{1.945157in}}{\pgfqpoint{4.033315in}{1.949547in}}{\pgfqpoint{4.041128in}{1.957360in}}%
\pgfpathcurveto{\pgfqpoint{4.048942in}{1.965174in}}{\pgfqpoint{4.053332in}{1.975773in}}{\pgfqpoint{4.053332in}{1.986823in}}%
\pgfpathcurveto{\pgfqpoint{4.053332in}{1.997873in}}{\pgfqpoint{4.048942in}{2.008472in}}{\pgfqpoint{4.041128in}{2.016286in}}%
\pgfpathcurveto{\pgfqpoint{4.033315in}{2.024100in}}{\pgfqpoint{4.022716in}{2.028490in}}{\pgfqpoint{4.011666in}{2.028490in}}%
\pgfpathcurveto{\pgfqpoint{4.000616in}{2.028490in}}{\pgfqpoint{3.990016in}{2.024100in}}{\pgfqpoint{3.982203in}{2.016286in}}%
\pgfpathcurveto{\pgfqpoint{3.974389in}{2.008472in}}{\pgfqpoint{3.969999in}{1.997873in}}{\pgfqpoint{3.969999in}{1.986823in}}%
\pgfpathcurveto{\pgfqpoint{3.969999in}{1.975773in}}{\pgfqpoint{3.974389in}{1.965174in}}{\pgfqpoint{3.982203in}{1.957360in}}%
\pgfpathcurveto{\pgfqpoint{3.990016in}{1.949547in}}{\pgfqpoint{4.000616in}{1.945157in}}{\pgfqpoint{4.011666in}{1.945157in}}%
\pgfpathclose%
\pgfusepath{stroke,fill}%
\end{pgfscope}%
\begin{pgfscope}%
\pgfpathrectangle{\pgfqpoint{0.800000in}{0.528000in}}{\pgfqpoint{4.960000in}{3.696000in}}%
\pgfusepath{clip}%
\pgfsetbuttcap%
\pgfsetroundjoin%
\definecolor{currentfill}{rgb}{0.000000,0.000000,0.000000}%
\pgfsetfillcolor{currentfill}%
\pgfsetlinewidth{1.003750pt}%
\definecolor{currentstroke}{rgb}{0.000000,0.000000,0.000000}%
\pgfsetstrokecolor{currentstroke}%
\pgfsetdash{}{0pt}%
\pgfpathmoveto{\pgfqpoint{4.011666in}{1.988151in}}%
\pgfpathcurveto{\pgfqpoint{4.022716in}{1.988151in}}{\pgfqpoint{4.033315in}{1.992542in}}{\pgfqpoint{4.041128in}{2.000355in}}%
\pgfpathcurveto{\pgfqpoint{4.048942in}{2.008169in}}{\pgfqpoint{4.053332in}{2.018768in}}{\pgfqpoint{4.053332in}{2.029818in}}%
\pgfpathcurveto{\pgfqpoint{4.053332in}{2.040868in}}{\pgfqpoint{4.048942in}{2.051467in}}{\pgfqpoint{4.041128in}{2.059281in}}%
\pgfpathcurveto{\pgfqpoint{4.033315in}{2.067095in}}{\pgfqpoint{4.022716in}{2.071485in}}{\pgfqpoint{4.011666in}{2.071485in}}%
\pgfpathcurveto{\pgfqpoint{4.000616in}{2.071485in}}{\pgfqpoint{3.990016in}{2.067095in}}{\pgfqpoint{3.982203in}{2.059281in}}%
\pgfpathcurveto{\pgfqpoint{3.974389in}{2.051467in}}{\pgfqpoint{3.969999in}{2.040868in}}{\pgfqpoint{3.969999in}{2.029818in}}%
\pgfpathcurveto{\pgfqpoint{3.969999in}{2.018768in}}{\pgfqpoint{3.974389in}{2.008169in}}{\pgfqpoint{3.982203in}{2.000355in}}%
\pgfpathcurveto{\pgfqpoint{3.990016in}{1.992542in}}{\pgfqpoint{4.000616in}{1.988151in}}{\pgfqpoint{4.011666in}{1.988151in}}%
\pgfpathclose%
\pgfusepath{stroke,fill}%
\end{pgfscope}%
\begin{pgfscope}%
\pgfpathrectangle{\pgfqpoint{0.800000in}{0.528000in}}{\pgfqpoint{4.960000in}{3.696000in}}%
\pgfusepath{clip}%
\pgfsetbuttcap%
\pgfsetroundjoin%
\definecolor{currentfill}{rgb}{0.000000,0.000000,0.000000}%
\pgfsetfillcolor{currentfill}%
\pgfsetlinewidth{1.003750pt}%
\definecolor{currentstroke}{rgb}{0.000000,0.000000,0.000000}%
\pgfsetstrokecolor{currentstroke}%
\pgfsetdash{}{0pt}%
\pgfpathmoveto{\pgfqpoint{4.011666in}{1.837669in}}%
\pgfpathcurveto{\pgfqpoint{4.022716in}{1.837669in}}{\pgfqpoint{4.033315in}{1.842059in}}{\pgfqpoint{4.041128in}{1.849873in}}%
\pgfpathcurveto{\pgfqpoint{4.048942in}{1.857687in}}{\pgfqpoint{4.053332in}{1.868286in}}{\pgfqpoint{4.053332in}{1.879336in}}%
\pgfpathcurveto{\pgfqpoint{4.053332in}{1.890386in}}{\pgfqpoint{4.048942in}{1.900985in}}{\pgfqpoint{4.041128in}{1.908798in}}%
\pgfpathcurveto{\pgfqpoint{4.033315in}{1.916612in}}{\pgfqpoint{4.022716in}{1.921002in}}{\pgfqpoint{4.011666in}{1.921002in}}%
\pgfpathcurveto{\pgfqpoint{4.000616in}{1.921002in}}{\pgfqpoint{3.990016in}{1.916612in}}{\pgfqpoint{3.982203in}{1.908798in}}%
\pgfpathcurveto{\pgfqpoint{3.974389in}{1.900985in}}{\pgfqpoint{3.969999in}{1.890386in}}{\pgfqpoint{3.969999in}{1.879336in}}%
\pgfpathcurveto{\pgfqpoint{3.969999in}{1.868286in}}{\pgfqpoint{3.974389in}{1.857687in}}{\pgfqpoint{3.982203in}{1.849873in}}%
\pgfpathcurveto{\pgfqpoint{3.990016in}{1.842059in}}{\pgfqpoint{4.000616in}{1.837669in}}{\pgfqpoint{4.011666in}{1.837669in}}%
\pgfpathclose%
\pgfusepath{stroke,fill}%
\end{pgfscope}%
\begin{pgfscope}%
\pgfpathrectangle{\pgfqpoint{0.800000in}{0.528000in}}{\pgfqpoint{4.960000in}{3.696000in}}%
\pgfusepath{clip}%
\pgfsetbuttcap%
\pgfsetroundjoin%
\definecolor{currentfill}{rgb}{0.000000,0.000000,0.000000}%
\pgfsetfillcolor{currentfill}%
\pgfsetlinewidth{1.003750pt}%
\definecolor{currentstroke}{rgb}{0.000000,0.000000,0.000000}%
\pgfsetstrokecolor{currentstroke}%
\pgfsetdash{}{0pt}%
\pgfpathmoveto{\pgfqpoint{4.011666in}{1.794674in}}%
\pgfpathcurveto{\pgfqpoint{4.022716in}{1.794674in}}{\pgfqpoint{4.033315in}{1.799064in}}{\pgfqpoint{4.041128in}{1.806878in}}%
\pgfpathcurveto{\pgfqpoint{4.048942in}{1.814692in}}{\pgfqpoint{4.053332in}{1.825291in}}{\pgfqpoint{4.053332in}{1.836341in}}%
\pgfpathcurveto{\pgfqpoint{4.053332in}{1.847391in}}{\pgfqpoint{4.048942in}{1.857990in}}{\pgfqpoint{4.041128in}{1.865804in}}%
\pgfpathcurveto{\pgfqpoint{4.033315in}{1.873617in}}{\pgfqpoint{4.022716in}{1.878007in}}{\pgfqpoint{4.011666in}{1.878007in}}%
\pgfpathcurveto{\pgfqpoint{4.000616in}{1.878007in}}{\pgfqpoint{3.990016in}{1.873617in}}{\pgfqpoint{3.982203in}{1.865804in}}%
\pgfpathcurveto{\pgfqpoint{3.974389in}{1.857990in}}{\pgfqpoint{3.969999in}{1.847391in}}{\pgfqpoint{3.969999in}{1.836341in}}%
\pgfpathcurveto{\pgfqpoint{3.969999in}{1.825291in}}{\pgfqpoint{3.974389in}{1.814692in}}{\pgfqpoint{3.982203in}{1.806878in}}%
\pgfpathcurveto{\pgfqpoint{3.990016in}{1.799064in}}{\pgfqpoint{4.000616in}{1.794674in}}{\pgfqpoint{4.011666in}{1.794674in}}%
\pgfpathclose%
\pgfusepath{stroke,fill}%
\end{pgfscope}%
\begin{pgfscope}%
\pgfpathrectangle{\pgfqpoint{0.800000in}{0.528000in}}{\pgfqpoint{4.960000in}{3.696000in}}%
\pgfusepath{clip}%
\pgfsetbuttcap%
\pgfsetroundjoin%
\definecolor{currentfill}{rgb}{0.000000,0.000000,0.000000}%
\pgfsetfillcolor{currentfill}%
\pgfsetlinewidth{1.003750pt}%
\definecolor{currentstroke}{rgb}{0.000000,0.000000,0.000000}%
\pgfsetstrokecolor{currentstroke}%
\pgfsetdash{}{0pt}%
\pgfpathmoveto{\pgfqpoint{4.011666in}{1.923659in}}%
\pgfpathcurveto{\pgfqpoint{4.022716in}{1.923659in}}{\pgfqpoint{4.033315in}{1.928049in}}{\pgfqpoint{4.041128in}{1.935863in}}%
\pgfpathcurveto{\pgfqpoint{4.048942in}{1.943677in}}{\pgfqpoint{4.053332in}{1.954276in}}{\pgfqpoint{4.053332in}{1.965326in}}%
\pgfpathcurveto{\pgfqpoint{4.053332in}{1.976376in}}{\pgfqpoint{4.048942in}{1.986975in}}{\pgfqpoint{4.041128in}{1.994788in}}%
\pgfpathcurveto{\pgfqpoint{4.033315in}{2.002602in}}{\pgfqpoint{4.022716in}{2.006992in}}{\pgfqpoint{4.011666in}{2.006992in}}%
\pgfpathcurveto{\pgfqpoint{4.000616in}{2.006992in}}{\pgfqpoint{3.990016in}{2.002602in}}{\pgfqpoint{3.982203in}{1.994788in}}%
\pgfpathcurveto{\pgfqpoint{3.974389in}{1.986975in}}{\pgfqpoint{3.969999in}{1.976376in}}{\pgfqpoint{3.969999in}{1.965326in}}%
\pgfpathcurveto{\pgfqpoint{3.969999in}{1.954276in}}{\pgfqpoint{3.974389in}{1.943677in}}{\pgfqpoint{3.982203in}{1.935863in}}%
\pgfpathcurveto{\pgfqpoint{3.990016in}{1.928049in}}{\pgfqpoint{4.000616in}{1.923659in}}{\pgfqpoint{4.011666in}{1.923659in}}%
\pgfpathclose%
\pgfusepath{stroke,fill}%
\end{pgfscope}%
\begin{pgfscope}%
\pgfpathrectangle{\pgfqpoint{0.800000in}{0.528000in}}{\pgfqpoint{4.960000in}{3.696000in}}%
\pgfusepath{clip}%
\pgfsetbuttcap%
\pgfsetroundjoin%
\definecolor{currentfill}{rgb}{0.000000,0.000000,0.000000}%
\pgfsetfillcolor{currentfill}%
\pgfsetlinewidth{1.003750pt}%
\definecolor{currentstroke}{rgb}{0.000000,0.000000,0.000000}%
\pgfsetstrokecolor{currentstroke}%
\pgfsetdash{}{0pt}%
\pgfpathmoveto{\pgfqpoint{4.011666in}{2.009649in}}%
\pgfpathcurveto{\pgfqpoint{4.022716in}{2.009649in}}{\pgfqpoint{4.033315in}{2.014039in}}{\pgfqpoint{4.041128in}{2.021853in}}%
\pgfpathcurveto{\pgfqpoint{4.048942in}{2.029666in}}{\pgfqpoint{4.053332in}{2.040266in}}{\pgfqpoint{4.053332in}{2.051316in}}%
\pgfpathcurveto{\pgfqpoint{4.053332in}{2.062366in}}{\pgfqpoint{4.048942in}{2.072965in}}{\pgfqpoint{4.041128in}{2.080778in}}%
\pgfpathcurveto{\pgfqpoint{4.033315in}{2.088592in}}{\pgfqpoint{4.022716in}{2.092982in}}{\pgfqpoint{4.011666in}{2.092982in}}%
\pgfpathcurveto{\pgfqpoint{4.000616in}{2.092982in}}{\pgfqpoint{3.990016in}{2.088592in}}{\pgfqpoint{3.982203in}{2.080778in}}%
\pgfpathcurveto{\pgfqpoint{3.974389in}{2.072965in}}{\pgfqpoint{3.969999in}{2.062366in}}{\pgfqpoint{3.969999in}{2.051316in}}%
\pgfpathcurveto{\pgfqpoint{3.969999in}{2.040266in}}{\pgfqpoint{3.974389in}{2.029666in}}{\pgfqpoint{3.982203in}{2.021853in}}%
\pgfpathcurveto{\pgfqpoint{3.990016in}{2.014039in}}{\pgfqpoint{4.000616in}{2.009649in}}{\pgfqpoint{4.011666in}{2.009649in}}%
\pgfpathclose%
\pgfusepath{stroke,fill}%
\end{pgfscope}%
\begin{pgfscope}%
\pgfpathrectangle{\pgfqpoint{0.800000in}{0.528000in}}{\pgfqpoint{4.960000in}{3.696000in}}%
\pgfusepath{clip}%
\pgfsetbuttcap%
\pgfsetroundjoin%
\definecolor{currentfill}{rgb}{0.000000,0.000000,0.000000}%
\pgfsetfillcolor{currentfill}%
\pgfsetlinewidth{1.003750pt}%
\definecolor{currentstroke}{rgb}{0.000000,0.000000,0.000000}%
\pgfsetstrokecolor{currentstroke}%
\pgfsetdash{}{0pt}%
\pgfpathmoveto{\pgfqpoint{5.504545in}{2.912544in}}%
\pgfpathcurveto{\pgfqpoint{5.515596in}{2.912544in}}{\pgfqpoint{5.526195in}{2.916934in}}{\pgfqpoint{5.534008in}{2.924748in}}%
\pgfpathcurveto{\pgfqpoint{5.541822in}{2.932561in}}{\pgfqpoint{5.546212in}{2.943160in}}{\pgfqpoint{5.546212in}{2.954210in}}%
\pgfpathcurveto{\pgfqpoint{5.546212in}{2.965260in}}{\pgfqpoint{5.541822in}{2.975859in}}{\pgfqpoint{5.534008in}{2.983673in}}%
\pgfpathcurveto{\pgfqpoint{5.526195in}{2.991487in}}{\pgfqpoint{5.515596in}{2.995877in}}{\pgfqpoint{5.504545in}{2.995877in}}%
\pgfpathcurveto{\pgfqpoint{5.493495in}{2.995877in}}{\pgfqpoint{5.482896in}{2.991487in}}{\pgfqpoint{5.475083in}{2.983673in}}%
\pgfpathcurveto{\pgfqpoint{5.467269in}{2.975859in}}{\pgfqpoint{5.462879in}{2.965260in}}{\pgfqpoint{5.462879in}{2.954210in}}%
\pgfpathcurveto{\pgfqpoint{5.462879in}{2.943160in}}{\pgfqpoint{5.467269in}{2.932561in}}{\pgfqpoint{5.475083in}{2.924748in}}%
\pgfpathcurveto{\pgfqpoint{5.482896in}{2.916934in}}{\pgfqpoint{5.493495in}{2.912544in}}{\pgfqpoint{5.504545in}{2.912544in}}%
\pgfpathclose%
\pgfusepath{stroke,fill}%
\end{pgfscope}%
\begin{pgfscope}%
\pgfpathrectangle{\pgfqpoint{0.800000in}{0.528000in}}{\pgfqpoint{4.960000in}{3.696000in}}%
\pgfusepath{clip}%
\pgfsetbuttcap%
\pgfsetroundjoin%
\definecolor{currentfill}{rgb}{0.000000,0.000000,0.000000}%
\pgfsetfillcolor{currentfill}%
\pgfsetlinewidth{1.003750pt}%
\definecolor{currentstroke}{rgb}{0.000000,0.000000,0.000000}%
\pgfsetstrokecolor{currentstroke}%
\pgfsetdash{}{0pt}%
\pgfpathmoveto{\pgfqpoint{5.504545in}{2.762061in}}%
\pgfpathcurveto{\pgfqpoint{5.515596in}{2.762061in}}{\pgfqpoint{5.526195in}{2.766451in}}{\pgfqpoint{5.534008in}{2.774265in}}%
\pgfpathcurveto{\pgfqpoint{5.541822in}{2.782079in}}{\pgfqpoint{5.546212in}{2.792678in}}{\pgfqpoint{5.546212in}{2.803728in}}%
\pgfpathcurveto{\pgfqpoint{5.546212in}{2.814778in}}{\pgfqpoint{5.541822in}{2.825377in}}{\pgfqpoint{5.534008in}{2.833191in}}%
\pgfpathcurveto{\pgfqpoint{5.526195in}{2.841004in}}{\pgfqpoint{5.515596in}{2.845395in}}{\pgfqpoint{5.504545in}{2.845395in}}%
\pgfpathcurveto{\pgfqpoint{5.493495in}{2.845395in}}{\pgfqpoint{5.482896in}{2.841004in}}{\pgfqpoint{5.475083in}{2.833191in}}%
\pgfpathcurveto{\pgfqpoint{5.467269in}{2.825377in}}{\pgfqpoint{5.462879in}{2.814778in}}{\pgfqpoint{5.462879in}{2.803728in}}%
\pgfpathcurveto{\pgfqpoint{5.462879in}{2.792678in}}{\pgfqpoint{5.467269in}{2.782079in}}{\pgfqpoint{5.475083in}{2.774265in}}%
\pgfpathcurveto{\pgfqpoint{5.482896in}{2.766451in}}{\pgfqpoint{5.493495in}{2.762061in}}{\pgfqpoint{5.504545in}{2.762061in}}%
\pgfpathclose%
\pgfusepath{stroke,fill}%
\end{pgfscope}%
\begin{pgfscope}%
\pgfpathrectangle{\pgfqpoint{0.800000in}{0.528000in}}{\pgfqpoint{4.960000in}{3.696000in}}%
\pgfusepath{clip}%
\pgfsetbuttcap%
\pgfsetroundjoin%
\definecolor{currentfill}{rgb}{0.000000,0.000000,0.000000}%
\pgfsetfillcolor{currentfill}%
\pgfsetlinewidth{1.003750pt}%
\definecolor{currentstroke}{rgb}{0.000000,0.000000,0.000000}%
\pgfsetstrokecolor{currentstroke}%
\pgfsetdash{}{0pt}%
\pgfpathmoveto{\pgfqpoint{5.504545in}{2.998534in}}%
\pgfpathcurveto{\pgfqpoint{5.515596in}{2.998534in}}{\pgfqpoint{5.526195in}{3.002924in}}{\pgfqpoint{5.534008in}{3.010738in}}%
\pgfpathcurveto{\pgfqpoint{5.541822in}{3.018551in}}{\pgfqpoint{5.546212in}{3.029150in}}{\pgfqpoint{5.546212in}{3.040200in}}%
\pgfpathcurveto{\pgfqpoint{5.546212in}{3.051250in}}{\pgfqpoint{5.541822in}{3.061849in}}{\pgfqpoint{5.534008in}{3.069663in}}%
\pgfpathcurveto{\pgfqpoint{5.526195in}{3.077477in}}{\pgfqpoint{5.515596in}{3.081867in}}{\pgfqpoint{5.504545in}{3.081867in}}%
\pgfpathcurveto{\pgfqpoint{5.493495in}{3.081867in}}{\pgfqpoint{5.482896in}{3.077477in}}{\pgfqpoint{5.475083in}{3.069663in}}%
\pgfpathcurveto{\pgfqpoint{5.467269in}{3.061849in}}{\pgfqpoint{5.462879in}{3.051250in}}{\pgfqpoint{5.462879in}{3.040200in}}%
\pgfpathcurveto{\pgfqpoint{5.462879in}{3.029150in}}{\pgfqpoint{5.467269in}{3.018551in}}{\pgfqpoint{5.475083in}{3.010738in}}%
\pgfpathcurveto{\pgfqpoint{5.482896in}{3.002924in}}{\pgfqpoint{5.493495in}{2.998534in}}{\pgfqpoint{5.504545in}{2.998534in}}%
\pgfpathclose%
\pgfusepath{stroke,fill}%
\end{pgfscope}%
\begin{pgfscope}%
\pgfpathrectangle{\pgfqpoint{0.800000in}{0.528000in}}{\pgfqpoint{4.960000in}{3.696000in}}%
\pgfusepath{clip}%
\pgfsetbuttcap%
\pgfsetroundjoin%
\definecolor{currentfill}{rgb}{0.000000,0.000000,0.000000}%
\pgfsetfillcolor{currentfill}%
\pgfsetlinewidth{1.003750pt}%
\definecolor{currentstroke}{rgb}{0.000000,0.000000,0.000000}%
\pgfsetstrokecolor{currentstroke}%
\pgfsetdash{}{0pt}%
\pgfpathmoveto{\pgfqpoint{5.504545in}{2.697569in}}%
\pgfpathcurveto{\pgfqpoint{5.515596in}{2.697569in}}{\pgfqpoint{5.526195in}{2.701959in}}{\pgfqpoint{5.534008in}{2.709773in}}%
\pgfpathcurveto{\pgfqpoint{5.541822in}{2.717586in}}{\pgfqpoint{5.546212in}{2.728185in}}{\pgfqpoint{5.546212in}{2.739235in}}%
\pgfpathcurveto{\pgfqpoint{5.546212in}{2.750286in}}{\pgfqpoint{5.541822in}{2.760885in}}{\pgfqpoint{5.534008in}{2.768698in}}%
\pgfpathcurveto{\pgfqpoint{5.526195in}{2.776512in}}{\pgfqpoint{5.515596in}{2.780902in}}{\pgfqpoint{5.504545in}{2.780902in}}%
\pgfpathcurveto{\pgfqpoint{5.493495in}{2.780902in}}{\pgfqpoint{5.482896in}{2.776512in}}{\pgfqpoint{5.475083in}{2.768698in}}%
\pgfpathcurveto{\pgfqpoint{5.467269in}{2.760885in}}{\pgfqpoint{5.462879in}{2.750286in}}{\pgfqpoint{5.462879in}{2.739235in}}%
\pgfpathcurveto{\pgfqpoint{5.462879in}{2.728185in}}{\pgfqpoint{5.467269in}{2.717586in}}{\pgfqpoint{5.475083in}{2.709773in}}%
\pgfpathcurveto{\pgfqpoint{5.482896in}{2.701959in}}{\pgfqpoint{5.493495in}{2.697569in}}{\pgfqpoint{5.504545in}{2.697569in}}%
\pgfpathclose%
\pgfusepath{stroke,fill}%
\end{pgfscope}%
\begin{pgfscope}%
\pgfpathrectangle{\pgfqpoint{0.800000in}{0.528000in}}{\pgfqpoint{4.960000in}{3.696000in}}%
\pgfusepath{clip}%
\pgfsetbuttcap%
\pgfsetroundjoin%
\definecolor{currentfill}{rgb}{0.000000,0.000000,0.000000}%
\pgfsetfillcolor{currentfill}%
\pgfsetlinewidth{1.003750pt}%
\definecolor{currentstroke}{rgb}{0.000000,0.000000,0.000000}%
\pgfsetstrokecolor{currentstroke}%
\pgfsetdash{}{0pt}%
\pgfpathmoveto{\pgfqpoint{5.504545in}{2.697569in}}%
\pgfpathcurveto{\pgfqpoint{5.515596in}{2.697569in}}{\pgfqpoint{5.526195in}{2.701959in}}{\pgfqpoint{5.534008in}{2.709773in}}%
\pgfpathcurveto{\pgfqpoint{5.541822in}{2.717586in}}{\pgfqpoint{5.546212in}{2.728185in}}{\pgfqpoint{5.546212in}{2.739235in}}%
\pgfpathcurveto{\pgfqpoint{5.546212in}{2.750286in}}{\pgfqpoint{5.541822in}{2.760885in}}{\pgfqpoint{5.534008in}{2.768698in}}%
\pgfpathcurveto{\pgfqpoint{5.526195in}{2.776512in}}{\pgfqpoint{5.515596in}{2.780902in}}{\pgfqpoint{5.504545in}{2.780902in}}%
\pgfpathcurveto{\pgfqpoint{5.493495in}{2.780902in}}{\pgfqpoint{5.482896in}{2.776512in}}{\pgfqpoint{5.475083in}{2.768698in}}%
\pgfpathcurveto{\pgfqpoint{5.467269in}{2.760885in}}{\pgfqpoint{5.462879in}{2.750286in}}{\pgfqpoint{5.462879in}{2.739235in}}%
\pgfpathcurveto{\pgfqpoint{5.462879in}{2.728185in}}{\pgfqpoint{5.467269in}{2.717586in}}{\pgfqpoint{5.475083in}{2.709773in}}%
\pgfpathcurveto{\pgfqpoint{5.482896in}{2.701959in}}{\pgfqpoint{5.493495in}{2.697569in}}{\pgfqpoint{5.504545in}{2.697569in}}%
\pgfpathclose%
\pgfusepath{stroke,fill}%
\end{pgfscope}%
\begin{pgfscope}%
\pgfpathrectangle{\pgfqpoint{0.800000in}{0.528000in}}{\pgfqpoint{4.960000in}{3.696000in}}%
\pgfusepath{clip}%
\pgfsetbuttcap%
\pgfsetroundjoin%
\definecolor{currentfill}{rgb}{0.000000,0.000000,0.000000}%
\pgfsetfillcolor{currentfill}%
\pgfsetlinewidth{1.003750pt}%
\definecolor{currentstroke}{rgb}{0.000000,0.000000,0.000000}%
\pgfsetstrokecolor{currentstroke}%
\pgfsetdash{}{0pt}%
\pgfpathmoveto{\pgfqpoint{5.504545in}{2.633076in}}%
\pgfpathcurveto{\pgfqpoint{5.515596in}{2.633076in}}{\pgfqpoint{5.526195in}{2.637467in}}{\pgfqpoint{5.534008in}{2.645280in}}%
\pgfpathcurveto{\pgfqpoint{5.541822in}{2.653094in}}{\pgfqpoint{5.546212in}{2.663693in}}{\pgfqpoint{5.546212in}{2.674743in}}%
\pgfpathcurveto{\pgfqpoint{5.546212in}{2.685793in}}{\pgfqpoint{5.541822in}{2.696392in}}{\pgfqpoint{5.534008in}{2.704206in}}%
\pgfpathcurveto{\pgfqpoint{5.526195in}{2.712019in}}{\pgfqpoint{5.515596in}{2.716410in}}{\pgfqpoint{5.504545in}{2.716410in}}%
\pgfpathcurveto{\pgfqpoint{5.493495in}{2.716410in}}{\pgfqpoint{5.482896in}{2.712019in}}{\pgfqpoint{5.475083in}{2.704206in}}%
\pgfpathcurveto{\pgfqpoint{5.467269in}{2.696392in}}{\pgfqpoint{5.462879in}{2.685793in}}{\pgfqpoint{5.462879in}{2.674743in}}%
\pgfpathcurveto{\pgfqpoint{5.462879in}{2.663693in}}{\pgfqpoint{5.467269in}{2.653094in}}{\pgfqpoint{5.475083in}{2.645280in}}%
\pgfpathcurveto{\pgfqpoint{5.482896in}{2.637467in}}{\pgfqpoint{5.493495in}{2.633076in}}{\pgfqpoint{5.504545in}{2.633076in}}%
\pgfpathclose%
\pgfusepath{stroke,fill}%
\end{pgfscope}%
\begin{pgfscope}%
\pgfpathrectangle{\pgfqpoint{0.800000in}{0.528000in}}{\pgfqpoint{4.960000in}{3.696000in}}%
\pgfusepath{clip}%
\pgfsetbuttcap%
\pgfsetroundjoin%
\definecolor{currentfill}{rgb}{0.000000,0.000000,0.000000}%
\pgfsetfillcolor{currentfill}%
\pgfsetlinewidth{1.003750pt}%
\definecolor{currentstroke}{rgb}{0.000000,0.000000,0.000000}%
\pgfsetstrokecolor{currentstroke}%
\pgfsetdash{}{0pt}%
\pgfpathmoveto{\pgfqpoint{5.504545in}{2.977036in}}%
\pgfpathcurveto{\pgfqpoint{5.515596in}{2.977036in}}{\pgfqpoint{5.526195in}{2.981426in}}{\pgfqpoint{5.534008in}{2.989240in}}%
\pgfpathcurveto{\pgfqpoint{5.541822in}{2.997054in}}{\pgfqpoint{5.546212in}{3.007653in}}{\pgfqpoint{5.546212in}{3.018703in}}%
\pgfpathcurveto{\pgfqpoint{5.546212in}{3.029753in}}{\pgfqpoint{5.541822in}{3.040352in}}{\pgfqpoint{5.534008in}{3.048166in}}%
\pgfpathcurveto{\pgfqpoint{5.526195in}{3.055979in}}{\pgfqpoint{5.515596in}{3.060369in}}{\pgfqpoint{5.504545in}{3.060369in}}%
\pgfpathcurveto{\pgfqpoint{5.493495in}{3.060369in}}{\pgfqpoint{5.482896in}{3.055979in}}{\pgfqpoint{5.475083in}{3.048166in}}%
\pgfpathcurveto{\pgfqpoint{5.467269in}{3.040352in}}{\pgfqpoint{5.462879in}{3.029753in}}{\pgfqpoint{5.462879in}{3.018703in}}%
\pgfpathcurveto{\pgfqpoint{5.462879in}{3.007653in}}{\pgfqpoint{5.467269in}{2.997054in}}{\pgfqpoint{5.475083in}{2.989240in}}%
\pgfpathcurveto{\pgfqpoint{5.482896in}{2.981426in}}{\pgfqpoint{5.493495in}{2.977036in}}{\pgfqpoint{5.504545in}{2.977036in}}%
\pgfpathclose%
\pgfusepath{stroke,fill}%
\end{pgfscope}%
\begin{pgfscope}%
\pgfpathrectangle{\pgfqpoint{0.800000in}{0.528000in}}{\pgfqpoint{4.960000in}{3.696000in}}%
\pgfusepath{clip}%
\pgfsetbuttcap%
\pgfsetroundjoin%
\definecolor{currentfill}{rgb}{0.000000,0.000000,0.000000}%
\pgfsetfillcolor{currentfill}%
\pgfsetlinewidth{1.003750pt}%
\definecolor{currentstroke}{rgb}{0.000000,0.000000,0.000000}%
\pgfsetstrokecolor{currentstroke}%
\pgfsetdash{}{0pt}%
\pgfpathmoveto{\pgfqpoint{5.504545in}{3.041529in}}%
\pgfpathcurveto{\pgfqpoint{5.515596in}{3.041529in}}{\pgfqpoint{5.526195in}{3.045919in}}{\pgfqpoint{5.534008in}{3.053732in}}%
\pgfpathcurveto{\pgfqpoint{5.541822in}{3.061546in}}{\pgfqpoint{5.546212in}{3.072145in}}{\pgfqpoint{5.546212in}{3.083195in}}%
\pgfpathcurveto{\pgfqpoint{5.546212in}{3.094245in}}{\pgfqpoint{5.541822in}{3.104844in}}{\pgfqpoint{5.534008in}{3.112658in}}%
\pgfpathcurveto{\pgfqpoint{5.526195in}{3.120472in}}{\pgfqpoint{5.515596in}{3.124862in}}{\pgfqpoint{5.504545in}{3.124862in}}%
\pgfpathcurveto{\pgfqpoint{5.493495in}{3.124862in}}{\pgfqpoint{5.482896in}{3.120472in}}{\pgfqpoint{5.475083in}{3.112658in}}%
\pgfpathcurveto{\pgfqpoint{5.467269in}{3.104844in}}{\pgfqpoint{5.462879in}{3.094245in}}{\pgfqpoint{5.462879in}{3.083195in}}%
\pgfpathcurveto{\pgfqpoint{5.462879in}{3.072145in}}{\pgfqpoint{5.467269in}{3.061546in}}{\pgfqpoint{5.475083in}{3.053732in}}%
\pgfpathcurveto{\pgfqpoint{5.482896in}{3.045919in}}{\pgfqpoint{5.493495in}{3.041529in}}{\pgfqpoint{5.504545in}{3.041529in}}%
\pgfpathclose%
\pgfusepath{stroke,fill}%
\end{pgfscope}%
\begin{pgfscope}%
\pgfpathrectangle{\pgfqpoint{0.800000in}{0.528000in}}{\pgfqpoint{4.960000in}{3.696000in}}%
\pgfusepath{clip}%
\pgfsetbuttcap%
\pgfsetroundjoin%
\definecolor{currentfill}{rgb}{0.000000,0.000000,0.000000}%
\pgfsetfillcolor{currentfill}%
\pgfsetlinewidth{1.003750pt}%
\definecolor{currentstroke}{rgb}{0.000000,0.000000,0.000000}%
\pgfsetstrokecolor{currentstroke}%
\pgfsetdash{}{0pt}%
\pgfpathmoveto{\pgfqpoint{5.504545in}{2.912544in}}%
\pgfpathcurveto{\pgfqpoint{5.515596in}{2.912544in}}{\pgfqpoint{5.526195in}{2.916934in}}{\pgfqpoint{5.534008in}{2.924748in}}%
\pgfpathcurveto{\pgfqpoint{5.541822in}{2.932561in}}{\pgfqpoint{5.546212in}{2.943160in}}{\pgfqpoint{5.546212in}{2.954210in}}%
\pgfpathcurveto{\pgfqpoint{5.546212in}{2.965260in}}{\pgfqpoint{5.541822in}{2.975859in}}{\pgfqpoint{5.534008in}{2.983673in}}%
\pgfpathcurveto{\pgfqpoint{5.526195in}{2.991487in}}{\pgfqpoint{5.515596in}{2.995877in}}{\pgfqpoint{5.504545in}{2.995877in}}%
\pgfpathcurveto{\pgfqpoint{5.493495in}{2.995877in}}{\pgfqpoint{5.482896in}{2.991487in}}{\pgfqpoint{5.475083in}{2.983673in}}%
\pgfpathcurveto{\pgfqpoint{5.467269in}{2.975859in}}{\pgfqpoint{5.462879in}{2.965260in}}{\pgfqpoint{5.462879in}{2.954210in}}%
\pgfpathcurveto{\pgfqpoint{5.462879in}{2.943160in}}{\pgfqpoint{5.467269in}{2.932561in}}{\pgfqpoint{5.475083in}{2.924748in}}%
\pgfpathcurveto{\pgfqpoint{5.482896in}{2.916934in}}{\pgfqpoint{5.493495in}{2.912544in}}{\pgfqpoint{5.504545in}{2.912544in}}%
\pgfpathclose%
\pgfusepath{stroke,fill}%
\end{pgfscope}%
\begin{pgfscope}%
\pgfpathrectangle{\pgfqpoint{0.800000in}{0.528000in}}{\pgfqpoint{4.960000in}{3.696000in}}%
\pgfusepath{clip}%
\pgfsetbuttcap%
\pgfsetroundjoin%
\definecolor{currentfill}{rgb}{0.000000,0.000000,0.000000}%
\pgfsetfillcolor{currentfill}%
\pgfsetlinewidth{1.003750pt}%
\definecolor{currentstroke}{rgb}{0.000000,0.000000,0.000000}%
\pgfsetstrokecolor{currentstroke}%
\pgfsetdash{}{0pt}%
\pgfpathmoveto{\pgfqpoint{5.504545in}{2.740564in}}%
\pgfpathcurveto{\pgfqpoint{5.515596in}{2.740564in}}{\pgfqpoint{5.526195in}{2.744954in}}{\pgfqpoint{5.534008in}{2.752768in}}%
\pgfpathcurveto{\pgfqpoint{5.541822in}{2.760581in}}{\pgfqpoint{5.546212in}{2.771180in}}{\pgfqpoint{5.546212in}{2.782230in}}%
\pgfpathcurveto{\pgfqpoint{5.546212in}{2.793281in}}{\pgfqpoint{5.541822in}{2.803880in}}{\pgfqpoint{5.534008in}{2.811693in}}%
\pgfpathcurveto{\pgfqpoint{5.526195in}{2.819507in}}{\pgfqpoint{5.515596in}{2.823897in}}{\pgfqpoint{5.504545in}{2.823897in}}%
\pgfpathcurveto{\pgfqpoint{5.493495in}{2.823897in}}{\pgfqpoint{5.482896in}{2.819507in}}{\pgfqpoint{5.475083in}{2.811693in}}%
\pgfpathcurveto{\pgfqpoint{5.467269in}{2.803880in}}{\pgfqpoint{5.462879in}{2.793281in}}{\pgfqpoint{5.462879in}{2.782230in}}%
\pgfpathcurveto{\pgfqpoint{5.462879in}{2.771180in}}{\pgfqpoint{5.467269in}{2.760581in}}{\pgfqpoint{5.475083in}{2.752768in}}%
\pgfpathcurveto{\pgfqpoint{5.482896in}{2.744954in}}{\pgfqpoint{5.493495in}{2.740564in}}{\pgfqpoint{5.504545in}{2.740564in}}%
\pgfpathclose%
\pgfusepath{stroke,fill}%
\end{pgfscope}%
\begin{pgfscope}%
\pgfpathrectangle{\pgfqpoint{0.800000in}{0.528000in}}{\pgfqpoint{4.960000in}{3.696000in}}%
\pgfusepath{clip}%
\pgfsetbuttcap%
\pgfsetroundjoin%
\definecolor{currentfill}{rgb}{0.000000,0.000000,0.000000}%
\pgfsetfillcolor{currentfill}%
\pgfsetlinewidth{1.003750pt}%
\definecolor{currentstroke}{rgb}{0.000000,0.000000,0.000000}%
\pgfsetstrokecolor{currentstroke}%
\pgfsetdash{}{0pt}%
\pgfpathmoveto{\pgfqpoint{5.504545in}{2.740564in}}%
\pgfpathcurveto{\pgfqpoint{5.515596in}{2.740564in}}{\pgfqpoint{5.526195in}{2.744954in}}{\pgfqpoint{5.534008in}{2.752768in}}%
\pgfpathcurveto{\pgfqpoint{5.541822in}{2.760581in}}{\pgfqpoint{5.546212in}{2.771180in}}{\pgfqpoint{5.546212in}{2.782230in}}%
\pgfpathcurveto{\pgfqpoint{5.546212in}{2.793281in}}{\pgfqpoint{5.541822in}{2.803880in}}{\pgfqpoint{5.534008in}{2.811693in}}%
\pgfpathcurveto{\pgfqpoint{5.526195in}{2.819507in}}{\pgfqpoint{5.515596in}{2.823897in}}{\pgfqpoint{5.504545in}{2.823897in}}%
\pgfpathcurveto{\pgfqpoint{5.493495in}{2.823897in}}{\pgfqpoint{5.482896in}{2.819507in}}{\pgfqpoint{5.475083in}{2.811693in}}%
\pgfpathcurveto{\pgfqpoint{5.467269in}{2.803880in}}{\pgfqpoint{5.462879in}{2.793281in}}{\pgfqpoint{5.462879in}{2.782230in}}%
\pgfpathcurveto{\pgfqpoint{5.462879in}{2.771180in}}{\pgfqpoint{5.467269in}{2.760581in}}{\pgfqpoint{5.475083in}{2.752768in}}%
\pgfpathcurveto{\pgfqpoint{5.482896in}{2.744954in}}{\pgfqpoint{5.493495in}{2.740564in}}{\pgfqpoint{5.504545in}{2.740564in}}%
\pgfpathclose%
\pgfusepath{stroke,fill}%
\end{pgfscope}%
\begin{pgfscope}%
\pgfpathrectangle{\pgfqpoint{0.800000in}{0.528000in}}{\pgfqpoint{4.960000in}{3.696000in}}%
\pgfusepath{clip}%
\pgfsetbuttcap%
\pgfsetroundjoin%
\definecolor{currentfill}{rgb}{0.000000,0.000000,0.000000}%
\pgfsetfillcolor{currentfill}%
\pgfsetlinewidth{1.003750pt}%
\definecolor{currentstroke}{rgb}{0.000000,0.000000,0.000000}%
\pgfsetstrokecolor{currentstroke}%
\pgfsetdash{}{0pt}%
\pgfpathmoveto{\pgfqpoint{5.504545in}{2.912544in}}%
\pgfpathcurveto{\pgfqpoint{5.515596in}{2.912544in}}{\pgfqpoint{5.526195in}{2.916934in}}{\pgfqpoint{5.534008in}{2.924748in}}%
\pgfpathcurveto{\pgfqpoint{5.541822in}{2.932561in}}{\pgfqpoint{5.546212in}{2.943160in}}{\pgfqpoint{5.546212in}{2.954210in}}%
\pgfpathcurveto{\pgfqpoint{5.546212in}{2.965260in}}{\pgfqpoint{5.541822in}{2.975859in}}{\pgfqpoint{5.534008in}{2.983673in}}%
\pgfpathcurveto{\pgfqpoint{5.526195in}{2.991487in}}{\pgfqpoint{5.515596in}{2.995877in}}{\pgfqpoint{5.504545in}{2.995877in}}%
\pgfpathcurveto{\pgfqpoint{5.493495in}{2.995877in}}{\pgfqpoint{5.482896in}{2.991487in}}{\pgfqpoint{5.475083in}{2.983673in}}%
\pgfpathcurveto{\pgfqpoint{5.467269in}{2.975859in}}{\pgfqpoint{5.462879in}{2.965260in}}{\pgfqpoint{5.462879in}{2.954210in}}%
\pgfpathcurveto{\pgfqpoint{5.462879in}{2.943160in}}{\pgfqpoint{5.467269in}{2.932561in}}{\pgfqpoint{5.475083in}{2.924748in}}%
\pgfpathcurveto{\pgfqpoint{5.482896in}{2.916934in}}{\pgfqpoint{5.493495in}{2.912544in}}{\pgfqpoint{5.504545in}{2.912544in}}%
\pgfpathclose%
\pgfusepath{stroke,fill}%
\end{pgfscope}%
\begin{pgfscope}%
\pgfpathrectangle{\pgfqpoint{0.800000in}{0.528000in}}{\pgfqpoint{4.960000in}{3.696000in}}%
\pgfusepath{clip}%
\pgfsetbuttcap%
\pgfsetroundjoin%
\definecolor{currentfill}{rgb}{0.000000,0.000000,0.000000}%
\pgfsetfillcolor{currentfill}%
\pgfsetlinewidth{1.003750pt}%
\definecolor{currentstroke}{rgb}{0.000000,0.000000,0.000000}%
\pgfsetstrokecolor{currentstroke}%
\pgfsetdash{}{0pt}%
\pgfpathmoveto{\pgfqpoint{5.504545in}{2.783559in}}%
\pgfpathcurveto{\pgfqpoint{5.515596in}{2.783559in}}{\pgfqpoint{5.526195in}{2.787949in}}{\pgfqpoint{5.534008in}{2.795763in}}%
\pgfpathcurveto{\pgfqpoint{5.541822in}{2.803576in}}{\pgfqpoint{5.546212in}{2.814175in}}{\pgfqpoint{5.546212in}{2.825225in}}%
\pgfpathcurveto{\pgfqpoint{5.546212in}{2.836275in}}{\pgfqpoint{5.541822in}{2.846875in}}{\pgfqpoint{5.534008in}{2.854688in}}%
\pgfpathcurveto{\pgfqpoint{5.526195in}{2.862502in}}{\pgfqpoint{5.515596in}{2.866892in}}{\pgfqpoint{5.504545in}{2.866892in}}%
\pgfpathcurveto{\pgfqpoint{5.493495in}{2.866892in}}{\pgfqpoint{5.482896in}{2.862502in}}{\pgfqpoint{5.475083in}{2.854688in}}%
\pgfpathcurveto{\pgfqpoint{5.467269in}{2.846875in}}{\pgfqpoint{5.462879in}{2.836275in}}{\pgfqpoint{5.462879in}{2.825225in}}%
\pgfpathcurveto{\pgfqpoint{5.462879in}{2.814175in}}{\pgfqpoint{5.467269in}{2.803576in}}{\pgfqpoint{5.475083in}{2.795763in}}%
\pgfpathcurveto{\pgfqpoint{5.482896in}{2.787949in}}{\pgfqpoint{5.493495in}{2.783559in}}{\pgfqpoint{5.504545in}{2.783559in}}%
\pgfpathclose%
\pgfusepath{stroke,fill}%
\end{pgfscope}%
\begin{pgfscope}%
\pgfpathrectangle{\pgfqpoint{0.800000in}{0.528000in}}{\pgfqpoint{4.960000in}{3.696000in}}%
\pgfusepath{clip}%
\pgfsetbuttcap%
\pgfsetroundjoin%
\definecolor{currentfill}{rgb}{0.000000,0.000000,0.000000}%
\pgfsetfillcolor{currentfill}%
\pgfsetlinewidth{1.003750pt}%
\definecolor{currentstroke}{rgb}{0.000000,0.000000,0.000000}%
\pgfsetstrokecolor{currentstroke}%
\pgfsetdash{}{0pt}%
\pgfpathmoveto{\pgfqpoint{5.504545in}{2.762061in}}%
\pgfpathcurveto{\pgfqpoint{5.515596in}{2.762061in}}{\pgfqpoint{5.526195in}{2.766451in}}{\pgfqpoint{5.534008in}{2.774265in}}%
\pgfpathcurveto{\pgfqpoint{5.541822in}{2.782079in}}{\pgfqpoint{5.546212in}{2.792678in}}{\pgfqpoint{5.546212in}{2.803728in}}%
\pgfpathcurveto{\pgfqpoint{5.546212in}{2.814778in}}{\pgfqpoint{5.541822in}{2.825377in}}{\pgfqpoint{5.534008in}{2.833191in}}%
\pgfpathcurveto{\pgfqpoint{5.526195in}{2.841004in}}{\pgfqpoint{5.515596in}{2.845395in}}{\pgfqpoint{5.504545in}{2.845395in}}%
\pgfpathcurveto{\pgfqpoint{5.493495in}{2.845395in}}{\pgfqpoint{5.482896in}{2.841004in}}{\pgfqpoint{5.475083in}{2.833191in}}%
\pgfpathcurveto{\pgfqpoint{5.467269in}{2.825377in}}{\pgfqpoint{5.462879in}{2.814778in}}{\pgfqpoint{5.462879in}{2.803728in}}%
\pgfpathcurveto{\pgfqpoint{5.462879in}{2.792678in}}{\pgfqpoint{5.467269in}{2.782079in}}{\pgfqpoint{5.475083in}{2.774265in}}%
\pgfpathcurveto{\pgfqpoint{5.482896in}{2.766451in}}{\pgfqpoint{5.493495in}{2.762061in}}{\pgfqpoint{5.504545in}{2.762061in}}%
\pgfpathclose%
\pgfusepath{stroke,fill}%
\end{pgfscope}%
\begin{pgfscope}%
\pgfpathrectangle{\pgfqpoint{0.800000in}{0.528000in}}{\pgfqpoint{4.960000in}{3.696000in}}%
\pgfusepath{clip}%
\pgfsetbuttcap%
\pgfsetroundjoin%
\definecolor{currentfill}{rgb}{0.000000,0.000000,0.000000}%
\pgfsetfillcolor{currentfill}%
\pgfsetlinewidth{1.003750pt}%
\definecolor{currentstroke}{rgb}{0.000000,0.000000,0.000000}%
\pgfsetstrokecolor{currentstroke}%
\pgfsetdash{}{0pt}%
\pgfpathmoveto{\pgfqpoint{5.504545in}{2.762061in}}%
\pgfpathcurveto{\pgfqpoint{5.515596in}{2.762061in}}{\pgfqpoint{5.526195in}{2.766451in}}{\pgfqpoint{5.534008in}{2.774265in}}%
\pgfpathcurveto{\pgfqpoint{5.541822in}{2.782079in}}{\pgfqpoint{5.546212in}{2.792678in}}{\pgfqpoint{5.546212in}{2.803728in}}%
\pgfpathcurveto{\pgfqpoint{5.546212in}{2.814778in}}{\pgfqpoint{5.541822in}{2.825377in}}{\pgfqpoint{5.534008in}{2.833191in}}%
\pgfpathcurveto{\pgfqpoint{5.526195in}{2.841004in}}{\pgfqpoint{5.515596in}{2.845395in}}{\pgfqpoint{5.504545in}{2.845395in}}%
\pgfpathcurveto{\pgfqpoint{5.493495in}{2.845395in}}{\pgfqpoint{5.482896in}{2.841004in}}{\pgfqpoint{5.475083in}{2.833191in}}%
\pgfpathcurveto{\pgfqpoint{5.467269in}{2.825377in}}{\pgfqpoint{5.462879in}{2.814778in}}{\pgfqpoint{5.462879in}{2.803728in}}%
\pgfpathcurveto{\pgfqpoint{5.462879in}{2.792678in}}{\pgfqpoint{5.467269in}{2.782079in}}{\pgfqpoint{5.475083in}{2.774265in}}%
\pgfpathcurveto{\pgfqpoint{5.482896in}{2.766451in}}{\pgfqpoint{5.493495in}{2.762061in}}{\pgfqpoint{5.504545in}{2.762061in}}%
\pgfpathclose%
\pgfusepath{stroke,fill}%
\end{pgfscope}%
\begin{pgfscope}%
\pgfpathrectangle{\pgfqpoint{0.800000in}{0.528000in}}{\pgfqpoint{4.960000in}{3.696000in}}%
\pgfusepath{clip}%
\pgfsetbuttcap%
\pgfsetroundjoin%
\definecolor{currentfill}{rgb}{0.000000,0.000000,0.000000}%
\pgfsetfillcolor{currentfill}%
\pgfsetlinewidth{1.003750pt}%
\definecolor{currentstroke}{rgb}{0.000000,0.000000,0.000000}%
\pgfsetstrokecolor{currentstroke}%
\pgfsetdash{}{0pt}%
\pgfpathmoveto{\pgfqpoint{5.504545in}{2.762061in}}%
\pgfpathcurveto{\pgfqpoint{5.515596in}{2.762061in}}{\pgfqpoint{5.526195in}{2.766451in}}{\pgfqpoint{5.534008in}{2.774265in}}%
\pgfpathcurveto{\pgfqpoint{5.541822in}{2.782079in}}{\pgfqpoint{5.546212in}{2.792678in}}{\pgfqpoint{5.546212in}{2.803728in}}%
\pgfpathcurveto{\pgfqpoint{5.546212in}{2.814778in}}{\pgfqpoint{5.541822in}{2.825377in}}{\pgfqpoint{5.534008in}{2.833191in}}%
\pgfpathcurveto{\pgfqpoint{5.526195in}{2.841004in}}{\pgfqpoint{5.515596in}{2.845395in}}{\pgfqpoint{5.504545in}{2.845395in}}%
\pgfpathcurveto{\pgfqpoint{5.493495in}{2.845395in}}{\pgfqpoint{5.482896in}{2.841004in}}{\pgfqpoint{5.475083in}{2.833191in}}%
\pgfpathcurveto{\pgfqpoint{5.467269in}{2.825377in}}{\pgfqpoint{5.462879in}{2.814778in}}{\pgfqpoint{5.462879in}{2.803728in}}%
\pgfpathcurveto{\pgfqpoint{5.462879in}{2.792678in}}{\pgfqpoint{5.467269in}{2.782079in}}{\pgfqpoint{5.475083in}{2.774265in}}%
\pgfpathcurveto{\pgfqpoint{5.482896in}{2.766451in}}{\pgfqpoint{5.493495in}{2.762061in}}{\pgfqpoint{5.504545in}{2.762061in}}%
\pgfpathclose%
\pgfusepath{stroke,fill}%
\end{pgfscope}%
\begin{pgfscope}%
\pgfpathrectangle{\pgfqpoint{0.800000in}{0.528000in}}{\pgfqpoint{4.960000in}{3.696000in}}%
\pgfusepath{clip}%
\pgfsetbuttcap%
\pgfsetroundjoin%
\definecolor{currentfill}{rgb}{0.000000,0.000000,0.000000}%
\pgfsetfillcolor{currentfill}%
\pgfsetlinewidth{1.003750pt}%
\definecolor{currentstroke}{rgb}{0.000000,0.000000,0.000000}%
\pgfsetstrokecolor{currentstroke}%
\pgfsetdash{}{0pt}%
\pgfpathmoveto{\pgfqpoint{5.504545in}{3.256504in}}%
\pgfpathcurveto{\pgfqpoint{5.515596in}{3.256504in}}{\pgfqpoint{5.526195in}{3.260894in}}{\pgfqpoint{5.534008in}{3.268707in}}%
\pgfpathcurveto{\pgfqpoint{5.541822in}{3.276521in}}{\pgfqpoint{5.546212in}{3.287120in}}{\pgfqpoint{5.546212in}{3.298170in}}%
\pgfpathcurveto{\pgfqpoint{5.546212in}{3.309220in}}{\pgfqpoint{5.541822in}{3.319819in}}{\pgfqpoint{5.534008in}{3.327633in}}%
\pgfpathcurveto{\pgfqpoint{5.526195in}{3.335447in}}{\pgfqpoint{5.515596in}{3.339837in}}{\pgfqpoint{5.504545in}{3.339837in}}%
\pgfpathcurveto{\pgfqpoint{5.493495in}{3.339837in}}{\pgfqpoint{5.482896in}{3.335447in}}{\pgfqpoint{5.475083in}{3.327633in}}%
\pgfpathcurveto{\pgfqpoint{5.467269in}{3.319819in}}{\pgfqpoint{5.462879in}{3.309220in}}{\pgfqpoint{5.462879in}{3.298170in}}%
\pgfpathcurveto{\pgfqpoint{5.462879in}{3.287120in}}{\pgfqpoint{5.467269in}{3.276521in}}{\pgfqpoint{5.475083in}{3.268707in}}%
\pgfpathcurveto{\pgfqpoint{5.482896in}{3.260894in}}{\pgfqpoint{5.493495in}{3.256504in}}{\pgfqpoint{5.504545in}{3.256504in}}%
\pgfpathclose%
\pgfusepath{stroke,fill}%
\end{pgfscope}%
\begin{pgfscope}%
\pgfpathrectangle{\pgfqpoint{0.800000in}{0.528000in}}{\pgfqpoint{4.960000in}{3.696000in}}%
\pgfusepath{clip}%
\pgfsetbuttcap%
\pgfsetroundjoin%
\definecolor{currentfill}{rgb}{0.000000,0.000000,0.000000}%
\pgfsetfillcolor{currentfill}%
\pgfsetlinewidth{1.003750pt}%
\definecolor{currentstroke}{rgb}{0.000000,0.000000,0.000000}%
\pgfsetstrokecolor{currentstroke}%
\pgfsetdash{}{0pt}%
\pgfpathmoveto{\pgfqpoint{5.504545in}{3.041529in}}%
\pgfpathcurveto{\pgfqpoint{5.515596in}{3.041529in}}{\pgfqpoint{5.526195in}{3.045919in}}{\pgfqpoint{5.534008in}{3.053732in}}%
\pgfpathcurveto{\pgfqpoint{5.541822in}{3.061546in}}{\pgfqpoint{5.546212in}{3.072145in}}{\pgfqpoint{5.546212in}{3.083195in}}%
\pgfpathcurveto{\pgfqpoint{5.546212in}{3.094245in}}{\pgfqpoint{5.541822in}{3.104844in}}{\pgfqpoint{5.534008in}{3.112658in}}%
\pgfpathcurveto{\pgfqpoint{5.526195in}{3.120472in}}{\pgfqpoint{5.515596in}{3.124862in}}{\pgfqpoint{5.504545in}{3.124862in}}%
\pgfpathcurveto{\pgfqpoint{5.493495in}{3.124862in}}{\pgfqpoint{5.482896in}{3.120472in}}{\pgfqpoint{5.475083in}{3.112658in}}%
\pgfpathcurveto{\pgfqpoint{5.467269in}{3.104844in}}{\pgfqpoint{5.462879in}{3.094245in}}{\pgfqpoint{5.462879in}{3.083195in}}%
\pgfpathcurveto{\pgfqpoint{5.462879in}{3.072145in}}{\pgfqpoint{5.467269in}{3.061546in}}{\pgfqpoint{5.475083in}{3.053732in}}%
\pgfpathcurveto{\pgfqpoint{5.482896in}{3.045919in}}{\pgfqpoint{5.493495in}{3.041529in}}{\pgfqpoint{5.504545in}{3.041529in}}%
\pgfpathclose%
\pgfusepath{stroke,fill}%
\end{pgfscope}%
\begin{pgfscope}%
\pgfpathrectangle{\pgfqpoint{0.800000in}{0.528000in}}{\pgfqpoint{4.960000in}{3.696000in}}%
\pgfusepath{clip}%
\pgfsetbuttcap%
\pgfsetroundjoin%
\definecolor{currentfill}{rgb}{0.000000,0.000000,0.000000}%
\pgfsetfillcolor{currentfill}%
\pgfsetlinewidth{1.003750pt}%
\definecolor{currentstroke}{rgb}{0.000000,0.000000,0.000000}%
\pgfsetstrokecolor{currentstroke}%
\pgfsetdash{}{0pt}%
\pgfpathmoveto{\pgfqpoint{5.504545in}{2.719066in}}%
\pgfpathcurveto{\pgfqpoint{5.515596in}{2.719066in}}{\pgfqpoint{5.526195in}{2.723456in}}{\pgfqpoint{5.534008in}{2.731270in}}%
\pgfpathcurveto{\pgfqpoint{5.541822in}{2.739084in}}{\pgfqpoint{5.546212in}{2.749683in}}{\pgfqpoint{5.546212in}{2.760733in}}%
\pgfpathcurveto{\pgfqpoint{5.546212in}{2.771783in}}{\pgfqpoint{5.541822in}{2.782382in}}{\pgfqpoint{5.534008in}{2.790196in}}%
\pgfpathcurveto{\pgfqpoint{5.526195in}{2.798009in}}{\pgfqpoint{5.515596in}{2.802400in}}{\pgfqpoint{5.504545in}{2.802400in}}%
\pgfpathcurveto{\pgfqpoint{5.493495in}{2.802400in}}{\pgfqpoint{5.482896in}{2.798009in}}{\pgfqpoint{5.475083in}{2.790196in}}%
\pgfpathcurveto{\pgfqpoint{5.467269in}{2.782382in}}{\pgfqpoint{5.462879in}{2.771783in}}{\pgfqpoint{5.462879in}{2.760733in}}%
\pgfpathcurveto{\pgfqpoint{5.462879in}{2.749683in}}{\pgfqpoint{5.467269in}{2.739084in}}{\pgfqpoint{5.475083in}{2.731270in}}%
\pgfpathcurveto{\pgfqpoint{5.482896in}{2.723456in}}{\pgfqpoint{5.493495in}{2.719066in}}{\pgfqpoint{5.504545in}{2.719066in}}%
\pgfpathclose%
\pgfusepath{stroke,fill}%
\end{pgfscope}%
\begin{pgfscope}%
\pgfpathrectangle{\pgfqpoint{0.800000in}{0.528000in}}{\pgfqpoint{4.960000in}{3.696000in}}%
\pgfusepath{clip}%
\pgfsetbuttcap%
\pgfsetroundjoin%
\definecolor{currentfill}{rgb}{0.000000,0.000000,0.000000}%
\pgfsetfillcolor{currentfill}%
\pgfsetlinewidth{1.003750pt}%
\definecolor{currentstroke}{rgb}{0.000000,0.000000,0.000000}%
\pgfsetstrokecolor{currentstroke}%
\pgfsetdash{}{0pt}%
\pgfpathmoveto{\pgfqpoint{5.504545in}{2.654574in}}%
\pgfpathcurveto{\pgfqpoint{5.515596in}{2.654574in}}{\pgfqpoint{5.526195in}{2.658964in}}{\pgfqpoint{5.534008in}{2.666778in}}%
\pgfpathcurveto{\pgfqpoint{5.541822in}{2.674591in}}{\pgfqpoint{5.546212in}{2.685190in}}{\pgfqpoint{5.546212in}{2.696240in}}%
\pgfpathcurveto{\pgfqpoint{5.546212in}{2.707291in}}{\pgfqpoint{5.541822in}{2.717890in}}{\pgfqpoint{5.534008in}{2.725703in}}%
\pgfpathcurveto{\pgfqpoint{5.526195in}{2.733517in}}{\pgfqpoint{5.515596in}{2.737907in}}{\pgfqpoint{5.504545in}{2.737907in}}%
\pgfpathcurveto{\pgfqpoint{5.493495in}{2.737907in}}{\pgfqpoint{5.482896in}{2.733517in}}{\pgfqpoint{5.475083in}{2.725703in}}%
\pgfpathcurveto{\pgfqpoint{5.467269in}{2.717890in}}{\pgfqpoint{5.462879in}{2.707291in}}{\pgfqpoint{5.462879in}{2.696240in}}%
\pgfpathcurveto{\pgfqpoint{5.462879in}{2.685190in}}{\pgfqpoint{5.467269in}{2.674591in}}{\pgfqpoint{5.475083in}{2.666778in}}%
\pgfpathcurveto{\pgfqpoint{5.482896in}{2.658964in}}{\pgfqpoint{5.493495in}{2.654574in}}{\pgfqpoint{5.504545in}{2.654574in}}%
\pgfpathclose%
\pgfusepath{stroke,fill}%
\end{pgfscope}%
\begin{pgfscope}%
\pgfpathrectangle{\pgfqpoint{0.800000in}{0.528000in}}{\pgfqpoint{4.960000in}{3.696000in}}%
\pgfusepath{clip}%
\pgfsetbuttcap%
\pgfsetroundjoin%
\definecolor{currentfill}{rgb}{0.000000,0.000000,0.000000}%
\pgfsetfillcolor{currentfill}%
\pgfsetlinewidth{1.003750pt}%
\definecolor{currentstroke}{rgb}{0.000000,0.000000,0.000000}%
\pgfsetstrokecolor{currentstroke}%
\pgfsetdash{}{0pt}%
\pgfpathmoveto{\pgfqpoint{5.504545in}{2.977036in}}%
\pgfpathcurveto{\pgfqpoint{5.515596in}{2.977036in}}{\pgfqpoint{5.526195in}{2.981426in}}{\pgfqpoint{5.534008in}{2.989240in}}%
\pgfpathcurveto{\pgfqpoint{5.541822in}{2.997054in}}{\pgfqpoint{5.546212in}{3.007653in}}{\pgfqpoint{5.546212in}{3.018703in}}%
\pgfpathcurveto{\pgfqpoint{5.546212in}{3.029753in}}{\pgfqpoint{5.541822in}{3.040352in}}{\pgfqpoint{5.534008in}{3.048166in}}%
\pgfpathcurveto{\pgfqpoint{5.526195in}{3.055979in}}{\pgfqpoint{5.515596in}{3.060369in}}{\pgfqpoint{5.504545in}{3.060369in}}%
\pgfpathcurveto{\pgfqpoint{5.493495in}{3.060369in}}{\pgfqpoint{5.482896in}{3.055979in}}{\pgfqpoint{5.475083in}{3.048166in}}%
\pgfpathcurveto{\pgfqpoint{5.467269in}{3.040352in}}{\pgfqpoint{5.462879in}{3.029753in}}{\pgfqpoint{5.462879in}{3.018703in}}%
\pgfpathcurveto{\pgfqpoint{5.462879in}{3.007653in}}{\pgfqpoint{5.467269in}{2.997054in}}{\pgfqpoint{5.475083in}{2.989240in}}%
\pgfpathcurveto{\pgfqpoint{5.482896in}{2.981426in}}{\pgfqpoint{5.493495in}{2.977036in}}{\pgfqpoint{5.504545in}{2.977036in}}%
\pgfpathclose%
\pgfusepath{stroke,fill}%
\end{pgfscope}%
\begin{pgfscope}%
\pgfpathrectangle{\pgfqpoint{0.800000in}{0.528000in}}{\pgfqpoint{4.960000in}{3.696000in}}%
\pgfusepath{clip}%
\pgfsetbuttcap%
\pgfsetroundjoin%
\definecolor{currentfill}{rgb}{0.000000,0.000000,0.000000}%
\pgfsetfillcolor{currentfill}%
\pgfsetlinewidth{1.003750pt}%
\definecolor{currentstroke}{rgb}{0.000000,0.000000,0.000000}%
\pgfsetstrokecolor{currentstroke}%
\pgfsetdash{}{0pt}%
\pgfpathmoveto{\pgfqpoint{5.504545in}{2.762061in}}%
\pgfpathcurveto{\pgfqpoint{5.515596in}{2.762061in}}{\pgfqpoint{5.526195in}{2.766451in}}{\pgfqpoint{5.534008in}{2.774265in}}%
\pgfpathcurveto{\pgfqpoint{5.541822in}{2.782079in}}{\pgfqpoint{5.546212in}{2.792678in}}{\pgfqpoint{5.546212in}{2.803728in}}%
\pgfpathcurveto{\pgfqpoint{5.546212in}{2.814778in}}{\pgfqpoint{5.541822in}{2.825377in}}{\pgfqpoint{5.534008in}{2.833191in}}%
\pgfpathcurveto{\pgfqpoint{5.526195in}{2.841004in}}{\pgfqpoint{5.515596in}{2.845395in}}{\pgfqpoint{5.504545in}{2.845395in}}%
\pgfpathcurveto{\pgfqpoint{5.493495in}{2.845395in}}{\pgfqpoint{5.482896in}{2.841004in}}{\pgfqpoint{5.475083in}{2.833191in}}%
\pgfpathcurveto{\pgfqpoint{5.467269in}{2.825377in}}{\pgfqpoint{5.462879in}{2.814778in}}{\pgfqpoint{5.462879in}{2.803728in}}%
\pgfpathcurveto{\pgfqpoint{5.462879in}{2.792678in}}{\pgfqpoint{5.467269in}{2.782079in}}{\pgfqpoint{5.475083in}{2.774265in}}%
\pgfpathcurveto{\pgfqpoint{5.482896in}{2.766451in}}{\pgfqpoint{5.493495in}{2.762061in}}{\pgfqpoint{5.504545in}{2.762061in}}%
\pgfpathclose%
\pgfusepath{stroke,fill}%
\end{pgfscope}%
\begin{pgfscope}%
\pgfpathrectangle{\pgfqpoint{0.800000in}{0.528000in}}{\pgfqpoint{4.960000in}{3.696000in}}%
\pgfusepath{clip}%
\pgfsetbuttcap%
\pgfsetroundjoin%
\definecolor{currentfill}{rgb}{0.000000,0.000000,0.000000}%
\pgfsetfillcolor{currentfill}%
\pgfsetlinewidth{1.003750pt}%
\definecolor{currentstroke}{rgb}{0.000000,0.000000,0.000000}%
\pgfsetstrokecolor{currentstroke}%
\pgfsetdash{}{0pt}%
\pgfpathmoveto{\pgfqpoint{5.504545in}{3.471478in}}%
\pgfpathcurveto{\pgfqpoint{5.515596in}{3.471478in}}{\pgfqpoint{5.526195in}{3.475869in}}{\pgfqpoint{5.534008in}{3.483682in}}%
\pgfpathcurveto{\pgfqpoint{5.541822in}{3.491496in}}{\pgfqpoint{5.546212in}{3.502095in}}{\pgfqpoint{5.546212in}{3.513145in}}%
\pgfpathcurveto{\pgfqpoint{5.546212in}{3.524195in}}{\pgfqpoint{5.541822in}{3.534794in}}{\pgfqpoint{5.534008in}{3.542608in}}%
\pgfpathcurveto{\pgfqpoint{5.526195in}{3.550422in}}{\pgfqpoint{5.515596in}{3.554812in}}{\pgfqpoint{5.504545in}{3.554812in}}%
\pgfpathcurveto{\pgfqpoint{5.493495in}{3.554812in}}{\pgfqpoint{5.482896in}{3.550422in}}{\pgfqpoint{5.475083in}{3.542608in}}%
\pgfpathcurveto{\pgfqpoint{5.467269in}{3.534794in}}{\pgfqpoint{5.462879in}{3.524195in}}{\pgfqpoint{5.462879in}{3.513145in}}%
\pgfpathcurveto{\pgfqpoint{5.462879in}{3.502095in}}{\pgfqpoint{5.467269in}{3.491496in}}{\pgfqpoint{5.475083in}{3.483682in}}%
\pgfpathcurveto{\pgfqpoint{5.482896in}{3.475869in}}{\pgfqpoint{5.493495in}{3.471478in}}{\pgfqpoint{5.504545in}{3.471478in}}%
\pgfpathclose%
\pgfusepath{stroke,fill}%
\end{pgfscope}%
\begin{pgfscope}%
\pgfpathrectangle{\pgfqpoint{0.800000in}{0.528000in}}{\pgfqpoint{4.960000in}{3.696000in}}%
\pgfusepath{clip}%
\pgfsetbuttcap%
\pgfsetroundjoin%
\definecolor{currentfill}{rgb}{0.000000,0.000000,0.000000}%
\pgfsetfillcolor{currentfill}%
\pgfsetlinewidth{1.003750pt}%
\definecolor{currentstroke}{rgb}{0.000000,0.000000,0.000000}%
\pgfsetstrokecolor{currentstroke}%
\pgfsetdash{}{0pt}%
\pgfpathmoveto{\pgfqpoint{5.504545in}{3.643458in}}%
\pgfpathcurveto{\pgfqpoint{5.515596in}{3.643458in}}{\pgfqpoint{5.526195in}{3.647849in}}{\pgfqpoint{5.534008in}{3.655662in}}%
\pgfpathcurveto{\pgfqpoint{5.541822in}{3.663476in}}{\pgfqpoint{5.546212in}{3.674075in}}{\pgfqpoint{5.546212in}{3.685125in}}%
\pgfpathcurveto{\pgfqpoint{5.546212in}{3.696175in}}{\pgfqpoint{5.541822in}{3.706774in}}{\pgfqpoint{5.534008in}{3.714588in}}%
\pgfpathcurveto{\pgfqpoint{5.526195in}{3.722401in}}{\pgfqpoint{5.515596in}{3.726792in}}{\pgfqpoint{5.504545in}{3.726792in}}%
\pgfpathcurveto{\pgfqpoint{5.493495in}{3.726792in}}{\pgfqpoint{5.482896in}{3.722401in}}{\pgfqpoint{5.475083in}{3.714588in}}%
\pgfpathcurveto{\pgfqpoint{5.467269in}{3.706774in}}{\pgfqpoint{5.462879in}{3.696175in}}{\pgfqpoint{5.462879in}{3.685125in}}%
\pgfpathcurveto{\pgfqpoint{5.462879in}{3.674075in}}{\pgfqpoint{5.467269in}{3.663476in}}{\pgfqpoint{5.475083in}{3.655662in}}%
\pgfpathcurveto{\pgfqpoint{5.482896in}{3.647849in}}{\pgfqpoint{5.493495in}{3.643458in}}{\pgfqpoint{5.504545in}{3.643458in}}%
\pgfpathclose%
\pgfusepath{stroke,fill}%
\end{pgfscope}%
\begin{pgfscope}%
\pgfpathrectangle{\pgfqpoint{0.800000in}{0.528000in}}{\pgfqpoint{4.960000in}{3.696000in}}%
\pgfusepath{clip}%
\pgfsetbuttcap%
\pgfsetroundjoin%
\definecolor{currentfill}{rgb}{0.000000,0.000000,0.000000}%
\pgfsetfillcolor{currentfill}%
\pgfsetlinewidth{1.003750pt}%
\definecolor{currentstroke}{rgb}{0.000000,0.000000,0.000000}%
\pgfsetstrokecolor{currentstroke}%
\pgfsetdash{}{0pt}%
\pgfpathmoveto{\pgfqpoint{5.504545in}{2.826554in}}%
\pgfpathcurveto{\pgfqpoint{5.515596in}{2.826554in}}{\pgfqpoint{5.526195in}{2.830944in}}{\pgfqpoint{5.534008in}{2.838758in}}%
\pgfpathcurveto{\pgfqpoint{5.541822in}{2.846571in}}{\pgfqpoint{5.546212in}{2.857170in}}{\pgfqpoint{5.546212in}{2.868220in}}%
\pgfpathcurveto{\pgfqpoint{5.546212in}{2.879270in}}{\pgfqpoint{5.541822in}{2.889870in}}{\pgfqpoint{5.534008in}{2.897683in}}%
\pgfpathcurveto{\pgfqpoint{5.526195in}{2.905497in}}{\pgfqpoint{5.515596in}{2.909887in}}{\pgfqpoint{5.504545in}{2.909887in}}%
\pgfpathcurveto{\pgfqpoint{5.493495in}{2.909887in}}{\pgfqpoint{5.482896in}{2.905497in}}{\pgfqpoint{5.475083in}{2.897683in}}%
\pgfpathcurveto{\pgfqpoint{5.467269in}{2.889870in}}{\pgfqpoint{5.462879in}{2.879270in}}{\pgfqpoint{5.462879in}{2.868220in}}%
\pgfpathcurveto{\pgfqpoint{5.462879in}{2.857170in}}{\pgfqpoint{5.467269in}{2.846571in}}{\pgfqpoint{5.475083in}{2.838758in}}%
\pgfpathcurveto{\pgfqpoint{5.482896in}{2.830944in}}{\pgfqpoint{5.493495in}{2.826554in}}{\pgfqpoint{5.504545in}{2.826554in}}%
\pgfpathclose%
\pgfusepath{stroke,fill}%
\end{pgfscope}%
\begin{pgfscope}%
\pgfpathrectangle{\pgfqpoint{0.800000in}{0.528000in}}{\pgfqpoint{4.960000in}{3.696000in}}%
\pgfusepath{clip}%
\pgfsetbuttcap%
\pgfsetroundjoin%
\definecolor{currentfill}{rgb}{0.000000,0.000000,0.000000}%
\pgfsetfillcolor{currentfill}%
\pgfsetlinewidth{1.003750pt}%
\definecolor{currentstroke}{rgb}{0.000000,0.000000,0.000000}%
\pgfsetstrokecolor{currentstroke}%
\pgfsetdash{}{0pt}%
\pgfpathmoveto{\pgfqpoint{5.504545in}{2.740564in}}%
\pgfpathcurveto{\pgfqpoint{5.515596in}{2.740564in}}{\pgfqpoint{5.526195in}{2.744954in}}{\pgfqpoint{5.534008in}{2.752768in}}%
\pgfpathcurveto{\pgfqpoint{5.541822in}{2.760581in}}{\pgfqpoint{5.546212in}{2.771180in}}{\pgfqpoint{5.546212in}{2.782230in}}%
\pgfpathcurveto{\pgfqpoint{5.546212in}{2.793281in}}{\pgfqpoint{5.541822in}{2.803880in}}{\pgfqpoint{5.534008in}{2.811693in}}%
\pgfpathcurveto{\pgfqpoint{5.526195in}{2.819507in}}{\pgfqpoint{5.515596in}{2.823897in}}{\pgfqpoint{5.504545in}{2.823897in}}%
\pgfpathcurveto{\pgfqpoint{5.493495in}{2.823897in}}{\pgfqpoint{5.482896in}{2.819507in}}{\pgfqpoint{5.475083in}{2.811693in}}%
\pgfpathcurveto{\pgfqpoint{5.467269in}{2.803880in}}{\pgfqpoint{5.462879in}{2.793281in}}{\pgfqpoint{5.462879in}{2.782230in}}%
\pgfpathcurveto{\pgfqpoint{5.462879in}{2.771180in}}{\pgfqpoint{5.467269in}{2.760581in}}{\pgfqpoint{5.475083in}{2.752768in}}%
\pgfpathcurveto{\pgfqpoint{5.482896in}{2.744954in}}{\pgfqpoint{5.493495in}{2.740564in}}{\pgfqpoint{5.504545in}{2.740564in}}%
\pgfpathclose%
\pgfusepath{stroke,fill}%
\end{pgfscope}%
\begin{pgfscope}%
\pgfpathrectangle{\pgfqpoint{0.800000in}{0.528000in}}{\pgfqpoint{4.960000in}{3.696000in}}%
\pgfusepath{clip}%
\pgfsetbuttcap%
\pgfsetroundjoin%
\definecolor{currentfill}{rgb}{0.000000,0.000000,0.000000}%
\pgfsetfillcolor{currentfill}%
\pgfsetlinewidth{1.003750pt}%
\definecolor{currentstroke}{rgb}{0.000000,0.000000,0.000000}%
\pgfsetstrokecolor{currentstroke}%
\pgfsetdash{}{0pt}%
\pgfpathmoveto{\pgfqpoint{5.504545in}{2.912544in}}%
\pgfpathcurveto{\pgfqpoint{5.515596in}{2.912544in}}{\pgfqpoint{5.526195in}{2.916934in}}{\pgfqpoint{5.534008in}{2.924748in}}%
\pgfpathcurveto{\pgfqpoint{5.541822in}{2.932561in}}{\pgfqpoint{5.546212in}{2.943160in}}{\pgfqpoint{5.546212in}{2.954210in}}%
\pgfpathcurveto{\pgfqpoint{5.546212in}{2.965260in}}{\pgfqpoint{5.541822in}{2.975859in}}{\pgfqpoint{5.534008in}{2.983673in}}%
\pgfpathcurveto{\pgfqpoint{5.526195in}{2.991487in}}{\pgfqpoint{5.515596in}{2.995877in}}{\pgfqpoint{5.504545in}{2.995877in}}%
\pgfpathcurveto{\pgfqpoint{5.493495in}{2.995877in}}{\pgfqpoint{5.482896in}{2.991487in}}{\pgfqpoint{5.475083in}{2.983673in}}%
\pgfpathcurveto{\pgfqpoint{5.467269in}{2.975859in}}{\pgfqpoint{5.462879in}{2.965260in}}{\pgfqpoint{5.462879in}{2.954210in}}%
\pgfpathcurveto{\pgfqpoint{5.462879in}{2.943160in}}{\pgfqpoint{5.467269in}{2.932561in}}{\pgfqpoint{5.475083in}{2.924748in}}%
\pgfpathcurveto{\pgfqpoint{5.482896in}{2.916934in}}{\pgfqpoint{5.493495in}{2.912544in}}{\pgfqpoint{5.504545in}{2.912544in}}%
\pgfpathclose%
\pgfusepath{stroke,fill}%
\end{pgfscope}%
\begin{pgfscope}%
\pgfpathrectangle{\pgfqpoint{0.800000in}{0.528000in}}{\pgfqpoint{4.960000in}{3.696000in}}%
\pgfusepath{clip}%
\pgfsetbuttcap%
\pgfsetroundjoin%
\definecolor{currentfill}{rgb}{0.000000,0.000000,0.000000}%
\pgfsetfillcolor{currentfill}%
\pgfsetlinewidth{1.003750pt}%
\definecolor{currentstroke}{rgb}{0.000000,0.000000,0.000000}%
\pgfsetstrokecolor{currentstroke}%
\pgfsetdash{}{0pt}%
\pgfpathmoveto{\pgfqpoint{5.504545in}{2.869549in}}%
\pgfpathcurveto{\pgfqpoint{5.515596in}{2.869549in}}{\pgfqpoint{5.526195in}{2.873939in}}{\pgfqpoint{5.534008in}{2.881753in}}%
\pgfpathcurveto{\pgfqpoint{5.541822in}{2.889566in}}{\pgfqpoint{5.546212in}{2.900165in}}{\pgfqpoint{5.546212in}{2.911215in}}%
\pgfpathcurveto{\pgfqpoint{5.546212in}{2.922265in}}{\pgfqpoint{5.541822in}{2.932864in}}{\pgfqpoint{5.534008in}{2.940678in}}%
\pgfpathcurveto{\pgfqpoint{5.526195in}{2.948492in}}{\pgfqpoint{5.515596in}{2.952882in}}{\pgfqpoint{5.504545in}{2.952882in}}%
\pgfpathcurveto{\pgfqpoint{5.493495in}{2.952882in}}{\pgfqpoint{5.482896in}{2.948492in}}{\pgfqpoint{5.475083in}{2.940678in}}%
\pgfpathcurveto{\pgfqpoint{5.467269in}{2.932864in}}{\pgfqpoint{5.462879in}{2.922265in}}{\pgfqpoint{5.462879in}{2.911215in}}%
\pgfpathcurveto{\pgfqpoint{5.462879in}{2.900165in}}{\pgfqpoint{5.467269in}{2.889566in}}{\pgfqpoint{5.475083in}{2.881753in}}%
\pgfpathcurveto{\pgfqpoint{5.482896in}{2.873939in}}{\pgfqpoint{5.493495in}{2.869549in}}{\pgfqpoint{5.504545in}{2.869549in}}%
\pgfpathclose%
\pgfusepath{stroke,fill}%
\end{pgfscope}%
\begin{pgfscope}%
\pgfpathrectangle{\pgfqpoint{0.800000in}{0.528000in}}{\pgfqpoint{4.960000in}{3.696000in}}%
\pgfusepath{clip}%
\pgfsetbuttcap%
\pgfsetroundjoin%
\definecolor{currentfill}{rgb}{0.000000,0.000000,0.000000}%
\pgfsetfillcolor{currentfill}%
\pgfsetlinewidth{1.003750pt}%
\definecolor{currentstroke}{rgb}{0.000000,0.000000,0.000000}%
\pgfsetstrokecolor{currentstroke}%
\pgfsetdash{}{0pt}%
\pgfpathmoveto{\pgfqpoint{5.504545in}{2.719066in}}%
\pgfpathcurveto{\pgfqpoint{5.515596in}{2.719066in}}{\pgfqpoint{5.526195in}{2.723456in}}{\pgfqpoint{5.534008in}{2.731270in}}%
\pgfpathcurveto{\pgfqpoint{5.541822in}{2.739084in}}{\pgfqpoint{5.546212in}{2.749683in}}{\pgfqpoint{5.546212in}{2.760733in}}%
\pgfpathcurveto{\pgfqpoint{5.546212in}{2.771783in}}{\pgfqpoint{5.541822in}{2.782382in}}{\pgfqpoint{5.534008in}{2.790196in}}%
\pgfpathcurveto{\pgfqpoint{5.526195in}{2.798009in}}{\pgfqpoint{5.515596in}{2.802400in}}{\pgfqpoint{5.504545in}{2.802400in}}%
\pgfpathcurveto{\pgfqpoint{5.493495in}{2.802400in}}{\pgfqpoint{5.482896in}{2.798009in}}{\pgfqpoint{5.475083in}{2.790196in}}%
\pgfpathcurveto{\pgfqpoint{5.467269in}{2.782382in}}{\pgfqpoint{5.462879in}{2.771783in}}{\pgfqpoint{5.462879in}{2.760733in}}%
\pgfpathcurveto{\pgfqpoint{5.462879in}{2.749683in}}{\pgfqpoint{5.467269in}{2.739084in}}{\pgfqpoint{5.475083in}{2.731270in}}%
\pgfpathcurveto{\pgfqpoint{5.482896in}{2.723456in}}{\pgfqpoint{5.493495in}{2.719066in}}{\pgfqpoint{5.504545in}{2.719066in}}%
\pgfpathclose%
\pgfusepath{stroke,fill}%
\end{pgfscope}%
\begin{pgfscope}%
\pgfpathrectangle{\pgfqpoint{0.800000in}{0.528000in}}{\pgfqpoint{4.960000in}{3.696000in}}%
\pgfusepath{clip}%
\pgfsetbuttcap%
\pgfsetroundjoin%
\definecolor{currentfill}{rgb}{0.000000,0.000000,0.000000}%
\pgfsetfillcolor{currentfill}%
\pgfsetlinewidth{1.003750pt}%
\definecolor{currentstroke}{rgb}{0.000000,0.000000,0.000000}%
\pgfsetstrokecolor{currentstroke}%
\pgfsetdash{}{0pt}%
\pgfpathmoveto{\pgfqpoint{5.504545in}{3.084524in}}%
\pgfpathcurveto{\pgfqpoint{5.515596in}{3.084524in}}{\pgfqpoint{5.526195in}{3.088914in}}{\pgfqpoint{5.534008in}{3.096727in}}%
\pgfpathcurveto{\pgfqpoint{5.541822in}{3.104541in}}{\pgfqpoint{5.546212in}{3.115140in}}{\pgfqpoint{5.546212in}{3.126190in}}%
\pgfpathcurveto{\pgfqpoint{5.546212in}{3.137240in}}{\pgfqpoint{5.541822in}{3.147839in}}{\pgfqpoint{5.534008in}{3.155653in}}%
\pgfpathcurveto{\pgfqpoint{5.526195in}{3.163467in}}{\pgfqpoint{5.515596in}{3.167857in}}{\pgfqpoint{5.504545in}{3.167857in}}%
\pgfpathcurveto{\pgfqpoint{5.493495in}{3.167857in}}{\pgfqpoint{5.482896in}{3.163467in}}{\pgfqpoint{5.475083in}{3.155653in}}%
\pgfpathcurveto{\pgfqpoint{5.467269in}{3.147839in}}{\pgfqpoint{5.462879in}{3.137240in}}{\pgfqpoint{5.462879in}{3.126190in}}%
\pgfpathcurveto{\pgfqpoint{5.462879in}{3.115140in}}{\pgfqpoint{5.467269in}{3.104541in}}{\pgfqpoint{5.475083in}{3.096727in}}%
\pgfpathcurveto{\pgfqpoint{5.482896in}{3.088914in}}{\pgfqpoint{5.493495in}{3.084524in}}{\pgfqpoint{5.504545in}{3.084524in}}%
\pgfpathclose%
\pgfusepath{stroke,fill}%
\end{pgfscope}%
\begin{pgfscope}%
\pgfpathrectangle{\pgfqpoint{0.800000in}{0.528000in}}{\pgfqpoint{4.960000in}{3.696000in}}%
\pgfusepath{clip}%
\pgfsetbuttcap%
\pgfsetroundjoin%
\definecolor{currentfill}{rgb}{0.000000,0.000000,0.000000}%
\pgfsetfillcolor{currentfill}%
\pgfsetlinewidth{1.003750pt}%
\definecolor{currentstroke}{rgb}{0.000000,0.000000,0.000000}%
\pgfsetstrokecolor{currentstroke}%
\pgfsetdash{}{0pt}%
\pgfpathmoveto{\pgfqpoint{5.504545in}{2.934041in}}%
\pgfpathcurveto{\pgfqpoint{5.515596in}{2.934041in}}{\pgfqpoint{5.526195in}{2.938431in}}{\pgfqpoint{5.534008in}{2.946245in}}%
\pgfpathcurveto{\pgfqpoint{5.541822in}{2.954059in}}{\pgfqpoint{5.546212in}{2.964658in}}{\pgfqpoint{5.546212in}{2.975708in}}%
\pgfpathcurveto{\pgfqpoint{5.546212in}{2.986758in}}{\pgfqpoint{5.541822in}{2.997357in}}{\pgfqpoint{5.534008in}{3.005171in}}%
\pgfpathcurveto{\pgfqpoint{5.526195in}{3.012984in}}{\pgfqpoint{5.515596in}{3.017374in}}{\pgfqpoint{5.504545in}{3.017374in}}%
\pgfpathcurveto{\pgfqpoint{5.493495in}{3.017374in}}{\pgfqpoint{5.482896in}{3.012984in}}{\pgfqpoint{5.475083in}{3.005171in}}%
\pgfpathcurveto{\pgfqpoint{5.467269in}{2.997357in}}{\pgfqpoint{5.462879in}{2.986758in}}{\pgfqpoint{5.462879in}{2.975708in}}%
\pgfpathcurveto{\pgfqpoint{5.462879in}{2.964658in}}{\pgfqpoint{5.467269in}{2.954059in}}{\pgfqpoint{5.475083in}{2.946245in}}%
\pgfpathcurveto{\pgfqpoint{5.482896in}{2.938431in}}{\pgfqpoint{5.493495in}{2.934041in}}{\pgfqpoint{5.504545in}{2.934041in}}%
\pgfpathclose%
\pgfusepath{stroke,fill}%
\end{pgfscope}%
\begin{pgfscope}%
\pgfpathrectangle{\pgfqpoint{0.800000in}{0.528000in}}{\pgfqpoint{4.960000in}{3.696000in}}%
\pgfusepath{clip}%
\pgfsetbuttcap%
\pgfsetroundjoin%
\definecolor{currentfill}{rgb}{0.000000,0.000000,0.000000}%
\pgfsetfillcolor{currentfill}%
\pgfsetlinewidth{1.003750pt}%
\definecolor{currentstroke}{rgb}{0.000000,0.000000,0.000000}%
\pgfsetstrokecolor{currentstroke}%
\pgfsetdash{}{0pt}%
\pgfpathmoveto{\pgfqpoint{5.504545in}{2.719066in}}%
\pgfpathcurveto{\pgfqpoint{5.515596in}{2.719066in}}{\pgfqpoint{5.526195in}{2.723456in}}{\pgfqpoint{5.534008in}{2.731270in}}%
\pgfpathcurveto{\pgfqpoint{5.541822in}{2.739084in}}{\pgfqpoint{5.546212in}{2.749683in}}{\pgfqpoint{5.546212in}{2.760733in}}%
\pgfpathcurveto{\pgfqpoint{5.546212in}{2.771783in}}{\pgfqpoint{5.541822in}{2.782382in}}{\pgfqpoint{5.534008in}{2.790196in}}%
\pgfpathcurveto{\pgfqpoint{5.526195in}{2.798009in}}{\pgfqpoint{5.515596in}{2.802400in}}{\pgfqpoint{5.504545in}{2.802400in}}%
\pgfpathcurveto{\pgfqpoint{5.493495in}{2.802400in}}{\pgfqpoint{5.482896in}{2.798009in}}{\pgfqpoint{5.475083in}{2.790196in}}%
\pgfpathcurveto{\pgfqpoint{5.467269in}{2.782382in}}{\pgfqpoint{5.462879in}{2.771783in}}{\pgfqpoint{5.462879in}{2.760733in}}%
\pgfpathcurveto{\pgfqpoint{5.462879in}{2.749683in}}{\pgfqpoint{5.467269in}{2.739084in}}{\pgfqpoint{5.475083in}{2.731270in}}%
\pgfpathcurveto{\pgfqpoint{5.482896in}{2.723456in}}{\pgfqpoint{5.493495in}{2.719066in}}{\pgfqpoint{5.504545in}{2.719066in}}%
\pgfpathclose%
\pgfusepath{stroke,fill}%
\end{pgfscope}%
\begin{pgfscope}%
\pgfpathrectangle{\pgfqpoint{0.800000in}{0.528000in}}{\pgfqpoint{4.960000in}{3.696000in}}%
\pgfusepath{clip}%
\pgfsetbuttcap%
\pgfsetroundjoin%
\definecolor{currentfill}{rgb}{0.000000,0.000000,0.000000}%
\pgfsetfillcolor{currentfill}%
\pgfsetlinewidth{1.003750pt}%
\definecolor{currentstroke}{rgb}{0.000000,0.000000,0.000000}%
\pgfsetstrokecolor{currentstroke}%
\pgfsetdash{}{0pt}%
\pgfpathmoveto{\pgfqpoint{5.504545in}{3.170514in}}%
\pgfpathcurveto{\pgfqpoint{5.515596in}{3.170514in}}{\pgfqpoint{5.526195in}{3.174904in}}{\pgfqpoint{5.534008in}{3.182717in}}%
\pgfpathcurveto{\pgfqpoint{5.541822in}{3.190531in}}{\pgfqpoint{5.546212in}{3.201130in}}{\pgfqpoint{5.546212in}{3.212180in}}%
\pgfpathcurveto{\pgfqpoint{5.546212in}{3.223230in}}{\pgfqpoint{5.541822in}{3.233829in}}{\pgfqpoint{5.534008in}{3.241643in}}%
\pgfpathcurveto{\pgfqpoint{5.526195in}{3.249457in}}{\pgfqpoint{5.515596in}{3.253847in}}{\pgfqpoint{5.504545in}{3.253847in}}%
\pgfpathcurveto{\pgfqpoint{5.493495in}{3.253847in}}{\pgfqpoint{5.482896in}{3.249457in}}{\pgfqpoint{5.475083in}{3.241643in}}%
\pgfpathcurveto{\pgfqpoint{5.467269in}{3.233829in}}{\pgfqpoint{5.462879in}{3.223230in}}{\pgfqpoint{5.462879in}{3.212180in}}%
\pgfpathcurveto{\pgfqpoint{5.462879in}{3.201130in}}{\pgfqpoint{5.467269in}{3.190531in}}{\pgfqpoint{5.475083in}{3.182717in}}%
\pgfpathcurveto{\pgfqpoint{5.482896in}{3.174904in}}{\pgfqpoint{5.493495in}{3.170514in}}{\pgfqpoint{5.504545in}{3.170514in}}%
\pgfpathclose%
\pgfusepath{stroke,fill}%
\end{pgfscope}%
\begin{pgfscope}%
\pgfpathrectangle{\pgfqpoint{0.800000in}{0.528000in}}{\pgfqpoint{4.960000in}{3.696000in}}%
\pgfusepath{clip}%
\pgfsetbuttcap%
\pgfsetroundjoin%
\definecolor{currentfill}{rgb}{0.000000,0.000000,0.000000}%
\pgfsetfillcolor{currentfill}%
\pgfsetlinewidth{1.003750pt}%
\definecolor{currentstroke}{rgb}{0.000000,0.000000,0.000000}%
\pgfsetstrokecolor{currentstroke}%
\pgfsetdash{}{0pt}%
\pgfpathmoveto{\pgfqpoint{5.504545in}{2.740564in}}%
\pgfpathcurveto{\pgfqpoint{5.515596in}{2.740564in}}{\pgfqpoint{5.526195in}{2.744954in}}{\pgfqpoint{5.534008in}{2.752768in}}%
\pgfpathcurveto{\pgfqpoint{5.541822in}{2.760581in}}{\pgfqpoint{5.546212in}{2.771180in}}{\pgfqpoint{5.546212in}{2.782230in}}%
\pgfpathcurveto{\pgfqpoint{5.546212in}{2.793281in}}{\pgfqpoint{5.541822in}{2.803880in}}{\pgfqpoint{5.534008in}{2.811693in}}%
\pgfpathcurveto{\pgfqpoint{5.526195in}{2.819507in}}{\pgfqpoint{5.515596in}{2.823897in}}{\pgfqpoint{5.504545in}{2.823897in}}%
\pgfpathcurveto{\pgfqpoint{5.493495in}{2.823897in}}{\pgfqpoint{5.482896in}{2.819507in}}{\pgfqpoint{5.475083in}{2.811693in}}%
\pgfpathcurveto{\pgfqpoint{5.467269in}{2.803880in}}{\pgfqpoint{5.462879in}{2.793281in}}{\pgfqpoint{5.462879in}{2.782230in}}%
\pgfpathcurveto{\pgfqpoint{5.462879in}{2.771180in}}{\pgfqpoint{5.467269in}{2.760581in}}{\pgfqpoint{5.475083in}{2.752768in}}%
\pgfpathcurveto{\pgfqpoint{5.482896in}{2.744954in}}{\pgfqpoint{5.493495in}{2.740564in}}{\pgfqpoint{5.504545in}{2.740564in}}%
\pgfpathclose%
\pgfusepath{stroke,fill}%
\end{pgfscope}%
\begin{pgfscope}%
\pgfpathrectangle{\pgfqpoint{0.800000in}{0.528000in}}{\pgfqpoint{4.960000in}{3.696000in}}%
\pgfusepath{clip}%
\pgfsetbuttcap%
\pgfsetroundjoin%
\definecolor{currentfill}{rgb}{0.000000,0.000000,0.000000}%
\pgfsetfillcolor{currentfill}%
\pgfsetlinewidth{1.003750pt}%
\definecolor{currentstroke}{rgb}{0.000000,0.000000,0.000000}%
\pgfsetstrokecolor{currentstroke}%
\pgfsetdash{}{0pt}%
\pgfpathmoveto{\pgfqpoint{5.504545in}{3.127519in}}%
\pgfpathcurveto{\pgfqpoint{5.515596in}{3.127519in}}{\pgfqpoint{5.526195in}{3.131909in}}{\pgfqpoint{5.534008in}{3.139722in}}%
\pgfpathcurveto{\pgfqpoint{5.541822in}{3.147536in}}{\pgfqpoint{5.546212in}{3.158135in}}{\pgfqpoint{5.546212in}{3.169185in}}%
\pgfpathcurveto{\pgfqpoint{5.546212in}{3.180235in}}{\pgfqpoint{5.541822in}{3.190834in}}{\pgfqpoint{5.534008in}{3.198648in}}%
\pgfpathcurveto{\pgfqpoint{5.526195in}{3.206462in}}{\pgfqpoint{5.515596in}{3.210852in}}{\pgfqpoint{5.504545in}{3.210852in}}%
\pgfpathcurveto{\pgfqpoint{5.493495in}{3.210852in}}{\pgfqpoint{5.482896in}{3.206462in}}{\pgfqpoint{5.475083in}{3.198648in}}%
\pgfpathcurveto{\pgfqpoint{5.467269in}{3.190834in}}{\pgfqpoint{5.462879in}{3.180235in}}{\pgfqpoint{5.462879in}{3.169185in}}%
\pgfpathcurveto{\pgfqpoint{5.462879in}{3.158135in}}{\pgfqpoint{5.467269in}{3.147536in}}{\pgfqpoint{5.475083in}{3.139722in}}%
\pgfpathcurveto{\pgfqpoint{5.482896in}{3.131909in}}{\pgfqpoint{5.493495in}{3.127519in}}{\pgfqpoint{5.504545in}{3.127519in}}%
\pgfpathclose%
\pgfusepath{stroke,fill}%
\end{pgfscope}%
\begin{pgfscope}%
\pgfpathrectangle{\pgfqpoint{0.800000in}{0.528000in}}{\pgfqpoint{4.960000in}{3.696000in}}%
\pgfusepath{clip}%
\pgfsetbuttcap%
\pgfsetroundjoin%
\definecolor{currentfill}{rgb}{0.000000,0.000000,0.000000}%
\pgfsetfillcolor{currentfill}%
\pgfsetlinewidth{1.003750pt}%
\definecolor{currentstroke}{rgb}{0.000000,0.000000,0.000000}%
\pgfsetstrokecolor{currentstroke}%
\pgfsetdash{}{0pt}%
\pgfpathmoveto{\pgfqpoint{5.504545in}{2.998534in}}%
\pgfpathcurveto{\pgfqpoint{5.515596in}{2.998534in}}{\pgfqpoint{5.526195in}{3.002924in}}{\pgfqpoint{5.534008in}{3.010738in}}%
\pgfpathcurveto{\pgfqpoint{5.541822in}{3.018551in}}{\pgfqpoint{5.546212in}{3.029150in}}{\pgfqpoint{5.546212in}{3.040200in}}%
\pgfpathcurveto{\pgfqpoint{5.546212in}{3.051250in}}{\pgfqpoint{5.541822in}{3.061849in}}{\pgfqpoint{5.534008in}{3.069663in}}%
\pgfpathcurveto{\pgfqpoint{5.526195in}{3.077477in}}{\pgfqpoint{5.515596in}{3.081867in}}{\pgfqpoint{5.504545in}{3.081867in}}%
\pgfpathcurveto{\pgfqpoint{5.493495in}{3.081867in}}{\pgfqpoint{5.482896in}{3.077477in}}{\pgfqpoint{5.475083in}{3.069663in}}%
\pgfpathcurveto{\pgfqpoint{5.467269in}{3.061849in}}{\pgfqpoint{5.462879in}{3.051250in}}{\pgfqpoint{5.462879in}{3.040200in}}%
\pgfpathcurveto{\pgfqpoint{5.462879in}{3.029150in}}{\pgfqpoint{5.467269in}{3.018551in}}{\pgfqpoint{5.475083in}{3.010738in}}%
\pgfpathcurveto{\pgfqpoint{5.482896in}{3.002924in}}{\pgfqpoint{5.493495in}{2.998534in}}{\pgfqpoint{5.504545in}{2.998534in}}%
\pgfpathclose%
\pgfusepath{stroke,fill}%
\end{pgfscope}%
\begin{pgfscope}%
\pgfpathrectangle{\pgfqpoint{0.800000in}{0.528000in}}{\pgfqpoint{4.960000in}{3.696000in}}%
\pgfusepath{clip}%
\pgfsetbuttcap%
\pgfsetroundjoin%
\definecolor{currentfill}{rgb}{0.000000,0.000000,0.000000}%
\pgfsetfillcolor{currentfill}%
\pgfsetlinewidth{1.003750pt}%
\definecolor{currentstroke}{rgb}{0.000000,0.000000,0.000000}%
\pgfsetstrokecolor{currentstroke}%
\pgfsetdash{}{0pt}%
\pgfpathmoveto{\pgfqpoint{5.504545in}{2.848051in}}%
\pgfpathcurveto{\pgfqpoint{5.515596in}{2.848051in}}{\pgfqpoint{5.526195in}{2.852441in}}{\pgfqpoint{5.534008in}{2.860255in}}%
\pgfpathcurveto{\pgfqpoint{5.541822in}{2.868069in}}{\pgfqpoint{5.546212in}{2.878668in}}{\pgfqpoint{5.546212in}{2.889718in}}%
\pgfpathcurveto{\pgfqpoint{5.546212in}{2.900768in}}{\pgfqpoint{5.541822in}{2.911367in}}{\pgfqpoint{5.534008in}{2.919181in}}%
\pgfpathcurveto{\pgfqpoint{5.526195in}{2.926994in}}{\pgfqpoint{5.515596in}{2.931385in}}{\pgfqpoint{5.504545in}{2.931385in}}%
\pgfpathcurveto{\pgfqpoint{5.493495in}{2.931385in}}{\pgfqpoint{5.482896in}{2.926994in}}{\pgfqpoint{5.475083in}{2.919181in}}%
\pgfpathcurveto{\pgfqpoint{5.467269in}{2.911367in}}{\pgfqpoint{5.462879in}{2.900768in}}{\pgfqpoint{5.462879in}{2.889718in}}%
\pgfpathcurveto{\pgfqpoint{5.462879in}{2.878668in}}{\pgfqpoint{5.467269in}{2.868069in}}{\pgfqpoint{5.475083in}{2.860255in}}%
\pgfpathcurveto{\pgfqpoint{5.482896in}{2.852441in}}{\pgfqpoint{5.493495in}{2.848051in}}{\pgfqpoint{5.504545in}{2.848051in}}%
\pgfpathclose%
\pgfusepath{stroke,fill}%
\end{pgfscope}%
\begin{pgfscope}%
\pgfpathrectangle{\pgfqpoint{0.800000in}{0.528000in}}{\pgfqpoint{4.960000in}{3.696000in}}%
\pgfusepath{clip}%
\pgfsetbuttcap%
\pgfsetroundjoin%
\definecolor{currentfill}{rgb}{0.000000,0.000000,0.000000}%
\pgfsetfillcolor{currentfill}%
\pgfsetlinewidth{1.003750pt}%
\definecolor{currentstroke}{rgb}{0.000000,0.000000,0.000000}%
\pgfsetstrokecolor{currentstroke}%
\pgfsetdash{}{0pt}%
\pgfpathmoveto{\pgfqpoint{5.504545in}{3.320996in}}%
\pgfpathcurveto{\pgfqpoint{5.515596in}{3.320996in}}{\pgfqpoint{5.526195in}{3.325386in}}{\pgfqpoint{5.534008in}{3.333200in}}%
\pgfpathcurveto{\pgfqpoint{5.541822in}{3.341014in}}{\pgfqpoint{5.546212in}{3.351613in}}{\pgfqpoint{5.546212in}{3.362663in}}%
\pgfpathcurveto{\pgfqpoint{5.546212in}{3.373713in}}{\pgfqpoint{5.541822in}{3.384312in}}{\pgfqpoint{5.534008in}{3.392125in}}%
\pgfpathcurveto{\pgfqpoint{5.526195in}{3.399939in}}{\pgfqpoint{5.515596in}{3.404329in}}{\pgfqpoint{5.504545in}{3.404329in}}%
\pgfpathcurveto{\pgfqpoint{5.493495in}{3.404329in}}{\pgfqpoint{5.482896in}{3.399939in}}{\pgfqpoint{5.475083in}{3.392125in}}%
\pgfpathcurveto{\pgfqpoint{5.467269in}{3.384312in}}{\pgfqpoint{5.462879in}{3.373713in}}{\pgfqpoint{5.462879in}{3.362663in}}%
\pgfpathcurveto{\pgfqpoint{5.462879in}{3.351613in}}{\pgfqpoint{5.467269in}{3.341014in}}{\pgfqpoint{5.475083in}{3.333200in}}%
\pgfpathcurveto{\pgfqpoint{5.482896in}{3.325386in}}{\pgfqpoint{5.493495in}{3.320996in}}{\pgfqpoint{5.504545in}{3.320996in}}%
\pgfpathclose%
\pgfusepath{stroke,fill}%
\end{pgfscope}%
\begin{pgfscope}%
\pgfpathrectangle{\pgfqpoint{0.800000in}{0.528000in}}{\pgfqpoint{4.960000in}{3.696000in}}%
\pgfusepath{clip}%
\pgfsetbuttcap%
\pgfsetroundjoin%
\definecolor{currentfill}{rgb}{0.000000,0.000000,0.000000}%
\pgfsetfillcolor{currentfill}%
\pgfsetlinewidth{1.003750pt}%
\definecolor{currentstroke}{rgb}{0.000000,0.000000,0.000000}%
\pgfsetstrokecolor{currentstroke}%
\pgfsetdash{}{0pt}%
\pgfpathmoveto{\pgfqpoint{5.504545in}{2.826554in}}%
\pgfpathcurveto{\pgfqpoint{5.515596in}{2.826554in}}{\pgfqpoint{5.526195in}{2.830944in}}{\pgfqpoint{5.534008in}{2.838758in}}%
\pgfpathcurveto{\pgfqpoint{5.541822in}{2.846571in}}{\pgfqpoint{5.546212in}{2.857170in}}{\pgfqpoint{5.546212in}{2.868220in}}%
\pgfpathcurveto{\pgfqpoint{5.546212in}{2.879270in}}{\pgfqpoint{5.541822in}{2.889870in}}{\pgfqpoint{5.534008in}{2.897683in}}%
\pgfpathcurveto{\pgfqpoint{5.526195in}{2.905497in}}{\pgfqpoint{5.515596in}{2.909887in}}{\pgfqpoint{5.504545in}{2.909887in}}%
\pgfpathcurveto{\pgfqpoint{5.493495in}{2.909887in}}{\pgfqpoint{5.482896in}{2.905497in}}{\pgfqpoint{5.475083in}{2.897683in}}%
\pgfpathcurveto{\pgfqpoint{5.467269in}{2.889870in}}{\pgfqpoint{5.462879in}{2.879270in}}{\pgfqpoint{5.462879in}{2.868220in}}%
\pgfpathcurveto{\pgfqpoint{5.462879in}{2.857170in}}{\pgfqpoint{5.467269in}{2.846571in}}{\pgfqpoint{5.475083in}{2.838758in}}%
\pgfpathcurveto{\pgfqpoint{5.482896in}{2.830944in}}{\pgfqpoint{5.493495in}{2.826554in}}{\pgfqpoint{5.504545in}{2.826554in}}%
\pgfpathclose%
\pgfusepath{stroke,fill}%
\end{pgfscope}%
\begin{pgfscope}%
\pgfpathrectangle{\pgfqpoint{0.800000in}{0.528000in}}{\pgfqpoint{4.960000in}{3.696000in}}%
\pgfusepath{clip}%
\pgfsetbuttcap%
\pgfsetroundjoin%
\definecolor{currentfill}{rgb}{0.000000,0.000000,0.000000}%
\pgfsetfillcolor{currentfill}%
\pgfsetlinewidth{1.003750pt}%
\definecolor{currentstroke}{rgb}{0.000000,0.000000,0.000000}%
\pgfsetstrokecolor{currentstroke}%
\pgfsetdash{}{0pt}%
\pgfpathmoveto{\pgfqpoint{5.504545in}{3.987418in}}%
\pgfpathcurveto{\pgfqpoint{5.515596in}{3.987418in}}{\pgfqpoint{5.526195in}{3.991809in}}{\pgfqpoint{5.534008in}{3.999622in}}%
\pgfpathcurveto{\pgfqpoint{5.541822in}{4.007436in}}{\pgfqpoint{5.546212in}{4.018035in}}{\pgfqpoint{5.546212in}{4.029085in}}%
\pgfpathcurveto{\pgfqpoint{5.546212in}{4.040135in}}{\pgfqpoint{5.541822in}{4.050734in}}{\pgfqpoint{5.534008in}{4.058548in}}%
\pgfpathcurveto{\pgfqpoint{5.526195in}{4.066361in}}{\pgfqpoint{5.515596in}{4.070752in}}{\pgfqpoint{5.504545in}{4.070752in}}%
\pgfpathcurveto{\pgfqpoint{5.493495in}{4.070752in}}{\pgfqpoint{5.482896in}{4.066361in}}{\pgfqpoint{5.475083in}{4.058548in}}%
\pgfpathcurveto{\pgfqpoint{5.467269in}{4.050734in}}{\pgfqpoint{5.462879in}{4.040135in}}{\pgfqpoint{5.462879in}{4.029085in}}%
\pgfpathcurveto{\pgfqpoint{5.462879in}{4.018035in}}{\pgfqpoint{5.467269in}{4.007436in}}{\pgfqpoint{5.475083in}{3.999622in}}%
\pgfpathcurveto{\pgfqpoint{5.482896in}{3.991809in}}{\pgfqpoint{5.493495in}{3.987418in}}{\pgfqpoint{5.504545in}{3.987418in}}%
\pgfpathclose%
\pgfusepath{stroke,fill}%
\end{pgfscope}%
\begin{pgfscope}%
\pgfpathrectangle{\pgfqpoint{0.800000in}{0.528000in}}{\pgfqpoint{4.960000in}{3.696000in}}%
\pgfusepath{clip}%
\pgfsetbuttcap%
\pgfsetroundjoin%
\definecolor{currentfill}{rgb}{0.000000,0.000000,0.000000}%
\pgfsetfillcolor{currentfill}%
\pgfsetlinewidth{1.003750pt}%
\definecolor{currentstroke}{rgb}{0.000000,0.000000,0.000000}%
\pgfsetstrokecolor{currentstroke}%
\pgfsetdash{}{0pt}%
\pgfpathmoveto{\pgfqpoint{5.504545in}{3.063026in}}%
\pgfpathcurveto{\pgfqpoint{5.515596in}{3.063026in}}{\pgfqpoint{5.526195in}{3.067416in}}{\pgfqpoint{5.534008in}{3.075230in}}%
\pgfpathcurveto{\pgfqpoint{5.541822in}{3.083044in}}{\pgfqpoint{5.546212in}{3.093643in}}{\pgfqpoint{5.546212in}{3.104693in}}%
\pgfpathcurveto{\pgfqpoint{5.546212in}{3.115743in}}{\pgfqpoint{5.541822in}{3.126342in}}{\pgfqpoint{5.534008in}{3.134156in}}%
\pgfpathcurveto{\pgfqpoint{5.526195in}{3.141969in}}{\pgfqpoint{5.515596in}{3.146359in}}{\pgfqpoint{5.504545in}{3.146359in}}%
\pgfpathcurveto{\pgfqpoint{5.493495in}{3.146359in}}{\pgfqpoint{5.482896in}{3.141969in}}{\pgfqpoint{5.475083in}{3.134156in}}%
\pgfpathcurveto{\pgfqpoint{5.467269in}{3.126342in}}{\pgfqpoint{5.462879in}{3.115743in}}{\pgfqpoint{5.462879in}{3.104693in}}%
\pgfpathcurveto{\pgfqpoint{5.462879in}{3.093643in}}{\pgfqpoint{5.467269in}{3.083044in}}{\pgfqpoint{5.475083in}{3.075230in}}%
\pgfpathcurveto{\pgfqpoint{5.482896in}{3.067416in}}{\pgfqpoint{5.493495in}{3.063026in}}{\pgfqpoint{5.504545in}{3.063026in}}%
\pgfpathclose%
\pgfusepath{stroke,fill}%
\end{pgfscope}%
\begin{pgfscope}%
\pgfpathrectangle{\pgfqpoint{0.800000in}{0.528000in}}{\pgfqpoint{4.960000in}{3.696000in}}%
\pgfusepath{clip}%
\pgfsetbuttcap%
\pgfsetroundjoin%
\definecolor{currentfill}{rgb}{0.000000,0.000000,0.000000}%
\pgfsetfillcolor{currentfill}%
\pgfsetlinewidth{1.003750pt}%
\definecolor{currentstroke}{rgb}{0.000000,0.000000,0.000000}%
\pgfsetstrokecolor{currentstroke}%
\pgfsetdash{}{0pt}%
\pgfpathmoveto{\pgfqpoint{5.504545in}{2.848051in}}%
\pgfpathcurveto{\pgfqpoint{5.515596in}{2.848051in}}{\pgfqpoint{5.526195in}{2.852441in}}{\pgfqpoint{5.534008in}{2.860255in}}%
\pgfpathcurveto{\pgfqpoint{5.541822in}{2.868069in}}{\pgfqpoint{5.546212in}{2.878668in}}{\pgfqpoint{5.546212in}{2.889718in}}%
\pgfpathcurveto{\pgfqpoint{5.546212in}{2.900768in}}{\pgfqpoint{5.541822in}{2.911367in}}{\pgfqpoint{5.534008in}{2.919181in}}%
\pgfpathcurveto{\pgfqpoint{5.526195in}{2.926994in}}{\pgfqpoint{5.515596in}{2.931385in}}{\pgfqpoint{5.504545in}{2.931385in}}%
\pgfpathcurveto{\pgfqpoint{5.493495in}{2.931385in}}{\pgfqpoint{5.482896in}{2.926994in}}{\pgfqpoint{5.475083in}{2.919181in}}%
\pgfpathcurveto{\pgfqpoint{5.467269in}{2.911367in}}{\pgfqpoint{5.462879in}{2.900768in}}{\pgfqpoint{5.462879in}{2.889718in}}%
\pgfpathcurveto{\pgfqpoint{5.462879in}{2.878668in}}{\pgfqpoint{5.467269in}{2.868069in}}{\pgfqpoint{5.475083in}{2.860255in}}%
\pgfpathcurveto{\pgfqpoint{5.482896in}{2.852441in}}{\pgfqpoint{5.493495in}{2.848051in}}{\pgfqpoint{5.504545in}{2.848051in}}%
\pgfpathclose%
\pgfusepath{stroke,fill}%
\end{pgfscope}%
\begin{pgfscope}%
\pgfpathrectangle{\pgfqpoint{0.800000in}{0.528000in}}{\pgfqpoint{4.960000in}{3.696000in}}%
\pgfusepath{clip}%
\pgfsetbuttcap%
\pgfsetroundjoin%
\definecolor{currentfill}{rgb}{0.000000,0.000000,0.000000}%
\pgfsetfillcolor{currentfill}%
\pgfsetlinewidth{1.003750pt}%
\definecolor{currentstroke}{rgb}{0.000000,0.000000,0.000000}%
\pgfsetstrokecolor{currentstroke}%
\pgfsetdash{}{0pt}%
\pgfpathmoveto{\pgfqpoint{5.504545in}{2.762061in}}%
\pgfpathcurveto{\pgfqpoint{5.515596in}{2.762061in}}{\pgfqpoint{5.526195in}{2.766451in}}{\pgfqpoint{5.534008in}{2.774265in}}%
\pgfpathcurveto{\pgfqpoint{5.541822in}{2.782079in}}{\pgfqpoint{5.546212in}{2.792678in}}{\pgfqpoint{5.546212in}{2.803728in}}%
\pgfpathcurveto{\pgfqpoint{5.546212in}{2.814778in}}{\pgfqpoint{5.541822in}{2.825377in}}{\pgfqpoint{5.534008in}{2.833191in}}%
\pgfpathcurveto{\pgfqpoint{5.526195in}{2.841004in}}{\pgfqpoint{5.515596in}{2.845395in}}{\pgfqpoint{5.504545in}{2.845395in}}%
\pgfpathcurveto{\pgfqpoint{5.493495in}{2.845395in}}{\pgfqpoint{5.482896in}{2.841004in}}{\pgfqpoint{5.475083in}{2.833191in}}%
\pgfpathcurveto{\pgfqpoint{5.467269in}{2.825377in}}{\pgfqpoint{5.462879in}{2.814778in}}{\pgfqpoint{5.462879in}{2.803728in}}%
\pgfpathcurveto{\pgfqpoint{5.462879in}{2.792678in}}{\pgfqpoint{5.467269in}{2.782079in}}{\pgfqpoint{5.475083in}{2.774265in}}%
\pgfpathcurveto{\pgfqpoint{5.482896in}{2.766451in}}{\pgfqpoint{5.493495in}{2.762061in}}{\pgfqpoint{5.504545in}{2.762061in}}%
\pgfpathclose%
\pgfusepath{stroke,fill}%
\end{pgfscope}%
\begin{pgfscope}%
\pgfpathrectangle{\pgfqpoint{0.800000in}{0.528000in}}{\pgfqpoint{4.960000in}{3.696000in}}%
\pgfusepath{clip}%
\pgfsetbuttcap%
\pgfsetroundjoin%
\definecolor{currentfill}{rgb}{0.000000,0.000000,0.000000}%
\pgfsetfillcolor{currentfill}%
\pgfsetlinewidth{1.003750pt}%
\definecolor{currentstroke}{rgb}{0.000000,0.000000,0.000000}%
\pgfsetstrokecolor{currentstroke}%
\pgfsetdash{}{0pt}%
\pgfpathmoveto{\pgfqpoint{5.504545in}{2.719066in}}%
\pgfpathcurveto{\pgfqpoint{5.515596in}{2.719066in}}{\pgfqpoint{5.526195in}{2.723456in}}{\pgfqpoint{5.534008in}{2.731270in}}%
\pgfpathcurveto{\pgfqpoint{5.541822in}{2.739084in}}{\pgfqpoint{5.546212in}{2.749683in}}{\pgfqpoint{5.546212in}{2.760733in}}%
\pgfpathcurveto{\pgfqpoint{5.546212in}{2.771783in}}{\pgfqpoint{5.541822in}{2.782382in}}{\pgfqpoint{5.534008in}{2.790196in}}%
\pgfpathcurveto{\pgfqpoint{5.526195in}{2.798009in}}{\pgfqpoint{5.515596in}{2.802400in}}{\pgfqpoint{5.504545in}{2.802400in}}%
\pgfpathcurveto{\pgfqpoint{5.493495in}{2.802400in}}{\pgfqpoint{5.482896in}{2.798009in}}{\pgfqpoint{5.475083in}{2.790196in}}%
\pgfpathcurveto{\pgfqpoint{5.467269in}{2.782382in}}{\pgfqpoint{5.462879in}{2.771783in}}{\pgfqpoint{5.462879in}{2.760733in}}%
\pgfpathcurveto{\pgfqpoint{5.462879in}{2.749683in}}{\pgfqpoint{5.467269in}{2.739084in}}{\pgfqpoint{5.475083in}{2.731270in}}%
\pgfpathcurveto{\pgfqpoint{5.482896in}{2.723456in}}{\pgfqpoint{5.493495in}{2.719066in}}{\pgfqpoint{5.504545in}{2.719066in}}%
\pgfpathclose%
\pgfusepath{stroke,fill}%
\end{pgfscope}%
\begin{pgfscope}%
\pgfpathrectangle{\pgfqpoint{0.800000in}{0.528000in}}{\pgfqpoint{4.960000in}{3.696000in}}%
\pgfusepath{clip}%
\pgfsetbuttcap%
\pgfsetroundjoin%
\definecolor{currentfill}{rgb}{0.000000,0.000000,0.000000}%
\pgfsetfillcolor{currentfill}%
\pgfsetlinewidth{1.003750pt}%
\definecolor{currentstroke}{rgb}{0.000000,0.000000,0.000000}%
\pgfsetstrokecolor{currentstroke}%
\pgfsetdash{}{0pt}%
\pgfpathmoveto{\pgfqpoint{5.504545in}{3.106021in}}%
\pgfpathcurveto{\pgfqpoint{5.515596in}{3.106021in}}{\pgfqpoint{5.526195in}{3.110411in}}{\pgfqpoint{5.534008in}{3.118225in}}%
\pgfpathcurveto{\pgfqpoint{5.541822in}{3.126039in}}{\pgfqpoint{5.546212in}{3.136638in}}{\pgfqpoint{5.546212in}{3.147688in}}%
\pgfpathcurveto{\pgfqpoint{5.546212in}{3.158738in}}{\pgfqpoint{5.541822in}{3.169337in}}{\pgfqpoint{5.534008in}{3.177151in}}%
\pgfpathcurveto{\pgfqpoint{5.526195in}{3.184964in}}{\pgfqpoint{5.515596in}{3.189354in}}{\pgfqpoint{5.504545in}{3.189354in}}%
\pgfpathcurveto{\pgfqpoint{5.493495in}{3.189354in}}{\pgfqpoint{5.482896in}{3.184964in}}{\pgfqpoint{5.475083in}{3.177151in}}%
\pgfpathcurveto{\pgfqpoint{5.467269in}{3.169337in}}{\pgfqpoint{5.462879in}{3.158738in}}{\pgfqpoint{5.462879in}{3.147688in}}%
\pgfpathcurveto{\pgfqpoint{5.462879in}{3.136638in}}{\pgfqpoint{5.467269in}{3.126039in}}{\pgfqpoint{5.475083in}{3.118225in}}%
\pgfpathcurveto{\pgfqpoint{5.482896in}{3.110411in}}{\pgfqpoint{5.493495in}{3.106021in}}{\pgfqpoint{5.504545in}{3.106021in}}%
\pgfpathclose%
\pgfusepath{stroke,fill}%
\end{pgfscope}%
\begin{pgfscope}%
\pgfpathrectangle{\pgfqpoint{0.800000in}{0.528000in}}{\pgfqpoint{4.960000in}{3.696000in}}%
\pgfusepath{clip}%
\pgfsetbuttcap%
\pgfsetroundjoin%
\definecolor{currentfill}{rgb}{0.000000,0.000000,0.000000}%
\pgfsetfillcolor{currentfill}%
\pgfsetlinewidth{1.003750pt}%
\definecolor{currentstroke}{rgb}{0.000000,0.000000,0.000000}%
\pgfsetstrokecolor{currentstroke}%
\pgfsetdash{}{0pt}%
\pgfpathmoveto{\pgfqpoint{5.504545in}{2.740564in}}%
\pgfpathcurveto{\pgfqpoint{5.515596in}{2.740564in}}{\pgfqpoint{5.526195in}{2.744954in}}{\pgfqpoint{5.534008in}{2.752768in}}%
\pgfpathcurveto{\pgfqpoint{5.541822in}{2.760581in}}{\pgfqpoint{5.546212in}{2.771180in}}{\pgfqpoint{5.546212in}{2.782230in}}%
\pgfpathcurveto{\pgfqpoint{5.546212in}{2.793281in}}{\pgfqpoint{5.541822in}{2.803880in}}{\pgfqpoint{5.534008in}{2.811693in}}%
\pgfpathcurveto{\pgfqpoint{5.526195in}{2.819507in}}{\pgfqpoint{5.515596in}{2.823897in}}{\pgfqpoint{5.504545in}{2.823897in}}%
\pgfpathcurveto{\pgfqpoint{5.493495in}{2.823897in}}{\pgfqpoint{5.482896in}{2.819507in}}{\pgfqpoint{5.475083in}{2.811693in}}%
\pgfpathcurveto{\pgfqpoint{5.467269in}{2.803880in}}{\pgfqpoint{5.462879in}{2.793281in}}{\pgfqpoint{5.462879in}{2.782230in}}%
\pgfpathcurveto{\pgfqpoint{5.462879in}{2.771180in}}{\pgfqpoint{5.467269in}{2.760581in}}{\pgfqpoint{5.475083in}{2.752768in}}%
\pgfpathcurveto{\pgfqpoint{5.482896in}{2.744954in}}{\pgfqpoint{5.493495in}{2.740564in}}{\pgfqpoint{5.504545in}{2.740564in}}%
\pgfpathclose%
\pgfusepath{stroke,fill}%
\end{pgfscope}%
\begin{pgfscope}%
\pgfpathrectangle{\pgfqpoint{0.800000in}{0.528000in}}{\pgfqpoint{4.960000in}{3.696000in}}%
\pgfusepath{clip}%
\pgfsetbuttcap%
\pgfsetroundjoin%
\definecolor{currentfill}{rgb}{0.000000,0.000000,0.000000}%
\pgfsetfillcolor{currentfill}%
\pgfsetlinewidth{1.003750pt}%
\definecolor{currentstroke}{rgb}{0.000000,0.000000,0.000000}%
\pgfsetstrokecolor{currentstroke}%
\pgfsetdash{}{0pt}%
\pgfpathmoveto{\pgfqpoint{5.504545in}{3.535971in}}%
\pgfpathcurveto{\pgfqpoint{5.515596in}{3.535971in}}{\pgfqpoint{5.526195in}{3.540361in}}{\pgfqpoint{5.534008in}{3.548175in}}%
\pgfpathcurveto{\pgfqpoint{5.541822in}{3.555988in}}{\pgfqpoint{5.546212in}{3.566587in}}{\pgfqpoint{5.546212in}{3.577638in}}%
\pgfpathcurveto{\pgfqpoint{5.546212in}{3.588688in}}{\pgfqpoint{5.541822in}{3.599287in}}{\pgfqpoint{5.534008in}{3.607100in}}%
\pgfpathcurveto{\pgfqpoint{5.526195in}{3.614914in}}{\pgfqpoint{5.515596in}{3.619304in}}{\pgfqpoint{5.504545in}{3.619304in}}%
\pgfpathcurveto{\pgfqpoint{5.493495in}{3.619304in}}{\pgfqpoint{5.482896in}{3.614914in}}{\pgfqpoint{5.475083in}{3.607100in}}%
\pgfpathcurveto{\pgfqpoint{5.467269in}{3.599287in}}{\pgfqpoint{5.462879in}{3.588688in}}{\pgfqpoint{5.462879in}{3.577638in}}%
\pgfpathcurveto{\pgfqpoint{5.462879in}{3.566587in}}{\pgfqpoint{5.467269in}{3.555988in}}{\pgfqpoint{5.475083in}{3.548175in}}%
\pgfpathcurveto{\pgfqpoint{5.482896in}{3.540361in}}{\pgfqpoint{5.493495in}{3.535971in}}{\pgfqpoint{5.504545in}{3.535971in}}%
\pgfpathclose%
\pgfusepath{stroke,fill}%
\end{pgfscope}%
\begin{pgfscope}%
\pgfpathrectangle{\pgfqpoint{0.800000in}{0.528000in}}{\pgfqpoint{4.960000in}{3.696000in}}%
\pgfusepath{clip}%
\pgfsetbuttcap%
\pgfsetroundjoin%
\definecolor{currentfill}{rgb}{0.000000,0.000000,0.000000}%
\pgfsetfillcolor{currentfill}%
\pgfsetlinewidth{1.003750pt}%
\definecolor{currentstroke}{rgb}{0.000000,0.000000,0.000000}%
\pgfsetstrokecolor{currentstroke}%
\pgfsetdash{}{0pt}%
\pgfpathmoveto{\pgfqpoint{5.504545in}{2.934041in}}%
\pgfpathcurveto{\pgfqpoint{5.515596in}{2.934041in}}{\pgfqpoint{5.526195in}{2.938431in}}{\pgfqpoint{5.534008in}{2.946245in}}%
\pgfpathcurveto{\pgfqpoint{5.541822in}{2.954059in}}{\pgfqpoint{5.546212in}{2.964658in}}{\pgfqpoint{5.546212in}{2.975708in}}%
\pgfpathcurveto{\pgfqpoint{5.546212in}{2.986758in}}{\pgfqpoint{5.541822in}{2.997357in}}{\pgfqpoint{5.534008in}{3.005171in}}%
\pgfpathcurveto{\pgfqpoint{5.526195in}{3.012984in}}{\pgfqpoint{5.515596in}{3.017374in}}{\pgfqpoint{5.504545in}{3.017374in}}%
\pgfpathcurveto{\pgfqpoint{5.493495in}{3.017374in}}{\pgfqpoint{5.482896in}{3.012984in}}{\pgfqpoint{5.475083in}{3.005171in}}%
\pgfpathcurveto{\pgfqpoint{5.467269in}{2.997357in}}{\pgfqpoint{5.462879in}{2.986758in}}{\pgfqpoint{5.462879in}{2.975708in}}%
\pgfpathcurveto{\pgfqpoint{5.462879in}{2.964658in}}{\pgfqpoint{5.467269in}{2.954059in}}{\pgfqpoint{5.475083in}{2.946245in}}%
\pgfpathcurveto{\pgfqpoint{5.482896in}{2.938431in}}{\pgfqpoint{5.493495in}{2.934041in}}{\pgfqpoint{5.504545in}{2.934041in}}%
\pgfpathclose%
\pgfusepath{stroke,fill}%
\end{pgfscope}%
\begin{pgfscope}%
\pgfpathrectangle{\pgfqpoint{0.800000in}{0.528000in}}{\pgfqpoint{4.960000in}{3.696000in}}%
\pgfusepath{clip}%
\pgfsetbuttcap%
\pgfsetroundjoin%
\definecolor{currentfill}{rgb}{0.000000,0.000000,0.000000}%
\pgfsetfillcolor{currentfill}%
\pgfsetlinewidth{1.003750pt}%
\definecolor{currentstroke}{rgb}{0.000000,0.000000,0.000000}%
\pgfsetstrokecolor{currentstroke}%
\pgfsetdash{}{0pt}%
\pgfpathmoveto{\pgfqpoint{5.504545in}{2.676071in}}%
\pgfpathcurveto{\pgfqpoint{5.515596in}{2.676071in}}{\pgfqpoint{5.526195in}{2.680462in}}{\pgfqpoint{5.534008in}{2.688275in}}%
\pgfpathcurveto{\pgfqpoint{5.541822in}{2.696089in}}{\pgfqpoint{5.546212in}{2.706688in}}{\pgfqpoint{5.546212in}{2.717738in}}%
\pgfpathcurveto{\pgfqpoint{5.546212in}{2.728788in}}{\pgfqpoint{5.541822in}{2.739387in}}{\pgfqpoint{5.534008in}{2.747201in}}%
\pgfpathcurveto{\pgfqpoint{5.526195in}{2.755014in}}{\pgfqpoint{5.515596in}{2.759405in}}{\pgfqpoint{5.504545in}{2.759405in}}%
\pgfpathcurveto{\pgfqpoint{5.493495in}{2.759405in}}{\pgfqpoint{5.482896in}{2.755014in}}{\pgfqpoint{5.475083in}{2.747201in}}%
\pgfpathcurveto{\pgfqpoint{5.467269in}{2.739387in}}{\pgfqpoint{5.462879in}{2.728788in}}{\pgfqpoint{5.462879in}{2.717738in}}%
\pgfpathcurveto{\pgfqpoint{5.462879in}{2.706688in}}{\pgfqpoint{5.467269in}{2.696089in}}{\pgfqpoint{5.475083in}{2.688275in}}%
\pgfpathcurveto{\pgfqpoint{5.482896in}{2.680462in}}{\pgfqpoint{5.493495in}{2.676071in}}{\pgfqpoint{5.504545in}{2.676071in}}%
\pgfpathclose%
\pgfusepath{stroke,fill}%
\end{pgfscope}%
\begin{pgfscope}%
\pgfpathrectangle{\pgfqpoint{0.800000in}{0.528000in}}{\pgfqpoint{4.960000in}{3.696000in}}%
\pgfusepath{clip}%
\pgfsetbuttcap%
\pgfsetroundjoin%
\definecolor{currentfill}{rgb}{0.000000,0.000000,0.000000}%
\pgfsetfillcolor{currentfill}%
\pgfsetlinewidth{1.003750pt}%
\definecolor{currentstroke}{rgb}{0.000000,0.000000,0.000000}%
\pgfsetstrokecolor{currentstroke}%
\pgfsetdash{}{0pt}%
\pgfpathmoveto{\pgfqpoint{5.504545in}{2.740564in}}%
\pgfpathcurveto{\pgfqpoint{5.515596in}{2.740564in}}{\pgfqpoint{5.526195in}{2.744954in}}{\pgfqpoint{5.534008in}{2.752768in}}%
\pgfpathcurveto{\pgfqpoint{5.541822in}{2.760581in}}{\pgfqpoint{5.546212in}{2.771180in}}{\pgfqpoint{5.546212in}{2.782230in}}%
\pgfpathcurveto{\pgfqpoint{5.546212in}{2.793281in}}{\pgfqpoint{5.541822in}{2.803880in}}{\pgfqpoint{5.534008in}{2.811693in}}%
\pgfpathcurveto{\pgfqpoint{5.526195in}{2.819507in}}{\pgfqpoint{5.515596in}{2.823897in}}{\pgfqpoint{5.504545in}{2.823897in}}%
\pgfpathcurveto{\pgfqpoint{5.493495in}{2.823897in}}{\pgfqpoint{5.482896in}{2.819507in}}{\pgfqpoint{5.475083in}{2.811693in}}%
\pgfpathcurveto{\pgfqpoint{5.467269in}{2.803880in}}{\pgfqpoint{5.462879in}{2.793281in}}{\pgfqpoint{5.462879in}{2.782230in}}%
\pgfpathcurveto{\pgfqpoint{5.462879in}{2.771180in}}{\pgfqpoint{5.467269in}{2.760581in}}{\pgfqpoint{5.475083in}{2.752768in}}%
\pgfpathcurveto{\pgfqpoint{5.482896in}{2.744954in}}{\pgfqpoint{5.493495in}{2.740564in}}{\pgfqpoint{5.504545in}{2.740564in}}%
\pgfpathclose%
\pgfusepath{stroke,fill}%
\end{pgfscope}%
\begin{pgfscope}%
\pgfpathrectangle{\pgfqpoint{0.800000in}{0.528000in}}{\pgfqpoint{4.960000in}{3.696000in}}%
\pgfusepath{clip}%
\pgfsetbuttcap%
\pgfsetroundjoin%
\definecolor{currentfill}{rgb}{0.000000,0.000000,0.000000}%
\pgfsetfillcolor{currentfill}%
\pgfsetlinewidth{1.003750pt}%
\definecolor{currentstroke}{rgb}{0.000000,0.000000,0.000000}%
\pgfsetstrokecolor{currentstroke}%
\pgfsetdash{}{0pt}%
\pgfpathmoveto{\pgfqpoint{5.504545in}{2.891046in}}%
\pgfpathcurveto{\pgfqpoint{5.515596in}{2.891046in}}{\pgfqpoint{5.526195in}{2.895436in}}{\pgfqpoint{5.534008in}{2.903250in}}%
\pgfpathcurveto{\pgfqpoint{5.541822in}{2.911064in}}{\pgfqpoint{5.546212in}{2.921663in}}{\pgfqpoint{5.546212in}{2.932713in}}%
\pgfpathcurveto{\pgfqpoint{5.546212in}{2.943763in}}{\pgfqpoint{5.541822in}{2.954362in}}{\pgfqpoint{5.534008in}{2.962176in}}%
\pgfpathcurveto{\pgfqpoint{5.526195in}{2.969989in}}{\pgfqpoint{5.515596in}{2.974379in}}{\pgfqpoint{5.504545in}{2.974379in}}%
\pgfpathcurveto{\pgfqpoint{5.493495in}{2.974379in}}{\pgfqpoint{5.482896in}{2.969989in}}{\pgfqpoint{5.475083in}{2.962176in}}%
\pgfpathcurveto{\pgfqpoint{5.467269in}{2.954362in}}{\pgfqpoint{5.462879in}{2.943763in}}{\pgfqpoint{5.462879in}{2.932713in}}%
\pgfpathcurveto{\pgfqpoint{5.462879in}{2.921663in}}{\pgfqpoint{5.467269in}{2.911064in}}{\pgfqpoint{5.475083in}{2.903250in}}%
\pgfpathcurveto{\pgfqpoint{5.482896in}{2.895436in}}{\pgfqpoint{5.493495in}{2.891046in}}{\pgfqpoint{5.504545in}{2.891046in}}%
\pgfpathclose%
\pgfusepath{stroke,fill}%
\end{pgfscope}%
\begin{pgfscope}%
\pgfpathrectangle{\pgfqpoint{0.800000in}{0.528000in}}{\pgfqpoint{4.960000in}{3.696000in}}%
\pgfusepath{clip}%
\pgfsetbuttcap%
\pgfsetroundjoin%
\definecolor{currentfill}{rgb}{0.000000,0.000000,0.000000}%
\pgfsetfillcolor{currentfill}%
\pgfsetlinewidth{1.003750pt}%
\definecolor{currentstroke}{rgb}{0.000000,0.000000,0.000000}%
\pgfsetstrokecolor{currentstroke}%
\pgfsetdash{}{0pt}%
\pgfpathmoveto{\pgfqpoint{5.504545in}{3.299499in}}%
\pgfpathcurveto{\pgfqpoint{5.515596in}{3.299499in}}{\pgfqpoint{5.526195in}{3.303889in}}{\pgfqpoint{5.534008in}{3.311702in}}%
\pgfpathcurveto{\pgfqpoint{5.541822in}{3.319516in}}{\pgfqpoint{5.546212in}{3.330115in}}{\pgfqpoint{5.546212in}{3.341165in}}%
\pgfpathcurveto{\pgfqpoint{5.546212in}{3.352215in}}{\pgfqpoint{5.541822in}{3.362814in}}{\pgfqpoint{5.534008in}{3.370628in}}%
\pgfpathcurveto{\pgfqpoint{5.526195in}{3.378442in}}{\pgfqpoint{5.515596in}{3.382832in}}{\pgfqpoint{5.504545in}{3.382832in}}%
\pgfpathcurveto{\pgfqpoint{5.493495in}{3.382832in}}{\pgfqpoint{5.482896in}{3.378442in}}{\pgfqpoint{5.475083in}{3.370628in}}%
\pgfpathcurveto{\pgfqpoint{5.467269in}{3.362814in}}{\pgfqpoint{5.462879in}{3.352215in}}{\pgfqpoint{5.462879in}{3.341165in}}%
\pgfpathcurveto{\pgfqpoint{5.462879in}{3.330115in}}{\pgfqpoint{5.467269in}{3.319516in}}{\pgfqpoint{5.475083in}{3.311702in}}%
\pgfpathcurveto{\pgfqpoint{5.482896in}{3.303889in}}{\pgfqpoint{5.493495in}{3.299499in}}{\pgfqpoint{5.504545in}{3.299499in}}%
\pgfpathclose%
\pgfusepath{stroke,fill}%
\end{pgfscope}%
\begin{pgfscope}%
\pgfpathrectangle{\pgfqpoint{0.800000in}{0.528000in}}{\pgfqpoint{4.960000in}{3.696000in}}%
\pgfusepath{clip}%
\pgfsetbuttcap%
\pgfsetroundjoin%
\definecolor{currentfill}{rgb}{0.000000,0.000000,0.000000}%
\pgfsetfillcolor{currentfill}%
\pgfsetlinewidth{1.003750pt}%
\definecolor{currentstroke}{rgb}{0.000000,0.000000,0.000000}%
\pgfsetstrokecolor{currentstroke}%
\pgfsetdash{}{0pt}%
\pgfpathmoveto{\pgfqpoint{5.504545in}{2.826554in}}%
\pgfpathcurveto{\pgfqpoint{5.515596in}{2.826554in}}{\pgfqpoint{5.526195in}{2.830944in}}{\pgfqpoint{5.534008in}{2.838758in}}%
\pgfpathcurveto{\pgfqpoint{5.541822in}{2.846571in}}{\pgfqpoint{5.546212in}{2.857170in}}{\pgfqpoint{5.546212in}{2.868220in}}%
\pgfpathcurveto{\pgfqpoint{5.546212in}{2.879270in}}{\pgfqpoint{5.541822in}{2.889870in}}{\pgfqpoint{5.534008in}{2.897683in}}%
\pgfpathcurveto{\pgfqpoint{5.526195in}{2.905497in}}{\pgfqpoint{5.515596in}{2.909887in}}{\pgfqpoint{5.504545in}{2.909887in}}%
\pgfpathcurveto{\pgfqpoint{5.493495in}{2.909887in}}{\pgfqpoint{5.482896in}{2.905497in}}{\pgfqpoint{5.475083in}{2.897683in}}%
\pgfpathcurveto{\pgfqpoint{5.467269in}{2.889870in}}{\pgfqpoint{5.462879in}{2.879270in}}{\pgfqpoint{5.462879in}{2.868220in}}%
\pgfpathcurveto{\pgfqpoint{5.462879in}{2.857170in}}{\pgfqpoint{5.467269in}{2.846571in}}{\pgfqpoint{5.475083in}{2.838758in}}%
\pgfpathcurveto{\pgfqpoint{5.482896in}{2.830944in}}{\pgfqpoint{5.493495in}{2.826554in}}{\pgfqpoint{5.504545in}{2.826554in}}%
\pgfpathclose%
\pgfusepath{stroke,fill}%
\end{pgfscope}%
\begin{pgfscope}%
\pgfpathrectangle{\pgfqpoint{0.800000in}{0.528000in}}{\pgfqpoint{4.960000in}{3.696000in}}%
\pgfusepath{clip}%
\pgfsetbuttcap%
\pgfsetroundjoin%
\definecolor{currentfill}{rgb}{0.000000,0.000000,0.000000}%
\pgfsetfillcolor{currentfill}%
\pgfsetlinewidth{1.003750pt}%
\definecolor{currentstroke}{rgb}{0.000000,0.000000,0.000000}%
\pgfsetstrokecolor{currentstroke}%
\pgfsetdash{}{0pt}%
\pgfpathmoveto{\pgfqpoint{5.504545in}{2.740564in}}%
\pgfpathcurveto{\pgfqpoint{5.515596in}{2.740564in}}{\pgfqpoint{5.526195in}{2.744954in}}{\pgfqpoint{5.534008in}{2.752768in}}%
\pgfpathcurveto{\pgfqpoint{5.541822in}{2.760581in}}{\pgfqpoint{5.546212in}{2.771180in}}{\pgfqpoint{5.546212in}{2.782230in}}%
\pgfpathcurveto{\pgfqpoint{5.546212in}{2.793281in}}{\pgfqpoint{5.541822in}{2.803880in}}{\pgfqpoint{5.534008in}{2.811693in}}%
\pgfpathcurveto{\pgfqpoint{5.526195in}{2.819507in}}{\pgfqpoint{5.515596in}{2.823897in}}{\pgfqpoint{5.504545in}{2.823897in}}%
\pgfpathcurveto{\pgfqpoint{5.493495in}{2.823897in}}{\pgfqpoint{5.482896in}{2.819507in}}{\pgfqpoint{5.475083in}{2.811693in}}%
\pgfpathcurveto{\pgfqpoint{5.467269in}{2.803880in}}{\pgfqpoint{5.462879in}{2.793281in}}{\pgfqpoint{5.462879in}{2.782230in}}%
\pgfpathcurveto{\pgfqpoint{5.462879in}{2.771180in}}{\pgfqpoint{5.467269in}{2.760581in}}{\pgfqpoint{5.475083in}{2.752768in}}%
\pgfpathcurveto{\pgfqpoint{5.482896in}{2.744954in}}{\pgfqpoint{5.493495in}{2.740564in}}{\pgfqpoint{5.504545in}{2.740564in}}%
\pgfpathclose%
\pgfusepath{stroke,fill}%
\end{pgfscope}%
\begin{pgfscope}%
\pgfpathrectangle{\pgfqpoint{0.800000in}{0.528000in}}{\pgfqpoint{4.960000in}{3.696000in}}%
\pgfusepath{clip}%
\pgfsetbuttcap%
\pgfsetroundjoin%
\definecolor{currentfill}{rgb}{0.000000,0.000000,0.000000}%
\pgfsetfillcolor{currentfill}%
\pgfsetlinewidth{1.003750pt}%
\definecolor{currentstroke}{rgb}{0.000000,0.000000,0.000000}%
\pgfsetstrokecolor{currentstroke}%
\pgfsetdash{}{0pt}%
\pgfpathmoveto{\pgfqpoint{5.504545in}{2.740564in}}%
\pgfpathcurveto{\pgfqpoint{5.515596in}{2.740564in}}{\pgfqpoint{5.526195in}{2.744954in}}{\pgfqpoint{5.534008in}{2.752768in}}%
\pgfpathcurveto{\pgfqpoint{5.541822in}{2.760581in}}{\pgfqpoint{5.546212in}{2.771180in}}{\pgfqpoint{5.546212in}{2.782230in}}%
\pgfpathcurveto{\pgfqpoint{5.546212in}{2.793281in}}{\pgfqpoint{5.541822in}{2.803880in}}{\pgfqpoint{5.534008in}{2.811693in}}%
\pgfpathcurveto{\pgfqpoint{5.526195in}{2.819507in}}{\pgfqpoint{5.515596in}{2.823897in}}{\pgfqpoint{5.504545in}{2.823897in}}%
\pgfpathcurveto{\pgfqpoint{5.493495in}{2.823897in}}{\pgfqpoint{5.482896in}{2.819507in}}{\pgfqpoint{5.475083in}{2.811693in}}%
\pgfpathcurveto{\pgfqpoint{5.467269in}{2.803880in}}{\pgfqpoint{5.462879in}{2.793281in}}{\pgfqpoint{5.462879in}{2.782230in}}%
\pgfpathcurveto{\pgfqpoint{5.462879in}{2.771180in}}{\pgfqpoint{5.467269in}{2.760581in}}{\pgfqpoint{5.475083in}{2.752768in}}%
\pgfpathcurveto{\pgfqpoint{5.482896in}{2.744954in}}{\pgfqpoint{5.493495in}{2.740564in}}{\pgfqpoint{5.504545in}{2.740564in}}%
\pgfpathclose%
\pgfusepath{stroke,fill}%
\end{pgfscope}%
\begin{pgfscope}%
\pgfpathrectangle{\pgfqpoint{0.800000in}{0.528000in}}{\pgfqpoint{4.960000in}{3.696000in}}%
\pgfusepath{clip}%
\pgfsetbuttcap%
\pgfsetroundjoin%
\definecolor{currentfill}{rgb}{0.000000,0.000000,0.000000}%
\pgfsetfillcolor{currentfill}%
\pgfsetlinewidth{1.003750pt}%
\definecolor{currentstroke}{rgb}{0.000000,0.000000,0.000000}%
\pgfsetstrokecolor{currentstroke}%
\pgfsetdash{}{0pt}%
\pgfpathmoveto{\pgfqpoint{5.504545in}{3.084524in}}%
\pgfpathcurveto{\pgfqpoint{5.515596in}{3.084524in}}{\pgfqpoint{5.526195in}{3.088914in}}{\pgfqpoint{5.534008in}{3.096727in}}%
\pgfpathcurveto{\pgfqpoint{5.541822in}{3.104541in}}{\pgfqpoint{5.546212in}{3.115140in}}{\pgfqpoint{5.546212in}{3.126190in}}%
\pgfpathcurveto{\pgfqpoint{5.546212in}{3.137240in}}{\pgfqpoint{5.541822in}{3.147839in}}{\pgfqpoint{5.534008in}{3.155653in}}%
\pgfpathcurveto{\pgfqpoint{5.526195in}{3.163467in}}{\pgfqpoint{5.515596in}{3.167857in}}{\pgfqpoint{5.504545in}{3.167857in}}%
\pgfpathcurveto{\pgfqpoint{5.493495in}{3.167857in}}{\pgfqpoint{5.482896in}{3.163467in}}{\pgfqpoint{5.475083in}{3.155653in}}%
\pgfpathcurveto{\pgfqpoint{5.467269in}{3.147839in}}{\pgfqpoint{5.462879in}{3.137240in}}{\pgfqpoint{5.462879in}{3.126190in}}%
\pgfpathcurveto{\pgfqpoint{5.462879in}{3.115140in}}{\pgfqpoint{5.467269in}{3.104541in}}{\pgfqpoint{5.475083in}{3.096727in}}%
\pgfpathcurveto{\pgfqpoint{5.482896in}{3.088914in}}{\pgfqpoint{5.493495in}{3.084524in}}{\pgfqpoint{5.504545in}{3.084524in}}%
\pgfpathclose%
\pgfusepath{stroke,fill}%
\end{pgfscope}%
\begin{pgfscope}%
\pgfpathrectangle{\pgfqpoint{0.800000in}{0.528000in}}{\pgfqpoint{4.960000in}{3.696000in}}%
\pgfusepath{clip}%
\pgfsetbuttcap%
\pgfsetroundjoin%
\definecolor{currentfill}{rgb}{0.000000,0.000000,0.000000}%
\pgfsetfillcolor{currentfill}%
\pgfsetlinewidth{1.003750pt}%
\definecolor{currentstroke}{rgb}{0.000000,0.000000,0.000000}%
\pgfsetstrokecolor{currentstroke}%
\pgfsetdash{}{0pt}%
\pgfpathmoveto{\pgfqpoint{5.504545in}{2.740564in}}%
\pgfpathcurveto{\pgfqpoint{5.515596in}{2.740564in}}{\pgfqpoint{5.526195in}{2.744954in}}{\pgfqpoint{5.534008in}{2.752768in}}%
\pgfpathcurveto{\pgfqpoint{5.541822in}{2.760581in}}{\pgfqpoint{5.546212in}{2.771180in}}{\pgfqpoint{5.546212in}{2.782230in}}%
\pgfpathcurveto{\pgfqpoint{5.546212in}{2.793281in}}{\pgfqpoint{5.541822in}{2.803880in}}{\pgfqpoint{5.534008in}{2.811693in}}%
\pgfpathcurveto{\pgfqpoint{5.526195in}{2.819507in}}{\pgfqpoint{5.515596in}{2.823897in}}{\pgfqpoint{5.504545in}{2.823897in}}%
\pgfpathcurveto{\pgfqpoint{5.493495in}{2.823897in}}{\pgfqpoint{5.482896in}{2.819507in}}{\pgfqpoint{5.475083in}{2.811693in}}%
\pgfpathcurveto{\pgfqpoint{5.467269in}{2.803880in}}{\pgfqpoint{5.462879in}{2.793281in}}{\pgfqpoint{5.462879in}{2.782230in}}%
\pgfpathcurveto{\pgfqpoint{5.462879in}{2.771180in}}{\pgfqpoint{5.467269in}{2.760581in}}{\pgfqpoint{5.475083in}{2.752768in}}%
\pgfpathcurveto{\pgfqpoint{5.482896in}{2.744954in}}{\pgfqpoint{5.493495in}{2.740564in}}{\pgfqpoint{5.504545in}{2.740564in}}%
\pgfpathclose%
\pgfusepath{stroke,fill}%
\end{pgfscope}%
\begin{pgfscope}%
\pgfpathrectangle{\pgfqpoint{0.800000in}{0.528000in}}{\pgfqpoint{4.960000in}{3.696000in}}%
\pgfusepath{clip}%
\pgfsetbuttcap%
\pgfsetroundjoin%
\definecolor{currentfill}{rgb}{0.000000,0.000000,0.000000}%
\pgfsetfillcolor{currentfill}%
\pgfsetlinewidth{1.003750pt}%
\definecolor{currentstroke}{rgb}{0.000000,0.000000,0.000000}%
\pgfsetstrokecolor{currentstroke}%
\pgfsetdash{}{0pt}%
\pgfpathmoveto{\pgfqpoint{5.504545in}{2.740564in}}%
\pgfpathcurveto{\pgfqpoint{5.515596in}{2.740564in}}{\pgfqpoint{5.526195in}{2.744954in}}{\pgfqpoint{5.534008in}{2.752768in}}%
\pgfpathcurveto{\pgfqpoint{5.541822in}{2.760581in}}{\pgfqpoint{5.546212in}{2.771180in}}{\pgfqpoint{5.546212in}{2.782230in}}%
\pgfpathcurveto{\pgfqpoint{5.546212in}{2.793281in}}{\pgfqpoint{5.541822in}{2.803880in}}{\pgfqpoint{5.534008in}{2.811693in}}%
\pgfpathcurveto{\pgfqpoint{5.526195in}{2.819507in}}{\pgfqpoint{5.515596in}{2.823897in}}{\pgfqpoint{5.504545in}{2.823897in}}%
\pgfpathcurveto{\pgfqpoint{5.493495in}{2.823897in}}{\pgfqpoint{5.482896in}{2.819507in}}{\pgfqpoint{5.475083in}{2.811693in}}%
\pgfpathcurveto{\pgfqpoint{5.467269in}{2.803880in}}{\pgfqpoint{5.462879in}{2.793281in}}{\pgfqpoint{5.462879in}{2.782230in}}%
\pgfpathcurveto{\pgfqpoint{5.462879in}{2.771180in}}{\pgfqpoint{5.467269in}{2.760581in}}{\pgfqpoint{5.475083in}{2.752768in}}%
\pgfpathcurveto{\pgfqpoint{5.482896in}{2.744954in}}{\pgfqpoint{5.493495in}{2.740564in}}{\pgfqpoint{5.504545in}{2.740564in}}%
\pgfpathclose%
\pgfusepath{stroke,fill}%
\end{pgfscope}%
\begin{pgfscope}%
\pgfpathrectangle{\pgfqpoint{0.800000in}{0.528000in}}{\pgfqpoint{4.960000in}{3.696000in}}%
\pgfusepath{clip}%
\pgfsetbuttcap%
\pgfsetroundjoin%
\definecolor{currentfill}{rgb}{0.000000,0.000000,0.000000}%
\pgfsetfillcolor{currentfill}%
\pgfsetlinewidth{1.003750pt}%
\definecolor{currentstroke}{rgb}{0.000000,0.000000,0.000000}%
\pgfsetstrokecolor{currentstroke}%
\pgfsetdash{}{0pt}%
\pgfpathmoveto{\pgfqpoint{5.504545in}{2.783559in}}%
\pgfpathcurveto{\pgfqpoint{5.515596in}{2.783559in}}{\pgfqpoint{5.526195in}{2.787949in}}{\pgfqpoint{5.534008in}{2.795763in}}%
\pgfpathcurveto{\pgfqpoint{5.541822in}{2.803576in}}{\pgfqpoint{5.546212in}{2.814175in}}{\pgfqpoint{5.546212in}{2.825225in}}%
\pgfpathcurveto{\pgfqpoint{5.546212in}{2.836275in}}{\pgfqpoint{5.541822in}{2.846875in}}{\pgfqpoint{5.534008in}{2.854688in}}%
\pgfpathcurveto{\pgfqpoint{5.526195in}{2.862502in}}{\pgfqpoint{5.515596in}{2.866892in}}{\pgfqpoint{5.504545in}{2.866892in}}%
\pgfpathcurveto{\pgfqpoint{5.493495in}{2.866892in}}{\pgfqpoint{5.482896in}{2.862502in}}{\pgfqpoint{5.475083in}{2.854688in}}%
\pgfpathcurveto{\pgfqpoint{5.467269in}{2.846875in}}{\pgfqpoint{5.462879in}{2.836275in}}{\pgfqpoint{5.462879in}{2.825225in}}%
\pgfpathcurveto{\pgfqpoint{5.462879in}{2.814175in}}{\pgfqpoint{5.467269in}{2.803576in}}{\pgfqpoint{5.475083in}{2.795763in}}%
\pgfpathcurveto{\pgfqpoint{5.482896in}{2.787949in}}{\pgfqpoint{5.493495in}{2.783559in}}{\pgfqpoint{5.504545in}{2.783559in}}%
\pgfpathclose%
\pgfusepath{stroke,fill}%
\end{pgfscope}%
\begin{pgfscope}%
\pgfpathrectangle{\pgfqpoint{0.800000in}{0.528000in}}{\pgfqpoint{4.960000in}{3.696000in}}%
\pgfusepath{clip}%
\pgfsetbuttcap%
\pgfsetroundjoin%
\definecolor{currentfill}{rgb}{0.000000,0.000000,0.000000}%
\pgfsetfillcolor{currentfill}%
\pgfsetlinewidth{1.003750pt}%
\definecolor{currentstroke}{rgb}{0.000000,0.000000,0.000000}%
\pgfsetstrokecolor{currentstroke}%
\pgfsetdash{}{0pt}%
\pgfpathmoveto{\pgfqpoint{5.504545in}{2.891046in}}%
\pgfpathcurveto{\pgfqpoint{5.515596in}{2.891046in}}{\pgfqpoint{5.526195in}{2.895436in}}{\pgfqpoint{5.534008in}{2.903250in}}%
\pgfpathcurveto{\pgfqpoint{5.541822in}{2.911064in}}{\pgfqpoint{5.546212in}{2.921663in}}{\pgfqpoint{5.546212in}{2.932713in}}%
\pgfpathcurveto{\pgfqpoint{5.546212in}{2.943763in}}{\pgfqpoint{5.541822in}{2.954362in}}{\pgfqpoint{5.534008in}{2.962176in}}%
\pgfpathcurveto{\pgfqpoint{5.526195in}{2.969989in}}{\pgfqpoint{5.515596in}{2.974379in}}{\pgfqpoint{5.504545in}{2.974379in}}%
\pgfpathcurveto{\pgfqpoint{5.493495in}{2.974379in}}{\pgfqpoint{5.482896in}{2.969989in}}{\pgfqpoint{5.475083in}{2.962176in}}%
\pgfpathcurveto{\pgfqpoint{5.467269in}{2.954362in}}{\pgfqpoint{5.462879in}{2.943763in}}{\pgfqpoint{5.462879in}{2.932713in}}%
\pgfpathcurveto{\pgfqpoint{5.462879in}{2.921663in}}{\pgfqpoint{5.467269in}{2.911064in}}{\pgfqpoint{5.475083in}{2.903250in}}%
\pgfpathcurveto{\pgfqpoint{5.482896in}{2.895436in}}{\pgfqpoint{5.493495in}{2.891046in}}{\pgfqpoint{5.504545in}{2.891046in}}%
\pgfpathclose%
\pgfusepath{stroke,fill}%
\end{pgfscope}%
\begin{pgfscope}%
\pgfpathrectangle{\pgfqpoint{0.800000in}{0.528000in}}{\pgfqpoint{4.960000in}{3.696000in}}%
\pgfusepath{clip}%
\pgfsetbuttcap%
\pgfsetroundjoin%
\definecolor{currentfill}{rgb}{0.000000,0.000000,0.000000}%
\pgfsetfillcolor{currentfill}%
\pgfsetlinewidth{1.003750pt}%
\definecolor{currentstroke}{rgb}{0.000000,0.000000,0.000000}%
\pgfsetstrokecolor{currentstroke}%
\pgfsetdash{}{0pt}%
\pgfpathmoveto{\pgfqpoint{5.504545in}{2.740564in}}%
\pgfpathcurveto{\pgfqpoint{5.515596in}{2.740564in}}{\pgfqpoint{5.526195in}{2.744954in}}{\pgfqpoint{5.534008in}{2.752768in}}%
\pgfpathcurveto{\pgfqpoint{5.541822in}{2.760581in}}{\pgfqpoint{5.546212in}{2.771180in}}{\pgfqpoint{5.546212in}{2.782230in}}%
\pgfpathcurveto{\pgfqpoint{5.546212in}{2.793281in}}{\pgfqpoint{5.541822in}{2.803880in}}{\pgfqpoint{5.534008in}{2.811693in}}%
\pgfpathcurveto{\pgfqpoint{5.526195in}{2.819507in}}{\pgfqpoint{5.515596in}{2.823897in}}{\pgfqpoint{5.504545in}{2.823897in}}%
\pgfpathcurveto{\pgfqpoint{5.493495in}{2.823897in}}{\pgfqpoint{5.482896in}{2.819507in}}{\pgfqpoint{5.475083in}{2.811693in}}%
\pgfpathcurveto{\pgfqpoint{5.467269in}{2.803880in}}{\pgfqpoint{5.462879in}{2.793281in}}{\pgfqpoint{5.462879in}{2.782230in}}%
\pgfpathcurveto{\pgfqpoint{5.462879in}{2.771180in}}{\pgfqpoint{5.467269in}{2.760581in}}{\pgfqpoint{5.475083in}{2.752768in}}%
\pgfpathcurveto{\pgfqpoint{5.482896in}{2.744954in}}{\pgfqpoint{5.493495in}{2.740564in}}{\pgfqpoint{5.504545in}{2.740564in}}%
\pgfpathclose%
\pgfusepath{stroke,fill}%
\end{pgfscope}%
\begin{pgfscope}%
\pgfpathrectangle{\pgfqpoint{0.800000in}{0.528000in}}{\pgfqpoint{4.960000in}{3.696000in}}%
\pgfusepath{clip}%
\pgfsetbuttcap%
\pgfsetroundjoin%
\definecolor{currentfill}{rgb}{0.000000,0.000000,0.000000}%
\pgfsetfillcolor{currentfill}%
\pgfsetlinewidth{1.003750pt}%
\definecolor{currentstroke}{rgb}{0.000000,0.000000,0.000000}%
\pgfsetstrokecolor{currentstroke}%
\pgfsetdash{}{0pt}%
\pgfpathmoveto{\pgfqpoint{5.504545in}{2.826554in}}%
\pgfpathcurveto{\pgfqpoint{5.515596in}{2.826554in}}{\pgfqpoint{5.526195in}{2.830944in}}{\pgfqpoint{5.534008in}{2.838758in}}%
\pgfpathcurveto{\pgfqpoint{5.541822in}{2.846571in}}{\pgfqpoint{5.546212in}{2.857170in}}{\pgfqpoint{5.546212in}{2.868220in}}%
\pgfpathcurveto{\pgfqpoint{5.546212in}{2.879270in}}{\pgfqpoint{5.541822in}{2.889870in}}{\pgfqpoint{5.534008in}{2.897683in}}%
\pgfpathcurveto{\pgfqpoint{5.526195in}{2.905497in}}{\pgfqpoint{5.515596in}{2.909887in}}{\pgfqpoint{5.504545in}{2.909887in}}%
\pgfpathcurveto{\pgfqpoint{5.493495in}{2.909887in}}{\pgfqpoint{5.482896in}{2.905497in}}{\pgfqpoint{5.475083in}{2.897683in}}%
\pgfpathcurveto{\pgfqpoint{5.467269in}{2.889870in}}{\pgfqpoint{5.462879in}{2.879270in}}{\pgfqpoint{5.462879in}{2.868220in}}%
\pgfpathcurveto{\pgfqpoint{5.462879in}{2.857170in}}{\pgfqpoint{5.467269in}{2.846571in}}{\pgfqpoint{5.475083in}{2.838758in}}%
\pgfpathcurveto{\pgfqpoint{5.482896in}{2.830944in}}{\pgfqpoint{5.493495in}{2.826554in}}{\pgfqpoint{5.504545in}{2.826554in}}%
\pgfpathclose%
\pgfusepath{stroke,fill}%
\end{pgfscope}%
\begin{pgfscope}%
\pgfpathrectangle{\pgfqpoint{0.800000in}{0.528000in}}{\pgfqpoint{4.960000in}{3.696000in}}%
\pgfusepath{clip}%
\pgfsetbuttcap%
\pgfsetroundjoin%
\definecolor{currentfill}{rgb}{0.000000,0.000000,0.000000}%
\pgfsetfillcolor{currentfill}%
\pgfsetlinewidth{1.003750pt}%
\definecolor{currentstroke}{rgb}{0.000000,0.000000,0.000000}%
\pgfsetstrokecolor{currentstroke}%
\pgfsetdash{}{0pt}%
\pgfpathmoveto{\pgfqpoint{5.504545in}{3.041529in}}%
\pgfpathcurveto{\pgfqpoint{5.515596in}{3.041529in}}{\pgfqpoint{5.526195in}{3.045919in}}{\pgfqpoint{5.534008in}{3.053732in}}%
\pgfpathcurveto{\pgfqpoint{5.541822in}{3.061546in}}{\pgfqpoint{5.546212in}{3.072145in}}{\pgfqpoint{5.546212in}{3.083195in}}%
\pgfpathcurveto{\pgfqpoint{5.546212in}{3.094245in}}{\pgfqpoint{5.541822in}{3.104844in}}{\pgfqpoint{5.534008in}{3.112658in}}%
\pgfpathcurveto{\pgfqpoint{5.526195in}{3.120472in}}{\pgfqpoint{5.515596in}{3.124862in}}{\pgfqpoint{5.504545in}{3.124862in}}%
\pgfpathcurveto{\pgfqpoint{5.493495in}{3.124862in}}{\pgfqpoint{5.482896in}{3.120472in}}{\pgfqpoint{5.475083in}{3.112658in}}%
\pgfpathcurveto{\pgfqpoint{5.467269in}{3.104844in}}{\pgfqpoint{5.462879in}{3.094245in}}{\pgfqpoint{5.462879in}{3.083195in}}%
\pgfpathcurveto{\pgfqpoint{5.462879in}{3.072145in}}{\pgfqpoint{5.467269in}{3.061546in}}{\pgfqpoint{5.475083in}{3.053732in}}%
\pgfpathcurveto{\pgfqpoint{5.482896in}{3.045919in}}{\pgfqpoint{5.493495in}{3.041529in}}{\pgfqpoint{5.504545in}{3.041529in}}%
\pgfpathclose%
\pgfusepath{stroke,fill}%
\end{pgfscope}%
\begin{pgfscope}%
\pgfpathrectangle{\pgfqpoint{0.800000in}{0.528000in}}{\pgfqpoint{4.960000in}{3.696000in}}%
\pgfusepath{clip}%
\pgfsetbuttcap%
\pgfsetroundjoin%
\definecolor{currentfill}{rgb}{0.000000,0.000000,0.000000}%
\pgfsetfillcolor{currentfill}%
\pgfsetlinewidth{1.003750pt}%
\definecolor{currentstroke}{rgb}{0.000000,0.000000,0.000000}%
\pgfsetstrokecolor{currentstroke}%
\pgfsetdash{}{0pt}%
\pgfpathmoveto{\pgfqpoint{5.504545in}{3.299499in}}%
\pgfpathcurveto{\pgfqpoint{5.515596in}{3.299499in}}{\pgfqpoint{5.526195in}{3.303889in}}{\pgfqpoint{5.534008in}{3.311702in}}%
\pgfpathcurveto{\pgfqpoint{5.541822in}{3.319516in}}{\pgfqpoint{5.546212in}{3.330115in}}{\pgfqpoint{5.546212in}{3.341165in}}%
\pgfpathcurveto{\pgfqpoint{5.546212in}{3.352215in}}{\pgfqpoint{5.541822in}{3.362814in}}{\pgfqpoint{5.534008in}{3.370628in}}%
\pgfpathcurveto{\pgfqpoint{5.526195in}{3.378442in}}{\pgfqpoint{5.515596in}{3.382832in}}{\pgfqpoint{5.504545in}{3.382832in}}%
\pgfpathcurveto{\pgfqpoint{5.493495in}{3.382832in}}{\pgfqpoint{5.482896in}{3.378442in}}{\pgfqpoint{5.475083in}{3.370628in}}%
\pgfpathcurveto{\pgfqpoint{5.467269in}{3.362814in}}{\pgfqpoint{5.462879in}{3.352215in}}{\pgfqpoint{5.462879in}{3.341165in}}%
\pgfpathcurveto{\pgfqpoint{5.462879in}{3.330115in}}{\pgfqpoint{5.467269in}{3.319516in}}{\pgfqpoint{5.475083in}{3.311702in}}%
\pgfpathcurveto{\pgfqpoint{5.482896in}{3.303889in}}{\pgfqpoint{5.493495in}{3.299499in}}{\pgfqpoint{5.504545in}{3.299499in}}%
\pgfpathclose%
\pgfusepath{stroke,fill}%
\end{pgfscope}%
\begin{pgfscope}%
\pgfpathrectangle{\pgfqpoint{0.800000in}{0.528000in}}{\pgfqpoint{4.960000in}{3.696000in}}%
\pgfusepath{clip}%
\pgfsetbuttcap%
\pgfsetroundjoin%
\definecolor{currentfill}{rgb}{0.000000,0.000000,0.000000}%
\pgfsetfillcolor{currentfill}%
\pgfsetlinewidth{1.003750pt}%
\definecolor{currentstroke}{rgb}{0.000000,0.000000,0.000000}%
\pgfsetstrokecolor{currentstroke}%
\pgfsetdash{}{0pt}%
\pgfpathmoveto{\pgfqpoint{5.504545in}{2.676071in}}%
\pgfpathcurveto{\pgfqpoint{5.515596in}{2.676071in}}{\pgfqpoint{5.526195in}{2.680462in}}{\pgfqpoint{5.534008in}{2.688275in}}%
\pgfpathcurveto{\pgfqpoint{5.541822in}{2.696089in}}{\pgfqpoint{5.546212in}{2.706688in}}{\pgfqpoint{5.546212in}{2.717738in}}%
\pgfpathcurveto{\pgfqpoint{5.546212in}{2.728788in}}{\pgfqpoint{5.541822in}{2.739387in}}{\pgfqpoint{5.534008in}{2.747201in}}%
\pgfpathcurveto{\pgfqpoint{5.526195in}{2.755014in}}{\pgfqpoint{5.515596in}{2.759405in}}{\pgfqpoint{5.504545in}{2.759405in}}%
\pgfpathcurveto{\pgfqpoint{5.493495in}{2.759405in}}{\pgfqpoint{5.482896in}{2.755014in}}{\pgfqpoint{5.475083in}{2.747201in}}%
\pgfpathcurveto{\pgfqpoint{5.467269in}{2.739387in}}{\pgfqpoint{5.462879in}{2.728788in}}{\pgfqpoint{5.462879in}{2.717738in}}%
\pgfpathcurveto{\pgfqpoint{5.462879in}{2.706688in}}{\pgfqpoint{5.467269in}{2.696089in}}{\pgfqpoint{5.475083in}{2.688275in}}%
\pgfpathcurveto{\pgfqpoint{5.482896in}{2.680462in}}{\pgfqpoint{5.493495in}{2.676071in}}{\pgfqpoint{5.504545in}{2.676071in}}%
\pgfpathclose%
\pgfusepath{stroke,fill}%
\end{pgfscope}%
\begin{pgfscope}%
\pgfpathrectangle{\pgfqpoint{0.800000in}{0.528000in}}{\pgfqpoint{4.960000in}{3.696000in}}%
\pgfusepath{clip}%
\pgfsetbuttcap%
\pgfsetroundjoin%
\definecolor{currentfill}{rgb}{0.000000,0.000000,0.000000}%
\pgfsetfillcolor{currentfill}%
\pgfsetlinewidth{1.003750pt}%
\definecolor{currentstroke}{rgb}{0.000000,0.000000,0.000000}%
\pgfsetstrokecolor{currentstroke}%
\pgfsetdash{}{0pt}%
\pgfpathmoveto{\pgfqpoint{5.504545in}{2.719066in}}%
\pgfpathcurveto{\pgfqpoint{5.515596in}{2.719066in}}{\pgfqpoint{5.526195in}{2.723456in}}{\pgfqpoint{5.534008in}{2.731270in}}%
\pgfpathcurveto{\pgfqpoint{5.541822in}{2.739084in}}{\pgfqpoint{5.546212in}{2.749683in}}{\pgfqpoint{5.546212in}{2.760733in}}%
\pgfpathcurveto{\pgfqpoint{5.546212in}{2.771783in}}{\pgfqpoint{5.541822in}{2.782382in}}{\pgfqpoint{5.534008in}{2.790196in}}%
\pgfpathcurveto{\pgfqpoint{5.526195in}{2.798009in}}{\pgfqpoint{5.515596in}{2.802400in}}{\pgfqpoint{5.504545in}{2.802400in}}%
\pgfpathcurveto{\pgfqpoint{5.493495in}{2.802400in}}{\pgfqpoint{5.482896in}{2.798009in}}{\pgfqpoint{5.475083in}{2.790196in}}%
\pgfpathcurveto{\pgfqpoint{5.467269in}{2.782382in}}{\pgfqpoint{5.462879in}{2.771783in}}{\pgfqpoint{5.462879in}{2.760733in}}%
\pgfpathcurveto{\pgfqpoint{5.462879in}{2.749683in}}{\pgfqpoint{5.467269in}{2.739084in}}{\pgfqpoint{5.475083in}{2.731270in}}%
\pgfpathcurveto{\pgfqpoint{5.482896in}{2.723456in}}{\pgfqpoint{5.493495in}{2.719066in}}{\pgfqpoint{5.504545in}{2.719066in}}%
\pgfpathclose%
\pgfusepath{stroke,fill}%
\end{pgfscope}%
\begin{pgfscope}%
\pgfpathrectangle{\pgfqpoint{0.800000in}{0.528000in}}{\pgfqpoint{4.960000in}{3.696000in}}%
\pgfusepath{clip}%
\pgfsetbuttcap%
\pgfsetroundjoin%
\definecolor{currentfill}{rgb}{0.000000,0.000000,0.000000}%
\pgfsetfillcolor{currentfill}%
\pgfsetlinewidth{1.003750pt}%
\definecolor{currentstroke}{rgb}{0.000000,0.000000,0.000000}%
\pgfsetstrokecolor{currentstroke}%
\pgfsetdash{}{0pt}%
\pgfpathmoveto{\pgfqpoint{5.504545in}{3.707951in}}%
\pgfpathcurveto{\pgfqpoint{5.515596in}{3.707951in}}{\pgfqpoint{5.526195in}{3.712341in}}{\pgfqpoint{5.534008in}{3.720155in}}%
\pgfpathcurveto{\pgfqpoint{5.541822in}{3.727968in}}{\pgfqpoint{5.546212in}{3.738567in}}{\pgfqpoint{5.546212in}{3.749618in}}%
\pgfpathcurveto{\pgfqpoint{5.546212in}{3.760668in}}{\pgfqpoint{5.541822in}{3.771267in}}{\pgfqpoint{5.534008in}{3.779080in}}%
\pgfpathcurveto{\pgfqpoint{5.526195in}{3.786894in}}{\pgfqpoint{5.515596in}{3.791284in}}{\pgfqpoint{5.504545in}{3.791284in}}%
\pgfpathcurveto{\pgfqpoint{5.493495in}{3.791284in}}{\pgfqpoint{5.482896in}{3.786894in}}{\pgfqpoint{5.475083in}{3.779080in}}%
\pgfpathcurveto{\pgfqpoint{5.467269in}{3.771267in}}{\pgfqpoint{5.462879in}{3.760668in}}{\pgfqpoint{5.462879in}{3.749618in}}%
\pgfpathcurveto{\pgfqpoint{5.462879in}{3.738567in}}{\pgfqpoint{5.467269in}{3.727968in}}{\pgfqpoint{5.475083in}{3.720155in}}%
\pgfpathcurveto{\pgfqpoint{5.482896in}{3.712341in}}{\pgfqpoint{5.493495in}{3.707951in}}{\pgfqpoint{5.504545in}{3.707951in}}%
\pgfpathclose%
\pgfusepath{stroke,fill}%
\end{pgfscope}%
\begin{pgfscope}%
\pgfpathrectangle{\pgfqpoint{0.800000in}{0.528000in}}{\pgfqpoint{4.960000in}{3.696000in}}%
\pgfusepath{clip}%
\pgfsetbuttcap%
\pgfsetroundjoin%
\definecolor{currentfill}{rgb}{0.000000,0.000000,0.000000}%
\pgfsetfillcolor{currentfill}%
\pgfsetlinewidth{1.003750pt}%
\definecolor{currentstroke}{rgb}{0.000000,0.000000,0.000000}%
\pgfsetstrokecolor{currentstroke}%
\pgfsetdash{}{0pt}%
\pgfpathmoveto{\pgfqpoint{5.504545in}{2.676071in}}%
\pgfpathcurveto{\pgfqpoint{5.515596in}{2.676071in}}{\pgfqpoint{5.526195in}{2.680462in}}{\pgfqpoint{5.534008in}{2.688275in}}%
\pgfpathcurveto{\pgfqpoint{5.541822in}{2.696089in}}{\pgfqpoint{5.546212in}{2.706688in}}{\pgfqpoint{5.546212in}{2.717738in}}%
\pgfpathcurveto{\pgfqpoint{5.546212in}{2.728788in}}{\pgfqpoint{5.541822in}{2.739387in}}{\pgfqpoint{5.534008in}{2.747201in}}%
\pgfpathcurveto{\pgfqpoint{5.526195in}{2.755014in}}{\pgfqpoint{5.515596in}{2.759405in}}{\pgfqpoint{5.504545in}{2.759405in}}%
\pgfpathcurveto{\pgfqpoint{5.493495in}{2.759405in}}{\pgfqpoint{5.482896in}{2.755014in}}{\pgfqpoint{5.475083in}{2.747201in}}%
\pgfpathcurveto{\pgfqpoint{5.467269in}{2.739387in}}{\pgfqpoint{5.462879in}{2.728788in}}{\pgfqpoint{5.462879in}{2.717738in}}%
\pgfpathcurveto{\pgfqpoint{5.462879in}{2.706688in}}{\pgfqpoint{5.467269in}{2.696089in}}{\pgfqpoint{5.475083in}{2.688275in}}%
\pgfpathcurveto{\pgfqpoint{5.482896in}{2.680462in}}{\pgfqpoint{5.493495in}{2.676071in}}{\pgfqpoint{5.504545in}{2.676071in}}%
\pgfpathclose%
\pgfusepath{stroke,fill}%
\end{pgfscope}%
\begin{pgfscope}%
\pgfpathrectangle{\pgfqpoint{0.800000in}{0.528000in}}{\pgfqpoint{4.960000in}{3.696000in}}%
\pgfusepath{clip}%
\pgfsetbuttcap%
\pgfsetroundjoin%
\definecolor{currentfill}{rgb}{0.000000,0.000000,0.000000}%
\pgfsetfillcolor{currentfill}%
\pgfsetlinewidth{1.003750pt}%
\definecolor{currentstroke}{rgb}{0.000000,0.000000,0.000000}%
\pgfsetstrokecolor{currentstroke}%
\pgfsetdash{}{0pt}%
\pgfpathmoveto{\pgfqpoint{5.504545in}{3.106021in}}%
\pgfpathcurveto{\pgfqpoint{5.515596in}{3.106021in}}{\pgfqpoint{5.526195in}{3.110411in}}{\pgfqpoint{5.534008in}{3.118225in}}%
\pgfpathcurveto{\pgfqpoint{5.541822in}{3.126039in}}{\pgfqpoint{5.546212in}{3.136638in}}{\pgfqpoint{5.546212in}{3.147688in}}%
\pgfpathcurveto{\pgfqpoint{5.546212in}{3.158738in}}{\pgfqpoint{5.541822in}{3.169337in}}{\pgfqpoint{5.534008in}{3.177151in}}%
\pgfpathcurveto{\pgfqpoint{5.526195in}{3.184964in}}{\pgfqpoint{5.515596in}{3.189354in}}{\pgfqpoint{5.504545in}{3.189354in}}%
\pgfpathcurveto{\pgfqpoint{5.493495in}{3.189354in}}{\pgfqpoint{5.482896in}{3.184964in}}{\pgfqpoint{5.475083in}{3.177151in}}%
\pgfpathcurveto{\pgfqpoint{5.467269in}{3.169337in}}{\pgfqpoint{5.462879in}{3.158738in}}{\pgfqpoint{5.462879in}{3.147688in}}%
\pgfpathcurveto{\pgfqpoint{5.462879in}{3.136638in}}{\pgfqpoint{5.467269in}{3.126039in}}{\pgfqpoint{5.475083in}{3.118225in}}%
\pgfpathcurveto{\pgfqpoint{5.482896in}{3.110411in}}{\pgfqpoint{5.493495in}{3.106021in}}{\pgfqpoint{5.504545in}{3.106021in}}%
\pgfpathclose%
\pgfusepath{stroke,fill}%
\end{pgfscope}%
\begin{pgfscope}%
\pgfpathrectangle{\pgfqpoint{0.800000in}{0.528000in}}{\pgfqpoint{4.960000in}{3.696000in}}%
\pgfusepath{clip}%
\pgfsetbuttcap%
\pgfsetroundjoin%
\definecolor{currentfill}{rgb}{0.000000,0.000000,0.000000}%
\pgfsetfillcolor{currentfill}%
\pgfsetlinewidth{1.003750pt}%
\definecolor{currentstroke}{rgb}{0.000000,0.000000,0.000000}%
\pgfsetstrokecolor{currentstroke}%
\pgfsetdash{}{0pt}%
\pgfpathmoveto{\pgfqpoint{5.504545in}{2.654574in}}%
\pgfpathcurveto{\pgfqpoint{5.515596in}{2.654574in}}{\pgfqpoint{5.526195in}{2.658964in}}{\pgfqpoint{5.534008in}{2.666778in}}%
\pgfpathcurveto{\pgfqpoint{5.541822in}{2.674591in}}{\pgfqpoint{5.546212in}{2.685190in}}{\pgfqpoint{5.546212in}{2.696240in}}%
\pgfpathcurveto{\pgfqpoint{5.546212in}{2.707291in}}{\pgfqpoint{5.541822in}{2.717890in}}{\pgfqpoint{5.534008in}{2.725703in}}%
\pgfpathcurveto{\pgfqpoint{5.526195in}{2.733517in}}{\pgfqpoint{5.515596in}{2.737907in}}{\pgfqpoint{5.504545in}{2.737907in}}%
\pgfpathcurveto{\pgfqpoint{5.493495in}{2.737907in}}{\pgfqpoint{5.482896in}{2.733517in}}{\pgfqpoint{5.475083in}{2.725703in}}%
\pgfpathcurveto{\pgfqpoint{5.467269in}{2.717890in}}{\pgfqpoint{5.462879in}{2.707291in}}{\pgfqpoint{5.462879in}{2.696240in}}%
\pgfpathcurveto{\pgfqpoint{5.462879in}{2.685190in}}{\pgfqpoint{5.467269in}{2.674591in}}{\pgfqpoint{5.475083in}{2.666778in}}%
\pgfpathcurveto{\pgfqpoint{5.482896in}{2.658964in}}{\pgfqpoint{5.493495in}{2.654574in}}{\pgfqpoint{5.504545in}{2.654574in}}%
\pgfpathclose%
\pgfusepath{stroke,fill}%
\end{pgfscope}%
\begin{pgfscope}%
\pgfpathrectangle{\pgfqpoint{0.800000in}{0.528000in}}{\pgfqpoint{4.960000in}{3.696000in}}%
\pgfusepath{clip}%
\pgfsetbuttcap%
\pgfsetroundjoin%
\definecolor{currentfill}{rgb}{0.000000,0.000000,0.000000}%
\pgfsetfillcolor{currentfill}%
\pgfsetlinewidth{1.003750pt}%
\definecolor{currentstroke}{rgb}{0.000000,0.000000,0.000000}%
\pgfsetstrokecolor{currentstroke}%
\pgfsetdash{}{0pt}%
\pgfpathmoveto{\pgfqpoint{5.504545in}{2.762061in}}%
\pgfpathcurveto{\pgfqpoint{5.515596in}{2.762061in}}{\pgfqpoint{5.526195in}{2.766451in}}{\pgfqpoint{5.534008in}{2.774265in}}%
\pgfpathcurveto{\pgfqpoint{5.541822in}{2.782079in}}{\pgfqpoint{5.546212in}{2.792678in}}{\pgfqpoint{5.546212in}{2.803728in}}%
\pgfpathcurveto{\pgfqpoint{5.546212in}{2.814778in}}{\pgfqpoint{5.541822in}{2.825377in}}{\pgfqpoint{5.534008in}{2.833191in}}%
\pgfpathcurveto{\pgfqpoint{5.526195in}{2.841004in}}{\pgfqpoint{5.515596in}{2.845395in}}{\pgfqpoint{5.504545in}{2.845395in}}%
\pgfpathcurveto{\pgfqpoint{5.493495in}{2.845395in}}{\pgfqpoint{5.482896in}{2.841004in}}{\pgfqpoint{5.475083in}{2.833191in}}%
\pgfpathcurveto{\pgfqpoint{5.467269in}{2.825377in}}{\pgfqpoint{5.462879in}{2.814778in}}{\pgfqpoint{5.462879in}{2.803728in}}%
\pgfpathcurveto{\pgfqpoint{5.462879in}{2.792678in}}{\pgfqpoint{5.467269in}{2.782079in}}{\pgfqpoint{5.475083in}{2.774265in}}%
\pgfpathcurveto{\pgfqpoint{5.482896in}{2.766451in}}{\pgfqpoint{5.493495in}{2.762061in}}{\pgfqpoint{5.504545in}{2.762061in}}%
\pgfpathclose%
\pgfusepath{stroke,fill}%
\end{pgfscope}%
\begin{pgfscope}%
\pgfpathrectangle{\pgfqpoint{0.800000in}{0.528000in}}{\pgfqpoint{4.960000in}{3.696000in}}%
\pgfusepath{clip}%
\pgfsetbuttcap%
\pgfsetroundjoin%
\definecolor{currentfill}{rgb}{0.000000,0.000000,0.000000}%
\pgfsetfillcolor{currentfill}%
\pgfsetlinewidth{1.003750pt}%
\definecolor{currentstroke}{rgb}{0.000000,0.000000,0.000000}%
\pgfsetstrokecolor{currentstroke}%
\pgfsetdash{}{0pt}%
\pgfpathmoveto{\pgfqpoint{5.504545in}{2.934041in}}%
\pgfpathcurveto{\pgfqpoint{5.515596in}{2.934041in}}{\pgfqpoint{5.526195in}{2.938431in}}{\pgfqpoint{5.534008in}{2.946245in}}%
\pgfpathcurveto{\pgfqpoint{5.541822in}{2.954059in}}{\pgfqpoint{5.546212in}{2.964658in}}{\pgfqpoint{5.546212in}{2.975708in}}%
\pgfpathcurveto{\pgfqpoint{5.546212in}{2.986758in}}{\pgfqpoint{5.541822in}{2.997357in}}{\pgfqpoint{5.534008in}{3.005171in}}%
\pgfpathcurveto{\pgfqpoint{5.526195in}{3.012984in}}{\pgfqpoint{5.515596in}{3.017374in}}{\pgfqpoint{5.504545in}{3.017374in}}%
\pgfpathcurveto{\pgfqpoint{5.493495in}{3.017374in}}{\pgfqpoint{5.482896in}{3.012984in}}{\pgfqpoint{5.475083in}{3.005171in}}%
\pgfpathcurveto{\pgfqpoint{5.467269in}{2.997357in}}{\pgfqpoint{5.462879in}{2.986758in}}{\pgfqpoint{5.462879in}{2.975708in}}%
\pgfpathcurveto{\pgfqpoint{5.462879in}{2.964658in}}{\pgfqpoint{5.467269in}{2.954059in}}{\pgfqpoint{5.475083in}{2.946245in}}%
\pgfpathcurveto{\pgfqpoint{5.482896in}{2.938431in}}{\pgfqpoint{5.493495in}{2.934041in}}{\pgfqpoint{5.504545in}{2.934041in}}%
\pgfpathclose%
\pgfusepath{stroke,fill}%
\end{pgfscope}%
\begin{pgfscope}%
\pgfpathrectangle{\pgfqpoint{0.800000in}{0.528000in}}{\pgfqpoint{4.960000in}{3.696000in}}%
\pgfusepath{clip}%
\pgfsetbuttcap%
\pgfsetroundjoin%
\definecolor{currentfill}{rgb}{0.000000,0.000000,0.000000}%
\pgfsetfillcolor{currentfill}%
\pgfsetlinewidth{1.003750pt}%
\definecolor{currentstroke}{rgb}{0.000000,0.000000,0.000000}%
\pgfsetstrokecolor{currentstroke}%
\pgfsetdash{}{0pt}%
\pgfpathmoveto{\pgfqpoint{5.504545in}{3.256504in}}%
\pgfpathcurveto{\pgfqpoint{5.515596in}{3.256504in}}{\pgfqpoint{5.526195in}{3.260894in}}{\pgfqpoint{5.534008in}{3.268707in}}%
\pgfpathcurveto{\pgfqpoint{5.541822in}{3.276521in}}{\pgfqpoint{5.546212in}{3.287120in}}{\pgfqpoint{5.546212in}{3.298170in}}%
\pgfpathcurveto{\pgfqpoint{5.546212in}{3.309220in}}{\pgfqpoint{5.541822in}{3.319819in}}{\pgfqpoint{5.534008in}{3.327633in}}%
\pgfpathcurveto{\pgfqpoint{5.526195in}{3.335447in}}{\pgfqpoint{5.515596in}{3.339837in}}{\pgfqpoint{5.504545in}{3.339837in}}%
\pgfpathcurveto{\pgfqpoint{5.493495in}{3.339837in}}{\pgfqpoint{5.482896in}{3.335447in}}{\pgfqpoint{5.475083in}{3.327633in}}%
\pgfpathcurveto{\pgfqpoint{5.467269in}{3.319819in}}{\pgfqpoint{5.462879in}{3.309220in}}{\pgfqpoint{5.462879in}{3.298170in}}%
\pgfpathcurveto{\pgfqpoint{5.462879in}{3.287120in}}{\pgfqpoint{5.467269in}{3.276521in}}{\pgfqpoint{5.475083in}{3.268707in}}%
\pgfpathcurveto{\pgfqpoint{5.482896in}{3.260894in}}{\pgfqpoint{5.493495in}{3.256504in}}{\pgfqpoint{5.504545in}{3.256504in}}%
\pgfpathclose%
\pgfusepath{stroke,fill}%
\end{pgfscope}%
\begin{pgfscope}%
\pgfpathrectangle{\pgfqpoint{0.800000in}{0.528000in}}{\pgfqpoint{4.960000in}{3.696000in}}%
\pgfusepath{clip}%
\pgfsetbuttcap%
\pgfsetroundjoin%
\definecolor{currentfill}{rgb}{0.000000,0.000000,0.000000}%
\pgfsetfillcolor{currentfill}%
\pgfsetlinewidth{1.003750pt}%
\definecolor{currentstroke}{rgb}{0.000000,0.000000,0.000000}%
\pgfsetstrokecolor{currentstroke}%
\pgfsetdash{}{0pt}%
\pgfpathmoveto{\pgfqpoint{5.504545in}{2.869549in}}%
\pgfpathcurveto{\pgfqpoint{5.515596in}{2.869549in}}{\pgfqpoint{5.526195in}{2.873939in}}{\pgfqpoint{5.534008in}{2.881753in}}%
\pgfpathcurveto{\pgfqpoint{5.541822in}{2.889566in}}{\pgfqpoint{5.546212in}{2.900165in}}{\pgfqpoint{5.546212in}{2.911215in}}%
\pgfpathcurveto{\pgfqpoint{5.546212in}{2.922265in}}{\pgfqpoint{5.541822in}{2.932864in}}{\pgfqpoint{5.534008in}{2.940678in}}%
\pgfpathcurveto{\pgfqpoint{5.526195in}{2.948492in}}{\pgfqpoint{5.515596in}{2.952882in}}{\pgfqpoint{5.504545in}{2.952882in}}%
\pgfpathcurveto{\pgfqpoint{5.493495in}{2.952882in}}{\pgfqpoint{5.482896in}{2.948492in}}{\pgfqpoint{5.475083in}{2.940678in}}%
\pgfpathcurveto{\pgfqpoint{5.467269in}{2.932864in}}{\pgfqpoint{5.462879in}{2.922265in}}{\pgfqpoint{5.462879in}{2.911215in}}%
\pgfpathcurveto{\pgfqpoint{5.462879in}{2.900165in}}{\pgfqpoint{5.467269in}{2.889566in}}{\pgfqpoint{5.475083in}{2.881753in}}%
\pgfpathcurveto{\pgfqpoint{5.482896in}{2.873939in}}{\pgfqpoint{5.493495in}{2.869549in}}{\pgfqpoint{5.504545in}{2.869549in}}%
\pgfpathclose%
\pgfusepath{stroke,fill}%
\end{pgfscope}%
\begin{pgfscope}%
\pgfpathrectangle{\pgfqpoint{0.800000in}{0.528000in}}{\pgfqpoint{4.960000in}{3.696000in}}%
\pgfusepath{clip}%
\pgfsetbuttcap%
\pgfsetroundjoin%
\definecolor{currentfill}{rgb}{0.000000,0.000000,0.000000}%
\pgfsetfillcolor{currentfill}%
\pgfsetlinewidth{1.003750pt}%
\definecolor{currentstroke}{rgb}{0.000000,0.000000,0.000000}%
\pgfsetstrokecolor{currentstroke}%
\pgfsetdash{}{0pt}%
\pgfpathmoveto{\pgfqpoint{5.504545in}{2.783559in}}%
\pgfpathcurveto{\pgfqpoint{5.515596in}{2.783559in}}{\pgfqpoint{5.526195in}{2.787949in}}{\pgfqpoint{5.534008in}{2.795763in}}%
\pgfpathcurveto{\pgfqpoint{5.541822in}{2.803576in}}{\pgfqpoint{5.546212in}{2.814175in}}{\pgfqpoint{5.546212in}{2.825225in}}%
\pgfpathcurveto{\pgfqpoint{5.546212in}{2.836275in}}{\pgfqpoint{5.541822in}{2.846875in}}{\pgfqpoint{5.534008in}{2.854688in}}%
\pgfpathcurveto{\pgfqpoint{5.526195in}{2.862502in}}{\pgfqpoint{5.515596in}{2.866892in}}{\pgfqpoint{5.504545in}{2.866892in}}%
\pgfpathcurveto{\pgfqpoint{5.493495in}{2.866892in}}{\pgfqpoint{5.482896in}{2.862502in}}{\pgfqpoint{5.475083in}{2.854688in}}%
\pgfpathcurveto{\pgfqpoint{5.467269in}{2.846875in}}{\pgfqpoint{5.462879in}{2.836275in}}{\pgfqpoint{5.462879in}{2.825225in}}%
\pgfpathcurveto{\pgfqpoint{5.462879in}{2.814175in}}{\pgfqpoint{5.467269in}{2.803576in}}{\pgfqpoint{5.475083in}{2.795763in}}%
\pgfpathcurveto{\pgfqpoint{5.482896in}{2.787949in}}{\pgfqpoint{5.493495in}{2.783559in}}{\pgfqpoint{5.504545in}{2.783559in}}%
\pgfpathclose%
\pgfusepath{stroke,fill}%
\end{pgfscope}%
\begin{pgfscope}%
\pgfpathrectangle{\pgfqpoint{0.800000in}{0.528000in}}{\pgfqpoint{4.960000in}{3.696000in}}%
\pgfusepath{clip}%
\pgfsetbuttcap%
\pgfsetroundjoin%
\definecolor{currentfill}{rgb}{0.000000,0.000000,0.000000}%
\pgfsetfillcolor{currentfill}%
\pgfsetlinewidth{1.003750pt}%
\definecolor{currentstroke}{rgb}{0.000000,0.000000,0.000000}%
\pgfsetstrokecolor{currentstroke}%
\pgfsetdash{}{0pt}%
\pgfpathmoveto{\pgfqpoint{5.504545in}{2.719066in}}%
\pgfpathcurveto{\pgfqpoint{5.515596in}{2.719066in}}{\pgfqpoint{5.526195in}{2.723456in}}{\pgfqpoint{5.534008in}{2.731270in}}%
\pgfpathcurveto{\pgfqpoint{5.541822in}{2.739084in}}{\pgfqpoint{5.546212in}{2.749683in}}{\pgfqpoint{5.546212in}{2.760733in}}%
\pgfpathcurveto{\pgfqpoint{5.546212in}{2.771783in}}{\pgfqpoint{5.541822in}{2.782382in}}{\pgfqpoint{5.534008in}{2.790196in}}%
\pgfpathcurveto{\pgfqpoint{5.526195in}{2.798009in}}{\pgfqpoint{5.515596in}{2.802400in}}{\pgfqpoint{5.504545in}{2.802400in}}%
\pgfpathcurveto{\pgfqpoint{5.493495in}{2.802400in}}{\pgfqpoint{5.482896in}{2.798009in}}{\pgfqpoint{5.475083in}{2.790196in}}%
\pgfpathcurveto{\pgfqpoint{5.467269in}{2.782382in}}{\pgfqpoint{5.462879in}{2.771783in}}{\pgfqpoint{5.462879in}{2.760733in}}%
\pgfpathcurveto{\pgfqpoint{5.462879in}{2.749683in}}{\pgfqpoint{5.467269in}{2.739084in}}{\pgfqpoint{5.475083in}{2.731270in}}%
\pgfpathcurveto{\pgfqpoint{5.482896in}{2.723456in}}{\pgfqpoint{5.493495in}{2.719066in}}{\pgfqpoint{5.504545in}{2.719066in}}%
\pgfpathclose%
\pgfusepath{stroke,fill}%
\end{pgfscope}%
\begin{pgfscope}%
\pgfpathrectangle{\pgfqpoint{0.800000in}{0.528000in}}{\pgfqpoint{4.960000in}{3.696000in}}%
\pgfusepath{clip}%
\pgfsetbuttcap%
\pgfsetroundjoin%
\definecolor{currentfill}{rgb}{0.000000,0.000000,0.000000}%
\pgfsetfillcolor{currentfill}%
\pgfsetlinewidth{1.003750pt}%
\definecolor{currentstroke}{rgb}{0.000000,0.000000,0.000000}%
\pgfsetstrokecolor{currentstroke}%
\pgfsetdash{}{0pt}%
\pgfpathmoveto{\pgfqpoint{5.504545in}{2.740564in}}%
\pgfpathcurveto{\pgfqpoint{5.515596in}{2.740564in}}{\pgfqpoint{5.526195in}{2.744954in}}{\pgfqpoint{5.534008in}{2.752768in}}%
\pgfpathcurveto{\pgfqpoint{5.541822in}{2.760581in}}{\pgfqpoint{5.546212in}{2.771180in}}{\pgfqpoint{5.546212in}{2.782230in}}%
\pgfpathcurveto{\pgfqpoint{5.546212in}{2.793281in}}{\pgfqpoint{5.541822in}{2.803880in}}{\pgfqpoint{5.534008in}{2.811693in}}%
\pgfpathcurveto{\pgfqpoint{5.526195in}{2.819507in}}{\pgfqpoint{5.515596in}{2.823897in}}{\pgfqpoint{5.504545in}{2.823897in}}%
\pgfpathcurveto{\pgfqpoint{5.493495in}{2.823897in}}{\pgfqpoint{5.482896in}{2.819507in}}{\pgfqpoint{5.475083in}{2.811693in}}%
\pgfpathcurveto{\pgfqpoint{5.467269in}{2.803880in}}{\pgfqpoint{5.462879in}{2.793281in}}{\pgfqpoint{5.462879in}{2.782230in}}%
\pgfpathcurveto{\pgfqpoint{5.462879in}{2.771180in}}{\pgfqpoint{5.467269in}{2.760581in}}{\pgfqpoint{5.475083in}{2.752768in}}%
\pgfpathcurveto{\pgfqpoint{5.482896in}{2.744954in}}{\pgfqpoint{5.493495in}{2.740564in}}{\pgfqpoint{5.504545in}{2.740564in}}%
\pgfpathclose%
\pgfusepath{stroke,fill}%
\end{pgfscope}%
\begin{pgfscope}%
\pgfpathrectangle{\pgfqpoint{0.800000in}{0.528000in}}{\pgfqpoint{4.960000in}{3.696000in}}%
\pgfusepath{clip}%
\pgfsetbuttcap%
\pgfsetroundjoin%
\definecolor{currentfill}{rgb}{0.000000,0.000000,0.000000}%
\pgfsetfillcolor{currentfill}%
\pgfsetlinewidth{1.003750pt}%
\definecolor{currentstroke}{rgb}{0.000000,0.000000,0.000000}%
\pgfsetstrokecolor{currentstroke}%
\pgfsetdash{}{0pt}%
\pgfpathmoveto{\pgfqpoint{5.504545in}{3.170514in}}%
\pgfpathcurveto{\pgfqpoint{5.515596in}{3.170514in}}{\pgfqpoint{5.526195in}{3.174904in}}{\pgfqpoint{5.534008in}{3.182717in}}%
\pgfpathcurveto{\pgfqpoint{5.541822in}{3.190531in}}{\pgfqpoint{5.546212in}{3.201130in}}{\pgfqpoint{5.546212in}{3.212180in}}%
\pgfpathcurveto{\pgfqpoint{5.546212in}{3.223230in}}{\pgfqpoint{5.541822in}{3.233829in}}{\pgfqpoint{5.534008in}{3.241643in}}%
\pgfpathcurveto{\pgfqpoint{5.526195in}{3.249457in}}{\pgfqpoint{5.515596in}{3.253847in}}{\pgfqpoint{5.504545in}{3.253847in}}%
\pgfpathcurveto{\pgfqpoint{5.493495in}{3.253847in}}{\pgfqpoint{5.482896in}{3.249457in}}{\pgfqpoint{5.475083in}{3.241643in}}%
\pgfpathcurveto{\pgfqpoint{5.467269in}{3.233829in}}{\pgfqpoint{5.462879in}{3.223230in}}{\pgfqpoint{5.462879in}{3.212180in}}%
\pgfpathcurveto{\pgfqpoint{5.462879in}{3.201130in}}{\pgfqpoint{5.467269in}{3.190531in}}{\pgfqpoint{5.475083in}{3.182717in}}%
\pgfpathcurveto{\pgfqpoint{5.482896in}{3.174904in}}{\pgfqpoint{5.493495in}{3.170514in}}{\pgfqpoint{5.504545in}{3.170514in}}%
\pgfpathclose%
\pgfusepath{stroke,fill}%
\end{pgfscope}%
\begin{pgfscope}%
\pgfpathrectangle{\pgfqpoint{0.800000in}{0.528000in}}{\pgfqpoint{4.960000in}{3.696000in}}%
\pgfusepath{clip}%
\pgfsetbuttcap%
\pgfsetroundjoin%
\definecolor{currentfill}{rgb}{0.000000,0.000000,0.000000}%
\pgfsetfillcolor{currentfill}%
\pgfsetlinewidth{1.003750pt}%
\definecolor{currentstroke}{rgb}{0.000000,0.000000,0.000000}%
\pgfsetstrokecolor{currentstroke}%
\pgfsetdash{}{0pt}%
\pgfpathmoveto{\pgfqpoint{5.504545in}{2.590081in}}%
\pgfpathcurveto{\pgfqpoint{5.515596in}{2.590081in}}{\pgfqpoint{5.526195in}{2.594472in}}{\pgfqpoint{5.534008in}{2.602285in}}%
\pgfpathcurveto{\pgfqpoint{5.541822in}{2.610099in}}{\pgfqpoint{5.546212in}{2.620698in}}{\pgfqpoint{5.546212in}{2.631748in}}%
\pgfpathcurveto{\pgfqpoint{5.546212in}{2.642798in}}{\pgfqpoint{5.541822in}{2.653397in}}{\pgfqpoint{5.534008in}{2.661211in}}%
\pgfpathcurveto{\pgfqpoint{5.526195in}{2.669024in}}{\pgfqpoint{5.515596in}{2.673415in}}{\pgfqpoint{5.504545in}{2.673415in}}%
\pgfpathcurveto{\pgfqpoint{5.493495in}{2.673415in}}{\pgfqpoint{5.482896in}{2.669024in}}{\pgfqpoint{5.475083in}{2.661211in}}%
\pgfpathcurveto{\pgfqpoint{5.467269in}{2.653397in}}{\pgfqpoint{5.462879in}{2.642798in}}{\pgfqpoint{5.462879in}{2.631748in}}%
\pgfpathcurveto{\pgfqpoint{5.462879in}{2.620698in}}{\pgfqpoint{5.467269in}{2.610099in}}{\pgfqpoint{5.475083in}{2.602285in}}%
\pgfpathcurveto{\pgfqpoint{5.482896in}{2.594472in}}{\pgfqpoint{5.493495in}{2.590081in}}{\pgfqpoint{5.504545in}{2.590081in}}%
\pgfpathclose%
\pgfusepath{stroke,fill}%
\end{pgfscope}%
\begin{pgfscope}%
\pgfpathrectangle{\pgfqpoint{0.800000in}{0.528000in}}{\pgfqpoint{4.960000in}{3.696000in}}%
\pgfusepath{clip}%
\pgfsetbuttcap%
\pgfsetroundjoin%
\definecolor{currentfill}{rgb}{0.000000,0.000000,0.000000}%
\pgfsetfillcolor{currentfill}%
\pgfsetlinewidth{1.003750pt}%
\definecolor{currentstroke}{rgb}{0.000000,0.000000,0.000000}%
\pgfsetstrokecolor{currentstroke}%
\pgfsetdash{}{0pt}%
\pgfpathmoveto{\pgfqpoint{5.504545in}{3.084524in}}%
\pgfpathcurveto{\pgfqpoint{5.515596in}{3.084524in}}{\pgfqpoint{5.526195in}{3.088914in}}{\pgfqpoint{5.534008in}{3.096727in}}%
\pgfpathcurveto{\pgfqpoint{5.541822in}{3.104541in}}{\pgfqpoint{5.546212in}{3.115140in}}{\pgfqpoint{5.546212in}{3.126190in}}%
\pgfpathcurveto{\pgfqpoint{5.546212in}{3.137240in}}{\pgfqpoint{5.541822in}{3.147839in}}{\pgfqpoint{5.534008in}{3.155653in}}%
\pgfpathcurveto{\pgfqpoint{5.526195in}{3.163467in}}{\pgfqpoint{5.515596in}{3.167857in}}{\pgfqpoint{5.504545in}{3.167857in}}%
\pgfpathcurveto{\pgfqpoint{5.493495in}{3.167857in}}{\pgfqpoint{5.482896in}{3.163467in}}{\pgfqpoint{5.475083in}{3.155653in}}%
\pgfpathcurveto{\pgfqpoint{5.467269in}{3.147839in}}{\pgfqpoint{5.462879in}{3.137240in}}{\pgfqpoint{5.462879in}{3.126190in}}%
\pgfpathcurveto{\pgfqpoint{5.462879in}{3.115140in}}{\pgfqpoint{5.467269in}{3.104541in}}{\pgfqpoint{5.475083in}{3.096727in}}%
\pgfpathcurveto{\pgfqpoint{5.482896in}{3.088914in}}{\pgfqpoint{5.493495in}{3.084524in}}{\pgfqpoint{5.504545in}{3.084524in}}%
\pgfpathclose%
\pgfusepath{stroke,fill}%
\end{pgfscope}%
\begin{pgfscope}%
\pgfpathrectangle{\pgfqpoint{0.800000in}{0.528000in}}{\pgfqpoint{4.960000in}{3.696000in}}%
\pgfusepath{clip}%
\pgfsetbuttcap%
\pgfsetroundjoin%
\definecolor{currentfill}{rgb}{0.000000,0.000000,0.000000}%
\pgfsetfillcolor{currentfill}%
\pgfsetlinewidth{1.003750pt}%
\definecolor{currentstroke}{rgb}{0.000000,0.000000,0.000000}%
\pgfsetstrokecolor{currentstroke}%
\pgfsetdash{}{0pt}%
\pgfpathmoveto{\pgfqpoint{5.504545in}{3.106021in}}%
\pgfpathcurveto{\pgfqpoint{5.515596in}{3.106021in}}{\pgfqpoint{5.526195in}{3.110411in}}{\pgfqpoint{5.534008in}{3.118225in}}%
\pgfpathcurveto{\pgfqpoint{5.541822in}{3.126039in}}{\pgfqpoint{5.546212in}{3.136638in}}{\pgfqpoint{5.546212in}{3.147688in}}%
\pgfpathcurveto{\pgfqpoint{5.546212in}{3.158738in}}{\pgfqpoint{5.541822in}{3.169337in}}{\pgfqpoint{5.534008in}{3.177151in}}%
\pgfpathcurveto{\pgfqpoint{5.526195in}{3.184964in}}{\pgfqpoint{5.515596in}{3.189354in}}{\pgfqpoint{5.504545in}{3.189354in}}%
\pgfpathcurveto{\pgfqpoint{5.493495in}{3.189354in}}{\pgfqpoint{5.482896in}{3.184964in}}{\pgfqpoint{5.475083in}{3.177151in}}%
\pgfpathcurveto{\pgfqpoint{5.467269in}{3.169337in}}{\pgfqpoint{5.462879in}{3.158738in}}{\pgfqpoint{5.462879in}{3.147688in}}%
\pgfpathcurveto{\pgfqpoint{5.462879in}{3.136638in}}{\pgfqpoint{5.467269in}{3.126039in}}{\pgfqpoint{5.475083in}{3.118225in}}%
\pgfpathcurveto{\pgfqpoint{5.482896in}{3.110411in}}{\pgfqpoint{5.493495in}{3.106021in}}{\pgfqpoint{5.504545in}{3.106021in}}%
\pgfpathclose%
\pgfusepath{stroke,fill}%
\end{pgfscope}%
\begin{pgfscope}%
\pgfpathrectangle{\pgfqpoint{0.800000in}{0.528000in}}{\pgfqpoint{4.960000in}{3.696000in}}%
\pgfusepath{clip}%
\pgfsetbuttcap%
\pgfsetroundjoin%
\definecolor{currentfill}{rgb}{0.000000,0.000000,0.000000}%
\pgfsetfillcolor{currentfill}%
\pgfsetlinewidth{1.003750pt}%
\definecolor{currentstroke}{rgb}{0.000000,0.000000,0.000000}%
\pgfsetstrokecolor{currentstroke}%
\pgfsetdash{}{0pt}%
\pgfpathmoveto{\pgfqpoint{5.504545in}{2.783559in}}%
\pgfpathcurveto{\pgfqpoint{5.515596in}{2.783559in}}{\pgfqpoint{5.526195in}{2.787949in}}{\pgfqpoint{5.534008in}{2.795763in}}%
\pgfpathcurveto{\pgfqpoint{5.541822in}{2.803576in}}{\pgfqpoint{5.546212in}{2.814175in}}{\pgfqpoint{5.546212in}{2.825225in}}%
\pgfpathcurveto{\pgfqpoint{5.546212in}{2.836275in}}{\pgfqpoint{5.541822in}{2.846875in}}{\pgfqpoint{5.534008in}{2.854688in}}%
\pgfpathcurveto{\pgfqpoint{5.526195in}{2.862502in}}{\pgfqpoint{5.515596in}{2.866892in}}{\pgfqpoint{5.504545in}{2.866892in}}%
\pgfpathcurveto{\pgfqpoint{5.493495in}{2.866892in}}{\pgfqpoint{5.482896in}{2.862502in}}{\pgfqpoint{5.475083in}{2.854688in}}%
\pgfpathcurveto{\pgfqpoint{5.467269in}{2.846875in}}{\pgfqpoint{5.462879in}{2.836275in}}{\pgfqpoint{5.462879in}{2.825225in}}%
\pgfpathcurveto{\pgfqpoint{5.462879in}{2.814175in}}{\pgfqpoint{5.467269in}{2.803576in}}{\pgfqpoint{5.475083in}{2.795763in}}%
\pgfpathcurveto{\pgfqpoint{5.482896in}{2.787949in}}{\pgfqpoint{5.493495in}{2.783559in}}{\pgfqpoint{5.504545in}{2.783559in}}%
\pgfpathclose%
\pgfusepath{stroke,fill}%
\end{pgfscope}%
\begin{pgfscope}%
\pgfpathrectangle{\pgfqpoint{0.800000in}{0.528000in}}{\pgfqpoint{4.960000in}{3.696000in}}%
\pgfusepath{clip}%
\pgfsetbuttcap%
\pgfsetroundjoin%
\definecolor{currentfill}{rgb}{0.000000,0.000000,0.000000}%
\pgfsetfillcolor{currentfill}%
\pgfsetlinewidth{1.003750pt}%
\definecolor{currentstroke}{rgb}{0.000000,0.000000,0.000000}%
\pgfsetstrokecolor{currentstroke}%
\pgfsetdash{}{0pt}%
\pgfpathmoveto{\pgfqpoint{5.504545in}{2.762061in}}%
\pgfpathcurveto{\pgfqpoint{5.515596in}{2.762061in}}{\pgfqpoint{5.526195in}{2.766451in}}{\pgfqpoint{5.534008in}{2.774265in}}%
\pgfpathcurveto{\pgfqpoint{5.541822in}{2.782079in}}{\pgfqpoint{5.546212in}{2.792678in}}{\pgfqpoint{5.546212in}{2.803728in}}%
\pgfpathcurveto{\pgfqpoint{5.546212in}{2.814778in}}{\pgfqpoint{5.541822in}{2.825377in}}{\pgfqpoint{5.534008in}{2.833191in}}%
\pgfpathcurveto{\pgfqpoint{5.526195in}{2.841004in}}{\pgfqpoint{5.515596in}{2.845395in}}{\pgfqpoint{5.504545in}{2.845395in}}%
\pgfpathcurveto{\pgfqpoint{5.493495in}{2.845395in}}{\pgfqpoint{5.482896in}{2.841004in}}{\pgfqpoint{5.475083in}{2.833191in}}%
\pgfpathcurveto{\pgfqpoint{5.467269in}{2.825377in}}{\pgfqpoint{5.462879in}{2.814778in}}{\pgfqpoint{5.462879in}{2.803728in}}%
\pgfpathcurveto{\pgfqpoint{5.462879in}{2.792678in}}{\pgfqpoint{5.467269in}{2.782079in}}{\pgfqpoint{5.475083in}{2.774265in}}%
\pgfpathcurveto{\pgfqpoint{5.482896in}{2.766451in}}{\pgfqpoint{5.493495in}{2.762061in}}{\pgfqpoint{5.504545in}{2.762061in}}%
\pgfpathclose%
\pgfusepath{stroke,fill}%
\end{pgfscope}%
\begin{pgfscope}%
\pgfpathrectangle{\pgfqpoint{0.800000in}{0.528000in}}{\pgfqpoint{4.960000in}{3.696000in}}%
\pgfusepath{clip}%
\pgfsetbuttcap%
\pgfsetroundjoin%
\definecolor{currentfill}{rgb}{0.000000,0.000000,0.000000}%
\pgfsetfillcolor{currentfill}%
\pgfsetlinewidth{1.003750pt}%
\definecolor{currentstroke}{rgb}{0.000000,0.000000,0.000000}%
\pgfsetstrokecolor{currentstroke}%
\pgfsetdash{}{0pt}%
\pgfpathmoveto{\pgfqpoint{5.504545in}{2.869549in}}%
\pgfpathcurveto{\pgfqpoint{5.515596in}{2.869549in}}{\pgfqpoint{5.526195in}{2.873939in}}{\pgfqpoint{5.534008in}{2.881753in}}%
\pgfpathcurveto{\pgfqpoint{5.541822in}{2.889566in}}{\pgfqpoint{5.546212in}{2.900165in}}{\pgfqpoint{5.546212in}{2.911215in}}%
\pgfpathcurveto{\pgfqpoint{5.546212in}{2.922265in}}{\pgfqpoint{5.541822in}{2.932864in}}{\pgfqpoint{5.534008in}{2.940678in}}%
\pgfpathcurveto{\pgfqpoint{5.526195in}{2.948492in}}{\pgfqpoint{5.515596in}{2.952882in}}{\pgfqpoint{5.504545in}{2.952882in}}%
\pgfpathcurveto{\pgfqpoint{5.493495in}{2.952882in}}{\pgfqpoint{5.482896in}{2.948492in}}{\pgfqpoint{5.475083in}{2.940678in}}%
\pgfpathcurveto{\pgfqpoint{5.467269in}{2.932864in}}{\pgfqpoint{5.462879in}{2.922265in}}{\pgfqpoint{5.462879in}{2.911215in}}%
\pgfpathcurveto{\pgfqpoint{5.462879in}{2.900165in}}{\pgfqpoint{5.467269in}{2.889566in}}{\pgfqpoint{5.475083in}{2.881753in}}%
\pgfpathcurveto{\pgfqpoint{5.482896in}{2.873939in}}{\pgfqpoint{5.493495in}{2.869549in}}{\pgfqpoint{5.504545in}{2.869549in}}%
\pgfpathclose%
\pgfusepath{stroke,fill}%
\end{pgfscope}%
\begin{pgfscope}%
\pgfpathrectangle{\pgfqpoint{0.800000in}{0.528000in}}{\pgfqpoint{4.960000in}{3.696000in}}%
\pgfusepath{clip}%
\pgfsetbuttcap%
\pgfsetroundjoin%
\definecolor{currentfill}{rgb}{0.000000,0.000000,0.000000}%
\pgfsetfillcolor{currentfill}%
\pgfsetlinewidth{1.003750pt}%
\definecolor{currentstroke}{rgb}{0.000000,0.000000,0.000000}%
\pgfsetstrokecolor{currentstroke}%
\pgfsetdash{}{0pt}%
\pgfpathmoveto{\pgfqpoint{5.504545in}{2.697569in}}%
\pgfpathcurveto{\pgfqpoint{5.515596in}{2.697569in}}{\pgfqpoint{5.526195in}{2.701959in}}{\pgfqpoint{5.534008in}{2.709773in}}%
\pgfpathcurveto{\pgfqpoint{5.541822in}{2.717586in}}{\pgfqpoint{5.546212in}{2.728185in}}{\pgfqpoint{5.546212in}{2.739235in}}%
\pgfpathcurveto{\pgfqpoint{5.546212in}{2.750286in}}{\pgfqpoint{5.541822in}{2.760885in}}{\pgfqpoint{5.534008in}{2.768698in}}%
\pgfpathcurveto{\pgfqpoint{5.526195in}{2.776512in}}{\pgfqpoint{5.515596in}{2.780902in}}{\pgfqpoint{5.504545in}{2.780902in}}%
\pgfpathcurveto{\pgfqpoint{5.493495in}{2.780902in}}{\pgfqpoint{5.482896in}{2.776512in}}{\pgfqpoint{5.475083in}{2.768698in}}%
\pgfpathcurveto{\pgfqpoint{5.467269in}{2.760885in}}{\pgfqpoint{5.462879in}{2.750286in}}{\pgfqpoint{5.462879in}{2.739235in}}%
\pgfpathcurveto{\pgfqpoint{5.462879in}{2.728185in}}{\pgfqpoint{5.467269in}{2.717586in}}{\pgfqpoint{5.475083in}{2.709773in}}%
\pgfpathcurveto{\pgfqpoint{5.482896in}{2.701959in}}{\pgfqpoint{5.493495in}{2.697569in}}{\pgfqpoint{5.504545in}{2.697569in}}%
\pgfpathclose%
\pgfusepath{stroke,fill}%
\end{pgfscope}%
\begin{pgfscope}%
\pgfpathrectangle{\pgfqpoint{0.800000in}{0.528000in}}{\pgfqpoint{4.960000in}{3.696000in}}%
\pgfusepath{clip}%
\pgfsetbuttcap%
\pgfsetroundjoin%
\definecolor{currentfill}{rgb}{0.000000,0.000000,0.000000}%
\pgfsetfillcolor{currentfill}%
\pgfsetlinewidth{1.003750pt}%
\definecolor{currentstroke}{rgb}{0.000000,0.000000,0.000000}%
\pgfsetstrokecolor{currentstroke}%
\pgfsetdash{}{0pt}%
\pgfpathmoveto{\pgfqpoint{5.504545in}{3.428483in}}%
\pgfpathcurveto{\pgfqpoint{5.515596in}{3.428483in}}{\pgfqpoint{5.526195in}{3.432874in}}{\pgfqpoint{5.534008in}{3.440687in}}%
\pgfpathcurveto{\pgfqpoint{5.541822in}{3.448501in}}{\pgfqpoint{5.546212in}{3.459100in}}{\pgfqpoint{5.546212in}{3.470150in}}%
\pgfpathcurveto{\pgfqpoint{5.546212in}{3.481200in}}{\pgfqpoint{5.541822in}{3.491799in}}{\pgfqpoint{5.534008in}{3.499613in}}%
\pgfpathcurveto{\pgfqpoint{5.526195in}{3.507427in}}{\pgfqpoint{5.515596in}{3.511817in}}{\pgfqpoint{5.504545in}{3.511817in}}%
\pgfpathcurveto{\pgfqpoint{5.493495in}{3.511817in}}{\pgfqpoint{5.482896in}{3.507427in}}{\pgfqpoint{5.475083in}{3.499613in}}%
\pgfpathcurveto{\pgfqpoint{5.467269in}{3.491799in}}{\pgfqpoint{5.462879in}{3.481200in}}{\pgfqpoint{5.462879in}{3.470150in}}%
\pgfpathcurveto{\pgfqpoint{5.462879in}{3.459100in}}{\pgfqpoint{5.467269in}{3.448501in}}{\pgfqpoint{5.475083in}{3.440687in}}%
\pgfpathcurveto{\pgfqpoint{5.482896in}{3.432874in}}{\pgfqpoint{5.493495in}{3.428483in}}{\pgfqpoint{5.504545in}{3.428483in}}%
\pgfpathclose%
\pgfusepath{stroke,fill}%
\end{pgfscope}%
\begin{pgfscope}%
\pgfpathrectangle{\pgfqpoint{0.800000in}{0.528000in}}{\pgfqpoint{4.960000in}{3.696000in}}%
\pgfusepath{clip}%
\pgfsetbuttcap%
\pgfsetroundjoin%
\definecolor{currentfill}{rgb}{0.000000,0.000000,0.000000}%
\pgfsetfillcolor{currentfill}%
\pgfsetlinewidth{1.003750pt}%
\definecolor{currentstroke}{rgb}{0.000000,0.000000,0.000000}%
\pgfsetstrokecolor{currentstroke}%
\pgfsetdash{}{0pt}%
\pgfpathmoveto{\pgfqpoint{5.504545in}{2.848051in}}%
\pgfpathcurveto{\pgfqpoint{5.515596in}{2.848051in}}{\pgfqpoint{5.526195in}{2.852441in}}{\pgfqpoint{5.534008in}{2.860255in}}%
\pgfpathcurveto{\pgfqpoint{5.541822in}{2.868069in}}{\pgfqpoint{5.546212in}{2.878668in}}{\pgfqpoint{5.546212in}{2.889718in}}%
\pgfpathcurveto{\pgfqpoint{5.546212in}{2.900768in}}{\pgfqpoint{5.541822in}{2.911367in}}{\pgfqpoint{5.534008in}{2.919181in}}%
\pgfpathcurveto{\pgfqpoint{5.526195in}{2.926994in}}{\pgfqpoint{5.515596in}{2.931385in}}{\pgfqpoint{5.504545in}{2.931385in}}%
\pgfpathcurveto{\pgfqpoint{5.493495in}{2.931385in}}{\pgfqpoint{5.482896in}{2.926994in}}{\pgfqpoint{5.475083in}{2.919181in}}%
\pgfpathcurveto{\pgfqpoint{5.467269in}{2.911367in}}{\pgfqpoint{5.462879in}{2.900768in}}{\pgfqpoint{5.462879in}{2.889718in}}%
\pgfpathcurveto{\pgfqpoint{5.462879in}{2.878668in}}{\pgfqpoint{5.467269in}{2.868069in}}{\pgfqpoint{5.475083in}{2.860255in}}%
\pgfpathcurveto{\pgfqpoint{5.482896in}{2.852441in}}{\pgfqpoint{5.493495in}{2.848051in}}{\pgfqpoint{5.504545in}{2.848051in}}%
\pgfpathclose%
\pgfusepath{stroke,fill}%
\end{pgfscope}%
\begin{pgfscope}%
\pgfpathrectangle{\pgfqpoint{0.800000in}{0.528000in}}{\pgfqpoint{4.960000in}{3.696000in}}%
\pgfusepath{clip}%
\pgfsetbuttcap%
\pgfsetroundjoin%
\definecolor{currentfill}{rgb}{0.000000,0.000000,0.000000}%
\pgfsetfillcolor{currentfill}%
\pgfsetlinewidth{1.003750pt}%
\definecolor{currentstroke}{rgb}{0.000000,0.000000,0.000000}%
\pgfsetstrokecolor{currentstroke}%
\pgfsetdash{}{0pt}%
\pgfpathmoveto{\pgfqpoint{5.504545in}{3.278001in}}%
\pgfpathcurveto{\pgfqpoint{5.515596in}{3.278001in}}{\pgfqpoint{5.526195in}{3.282391in}}{\pgfqpoint{5.534008in}{3.290205in}}%
\pgfpathcurveto{\pgfqpoint{5.541822in}{3.298019in}}{\pgfqpoint{5.546212in}{3.308618in}}{\pgfqpoint{5.546212in}{3.319668in}}%
\pgfpathcurveto{\pgfqpoint{5.546212in}{3.330718in}}{\pgfqpoint{5.541822in}{3.341317in}}{\pgfqpoint{5.534008in}{3.349130in}}%
\pgfpathcurveto{\pgfqpoint{5.526195in}{3.356944in}}{\pgfqpoint{5.515596in}{3.361334in}}{\pgfqpoint{5.504545in}{3.361334in}}%
\pgfpathcurveto{\pgfqpoint{5.493495in}{3.361334in}}{\pgfqpoint{5.482896in}{3.356944in}}{\pgfqpoint{5.475083in}{3.349130in}}%
\pgfpathcurveto{\pgfqpoint{5.467269in}{3.341317in}}{\pgfqpoint{5.462879in}{3.330718in}}{\pgfqpoint{5.462879in}{3.319668in}}%
\pgfpathcurveto{\pgfqpoint{5.462879in}{3.308618in}}{\pgfqpoint{5.467269in}{3.298019in}}{\pgfqpoint{5.475083in}{3.290205in}}%
\pgfpathcurveto{\pgfqpoint{5.482896in}{3.282391in}}{\pgfqpoint{5.493495in}{3.278001in}}{\pgfqpoint{5.504545in}{3.278001in}}%
\pgfpathclose%
\pgfusepath{stroke,fill}%
\end{pgfscope}%
\begin{pgfscope}%
\pgfpathrectangle{\pgfqpoint{0.800000in}{0.528000in}}{\pgfqpoint{4.960000in}{3.696000in}}%
\pgfusepath{clip}%
\pgfsetbuttcap%
\pgfsetroundjoin%
\definecolor{currentfill}{rgb}{0.000000,0.000000,0.000000}%
\pgfsetfillcolor{currentfill}%
\pgfsetlinewidth{1.003750pt}%
\definecolor{currentstroke}{rgb}{0.000000,0.000000,0.000000}%
\pgfsetstrokecolor{currentstroke}%
\pgfsetdash{}{0pt}%
\pgfpathmoveto{\pgfqpoint{5.504545in}{2.891046in}}%
\pgfpathcurveto{\pgfqpoint{5.515596in}{2.891046in}}{\pgfqpoint{5.526195in}{2.895436in}}{\pgfqpoint{5.534008in}{2.903250in}}%
\pgfpathcurveto{\pgfqpoint{5.541822in}{2.911064in}}{\pgfqpoint{5.546212in}{2.921663in}}{\pgfqpoint{5.546212in}{2.932713in}}%
\pgfpathcurveto{\pgfqpoint{5.546212in}{2.943763in}}{\pgfqpoint{5.541822in}{2.954362in}}{\pgfqpoint{5.534008in}{2.962176in}}%
\pgfpathcurveto{\pgfqpoint{5.526195in}{2.969989in}}{\pgfqpoint{5.515596in}{2.974379in}}{\pgfqpoint{5.504545in}{2.974379in}}%
\pgfpathcurveto{\pgfqpoint{5.493495in}{2.974379in}}{\pgfqpoint{5.482896in}{2.969989in}}{\pgfqpoint{5.475083in}{2.962176in}}%
\pgfpathcurveto{\pgfqpoint{5.467269in}{2.954362in}}{\pgfqpoint{5.462879in}{2.943763in}}{\pgfqpoint{5.462879in}{2.932713in}}%
\pgfpathcurveto{\pgfqpoint{5.462879in}{2.921663in}}{\pgfqpoint{5.467269in}{2.911064in}}{\pgfqpoint{5.475083in}{2.903250in}}%
\pgfpathcurveto{\pgfqpoint{5.482896in}{2.895436in}}{\pgfqpoint{5.493495in}{2.891046in}}{\pgfqpoint{5.504545in}{2.891046in}}%
\pgfpathclose%
\pgfusepath{stroke,fill}%
\end{pgfscope}%
\begin{pgfscope}%
\pgfpathrectangle{\pgfqpoint{0.800000in}{0.528000in}}{\pgfqpoint{4.960000in}{3.696000in}}%
\pgfusepath{clip}%
\pgfsetbuttcap%
\pgfsetroundjoin%
\definecolor{currentfill}{rgb}{0.000000,0.000000,0.000000}%
\pgfsetfillcolor{currentfill}%
\pgfsetlinewidth{1.003750pt}%
\definecolor{currentstroke}{rgb}{0.000000,0.000000,0.000000}%
\pgfsetstrokecolor{currentstroke}%
\pgfsetdash{}{0pt}%
\pgfpathmoveto{\pgfqpoint{5.504545in}{2.955539in}}%
\pgfpathcurveto{\pgfqpoint{5.515596in}{2.955539in}}{\pgfqpoint{5.526195in}{2.959929in}}{\pgfqpoint{5.534008in}{2.967743in}}%
\pgfpathcurveto{\pgfqpoint{5.541822in}{2.975556in}}{\pgfqpoint{5.546212in}{2.986155in}}{\pgfqpoint{5.546212in}{2.997205in}}%
\pgfpathcurveto{\pgfqpoint{5.546212in}{3.008255in}}{\pgfqpoint{5.541822in}{3.018854in}}{\pgfqpoint{5.534008in}{3.026668in}}%
\pgfpathcurveto{\pgfqpoint{5.526195in}{3.034482in}}{\pgfqpoint{5.515596in}{3.038872in}}{\pgfqpoint{5.504545in}{3.038872in}}%
\pgfpathcurveto{\pgfqpoint{5.493495in}{3.038872in}}{\pgfqpoint{5.482896in}{3.034482in}}{\pgfqpoint{5.475083in}{3.026668in}}%
\pgfpathcurveto{\pgfqpoint{5.467269in}{3.018854in}}{\pgfqpoint{5.462879in}{3.008255in}}{\pgfqpoint{5.462879in}{2.997205in}}%
\pgfpathcurveto{\pgfqpoint{5.462879in}{2.986155in}}{\pgfqpoint{5.467269in}{2.975556in}}{\pgfqpoint{5.475083in}{2.967743in}}%
\pgfpathcurveto{\pgfqpoint{5.482896in}{2.959929in}}{\pgfqpoint{5.493495in}{2.955539in}}{\pgfqpoint{5.504545in}{2.955539in}}%
\pgfpathclose%
\pgfusepath{stroke,fill}%
\end{pgfscope}%
\begin{pgfscope}%
\pgfpathrectangle{\pgfqpoint{0.800000in}{0.528000in}}{\pgfqpoint{4.960000in}{3.696000in}}%
\pgfusepath{clip}%
\pgfsetbuttcap%
\pgfsetroundjoin%
\definecolor{currentfill}{rgb}{0.000000,0.000000,0.000000}%
\pgfsetfillcolor{currentfill}%
\pgfsetlinewidth{1.003750pt}%
\definecolor{currentstroke}{rgb}{0.000000,0.000000,0.000000}%
\pgfsetstrokecolor{currentstroke}%
\pgfsetdash{}{0pt}%
\pgfpathmoveto{\pgfqpoint{5.504545in}{3.299499in}}%
\pgfpathcurveto{\pgfqpoint{5.515596in}{3.299499in}}{\pgfqpoint{5.526195in}{3.303889in}}{\pgfqpoint{5.534008in}{3.311702in}}%
\pgfpathcurveto{\pgfqpoint{5.541822in}{3.319516in}}{\pgfqpoint{5.546212in}{3.330115in}}{\pgfqpoint{5.546212in}{3.341165in}}%
\pgfpathcurveto{\pgfqpoint{5.546212in}{3.352215in}}{\pgfqpoint{5.541822in}{3.362814in}}{\pgfqpoint{5.534008in}{3.370628in}}%
\pgfpathcurveto{\pgfqpoint{5.526195in}{3.378442in}}{\pgfqpoint{5.515596in}{3.382832in}}{\pgfqpoint{5.504545in}{3.382832in}}%
\pgfpathcurveto{\pgfqpoint{5.493495in}{3.382832in}}{\pgfqpoint{5.482896in}{3.378442in}}{\pgfqpoint{5.475083in}{3.370628in}}%
\pgfpathcurveto{\pgfqpoint{5.467269in}{3.362814in}}{\pgfqpoint{5.462879in}{3.352215in}}{\pgfqpoint{5.462879in}{3.341165in}}%
\pgfpathcurveto{\pgfqpoint{5.462879in}{3.330115in}}{\pgfqpoint{5.467269in}{3.319516in}}{\pgfqpoint{5.475083in}{3.311702in}}%
\pgfpathcurveto{\pgfqpoint{5.482896in}{3.303889in}}{\pgfqpoint{5.493495in}{3.299499in}}{\pgfqpoint{5.504545in}{3.299499in}}%
\pgfpathclose%
\pgfusepath{stroke,fill}%
\end{pgfscope}%
\begin{pgfscope}%
\pgfpathrectangle{\pgfqpoint{0.800000in}{0.528000in}}{\pgfqpoint{4.960000in}{3.696000in}}%
\pgfusepath{clip}%
\pgfsetbuttcap%
\pgfsetroundjoin%
\definecolor{currentfill}{rgb}{0.000000,0.000000,0.000000}%
\pgfsetfillcolor{currentfill}%
\pgfsetlinewidth{1.003750pt}%
\definecolor{currentstroke}{rgb}{0.000000,0.000000,0.000000}%
\pgfsetstrokecolor{currentstroke}%
\pgfsetdash{}{0pt}%
\pgfpathmoveto{\pgfqpoint{5.504545in}{3.772443in}}%
\pgfpathcurveto{\pgfqpoint{5.515596in}{3.772443in}}{\pgfqpoint{5.526195in}{3.776834in}}{\pgfqpoint{5.534008in}{3.784647in}}%
\pgfpathcurveto{\pgfqpoint{5.541822in}{3.792461in}}{\pgfqpoint{5.546212in}{3.803060in}}{\pgfqpoint{5.546212in}{3.814110in}}%
\pgfpathcurveto{\pgfqpoint{5.546212in}{3.825160in}}{\pgfqpoint{5.541822in}{3.835759in}}{\pgfqpoint{5.534008in}{3.843573in}}%
\pgfpathcurveto{\pgfqpoint{5.526195in}{3.851386in}}{\pgfqpoint{5.515596in}{3.855777in}}{\pgfqpoint{5.504545in}{3.855777in}}%
\pgfpathcurveto{\pgfqpoint{5.493495in}{3.855777in}}{\pgfqpoint{5.482896in}{3.851386in}}{\pgfqpoint{5.475083in}{3.843573in}}%
\pgfpathcurveto{\pgfqpoint{5.467269in}{3.835759in}}{\pgfqpoint{5.462879in}{3.825160in}}{\pgfqpoint{5.462879in}{3.814110in}}%
\pgfpathcurveto{\pgfqpoint{5.462879in}{3.803060in}}{\pgfqpoint{5.467269in}{3.792461in}}{\pgfqpoint{5.475083in}{3.784647in}}%
\pgfpathcurveto{\pgfqpoint{5.482896in}{3.776834in}}{\pgfqpoint{5.493495in}{3.772443in}}{\pgfqpoint{5.504545in}{3.772443in}}%
\pgfpathclose%
\pgfusepath{stroke,fill}%
\end{pgfscope}%
\begin{pgfscope}%
\pgfpathrectangle{\pgfqpoint{0.800000in}{0.528000in}}{\pgfqpoint{4.960000in}{3.696000in}}%
\pgfusepath{clip}%
\pgfsetbuttcap%
\pgfsetroundjoin%
\definecolor{currentfill}{rgb}{0.000000,0.000000,0.000000}%
\pgfsetfillcolor{currentfill}%
\pgfsetlinewidth{1.003750pt}%
\definecolor{currentstroke}{rgb}{0.000000,0.000000,0.000000}%
\pgfsetstrokecolor{currentstroke}%
\pgfsetdash{}{0pt}%
\pgfpathmoveto{\pgfqpoint{5.504545in}{2.719066in}}%
\pgfpathcurveto{\pgfqpoint{5.515596in}{2.719066in}}{\pgfqpoint{5.526195in}{2.723456in}}{\pgfqpoint{5.534008in}{2.731270in}}%
\pgfpathcurveto{\pgfqpoint{5.541822in}{2.739084in}}{\pgfqpoint{5.546212in}{2.749683in}}{\pgfqpoint{5.546212in}{2.760733in}}%
\pgfpathcurveto{\pgfqpoint{5.546212in}{2.771783in}}{\pgfqpoint{5.541822in}{2.782382in}}{\pgfqpoint{5.534008in}{2.790196in}}%
\pgfpathcurveto{\pgfqpoint{5.526195in}{2.798009in}}{\pgfqpoint{5.515596in}{2.802400in}}{\pgfqpoint{5.504545in}{2.802400in}}%
\pgfpathcurveto{\pgfqpoint{5.493495in}{2.802400in}}{\pgfqpoint{5.482896in}{2.798009in}}{\pgfqpoint{5.475083in}{2.790196in}}%
\pgfpathcurveto{\pgfqpoint{5.467269in}{2.782382in}}{\pgfqpoint{5.462879in}{2.771783in}}{\pgfqpoint{5.462879in}{2.760733in}}%
\pgfpathcurveto{\pgfqpoint{5.462879in}{2.749683in}}{\pgfqpoint{5.467269in}{2.739084in}}{\pgfqpoint{5.475083in}{2.731270in}}%
\pgfpathcurveto{\pgfqpoint{5.482896in}{2.723456in}}{\pgfqpoint{5.493495in}{2.719066in}}{\pgfqpoint{5.504545in}{2.719066in}}%
\pgfpathclose%
\pgfusepath{stroke,fill}%
\end{pgfscope}%
\begin{pgfscope}%
\pgfpathrectangle{\pgfqpoint{0.800000in}{0.528000in}}{\pgfqpoint{4.960000in}{3.696000in}}%
\pgfusepath{clip}%
\pgfsetbuttcap%
\pgfsetroundjoin%
\definecolor{currentfill}{rgb}{0.000000,0.000000,0.000000}%
\pgfsetfillcolor{currentfill}%
\pgfsetlinewidth{1.003750pt}%
\definecolor{currentstroke}{rgb}{0.000000,0.000000,0.000000}%
\pgfsetstrokecolor{currentstroke}%
\pgfsetdash{}{0pt}%
\pgfpathmoveto{\pgfqpoint{5.504545in}{2.719066in}}%
\pgfpathcurveto{\pgfqpoint{5.515596in}{2.719066in}}{\pgfqpoint{5.526195in}{2.723456in}}{\pgfqpoint{5.534008in}{2.731270in}}%
\pgfpathcurveto{\pgfqpoint{5.541822in}{2.739084in}}{\pgfqpoint{5.546212in}{2.749683in}}{\pgfqpoint{5.546212in}{2.760733in}}%
\pgfpathcurveto{\pgfqpoint{5.546212in}{2.771783in}}{\pgfqpoint{5.541822in}{2.782382in}}{\pgfqpoint{5.534008in}{2.790196in}}%
\pgfpathcurveto{\pgfqpoint{5.526195in}{2.798009in}}{\pgfqpoint{5.515596in}{2.802400in}}{\pgfqpoint{5.504545in}{2.802400in}}%
\pgfpathcurveto{\pgfqpoint{5.493495in}{2.802400in}}{\pgfqpoint{5.482896in}{2.798009in}}{\pgfqpoint{5.475083in}{2.790196in}}%
\pgfpathcurveto{\pgfqpoint{5.467269in}{2.782382in}}{\pgfqpoint{5.462879in}{2.771783in}}{\pgfqpoint{5.462879in}{2.760733in}}%
\pgfpathcurveto{\pgfqpoint{5.462879in}{2.749683in}}{\pgfqpoint{5.467269in}{2.739084in}}{\pgfqpoint{5.475083in}{2.731270in}}%
\pgfpathcurveto{\pgfqpoint{5.482896in}{2.723456in}}{\pgfqpoint{5.493495in}{2.719066in}}{\pgfqpoint{5.504545in}{2.719066in}}%
\pgfpathclose%
\pgfusepath{stroke,fill}%
\end{pgfscope}%
\begin{pgfscope}%
\pgfpathrectangle{\pgfqpoint{0.800000in}{0.528000in}}{\pgfqpoint{4.960000in}{3.696000in}}%
\pgfusepath{clip}%
\pgfsetbuttcap%
\pgfsetroundjoin%
\definecolor{currentfill}{rgb}{0.000000,0.000000,0.000000}%
\pgfsetfillcolor{currentfill}%
\pgfsetlinewidth{1.003750pt}%
\definecolor{currentstroke}{rgb}{0.000000,0.000000,0.000000}%
\pgfsetstrokecolor{currentstroke}%
\pgfsetdash{}{0pt}%
\pgfpathmoveto{\pgfqpoint{5.504545in}{3.063026in}}%
\pgfpathcurveto{\pgfqpoint{5.515596in}{3.063026in}}{\pgfqpoint{5.526195in}{3.067416in}}{\pgfqpoint{5.534008in}{3.075230in}}%
\pgfpathcurveto{\pgfqpoint{5.541822in}{3.083044in}}{\pgfqpoint{5.546212in}{3.093643in}}{\pgfqpoint{5.546212in}{3.104693in}}%
\pgfpathcurveto{\pgfqpoint{5.546212in}{3.115743in}}{\pgfqpoint{5.541822in}{3.126342in}}{\pgfqpoint{5.534008in}{3.134156in}}%
\pgfpathcurveto{\pgfqpoint{5.526195in}{3.141969in}}{\pgfqpoint{5.515596in}{3.146359in}}{\pgfqpoint{5.504545in}{3.146359in}}%
\pgfpathcurveto{\pgfqpoint{5.493495in}{3.146359in}}{\pgfqpoint{5.482896in}{3.141969in}}{\pgfqpoint{5.475083in}{3.134156in}}%
\pgfpathcurveto{\pgfqpoint{5.467269in}{3.126342in}}{\pgfqpoint{5.462879in}{3.115743in}}{\pgfqpoint{5.462879in}{3.104693in}}%
\pgfpathcurveto{\pgfqpoint{5.462879in}{3.093643in}}{\pgfqpoint{5.467269in}{3.083044in}}{\pgfqpoint{5.475083in}{3.075230in}}%
\pgfpathcurveto{\pgfqpoint{5.482896in}{3.067416in}}{\pgfqpoint{5.493495in}{3.063026in}}{\pgfqpoint{5.504545in}{3.063026in}}%
\pgfpathclose%
\pgfusepath{stroke,fill}%
\end{pgfscope}%
\begin{pgfscope}%
\pgfpathrectangle{\pgfqpoint{0.800000in}{0.528000in}}{\pgfqpoint{4.960000in}{3.696000in}}%
\pgfusepath{clip}%
\pgfsetbuttcap%
\pgfsetroundjoin%
\definecolor{currentfill}{rgb}{0.000000,0.000000,0.000000}%
\pgfsetfillcolor{currentfill}%
\pgfsetlinewidth{1.003750pt}%
\definecolor{currentstroke}{rgb}{0.000000,0.000000,0.000000}%
\pgfsetstrokecolor{currentstroke}%
\pgfsetdash{}{0pt}%
\pgfpathmoveto{\pgfqpoint{5.504545in}{2.762061in}}%
\pgfpathcurveto{\pgfqpoint{5.515596in}{2.762061in}}{\pgfqpoint{5.526195in}{2.766451in}}{\pgfqpoint{5.534008in}{2.774265in}}%
\pgfpathcurveto{\pgfqpoint{5.541822in}{2.782079in}}{\pgfqpoint{5.546212in}{2.792678in}}{\pgfqpoint{5.546212in}{2.803728in}}%
\pgfpathcurveto{\pgfqpoint{5.546212in}{2.814778in}}{\pgfqpoint{5.541822in}{2.825377in}}{\pgfqpoint{5.534008in}{2.833191in}}%
\pgfpathcurveto{\pgfqpoint{5.526195in}{2.841004in}}{\pgfqpoint{5.515596in}{2.845395in}}{\pgfqpoint{5.504545in}{2.845395in}}%
\pgfpathcurveto{\pgfqpoint{5.493495in}{2.845395in}}{\pgfqpoint{5.482896in}{2.841004in}}{\pgfqpoint{5.475083in}{2.833191in}}%
\pgfpathcurveto{\pgfqpoint{5.467269in}{2.825377in}}{\pgfqpoint{5.462879in}{2.814778in}}{\pgfqpoint{5.462879in}{2.803728in}}%
\pgfpathcurveto{\pgfqpoint{5.462879in}{2.792678in}}{\pgfqpoint{5.467269in}{2.782079in}}{\pgfqpoint{5.475083in}{2.774265in}}%
\pgfpathcurveto{\pgfqpoint{5.482896in}{2.766451in}}{\pgfqpoint{5.493495in}{2.762061in}}{\pgfqpoint{5.504545in}{2.762061in}}%
\pgfpathclose%
\pgfusepath{stroke,fill}%
\end{pgfscope}%
\begin{pgfscope}%
\pgfpathrectangle{\pgfqpoint{0.800000in}{0.528000in}}{\pgfqpoint{4.960000in}{3.696000in}}%
\pgfusepath{clip}%
\pgfsetbuttcap%
\pgfsetroundjoin%
\definecolor{currentfill}{rgb}{0.000000,0.000000,0.000000}%
\pgfsetfillcolor{currentfill}%
\pgfsetlinewidth{1.003750pt}%
\definecolor{currentstroke}{rgb}{0.000000,0.000000,0.000000}%
\pgfsetstrokecolor{currentstroke}%
\pgfsetdash{}{0pt}%
\pgfpathmoveto{\pgfqpoint{5.504545in}{2.740564in}}%
\pgfpathcurveto{\pgfqpoint{5.515596in}{2.740564in}}{\pgfqpoint{5.526195in}{2.744954in}}{\pgfqpoint{5.534008in}{2.752768in}}%
\pgfpathcurveto{\pgfqpoint{5.541822in}{2.760581in}}{\pgfqpoint{5.546212in}{2.771180in}}{\pgfqpoint{5.546212in}{2.782230in}}%
\pgfpathcurveto{\pgfqpoint{5.546212in}{2.793281in}}{\pgfqpoint{5.541822in}{2.803880in}}{\pgfqpoint{5.534008in}{2.811693in}}%
\pgfpathcurveto{\pgfqpoint{5.526195in}{2.819507in}}{\pgfqpoint{5.515596in}{2.823897in}}{\pgfqpoint{5.504545in}{2.823897in}}%
\pgfpathcurveto{\pgfqpoint{5.493495in}{2.823897in}}{\pgfqpoint{5.482896in}{2.819507in}}{\pgfqpoint{5.475083in}{2.811693in}}%
\pgfpathcurveto{\pgfqpoint{5.467269in}{2.803880in}}{\pgfqpoint{5.462879in}{2.793281in}}{\pgfqpoint{5.462879in}{2.782230in}}%
\pgfpathcurveto{\pgfqpoint{5.462879in}{2.771180in}}{\pgfqpoint{5.467269in}{2.760581in}}{\pgfqpoint{5.475083in}{2.752768in}}%
\pgfpathcurveto{\pgfqpoint{5.482896in}{2.744954in}}{\pgfqpoint{5.493495in}{2.740564in}}{\pgfqpoint{5.504545in}{2.740564in}}%
\pgfpathclose%
\pgfusepath{stroke,fill}%
\end{pgfscope}%
\begin{pgfscope}%
\pgfpathrectangle{\pgfqpoint{0.800000in}{0.528000in}}{\pgfqpoint{4.960000in}{3.696000in}}%
\pgfusepath{clip}%
\pgfsetbuttcap%
\pgfsetroundjoin%
\definecolor{currentfill}{rgb}{0.000000,0.000000,0.000000}%
\pgfsetfillcolor{currentfill}%
\pgfsetlinewidth{1.003750pt}%
\definecolor{currentstroke}{rgb}{0.000000,0.000000,0.000000}%
\pgfsetstrokecolor{currentstroke}%
\pgfsetdash{}{0pt}%
\pgfpathmoveto{\pgfqpoint{5.504545in}{3.192011in}}%
\pgfpathcurveto{\pgfqpoint{5.515596in}{3.192011in}}{\pgfqpoint{5.526195in}{3.196401in}}{\pgfqpoint{5.534008in}{3.204215in}}%
\pgfpathcurveto{\pgfqpoint{5.541822in}{3.212029in}}{\pgfqpoint{5.546212in}{3.222628in}}{\pgfqpoint{5.546212in}{3.233678in}}%
\pgfpathcurveto{\pgfqpoint{5.546212in}{3.244728in}}{\pgfqpoint{5.541822in}{3.255327in}}{\pgfqpoint{5.534008in}{3.263141in}}%
\pgfpathcurveto{\pgfqpoint{5.526195in}{3.270954in}}{\pgfqpoint{5.515596in}{3.275344in}}{\pgfqpoint{5.504545in}{3.275344in}}%
\pgfpathcurveto{\pgfqpoint{5.493495in}{3.275344in}}{\pgfqpoint{5.482896in}{3.270954in}}{\pgfqpoint{5.475083in}{3.263141in}}%
\pgfpathcurveto{\pgfqpoint{5.467269in}{3.255327in}}{\pgfqpoint{5.462879in}{3.244728in}}{\pgfqpoint{5.462879in}{3.233678in}}%
\pgfpathcurveto{\pgfqpoint{5.462879in}{3.222628in}}{\pgfqpoint{5.467269in}{3.212029in}}{\pgfqpoint{5.475083in}{3.204215in}}%
\pgfpathcurveto{\pgfqpoint{5.482896in}{3.196401in}}{\pgfqpoint{5.493495in}{3.192011in}}{\pgfqpoint{5.504545in}{3.192011in}}%
\pgfpathclose%
\pgfusepath{stroke,fill}%
\end{pgfscope}%
\begin{pgfscope}%
\pgfpathrectangle{\pgfqpoint{0.800000in}{0.528000in}}{\pgfqpoint{4.960000in}{3.696000in}}%
\pgfusepath{clip}%
\pgfsetbuttcap%
\pgfsetroundjoin%
\definecolor{currentfill}{rgb}{0.000000,0.000000,0.000000}%
\pgfsetfillcolor{currentfill}%
\pgfsetlinewidth{1.003750pt}%
\definecolor{currentstroke}{rgb}{0.000000,0.000000,0.000000}%
\pgfsetstrokecolor{currentstroke}%
\pgfsetdash{}{0pt}%
\pgfpathmoveto{\pgfqpoint{5.504545in}{2.805056in}}%
\pgfpathcurveto{\pgfqpoint{5.515596in}{2.805056in}}{\pgfqpoint{5.526195in}{2.809446in}}{\pgfqpoint{5.534008in}{2.817260in}}%
\pgfpathcurveto{\pgfqpoint{5.541822in}{2.825074in}}{\pgfqpoint{5.546212in}{2.835673in}}{\pgfqpoint{5.546212in}{2.846723in}}%
\pgfpathcurveto{\pgfqpoint{5.546212in}{2.857773in}}{\pgfqpoint{5.541822in}{2.868372in}}{\pgfqpoint{5.534008in}{2.876186in}}%
\pgfpathcurveto{\pgfqpoint{5.526195in}{2.883999in}}{\pgfqpoint{5.515596in}{2.888390in}}{\pgfqpoint{5.504545in}{2.888390in}}%
\pgfpathcurveto{\pgfqpoint{5.493495in}{2.888390in}}{\pgfqpoint{5.482896in}{2.883999in}}{\pgfqpoint{5.475083in}{2.876186in}}%
\pgfpathcurveto{\pgfqpoint{5.467269in}{2.868372in}}{\pgfqpoint{5.462879in}{2.857773in}}{\pgfqpoint{5.462879in}{2.846723in}}%
\pgfpathcurveto{\pgfqpoint{5.462879in}{2.835673in}}{\pgfqpoint{5.467269in}{2.825074in}}{\pgfqpoint{5.475083in}{2.817260in}}%
\pgfpathcurveto{\pgfqpoint{5.482896in}{2.809446in}}{\pgfqpoint{5.493495in}{2.805056in}}{\pgfqpoint{5.504545in}{2.805056in}}%
\pgfpathclose%
\pgfusepath{stroke,fill}%
\end{pgfscope}%
\begin{pgfscope}%
\pgfpathrectangle{\pgfqpoint{0.800000in}{0.528000in}}{\pgfqpoint{4.960000in}{3.696000in}}%
\pgfusepath{clip}%
\pgfsetbuttcap%
\pgfsetroundjoin%
\definecolor{currentfill}{rgb}{0.000000,0.000000,0.000000}%
\pgfsetfillcolor{currentfill}%
\pgfsetlinewidth{1.003750pt}%
\definecolor{currentstroke}{rgb}{0.000000,0.000000,0.000000}%
\pgfsetstrokecolor{currentstroke}%
\pgfsetdash{}{0pt}%
\pgfpathmoveto{\pgfqpoint{5.504545in}{3.664956in}}%
\pgfpathcurveto{\pgfqpoint{5.515596in}{3.664956in}}{\pgfqpoint{5.526195in}{3.669346in}}{\pgfqpoint{5.534008in}{3.677160in}}%
\pgfpathcurveto{\pgfqpoint{5.541822in}{3.684973in}}{\pgfqpoint{5.546212in}{3.695572in}}{\pgfqpoint{5.546212in}{3.706623in}}%
\pgfpathcurveto{\pgfqpoint{5.546212in}{3.717673in}}{\pgfqpoint{5.541822in}{3.728272in}}{\pgfqpoint{5.534008in}{3.736085in}}%
\pgfpathcurveto{\pgfqpoint{5.526195in}{3.743899in}}{\pgfqpoint{5.515596in}{3.748289in}}{\pgfqpoint{5.504545in}{3.748289in}}%
\pgfpathcurveto{\pgfqpoint{5.493495in}{3.748289in}}{\pgfqpoint{5.482896in}{3.743899in}}{\pgfqpoint{5.475083in}{3.736085in}}%
\pgfpathcurveto{\pgfqpoint{5.467269in}{3.728272in}}{\pgfqpoint{5.462879in}{3.717673in}}{\pgfqpoint{5.462879in}{3.706623in}}%
\pgfpathcurveto{\pgfqpoint{5.462879in}{3.695572in}}{\pgfqpoint{5.467269in}{3.684973in}}{\pgfqpoint{5.475083in}{3.677160in}}%
\pgfpathcurveto{\pgfqpoint{5.482896in}{3.669346in}}{\pgfqpoint{5.493495in}{3.664956in}}{\pgfqpoint{5.504545in}{3.664956in}}%
\pgfpathclose%
\pgfusepath{stroke,fill}%
\end{pgfscope}%
\begin{pgfscope}%
\pgfsetbuttcap%
\pgfsetroundjoin%
\definecolor{currentfill}{rgb}{0.000000,0.000000,0.000000}%
\pgfsetfillcolor{currentfill}%
\pgfsetlinewidth{0.803000pt}%
\definecolor{currentstroke}{rgb}{0.000000,0.000000,0.000000}%
\pgfsetstrokecolor{currentstroke}%
\pgfsetdash{}{0pt}%
\pgfsys@defobject{currentmarker}{\pgfqpoint{0.000000in}{-0.048611in}}{\pgfqpoint{0.000000in}{0.000000in}}{%
\pgfpathmoveto{\pgfqpoint{0.000000in}{0.000000in}}%
\pgfpathlineto{\pgfqpoint{0.000000in}{-0.048611in}}%
\pgfusepath{stroke,fill}%
}%
\begin{pgfscope}%
\pgfsys@transformshift{1.025906in}{0.528000in}%
\pgfsys@useobject{currentmarker}{}%
\end{pgfscope}%
\end{pgfscope}%
\begin{pgfscope}%
\definecolor{textcolor}{rgb}{0.000000,0.000000,0.000000}%
\pgfsetstrokecolor{textcolor}%
\pgfsetfillcolor{textcolor}%
\pgftext[x=1.025906in,y=0.430778in,,top]{\color{textcolor}\sffamily\fontsize{10.000000}{12.000000}\selectfont 20}%
\end{pgfscope}%
\begin{pgfscope}%
\pgfsetbuttcap%
\pgfsetroundjoin%
\definecolor{currentfill}{rgb}{0.000000,0.000000,0.000000}%
\pgfsetfillcolor{currentfill}%
\pgfsetlinewidth{0.803000pt}%
\definecolor{currentstroke}{rgb}{0.000000,0.000000,0.000000}%
\pgfsetstrokecolor{currentstroke}%
\pgfsetdash{}{0pt}%
\pgfsys@defobject{currentmarker}{\pgfqpoint{0.000000in}{-0.048611in}}{\pgfqpoint{0.000000in}{0.000000in}}{%
\pgfpathmoveto{\pgfqpoint{0.000000in}{0.000000in}}%
\pgfpathlineto{\pgfqpoint{0.000000in}{-0.048611in}}%
\pgfusepath{stroke,fill}%
}%
\begin{pgfscope}%
\pgfsys@transformshift{2.518786in}{0.528000in}%
\pgfsys@useobject{currentmarker}{}%
\end{pgfscope}%
\end{pgfscope}%
\begin{pgfscope}%
\definecolor{textcolor}{rgb}{0.000000,0.000000,0.000000}%
\pgfsetstrokecolor{textcolor}%
\pgfsetfillcolor{textcolor}%
\pgftext[x=2.518786in,y=0.430778in,,top]{\color{textcolor}\sffamily\fontsize{10.000000}{12.000000}\selectfont 40}%
\end{pgfscope}%
\begin{pgfscope}%
\pgfsetbuttcap%
\pgfsetroundjoin%
\definecolor{currentfill}{rgb}{0.000000,0.000000,0.000000}%
\pgfsetfillcolor{currentfill}%
\pgfsetlinewidth{0.803000pt}%
\definecolor{currentstroke}{rgb}{0.000000,0.000000,0.000000}%
\pgfsetstrokecolor{currentstroke}%
\pgfsetdash{}{0pt}%
\pgfsys@defobject{currentmarker}{\pgfqpoint{0.000000in}{-0.048611in}}{\pgfqpoint{0.000000in}{0.000000in}}{%
\pgfpathmoveto{\pgfqpoint{0.000000in}{0.000000in}}%
\pgfpathlineto{\pgfqpoint{0.000000in}{-0.048611in}}%
\pgfusepath{stroke,fill}%
}%
\begin{pgfscope}%
\pgfsys@transformshift{4.011666in}{0.528000in}%
\pgfsys@useobject{currentmarker}{}%
\end{pgfscope}%
\end{pgfscope}%
\begin{pgfscope}%
\definecolor{textcolor}{rgb}{0.000000,0.000000,0.000000}%
\pgfsetstrokecolor{textcolor}%
\pgfsetfillcolor{textcolor}%
\pgftext[x=4.011666in,y=0.430778in,,top]{\color{textcolor}\sffamily\fontsize{10.000000}{12.000000}\selectfont 60}%
\end{pgfscope}%
\begin{pgfscope}%
\pgfsetbuttcap%
\pgfsetroundjoin%
\definecolor{currentfill}{rgb}{0.000000,0.000000,0.000000}%
\pgfsetfillcolor{currentfill}%
\pgfsetlinewidth{0.803000pt}%
\definecolor{currentstroke}{rgb}{0.000000,0.000000,0.000000}%
\pgfsetstrokecolor{currentstroke}%
\pgfsetdash{}{0pt}%
\pgfsys@defobject{currentmarker}{\pgfqpoint{0.000000in}{-0.048611in}}{\pgfqpoint{0.000000in}{0.000000in}}{%
\pgfpathmoveto{\pgfqpoint{0.000000in}{0.000000in}}%
\pgfpathlineto{\pgfqpoint{0.000000in}{-0.048611in}}%
\pgfusepath{stroke,fill}%
}%
\begin{pgfscope}%
\pgfsys@transformshift{5.504545in}{0.528000in}%
\pgfsys@useobject{currentmarker}{}%
\end{pgfscope}%
\end{pgfscope}%
\begin{pgfscope}%
\definecolor{textcolor}{rgb}{0.000000,0.000000,0.000000}%
\pgfsetstrokecolor{textcolor}%
\pgfsetfillcolor{textcolor}%
\pgftext[x=5.504545in,y=0.430778in,,top]{\color{textcolor}\sffamily\fontsize{10.000000}{12.000000}\selectfont 80}%
\end{pgfscope}%
\begin{pgfscope}%
\definecolor{textcolor}{rgb}{0.000000,0.000000,0.000000}%
\pgfsetstrokecolor{textcolor}%
\pgfsetfillcolor{textcolor}%
\pgftext[x=3.280000in,y=0.240809in,,top]{\color{textcolor}\sffamily\fontsize{10.000000}{12.000000}\selectfont \(\displaystyle k\)}%
\end{pgfscope}%
\begin{pgfscope}%
\pgfsetbuttcap%
\pgfsetroundjoin%
\definecolor{currentfill}{rgb}{0.000000,0.000000,0.000000}%
\pgfsetfillcolor{currentfill}%
\pgfsetlinewidth{0.803000pt}%
\definecolor{currentstroke}{rgb}{0.000000,0.000000,0.000000}%
\pgfsetstrokecolor{currentstroke}%
\pgfsetdash{}{0pt}%
\pgfsys@defobject{currentmarker}{\pgfqpoint{-0.048611in}{0.000000in}}{\pgfqpoint{0.000000in}{0.000000in}}{%
\pgfpathmoveto{\pgfqpoint{0.000000in}{0.000000in}}%
\pgfpathlineto{\pgfqpoint{-0.048611in}{0.000000in}}%
\pgfusepath{stroke,fill}%
}%
\begin{pgfscope}%
\pgfsys@transformshift{0.800000in}{0.675476in}%
\pgfsys@useobject{currentmarker}{}%
\end{pgfscope}%
\end{pgfscope}%
\begin{pgfscope}%
\definecolor{textcolor}{rgb}{0.000000,0.000000,0.000000}%
\pgfsetstrokecolor{textcolor}%
\pgfsetfillcolor{textcolor}%
\pgftext[x=0.526047in,y=0.622715in,left,base]{\color{textcolor}\sffamily\fontsize{10.000000}{12.000000}\selectfont 20}%
\end{pgfscope}%
\begin{pgfscope}%
\pgfsetbuttcap%
\pgfsetroundjoin%
\definecolor{currentfill}{rgb}{0.000000,0.000000,0.000000}%
\pgfsetfillcolor{currentfill}%
\pgfsetlinewidth{0.803000pt}%
\definecolor{currentstroke}{rgb}{0.000000,0.000000,0.000000}%
\pgfsetstrokecolor{currentstroke}%
\pgfsetdash{}{0pt}%
\pgfsys@defobject{currentmarker}{\pgfqpoint{-0.048611in}{0.000000in}}{\pgfqpoint{0.000000in}{0.000000in}}{%
\pgfpathmoveto{\pgfqpoint{0.000000in}{0.000000in}}%
\pgfpathlineto{\pgfqpoint{-0.048611in}{0.000000in}}%
\pgfusepath{stroke,fill}%
}%
\begin{pgfscope}%
\pgfsys@transformshift{0.800000in}{1.105426in}%
\pgfsys@useobject{currentmarker}{}%
\end{pgfscope}%
\end{pgfscope}%
\begin{pgfscope}%
\definecolor{textcolor}{rgb}{0.000000,0.000000,0.000000}%
\pgfsetstrokecolor{textcolor}%
\pgfsetfillcolor{textcolor}%
\pgftext[x=0.526047in,y=1.052664in,left,base]{\color{textcolor}\sffamily\fontsize{10.000000}{12.000000}\selectfont 40}%
\end{pgfscope}%
\begin{pgfscope}%
\pgfsetbuttcap%
\pgfsetroundjoin%
\definecolor{currentfill}{rgb}{0.000000,0.000000,0.000000}%
\pgfsetfillcolor{currentfill}%
\pgfsetlinewidth{0.803000pt}%
\definecolor{currentstroke}{rgb}{0.000000,0.000000,0.000000}%
\pgfsetstrokecolor{currentstroke}%
\pgfsetdash{}{0pt}%
\pgfsys@defobject{currentmarker}{\pgfqpoint{-0.048611in}{0.000000in}}{\pgfqpoint{0.000000in}{0.000000in}}{%
\pgfpathmoveto{\pgfqpoint{0.000000in}{0.000000in}}%
\pgfpathlineto{\pgfqpoint{-0.048611in}{0.000000in}}%
\pgfusepath{stroke,fill}%
}%
\begin{pgfscope}%
\pgfsys@transformshift{0.800000in}{1.535376in}%
\pgfsys@useobject{currentmarker}{}%
\end{pgfscope}%
\end{pgfscope}%
\begin{pgfscope}%
\definecolor{textcolor}{rgb}{0.000000,0.000000,0.000000}%
\pgfsetstrokecolor{textcolor}%
\pgfsetfillcolor{textcolor}%
\pgftext[x=0.526047in,y=1.482614in,left,base]{\color{textcolor}\sffamily\fontsize{10.000000}{12.000000}\selectfont 60}%
\end{pgfscope}%
\begin{pgfscope}%
\pgfsetbuttcap%
\pgfsetroundjoin%
\definecolor{currentfill}{rgb}{0.000000,0.000000,0.000000}%
\pgfsetfillcolor{currentfill}%
\pgfsetlinewidth{0.803000pt}%
\definecolor{currentstroke}{rgb}{0.000000,0.000000,0.000000}%
\pgfsetstrokecolor{currentstroke}%
\pgfsetdash{}{0pt}%
\pgfsys@defobject{currentmarker}{\pgfqpoint{-0.048611in}{0.000000in}}{\pgfqpoint{0.000000in}{0.000000in}}{%
\pgfpathmoveto{\pgfqpoint{0.000000in}{0.000000in}}%
\pgfpathlineto{\pgfqpoint{-0.048611in}{0.000000in}}%
\pgfusepath{stroke,fill}%
}%
\begin{pgfscope}%
\pgfsys@transformshift{0.800000in}{1.965326in}%
\pgfsys@useobject{currentmarker}{}%
\end{pgfscope}%
\end{pgfscope}%
\begin{pgfscope}%
\definecolor{textcolor}{rgb}{0.000000,0.000000,0.000000}%
\pgfsetstrokecolor{textcolor}%
\pgfsetfillcolor{textcolor}%
\pgftext[x=0.526047in,y=1.912564in,left,base]{\color{textcolor}\sffamily\fontsize{10.000000}{12.000000}\selectfont 80}%
\end{pgfscope}%
\begin{pgfscope}%
\pgfsetbuttcap%
\pgfsetroundjoin%
\definecolor{currentfill}{rgb}{0.000000,0.000000,0.000000}%
\pgfsetfillcolor{currentfill}%
\pgfsetlinewidth{0.803000pt}%
\definecolor{currentstroke}{rgb}{0.000000,0.000000,0.000000}%
\pgfsetstrokecolor{currentstroke}%
\pgfsetdash{}{0pt}%
\pgfsys@defobject{currentmarker}{\pgfqpoint{-0.048611in}{0.000000in}}{\pgfqpoint{0.000000in}{0.000000in}}{%
\pgfpathmoveto{\pgfqpoint{0.000000in}{0.000000in}}%
\pgfpathlineto{\pgfqpoint{-0.048611in}{0.000000in}}%
\pgfusepath{stroke,fill}%
}%
\begin{pgfscope}%
\pgfsys@transformshift{0.800000in}{2.395276in}%
\pgfsys@useobject{currentmarker}{}%
\end{pgfscope}%
\end{pgfscope}%
\begin{pgfscope}%
\definecolor{textcolor}{rgb}{0.000000,0.000000,0.000000}%
\pgfsetstrokecolor{textcolor}%
\pgfsetfillcolor{textcolor}%
\pgftext[x=0.437682in,y=2.342514in,left,base]{\color{textcolor}\sffamily\fontsize{10.000000}{12.000000}\selectfont 100}%
\end{pgfscope}%
\begin{pgfscope}%
\pgfsetbuttcap%
\pgfsetroundjoin%
\definecolor{currentfill}{rgb}{0.000000,0.000000,0.000000}%
\pgfsetfillcolor{currentfill}%
\pgfsetlinewidth{0.803000pt}%
\definecolor{currentstroke}{rgb}{0.000000,0.000000,0.000000}%
\pgfsetstrokecolor{currentstroke}%
\pgfsetdash{}{0pt}%
\pgfsys@defobject{currentmarker}{\pgfqpoint{-0.048611in}{0.000000in}}{\pgfqpoint{0.000000in}{0.000000in}}{%
\pgfpathmoveto{\pgfqpoint{0.000000in}{0.000000in}}%
\pgfpathlineto{\pgfqpoint{-0.048611in}{0.000000in}}%
\pgfusepath{stroke,fill}%
}%
\begin{pgfscope}%
\pgfsys@transformshift{0.800000in}{2.825225in}%
\pgfsys@useobject{currentmarker}{}%
\end{pgfscope}%
\end{pgfscope}%
\begin{pgfscope}%
\definecolor{textcolor}{rgb}{0.000000,0.000000,0.000000}%
\pgfsetstrokecolor{textcolor}%
\pgfsetfillcolor{textcolor}%
\pgftext[x=0.437682in,y=2.772464in,left,base]{\color{textcolor}\sffamily\fontsize{10.000000}{12.000000}\selectfont 120}%
\end{pgfscope}%
\begin{pgfscope}%
\pgfsetbuttcap%
\pgfsetroundjoin%
\definecolor{currentfill}{rgb}{0.000000,0.000000,0.000000}%
\pgfsetfillcolor{currentfill}%
\pgfsetlinewidth{0.803000pt}%
\definecolor{currentstroke}{rgb}{0.000000,0.000000,0.000000}%
\pgfsetstrokecolor{currentstroke}%
\pgfsetdash{}{0pt}%
\pgfsys@defobject{currentmarker}{\pgfqpoint{-0.048611in}{0.000000in}}{\pgfqpoint{0.000000in}{0.000000in}}{%
\pgfpathmoveto{\pgfqpoint{0.000000in}{0.000000in}}%
\pgfpathlineto{\pgfqpoint{-0.048611in}{0.000000in}}%
\pgfusepath{stroke,fill}%
}%
\begin{pgfscope}%
\pgfsys@transformshift{0.800000in}{3.255175in}%
\pgfsys@useobject{currentmarker}{}%
\end{pgfscope}%
\end{pgfscope}%
\begin{pgfscope}%
\definecolor{textcolor}{rgb}{0.000000,0.000000,0.000000}%
\pgfsetstrokecolor{textcolor}%
\pgfsetfillcolor{textcolor}%
\pgftext[x=0.437682in,y=3.202414in,left,base]{\color{textcolor}\sffamily\fontsize{10.000000}{12.000000}\selectfont 140}%
\end{pgfscope}%
\begin{pgfscope}%
\pgfsetbuttcap%
\pgfsetroundjoin%
\definecolor{currentfill}{rgb}{0.000000,0.000000,0.000000}%
\pgfsetfillcolor{currentfill}%
\pgfsetlinewidth{0.803000pt}%
\definecolor{currentstroke}{rgb}{0.000000,0.000000,0.000000}%
\pgfsetstrokecolor{currentstroke}%
\pgfsetdash{}{0pt}%
\pgfsys@defobject{currentmarker}{\pgfqpoint{-0.048611in}{0.000000in}}{\pgfqpoint{0.000000in}{0.000000in}}{%
\pgfpathmoveto{\pgfqpoint{0.000000in}{0.000000in}}%
\pgfpathlineto{\pgfqpoint{-0.048611in}{0.000000in}}%
\pgfusepath{stroke,fill}%
}%
\begin{pgfscope}%
\pgfsys@transformshift{0.800000in}{3.685125in}%
\pgfsys@useobject{currentmarker}{}%
\end{pgfscope}%
\end{pgfscope}%
\begin{pgfscope}%
\definecolor{textcolor}{rgb}{0.000000,0.000000,0.000000}%
\pgfsetstrokecolor{textcolor}%
\pgfsetfillcolor{textcolor}%
\pgftext[x=0.437682in,y=3.632364in,left,base]{\color{textcolor}\sffamily\fontsize{10.000000}{12.000000}\selectfont 160}%
\end{pgfscope}%
\begin{pgfscope}%
\pgfsetbuttcap%
\pgfsetroundjoin%
\definecolor{currentfill}{rgb}{0.000000,0.000000,0.000000}%
\pgfsetfillcolor{currentfill}%
\pgfsetlinewidth{0.803000pt}%
\definecolor{currentstroke}{rgb}{0.000000,0.000000,0.000000}%
\pgfsetstrokecolor{currentstroke}%
\pgfsetdash{}{0pt}%
\pgfsys@defobject{currentmarker}{\pgfqpoint{-0.048611in}{0.000000in}}{\pgfqpoint{0.000000in}{0.000000in}}{%
\pgfpathmoveto{\pgfqpoint{0.000000in}{0.000000in}}%
\pgfpathlineto{\pgfqpoint{-0.048611in}{0.000000in}}%
\pgfusepath{stroke,fill}%
}%
\begin{pgfscope}%
\pgfsys@transformshift{0.800000in}{4.115075in}%
\pgfsys@useobject{currentmarker}{}%
\end{pgfscope}%
\end{pgfscope}%
\begin{pgfscope}%
\definecolor{textcolor}{rgb}{0.000000,0.000000,0.000000}%
\pgfsetstrokecolor{textcolor}%
\pgfsetfillcolor{textcolor}%
\pgftext[x=0.437682in,y=4.062313in,left,base]{\color{textcolor}\sffamily\fontsize{10.000000}{12.000000}\selectfont 180}%
\end{pgfscope}%
\begin{pgfscope}%
\definecolor{textcolor}{rgb}{0.000000,0.000000,0.000000}%
\pgfsetstrokecolor{textcolor}%
\pgfsetfillcolor{textcolor}%
\pgftext[x=0.382126in,y=2.376000in,,bottom,rotate=90.000000]{\color{textcolor}\sffamily\fontsize{10.000000}{12.000000}\selectfont Number of GMRES Iterations}%
\end{pgfscope}%
\begin{pgfscope}%
\pgfsetrectcap%
\pgfsetmiterjoin%
\pgfsetlinewidth{0.803000pt}%
\definecolor{currentstroke}{rgb}{0.000000,0.000000,0.000000}%
\pgfsetstrokecolor{currentstroke}%
\pgfsetdash{}{0pt}%
\pgfpathmoveto{\pgfqpoint{0.800000in}{0.528000in}}%
\pgfpathlineto{\pgfqpoint{0.800000in}{4.224000in}}%
\pgfusepath{stroke}%
\end{pgfscope}%
\begin{pgfscope}%
\pgfsetrectcap%
\pgfsetmiterjoin%
\pgfsetlinewidth{0.803000pt}%
\definecolor{currentstroke}{rgb}{0.000000,0.000000,0.000000}%
\pgfsetstrokecolor{currentstroke}%
\pgfsetdash{}{0pt}%
\pgfpathmoveto{\pgfqpoint{5.760000in}{0.528000in}}%
\pgfpathlineto{\pgfqpoint{5.760000in}{4.224000in}}%
\pgfusepath{stroke}%
\end{pgfscope}%
\begin{pgfscope}%
\pgfsetrectcap%
\pgfsetmiterjoin%
\pgfsetlinewidth{0.803000pt}%
\definecolor{currentstroke}{rgb}{0.000000,0.000000,0.000000}%
\pgfsetstrokecolor{currentstroke}%
\pgfsetdash{}{0pt}%
\pgfpathmoveto{\pgfqpoint{0.800000in}{0.528000in}}%
\pgfpathlineto{\pgfqpoint{5.760000in}{0.528000in}}%
\pgfusepath{stroke}%
\end{pgfscope}%
\begin{pgfscope}%
\pgfsetrectcap%
\pgfsetmiterjoin%
\pgfsetlinewidth{0.803000pt}%
\definecolor{currentstroke}{rgb}{0.000000,0.000000,0.000000}%
\pgfsetstrokecolor{currentstroke}%
\pgfsetdash{}{0pt}%
\pgfpathmoveto{\pgfqpoint{0.800000in}{4.224000in}}%
\pgfpathlineto{\pgfqpoint{5.760000in}{4.224000in}}%
\pgfusepath{stroke}%
\end{pgfscope}%
\end{pgfpicture}%
\makeatother%
\endgroup%

\caption{GMRES iteration counts for $\alpha = 0.5$}\label{fig:linfinityA0}
    \end{subfigure}
    
    \begin{subfigure}{\textwidth}
      \centering
%% Creator: Matplotlib, PGF backend
%%
%% To include the figure in your LaTeX document, write
%%   \input{<filename>.pgf}
%%
%% Make sure the required packages are loaded in your preamble
%%   \usepackage{pgf}
%%
%% Figures using additional raster images can only be included by \input if
%% they are in the same directory as the main LaTeX file. For loading figures
%% from other directories you can use the `import` package
%%   \usepackage{import}
%% and then include the figures with
%%   \import{<path to file>}{<filename>.pgf}
%%
%% Matplotlib used the following preamble
%%   \usepackage{fontspec}
%%   \setmainfont{DejaVuSerif.ttf}[Path=/home/owen/progs/firedrake-complex/firedrake/lib/python3.5/site-packages/matplotlib/mpl-data/fonts/ttf/]
%%   \setsansfont{DejaVuSans.ttf}[Path=/home/owen/progs/firedrake-complex/firedrake/lib/python3.5/site-packages/matplotlib/mpl-data/fonts/ttf/]
%%   \setmonofont{DejaVuSansMono.ttf}[Path=/home/owen/progs/firedrake-complex/firedrake/lib/python3.5/site-packages/matplotlib/mpl-data/fonts/ttf/]
%%
\begingroup%
\makeatletter%
\begin{pgfpicture}%
\pgfpathrectangle{\pgfpointorigin}{\pgfqpoint{6.400000in}{4.800000in}}%
\pgfusepath{use as bounding box, clip}%
\begin{pgfscope}%
\pgfsetbuttcap%
\pgfsetmiterjoin%
\definecolor{currentfill}{rgb}{1.000000,1.000000,1.000000}%
\pgfsetfillcolor{currentfill}%
\pgfsetlinewidth{0.000000pt}%
\definecolor{currentstroke}{rgb}{1.000000,1.000000,1.000000}%
\pgfsetstrokecolor{currentstroke}%
\pgfsetdash{}{0pt}%
\pgfpathmoveto{\pgfqpoint{0.000000in}{0.000000in}}%
\pgfpathlineto{\pgfqpoint{6.400000in}{0.000000in}}%
\pgfpathlineto{\pgfqpoint{6.400000in}{4.800000in}}%
\pgfpathlineto{\pgfqpoint{0.000000in}{4.800000in}}%
\pgfpathclose%
\pgfusepath{fill}%
\end{pgfscope}%
\begin{pgfscope}%
\pgfsetbuttcap%
\pgfsetmiterjoin%
\definecolor{currentfill}{rgb}{1.000000,1.000000,1.000000}%
\pgfsetfillcolor{currentfill}%
\pgfsetlinewidth{0.000000pt}%
\definecolor{currentstroke}{rgb}{0.000000,0.000000,0.000000}%
\pgfsetstrokecolor{currentstroke}%
\pgfsetstrokeopacity{0.000000}%
\pgfsetdash{}{0pt}%
\pgfpathmoveto{\pgfqpoint{0.800000in}{0.528000in}}%
\pgfpathlineto{\pgfqpoint{5.760000in}{0.528000in}}%
\pgfpathlineto{\pgfqpoint{5.760000in}{4.224000in}}%
\pgfpathlineto{\pgfqpoint{0.800000in}{4.224000in}}%
\pgfpathclose%
\pgfusepath{fill}%
\end{pgfscope}%
\begin{pgfscope}%
\pgfpathrectangle{\pgfqpoint{0.800000in}{0.528000in}}{\pgfqpoint{4.960000in}{3.696000in}}%
\pgfusepath{clip}%
\pgfsetbuttcap%
\pgfsetroundjoin%
\definecolor{currentfill}{rgb}{0.000000,0.000000,0.000000}%
\pgfsetfillcolor{currentfill}%
\pgfsetlinewidth{1.003750pt}%
\definecolor{currentstroke}{rgb}{0.000000,0.000000,0.000000}%
\pgfsetstrokecolor{currentstroke}%
\pgfsetdash{}{0pt}%
\pgfpathmoveto{\pgfqpoint{1.025906in}{0.664394in}}%
\pgfpathcurveto{\pgfqpoint{1.036956in}{0.664394in}}{\pgfqpoint{1.047555in}{0.668784in}}{\pgfqpoint{1.055369in}{0.676598in}}%
\pgfpathcurveto{\pgfqpoint{1.063182in}{0.684411in}}{\pgfqpoint{1.067573in}{0.695010in}}{\pgfqpoint{1.067573in}{0.706060in}}%
\pgfpathcurveto{\pgfqpoint{1.067573in}{0.717111in}}{\pgfqpoint{1.063182in}{0.727710in}}{\pgfqpoint{1.055369in}{0.735523in}}%
\pgfpathcurveto{\pgfqpoint{1.047555in}{0.743337in}}{\pgfqpoint{1.036956in}{0.747727in}}{\pgfqpoint{1.025906in}{0.747727in}}%
\pgfpathcurveto{\pgfqpoint{1.014856in}{0.747727in}}{\pgfqpoint{1.004257in}{0.743337in}}{\pgfqpoint{0.996443in}{0.735523in}}%
\pgfpathcurveto{\pgfqpoint{0.988630in}{0.727710in}}{\pgfqpoint{0.984239in}{0.717111in}}{\pgfqpoint{0.984239in}{0.706060in}}%
\pgfpathcurveto{\pgfqpoint{0.984239in}{0.695010in}}{\pgfqpoint{0.988630in}{0.684411in}}{\pgfqpoint{0.996443in}{0.676598in}}%
\pgfpathcurveto{\pgfqpoint{1.004257in}{0.668784in}}{\pgfqpoint{1.014856in}{0.664394in}}{\pgfqpoint{1.025906in}{0.664394in}}%
\pgfpathclose%
\pgfusepath{stroke,fill}%
\end{pgfscope}%
\begin{pgfscope}%
\pgfpathrectangle{\pgfqpoint{0.800000in}{0.528000in}}{\pgfqpoint{4.960000in}{3.696000in}}%
\pgfusepath{clip}%
\pgfsetbuttcap%
\pgfsetroundjoin%
\definecolor{currentfill}{rgb}{0.000000,0.000000,0.000000}%
\pgfsetfillcolor{currentfill}%
\pgfsetlinewidth{1.003750pt}%
\definecolor{currentstroke}{rgb}{0.000000,0.000000,0.000000}%
\pgfsetstrokecolor{currentstroke}%
\pgfsetdash{}{0pt}%
\pgfpathmoveto{\pgfqpoint{1.025906in}{0.664394in}}%
\pgfpathcurveto{\pgfqpoint{1.036956in}{0.664394in}}{\pgfqpoint{1.047555in}{0.668784in}}{\pgfqpoint{1.055369in}{0.676598in}}%
\pgfpathcurveto{\pgfqpoint{1.063182in}{0.684411in}}{\pgfqpoint{1.067573in}{0.695010in}}{\pgfqpoint{1.067573in}{0.706060in}}%
\pgfpathcurveto{\pgfqpoint{1.067573in}{0.717111in}}{\pgfqpoint{1.063182in}{0.727710in}}{\pgfqpoint{1.055369in}{0.735523in}}%
\pgfpathcurveto{\pgfqpoint{1.047555in}{0.743337in}}{\pgfqpoint{1.036956in}{0.747727in}}{\pgfqpoint{1.025906in}{0.747727in}}%
\pgfpathcurveto{\pgfqpoint{1.014856in}{0.747727in}}{\pgfqpoint{1.004257in}{0.743337in}}{\pgfqpoint{0.996443in}{0.735523in}}%
\pgfpathcurveto{\pgfqpoint{0.988630in}{0.727710in}}{\pgfqpoint{0.984239in}{0.717111in}}{\pgfqpoint{0.984239in}{0.706060in}}%
\pgfpathcurveto{\pgfqpoint{0.984239in}{0.695010in}}{\pgfqpoint{0.988630in}{0.684411in}}{\pgfqpoint{0.996443in}{0.676598in}}%
\pgfpathcurveto{\pgfqpoint{1.004257in}{0.668784in}}{\pgfqpoint{1.014856in}{0.664394in}}{\pgfqpoint{1.025906in}{0.664394in}}%
\pgfpathclose%
\pgfusepath{stroke,fill}%
\end{pgfscope}%
\begin{pgfscope}%
\pgfpathrectangle{\pgfqpoint{0.800000in}{0.528000in}}{\pgfqpoint{4.960000in}{3.696000in}}%
\pgfusepath{clip}%
\pgfsetbuttcap%
\pgfsetroundjoin%
\definecolor{currentfill}{rgb}{0.000000,0.000000,0.000000}%
\pgfsetfillcolor{currentfill}%
\pgfsetlinewidth{1.003750pt}%
\definecolor{currentstroke}{rgb}{0.000000,0.000000,0.000000}%
\pgfsetstrokecolor{currentstroke}%
\pgfsetdash{}{0pt}%
\pgfpathmoveto{\pgfqpoint{1.025906in}{0.664394in}}%
\pgfpathcurveto{\pgfqpoint{1.036956in}{0.664394in}}{\pgfqpoint{1.047555in}{0.668784in}}{\pgfqpoint{1.055369in}{0.676598in}}%
\pgfpathcurveto{\pgfqpoint{1.063182in}{0.684411in}}{\pgfqpoint{1.067573in}{0.695010in}}{\pgfqpoint{1.067573in}{0.706060in}}%
\pgfpathcurveto{\pgfqpoint{1.067573in}{0.717111in}}{\pgfqpoint{1.063182in}{0.727710in}}{\pgfqpoint{1.055369in}{0.735523in}}%
\pgfpathcurveto{\pgfqpoint{1.047555in}{0.743337in}}{\pgfqpoint{1.036956in}{0.747727in}}{\pgfqpoint{1.025906in}{0.747727in}}%
\pgfpathcurveto{\pgfqpoint{1.014856in}{0.747727in}}{\pgfqpoint{1.004257in}{0.743337in}}{\pgfqpoint{0.996443in}{0.735523in}}%
\pgfpathcurveto{\pgfqpoint{0.988630in}{0.727710in}}{\pgfqpoint{0.984239in}{0.717111in}}{\pgfqpoint{0.984239in}{0.706060in}}%
\pgfpathcurveto{\pgfqpoint{0.984239in}{0.695010in}}{\pgfqpoint{0.988630in}{0.684411in}}{\pgfqpoint{0.996443in}{0.676598in}}%
\pgfpathcurveto{\pgfqpoint{1.004257in}{0.668784in}}{\pgfqpoint{1.014856in}{0.664394in}}{\pgfqpoint{1.025906in}{0.664394in}}%
\pgfpathclose%
\pgfusepath{stroke,fill}%
\end{pgfscope}%
\begin{pgfscope}%
\pgfpathrectangle{\pgfqpoint{0.800000in}{0.528000in}}{\pgfqpoint{4.960000in}{3.696000in}}%
\pgfusepath{clip}%
\pgfsetbuttcap%
\pgfsetroundjoin%
\definecolor{currentfill}{rgb}{0.000000,0.000000,0.000000}%
\pgfsetfillcolor{currentfill}%
\pgfsetlinewidth{1.003750pt}%
\definecolor{currentstroke}{rgb}{0.000000,0.000000,0.000000}%
\pgfsetstrokecolor{currentstroke}%
\pgfsetdash{}{0pt}%
\pgfpathmoveto{\pgfqpoint{1.025906in}{0.664394in}}%
\pgfpathcurveto{\pgfqpoint{1.036956in}{0.664394in}}{\pgfqpoint{1.047555in}{0.668784in}}{\pgfqpoint{1.055369in}{0.676598in}}%
\pgfpathcurveto{\pgfqpoint{1.063182in}{0.684411in}}{\pgfqpoint{1.067573in}{0.695010in}}{\pgfqpoint{1.067573in}{0.706060in}}%
\pgfpathcurveto{\pgfqpoint{1.067573in}{0.717111in}}{\pgfqpoint{1.063182in}{0.727710in}}{\pgfqpoint{1.055369in}{0.735523in}}%
\pgfpathcurveto{\pgfqpoint{1.047555in}{0.743337in}}{\pgfqpoint{1.036956in}{0.747727in}}{\pgfqpoint{1.025906in}{0.747727in}}%
\pgfpathcurveto{\pgfqpoint{1.014856in}{0.747727in}}{\pgfqpoint{1.004257in}{0.743337in}}{\pgfqpoint{0.996443in}{0.735523in}}%
\pgfpathcurveto{\pgfqpoint{0.988630in}{0.727710in}}{\pgfqpoint{0.984239in}{0.717111in}}{\pgfqpoint{0.984239in}{0.706060in}}%
\pgfpathcurveto{\pgfqpoint{0.984239in}{0.695010in}}{\pgfqpoint{0.988630in}{0.684411in}}{\pgfqpoint{0.996443in}{0.676598in}}%
\pgfpathcurveto{\pgfqpoint{1.004257in}{0.668784in}}{\pgfqpoint{1.014856in}{0.664394in}}{\pgfqpoint{1.025906in}{0.664394in}}%
\pgfpathclose%
\pgfusepath{stroke,fill}%
\end{pgfscope}%
\begin{pgfscope}%
\pgfpathrectangle{\pgfqpoint{0.800000in}{0.528000in}}{\pgfqpoint{4.960000in}{3.696000in}}%
\pgfusepath{clip}%
\pgfsetbuttcap%
\pgfsetroundjoin%
\definecolor{currentfill}{rgb}{0.000000,0.000000,0.000000}%
\pgfsetfillcolor{currentfill}%
\pgfsetlinewidth{1.003750pt}%
\definecolor{currentstroke}{rgb}{0.000000,0.000000,0.000000}%
\pgfsetstrokecolor{currentstroke}%
\pgfsetdash{}{0pt}%
\pgfpathmoveto{\pgfqpoint{1.025906in}{0.664394in}}%
\pgfpathcurveto{\pgfqpoint{1.036956in}{0.664394in}}{\pgfqpoint{1.047555in}{0.668784in}}{\pgfqpoint{1.055369in}{0.676598in}}%
\pgfpathcurveto{\pgfqpoint{1.063182in}{0.684411in}}{\pgfqpoint{1.067573in}{0.695010in}}{\pgfqpoint{1.067573in}{0.706060in}}%
\pgfpathcurveto{\pgfqpoint{1.067573in}{0.717111in}}{\pgfqpoint{1.063182in}{0.727710in}}{\pgfqpoint{1.055369in}{0.735523in}}%
\pgfpathcurveto{\pgfqpoint{1.047555in}{0.743337in}}{\pgfqpoint{1.036956in}{0.747727in}}{\pgfqpoint{1.025906in}{0.747727in}}%
\pgfpathcurveto{\pgfqpoint{1.014856in}{0.747727in}}{\pgfqpoint{1.004257in}{0.743337in}}{\pgfqpoint{0.996443in}{0.735523in}}%
\pgfpathcurveto{\pgfqpoint{0.988630in}{0.727710in}}{\pgfqpoint{0.984239in}{0.717111in}}{\pgfqpoint{0.984239in}{0.706060in}}%
\pgfpathcurveto{\pgfqpoint{0.984239in}{0.695010in}}{\pgfqpoint{0.988630in}{0.684411in}}{\pgfqpoint{0.996443in}{0.676598in}}%
\pgfpathcurveto{\pgfqpoint{1.004257in}{0.668784in}}{\pgfqpoint{1.014856in}{0.664394in}}{\pgfqpoint{1.025906in}{0.664394in}}%
\pgfpathclose%
\pgfusepath{stroke,fill}%
\end{pgfscope}%
\begin{pgfscope}%
\pgfpathrectangle{\pgfqpoint{0.800000in}{0.528000in}}{\pgfqpoint{4.960000in}{3.696000in}}%
\pgfusepath{clip}%
\pgfsetbuttcap%
\pgfsetroundjoin%
\definecolor{currentfill}{rgb}{0.000000,0.000000,0.000000}%
\pgfsetfillcolor{currentfill}%
\pgfsetlinewidth{1.003750pt}%
\definecolor{currentstroke}{rgb}{0.000000,0.000000,0.000000}%
\pgfsetstrokecolor{currentstroke}%
\pgfsetdash{}{0pt}%
\pgfpathmoveto{\pgfqpoint{1.025906in}{0.664394in}}%
\pgfpathcurveto{\pgfqpoint{1.036956in}{0.664394in}}{\pgfqpoint{1.047555in}{0.668784in}}{\pgfqpoint{1.055369in}{0.676598in}}%
\pgfpathcurveto{\pgfqpoint{1.063182in}{0.684411in}}{\pgfqpoint{1.067573in}{0.695010in}}{\pgfqpoint{1.067573in}{0.706060in}}%
\pgfpathcurveto{\pgfqpoint{1.067573in}{0.717111in}}{\pgfqpoint{1.063182in}{0.727710in}}{\pgfqpoint{1.055369in}{0.735523in}}%
\pgfpathcurveto{\pgfqpoint{1.047555in}{0.743337in}}{\pgfqpoint{1.036956in}{0.747727in}}{\pgfqpoint{1.025906in}{0.747727in}}%
\pgfpathcurveto{\pgfqpoint{1.014856in}{0.747727in}}{\pgfqpoint{1.004257in}{0.743337in}}{\pgfqpoint{0.996443in}{0.735523in}}%
\pgfpathcurveto{\pgfqpoint{0.988630in}{0.727710in}}{\pgfqpoint{0.984239in}{0.717111in}}{\pgfqpoint{0.984239in}{0.706060in}}%
\pgfpathcurveto{\pgfqpoint{0.984239in}{0.695010in}}{\pgfqpoint{0.988630in}{0.684411in}}{\pgfqpoint{0.996443in}{0.676598in}}%
\pgfpathcurveto{\pgfqpoint{1.004257in}{0.668784in}}{\pgfqpoint{1.014856in}{0.664394in}}{\pgfqpoint{1.025906in}{0.664394in}}%
\pgfpathclose%
\pgfusepath{stroke,fill}%
\end{pgfscope}%
\begin{pgfscope}%
\pgfpathrectangle{\pgfqpoint{0.800000in}{0.528000in}}{\pgfqpoint{4.960000in}{3.696000in}}%
\pgfusepath{clip}%
\pgfsetbuttcap%
\pgfsetroundjoin%
\definecolor{currentfill}{rgb}{0.000000,0.000000,0.000000}%
\pgfsetfillcolor{currentfill}%
\pgfsetlinewidth{1.003750pt}%
\definecolor{currentstroke}{rgb}{0.000000,0.000000,0.000000}%
\pgfsetstrokecolor{currentstroke}%
\pgfsetdash{}{0pt}%
\pgfpathmoveto{\pgfqpoint{1.025906in}{0.664394in}}%
\pgfpathcurveto{\pgfqpoint{1.036956in}{0.664394in}}{\pgfqpoint{1.047555in}{0.668784in}}{\pgfqpoint{1.055369in}{0.676598in}}%
\pgfpathcurveto{\pgfqpoint{1.063182in}{0.684411in}}{\pgfqpoint{1.067573in}{0.695010in}}{\pgfqpoint{1.067573in}{0.706060in}}%
\pgfpathcurveto{\pgfqpoint{1.067573in}{0.717111in}}{\pgfqpoint{1.063182in}{0.727710in}}{\pgfqpoint{1.055369in}{0.735523in}}%
\pgfpathcurveto{\pgfqpoint{1.047555in}{0.743337in}}{\pgfqpoint{1.036956in}{0.747727in}}{\pgfqpoint{1.025906in}{0.747727in}}%
\pgfpathcurveto{\pgfqpoint{1.014856in}{0.747727in}}{\pgfqpoint{1.004257in}{0.743337in}}{\pgfqpoint{0.996443in}{0.735523in}}%
\pgfpathcurveto{\pgfqpoint{0.988630in}{0.727710in}}{\pgfqpoint{0.984239in}{0.717111in}}{\pgfqpoint{0.984239in}{0.706060in}}%
\pgfpathcurveto{\pgfqpoint{0.984239in}{0.695010in}}{\pgfqpoint{0.988630in}{0.684411in}}{\pgfqpoint{0.996443in}{0.676598in}}%
\pgfpathcurveto{\pgfqpoint{1.004257in}{0.668784in}}{\pgfqpoint{1.014856in}{0.664394in}}{\pgfqpoint{1.025906in}{0.664394in}}%
\pgfpathclose%
\pgfusepath{stroke,fill}%
\end{pgfscope}%
\begin{pgfscope}%
\pgfpathrectangle{\pgfqpoint{0.800000in}{0.528000in}}{\pgfqpoint{4.960000in}{3.696000in}}%
\pgfusepath{clip}%
\pgfsetbuttcap%
\pgfsetroundjoin%
\definecolor{currentfill}{rgb}{0.000000,0.000000,0.000000}%
\pgfsetfillcolor{currentfill}%
\pgfsetlinewidth{1.003750pt}%
\definecolor{currentstroke}{rgb}{0.000000,0.000000,0.000000}%
\pgfsetstrokecolor{currentstroke}%
\pgfsetdash{}{0pt}%
\pgfpathmoveto{\pgfqpoint{1.025906in}{0.664394in}}%
\pgfpathcurveto{\pgfqpoint{1.036956in}{0.664394in}}{\pgfqpoint{1.047555in}{0.668784in}}{\pgfqpoint{1.055369in}{0.676598in}}%
\pgfpathcurveto{\pgfqpoint{1.063182in}{0.684411in}}{\pgfqpoint{1.067573in}{0.695010in}}{\pgfqpoint{1.067573in}{0.706060in}}%
\pgfpathcurveto{\pgfqpoint{1.067573in}{0.717111in}}{\pgfqpoint{1.063182in}{0.727710in}}{\pgfqpoint{1.055369in}{0.735523in}}%
\pgfpathcurveto{\pgfqpoint{1.047555in}{0.743337in}}{\pgfqpoint{1.036956in}{0.747727in}}{\pgfqpoint{1.025906in}{0.747727in}}%
\pgfpathcurveto{\pgfqpoint{1.014856in}{0.747727in}}{\pgfqpoint{1.004257in}{0.743337in}}{\pgfqpoint{0.996443in}{0.735523in}}%
\pgfpathcurveto{\pgfqpoint{0.988630in}{0.727710in}}{\pgfqpoint{0.984239in}{0.717111in}}{\pgfqpoint{0.984239in}{0.706060in}}%
\pgfpathcurveto{\pgfqpoint{0.984239in}{0.695010in}}{\pgfqpoint{0.988630in}{0.684411in}}{\pgfqpoint{0.996443in}{0.676598in}}%
\pgfpathcurveto{\pgfqpoint{1.004257in}{0.668784in}}{\pgfqpoint{1.014856in}{0.664394in}}{\pgfqpoint{1.025906in}{0.664394in}}%
\pgfpathclose%
\pgfusepath{stroke,fill}%
\end{pgfscope}%
\begin{pgfscope}%
\pgfpathrectangle{\pgfqpoint{0.800000in}{0.528000in}}{\pgfqpoint{4.960000in}{3.696000in}}%
\pgfusepath{clip}%
\pgfsetbuttcap%
\pgfsetroundjoin%
\definecolor{currentfill}{rgb}{0.000000,0.000000,0.000000}%
\pgfsetfillcolor{currentfill}%
\pgfsetlinewidth{1.003750pt}%
\definecolor{currentstroke}{rgb}{0.000000,0.000000,0.000000}%
\pgfsetstrokecolor{currentstroke}%
\pgfsetdash{}{0pt}%
\pgfpathmoveto{\pgfqpoint{1.025906in}{0.664394in}}%
\pgfpathcurveto{\pgfqpoint{1.036956in}{0.664394in}}{\pgfqpoint{1.047555in}{0.668784in}}{\pgfqpoint{1.055369in}{0.676598in}}%
\pgfpathcurveto{\pgfqpoint{1.063182in}{0.684411in}}{\pgfqpoint{1.067573in}{0.695010in}}{\pgfqpoint{1.067573in}{0.706060in}}%
\pgfpathcurveto{\pgfqpoint{1.067573in}{0.717111in}}{\pgfqpoint{1.063182in}{0.727710in}}{\pgfqpoint{1.055369in}{0.735523in}}%
\pgfpathcurveto{\pgfqpoint{1.047555in}{0.743337in}}{\pgfqpoint{1.036956in}{0.747727in}}{\pgfqpoint{1.025906in}{0.747727in}}%
\pgfpathcurveto{\pgfqpoint{1.014856in}{0.747727in}}{\pgfqpoint{1.004257in}{0.743337in}}{\pgfqpoint{0.996443in}{0.735523in}}%
\pgfpathcurveto{\pgfqpoint{0.988630in}{0.727710in}}{\pgfqpoint{0.984239in}{0.717111in}}{\pgfqpoint{0.984239in}{0.706060in}}%
\pgfpathcurveto{\pgfqpoint{0.984239in}{0.695010in}}{\pgfqpoint{0.988630in}{0.684411in}}{\pgfqpoint{0.996443in}{0.676598in}}%
\pgfpathcurveto{\pgfqpoint{1.004257in}{0.668784in}}{\pgfqpoint{1.014856in}{0.664394in}}{\pgfqpoint{1.025906in}{0.664394in}}%
\pgfpathclose%
\pgfusepath{stroke,fill}%
\end{pgfscope}%
\begin{pgfscope}%
\pgfpathrectangle{\pgfqpoint{0.800000in}{0.528000in}}{\pgfqpoint{4.960000in}{3.696000in}}%
\pgfusepath{clip}%
\pgfsetbuttcap%
\pgfsetroundjoin%
\definecolor{currentfill}{rgb}{0.000000,0.000000,0.000000}%
\pgfsetfillcolor{currentfill}%
\pgfsetlinewidth{1.003750pt}%
\definecolor{currentstroke}{rgb}{0.000000,0.000000,0.000000}%
\pgfsetstrokecolor{currentstroke}%
\pgfsetdash{}{0pt}%
\pgfpathmoveto{\pgfqpoint{1.025906in}{0.664394in}}%
\pgfpathcurveto{\pgfqpoint{1.036956in}{0.664394in}}{\pgfqpoint{1.047555in}{0.668784in}}{\pgfqpoint{1.055369in}{0.676598in}}%
\pgfpathcurveto{\pgfqpoint{1.063182in}{0.684411in}}{\pgfqpoint{1.067573in}{0.695010in}}{\pgfqpoint{1.067573in}{0.706060in}}%
\pgfpathcurveto{\pgfqpoint{1.067573in}{0.717111in}}{\pgfqpoint{1.063182in}{0.727710in}}{\pgfqpoint{1.055369in}{0.735523in}}%
\pgfpathcurveto{\pgfqpoint{1.047555in}{0.743337in}}{\pgfqpoint{1.036956in}{0.747727in}}{\pgfqpoint{1.025906in}{0.747727in}}%
\pgfpathcurveto{\pgfqpoint{1.014856in}{0.747727in}}{\pgfqpoint{1.004257in}{0.743337in}}{\pgfqpoint{0.996443in}{0.735523in}}%
\pgfpathcurveto{\pgfqpoint{0.988630in}{0.727710in}}{\pgfqpoint{0.984239in}{0.717111in}}{\pgfqpoint{0.984239in}{0.706060in}}%
\pgfpathcurveto{\pgfqpoint{0.984239in}{0.695010in}}{\pgfqpoint{0.988630in}{0.684411in}}{\pgfqpoint{0.996443in}{0.676598in}}%
\pgfpathcurveto{\pgfqpoint{1.004257in}{0.668784in}}{\pgfqpoint{1.014856in}{0.664394in}}{\pgfqpoint{1.025906in}{0.664394in}}%
\pgfpathclose%
\pgfusepath{stroke,fill}%
\end{pgfscope}%
\begin{pgfscope}%
\pgfpathrectangle{\pgfqpoint{0.800000in}{0.528000in}}{\pgfqpoint{4.960000in}{3.696000in}}%
\pgfusepath{clip}%
\pgfsetbuttcap%
\pgfsetroundjoin%
\definecolor{currentfill}{rgb}{0.000000,0.000000,0.000000}%
\pgfsetfillcolor{currentfill}%
\pgfsetlinewidth{1.003750pt}%
\definecolor{currentstroke}{rgb}{0.000000,0.000000,0.000000}%
\pgfsetstrokecolor{currentstroke}%
\pgfsetdash{}{0pt}%
\pgfpathmoveto{\pgfqpoint{1.025906in}{1.771040in}}%
\pgfpathcurveto{\pgfqpoint{1.036956in}{1.771040in}}{\pgfqpoint{1.047555in}{1.775431in}}{\pgfqpoint{1.055369in}{1.783244in}}%
\pgfpathcurveto{\pgfqpoint{1.063182in}{1.791058in}}{\pgfqpoint{1.067573in}{1.801657in}}{\pgfqpoint{1.067573in}{1.812707in}}%
\pgfpathcurveto{\pgfqpoint{1.067573in}{1.823757in}}{\pgfqpoint{1.063182in}{1.834356in}}{\pgfqpoint{1.055369in}{1.842170in}}%
\pgfpathcurveto{\pgfqpoint{1.047555in}{1.849983in}}{\pgfqpoint{1.036956in}{1.854374in}}{\pgfqpoint{1.025906in}{1.854374in}}%
\pgfpathcurveto{\pgfqpoint{1.014856in}{1.854374in}}{\pgfqpoint{1.004257in}{1.849983in}}{\pgfqpoint{0.996443in}{1.842170in}}%
\pgfpathcurveto{\pgfqpoint{0.988630in}{1.834356in}}{\pgfqpoint{0.984239in}{1.823757in}}{\pgfqpoint{0.984239in}{1.812707in}}%
\pgfpathcurveto{\pgfqpoint{0.984239in}{1.801657in}}{\pgfqpoint{0.988630in}{1.791058in}}{\pgfqpoint{0.996443in}{1.783244in}}%
\pgfpathcurveto{\pgfqpoint{1.004257in}{1.775431in}}{\pgfqpoint{1.014856in}{1.771040in}}{\pgfqpoint{1.025906in}{1.771040in}}%
\pgfpathclose%
\pgfusepath{stroke,fill}%
\end{pgfscope}%
\begin{pgfscope}%
\pgfpathrectangle{\pgfqpoint{0.800000in}{0.528000in}}{\pgfqpoint{4.960000in}{3.696000in}}%
\pgfusepath{clip}%
\pgfsetbuttcap%
\pgfsetroundjoin%
\definecolor{currentfill}{rgb}{0.000000,0.000000,0.000000}%
\pgfsetfillcolor{currentfill}%
\pgfsetlinewidth{1.003750pt}%
\definecolor{currentstroke}{rgb}{0.000000,0.000000,0.000000}%
\pgfsetstrokecolor{currentstroke}%
\pgfsetdash{}{0pt}%
\pgfpathmoveto{\pgfqpoint{1.025906in}{0.664394in}}%
\pgfpathcurveto{\pgfqpoint{1.036956in}{0.664394in}}{\pgfqpoint{1.047555in}{0.668784in}}{\pgfqpoint{1.055369in}{0.676598in}}%
\pgfpathcurveto{\pgfqpoint{1.063182in}{0.684411in}}{\pgfqpoint{1.067573in}{0.695010in}}{\pgfqpoint{1.067573in}{0.706060in}}%
\pgfpathcurveto{\pgfqpoint{1.067573in}{0.717111in}}{\pgfqpoint{1.063182in}{0.727710in}}{\pgfqpoint{1.055369in}{0.735523in}}%
\pgfpathcurveto{\pgfqpoint{1.047555in}{0.743337in}}{\pgfqpoint{1.036956in}{0.747727in}}{\pgfqpoint{1.025906in}{0.747727in}}%
\pgfpathcurveto{\pgfqpoint{1.014856in}{0.747727in}}{\pgfqpoint{1.004257in}{0.743337in}}{\pgfqpoint{0.996443in}{0.735523in}}%
\pgfpathcurveto{\pgfqpoint{0.988630in}{0.727710in}}{\pgfqpoint{0.984239in}{0.717111in}}{\pgfqpoint{0.984239in}{0.706060in}}%
\pgfpathcurveto{\pgfqpoint{0.984239in}{0.695010in}}{\pgfqpoint{0.988630in}{0.684411in}}{\pgfqpoint{0.996443in}{0.676598in}}%
\pgfpathcurveto{\pgfqpoint{1.004257in}{0.668784in}}{\pgfqpoint{1.014856in}{0.664394in}}{\pgfqpoint{1.025906in}{0.664394in}}%
\pgfpathclose%
\pgfusepath{stroke,fill}%
\end{pgfscope}%
\begin{pgfscope}%
\pgfpathrectangle{\pgfqpoint{0.800000in}{0.528000in}}{\pgfqpoint{4.960000in}{3.696000in}}%
\pgfusepath{clip}%
\pgfsetbuttcap%
\pgfsetroundjoin%
\definecolor{currentfill}{rgb}{0.000000,0.000000,0.000000}%
\pgfsetfillcolor{currentfill}%
\pgfsetlinewidth{1.003750pt}%
\definecolor{currentstroke}{rgb}{0.000000,0.000000,0.000000}%
\pgfsetstrokecolor{currentstroke}%
\pgfsetdash{}{0pt}%
\pgfpathmoveto{\pgfqpoint{1.025906in}{0.664394in}}%
\pgfpathcurveto{\pgfqpoint{1.036956in}{0.664394in}}{\pgfqpoint{1.047555in}{0.668784in}}{\pgfqpoint{1.055369in}{0.676598in}}%
\pgfpathcurveto{\pgfqpoint{1.063182in}{0.684411in}}{\pgfqpoint{1.067573in}{0.695010in}}{\pgfqpoint{1.067573in}{0.706060in}}%
\pgfpathcurveto{\pgfqpoint{1.067573in}{0.717111in}}{\pgfqpoint{1.063182in}{0.727710in}}{\pgfqpoint{1.055369in}{0.735523in}}%
\pgfpathcurveto{\pgfqpoint{1.047555in}{0.743337in}}{\pgfqpoint{1.036956in}{0.747727in}}{\pgfqpoint{1.025906in}{0.747727in}}%
\pgfpathcurveto{\pgfqpoint{1.014856in}{0.747727in}}{\pgfqpoint{1.004257in}{0.743337in}}{\pgfqpoint{0.996443in}{0.735523in}}%
\pgfpathcurveto{\pgfqpoint{0.988630in}{0.727710in}}{\pgfqpoint{0.984239in}{0.717111in}}{\pgfqpoint{0.984239in}{0.706060in}}%
\pgfpathcurveto{\pgfqpoint{0.984239in}{0.695010in}}{\pgfqpoint{0.988630in}{0.684411in}}{\pgfqpoint{0.996443in}{0.676598in}}%
\pgfpathcurveto{\pgfqpoint{1.004257in}{0.668784in}}{\pgfqpoint{1.014856in}{0.664394in}}{\pgfqpoint{1.025906in}{0.664394in}}%
\pgfpathclose%
\pgfusepath{stroke,fill}%
\end{pgfscope}%
\begin{pgfscope}%
\pgfpathrectangle{\pgfqpoint{0.800000in}{0.528000in}}{\pgfqpoint{4.960000in}{3.696000in}}%
\pgfusepath{clip}%
\pgfsetbuttcap%
\pgfsetroundjoin%
\definecolor{currentfill}{rgb}{0.000000,0.000000,0.000000}%
\pgfsetfillcolor{currentfill}%
\pgfsetlinewidth{1.003750pt}%
\definecolor{currentstroke}{rgb}{0.000000,0.000000,0.000000}%
\pgfsetstrokecolor{currentstroke}%
\pgfsetdash{}{0pt}%
\pgfpathmoveto{\pgfqpoint{1.025906in}{0.664394in}}%
\pgfpathcurveto{\pgfqpoint{1.036956in}{0.664394in}}{\pgfqpoint{1.047555in}{0.668784in}}{\pgfqpoint{1.055369in}{0.676598in}}%
\pgfpathcurveto{\pgfqpoint{1.063182in}{0.684411in}}{\pgfqpoint{1.067573in}{0.695010in}}{\pgfqpoint{1.067573in}{0.706060in}}%
\pgfpathcurveto{\pgfqpoint{1.067573in}{0.717111in}}{\pgfqpoint{1.063182in}{0.727710in}}{\pgfqpoint{1.055369in}{0.735523in}}%
\pgfpathcurveto{\pgfqpoint{1.047555in}{0.743337in}}{\pgfqpoint{1.036956in}{0.747727in}}{\pgfqpoint{1.025906in}{0.747727in}}%
\pgfpathcurveto{\pgfqpoint{1.014856in}{0.747727in}}{\pgfqpoint{1.004257in}{0.743337in}}{\pgfqpoint{0.996443in}{0.735523in}}%
\pgfpathcurveto{\pgfqpoint{0.988630in}{0.727710in}}{\pgfqpoint{0.984239in}{0.717111in}}{\pgfqpoint{0.984239in}{0.706060in}}%
\pgfpathcurveto{\pgfqpoint{0.984239in}{0.695010in}}{\pgfqpoint{0.988630in}{0.684411in}}{\pgfqpoint{0.996443in}{0.676598in}}%
\pgfpathcurveto{\pgfqpoint{1.004257in}{0.668784in}}{\pgfqpoint{1.014856in}{0.664394in}}{\pgfqpoint{1.025906in}{0.664394in}}%
\pgfpathclose%
\pgfusepath{stroke,fill}%
\end{pgfscope}%
\begin{pgfscope}%
\pgfpathrectangle{\pgfqpoint{0.800000in}{0.528000in}}{\pgfqpoint{4.960000in}{3.696000in}}%
\pgfusepath{clip}%
\pgfsetbuttcap%
\pgfsetroundjoin%
\definecolor{currentfill}{rgb}{0.000000,0.000000,0.000000}%
\pgfsetfillcolor{currentfill}%
\pgfsetlinewidth{1.003750pt}%
\definecolor{currentstroke}{rgb}{0.000000,0.000000,0.000000}%
\pgfsetstrokecolor{currentstroke}%
\pgfsetdash{}{0pt}%
\pgfpathmoveto{\pgfqpoint{1.025906in}{0.664394in}}%
\pgfpathcurveto{\pgfqpoint{1.036956in}{0.664394in}}{\pgfqpoint{1.047555in}{0.668784in}}{\pgfqpoint{1.055369in}{0.676598in}}%
\pgfpathcurveto{\pgfqpoint{1.063182in}{0.684411in}}{\pgfqpoint{1.067573in}{0.695010in}}{\pgfqpoint{1.067573in}{0.706060in}}%
\pgfpathcurveto{\pgfqpoint{1.067573in}{0.717111in}}{\pgfqpoint{1.063182in}{0.727710in}}{\pgfqpoint{1.055369in}{0.735523in}}%
\pgfpathcurveto{\pgfqpoint{1.047555in}{0.743337in}}{\pgfqpoint{1.036956in}{0.747727in}}{\pgfqpoint{1.025906in}{0.747727in}}%
\pgfpathcurveto{\pgfqpoint{1.014856in}{0.747727in}}{\pgfqpoint{1.004257in}{0.743337in}}{\pgfqpoint{0.996443in}{0.735523in}}%
\pgfpathcurveto{\pgfqpoint{0.988630in}{0.727710in}}{\pgfqpoint{0.984239in}{0.717111in}}{\pgfqpoint{0.984239in}{0.706060in}}%
\pgfpathcurveto{\pgfqpoint{0.984239in}{0.695010in}}{\pgfqpoint{0.988630in}{0.684411in}}{\pgfqpoint{0.996443in}{0.676598in}}%
\pgfpathcurveto{\pgfqpoint{1.004257in}{0.668784in}}{\pgfqpoint{1.014856in}{0.664394in}}{\pgfqpoint{1.025906in}{0.664394in}}%
\pgfpathclose%
\pgfusepath{stroke,fill}%
\end{pgfscope}%
\begin{pgfscope}%
\pgfpathrectangle{\pgfqpoint{0.800000in}{0.528000in}}{\pgfqpoint{4.960000in}{3.696000in}}%
\pgfusepath{clip}%
\pgfsetbuttcap%
\pgfsetroundjoin%
\definecolor{currentfill}{rgb}{0.000000,0.000000,0.000000}%
\pgfsetfillcolor{currentfill}%
\pgfsetlinewidth{1.003750pt}%
\definecolor{currentstroke}{rgb}{0.000000,0.000000,0.000000}%
\pgfsetstrokecolor{currentstroke}%
\pgfsetdash{}{0pt}%
\pgfpathmoveto{\pgfqpoint{1.025906in}{0.664394in}}%
\pgfpathcurveto{\pgfqpoint{1.036956in}{0.664394in}}{\pgfqpoint{1.047555in}{0.668784in}}{\pgfqpoint{1.055369in}{0.676598in}}%
\pgfpathcurveto{\pgfqpoint{1.063182in}{0.684411in}}{\pgfqpoint{1.067573in}{0.695010in}}{\pgfqpoint{1.067573in}{0.706060in}}%
\pgfpathcurveto{\pgfqpoint{1.067573in}{0.717111in}}{\pgfqpoint{1.063182in}{0.727710in}}{\pgfqpoint{1.055369in}{0.735523in}}%
\pgfpathcurveto{\pgfqpoint{1.047555in}{0.743337in}}{\pgfqpoint{1.036956in}{0.747727in}}{\pgfqpoint{1.025906in}{0.747727in}}%
\pgfpathcurveto{\pgfqpoint{1.014856in}{0.747727in}}{\pgfqpoint{1.004257in}{0.743337in}}{\pgfqpoint{0.996443in}{0.735523in}}%
\pgfpathcurveto{\pgfqpoint{0.988630in}{0.727710in}}{\pgfqpoint{0.984239in}{0.717111in}}{\pgfqpoint{0.984239in}{0.706060in}}%
\pgfpathcurveto{\pgfqpoint{0.984239in}{0.695010in}}{\pgfqpoint{0.988630in}{0.684411in}}{\pgfqpoint{0.996443in}{0.676598in}}%
\pgfpathcurveto{\pgfqpoint{1.004257in}{0.668784in}}{\pgfqpoint{1.014856in}{0.664394in}}{\pgfqpoint{1.025906in}{0.664394in}}%
\pgfpathclose%
\pgfusepath{stroke,fill}%
\end{pgfscope}%
\begin{pgfscope}%
\pgfpathrectangle{\pgfqpoint{0.800000in}{0.528000in}}{\pgfqpoint{4.960000in}{3.696000in}}%
\pgfusepath{clip}%
\pgfsetbuttcap%
\pgfsetroundjoin%
\definecolor{currentfill}{rgb}{0.000000,0.000000,0.000000}%
\pgfsetfillcolor{currentfill}%
\pgfsetlinewidth{1.003750pt}%
\definecolor{currentstroke}{rgb}{0.000000,0.000000,0.000000}%
\pgfsetstrokecolor{currentstroke}%
\pgfsetdash{}{0pt}%
\pgfpathmoveto{\pgfqpoint{1.025906in}{0.664394in}}%
\pgfpathcurveto{\pgfqpoint{1.036956in}{0.664394in}}{\pgfqpoint{1.047555in}{0.668784in}}{\pgfqpoint{1.055369in}{0.676598in}}%
\pgfpathcurveto{\pgfqpoint{1.063182in}{0.684411in}}{\pgfqpoint{1.067573in}{0.695010in}}{\pgfqpoint{1.067573in}{0.706060in}}%
\pgfpathcurveto{\pgfqpoint{1.067573in}{0.717111in}}{\pgfqpoint{1.063182in}{0.727710in}}{\pgfqpoint{1.055369in}{0.735523in}}%
\pgfpathcurveto{\pgfqpoint{1.047555in}{0.743337in}}{\pgfqpoint{1.036956in}{0.747727in}}{\pgfqpoint{1.025906in}{0.747727in}}%
\pgfpathcurveto{\pgfqpoint{1.014856in}{0.747727in}}{\pgfqpoint{1.004257in}{0.743337in}}{\pgfqpoint{0.996443in}{0.735523in}}%
\pgfpathcurveto{\pgfqpoint{0.988630in}{0.727710in}}{\pgfqpoint{0.984239in}{0.717111in}}{\pgfqpoint{0.984239in}{0.706060in}}%
\pgfpathcurveto{\pgfqpoint{0.984239in}{0.695010in}}{\pgfqpoint{0.988630in}{0.684411in}}{\pgfqpoint{0.996443in}{0.676598in}}%
\pgfpathcurveto{\pgfqpoint{1.004257in}{0.668784in}}{\pgfqpoint{1.014856in}{0.664394in}}{\pgfqpoint{1.025906in}{0.664394in}}%
\pgfpathclose%
\pgfusepath{stroke,fill}%
\end{pgfscope}%
\begin{pgfscope}%
\pgfpathrectangle{\pgfqpoint{0.800000in}{0.528000in}}{\pgfqpoint{4.960000in}{3.696000in}}%
\pgfusepath{clip}%
\pgfsetbuttcap%
\pgfsetroundjoin%
\definecolor{currentfill}{rgb}{0.000000,0.000000,0.000000}%
\pgfsetfillcolor{currentfill}%
\pgfsetlinewidth{1.003750pt}%
\definecolor{currentstroke}{rgb}{0.000000,0.000000,0.000000}%
\pgfsetstrokecolor{currentstroke}%
\pgfsetdash{}{0pt}%
\pgfpathmoveto{\pgfqpoint{1.025906in}{0.664394in}}%
\pgfpathcurveto{\pgfqpoint{1.036956in}{0.664394in}}{\pgfqpoint{1.047555in}{0.668784in}}{\pgfqpoint{1.055369in}{0.676598in}}%
\pgfpathcurveto{\pgfqpoint{1.063182in}{0.684411in}}{\pgfqpoint{1.067573in}{0.695010in}}{\pgfqpoint{1.067573in}{0.706060in}}%
\pgfpathcurveto{\pgfqpoint{1.067573in}{0.717111in}}{\pgfqpoint{1.063182in}{0.727710in}}{\pgfqpoint{1.055369in}{0.735523in}}%
\pgfpathcurveto{\pgfqpoint{1.047555in}{0.743337in}}{\pgfqpoint{1.036956in}{0.747727in}}{\pgfqpoint{1.025906in}{0.747727in}}%
\pgfpathcurveto{\pgfqpoint{1.014856in}{0.747727in}}{\pgfqpoint{1.004257in}{0.743337in}}{\pgfqpoint{0.996443in}{0.735523in}}%
\pgfpathcurveto{\pgfqpoint{0.988630in}{0.727710in}}{\pgfqpoint{0.984239in}{0.717111in}}{\pgfqpoint{0.984239in}{0.706060in}}%
\pgfpathcurveto{\pgfqpoint{0.984239in}{0.695010in}}{\pgfqpoint{0.988630in}{0.684411in}}{\pgfqpoint{0.996443in}{0.676598in}}%
\pgfpathcurveto{\pgfqpoint{1.004257in}{0.668784in}}{\pgfqpoint{1.014856in}{0.664394in}}{\pgfqpoint{1.025906in}{0.664394in}}%
\pgfpathclose%
\pgfusepath{stroke,fill}%
\end{pgfscope}%
\begin{pgfscope}%
\pgfpathrectangle{\pgfqpoint{0.800000in}{0.528000in}}{\pgfqpoint{4.960000in}{3.696000in}}%
\pgfusepath{clip}%
\pgfsetbuttcap%
\pgfsetroundjoin%
\definecolor{currentfill}{rgb}{0.000000,0.000000,0.000000}%
\pgfsetfillcolor{currentfill}%
\pgfsetlinewidth{1.003750pt}%
\definecolor{currentstroke}{rgb}{0.000000,0.000000,0.000000}%
\pgfsetstrokecolor{currentstroke}%
\pgfsetdash{}{0pt}%
\pgfpathmoveto{\pgfqpoint{1.025906in}{0.664394in}}%
\pgfpathcurveto{\pgfqpoint{1.036956in}{0.664394in}}{\pgfqpoint{1.047555in}{0.668784in}}{\pgfqpoint{1.055369in}{0.676598in}}%
\pgfpathcurveto{\pgfqpoint{1.063182in}{0.684411in}}{\pgfqpoint{1.067573in}{0.695010in}}{\pgfqpoint{1.067573in}{0.706060in}}%
\pgfpathcurveto{\pgfqpoint{1.067573in}{0.717111in}}{\pgfqpoint{1.063182in}{0.727710in}}{\pgfqpoint{1.055369in}{0.735523in}}%
\pgfpathcurveto{\pgfqpoint{1.047555in}{0.743337in}}{\pgfqpoint{1.036956in}{0.747727in}}{\pgfqpoint{1.025906in}{0.747727in}}%
\pgfpathcurveto{\pgfqpoint{1.014856in}{0.747727in}}{\pgfqpoint{1.004257in}{0.743337in}}{\pgfqpoint{0.996443in}{0.735523in}}%
\pgfpathcurveto{\pgfqpoint{0.988630in}{0.727710in}}{\pgfqpoint{0.984239in}{0.717111in}}{\pgfqpoint{0.984239in}{0.706060in}}%
\pgfpathcurveto{\pgfqpoint{0.984239in}{0.695010in}}{\pgfqpoint{0.988630in}{0.684411in}}{\pgfqpoint{0.996443in}{0.676598in}}%
\pgfpathcurveto{\pgfqpoint{1.004257in}{0.668784in}}{\pgfqpoint{1.014856in}{0.664394in}}{\pgfqpoint{1.025906in}{0.664394in}}%
\pgfpathclose%
\pgfusepath{stroke,fill}%
\end{pgfscope}%
\begin{pgfscope}%
\pgfpathrectangle{\pgfqpoint{0.800000in}{0.528000in}}{\pgfqpoint{4.960000in}{3.696000in}}%
\pgfusepath{clip}%
\pgfsetbuttcap%
\pgfsetroundjoin%
\definecolor{currentfill}{rgb}{0.000000,0.000000,0.000000}%
\pgfsetfillcolor{currentfill}%
\pgfsetlinewidth{1.003750pt}%
\definecolor{currentstroke}{rgb}{0.000000,0.000000,0.000000}%
\pgfsetstrokecolor{currentstroke}%
\pgfsetdash{}{0pt}%
\pgfpathmoveto{\pgfqpoint{1.025906in}{0.664394in}}%
\pgfpathcurveto{\pgfqpoint{1.036956in}{0.664394in}}{\pgfqpoint{1.047555in}{0.668784in}}{\pgfqpoint{1.055369in}{0.676598in}}%
\pgfpathcurveto{\pgfqpoint{1.063182in}{0.684411in}}{\pgfqpoint{1.067573in}{0.695010in}}{\pgfqpoint{1.067573in}{0.706060in}}%
\pgfpathcurveto{\pgfqpoint{1.067573in}{0.717111in}}{\pgfqpoint{1.063182in}{0.727710in}}{\pgfqpoint{1.055369in}{0.735523in}}%
\pgfpathcurveto{\pgfqpoint{1.047555in}{0.743337in}}{\pgfqpoint{1.036956in}{0.747727in}}{\pgfqpoint{1.025906in}{0.747727in}}%
\pgfpathcurveto{\pgfqpoint{1.014856in}{0.747727in}}{\pgfqpoint{1.004257in}{0.743337in}}{\pgfqpoint{0.996443in}{0.735523in}}%
\pgfpathcurveto{\pgfqpoint{0.988630in}{0.727710in}}{\pgfqpoint{0.984239in}{0.717111in}}{\pgfqpoint{0.984239in}{0.706060in}}%
\pgfpathcurveto{\pgfqpoint{0.984239in}{0.695010in}}{\pgfqpoint{0.988630in}{0.684411in}}{\pgfqpoint{0.996443in}{0.676598in}}%
\pgfpathcurveto{\pgfqpoint{1.004257in}{0.668784in}}{\pgfqpoint{1.014856in}{0.664394in}}{\pgfqpoint{1.025906in}{0.664394in}}%
\pgfpathclose%
\pgfusepath{stroke,fill}%
\end{pgfscope}%
\begin{pgfscope}%
\pgfpathrectangle{\pgfqpoint{0.800000in}{0.528000in}}{\pgfqpoint{4.960000in}{3.696000in}}%
\pgfusepath{clip}%
\pgfsetbuttcap%
\pgfsetroundjoin%
\definecolor{currentfill}{rgb}{0.000000,0.000000,0.000000}%
\pgfsetfillcolor{currentfill}%
\pgfsetlinewidth{1.003750pt}%
\definecolor{currentstroke}{rgb}{0.000000,0.000000,0.000000}%
\pgfsetstrokecolor{currentstroke}%
\pgfsetdash{}{0pt}%
\pgfpathmoveto{\pgfqpoint{1.025906in}{0.664394in}}%
\pgfpathcurveto{\pgfqpoint{1.036956in}{0.664394in}}{\pgfqpoint{1.047555in}{0.668784in}}{\pgfqpoint{1.055369in}{0.676598in}}%
\pgfpathcurveto{\pgfqpoint{1.063182in}{0.684411in}}{\pgfqpoint{1.067573in}{0.695010in}}{\pgfqpoint{1.067573in}{0.706060in}}%
\pgfpathcurveto{\pgfqpoint{1.067573in}{0.717111in}}{\pgfqpoint{1.063182in}{0.727710in}}{\pgfqpoint{1.055369in}{0.735523in}}%
\pgfpathcurveto{\pgfqpoint{1.047555in}{0.743337in}}{\pgfqpoint{1.036956in}{0.747727in}}{\pgfqpoint{1.025906in}{0.747727in}}%
\pgfpathcurveto{\pgfqpoint{1.014856in}{0.747727in}}{\pgfqpoint{1.004257in}{0.743337in}}{\pgfqpoint{0.996443in}{0.735523in}}%
\pgfpathcurveto{\pgfqpoint{0.988630in}{0.727710in}}{\pgfqpoint{0.984239in}{0.717111in}}{\pgfqpoint{0.984239in}{0.706060in}}%
\pgfpathcurveto{\pgfqpoint{0.984239in}{0.695010in}}{\pgfqpoint{0.988630in}{0.684411in}}{\pgfqpoint{0.996443in}{0.676598in}}%
\pgfpathcurveto{\pgfqpoint{1.004257in}{0.668784in}}{\pgfqpoint{1.014856in}{0.664394in}}{\pgfqpoint{1.025906in}{0.664394in}}%
\pgfpathclose%
\pgfusepath{stroke,fill}%
\end{pgfscope}%
\begin{pgfscope}%
\pgfpathrectangle{\pgfqpoint{0.800000in}{0.528000in}}{\pgfqpoint{4.960000in}{3.696000in}}%
\pgfusepath{clip}%
\pgfsetbuttcap%
\pgfsetroundjoin%
\definecolor{currentfill}{rgb}{0.000000,0.000000,0.000000}%
\pgfsetfillcolor{currentfill}%
\pgfsetlinewidth{1.003750pt}%
\definecolor{currentstroke}{rgb}{0.000000,0.000000,0.000000}%
\pgfsetstrokecolor{currentstroke}%
\pgfsetdash{}{0pt}%
\pgfpathmoveto{\pgfqpoint{1.025906in}{0.664394in}}%
\pgfpathcurveto{\pgfqpoint{1.036956in}{0.664394in}}{\pgfqpoint{1.047555in}{0.668784in}}{\pgfqpoint{1.055369in}{0.676598in}}%
\pgfpathcurveto{\pgfqpoint{1.063182in}{0.684411in}}{\pgfqpoint{1.067573in}{0.695010in}}{\pgfqpoint{1.067573in}{0.706060in}}%
\pgfpathcurveto{\pgfqpoint{1.067573in}{0.717111in}}{\pgfqpoint{1.063182in}{0.727710in}}{\pgfqpoint{1.055369in}{0.735523in}}%
\pgfpathcurveto{\pgfqpoint{1.047555in}{0.743337in}}{\pgfqpoint{1.036956in}{0.747727in}}{\pgfqpoint{1.025906in}{0.747727in}}%
\pgfpathcurveto{\pgfqpoint{1.014856in}{0.747727in}}{\pgfqpoint{1.004257in}{0.743337in}}{\pgfqpoint{0.996443in}{0.735523in}}%
\pgfpathcurveto{\pgfqpoint{0.988630in}{0.727710in}}{\pgfqpoint{0.984239in}{0.717111in}}{\pgfqpoint{0.984239in}{0.706060in}}%
\pgfpathcurveto{\pgfqpoint{0.984239in}{0.695010in}}{\pgfqpoint{0.988630in}{0.684411in}}{\pgfqpoint{0.996443in}{0.676598in}}%
\pgfpathcurveto{\pgfqpoint{1.004257in}{0.668784in}}{\pgfqpoint{1.014856in}{0.664394in}}{\pgfqpoint{1.025906in}{0.664394in}}%
\pgfpathclose%
\pgfusepath{stroke,fill}%
\end{pgfscope}%
\begin{pgfscope}%
\pgfpathrectangle{\pgfqpoint{0.800000in}{0.528000in}}{\pgfqpoint{4.960000in}{3.696000in}}%
\pgfusepath{clip}%
\pgfsetbuttcap%
\pgfsetroundjoin%
\definecolor{currentfill}{rgb}{0.000000,0.000000,0.000000}%
\pgfsetfillcolor{currentfill}%
\pgfsetlinewidth{1.003750pt}%
\definecolor{currentstroke}{rgb}{0.000000,0.000000,0.000000}%
\pgfsetstrokecolor{currentstroke}%
\pgfsetdash{}{0pt}%
\pgfpathmoveto{\pgfqpoint{1.025906in}{0.664394in}}%
\pgfpathcurveto{\pgfqpoint{1.036956in}{0.664394in}}{\pgfqpoint{1.047555in}{0.668784in}}{\pgfqpoint{1.055369in}{0.676598in}}%
\pgfpathcurveto{\pgfqpoint{1.063182in}{0.684411in}}{\pgfqpoint{1.067573in}{0.695010in}}{\pgfqpoint{1.067573in}{0.706060in}}%
\pgfpathcurveto{\pgfqpoint{1.067573in}{0.717111in}}{\pgfqpoint{1.063182in}{0.727710in}}{\pgfqpoint{1.055369in}{0.735523in}}%
\pgfpathcurveto{\pgfqpoint{1.047555in}{0.743337in}}{\pgfqpoint{1.036956in}{0.747727in}}{\pgfqpoint{1.025906in}{0.747727in}}%
\pgfpathcurveto{\pgfqpoint{1.014856in}{0.747727in}}{\pgfqpoint{1.004257in}{0.743337in}}{\pgfqpoint{0.996443in}{0.735523in}}%
\pgfpathcurveto{\pgfqpoint{0.988630in}{0.727710in}}{\pgfqpoint{0.984239in}{0.717111in}}{\pgfqpoint{0.984239in}{0.706060in}}%
\pgfpathcurveto{\pgfqpoint{0.984239in}{0.695010in}}{\pgfqpoint{0.988630in}{0.684411in}}{\pgfqpoint{0.996443in}{0.676598in}}%
\pgfpathcurveto{\pgfqpoint{1.004257in}{0.668784in}}{\pgfqpoint{1.014856in}{0.664394in}}{\pgfqpoint{1.025906in}{0.664394in}}%
\pgfpathclose%
\pgfusepath{stroke,fill}%
\end{pgfscope}%
\begin{pgfscope}%
\pgfpathrectangle{\pgfqpoint{0.800000in}{0.528000in}}{\pgfqpoint{4.960000in}{3.696000in}}%
\pgfusepath{clip}%
\pgfsetbuttcap%
\pgfsetroundjoin%
\definecolor{currentfill}{rgb}{0.000000,0.000000,0.000000}%
\pgfsetfillcolor{currentfill}%
\pgfsetlinewidth{1.003750pt}%
\definecolor{currentstroke}{rgb}{0.000000,0.000000,0.000000}%
\pgfsetstrokecolor{currentstroke}%
\pgfsetdash{}{0pt}%
\pgfpathmoveto{\pgfqpoint{1.025906in}{0.664394in}}%
\pgfpathcurveto{\pgfqpoint{1.036956in}{0.664394in}}{\pgfqpoint{1.047555in}{0.668784in}}{\pgfqpoint{1.055369in}{0.676598in}}%
\pgfpathcurveto{\pgfqpoint{1.063182in}{0.684411in}}{\pgfqpoint{1.067573in}{0.695010in}}{\pgfqpoint{1.067573in}{0.706060in}}%
\pgfpathcurveto{\pgfqpoint{1.067573in}{0.717111in}}{\pgfqpoint{1.063182in}{0.727710in}}{\pgfqpoint{1.055369in}{0.735523in}}%
\pgfpathcurveto{\pgfqpoint{1.047555in}{0.743337in}}{\pgfqpoint{1.036956in}{0.747727in}}{\pgfqpoint{1.025906in}{0.747727in}}%
\pgfpathcurveto{\pgfqpoint{1.014856in}{0.747727in}}{\pgfqpoint{1.004257in}{0.743337in}}{\pgfqpoint{0.996443in}{0.735523in}}%
\pgfpathcurveto{\pgfqpoint{0.988630in}{0.727710in}}{\pgfqpoint{0.984239in}{0.717111in}}{\pgfqpoint{0.984239in}{0.706060in}}%
\pgfpathcurveto{\pgfqpoint{0.984239in}{0.695010in}}{\pgfqpoint{0.988630in}{0.684411in}}{\pgfqpoint{0.996443in}{0.676598in}}%
\pgfpathcurveto{\pgfqpoint{1.004257in}{0.668784in}}{\pgfqpoint{1.014856in}{0.664394in}}{\pgfqpoint{1.025906in}{0.664394in}}%
\pgfpathclose%
\pgfusepath{stroke,fill}%
\end{pgfscope}%
\begin{pgfscope}%
\pgfpathrectangle{\pgfqpoint{0.800000in}{0.528000in}}{\pgfqpoint{4.960000in}{3.696000in}}%
\pgfusepath{clip}%
\pgfsetbuttcap%
\pgfsetroundjoin%
\definecolor{currentfill}{rgb}{0.000000,0.000000,0.000000}%
\pgfsetfillcolor{currentfill}%
\pgfsetlinewidth{1.003750pt}%
\definecolor{currentstroke}{rgb}{0.000000,0.000000,0.000000}%
\pgfsetstrokecolor{currentstroke}%
\pgfsetdash{}{0pt}%
\pgfpathmoveto{\pgfqpoint{1.025906in}{0.664394in}}%
\pgfpathcurveto{\pgfqpoint{1.036956in}{0.664394in}}{\pgfqpoint{1.047555in}{0.668784in}}{\pgfqpoint{1.055369in}{0.676598in}}%
\pgfpathcurveto{\pgfqpoint{1.063182in}{0.684411in}}{\pgfqpoint{1.067573in}{0.695010in}}{\pgfqpoint{1.067573in}{0.706060in}}%
\pgfpathcurveto{\pgfqpoint{1.067573in}{0.717111in}}{\pgfqpoint{1.063182in}{0.727710in}}{\pgfqpoint{1.055369in}{0.735523in}}%
\pgfpathcurveto{\pgfqpoint{1.047555in}{0.743337in}}{\pgfqpoint{1.036956in}{0.747727in}}{\pgfqpoint{1.025906in}{0.747727in}}%
\pgfpathcurveto{\pgfqpoint{1.014856in}{0.747727in}}{\pgfqpoint{1.004257in}{0.743337in}}{\pgfqpoint{0.996443in}{0.735523in}}%
\pgfpathcurveto{\pgfqpoint{0.988630in}{0.727710in}}{\pgfqpoint{0.984239in}{0.717111in}}{\pgfqpoint{0.984239in}{0.706060in}}%
\pgfpathcurveto{\pgfqpoint{0.984239in}{0.695010in}}{\pgfqpoint{0.988630in}{0.684411in}}{\pgfqpoint{0.996443in}{0.676598in}}%
\pgfpathcurveto{\pgfqpoint{1.004257in}{0.668784in}}{\pgfqpoint{1.014856in}{0.664394in}}{\pgfqpoint{1.025906in}{0.664394in}}%
\pgfpathclose%
\pgfusepath{stroke,fill}%
\end{pgfscope}%
\begin{pgfscope}%
\pgfpathrectangle{\pgfqpoint{0.800000in}{0.528000in}}{\pgfqpoint{4.960000in}{3.696000in}}%
\pgfusepath{clip}%
\pgfsetbuttcap%
\pgfsetroundjoin%
\definecolor{currentfill}{rgb}{0.000000,0.000000,0.000000}%
\pgfsetfillcolor{currentfill}%
\pgfsetlinewidth{1.003750pt}%
\definecolor{currentstroke}{rgb}{0.000000,0.000000,0.000000}%
\pgfsetstrokecolor{currentstroke}%
\pgfsetdash{}{0pt}%
\pgfpathmoveto{\pgfqpoint{1.025906in}{0.664394in}}%
\pgfpathcurveto{\pgfqpoint{1.036956in}{0.664394in}}{\pgfqpoint{1.047555in}{0.668784in}}{\pgfqpoint{1.055369in}{0.676598in}}%
\pgfpathcurveto{\pgfqpoint{1.063182in}{0.684411in}}{\pgfqpoint{1.067573in}{0.695010in}}{\pgfqpoint{1.067573in}{0.706060in}}%
\pgfpathcurveto{\pgfqpoint{1.067573in}{0.717111in}}{\pgfqpoint{1.063182in}{0.727710in}}{\pgfqpoint{1.055369in}{0.735523in}}%
\pgfpathcurveto{\pgfqpoint{1.047555in}{0.743337in}}{\pgfqpoint{1.036956in}{0.747727in}}{\pgfqpoint{1.025906in}{0.747727in}}%
\pgfpathcurveto{\pgfqpoint{1.014856in}{0.747727in}}{\pgfqpoint{1.004257in}{0.743337in}}{\pgfqpoint{0.996443in}{0.735523in}}%
\pgfpathcurveto{\pgfqpoint{0.988630in}{0.727710in}}{\pgfqpoint{0.984239in}{0.717111in}}{\pgfqpoint{0.984239in}{0.706060in}}%
\pgfpathcurveto{\pgfqpoint{0.984239in}{0.695010in}}{\pgfqpoint{0.988630in}{0.684411in}}{\pgfqpoint{0.996443in}{0.676598in}}%
\pgfpathcurveto{\pgfqpoint{1.004257in}{0.668784in}}{\pgfqpoint{1.014856in}{0.664394in}}{\pgfqpoint{1.025906in}{0.664394in}}%
\pgfpathclose%
\pgfusepath{stroke,fill}%
\end{pgfscope}%
\begin{pgfscope}%
\pgfpathrectangle{\pgfqpoint{0.800000in}{0.528000in}}{\pgfqpoint{4.960000in}{3.696000in}}%
\pgfusepath{clip}%
\pgfsetbuttcap%
\pgfsetroundjoin%
\definecolor{currentfill}{rgb}{0.000000,0.000000,0.000000}%
\pgfsetfillcolor{currentfill}%
\pgfsetlinewidth{1.003750pt}%
\definecolor{currentstroke}{rgb}{0.000000,0.000000,0.000000}%
\pgfsetstrokecolor{currentstroke}%
\pgfsetdash{}{0pt}%
\pgfpathmoveto{\pgfqpoint{1.025906in}{0.664394in}}%
\pgfpathcurveto{\pgfqpoint{1.036956in}{0.664394in}}{\pgfqpoint{1.047555in}{0.668784in}}{\pgfqpoint{1.055369in}{0.676598in}}%
\pgfpathcurveto{\pgfqpoint{1.063182in}{0.684411in}}{\pgfqpoint{1.067573in}{0.695010in}}{\pgfqpoint{1.067573in}{0.706060in}}%
\pgfpathcurveto{\pgfqpoint{1.067573in}{0.717111in}}{\pgfqpoint{1.063182in}{0.727710in}}{\pgfqpoint{1.055369in}{0.735523in}}%
\pgfpathcurveto{\pgfqpoint{1.047555in}{0.743337in}}{\pgfqpoint{1.036956in}{0.747727in}}{\pgfqpoint{1.025906in}{0.747727in}}%
\pgfpathcurveto{\pgfqpoint{1.014856in}{0.747727in}}{\pgfqpoint{1.004257in}{0.743337in}}{\pgfqpoint{0.996443in}{0.735523in}}%
\pgfpathcurveto{\pgfqpoint{0.988630in}{0.727710in}}{\pgfqpoint{0.984239in}{0.717111in}}{\pgfqpoint{0.984239in}{0.706060in}}%
\pgfpathcurveto{\pgfqpoint{0.984239in}{0.695010in}}{\pgfqpoint{0.988630in}{0.684411in}}{\pgfqpoint{0.996443in}{0.676598in}}%
\pgfpathcurveto{\pgfqpoint{1.004257in}{0.668784in}}{\pgfqpoint{1.014856in}{0.664394in}}{\pgfqpoint{1.025906in}{0.664394in}}%
\pgfpathclose%
\pgfusepath{stroke,fill}%
\end{pgfscope}%
\begin{pgfscope}%
\pgfpathrectangle{\pgfqpoint{0.800000in}{0.528000in}}{\pgfqpoint{4.960000in}{3.696000in}}%
\pgfusepath{clip}%
\pgfsetbuttcap%
\pgfsetroundjoin%
\definecolor{currentfill}{rgb}{0.000000,0.000000,0.000000}%
\pgfsetfillcolor{currentfill}%
\pgfsetlinewidth{1.003750pt}%
\definecolor{currentstroke}{rgb}{0.000000,0.000000,0.000000}%
\pgfsetstrokecolor{currentstroke}%
\pgfsetdash{}{0pt}%
\pgfpathmoveto{\pgfqpoint{1.025906in}{0.664394in}}%
\pgfpathcurveto{\pgfqpoint{1.036956in}{0.664394in}}{\pgfqpoint{1.047555in}{0.668784in}}{\pgfqpoint{1.055369in}{0.676598in}}%
\pgfpathcurveto{\pgfqpoint{1.063182in}{0.684411in}}{\pgfqpoint{1.067573in}{0.695010in}}{\pgfqpoint{1.067573in}{0.706060in}}%
\pgfpathcurveto{\pgfqpoint{1.067573in}{0.717111in}}{\pgfqpoint{1.063182in}{0.727710in}}{\pgfqpoint{1.055369in}{0.735523in}}%
\pgfpathcurveto{\pgfqpoint{1.047555in}{0.743337in}}{\pgfqpoint{1.036956in}{0.747727in}}{\pgfqpoint{1.025906in}{0.747727in}}%
\pgfpathcurveto{\pgfqpoint{1.014856in}{0.747727in}}{\pgfqpoint{1.004257in}{0.743337in}}{\pgfqpoint{0.996443in}{0.735523in}}%
\pgfpathcurveto{\pgfqpoint{0.988630in}{0.727710in}}{\pgfqpoint{0.984239in}{0.717111in}}{\pgfqpoint{0.984239in}{0.706060in}}%
\pgfpathcurveto{\pgfqpoint{0.984239in}{0.695010in}}{\pgfqpoint{0.988630in}{0.684411in}}{\pgfqpoint{0.996443in}{0.676598in}}%
\pgfpathcurveto{\pgfqpoint{1.004257in}{0.668784in}}{\pgfqpoint{1.014856in}{0.664394in}}{\pgfqpoint{1.025906in}{0.664394in}}%
\pgfpathclose%
\pgfusepath{stroke,fill}%
\end{pgfscope}%
\begin{pgfscope}%
\pgfpathrectangle{\pgfqpoint{0.800000in}{0.528000in}}{\pgfqpoint{4.960000in}{3.696000in}}%
\pgfusepath{clip}%
\pgfsetbuttcap%
\pgfsetroundjoin%
\definecolor{currentfill}{rgb}{0.000000,0.000000,0.000000}%
\pgfsetfillcolor{currentfill}%
\pgfsetlinewidth{1.003750pt}%
\definecolor{currentstroke}{rgb}{0.000000,0.000000,0.000000}%
\pgfsetstrokecolor{currentstroke}%
\pgfsetdash{}{0pt}%
\pgfpathmoveto{\pgfqpoint{1.025906in}{0.664394in}}%
\pgfpathcurveto{\pgfqpoint{1.036956in}{0.664394in}}{\pgfqpoint{1.047555in}{0.668784in}}{\pgfqpoint{1.055369in}{0.676598in}}%
\pgfpathcurveto{\pgfqpoint{1.063182in}{0.684411in}}{\pgfqpoint{1.067573in}{0.695010in}}{\pgfqpoint{1.067573in}{0.706060in}}%
\pgfpathcurveto{\pgfqpoint{1.067573in}{0.717111in}}{\pgfqpoint{1.063182in}{0.727710in}}{\pgfqpoint{1.055369in}{0.735523in}}%
\pgfpathcurveto{\pgfqpoint{1.047555in}{0.743337in}}{\pgfqpoint{1.036956in}{0.747727in}}{\pgfqpoint{1.025906in}{0.747727in}}%
\pgfpathcurveto{\pgfqpoint{1.014856in}{0.747727in}}{\pgfqpoint{1.004257in}{0.743337in}}{\pgfqpoint{0.996443in}{0.735523in}}%
\pgfpathcurveto{\pgfqpoint{0.988630in}{0.727710in}}{\pgfqpoint{0.984239in}{0.717111in}}{\pgfqpoint{0.984239in}{0.706060in}}%
\pgfpathcurveto{\pgfqpoint{0.984239in}{0.695010in}}{\pgfqpoint{0.988630in}{0.684411in}}{\pgfqpoint{0.996443in}{0.676598in}}%
\pgfpathcurveto{\pgfqpoint{1.004257in}{0.668784in}}{\pgfqpoint{1.014856in}{0.664394in}}{\pgfqpoint{1.025906in}{0.664394in}}%
\pgfpathclose%
\pgfusepath{stroke,fill}%
\end{pgfscope}%
\begin{pgfscope}%
\pgfpathrectangle{\pgfqpoint{0.800000in}{0.528000in}}{\pgfqpoint{4.960000in}{3.696000in}}%
\pgfusepath{clip}%
\pgfsetbuttcap%
\pgfsetroundjoin%
\definecolor{currentfill}{rgb}{0.000000,0.000000,0.000000}%
\pgfsetfillcolor{currentfill}%
\pgfsetlinewidth{1.003750pt}%
\definecolor{currentstroke}{rgb}{0.000000,0.000000,0.000000}%
\pgfsetstrokecolor{currentstroke}%
\pgfsetdash{}{0pt}%
\pgfpathmoveto{\pgfqpoint{1.025906in}{0.664394in}}%
\pgfpathcurveto{\pgfqpoint{1.036956in}{0.664394in}}{\pgfqpoint{1.047555in}{0.668784in}}{\pgfqpoint{1.055369in}{0.676598in}}%
\pgfpathcurveto{\pgfqpoint{1.063182in}{0.684411in}}{\pgfqpoint{1.067573in}{0.695010in}}{\pgfqpoint{1.067573in}{0.706060in}}%
\pgfpathcurveto{\pgfqpoint{1.067573in}{0.717111in}}{\pgfqpoint{1.063182in}{0.727710in}}{\pgfqpoint{1.055369in}{0.735523in}}%
\pgfpathcurveto{\pgfqpoint{1.047555in}{0.743337in}}{\pgfqpoint{1.036956in}{0.747727in}}{\pgfqpoint{1.025906in}{0.747727in}}%
\pgfpathcurveto{\pgfqpoint{1.014856in}{0.747727in}}{\pgfqpoint{1.004257in}{0.743337in}}{\pgfqpoint{0.996443in}{0.735523in}}%
\pgfpathcurveto{\pgfqpoint{0.988630in}{0.727710in}}{\pgfqpoint{0.984239in}{0.717111in}}{\pgfqpoint{0.984239in}{0.706060in}}%
\pgfpathcurveto{\pgfqpoint{0.984239in}{0.695010in}}{\pgfqpoint{0.988630in}{0.684411in}}{\pgfqpoint{0.996443in}{0.676598in}}%
\pgfpathcurveto{\pgfqpoint{1.004257in}{0.668784in}}{\pgfqpoint{1.014856in}{0.664394in}}{\pgfqpoint{1.025906in}{0.664394in}}%
\pgfpathclose%
\pgfusepath{stroke,fill}%
\end{pgfscope}%
\begin{pgfscope}%
\pgfpathrectangle{\pgfqpoint{0.800000in}{0.528000in}}{\pgfqpoint{4.960000in}{3.696000in}}%
\pgfusepath{clip}%
\pgfsetbuttcap%
\pgfsetroundjoin%
\definecolor{currentfill}{rgb}{0.000000,0.000000,0.000000}%
\pgfsetfillcolor{currentfill}%
\pgfsetlinewidth{1.003750pt}%
\definecolor{currentstroke}{rgb}{0.000000,0.000000,0.000000}%
\pgfsetstrokecolor{currentstroke}%
\pgfsetdash{}{0pt}%
\pgfpathmoveto{\pgfqpoint{1.025906in}{0.664394in}}%
\pgfpathcurveto{\pgfqpoint{1.036956in}{0.664394in}}{\pgfqpoint{1.047555in}{0.668784in}}{\pgfqpoint{1.055369in}{0.676598in}}%
\pgfpathcurveto{\pgfqpoint{1.063182in}{0.684411in}}{\pgfqpoint{1.067573in}{0.695010in}}{\pgfqpoint{1.067573in}{0.706060in}}%
\pgfpathcurveto{\pgfqpoint{1.067573in}{0.717111in}}{\pgfqpoint{1.063182in}{0.727710in}}{\pgfqpoint{1.055369in}{0.735523in}}%
\pgfpathcurveto{\pgfqpoint{1.047555in}{0.743337in}}{\pgfqpoint{1.036956in}{0.747727in}}{\pgfqpoint{1.025906in}{0.747727in}}%
\pgfpathcurveto{\pgfqpoint{1.014856in}{0.747727in}}{\pgfqpoint{1.004257in}{0.743337in}}{\pgfqpoint{0.996443in}{0.735523in}}%
\pgfpathcurveto{\pgfqpoint{0.988630in}{0.727710in}}{\pgfqpoint{0.984239in}{0.717111in}}{\pgfqpoint{0.984239in}{0.706060in}}%
\pgfpathcurveto{\pgfqpoint{0.984239in}{0.695010in}}{\pgfqpoint{0.988630in}{0.684411in}}{\pgfqpoint{0.996443in}{0.676598in}}%
\pgfpathcurveto{\pgfqpoint{1.004257in}{0.668784in}}{\pgfqpoint{1.014856in}{0.664394in}}{\pgfqpoint{1.025906in}{0.664394in}}%
\pgfpathclose%
\pgfusepath{stroke,fill}%
\end{pgfscope}%
\begin{pgfscope}%
\pgfpathrectangle{\pgfqpoint{0.800000in}{0.528000in}}{\pgfqpoint{4.960000in}{3.696000in}}%
\pgfusepath{clip}%
\pgfsetbuttcap%
\pgfsetroundjoin%
\definecolor{currentfill}{rgb}{0.000000,0.000000,0.000000}%
\pgfsetfillcolor{currentfill}%
\pgfsetlinewidth{1.003750pt}%
\definecolor{currentstroke}{rgb}{0.000000,0.000000,0.000000}%
\pgfsetstrokecolor{currentstroke}%
\pgfsetdash{}{0pt}%
\pgfpathmoveto{\pgfqpoint{1.025906in}{0.664394in}}%
\pgfpathcurveto{\pgfqpoint{1.036956in}{0.664394in}}{\pgfqpoint{1.047555in}{0.668784in}}{\pgfqpoint{1.055369in}{0.676598in}}%
\pgfpathcurveto{\pgfqpoint{1.063182in}{0.684411in}}{\pgfqpoint{1.067573in}{0.695010in}}{\pgfqpoint{1.067573in}{0.706060in}}%
\pgfpathcurveto{\pgfqpoint{1.067573in}{0.717111in}}{\pgfqpoint{1.063182in}{0.727710in}}{\pgfqpoint{1.055369in}{0.735523in}}%
\pgfpathcurveto{\pgfqpoint{1.047555in}{0.743337in}}{\pgfqpoint{1.036956in}{0.747727in}}{\pgfqpoint{1.025906in}{0.747727in}}%
\pgfpathcurveto{\pgfqpoint{1.014856in}{0.747727in}}{\pgfqpoint{1.004257in}{0.743337in}}{\pgfqpoint{0.996443in}{0.735523in}}%
\pgfpathcurveto{\pgfqpoint{0.988630in}{0.727710in}}{\pgfqpoint{0.984239in}{0.717111in}}{\pgfqpoint{0.984239in}{0.706060in}}%
\pgfpathcurveto{\pgfqpoint{0.984239in}{0.695010in}}{\pgfqpoint{0.988630in}{0.684411in}}{\pgfqpoint{0.996443in}{0.676598in}}%
\pgfpathcurveto{\pgfqpoint{1.004257in}{0.668784in}}{\pgfqpoint{1.014856in}{0.664394in}}{\pgfqpoint{1.025906in}{0.664394in}}%
\pgfpathclose%
\pgfusepath{stroke,fill}%
\end{pgfscope}%
\begin{pgfscope}%
\pgfpathrectangle{\pgfqpoint{0.800000in}{0.528000in}}{\pgfqpoint{4.960000in}{3.696000in}}%
\pgfusepath{clip}%
\pgfsetbuttcap%
\pgfsetroundjoin%
\definecolor{currentfill}{rgb}{0.000000,0.000000,0.000000}%
\pgfsetfillcolor{currentfill}%
\pgfsetlinewidth{1.003750pt}%
\definecolor{currentstroke}{rgb}{0.000000,0.000000,0.000000}%
\pgfsetstrokecolor{currentstroke}%
\pgfsetdash{}{0pt}%
\pgfpathmoveto{\pgfqpoint{1.025906in}{0.664394in}}%
\pgfpathcurveto{\pgfqpoint{1.036956in}{0.664394in}}{\pgfqpoint{1.047555in}{0.668784in}}{\pgfqpoint{1.055369in}{0.676598in}}%
\pgfpathcurveto{\pgfqpoint{1.063182in}{0.684411in}}{\pgfqpoint{1.067573in}{0.695010in}}{\pgfqpoint{1.067573in}{0.706060in}}%
\pgfpathcurveto{\pgfqpoint{1.067573in}{0.717111in}}{\pgfqpoint{1.063182in}{0.727710in}}{\pgfqpoint{1.055369in}{0.735523in}}%
\pgfpathcurveto{\pgfqpoint{1.047555in}{0.743337in}}{\pgfqpoint{1.036956in}{0.747727in}}{\pgfqpoint{1.025906in}{0.747727in}}%
\pgfpathcurveto{\pgfqpoint{1.014856in}{0.747727in}}{\pgfqpoint{1.004257in}{0.743337in}}{\pgfqpoint{0.996443in}{0.735523in}}%
\pgfpathcurveto{\pgfqpoint{0.988630in}{0.727710in}}{\pgfqpoint{0.984239in}{0.717111in}}{\pgfqpoint{0.984239in}{0.706060in}}%
\pgfpathcurveto{\pgfqpoint{0.984239in}{0.695010in}}{\pgfqpoint{0.988630in}{0.684411in}}{\pgfqpoint{0.996443in}{0.676598in}}%
\pgfpathcurveto{\pgfqpoint{1.004257in}{0.668784in}}{\pgfqpoint{1.014856in}{0.664394in}}{\pgfqpoint{1.025906in}{0.664394in}}%
\pgfpathclose%
\pgfusepath{stroke,fill}%
\end{pgfscope}%
\begin{pgfscope}%
\pgfpathrectangle{\pgfqpoint{0.800000in}{0.528000in}}{\pgfqpoint{4.960000in}{3.696000in}}%
\pgfusepath{clip}%
\pgfsetbuttcap%
\pgfsetroundjoin%
\definecolor{currentfill}{rgb}{0.000000,0.000000,0.000000}%
\pgfsetfillcolor{currentfill}%
\pgfsetlinewidth{1.003750pt}%
\definecolor{currentstroke}{rgb}{0.000000,0.000000,0.000000}%
\pgfsetstrokecolor{currentstroke}%
\pgfsetdash{}{0pt}%
\pgfpathmoveto{\pgfqpoint{1.025906in}{0.664394in}}%
\pgfpathcurveto{\pgfqpoint{1.036956in}{0.664394in}}{\pgfqpoint{1.047555in}{0.668784in}}{\pgfqpoint{1.055369in}{0.676598in}}%
\pgfpathcurveto{\pgfqpoint{1.063182in}{0.684411in}}{\pgfqpoint{1.067573in}{0.695010in}}{\pgfqpoint{1.067573in}{0.706060in}}%
\pgfpathcurveto{\pgfqpoint{1.067573in}{0.717111in}}{\pgfqpoint{1.063182in}{0.727710in}}{\pgfqpoint{1.055369in}{0.735523in}}%
\pgfpathcurveto{\pgfqpoint{1.047555in}{0.743337in}}{\pgfqpoint{1.036956in}{0.747727in}}{\pgfqpoint{1.025906in}{0.747727in}}%
\pgfpathcurveto{\pgfqpoint{1.014856in}{0.747727in}}{\pgfqpoint{1.004257in}{0.743337in}}{\pgfqpoint{0.996443in}{0.735523in}}%
\pgfpathcurveto{\pgfqpoint{0.988630in}{0.727710in}}{\pgfqpoint{0.984239in}{0.717111in}}{\pgfqpoint{0.984239in}{0.706060in}}%
\pgfpathcurveto{\pgfqpoint{0.984239in}{0.695010in}}{\pgfqpoint{0.988630in}{0.684411in}}{\pgfqpoint{0.996443in}{0.676598in}}%
\pgfpathcurveto{\pgfqpoint{1.004257in}{0.668784in}}{\pgfqpoint{1.014856in}{0.664394in}}{\pgfqpoint{1.025906in}{0.664394in}}%
\pgfpathclose%
\pgfusepath{stroke,fill}%
\end{pgfscope}%
\begin{pgfscope}%
\pgfpathrectangle{\pgfqpoint{0.800000in}{0.528000in}}{\pgfqpoint{4.960000in}{3.696000in}}%
\pgfusepath{clip}%
\pgfsetbuttcap%
\pgfsetroundjoin%
\definecolor{currentfill}{rgb}{0.000000,0.000000,0.000000}%
\pgfsetfillcolor{currentfill}%
\pgfsetlinewidth{1.003750pt}%
\definecolor{currentstroke}{rgb}{0.000000,0.000000,0.000000}%
\pgfsetstrokecolor{currentstroke}%
\pgfsetdash{}{0pt}%
\pgfpathmoveto{\pgfqpoint{1.025906in}{0.664394in}}%
\pgfpathcurveto{\pgfqpoint{1.036956in}{0.664394in}}{\pgfqpoint{1.047555in}{0.668784in}}{\pgfqpoint{1.055369in}{0.676598in}}%
\pgfpathcurveto{\pgfqpoint{1.063182in}{0.684411in}}{\pgfqpoint{1.067573in}{0.695010in}}{\pgfqpoint{1.067573in}{0.706060in}}%
\pgfpathcurveto{\pgfqpoint{1.067573in}{0.717111in}}{\pgfqpoint{1.063182in}{0.727710in}}{\pgfqpoint{1.055369in}{0.735523in}}%
\pgfpathcurveto{\pgfqpoint{1.047555in}{0.743337in}}{\pgfqpoint{1.036956in}{0.747727in}}{\pgfqpoint{1.025906in}{0.747727in}}%
\pgfpathcurveto{\pgfqpoint{1.014856in}{0.747727in}}{\pgfqpoint{1.004257in}{0.743337in}}{\pgfqpoint{0.996443in}{0.735523in}}%
\pgfpathcurveto{\pgfqpoint{0.988630in}{0.727710in}}{\pgfqpoint{0.984239in}{0.717111in}}{\pgfqpoint{0.984239in}{0.706060in}}%
\pgfpathcurveto{\pgfqpoint{0.984239in}{0.695010in}}{\pgfqpoint{0.988630in}{0.684411in}}{\pgfqpoint{0.996443in}{0.676598in}}%
\pgfpathcurveto{\pgfqpoint{1.004257in}{0.668784in}}{\pgfqpoint{1.014856in}{0.664394in}}{\pgfqpoint{1.025906in}{0.664394in}}%
\pgfpathclose%
\pgfusepath{stroke,fill}%
\end{pgfscope}%
\begin{pgfscope}%
\pgfpathrectangle{\pgfqpoint{0.800000in}{0.528000in}}{\pgfqpoint{4.960000in}{3.696000in}}%
\pgfusepath{clip}%
\pgfsetbuttcap%
\pgfsetroundjoin%
\definecolor{currentfill}{rgb}{0.000000,0.000000,0.000000}%
\pgfsetfillcolor{currentfill}%
\pgfsetlinewidth{1.003750pt}%
\definecolor{currentstroke}{rgb}{0.000000,0.000000,0.000000}%
\pgfsetstrokecolor{currentstroke}%
\pgfsetdash{}{0pt}%
\pgfpathmoveto{\pgfqpoint{1.025906in}{0.664394in}}%
\pgfpathcurveto{\pgfqpoint{1.036956in}{0.664394in}}{\pgfqpoint{1.047555in}{0.668784in}}{\pgfqpoint{1.055369in}{0.676598in}}%
\pgfpathcurveto{\pgfqpoint{1.063182in}{0.684411in}}{\pgfqpoint{1.067573in}{0.695010in}}{\pgfqpoint{1.067573in}{0.706060in}}%
\pgfpathcurveto{\pgfqpoint{1.067573in}{0.717111in}}{\pgfqpoint{1.063182in}{0.727710in}}{\pgfqpoint{1.055369in}{0.735523in}}%
\pgfpathcurveto{\pgfqpoint{1.047555in}{0.743337in}}{\pgfqpoint{1.036956in}{0.747727in}}{\pgfqpoint{1.025906in}{0.747727in}}%
\pgfpathcurveto{\pgfqpoint{1.014856in}{0.747727in}}{\pgfqpoint{1.004257in}{0.743337in}}{\pgfqpoint{0.996443in}{0.735523in}}%
\pgfpathcurveto{\pgfqpoint{0.988630in}{0.727710in}}{\pgfqpoint{0.984239in}{0.717111in}}{\pgfqpoint{0.984239in}{0.706060in}}%
\pgfpathcurveto{\pgfqpoint{0.984239in}{0.695010in}}{\pgfqpoint{0.988630in}{0.684411in}}{\pgfqpoint{0.996443in}{0.676598in}}%
\pgfpathcurveto{\pgfqpoint{1.004257in}{0.668784in}}{\pgfqpoint{1.014856in}{0.664394in}}{\pgfqpoint{1.025906in}{0.664394in}}%
\pgfpathclose%
\pgfusepath{stroke,fill}%
\end{pgfscope}%
\begin{pgfscope}%
\pgfpathrectangle{\pgfqpoint{0.800000in}{0.528000in}}{\pgfqpoint{4.960000in}{3.696000in}}%
\pgfusepath{clip}%
\pgfsetbuttcap%
\pgfsetroundjoin%
\definecolor{currentfill}{rgb}{0.000000,0.000000,0.000000}%
\pgfsetfillcolor{currentfill}%
\pgfsetlinewidth{1.003750pt}%
\definecolor{currentstroke}{rgb}{0.000000,0.000000,0.000000}%
\pgfsetstrokecolor{currentstroke}%
\pgfsetdash{}{0pt}%
\pgfpathmoveto{\pgfqpoint{1.025906in}{0.664394in}}%
\pgfpathcurveto{\pgfqpoint{1.036956in}{0.664394in}}{\pgfqpoint{1.047555in}{0.668784in}}{\pgfqpoint{1.055369in}{0.676598in}}%
\pgfpathcurveto{\pgfqpoint{1.063182in}{0.684411in}}{\pgfqpoint{1.067573in}{0.695010in}}{\pgfqpoint{1.067573in}{0.706060in}}%
\pgfpathcurveto{\pgfqpoint{1.067573in}{0.717111in}}{\pgfqpoint{1.063182in}{0.727710in}}{\pgfqpoint{1.055369in}{0.735523in}}%
\pgfpathcurveto{\pgfqpoint{1.047555in}{0.743337in}}{\pgfqpoint{1.036956in}{0.747727in}}{\pgfqpoint{1.025906in}{0.747727in}}%
\pgfpathcurveto{\pgfqpoint{1.014856in}{0.747727in}}{\pgfqpoint{1.004257in}{0.743337in}}{\pgfqpoint{0.996443in}{0.735523in}}%
\pgfpathcurveto{\pgfqpoint{0.988630in}{0.727710in}}{\pgfqpoint{0.984239in}{0.717111in}}{\pgfqpoint{0.984239in}{0.706060in}}%
\pgfpathcurveto{\pgfqpoint{0.984239in}{0.695010in}}{\pgfqpoint{0.988630in}{0.684411in}}{\pgfqpoint{0.996443in}{0.676598in}}%
\pgfpathcurveto{\pgfqpoint{1.004257in}{0.668784in}}{\pgfqpoint{1.014856in}{0.664394in}}{\pgfqpoint{1.025906in}{0.664394in}}%
\pgfpathclose%
\pgfusepath{stroke,fill}%
\end{pgfscope}%
\begin{pgfscope}%
\pgfpathrectangle{\pgfqpoint{0.800000in}{0.528000in}}{\pgfqpoint{4.960000in}{3.696000in}}%
\pgfusepath{clip}%
\pgfsetbuttcap%
\pgfsetroundjoin%
\definecolor{currentfill}{rgb}{0.000000,0.000000,0.000000}%
\pgfsetfillcolor{currentfill}%
\pgfsetlinewidth{1.003750pt}%
\definecolor{currentstroke}{rgb}{0.000000,0.000000,0.000000}%
\pgfsetstrokecolor{currentstroke}%
\pgfsetdash{}{0pt}%
\pgfpathmoveto{\pgfqpoint{1.025906in}{0.664394in}}%
\pgfpathcurveto{\pgfqpoint{1.036956in}{0.664394in}}{\pgfqpoint{1.047555in}{0.668784in}}{\pgfqpoint{1.055369in}{0.676598in}}%
\pgfpathcurveto{\pgfqpoint{1.063182in}{0.684411in}}{\pgfqpoint{1.067573in}{0.695010in}}{\pgfqpoint{1.067573in}{0.706060in}}%
\pgfpathcurveto{\pgfqpoint{1.067573in}{0.717111in}}{\pgfqpoint{1.063182in}{0.727710in}}{\pgfqpoint{1.055369in}{0.735523in}}%
\pgfpathcurveto{\pgfqpoint{1.047555in}{0.743337in}}{\pgfqpoint{1.036956in}{0.747727in}}{\pgfqpoint{1.025906in}{0.747727in}}%
\pgfpathcurveto{\pgfqpoint{1.014856in}{0.747727in}}{\pgfqpoint{1.004257in}{0.743337in}}{\pgfqpoint{0.996443in}{0.735523in}}%
\pgfpathcurveto{\pgfqpoint{0.988630in}{0.727710in}}{\pgfqpoint{0.984239in}{0.717111in}}{\pgfqpoint{0.984239in}{0.706060in}}%
\pgfpathcurveto{\pgfqpoint{0.984239in}{0.695010in}}{\pgfqpoint{0.988630in}{0.684411in}}{\pgfqpoint{0.996443in}{0.676598in}}%
\pgfpathcurveto{\pgfqpoint{1.004257in}{0.668784in}}{\pgfqpoint{1.014856in}{0.664394in}}{\pgfqpoint{1.025906in}{0.664394in}}%
\pgfpathclose%
\pgfusepath{stroke,fill}%
\end{pgfscope}%
\begin{pgfscope}%
\pgfpathrectangle{\pgfqpoint{0.800000in}{0.528000in}}{\pgfqpoint{4.960000in}{3.696000in}}%
\pgfusepath{clip}%
\pgfsetbuttcap%
\pgfsetroundjoin%
\definecolor{currentfill}{rgb}{0.000000,0.000000,0.000000}%
\pgfsetfillcolor{currentfill}%
\pgfsetlinewidth{1.003750pt}%
\definecolor{currentstroke}{rgb}{0.000000,0.000000,0.000000}%
\pgfsetstrokecolor{currentstroke}%
\pgfsetdash{}{0pt}%
\pgfpathmoveto{\pgfqpoint{1.025906in}{0.664394in}}%
\pgfpathcurveto{\pgfqpoint{1.036956in}{0.664394in}}{\pgfqpoint{1.047555in}{0.668784in}}{\pgfqpoint{1.055369in}{0.676598in}}%
\pgfpathcurveto{\pgfqpoint{1.063182in}{0.684411in}}{\pgfqpoint{1.067573in}{0.695010in}}{\pgfqpoint{1.067573in}{0.706060in}}%
\pgfpathcurveto{\pgfqpoint{1.067573in}{0.717111in}}{\pgfqpoint{1.063182in}{0.727710in}}{\pgfqpoint{1.055369in}{0.735523in}}%
\pgfpathcurveto{\pgfqpoint{1.047555in}{0.743337in}}{\pgfqpoint{1.036956in}{0.747727in}}{\pgfqpoint{1.025906in}{0.747727in}}%
\pgfpathcurveto{\pgfqpoint{1.014856in}{0.747727in}}{\pgfqpoint{1.004257in}{0.743337in}}{\pgfqpoint{0.996443in}{0.735523in}}%
\pgfpathcurveto{\pgfqpoint{0.988630in}{0.727710in}}{\pgfqpoint{0.984239in}{0.717111in}}{\pgfqpoint{0.984239in}{0.706060in}}%
\pgfpathcurveto{\pgfqpoint{0.984239in}{0.695010in}}{\pgfqpoint{0.988630in}{0.684411in}}{\pgfqpoint{0.996443in}{0.676598in}}%
\pgfpathcurveto{\pgfqpoint{1.004257in}{0.668784in}}{\pgfqpoint{1.014856in}{0.664394in}}{\pgfqpoint{1.025906in}{0.664394in}}%
\pgfpathclose%
\pgfusepath{stroke,fill}%
\end{pgfscope}%
\begin{pgfscope}%
\pgfpathrectangle{\pgfqpoint{0.800000in}{0.528000in}}{\pgfqpoint{4.960000in}{3.696000in}}%
\pgfusepath{clip}%
\pgfsetbuttcap%
\pgfsetroundjoin%
\definecolor{currentfill}{rgb}{0.000000,0.000000,0.000000}%
\pgfsetfillcolor{currentfill}%
\pgfsetlinewidth{1.003750pt}%
\definecolor{currentstroke}{rgb}{0.000000,0.000000,0.000000}%
\pgfsetstrokecolor{currentstroke}%
\pgfsetdash{}{0pt}%
\pgfpathmoveto{\pgfqpoint{1.025906in}{0.664394in}}%
\pgfpathcurveto{\pgfqpoint{1.036956in}{0.664394in}}{\pgfqpoint{1.047555in}{0.668784in}}{\pgfqpoint{1.055369in}{0.676598in}}%
\pgfpathcurveto{\pgfqpoint{1.063182in}{0.684411in}}{\pgfqpoint{1.067573in}{0.695010in}}{\pgfqpoint{1.067573in}{0.706060in}}%
\pgfpathcurveto{\pgfqpoint{1.067573in}{0.717111in}}{\pgfqpoint{1.063182in}{0.727710in}}{\pgfqpoint{1.055369in}{0.735523in}}%
\pgfpathcurveto{\pgfqpoint{1.047555in}{0.743337in}}{\pgfqpoint{1.036956in}{0.747727in}}{\pgfqpoint{1.025906in}{0.747727in}}%
\pgfpathcurveto{\pgfqpoint{1.014856in}{0.747727in}}{\pgfqpoint{1.004257in}{0.743337in}}{\pgfqpoint{0.996443in}{0.735523in}}%
\pgfpathcurveto{\pgfqpoint{0.988630in}{0.727710in}}{\pgfqpoint{0.984239in}{0.717111in}}{\pgfqpoint{0.984239in}{0.706060in}}%
\pgfpathcurveto{\pgfqpoint{0.984239in}{0.695010in}}{\pgfqpoint{0.988630in}{0.684411in}}{\pgfqpoint{0.996443in}{0.676598in}}%
\pgfpathcurveto{\pgfqpoint{1.004257in}{0.668784in}}{\pgfqpoint{1.014856in}{0.664394in}}{\pgfqpoint{1.025906in}{0.664394in}}%
\pgfpathclose%
\pgfusepath{stroke,fill}%
\end{pgfscope}%
\begin{pgfscope}%
\pgfpathrectangle{\pgfqpoint{0.800000in}{0.528000in}}{\pgfqpoint{4.960000in}{3.696000in}}%
\pgfusepath{clip}%
\pgfsetbuttcap%
\pgfsetroundjoin%
\definecolor{currentfill}{rgb}{0.000000,0.000000,0.000000}%
\pgfsetfillcolor{currentfill}%
\pgfsetlinewidth{1.003750pt}%
\definecolor{currentstroke}{rgb}{0.000000,0.000000,0.000000}%
\pgfsetstrokecolor{currentstroke}%
\pgfsetdash{}{0pt}%
\pgfpathmoveto{\pgfqpoint{1.025906in}{0.664394in}}%
\pgfpathcurveto{\pgfqpoint{1.036956in}{0.664394in}}{\pgfqpoint{1.047555in}{0.668784in}}{\pgfqpoint{1.055369in}{0.676598in}}%
\pgfpathcurveto{\pgfqpoint{1.063182in}{0.684411in}}{\pgfqpoint{1.067573in}{0.695010in}}{\pgfqpoint{1.067573in}{0.706060in}}%
\pgfpathcurveto{\pgfqpoint{1.067573in}{0.717111in}}{\pgfqpoint{1.063182in}{0.727710in}}{\pgfqpoint{1.055369in}{0.735523in}}%
\pgfpathcurveto{\pgfqpoint{1.047555in}{0.743337in}}{\pgfqpoint{1.036956in}{0.747727in}}{\pgfqpoint{1.025906in}{0.747727in}}%
\pgfpathcurveto{\pgfqpoint{1.014856in}{0.747727in}}{\pgfqpoint{1.004257in}{0.743337in}}{\pgfqpoint{0.996443in}{0.735523in}}%
\pgfpathcurveto{\pgfqpoint{0.988630in}{0.727710in}}{\pgfqpoint{0.984239in}{0.717111in}}{\pgfqpoint{0.984239in}{0.706060in}}%
\pgfpathcurveto{\pgfqpoint{0.984239in}{0.695010in}}{\pgfqpoint{0.988630in}{0.684411in}}{\pgfqpoint{0.996443in}{0.676598in}}%
\pgfpathcurveto{\pgfqpoint{1.004257in}{0.668784in}}{\pgfqpoint{1.014856in}{0.664394in}}{\pgfqpoint{1.025906in}{0.664394in}}%
\pgfpathclose%
\pgfusepath{stroke,fill}%
\end{pgfscope}%
\begin{pgfscope}%
\pgfpathrectangle{\pgfqpoint{0.800000in}{0.528000in}}{\pgfqpoint{4.960000in}{3.696000in}}%
\pgfusepath{clip}%
\pgfsetbuttcap%
\pgfsetroundjoin%
\definecolor{currentfill}{rgb}{0.000000,0.000000,0.000000}%
\pgfsetfillcolor{currentfill}%
\pgfsetlinewidth{1.003750pt}%
\definecolor{currentstroke}{rgb}{0.000000,0.000000,0.000000}%
\pgfsetstrokecolor{currentstroke}%
\pgfsetdash{}{0pt}%
\pgfpathmoveto{\pgfqpoint{1.025906in}{0.664394in}}%
\pgfpathcurveto{\pgfqpoint{1.036956in}{0.664394in}}{\pgfqpoint{1.047555in}{0.668784in}}{\pgfqpoint{1.055369in}{0.676598in}}%
\pgfpathcurveto{\pgfqpoint{1.063182in}{0.684411in}}{\pgfqpoint{1.067573in}{0.695010in}}{\pgfqpoint{1.067573in}{0.706060in}}%
\pgfpathcurveto{\pgfqpoint{1.067573in}{0.717111in}}{\pgfqpoint{1.063182in}{0.727710in}}{\pgfqpoint{1.055369in}{0.735523in}}%
\pgfpathcurveto{\pgfqpoint{1.047555in}{0.743337in}}{\pgfqpoint{1.036956in}{0.747727in}}{\pgfqpoint{1.025906in}{0.747727in}}%
\pgfpathcurveto{\pgfqpoint{1.014856in}{0.747727in}}{\pgfqpoint{1.004257in}{0.743337in}}{\pgfqpoint{0.996443in}{0.735523in}}%
\pgfpathcurveto{\pgfqpoint{0.988630in}{0.727710in}}{\pgfqpoint{0.984239in}{0.717111in}}{\pgfqpoint{0.984239in}{0.706060in}}%
\pgfpathcurveto{\pgfqpoint{0.984239in}{0.695010in}}{\pgfqpoint{0.988630in}{0.684411in}}{\pgfqpoint{0.996443in}{0.676598in}}%
\pgfpathcurveto{\pgfqpoint{1.004257in}{0.668784in}}{\pgfqpoint{1.014856in}{0.664394in}}{\pgfqpoint{1.025906in}{0.664394in}}%
\pgfpathclose%
\pgfusepath{stroke,fill}%
\end{pgfscope}%
\begin{pgfscope}%
\pgfpathrectangle{\pgfqpoint{0.800000in}{0.528000in}}{\pgfqpoint{4.960000in}{3.696000in}}%
\pgfusepath{clip}%
\pgfsetbuttcap%
\pgfsetroundjoin%
\definecolor{currentfill}{rgb}{0.000000,0.000000,0.000000}%
\pgfsetfillcolor{currentfill}%
\pgfsetlinewidth{1.003750pt}%
\definecolor{currentstroke}{rgb}{0.000000,0.000000,0.000000}%
\pgfsetstrokecolor{currentstroke}%
\pgfsetdash{}{0pt}%
\pgfpathmoveto{\pgfqpoint{1.025906in}{0.664394in}}%
\pgfpathcurveto{\pgfqpoint{1.036956in}{0.664394in}}{\pgfqpoint{1.047555in}{0.668784in}}{\pgfqpoint{1.055369in}{0.676598in}}%
\pgfpathcurveto{\pgfqpoint{1.063182in}{0.684411in}}{\pgfqpoint{1.067573in}{0.695010in}}{\pgfqpoint{1.067573in}{0.706060in}}%
\pgfpathcurveto{\pgfqpoint{1.067573in}{0.717111in}}{\pgfqpoint{1.063182in}{0.727710in}}{\pgfqpoint{1.055369in}{0.735523in}}%
\pgfpathcurveto{\pgfqpoint{1.047555in}{0.743337in}}{\pgfqpoint{1.036956in}{0.747727in}}{\pgfqpoint{1.025906in}{0.747727in}}%
\pgfpathcurveto{\pgfqpoint{1.014856in}{0.747727in}}{\pgfqpoint{1.004257in}{0.743337in}}{\pgfqpoint{0.996443in}{0.735523in}}%
\pgfpathcurveto{\pgfqpoint{0.988630in}{0.727710in}}{\pgfqpoint{0.984239in}{0.717111in}}{\pgfqpoint{0.984239in}{0.706060in}}%
\pgfpathcurveto{\pgfqpoint{0.984239in}{0.695010in}}{\pgfqpoint{0.988630in}{0.684411in}}{\pgfqpoint{0.996443in}{0.676598in}}%
\pgfpathcurveto{\pgfqpoint{1.004257in}{0.668784in}}{\pgfqpoint{1.014856in}{0.664394in}}{\pgfqpoint{1.025906in}{0.664394in}}%
\pgfpathclose%
\pgfusepath{stroke,fill}%
\end{pgfscope}%
\begin{pgfscope}%
\pgfpathrectangle{\pgfqpoint{0.800000in}{0.528000in}}{\pgfqpoint{4.960000in}{3.696000in}}%
\pgfusepath{clip}%
\pgfsetbuttcap%
\pgfsetroundjoin%
\definecolor{currentfill}{rgb}{0.000000,0.000000,0.000000}%
\pgfsetfillcolor{currentfill}%
\pgfsetlinewidth{1.003750pt}%
\definecolor{currentstroke}{rgb}{0.000000,0.000000,0.000000}%
\pgfsetstrokecolor{currentstroke}%
\pgfsetdash{}{0pt}%
\pgfpathmoveto{\pgfqpoint{1.025906in}{0.664394in}}%
\pgfpathcurveto{\pgfqpoint{1.036956in}{0.664394in}}{\pgfqpoint{1.047555in}{0.668784in}}{\pgfqpoint{1.055369in}{0.676598in}}%
\pgfpathcurveto{\pgfqpoint{1.063182in}{0.684411in}}{\pgfqpoint{1.067573in}{0.695010in}}{\pgfqpoint{1.067573in}{0.706060in}}%
\pgfpathcurveto{\pgfqpoint{1.067573in}{0.717111in}}{\pgfqpoint{1.063182in}{0.727710in}}{\pgfqpoint{1.055369in}{0.735523in}}%
\pgfpathcurveto{\pgfqpoint{1.047555in}{0.743337in}}{\pgfqpoint{1.036956in}{0.747727in}}{\pgfqpoint{1.025906in}{0.747727in}}%
\pgfpathcurveto{\pgfqpoint{1.014856in}{0.747727in}}{\pgfqpoint{1.004257in}{0.743337in}}{\pgfqpoint{0.996443in}{0.735523in}}%
\pgfpathcurveto{\pgfqpoint{0.988630in}{0.727710in}}{\pgfqpoint{0.984239in}{0.717111in}}{\pgfqpoint{0.984239in}{0.706060in}}%
\pgfpathcurveto{\pgfqpoint{0.984239in}{0.695010in}}{\pgfqpoint{0.988630in}{0.684411in}}{\pgfqpoint{0.996443in}{0.676598in}}%
\pgfpathcurveto{\pgfqpoint{1.004257in}{0.668784in}}{\pgfqpoint{1.014856in}{0.664394in}}{\pgfqpoint{1.025906in}{0.664394in}}%
\pgfpathclose%
\pgfusepath{stroke,fill}%
\end{pgfscope}%
\begin{pgfscope}%
\pgfpathrectangle{\pgfqpoint{0.800000in}{0.528000in}}{\pgfqpoint{4.960000in}{3.696000in}}%
\pgfusepath{clip}%
\pgfsetbuttcap%
\pgfsetroundjoin%
\definecolor{currentfill}{rgb}{0.000000,0.000000,0.000000}%
\pgfsetfillcolor{currentfill}%
\pgfsetlinewidth{1.003750pt}%
\definecolor{currentstroke}{rgb}{0.000000,0.000000,0.000000}%
\pgfsetstrokecolor{currentstroke}%
\pgfsetdash{}{0pt}%
\pgfpathmoveto{\pgfqpoint{1.025906in}{0.664394in}}%
\pgfpathcurveto{\pgfqpoint{1.036956in}{0.664394in}}{\pgfqpoint{1.047555in}{0.668784in}}{\pgfqpoint{1.055369in}{0.676598in}}%
\pgfpathcurveto{\pgfqpoint{1.063182in}{0.684411in}}{\pgfqpoint{1.067573in}{0.695010in}}{\pgfqpoint{1.067573in}{0.706060in}}%
\pgfpathcurveto{\pgfqpoint{1.067573in}{0.717111in}}{\pgfqpoint{1.063182in}{0.727710in}}{\pgfqpoint{1.055369in}{0.735523in}}%
\pgfpathcurveto{\pgfqpoint{1.047555in}{0.743337in}}{\pgfqpoint{1.036956in}{0.747727in}}{\pgfqpoint{1.025906in}{0.747727in}}%
\pgfpathcurveto{\pgfqpoint{1.014856in}{0.747727in}}{\pgfqpoint{1.004257in}{0.743337in}}{\pgfqpoint{0.996443in}{0.735523in}}%
\pgfpathcurveto{\pgfqpoint{0.988630in}{0.727710in}}{\pgfqpoint{0.984239in}{0.717111in}}{\pgfqpoint{0.984239in}{0.706060in}}%
\pgfpathcurveto{\pgfqpoint{0.984239in}{0.695010in}}{\pgfqpoint{0.988630in}{0.684411in}}{\pgfqpoint{0.996443in}{0.676598in}}%
\pgfpathcurveto{\pgfqpoint{1.004257in}{0.668784in}}{\pgfqpoint{1.014856in}{0.664394in}}{\pgfqpoint{1.025906in}{0.664394in}}%
\pgfpathclose%
\pgfusepath{stroke,fill}%
\end{pgfscope}%
\begin{pgfscope}%
\pgfpathrectangle{\pgfqpoint{0.800000in}{0.528000in}}{\pgfqpoint{4.960000in}{3.696000in}}%
\pgfusepath{clip}%
\pgfsetbuttcap%
\pgfsetroundjoin%
\definecolor{currentfill}{rgb}{0.000000,0.000000,0.000000}%
\pgfsetfillcolor{currentfill}%
\pgfsetlinewidth{1.003750pt}%
\definecolor{currentstroke}{rgb}{0.000000,0.000000,0.000000}%
\pgfsetstrokecolor{currentstroke}%
\pgfsetdash{}{0pt}%
\pgfpathmoveto{\pgfqpoint{1.025906in}{0.664394in}}%
\pgfpathcurveto{\pgfqpoint{1.036956in}{0.664394in}}{\pgfqpoint{1.047555in}{0.668784in}}{\pgfqpoint{1.055369in}{0.676598in}}%
\pgfpathcurveto{\pgfqpoint{1.063182in}{0.684411in}}{\pgfqpoint{1.067573in}{0.695010in}}{\pgfqpoint{1.067573in}{0.706060in}}%
\pgfpathcurveto{\pgfqpoint{1.067573in}{0.717111in}}{\pgfqpoint{1.063182in}{0.727710in}}{\pgfqpoint{1.055369in}{0.735523in}}%
\pgfpathcurveto{\pgfqpoint{1.047555in}{0.743337in}}{\pgfqpoint{1.036956in}{0.747727in}}{\pgfqpoint{1.025906in}{0.747727in}}%
\pgfpathcurveto{\pgfqpoint{1.014856in}{0.747727in}}{\pgfqpoint{1.004257in}{0.743337in}}{\pgfqpoint{0.996443in}{0.735523in}}%
\pgfpathcurveto{\pgfqpoint{0.988630in}{0.727710in}}{\pgfqpoint{0.984239in}{0.717111in}}{\pgfqpoint{0.984239in}{0.706060in}}%
\pgfpathcurveto{\pgfqpoint{0.984239in}{0.695010in}}{\pgfqpoint{0.988630in}{0.684411in}}{\pgfqpoint{0.996443in}{0.676598in}}%
\pgfpathcurveto{\pgfqpoint{1.004257in}{0.668784in}}{\pgfqpoint{1.014856in}{0.664394in}}{\pgfqpoint{1.025906in}{0.664394in}}%
\pgfpathclose%
\pgfusepath{stroke,fill}%
\end{pgfscope}%
\begin{pgfscope}%
\pgfpathrectangle{\pgfqpoint{0.800000in}{0.528000in}}{\pgfqpoint{4.960000in}{3.696000in}}%
\pgfusepath{clip}%
\pgfsetbuttcap%
\pgfsetroundjoin%
\definecolor{currentfill}{rgb}{0.000000,0.000000,0.000000}%
\pgfsetfillcolor{currentfill}%
\pgfsetlinewidth{1.003750pt}%
\definecolor{currentstroke}{rgb}{0.000000,0.000000,0.000000}%
\pgfsetstrokecolor{currentstroke}%
\pgfsetdash{}{0pt}%
\pgfpathmoveto{\pgfqpoint{1.025906in}{0.664394in}}%
\pgfpathcurveto{\pgfqpoint{1.036956in}{0.664394in}}{\pgfqpoint{1.047555in}{0.668784in}}{\pgfqpoint{1.055369in}{0.676598in}}%
\pgfpathcurveto{\pgfqpoint{1.063182in}{0.684411in}}{\pgfqpoint{1.067573in}{0.695010in}}{\pgfqpoint{1.067573in}{0.706060in}}%
\pgfpathcurveto{\pgfqpoint{1.067573in}{0.717111in}}{\pgfqpoint{1.063182in}{0.727710in}}{\pgfqpoint{1.055369in}{0.735523in}}%
\pgfpathcurveto{\pgfqpoint{1.047555in}{0.743337in}}{\pgfqpoint{1.036956in}{0.747727in}}{\pgfqpoint{1.025906in}{0.747727in}}%
\pgfpathcurveto{\pgfqpoint{1.014856in}{0.747727in}}{\pgfqpoint{1.004257in}{0.743337in}}{\pgfqpoint{0.996443in}{0.735523in}}%
\pgfpathcurveto{\pgfqpoint{0.988630in}{0.727710in}}{\pgfqpoint{0.984239in}{0.717111in}}{\pgfqpoint{0.984239in}{0.706060in}}%
\pgfpathcurveto{\pgfqpoint{0.984239in}{0.695010in}}{\pgfqpoint{0.988630in}{0.684411in}}{\pgfqpoint{0.996443in}{0.676598in}}%
\pgfpathcurveto{\pgfqpoint{1.004257in}{0.668784in}}{\pgfqpoint{1.014856in}{0.664394in}}{\pgfqpoint{1.025906in}{0.664394in}}%
\pgfpathclose%
\pgfusepath{stroke,fill}%
\end{pgfscope}%
\begin{pgfscope}%
\pgfpathrectangle{\pgfqpoint{0.800000in}{0.528000in}}{\pgfqpoint{4.960000in}{3.696000in}}%
\pgfusepath{clip}%
\pgfsetbuttcap%
\pgfsetroundjoin%
\definecolor{currentfill}{rgb}{0.000000,0.000000,0.000000}%
\pgfsetfillcolor{currentfill}%
\pgfsetlinewidth{1.003750pt}%
\definecolor{currentstroke}{rgb}{0.000000,0.000000,0.000000}%
\pgfsetstrokecolor{currentstroke}%
\pgfsetdash{}{0pt}%
\pgfpathmoveto{\pgfqpoint{1.025906in}{0.664394in}}%
\pgfpathcurveto{\pgfqpoint{1.036956in}{0.664394in}}{\pgfqpoint{1.047555in}{0.668784in}}{\pgfqpoint{1.055369in}{0.676598in}}%
\pgfpathcurveto{\pgfqpoint{1.063182in}{0.684411in}}{\pgfqpoint{1.067573in}{0.695010in}}{\pgfqpoint{1.067573in}{0.706060in}}%
\pgfpathcurveto{\pgfqpoint{1.067573in}{0.717111in}}{\pgfqpoint{1.063182in}{0.727710in}}{\pgfqpoint{1.055369in}{0.735523in}}%
\pgfpathcurveto{\pgfqpoint{1.047555in}{0.743337in}}{\pgfqpoint{1.036956in}{0.747727in}}{\pgfqpoint{1.025906in}{0.747727in}}%
\pgfpathcurveto{\pgfqpoint{1.014856in}{0.747727in}}{\pgfqpoint{1.004257in}{0.743337in}}{\pgfqpoint{0.996443in}{0.735523in}}%
\pgfpathcurveto{\pgfqpoint{0.988630in}{0.727710in}}{\pgfqpoint{0.984239in}{0.717111in}}{\pgfqpoint{0.984239in}{0.706060in}}%
\pgfpathcurveto{\pgfqpoint{0.984239in}{0.695010in}}{\pgfqpoint{0.988630in}{0.684411in}}{\pgfqpoint{0.996443in}{0.676598in}}%
\pgfpathcurveto{\pgfqpoint{1.004257in}{0.668784in}}{\pgfqpoint{1.014856in}{0.664394in}}{\pgfqpoint{1.025906in}{0.664394in}}%
\pgfpathclose%
\pgfusepath{stroke,fill}%
\end{pgfscope}%
\begin{pgfscope}%
\pgfpathrectangle{\pgfqpoint{0.800000in}{0.528000in}}{\pgfqpoint{4.960000in}{3.696000in}}%
\pgfusepath{clip}%
\pgfsetbuttcap%
\pgfsetroundjoin%
\definecolor{currentfill}{rgb}{0.000000,0.000000,0.000000}%
\pgfsetfillcolor{currentfill}%
\pgfsetlinewidth{1.003750pt}%
\definecolor{currentstroke}{rgb}{0.000000,0.000000,0.000000}%
\pgfsetstrokecolor{currentstroke}%
\pgfsetdash{}{0pt}%
\pgfpathmoveto{\pgfqpoint{1.025906in}{0.664394in}}%
\pgfpathcurveto{\pgfqpoint{1.036956in}{0.664394in}}{\pgfqpoint{1.047555in}{0.668784in}}{\pgfqpoint{1.055369in}{0.676598in}}%
\pgfpathcurveto{\pgfqpoint{1.063182in}{0.684411in}}{\pgfqpoint{1.067573in}{0.695010in}}{\pgfqpoint{1.067573in}{0.706060in}}%
\pgfpathcurveto{\pgfqpoint{1.067573in}{0.717111in}}{\pgfqpoint{1.063182in}{0.727710in}}{\pgfqpoint{1.055369in}{0.735523in}}%
\pgfpathcurveto{\pgfqpoint{1.047555in}{0.743337in}}{\pgfqpoint{1.036956in}{0.747727in}}{\pgfqpoint{1.025906in}{0.747727in}}%
\pgfpathcurveto{\pgfqpoint{1.014856in}{0.747727in}}{\pgfqpoint{1.004257in}{0.743337in}}{\pgfqpoint{0.996443in}{0.735523in}}%
\pgfpathcurveto{\pgfqpoint{0.988630in}{0.727710in}}{\pgfqpoint{0.984239in}{0.717111in}}{\pgfqpoint{0.984239in}{0.706060in}}%
\pgfpathcurveto{\pgfqpoint{0.984239in}{0.695010in}}{\pgfqpoint{0.988630in}{0.684411in}}{\pgfqpoint{0.996443in}{0.676598in}}%
\pgfpathcurveto{\pgfqpoint{1.004257in}{0.668784in}}{\pgfqpoint{1.014856in}{0.664394in}}{\pgfqpoint{1.025906in}{0.664394in}}%
\pgfpathclose%
\pgfusepath{stroke,fill}%
\end{pgfscope}%
\begin{pgfscope}%
\pgfpathrectangle{\pgfqpoint{0.800000in}{0.528000in}}{\pgfqpoint{4.960000in}{3.696000in}}%
\pgfusepath{clip}%
\pgfsetbuttcap%
\pgfsetroundjoin%
\definecolor{currentfill}{rgb}{0.000000,0.000000,0.000000}%
\pgfsetfillcolor{currentfill}%
\pgfsetlinewidth{1.003750pt}%
\definecolor{currentstroke}{rgb}{0.000000,0.000000,0.000000}%
\pgfsetstrokecolor{currentstroke}%
\pgfsetdash{}{0pt}%
\pgfpathmoveto{\pgfqpoint{1.025906in}{0.664394in}}%
\pgfpathcurveto{\pgfqpoint{1.036956in}{0.664394in}}{\pgfqpoint{1.047555in}{0.668784in}}{\pgfqpoint{1.055369in}{0.676598in}}%
\pgfpathcurveto{\pgfqpoint{1.063182in}{0.684411in}}{\pgfqpoint{1.067573in}{0.695010in}}{\pgfqpoint{1.067573in}{0.706060in}}%
\pgfpathcurveto{\pgfqpoint{1.067573in}{0.717111in}}{\pgfqpoint{1.063182in}{0.727710in}}{\pgfqpoint{1.055369in}{0.735523in}}%
\pgfpathcurveto{\pgfqpoint{1.047555in}{0.743337in}}{\pgfqpoint{1.036956in}{0.747727in}}{\pgfqpoint{1.025906in}{0.747727in}}%
\pgfpathcurveto{\pgfqpoint{1.014856in}{0.747727in}}{\pgfqpoint{1.004257in}{0.743337in}}{\pgfqpoint{0.996443in}{0.735523in}}%
\pgfpathcurveto{\pgfqpoint{0.988630in}{0.727710in}}{\pgfqpoint{0.984239in}{0.717111in}}{\pgfqpoint{0.984239in}{0.706060in}}%
\pgfpathcurveto{\pgfqpoint{0.984239in}{0.695010in}}{\pgfqpoint{0.988630in}{0.684411in}}{\pgfqpoint{0.996443in}{0.676598in}}%
\pgfpathcurveto{\pgfqpoint{1.004257in}{0.668784in}}{\pgfqpoint{1.014856in}{0.664394in}}{\pgfqpoint{1.025906in}{0.664394in}}%
\pgfpathclose%
\pgfusepath{stroke,fill}%
\end{pgfscope}%
\begin{pgfscope}%
\pgfpathrectangle{\pgfqpoint{0.800000in}{0.528000in}}{\pgfqpoint{4.960000in}{3.696000in}}%
\pgfusepath{clip}%
\pgfsetbuttcap%
\pgfsetroundjoin%
\definecolor{currentfill}{rgb}{0.000000,0.000000,0.000000}%
\pgfsetfillcolor{currentfill}%
\pgfsetlinewidth{1.003750pt}%
\definecolor{currentstroke}{rgb}{0.000000,0.000000,0.000000}%
\pgfsetstrokecolor{currentstroke}%
\pgfsetdash{}{0pt}%
\pgfpathmoveto{\pgfqpoint{1.025906in}{1.771040in}}%
\pgfpathcurveto{\pgfqpoint{1.036956in}{1.771040in}}{\pgfqpoint{1.047555in}{1.775431in}}{\pgfqpoint{1.055369in}{1.783244in}}%
\pgfpathcurveto{\pgfqpoint{1.063182in}{1.791058in}}{\pgfqpoint{1.067573in}{1.801657in}}{\pgfqpoint{1.067573in}{1.812707in}}%
\pgfpathcurveto{\pgfqpoint{1.067573in}{1.823757in}}{\pgfqpoint{1.063182in}{1.834356in}}{\pgfqpoint{1.055369in}{1.842170in}}%
\pgfpathcurveto{\pgfqpoint{1.047555in}{1.849983in}}{\pgfqpoint{1.036956in}{1.854374in}}{\pgfqpoint{1.025906in}{1.854374in}}%
\pgfpathcurveto{\pgfqpoint{1.014856in}{1.854374in}}{\pgfqpoint{1.004257in}{1.849983in}}{\pgfqpoint{0.996443in}{1.842170in}}%
\pgfpathcurveto{\pgfqpoint{0.988630in}{1.834356in}}{\pgfqpoint{0.984239in}{1.823757in}}{\pgfqpoint{0.984239in}{1.812707in}}%
\pgfpathcurveto{\pgfqpoint{0.984239in}{1.801657in}}{\pgfqpoint{0.988630in}{1.791058in}}{\pgfqpoint{0.996443in}{1.783244in}}%
\pgfpathcurveto{\pgfqpoint{1.004257in}{1.775431in}}{\pgfqpoint{1.014856in}{1.771040in}}{\pgfqpoint{1.025906in}{1.771040in}}%
\pgfpathclose%
\pgfusepath{stroke,fill}%
\end{pgfscope}%
\begin{pgfscope}%
\pgfpathrectangle{\pgfqpoint{0.800000in}{0.528000in}}{\pgfqpoint{4.960000in}{3.696000in}}%
\pgfusepath{clip}%
\pgfsetbuttcap%
\pgfsetroundjoin%
\definecolor{currentfill}{rgb}{0.000000,0.000000,0.000000}%
\pgfsetfillcolor{currentfill}%
\pgfsetlinewidth{1.003750pt}%
\definecolor{currentstroke}{rgb}{0.000000,0.000000,0.000000}%
\pgfsetstrokecolor{currentstroke}%
\pgfsetdash{}{0pt}%
\pgfpathmoveto{\pgfqpoint{1.025906in}{0.664394in}}%
\pgfpathcurveto{\pgfqpoint{1.036956in}{0.664394in}}{\pgfqpoint{1.047555in}{0.668784in}}{\pgfqpoint{1.055369in}{0.676598in}}%
\pgfpathcurveto{\pgfqpoint{1.063182in}{0.684411in}}{\pgfqpoint{1.067573in}{0.695010in}}{\pgfqpoint{1.067573in}{0.706060in}}%
\pgfpathcurveto{\pgfqpoint{1.067573in}{0.717111in}}{\pgfqpoint{1.063182in}{0.727710in}}{\pgfqpoint{1.055369in}{0.735523in}}%
\pgfpathcurveto{\pgfqpoint{1.047555in}{0.743337in}}{\pgfqpoint{1.036956in}{0.747727in}}{\pgfqpoint{1.025906in}{0.747727in}}%
\pgfpathcurveto{\pgfqpoint{1.014856in}{0.747727in}}{\pgfqpoint{1.004257in}{0.743337in}}{\pgfqpoint{0.996443in}{0.735523in}}%
\pgfpathcurveto{\pgfqpoint{0.988630in}{0.727710in}}{\pgfqpoint{0.984239in}{0.717111in}}{\pgfqpoint{0.984239in}{0.706060in}}%
\pgfpathcurveto{\pgfqpoint{0.984239in}{0.695010in}}{\pgfqpoint{0.988630in}{0.684411in}}{\pgfqpoint{0.996443in}{0.676598in}}%
\pgfpathcurveto{\pgfqpoint{1.004257in}{0.668784in}}{\pgfqpoint{1.014856in}{0.664394in}}{\pgfqpoint{1.025906in}{0.664394in}}%
\pgfpathclose%
\pgfusepath{stroke,fill}%
\end{pgfscope}%
\begin{pgfscope}%
\pgfpathrectangle{\pgfqpoint{0.800000in}{0.528000in}}{\pgfqpoint{4.960000in}{3.696000in}}%
\pgfusepath{clip}%
\pgfsetbuttcap%
\pgfsetroundjoin%
\definecolor{currentfill}{rgb}{0.000000,0.000000,0.000000}%
\pgfsetfillcolor{currentfill}%
\pgfsetlinewidth{1.003750pt}%
\definecolor{currentstroke}{rgb}{0.000000,0.000000,0.000000}%
\pgfsetstrokecolor{currentstroke}%
\pgfsetdash{}{0pt}%
\pgfpathmoveto{\pgfqpoint{1.025906in}{0.664394in}}%
\pgfpathcurveto{\pgfqpoint{1.036956in}{0.664394in}}{\pgfqpoint{1.047555in}{0.668784in}}{\pgfqpoint{1.055369in}{0.676598in}}%
\pgfpathcurveto{\pgfqpoint{1.063182in}{0.684411in}}{\pgfqpoint{1.067573in}{0.695010in}}{\pgfqpoint{1.067573in}{0.706060in}}%
\pgfpathcurveto{\pgfqpoint{1.067573in}{0.717111in}}{\pgfqpoint{1.063182in}{0.727710in}}{\pgfqpoint{1.055369in}{0.735523in}}%
\pgfpathcurveto{\pgfqpoint{1.047555in}{0.743337in}}{\pgfqpoint{1.036956in}{0.747727in}}{\pgfqpoint{1.025906in}{0.747727in}}%
\pgfpathcurveto{\pgfqpoint{1.014856in}{0.747727in}}{\pgfqpoint{1.004257in}{0.743337in}}{\pgfqpoint{0.996443in}{0.735523in}}%
\pgfpathcurveto{\pgfqpoint{0.988630in}{0.727710in}}{\pgfqpoint{0.984239in}{0.717111in}}{\pgfqpoint{0.984239in}{0.706060in}}%
\pgfpathcurveto{\pgfqpoint{0.984239in}{0.695010in}}{\pgfqpoint{0.988630in}{0.684411in}}{\pgfqpoint{0.996443in}{0.676598in}}%
\pgfpathcurveto{\pgfqpoint{1.004257in}{0.668784in}}{\pgfqpoint{1.014856in}{0.664394in}}{\pgfqpoint{1.025906in}{0.664394in}}%
\pgfpathclose%
\pgfusepath{stroke,fill}%
\end{pgfscope}%
\begin{pgfscope}%
\pgfpathrectangle{\pgfqpoint{0.800000in}{0.528000in}}{\pgfqpoint{4.960000in}{3.696000in}}%
\pgfusepath{clip}%
\pgfsetbuttcap%
\pgfsetroundjoin%
\definecolor{currentfill}{rgb}{0.000000,0.000000,0.000000}%
\pgfsetfillcolor{currentfill}%
\pgfsetlinewidth{1.003750pt}%
\definecolor{currentstroke}{rgb}{0.000000,0.000000,0.000000}%
\pgfsetstrokecolor{currentstroke}%
\pgfsetdash{}{0pt}%
\pgfpathmoveto{\pgfqpoint{1.025906in}{0.664394in}}%
\pgfpathcurveto{\pgfqpoint{1.036956in}{0.664394in}}{\pgfqpoint{1.047555in}{0.668784in}}{\pgfqpoint{1.055369in}{0.676598in}}%
\pgfpathcurveto{\pgfqpoint{1.063182in}{0.684411in}}{\pgfqpoint{1.067573in}{0.695010in}}{\pgfqpoint{1.067573in}{0.706060in}}%
\pgfpathcurveto{\pgfqpoint{1.067573in}{0.717111in}}{\pgfqpoint{1.063182in}{0.727710in}}{\pgfqpoint{1.055369in}{0.735523in}}%
\pgfpathcurveto{\pgfqpoint{1.047555in}{0.743337in}}{\pgfqpoint{1.036956in}{0.747727in}}{\pgfqpoint{1.025906in}{0.747727in}}%
\pgfpathcurveto{\pgfqpoint{1.014856in}{0.747727in}}{\pgfqpoint{1.004257in}{0.743337in}}{\pgfqpoint{0.996443in}{0.735523in}}%
\pgfpathcurveto{\pgfqpoint{0.988630in}{0.727710in}}{\pgfqpoint{0.984239in}{0.717111in}}{\pgfqpoint{0.984239in}{0.706060in}}%
\pgfpathcurveto{\pgfqpoint{0.984239in}{0.695010in}}{\pgfqpoint{0.988630in}{0.684411in}}{\pgfqpoint{0.996443in}{0.676598in}}%
\pgfpathcurveto{\pgfqpoint{1.004257in}{0.668784in}}{\pgfqpoint{1.014856in}{0.664394in}}{\pgfqpoint{1.025906in}{0.664394in}}%
\pgfpathclose%
\pgfusepath{stroke,fill}%
\end{pgfscope}%
\begin{pgfscope}%
\pgfpathrectangle{\pgfqpoint{0.800000in}{0.528000in}}{\pgfqpoint{4.960000in}{3.696000in}}%
\pgfusepath{clip}%
\pgfsetbuttcap%
\pgfsetroundjoin%
\definecolor{currentfill}{rgb}{0.000000,0.000000,0.000000}%
\pgfsetfillcolor{currentfill}%
\pgfsetlinewidth{1.003750pt}%
\definecolor{currentstroke}{rgb}{0.000000,0.000000,0.000000}%
\pgfsetstrokecolor{currentstroke}%
\pgfsetdash{}{0pt}%
\pgfpathmoveto{\pgfqpoint{1.025906in}{0.664394in}}%
\pgfpathcurveto{\pgfqpoint{1.036956in}{0.664394in}}{\pgfqpoint{1.047555in}{0.668784in}}{\pgfqpoint{1.055369in}{0.676598in}}%
\pgfpathcurveto{\pgfqpoint{1.063182in}{0.684411in}}{\pgfqpoint{1.067573in}{0.695010in}}{\pgfqpoint{1.067573in}{0.706060in}}%
\pgfpathcurveto{\pgfqpoint{1.067573in}{0.717111in}}{\pgfqpoint{1.063182in}{0.727710in}}{\pgfqpoint{1.055369in}{0.735523in}}%
\pgfpathcurveto{\pgfqpoint{1.047555in}{0.743337in}}{\pgfqpoint{1.036956in}{0.747727in}}{\pgfqpoint{1.025906in}{0.747727in}}%
\pgfpathcurveto{\pgfqpoint{1.014856in}{0.747727in}}{\pgfqpoint{1.004257in}{0.743337in}}{\pgfqpoint{0.996443in}{0.735523in}}%
\pgfpathcurveto{\pgfqpoint{0.988630in}{0.727710in}}{\pgfqpoint{0.984239in}{0.717111in}}{\pgfqpoint{0.984239in}{0.706060in}}%
\pgfpathcurveto{\pgfqpoint{0.984239in}{0.695010in}}{\pgfqpoint{0.988630in}{0.684411in}}{\pgfqpoint{0.996443in}{0.676598in}}%
\pgfpathcurveto{\pgfqpoint{1.004257in}{0.668784in}}{\pgfqpoint{1.014856in}{0.664394in}}{\pgfqpoint{1.025906in}{0.664394in}}%
\pgfpathclose%
\pgfusepath{stroke,fill}%
\end{pgfscope}%
\begin{pgfscope}%
\pgfpathrectangle{\pgfqpoint{0.800000in}{0.528000in}}{\pgfqpoint{4.960000in}{3.696000in}}%
\pgfusepath{clip}%
\pgfsetbuttcap%
\pgfsetroundjoin%
\definecolor{currentfill}{rgb}{0.000000,0.000000,0.000000}%
\pgfsetfillcolor{currentfill}%
\pgfsetlinewidth{1.003750pt}%
\definecolor{currentstroke}{rgb}{0.000000,0.000000,0.000000}%
\pgfsetstrokecolor{currentstroke}%
\pgfsetdash{}{0pt}%
\pgfpathmoveto{\pgfqpoint{1.025906in}{0.664394in}}%
\pgfpathcurveto{\pgfqpoint{1.036956in}{0.664394in}}{\pgfqpoint{1.047555in}{0.668784in}}{\pgfqpoint{1.055369in}{0.676598in}}%
\pgfpathcurveto{\pgfqpoint{1.063182in}{0.684411in}}{\pgfqpoint{1.067573in}{0.695010in}}{\pgfqpoint{1.067573in}{0.706060in}}%
\pgfpathcurveto{\pgfqpoint{1.067573in}{0.717111in}}{\pgfqpoint{1.063182in}{0.727710in}}{\pgfqpoint{1.055369in}{0.735523in}}%
\pgfpathcurveto{\pgfqpoint{1.047555in}{0.743337in}}{\pgfqpoint{1.036956in}{0.747727in}}{\pgfqpoint{1.025906in}{0.747727in}}%
\pgfpathcurveto{\pgfqpoint{1.014856in}{0.747727in}}{\pgfqpoint{1.004257in}{0.743337in}}{\pgfqpoint{0.996443in}{0.735523in}}%
\pgfpathcurveto{\pgfqpoint{0.988630in}{0.727710in}}{\pgfqpoint{0.984239in}{0.717111in}}{\pgfqpoint{0.984239in}{0.706060in}}%
\pgfpathcurveto{\pgfqpoint{0.984239in}{0.695010in}}{\pgfqpoint{0.988630in}{0.684411in}}{\pgfqpoint{0.996443in}{0.676598in}}%
\pgfpathcurveto{\pgfqpoint{1.004257in}{0.668784in}}{\pgfqpoint{1.014856in}{0.664394in}}{\pgfqpoint{1.025906in}{0.664394in}}%
\pgfpathclose%
\pgfusepath{stroke,fill}%
\end{pgfscope}%
\begin{pgfscope}%
\pgfpathrectangle{\pgfqpoint{0.800000in}{0.528000in}}{\pgfqpoint{4.960000in}{3.696000in}}%
\pgfusepath{clip}%
\pgfsetbuttcap%
\pgfsetroundjoin%
\definecolor{currentfill}{rgb}{0.000000,0.000000,0.000000}%
\pgfsetfillcolor{currentfill}%
\pgfsetlinewidth{1.003750pt}%
\definecolor{currentstroke}{rgb}{0.000000,0.000000,0.000000}%
\pgfsetstrokecolor{currentstroke}%
\pgfsetdash{}{0pt}%
\pgfpathmoveto{\pgfqpoint{1.025906in}{0.664394in}}%
\pgfpathcurveto{\pgfqpoint{1.036956in}{0.664394in}}{\pgfqpoint{1.047555in}{0.668784in}}{\pgfqpoint{1.055369in}{0.676598in}}%
\pgfpathcurveto{\pgfqpoint{1.063182in}{0.684411in}}{\pgfqpoint{1.067573in}{0.695010in}}{\pgfqpoint{1.067573in}{0.706060in}}%
\pgfpathcurveto{\pgfqpoint{1.067573in}{0.717111in}}{\pgfqpoint{1.063182in}{0.727710in}}{\pgfqpoint{1.055369in}{0.735523in}}%
\pgfpathcurveto{\pgfqpoint{1.047555in}{0.743337in}}{\pgfqpoint{1.036956in}{0.747727in}}{\pgfqpoint{1.025906in}{0.747727in}}%
\pgfpathcurveto{\pgfqpoint{1.014856in}{0.747727in}}{\pgfqpoint{1.004257in}{0.743337in}}{\pgfqpoint{0.996443in}{0.735523in}}%
\pgfpathcurveto{\pgfqpoint{0.988630in}{0.727710in}}{\pgfqpoint{0.984239in}{0.717111in}}{\pgfqpoint{0.984239in}{0.706060in}}%
\pgfpathcurveto{\pgfqpoint{0.984239in}{0.695010in}}{\pgfqpoint{0.988630in}{0.684411in}}{\pgfqpoint{0.996443in}{0.676598in}}%
\pgfpathcurveto{\pgfqpoint{1.004257in}{0.668784in}}{\pgfqpoint{1.014856in}{0.664394in}}{\pgfqpoint{1.025906in}{0.664394in}}%
\pgfpathclose%
\pgfusepath{stroke,fill}%
\end{pgfscope}%
\begin{pgfscope}%
\pgfpathrectangle{\pgfqpoint{0.800000in}{0.528000in}}{\pgfqpoint{4.960000in}{3.696000in}}%
\pgfusepath{clip}%
\pgfsetbuttcap%
\pgfsetroundjoin%
\definecolor{currentfill}{rgb}{0.000000,0.000000,0.000000}%
\pgfsetfillcolor{currentfill}%
\pgfsetlinewidth{1.003750pt}%
\definecolor{currentstroke}{rgb}{0.000000,0.000000,0.000000}%
\pgfsetstrokecolor{currentstroke}%
\pgfsetdash{}{0pt}%
\pgfpathmoveto{\pgfqpoint{1.025906in}{0.664394in}}%
\pgfpathcurveto{\pgfqpoint{1.036956in}{0.664394in}}{\pgfqpoint{1.047555in}{0.668784in}}{\pgfqpoint{1.055369in}{0.676598in}}%
\pgfpathcurveto{\pgfqpoint{1.063182in}{0.684411in}}{\pgfqpoint{1.067573in}{0.695010in}}{\pgfqpoint{1.067573in}{0.706060in}}%
\pgfpathcurveto{\pgfqpoint{1.067573in}{0.717111in}}{\pgfqpoint{1.063182in}{0.727710in}}{\pgfqpoint{1.055369in}{0.735523in}}%
\pgfpathcurveto{\pgfqpoint{1.047555in}{0.743337in}}{\pgfqpoint{1.036956in}{0.747727in}}{\pgfqpoint{1.025906in}{0.747727in}}%
\pgfpathcurveto{\pgfqpoint{1.014856in}{0.747727in}}{\pgfqpoint{1.004257in}{0.743337in}}{\pgfqpoint{0.996443in}{0.735523in}}%
\pgfpathcurveto{\pgfqpoint{0.988630in}{0.727710in}}{\pgfqpoint{0.984239in}{0.717111in}}{\pgfqpoint{0.984239in}{0.706060in}}%
\pgfpathcurveto{\pgfqpoint{0.984239in}{0.695010in}}{\pgfqpoint{0.988630in}{0.684411in}}{\pgfqpoint{0.996443in}{0.676598in}}%
\pgfpathcurveto{\pgfqpoint{1.004257in}{0.668784in}}{\pgfqpoint{1.014856in}{0.664394in}}{\pgfqpoint{1.025906in}{0.664394in}}%
\pgfpathclose%
\pgfusepath{stroke,fill}%
\end{pgfscope}%
\begin{pgfscope}%
\pgfpathrectangle{\pgfqpoint{0.800000in}{0.528000in}}{\pgfqpoint{4.960000in}{3.696000in}}%
\pgfusepath{clip}%
\pgfsetbuttcap%
\pgfsetroundjoin%
\definecolor{currentfill}{rgb}{0.000000,0.000000,0.000000}%
\pgfsetfillcolor{currentfill}%
\pgfsetlinewidth{1.003750pt}%
\definecolor{currentstroke}{rgb}{0.000000,0.000000,0.000000}%
\pgfsetstrokecolor{currentstroke}%
\pgfsetdash{}{0pt}%
\pgfpathmoveto{\pgfqpoint{1.025906in}{0.664394in}}%
\pgfpathcurveto{\pgfqpoint{1.036956in}{0.664394in}}{\pgfqpoint{1.047555in}{0.668784in}}{\pgfqpoint{1.055369in}{0.676598in}}%
\pgfpathcurveto{\pgfqpoint{1.063182in}{0.684411in}}{\pgfqpoint{1.067573in}{0.695010in}}{\pgfqpoint{1.067573in}{0.706060in}}%
\pgfpathcurveto{\pgfqpoint{1.067573in}{0.717111in}}{\pgfqpoint{1.063182in}{0.727710in}}{\pgfqpoint{1.055369in}{0.735523in}}%
\pgfpathcurveto{\pgfqpoint{1.047555in}{0.743337in}}{\pgfqpoint{1.036956in}{0.747727in}}{\pgfqpoint{1.025906in}{0.747727in}}%
\pgfpathcurveto{\pgfqpoint{1.014856in}{0.747727in}}{\pgfqpoint{1.004257in}{0.743337in}}{\pgfqpoint{0.996443in}{0.735523in}}%
\pgfpathcurveto{\pgfqpoint{0.988630in}{0.727710in}}{\pgfqpoint{0.984239in}{0.717111in}}{\pgfqpoint{0.984239in}{0.706060in}}%
\pgfpathcurveto{\pgfqpoint{0.984239in}{0.695010in}}{\pgfqpoint{0.988630in}{0.684411in}}{\pgfqpoint{0.996443in}{0.676598in}}%
\pgfpathcurveto{\pgfqpoint{1.004257in}{0.668784in}}{\pgfqpoint{1.014856in}{0.664394in}}{\pgfqpoint{1.025906in}{0.664394in}}%
\pgfpathclose%
\pgfusepath{stroke,fill}%
\end{pgfscope}%
\begin{pgfscope}%
\pgfpathrectangle{\pgfqpoint{0.800000in}{0.528000in}}{\pgfqpoint{4.960000in}{3.696000in}}%
\pgfusepath{clip}%
\pgfsetbuttcap%
\pgfsetroundjoin%
\definecolor{currentfill}{rgb}{0.000000,0.000000,0.000000}%
\pgfsetfillcolor{currentfill}%
\pgfsetlinewidth{1.003750pt}%
\definecolor{currentstroke}{rgb}{0.000000,0.000000,0.000000}%
\pgfsetstrokecolor{currentstroke}%
\pgfsetdash{}{0pt}%
\pgfpathmoveto{\pgfqpoint{1.025906in}{0.664394in}}%
\pgfpathcurveto{\pgfqpoint{1.036956in}{0.664394in}}{\pgfqpoint{1.047555in}{0.668784in}}{\pgfqpoint{1.055369in}{0.676598in}}%
\pgfpathcurveto{\pgfqpoint{1.063182in}{0.684411in}}{\pgfqpoint{1.067573in}{0.695010in}}{\pgfqpoint{1.067573in}{0.706060in}}%
\pgfpathcurveto{\pgfqpoint{1.067573in}{0.717111in}}{\pgfqpoint{1.063182in}{0.727710in}}{\pgfqpoint{1.055369in}{0.735523in}}%
\pgfpathcurveto{\pgfqpoint{1.047555in}{0.743337in}}{\pgfqpoint{1.036956in}{0.747727in}}{\pgfqpoint{1.025906in}{0.747727in}}%
\pgfpathcurveto{\pgfqpoint{1.014856in}{0.747727in}}{\pgfqpoint{1.004257in}{0.743337in}}{\pgfqpoint{0.996443in}{0.735523in}}%
\pgfpathcurveto{\pgfqpoint{0.988630in}{0.727710in}}{\pgfqpoint{0.984239in}{0.717111in}}{\pgfqpoint{0.984239in}{0.706060in}}%
\pgfpathcurveto{\pgfqpoint{0.984239in}{0.695010in}}{\pgfqpoint{0.988630in}{0.684411in}}{\pgfqpoint{0.996443in}{0.676598in}}%
\pgfpathcurveto{\pgfqpoint{1.004257in}{0.668784in}}{\pgfqpoint{1.014856in}{0.664394in}}{\pgfqpoint{1.025906in}{0.664394in}}%
\pgfpathclose%
\pgfusepath{stroke,fill}%
\end{pgfscope}%
\begin{pgfscope}%
\pgfpathrectangle{\pgfqpoint{0.800000in}{0.528000in}}{\pgfqpoint{4.960000in}{3.696000in}}%
\pgfusepath{clip}%
\pgfsetbuttcap%
\pgfsetroundjoin%
\definecolor{currentfill}{rgb}{0.000000,0.000000,0.000000}%
\pgfsetfillcolor{currentfill}%
\pgfsetlinewidth{1.003750pt}%
\definecolor{currentstroke}{rgb}{0.000000,0.000000,0.000000}%
\pgfsetstrokecolor{currentstroke}%
\pgfsetdash{}{0pt}%
\pgfpathmoveto{\pgfqpoint{1.025906in}{0.664394in}}%
\pgfpathcurveto{\pgfqpoint{1.036956in}{0.664394in}}{\pgfqpoint{1.047555in}{0.668784in}}{\pgfqpoint{1.055369in}{0.676598in}}%
\pgfpathcurveto{\pgfqpoint{1.063182in}{0.684411in}}{\pgfqpoint{1.067573in}{0.695010in}}{\pgfqpoint{1.067573in}{0.706060in}}%
\pgfpathcurveto{\pgfqpoint{1.067573in}{0.717111in}}{\pgfqpoint{1.063182in}{0.727710in}}{\pgfqpoint{1.055369in}{0.735523in}}%
\pgfpathcurveto{\pgfqpoint{1.047555in}{0.743337in}}{\pgfqpoint{1.036956in}{0.747727in}}{\pgfqpoint{1.025906in}{0.747727in}}%
\pgfpathcurveto{\pgfqpoint{1.014856in}{0.747727in}}{\pgfqpoint{1.004257in}{0.743337in}}{\pgfqpoint{0.996443in}{0.735523in}}%
\pgfpathcurveto{\pgfqpoint{0.988630in}{0.727710in}}{\pgfqpoint{0.984239in}{0.717111in}}{\pgfqpoint{0.984239in}{0.706060in}}%
\pgfpathcurveto{\pgfqpoint{0.984239in}{0.695010in}}{\pgfqpoint{0.988630in}{0.684411in}}{\pgfqpoint{0.996443in}{0.676598in}}%
\pgfpathcurveto{\pgfqpoint{1.004257in}{0.668784in}}{\pgfqpoint{1.014856in}{0.664394in}}{\pgfqpoint{1.025906in}{0.664394in}}%
\pgfpathclose%
\pgfusepath{stroke,fill}%
\end{pgfscope}%
\begin{pgfscope}%
\pgfpathrectangle{\pgfqpoint{0.800000in}{0.528000in}}{\pgfqpoint{4.960000in}{3.696000in}}%
\pgfusepath{clip}%
\pgfsetbuttcap%
\pgfsetroundjoin%
\definecolor{currentfill}{rgb}{0.000000,0.000000,0.000000}%
\pgfsetfillcolor{currentfill}%
\pgfsetlinewidth{1.003750pt}%
\definecolor{currentstroke}{rgb}{0.000000,0.000000,0.000000}%
\pgfsetstrokecolor{currentstroke}%
\pgfsetdash{}{0pt}%
\pgfpathmoveto{\pgfqpoint{1.025906in}{0.664394in}}%
\pgfpathcurveto{\pgfqpoint{1.036956in}{0.664394in}}{\pgfqpoint{1.047555in}{0.668784in}}{\pgfqpoint{1.055369in}{0.676598in}}%
\pgfpathcurveto{\pgfqpoint{1.063182in}{0.684411in}}{\pgfqpoint{1.067573in}{0.695010in}}{\pgfqpoint{1.067573in}{0.706060in}}%
\pgfpathcurveto{\pgfqpoint{1.067573in}{0.717111in}}{\pgfqpoint{1.063182in}{0.727710in}}{\pgfqpoint{1.055369in}{0.735523in}}%
\pgfpathcurveto{\pgfqpoint{1.047555in}{0.743337in}}{\pgfqpoint{1.036956in}{0.747727in}}{\pgfqpoint{1.025906in}{0.747727in}}%
\pgfpathcurveto{\pgfqpoint{1.014856in}{0.747727in}}{\pgfqpoint{1.004257in}{0.743337in}}{\pgfqpoint{0.996443in}{0.735523in}}%
\pgfpathcurveto{\pgfqpoint{0.988630in}{0.727710in}}{\pgfqpoint{0.984239in}{0.717111in}}{\pgfqpoint{0.984239in}{0.706060in}}%
\pgfpathcurveto{\pgfqpoint{0.984239in}{0.695010in}}{\pgfqpoint{0.988630in}{0.684411in}}{\pgfqpoint{0.996443in}{0.676598in}}%
\pgfpathcurveto{\pgfqpoint{1.004257in}{0.668784in}}{\pgfqpoint{1.014856in}{0.664394in}}{\pgfqpoint{1.025906in}{0.664394in}}%
\pgfpathclose%
\pgfusepath{stroke,fill}%
\end{pgfscope}%
\begin{pgfscope}%
\pgfpathrectangle{\pgfqpoint{0.800000in}{0.528000in}}{\pgfqpoint{4.960000in}{3.696000in}}%
\pgfusepath{clip}%
\pgfsetbuttcap%
\pgfsetroundjoin%
\definecolor{currentfill}{rgb}{0.000000,0.000000,0.000000}%
\pgfsetfillcolor{currentfill}%
\pgfsetlinewidth{1.003750pt}%
\definecolor{currentstroke}{rgb}{0.000000,0.000000,0.000000}%
\pgfsetstrokecolor{currentstroke}%
\pgfsetdash{}{0pt}%
\pgfpathmoveto{\pgfqpoint{1.025906in}{0.664394in}}%
\pgfpathcurveto{\pgfqpoint{1.036956in}{0.664394in}}{\pgfqpoint{1.047555in}{0.668784in}}{\pgfqpoint{1.055369in}{0.676598in}}%
\pgfpathcurveto{\pgfqpoint{1.063182in}{0.684411in}}{\pgfqpoint{1.067573in}{0.695010in}}{\pgfqpoint{1.067573in}{0.706060in}}%
\pgfpathcurveto{\pgfqpoint{1.067573in}{0.717111in}}{\pgfqpoint{1.063182in}{0.727710in}}{\pgfqpoint{1.055369in}{0.735523in}}%
\pgfpathcurveto{\pgfqpoint{1.047555in}{0.743337in}}{\pgfqpoint{1.036956in}{0.747727in}}{\pgfqpoint{1.025906in}{0.747727in}}%
\pgfpathcurveto{\pgfqpoint{1.014856in}{0.747727in}}{\pgfqpoint{1.004257in}{0.743337in}}{\pgfqpoint{0.996443in}{0.735523in}}%
\pgfpathcurveto{\pgfqpoint{0.988630in}{0.727710in}}{\pgfqpoint{0.984239in}{0.717111in}}{\pgfqpoint{0.984239in}{0.706060in}}%
\pgfpathcurveto{\pgfqpoint{0.984239in}{0.695010in}}{\pgfqpoint{0.988630in}{0.684411in}}{\pgfqpoint{0.996443in}{0.676598in}}%
\pgfpathcurveto{\pgfqpoint{1.004257in}{0.668784in}}{\pgfqpoint{1.014856in}{0.664394in}}{\pgfqpoint{1.025906in}{0.664394in}}%
\pgfpathclose%
\pgfusepath{stroke,fill}%
\end{pgfscope}%
\begin{pgfscope}%
\pgfpathrectangle{\pgfqpoint{0.800000in}{0.528000in}}{\pgfqpoint{4.960000in}{3.696000in}}%
\pgfusepath{clip}%
\pgfsetbuttcap%
\pgfsetroundjoin%
\definecolor{currentfill}{rgb}{0.000000,0.000000,0.000000}%
\pgfsetfillcolor{currentfill}%
\pgfsetlinewidth{1.003750pt}%
\definecolor{currentstroke}{rgb}{0.000000,0.000000,0.000000}%
\pgfsetstrokecolor{currentstroke}%
\pgfsetdash{}{0pt}%
\pgfpathmoveto{\pgfqpoint{1.025906in}{0.664394in}}%
\pgfpathcurveto{\pgfqpoint{1.036956in}{0.664394in}}{\pgfqpoint{1.047555in}{0.668784in}}{\pgfqpoint{1.055369in}{0.676598in}}%
\pgfpathcurveto{\pgfqpoint{1.063182in}{0.684411in}}{\pgfqpoint{1.067573in}{0.695010in}}{\pgfqpoint{1.067573in}{0.706060in}}%
\pgfpathcurveto{\pgfqpoint{1.067573in}{0.717111in}}{\pgfqpoint{1.063182in}{0.727710in}}{\pgfqpoint{1.055369in}{0.735523in}}%
\pgfpathcurveto{\pgfqpoint{1.047555in}{0.743337in}}{\pgfqpoint{1.036956in}{0.747727in}}{\pgfqpoint{1.025906in}{0.747727in}}%
\pgfpathcurveto{\pgfqpoint{1.014856in}{0.747727in}}{\pgfqpoint{1.004257in}{0.743337in}}{\pgfqpoint{0.996443in}{0.735523in}}%
\pgfpathcurveto{\pgfqpoint{0.988630in}{0.727710in}}{\pgfqpoint{0.984239in}{0.717111in}}{\pgfqpoint{0.984239in}{0.706060in}}%
\pgfpathcurveto{\pgfqpoint{0.984239in}{0.695010in}}{\pgfqpoint{0.988630in}{0.684411in}}{\pgfqpoint{0.996443in}{0.676598in}}%
\pgfpathcurveto{\pgfqpoint{1.004257in}{0.668784in}}{\pgfqpoint{1.014856in}{0.664394in}}{\pgfqpoint{1.025906in}{0.664394in}}%
\pgfpathclose%
\pgfusepath{stroke,fill}%
\end{pgfscope}%
\begin{pgfscope}%
\pgfpathrectangle{\pgfqpoint{0.800000in}{0.528000in}}{\pgfqpoint{4.960000in}{3.696000in}}%
\pgfusepath{clip}%
\pgfsetbuttcap%
\pgfsetroundjoin%
\definecolor{currentfill}{rgb}{0.000000,0.000000,0.000000}%
\pgfsetfillcolor{currentfill}%
\pgfsetlinewidth{1.003750pt}%
\definecolor{currentstroke}{rgb}{0.000000,0.000000,0.000000}%
\pgfsetstrokecolor{currentstroke}%
\pgfsetdash{}{0pt}%
\pgfpathmoveto{\pgfqpoint{1.025906in}{0.664394in}}%
\pgfpathcurveto{\pgfqpoint{1.036956in}{0.664394in}}{\pgfqpoint{1.047555in}{0.668784in}}{\pgfqpoint{1.055369in}{0.676598in}}%
\pgfpathcurveto{\pgfqpoint{1.063182in}{0.684411in}}{\pgfqpoint{1.067573in}{0.695010in}}{\pgfqpoint{1.067573in}{0.706060in}}%
\pgfpathcurveto{\pgfqpoint{1.067573in}{0.717111in}}{\pgfqpoint{1.063182in}{0.727710in}}{\pgfqpoint{1.055369in}{0.735523in}}%
\pgfpathcurveto{\pgfqpoint{1.047555in}{0.743337in}}{\pgfqpoint{1.036956in}{0.747727in}}{\pgfqpoint{1.025906in}{0.747727in}}%
\pgfpathcurveto{\pgfqpoint{1.014856in}{0.747727in}}{\pgfqpoint{1.004257in}{0.743337in}}{\pgfqpoint{0.996443in}{0.735523in}}%
\pgfpathcurveto{\pgfqpoint{0.988630in}{0.727710in}}{\pgfqpoint{0.984239in}{0.717111in}}{\pgfqpoint{0.984239in}{0.706060in}}%
\pgfpathcurveto{\pgfqpoint{0.984239in}{0.695010in}}{\pgfqpoint{0.988630in}{0.684411in}}{\pgfqpoint{0.996443in}{0.676598in}}%
\pgfpathcurveto{\pgfqpoint{1.004257in}{0.668784in}}{\pgfqpoint{1.014856in}{0.664394in}}{\pgfqpoint{1.025906in}{0.664394in}}%
\pgfpathclose%
\pgfusepath{stroke,fill}%
\end{pgfscope}%
\begin{pgfscope}%
\pgfpathrectangle{\pgfqpoint{0.800000in}{0.528000in}}{\pgfqpoint{4.960000in}{3.696000in}}%
\pgfusepath{clip}%
\pgfsetbuttcap%
\pgfsetroundjoin%
\definecolor{currentfill}{rgb}{0.000000,0.000000,0.000000}%
\pgfsetfillcolor{currentfill}%
\pgfsetlinewidth{1.003750pt}%
\definecolor{currentstroke}{rgb}{0.000000,0.000000,0.000000}%
\pgfsetstrokecolor{currentstroke}%
\pgfsetdash{}{0pt}%
\pgfpathmoveto{\pgfqpoint{1.025906in}{0.664394in}}%
\pgfpathcurveto{\pgfqpoint{1.036956in}{0.664394in}}{\pgfqpoint{1.047555in}{0.668784in}}{\pgfqpoint{1.055369in}{0.676598in}}%
\pgfpathcurveto{\pgfqpoint{1.063182in}{0.684411in}}{\pgfqpoint{1.067573in}{0.695010in}}{\pgfqpoint{1.067573in}{0.706060in}}%
\pgfpathcurveto{\pgfqpoint{1.067573in}{0.717111in}}{\pgfqpoint{1.063182in}{0.727710in}}{\pgfqpoint{1.055369in}{0.735523in}}%
\pgfpathcurveto{\pgfqpoint{1.047555in}{0.743337in}}{\pgfqpoint{1.036956in}{0.747727in}}{\pgfqpoint{1.025906in}{0.747727in}}%
\pgfpathcurveto{\pgfqpoint{1.014856in}{0.747727in}}{\pgfqpoint{1.004257in}{0.743337in}}{\pgfqpoint{0.996443in}{0.735523in}}%
\pgfpathcurveto{\pgfqpoint{0.988630in}{0.727710in}}{\pgfqpoint{0.984239in}{0.717111in}}{\pgfqpoint{0.984239in}{0.706060in}}%
\pgfpathcurveto{\pgfqpoint{0.984239in}{0.695010in}}{\pgfqpoint{0.988630in}{0.684411in}}{\pgfqpoint{0.996443in}{0.676598in}}%
\pgfpathcurveto{\pgfqpoint{1.004257in}{0.668784in}}{\pgfqpoint{1.014856in}{0.664394in}}{\pgfqpoint{1.025906in}{0.664394in}}%
\pgfpathclose%
\pgfusepath{stroke,fill}%
\end{pgfscope}%
\begin{pgfscope}%
\pgfpathrectangle{\pgfqpoint{0.800000in}{0.528000in}}{\pgfqpoint{4.960000in}{3.696000in}}%
\pgfusepath{clip}%
\pgfsetbuttcap%
\pgfsetroundjoin%
\definecolor{currentfill}{rgb}{0.000000,0.000000,0.000000}%
\pgfsetfillcolor{currentfill}%
\pgfsetlinewidth{1.003750pt}%
\definecolor{currentstroke}{rgb}{0.000000,0.000000,0.000000}%
\pgfsetstrokecolor{currentstroke}%
\pgfsetdash{}{0pt}%
\pgfpathmoveto{\pgfqpoint{1.025906in}{0.664394in}}%
\pgfpathcurveto{\pgfqpoint{1.036956in}{0.664394in}}{\pgfqpoint{1.047555in}{0.668784in}}{\pgfqpoint{1.055369in}{0.676598in}}%
\pgfpathcurveto{\pgfqpoint{1.063182in}{0.684411in}}{\pgfqpoint{1.067573in}{0.695010in}}{\pgfqpoint{1.067573in}{0.706060in}}%
\pgfpathcurveto{\pgfqpoint{1.067573in}{0.717111in}}{\pgfqpoint{1.063182in}{0.727710in}}{\pgfqpoint{1.055369in}{0.735523in}}%
\pgfpathcurveto{\pgfqpoint{1.047555in}{0.743337in}}{\pgfqpoint{1.036956in}{0.747727in}}{\pgfqpoint{1.025906in}{0.747727in}}%
\pgfpathcurveto{\pgfqpoint{1.014856in}{0.747727in}}{\pgfqpoint{1.004257in}{0.743337in}}{\pgfqpoint{0.996443in}{0.735523in}}%
\pgfpathcurveto{\pgfqpoint{0.988630in}{0.727710in}}{\pgfqpoint{0.984239in}{0.717111in}}{\pgfqpoint{0.984239in}{0.706060in}}%
\pgfpathcurveto{\pgfqpoint{0.984239in}{0.695010in}}{\pgfqpoint{0.988630in}{0.684411in}}{\pgfqpoint{0.996443in}{0.676598in}}%
\pgfpathcurveto{\pgfqpoint{1.004257in}{0.668784in}}{\pgfqpoint{1.014856in}{0.664394in}}{\pgfqpoint{1.025906in}{0.664394in}}%
\pgfpathclose%
\pgfusepath{stroke,fill}%
\end{pgfscope}%
\begin{pgfscope}%
\pgfpathrectangle{\pgfqpoint{0.800000in}{0.528000in}}{\pgfqpoint{4.960000in}{3.696000in}}%
\pgfusepath{clip}%
\pgfsetbuttcap%
\pgfsetroundjoin%
\definecolor{currentfill}{rgb}{0.000000,0.000000,0.000000}%
\pgfsetfillcolor{currentfill}%
\pgfsetlinewidth{1.003750pt}%
\definecolor{currentstroke}{rgb}{0.000000,0.000000,0.000000}%
\pgfsetstrokecolor{currentstroke}%
\pgfsetdash{}{0pt}%
\pgfpathmoveto{\pgfqpoint{1.025906in}{0.664394in}}%
\pgfpathcurveto{\pgfqpoint{1.036956in}{0.664394in}}{\pgfqpoint{1.047555in}{0.668784in}}{\pgfqpoint{1.055369in}{0.676598in}}%
\pgfpathcurveto{\pgfqpoint{1.063182in}{0.684411in}}{\pgfqpoint{1.067573in}{0.695010in}}{\pgfqpoint{1.067573in}{0.706060in}}%
\pgfpathcurveto{\pgfqpoint{1.067573in}{0.717111in}}{\pgfqpoint{1.063182in}{0.727710in}}{\pgfqpoint{1.055369in}{0.735523in}}%
\pgfpathcurveto{\pgfqpoint{1.047555in}{0.743337in}}{\pgfqpoint{1.036956in}{0.747727in}}{\pgfqpoint{1.025906in}{0.747727in}}%
\pgfpathcurveto{\pgfqpoint{1.014856in}{0.747727in}}{\pgfqpoint{1.004257in}{0.743337in}}{\pgfqpoint{0.996443in}{0.735523in}}%
\pgfpathcurveto{\pgfqpoint{0.988630in}{0.727710in}}{\pgfqpoint{0.984239in}{0.717111in}}{\pgfqpoint{0.984239in}{0.706060in}}%
\pgfpathcurveto{\pgfqpoint{0.984239in}{0.695010in}}{\pgfqpoint{0.988630in}{0.684411in}}{\pgfqpoint{0.996443in}{0.676598in}}%
\pgfpathcurveto{\pgfqpoint{1.004257in}{0.668784in}}{\pgfqpoint{1.014856in}{0.664394in}}{\pgfqpoint{1.025906in}{0.664394in}}%
\pgfpathclose%
\pgfusepath{stroke,fill}%
\end{pgfscope}%
\begin{pgfscope}%
\pgfpathrectangle{\pgfqpoint{0.800000in}{0.528000in}}{\pgfqpoint{4.960000in}{3.696000in}}%
\pgfusepath{clip}%
\pgfsetbuttcap%
\pgfsetroundjoin%
\definecolor{currentfill}{rgb}{0.000000,0.000000,0.000000}%
\pgfsetfillcolor{currentfill}%
\pgfsetlinewidth{1.003750pt}%
\definecolor{currentstroke}{rgb}{0.000000,0.000000,0.000000}%
\pgfsetstrokecolor{currentstroke}%
\pgfsetdash{}{0pt}%
\pgfpathmoveto{\pgfqpoint{1.025906in}{0.664394in}}%
\pgfpathcurveto{\pgfqpoint{1.036956in}{0.664394in}}{\pgfqpoint{1.047555in}{0.668784in}}{\pgfqpoint{1.055369in}{0.676598in}}%
\pgfpathcurveto{\pgfqpoint{1.063182in}{0.684411in}}{\pgfqpoint{1.067573in}{0.695010in}}{\pgfqpoint{1.067573in}{0.706060in}}%
\pgfpathcurveto{\pgfqpoint{1.067573in}{0.717111in}}{\pgfqpoint{1.063182in}{0.727710in}}{\pgfqpoint{1.055369in}{0.735523in}}%
\pgfpathcurveto{\pgfqpoint{1.047555in}{0.743337in}}{\pgfqpoint{1.036956in}{0.747727in}}{\pgfqpoint{1.025906in}{0.747727in}}%
\pgfpathcurveto{\pgfqpoint{1.014856in}{0.747727in}}{\pgfqpoint{1.004257in}{0.743337in}}{\pgfqpoint{0.996443in}{0.735523in}}%
\pgfpathcurveto{\pgfqpoint{0.988630in}{0.727710in}}{\pgfqpoint{0.984239in}{0.717111in}}{\pgfqpoint{0.984239in}{0.706060in}}%
\pgfpathcurveto{\pgfqpoint{0.984239in}{0.695010in}}{\pgfqpoint{0.988630in}{0.684411in}}{\pgfqpoint{0.996443in}{0.676598in}}%
\pgfpathcurveto{\pgfqpoint{1.004257in}{0.668784in}}{\pgfqpoint{1.014856in}{0.664394in}}{\pgfqpoint{1.025906in}{0.664394in}}%
\pgfpathclose%
\pgfusepath{stroke,fill}%
\end{pgfscope}%
\begin{pgfscope}%
\pgfpathrectangle{\pgfqpoint{0.800000in}{0.528000in}}{\pgfqpoint{4.960000in}{3.696000in}}%
\pgfusepath{clip}%
\pgfsetbuttcap%
\pgfsetroundjoin%
\definecolor{currentfill}{rgb}{0.000000,0.000000,0.000000}%
\pgfsetfillcolor{currentfill}%
\pgfsetlinewidth{1.003750pt}%
\definecolor{currentstroke}{rgb}{0.000000,0.000000,0.000000}%
\pgfsetstrokecolor{currentstroke}%
\pgfsetdash{}{0pt}%
\pgfpathmoveto{\pgfqpoint{1.025906in}{0.664394in}}%
\pgfpathcurveto{\pgfqpoint{1.036956in}{0.664394in}}{\pgfqpoint{1.047555in}{0.668784in}}{\pgfqpoint{1.055369in}{0.676598in}}%
\pgfpathcurveto{\pgfqpoint{1.063182in}{0.684411in}}{\pgfqpoint{1.067573in}{0.695010in}}{\pgfqpoint{1.067573in}{0.706060in}}%
\pgfpathcurveto{\pgfqpoint{1.067573in}{0.717111in}}{\pgfqpoint{1.063182in}{0.727710in}}{\pgfqpoint{1.055369in}{0.735523in}}%
\pgfpathcurveto{\pgfqpoint{1.047555in}{0.743337in}}{\pgfqpoint{1.036956in}{0.747727in}}{\pgfqpoint{1.025906in}{0.747727in}}%
\pgfpathcurveto{\pgfqpoint{1.014856in}{0.747727in}}{\pgfqpoint{1.004257in}{0.743337in}}{\pgfqpoint{0.996443in}{0.735523in}}%
\pgfpathcurveto{\pgfqpoint{0.988630in}{0.727710in}}{\pgfqpoint{0.984239in}{0.717111in}}{\pgfqpoint{0.984239in}{0.706060in}}%
\pgfpathcurveto{\pgfqpoint{0.984239in}{0.695010in}}{\pgfqpoint{0.988630in}{0.684411in}}{\pgfqpoint{0.996443in}{0.676598in}}%
\pgfpathcurveto{\pgfqpoint{1.004257in}{0.668784in}}{\pgfqpoint{1.014856in}{0.664394in}}{\pgfqpoint{1.025906in}{0.664394in}}%
\pgfpathclose%
\pgfusepath{stroke,fill}%
\end{pgfscope}%
\begin{pgfscope}%
\pgfpathrectangle{\pgfqpoint{0.800000in}{0.528000in}}{\pgfqpoint{4.960000in}{3.696000in}}%
\pgfusepath{clip}%
\pgfsetbuttcap%
\pgfsetroundjoin%
\definecolor{currentfill}{rgb}{0.000000,0.000000,0.000000}%
\pgfsetfillcolor{currentfill}%
\pgfsetlinewidth{1.003750pt}%
\definecolor{currentstroke}{rgb}{0.000000,0.000000,0.000000}%
\pgfsetstrokecolor{currentstroke}%
\pgfsetdash{}{0pt}%
\pgfpathmoveto{\pgfqpoint{1.025906in}{0.664394in}}%
\pgfpathcurveto{\pgfqpoint{1.036956in}{0.664394in}}{\pgfqpoint{1.047555in}{0.668784in}}{\pgfqpoint{1.055369in}{0.676598in}}%
\pgfpathcurveto{\pgfqpoint{1.063182in}{0.684411in}}{\pgfqpoint{1.067573in}{0.695010in}}{\pgfqpoint{1.067573in}{0.706060in}}%
\pgfpathcurveto{\pgfqpoint{1.067573in}{0.717111in}}{\pgfqpoint{1.063182in}{0.727710in}}{\pgfqpoint{1.055369in}{0.735523in}}%
\pgfpathcurveto{\pgfqpoint{1.047555in}{0.743337in}}{\pgfqpoint{1.036956in}{0.747727in}}{\pgfqpoint{1.025906in}{0.747727in}}%
\pgfpathcurveto{\pgfqpoint{1.014856in}{0.747727in}}{\pgfqpoint{1.004257in}{0.743337in}}{\pgfqpoint{0.996443in}{0.735523in}}%
\pgfpathcurveto{\pgfqpoint{0.988630in}{0.727710in}}{\pgfqpoint{0.984239in}{0.717111in}}{\pgfqpoint{0.984239in}{0.706060in}}%
\pgfpathcurveto{\pgfqpoint{0.984239in}{0.695010in}}{\pgfqpoint{0.988630in}{0.684411in}}{\pgfqpoint{0.996443in}{0.676598in}}%
\pgfpathcurveto{\pgfqpoint{1.004257in}{0.668784in}}{\pgfqpoint{1.014856in}{0.664394in}}{\pgfqpoint{1.025906in}{0.664394in}}%
\pgfpathclose%
\pgfusepath{stroke,fill}%
\end{pgfscope}%
\begin{pgfscope}%
\pgfpathrectangle{\pgfqpoint{0.800000in}{0.528000in}}{\pgfqpoint{4.960000in}{3.696000in}}%
\pgfusepath{clip}%
\pgfsetbuttcap%
\pgfsetroundjoin%
\definecolor{currentfill}{rgb}{0.000000,0.000000,0.000000}%
\pgfsetfillcolor{currentfill}%
\pgfsetlinewidth{1.003750pt}%
\definecolor{currentstroke}{rgb}{0.000000,0.000000,0.000000}%
\pgfsetstrokecolor{currentstroke}%
\pgfsetdash{}{0pt}%
\pgfpathmoveto{\pgfqpoint{1.025906in}{0.664394in}}%
\pgfpathcurveto{\pgfqpoint{1.036956in}{0.664394in}}{\pgfqpoint{1.047555in}{0.668784in}}{\pgfqpoint{1.055369in}{0.676598in}}%
\pgfpathcurveto{\pgfqpoint{1.063182in}{0.684411in}}{\pgfqpoint{1.067573in}{0.695010in}}{\pgfqpoint{1.067573in}{0.706060in}}%
\pgfpathcurveto{\pgfqpoint{1.067573in}{0.717111in}}{\pgfqpoint{1.063182in}{0.727710in}}{\pgfqpoint{1.055369in}{0.735523in}}%
\pgfpathcurveto{\pgfqpoint{1.047555in}{0.743337in}}{\pgfqpoint{1.036956in}{0.747727in}}{\pgfqpoint{1.025906in}{0.747727in}}%
\pgfpathcurveto{\pgfqpoint{1.014856in}{0.747727in}}{\pgfqpoint{1.004257in}{0.743337in}}{\pgfqpoint{0.996443in}{0.735523in}}%
\pgfpathcurveto{\pgfqpoint{0.988630in}{0.727710in}}{\pgfqpoint{0.984239in}{0.717111in}}{\pgfqpoint{0.984239in}{0.706060in}}%
\pgfpathcurveto{\pgfqpoint{0.984239in}{0.695010in}}{\pgfqpoint{0.988630in}{0.684411in}}{\pgfqpoint{0.996443in}{0.676598in}}%
\pgfpathcurveto{\pgfqpoint{1.004257in}{0.668784in}}{\pgfqpoint{1.014856in}{0.664394in}}{\pgfqpoint{1.025906in}{0.664394in}}%
\pgfpathclose%
\pgfusepath{stroke,fill}%
\end{pgfscope}%
\begin{pgfscope}%
\pgfpathrectangle{\pgfqpoint{0.800000in}{0.528000in}}{\pgfqpoint{4.960000in}{3.696000in}}%
\pgfusepath{clip}%
\pgfsetbuttcap%
\pgfsetroundjoin%
\definecolor{currentfill}{rgb}{0.000000,0.000000,0.000000}%
\pgfsetfillcolor{currentfill}%
\pgfsetlinewidth{1.003750pt}%
\definecolor{currentstroke}{rgb}{0.000000,0.000000,0.000000}%
\pgfsetstrokecolor{currentstroke}%
\pgfsetdash{}{0pt}%
\pgfpathmoveto{\pgfqpoint{1.025906in}{0.664394in}}%
\pgfpathcurveto{\pgfqpoint{1.036956in}{0.664394in}}{\pgfqpoint{1.047555in}{0.668784in}}{\pgfqpoint{1.055369in}{0.676598in}}%
\pgfpathcurveto{\pgfqpoint{1.063182in}{0.684411in}}{\pgfqpoint{1.067573in}{0.695010in}}{\pgfqpoint{1.067573in}{0.706060in}}%
\pgfpathcurveto{\pgfqpoint{1.067573in}{0.717111in}}{\pgfqpoint{1.063182in}{0.727710in}}{\pgfqpoint{1.055369in}{0.735523in}}%
\pgfpathcurveto{\pgfqpoint{1.047555in}{0.743337in}}{\pgfqpoint{1.036956in}{0.747727in}}{\pgfqpoint{1.025906in}{0.747727in}}%
\pgfpathcurveto{\pgfqpoint{1.014856in}{0.747727in}}{\pgfqpoint{1.004257in}{0.743337in}}{\pgfqpoint{0.996443in}{0.735523in}}%
\pgfpathcurveto{\pgfqpoint{0.988630in}{0.727710in}}{\pgfqpoint{0.984239in}{0.717111in}}{\pgfqpoint{0.984239in}{0.706060in}}%
\pgfpathcurveto{\pgfqpoint{0.984239in}{0.695010in}}{\pgfqpoint{0.988630in}{0.684411in}}{\pgfqpoint{0.996443in}{0.676598in}}%
\pgfpathcurveto{\pgfqpoint{1.004257in}{0.668784in}}{\pgfqpoint{1.014856in}{0.664394in}}{\pgfqpoint{1.025906in}{0.664394in}}%
\pgfpathclose%
\pgfusepath{stroke,fill}%
\end{pgfscope}%
\begin{pgfscope}%
\pgfpathrectangle{\pgfqpoint{0.800000in}{0.528000in}}{\pgfqpoint{4.960000in}{3.696000in}}%
\pgfusepath{clip}%
\pgfsetbuttcap%
\pgfsetroundjoin%
\definecolor{currentfill}{rgb}{0.000000,0.000000,0.000000}%
\pgfsetfillcolor{currentfill}%
\pgfsetlinewidth{1.003750pt}%
\definecolor{currentstroke}{rgb}{0.000000,0.000000,0.000000}%
\pgfsetstrokecolor{currentstroke}%
\pgfsetdash{}{0pt}%
\pgfpathmoveto{\pgfqpoint{1.025906in}{0.664394in}}%
\pgfpathcurveto{\pgfqpoint{1.036956in}{0.664394in}}{\pgfqpoint{1.047555in}{0.668784in}}{\pgfqpoint{1.055369in}{0.676598in}}%
\pgfpathcurveto{\pgfqpoint{1.063182in}{0.684411in}}{\pgfqpoint{1.067573in}{0.695010in}}{\pgfqpoint{1.067573in}{0.706060in}}%
\pgfpathcurveto{\pgfqpoint{1.067573in}{0.717111in}}{\pgfqpoint{1.063182in}{0.727710in}}{\pgfqpoint{1.055369in}{0.735523in}}%
\pgfpathcurveto{\pgfqpoint{1.047555in}{0.743337in}}{\pgfqpoint{1.036956in}{0.747727in}}{\pgfqpoint{1.025906in}{0.747727in}}%
\pgfpathcurveto{\pgfqpoint{1.014856in}{0.747727in}}{\pgfqpoint{1.004257in}{0.743337in}}{\pgfqpoint{0.996443in}{0.735523in}}%
\pgfpathcurveto{\pgfqpoint{0.988630in}{0.727710in}}{\pgfqpoint{0.984239in}{0.717111in}}{\pgfqpoint{0.984239in}{0.706060in}}%
\pgfpathcurveto{\pgfqpoint{0.984239in}{0.695010in}}{\pgfqpoint{0.988630in}{0.684411in}}{\pgfqpoint{0.996443in}{0.676598in}}%
\pgfpathcurveto{\pgfqpoint{1.004257in}{0.668784in}}{\pgfqpoint{1.014856in}{0.664394in}}{\pgfqpoint{1.025906in}{0.664394in}}%
\pgfpathclose%
\pgfusepath{stroke,fill}%
\end{pgfscope}%
\begin{pgfscope}%
\pgfpathrectangle{\pgfqpoint{0.800000in}{0.528000in}}{\pgfqpoint{4.960000in}{3.696000in}}%
\pgfusepath{clip}%
\pgfsetbuttcap%
\pgfsetroundjoin%
\definecolor{currentfill}{rgb}{0.000000,0.000000,0.000000}%
\pgfsetfillcolor{currentfill}%
\pgfsetlinewidth{1.003750pt}%
\definecolor{currentstroke}{rgb}{0.000000,0.000000,0.000000}%
\pgfsetstrokecolor{currentstroke}%
\pgfsetdash{}{0pt}%
\pgfpathmoveto{\pgfqpoint{1.025906in}{0.664394in}}%
\pgfpathcurveto{\pgfqpoint{1.036956in}{0.664394in}}{\pgfqpoint{1.047555in}{0.668784in}}{\pgfqpoint{1.055369in}{0.676598in}}%
\pgfpathcurveto{\pgfqpoint{1.063182in}{0.684411in}}{\pgfqpoint{1.067573in}{0.695010in}}{\pgfqpoint{1.067573in}{0.706060in}}%
\pgfpathcurveto{\pgfqpoint{1.067573in}{0.717111in}}{\pgfqpoint{1.063182in}{0.727710in}}{\pgfqpoint{1.055369in}{0.735523in}}%
\pgfpathcurveto{\pgfqpoint{1.047555in}{0.743337in}}{\pgfqpoint{1.036956in}{0.747727in}}{\pgfqpoint{1.025906in}{0.747727in}}%
\pgfpathcurveto{\pgfqpoint{1.014856in}{0.747727in}}{\pgfqpoint{1.004257in}{0.743337in}}{\pgfqpoint{0.996443in}{0.735523in}}%
\pgfpathcurveto{\pgfqpoint{0.988630in}{0.727710in}}{\pgfqpoint{0.984239in}{0.717111in}}{\pgfqpoint{0.984239in}{0.706060in}}%
\pgfpathcurveto{\pgfqpoint{0.984239in}{0.695010in}}{\pgfqpoint{0.988630in}{0.684411in}}{\pgfqpoint{0.996443in}{0.676598in}}%
\pgfpathcurveto{\pgfqpoint{1.004257in}{0.668784in}}{\pgfqpoint{1.014856in}{0.664394in}}{\pgfqpoint{1.025906in}{0.664394in}}%
\pgfpathclose%
\pgfusepath{stroke,fill}%
\end{pgfscope}%
\begin{pgfscope}%
\pgfpathrectangle{\pgfqpoint{0.800000in}{0.528000in}}{\pgfqpoint{4.960000in}{3.696000in}}%
\pgfusepath{clip}%
\pgfsetbuttcap%
\pgfsetroundjoin%
\definecolor{currentfill}{rgb}{0.000000,0.000000,0.000000}%
\pgfsetfillcolor{currentfill}%
\pgfsetlinewidth{1.003750pt}%
\definecolor{currentstroke}{rgb}{0.000000,0.000000,0.000000}%
\pgfsetstrokecolor{currentstroke}%
\pgfsetdash{}{0pt}%
\pgfpathmoveto{\pgfqpoint{1.025906in}{0.664394in}}%
\pgfpathcurveto{\pgfqpoint{1.036956in}{0.664394in}}{\pgfqpoint{1.047555in}{0.668784in}}{\pgfqpoint{1.055369in}{0.676598in}}%
\pgfpathcurveto{\pgfqpoint{1.063182in}{0.684411in}}{\pgfqpoint{1.067573in}{0.695010in}}{\pgfqpoint{1.067573in}{0.706060in}}%
\pgfpathcurveto{\pgfqpoint{1.067573in}{0.717111in}}{\pgfqpoint{1.063182in}{0.727710in}}{\pgfqpoint{1.055369in}{0.735523in}}%
\pgfpathcurveto{\pgfqpoint{1.047555in}{0.743337in}}{\pgfqpoint{1.036956in}{0.747727in}}{\pgfqpoint{1.025906in}{0.747727in}}%
\pgfpathcurveto{\pgfqpoint{1.014856in}{0.747727in}}{\pgfqpoint{1.004257in}{0.743337in}}{\pgfqpoint{0.996443in}{0.735523in}}%
\pgfpathcurveto{\pgfqpoint{0.988630in}{0.727710in}}{\pgfqpoint{0.984239in}{0.717111in}}{\pgfqpoint{0.984239in}{0.706060in}}%
\pgfpathcurveto{\pgfqpoint{0.984239in}{0.695010in}}{\pgfqpoint{0.988630in}{0.684411in}}{\pgfqpoint{0.996443in}{0.676598in}}%
\pgfpathcurveto{\pgfqpoint{1.004257in}{0.668784in}}{\pgfqpoint{1.014856in}{0.664394in}}{\pgfqpoint{1.025906in}{0.664394in}}%
\pgfpathclose%
\pgfusepath{stroke,fill}%
\end{pgfscope}%
\begin{pgfscope}%
\pgfpathrectangle{\pgfqpoint{0.800000in}{0.528000in}}{\pgfqpoint{4.960000in}{3.696000in}}%
\pgfusepath{clip}%
\pgfsetbuttcap%
\pgfsetroundjoin%
\definecolor{currentfill}{rgb}{0.000000,0.000000,0.000000}%
\pgfsetfillcolor{currentfill}%
\pgfsetlinewidth{1.003750pt}%
\definecolor{currentstroke}{rgb}{0.000000,0.000000,0.000000}%
\pgfsetstrokecolor{currentstroke}%
\pgfsetdash{}{0pt}%
\pgfpathmoveto{\pgfqpoint{1.025906in}{0.664394in}}%
\pgfpathcurveto{\pgfqpoint{1.036956in}{0.664394in}}{\pgfqpoint{1.047555in}{0.668784in}}{\pgfqpoint{1.055369in}{0.676598in}}%
\pgfpathcurveto{\pgfqpoint{1.063182in}{0.684411in}}{\pgfqpoint{1.067573in}{0.695010in}}{\pgfqpoint{1.067573in}{0.706060in}}%
\pgfpathcurveto{\pgfqpoint{1.067573in}{0.717111in}}{\pgfqpoint{1.063182in}{0.727710in}}{\pgfqpoint{1.055369in}{0.735523in}}%
\pgfpathcurveto{\pgfqpoint{1.047555in}{0.743337in}}{\pgfqpoint{1.036956in}{0.747727in}}{\pgfqpoint{1.025906in}{0.747727in}}%
\pgfpathcurveto{\pgfqpoint{1.014856in}{0.747727in}}{\pgfqpoint{1.004257in}{0.743337in}}{\pgfqpoint{0.996443in}{0.735523in}}%
\pgfpathcurveto{\pgfqpoint{0.988630in}{0.727710in}}{\pgfqpoint{0.984239in}{0.717111in}}{\pgfqpoint{0.984239in}{0.706060in}}%
\pgfpathcurveto{\pgfqpoint{0.984239in}{0.695010in}}{\pgfqpoint{0.988630in}{0.684411in}}{\pgfqpoint{0.996443in}{0.676598in}}%
\pgfpathcurveto{\pgfqpoint{1.004257in}{0.668784in}}{\pgfqpoint{1.014856in}{0.664394in}}{\pgfqpoint{1.025906in}{0.664394in}}%
\pgfpathclose%
\pgfusepath{stroke,fill}%
\end{pgfscope}%
\begin{pgfscope}%
\pgfpathrectangle{\pgfqpoint{0.800000in}{0.528000in}}{\pgfqpoint{4.960000in}{3.696000in}}%
\pgfusepath{clip}%
\pgfsetbuttcap%
\pgfsetroundjoin%
\definecolor{currentfill}{rgb}{0.000000,0.000000,0.000000}%
\pgfsetfillcolor{currentfill}%
\pgfsetlinewidth{1.003750pt}%
\definecolor{currentstroke}{rgb}{0.000000,0.000000,0.000000}%
\pgfsetstrokecolor{currentstroke}%
\pgfsetdash{}{0pt}%
\pgfpathmoveto{\pgfqpoint{1.025906in}{0.664394in}}%
\pgfpathcurveto{\pgfqpoint{1.036956in}{0.664394in}}{\pgfqpoint{1.047555in}{0.668784in}}{\pgfqpoint{1.055369in}{0.676598in}}%
\pgfpathcurveto{\pgfqpoint{1.063182in}{0.684411in}}{\pgfqpoint{1.067573in}{0.695010in}}{\pgfqpoint{1.067573in}{0.706060in}}%
\pgfpathcurveto{\pgfqpoint{1.067573in}{0.717111in}}{\pgfqpoint{1.063182in}{0.727710in}}{\pgfqpoint{1.055369in}{0.735523in}}%
\pgfpathcurveto{\pgfqpoint{1.047555in}{0.743337in}}{\pgfqpoint{1.036956in}{0.747727in}}{\pgfqpoint{1.025906in}{0.747727in}}%
\pgfpathcurveto{\pgfqpoint{1.014856in}{0.747727in}}{\pgfqpoint{1.004257in}{0.743337in}}{\pgfqpoint{0.996443in}{0.735523in}}%
\pgfpathcurveto{\pgfqpoint{0.988630in}{0.727710in}}{\pgfqpoint{0.984239in}{0.717111in}}{\pgfqpoint{0.984239in}{0.706060in}}%
\pgfpathcurveto{\pgfqpoint{0.984239in}{0.695010in}}{\pgfqpoint{0.988630in}{0.684411in}}{\pgfqpoint{0.996443in}{0.676598in}}%
\pgfpathcurveto{\pgfqpoint{1.004257in}{0.668784in}}{\pgfqpoint{1.014856in}{0.664394in}}{\pgfqpoint{1.025906in}{0.664394in}}%
\pgfpathclose%
\pgfusepath{stroke,fill}%
\end{pgfscope}%
\begin{pgfscope}%
\pgfpathrectangle{\pgfqpoint{0.800000in}{0.528000in}}{\pgfqpoint{4.960000in}{3.696000in}}%
\pgfusepath{clip}%
\pgfsetbuttcap%
\pgfsetroundjoin%
\definecolor{currentfill}{rgb}{0.000000,0.000000,0.000000}%
\pgfsetfillcolor{currentfill}%
\pgfsetlinewidth{1.003750pt}%
\definecolor{currentstroke}{rgb}{0.000000,0.000000,0.000000}%
\pgfsetstrokecolor{currentstroke}%
\pgfsetdash{}{0pt}%
\pgfpathmoveto{\pgfqpoint{1.025906in}{0.664394in}}%
\pgfpathcurveto{\pgfqpoint{1.036956in}{0.664394in}}{\pgfqpoint{1.047555in}{0.668784in}}{\pgfqpoint{1.055369in}{0.676598in}}%
\pgfpathcurveto{\pgfqpoint{1.063182in}{0.684411in}}{\pgfqpoint{1.067573in}{0.695010in}}{\pgfqpoint{1.067573in}{0.706060in}}%
\pgfpathcurveto{\pgfqpoint{1.067573in}{0.717111in}}{\pgfqpoint{1.063182in}{0.727710in}}{\pgfqpoint{1.055369in}{0.735523in}}%
\pgfpathcurveto{\pgfqpoint{1.047555in}{0.743337in}}{\pgfqpoint{1.036956in}{0.747727in}}{\pgfqpoint{1.025906in}{0.747727in}}%
\pgfpathcurveto{\pgfqpoint{1.014856in}{0.747727in}}{\pgfqpoint{1.004257in}{0.743337in}}{\pgfqpoint{0.996443in}{0.735523in}}%
\pgfpathcurveto{\pgfqpoint{0.988630in}{0.727710in}}{\pgfqpoint{0.984239in}{0.717111in}}{\pgfqpoint{0.984239in}{0.706060in}}%
\pgfpathcurveto{\pgfqpoint{0.984239in}{0.695010in}}{\pgfqpoint{0.988630in}{0.684411in}}{\pgfqpoint{0.996443in}{0.676598in}}%
\pgfpathcurveto{\pgfqpoint{1.004257in}{0.668784in}}{\pgfqpoint{1.014856in}{0.664394in}}{\pgfqpoint{1.025906in}{0.664394in}}%
\pgfpathclose%
\pgfusepath{stroke,fill}%
\end{pgfscope}%
\begin{pgfscope}%
\pgfpathrectangle{\pgfqpoint{0.800000in}{0.528000in}}{\pgfqpoint{4.960000in}{3.696000in}}%
\pgfusepath{clip}%
\pgfsetbuttcap%
\pgfsetroundjoin%
\definecolor{currentfill}{rgb}{0.000000,0.000000,0.000000}%
\pgfsetfillcolor{currentfill}%
\pgfsetlinewidth{1.003750pt}%
\definecolor{currentstroke}{rgb}{0.000000,0.000000,0.000000}%
\pgfsetstrokecolor{currentstroke}%
\pgfsetdash{}{0pt}%
\pgfpathmoveto{\pgfqpoint{1.025906in}{0.664394in}}%
\pgfpathcurveto{\pgfqpoint{1.036956in}{0.664394in}}{\pgfqpoint{1.047555in}{0.668784in}}{\pgfqpoint{1.055369in}{0.676598in}}%
\pgfpathcurveto{\pgfqpoint{1.063182in}{0.684411in}}{\pgfqpoint{1.067573in}{0.695010in}}{\pgfqpoint{1.067573in}{0.706060in}}%
\pgfpathcurveto{\pgfqpoint{1.067573in}{0.717111in}}{\pgfqpoint{1.063182in}{0.727710in}}{\pgfqpoint{1.055369in}{0.735523in}}%
\pgfpathcurveto{\pgfqpoint{1.047555in}{0.743337in}}{\pgfqpoint{1.036956in}{0.747727in}}{\pgfqpoint{1.025906in}{0.747727in}}%
\pgfpathcurveto{\pgfqpoint{1.014856in}{0.747727in}}{\pgfqpoint{1.004257in}{0.743337in}}{\pgfqpoint{0.996443in}{0.735523in}}%
\pgfpathcurveto{\pgfqpoint{0.988630in}{0.727710in}}{\pgfqpoint{0.984239in}{0.717111in}}{\pgfqpoint{0.984239in}{0.706060in}}%
\pgfpathcurveto{\pgfqpoint{0.984239in}{0.695010in}}{\pgfqpoint{0.988630in}{0.684411in}}{\pgfqpoint{0.996443in}{0.676598in}}%
\pgfpathcurveto{\pgfqpoint{1.004257in}{0.668784in}}{\pgfqpoint{1.014856in}{0.664394in}}{\pgfqpoint{1.025906in}{0.664394in}}%
\pgfpathclose%
\pgfusepath{stroke,fill}%
\end{pgfscope}%
\begin{pgfscope}%
\pgfpathrectangle{\pgfqpoint{0.800000in}{0.528000in}}{\pgfqpoint{4.960000in}{3.696000in}}%
\pgfusepath{clip}%
\pgfsetbuttcap%
\pgfsetroundjoin%
\definecolor{currentfill}{rgb}{0.000000,0.000000,0.000000}%
\pgfsetfillcolor{currentfill}%
\pgfsetlinewidth{1.003750pt}%
\definecolor{currentstroke}{rgb}{0.000000,0.000000,0.000000}%
\pgfsetstrokecolor{currentstroke}%
\pgfsetdash{}{0pt}%
\pgfpathmoveto{\pgfqpoint{1.025906in}{0.664394in}}%
\pgfpathcurveto{\pgfqpoint{1.036956in}{0.664394in}}{\pgfqpoint{1.047555in}{0.668784in}}{\pgfqpoint{1.055369in}{0.676598in}}%
\pgfpathcurveto{\pgfqpoint{1.063182in}{0.684411in}}{\pgfqpoint{1.067573in}{0.695010in}}{\pgfqpoint{1.067573in}{0.706060in}}%
\pgfpathcurveto{\pgfqpoint{1.067573in}{0.717111in}}{\pgfqpoint{1.063182in}{0.727710in}}{\pgfqpoint{1.055369in}{0.735523in}}%
\pgfpathcurveto{\pgfqpoint{1.047555in}{0.743337in}}{\pgfqpoint{1.036956in}{0.747727in}}{\pgfqpoint{1.025906in}{0.747727in}}%
\pgfpathcurveto{\pgfqpoint{1.014856in}{0.747727in}}{\pgfqpoint{1.004257in}{0.743337in}}{\pgfqpoint{0.996443in}{0.735523in}}%
\pgfpathcurveto{\pgfqpoint{0.988630in}{0.727710in}}{\pgfqpoint{0.984239in}{0.717111in}}{\pgfqpoint{0.984239in}{0.706060in}}%
\pgfpathcurveto{\pgfqpoint{0.984239in}{0.695010in}}{\pgfqpoint{0.988630in}{0.684411in}}{\pgfqpoint{0.996443in}{0.676598in}}%
\pgfpathcurveto{\pgfqpoint{1.004257in}{0.668784in}}{\pgfqpoint{1.014856in}{0.664394in}}{\pgfqpoint{1.025906in}{0.664394in}}%
\pgfpathclose%
\pgfusepath{stroke,fill}%
\end{pgfscope}%
\begin{pgfscope}%
\pgfpathrectangle{\pgfqpoint{0.800000in}{0.528000in}}{\pgfqpoint{4.960000in}{3.696000in}}%
\pgfusepath{clip}%
\pgfsetbuttcap%
\pgfsetroundjoin%
\definecolor{currentfill}{rgb}{0.000000,0.000000,0.000000}%
\pgfsetfillcolor{currentfill}%
\pgfsetlinewidth{1.003750pt}%
\definecolor{currentstroke}{rgb}{0.000000,0.000000,0.000000}%
\pgfsetstrokecolor{currentstroke}%
\pgfsetdash{}{0pt}%
\pgfpathmoveto{\pgfqpoint{1.025906in}{0.664394in}}%
\pgfpathcurveto{\pgfqpoint{1.036956in}{0.664394in}}{\pgfqpoint{1.047555in}{0.668784in}}{\pgfqpoint{1.055369in}{0.676598in}}%
\pgfpathcurveto{\pgfqpoint{1.063182in}{0.684411in}}{\pgfqpoint{1.067573in}{0.695010in}}{\pgfqpoint{1.067573in}{0.706060in}}%
\pgfpathcurveto{\pgfqpoint{1.067573in}{0.717111in}}{\pgfqpoint{1.063182in}{0.727710in}}{\pgfqpoint{1.055369in}{0.735523in}}%
\pgfpathcurveto{\pgfqpoint{1.047555in}{0.743337in}}{\pgfqpoint{1.036956in}{0.747727in}}{\pgfqpoint{1.025906in}{0.747727in}}%
\pgfpathcurveto{\pgfqpoint{1.014856in}{0.747727in}}{\pgfqpoint{1.004257in}{0.743337in}}{\pgfqpoint{0.996443in}{0.735523in}}%
\pgfpathcurveto{\pgfqpoint{0.988630in}{0.727710in}}{\pgfqpoint{0.984239in}{0.717111in}}{\pgfqpoint{0.984239in}{0.706060in}}%
\pgfpathcurveto{\pgfqpoint{0.984239in}{0.695010in}}{\pgfqpoint{0.988630in}{0.684411in}}{\pgfqpoint{0.996443in}{0.676598in}}%
\pgfpathcurveto{\pgfqpoint{1.004257in}{0.668784in}}{\pgfqpoint{1.014856in}{0.664394in}}{\pgfqpoint{1.025906in}{0.664394in}}%
\pgfpathclose%
\pgfusepath{stroke,fill}%
\end{pgfscope}%
\begin{pgfscope}%
\pgfpathrectangle{\pgfqpoint{0.800000in}{0.528000in}}{\pgfqpoint{4.960000in}{3.696000in}}%
\pgfusepath{clip}%
\pgfsetbuttcap%
\pgfsetroundjoin%
\definecolor{currentfill}{rgb}{0.000000,0.000000,0.000000}%
\pgfsetfillcolor{currentfill}%
\pgfsetlinewidth{1.003750pt}%
\definecolor{currentstroke}{rgb}{0.000000,0.000000,0.000000}%
\pgfsetstrokecolor{currentstroke}%
\pgfsetdash{}{0pt}%
\pgfpathmoveto{\pgfqpoint{1.025906in}{0.664394in}}%
\pgfpathcurveto{\pgfqpoint{1.036956in}{0.664394in}}{\pgfqpoint{1.047555in}{0.668784in}}{\pgfqpoint{1.055369in}{0.676598in}}%
\pgfpathcurveto{\pgfqpoint{1.063182in}{0.684411in}}{\pgfqpoint{1.067573in}{0.695010in}}{\pgfqpoint{1.067573in}{0.706060in}}%
\pgfpathcurveto{\pgfqpoint{1.067573in}{0.717111in}}{\pgfqpoint{1.063182in}{0.727710in}}{\pgfqpoint{1.055369in}{0.735523in}}%
\pgfpathcurveto{\pgfqpoint{1.047555in}{0.743337in}}{\pgfqpoint{1.036956in}{0.747727in}}{\pgfqpoint{1.025906in}{0.747727in}}%
\pgfpathcurveto{\pgfqpoint{1.014856in}{0.747727in}}{\pgfqpoint{1.004257in}{0.743337in}}{\pgfqpoint{0.996443in}{0.735523in}}%
\pgfpathcurveto{\pgfqpoint{0.988630in}{0.727710in}}{\pgfqpoint{0.984239in}{0.717111in}}{\pgfqpoint{0.984239in}{0.706060in}}%
\pgfpathcurveto{\pgfqpoint{0.984239in}{0.695010in}}{\pgfqpoint{0.988630in}{0.684411in}}{\pgfqpoint{0.996443in}{0.676598in}}%
\pgfpathcurveto{\pgfqpoint{1.004257in}{0.668784in}}{\pgfqpoint{1.014856in}{0.664394in}}{\pgfqpoint{1.025906in}{0.664394in}}%
\pgfpathclose%
\pgfusepath{stroke,fill}%
\end{pgfscope}%
\begin{pgfscope}%
\pgfpathrectangle{\pgfqpoint{0.800000in}{0.528000in}}{\pgfqpoint{4.960000in}{3.696000in}}%
\pgfusepath{clip}%
\pgfsetbuttcap%
\pgfsetroundjoin%
\definecolor{currentfill}{rgb}{0.000000,0.000000,0.000000}%
\pgfsetfillcolor{currentfill}%
\pgfsetlinewidth{1.003750pt}%
\definecolor{currentstroke}{rgb}{0.000000,0.000000,0.000000}%
\pgfsetstrokecolor{currentstroke}%
\pgfsetdash{}{0pt}%
\pgfpathmoveto{\pgfqpoint{1.025906in}{0.664394in}}%
\pgfpathcurveto{\pgfqpoint{1.036956in}{0.664394in}}{\pgfqpoint{1.047555in}{0.668784in}}{\pgfqpoint{1.055369in}{0.676598in}}%
\pgfpathcurveto{\pgfqpoint{1.063182in}{0.684411in}}{\pgfqpoint{1.067573in}{0.695010in}}{\pgfqpoint{1.067573in}{0.706060in}}%
\pgfpathcurveto{\pgfqpoint{1.067573in}{0.717111in}}{\pgfqpoint{1.063182in}{0.727710in}}{\pgfqpoint{1.055369in}{0.735523in}}%
\pgfpathcurveto{\pgfqpoint{1.047555in}{0.743337in}}{\pgfqpoint{1.036956in}{0.747727in}}{\pgfqpoint{1.025906in}{0.747727in}}%
\pgfpathcurveto{\pgfqpoint{1.014856in}{0.747727in}}{\pgfqpoint{1.004257in}{0.743337in}}{\pgfqpoint{0.996443in}{0.735523in}}%
\pgfpathcurveto{\pgfqpoint{0.988630in}{0.727710in}}{\pgfqpoint{0.984239in}{0.717111in}}{\pgfqpoint{0.984239in}{0.706060in}}%
\pgfpathcurveto{\pgfqpoint{0.984239in}{0.695010in}}{\pgfqpoint{0.988630in}{0.684411in}}{\pgfqpoint{0.996443in}{0.676598in}}%
\pgfpathcurveto{\pgfqpoint{1.004257in}{0.668784in}}{\pgfqpoint{1.014856in}{0.664394in}}{\pgfqpoint{1.025906in}{0.664394in}}%
\pgfpathclose%
\pgfusepath{stroke,fill}%
\end{pgfscope}%
\begin{pgfscope}%
\pgfpathrectangle{\pgfqpoint{0.800000in}{0.528000in}}{\pgfqpoint{4.960000in}{3.696000in}}%
\pgfusepath{clip}%
\pgfsetbuttcap%
\pgfsetroundjoin%
\definecolor{currentfill}{rgb}{0.000000,0.000000,0.000000}%
\pgfsetfillcolor{currentfill}%
\pgfsetlinewidth{1.003750pt}%
\definecolor{currentstroke}{rgb}{0.000000,0.000000,0.000000}%
\pgfsetstrokecolor{currentstroke}%
\pgfsetdash{}{0pt}%
\pgfpathmoveto{\pgfqpoint{1.025906in}{0.664394in}}%
\pgfpathcurveto{\pgfqpoint{1.036956in}{0.664394in}}{\pgfqpoint{1.047555in}{0.668784in}}{\pgfqpoint{1.055369in}{0.676598in}}%
\pgfpathcurveto{\pgfqpoint{1.063182in}{0.684411in}}{\pgfqpoint{1.067573in}{0.695010in}}{\pgfqpoint{1.067573in}{0.706060in}}%
\pgfpathcurveto{\pgfqpoint{1.067573in}{0.717111in}}{\pgfqpoint{1.063182in}{0.727710in}}{\pgfqpoint{1.055369in}{0.735523in}}%
\pgfpathcurveto{\pgfqpoint{1.047555in}{0.743337in}}{\pgfqpoint{1.036956in}{0.747727in}}{\pgfqpoint{1.025906in}{0.747727in}}%
\pgfpathcurveto{\pgfqpoint{1.014856in}{0.747727in}}{\pgfqpoint{1.004257in}{0.743337in}}{\pgfqpoint{0.996443in}{0.735523in}}%
\pgfpathcurveto{\pgfqpoint{0.988630in}{0.727710in}}{\pgfqpoint{0.984239in}{0.717111in}}{\pgfqpoint{0.984239in}{0.706060in}}%
\pgfpathcurveto{\pgfqpoint{0.984239in}{0.695010in}}{\pgfqpoint{0.988630in}{0.684411in}}{\pgfqpoint{0.996443in}{0.676598in}}%
\pgfpathcurveto{\pgfqpoint{1.004257in}{0.668784in}}{\pgfqpoint{1.014856in}{0.664394in}}{\pgfqpoint{1.025906in}{0.664394in}}%
\pgfpathclose%
\pgfusepath{stroke,fill}%
\end{pgfscope}%
\begin{pgfscope}%
\pgfpathrectangle{\pgfqpoint{0.800000in}{0.528000in}}{\pgfqpoint{4.960000in}{3.696000in}}%
\pgfusepath{clip}%
\pgfsetbuttcap%
\pgfsetroundjoin%
\definecolor{currentfill}{rgb}{0.000000,0.000000,0.000000}%
\pgfsetfillcolor{currentfill}%
\pgfsetlinewidth{1.003750pt}%
\definecolor{currentstroke}{rgb}{0.000000,0.000000,0.000000}%
\pgfsetstrokecolor{currentstroke}%
\pgfsetdash{}{0pt}%
\pgfpathmoveto{\pgfqpoint{1.025906in}{0.664394in}}%
\pgfpathcurveto{\pgfqpoint{1.036956in}{0.664394in}}{\pgfqpoint{1.047555in}{0.668784in}}{\pgfqpoint{1.055369in}{0.676598in}}%
\pgfpathcurveto{\pgfqpoint{1.063182in}{0.684411in}}{\pgfqpoint{1.067573in}{0.695010in}}{\pgfqpoint{1.067573in}{0.706060in}}%
\pgfpathcurveto{\pgfqpoint{1.067573in}{0.717111in}}{\pgfqpoint{1.063182in}{0.727710in}}{\pgfqpoint{1.055369in}{0.735523in}}%
\pgfpathcurveto{\pgfqpoint{1.047555in}{0.743337in}}{\pgfqpoint{1.036956in}{0.747727in}}{\pgfqpoint{1.025906in}{0.747727in}}%
\pgfpathcurveto{\pgfqpoint{1.014856in}{0.747727in}}{\pgfqpoint{1.004257in}{0.743337in}}{\pgfqpoint{0.996443in}{0.735523in}}%
\pgfpathcurveto{\pgfqpoint{0.988630in}{0.727710in}}{\pgfqpoint{0.984239in}{0.717111in}}{\pgfqpoint{0.984239in}{0.706060in}}%
\pgfpathcurveto{\pgfqpoint{0.984239in}{0.695010in}}{\pgfqpoint{0.988630in}{0.684411in}}{\pgfqpoint{0.996443in}{0.676598in}}%
\pgfpathcurveto{\pgfqpoint{1.004257in}{0.668784in}}{\pgfqpoint{1.014856in}{0.664394in}}{\pgfqpoint{1.025906in}{0.664394in}}%
\pgfpathclose%
\pgfusepath{stroke,fill}%
\end{pgfscope}%
\begin{pgfscope}%
\pgfpathrectangle{\pgfqpoint{0.800000in}{0.528000in}}{\pgfqpoint{4.960000in}{3.696000in}}%
\pgfusepath{clip}%
\pgfsetbuttcap%
\pgfsetroundjoin%
\definecolor{currentfill}{rgb}{0.000000,0.000000,0.000000}%
\pgfsetfillcolor{currentfill}%
\pgfsetlinewidth{1.003750pt}%
\definecolor{currentstroke}{rgb}{0.000000,0.000000,0.000000}%
\pgfsetstrokecolor{currentstroke}%
\pgfsetdash{}{0pt}%
\pgfpathmoveto{\pgfqpoint{1.025906in}{0.664394in}}%
\pgfpathcurveto{\pgfqpoint{1.036956in}{0.664394in}}{\pgfqpoint{1.047555in}{0.668784in}}{\pgfqpoint{1.055369in}{0.676598in}}%
\pgfpathcurveto{\pgfqpoint{1.063182in}{0.684411in}}{\pgfqpoint{1.067573in}{0.695010in}}{\pgfqpoint{1.067573in}{0.706060in}}%
\pgfpathcurveto{\pgfqpoint{1.067573in}{0.717111in}}{\pgfqpoint{1.063182in}{0.727710in}}{\pgfqpoint{1.055369in}{0.735523in}}%
\pgfpathcurveto{\pgfqpoint{1.047555in}{0.743337in}}{\pgfqpoint{1.036956in}{0.747727in}}{\pgfqpoint{1.025906in}{0.747727in}}%
\pgfpathcurveto{\pgfqpoint{1.014856in}{0.747727in}}{\pgfqpoint{1.004257in}{0.743337in}}{\pgfqpoint{0.996443in}{0.735523in}}%
\pgfpathcurveto{\pgfqpoint{0.988630in}{0.727710in}}{\pgfqpoint{0.984239in}{0.717111in}}{\pgfqpoint{0.984239in}{0.706060in}}%
\pgfpathcurveto{\pgfqpoint{0.984239in}{0.695010in}}{\pgfqpoint{0.988630in}{0.684411in}}{\pgfqpoint{0.996443in}{0.676598in}}%
\pgfpathcurveto{\pgfqpoint{1.004257in}{0.668784in}}{\pgfqpoint{1.014856in}{0.664394in}}{\pgfqpoint{1.025906in}{0.664394in}}%
\pgfpathclose%
\pgfusepath{stroke,fill}%
\end{pgfscope}%
\begin{pgfscope}%
\pgfpathrectangle{\pgfqpoint{0.800000in}{0.528000in}}{\pgfqpoint{4.960000in}{3.696000in}}%
\pgfusepath{clip}%
\pgfsetbuttcap%
\pgfsetroundjoin%
\definecolor{currentfill}{rgb}{0.000000,0.000000,0.000000}%
\pgfsetfillcolor{currentfill}%
\pgfsetlinewidth{1.003750pt}%
\definecolor{currentstroke}{rgb}{0.000000,0.000000,0.000000}%
\pgfsetstrokecolor{currentstroke}%
\pgfsetdash{}{0pt}%
\pgfpathmoveto{\pgfqpoint{1.025906in}{0.664394in}}%
\pgfpathcurveto{\pgfqpoint{1.036956in}{0.664394in}}{\pgfqpoint{1.047555in}{0.668784in}}{\pgfqpoint{1.055369in}{0.676598in}}%
\pgfpathcurveto{\pgfqpoint{1.063182in}{0.684411in}}{\pgfqpoint{1.067573in}{0.695010in}}{\pgfqpoint{1.067573in}{0.706060in}}%
\pgfpathcurveto{\pgfqpoint{1.067573in}{0.717111in}}{\pgfqpoint{1.063182in}{0.727710in}}{\pgfqpoint{1.055369in}{0.735523in}}%
\pgfpathcurveto{\pgfqpoint{1.047555in}{0.743337in}}{\pgfqpoint{1.036956in}{0.747727in}}{\pgfqpoint{1.025906in}{0.747727in}}%
\pgfpathcurveto{\pgfqpoint{1.014856in}{0.747727in}}{\pgfqpoint{1.004257in}{0.743337in}}{\pgfqpoint{0.996443in}{0.735523in}}%
\pgfpathcurveto{\pgfqpoint{0.988630in}{0.727710in}}{\pgfqpoint{0.984239in}{0.717111in}}{\pgfqpoint{0.984239in}{0.706060in}}%
\pgfpathcurveto{\pgfqpoint{0.984239in}{0.695010in}}{\pgfqpoint{0.988630in}{0.684411in}}{\pgfqpoint{0.996443in}{0.676598in}}%
\pgfpathcurveto{\pgfqpoint{1.004257in}{0.668784in}}{\pgfqpoint{1.014856in}{0.664394in}}{\pgfqpoint{1.025906in}{0.664394in}}%
\pgfpathclose%
\pgfusepath{stroke,fill}%
\end{pgfscope}%
\begin{pgfscope}%
\pgfpathrectangle{\pgfqpoint{0.800000in}{0.528000in}}{\pgfqpoint{4.960000in}{3.696000in}}%
\pgfusepath{clip}%
\pgfsetbuttcap%
\pgfsetroundjoin%
\definecolor{currentfill}{rgb}{0.000000,0.000000,0.000000}%
\pgfsetfillcolor{currentfill}%
\pgfsetlinewidth{1.003750pt}%
\definecolor{currentstroke}{rgb}{0.000000,0.000000,0.000000}%
\pgfsetstrokecolor{currentstroke}%
\pgfsetdash{}{0pt}%
\pgfpathmoveto{\pgfqpoint{1.025906in}{0.664394in}}%
\pgfpathcurveto{\pgfqpoint{1.036956in}{0.664394in}}{\pgfqpoint{1.047555in}{0.668784in}}{\pgfqpoint{1.055369in}{0.676598in}}%
\pgfpathcurveto{\pgfqpoint{1.063182in}{0.684411in}}{\pgfqpoint{1.067573in}{0.695010in}}{\pgfqpoint{1.067573in}{0.706060in}}%
\pgfpathcurveto{\pgfqpoint{1.067573in}{0.717111in}}{\pgfqpoint{1.063182in}{0.727710in}}{\pgfqpoint{1.055369in}{0.735523in}}%
\pgfpathcurveto{\pgfqpoint{1.047555in}{0.743337in}}{\pgfqpoint{1.036956in}{0.747727in}}{\pgfqpoint{1.025906in}{0.747727in}}%
\pgfpathcurveto{\pgfqpoint{1.014856in}{0.747727in}}{\pgfqpoint{1.004257in}{0.743337in}}{\pgfqpoint{0.996443in}{0.735523in}}%
\pgfpathcurveto{\pgfqpoint{0.988630in}{0.727710in}}{\pgfqpoint{0.984239in}{0.717111in}}{\pgfqpoint{0.984239in}{0.706060in}}%
\pgfpathcurveto{\pgfqpoint{0.984239in}{0.695010in}}{\pgfqpoint{0.988630in}{0.684411in}}{\pgfqpoint{0.996443in}{0.676598in}}%
\pgfpathcurveto{\pgfqpoint{1.004257in}{0.668784in}}{\pgfqpoint{1.014856in}{0.664394in}}{\pgfqpoint{1.025906in}{0.664394in}}%
\pgfpathclose%
\pgfusepath{stroke,fill}%
\end{pgfscope}%
\begin{pgfscope}%
\pgfpathrectangle{\pgfqpoint{0.800000in}{0.528000in}}{\pgfqpoint{4.960000in}{3.696000in}}%
\pgfusepath{clip}%
\pgfsetbuttcap%
\pgfsetroundjoin%
\definecolor{currentfill}{rgb}{0.000000,0.000000,0.000000}%
\pgfsetfillcolor{currentfill}%
\pgfsetlinewidth{1.003750pt}%
\definecolor{currentstroke}{rgb}{0.000000,0.000000,0.000000}%
\pgfsetstrokecolor{currentstroke}%
\pgfsetdash{}{0pt}%
\pgfpathmoveto{\pgfqpoint{1.025906in}{0.664394in}}%
\pgfpathcurveto{\pgfqpoint{1.036956in}{0.664394in}}{\pgfqpoint{1.047555in}{0.668784in}}{\pgfqpoint{1.055369in}{0.676598in}}%
\pgfpathcurveto{\pgfqpoint{1.063182in}{0.684411in}}{\pgfqpoint{1.067573in}{0.695010in}}{\pgfqpoint{1.067573in}{0.706060in}}%
\pgfpathcurveto{\pgfqpoint{1.067573in}{0.717111in}}{\pgfqpoint{1.063182in}{0.727710in}}{\pgfqpoint{1.055369in}{0.735523in}}%
\pgfpathcurveto{\pgfqpoint{1.047555in}{0.743337in}}{\pgfqpoint{1.036956in}{0.747727in}}{\pgfqpoint{1.025906in}{0.747727in}}%
\pgfpathcurveto{\pgfqpoint{1.014856in}{0.747727in}}{\pgfqpoint{1.004257in}{0.743337in}}{\pgfqpoint{0.996443in}{0.735523in}}%
\pgfpathcurveto{\pgfqpoint{0.988630in}{0.727710in}}{\pgfqpoint{0.984239in}{0.717111in}}{\pgfqpoint{0.984239in}{0.706060in}}%
\pgfpathcurveto{\pgfqpoint{0.984239in}{0.695010in}}{\pgfqpoint{0.988630in}{0.684411in}}{\pgfqpoint{0.996443in}{0.676598in}}%
\pgfpathcurveto{\pgfqpoint{1.004257in}{0.668784in}}{\pgfqpoint{1.014856in}{0.664394in}}{\pgfqpoint{1.025906in}{0.664394in}}%
\pgfpathclose%
\pgfusepath{stroke,fill}%
\end{pgfscope}%
\begin{pgfscope}%
\pgfpathrectangle{\pgfqpoint{0.800000in}{0.528000in}}{\pgfqpoint{4.960000in}{3.696000in}}%
\pgfusepath{clip}%
\pgfsetbuttcap%
\pgfsetroundjoin%
\definecolor{currentfill}{rgb}{0.000000,0.000000,0.000000}%
\pgfsetfillcolor{currentfill}%
\pgfsetlinewidth{1.003750pt}%
\definecolor{currentstroke}{rgb}{0.000000,0.000000,0.000000}%
\pgfsetstrokecolor{currentstroke}%
\pgfsetdash{}{0pt}%
\pgfpathmoveto{\pgfqpoint{1.025906in}{0.664394in}}%
\pgfpathcurveto{\pgfqpoint{1.036956in}{0.664394in}}{\pgfqpoint{1.047555in}{0.668784in}}{\pgfqpoint{1.055369in}{0.676598in}}%
\pgfpathcurveto{\pgfqpoint{1.063182in}{0.684411in}}{\pgfqpoint{1.067573in}{0.695010in}}{\pgfqpoint{1.067573in}{0.706060in}}%
\pgfpathcurveto{\pgfqpoint{1.067573in}{0.717111in}}{\pgfqpoint{1.063182in}{0.727710in}}{\pgfqpoint{1.055369in}{0.735523in}}%
\pgfpathcurveto{\pgfqpoint{1.047555in}{0.743337in}}{\pgfqpoint{1.036956in}{0.747727in}}{\pgfqpoint{1.025906in}{0.747727in}}%
\pgfpathcurveto{\pgfqpoint{1.014856in}{0.747727in}}{\pgfqpoint{1.004257in}{0.743337in}}{\pgfqpoint{0.996443in}{0.735523in}}%
\pgfpathcurveto{\pgfqpoint{0.988630in}{0.727710in}}{\pgfqpoint{0.984239in}{0.717111in}}{\pgfqpoint{0.984239in}{0.706060in}}%
\pgfpathcurveto{\pgfqpoint{0.984239in}{0.695010in}}{\pgfqpoint{0.988630in}{0.684411in}}{\pgfqpoint{0.996443in}{0.676598in}}%
\pgfpathcurveto{\pgfqpoint{1.004257in}{0.668784in}}{\pgfqpoint{1.014856in}{0.664394in}}{\pgfqpoint{1.025906in}{0.664394in}}%
\pgfpathclose%
\pgfusepath{stroke,fill}%
\end{pgfscope}%
\begin{pgfscope}%
\pgfpathrectangle{\pgfqpoint{0.800000in}{0.528000in}}{\pgfqpoint{4.960000in}{3.696000in}}%
\pgfusepath{clip}%
\pgfsetbuttcap%
\pgfsetroundjoin%
\definecolor{currentfill}{rgb}{0.000000,0.000000,0.000000}%
\pgfsetfillcolor{currentfill}%
\pgfsetlinewidth{1.003750pt}%
\definecolor{currentstroke}{rgb}{0.000000,0.000000,0.000000}%
\pgfsetstrokecolor{currentstroke}%
\pgfsetdash{}{0pt}%
\pgfpathmoveto{\pgfqpoint{1.025906in}{0.664394in}}%
\pgfpathcurveto{\pgfqpoint{1.036956in}{0.664394in}}{\pgfqpoint{1.047555in}{0.668784in}}{\pgfqpoint{1.055369in}{0.676598in}}%
\pgfpathcurveto{\pgfqpoint{1.063182in}{0.684411in}}{\pgfqpoint{1.067573in}{0.695010in}}{\pgfqpoint{1.067573in}{0.706060in}}%
\pgfpathcurveto{\pgfqpoint{1.067573in}{0.717111in}}{\pgfqpoint{1.063182in}{0.727710in}}{\pgfqpoint{1.055369in}{0.735523in}}%
\pgfpathcurveto{\pgfqpoint{1.047555in}{0.743337in}}{\pgfqpoint{1.036956in}{0.747727in}}{\pgfqpoint{1.025906in}{0.747727in}}%
\pgfpathcurveto{\pgfqpoint{1.014856in}{0.747727in}}{\pgfqpoint{1.004257in}{0.743337in}}{\pgfqpoint{0.996443in}{0.735523in}}%
\pgfpathcurveto{\pgfqpoint{0.988630in}{0.727710in}}{\pgfqpoint{0.984239in}{0.717111in}}{\pgfqpoint{0.984239in}{0.706060in}}%
\pgfpathcurveto{\pgfqpoint{0.984239in}{0.695010in}}{\pgfqpoint{0.988630in}{0.684411in}}{\pgfqpoint{0.996443in}{0.676598in}}%
\pgfpathcurveto{\pgfqpoint{1.004257in}{0.668784in}}{\pgfqpoint{1.014856in}{0.664394in}}{\pgfqpoint{1.025906in}{0.664394in}}%
\pgfpathclose%
\pgfusepath{stroke,fill}%
\end{pgfscope}%
\begin{pgfscope}%
\pgfpathrectangle{\pgfqpoint{0.800000in}{0.528000in}}{\pgfqpoint{4.960000in}{3.696000in}}%
\pgfusepath{clip}%
\pgfsetbuttcap%
\pgfsetroundjoin%
\definecolor{currentfill}{rgb}{0.000000,0.000000,0.000000}%
\pgfsetfillcolor{currentfill}%
\pgfsetlinewidth{1.003750pt}%
\definecolor{currentstroke}{rgb}{0.000000,0.000000,0.000000}%
\pgfsetstrokecolor{currentstroke}%
\pgfsetdash{}{0pt}%
\pgfpathmoveto{\pgfqpoint{1.025906in}{0.664394in}}%
\pgfpathcurveto{\pgfqpoint{1.036956in}{0.664394in}}{\pgfqpoint{1.047555in}{0.668784in}}{\pgfqpoint{1.055369in}{0.676598in}}%
\pgfpathcurveto{\pgfqpoint{1.063182in}{0.684411in}}{\pgfqpoint{1.067573in}{0.695010in}}{\pgfqpoint{1.067573in}{0.706060in}}%
\pgfpathcurveto{\pgfqpoint{1.067573in}{0.717111in}}{\pgfqpoint{1.063182in}{0.727710in}}{\pgfqpoint{1.055369in}{0.735523in}}%
\pgfpathcurveto{\pgfqpoint{1.047555in}{0.743337in}}{\pgfqpoint{1.036956in}{0.747727in}}{\pgfqpoint{1.025906in}{0.747727in}}%
\pgfpathcurveto{\pgfqpoint{1.014856in}{0.747727in}}{\pgfqpoint{1.004257in}{0.743337in}}{\pgfqpoint{0.996443in}{0.735523in}}%
\pgfpathcurveto{\pgfqpoint{0.988630in}{0.727710in}}{\pgfqpoint{0.984239in}{0.717111in}}{\pgfqpoint{0.984239in}{0.706060in}}%
\pgfpathcurveto{\pgfqpoint{0.984239in}{0.695010in}}{\pgfqpoint{0.988630in}{0.684411in}}{\pgfqpoint{0.996443in}{0.676598in}}%
\pgfpathcurveto{\pgfqpoint{1.004257in}{0.668784in}}{\pgfqpoint{1.014856in}{0.664394in}}{\pgfqpoint{1.025906in}{0.664394in}}%
\pgfpathclose%
\pgfusepath{stroke,fill}%
\end{pgfscope}%
\begin{pgfscope}%
\pgfpathrectangle{\pgfqpoint{0.800000in}{0.528000in}}{\pgfqpoint{4.960000in}{3.696000in}}%
\pgfusepath{clip}%
\pgfsetbuttcap%
\pgfsetroundjoin%
\definecolor{currentfill}{rgb}{0.000000,0.000000,0.000000}%
\pgfsetfillcolor{currentfill}%
\pgfsetlinewidth{1.003750pt}%
\definecolor{currentstroke}{rgb}{0.000000,0.000000,0.000000}%
\pgfsetstrokecolor{currentstroke}%
\pgfsetdash{}{0pt}%
\pgfpathmoveto{\pgfqpoint{1.025906in}{0.664394in}}%
\pgfpathcurveto{\pgfqpoint{1.036956in}{0.664394in}}{\pgfqpoint{1.047555in}{0.668784in}}{\pgfqpoint{1.055369in}{0.676598in}}%
\pgfpathcurveto{\pgfqpoint{1.063182in}{0.684411in}}{\pgfqpoint{1.067573in}{0.695010in}}{\pgfqpoint{1.067573in}{0.706060in}}%
\pgfpathcurveto{\pgfqpoint{1.067573in}{0.717111in}}{\pgfqpoint{1.063182in}{0.727710in}}{\pgfqpoint{1.055369in}{0.735523in}}%
\pgfpathcurveto{\pgfqpoint{1.047555in}{0.743337in}}{\pgfqpoint{1.036956in}{0.747727in}}{\pgfqpoint{1.025906in}{0.747727in}}%
\pgfpathcurveto{\pgfqpoint{1.014856in}{0.747727in}}{\pgfqpoint{1.004257in}{0.743337in}}{\pgfqpoint{0.996443in}{0.735523in}}%
\pgfpathcurveto{\pgfqpoint{0.988630in}{0.727710in}}{\pgfqpoint{0.984239in}{0.717111in}}{\pgfqpoint{0.984239in}{0.706060in}}%
\pgfpathcurveto{\pgfqpoint{0.984239in}{0.695010in}}{\pgfqpoint{0.988630in}{0.684411in}}{\pgfqpoint{0.996443in}{0.676598in}}%
\pgfpathcurveto{\pgfqpoint{1.004257in}{0.668784in}}{\pgfqpoint{1.014856in}{0.664394in}}{\pgfqpoint{1.025906in}{0.664394in}}%
\pgfpathclose%
\pgfusepath{stroke,fill}%
\end{pgfscope}%
\begin{pgfscope}%
\pgfpathrectangle{\pgfqpoint{0.800000in}{0.528000in}}{\pgfqpoint{4.960000in}{3.696000in}}%
\pgfusepath{clip}%
\pgfsetbuttcap%
\pgfsetroundjoin%
\definecolor{currentfill}{rgb}{0.000000,0.000000,0.000000}%
\pgfsetfillcolor{currentfill}%
\pgfsetlinewidth{1.003750pt}%
\definecolor{currentstroke}{rgb}{0.000000,0.000000,0.000000}%
\pgfsetstrokecolor{currentstroke}%
\pgfsetdash{}{0pt}%
\pgfpathmoveto{\pgfqpoint{1.025906in}{0.664394in}}%
\pgfpathcurveto{\pgfqpoint{1.036956in}{0.664394in}}{\pgfqpoint{1.047555in}{0.668784in}}{\pgfqpoint{1.055369in}{0.676598in}}%
\pgfpathcurveto{\pgfqpoint{1.063182in}{0.684411in}}{\pgfqpoint{1.067573in}{0.695010in}}{\pgfqpoint{1.067573in}{0.706060in}}%
\pgfpathcurveto{\pgfqpoint{1.067573in}{0.717111in}}{\pgfqpoint{1.063182in}{0.727710in}}{\pgfqpoint{1.055369in}{0.735523in}}%
\pgfpathcurveto{\pgfqpoint{1.047555in}{0.743337in}}{\pgfqpoint{1.036956in}{0.747727in}}{\pgfqpoint{1.025906in}{0.747727in}}%
\pgfpathcurveto{\pgfqpoint{1.014856in}{0.747727in}}{\pgfqpoint{1.004257in}{0.743337in}}{\pgfqpoint{0.996443in}{0.735523in}}%
\pgfpathcurveto{\pgfqpoint{0.988630in}{0.727710in}}{\pgfqpoint{0.984239in}{0.717111in}}{\pgfqpoint{0.984239in}{0.706060in}}%
\pgfpathcurveto{\pgfqpoint{0.984239in}{0.695010in}}{\pgfqpoint{0.988630in}{0.684411in}}{\pgfqpoint{0.996443in}{0.676598in}}%
\pgfpathcurveto{\pgfqpoint{1.004257in}{0.668784in}}{\pgfqpoint{1.014856in}{0.664394in}}{\pgfqpoint{1.025906in}{0.664394in}}%
\pgfpathclose%
\pgfusepath{stroke,fill}%
\end{pgfscope}%
\begin{pgfscope}%
\pgfpathrectangle{\pgfqpoint{0.800000in}{0.528000in}}{\pgfqpoint{4.960000in}{3.696000in}}%
\pgfusepath{clip}%
\pgfsetbuttcap%
\pgfsetroundjoin%
\definecolor{currentfill}{rgb}{0.000000,0.000000,0.000000}%
\pgfsetfillcolor{currentfill}%
\pgfsetlinewidth{1.003750pt}%
\definecolor{currentstroke}{rgb}{0.000000,0.000000,0.000000}%
\pgfsetstrokecolor{currentstroke}%
\pgfsetdash{}{0pt}%
\pgfpathmoveto{\pgfqpoint{1.025906in}{0.664394in}}%
\pgfpathcurveto{\pgfqpoint{1.036956in}{0.664394in}}{\pgfqpoint{1.047555in}{0.668784in}}{\pgfqpoint{1.055369in}{0.676598in}}%
\pgfpathcurveto{\pgfqpoint{1.063182in}{0.684411in}}{\pgfqpoint{1.067573in}{0.695010in}}{\pgfqpoint{1.067573in}{0.706060in}}%
\pgfpathcurveto{\pgfqpoint{1.067573in}{0.717111in}}{\pgfqpoint{1.063182in}{0.727710in}}{\pgfqpoint{1.055369in}{0.735523in}}%
\pgfpathcurveto{\pgfqpoint{1.047555in}{0.743337in}}{\pgfqpoint{1.036956in}{0.747727in}}{\pgfqpoint{1.025906in}{0.747727in}}%
\pgfpathcurveto{\pgfqpoint{1.014856in}{0.747727in}}{\pgfqpoint{1.004257in}{0.743337in}}{\pgfqpoint{0.996443in}{0.735523in}}%
\pgfpathcurveto{\pgfqpoint{0.988630in}{0.727710in}}{\pgfqpoint{0.984239in}{0.717111in}}{\pgfqpoint{0.984239in}{0.706060in}}%
\pgfpathcurveto{\pgfqpoint{0.984239in}{0.695010in}}{\pgfqpoint{0.988630in}{0.684411in}}{\pgfqpoint{0.996443in}{0.676598in}}%
\pgfpathcurveto{\pgfqpoint{1.004257in}{0.668784in}}{\pgfqpoint{1.014856in}{0.664394in}}{\pgfqpoint{1.025906in}{0.664394in}}%
\pgfpathclose%
\pgfusepath{stroke,fill}%
\end{pgfscope}%
\begin{pgfscope}%
\pgfpathrectangle{\pgfqpoint{0.800000in}{0.528000in}}{\pgfqpoint{4.960000in}{3.696000in}}%
\pgfusepath{clip}%
\pgfsetbuttcap%
\pgfsetroundjoin%
\definecolor{currentfill}{rgb}{0.000000,0.000000,0.000000}%
\pgfsetfillcolor{currentfill}%
\pgfsetlinewidth{1.003750pt}%
\definecolor{currentstroke}{rgb}{0.000000,0.000000,0.000000}%
\pgfsetstrokecolor{currentstroke}%
\pgfsetdash{}{0pt}%
\pgfpathmoveto{\pgfqpoint{1.025906in}{0.664394in}}%
\pgfpathcurveto{\pgfqpoint{1.036956in}{0.664394in}}{\pgfqpoint{1.047555in}{0.668784in}}{\pgfqpoint{1.055369in}{0.676598in}}%
\pgfpathcurveto{\pgfqpoint{1.063182in}{0.684411in}}{\pgfqpoint{1.067573in}{0.695010in}}{\pgfqpoint{1.067573in}{0.706060in}}%
\pgfpathcurveto{\pgfqpoint{1.067573in}{0.717111in}}{\pgfqpoint{1.063182in}{0.727710in}}{\pgfqpoint{1.055369in}{0.735523in}}%
\pgfpathcurveto{\pgfqpoint{1.047555in}{0.743337in}}{\pgfqpoint{1.036956in}{0.747727in}}{\pgfqpoint{1.025906in}{0.747727in}}%
\pgfpathcurveto{\pgfqpoint{1.014856in}{0.747727in}}{\pgfqpoint{1.004257in}{0.743337in}}{\pgfqpoint{0.996443in}{0.735523in}}%
\pgfpathcurveto{\pgfqpoint{0.988630in}{0.727710in}}{\pgfqpoint{0.984239in}{0.717111in}}{\pgfqpoint{0.984239in}{0.706060in}}%
\pgfpathcurveto{\pgfqpoint{0.984239in}{0.695010in}}{\pgfqpoint{0.988630in}{0.684411in}}{\pgfqpoint{0.996443in}{0.676598in}}%
\pgfpathcurveto{\pgfqpoint{1.004257in}{0.668784in}}{\pgfqpoint{1.014856in}{0.664394in}}{\pgfqpoint{1.025906in}{0.664394in}}%
\pgfpathclose%
\pgfusepath{stroke,fill}%
\end{pgfscope}%
\begin{pgfscope}%
\pgfpathrectangle{\pgfqpoint{0.800000in}{0.528000in}}{\pgfqpoint{4.960000in}{3.696000in}}%
\pgfusepath{clip}%
\pgfsetbuttcap%
\pgfsetroundjoin%
\definecolor{currentfill}{rgb}{0.000000,0.000000,0.000000}%
\pgfsetfillcolor{currentfill}%
\pgfsetlinewidth{1.003750pt}%
\definecolor{currentstroke}{rgb}{0.000000,0.000000,0.000000}%
\pgfsetstrokecolor{currentstroke}%
\pgfsetdash{}{0pt}%
\pgfpathmoveto{\pgfqpoint{1.025906in}{0.664394in}}%
\pgfpathcurveto{\pgfqpoint{1.036956in}{0.664394in}}{\pgfqpoint{1.047555in}{0.668784in}}{\pgfqpoint{1.055369in}{0.676598in}}%
\pgfpathcurveto{\pgfqpoint{1.063182in}{0.684411in}}{\pgfqpoint{1.067573in}{0.695010in}}{\pgfqpoint{1.067573in}{0.706060in}}%
\pgfpathcurveto{\pgfqpoint{1.067573in}{0.717111in}}{\pgfqpoint{1.063182in}{0.727710in}}{\pgfqpoint{1.055369in}{0.735523in}}%
\pgfpathcurveto{\pgfqpoint{1.047555in}{0.743337in}}{\pgfqpoint{1.036956in}{0.747727in}}{\pgfqpoint{1.025906in}{0.747727in}}%
\pgfpathcurveto{\pgfqpoint{1.014856in}{0.747727in}}{\pgfqpoint{1.004257in}{0.743337in}}{\pgfqpoint{0.996443in}{0.735523in}}%
\pgfpathcurveto{\pgfqpoint{0.988630in}{0.727710in}}{\pgfqpoint{0.984239in}{0.717111in}}{\pgfqpoint{0.984239in}{0.706060in}}%
\pgfpathcurveto{\pgfqpoint{0.984239in}{0.695010in}}{\pgfqpoint{0.988630in}{0.684411in}}{\pgfqpoint{0.996443in}{0.676598in}}%
\pgfpathcurveto{\pgfqpoint{1.004257in}{0.668784in}}{\pgfqpoint{1.014856in}{0.664394in}}{\pgfqpoint{1.025906in}{0.664394in}}%
\pgfpathclose%
\pgfusepath{stroke,fill}%
\end{pgfscope}%
\begin{pgfscope}%
\pgfpathrectangle{\pgfqpoint{0.800000in}{0.528000in}}{\pgfqpoint{4.960000in}{3.696000in}}%
\pgfusepath{clip}%
\pgfsetbuttcap%
\pgfsetroundjoin%
\definecolor{currentfill}{rgb}{0.000000,0.000000,0.000000}%
\pgfsetfillcolor{currentfill}%
\pgfsetlinewidth{1.003750pt}%
\definecolor{currentstroke}{rgb}{0.000000,0.000000,0.000000}%
\pgfsetstrokecolor{currentstroke}%
\pgfsetdash{}{0pt}%
\pgfpathmoveto{\pgfqpoint{1.025906in}{0.664394in}}%
\pgfpathcurveto{\pgfqpoint{1.036956in}{0.664394in}}{\pgfqpoint{1.047555in}{0.668784in}}{\pgfqpoint{1.055369in}{0.676598in}}%
\pgfpathcurveto{\pgfqpoint{1.063182in}{0.684411in}}{\pgfqpoint{1.067573in}{0.695010in}}{\pgfqpoint{1.067573in}{0.706060in}}%
\pgfpathcurveto{\pgfqpoint{1.067573in}{0.717111in}}{\pgfqpoint{1.063182in}{0.727710in}}{\pgfqpoint{1.055369in}{0.735523in}}%
\pgfpathcurveto{\pgfqpoint{1.047555in}{0.743337in}}{\pgfqpoint{1.036956in}{0.747727in}}{\pgfqpoint{1.025906in}{0.747727in}}%
\pgfpathcurveto{\pgfqpoint{1.014856in}{0.747727in}}{\pgfqpoint{1.004257in}{0.743337in}}{\pgfqpoint{0.996443in}{0.735523in}}%
\pgfpathcurveto{\pgfqpoint{0.988630in}{0.727710in}}{\pgfqpoint{0.984239in}{0.717111in}}{\pgfqpoint{0.984239in}{0.706060in}}%
\pgfpathcurveto{\pgfqpoint{0.984239in}{0.695010in}}{\pgfqpoint{0.988630in}{0.684411in}}{\pgfqpoint{0.996443in}{0.676598in}}%
\pgfpathcurveto{\pgfqpoint{1.004257in}{0.668784in}}{\pgfqpoint{1.014856in}{0.664394in}}{\pgfqpoint{1.025906in}{0.664394in}}%
\pgfpathclose%
\pgfusepath{stroke,fill}%
\end{pgfscope}%
\begin{pgfscope}%
\pgfpathrectangle{\pgfqpoint{0.800000in}{0.528000in}}{\pgfqpoint{4.960000in}{3.696000in}}%
\pgfusepath{clip}%
\pgfsetbuttcap%
\pgfsetroundjoin%
\definecolor{currentfill}{rgb}{0.000000,0.000000,0.000000}%
\pgfsetfillcolor{currentfill}%
\pgfsetlinewidth{1.003750pt}%
\definecolor{currentstroke}{rgb}{0.000000,0.000000,0.000000}%
\pgfsetstrokecolor{currentstroke}%
\pgfsetdash{}{0pt}%
\pgfpathmoveto{\pgfqpoint{1.025906in}{0.664394in}}%
\pgfpathcurveto{\pgfqpoint{1.036956in}{0.664394in}}{\pgfqpoint{1.047555in}{0.668784in}}{\pgfqpoint{1.055369in}{0.676598in}}%
\pgfpathcurveto{\pgfqpoint{1.063182in}{0.684411in}}{\pgfqpoint{1.067573in}{0.695010in}}{\pgfqpoint{1.067573in}{0.706060in}}%
\pgfpathcurveto{\pgfqpoint{1.067573in}{0.717111in}}{\pgfqpoint{1.063182in}{0.727710in}}{\pgfqpoint{1.055369in}{0.735523in}}%
\pgfpathcurveto{\pgfqpoint{1.047555in}{0.743337in}}{\pgfqpoint{1.036956in}{0.747727in}}{\pgfqpoint{1.025906in}{0.747727in}}%
\pgfpathcurveto{\pgfqpoint{1.014856in}{0.747727in}}{\pgfqpoint{1.004257in}{0.743337in}}{\pgfqpoint{0.996443in}{0.735523in}}%
\pgfpathcurveto{\pgfqpoint{0.988630in}{0.727710in}}{\pgfqpoint{0.984239in}{0.717111in}}{\pgfqpoint{0.984239in}{0.706060in}}%
\pgfpathcurveto{\pgfqpoint{0.984239in}{0.695010in}}{\pgfqpoint{0.988630in}{0.684411in}}{\pgfqpoint{0.996443in}{0.676598in}}%
\pgfpathcurveto{\pgfqpoint{1.004257in}{0.668784in}}{\pgfqpoint{1.014856in}{0.664394in}}{\pgfqpoint{1.025906in}{0.664394in}}%
\pgfpathclose%
\pgfusepath{stroke,fill}%
\end{pgfscope}%
\begin{pgfscope}%
\pgfpathrectangle{\pgfqpoint{0.800000in}{0.528000in}}{\pgfqpoint{4.960000in}{3.696000in}}%
\pgfusepath{clip}%
\pgfsetbuttcap%
\pgfsetroundjoin%
\definecolor{currentfill}{rgb}{0.000000,0.000000,0.000000}%
\pgfsetfillcolor{currentfill}%
\pgfsetlinewidth{1.003750pt}%
\definecolor{currentstroke}{rgb}{0.000000,0.000000,0.000000}%
\pgfsetstrokecolor{currentstroke}%
\pgfsetdash{}{0pt}%
\pgfpathmoveto{\pgfqpoint{2.518786in}{1.771040in}}%
\pgfpathcurveto{\pgfqpoint{2.529836in}{1.771040in}}{\pgfqpoint{2.540435in}{1.775431in}}{\pgfqpoint{2.548249in}{1.783244in}}%
\pgfpathcurveto{\pgfqpoint{2.556062in}{1.791058in}}{\pgfqpoint{2.560452in}{1.801657in}}{\pgfqpoint{2.560452in}{1.812707in}}%
\pgfpathcurveto{\pgfqpoint{2.560452in}{1.823757in}}{\pgfqpoint{2.556062in}{1.834356in}}{\pgfqpoint{2.548249in}{1.842170in}}%
\pgfpathcurveto{\pgfqpoint{2.540435in}{1.849983in}}{\pgfqpoint{2.529836in}{1.854374in}}{\pgfqpoint{2.518786in}{1.854374in}}%
\pgfpathcurveto{\pgfqpoint{2.507736in}{1.854374in}}{\pgfqpoint{2.497137in}{1.849983in}}{\pgfqpoint{2.489323in}{1.842170in}}%
\pgfpathcurveto{\pgfqpoint{2.481509in}{1.834356in}}{\pgfqpoint{2.477119in}{1.823757in}}{\pgfqpoint{2.477119in}{1.812707in}}%
\pgfpathcurveto{\pgfqpoint{2.477119in}{1.801657in}}{\pgfqpoint{2.481509in}{1.791058in}}{\pgfqpoint{2.489323in}{1.783244in}}%
\pgfpathcurveto{\pgfqpoint{2.497137in}{1.775431in}}{\pgfqpoint{2.507736in}{1.771040in}}{\pgfqpoint{2.518786in}{1.771040in}}%
\pgfpathclose%
\pgfusepath{stroke,fill}%
\end{pgfscope}%
\begin{pgfscope}%
\pgfpathrectangle{\pgfqpoint{0.800000in}{0.528000in}}{\pgfqpoint{4.960000in}{3.696000in}}%
\pgfusepath{clip}%
\pgfsetbuttcap%
\pgfsetroundjoin%
\definecolor{currentfill}{rgb}{0.000000,0.000000,0.000000}%
\pgfsetfillcolor{currentfill}%
\pgfsetlinewidth{1.003750pt}%
\definecolor{currentstroke}{rgb}{0.000000,0.000000,0.000000}%
\pgfsetstrokecolor{currentstroke}%
\pgfsetdash{}{0pt}%
\pgfpathmoveto{\pgfqpoint{2.518786in}{1.771040in}}%
\pgfpathcurveto{\pgfqpoint{2.529836in}{1.771040in}}{\pgfqpoint{2.540435in}{1.775431in}}{\pgfqpoint{2.548249in}{1.783244in}}%
\pgfpathcurveto{\pgfqpoint{2.556062in}{1.791058in}}{\pgfqpoint{2.560452in}{1.801657in}}{\pgfqpoint{2.560452in}{1.812707in}}%
\pgfpathcurveto{\pgfqpoint{2.560452in}{1.823757in}}{\pgfqpoint{2.556062in}{1.834356in}}{\pgfqpoint{2.548249in}{1.842170in}}%
\pgfpathcurveto{\pgfqpoint{2.540435in}{1.849983in}}{\pgfqpoint{2.529836in}{1.854374in}}{\pgfqpoint{2.518786in}{1.854374in}}%
\pgfpathcurveto{\pgfqpoint{2.507736in}{1.854374in}}{\pgfqpoint{2.497137in}{1.849983in}}{\pgfqpoint{2.489323in}{1.842170in}}%
\pgfpathcurveto{\pgfqpoint{2.481509in}{1.834356in}}{\pgfqpoint{2.477119in}{1.823757in}}{\pgfqpoint{2.477119in}{1.812707in}}%
\pgfpathcurveto{\pgfqpoint{2.477119in}{1.801657in}}{\pgfqpoint{2.481509in}{1.791058in}}{\pgfqpoint{2.489323in}{1.783244in}}%
\pgfpathcurveto{\pgfqpoint{2.497137in}{1.775431in}}{\pgfqpoint{2.507736in}{1.771040in}}{\pgfqpoint{2.518786in}{1.771040in}}%
\pgfpathclose%
\pgfusepath{stroke,fill}%
\end{pgfscope}%
\begin{pgfscope}%
\pgfpathrectangle{\pgfqpoint{0.800000in}{0.528000in}}{\pgfqpoint{4.960000in}{3.696000in}}%
\pgfusepath{clip}%
\pgfsetbuttcap%
\pgfsetroundjoin%
\definecolor{currentfill}{rgb}{0.000000,0.000000,0.000000}%
\pgfsetfillcolor{currentfill}%
\pgfsetlinewidth{1.003750pt}%
\definecolor{currentstroke}{rgb}{0.000000,0.000000,0.000000}%
\pgfsetstrokecolor{currentstroke}%
\pgfsetdash{}{0pt}%
\pgfpathmoveto{\pgfqpoint{2.518786in}{1.771040in}}%
\pgfpathcurveto{\pgfqpoint{2.529836in}{1.771040in}}{\pgfqpoint{2.540435in}{1.775431in}}{\pgfqpoint{2.548249in}{1.783244in}}%
\pgfpathcurveto{\pgfqpoint{2.556062in}{1.791058in}}{\pgfqpoint{2.560452in}{1.801657in}}{\pgfqpoint{2.560452in}{1.812707in}}%
\pgfpathcurveto{\pgfqpoint{2.560452in}{1.823757in}}{\pgfqpoint{2.556062in}{1.834356in}}{\pgfqpoint{2.548249in}{1.842170in}}%
\pgfpathcurveto{\pgfqpoint{2.540435in}{1.849983in}}{\pgfqpoint{2.529836in}{1.854374in}}{\pgfqpoint{2.518786in}{1.854374in}}%
\pgfpathcurveto{\pgfqpoint{2.507736in}{1.854374in}}{\pgfqpoint{2.497137in}{1.849983in}}{\pgfqpoint{2.489323in}{1.842170in}}%
\pgfpathcurveto{\pgfqpoint{2.481509in}{1.834356in}}{\pgfqpoint{2.477119in}{1.823757in}}{\pgfqpoint{2.477119in}{1.812707in}}%
\pgfpathcurveto{\pgfqpoint{2.477119in}{1.801657in}}{\pgfqpoint{2.481509in}{1.791058in}}{\pgfqpoint{2.489323in}{1.783244in}}%
\pgfpathcurveto{\pgfqpoint{2.497137in}{1.775431in}}{\pgfqpoint{2.507736in}{1.771040in}}{\pgfqpoint{2.518786in}{1.771040in}}%
\pgfpathclose%
\pgfusepath{stroke,fill}%
\end{pgfscope}%
\begin{pgfscope}%
\pgfpathrectangle{\pgfqpoint{0.800000in}{0.528000in}}{\pgfqpoint{4.960000in}{3.696000in}}%
\pgfusepath{clip}%
\pgfsetbuttcap%
\pgfsetroundjoin%
\definecolor{currentfill}{rgb}{0.000000,0.000000,0.000000}%
\pgfsetfillcolor{currentfill}%
\pgfsetlinewidth{1.003750pt}%
\definecolor{currentstroke}{rgb}{0.000000,0.000000,0.000000}%
\pgfsetstrokecolor{currentstroke}%
\pgfsetdash{}{0pt}%
\pgfpathmoveto{\pgfqpoint{2.518786in}{1.771040in}}%
\pgfpathcurveto{\pgfqpoint{2.529836in}{1.771040in}}{\pgfqpoint{2.540435in}{1.775431in}}{\pgfqpoint{2.548249in}{1.783244in}}%
\pgfpathcurveto{\pgfqpoint{2.556062in}{1.791058in}}{\pgfqpoint{2.560452in}{1.801657in}}{\pgfqpoint{2.560452in}{1.812707in}}%
\pgfpathcurveto{\pgfqpoint{2.560452in}{1.823757in}}{\pgfqpoint{2.556062in}{1.834356in}}{\pgfqpoint{2.548249in}{1.842170in}}%
\pgfpathcurveto{\pgfqpoint{2.540435in}{1.849983in}}{\pgfqpoint{2.529836in}{1.854374in}}{\pgfqpoint{2.518786in}{1.854374in}}%
\pgfpathcurveto{\pgfqpoint{2.507736in}{1.854374in}}{\pgfqpoint{2.497137in}{1.849983in}}{\pgfqpoint{2.489323in}{1.842170in}}%
\pgfpathcurveto{\pgfqpoint{2.481509in}{1.834356in}}{\pgfqpoint{2.477119in}{1.823757in}}{\pgfqpoint{2.477119in}{1.812707in}}%
\pgfpathcurveto{\pgfqpoint{2.477119in}{1.801657in}}{\pgfqpoint{2.481509in}{1.791058in}}{\pgfqpoint{2.489323in}{1.783244in}}%
\pgfpathcurveto{\pgfqpoint{2.497137in}{1.775431in}}{\pgfqpoint{2.507736in}{1.771040in}}{\pgfqpoint{2.518786in}{1.771040in}}%
\pgfpathclose%
\pgfusepath{stroke,fill}%
\end{pgfscope}%
\begin{pgfscope}%
\pgfpathrectangle{\pgfqpoint{0.800000in}{0.528000in}}{\pgfqpoint{4.960000in}{3.696000in}}%
\pgfusepath{clip}%
\pgfsetbuttcap%
\pgfsetroundjoin%
\definecolor{currentfill}{rgb}{0.000000,0.000000,0.000000}%
\pgfsetfillcolor{currentfill}%
\pgfsetlinewidth{1.003750pt}%
\definecolor{currentstroke}{rgb}{0.000000,0.000000,0.000000}%
\pgfsetstrokecolor{currentstroke}%
\pgfsetdash{}{0pt}%
\pgfpathmoveto{\pgfqpoint{2.518786in}{1.771040in}}%
\pgfpathcurveto{\pgfqpoint{2.529836in}{1.771040in}}{\pgfqpoint{2.540435in}{1.775431in}}{\pgfqpoint{2.548249in}{1.783244in}}%
\pgfpathcurveto{\pgfqpoint{2.556062in}{1.791058in}}{\pgfqpoint{2.560452in}{1.801657in}}{\pgfqpoint{2.560452in}{1.812707in}}%
\pgfpathcurveto{\pgfqpoint{2.560452in}{1.823757in}}{\pgfqpoint{2.556062in}{1.834356in}}{\pgfqpoint{2.548249in}{1.842170in}}%
\pgfpathcurveto{\pgfqpoint{2.540435in}{1.849983in}}{\pgfqpoint{2.529836in}{1.854374in}}{\pgfqpoint{2.518786in}{1.854374in}}%
\pgfpathcurveto{\pgfqpoint{2.507736in}{1.854374in}}{\pgfqpoint{2.497137in}{1.849983in}}{\pgfqpoint{2.489323in}{1.842170in}}%
\pgfpathcurveto{\pgfqpoint{2.481509in}{1.834356in}}{\pgfqpoint{2.477119in}{1.823757in}}{\pgfqpoint{2.477119in}{1.812707in}}%
\pgfpathcurveto{\pgfqpoint{2.477119in}{1.801657in}}{\pgfqpoint{2.481509in}{1.791058in}}{\pgfqpoint{2.489323in}{1.783244in}}%
\pgfpathcurveto{\pgfqpoint{2.497137in}{1.775431in}}{\pgfqpoint{2.507736in}{1.771040in}}{\pgfqpoint{2.518786in}{1.771040in}}%
\pgfpathclose%
\pgfusepath{stroke,fill}%
\end{pgfscope}%
\begin{pgfscope}%
\pgfpathrectangle{\pgfqpoint{0.800000in}{0.528000in}}{\pgfqpoint{4.960000in}{3.696000in}}%
\pgfusepath{clip}%
\pgfsetbuttcap%
\pgfsetroundjoin%
\definecolor{currentfill}{rgb}{0.000000,0.000000,0.000000}%
\pgfsetfillcolor{currentfill}%
\pgfsetlinewidth{1.003750pt}%
\definecolor{currentstroke}{rgb}{0.000000,0.000000,0.000000}%
\pgfsetstrokecolor{currentstroke}%
\pgfsetdash{}{0pt}%
\pgfpathmoveto{\pgfqpoint{2.518786in}{1.771040in}}%
\pgfpathcurveto{\pgfqpoint{2.529836in}{1.771040in}}{\pgfqpoint{2.540435in}{1.775431in}}{\pgfqpoint{2.548249in}{1.783244in}}%
\pgfpathcurveto{\pgfqpoint{2.556062in}{1.791058in}}{\pgfqpoint{2.560452in}{1.801657in}}{\pgfqpoint{2.560452in}{1.812707in}}%
\pgfpathcurveto{\pgfqpoint{2.560452in}{1.823757in}}{\pgfqpoint{2.556062in}{1.834356in}}{\pgfqpoint{2.548249in}{1.842170in}}%
\pgfpathcurveto{\pgfqpoint{2.540435in}{1.849983in}}{\pgfqpoint{2.529836in}{1.854374in}}{\pgfqpoint{2.518786in}{1.854374in}}%
\pgfpathcurveto{\pgfqpoint{2.507736in}{1.854374in}}{\pgfqpoint{2.497137in}{1.849983in}}{\pgfqpoint{2.489323in}{1.842170in}}%
\pgfpathcurveto{\pgfqpoint{2.481509in}{1.834356in}}{\pgfqpoint{2.477119in}{1.823757in}}{\pgfqpoint{2.477119in}{1.812707in}}%
\pgfpathcurveto{\pgfqpoint{2.477119in}{1.801657in}}{\pgfqpoint{2.481509in}{1.791058in}}{\pgfqpoint{2.489323in}{1.783244in}}%
\pgfpathcurveto{\pgfqpoint{2.497137in}{1.775431in}}{\pgfqpoint{2.507736in}{1.771040in}}{\pgfqpoint{2.518786in}{1.771040in}}%
\pgfpathclose%
\pgfusepath{stroke,fill}%
\end{pgfscope}%
\begin{pgfscope}%
\pgfpathrectangle{\pgfqpoint{0.800000in}{0.528000in}}{\pgfqpoint{4.960000in}{3.696000in}}%
\pgfusepath{clip}%
\pgfsetbuttcap%
\pgfsetroundjoin%
\definecolor{currentfill}{rgb}{0.000000,0.000000,0.000000}%
\pgfsetfillcolor{currentfill}%
\pgfsetlinewidth{1.003750pt}%
\definecolor{currentstroke}{rgb}{0.000000,0.000000,0.000000}%
\pgfsetstrokecolor{currentstroke}%
\pgfsetdash{}{0pt}%
\pgfpathmoveto{\pgfqpoint{2.518786in}{1.771040in}}%
\pgfpathcurveto{\pgfqpoint{2.529836in}{1.771040in}}{\pgfqpoint{2.540435in}{1.775431in}}{\pgfqpoint{2.548249in}{1.783244in}}%
\pgfpathcurveto{\pgfqpoint{2.556062in}{1.791058in}}{\pgfqpoint{2.560452in}{1.801657in}}{\pgfqpoint{2.560452in}{1.812707in}}%
\pgfpathcurveto{\pgfqpoint{2.560452in}{1.823757in}}{\pgfqpoint{2.556062in}{1.834356in}}{\pgfqpoint{2.548249in}{1.842170in}}%
\pgfpathcurveto{\pgfqpoint{2.540435in}{1.849983in}}{\pgfqpoint{2.529836in}{1.854374in}}{\pgfqpoint{2.518786in}{1.854374in}}%
\pgfpathcurveto{\pgfqpoint{2.507736in}{1.854374in}}{\pgfqpoint{2.497137in}{1.849983in}}{\pgfqpoint{2.489323in}{1.842170in}}%
\pgfpathcurveto{\pgfqpoint{2.481509in}{1.834356in}}{\pgfqpoint{2.477119in}{1.823757in}}{\pgfqpoint{2.477119in}{1.812707in}}%
\pgfpathcurveto{\pgfqpoint{2.477119in}{1.801657in}}{\pgfqpoint{2.481509in}{1.791058in}}{\pgfqpoint{2.489323in}{1.783244in}}%
\pgfpathcurveto{\pgfqpoint{2.497137in}{1.775431in}}{\pgfqpoint{2.507736in}{1.771040in}}{\pgfqpoint{2.518786in}{1.771040in}}%
\pgfpathclose%
\pgfusepath{stroke,fill}%
\end{pgfscope}%
\begin{pgfscope}%
\pgfpathrectangle{\pgfqpoint{0.800000in}{0.528000in}}{\pgfqpoint{4.960000in}{3.696000in}}%
\pgfusepath{clip}%
\pgfsetbuttcap%
\pgfsetroundjoin%
\definecolor{currentfill}{rgb}{0.000000,0.000000,0.000000}%
\pgfsetfillcolor{currentfill}%
\pgfsetlinewidth{1.003750pt}%
\definecolor{currentstroke}{rgb}{0.000000,0.000000,0.000000}%
\pgfsetstrokecolor{currentstroke}%
\pgfsetdash{}{0pt}%
\pgfpathmoveto{\pgfqpoint{2.518786in}{0.664394in}}%
\pgfpathcurveto{\pgfqpoint{2.529836in}{0.664394in}}{\pgfqpoint{2.540435in}{0.668784in}}{\pgfqpoint{2.548249in}{0.676598in}}%
\pgfpathcurveto{\pgfqpoint{2.556062in}{0.684411in}}{\pgfqpoint{2.560452in}{0.695010in}}{\pgfqpoint{2.560452in}{0.706060in}}%
\pgfpathcurveto{\pgfqpoint{2.560452in}{0.717111in}}{\pgfqpoint{2.556062in}{0.727710in}}{\pgfqpoint{2.548249in}{0.735523in}}%
\pgfpathcurveto{\pgfqpoint{2.540435in}{0.743337in}}{\pgfqpoint{2.529836in}{0.747727in}}{\pgfqpoint{2.518786in}{0.747727in}}%
\pgfpathcurveto{\pgfqpoint{2.507736in}{0.747727in}}{\pgfqpoint{2.497137in}{0.743337in}}{\pgfqpoint{2.489323in}{0.735523in}}%
\pgfpathcurveto{\pgfqpoint{2.481509in}{0.727710in}}{\pgfqpoint{2.477119in}{0.717111in}}{\pgfqpoint{2.477119in}{0.706060in}}%
\pgfpathcurveto{\pgfqpoint{2.477119in}{0.695010in}}{\pgfqpoint{2.481509in}{0.684411in}}{\pgfqpoint{2.489323in}{0.676598in}}%
\pgfpathcurveto{\pgfqpoint{2.497137in}{0.668784in}}{\pgfqpoint{2.507736in}{0.664394in}}{\pgfqpoint{2.518786in}{0.664394in}}%
\pgfpathclose%
\pgfusepath{stroke,fill}%
\end{pgfscope}%
\begin{pgfscope}%
\pgfpathrectangle{\pgfqpoint{0.800000in}{0.528000in}}{\pgfqpoint{4.960000in}{3.696000in}}%
\pgfusepath{clip}%
\pgfsetbuttcap%
\pgfsetroundjoin%
\definecolor{currentfill}{rgb}{0.000000,0.000000,0.000000}%
\pgfsetfillcolor{currentfill}%
\pgfsetlinewidth{1.003750pt}%
\definecolor{currentstroke}{rgb}{0.000000,0.000000,0.000000}%
\pgfsetstrokecolor{currentstroke}%
\pgfsetdash{}{0pt}%
\pgfpathmoveto{\pgfqpoint{2.518786in}{1.771040in}}%
\pgfpathcurveto{\pgfqpoint{2.529836in}{1.771040in}}{\pgfqpoint{2.540435in}{1.775431in}}{\pgfqpoint{2.548249in}{1.783244in}}%
\pgfpathcurveto{\pgfqpoint{2.556062in}{1.791058in}}{\pgfqpoint{2.560452in}{1.801657in}}{\pgfqpoint{2.560452in}{1.812707in}}%
\pgfpathcurveto{\pgfqpoint{2.560452in}{1.823757in}}{\pgfqpoint{2.556062in}{1.834356in}}{\pgfqpoint{2.548249in}{1.842170in}}%
\pgfpathcurveto{\pgfqpoint{2.540435in}{1.849983in}}{\pgfqpoint{2.529836in}{1.854374in}}{\pgfqpoint{2.518786in}{1.854374in}}%
\pgfpathcurveto{\pgfqpoint{2.507736in}{1.854374in}}{\pgfqpoint{2.497137in}{1.849983in}}{\pgfqpoint{2.489323in}{1.842170in}}%
\pgfpathcurveto{\pgfqpoint{2.481509in}{1.834356in}}{\pgfqpoint{2.477119in}{1.823757in}}{\pgfqpoint{2.477119in}{1.812707in}}%
\pgfpathcurveto{\pgfqpoint{2.477119in}{1.801657in}}{\pgfqpoint{2.481509in}{1.791058in}}{\pgfqpoint{2.489323in}{1.783244in}}%
\pgfpathcurveto{\pgfqpoint{2.497137in}{1.775431in}}{\pgfqpoint{2.507736in}{1.771040in}}{\pgfqpoint{2.518786in}{1.771040in}}%
\pgfpathclose%
\pgfusepath{stroke,fill}%
\end{pgfscope}%
\begin{pgfscope}%
\pgfpathrectangle{\pgfqpoint{0.800000in}{0.528000in}}{\pgfqpoint{4.960000in}{3.696000in}}%
\pgfusepath{clip}%
\pgfsetbuttcap%
\pgfsetroundjoin%
\definecolor{currentfill}{rgb}{0.000000,0.000000,0.000000}%
\pgfsetfillcolor{currentfill}%
\pgfsetlinewidth{1.003750pt}%
\definecolor{currentstroke}{rgb}{0.000000,0.000000,0.000000}%
\pgfsetstrokecolor{currentstroke}%
\pgfsetdash{}{0pt}%
\pgfpathmoveto{\pgfqpoint{2.518786in}{1.771040in}}%
\pgfpathcurveto{\pgfqpoint{2.529836in}{1.771040in}}{\pgfqpoint{2.540435in}{1.775431in}}{\pgfqpoint{2.548249in}{1.783244in}}%
\pgfpathcurveto{\pgfqpoint{2.556062in}{1.791058in}}{\pgfqpoint{2.560452in}{1.801657in}}{\pgfqpoint{2.560452in}{1.812707in}}%
\pgfpathcurveto{\pgfqpoint{2.560452in}{1.823757in}}{\pgfqpoint{2.556062in}{1.834356in}}{\pgfqpoint{2.548249in}{1.842170in}}%
\pgfpathcurveto{\pgfqpoint{2.540435in}{1.849983in}}{\pgfqpoint{2.529836in}{1.854374in}}{\pgfqpoint{2.518786in}{1.854374in}}%
\pgfpathcurveto{\pgfqpoint{2.507736in}{1.854374in}}{\pgfqpoint{2.497137in}{1.849983in}}{\pgfqpoint{2.489323in}{1.842170in}}%
\pgfpathcurveto{\pgfqpoint{2.481509in}{1.834356in}}{\pgfqpoint{2.477119in}{1.823757in}}{\pgfqpoint{2.477119in}{1.812707in}}%
\pgfpathcurveto{\pgfqpoint{2.477119in}{1.801657in}}{\pgfqpoint{2.481509in}{1.791058in}}{\pgfqpoint{2.489323in}{1.783244in}}%
\pgfpathcurveto{\pgfqpoint{2.497137in}{1.775431in}}{\pgfqpoint{2.507736in}{1.771040in}}{\pgfqpoint{2.518786in}{1.771040in}}%
\pgfpathclose%
\pgfusepath{stroke,fill}%
\end{pgfscope}%
\begin{pgfscope}%
\pgfpathrectangle{\pgfqpoint{0.800000in}{0.528000in}}{\pgfqpoint{4.960000in}{3.696000in}}%
\pgfusepath{clip}%
\pgfsetbuttcap%
\pgfsetroundjoin%
\definecolor{currentfill}{rgb}{0.000000,0.000000,0.000000}%
\pgfsetfillcolor{currentfill}%
\pgfsetlinewidth{1.003750pt}%
\definecolor{currentstroke}{rgb}{0.000000,0.000000,0.000000}%
\pgfsetstrokecolor{currentstroke}%
\pgfsetdash{}{0pt}%
\pgfpathmoveto{\pgfqpoint{2.518786in}{1.771040in}}%
\pgfpathcurveto{\pgfqpoint{2.529836in}{1.771040in}}{\pgfqpoint{2.540435in}{1.775431in}}{\pgfqpoint{2.548249in}{1.783244in}}%
\pgfpathcurveto{\pgfqpoint{2.556062in}{1.791058in}}{\pgfqpoint{2.560452in}{1.801657in}}{\pgfqpoint{2.560452in}{1.812707in}}%
\pgfpathcurveto{\pgfqpoint{2.560452in}{1.823757in}}{\pgfqpoint{2.556062in}{1.834356in}}{\pgfqpoint{2.548249in}{1.842170in}}%
\pgfpathcurveto{\pgfqpoint{2.540435in}{1.849983in}}{\pgfqpoint{2.529836in}{1.854374in}}{\pgfqpoint{2.518786in}{1.854374in}}%
\pgfpathcurveto{\pgfqpoint{2.507736in}{1.854374in}}{\pgfqpoint{2.497137in}{1.849983in}}{\pgfqpoint{2.489323in}{1.842170in}}%
\pgfpathcurveto{\pgfqpoint{2.481509in}{1.834356in}}{\pgfqpoint{2.477119in}{1.823757in}}{\pgfqpoint{2.477119in}{1.812707in}}%
\pgfpathcurveto{\pgfqpoint{2.477119in}{1.801657in}}{\pgfqpoint{2.481509in}{1.791058in}}{\pgfqpoint{2.489323in}{1.783244in}}%
\pgfpathcurveto{\pgfqpoint{2.497137in}{1.775431in}}{\pgfqpoint{2.507736in}{1.771040in}}{\pgfqpoint{2.518786in}{1.771040in}}%
\pgfpathclose%
\pgfusepath{stroke,fill}%
\end{pgfscope}%
\begin{pgfscope}%
\pgfpathrectangle{\pgfqpoint{0.800000in}{0.528000in}}{\pgfqpoint{4.960000in}{3.696000in}}%
\pgfusepath{clip}%
\pgfsetbuttcap%
\pgfsetroundjoin%
\definecolor{currentfill}{rgb}{0.000000,0.000000,0.000000}%
\pgfsetfillcolor{currentfill}%
\pgfsetlinewidth{1.003750pt}%
\definecolor{currentstroke}{rgb}{0.000000,0.000000,0.000000}%
\pgfsetstrokecolor{currentstroke}%
\pgfsetdash{}{0pt}%
\pgfpathmoveto{\pgfqpoint{2.518786in}{1.771040in}}%
\pgfpathcurveto{\pgfqpoint{2.529836in}{1.771040in}}{\pgfqpoint{2.540435in}{1.775431in}}{\pgfqpoint{2.548249in}{1.783244in}}%
\pgfpathcurveto{\pgfqpoint{2.556062in}{1.791058in}}{\pgfqpoint{2.560452in}{1.801657in}}{\pgfqpoint{2.560452in}{1.812707in}}%
\pgfpathcurveto{\pgfqpoint{2.560452in}{1.823757in}}{\pgfqpoint{2.556062in}{1.834356in}}{\pgfqpoint{2.548249in}{1.842170in}}%
\pgfpathcurveto{\pgfqpoint{2.540435in}{1.849983in}}{\pgfqpoint{2.529836in}{1.854374in}}{\pgfqpoint{2.518786in}{1.854374in}}%
\pgfpathcurveto{\pgfqpoint{2.507736in}{1.854374in}}{\pgfqpoint{2.497137in}{1.849983in}}{\pgfqpoint{2.489323in}{1.842170in}}%
\pgfpathcurveto{\pgfqpoint{2.481509in}{1.834356in}}{\pgfqpoint{2.477119in}{1.823757in}}{\pgfqpoint{2.477119in}{1.812707in}}%
\pgfpathcurveto{\pgfqpoint{2.477119in}{1.801657in}}{\pgfqpoint{2.481509in}{1.791058in}}{\pgfqpoint{2.489323in}{1.783244in}}%
\pgfpathcurveto{\pgfqpoint{2.497137in}{1.775431in}}{\pgfqpoint{2.507736in}{1.771040in}}{\pgfqpoint{2.518786in}{1.771040in}}%
\pgfpathclose%
\pgfusepath{stroke,fill}%
\end{pgfscope}%
\begin{pgfscope}%
\pgfpathrectangle{\pgfqpoint{0.800000in}{0.528000in}}{\pgfqpoint{4.960000in}{3.696000in}}%
\pgfusepath{clip}%
\pgfsetbuttcap%
\pgfsetroundjoin%
\definecolor{currentfill}{rgb}{0.000000,0.000000,0.000000}%
\pgfsetfillcolor{currentfill}%
\pgfsetlinewidth{1.003750pt}%
\definecolor{currentstroke}{rgb}{0.000000,0.000000,0.000000}%
\pgfsetstrokecolor{currentstroke}%
\pgfsetdash{}{0pt}%
\pgfpathmoveto{\pgfqpoint{2.518786in}{1.771040in}}%
\pgfpathcurveto{\pgfqpoint{2.529836in}{1.771040in}}{\pgfqpoint{2.540435in}{1.775431in}}{\pgfqpoint{2.548249in}{1.783244in}}%
\pgfpathcurveto{\pgfqpoint{2.556062in}{1.791058in}}{\pgfqpoint{2.560452in}{1.801657in}}{\pgfqpoint{2.560452in}{1.812707in}}%
\pgfpathcurveto{\pgfqpoint{2.560452in}{1.823757in}}{\pgfqpoint{2.556062in}{1.834356in}}{\pgfqpoint{2.548249in}{1.842170in}}%
\pgfpathcurveto{\pgfqpoint{2.540435in}{1.849983in}}{\pgfqpoint{2.529836in}{1.854374in}}{\pgfqpoint{2.518786in}{1.854374in}}%
\pgfpathcurveto{\pgfqpoint{2.507736in}{1.854374in}}{\pgfqpoint{2.497137in}{1.849983in}}{\pgfqpoint{2.489323in}{1.842170in}}%
\pgfpathcurveto{\pgfqpoint{2.481509in}{1.834356in}}{\pgfqpoint{2.477119in}{1.823757in}}{\pgfqpoint{2.477119in}{1.812707in}}%
\pgfpathcurveto{\pgfqpoint{2.477119in}{1.801657in}}{\pgfqpoint{2.481509in}{1.791058in}}{\pgfqpoint{2.489323in}{1.783244in}}%
\pgfpathcurveto{\pgfqpoint{2.497137in}{1.775431in}}{\pgfqpoint{2.507736in}{1.771040in}}{\pgfqpoint{2.518786in}{1.771040in}}%
\pgfpathclose%
\pgfusepath{stroke,fill}%
\end{pgfscope}%
\begin{pgfscope}%
\pgfpathrectangle{\pgfqpoint{0.800000in}{0.528000in}}{\pgfqpoint{4.960000in}{3.696000in}}%
\pgfusepath{clip}%
\pgfsetbuttcap%
\pgfsetroundjoin%
\definecolor{currentfill}{rgb}{0.000000,0.000000,0.000000}%
\pgfsetfillcolor{currentfill}%
\pgfsetlinewidth{1.003750pt}%
\definecolor{currentstroke}{rgb}{0.000000,0.000000,0.000000}%
\pgfsetstrokecolor{currentstroke}%
\pgfsetdash{}{0pt}%
\pgfpathmoveto{\pgfqpoint{2.518786in}{1.771040in}}%
\pgfpathcurveto{\pgfqpoint{2.529836in}{1.771040in}}{\pgfqpoint{2.540435in}{1.775431in}}{\pgfqpoint{2.548249in}{1.783244in}}%
\pgfpathcurveto{\pgfqpoint{2.556062in}{1.791058in}}{\pgfqpoint{2.560452in}{1.801657in}}{\pgfqpoint{2.560452in}{1.812707in}}%
\pgfpathcurveto{\pgfqpoint{2.560452in}{1.823757in}}{\pgfqpoint{2.556062in}{1.834356in}}{\pgfqpoint{2.548249in}{1.842170in}}%
\pgfpathcurveto{\pgfqpoint{2.540435in}{1.849983in}}{\pgfqpoint{2.529836in}{1.854374in}}{\pgfqpoint{2.518786in}{1.854374in}}%
\pgfpathcurveto{\pgfqpoint{2.507736in}{1.854374in}}{\pgfqpoint{2.497137in}{1.849983in}}{\pgfqpoint{2.489323in}{1.842170in}}%
\pgfpathcurveto{\pgfqpoint{2.481509in}{1.834356in}}{\pgfqpoint{2.477119in}{1.823757in}}{\pgfqpoint{2.477119in}{1.812707in}}%
\pgfpathcurveto{\pgfqpoint{2.477119in}{1.801657in}}{\pgfqpoint{2.481509in}{1.791058in}}{\pgfqpoint{2.489323in}{1.783244in}}%
\pgfpathcurveto{\pgfqpoint{2.497137in}{1.775431in}}{\pgfqpoint{2.507736in}{1.771040in}}{\pgfqpoint{2.518786in}{1.771040in}}%
\pgfpathclose%
\pgfusepath{stroke,fill}%
\end{pgfscope}%
\begin{pgfscope}%
\pgfpathrectangle{\pgfqpoint{0.800000in}{0.528000in}}{\pgfqpoint{4.960000in}{3.696000in}}%
\pgfusepath{clip}%
\pgfsetbuttcap%
\pgfsetroundjoin%
\definecolor{currentfill}{rgb}{0.000000,0.000000,0.000000}%
\pgfsetfillcolor{currentfill}%
\pgfsetlinewidth{1.003750pt}%
\definecolor{currentstroke}{rgb}{0.000000,0.000000,0.000000}%
\pgfsetstrokecolor{currentstroke}%
\pgfsetdash{}{0pt}%
\pgfpathmoveto{\pgfqpoint{2.518786in}{0.664394in}}%
\pgfpathcurveto{\pgfqpoint{2.529836in}{0.664394in}}{\pgfqpoint{2.540435in}{0.668784in}}{\pgfqpoint{2.548249in}{0.676598in}}%
\pgfpathcurveto{\pgfqpoint{2.556062in}{0.684411in}}{\pgfqpoint{2.560452in}{0.695010in}}{\pgfqpoint{2.560452in}{0.706060in}}%
\pgfpathcurveto{\pgfqpoint{2.560452in}{0.717111in}}{\pgfqpoint{2.556062in}{0.727710in}}{\pgfqpoint{2.548249in}{0.735523in}}%
\pgfpathcurveto{\pgfqpoint{2.540435in}{0.743337in}}{\pgfqpoint{2.529836in}{0.747727in}}{\pgfqpoint{2.518786in}{0.747727in}}%
\pgfpathcurveto{\pgfqpoint{2.507736in}{0.747727in}}{\pgfqpoint{2.497137in}{0.743337in}}{\pgfqpoint{2.489323in}{0.735523in}}%
\pgfpathcurveto{\pgfqpoint{2.481509in}{0.727710in}}{\pgfqpoint{2.477119in}{0.717111in}}{\pgfqpoint{2.477119in}{0.706060in}}%
\pgfpathcurveto{\pgfqpoint{2.477119in}{0.695010in}}{\pgfqpoint{2.481509in}{0.684411in}}{\pgfqpoint{2.489323in}{0.676598in}}%
\pgfpathcurveto{\pgfqpoint{2.497137in}{0.668784in}}{\pgfqpoint{2.507736in}{0.664394in}}{\pgfqpoint{2.518786in}{0.664394in}}%
\pgfpathclose%
\pgfusepath{stroke,fill}%
\end{pgfscope}%
\begin{pgfscope}%
\pgfpathrectangle{\pgfqpoint{0.800000in}{0.528000in}}{\pgfqpoint{4.960000in}{3.696000in}}%
\pgfusepath{clip}%
\pgfsetbuttcap%
\pgfsetroundjoin%
\definecolor{currentfill}{rgb}{0.000000,0.000000,0.000000}%
\pgfsetfillcolor{currentfill}%
\pgfsetlinewidth{1.003750pt}%
\definecolor{currentstroke}{rgb}{0.000000,0.000000,0.000000}%
\pgfsetstrokecolor{currentstroke}%
\pgfsetdash{}{0pt}%
\pgfpathmoveto{\pgfqpoint{2.518786in}{1.771040in}}%
\pgfpathcurveto{\pgfqpoint{2.529836in}{1.771040in}}{\pgfqpoint{2.540435in}{1.775431in}}{\pgfqpoint{2.548249in}{1.783244in}}%
\pgfpathcurveto{\pgfqpoint{2.556062in}{1.791058in}}{\pgfqpoint{2.560452in}{1.801657in}}{\pgfqpoint{2.560452in}{1.812707in}}%
\pgfpathcurveto{\pgfqpoint{2.560452in}{1.823757in}}{\pgfqpoint{2.556062in}{1.834356in}}{\pgfqpoint{2.548249in}{1.842170in}}%
\pgfpathcurveto{\pgfqpoint{2.540435in}{1.849983in}}{\pgfqpoint{2.529836in}{1.854374in}}{\pgfqpoint{2.518786in}{1.854374in}}%
\pgfpathcurveto{\pgfqpoint{2.507736in}{1.854374in}}{\pgfqpoint{2.497137in}{1.849983in}}{\pgfqpoint{2.489323in}{1.842170in}}%
\pgfpathcurveto{\pgfqpoint{2.481509in}{1.834356in}}{\pgfqpoint{2.477119in}{1.823757in}}{\pgfqpoint{2.477119in}{1.812707in}}%
\pgfpathcurveto{\pgfqpoint{2.477119in}{1.801657in}}{\pgfqpoint{2.481509in}{1.791058in}}{\pgfqpoint{2.489323in}{1.783244in}}%
\pgfpathcurveto{\pgfqpoint{2.497137in}{1.775431in}}{\pgfqpoint{2.507736in}{1.771040in}}{\pgfqpoint{2.518786in}{1.771040in}}%
\pgfpathclose%
\pgfusepath{stroke,fill}%
\end{pgfscope}%
\begin{pgfscope}%
\pgfpathrectangle{\pgfqpoint{0.800000in}{0.528000in}}{\pgfqpoint{4.960000in}{3.696000in}}%
\pgfusepath{clip}%
\pgfsetbuttcap%
\pgfsetroundjoin%
\definecolor{currentfill}{rgb}{0.000000,0.000000,0.000000}%
\pgfsetfillcolor{currentfill}%
\pgfsetlinewidth{1.003750pt}%
\definecolor{currentstroke}{rgb}{0.000000,0.000000,0.000000}%
\pgfsetstrokecolor{currentstroke}%
\pgfsetdash{}{0pt}%
\pgfpathmoveto{\pgfqpoint{2.518786in}{1.771040in}}%
\pgfpathcurveto{\pgfqpoint{2.529836in}{1.771040in}}{\pgfqpoint{2.540435in}{1.775431in}}{\pgfqpoint{2.548249in}{1.783244in}}%
\pgfpathcurveto{\pgfqpoint{2.556062in}{1.791058in}}{\pgfqpoint{2.560452in}{1.801657in}}{\pgfqpoint{2.560452in}{1.812707in}}%
\pgfpathcurveto{\pgfqpoint{2.560452in}{1.823757in}}{\pgfqpoint{2.556062in}{1.834356in}}{\pgfqpoint{2.548249in}{1.842170in}}%
\pgfpathcurveto{\pgfqpoint{2.540435in}{1.849983in}}{\pgfqpoint{2.529836in}{1.854374in}}{\pgfqpoint{2.518786in}{1.854374in}}%
\pgfpathcurveto{\pgfqpoint{2.507736in}{1.854374in}}{\pgfqpoint{2.497137in}{1.849983in}}{\pgfqpoint{2.489323in}{1.842170in}}%
\pgfpathcurveto{\pgfqpoint{2.481509in}{1.834356in}}{\pgfqpoint{2.477119in}{1.823757in}}{\pgfqpoint{2.477119in}{1.812707in}}%
\pgfpathcurveto{\pgfqpoint{2.477119in}{1.801657in}}{\pgfqpoint{2.481509in}{1.791058in}}{\pgfqpoint{2.489323in}{1.783244in}}%
\pgfpathcurveto{\pgfqpoint{2.497137in}{1.775431in}}{\pgfqpoint{2.507736in}{1.771040in}}{\pgfqpoint{2.518786in}{1.771040in}}%
\pgfpathclose%
\pgfusepath{stroke,fill}%
\end{pgfscope}%
\begin{pgfscope}%
\pgfpathrectangle{\pgfqpoint{0.800000in}{0.528000in}}{\pgfqpoint{4.960000in}{3.696000in}}%
\pgfusepath{clip}%
\pgfsetbuttcap%
\pgfsetroundjoin%
\definecolor{currentfill}{rgb}{0.000000,0.000000,0.000000}%
\pgfsetfillcolor{currentfill}%
\pgfsetlinewidth{1.003750pt}%
\definecolor{currentstroke}{rgb}{0.000000,0.000000,0.000000}%
\pgfsetstrokecolor{currentstroke}%
\pgfsetdash{}{0pt}%
\pgfpathmoveto{\pgfqpoint{2.518786in}{0.664394in}}%
\pgfpathcurveto{\pgfqpoint{2.529836in}{0.664394in}}{\pgfqpoint{2.540435in}{0.668784in}}{\pgfqpoint{2.548249in}{0.676598in}}%
\pgfpathcurveto{\pgfqpoint{2.556062in}{0.684411in}}{\pgfqpoint{2.560452in}{0.695010in}}{\pgfqpoint{2.560452in}{0.706060in}}%
\pgfpathcurveto{\pgfqpoint{2.560452in}{0.717111in}}{\pgfqpoint{2.556062in}{0.727710in}}{\pgfqpoint{2.548249in}{0.735523in}}%
\pgfpathcurveto{\pgfqpoint{2.540435in}{0.743337in}}{\pgfqpoint{2.529836in}{0.747727in}}{\pgfqpoint{2.518786in}{0.747727in}}%
\pgfpathcurveto{\pgfqpoint{2.507736in}{0.747727in}}{\pgfqpoint{2.497137in}{0.743337in}}{\pgfqpoint{2.489323in}{0.735523in}}%
\pgfpathcurveto{\pgfqpoint{2.481509in}{0.727710in}}{\pgfqpoint{2.477119in}{0.717111in}}{\pgfqpoint{2.477119in}{0.706060in}}%
\pgfpathcurveto{\pgfqpoint{2.477119in}{0.695010in}}{\pgfqpoint{2.481509in}{0.684411in}}{\pgfqpoint{2.489323in}{0.676598in}}%
\pgfpathcurveto{\pgfqpoint{2.497137in}{0.668784in}}{\pgfqpoint{2.507736in}{0.664394in}}{\pgfqpoint{2.518786in}{0.664394in}}%
\pgfpathclose%
\pgfusepath{stroke,fill}%
\end{pgfscope}%
\begin{pgfscope}%
\pgfpathrectangle{\pgfqpoint{0.800000in}{0.528000in}}{\pgfqpoint{4.960000in}{3.696000in}}%
\pgfusepath{clip}%
\pgfsetbuttcap%
\pgfsetroundjoin%
\definecolor{currentfill}{rgb}{0.000000,0.000000,0.000000}%
\pgfsetfillcolor{currentfill}%
\pgfsetlinewidth{1.003750pt}%
\definecolor{currentstroke}{rgb}{0.000000,0.000000,0.000000}%
\pgfsetstrokecolor{currentstroke}%
\pgfsetdash{}{0pt}%
\pgfpathmoveto{\pgfqpoint{2.518786in}{1.771040in}}%
\pgfpathcurveto{\pgfqpoint{2.529836in}{1.771040in}}{\pgfqpoint{2.540435in}{1.775431in}}{\pgfqpoint{2.548249in}{1.783244in}}%
\pgfpathcurveto{\pgfqpoint{2.556062in}{1.791058in}}{\pgfqpoint{2.560452in}{1.801657in}}{\pgfqpoint{2.560452in}{1.812707in}}%
\pgfpathcurveto{\pgfqpoint{2.560452in}{1.823757in}}{\pgfqpoint{2.556062in}{1.834356in}}{\pgfqpoint{2.548249in}{1.842170in}}%
\pgfpathcurveto{\pgfqpoint{2.540435in}{1.849983in}}{\pgfqpoint{2.529836in}{1.854374in}}{\pgfqpoint{2.518786in}{1.854374in}}%
\pgfpathcurveto{\pgfqpoint{2.507736in}{1.854374in}}{\pgfqpoint{2.497137in}{1.849983in}}{\pgfqpoint{2.489323in}{1.842170in}}%
\pgfpathcurveto{\pgfqpoint{2.481509in}{1.834356in}}{\pgfqpoint{2.477119in}{1.823757in}}{\pgfqpoint{2.477119in}{1.812707in}}%
\pgfpathcurveto{\pgfqpoint{2.477119in}{1.801657in}}{\pgfqpoint{2.481509in}{1.791058in}}{\pgfqpoint{2.489323in}{1.783244in}}%
\pgfpathcurveto{\pgfqpoint{2.497137in}{1.775431in}}{\pgfqpoint{2.507736in}{1.771040in}}{\pgfqpoint{2.518786in}{1.771040in}}%
\pgfpathclose%
\pgfusepath{stroke,fill}%
\end{pgfscope}%
\begin{pgfscope}%
\pgfpathrectangle{\pgfqpoint{0.800000in}{0.528000in}}{\pgfqpoint{4.960000in}{3.696000in}}%
\pgfusepath{clip}%
\pgfsetbuttcap%
\pgfsetroundjoin%
\definecolor{currentfill}{rgb}{0.000000,0.000000,0.000000}%
\pgfsetfillcolor{currentfill}%
\pgfsetlinewidth{1.003750pt}%
\definecolor{currentstroke}{rgb}{0.000000,0.000000,0.000000}%
\pgfsetstrokecolor{currentstroke}%
\pgfsetdash{}{0pt}%
\pgfpathmoveto{\pgfqpoint{2.518786in}{1.771040in}}%
\pgfpathcurveto{\pgfqpoint{2.529836in}{1.771040in}}{\pgfqpoint{2.540435in}{1.775431in}}{\pgfqpoint{2.548249in}{1.783244in}}%
\pgfpathcurveto{\pgfqpoint{2.556062in}{1.791058in}}{\pgfqpoint{2.560452in}{1.801657in}}{\pgfqpoint{2.560452in}{1.812707in}}%
\pgfpathcurveto{\pgfqpoint{2.560452in}{1.823757in}}{\pgfqpoint{2.556062in}{1.834356in}}{\pgfqpoint{2.548249in}{1.842170in}}%
\pgfpathcurveto{\pgfqpoint{2.540435in}{1.849983in}}{\pgfqpoint{2.529836in}{1.854374in}}{\pgfqpoint{2.518786in}{1.854374in}}%
\pgfpathcurveto{\pgfqpoint{2.507736in}{1.854374in}}{\pgfqpoint{2.497137in}{1.849983in}}{\pgfqpoint{2.489323in}{1.842170in}}%
\pgfpathcurveto{\pgfqpoint{2.481509in}{1.834356in}}{\pgfqpoint{2.477119in}{1.823757in}}{\pgfqpoint{2.477119in}{1.812707in}}%
\pgfpathcurveto{\pgfqpoint{2.477119in}{1.801657in}}{\pgfqpoint{2.481509in}{1.791058in}}{\pgfqpoint{2.489323in}{1.783244in}}%
\pgfpathcurveto{\pgfqpoint{2.497137in}{1.775431in}}{\pgfqpoint{2.507736in}{1.771040in}}{\pgfqpoint{2.518786in}{1.771040in}}%
\pgfpathclose%
\pgfusepath{stroke,fill}%
\end{pgfscope}%
\begin{pgfscope}%
\pgfpathrectangle{\pgfqpoint{0.800000in}{0.528000in}}{\pgfqpoint{4.960000in}{3.696000in}}%
\pgfusepath{clip}%
\pgfsetbuttcap%
\pgfsetroundjoin%
\definecolor{currentfill}{rgb}{0.000000,0.000000,0.000000}%
\pgfsetfillcolor{currentfill}%
\pgfsetlinewidth{1.003750pt}%
\definecolor{currentstroke}{rgb}{0.000000,0.000000,0.000000}%
\pgfsetstrokecolor{currentstroke}%
\pgfsetdash{}{0pt}%
\pgfpathmoveto{\pgfqpoint{2.518786in}{1.771040in}}%
\pgfpathcurveto{\pgfqpoint{2.529836in}{1.771040in}}{\pgfqpoint{2.540435in}{1.775431in}}{\pgfqpoint{2.548249in}{1.783244in}}%
\pgfpathcurveto{\pgfqpoint{2.556062in}{1.791058in}}{\pgfqpoint{2.560452in}{1.801657in}}{\pgfqpoint{2.560452in}{1.812707in}}%
\pgfpathcurveto{\pgfqpoint{2.560452in}{1.823757in}}{\pgfqpoint{2.556062in}{1.834356in}}{\pgfqpoint{2.548249in}{1.842170in}}%
\pgfpathcurveto{\pgfqpoint{2.540435in}{1.849983in}}{\pgfqpoint{2.529836in}{1.854374in}}{\pgfqpoint{2.518786in}{1.854374in}}%
\pgfpathcurveto{\pgfqpoint{2.507736in}{1.854374in}}{\pgfqpoint{2.497137in}{1.849983in}}{\pgfqpoint{2.489323in}{1.842170in}}%
\pgfpathcurveto{\pgfqpoint{2.481509in}{1.834356in}}{\pgfqpoint{2.477119in}{1.823757in}}{\pgfqpoint{2.477119in}{1.812707in}}%
\pgfpathcurveto{\pgfqpoint{2.477119in}{1.801657in}}{\pgfqpoint{2.481509in}{1.791058in}}{\pgfqpoint{2.489323in}{1.783244in}}%
\pgfpathcurveto{\pgfqpoint{2.497137in}{1.775431in}}{\pgfqpoint{2.507736in}{1.771040in}}{\pgfqpoint{2.518786in}{1.771040in}}%
\pgfpathclose%
\pgfusepath{stroke,fill}%
\end{pgfscope}%
\begin{pgfscope}%
\pgfpathrectangle{\pgfqpoint{0.800000in}{0.528000in}}{\pgfqpoint{4.960000in}{3.696000in}}%
\pgfusepath{clip}%
\pgfsetbuttcap%
\pgfsetroundjoin%
\definecolor{currentfill}{rgb}{0.000000,0.000000,0.000000}%
\pgfsetfillcolor{currentfill}%
\pgfsetlinewidth{1.003750pt}%
\definecolor{currentstroke}{rgb}{0.000000,0.000000,0.000000}%
\pgfsetstrokecolor{currentstroke}%
\pgfsetdash{}{0pt}%
\pgfpathmoveto{\pgfqpoint{2.518786in}{1.771040in}}%
\pgfpathcurveto{\pgfqpoint{2.529836in}{1.771040in}}{\pgfqpoint{2.540435in}{1.775431in}}{\pgfqpoint{2.548249in}{1.783244in}}%
\pgfpathcurveto{\pgfqpoint{2.556062in}{1.791058in}}{\pgfqpoint{2.560452in}{1.801657in}}{\pgfqpoint{2.560452in}{1.812707in}}%
\pgfpathcurveto{\pgfqpoint{2.560452in}{1.823757in}}{\pgfqpoint{2.556062in}{1.834356in}}{\pgfqpoint{2.548249in}{1.842170in}}%
\pgfpathcurveto{\pgfqpoint{2.540435in}{1.849983in}}{\pgfqpoint{2.529836in}{1.854374in}}{\pgfqpoint{2.518786in}{1.854374in}}%
\pgfpathcurveto{\pgfqpoint{2.507736in}{1.854374in}}{\pgfqpoint{2.497137in}{1.849983in}}{\pgfqpoint{2.489323in}{1.842170in}}%
\pgfpathcurveto{\pgfqpoint{2.481509in}{1.834356in}}{\pgfqpoint{2.477119in}{1.823757in}}{\pgfqpoint{2.477119in}{1.812707in}}%
\pgfpathcurveto{\pgfqpoint{2.477119in}{1.801657in}}{\pgfqpoint{2.481509in}{1.791058in}}{\pgfqpoint{2.489323in}{1.783244in}}%
\pgfpathcurveto{\pgfqpoint{2.497137in}{1.775431in}}{\pgfqpoint{2.507736in}{1.771040in}}{\pgfqpoint{2.518786in}{1.771040in}}%
\pgfpathclose%
\pgfusepath{stroke,fill}%
\end{pgfscope}%
\begin{pgfscope}%
\pgfpathrectangle{\pgfqpoint{0.800000in}{0.528000in}}{\pgfqpoint{4.960000in}{3.696000in}}%
\pgfusepath{clip}%
\pgfsetbuttcap%
\pgfsetroundjoin%
\definecolor{currentfill}{rgb}{0.000000,0.000000,0.000000}%
\pgfsetfillcolor{currentfill}%
\pgfsetlinewidth{1.003750pt}%
\definecolor{currentstroke}{rgb}{0.000000,0.000000,0.000000}%
\pgfsetstrokecolor{currentstroke}%
\pgfsetdash{}{0pt}%
\pgfpathmoveto{\pgfqpoint{2.518786in}{1.771040in}}%
\pgfpathcurveto{\pgfqpoint{2.529836in}{1.771040in}}{\pgfqpoint{2.540435in}{1.775431in}}{\pgfqpoint{2.548249in}{1.783244in}}%
\pgfpathcurveto{\pgfqpoint{2.556062in}{1.791058in}}{\pgfqpoint{2.560452in}{1.801657in}}{\pgfqpoint{2.560452in}{1.812707in}}%
\pgfpathcurveto{\pgfqpoint{2.560452in}{1.823757in}}{\pgfqpoint{2.556062in}{1.834356in}}{\pgfqpoint{2.548249in}{1.842170in}}%
\pgfpathcurveto{\pgfqpoint{2.540435in}{1.849983in}}{\pgfqpoint{2.529836in}{1.854374in}}{\pgfqpoint{2.518786in}{1.854374in}}%
\pgfpathcurveto{\pgfqpoint{2.507736in}{1.854374in}}{\pgfqpoint{2.497137in}{1.849983in}}{\pgfqpoint{2.489323in}{1.842170in}}%
\pgfpathcurveto{\pgfqpoint{2.481509in}{1.834356in}}{\pgfqpoint{2.477119in}{1.823757in}}{\pgfqpoint{2.477119in}{1.812707in}}%
\pgfpathcurveto{\pgfqpoint{2.477119in}{1.801657in}}{\pgfqpoint{2.481509in}{1.791058in}}{\pgfqpoint{2.489323in}{1.783244in}}%
\pgfpathcurveto{\pgfqpoint{2.497137in}{1.775431in}}{\pgfqpoint{2.507736in}{1.771040in}}{\pgfqpoint{2.518786in}{1.771040in}}%
\pgfpathclose%
\pgfusepath{stroke,fill}%
\end{pgfscope}%
\begin{pgfscope}%
\pgfpathrectangle{\pgfqpoint{0.800000in}{0.528000in}}{\pgfqpoint{4.960000in}{3.696000in}}%
\pgfusepath{clip}%
\pgfsetbuttcap%
\pgfsetroundjoin%
\definecolor{currentfill}{rgb}{0.000000,0.000000,0.000000}%
\pgfsetfillcolor{currentfill}%
\pgfsetlinewidth{1.003750pt}%
\definecolor{currentstroke}{rgb}{0.000000,0.000000,0.000000}%
\pgfsetstrokecolor{currentstroke}%
\pgfsetdash{}{0pt}%
\pgfpathmoveto{\pgfqpoint{2.518786in}{1.771040in}}%
\pgfpathcurveto{\pgfqpoint{2.529836in}{1.771040in}}{\pgfqpoint{2.540435in}{1.775431in}}{\pgfqpoint{2.548249in}{1.783244in}}%
\pgfpathcurveto{\pgfqpoint{2.556062in}{1.791058in}}{\pgfqpoint{2.560452in}{1.801657in}}{\pgfqpoint{2.560452in}{1.812707in}}%
\pgfpathcurveto{\pgfqpoint{2.560452in}{1.823757in}}{\pgfqpoint{2.556062in}{1.834356in}}{\pgfqpoint{2.548249in}{1.842170in}}%
\pgfpathcurveto{\pgfqpoint{2.540435in}{1.849983in}}{\pgfqpoint{2.529836in}{1.854374in}}{\pgfqpoint{2.518786in}{1.854374in}}%
\pgfpathcurveto{\pgfqpoint{2.507736in}{1.854374in}}{\pgfqpoint{2.497137in}{1.849983in}}{\pgfqpoint{2.489323in}{1.842170in}}%
\pgfpathcurveto{\pgfqpoint{2.481509in}{1.834356in}}{\pgfqpoint{2.477119in}{1.823757in}}{\pgfqpoint{2.477119in}{1.812707in}}%
\pgfpathcurveto{\pgfqpoint{2.477119in}{1.801657in}}{\pgfqpoint{2.481509in}{1.791058in}}{\pgfqpoint{2.489323in}{1.783244in}}%
\pgfpathcurveto{\pgfqpoint{2.497137in}{1.775431in}}{\pgfqpoint{2.507736in}{1.771040in}}{\pgfqpoint{2.518786in}{1.771040in}}%
\pgfpathclose%
\pgfusepath{stroke,fill}%
\end{pgfscope}%
\begin{pgfscope}%
\pgfpathrectangle{\pgfqpoint{0.800000in}{0.528000in}}{\pgfqpoint{4.960000in}{3.696000in}}%
\pgfusepath{clip}%
\pgfsetbuttcap%
\pgfsetroundjoin%
\definecolor{currentfill}{rgb}{0.000000,0.000000,0.000000}%
\pgfsetfillcolor{currentfill}%
\pgfsetlinewidth{1.003750pt}%
\definecolor{currentstroke}{rgb}{0.000000,0.000000,0.000000}%
\pgfsetstrokecolor{currentstroke}%
\pgfsetdash{}{0pt}%
\pgfpathmoveto{\pgfqpoint{2.518786in}{1.771040in}}%
\pgfpathcurveto{\pgfqpoint{2.529836in}{1.771040in}}{\pgfqpoint{2.540435in}{1.775431in}}{\pgfqpoint{2.548249in}{1.783244in}}%
\pgfpathcurveto{\pgfqpoint{2.556062in}{1.791058in}}{\pgfqpoint{2.560452in}{1.801657in}}{\pgfqpoint{2.560452in}{1.812707in}}%
\pgfpathcurveto{\pgfqpoint{2.560452in}{1.823757in}}{\pgfqpoint{2.556062in}{1.834356in}}{\pgfqpoint{2.548249in}{1.842170in}}%
\pgfpathcurveto{\pgfqpoint{2.540435in}{1.849983in}}{\pgfqpoint{2.529836in}{1.854374in}}{\pgfqpoint{2.518786in}{1.854374in}}%
\pgfpathcurveto{\pgfqpoint{2.507736in}{1.854374in}}{\pgfqpoint{2.497137in}{1.849983in}}{\pgfqpoint{2.489323in}{1.842170in}}%
\pgfpathcurveto{\pgfqpoint{2.481509in}{1.834356in}}{\pgfqpoint{2.477119in}{1.823757in}}{\pgfqpoint{2.477119in}{1.812707in}}%
\pgfpathcurveto{\pgfqpoint{2.477119in}{1.801657in}}{\pgfqpoint{2.481509in}{1.791058in}}{\pgfqpoint{2.489323in}{1.783244in}}%
\pgfpathcurveto{\pgfqpoint{2.497137in}{1.775431in}}{\pgfqpoint{2.507736in}{1.771040in}}{\pgfqpoint{2.518786in}{1.771040in}}%
\pgfpathclose%
\pgfusepath{stroke,fill}%
\end{pgfscope}%
\begin{pgfscope}%
\pgfpathrectangle{\pgfqpoint{0.800000in}{0.528000in}}{\pgfqpoint{4.960000in}{3.696000in}}%
\pgfusepath{clip}%
\pgfsetbuttcap%
\pgfsetroundjoin%
\definecolor{currentfill}{rgb}{0.000000,0.000000,0.000000}%
\pgfsetfillcolor{currentfill}%
\pgfsetlinewidth{1.003750pt}%
\definecolor{currentstroke}{rgb}{0.000000,0.000000,0.000000}%
\pgfsetstrokecolor{currentstroke}%
\pgfsetdash{}{0pt}%
\pgfpathmoveto{\pgfqpoint{2.518786in}{1.771040in}}%
\pgfpathcurveto{\pgfqpoint{2.529836in}{1.771040in}}{\pgfqpoint{2.540435in}{1.775431in}}{\pgfqpoint{2.548249in}{1.783244in}}%
\pgfpathcurveto{\pgfqpoint{2.556062in}{1.791058in}}{\pgfqpoint{2.560452in}{1.801657in}}{\pgfqpoint{2.560452in}{1.812707in}}%
\pgfpathcurveto{\pgfqpoint{2.560452in}{1.823757in}}{\pgfqpoint{2.556062in}{1.834356in}}{\pgfqpoint{2.548249in}{1.842170in}}%
\pgfpathcurveto{\pgfqpoint{2.540435in}{1.849983in}}{\pgfqpoint{2.529836in}{1.854374in}}{\pgfqpoint{2.518786in}{1.854374in}}%
\pgfpathcurveto{\pgfqpoint{2.507736in}{1.854374in}}{\pgfqpoint{2.497137in}{1.849983in}}{\pgfqpoint{2.489323in}{1.842170in}}%
\pgfpathcurveto{\pgfqpoint{2.481509in}{1.834356in}}{\pgfqpoint{2.477119in}{1.823757in}}{\pgfqpoint{2.477119in}{1.812707in}}%
\pgfpathcurveto{\pgfqpoint{2.477119in}{1.801657in}}{\pgfqpoint{2.481509in}{1.791058in}}{\pgfqpoint{2.489323in}{1.783244in}}%
\pgfpathcurveto{\pgfqpoint{2.497137in}{1.775431in}}{\pgfqpoint{2.507736in}{1.771040in}}{\pgfqpoint{2.518786in}{1.771040in}}%
\pgfpathclose%
\pgfusepath{stroke,fill}%
\end{pgfscope}%
\begin{pgfscope}%
\pgfpathrectangle{\pgfqpoint{0.800000in}{0.528000in}}{\pgfqpoint{4.960000in}{3.696000in}}%
\pgfusepath{clip}%
\pgfsetbuttcap%
\pgfsetroundjoin%
\definecolor{currentfill}{rgb}{0.000000,0.000000,0.000000}%
\pgfsetfillcolor{currentfill}%
\pgfsetlinewidth{1.003750pt}%
\definecolor{currentstroke}{rgb}{0.000000,0.000000,0.000000}%
\pgfsetstrokecolor{currentstroke}%
\pgfsetdash{}{0pt}%
\pgfpathmoveto{\pgfqpoint{2.518786in}{1.771040in}}%
\pgfpathcurveto{\pgfqpoint{2.529836in}{1.771040in}}{\pgfqpoint{2.540435in}{1.775431in}}{\pgfqpoint{2.548249in}{1.783244in}}%
\pgfpathcurveto{\pgfqpoint{2.556062in}{1.791058in}}{\pgfqpoint{2.560452in}{1.801657in}}{\pgfqpoint{2.560452in}{1.812707in}}%
\pgfpathcurveto{\pgfqpoint{2.560452in}{1.823757in}}{\pgfqpoint{2.556062in}{1.834356in}}{\pgfqpoint{2.548249in}{1.842170in}}%
\pgfpathcurveto{\pgfqpoint{2.540435in}{1.849983in}}{\pgfqpoint{2.529836in}{1.854374in}}{\pgfqpoint{2.518786in}{1.854374in}}%
\pgfpathcurveto{\pgfqpoint{2.507736in}{1.854374in}}{\pgfqpoint{2.497137in}{1.849983in}}{\pgfqpoint{2.489323in}{1.842170in}}%
\pgfpathcurveto{\pgfqpoint{2.481509in}{1.834356in}}{\pgfqpoint{2.477119in}{1.823757in}}{\pgfqpoint{2.477119in}{1.812707in}}%
\pgfpathcurveto{\pgfqpoint{2.477119in}{1.801657in}}{\pgfqpoint{2.481509in}{1.791058in}}{\pgfqpoint{2.489323in}{1.783244in}}%
\pgfpathcurveto{\pgfqpoint{2.497137in}{1.775431in}}{\pgfqpoint{2.507736in}{1.771040in}}{\pgfqpoint{2.518786in}{1.771040in}}%
\pgfpathclose%
\pgfusepath{stroke,fill}%
\end{pgfscope}%
\begin{pgfscope}%
\pgfpathrectangle{\pgfqpoint{0.800000in}{0.528000in}}{\pgfqpoint{4.960000in}{3.696000in}}%
\pgfusepath{clip}%
\pgfsetbuttcap%
\pgfsetroundjoin%
\definecolor{currentfill}{rgb}{0.000000,0.000000,0.000000}%
\pgfsetfillcolor{currentfill}%
\pgfsetlinewidth{1.003750pt}%
\definecolor{currentstroke}{rgb}{0.000000,0.000000,0.000000}%
\pgfsetstrokecolor{currentstroke}%
\pgfsetdash{}{0pt}%
\pgfpathmoveto{\pgfqpoint{2.518786in}{1.771040in}}%
\pgfpathcurveto{\pgfqpoint{2.529836in}{1.771040in}}{\pgfqpoint{2.540435in}{1.775431in}}{\pgfqpoint{2.548249in}{1.783244in}}%
\pgfpathcurveto{\pgfqpoint{2.556062in}{1.791058in}}{\pgfqpoint{2.560452in}{1.801657in}}{\pgfqpoint{2.560452in}{1.812707in}}%
\pgfpathcurveto{\pgfqpoint{2.560452in}{1.823757in}}{\pgfqpoint{2.556062in}{1.834356in}}{\pgfqpoint{2.548249in}{1.842170in}}%
\pgfpathcurveto{\pgfqpoint{2.540435in}{1.849983in}}{\pgfqpoint{2.529836in}{1.854374in}}{\pgfqpoint{2.518786in}{1.854374in}}%
\pgfpathcurveto{\pgfqpoint{2.507736in}{1.854374in}}{\pgfqpoint{2.497137in}{1.849983in}}{\pgfqpoint{2.489323in}{1.842170in}}%
\pgfpathcurveto{\pgfqpoint{2.481509in}{1.834356in}}{\pgfqpoint{2.477119in}{1.823757in}}{\pgfqpoint{2.477119in}{1.812707in}}%
\pgfpathcurveto{\pgfqpoint{2.477119in}{1.801657in}}{\pgfqpoint{2.481509in}{1.791058in}}{\pgfqpoint{2.489323in}{1.783244in}}%
\pgfpathcurveto{\pgfqpoint{2.497137in}{1.775431in}}{\pgfqpoint{2.507736in}{1.771040in}}{\pgfqpoint{2.518786in}{1.771040in}}%
\pgfpathclose%
\pgfusepath{stroke,fill}%
\end{pgfscope}%
\begin{pgfscope}%
\pgfpathrectangle{\pgfqpoint{0.800000in}{0.528000in}}{\pgfqpoint{4.960000in}{3.696000in}}%
\pgfusepath{clip}%
\pgfsetbuttcap%
\pgfsetroundjoin%
\definecolor{currentfill}{rgb}{0.000000,0.000000,0.000000}%
\pgfsetfillcolor{currentfill}%
\pgfsetlinewidth{1.003750pt}%
\definecolor{currentstroke}{rgb}{0.000000,0.000000,0.000000}%
\pgfsetstrokecolor{currentstroke}%
\pgfsetdash{}{0pt}%
\pgfpathmoveto{\pgfqpoint{2.518786in}{1.771040in}}%
\pgfpathcurveto{\pgfqpoint{2.529836in}{1.771040in}}{\pgfqpoint{2.540435in}{1.775431in}}{\pgfqpoint{2.548249in}{1.783244in}}%
\pgfpathcurveto{\pgfqpoint{2.556062in}{1.791058in}}{\pgfqpoint{2.560452in}{1.801657in}}{\pgfqpoint{2.560452in}{1.812707in}}%
\pgfpathcurveto{\pgfqpoint{2.560452in}{1.823757in}}{\pgfqpoint{2.556062in}{1.834356in}}{\pgfqpoint{2.548249in}{1.842170in}}%
\pgfpathcurveto{\pgfqpoint{2.540435in}{1.849983in}}{\pgfqpoint{2.529836in}{1.854374in}}{\pgfqpoint{2.518786in}{1.854374in}}%
\pgfpathcurveto{\pgfqpoint{2.507736in}{1.854374in}}{\pgfqpoint{2.497137in}{1.849983in}}{\pgfqpoint{2.489323in}{1.842170in}}%
\pgfpathcurveto{\pgfqpoint{2.481509in}{1.834356in}}{\pgfqpoint{2.477119in}{1.823757in}}{\pgfqpoint{2.477119in}{1.812707in}}%
\pgfpathcurveto{\pgfqpoint{2.477119in}{1.801657in}}{\pgfqpoint{2.481509in}{1.791058in}}{\pgfqpoint{2.489323in}{1.783244in}}%
\pgfpathcurveto{\pgfqpoint{2.497137in}{1.775431in}}{\pgfqpoint{2.507736in}{1.771040in}}{\pgfqpoint{2.518786in}{1.771040in}}%
\pgfpathclose%
\pgfusepath{stroke,fill}%
\end{pgfscope}%
\begin{pgfscope}%
\pgfpathrectangle{\pgfqpoint{0.800000in}{0.528000in}}{\pgfqpoint{4.960000in}{3.696000in}}%
\pgfusepath{clip}%
\pgfsetbuttcap%
\pgfsetroundjoin%
\definecolor{currentfill}{rgb}{0.000000,0.000000,0.000000}%
\pgfsetfillcolor{currentfill}%
\pgfsetlinewidth{1.003750pt}%
\definecolor{currentstroke}{rgb}{0.000000,0.000000,0.000000}%
\pgfsetstrokecolor{currentstroke}%
\pgfsetdash{}{0pt}%
\pgfpathmoveto{\pgfqpoint{2.518786in}{1.771040in}}%
\pgfpathcurveto{\pgfqpoint{2.529836in}{1.771040in}}{\pgfqpoint{2.540435in}{1.775431in}}{\pgfqpoint{2.548249in}{1.783244in}}%
\pgfpathcurveto{\pgfqpoint{2.556062in}{1.791058in}}{\pgfqpoint{2.560452in}{1.801657in}}{\pgfqpoint{2.560452in}{1.812707in}}%
\pgfpathcurveto{\pgfqpoint{2.560452in}{1.823757in}}{\pgfqpoint{2.556062in}{1.834356in}}{\pgfqpoint{2.548249in}{1.842170in}}%
\pgfpathcurveto{\pgfqpoint{2.540435in}{1.849983in}}{\pgfqpoint{2.529836in}{1.854374in}}{\pgfqpoint{2.518786in}{1.854374in}}%
\pgfpathcurveto{\pgfqpoint{2.507736in}{1.854374in}}{\pgfqpoint{2.497137in}{1.849983in}}{\pgfqpoint{2.489323in}{1.842170in}}%
\pgfpathcurveto{\pgfqpoint{2.481509in}{1.834356in}}{\pgfqpoint{2.477119in}{1.823757in}}{\pgfqpoint{2.477119in}{1.812707in}}%
\pgfpathcurveto{\pgfqpoint{2.477119in}{1.801657in}}{\pgfqpoint{2.481509in}{1.791058in}}{\pgfqpoint{2.489323in}{1.783244in}}%
\pgfpathcurveto{\pgfqpoint{2.497137in}{1.775431in}}{\pgfqpoint{2.507736in}{1.771040in}}{\pgfqpoint{2.518786in}{1.771040in}}%
\pgfpathclose%
\pgfusepath{stroke,fill}%
\end{pgfscope}%
\begin{pgfscope}%
\pgfpathrectangle{\pgfqpoint{0.800000in}{0.528000in}}{\pgfqpoint{4.960000in}{3.696000in}}%
\pgfusepath{clip}%
\pgfsetbuttcap%
\pgfsetroundjoin%
\definecolor{currentfill}{rgb}{0.000000,0.000000,0.000000}%
\pgfsetfillcolor{currentfill}%
\pgfsetlinewidth{1.003750pt}%
\definecolor{currentstroke}{rgb}{0.000000,0.000000,0.000000}%
\pgfsetstrokecolor{currentstroke}%
\pgfsetdash{}{0pt}%
\pgfpathmoveto{\pgfqpoint{2.518786in}{1.771040in}}%
\pgfpathcurveto{\pgfqpoint{2.529836in}{1.771040in}}{\pgfqpoint{2.540435in}{1.775431in}}{\pgfqpoint{2.548249in}{1.783244in}}%
\pgfpathcurveto{\pgfqpoint{2.556062in}{1.791058in}}{\pgfqpoint{2.560452in}{1.801657in}}{\pgfqpoint{2.560452in}{1.812707in}}%
\pgfpathcurveto{\pgfqpoint{2.560452in}{1.823757in}}{\pgfqpoint{2.556062in}{1.834356in}}{\pgfqpoint{2.548249in}{1.842170in}}%
\pgfpathcurveto{\pgfqpoint{2.540435in}{1.849983in}}{\pgfqpoint{2.529836in}{1.854374in}}{\pgfqpoint{2.518786in}{1.854374in}}%
\pgfpathcurveto{\pgfqpoint{2.507736in}{1.854374in}}{\pgfqpoint{2.497137in}{1.849983in}}{\pgfqpoint{2.489323in}{1.842170in}}%
\pgfpathcurveto{\pgfqpoint{2.481509in}{1.834356in}}{\pgfqpoint{2.477119in}{1.823757in}}{\pgfqpoint{2.477119in}{1.812707in}}%
\pgfpathcurveto{\pgfqpoint{2.477119in}{1.801657in}}{\pgfqpoint{2.481509in}{1.791058in}}{\pgfqpoint{2.489323in}{1.783244in}}%
\pgfpathcurveto{\pgfqpoint{2.497137in}{1.775431in}}{\pgfqpoint{2.507736in}{1.771040in}}{\pgfqpoint{2.518786in}{1.771040in}}%
\pgfpathclose%
\pgfusepath{stroke,fill}%
\end{pgfscope}%
\begin{pgfscope}%
\pgfpathrectangle{\pgfqpoint{0.800000in}{0.528000in}}{\pgfqpoint{4.960000in}{3.696000in}}%
\pgfusepath{clip}%
\pgfsetbuttcap%
\pgfsetroundjoin%
\definecolor{currentfill}{rgb}{0.000000,0.000000,0.000000}%
\pgfsetfillcolor{currentfill}%
\pgfsetlinewidth{1.003750pt}%
\definecolor{currentstroke}{rgb}{0.000000,0.000000,0.000000}%
\pgfsetstrokecolor{currentstroke}%
\pgfsetdash{}{0pt}%
\pgfpathmoveto{\pgfqpoint{2.518786in}{1.771040in}}%
\pgfpathcurveto{\pgfqpoint{2.529836in}{1.771040in}}{\pgfqpoint{2.540435in}{1.775431in}}{\pgfqpoint{2.548249in}{1.783244in}}%
\pgfpathcurveto{\pgfqpoint{2.556062in}{1.791058in}}{\pgfqpoint{2.560452in}{1.801657in}}{\pgfqpoint{2.560452in}{1.812707in}}%
\pgfpathcurveto{\pgfqpoint{2.560452in}{1.823757in}}{\pgfqpoint{2.556062in}{1.834356in}}{\pgfqpoint{2.548249in}{1.842170in}}%
\pgfpathcurveto{\pgfqpoint{2.540435in}{1.849983in}}{\pgfqpoint{2.529836in}{1.854374in}}{\pgfqpoint{2.518786in}{1.854374in}}%
\pgfpathcurveto{\pgfqpoint{2.507736in}{1.854374in}}{\pgfqpoint{2.497137in}{1.849983in}}{\pgfqpoint{2.489323in}{1.842170in}}%
\pgfpathcurveto{\pgfqpoint{2.481509in}{1.834356in}}{\pgfqpoint{2.477119in}{1.823757in}}{\pgfqpoint{2.477119in}{1.812707in}}%
\pgfpathcurveto{\pgfqpoint{2.477119in}{1.801657in}}{\pgfqpoint{2.481509in}{1.791058in}}{\pgfqpoint{2.489323in}{1.783244in}}%
\pgfpathcurveto{\pgfqpoint{2.497137in}{1.775431in}}{\pgfqpoint{2.507736in}{1.771040in}}{\pgfqpoint{2.518786in}{1.771040in}}%
\pgfpathclose%
\pgfusepath{stroke,fill}%
\end{pgfscope}%
\begin{pgfscope}%
\pgfpathrectangle{\pgfqpoint{0.800000in}{0.528000in}}{\pgfqpoint{4.960000in}{3.696000in}}%
\pgfusepath{clip}%
\pgfsetbuttcap%
\pgfsetroundjoin%
\definecolor{currentfill}{rgb}{0.000000,0.000000,0.000000}%
\pgfsetfillcolor{currentfill}%
\pgfsetlinewidth{1.003750pt}%
\definecolor{currentstroke}{rgb}{0.000000,0.000000,0.000000}%
\pgfsetstrokecolor{currentstroke}%
\pgfsetdash{}{0pt}%
\pgfpathmoveto{\pgfqpoint{2.518786in}{1.771040in}}%
\pgfpathcurveto{\pgfqpoint{2.529836in}{1.771040in}}{\pgfqpoint{2.540435in}{1.775431in}}{\pgfqpoint{2.548249in}{1.783244in}}%
\pgfpathcurveto{\pgfqpoint{2.556062in}{1.791058in}}{\pgfqpoint{2.560452in}{1.801657in}}{\pgfqpoint{2.560452in}{1.812707in}}%
\pgfpathcurveto{\pgfqpoint{2.560452in}{1.823757in}}{\pgfqpoint{2.556062in}{1.834356in}}{\pgfqpoint{2.548249in}{1.842170in}}%
\pgfpathcurveto{\pgfqpoint{2.540435in}{1.849983in}}{\pgfqpoint{2.529836in}{1.854374in}}{\pgfqpoint{2.518786in}{1.854374in}}%
\pgfpathcurveto{\pgfqpoint{2.507736in}{1.854374in}}{\pgfqpoint{2.497137in}{1.849983in}}{\pgfqpoint{2.489323in}{1.842170in}}%
\pgfpathcurveto{\pgfqpoint{2.481509in}{1.834356in}}{\pgfqpoint{2.477119in}{1.823757in}}{\pgfqpoint{2.477119in}{1.812707in}}%
\pgfpathcurveto{\pgfqpoint{2.477119in}{1.801657in}}{\pgfqpoint{2.481509in}{1.791058in}}{\pgfqpoint{2.489323in}{1.783244in}}%
\pgfpathcurveto{\pgfqpoint{2.497137in}{1.775431in}}{\pgfqpoint{2.507736in}{1.771040in}}{\pgfqpoint{2.518786in}{1.771040in}}%
\pgfpathclose%
\pgfusepath{stroke,fill}%
\end{pgfscope}%
\begin{pgfscope}%
\pgfpathrectangle{\pgfqpoint{0.800000in}{0.528000in}}{\pgfqpoint{4.960000in}{3.696000in}}%
\pgfusepath{clip}%
\pgfsetbuttcap%
\pgfsetroundjoin%
\definecolor{currentfill}{rgb}{0.000000,0.000000,0.000000}%
\pgfsetfillcolor{currentfill}%
\pgfsetlinewidth{1.003750pt}%
\definecolor{currentstroke}{rgb}{0.000000,0.000000,0.000000}%
\pgfsetstrokecolor{currentstroke}%
\pgfsetdash{}{0pt}%
\pgfpathmoveto{\pgfqpoint{2.518786in}{1.771040in}}%
\pgfpathcurveto{\pgfqpoint{2.529836in}{1.771040in}}{\pgfqpoint{2.540435in}{1.775431in}}{\pgfqpoint{2.548249in}{1.783244in}}%
\pgfpathcurveto{\pgfqpoint{2.556062in}{1.791058in}}{\pgfqpoint{2.560452in}{1.801657in}}{\pgfqpoint{2.560452in}{1.812707in}}%
\pgfpathcurveto{\pgfqpoint{2.560452in}{1.823757in}}{\pgfqpoint{2.556062in}{1.834356in}}{\pgfqpoint{2.548249in}{1.842170in}}%
\pgfpathcurveto{\pgfqpoint{2.540435in}{1.849983in}}{\pgfqpoint{2.529836in}{1.854374in}}{\pgfqpoint{2.518786in}{1.854374in}}%
\pgfpathcurveto{\pgfqpoint{2.507736in}{1.854374in}}{\pgfqpoint{2.497137in}{1.849983in}}{\pgfqpoint{2.489323in}{1.842170in}}%
\pgfpathcurveto{\pgfqpoint{2.481509in}{1.834356in}}{\pgfqpoint{2.477119in}{1.823757in}}{\pgfqpoint{2.477119in}{1.812707in}}%
\pgfpathcurveto{\pgfqpoint{2.477119in}{1.801657in}}{\pgfqpoint{2.481509in}{1.791058in}}{\pgfqpoint{2.489323in}{1.783244in}}%
\pgfpathcurveto{\pgfqpoint{2.497137in}{1.775431in}}{\pgfqpoint{2.507736in}{1.771040in}}{\pgfqpoint{2.518786in}{1.771040in}}%
\pgfpathclose%
\pgfusepath{stroke,fill}%
\end{pgfscope}%
\begin{pgfscope}%
\pgfpathrectangle{\pgfqpoint{0.800000in}{0.528000in}}{\pgfqpoint{4.960000in}{3.696000in}}%
\pgfusepath{clip}%
\pgfsetbuttcap%
\pgfsetroundjoin%
\definecolor{currentfill}{rgb}{0.000000,0.000000,0.000000}%
\pgfsetfillcolor{currentfill}%
\pgfsetlinewidth{1.003750pt}%
\definecolor{currentstroke}{rgb}{0.000000,0.000000,0.000000}%
\pgfsetstrokecolor{currentstroke}%
\pgfsetdash{}{0pt}%
\pgfpathmoveto{\pgfqpoint{2.518786in}{1.771040in}}%
\pgfpathcurveto{\pgfqpoint{2.529836in}{1.771040in}}{\pgfqpoint{2.540435in}{1.775431in}}{\pgfqpoint{2.548249in}{1.783244in}}%
\pgfpathcurveto{\pgfqpoint{2.556062in}{1.791058in}}{\pgfqpoint{2.560452in}{1.801657in}}{\pgfqpoint{2.560452in}{1.812707in}}%
\pgfpathcurveto{\pgfqpoint{2.560452in}{1.823757in}}{\pgfqpoint{2.556062in}{1.834356in}}{\pgfqpoint{2.548249in}{1.842170in}}%
\pgfpathcurveto{\pgfqpoint{2.540435in}{1.849983in}}{\pgfqpoint{2.529836in}{1.854374in}}{\pgfqpoint{2.518786in}{1.854374in}}%
\pgfpathcurveto{\pgfqpoint{2.507736in}{1.854374in}}{\pgfqpoint{2.497137in}{1.849983in}}{\pgfqpoint{2.489323in}{1.842170in}}%
\pgfpathcurveto{\pgfqpoint{2.481509in}{1.834356in}}{\pgfqpoint{2.477119in}{1.823757in}}{\pgfqpoint{2.477119in}{1.812707in}}%
\pgfpathcurveto{\pgfqpoint{2.477119in}{1.801657in}}{\pgfqpoint{2.481509in}{1.791058in}}{\pgfqpoint{2.489323in}{1.783244in}}%
\pgfpathcurveto{\pgfqpoint{2.497137in}{1.775431in}}{\pgfqpoint{2.507736in}{1.771040in}}{\pgfqpoint{2.518786in}{1.771040in}}%
\pgfpathclose%
\pgfusepath{stroke,fill}%
\end{pgfscope}%
\begin{pgfscope}%
\pgfpathrectangle{\pgfqpoint{0.800000in}{0.528000in}}{\pgfqpoint{4.960000in}{3.696000in}}%
\pgfusepath{clip}%
\pgfsetbuttcap%
\pgfsetroundjoin%
\definecolor{currentfill}{rgb}{0.000000,0.000000,0.000000}%
\pgfsetfillcolor{currentfill}%
\pgfsetlinewidth{1.003750pt}%
\definecolor{currentstroke}{rgb}{0.000000,0.000000,0.000000}%
\pgfsetstrokecolor{currentstroke}%
\pgfsetdash{}{0pt}%
\pgfpathmoveto{\pgfqpoint{2.518786in}{1.771040in}}%
\pgfpathcurveto{\pgfqpoint{2.529836in}{1.771040in}}{\pgfqpoint{2.540435in}{1.775431in}}{\pgfqpoint{2.548249in}{1.783244in}}%
\pgfpathcurveto{\pgfqpoint{2.556062in}{1.791058in}}{\pgfqpoint{2.560452in}{1.801657in}}{\pgfqpoint{2.560452in}{1.812707in}}%
\pgfpathcurveto{\pgfqpoint{2.560452in}{1.823757in}}{\pgfqpoint{2.556062in}{1.834356in}}{\pgfqpoint{2.548249in}{1.842170in}}%
\pgfpathcurveto{\pgfqpoint{2.540435in}{1.849983in}}{\pgfqpoint{2.529836in}{1.854374in}}{\pgfqpoint{2.518786in}{1.854374in}}%
\pgfpathcurveto{\pgfqpoint{2.507736in}{1.854374in}}{\pgfqpoint{2.497137in}{1.849983in}}{\pgfqpoint{2.489323in}{1.842170in}}%
\pgfpathcurveto{\pgfqpoint{2.481509in}{1.834356in}}{\pgfqpoint{2.477119in}{1.823757in}}{\pgfqpoint{2.477119in}{1.812707in}}%
\pgfpathcurveto{\pgfqpoint{2.477119in}{1.801657in}}{\pgfqpoint{2.481509in}{1.791058in}}{\pgfqpoint{2.489323in}{1.783244in}}%
\pgfpathcurveto{\pgfqpoint{2.497137in}{1.775431in}}{\pgfqpoint{2.507736in}{1.771040in}}{\pgfqpoint{2.518786in}{1.771040in}}%
\pgfpathclose%
\pgfusepath{stroke,fill}%
\end{pgfscope}%
\begin{pgfscope}%
\pgfpathrectangle{\pgfqpoint{0.800000in}{0.528000in}}{\pgfqpoint{4.960000in}{3.696000in}}%
\pgfusepath{clip}%
\pgfsetbuttcap%
\pgfsetroundjoin%
\definecolor{currentfill}{rgb}{0.000000,0.000000,0.000000}%
\pgfsetfillcolor{currentfill}%
\pgfsetlinewidth{1.003750pt}%
\definecolor{currentstroke}{rgb}{0.000000,0.000000,0.000000}%
\pgfsetstrokecolor{currentstroke}%
\pgfsetdash{}{0pt}%
\pgfpathmoveto{\pgfqpoint{2.518786in}{1.771040in}}%
\pgfpathcurveto{\pgfqpoint{2.529836in}{1.771040in}}{\pgfqpoint{2.540435in}{1.775431in}}{\pgfqpoint{2.548249in}{1.783244in}}%
\pgfpathcurveto{\pgfqpoint{2.556062in}{1.791058in}}{\pgfqpoint{2.560452in}{1.801657in}}{\pgfqpoint{2.560452in}{1.812707in}}%
\pgfpathcurveto{\pgfqpoint{2.560452in}{1.823757in}}{\pgfqpoint{2.556062in}{1.834356in}}{\pgfqpoint{2.548249in}{1.842170in}}%
\pgfpathcurveto{\pgfqpoint{2.540435in}{1.849983in}}{\pgfqpoint{2.529836in}{1.854374in}}{\pgfqpoint{2.518786in}{1.854374in}}%
\pgfpathcurveto{\pgfqpoint{2.507736in}{1.854374in}}{\pgfqpoint{2.497137in}{1.849983in}}{\pgfqpoint{2.489323in}{1.842170in}}%
\pgfpathcurveto{\pgfqpoint{2.481509in}{1.834356in}}{\pgfqpoint{2.477119in}{1.823757in}}{\pgfqpoint{2.477119in}{1.812707in}}%
\pgfpathcurveto{\pgfqpoint{2.477119in}{1.801657in}}{\pgfqpoint{2.481509in}{1.791058in}}{\pgfqpoint{2.489323in}{1.783244in}}%
\pgfpathcurveto{\pgfqpoint{2.497137in}{1.775431in}}{\pgfqpoint{2.507736in}{1.771040in}}{\pgfqpoint{2.518786in}{1.771040in}}%
\pgfpathclose%
\pgfusepath{stroke,fill}%
\end{pgfscope}%
\begin{pgfscope}%
\pgfpathrectangle{\pgfqpoint{0.800000in}{0.528000in}}{\pgfqpoint{4.960000in}{3.696000in}}%
\pgfusepath{clip}%
\pgfsetbuttcap%
\pgfsetroundjoin%
\definecolor{currentfill}{rgb}{0.000000,0.000000,0.000000}%
\pgfsetfillcolor{currentfill}%
\pgfsetlinewidth{1.003750pt}%
\definecolor{currentstroke}{rgb}{0.000000,0.000000,0.000000}%
\pgfsetstrokecolor{currentstroke}%
\pgfsetdash{}{0pt}%
\pgfpathmoveto{\pgfqpoint{2.518786in}{1.771040in}}%
\pgfpathcurveto{\pgfqpoint{2.529836in}{1.771040in}}{\pgfqpoint{2.540435in}{1.775431in}}{\pgfqpoint{2.548249in}{1.783244in}}%
\pgfpathcurveto{\pgfqpoint{2.556062in}{1.791058in}}{\pgfqpoint{2.560452in}{1.801657in}}{\pgfqpoint{2.560452in}{1.812707in}}%
\pgfpathcurveto{\pgfqpoint{2.560452in}{1.823757in}}{\pgfqpoint{2.556062in}{1.834356in}}{\pgfqpoint{2.548249in}{1.842170in}}%
\pgfpathcurveto{\pgfqpoint{2.540435in}{1.849983in}}{\pgfqpoint{2.529836in}{1.854374in}}{\pgfqpoint{2.518786in}{1.854374in}}%
\pgfpathcurveto{\pgfqpoint{2.507736in}{1.854374in}}{\pgfqpoint{2.497137in}{1.849983in}}{\pgfqpoint{2.489323in}{1.842170in}}%
\pgfpathcurveto{\pgfqpoint{2.481509in}{1.834356in}}{\pgfqpoint{2.477119in}{1.823757in}}{\pgfqpoint{2.477119in}{1.812707in}}%
\pgfpathcurveto{\pgfqpoint{2.477119in}{1.801657in}}{\pgfqpoint{2.481509in}{1.791058in}}{\pgfqpoint{2.489323in}{1.783244in}}%
\pgfpathcurveto{\pgfqpoint{2.497137in}{1.775431in}}{\pgfqpoint{2.507736in}{1.771040in}}{\pgfqpoint{2.518786in}{1.771040in}}%
\pgfpathclose%
\pgfusepath{stroke,fill}%
\end{pgfscope}%
\begin{pgfscope}%
\pgfpathrectangle{\pgfqpoint{0.800000in}{0.528000in}}{\pgfqpoint{4.960000in}{3.696000in}}%
\pgfusepath{clip}%
\pgfsetbuttcap%
\pgfsetroundjoin%
\definecolor{currentfill}{rgb}{0.000000,0.000000,0.000000}%
\pgfsetfillcolor{currentfill}%
\pgfsetlinewidth{1.003750pt}%
\definecolor{currentstroke}{rgb}{0.000000,0.000000,0.000000}%
\pgfsetstrokecolor{currentstroke}%
\pgfsetdash{}{0pt}%
\pgfpathmoveto{\pgfqpoint{2.518786in}{1.771040in}}%
\pgfpathcurveto{\pgfqpoint{2.529836in}{1.771040in}}{\pgfqpoint{2.540435in}{1.775431in}}{\pgfqpoint{2.548249in}{1.783244in}}%
\pgfpathcurveto{\pgfqpoint{2.556062in}{1.791058in}}{\pgfqpoint{2.560452in}{1.801657in}}{\pgfqpoint{2.560452in}{1.812707in}}%
\pgfpathcurveto{\pgfqpoint{2.560452in}{1.823757in}}{\pgfqpoint{2.556062in}{1.834356in}}{\pgfqpoint{2.548249in}{1.842170in}}%
\pgfpathcurveto{\pgfqpoint{2.540435in}{1.849983in}}{\pgfqpoint{2.529836in}{1.854374in}}{\pgfqpoint{2.518786in}{1.854374in}}%
\pgfpathcurveto{\pgfqpoint{2.507736in}{1.854374in}}{\pgfqpoint{2.497137in}{1.849983in}}{\pgfqpoint{2.489323in}{1.842170in}}%
\pgfpathcurveto{\pgfqpoint{2.481509in}{1.834356in}}{\pgfqpoint{2.477119in}{1.823757in}}{\pgfqpoint{2.477119in}{1.812707in}}%
\pgfpathcurveto{\pgfqpoint{2.477119in}{1.801657in}}{\pgfqpoint{2.481509in}{1.791058in}}{\pgfqpoint{2.489323in}{1.783244in}}%
\pgfpathcurveto{\pgfqpoint{2.497137in}{1.775431in}}{\pgfqpoint{2.507736in}{1.771040in}}{\pgfqpoint{2.518786in}{1.771040in}}%
\pgfpathclose%
\pgfusepath{stroke,fill}%
\end{pgfscope}%
\begin{pgfscope}%
\pgfpathrectangle{\pgfqpoint{0.800000in}{0.528000in}}{\pgfqpoint{4.960000in}{3.696000in}}%
\pgfusepath{clip}%
\pgfsetbuttcap%
\pgfsetroundjoin%
\definecolor{currentfill}{rgb}{0.000000,0.000000,0.000000}%
\pgfsetfillcolor{currentfill}%
\pgfsetlinewidth{1.003750pt}%
\definecolor{currentstroke}{rgb}{0.000000,0.000000,0.000000}%
\pgfsetstrokecolor{currentstroke}%
\pgfsetdash{}{0pt}%
\pgfpathmoveto{\pgfqpoint{2.518786in}{1.771040in}}%
\pgfpathcurveto{\pgfqpoint{2.529836in}{1.771040in}}{\pgfqpoint{2.540435in}{1.775431in}}{\pgfqpoint{2.548249in}{1.783244in}}%
\pgfpathcurveto{\pgfqpoint{2.556062in}{1.791058in}}{\pgfqpoint{2.560452in}{1.801657in}}{\pgfqpoint{2.560452in}{1.812707in}}%
\pgfpathcurveto{\pgfqpoint{2.560452in}{1.823757in}}{\pgfqpoint{2.556062in}{1.834356in}}{\pgfqpoint{2.548249in}{1.842170in}}%
\pgfpathcurveto{\pgfqpoint{2.540435in}{1.849983in}}{\pgfqpoint{2.529836in}{1.854374in}}{\pgfqpoint{2.518786in}{1.854374in}}%
\pgfpathcurveto{\pgfqpoint{2.507736in}{1.854374in}}{\pgfqpoint{2.497137in}{1.849983in}}{\pgfqpoint{2.489323in}{1.842170in}}%
\pgfpathcurveto{\pgfqpoint{2.481509in}{1.834356in}}{\pgfqpoint{2.477119in}{1.823757in}}{\pgfqpoint{2.477119in}{1.812707in}}%
\pgfpathcurveto{\pgfqpoint{2.477119in}{1.801657in}}{\pgfqpoint{2.481509in}{1.791058in}}{\pgfqpoint{2.489323in}{1.783244in}}%
\pgfpathcurveto{\pgfqpoint{2.497137in}{1.775431in}}{\pgfqpoint{2.507736in}{1.771040in}}{\pgfqpoint{2.518786in}{1.771040in}}%
\pgfpathclose%
\pgfusepath{stroke,fill}%
\end{pgfscope}%
\begin{pgfscope}%
\pgfpathrectangle{\pgfqpoint{0.800000in}{0.528000in}}{\pgfqpoint{4.960000in}{3.696000in}}%
\pgfusepath{clip}%
\pgfsetbuttcap%
\pgfsetroundjoin%
\definecolor{currentfill}{rgb}{0.000000,0.000000,0.000000}%
\pgfsetfillcolor{currentfill}%
\pgfsetlinewidth{1.003750pt}%
\definecolor{currentstroke}{rgb}{0.000000,0.000000,0.000000}%
\pgfsetstrokecolor{currentstroke}%
\pgfsetdash{}{0pt}%
\pgfpathmoveto{\pgfqpoint{2.518786in}{1.771040in}}%
\pgfpathcurveto{\pgfqpoint{2.529836in}{1.771040in}}{\pgfqpoint{2.540435in}{1.775431in}}{\pgfqpoint{2.548249in}{1.783244in}}%
\pgfpathcurveto{\pgfqpoint{2.556062in}{1.791058in}}{\pgfqpoint{2.560452in}{1.801657in}}{\pgfqpoint{2.560452in}{1.812707in}}%
\pgfpathcurveto{\pgfqpoint{2.560452in}{1.823757in}}{\pgfqpoint{2.556062in}{1.834356in}}{\pgfqpoint{2.548249in}{1.842170in}}%
\pgfpathcurveto{\pgfqpoint{2.540435in}{1.849983in}}{\pgfqpoint{2.529836in}{1.854374in}}{\pgfqpoint{2.518786in}{1.854374in}}%
\pgfpathcurveto{\pgfqpoint{2.507736in}{1.854374in}}{\pgfqpoint{2.497137in}{1.849983in}}{\pgfqpoint{2.489323in}{1.842170in}}%
\pgfpathcurveto{\pgfqpoint{2.481509in}{1.834356in}}{\pgfqpoint{2.477119in}{1.823757in}}{\pgfqpoint{2.477119in}{1.812707in}}%
\pgfpathcurveto{\pgfqpoint{2.477119in}{1.801657in}}{\pgfqpoint{2.481509in}{1.791058in}}{\pgfqpoint{2.489323in}{1.783244in}}%
\pgfpathcurveto{\pgfqpoint{2.497137in}{1.775431in}}{\pgfqpoint{2.507736in}{1.771040in}}{\pgfqpoint{2.518786in}{1.771040in}}%
\pgfpathclose%
\pgfusepath{stroke,fill}%
\end{pgfscope}%
\begin{pgfscope}%
\pgfpathrectangle{\pgfqpoint{0.800000in}{0.528000in}}{\pgfqpoint{4.960000in}{3.696000in}}%
\pgfusepath{clip}%
\pgfsetbuttcap%
\pgfsetroundjoin%
\definecolor{currentfill}{rgb}{0.000000,0.000000,0.000000}%
\pgfsetfillcolor{currentfill}%
\pgfsetlinewidth{1.003750pt}%
\definecolor{currentstroke}{rgb}{0.000000,0.000000,0.000000}%
\pgfsetstrokecolor{currentstroke}%
\pgfsetdash{}{0pt}%
\pgfpathmoveto{\pgfqpoint{2.518786in}{1.771040in}}%
\pgfpathcurveto{\pgfqpoint{2.529836in}{1.771040in}}{\pgfqpoint{2.540435in}{1.775431in}}{\pgfqpoint{2.548249in}{1.783244in}}%
\pgfpathcurveto{\pgfqpoint{2.556062in}{1.791058in}}{\pgfqpoint{2.560452in}{1.801657in}}{\pgfqpoint{2.560452in}{1.812707in}}%
\pgfpathcurveto{\pgfqpoint{2.560452in}{1.823757in}}{\pgfqpoint{2.556062in}{1.834356in}}{\pgfqpoint{2.548249in}{1.842170in}}%
\pgfpathcurveto{\pgfqpoint{2.540435in}{1.849983in}}{\pgfqpoint{2.529836in}{1.854374in}}{\pgfqpoint{2.518786in}{1.854374in}}%
\pgfpathcurveto{\pgfqpoint{2.507736in}{1.854374in}}{\pgfqpoint{2.497137in}{1.849983in}}{\pgfqpoint{2.489323in}{1.842170in}}%
\pgfpathcurveto{\pgfqpoint{2.481509in}{1.834356in}}{\pgfqpoint{2.477119in}{1.823757in}}{\pgfqpoint{2.477119in}{1.812707in}}%
\pgfpathcurveto{\pgfqpoint{2.477119in}{1.801657in}}{\pgfqpoint{2.481509in}{1.791058in}}{\pgfqpoint{2.489323in}{1.783244in}}%
\pgfpathcurveto{\pgfqpoint{2.497137in}{1.775431in}}{\pgfqpoint{2.507736in}{1.771040in}}{\pgfqpoint{2.518786in}{1.771040in}}%
\pgfpathclose%
\pgfusepath{stroke,fill}%
\end{pgfscope}%
\begin{pgfscope}%
\pgfpathrectangle{\pgfqpoint{0.800000in}{0.528000in}}{\pgfqpoint{4.960000in}{3.696000in}}%
\pgfusepath{clip}%
\pgfsetbuttcap%
\pgfsetroundjoin%
\definecolor{currentfill}{rgb}{0.000000,0.000000,0.000000}%
\pgfsetfillcolor{currentfill}%
\pgfsetlinewidth{1.003750pt}%
\definecolor{currentstroke}{rgb}{0.000000,0.000000,0.000000}%
\pgfsetstrokecolor{currentstroke}%
\pgfsetdash{}{0pt}%
\pgfpathmoveto{\pgfqpoint{2.518786in}{0.664394in}}%
\pgfpathcurveto{\pgfqpoint{2.529836in}{0.664394in}}{\pgfqpoint{2.540435in}{0.668784in}}{\pgfqpoint{2.548249in}{0.676598in}}%
\pgfpathcurveto{\pgfqpoint{2.556062in}{0.684411in}}{\pgfqpoint{2.560452in}{0.695010in}}{\pgfqpoint{2.560452in}{0.706060in}}%
\pgfpathcurveto{\pgfqpoint{2.560452in}{0.717111in}}{\pgfqpoint{2.556062in}{0.727710in}}{\pgfqpoint{2.548249in}{0.735523in}}%
\pgfpathcurveto{\pgfqpoint{2.540435in}{0.743337in}}{\pgfqpoint{2.529836in}{0.747727in}}{\pgfqpoint{2.518786in}{0.747727in}}%
\pgfpathcurveto{\pgfqpoint{2.507736in}{0.747727in}}{\pgfqpoint{2.497137in}{0.743337in}}{\pgfqpoint{2.489323in}{0.735523in}}%
\pgfpathcurveto{\pgfqpoint{2.481509in}{0.727710in}}{\pgfqpoint{2.477119in}{0.717111in}}{\pgfqpoint{2.477119in}{0.706060in}}%
\pgfpathcurveto{\pgfqpoint{2.477119in}{0.695010in}}{\pgfqpoint{2.481509in}{0.684411in}}{\pgfqpoint{2.489323in}{0.676598in}}%
\pgfpathcurveto{\pgfqpoint{2.497137in}{0.668784in}}{\pgfqpoint{2.507736in}{0.664394in}}{\pgfqpoint{2.518786in}{0.664394in}}%
\pgfpathclose%
\pgfusepath{stroke,fill}%
\end{pgfscope}%
\begin{pgfscope}%
\pgfpathrectangle{\pgfqpoint{0.800000in}{0.528000in}}{\pgfqpoint{4.960000in}{3.696000in}}%
\pgfusepath{clip}%
\pgfsetbuttcap%
\pgfsetroundjoin%
\definecolor{currentfill}{rgb}{0.000000,0.000000,0.000000}%
\pgfsetfillcolor{currentfill}%
\pgfsetlinewidth{1.003750pt}%
\definecolor{currentstroke}{rgb}{0.000000,0.000000,0.000000}%
\pgfsetstrokecolor{currentstroke}%
\pgfsetdash{}{0pt}%
\pgfpathmoveto{\pgfqpoint{2.518786in}{1.771040in}}%
\pgfpathcurveto{\pgfqpoint{2.529836in}{1.771040in}}{\pgfqpoint{2.540435in}{1.775431in}}{\pgfqpoint{2.548249in}{1.783244in}}%
\pgfpathcurveto{\pgfqpoint{2.556062in}{1.791058in}}{\pgfqpoint{2.560452in}{1.801657in}}{\pgfqpoint{2.560452in}{1.812707in}}%
\pgfpathcurveto{\pgfqpoint{2.560452in}{1.823757in}}{\pgfqpoint{2.556062in}{1.834356in}}{\pgfqpoint{2.548249in}{1.842170in}}%
\pgfpathcurveto{\pgfqpoint{2.540435in}{1.849983in}}{\pgfqpoint{2.529836in}{1.854374in}}{\pgfqpoint{2.518786in}{1.854374in}}%
\pgfpathcurveto{\pgfqpoint{2.507736in}{1.854374in}}{\pgfqpoint{2.497137in}{1.849983in}}{\pgfqpoint{2.489323in}{1.842170in}}%
\pgfpathcurveto{\pgfqpoint{2.481509in}{1.834356in}}{\pgfqpoint{2.477119in}{1.823757in}}{\pgfqpoint{2.477119in}{1.812707in}}%
\pgfpathcurveto{\pgfqpoint{2.477119in}{1.801657in}}{\pgfqpoint{2.481509in}{1.791058in}}{\pgfqpoint{2.489323in}{1.783244in}}%
\pgfpathcurveto{\pgfqpoint{2.497137in}{1.775431in}}{\pgfqpoint{2.507736in}{1.771040in}}{\pgfqpoint{2.518786in}{1.771040in}}%
\pgfpathclose%
\pgfusepath{stroke,fill}%
\end{pgfscope}%
\begin{pgfscope}%
\pgfpathrectangle{\pgfqpoint{0.800000in}{0.528000in}}{\pgfqpoint{4.960000in}{3.696000in}}%
\pgfusepath{clip}%
\pgfsetbuttcap%
\pgfsetroundjoin%
\definecolor{currentfill}{rgb}{0.000000,0.000000,0.000000}%
\pgfsetfillcolor{currentfill}%
\pgfsetlinewidth{1.003750pt}%
\definecolor{currentstroke}{rgb}{0.000000,0.000000,0.000000}%
\pgfsetstrokecolor{currentstroke}%
\pgfsetdash{}{0pt}%
\pgfpathmoveto{\pgfqpoint{2.518786in}{1.771040in}}%
\pgfpathcurveto{\pgfqpoint{2.529836in}{1.771040in}}{\pgfqpoint{2.540435in}{1.775431in}}{\pgfqpoint{2.548249in}{1.783244in}}%
\pgfpathcurveto{\pgfqpoint{2.556062in}{1.791058in}}{\pgfqpoint{2.560452in}{1.801657in}}{\pgfqpoint{2.560452in}{1.812707in}}%
\pgfpathcurveto{\pgfqpoint{2.560452in}{1.823757in}}{\pgfqpoint{2.556062in}{1.834356in}}{\pgfqpoint{2.548249in}{1.842170in}}%
\pgfpathcurveto{\pgfqpoint{2.540435in}{1.849983in}}{\pgfqpoint{2.529836in}{1.854374in}}{\pgfqpoint{2.518786in}{1.854374in}}%
\pgfpathcurveto{\pgfqpoint{2.507736in}{1.854374in}}{\pgfqpoint{2.497137in}{1.849983in}}{\pgfqpoint{2.489323in}{1.842170in}}%
\pgfpathcurveto{\pgfqpoint{2.481509in}{1.834356in}}{\pgfqpoint{2.477119in}{1.823757in}}{\pgfqpoint{2.477119in}{1.812707in}}%
\pgfpathcurveto{\pgfqpoint{2.477119in}{1.801657in}}{\pgfqpoint{2.481509in}{1.791058in}}{\pgfqpoint{2.489323in}{1.783244in}}%
\pgfpathcurveto{\pgfqpoint{2.497137in}{1.775431in}}{\pgfqpoint{2.507736in}{1.771040in}}{\pgfqpoint{2.518786in}{1.771040in}}%
\pgfpathclose%
\pgfusepath{stroke,fill}%
\end{pgfscope}%
\begin{pgfscope}%
\pgfpathrectangle{\pgfqpoint{0.800000in}{0.528000in}}{\pgfqpoint{4.960000in}{3.696000in}}%
\pgfusepath{clip}%
\pgfsetbuttcap%
\pgfsetroundjoin%
\definecolor{currentfill}{rgb}{0.000000,0.000000,0.000000}%
\pgfsetfillcolor{currentfill}%
\pgfsetlinewidth{1.003750pt}%
\definecolor{currentstroke}{rgb}{0.000000,0.000000,0.000000}%
\pgfsetstrokecolor{currentstroke}%
\pgfsetdash{}{0pt}%
\pgfpathmoveto{\pgfqpoint{2.518786in}{1.771040in}}%
\pgfpathcurveto{\pgfqpoint{2.529836in}{1.771040in}}{\pgfqpoint{2.540435in}{1.775431in}}{\pgfqpoint{2.548249in}{1.783244in}}%
\pgfpathcurveto{\pgfqpoint{2.556062in}{1.791058in}}{\pgfqpoint{2.560452in}{1.801657in}}{\pgfqpoint{2.560452in}{1.812707in}}%
\pgfpathcurveto{\pgfqpoint{2.560452in}{1.823757in}}{\pgfqpoint{2.556062in}{1.834356in}}{\pgfqpoint{2.548249in}{1.842170in}}%
\pgfpathcurveto{\pgfqpoint{2.540435in}{1.849983in}}{\pgfqpoint{2.529836in}{1.854374in}}{\pgfqpoint{2.518786in}{1.854374in}}%
\pgfpathcurveto{\pgfqpoint{2.507736in}{1.854374in}}{\pgfqpoint{2.497137in}{1.849983in}}{\pgfqpoint{2.489323in}{1.842170in}}%
\pgfpathcurveto{\pgfqpoint{2.481509in}{1.834356in}}{\pgfqpoint{2.477119in}{1.823757in}}{\pgfqpoint{2.477119in}{1.812707in}}%
\pgfpathcurveto{\pgfqpoint{2.477119in}{1.801657in}}{\pgfqpoint{2.481509in}{1.791058in}}{\pgfqpoint{2.489323in}{1.783244in}}%
\pgfpathcurveto{\pgfqpoint{2.497137in}{1.775431in}}{\pgfqpoint{2.507736in}{1.771040in}}{\pgfqpoint{2.518786in}{1.771040in}}%
\pgfpathclose%
\pgfusepath{stroke,fill}%
\end{pgfscope}%
\begin{pgfscope}%
\pgfpathrectangle{\pgfqpoint{0.800000in}{0.528000in}}{\pgfqpoint{4.960000in}{3.696000in}}%
\pgfusepath{clip}%
\pgfsetbuttcap%
\pgfsetroundjoin%
\definecolor{currentfill}{rgb}{0.000000,0.000000,0.000000}%
\pgfsetfillcolor{currentfill}%
\pgfsetlinewidth{1.003750pt}%
\definecolor{currentstroke}{rgb}{0.000000,0.000000,0.000000}%
\pgfsetstrokecolor{currentstroke}%
\pgfsetdash{}{0pt}%
\pgfpathmoveto{\pgfqpoint{2.518786in}{1.771040in}}%
\pgfpathcurveto{\pgfqpoint{2.529836in}{1.771040in}}{\pgfqpoint{2.540435in}{1.775431in}}{\pgfqpoint{2.548249in}{1.783244in}}%
\pgfpathcurveto{\pgfqpoint{2.556062in}{1.791058in}}{\pgfqpoint{2.560452in}{1.801657in}}{\pgfqpoint{2.560452in}{1.812707in}}%
\pgfpathcurveto{\pgfqpoint{2.560452in}{1.823757in}}{\pgfqpoint{2.556062in}{1.834356in}}{\pgfqpoint{2.548249in}{1.842170in}}%
\pgfpathcurveto{\pgfqpoint{2.540435in}{1.849983in}}{\pgfqpoint{2.529836in}{1.854374in}}{\pgfqpoint{2.518786in}{1.854374in}}%
\pgfpathcurveto{\pgfqpoint{2.507736in}{1.854374in}}{\pgfqpoint{2.497137in}{1.849983in}}{\pgfqpoint{2.489323in}{1.842170in}}%
\pgfpathcurveto{\pgfqpoint{2.481509in}{1.834356in}}{\pgfqpoint{2.477119in}{1.823757in}}{\pgfqpoint{2.477119in}{1.812707in}}%
\pgfpathcurveto{\pgfqpoint{2.477119in}{1.801657in}}{\pgfqpoint{2.481509in}{1.791058in}}{\pgfqpoint{2.489323in}{1.783244in}}%
\pgfpathcurveto{\pgfqpoint{2.497137in}{1.775431in}}{\pgfqpoint{2.507736in}{1.771040in}}{\pgfqpoint{2.518786in}{1.771040in}}%
\pgfpathclose%
\pgfusepath{stroke,fill}%
\end{pgfscope}%
\begin{pgfscope}%
\pgfpathrectangle{\pgfqpoint{0.800000in}{0.528000in}}{\pgfqpoint{4.960000in}{3.696000in}}%
\pgfusepath{clip}%
\pgfsetbuttcap%
\pgfsetroundjoin%
\definecolor{currentfill}{rgb}{0.000000,0.000000,0.000000}%
\pgfsetfillcolor{currentfill}%
\pgfsetlinewidth{1.003750pt}%
\definecolor{currentstroke}{rgb}{0.000000,0.000000,0.000000}%
\pgfsetstrokecolor{currentstroke}%
\pgfsetdash{}{0pt}%
\pgfpathmoveto{\pgfqpoint{2.518786in}{1.771040in}}%
\pgfpathcurveto{\pgfqpoint{2.529836in}{1.771040in}}{\pgfqpoint{2.540435in}{1.775431in}}{\pgfqpoint{2.548249in}{1.783244in}}%
\pgfpathcurveto{\pgfqpoint{2.556062in}{1.791058in}}{\pgfqpoint{2.560452in}{1.801657in}}{\pgfqpoint{2.560452in}{1.812707in}}%
\pgfpathcurveto{\pgfqpoint{2.560452in}{1.823757in}}{\pgfqpoint{2.556062in}{1.834356in}}{\pgfqpoint{2.548249in}{1.842170in}}%
\pgfpathcurveto{\pgfqpoint{2.540435in}{1.849983in}}{\pgfqpoint{2.529836in}{1.854374in}}{\pgfqpoint{2.518786in}{1.854374in}}%
\pgfpathcurveto{\pgfqpoint{2.507736in}{1.854374in}}{\pgfqpoint{2.497137in}{1.849983in}}{\pgfqpoint{2.489323in}{1.842170in}}%
\pgfpathcurveto{\pgfqpoint{2.481509in}{1.834356in}}{\pgfqpoint{2.477119in}{1.823757in}}{\pgfqpoint{2.477119in}{1.812707in}}%
\pgfpathcurveto{\pgfqpoint{2.477119in}{1.801657in}}{\pgfqpoint{2.481509in}{1.791058in}}{\pgfqpoint{2.489323in}{1.783244in}}%
\pgfpathcurveto{\pgfqpoint{2.497137in}{1.775431in}}{\pgfqpoint{2.507736in}{1.771040in}}{\pgfqpoint{2.518786in}{1.771040in}}%
\pgfpathclose%
\pgfusepath{stroke,fill}%
\end{pgfscope}%
\begin{pgfscope}%
\pgfpathrectangle{\pgfqpoint{0.800000in}{0.528000in}}{\pgfqpoint{4.960000in}{3.696000in}}%
\pgfusepath{clip}%
\pgfsetbuttcap%
\pgfsetroundjoin%
\definecolor{currentfill}{rgb}{0.000000,0.000000,0.000000}%
\pgfsetfillcolor{currentfill}%
\pgfsetlinewidth{1.003750pt}%
\definecolor{currentstroke}{rgb}{0.000000,0.000000,0.000000}%
\pgfsetstrokecolor{currentstroke}%
\pgfsetdash{}{0pt}%
\pgfpathmoveto{\pgfqpoint{2.518786in}{1.771040in}}%
\pgfpathcurveto{\pgfqpoint{2.529836in}{1.771040in}}{\pgfqpoint{2.540435in}{1.775431in}}{\pgfqpoint{2.548249in}{1.783244in}}%
\pgfpathcurveto{\pgfqpoint{2.556062in}{1.791058in}}{\pgfqpoint{2.560452in}{1.801657in}}{\pgfqpoint{2.560452in}{1.812707in}}%
\pgfpathcurveto{\pgfqpoint{2.560452in}{1.823757in}}{\pgfqpoint{2.556062in}{1.834356in}}{\pgfqpoint{2.548249in}{1.842170in}}%
\pgfpathcurveto{\pgfqpoint{2.540435in}{1.849983in}}{\pgfqpoint{2.529836in}{1.854374in}}{\pgfqpoint{2.518786in}{1.854374in}}%
\pgfpathcurveto{\pgfqpoint{2.507736in}{1.854374in}}{\pgfqpoint{2.497137in}{1.849983in}}{\pgfqpoint{2.489323in}{1.842170in}}%
\pgfpathcurveto{\pgfqpoint{2.481509in}{1.834356in}}{\pgfqpoint{2.477119in}{1.823757in}}{\pgfqpoint{2.477119in}{1.812707in}}%
\pgfpathcurveto{\pgfqpoint{2.477119in}{1.801657in}}{\pgfqpoint{2.481509in}{1.791058in}}{\pgfqpoint{2.489323in}{1.783244in}}%
\pgfpathcurveto{\pgfqpoint{2.497137in}{1.775431in}}{\pgfqpoint{2.507736in}{1.771040in}}{\pgfqpoint{2.518786in}{1.771040in}}%
\pgfpathclose%
\pgfusepath{stroke,fill}%
\end{pgfscope}%
\begin{pgfscope}%
\pgfpathrectangle{\pgfqpoint{0.800000in}{0.528000in}}{\pgfqpoint{4.960000in}{3.696000in}}%
\pgfusepath{clip}%
\pgfsetbuttcap%
\pgfsetroundjoin%
\definecolor{currentfill}{rgb}{0.000000,0.000000,0.000000}%
\pgfsetfillcolor{currentfill}%
\pgfsetlinewidth{1.003750pt}%
\definecolor{currentstroke}{rgb}{0.000000,0.000000,0.000000}%
\pgfsetstrokecolor{currentstroke}%
\pgfsetdash{}{0pt}%
\pgfpathmoveto{\pgfqpoint{2.518786in}{1.771040in}}%
\pgfpathcurveto{\pgfqpoint{2.529836in}{1.771040in}}{\pgfqpoint{2.540435in}{1.775431in}}{\pgfqpoint{2.548249in}{1.783244in}}%
\pgfpathcurveto{\pgfqpoint{2.556062in}{1.791058in}}{\pgfqpoint{2.560452in}{1.801657in}}{\pgfqpoint{2.560452in}{1.812707in}}%
\pgfpathcurveto{\pgfqpoint{2.560452in}{1.823757in}}{\pgfqpoint{2.556062in}{1.834356in}}{\pgfqpoint{2.548249in}{1.842170in}}%
\pgfpathcurveto{\pgfqpoint{2.540435in}{1.849983in}}{\pgfqpoint{2.529836in}{1.854374in}}{\pgfqpoint{2.518786in}{1.854374in}}%
\pgfpathcurveto{\pgfqpoint{2.507736in}{1.854374in}}{\pgfqpoint{2.497137in}{1.849983in}}{\pgfqpoint{2.489323in}{1.842170in}}%
\pgfpathcurveto{\pgfqpoint{2.481509in}{1.834356in}}{\pgfqpoint{2.477119in}{1.823757in}}{\pgfqpoint{2.477119in}{1.812707in}}%
\pgfpathcurveto{\pgfqpoint{2.477119in}{1.801657in}}{\pgfqpoint{2.481509in}{1.791058in}}{\pgfqpoint{2.489323in}{1.783244in}}%
\pgfpathcurveto{\pgfqpoint{2.497137in}{1.775431in}}{\pgfqpoint{2.507736in}{1.771040in}}{\pgfqpoint{2.518786in}{1.771040in}}%
\pgfpathclose%
\pgfusepath{stroke,fill}%
\end{pgfscope}%
\begin{pgfscope}%
\pgfpathrectangle{\pgfqpoint{0.800000in}{0.528000in}}{\pgfqpoint{4.960000in}{3.696000in}}%
\pgfusepath{clip}%
\pgfsetbuttcap%
\pgfsetroundjoin%
\definecolor{currentfill}{rgb}{0.000000,0.000000,0.000000}%
\pgfsetfillcolor{currentfill}%
\pgfsetlinewidth{1.003750pt}%
\definecolor{currentstroke}{rgb}{0.000000,0.000000,0.000000}%
\pgfsetstrokecolor{currentstroke}%
\pgfsetdash{}{0pt}%
\pgfpathmoveto{\pgfqpoint{2.518786in}{1.771040in}}%
\pgfpathcurveto{\pgfqpoint{2.529836in}{1.771040in}}{\pgfqpoint{2.540435in}{1.775431in}}{\pgfqpoint{2.548249in}{1.783244in}}%
\pgfpathcurveto{\pgfqpoint{2.556062in}{1.791058in}}{\pgfqpoint{2.560452in}{1.801657in}}{\pgfqpoint{2.560452in}{1.812707in}}%
\pgfpathcurveto{\pgfqpoint{2.560452in}{1.823757in}}{\pgfqpoint{2.556062in}{1.834356in}}{\pgfqpoint{2.548249in}{1.842170in}}%
\pgfpathcurveto{\pgfqpoint{2.540435in}{1.849983in}}{\pgfqpoint{2.529836in}{1.854374in}}{\pgfqpoint{2.518786in}{1.854374in}}%
\pgfpathcurveto{\pgfqpoint{2.507736in}{1.854374in}}{\pgfqpoint{2.497137in}{1.849983in}}{\pgfqpoint{2.489323in}{1.842170in}}%
\pgfpathcurveto{\pgfqpoint{2.481509in}{1.834356in}}{\pgfqpoint{2.477119in}{1.823757in}}{\pgfqpoint{2.477119in}{1.812707in}}%
\pgfpathcurveto{\pgfqpoint{2.477119in}{1.801657in}}{\pgfqpoint{2.481509in}{1.791058in}}{\pgfqpoint{2.489323in}{1.783244in}}%
\pgfpathcurveto{\pgfqpoint{2.497137in}{1.775431in}}{\pgfqpoint{2.507736in}{1.771040in}}{\pgfqpoint{2.518786in}{1.771040in}}%
\pgfpathclose%
\pgfusepath{stroke,fill}%
\end{pgfscope}%
\begin{pgfscope}%
\pgfpathrectangle{\pgfqpoint{0.800000in}{0.528000in}}{\pgfqpoint{4.960000in}{3.696000in}}%
\pgfusepath{clip}%
\pgfsetbuttcap%
\pgfsetroundjoin%
\definecolor{currentfill}{rgb}{0.000000,0.000000,0.000000}%
\pgfsetfillcolor{currentfill}%
\pgfsetlinewidth{1.003750pt}%
\definecolor{currentstroke}{rgb}{0.000000,0.000000,0.000000}%
\pgfsetstrokecolor{currentstroke}%
\pgfsetdash{}{0pt}%
\pgfpathmoveto{\pgfqpoint{2.518786in}{1.771040in}}%
\pgfpathcurveto{\pgfqpoint{2.529836in}{1.771040in}}{\pgfqpoint{2.540435in}{1.775431in}}{\pgfqpoint{2.548249in}{1.783244in}}%
\pgfpathcurveto{\pgfqpoint{2.556062in}{1.791058in}}{\pgfqpoint{2.560452in}{1.801657in}}{\pgfqpoint{2.560452in}{1.812707in}}%
\pgfpathcurveto{\pgfqpoint{2.560452in}{1.823757in}}{\pgfqpoint{2.556062in}{1.834356in}}{\pgfqpoint{2.548249in}{1.842170in}}%
\pgfpathcurveto{\pgfqpoint{2.540435in}{1.849983in}}{\pgfqpoint{2.529836in}{1.854374in}}{\pgfqpoint{2.518786in}{1.854374in}}%
\pgfpathcurveto{\pgfqpoint{2.507736in}{1.854374in}}{\pgfqpoint{2.497137in}{1.849983in}}{\pgfqpoint{2.489323in}{1.842170in}}%
\pgfpathcurveto{\pgfqpoint{2.481509in}{1.834356in}}{\pgfqpoint{2.477119in}{1.823757in}}{\pgfqpoint{2.477119in}{1.812707in}}%
\pgfpathcurveto{\pgfqpoint{2.477119in}{1.801657in}}{\pgfqpoint{2.481509in}{1.791058in}}{\pgfqpoint{2.489323in}{1.783244in}}%
\pgfpathcurveto{\pgfqpoint{2.497137in}{1.775431in}}{\pgfqpoint{2.507736in}{1.771040in}}{\pgfqpoint{2.518786in}{1.771040in}}%
\pgfpathclose%
\pgfusepath{stroke,fill}%
\end{pgfscope}%
\begin{pgfscope}%
\pgfpathrectangle{\pgfqpoint{0.800000in}{0.528000in}}{\pgfqpoint{4.960000in}{3.696000in}}%
\pgfusepath{clip}%
\pgfsetbuttcap%
\pgfsetroundjoin%
\definecolor{currentfill}{rgb}{0.000000,0.000000,0.000000}%
\pgfsetfillcolor{currentfill}%
\pgfsetlinewidth{1.003750pt}%
\definecolor{currentstroke}{rgb}{0.000000,0.000000,0.000000}%
\pgfsetstrokecolor{currentstroke}%
\pgfsetdash{}{0pt}%
\pgfpathmoveto{\pgfqpoint{2.518786in}{1.771040in}}%
\pgfpathcurveto{\pgfqpoint{2.529836in}{1.771040in}}{\pgfqpoint{2.540435in}{1.775431in}}{\pgfqpoint{2.548249in}{1.783244in}}%
\pgfpathcurveto{\pgfqpoint{2.556062in}{1.791058in}}{\pgfqpoint{2.560452in}{1.801657in}}{\pgfqpoint{2.560452in}{1.812707in}}%
\pgfpathcurveto{\pgfqpoint{2.560452in}{1.823757in}}{\pgfqpoint{2.556062in}{1.834356in}}{\pgfqpoint{2.548249in}{1.842170in}}%
\pgfpathcurveto{\pgfqpoint{2.540435in}{1.849983in}}{\pgfqpoint{2.529836in}{1.854374in}}{\pgfqpoint{2.518786in}{1.854374in}}%
\pgfpathcurveto{\pgfqpoint{2.507736in}{1.854374in}}{\pgfqpoint{2.497137in}{1.849983in}}{\pgfqpoint{2.489323in}{1.842170in}}%
\pgfpathcurveto{\pgfqpoint{2.481509in}{1.834356in}}{\pgfqpoint{2.477119in}{1.823757in}}{\pgfqpoint{2.477119in}{1.812707in}}%
\pgfpathcurveto{\pgfqpoint{2.477119in}{1.801657in}}{\pgfqpoint{2.481509in}{1.791058in}}{\pgfqpoint{2.489323in}{1.783244in}}%
\pgfpathcurveto{\pgfqpoint{2.497137in}{1.775431in}}{\pgfqpoint{2.507736in}{1.771040in}}{\pgfqpoint{2.518786in}{1.771040in}}%
\pgfpathclose%
\pgfusepath{stroke,fill}%
\end{pgfscope}%
\begin{pgfscope}%
\pgfpathrectangle{\pgfqpoint{0.800000in}{0.528000in}}{\pgfqpoint{4.960000in}{3.696000in}}%
\pgfusepath{clip}%
\pgfsetbuttcap%
\pgfsetroundjoin%
\definecolor{currentfill}{rgb}{0.000000,0.000000,0.000000}%
\pgfsetfillcolor{currentfill}%
\pgfsetlinewidth{1.003750pt}%
\definecolor{currentstroke}{rgb}{0.000000,0.000000,0.000000}%
\pgfsetstrokecolor{currentstroke}%
\pgfsetdash{}{0pt}%
\pgfpathmoveto{\pgfqpoint{2.518786in}{1.771040in}}%
\pgfpathcurveto{\pgfqpoint{2.529836in}{1.771040in}}{\pgfqpoint{2.540435in}{1.775431in}}{\pgfqpoint{2.548249in}{1.783244in}}%
\pgfpathcurveto{\pgfqpoint{2.556062in}{1.791058in}}{\pgfqpoint{2.560452in}{1.801657in}}{\pgfqpoint{2.560452in}{1.812707in}}%
\pgfpathcurveto{\pgfqpoint{2.560452in}{1.823757in}}{\pgfqpoint{2.556062in}{1.834356in}}{\pgfqpoint{2.548249in}{1.842170in}}%
\pgfpathcurveto{\pgfqpoint{2.540435in}{1.849983in}}{\pgfqpoint{2.529836in}{1.854374in}}{\pgfqpoint{2.518786in}{1.854374in}}%
\pgfpathcurveto{\pgfqpoint{2.507736in}{1.854374in}}{\pgfqpoint{2.497137in}{1.849983in}}{\pgfqpoint{2.489323in}{1.842170in}}%
\pgfpathcurveto{\pgfqpoint{2.481509in}{1.834356in}}{\pgfqpoint{2.477119in}{1.823757in}}{\pgfqpoint{2.477119in}{1.812707in}}%
\pgfpathcurveto{\pgfqpoint{2.477119in}{1.801657in}}{\pgfqpoint{2.481509in}{1.791058in}}{\pgfqpoint{2.489323in}{1.783244in}}%
\pgfpathcurveto{\pgfqpoint{2.497137in}{1.775431in}}{\pgfqpoint{2.507736in}{1.771040in}}{\pgfqpoint{2.518786in}{1.771040in}}%
\pgfpathclose%
\pgfusepath{stroke,fill}%
\end{pgfscope}%
\begin{pgfscope}%
\pgfpathrectangle{\pgfqpoint{0.800000in}{0.528000in}}{\pgfqpoint{4.960000in}{3.696000in}}%
\pgfusepath{clip}%
\pgfsetbuttcap%
\pgfsetroundjoin%
\definecolor{currentfill}{rgb}{0.000000,0.000000,0.000000}%
\pgfsetfillcolor{currentfill}%
\pgfsetlinewidth{1.003750pt}%
\definecolor{currentstroke}{rgb}{0.000000,0.000000,0.000000}%
\pgfsetstrokecolor{currentstroke}%
\pgfsetdash{}{0pt}%
\pgfpathmoveto{\pgfqpoint{2.518786in}{1.771040in}}%
\pgfpathcurveto{\pgfqpoint{2.529836in}{1.771040in}}{\pgfqpoint{2.540435in}{1.775431in}}{\pgfqpoint{2.548249in}{1.783244in}}%
\pgfpathcurveto{\pgfqpoint{2.556062in}{1.791058in}}{\pgfqpoint{2.560452in}{1.801657in}}{\pgfqpoint{2.560452in}{1.812707in}}%
\pgfpathcurveto{\pgfqpoint{2.560452in}{1.823757in}}{\pgfqpoint{2.556062in}{1.834356in}}{\pgfqpoint{2.548249in}{1.842170in}}%
\pgfpathcurveto{\pgfqpoint{2.540435in}{1.849983in}}{\pgfqpoint{2.529836in}{1.854374in}}{\pgfqpoint{2.518786in}{1.854374in}}%
\pgfpathcurveto{\pgfqpoint{2.507736in}{1.854374in}}{\pgfqpoint{2.497137in}{1.849983in}}{\pgfqpoint{2.489323in}{1.842170in}}%
\pgfpathcurveto{\pgfqpoint{2.481509in}{1.834356in}}{\pgfqpoint{2.477119in}{1.823757in}}{\pgfqpoint{2.477119in}{1.812707in}}%
\pgfpathcurveto{\pgfqpoint{2.477119in}{1.801657in}}{\pgfqpoint{2.481509in}{1.791058in}}{\pgfqpoint{2.489323in}{1.783244in}}%
\pgfpathcurveto{\pgfqpoint{2.497137in}{1.775431in}}{\pgfqpoint{2.507736in}{1.771040in}}{\pgfqpoint{2.518786in}{1.771040in}}%
\pgfpathclose%
\pgfusepath{stroke,fill}%
\end{pgfscope}%
\begin{pgfscope}%
\pgfpathrectangle{\pgfqpoint{0.800000in}{0.528000in}}{\pgfqpoint{4.960000in}{3.696000in}}%
\pgfusepath{clip}%
\pgfsetbuttcap%
\pgfsetroundjoin%
\definecolor{currentfill}{rgb}{0.000000,0.000000,0.000000}%
\pgfsetfillcolor{currentfill}%
\pgfsetlinewidth{1.003750pt}%
\definecolor{currentstroke}{rgb}{0.000000,0.000000,0.000000}%
\pgfsetstrokecolor{currentstroke}%
\pgfsetdash{}{0pt}%
\pgfpathmoveto{\pgfqpoint{2.518786in}{1.771040in}}%
\pgfpathcurveto{\pgfqpoint{2.529836in}{1.771040in}}{\pgfqpoint{2.540435in}{1.775431in}}{\pgfqpoint{2.548249in}{1.783244in}}%
\pgfpathcurveto{\pgfqpoint{2.556062in}{1.791058in}}{\pgfqpoint{2.560452in}{1.801657in}}{\pgfqpoint{2.560452in}{1.812707in}}%
\pgfpathcurveto{\pgfqpoint{2.560452in}{1.823757in}}{\pgfqpoint{2.556062in}{1.834356in}}{\pgfqpoint{2.548249in}{1.842170in}}%
\pgfpathcurveto{\pgfqpoint{2.540435in}{1.849983in}}{\pgfqpoint{2.529836in}{1.854374in}}{\pgfqpoint{2.518786in}{1.854374in}}%
\pgfpathcurveto{\pgfqpoint{2.507736in}{1.854374in}}{\pgfqpoint{2.497137in}{1.849983in}}{\pgfqpoint{2.489323in}{1.842170in}}%
\pgfpathcurveto{\pgfqpoint{2.481509in}{1.834356in}}{\pgfqpoint{2.477119in}{1.823757in}}{\pgfqpoint{2.477119in}{1.812707in}}%
\pgfpathcurveto{\pgfqpoint{2.477119in}{1.801657in}}{\pgfqpoint{2.481509in}{1.791058in}}{\pgfqpoint{2.489323in}{1.783244in}}%
\pgfpathcurveto{\pgfqpoint{2.497137in}{1.775431in}}{\pgfqpoint{2.507736in}{1.771040in}}{\pgfqpoint{2.518786in}{1.771040in}}%
\pgfpathclose%
\pgfusepath{stroke,fill}%
\end{pgfscope}%
\begin{pgfscope}%
\pgfpathrectangle{\pgfqpoint{0.800000in}{0.528000in}}{\pgfqpoint{4.960000in}{3.696000in}}%
\pgfusepath{clip}%
\pgfsetbuttcap%
\pgfsetroundjoin%
\definecolor{currentfill}{rgb}{0.000000,0.000000,0.000000}%
\pgfsetfillcolor{currentfill}%
\pgfsetlinewidth{1.003750pt}%
\definecolor{currentstroke}{rgb}{0.000000,0.000000,0.000000}%
\pgfsetstrokecolor{currentstroke}%
\pgfsetdash{}{0pt}%
\pgfpathmoveto{\pgfqpoint{2.518786in}{1.771040in}}%
\pgfpathcurveto{\pgfqpoint{2.529836in}{1.771040in}}{\pgfqpoint{2.540435in}{1.775431in}}{\pgfqpoint{2.548249in}{1.783244in}}%
\pgfpathcurveto{\pgfqpoint{2.556062in}{1.791058in}}{\pgfqpoint{2.560452in}{1.801657in}}{\pgfqpoint{2.560452in}{1.812707in}}%
\pgfpathcurveto{\pgfqpoint{2.560452in}{1.823757in}}{\pgfqpoint{2.556062in}{1.834356in}}{\pgfqpoint{2.548249in}{1.842170in}}%
\pgfpathcurveto{\pgfqpoint{2.540435in}{1.849983in}}{\pgfqpoint{2.529836in}{1.854374in}}{\pgfqpoint{2.518786in}{1.854374in}}%
\pgfpathcurveto{\pgfqpoint{2.507736in}{1.854374in}}{\pgfqpoint{2.497137in}{1.849983in}}{\pgfqpoint{2.489323in}{1.842170in}}%
\pgfpathcurveto{\pgfqpoint{2.481509in}{1.834356in}}{\pgfqpoint{2.477119in}{1.823757in}}{\pgfqpoint{2.477119in}{1.812707in}}%
\pgfpathcurveto{\pgfqpoint{2.477119in}{1.801657in}}{\pgfqpoint{2.481509in}{1.791058in}}{\pgfqpoint{2.489323in}{1.783244in}}%
\pgfpathcurveto{\pgfqpoint{2.497137in}{1.775431in}}{\pgfqpoint{2.507736in}{1.771040in}}{\pgfqpoint{2.518786in}{1.771040in}}%
\pgfpathclose%
\pgfusepath{stroke,fill}%
\end{pgfscope}%
\begin{pgfscope}%
\pgfpathrectangle{\pgfqpoint{0.800000in}{0.528000in}}{\pgfqpoint{4.960000in}{3.696000in}}%
\pgfusepath{clip}%
\pgfsetbuttcap%
\pgfsetroundjoin%
\definecolor{currentfill}{rgb}{0.000000,0.000000,0.000000}%
\pgfsetfillcolor{currentfill}%
\pgfsetlinewidth{1.003750pt}%
\definecolor{currentstroke}{rgb}{0.000000,0.000000,0.000000}%
\pgfsetstrokecolor{currentstroke}%
\pgfsetdash{}{0pt}%
\pgfpathmoveto{\pgfqpoint{2.518786in}{1.771040in}}%
\pgfpathcurveto{\pgfqpoint{2.529836in}{1.771040in}}{\pgfqpoint{2.540435in}{1.775431in}}{\pgfqpoint{2.548249in}{1.783244in}}%
\pgfpathcurveto{\pgfqpoint{2.556062in}{1.791058in}}{\pgfqpoint{2.560452in}{1.801657in}}{\pgfqpoint{2.560452in}{1.812707in}}%
\pgfpathcurveto{\pgfqpoint{2.560452in}{1.823757in}}{\pgfqpoint{2.556062in}{1.834356in}}{\pgfqpoint{2.548249in}{1.842170in}}%
\pgfpathcurveto{\pgfqpoint{2.540435in}{1.849983in}}{\pgfqpoint{2.529836in}{1.854374in}}{\pgfqpoint{2.518786in}{1.854374in}}%
\pgfpathcurveto{\pgfqpoint{2.507736in}{1.854374in}}{\pgfqpoint{2.497137in}{1.849983in}}{\pgfqpoint{2.489323in}{1.842170in}}%
\pgfpathcurveto{\pgfqpoint{2.481509in}{1.834356in}}{\pgfqpoint{2.477119in}{1.823757in}}{\pgfqpoint{2.477119in}{1.812707in}}%
\pgfpathcurveto{\pgfqpoint{2.477119in}{1.801657in}}{\pgfqpoint{2.481509in}{1.791058in}}{\pgfqpoint{2.489323in}{1.783244in}}%
\pgfpathcurveto{\pgfqpoint{2.497137in}{1.775431in}}{\pgfqpoint{2.507736in}{1.771040in}}{\pgfqpoint{2.518786in}{1.771040in}}%
\pgfpathclose%
\pgfusepath{stroke,fill}%
\end{pgfscope}%
\begin{pgfscope}%
\pgfpathrectangle{\pgfqpoint{0.800000in}{0.528000in}}{\pgfqpoint{4.960000in}{3.696000in}}%
\pgfusepath{clip}%
\pgfsetbuttcap%
\pgfsetroundjoin%
\definecolor{currentfill}{rgb}{0.000000,0.000000,0.000000}%
\pgfsetfillcolor{currentfill}%
\pgfsetlinewidth{1.003750pt}%
\definecolor{currentstroke}{rgb}{0.000000,0.000000,0.000000}%
\pgfsetstrokecolor{currentstroke}%
\pgfsetdash{}{0pt}%
\pgfpathmoveto{\pgfqpoint{2.518786in}{1.771040in}}%
\pgfpathcurveto{\pgfqpoint{2.529836in}{1.771040in}}{\pgfqpoint{2.540435in}{1.775431in}}{\pgfqpoint{2.548249in}{1.783244in}}%
\pgfpathcurveto{\pgfqpoint{2.556062in}{1.791058in}}{\pgfqpoint{2.560452in}{1.801657in}}{\pgfqpoint{2.560452in}{1.812707in}}%
\pgfpathcurveto{\pgfqpoint{2.560452in}{1.823757in}}{\pgfqpoint{2.556062in}{1.834356in}}{\pgfqpoint{2.548249in}{1.842170in}}%
\pgfpathcurveto{\pgfqpoint{2.540435in}{1.849983in}}{\pgfqpoint{2.529836in}{1.854374in}}{\pgfqpoint{2.518786in}{1.854374in}}%
\pgfpathcurveto{\pgfqpoint{2.507736in}{1.854374in}}{\pgfqpoint{2.497137in}{1.849983in}}{\pgfqpoint{2.489323in}{1.842170in}}%
\pgfpathcurveto{\pgfqpoint{2.481509in}{1.834356in}}{\pgfqpoint{2.477119in}{1.823757in}}{\pgfqpoint{2.477119in}{1.812707in}}%
\pgfpathcurveto{\pgfqpoint{2.477119in}{1.801657in}}{\pgfqpoint{2.481509in}{1.791058in}}{\pgfqpoint{2.489323in}{1.783244in}}%
\pgfpathcurveto{\pgfqpoint{2.497137in}{1.775431in}}{\pgfqpoint{2.507736in}{1.771040in}}{\pgfqpoint{2.518786in}{1.771040in}}%
\pgfpathclose%
\pgfusepath{stroke,fill}%
\end{pgfscope}%
\begin{pgfscope}%
\pgfpathrectangle{\pgfqpoint{0.800000in}{0.528000in}}{\pgfqpoint{4.960000in}{3.696000in}}%
\pgfusepath{clip}%
\pgfsetbuttcap%
\pgfsetroundjoin%
\definecolor{currentfill}{rgb}{0.000000,0.000000,0.000000}%
\pgfsetfillcolor{currentfill}%
\pgfsetlinewidth{1.003750pt}%
\definecolor{currentstroke}{rgb}{0.000000,0.000000,0.000000}%
\pgfsetstrokecolor{currentstroke}%
\pgfsetdash{}{0pt}%
\pgfpathmoveto{\pgfqpoint{2.518786in}{1.771040in}}%
\pgfpathcurveto{\pgfqpoint{2.529836in}{1.771040in}}{\pgfqpoint{2.540435in}{1.775431in}}{\pgfqpoint{2.548249in}{1.783244in}}%
\pgfpathcurveto{\pgfqpoint{2.556062in}{1.791058in}}{\pgfqpoint{2.560452in}{1.801657in}}{\pgfqpoint{2.560452in}{1.812707in}}%
\pgfpathcurveto{\pgfqpoint{2.560452in}{1.823757in}}{\pgfqpoint{2.556062in}{1.834356in}}{\pgfqpoint{2.548249in}{1.842170in}}%
\pgfpathcurveto{\pgfqpoint{2.540435in}{1.849983in}}{\pgfqpoint{2.529836in}{1.854374in}}{\pgfqpoint{2.518786in}{1.854374in}}%
\pgfpathcurveto{\pgfqpoint{2.507736in}{1.854374in}}{\pgfqpoint{2.497137in}{1.849983in}}{\pgfqpoint{2.489323in}{1.842170in}}%
\pgfpathcurveto{\pgfqpoint{2.481509in}{1.834356in}}{\pgfqpoint{2.477119in}{1.823757in}}{\pgfqpoint{2.477119in}{1.812707in}}%
\pgfpathcurveto{\pgfqpoint{2.477119in}{1.801657in}}{\pgfqpoint{2.481509in}{1.791058in}}{\pgfqpoint{2.489323in}{1.783244in}}%
\pgfpathcurveto{\pgfqpoint{2.497137in}{1.775431in}}{\pgfqpoint{2.507736in}{1.771040in}}{\pgfqpoint{2.518786in}{1.771040in}}%
\pgfpathclose%
\pgfusepath{stroke,fill}%
\end{pgfscope}%
\begin{pgfscope}%
\pgfpathrectangle{\pgfqpoint{0.800000in}{0.528000in}}{\pgfqpoint{4.960000in}{3.696000in}}%
\pgfusepath{clip}%
\pgfsetbuttcap%
\pgfsetroundjoin%
\definecolor{currentfill}{rgb}{0.000000,0.000000,0.000000}%
\pgfsetfillcolor{currentfill}%
\pgfsetlinewidth{1.003750pt}%
\definecolor{currentstroke}{rgb}{0.000000,0.000000,0.000000}%
\pgfsetstrokecolor{currentstroke}%
\pgfsetdash{}{0pt}%
\pgfpathmoveto{\pgfqpoint{2.518786in}{1.771040in}}%
\pgfpathcurveto{\pgfqpoint{2.529836in}{1.771040in}}{\pgfqpoint{2.540435in}{1.775431in}}{\pgfqpoint{2.548249in}{1.783244in}}%
\pgfpathcurveto{\pgfqpoint{2.556062in}{1.791058in}}{\pgfqpoint{2.560452in}{1.801657in}}{\pgfqpoint{2.560452in}{1.812707in}}%
\pgfpathcurveto{\pgfqpoint{2.560452in}{1.823757in}}{\pgfqpoint{2.556062in}{1.834356in}}{\pgfqpoint{2.548249in}{1.842170in}}%
\pgfpathcurveto{\pgfqpoint{2.540435in}{1.849983in}}{\pgfqpoint{2.529836in}{1.854374in}}{\pgfqpoint{2.518786in}{1.854374in}}%
\pgfpathcurveto{\pgfqpoint{2.507736in}{1.854374in}}{\pgfqpoint{2.497137in}{1.849983in}}{\pgfqpoint{2.489323in}{1.842170in}}%
\pgfpathcurveto{\pgfqpoint{2.481509in}{1.834356in}}{\pgfqpoint{2.477119in}{1.823757in}}{\pgfqpoint{2.477119in}{1.812707in}}%
\pgfpathcurveto{\pgfqpoint{2.477119in}{1.801657in}}{\pgfqpoint{2.481509in}{1.791058in}}{\pgfqpoint{2.489323in}{1.783244in}}%
\pgfpathcurveto{\pgfqpoint{2.497137in}{1.775431in}}{\pgfqpoint{2.507736in}{1.771040in}}{\pgfqpoint{2.518786in}{1.771040in}}%
\pgfpathclose%
\pgfusepath{stroke,fill}%
\end{pgfscope}%
\begin{pgfscope}%
\pgfpathrectangle{\pgfqpoint{0.800000in}{0.528000in}}{\pgfqpoint{4.960000in}{3.696000in}}%
\pgfusepath{clip}%
\pgfsetbuttcap%
\pgfsetroundjoin%
\definecolor{currentfill}{rgb}{0.000000,0.000000,0.000000}%
\pgfsetfillcolor{currentfill}%
\pgfsetlinewidth{1.003750pt}%
\definecolor{currentstroke}{rgb}{0.000000,0.000000,0.000000}%
\pgfsetstrokecolor{currentstroke}%
\pgfsetdash{}{0pt}%
\pgfpathmoveto{\pgfqpoint{2.518786in}{1.771040in}}%
\pgfpathcurveto{\pgfqpoint{2.529836in}{1.771040in}}{\pgfqpoint{2.540435in}{1.775431in}}{\pgfqpoint{2.548249in}{1.783244in}}%
\pgfpathcurveto{\pgfqpoint{2.556062in}{1.791058in}}{\pgfqpoint{2.560452in}{1.801657in}}{\pgfqpoint{2.560452in}{1.812707in}}%
\pgfpathcurveto{\pgfqpoint{2.560452in}{1.823757in}}{\pgfqpoint{2.556062in}{1.834356in}}{\pgfqpoint{2.548249in}{1.842170in}}%
\pgfpathcurveto{\pgfqpoint{2.540435in}{1.849983in}}{\pgfqpoint{2.529836in}{1.854374in}}{\pgfqpoint{2.518786in}{1.854374in}}%
\pgfpathcurveto{\pgfqpoint{2.507736in}{1.854374in}}{\pgfqpoint{2.497137in}{1.849983in}}{\pgfqpoint{2.489323in}{1.842170in}}%
\pgfpathcurveto{\pgfqpoint{2.481509in}{1.834356in}}{\pgfqpoint{2.477119in}{1.823757in}}{\pgfqpoint{2.477119in}{1.812707in}}%
\pgfpathcurveto{\pgfqpoint{2.477119in}{1.801657in}}{\pgfqpoint{2.481509in}{1.791058in}}{\pgfqpoint{2.489323in}{1.783244in}}%
\pgfpathcurveto{\pgfqpoint{2.497137in}{1.775431in}}{\pgfqpoint{2.507736in}{1.771040in}}{\pgfqpoint{2.518786in}{1.771040in}}%
\pgfpathclose%
\pgfusepath{stroke,fill}%
\end{pgfscope}%
\begin{pgfscope}%
\pgfpathrectangle{\pgfqpoint{0.800000in}{0.528000in}}{\pgfqpoint{4.960000in}{3.696000in}}%
\pgfusepath{clip}%
\pgfsetbuttcap%
\pgfsetroundjoin%
\definecolor{currentfill}{rgb}{0.000000,0.000000,0.000000}%
\pgfsetfillcolor{currentfill}%
\pgfsetlinewidth{1.003750pt}%
\definecolor{currentstroke}{rgb}{0.000000,0.000000,0.000000}%
\pgfsetstrokecolor{currentstroke}%
\pgfsetdash{}{0pt}%
\pgfpathmoveto{\pgfqpoint{2.518786in}{1.771040in}}%
\pgfpathcurveto{\pgfqpoint{2.529836in}{1.771040in}}{\pgfqpoint{2.540435in}{1.775431in}}{\pgfqpoint{2.548249in}{1.783244in}}%
\pgfpathcurveto{\pgfqpoint{2.556062in}{1.791058in}}{\pgfqpoint{2.560452in}{1.801657in}}{\pgfqpoint{2.560452in}{1.812707in}}%
\pgfpathcurveto{\pgfqpoint{2.560452in}{1.823757in}}{\pgfqpoint{2.556062in}{1.834356in}}{\pgfqpoint{2.548249in}{1.842170in}}%
\pgfpathcurveto{\pgfqpoint{2.540435in}{1.849983in}}{\pgfqpoint{2.529836in}{1.854374in}}{\pgfqpoint{2.518786in}{1.854374in}}%
\pgfpathcurveto{\pgfqpoint{2.507736in}{1.854374in}}{\pgfqpoint{2.497137in}{1.849983in}}{\pgfqpoint{2.489323in}{1.842170in}}%
\pgfpathcurveto{\pgfqpoint{2.481509in}{1.834356in}}{\pgfqpoint{2.477119in}{1.823757in}}{\pgfqpoint{2.477119in}{1.812707in}}%
\pgfpathcurveto{\pgfqpoint{2.477119in}{1.801657in}}{\pgfqpoint{2.481509in}{1.791058in}}{\pgfqpoint{2.489323in}{1.783244in}}%
\pgfpathcurveto{\pgfqpoint{2.497137in}{1.775431in}}{\pgfqpoint{2.507736in}{1.771040in}}{\pgfqpoint{2.518786in}{1.771040in}}%
\pgfpathclose%
\pgfusepath{stroke,fill}%
\end{pgfscope}%
\begin{pgfscope}%
\pgfpathrectangle{\pgfqpoint{0.800000in}{0.528000in}}{\pgfqpoint{4.960000in}{3.696000in}}%
\pgfusepath{clip}%
\pgfsetbuttcap%
\pgfsetroundjoin%
\definecolor{currentfill}{rgb}{0.000000,0.000000,0.000000}%
\pgfsetfillcolor{currentfill}%
\pgfsetlinewidth{1.003750pt}%
\definecolor{currentstroke}{rgb}{0.000000,0.000000,0.000000}%
\pgfsetstrokecolor{currentstroke}%
\pgfsetdash{}{0pt}%
\pgfpathmoveto{\pgfqpoint{2.518786in}{1.771040in}}%
\pgfpathcurveto{\pgfqpoint{2.529836in}{1.771040in}}{\pgfqpoint{2.540435in}{1.775431in}}{\pgfqpoint{2.548249in}{1.783244in}}%
\pgfpathcurveto{\pgfqpoint{2.556062in}{1.791058in}}{\pgfqpoint{2.560452in}{1.801657in}}{\pgfqpoint{2.560452in}{1.812707in}}%
\pgfpathcurveto{\pgfqpoint{2.560452in}{1.823757in}}{\pgfqpoint{2.556062in}{1.834356in}}{\pgfqpoint{2.548249in}{1.842170in}}%
\pgfpathcurveto{\pgfqpoint{2.540435in}{1.849983in}}{\pgfqpoint{2.529836in}{1.854374in}}{\pgfqpoint{2.518786in}{1.854374in}}%
\pgfpathcurveto{\pgfqpoint{2.507736in}{1.854374in}}{\pgfqpoint{2.497137in}{1.849983in}}{\pgfqpoint{2.489323in}{1.842170in}}%
\pgfpathcurveto{\pgfqpoint{2.481509in}{1.834356in}}{\pgfqpoint{2.477119in}{1.823757in}}{\pgfqpoint{2.477119in}{1.812707in}}%
\pgfpathcurveto{\pgfqpoint{2.477119in}{1.801657in}}{\pgfqpoint{2.481509in}{1.791058in}}{\pgfqpoint{2.489323in}{1.783244in}}%
\pgfpathcurveto{\pgfqpoint{2.497137in}{1.775431in}}{\pgfqpoint{2.507736in}{1.771040in}}{\pgfqpoint{2.518786in}{1.771040in}}%
\pgfpathclose%
\pgfusepath{stroke,fill}%
\end{pgfscope}%
\begin{pgfscope}%
\pgfpathrectangle{\pgfqpoint{0.800000in}{0.528000in}}{\pgfqpoint{4.960000in}{3.696000in}}%
\pgfusepath{clip}%
\pgfsetbuttcap%
\pgfsetroundjoin%
\definecolor{currentfill}{rgb}{0.000000,0.000000,0.000000}%
\pgfsetfillcolor{currentfill}%
\pgfsetlinewidth{1.003750pt}%
\definecolor{currentstroke}{rgb}{0.000000,0.000000,0.000000}%
\pgfsetstrokecolor{currentstroke}%
\pgfsetdash{}{0pt}%
\pgfpathmoveto{\pgfqpoint{2.518786in}{1.771040in}}%
\pgfpathcurveto{\pgfqpoint{2.529836in}{1.771040in}}{\pgfqpoint{2.540435in}{1.775431in}}{\pgfqpoint{2.548249in}{1.783244in}}%
\pgfpathcurveto{\pgfqpoint{2.556062in}{1.791058in}}{\pgfqpoint{2.560452in}{1.801657in}}{\pgfqpoint{2.560452in}{1.812707in}}%
\pgfpathcurveto{\pgfqpoint{2.560452in}{1.823757in}}{\pgfqpoint{2.556062in}{1.834356in}}{\pgfqpoint{2.548249in}{1.842170in}}%
\pgfpathcurveto{\pgfqpoint{2.540435in}{1.849983in}}{\pgfqpoint{2.529836in}{1.854374in}}{\pgfqpoint{2.518786in}{1.854374in}}%
\pgfpathcurveto{\pgfqpoint{2.507736in}{1.854374in}}{\pgfqpoint{2.497137in}{1.849983in}}{\pgfqpoint{2.489323in}{1.842170in}}%
\pgfpathcurveto{\pgfqpoint{2.481509in}{1.834356in}}{\pgfqpoint{2.477119in}{1.823757in}}{\pgfqpoint{2.477119in}{1.812707in}}%
\pgfpathcurveto{\pgfqpoint{2.477119in}{1.801657in}}{\pgfqpoint{2.481509in}{1.791058in}}{\pgfqpoint{2.489323in}{1.783244in}}%
\pgfpathcurveto{\pgfqpoint{2.497137in}{1.775431in}}{\pgfqpoint{2.507736in}{1.771040in}}{\pgfqpoint{2.518786in}{1.771040in}}%
\pgfpathclose%
\pgfusepath{stroke,fill}%
\end{pgfscope}%
\begin{pgfscope}%
\pgfpathrectangle{\pgfqpoint{0.800000in}{0.528000in}}{\pgfqpoint{4.960000in}{3.696000in}}%
\pgfusepath{clip}%
\pgfsetbuttcap%
\pgfsetroundjoin%
\definecolor{currentfill}{rgb}{0.000000,0.000000,0.000000}%
\pgfsetfillcolor{currentfill}%
\pgfsetlinewidth{1.003750pt}%
\definecolor{currentstroke}{rgb}{0.000000,0.000000,0.000000}%
\pgfsetstrokecolor{currentstroke}%
\pgfsetdash{}{0pt}%
\pgfpathmoveto{\pgfqpoint{2.518786in}{1.771040in}}%
\pgfpathcurveto{\pgfqpoint{2.529836in}{1.771040in}}{\pgfqpoint{2.540435in}{1.775431in}}{\pgfqpoint{2.548249in}{1.783244in}}%
\pgfpathcurveto{\pgfqpoint{2.556062in}{1.791058in}}{\pgfqpoint{2.560452in}{1.801657in}}{\pgfqpoint{2.560452in}{1.812707in}}%
\pgfpathcurveto{\pgfqpoint{2.560452in}{1.823757in}}{\pgfqpoint{2.556062in}{1.834356in}}{\pgfqpoint{2.548249in}{1.842170in}}%
\pgfpathcurveto{\pgfqpoint{2.540435in}{1.849983in}}{\pgfqpoint{2.529836in}{1.854374in}}{\pgfqpoint{2.518786in}{1.854374in}}%
\pgfpathcurveto{\pgfqpoint{2.507736in}{1.854374in}}{\pgfqpoint{2.497137in}{1.849983in}}{\pgfqpoint{2.489323in}{1.842170in}}%
\pgfpathcurveto{\pgfqpoint{2.481509in}{1.834356in}}{\pgfqpoint{2.477119in}{1.823757in}}{\pgfqpoint{2.477119in}{1.812707in}}%
\pgfpathcurveto{\pgfqpoint{2.477119in}{1.801657in}}{\pgfqpoint{2.481509in}{1.791058in}}{\pgfqpoint{2.489323in}{1.783244in}}%
\pgfpathcurveto{\pgfqpoint{2.497137in}{1.775431in}}{\pgfqpoint{2.507736in}{1.771040in}}{\pgfqpoint{2.518786in}{1.771040in}}%
\pgfpathclose%
\pgfusepath{stroke,fill}%
\end{pgfscope}%
\begin{pgfscope}%
\pgfpathrectangle{\pgfqpoint{0.800000in}{0.528000in}}{\pgfqpoint{4.960000in}{3.696000in}}%
\pgfusepath{clip}%
\pgfsetbuttcap%
\pgfsetroundjoin%
\definecolor{currentfill}{rgb}{0.000000,0.000000,0.000000}%
\pgfsetfillcolor{currentfill}%
\pgfsetlinewidth{1.003750pt}%
\definecolor{currentstroke}{rgb}{0.000000,0.000000,0.000000}%
\pgfsetstrokecolor{currentstroke}%
\pgfsetdash{}{0pt}%
\pgfpathmoveto{\pgfqpoint{2.518786in}{1.771040in}}%
\pgfpathcurveto{\pgfqpoint{2.529836in}{1.771040in}}{\pgfqpoint{2.540435in}{1.775431in}}{\pgfqpoint{2.548249in}{1.783244in}}%
\pgfpathcurveto{\pgfqpoint{2.556062in}{1.791058in}}{\pgfqpoint{2.560452in}{1.801657in}}{\pgfqpoint{2.560452in}{1.812707in}}%
\pgfpathcurveto{\pgfqpoint{2.560452in}{1.823757in}}{\pgfqpoint{2.556062in}{1.834356in}}{\pgfqpoint{2.548249in}{1.842170in}}%
\pgfpathcurveto{\pgfqpoint{2.540435in}{1.849983in}}{\pgfqpoint{2.529836in}{1.854374in}}{\pgfqpoint{2.518786in}{1.854374in}}%
\pgfpathcurveto{\pgfqpoint{2.507736in}{1.854374in}}{\pgfqpoint{2.497137in}{1.849983in}}{\pgfqpoint{2.489323in}{1.842170in}}%
\pgfpathcurveto{\pgfqpoint{2.481509in}{1.834356in}}{\pgfqpoint{2.477119in}{1.823757in}}{\pgfqpoint{2.477119in}{1.812707in}}%
\pgfpathcurveto{\pgfqpoint{2.477119in}{1.801657in}}{\pgfqpoint{2.481509in}{1.791058in}}{\pgfqpoint{2.489323in}{1.783244in}}%
\pgfpathcurveto{\pgfqpoint{2.497137in}{1.775431in}}{\pgfqpoint{2.507736in}{1.771040in}}{\pgfqpoint{2.518786in}{1.771040in}}%
\pgfpathclose%
\pgfusepath{stroke,fill}%
\end{pgfscope}%
\begin{pgfscope}%
\pgfpathrectangle{\pgfqpoint{0.800000in}{0.528000in}}{\pgfqpoint{4.960000in}{3.696000in}}%
\pgfusepath{clip}%
\pgfsetbuttcap%
\pgfsetroundjoin%
\definecolor{currentfill}{rgb}{0.000000,0.000000,0.000000}%
\pgfsetfillcolor{currentfill}%
\pgfsetlinewidth{1.003750pt}%
\definecolor{currentstroke}{rgb}{0.000000,0.000000,0.000000}%
\pgfsetstrokecolor{currentstroke}%
\pgfsetdash{}{0pt}%
\pgfpathmoveto{\pgfqpoint{2.518786in}{1.771040in}}%
\pgfpathcurveto{\pgfqpoint{2.529836in}{1.771040in}}{\pgfqpoint{2.540435in}{1.775431in}}{\pgfqpoint{2.548249in}{1.783244in}}%
\pgfpathcurveto{\pgfqpoint{2.556062in}{1.791058in}}{\pgfqpoint{2.560452in}{1.801657in}}{\pgfqpoint{2.560452in}{1.812707in}}%
\pgfpathcurveto{\pgfqpoint{2.560452in}{1.823757in}}{\pgfqpoint{2.556062in}{1.834356in}}{\pgfqpoint{2.548249in}{1.842170in}}%
\pgfpathcurveto{\pgfqpoint{2.540435in}{1.849983in}}{\pgfqpoint{2.529836in}{1.854374in}}{\pgfqpoint{2.518786in}{1.854374in}}%
\pgfpathcurveto{\pgfqpoint{2.507736in}{1.854374in}}{\pgfqpoint{2.497137in}{1.849983in}}{\pgfqpoint{2.489323in}{1.842170in}}%
\pgfpathcurveto{\pgfqpoint{2.481509in}{1.834356in}}{\pgfqpoint{2.477119in}{1.823757in}}{\pgfqpoint{2.477119in}{1.812707in}}%
\pgfpathcurveto{\pgfqpoint{2.477119in}{1.801657in}}{\pgfqpoint{2.481509in}{1.791058in}}{\pgfqpoint{2.489323in}{1.783244in}}%
\pgfpathcurveto{\pgfqpoint{2.497137in}{1.775431in}}{\pgfqpoint{2.507736in}{1.771040in}}{\pgfqpoint{2.518786in}{1.771040in}}%
\pgfpathclose%
\pgfusepath{stroke,fill}%
\end{pgfscope}%
\begin{pgfscope}%
\pgfpathrectangle{\pgfqpoint{0.800000in}{0.528000in}}{\pgfqpoint{4.960000in}{3.696000in}}%
\pgfusepath{clip}%
\pgfsetbuttcap%
\pgfsetroundjoin%
\definecolor{currentfill}{rgb}{0.000000,0.000000,0.000000}%
\pgfsetfillcolor{currentfill}%
\pgfsetlinewidth{1.003750pt}%
\definecolor{currentstroke}{rgb}{0.000000,0.000000,0.000000}%
\pgfsetstrokecolor{currentstroke}%
\pgfsetdash{}{0pt}%
\pgfpathmoveto{\pgfqpoint{2.518786in}{1.771040in}}%
\pgfpathcurveto{\pgfqpoint{2.529836in}{1.771040in}}{\pgfqpoint{2.540435in}{1.775431in}}{\pgfqpoint{2.548249in}{1.783244in}}%
\pgfpathcurveto{\pgfqpoint{2.556062in}{1.791058in}}{\pgfqpoint{2.560452in}{1.801657in}}{\pgfqpoint{2.560452in}{1.812707in}}%
\pgfpathcurveto{\pgfqpoint{2.560452in}{1.823757in}}{\pgfqpoint{2.556062in}{1.834356in}}{\pgfqpoint{2.548249in}{1.842170in}}%
\pgfpathcurveto{\pgfqpoint{2.540435in}{1.849983in}}{\pgfqpoint{2.529836in}{1.854374in}}{\pgfqpoint{2.518786in}{1.854374in}}%
\pgfpathcurveto{\pgfqpoint{2.507736in}{1.854374in}}{\pgfqpoint{2.497137in}{1.849983in}}{\pgfqpoint{2.489323in}{1.842170in}}%
\pgfpathcurveto{\pgfqpoint{2.481509in}{1.834356in}}{\pgfqpoint{2.477119in}{1.823757in}}{\pgfqpoint{2.477119in}{1.812707in}}%
\pgfpathcurveto{\pgfqpoint{2.477119in}{1.801657in}}{\pgfqpoint{2.481509in}{1.791058in}}{\pgfqpoint{2.489323in}{1.783244in}}%
\pgfpathcurveto{\pgfqpoint{2.497137in}{1.775431in}}{\pgfqpoint{2.507736in}{1.771040in}}{\pgfqpoint{2.518786in}{1.771040in}}%
\pgfpathclose%
\pgfusepath{stroke,fill}%
\end{pgfscope}%
\begin{pgfscope}%
\pgfpathrectangle{\pgfqpoint{0.800000in}{0.528000in}}{\pgfqpoint{4.960000in}{3.696000in}}%
\pgfusepath{clip}%
\pgfsetbuttcap%
\pgfsetroundjoin%
\definecolor{currentfill}{rgb}{0.000000,0.000000,0.000000}%
\pgfsetfillcolor{currentfill}%
\pgfsetlinewidth{1.003750pt}%
\definecolor{currentstroke}{rgb}{0.000000,0.000000,0.000000}%
\pgfsetstrokecolor{currentstroke}%
\pgfsetdash{}{0pt}%
\pgfpathmoveto{\pgfqpoint{2.518786in}{1.771040in}}%
\pgfpathcurveto{\pgfqpoint{2.529836in}{1.771040in}}{\pgfqpoint{2.540435in}{1.775431in}}{\pgfqpoint{2.548249in}{1.783244in}}%
\pgfpathcurveto{\pgfqpoint{2.556062in}{1.791058in}}{\pgfqpoint{2.560452in}{1.801657in}}{\pgfqpoint{2.560452in}{1.812707in}}%
\pgfpathcurveto{\pgfqpoint{2.560452in}{1.823757in}}{\pgfqpoint{2.556062in}{1.834356in}}{\pgfqpoint{2.548249in}{1.842170in}}%
\pgfpathcurveto{\pgfqpoint{2.540435in}{1.849983in}}{\pgfqpoint{2.529836in}{1.854374in}}{\pgfqpoint{2.518786in}{1.854374in}}%
\pgfpathcurveto{\pgfqpoint{2.507736in}{1.854374in}}{\pgfqpoint{2.497137in}{1.849983in}}{\pgfqpoint{2.489323in}{1.842170in}}%
\pgfpathcurveto{\pgfqpoint{2.481509in}{1.834356in}}{\pgfqpoint{2.477119in}{1.823757in}}{\pgfqpoint{2.477119in}{1.812707in}}%
\pgfpathcurveto{\pgfqpoint{2.477119in}{1.801657in}}{\pgfqpoint{2.481509in}{1.791058in}}{\pgfqpoint{2.489323in}{1.783244in}}%
\pgfpathcurveto{\pgfqpoint{2.497137in}{1.775431in}}{\pgfqpoint{2.507736in}{1.771040in}}{\pgfqpoint{2.518786in}{1.771040in}}%
\pgfpathclose%
\pgfusepath{stroke,fill}%
\end{pgfscope}%
\begin{pgfscope}%
\pgfpathrectangle{\pgfqpoint{0.800000in}{0.528000in}}{\pgfqpoint{4.960000in}{3.696000in}}%
\pgfusepath{clip}%
\pgfsetbuttcap%
\pgfsetroundjoin%
\definecolor{currentfill}{rgb}{0.000000,0.000000,0.000000}%
\pgfsetfillcolor{currentfill}%
\pgfsetlinewidth{1.003750pt}%
\definecolor{currentstroke}{rgb}{0.000000,0.000000,0.000000}%
\pgfsetstrokecolor{currentstroke}%
\pgfsetdash{}{0pt}%
\pgfpathmoveto{\pgfqpoint{2.518786in}{1.771040in}}%
\pgfpathcurveto{\pgfqpoint{2.529836in}{1.771040in}}{\pgfqpoint{2.540435in}{1.775431in}}{\pgfqpoint{2.548249in}{1.783244in}}%
\pgfpathcurveto{\pgfqpoint{2.556062in}{1.791058in}}{\pgfqpoint{2.560452in}{1.801657in}}{\pgfqpoint{2.560452in}{1.812707in}}%
\pgfpathcurveto{\pgfqpoint{2.560452in}{1.823757in}}{\pgfqpoint{2.556062in}{1.834356in}}{\pgfqpoint{2.548249in}{1.842170in}}%
\pgfpathcurveto{\pgfqpoint{2.540435in}{1.849983in}}{\pgfqpoint{2.529836in}{1.854374in}}{\pgfqpoint{2.518786in}{1.854374in}}%
\pgfpathcurveto{\pgfqpoint{2.507736in}{1.854374in}}{\pgfqpoint{2.497137in}{1.849983in}}{\pgfqpoint{2.489323in}{1.842170in}}%
\pgfpathcurveto{\pgfqpoint{2.481509in}{1.834356in}}{\pgfqpoint{2.477119in}{1.823757in}}{\pgfqpoint{2.477119in}{1.812707in}}%
\pgfpathcurveto{\pgfqpoint{2.477119in}{1.801657in}}{\pgfqpoint{2.481509in}{1.791058in}}{\pgfqpoint{2.489323in}{1.783244in}}%
\pgfpathcurveto{\pgfqpoint{2.497137in}{1.775431in}}{\pgfqpoint{2.507736in}{1.771040in}}{\pgfqpoint{2.518786in}{1.771040in}}%
\pgfpathclose%
\pgfusepath{stroke,fill}%
\end{pgfscope}%
\begin{pgfscope}%
\pgfpathrectangle{\pgfqpoint{0.800000in}{0.528000in}}{\pgfqpoint{4.960000in}{3.696000in}}%
\pgfusepath{clip}%
\pgfsetbuttcap%
\pgfsetroundjoin%
\definecolor{currentfill}{rgb}{0.000000,0.000000,0.000000}%
\pgfsetfillcolor{currentfill}%
\pgfsetlinewidth{1.003750pt}%
\definecolor{currentstroke}{rgb}{0.000000,0.000000,0.000000}%
\pgfsetstrokecolor{currentstroke}%
\pgfsetdash{}{0pt}%
\pgfpathmoveto{\pgfqpoint{2.518786in}{1.771040in}}%
\pgfpathcurveto{\pgfqpoint{2.529836in}{1.771040in}}{\pgfqpoint{2.540435in}{1.775431in}}{\pgfqpoint{2.548249in}{1.783244in}}%
\pgfpathcurveto{\pgfqpoint{2.556062in}{1.791058in}}{\pgfqpoint{2.560452in}{1.801657in}}{\pgfqpoint{2.560452in}{1.812707in}}%
\pgfpathcurveto{\pgfqpoint{2.560452in}{1.823757in}}{\pgfqpoint{2.556062in}{1.834356in}}{\pgfqpoint{2.548249in}{1.842170in}}%
\pgfpathcurveto{\pgfqpoint{2.540435in}{1.849983in}}{\pgfqpoint{2.529836in}{1.854374in}}{\pgfqpoint{2.518786in}{1.854374in}}%
\pgfpathcurveto{\pgfqpoint{2.507736in}{1.854374in}}{\pgfqpoint{2.497137in}{1.849983in}}{\pgfqpoint{2.489323in}{1.842170in}}%
\pgfpathcurveto{\pgfqpoint{2.481509in}{1.834356in}}{\pgfqpoint{2.477119in}{1.823757in}}{\pgfqpoint{2.477119in}{1.812707in}}%
\pgfpathcurveto{\pgfqpoint{2.477119in}{1.801657in}}{\pgfqpoint{2.481509in}{1.791058in}}{\pgfqpoint{2.489323in}{1.783244in}}%
\pgfpathcurveto{\pgfqpoint{2.497137in}{1.775431in}}{\pgfqpoint{2.507736in}{1.771040in}}{\pgfqpoint{2.518786in}{1.771040in}}%
\pgfpathclose%
\pgfusepath{stroke,fill}%
\end{pgfscope}%
\begin{pgfscope}%
\pgfpathrectangle{\pgfqpoint{0.800000in}{0.528000in}}{\pgfqpoint{4.960000in}{3.696000in}}%
\pgfusepath{clip}%
\pgfsetbuttcap%
\pgfsetroundjoin%
\definecolor{currentfill}{rgb}{0.000000,0.000000,0.000000}%
\pgfsetfillcolor{currentfill}%
\pgfsetlinewidth{1.003750pt}%
\definecolor{currentstroke}{rgb}{0.000000,0.000000,0.000000}%
\pgfsetstrokecolor{currentstroke}%
\pgfsetdash{}{0pt}%
\pgfpathmoveto{\pgfqpoint{2.518786in}{1.771040in}}%
\pgfpathcurveto{\pgfqpoint{2.529836in}{1.771040in}}{\pgfqpoint{2.540435in}{1.775431in}}{\pgfqpoint{2.548249in}{1.783244in}}%
\pgfpathcurveto{\pgfqpoint{2.556062in}{1.791058in}}{\pgfqpoint{2.560452in}{1.801657in}}{\pgfqpoint{2.560452in}{1.812707in}}%
\pgfpathcurveto{\pgfqpoint{2.560452in}{1.823757in}}{\pgfqpoint{2.556062in}{1.834356in}}{\pgfqpoint{2.548249in}{1.842170in}}%
\pgfpathcurveto{\pgfqpoint{2.540435in}{1.849983in}}{\pgfqpoint{2.529836in}{1.854374in}}{\pgfqpoint{2.518786in}{1.854374in}}%
\pgfpathcurveto{\pgfqpoint{2.507736in}{1.854374in}}{\pgfqpoint{2.497137in}{1.849983in}}{\pgfqpoint{2.489323in}{1.842170in}}%
\pgfpathcurveto{\pgfqpoint{2.481509in}{1.834356in}}{\pgfqpoint{2.477119in}{1.823757in}}{\pgfqpoint{2.477119in}{1.812707in}}%
\pgfpathcurveto{\pgfqpoint{2.477119in}{1.801657in}}{\pgfqpoint{2.481509in}{1.791058in}}{\pgfqpoint{2.489323in}{1.783244in}}%
\pgfpathcurveto{\pgfqpoint{2.497137in}{1.775431in}}{\pgfqpoint{2.507736in}{1.771040in}}{\pgfqpoint{2.518786in}{1.771040in}}%
\pgfpathclose%
\pgfusepath{stroke,fill}%
\end{pgfscope}%
\begin{pgfscope}%
\pgfpathrectangle{\pgfqpoint{0.800000in}{0.528000in}}{\pgfqpoint{4.960000in}{3.696000in}}%
\pgfusepath{clip}%
\pgfsetbuttcap%
\pgfsetroundjoin%
\definecolor{currentfill}{rgb}{0.000000,0.000000,0.000000}%
\pgfsetfillcolor{currentfill}%
\pgfsetlinewidth{1.003750pt}%
\definecolor{currentstroke}{rgb}{0.000000,0.000000,0.000000}%
\pgfsetstrokecolor{currentstroke}%
\pgfsetdash{}{0pt}%
\pgfpathmoveto{\pgfqpoint{2.518786in}{1.771040in}}%
\pgfpathcurveto{\pgfqpoint{2.529836in}{1.771040in}}{\pgfqpoint{2.540435in}{1.775431in}}{\pgfqpoint{2.548249in}{1.783244in}}%
\pgfpathcurveto{\pgfqpoint{2.556062in}{1.791058in}}{\pgfqpoint{2.560452in}{1.801657in}}{\pgfqpoint{2.560452in}{1.812707in}}%
\pgfpathcurveto{\pgfqpoint{2.560452in}{1.823757in}}{\pgfqpoint{2.556062in}{1.834356in}}{\pgfqpoint{2.548249in}{1.842170in}}%
\pgfpathcurveto{\pgfqpoint{2.540435in}{1.849983in}}{\pgfqpoint{2.529836in}{1.854374in}}{\pgfqpoint{2.518786in}{1.854374in}}%
\pgfpathcurveto{\pgfqpoint{2.507736in}{1.854374in}}{\pgfqpoint{2.497137in}{1.849983in}}{\pgfqpoint{2.489323in}{1.842170in}}%
\pgfpathcurveto{\pgfqpoint{2.481509in}{1.834356in}}{\pgfqpoint{2.477119in}{1.823757in}}{\pgfqpoint{2.477119in}{1.812707in}}%
\pgfpathcurveto{\pgfqpoint{2.477119in}{1.801657in}}{\pgfqpoint{2.481509in}{1.791058in}}{\pgfqpoint{2.489323in}{1.783244in}}%
\pgfpathcurveto{\pgfqpoint{2.497137in}{1.775431in}}{\pgfqpoint{2.507736in}{1.771040in}}{\pgfqpoint{2.518786in}{1.771040in}}%
\pgfpathclose%
\pgfusepath{stroke,fill}%
\end{pgfscope}%
\begin{pgfscope}%
\pgfpathrectangle{\pgfqpoint{0.800000in}{0.528000in}}{\pgfqpoint{4.960000in}{3.696000in}}%
\pgfusepath{clip}%
\pgfsetbuttcap%
\pgfsetroundjoin%
\definecolor{currentfill}{rgb}{0.000000,0.000000,0.000000}%
\pgfsetfillcolor{currentfill}%
\pgfsetlinewidth{1.003750pt}%
\definecolor{currentstroke}{rgb}{0.000000,0.000000,0.000000}%
\pgfsetstrokecolor{currentstroke}%
\pgfsetdash{}{0pt}%
\pgfpathmoveto{\pgfqpoint{2.518786in}{1.771040in}}%
\pgfpathcurveto{\pgfqpoint{2.529836in}{1.771040in}}{\pgfqpoint{2.540435in}{1.775431in}}{\pgfqpoint{2.548249in}{1.783244in}}%
\pgfpathcurveto{\pgfqpoint{2.556062in}{1.791058in}}{\pgfqpoint{2.560452in}{1.801657in}}{\pgfqpoint{2.560452in}{1.812707in}}%
\pgfpathcurveto{\pgfqpoint{2.560452in}{1.823757in}}{\pgfqpoint{2.556062in}{1.834356in}}{\pgfqpoint{2.548249in}{1.842170in}}%
\pgfpathcurveto{\pgfqpoint{2.540435in}{1.849983in}}{\pgfqpoint{2.529836in}{1.854374in}}{\pgfqpoint{2.518786in}{1.854374in}}%
\pgfpathcurveto{\pgfqpoint{2.507736in}{1.854374in}}{\pgfqpoint{2.497137in}{1.849983in}}{\pgfqpoint{2.489323in}{1.842170in}}%
\pgfpathcurveto{\pgfqpoint{2.481509in}{1.834356in}}{\pgfqpoint{2.477119in}{1.823757in}}{\pgfqpoint{2.477119in}{1.812707in}}%
\pgfpathcurveto{\pgfqpoint{2.477119in}{1.801657in}}{\pgfqpoint{2.481509in}{1.791058in}}{\pgfqpoint{2.489323in}{1.783244in}}%
\pgfpathcurveto{\pgfqpoint{2.497137in}{1.775431in}}{\pgfqpoint{2.507736in}{1.771040in}}{\pgfqpoint{2.518786in}{1.771040in}}%
\pgfpathclose%
\pgfusepath{stroke,fill}%
\end{pgfscope}%
\begin{pgfscope}%
\pgfpathrectangle{\pgfqpoint{0.800000in}{0.528000in}}{\pgfqpoint{4.960000in}{3.696000in}}%
\pgfusepath{clip}%
\pgfsetbuttcap%
\pgfsetroundjoin%
\definecolor{currentfill}{rgb}{0.000000,0.000000,0.000000}%
\pgfsetfillcolor{currentfill}%
\pgfsetlinewidth{1.003750pt}%
\definecolor{currentstroke}{rgb}{0.000000,0.000000,0.000000}%
\pgfsetstrokecolor{currentstroke}%
\pgfsetdash{}{0pt}%
\pgfpathmoveto{\pgfqpoint{2.518786in}{1.771040in}}%
\pgfpathcurveto{\pgfqpoint{2.529836in}{1.771040in}}{\pgfqpoint{2.540435in}{1.775431in}}{\pgfqpoint{2.548249in}{1.783244in}}%
\pgfpathcurveto{\pgfqpoint{2.556062in}{1.791058in}}{\pgfqpoint{2.560452in}{1.801657in}}{\pgfqpoint{2.560452in}{1.812707in}}%
\pgfpathcurveto{\pgfqpoint{2.560452in}{1.823757in}}{\pgfqpoint{2.556062in}{1.834356in}}{\pgfqpoint{2.548249in}{1.842170in}}%
\pgfpathcurveto{\pgfqpoint{2.540435in}{1.849983in}}{\pgfqpoint{2.529836in}{1.854374in}}{\pgfqpoint{2.518786in}{1.854374in}}%
\pgfpathcurveto{\pgfqpoint{2.507736in}{1.854374in}}{\pgfqpoint{2.497137in}{1.849983in}}{\pgfqpoint{2.489323in}{1.842170in}}%
\pgfpathcurveto{\pgfqpoint{2.481509in}{1.834356in}}{\pgfqpoint{2.477119in}{1.823757in}}{\pgfqpoint{2.477119in}{1.812707in}}%
\pgfpathcurveto{\pgfqpoint{2.477119in}{1.801657in}}{\pgfqpoint{2.481509in}{1.791058in}}{\pgfqpoint{2.489323in}{1.783244in}}%
\pgfpathcurveto{\pgfqpoint{2.497137in}{1.775431in}}{\pgfqpoint{2.507736in}{1.771040in}}{\pgfqpoint{2.518786in}{1.771040in}}%
\pgfpathclose%
\pgfusepath{stroke,fill}%
\end{pgfscope}%
\begin{pgfscope}%
\pgfpathrectangle{\pgfqpoint{0.800000in}{0.528000in}}{\pgfqpoint{4.960000in}{3.696000in}}%
\pgfusepath{clip}%
\pgfsetbuttcap%
\pgfsetroundjoin%
\definecolor{currentfill}{rgb}{0.000000,0.000000,0.000000}%
\pgfsetfillcolor{currentfill}%
\pgfsetlinewidth{1.003750pt}%
\definecolor{currentstroke}{rgb}{0.000000,0.000000,0.000000}%
\pgfsetstrokecolor{currentstroke}%
\pgfsetdash{}{0pt}%
\pgfpathmoveto{\pgfqpoint{2.518786in}{1.771040in}}%
\pgfpathcurveto{\pgfqpoint{2.529836in}{1.771040in}}{\pgfqpoint{2.540435in}{1.775431in}}{\pgfqpoint{2.548249in}{1.783244in}}%
\pgfpathcurveto{\pgfqpoint{2.556062in}{1.791058in}}{\pgfqpoint{2.560452in}{1.801657in}}{\pgfqpoint{2.560452in}{1.812707in}}%
\pgfpathcurveto{\pgfqpoint{2.560452in}{1.823757in}}{\pgfqpoint{2.556062in}{1.834356in}}{\pgfqpoint{2.548249in}{1.842170in}}%
\pgfpathcurveto{\pgfqpoint{2.540435in}{1.849983in}}{\pgfqpoint{2.529836in}{1.854374in}}{\pgfqpoint{2.518786in}{1.854374in}}%
\pgfpathcurveto{\pgfqpoint{2.507736in}{1.854374in}}{\pgfqpoint{2.497137in}{1.849983in}}{\pgfqpoint{2.489323in}{1.842170in}}%
\pgfpathcurveto{\pgfqpoint{2.481509in}{1.834356in}}{\pgfqpoint{2.477119in}{1.823757in}}{\pgfqpoint{2.477119in}{1.812707in}}%
\pgfpathcurveto{\pgfqpoint{2.477119in}{1.801657in}}{\pgfqpoint{2.481509in}{1.791058in}}{\pgfqpoint{2.489323in}{1.783244in}}%
\pgfpathcurveto{\pgfqpoint{2.497137in}{1.775431in}}{\pgfqpoint{2.507736in}{1.771040in}}{\pgfqpoint{2.518786in}{1.771040in}}%
\pgfpathclose%
\pgfusepath{stroke,fill}%
\end{pgfscope}%
\begin{pgfscope}%
\pgfpathrectangle{\pgfqpoint{0.800000in}{0.528000in}}{\pgfqpoint{4.960000in}{3.696000in}}%
\pgfusepath{clip}%
\pgfsetbuttcap%
\pgfsetroundjoin%
\definecolor{currentfill}{rgb}{0.000000,0.000000,0.000000}%
\pgfsetfillcolor{currentfill}%
\pgfsetlinewidth{1.003750pt}%
\definecolor{currentstroke}{rgb}{0.000000,0.000000,0.000000}%
\pgfsetstrokecolor{currentstroke}%
\pgfsetdash{}{0pt}%
\pgfpathmoveto{\pgfqpoint{2.518786in}{1.771040in}}%
\pgfpathcurveto{\pgfqpoint{2.529836in}{1.771040in}}{\pgfqpoint{2.540435in}{1.775431in}}{\pgfqpoint{2.548249in}{1.783244in}}%
\pgfpathcurveto{\pgfqpoint{2.556062in}{1.791058in}}{\pgfqpoint{2.560452in}{1.801657in}}{\pgfqpoint{2.560452in}{1.812707in}}%
\pgfpathcurveto{\pgfqpoint{2.560452in}{1.823757in}}{\pgfqpoint{2.556062in}{1.834356in}}{\pgfqpoint{2.548249in}{1.842170in}}%
\pgfpathcurveto{\pgfqpoint{2.540435in}{1.849983in}}{\pgfqpoint{2.529836in}{1.854374in}}{\pgfqpoint{2.518786in}{1.854374in}}%
\pgfpathcurveto{\pgfqpoint{2.507736in}{1.854374in}}{\pgfqpoint{2.497137in}{1.849983in}}{\pgfqpoint{2.489323in}{1.842170in}}%
\pgfpathcurveto{\pgfqpoint{2.481509in}{1.834356in}}{\pgfqpoint{2.477119in}{1.823757in}}{\pgfqpoint{2.477119in}{1.812707in}}%
\pgfpathcurveto{\pgfqpoint{2.477119in}{1.801657in}}{\pgfqpoint{2.481509in}{1.791058in}}{\pgfqpoint{2.489323in}{1.783244in}}%
\pgfpathcurveto{\pgfqpoint{2.497137in}{1.775431in}}{\pgfqpoint{2.507736in}{1.771040in}}{\pgfqpoint{2.518786in}{1.771040in}}%
\pgfpathclose%
\pgfusepath{stroke,fill}%
\end{pgfscope}%
\begin{pgfscope}%
\pgfpathrectangle{\pgfqpoint{0.800000in}{0.528000in}}{\pgfqpoint{4.960000in}{3.696000in}}%
\pgfusepath{clip}%
\pgfsetbuttcap%
\pgfsetroundjoin%
\definecolor{currentfill}{rgb}{0.000000,0.000000,0.000000}%
\pgfsetfillcolor{currentfill}%
\pgfsetlinewidth{1.003750pt}%
\definecolor{currentstroke}{rgb}{0.000000,0.000000,0.000000}%
\pgfsetstrokecolor{currentstroke}%
\pgfsetdash{}{0pt}%
\pgfpathmoveto{\pgfqpoint{2.518786in}{1.771040in}}%
\pgfpathcurveto{\pgfqpoint{2.529836in}{1.771040in}}{\pgfqpoint{2.540435in}{1.775431in}}{\pgfqpoint{2.548249in}{1.783244in}}%
\pgfpathcurveto{\pgfqpoint{2.556062in}{1.791058in}}{\pgfqpoint{2.560452in}{1.801657in}}{\pgfqpoint{2.560452in}{1.812707in}}%
\pgfpathcurveto{\pgfqpoint{2.560452in}{1.823757in}}{\pgfqpoint{2.556062in}{1.834356in}}{\pgfqpoint{2.548249in}{1.842170in}}%
\pgfpathcurveto{\pgfqpoint{2.540435in}{1.849983in}}{\pgfqpoint{2.529836in}{1.854374in}}{\pgfqpoint{2.518786in}{1.854374in}}%
\pgfpathcurveto{\pgfqpoint{2.507736in}{1.854374in}}{\pgfqpoint{2.497137in}{1.849983in}}{\pgfqpoint{2.489323in}{1.842170in}}%
\pgfpathcurveto{\pgfqpoint{2.481509in}{1.834356in}}{\pgfqpoint{2.477119in}{1.823757in}}{\pgfqpoint{2.477119in}{1.812707in}}%
\pgfpathcurveto{\pgfqpoint{2.477119in}{1.801657in}}{\pgfqpoint{2.481509in}{1.791058in}}{\pgfqpoint{2.489323in}{1.783244in}}%
\pgfpathcurveto{\pgfqpoint{2.497137in}{1.775431in}}{\pgfqpoint{2.507736in}{1.771040in}}{\pgfqpoint{2.518786in}{1.771040in}}%
\pgfpathclose%
\pgfusepath{stroke,fill}%
\end{pgfscope}%
\begin{pgfscope}%
\pgfpathrectangle{\pgfqpoint{0.800000in}{0.528000in}}{\pgfqpoint{4.960000in}{3.696000in}}%
\pgfusepath{clip}%
\pgfsetbuttcap%
\pgfsetroundjoin%
\definecolor{currentfill}{rgb}{0.000000,0.000000,0.000000}%
\pgfsetfillcolor{currentfill}%
\pgfsetlinewidth{1.003750pt}%
\definecolor{currentstroke}{rgb}{0.000000,0.000000,0.000000}%
\pgfsetstrokecolor{currentstroke}%
\pgfsetdash{}{0pt}%
\pgfpathmoveto{\pgfqpoint{2.518786in}{1.771040in}}%
\pgfpathcurveto{\pgfqpoint{2.529836in}{1.771040in}}{\pgfqpoint{2.540435in}{1.775431in}}{\pgfqpoint{2.548249in}{1.783244in}}%
\pgfpathcurveto{\pgfqpoint{2.556062in}{1.791058in}}{\pgfqpoint{2.560452in}{1.801657in}}{\pgfqpoint{2.560452in}{1.812707in}}%
\pgfpathcurveto{\pgfqpoint{2.560452in}{1.823757in}}{\pgfqpoint{2.556062in}{1.834356in}}{\pgfqpoint{2.548249in}{1.842170in}}%
\pgfpathcurveto{\pgfqpoint{2.540435in}{1.849983in}}{\pgfqpoint{2.529836in}{1.854374in}}{\pgfqpoint{2.518786in}{1.854374in}}%
\pgfpathcurveto{\pgfqpoint{2.507736in}{1.854374in}}{\pgfqpoint{2.497137in}{1.849983in}}{\pgfqpoint{2.489323in}{1.842170in}}%
\pgfpathcurveto{\pgfqpoint{2.481509in}{1.834356in}}{\pgfqpoint{2.477119in}{1.823757in}}{\pgfqpoint{2.477119in}{1.812707in}}%
\pgfpathcurveto{\pgfqpoint{2.477119in}{1.801657in}}{\pgfqpoint{2.481509in}{1.791058in}}{\pgfqpoint{2.489323in}{1.783244in}}%
\pgfpathcurveto{\pgfqpoint{2.497137in}{1.775431in}}{\pgfqpoint{2.507736in}{1.771040in}}{\pgfqpoint{2.518786in}{1.771040in}}%
\pgfpathclose%
\pgfusepath{stroke,fill}%
\end{pgfscope}%
\begin{pgfscope}%
\pgfpathrectangle{\pgfqpoint{0.800000in}{0.528000in}}{\pgfqpoint{4.960000in}{3.696000in}}%
\pgfusepath{clip}%
\pgfsetbuttcap%
\pgfsetroundjoin%
\definecolor{currentfill}{rgb}{0.000000,0.000000,0.000000}%
\pgfsetfillcolor{currentfill}%
\pgfsetlinewidth{1.003750pt}%
\definecolor{currentstroke}{rgb}{0.000000,0.000000,0.000000}%
\pgfsetstrokecolor{currentstroke}%
\pgfsetdash{}{0pt}%
\pgfpathmoveto{\pgfqpoint{2.518786in}{0.664394in}}%
\pgfpathcurveto{\pgfqpoint{2.529836in}{0.664394in}}{\pgfqpoint{2.540435in}{0.668784in}}{\pgfqpoint{2.548249in}{0.676598in}}%
\pgfpathcurveto{\pgfqpoint{2.556062in}{0.684411in}}{\pgfqpoint{2.560452in}{0.695010in}}{\pgfqpoint{2.560452in}{0.706060in}}%
\pgfpathcurveto{\pgfqpoint{2.560452in}{0.717111in}}{\pgfqpoint{2.556062in}{0.727710in}}{\pgfqpoint{2.548249in}{0.735523in}}%
\pgfpathcurveto{\pgfqpoint{2.540435in}{0.743337in}}{\pgfqpoint{2.529836in}{0.747727in}}{\pgfqpoint{2.518786in}{0.747727in}}%
\pgfpathcurveto{\pgfqpoint{2.507736in}{0.747727in}}{\pgfqpoint{2.497137in}{0.743337in}}{\pgfqpoint{2.489323in}{0.735523in}}%
\pgfpathcurveto{\pgfqpoint{2.481509in}{0.727710in}}{\pgfqpoint{2.477119in}{0.717111in}}{\pgfqpoint{2.477119in}{0.706060in}}%
\pgfpathcurveto{\pgfqpoint{2.477119in}{0.695010in}}{\pgfqpoint{2.481509in}{0.684411in}}{\pgfqpoint{2.489323in}{0.676598in}}%
\pgfpathcurveto{\pgfqpoint{2.497137in}{0.668784in}}{\pgfqpoint{2.507736in}{0.664394in}}{\pgfqpoint{2.518786in}{0.664394in}}%
\pgfpathclose%
\pgfusepath{stroke,fill}%
\end{pgfscope}%
\begin{pgfscope}%
\pgfpathrectangle{\pgfqpoint{0.800000in}{0.528000in}}{\pgfqpoint{4.960000in}{3.696000in}}%
\pgfusepath{clip}%
\pgfsetbuttcap%
\pgfsetroundjoin%
\definecolor{currentfill}{rgb}{0.000000,0.000000,0.000000}%
\pgfsetfillcolor{currentfill}%
\pgfsetlinewidth{1.003750pt}%
\definecolor{currentstroke}{rgb}{0.000000,0.000000,0.000000}%
\pgfsetstrokecolor{currentstroke}%
\pgfsetdash{}{0pt}%
\pgfpathmoveto{\pgfqpoint{2.518786in}{1.771040in}}%
\pgfpathcurveto{\pgfqpoint{2.529836in}{1.771040in}}{\pgfqpoint{2.540435in}{1.775431in}}{\pgfqpoint{2.548249in}{1.783244in}}%
\pgfpathcurveto{\pgfqpoint{2.556062in}{1.791058in}}{\pgfqpoint{2.560452in}{1.801657in}}{\pgfqpoint{2.560452in}{1.812707in}}%
\pgfpathcurveto{\pgfqpoint{2.560452in}{1.823757in}}{\pgfqpoint{2.556062in}{1.834356in}}{\pgfqpoint{2.548249in}{1.842170in}}%
\pgfpathcurveto{\pgfqpoint{2.540435in}{1.849983in}}{\pgfqpoint{2.529836in}{1.854374in}}{\pgfqpoint{2.518786in}{1.854374in}}%
\pgfpathcurveto{\pgfqpoint{2.507736in}{1.854374in}}{\pgfqpoint{2.497137in}{1.849983in}}{\pgfqpoint{2.489323in}{1.842170in}}%
\pgfpathcurveto{\pgfqpoint{2.481509in}{1.834356in}}{\pgfqpoint{2.477119in}{1.823757in}}{\pgfqpoint{2.477119in}{1.812707in}}%
\pgfpathcurveto{\pgfqpoint{2.477119in}{1.801657in}}{\pgfqpoint{2.481509in}{1.791058in}}{\pgfqpoint{2.489323in}{1.783244in}}%
\pgfpathcurveto{\pgfqpoint{2.497137in}{1.775431in}}{\pgfqpoint{2.507736in}{1.771040in}}{\pgfqpoint{2.518786in}{1.771040in}}%
\pgfpathclose%
\pgfusepath{stroke,fill}%
\end{pgfscope}%
\begin{pgfscope}%
\pgfpathrectangle{\pgfqpoint{0.800000in}{0.528000in}}{\pgfqpoint{4.960000in}{3.696000in}}%
\pgfusepath{clip}%
\pgfsetbuttcap%
\pgfsetroundjoin%
\definecolor{currentfill}{rgb}{0.000000,0.000000,0.000000}%
\pgfsetfillcolor{currentfill}%
\pgfsetlinewidth{1.003750pt}%
\definecolor{currentstroke}{rgb}{0.000000,0.000000,0.000000}%
\pgfsetstrokecolor{currentstroke}%
\pgfsetdash{}{0pt}%
\pgfpathmoveto{\pgfqpoint{2.518786in}{1.771040in}}%
\pgfpathcurveto{\pgfqpoint{2.529836in}{1.771040in}}{\pgfqpoint{2.540435in}{1.775431in}}{\pgfqpoint{2.548249in}{1.783244in}}%
\pgfpathcurveto{\pgfqpoint{2.556062in}{1.791058in}}{\pgfqpoint{2.560452in}{1.801657in}}{\pgfqpoint{2.560452in}{1.812707in}}%
\pgfpathcurveto{\pgfqpoint{2.560452in}{1.823757in}}{\pgfqpoint{2.556062in}{1.834356in}}{\pgfqpoint{2.548249in}{1.842170in}}%
\pgfpathcurveto{\pgfqpoint{2.540435in}{1.849983in}}{\pgfqpoint{2.529836in}{1.854374in}}{\pgfqpoint{2.518786in}{1.854374in}}%
\pgfpathcurveto{\pgfqpoint{2.507736in}{1.854374in}}{\pgfqpoint{2.497137in}{1.849983in}}{\pgfqpoint{2.489323in}{1.842170in}}%
\pgfpathcurveto{\pgfqpoint{2.481509in}{1.834356in}}{\pgfqpoint{2.477119in}{1.823757in}}{\pgfqpoint{2.477119in}{1.812707in}}%
\pgfpathcurveto{\pgfqpoint{2.477119in}{1.801657in}}{\pgfqpoint{2.481509in}{1.791058in}}{\pgfqpoint{2.489323in}{1.783244in}}%
\pgfpathcurveto{\pgfqpoint{2.497137in}{1.775431in}}{\pgfqpoint{2.507736in}{1.771040in}}{\pgfqpoint{2.518786in}{1.771040in}}%
\pgfpathclose%
\pgfusepath{stroke,fill}%
\end{pgfscope}%
\begin{pgfscope}%
\pgfpathrectangle{\pgfqpoint{0.800000in}{0.528000in}}{\pgfqpoint{4.960000in}{3.696000in}}%
\pgfusepath{clip}%
\pgfsetbuttcap%
\pgfsetroundjoin%
\definecolor{currentfill}{rgb}{0.000000,0.000000,0.000000}%
\pgfsetfillcolor{currentfill}%
\pgfsetlinewidth{1.003750pt}%
\definecolor{currentstroke}{rgb}{0.000000,0.000000,0.000000}%
\pgfsetstrokecolor{currentstroke}%
\pgfsetdash{}{0pt}%
\pgfpathmoveto{\pgfqpoint{2.518786in}{1.771040in}}%
\pgfpathcurveto{\pgfqpoint{2.529836in}{1.771040in}}{\pgfqpoint{2.540435in}{1.775431in}}{\pgfqpoint{2.548249in}{1.783244in}}%
\pgfpathcurveto{\pgfqpoint{2.556062in}{1.791058in}}{\pgfqpoint{2.560452in}{1.801657in}}{\pgfqpoint{2.560452in}{1.812707in}}%
\pgfpathcurveto{\pgfqpoint{2.560452in}{1.823757in}}{\pgfqpoint{2.556062in}{1.834356in}}{\pgfqpoint{2.548249in}{1.842170in}}%
\pgfpathcurveto{\pgfqpoint{2.540435in}{1.849983in}}{\pgfqpoint{2.529836in}{1.854374in}}{\pgfqpoint{2.518786in}{1.854374in}}%
\pgfpathcurveto{\pgfqpoint{2.507736in}{1.854374in}}{\pgfqpoint{2.497137in}{1.849983in}}{\pgfqpoint{2.489323in}{1.842170in}}%
\pgfpathcurveto{\pgfqpoint{2.481509in}{1.834356in}}{\pgfqpoint{2.477119in}{1.823757in}}{\pgfqpoint{2.477119in}{1.812707in}}%
\pgfpathcurveto{\pgfqpoint{2.477119in}{1.801657in}}{\pgfqpoint{2.481509in}{1.791058in}}{\pgfqpoint{2.489323in}{1.783244in}}%
\pgfpathcurveto{\pgfqpoint{2.497137in}{1.775431in}}{\pgfqpoint{2.507736in}{1.771040in}}{\pgfqpoint{2.518786in}{1.771040in}}%
\pgfpathclose%
\pgfusepath{stroke,fill}%
\end{pgfscope}%
\begin{pgfscope}%
\pgfpathrectangle{\pgfqpoint{0.800000in}{0.528000in}}{\pgfqpoint{4.960000in}{3.696000in}}%
\pgfusepath{clip}%
\pgfsetbuttcap%
\pgfsetroundjoin%
\definecolor{currentfill}{rgb}{0.000000,0.000000,0.000000}%
\pgfsetfillcolor{currentfill}%
\pgfsetlinewidth{1.003750pt}%
\definecolor{currentstroke}{rgb}{0.000000,0.000000,0.000000}%
\pgfsetstrokecolor{currentstroke}%
\pgfsetdash{}{0pt}%
\pgfpathmoveto{\pgfqpoint{2.518786in}{1.771040in}}%
\pgfpathcurveto{\pgfqpoint{2.529836in}{1.771040in}}{\pgfqpoint{2.540435in}{1.775431in}}{\pgfqpoint{2.548249in}{1.783244in}}%
\pgfpathcurveto{\pgfqpoint{2.556062in}{1.791058in}}{\pgfqpoint{2.560452in}{1.801657in}}{\pgfqpoint{2.560452in}{1.812707in}}%
\pgfpathcurveto{\pgfqpoint{2.560452in}{1.823757in}}{\pgfqpoint{2.556062in}{1.834356in}}{\pgfqpoint{2.548249in}{1.842170in}}%
\pgfpathcurveto{\pgfqpoint{2.540435in}{1.849983in}}{\pgfqpoint{2.529836in}{1.854374in}}{\pgfqpoint{2.518786in}{1.854374in}}%
\pgfpathcurveto{\pgfqpoint{2.507736in}{1.854374in}}{\pgfqpoint{2.497137in}{1.849983in}}{\pgfqpoint{2.489323in}{1.842170in}}%
\pgfpathcurveto{\pgfqpoint{2.481509in}{1.834356in}}{\pgfqpoint{2.477119in}{1.823757in}}{\pgfqpoint{2.477119in}{1.812707in}}%
\pgfpathcurveto{\pgfqpoint{2.477119in}{1.801657in}}{\pgfqpoint{2.481509in}{1.791058in}}{\pgfqpoint{2.489323in}{1.783244in}}%
\pgfpathcurveto{\pgfqpoint{2.497137in}{1.775431in}}{\pgfqpoint{2.507736in}{1.771040in}}{\pgfqpoint{2.518786in}{1.771040in}}%
\pgfpathclose%
\pgfusepath{stroke,fill}%
\end{pgfscope}%
\begin{pgfscope}%
\pgfpathrectangle{\pgfqpoint{0.800000in}{0.528000in}}{\pgfqpoint{4.960000in}{3.696000in}}%
\pgfusepath{clip}%
\pgfsetbuttcap%
\pgfsetroundjoin%
\definecolor{currentfill}{rgb}{0.000000,0.000000,0.000000}%
\pgfsetfillcolor{currentfill}%
\pgfsetlinewidth{1.003750pt}%
\definecolor{currentstroke}{rgb}{0.000000,0.000000,0.000000}%
\pgfsetstrokecolor{currentstroke}%
\pgfsetdash{}{0pt}%
\pgfpathmoveto{\pgfqpoint{2.518786in}{1.771040in}}%
\pgfpathcurveto{\pgfqpoint{2.529836in}{1.771040in}}{\pgfqpoint{2.540435in}{1.775431in}}{\pgfqpoint{2.548249in}{1.783244in}}%
\pgfpathcurveto{\pgfqpoint{2.556062in}{1.791058in}}{\pgfqpoint{2.560452in}{1.801657in}}{\pgfqpoint{2.560452in}{1.812707in}}%
\pgfpathcurveto{\pgfqpoint{2.560452in}{1.823757in}}{\pgfqpoint{2.556062in}{1.834356in}}{\pgfqpoint{2.548249in}{1.842170in}}%
\pgfpathcurveto{\pgfqpoint{2.540435in}{1.849983in}}{\pgfqpoint{2.529836in}{1.854374in}}{\pgfqpoint{2.518786in}{1.854374in}}%
\pgfpathcurveto{\pgfqpoint{2.507736in}{1.854374in}}{\pgfqpoint{2.497137in}{1.849983in}}{\pgfqpoint{2.489323in}{1.842170in}}%
\pgfpathcurveto{\pgfqpoint{2.481509in}{1.834356in}}{\pgfqpoint{2.477119in}{1.823757in}}{\pgfqpoint{2.477119in}{1.812707in}}%
\pgfpathcurveto{\pgfqpoint{2.477119in}{1.801657in}}{\pgfqpoint{2.481509in}{1.791058in}}{\pgfqpoint{2.489323in}{1.783244in}}%
\pgfpathcurveto{\pgfqpoint{2.497137in}{1.775431in}}{\pgfqpoint{2.507736in}{1.771040in}}{\pgfqpoint{2.518786in}{1.771040in}}%
\pgfpathclose%
\pgfusepath{stroke,fill}%
\end{pgfscope}%
\begin{pgfscope}%
\pgfpathrectangle{\pgfqpoint{0.800000in}{0.528000in}}{\pgfqpoint{4.960000in}{3.696000in}}%
\pgfusepath{clip}%
\pgfsetbuttcap%
\pgfsetroundjoin%
\definecolor{currentfill}{rgb}{0.000000,0.000000,0.000000}%
\pgfsetfillcolor{currentfill}%
\pgfsetlinewidth{1.003750pt}%
\definecolor{currentstroke}{rgb}{0.000000,0.000000,0.000000}%
\pgfsetstrokecolor{currentstroke}%
\pgfsetdash{}{0pt}%
\pgfpathmoveto{\pgfqpoint{2.518786in}{1.771040in}}%
\pgfpathcurveto{\pgfqpoint{2.529836in}{1.771040in}}{\pgfqpoint{2.540435in}{1.775431in}}{\pgfqpoint{2.548249in}{1.783244in}}%
\pgfpathcurveto{\pgfqpoint{2.556062in}{1.791058in}}{\pgfqpoint{2.560452in}{1.801657in}}{\pgfqpoint{2.560452in}{1.812707in}}%
\pgfpathcurveto{\pgfqpoint{2.560452in}{1.823757in}}{\pgfqpoint{2.556062in}{1.834356in}}{\pgfqpoint{2.548249in}{1.842170in}}%
\pgfpathcurveto{\pgfqpoint{2.540435in}{1.849983in}}{\pgfqpoint{2.529836in}{1.854374in}}{\pgfqpoint{2.518786in}{1.854374in}}%
\pgfpathcurveto{\pgfqpoint{2.507736in}{1.854374in}}{\pgfqpoint{2.497137in}{1.849983in}}{\pgfqpoint{2.489323in}{1.842170in}}%
\pgfpathcurveto{\pgfqpoint{2.481509in}{1.834356in}}{\pgfqpoint{2.477119in}{1.823757in}}{\pgfqpoint{2.477119in}{1.812707in}}%
\pgfpathcurveto{\pgfqpoint{2.477119in}{1.801657in}}{\pgfqpoint{2.481509in}{1.791058in}}{\pgfqpoint{2.489323in}{1.783244in}}%
\pgfpathcurveto{\pgfqpoint{2.497137in}{1.775431in}}{\pgfqpoint{2.507736in}{1.771040in}}{\pgfqpoint{2.518786in}{1.771040in}}%
\pgfpathclose%
\pgfusepath{stroke,fill}%
\end{pgfscope}%
\begin{pgfscope}%
\pgfpathrectangle{\pgfqpoint{0.800000in}{0.528000in}}{\pgfqpoint{4.960000in}{3.696000in}}%
\pgfusepath{clip}%
\pgfsetbuttcap%
\pgfsetroundjoin%
\definecolor{currentfill}{rgb}{0.000000,0.000000,0.000000}%
\pgfsetfillcolor{currentfill}%
\pgfsetlinewidth{1.003750pt}%
\definecolor{currentstroke}{rgb}{0.000000,0.000000,0.000000}%
\pgfsetstrokecolor{currentstroke}%
\pgfsetdash{}{0pt}%
\pgfpathmoveto{\pgfqpoint{2.518786in}{1.771040in}}%
\pgfpathcurveto{\pgfqpoint{2.529836in}{1.771040in}}{\pgfqpoint{2.540435in}{1.775431in}}{\pgfqpoint{2.548249in}{1.783244in}}%
\pgfpathcurveto{\pgfqpoint{2.556062in}{1.791058in}}{\pgfqpoint{2.560452in}{1.801657in}}{\pgfqpoint{2.560452in}{1.812707in}}%
\pgfpathcurveto{\pgfqpoint{2.560452in}{1.823757in}}{\pgfqpoint{2.556062in}{1.834356in}}{\pgfqpoint{2.548249in}{1.842170in}}%
\pgfpathcurveto{\pgfqpoint{2.540435in}{1.849983in}}{\pgfqpoint{2.529836in}{1.854374in}}{\pgfqpoint{2.518786in}{1.854374in}}%
\pgfpathcurveto{\pgfqpoint{2.507736in}{1.854374in}}{\pgfqpoint{2.497137in}{1.849983in}}{\pgfqpoint{2.489323in}{1.842170in}}%
\pgfpathcurveto{\pgfqpoint{2.481509in}{1.834356in}}{\pgfqpoint{2.477119in}{1.823757in}}{\pgfqpoint{2.477119in}{1.812707in}}%
\pgfpathcurveto{\pgfqpoint{2.477119in}{1.801657in}}{\pgfqpoint{2.481509in}{1.791058in}}{\pgfqpoint{2.489323in}{1.783244in}}%
\pgfpathcurveto{\pgfqpoint{2.497137in}{1.775431in}}{\pgfqpoint{2.507736in}{1.771040in}}{\pgfqpoint{2.518786in}{1.771040in}}%
\pgfpathclose%
\pgfusepath{stroke,fill}%
\end{pgfscope}%
\begin{pgfscope}%
\pgfpathrectangle{\pgfqpoint{0.800000in}{0.528000in}}{\pgfqpoint{4.960000in}{3.696000in}}%
\pgfusepath{clip}%
\pgfsetbuttcap%
\pgfsetroundjoin%
\definecolor{currentfill}{rgb}{0.000000,0.000000,0.000000}%
\pgfsetfillcolor{currentfill}%
\pgfsetlinewidth{1.003750pt}%
\definecolor{currentstroke}{rgb}{0.000000,0.000000,0.000000}%
\pgfsetstrokecolor{currentstroke}%
\pgfsetdash{}{0pt}%
\pgfpathmoveto{\pgfqpoint{2.518786in}{1.771040in}}%
\pgfpathcurveto{\pgfqpoint{2.529836in}{1.771040in}}{\pgfqpoint{2.540435in}{1.775431in}}{\pgfqpoint{2.548249in}{1.783244in}}%
\pgfpathcurveto{\pgfqpoint{2.556062in}{1.791058in}}{\pgfqpoint{2.560452in}{1.801657in}}{\pgfqpoint{2.560452in}{1.812707in}}%
\pgfpathcurveto{\pgfqpoint{2.560452in}{1.823757in}}{\pgfqpoint{2.556062in}{1.834356in}}{\pgfqpoint{2.548249in}{1.842170in}}%
\pgfpathcurveto{\pgfqpoint{2.540435in}{1.849983in}}{\pgfqpoint{2.529836in}{1.854374in}}{\pgfqpoint{2.518786in}{1.854374in}}%
\pgfpathcurveto{\pgfqpoint{2.507736in}{1.854374in}}{\pgfqpoint{2.497137in}{1.849983in}}{\pgfqpoint{2.489323in}{1.842170in}}%
\pgfpathcurveto{\pgfqpoint{2.481509in}{1.834356in}}{\pgfqpoint{2.477119in}{1.823757in}}{\pgfqpoint{2.477119in}{1.812707in}}%
\pgfpathcurveto{\pgfqpoint{2.477119in}{1.801657in}}{\pgfqpoint{2.481509in}{1.791058in}}{\pgfqpoint{2.489323in}{1.783244in}}%
\pgfpathcurveto{\pgfqpoint{2.497137in}{1.775431in}}{\pgfqpoint{2.507736in}{1.771040in}}{\pgfqpoint{2.518786in}{1.771040in}}%
\pgfpathclose%
\pgfusepath{stroke,fill}%
\end{pgfscope}%
\begin{pgfscope}%
\pgfpathrectangle{\pgfqpoint{0.800000in}{0.528000in}}{\pgfqpoint{4.960000in}{3.696000in}}%
\pgfusepath{clip}%
\pgfsetbuttcap%
\pgfsetroundjoin%
\definecolor{currentfill}{rgb}{0.000000,0.000000,0.000000}%
\pgfsetfillcolor{currentfill}%
\pgfsetlinewidth{1.003750pt}%
\definecolor{currentstroke}{rgb}{0.000000,0.000000,0.000000}%
\pgfsetstrokecolor{currentstroke}%
\pgfsetdash{}{0pt}%
\pgfpathmoveto{\pgfqpoint{2.518786in}{0.664394in}}%
\pgfpathcurveto{\pgfqpoint{2.529836in}{0.664394in}}{\pgfqpoint{2.540435in}{0.668784in}}{\pgfqpoint{2.548249in}{0.676598in}}%
\pgfpathcurveto{\pgfqpoint{2.556062in}{0.684411in}}{\pgfqpoint{2.560452in}{0.695010in}}{\pgfqpoint{2.560452in}{0.706060in}}%
\pgfpathcurveto{\pgfqpoint{2.560452in}{0.717111in}}{\pgfqpoint{2.556062in}{0.727710in}}{\pgfqpoint{2.548249in}{0.735523in}}%
\pgfpathcurveto{\pgfqpoint{2.540435in}{0.743337in}}{\pgfqpoint{2.529836in}{0.747727in}}{\pgfqpoint{2.518786in}{0.747727in}}%
\pgfpathcurveto{\pgfqpoint{2.507736in}{0.747727in}}{\pgfqpoint{2.497137in}{0.743337in}}{\pgfqpoint{2.489323in}{0.735523in}}%
\pgfpathcurveto{\pgfqpoint{2.481509in}{0.727710in}}{\pgfqpoint{2.477119in}{0.717111in}}{\pgfqpoint{2.477119in}{0.706060in}}%
\pgfpathcurveto{\pgfqpoint{2.477119in}{0.695010in}}{\pgfqpoint{2.481509in}{0.684411in}}{\pgfqpoint{2.489323in}{0.676598in}}%
\pgfpathcurveto{\pgfqpoint{2.497137in}{0.668784in}}{\pgfqpoint{2.507736in}{0.664394in}}{\pgfqpoint{2.518786in}{0.664394in}}%
\pgfpathclose%
\pgfusepath{stroke,fill}%
\end{pgfscope}%
\begin{pgfscope}%
\pgfpathrectangle{\pgfqpoint{0.800000in}{0.528000in}}{\pgfqpoint{4.960000in}{3.696000in}}%
\pgfusepath{clip}%
\pgfsetbuttcap%
\pgfsetroundjoin%
\definecolor{currentfill}{rgb}{0.000000,0.000000,0.000000}%
\pgfsetfillcolor{currentfill}%
\pgfsetlinewidth{1.003750pt}%
\definecolor{currentstroke}{rgb}{0.000000,0.000000,0.000000}%
\pgfsetstrokecolor{currentstroke}%
\pgfsetdash{}{0pt}%
\pgfpathmoveto{\pgfqpoint{2.518786in}{1.771040in}}%
\pgfpathcurveto{\pgfqpoint{2.529836in}{1.771040in}}{\pgfqpoint{2.540435in}{1.775431in}}{\pgfqpoint{2.548249in}{1.783244in}}%
\pgfpathcurveto{\pgfqpoint{2.556062in}{1.791058in}}{\pgfqpoint{2.560452in}{1.801657in}}{\pgfqpoint{2.560452in}{1.812707in}}%
\pgfpathcurveto{\pgfqpoint{2.560452in}{1.823757in}}{\pgfqpoint{2.556062in}{1.834356in}}{\pgfqpoint{2.548249in}{1.842170in}}%
\pgfpathcurveto{\pgfqpoint{2.540435in}{1.849983in}}{\pgfqpoint{2.529836in}{1.854374in}}{\pgfqpoint{2.518786in}{1.854374in}}%
\pgfpathcurveto{\pgfqpoint{2.507736in}{1.854374in}}{\pgfqpoint{2.497137in}{1.849983in}}{\pgfqpoint{2.489323in}{1.842170in}}%
\pgfpathcurveto{\pgfqpoint{2.481509in}{1.834356in}}{\pgfqpoint{2.477119in}{1.823757in}}{\pgfqpoint{2.477119in}{1.812707in}}%
\pgfpathcurveto{\pgfqpoint{2.477119in}{1.801657in}}{\pgfqpoint{2.481509in}{1.791058in}}{\pgfqpoint{2.489323in}{1.783244in}}%
\pgfpathcurveto{\pgfqpoint{2.497137in}{1.775431in}}{\pgfqpoint{2.507736in}{1.771040in}}{\pgfqpoint{2.518786in}{1.771040in}}%
\pgfpathclose%
\pgfusepath{stroke,fill}%
\end{pgfscope}%
\begin{pgfscope}%
\pgfpathrectangle{\pgfqpoint{0.800000in}{0.528000in}}{\pgfqpoint{4.960000in}{3.696000in}}%
\pgfusepath{clip}%
\pgfsetbuttcap%
\pgfsetroundjoin%
\definecolor{currentfill}{rgb}{0.000000,0.000000,0.000000}%
\pgfsetfillcolor{currentfill}%
\pgfsetlinewidth{1.003750pt}%
\definecolor{currentstroke}{rgb}{0.000000,0.000000,0.000000}%
\pgfsetstrokecolor{currentstroke}%
\pgfsetdash{}{0pt}%
\pgfpathmoveto{\pgfqpoint{2.518786in}{1.771040in}}%
\pgfpathcurveto{\pgfqpoint{2.529836in}{1.771040in}}{\pgfqpoint{2.540435in}{1.775431in}}{\pgfqpoint{2.548249in}{1.783244in}}%
\pgfpathcurveto{\pgfqpoint{2.556062in}{1.791058in}}{\pgfqpoint{2.560452in}{1.801657in}}{\pgfqpoint{2.560452in}{1.812707in}}%
\pgfpathcurveto{\pgfqpoint{2.560452in}{1.823757in}}{\pgfqpoint{2.556062in}{1.834356in}}{\pgfqpoint{2.548249in}{1.842170in}}%
\pgfpathcurveto{\pgfqpoint{2.540435in}{1.849983in}}{\pgfqpoint{2.529836in}{1.854374in}}{\pgfqpoint{2.518786in}{1.854374in}}%
\pgfpathcurveto{\pgfqpoint{2.507736in}{1.854374in}}{\pgfqpoint{2.497137in}{1.849983in}}{\pgfqpoint{2.489323in}{1.842170in}}%
\pgfpathcurveto{\pgfqpoint{2.481509in}{1.834356in}}{\pgfqpoint{2.477119in}{1.823757in}}{\pgfqpoint{2.477119in}{1.812707in}}%
\pgfpathcurveto{\pgfqpoint{2.477119in}{1.801657in}}{\pgfqpoint{2.481509in}{1.791058in}}{\pgfqpoint{2.489323in}{1.783244in}}%
\pgfpathcurveto{\pgfqpoint{2.497137in}{1.775431in}}{\pgfqpoint{2.507736in}{1.771040in}}{\pgfqpoint{2.518786in}{1.771040in}}%
\pgfpathclose%
\pgfusepath{stroke,fill}%
\end{pgfscope}%
\begin{pgfscope}%
\pgfpathrectangle{\pgfqpoint{0.800000in}{0.528000in}}{\pgfqpoint{4.960000in}{3.696000in}}%
\pgfusepath{clip}%
\pgfsetbuttcap%
\pgfsetroundjoin%
\definecolor{currentfill}{rgb}{0.000000,0.000000,0.000000}%
\pgfsetfillcolor{currentfill}%
\pgfsetlinewidth{1.003750pt}%
\definecolor{currentstroke}{rgb}{0.000000,0.000000,0.000000}%
\pgfsetstrokecolor{currentstroke}%
\pgfsetdash{}{0pt}%
\pgfpathmoveto{\pgfqpoint{2.518786in}{1.771040in}}%
\pgfpathcurveto{\pgfqpoint{2.529836in}{1.771040in}}{\pgfqpoint{2.540435in}{1.775431in}}{\pgfqpoint{2.548249in}{1.783244in}}%
\pgfpathcurveto{\pgfqpoint{2.556062in}{1.791058in}}{\pgfqpoint{2.560452in}{1.801657in}}{\pgfqpoint{2.560452in}{1.812707in}}%
\pgfpathcurveto{\pgfqpoint{2.560452in}{1.823757in}}{\pgfqpoint{2.556062in}{1.834356in}}{\pgfqpoint{2.548249in}{1.842170in}}%
\pgfpathcurveto{\pgfqpoint{2.540435in}{1.849983in}}{\pgfqpoint{2.529836in}{1.854374in}}{\pgfqpoint{2.518786in}{1.854374in}}%
\pgfpathcurveto{\pgfqpoint{2.507736in}{1.854374in}}{\pgfqpoint{2.497137in}{1.849983in}}{\pgfqpoint{2.489323in}{1.842170in}}%
\pgfpathcurveto{\pgfqpoint{2.481509in}{1.834356in}}{\pgfqpoint{2.477119in}{1.823757in}}{\pgfqpoint{2.477119in}{1.812707in}}%
\pgfpathcurveto{\pgfqpoint{2.477119in}{1.801657in}}{\pgfqpoint{2.481509in}{1.791058in}}{\pgfqpoint{2.489323in}{1.783244in}}%
\pgfpathcurveto{\pgfqpoint{2.497137in}{1.775431in}}{\pgfqpoint{2.507736in}{1.771040in}}{\pgfqpoint{2.518786in}{1.771040in}}%
\pgfpathclose%
\pgfusepath{stroke,fill}%
\end{pgfscope}%
\begin{pgfscope}%
\pgfpathrectangle{\pgfqpoint{0.800000in}{0.528000in}}{\pgfqpoint{4.960000in}{3.696000in}}%
\pgfusepath{clip}%
\pgfsetbuttcap%
\pgfsetroundjoin%
\definecolor{currentfill}{rgb}{0.000000,0.000000,0.000000}%
\pgfsetfillcolor{currentfill}%
\pgfsetlinewidth{1.003750pt}%
\definecolor{currentstroke}{rgb}{0.000000,0.000000,0.000000}%
\pgfsetstrokecolor{currentstroke}%
\pgfsetdash{}{0pt}%
\pgfpathmoveto{\pgfqpoint{2.518786in}{1.771040in}}%
\pgfpathcurveto{\pgfqpoint{2.529836in}{1.771040in}}{\pgfqpoint{2.540435in}{1.775431in}}{\pgfqpoint{2.548249in}{1.783244in}}%
\pgfpathcurveto{\pgfqpoint{2.556062in}{1.791058in}}{\pgfqpoint{2.560452in}{1.801657in}}{\pgfqpoint{2.560452in}{1.812707in}}%
\pgfpathcurveto{\pgfqpoint{2.560452in}{1.823757in}}{\pgfqpoint{2.556062in}{1.834356in}}{\pgfqpoint{2.548249in}{1.842170in}}%
\pgfpathcurveto{\pgfqpoint{2.540435in}{1.849983in}}{\pgfqpoint{2.529836in}{1.854374in}}{\pgfqpoint{2.518786in}{1.854374in}}%
\pgfpathcurveto{\pgfqpoint{2.507736in}{1.854374in}}{\pgfqpoint{2.497137in}{1.849983in}}{\pgfqpoint{2.489323in}{1.842170in}}%
\pgfpathcurveto{\pgfqpoint{2.481509in}{1.834356in}}{\pgfqpoint{2.477119in}{1.823757in}}{\pgfqpoint{2.477119in}{1.812707in}}%
\pgfpathcurveto{\pgfqpoint{2.477119in}{1.801657in}}{\pgfqpoint{2.481509in}{1.791058in}}{\pgfqpoint{2.489323in}{1.783244in}}%
\pgfpathcurveto{\pgfqpoint{2.497137in}{1.775431in}}{\pgfqpoint{2.507736in}{1.771040in}}{\pgfqpoint{2.518786in}{1.771040in}}%
\pgfpathclose%
\pgfusepath{stroke,fill}%
\end{pgfscope}%
\begin{pgfscope}%
\pgfpathrectangle{\pgfqpoint{0.800000in}{0.528000in}}{\pgfqpoint{4.960000in}{3.696000in}}%
\pgfusepath{clip}%
\pgfsetbuttcap%
\pgfsetroundjoin%
\definecolor{currentfill}{rgb}{0.000000,0.000000,0.000000}%
\pgfsetfillcolor{currentfill}%
\pgfsetlinewidth{1.003750pt}%
\definecolor{currentstroke}{rgb}{0.000000,0.000000,0.000000}%
\pgfsetstrokecolor{currentstroke}%
\pgfsetdash{}{0pt}%
\pgfpathmoveto{\pgfqpoint{2.518786in}{1.771040in}}%
\pgfpathcurveto{\pgfqpoint{2.529836in}{1.771040in}}{\pgfqpoint{2.540435in}{1.775431in}}{\pgfqpoint{2.548249in}{1.783244in}}%
\pgfpathcurveto{\pgfqpoint{2.556062in}{1.791058in}}{\pgfqpoint{2.560452in}{1.801657in}}{\pgfqpoint{2.560452in}{1.812707in}}%
\pgfpathcurveto{\pgfqpoint{2.560452in}{1.823757in}}{\pgfqpoint{2.556062in}{1.834356in}}{\pgfqpoint{2.548249in}{1.842170in}}%
\pgfpathcurveto{\pgfqpoint{2.540435in}{1.849983in}}{\pgfqpoint{2.529836in}{1.854374in}}{\pgfqpoint{2.518786in}{1.854374in}}%
\pgfpathcurveto{\pgfqpoint{2.507736in}{1.854374in}}{\pgfqpoint{2.497137in}{1.849983in}}{\pgfqpoint{2.489323in}{1.842170in}}%
\pgfpathcurveto{\pgfqpoint{2.481509in}{1.834356in}}{\pgfqpoint{2.477119in}{1.823757in}}{\pgfqpoint{2.477119in}{1.812707in}}%
\pgfpathcurveto{\pgfqpoint{2.477119in}{1.801657in}}{\pgfqpoint{2.481509in}{1.791058in}}{\pgfqpoint{2.489323in}{1.783244in}}%
\pgfpathcurveto{\pgfqpoint{2.497137in}{1.775431in}}{\pgfqpoint{2.507736in}{1.771040in}}{\pgfqpoint{2.518786in}{1.771040in}}%
\pgfpathclose%
\pgfusepath{stroke,fill}%
\end{pgfscope}%
\begin{pgfscope}%
\pgfpathrectangle{\pgfqpoint{0.800000in}{0.528000in}}{\pgfqpoint{4.960000in}{3.696000in}}%
\pgfusepath{clip}%
\pgfsetbuttcap%
\pgfsetroundjoin%
\definecolor{currentfill}{rgb}{0.000000,0.000000,0.000000}%
\pgfsetfillcolor{currentfill}%
\pgfsetlinewidth{1.003750pt}%
\definecolor{currentstroke}{rgb}{0.000000,0.000000,0.000000}%
\pgfsetstrokecolor{currentstroke}%
\pgfsetdash{}{0pt}%
\pgfpathmoveto{\pgfqpoint{2.518786in}{1.771040in}}%
\pgfpathcurveto{\pgfqpoint{2.529836in}{1.771040in}}{\pgfqpoint{2.540435in}{1.775431in}}{\pgfqpoint{2.548249in}{1.783244in}}%
\pgfpathcurveto{\pgfqpoint{2.556062in}{1.791058in}}{\pgfqpoint{2.560452in}{1.801657in}}{\pgfqpoint{2.560452in}{1.812707in}}%
\pgfpathcurveto{\pgfqpoint{2.560452in}{1.823757in}}{\pgfqpoint{2.556062in}{1.834356in}}{\pgfqpoint{2.548249in}{1.842170in}}%
\pgfpathcurveto{\pgfqpoint{2.540435in}{1.849983in}}{\pgfqpoint{2.529836in}{1.854374in}}{\pgfqpoint{2.518786in}{1.854374in}}%
\pgfpathcurveto{\pgfqpoint{2.507736in}{1.854374in}}{\pgfqpoint{2.497137in}{1.849983in}}{\pgfqpoint{2.489323in}{1.842170in}}%
\pgfpathcurveto{\pgfqpoint{2.481509in}{1.834356in}}{\pgfqpoint{2.477119in}{1.823757in}}{\pgfqpoint{2.477119in}{1.812707in}}%
\pgfpathcurveto{\pgfqpoint{2.477119in}{1.801657in}}{\pgfqpoint{2.481509in}{1.791058in}}{\pgfqpoint{2.489323in}{1.783244in}}%
\pgfpathcurveto{\pgfqpoint{2.497137in}{1.775431in}}{\pgfqpoint{2.507736in}{1.771040in}}{\pgfqpoint{2.518786in}{1.771040in}}%
\pgfpathclose%
\pgfusepath{stroke,fill}%
\end{pgfscope}%
\begin{pgfscope}%
\pgfpathrectangle{\pgfqpoint{0.800000in}{0.528000in}}{\pgfqpoint{4.960000in}{3.696000in}}%
\pgfusepath{clip}%
\pgfsetbuttcap%
\pgfsetroundjoin%
\definecolor{currentfill}{rgb}{0.000000,0.000000,0.000000}%
\pgfsetfillcolor{currentfill}%
\pgfsetlinewidth{1.003750pt}%
\definecolor{currentstroke}{rgb}{0.000000,0.000000,0.000000}%
\pgfsetstrokecolor{currentstroke}%
\pgfsetdash{}{0pt}%
\pgfpathmoveto{\pgfqpoint{2.518786in}{1.771040in}}%
\pgfpathcurveto{\pgfqpoint{2.529836in}{1.771040in}}{\pgfqpoint{2.540435in}{1.775431in}}{\pgfqpoint{2.548249in}{1.783244in}}%
\pgfpathcurveto{\pgfqpoint{2.556062in}{1.791058in}}{\pgfqpoint{2.560452in}{1.801657in}}{\pgfqpoint{2.560452in}{1.812707in}}%
\pgfpathcurveto{\pgfqpoint{2.560452in}{1.823757in}}{\pgfqpoint{2.556062in}{1.834356in}}{\pgfqpoint{2.548249in}{1.842170in}}%
\pgfpathcurveto{\pgfqpoint{2.540435in}{1.849983in}}{\pgfqpoint{2.529836in}{1.854374in}}{\pgfqpoint{2.518786in}{1.854374in}}%
\pgfpathcurveto{\pgfqpoint{2.507736in}{1.854374in}}{\pgfqpoint{2.497137in}{1.849983in}}{\pgfqpoint{2.489323in}{1.842170in}}%
\pgfpathcurveto{\pgfqpoint{2.481509in}{1.834356in}}{\pgfqpoint{2.477119in}{1.823757in}}{\pgfqpoint{2.477119in}{1.812707in}}%
\pgfpathcurveto{\pgfqpoint{2.477119in}{1.801657in}}{\pgfqpoint{2.481509in}{1.791058in}}{\pgfqpoint{2.489323in}{1.783244in}}%
\pgfpathcurveto{\pgfqpoint{2.497137in}{1.775431in}}{\pgfqpoint{2.507736in}{1.771040in}}{\pgfqpoint{2.518786in}{1.771040in}}%
\pgfpathclose%
\pgfusepath{stroke,fill}%
\end{pgfscope}%
\begin{pgfscope}%
\pgfpathrectangle{\pgfqpoint{0.800000in}{0.528000in}}{\pgfqpoint{4.960000in}{3.696000in}}%
\pgfusepath{clip}%
\pgfsetbuttcap%
\pgfsetroundjoin%
\definecolor{currentfill}{rgb}{0.000000,0.000000,0.000000}%
\pgfsetfillcolor{currentfill}%
\pgfsetlinewidth{1.003750pt}%
\definecolor{currentstroke}{rgb}{0.000000,0.000000,0.000000}%
\pgfsetstrokecolor{currentstroke}%
\pgfsetdash{}{0pt}%
\pgfpathmoveto{\pgfqpoint{2.518786in}{1.771040in}}%
\pgfpathcurveto{\pgfqpoint{2.529836in}{1.771040in}}{\pgfqpoint{2.540435in}{1.775431in}}{\pgfqpoint{2.548249in}{1.783244in}}%
\pgfpathcurveto{\pgfqpoint{2.556062in}{1.791058in}}{\pgfqpoint{2.560452in}{1.801657in}}{\pgfqpoint{2.560452in}{1.812707in}}%
\pgfpathcurveto{\pgfqpoint{2.560452in}{1.823757in}}{\pgfqpoint{2.556062in}{1.834356in}}{\pgfqpoint{2.548249in}{1.842170in}}%
\pgfpathcurveto{\pgfqpoint{2.540435in}{1.849983in}}{\pgfqpoint{2.529836in}{1.854374in}}{\pgfqpoint{2.518786in}{1.854374in}}%
\pgfpathcurveto{\pgfqpoint{2.507736in}{1.854374in}}{\pgfqpoint{2.497137in}{1.849983in}}{\pgfqpoint{2.489323in}{1.842170in}}%
\pgfpathcurveto{\pgfqpoint{2.481509in}{1.834356in}}{\pgfqpoint{2.477119in}{1.823757in}}{\pgfqpoint{2.477119in}{1.812707in}}%
\pgfpathcurveto{\pgfqpoint{2.477119in}{1.801657in}}{\pgfqpoint{2.481509in}{1.791058in}}{\pgfqpoint{2.489323in}{1.783244in}}%
\pgfpathcurveto{\pgfqpoint{2.497137in}{1.775431in}}{\pgfqpoint{2.507736in}{1.771040in}}{\pgfqpoint{2.518786in}{1.771040in}}%
\pgfpathclose%
\pgfusepath{stroke,fill}%
\end{pgfscope}%
\begin{pgfscope}%
\pgfpathrectangle{\pgfqpoint{0.800000in}{0.528000in}}{\pgfqpoint{4.960000in}{3.696000in}}%
\pgfusepath{clip}%
\pgfsetbuttcap%
\pgfsetroundjoin%
\definecolor{currentfill}{rgb}{0.000000,0.000000,0.000000}%
\pgfsetfillcolor{currentfill}%
\pgfsetlinewidth{1.003750pt}%
\definecolor{currentstroke}{rgb}{0.000000,0.000000,0.000000}%
\pgfsetstrokecolor{currentstroke}%
\pgfsetdash{}{0pt}%
\pgfpathmoveto{\pgfqpoint{2.518786in}{1.771040in}}%
\pgfpathcurveto{\pgfqpoint{2.529836in}{1.771040in}}{\pgfqpoint{2.540435in}{1.775431in}}{\pgfqpoint{2.548249in}{1.783244in}}%
\pgfpathcurveto{\pgfqpoint{2.556062in}{1.791058in}}{\pgfqpoint{2.560452in}{1.801657in}}{\pgfqpoint{2.560452in}{1.812707in}}%
\pgfpathcurveto{\pgfqpoint{2.560452in}{1.823757in}}{\pgfqpoint{2.556062in}{1.834356in}}{\pgfqpoint{2.548249in}{1.842170in}}%
\pgfpathcurveto{\pgfqpoint{2.540435in}{1.849983in}}{\pgfqpoint{2.529836in}{1.854374in}}{\pgfqpoint{2.518786in}{1.854374in}}%
\pgfpathcurveto{\pgfqpoint{2.507736in}{1.854374in}}{\pgfqpoint{2.497137in}{1.849983in}}{\pgfqpoint{2.489323in}{1.842170in}}%
\pgfpathcurveto{\pgfqpoint{2.481509in}{1.834356in}}{\pgfqpoint{2.477119in}{1.823757in}}{\pgfqpoint{2.477119in}{1.812707in}}%
\pgfpathcurveto{\pgfqpoint{2.477119in}{1.801657in}}{\pgfqpoint{2.481509in}{1.791058in}}{\pgfqpoint{2.489323in}{1.783244in}}%
\pgfpathcurveto{\pgfqpoint{2.497137in}{1.775431in}}{\pgfqpoint{2.507736in}{1.771040in}}{\pgfqpoint{2.518786in}{1.771040in}}%
\pgfpathclose%
\pgfusepath{stroke,fill}%
\end{pgfscope}%
\begin{pgfscope}%
\pgfpathrectangle{\pgfqpoint{0.800000in}{0.528000in}}{\pgfqpoint{4.960000in}{3.696000in}}%
\pgfusepath{clip}%
\pgfsetbuttcap%
\pgfsetroundjoin%
\definecolor{currentfill}{rgb}{0.000000,0.000000,0.000000}%
\pgfsetfillcolor{currentfill}%
\pgfsetlinewidth{1.003750pt}%
\definecolor{currentstroke}{rgb}{0.000000,0.000000,0.000000}%
\pgfsetstrokecolor{currentstroke}%
\pgfsetdash{}{0pt}%
\pgfpathmoveto{\pgfqpoint{2.518786in}{1.771040in}}%
\pgfpathcurveto{\pgfqpoint{2.529836in}{1.771040in}}{\pgfqpoint{2.540435in}{1.775431in}}{\pgfqpoint{2.548249in}{1.783244in}}%
\pgfpathcurveto{\pgfqpoint{2.556062in}{1.791058in}}{\pgfqpoint{2.560452in}{1.801657in}}{\pgfqpoint{2.560452in}{1.812707in}}%
\pgfpathcurveto{\pgfqpoint{2.560452in}{1.823757in}}{\pgfqpoint{2.556062in}{1.834356in}}{\pgfqpoint{2.548249in}{1.842170in}}%
\pgfpathcurveto{\pgfqpoint{2.540435in}{1.849983in}}{\pgfqpoint{2.529836in}{1.854374in}}{\pgfqpoint{2.518786in}{1.854374in}}%
\pgfpathcurveto{\pgfqpoint{2.507736in}{1.854374in}}{\pgfqpoint{2.497137in}{1.849983in}}{\pgfqpoint{2.489323in}{1.842170in}}%
\pgfpathcurveto{\pgfqpoint{2.481509in}{1.834356in}}{\pgfqpoint{2.477119in}{1.823757in}}{\pgfqpoint{2.477119in}{1.812707in}}%
\pgfpathcurveto{\pgfqpoint{2.477119in}{1.801657in}}{\pgfqpoint{2.481509in}{1.791058in}}{\pgfqpoint{2.489323in}{1.783244in}}%
\pgfpathcurveto{\pgfqpoint{2.497137in}{1.775431in}}{\pgfqpoint{2.507736in}{1.771040in}}{\pgfqpoint{2.518786in}{1.771040in}}%
\pgfpathclose%
\pgfusepath{stroke,fill}%
\end{pgfscope}%
\begin{pgfscope}%
\pgfpathrectangle{\pgfqpoint{0.800000in}{0.528000in}}{\pgfqpoint{4.960000in}{3.696000in}}%
\pgfusepath{clip}%
\pgfsetbuttcap%
\pgfsetroundjoin%
\definecolor{currentfill}{rgb}{0.000000,0.000000,0.000000}%
\pgfsetfillcolor{currentfill}%
\pgfsetlinewidth{1.003750pt}%
\definecolor{currentstroke}{rgb}{0.000000,0.000000,0.000000}%
\pgfsetstrokecolor{currentstroke}%
\pgfsetdash{}{0pt}%
\pgfpathmoveto{\pgfqpoint{4.011666in}{2.877687in}}%
\pgfpathcurveto{\pgfqpoint{4.022716in}{2.877687in}}{\pgfqpoint{4.033315in}{2.882077in}}{\pgfqpoint{4.041128in}{2.889891in}}%
\pgfpathcurveto{\pgfqpoint{4.048942in}{2.897704in}}{\pgfqpoint{4.053332in}{2.908303in}}{\pgfqpoint{4.053332in}{2.919353in}}%
\pgfpathcurveto{\pgfqpoint{4.053332in}{2.930404in}}{\pgfqpoint{4.048942in}{2.941003in}}{\pgfqpoint{4.041128in}{2.948816in}}%
\pgfpathcurveto{\pgfqpoint{4.033315in}{2.956630in}}{\pgfqpoint{4.022716in}{2.961020in}}{\pgfqpoint{4.011666in}{2.961020in}}%
\pgfpathcurveto{\pgfqpoint{4.000616in}{2.961020in}}{\pgfqpoint{3.990016in}{2.956630in}}{\pgfqpoint{3.982203in}{2.948816in}}%
\pgfpathcurveto{\pgfqpoint{3.974389in}{2.941003in}}{\pgfqpoint{3.969999in}{2.930404in}}{\pgfqpoint{3.969999in}{2.919353in}}%
\pgfpathcurveto{\pgfqpoint{3.969999in}{2.908303in}}{\pgfqpoint{3.974389in}{2.897704in}}{\pgfqpoint{3.982203in}{2.889891in}}%
\pgfpathcurveto{\pgfqpoint{3.990016in}{2.882077in}}{\pgfqpoint{4.000616in}{2.877687in}}{\pgfqpoint{4.011666in}{2.877687in}}%
\pgfpathclose%
\pgfusepath{stroke,fill}%
\end{pgfscope}%
\begin{pgfscope}%
\pgfpathrectangle{\pgfqpoint{0.800000in}{0.528000in}}{\pgfqpoint{4.960000in}{3.696000in}}%
\pgfusepath{clip}%
\pgfsetbuttcap%
\pgfsetroundjoin%
\definecolor{currentfill}{rgb}{0.000000,0.000000,0.000000}%
\pgfsetfillcolor{currentfill}%
\pgfsetlinewidth{1.003750pt}%
\definecolor{currentstroke}{rgb}{0.000000,0.000000,0.000000}%
\pgfsetstrokecolor{currentstroke}%
\pgfsetdash{}{0pt}%
\pgfpathmoveto{\pgfqpoint{4.011666in}{1.771040in}}%
\pgfpathcurveto{\pgfqpoint{4.022716in}{1.771040in}}{\pgfqpoint{4.033315in}{1.775431in}}{\pgfqpoint{4.041128in}{1.783244in}}%
\pgfpathcurveto{\pgfqpoint{4.048942in}{1.791058in}}{\pgfqpoint{4.053332in}{1.801657in}}{\pgfqpoint{4.053332in}{1.812707in}}%
\pgfpathcurveto{\pgfqpoint{4.053332in}{1.823757in}}{\pgfqpoint{4.048942in}{1.834356in}}{\pgfqpoint{4.041128in}{1.842170in}}%
\pgfpathcurveto{\pgfqpoint{4.033315in}{1.849983in}}{\pgfqpoint{4.022716in}{1.854374in}}{\pgfqpoint{4.011666in}{1.854374in}}%
\pgfpathcurveto{\pgfqpoint{4.000616in}{1.854374in}}{\pgfqpoint{3.990016in}{1.849983in}}{\pgfqpoint{3.982203in}{1.842170in}}%
\pgfpathcurveto{\pgfqpoint{3.974389in}{1.834356in}}{\pgfqpoint{3.969999in}{1.823757in}}{\pgfqpoint{3.969999in}{1.812707in}}%
\pgfpathcurveto{\pgfqpoint{3.969999in}{1.801657in}}{\pgfqpoint{3.974389in}{1.791058in}}{\pgfqpoint{3.982203in}{1.783244in}}%
\pgfpathcurveto{\pgfqpoint{3.990016in}{1.775431in}}{\pgfqpoint{4.000616in}{1.771040in}}{\pgfqpoint{4.011666in}{1.771040in}}%
\pgfpathclose%
\pgfusepath{stroke,fill}%
\end{pgfscope}%
\begin{pgfscope}%
\pgfpathrectangle{\pgfqpoint{0.800000in}{0.528000in}}{\pgfqpoint{4.960000in}{3.696000in}}%
\pgfusepath{clip}%
\pgfsetbuttcap%
\pgfsetroundjoin%
\definecolor{currentfill}{rgb}{0.000000,0.000000,0.000000}%
\pgfsetfillcolor{currentfill}%
\pgfsetlinewidth{1.003750pt}%
\definecolor{currentstroke}{rgb}{0.000000,0.000000,0.000000}%
\pgfsetstrokecolor{currentstroke}%
\pgfsetdash{}{0pt}%
\pgfpathmoveto{\pgfqpoint{4.011666in}{1.771040in}}%
\pgfpathcurveto{\pgfqpoint{4.022716in}{1.771040in}}{\pgfqpoint{4.033315in}{1.775431in}}{\pgfqpoint{4.041128in}{1.783244in}}%
\pgfpathcurveto{\pgfqpoint{4.048942in}{1.791058in}}{\pgfqpoint{4.053332in}{1.801657in}}{\pgfqpoint{4.053332in}{1.812707in}}%
\pgfpathcurveto{\pgfqpoint{4.053332in}{1.823757in}}{\pgfqpoint{4.048942in}{1.834356in}}{\pgfqpoint{4.041128in}{1.842170in}}%
\pgfpathcurveto{\pgfqpoint{4.033315in}{1.849983in}}{\pgfqpoint{4.022716in}{1.854374in}}{\pgfqpoint{4.011666in}{1.854374in}}%
\pgfpathcurveto{\pgfqpoint{4.000616in}{1.854374in}}{\pgfqpoint{3.990016in}{1.849983in}}{\pgfqpoint{3.982203in}{1.842170in}}%
\pgfpathcurveto{\pgfqpoint{3.974389in}{1.834356in}}{\pgfqpoint{3.969999in}{1.823757in}}{\pgfqpoint{3.969999in}{1.812707in}}%
\pgfpathcurveto{\pgfqpoint{3.969999in}{1.801657in}}{\pgfqpoint{3.974389in}{1.791058in}}{\pgfqpoint{3.982203in}{1.783244in}}%
\pgfpathcurveto{\pgfqpoint{3.990016in}{1.775431in}}{\pgfqpoint{4.000616in}{1.771040in}}{\pgfqpoint{4.011666in}{1.771040in}}%
\pgfpathclose%
\pgfusepath{stroke,fill}%
\end{pgfscope}%
\begin{pgfscope}%
\pgfpathrectangle{\pgfqpoint{0.800000in}{0.528000in}}{\pgfqpoint{4.960000in}{3.696000in}}%
\pgfusepath{clip}%
\pgfsetbuttcap%
\pgfsetroundjoin%
\definecolor{currentfill}{rgb}{0.000000,0.000000,0.000000}%
\pgfsetfillcolor{currentfill}%
\pgfsetlinewidth{1.003750pt}%
\definecolor{currentstroke}{rgb}{0.000000,0.000000,0.000000}%
\pgfsetstrokecolor{currentstroke}%
\pgfsetdash{}{0pt}%
\pgfpathmoveto{\pgfqpoint{4.011666in}{1.771040in}}%
\pgfpathcurveto{\pgfqpoint{4.022716in}{1.771040in}}{\pgfqpoint{4.033315in}{1.775431in}}{\pgfqpoint{4.041128in}{1.783244in}}%
\pgfpathcurveto{\pgfqpoint{4.048942in}{1.791058in}}{\pgfqpoint{4.053332in}{1.801657in}}{\pgfqpoint{4.053332in}{1.812707in}}%
\pgfpathcurveto{\pgfqpoint{4.053332in}{1.823757in}}{\pgfqpoint{4.048942in}{1.834356in}}{\pgfqpoint{4.041128in}{1.842170in}}%
\pgfpathcurveto{\pgfqpoint{4.033315in}{1.849983in}}{\pgfqpoint{4.022716in}{1.854374in}}{\pgfqpoint{4.011666in}{1.854374in}}%
\pgfpathcurveto{\pgfqpoint{4.000616in}{1.854374in}}{\pgfqpoint{3.990016in}{1.849983in}}{\pgfqpoint{3.982203in}{1.842170in}}%
\pgfpathcurveto{\pgfqpoint{3.974389in}{1.834356in}}{\pgfqpoint{3.969999in}{1.823757in}}{\pgfqpoint{3.969999in}{1.812707in}}%
\pgfpathcurveto{\pgfqpoint{3.969999in}{1.801657in}}{\pgfqpoint{3.974389in}{1.791058in}}{\pgfqpoint{3.982203in}{1.783244in}}%
\pgfpathcurveto{\pgfqpoint{3.990016in}{1.775431in}}{\pgfqpoint{4.000616in}{1.771040in}}{\pgfqpoint{4.011666in}{1.771040in}}%
\pgfpathclose%
\pgfusepath{stroke,fill}%
\end{pgfscope}%
\begin{pgfscope}%
\pgfpathrectangle{\pgfqpoint{0.800000in}{0.528000in}}{\pgfqpoint{4.960000in}{3.696000in}}%
\pgfusepath{clip}%
\pgfsetbuttcap%
\pgfsetroundjoin%
\definecolor{currentfill}{rgb}{0.000000,0.000000,0.000000}%
\pgfsetfillcolor{currentfill}%
\pgfsetlinewidth{1.003750pt}%
\definecolor{currentstroke}{rgb}{0.000000,0.000000,0.000000}%
\pgfsetstrokecolor{currentstroke}%
\pgfsetdash{}{0pt}%
\pgfpathmoveto{\pgfqpoint{4.011666in}{2.877687in}}%
\pgfpathcurveto{\pgfqpoint{4.022716in}{2.877687in}}{\pgfqpoint{4.033315in}{2.882077in}}{\pgfqpoint{4.041128in}{2.889891in}}%
\pgfpathcurveto{\pgfqpoint{4.048942in}{2.897704in}}{\pgfqpoint{4.053332in}{2.908303in}}{\pgfqpoint{4.053332in}{2.919353in}}%
\pgfpathcurveto{\pgfqpoint{4.053332in}{2.930404in}}{\pgfqpoint{4.048942in}{2.941003in}}{\pgfqpoint{4.041128in}{2.948816in}}%
\pgfpathcurveto{\pgfqpoint{4.033315in}{2.956630in}}{\pgfqpoint{4.022716in}{2.961020in}}{\pgfqpoint{4.011666in}{2.961020in}}%
\pgfpathcurveto{\pgfqpoint{4.000616in}{2.961020in}}{\pgfqpoint{3.990016in}{2.956630in}}{\pgfqpoint{3.982203in}{2.948816in}}%
\pgfpathcurveto{\pgfqpoint{3.974389in}{2.941003in}}{\pgfqpoint{3.969999in}{2.930404in}}{\pgfqpoint{3.969999in}{2.919353in}}%
\pgfpathcurveto{\pgfqpoint{3.969999in}{2.908303in}}{\pgfqpoint{3.974389in}{2.897704in}}{\pgfqpoint{3.982203in}{2.889891in}}%
\pgfpathcurveto{\pgfqpoint{3.990016in}{2.882077in}}{\pgfqpoint{4.000616in}{2.877687in}}{\pgfqpoint{4.011666in}{2.877687in}}%
\pgfpathclose%
\pgfusepath{stroke,fill}%
\end{pgfscope}%
\begin{pgfscope}%
\pgfpathrectangle{\pgfqpoint{0.800000in}{0.528000in}}{\pgfqpoint{4.960000in}{3.696000in}}%
\pgfusepath{clip}%
\pgfsetbuttcap%
\pgfsetroundjoin%
\definecolor{currentfill}{rgb}{0.000000,0.000000,0.000000}%
\pgfsetfillcolor{currentfill}%
\pgfsetlinewidth{1.003750pt}%
\definecolor{currentstroke}{rgb}{0.000000,0.000000,0.000000}%
\pgfsetstrokecolor{currentstroke}%
\pgfsetdash{}{0pt}%
\pgfpathmoveto{\pgfqpoint{4.011666in}{1.771040in}}%
\pgfpathcurveto{\pgfqpoint{4.022716in}{1.771040in}}{\pgfqpoint{4.033315in}{1.775431in}}{\pgfqpoint{4.041128in}{1.783244in}}%
\pgfpathcurveto{\pgfqpoint{4.048942in}{1.791058in}}{\pgfqpoint{4.053332in}{1.801657in}}{\pgfqpoint{4.053332in}{1.812707in}}%
\pgfpathcurveto{\pgfqpoint{4.053332in}{1.823757in}}{\pgfqpoint{4.048942in}{1.834356in}}{\pgfqpoint{4.041128in}{1.842170in}}%
\pgfpathcurveto{\pgfqpoint{4.033315in}{1.849983in}}{\pgfqpoint{4.022716in}{1.854374in}}{\pgfqpoint{4.011666in}{1.854374in}}%
\pgfpathcurveto{\pgfqpoint{4.000616in}{1.854374in}}{\pgfqpoint{3.990016in}{1.849983in}}{\pgfqpoint{3.982203in}{1.842170in}}%
\pgfpathcurveto{\pgfqpoint{3.974389in}{1.834356in}}{\pgfqpoint{3.969999in}{1.823757in}}{\pgfqpoint{3.969999in}{1.812707in}}%
\pgfpathcurveto{\pgfqpoint{3.969999in}{1.801657in}}{\pgfqpoint{3.974389in}{1.791058in}}{\pgfqpoint{3.982203in}{1.783244in}}%
\pgfpathcurveto{\pgfqpoint{3.990016in}{1.775431in}}{\pgfqpoint{4.000616in}{1.771040in}}{\pgfqpoint{4.011666in}{1.771040in}}%
\pgfpathclose%
\pgfusepath{stroke,fill}%
\end{pgfscope}%
\begin{pgfscope}%
\pgfpathrectangle{\pgfqpoint{0.800000in}{0.528000in}}{\pgfqpoint{4.960000in}{3.696000in}}%
\pgfusepath{clip}%
\pgfsetbuttcap%
\pgfsetroundjoin%
\definecolor{currentfill}{rgb}{0.000000,0.000000,0.000000}%
\pgfsetfillcolor{currentfill}%
\pgfsetlinewidth{1.003750pt}%
\definecolor{currentstroke}{rgb}{0.000000,0.000000,0.000000}%
\pgfsetstrokecolor{currentstroke}%
\pgfsetdash{}{0pt}%
\pgfpathmoveto{\pgfqpoint{4.011666in}{1.771040in}}%
\pgfpathcurveto{\pgfqpoint{4.022716in}{1.771040in}}{\pgfqpoint{4.033315in}{1.775431in}}{\pgfqpoint{4.041128in}{1.783244in}}%
\pgfpathcurveto{\pgfqpoint{4.048942in}{1.791058in}}{\pgfqpoint{4.053332in}{1.801657in}}{\pgfqpoint{4.053332in}{1.812707in}}%
\pgfpathcurveto{\pgfqpoint{4.053332in}{1.823757in}}{\pgfqpoint{4.048942in}{1.834356in}}{\pgfqpoint{4.041128in}{1.842170in}}%
\pgfpathcurveto{\pgfqpoint{4.033315in}{1.849983in}}{\pgfqpoint{4.022716in}{1.854374in}}{\pgfqpoint{4.011666in}{1.854374in}}%
\pgfpathcurveto{\pgfqpoint{4.000616in}{1.854374in}}{\pgfqpoint{3.990016in}{1.849983in}}{\pgfqpoint{3.982203in}{1.842170in}}%
\pgfpathcurveto{\pgfqpoint{3.974389in}{1.834356in}}{\pgfqpoint{3.969999in}{1.823757in}}{\pgfqpoint{3.969999in}{1.812707in}}%
\pgfpathcurveto{\pgfqpoint{3.969999in}{1.801657in}}{\pgfqpoint{3.974389in}{1.791058in}}{\pgfqpoint{3.982203in}{1.783244in}}%
\pgfpathcurveto{\pgfqpoint{3.990016in}{1.775431in}}{\pgfqpoint{4.000616in}{1.771040in}}{\pgfqpoint{4.011666in}{1.771040in}}%
\pgfpathclose%
\pgfusepath{stroke,fill}%
\end{pgfscope}%
\begin{pgfscope}%
\pgfpathrectangle{\pgfqpoint{0.800000in}{0.528000in}}{\pgfqpoint{4.960000in}{3.696000in}}%
\pgfusepath{clip}%
\pgfsetbuttcap%
\pgfsetroundjoin%
\definecolor{currentfill}{rgb}{0.000000,0.000000,0.000000}%
\pgfsetfillcolor{currentfill}%
\pgfsetlinewidth{1.003750pt}%
\definecolor{currentstroke}{rgb}{0.000000,0.000000,0.000000}%
\pgfsetstrokecolor{currentstroke}%
\pgfsetdash{}{0pt}%
\pgfpathmoveto{\pgfqpoint{4.011666in}{2.877687in}}%
\pgfpathcurveto{\pgfqpoint{4.022716in}{2.877687in}}{\pgfqpoint{4.033315in}{2.882077in}}{\pgfqpoint{4.041128in}{2.889891in}}%
\pgfpathcurveto{\pgfqpoint{4.048942in}{2.897704in}}{\pgfqpoint{4.053332in}{2.908303in}}{\pgfqpoint{4.053332in}{2.919353in}}%
\pgfpathcurveto{\pgfqpoint{4.053332in}{2.930404in}}{\pgfqpoint{4.048942in}{2.941003in}}{\pgfqpoint{4.041128in}{2.948816in}}%
\pgfpathcurveto{\pgfqpoint{4.033315in}{2.956630in}}{\pgfqpoint{4.022716in}{2.961020in}}{\pgfqpoint{4.011666in}{2.961020in}}%
\pgfpathcurveto{\pgfqpoint{4.000616in}{2.961020in}}{\pgfqpoint{3.990016in}{2.956630in}}{\pgfqpoint{3.982203in}{2.948816in}}%
\pgfpathcurveto{\pgfqpoint{3.974389in}{2.941003in}}{\pgfqpoint{3.969999in}{2.930404in}}{\pgfqpoint{3.969999in}{2.919353in}}%
\pgfpathcurveto{\pgfqpoint{3.969999in}{2.908303in}}{\pgfqpoint{3.974389in}{2.897704in}}{\pgfqpoint{3.982203in}{2.889891in}}%
\pgfpathcurveto{\pgfqpoint{3.990016in}{2.882077in}}{\pgfqpoint{4.000616in}{2.877687in}}{\pgfqpoint{4.011666in}{2.877687in}}%
\pgfpathclose%
\pgfusepath{stroke,fill}%
\end{pgfscope}%
\begin{pgfscope}%
\pgfpathrectangle{\pgfqpoint{0.800000in}{0.528000in}}{\pgfqpoint{4.960000in}{3.696000in}}%
\pgfusepath{clip}%
\pgfsetbuttcap%
\pgfsetroundjoin%
\definecolor{currentfill}{rgb}{0.000000,0.000000,0.000000}%
\pgfsetfillcolor{currentfill}%
\pgfsetlinewidth{1.003750pt}%
\definecolor{currentstroke}{rgb}{0.000000,0.000000,0.000000}%
\pgfsetstrokecolor{currentstroke}%
\pgfsetdash{}{0pt}%
\pgfpathmoveto{\pgfqpoint{4.011666in}{2.877687in}}%
\pgfpathcurveto{\pgfqpoint{4.022716in}{2.877687in}}{\pgfqpoint{4.033315in}{2.882077in}}{\pgfqpoint{4.041128in}{2.889891in}}%
\pgfpathcurveto{\pgfqpoint{4.048942in}{2.897704in}}{\pgfqpoint{4.053332in}{2.908303in}}{\pgfqpoint{4.053332in}{2.919353in}}%
\pgfpathcurveto{\pgfqpoint{4.053332in}{2.930404in}}{\pgfqpoint{4.048942in}{2.941003in}}{\pgfqpoint{4.041128in}{2.948816in}}%
\pgfpathcurveto{\pgfqpoint{4.033315in}{2.956630in}}{\pgfqpoint{4.022716in}{2.961020in}}{\pgfqpoint{4.011666in}{2.961020in}}%
\pgfpathcurveto{\pgfqpoint{4.000616in}{2.961020in}}{\pgfqpoint{3.990016in}{2.956630in}}{\pgfqpoint{3.982203in}{2.948816in}}%
\pgfpathcurveto{\pgfqpoint{3.974389in}{2.941003in}}{\pgfqpoint{3.969999in}{2.930404in}}{\pgfqpoint{3.969999in}{2.919353in}}%
\pgfpathcurveto{\pgfqpoint{3.969999in}{2.908303in}}{\pgfqpoint{3.974389in}{2.897704in}}{\pgfqpoint{3.982203in}{2.889891in}}%
\pgfpathcurveto{\pgfqpoint{3.990016in}{2.882077in}}{\pgfqpoint{4.000616in}{2.877687in}}{\pgfqpoint{4.011666in}{2.877687in}}%
\pgfpathclose%
\pgfusepath{stroke,fill}%
\end{pgfscope}%
\begin{pgfscope}%
\pgfpathrectangle{\pgfqpoint{0.800000in}{0.528000in}}{\pgfqpoint{4.960000in}{3.696000in}}%
\pgfusepath{clip}%
\pgfsetbuttcap%
\pgfsetroundjoin%
\definecolor{currentfill}{rgb}{0.000000,0.000000,0.000000}%
\pgfsetfillcolor{currentfill}%
\pgfsetlinewidth{1.003750pt}%
\definecolor{currentstroke}{rgb}{0.000000,0.000000,0.000000}%
\pgfsetstrokecolor{currentstroke}%
\pgfsetdash{}{0pt}%
\pgfpathmoveto{\pgfqpoint{4.011666in}{1.771040in}}%
\pgfpathcurveto{\pgfqpoint{4.022716in}{1.771040in}}{\pgfqpoint{4.033315in}{1.775431in}}{\pgfqpoint{4.041128in}{1.783244in}}%
\pgfpathcurveto{\pgfqpoint{4.048942in}{1.791058in}}{\pgfqpoint{4.053332in}{1.801657in}}{\pgfqpoint{4.053332in}{1.812707in}}%
\pgfpathcurveto{\pgfqpoint{4.053332in}{1.823757in}}{\pgfqpoint{4.048942in}{1.834356in}}{\pgfqpoint{4.041128in}{1.842170in}}%
\pgfpathcurveto{\pgfqpoint{4.033315in}{1.849983in}}{\pgfqpoint{4.022716in}{1.854374in}}{\pgfqpoint{4.011666in}{1.854374in}}%
\pgfpathcurveto{\pgfqpoint{4.000616in}{1.854374in}}{\pgfqpoint{3.990016in}{1.849983in}}{\pgfqpoint{3.982203in}{1.842170in}}%
\pgfpathcurveto{\pgfqpoint{3.974389in}{1.834356in}}{\pgfqpoint{3.969999in}{1.823757in}}{\pgfqpoint{3.969999in}{1.812707in}}%
\pgfpathcurveto{\pgfqpoint{3.969999in}{1.801657in}}{\pgfqpoint{3.974389in}{1.791058in}}{\pgfqpoint{3.982203in}{1.783244in}}%
\pgfpathcurveto{\pgfqpoint{3.990016in}{1.775431in}}{\pgfqpoint{4.000616in}{1.771040in}}{\pgfqpoint{4.011666in}{1.771040in}}%
\pgfpathclose%
\pgfusepath{stroke,fill}%
\end{pgfscope}%
\begin{pgfscope}%
\pgfpathrectangle{\pgfqpoint{0.800000in}{0.528000in}}{\pgfqpoint{4.960000in}{3.696000in}}%
\pgfusepath{clip}%
\pgfsetbuttcap%
\pgfsetroundjoin%
\definecolor{currentfill}{rgb}{0.000000,0.000000,0.000000}%
\pgfsetfillcolor{currentfill}%
\pgfsetlinewidth{1.003750pt}%
\definecolor{currentstroke}{rgb}{0.000000,0.000000,0.000000}%
\pgfsetstrokecolor{currentstroke}%
\pgfsetdash{}{0pt}%
\pgfpathmoveto{\pgfqpoint{4.011666in}{2.877687in}}%
\pgfpathcurveto{\pgfqpoint{4.022716in}{2.877687in}}{\pgfqpoint{4.033315in}{2.882077in}}{\pgfqpoint{4.041128in}{2.889891in}}%
\pgfpathcurveto{\pgfqpoint{4.048942in}{2.897704in}}{\pgfqpoint{4.053332in}{2.908303in}}{\pgfqpoint{4.053332in}{2.919353in}}%
\pgfpathcurveto{\pgfqpoint{4.053332in}{2.930404in}}{\pgfqpoint{4.048942in}{2.941003in}}{\pgfqpoint{4.041128in}{2.948816in}}%
\pgfpathcurveto{\pgfqpoint{4.033315in}{2.956630in}}{\pgfqpoint{4.022716in}{2.961020in}}{\pgfqpoint{4.011666in}{2.961020in}}%
\pgfpathcurveto{\pgfqpoint{4.000616in}{2.961020in}}{\pgfqpoint{3.990016in}{2.956630in}}{\pgfqpoint{3.982203in}{2.948816in}}%
\pgfpathcurveto{\pgfqpoint{3.974389in}{2.941003in}}{\pgfqpoint{3.969999in}{2.930404in}}{\pgfqpoint{3.969999in}{2.919353in}}%
\pgfpathcurveto{\pgfqpoint{3.969999in}{2.908303in}}{\pgfqpoint{3.974389in}{2.897704in}}{\pgfqpoint{3.982203in}{2.889891in}}%
\pgfpathcurveto{\pgfqpoint{3.990016in}{2.882077in}}{\pgfqpoint{4.000616in}{2.877687in}}{\pgfqpoint{4.011666in}{2.877687in}}%
\pgfpathclose%
\pgfusepath{stroke,fill}%
\end{pgfscope}%
\begin{pgfscope}%
\pgfpathrectangle{\pgfqpoint{0.800000in}{0.528000in}}{\pgfqpoint{4.960000in}{3.696000in}}%
\pgfusepath{clip}%
\pgfsetbuttcap%
\pgfsetroundjoin%
\definecolor{currentfill}{rgb}{0.000000,0.000000,0.000000}%
\pgfsetfillcolor{currentfill}%
\pgfsetlinewidth{1.003750pt}%
\definecolor{currentstroke}{rgb}{0.000000,0.000000,0.000000}%
\pgfsetstrokecolor{currentstroke}%
\pgfsetdash{}{0pt}%
\pgfpathmoveto{\pgfqpoint{4.011666in}{1.771040in}}%
\pgfpathcurveto{\pgfqpoint{4.022716in}{1.771040in}}{\pgfqpoint{4.033315in}{1.775431in}}{\pgfqpoint{4.041128in}{1.783244in}}%
\pgfpathcurveto{\pgfqpoint{4.048942in}{1.791058in}}{\pgfqpoint{4.053332in}{1.801657in}}{\pgfqpoint{4.053332in}{1.812707in}}%
\pgfpathcurveto{\pgfqpoint{4.053332in}{1.823757in}}{\pgfqpoint{4.048942in}{1.834356in}}{\pgfqpoint{4.041128in}{1.842170in}}%
\pgfpathcurveto{\pgfqpoint{4.033315in}{1.849983in}}{\pgfqpoint{4.022716in}{1.854374in}}{\pgfqpoint{4.011666in}{1.854374in}}%
\pgfpathcurveto{\pgfqpoint{4.000616in}{1.854374in}}{\pgfqpoint{3.990016in}{1.849983in}}{\pgfqpoint{3.982203in}{1.842170in}}%
\pgfpathcurveto{\pgfqpoint{3.974389in}{1.834356in}}{\pgfqpoint{3.969999in}{1.823757in}}{\pgfqpoint{3.969999in}{1.812707in}}%
\pgfpathcurveto{\pgfqpoint{3.969999in}{1.801657in}}{\pgfqpoint{3.974389in}{1.791058in}}{\pgfqpoint{3.982203in}{1.783244in}}%
\pgfpathcurveto{\pgfqpoint{3.990016in}{1.775431in}}{\pgfqpoint{4.000616in}{1.771040in}}{\pgfqpoint{4.011666in}{1.771040in}}%
\pgfpathclose%
\pgfusepath{stroke,fill}%
\end{pgfscope}%
\begin{pgfscope}%
\pgfpathrectangle{\pgfqpoint{0.800000in}{0.528000in}}{\pgfqpoint{4.960000in}{3.696000in}}%
\pgfusepath{clip}%
\pgfsetbuttcap%
\pgfsetroundjoin%
\definecolor{currentfill}{rgb}{0.000000,0.000000,0.000000}%
\pgfsetfillcolor{currentfill}%
\pgfsetlinewidth{1.003750pt}%
\definecolor{currentstroke}{rgb}{0.000000,0.000000,0.000000}%
\pgfsetstrokecolor{currentstroke}%
\pgfsetdash{}{0pt}%
\pgfpathmoveto{\pgfqpoint{4.011666in}{1.771040in}}%
\pgfpathcurveto{\pgfqpoint{4.022716in}{1.771040in}}{\pgfqpoint{4.033315in}{1.775431in}}{\pgfqpoint{4.041128in}{1.783244in}}%
\pgfpathcurveto{\pgfqpoint{4.048942in}{1.791058in}}{\pgfqpoint{4.053332in}{1.801657in}}{\pgfqpoint{4.053332in}{1.812707in}}%
\pgfpathcurveto{\pgfqpoint{4.053332in}{1.823757in}}{\pgfqpoint{4.048942in}{1.834356in}}{\pgfqpoint{4.041128in}{1.842170in}}%
\pgfpathcurveto{\pgfqpoint{4.033315in}{1.849983in}}{\pgfqpoint{4.022716in}{1.854374in}}{\pgfqpoint{4.011666in}{1.854374in}}%
\pgfpathcurveto{\pgfqpoint{4.000616in}{1.854374in}}{\pgfqpoint{3.990016in}{1.849983in}}{\pgfqpoint{3.982203in}{1.842170in}}%
\pgfpathcurveto{\pgfqpoint{3.974389in}{1.834356in}}{\pgfqpoint{3.969999in}{1.823757in}}{\pgfqpoint{3.969999in}{1.812707in}}%
\pgfpathcurveto{\pgfqpoint{3.969999in}{1.801657in}}{\pgfqpoint{3.974389in}{1.791058in}}{\pgfqpoint{3.982203in}{1.783244in}}%
\pgfpathcurveto{\pgfqpoint{3.990016in}{1.775431in}}{\pgfqpoint{4.000616in}{1.771040in}}{\pgfqpoint{4.011666in}{1.771040in}}%
\pgfpathclose%
\pgfusepath{stroke,fill}%
\end{pgfscope}%
\begin{pgfscope}%
\pgfpathrectangle{\pgfqpoint{0.800000in}{0.528000in}}{\pgfqpoint{4.960000in}{3.696000in}}%
\pgfusepath{clip}%
\pgfsetbuttcap%
\pgfsetroundjoin%
\definecolor{currentfill}{rgb}{0.000000,0.000000,0.000000}%
\pgfsetfillcolor{currentfill}%
\pgfsetlinewidth{1.003750pt}%
\definecolor{currentstroke}{rgb}{0.000000,0.000000,0.000000}%
\pgfsetstrokecolor{currentstroke}%
\pgfsetdash{}{0pt}%
\pgfpathmoveto{\pgfqpoint{4.011666in}{1.771040in}}%
\pgfpathcurveto{\pgfqpoint{4.022716in}{1.771040in}}{\pgfqpoint{4.033315in}{1.775431in}}{\pgfqpoint{4.041128in}{1.783244in}}%
\pgfpathcurveto{\pgfqpoint{4.048942in}{1.791058in}}{\pgfqpoint{4.053332in}{1.801657in}}{\pgfqpoint{4.053332in}{1.812707in}}%
\pgfpathcurveto{\pgfqpoint{4.053332in}{1.823757in}}{\pgfqpoint{4.048942in}{1.834356in}}{\pgfqpoint{4.041128in}{1.842170in}}%
\pgfpathcurveto{\pgfqpoint{4.033315in}{1.849983in}}{\pgfqpoint{4.022716in}{1.854374in}}{\pgfqpoint{4.011666in}{1.854374in}}%
\pgfpathcurveto{\pgfqpoint{4.000616in}{1.854374in}}{\pgfqpoint{3.990016in}{1.849983in}}{\pgfqpoint{3.982203in}{1.842170in}}%
\pgfpathcurveto{\pgfqpoint{3.974389in}{1.834356in}}{\pgfqpoint{3.969999in}{1.823757in}}{\pgfqpoint{3.969999in}{1.812707in}}%
\pgfpathcurveto{\pgfqpoint{3.969999in}{1.801657in}}{\pgfqpoint{3.974389in}{1.791058in}}{\pgfqpoint{3.982203in}{1.783244in}}%
\pgfpathcurveto{\pgfqpoint{3.990016in}{1.775431in}}{\pgfqpoint{4.000616in}{1.771040in}}{\pgfqpoint{4.011666in}{1.771040in}}%
\pgfpathclose%
\pgfusepath{stroke,fill}%
\end{pgfscope}%
\begin{pgfscope}%
\pgfpathrectangle{\pgfqpoint{0.800000in}{0.528000in}}{\pgfqpoint{4.960000in}{3.696000in}}%
\pgfusepath{clip}%
\pgfsetbuttcap%
\pgfsetroundjoin%
\definecolor{currentfill}{rgb}{0.000000,0.000000,0.000000}%
\pgfsetfillcolor{currentfill}%
\pgfsetlinewidth{1.003750pt}%
\definecolor{currentstroke}{rgb}{0.000000,0.000000,0.000000}%
\pgfsetstrokecolor{currentstroke}%
\pgfsetdash{}{0pt}%
\pgfpathmoveto{\pgfqpoint{4.011666in}{1.771040in}}%
\pgfpathcurveto{\pgfqpoint{4.022716in}{1.771040in}}{\pgfqpoint{4.033315in}{1.775431in}}{\pgfqpoint{4.041128in}{1.783244in}}%
\pgfpathcurveto{\pgfqpoint{4.048942in}{1.791058in}}{\pgfqpoint{4.053332in}{1.801657in}}{\pgfqpoint{4.053332in}{1.812707in}}%
\pgfpathcurveto{\pgfqpoint{4.053332in}{1.823757in}}{\pgfqpoint{4.048942in}{1.834356in}}{\pgfqpoint{4.041128in}{1.842170in}}%
\pgfpathcurveto{\pgfqpoint{4.033315in}{1.849983in}}{\pgfqpoint{4.022716in}{1.854374in}}{\pgfqpoint{4.011666in}{1.854374in}}%
\pgfpathcurveto{\pgfqpoint{4.000616in}{1.854374in}}{\pgfqpoint{3.990016in}{1.849983in}}{\pgfqpoint{3.982203in}{1.842170in}}%
\pgfpathcurveto{\pgfqpoint{3.974389in}{1.834356in}}{\pgfqpoint{3.969999in}{1.823757in}}{\pgfqpoint{3.969999in}{1.812707in}}%
\pgfpathcurveto{\pgfqpoint{3.969999in}{1.801657in}}{\pgfqpoint{3.974389in}{1.791058in}}{\pgfqpoint{3.982203in}{1.783244in}}%
\pgfpathcurveto{\pgfqpoint{3.990016in}{1.775431in}}{\pgfqpoint{4.000616in}{1.771040in}}{\pgfqpoint{4.011666in}{1.771040in}}%
\pgfpathclose%
\pgfusepath{stroke,fill}%
\end{pgfscope}%
\begin{pgfscope}%
\pgfpathrectangle{\pgfqpoint{0.800000in}{0.528000in}}{\pgfqpoint{4.960000in}{3.696000in}}%
\pgfusepath{clip}%
\pgfsetbuttcap%
\pgfsetroundjoin%
\definecolor{currentfill}{rgb}{0.000000,0.000000,0.000000}%
\pgfsetfillcolor{currentfill}%
\pgfsetlinewidth{1.003750pt}%
\definecolor{currentstroke}{rgb}{0.000000,0.000000,0.000000}%
\pgfsetstrokecolor{currentstroke}%
\pgfsetdash{}{0pt}%
\pgfpathmoveto{\pgfqpoint{4.011666in}{2.877687in}}%
\pgfpathcurveto{\pgfqpoint{4.022716in}{2.877687in}}{\pgfqpoint{4.033315in}{2.882077in}}{\pgfqpoint{4.041128in}{2.889891in}}%
\pgfpathcurveto{\pgfqpoint{4.048942in}{2.897704in}}{\pgfqpoint{4.053332in}{2.908303in}}{\pgfqpoint{4.053332in}{2.919353in}}%
\pgfpathcurveto{\pgfqpoint{4.053332in}{2.930404in}}{\pgfqpoint{4.048942in}{2.941003in}}{\pgfqpoint{4.041128in}{2.948816in}}%
\pgfpathcurveto{\pgfqpoint{4.033315in}{2.956630in}}{\pgfqpoint{4.022716in}{2.961020in}}{\pgfqpoint{4.011666in}{2.961020in}}%
\pgfpathcurveto{\pgfqpoint{4.000616in}{2.961020in}}{\pgfqpoint{3.990016in}{2.956630in}}{\pgfqpoint{3.982203in}{2.948816in}}%
\pgfpathcurveto{\pgfqpoint{3.974389in}{2.941003in}}{\pgfqpoint{3.969999in}{2.930404in}}{\pgfqpoint{3.969999in}{2.919353in}}%
\pgfpathcurveto{\pgfqpoint{3.969999in}{2.908303in}}{\pgfqpoint{3.974389in}{2.897704in}}{\pgfqpoint{3.982203in}{2.889891in}}%
\pgfpathcurveto{\pgfqpoint{3.990016in}{2.882077in}}{\pgfqpoint{4.000616in}{2.877687in}}{\pgfqpoint{4.011666in}{2.877687in}}%
\pgfpathclose%
\pgfusepath{stroke,fill}%
\end{pgfscope}%
\begin{pgfscope}%
\pgfpathrectangle{\pgfqpoint{0.800000in}{0.528000in}}{\pgfqpoint{4.960000in}{3.696000in}}%
\pgfusepath{clip}%
\pgfsetbuttcap%
\pgfsetroundjoin%
\definecolor{currentfill}{rgb}{0.000000,0.000000,0.000000}%
\pgfsetfillcolor{currentfill}%
\pgfsetlinewidth{1.003750pt}%
\definecolor{currentstroke}{rgb}{0.000000,0.000000,0.000000}%
\pgfsetstrokecolor{currentstroke}%
\pgfsetdash{}{0pt}%
\pgfpathmoveto{\pgfqpoint{4.011666in}{1.771040in}}%
\pgfpathcurveto{\pgfqpoint{4.022716in}{1.771040in}}{\pgfqpoint{4.033315in}{1.775431in}}{\pgfqpoint{4.041128in}{1.783244in}}%
\pgfpathcurveto{\pgfqpoint{4.048942in}{1.791058in}}{\pgfqpoint{4.053332in}{1.801657in}}{\pgfqpoint{4.053332in}{1.812707in}}%
\pgfpathcurveto{\pgfqpoint{4.053332in}{1.823757in}}{\pgfqpoint{4.048942in}{1.834356in}}{\pgfqpoint{4.041128in}{1.842170in}}%
\pgfpathcurveto{\pgfqpoint{4.033315in}{1.849983in}}{\pgfqpoint{4.022716in}{1.854374in}}{\pgfqpoint{4.011666in}{1.854374in}}%
\pgfpathcurveto{\pgfqpoint{4.000616in}{1.854374in}}{\pgfqpoint{3.990016in}{1.849983in}}{\pgfqpoint{3.982203in}{1.842170in}}%
\pgfpathcurveto{\pgfqpoint{3.974389in}{1.834356in}}{\pgfqpoint{3.969999in}{1.823757in}}{\pgfqpoint{3.969999in}{1.812707in}}%
\pgfpathcurveto{\pgfqpoint{3.969999in}{1.801657in}}{\pgfqpoint{3.974389in}{1.791058in}}{\pgfqpoint{3.982203in}{1.783244in}}%
\pgfpathcurveto{\pgfqpoint{3.990016in}{1.775431in}}{\pgfqpoint{4.000616in}{1.771040in}}{\pgfqpoint{4.011666in}{1.771040in}}%
\pgfpathclose%
\pgfusepath{stroke,fill}%
\end{pgfscope}%
\begin{pgfscope}%
\pgfpathrectangle{\pgfqpoint{0.800000in}{0.528000in}}{\pgfqpoint{4.960000in}{3.696000in}}%
\pgfusepath{clip}%
\pgfsetbuttcap%
\pgfsetroundjoin%
\definecolor{currentfill}{rgb}{0.000000,0.000000,0.000000}%
\pgfsetfillcolor{currentfill}%
\pgfsetlinewidth{1.003750pt}%
\definecolor{currentstroke}{rgb}{0.000000,0.000000,0.000000}%
\pgfsetstrokecolor{currentstroke}%
\pgfsetdash{}{0pt}%
\pgfpathmoveto{\pgfqpoint{4.011666in}{1.771040in}}%
\pgfpathcurveto{\pgfqpoint{4.022716in}{1.771040in}}{\pgfqpoint{4.033315in}{1.775431in}}{\pgfqpoint{4.041128in}{1.783244in}}%
\pgfpathcurveto{\pgfqpoint{4.048942in}{1.791058in}}{\pgfqpoint{4.053332in}{1.801657in}}{\pgfqpoint{4.053332in}{1.812707in}}%
\pgfpathcurveto{\pgfqpoint{4.053332in}{1.823757in}}{\pgfqpoint{4.048942in}{1.834356in}}{\pgfqpoint{4.041128in}{1.842170in}}%
\pgfpathcurveto{\pgfqpoint{4.033315in}{1.849983in}}{\pgfqpoint{4.022716in}{1.854374in}}{\pgfqpoint{4.011666in}{1.854374in}}%
\pgfpathcurveto{\pgfqpoint{4.000616in}{1.854374in}}{\pgfqpoint{3.990016in}{1.849983in}}{\pgfqpoint{3.982203in}{1.842170in}}%
\pgfpathcurveto{\pgfqpoint{3.974389in}{1.834356in}}{\pgfqpoint{3.969999in}{1.823757in}}{\pgfqpoint{3.969999in}{1.812707in}}%
\pgfpathcurveto{\pgfqpoint{3.969999in}{1.801657in}}{\pgfqpoint{3.974389in}{1.791058in}}{\pgfqpoint{3.982203in}{1.783244in}}%
\pgfpathcurveto{\pgfqpoint{3.990016in}{1.775431in}}{\pgfqpoint{4.000616in}{1.771040in}}{\pgfqpoint{4.011666in}{1.771040in}}%
\pgfpathclose%
\pgfusepath{stroke,fill}%
\end{pgfscope}%
\begin{pgfscope}%
\pgfpathrectangle{\pgfqpoint{0.800000in}{0.528000in}}{\pgfqpoint{4.960000in}{3.696000in}}%
\pgfusepath{clip}%
\pgfsetbuttcap%
\pgfsetroundjoin%
\definecolor{currentfill}{rgb}{0.000000,0.000000,0.000000}%
\pgfsetfillcolor{currentfill}%
\pgfsetlinewidth{1.003750pt}%
\definecolor{currentstroke}{rgb}{0.000000,0.000000,0.000000}%
\pgfsetstrokecolor{currentstroke}%
\pgfsetdash{}{0pt}%
\pgfpathmoveto{\pgfqpoint{4.011666in}{1.771040in}}%
\pgfpathcurveto{\pgfqpoint{4.022716in}{1.771040in}}{\pgfqpoint{4.033315in}{1.775431in}}{\pgfqpoint{4.041128in}{1.783244in}}%
\pgfpathcurveto{\pgfqpoint{4.048942in}{1.791058in}}{\pgfqpoint{4.053332in}{1.801657in}}{\pgfqpoint{4.053332in}{1.812707in}}%
\pgfpathcurveto{\pgfqpoint{4.053332in}{1.823757in}}{\pgfqpoint{4.048942in}{1.834356in}}{\pgfqpoint{4.041128in}{1.842170in}}%
\pgfpathcurveto{\pgfqpoint{4.033315in}{1.849983in}}{\pgfqpoint{4.022716in}{1.854374in}}{\pgfqpoint{4.011666in}{1.854374in}}%
\pgfpathcurveto{\pgfqpoint{4.000616in}{1.854374in}}{\pgfqpoint{3.990016in}{1.849983in}}{\pgfqpoint{3.982203in}{1.842170in}}%
\pgfpathcurveto{\pgfqpoint{3.974389in}{1.834356in}}{\pgfqpoint{3.969999in}{1.823757in}}{\pgfqpoint{3.969999in}{1.812707in}}%
\pgfpathcurveto{\pgfqpoint{3.969999in}{1.801657in}}{\pgfqpoint{3.974389in}{1.791058in}}{\pgfqpoint{3.982203in}{1.783244in}}%
\pgfpathcurveto{\pgfqpoint{3.990016in}{1.775431in}}{\pgfqpoint{4.000616in}{1.771040in}}{\pgfqpoint{4.011666in}{1.771040in}}%
\pgfpathclose%
\pgfusepath{stroke,fill}%
\end{pgfscope}%
\begin{pgfscope}%
\pgfpathrectangle{\pgfqpoint{0.800000in}{0.528000in}}{\pgfqpoint{4.960000in}{3.696000in}}%
\pgfusepath{clip}%
\pgfsetbuttcap%
\pgfsetroundjoin%
\definecolor{currentfill}{rgb}{0.000000,0.000000,0.000000}%
\pgfsetfillcolor{currentfill}%
\pgfsetlinewidth{1.003750pt}%
\definecolor{currentstroke}{rgb}{0.000000,0.000000,0.000000}%
\pgfsetstrokecolor{currentstroke}%
\pgfsetdash{}{0pt}%
\pgfpathmoveto{\pgfqpoint{4.011666in}{1.771040in}}%
\pgfpathcurveto{\pgfqpoint{4.022716in}{1.771040in}}{\pgfqpoint{4.033315in}{1.775431in}}{\pgfqpoint{4.041128in}{1.783244in}}%
\pgfpathcurveto{\pgfqpoint{4.048942in}{1.791058in}}{\pgfqpoint{4.053332in}{1.801657in}}{\pgfqpoint{4.053332in}{1.812707in}}%
\pgfpathcurveto{\pgfqpoint{4.053332in}{1.823757in}}{\pgfqpoint{4.048942in}{1.834356in}}{\pgfqpoint{4.041128in}{1.842170in}}%
\pgfpathcurveto{\pgfqpoint{4.033315in}{1.849983in}}{\pgfqpoint{4.022716in}{1.854374in}}{\pgfqpoint{4.011666in}{1.854374in}}%
\pgfpathcurveto{\pgfqpoint{4.000616in}{1.854374in}}{\pgfqpoint{3.990016in}{1.849983in}}{\pgfqpoint{3.982203in}{1.842170in}}%
\pgfpathcurveto{\pgfqpoint{3.974389in}{1.834356in}}{\pgfqpoint{3.969999in}{1.823757in}}{\pgfqpoint{3.969999in}{1.812707in}}%
\pgfpathcurveto{\pgfqpoint{3.969999in}{1.801657in}}{\pgfqpoint{3.974389in}{1.791058in}}{\pgfqpoint{3.982203in}{1.783244in}}%
\pgfpathcurveto{\pgfqpoint{3.990016in}{1.775431in}}{\pgfqpoint{4.000616in}{1.771040in}}{\pgfqpoint{4.011666in}{1.771040in}}%
\pgfpathclose%
\pgfusepath{stroke,fill}%
\end{pgfscope}%
\begin{pgfscope}%
\pgfpathrectangle{\pgfqpoint{0.800000in}{0.528000in}}{\pgfqpoint{4.960000in}{3.696000in}}%
\pgfusepath{clip}%
\pgfsetbuttcap%
\pgfsetroundjoin%
\definecolor{currentfill}{rgb}{0.000000,0.000000,0.000000}%
\pgfsetfillcolor{currentfill}%
\pgfsetlinewidth{1.003750pt}%
\definecolor{currentstroke}{rgb}{0.000000,0.000000,0.000000}%
\pgfsetstrokecolor{currentstroke}%
\pgfsetdash{}{0pt}%
\pgfpathmoveto{\pgfqpoint{4.011666in}{1.771040in}}%
\pgfpathcurveto{\pgfqpoint{4.022716in}{1.771040in}}{\pgfqpoint{4.033315in}{1.775431in}}{\pgfqpoint{4.041128in}{1.783244in}}%
\pgfpathcurveto{\pgfqpoint{4.048942in}{1.791058in}}{\pgfqpoint{4.053332in}{1.801657in}}{\pgfqpoint{4.053332in}{1.812707in}}%
\pgfpathcurveto{\pgfqpoint{4.053332in}{1.823757in}}{\pgfqpoint{4.048942in}{1.834356in}}{\pgfqpoint{4.041128in}{1.842170in}}%
\pgfpathcurveto{\pgfqpoint{4.033315in}{1.849983in}}{\pgfqpoint{4.022716in}{1.854374in}}{\pgfqpoint{4.011666in}{1.854374in}}%
\pgfpathcurveto{\pgfqpoint{4.000616in}{1.854374in}}{\pgfqpoint{3.990016in}{1.849983in}}{\pgfqpoint{3.982203in}{1.842170in}}%
\pgfpathcurveto{\pgfqpoint{3.974389in}{1.834356in}}{\pgfqpoint{3.969999in}{1.823757in}}{\pgfqpoint{3.969999in}{1.812707in}}%
\pgfpathcurveto{\pgfqpoint{3.969999in}{1.801657in}}{\pgfqpoint{3.974389in}{1.791058in}}{\pgfqpoint{3.982203in}{1.783244in}}%
\pgfpathcurveto{\pgfqpoint{3.990016in}{1.775431in}}{\pgfqpoint{4.000616in}{1.771040in}}{\pgfqpoint{4.011666in}{1.771040in}}%
\pgfpathclose%
\pgfusepath{stroke,fill}%
\end{pgfscope}%
\begin{pgfscope}%
\pgfpathrectangle{\pgfqpoint{0.800000in}{0.528000in}}{\pgfqpoint{4.960000in}{3.696000in}}%
\pgfusepath{clip}%
\pgfsetbuttcap%
\pgfsetroundjoin%
\definecolor{currentfill}{rgb}{0.000000,0.000000,0.000000}%
\pgfsetfillcolor{currentfill}%
\pgfsetlinewidth{1.003750pt}%
\definecolor{currentstroke}{rgb}{0.000000,0.000000,0.000000}%
\pgfsetstrokecolor{currentstroke}%
\pgfsetdash{}{0pt}%
\pgfpathmoveto{\pgfqpoint{4.011666in}{1.771040in}}%
\pgfpathcurveto{\pgfqpoint{4.022716in}{1.771040in}}{\pgfqpoint{4.033315in}{1.775431in}}{\pgfqpoint{4.041128in}{1.783244in}}%
\pgfpathcurveto{\pgfqpoint{4.048942in}{1.791058in}}{\pgfqpoint{4.053332in}{1.801657in}}{\pgfqpoint{4.053332in}{1.812707in}}%
\pgfpathcurveto{\pgfqpoint{4.053332in}{1.823757in}}{\pgfqpoint{4.048942in}{1.834356in}}{\pgfqpoint{4.041128in}{1.842170in}}%
\pgfpathcurveto{\pgfqpoint{4.033315in}{1.849983in}}{\pgfqpoint{4.022716in}{1.854374in}}{\pgfqpoint{4.011666in}{1.854374in}}%
\pgfpathcurveto{\pgfqpoint{4.000616in}{1.854374in}}{\pgfqpoint{3.990016in}{1.849983in}}{\pgfqpoint{3.982203in}{1.842170in}}%
\pgfpathcurveto{\pgfqpoint{3.974389in}{1.834356in}}{\pgfqpoint{3.969999in}{1.823757in}}{\pgfqpoint{3.969999in}{1.812707in}}%
\pgfpathcurveto{\pgfqpoint{3.969999in}{1.801657in}}{\pgfqpoint{3.974389in}{1.791058in}}{\pgfqpoint{3.982203in}{1.783244in}}%
\pgfpathcurveto{\pgfqpoint{3.990016in}{1.775431in}}{\pgfqpoint{4.000616in}{1.771040in}}{\pgfqpoint{4.011666in}{1.771040in}}%
\pgfpathclose%
\pgfusepath{stroke,fill}%
\end{pgfscope}%
\begin{pgfscope}%
\pgfpathrectangle{\pgfqpoint{0.800000in}{0.528000in}}{\pgfqpoint{4.960000in}{3.696000in}}%
\pgfusepath{clip}%
\pgfsetbuttcap%
\pgfsetroundjoin%
\definecolor{currentfill}{rgb}{0.000000,0.000000,0.000000}%
\pgfsetfillcolor{currentfill}%
\pgfsetlinewidth{1.003750pt}%
\definecolor{currentstroke}{rgb}{0.000000,0.000000,0.000000}%
\pgfsetstrokecolor{currentstroke}%
\pgfsetdash{}{0pt}%
\pgfpathmoveto{\pgfqpoint{4.011666in}{2.877687in}}%
\pgfpathcurveto{\pgfqpoint{4.022716in}{2.877687in}}{\pgfqpoint{4.033315in}{2.882077in}}{\pgfqpoint{4.041128in}{2.889891in}}%
\pgfpathcurveto{\pgfqpoint{4.048942in}{2.897704in}}{\pgfqpoint{4.053332in}{2.908303in}}{\pgfqpoint{4.053332in}{2.919353in}}%
\pgfpathcurveto{\pgfqpoint{4.053332in}{2.930404in}}{\pgfqpoint{4.048942in}{2.941003in}}{\pgfqpoint{4.041128in}{2.948816in}}%
\pgfpathcurveto{\pgfqpoint{4.033315in}{2.956630in}}{\pgfqpoint{4.022716in}{2.961020in}}{\pgfqpoint{4.011666in}{2.961020in}}%
\pgfpathcurveto{\pgfqpoint{4.000616in}{2.961020in}}{\pgfqpoint{3.990016in}{2.956630in}}{\pgfqpoint{3.982203in}{2.948816in}}%
\pgfpathcurveto{\pgfqpoint{3.974389in}{2.941003in}}{\pgfqpoint{3.969999in}{2.930404in}}{\pgfqpoint{3.969999in}{2.919353in}}%
\pgfpathcurveto{\pgfqpoint{3.969999in}{2.908303in}}{\pgfqpoint{3.974389in}{2.897704in}}{\pgfqpoint{3.982203in}{2.889891in}}%
\pgfpathcurveto{\pgfqpoint{3.990016in}{2.882077in}}{\pgfqpoint{4.000616in}{2.877687in}}{\pgfqpoint{4.011666in}{2.877687in}}%
\pgfpathclose%
\pgfusepath{stroke,fill}%
\end{pgfscope}%
\begin{pgfscope}%
\pgfpathrectangle{\pgfqpoint{0.800000in}{0.528000in}}{\pgfqpoint{4.960000in}{3.696000in}}%
\pgfusepath{clip}%
\pgfsetbuttcap%
\pgfsetroundjoin%
\definecolor{currentfill}{rgb}{0.000000,0.000000,0.000000}%
\pgfsetfillcolor{currentfill}%
\pgfsetlinewidth{1.003750pt}%
\definecolor{currentstroke}{rgb}{0.000000,0.000000,0.000000}%
\pgfsetstrokecolor{currentstroke}%
\pgfsetdash{}{0pt}%
\pgfpathmoveto{\pgfqpoint{4.011666in}{1.771040in}}%
\pgfpathcurveto{\pgfqpoint{4.022716in}{1.771040in}}{\pgfqpoint{4.033315in}{1.775431in}}{\pgfqpoint{4.041128in}{1.783244in}}%
\pgfpathcurveto{\pgfqpoint{4.048942in}{1.791058in}}{\pgfqpoint{4.053332in}{1.801657in}}{\pgfqpoint{4.053332in}{1.812707in}}%
\pgfpathcurveto{\pgfqpoint{4.053332in}{1.823757in}}{\pgfqpoint{4.048942in}{1.834356in}}{\pgfqpoint{4.041128in}{1.842170in}}%
\pgfpathcurveto{\pgfqpoint{4.033315in}{1.849983in}}{\pgfqpoint{4.022716in}{1.854374in}}{\pgfqpoint{4.011666in}{1.854374in}}%
\pgfpathcurveto{\pgfqpoint{4.000616in}{1.854374in}}{\pgfqpoint{3.990016in}{1.849983in}}{\pgfqpoint{3.982203in}{1.842170in}}%
\pgfpathcurveto{\pgfqpoint{3.974389in}{1.834356in}}{\pgfqpoint{3.969999in}{1.823757in}}{\pgfqpoint{3.969999in}{1.812707in}}%
\pgfpathcurveto{\pgfqpoint{3.969999in}{1.801657in}}{\pgfqpoint{3.974389in}{1.791058in}}{\pgfqpoint{3.982203in}{1.783244in}}%
\pgfpathcurveto{\pgfqpoint{3.990016in}{1.775431in}}{\pgfqpoint{4.000616in}{1.771040in}}{\pgfqpoint{4.011666in}{1.771040in}}%
\pgfpathclose%
\pgfusepath{stroke,fill}%
\end{pgfscope}%
\begin{pgfscope}%
\pgfpathrectangle{\pgfqpoint{0.800000in}{0.528000in}}{\pgfqpoint{4.960000in}{3.696000in}}%
\pgfusepath{clip}%
\pgfsetbuttcap%
\pgfsetroundjoin%
\definecolor{currentfill}{rgb}{0.000000,0.000000,0.000000}%
\pgfsetfillcolor{currentfill}%
\pgfsetlinewidth{1.003750pt}%
\definecolor{currentstroke}{rgb}{0.000000,0.000000,0.000000}%
\pgfsetstrokecolor{currentstroke}%
\pgfsetdash{}{0pt}%
\pgfpathmoveto{\pgfqpoint{4.011666in}{1.771040in}}%
\pgfpathcurveto{\pgfqpoint{4.022716in}{1.771040in}}{\pgfqpoint{4.033315in}{1.775431in}}{\pgfqpoint{4.041128in}{1.783244in}}%
\pgfpathcurveto{\pgfqpoint{4.048942in}{1.791058in}}{\pgfqpoint{4.053332in}{1.801657in}}{\pgfqpoint{4.053332in}{1.812707in}}%
\pgfpathcurveto{\pgfqpoint{4.053332in}{1.823757in}}{\pgfqpoint{4.048942in}{1.834356in}}{\pgfqpoint{4.041128in}{1.842170in}}%
\pgfpathcurveto{\pgfqpoint{4.033315in}{1.849983in}}{\pgfqpoint{4.022716in}{1.854374in}}{\pgfqpoint{4.011666in}{1.854374in}}%
\pgfpathcurveto{\pgfqpoint{4.000616in}{1.854374in}}{\pgfqpoint{3.990016in}{1.849983in}}{\pgfqpoint{3.982203in}{1.842170in}}%
\pgfpathcurveto{\pgfqpoint{3.974389in}{1.834356in}}{\pgfqpoint{3.969999in}{1.823757in}}{\pgfqpoint{3.969999in}{1.812707in}}%
\pgfpathcurveto{\pgfqpoint{3.969999in}{1.801657in}}{\pgfqpoint{3.974389in}{1.791058in}}{\pgfqpoint{3.982203in}{1.783244in}}%
\pgfpathcurveto{\pgfqpoint{3.990016in}{1.775431in}}{\pgfqpoint{4.000616in}{1.771040in}}{\pgfqpoint{4.011666in}{1.771040in}}%
\pgfpathclose%
\pgfusepath{stroke,fill}%
\end{pgfscope}%
\begin{pgfscope}%
\pgfpathrectangle{\pgfqpoint{0.800000in}{0.528000in}}{\pgfqpoint{4.960000in}{3.696000in}}%
\pgfusepath{clip}%
\pgfsetbuttcap%
\pgfsetroundjoin%
\definecolor{currentfill}{rgb}{0.000000,0.000000,0.000000}%
\pgfsetfillcolor{currentfill}%
\pgfsetlinewidth{1.003750pt}%
\definecolor{currentstroke}{rgb}{0.000000,0.000000,0.000000}%
\pgfsetstrokecolor{currentstroke}%
\pgfsetdash{}{0pt}%
\pgfpathmoveto{\pgfqpoint{4.011666in}{2.877687in}}%
\pgfpathcurveto{\pgfqpoint{4.022716in}{2.877687in}}{\pgfqpoint{4.033315in}{2.882077in}}{\pgfqpoint{4.041128in}{2.889891in}}%
\pgfpathcurveto{\pgfqpoint{4.048942in}{2.897704in}}{\pgfqpoint{4.053332in}{2.908303in}}{\pgfqpoint{4.053332in}{2.919353in}}%
\pgfpathcurveto{\pgfqpoint{4.053332in}{2.930404in}}{\pgfqpoint{4.048942in}{2.941003in}}{\pgfqpoint{4.041128in}{2.948816in}}%
\pgfpathcurveto{\pgfqpoint{4.033315in}{2.956630in}}{\pgfqpoint{4.022716in}{2.961020in}}{\pgfqpoint{4.011666in}{2.961020in}}%
\pgfpathcurveto{\pgfqpoint{4.000616in}{2.961020in}}{\pgfqpoint{3.990016in}{2.956630in}}{\pgfqpoint{3.982203in}{2.948816in}}%
\pgfpathcurveto{\pgfqpoint{3.974389in}{2.941003in}}{\pgfqpoint{3.969999in}{2.930404in}}{\pgfqpoint{3.969999in}{2.919353in}}%
\pgfpathcurveto{\pgfqpoint{3.969999in}{2.908303in}}{\pgfqpoint{3.974389in}{2.897704in}}{\pgfqpoint{3.982203in}{2.889891in}}%
\pgfpathcurveto{\pgfqpoint{3.990016in}{2.882077in}}{\pgfqpoint{4.000616in}{2.877687in}}{\pgfqpoint{4.011666in}{2.877687in}}%
\pgfpathclose%
\pgfusepath{stroke,fill}%
\end{pgfscope}%
\begin{pgfscope}%
\pgfpathrectangle{\pgfqpoint{0.800000in}{0.528000in}}{\pgfqpoint{4.960000in}{3.696000in}}%
\pgfusepath{clip}%
\pgfsetbuttcap%
\pgfsetroundjoin%
\definecolor{currentfill}{rgb}{0.000000,0.000000,0.000000}%
\pgfsetfillcolor{currentfill}%
\pgfsetlinewidth{1.003750pt}%
\definecolor{currentstroke}{rgb}{0.000000,0.000000,0.000000}%
\pgfsetstrokecolor{currentstroke}%
\pgfsetdash{}{0pt}%
\pgfpathmoveto{\pgfqpoint{4.011666in}{2.877687in}}%
\pgfpathcurveto{\pgfqpoint{4.022716in}{2.877687in}}{\pgfqpoint{4.033315in}{2.882077in}}{\pgfqpoint{4.041128in}{2.889891in}}%
\pgfpathcurveto{\pgfqpoint{4.048942in}{2.897704in}}{\pgfqpoint{4.053332in}{2.908303in}}{\pgfqpoint{4.053332in}{2.919353in}}%
\pgfpathcurveto{\pgfqpoint{4.053332in}{2.930404in}}{\pgfqpoint{4.048942in}{2.941003in}}{\pgfqpoint{4.041128in}{2.948816in}}%
\pgfpathcurveto{\pgfqpoint{4.033315in}{2.956630in}}{\pgfqpoint{4.022716in}{2.961020in}}{\pgfqpoint{4.011666in}{2.961020in}}%
\pgfpathcurveto{\pgfqpoint{4.000616in}{2.961020in}}{\pgfqpoint{3.990016in}{2.956630in}}{\pgfqpoint{3.982203in}{2.948816in}}%
\pgfpathcurveto{\pgfqpoint{3.974389in}{2.941003in}}{\pgfqpoint{3.969999in}{2.930404in}}{\pgfqpoint{3.969999in}{2.919353in}}%
\pgfpathcurveto{\pgfqpoint{3.969999in}{2.908303in}}{\pgfqpoint{3.974389in}{2.897704in}}{\pgfqpoint{3.982203in}{2.889891in}}%
\pgfpathcurveto{\pgfqpoint{3.990016in}{2.882077in}}{\pgfqpoint{4.000616in}{2.877687in}}{\pgfqpoint{4.011666in}{2.877687in}}%
\pgfpathclose%
\pgfusepath{stroke,fill}%
\end{pgfscope}%
\begin{pgfscope}%
\pgfpathrectangle{\pgfqpoint{0.800000in}{0.528000in}}{\pgfqpoint{4.960000in}{3.696000in}}%
\pgfusepath{clip}%
\pgfsetbuttcap%
\pgfsetroundjoin%
\definecolor{currentfill}{rgb}{0.000000,0.000000,0.000000}%
\pgfsetfillcolor{currentfill}%
\pgfsetlinewidth{1.003750pt}%
\definecolor{currentstroke}{rgb}{0.000000,0.000000,0.000000}%
\pgfsetstrokecolor{currentstroke}%
\pgfsetdash{}{0pt}%
\pgfpathmoveto{\pgfqpoint{4.011666in}{1.771040in}}%
\pgfpathcurveto{\pgfqpoint{4.022716in}{1.771040in}}{\pgfqpoint{4.033315in}{1.775431in}}{\pgfqpoint{4.041128in}{1.783244in}}%
\pgfpathcurveto{\pgfqpoint{4.048942in}{1.791058in}}{\pgfqpoint{4.053332in}{1.801657in}}{\pgfqpoint{4.053332in}{1.812707in}}%
\pgfpathcurveto{\pgfqpoint{4.053332in}{1.823757in}}{\pgfqpoint{4.048942in}{1.834356in}}{\pgfqpoint{4.041128in}{1.842170in}}%
\pgfpathcurveto{\pgfqpoint{4.033315in}{1.849983in}}{\pgfqpoint{4.022716in}{1.854374in}}{\pgfqpoint{4.011666in}{1.854374in}}%
\pgfpathcurveto{\pgfqpoint{4.000616in}{1.854374in}}{\pgfqpoint{3.990016in}{1.849983in}}{\pgfqpoint{3.982203in}{1.842170in}}%
\pgfpathcurveto{\pgfqpoint{3.974389in}{1.834356in}}{\pgfqpoint{3.969999in}{1.823757in}}{\pgfqpoint{3.969999in}{1.812707in}}%
\pgfpathcurveto{\pgfqpoint{3.969999in}{1.801657in}}{\pgfqpoint{3.974389in}{1.791058in}}{\pgfqpoint{3.982203in}{1.783244in}}%
\pgfpathcurveto{\pgfqpoint{3.990016in}{1.775431in}}{\pgfqpoint{4.000616in}{1.771040in}}{\pgfqpoint{4.011666in}{1.771040in}}%
\pgfpathclose%
\pgfusepath{stroke,fill}%
\end{pgfscope}%
\begin{pgfscope}%
\pgfpathrectangle{\pgfqpoint{0.800000in}{0.528000in}}{\pgfqpoint{4.960000in}{3.696000in}}%
\pgfusepath{clip}%
\pgfsetbuttcap%
\pgfsetroundjoin%
\definecolor{currentfill}{rgb}{0.000000,0.000000,0.000000}%
\pgfsetfillcolor{currentfill}%
\pgfsetlinewidth{1.003750pt}%
\definecolor{currentstroke}{rgb}{0.000000,0.000000,0.000000}%
\pgfsetstrokecolor{currentstroke}%
\pgfsetdash{}{0pt}%
\pgfpathmoveto{\pgfqpoint{4.011666in}{2.877687in}}%
\pgfpathcurveto{\pgfqpoint{4.022716in}{2.877687in}}{\pgfqpoint{4.033315in}{2.882077in}}{\pgfqpoint{4.041128in}{2.889891in}}%
\pgfpathcurveto{\pgfqpoint{4.048942in}{2.897704in}}{\pgfqpoint{4.053332in}{2.908303in}}{\pgfqpoint{4.053332in}{2.919353in}}%
\pgfpathcurveto{\pgfqpoint{4.053332in}{2.930404in}}{\pgfqpoint{4.048942in}{2.941003in}}{\pgfqpoint{4.041128in}{2.948816in}}%
\pgfpathcurveto{\pgfqpoint{4.033315in}{2.956630in}}{\pgfqpoint{4.022716in}{2.961020in}}{\pgfqpoint{4.011666in}{2.961020in}}%
\pgfpathcurveto{\pgfqpoint{4.000616in}{2.961020in}}{\pgfqpoint{3.990016in}{2.956630in}}{\pgfqpoint{3.982203in}{2.948816in}}%
\pgfpathcurveto{\pgfqpoint{3.974389in}{2.941003in}}{\pgfqpoint{3.969999in}{2.930404in}}{\pgfqpoint{3.969999in}{2.919353in}}%
\pgfpathcurveto{\pgfqpoint{3.969999in}{2.908303in}}{\pgfqpoint{3.974389in}{2.897704in}}{\pgfqpoint{3.982203in}{2.889891in}}%
\pgfpathcurveto{\pgfqpoint{3.990016in}{2.882077in}}{\pgfqpoint{4.000616in}{2.877687in}}{\pgfqpoint{4.011666in}{2.877687in}}%
\pgfpathclose%
\pgfusepath{stroke,fill}%
\end{pgfscope}%
\begin{pgfscope}%
\pgfpathrectangle{\pgfqpoint{0.800000in}{0.528000in}}{\pgfqpoint{4.960000in}{3.696000in}}%
\pgfusepath{clip}%
\pgfsetbuttcap%
\pgfsetroundjoin%
\definecolor{currentfill}{rgb}{0.000000,0.000000,0.000000}%
\pgfsetfillcolor{currentfill}%
\pgfsetlinewidth{1.003750pt}%
\definecolor{currentstroke}{rgb}{0.000000,0.000000,0.000000}%
\pgfsetstrokecolor{currentstroke}%
\pgfsetdash{}{0pt}%
\pgfpathmoveto{\pgfqpoint{4.011666in}{2.877687in}}%
\pgfpathcurveto{\pgfqpoint{4.022716in}{2.877687in}}{\pgfqpoint{4.033315in}{2.882077in}}{\pgfqpoint{4.041128in}{2.889891in}}%
\pgfpathcurveto{\pgfqpoint{4.048942in}{2.897704in}}{\pgfqpoint{4.053332in}{2.908303in}}{\pgfqpoint{4.053332in}{2.919353in}}%
\pgfpathcurveto{\pgfqpoint{4.053332in}{2.930404in}}{\pgfqpoint{4.048942in}{2.941003in}}{\pgfqpoint{4.041128in}{2.948816in}}%
\pgfpathcurveto{\pgfqpoint{4.033315in}{2.956630in}}{\pgfqpoint{4.022716in}{2.961020in}}{\pgfqpoint{4.011666in}{2.961020in}}%
\pgfpathcurveto{\pgfqpoint{4.000616in}{2.961020in}}{\pgfqpoint{3.990016in}{2.956630in}}{\pgfqpoint{3.982203in}{2.948816in}}%
\pgfpathcurveto{\pgfqpoint{3.974389in}{2.941003in}}{\pgfqpoint{3.969999in}{2.930404in}}{\pgfqpoint{3.969999in}{2.919353in}}%
\pgfpathcurveto{\pgfqpoint{3.969999in}{2.908303in}}{\pgfqpoint{3.974389in}{2.897704in}}{\pgfqpoint{3.982203in}{2.889891in}}%
\pgfpathcurveto{\pgfqpoint{3.990016in}{2.882077in}}{\pgfqpoint{4.000616in}{2.877687in}}{\pgfqpoint{4.011666in}{2.877687in}}%
\pgfpathclose%
\pgfusepath{stroke,fill}%
\end{pgfscope}%
\begin{pgfscope}%
\pgfpathrectangle{\pgfqpoint{0.800000in}{0.528000in}}{\pgfqpoint{4.960000in}{3.696000in}}%
\pgfusepath{clip}%
\pgfsetbuttcap%
\pgfsetroundjoin%
\definecolor{currentfill}{rgb}{0.000000,0.000000,0.000000}%
\pgfsetfillcolor{currentfill}%
\pgfsetlinewidth{1.003750pt}%
\definecolor{currentstroke}{rgb}{0.000000,0.000000,0.000000}%
\pgfsetstrokecolor{currentstroke}%
\pgfsetdash{}{0pt}%
\pgfpathmoveto{\pgfqpoint{4.011666in}{1.771040in}}%
\pgfpathcurveto{\pgfqpoint{4.022716in}{1.771040in}}{\pgfqpoint{4.033315in}{1.775431in}}{\pgfqpoint{4.041128in}{1.783244in}}%
\pgfpathcurveto{\pgfqpoint{4.048942in}{1.791058in}}{\pgfqpoint{4.053332in}{1.801657in}}{\pgfqpoint{4.053332in}{1.812707in}}%
\pgfpathcurveto{\pgfqpoint{4.053332in}{1.823757in}}{\pgfqpoint{4.048942in}{1.834356in}}{\pgfqpoint{4.041128in}{1.842170in}}%
\pgfpathcurveto{\pgfqpoint{4.033315in}{1.849983in}}{\pgfqpoint{4.022716in}{1.854374in}}{\pgfqpoint{4.011666in}{1.854374in}}%
\pgfpathcurveto{\pgfqpoint{4.000616in}{1.854374in}}{\pgfqpoint{3.990016in}{1.849983in}}{\pgfqpoint{3.982203in}{1.842170in}}%
\pgfpathcurveto{\pgfqpoint{3.974389in}{1.834356in}}{\pgfqpoint{3.969999in}{1.823757in}}{\pgfqpoint{3.969999in}{1.812707in}}%
\pgfpathcurveto{\pgfqpoint{3.969999in}{1.801657in}}{\pgfqpoint{3.974389in}{1.791058in}}{\pgfqpoint{3.982203in}{1.783244in}}%
\pgfpathcurveto{\pgfqpoint{3.990016in}{1.775431in}}{\pgfqpoint{4.000616in}{1.771040in}}{\pgfqpoint{4.011666in}{1.771040in}}%
\pgfpathclose%
\pgfusepath{stroke,fill}%
\end{pgfscope}%
\begin{pgfscope}%
\pgfpathrectangle{\pgfqpoint{0.800000in}{0.528000in}}{\pgfqpoint{4.960000in}{3.696000in}}%
\pgfusepath{clip}%
\pgfsetbuttcap%
\pgfsetroundjoin%
\definecolor{currentfill}{rgb}{0.000000,0.000000,0.000000}%
\pgfsetfillcolor{currentfill}%
\pgfsetlinewidth{1.003750pt}%
\definecolor{currentstroke}{rgb}{0.000000,0.000000,0.000000}%
\pgfsetstrokecolor{currentstroke}%
\pgfsetdash{}{0pt}%
\pgfpathmoveto{\pgfqpoint{4.011666in}{2.877687in}}%
\pgfpathcurveto{\pgfqpoint{4.022716in}{2.877687in}}{\pgfqpoint{4.033315in}{2.882077in}}{\pgfqpoint{4.041128in}{2.889891in}}%
\pgfpathcurveto{\pgfqpoint{4.048942in}{2.897704in}}{\pgfqpoint{4.053332in}{2.908303in}}{\pgfqpoint{4.053332in}{2.919353in}}%
\pgfpathcurveto{\pgfqpoint{4.053332in}{2.930404in}}{\pgfqpoint{4.048942in}{2.941003in}}{\pgfqpoint{4.041128in}{2.948816in}}%
\pgfpathcurveto{\pgfqpoint{4.033315in}{2.956630in}}{\pgfqpoint{4.022716in}{2.961020in}}{\pgfqpoint{4.011666in}{2.961020in}}%
\pgfpathcurveto{\pgfqpoint{4.000616in}{2.961020in}}{\pgfqpoint{3.990016in}{2.956630in}}{\pgfqpoint{3.982203in}{2.948816in}}%
\pgfpathcurveto{\pgfqpoint{3.974389in}{2.941003in}}{\pgfqpoint{3.969999in}{2.930404in}}{\pgfqpoint{3.969999in}{2.919353in}}%
\pgfpathcurveto{\pgfqpoint{3.969999in}{2.908303in}}{\pgfqpoint{3.974389in}{2.897704in}}{\pgfqpoint{3.982203in}{2.889891in}}%
\pgfpathcurveto{\pgfqpoint{3.990016in}{2.882077in}}{\pgfqpoint{4.000616in}{2.877687in}}{\pgfqpoint{4.011666in}{2.877687in}}%
\pgfpathclose%
\pgfusepath{stroke,fill}%
\end{pgfscope}%
\begin{pgfscope}%
\pgfpathrectangle{\pgfqpoint{0.800000in}{0.528000in}}{\pgfqpoint{4.960000in}{3.696000in}}%
\pgfusepath{clip}%
\pgfsetbuttcap%
\pgfsetroundjoin%
\definecolor{currentfill}{rgb}{0.000000,0.000000,0.000000}%
\pgfsetfillcolor{currentfill}%
\pgfsetlinewidth{1.003750pt}%
\definecolor{currentstroke}{rgb}{0.000000,0.000000,0.000000}%
\pgfsetstrokecolor{currentstroke}%
\pgfsetdash{}{0pt}%
\pgfpathmoveto{\pgfqpoint{4.011666in}{1.771040in}}%
\pgfpathcurveto{\pgfqpoint{4.022716in}{1.771040in}}{\pgfqpoint{4.033315in}{1.775431in}}{\pgfqpoint{4.041128in}{1.783244in}}%
\pgfpathcurveto{\pgfqpoint{4.048942in}{1.791058in}}{\pgfqpoint{4.053332in}{1.801657in}}{\pgfqpoint{4.053332in}{1.812707in}}%
\pgfpathcurveto{\pgfqpoint{4.053332in}{1.823757in}}{\pgfqpoint{4.048942in}{1.834356in}}{\pgfqpoint{4.041128in}{1.842170in}}%
\pgfpathcurveto{\pgfqpoint{4.033315in}{1.849983in}}{\pgfqpoint{4.022716in}{1.854374in}}{\pgfqpoint{4.011666in}{1.854374in}}%
\pgfpathcurveto{\pgfqpoint{4.000616in}{1.854374in}}{\pgfqpoint{3.990016in}{1.849983in}}{\pgfqpoint{3.982203in}{1.842170in}}%
\pgfpathcurveto{\pgfqpoint{3.974389in}{1.834356in}}{\pgfqpoint{3.969999in}{1.823757in}}{\pgfqpoint{3.969999in}{1.812707in}}%
\pgfpathcurveto{\pgfqpoint{3.969999in}{1.801657in}}{\pgfqpoint{3.974389in}{1.791058in}}{\pgfqpoint{3.982203in}{1.783244in}}%
\pgfpathcurveto{\pgfqpoint{3.990016in}{1.775431in}}{\pgfqpoint{4.000616in}{1.771040in}}{\pgfqpoint{4.011666in}{1.771040in}}%
\pgfpathclose%
\pgfusepath{stroke,fill}%
\end{pgfscope}%
\begin{pgfscope}%
\pgfpathrectangle{\pgfqpoint{0.800000in}{0.528000in}}{\pgfqpoint{4.960000in}{3.696000in}}%
\pgfusepath{clip}%
\pgfsetbuttcap%
\pgfsetroundjoin%
\definecolor{currentfill}{rgb}{0.000000,0.000000,0.000000}%
\pgfsetfillcolor{currentfill}%
\pgfsetlinewidth{1.003750pt}%
\definecolor{currentstroke}{rgb}{0.000000,0.000000,0.000000}%
\pgfsetstrokecolor{currentstroke}%
\pgfsetdash{}{0pt}%
\pgfpathmoveto{\pgfqpoint{4.011666in}{1.771040in}}%
\pgfpathcurveto{\pgfqpoint{4.022716in}{1.771040in}}{\pgfqpoint{4.033315in}{1.775431in}}{\pgfqpoint{4.041128in}{1.783244in}}%
\pgfpathcurveto{\pgfqpoint{4.048942in}{1.791058in}}{\pgfqpoint{4.053332in}{1.801657in}}{\pgfqpoint{4.053332in}{1.812707in}}%
\pgfpathcurveto{\pgfqpoint{4.053332in}{1.823757in}}{\pgfqpoint{4.048942in}{1.834356in}}{\pgfqpoint{4.041128in}{1.842170in}}%
\pgfpathcurveto{\pgfqpoint{4.033315in}{1.849983in}}{\pgfqpoint{4.022716in}{1.854374in}}{\pgfqpoint{4.011666in}{1.854374in}}%
\pgfpathcurveto{\pgfqpoint{4.000616in}{1.854374in}}{\pgfqpoint{3.990016in}{1.849983in}}{\pgfqpoint{3.982203in}{1.842170in}}%
\pgfpathcurveto{\pgfqpoint{3.974389in}{1.834356in}}{\pgfqpoint{3.969999in}{1.823757in}}{\pgfqpoint{3.969999in}{1.812707in}}%
\pgfpathcurveto{\pgfqpoint{3.969999in}{1.801657in}}{\pgfqpoint{3.974389in}{1.791058in}}{\pgfqpoint{3.982203in}{1.783244in}}%
\pgfpathcurveto{\pgfqpoint{3.990016in}{1.775431in}}{\pgfqpoint{4.000616in}{1.771040in}}{\pgfqpoint{4.011666in}{1.771040in}}%
\pgfpathclose%
\pgfusepath{stroke,fill}%
\end{pgfscope}%
\begin{pgfscope}%
\pgfpathrectangle{\pgfqpoint{0.800000in}{0.528000in}}{\pgfqpoint{4.960000in}{3.696000in}}%
\pgfusepath{clip}%
\pgfsetbuttcap%
\pgfsetroundjoin%
\definecolor{currentfill}{rgb}{0.000000,0.000000,0.000000}%
\pgfsetfillcolor{currentfill}%
\pgfsetlinewidth{1.003750pt}%
\definecolor{currentstroke}{rgb}{0.000000,0.000000,0.000000}%
\pgfsetstrokecolor{currentstroke}%
\pgfsetdash{}{0pt}%
\pgfpathmoveto{\pgfqpoint{4.011666in}{1.771040in}}%
\pgfpathcurveto{\pgfqpoint{4.022716in}{1.771040in}}{\pgfqpoint{4.033315in}{1.775431in}}{\pgfqpoint{4.041128in}{1.783244in}}%
\pgfpathcurveto{\pgfqpoint{4.048942in}{1.791058in}}{\pgfqpoint{4.053332in}{1.801657in}}{\pgfqpoint{4.053332in}{1.812707in}}%
\pgfpathcurveto{\pgfqpoint{4.053332in}{1.823757in}}{\pgfqpoint{4.048942in}{1.834356in}}{\pgfqpoint{4.041128in}{1.842170in}}%
\pgfpathcurveto{\pgfqpoint{4.033315in}{1.849983in}}{\pgfqpoint{4.022716in}{1.854374in}}{\pgfqpoint{4.011666in}{1.854374in}}%
\pgfpathcurveto{\pgfqpoint{4.000616in}{1.854374in}}{\pgfqpoint{3.990016in}{1.849983in}}{\pgfqpoint{3.982203in}{1.842170in}}%
\pgfpathcurveto{\pgfqpoint{3.974389in}{1.834356in}}{\pgfqpoint{3.969999in}{1.823757in}}{\pgfqpoint{3.969999in}{1.812707in}}%
\pgfpathcurveto{\pgfqpoint{3.969999in}{1.801657in}}{\pgfqpoint{3.974389in}{1.791058in}}{\pgfqpoint{3.982203in}{1.783244in}}%
\pgfpathcurveto{\pgfqpoint{3.990016in}{1.775431in}}{\pgfqpoint{4.000616in}{1.771040in}}{\pgfqpoint{4.011666in}{1.771040in}}%
\pgfpathclose%
\pgfusepath{stroke,fill}%
\end{pgfscope}%
\begin{pgfscope}%
\pgfpathrectangle{\pgfqpoint{0.800000in}{0.528000in}}{\pgfqpoint{4.960000in}{3.696000in}}%
\pgfusepath{clip}%
\pgfsetbuttcap%
\pgfsetroundjoin%
\definecolor{currentfill}{rgb}{0.000000,0.000000,0.000000}%
\pgfsetfillcolor{currentfill}%
\pgfsetlinewidth{1.003750pt}%
\definecolor{currentstroke}{rgb}{0.000000,0.000000,0.000000}%
\pgfsetstrokecolor{currentstroke}%
\pgfsetdash{}{0pt}%
\pgfpathmoveto{\pgfqpoint{4.011666in}{2.877687in}}%
\pgfpathcurveto{\pgfqpoint{4.022716in}{2.877687in}}{\pgfqpoint{4.033315in}{2.882077in}}{\pgfqpoint{4.041128in}{2.889891in}}%
\pgfpathcurveto{\pgfqpoint{4.048942in}{2.897704in}}{\pgfqpoint{4.053332in}{2.908303in}}{\pgfqpoint{4.053332in}{2.919353in}}%
\pgfpathcurveto{\pgfqpoint{4.053332in}{2.930404in}}{\pgfqpoint{4.048942in}{2.941003in}}{\pgfqpoint{4.041128in}{2.948816in}}%
\pgfpathcurveto{\pgfqpoint{4.033315in}{2.956630in}}{\pgfqpoint{4.022716in}{2.961020in}}{\pgfqpoint{4.011666in}{2.961020in}}%
\pgfpathcurveto{\pgfqpoint{4.000616in}{2.961020in}}{\pgfqpoint{3.990016in}{2.956630in}}{\pgfqpoint{3.982203in}{2.948816in}}%
\pgfpathcurveto{\pgfqpoint{3.974389in}{2.941003in}}{\pgfqpoint{3.969999in}{2.930404in}}{\pgfqpoint{3.969999in}{2.919353in}}%
\pgfpathcurveto{\pgfqpoint{3.969999in}{2.908303in}}{\pgfqpoint{3.974389in}{2.897704in}}{\pgfqpoint{3.982203in}{2.889891in}}%
\pgfpathcurveto{\pgfqpoint{3.990016in}{2.882077in}}{\pgfqpoint{4.000616in}{2.877687in}}{\pgfqpoint{4.011666in}{2.877687in}}%
\pgfpathclose%
\pgfusepath{stroke,fill}%
\end{pgfscope}%
\begin{pgfscope}%
\pgfpathrectangle{\pgfqpoint{0.800000in}{0.528000in}}{\pgfqpoint{4.960000in}{3.696000in}}%
\pgfusepath{clip}%
\pgfsetbuttcap%
\pgfsetroundjoin%
\definecolor{currentfill}{rgb}{0.000000,0.000000,0.000000}%
\pgfsetfillcolor{currentfill}%
\pgfsetlinewidth{1.003750pt}%
\definecolor{currentstroke}{rgb}{0.000000,0.000000,0.000000}%
\pgfsetstrokecolor{currentstroke}%
\pgfsetdash{}{0pt}%
\pgfpathmoveto{\pgfqpoint{4.011666in}{1.771040in}}%
\pgfpathcurveto{\pgfqpoint{4.022716in}{1.771040in}}{\pgfqpoint{4.033315in}{1.775431in}}{\pgfqpoint{4.041128in}{1.783244in}}%
\pgfpathcurveto{\pgfqpoint{4.048942in}{1.791058in}}{\pgfqpoint{4.053332in}{1.801657in}}{\pgfqpoint{4.053332in}{1.812707in}}%
\pgfpathcurveto{\pgfqpoint{4.053332in}{1.823757in}}{\pgfqpoint{4.048942in}{1.834356in}}{\pgfqpoint{4.041128in}{1.842170in}}%
\pgfpathcurveto{\pgfqpoint{4.033315in}{1.849983in}}{\pgfqpoint{4.022716in}{1.854374in}}{\pgfqpoint{4.011666in}{1.854374in}}%
\pgfpathcurveto{\pgfqpoint{4.000616in}{1.854374in}}{\pgfqpoint{3.990016in}{1.849983in}}{\pgfqpoint{3.982203in}{1.842170in}}%
\pgfpathcurveto{\pgfqpoint{3.974389in}{1.834356in}}{\pgfqpoint{3.969999in}{1.823757in}}{\pgfqpoint{3.969999in}{1.812707in}}%
\pgfpathcurveto{\pgfqpoint{3.969999in}{1.801657in}}{\pgfqpoint{3.974389in}{1.791058in}}{\pgfqpoint{3.982203in}{1.783244in}}%
\pgfpathcurveto{\pgfqpoint{3.990016in}{1.775431in}}{\pgfqpoint{4.000616in}{1.771040in}}{\pgfqpoint{4.011666in}{1.771040in}}%
\pgfpathclose%
\pgfusepath{stroke,fill}%
\end{pgfscope}%
\begin{pgfscope}%
\pgfpathrectangle{\pgfqpoint{0.800000in}{0.528000in}}{\pgfqpoint{4.960000in}{3.696000in}}%
\pgfusepath{clip}%
\pgfsetbuttcap%
\pgfsetroundjoin%
\definecolor{currentfill}{rgb}{0.000000,0.000000,0.000000}%
\pgfsetfillcolor{currentfill}%
\pgfsetlinewidth{1.003750pt}%
\definecolor{currentstroke}{rgb}{0.000000,0.000000,0.000000}%
\pgfsetstrokecolor{currentstroke}%
\pgfsetdash{}{0pt}%
\pgfpathmoveto{\pgfqpoint{4.011666in}{2.877687in}}%
\pgfpathcurveto{\pgfqpoint{4.022716in}{2.877687in}}{\pgfqpoint{4.033315in}{2.882077in}}{\pgfqpoint{4.041128in}{2.889891in}}%
\pgfpathcurveto{\pgfqpoint{4.048942in}{2.897704in}}{\pgfqpoint{4.053332in}{2.908303in}}{\pgfqpoint{4.053332in}{2.919353in}}%
\pgfpathcurveto{\pgfqpoint{4.053332in}{2.930404in}}{\pgfqpoint{4.048942in}{2.941003in}}{\pgfqpoint{4.041128in}{2.948816in}}%
\pgfpathcurveto{\pgfqpoint{4.033315in}{2.956630in}}{\pgfqpoint{4.022716in}{2.961020in}}{\pgfqpoint{4.011666in}{2.961020in}}%
\pgfpathcurveto{\pgfqpoint{4.000616in}{2.961020in}}{\pgfqpoint{3.990016in}{2.956630in}}{\pgfqpoint{3.982203in}{2.948816in}}%
\pgfpathcurveto{\pgfqpoint{3.974389in}{2.941003in}}{\pgfqpoint{3.969999in}{2.930404in}}{\pgfqpoint{3.969999in}{2.919353in}}%
\pgfpathcurveto{\pgfqpoint{3.969999in}{2.908303in}}{\pgfqpoint{3.974389in}{2.897704in}}{\pgfqpoint{3.982203in}{2.889891in}}%
\pgfpathcurveto{\pgfqpoint{3.990016in}{2.882077in}}{\pgfqpoint{4.000616in}{2.877687in}}{\pgfqpoint{4.011666in}{2.877687in}}%
\pgfpathclose%
\pgfusepath{stroke,fill}%
\end{pgfscope}%
\begin{pgfscope}%
\pgfpathrectangle{\pgfqpoint{0.800000in}{0.528000in}}{\pgfqpoint{4.960000in}{3.696000in}}%
\pgfusepath{clip}%
\pgfsetbuttcap%
\pgfsetroundjoin%
\definecolor{currentfill}{rgb}{0.000000,0.000000,0.000000}%
\pgfsetfillcolor{currentfill}%
\pgfsetlinewidth{1.003750pt}%
\definecolor{currentstroke}{rgb}{0.000000,0.000000,0.000000}%
\pgfsetstrokecolor{currentstroke}%
\pgfsetdash{}{0pt}%
\pgfpathmoveto{\pgfqpoint{4.011666in}{1.771040in}}%
\pgfpathcurveto{\pgfqpoint{4.022716in}{1.771040in}}{\pgfqpoint{4.033315in}{1.775431in}}{\pgfqpoint{4.041128in}{1.783244in}}%
\pgfpathcurveto{\pgfqpoint{4.048942in}{1.791058in}}{\pgfqpoint{4.053332in}{1.801657in}}{\pgfqpoint{4.053332in}{1.812707in}}%
\pgfpathcurveto{\pgfqpoint{4.053332in}{1.823757in}}{\pgfqpoint{4.048942in}{1.834356in}}{\pgfqpoint{4.041128in}{1.842170in}}%
\pgfpathcurveto{\pgfqpoint{4.033315in}{1.849983in}}{\pgfqpoint{4.022716in}{1.854374in}}{\pgfqpoint{4.011666in}{1.854374in}}%
\pgfpathcurveto{\pgfqpoint{4.000616in}{1.854374in}}{\pgfqpoint{3.990016in}{1.849983in}}{\pgfqpoint{3.982203in}{1.842170in}}%
\pgfpathcurveto{\pgfqpoint{3.974389in}{1.834356in}}{\pgfqpoint{3.969999in}{1.823757in}}{\pgfqpoint{3.969999in}{1.812707in}}%
\pgfpathcurveto{\pgfqpoint{3.969999in}{1.801657in}}{\pgfqpoint{3.974389in}{1.791058in}}{\pgfqpoint{3.982203in}{1.783244in}}%
\pgfpathcurveto{\pgfqpoint{3.990016in}{1.775431in}}{\pgfqpoint{4.000616in}{1.771040in}}{\pgfqpoint{4.011666in}{1.771040in}}%
\pgfpathclose%
\pgfusepath{stroke,fill}%
\end{pgfscope}%
\begin{pgfscope}%
\pgfpathrectangle{\pgfqpoint{0.800000in}{0.528000in}}{\pgfqpoint{4.960000in}{3.696000in}}%
\pgfusepath{clip}%
\pgfsetbuttcap%
\pgfsetroundjoin%
\definecolor{currentfill}{rgb}{0.000000,0.000000,0.000000}%
\pgfsetfillcolor{currentfill}%
\pgfsetlinewidth{1.003750pt}%
\definecolor{currentstroke}{rgb}{0.000000,0.000000,0.000000}%
\pgfsetstrokecolor{currentstroke}%
\pgfsetdash{}{0pt}%
\pgfpathmoveto{\pgfqpoint{4.011666in}{2.877687in}}%
\pgfpathcurveto{\pgfqpoint{4.022716in}{2.877687in}}{\pgfqpoint{4.033315in}{2.882077in}}{\pgfqpoint{4.041128in}{2.889891in}}%
\pgfpathcurveto{\pgfqpoint{4.048942in}{2.897704in}}{\pgfqpoint{4.053332in}{2.908303in}}{\pgfqpoint{4.053332in}{2.919353in}}%
\pgfpathcurveto{\pgfqpoint{4.053332in}{2.930404in}}{\pgfqpoint{4.048942in}{2.941003in}}{\pgfqpoint{4.041128in}{2.948816in}}%
\pgfpathcurveto{\pgfqpoint{4.033315in}{2.956630in}}{\pgfqpoint{4.022716in}{2.961020in}}{\pgfqpoint{4.011666in}{2.961020in}}%
\pgfpathcurveto{\pgfqpoint{4.000616in}{2.961020in}}{\pgfqpoint{3.990016in}{2.956630in}}{\pgfqpoint{3.982203in}{2.948816in}}%
\pgfpathcurveto{\pgfqpoint{3.974389in}{2.941003in}}{\pgfqpoint{3.969999in}{2.930404in}}{\pgfqpoint{3.969999in}{2.919353in}}%
\pgfpathcurveto{\pgfqpoint{3.969999in}{2.908303in}}{\pgfqpoint{3.974389in}{2.897704in}}{\pgfqpoint{3.982203in}{2.889891in}}%
\pgfpathcurveto{\pgfqpoint{3.990016in}{2.882077in}}{\pgfqpoint{4.000616in}{2.877687in}}{\pgfqpoint{4.011666in}{2.877687in}}%
\pgfpathclose%
\pgfusepath{stroke,fill}%
\end{pgfscope}%
\begin{pgfscope}%
\pgfpathrectangle{\pgfqpoint{0.800000in}{0.528000in}}{\pgfqpoint{4.960000in}{3.696000in}}%
\pgfusepath{clip}%
\pgfsetbuttcap%
\pgfsetroundjoin%
\definecolor{currentfill}{rgb}{0.000000,0.000000,0.000000}%
\pgfsetfillcolor{currentfill}%
\pgfsetlinewidth{1.003750pt}%
\definecolor{currentstroke}{rgb}{0.000000,0.000000,0.000000}%
\pgfsetstrokecolor{currentstroke}%
\pgfsetdash{}{0pt}%
\pgfpathmoveto{\pgfqpoint{4.011666in}{1.771040in}}%
\pgfpathcurveto{\pgfqpoint{4.022716in}{1.771040in}}{\pgfqpoint{4.033315in}{1.775431in}}{\pgfqpoint{4.041128in}{1.783244in}}%
\pgfpathcurveto{\pgfqpoint{4.048942in}{1.791058in}}{\pgfqpoint{4.053332in}{1.801657in}}{\pgfqpoint{4.053332in}{1.812707in}}%
\pgfpathcurveto{\pgfqpoint{4.053332in}{1.823757in}}{\pgfqpoint{4.048942in}{1.834356in}}{\pgfqpoint{4.041128in}{1.842170in}}%
\pgfpathcurveto{\pgfqpoint{4.033315in}{1.849983in}}{\pgfqpoint{4.022716in}{1.854374in}}{\pgfqpoint{4.011666in}{1.854374in}}%
\pgfpathcurveto{\pgfqpoint{4.000616in}{1.854374in}}{\pgfqpoint{3.990016in}{1.849983in}}{\pgfqpoint{3.982203in}{1.842170in}}%
\pgfpathcurveto{\pgfqpoint{3.974389in}{1.834356in}}{\pgfqpoint{3.969999in}{1.823757in}}{\pgfqpoint{3.969999in}{1.812707in}}%
\pgfpathcurveto{\pgfqpoint{3.969999in}{1.801657in}}{\pgfqpoint{3.974389in}{1.791058in}}{\pgfqpoint{3.982203in}{1.783244in}}%
\pgfpathcurveto{\pgfqpoint{3.990016in}{1.775431in}}{\pgfqpoint{4.000616in}{1.771040in}}{\pgfqpoint{4.011666in}{1.771040in}}%
\pgfpathclose%
\pgfusepath{stroke,fill}%
\end{pgfscope}%
\begin{pgfscope}%
\pgfpathrectangle{\pgfqpoint{0.800000in}{0.528000in}}{\pgfqpoint{4.960000in}{3.696000in}}%
\pgfusepath{clip}%
\pgfsetbuttcap%
\pgfsetroundjoin%
\definecolor{currentfill}{rgb}{0.000000,0.000000,0.000000}%
\pgfsetfillcolor{currentfill}%
\pgfsetlinewidth{1.003750pt}%
\definecolor{currentstroke}{rgb}{0.000000,0.000000,0.000000}%
\pgfsetstrokecolor{currentstroke}%
\pgfsetdash{}{0pt}%
\pgfpathmoveto{\pgfqpoint{4.011666in}{1.771040in}}%
\pgfpathcurveto{\pgfqpoint{4.022716in}{1.771040in}}{\pgfqpoint{4.033315in}{1.775431in}}{\pgfqpoint{4.041128in}{1.783244in}}%
\pgfpathcurveto{\pgfqpoint{4.048942in}{1.791058in}}{\pgfqpoint{4.053332in}{1.801657in}}{\pgfqpoint{4.053332in}{1.812707in}}%
\pgfpathcurveto{\pgfqpoint{4.053332in}{1.823757in}}{\pgfqpoint{4.048942in}{1.834356in}}{\pgfqpoint{4.041128in}{1.842170in}}%
\pgfpathcurveto{\pgfqpoint{4.033315in}{1.849983in}}{\pgfqpoint{4.022716in}{1.854374in}}{\pgfqpoint{4.011666in}{1.854374in}}%
\pgfpathcurveto{\pgfqpoint{4.000616in}{1.854374in}}{\pgfqpoint{3.990016in}{1.849983in}}{\pgfqpoint{3.982203in}{1.842170in}}%
\pgfpathcurveto{\pgfqpoint{3.974389in}{1.834356in}}{\pgfqpoint{3.969999in}{1.823757in}}{\pgfqpoint{3.969999in}{1.812707in}}%
\pgfpathcurveto{\pgfqpoint{3.969999in}{1.801657in}}{\pgfqpoint{3.974389in}{1.791058in}}{\pgfqpoint{3.982203in}{1.783244in}}%
\pgfpathcurveto{\pgfqpoint{3.990016in}{1.775431in}}{\pgfqpoint{4.000616in}{1.771040in}}{\pgfqpoint{4.011666in}{1.771040in}}%
\pgfpathclose%
\pgfusepath{stroke,fill}%
\end{pgfscope}%
\begin{pgfscope}%
\pgfpathrectangle{\pgfqpoint{0.800000in}{0.528000in}}{\pgfqpoint{4.960000in}{3.696000in}}%
\pgfusepath{clip}%
\pgfsetbuttcap%
\pgfsetroundjoin%
\definecolor{currentfill}{rgb}{0.000000,0.000000,0.000000}%
\pgfsetfillcolor{currentfill}%
\pgfsetlinewidth{1.003750pt}%
\definecolor{currentstroke}{rgb}{0.000000,0.000000,0.000000}%
\pgfsetstrokecolor{currentstroke}%
\pgfsetdash{}{0pt}%
\pgfpathmoveto{\pgfqpoint{4.011666in}{1.771040in}}%
\pgfpathcurveto{\pgfqpoint{4.022716in}{1.771040in}}{\pgfqpoint{4.033315in}{1.775431in}}{\pgfqpoint{4.041128in}{1.783244in}}%
\pgfpathcurveto{\pgfqpoint{4.048942in}{1.791058in}}{\pgfqpoint{4.053332in}{1.801657in}}{\pgfqpoint{4.053332in}{1.812707in}}%
\pgfpathcurveto{\pgfqpoint{4.053332in}{1.823757in}}{\pgfqpoint{4.048942in}{1.834356in}}{\pgfqpoint{4.041128in}{1.842170in}}%
\pgfpathcurveto{\pgfqpoint{4.033315in}{1.849983in}}{\pgfqpoint{4.022716in}{1.854374in}}{\pgfqpoint{4.011666in}{1.854374in}}%
\pgfpathcurveto{\pgfqpoint{4.000616in}{1.854374in}}{\pgfqpoint{3.990016in}{1.849983in}}{\pgfqpoint{3.982203in}{1.842170in}}%
\pgfpathcurveto{\pgfqpoint{3.974389in}{1.834356in}}{\pgfqpoint{3.969999in}{1.823757in}}{\pgfqpoint{3.969999in}{1.812707in}}%
\pgfpathcurveto{\pgfqpoint{3.969999in}{1.801657in}}{\pgfqpoint{3.974389in}{1.791058in}}{\pgfqpoint{3.982203in}{1.783244in}}%
\pgfpathcurveto{\pgfqpoint{3.990016in}{1.775431in}}{\pgfqpoint{4.000616in}{1.771040in}}{\pgfqpoint{4.011666in}{1.771040in}}%
\pgfpathclose%
\pgfusepath{stroke,fill}%
\end{pgfscope}%
\begin{pgfscope}%
\pgfpathrectangle{\pgfqpoint{0.800000in}{0.528000in}}{\pgfqpoint{4.960000in}{3.696000in}}%
\pgfusepath{clip}%
\pgfsetbuttcap%
\pgfsetroundjoin%
\definecolor{currentfill}{rgb}{0.000000,0.000000,0.000000}%
\pgfsetfillcolor{currentfill}%
\pgfsetlinewidth{1.003750pt}%
\definecolor{currentstroke}{rgb}{0.000000,0.000000,0.000000}%
\pgfsetstrokecolor{currentstroke}%
\pgfsetdash{}{0pt}%
\pgfpathmoveto{\pgfqpoint{4.011666in}{1.771040in}}%
\pgfpathcurveto{\pgfqpoint{4.022716in}{1.771040in}}{\pgfqpoint{4.033315in}{1.775431in}}{\pgfqpoint{4.041128in}{1.783244in}}%
\pgfpathcurveto{\pgfqpoint{4.048942in}{1.791058in}}{\pgfqpoint{4.053332in}{1.801657in}}{\pgfqpoint{4.053332in}{1.812707in}}%
\pgfpathcurveto{\pgfqpoint{4.053332in}{1.823757in}}{\pgfqpoint{4.048942in}{1.834356in}}{\pgfqpoint{4.041128in}{1.842170in}}%
\pgfpathcurveto{\pgfqpoint{4.033315in}{1.849983in}}{\pgfqpoint{4.022716in}{1.854374in}}{\pgfqpoint{4.011666in}{1.854374in}}%
\pgfpathcurveto{\pgfqpoint{4.000616in}{1.854374in}}{\pgfqpoint{3.990016in}{1.849983in}}{\pgfqpoint{3.982203in}{1.842170in}}%
\pgfpathcurveto{\pgfqpoint{3.974389in}{1.834356in}}{\pgfqpoint{3.969999in}{1.823757in}}{\pgfqpoint{3.969999in}{1.812707in}}%
\pgfpathcurveto{\pgfqpoint{3.969999in}{1.801657in}}{\pgfqpoint{3.974389in}{1.791058in}}{\pgfqpoint{3.982203in}{1.783244in}}%
\pgfpathcurveto{\pgfqpoint{3.990016in}{1.775431in}}{\pgfqpoint{4.000616in}{1.771040in}}{\pgfqpoint{4.011666in}{1.771040in}}%
\pgfpathclose%
\pgfusepath{stroke,fill}%
\end{pgfscope}%
\begin{pgfscope}%
\pgfpathrectangle{\pgfqpoint{0.800000in}{0.528000in}}{\pgfqpoint{4.960000in}{3.696000in}}%
\pgfusepath{clip}%
\pgfsetbuttcap%
\pgfsetroundjoin%
\definecolor{currentfill}{rgb}{0.000000,0.000000,0.000000}%
\pgfsetfillcolor{currentfill}%
\pgfsetlinewidth{1.003750pt}%
\definecolor{currentstroke}{rgb}{0.000000,0.000000,0.000000}%
\pgfsetstrokecolor{currentstroke}%
\pgfsetdash{}{0pt}%
\pgfpathmoveto{\pgfqpoint{4.011666in}{1.771040in}}%
\pgfpathcurveto{\pgfqpoint{4.022716in}{1.771040in}}{\pgfqpoint{4.033315in}{1.775431in}}{\pgfqpoint{4.041128in}{1.783244in}}%
\pgfpathcurveto{\pgfqpoint{4.048942in}{1.791058in}}{\pgfqpoint{4.053332in}{1.801657in}}{\pgfqpoint{4.053332in}{1.812707in}}%
\pgfpathcurveto{\pgfqpoint{4.053332in}{1.823757in}}{\pgfqpoint{4.048942in}{1.834356in}}{\pgfqpoint{4.041128in}{1.842170in}}%
\pgfpathcurveto{\pgfqpoint{4.033315in}{1.849983in}}{\pgfqpoint{4.022716in}{1.854374in}}{\pgfqpoint{4.011666in}{1.854374in}}%
\pgfpathcurveto{\pgfqpoint{4.000616in}{1.854374in}}{\pgfqpoint{3.990016in}{1.849983in}}{\pgfqpoint{3.982203in}{1.842170in}}%
\pgfpathcurveto{\pgfqpoint{3.974389in}{1.834356in}}{\pgfqpoint{3.969999in}{1.823757in}}{\pgfqpoint{3.969999in}{1.812707in}}%
\pgfpathcurveto{\pgfqpoint{3.969999in}{1.801657in}}{\pgfqpoint{3.974389in}{1.791058in}}{\pgfqpoint{3.982203in}{1.783244in}}%
\pgfpathcurveto{\pgfqpoint{3.990016in}{1.775431in}}{\pgfqpoint{4.000616in}{1.771040in}}{\pgfqpoint{4.011666in}{1.771040in}}%
\pgfpathclose%
\pgfusepath{stroke,fill}%
\end{pgfscope}%
\begin{pgfscope}%
\pgfpathrectangle{\pgfqpoint{0.800000in}{0.528000in}}{\pgfqpoint{4.960000in}{3.696000in}}%
\pgfusepath{clip}%
\pgfsetbuttcap%
\pgfsetroundjoin%
\definecolor{currentfill}{rgb}{0.000000,0.000000,0.000000}%
\pgfsetfillcolor{currentfill}%
\pgfsetlinewidth{1.003750pt}%
\definecolor{currentstroke}{rgb}{0.000000,0.000000,0.000000}%
\pgfsetstrokecolor{currentstroke}%
\pgfsetdash{}{0pt}%
\pgfpathmoveto{\pgfqpoint{4.011666in}{2.877687in}}%
\pgfpathcurveto{\pgfqpoint{4.022716in}{2.877687in}}{\pgfqpoint{4.033315in}{2.882077in}}{\pgfqpoint{4.041128in}{2.889891in}}%
\pgfpathcurveto{\pgfqpoint{4.048942in}{2.897704in}}{\pgfqpoint{4.053332in}{2.908303in}}{\pgfqpoint{4.053332in}{2.919353in}}%
\pgfpathcurveto{\pgfqpoint{4.053332in}{2.930404in}}{\pgfqpoint{4.048942in}{2.941003in}}{\pgfqpoint{4.041128in}{2.948816in}}%
\pgfpathcurveto{\pgfqpoint{4.033315in}{2.956630in}}{\pgfqpoint{4.022716in}{2.961020in}}{\pgfqpoint{4.011666in}{2.961020in}}%
\pgfpathcurveto{\pgfqpoint{4.000616in}{2.961020in}}{\pgfqpoint{3.990016in}{2.956630in}}{\pgfqpoint{3.982203in}{2.948816in}}%
\pgfpathcurveto{\pgfqpoint{3.974389in}{2.941003in}}{\pgfqpoint{3.969999in}{2.930404in}}{\pgfqpoint{3.969999in}{2.919353in}}%
\pgfpathcurveto{\pgfqpoint{3.969999in}{2.908303in}}{\pgfqpoint{3.974389in}{2.897704in}}{\pgfqpoint{3.982203in}{2.889891in}}%
\pgfpathcurveto{\pgfqpoint{3.990016in}{2.882077in}}{\pgfqpoint{4.000616in}{2.877687in}}{\pgfqpoint{4.011666in}{2.877687in}}%
\pgfpathclose%
\pgfusepath{stroke,fill}%
\end{pgfscope}%
\begin{pgfscope}%
\pgfpathrectangle{\pgfqpoint{0.800000in}{0.528000in}}{\pgfqpoint{4.960000in}{3.696000in}}%
\pgfusepath{clip}%
\pgfsetbuttcap%
\pgfsetroundjoin%
\definecolor{currentfill}{rgb}{0.000000,0.000000,0.000000}%
\pgfsetfillcolor{currentfill}%
\pgfsetlinewidth{1.003750pt}%
\definecolor{currentstroke}{rgb}{0.000000,0.000000,0.000000}%
\pgfsetstrokecolor{currentstroke}%
\pgfsetdash{}{0pt}%
\pgfpathmoveto{\pgfqpoint{4.011666in}{1.771040in}}%
\pgfpathcurveto{\pgfqpoint{4.022716in}{1.771040in}}{\pgfqpoint{4.033315in}{1.775431in}}{\pgfqpoint{4.041128in}{1.783244in}}%
\pgfpathcurveto{\pgfqpoint{4.048942in}{1.791058in}}{\pgfqpoint{4.053332in}{1.801657in}}{\pgfqpoint{4.053332in}{1.812707in}}%
\pgfpathcurveto{\pgfqpoint{4.053332in}{1.823757in}}{\pgfqpoint{4.048942in}{1.834356in}}{\pgfqpoint{4.041128in}{1.842170in}}%
\pgfpathcurveto{\pgfqpoint{4.033315in}{1.849983in}}{\pgfqpoint{4.022716in}{1.854374in}}{\pgfqpoint{4.011666in}{1.854374in}}%
\pgfpathcurveto{\pgfqpoint{4.000616in}{1.854374in}}{\pgfqpoint{3.990016in}{1.849983in}}{\pgfqpoint{3.982203in}{1.842170in}}%
\pgfpathcurveto{\pgfqpoint{3.974389in}{1.834356in}}{\pgfqpoint{3.969999in}{1.823757in}}{\pgfqpoint{3.969999in}{1.812707in}}%
\pgfpathcurveto{\pgfqpoint{3.969999in}{1.801657in}}{\pgfqpoint{3.974389in}{1.791058in}}{\pgfqpoint{3.982203in}{1.783244in}}%
\pgfpathcurveto{\pgfqpoint{3.990016in}{1.775431in}}{\pgfqpoint{4.000616in}{1.771040in}}{\pgfqpoint{4.011666in}{1.771040in}}%
\pgfpathclose%
\pgfusepath{stroke,fill}%
\end{pgfscope}%
\begin{pgfscope}%
\pgfpathrectangle{\pgfqpoint{0.800000in}{0.528000in}}{\pgfqpoint{4.960000in}{3.696000in}}%
\pgfusepath{clip}%
\pgfsetbuttcap%
\pgfsetroundjoin%
\definecolor{currentfill}{rgb}{0.000000,0.000000,0.000000}%
\pgfsetfillcolor{currentfill}%
\pgfsetlinewidth{1.003750pt}%
\definecolor{currentstroke}{rgb}{0.000000,0.000000,0.000000}%
\pgfsetstrokecolor{currentstroke}%
\pgfsetdash{}{0pt}%
\pgfpathmoveto{\pgfqpoint{4.011666in}{1.771040in}}%
\pgfpathcurveto{\pgfqpoint{4.022716in}{1.771040in}}{\pgfqpoint{4.033315in}{1.775431in}}{\pgfqpoint{4.041128in}{1.783244in}}%
\pgfpathcurveto{\pgfqpoint{4.048942in}{1.791058in}}{\pgfqpoint{4.053332in}{1.801657in}}{\pgfqpoint{4.053332in}{1.812707in}}%
\pgfpathcurveto{\pgfqpoint{4.053332in}{1.823757in}}{\pgfqpoint{4.048942in}{1.834356in}}{\pgfqpoint{4.041128in}{1.842170in}}%
\pgfpathcurveto{\pgfqpoint{4.033315in}{1.849983in}}{\pgfqpoint{4.022716in}{1.854374in}}{\pgfqpoint{4.011666in}{1.854374in}}%
\pgfpathcurveto{\pgfqpoint{4.000616in}{1.854374in}}{\pgfqpoint{3.990016in}{1.849983in}}{\pgfqpoint{3.982203in}{1.842170in}}%
\pgfpathcurveto{\pgfqpoint{3.974389in}{1.834356in}}{\pgfqpoint{3.969999in}{1.823757in}}{\pgfqpoint{3.969999in}{1.812707in}}%
\pgfpathcurveto{\pgfqpoint{3.969999in}{1.801657in}}{\pgfqpoint{3.974389in}{1.791058in}}{\pgfqpoint{3.982203in}{1.783244in}}%
\pgfpathcurveto{\pgfqpoint{3.990016in}{1.775431in}}{\pgfqpoint{4.000616in}{1.771040in}}{\pgfqpoint{4.011666in}{1.771040in}}%
\pgfpathclose%
\pgfusepath{stroke,fill}%
\end{pgfscope}%
\begin{pgfscope}%
\pgfpathrectangle{\pgfqpoint{0.800000in}{0.528000in}}{\pgfqpoint{4.960000in}{3.696000in}}%
\pgfusepath{clip}%
\pgfsetbuttcap%
\pgfsetroundjoin%
\definecolor{currentfill}{rgb}{0.000000,0.000000,0.000000}%
\pgfsetfillcolor{currentfill}%
\pgfsetlinewidth{1.003750pt}%
\definecolor{currentstroke}{rgb}{0.000000,0.000000,0.000000}%
\pgfsetstrokecolor{currentstroke}%
\pgfsetdash{}{0pt}%
\pgfpathmoveto{\pgfqpoint{4.011666in}{2.877687in}}%
\pgfpathcurveto{\pgfqpoint{4.022716in}{2.877687in}}{\pgfqpoint{4.033315in}{2.882077in}}{\pgfqpoint{4.041128in}{2.889891in}}%
\pgfpathcurveto{\pgfqpoint{4.048942in}{2.897704in}}{\pgfqpoint{4.053332in}{2.908303in}}{\pgfqpoint{4.053332in}{2.919353in}}%
\pgfpathcurveto{\pgfqpoint{4.053332in}{2.930404in}}{\pgfqpoint{4.048942in}{2.941003in}}{\pgfqpoint{4.041128in}{2.948816in}}%
\pgfpathcurveto{\pgfqpoint{4.033315in}{2.956630in}}{\pgfqpoint{4.022716in}{2.961020in}}{\pgfqpoint{4.011666in}{2.961020in}}%
\pgfpathcurveto{\pgfqpoint{4.000616in}{2.961020in}}{\pgfqpoint{3.990016in}{2.956630in}}{\pgfqpoint{3.982203in}{2.948816in}}%
\pgfpathcurveto{\pgfqpoint{3.974389in}{2.941003in}}{\pgfqpoint{3.969999in}{2.930404in}}{\pgfqpoint{3.969999in}{2.919353in}}%
\pgfpathcurveto{\pgfqpoint{3.969999in}{2.908303in}}{\pgfqpoint{3.974389in}{2.897704in}}{\pgfqpoint{3.982203in}{2.889891in}}%
\pgfpathcurveto{\pgfqpoint{3.990016in}{2.882077in}}{\pgfqpoint{4.000616in}{2.877687in}}{\pgfqpoint{4.011666in}{2.877687in}}%
\pgfpathclose%
\pgfusepath{stroke,fill}%
\end{pgfscope}%
\begin{pgfscope}%
\pgfpathrectangle{\pgfqpoint{0.800000in}{0.528000in}}{\pgfqpoint{4.960000in}{3.696000in}}%
\pgfusepath{clip}%
\pgfsetbuttcap%
\pgfsetroundjoin%
\definecolor{currentfill}{rgb}{0.000000,0.000000,0.000000}%
\pgfsetfillcolor{currentfill}%
\pgfsetlinewidth{1.003750pt}%
\definecolor{currentstroke}{rgb}{0.000000,0.000000,0.000000}%
\pgfsetstrokecolor{currentstroke}%
\pgfsetdash{}{0pt}%
\pgfpathmoveto{\pgfqpoint{4.011666in}{2.877687in}}%
\pgfpathcurveto{\pgfqpoint{4.022716in}{2.877687in}}{\pgfqpoint{4.033315in}{2.882077in}}{\pgfqpoint{4.041128in}{2.889891in}}%
\pgfpathcurveto{\pgfqpoint{4.048942in}{2.897704in}}{\pgfqpoint{4.053332in}{2.908303in}}{\pgfqpoint{4.053332in}{2.919353in}}%
\pgfpathcurveto{\pgfqpoint{4.053332in}{2.930404in}}{\pgfqpoint{4.048942in}{2.941003in}}{\pgfqpoint{4.041128in}{2.948816in}}%
\pgfpathcurveto{\pgfqpoint{4.033315in}{2.956630in}}{\pgfqpoint{4.022716in}{2.961020in}}{\pgfqpoint{4.011666in}{2.961020in}}%
\pgfpathcurveto{\pgfqpoint{4.000616in}{2.961020in}}{\pgfqpoint{3.990016in}{2.956630in}}{\pgfqpoint{3.982203in}{2.948816in}}%
\pgfpathcurveto{\pgfqpoint{3.974389in}{2.941003in}}{\pgfqpoint{3.969999in}{2.930404in}}{\pgfqpoint{3.969999in}{2.919353in}}%
\pgfpathcurveto{\pgfqpoint{3.969999in}{2.908303in}}{\pgfqpoint{3.974389in}{2.897704in}}{\pgfqpoint{3.982203in}{2.889891in}}%
\pgfpathcurveto{\pgfqpoint{3.990016in}{2.882077in}}{\pgfqpoint{4.000616in}{2.877687in}}{\pgfqpoint{4.011666in}{2.877687in}}%
\pgfpathclose%
\pgfusepath{stroke,fill}%
\end{pgfscope}%
\begin{pgfscope}%
\pgfpathrectangle{\pgfqpoint{0.800000in}{0.528000in}}{\pgfqpoint{4.960000in}{3.696000in}}%
\pgfusepath{clip}%
\pgfsetbuttcap%
\pgfsetroundjoin%
\definecolor{currentfill}{rgb}{0.000000,0.000000,0.000000}%
\pgfsetfillcolor{currentfill}%
\pgfsetlinewidth{1.003750pt}%
\definecolor{currentstroke}{rgb}{0.000000,0.000000,0.000000}%
\pgfsetstrokecolor{currentstroke}%
\pgfsetdash{}{0pt}%
\pgfpathmoveto{\pgfqpoint{4.011666in}{1.771040in}}%
\pgfpathcurveto{\pgfqpoint{4.022716in}{1.771040in}}{\pgfqpoint{4.033315in}{1.775431in}}{\pgfqpoint{4.041128in}{1.783244in}}%
\pgfpathcurveto{\pgfqpoint{4.048942in}{1.791058in}}{\pgfqpoint{4.053332in}{1.801657in}}{\pgfqpoint{4.053332in}{1.812707in}}%
\pgfpathcurveto{\pgfqpoint{4.053332in}{1.823757in}}{\pgfqpoint{4.048942in}{1.834356in}}{\pgfqpoint{4.041128in}{1.842170in}}%
\pgfpathcurveto{\pgfqpoint{4.033315in}{1.849983in}}{\pgfqpoint{4.022716in}{1.854374in}}{\pgfqpoint{4.011666in}{1.854374in}}%
\pgfpathcurveto{\pgfqpoint{4.000616in}{1.854374in}}{\pgfqpoint{3.990016in}{1.849983in}}{\pgfqpoint{3.982203in}{1.842170in}}%
\pgfpathcurveto{\pgfqpoint{3.974389in}{1.834356in}}{\pgfqpoint{3.969999in}{1.823757in}}{\pgfqpoint{3.969999in}{1.812707in}}%
\pgfpathcurveto{\pgfqpoint{3.969999in}{1.801657in}}{\pgfqpoint{3.974389in}{1.791058in}}{\pgfqpoint{3.982203in}{1.783244in}}%
\pgfpathcurveto{\pgfqpoint{3.990016in}{1.775431in}}{\pgfqpoint{4.000616in}{1.771040in}}{\pgfqpoint{4.011666in}{1.771040in}}%
\pgfpathclose%
\pgfusepath{stroke,fill}%
\end{pgfscope}%
\begin{pgfscope}%
\pgfpathrectangle{\pgfqpoint{0.800000in}{0.528000in}}{\pgfqpoint{4.960000in}{3.696000in}}%
\pgfusepath{clip}%
\pgfsetbuttcap%
\pgfsetroundjoin%
\definecolor{currentfill}{rgb}{0.000000,0.000000,0.000000}%
\pgfsetfillcolor{currentfill}%
\pgfsetlinewidth{1.003750pt}%
\definecolor{currentstroke}{rgb}{0.000000,0.000000,0.000000}%
\pgfsetstrokecolor{currentstroke}%
\pgfsetdash{}{0pt}%
\pgfpathmoveto{\pgfqpoint{4.011666in}{1.771040in}}%
\pgfpathcurveto{\pgfqpoint{4.022716in}{1.771040in}}{\pgfqpoint{4.033315in}{1.775431in}}{\pgfqpoint{4.041128in}{1.783244in}}%
\pgfpathcurveto{\pgfqpoint{4.048942in}{1.791058in}}{\pgfqpoint{4.053332in}{1.801657in}}{\pgfqpoint{4.053332in}{1.812707in}}%
\pgfpathcurveto{\pgfqpoint{4.053332in}{1.823757in}}{\pgfqpoint{4.048942in}{1.834356in}}{\pgfqpoint{4.041128in}{1.842170in}}%
\pgfpathcurveto{\pgfqpoint{4.033315in}{1.849983in}}{\pgfqpoint{4.022716in}{1.854374in}}{\pgfqpoint{4.011666in}{1.854374in}}%
\pgfpathcurveto{\pgfqpoint{4.000616in}{1.854374in}}{\pgfqpoint{3.990016in}{1.849983in}}{\pgfqpoint{3.982203in}{1.842170in}}%
\pgfpathcurveto{\pgfqpoint{3.974389in}{1.834356in}}{\pgfqpoint{3.969999in}{1.823757in}}{\pgfqpoint{3.969999in}{1.812707in}}%
\pgfpathcurveto{\pgfqpoint{3.969999in}{1.801657in}}{\pgfqpoint{3.974389in}{1.791058in}}{\pgfqpoint{3.982203in}{1.783244in}}%
\pgfpathcurveto{\pgfqpoint{3.990016in}{1.775431in}}{\pgfqpoint{4.000616in}{1.771040in}}{\pgfqpoint{4.011666in}{1.771040in}}%
\pgfpathclose%
\pgfusepath{stroke,fill}%
\end{pgfscope}%
\begin{pgfscope}%
\pgfpathrectangle{\pgfqpoint{0.800000in}{0.528000in}}{\pgfqpoint{4.960000in}{3.696000in}}%
\pgfusepath{clip}%
\pgfsetbuttcap%
\pgfsetroundjoin%
\definecolor{currentfill}{rgb}{0.000000,0.000000,0.000000}%
\pgfsetfillcolor{currentfill}%
\pgfsetlinewidth{1.003750pt}%
\definecolor{currentstroke}{rgb}{0.000000,0.000000,0.000000}%
\pgfsetstrokecolor{currentstroke}%
\pgfsetdash{}{0pt}%
\pgfpathmoveto{\pgfqpoint{4.011666in}{1.771040in}}%
\pgfpathcurveto{\pgfqpoint{4.022716in}{1.771040in}}{\pgfqpoint{4.033315in}{1.775431in}}{\pgfqpoint{4.041128in}{1.783244in}}%
\pgfpathcurveto{\pgfqpoint{4.048942in}{1.791058in}}{\pgfqpoint{4.053332in}{1.801657in}}{\pgfqpoint{4.053332in}{1.812707in}}%
\pgfpathcurveto{\pgfqpoint{4.053332in}{1.823757in}}{\pgfqpoint{4.048942in}{1.834356in}}{\pgfqpoint{4.041128in}{1.842170in}}%
\pgfpathcurveto{\pgfqpoint{4.033315in}{1.849983in}}{\pgfqpoint{4.022716in}{1.854374in}}{\pgfqpoint{4.011666in}{1.854374in}}%
\pgfpathcurveto{\pgfqpoint{4.000616in}{1.854374in}}{\pgfqpoint{3.990016in}{1.849983in}}{\pgfqpoint{3.982203in}{1.842170in}}%
\pgfpathcurveto{\pgfqpoint{3.974389in}{1.834356in}}{\pgfqpoint{3.969999in}{1.823757in}}{\pgfqpoint{3.969999in}{1.812707in}}%
\pgfpathcurveto{\pgfqpoint{3.969999in}{1.801657in}}{\pgfqpoint{3.974389in}{1.791058in}}{\pgfqpoint{3.982203in}{1.783244in}}%
\pgfpathcurveto{\pgfqpoint{3.990016in}{1.775431in}}{\pgfqpoint{4.000616in}{1.771040in}}{\pgfqpoint{4.011666in}{1.771040in}}%
\pgfpathclose%
\pgfusepath{stroke,fill}%
\end{pgfscope}%
\begin{pgfscope}%
\pgfpathrectangle{\pgfqpoint{0.800000in}{0.528000in}}{\pgfqpoint{4.960000in}{3.696000in}}%
\pgfusepath{clip}%
\pgfsetbuttcap%
\pgfsetroundjoin%
\definecolor{currentfill}{rgb}{0.000000,0.000000,0.000000}%
\pgfsetfillcolor{currentfill}%
\pgfsetlinewidth{1.003750pt}%
\definecolor{currentstroke}{rgb}{0.000000,0.000000,0.000000}%
\pgfsetstrokecolor{currentstroke}%
\pgfsetdash{}{0pt}%
\pgfpathmoveto{\pgfqpoint{4.011666in}{1.771040in}}%
\pgfpathcurveto{\pgfqpoint{4.022716in}{1.771040in}}{\pgfqpoint{4.033315in}{1.775431in}}{\pgfqpoint{4.041128in}{1.783244in}}%
\pgfpathcurveto{\pgfqpoint{4.048942in}{1.791058in}}{\pgfqpoint{4.053332in}{1.801657in}}{\pgfqpoint{4.053332in}{1.812707in}}%
\pgfpathcurveto{\pgfqpoint{4.053332in}{1.823757in}}{\pgfqpoint{4.048942in}{1.834356in}}{\pgfqpoint{4.041128in}{1.842170in}}%
\pgfpathcurveto{\pgfqpoint{4.033315in}{1.849983in}}{\pgfqpoint{4.022716in}{1.854374in}}{\pgfqpoint{4.011666in}{1.854374in}}%
\pgfpathcurveto{\pgfqpoint{4.000616in}{1.854374in}}{\pgfqpoint{3.990016in}{1.849983in}}{\pgfqpoint{3.982203in}{1.842170in}}%
\pgfpathcurveto{\pgfqpoint{3.974389in}{1.834356in}}{\pgfqpoint{3.969999in}{1.823757in}}{\pgfqpoint{3.969999in}{1.812707in}}%
\pgfpathcurveto{\pgfqpoint{3.969999in}{1.801657in}}{\pgfqpoint{3.974389in}{1.791058in}}{\pgfqpoint{3.982203in}{1.783244in}}%
\pgfpathcurveto{\pgfqpoint{3.990016in}{1.775431in}}{\pgfqpoint{4.000616in}{1.771040in}}{\pgfqpoint{4.011666in}{1.771040in}}%
\pgfpathclose%
\pgfusepath{stroke,fill}%
\end{pgfscope}%
\begin{pgfscope}%
\pgfpathrectangle{\pgfqpoint{0.800000in}{0.528000in}}{\pgfqpoint{4.960000in}{3.696000in}}%
\pgfusepath{clip}%
\pgfsetbuttcap%
\pgfsetroundjoin%
\definecolor{currentfill}{rgb}{0.000000,0.000000,0.000000}%
\pgfsetfillcolor{currentfill}%
\pgfsetlinewidth{1.003750pt}%
\definecolor{currentstroke}{rgb}{0.000000,0.000000,0.000000}%
\pgfsetstrokecolor{currentstroke}%
\pgfsetdash{}{0pt}%
\pgfpathmoveto{\pgfqpoint{4.011666in}{2.877687in}}%
\pgfpathcurveto{\pgfqpoint{4.022716in}{2.877687in}}{\pgfqpoint{4.033315in}{2.882077in}}{\pgfqpoint{4.041128in}{2.889891in}}%
\pgfpathcurveto{\pgfqpoint{4.048942in}{2.897704in}}{\pgfqpoint{4.053332in}{2.908303in}}{\pgfqpoint{4.053332in}{2.919353in}}%
\pgfpathcurveto{\pgfqpoint{4.053332in}{2.930404in}}{\pgfqpoint{4.048942in}{2.941003in}}{\pgfqpoint{4.041128in}{2.948816in}}%
\pgfpathcurveto{\pgfqpoint{4.033315in}{2.956630in}}{\pgfqpoint{4.022716in}{2.961020in}}{\pgfqpoint{4.011666in}{2.961020in}}%
\pgfpathcurveto{\pgfqpoint{4.000616in}{2.961020in}}{\pgfqpoint{3.990016in}{2.956630in}}{\pgfqpoint{3.982203in}{2.948816in}}%
\pgfpathcurveto{\pgfqpoint{3.974389in}{2.941003in}}{\pgfqpoint{3.969999in}{2.930404in}}{\pgfqpoint{3.969999in}{2.919353in}}%
\pgfpathcurveto{\pgfqpoint{3.969999in}{2.908303in}}{\pgfqpoint{3.974389in}{2.897704in}}{\pgfqpoint{3.982203in}{2.889891in}}%
\pgfpathcurveto{\pgfqpoint{3.990016in}{2.882077in}}{\pgfqpoint{4.000616in}{2.877687in}}{\pgfqpoint{4.011666in}{2.877687in}}%
\pgfpathclose%
\pgfusepath{stroke,fill}%
\end{pgfscope}%
\begin{pgfscope}%
\pgfpathrectangle{\pgfqpoint{0.800000in}{0.528000in}}{\pgfqpoint{4.960000in}{3.696000in}}%
\pgfusepath{clip}%
\pgfsetbuttcap%
\pgfsetroundjoin%
\definecolor{currentfill}{rgb}{0.000000,0.000000,0.000000}%
\pgfsetfillcolor{currentfill}%
\pgfsetlinewidth{1.003750pt}%
\definecolor{currentstroke}{rgb}{0.000000,0.000000,0.000000}%
\pgfsetstrokecolor{currentstroke}%
\pgfsetdash{}{0pt}%
\pgfpathmoveto{\pgfqpoint{4.011666in}{2.877687in}}%
\pgfpathcurveto{\pgfqpoint{4.022716in}{2.877687in}}{\pgfqpoint{4.033315in}{2.882077in}}{\pgfqpoint{4.041128in}{2.889891in}}%
\pgfpathcurveto{\pgfqpoint{4.048942in}{2.897704in}}{\pgfqpoint{4.053332in}{2.908303in}}{\pgfqpoint{4.053332in}{2.919353in}}%
\pgfpathcurveto{\pgfqpoint{4.053332in}{2.930404in}}{\pgfqpoint{4.048942in}{2.941003in}}{\pgfqpoint{4.041128in}{2.948816in}}%
\pgfpathcurveto{\pgfqpoint{4.033315in}{2.956630in}}{\pgfqpoint{4.022716in}{2.961020in}}{\pgfqpoint{4.011666in}{2.961020in}}%
\pgfpathcurveto{\pgfqpoint{4.000616in}{2.961020in}}{\pgfqpoint{3.990016in}{2.956630in}}{\pgfqpoint{3.982203in}{2.948816in}}%
\pgfpathcurveto{\pgfqpoint{3.974389in}{2.941003in}}{\pgfqpoint{3.969999in}{2.930404in}}{\pgfqpoint{3.969999in}{2.919353in}}%
\pgfpathcurveto{\pgfqpoint{3.969999in}{2.908303in}}{\pgfqpoint{3.974389in}{2.897704in}}{\pgfqpoint{3.982203in}{2.889891in}}%
\pgfpathcurveto{\pgfqpoint{3.990016in}{2.882077in}}{\pgfqpoint{4.000616in}{2.877687in}}{\pgfqpoint{4.011666in}{2.877687in}}%
\pgfpathclose%
\pgfusepath{stroke,fill}%
\end{pgfscope}%
\begin{pgfscope}%
\pgfpathrectangle{\pgfqpoint{0.800000in}{0.528000in}}{\pgfqpoint{4.960000in}{3.696000in}}%
\pgfusepath{clip}%
\pgfsetbuttcap%
\pgfsetroundjoin%
\definecolor{currentfill}{rgb}{0.000000,0.000000,0.000000}%
\pgfsetfillcolor{currentfill}%
\pgfsetlinewidth{1.003750pt}%
\definecolor{currentstroke}{rgb}{0.000000,0.000000,0.000000}%
\pgfsetstrokecolor{currentstroke}%
\pgfsetdash{}{0pt}%
\pgfpathmoveto{\pgfqpoint{4.011666in}{1.771040in}}%
\pgfpathcurveto{\pgfqpoint{4.022716in}{1.771040in}}{\pgfqpoint{4.033315in}{1.775431in}}{\pgfqpoint{4.041128in}{1.783244in}}%
\pgfpathcurveto{\pgfqpoint{4.048942in}{1.791058in}}{\pgfqpoint{4.053332in}{1.801657in}}{\pgfqpoint{4.053332in}{1.812707in}}%
\pgfpathcurveto{\pgfqpoint{4.053332in}{1.823757in}}{\pgfqpoint{4.048942in}{1.834356in}}{\pgfqpoint{4.041128in}{1.842170in}}%
\pgfpathcurveto{\pgfqpoint{4.033315in}{1.849983in}}{\pgfqpoint{4.022716in}{1.854374in}}{\pgfqpoint{4.011666in}{1.854374in}}%
\pgfpathcurveto{\pgfqpoint{4.000616in}{1.854374in}}{\pgfqpoint{3.990016in}{1.849983in}}{\pgfqpoint{3.982203in}{1.842170in}}%
\pgfpathcurveto{\pgfqpoint{3.974389in}{1.834356in}}{\pgfqpoint{3.969999in}{1.823757in}}{\pgfqpoint{3.969999in}{1.812707in}}%
\pgfpathcurveto{\pgfqpoint{3.969999in}{1.801657in}}{\pgfqpoint{3.974389in}{1.791058in}}{\pgfqpoint{3.982203in}{1.783244in}}%
\pgfpathcurveto{\pgfqpoint{3.990016in}{1.775431in}}{\pgfqpoint{4.000616in}{1.771040in}}{\pgfqpoint{4.011666in}{1.771040in}}%
\pgfpathclose%
\pgfusepath{stroke,fill}%
\end{pgfscope}%
\begin{pgfscope}%
\pgfpathrectangle{\pgfqpoint{0.800000in}{0.528000in}}{\pgfqpoint{4.960000in}{3.696000in}}%
\pgfusepath{clip}%
\pgfsetbuttcap%
\pgfsetroundjoin%
\definecolor{currentfill}{rgb}{0.000000,0.000000,0.000000}%
\pgfsetfillcolor{currentfill}%
\pgfsetlinewidth{1.003750pt}%
\definecolor{currentstroke}{rgb}{0.000000,0.000000,0.000000}%
\pgfsetstrokecolor{currentstroke}%
\pgfsetdash{}{0pt}%
\pgfpathmoveto{\pgfqpoint{4.011666in}{1.771040in}}%
\pgfpathcurveto{\pgfqpoint{4.022716in}{1.771040in}}{\pgfqpoint{4.033315in}{1.775431in}}{\pgfqpoint{4.041128in}{1.783244in}}%
\pgfpathcurveto{\pgfqpoint{4.048942in}{1.791058in}}{\pgfqpoint{4.053332in}{1.801657in}}{\pgfqpoint{4.053332in}{1.812707in}}%
\pgfpathcurveto{\pgfqpoint{4.053332in}{1.823757in}}{\pgfqpoint{4.048942in}{1.834356in}}{\pgfqpoint{4.041128in}{1.842170in}}%
\pgfpathcurveto{\pgfqpoint{4.033315in}{1.849983in}}{\pgfqpoint{4.022716in}{1.854374in}}{\pgfqpoint{4.011666in}{1.854374in}}%
\pgfpathcurveto{\pgfqpoint{4.000616in}{1.854374in}}{\pgfqpoint{3.990016in}{1.849983in}}{\pgfqpoint{3.982203in}{1.842170in}}%
\pgfpathcurveto{\pgfqpoint{3.974389in}{1.834356in}}{\pgfqpoint{3.969999in}{1.823757in}}{\pgfqpoint{3.969999in}{1.812707in}}%
\pgfpathcurveto{\pgfqpoint{3.969999in}{1.801657in}}{\pgfqpoint{3.974389in}{1.791058in}}{\pgfqpoint{3.982203in}{1.783244in}}%
\pgfpathcurveto{\pgfqpoint{3.990016in}{1.775431in}}{\pgfqpoint{4.000616in}{1.771040in}}{\pgfqpoint{4.011666in}{1.771040in}}%
\pgfpathclose%
\pgfusepath{stroke,fill}%
\end{pgfscope}%
\begin{pgfscope}%
\pgfpathrectangle{\pgfqpoint{0.800000in}{0.528000in}}{\pgfqpoint{4.960000in}{3.696000in}}%
\pgfusepath{clip}%
\pgfsetbuttcap%
\pgfsetroundjoin%
\definecolor{currentfill}{rgb}{0.000000,0.000000,0.000000}%
\pgfsetfillcolor{currentfill}%
\pgfsetlinewidth{1.003750pt}%
\definecolor{currentstroke}{rgb}{0.000000,0.000000,0.000000}%
\pgfsetstrokecolor{currentstroke}%
\pgfsetdash{}{0pt}%
\pgfpathmoveto{\pgfqpoint{4.011666in}{2.877687in}}%
\pgfpathcurveto{\pgfqpoint{4.022716in}{2.877687in}}{\pgfqpoint{4.033315in}{2.882077in}}{\pgfqpoint{4.041128in}{2.889891in}}%
\pgfpathcurveto{\pgfqpoint{4.048942in}{2.897704in}}{\pgfqpoint{4.053332in}{2.908303in}}{\pgfqpoint{4.053332in}{2.919353in}}%
\pgfpathcurveto{\pgfqpoint{4.053332in}{2.930404in}}{\pgfqpoint{4.048942in}{2.941003in}}{\pgfqpoint{4.041128in}{2.948816in}}%
\pgfpathcurveto{\pgfqpoint{4.033315in}{2.956630in}}{\pgfqpoint{4.022716in}{2.961020in}}{\pgfqpoint{4.011666in}{2.961020in}}%
\pgfpathcurveto{\pgfqpoint{4.000616in}{2.961020in}}{\pgfqpoint{3.990016in}{2.956630in}}{\pgfqpoint{3.982203in}{2.948816in}}%
\pgfpathcurveto{\pgfqpoint{3.974389in}{2.941003in}}{\pgfqpoint{3.969999in}{2.930404in}}{\pgfqpoint{3.969999in}{2.919353in}}%
\pgfpathcurveto{\pgfqpoint{3.969999in}{2.908303in}}{\pgfqpoint{3.974389in}{2.897704in}}{\pgfqpoint{3.982203in}{2.889891in}}%
\pgfpathcurveto{\pgfqpoint{3.990016in}{2.882077in}}{\pgfqpoint{4.000616in}{2.877687in}}{\pgfqpoint{4.011666in}{2.877687in}}%
\pgfpathclose%
\pgfusepath{stroke,fill}%
\end{pgfscope}%
\begin{pgfscope}%
\pgfpathrectangle{\pgfqpoint{0.800000in}{0.528000in}}{\pgfqpoint{4.960000in}{3.696000in}}%
\pgfusepath{clip}%
\pgfsetbuttcap%
\pgfsetroundjoin%
\definecolor{currentfill}{rgb}{0.000000,0.000000,0.000000}%
\pgfsetfillcolor{currentfill}%
\pgfsetlinewidth{1.003750pt}%
\definecolor{currentstroke}{rgb}{0.000000,0.000000,0.000000}%
\pgfsetstrokecolor{currentstroke}%
\pgfsetdash{}{0pt}%
\pgfpathmoveto{\pgfqpoint{4.011666in}{1.771040in}}%
\pgfpathcurveto{\pgfqpoint{4.022716in}{1.771040in}}{\pgfqpoint{4.033315in}{1.775431in}}{\pgfqpoint{4.041128in}{1.783244in}}%
\pgfpathcurveto{\pgfqpoint{4.048942in}{1.791058in}}{\pgfqpoint{4.053332in}{1.801657in}}{\pgfqpoint{4.053332in}{1.812707in}}%
\pgfpathcurveto{\pgfqpoint{4.053332in}{1.823757in}}{\pgfqpoint{4.048942in}{1.834356in}}{\pgfqpoint{4.041128in}{1.842170in}}%
\pgfpathcurveto{\pgfqpoint{4.033315in}{1.849983in}}{\pgfqpoint{4.022716in}{1.854374in}}{\pgfqpoint{4.011666in}{1.854374in}}%
\pgfpathcurveto{\pgfqpoint{4.000616in}{1.854374in}}{\pgfqpoint{3.990016in}{1.849983in}}{\pgfqpoint{3.982203in}{1.842170in}}%
\pgfpathcurveto{\pgfqpoint{3.974389in}{1.834356in}}{\pgfqpoint{3.969999in}{1.823757in}}{\pgfqpoint{3.969999in}{1.812707in}}%
\pgfpathcurveto{\pgfqpoint{3.969999in}{1.801657in}}{\pgfqpoint{3.974389in}{1.791058in}}{\pgfqpoint{3.982203in}{1.783244in}}%
\pgfpathcurveto{\pgfqpoint{3.990016in}{1.775431in}}{\pgfqpoint{4.000616in}{1.771040in}}{\pgfqpoint{4.011666in}{1.771040in}}%
\pgfpathclose%
\pgfusepath{stroke,fill}%
\end{pgfscope}%
\begin{pgfscope}%
\pgfpathrectangle{\pgfqpoint{0.800000in}{0.528000in}}{\pgfqpoint{4.960000in}{3.696000in}}%
\pgfusepath{clip}%
\pgfsetbuttcap%
\pgfsetroundjoin%
\definecolor{currentfill}{rgb}{0.000000,0.000000,0.000000}%
\pgfsetfillcolor{currentfill}%
\pgfsetlinewidth{1.003750pt}%
\definecolor{currentstroke}{rgb}{0.000000,0.000000,0.000000}%
\pgfsetstrokecolor{currentstroke}%
\pgfsetdash{}{0pt}%
\pgfpathmoveto{\pgfqpoint{4.011666in}{1.771040in}}%
\pgfpathcurveto{\pgfqpoint{4.022716in}{1.771040in}}{\pgfqpoint{4.033315in}{1.775431in}}{\pgfqpoint{4.041128in}{1.783244in}}%
\pgfpathcurveto{\pgfqpoint{4.048942in}{1.791058in}}{\pgfqpoint{4.053332in}{1.801657in}}{\pgfqpoint{4.053332in}{1.812707in}}%
\pgfpathcurveto{\pgfqpoint{4.053332in}{1.823757in}}{\pgfqpoint{4.048942in}{1.834356in}}{\pgfqpoint{4.041128in}{1.842170in}}%
\pgfpathcurveto{\pgfqpoint{4.033315in}{1.849983in}}{\pgfqpoint{4.022716in}{1.854374in}}{\pgfqpoint{4.011666in}{1.854374in}}%
\pgfpathcurveto{\pgfqpoint{4.000616in}{1.854374in}}{\pgfqpoint{3.990016in}{1.849983in}}{\pgfqpoint{3.982203in}{1.842170in}}%
\pgfpathcurveto{\pgfqpoint{3.974389in}{1.834356in}}{\pgfqpoint{3.969999in}{1.823757in}}{\pgfqpoint{3.969999in}{1.812707in}}%
\pgfpathcurveto{\pgfqpoint{3.969999in}{1.801657in}}{\pgfqpoint{3.974389in}{1.791058in}}{\pgfqpoint{3.982203in}{1.783244in}}%
\pgfpathcurveto{\pgfqpoint{3.990016in}{1.775431in}}{\pgfqpoint{4.000616in}{1.771040in}}{\pgfqpoint{4.011666in}{1.771040in}}%
\pgfpathclose%
\pgfusepath{stroke,fill}%
\end{pgfscope}%
\begin{pgfscope}%
\pgfpathrectangle{\pgfqpoint{0.800000in}{0.528000in}}{\pgfqpoint{4.960000in}{3.696000in}}%
\pgfusepath{clip}%
\pgfsetbuttcap%
\pgfsetroundjoin%
\definecolor{currentfill}{rgb}{0.000000,0.000000,0.000000}%
\pgfsetfillcolor{currentfill}%
\pgfsetlinewidth{1.003750pt}%
\definecolor{currentstroke}{rgb}{0.000000,0.000000,0.000000}%
\pgfsetstrokecolor{currentstroke}%
\pgfsetdash{}{0pt}%
\pgfpathmoveto{\pgfqpoint{4.011666in}{1.771040in}}%
\pgfpathcurveto{\pgfqpoint{4.022716in}{1.771040in}}{\pgfqpoint{4.033315in}{1.775431in}}{\pgfqpoint{4.041128in}{1.783244in}}%
\pgfpathcurveto{\pgfqpoint{4.048942in}{1.791058in}}{\pgfqpoint{4.053332in}{1.801657in}}{\pgfqpoint{4.053332in}{1.812707in}}%
\pgfpathcurveto{\pgfqpoint{4.053332in}{1.823757in}}{\pgfqpoint{4.048942in}{1.834356in}}{\pgfqpoint{4.041128in}{1.842170in}}%
\pgfpathcurveto{\pgfqpoint{4.033315in}{1.849983in}}{\pgfqpoint{4.022716in}{1.854374in}}{\pgfqpoint{4.011666in}{1.854374in}}%
\pgfpathcurveto{\pgfqpoint{4.000616in}{1.854374in}}{\pgfqpoint{3.990016in}{1.849983in}}{\pgfqpoint{3.982203in}{1.842170in}}%
\pgfpathcurveto{\pgfqpoint{3.974389in}{1.834356in}}{\pgfqpoint{3.969999in}{1.823757in}}{\pgfqpoint{3.969999in}{1.812707in}}%
\pgfpathcurveto{\pgfqpoint{3.969999in}{1.801657in}}{\pgfqpoint{3.974389in}{1.791058in}}{\pgfqpoint{3.982203in}{1.783244in}}%
\pgfpathcurveto{\pgfqpoint{3.990016in}{1.775431in}}{\pgfqpoint{4.000616in}{1.771040in}}{\pgfqpoint{4.011666in}{1.771040in}}%
\pgfpathclose%
\pgfusepath{stroke,fill}%
\end{pgfscope}%
\begin{pgfscope}%
\pgfpathrectangle{\pgfqpoint{0.800000in}{0.528000in}}{\pgfqpoint{4.960000in}{3.696000in}}%
\pgfusepath{clip}%
\pgfsetbuttcap%
\pgfsetroundjoin%
\definecolor{currentfill}{rgb}{0.000000,0.000000,0.000000}%
\pgfsetfillcolor{currentfill}%
\pgfsetlinewidth{1.003750pt}%
\definecolor{currentstroke}{rgb}{0.000000,0.000000,0.000000}%
\pgfsetstrokecolor{currentstroke}%
\pgfsetdash{}{0pt}%
\pgfpathmoveto{\pgfqpoint{4.011666in}{2.877687in}}%
\pgfpathcurveto{\pgfqpoint{4.022716in}{2.877687in}}{\pgfqpoint{4.033315in}{2.882077in}}{\pgfqpoint{4.041128in}{2.889891in}}%
\pgfpathcurveto{\pgfqpoint{4.048942in}{2.897704in}}{\pgfqpoint{4.053332in}{2.908303in}}{\pgfqpoint{4.053332in}{2.919353in}}%
\pgfpathcurveto{\pgfqpoint{4.053332in}{2.930404in}}{\pgfqpoint{4.048942in}{2.941003in}}{\pgfqpoint{4.041128in}{2.948816in}}%
\pgfpathcurveto{\pgfqpoint{4.033315in}{2.956630in}}{\pgfqpoint{4.022716in}{2.961020in}}{\pgfqpoint{4.011666in}{2.961020in}}%
\pgfpathcurveto{\pgfqpoint{4.000616in}{2.961020in}}{\pgfqpoint{3.990016in}{2.956630in}}{\pgfqpoint{3.982203in}{2.948816in}}%
\pgfpathcurveto{\pgfqpoint{3.974389in}{2.941003in}}{\pgfqpoint{3.969999in}{2.930404in}}{\pgfqpoint{3.969999in}{2.919353in}}%
\pgfpathcurveto{\pgfqpoint{3.969999in}{2.908303in}}{\pgfqpoint{3.974389in}{2.897704in}}{\pgfqpoint{3.982203in}{2.889891in}}%
\pgfpathcurveto{\pgfqpoint{3.990016in}{2.882077in}}{\pgfqpoint{4.000616in}{2.877687in}}{\pgfqpoint{4.011666in}{2.877687in}}%
\pgfpathclose%
\pgfusepath{stroke,fill}%
\end{pgfscope}%
\begin{pgfscope}%
\pgfpathrectangle{\pgfqpoint{0.800000in}{0.528000in}}{\pgfqpoint{4.960000in}{3.696000in}}%
\pgfusepath{clip}%
\pgfsetbuttcap%
\pgfsetroundjoin%
\definecolor{currentfill}{rgb}{0.000000,0.000000,0.000000}%
\pgfsetfillcolor{currentfill}%
\pgfsetlinewidth{1.003750pt}%
\definecolor{currentstroke}{rgb}{0.000000,0.000000,0.000000}%
\pgfsetstrokecolor{currentstroke}%
\pgfsetdash{}{0pt}%
\pgfpathmoveto{\pgfqpoint{4.011666in}{1.771040in}}%
\pgfpathcurveto{\pgfqpoint{4.022716in}{1.771040in}}{\pgfqpoint{4.033315in}{1.775431in}}{\pgfqpoint{4.041128in}{1.783244in}}%
\pgfpathcurveto{\pgfqpoint{4.048942in}{1.791058in}}{\pgfqpoint{4.053332in}{1.801657in}}{\pgfqpoint{4.053332in}{1.812707in}}%
\pgfpathcurveto{\pgfqpoint{4.053332in}{1.823757in}}{\pgfqpoint{4.048942in}{1.834356in}}{\pgfqpoint{4.041128in}{1.842170in}}%
\pgfpathcurveto{\pgfqpoint{4.033315in}{1.849983in}}{\pgfqpoint{4.022716in}{1.854374in}}{\pgfqpoint{4.011666in}{1.854374in}}%
\pgfpathcurveto{\pgfqpoint{4.000616in}{1.854374in}}{\pgfqpoint{3.990016in}{1.849983in}}{\pgfqpoint{3.982203in}{1.842170in}}%
\pgfpathcurveto{\pgfqpoint{3.974389in}{1.834356in}}{\pgfqpoint{3.969999in}{1.823757in}}{\pgfqpoint{3.969999in}{1.812707in}}%
\pgfpathcurveto{\pgfqpoint{3.969999in}{1.801657in}}{\pgfqpoint{3.974389in}{1.791058in}}{\pgfqpoint{3.982203in}{1.783244in}}%
\pgfpathcurveto{\pgfqpoint{3.990016in}{1.775431in}}{\pgfqpoint{4.000616in}{1.771040in}}{\pgfqpoint{4.011666in}{1.771040in}}%
\pgfpathclose%
\pgfusepath{stroke,fill}%
\end{pgfscope}%
\begin{pgfscope}%
\pgfpathrectangle{\pgfqpoint{0.800000in}{0.528000in}}{\pgfqpoint{4.960000in}{3.696000in}}%
\pgfusepath{clip}%
\pgfsetbuttcap%
\pgfsetroundjoin%
\definecolor{currentfill}{rgb}{0.000000,0.000000,0.000000}%
\pgfsetfillcolor{currentfill}%
\pgfsetlinewidth{1.003750pt}%
\definecolor{currentstroke}{rgb}{0.000000,0.000000,0.000000}%
\pgfsetstrokecolor{currentstroke}%
\pgfsetdash{}{0pt}%
\pgfpathmoveto{\pgfqpoint{4.011666in}{1.771040in}}%
\pgfpathcurveto{\pgfqpoint{4.022716in}{1.771040in}}{\pgfqpoint{4.033315in}{1.775431in}}{\pgfqpoint{4.041128in}{1.783244in}}%
\pgfpathcurveto{\pgfqpoint{4.048942in}{1.791058in}}{\pgfqpoint{4.053332in}{1.801657in}}{\pgfqpoint{4.053332in}{1.812707in}}%
\pgfpathcurveto{\pgfqpoint{4.053332in}{1.823757in}}{\pgfqpoint{4.048942in}{1.834356in}}{\pgfqpoint{4.041128in}{1.842170in}}%
\pgfpathcurveto{\pgfqpoint{4.033315in}{1.849983in}}{\pgfqpoint{4.022716in}{1.854374in}}{\pgfqpoint{4.011666in}{1.854374in}}%
\pgfpathcurveto{\pgfqpoint{4.000616in}{1.854374in}}{\pgfqpoint{3.990016in}{1.849983in}}{\pgfqpoint{3.982203in}{1.842170in}}%
\pgfpathcurveto{\pgfqpoint{3.974389in}{1.834356in}}{\pgfqpoint{3.969999in}{1.823757in}}{\pgfqpoint{3.969999in}{1.812707in}}%
\pgfpathcurveto{\pgfqpoint{3.969999in}{1.801657in}}{\pgfqpoint{3.974389in}{1.791058in}}{\pgfqpoint{3.982203in}{1.783244in}}%
\pgfpathcurveto{\pgfqpoint{3.990016in}{1.775431in}}{\pgfqpoint{4.000616in}{1.771040in}}{\pgfqpoint{4.011666in}{1.771040in}}%
\pgfpathclose%
\pgfusepath{stroke,fill}%
\end{pgfscope}%
\begin{pgfscope}%
\pgfpathrectangle{\pgfqpoint{0.800000in}{0.528000in}}{\pgfqpoint{4.960000in}{3.696000in}}%
\pgfusepath{clip}%
\pgfsetbuttcap%
\pgfsetroundjoin%
\definecolor{currentfill}{rgb}{0.000000,0.000000,0.000000}%
\pgfsetfillcolor{currentfill}%
\pgfsetlinewidth{1.003750pt}%
\definecolor{currentstroke}{rgb}{0.000000,0.000000,0.000000}%
\pgfsetstrokecolor{currentstroke}%
\pgfsetdash{}{0pt}%
\pgfpathmoveto{\pgfqpoint{4.011666in}{1.771040in}}%
\pgfpathcurveto{\pgfqpoint{4.022716in}{1.771040in}}{\pgfqpoint{4.033315in}{1.775431in}}{\pgfqpoint{4.041128in}{1.783244in}}%
\pgfpathcurveto{\pgfqpoint{4.048942in}{1.791058in}}{\pgfqpoint{4.053332in}{1.801657in}}{\pgfqpoint{4.053332in}{1.812707in}}%
\pgfpathcurveto{\pgfqpoint{4.053332in}{1.823757in}}{\pgfqpoint{4.048942in}{1.834356in}}{\pgfqpoint{4.041128in}{1.842170in}}%
\pgfpathcurveto{\pgfqpoint{4.033315in}{1.849983in}}{\pgfqpoint{4.022716in}{1.854374in}}{\pgfqpoint{4.011666in}{1.854374in}}%
\pgfpathcurveto{\pgfqpoint{4.000616in}{1.854374in}}{\pgfqpoint{3.990016in}{1.849983in}}{\pgfqpoint{3.982203in}{1.842170in}}%
\pgfpathcurveto{\pgfqpoint{3.974389in}{1.834356in}}{\pgfqpoint{3.969999in}{1.823757in}}{\pgfqpoint{3.969999in}{1.812707in}}%
\pgfpathcurveto{\pgfqpoint{3.969999in}{1.801657in}}{\pgfqpoint{3.974389in}{1.791058in}}{\pgfqpoint{3.982203in}{1.783244in}}%
\pgfpathcurveto{\pgfqpoint{3.990016in}{1.775431in}}{\pgfqpoint{4.000616in}{1.771040in}}{\pgfqpoint{4.011666in}{1.771040in}}%
\pgfpathclose%
\pgfusepath{stroke,fill}%
\end{pgfscope}%
\begin{pgfscope}%
\pgfpathrectangle{\pgfqpoint{0.800000in}{0.528000in}}{\pgfqpoint{4.960000in}{3.696000in}}%
\pgfusepath{clip}%
\pgfsetbuttcap%
\pgfsetroundjoin%
\definecolor{currentfill}{rgb}{0.000000,0.000000,0.000000}%
\pgfsetfillcolor{currentfill}%
\pgfsetlinewidth{1.003750pt}%
\definecolor{currentstroke}{rgb}{0.000000,0.000000,0.000000}%
\pgfsetstrokecolor{currentstroke}%
\pgfsetdash{}{0pt}%
\pgfpathmoveto{\pgfqpoint{4.011666in}{2.877687in}}%
\pgfpathcurveto{\pgfqpoint{4.022716in}{2.877687in}}{\pgfqpoint{4.033315in}{2.882077in}}{\pgfqpoint{4.041128in}{2.889891in}}%
\pgfpathcurveto{\pgfqpoint{4.048942in}{2.897704in}}{\pgfqpoint{4.053332in}{2.908303in}}{\pgfqpoint{4.053332in}{2.919353in}}%
\pgfpathcurveto{\pgfqpoint{4.053332in}{2.930404in}}{\pgfqpoint{4.048942in}{2.941003in}}{\pgfqpoint{4.041128in}{2.948816in}}%
\pgfpathcurveto{\pgfqpoint{4.033315in}{2.956630in}}{\pgfqpoint{4.022716in}{2.961020in}}{\pgfqpoint{4.011666in}{2.961020in}}%
\pgfpathcurveto{\pgfqpoint{4.000616in}{2.961020in}}{\pgfqpoint{3.990016in}{2.956630in}}{\pgfqpoint{3.982203in}{2.948816in}}%
\pgfpathcurveto{\pgfqpoint{3.974389in}{2.941003in}}{\pgfqpoint{3.969999in}{2.930404in}}{\pgfqpoint{3.969999in}{2.919353in}}%
\pgfpathcurveto{\pgfqpoint{3.969999in}{2.908303in}}{\pgfqpoint{3.974389in}{2.897704in}}{\pgfqpoint{3.982203in}{2.889891in}}%
\pgfpathcurveto{\pgfqpoint{3.990016in}{2.882077in}}{\pgfqpoint{4.000616in}{2.877687in}}{\pgfqpoint{4.011666in}{2.877687in}}%
\pgfpathclose%
\pgfusepath{stroke,fill}%
\end{pgfscope}%
\begin{pgfscope}%
\pgfpathrectangle{\pgfqpoint{0.800000in}{0.528000in}}{\pgfqpoint{4.960000in}{3.696000in}}%
\pgfusepath{clip}%
\pgfsetbuttcap%
\pgfsetroundjoin%
\definecolor{currentfill}{rgb}{0.000000,0.000000,0.000000}%
\pgfsetfillcolor{currentfill}%
\pgfsetlinewidth{1.003750pt}%
\definecolor{currentstroke}{rgb}{0.000000,0.000000,0.000000}%
\pgfsetstrokecolor{currentstroke}%
\pgfsetdash{}{0pt}%
\pgfpathmoveto{\pgfqpoint{4.011666in}{1.771040in}}%
\pgfpathcurveto{\pgfqpoint{4.022716in}{1.771040in}}{\pgfqpoint{4.033315in}{1.775431in}}{\pgfqpoint{4.041128in}{1.783244in}}%
\pgfpathcurveto{\pgfqpoint{4.048942in}{1.791058in}}{\pgfqpoint{4.053332in}{1.801657in}}{\pgfqpoint{4.053332in}{1.812707in}}%
\pgfpathcurveto{\pgfqpoint{4.053332in}{1.823757in}}{\pgfqpoint{4.048942in}{1.834356in}}{\pgfqpoint{4.041128in}{1.842170in}}%
\pgfpathcurveto{\pgfqpoint{4.033315in}{1.849983in}}{\pgfqpoint{4.022716in}{1.854374in}}{\pgfqpoint{4.011666in}{1.854374in}}%
\pgfpathcurveto{\pgfqpoint{4.000616in}{1.854374in}}{\pgfqpoint{3.990016in}{1.849983in}}{\pgfqpoint{3.982203in}{1.842170in}}%
\pgfpathcurveto{\pgfqpoint{3.974389in}{1.834356in}}{\pgfqpoint{3.969999in}{1.823757in}}{\pgfqpoint{3.969999in}{1.812707in}}%
\pgfpathcurveto{\pgfqpoint{3.969999in}{1.801657in}}{\pgfqpoint{3.974389in}{1.791058in}}{\pgfqpoint{3.982203in}{1.783244in}}%
\pgfpathcurveto{\pgfqpoint{3.990016in}{1.775431in}}{\pgfqpoint{4.000616in}{1.771040in}}{\pgfqpoint{4.011666in}{1.771040in}}%
\pgfpathclose%
\pgfusepath{stroke,fill}%
\end{pgfscope}%
\begin{pgfscope}%
\pgfpathrectangle{\pgfqpoint{0.800000in}{0.528000in}}{\pgfqpoint{4.960000in}{3.696000in}}%
\pgfusepath{clip}%
\pgfsetbuttcap%
\pgfsetroundjoin%
\definecolor{currentfill}{rgb}{0.000000,0.000000,0.000000}%
\pgfsetfillcolor{currentfill}%
\pgfsetlinewidth{1.003750pt}%
\definecolor{currentstroke}{rgb}{0.000000,0.000000,0.000000}%
\pgfsetstrokecolor{currentstroke}%
\pgfsetdash{}{0pt}%
\pgfpathmoveto{\pgfqpoint{4.011666in}{2.877687in}}%
\pgfpathcurveto{\pgfqpoint{4.022716in}{2.877687in}}{\pgfqpoint{4.033315in}{2.882077in}}{\pgfqpoint{4.041128in}{2.889891in}}%
\pgfpathcurveto{\pgfqpoint{4.048942in}{2.897704in}}{\pgfqpoint{4.053332in}{2.908303in}}{\pgfqpoint{4.053332in}{2.919353in}}%
\pgfpathcurveto{\pgfqpoint{4.053332in}{2.930404in}}{\pgfqpoint{4.048942in}{2.941003in}}{\pgfqpoint{4.041128in}{2.948816in}}%
\pgfpathcurveto{\pgfqpoint{4.033315in}{2.956630in}}{\pgfqpoint{4.022716in}{2.961020in}}{\pgfqpoint{4.011666in}{2.961020in}}%
\pgfpathcurveto{\pgfqpoint{4.000616in}{2.961020in}}{\pgfqpoint{3.990016in}{2.956630in}}{\pgfqpoint{3.982203in}{2.948816in}}%
\pgfpathcurveto{\pgfqpoint{3.974389in}{2.941003in}}{\pgfqpoint{3.969999in}{2.930404in}}{\pgfqpoint{3.969999in}{2.919353in}}%
\pgfpathcurveto{\pgfqpoint{3.969999in}{2.908303in}}{\pgfqpoint{3.974389in}{2.897704in}}{\pgfqpoint{3.982203in}{2.889891in}}%
\pgfpathcurveto{\pgfqpoint{3.990016in}{2.882077in}}{\pgfqpoint{4.000616in}{2.877687in}}{\pgfqpoint{4.011666in}{2.877687in}}%
\pgfpathclose%
\pgfusepath{stroke,fill}%
\end{pgfscope}%
\begin{pgfscope}%
\pgfpathrectangle{\pgfqpoint{0.800000in}{0.528000in}}{\pgfqpoint{4.960000in}{3.696000in}}%
\pgfusepath{clip}%
\pgfsetbuttcap%
\pgfsetroundjoin%
\definecolor{currentfill}{rgb}{0.000000,0.000000,0.000000}%
\pgfsetfillcolor{currentfill}%
\pgfsetlinewidth{1.003750pt}%
\definecolor{currentstroke}{rgb}{0.000000,0.000000,0.000000}%
\pgfsetstrokecolor{currentstroke}%
\pgfsetdash{}{0pt}%
\pgfpathmoveto{\pgfqpoint{4.011666in}{1.771040in}}%
\pgfpathcurveto{\pgfqpoint{4.022716in}{1.771040in}}{\pgfqpoint{4.033315in}{1.775431in}}{\pgfqpoint{4.041128in}{1.783244in}}%
\pgfpathcurveto{\pgfqpoint{4.048942in}{1.791058in}}{\pgfqpoint{4.053332in}{1.801657in}}{\pgfqpoint{4.053332in}{1.812707in}}%
\pgfpathcurveto{\pgfqpoint{4.053332in}{1.823757in}}{\pgfqpoint{4.048942in}{1.834356in}}{\pgfqpoint{4.041128in}{1.842170in}}%
\pgfpathcurveto{\pgfqpoint{4.033315in}{1.849983in}}{\pgfqpoint{4.022716in}{1.854374in}}{\pgfqpoint{4.011666in}{1.854374in}}%
\pgfpathcurveto{\pgfqpoint{4.000616in}{1.854374in}}{\pgfqpoint{3.990016in}{1.849983in}}{\pgfqpoint{3.982203in}{1.842170in}}%
\pgfpathcurveto{\pgfqpoint{3.974389in}{1.834356in}}{\pgfqpoint{3.969999in}{1.823757in}}{\pgfqpoint{3.969999in}{1.812707in}}%
\pgfpathcurveto{\pgfqpoint{3.969999in}{1.801657in}}{\pgfqpoint{3.974389in}{1.791058in}}{\pgfqpoint{3.982203in}{1.783244in}}%
\pgfpathcurveto{\pgfqpoint{3.990016in}{1.775431in}}{\pgfqpoint{4.000616in}{1.771040in}}{\pgfqpoint{4.011666in}{1.771040in}}%
\pgfpathclose%
\pgfusepath{stroke,fill}%
\end{pgfscope}%
\begin{pgfscope}%
\pgfpathrectangle{\pgfqpoint{0.800000in}{0.528000in}}{\pgfqpoint{4.960000in}{3.696000in}}%
\pgfusepath{clip}%
\pgfsetbuttcap%
\pgfsetroundjoin%
\definecolor{currentfill}{rgb}{0.000000,0.000000,0.000000}%
\pgfsetfillcolor{currentfill}%
\pgfsetlinewidth{1.003750pt}%
\definecolor{currentstroke}{rgb}{0.000000,0.000000,0.000000}%
\pgfsetstrokecolor{currentstroke}%
\pgfsetdash{}{0pt}%
\pgfpathmoveto{\pgfqpoint{4.011666in}{1.771040in}}%
\pgfpathcurveto{\pgfqpoint{4.022716in}{1.771040in}}{\pgfqpoint{4.033315in}{1.775431in}}{\pgfqpoint{4.041128in}{1.783244in}}%
\pgfpathcurveto{\pgfqpoint{4.048942in}{1.791058in}}{\pgfqpoint{4.053332in}{1.801657in}}{\pgfqpoint{4.053332in}{1.812707in}}%
\pgfpathcurveto{\pgfqpoint{4.053332in}{1.823757in}}{\pgfqpoint{4.048942in}{1.834356in}}{\pgfqpoint{4.041128in}{1.842170in}}%
\pgfpathcurveto{\pgfqpoint{4.033315in}{1.849983in}}{\pgfqpoint{4.022716in}{1.854374in}}{\pgfqpoint{4.011666in}{1.854374in}}%
\pgfpathcurveto{\pgfqpoint{4.000616in}{1.854374in}}{\pgfqpoint{3.990016in}{1.849983in}}{\pgfqpoint{3.982203in}{1.842170in}}%
\pgfpathcurveto{\pgfqpoint{3.974389in}{1.834356in}}{\pgfqpoint{3.969999in}{1.823757in}}{\pgfqpoint{3.969999in}{1.812707in}}%
\pgfpathcurveto{\pgfqpoint{3.969999in}{1.801657in}}{\pgfqpoint{3.974389in}{1.791058in}}{\pgfqpoint{3.982203in}{1.783244in}}%
\pgfpathcurveto{\pgfqpoint{3.990016in}{1.775431in}}{\pgfqpoint{4.000616in}{1.771040in}}{\pgfqpoint{4.011666in}{1.771040in}}%
\pgfpathclose%
\pgfusepath{stroke,fill}%
\end{pgfscope}%
\begin{pgfscope}%
\pgfpathrectangle{\pgfqpoint{0.800000in}{0.528000in}}{\pgfqpoint{4.960000in}{3.696000in}}%
\pgfusepath{clip}%
\pgfsetbuttcap%
\pgfsetroundjoin%
\definecolor{currentfill}{rgb}{0.000000,0.000000,0.000000}%
\pgfsetfillcolor{currentfill}%
\pgfsetlinewidth{1.003750pt}%
\definecolor{currentstroke}{rgb}{0.000000,0.000000,0.000000}%
\pgfsetstrokecolor{currentstroke}%
\pgfsetdash{}{0pt}%
\pgfpathmoveto{\pgfqpoint{4.011666in}{1.771040in}}%
\pgfpathcurveto{\pgfqpoint{4.022716in}{1.771040in}}{\pgfqpoint{4.033315in}{1.775431in}}{\pgfqpoint{4.041128in}{1.783244in}}%
\pgfpathcurveto{\pgfqpoint{4.048942in}{1.791058in}}{\pgfqpoint{4.053332in}{1.801657in}}{\pgfqpoint{4.053332in}{1.812707in}}%
\pgfpathcurveto{\pgfqpoint{4.053332in}{1.823757in}}{\pgfqpoint{4.048942in}{1.834356in}}{\pgfqpoint{4.041128in}{1.842170in}}%
\pgfpathcurveto{\pgfqpoint{4.033315in}{1.849983in}}{\pgfqpoint{4.022716in}{1.854374in}}{\pgfqpoint{4.011666in}{1.854374in}}%
\pgfpathcurveto{\pgfqpoint{4.000616in}{1.854374in}}{\pgfqpoint{3.990016in}{1.849983in}}{\pgfqpoint{3.982203in}{1.842170in}}%
\pgfpathcurveto{\pgfqpoint{3.974389in}{1.834356in}}{\pgfqpoint{3.969999in}{1.823757in}}{\pgfqpoint{3.969999in}{1.812707in}}%
\pgfpathcurveto{\pgfqpoint{3.969999in}{1.801657in}}{\pgfqpoint{3.974389in}{1.791058in}}{\pgfqpoint{3.982203in}{1.783244in}}%
\pgfpathcurveto{\pgfqpoint{3.990016in}{1.775431in}}{\pgfqpoint{4.000616in}{1.771040in}}{\pgfqpoint{4.011666in}{1.771040in}}%
\pgfpathclose%
\pgfusepath{stroke,fill}%
\end{pgfscope}%
\begin{pgfscope}%
\pgfpathrectangle{\pgfqpoint{0.800000in}{0.528000in}}{\pgfqpoint{4.960000in}{3.696000in}}%
\pgfusepath{clip}%
\pgfsetbuttcap%
\pgfsetroundjoin%
\definecolor{currentfill}{rgb}{0.000000,0.000000,0.000000}%
\pgfsetfillcolor{currentfill}%
\pgfsetlinewidth{1.003750pt}%
\definecolor{currentstroke}{rgb}{0.000000,0.000000,0.000000}%
\pgfsetstrokecolor{currentstroke}%
\pgfsetdash{}{0pt}%
\pgfpathmoveto{\pgfqpoint{4.011666in}{2.877687in}}%
\pgfpathcurveto{\pgfqpoint{4.022716in}{2.877687in}}{\pgfqpoint{4.033315in}{2.882077in}}{\pgfqpoint{4.041128in}{2.889891in}}%
\pgfpathcurveto{\pgfqpoint{4.048942in}{2.897704in}}{\pgfqpoint{4.053332in}{2.908303in}}{\pgfqpoint{4.053332in}{2.919353in}}%
\pgfpathcurveto{\pgfqpoint{4.053332in}{2.930404in}}{\pgfqpoint{4.048942in}{2.941003in}}{\pgfqpoint{4.041128in}{2.948816in}}%
\pgfpathcurveto{\pgfqpoint{4.033315in}{2.956630in}}{\pgfqpoint{4.022716in}{2.961020in}}{\pgfqpoint{4.011666in}{2.961020in}}%
\pgfpathcurveto{\pgfqpoint{4.000616in}{2.961020in}}{\pgfqpoint{3.990016in}{2.956630in}}{\pgfqpoint{3.982203in}{2.948816in}}%
\pgfpathcurveto{\pgfqpoint{3.974389in}{2.941003in}}{\pgfqpoint{3.969999in}{2.930404in}}{\pgfqpoint{3.969999in}{2.919353in}}%
\pgfpathcurveto{\pgfqpoint{3.969999in}{2.908303in}}{\pgfqpoint{3.974389in}{2.897704in}}{\pgfqpoint{3.982203in}{2.889891in}}%
\pgfpathcurveto{\pgfqpoint{3.990016in}{2.882077in}}{\pgfqpoint{4.000616in}{2.877687in}}{\pgfqpoint{4.011666in}{2.877687in}}%
\pgfpathclose%
\pgfusepath{stroke,fill}%
\end{pgfscope}%
\begin{pgfscope}%
\pgfpathrectangle{\pgfqpoint{0.800000in}{0.528000in}}{\pgfqpoint{4.960000in}{3.696000in}}%
\pgfusepath{clip}%
\pgfsetbuttcap%
\pgfsetroundjoin%
\definecolor{currentfill}{rgb}{0.000000,0.000000,0.000000}%
\pgfsetfillcolor{currentfill}%
\pgfsetlinewidth{1.003750pt}%
\definecolor{currentstroke}{rgb}{0.000000,0.000000,0.000000}%
\pgfsetstrokecolor{currentstroke}%
\pgfsetdash{}{0pt}%
\pgfpathmoveto{\pgfqpoint{4.011666in}{1.771040in}}%
\pgfpathcurveto{\pgfqpoint{4.022716in}{1.771040in}}{\pgfqpoint{4.033315in}{1.775431in}}{\pgfqpoint{4.041128in}{1.783244in}}%
\pgfpathcurveto{\pgfqpoint{4.048942in}{1.791058in}}{\pgfqpoint{4.053332in}{1.801657in}}{\pgfqpoint{4.053332in}{1.812707in}}%
\pgfpathcurveto{\pgfqpoint{4.053332in}{1.823757in}}{\pgfqpoint{4.048942in}{1.834356in}}{\pgfqpoint{4.041128in}{1.842170in}}%
\pgfpathcurveto{\pgfqpoint{4.033315in}{1.849983in}}{\pgfqpoint{4.022716in}{1.854374in}}{\pgfqpoint{4.011666in}{1.854374in}}%
\pgfpathcurveto{\pgfqpoint{4.000616in}{1.854374in}}{\pgfqpoint{3.990016in}{1.849983in}}{\pgfqpoint{3.982203in}{1.842170in}}%
\pgfpathcurveto{\pgfqpoint{3.974389in}{1.834356in}}{\pgfqpoint{3.969999in}{1.823757in}}{\pgfqpoint{3.969999in}{1.812707in}}%
\pgfpathcurveto{\pgfqpoint{3.969999in}{1.801657in}}{\pgfqpoint{3.974389in}{1.791058in}}{\pgfqpoint{3.982203in}{1.783244in}}%
\pgfpathcurveto{\pgfqpoint{3.990016in}{1.775431in}}{\pgfqpoint{4.000616in}{1.771040in}}{\pgfqpoint{4.011666in}{1.771040in}}%
\pgfpathclose%
\pgfusepath{stroke,fill}%
\end{pgfscope}%
\begin{pgfscope}%
\pgfpathrectangle{\pgfqpoint{0.800000in}{0.528000in}}{\pgfqpoint{4.960000in}{3.696000in}}%
\pgfusepath{clip}%
\pgfsetbuttcap%
\pgfsetroundjoin%
\definecolor{currentfill}{rgb}{0.000000,0.000000,0.000000}%
\pgfsetfillcolor{currentfill}%
\pgfsetlinewidth{1.003750pt}%
\definecolor{currentstroke}{rgb}{0.000000,0.000000,0.000000}%
\pgfsetstrokecolor{currentstroke}%
\pgfsetdash{}{0pt}%
\pgfpathmoveto{\pgfqpoint{4.011666in}{1.771040in}}%
\pgfpathcurveto{\pgfqpoint{4.022716in}{1.771040in}}{\pgfqpoint{4.033315in}{1.775431in}}{\pgfqpoint{4.041128in}{1.783244in}}%
\pgfpathcurveto{\pgfqpoint{4.048942in}{1.791058in}}{\pgfqpoint{4.053332in}{1.801657in}}{\pgfqpoint{4.053332in}{1.812707in}}%
\pgfpathcurveto{\pgfqpoint{4.053332in}{1.823757in}}{\pgfqpoint{4.048942in}{1.834356in}}{\pgfqpoint{4.041128in}{1.842170in}}%
\pgfpathcurveto{\pgfqpoint{4.033315in}{1.849983in}}{\pgfqpoint{4.022716in}{1.854374in}}{\pgfqpoint{4.011666in}{1.854374in}}%
\pgfpathcurveto{\pgfqpoint{4.000616in}{1.854374in}}{\pgfqpoint{3.990016in}{1.849983in}}{\pgfqpoint{3.982203in}{1.842170in}}%
\pgfpathcurveto{\pgfqpoint{3.974389in}{1.834356in}}{\pgfqpoint{3.969999in}{1.823757in}}{\pgfqpoint{3.969999in}{1.812707in}}%
\pgfpathcurveto{\pgfqpoint{3.969999in}{1.801657in}}{\pgfqpoint{3.974389in}{1.791058in}}{\pgfqpoint{3.982203in}{1.783244in}}%
\pgfpathcurveto{\pgfqpoint{3.990016in}{1.775431in}}{\pgfqpoint{4.000616in}{1.771040in}}{\pgfqpoint{4.011666in}{1.771040in}}%
\pgfpathclose%
\pgfusepath{stroke,fill}%
\end{pgfscope}%
\begin{pgfscope}%
\pgfpathrectangle{\pgfqpoint{0.800000in}{0.528000in}}{\pgfqpoint{4.960000in}{3.696000in}}%
\pgfusepath{clip}%
\pgfsetbuttcap%
\pgfsetroundjoin%
\definecolor{currentfill}{rgb}{0.000000,0.000000,0.000000}%
\pgfsetfillcolor{currentfill}%
\pgfsetlinewidth{1.003750pt}%
\definecolor{currentstroke}{rgb}{0.000000,0.000000,0.000000}%
\pgfsetstrokecolor{currentstroke}%
\pgfsetdash{}{0pt}%
\pgfpathmoveto{\pgfqpoint{4.011666in}{1.771040in}}%
\pgfpathcurveto{\pgfqpoint{4.022716in}{1.771040in}}{\pgfqpoint{4.033315in}{1.775431in}}{\pgfqpoint{4.041128in}{1.783244in}}%
\pgfpathcurveto{\pgfqpoint{4.048942in}{1.791058in}}{\pgfqpoint{4.053332in}{1.801657in}}{\pgfqpoint{4.053332in}{1.812707in}}%
\pgfpathcurveto{\pgfqpoint{4.053332in}{1.823757in}}{\pgfqpoint{4.048942in}{1.834356in}}{\pgfqpoint{4.041128in}{1.842170in}}%
\pgfpathcurveto{\pgfqpoint{4.033315in}{1.849983in}}{\pgfqpoint{4.022716in}{1.854374in}}{\pgfqpoint{4.011666in}{1.854374in}}%
\pgfpathcurveto{\pgfqpoint{4.000616in}{1.854374in}}{\pgfqpoint{3.990016in}{1.849983in}}{\pgfqpoint{3.982203in}{1.842170in}}%
\pgfpathcurveto{\pgfqpoint{3.974389in}{1.834356in}}{\pgfqpoint{3.969999in}{1.823757in}}{\pgfqpoint{3.969999in}{1.812707in}}%
\pgfpathcurveto{\pgfqpoint{3.969999in}{1.801657in}}{\pgfqpoint{3.974389in}{1.791058in}}{\pgfqpoint{3.982203in}{1.783244in}}%
\pgfpathcurveto{\pgfqpoint{3.990016in}{1.775431in}}{\pgfqpoint{4.000616in}{1.771040in}}{\pgfqpoint{4.011666in}{1.771040in}}%
\pgfpathclose%
\pgfusepath{stroke,fill}%
\end{pgfscope}%
\begin{pgfscope}%
\pgfpathrectangle{\pgfqpoint{0.800000in}{0.528000in}}{\pgfqpoint{4.960000in}{3.696000in}}%
\pgfusepath{clip}%
\pgfsetbuttcap%
\pgfsetroundjoin%
\definecolor{currentfill}{rgb}{0.000000,0.000000,0.000000}%
\pgfsetfillcolor{currentfill}%
\pgfsetlinewidth{1.003750pt}%
\definecolor{currentstroke}{rgb}{0.000000,0.000000,0.000000}%
\pgfsetstrokecolor{currentstroke}%
\pgfsetdash{}{0pt}%
\pgfpathmoveto{\pgfqpoint{4.011666in}{1.771040in}}%
\pgfpathcurveto{\pgfqpoint{4.022716in}{1.771040in}}{\pgfqpoint{4.033315in}{1.775431in}}{\pgfqpoint{4.041128in}{1.783244in}}%
\pgfpathcurveto{\pgfqpoint{4.048942in}{1.791058in}}{\pgfqpoint{4.053332in}{1.801657in}}{\pgfqpoint{4.053332in}{1.812707in}}%
\pgfpathcurveto{\pgfqpoint{4.053332in}{1.823757in}}{\pgfqpoint{4.048942in}{1.834356in}}{\pgfqpoint{4.041128in}{1.842170in}}%
\pgfpathcurveto{\pgfqpoint{4.033315in}{1.849983in}}{\pgfqpoint{4.022716in}{1.854374in}}{\pgfqpoint{4.011666in}{1.854374in}}%
\pgfpathcurveto{\pgfqpoint{4.000616in}{1.854374in}}{\pgfqpoint{3.990016in}{1.849983in}}{\pgfqpoint{3.982203in}{1.842170in}}%
\pgfpathcurveto{\pgfqpoint{3.974389in}{1.834356in}}{\pgfqpoint{3.969999in}{1.823757in}}{\pgfqpoint{3.969999in}{1.812707in}}%
\pgfpathcurveto{\pgfqpoint{3.969999in}{1.801657in}}{\pgfqpoint{3.974389in}{1.791058in}}{\pgfqpoint{3.982203in}{1.783244in}}%
\pgfpathcurveto{\pgfqpoint{3.990016in}{1.775431in}}{\pgfqpoint{4.000616in}{1.771040in}}{\pgfqpoint{4.011666in}{1.771040in}}%
\pgfpathclose%
\pgfusepath{stroke,fill}%
\end{pgfscope}%
\begin{pgfscope}%
\pgfpathrectangle{\pgfqpoint{0.800000in}{0.528000in}}{\pgfqpoint{4.960000in}{3.696000in}}%
\pgfusepath{clip}%
\pgfsetbuttcap%
\pgfsetroundjoin%
\definecolor{currentfill}{rgb}{0.000000,0.000000,0.000000}%
\pgfsetfillcolor{currentfill}%
\pgfsetlinewidth{1.003750pt}%
\definecolor{currentstroke}{rgb}{0.000000,0.000000,0.000000}%
\pgfsetstrokecolor{currentstroke}%
\pgfsetdash{}{0pt}%
\pgfpathmoveto{\pgfqpoint{4.011666in}{2.877687in}}%
\pgfpathcurveto{\pgfqpoint{4.022716in}{2.877687in}}{\pgfqpoint{4.033315in}{2.882077in}}{\pgfqpoint{4.041128in}{2.889891in}}%
\pgfpathcurveto{\pgfqpoint{4.048942in}{2.897704in}}{\pgfqpoint{4.053332in}{2.908303in}}{\pgfqpoint{4.053332in}{2.919353in}}%
\pgfpathcurveto{\pgfqpoint{4.053332in}{2.930404in}}{\pgfqpoint{4.048942in}{2.941003in}}{\pgfqpoint{4.041128in}{2.948816in}}%
\pgfpathcurveto{\pgfqpoint{4.033315in}{2.956630in}}{\pgfqpoint{4.022716in}{2.961020in}}{\pgfqpoint{4.011666in}{2.961020in}}%
\pgfpathcurveto{\pgfqpoint{4.000616in}{2.961020in}}{\pgfqpoint{3.990016in}{2.956630in}}{\pgfqpoint{3.982203in}{2.948816in}}%
\pgfpathcurveto{\pgfqpoint{3.974389in}{2.941003in}}{\pgfqpoint{3.969999in}{2.930404in}}{\pgfqpoint{3.969999in}{2.919353in}}%
\pgfpathcurveto{\pgfqpoint{3.969999in}{2.908303in}}{\pgfqpoint{3.974389in}{2.897704in}}{\pgfqpoint{3.982203in}{2.889891in}}%
\pgfpathcurveto{\pgfqpoint{3.990016in}{2.882077in}}{\pgfqpoint{4.000616in}{2.877687in}}{\pgfqpoint{4.011666in}{2.877687in}}%
\pgfpathclose%
\pgfusepath{stroke,fill}%
\end{pgfscope}%
\begin{pgfscope}%
\pgfpathrectangle{\pgfqpoint{0.800000in}{0.528000in}}{\pgfqpoint{4.960000in}{3.696000in}}%
\pgfusepath{clip}%
\pgfsetbuttcap%
\pgfsetroundjoin%
\definecolor{currentfill}{rgb}{0.000000,0.000000,0.000000}%
\pgfsetfillcolor{currentfill}%
\pgfsetlinewidth{1.003750pt}%
\definecolor{currentstroke}{rgb}{0.000000,0.000000,0.000000}%
\pgfsetstrokecolor{currentstroke}%
\pgfsetdash{}{0pt}%
\pgfpathmoveto{\pgfqpoint{4.011666in}{2.877687in}}%
\pgfpathcurveto{\pgfqpoint{4.022716in}{2.877687in}}{\pgfqpoint{4.033315in}{2.882077in}}{\pgfqpoint{4.041128in}{2.889891in}}%
\pgfpathcurveto{\pgfqpoint{4.048942in}{2.897704in}}{\pgfqpoint{4.053332in}{2.908303in}}{\pgfqpoint{4.053332in}{2.919353in}}%
\pgfpathcurveto{\pgfqpoint{4.053332in}{2.930404in}}{\pgfqpoint{4.048942in}{2.941003in}}{\pgfqpoint{4.041128in}{2.948816in}}%
\pgfpathcurveto{\pgfqpoint{4.033315in}{2.956630in}}{\pgfqpoint{4.022716in}{2.961020in}}{\pgfqpoint{4.011666in}{2.961020in}}%
\pgfpathcurveto{\pgfqpoint{4.000616in}{2.961020in}}{\pgfqpoint{3.990016in}{2.956630in}}{\pgfqpoint{3.982203in}{2.948816in}}%
\pgfpathcurveto{\pgfqpoint{3.974389in}{2.941003in}}{\pgfqpoint{3.969999in}{2.930404in}}{\pgfqpoint{3.969999in}{2.919353in}}%
\pgfpathcurveto{\pgfqpoint{3.969999in}{2.908303in}}{\pgfqpoint{3.974389in}{2.897704in}}{\pgfqpoint{3.982203in}{2.889891in}}%
\pgfpathcurveto{\pgfqpoint{3.990016in}{2.882077in}}{\pgfqpoint{4.000616in}{2.877687in}}{\pgfqpoint{4.011666in}{2.877687in}}%
\pgfpathclose%
\pgfusepath{stroke,fill}%
\end{pgfscope}%
\begin{pgfscope}%
\pgfpathrectangle{\pgfqpoint{0.800000in}{0.528000in}}{\pgfqpoint{4.960000in}{3.696000in}}%
\pgfusepath{clip}%
\pgfsetbuttcap%
\pgfsetroundjoin%
\definecolor{currentfill}{rgb}{0.000000,0.000000,0.000000}%
\pgfsetfillcolor{currentfill}%
\pgfsetlinewidth{1.003750pt}%
\definecolor{currentstroke}{rgb}{0.000000,0.000000,0.000000}%
\pgfsetstrokecolor{currentstroke}%
\pgfsetdash{}{0pt}%
\pgfpathmoveto{\pgfqpoint{4.011666in}{2.877687in}}%
\pgfpathcurveto{\pgfqpoint{4.022716in}{2.877687in}}{\pgfqpoint{4.033315in}{2.882077in}}{\pgfqpoint{4.041128in}{2.889891in}}%
\pgfpathcurveto{\pgfqpoint{4.048942in}{2.897704in}}{\pgfqpoint{4.053332in}{2.908303in}}{\pgfqpoint{4.053332in}{2.919353in}}%
\pgfpathcurveto{\pgfqpoint{4.053332in}{2.930404in}}{\pgfqpoint{4.048942in}{2.941003in}}{\pgfqpoint{4.041128in}{2.948816in}}%
\pgfpathcurveto{\pgfqpoint{4.033315in}{2.956630in}}{\pgfqpoint{4.022716in}{2.961020in}}{\pgfqpoint{4.011666in}{2.961020in}}%
\pgfpathcurveto{\pgfqpoint{4.000616in}{2.961020in}}{\pgfqpoint{3.990016in}{2.956630in}}{\pgfqpoint{3.982203in}{2.948816in}}%
\pgfpathcurveto{\pgfqpoint{3.974389in}{2.941003in}}{\pgfqpoint{3.969999in}{2.930404in}}{\pgfqpoint{3.969999in}{2.919353in}}%
\pgfpathcurveto{\pgfqpoint{3.969999in}{2.908303in}}{\pgfqpoint{3.974389in}{2.897704in}}{\pgfqpoint{3.982203in}{2.889891in}}%
\pgfpathcurveto{\pgfqpoint{3.990016in}{2.882077in}}{\pgfqpoint{4.000616in}{2.877687in}}{\pgfqpoint{4.011666in}{2.877687in}}%
\pgfpathclose%
\pgfusepath{stroke,fill}%
\end{pgfscope}%
\begin{pgfscope}%
\pgfpathrectangle{\pgfqpoint{0.800000in}{0.528000in}}{\pgfqpoint{4.960000in}{3.696000in}}%
\pgfusepath{clip}%
\pgfsetbuttcap%
\pgfsetroundjoin%
\definecolor{currentfill}{rgb}{0.000000,0.000000,0.000000}%
\pgfsetfillcolor{currentfill}%
\pgfsetlinewidth{1.003750pt}%
\definecolor{currentstroke}{rgb}{0.000000,0.000000,0.000000}%
\pgfsetstrokecolor{currentstroke}%
\pgfsetdash{}{0pt}%
\pgfpathmoveto{\pgfqpoint{4.011666in}{1.771040in}}%
\pgfpathcurveto{\pgfqpoint{4.022716in}{1.771040in}}{\pgfqpoint{4.033315in}{1.775431in}}{\pgfqpoint{4.041128in}{1.783244in}}%
\pgfpathcurveto{\pgfqpoint{4.048942in}{1.791058in}}{\pgfqpoint{4.053332in}{1.801657in}}{\pgfqpoint{4.053332in}{1.812707in}}%
\pgfpathcurveto{\pgfqpoint{4.053332in}{1.823757in}}{\pgfqpoint{4.048942in}{1.834356in}}{\pgfqpoint{4.041128in}{1.842170in}}%
\pgfpathcurveto{\pgfqpoint{4.033315in}{1.849983in}}{\pgfqpoint{4.022716in}{1.854374in}}{\pgfqpoint{4.011666in}{1.854374in}}%
\pgfpathcurveto{\pgfqpoint{4.000616in}{1.854374in}}{\pgfqpoint{3.990016in}{1.849983in}}{\pgfqpoint{3.982203in}{1.842170in}}%
\pgfpathcurveto{\pgfqpoint{3.974389in}{1.834356in}}{\pgfqpoint{3.969999in}{1.823757in}}{\pgfqpoint{3.969999in}{1.812707in}}%
\pgfpathcurveto{\pgfqpoint{3.969999in}{1.801657in}}{\pgfqpoint{3.974389in}{1.791058in}}{\pgfqpoint{3.982203in}{1.783244in}}%
\pgfpathcurveto{\pgfqpoint{3.990016in}{1.775431in}}{\pgfqpoint{4.000616in}{1.771040in}}{\pgfqpoint{4.011666in}{1.771040in}}%
\pgfpathclose%
\pgfusepath{stroke,fill}%
\end{pgfscope}%
\begin{pgfscope}%
\pgfpathrectangle{\pgfqpoint{0.800000in}{0.528000in}}{\pgfqpoint{4.960000in}{3.696000in}}%
\pgfusepath{clip}%
\pgfsetbuttcap%
\pgfsetroundjoin%
\definecolor{currentfill}{rgb}{0.000000,0.000000,0.000000}%
\pgfsetfillcolor{currentfill}%
\pgfsetlinewidth{1.003750pt}%
\definecolor{currentstroke}{rgb}{0.000000,0.000000,0.000000}%
\pgfsetstrokecolor{currentstroke}%
\pgfsetdash{}{0pt}%
\pgfpathmoveto{\pgfqpoint{4.011666in}{2.877687in}}%
\pgfpathcurveto{\pgfqpoint{4.022716in}{2.877687in}}{\pgfqpoint{4.033315in}{2.882077in}}{\pgfqpoint{4.041128in}{2.889891in}}%
\pgfpathcurveto{\pgfqpoint{4.048942in}{2.897704in}}{\pgfqpoint{4.053332in}{2.908303in}}{\pgfqpoint{4.053332in}{2.919353in}}%
\pgfpathcurveto{\pgfqpoint{4.053332in}{2.930404in}}{\pgfqpoint{4.048942in}{2.941003in}}{\pgfqpoint{4.041128in}{2.948816in}}%
\pgfpathcurveto{\pgfqpoint{4.033315in}{2.956630in}}{\pgfqpoint{4.022716in}{2.961020in}}{\pgfqpoint{4.011666in}{2.961020in}}%
\pgfpathcurveto{\pgfqpoint{4.000616in}{2.961020in}}{\pgfqpoint{3.990016in}{2.956630in}}{\pgfqpoint{3.982203in}{2.948816in}}%
\pgfpathcurveto{\pgfqpoint{3.974389in}{2.941003in}}{\pgfqpoint{3.969999in}{2.930404in}}{\pgfqpoint{3.969999in}{2.919353in}}%
\pgfpathcurveto{\pgfqpoint{3.969999in}{2.908303in}}{\pgfqpoint{3.974389in}{2.897704in}}{\pgfqpoint{3.982203in}{2.889891in}}%
\pgfpathcurveto{\pgfqpoint{3.990016in}{2.882077in}}{\pgfqpoint{4.000616in}{2.877687in}}{\pgfqpoint{4.011666in}{2.877687in}}%
\pgfpathclose%
\pgfusepath{stroke,fill}%
\end{pgfscope}%
\begin{pgfscope}%
\pgfpathrectangle{\pgfqpoint{0.800000in}{0.528000in}}{\pgfqpoint{4.960000in}{3.696000in}}%
\pgfusepath{clip}%
\pgfsetbuttcap%
\pgfsetroundjoin%
\definecolor{currentfill}{rgb}{0.000000,0.000000,0.000000}%
\pgfsetfillcolor{currentfill}%
\pgfsetlinewidth{1.003750pt}%
\definecolor{currentstroke}{rgb}{0.000000,0.000000,0.000000}%
\pgfsetstrokecolor{currentstroke}%
\pgfsetdash{}{0pt}%
\pgfpathmoveto{\pgfqpoint{4.011666in}{1.771040in}}%
\pgfpathcurveto{\pgfqpoint{4.022716in}{1.771040in}}{\pgfqpoint{4.033315in}{1.775431in}}{\pgfqpoint{4.041128in}{1.783244in}}%
\pgfpathcurveto{\pgfqpoint{4.048942in}{1.791058in}}{\pgfqpoint{4.053332in}{1.801657in}}{\pgfqpoint{4.053332in}{1.812707in}}%
\pgfpathcurveto{\pgfqpoint{4.053332in}{1.823757in}}{\pgfqpoint{4.048942in}{1.834356in}}{\pgfqpoint{4.041128in}{1.842170in}}%
\pgfpathcurveto{\pgfqpoint{4.033315in}{1.849983in}}{\pgfqpoint{4.022716in}{1.854374in}}{\pgfqpoint{4.011666in}{1.854374in}}%
\pgfpathcurveto{\pgfqpoint{4.000616in}{1.854374in}}{\pgfqpoint{3.990016in}{1.849983in}}{\pgfqpoint{3.982203in}{1.842170in}}%
\pgfpathcurveto{\pgfqpoint{3.974389in}{1.834356in}}{\pgfqpoint{3.969999in}{1.823757in}}{\pgfqpoint{3.969999in}{1.812707in}}%
\pgfpathcurveto{\pgfqpoint{3.969999in}{1.801657in}}{\pgfqpoint{3.974389in}{1.791058in}}{\pgfqpoint{3.982203in}{1.783244in}}%
\pgfpathcurveto{\pgfqpoint{3.990016in}{1.775431in}}{\pgfqpoint{4.000616in}{1.771040in}}{\pgfqpoint{4.011666in}{1.771040in}}%
\pgfpathclose%
\pgfusepath{stroke,fill}%
\end{pgfscope}%
\begin{pgfscope}%
\pgfpathrectangle{\pgfqpoint{0.800000in}{0.528000in}}{\pgfqpoint{4.960000in}{3.696000in}}%
\pgfusepath{clip}%
\pgfsetbuttcap%
\pgfsetroundjoin%
\definecolor{currentfill}{rgb}{0.000000,0.000000,0.000000}%
\pgfsetfillcolor{currentfill}%
\pgfsetlinewidth{1.003750pt}%
\definecolor{currentstroke}{rgb}{0.000000,0.000000,0.000000}%
\pgfsetstrokecolor{currentstroke}%
\pgfsetdash{}{0pt}%
\pgfpathmoveto{\pgfqpoint{4.011666in}{1.771040in}}%
\pgfpathcurveto{\pgfqpoint{4.022716in}{1.771040in}}{\pgfqpoint{4.033315in}{1.775431in}}{\pgfqpoint{4.041128in}{1.783244in}}%
\pgfpathcurveto{\pgfqpoint{4.048942in}{1.791058in}}{\pgfqpoint{4.053332in}{1.801657in}}{\pgfqpoint{4.053332in}{1.812707in}}%
\pgfpathcurveto{\pgfqpoint{4.053332in}{1.823757in}}{\pgfqpoint{4.048942in}{1.834356in}}{\pgfqpoint{4.041128in}{1.842170in}}%
\pgfpathcurveto{\pgfqpoint{4.033315in}{1.849983in}}{\pgfqpoint{4.022716in}{1.854374in}}{\pgfqpoint{4.011666in}{1.854374in}}%
\pgfpathcurveto{\pgfqpoint{4.000616in}{1.854374in}}{\pgfqpoint{3.990016in}{1.849983in}}{\pgfqpoint{3.982203in}{1.842170in}}%
\pgfpathcurveto{\pgfqpoint{3.974389in}{1.834356in}}{\pgfqpoint{3.969999in}{1.823757in}}{\pgfqpoint{3.969999in}{1.812707in}}%
\pgfpathcurveto{\pgfqpoint{3.969999in}{1.801657in}}{\pgfqpoint{3.974389in}{1.791058in}}{\pgfqpoint{3.982203in}{1.783244in}}%
\pgfpathcurveto{\pgfqpoint{3.990016in}{1.775431in}}{\pgfqpoint{4.000616in}{1.771040in}}{\pgfqpoint{4.011666in}{1.771040in}}%
\pgfpathclose%
\pgfusepath{stroke,fill}%
\end{pgfscope}%
\begin{pgfscope}%
\pgfpathrectangle{\pgfqpoint{0.800000in}{0.528000in}}{\pgfqpoint{4.960000in}{3.696000in}}%
\pgfusepath{clip}%
\pgfsetbuttcap%
\pgfsetroundjoin%
\definecolor{currentfill}{rgb}{0.000000,0.000000,0.000000}%
\pgfsetfillcolor{currentfill}%
\pgfsetlinewidth{1.003750pt}%
\definecolor{currentstroke}{rgb}{0.000000,0.000000,0.000000}%
\pgfsetstrokecolor{currentstroke}%
\pgfsetdash{}{0pt}%
\pgfpathmoveto{\pgfqpoint{4.011666in}{1.771040in}}%
\pgfpathcurveto{\pgfqpoint{4.022716in}{1.771040in}}{\pgfqpoint{4.033315in}{1.775431in}}{\pgfqpoint{4.041128in}{1.783244in}}%
\pgfpathcurveto{\pgfqpoint{4.048942in}{1.791058in}}{\pgfqpoint{4.053332in}{1.801657in}}{\pgfqpoint{4.053332in}{1.812707in}}%
\pgfpathcurveto{\pgfqpoint{4.053332in}{1.823757in}}{\pgfqpoint{4.048942in}{1.834356in}}{\pgfqpoint{4.041128in}{1.842170in}}%
\pgfpathcurveto{\pgfqpoint{4.033315in}{1.849983in}}{\pgfqpoint{4.022716in}{1.854374in}}{\pgfqpoint{4.011666in}{1.854374in}}%
\pgfpathcurveto{\pgfqpoint{4.000616in}{1.854374in}}{\pgfqpoint{3.990016in}{1.849983in}}{\pgfqpoint{3.982203in}{1.842170in}}%
\pgfpathcurveto{\pgfqpoint{3.974389in}{1.834356in}}{\pgfqpoint{3.969999in}{1.823757in}}{\pgfqpoint{3.969999in}{1.812707in}}%
\pgfpathcurveto{\pgfqpoint{3.969999in}{1.801657in}}{\pgfqpoint{3.974389in}{1.791058in}}{\pgfqpoint{3.982203in}{1.783244in}}%
\pgfpathcurveto{\pgfqpoint{3.990016in}{1.775431in}}{\pgfqpoint{4.000616in}{1.771040in}}{\pgfqpoint{4.011666in}{1.771040in}}%
\pgfpathclose%
\pgfusepath{stroke,fill}%
\end{pgfscope}%
\begin{pgfscope}%
\pgfpathrectangle{\pgfqpoint{0.800000in}{0.528000in}}{\pgfqpoint{4.960000in}{3.696000in}}%
\pgfusepath{clip}%
\pgfsetbuttcap%
\pgfsetroundjoin%
\definecolor{currentfill}{rgb}{0.000000,0.000000,0.000000}%
\pgfsetfillcolor{currentfill}%
\pgfsetlinewidth{1.003750pt}%
\definecolor{currentstroke}{rgb}{0.000000,0.000000,0.000000}%
\pgfsetstrokecolor{currentstroke}%
\pgfsetdash{}{0pt}%
\pgfpathmoveto{\pgfqpoint{4.011666in}{1.771040in}}%
\pgfpathcurveto{\pgfqpoint{4.022716in}{1.771040in}}{\pgfqpoint{4.033315in}{1.775431in}}{\pgfqpoint{4.041128in}{1.783244in}}%
\pgfpathcurveto{\pgfqpoint{4.048942in}{1.791058in}}{\pgfqpoint{4.053332in}{1.801657in}}{\pgfqpoint{4.053332in}{1.812707in}}%
\pgfpathcurveto{\pgfqpoint{4.053332in}{1.823757in}}{\pgfqpoint{4.048942in}{1.834356in}}{\pgfqpoint{4.041128in}{1.842170in}}%
\pgfpathcurveto{\pgfqpoint{4.033315in}{1.849983in}}{\pgfqpoint{4.022716in}{1.854374in}}{\pgfqpoint{4.011666in}{1.854374in}}%
\pgfpathcurveto{\pgfqpoint{4.000616in}{1.854374in}}{\pgfqpoint{3.990016in}{1.849983in}}{\pgfqpoint{3.982203in}{1.842170in}}%
\pgfpathcurveto{\pgfqpoint{3.974389in}{1.834356in}}{\pgfqpoint{3.969999in}{1.823757in}}{\pgfqpoint{3.969999in}{1.812707in}}%
\pgfpathcurveto{\pgfqpoint{3.969999in}{1.801657in}}{\pgfqpoint{3.974389in}{1.791058in}}{\pgfqpoint{3.982203in}{1.783244in}}%
\pgfpathcurveto{\pgfqpoint{3.990016in}{1.775431in}}{\pgfqpoint{4.000616in}{1.771040in}}{\pgfqpoint{4.011666in}{1.771040in}}%
\pgfpathclose%
\pgfusepath{stroke,fill}%
\end{pgfscope}%
\begin{pgfscope}%
\pgfpathrectangle{\pgfqpoint{0.800000in}{0.528000in}}{\pgfqpoint{4.960000in}{3.696000in}}%
\pgfusepath{clip}%
\pgfsetbuttcap%
\pgfsetroundjoin%
\definecolor{currentfill}{rgb}{0.000000,0.000000,0.000000}%
\pgfsetfillcolor{currentfill}%
\pgfsetlinewidth{1.003750pt}%
\definecolor{currentstroke}{rgb}{0.000000,0.000000,0.000000}%
\pgfsetstrokecolor{currentstroke}%
\pgfsetdash{}{0pt}%
\pgfpathmoveto{\pgfqpoint{4.011666in}{1.771040in}}%
\pgfpathcurveto{\pgfqpoint{4.022716in}{1.771040in}}{\pgfqpoint{4.033315in}{1.775431in}}{\pgfqpoint{4.041128in}{1.783244in}}%
\pgfpathcurveto{\pgfqpoint{4.048942in}{1.791058in}}{\pgfqpoint{4.053332in}{1.801657in}}{\pgfqpoint{4.053332in}{1.812707in}}%
\pgfpathcurveto{\pgfqpoint{4.053332in}{1.823757in}}{\pgfqpoint{4.048942in}{1.834356in}}{\pgfqpoint{4.041128in}{1.842170in}}%
\pgfpathcurveto{\pgfqpoint{4.033315in}{1.849983in}}{\pgfqpoint{4.022716in}{1.854374in}}{\pgfqpoint{4.011666in}{1.854374in}}%
\pgfpathcurveto{\pgfqpoint{4.000616in}{1.854374in}}{\pgfqpoint{3.990016in}{1.849983in}}{\pgfqpoint{3.982203in}{1.842170in}}%
\pgfpathcurveto{\pgfqpoint{3.974389in}{1.834356in}}{\pgfqpoint{3.969999in}{1.823757in}}{\pgfqpoint{3.969999in}{1.812707in}}%
\pgfpathcurveto{\pgfqpoint{3.969999in}{1.801657in}}{\pgfqpoint{3.974389in}{1.791058in}}{\pgfqpoint{3.982203in}{1.783244in}}%
\pgfpathcurveto{\pgfqpoint{3.990016in}{1.775431in}}{\pgfqpoint{4.000616in}{1.771040in}}{\pgfqpoint{4.011666in}{1.771040in}}%
\pgfpathclose%
\pgfusepath{stroke,fill}%
\end{pgfscope}%
\begin{pgfscope}%
\pgfpathrectangle{\pgfqpoint{0.800000in}{0.528000in}}{\pgfqpoint{4.960000in}{3.696000in}}%
\pgfusepath{clip}%
\pgfsetbuttcap%
\pgfsetroundjoin%
\definecolor{currentfill}{rgb}{0.000000,0.000000,0.000000}%
\pgfsetfillcolor{currentfill}%
\pgfsetlinewidth{1.003750pt}%
\definecolor{currentstroke}{rgb}{0.000000,0.000000,0.000000}%
\pgfsetstrokecolor{currentstroke}%
\pgfsetdash{}{0pt}%
\pgfpathmoveto{\pgfqpoint{4.011666in}{2.877687in}}%
\pgfpathcurveto{\pgfqpoint{4.022716in}{2.877687in}}{\pgfqpoint{4.033315in}{2.882077in}}{\pgfqpoint{4.041128in}{2.889891in}}%
\pgfpathcurveto{\pgfqpoint{4.048942in}{2.897704in}}{\pgfqpoint{4.053332in}{2.908303in}}{\pgfqpoint{4.053332in}{2.919353in}}%
\pgfpathcurveto{\pgfqpoint{4.053332in}{2.930404in}}{\pgfqpoint{4.048942in}{2.941003in}}{\pgfqpoint{4.041128in}{2.948816in}}%
\pgfpathcurveto{\pgfqpoint{4.033315in}{2.956630in}}{\pgfqpoint{4.022716in}{2.961020in}}{\pgfqpoint{4.011666in}{2.961020in}}%
\pgfpathcurveto{\pgfqpoint{4.000616in}{2.961020in}}{\pgfqpoint{3.990016in}{2.956630in}}{\pgfqpoint{3.982203in}{2.948816in}}%
\pgfpathcurveto{\pgfqpoint{3.974389in}{2.941003in}}{\pgfqpoint{3.969999in}{2.930404in}}{\pgfqpoint{3.969999in}{2.919353in}}%
\pgfpathcurveto{\pgfqpoint{3.969999in}{2.908303in}}{\pgfqpoint{3.974389in}{2.897704in}}{\pgfqpoint{3.982203in}{2.889891in}}%
\pgfpathcurveto{\pgfqpoint{3.990016in}{2.882077in}}{\pgfqpoint{4.000616in}{2.877687in}}{\pgfqpoint{4.011666in}{2.877687in}}%
\pgfpathclose%
\pgfusepath{stroke,fill}%
\end{pgfscope}%
\begin{pgfscope}%
\pgfpathrectangle{\pgfqpoint{0.800000in}{0.528000in}}{\pgfqpoint{4.960000in}{3.696000in}}%
\pgfusepath{clip}%
\pgfsetbuttcap%
\pgfsetroundjoin%
\definecolor{currentfill}{rgb}{0.000000,0.000000,0.000000}%
\pgfsetfillcolor{currentfill}%
\pgfsetlinewidth{1.003750pt}%
\definecolor{currentstroke}{rgb}{0.000000,0.000000,0.000000}%
\pgfsetstrokecolor{currentstroke}%
\pgfsetdash{}{0pt}%
\pgfpathmoveto{\pgfqpoint{4.011666in}{2.877687in}}%
\pgfpathcurveto{\pgfqpoint{4.022716in}{2.877687in}}{\pgfqpoint{4.033315in}{2.882077in}}{\pgfqpoint{4.041128in}{2.889891in}}%
\pgfpathcurveto{\pgfqpoint{4.048942in}{2.897704in}}{\pgfqpoint{4.053332in}{2.908303in}}{\pgfqpoint{4.053332in}{2.919353in}}%
\pgfpathcurveto{\pgfqpoint{4.053332in}{2.930404in}}{\pgfqpoint{4.048942in}{2.941003in}}{\pgfqpoint{4.041128in}{2.948816in}}%
\pgfpathcurveto{\pgfqpoint{4.033315in}{2.956630in}}{\pgfqpoint{4.022716in}{2.961020in}}{\pgfqpoint{4.011666in}{2.961020in}}%
\pgfpathcurveto{\pgfqpoint{4.000616in}{2.961020in}}{\pgfqpoint{3.990016in}{2.956630in}}{\pgfqpoint{3.982203in}{2.948816in}}%
\pgfpathcurveto{\pgfqpoint{3.974389in}{2.941003in}}{\pgfqpoint{3.969999in}{2.930404in}}{\pgfqpoint{3.969999in}{2.919353in}}%
\pgfpathcurveto{\pgfqpoint{3.969999in}{2.908303in}}{\pgfqpoint{3.974389in}{2.897704in}}{\pgfqpoint{3.982203in}{2.889891in}}%
\pgfpathcurveto{\pgfqpoint{3.990016in}{2.882077in}}{\pgfqpoint{4.000616in}{2.877687in}}{\pgfqpoint{4.011666in}{2.877687in}}%
\pgfpathclose%
\pgfusepath{stroke,fill}%
\end{pgfscope}%
\begin{pgfscope}%
\pgfpathrectangle{\pgfqpoint{0.800000in}{0.528000in}}{\pgfqpoint{4.960000in}{3.696000in}}%
\pgfusepath{clip}%
\pgfsetbuttcap%
\pgfsetroundjoin%
\definecolor{currentfill}{rgb}{0.000000,0.000000,0.000000}%
\pgfsetfillcolor{currentfill}%
\pgfsetlinewidth{1.003750pt}%
\definecolor{currentstroke}{rgb}{0.000000,0.000000,0.000000}%
\pgfsetstrokecolor{currentstroke}%
\pgfsetdash{}{0pt}%
\pgfpathmoveto{\pgfqpoint{4.011666in}{1.771040in}}%
\pgfpathcurveto{\pgfqpoint{4.022716in}{1.771040in}}{\pgfqpoint{4.033315in}{1.775431in}}{\pgfqpoint{4.041128in}{1.783244in}}%
\pgfpathcurveto{\pgfqpoint{4.048942in}{1.791058in}}{\pgfqpoint{4.053332in}{1.801657in}}{\pgfqpoint{4.053332in}{1.812707in}}%
\pgfpathcurveto{\pgfqpoint{4.053332in}{1.823757in}}{\pgfqpoint{4.048942in}{1.834356in}}{\pgfqpoint{4.041128in}{1.842170in}}%
\pgfpathcurveto{\pgfqpoint{4.033315in}{1.849983in}}{\pgfqpoint{4.022716in}{1.854374in}}{\pgfqpoint{4.011666in}{1.854374in}}%
\pgfpathcurveto{\pgfqpoint{4.000616in}{1.854374in}}{\pgfqpoint{3.990016in}{1.849983in}}{\pgfqpoint{3.982203in}{1.842170in}}%
\pgfpathcurveto{\pgfqpoint{3.974389in}{1.834356in}}{\pgfqpoint{3.969999in}{1.823757in}}{\pgfqpoint{3.969999in}{1.812707in}}%
\pgfpathcurveto{\pgfqpoint{3.969999in}{1.801657in}}{\pgfqpoint{3.974389in}{1.791058in}}{\pgfqpoint{3.982203in}{1.783244in}}%
\pgfpathcurveto{\pgfqpoint{3.990016in}{1.775431in}}{\pgfqpoint{4.000616in}{1.771040in}}{\pgfqpoint{4.011666in}{1.771040in}}%
\pgfpathclose%
\pgfusepath{stroke,fill}%
\end{pgfscope}%
\begin{pgfscope}%
\pgfpathrectangle{\pgfqpoint{0.800000in}{0.528000in}}{\pgfqpoint{4.960000in}{3.696000in}}%
\pgfusepath{clip}%
\pgfsetbuttcap%
\pgfsetroundjoin%
\definecolor{currentfill}{rgb}{0.000000,0.000000,0.000000}%
\pgfsetfillcolor{currentfill}%
\pgfsetlinewidth{1.003750pt}%
\definecolor{currentstroke}{rgb}{0.000000,0.000000,0.000000}%
\pgfsetstrokecolor{currentstroke}%
\pgfsetdash{}{0pt}%
\pgfpathmoveto{\pgfqpoint{4.011666in}{1.771040in}}%
\pgfpathcurveto{\pgfqpoint{4.022716in}{1.771040in}}{\pgfqpoint{4.033315in}{1.775431in}}{\pgfqpoint{4.041128in}{1.783244in}}%
\pgfpathcurveto{\pgfqpoint{4.048942in}{1.791058in}}{\pgfqpoint{4.053332in}{1.801657in}}{\pgfqpoint{4.053332in}{1.812707in}}%
\pgfpathcurveto{\pgfqpoint{4.053332in}{1.823757in}}{\pgfqpoint{4.048942in}{1.834356in}}{\pgfqpoint{4.041128in}{1.842170in}}%
\pgfpathcurveto{\pgfqpoint{4.033315in}{1.849983in}}{\pgfqpoint{4.022716in}{1.854374in}}{\pgfqpoint{4.011666in}{1.854374in}}%
\pgfpathcurveto{\pgfqpoint{4.000616in}{1.854374in}}{\pgfqpoint{3.990016in}{1.849983in}}{\pgfqpoint{3.982203in}{1.842170in}}%
\pgfpathcurveto{\pgfqpoint{3.974389in}{1.834356in}}{\pgfqpoint{3.969999in}{1.823757in}}{\pgfqpoint{3.969999in}{1.812707in}}%
\pgfpathcurveto{\pgfqpoint{3.969999in}{1.801657in}}{\pgfqpoint{3.974389in}{1.791058in}}{\pgfqpoint{3.982203in}{1.783244in}}%
\pgfpathcurveto{\pgfqpoint{3.990016in}{1.775431in}}{\pgfqpoint{4.000616in}{1.771040in}}{\pgfqpoint{4.011666in}{1.771040in}}%
\pgfpathclose%
\pgfusepath{stroke,fill}%
\end{pgfscope}%
\begin{pgfscope}%
\pgfpathrectangle{\pgfqpoint{0.800000in}{0.528000in}}{\pgfqpoint{4.960000in}{3.696000in}}%
\pgfusepath{clip}%
\pgfsetbuttcap%
\pgfsetroundjoin%
\definecolor{currentfill}{rgb}{0.000000,0.000000,0.000000}%
\pgfsetfillcolor{currentfill}%
\pgfsetlinewidth{1.003750pt}%
\definecolor{currentstroke}{rgb}{0.000000,0.000000,0.000000}%
\pgfsetstrokecolor{currentstroke}%
\pgfsetdash{}{0pt}%
\pgfpathmoveto{\pgfqpoint{4.011666in}{1.771040in}}%
\pgfpathcurveto{\pgfqpoint{4.022716in}{1.771040in}}{\pgfqpoint{4.033315in}{1.775431in}}{\pgfqpoint{4.041128in}{1.783244in}}%
\pgfpathcurveto{\pgfqpoint{4.048942in}{1.791058in}}{\pgfqpoint{4.053332in}{1.801657in}}{\pgfqpoint{4.053332in}{1.812707in}}%
\pgfpathcurveto{\pgfqpoint{4.053332in}{1.823757in}}{\pgfqpoint{4.048942in}{1.834356in}}{\pgfqpoint{4.041128in}{1.842170in}}%
\pgfpathcurveto{\pgfqpoint{4.033315in}{1.849983in}}{\pgfqpoint{4.022716in}{1.854374in}}{\pgfqpoint{4.011666in}{1.854374in}}%
\pgfpathcurveto{\pgfqpoint{4.000616in}{1.854374in}}{\pgfqpoint{3.990016in}{1.849983in}}{\pgfqpoint{3.982203in}{1.842170in}}%
\pgfpathcurveto{\pgfqpoint{3.974389in}{1.834356in}}{\pgfqpoint{3.969999in}{1.823757in}}{\pgfqpoint{3.969999in}{1.812707in}}%
\pgfpathcurveto{\pgfqpoint{3.969999in}{1.801657in}}{\pgfqpoint{3.974389in}{1.791058in}}{\pgfqpoint{3.982203in}{1.783244in}}%
\pgfpathcurveto{\pgfqpoint{3.990016in}{1.775431in}}{\pgfqpoint{4.000616in}{1.771040in}}{\pgfqpoint{4.011666in}{1.771040in}}%
\pgfpathclose%
\pgfusepath{stroke,fill}%
\end{pgfscope}%
\begin{pgfscope}%
\pgfpathrectangle{\pgfqpoint{0.800000in}{0.528000in}}{\pgfqpoint{4.960000in}{3.696000in}}%
\pgfusepath{clip}%
\pgfsetbuttcap%
\pgfsetroundjoin%
\definecolor{currentfill}{rgb}{0.000000,0.000000,0.000000}%
\pgfsetfillcolor{currentfill}%
\pgfsetlinewidth{1.003750pt}%
\definecolor{currentstroke}{rgb}{0.000000,0.000000,0.000000}%
\pgfsetstrokecolor{currentstroke}%
\pgfsetdash{}{0pt}%
\pgfpathmoveto{\pgfqpoint{4.011666in}{1.771040in}}%
\pgfpathcurveto{\pgfqpoint{4.022716in}{1.771040in}}{\pgfqpoint{4.033315in}{1.775431in}}{\pgfqpoint{4.041128in}{1.783244in}}%
\pgfpathcurveto{\pgfqpoint{4.048942in}{1.791058in}}{\pgfqpoint{4.053332in}{1.801657in}}{\pgfqpoint{4.053332in}{1.812707in}}%
\pgfpathcurveto{\pgfqpoint{4.053332in}{1.823757in}}{\pgfqpoint{4.048942in}{1.834356in}}{\pgfqpoint{4.041128in}{1.842170in}}%
\pgfpathcurveto{\pgfqpoint{4.033315in}{1.849983in}}{\pgfqpoint{4.022716in}{1.854374in}}{\pgfqpoint{4.011666in}{1.854374in}}%
\pgfpathcurveto{\pgfqpoint{4.000616in}{1.854374in}}{\pgfqpoint{3.990016in}{1.849983in}}{\pgfqpoint{3.982203in}{1.842170in}}%
\pgfpathcurveto{\pgfqpoint{3.974389in}{1.834356in}}{\pgfqpoint{3.969999in}{1.823757in}}{\pgfqpoint{3.969999in}{1.812707in}}%
\pgfpathcurveto{\pgfqpoint{3.969999in}{1.801657in}}{\pgfqpoint{3.974389in}{1.791058in}}{\pgfqpoint{3.982203in}{1.783244in}}%
\pgfpathcurveto{\pgfqpoint{3.990016in}{1.775431in}}{\pgfqpoint{4.000616in}{1.771040in}}{\pgfqpoint{4.011666in}{1.771040in}}%
\pgfpathclose%
\pgfusepath{stroke,fill}%
\end{pgfscope}%
\begin{pgfscope}%
\pgfpathrectangle{\pgfqpoint{0.800000in}{0.528000in}}{\pgfqpoint{4.960000in}{3.696000in}}%
\pgfusepath{clip}%
\pgfsetbuttcap%
\pgfsetroundjoin%
\definecolor{currentfill}{rgb}{0.000000,0.000000,0.000000}%
\pgfsetfillcolor{currentfill}%
\pgfsetlinewidth{1.003750pt}%
\definecolor{currentstroke}{rgb}{0.000000,0.000000,0.000000}%
\pgfsetstrokecolor{currentstroke}%
\pgfsetdash{}{0pt}%
\pgfpathmoveto{\pgfqpoint{4.011666in}{2.877687in}}%
\pgfpathcurveto{\pgfqpoint{4.022716in}{2.877687in}}{\pgfqpoint{4.033315in}{2.882077in}}{\pgfqpoint{4.041128in}{2.889891in}}%
\pgfpathcurveto{\pgfqpoint{4.048942in}{2.897704in}}{\pgfqpoint{4.053332in}{2.908303in}}{\pgfqpoint{4.053332in}{2.919353in}}%
\pgfpathcurveto{\pgfqpoint{4.053332in}{2.930404in}}{\pgfqpoint{4.048942in}{2.941003in}}{\pgfqpoint{4.041128in}{2.948816in}}%
\pgfpathcurveto{\pgfqpoint{4.033315in}{2.956630in}}{\pgfqpoint{4.022716in}{2.961020in}}{\pgfqpoint{4.011666in}{2.961020in}}%
\pgfpathcurveto{\pgfqpoint{4.000616in}{2.961020in}}{\pgfqpoint{3.990016in}{2.956630in}}{\pgfqpoint{3.982203in}{2.948816in}}%
\pgfpathcurveto{\pgfqpoint{3.974389in}{2.941003in}}{\pgfqpoint{3.969999in}{2.930404in}}{\pgfqpoint{3.969999in}{2.919353in}}%
\pgfpathcurveto{\pgfqpoint{3.969999in}{2.908303in}}{\pgfqpoint{3.974389in}{2.897704in}}{\pgfqpoint{3.982203in}{2.889891in}}%
\pgfpathcurveto{\pgfqpoint{3.990016in}{2.882077in}}{\pgfqpoint{4.000616in}{2.877687in}}{\pgfqpoint{4.011666in}{2.877687in}}%
\pgfpathclose%
\pgfusepath{stroke,fill}%
\end{pgfscope}%
\begin{pgfscope}%
\pgfpathrectangle{\pgfqpoint{0.800000in}{0.528000in}}{\pgfqpoint{4.960000in}{3.696000in}}%
\pgfusepath{clip}%
\pgfsetbuttcap%
\pgfsetroundjoin%
\definecolor{currentfill}{rgb}{0.000000,0.000000,0.000000}%
\pgfsetfillcolor{currentfill}%
\pgfsetlinewidth{1.003750pt}%
\definecolor{currentstroke}{rgb}{0.000000,0.000000,0.000000}%
\pgfsetstrokecolor{currentstroke}%
\pgfsetdash{}{0pt}%
\pgfpathmoveto{\pgfqpoint{4.011666in}{2.877687in}}%
\pgfpathcurveto{\pgfqpoint{4.022716in}{2.877687in}}{\pgfqpoint{4.033315in}{2.882077in}}{\pgfqpoint{4.041128in}{2.889891in}}%
\pgfpathcurveto{\pgfqpoint{4.048942in}{2.897704in}}{\pgfqpoint{4.053332in}{2.908303in}}{\pgfqpoint{4.053332in}{2.919353in}}%
\pgfpathcurveto{\pgfqpoint{4.053332in}{2.930404in}}{\pgfqpoint{4.048942in}{2.941003in}}{\pgfqpoint{4.041128in}{2.948816in}}%
\pgfpathcurveto{\pgfqpoint{4.033315in}{2.956630in}}{\pgfqpoint{4.022716in}{2.961020in}}{\pgfqpoint{4.011666in}{2.961020in}}%
\pgfpathcurveto{\pgfqpoint{4.000616in}{2.961020in}}{\pgfqpoint{3.990016in}{2.956630in}}{\pgfqpoint{3.982203in}{2.948816in}}%
\pgfpathcurveto{\pgfqpoint{3.974389in}{2.941003in}}{\pgfqpoint{3.969999in}{2.930404in}}{\pgfqpoint{3.969999in}{2.919353in}}%
\pgfpathcurveto{\pgfqpoint{3.969999in}{2.908303in}}{\pgfqpoint{3.974389in}{2.897704in}}{\pgfqpoint{3.982203in}{2.889891in}}%
\pgfpathcurveto{\pgfqpoint{3.990016in}{2.882077in}}{\pgfqpoint{4.000616in}{2.877687in}}{\pgfqpoint{4.011666in}{2.877687in}}%
\pgfpathclose%
\pgfusepath{stroke,fill}%
\end{pgfscope}%
\begin{pgfscope}%
\pgfpathrectangle{\pgfqpoint{0.800000in}{0.528000in}}{\pgfqpoint{4.960000in}{3.696000in}}%
\pgfusepath{clip}%
\pgfsetbuttcap%
\pgfsetroundjoin%
\definecolor{currentfill}{rgb}{0.000000,0.000000,0.000000}%
\pgfsetfillcolor{currentfill}%
\pgfsetlinewidth{1.003750pt}%
\definecolor{currentstroke}{rgb}{0.000000,0.000000,0.000000}%
\pgfsetstrokecolor{currentstroke}%
\pgfsetdash{}{0pt}%
\pgfpathmoveto{\pgfqpoint{4.011666in}{2.877687in}}%
\pgfpathcurveto{\pgfqpoint{4.022716in}{2.877687in}}{\pgfqpoint{4.033315in}{2.882077in}}{\pgfqpoint{4.041128in}{2.889891in}}%
\pgfpathcurveto{\pgfqpoint{4.048942in}{2.897704in}}{\pgfqpoint{4.053332in}{2.908303in}}{\pgfqpoint{4.053332in}{2.919353in}}%
\pgfpathcurveto{\pgfqpoint{4.053332in}{2.930404in}}{\pgfqpoint{4.048942in}{2.941003in}}{\pgfqpoint{4.041128in}{2.948816in}}%
\pgfpathcurveto{\pgfqpoint{4.033315in}{2.956630in}}{\pgfqpoint{4.022716in}{2.961020in}}{\pgfqpoint{4.011666in}{2.961020in}}%
\pgfpathcurveto{\pgfqpoint{4.000616in}{2.961020in}}{\pgfqpoint{3.990016in}{2.956630in}}{\pgfqpoint{3.982203in}{2.948816in}}%
\pgfpathcurveto{\pgfqpoint{3.974389in}{2.941003in}}{\pgfqpoint{3.969999in}{2.930404in}}{\pgfqpoint{3.969999in}{2.919353in}}%
\pgfpathcurveto{\pgfqpoint{3.969999in}{2.908303in}}{\pgfqpoint{3.974389in}{2.897704in}}{\pgfqpoint{3.982203in}{2.889891in}}%
\pgfpathcurveto{\pgfqpoint{3.990016in}{2.882077in}}{\pgfqpoint{4.000616in}{2.877687in}}{\pgfqpoint{4.011666in}{2.877687in}}%
\pgfpathclose%
\pgfusepath{stroke,fill}%
\end{pgfscope}%
\begin{pgfscope}%
\pgfpathrectangle{\pgfqpoint{0.800000in}{0.528000in}}{\pgfqpoint{4.960000in}{3.696000in}}%
\pgfusepath{clip}%
\pgfsetbuttcap%
\pgfsetroundjoin%
\definecolor{currentfill}{rgb}{0.000000,0.000000,0.000000}%
\pgfsetfillcolor{currentfill}%
\pgfsetlinewidth{1.003750pt}%
\definecolor{currentstroke}{rgb}{0.000000,0.000000,0.000000}%
\pgfsetstrokecolor{currentstroke}%
\pgfsetdash{}{0pt}%
\pgfpathmoveto{\pgfqpoint{4.011666in}{2.877687in}}%
\pgfpathcurveto{\pgfqpoint{4.022716in}{2.877687in}}{\pgfqpoint{4.033315in}{2.882077in}}{\pgfqpoint{4.041128in}{2.889891in}}%
\pgfpathcurveto{\pgfqpoint{4.048942in}{2.897704in}}{\pgfqpoint{4.053332in}{2.908303in}}{\pgfqpoint{4.053332in}{2.919353in}}%
\pgfpathcurveto{\pgfqpoint{4.053332in}{2.930404in}}{\pgfqpoint{4.048942in}{2.941003in}}{\pgfqpoint{4.041128in}{2.948816in}}%
\pgfpathcurveto{\pgfqpoint{4.033315in}{2.956630in}}{\pgfqpoint{4.022716in}{2.961020in}}{\pgfqpoint{4.011666in}{2.961020in}}%
\pgfpathcurveto{\pgfqpoint{4.000616in}{2.961020in}}{\pgfqpoint{3.990016in}{2.956630in}}{\pgfqpoint{3.982203in}{2.948816in}}%
\pgfpathcurveto{\pgfqpoint{3.974389in}{2.941003in}}{\pgfqpoint{3.969999in}{2.930404in}}{\pgfqpoint{3.969999in}{2.919353in}}%
\pgfpathcurveto{\pgfqpoint{3.969999in}{2.908303in}}{\pgfqpoint{3.974389in}{2.897704in}}{\pgfqpoint{3.982203in}{2.889891in}}%
\pgfpathcurveto{\pgfqpoint{3.990016in}{2.882077in}}{\pgfqpoint{4.000616in}{2.877687in}}{\pgfqpoint{4.011666in}{2.877687in}}%
\pgfpathclose%
\pgfusepath{stroke,fill}%
\end{pgfscope}%
\begin{pgfscope}%
\pgfpathrectangle{\pgfqpoint{0.800000in}{0.528000in}}{\pgfqpoint{4.960000in}{3.696000in}}%
\pgfusepath{clip}%
\pgfsetbuttcap%
\pgfsetroundjoin%
\definecolor{currentfill}{rgb}{0.000000,0.000000,0.000000}%
\pgfsetfillcolor{currentfill}%
\pgfsetlinewidth{1.003750pt}%
\definecolor{currentstroke}{rgb}{0.000000,0.000000,0.000000}%
\pgfsetstrokecolor{currentstroke}%
\pgfsetdash{}{0pt}%
\pgfpathmoveto{\pgfqpoint{4.011666in}{1.771040in}}%
\pgfpathcurveto{\pgfqpoint{4.022716in}{1.771040in}}{\pgfqpoint{4.033315in}{1.775431in}}{\pgfqpoint{4.041128in}{1.783244in}}%
\pgfpathcurveto{\pgfqpoint{4.048942in}{1.791058in}}{\pgfqpoint{4.053332in}{1.801657in}}{\pgfqpoint{4.053332in}{1.812707in}}%
\pgfpathcurveto{\pgfqpoint{4.053332in}{1.823757in}}{\pgfqpoint{4.048942in}{1.834356in}}{\pgfqpoint{4.041128in}{1.842170in}}%
\pgfpathcurveto{\pgfqpoint{4.033315in}{1.849983in}}{\pgfqpoint{4.022716in}{1.854374in}}{\pgfqpoint{4.011666in}{1.854374in}}%
\pgfpathcurveto{\pgfqpoint{4.000616in}{1.854374in}}{\pgfqpoint{3.990016in}{1.849983in}}{\pgfqpoint{3.982203in}{1.842170in}}%
\pgfpathcurveto{\pgfqpoint{3.974389in}{1.834356in}}{\pgfqpoint{3.969999in}{1.823757in}}{\pgfqpoint{3.969999in}{1.812707in}}%
\pgfpathcurveto{\pgfqpoint{3.969999in}{1.801657in}}{\pgfqpoint{3.974389in}{1.791058in}}{\pgfqpoint{3.982203in}{1.783244in}}%
\pgfpathcurveto{\pgfqpoint{3.990016in}{1.775431in}}{\pgfqpoint{4.000616in}{1.771040in}}{\pgfqpoint{4.011666in}{1.771040in}}%
\pgfpathclose%
\pgfusepath{stroke,fill}%
\end{pgfscope}%
\begin{pgfscope}%
\pgfpathrectangle{\pgfqpoint{0.800000in}{0.528000in}}{\pgfqpoint{4.960000in}{3.696000in}}%
\pgfusepath{clip}%
\pgfsetbuttcap%
\pgfsetroundjoin%
\definecolor{currentfill}{rgb}{0.000000,0.000000,0.000000}%
\pgfsetfillcolor{currentfill}%
\pgfsetlinewidth{1.003750pt}%
\definecolor{currentstroke}{rgb}{0.000000,0.000000,0.000000}%
\pgfsetstrokecolor{currentstroke}%
\pgfsetdash{}{0pt}%
\pgfpathmoveto{\pgfqpoint{4.011666in}{2.877687in}}%
\pgfpathcurveto{\pgfqpoint{4.022716in}{2.877687in}}{\pgfqpoint{4.033315in}{2.882077in}}{\pgfqpoint{4.041128in}{2.889891in}}%
\pgfpathcurveto{\pgfqpoint{4.048942in}{2.897704in}}{\pgfqpoint{4.053332in}{2.908303in}}{\pgfqpoint{4.053332in}{2.919353in}}%
\pgfpathcurveto{\pgfqpoint{4.053332in}{2.930404in}}{\pgfqpoint{4.048942in}{2.941003in}}{\pgfqpoint{4.041128in}{2.948816in}}%
\pgfpathcurveto{\pgfqpoint{4.033315in}{2.956630in}}{\pgfqpoint{4.022716in}{2.961020in}}{\pgfqpoint{4.011666in}{2.961020in}}%
\pgfpathcurveto{\pgfqpoint{4.000616in}{2.961020in}}{\pgfqpoint{3.990016in}{2.956630in}}{\pgfqpoint{3.982203in}{2.948816in}}%
\pgfpathcurveto{\pgfqpoint{3.974389in}{2.941003in}}{\pgfqpoint{3.969999in}{2.930404in}}{\pgfqpoint{3.969999in}{2.919353in}}%
\pgfpathcurveto{\pgfqpoint{3.969999in}{2.908303in}}{\pgfqpoint{3.974389in}{2.897704in}}{\pgfqpoint{3.982203in}{2.889891in}}%
\pgfpathcurveto{\pgfqpoint{3.990016in}{2.882077in}}{\pgfqpoint{4.000616in}{2.877687in}}{\pgfqpoint{4.011666in}{2.877687in}}%
\pgfpathclose%
\pgfusepath{stroke,fill}%
\end{pgfscope}%
\begin{pgfscope}%
\pgfpathrectangle{\pgfqpoint{0.800000in}{0.528000in}}{\pgfqpoint{4.960000in}{3.696000in}}%
\pgfusepath{clip}%
\pgfsetbuttcap%
\pgfsetroundjoin%
\definecolor{currentfill}{rgb}{0.000000,0.000000,0.000000}%
\pgfsetfillcolor{currentfill}%
\pgfsetlinewidth{1.003750pt}%
\definecolor{currentstroke}{rgb}{0.000000,0.000000,0.000000}%
\pgfsetstrokecolor{currentstroke}%
\pgfsetdash{}{0pt}%
\pgfpathmoveto{\pgfqpoint{4.011666in}{2.877687in}}%
\pgfpathcurveto{\pgfqpoint{4.022716in}{2.877687in}}{\pgfqpoint{4.033315in}{2.882077in}}{\pgfqpoint{4.041128in}{2.889891in}}%
\pgfpathcurveto{\pgfqpoint{4.048942in}{2.897704in}}{\pgfqpoint{4.053332in}{2.908303in}}{\pgfqpoint{4.053332in}{2.919353in}}%
\pgfpathcurveto{\pgfqpoint{4.053332in}{2.930404in}}{\pgfqpoint{4.048942in}{2.941003in}}{\pgfqpoint{4.041128in}{2.948816in}}%
\pgfpathcurveto{\pgfqpoint{4.033315in}{2.956630in}}{\pgfqpoint{4.022716in}{2.961020in}}{\pgfqpoint{4.011666in}{2.961020in}}%
\pgfpathcurveto{\pgfqpoint{4.000616in}{2.961020in}}{\pgfqpoint{3.990016in}{2.956630in}}{\pgfqpoint{3.982203in}{2.948816in}}%
\pgfpathcurveto{\pgfqpoint{3.974389in}{2.941003in}}{\pgfqpoint{3.969999in}{2.930404in}}{\pgfqpoint{3.969999in}{2.919353in}}%
\pgfpathcurveto{\pgfqpoint{3.969999in}{2.908303in}}{\pgfqpoint{3.974389in}{2.897704in}}{\pgfqpoint{3.982203in}{2.889891in}}%
\pgfpathcurveto{\pgfqpoint{3.990016in}{2.882077in}}{\pgfqpoint{4.000616in}{2.877687in}}{\pgfqpoint{4.011666in}{2.877687in}}%
\pgfpathclose%
\pgfusepath{stroke,fill}%
\end{pgfscope}%
\begin{pgfscope}%
\pgfpathrectangle{\pgfqpoint{0.800000in}{0.528000in}}{\pgfqpoint{4.960000in}{3.696000in}}%
\pgfusepath{clip}%
\pgfsetbuttcap%
\pgfsetroundjoin%
\definecolor{currentfill}{rgb}{0.000000,0.000000,0.000000}%
\pgfsetfillcolor{currentfill}%
\pgfsetlinewidth{1.003750pt}%
\definecolor{currentstroke}{rgb}{0.000000,0.000000,0.000000}%
\pgfsetstrokecolor{currentstroke}%
\pgfsetdash{}{0pt}%
\pgfpathmoveto{\pgfqpoint{5.504545in}{2.877687in}}%
\pgfpathcurveto{\pgfqpoint{5.515596in}{2.877687in}}{\pgfqpoint{5.526195in}{2.882077in}}{\pgfqpoint{5.534008in}{2.889891in}}%
\pgfpathcurveto{\pgfqpoint{5.541822in}{2.897704in}}{\pgfqpoint{5.546212in}{2.908303in}}{\pgfqpoint{5.546212in}{2.919353in}}%
\pgfpathcurveto{\pgfqpoint{5.546212in}{2.930404in}}{\pgfqpoint{5.541822in}{2.941003in}}{\pgfqpoint{5.534008in}{2.948816in}}%
\pgfpathcurveto{\pgfqpoint{5.526195in}{2.956630in}}{\pgfqpoint{5.515596in}{2.961020in}}{\pgfqpoint{5.504545in}{2.961020in}}%
\pgfpathcurveto{\pgfqpoint{5.493495in}{2.961020in}}{\pgfqpoint{5.482896in}{2.956630in}}{\pgfqpoint{5.475083in}{2.948816in}}%
\pgfpathcurveto{\pgfqpoint{5.467269in}{2.941003in}}{\pgfqpoint{5.462879in}{2.930404in}}{\pgfqpoint{5.462879in}{2.919353in}}%
\pgfpathcurveto{\pgfqpoint{5.462879in}{2.908303in}}{\pgfqpoint{5.467269in}{2.897704in}}{\pgfqpoint{5.475083in}{2.889891in}}%
\pgfpathcurveto{\pgfqpoint{5.482896in}{2.882077in}}{\pgfqpoint{5.493495in}{2.877687in}}{\pgfqpoint{5.504545in}{2.877687in}}%
\pgfpathclose%
\pgfusepath{stroke,fill}%
\end{pgfscope}%
\begin{pgfscope}%
\pgfpathrectangle{\pgfqpoint{0.800000in}{0.528000in}}{\pgfqpoint{4.960000in}{3.696000in}}%
\pgfusepath{clip}%
\pgfsetbuttcap%
\pgfsetroundjoin%
\definecolor{currentfill}{rgb}{0.000000,0.000000,0.000000}%
\pgfsetfillcolor{currentfill}%
\pgfsetlinewidth{1.003750pt}%
\definecolor{currentstroke}{rgb}{0.000000,0.000000,0.000000}%
\pgfsetstrokecolor{currentstroke}%
\pgfsetdash{}{0pt}%
\pgfpathmoveto{\pgfqpoint{5.504545in}{2.877687in}}%
\pgfpathcurveto{\pgfqpoint{5.515596in}{2.877687in}}{\pgfqpoint{5.526195in}{2.882077in}}{\pgfqpoint{5.534008in}{2.889891in}}%
\pgfpathcurveto{\pgfqpoint{5.541822in}{2.897704in}}{\pgfqpoint{5.546212in}{2.908303in}}{\pgfqpoint{5.546212in}{2.919353in}}%
\pgfpathcurveto{\pgfqpoint{5.546212in}{2.930404in}}{\pgfqpoint{5.541822in}{2.941003in}}{\pgfqpoint{5.534008in}{2.948816in}}%
\pgfpathcurveto{\pgfqpoint{5.526195in}{2.956630in}}{\pgfqpoint{5.515596in}{2.961020in}}{\pgfqpoint{5.504545in}{2.961020in}}%
\pgfpathcurveto{\pgfqpoint{5.493495in}{2.961020in}}{\pgfqpoint{5.482896in}{2.956630in}}{\pgfqpoint{5.475083in}{2.948816in}}%
\pgfpathcurveto{\pgfqpoint{5.467269in}{2.941003in}}{\pgfqpoint{5.462879in}{2.930404in}}{\pgfqpoint{5.462879in}{2.919353in}}%
\pgfpathcurveto{\pgfqpoint{5.462879in}{2.908303in}}{\pgfqpoint{5.467269in}{2.897704in}}{\pgfqpoint{5.475083in}{2.889891in}}%
\pgfpathcurveto{\pgfqpoint{5.482896in}{2.882077in}}{\pgfqpoint{5.493495in}{2.877687in}}{\pgfqpoint{5.504545in}{2.877687in}}%
\pgfpathclose%
\pgfusepath{stroke,fill}%
\end{pgfscope}%
\begin{pgfscope}%
\pgfpathrectangle{\pgfqpoint{0.800000in}{0.528000in}}{\pgfqpoint{4.960000in}{3.696000in}}%
\pgfusepath{clip}%
\pgfsetbuttcap%
\pgfsetroundjoin%
\definecolor{currentfill}{rgb}{0.000000,0.000000,0.000000}%
\pgfsetfillcolor{currentfill}%
\pgfsetlinewidth{1.003750pt}%
\definecolor{currentstroke}{rgb}{0.000000,0.000000,0.000000}%
\pgfsetstrokecolor{currentstroke}%
\pgfsetdash{}{0pt}%
\pgfpathmoveto{\pgfqpoint{5.504545in}{2.877687in}}%
\pgfpathcurveto{\pgfqpoint{5.515596in}{2.877687in}}{\pgfqpoint{5.526195in}{2.882077in}}{\pgfqpoint{5.534008in}{2.889891in}}%
\pgfpathcurveto{\pgfqpoint{5.541822in}{2.897704in}}{\pgfqpoint{5.546212in}{2.908303in}}{\pgfqpoint{5.546212in}{2.919353in}}%
\pgfpathcurveto{\pgfqpoint{5.546212in}{2.930404in}}{\pgfqpoint{5.541822in}{2.941003in}}{\pgfqpoint{5.534008in}{2.948816in}}%
\pgfpathcurveto{\pgfqpoint{5.526195in}{2.956630in}}{\pgfqpoint{5.515596in}{2.961020in}}{\pgfqpoint{5.504545in}{2.961020in}}%
\pgfpathcurveto{\pgfqpoint{5.493495in}{2.961020in}}{\pgfqpoint{5.482896in}{2.956630in}}{\pgfqpoint{5.475083in}{2.948816in}}%
\pgfpathcurveto{\pgfqpoint{5.467269in}{2.941003in}}{\pgfqpoint{5.462879in}{2.930404in}}{\pgfqpoint{5.462879in}{2.919353in}}%
\pgfpathcurveto{\pgfqpoint{5.462879in}{2.908303in}}{\pgfqpoint{5.467269in}{2.897704in}}{\pgfqpoint{5.475083in}{2.889891in}}%
\pgfpathcurveto{\pgfqpoint{5.482896in}{2.882077in}}{\pgfqpoint{5.493495in}{2.877687in}}{\pgfqpoint{5.504545in}{2.877687in}}%
\pgfpathclose%
\pgfusepath{stroke,fill}%
\end{pgfscope}%
\begin{pgfscope}%
\pgfpathrectangle{\pgfqpoint{0.800000in}{0.528000in}}{\pgfqpoint{4.960000in}{3.696000in}}%
\pgfusepath{clip}%
\pgfsetbuttcap%
\pgfsetroundjoin%
\definecolor{currentfill}{rgb}{0.000000,0.000000,0.000000}%
\pgfsetfillcolor{currentfill}%
\pgfsetlinewidth{1.003750pt}%
\definecolor{currentstroke}{rgb}{0.000000,0.000000,0.000000}%
\pgfsetstrokecolor{currentstroke}%
\pgfsetdash{}{0pt}%
\pgfpathmoveto{\pgfqpoint{5.504545in}{2.877687in}}%
\pgfpathcurveto{\pgfqpoint{5.515596in}{2.877687in}}{\pgfqpoint{5.526195in}{2.882077in}}{\pgfqpoint{5.534008in}{2.889891in}}%
\pgfpathcurveto{\pgfqpoint{5.541822in}{2.897704in}}{\pgfqpoint{5.546212in}{2.908303in}}{\pgfqpoint{5.546212in}{2.919353in}}%
\pgfpathcurveto{\pgfqpoint{5.546212in}{2.930404in}}{\pgfqpoint{5.541822in}{2.941003in}}{\pgfqpoint{5.534008in}{2.948816in}}%
\pgfpathcurveto{\pgfqpoint{5.526195in}{2.956630in}}{\pgfqpoint{5.515596in}{2.961020in}}{\pgfqpoint{5.504545in}{2.961020in}}%
\pgfpathcurveto{\pgfqpoint{5.493495in}{2.961020in}}{\pgfqpoint{5.482896in}{2.956630in}}{\pgfqpoint{5.475083in}{2.948816in}}%
\pgfpathcurveto{\pgfqpoint{5.467269in}{2.941003in}}{\pgfqpoint{5.462879in}{2.930404in}}{\pgfqpoint{5.462879in}{2.919353in}}%
\pgfpathcurveto{\pgfqpoint{5.462879in}{2.908303in}}{\pgfqpoint{5.467269in}{2.897704in}}{\pgfqpoint{5.475083in}{2.889891in}}%
\pgfpathcurveto{\pgfqpoint{5.482896in}{2.882077in}}{\pgfqpoint{5.493495in}{2.877687in}}{\pgfqpoint{5.504545in}{2.877687in}}%
\pgfpathclose%
\pgfusepath{stroke,fill}%
\end{pgfscope}%
\begin{pgfscope}%
\pgfpathrectangle{\pgfqpoint{0.800000in}{0.528000in}}{\pgfqpoint{4.960000in}{3.696000in}}%
\pgfusepath{clip}%
\pgfsetbuttcap%
\pgfsetroundjoin%
\definecolor{currentfill}{rgb}{0.000000,0.000000,0.000000}%
\pgfsetfillcolor{currentfill}%
\pgfsetlinewidth{1.003750pt}%
\definecolor{currentstroke}{rgb}{0.000000,0.000000,0.000000}%
\pgfsetstrokecolor{currentstroke}%
\pgfsetdash{}{0pt}%
\pgfpathmoveto{\pgfqpoint{5.504545in}{2.877687in}}%
\pgfpathcurveto{\pgfqpoint{5.515596in}{2.877687in}}{\pgfqpoint{5.526195in}{2.882077in}}{\pgfqpoint{5.534008in}{2.889891in}}%
\pgfpathcurveto{\pgfqpoint{5.541822in}{2.897704in}}{\pgfqpoint{5.546212in}{2.908303in}}{\pgfqpoint{5.546212in}{2.919353in}}%
\pgfpathcurveto{\pgfqpoint{5.546212in}{2.930404in}}{\pgfqpoint{5.541822in}{2.941003in}}{\pgfqpoint{5.534008in}{2.948816in}}%
\pgfpathcurveto{\pgfqpoint{5.526195in}{2.956630in}}{\pgfqpoint{5.515596in}{2.961020in}}{\pgfqpoint{5.504545in}{2.961020in}}%
\pgfpathcurveto{\pgfqpoint{5.493495in}{2.961020in}}{\pgfqpoint{5.482896in}{2.956630in}}{\pgfqpoint{5.475083in}{2.948816in}}%
\pgfpathcurveto{\pgfqpoint{5.467269in}{2.941003in}}{\pgfqpoint{5.462879in}{2.930404in}}{\pgfqpoint{5.462879in}{2.919353in}}%
\pgfpathcurveto{\pgfqpoint{5.462879in}{2.908303in}}{\pgfqpoint{5.467269in}{2.897704in}}{\pgfqpoint{5.475083in}{2.889891in}}%
\pgfpathcurveto{\pgfqpoint{5.482896in}{2.882077in}}{\pgfqpoint{5.493495in}{2.877687in}}{\pgfqpoint{5.504545in}{2.877687in}}%
\pgfpathclose%
\pgfusepath{stroke,fill}%
\end{pgfscope}%
\begin{pgfscope}%
\pgfpathrectangle{\pgfqpoint{0.800000in}{0.528000in}}{\pgfqpoint{4.960000in}{3.696000in}}%
\pgfusepath{clip}%
\pgfsetbuttcap%
\pgfsetroundjoin%
\definecolor{currentfill}{rgb}{0.000000,0.000000,0.000000}%
\pgfsetfillcolor{currentfill}%
\pgfsetlinewidth{1.003750pt}%
\definecolor{currentstroke}{rgb}{0.000000,0.000000,0.000000}%
\pgfsetstrokecolor{currentstroke}%
\pgfsetdash{}{0pt}%
\pgfpathmoveto{\pgfqpoint{5.504545in}{2.877687in}}%
\pgfpathcurveto{\pgfqpoint{5.515596in}{2.877687in}}{\pgfqpoint{5.526195in}{2.882077in}}{\pgfqpoint{5.534008in}{2.889891in}}%
\pgfpathcurveto{\pgfqpoint{5.541822in}{2.897704in}}{\pgfqpoint{5.546212in}{2.908303in}}{\pgfqpoint{5.546212in}{2.919353in}}%
\pgfpathcurveto{\pgfqpoint{5.546212in}{2.930404in}}{\pgfqpoint{5.541822in}{2.941003in}}{\pgfqpoint{5.534008in}{2.948816in}}%
\pgfpathcurveto{\pgfqpoint{5.526195in}{2.956630in}}{\pgfqpoint{5.515596in}{2.961020in}}{\pgfqpoint{5.504545in}{2.961020in}}%
\pgfpathcurveto{\pgfqpoint{5.493495in}{2.961020in}}{\pgfqpoint{5.482896in}{2.956630in}}{\pgfqpoint{5.475083in}{2.948816in}}%
\pgfpathcurveto{\pgfqpoint{5.467269in}{2.941003in}}{\pgfqpoint{5.462879in}{2.930404in}}{\pgfqpoint{5.462879in}{2.919353in}}%
\pgfpathcurveto{\pgfqpoint{5.462879in}{2.908303in}}{\pgfqpoint{5.467269in}{2.897704in}}{\pgfqpoint{5.475083in}{2.889891in}}%
\pgfpathcurveto{\pgfqpoint{5.482896in}{2.882077in}}{\pgfqpoint{5.493495in}{2.877687in}}{\pgfqpoint{5.504545in}{2.877687in}}%
\pgfpathclose%
\pgfusepath{stroke,fill}%
\end{pgfscope}%
\begin{pgfscope}%
\pgfpathrectangle{\pgfqpoint{0.800000in}{0.528000in}}{\pgfqpoint{4.960000in}{3.696000in}}%
\pgfusepath{clip}%
\pgfsetbuttcap%
\pgfsetroundjoin%
\definecolor{currentfill}{rgb}{0.000000,0.000000,0.000000}%
\pgfsetfillcolor{currentfill}%
\pgfsetlinewidth{1.003750pt}%
\definecolor{currentstroke}{rgb}{0.000000,0.000000,0.000000}%
\pgfsetstrokecolor{currentstroke}%
\pgfsetdash{}{0pt}%
\pgfpathmoveto{\pgfqpoint{5.504545in}{2.877687in}}%
\pgfpathcurveto{\pgfqpoint{5.515596in}{2.877687in}}{\pgfqpoint{5.526195in}{2.882077in}}{\pgfqpoint{5.534008in}{2.889891in}}%
\pgfpathcurveto{\pgfqpoint{5.541822in}{2.897704in}}{\pgfqpoint{5.546212in}{2.908303in}}{\pgfqpoint{5.546212in}{2.919353in}}%
\pgfpathcurveto{\pgfqpoint{5.546212in}{2.930404in}}{\pgfqpoint{5.541822in}{2.941003in}}{\pgfqpoint{5.534008in}{2.948816in}}%
\pgfpathcurveto{\pgfqpoint{5.526195in}{2.956630in}}{\pgfqpoint{5.515596in}{2.961020in}}{\pgfqpoint{5.504545in}{2.961020in}}%
\pgfpathcurveto{\pgfqpoint{5.493495in}{2.961020in}}{\pgfqpoint{5.482896in}{2.956630in}}{\pgfqpoint{5.475083in}{2.948816in}}%
\pgfpathcurveto{\pgfqpoint{5.467269in}{2.941003in}}{\pgfqpoint{5.462879in}{2.930404in}}{\pgfqpoint{5.462879in}{2.919353in}}%
\pgfpathcurveto{\pgfqpoint{5.462879in}{2.908303in}}{\pgfqpoint{5.467269in}{2.897704in}}{\pgfqpoint{5.475083in}{2.889891in}}%
\pgfpathcurveto{\pgfqpoint{5.482896in}{2.882077in}}{\pgfqpoint{5.493495in}{2.877687in}}{\pgfqpoint{5.504545in}{2.877687in}}%
\pgfpathclose%
\pgfusepath{stroke,fill}%
\end{pgfscope}%
\begin{pgfscope}%
\pgfpathrectangle{\pgfqpoint{0.800000in}{0.528000in}}{\pgfqpoint{4.960000in}{3.696000in}}%
\pgfusepath{clip}%
\pgfsetbuttcap%
\pgfsetroundjoin%
\definecolor{currentfill}{rgb}{0.000000,0.000000,0.000000}%
\pgfsetfillcolor{currentfill}%
\pgfsetlinewidth{1.003750pt}%
\definecolor{currentstroke}{rgb}{0.000000,0.000000,0.000000}%
\pgfsetstrokecolor{currentstroke}%
\pgfsetdash{}{0pt}%
\pgfpathmoveto{\pgfqpoint{5.504545in}{2.877687in}}%
\pgfpathcurveto{\pgfqpoint{5.515596in}{2.877687in}}{\pgfqpoint{5.526195in}{2.882077in}}{\pgfqpoint{5.534008in}{2.889891in}}%
\pgfpathcurveto{\pgfqpoint{5.541822in}{2.897704in}}{\pgfqpoint{5.546212in}{2.908303in}}{\pgfqpoint{5.546212in}{2.919353in}}%
\pgfpathcurveto{\pgfqpoint{5.546212in}{2.930404in}}{\pgfqpoint{5.541822in}{2.941003in}}{\pgfqpoint{5.534008in}{2.948816in}}%
\pgfpathcurveto{\pgfqpoint{5.526195in}{2.956630in}}{\pgfqpoint{5.515596in}{2.961020in}}{\pgfqpoint{5.504545in}{2.961020in}}%
\pgfpathcurveto{\pgfqpoint{5.493495in}{2.961020in}}{\pgfqpoint{5.482896in}{2.956630in}}{\pgfqpoint{5.475083in}{2.948816in}}%
\pgfpathcurveto{\pgfqpoint{5.467269in}{2.941003in}}{\pgfqpoint{5.462879in}{2.930404in}}{\pgfqpoint{5.462879in}{2.919353in}}%
\pgfpathcurveto{\pgfqpoint{5.462879in}{2.908303in}}{\pgfqpoint{5.467269in}{2.897704in}}{\pgfqpoint{5.475083in}{2.889891in}}%
\pgfpathcurveto{\pgfqpoint{5.482896in}{2.882077in}}{\pgfqpoint{5.493495in}{2.877687in}}{\pgfqpoint{5.504545in}{2.877687in}}%
\pgfpathclose%
\pgfusepath{stroke,fill}%
\end{pgfscope}%
\begin{pgfscope}%
\pgfpathrectangle{\pgfqpoint{0.800000in}{0.528000in}}{\pgfqpoint{4.960000in}{3.696000in}}%
\pgfusepath{clip}%
\pgfsetbuttcap%
\pgfsetroundjoin%
\definecolor{currentfill}{rgb}{0.000000,0.000000,0.000000}%
\pgfsetfillcolor{currentfill}%
\pgfsetlinewidth{1.003750pt}%
\definecolor{currentstroke}{rgb}{0.000000,0.000000,0.000000}%
\pgfsetstrokecolor{currentstroke}%
\pgfsetdash{}{0pt}%
\pgfpathmoveto{\pgfqpoint{5.504545in}{2.877687in}}%
\pgfpathcurveto{\pgfqpoint{5.515596in}{2.877687in}}{\pgfqpoint{5.526195in}{2.882077in}}{\pgfqpoint{5.534008in}{2.889891in}}%
\pgfpathcurveto{\pgfqpoint{5.541822in}{2.897704in}}{\pgfqpoint{5.546212in}{2.908303in}}{\pgfqpoint{5.546212in}{2.919353in}}%
\pgfpathcurveto{\pgfqpoint{5.546212in}{2.930404in}}{\pgfqpoint{5.541822in}{2.941003in}}{\pgfqpoint{5.534008in}{2.948816in}}%
\pgfpathcurveto{\pgfqpoint{5.526195in}{2.956630in}}{\pgfqpoint{5.515596in}{2.961020in}}{\pgfqpoint{5.504545in}{2.961020in}}%
\pgfpathcurveto{\pgfqpoint{5.493495in}{2.961020in}}{\pgfqpoint{5.482896in}{2.956630in}}{\pgfqpoint{5.475083in}{2.948816in}}%
\pgfpathcurveto{\pgfqpoint{5.467269in}{2.941003in}}{\pgfqpoint{5.462879in}{2.930404in}}{\pgfqpoint{5.462879in}{2.919353in}}%
\pgfpathcurveto{\pgfqpoint{5.462879in}{2.908303in}}{\pgfqpoint{5.467269in}{2.897704in}}{\pgfqpoint{5.475083in}{2.889891in}}%
\pgfpathcurveto{\pgfqpoint{5.482896in}{2.882077in}}{\pgfqpoint{5.493495in}{2.877687in}}{\pgfqpoint{5.504545in}{2.877687in}}%
\pgfpathclose%
\pgfusepath{stroke,fill}%
\end{pgfscope}%
\begin{pgfscope}%
\pgfpathrectangle{\pgfqpoint{0.800000in}{0.528000in}}{\pgfqpoint{4.960000in}{3.696000in}}%
\pgfusepath{clip}%
\pgfsetbuttcap%
\pgfsetroundjoin%
\definecolor{currentfill}{rgb}{0.000000,0.000000,0.000000}%
\pgfsetfillcolor{currentfill}%
\pgfsetlinewidth{1.003750pt}%
\definecolor{currentstroke}{rgb}{0.000000,0.000000,0.000000}%
\pgfsetstrokecolor{currentstroke}%
\pgfsetdash{}{0pt}%
\pgfpathmoveto{\pgfqpoint{5.504545in}{2.877687in}}%
\pgfpathcurveto{\pgfqpoint{5.515596in}{2.877687in}}{\pgfqpoint{5.526195in}{2.882077in}}{\pgfqpoint{5.534008in}{2.889891in}}%
\pgfpathcurveto{\pgfqpoint{5.541822in}{2.897704in}}{\pgfqpoint{5.546212in}{2.908303in}}{\pgfqpoint{5.546212in}{2.919353in}}%
\pgfpathcurveto{\pgfqpoint{5.546212in}{2.930404in}}{\pgfqpoint{5.541822in}{2.941003in}}{\pgfqpoint{5.534008in}{2.948816in}}%
\pgfpathcurveto{\pgfqpoint{5.526195in}{2.956630in}}{\pgfqpoint{5.515596in}{2.961020in}}{\pgfqpoint{5.504545in}{2.961020in}}%
\pgfpathcurveto{\pgfqpoint{5.493495in}{2.961020in}}{\pgfqpoint{5.482896in}{2.956630in}}{\pgfqpoint{5.475083in}{2.948816in}}%
\pgfpathcurveto{\pgfqpoint{5.467269in}{2.941003in}}{\pgfqpoint{5.462879in}{2.930404in}}{\pgfqpoint{5.462879in}{2.919353in}}%
\pgfpathcurveto{\pgfqpoint{5.462879in}{2.908303in}}{\pgfqpoint{5.467269in}{2.897704in}}{\pgfqpoint{5.475083in}{2.889891in}}%
\pgfpathcurveto{\pgfqpoint{5.482896in}{2.882077in}}{\pgfqpoint{5.493495in}{2.877687in}}{\pgfqpoint{5.504545in}{2.877687in}}%
\pgfpathclose%
\pgfusepath{stroke,fill}%
\end{pgfscope}%
\begin{pgfscope}%
\pgfpathrectangle{\pgfqpoint{0.800000in}{0.528000in}}{\pgfqpoint{4.960000in}{3.696000in}}%
\pgfusepath{clip}%
\pgfsetbuttcap%
\pgfsetroundjoin%
\definecolor{currentfill}{rgb}{0.000000,0.000000,0.000000}%
\pgfsetfillcolor{currentfill}%
\pgfsetlinewidth{1.003750pt}%
\definecolor{currentstroke}{rgb}{0.000000,0.000000,0.000000}%
\pgfsetstrokecolor{currentstroke}%
\pgfsetdash{}{0pt}%
\pgfpathmoveto{\pgfqpoint{5.504545in}{2.877687in}}%
\pgfpathcurveto{\pgfqpoint{5.515596in}{2.877687in}}{\pgfqpoint{5.526195in}{2.882077in}}{\pgfqpoint{5.534008in}{2.889891in}}%
\pgfpathcurveto{\pgfqpoint{5.541822in}{2.897704in}}{\pgfqpoint{5.546212in}{2.908303in}}{\pgfqpoint{5.546212in}{2.919353in}}%
\pgfpathcurveto{\pgfqpoint{5.546212in}{2.930404in}}{\pgfqpoint{5.541822in}{2.941003in}}{\pgfqpoint{5.534008in}{2.948816in}}%
\pgfpathcurveto{\pgfqpoint{5.526195in}{2.956630in}}{\pgfqpoint{5.515596in}{2.961020in}}{\pgfqpoint{5.504545in}{2.961020in}}%
\pgfpathcurveto{\pgfqpoint{5.493495in}{2.961020in}}{\pgfqpoint{5.482896in}{2.956630in}}{\pgfqpoint{5.475083in}{2.948816in}}%
\pgfpathcurveto{\pgfqpoint{5.467269in}{2.941003in}}{\pgfqpoint{5.462879in}{2.930404in}}{\pgfqpoint{5.462879in}{2.919353in}}%
\pgfpathcurveto{\pgfqpoint{5.462879in}{2.908303in}}{\pgfqpoint{5.467269in}{2.897704in}}{\pgfqpoint{5.475083in}{2.889891in}}%
\pgfpathcurveto{\pgfqpoint{5.482896in}{2.882077in}}{\pgfqpoint{5.493495in}{2.877687in}}{\pgfqpoint{5.504545in}{2.877687in}}%
\pgfpathclose%
\pgfusepath{stroke,fill}%
\end{pgfscope}%
\begin{pgfscope}%
\pgfpathrectangle{\pgfqpoint{0.800000in}{0.528000in}}{\pgfqpoint{4.960000in}{3.696000in}}%
\pgfusepath{clip}%
\pgfsetbuttcap%
\pgfsetroundjoin%
\definecolor{currentfill}{rgb}{0.000000,0.000000,0.000000}%
\pgfsetfillcolor{currentfill}%
\pgfsetlinewidth{1.003750pt}%
\definecolor{currentstroke}{rgb}{0.000000,0.000000,0.000000}%
\pgfsetstrokecolor{currentstroke}%
\pgfsetdash{}{0pt}%
\pgfpathmoveto{\pgfqpoint{5.504545in}{2.877687in}}%
\pgfpathcurveto{\pgfqpoint{5.515596in}{2.877687in}}{\pgfqpoint{5.526195in}{2.882077in}}{\pgfqpoint{5.534008in}{2.889891in}}%
\pgfpathcurveto{\pgfqpoint{5.541822in}{2.897704in}}{\pgfqpoint{5.546212in}{2.908303in}}{\pgfqpoint{5.546212in}{2.919353in}}%
\pgfpathcurveto{\pgfqpoint{5.546212in}{2.930404in}}{\pgfqpoint{5.541822in}{2.941003in}}{\pgfqpoint{5.534008in}{2.948816in}}%
\pgfpathcurveto{\pgfqpoint{5.526195in}{2.956630in}}{\pgfqpoint{5.515596in}{2.961020in}}{\pgfqpoint{5.504545in}{2.961020in}}%
\pgfpathcurveto{\pgfqpoint{5.493495in}{2.961020in}}{\pgfqpoint{5.482896in}{2.956630in}}{\pgfqpoint{5.475083in}{2.948816in}}%
\pgfpathcurveto{\pgfqpoint{5.467269in}{2.941003in}}{\pgfqpoint{5.462879in}{2.930404in}}{\pgfqpoint{5.462879in}{2.919353in}}%
\pgfpathcurveto{\pgfqpoint{5.462879in}{2.908303in}}{\pgfqpoint{5.467269in}{2.897704in}}{\pgfqpoint{5.475083in}{2.889891in}}%
\pgfpathcurveto{\pgfqpoint{5.482896in}{2.882077in}}{\pgfqpoint{5.493495in}{2.877687in}}{\pgfqpoint{5.504545in}{2.877687in}}%
\pgfpathclose%
\pgfusepath{stroke,fill}%
\end{pgfscope}%
\begin{pgfscope}%
\pgfpathrectangle{\pgfqpoint{0.800000in}{0.528000in}}{\pgfqpoint{4.960000in}{3.696000in}}%
\pgfusepath{clip}%
\pgfsetbuttcap%
\pgfsetroundjoin%
\definecolor{currentfill}{rgb}{0.000000,0.000000,0.000000}%
\pgfsetfillcolor{currentfill}%
\pgfsetlinewidth{1.003750pt}%
\definecolor{currentstroke}{rgb}{0.000000,0.000000,0.000000}%
\pgfsetstrokecolor{currentstroke}%
\pgfsetdash{}{0pt}%
\pgfpathmoveto{\pgfqpoint{5.504545in}{2.877687in}}%
\pgfpathcurveto{\pgfqpoint{5.515596in}{2.877687in}}{\pgfqpoint{5.526195in}{2.882077in}}{\pgfqpoint{5.534008in}{2.889891in}}%
\pgfpathcurveto{\pgfqpoint{5.541822in}{2.897704in}}{\pgfqpoint{5.546212in}{2.908303in}}{\pgfqpoint{5.546212in}{2.919353in}}%
\pgfpathcurveto{\pgfqpoint{5.546212in}{2.930404in}}{\pgfqpoint{5.541822in}{2.941003in}}{\pgfqpoint{5.534008in}{2.948816in}}%
\pgfpathcurveto{\pgfqpoint{5.526195in}{2.956630in}}{\pgfqpoint{5.515596in}{2.961020in}}{\pgfqpoint{5.504545in}{2.961020in}}%
\pgfpathcurveto{\pgfqpoint{5.493495in}{2.961020in}}{\pgfqpoint{5.482896in}{2.956630in}}{\pgfqpoint{5.475083in}{2.948816in}}%
\pgfpathcurveto{\pgfqpoint{5.467269in}{2.941003in}}{\pgfqpoint{5.462879in}{2.930404in}}{\pgfqpoint{5.462879in}{2.919353in}}%
\pgfpathcurveto{\pgfqpoint{5.462879in}{2.908303in}}{\pgfqpoint{5.467269in}{2.897704in}}{\pgfqpoint{5.475083in}{2.889891in}}%
\pgfpathcurveto{\pgfqpoint{5.482896in}{2.882077in}}{\pgfqpoint{5.493495in}{2.877687in}}{\pgfqpoint{5.504545in}{2.877687in}}%
\pgfpathclose%
\pgfusepath{stroke,fill}%
\end{pgfscope}%
\begin{pgfscope}%
\pgfpathrectangle{\pgfqpoint{0.800000in}{0.528000in}}{\pgfqpoint{4.960000in}{3.696000in}}%
\pgfusepath{clip}%
\pgfsetbuttcap%
\pgfsetroundjoin%
\definecolor{currentfill}{rgb}{0.000000,0.000000,0.000000}%
\pgfsetfillcolor{currentfill}%
\pgfsetlinewidth{1.003750pt}%
\definecolor{currentstroke}{rgb}{0.000000,0.000000,0.000000}%
\pgfsetstrokecolor{currentstroke}%
\pgfsetdash{}{0pt}%
\pgfpathmoveto{\pgfqpoint{5.504545in}{2.877687in}}%
\pgfpathcurveto{\pgfqpoint{5.515596in}{2.877687in}}{\pgfqpoint{5.526195in}{2.882077in}}{\pgfqpoint{5.534008in}{2.889891in}}%
\pgfpathcurveto{\pgfqpoint{5.541822in}{2.897704in}}{\pgfqpoint{5.546212in}{2.908303in}}{\pgfqpoint{5.546212in}{2.919353in}}%
\pgfpathcurveto{\pgfqpoint{5.546212in}{2.930404in}}{\pgfqpoint{5.541822in}{2.941003in}}{\pgfqpoint{5.534008in}{2.948816in}}%
\pgfpathcurveto{\pgfqpoint{5.526195in}{2.956630in}}{\pgfqpoint{5.515596in}{2.961020in}}{\pgfqpoint{5.504545in}{2.961020in}}%
\pgfpathcurveto{\pgfqpoint{5.493495in}{2.961020in}}{\pgfqpoint{5.482896in}{2.956630in}}{\pgfqpoint{5.475083in}{2.948816in}}%
\pgfpathcurveto{\pgfqpoint{5.467269in}{2.941003in}}{\pgfqpoint{5.462879in}{2.930404in}}{\pgfqpoint{5.462879in}{2.919353in}}%
\pgfpathcurveto{\pgfqpoint{5.462879in}{2.908303in}}{\pgfqpoint{5.467269in}{2.897704in}}{\pgfqpoint{5.475083in}{2.889891in}}%
\pgfpathcurveto{\pgfqpoint{5.482896in}{2.882077in}}{\pgfqpoint{5.493495in}{2.877687in}}{\pgfqpoint{5.504545in}{2.877687in}}%
\pgfpathclose%
\pgfusepath{stroke,fill}%
\end{pgfscope}%
\begin{pgfscope}%
\pgfpathrectangle{\pgfqpoint{0.800000in}{0.528000in}}{\pgfqpoint{4.960000in}{3.696000in}}%
\pgfusepath{clip}%
\pgfsetbuttcap%
\pgfsetroundjoin%
\definecolor{currentfill}{rgb}{0.000000,0.000000,0.000000}%
\pgfsetfillcolor{currentfill}%
\pgfsetlinewidth{1.003750pt}%
\definecolor{currentstroke}{rgb}{0.000000,0.000000,0.000000}%
\pgfsetstrokecolor{currentstroke}%
\pgfsetdash{}{0pt}%
\pgfpathmoveto{\pgfqpoint{5.504545in}{2.877687in}}%
\pgfpathcurveto{\pgfqpoint{5.515596in}{2.877687in}}{\pgfqpoint{5.526195in}{2.882077in}}{\pgfqpoint{5.534008in}{2.889891in}}%
\pgfpathcurveto{\pgfqpoint{5.541822in}{2.897704in}}{\pgfqpoint{5.546212in}{2.908303in}}{\pgfqpoint{5.546212in}{2.919353in}}%
\pgfpathcurveto{\pgfqpoint{5.546212in}{2.930404in}}{\pgfqpoint{5.541822in}{2.941003in}}{\pgfqpoint{5.534008in}{2.948816in}}%
\pgfpathcurveto{\pgfqpoint{5.526195in}{2.956630in}}{\pgfqpoint{5.515596in}{2.961020in}}{\pgfqpoint{5.504545in}{2.961020in}}%
\pgfpathcurveto{\pgfqpoint{5.493495in}{2.961020in}}{\pgfqpoint{5.482896in}{2.956630in}}{\pgfqpoint{5.475083in}{2.948816in}}%
\pgfpathcurveto{\pgfqpoint{5.467269in}{2.941003in}}{\pgfqpoint{5.462879in}{2.930404in}}{\pgfqpoint{5.462879in}{2.919353in}}%
\pgfpathcurveto{\pgfqpoint{5.462879in}{2.908303in}}{\pgfqpoint{5.467269in}{2.897704in}}{\pgfqpoint{5.475083in}{2.889891in}}%
\pgfpathcurveto{\pgfqpoint{5.482896in}{2.882077in}}{\pgfqpoint{5.493495in}{2.877687in}}{\pgfqpoint{5.504545in}{2.877687in}}%
\pgfpathclose%
\pgfusepath{stroke,fill}%
\end{pgfscope}%
\begin{pgfscope}%
\pgfpathrectangle{\pgfqpoint{0.800000in}{0.528000in}}{\pgfqpoint{4.960000in}{3.696000in}}%
\pgfusepath{clip}%
\pgfsetbuttcap%
\pgfsetroundjoin%
\definecolor{currentfill}{rgb}{0.000000,0.000000,0.000000}%
\pgfsetfillcolor{currentfill}%
\pgfsetlinewidth{1.003750pt}%
\definecolor{currentstroke}{rgb}{0.000000,0.000000,0.000000}%
\pgfsetstrokecolor{currentstroke}%
\pgfsetdash{}{0pt}%
\pgfpathmoveto{\pgfqpoint{5.504545in}{2.877687in}}%
\pgfpathcurveto{\pgfqpoint{5.515596in}{2.877687in}}{\pgfqpoint{5.526195in}{2.882077in}}{\pgfqpoint{5.534008in}{2.889891in}}%
\pgfpathcurveto{\pgfqpoint{5.541822in}{2.897704in}}{\pgfqpoint{5.546212in}{2.908303in}}{\pgfqpoint{5.546212in}{2.919353in}}%
\pgfpathcurveto{\pgfqpoint{5.546212in}{2.930404in}}{\pgfqpoint{5.541822in}{2.941003in}}{\pgfqpoint{5.534008in}{2.948816in}}%
\pgfpathcurveto{\pgfqpoint{5.526195in}{2.956630in}}{\pgfqpoint{5.515596in}{2.961020in}}{\pgfqpoint{5.504545in}{2.961020in}}%
\pgfpathcurveto{\pgfqpoint{5.493495in}{2.961020in}}{\pgfqpoint{5.482896in}{2.956630in}}{\pgfqpoint{5.475083in}{2.948816in}}%
\pgfpathcurveto{\pgfqpoint{5.467269in}{2.941003in}}{\pgfqpoint{5.462879in}{2.930404in}}{\pgfqpoint{5.462879in}{2.919353in}}%
\pgfpathcurveto{\pgfqpoint{5.462879in}{2.908303in}}{\pgfqpoint{5.467269in}{2.897704in}}{\pgfqpoint{5.475083in}{2.889891in}}%
\pgfpathcurveto{\pgfqpoint{5.482896in}{2.882077in}}{\pgfqpoint{5.493495in}{2.877687in}}{\pgfqpoint{5.504545in}{2.877687in}}%
\pgfpathclose%
\pgfusepath{stroke,fill}%
\end{pgfscope}%
\begin{pgfscope}%
\pgfpathrectangle{\pgfqpoint{0.800000in}{0.528000in}}{\pgfqpoint{4.960000in}{3.696000in}}%
\pgfusepath{clip}%
\pgfsetbuttcap%
\pgfsetroundjoin%
\definecolor{currentfill}{rgb}{0.000000,0.000000,0.000000}%
\pgfsetfillcolor{currentfill}%
\pgfsetlinewidth{1.003750pt}%
\definecolor{currentstroke}{rgb}{0.000000,0.000000,0.000000}%
\pgfsetstrokecolor{currentstroke}%
\pgfsetdash{}{0pt}%
\pgfpathmoveto{\pgfqpoint{5.504545in}{2.877687in}}%
\pgfpathcurveto{\pgfqpoint{5.515596in}{2.877687in}}{\pgfqpoint{5.526195in}{2.882077in}}{\pgfqpoint{5.534008in}{2.889891in}}%
\pgfpathcurveto{\pgfqpoint{5.541822in}{2.897704in}}{\pgfqpoint{5.546212in}{2.908303in}}{\pgfqpoint{5.546212in}{2.919353in}}%
\pgfpathcurveto{\pgfqpoint{5.546212in}{2.930404in}}{\pgfqpoint{5.541822in}{2.941003in}}{\pgfqpoint{5.534008in}{2.948816in}}%
\pgfpathcurveto{\pgfqpoint{5.526195in}{2.956630in}}{\pgfqpoint{5.515596in}{2.961020in}}{\pgfqpoint{5.504545in}{2.961020in}}%
\pgfpathcurveto{\pgfqpoint{5.493495in}{2.961020in}}{\pgfqpoint{5.482896in}{2.956630in}}{\pgfqpoint{5.475083in}{2.948816in}}%
\pgfpathcurveto{\pgfqpoint{5.467269in}{2.941003in}}{\pgfqpoint{5.462879in}{2.930404in}}{\pgfqpoint{5.462879in}{2.919353in}}%
\pgfpathcurveto{\pgfqpoint{5.462879in}{2.908303in}}{\pgfqpoint{5.467269in}{2.897704in}}{\pgfqpoint{5.475083in}{2.889891in}}%
\pgfpathcurveto{\pgfqpoint{5.482896in}{2.882077in}}{\pgfqpoint{5.493495in}{2.877687in}}{\pgfqpoint{5.504545in}{2.877687in}}%
\pgfpathclose%
\pgfusepath{stroke,fill}%
\end{pgfscope}%
\begin{pgfscope}%
\pgfpathrectangle{\pgfqpoint{0.800000in}{0.528000in}}{\pgfqpoint{4.960000in}{3.696000in}}%
\pgfusepath{clip}%
\pgfsetbuttcap%
\pgfsetroundjoin%
\definecolor{currentfill}{rgb}{0.000000,0.000000,0.000000}%
\pgfsetfillcolor{currentfill}%
\pgfsetlinewidth{1.003750pt}%
\definecolor{currentstroke}{rgb}{0.000000,0.000000,0.000000}%
\pgfsetstrokecolor{currentstroke}%
\pgfsetdash{}{0pt}%
\pgfpathmoveto{\pgfqpoint{5.504545in}{2.877687in}}%
\pgfpathcurveto{\pgfqpoint{5.515596in}{2.877687in}}{\pgfqpoint{5.526195in}{2.882077in}}{\pgfqpoint{5.534008in}{2.889891in}}%
\pgfpathcurveto{\pgfqpoint{5.541822in}{2.897704in}}{\pgfqpoint{5.546212in}{2.908303in}}{\pgfqpoint{5.546212in}{2.919353in}}%
\pgfpathcurveto{\pgfqpoint{5.546212in}{2.930404in}}{\pgfqpoint{5.541822in}{2.941003in}}{\pgfqpoint{5.534008in}{2.948816in}}%
\pgfpathcurveto{\pgfqpoint{5.526195in}{2.956630in}}{\pgfqpoint{5.515596in}{2.961020in}}{\pgfqpoint{5.504545in}{2.961020in}}%
\pgfpathcurveto{\pgfqpoint{5.493495in}{2.961020in}}{\pgfqpoint{5.482896in}{2.956630in}}{\pgfqpoint{5.475083in}{2.948816in}}%
\pgfpathcurveto{\pgfqpoint{5.467269in}{2.941003in}}{\pgfqpoint{5.462879in}{2.930404in}}{\pgfqpoint{5.462879in}{2.919353in}}%
\pgfpathcurveto{\pgfqpoint{5.462879in}{2.908303in}}{\pgfqpoint{5.467269in}{2.897704in}}{\pgfqpoint{5.475083in}{2.889891in}}%
\pgfpathcurveto{\pgfqpoint{5.482896in}{2.882077in}}{\pgfqpoint{5.493495in}{2.877687in}}{\pgfqpoint{5.504545in}{2.877687in}}%
\pgfpathclose%
\pgfusepath{stroke,fill}%
\end{pgfscope}%
\begin{pgfscope}%
\pgfpathrectangle{\pgfqpoint{0.800000in}{0.528000in}}{\pgfqpoint{4.960000in}{3.696000in}}%
\pgfusepath{clip}%
\pgfsetbuttcap%
\pgfsetroundjoin%
\definecolor{currentfill}{rgb}{0.000000,0.000000,0.000000}%
\pgfsetfillcolor{currentfill}%
\pgfsetlinewidth{1.003750pt}%
\definecolor{currentstroke}{rgb}{0.000000,0.000000,0.000000}%
\pgfsetstrokecolor{currentstroke}%
\pgfsetdash{}{0pt}%
\pgfpathmoveto{\pgfqpoint{5.504545in}{2.877687in}}%
\pgfpathcurveto{\pgfqpoint{5.515596in}{2.877687in}}{\pgfqpoint{5.526195in}{2.882077in}}{\pgfqpoint{5.534008in}{2.889891in}}%
\pgfpathcurveto{\pgfqpoint{5.541822in}{2.897704in}}{\pgfqpoint{5.546212in}{2.908303in}}{\pgfqpoint{5.546212in}{2.919353in}}%
\pgfpathcurveto{\pgfqpoint{5.546212in}{2.930404in}}{\pgfqpoint{5.541822in}{2.941003in}}{\pgfqpoint{5.534008in}{2.948816in}}%
\pgfpathcurveto{\pgfqpoint{5.526195in}{2.956630in}}{\pgfqpoint{5.515596in}{2.961020in}}{\pgfqpoint{5.504545in}{2.961020in}}%
\pgfpathcurveto{\pgfqpoint{5.493495in}{2.961020in}}{\pgfqpoint{5.482896in}{2.956630in}}{\pgfqpoint{5.475083in}{2.948816in}}%
\pgfpathcurveto{\pgfqpoint{5.467269in}{2.941003in}}{\pgfqpoint{5.462879in}{2.930404in}}{\pgfqpoint{5.462879in}{2.919353in}}%
\pgfpathcurveto{\pgfqpoint{5.462879in}{2.908303in}}{\pgfqpoint{5.467269in}{2.897704in}}{\pgfqpoint{5.475083in}{2.889891in}}%
\pgfpathcurveto{\pgfqpoint{5.482896in}{2.882077in}}{\pgfqpoint{5.493495in}{2.877687in}}{\pgfqpoint{5.504545in}{2.877687in}}%
\pgfpathclose%
\pgfusepath{stroke,fill}%
\end{pgfscope}%
\begin{pgfscope}%
\pgfpathrectangle{\pgfqpoint{0.800000in}{0.528000in}}{\pgfqpoint{4.960000in}{3.696000in}}%
\pgfusepath{clip}%
\pgfsetbuttcap%
\pgfsetroundjoin%
\definecolor{currentfill}{rgb}{0.000000,0.000000,0.000000}%
\pgfsetfillcolor{currentfill}%
\pgfsetlinewidth{1.003750pt}%
\definecolor{currentstroke}{rgb}{0.000000,0.000000,0.000000}%
\pgfsetstrokecolor{currentstroke}%
\pgfsetdash{}{0pt}%
\pgfpathmoveto{\pgfqpoint{5.504545in}{2.877687in}}%
\pgfpathcurveto{\pgfqpoint{5.515596in}{2.877687in}}{\pgfqpoint{5.526195in}{2.882077in}}{\pgfqpoint{5.534008in}{2.889891in}}%
\pgfpathcurveto{\pgfqpoint{5.541822in}{2.897704in}}{\pgfqpoint{5.546212in}{2.908303in}}{\pgfqpoint{5.546212in}{2.919353in}}%
\pgfpathcurveto{\pgfqpoint{5.546212in}{2.930404in}}{\pgfqpoint{5.541822in}{2.941003in}}{\pgfqpoint{5.534008in}{2.948816in}}%
\pgfpathcurveto{\pgfqpoint{5.526195in}{2.956630in}}{\pgfqpoint{5.515596in}{2.961020in}}{\pgfqpoint{5.504545in}{2.961020in}}%
\pgfpathcurveto{\pgfqpoint{5.493495in}{2.961020in}}{\pgfqpoint{5.482896in}{2.956630in}}{\pgfqpoint{5.475083in}{2.948816in}}%
\pgfpathcurveto{\pgfqpoint{5.467269in}{2.941003in}}{\pgfqpoint{5.462879in}{2.930404in}}{\pgfqpoint{5.462879in}{2.919353in}}%
\pgfpathcurveto{\pgfqpoint{5.462879in}{2.908303in}}{\pgfqpoint{5.467269in}{2.897704in}}{\pgfqpoint{5.475083in}{2.889891in}}%
\pgfpathcurveto{\pgfqpoint{5.482896in}{2.882077in}}{\pgfqpoint{5.493495in}{2.877687in}}{\pgfqpoint{5.504545in}{2.877687in}}%
\pgfpathclose%
\pgfusepath{stroke,fill}%
\end{pgfscope}%
\begin{pgfscope}%
\pgfpathrectangle{\pgfqpoint{0.800000in}{0.528000in}}{\pgfqpoint{4.960000in}{3.696000in}}%
\pgfusepath{clip}%
\pgfsetbuttcap%
\pgfsetroundjoin%
\definecolor{currentfill}{rgb}{0.000000,0.000000,0.000000}%
\pgfsetfillcolor{currentfill}%
\pgfsetlinewidth{1.003750pt}%
\definecolor{currentstroke}{rgb}{0.000000,0.000000,0.000000}%
\pgfsetstrokecolor{currentstroke}%
\pgfsetdash{}{0pt}%
\pgfpathmoveto{\pgfqpoint{5.504545in}{2.877687in}}%
\pgfpathcurveto{\pgfqpoint{5.515596in}{2.877687in}}{\pgfqpoint{5.526195in}{2.882077in}}{\pgfqpoint{5.534008in}{2.889891in}}%
\pgfpathcurveto{\pgfqpoint{5.541822in}{2.897704in}}{\pgfqpoint{5.546212in}{2.908303in}}{\pgfqpoint{5.546212in}{2.919353in}}%
\pgfpathcurveto{\pgfqpoint{5.546212in}{2.930404in}}{\pgfqpoint{5.541822in}{2.941003in}}{\pgfqpoint{5.534008in}{2.948816in}}%
\pgfpathcurveto{\pgfqpoint{5.526195in}{2.956630in}}{\pgfqpoint{5.515596in}{2.961020in}}{\pgfqpoint{5.504545in}{2.961020in}}%
\pgfpathcurveto{\pgfqpoint{5.493495in}{2.961020in}}{\pgfqpoint{5.482896in}{2.956630in}}{\pgfqpoint{5.475083in}{2.948816in}}%
\pgfpathcurveto{\pgfqpoint{5.467269in}{2.941003in}}{\pgfqpoint{5.462879in}{2.930404in}}{\pgfqpoint{5.462879in}{2.919353in}}%
\pgfpathcurveto{\pgfqpoint{5.462879in}{2.908303in}}{\pgfqpoint{5.467269in}{2.897704in}}{\pgfqpoint{5.475083in}{2.889891in}}%
\pgfpathcurveto{\pgfqpoint{5.482896in}{2.882077in}}{\pgfqpoint{5.493495in}{2.877687in}}{\pgfqpoint{5.504545in}{2.877687in}}%
\pgfpathclose%
\pgfusepath{stroke,fill}%
\end{pgfscope}%
\begin{pgfscope}%
\pgfpathrectangle{\pgfqpoint{0.800000in}{0.528000in}}{\pgfqpoint{4.960000in}{3.696000in}}%
\pgfusepath{clip}%
\pgfsetbuttcap%
\pgfsetroundjoin%
\definecolor{currentfill}{rgb}{0.000000,0.000000,0.000000}%
\pgfsetfillcolor{currentfill}%
\pgfsetlinewidth{1.003750pt}%
\definecolor{currentstroke}{rgb}{0.000000,0.000000,0.000000}%
\pgfsetstrokecolor{currentstroke}%
\pgfsetdash{}{0pt}%
\pgfpathmoveto{\pgfqpoint{5.504545in}{2.877687in}}%
\pgfpathcurveto{\pgfqpoint{5.515596in}{2.877687in}}{\pgfqpoint{5.526195in}{2.882077in}}{\pgfqpoint{5.534008in}{2.889891in}}%
\pgfpathcurveto{\pgfqpoint{5.541822in}{2.897704in}}{\pgfqpoint{5.546212in}{2.908303in}}{\pgfqpoint{5.546212in}{2.919353in}}%
\pgfpathcurveto{\pgfqpoint{5.546212in}{2.930404in}}{\pgfqpoint{5.541822in}{2.941003in}}{\pgfqpoint{5.534008in}{2.948816in}}%
\pgfpathcurveto{\pgfqpoint{5.526195in}{2.956630in}}{\pgfqpoint{5.515596in}{2.961020in}}{\pgfqpoint{5.504545in}{2.961020in}}%
\pgfpathcurveto{\pgfqpoint{5.493495in}{2.961020in}}{\pgfqpoint{5.482896in}{2.956630in}}{\pgfqpoint{5.475083in}{2.948816in}}%
\pgfpathcurveto{\pgfqpoint{5.467269in}{2.941003in}}{\pgfqpoint{5.462879in}{2.930404in}}{\pgfqpoint{5.462879in}{2.919353in}}%
\pgfpathcurveto{\pgfqpoint{5.462879in}{2.908303in}}{\pgfqpoint{5.467269in}{2.897704in}}{\pgfqpoint{5.475083in}{2.889891in}}%
\pgfpathcurveto{\pgfqpoint{5.482896in}{2.882077in}}{\pgfqpoint{5.493495in}{2.877687in}}{\pgfqpoint{5.504545in}{2.877687in}}%
\pgfpathclose%
\pgfusepath{stroke,fill}%
\end{pgfscope}%
\begin{pgfscope}%
\pgfpathrectangle{\pgfqpoint{0.800000in}{0.528000in}}{\pgfqpoint{4.960000in}{3.696000in}}%
\pgfusepath{clip}%
\pgfsetbuttcap%
\pgfsetroundjoin%
\definecolor{currentfill}{rgb}{0.000000,0.000000,0.000000}%
\pgfsetfillcolor{currentfill}%
\pgfsetlinewidth{1.003750pt}%
\definecolor{currentstroke}{rgb}{0.000000,0.000000,0.000000}%
\pgfsetstrokecolor{currentstroke}%
\pgfsetdash{}{0pt}%
\pgfpathmoveto{\pgfqpoint{5.504545in}{3.984333in}}%
\pgfpathcurveto{\pgfqpoint{5.515596in}{3.984333in}}{\pgfqpoint{5.526195in}{3.988724in}}{\pgfqpoint{5.534008in}{3.996537in}}%
\pgfpathcurveto{\pgfqpoint{5.541822in}{4.004351in}}{\pgfqpoint{5.546212in}{4.014950in}}{\pgfqpoint{5.546212in}{4.026000in}}%
\pgfpathcurveto{\pgfqpoint{5.546212in}{4.037050in}}{\pgfqpoint{5.541822in}{4.047649in}}{\pgfqpoint{5.534008in}{4.055463in}}%
\pgfpathcurveto{\pgfqpoint{5.526195in}{4.063276in}}{\pgfqpoint{5.515596in}{4.067667in}}{\pgfqpoint{5.504545in}{4.067667in}}%
\pgfpathcurveto{\pgfqpoint{5.493495in}{4.067667in}}{\pgfqpoint{5.482896in}{4.063276in}}{\pgfqpoint{5.475083in}{4.055463in}}%
\pgfpathcurveto{\pgfqpoint{5.467269in}{4.047649in}}{\pgfqpoint{5.462879in}{4.037050in}}{\pgfqpoint{5.462879in}{4.026000in}}%
\pgfpathcurveto{\pgfqpoint{5.462879in}{4.014950in}}{\pgfqpoint{5.467269in}{4.004351in}}{\pgfqpoint{5.475083in}{3.996537in}}%
\pgfpathcurveto{\pgfqpoint{5.482896in}{3.988724in}}{\pgfqpoint{5.493495in}{3.984333in}}{\pgfqpoint{5.504545in}{3.984333in}}%
\pgfpathclose%
\pgfusepath{stroke,fill}%
\end{pgfscope}%
\begin{pgfscope}%
\pgfpathrectangle{\pgfqpoint{0.800000in}{0.528000in}}{\pgfqpoint{4.960000in}{3.696000in}}%
\pgfusepath{clip}%
\pgfsetbuttcap%
\pgfsetroundjoin%
\definecolor{currentfill}{rgb}{0.000000,0.000000,0.000000}%
\pgfsetfillcolor{currentfill}%
\pgfsetlinewidth{1.003750pt}%
\definecolor{currentstroke}{rgb}{0.000000,0.000000,0.000000}%
\pgfsetstrokecolor{currentstroke}%
\pgfsetdash{}{0pt}%
\pgfpathmoveto{\pgfqpoint{5.504545in}{2.877687in}}%
\pgfpathcurveto{\pgfqpoint{5.515596in}{2.877687in}}{\pgfqpoint{5.526195in}{2.882077in}}{\pgfqpoint{5.534008in}{2.889891in}}%
\pgfpathcurveto{\pgfqpoint{5.541822in}{2.897704in}}{\pgfqpoint{5.546212in}{2.908303in}}{\pgfqpoint{5.546212in}{2.919353in}}%
\pgfpathcurveto{\pgfqpoint{5.546212in}{2.930404in}}{\pgfqpoint{5.541822in}{2.941003in}}{\pgfqpoint{5.534008in}{2.948816in}}%
\pgfpathcurveto{\pgfqpoint{5.526195in}{2.956630in}}{\pgfqpoint{5.515596in}{2.961020in}}{\pgfqpoint{5.504545in}{2.961020in}}%
\pgfpathcurveto{\pgfqpoint{5.493495in}{2.961020in}}{\pgfqpoint{5.482896in}{2.956630in}}{\pgfqpoint{5.475083in}{2.948816in}}%
\pgfpathcurveto{\pgfqpoint{5.467269in}{2.941003in}}{\pgfqpoint{5.462879in}{2.930404in}}{\pgfqpoint{5.462879in}{2.919353in}}%
\pgfpathcurveto{\pgfqpoint{5.462879in}{2.908303in}}{\pgfqpoint{5.467269in}{2.897704in}}{\pgfqpoint{5.475083in}{2.889891in}}%
\pgfpathcurveto{\pgfqpoint{5.482896in}{2.882077in}}{\pgfqpoint{5.493495in}{2.877687in}}{\pgfqpoint{5.504545in}{2.877687in}}%
\pgfpathclose%
\pgfusepath{stroke,fill}%
\end{pgfscope}%
\begin{pgfscope}%
\pgfpathrectangle{\pgfqpoint{0.800000in}{0.528000in}}{\pgfqpoint{4.960000in}{3.696000in}}%
\pgfusepath{clip}%
\pgfsetbuttcap%
\pgfsetroundjoin%
\definecolor{currentfill}{rgb}{0.000000,0.000000,0.000000}%
\pgfsetfillcolor{currentfill}%
\pgfsetlinewidth{1.003750pt}%
\definecolor{currentstroke}{rgb}{0.000000,0.000000,0.000000}%
\pgfsetstrokecolor{currentstroke}%
\pgfsetdash{}{0pt}%
\pgfpathmoveto{\pgfqpoint{5.504545in}{2.877687in}}%
\pgfpathcurveto{\pgfqpoint{5.515596in}{2.877687in}}{\pgfqpoint{5.526195in}{2.882077in}}{\pgfqpoint{5.534008in}{2.889891in}}%
\pgfpathcurveto{\pgfqpoint{5.541822in}{2.897704in}}{\pgfqpoint{5.546212in}{2.908303in}}{\pgfqpoint{5.546212in}{2.919353in}}%
\pgfpathcurveto{\pgfqpoint{5.546212in}{2.930404in}}{\pgfqpoint{5.541822in}{2.941003in}}{\pgfqpoint{5.534008in}{2.948816in}}%
\pgfpathcurveto{\pgfqpoint{5.526195in}{2.956630in}}{\pgfqpoint{5.515596in}{2.961020in}}{\pgfqpoint{5.504545in}{2.961020in}}%
\pgfpathcurveto{\pgfqpoint{5.493495in}{2.961020in}}{\pgfqpoint{5.482896in}{2.956630in}}{\pgfqpoint{5.475083in}{2.948816in}}%
\pgfpathcurveto{\pgfqpoint{5.467269in}{2.941003in}}{\pgfqpoint{5.462879in}{2.930404in}}{\pgfqpoint{5.462879in}{2.919353in}}%
\pgfpathcurveto{\pgfqpoint{5.462879in}{2.908303in}}{\pgfqpoint{5.467269in}{2.897704in}}{\pgfqpoint{5.475083in}{2.889891in}}%
\pgfpathcurveto{\pgfqpoint{5.482896in}{2.882077in}}{\pgfqpoint{5.493495in}{2.877687in}}{\pgfqpoint{5.504545in}{2.877687in}}%
\pgfpathclose%
\pgfusepath{stroke,fill}%
\end{pgfscope}%
\begin{pgfscope}%
\pgfpathrectangle{\pgfqpoint{0.800000in}{0.528000in}}{\pgfqpoint{4.960000in}{3.696000in}}%
\pgfusepath{clip}%
\pgfsetbuttcap%
\pgfsetroundjoin%
\definecolor{currentfill}{rgb}{0.000000,0.000000,0.000000}%
\pgfsetfillcolor{currentfill}%
\pgfsetlinewidth{1.003750pt}%
\definecolor{currentstroke}{rgb}{0.000000,0.000000,0.000000}%
\pgfsetstrokecolor{currentstroke}%
\pgfsetdash{}{0pt}%
\pgfpathmoveto{\pgfqpoint{5.504545in}{2.877687in}}%
\pgfpathcurveto{\pgfqpoint{5.515596in}{2.877687in}}{\pgfqpoint{5.526195in}{2.882077in}}{\pgfqpoint{5.534008in}{2.889891in}}%
\pgfpathcurveto{\pgfqpoint{5.541822in}{2.897704in}}{\pgfqpoint{5.546212in}{2.908303in}}{\pgfqpoint{5.546212in}{2.919353in}}%
\pgfpathcurveto{\pgfqpoint{5.546212in}{2.930404in}}{\pgfqpoint{5.541822in}{2.941003in}}{\pgfqpoint{5.534008in}{2.948816in}}%
\pgfpathcurveto{\pgfqpoint{5.526195in}{2.956630in}}{\pgfqpoint{5.515596in}{2.961020in}}{\pgfqpoint{5.504545in}{2.961020in}}%
\pgfpathcurveto{\pgfqpoint{5.493495in}{2.961020in}}{\pgfqpoint{5.482896in}{2.956630in}}{\pgfqpoint{5.475083in}{2.948816in}}%
\pgfpathcurveto{\pgfqpoint{5.467269in}{2.941003in}}{\pgfqpoint{5.462879in}{2.930404in}}{\pgfqpoint{5.462879in}{2.919353in}}%
\pgfpathcurveto{\pgfqpoint{5.462879in}{2.908303in}}{\pgfqpoint{5.467269in}{2.897704in}}{\pgfqpoint{5.475083in}{2.889891in}}%
\pgfpathcurveto{\pgfqpoint{5.482896in}{2.882077in}}{\pgfqpoint{5.493495in}{2.877687in}}{\pgfqpoint{5.504545in}{2.877687in}}%
\pgfpathclose%
\pgfusepath{stroke,fill}%
\end{pgfscope}%
\begin{pgfscope}%
\pgfpathrectangle{\pgfqpoint{0.800000in}{0.528000in}}{\pgfqpoint{4.960000in}{3.696000in}}%
\pgfusepath{clip}%
\pgfsetbuttcap%
\pgfsetroundjoin%
\definecolor{currentfill}{rgb}{0.000000,0.000000,0.000000}%
\pgfsetfillcolor{currentfill}%
\pgfsetlinewidth{1.003750pt}%
\definecolor{currentstroke}{rgb}{0.000000,0.000000,0.000000}%
\pgfsetstrokecolor{currentstroke}%
\pgfsetdash{}{0pt}%
\pgfpathmoveto{\pgfqpoint{5.504545in}{2.877687in}}%
\pgfpathcurveto{\pgfqpoint{5.515596in}{2.877687in}}{\pgfqpoint{5.526195in}{2.882077in}}{\pgfqpoint{5.534008in}{2.889891in}}%
\pgfpathcurveto{\pgfqpoint{5.541822in}{2.897704in}}{\pgfqpoint{5.546212in}{2.908303in}}{\pgfqpoint{5.546212in}{2.919353in}}%
\pgfpathcurveto{\pgfqpoint{5.546212in}{2.930404in}}{\pgfqpoint{5.541822in}{2.941003in}}{\pgfqpoint{5.534008in}{2.948816in}}%
\pgfpathcurveto{\pgfqpoint{5.526195in}{2.956630in}}{\pgfqpoint{5.515596in}{2.961020in}}{\pgfqpoint{5.504545in}{2.961020in}}%
\pgfpathcurveto{\pgfqpoint{5.493495in}{2.961020in}}{\pgfqpoint{5.482896in}{2.956630in}}{\pgfqpoint{5.475083in}{2.948816in}}%
\pgfpathcurveto{\pgfqpoint{5.467269in}{2.941003in}}{\pgfqpoint{5.462879in}{2.930404in}}{\pgfqpoint{5.462879in}{2.919353in}}%
\pgfpathcurveto{\pgfqpoint{5.462879in}{2.908303in}}{\pgfqpoint{5.467269in}{2.897704in}}{\pgfqpoint{5.475083in}{2.889891in}}%
\pgfpathcurveto{\pgfqpoint{5.482896in}{2.882077in}}{\pgfqpoint{5.493495in}{2.877687in}}{\pgfqpoint{5.504545in}{2.877687in}}%
\pgfpathclose%
\pgfusepath{stroke,fill}%
\end{pgfscope}%
\begin{pgfscope}%
\pgfpathrectangle{\pgfqpoint{0.800000in}{0.528000in}}{\pgfqpoint{4.960000in}{3.696000in}}%
\pgfusepath{clip}%
\pgfsetbuttcap%
\pgfsetroundjoin%
\definecolor{currentfill}{rgb}{0.000000,0.000000,0.000000}%
\pgfsetfillcolor{currentfill}%
\pgfsetlinewidth{1.003750pt}%
\definecolor{currentstroke}{rgb}{0.000000,0.000000,0.000000}%
\pgfsetstrokecolor{currentstroke}%
\pgfsetdash{}{0pt}%
\pgfpathmoveto{\pgfqpoint{5.504545in}{2.877687in}}%
\pgfpathcurveto{\pgfqpoint{5.515596in}{2.877687in}}{\pgfqpoint{5.526195in}{2.882077in}}{\pgfqpoint{5.534008in}{2.889891in}}%
\pgfpathcurveto{\pgfqpoint{5.541822in}{2.897704in}}{\pgfqpoint{5.546212in}{2.908303in}}{\pgfqpoint{5.546212in}{2.919353in}}%
\pgfpathcurveto{\pgfqpoint{5.546212in}{2.930404in}}{\pgfqpoint{5.541822in}{2.941003in}}{\pgfqpoint{5.534008in}{2.948816in}}%
\pgfpathcurveto{\pgfqpoint{5.526195in}{2.956630in}}{\pgfqpoint{5.515596in}{2.961020in}}{\pgfqpoint{5.504545in}{2.961020in}}%
\pgfpathcurveto{\pgfqpoint{5.493495in}{2.961020in}}{\pgfqpoint{5.482896in}{2.956630in}}{\pgfqpoint{5.475083in}{2.948816in}}%
\pgfpathcurveto{\pgfqpoint{5.467269in}{2.941003in}}{\pgfqpoint{5.462879in}{2.930404in}}{\pgfqpoint{5.462879in}{2.919353in}}%
\pgfpathcurveto{\pgfqpoint{5.462879in}{2.908303in}}{\pgfqpoint{5.467269in}{2.897704in}}{\pgfqpoint{5.475083in}{2.889891in}}%
\pgfpathcurveto{\pgfqpoint{5.482896in}{2.882077in}}{\pgfqpoint{5.493495in}{2.877687in}}{\pgfqpoint{5.504545in}{2.877687in}}%
\pgfpathclose%
\pgfusepath{stroke,fill}%
\end{pgfscope}%
\begin{pgfscope}%
\pgfpathrectangle{\pgfqpoint{0.800000in}{0.528000in}}{\pgfqpoint{4.960000in}{3.696000in}}%
\pgfusepath{clip}%
\pgfsetbuttcap%
\pgfsetroundjoin%
\definecolor{currentfill}{rgb}{0.000000,0.000000,0.000000}%
\pgfsetfillcolor{currentfill}%
\pgfsetlinewidth{1.003750pt}%
\definecolor{currentstroke}{rgb}{0.000000,0.000000,0.000000}%
\pgfsetstrokecolor{currentstroke}%
\pgfsetdash{}{0pt}%
\pgfpathmoveto{\pgfqpoint{5.504545in}{2.877687in}}%
\pgfpathcurveto{\pgfqpoint{5.515596in}{2.877687in}}{\pgfqpoint{5.526195in}{2.882077in}}{\pgfqpoint{5.534008in}{2.889891in}}%
\pgfpathcurveto{\pgfqpoint{5.541822in}{2.897704in}}{\pgfqpoint{5.546212in}{2.908303in}}{\pgfqpoint{5.546212in}{2.919353in}}%
\pgfpathcurveto{\pgfqpoint{5.546212in}{2.930404in}}{\pgfqpoint{5.541822in}{2.941003in}}{\pgfqpoint{5.534008in}{2.948816in}}%
\pgfpathcurveto{\pgfqpoint{5.526195in}{2.956630in}}{\pgfqpoint{5.515596in}{2.961020in}}{\pgfqpoint{5.504545in}{2.961020in}}%
\pgfpathcurveto{\pgfqpoint{5.493495in}{2.961020in}}{\pgfqpoint{5.482896in}{2.956630in}}{\pgfqpoint{5.475083in}{2.948816in}}%
\pgfpathcurveto{\pgfqpoint{5.467269in}{2.941003in}}{\pgfqpoint{5.462879in}{2.930404in}}{\pgfqpoint{5.462879in}{2.919353in}}%
\pgfpathcurveto{\pgfqpoint{5.462879in}{2.908303in}}{\pgfqpoint{5.467269in}{2.897704in}}{\pgfqpoint{5.475083in}{2.889891in}}%
\pgfpathcurveto{\pgfqpoint{5.482896in}{2.882077in}}{\pgfqpoint{5.493495in}{2.877687in}}{\pgfqpoint{5.504545in}{2.877687in}}%
\pgfpathclose%
\pgfusepath{stroke,fill}%
\end{pgfscope}%
\begin{pgfscope}%
\pgfpathrectangle{\pgfqpoint{0.800000in}{0.528000in}}{\pgfqpoint{4.960000in}{3.696000in}}%
\pgfusepath{clip}%
\pgfsetbuttcap%
\pgfsetroundjoin%
\definecolor{currentfill}{rgb}{0.000000,0.000000,0.000000}%
\pgfsetfillcolor{currentfill}%
\pgfsetlinewidth{1.003750pt}%
\definecolor{currentstroke}{rgb}{0.000000,0.000000,0.000000}%
\pgfsetstrokecolor{currentstroke}%
\pgfsetdash{}{0pt}%
\pgfpathmoveto{\pgfqpoint{5.504545in}{2.877687in}}%
\pgfpathcurveto{\pgfqpoint{5.515596in}{2.877687in}}{\pgfqpoint{5.526195in}{2.882077in}}{\pgfqpoint{5.534008in}{2.889891in}}%
\pgfpathcurveto{\pgfqpoint{5.541822in}{2.897704in}}{\pgfqpoint{5.546212in}{2.908303in}}{\pgfqpoint{5.546212in}{2.919353in}}%
\pgfpathcurveto{\pgfqpoint{5.546212in}{2.930404in}}{\pgfqpoint{5.541822in}{2.941003in}}{\pgfqpoint{5.534008in}{2.948816in}}%
\pgfpathcurveto{\pgfqpoint{5.526195in}{2.956630in}}{\pgfqpoint{5.515596in}{2.961020in}}{\pgfqpoint{5.504545in}{2.961020in}}%
\pgfpathcurveto{\pgfqpoint{5.493495in}{2.961020in}}{\pgfqpoint{5.482896in}{2.956630in}}{\pgfqpoint{5.475083in}{2.948816in}}%
\pgfpathcurveto{\pgfqpoint{5.467269in}{2.941003in}}{\pgfqpoint{5.462879in}{2.930404in}}{\pgfqpoint{5.462879in}{2.919353in}}%
\pgfpathcurveto{\pgfqpoint{5.462879in}{2.908303in}}{\pgfqpoint{5.467269in}{2.897704in}}{\pgfqpoint{5.475083in}{2.889891in}}%
\pgfpathcurveto{\pgfqpoint{5.482896in}{2.882077in}}{\pgfqpoint{5.493495in}{2.877687in}}{\pgfqpoint{5.504545in}{2.877687in}}%
\pgfpathclose%
\pgfusepath{stroke,fill}%
\end{pgfscope}%
\begin{pgfscope}%
\pgfpathrectangle{\pgfqpoint{0.800000in}{0.528000in}}{\pgfqpoint{4.960000in}{3.696000in}}%
\pgfusepath{clip}%
\pgfsetbuttcap%
\pgfsetroundjoin%
\definecolor{currentfill}{rgb}{0.000000,0.000000,0.000000}%
\pgfsetfillcolor{currentfill}%
\pgfsetlinewidth{1.003750pt}%
\definecolor{currentstroke}{rgb}{0.000000,0.000000,0.000000}%
\pgfsetstrokecolor{currentstroke}%
\pgfsetdash{}{0pt}%
\pgfpathmoveto{\pgfqpoint{5.504545in}{2.877687in}}%
\pgfpathcurveto{\pgfqpoint{5.515596in}{2.877687in}}{\pgfqpoint{5.526195in}{2.882077in}}{\pgfqpoint{5.534008in}{2.889891in}}%
\pgfpathcurveto{\pgfqpoint{5.541822in}{2.897704in}}{\pgfqpoint{5.546212in}{2.908303in}}{\pgfqpoint{5.546212in}{2.919353in}}%
\pgfpathcurveto{\pgfqpoint{5.546212in}{2.930404in}}{\pgfqpoint{5.541822in}{2.941003in}}{\pgfqpoint{5.534008in}{2.948816in}}%
\pgfpathcurveto{\pgfqpoint{5.526195in}{2.956630in}}{\pgfqpoint{5.515596in}{2.961020in}}{\pgfqpoint{5.504545in}{2.961020in}}%
\pgfpathcurveto{\pgfqpoint{5.493495in}{2.961020in}}{\pgfqpoint{5.482896in}{2.956630in}}{\pgfqpoint{5.475083in}{2.948816in}}%
\pgfpathcurveto{\pgfqpoint{5.467269in}{2.941003in}}{\pgfqpoint{5.462879in}{2.930404in}}{\pgfqpoint{5.462879in}{2.919353in}}%
\pgfpathcurveto{\pgfqpoint{5.462879in}{2.908303in}}{\pgfqpoint{5.467269in}{2.897704in}}{\pgfqpoint{5.475083in}{2.889891in}}%
\pgfpathcurveto{\pgfqpoint{5.482896in}{2.882077in}}{\pgfqpoint{5.493495in}{2.877687in}}{\pgfqpoint{5.504545in}{2.877687in}}%
\pgfpathclose%
\pgfusepath{stroke,fill}%
\end{pgfscope}%
\begin{pgfscope}%
\pgfpathrectangle{\pgfqpoint{0.800000in}{0.528000in}}{\pgfqpoint{4.960000in}{3.696000in}}%
\pgfusepath{clip}%
\pgfsetbuttcap%
\pgfsetroundjoin%
\definecolor{currentfill}{rgb}{0.000000,0.000000,0.000000}%
\pgfsetfillcolor{currentfill}%
\pgfsetlinewidth{1.003750pt}%
\definecolor{currentstroke}{rgb}{0.000000,0.000000,0.000000}%
\pgfsetstrokecolor{currentstroke}%
\pgfsetdash{}{0pt}%
\pgfpathmoveto{\pgfqpoint{5.504545in}{2.877687in}}%
\pgfpathcurveto{\pgfqpoint{5.515596in}{2.877687in}}{\pgfqpoint{5.526195in}{2.882077in}}{\pgfqpoint{5.534008in}{2.889891in}}%
\pgfpathcurveto{\pgfqpoint{5.541822in}{2.897704in}}{\pgfqpoint{5.546212in}{2.908303in}}{\pgfqpoint{5.546212in}{2.919353in}}%
\pgfpathcurveto{\pgfqpoint{5.546212in}{2.930404in}}{\pgfqpoint{5.541822in}{2.941003in}}{\pgfqpoint{5.534008in}{2.948816in}}%
\pgfpathcurveto{\pgfqpoint{5.526195in}{2.956630in}}{\pgfqpoint{5.515596in}{2.961020in}}{\pgfqpoint{5.504545in}{2.961020in}}%
\pgfpathcurveto{\pgfqpoint{5.493495in}{2.961020in}}{\pgfqpoint{5.482896in}{2.956630in}}{\pgfqpoint{5.475083in}{2.948816in}}%
\pgfpathcurveto{\pgfqpoint{5.467269in}{2.941003in}}{\pgfqpoint{5.462879in}{2.930404in}}{\pgfqpoint{5.462879in}{2.919353in}}%
\pgfpathcurveto{\pgfqpoint{5.462879in}{2.908303in}}{\pgfqpoint{5.467269in}{2.897704in}}{\pgfqpoint{5.475083in}{2.889891in}}%
\pgfpathcurveto{\pgfqpoint{5.482896in}{2.882077in}}{\pgfqpoint{5.493495in}{2.877687in}}{\pgfqpoint{5.504545in}{2.877687in}}%
\pgfpathclose%
\pgfusepath{stroke,fill}%
\end{pgfscope}%
\begin{pgfscope}%
\pgfpathrectangle{\pgfqpoint{0.800000in}{0.528000in}}{\pgfqpoint{4.960000in}{3.696000in}}%
\pgfusepath{clip}%
\pgfsetbuttcap%
\pgfsetroundjoin%
\definecolor{currentfill}{rgb}{0.000000,0.000000,0.000000}%
\pgfsetfillcolor{currentfill}%
\pgfsetlinewidth{1.003750pt}%
\definecolor{currentstroke}{rgb}{0.000000,0.000000,0.000000}%
\pgfsetstrokecolor{currentstroke}%
\pgfsetdash{}{0pt}%
\pgfpathmoveto{\pgfqpoint{5.504545in}{2.877687in}}%
\pgfpathcurveto{\pgfqpoint{5.515596in}{2.877687in}}{\pgfqpoint{5.526195in}{2.882077in}}{\pgfqpoint{5.534008in}{2.889891in}}%
\pgfpathcurveto{\pgfqpoint{5.541822in}{2.897704in}}{\pgfqpoint{5.546212in}{2.908303in}}{\pgfqpoint{5.546212in}{2.919353in}}%
\pgfpathcurveto{\pgfqpoint{5.546212in}{2.930404in}}{\pgfqpoint{5.541822in}{2.941003in}}{\pgfqpoint{5.534008in}{2.948816in}}%
\pgfpathcurveto{\pgfqpoint{5.526195in}{2.956630in}}{\pgfqpoint{5.515596in}{2.961020in}}{\pgfqpoint{5.504545in}{2.961020in}}%
\pgfpathcurveto{\pgfqpoint{5.493495in}{2.961020in}}{\pgfqpoint{5.482896in}{2.956630in}}{\pgfqpoint{5.475083in}{2.948816in}}%
\pgfpathcurveto{\pgfqpoint{5.467269in}{2.941003in}}{\pgfqpoint{5.462879in}{2.930404in}}{\pgfqpoint{5.462879in}{2.919353in}}%
\pgfpathcurveto{\pgfqpoint{5.462879in}{2.908303in}}{\pgfqpoint{5.467269in}{2.897704in}}{\pgfqpoint{5.475083in}{2.889891in}}%
\pgfpathcurveto{\pgfqpoint{5.482896in}{2.882077in}}{\pgfqpoint{5.493495in}{2.877687in}}{\pgfqpoint{5.504545in}{2.877687in}}%
\pgfpathclose%
\pgfusepath{stroke,fill}%
\end{pgfscope}%
\begin{pgfscope}%
\pgfpathrectangle{\pgfqpoint{0.800000in}{0.528000in}}{\pgfqpoint{4.960000in}{3.696000in}}%
\pgfusepath{clip}%
\pgfsetbuttcap%
\pgfsetroundjoin%
\definecolor{currentfill}{rgb}{0.000000,0.000000,0.000000}%
\pgfsetfillcolor{currentfill}%
\pgfsetlinewidth{1.003750pt}%
\definecolor{currentstroke}{rgb}{0.000000,0.000000,0.000000}%
\pgfsetstrokecolor{currentstroke}%
\pgfsetdash{}{0pt}%
\pgfpathmoveto{\pgfqpoint{5.504545in}{2.877687in}}%
\pgfpathcurveto{\pgfqpoint{5.515596in}{2.877687in}}{\pgfqpoint{5.526195in}{2.882077in}}{\pgfqpoint{5.534008in}{2.889891in}}%
\pgfpathcurveto{\pgfqpoint{5.541822in}{2.897704in}}{\pgfqpoint{5.546212in}{2.908303in}}{\pgfqpoint{5.546212in}{2.919353in}}%
\pgfpathcurveto{\pgfqpoint{5.546212in}{2.930404in}}{\pgfqpoint{5.541822in}{2.941003in}}{\pgfqpoint{5.534008in}{2.948816in}}%
\pgfpathcurveto{\pgfqpoint{5.526195in}{2.956630in}}{\pgfqpoint{5.515596in}{2.961020in}}{\pgfqpoint{5.504545in}{2.961020in}}%
\pgfpathcurveto{\pgfqpoint{5.493495in}{2.961020in}}{\pgfqpoint{5.482896in}{2.956630in}}{\pgfqpoint{5.475083in}{2.948816in}}%
\pgfpathcurveto{\pgfqpoint{5.467269in}{2.941003in}}{\pgfqpoint{5.462879in}{2.930404in}}{\pgfqpoint{5.462879in}{2.919353in}}%
\pgfpathcurveto{\pgfqpoint{5.462879in}{2.908303in}}{\pgfqpoint{5.467269in}{2.897704in}}{\pgfqpoint{5.475083in}{2.889891in}}%
\pgfpathcurveto{\pgfqpoint{5.482896in}{2.882077in}}{\pgfqpoint{5.493495in}{2.877687in}}{\pgfqpoint{5.504545in}{2.877687in}}%
\pgfpathclose%
\pgfusepath{stroke,fill}%
\end{pgfscope}%
\begin{pgfscope}%
\pgfpathrectangle{\pgfqpoint{0.800000in}{0.528000in}}{\pgfqpoint{4.960000in}{3.696000in}}%
\pgfusepath{clip}%
\pgfsetbuttcap%
\pgfsetroundjoin%
\definecolor{currentfill}{rgb}{0.000000,0.000000,0.000000}%
\pgfsetfillcolor{currentfill}%
\pgfsetlinewidth{1.003750pt}%
\definecolor{currentstroke}{rgb}{0.000000,0.000000,0.000000}%
\pgfsetstrokecolor{currentstroke}%
\pgfsetdash{}{0pt}%
\pgfpathmoveto{\pgfqpoint{5.504545in}{2.877687in}}%
\pgfpathcurveto{\pgfqpoint{5.515596in}{2.877687in}}{\pgfqpoint{5.526195in}{2.882077in}}{\pgfqpoint{5.534008in}{2.889891in}}%
\pgfpathcurveto{\pgfqpoint{5.541822in}{2.897704in}}{\pgfqpoint{5.546212in}{2.908303in}}{\pgfqpoint{5.546212in}{2.919353in}}%
\pgfpathcurveto{\pgfqpoint{5.546212in}{2.930404in}}{\pgfqpoint{5.541822in}{2.941003in}}{\pgfqpoint{5.534008in}{2.948816in}}%
\pgfpathcurveto{\pgfqpoint{5.526195in}{2.956630in}}{\pgfqpoint{5.515596in}{2.961020in}}{\pgfqpoint{5.504545in}{2.961020in}}%
\pgfpathcurveto{\pgfqpoint{5.493495in}{2.961020in}}{\pgfqpoint{5.482896in}{2.956630in}}{\pgfqpoint{5.475083in}{2.948816in}}%
\pgfpathcurveto{\pgfqpoint{5.467269in}{2.941003in}}{\pgfqpoint{5.462879in}{2.930404in}}{\pgfqpoint{5.462879in}{2.919353in}}%
\pgfpathcurveto{\pgfqpoint{5.462879in}{2.908303in}}{\pgfqpoint{5.467269in}{2.897704in}}{\pgfqpoint{5.475083in}{2.889891in}}%
\pgfpathcurveto{\pgfqpoint{5.482896in}{2.882077in}}{\pgfqpoint{5.493495in}{2.877687in}}{\pgfqpoint{5.504545in}{2.877687in}}%
\pgfpathclose%
\pgfusepath{stroke,fill}%
\end{pgfscope}%
\begin{pgfscope}%
\pgfpathrectangle{\pgfqpoint{0.800000in}{0.528000in}}{\pgfqpoint{4.960000in}{3.696000in}}%
\pgfusepath{clip}%
\pgfsetbuttcap%
\pgfsetroundjoin%
\definecolor{currentfill}{rgb}{0.000000,0.000000,0.000000}%
\pgfsetfillcolor{currentfill}%
\pgfsetlinewidth{1.003750pt}%
\definecolor{currentstroke}{rgb}{0.000000,0.000000,0.000000}%
\pgfsetstrokecolor{currentstroke}%
\pgfsetdash{}{0pt}%
\pgfpathmoveto{\pgfqpoint{5.504545in}{2.877687in}}%
\pgfpathcurveto{\pgfqpoint{5.515596in}{2.877687in}}{\pgfqpoint{5.526195in}{2.882077in}}{\pgfqpoint{5.534008in}{2.889891in}}%
\pgfpathcurveto{\pgfqpoint{5.541822in}{2.897704in}}{\pgfqpoint{5.546212in}{2.908303in}}{\pgfqpoint{5.546212in}{2.919353in}}%
\pgfpathcurveto{\pgfqpoint{5.546212in}{2.930404in}}{\pgfqpoint{5.541822in}{2.941003in}}{\pgfqpoint{5.534008in}{2.948816in}}%
\pgfpathcurveto{\pgfqpoint{5.526195in}{2.956630in}}{\pgfqpoint{5.515596in}{2.961020in}}{\pgfqpoint{5.504545in}{2.961020in}}%
\pgfpathcurveto{\pgfqpoint{5.493495in}{2.961020in}}{\pgfqpoint{5.482896in}{2.956630in}}{\pgfqpoint{5.475083in}{2.948816in}}%
\pgfpathcurveto{\pgfqpoint{5.467269in}{2.941003in}}{\pgfqpoint{5.462879in}{2.930404in}}{\pgfqpoint{5.462879in}{2.919353in}}%
\pgfpathcurveto{\pgfqpoint{5.462879in}{2.908303in}}{\pgfqpoint{5.467269in}{2.897704in}}{\pgfqpoint{5.475083in}{2.889891in}}%
\pgfpathcurveto{\pgfqpoint{5.482896in}{2.882077in}}{\pgfqpoint{5.493495in}{2.877687in}}{\pgfqpoint{5.504545in}{2.877687in}}%
\pgfpathclose%
\pgfusepath{stroke,fill}%
\end{pgfscope}%
\begin{pgfscope}%
\pgfpathrectangle{\pgfqpoint{0.800000in}{0.528000in}}{\pgfqpoint{4.960000in}{3.696000in}}%
\pgfusepath{clip}%
\pgfsetbuttcap%
\pgfsetroundjoin%
\definecolor{currentfill}{rgb}{0.000000,0.000000,0.000000}%
\pgfsetfillcolor{currentfill}%
\pgfsetlinewidth{1.003750pt}%
\definecolor{currentstroke}{rgb}{0.000000,0.000000,0.000000}%
\pgfsetstrokecolor{currentstroke}%
\pgfsetdash{}{0pt}%
\pgfpathmoveto{\pgfqpoint{5.504545in}{3.984333in}}%
\pgfpathcurveto{\pgfqpoint{5.515596in}{3.984333in}}{\pgfqpoint{5.526195in}{3.988724in}}{\pgfqpoint{5.534008in}{3.996537in}}%
\pgfpathcurveto{\pgfqpoint{5.541822in}{4.004351in}}{\pgfqpoint{5.546212in}{4.014950in}}{\pgfqpoint{5.546212in}{4.026000in}}%
\pgfpathcurveto{\pgfqpoint{5.546212in}{4.037050in}}{\pgfqpoint{5.541822in}{4.047649in}}{\pgfqpoint{5.534008in}{4.055463in}}%
\pgfpathcurveto{\pgfqpoint{5.526195in}{4.063276in}}{\pgfqpoint{5.515596in}{4.067667in}}{\pgfqpoint{5.504545in}{4.067667in}}%
\pgfpathcurveto{\pgfqpoint{5.493495in}{4.067667in}}{\pgfqpoint{5.482896in}{4.063276in}}{\pgfqpoint{5.475083in}{4.055463in}}%
\pgfpathcurveto{\pgfqpoint{5.467269in}{4.047649in}}{\pgfqpoint{5.462879in}{4.037050in}}{\pgfqpoint{5.462879in}{4.026000in}}%
\pgfpathcurveto{\pgfqpoint{5.462879in}{4.014950in}}{\pgfqpoint{5.467269in}{4.004351in}}{\pgfqpoint{5.475083in}{3.996537in}}%
\pgfpathcurveto{\pgfqpoint{5.482896in}{3.988724in}}{\pgfqpoint{5.493495in}{3.984333in}}{\pgfqpoint{5.504545in}{3.984333in}}%
\pgfpathclose%
\pgfusepath{stroke,fill}%
\end{pgfscope}%
\begin{pgfscope}%
\pgfpathrectangle{\pgfqpoint{0.800000in}{0.528000in}}{\pgfqpoint{4.960000in}{3.696000in}}%
\pgfusepath{clip}%
\pgfsetbuttcap%
\pgfsetroundjoin%
\definecolor{currentfill}{rgb}{0.000000,0.000000,0.000000}%
\pgfsetfillcolor{currentfill}%
\pgfsetlinewidth{1.003750pt}%
\definecolor{currentstroke}{rgb}{0.000000,0.000000,0.000000}%
\pgfsetstrokecolor{currentstroke}%
\pgfsetdash{}{0pt}%
\pgfpathmoveto{\pgfqpoint{5.504545in}{2.877687in}}%
\pgfpathcurveto{\pgfqpoint{5.515596in}{2.877687in}}{\pgfqpoint{5.526195in}{2.882077in}}{\pgfqpoint{5.534008in}{2.889891in}}%
\pgfpathcurveto{\pgfqpoint{5.541822in}{2.897704in}}{\pgfqpoint{5.546212in}{2.908303in}}{\pgfqpoint{5.546212in}{2.919353in}}%
\pgfpathcurveto{\pgfqpoint{5.546212in}{2.930404in}}{\pgfqpoint{5.541822in}{2.941003in}}{\pgfqpoint{5.534008in}{2.948816in}}%
\pgfpathcurveto{\pgfqpoint{5.526195in}{2.956630in}}{\pgfqpoint{5.515596in}{2.961020in}}{\pgfqpoint{5.504545in}{2.961020in}}%
\pgfpathcurveto{\pgfqpoint{5.493495in}{2.961020in}}{\pgfqpoint{5.482896in}{2.956630in}}{\pgfqpoint{5.475083in}{2.948816in}}%
\pgfpathcurveto{\pgfqpoint{5.467269in}{2.941003in}}{\pgfqpoint{5.462879in}{2.930404in}}{\pgfqpoint{5.462879in}{2.919353in}}%
\pgfpathcurveto{\pgfqpoint{5.462879in}{2.908303in}}{\pgfqpoint{5.467269in}{2.897704in}}{\pgfqpoint{5.475083in}{2.889891in}}%
\pgfpathcurveto{\pgfqpoint{5.482896in}{2.882077in}}{\pgfqpoint{5.493495in}{2.877687in}}{\pgfqpoint{5.504545in}{2.877687in}}%
\pgfpathclose%
\pgfusepath{stroke,fill}%
\end{pgfscope}%
\begin{pgfscope}%
\pgfpathrectangle{\pgfqpoint{0.800000in}{0.528000in}}{\pgfqpoint{4.960000in}{3.696000in}}%
\pgfusepath{clip}%
\pgfsetbuttcap%
\pgfsetroundjoin%
\definecolor{currentfill}{rgb}{0.000000,0.000000,0.000000}%
\pgfsetfillcolor{currentfill}%
\pgfsetlinewidth{1.003750pt}%
\definecolor{currentstroke}{rgb}{0.000000,0.000000,0.000000}%
\pgfsetstrokecolor{currentstroke}%
\pgfsetdash{}{0pt}%
\pgfpathmoveto{\pgfqpoint{5.504545in}{2.877687in}}%
\pgfpathcurveto{\pgfqpoint{5.515596in}{2.877687in}}{\pgfqpoint{5.526195in}{2.882077in}}{\pgfqpoint{5.534008in}{2.889891in}}%
\pgfpathcurveto{\pgfqpoint{5.541822in}{2.897704in}}{\pgfqpoint{5.546212in}{2.908303in}}{\pgfqpoint{5.546212in}{2.919353in}}%
\pgfpathcurveto{\pgfqpoint{5.546212in}{2.930404in}}{\pgfqpoint{5.541822in}{2.941003in}}{\pgfqpoint{5.534008in}{2.948816in}}%
\pgfpathcurveto{\pgfqpoint{5.526195in}{2.956630in}}{\pgfqpoint{5.515596in}{2.961020in}}{\pgfqpoint{5.504545in}{2.961020in}}%
\pgfpathcurveto{\pgfqpoint{5.493495in}{2.961020in}}{\pgfqpoint{5.482896in}{2.956630in}}{\pgfqpoint{5.475083in}{2.948816in}}%
\pgfpathcurveto{\pgfqpoint{5.467269in}{2.941003in}}{\pgfqpoint{5.462879in}{2.930404in}}{\pgfqpoint{5.462879in}{2.919353in}}%
\pgfpathcurveto{\pgfqpoint{5.462879in}{2.908303in}}{\pgfqpoint{5.467269in}{2.897704in}}{\pgfqpoint{5.475083in}{2.889891in}}%
\pgfpathcurveto{\pgfqpoint{5.482896in}{2.882077in}}{\pgfqpoint{5.493495in}{2.877687in}}{\pgfqpoint{5.504545in}{2.877687in}}%
\pgfpathclose%
\pgfusepath{stroke,fill}%
\end{pgfscope}%
\begin{pgfscope}%
\pgfpathrectangle{\pgfqpoint{0.800000in}{0.528000in}}{\pgfqpoint{4.960000in}{3.696000in}}%
\pgfusepath{clip}%
\pgfsetbuttcap%
\pgfsetroundjoin%
\definecolor{currentfill}{rgb}{0.000000,0.000000,0.000000}%
\pgfsetfillcolor{currentfill}%
\pgfsetlinewidth{1.003750pt}%
\definecolor{currentstroke}{rgb}{0.000000,0.000000,0.000000}%
\pgfsetstrokecolor{currentstroke}%
\pgfsetdash{}{0pt}%
\pgfpathmoveto{\pgfqpoint{5.504545in}{3.984333in}}%
\pgfpathcurveto{\pgfqpoint{5.515596in}{3.984333in}}{\pgfqpoint{5.526195in}{3.988724in}}{\pgfqpoint{5.534008in}{3.996537in}}%
\pgfpathcurveto{\pgfqpoint{5.541822in}{4.004351in}}{\pgfqpoint{5.546212in}{4.014950in}}{\pgfqpoint{5.546212in}{4.026000in}}%
\pgfpathcurveto{\pgfqpoint{5.546212in}{4.037050in}}{\pgfqpoint{5.541822in}{4.047649in}}{\pgfqpoint{5.534008in}{4.055463in}}%
\pgfpathcurveto{\pgfqpoint{5.526195in}{4.063276in}}{\pgfqpoint{5.515596in}{4.067667in}}{\pgfqpoint{5.504545in}{4.067667in}}%
\pgfpathcurveto{\pgfqpoint{5.493495in}{4.067667in}}{\pgfqpoint{5.482896in}{4.063276in}}{\pgfqpoint{5.475083in}{4.055463in}}%
\pgfpathcurveto{\pgfqpoint{5.467269in}{4.047649in}}{\pgfqpoint{5.462879in}{4.037050in}}{\pgfqpoint{5.462879in}{4.026000in}}%
\pgfpathcurveto{\pgfqpoint{5.462879in}{4.014950in}}{\pgfqpoint{5.467269in}{4.004351in}}{\pgfqpoint{5.475083in}{3.996537in}}%
\pgfpathcurveto{\pgfqpoint{5.482896in}{3.988724in}}{\pgfqpoint{5.493495in}{3.984333in}}{\pgfqpoint{5.504545in}{3.984333in}}%
\pgfpathclose%
\pgfusepath{stroke,fill}%
\end{pgfscope}%
\begin{pgfscope}%
\pgfpathrectangle{\pgfqpoint{0.800000in}{0.528000in}}{\pgfqpoint{4.960000in}{3.696000in}}%
\pgfusepath{clip}%
\pgfsetbuttcap%
\pgfsetroundjoin%
\definecolor{currentfill}{rgb}{0.000000,0.000000,0.000000}%
\pgfsetfillcolor{currentfill}%
\pgfsetlinewidth{1.003750pt}%
\definecolor{currentstroke}{rgb}{0.000000,0.000000,0.000000}%
\pgfsetstrokecolor{currentstroke}%
\pgfsetdash{}{0pt}%
\pgfpathmoveto{\pgfqpoint{5.504545in}{2.877687in}}%
\pgfpathcurveto{\pgfqpoint{5.515596in}{2.877687in}}{\pgfqpoint{5.526195in}{2.882077in}}{\pgfqpoint{5.534008in}{2.889891in}}%
\pgfpathcurveto{\pgfqpoint{5.541822in}{2.897704in}}{\pgfqpoint{5.546212in}{2.908303in}}{\pgfqpoint{5.546212in}{2.919353in}}%
\pgfpathcurveto{\pgfqpoint{5.546212in}{2.930404in}}{\pgfqpoint{5.541822in}{2.941003in}}{\pgfqpoint{5.534008in}{2.948816in}}%
\pgfpathcurveto{\pgfqpoint{5.526195in}{2.956630in}}{\pgfqpoint{5.515596in}{2.961020in}}{\pgfqpoint{5.504545in}{2.961020in}}%
\pgfpathcurveto{\pgfqpoint{5.493495in}{2.961020in}}{\pgfqpoint{5.482896in}{2.956630in}}{\pgfqpoint{5.475083in}{2.948816in}}%
\pgfpathcurveto{\pgfqpoint{5.467269in}{2.941003in}}{\pgfqpoint{5.462879in}{2.930404in}}{\pgfqpoint{5.462879in}{2.919353in}}%
\pgfpathcurveto{\pgfqpoint{5.462879in}{2.908303in}}{\pgfqpoint{5.467269in}{2.897704in}}{\pgfqpoint{5.475083in}{2.889891in}}%
\pgfpathcurveto{\pgfqpoint{5.482896in}{2.882077in}}{\pgfqpoint{5.493495in}{2.877687in}}{\pgfqpoint{5.504545in}{2.877687in}}%
\pgfpathclose%
\pgfusepath{stroke,fill}%
\end{pgfscope}%
\begin{pgfscope}%
\pgfpathrectangle{\pgfqpoint{0.800000in}{0.528000in}}{\pgfqpoint{4.960000in}{3.696000in}}%
\pgfusepath{clip}%
\pgfsetbuttcap%
\pgfsetroundjoin%
\definecolor{currentfill}{rgb}{0.000000,0.000000,0.000000}%
\pgfsetfillcolor{currentfill}%
\pgfsetlinewidth{1.003750pt}%
\definecolor{currentstroke}{rgb}{0.000000,0.000000,0.000000}%
\pgfsetstrokecolor{currentstroke}%
\pgfsetdash{}{0pt}%
\pgfpathmoveto{\pgfqpoint{5.504545in}{2.877687in}}%
\pgfpathcurveto{\pgfqpoint{5.515596in}{2.877687in}}{\pgfqpoint{5.526195in}{2.882077in}}{\pgfqpoint{5.534008in}{2.889891in}}%
\pgfpathcurveto{\pgfqpoint{5.541822in}{2.897704in}}{\pgfqpoint{5.546212in}{2.908303in}}{\pgfqpoint{5.546212in}{2.919353in}}%
\pgfpathcurveto{\pgfqpoint{5.546212in}{2.930404in}}{\pgfqpoint{5.541822in}{2.941003in}}{\pgfqpoint{5.534008in}{2.948816in}}%
\pgfpathcurveto{\pgfqpoint{5.526195in}{2.956630in}}{\pgfqpoint{5.515596in}{2.961020in}}{\pgfqpoint{5.504545in}{2.961020in}}%
\pgfpathcurveto{\pgfqpoint{5.493495in}{2.961020in}}{\pgfqpoint{5.482896in}{2.956630in}}{\pgfqpoint{5.475083in}{2.948816in}}%
\pgfpathcurveto{\pgfqpoint{5.467269in}{2.941003in}}{\pgfqpoint{5.462879in}{2.930404in}}{\pgfqpoint{5.462879in}{2.919353in}}%
\pgfpathcurveto{\pgfqpoint{5.462879in}{2.908303in}}{\pgfqpoint{5.467269in}{2.897704in}}{\pgfqpoint{5.475083in}{2.889891in}}%
\pgfpathcurveto{\pgfqpoint{5.482896in}{2.882077in}}{\pgfqpoint{5.493495in}{2.877687in}}{\pgfqpoint{5.504545in}{2.877687in}}%
\pgfpathclose%
\pgfusepath{stroke,fill}%
\end{pgfscope}%
\begin{pgfscope}%
\pgfpathrectangle{\pgfqpoint{0.800000in}{0.528000in}}{\pgfqpoint{4.960000in}{3.696000in}}%
\pgfusepath{clip}%
\pgfsetbuttcap%
\pgfsetroundjoin%
\definecolor{currentfill}{rgb}{0.000000,0.000000,0.000000}%
\pgfsetfillcolor{currentfill}%
\pgfsetlinewidth{1.003750pt}%
\definecolor{currentstroke}{rgb}{0.000000,0.000000,0.000000}%
\pgfsetstrokecolor{currentstroke}%
\pgfsetdash{}{0pt}%
\pgfpathmoveto{\pgfqpoint{5.504545in}{2.877687in}}%
\pgfpathcurveto{\pgfqpoint{5.515596in}{2.877687in}}{\pgfqpoint{5.526195in}{2.882077in}}{\pgfqpoint{5.534008in}{2.889891in}}%
\pgfpathcurveto{\pgfqpoint{5.541822in}{2.897704in}}{\pgfqpoint{5.546212in}{2.908303in}}{\pgfqpoint{5.546212in}{2.919353in}}%
\pgfpathcurveto{\pgfqpoint{5.546212in}{2.930404in}}{\pgfqpoint{5.541822in}{2.941003in}}{\pgfqpoint{5.534008in}{2.948816in}}%
\pgfpathcurveto{\pgfqpoint{5.526195in}{2.956630in}}{\pgfqpoint{5.515596in}{2.961020in}}{\pgfqpoint{5.504545in}{2.961020in}}%
\pgfpathcurveto{\pgfqpoint{5.493495in}{2.961020in}}{\pgfqpoint{5.482896in}{2.956630in}}{\pgfqpoint{5.475083in}{2.948816in}}%
\pgfpathcurveto{\pgfqpoint{5.467269in}{2.941003in}}{\pgfqpoint{5.462879in}{2.930404in}}{\pgfqpoint{5.462879in}{2.919353in}}%
\pgfpathcurveto{\pgfqpoint{5.462879in}{2.908303in}}{\pgfqpoint{5.467269in}{2.897704in}}{\pgfqpoint{5.475083in}{2.889891in}}%
\pgfpathcurveto{\pgfqpoint{5.482896in}{2.882077in}}{\pgfqpoint{5.493495in}{2.877687in}}{\pgfqpoint{5.504545in}{2.877687in}}%
\pgfpathclose%
\pgfusepath{stroke,fill}%
\end{pgfscope}%
\begin{pgfscope}%
\pgfpathrectangle{\pgfqpoint{0.800000in}{0.528000in}}{\pgfqpoint{4.960000in}{3.696000in}}%
\pgfusepath{clip}%
\pgfsetbuttcap%
\pgfsetroundjoin%
\definecolor{currentfill}{rgb}{0.000000,0.000000,0.000000}%
\pgfsetfillcolor{currentfill}%
\pgfsetlinewidth{1.003750pt}%
\definecolor{currentstroke}{rgb}{0.000000,0.000000,0.000000}%
\pgfsetstrokecolor{currentstroke}%
\pgfsetdash{}{0pt}%
\pgfpathmoveto{\pgfqpoint{5.504545in}{2.877687in}}%
\pgfpathcurveto{\pgfqpoint{5.515596in}{2.877687in}}{\pgfqpoint{5.526195in}{2.882077in}}{\pgfqpoint{5.534008in}{2.889891in}}%
\pgfpathcurveto{\pgfqpoint{5.541822in}{2.897704in}}{\pgfqpoint{5.546212in}{2.908303in}}{\pgfqpoint{5.546212in}{2.919353in}}%
\pgfpathcurveto{\pgfqpoint{5.546212in}{2.930404in}}{\pgfqpoint{5.541822in}{2.941003in}}{\pgfqpoint{5.534008in}{2.948816in}}%
\pgfpathcurveto{\pgfqpoint{5.526195in}{2.956630in}}{\pgfqpoint{5.515596in}{2.961020in}}{\pgfqpoint{5.504545in}{2.961020in}}%
\pgfpathcurveto{\pgfqpoint{5.493495in}{2.961020in}}{\pgfqpoint{5.482896in}{2.956630in}}{\pgfqpoint{5.475083in}{2.948816in}}%
\pgfpathcurveto{\pgfqpoint{5.467269in}{2.941003in}}{\pgfqpoint{5.462879in}{2.930404in}}{\pgfqpoint{5.462879in}{2.919353in}}%
\pgfpathcurveto{\pgfqpoint{5.462879in}{2.908303in}}{\pgfqpoint{5.467269in}{2.897704in}}{\pgfqpoint{5.475083in}{2.889891in}}%
\pgfpathcurveto{\pgfqpoint{5.482896in}{2.882077in}}{\pgfqpoint{5.493495in}{2.877687in}}{\pgfqpoint{5.504545in}{2.877687in}}%
\pgfpathclose%
\pgfusepath{stroke,fill}%
\end{pgfscope}%
\begin{pgfscope}%
\pgfpathrectangle{\pgfqpoint{0.800000in}{0.528000in}}{\pgfqpoint{4.960000in}{3.696000in}}%
\pgfusepath{clip}%
\pgfsetbuttcap%
\pgfsetroundjoin%
\definecolor{currentfill}{rgb}{0.000000,0.000000,0.000000}%
\pgfsetfillcolor{currentfill}%
\pgfsetlinewidth{1.003750pt}%
\definecolor{currentstroke}{rgb}{0.000000,0.000000,0.000000}%
\pgfsetstrokecolor{currentstroke}%
\pgfsetdash{}{0pt}%
\pgfpathmoveto{\pgfqpoint{5.504545in}{2.877687in}}%
\pgfpathcurveto{\pgfqpoint{5.515596in}{2.877687in}}{\pgfqpoint{5.526195in}{2.882077in}}{\pgfqpoint{5.534008in}{2.889891in}}%
\pgfpathcurveto{\pgfqpoint{5.541822in}{2.897704in}}{\pgfqpoint{5.546212in}{2.908303in}}{\pgfqpoint{5.546212in}{2.919353in}}%
\pgfpathcurveto{\pgfqpoint{5.546212in}{2.930404in}}{\pgfqpoint{5.541822in}{2.941003in}}{\pgfqpoint{5.534008in}{2.948816in}}%
\pgfpathcurveto{\pgfqpoint{5.526195in}{2.956630in}}{\pgfqpoint{5.515596in}{2.961020in}}{\pgfqpoint{5.504545in}{2.961020in}}%
\pgfpathcurveto{\pgfqpoint{5.493495in}{2.961020in}}{\pgfqpoint{5.482896in}{2.956630in}}{\pgfqpoint{5.475083in}{2.948816in}}%
\pgfpathcurveto{\pgfqpoint{5.467269in}{2.941003in}}{\pgfqpoint{5.462879in}{2.930404in}}{\pgfqpoint{5.462879in}{2.919353in}}%
\pgfpathcurveto{\pgfqpoint{5.462879in}{2.908303in}}{\pgfqpoint{5.467269in}{2.897704in}}{\pgfqpoint{5.475083in}{2.889891in}}%
\pgfpathcurveto{\pgfqpoint{5.482896in}{2.882077in}}{\pgfqpoint{5.493495in}{2.877687in}}{\pgfqpoint{5.504545in}{2.877687in}}%
\pgfpathclose%
\pgfusepath{stroke,fill}%
\end{pgfscope}%
\begin{pgfscope}%
\pgfpathrectangle{\pgfqpoint{0.800000in}{0.528000in}}{\pgfqpoint{4.960000in}{3.696000in}}%
\pgfusepath{clip}%
\pgfsetbuttcap%
\pgfsetroundjoin%
\definecolor{currentfill}{rgb}{0.000000,0.000000,0.000000}%
\pgfsetfillcolor{currentfill}%
\pgfsetlinewidth{1.003750pt}%
\definecolor{currentstroke}{rgb}{0.000000,0.000000,0.000000}%
\pgfsetstrokecolor{currentstroke}%
\pgfsetdash{}{0pt}%
\pgfpathmoveto{\pgfqpoint{5.504545in}{2.877687in}}%
\pgfpathcurveto{\pgfqpoint{5.515596in}{2.877687in}}{\pgfqpoint{5.526195in}{2.882077in}}{\pgfqpoint{5.534008in}{2.889891in}}%
\pgfpathcurveto{\pgfqpoint{5.541822in}{2.897704in}}{\pgfqpoint{5.546212in}{2.908303in}}{\pgfqpoint{5.546212in}{2.919353in}}%
\pgfpathcurveto{\pgfqpoint{5.546212in}{2.930404in}}{\pgfqpoint{5.541822in}{2.941003in}}{\pgfqpoint{5.534008in}{2.948816in}}%
\pgfpathcurveto{\pgfqpoint{5.526195in}{2.956630in}}{\pgfqpoint{5.515596in}{2.961020in}}{\pgfqpoint{5.504545in}{2.961020in}}%
\pgfpathcurveto{\pgfqpoint{5.493495in}{2.961020in}}{\pgfqpoint{5.482896in}{2.956630in}}{\pgfqpoint{5.475083in}{2.948816in}}%
\pgfpathcurveto{\pgfqpoint{5.467269in}{2.941003in}}{\pgfqpoint{5.462879in}{2.930404in}}{\pgfqpoint{5.462879in}{2.919353in}}%
\pgfpathcurveto{\pgfqpoint{5.462879in}{2.908303in}}{\pgfqpoint{5.467269in}{2.897704in}}{\pgfqpoint{5.475083in}{2.889891in}}%
\pgfpathcurveto{\pgfqpoint{5.482896in}{2.882077in}}{\pgfqpoint{5.493495in}{2.877687in}}{\pgfqpoint{5.504545in}{2.877687in}}%
\pgfpathclose%
\pgfusepath{stroke,fill}%
\end{pgfscope}%
\begin{pgfscope}%
\pgfpathrectangle{\pgfqpoint{0.800000in}{0.528000in}}{\pgfqpoint{4.960000in}{3.696000in}}%
\pgfusepath{clip}%
\pgfsetbuttcap%
\pgfsetroundjoin%
\definecolor{currentfill}{rgb}{0.000000,0.000000,0.000000}%
\pgfsetfillcolor{currentfill}%
\pgfsetlinewidth{1.003750pt}%
\definecolor{currentstroke}{rgb}{0.000000,0.000000,0.000000}%
\pgfsetstrokecolor{currentstroke}%
\pgfsetdash{}{0pt}%
\pgfpathmoveto{\pgfqpoint{5.504545in}{2.877687in}}%
\pgfpathcurveto{\pgfqpoint{5.515596in}{2.877687in}}{\pgfqpoint{5.526195in}{2.882077in}}{\pgfqpoint{5.534008in}{2.889891in}}%
\pgfpathcurveto{\pgfqpoint{5.541822in}{2.897704in}}{\pgfqpoint{5.546212in}{2.908303in}}{\pgfqpoint{5.546212in}{2.919353in}}%
\pgfpathcurveto{\pgfqpoint{5.546212in}{2.930404in}}{\pgfqpoint{5.541822in}{2.941003in}}{\pgfqpoint{5.534008in}{2.948816in}}%
\pgfpathcurveto{\pgfqpoint{5.526195in}{2.956630in}}{\pgfqpoint{5.515596in}{2.961020in}}{\pgfqpoint{5.504545in}{2.961020in}}%
\pgfpathcurveto{\pgfqpoint{5.493495in}{2.961020in}}{\pgfqpoint{5.482896in}{2.956630in}}{\pgfqpoint{5.475083in}{2.948816in}}%
\pgfpathcurveto{\pgfqpoint{5.467269in}{2.941003in}}{\pgfqpoint{5.462879in}{2.930404in}}{\pgfqpoint{5.462879in}{2.919353in}}%
\pgfpathcurveto{\pgfqpoint{5.462879in}{2.908303in}}{\pgfqpoint{5.467269in}{2.897704in}}{\pgfqpoint{5.475083in}{2.889891in}}%
\pgfpathcurveto{\pgfqpoint{5.482896in}{2.882077in}}{\pgfqpoint{5.493495in}{2.877687in}}{\pgfqpoint{5.504545in}{2.877687in}}%
\pgfpathclose%
\pgfusepath{stroke,fill}%
\end{pgfscope}%
\begin{pgfscope}%
\pgfpathrectangle{\pgfqpoint{0.800000in}{0.528000in}}{\pgfqpoint{4.960000in}{3.696000in}}%
\pgfusepath{clip}%
\pgfsetbuttcap%
\pgfsetroundjoin%
\definecolor{currentfill}{rgb}{0.000000,0.000000,0.000000}%
\pgfsetfillcolor{currentfill}%
\pgfsetlinewidth{1.003750pt}%
\definecolor{currentstroke}{rgb}{0.000000,0.000000,0.000000}%
\pgfsetstrokecolor{currentstroke}%
\pgfsetdash{}{0pt}%
\pgfpathmoveto{\pgfqpoint{5.504545in}{2.877687in}}%
\pgfpathcurveto{\pgfqpoint{5.515596in}{2.877687in}}{\pgfqpoint{5.526195in}{2.882077in}}{\pgfqpoint{5.534008in}{2.889891in}}%
\pgfpathcurveto{\pgfqpoint{5.541822in}{2.897704in}}{\pgfqpoint{5.546212in}{2.908303in}}{\pgfqpoint{5.546212in}{2.919353in}}%
\pgfpathcurveto{\pgfqpoint{5.546212in}{2.930404in}}{\pgfqpoint{5.541822in}{2.941003in}}{\pgfqpoint{5.534008in}{2.948816in}}%
\pgfpathcurveto{\pgfqpoint{5.526195in}{2.956630in}}{\pgfqpoint{5.515596in}{2.961020in}}{\pgfqpoint{5.504545in}{2.961020in}}%
\pgfpathcurveto{\pgfqpoint{5.493495in}{2.961020in}}{\pgfqpoint{5.482896in}{2.956630in}}{\pgfqpoint{5.475083in}{2.948816in}}%
\pgfpathcurveto{\pgfqpoint{5.467269in}{2.941003in}}{\pgfqpoint{5.462879in}{2.930404in}}{\pgfqpoint{5.462879in}{2.919353in}}%
\pgfpathcurveto{\pgfqpoint{5.462879in}{2.908303in}}{\pgfqpoint{5.467269in}{2.897704in}}{\pgfqpoint{5.475083in}{2.889891in}}%
\pgfpathcurveto{\pgfqpoint{5.482896in}{2.882077in}}{\pgfqpoint{5.493495in}{2.877687in}}{\pgfqpoint{5.504545in}{2.877687in}}%
\pgfpathclose%
\pgfusepath{stroke,fill}%
\end{pgfscope}%
\begin{pgfscope}%
\pgfpathrectangle{\pgfqpoint{0.800000in}{0.528000in}}{\pgfqpoint{4.960000in}{3.696000in}}%
\pgfusepath{clip}%
\pgfsetbuttcap%
\pgfsetroundjoin%
\definecolor{currentfill}{rgb}{0.000000,0.000000,0.000000}%
\pgfsetfillcolor{currentfill}%
\pgfsetlinewidth{1.003750pt}%
\definecolor{currentstroke}{rgb}{0.000000,0.000000,0.000000}%
\pgfsetstrokecolor{currentstroke}%
\pgfsetdash{}{0pt}%
\pgfpathmoveto{\pgfqpoint{5.504545in}{2.877687in}}%
\pgfpathcurveto{\pgfqpoint{5.515596in}{2.877687in}}{\pgfqpoint{5.526195in}{2.882077in}}{\pgfqpoint{5.534008in}{2.889891in}}%
\pgfpathcurveto{\pgfqpoint{5.541822in}{2.897704in}}{\pgfqpoint{5.546212in}{2.908303in}}{\pgfqpoint{5.546212in}{2.919353in}}%
\pgfpathcurveto{\pgfqpoint{5.546212in}{2.930404in}}{\pgfqpoint{5.541822in}{2.941003in}}{\pgfqpoint{5.534008in}{2.948816in}}%
\pgfpathcurveto{\pgfqpoint{5.526195in}{2.956630in}}{\pgfqpoint{5.515596in}{2.961020in}}{\pgfqpoint{5.504545in}{2.961020in}}%
\pgfpathcurveto{\pgfqpoint{5.493495in}{2.961020in}}{\pgfqpoint{5.482896in}{2.956630in}}{\pgfqpoint{5.475083in}{2.948816in}}%
\pgfpathcurveto{\pgfqpoint{5.467269in}{2.941003in}}{\pgfqpoint{5.462879in}{2.930404in}}{\pgfqpoint{5.462879in}{2.919353in}}%
\pgfpathcurveto{\pgfqpoint{5.462879in}{2.908303in}}{\pgfqpoint{5.467269in}{2.897704in}}{\pgfqpoint{5.475083in}{2.889891in}}%
\pgfpathcurveto{\pgfqpoint{5.482896in}{2.882077in}}{\pgfqpoint{5.493495in}{2.877687in}}{\pgfqpoint{5.504545in}{2.877687in}}%
\pgfpathclose%
\pgfusepath{stroke,fill}%
\end{pgfscope}%
\begin{pgfscope}%
\pgfpathrectangle{\pgfqpoint{0.800000in}{0.528000in}}{\pgfqpoint{4.960000in}{3.696000in}}%
\pgfusepath{clip}%
\pgfsetbuttcap%
\pgfsetroundjoin%
\definecolor{currentfill}{rgb}{0.000000,0.000000,0.000000}%
\pgfsetfillcolor{currentfill}%
\pgfsetlinewidth{1.003750pt}%
\definecolor{currentstroke}{rgb}{0.000000,0.000000,0.000000}%
\pgfsetstrokecolor{currentstroke}%
\pgfsetdash{}{0pt}%
\pgfpathmoveto{\pgfqpoint{5.504545in}{2.877687in}}%
\pgfpathcurveto{\pgfqpoint{5.515596in}{2.877687in}}{\pgfqpoint{5.526195in}{2.882077in}}{\pgfqpoint{5.534008in}{2.889891in}}%
\pgfpathcurveto{\pgfqpoint{5.541822in}{2.897704in}}{\pgfqpoint{5.546212in}{2.908303in}}{\pgfqpoint{5.546212in}{2.919353in}}%
\pgfpathcurveto{\pgfqpoint{5.546212in}{2.930404in}}{\pgfqpoint{5.541822in}{2.941003in}}{\pgfqpoint{5.534008in}{2.948816in}}%
\pgfpathcurveto{\pgfqpoint{5.526195in}{2.956630in}}{\pgfqpoint{5.515596in}{2.961020in}}{\pgfqpoint{5.504545in}{2.961020in}}%
\pgfpathcurveto{\pgfqpoint{5.493495in}{2.961020in}}{\pgfqpoint{5.482896in}{2.956630in}}{\pgfqpoint{5.475083in}{2.948816in}}%
\pgfpathcurveto{\pgfqpoint{5.467269in}{2.941003in}}{\pgfqpoint{5.462879in}{2.930404in}}{\pgfqpoint{5.462879in}{2.919353in}}%
\pgfpathcurveto{\pgfqpoint{5.462879in}{2.908303in}}{\pgfqpoint{5.467269in}{2.897704in}}{\pgfqpoint{5.475083in}{2.889891in}}%
\pgfpathcurveto{\pgfqpoint{5.482896in}{2.882077in}}{\pgfqpoint{5.493495in}{2.877687in}}{\pgfqpoint{5.504545in}{2.877687in}}%
\pgfpathclose%
\pgfusepath{stroke,fill}%
\end{pgfscope}%
\begin{pgfscope}%
\pgfpathrectangle{\pgfqpoint{0.800000in}{0.528000in}}{\pgfqpoint{4.960000in}{3.696000in}}%
\pgfusepath{clip}%
\pgfsetbuttcap%
\pgfsetroundjoin%
\definecolor{currentfill}{rgb}{0.000000,0.000000,0.000000}%
\pgfsetfillcolor{currentfill}%
\pgfsetlinewidth{1.003750pt}%
\definecolor{currentstroke}{rgb}{0.000000,0.000000,0.000000}%
\pgfsetstrokecolor{currentstroke}%
\pgfsetdash{}{0pt}%
\pgfpathmoveto{\pgfqpoint{5.504545in}{2.877687in}}%
\pgfpathcurveto{\pgfqpoint{5.515596in}{2.877687in}}{\pgfqpoint{5.526195in}{2.882077in}}{\pgfqpoint{5.534008in}{2.889891in}}%
\pgfpathcurveto{\pgfqpoint{5.541822in}{2.897704in}}{\pgfqpoint{5.546212in}{2.908303in}}{\pgfqpoint{5.546212in}{2.919353in}}%
\pgfpathcurveto{\pgfqpoint{5.546212in}{2.930404in}}{\pgfqpoint{5.541822in}{2.941003in}}{\pgfqpoint{5.534008in}{2.948816in}}%
\pgfpathcurveto{\pgfqpoint{5.526195in}{2.956630in}}{\pgfqpoint{5.515596in}{2.961020in}}{\pgfqpoint{5.504545in}{2.961020in}}%
\pgfpathcurveto{\pgfqpoint{5.493495in}{2.961020in}}{\pgfqpoint{5.482896in}{2.956630in}}{\pgfqpoint{5.475083in}{2.948816in}}%
\pgfpathcurveto{\pgfqpoint{5.467269in}{2.941003in}}{\pgfqpoint{5.462879in}{2.930404in}}{\pgfqpoint{5.462879in}{2.919353in}}%
\pgfpathcurveto{\pgfqpoint{5.462879in}{2.908303in}}{\pgfqpoint{5.467269in}{2.897704in}}{\pgfqpoint{5.475083in}{2.889891in}}%
\pgfpathcurveto{\pgfqpoint{5.482896in}{2.882077in}}{\pgfqpoint{5.493495in}{2.877687in}}{\pgfqpoint{5.504545in}{2.877687in}}%
\pgfpathclose%
\pgfusepath{stroke,fill}%
\end{pgfscope}%
\begin{pgfscope}%
\pgfpathrectangle{\pgfqpoint{0.800000in}{0.528000in}}{\pgfqpoint{4.960000in}{3.696000in}}%
\pgfusepath{clip}%
\pgfsetbuttcap%
\pgfsetroundjoin%
\definecolor{currentfill}{rgb}{0.000000,0.000000,0.000000}%
\pgfsetfillcolor{currentfill}%
\pgfsetlinewidth{1.003750pt}%
\definecolor{currentstroke}{rgb}{0.000000,0.000000,0.000000}%
\pgfsetstrokecolor{currentstroke}%
\pgfsetdash{}{0pt}%
\pgfpathmoveto{\pgfqpoint{5.504545in}{2.877687in}}%
\pgfpathcurveto{\pgfqpoint{5.515596in}{2.877687in}}{\pgfqpoint{5.526195in}{2.882077in}}{\pgfqpoint{5.534008in}{2.889891in}}%
\pgfpathcurveto{\pgfqpoint{5.541822in}{2.897704in}}{\pgfqpoint{5.546212in}{2.908303in}}{\pgfqpoint{5.546212in}{2.919353in}}%
\pgfpathcurveto{\pgfqpoint{5.546212in}{2.930404in}}{\pgfqpoint{5.541822in}{2.941003in}}{\pgfqpoint{5.534008in}{2.948816in}}%
\pgfpathcurveto{\pgfqpoint{5.526195in}{2.956630in}}{\pgfqpoint{5.515596in}{2.961020in}}{\pgfqpoint{5.504545in}{2.961020in}}%
\pgfpathcurveto{\pgfqpoint{5.493495in}{2.961020in}}{\pgfqpoint{5.482896in}{2.956630in}}{\pgfqpoint{5.475083in}{2.948816in}}%
\pgfpathcurveto{\pgfqpoint{5.467269in}{2.941003in}}{\pgfqpoint{5.462879in}{2.930404in}}{\pgfqpoint{5.462879in}{2.919353in}}%
\pgfpathcurveto{\pgfqpoint{5.462879in}{2.908303in}}{\pgfqpoint{5.467269in}{2.897704in}}{\pgfqpoint{5.475083in}{2.889891in}}%
\pgfpathcurveto{\pgfqpoint{5.482896in}{2.882077in}}{\pgfqpoint{5.493495in}{2.877687in}}{\pgfqpoint{5.504545in}{2.877687in}}%
\pgfpathclose%
\pgfusepath{stroke,fill}%
\end{pgfscope}%
\begin{pgfscope}%
\pgfpathrectangle{\pgfqpoint{0.800000in}{0.528000in}}{\pgfqpoint{4.960000in}{3.696000in}}%
\pgfusepath{clip}%
\pgfsetbuttcap%
\pgfsetroundjoin%
\definecolor{currentfill}{rgb}{0.000000,0.000000,0.000000}%
\pgfsetfillcolor{currentfill}%
\pgfsetlinewidth{1.003750pt}%
\definecolor{currentstroke}{rgb}{0.000000,0.000000,0.000000}%
\pgfsetstrokecolor{currentstroke}%
\pgfsetdash{}{0pt}%
\pgfpathmoveto{\pgfqpoint{5.504545in}{2.877687in}}%
\pgfpathcurveto{\pgfqpoint{5.515596in}{2.877687in}}{\pgfqpoint{5.526195in}{2.882077in}}{\pgfqpoint{5.534008in}{2.889891in}}%
\pgfpathcurveto{\pgfqpoint{5.541822in}{2.897704in}}{\pgfqpoint{5.546212in}{2.908303in}}{\pgfqpoint{5.546212in}{2.919353in}}%
\pgfpathcurveto{\pgfqpoint{5.546212in}{2.930404in}}{\pgfqpoint{5.541822in}{2.941003in}}{\pgfqpoint{5.534008in}{2.948816in}}%
\pgfpathcurveto{\pgfqpoint{5.526195in}{2.956630in}}{\pgfqpoint{5.515596in}{2.961020in}}{\pgfqpoint{5.504545in}{2.961020in}}%
\pgfpathcurveto{\pgfqpoint{5.493495in}{2.961020in}}{\pgfqpoint{5.482896in}{2.956630in}}{\pgfqpoint{5.475083in}{2.948816in}}%
\pgfpathcurveto{\pgfqpoint{5.467269in}{2.941003in}}{\pgfqpoint{5.462879in}{2.930404in}}{\pgfqpoint{5.462879in}{2.919353in}}%
\pgfpathcurveto{\pgfqpoint{5.462879in}{2.908303in}}{\pgfqpoint{5.467269in}{2.897704in}}{\pgfqpoint{5.475083in}{2.889891in}}%
\pgfpathcurveto{\pgfqpoint{5.482896in}{2.882077in}}{\pgfqpoint{5.493495in}{2.877687in}}{\pgfqpoint{5.504545in}{2.877687in}}%
\pgfpathclose%
\pgfusepath{stroke,fill}%
\end{pgfscope}%
\begin{pgfscope}%
\pgfpathrectangle{\pgfqpoint{0.800000in}{0.528000in}}{\pgfqpoint{4.960000in}{3.696000in}}%
\pgfusepath{clip}%
\pgfsetbuttcap%
\pgfsetroundjoin%
\definecolor{currentfill}{rgb}{0.000000,0.000000,0.000000}%
\pgfsetfillcolor{currentfill}%
\pgfsetlinewidth{1.003750pt}%
\definecolor{currentstroke}{rgb}{0.000000,0.000000,0.000000}%
\pgfsetstrokecolor{currentstroke}%
\pgfsetdash{}{0pt}%
\pgfpathmoveto{\pgfqpoint{5.504545in}{2.877687in}}%
\pgfpathcurveto{\pgfqpoint{5.515596in}{2.877687in}}{\pgfqpoint{5.526195in}{2.882077in}}{\pgfqpoint{5.534008in}{2.889891in}}%
\pgfpathcurveto{\pgfqpoint{5.541822in}{2.897704in}}{\pgfqpoint{5.546212in}{2.908303in}}{\pgfqpoint{5.546212in}{2.919353in}}%
\pgfpathcurveto{\pgfqpoint{5.546212in}{2.930404in}}{\pgfqpoint{5.541822in}{2.941003in}}{\pgfqpoint{5.534008in}{2.948816in}}%
\pgfpathcurveto{\pgfqpoint{5.526195in}{2.956630in}}{\pgfqpoint{5.515596in}{2.961020in}}{\pgfqpoint{5.504545in}{2.961020in}}%
\pgfpathcurveto{\pgfqpoint{5.493495in}{2.961020in}}{\pgfqpoint{5.482896in}{2.956630in}}{\pgfqpoint{5.475083in}{2.948816in}}%
\pgfpathcurveto{\pgfqpoint{5.467269in}{2.941003in}}{\pgfqpoint{5.462879in}{2.930404in}}{\pgfqpoint{5.462879in}{2.919353in}}%
\pgfpathcurveto{\pgfqpoint{5.462879in}{2.908303in}}{\pgfqpoint{5.467269in}{2.897704in}}{\pgfqpoint{5.475083in}{2.889891in}}%
\pgfpathcurveto{\pgfqpoint{5.482896in}{2.882077in}}{\pgfqpoint{5.493495in}{2.877687in}}{\pgfqpoint{5.504545in}{2.877687in}}%
\pgfpathclose%
\pgfusepath{stroke,fill}%
\end{pgfscope}%
\begin{pgfscope}%
\pgfpathrectangle{\pgfqpoint{0.800000in}{0.528000in}}{\pgfqpoint{4.960000in}{3.696000in}}%
\pgfusepath{clip}%
\pgfsetbuttcap%
\pgfsetroundjoin%
\definecolor{currentfill}{rgb}{0.000000,0.000000,0.000000}%
\pgfsetfillcolor{currentfill}%
\pgfsetlinewidth{1.003750pt}%
\definecolor{currentstroke}{rgb}{0.000000,0.000000,0.000000}%
\pgfsetstrokecolor{currentstroke}%
\pgfsetdash{}{0pt}%
\pgfpathmoveto{\pgfqpoint{5.504545in}{2.877687in}}%
\pgfpathcurveto{\pgfqpoint{5.515596in}{2.877687in}}{\pgfqpoint{5.526195in}{2.882077in}}{\pgfqpoint{5.534008in}{2.889891in}}%
\pgfpathcurveto{\pgfqpoint{5.541822in}{2.897704in}}{\pgfqpoint{5.546212in}{2.908303in}}{\pgfqpoint{5.546212in}{2.919353in}}%
\pgfpathcurveto{\pgfqpoint{5.546212in}{2.930404in}}{\pgfqpoint{5.541822in}{2.941003in}}{\pgfqpoint{5.534008in}{2.948816in}}%
\pgfpathcurveto{\pgfqpoint{5.526195in}{2.956630in}}{\pgfqpoint{5.515596in}{2.961020in}}{\pgfqpoint{5.504545in}{2.961020in}}%
\pgfpathcurveto{\pgfqpoint{5.493495in}{2.961020in}}{\pgfqpoint{5.482896in}{2.956630in}}{\pgfqpoint{5.475083in}{2.948816in}}%
\pgfpathcurveto{\pgfqpoint{5.467269in}{2.941003in}}{\pgfqpoint{5.462879in}{2.930404in}}{\pgfqpoint{5.462879in}{2.919353in}}%
\pgfpathcurveto{\pgfqpoint{5.462879in}{2.908303in}}{\pgfqpoint{5.467269in}{2.897704in}}{\pgfqpoint{5.475083in}{2.889891in}}%
\pgfpathcurveto{\pgfqpoint{5.482896in}{2.882077in}}{\pgfqpoint{5.493495in}{2.877687in}}{\pgfqpoint{5.504545in}{2.877687in}}%
\pgfpathclose%
\pgfusepath{stroke,fill}%
\end{pgfscope}%
\begin{pgfscope}%
\pgfpathrectangle{\pgfqpoint{0.800000in}{0.528000in}}{\pgfqpoint{4.960000in}{3.696000in}}%
\pgfusepath{clip}%
\pgfsetbuttcap%
\pgfsetroundjoin%
\definecolor{currentfill}{rgb}{0.000000,0.000000,0.000000}%
\pgfsetfillcolor{currentfill}%
\pgfsetlinewidth{1.003750pt}%
\definecolor{currentstroke}{rgb}{0.000000,0.000000,0.000000}%
\pgfsetstrokecolor{currentstroke}%
\pgfsetdash{}{0pt}%
\pgfpathmoveto{\pgfqpoint{5.504545in}{2.877687in}}%
\pgfpathcurveto{\pgfqpoint{5.515596in}{2.877687in}}{\pgfqpoint{5.526195in}{2.882077in}}{\pgfqpoint{5.534008in}{2.889891in}}%
\pgfpathcurveto{\pgfqpoint{5.541822in}{2.897704in}}{\pgfqpoint{5.546212in}{2.908303in}}{\pgfqpoint{5.546212in}{2.919353in}}%
\pgfpathcurveto{\pgfqpoint{5.546212in}{2.930404in}}{\pgfqpoint{5.541822in}{2.941003in}}{\pgfqpoint{5.534008in}{2.948816in}}%
\pgfpathcurveto{\pgfqpoint{5.526195in}{2.956630in}}{\pgfqpoint{5.515596in}{2.961020in}}{\pgfqpoint{5.504545in}{2.961020in}}%
\pgfpathcurveto{\pgfqpoint{5.493495in}{2.961020in}}{\pgfqpoint{5.482896in}{2.956630in}}{\pgfqpoint{5.475083in}{2.948816in}}%
\pgfpathcurveto{\pgfqpoint{5.467269in}{2.941003in}}{\pgfqpoint{5.462879in}{2.930404in}}{\pgfqpoint{5.462879in}{2.919353in}}%
\pgfpathcurveto{\pgfqpoint{5.462879in}{2.908303in}}{\pgfqpoint{5.467269in}{2.897704in}}{\pgfqpoint{5.475083in}{2.889891in}}%
\pgfpathcurveto{\pgfqpoint{5.482896in}{2.882077in}}{\pgfqpoint{5.493495in}{2.877687in}}{\pgfqpoint{5.504545in}{2.877687in}}%
\pgfpathclose%
\pgfusepath{stroke,fill}%
\end{pgfscope}%
\begin{pgfscope}%
\pgfpathrectangle{\pgfqpoint{0.800000in}{0.528000in}}{\pgfqpoint{4.960000in}{3.696000in}}%
\pgfusepath{clip}%
\pgfsetbuttcap%
\pgfsetroundjoin%
\definecolor{currentfill}{rgb}{0.000000,0.000000,0.000000}%
\pgfsetfillcolor{currentfill}%
\pgfsetlinewidth{1.003750pt}%
\definecolor{currentstroke}{rgb}{0.000000,0.000000,0.000000}%
\pgfsetstrokecolor{currentstroke}%
\pgfsetdash{}{0pt}%
\pgfpathmoveto{\pgfqpoint{5.504545in}{2.877687in}}%
\pgfpathcurveto{\pgfqpoint{5.515596in}{2.877687in}}{\pgfqpoint{5.526195in}{2.882077in}}{\pgfqpoint{5.534008in}{2.889891in}}%
\pgfpathcurveto{\pgfqpoint{5.541822in}{2.897704in}}{\pgfqpoint{5.546212in}{2.908303in}}{\pgfqpoint{5.546212in}{2.919353in}}%
\pgfpathcurveto{\pgfqpoint{5.546212in}{2.930404in}}{\pgfqpoint{5.541822in}{2.941003in}}{\pgfqpoint{5.534008in}{2.948816in}}%
\pgfpathcurveto{\pgfqpoint{5.526195in}{2.956630in}}{\pgfqpoint{5.515596in}{2.961020in}}{\pgfqpoint{5.504545in}{2.961020in}}%
\pgfpathcurveto{\pgfqpoint{5.493495in}{2.961020in}}{\pgfqpoint{5.482896in}{2.956630in}}{\pgfqpoint{5.475083in}{2.948816in}}%
\pgfpathcurveto{\pgfqpoint{5.467269in}{2.941003in}}{\pgfqpoint{5.462879in}{2.930404in}}{\pgfqpoint{5.462879in}{2.919353in}}%
\pgfpathcurveto{\pgfqpoint{5.462879in}{2.908303in}}{\pgfqpoint{5.467269in}{2.897704in}}{\pgfqpoint{5.475083in}{2.889891in}}%
\pgfpathcurveto{\pgfqpoint{5.482896in}{2.882077in}}{\pgfqpoint{5.493495in}{2.877687in}}{\pgfqpoint{5.504545in}{2.877687in}}%
\pgfpathclose%
\pgfusepath{stroke,fill}%
\end{pgfscope}%
\begin{pgfscope}%
\pgfpathrectangle{\pgfqpoint{0.800000in}{0.528000in}}{\pgfqpoint{4.960000in}{3.696000in}}%
\pgfusepath{clip}%
\pgfsetbuttcap%
\pgfsetroundjoin%
\definecolor{currentfill}{rgb}{0.000000,0.000000,0.000000}%
\pgfsetfillcolor{currentfill}%
\pgfsetlinewidth{1.003750pt}%
\definecolor{currentstroke}{rgb}{0.000000,0.000000,0.000000}%
\pgfsetstrokecolor{currentstroke}%
\pgfsetdash{}{0pt}%
\pgfpathmoveto{\pgfqpoint{5.504545in}{2.877687in}}%
\pgfpathcurveto{\pgfqpoint{5.515596in}{2.877687in}}{\pgfqpoint{5.526195in}{2.882077in}}{\pgfqpoint{5.534008in}{2.889891in}}%
\pgfpathcurveto{\pgfqpoint{5.541822in}{2.897704in}}{\pgfqpoint{5.546212in}{2.908303in}}{\pgfqpoint{5.546212in}{2.919353in}}%
\pgfpathcurveto{\pgfqpoint{5.546212in}{2.930404in}}{\pgfqpoint{5.541822in}{2.941003in}}{\pgfqpoint{5.534008in}{2.948816in}}%
\pgfpathcurveto{\pgfqpoint{5.526195in}{2.956630in}}{\pgfqpoint{5.515596in}{2.961020in}}{\pgfqpoint{5.504545in}{2.961020in}}%
\pgfpathcurveto{\pgfqpoint{5.493495in}{2.961020in}}{\pgfqpoint{5.482896in}{2.956630in}}{\pgfqpoint{5.475083in}{2.948816in}}%
\pgfpathcurveto{\pgfqpoint{5.467269in}{2.941003in}}{\pgfqpoint{5.462879in}{2.930404in}}{\pgfqpoint{5.462879in}{2.919353in}}%
\pgfpathcurveto{\pgfqpoint{5.462879in}{2.908303in}}{\pgfqpoint{5.467269in}{2.897704in}}{\pgfqpoint{5.475083in}{2.889891in}}%
\pgfpathcurveto{\pgfqpoint{5.482896in}{2.882077in}}{\pgfqpoint{5.493495in}{2.877687in}}{\pgfqpoint{5.504545in}{2.877687in}}%
\pgfpathclose%
\pgfusepath{stroke,fill}%
\end{pgfscope}%
\begin{pgfscope}%
\pgfpathrectangle{\pgfqpoint{0.800000in}{0.528000in}}{\pgfqpoint{4.960000in}{3.696000in}}%
\pgfusepath{clip}%
\pgfsetbuttcap%
\pgfsetroundjoin%
\definecolor{currentfill}{rgb}{0.000000,0.000000,0.000000}%
\pgfsetfillcolor{currentfill}%
\pgfsetlinewidth{1.003750pt}%
\definecolor{currentstroke}{rgb}{0.000000,0.000000,0.000000}%
\pgfsetstrokecolor{currentstroke}%
\pgfsetdash{}{0pt}%
\pgfpathmoveto{\pgfqpoint{5.504545in}{2.877687in}}%
\pgfpathcurveto{\pgfqpoint{5.515596in}{2.877687in}}{\pgfqpoint{5.526195in}{2.882077in}}{\pgfqpoint{5.534008in}{2.889891in}}%
\pgfpathcurveto{\pgfqpoint{5.541822in}{2.897704in}}{\pgfqpoint{5.546212in}{2.908303in}}{\pgfqpoint{5.546212in}{2.919353in}}%
\pgfpathcurveto{\pgfqpoint{5.546212in}{2.930404in}}{\pgfqpoint{5.541822in}{2.941003in}}{\pgfqpoint{5.534008in}{2.948816in}}%
\pgfpathcurveto{\pgfqpoint{5.526195in}{2.956630in}}{\pgfqpoint{5.515596in}{2.961020in}}{\pgfqpoint{5.504545in}{2.961020in}}%
\pgfpathcurveto{\pgfqpoint{5.493495in}{2.961020in}}{\pgfqpoint{5.482896in}{2.956630in}}{\pgfqpoint{5.475083in}{2.948816in}}%
\pgfpathcurveto{\pgfqpoint{5.467269in}{2.941003in}}{\pgfqpoint{5.462879in}{2.930404in}}{\pgfqpoint{5.462879in}{2.919353in}}%
\pgfpathcurveto{\pgfqpoint{5.462879in}{2.908303in}}{\pgfqpoint{5.467269in}{2.897704in}}{\pgfqpoint{5.475083in}{2.889891in}}%
\pgfpathcurveto{\pgfqpoint{5.482896in}{2.882077in}}{\pgfqpoint{5.493495in}{2.877687in}}{\pgfqpoint{5.504545in}{2.877687in}}%
\pgfpathclose%
\pgfusepath{stroke,fill}%
\end{pgfscope}%
\begin{pgfscope}%
\pgfpathrectangle{\pgfqpoint{0.800000in}{0.528000in}}{\pgfqpoint{4.960000in}{3.696000in}}%
\pgfusepath{clip}%
\pgfsetbuttcap%
\pgfsetroundjoin%
\definecolor{currentfill}{rgb}{0.000000,0.000000,0.000000}%
\pgfsetfillcolor{currentfill}%
\pgfsetlinewidth{1.003750pt}%
\definecolor{currentstroke}{rgb}{0.000000,0.000000,0.000000}%
\pgfsetstrokecolor{currentstroke}%
\pgfsetdash{}{0pt}%
\pgfpathmoveto{\pgfqpoint{5.504545in}{2.877687in}}%
\pgfpathcurveto{\pgfqpoint{5.515596in}{2.877687in}}{\pgfqpoint{5.526195in}{2.882077in}}{\pgfqpoint{5.534008in}{2.889891in}}%
\pgfpathcurveto{\pgfqpoint{5.541822in}{2.897704in}}{\pgfqpoint{5.546212in}{2.908303in}}{\pgfqpoint{5.546212in}{2.919353in}}%
\pgfpathcurveto{\pgfqpoint{5.546212in}{2.930404in}}{\pgfqpoint{5.541822in}{2.941003in}}{\pgfqpoint{5.534008in}{2.948816in}}%
\pgfpathcurveto{\pgfqpoint{5.526195in}{2.956630in}}{\pgfqpoint{5.515596in}{2.961020in}}{\pgfqpoint{5.504545in}{2.961020in}}%
\pgfpathcurveto{\pgfqpoint{5.493495in}{2.961020in}}{\pgfqpoint{5.482896in}{2.956630in}}{\pgfqpoint{5.475083in}{2.948816in}}%
\pgfpathcurveto{\pgfqpoint{5.467269in}{2.941003in}}{\pgfqpoint{5.462879in}{2.930404in}}{\pgfqpoint{5.462879in}{2.919353in}}%
\pgfpathcurveto{\pgfqpoint{5.462879in}{2.908303in}}{\pgfqpoint{5.467269in}{2.897704in}}{\pgfqpoint{5.475083in}{2.889891in}}%
\pgfpathcurveto{\pgfqpoint{5.482896in}{2.882077in}}{\pgfqpoint{5.493495in}{2.877687in}}{\pgfqpoint{5.504545in}{2.877687in}}%
\pgfpathclose%
\pgfusepath{stroke,fill}%
\end{pgfscope}%
\begin{pgfscope}%
\pgfpathrectangle{\pgfqpoint{0.800000in}{0.528000in}}{\pgfqpoint{4.960000in}{3.696000in}}%
\pgfusepath{clip}%
\pgfsetbuttcap%
\pgfsetroundjoin%
\definecolor{currentfill}{rgb}{0.000000,0.000000,0.000000}%
\pgfsetfillcolor{currentfill}%
\pgfsetlinewidth{1.003750pt}%
\definecolor{currentstroke}{rgb}{0.000000,0.000000,0.000000}%
\pgfsetstrokecolor{currentstroke}%
\pgfsetdash{}{0pt}%
\pgfpathmoveto{\pgfqpoint{5.504545in}{2.877687in}}%
\pgfpathcurveto{\pgfqpoint{5.515596in}{2.877687in}}{\pgfqpoint{5.526195in}{2.882077in}}{\pgfqpoint{5.534008in}{2.889891in}}%
\pgfpathcurveto{\pgfqpoint{5.541822in}{2.897704in}}{\pgfqpoint{5.546212in}{2.908303in}}{\pgfqpoint{5.546212in}{2.919353in}}%
\pgfpathcurveto{\pgfqpoint{5.546212in}{2.930404in}}{\pgfqpoint{5.541822in}{2.941003in}}{\pgfqpoint{5.534008in}{2.948816in}}%
\pgfpathcurveto{\pgfqpoint{5.526195in}{2.956630in}}{\pgfqpoint{5.515596in}{2.961020in}}{\pgfqpoint{5.504545in}{2.961020in}}%
\pgfpathcurveto{\pgfqpoint{5.493495in}{2.961020in}}{\pgfqpoint{5.482896in}{2.956630in}}{\pgfqpoint{5.475083in}{2.948816in}}%
\pgfpathcurveto{\pgfqpoint{5.467269in}{2.941003in}}{\pgfqpoint{5.462879in}{2.930404in}}{\pgfqpoint{5.462879in}{2.919353in}}%
\pgfpathcurveto{\pgfqpoint{5.462879in}{2.908303in}}{\pgfqpoint{5.467269in}{2.897704in}}{\pgfqpoint{5.475083in}{2.889891in}}%
\pgfpathcurveto{\pgfqpoint{5.482896in}{2.882077in}}{\pgfqpoint{5.493495in}{2.877687in}}{\pgfqpoint{5.504545in}{2.877687in}}%
\pgfpathclose%
\pgfusepath{stroke,fill}%
\end{pgfscope}%
\begin{pgfscope}%
\pgfpathrectangle{\pgfqpoint{0.800000in}{0.528000in}}{\pgfqpoint{4.960000in}{3.696000in}}%
\pgfusepath{clip}%
\pgfsetbuttcap%
\pgfsetroundjoin%
\definecolor{currentfill}{rgb}{0.000000,0.000000,0.000000}%
\pgfsetfillcolor{currentfill}%
\pgfsetlinewidth{1.003750pt}%
\definecolor{currentstroke}{rgb}{0.000000,0.000000,0.000000}%
\pgfsetstrokecolor{currentstroke}%
\pgfsetdash{}{0pt}%
\pgfpathmoveto{\pgfqpoint{5.504545in}{2.877687in}}%
\pgfpathcurveto{\pgfqpoint{5.515596in}{2.877687in}}{\pgfqpoint{5.526195in}{2.882077in}}{\pgfqpoint{5.534008in}{2.889891in}}%
\pgfpathcurveto{\pgfqpoint{5.541822in}{2.897704in}}{\pgfqpoint{5.546212in}{2.908303in}}{\pgfqpoint{5.546212in}{2.919353in}}%
\pgfpathcurveto{\pgfqpoint{5.546212in}{2.930404in}}{\pgfqpoint{5.541822in}{2.941003in}}{\pgfqpoint{5.534008in}{2.948816in}}%
\pgfpathcurveto{\pgfqpoint{5.526195in}{2.956630in}}{\pgfqpoint{5.515596in}{2.961020in}}{\pgfqpoint{5.504545in}{2.961020in}}%
\pgfpathcurveto{\pgfqpoint{5.493495in}{2.961020in}}{\pgfqpoint{5.482896in}{2.956630in}}{\pgfqpoint{5.475083in}{2.948816in}}%
\pgfpathcurveto{\pgfqpoint{5.467269in}{2.941003in}}{\pgfqpoint{5.462879in}{2.930404in}}{\pgfqpoint{5.462879in}{2.919353in}}%
\pgfpathcurveto{\pgfqpoint{5.462879in}{2.908303in}}{\pgfqpoint{5.467269in}{2.897704in}}{\pgfqpoint{5.475083in}{2.889891in}}%
\pgfpathcurveto{\pgfqpoint{5.482896in}{2.882077in}}{\pgfqpoint{5.493495in}{2.877687in}}{\pgfqpoint{5.504545in}{2.877687in}}%
\pgfpathclose%
\pgfusepath{stroke,fill}%
\end{pgfscope}%
\begin{pgfscope}%
\pgfpathrectangle{\pgfqpoint{0.800000in}{0.528000in}}{\pgfqpoint{4.960000in}{3.696000in}}%
\pgfusepath{clip}%
\pgfsetbuttcap%
\pgfsetroundjoin%
\definecolor{currentfill}{rgb}{0.000000,0.000000,0.000000}%
\pgfsetfillcolor{currentfill}%
\pgfsetlinewidth{1.003750pt}%
\definecolor{currentstroke}{rgb}{0.000000,0.000000,0.000000}%
\pgfsetstrokecolor{currentstroke}%
\pgfsetdash{}{0pt}%
\pgfpathmoveto{\pgfqpoint{5.504545in}{2.877687in}}%
\pgfpathcurveto{\pgfqpoint{5.515596in}{2.877687in}}{\pgfqpoint{5.526195in}{2.882077in}}{\pgfqpoint{5.534008in}{2.889891in}}%
\pgfpathcurveto{\pgfqpoint{5.541822in}{2.897704in}}{\pgfqpoint{5.546212in}{2.908303in}}{\pgfqpoint{5.546212in}{2.919353in}}%
\pgfpathcurveto{\pgfqpoint{5.546212in}{2.930404in}}{\pgfqpoint{5.541822in}{2.941003in}}{\pgfqpoint{5.534008in}{2.948816in}}%
\pgfpathcurveto{\pgfqpoint{5.526195in}{2.956630in}}{\pgfqpoint{5.515596in}{2.961020in}}{\pgfqpoint{5.504545in}{2.961020in}}%
\pgfpathcurveto{\pgfqpoint{5.493495in}{2.961020in}}{\pgfqpoint{5.482896in}{2.956630in}}{\pgfqpoint{5.475083in}{2.948816in}}%
\pgfpathcurveto{\pgfqpoint{5.467269in}{2.941003in}}{\pgfqpoint{5.462879in}{2.930404in}}{\pgfqpoint{5.462879in}{2.919353in}}%
\pgfpathcurveto{\pgfqpoint{5.462879in}{2.908303in}}{\pgfqpoint{5.467269in}{2.897704in}}{\pgfqpoint{5.475083in}{2.889891in}}%
\pgfpathcurveto{\pgfqpoint{5.482896in}{2.882077in}}{\pgfqpoint{5.493495in}{2.877687in}}{\pgfqpoint{5.504545in}{2.877687in}}%
\pgfpathclose%
\pgfusepath{stroke,fill}%
\end{pgfscope}%
\begin{pgfscope}%
\pgfpathrectangle{\pgfqpoint{0.800000in}{0.528000in}}{\pgfqpoint{4.960000in}{3.696000in}}%
\pgfusepath{clip}%
\pgfsetbuttcap%
\pgfsetroundjoin%
\definecolor{currentfill}{rgb}{0.000000,0.000000,0.000000}%
\pgfsetfillcolor{currentfill}%
\pgfsetlinewidth{1.003750pt}%
\definecolor{currentstroke}{rgb}{0.000000,0.000000,0.000000}%
\pgfsetstrokecolor{currentstroke}%
\pgfsetdash{}{0pt}%
\pgfpathmoveto{\pgfqpoint{5.504545in}{3.984333in}}%
\pgfpathcurveto{\pgfqpoint{5.515596in}{3.984333in}}{\pgfqpoint{5.526195in}{3.988724in}}{\pgfqpoint{5.534008in}{3.996537in}}%
\pgfpathcurveto{\pgfqpoint{5.541822in}{4.004351in}}{\pgfqpoint{5.546212in}{4.014950in}}{\pgfqpoint{5.546212in}{4.026000in}}%
\pgfpathcurveto{\pgfqpoint{5.546212in}{4.037050in}}{\pgfqpoint{5.541822in}{4.047649in}}{\pgfqpoint{5.534008in}{4.055463in}}%
\pgfpathcurveto{\pgfqpoint{5.526195in}{4.063276in}}{\pgfqpoint{5.515596in}{4.067667in}}{\pgfqpoint{5.504545in}{4.067667in}}%
\pgfpathcurveto{\pgfqpoint{5.493495in}{4.067667in}}{\pgfqpoint{5.482896in}{4.063276in}}{\pgfqpoint{5.475083in}{4.055463in}}%
\pgfpathcurveto{\pgfqpoint{5.467269in}{4.047649in}}{\pgfqpoint{5.462879in}{4.037050in}}{\pgfqpoint{5.462879in}{4.026000in}}%
\pgfpathcurveto{\pgfqpoint{5.462879in}{4.014950in}}{\pgfqpoint{5.467269in}{4.004351in}}{\pgfqpoint{5.475083in}{3.996537in}}%
\pgfpathcurveto{\pgfqpoint{5.482896in}{3.988724in}}{\pgfqpoint{5.493495in}{3.984333in}}{\pgfqpoint{5.504545in}{3.984333in}}%
\pgfpathclose%
\pgfusepath{stroke,fill}%
\end{pgfscope}%
\begin{pgfscope}%
\pgfpathrectangle{\pgfqpoint{0.800000in}{0.528000in}}{\pgfqpoint{4.960000in}{3.696000in}}%
\pgfusepath{clip}%
\pgfsetbuttcap%
\pgfsetroundjoin%
\definecolor{currentfill}{rgb}{0.000000,0.000000,0.000000}%
\pgfsetfillcolor{currentfill}%
\pgfsetlinewidth{1.003750pt}%
\definecolor{currentstroke}{rgb}{0.000000,0.000000,0.000000}%
\pgfsetstrokecolor{currentstroke}%
\pgfsetdash{}{0pt}%
\pgfpathmoveto{\pgfqpoint{5.504545in}{2.877687in}}%
\pgfpathcurveto{\pgfqpoint{5.515596in}{2.877687in}}{\pgfqpoint{5.526195in}{2.882077in}}{\pgfqpoint{5.534008in}{2.889891in}}%
\pgfpathcurveto{\pgfqpoint{5.541822in}{2.897704in}}{\pgfqpoint{5.546212in}{2.908303in}}{\pgfqpoint{5.546212in}{2.919353in}}%
\pgfpathcurveto{\pgfqpoint{5.546212in}{2.930404in}}{\pgfqpoint{5.541822in}{2.941003in}}{\pgfqpoint{5.534008in}{2.948816in}}%
\pgfpathcurveto{\pgfqpoint{5.526195in}{2.956630in}}{\pgfqpoint{5.515596in}{2.961020in}}{\pgfqpoint{5.504545in}{2.961020in}}%
\pgfpathcurveto{\pgfqpoint{5.493495in}{2.961020in}}{\pgfqpoint{5.482896in}{2.956630in}}{\pgfqpoint{5.475083in}{2.948816in}}%
\pgfpathcurveto{\pgfqpoint{5.467269in}{2.941003in}}{\pgfqpoint{5.462879in}{2.930404in}}{\pgfqpoint{5.462879in}{2.919353in}}%
\pgfpathcurveto{\pgfqpoint{5.462879in}{2.908303in}}{\pgfqpoint{5.467269in}{2.897704in}}{\pgfqpoint{5.475083in}{2.889891in}}%
\pgfpathcurveto{\pgfqpoint{5.482896in}{2.882077in}}{\pgfqpoint{5.493495in}{2.877687in}}{\pgfqpoint{5.504545in}{2.877687in}}%
\pgfpathclose%
\pgfusepath{stroke,fill}%
\end{pgfscope}%
\begin{pgfscope}%
\pgfpathrectangle{\pgfqpoint{0.800000in}{0.528000in}}{\pgfqpoint{4.960000in}{3.696000in}}%
\pgfusepath{clip}%
\pgfsetbuttcap%
\pgfsetroundjoin%
\definecolor{currentfill}{rgb}{0.000000,0.000000,0.000000}%
\pgfsetfillcolor{currentfill}%
\pgfsetlinewidth{1.003750pt}%
\definecolor{currentstroke}{rgb}{0.000000,0.000000,0.000000}%
\pgfsetstrokecolor{currentstroke}%
\pgfsetdash{}{0pt}%
\pgfpathmoveto{\pgfqpoint{5.504545in}{2.877687in}}%
\pgfpathcurveto{\pgfqpoint{5.515596in}{2.877687in}}{\pgfqpoint{5.526195in}{2.882077in}}{\pgfqpoint{5.534008in}{2.889891in}}%
\pgfpathcurveto{\pgfqpoint{5.541822in}{2.897704in}}{\pgfqpoint{5.546212in}{2.908303in}}{\pgfqpoint{5.546212in}{2.919353in}}%
\pgfpathcurveto{\pgfqpoint{5.546212in}{2.930404in}}{\pgfqpoint{5.541822in}{2.941003in}}{\pgfqpoint{5.534008in}{2.948816in}}%
\pgfpathcurveto{\pgfqpoint{5.526195in}{2.956630in}}{\pgfqpoint{5.515596in}{2.961020in}}{\pgfqpoint{5.504545in}{2.961020in}}%
\pgfpathcurveto{\pgfqpoint{5.493495in}{2.961020in}}{\pgfqpoint{5.482896in}{2.956630in}}{\pgfqpoint{5.475083in}{2.948816in}}%
\pgfpathcurveto{\pgfqpoint{5.467269in}{2.941003in}}{\pgfqpoint{5.462879in}{2.930404in}}{\pgfqpoint{5.462879in}{2.919353in}}%
\pgfpathcurveto{\pgfqpoint{5.462879in}{2.908303in}}{\pgfqpoint{5.467269in}{2.897704in}}{\pgfqpoint{5.475083in}{2.889891in}}%
\pgfpathcurveto{\pgfqpoint{5.482896in}{2.882077in}}{\pgfqpoint{5.493495in}{2.877687in}}{\pgfqpoint{5.504545in}{2.877687in}}%
\pgfpathclose%
\pgfusepath{stroke,fill}%
\end{pgfscope}%
\begin{pgfscope}%
\pgfpathrectangle{\pgfqpoint{0.800000in}{0.528000in}}{\pgfqpoint{4.960000in}{3.696000in}}%
\pgfusepath{clip}%
\pgfsetbuttcap%
\pgfsetroundjoin%
\definecolor{currentfill}{rgb}{0.000000,0.000000,0.000000}%
\pgfsetfillcolor{currentfill}%
\pgfsetlinewidth{1.003750pt}%
\definecolor{currentstroke}{rgb}{0.000000,0.000000,0.000000}%
\pgfsetstrokecolor{currentstroke}%
\pgfsetdash{}{0pt}%
\pgfpathmoveto{\pgfqpoint{5.504545in}{3.984333in}}%
\pgfpathcurveto{\pgfqpoint{5.515596in}{3.984333in}}{\pgfqpoint{5.526195in}{3.988724in}}{\pgfqpoint{5.534008in}{3.996537in}}%
\pgfpathcurveto{\pgfqpoint{5.541822in}{4.004351in}}{\pgfqpoint{5.546212in}{4.014950in}}{\pgfqpoint{5.546212in}{4.026000in}}%
\pgfpathcurveto{\pgfqpoint{5.546212in}{4.037050in}}{\pgfqpoint{5.541822in}{4.047649in}}{\pgfqpoint{5.534008in}{4.055463in}}%
\pgfpathcurveto{\pgfqpoint{5.526195in}{4.063276in}}{\pgfqpoint{5.515596in}{4.067667in}}{\pgfqpoint{5.504545in}{4.067667in}}%
\pgfpathcurveto{\pgfqpoint{5.493495in}{4.067667in}}{\pgfqpoint{5.482896in}{4.063276in}}{\pgfqpoint{5.475083in}{4.055463in}}%
\pgfpathcurveto{\pgfqpoint{5.467269in}{4.047649in}}{\pgfqpoint{5.462879in}{4.037050in}}{\pgfqpoint{5.462879in}{4.026000in}}%
\pgfpathcurveto{\pgfqpoint{5.462879in}{4.014950in}}{\pgfqpoint{5.467269in}{4.004351in}}{\pgfqpoint{5.475083in}{3.996537in}}%
\pgfpathcurveto{\pgfqpoint{5.482896in}{3.988724in}}{\pgfqpoint{5.493495in}{3.984333in}}{\pgfqpoint{5.504545in}{3.984333in}}%
\pgfpathclose%
\pgfusepath{stroke,fill}%
\end{pgfscope}%
\begin{pgfscope}%
\pgfpathrectangle{\pgfqpoint{0.800000in}{0.528000in}}{\pgfqpoint{4.960000in}{3.696000in}}%
\pgfusepath{clip}%
\pgfsetbuttcap%
\pgfsetroundjoin%
\definecolor{currentfill}{rgb}{0.000000,0.000000,0.000000}%
\pgfsetfillcolor{currentfill}%
\pgfsetlinewidth{1.003750pt}%
\definecolor{currentstroke}{rgb}{0.000000,0.000000,0.000000}%
\pgfsetstrokecolor{currentstroke}%
\pgfsetdash{}{0pt}%
\pgfpathmoveto{\pgfqpoint{5.504545in}{2.877687in}}%
\pgfpathcurveto{\pgfqpoint{5.515596in}{2.877687in}}{\pgfqpoint{5.526195in}{2.882077in}}{\pgfqpoint{5.534008in}{2.889891in}}%
\pgfpathcurveto{\pgfqpoint{5.541822in}{2.897704in}}{\pgfqpoint{5.546212in}{2.908303in}}{\pgfqpoint{5.546212in}{2.919353in}}%
\pgfpathcurveto{\pgfqpoint{5.546212in}{2.930404in}}{\pgfqpoint{5.541822in}{2.941003in}}{\pgfqpoint{5.534008in}{2.948816in}}%
\pgfpathcurveto{\pgfqpoint{5.526195in}{2.956630in}}{\pgfqpoint{5.515596in}{2.961020in}}{\pgfqpoint{5.504545in}{2.961020in}}%
\pgfpathcurveto{\pgfqpoint{5.493495in}{2.961020in}}{\pgfqpoint{5.482896in}{2.956630in}}{\pgfqpoint{5.475083in}{2.948816in}}%
\pgfpathcurveto{\pgfqpoint{5.467269in}{2.941003in}}{\pgfqpoint{5.462879in}{2.930404in}}{\pgfqpoint{5.462879in}{2.919353in}}%
\pgfpathcurveto{\pgfqpoint{5.462879in}{2.908303in}}{\pgfqpoint{5.467269in}{2.897704in}}{\pgfqpoint{5.475083in}{2.889891in}}%
\pgfpathcurveto{\pgfqpoint{5.482896in}{2.882077in}}{\pgfqpoint{5.493495in}{2.877687in}}{\pgfqpoint{5.504545in}{2.877687in}}%
\pgfpathclose%
\pgfusepath{stroke,fill}%
\end{pgfscope}%
\begin{pgfscope}%
\pgfpathrectangle{\pgfqpoint{0.800000in}{0.528000in}}{\pgfqpoint{4.960000in}{3.696000in}}%
\pgfusepath{clip}%
\pgfsetbuttcap%
\pgfsetroundjoin%
\definecolor{currentfill}{rgb}{0.000000,0.000000,0.000000}%
\pgfsetfillcolor{currentfill}%
\pgfsetlinewidth{1.003750pt}%
\definecolor{currentstroke}{rgb}{0.000000,0.000000,0.000000}%
\pgfsetstrokecolor{currentstroke}%
\pgfsetdash{}{0pt}%
\pgfpathmoveto{\pgfqpoint{5.504545in}{2.877687in}}%
\pgfpathcurveto{\pgfqpoint{5.515596in}{2.877687in}}{\pgfqpoint{5.526195in}{2.882077in}}{\pgfqpoint{5.534008in}{2.889891in}}%
\pgfpathcurveto{\pgfqpoint{5.541822in}{2.897704in}}{\pgfqpoint{5.546212in}{2.908303in}}{\pgfqpoint{5.546212in}{2.919353in}}%
\pgfpathcurveto{\pgfqpoint{5.546212in}{2.930404in}}{\pgfqpoint{5.541822in}{2.941003in}}{\pgfqpoint{5.534008in}{2.948816in}}%
\pgfpathcurveto{\pgfqpoint{5.526195in}{2.956630in}}{\pgfqpoint{5.515596in}{2.961020in}}{\pgfqpoint{5.504545in}{2.961020in}}%
\pgfpathcurveto{\pgfqpoint{5.493495in}{2.961020in}}{\pgfqpoint{5.482896in}{2.956630in}}{\pgfqpoint{5.475083in}{2.948816in}}%
\pgfpathcurveto{\pgfqpoint{5.467269in}{2.941003in}}{\pgfqpoint{5.462879in}{2.930404in}}{\pgfqpoint{5.462879in}{2.919353in}}%
\pgfpathcurveto{\pgfqpoint{5.462879in}{2.908303in}}{\pgfqpoint{5.467269in}{2.897704in}}{\pgfqpoint{5.475083in}{2.889891in}}%
\pgfpathcurveto{\pgfqpoint{5.482896in}{2.882077in}}{\pgfqpoint{5.493495in}{2.877687in}}{\pgfqpoint{5.504545in}{2.877687in}}%
\pgfpathclose%
\pgfusepath{stroke,fill}%
\end{pgfscope}%
\begin{pgfscope}%
\pgfpathrectangle{\pgfqpoint{0.800000in}{0.528000in}}{\pgfqpoint{4.960000in}{3.696000in}}%
\pgfusepath{clip}%
\pgfsetbuttcap%
\pgfsetroundjoin%
\definecolor{currentfill}{rgb}{0.000000,0.000000,0.000000}%
\pgfsetfillcolor{currentfill}%
\pgfsetlinewidth{1.003750pt}%
\definecolor{currentstroke}{rgb}{0.000000,0.000000,0.000000}%
\pgfsetstrokecolor{currentstroke}%
\pgfsetdash{}{0pt}%
\pgfpathmoveto{\pgfqpoint{5.504545in}{2.877687in}}%
\pgfpathcurveto{\pgfqpoint{5.515596in}{2.877687in}}{\pgfqpoint{5.526195in}{2.882077in}}{\pgfqpoint{5.534008in}{2.889891in}}%
\pgfpathcurveto{\pgfqpoint{5.541822in}{2.897704in}}{\pgfqpoint{5.546212in}{2.908303in}}{\pgfqpoint{5.546212in}{2.919353in}}%
\pgfpathcurveto{\pgfqpoint{5.546212in}{2.930404in}}{\pgfqpoint{5.541822in}{2.941003in}}{\pgfqpoint{5.534008in}{2.948816in}}%
\pgfpathcurveto{\pgfqpoint{5.526195in}{2.956630in}}{\pgfqpoint{5.515596in}{2.961020in}}{\pgfqpoint{5.504545in}{2.961020in}}%
\pgfpathcurveto{\pgfqpoint{5.493495in}{2.961020in}}{\pgfqpoint{5.482896in}{2.956630in}}{\pgfqpoint{5.475083in}{2.948816in}}%
\pgfpathcurveto{\pgfqpoint{5.467269in}{2.941003in}}{\pgfqpoint{5.462879in}{2.930404in}}{\pgfqpoint{5.462879in}{2.919353in}}%
\pgfpathcurveto{\pgfqpoint{5.462879in}{2.908303in}}{\pgfqpoint{5.467269in}{2.897704in}}{\pgfqpoint{5.475083in}{2.889891in}}%
\pgfpathcurveto{\pgfqpoint{5.482896in}{2.882077in}}{\pgfqpoint{5.493495in}{2.877687in}}{\pgfqpoint{5.504545in}{2.877687in}}%
\pgfpathclose%
\pgfusepath{stroke,fill}%
\end{pgfscope}%
\begin{pgfscope}%
\pgfpathrectangle{\pgfqpoint{0.800000in}{0.528000in}}{\pgfqpoint{4.960000in}{3.696000in}}%
\pgfusepath{clip}%
\pgfsetbuttcap%
\pgfsetroundjoin%
\definecolor{currentfill}{rgb}{0.000000,0.000000,0.000000}%
\pgfsetfillcolor{currentfill}%
\pgfsetlinewidth{1.003750pt}%
\definecolor{currentstroke}{rgb}{0.000000,0.000000,0.000000}%
\pgfsetstrokecolor{currentstroke}%
\pgfsetdash{}{0pt}%
\pgfpathmoveto{\pgfqpoint{5.504545in}{2.877687in}}%
\pgfpathcurveto{\pgfqpoint{5.515596in}{2.877687in}}{\pgfqpoint{5.526195in}{2.882077in}}{\pgfqpoint{5.534008in}{2.889891in}}%
\pgfpathcurveto{\pgfqpoint{5.541822in}{2.897704in}}{\pgfqpoint{5.546212in}{2.908303in}}{\pgfqpoint{5.546212in}{2.919353in}}%
\pgfpathcurveto{\pgfqpoint{5.546212in}{2.930404in}}{\pgfqpoint{5.541822in}{2.941003in}}{\pgfqpoint{5.534008in}{2.948816in}}%
\pgfpathcurveto{\pgfqpoint{5.526195in}{2.956630in}}{\pgfqpoint{5.515596in}{2.961020in}}{\pgfqpoint{5.504545in}{2.961020in}}%
\pgfpathcurveto{\pgfqpoint{5.493495in}{2.961020in}}{\pgfqpoint{5.482896in}{2.956630in}}{\pgfqpoint{5.475083in}{2.948816in}}%
\pgfpathcurveto{\pgfqpoint{5.467269in}{2.941003in}}{\pgfqpoint{5.462879in}{2.930404in}}{\pgfqpoint{5.462879in}{2.919353in}}%
\pgfpathcurveto{\pgfqpoint{5.462879in}{2.908303in}}{\pgfqpoint{5.467269in}{2.897704in}}{\pgfqpoint{5.475083in}{2.889891in}}%
\pgfpathcurveto{\pgfqpoint{5.482896in}{2.882077in}}{\pgfqpoint{5.493495in}{2.877687in}}{\pgfqpoint{5.504545in}{2.877687in}}%
\pgfpathclose%
\pgfusepath{stroke,fill}%
\end{pgfscope}%
\begin{pgfscope}%
\pgfpathrectangle{\pgfqpoint{0.800000in}{0.528000in}}{\pgfqpoint{4.960000in}{3.696000in}}%
\pgfusepath{clip}%
\pgfsetbuttcap%
\pgfsetroundjoin%
\definecolor{currentfill}{rgb}{0.000000,0.000000,0.000000}%
\pgfsetfillcolor{currentfill}%
\pgfsetlinewidth{1.003750pt}%
\definecolor{currentstroke}{rgb}{0.000000,0.000000,0.000000}%
\pgfsetstrokecolor{currentstroke}%
\pgfsetdash{}{0pt}%
\pgfpathmoveto{\pgfqpoint{5.504545in}{2.877687in}}%
\pgfpathcurveto{\pgfqpoint{5.515596in}{2.877687in}}{\pgfqpoint{5.526195in}{2.882077in}}{\pgfqpoint{5.534008in}{2.889891in}}%
\pgfpathcurveto{\pgfqpoint{5.541822in}{2.897704in}}{\pgfqpoint{5.546212in}{2.908303in}}{\pgfqpoint{5.546212in}{2.919353in}}%
\pgfpathcurveto{\pgfqpoint{5.546212in}{2.930404in}}{\pgfqpoint{5.541822in}{2.941003in}}{\pgfqpoint{5.534008in}{2.948816in}}%
\pgfpathcurveto{\pgfqpoint{5.526195in}{2.956630in}}{\pgfqpoint{5.515596in}{2.961020in}}{\pgfqpoint{5.504545in}{2.961020in}}%
\pgfpathcurveto{\pgfqpoint{5.493495in}{2.961020in}}{\pgfqpoint{5.482896in}{2.956630in}}{\pgfqpoint{5.475083in}{2.948816in}}%
\pgfpathcurveto{\pgfqpoint{5.467269in}{2.941003in}}{\pgfqpoint{5.462879in}{2.930404in}}{\pgfqpoint{5.462879in}{2.919353in}}%
\pgfpathcurveto{\pgfqpoint{5.462879in}{2.908303in}}{\pgfqpoint{5.467269in}{2.897704in}}{\pgfqpoint{5.475083in}{2.889891in}}%
\pgfpathcurveto{\pgfqpoint{5.482896in}{2.882077in}}{\pgfqpoint{5.493495in}{2.877687in}}{\pgfqpoint{5.504545in}{2.877687in}}%
\pgfpathclose%
\pgfusepath{stroke,fill}%
\end{pgfscope}%
\begin{pgfscope}%
\pgfpathrectangle{\pgfqpoint{0.800000in}{0.528000in}}{\pgfqpoint{4.960000in}{3.696000in}}%
\pgfusepath{clip}%
\pgfsetbuttcap%
\pgfsetroundjoin%
\definecolor{currentfill}{rgb}{0.000000,0.000000,0.000000}%
\pgfsetfillcolor{currentfill}%
\pgfsetlinewidth{1.003750pt}%
\definecolor{currentstroke}{rgb}{0.000000,0.000000,0.000000}%
\pgfsetstrokecolor{currentstroke}%
\pgfsetdash{}{0pt}%
\pgfpathmoveto{\pgfqpoint{5.504545in}{2.877687in}}%
\pgfpathcurveto{\pgfqpoint{5.515596in}{2.877687in}}{\pgfqpoint{5.526195in}{2.882077in}}{\pgfqpoint{5.534008in}{2.889891in}}%
\pgfpathcurveto{\pgfqpoint{5.541822in}{2.897704in}}{\pgfqpoint{5.546212in}{2.908303in}}{\pgfqpoint{5.546212in}{2.919353in}}%
\pgfpathcurveto{\pgfqpoint{5.546212in}{2.930404in}}{\pgfqpoint{5.541822in}{2.941003in}}{\pgfqpoint{5.534008in}{2.948816in}}%
\pgfpathcurveto{\pgfqpoint{5.526195in}{2.956630in}}{\pgfqpoint{5.515596in}{2.961020in}}{\pgfqpoint{5.504545in}{2.961020in}}%
\pgfpathcurveto{\pgfqpoint{5.493495in}{2.961020in}}{\pgfqpoint{5.482896in}{2.956630in}}{\pgfqpoint{5.475083in}{2.948816in}}%
\pgfpathcurveto{\pgfqpoint{5.467269in}{2.941003in}}{\pgfqpoint{5.462879in}{2.930404in}}{\pgfqpoint{5.462879in}{2.919353in}}%
\pgfpathcurveto{\pgfqpoint{5.462879in}{2.908303in}}{\pgfqpoint{5.467269in}{2.897704in}}{\pgfqpoint{5.475083in}{2.889891in}}%
\pgfpathcurveto{\pgfqpoint{5.482896in}{2.882077in}}{\pgfqpoint{5.493495in}{2.877687in}}{\pgfqpoint{5.504545in}{2.877687in}}%
\pgfpathclose%
\pgfusepath{stroke,fill}%
\end{pgfscope}%
\begin{pgfscope}%
\pgfpathrectangle{\pgfqpoint{0.800000in}{0.528000in}}{\pgfqpoint{4.960000in}{3.696000in}}%
\pgfusepath{clip}%
\pgfsetbuttcap%
\pgfsetroundjoin%
\definecolor{currentfill}{rgb}{0.000000,0.000000,0.000000}%
\pgfsetfillcolor{currentfill}%
\pgfsetlinewidth{1.003750pt}%
\definecolor{currentstroke}{rgb}{0.000000,0.000000,0.000000}%
\pgfsetstrokecolor{currentstroke}%
\pgfsetdash{}{0pt}%
\pgfpathmoveto{\pgfqpoint{5.504545in}{2.877687in}}%
\pgfpathcurveto{\pgfqpoint{5.515596in}{2.877687in}}{\pgfqpoint{5.526195in}{2.882077in}}{\pgfqpoint{5.534008in}{2.889891in}}%
\pgfpathcurveto{\pgfqpoint{5.541822in}{2.897704in}}{\pgfqpoint{5.546212in}{2.908303in}}{\pgfqpoint{5.546212in}{2.919353in}}%
\pgfpathcurveto{\pgfqpoint{5.546212in}{2.930404in}}{\pgfqpoint{5.541822in}{2.941003in}}{\pgfqpoint{5.534008in}{2.948816in}}%
\pgfpathcurveto{\pgfqpoint{5.526195in}{2.956630in}}{\pgfqpoint{5.515596in}{2.961020in}}{\pgfqpoint{5.504545in}{2.961020in}}%
\pgfpathcurveto{\pgfqpoint{5.493495in}{2.961020in}}{\pgfqpoint{5.482896in}{2.956630in}}{\pgfqpoint{5.475083in}{2.948816in}}%
\pgfpathcurveto{\pgfqpoint{5.467269in}{2.941003in}}{\pgfqpoint{5.462879in}{2.930404in}}{\pgfqpoint{5.462879in}{2.919353in}}%
\pgfpathcurveto{\pgfqpoint{5.462879in}{2.908303in}}{\pgfqpoint{5.467269in}{2.897704in}}{\pgfqpoint{5.475083in}{2.889891in}}%
\pgfpathcurveto{\pgfqpoint{5.482896in}{2.882077in}}{\pgfqpoint{5.493495in}{2.877687in}}{\pgfqpoint{5.504545in}{2.877687in}}%
\pgfpathclose%
\pgfusepath{stroke,fill}%
\end{pgfscope}%
\begin{pgfscope}%
\pgfpathrectangle{\pgfqpoint{0.800000in}{0.528000in}}{\pgfqpoint{4.960000in}{3.696000in}}%
\pgfusepath{clip}%
\pgfsetbuttcap%
\pgfsetroundjoin%
\definecolor{currentfill}{rgb}{0.000000,0.000000,0.000000}%
\pgfsetfillcolor{currentfill}%
\pgfsetlinewidth{1.003750pt}%
\definecolor{currentstroke}{rgb}{0.000000,0.000000,0.000000}%
\pgfsetstrokecolor{currentstroke}%
\pgfsetdash{}{0pt}%
\pgfpathmoveto{\pgfqpoint{5.504545in}{2.877687in}}%
\pgfpathcurveto{\pgfqpoint{5.515596in}{2.877687in}}{\pgfqpoint{5.526195in}{2.882077in}}{\pgfqpoint{5.534008in}{2.889891in}}%
\pgfpathcurveto{\pgfqpoint{5.541822in}{2.897704in}}{\pgfqpoint{5.546212in}{2.908303in}}{\pgfqpoint{5.546212in}{2.919353in}}%
\pgfpathcurveto{\pgfqpoint{5.546212in}{2.930404in}}{\pgfqpoint{5.541822in}{2.941003in}}{\pgfqpoint{5.534008in}{2.948816in}}%
\pgfpathcurveto{\pgfqpoint{5.526195in}{2.956630in}}{\pgfqpoint{5.515596in}{2.961020in}}{\pgfqpoint{5.504545in}{2.961020in}}%
\pgfpathcurveto{\pgfqpoint{5.493495in}{2.961020in}}{\pgfqpoint{5.482896in}{2.956630in}}{\pgfqpoint{5.475083in}{2.948816in}}%
\pgfpathcurveto{\pgfqpoint{5.467269in}{2.941003in}}{\pgfqpoint{5.462879in}{2.930404in}}{\pgfqpoint{5.462879in}{2.919353in}}%
\pgfpathcurveto{\pgfqpoint{5.462879in}{2.908303in}}{\pgfqpoint{5.467269in}{2.897704in}}{\pgfqpoint{5.475083in}{2.889891in}}%
\pgfpathcurveto{\pgfqpoint{5.482896in}{2.882077in}}{\pgfqpoint{5.493495in}{2.877687in}}{\pgfqpoint{5.504545in}{2.877687in}}%
\pgfpathclose%
\pgfusepath{stroke,fill}%
\end{pgfscope}%
\begin{pgfscope}%
\pgfpathrectangle{\pgfqpoint{0.800000in}{0.528000in}}{\pgfqpoint{4.960000in}{3.696000in}}%
\pgfusepath{clip}%
\pgfsetbuttcap%
\pgfsetroundjoin%
\definecolor{currentfill}{rgb}{0.000000,0.000000,0.000000}%
\pgfsetfillcolor{currentfill}%
\pgfsetlinewidth{1.003750pt}%
\definecolor{currentstroke}{rgb}{0.000000,0.000000,0.000000}%
\pgfsetstrokecolor{currentstroke}%
\pgfsetdash{}{0pt}%
\pgfpathmoveto{\pgfqpoint{5.504545in}{2.877687in}}%
\pgfpathcurveto{\pgfqpoint{5.515596in}{2.877687in}}{\pgfqpoint{5.526195in}{2.882077in}}{\pgfqpoint{5.534008in}{2.889891in}}%
\pgfpathcurveto{\pgfqpoint{5.541822in}{2.897704in}}{\pgfqpoint{5.546212in}{2.908303in}}{\pgfqpoint{5.546212in}{2.919353in}}%
\pgfpathcurveto{\pgfqpoint{5.546212in}{2.930404in}}{\pgfqpoint{5.541822in}{2.941003in}}{\pgfqpoint{5.534008in}{2.948816in}}%
\pgfpathcurveto{\pgfqpoint{5.526195in}{2.956630in}}{\pgfqpoint{5.515596in}{2.961020in}}{\pgfqpoint{5.504545in}{2.961020in}}%
\pgfpathcurveto{\pgfqpoint{5.493495in}{2.961020in}}{\pgfqpoint{5.482896in}{2.956630in}}{\pgfqpoint{5.475083in}{2.948816in}}%
\pgfpathcurveto{\pgfqpoint{5.467269in}{2.941003in}}{\pgfqpoint{5.462879in}{2.930404in}}{\pgfqpoint{5.462879in}{2.919353in}}%
\pgfpathcurveto{\pgfqpoint{5.462879in}{2.908303in}}{\pgfqpoint{5.467269in}{2.897704in}}{\pgfqpoint{5.475083in}{2.889891in}}%
\pgfpathcurveto{\pgfqpoint{5.482896in}{2.882077in}}{\pgfqpoint{5.493495in}{2.877687in}}{\pgfqpoint{5.504545in}{2.877687in}}%
\pgfpathclose%
\pgfusepath{stroke,fill}%
\end{pgfscope}%
\begin{pgfscope}%
\pgfpathrectangle{\pgfqpoint{0.800000in}{0.528000in}}{\pgfqpoint{4.960000in}{3.696000in}}%
\pgfusepath{clip}%
\pgfsetbuttcap%
\pgfsetroundjoin%
\definecolor{currentfill}{rgb}{0.000000,0.000000,0.000000}%
\pgfsetfillcolor{currentfill}%
\pgfsetlinewidth{1.003750pt}%
\definecolor{currentstroke}{rgb}{0.000000,0.000000,0.000000}%
\pgfsetstrokecolor{currentstroke}%
\pgfsetdash{}{0pt}%
\pgfpathmoveto{\pgfqpoint{5.504545in}{2.877687in}}%
\pgfpathcurveto{\pgfqpoint{5.515596in}{2.877687in}}{\pgfqpoint{5.526195in}{2.882077in}}{\pgfqpoint{5.534008in}{2.889891in}}%
\pgfpathcurveto{\pgfqpoint{5.541822in}{2.897704in}}{\pgfqpoint{5.546212in}{2.908303in}}{\pgfqpoint{5.546212in}{2.919353in}}%
\pgfpathcurveto{\pgfqpoint{5.546212in}{2.930404in}}{\pgfqpoint{5.541822in}{2.941003in}}{\pgfqpoint{5.534008in}{2.948816in}}%
\pgfpathcurveto{\pgfqpoint{5.526195in}{2.956630in}}{\pgfqpoint{5.515596in}{2.961020in}}{\pgfqpoint{5.504545in}{2.961020in}}%
\pgfpathcurveto{\pgfqpoint{5.493495in}{2.961020in}}{\pgfqpoint{5.482896in}{2.956630in}}{\pgfqpoint{5.475083in}{2.948816in}}%
\pgfpathcurveto{\pgfqpoint{5.467269in}{2.941003in}}{\pgfqpoint{5.462879in}{2.930404in}}{\pgfqpoint{5.462879in}{2.919353in}}%
\pgfpathcurveto{\pgfqpoint{5.462879in}{2.908303in}}{\pgfqpoint{5.467269in}{2.897704in}}{\pgfqpoint{5.475083in}{2.889891in}}%
\pgfpathcurveto{\pgfqpoint{5.482896in}{2.882077in}}{\pgfqpoint{5.493495in}{2.877687in}}{\pgfqpoint{5.504545in}{2.877687in}}%
\pgfpathclose%
\pgfusepath{stroke,fill}%
\end{pgfscope}%
\begin{pgfscope}%
\pgfpathrectangle{\pgfqpoint{0.800000in}{0.528000in}}{\pgfqpoint{4.960000in}{3.696000in}}%
\pgfusepath{clip}%
\pgfsetbuttcap%
\pgfsetroundjoin%
\definecolor{currentfill}{rgb}{0.000000,0.000000,0.000000}%
\pgfsetfillcolor{currentfill}%
\pgfsetlinewidth{1.003750pt}%
\definecolor{currentstroke}{rgb}{0.000000,0.000000,0.000000}%
\pgfsetstrokecolor{currentstroke}%
\pgfsetdash{}{0pt}%
\pgfpathmoveto{\pgfqpoint{5.504545in}{2.877687in}}%
\pgfpathcurveto{\pgfqpoint{5.515596in}{2.877687in}}{\pgfqpoint{5.526195in}{2.882077in}}{\pgfqpoint{5.534008in}{2.889891in}}%
\pgfpathcurveto{\pgfqpoint{5.541822in}{2.897704in}}{\pgfqpoint{5.546212in}{2.908303in}}{\pgfqpoint{5.546212in}{2.919353in}}%
\pgfpathcurveto{\pgfqpoint{5.546212in}{2.930404in}}{\pgfqpoint{5.541822in}{2.941003in}}{\pgfqpoint{5.534008in}{2.948816in}}%
\pgfpathcurveto{\pgfqpoint{5.526195in}{2.956630in}}{\pgfqpoint{5.515596in}{2.961020in}}{\pgfqpoint{5.504545in}{2.961020in}}%
\pgfpathcurveto{\pgfqpoint{5.493495in}{2.961020in}}{\pgfqpoint{5.482896in}{2.956630in}}{\pgfqpoint{5.475083in}{2.948816in}}%
\pgfpathcurveto{\pgfqpoint{5.467269in}{2.941003in}}{\pgfqpoint{5.462879in}{2.930404in}}{\pgfqpoint{5.462879in}{2.919353in}}%
\pgfpathcurveto{\pgfqpoint{5.462879in}{2.908303in}}{\pgfqpoint{5.467269in}{2.897704in}}{\pgfqpoint{5.475083in}{2.889891in}}%
\pgfpathcurveto{\pgfqpoint{5.482896in}{2.882077in}}{\pgfqpoint{5.493495in}{2.877687in}}{\pgfqpoint{5.504545in}{2.877687in}}%
\pgfpathclose%
\pgfusepath{stroke,fill}%
\end{pgfscope}%
\begin{pgfscope}%
\pgfpathrectangle{\pgfqpoint{0.800000in}{0.528000in}}{\pgfqpoint{4.960000in}{3.696000in}}%
\pgfusepath{clip}%
\pgfsetbuttcap%
\pgfsetroundjoin%
\definecolor{currentfill}{rgb}{0.000000,0.000000,0.000000}%
\pgfsetfillcolor{currentfill}%
\pgfsetlinewidth{1.003750pt}%
\definecolor{currentstroke}{rgb}{0.000000,0.000000,0.000000}%
\pgfsetstrokecolor{currentstroke}%
\pgfsetdash{}{0pt}%
\pgfpathmoveto{\pgfqpoint{5.504545in}{2.877687in}}%
\pgfpathcurveto{\pgfqpoint{5.515596in}{2.877687in}}{\pgfqpoint{5.526195in}{2.882077in}}{\pgfqpoint{5.534008in}{2.889891in}}%
\pgfpathcurveto{\pgfqpoint{5.541822in}{2.897704in}}{\pgfqpoint{5.546212in}{2.908303in}}{\pgfqpoint{5.546212in}{2.919353in}}%
\pgfpathcurveto{\pgfqpoint{5.546212in}{2.930404in}}{\pgfqpoint{5.541822in}{2.941003in}}{\pgfqpoint{5.534008in}{2.948816in}}%
\pgfpathcurveto{\pgfqpoint{5.526195in}{2.956630in}}{\pgfqpoint{5.515596in}{2.961020in}}{\pgfqpoint{5.504545in}{2.961020in}}%
\pgfpathcurveto{\pgfqpoint{5.493495in}{2.961020in}}{\pgfqpoint{5.482896in}{2.956630in}}{\pgfqpoint{5.475083in}{2.948816in}}%
\pgfpathcurveto{\pgfqpoint{5.467269in}{2.941003in}}{\pgfqpoint{5.462879in}{2.930404in}}{\pgfqpoint{5.462879in}{2.919353in}}%
\pgfpathcurveto{\pgfqpoint{5.462879in}{2.908303in}}{\pgfqpoint{5.467269in}{2.897704in}}{\pgfqpoint{5.475083in}{2.889891in}}%
\pgfpathcurveto{\pgfqpoint{5.482896in}{2.882077in}}{\pgfqpoint{5.493495in}{2.877687in}}{\pgfqpoint{5.504545in}{2.877687in}}%
\pgfpathclose%
\pgfusepath{stroke,fill}%
\end{pgfscope}%
\begin{pgfscope}%
\pgfpathrectangle{\pgfqpoint{0.800000in}{0.528000in}}{\pgfqpoint{4.960000in}{3.696000in}}%
\pgfusepath{clip}%
\pgfsetbuttcap%
\pgfsetroundjoin%
\definecolor{currentfill}{rgb}{0.000000,0.000000,0.000000}%
\pgfsetfillcolor{currentfill}%
\pgfsetlinewidth{1.003750pt}%
\definecolor{currentstroke}{rgb}{0.000000,0.000000,0.000000}%
\pgfsetstrokecolor{currentstroke}%
\pgfsetdash{}{0pt}%
\pgfpathmoveto{\pgfqpoint{5.504545in}{2.877687in}}%
\pgfpathcurveto{\pgfqpoint{5.515596in}{2.877687in}}{\pgfqpoint{5.526195in}{2.882077in}}{\pgfqpoint{5.534008in}{2.889891in}}%
\pgfpathcurveto{\pgfqpoint{5.541822in}{2.897704in}}{\pgfqpoint{5.546212in}{2.908303in}}{\pgfqpoint{5.546212in}{2.919353in}}%
\pgfpathcurveto{\pgfqpoint{5.546212in}{2.930404in}}{\pgfqpoint{5.541822in}{2.941003in}}{\pgfqpoint{5.534008in}{2.948816in}}%
\pgfpathcurveto{\pgfqpoint{5.526195in}{2.956630in}}{\pgfqpoint{5.515596in}{2.961020in}}{\pgfqpoint{5.504545in}{2.961020in}}%
\pgfpathcurveto{\pgfqpoint{5.493495in}{2.961020in}}{\pgfqpoint{5.482896in}{2.956630in}}{\pgfqpoint{5.475083in}{2.948816in}}%
\pgfpathcurveto{\pgfqpoint{5.467269in}{2.941003in}}{\pgfqpoint{5.462879in}{2.930404in}}{\pgfqpoint{5.462879in}{2.919353in}}%
\pgfpathcurveto{\pgfqpoint{5.462879in}{2.908303in}}{\pgfqpoint{5.467269in}{2.897704in}}{\pgfqpoint{5.475083in}{2.889891in}}%
\pgfpathcurveto{\pgfqpoint{5.482896in}{2.882077in}}{\pgfqpoint{5.493495in}{2.877687in}}{\pgfqpoint{5.504545in}{2.877687in}}%
\pgfpathclose%
\pgfusepath{stroke,fill}%
\end{pgfscope}%
\begin{pgfscope}%
\pgfpathrectangle{\pgfqpoint{0.800000in}{0.528000in}}{\pgfqpoint{4.960000in}{3.696000in}}%
\pgfusepath{clip}%
\pgfsetbuttcap%
\pgfsetroundjoin%
\definecolor{currentfill}{rgb}{0.000000,0.000000,0.000000}%
\pgfsetfillcolor{currentfill}%
\pgfsetlinewidth{1.003750pt}%
\definecolor{currentstroke}{rgb}{0.000000,0.000000,0.000000}%
\pgfsetstrokecolor{currentstroke}%
\pgfsetdash{}{0pt}%
\pgfpathmoveto{\pgfqpoint{5.504545in}{2.877687in}}%
\pgfpathcurveto{\pgfqpoint{5.515596in}{2.877687in}}{\pgfqpoint{5.526195in}{2.882077in}}{\pgfqpoint{5.534008in}{2.889891in}}%
\pgfpathcurveto{\pgfqpoint{5.541822in}{2.897704in}}{\pgfqpoint{5.546212in}{2.908303in}}{\pgfqpoint{5.546212in}{2.919353in}}%
\pgfpathcurveto{\pgfqpoint{5.546212in}{2.930404in}}{\pgfqpoint{5.541822in}{2.941003in}}{\pgfqpoint{5.534008in}{2.948816in}}%
\pgfpathcurveto{\pgfqpoint{5.526195in}{2.956630in}}{\pgfqpoint{5.515596in}{2.961020in}}{\pgfqpoint{5.504545in}{2.961020in}}%
\pgfpathcurveto{\pgfqpoint{5.493495in}{2.961020in}}{\pgfqpoint{5.482896in}{2.956630in}}{\pgfqpoint{5.475083in}{2.948816in}}%
\pgfpathcurveto{\pgfqpoint{5.467269in}{2.941003in}}{\pgfqpoint{5.462879in}{2.930404in}}{\pgfqpoint{5.462879in}{2.919353in}}%
\pgfpathcurveto{\pgfqpoint{5.462879in}{2.908303in}}{\pgfqpoint{5.467269in}{2.897704in}}{\pgfqpoint{5.475083in}{2.889891in}}%
\pgfpathcurveto{\pgfqpoint{5.482896in}{2.882077in}}{\pgfqpoint{5.493495in}{2.877687in}}{\pgfqpoint{5.504545in}{2.877687in}}%
\pgfpathclose%
\pgfusepath{stroke,fill}%
\end{pgfscope}%
\begin{pgfscope}%
\pgfpathrectangle{\pgfqpoint{0.800000in}{0.528000in}}{\pgfqpoint{4.960000in}{3.696000in}}%
\pgfusepath{clip}%
\pgfsetbuttcap%
\pgfsetroundjoin%
\definecolor{currentfill}{rgb}{0.000000,0.000000,0.000000}%
\pgfsetfillcolor{currentfill}%
\pgfsetlinewidth{1.003750pt}%
\definecolor{currentstroke}{rgb}{0.000000,0.000000,0.000000}%
\pgfsetstrokecolor{currentstroke}%
\pgfsetdash{}{0pt}%
\pgfpathmoveto{\pgfqpoint{5.504545in}{2.877687in}}%
\pgfpathcurveto{\pgfqpoint{5.515596in}{2.877687in}}{\pgfqpoint{5.526195in}{2.882077in}}{\pgfqpoint{5.534008in}{2.889891in}}%
\pgfpathcurveto{\pgfqpoint{5.541822in}{2.897704in}}{\pgfqpoint{5.546212in}{2.908303in}}{\pgfqpoint{5.546212in}{2.919353in}}%
\pgfpathcurveto{\pgfqpoint{5.546212in}{2.930404in}}{\pgfqpoint{5.541822in}{2.941003in}}{\pgfqpoint{5.534008in}{2.948816in}}%
\pgfpathcurveto{\pgfqpoint{5.526195in}{2.956630in}}{\pgfqpoint{5.515596in}{2.961020in}}{\pgfqpoint{5.504545in}{2.961020in}}%
\pgfpathcurveto{\pgfqpoint{5.493495in}{2.961020in}}{\pgfqpoint{5.482896in}{2.956630in}}{\pgfqpoint{5.475083in}{2.948816in}}%
\pgfpathcurveto{\pgfqpoint{5.467269in}{2.941003in}}{\pgfqpoint{5.462879in}{2.930404in}}{\pgfqpoint{5.462879in}{2.919353in}}%
\pgfpathcurveto{\pgfqpoint{5.462879in}{2.908303in}}{\pgfqpoint{5.467269in}{2.897704in}}{\pgfqpoint{5.475083in}{2.889891in}}%
\pgfpathcurveto{\pgfqpoint{5.482896in}{2.882077in}}{\pgfqpoint{5.493495in}{2.877687in}}{\pgfqpoint{5.504545in}{2.877687in}}%
\pgfpathclose%
\pgfusepath{stroke,fill}%
\end{pgfscope}%
\begin{pgfscope}%
\pgfpathrectangle{\pgfqpoint{0.800000in}{0.528000in}}{\pgfqpoint{4.960000in}{3.696000in}}%
\pgfusepath{clip}%
\pgfsetbuttcap%
\pgfsetroundjoin%
\definecolor{currentfill}{rgb}{0.000000,0.000000,0.000000}%
\pgfsetfillcolor{currentfill}%
\pgfsetlinewidth{1.003750pt}%
\definecolor{currentstroke}{rgb}{0.000000,0.000000,0.000000}%
\pgfsetstrokecolor{currentstroke}%
\pgfsetdash{}{0pt}%
\pgfpathmoveto{\pgfqpoint{5.504545in}{2.877687in}}%
\pgfpathcurveto{\pgfqpoint{5.515596in}{2.877687in}}{\pgfqpoint{5.526195in}{2.882077in}}{\pgfqpoint{5.534008in}{2.889891in}}%
\pgfpathcurveto{\pgfqpoint{5.541822in}{2.897704in}}{\pgfqpoint{5.546212in}{2.908303in}}{\pgfqpoint{5.546212in}{2.919353in}}%
\pgfpathcurveto{\pgfqpoint{5.546212in}{2.930404in}}{\pgfqpoint{5.541822in}{2.941003in}}{\pgfqpoint{5.534008in}{2.948816in}}%
\pgfpathcurveto{\pgfqpoint{5.526195in}{2.956630in}}{\pgfqpoint{5.515596in}{2.961020in}}{\pgfqpoint{5.504545in}{2.961020in}}%
\pgfpathcurveto{\pgfqpoint{5.493495in}{2.961020in}}{\pgfqpoint{5.482896in}{2.956630in}}{\pgfqpoint{5.475083in}{2.948816in}}%
\pgfpathcurveto{\pgfqpoint{5.467269in}{2.941003in}}{\pgfqpoint{5.462879in}{2.930404in}}{\pgfqpoint{5.462879in}{2.919353in}}%
\pgfpathcurveto{\pgfqpoint{5.462879in}{2.908303in}}{\pgfqpoint{5.467269in}{2.897704in}}{\pgfqpoint{5.475083in}{2.889891in}}%
\pgfpathcurveto{\pgfqpoint{5.482896in}{2.882077in}}{\pgfqpoint{5.493495in}{2.877687in}}{\pgfqpoint{5.504545in}{2.877687in}}%
\pgfpathclose%
\pgfusepath{stroke,fill}%
\end{pgfscope}%
\begin{pgfscope}%
\pgfpathrectangle{\pgfqpoint{0.800000in}{0.528000in}}{\pgfqpoint{4.960000in}{3.696000in}}%
\pgfusepath{clip}%
\pgfsetbuttcap%
\pgfsetroundjoin%
\definecolor{currentfill}{rgb}{0.000000,0.000000,0.000000}%
\pgfsetfillcolor{currentfill}%
\pgfsetlinewidth{1.003750pt}%
\definecolor{currentstroke}{rgb}{0.000000,0.000000,0.000000}%
\pgfsetstrokecolor{currentstroke}%
\pgfsetdash{}{0pt}%
\pgfpathmoveto{\pgfqpoint{5.504545in}{2.877687in}}%
\pgfpathcurveto{\pgfqpoint{5.515596in}{2.877687in}}{\pgfqpoint{5.526195in}{2.882077in}}{\pgfqpoint{5.534008in}{2.889891in}}%
\pgfpathcurveto{\pgfqpoint{5.541822in}{2.897704in}}{\pgfqpoint{5.546212in}{2.908303in}}{\pgfqpoint{5.546212in}{2.919353in}}%
\pgfpathcurveto{\pgfqpoint{5.546212in}{2.930404in}}{\pgfqpoint{5.541822in}{2.941003in}}{\pgfqpoint{5.534008in}{2.948816in}}%
\pgfpathcurveto{\pgfqpoint{5.526195in}{2.956630in}}{\pgfqpoint{5.515596in}{2.961020in}}{\pgfqpoint{5.504545in}{2.961020in}}%
\pgfpathcurveto{\pgfqpoint{5.493495in}{2.961020in}}{\pgfqpoint{5.482896in}{2.956630in}}{\pgfqpoint{5.475083in}{2.948816in}}%
\pgfpathcurveto{\pgfqpoint{5.467269in}{2.941003in}}{\pgfqpoint{5.462879in}{2.930404in}}{\pgfqpoint{5.462879in}{2.919353in}}%
\pgfpathcurveto{\pgfqpoint{5.462879in}{2.908303in}}{\pgfqpoint{5.467269in}{2.897704in}}{\pgfqpoint{5.475083in}{2.889891in}}%
\pgfpathcurveto{\pgfqpoint{5.482896in}{2.882077in}}{\pgfqpoint{5.493495in}{2.877687in}}{\pgfqpoint{5.504545in}{2.877687in}}%
\pgfpathclose%
\pgfusepath{stroke,fill}%
\end{pgfscope}%
\begin{pgfscope}%
\pgfpathrectangle{\pgfqpoint{0.800000in}{0.528000in}}{\pgfqpoint{4.960000in}{3.696000in}}%
\pgfusepath{clip}%
\pgfsetbuttcap%
\pgfsetroundjoin%
\definecolor{currentfill}{rgb}{0.000000,0.000000,0.000000}%
\pgfsetfillcolor{currentfill}%
\pgfsetlinewidth{1.003750pt}%
\definecolor{currentstroke}{rgb}{0.000000,0.000000,0.000000}%
\pgfsetstrokecolor{currentstroke}%
\pgfsetdash{}{0pt}%
\pgfpathmoveto{\pgfqpoint{5.504545in}{2.877687in}}%
\pgfpathcurveto{\pgfqpoint{5.515596in}{2.877687in}}{\pgfqpoint{5.526195in}{2.882077in}}{\pgfqpoint{5.534008in}{2.889891in}}%
\pgfpathcurveto{\pgfqpoint{5.541822in}{2.897704in}}{\pgfqpoint{5.546212in}{2.908303in}}{\pgfqpoint{5.546212in}{2.919353in}}%
\pgfpathcurveto{\pgfqpoint{5.546212in}{2.930404in}}{\pgfqpoint{5.541822in}{2.941003in}}{\pgfqpoint{5.534008in}{2.948816in}}%
\pgfpathcurveto{\pgfqpoint{5.526195in}{2.956630in}}{\pgfqpoint{5.515596in}{2.961020in}}{\pgfqpoint{5.504545in}{2.961020in}}%
\pgfpathcurveto{\pgfqpoint{5.493495in}{2.961020in}}{\pgfqpoint{5.482896in}{2.956630in}}{\pgfqpoint{5.475083in}{2.948816in}}%
\pgfpathcurveto{\pgfqpoint{5.467269in}{2.941003in}}{\pgfqpoint{5.462879in}{2.930404in}}{\pgfqpoint{5.462879in}{2.919353in}}%
\pgfpathcurveto{\pgfqpoint{5.462879in}{2.908303in}}{\pgfqpoint{5.467269in}{2.897704in}}{\pgfqpoint{5.475083in}{2.889891in}}%
\pgfpathcurveto{\pgfqpoint{5.482896in}{2.882077in}}{\pgfqpoint{5.493495in}{2.877687in}}{\pgfqpoint{5.504545in}{2.877687in}}%
\pgfpathclose%
\pgfusepath{stroke,fill}%
\end{pgfscope}%
\begin{pgfscope}%
\pgfpathrectangle{\pgfqpoint{0.800000in}{0.528000in}}{\pgfqpoint{4.960000in}{3.696000in}}%
\pgfusepath{clip}%
\pgfsetbuttcap%
\pgfsetroundjoin%
\definecolor{currentfill}{rgb}{0.000000,0.000000,0.000000}%
\pgfsetfillcolor{currentfill}%
\pgfsetlinewidth{1.003750pt}%
\definecolor{currentstroke}{rgb}{0.000000,0.000000,0.000000}%
\pgfsetstrokecolor{currentstroke}%
\pgfsetdash{}{0pt}%
\pgfpathmoveto{\pgfqpoint{5.504545in}{2.877687in}}%
\pgfpathcurveto{\pgfqpoint{5.515596in}{2.877687in}}{\pgfqpoint{5.526195in}{2.882077in}}{\pgfqpoint{5.534008in}{2.889891in}}%
\pgfpathcurveto{\pgfqpoint{5.541822in}{2.897704in}}{\pgfqpoint{5.546212in}{2.908303in}}{\pgfqpoint{5.546212in}{2.919353in}}%
\pgfpathcurveto{\pgfqpoint{5.546212in}{2.930404in}}{\pgfqpoint{5.541822in}{2.941003in}}{\pgfqpoint{5.534008in}{2.948816in}}%
\pgfpathcurveto{\pgfqpoint{5.526195in}{2.956630in}}{\pgfqpoint{5.515596in}{2.961020in}}{\pgfqpoint{5.504545in}{2.961020in}}%
\pgfpathcurveto{\pgfqpoint{5.493495in}{2.961020in}}{\pgfqpoint{5.482896in}{2.956630in}}{\pgfqpoint{5.475083in}{2.948816in}}%
\pgfpathcurveto{\pgfqpoint{5.467269in}{2.941003in}}{\pgfqpoint{5.462879in}{2.930404in}}{\pgfqpoint{5.462879in}{2.919353in}}%
\pgfpathcurveto{\pgfqpoint{5.462879in}{2.908303in}}{\pgfqpoint{5.467269in}{2.897704in}}{\pgfqpoint{5.475083in}{2.889891in}}%
\pgfpathcurveto{\pgfqpoint{5.482896in}{2.882077in}}{\pgfqpoint{5.493495in}{2.877687in}}{\pgfqpoint{5.504545in}{2.877687in}}%
\pgfpathclose%
\pgfusepath{stroke,fill}%
\end{pgfscope}%
\begin{pgfscope}%
\pgfpathrectangle{\pgfqpoint{0.800000in}{0.528000in}}{\pgfqpoint{4.960000in}{3.696000in}}%
\pgfusepath{clip}%
\pgfsetbuttcap%
\pgfsetroundjoin%
\definecolor{currentfill}{rgb}{0.000000,0.000000,0.000000}%
\pgfsetfillcolor{currentfill}%
\pgfsetlinewidth{1.003750pt}%
\definecolor{currentstroke}{rgb}{0.000000,0.000000,0.000000}%
\pgfsetstrokecolor{currentstroke}%
\pgfsetdash{}{0pt}%
\pgfpathmoveto{\pgfqpoint{5.504545in}{2.877687in}}%
\pgfpathcurveto{\pgfqpoint{5.515596in}{2.877687in}}{\pgfqpoint{5.526195in}{2.882077in}}{\pgfqpoint{5.534008in}{2.889891in}}%
\pgfpathcurveto{\pgfqpoint{5.541822in}{2.897704in}}{\pgfqpoint{5.546212in}{2.908303in}}{\pgfqpoint{5.546212in}{2.919353in}}%
\pgfpathcurveto{\pgfqpoint{5.546212in}{2.930404in}}{\pgfqpoint{5.541822in}{2.941003in}}{\pgfqpoint{5.534008in}{2.948816in}}%
\pgfpathcurveto{\pgfqpoint{5.526195in}{2.956630in}}{\pgfqpoint{5.515596in}{2.961020in}}{\pgfqpoint{5.504545in}{2.961020in}}%
\pgfpathcurveto{\pgfqpoint{5.493495in}{2.961020in}}{\pgfqpoint{5.482896in}{2.956630in}}{\pgfqpoint{5.475083in}{2.948816in}}%
\pgfpathcurveto{\pgfqpoint{5.467269in}{2.941003in}}{\pgfqpoint{5.462879in}{2.930404in}}{\pgfqpoint{5.462879in}{2.919353in}}%
\pgfpathcurveto{\pgfqpoint{5.462879in}{2.908303in}}{\pgfqpoint{5.467269in}{2.897704in}}{\pgfqpoint{5.475083in}{2.889891in}}%
\pgfpathcurveto{\pgfqpoint{5.482896in}{2.882077in}}{\pgfqpoint{5.493495in}{2.877687in}}{\pgfqpoint{5.504545in}{2.877687in}}%
\pgfpathclose%
\pgfusepath{stroke,fill}%
\end{pgfscope}%
\begin{pgfscope}%
\pgfpathrectangle{\pgfqpoint{0.800000in}{0.528000in}}{\pgfqpoint{4.960000in}{3.696000in}}%
\pgfusepath{clip}%
\pgfsetbuttcap%
\pgfsetroundjoin%
\definecolor{currentfill}{rgb}{0.000000,0.000000,0.000000}%
\pgfsetfillcolor{currentfill}%
\pgfsetlinewidth{1.003750pt}%
\definecolor{currentstroke}{rgb}{0.000000,0.000000,0.000000}%
\pgfsetstrokecolor{currentstroke}%
\pgfsetdash{}{0pt}%
\pgfpathmoveto{\pgfqpoint{5.504545in}{2.877687in}}%
\pgfpathcurveto{\pgfqpoint{5.515596in}{2.877687in}}{\pgfqpoint{5.526195in}{2.882077in}}{\pgfqpoint{5.534008in}{2.889891in}}%
\pgfpathcurveto{\pgfqpoint{5.541822in}{2.897704in}}{\pgfqpoint{5.546212in}{2.908303in}}{\pgfqpoint{5.546212in}{2.919353in}}%
\pgfpathcurveto{\pgfqpoint{5.546212in}{2.930404in}}{\pgfqpoint{5.541822in}{2.941003in}}{\pgfqpoint{5.534008in}{2.948816in}}%
\pgfpathcurveto{\pgfqpoint{5.526195in}{2.956630in}}{\pgfqpoint{5.515596in}{2.961020in}}{\pgfqpoint{5.504545in}{2.961020in}}%
\pgfpathcurveto{\pgfqpoint{5.493495in}{2.961020in}}{\pgfqpoint{5.482896in}{2.956630in}}{\pgfqpoint{5.475083in}{2.948816in}}%
\pgfpathcurveto{\pgfqpoint{5.467269in}{2.941003in}}{\pgfqpoint{5.462879in}{2.930404in}}{\pgfqpoint{5.462879in}{2.919353in}}%
\pgfpathcurveto{\pgfqpoint{5.462879in}{2.908303in}}{\pgfqpoint{5.467269in}{2.897704in}}{\pgfqpoint{5.475083in}{2.889891in}}%
\pgfpathcurveto{\pgfqpoint{5.482896in}{2.882077in}}{\pgfqpoint{5.493495in}{2.877687in}}{\pgfqpoint{5.504545in}{2.877687in}}%
\pgfpathclose%
\pgfusepath{stroke,fill}%
\end{pgfscope}%
\begin{pgfscope}%
\pgfpathrectangle{\pgfqpoint{0.800000in}{0.528000in}}{\pgfqpoint{4.960000in}{3.696000in}}%
\pgfusepath{clip}%
\pgfsetbuttcap%
\pgfsetroundjoin%
\definecolor{currentfill}{rgb}{0.000000,0.000000,0.000000}%
\pgfsetfillcolor{currentfill}%
\pgfsetlinewidth{1.003750pt}%
\definecolor{currentstroke}{rgb}{0.000000,0.000000,0.000000}%
\pgfsetstrokecolor{currentstroke}%
\pgfsetdash{}{0pt}%
\pgfpathmoveto{\pgfqpoint{5.504545in}{2.877687in}}%
\pgfpathcurveto{\pgfqpoint{5.515596in}{2.877687in}}{\pgfqpoint{5.526195in}{2.882077in}}{\pgfqpoint{5.534008in}{2.889891in}}%
\pgfpathcurveto{\pgfqpoint{5.541822in}{2.897704in}}{\pgfqpoint{5.546212in}{2.908303in}}{\pgfqpoint{5.546212in}{2.919353in}}%
\pgfpathcurveto{\pgfqpoint{5.546212in}{2.930404in}}{\pgfqpoint{5.541822in}{2.941003in}}{\pgfqpoint{5.534008in}{2.948816in}}%
\pgfpathcurveto{\pgfqpoint{5.526195in}{2.956630in}}{\pgfqpoint{5.515596in}{2.961020in}}{\pgfqpoint{5.504545in}{2.961020in}}%
\pgfpathcurveto{\pgfqpoint{5.493495in}{2.961020in}}{\pgfqpoint{5.482896in}{2.956630in}}{\pgfqpoint{5.475083in}{2.948816in}}%
\pgfpathcurveto{\pgfqpoint{5.467269in}{2.941003in}}{\pgfqpoint{5.462879in}{2.930404in}}{\pgfqpoint{5.462879in}{2.919353in}}%
\pgfpathcurveto{\pgfqpoint{5.462879in}{2.908303in}}{\pgfqpoint{5.467269in}{2.897704in}}{\pgfqpoint{5.475083in}{2.889891in}}%
\pgfpathcurveto{\pgfqpoint{5.482896in}{2.882077in}}{\pgfqpoint{5.493495in}{2.877687in}}{\pgfqpoint{5.504545in}{2.877687in}}%
\pgfpathclose%
\pgfusepath{stroke,fill}%
\end{pgfscope}%
\begin{pgfscope}%
\pgfpathrectangle{\pgfqpoint{0.800000in}{0.528000in}}{\pgfqpoint{4.960000in}{3.696000in}}%
\pgfusepath{clip}%
\pgfsetbuttcap%
\pgfsetroundjoin%
\definecolor{currentfill}{rgb}{0.000000,0.000000,0.000000}%
\pgfsetfillcolor{currentfill}%
\pgfsetlinewidth{1.003750pt}%
\definecolor{currentstroke}{rgb}{0.000000,0.000000,0.000000}%
\pgfsetstrokecolor{currentstroke}%
\pgfsetdash{}{0pt}%
\pgfpathmoveto{\pgfqpoint{5.504545in}{2.877687in}}%
\pgfpathcurveto{\pgfqpoint{5.515596in}{2.877687in}}{\pgfqpoint{5.526195in}{2.882077in}}{\pgfqpoint{5.534008in}{2.889891in}}%
\pgfpathcurveto{\pgfqpoint{5.541822in}{2.897704in}}{\pgfqpoint{5.546212in}{2.908303in}}{\pgfqpoint{5.546212in}{2.919353in}}%
\pgfpathcurveto{\pgfqpoint{5.546212in}{2.930404in}}{\pgfqpoint{5.541822in}{2.941003in}}{\pgfqpoint{5.534008in}{2.948816in}}%
\pgfpathcurveto{\pgfqpoint{5.526195in}{2.956630in}}{\pgfqpoint{5.515596in}{2.961020in}}{\pgfqpoint{5.504545in}{2.961020in}}%
\pgfpathcurveto{\pgfqpoint{5.493495in}{2.961020in}}{\pgfqpoint{5.482896in}{2.956630in}}{\pgfqpoint{5.475083in}{2.948816in}}%
\pgfpathcurveto{\pgfqpoint{5.467269in}{2.941003in}}{\pgfqpoint{5.462879in}{2.930404in}}{\pgfqpoint{5.462879in}{2.919353in}}%
\pgfpathcurveto{\pgfqpoint{5.462879in}{2.908303in}}{\pgfqpoint{5.467269in}{2.897704in}}{\pgfqpoint{5.475083in}{2.889891in}}%
\pgfpathcurveto{\pgfqpoint{5.482896in}{2.882077in}}{\pgfqpoint{5.493495in}{2.877687in}}{\pgfqpoint{5.504545in}{2.877687in}}%
\pgfpathclose%
\pgfusepath{stroke,fill}%
\end{pgfscope}%
\begin{pgfscope}%
\pgfpathrectangle{\pgfqpoint{0.800000in}{0.528000in}}{\pgfqpoint{4.960000in}{3.696000in}}%
\pgfusepath{clip}%
\pgfsetbuttcap%
\pgfsetroundjoin%
\definecolor{currentfill}{rgb}{0.000000,0.000000,0.000000}%
\pgfsetfillcolor{currentfill}%
\pgfsetlinewidth{1.003750pt}%
\definecolor{currentstroke}{rgb}{0.000000,0.000000,0.000000}%
\pgfsetstrokecolor{currentstroke}%
\pgfsetdash{}{0pt}%
\pgfpathmoveto{\pgfqpoint{5.504545in}{3.984333in}}%
\pgfpathcurveto{\pgfqpoint{5.515596in}{3.984333in}}{\pgfqpoint{5.526195in}{3.988724in}}{\pgfqpoint{5.534008in}{3.996537in}}%
\pgfpathcurveto{\pgfqpoint{5.541822in}{4.004351in}}{\pgfqpoint{5.546212in}{4.014950in}}{\pgfqpoint{5.546212in}{4.026000in}}%
\pgfpathcurveto{\pgfqpoint{5.546212in}{4.037050in}}{\pgfqpoint{5.541822in}{4.047649in}}{\pgfqpoint{5.534008in}{4.055463in}}%
\pgfpathcurveto{\pgfqpoint{5.526195in}{4.063276in}}{\pgfqpoint{5.515596in}{4.067667in}}{\pgfqpoint{5.504545in}{4.067667in}}%
\pgfpathcurveto{\pgfqpoint{5.493495in}{4.067667in}}{\pgfqpoint{5.482896in}{4.063276in}}{\pgfqpoint{5.475083in}{4.055463in}}%
\pgfpathcurveto{\pgfqpoint{5.467269in}{4.047649in}}{\pgfqpoint{5.462879in}{4.037050in}}{\pgfqpoint{5.462879in}{4.026000in}}%
\pgfpathcurveto{\pgfqpoint{5.462879in}{4.014950in}}{\pgfqpoint{5.467269in}{4.004351in}}{\pgfqpoint{5.475083in}{3.996537in}}%
\pgfpathcurveto{\pgfqpoint{5.482896in}{3.988724in}}{\pgfqpoint{5.493495in}{3.984333in}}{\pgfqpoint{5.504545in}{3.984333in}}%
\pgfpathclose%
\pgfusepath{stroke,fill}%
\end{pgfscope}%
\begin{pgfscope}%
\pgfpathrectangle{\pgfqpoint{0.800000in}{0.528000in}}{\pgfqpoint{4.960000in}{3.696000in}}%
\pgfusepath{clip}%
\pgfsetbuttcap%
\pgfsetroundjoin%
\definecolor{currentfill}{rgb}{0.000000,0.000000,0.000000}%
\pgfsetfillcolor{currentfill}%
\pgfsetlinewidth{1.003750pt}%
\definecolor{currentstroke}{rgb}{0.000000,0.000000,0.000000}%
\pgfsetstrokecolor{currentstroke}%
\pgfsetdash{}{0pt}%
\pgfpathmoveto{\pgfqpoint{5.504545in}{2.877687in}}%
\pgfpathcurveto{\pgfqpoint{5.515596in}{2.877687in}}{\pgfqpoint{5.526195in}{2.882077in}}{\pgfqpoint{5.534008in}{2.889891in}}%
\pgfpathcurveto{\pgfqpoint{5.541822in}{2.897704in}}{\pgfqpoint{5.546212in}{2.908303in}}{\pgfqpoint{5.546212in}{2.919353in}}%
\pgfpathcurveto{\pgfqpoint{5.546212in}{2.930404in}}{\pgfqpoint{5.541822in}{2.941003in}}{\pgfqpoint{5.534008in}{2.948816in}}%
\pgfpathcurveto{\pgfqpoint{5.526195in}{2.956630in}}{\pgfqpoint{5.515596in}{2.961020in}}{\pgfqpoint{5.504545in}{2.961020in}}%
\pgfpathcurveto{\pgfqpoint{5.493495in}{2.961020in}}{\pgfqpoint{5.482896in}{2.956630in}}{\pgfqpoint{5.475083in}{2.948816in}}%
\pgfpathcurveto{\pgfqpoint{5.467269in}{2.941003in}}{\pgfqpoint{5.462879in}{2.930404in}}{\pgfqpoint{5.462879in}{2.919353in}}%
\pgfpathcurveto{\pgfqpoint{5.462879in}{2.908303in}}{\pgfqpoint{5.467269in}{2.897704in}}{\pgfqpoint{5.475083in}{2.889891in}}%
\pgfpathcurveto{\pgfqpoint{5.482896in}{2.882077in}}{\pgfqpoint{5.493495in}{2.877687in}}{\pgfqpoint{5.504545in}{2.877687in}}%
\pgfpathclose%
\pgfusepath{stroke,fill}%
\end{pgfscope}%
\begin{pgfscope}%
\pgfpathrectangle{\pgfqpoint{0.800000in}{0.528000in}}{\pgfqpoint{4.960000in}{3.696000in}}%
\pgfusepath{clip}%
\pgfsetbuttcap%
\pgfsetroundjoin%
\definecolor{currentfill}{rgb}{0.000000,0.000000,0.000000}%
\pgfsetfillcolor{currentfill}%
\pgfsetlinewidth{1.003750pt}%
\definecolor{currentstroke}{rgb}{0.000000,0.000000,0.000000}%
\pgfsetstrokecolor{currentstroke}%
\pgfsetdash{}{0pt}%
\pgfpathmoveto{\pgfqpoint{5.504545in}{2.877687in}}%
\pgfpathcurveto{\pgfqpoint{5.515596in}{2.877687in}}{\pgfqpoint{5.526195in}{2.882077in}}{\pgfqpoint{5.534008in}{2.889891in}}%
\pgfpathcurveto{\pgfqpoint{5.541822in}{2.897704in}}{\pgfqpoint{5.546212in}{2.908303in}}{\pgfqpoint{5.546212in}{2.919353in}}%
\pgfpathcurveto{\pgfqpoint{5.546212in}{2.930404in}}{\pgfqpoint{5.541822in}{2.941003in}}{\pgfqpoint{5.534008in}{2.948816in}}%
\pgfpathcurveto{\pgfqpoint{5.526195in}{2.956630in}}{\pgfqpoint{5.515596in}{2.961020in}}{\pgfqpoint{5.504545in}{2.961020in}}%
\pgfpathcurveto{\pgfqpoint{5.493495in}{2.961020in}}{\pgfqpoint{5.482896in}{2.956630in}}{\pgfqpoint{5.475083in}{2.948816in}}%
\pgfpathcurveto{\pgfqpoint{5.467269in}{2.941003in}}{\pgfqpoint{5.462879in}{2.930404in}}{\pgfqpoint{5.462879in}{2.919353in}}%
\pgfpathcurveto{\pgfqpoint{5.462879in}{2.908303in}}{\pgfqpoint{5.467269in}{2.897704in}}{\pgfqpoint{5.475083in}{2.889891in}}%
\pgfpathcurveto{\pgfqpoint{5.482896in}{2.882077in}}{\pgfqpoint{5.493495in}{2.877687in}}{\pgfqpoint{5.504545in}{2.877687in}}%
\pgfpathclose%
\pgfusepath{stroke,fill}%
\end{pgfscope}%
\begin{pgfscope}%
\pgfpathrectangle{\pgfqpoint{0.800000in}{0.528000in}}{\pgfqpoint{4.960000in}{3.696000in}}%
\pgfusepath{clip}%
\pgfsetbuttcap%
\pgfsetroundjoin%
\definecolor{currentfill}{rgb}{0.000000,0.000000,0.000000}%
\pgfsetfillcolor{currentfill}%
\pgfsetlinewidth{1.003750pt}%
\definecolor{currentstroke}{rgb}{0.000000,0.000000,0.000000}%
\pgfsetstrokecolor{currentstroke}%
\pgfsetdash{}{0pt}%
\pgfpathmoveto{\pgfqpoint{5.504545in}{2.877687in}}%
\pgfpathcurveto{\pgfqpoint{5.515596in}{2.877687in}}{\pgfqpoint{5.526195in}{2.882077in}}{\pgfqpoint{5.534008in}{2.889891in}}%
\pgfpathcurveto{\pgfqpoint{5.541822in}{2.897704in}}{\pgfqpoint{5.546212in}{2.908303in}}{\pgfqpoint{5.546212in}{2.919353in}}%
\pgfpathcurveto{\pgfqpoint{5.546212in}{2.930404in}}{\pgfqpoint{5.541822in}{2.941003in}}{\pgfqpoint{5.534008in}{2.948816in}}%
\pgfpathcurveto{\pgfqpoint{5.526195in}{2.956630in}}{\pgfqpoint{5.515596in}{2.961020in}}{\pgfqpoint{5.504545in}{2.961020in}}%
\pgfpathcurveto{\pgfqpoint{5.493495in}{2.961020in}}{\pgfqpoint{5.482896in}{2.956630in}}{\pgfqpoint{5.475083in}{2.948816in}}%
\pgfpathcurveto{\pgfqpoint{5.467269in}{2.941003in}}{\pgfqpoint{5.462879in}{2.930404in}}{\pgfqpoint{5.462879in}{2.919353in}}%
\pgfpathcurveto{\pgfqpoint{5.462879in}{2.908303in}}{\pgfqpoint{5.467269in}{2.897704in}}{\pgfqpoint{5.475083in}{2.889891in}}%
\pgfpathcurveto{\pgfqpoint{5.482896in}{2.882077in}}{\pgfqpoint{5.493495in}{2.877687in}}{\pgfqpoint{5.504545in}{2.877687in}}%
\pgfpathclose%
\pgfusepath{stroke,fill}%
\end{pgfscope}%
\begin{pgfscope}%
\pgfpathrectangle{\pgfqpoint{0.800000in}{0.528000in}}{\pgfqpoint{4.960000in}{3.696000in}}%
\pgfusepath{clip}%
\pgfsetbuttcap%
\pgfsetroundjoin%
\definecolor{currentfill}{rgb}{0.000000,0.000000,0.000000}%
\pgfsetfillcolor{currentfill}%
\pgfsetlinewidth{1.003750pt}%
\definecolor{currentstroke}{rgb}{0.000000,0.000000,0.000000}%
\pgfsetstrokecolor{currentstroke}%
\pgfsetdash{}{0pt}%
\pgfpathmoveto{\pgfqpoint{5.504545in}{2.877687in}}%
\pgfpathcurveto{\pgfqpoint{5.515596in}{2.877687in}}{\pgfqpoint{5.526195in}{2.882077in}}{\pgfqpoint{5.534008in}{2.889891in}}%
\pgfpathcurveto{\pgfqpoint{5.541822in}{2.897704in}}{\pgfqpoint{5.546212in}{2.908303in}}{\pgfqpoint{5.546212in}{2.919353in}}%
\pgfpathcurveto{\pgfqpoint{5.546212in}{2.930404in}}{\pgfqpoint{5.541822in}{2.941003in}}{\pgfqpoint{5.534008in}{2.948816in}}%
\pgfpathcurveto{\pgfqpoint{5.526195in}{2.956630in}}{\pgfqpoint{5.515596in}{2.961020in}}{\pgfqpoint{5.504545in}{2.961020in}}%
\pgfpathcurveto{\pgfqpoint{5.493495in}{2.961020in}}{\pgfqpoint{5.482896in}{2.956630in}}{\pgfqpoint{5.475083in}{2.948816in}}%
\pgfpathcurveto{\pgfqpoint{5.467269in}{2.941003in}}{\pgfqpoint{5.462879in}{2.930404in}}{\pgfqpoint{5.462879in}{2.919353in}}%
\pgfpathcurveto{\pgfqpoint{5.462879in}{2.908303in}}{\pgfqpoint{5.467269in}{2.897704in}}{\pgfqpoint{5.475083in}{2.889891in}}%
\pgfpathcurveto{\pgfqpoint{5.482896in}{2.882077in}}{\pgfqpoint{5.493495in}{2.877687in}}{\pgfqpoint{5.504545in}{2.877687in}}%
\pgfpathclose%
\pgfusepath{stroke,fill}%
\end{pgfscope}%
\begin{pgfscope}%
\pgfpathrectangle{\pgfqpoint{0.800000in}{0.528000in}}{\pgfqpoint{4.960000in}{3.696000in}}%
\pgfusepath{clip}%
\pgfsetbuttcap%
\pgfsetroundjoin%
\definecolor{currentfill}{rgb}{0.000000,0.000000,0.000000}%
\pgfsetfillcolor{currentfill}%
\pgfsetlinewidth{1.003750pt}%
\definecolor{currentstroke}{rgb}{0.000000,0.000000,0.000000}%
\pgfsetstrokecolor{currentstroke}%
\pgfsetdash{}{0pt}%
\pgfpathmoveto{\pgfqpoint{5.504545in}{2.877687in}}%
\pgfpathcurveto{\pgfqpoint{5.515596in}{2.877687in}}{\pgfqpoint{5.526195in}{2.882077in}}{\pgfqpoint{5.534008in}{2.889891in}}%
\pgfpathcurveto{\pgfqpoint{5.541822in}{2.897704in}}{\pgfqpoint{5.546212in}{2.908303in}}{\pgfqpoint{5.546212in}{2.919353in}}%
\pgfpathcurveto{\pgfqpoint{5.546212in}{2.930404in}}{\pgfqpoint{5.541822in}{2.941003in}}{\pgfqpoint{5.534008in}{2.948816in}}%
\pgfpathcurveto{\pgfqpoint{5.526195in}{2.956630in}}{\pgfqpoint{5.515596in}{2.961020in}}{\pgfqpoint{5.504545in}{2.961020in}}%
\pgfpathcurveto{\pgfqpoint{5.493495in}{2.961020in}}{\pgfqpoint{5.482896in}{2.956630in}}{\pgfqpoint{5.475083in}{2.948816in}}%
\pgfpathcurveto{\pgfqpoint{5.467269in}{2.941003in}}{\pgfqpoint{5.462879in}{2.930404in}}{\pgfqpoint{5.462879in}{2.919353in}}%
\pgfpathcurveto{\pgfqpoint{5.462879in}{2.908303in}}{\pgfqpoint{5.467269in}{2.897704in}}{\pgfqpoint{5.475083in}{2.889891in}}%
\pgfpathcurveto{\pgfqpoint{5.482896in}{2.882077in}}{\pgfqpoint{5.493495in}{2.877687in}}{\pgfqpoint{5.504545in}{2.877687in}}%
\pgfpathclose%
\pgfusepath{stroke,fill}%
\end{pgfscope}%
\begin{pgfscope}%
\pgfpathrectangle{\pgfqpoint{0.800000in}{0.528000in}}{\pgfqpoint{4.960000in}{3.696000in}}%
\pgfusepath{clip}%
\pgfsetbuttcap%
\pgfsetroundjoin%
\definecolor{currentfill}{rgb}{0.000000,0.000000,0.000000}%
\pgfsetfillcolor{currentfill}%
\pgfsetlinewidth{1.003750pt}%
\definecolor{currentstroke}{rgb}{0.000000,0.000000,0.000000}%
\pgfsetstrokecolor{currentstroke}%
\pgfsetdash{}{0pt}%
\pgfpathmoveto{\pgfqpoint{5.504545in}{2.877687in}}%
\pgfpathcurveto{\pgfqpoint{5.515596in}{2.877687in}}{\pgfqpoint{5.526195in}{2.882077in}}{\pgfqpoint{5.534008in}{2.889891in}}%
\pgfpathcurveto{\pgfqpoint{5.541822in}{2.897704in}}{\pgfqpoint{5.546212in}{2.908303in}}{\pgfqpoint{5.546212in}{2.919353in}}%
\pgfpathcurveto{\pgfqpoint{5.546212in}{2.930404in}}{\pgfqpoint{5.541822in}{2.941003in}}{\pgfqpoint{5.534008in}{2.948816in}}%
\pgfpathcurveto{\pgfqpoint{5.526195in}{2.956630in}}{\pgfqpoint{5.515596in}{2.961020in}}{\pgfqpoint{5.504545in}{2.961020in}}%
\pgfpathcurveto{\pgfqpoint{5.493495in}{2.961020in}}{\pgfqpoint{5.482896in}{2.956630in}}{\pgfqpoint{5.475083in}{2.948816in}}%
\pgfpathcurveto{\pgfqpoint{5.467269in}{2.941003in}}{\pgfqpoint{5.462879in}{2.930404in}}{\pgfqpoint{5.462879in}{2.919353in}}%
\pgfpathcurveto{\pgfqpoint{5.462879in}{2.908303in}}{\pgfqpoint{5.467269in}{2.897704in}}{\pgfqpoint{5.475083in}{2.889891in}}%
\pgfpathcurveto{\pgfqpoint{5.482896in}{2.882077in}}{\pgfqpoint{5.493495in}{2.877687in}}{\pgfqpoint{5.504545in}{2.877687in}}%
\pgfpathclose%
\pgfusepath{stroke,fill}%
\end{pgfscope}%
\begin{pgfscope}%
\pgfpathrectangle{\pgfqpoint{0.800000in}{0.528000in}}{\pgfqpoint{4.960000in}{3.696000in}}%
\pgfusepath{clip}%
\pgfsetbuttcap%
\pgfsetroundjoin%
\definecolor{currentfill}{rgb}{0.000000,0.000000,0.000000}%
\pgfsetfillcolor{currentfill}%
\pgfsetlinewidth{1.003750pt}%
\definecolor{currentstroke}{rgb}{0.000000,0.000000,0.000000}%
\pgfsetstrokecolor{currentstroke}%
\pgfsetdash{}{0pt}%
\pgfpathmoveto{\pgfqpoint{5.504545in}{2.877687in}}%
\pgfpathcurveto{\pgfqpoint{5.515596in}{2.877687in}}{\pgfqpoint{5.526195in}{2.882077in}}{\pgfqpoint{5.534008in}{2.889891in}}%
\pgfpathcurveto{\pgfqpoint{5.541822in}{2.897704in}}{\pgfqpoint{5.546212in}{2.908303in}}{\pgfqpoint{5.546212in}{2.919353in}}%
\pgfpathcurveto{\pgfqpoint{5.546212in}{2.930404in}}{\pgfqpoint{5.541822in}{2.941003in}}{\pgfqpoint{5.534008in}{2.948816in}}%
\pgfpathcurveto{\pgfqpoint{5.526195in}{2.956630in}}{\pgfqpoint{5.515596in}{2.961020in}}{\pgfqpoint{5.504545in}{2.961020in}}%
\pgfpathcurveto{\pgfqpoint{5.493495in}{2.961020in}}{\pgfqpoint{5.482896in}{2.956630in}}{\pgfqpoint{5.475083in}{2.948816in}}%
\pgfpathcurveto{\pgfqpoint{5.467269in}{2.941003in}}{\pgfqpoint{5.462879in}{2.930404in}}{\pgfqpoint{5.462879in}{2.919353in}}%
\pgfpathcurveto{\pgfqpoint{5.462879in}{2.908303in}}{\pgfqpoint{5.467269in}{2.897704in}}{\pgfqpoint{5.475083in}{2.889891in}}%
\pgfpathcurveto{\pgfqpoint{5.482896in}{2.882077in}}{\pgfqpoint{5.493495in}{2.877687in}}{\pgfqpoint{5.504545in}{2.877687in}}%
\pgfpathclose%
\pgfusepath{stroke,fill}%
\end{pgfscope}%
\begin{pgfscope}%
\pgfpathrectangle{\pgfqpoint{0.800000in}{0.528000in}}{\pgfqpoint{4.960000in}{3.696000in}}%
\pgfusepath{clip}%
\pgfsetbuttcap%
\pgfsetroundjoin%
\definecolor{currentfill}{rgb}{0.000000,0.000000,0.000000}%
\pgfsetfillcolor{currentfill}%
\pgfsetlinewidth{1.003750pt}%
\definecolor{currentstroke}{rgb}{0.000000,0.000000,0.000000}%
\pgfsetstrokecolor{currentstroke}%
\pgfsetdash{}{0pt}%
\pgfpathmoveto{\pgfqpoint{5.504545in}{2.877687in}}%
\pgfpathcurveto{\pgfqpoint{5.515596in}{2.877687in}}{\pgfqpoint{5.526195in}{2.882077in}}{\pgfqpoint{5.534008in}{2.889891in}}%
\pgfpathcurveto{\pgfqpoint{5.541822in}{2.897704in}}{\pgfqpoint{5.546212in}{2.908303in}}{\pgfqpoint{5.546212in}{2.919353in}}%
\pgfpathcurveto{\pgfqpoint{5.546212in}{2.930404in}}{\pgfqpoint{5.541822in}{2.941003in}}{\pgfqpoint{5.534008in}{2.948816in}}%
\pgfpathcurveto{\pgfqpoint{5.526195in}{2.956630in}}{\pgfqpoint{5.515596in}{2.961020in}}{\pgfqpoint{5.504545in}{2.961020in}}%
\pgfpathcurveto{\pgfqpoint{5.493495in}{2.961020in}}{\pgfqpoint{5.482896in}{2.956630in}}{\pgfqpoint{5.475083in}{2.948816in}}%
\pgfpathcurveto{\pgfqpoint{5.467269in}{2.941003in}}{\pgfqpoint{5.462879in}{2.930404in}}{\pgfqpoint{5.462879in}{2.919353in}}%
\pgfpathcurveto{\pgfqpoint{5.462879in}{2.908303in}}{\pgfqpoint{5.467269in}{2.897704in}}{\pgfqpoint{5.475083in}{2.889891in}}%
\pgfpathcurveto{\pgfqpoint{5.482896in}{2.882077in}}{\pgfqpoint{5.493495in}{2.877687in}}{\pgfqpoint{5.504545in}{2.877687in}}%
\pgfpathclose%
\pgfusepath{stroke,fill}%
\end{pgfscope}%
\begin{pgfscope}%
\pgfpathrectangle{\pgfqpoint{0.800000in}{0.528000in}}{\pgfqpoint{4.960000in}{3.696000in}}%
\pgfusepath{clip}%
\pgfsetbuttcap%
\pgfsetroundjoin%
\definecolor{currentfill}{rgb}{0.000000,0.000000,0.000000}%
\pgfsetfillcolor{currentfill}%
\pgfsetlinewidth{1.003750pt}%
\definecolor{currentstroke}{rgb}{0.000000,0.000000,0.000000}%
\pgfsetstrokecolor{currentstroke}%
\pgfsetdash{}{0pt}%
\pgfpathmoveto{\pgfqpoint{5.504545in}{3.984333in}}%
\pgfpathcurveto{\pgfqpoint{5.515596in}{3.984333in}}{\pgfqpoint{5.526195in}{3.988724in}}{\pgfqpoint{5.534008in}{3.996537in}}%
\pgfpathcurveto{\pgfqpoint{5.541822in}{4.004351in}}{\pgfqpoint{5.546212in}{4.014950in}}{\pgfqpoint{5.546212in}{4.026000in}}%
\pgfpathcurveto{\pgfqpoint{5.546212in}{4.037050in}}{\pgfqpoint{5.541822in}{4.047649in}}{\pgfqpoint{5.534008in}{4.055463in}}%
\pgfpathcurveto{\pgfqpoint{5.526195in}{4.063276in}}{\pgfqpoint{5.515596in}{4.067667in}}{\pgfqpoint{5.504545in}{4.067667in}}%
\pgfpathcurveto{\pgfqpoint{5.493495in}{4.067667in}}{\pgfqpoint{5.482896in}{4.063276in}}{\pgfqpoint{5.475083in}{4.055463in}}%
\pgfpathcurveto{\pgfqpoint{5.467269in}{4.047649in}}{\pgfqpoint{5.462879in}{4.037050in}}{\pgfqpoint{5.462879in}{4.026000in}}%
\pgfpathcurveto{\pgfqpoint{5.462879in}{4.014950in}}{\pgfqpoint{5.467269in}{4.004351in}}{\pgfqpoint{5.475083in}{3.996537in}}%
\pgfpathcurveto{\pgfqpoint{5.482896in}{3.988724in}}{\pgfqpoint{5.493495in}{3.984333in}}{\pgfqpoint{5.504545in}{3.984333in}}%
\pgfpathclose%
\pgfusepath{stroke,fill}%
\end{pgfscope}%
\begin{pgfscope}%
\pgfsetbuttcap%
\pgfsetroundjoin%
\definecolor{currentfill}{rgb}{0.000000,0.000000,0.000000}%
\pgfsetfillcolor{currentfill}%
\pgfsetlinewidth{0.803000pt}%
\definecolor{currentstroke}{rgb}{0.000000,0.000000,0.000000}%
\pgfsetstrokecolor{currentstroke}%
\pgfsetdash{}{0pt}%
\pgfsys@defobject{currentmarker}{\pgfqpoint{0.000000in}{-0.048611in}}{\pgfqpoint{0.000000in}{0.000000in}}{%
\pgfpathmoveto{\pgfqpoint{0.000000in}{0.000000in}}%
\pgfpathlineto{\pgfqpoint{0.000000in}{-0.048611in}}%
\pgfusepath{stroke,fill}%
}%
\begin{pgfscope}%
\pgfsys@transformshift{1.025906in}{0.528000in}%
\pgfsys@useobject{currentmarker}{}%
\end{pgfscope}%
\end{pgfscope}%
\begin{pgfscope}%
\definecolor{textcolor}{rgb}{0.000000,0.000000,0.000000}%
\pgfsetstrokecolor{textcolor}%
\pgfsetfillcolor{textcolor}%
\pgftext[x=1.025906in,y=0.430778in,,top]{\color{textcolor}\sffamily\fontsize{10.000000}{12.000000}\selectfont 20}%
\end{pgfscope}%
\begin{pgfscope}%
\pgfsetbuttcap%
\pgfsetroundjoin%
\definecolor{currentfill}{rgb}{0.000000,0.000000,0.000000}%
\pgfsetfillcolor{currentfill}%
\pgfsetlinewidth{0.803000pt}%
\definecolor{currentstroke}{rgb}{0.000000,0.000000,0.000000}%
\pgfsetstrokecolor{currentstroke}%
\pgfsetdash{}{0pt}%
\pgfsys@defobject{currentmarker}{\pgfqpoint{0.000000in}{-0.048611in}}{\pgfqpoint{0.000000in}{0.000000in}}{%
\pgfpathmoveto{\pgfqpoint{0.000000in}{0.000000in}}%
\pgfpathlineto{\pgfqpoint{0.000000in}{-0.048611in}}%
\pgfusepath{stroke,fill}%
}%
\begin{pgfscope}%
\pgfsys@transformshift{2.518786in}{0.528000in}%
\pgfsys@useobject{currentmarker}{}%
\end{pgfscope}%
\end{pgfscope}%
\begin{pgfscope}%
\definecolor{textcolor}{rgb}{0.000000,0.000000,0.000000}%
\pgfsetstrokecolor{textcolor}%
\pgfsetfillcolor{textcolor}%
\pgftext[x=2.518786in,y=0.430778in,,top]{\color{textcolor}\sffamily\fontsize{10.000000}{12.000000}\selectfont 40}%
\end{pgfscope}%
\begin{pgfscope}%
\pgfsetbuttcap%
\pgfsetroundjoin%
\definecolor{currentfill}{rgb}{0.000000,0.000000,0.000000}%
\pgfsetfillcolor{currentfill}%
\pgfsetlinewidth{0.803000pt}%
\definecolor{currentstroke}{rgb}{0.000000,0.000000,0.000000}%
\pgfsetstrokecolor{currentstroke}%
\pgfsetdash{}{0pt}%
\pgfsys@defobject{currentmarker}{\pgfqpoint{0.000000in}{-0.048611in}}{\pgfqpoint{0.000000in}{0.000000in}}{%
\pgfpathmoveto{\pgfqpoint{0.000000in}{0.000000in}}%
\pgfpathlineto{\pgfqpoint{0.000000in}{-0.048611in}}%
\pgfusepath{stroke,fill}%
}%
\begin{pgfscope}%
\pgfsys@transformshift{4.011666in}{0.528000in}%
\pgfsys@useobject{currentmarker}{}%
\end{pgfscope}%
\end{pgfscope}%
\begin{pgfscope}%
\definecolor{textcolor}{rgb}{0.000000,0.000000,0.000000}%
\pgfsetstrokecolor{textcolor}%
\pgfsetfillcolor{textcolor}%
\pgftext[x=4.011666in,y=0.430778in,,top]{\color{textcolor}\sffamily\fontsize{10.000000}{12.000000}\selectfont 60}%
\end{pgfscope}%
\begin{pgfscope}%
\pgfsetbuttcap%
\pgfsetroundjoin%
\definecolor{currentfill}{rgb}{0.000000,0.000000,0.000000}%
\pgfsetfillcolor{currentfill}%
\pgfsetlinewidth{0.803000pt}%
\definecolor{currentstroke}{rgb}{0.000000,0.000000,0.000000}%
\pgfsetstrokecolor{currentstroke}%
\pgfsetdash{}{0pt}%
\pgfsys@defobject{currentmarker}{\pgfqpoint{0.000000in}{-0.048611in}}{\pgfqpoint{0.000000in}{0.000000in}}{%
\pgfpathmoveto{\pgfqpoint{0.000000in}{0.000000in}}%
\pgfpathlineto{\pgfqpoint{0.000000in}{-0.048611in}}%
\pgfusepath{stroke,fill}%
}%
\begin{pgfscope}%
\pgfsys@transformshift{5.504545in}{0.528000in}%
\pgfsys@useobject{currentmarker}{}%
\end{pgfscope}%
\end{pgfscope}%
\begin{pgfscope}%
\definecolor{textcolor}{rgb}{0.000000,0.000000,0.000000}%
\pgfsetstrokecolor{textcolor}%
\pgfsetfillcolor{textcolor}%
\pgftext[x=5.504545in,y=0.430778in,,top]{\color{textcolor}\sffamily\fontsize{10.000000}{12.000000}\selectfont 80}%
\end{pgfscope}%
\begin{pgfscope}%
\definecolor{textcolor}{rgb}{0.000000,0.000000,0.000000}%
\pgfsetstrokecolor{textcolor}%
\pgfsetfillcolor{textcolor}%
\pgftext[x=3.280000in,y=0.240809in,,top]{\color{textcolor}\sffamily\fontsize{10.000000}{12.000000}\selectfont \(\displaystyle k\)}%
\end{pgfscope}%
\begin{pgfscope}%
\pgfsetbuttcap%
\pgfsetroundjoin%
\definecolor{currentfill}{rgb}{0.000000,0.000000,0.000000}%
\pgfsetfillcolor{currentfill}%
\pgfsetlinewidth{0.803000pt}%
\definecolor{currentstroke}{rgb}{0.000000,0.000000,0.000000}%
\pgfsetstrokecolor{currentstroke}%
\pgfsetdash{}{0pt}%
\pgfsys@defobject{currentmarker}{\pgfqpoint{-0.048611in}{0.000000in}}{\pgfqpoint{0.000000in}{0.000000in}}{%
\pgfpathmoveto{\pgfqpoint{0.000000in}{0.000000in}}%
\pgfpathlineto{\pgfqpoint{-0.048611in}{0.000000in}}%
\pgfusepath{stroke,fill}%
}%
\begin{pgfscope}%
\pgfsys@transformshift{0.800000in}{0.706060in}%
\pgfsys@useobject{currentmarker}{}%
\end{pgfscope}%
\end{pgfscope}%
\begin{pgfscope}%
\definecolor{textcolor}{rgb}{0.000000,0.000000,0.000000}%
\pgfsetstrokecolor{textcolor}%
\pgfsetfillcolor{textcolor}%
\pgftext[x=0.614413in,y=0.653299in,left,base]{\color{textcolor}\sffamily\fontsize{10.000000}{12.000000}\selectfont 9}%
\end{pgfscope}%
\begin{pgfscope}%
\pgfsetbuttcap%
\pgfsetroundjoin%
\definecolor{currentfill}{rgb}{0.000000,0.000000,0.000000}%
\pgfsetfillcolor{currentfill}%
\pgfsetlinewidth{0.803000pt}%
\definecolor{currentstroke}{rgb}{0.000000,0.000000,0.000000}%
\pgfsetstrokecolor{currentstroke}%
\pgfsetdash{}{0pt}%
\pgfsys@defobject{currentmarker}{\pgfqpoint{-0.048611in}{0.000000in}}{\pgfqpoint{0.000000in}{0.000000in}}{%
\pgfpathmoveto{\pgfqpoint{0.000000in}{0.000000in}}%
\pgfpathlineto{\pgfqpoint{-0.048611in}{0.000000in}}%
\pgfusepath{stroke,fill}%
}%
\begin{pgfscope}%
\pgfsys@transformshift{0.800000in}{1.812707in}%
\pgfsys@useobject{currentmarker}{}%
\end{pgfscope}%
\end{pgfscope}%
\begin{pgfscope}%
\definecolor{textcolor}{rgb}{0.000000,0.000000,0.000000}%
\pgfsetstrokecolor{textcolor}%
\pgfsetfillcolor{textcolor}%
\pgftext[x=0.526047in,y=1.759945in,left,base]{\color{textcolor}\sffamily\fontsize{10.000000}{12.000000}\selectfont 10}%
\end{pgfscope}%
\begin{pgfscope}%
\pgfsetbuttcap%
\pgfsetroundjoin%
\definecolor{currentfill}{rgb}{0.000000,0.000000,0.000000}%
\pgfsetfillcolor{currentfill}%
\pgfsetlinewidth{0.803000pt}%
\definecolor{currentstroke}{rgb}{0.000000,0.000000,0.000000}%
\pgfsetstrokecolor{currentstroke}%
\pgfsetdash{}{0pt}%
\pgfsys@defobject{currentmarker}{\pgfqpoint{-0.048611in}{0.000000in}}{\pgfqpoint{0.000000in}{0.000000in}}{%
\pgfpathmoveto{\pgfqpoint{0.000000in}{0.000000in}}%
\pgfpathlineto{\pgfqpoint{-0.048611in}{0.000000in}}%
\pgfusepath{stroke,fill}%
}%
\begin{pgfscope}%
\pgfsys@transformshift{0.800000in}{2.919353in}%
\pgfsys@useobject{currentmarker}{}%
\end{pgfscope}%
\end{pgfscope}%
\begin{pgfscope}%
\definecolor{textcolor}{rgb}{0.000000,0.000000,0.000000}%
\pgfsetstrokecolor{textcolor}%
\pgfsetfillcolor{textcolor}%
\pgftext[x=0.526047in,y=2.866592in,left,base]{\color{textcolor}\sffamily\fontsize{10.000000}{12.000000}\selectfont 11}%
\end{pgfscope}%
\begin{pgfscope}%
\pgfsetbuttcap%
\pgfsetroundjoin%
\definecolor{currentfill}{rgb}{0.000000,0.000000,0.000000}%
\pgfsetfillcolor{currentfill}%
\pgfsetlinewidth{0.803000pt}%
\definecolor{currentstroke}{rgb}{0.000000,0.000000,0.000000}%
\pgfsetstrokecolor{currentstroke}%
\pgfsetdash{}{0pt}%
\pgfsys@defobject{currentmarker}{\pgfqpoint{-0.048611in}{0.000000in}}{\pgfqpoint{0.000000in}{0.000000in}}{%
\pgfpathmoveto{\pgfqpoint{0.000000in}{0.000000in}}%
\pgfpathlineto{\pgfqpoint{-0.048611in}{0.000000in}}%
\pgfusepath{stroke,fill}%
}%
\begin{pgfscope}%
\pgfsys@transformshift{0.800000in}{4.026000in}%
\pgfsys@useobject{currentmarker}{}%
\end{pgfscope}%
\end{pgfscope}%
\begin{pgfscope}%
\definecolor{textcolor}{rgb}{0.000000,0.000000,0.000000}%
\pgfsetstrokecolor{textcolor}%
\pgfsetfillcolor{textcolor}%
\pgftext[x=0.526047in,y=3.973238in,left,base]{\color{textcolor}\sffamily\fontsize{10.000000}{12.000000}\selectfont 12}%
\end{pgfscope}%
\begin{pgfscope}%
\definecolor{textcolor}{rgb}{0.000000,0.000000,0.000000}%
\pgfsetstrokecolor{textcolor}%
\pgfsetfillcolor{textcolor}%
\pgftext[x=0.470492in,y=2.376000in,,bottom,rotate=90.000000]{\color{textcolor}\sffamily\fontsize{10.000000}{12.000000}\selectfont Number of GMRES Iterations}%
\end{pgfscope}%
\begin{pgfscope}%
\pgfsetrectcap%
\pgfsetmiterjoin%
\pgfsetlinewidth{0.803000pt}%
\definecolor{currentstroke}{rgb}{0.000000,0.000000,0.000000}%
\pgfsetstrokecolor{currentstroke}%
\pgfsetdash{}{0pt}%
\pgfpathmoveto{\pgfqpoint{0.800000in}{0.528000in}}%
\pgfpathlineto{\pgfqpoint{0.800000in}{4.224000in}}%
\pgfusepath{stroke}%
\end{pgfscope}%
\begin{pgfscope}%
\pgfsetrectcap%
\pgfsetmiterjoin%
\pgfsetlinewidth{0.803000pt}%
\definecolor{currentstroke}{rgb}{0.000000,0.000000,0.000000}%
\pgfsetstrokecolor{currentstroke}%
\pgfsetdash{}{0pt}%
\pgfpathmoveto{\pgfqpoint{5.760000in}{0.528000in}}%
\pgfpathlineto{\pgfqpoint{5.760000in}{4.224000in}}%
\pgfusepath{stroke}%
\end{pgfscope}%
\begin{pgfscope}%
\pgfsetrectcap%
\pgfsetmiterjoin%
\pgfsetlinewidth{0.803000pt}%
\definecolor{currentstroke}{rgb}{0.000000,0.000000,0.000000}%
\pgfsetstrokecolor{currentstroke}%
\pgfsetdash{}{0pt}%
\pgfpathmoveto{\pgfqpoint{0.800000in}{0.528000in}}%
\pgfpathlineto{\pgfqpoint{5.760000in}{0.528000in}}%
\pgfusepath{stroke}%
\end{pgfscope}%
\begin{pgfscope}%
\pgfsetrectcap%
\pgfsetmiterjoin%
\pgfsetlinewidth{0.803000pt}%
\definecolor{currentstroke}{rgb}{0.000000,0.000000,0.000000}%
\pgfsetstrokecolor{currentstroke}%
\pgfsetdash{}{0pt}%
\pgfpathmoveto{\pgfqpoint{0.800000in}{4.224000in}}%
\pgfpathlineto{\pgfqpoint{5.760000in}{4.224000in}}%
\pgfusepath{stroke}%
\end{pgfscope}%
\end{pgfpicture}%
\makeatother%
\endgroup%

  \caption{GMRES iteration counts for $\alpha = 0.5/k^{1/2}$}\label{fig:linfinityA1}
\end{subfigure}

    \begin{subfigure}{\textwidth}
      \centering
%% Creator: Matplotlib, PGF backend
%%
%% To include the figure in your LaTeX document, write
%%   \input{<filename>.pgf}
%%
%% Make sure the required packages are loaded in your preamble
%%   \usepackage{pgf}
%%
%% Figures using additional raster images can only be included by \input if
%% they are in the same directory as the main LaTeX file. For loading figures
%% from other directories you can use the `import` package
%%   \usepackage{import}
%% and then include the figures with
%%   \import{<path to file>}{<filename>.pgf}
%%
%% Matplotlib used the following preamble
%%   \usepackage{fontspec}
%%   \setmainfont{DejaVuSerif.ttf}[Path=/home/owen/progs/firedrake-complex/firedrake/lib/python3.5/site-packages/matplotlib/mpl-data/fonts/ttf/]
%%   \setsansfont{DejaVuSans.ttf}[Path=/home/owen/progs/firedrake-complex/firedrake/lib/python3.5/site-packages/matplotlib/mpl-data/fonts/ttf/]
%%   \setmonofont{DejaVuSansMono.ttf}[Path=/home/owen/progs/firedrake-complex/firedrake/lib/python3.5/site-packages/matplotlib/mpl-data/fonts/ttf/]
%%
\begingroup%
\makeatletter%
\begin{pgfpicture}%
\pgfpathrectangle{\pgfpointorigin}{\pgfqpoint{6.400000in}{4.800000in}}%
\pgfusepath{use as bounding box, clip}%
\begin{pgfscope}%
\pgfsetbuttcap%
\pgfsetmiterjoin%
\definecolor{currentfill}{rgb}{1.000000,1.000000,1.000000}%
\pgfsetfillcolor{currentfill}%
\pgfsetlinewidth{0.000000pt}%
\definecolor{currentstroke}{rgb}{1.000000,1.000000,1.000000}%
\pgfsetstrokecolor{currentstroke}%
\pgfsetdash{}{0pt}%
\pgfpathmoveto{\pgfqpoint{0.000000in}{0.000000in}}%
\pgfpathlineto{\pgfqpoint{6.400000in}{0.000000in}}%
\pgfpathlineto{\pgfqpoint{6.400000in}{4.800000in}}%
\pgfpathlineto{\pgfqpoint{0.000000in}{4.800000in}}%
\pgfpathclose%
\pgfusepath{fill}%
\end{pgfscope}%
\begin{pgfscope}%
\pgfsetbuttcap%
\pgfsetmiterjoin%
\definecolor{currentfill}{rgb}{1.000000,1.000000,1.000000}%
\pgfsetfillcolor{currentfill}%
\pgfsetlinewidth{0.000000pt}%
\definecolor{currentstroke}{rgb}{0.000000,0.000000,0.000000}%
\pgfsetstrokecolor{currentstroke}%
\pgfsetstrokeopacity{0.000000}%
\pgfsetdash{}{0pt}%
\pgfpathmoveto{\pgfqpoint{0.800000in}{0.528000in}}%
\pgfpathlineto{\pgfqpoint{5.760000in}{0.528000in}}%
\pgfpathlineto{\pgfqpoint{5.760000in}{4.224000in}}%
\pgfpathlineto{\pgfqpoint{0.800000in}{4.224000in}}%
\pgfpathclose%
\pgfusepath{fill}%
\end{pgfscope}%
\begin{pgfscope}%
\pgfpathrectangle{\pgfqpoint{0.800000in}{0.528000in}}{\pgfqpoint{4.960000in}{3.696000in}}%
\pgfusepath{clip}%
\pgfsetbuttcap%
\pgfsetroundjoin%
\definecolor{currentfill}{rgb}{0.000000,0.000000,0.000000}%
\pgfsetfillcolor{currentfill}%
\pgfsetlinewidth{1.003750pt}%
\definecolor{currentstroke}{rgb}{0.000000,0.000000,0.000000}%
\pgfsetstrokecolor{currentstroke}%
\pgfsetdash{}{0pt}%
\pgfpathmoveto{\pgfqpoint{1.025906in}{0.684333in}}%
\pgfpathcurveto{\pgfqpoint{1.036956in}{0.684333in}}{\pgfqpoint{1.047555in}{0.688724in}}{\pgfqpoint{1.055369in}{0.696537in}}%
\pgfpathcurveto{\pgfqpoint{1.063182in}{0.704351in}}{\pgfqpoint{1.067573in}{0.714950in}}{\pgfqpoint{1.067573in}{0.726000in}}%
\pgfpathcurveto{\pgfqpoint{1.067573in}{0.737050in}}{\pgfqpoint{1.063182in}{0.747649in}}{\pgfqpoint{1.055369in}{0.755463in}}%
\pgfpathcurveto{\pgfqpoint{1.047555in}{0.763276in}}{\pgfqpoint{1.036956in}{0.767667in}}{\pgfqpoint{1.025906in}{0.767667in}}%
\pgfpathcurveto{\pgfqpoint{1.014856in}{0.767667in}}{\pgfqpoint{1.004257in}{0.763276in}}{\pgfqpoint{0.996443in}{0.755463in}}%
\pgfpathcurveto{\pgfqpoint{0.988630in}{0.747649in}}{\pgfqpoint{0.984239in}{0.737050in}}{\pgfqpoint{0.984239in}{0.726000in}}%
\pgfpathcurveto{\pgfqpoint{0.984239in}{0.714950in}}{\pgfqpoint{0.988630in}{0.704351in}}{\pgfqpoint{0.996443in}{0.696537in}}%
\pgfpathcurveto{\pgfqpoint{1.004257in}{0.688724in}}{\pgfqpoint{1.014856in}{0.684333in}}{\pgfqpoint{1.025906in}{0.684333in}}%
\pgfpathclose%
\pgfusepath{stroke,fill}%
\end{pgfscope}%
\begin{pgfscope}%
\pgfpathrectangle{\pgfqpoint{0.800000in}{0.528000in}}{\pgfqpoint{4.960000in}{3.696000in}}%
\pgfusepath{clip}%
\pgfsetbuttcap%
\pgfsetroundjoin%
\definecolor{currentfill}{rgb}{0.000000,0.000000,0.000000}%
\pgfsetfillcolor{currentfill}%
\pgfsetlinewidth{1.003750pt}%
\definecolor{currentstroke}{rgb}{0.000000,0.000000,0.000000}%
\pgfsetstrokecolor{currentstroke}%
\pgfsetdash{}{0pt}%
\pgfpathmoveto{\pgfqpoint{1.025906in}{0.684333in}}%
\pgfpathcurveto{\pgfqpoint{1.036956in}{0.684333in}}{\pgfqpoint{1.047555in}{0.688724in}}{\pgfqpoint{1.055369in}{0.696537in}}%
\pgfpathcurveto{\pgfqpoint{1.063182in}{0.704351in}}{\pgfqpoint{1.067573in}{0.714950in}}{\pgfqpoint{1.067573in}{0.726000in}}%
\pgfpathcurveto{\pgfqpoint{1.067573in}{0.737050in}}{\pgfqpoint{1.063182in}{0.747649in}}{\pgfqpoint{1.055369in}{0.755463in}}%
\pgfpathcurveto{\pgfqpoint{1.047555in}{0.763276in}}{\pgfqpoint{1.036956in}{0.767667in}}{\pgfqpoint{1.025906in}{0.767667in}}%
\pgfpathcurveto{\pgfqpoint{1.014856in}{0.767667in}}{\pgfqpoint{1.004257in}{0.763276in}}{\pgfqpoint{0.996443in}{0.755463in}}%
\pgfpathcurveto{\pgfqpoint{0.988630in}{0.747649in}}{\pgfqpoint{0.984239in}{0.737050in}}{\pgfqpoint{0.984239in}{0.726000in}}%
\pgfpathcurveto{\pgfqpoint{0.984239in}{0.714950in}}{\pgfqpoint{0.988630in}{0.704351in}}{\pgfqpoint{0.996443in}{0.696537in}}%
\pgfpathcurveto{\pgfqpoint{1.004257in}{0.688724in}}{\pgfqpoint{1.014856in}{0.684333in}}{\pgfqpoint{1.025906in}{0.684333in}}%
\pgfpathclose%
\pgfusepath{stroke,fill}%
\end{pgfscope}%
\begin{pgfscope}%
\pgfpathrectangle{\pgfqpoint{0.800000in}{0.528000in}}{\pgfqpoint{4.960000in}{3.696000in}}%
\pgfusepath{clip}%
\pgfsetbuttcap%
\pgfsetroundjoin%
\definecolor{currentfill}{rgb}{0.000000,0.000000,0.000000}%
\pgfsetfillcolor{currentfill}%
\pgfsetlinewidth{1.003750pt}%
\definecolor{currentstroke}{rgb}{0.000000,0.000000,0.000000}%
\pgfsetstrokecolor{currentstroke}%
\pgfsetdash{}{0pt}%
\pgfpathmoveto{\pgfqpoint{1.025906in}{0.684333in}}%
\pgfpathcurveto{\pgfqpoint{1.036956in}{0.684333in}}{\pgfqpoint{1.047555in}{0.688724in}}{\pgfqpoint{1.055369in}{0.696537in}}%
\pgfpathcurveto{\pgfqpoint{1.063182in}{0.704351in}}{\pgfqpoint{1.067573in}{0.714950in}}{\pgfqpoint{1.067573in}{0.726000in}}%
\pgfpathcurveto{\pgfqpoint{1.067573in}{0.737050in}}{\pgfqpoint{1.063182in}{0.747649in}}{\pgfqpoint{1.055369in}{0.755463in}}%
\pgfpathcurveto{\pgfqpoint{1.047555in}{0.763276in}}{\pgfqpoint{1.036956in}{0.767667in}}{\pgfqpoint{1.025906in}{0.767667in}}%
\pgfpathcurveto{\pgfqpoint{1.014856in}{0.767667in}}{\pgfqpoint{1.004257in}{0.763276in}}{\pgfqpoint{0.996443in}{0.755463in}}%
\pgfpathcurveto{\pgfqpoint{0.988630in}{0.747649in}}{\pgfqpoint{0.984239in}{0.737050in}}{\pgfqpoint{0.984239in}{0.726000in}}%
\pgfpathcurveto{\pgfqpoint{0.984239in}{0.714950in}}{\pgfqpoint{0.988630in}{0.704351in}}{\pgfqpoint{0.996443in}{0.696537in}}%
\pgfpathcurveto{\pgfqpoint{1.004257in}{0.688724in}}{\pgfqpoint{1.014856in}{0.684333in}}{\pgfqpoint{1.025906in}{0.684333in}}%
\pgfpathclose%
\pgfusepath{stroke,fill}%
\end{pgfscope}%
\begin{pgfscope}%
\pgfpathrectangle{\pgfqpoint{0.800000in}{0.528000in}}{\pgfqpoint{4.960000in}{3.696000in}}%
\pgfusepath{clip}%
\pgfsetbuttcap%
\pgfsetroundjoin%
\definecolor{currentfill}{rgb}{0.000000,0.000000,0.000000}%
\pgfsetfillcolor{currentfill}%
\pgfsetlinewidth{1.003750pt}%
\definecolor{currentstroke}{rgb}{0.000000,0.000000,0.000000}%
\pgfsetstrokecolor{currentstroke}%
\pgfsetdash{}{0pt}%
\pgfpathmoveto{\pgfqpoint{1.025906in}{0.684333in}}%
\pgfpathcurveto{\pgfqpoint{1.036956in}{0.684333in}}{\pgfqpoint{1.047555in}{0.688724in}}{\pgfqpoint{1.055369in}{0.696537in}}%
\pgfpathcurveto{\pgfqpoint{1.063182in}{0.704351in}}{\pgfqpoint{1.067573in}{0.714950in}}{\pgfqpoint{1.067573in}{0.726000in}}%
\pgfpathcurveto{\pgfqpoint{1.067573in}{0.737050in}}{\pgfqpoint{1.063182in}{0.747649in}}{\pgfqpoint{1.055369in}{0.755463in}}%
\pgfpathcurveto{\pgfqpoint{1.047555in}{0.763276in}}{\pgfqpoint{1.036956in}{0.767667in}}{\pgfqpoint{1.025906in}{0.767667in}}%
\pgfpathcurveto{\pgfqpoint{1.014856in}{0.767667in}}{\pgfqpoint{1.004257in}{0.763276in}}{\pgfqpoint{0.996443in}{0.755463in}}%
\pgfpathcurveto{\pgfqpoint{0.988630in}{0.747649in}}{\pgfqpoint{0.984239in}{0.737050in}}{\pgfqpoint{0.984239in}{0.726000in}}%
\pgfpathcurveto{\pgfqpoint{0.984239in}{0.714950in}}{\pgfqpoint{0.988630in}{0.704351in}}{\pgfqpoint{0.996443in}{0.696537in}}%
\pgfpathcurveto{\pgfqpoint{1.004257in}{0.688724in}}{\pgfqpoint{1.014856in}{0.684333in}}{\pgfqpoint{1.025906in}{0.684333in}}%
\pgfpathclose%
\pgfusepath{stroke,fill}%
\end{pgfscope}%
\begin{pgfscope}%
\pgfpathrectangle{\pgfqpoint{0.800000in}{0.528000in}}{\pgfqpoint{4.960000in}{3.696000in}}%
\pgfusepath{clip}%
\pgfsetbuttcap%
\pgfsetroundjoin%
\definecolor{currentfill}{rgb}{0.000000,0.000000,0.000000}%
\pgfsetfillcolor{currentfill}%
\pgfsetlinewidth{1.003750pt}%
\definecolor{currentstroke}{rgb}{0.000000,0.000000,0.000000}%
\pgfsetstrokecolor{currentstroke}%
\pgfsetdash{}{0pt}%
\pgfpathmoveto{\pgfqpoint{1.025906in}{0.684333in}}%
\pgfpathcurveto{\pgfqpoint{1.036956in}{0.684333in}}{\pgfqpoint{1.047555in}{0.688724in}}{\pgfqpoint{1.055369in}{0.696537in}}%
\pgfpathcurveto{\pgfqpoint{1.063182in}{0.704351in}}{\pgfqpoint{1.067573in}{0.714950in}}{\pgfqpoint{1.067573in}{0.726000in}}%
\pgfpathcurveto{\pgfqpoint{1.067573in}{0.737050in}}{\pgfqpoint{1.063182in}{0.747649in}}{\pgfqpoint{1.055369in}{0.755463in}}%
\pgfpathcurveto{\pgfqpoint{1.047555in}{0.763276in}}{\pgfqpoint{1.036956in}{0.767667in}}{\pgfqpoint{1.025906in}{0.767667in}}%
\pgfpathcurveto{\pgfqpoint{1.014856in}{0.767667in}}{\pgfqpoint{1.004257in}{0.763276in}}{\pgfqpoint{0.996443in}{0.755463in}}%
\pgfpathcurveto{\pgfqpoint{0.988630in}{0.747649in}}{\pgfqpoint{0.984239in}{0.737050in}}{\pgfqpoint{0.984239in}{0.726000in}}%
\pgfpathcurveto{\pgfqpoint{0.984239in}{0.714950in}}{\pgfqpoint{0.988630in}{0.704351in}}{\pgfqpoint{0.996443in}{0.696537in}}%
\pgfpathcurveto{\pgfqpoint{1.004257in}{0.688724in}}{\pgfqpoint{1.014856in}{0.684333in}}{\pgfqpoint{1.025906in}{0.684333in}}%
\pgfpathclose%
\pgfusepath{stroke,fill}%
\end{pgfscope}%
\begin{pgfscope}%
\pgfpathrectangle{\pgfqpoint{0.800000in}{0.528000in}}{\pgfqpoint{4.960000in}{3.696000in}}%
\pgfusepath{clip}%
\pgfsetbuttcap%
\pgfsetroundjoin%
\definecolor{currentfill}{rgb}{0.000000,0.000000,0.000000}%
\pgfsetfillcolor{currentfill}%
\pgfsetlinewidth{1.003750pt}%
\definecolor{currentstroke}{rgb}{0.000000,0.000000,0.000000}%
\pgfsetstrokecolor{currentstroke}%
\pgfsetdash{}{0pt}%
\pgfpathmoveto{\pgfqpoint{1.025906in}{0.684333in}}%
\pgfpathcurveto{\pgfqpoint{1.036956in}{0.684333in}}{\pgfqpoint{1.047555in}{0.688724in}}{\pgfqpoint{1.055369in}{0.696537in}}%
\pgfpathcurveto{\pgfqpoint{1.063182in}{0.704351in}}{\pgfqpoint{1.067573in}{0.714950in}}{\pgfqpoint{1.067573in}{0.726000in}}%
\pgfpathcurveto{\pgfqpoint{1.067573in}{0.737050in}}{\pgfqpoint{1.063182in}{0.747649in}}{\pgfqpoint{1.055369in}{0.755463in}}%
\pgfpathcurveto{\pgfqpoint{1.047555in}{0.763276in}}{\pgfqpoint{1.036956in}{0.767667in}}{\pgfqpoint{1.025906in}{0.767667in}}%
\pgfpathcurveto{\pgfqpoint{1.014856in}{0.767667in}}{\pgfqpoint{1.004257in}{0.763276in}}{\pgfqpoint{0.996443in}{0.755463in}}%
\pgfpathcurveto{\pgfqpoint{0.988630in}{0.747649in}}{\pgfqpoint{0.984239in}{0.737050in}}{\pgfqpoint{0.984239in}{0.726000in}}%
\pgfpathcurveto{\pgfqpoint{0.984239in}{0.714950in}}{\pgfqpoint{0.988630in}{0.704351in}}{\pgfqpoint{0.996443in}{0.696537in}}%
\pgfpathcurveto{\pgfqpoint{1.004257in}{0.688724in}}{\pgfqpoint{1.014856in}{0.684333in}}{\pgfqpoint{1.025906in}{0.684333in}}%
\pgfpathclose%
\pgfusepath{stroke,fill}%
\end{pgfscope}%
\begin{pgfscope}%
\pgfpathrectangle{\pgfqpoint{0.800000in}{0.528000in}}{\pgfqpoint{4.960000in}{3.696000in}}%
\pgfusepath{clip}%
\pgfsetbuttcap%
\pgfsetroundjoin%
\definecolor{currentfill}{rgb}{0.000000,0.000000,0.000000}%
\pgfsetfillcolor{currentfill}%
\pgfsetlinewidth{1.003750pt}%
\definecolor{currentstroke}{rgb}{0.000000,0.000000,0.000000}%
\pgfsetstrokecolor{currentstroke}%
\pgfsetdash{}{0pt}%
\pgfpathmoveto{\pgfqpoint{1.025906in}{0.684333in}}%
\pgfpathcurveto{\pgfqpoint{1.036956in}{0.684333in}}{\pgfqpoint{1.047555in}{0.688724in}}{\pgfqpoint{1.055369in}{0.696537in}}%
\pgfpathcurveto{\pgfqpoint{1.063182in}{0.704351in}}{\pgfqpoint{1.067573in}{0.714950in}}{\pgfqpoint{1.067573in}{0.726000in}}%
\pgfpathcurveto{\pgfqpoint{1.067573in}{0.737050in}}{\pgfqpoint{1.063182in}{0.747649in}}{\pgfqpoint{1.055369in}{0.755463in}}%
\pgfpathcurveto{\pgfqpoint{1.047555in}{0.763276in}}{\pgfqpoint{1.036956in}{0.767667in}}{\pgfqpoint{1.025906in}{0.767667in}}%
\pgfpathcurveto{\pgfqpoint{1.014856in}{0.767667in}}{\pgfqpoint{1.004257in}{0.763276in}}{\pgfqpoint{0.996443in}{0.755463in}}%
\pgfpathcurveto{\pgfqpoint{0.988630in}{0.747649in}}{\pgfqpoint{0.984239in}{0.737050in}}{\pgfqpoint{0.984239in}{0.726000in}}%
\pgfpathcurveto{\pgfqpoint{0.984239in}{0.714950in}}{\pgfqpoint{0.988630in}{0.704351in}}{\pgfqpoint{0.996443in}{0.696537in}}%
\pgfpathcurveto{\pgfqpoint{1.004257in}{0.688724in}}{\pgfqpoint{1.014856in}{0.684333in}}{\pgfqpoint{1.025906in}{0.684333in}}%
\pgfpathclose%
\pgfusepath{stroke,fill}%
\end{pgfscope}%
\begin{pgfscope}%
\pgfpathrectangle{\pgfqpoint{0.800000in}{0.528000in}}{\pgfqpoint{4.960000in}{3.696000in}}%
\pgfusepath{clip}%
\pgfsetbuttcap%
\pgfsetroundjoin%
\definecolor{currentfill}{rgb}{0.000000,0.000000,0.000000}%
\pgfsetfillcolor{currentfill}%
\pgfsetlinewidth{1.003750pt}%
\definecolor{currentstroke}{rgb}{0.000000,0.000000,0.000000}%
\pgfsetstrokecolor{currentstroke}%
\pgfsetdash{}{0pt}%
\pgfpathmoveto{\pgfqpoint{1.025906in}{0.684333in}}%
\pgfpathcurveto{\pgfqpoint{1.036956in}{0.684333in}}{\pgfqpoint{1.047555in}{0.688724in}}{\pgfqpoint{1.055369in}{0.696537in}}%
\pgfpathcurveto{\pgfqpoint{1.063182in}{0.704351in}}{\pgfqpoint{1.067573in}{0.714950in}}{\pgfqpoint{1.067573in}{0.726000in}}%
\pgfpathcurveto{\pgfqpoint{1.067573in}{0.737050in}}{\pgfqpoint{1.063182in}{0.747649in}}{\pgfqpoint{1.055369in}{0.755463in}}%
\pgfpathcurveto{\pgfqpoint{1.047555in}{0.763276in}}{\pgfqpoint{1.036956in}{0.767667in}}{\pgfqpoint{1.025906in}{0.767667in}}%
\pgfpathcurveto{\pgfqpoint{1.014856in}{0.767667in}}{\pgfqpoint{1.004257in}{0.763276in}}{\pgfqpoint{0.996443in}{0.755463in}}%
\pgfpathcurveto{\pgfqpoint{0.988630in}{0.747649in}}{\pgfqpoint{0.984239in}{0.737050in}}{\pgfqpoint{0.984239in}{0.726000in}}%
\pgfpathcurveto{\pgfqpoint{0.984239in}{0.714950in}}{\pgfqpoint{0.988630in}{0.704351in}}{\pgfqpoint{0.996443in}{0.696537in}}%
\pgfpathcurveto{\pgfqpoint{1.004257in}{0.688724in}}{\pgfqpoint{1.014856in}{0.684333in}}{\pgfqpoint{1.025906in}{0.684333in}}%
\pgfpathclose%
\pgfusepath{stroke,fill}%
\end{pgfscope}%
\begin{pgfscope}%
\pgfpathrectangle{\pgfqpoint{0.800000in}{0.528000in}}{\pgfqpoint{4.960000in}{3.696000in}}%
\pgfusepath{clip}%
\pgfsetbuttcap%
\pgfsetroundjoin%
\definecolor{currentfill}{rgb}{0.000000,0.000000,0.000000}%
\pgfsetfillcolor{currentfill}%
\pgfsetlinewidth{1.003750pt}%
\definecolor{currentstroke}{rgb}{0.000000,0.000000,0.000000}%
\pgfsetstrokecolor{currentstroke}%
\pgfsetdash{}{0pt}%
\pgfpathmoveto{\pgfqpoint{1.025906in}{0.684333in}}%
\pgfpathcurveto{\pgfqpoint{1.036956in}{0.684333in}}{\pgfqpoint{1.047555in}{0.688724in}}{\pgfqpoint{1.055369in}{0.696537in}}%
\pgfpathcurveto{\pgfqpoint{1.063182in}{0.704351in}}{\pgfqpoint{1.067573in}{0.714950in}}{\pgfqpoint{1.067573in}{0.726000in}}%
\pgfpathcurveto{\pgfqpoint{1.067573in}{0.737050in}}{\pgfqpoint{1.063182in}{0.747649in}}{\pgfqpoint{1.055369in}{0.755463in}}%
\pgfpathcurveto{\pgfqpoint{1.047555in}{0.763276in}}{\pgfqpoint{1.036956in}{0.767667in}}{\pgfqpoint{1.025906in}{0.767667in}}%
\pgfpathcurveto{\pgfqpoint{1.014856in}{0.767667in}}{\pgfqpoint{1.004257in}{0.763276in}}{\pgfqpoint{0.996443in}{0.755463in}}%
\pgfpathcurveto{\pgfqpoint{0.988630in}{0.747649in}}{\pgfqpoint{0.984239in}{0.737050in}}{\pgfqpoint{0.984239in}{0.726000in}}%
\pgfpathcurveto{\pgfqpoint{0.984239in}{0.714950in}}{\pgfqpoint{0.988630in}{0.704351in}}{\pgfqpoint{0.996443in}{0.696537in}}%
\pgfpathcurveto{\pgfqpoint{1.004257in}{0.688724in}}{\pgfqpoint{1.014856in}{0.684333in}}{\pgfqpoint{1.025906in}{0.684333in}}%
\pgfpathclose%
\pgfusepath{stroke,fill}%
\end{pgfscope}%
\begin{pgfscope}%
\pgfpathrectangle{\pgfqpoint{0.800000in}{0.528000in}}{\pgfqpoint{4.960000in}{3.696000in}}%
\pgfusepath{clip}%
\pgfsetbuttcap%
\pgfsetroundjoin%
\definecolor{currentfill}{rgb}{0.000000,0.000000,0.000000}%
\pgfsetfillcolor{currentfill}%
\pgfsetlinewidth{1.003750pt}%
\definecolor{currentstroke}{rgb}{0.000000,0.000000,0.000000}%
\pgfsetstrokecolor{currentstroke}%
\pgfsetdash{}{0pt}%
\pgfpathmoveto{\pgfqpoint{1.025906in}{0.684333in}}%
\pgfpathcurveto{\pgfqpoint{1.036956in}{0.684333in}}{\pgfqpoint{1.047555in}{0.688724in}}{\pgfqpoint{1.055369in}{0.696537in}}%
\pgfpathcurveto{\pgfqpoint{1.063182in}{0.704351in}}{\pgfqpoint{1.067573in}{0.714950in}}{\pgfqpoint{1.067573in}{0.726000in}}%
\pgfpathcurveto{\pgfqpoint{1.067573in}{0.737050in}}{\pgfqpoint{1.063182in}{0.747649in}}{\pgfqpoint{1.055369in}{0.755463in}}%
\pgfpathcurveto{\pgfqpoint{1.047555in}{0.763276in}}{\pgfqpoint{1.036956in}{0.767667in}}{\pgfqpoint{1.025906in}{0.767667in}}%
\pgfpathcurveto{\pgfqpoint{1.014856in}{0.767667in}}{\pgfqpoint{1.004257in}{0.763276in}}{\pgfqpoint{0.996443in}{0.755463in}}%
\pgfpathcurveto{\pgfqpoint{0.988630in}{0.747649in}}{\pgfqpoint{0.984239in}{0.737050in}}{\pgfqpoint{0.984239in}{0.726000in}}%
\pgfpathcurveto{\pgfqpoint{0.984239in}{0.714950in}}{\pgfqpoint{0.988630in}{0.704351in}}{\pgfqpoint{0.996443in}{0.696537in}}%
\pgfpathcurveto{\pgfqpoint{1.004257in}{0.688724in}}{\pgfqpoint{1.014856in}{0.684333in}}{\pgfqpoint{1.025906in}{0.684333in}}%
\pgfpathclose%
\pgfusepath{stroke,fill}%
\end{pgfscope}%
\begin{pgfscope}%
\pgfpathrectangle{\pgfqpoint{0.800000in}{0.528000in}}{\pgfqpoint{4.960000in}{3.696000in}}%
\pgfusepath{clip}%
\pgfsetbuttcap%
\pgfsetroundjoin%
\definecolor{currentfill}{rgb}{0.000000,0.000000,0.000000}%
\pgfsetfillcolor{currentfill}%
\pgfsetlinewidth{1.003750pt}%
\definecolor{currentstroke}{rgb}{0.000000,0.000000,0.000000}%
\pgfsetstrokecolor{currentstroke}%
\pgfsetdash{}{0pt}%
\pgfpathmoveto{\pgfqpoint{1.025906in}{3.984333in}}%
\pgfpathcurveto{\pgfqpoint{1.036956in}{3.984333in}}{\pgfqpoint{1.047555in}{3.988724in}}{\pgfqpoint{1.055369in}{3.996537in}}%
\pgfpathcurveto{\pgfqpoint{1.063182in}{4.004351in}}{\pgfqpoint{1.067573in}{4.014950in}}{\pgfqpoint{1.067573in}{4.026000in}}%
\pgfpathcurveto{\pgfqpoint{1.067573in}{4.037050in}}{\pgfqpoint{1.063182in}{4.047649in}}{\pgfqpoint{1.055369in}{4.055463in}}%
\pgfpathcurveto{\pgfqpoint{1.047555in}{4.063276in}}{\pgfqpoint{1.036956in}{4.067667in}}{\pgfqpoint{1.025906in}{4.067667in}}%
\pgfpathcurveto{\pgfqpoint{1.014856in}{4.067667in}}{\pgfqpoint{1.004257in}{4.063276in}}{\pgfqpoint{0.996443in}{4.055463in}}%
\pgfpathcurveto{\pgfqpoint{0.988630in}{4.047649in}}{\pgfqpoint{0.984239in}{4.037050in}}{\pgfqpoint{0.984239in}{4.026000in}}%
\pgfpathcurveto{\pgfqpoint{0.984239in}{4.014950in}}{\pgfqpoint{0.988630in}{4.004351in}}{\pgfqpoint{0.996443in}{3.996537in}}%
\pgfpathcurveto{\pgfqpoint{1.004257in}{3.988724in}}{\pgfqpoint{1.014856in}{3.984333in}}{\pgfqpoint{1.025906in}{3.984333in}}%
\pgfpathclose%
\pgfusepath{stroke,fill}%
\end{pgfscope}%
\begin{pgfscope}%
\pgfpathrectangle{\pgfqpoint{0.800000in}{0.528000in}}{\pgfqpoint{4.960000in}{3.696000in}}%
\pgfusepath{clip}%
\pgfsetbuttcap%
\pgfsetroundjoin%
\definecolor{currentfill}{rgb}{0.000000,0.000000,0.000000}%
\pgfsetfillcolor{currentfill}%
\pgfsetlinewidth{1.003750pt}%
\definecolor{currentstroke}{rgb}{0.000000,0.000000,0.000000}%
\pgfsetstrokecolor{currentstroke}%
\pgfsetdash{}{0pt}%
\pgfpathmoveto{\pgfqpoint{1.025906in}{0.684333in}}%
\pgfpathcurveto{\pgfqpoint{1.036956in}{0.684333in}}{\pgfqpoint{1.047555in}{0.688724in}}{\pgfqpoint{1.055369in}{0.696537in}}%
\pgfpathcurveto{\pgfqpoint{1.063182in}{0.704351in}}{\pgfqpoint{1.067573in}{0.714950in}}{\pgfqpoint{1.067573in}{0.726000in}}%
\pgfpathcurveto{\pgfqpoint{1.067573in}{0.737050in}}{\pgfqpoint{1.063182in}{0.747649in}}{\pgfqpoint{1.055369in}{0.755463in}}%
\pgfpathcurveto{\pgfqpoint{1.047555in}{0.763276in}}{\pgfqpoint{1.036956in}{0.767667in}}{\pgfqpoint{1.025906in}{0.767667in}}%
\pgfpathcurveto{\pgfqpoint{1.014856in}{0.767667in}}{\pgfqpoint{1.004257in}{0.763276in}}{\pgfqpoint{0.996443in}{0.755463in}}%
\pgfpathcurveto{\pgfqpoint{0.988630in}{0.747649in}}{\pgfqpoint{0.984239in}{0.737050in}}{\pgfqpoint{0.984239in}{0.726000in}}%
\pgfpathcurveto{\pgfqpoint{0.984239in}{0.714950in}}{\pgfqpoint{0.988630in}{0.704351in}}{\pgfqpoint{0.996443in}{0.696537in}}%
\pgfpathcurveto{\pgfqpoint{1.004257in}{0.688724in}}{\pgfqpoint{1.014856in}{0.684333in}}{\pgfqpoint{1.025906in}{0.684333in}}%
\pgfpathclose%
\pgfusepath{stroke,fill}%
\end{pgfscope}%
\begin{pgfscope}%
\pgfpathrectangle{\pgfqpoint{0.800000in}{0.528000in}}{\pgfqpoint{4.960000in}{3.696000in}}%
\pgfusepath{clip}%
\pgfsetbuttcap%
\pgfsetroundjoin%
\definecolor{currentfill}{rgb}{0.000000,0.000000,0.000000}%
\pgfsetfillcolor{currentfill}%
\pgfsetlinewidth{1.003750pt}%
\definecolor{currentstroke}{rgb}{0.000000,0.000000,0.000000}%
\pgfsetstrokecolor{currentstroke}%
\pgfsetdash{}{0pt}%
\pgfpathmoveto{\pgfqpoint{1.025906in}{0.684333in}}%
\pgfpathcurveto{\pgfqpoint{1.036956in}{0.684333in}}{\pgfqpoint{1.047555in}{0.688724in}}{\pgfqpoint{1.055369in}{0.696537in}}%
\pgfpathcurveto{\pgfqpoint{1.063182in}{0.704351in}}{\pgfqpoint{1.067573in}{0.714950in}}{\pgfqpoint{1.067573in}{0.726000in}}%
\pgfpathcurveto{\pgfqpoint{1.067573in}{0.737050in}}{\pgfqpoint{1.063182in}{0.747649in}}{\pgfqpoint{1.055369in}{0.755463in}}%
\pgfpathcurveto{\pgfqpoint{1.047555in}{0.763276in}}{\pgfqpoint{1.036956in}{0.767667in}}{\pgfqpoint{1.025906in}{0.767667in}}%
\pgfpathcurveto{\pgfqpoint{1.014856in}{0.767667in}}{\pgfqpoint{1.004257in}{0.763276in}}{\pgfqpoint{0.996443in}{0.755463in}}%
\pgfpathcurveto{\pgfqpoint{0.988630in}{0.747649in}}{\pgfqpoint{0.984239in}{0.737050in}}{\pgfqpoint{0.984239in}{0.726000in}}%
\pgfpathcurveto{\pgfqpoint{0.984239in}{0.714950in}}{\pgfqpoint{0.988630in}{0.704351in}}{\pgfqpoint{0.996443in}{0.696537in}}%
\pgfpathcurveto{\pgfqpoint{1.004257in}{0.688724in}}{\pgfqpoint{1.014856in}{0.684333in}}{\pgfqpoint{1.025906in}{0.684333in}}%
\pgfpathclose%
\pgfusepath{stroke,fill}%
\end{pgfscope}%
\begin{pgfscope}%
\pgfpathrectangle{\pgfqpoint{0.800000in}{0.528000in}}{\pgfqpoint{4.960000in}{3.696000in}}%
\pgfusepath{clip}%
\pgfsetbuttcap%
\pgfsetroundjoin%
\definecolor{currentfill}{rgb}{0.000000,0.000000,0.000000}%
\pgfsetfillcolor{currentfill}%
\pgfsetlinewidth{1.003750pt}%
\definecolor{currentstroke}{rgb}{0.000000,0.000000,0.000000}%
\pgfsetstrokecolor{currentstroke}%
\pgfsetdash{}{0pt}%
\pgfpathmoveto{\pgfqpoint{1.025906in}{0.684333in}}%
\pgfpathcurveto{\pgfqpoint{1.036956in}{0.684333in}}{\pgfqpoint{1.047555in}{0.688724in}}{\pgfqpoint{1.055369in}{0.696537in}}%
\pgfpathcurveto{\pgfqpoint{1.063182in}{0.704351in}}{\pgfqpoint{1.067573in}{0.714950in}}{\pgfqpoint{1.067573in}{0.726000in}}%
\pgfpathcurveto{\pgfqpoint{1.067573in}{0.737050in}}{\pgfqpoint{1.063182in}{0.747649in}}{\pgfqpoint{1.055369in}{0.755463in}}%
\pgfpathcurveto{\pgfqpoint{1.047555in}{0.763276in}}{\pgfqpoint{1.036956in}{0.767667in}}{\pgfqpoint{1.025906in}{0.767667in}}%
\pgfpathcurveto{\pgfqpoint{1.014856in}{0.767667in}}{\pgfqpoint{1.004257in}{0.763276in}}{\pgfqpoint{0.996443in}{0.755463in}}%
\pgfpathcurveto{\pgfqpoint{0.988630in}{0.747649in}}{\pgfqpoint{0.984239in}{0.737050in}}{\pgfqpoint{0.984239in}{0.726000in}}%
\pgfpathcurveto{\pgfqpoint{0.984239in}{0.714950in}}{\pgfqpoint{0.988630in}{0.704351in}}{\pgfqpoint{0.996443in}{0.696537in}}%
\pgfpathcurveto{\pgfqpoint{1.004257in}{0.688724in}}{\pgfqpoint{1.014856in}{0.684333in}}{\pgfqpoint{1.025906in}{0.684333in}}%
\pgfpathclose%
\pgfusepath{stroke,fill}%
\end{pgfscope}%
\begin{pgfscope}%
\pgfpathrectangle{\pgfqpoint{0.800000in}{0.528000in}}{\pgfqpoint{4.960000in}{3.696000in}}%
\pgfusepath{clip}%
\pgfsetbuttcap%
\pgfsetroundjoin%
\definecolor{currentfill}{rgb}{0.000000,0.000000,0.000000}%
\pgfsetfillcolor{currentfill}%
\pgfsetlinewidth{1.003750pt}%
\definecolor{currentstroke}{rgb}{0.000000,0.000000,0.000000}%
\pgfsetstrokecolor{currentstroke}%
\pgfsetdash{}{0pt}%
\pgfpathmoveto{\pgfqpoint{1.025906in}{0.684333in}}%
\pgfpathcurveto{\pgfqpoint{1.036956in}{0.684333in}}{\pgfqpoint{1.047555in}{0.688724in}}{\pgfqpoint{1.055369in}{0.696537in}}%
\pgfpathcurveto{\pgfqpoint{1.063182in}{0.704351in}}{\pgfqpoint{1.067573in}{0.714950in}}{\pgfqpoint{1.067573in}{0.726000in}}%
\pgfpathcurveto{\pgfqpoint{1.067573in}{0.737050in}}{\pgfqpoint{1.063182in}{0.747649in}}{\pgfqpoint{1.055369in}{0.755463in}}%
\pgfpathcurveto{\pgfqpoint{1.047555in}{0.763276in}}{\pgfqpoint{1.036956in}{0.767667in}}{\pgfqpoint{1.025906in}{0.767667in}}%
\pgfpathcurveto{\pgfqpoint{1.014856in}{0.767667in}}{\pgfqpoint{1.004257in}{0.763276in}}{\pgfqpoint{0.996443in}{0.755463in}}%
\pgfpathcurveto{\pgfqpoint{0.988630in}{0.747649in}}{\pgfqpoint{0.984239in}{0.737050in}}{\pgfqpoint{0.984239in}{0.726000in}}%
\pgfpathcurveto{\pgfqpoint{0.984239in}{0.714950in}}{\pgfqpoint{0.988630in}{0.704351in}}{\pgfqpoint{0.996443in}{0.696537in}}%
\pgfpathcurveto{\pgfqpoint{1.004257in}{0.688724in}}{\pgfqpoint{1.014856in}{0.684333in}}{\pgfqpoint{1.025906in}{0.684333in}}%
\pgfpathclose%
\pgfusepath{stroke,fill}%
\end{pgfscope}%
\begin{pgfscope}%
\pgfpathrectangle{\pgfqpoint{0.800000in}{0.528000in}}{\pgfqpoint{4.960000in}{3.696000in}}%
\pgfusepath{clip}%
\pgfsetbuttcap%
\pgfsetroundjoin%
\definecolor{currentfill}{rgb}{0.000000,0.000000,0.000000}%
\pgfsetfillcolor{currentfill}%
\pgfsetlinewidth{1.003750pt}%
\definecolor{currentstroke}{rgb}{0.000000,0.000000,0.000000}%
\pgfsetstrokecolor{currentstroke}%
\pgfsetdash{}{0pt}%
\pgfpathmoveto{\pgfqpoint{1.025906in}{0.684333in}}%
\pgfpathcurveto{\pgfqpoint{1.036956in}{0.684333in}}{\pgfqpoint{1.047555in}{0.688724in}}{\pgfqpoint{1.055369in}{0.696537in}}%
\pgfpathcurveto{\pgfqpoint{1.063182in}{0.704351in}}{\pgfqpoint{1.067573in}{0.714950in}}{\pgfqpoint{1.067573in}{0.726000in}}%
\pgfpathcurveto{\pgfqpoint{1.067573in}{0.737050in}}{\pgfqpoint{1.063182in}{0.747649in}}{\pgfqpoint{1.055369in}{0.755463in}}%
\pgfpathcurveto{\pgfqpoint{1.047555in}{0.763276in}}{\pgfqpoint{1.036956in}{0.767667in}}{\pgfqpoint{1.025906in}{0.767667in}}%
\pgfpathcurveto{\pgfqpoint{1.014856in}{0.767667in}}{\pgfqpoint{1.004257in}{0.763276in}}{\pgfqpoint{0.996443in}{0.755463in}}%
\pgfpathcurveto{\pgfqpoint{0.988630in}{0.747649in}}{\pgfqpoint{0.984239in}{0.737050in}}{\pgfqpoint{0.984239in}{0.726000in}}%
\pgfpathcurveto{\pgfqpoint{0.984239in}{0.714950in}}{\pgfqpoint{0.988630in}{0.704351in}}{\pgfqpoint{0.996443in}{0.696537in}}%
\pgfpathcurveto{\pgfqpoint{1.004257in}{0.688724in}}{\pgfqpoint{1.014856in}{0.684333in}}{\pgfqpoint{1.025906in}{0.684333in}}%
\pgfpathclose%
\pgfusepath{stroke,fill}%
\end{pgfscope}%
\begin{pgfscope}%
\pgfpathrectangle{\pgfqpoint{0.800000in}{0.528000in}}{\pgfqpoint{4.960000in}{3.696000in}}%
\pgfusepath{clip}%
\pgfsetbuttcap%
\pgfsetroundjoin%
\definecolor{currentfill}{rgb}{0.000000,0.000000,0.000000}%
\pgfsetfillcolor{currentfill}%
\pgfsetlinewidth{1.003750pt}%
\definecolor{currentstroke}{rgb}{0.000000,0.000000,0.000000}%
\pgfsetstrokecolor{currentstroke}%
\pgfsetdash{}{0pt}%
\pgfpathmoveto{\pgfqpoint{1.025906in}{0.684333in}}%
\pgfpathcurveto{\pgfqpoint{1.036956in}{0.684333in}}{\pgfqpoint{1.047555in}{0.688724in}}{\pgfqpoint{1.055369in}{0.696537in}}%
\pgfpathcurveto{\pgfqpoint{1.063182in}{0.704351in}}{\pgfqpoint{1.067573in}{0.714950in}}{\pgfqpoint{1.067573in}{0.726000in}}%
\pgfpathcurveto{\pgfqpoint{1.067573in}{0.737050in}}{\pgfqpoint{1.063182in}{0.747649in}}{\pgfqpoint{1.055369in}{0.755463in}}%
\pgfpathcurveto{\pgfqpoint{1.047555in}{0.763276in}}{\pgfqpoint{1.036956in}{0.767667in}}{\pgfqpoint{1.025906in}{0.767667in}}%
\pgfpathcurveto{\pgfqpoint{1.014856in}{0.767667in}}{\pgfqpoint{1.004257in}{0.763276in}}{\pgfqpoint{0.996443in}{0.755463in}}%
\pgfpathcurveto{\pgfqpoint{0.988630in}{0.747649in}}{\pgfqpoint{0.984239in}{0.737050in}}{\pgfqpoint{0.984239in}{0.726000in}}%
\pgfpathcurveto{\pgfqpoint{0.984239in}{0.714950in}}{\pgfqpoint{0.988630in}{0.704351in}}{\pgfqpoint{0.996443in}{0.696537in}}%
\pgfpathcurveto{\pgfqpoint{1.004257in}{0.688724in}}{\pgfqpoint{1.014856in}{0.684333in}}{\pgfqpoint{1.025906in}{0.684333in}}%
\pgfpathclose%
\pgfusepath{stroke,fill}%
\end{pgfscope}%
\begin{pgfscope}%
\pgfpathrectangle{\pgfqpoint{0.800000in}{0.528000in}}{\pgfqpoint{4.960000in}{3.696000in}}%
\pgfusepath{clip}%
\pgfsetbuttcap%
\pgfsetroundjoin%
\definecolor{currentfill}{rgb}{0.000000,0.000000,0.000000}%
\pgfsetfillcolor{currentfill}%
\pgfsetlinewidth{1.003750pt}%
\definecolor{currentstroke}{rgb}{0.000000,0.000000,0.000000}%
\pgfsetstrokecolor{currentstroke}%
\pgfsetdash{}{0pt}%
\pgfpathmoveto{\pgfqpoint{1.025906in}{0.684333in}}%
\pgfpathcurveto{\pgfqpoint{1.036956in}{0.684333in}}{\pgfqpoint{1.047555in}{0.688724in}}{\pgfqpoint{1.055369in}{0.696537in}}%
\pgfpathcurveto{\pgfqpoint{1.063182in}{0.704351in}}{\pgfqpoint{1.067573in}{0.714950in}}{\pgfqpoint{1.067573in}{0.726000in}}%
\pgfpathcurveto{\pgfqpoint{1.067573in}{0.737050in}}{\pgfqpoint{1.063182in}{0.747649in}}{\pgfqpoint{1.055369in}{0.755463in}}%
\pgfpathcurveto{\pgfqpoint{1.047555in}{0.763276in}}{\pgfqpoint{1.036956in}{0.767667in}}{\pgfqpoint{1.025906in}{0.767667in}}%
\pgfpathcurveto{\pgfqpoint{1.014856in}{0.767667in}}{\pgfqpoint{1.004257in}{0.763276in}}{\pgfqpoint{0.996443in}{0.755463in}}%
\pgfpathcurveto{\pgfqpoint{0.988630in}{0.747649in}}{\pgfqpoint{0.984239in}{0.737050in}}{\pgfqpoint{0.984239in}{0.726000in}}%
\pgfpathcurveto{\pgfqpoint{0.984239in}{0.714950in}}{\pgfqpoint{0.988630in}{0.704351in}}{\pgfqpoint{0.996443in}{0.696537in}}%
\pgfpathcurveto{\pgfqpoint{1.004257in}{0.688724in}}{\pgfqpoint{1.014856in}{0.684333in}}{\pgfqpoint{1.025906in}{0.684333in}}%
\pgfpathclose%
\pgfusepath{stroke,fill}%
\end{pgfscope}%
\begin{pgfscope}%
\pgfpathrectangle{\pgfqpoint{0.800000in}{0.528000in}}{\pgfqpoint{4.960000in}{3.696000in}}%
\pgfusepath{clip}%
\pgfsetbuttcap%
\pgfsetroundjoin%
\definecolor{currentfill}{rgb}{0.000000,0.000000,0.000000}%
\pgfsetfillcolor{currentfill}%
\pgfsetlinewidth{1.003750pt}%
\definecolor{currentstroke}{rgb}{0.000000,0.000000,0.000000}%
\pgfsetstrokecolor{currentstroke}%
\pgfsetdash{}{0pt}%
\pgfpathmoveto{\pgfqpoint{1.025906in}{0.684333in}}%
\pgfpathcurveto{\pgfqpoint{1.036956in}{0.684333in}}{\pgfqpoint{1.047555in}{0.688724in}}{\pgfqpoint{1.055369in}{0.696537in}}%
\pgfpathcurveto{\pgfqpoint{1.063182in}{0.704351in}}{\pgfqpoint{1.067573in}{0.714950in}}{\pgfqpoint{1.067573in}{0.726000in}}%
\pgfpathcurveto{\pgfqpoint{1.067573in}{0.737050in}}{\pgfqpoint{1.063182in}{0.747649in}}{\pgfqpoint{1.055369in}{0.755463in}}%
\pgfpathcurveto{\pgfqpoint{1.047555in}{0.763276in}}{\pgfqpoint{1.036956in}{0.767667in}}{\pgfqpoint{1.025906in}{0.767667in}}%
\pgfpathcurveto{\pgfqpoint{1.014856in}{0.767667in}}{\pgfqpoint{1.004257in}{0.763276in}}{\pgfqpoint{0.996443in}{0.755463in}}%
\pgfpathcurveto{\pgfqpoint{0.988630in}{0.747649in}}{\pgfqpoint{0.984239in}{0.737050in}}{\pgfqpoint{0.984239in}{0.726000in}}%
\pgfpathcurveto{\pgfqpoint{0.984239in}{0.714950in}}{\pgfqpoint{0.988630in}{0.704351in}}{\pgfqpoint{0.996443in}{0.696537in}}%
\pgfpathcurveto{\pgfqpoint{1.004257in}{0.688724in}}{\pgfqpoint{1.014856in}{0.684333in}}{\pgfqpoint{1.025906in}{0.684333in}}%
\pgfpathclose%
\pgfusepath{stroke,fill}%
\end{pgfscope}%
\begin{pgfscope}%
\pgfpathrectangle{\pgfqpoint{0.800000in}{0.528000in}}{\pgfqpoint{4.960000in}{3.696000in}}%
\pgfusepath{clip}%
\pgfsetbuttcap%
\pgfsetroundjoin%
\definecolor{currentfill}{rgb}{0.000000,0.000000,0.000000}%
\pgfsetfillcolor{currentfill}%
\pgfsetlinewidth{1.003750pt}%
\definecolor{currentstroke}{rgb}{0.000000,0.000000,0.000000}%
\pgfsetstrokecolor{currentstroke}%
\pgfsetdash{}{0pt}%
\pgfpathmoveto{\pgfqpoint{1.025906in}{0.684333in}}%
\pgfpathcurveto{\pgfqpoint{1.036956in}{0.684333in}}{\pgfqpoint{1.047555in}{0.688724in}}{\pgfqpoint{1.055369in}{0.696537in}}%
\pgfpathcurveto{\pgfqpoint{1.063182in}{0.704351in}}{\pgfqpoint{1.067573in}{0.714950in}}{\pgfqpoint{1.067573in}{0.726000in}}%
\pgfpathcurveto{\pgfqpoint{1.067573in}{0.737050in}}{\pgfqpoint{1.063182in}{0.747649in}}{\pgfqpoint{1.055369in}{0.755463in}}%
\pgfpathcurveto{\pgfqpoint{1.047555in}{0.763276in}}{\pgfqpoint{1.036956in}{0.767667in}}{\pgfqpoint{1.025906in}{0.767667in}}%
\pgfpathcurveto{\pgfqpoint{1.014856in}{0.767667in}}{\pgfqpoint{1.004257in}{0.763276in}}{\pgfqpoint{0.996443in}{0.755463in}}%
\pgfpathcurveto{\pgfqpoint{0.988630in}{0.747649in}}{\pgfqpoint{0.984239in}{0.737050in}}{\pgfqpoint{0.984239in}{0.726000in}}%
\pgfpathcurveto{\pgfqpoint{0.984239in}{0.714950in}}{\pgfqpoint{0.988630in}{0.704351in}}{\pgfqpoint{0.996443in}{0.696537in}}%
\pgfpathcurveto{\pgfqpoint{1.004257in}{0.688724in}}{\pgfqpoint{1.014856in}{0.684333in}}{\pgfqpoint{1.025906in}{0.684333in}}%
\pgfpathclose%
\pgfusepath{stroke,fill}%
\end{pgfscope}%
\begin{pgfscope}%
\pgfpathrectangle{\pgfqpoint{0.800000in}{0.528000in}}{\pgfqpoint{4.960000in}{3.696000in}}%
\pgfusepath{clip}%
\pgfsetbuttcap%
\pgfsetroundjoin%
\definecolor{currentfill}{rgb}{0.000000,0.000000,0.000000}%
\pgfsetfillcolor{currentfill}%
\pgfsetlinewidth{1.003750pt}%
\definecolor{currentstroke}{rgb}{0.000000,0.000000,0.000000}%
\pgfsetstrokecolor{currentstroke}%
\pgfsetdash{}{0pt}%
\pgfpathmoveto{\pgfqpoint{1.025906in}{0.684333in}}%
\pgfpathcurveto{\pgfqpoint{1.036956in}{0.684333in}}{\pgfqpoint{1.047555in}{0.688724in}}{\pgfqpoint{1.055369in}{0.696537in}}%
\pgfpathcurveto{\pgfqpoint{1.063182in}{0.704351in}}{\pgfqpoint{1.067573in}{0.714950in}}{\pgfqpoint{1.067573in}{0.726000in}}%
\pgfpathcurveto{\pgfqpoint{1.067573in}{0.737050in}}{\pgfqpoint{1.063182in}{0.747649in}}{\pgfqpoint{1.055369in}{0.755463in}}%
\pgfpathcurveto{\pgfqpoint{1.047555in}{0.763276in}}{\pgfqpoint{1.036956in}{0.767667in}}{\pgfqpoint{1.025906in}{0.767667in}}%
\pgfpathcurveto{\pgfqpoint{1.014856in}{0.767667in}}{\pgfqpoint{1.004257in}{0.763276in}}{\pgfqpoint{0.996443in}{0.755463in}}%
\pgfpathcurveto{\pgfqpoint{0.988630in}{0.747649in}}{\pgfqpoint{0.984239in}{0.737050in}}{\pgfqpoint{0.984239in}{0.726000in}}%
\pgfpathcurveto{\pgfqpoint{0.984239in}{0.714950in}}{\pgfqpoint{0.988630in}{0.704351in}}{\pgfqpoint{0.996443in}{0.696537in}}%
\pgfpathcurveto{\pgfqpoint{1.004257in}{0.688724in}}{\pgfqpoint{1.014856in}{0.684333in}}{\pgfqpoint{1.025906in}{0.684333in}}%
\pgfpathclose%
\pgfusepath{stroke,fill}%
\end{pgfscope}%
\begin{pgfscope}%
\pgfpathrectangle{\pgfqpoint{0.800000in}{0.528000in}}{\pgfqpoint{4.960000in}{3.696000in}}%
\pgfusepath{clip}%
\pgfsetbuttcap%
\pgfsetroundjoin%
\definecolor{currentfill}{rgb}{0.000000,0.000000,0.000000}%
\pgfsetfillcolor{currentfill}%
\pgfsetlinewidth{1.003750pt}%
\definecolor{currentstroke}{rgb}{0.000000,0.000000,0.000000}%
\pgfsetstrokecolor{currentstroke}%
\pgfsetdash{}{0pt}%
\pgfpathmoveto{\pgfqpoint{1.025906in}{0.684333in}}%
\pgfpathcurveto{\pgfqpoint{1.036956in}{0.684333in}}{\pgfqpoint{1.047555in}{0.688724in}}{\pgfqpoint{1.055369in}{0.696537in}}%
\pgfpathcurveto{\pgfqpoint{1.063182in}{0.704351in}}{\pgfqpoint{1.067573in}{0.714950in}}{\pgfqpoint{1.067573in}{0.726000in}}%
\pgfpathcurveto{\pgfqpoint{1.067573in}{0.737050in}}{\pgfqpoint{1.063182in}{0.747649in}}{\pgfqpoint{1.055369in}{0.755463in}}%
\pgfpathcurveto{\pgfqpoint{1.047555in}{0.763276in}}{\pgfqpoint{1.036956in}{0.767667in}}{\pgfqpoint{1.025906in}{0.767667in}}%
\pgfpathcurveto{\pgfqpoint{1.014856in}{0.767667in}}{\pgfqpoint{1.004257in}{0.763276in}}{\pgfqpoint{0.996443in}{0.755463in}}%
\pgfpathcurveto{\pgfqpoint{0.988630in}{0.747649in}}{\pgfqpoint{0.984239in}{0.737050in}}{\pgfqpoint{0.984239in}{0.726000in}}%
\pgfpathcurveto{\pgfqpoint{0.984239in}{0.714950in}}{\pgfqpoint{0.988630in}{0.704351in}}{\pgfqpoint{0.996443in}{0.696537in}}%
\pgfpathcurveto{\pgfqpoint{1.004257in}{0.688724in}}{\pgfqpoint{1.014856in}{0.684333in}}{\pgfqpoint{1.025906in}{0.684333in}}%
\pgfpathclose%
\pgfusepath{stroke,fill}%
\end{pgfscope}%
\begin{pgfscope}%
\pgfpathrectangle{\pgfqpoint{0.800000in}{0.528000in}}{\pgfqpoint{4.960000in}{3.696000in}}%
\pgfusepath{clip}%
\pgfsetbuttcap%
\pgfsetroundjoin%
\definecolor{currentfill}{rgb}{0.000000,0.000000,0.000000}%
\pgfsetfillcolor{currentfill}%
\pgfsetlinewidth{1.003750pt}%
\definecolor{currentstroke}{rgb}{0.000000,0.000000,0.000000}%
\pgfsetstrokecolor{currentstroke}%
\pgfsetdash{}{0pt}%
\pgfpathmoveto{\pgfqpoint{1.025906in}{0.684333in}}%
\pgfpathcurveto{\pgfqpoint{1.036956in}{0.684333in}}{\pgfqpoint{1.047555in}{0.688724in}}{\pgfqpoint{1.055369in}{0.696537in}}%
\pgfpathcurveto{\pgfqpoint{1.063182in}{0.704351in}}{\pgfqpoint{1.067573in}{0.714950in}}{\pgfqpoint{1.067573in}{0.726000in}}%
\pgfpathcurveto{\pgfqpoint{1.067573in}{0.737050in}}{\pgfqpoint{1.063182in}{0.747649in}}{\pgfqpoint{1.055369in}{0.755463in}}%
\pgfpathcurveto{\pgfqpoint{1.047555in}{0.763276in}}{\pgfqpoint{1.036956in}{0.767667in}}{\pgfqpoint{1.025906in}{0.767667in}}%
\pgfpathcurveto{\pgfqpoint{1.014856in}{0.767667in}}{\pgfqpoint{1.004257in}{0.763276in}}{\pgfqpoint{0.996443in}{0.755463in}}%
\pgfpathcurveto{\pgfqpoint{0.988630in}{0.747649in}}{\pgfqpoint{0.984239in}{0.737050in}}{\pgfqpoint{0.984239in}{0.726000in}}%
\pgfpathcurveto{\pgfqpoint{0.984239in}{0.714950in}}{\pgfqpoint{0.988630in}{0.704351in}}{\pgfqpoint{0.996443in}{0.696537in}}%
\pgfpathcurveto{\pgfqpoint{1.004257in}{0.688724in}}{\pgfqpoint{1.014856in}{0.684333in}}{\pgfqpoint{1.025906in}{0.684333in}}%
\pgfpathclose%
\pgfusepath{stroke,fill}%
\end{pgfscope}%
\begin{pgfscope}%
\pgfpathrectangle{\pgfqpoint{0.800000in}{0.528000in}}{\pgfqpoint{4.960000in}{3.696000in}}%
\pgfusepath{clip}%
\pgfsetbuttcap%
\pgfsetroundjoin%
\definecolor{currentfill}{rgb}{0.000000,0.000000,0.000000}%
\pgfsetfillcolor{currentfill}%
\pgfsetlinewidth{1.003750pt}%
\definecolor{currentstroke}{rgb}{0.000000,0.000000,0.000000}%
\pgfsetstrokecolor{currentstroke}%
\pgfsetdash{}{0pt}%
\pgfpathmoveto{\pgfqpoint{1.025906in}{0.684333in}}%
\pgfpathcurveto{\pgfqpoint{1.036956in}{0.684333in}}{\pgfqpoint{1.047555in}{0.688724in}}{\pgfqpoint{1.055369in}{0.696537in}}%
\pgfpathcurveto{\pgfqpoint{1.063182in}{0.704351in}}{\pgfqpoint{1.067573in}{0.714950in}}{\pgfqpoint{1.067573in}{0.726000in}}%
\pgfpathcurveto{\pgfqpoint{1.067573in}{0.737050in}}{\pgfqpoint{1.063182in}{0.747649in}}{\pgfqpoint{1.055369in}{0.755463in}}%
\pgfpathcurveto{\pgfqpoint{1.047555in}{0.763276in}}{\pgfqpoint{1.036956in}{0.767667in}}{\pgfqpoint{1.025906in}{0.767667in}}%
\pgfpathcurveto{\pgfqpoint{1.014856in}{0.767667in}}{\pgfqpoint{1.004257in}{0.763276in}}{\pgfqpoint{0.996443in}{0.755463in}}%
\pgfpathcurveto{\pgfqpoint{0.988630in}{0.747649in}}{\pgfqpoint{0.984239in}{0.737050in}}{\pgfqpoint{0.984239in}{0.726000in}}%
\pgfpathcurveto{\pgfqpoint{0.984239in}{0.714950in}}{\pgfqpoint{0.988630in}{0.704351in}}{\pgfqpoint{0.996443in}{0.696537in}}%
\pgfpathcurveto{\pgfqpoint{1.004257in}{0.688724in}}{\pgfqpoint{1.014856in}{0.684333in}}{\pgfqpoint{1.025906in}{0.684333in}}%
\pgfpathclose%
\pgfusepath{stroke,fill}%
\end{pgfscope}%
\begin{pgfscope}%
\pgfpathrectangle{\pgfqpoint{0.800000in}{0.528000in}}{\pgfqpoint{4.960000in}{3.696000in}}%
\pgfusepath{clip}%
\pgfsetbuttcap%
\pgfsetroundjoin%
\definecolor{currentfill}{rgb}{0.000000,0.000000,0.000000}%
\pgfsetfillcolor{currentfill}%
\pgfsetlinewidth{1.003750pt}%
\definecolor{currentstroke}{rgb}{0.000000,0.000000,0.000000}%
\pgfsetstrokecolor{currentstroke}%
\pgfsetdash{}{0pt}%
\pgfpathmoveto{\pgfqpoint{1.025906in}{0.684333in}}%
\pgfpathcurveto{\pgfqpoint{1.036956in}{0.684333in}}{\pgfqpoint{1.047555in}{0.688724in}}{\pgfqpoint{1.055369in}{0.696537in}}%
\pgfpathcurveto{\pgfqpoint{1.063182in}{0.704351in}}{\pgfqpoint{1.067573in}{0.714950in}}{\pgfqpoint{1.067573in}{0.726000in}}%
\pgfpathcurveto{\pgfqpoint{1.067573in}{0.737050in}}{\pgfqpoint{1.063182in}{0.747649in}}{\pgfqpoint{1.055369in}{0.755463in}}%
\pgfpathcurveto{\pgfqpoint{1.047555in}{0.763276in}}{\pgfqpoint{1.036956in}{0.767667in}}{\pgfqpoint{1.025906in}{0.767667in}}%
\pgfpathcurveto{\pgfqpoint{1.014856in}{0.767667in}}{\pgfqpoint{1.004257in}{0.763276in}}{\pgfqpoint{0.996443in}{0.755463in}}%
\pgfpathcurveto{\pgfqpoint{0.988630in}{0.747649in}}{\pgfqpoint{0.984239in}{0.737050in}}{\pgfqpoint{0.984239in}{0.726000in}}%
\pgfpathcurveto{\pgfqpoint{0.984239in}{0.714950in}}{\pgfqpoint{0.988630in}{0.704351in}}{\pgfqpoint{0.996443in}{0.696537in}}%
\pgfpathcurveto{\pgfqpoint{1.004257in}{0.688724in}}{\pgfqpoint{1.014856in}{0.684333in}}{\pgfqpoint{1.025906in}{0.684333in}}%
\pgfpathclose%
\pgfusepath{stroke,fill}%
\end{pgfscope}%
\begin{pgfscope}%
\pgfpathrectangle{\pgfqpoint{0.800000in}{0.528000in}}{\pgfqpoint{4.960000in}{3.696000in}}%
\pgfusepath{clip}%
\pgfsetbuttcap%
\pgfsetroundjoin%
\definecolor{currentfill}{rgb}{0.000000,0.000000,0.000000}%
\pgfsetfillcolor{currentfill}%
\pgfsetlinewidth{1.003750pt}%
\definecolor{currentstroke}{rgb}{0.000000,0.000000,0.000000}%
\pgfsetstrokecolor{currentstroke}%
\pgfsetdash{}{0pt}%
\pgfpathmoveto{\pgfqpoint{1.025906in}{0.684333in}}%
\pgfpathcurveto{\pgfqpoint{1.036956in}{0.684333in}}{\pgfqpoint{1.047555in}{0.688724in}}{\pgfqpoint{1.055369in}{0.696537in}}%
\pgfpathcurveto{\pgfqpoint{1.063182in}{0.704351in}}{\pgfqpoint{1.067573in}{0.714950in}}{\pgfqpoint{1.067573in}{0.726000in}}%
\pgfpathcurveto{\pgfqpoint{1.067573in}{0.737050in}}{\pgfqpoint{1.063182in}{0.747649in}}{\pgfqpoint{1.055369in}{0.755463in}}%
\pgfpathcurveto{\pgfqpoint{1.047555in}{0.763276in}}{\pgfqpoint{1.036956in}{0.767667in}}{\pgfqpoint{1.025906in}{0.767667in}}%
\pgfpathcurveto{\pgfqpoint{1.014856in}{0.767667in}}{\pgfqpoint{1.004257in}{0.763276in}}{\pgfqpoint{0.996443in}{0.755463in}}%
\pgfpathcurveto{\pgfqpoint{0.988630in}{0.747649in}}{\pgfqpoint{0.984239in}{0.737050in}}{\pgfqpoint{0.984239in}{0.726000in}}%
\pgfpathcurveto{\pgfqpoint{0.984239in}{0.714950in}}{\pgfqpoint{0.988630in}{0.704351in}}{\pgfqpoint{0.996443in}{0.696537in}}%
\pgfpathcurveto{\pgfqpoint{1.004257in}{0.688724in}}{\pgfqpoint{1.014856in}{0.684333in}}{\pgfqpoint{1.025906in}{0.684333in}}%
\pgfpathclose%
\pgfusepath{stroke,fill}%
\end{pgfscope}%
\begin{pgfscope}%
\pgfpathrectangle{\pgfqpoint{0.800000in}{0.528000in}}{\pgfqpoint{4.960000in}{3.696000in}}%
\pgfusepath{clip}%
\pgfsetbuttcap%
\pgfsetroundjoin%
\definecolor{currentfill}{rgb}{0.000000,0.000000,0.000000}%
\pgfsetfillcolor{currentfill}%
\pgfsetlinewidth{1.003750pt}%
\definecolor{currentstroke}{rgb}{0.000000,0.000000,0.000000}%
\pgfsetstrokecolor{currentstroke}%
\pgfsetdash{}{0pt}%
\pgfpathmoveto{\pgfqpoint{1.025906in}{0.684333in}}%
\pgfpathcurveto{\pgfqpoint{1.036956in}{0.684333in}}{\pgfqpoint{1.047555in}{0.688724in}}{\pgfqpoint{1.055369in}{0.696537in}}%
\pgfpathcurveto{\pgfqpoint{1.063182in}{0.704351in}}{\pgfqpoint{1.067573in}{0.714950in}}{\pgfqpoint{1.067573in}{0.726000in}}%
\pgfpathcurveto{\pgfqpoint{1.067573in}{0.737050in}}{\pgfqpoint{1.063182in}{0.747649in}}{\pgfqpoint{1.055369in}{0.755463in}}%
\pgfpathcurveto{\pgfqpoint{1.047555in}{0.763276in}}{\pgfqpoint{1.036956in}{0.767667in}}{\pgfqpoint{1.025906in}{0.767667in}}%
\pgfpathcurveto{\pgfqpoint{1.014856in}{0.767667in}}{\pgfqpoint{1.004257in}{0.763276in}}{\pgfqpoint{0.996443in}{0.755463in}}%
\pgfpathcurveto{\pgfqpoint{0.988630in}{0.747649in}}{\pgfqpoint{0.984239in}{0.737050in}}{\pgfqpoint{0.984239in}{0.726000in}}%
\pgfpathcurveto{\pgfqpoint{0.984239in}{0.714950in}}{\pgfqpoint{0.988630in}{0.704351in}}{\pgfqpoint{0.996443in}{0.696537in}}%
\pgfpathcurveto{\pgfqpoint{1.004257in}{0.688724in}}{\pgfqpoint{1.014856in}{0.684333in}}{\pgfqpoint{1.025906in}{0.684333in}}%
\pgfpathclose%
\pgfusepath{stroke,fill}%
\end{pgfscope}%
\begin{pgfscope}%
\pgfpathrectangle{\pgfqpoint{0.800000in}{0.528000in}}{\pgfqpoint{4.960000in}{3.696000in}}%
\pgfusepath{clip}%
\pgfsetbuttcap%
\pgfsetroundjoin%
\definecolor{currentfill}{rgb}{0.000000,0.000000,0.000000}%
\pgfsetfillcolor{currentfill}%
\pgfsetlinewidth{1.003750pt}%
\definecolor{currentstroke}{rgb}{0.000000,0.000000,0.000000}%
\pgfsetstrokecolor{currentstroke}%
\pgfsetdash{}{0pt}%
\pgfpathmoveto{\pgfqpoint{1.025906in}{0.684333in}}%
\pgfpathcurveto{\pgfqpoint{1.036956in}{0.684333in}}{\pgfqpoint{1.047555in}{0.688724in}}{\pgfqpoint{1.055369in}{0.696537in}}%
\pgfpathcurveto{\pgfqpoint{1.063182in}{0.704351in}}{\pgfqpoint{1.067573in}{0.714950in}}{\pgfqpoint{1.067573in}{0.726000in}}%
\pgfpathcurveto{\pgfqpoint{1.067573in}{0.737050in}}{\pgfqpoint{1.063182in}{0.747649in}}{\pgfqpoint{1.055369in}{0.755463in}}%
\pgfpathcurveto{\pgfqpoint{1.047555in}{0.763276in}}{\pgfqpoint{1.036956in}{0.767667in}}{\pgfqpoint{1.025906in}{0.767667in}}%
\pgfpathcurveto{\pgfqpoint{1.014856in}{0.767667in}}{\pgfqpoint{1.004257in}{0.763276in}}{\pgfqpoint{0.996443in}{0.755463in}}%
\pgfpathcurveto{\pgfqpoint{0.988630in}{0.747649in}}{\pgfqpoint{0.984239in}{0.737050in}}{\pgfqpoint{0.984239in}{0.726000in}}%
\pgfpathcurveto{\pgfqpoint{0.984239in}{0.714950in}}{\pgfqpoint{0.988630in}{0.704351in}}{\pgfqpoint{0.996443in}{0.696537in}}%
\pgfpathcurveto{\pgfqpoint{1.004257in}{0.688724in}}{\pgfqpoint{1.014856in}{0.684333in}}{\pgfqpoint{1.025906in}{0.684333in}}%
\pgfpathclose%
\pgfusepath{stroke,fill}%
\end{pgfscope}%
\begin{pgfscope}%
\pgfpathrectangle{\pgfqpoint{0.800000in}{0.528000in}}{\pgfqpoint{4.960000in}{3.696000in}}%
\pgfusepath{clip}%
\pgfsetbuttcap%
\pgfsetroundjoin%
\definecolor{currentfill}{rgb}{0.000000,0.000000,0.000000}%
\pgfsetfillcolor{currentfill}%
\pgfsetlinewidth{1.003750pt}%
\definecolor{currentstroke}{rgb}{0.000000,0.000000,0.000000}%
\pgfsetstrokecolor{currentstroke}%
\pgfsetdash{}{0pt}%
\pgfpathmoveto{\pgfqpoint{1.025906in}{0.684333in}}%
\pgfpathcurveto{\pgfqpoint{1.036956in}{0.684333in}}{\pgfqpoint{1.047555in}{0.688724in}}{\pgfqpoint{1.055369in}{0.696537in}}%
\pgfpathcurveto{\pgfqpoint{1.063182in}{0.704351in}}{\pgfqpoint{1.067573in}{0.714950in}}{\pgfqpoint{1.067573in}{0.726000in}}%
\pgfpathcurveto{\pgfqpoint{1.067573in}{0.737050in}}{\pgfqpoint{1.063182in}{0.747649in}}{\pgfqpoint{1.055369in}{0.755463in}}%
\pgfpathcurveto{\pgfqpoint{1.047555in}{0.763276in}}{\pgfqpoint{1.036956in}{0.767667in}}{\pgfqpoint{1.025906in}{0.767667in}}%
\pgfpathcurveto{\pgfqpoint{1.014856in}{0.767667in}}{\pgfqpoint{1.004257in}{0.763276in}}{\pgfqpoint{0.996443in}{0.755463in}}%
\pgfpathcurveto{\pgfqpoint{0.988630in}{0.747649in}}{\pgfqpoint{0.984239in}{0.737050in}}{\pgfqpoint{0.984239in}{0.726000in}}%
\pgfpathcurveto{\pgfqpoint{0.984239in}{0.714950in}}{\pgfqpoint{0.988630in}{0.704351in}}{\pgfqpoint{0.996443in}{0.696537in}}%
\pgfpathcurveto{\pgfqpoint{1.004257in}{0.688724in}}{\pgfqpoint{1.014856in}{0.684333in}}{\pgfqpoint{1.025906in}{0.684333in}}%
\pgfpathclose%
\pgfusepath{stroke,fill}%
\end{pgfscope}%
\begin{pgfscope}%
\pgfpathrectangle{\pgfqpoint{0.800000in}{0.528000in}}{\pgfqpoint{4.960000in}{3.696000in}}%
\pgfusepath{clip}%
\pgfsetbuttcap%
\pgfsetroundjoin%
\definecolor{currentfill}{rgb}{0.000000,0.000000,0.000000}%
\pgfsetfillcolor{currentfill}%
\pgfsetlinewidth{1.003750pt}%
\definecolor{currentstroke}{rgb}{0.000000,0.000000,0.000000}%
\pgfsetstrokecolor{currentstroke}%
\pgfsetdash{}{0pt}%
\pgfpathmoveto{\pgfqpoint{1.025906in}{0.684333in}}%
\pgfpathcurveto{\pgfqpoint{1.036956in}{0.684333in}}{\pgfqpoint{1.047555in}{0.688724in}}{\pgfqpoint{1.055369in}{0.696537in}}%
\pgfpathcurveto{\pgfqpoint{1.063182in}{0.704351in}}{\pgfqpoint{1.067573in}{0.714950in}}{\pgfqpoint{1.067573in}{0.726000in}}%
\pgfpathcurveto{\pgfqpoint{1.067573in}{0.737050in}}{\pgfqpoint{1.063182in}{0.747649in}}{\pgfqpoint{1.055369in}{0.755463in}}%
\pgfpathcurveto{\pgfqpoint{1.047555in}{0.763276in}}{\pgfqpoint{1.036956in}{0.767667in}}{\pgfqpoint{1.025906in}{0.767667in}}%
\pgfpathcurveto{\pgfqpoint{1.014856in}{0.767667in}}{\pgfqpoint{1.004257in}{0.763276in}}{\pgfqpoint{0.996443in}{0.755463in}}%
\pgfpathcurveto{\pgfqpoint{0.988630in}{0.747649in}}{\pgfqpoint{0.984239in}{0.737050in}}{\pgfqpoint{0.984239in}{0.726000in}}%
\pgfpathcurveto{\pgfqpoint{0.984239in}{0.714950in}}{\pgfqpoint{0.988630in}{0.704351in}}{\pgfqpoint{0.996443in}{0.696537in}}%
\pgfpathcurveto{\pgfqpoint{1.004257in}{0.688724in}}{\pgfqpoint{1.014856in}{0.684333in}}{\pgfqpoint{1.025906in}{0.684333in}}%
\pgfpathclose%
\pgfusepath{stroke,fill}%
\end{pgfscope}%
\begin{pgfscope}%
\pgfpathrectangle{\pgfqpoint{0.800000in}{0.528000in}}{\pgfqpoint{4.960000in}{3.696000in}}%
\pgfusepath{clip}%
\pgfsetbuttcap%
\pgfsetroundjoin%
\definecolor{currentfill}{rgb}{0.000000,0.000000,0.000000}%
\pgfsetfillcolor{currentfill}%
\pgfsetlinewidth{1.003750pt}%
\definecolor{currentstroke}{rgb}{0.000000,0.000000,0.000000}%
\pgfsetstrokecolor{currentstroke}%
\pgfsetdash{}{0pt}%
\pgfpathmoveto{\pgfqpoint{1.025906in}{0.684333in}}%
\pgfpathcurveto{\pgfqpoint{1.036956in}{0.684333in}}{\pgfqpoint{1.047555in}{0.688724in}}{\pgfqpoint{1.055369in}{0.696537in}}%
\pgfpathcurveto{\pgfqpoint{1.063182in}{0.704351in}}{\pgfqpoint{1.067573in}{0.714950in}}{\pgfqpoint{1.067573in}{0.726000in}}%
\pgfpathcurveto{\pgfqpoint{1.067573in}{0.737050in}}{\pgfqpoint{1.063182in}{0.747649in}}{\pgfqpoint{1.055369in}{0.755463in}}%
\pgfpathcurveto{\pgfqpoint{1.047555in}{0.763276in}}{\pgfqpoint{1.036956in}{0.767667in}}{\pgfqpoint{1.025906in}{0.767667in}}%
\pgfpathcurveto{\pgfqpoint{1.014856in}{0.767667in}}{\pgfqpoint{1.004257in}{0.763276in}}{\pgfqpoint{0.996443in}{0.755463in}}%
\pgfpathcurveto{\pgfqpoint{0.988630in}{0.747649in}}{\pgfqpoint{0.984239in}{0.737050in}}{\pgfqpoint{0.984239in}{0.726000in}}%
\pgfpathcurveto{\pgfqpoint{0.984239in}{0.714950in}}{\pgfqpoint{0.988630in}{0.704351in}}{\pgfqpoint{0.996443in}{0.696537in}}%
\pgfpathcurveto{\pgfqpoint{1.004257in}{0.688724in}}{\pgfqpoint{1.014856in}{0.684333in}}{\pgfqpoint{1.025906in}{0.684333in}}%
\pgfpathclose%
\pgfusepath{stroke,fill}%
\end{pgfscope}%
\begin{pgfscope}%
\pgfpathrectangle{\pgfqpoint{0.800000in}{0.528000in}}{\pgfqpoint{4.960000in}{3.696000in}}%
\pgfusepath{clip}%
\pgfsetbuttcap%
\pgfsetroundjoin%
\definecolor{currentfill}{rgb}{0.000000,0.000000,0.000000}%
\pgfsetfillcolor{currentfill}%
\pgfsetlinewidth{1.003750pt}%
\definecolor{currentstroke}{rgb}{0.000000,0.000000,0.000000}%
\pgfsetstrokecolor{currentstroke}%
\pgfsetdash{}{0pt}%
\pgfpathmoveto{\pgfqpoint{1.025906in}{0.684333in}}%
\pgfpathcurveto{\pgfqpoint{1.036956in}{0.684333in}}{\pgfqpoint{1.047555in}{0.688724in}}{\pgfqpoint{1.055369in}{0.696537in}}%
\pgfpathcurveto{\pgfqpoint{1.063182in}{0.704351in}}{\pgfqpoint{1.067573in}{0.714950in}}{\pgfqpoint{1.067573in}{0.726000in}}%
\pgfpathcurveto{\pgfqpoint{1.067573in}{0.737050in}}{\pgfqpoint{1.063182in}{0.747649in}}{\pgfqpoint{1.055369in}{0.755463in}}%
\pgfpathcurveto{\pgfqpoint{1.047555in}{0.763276in}}{\pgfqpoint{1.036956in}{0.767667in}}{\pgfqpoint{1.025906in}{0.767667in}}%
\pgfpathcurveto{\pgfqpoint{1.014856in}{0.767667in}}{\pgfqpoint{1.004257in}{0.763276in}}{\pgfqpoint{0.996443in}{0.755463in}}%
\pgfpathcurveto{\pgfqpoint{0.988630in}{0.747649in}}{\pgfqpoint{0.984239in}{0.737050in}}{\pgfqpoint{0.984239in}{0.726000in}}%
\pgfpathcurveto{\pgfqpoint{0.984239in}{0.714950in}}{\pgfqpoint{0.988630in}{0.704351in}}{\pgfqpoint{0.996443in}{0.696537in}}%
\pgfpathcurveto{\pgfqpoint{1.004257in}{0.688724in}}{\pgfqpoint{1.014856in}{0.684333in}}{\pgfqpoint{1.025906in}{0.684333in}}%
\pgfpathclose%
\pgfusepath{stroke,fill}%
\end{pgfscope}%
\begin{pgfscope}%
\pgfpathrectangle{\pgfqpoint{0.800000in}{0.528000in}}{\pgfqpoint{4.960000in}{3.696000in}}%
\pgfusepath{clip}%
\pgfsetbuttcap%
\pgfsetroundjoin%
\definecolor{currentfill}{rgb}{0.000000,0.000000,0.000000}%
\pgfsetfillcolor{currentfill}%
\pgfsetlinewidth{1.003750pt}%
\definecolor{currentstroke}{rgb}{0.000000,0.000000,0.000000}%
\pgfsetstrokecolor{currentstroke}%
\pgfsetdash{}{0pt}%
\pgfpathmoveto{\pgfqpoint{1.025906in}{0.684333in}}%
\pgfpathcurveto{\pgfqpoint{1.036956in}{0.684333in}}{\pgfqpoint{1.047555in}{0.688724in}}{\pgfqpoint{1.055369in}{0.696537in}}%
\pgfpathcurveto{\pgfqpoint{1.063182in}{0.704351in}}{\pgfqpoint{1.067573in}{0.714950in}}{\pgfqpoint{1.067573in}{0.726000in}}%
\pgfpathcurveto{\pgfqpoint{1.067573in}{0.737050in}}{\pgfqpoint{1.063182in}{0.747649in}}{\pgfqpoint{1.055369in}{0.755463in}}%
\pgfpathcurveto{\pgfqpoint{1.047555in}{0.763276in}}{\pgfqpoint{1.036956in}{0.767667in}}{\pgfqpoint{1.025906in}{0.767667in}}%
\pgfpathcurveto{\pgfqpoint{1.014856in}{0.767667in}}{\pgfqpoint{1.004257in}{0.763276in}}{\pgfqpoint{0.996443in}{0.755463in}}%
\pgfpathcurveto{\pgfqpoint{0.988630in}{0.747649in}}{\pgfqpoint{0.984239in}{0.737050in}}{\pgfqpoint{0.984239in}{0.726000in}}%
\pgfpathcurveto{\pgfqpoint{0.984239in}{0.714950in}}{\pgfqpoint{0.988630in}{0.704351in}}{\pgfqpoint{0.996443in}{0.696537in}}%
\pgfpathcurveto{\pgfqpoint{1.004257in}{0.688724in}}{\pgfqpoint{1.014856in}{0.684333in}}{\pgfqpoint{1.025906in}{0.684333in}}%
\pgfpathclose%
\pgfusepath{stroke,fill}%
\end{pgfscope}%
\begin{pgfscope}%
\pgfpathrectangle{\pgfqpoint{0.800000in}{0.528000in}}{\pgfqpoint{4.960000in}{3.696000in}}%
\pgfusepath{clip}%
\pgfsetbuttcap%
\pgfsetroundjoin%
\definecolor{currentfill}{rgb}{0.000000,0.000000,0.000000}%
\pgfsetfillcolor{currentfill}%
\pgfsetlinewidth{1.003750pt}%
\definecolor{currentstroke}{rgb}{0.000000,0.000000,0.000000}%
\pgfsetstrokecolor{currentstroke}%
\pgfsetdash{}{0pt}%
\pgfpathmoveto{\pgfqpoint{1.025906in}{0.684333in}}%
\pgfpathcurveto{\pgfqpoint{1.036956in}{0.684333in}}{\pgfqpoint{1.047555in}{0.688724in}}{\pgfqpoint{1.055369in}{0.696537in}}%
\pgfpathcurveto{\pgfqpoint{1.063182in}{0.704351in}}{\pgfqpoint{1.067573in}{0.714950in}}{\pgfqpoint{1.067573in}{0.726000in}}%
\pgfpathcurveto{\pgfqpoint{1.067573in}{0.737050in}}{\pgfqpoint{1.063182in}{0.747649in}}{\pgfqpoint{1.055369in}{0.755463in}}%
\pgfpathcurveto{\pgfqpoint{1.047555in}{0.763276in}}{\pgfqpoint{1.036956in}{0.767667in}}{\pgfqpoint{1.025906in}{0.767667in}}%
\pgfpathcurveto{\pgfqpoint{1.014856in}{0.767667in}}{\pgfqpoint{1.004257in}{0.763276in}}{\pgfqpoint{0.996443in}{0.755463in}}%
\pgfpathcurveto{\pgfqpoint{0.988630in}{0.747649in}}{\pgfqpoint{0.984239in}{0.737050in}}{\pgfqpoint{0.984239in}{0.726000in}}%
\pgfpathcurveto{\pgfqpoint{0.984239in}{0.714950in}}{\pgfqpoint{0.988630in}{0.704351in}}{\pgfqpoint{0.996443in}{0.696537in}}%
\pgfpathcurveto{\pgfqpoint{1.004257in}{0.688724in}}{\pgfqpoint{1.014856in}{0.684333in}}{\pgfqpoint{1.025906in}{0.684333in}}%
\pgfpathclose%
\pgfusepath{stroke,fill}%
\end{pgfscope}%
\begin{pgfscope}%
\pgfpathrectangle{\pgfqpoint{0.800000in}{0.528000in}}{\pgfqpoint{4.960000in}{3.696000in}}%
\pgfusepath{clip}%
\pgfsetbuttcap%
\pgfsetroundjoin%
\definecolor{currentfill}{rgb}{0.000000,0.000000,0.000000}%
\pgfsetfillcolor{currentfill}%
\pgfsetlinewidth{1.003750pt}%
\definecolor{currentstroke}{rgb}{0.000000,0.000000,0.000000}%
\pgfsetstrokecolor{currentstroke}%
\pgfsetdash{}{0pt}%
\pgfpathmoveto{\pgfqpoint{1.025906in}{0.684333in}}%
\pgfpathcurveto{\pgfqpoint{1.036956in}{0.684333in}}{\pgfqpoint{1.047555in}{0.688724in}}{\pgfqpoint{1.055369in}{0.696537in}}%
\pgfpathcurveto{\pgfqpoint{1.063182in}{0.704351in}}{\pgfqpoint{1.067573in}{0.714950in}}{\pgfqpoint{1.067573in}{0.726000in}}%
\pgfpathcurveto{\pgfqpoint{1.067573in}{0.737050in}}{\pgfqpoint{1.063182in}{0.747649in}}{\pgfqpoint{1.055369in}{0.755463in}}%
\pgfpathcurveto{\pgfqpoint{1.047555in}{0.763276in}}{\pgfqpoint{1.036956in}{0.767667in}}{\pgfqpoint{1.025906in}{0.767667in}}%
\pgfpathcurveto{\pgfqpoint{1.014856in}{0.767667in}}{\pgfqpoint{1.004257in}{0.763276in}}{\pgfqpoint{0.996443in}{0.755463in}}%
\pgfpathcurveto{\pgfqpoint{0.988630in}{0.747649in}}{\pgfqpoint{0.984239in}{0.737050in}}{\pgfqpoint{0.984239in}{0.726000in}}%
\pgfpathcurveto{\pgfqpoint{0.984239in}{0.714950in}}{\pgfqpoint{0.988630in}{0.704351in}}{\pgfqpoint{0.996443in}{0.696537in}}%
\pgfpathcurveto{\pgfqpoint{1.004257in}{0.688724in}}{\pgfqpoint{1.014856in}{0.684333in}}{\pgfqpoint{1.025906in}{0.684333in}}%
\pgfpathclose%
\pgfusepath{stroke,fill}%
\end{pgfscope}%
\begin{pgfscope}%
\pgfpathrectangle{\pgfqpoint{0.800000in}{0.528000in}}{\pgfqpoint{4.960000in}{3.696000in}}%
\pgfusepath{clip}%
\pgfsetbuttcap%
\pgfsetroundjoin%
\definecolor{currentfill}{rgb}{0.000000,0.000000,0.000000}%
\pgfsetfillcolor{currentfill}%
\pgfsetlinewidth{1.003750pt}%
\definecolor{currentstroke}{rgb}{0.000000,0.000000,0.000000}%
\pgfsetstrokecolor{currentstroke}%
\pgfsetdash{}{0pt}%
\pgfpathmoveto{\pgfqpoint{1.025906in}{0.684333in}}%
\pgfpathcurveto{\pgfqpoint{1.036956in}{0.684333in}}{\pgfqpoint{1.047555in}{0.688724in}}{\pgfqpoint{1.055369in}{0.696537in}}%
\pgfpathcurveto{\pgfqpoint{1.063182in}{0.704351in}}{\pgfqpoint{1.067573in}{0.714950in}}{\pgfqpoint{1.067573in}{0.726000in}}%
\pgfpathcurveto{\pgfqpoint{1.067573in}{0.737050in}}{\pgfqpoint{1.063182in}{0.747649in}}{\pgfqpoint{1.055369in}{0.755463in}}%
\pgfpathcurveto{\pgfqpoint{1.047555in}{0.763276in}}{\pgfqpoint{1.036956in}{0.767667in}}{\pgfqpoint{1.025906in}{0.767667in}}%
\pgfpathcurveto{\pgfqpoint{1.014856in}{0.767667in}}{\pgfqpoint{1.004257in}{0.763276in}}{\pgfqpoint{0.996443in}{0.755463in}}%
\pgfpathcurveto{\pgfqpoint{0.988630in}{0.747649in}}{\pgfqpoint{0.984239in}{0.737050in}}{\pgfqpoint{0.984239in}{0.726000in}}%
\pgfpathcurveto{\pgfqpoint{0.984239in}{0.714950in}}{\pgfqpoint{0.988630in}{0.704351in}}{\pgfqpoint{0.996443in}{0.696537in}}%
\pgfpathcurveto{\pgfqpoint{1.004257in}{0.688724in}}{\pgfqpoint{1.014856in}{0.684333in}}{\pgfqpoint{1.025906in}{0.684333in}}%
\pgfpathclose%
\pgfusepath{stroke,fill}%
\end{pgfscope}%
\begin{pgfscope}%
\pgfpathrectangle{\pgfqpoint{0.800000in}{0.528000in}}{\pgfqpoint{4.960000in}{3.696000in}}%
\pgfusepath{clip}%
\pgfsetbuttcap%
\pgfsetroundjoin%
\definecolor{currentfill}{rgb}{0.000000,0.000000,0.000000}%
\pgfsetfillcolor{currentfill}%
\pgfsetlinewidth{1.003750pt}%
\definecolor{currentstroke}{rgb}{0.000000,0.000000,0.000000}%
\pgfsetstrokecolor{currentstroke}%
\pgfsetdash{}{0pt}%
\pgfpathmoveto{\pgfqpoint{1.025906in}{0.684333in}}%
\pgfpathcurveto{\pgfqpoint{1.036956in}{0.684333in}}{\pgfqpoint{1.047555in}{0.688724in}}{\pgfqpoint{1.055369in}{0.696537in}}%
\pgfpathcurveto{\pgfqpoint{1.063182in}{0.704351in}}{\pgfqpoint{1.067573in}{0.714950in}}{\pgfqpoint{1.067573in}{0.726000in}}%
\pgfpathcurveto{\pgfqpoint{1.067573in}{0.737050in}}{\pgfqpoint{1.063182in}{0.747649in}}{\pgfqpoint{1.055369in}{0.755463in}}%
\pgfpathcurveto{\pgfqpoint{1.047555in}{0.763276in}}{\pgfqpoint{1.036956in}{0.767667in}}{\pgfqpoint{1.025906in}{0.767667in}}%
\pgfpathcurveto{\pgfqpoint{1.014856in}{0.767667in}}{\pgfqpoint{1.004257in}{0.763276in}}{\pgfqpoint{0.996443in}{0.755463in}}%
\pgfpathcurveto{\pgfqpoint{0.988630in}{0.747649in}}{\pgfqpoint{0.984239in}{0.737050in}}{\pgfqpoint{0.984239in}{0.726000in}}%
\pgfpathcurveto{\pgfqpoint{0.984239in}{0.714950in}}{\pgfqpoint{0.988630in}{0.704351in}}{\pgfqpoint{0.996443in}{0.696537in}}%
\pgfpathcurveto{\pgfqpoint{1.004257in}{0.688724in}}{\pgfqpoint{1.014856in}{0.684333in}}{\pgfqpoint{1.025906in}{0.684333in}}%
\pgfpathclose%
\pgfusepath{stroke,fill}%
\end{pgfscope}%
\begin{pgfscope}%
\pgfpathrectangle{\pgfqpoint{0.800000in}{0.528000in}}{\pgfqpoint{4.960000in}{3.696000in}}%
\pgfusepath{clip}%
\pgfsetbuttcap%
\pgfsetroundjoin%
\definecolor{currentfill}{rgb}{0.000000,0.000000,0.000000}%
\pgfsetfillcolor{currentfill}%
\pgfsetlinewidth{1.003750pt}%
\definecolor{currentstroke}{rgb}{0.000000,0.000000,0.000000}%
\pgfsetstrokecolor{currentstroke}%
\pgfsetdash{}{0pt}%
\pgfpathmoveto{\pgfqpoint{1.025906in}{0.684333in}}%
\pgfpathcurveto{\pgfqpoint{1.036956in}{0.684333in}}{\pgfqpoint{1.047555in}{0.688724in}}{\pgfqpoint{1.055369in}{0.696537in}}%
\pgfpathcurveto{\pgfqpoint{1.063182in}{0.704351in}}{\pgfqpoint{1.067573in}{0.714950in}}{\pgfqpoint{1.067573in}{0.726000in}}%
\pgfpathcurveto{\pgfqpoint{1.067573in}{0.737050in}}{\pgfqpoint{1.063182in}{0.747649in}}{\pgfqpoint{1.055369in}{0.755463in}}%
\pgfpathcurveto{\pgfqpoint{1.047555in}{0.763276in}}{\pgfqpoint{1.036956in}{0.767667in}}{\pgfqpoint{1.025906in}{0.767667in}}%
\pgfpathcurveto{\pgfqpoint{1.014856in}{0.767667in}}{\pgfqpoint{1.004257in}{0.763276in}}{\pgfqpoint{0.996443in}{0.755463in}}%
\pgfpathcurveto{\pgfqpoint{0.988630in}{0.747649in}}{\pgfqpoint{0.984239in}{0.737050in}}{\pgfqpoint{0.984239in}{0.726000in}}%
\pgfpathcurveto{\pgfqpoint{0.984239in}{0.714950in}}{\pgfqpoint{0.988630in}{0.704351in}}{\pgfqpoint{0.996443in}{0.696537in}}%
\pgfpathcurveto{\pgfqpoint{1.004257in}{0.688724in}}{\pgfqpoint{1.014856in}{0.684333in}}{\pgfqpoint{1.025906in}{0.684333in}}%
\pgfpathclose%
\pgfusepath{stroke,fill}%
\end{pgfscope}%
\begin{pgfscope}%
\pgfpathrectangle{\pgfqpoint{0.800000in}{0.528000in}}{\pgfqpoint{4.960000in}{3.696000in}}%
\pgfusepath{clip}%
\pgfsetbuttcap%
\pgfsetroundjoin%
\definecolor{currentfill}{rgb}{0.000000,0.000000,0.000000}%
\pgfsetfillcolor{currentfill}%
\pgfsetlinewidth{1.003750pt}%
\definecolor{currentstroke}{rgb}{0.000000,0.000000,0.000000}%
\pgfsetstrokecolor{currentstroke}%
\pgfsetdash{}{0pt}%
\pgfpathmoveto{\pgfqpoint{1.025906in}{0.684333in}}%
\pgfpathcurveto{\pgfqpoint{1.036956in}{0.684333in}}{\pgfqpoint{1.047555in}{0.688724in}}{\pgfqpoint{1.055369in}{0.696537in}}%
\pgfpathcurveto{\pgfqpoint{1.063182in}{0.704351in}}{\pgfqpoint{1.067573in}{0.714950in}}{\pgfqpoint{1.067573in}{0.726000in}}%
\pgfpathcurveto{\pgfqpoint{1.067573in}{0.737050in}}{\pgfqpoint{1.063182in}{0.747649in}}{\pgfqpoint{1.055369in}{0.755463in}}%
\pgfpathcurveto{\pgfqpoint{1.047555in}{0.763276in}}{\pgfqpoint{1.036956in}{0.767667in}}{\pgfqpoint{1.025906in}{0.767667in}}%
\pgfpathcurveto{\pgfqpoint{1.014856in}{0.767667in}}{\pgfqpoint{1.004257in}{0.763276in}}{\pgfqpoint{0.996443in}{0.755463in}}%
\pgfpathcurveto{\pgfqpoint{0.988630in}{0.747649in}}{\pgfqpoint{0.984239in}{0.737050in}}{\pgfqpoint{0.984239in}{0.726000in}}%
\pgfpathcurveto{\pgfqpoint{0.984239in}{0.714950in}}{\pgfqpoint{0.988630in}{0.704351in}}{\pgfqpoint{0.996443in}{0.696537in}}%
\pgfpathcurveto{\pgfqpoint{1.004257in}{0.688724in}}{\pgfqpoint{1.014856in}{0.684333in}}{\pgfqpoint{1.025906in}{0.684333in}}%
\pgfpathclose%
\pgfusepath{stroke,fill}%
\end{pgfscope}%
\begin{pgfscope}%
\pgfpathrectangle{\pgfqpoint{0.800000in}{0.528000in}}{\pgfqpoint{4.960000in}{3.696000in}}%
\pgfusepath{clip}%
\pgfsetbuttcap%
\pgfsetroundjoin%
\definecolor{currentfill}{rgb}{0.000000,0.000000,0.000000}%
\pgfsetfillcolor{currentfill}%
\pgfsetlinewidth{1.003750pt}%
\definecolor{currentstroke}{rgb}{0.000000,0.000000,0.000000}%
\pgfsetstrokecolor{currentstroke}%
\pgfsetdash{}{0pt}%
\pgfpathmoveto{\pgfqpoint{1.025906in}{0.684333in}}%
\pgfpathcurveto{\pgfqpoint{1.036956in}{0.684333in}}{\pgfqpoint{1.047555in}{0.688724in}}{\pgfqpoint{1.055369in}{0.696537in}}%
\pgfpathcurveto{\pgfqpoint{1.063182in}{0.704351in}}{\pgfqpoint{1.067573in}{0.714950in}}{\pgfqpoint{1.067573in}{0.726000in}}%
\pgfpathcurveto{\pgfqpoint{1.067573in}{0.737050in}}{\pgfqpoint{1.063182in}{0.747649in}}{\pgfqpoint{1.055369in}{0.755463in}}%
\pgfpathcurveto{\pgfqpoint{1.047555in}{0.763276in}}{\pgfqpoint{1.036956in}{0.767667in}}{\pgfqpoint{1.025906in}{0.767667in}}%
\pgfpathcurveto{\pgfqpoint{1.014856in}{0.767667in}}{\pgfqpoint{1.004257in}{0.763276in}}{\pgfqpoint{0.996443in}{0.755463in}}%
\pgfpathcurveto{\pgfqpoint{0.988630in}{0.747649in}}{\pgfqpoint{0.984239in}{0.737050in}}{\pgfqpoint{0.984239in}{0.726000in}}%
\pgfpathcurveto{\pgfqpoint{0.984239in}{0.714950in}}{\pgfqpoint{0.988630in}{0.704351in}}{\pgfqpoint{0.996443in}{0.696537in}}%
\pgfpathcurveto{\pgfqpoint{1.004257in}{0.688724in}}{\pgfqpoint{1.014856in}{0.684333in}}{\pgfqpoint{1.025906in}{0.684333in}}%
\pgfpathclose%
\pgfusepath{stroke,fill}%
\end{pgfscope}%
\begin{pgfscope}%
\pgfpathrectangle{\pgfqpoint{0.800000in}{0.528000in}}{\pgfqpoint{4.960000in}{3.696000in}}%
\pgfusepath{clip}%
\pgfsetbuttcap%
\pgfsetroundjoin%
\definecolor{currentfill}{rgb}{0.000000,0.000000,0.000000}%
\pgfsetfillcolor{currentfill}%
\pgfsetlinewidth{1.003750pt}%
\definecolor{currentstroke}{rgb}{0.000000,0.000000,0.000000}%
\pgfsetstrokecolor{currentstroke}%
\pgfsetdash{}{0pt}%
\pgfpathmoveto{\pgfqpoint{1.025906in}{0.684333in}}%
\pgfpathcurveto{\pgfqpoint{1.036956in}{0.684333in}}{\pgfqpoint{1.047555in}{0.688724in}}{\pgfqpoint{1.055369in}{0.696537in}}%
\pgfpathcurveto{\pgfqpoint{1.063182in}{0.704351in}}{\pgfqpoint{1.067573in}{0.714950in}}{\pgfqpoint{1.067573in}{0.726000in}}%
\pgfpathcurveto{\pgfqpoint{1.067573in}{0.737050in}}{\pgfqpoint{1.063182in}{0.747649in}}{\pgfqpoint{1.055369in}{0.755463in}}%
\pgfpathcurveto{\pgfqpoint{1.047555in}{0.763276in}}{\pgfqpoint{1.036956in}{0.767667in}}{\pgfqpoint{1.025906in}{0.767667in}}%
\pgfpathcurveto{\pgfqpoint{1.014856in}{0.767667in}}{\pgfqpoint{1.004257in}{0.763276in}}{\pgfqpoint{0.996443in}{0.755463in}}%
\pgfpathcurveto{\pgfqpoint{0.988630in}{0.747649in}}{\pgfqpoint{0.984239in}{0.737050in}}{\pgfqpoint{0.984239in}{0.726000in}}%
\pgfpathcurveto{\pgfqpoint{0.984239in}{0.714950in}}{\pgfqpoint{0.988630in}{0.704351in}}{\pgfqpoint{0.996443in}{0.696537in}}%
\pgfpathcurveto{\pgfqpoint{1.004257in}{0.688724in}}{\pgfqpoint{1.014856in}{0.684333in}}{\pgfqpoint{1.025906in}{0.684333in}}%
\pgfpathclose%
\pgfusepath{stroke,fill}%
\end{pgfscope}%
\begin{pgfscope}%
\pgfpathrectangle{\pgfqpoint{0.800000in}{0.528000in}}{\pgfqpoint{4.960000in}{3.696000in}}%
\pgfusepath{clip}%
\pgfsetbuttcap%
\pgfsetroundjoin%
\definecolor{currentfill}{rgb}{0.000000,0.000000,0.000000}%
\pgfsetfillcolor{currentfill}%
\pgfsetlinewidth{1.003750pt}%
\definecolor{currentstroke}{rgb}{0.000000,0.000000,0.000000}%
\pgfsetstrokecolor{currentstroke}%
\pgfsetdash{}{0pt}%
\pgfpathmoveto{\pgfqpoint{1.025906in}{0.684333in}}%
\pgfpathcurveto{\pgfqpoint{1.036956in}{0.684333in}}{\pgfqpoint{1.047555in}{0.688724in}}{\pgfqpoint{1.055369in}{0.696537in}}%
\pgfpathcurveto{\pgfqpoint{1.063182in}{0.704351in}}{\pgfqpoint{1.067573in}{0.714950in}}{\pgfqpoint{1.067573in}{0.726000in}}%
\pgfpathcurveto{\pgfqpoint{1.067573in}{0.737050in}}{\pgfqpoint{1.063182in}{0.747649in}}{\pgfqpoint{1.055369in}{0.755463in}}%
\pgfpathcurveto{\pgfqpoint{1.047555in}{0.763276in}}{\pgfqpoint{1.036956in}{0.767667in}}{\pgfqpoint{1.025906in}{0.767667in}}%
\pgfpathcurveto{\pgfqpoint{1.014856in}{0.767667in}}{\pgfqpoint{1.004257in}{0.763276in}}{\pgfqpoint{0.996443in}{0.755463in}}%
\pgfpathcurveto{\pgfqpoint{0.988630in}{0.747649in}}{\pgfqpoint{0.984239in}{0.737050in}}{\pgfqpoint{0.984239in}{0.726000in}}%
\pgfpathcurveto{\pgfqpoint{0.984239in}{0.714950in}}{\pgfqpoint{0.988630in}{0.704351in}}{\pgfqpoint{0.996443in}{0.696537in}}%
\pgfpathcurveto{\pgfqpoint{1.004257in}{0.688724in}}{\pgfqpoint{1.014856in}{0.684333in}}{\pgfqpoint{1.025906in}{0.684333in}}%
\pgfpathclose%
\pgfusepath{stroke,fill}%
\end{pgfscope}%
\begin{pgfscope}%
\pgfpathrectangle{\pgfqpoint{0.800000in}{0.528000in}}{\pgfqpoint{4.960000in}{3.696000in}}%
\pgfusepath{clip}%
\pgfsetbuttcap%
\pgfsetroundjoin%
\definecolor{currentfill}{rgb}{0.000000,0.000000,0.000000}%
\pgfsetfillcolor{currentfill}%
\pgfsetlinewidth{1.003750pt}%
\definecolor{currentstroke}{rgb}{0.000000,0.000000,0.000000}%
\pgfsetstrokecolor{currentstroke}%
\pgfsetdash{}{0pt}%
\pgfpathmoveto{\pgfqpoint{1.025906in}{0.684333in}}%
\pgfpathcurveto{\pgfqpoint{1.036956in}{0.684333in}}{\pgfqpoint{1.047555in}{0.688724in}}{\pgfqpoint{1.055369in}{0.696537in}}%
\pgfpathcurveto{\pgfqpoint{1.063182in}{0.704351in}}{\pgfqpoint{1.067573in}{0.714950in}}{\pgfqpoint{1.067573in}{0.726000in}}%
\pgfpathcurveto{\pgfqpoint{1.067573in}{0.737050in}}{\pgfqpoint{1.063182in}{0.747649in}}{\pgfqpoint{1.055369in}{0.755463in}}%
\pgfpathcurveto{\pgfqpoint{1.047555in}{0.763276in}}{\pgfqpoint{1.036956in}{0.767667in}}{\pgfqpoint{1.025906in}{0.767667in}}%
\pgfpathcurveto{\pgfqpoint{1.014856in}{0.767667in}}{\pgfqpoint{1.004257in}{0.763276in}}{\pgfqpoint{0.996443in}{0.755463in}}%
\pgfpathcurveto{\pgfqpoint{0.988630in}{0.747649in}}{\pgfqpoint{0.984239in}{0.737050in}}{\pgfqpoint{0.984239in}{0.726000in}}%
\pgfpathcurveto{\pgfqpoint{0.984239in}{0.714950in}}{\pgfqpoint{0.988630in}{0.704351in}}{\pgfqpoint{0.996443in}{0.696537in}}%
\pgfpathcurveto{\pgfqpoint{1.004257in}{0.688724in}}{\pgfqpoint{1.014856in}{0.684333in}}{\pgfqpoint{1.025906in}{0.684333in}}%
\pgfpathclose%
\pgfusepath{stroke,fill}%
\end{pgfscope}%
\begin{pgfscope}%
\pgfpathrectangle{\pgfqpoint{0.800000in}{0.528000in}}{\pgfqpoint{4.960000in}{3.696000in}}%
\pgfusepath{clip}%
\pgfsetbuttcap%
\pgfsetroundjoin%
\definecolor{currentfill}{rgb}{0.000000,0.000000,0.000000}%
\pgfsetfillcolor{currentfill}%
\pgfsetlinewidth{1.003750pt}%
\definecolor{currentstroke}{rgb}{0.000000,0.000000,0.000000}%
\pgfsetstrokecolor{currentstroke}%
\pgfsetdash{}{0pt}%
\pgfpathmoveto{\pgfqpoint{1.025906in}{0.684333in}}%
\pgfpathcurveto{\pgfqpoint{1.036956in}{0.684333in}}{\pgfqpoint{1.047555in}{0.688724in}}{\pgfqpoint{1.055369in}{0.696537in}}%
\pgfpathcurveto{\pgfqpoint{1.063182in}{0.704351in}}{\pgfqpoint{1.067573in}{0.714950in}}{\pgfqpoint{1.067573in}{0.726000in}}%
\pgfpathcurveto{\pgfqpoint{1.067573in}{0.737050in}}{\pgfqpoint{1.063182in}{0.747649in}}{\pgfqpoint{1.055369in}{0.755463in}}%
\pgfpathcurveto{\pgfqpoint{1.047555in}{0.763276in}}{\pgfqpoint{1.036956in}{0.767667in}}{\pgfqpoint{1.025906in}{0.767667in}}%
\pgfpathcurveto{\pgfqpoint{1.014856in}{0.767667in}}{\pgfqpoint{1.004257in}{0.763276in}}{\pgfqpoint{0.996443in}{0.755463in}}%
\pgfpathcurveto{\pgfqpoint{0.988630in}{0.747649in}}{\pgfqpoint{0.984239in}{0.737050in}}{\pgfqpoint{0.984239in}{0.726000in}}%
\pgfpathcurveto{\pgfqpoint{0.984239in}{0.714950in}}{\pgfqpoint{0.988630in}{0.704351in}}{\pgfqpoint{0.996443in}{0.696537in}}%
\pgfpathcurveto{\pgfqpoint{1.004257in}{0.688724in}}{\pgfqpoint{1.014856in}{0.684333in}}{\pgfqpoint{1.025906in}{0.684333in}}%
\pgfpathclose%
\pgfusepath{stroke,fill}%
\end{pgfscope}%
\begin{pgfscope}%
\pgfpathrectangle{\pgfqpoint{0.800000in}{0.528000in}}{\pgfqpoint{4.960000in}{3.696000in}}%
\pgfusepath{clip}%
\pgfsetbuttcap%
\pgfsetroundjoin%
\definecolor{currentfill}{rgb}{0.000000,0.000000,0.000000}%
\pgfsetfillcolor{currentfill}%
\pgfsetlinewidth{1.003750pt}%
\definecolor{currentstroke}{rgb}{0.000000,0.000000,0.000000}%
\pgfsetstrokecolor{currentstroke}%
\pgfsetdash{}{0pt}%
\pgfpathmoveto{\pgfqpoint{1.025906in}{0.684333in}}%
\pgfpathcurveto{\pgfqpoint{1.036956in}{0.684333in}}{\pgfqpoint{1.047555in}{0.688724in}}{\pgfqpoint{1.055369in}{0.696537in}}%
\pgfpathcurveto{\pgfqpoint{1.063182in}{0.704351in}}{\pgfqpoint{1.067573in}{0.714950in}}{\pgfqpoint{1.067573in}{0.726000in}}%
\pgfpathcurveto{\pgfqpoint{1.067573in}{0.737050in}}{\pgfqpoint{1.063182in}{0.747649in}}{\pgfqpoint{1.055369in}{0.755463in}}%
\pgfpathcurveto{\pgfqpoint{1.047555in}{0.763276in}}{\pgfqpoint{1.036956in}{0.767667in}}{\pgfqpoint{1.025906in}{0.767667in}}%
\pgfpathcurveto{\pgfqpoint{1.014856in}{0.767667in}}{\pgfqpoint{1.004257in}{0.763276in}}{\pgfqpoint{0.996443in}{0.755463in}}%
\pgfpathcurveto{\pgfqpoint{0.988630in}{0.747649in}}{\pgfqpoint{0.984239in}{0.737050in}}{\pgfqpoint{0.984239in}{0.726000in}}%
\pgfpathcurveto{\pgfqpoint{0.984239in}{0.714950in}}{\pgfqpoint{0.988630in}{0.704351in}}{\pgfqpoint{0.996443in}{0.696537in}}%
\pgfpathcurveto{\pgfqpoint{1.004257in}{0.688724in}}{\pgfqpoint{1.014856in}{0.684333in}}{\pgfqpoint{1.025906in}{0.684333in}}%
\pgfpathclose%
\pgfusepath{stroke,fill}%
\end{pgfscope}%
\begin{pgfscope}%
\pgfpathrectangle{\pgfqpoint{0.800000in}{0.528000in}}{\pgfqpoint{4.960000in}{3.696000in}}%
\pgfusepath{clip}%
\pgfsetbuttcap%
\pgfsetroundjoin%
\definecolor{currentfill}{rgb}{0.000000,0.000000,0.000000}%
\pgfsetfillcolor{currentfill}%
\pgfsetlinewidth{1.003750pt}%
\definecolor{currentstroke}{rgb}{0.000000,0.000000,0.000000}%
\pgfsetstrokecolor{currentstroke}%
\pgfsetdash{}{0pt}%
\pgfpathmoveto{\pgfqpoint{1.025906in}{0.684333in}}%
\pgfpathcurveto{\pgfqpoint{1.036956in}{0.684333in}}{\pgfqpoint{1.047555in}{0.688724in}}{\pgfqpoint{1.055369in}{0.696537in}}%
\pgfpathcurveto{\pgfqpoint{1.063182in}{0.704351in}}{\pgfqpoint{1.067573in}{0.714950in}}{\pgfqpoint{1.067573in}{0.726000in}}%
\pgfpathcurveto{\pgfqpoint{1.067573in}{0.737050in}}{\pgfqpoint{1.063182in}{0.747649in}}{\pgfqpoint{1.055369in}{0.755463in}}%
\pgfpathcurveto{\pgfqpoint{1.047555in}{0.763276in}}{\pgfqpoint{1.036956in}{0.767667in}}{\pgfqpoint{1.025906in}{0.767667in}}%
\pgfpathcurveto{\pgfqpoint{1.014856in}{0.767667in}}{\pgfqpoint{1.004257in}{0.763276in}}{\pgfqpoint{0.996443in}{0.755463in}}%
\pgfpathcurveto{\pgfqpoint{0.988630in}{0.747649in}}{\pgfqpoint{0.984239in}{0.737050in}}{\pgfqpoint{0.984239in}{0.726000in}}%
\pgfpathcurveto{\pgfqpoint{0.984239in}{0.714950in}}{\pgfqpoint{0.988630in}{0.704351in}}{\pgfqpoint{0.996443in}{0.696537in}}%
\pgfpathcurveto{\pgfqpoint{1.004257in}{0.688724in}}{\pgfqpoint{1.014856in}{0.684333in}}{\pgfqpoint{1.025906in}{0.684333in}}%
\pgfpathclose%
\pgfusepath{stroke,fill}%
\end{pgfscope}%
\begin{pgfscope}%
\pgfpathrectangle{\pgfqpoint{0.800000in}{0.528000in}}{\pgfqpoint{4.960000in}{3.696000in}}%
\pgfusepath{clip}%
\pgfsetbuttcap%
\pgfsetroundjoin%
\definecolor{currentfill}{rgb}{0.000000,0.000000,0.000000}%
\pgfsetfillcolor{currentfill}%
\pgfsetlinewidth{1.003750pt}%
\definecolor{currentstroke}{rgb}{0.000000,0.000000,0.000000}%
\pgfsetstrokecolor{currentstroke}%
\pgfsetdash{}{0pt}%
\pgfpathmoveto{\pgfqpoint{1.025906in}{0.684333in}}%
\pgfpathcurveto{\pgfqpoint{1.036956in}{0.684333in}}{\pgfqpoint{1.047555in}{0.688724in}}{\pgfqpoint{1.055369in}{0.696537in}}%
\pgfpathcurveto{\pgfqpoint{1.063182in}{0.704351in}}{\pgfqpoint{1.067573in}{0.714950in}}{\pgfqpoint{1.067573in}{0.726000in}}%
\pgfpathcurveto{\pgfqpoint{1.067573in}{0.737050in}}{\pgfqpoint{1.063182in}{0.747649in}}{\pgfqpoint{1.055369in}{0.755463in}}%
\pgfpathcurveto{\pgfqpoint{1.047555in}{0.763276in}}{\pgfqpoint{1.036956in}{0.767667in}}{\pgfqpoint{1.025906in}{0.767667in}}%
\pgfpathcurveto{\pgfqpoint{1.014856in}{0.767667in}}{\pgfqpoint{1.004257in}{0.763276in}}{\pgfqpoint{0.996443in}{0.755463in}}%
\pgfpathcurveto{\pgfqpoint{0.988630in}{0.747649in}}{\pgfqpoint{0.984239in}{0.737050in}}{\pgfqpoint{0.984239in}{0.726000in}}%
\pgfpathcurveto{\pgfqpoint{0.984239in}{0.714950in}}{\pgfqpoint{0.988630in}{0.704351in}}{\pgfqpoint{0.996443in}{0.696537in}}%
\pgfpathcurveto{\pgfqpoint{1.004257in}{0.688724in}}{\pgfqpoint{1.014856in}{0.684333in}}{\pgfqpoint{1.025906in}{0.684333in}}%
\pgfpathclose%
\pgfusepath{stroke,fill}%
\end{pgfscope}%
\begin{pgfscope}%
\pgfpathrectangle{\pgfqpoint{0.800000in}{0.528000in}}{\pgfqpoint{4.960000in}{3.696000in}}%
\pgfusepath{clip}%
\pgfsetbuttcap%
\pgfsetroundjoin%
\definecolor{currentfill}{rgb}{0.000000,0.000000,0.000000}%
\pgfsetfillcolor{currentfill}%
\pgfsetlinewidth{1.003750pt}%
\definecolor{currentstroke}{rgb}{0.000000,0.000000,0.000000}%
\pgfsetstrokecolor{currentstroke}%
\pgfsetdash{}{0pt}%
\pgfpathmoveto{\pgfqpoint{1.025906in}{0.684333in}}%
\pgfpathcurveto{\pgfqpoint{1.036956in}{0.684333in}}{\pgfqpoint{1.047555in}{0.688724in}}{\pgfqpoint{1.055369in}{0.696537in}}%
\pgfpathcurveto{\pgfqpoint{1.063182in}{0.704351in}}{\pgfqpoint{1.067573in}{0.714950in}}{\pgfqpoint{1.067573in}{0.726000in}}%
\pgfpathcurveto{\pgfqpoint{1.067573in}{0.737050in}}{\pgfqpoint{1.063182in}{0.747649in}}{\pgfqpoint{1.055369in}{0.755463in}}%
\pgfpathcurveto{\pgfqpoint{1.047555in}{0.763276in}}{\pgfqpoint{1.036956in}{0.767667in}}{\pgfqpoint{1.025906in}{0.767667in}}%
\pgfpathcurveto{\pgfqpoint{1.014856in}{0.767667in}}{\pgfqpoint{1.004257in}{0.763276in}}{\pgfqpoint{0.996443in}{0.755463in}}%
\pgfpathcurveto{\pgfqpoint{0.988630in}{0.747649in}}{\pgfqpoint{0.984239in}{0.737050in}}{\pgfqpoint{0.984239in}{0.726000in}}%
\pgfpathcurveto{\pgfqpoint{0.984239in}{0.714950in}}{\pgfqpoint{0.988630in}{0.704351in}}{\pgfqpoint{0.996443in}{0.696537in}}%
\pgfpathcurveto{\pgfqpoint{1.004257in}{0.688724in}}{\pgfqpoint{1.014856in}{0.684333in}}{\pgfqpoint{1.025906in}{0.684333in}}%
\pgfpathclose%
\pgfusepath{stroke,fill}%
\end{pgfscope}%
\begin{pgfscope}%
\pgfpathrectangle{\pgfqpoint{0.800000in}{0.528000in}}{\pgfqpoint{4.960000in}{3.696000in}}%
\pgfusepath{clip}%
\pgfsetbuttcap%
\pgfsetroundjoin%
\definecolor{currentfill}{rgb}{0.000000,0.000000,0.000000}%
\pgfsetfillcolor{currentfill}%
\pgfsetlinewidth{1.003750pt}%
\definecolor{currentstroke}{rgb}{0.000000,0.000000,0.000000}%
\pgfsetstrokecolor{currentstroke}%
\pgfsetdash{}{0pt}%
\pgfpathmoveto{\pgfqpoint{1.025906in}{3.984333in}}%
\pgfpathcurveto{\pgfqpoint{1.036956in}{3.984333in}}{\pgfqpoint{1.047555in}{3.988724in}}{\pgfqpoint{1.055369in}{3.996537in}}%
\pgfpathcurveto{\pgfqpoint{1.063182in}{4.004351in}}{\pgfqpoint{1.067573in}{4.014950in}}{\pgfqpoint{1.067573in}{4.026000in}}%
\pgfpathcurveto{\pgfqpoint{1.067573in}{4.037050in}}{\pgfqpoint{1.063182in}{4.047649in}}{\pgfqpoint{1.055369in}{4.055463in}}%
\pgfpathcurveto{\pgfqpoint{1.047555in}{4.063276in}}{\pgfqpoint{1.036956in}{4.067667in}}{\pgfqpoint{1.025906in}{4.067667in}}%
\pgfpathcurveto{\pgfqpoint{1.014856in}{4.067667in}}{\pgfqpoint{1.004257in}{4.063276in}}{\pgfqpoint{0.996443in}{4.055463in}}%
\pgfpathcurveto{\pgfqpoint{0.988630in}{4.047649in}}{\pgfqpoint{0.984239in}{4.037050in}}{\pgfqpoint{0.984239in}{4.026000in}}%
\pgfpathcurveto{\pgfqpoint{0.984239in}{4.014950in}}{\pgfqpoint{0.988630in}{4.004351in}}{\pgfqpoint{0.996443in}{3.996537in}}%
\pgfpathcurveto{\pgfqpoint{1.004257in}{3.988724in}}{\pgfqpoint{1.014856in}{3.984333in}}{\pgfqpoint{1.025906in}{3.984333in}}%
\pgfpathclose%
\pgfusepath{stroke,fill}%
\end{pgfscope}%
\begin{pgfscope}%
\pgfpathrectangle{\pgfqpoint{0.800000in}{0.528000in}}{\pgfqpoint{4.960000in}{3.696000in}}%
\pgfusepath{clip}%
\pgfsetbuttcap%
\pgfsetroundjoin%
\definecolor{currentfill}{rgb}{0.000000,0.000000,0.000000}%
\pgfsetfillcolor{currentfill}%
\pgfsetlinewidth{1.003750pt}%
\definecolor{currentstroke}{rgb}{0.000000,0.000000,0.000000}%
\pgfsetstrokecolor{currentstroke}%
\pgfsetdash{}{0pt}%
\pgfpathmoveto{\pgfqpoint{1.025906in}{0.684333in}}%
\pgfpathcurveto{\pgfqpoint{1.036956in}{0.684333in}}{\pgfqpoint{1.047555in}{0.688724in}}{\pgfqpoint{1.055369in}{0.696537in}}%
\pgfpathcurveto{\pgfqpoint{1.063182in}{0.704351in}}{\pgfqpoint{1.067573in}{0.714950in}}{\pgfqpoint{1.067573in}{0.726000in}}%
\pgfpathcurveto{\pgfqpoint{1.067573in}{0.737050in}}{\pgfqpoint{1.063182in}{0.747649in}}{\pgfqpoint{1.055369in}{0.755463in}}%
\pgfpathcurveto{\pgfqpoint{1.047555in}{0.763276in}}{\pgfqpoint{1.036956in}{0.767667in}}{\pgfqpoint{1.025906in}{0.767667in}}%
\pgfpathcurveto{\pgfqpoint{1.014856in}{0.767667in}}{\pgfqpoint{1.004257in}{0.763276in}}{\pgfqpoint{0.996443in}{0.755463in}}%
\pgfpathcurveto{\pgfqpoint{0.988630in}{0.747649in}}{\pgfqpoint{0.984239in}{0.737050in}}{\pgfqpoint{0.984239in}{0.726000in}}%
\pgfpathcurveto{\pgfqpoint{0.984239in}{0.714950in}}{\pgfqpoint{0.988630in}{0.704351in}}{\pgfqpoint{0.996443in}{0.696537in}}%
\pgfpathcurveto{\pgfqpoint{1.004257in}{0.688724in}}{\pgfqpoint{1.014856in}{0.684333in}}{\pgfqpoint{1.025906in}{0.684333in}}%
\pgfpathclose%
\pgfusepath{stroke,fill}%
\end{pgfscope}%
\begin{pgfscope}%
\pgfpathrectangle{\pgfqpoint{0.800000in}{0.528000in}}{\pgfqpoint{4.960000in}{3.696000in}}%
\pgfusepath{clip}%
\pgfsetbuttcap%
\pgfsetroundjoin%
\definecolor{currentfill}{rgb}{0.000000,0.000000,0.000000}%
\pgfsetfillcolor{currentfill}%
\pgfsetlinewidth{1.003750pt}%
\definecolor{currentstroke}{rgb}{0.000000,0.000000,0.000000}%
\pgfsetstrokecolor{currentstroke}%
\pgfsetdash{}{0pt}%
\pgfpathmoveto{\pgfqpoint{1.025906in}{3.984333in}}%
\pgfpathcurveto{\pgfqpoint{1.036956in}{3.984333in}}{\pgfqpoint{1.047555in}{3.988724in}}{\pgfqpoint{1.055369in}{3.996537in}}%
\pgfpathcurveto{\pgfqpoint{1.063182in}{4.004351in}}{\pgfqpoint{1.067573in}{4.014950in}}{\pgfqpoint{1.067573in}{4.026000in}}%
\pgfpathcurveto{\pgfqpoint{1.067573in}{4.037050in}}{\pgfqpoint{1.063182in}{4.047649in}}{\pgfqpoint{1.055369in}{4.055463in}}%
\pgfpathcurveto{\pgfqpoint{1.047555in}{4.063276in}}{\pgfqpoint{1.036956in}{4.067667in}}{\pgfqpoint{1.025906in}{4.067667in}}%
\pgfpathcurveto{\pgfqpoint{1.014856in}{4.067667in}}{\pgfqpoint{1.004257in}{4.063276in}}{\pgfqpoint{0.996443in}{4.055463in}}%
\pgfpathcurveto{\pgfqpoint{0.988630in}{4.047649in}}{\pgfqpoint{0.984239in}{4.037050in}}{\pgfqpoint{0.984239in}{4.026000in}}%
\pgfpathcurveto{\pgfqpoint{0.984239in}{4.014950in}}{\pgfqpoint{0.988630in}{4.004351in}}{\pgfqpoint{0.996443in}{3.996537in}}%
\pgfpathcurveto{\pgfqpoint{1.004257in}{3.988724in}}{\pgfqpoint{1.014856in}{3.984333in}}{\pgfqpoint{1.025906in}{3.984333in}}%
\pgfpathclose%
\pgfusepath{stroke,fill}%
\end{pgfscope}%
\begin{pgfscope}%
\pgfpathrectangle{\pgfqpoint{0.800000in}{0.528000in}}{\pgfqpoint{4.960000in}{3.696000in}}%
\pgfusepath{clip}%
\pgfsetbuttcap%
\pgfsetroundjoin%
\definecolor{currentfill}{rgb}{0.000000,0.000000,0.000000}%
\pgfsetfillcolor{currentfill}%
\pgfsetlinewidth{1.003750pt}%
\definecolor{currentstroke}{rgb}{0.000000,0.000000,0.000000}%
\pgfsetstrokecolor{currentstroke}%
\pgfsetdash{}{0pt}%
\pgfpathmoveto{\pgfqpoint{1.025906in}{0.684333in}}%
\pgfpathcurveto{\pgfqpoint{1.036956in}{0.684333in}}{\pgfqpoint{1.047555in}{0.688724in}}{\pgfqpoint{1.055369in}{0.696537in}}%
\pgfpathcurveto{\pgfqpoint{1.063182in}{0.704351in}}{\pgfqpoint{1.067573in}{0.714950in}}{\pgfqpoint{1.067573in}{0.726000in}}%
\pgfpathcurveto{\pgfqpoint{1.067573in}{0.737050in}}{\pgfqpoint{1.063182in}{0.747649in}}{\pgfqpoint{1.055369in}{0.755463in}}%
\pgfpathcurveto{\pgfqpoint{1.047555in}{0.763276in}}{\pgfqpoint{1.036956in}{0.767667in}}{\pgfqpoint{1.025906in}{0.767667in}}%
\pgfpathcurveto{\pgfqpoint{1.014856in}{0.767667in}}{\pgfqpoint{1.004257in}{0.763276in}}{\pgfqpoint{0.996443in}{0.755463in}}%
\pgfpathcurveto{\pgfqpoint{0.988630in}{0.747649in}}{\pgfqpoint{0.984239in}{0.737050in}}{\pgfqpoint{0.984239in}{0.726000in}}%
\pgfpathcurveto{\pgfqpoint{0.984239in}{0.714950in}}{\pgfqpoint{0.988630in}{0.704351in}}{\pgfqpoint{0.996443in}{0.696537in}}%
\pgfpathcurveto{\pgfqpoint{1.004257in}{0.688724in}}{\pgfqpoint{1.014856in}{0.684333in}}{\pgfqpoint{1.025906in}{0.684333in}}%
\pgfpathclose%
\pgfusepath{stroke,fill}%
\end{pgfscope}%
\begin{pgfscope}%
\pgfpathrectangle{\pgfqpoint{0.800000in}{0.528000in}}{\pgfqpoint{4.960000in}{3.696000in}}%
\pgfusepath{clip}%
\pgfsetbuttcap%
\pgfsetroundjoin%
\definecolor{currentfill}{rgb}{0.000000,0.000000,0.000000}%
\pgfsetfillcolor{currentfill}%
\pgfsetlinewidth{1.003750pt}%
\definecolor{currentstroke}{rgb}{0.000000,0.000000,0.000000}%
\pgfsetstrokecolor{currentstroke}%
\pgfsetdash{}{0pt}%
\pgfpathmoveto{\pgfqpoint{1.025906in}{0.684333in}}%
\pgfpathcurveto{\pgfqpoint{1.036956in}{0.684333in}}{\pgfqpoint{1.047555in}{0.688724in}}{\pgfqpoint{1.055369in}{0.696537in}}%
\pgfpathcurveto{\pgfqpoint{1.063182in}{0.704351in}}{\pgfqpoint{1.067573in}{0.714950in}}{\pgfqpoint{1.067573in}{0.726000in}}%
\pgfpathcurveto{\pgfqpoint{1.067573in}{0.737050in}}{\pgfqpoint{1.063182in}{0.747649in}}{\pgfqpoint{1.055369in}{0.755463in}}%
\pgfpathcurveto{\pgfqpoint{1.047555in}{0.763276in}}{\pgfqpoint{1.036956in}{0.767667in}}{\pgfqpoint{1.025906in}{0.767667in}}%
\pgfpathcurveto{\pgfqpoint{1.014856in}{0.767667in}}{\pgfqpoint{1.004257in}{0.763276in}}{\pgfqpoint{0.996443in}{0.755463in}}%
\pgfpathcurveto{\pgfqpoint{0.988630in}{0.747649in}}{\pgfqpoint{0.984239in}{0.737050in}}{\pgfqpoint{0.984239in}{0.726000in}}%
\pgfpathcurveto{\pgfqpoint{0.984239in}{0.714950in}}{\pgfqpoint{0.988630in}{0.704351in}}{\pgfqpoint{0.996443in}{0.696537in}}%
\pgfpathcurveto{\pgfqpoint{1.004257in}{0.688724in}}{\pgfqpoint{1.014856in}{0.684333in}}{\pgfqpoint{1.025906in}{0.684333in}}%
\pgfpathclose%
\pgfusepath{stroke,fill}%
\end{pgfscope}%
\begin{pgfscope}%
\pgfpathrectangle{\pgfqpoint{0.800000in}{0.528000in}}{\pgfqpoint{4.960000in}{3.696000in}}%
\pgfusepath{clip}%
\pgfsetbuttcap%
\pgfsetroundjoin%
\definecolor{currentfill}{rgb}{0.000000,0.000000,0.000000}%
\pgfsetfillcolor{currentfill}%
\pgfsetlinewidth{1.003750pt}%
\definecolor{currentstroke}{rgb}{0.000000,0.000000,0.000000}%
\pgfsetstrokecolor{currentstroke}%
\pgfsetdash{}{0pt}%
\pgfpathmoveto{\pgfqpoint{1.025906in}{0.684333in}}%
\pgfpathcurveto{\pgfqpoint{1.036956in}{0.684333in}}{\pgfqpoint{1.047555in}{0.688724in}}{\pgfqpoint{1.055369in}{0.696537in}}%
\pgfpathcurveto{\pgfqpoint{1.063182in}{0.704351in}}{\pgfqpoint{1.067573in}{0.714950in}}{\pgfqpoint{1.067573in}{0.726000in}}%
\pgfpathcurveto{\pgfqpoint{1.067573in}{0.737050in}}{\pgfqpoint{1.063182in}{0.747649in}}{\pgfqpoint{1.055369in}{0.755463in}}%
\pgfpathcurveto{\pgfqpoint{1.047555in}{0.763276in}}{\pgfqpoint{1.036956in}{0.767667in}}{\pgfqpoint{1.025906in}{0.767667in}}%
\pgfpathcurveto{\pgfqpoint{1.014856in}{0.767667in}}{\pgfqpoint{1.004257in}{0.763276in}}{\pgfqpoint{0.996443in}{0.755463in}}%
\pgfpathcurveto{\pgfqpoint{0.988630in}{0.747649in}}{\pgfqpoint{0.984239in}{0.737050in}}{\pgfqpoint{0.984239in}{0.726000in}}%
\pgfpathcurveto{\pgfqpoint{0.984239in}{0.714950in}}{\pgfqpoint{0.988630in}{0.704351in}}{\pgfqpoint{0.996443in}{0.696537in}}%
\pgfpathcurveto{\pgfqpoint{1.004257in}{0.688724in}}{\pgfqpoint{1.014856in}{0.684333in}}{\pgfqpoint{1.025906in}{0.684333in}}%
\pgfpathclose%
\pgfusepath{stroke,fill}%
\end{pgfscope}%
\begin{pgfscope}%
\pgfpathrectangle{\pgfqpoint{0.800000in}{0.528000in}}{\pgfqpoint{4.960000in}{3.696000in}}%
\pgfusepath{clip}%
\pgfsetbuttcap%
\pgfsetroundjoin%
\definecolor{currentfill}{rgb}{0.000000,0.000000,0.000000}%
\pgfsetfillcolor{currentfill}%
\pgfsetlinewidth{1.003750pt}%
\definecolor{currentstroke}{rgb}{0.000000,0.000000,0.000000}%
\pgfsetstrokecolor{currentstroke}%
\pgfsetdash{}{0pt}%
\pgfpathmoveto{\pgfqpoint{1.025906in}{3.984333in}}%
\pgfpathcurveto{\pgfqpoint{1.036956in}{3.984333in}}{\pgfqpoint{1.047555in}{3.988724in}}{\pgfqpoint{1.055369in}{3.996537in}}%
\pgfpathcurveto{\pgfqpoint{1.063182in}{4.004351in}}{\pgfqpoint{1.067573in}{4.014950in}}{\pgfqpoint{1.067573in}{4.026000in}}%
\pgfpathcurveto{\pgfqpoint{1.067573in}{4.037050in}}{\pgfqpoint{1.063182in}{4.047649in}}{\pgfqpoint{1.055369in}{4.055463in}}%
\pgfpathcurveto{\pgfqpoint{1.047555in}{4.063276in}}{\pgfqpoint{1.036956in}{4.067667in}}{\pgfqpoint{1.025906in}{4.067667in}}%
\pgfpathcurveto{\pgfqpoint{1.014856in}{4.067667in}}{\pgfqpoint{1.004257in}{4.063276in}}{\pgfqpoint{0.996443in}{4.055463in}}%
\pgfpathcurveto{\pgfqpoint{0.988630in}{4.047649in}}{\pgfqpoint{0.984239in}{4.037050in}}{\pgfqpoint{0.984239in}{4.026000in}}%
\pgfpathcurveto{\pgfqpoint{0.984239in}{4.014950in}}{\pgfqpoint{0.988630in}{4.004351in}}{\pgfqpoint{0.996443in}{3.996537in}}%
\pgfpathcurveto{\pgfqpoint{1.004257in}{3.988724in}}{\pgfqpoint{1.014856in}{3.984333in}}{\pgfqpoint{1.025906in}{3.984333in}}%
\pgfpathclose%
\pgfusepath{stroke,fill}%
\end{pgfscope}%
\begin{pgfscope}%
\pgfpathrectangle{\pgfqpoint{0.800000in}{0.528000in}}{\pgfqpoint{4.960000in}{3.696000in}}%
\pgfusepath{clip}%
\pgfsetbuttcap%
\pgfsetroundjoin%
\definecolor{currentfill}{rgb}{0.000000,0.000000,0.000000}%
\pgfsetfillcolor{currentfill}%
\pgfsetlinewidth{1.003750pt}%
\definecolor{currentstroke}{rgb}{0.000000,0.000000,0.000000}%
\pgfsetstrokecolor{currentstroke}%
\pgfsetdash{}{0pt}%
\pgfpathmoveto{\pgfqpoint{1.025906in}{0.684333in}}%
\pgfpathcurveto{\pgfqpoint{1.036956in}{0.684333in}}{\pgfqpoint{1.047555in}{0.688724in}}{\pgfqpoint{1.055369in}{0.696537in}}%
\pgfpathcurveto{\pgfqpoint{1.063182in}{0.704351in}}{\pgfqpoint{1.067573in}{0.714950in}}{\pgfqpoint{1.067573in}{0.726000in}}%
\pgfpathcurveto{\pgfqpoint{1.067573in}{0.737050in}}{\pgfqpoint{1.063182in}{0.747649in}}{\pgfqpoint{1.055369in}{0.755463in}}%
\pgfpathcurveto{\pgfqpoint{1.047555in}{0.763276in}}{\pgfqpoint{1.036956in}{0.767667in}}{\pgfqpoint{1.025906in}{0.767667in}}%
\pgfpathcurveto{\pgfqpoint{1.014856in}{0.767667in}}{\pgfqpoint{1.004257in}{0.763276in}}{\pgfqpoint{0.996443in}{0.755463in}}%
\pgfpathcurveto{\pgfqpoint{0.988630in}{0.747649in}}{\pgfqpoint{0.984239in}{0.737050in}}{\pgfqpoint{0.984239in}{0.726000in}}%
\pgfpathcurveto{\pgfqpoint{0.984239in}{0.714950in}}{\pgfqpoint{0.988630in}{0.704351in}}{\pgfqpoint{0.996443in}{0.696537in}}%
\pgfpathcurveto{\pgfqpoint{1.004257in}{0.688724in}}{\pgfqpoint{1.014856in}{0.684333in}}{\pgfqpoint{1.025906in}{0.684333in}}%
\pgfpathclose%
\pgfusepath{stroke,fill}%
\end{pgfscope}%
\begin{pgfscope}%
\pgfpathrectangle{\pgfqpoint{0.800000in}{0.528000in}}{\pgfqpoint{4.960000in}{3.696000in}}%
\pgfusepath{clip}%
\pgfsetbuttcap%
\pgfsetroundjoin%
\definecolor{currentfill}{rgb}{0.000000,0.000000,0.000000}%
\pgfsetfillcolor{currentfill}%
\pgfsetlinewidth{1.003750pt}%
\definecolor{currentstroke}{rgb}{0.000000,0.000000,0.000000}%
\pgfsetstrokecolor{currentstroke}%
\pgfsetdash{}{0pt}%
\pgfpathmoveto{\pgfqpoint{1.025906in}{0.684333in}}%
\pgfpathcurveto{\pgfqpoint{1.036956in}{0.684333in}}{\pgfqpoint{1.047555in}{0.688724in}}{\pgfqpoint{1.055369in}{0.696537in}}%
\pgfpathcurveto{\pgfqpoint{1.063182in}{0.704351in}}{\pgfqpoint{1.067573in}{0.714950in}}{\pgfqpoint{1.067573in}{0.726000in}}%
\pgfpathcurveto{\pgfqpoint{1.067573in}{0.737050in}}{\pgfqpoint{1.063182in}{0.747649in}}{\pgfqpoint{1.055369in}{0.755463in}}%
\pgfpathcurveto{\pgfqpoint{1.047555in}{0.763276in}}{\pgfqpoint{1.036956in}{0.767667in}}{\pgfqpoint{1.025906in}{0.767667in}}%
\pgfpathcurveto{\pgfqpoint{1.014856in}{0.767667in}}{\pgfqpoint{1.004257in}{0.763276in}}{\pgfqpoint{0.996443in}{0.755463in}}%
\pgfpathcurveto{\pgfqpoint{0.988630in}{0.747649in}}{\pgfqpoint{0.984239in}{0.737050in}}{\pgfqpoint{0.984239in}{0.726000in}}%
\pgfpathcurveto{\pgfqpoint{0.984239in}{0.714950in}}{\pgfqpoint{0.988630in}{0.704351in}}{\pgfqpoint{0.996443in}{0.696537in}}%
\pgfpathcurveto{\pgfqpoint{1.004257in}{0.688724in}}{\pgfqpoint{1.014856in}{0.684333in}}{\pgfqpoint{1.025906in}{0.684333in}}%
\pgfpathclose%
\pgfusepath{stroke,fill}%
\end{pgfscope}%
\begin{pgfscope}%
\pgfpathrectangle{\pgfqpoint{0.800000in}{0.528000in}}{\pgfqpoint{4.960000in}{3.696000in}}%
\pgfusepath{clip}%
\pgfsetbuttcap%
\pgfsetroundjoin%
\definecolor{currentfill}{rgb}{0.000000,0.000000,0.000000}%
\pgfsetfillcolor{currentfill}%
\pgfsetlinewidth{1.003750pt}%
\definecolor{currentstroke}{rgb}{0.000000,0.000000,0.000000}%
\pgfsetstrokecolor{currentstroke}%
\pgfsetdash{}{0pt}%
\pgfpathmoveto{\pgfqpoint{1.025906in}{0.684333in}}%
\pgfpathcurveto{\pgfqpoint{1.036956in}{0.684333in}}{\pgfqpoint{1.047555in}{0.688724in}}{\pgfqpoint{1.055369in}{0.696537in}}%
\pgfpathcurveto{\pgfqpoint{1.063182in}{0.704351in}}{\pgfqpoint{1.067573in}{0.714950in}}{\pgfqpoint{1.067573in}{0.726000in}}%
\pgfpathcurveto{\pgfqpoint{1.067573in}{0.737050in}}{\pgfqpoint{1.063182in}{0.747649in}}{\pgfqpoint{1.055369in}{0.755463in}}%
\pgfpathcurveto{\pgfqpoint{1.047555in}{0.763276in}}{\pgfqpoint{1.036956in}{0.767667in}}{\pgfqpoint{1.025906in}{0.767667in}}%
\pgfpathcurveto{\pgfqpoint{1.014856in}{0.767667in}}{\pgfqpoint{1.004257in}{0.763276in}}{\pgfqpoint{0.996443in}{0.755463in}}%
\pgfpathcurveto{\pgfqpoint{0.988630in}{0.747649in}}{\pgfqpoint{0.984239in}{0.737050in}}{\pgfqpoint{0.984239in}{0.726000in}}%
\pgfpathcurveto{\pgfqpoint{0.984239in}{0.714950in}}{\pgfqpoint{0.988630in}{0.704351in}}{\pgfqpoint{0.996443in}{0.696537in}}%
\pgfpathcurveto{\pgfqpoint{1.004257in}{0.688724in}}{\pgfqpoint{1.014856in}{0.684333in}}{\pgfqpoint{1.025906in}{0.684333in}}%
\pgfpathclose%
\pgfusepath{stroke,fill}%
\end{pgfscope}%
\begin{pgfscope}%
\pgfpathrectangle{\pgfqpoint{0.800000in}{0.528000in}}{\pgfqpoint{4.960000in}{3.696000in}}%
\pgfusepath{clip}%
\pgfsetbuttcap%
\pgfsetroundjoin%
\definecolor{currentfill}{rgb}{0.000000,0.000000,0.000000}%
\pgfsetfillcolor{currentfill}%
\pgfsetlinewidth{1.003750pt}%
\definecolor{currentstroke}{rgb}{0.000000,0.000000,0.000000}%
\pgfsetstrokecolor{currentstroke}%
\pgfsetdash{}{0pt}%
\pgfpathmoveto{\pgfqpoint{1.025906in}{0.684333in}}%
\pgfpathcurveto{\pgfqpoint{1.036956in}{0.684333in}}{\pgfqpoint{1.047555in}{0.688724in}}{\pgfqpoint{1.055369in}{0.696537in}}%
\pgfpathcurveto{\pgfqpoint{1.063182in}{0.704351in}}{\pgfqpoint{1.067573in}{0.714950in}}{\pgfqpoint{1.067573in}{0.726000in}}%
\pgfpathcurveto{\pgfqpoint{1.067573in}{0.737050in}}{\pgfqpoint{1.063182in}{0.747649in}}{\pgfqpoint{1.055369in}{0.755463in}}%
\pgfpathcurveto{\pgfqpoint{1.047555in}{0.763276in}}{\pgfqpoint{1.036956in}{0.767667in}}{\pgfqpoint{1.025906in}{0.767667in}}%
\pgfpathcurveto{\pgfqpoint{1.014856in}{0.767667in}}{\pgfqpoint{1.004257in}{0.763276in}}{\pgfqpoint{0.996443in}{0.755463in}}%
\pgfpathcurveto{\pgfqpoint{0.988630in}{0.747649in}}{\pgfqpoint{0.984239in}{0.737050in}}{\pgfqpoint{0.984239in}{0.726000in}}%
\pgfpathcurveto{\pgfqpoint{0.984239in}{0.714950in}}{\pgfqpoint{0.988630in}{0.704351in}}{\pgfqpoint{0.996443in}{0.696537in}}%
\pgfpathcurveto{\pgfqpoint{1.004257in}{0.688724in}}{\pgfqpoint{1.014856in}{0.684333in}}{\pgfqpoint{1.025906in}{0.684333in}}%
\pgfpathclose%
\pgfusepath{stroke,fill}%
\end{pgfscope}%
\begin{pgfscope}%
\pgfpathrectangle{\pgfqpoint{0.800000in}{0.528000in}}{\pgfqpoint{4.960000in}{3.696000in}}%
\pgfusepath{clip}%
\pgfsetbuttcap%
\pgfsetroundjoin%
\definecolor{currentfill}{rgb}{0.000000,0.000000,0.000000}%
\pgfsetfillcolor{currentfill}%
\pgfsetlinewidth{1.003750pt}%
\definecolor{currentstroke}{rgb}{0.000000,0.000000,0.000000}%
\pgfsetstrokecolor{currentstroke}%
\pgfsetdash{}{0pt}%
\pgfpathmoveto{\pgfqpoint{1.025906in}{0.684333in}}%
\pgfpathcurveto{\pgfqpoint{1.036956in}{0.684333in}}{\pgfqpoint{1.047555in}{0.688724in}}{\pgfqpoint{1.055369in}{0.696537in}}%
\pgfpathcurveto{\pgfqpoint{1.063182in}{0.704351in}}{\pgfqpoint{1.067573in}{0.714950in}}{\pgfqpoint{1.067573in}{0.726000in}}%
\pgfpathcurveto{\pgfqpoint{1.067573in}{0.737050in}}{\pgfqpoint{1.063182in}{0.747649in}}{\pgfqpoint{1.055369in}{0.755463in}}%
\pgfpathcurveto{\pgfqpoint{1.047555in}{0.763276in}}{\pgfqpoint{1.036956in}{0.767667in}}{\pgfqpoint{1.025906in}{0.767667in}}%
\pgfpathcurveto{\pgfqpoint{1.014856in}{0.767667in}}{\pgfqpoint{1.004257in}{0.763276in}}{\pgfqpoint{0.996443in}{0.755463in}}%
\pgfpathcurveto{\pgfqpoint{0.988630in}{0.747649in}}{\pgfqpoint{0.984239in}{0.737050in}}{\pgfqpoint{0.984239in}{0.726000in}}%
\pgfpathcurveto{\pgfqpoint{0.984239in}{0.714950in}}{\pgfqpoint{0.988630in}{0.704351in}}{\pgfqpoint{0.996443in}{0.696537in}}%
\pgfpathcurveto{\pgfqpoint{1.004257in}{0.688724in}}{\pgfqpoint{1.014856in}{0.684333in}}{\pgfqpoint{1.025906in}{0.684333in}}%
\pgfpathclose%
\pgfusepath{stroke,fill}%
\end{pgfscope}%
\begin{pgfscope}%
\pgfpathrectangle{\pgfqpoint{0.800000in}{0.528000in}}{\pgfqpoint{4.960000in}{3.696000in}}%
\pgfusepath{clip}%
\pgfsetbuttcap%
\pgfsetroundjoin%
\definecolor{currentfill}{rgb}{0.000000,0.000000,0.000000}%
\pgfsetfillcolor{currentfill}%
\pgfsetlinewidth{1.003750pt}%
\definecolor{currentstroke}{rgb}{0.000000,0.000000,0.000000}%
\pgfsetstrokecolor{currentstroke}%
\pgfsetdash{}{0pt}%
\pgfpathmoveto{\pgfqpoint{1.025906in}{0.684333in}}%
\pgfpathcurveto{\pgfqpoint{1.036956in}{0.684333in}}{\pgfqpoint{1.047555in}{0.688724in}}{\pgfqpoint{1.055369in}{0.696537in}}%
\pgfpathcurveto{\pgfqpoint{1.063182in}{0.704351in}}{\pgfqpoint{1.067573in}{0.714950in}}{\pgfqpoint{1.067573in}{0.726000in}}%
\pgfpathcurveto{\pgfqpoint{1.067573in}{0.737050in}}{\pgfqpoint{1.063182in}{0.747649in}}{\pgfqpoint{1.055369in}{0.755463in}}%
\pgfpathcurveto{\pgfqpoint{1.047555in}{0.763276in}}{\pgfqpoint{1.036956in}{0.767667in}}{\pgfqpoint{1.025906in}{0.767667in}}%
\pgfpathcurveto{\pgfqpoint{1.014856in}{0.767667in}}{\pgfqpoint{1.004257in}{0.763276in}}{\pgfqpoint{0.996443in}{0.755463in}}%
\pgfpathcurveto{\pgfqpoint{0.988630in}{0.747649in}}{\pgfqpoint{0.984239in}{0.737050in}}{\pgfqpoint{0.984239in}{0.726000in}}%
\pgfpathcurveto{\pgfqpoint{0.984239in}{0.714950in}}{\pgfqpoint{0.988630in}{0.704351in}}{\pgfqpoint{0.996443in}{0.696537in}}%
\pgfpathcurveto{\pgfqpoint{1.004257in}{0.688724in}}{\pgfqpoint{1.014856in}{0.684333in}}{\pgfqpoint{1.025906in}{0.684333in}}%
\pgfpathclose%
\pgfusepath{stroke,fill}%
\end{pgfscope}%
\begin{pgfscope}%
\pgfpathrectangle{\pgfqpoint{0.800000in}{0.528000in}}{\pgfqpoint{4.960000in}{3.696000in}}%
\pgfusepath{clip}%
\pgfsetbuttcap%
\pgfsetroundjoin%
\definecolor{currentfill}{rgb}{0.000000,0.000000,0.000000}%
\pgfsetfillcolor{currentfill}%
\pgfsetlinewidth{1.003750pt}%
\definecolor{currentstroke}{rgb}{0.000000,0.000000,0.000000}%
\pgfsetstrokecolor{currentstroke}%
\pgfsetdash{}{0pt}%
\pgfpathmoveto{\pgfqpoint{1.025906in}{0.684333in}}%
\pgfpathcurveto{\pgfqpoint{1.036956in}{0.684333in}}{\pgfqpoint{1.047555in}{0.688724in}}{\pgfqpoint{1.055369in}{0.696537in}}%
\pgfpathcurveto{\pgfqpoint{1.063182in}{0.704351in}}{\pgfqpoint{1.067573in}{0.714950in}}{\pgfqpoint{1.067573in}{0.726000in}}%
\pgfpathcurveto{\pgfqpoint{1.067573in}{0.737050in}}{\pgfqpoint{1.063182in}{0.747649in}}{\pgfqpoint{1.055369in}{0.755463in}}%
\pgfpathcurveto{\pgfqpoint{1.047555in}{0.763276in}}{\pgfqpoint{1.036956in}{0.767667in}}{\pgfqpoint{1.025906in}{0.767667in}}%
\pgfpathcurveto{\pgfqpoint{1.014856in}{0.767667in}}{\pgfqpoint{1.004257in}{0.763276in}}{\pgfqpoint{0.996443in}{0.755463in}}%
\pgfpathcurveto{\pgfqpoint{0.988630in}{0.747649in}}{\pgfqpoint{0.984239in}{0.737050in}}{\pgfqpoint{0.984239in}{0.726000in}}%
\pgfpathcurveto{\pgfqpoint{0.984239in}{0.714950in}}{\pgfqpoint{0.988630in}{0.704351in}}{\pgfqpoint{0.996443in}{0.696537in}}%
\pgfpathcurveto{\pgfqpoint{1.004257in}{0.688724in}}{\pgfqpoint{1.014856in}{0.684333in}}{\pgfqpoint{1.025906in}{0.684333in}}%
\pgfpathclose%
\pgfusepath{stroke,fill}%
\end{pgfscope}%
\begin{pgfscope}%
\pgfpathrectangle{\pgfqpoint{0.800000in}{0.528000in}}{\pgfqpoint{4.960000in}{3.696000in}}%
\pgfusepath{clip}%
\pgfsetbuttcap%
\pgfsetroundjoin%
\definecolor{currentfill}{rgb}{0.000000,0.000000,0.000000}%
\pgfsetfillcolor{currentfill}%
\pgfsetlinewidth{1.003750pt}%
\definecolor{currentstroke}{rgb}{0.000000,0.000000,0.000000}%
\pgfsetstrokecolor{currentstroke}%
\pgfsetdash{}{0pt}%
\pgfpathmoveto{\pgfqpoint{1.025906in}{0.684333in}}%
\pgfpathcurveto{\pgfqpoint{1.036956in}{0.684333in}}{\pgfqpoint{1.047555in}{0.688724in}}{\pgfqpoint{1.055369in}{0.696537in}}%
\pgfpathcurveto{\pgfqpoint{1.063182in}{0.704351in}}{\pgfqpoint{1.067573in}{0.714950in}}{\pgfqpoint{1.067573in}{0.726000in}}%
\pgfpathcurveto{\pgfqpoint{1.067573in}{0.737050in}}{\pgfqpoint{1.063182in}{0.747649in}}{\pgfqpoint{1.055369in}{0.755463in}}%
\pgfpathcurveto{\pgfqpoint{1.047555in}{0.763276in}}{\pgfqpoint{1.036956in}{0.767667in}}{\pgfqpoint{1.025906in}{0.767667in}}%
\pgfpathcurveto{\pgfqpoint{1.014856in}{0.767667in}}{\pgfqpoint{1.004257in}{0.763276in}}{\pgfqpoint{0.996443in}{0.755463in}}%
\pgfpathcurveto{\pgfqpoint{0.988630in}{0.747649in}}{\pgfqpoint{0.984239in}{0.737050in}}{\pgfqpoint{0.984239in}{0.726000in}}%
\pgfpathcurveto{\pgfqpoint{0.984239in}{0.714950in}}{\pgfqpoint{0.988630in}{0.704351in}}{\pgfqpoint{0.996443in}{0.696537in}}%
\pgfpathcurveto{\pgfqpoint{1.004257in}{0.688724in}}{\pgfqpoint{1.014856in}{0.684333in}}{\pgfqpoint{1.025906in}{0.684333in}}%
\pgfpathclose%
\pgfusepath{stroke,fill}%
\end{pgfscope}%
\begin{pgfscope}%
\pgfpathrectangle{\pgfqpoint{0.800000in}{0.528000in}}{\pgfqpoint{4.960000in}{3.696000in}}%
\pgfusepath{clip}%
\pgfsetbuttcap%
\pgfsetroundjoin%
\definecolor{currentfill}{rgb}{0.000000,0.000000,0.000000}%
\pgfsetfillcolor{currentfill}%
\pgfsetlinewidth{1.003750pt}%
\definecolor{currentstroke}{rgb}{0.000000,0.000000,0.000000}%
\pgfsetstrokecolor{currentstroke}%
\pgfsetdash{}{0pt}%
\pgfpathmoveto{\pgfqpoint{1.025906in}{0.684333in}}%
\pgfpathcurveto{\pgfqpoint{1.036956in}{0.684333in}}{\pgfqpoint{1.047555in}{0.688724in}}{\pgfqpoint{1.055369in}{0.696537in}}%
\pgfpathcurveto{\pgfqpoint{1.063182in}{0.704351in}}{\pgfqpoint{1.067573in}{0.714950in}}{\pgfqpoint{1.067573in}{0.726000in}}%
\pgfpathcurveto{\pgfqpoint{1.067573in}{0.737050in}}{\pgfqpoint{1.063182in}{0.747649in}}{\pgfqpoint{1.055369in}{0.755463in}}%
\pgfpathcurveto{\pgfqpoint{1.047555in}{0.763276in}}{\pgfqpoint{1.036956in}{0.767667in}}{\pgfqpoint{1.025906in}{0.767667in}}%
\pgfpathcurveto{\pgfqpoint{1.014856in}{0.767667in}}{\pgfqpoint{1.004257in}{0.763276in}}{\pgfqpoint{0.996443in}{0.755463in}}%
\pgfpathcurveto{\pgfqpoint{0.988630in}{0.747649in}}{\pgfqpoint{0.984239in}{0.737050in}}{\pgfqpoint{0.984239in}{0.726000in}}%
\pgfpathcurveto{\pgfqpoint{0.984239in}{0.714950in}}{\pgfqpoint{0.988630in}{0.704351in}}{\pgfqpoint{0.996443in}{0.696537in}}%
\pgfpathcurveto{\pgfqpoint{1.004257in}{0.688724in}}{\pgfqpoint{1.014856in}{0.684333in}}{\pgfqpoint{1.025906in}{0.684333in}}%
\pgfpathclose%
\pgfusepath{stroke,fill}%
\end{pgfscope}%
\begin{pgfscope}%
\pgfpathrectangle{\pgfqpoint{0.800000in}{0.528000in}}{\pgfqpoint{4.960000in}{3.696000in}}%
\pgfusepath{clip}%
\pgfsetbuttcap%
\pgfsetroundjoin%
\definecolor{currentfill}{rgb}{0.000000,0.000000,0.000000}%
\pgfsetfillcolor{currentfill}%
\pgfsetlinewidth{1.003750pt}%
\definecolor{currentstroke}{rgb}{0.000000,0.000000,0.000000}%
\pgfsetstrokecolor{currentstroke}%
\pgfsetdash{}{0pt}%
\pgfpathmoveto{\pgfqpoint{1.025906in}{0.684333in}}%
\pgfpathcurveto{\pgfqpoint{1.036956in}{0.684333in}}{\pgfqpoint{1.047555in}{0.688724in}}{\pgfqpoint{1.055369in}{0.696537in}}%
\pgfpathcurveto{\pgfqpoint{1.063182in}{0.704351in}}{\pgfqpoint{1.067573in}{0.714950in}}{\pgfqpoint{1.067573in}{0.726000in}}%
\pgfpathcurveto{\pgfqpoint{1.067573in}{0.737050in}}{\pgfqpoint{1.063182in}{0.747649in}}{\pgfqpoint{1.055369in}{0.755463in}}%
\pgfpathcurveto{\pgfqpoint{1.047555in}{0.763276in}}{\pgfqpoint{1.036956in}{0.767667in}}{\pgfqpoint{1.025906in}{0.767667in}}%
\pgfpathcurveto{\pgfqpoint{1.014856in}{0.767667in}}{\pgfqpoint{1.004257in}{0.763276in}}{\pgfqpoint{0.996443in}{0.755463in}}%
\pgfpathcurveto{\pgfqpoint{0.988630in}{0.747649in}}{\pgfqpoint{0.984239in}{0.737050in}}{\pgfqpoint{0.984239in}{0.726000in}}%
\pgfpathcurveto{\pgfqpoint{0.984239in}{0.714950in}}{\pgfqpoint{0.988630in}{0.704351in}}{\pgfqpoint{0.996443in}{0.696537in}}%
\pgfpathcurveto{\pgfqpoint{1.004257in}{0.688724in}}{\pgfqpoint{1.014856in}{0.684333in}}{\pgfqpoint{1.025906in}{0.684333in}}%
\pgfpathclose%
\pgfusepath{stroke,fill}%
\end{pgfscope}%
\begin{pgfscope}%
\pgfpathrectangle{\pgfqpoint{0.800000in}{0.528000in}}{\pgfqpoint{4.960000in}{3.696000in}}%
\pgfusepath{clip}%
\pgfsetbuttcap%
\pgfsetroundjoin%
\definecolor{currentfill}{rgb}{0.000000,0.000000,0.000000}%
\pgfsetfillcolor{currentfill}%
\pgfsetlinewidth{1.003750pt}%
\definecolor{currentstroke}{rgb}{0.000000,0.000000,0.000000}%
\pgfsetstrokecolor{currentstroke}%
\pgfsetdash{}{0pt}%
\pgfpathmoveto{\pgfqpoint{1.025906in}{0.684333in}}%
\pgfpathcurveto{\pgfqpoint{1.036956in}{0.684333in}}{\pgfqpoint{1.047555in}{0.688724in}}{\pgfqpoint{1.055369in}{0.696537in}}%
\pgfpathcurveto{\pgfqpoint{1.063182in}{0.704351in}}{\pgfqpoint{1.067573in}{0.714950in}}{\pgfqpoint{1.067573in}{0.726000in}}%
\pgfpathcurveto{\pgfqpoint{1.067573in}{0.737050in}}{\pgfqpoint{1.063182in}{0.747649in}}{\pgfqpoint{1.055369in}{0.755463in}}%
\pgfpathcurveto{\pgfqpoint{1.047555in}{0.763276in}}{\pgfqpoint{1.036956in}{0.767667in}}{\pgfqpoint{1.025906in}{0.767667in}}%
\pgfpathcurveto{\pgfqpoint{1.014856in}{0.767667in}}{\pgfqpoint{1.004257in}{0.763276in}}{\pgfqpoint{0.996443in}{0.755463in}}%
\pgfpathcurveto{\pgfqpoint{0.988630in}{0.747649in}}{\pgfqpoint{0.984239in}{0.737050in}}{\pgfqpoint{0.984239in}{0.726000in}}%
\pgfpathcurveto{\pgfqpoint{0.984239in}{0.714950in}}{\pgfqpoint{0.988630in}{0.704351in}}{\pgfqpoint{0.996443in}{0.696537in}}%
\pgfpathcurveto{\pgfqpoint{1.004257in}{0.688724in}}{\pgfqpoint{1.014856in}{0.684333in}}{\pgfqpoint{1.025906in}{0.684333in}}%
\pgfpathclose%
\pgfusepath{stroke,fill}%
\end{pgfscope}%
\begin{pgfscope}%
\pgfpathrectangle{\pgfqpoint{0.800000in}{0.528000in}}{\pgfqpoint{4.960000in}{3.696000in}}%
\pgfusepath{clip}%
\pgfsetbuttcap%
\pgfsetroundjoin%
\definecolor{currentfill}{rgb}{0.000000,0.000000,0.000000}%
\pgfsetfillcolor{currentfill}%
\pgfsetlinewidth{1.003750pt}%
\definecolor{currentstroke}{rgb}{0.000000,0.000000,0.000000}%
\pgfsetstrokecolor{currentstroke}%
\pgfsetdash{}{0pt}%
\pgfpathmoveto{\pgfqpoint{1.025906in}{0.684333in}}%
\pgfpathcurveto{\pgfqpoint{1.036956in}{0.684333in}}{\pgfqpoint{1.047555in}{0.688724in}}{\pgfqpoint{1.055369in}{0.696537in}}%
\pgfpathcurveto{\pgfqpoint{1.063182in}{0.704351in}}{\pgfqpoint{1.067573in}{0.714950in}}{\pgfqpoint{1.067573in}{0.726000in}}%
\pgfpathcurveto{\pgfqpoint{1.067573in}{0.737050in}}{\pgfqpoint{1.063182in}{0.747649in}}{\pgfqpoint{1.055369in}{0.755463in}}%
\pgfpathcurveto{\pgfqpoint{1.047555in}{0.763276in}}{\pgfqpoint{1.036956in}{0.767667in}}{\pgfqpoint{1.025906in}{0.767667in}}%
\pgfpathcurveto{\pgfqpoint{1.014856in}{0.767667in}}{\pgfqpoint{1.004257in}{0.763276in}}{\pgfqpoint{0.996443in}{0.755463in}}%
\pgfpathcurveto{\pgfqpoint{0.988630in}{0.747649in}}{\pgfqpoint{0.984239in}{0.737050in}}{\pgfqpoint{0.984239in}{0.726000in}}%
\pgfpathcurveto{\pgfqpoint{0.984239in}{0.714950in}}{\pgfqpoint{0.988630in}{0.704351in}}{\pgfqpoint{0.996443in}{0.696537in}}%
\pgfpathcurveto{\pgfqpoint{1.004257in}{0.688724in}}{\pgfqpoint{1.014856in}{0.684333in}}{\pgfqpoint{1.025906in}{0.684333in}}%
\pgfpathclose%
\pgfusepath{stroke,fill}%
\end{pgfscope}%
\begin{pgfscope}%
\pgfpathrectangle{\pgfqpoint{0.800000in}{0.528000in}}{\pgfqpoint{4.960000in}{3.696000in}}%
\pgfusepath{clip}%
\pgfsetbuttcap%
\pgfsetroundjoin%
\definecolor{currentfill}{rgb}{0.000000,0.000000,0.000000}%
\pgfsetfillcolor{currentfill}%
\pgfsetlinewidth{1.003750pt}%
\definecolor{currentstroke}{rgb}{0.000000,0.000000,0.000000}%
\pgfsetstrokecolor{currentstroke}%
\pgfsetdash{}{0pt}%
\pgfpathmoveto{\pgfqpoint{1.025906in}{0.684333in}}%
\pgfpathcurveto{\pgfqpoint{1.036956in}{0.684333in}}{\pgfqpoint{1.047555in}{0.688724in}}{\pgfqpoint{1.055369in}{0.696537in}}%
\pgfpathcurveto{\pgfqpoint{1.063182in}{0.704351in}}{\pgfqpoint{1.067573in}{0.714950in}}{\pgfqpoint{1.067573in}{0.726000in}}%
\pgfpathcurveto{\pgfqpoint{1.067573in}{0.737050in}}{\pgfqpoint{1.063182in}{0.747649in}}{\pgfqpoint{1.055369in}{0.755463in}}%
\pgfpathcurveto{\pgfqpoint{1.047555in}{0.763276in}}{\pgfqpoint{1.036956in}{0.767667in}}{\pgfqpoint{1.025906in}{0.767667in}}%
\pgfpathcurveto{\pgfqpoint{1.014856in}{0.767667in}}{\pgfqpoint{1.004257in}{0.763276in}}{\pgfqpoint{0.996443in}{0.755463in}}%
\pgfpathcurveto{\pgfqpoint{0.988630in}{0.747649in}}{\pgfqpoint{0.984239in}{0.737050in}}{\pgfqpoint{0.984239in}{0.726000in}}%
\pgfpathcurveto{\pgfqpoint{0.984239in}{0.714950in}}{\pgfqpoint{0.988630in}{0.704351in}}{\pgfqpoint{0.996443in}{0.696537in}}%
\pgfpathcurveto{\pgfqpoint{1.004257in}{0.688724in}}{\pgfqpoint{1.014856in}{0.684333in}}{\pgfqpoint{1.025906in}{0.684333in}}%
\pgfpathclose%
\pgfusepath{stroke,fill}%
\end{pgfscope}%
\begin{pgfscope}%
\pgfpathrectangle{\pgfqpoint{0.800000in}{0.528000in}}{\pgfqpoint{4.960000in}{3.696000in}}%
\pgfusepath{clip}%
\pgfsetbuttcap%
\pgfsetroundjoin%
\definecolor{currentfill}{rgb}{0.000000,0.000000,0.000000}%
\pgfsetfillcolor{currentfill}%
\pgfsetlinewidth{1.003750pt}%
\definecolor{currentstroke}{rgb}{0.000000,0.000000,0.000000}%
\pgfsetstrokecolor{currentstroke}%
\pgfsetdash{}{0pt}%
\pgfpathmoveto{\pgfqpoint{1.025906in}{0.684333in}}%
\pgfpathcurveto{\pgfqpoint{1.036956in}{0.684333in}}{\pgfqpoint{1.047555in}{0.688724in}}{\pgfqpoint{1.055369in}{0.696537in}}%
\pgfpathcurveto{\pgfqpoint{1.063182in}{0.704351in}}{\pgfqpoint{1.067573in}{0.714950in}}{\pgfqpoint{1.067573in}{0.726000in}}%
\pgfpathcurveto{\pgfqpoint{1.067573in}{0.737050in}}{\pgfqpoint{1.063182in}{0.747649in}}{\pgfqpoint{1.055369in}{0.755463in}}%
\pgfpathcurveto{\pgfqpoint{1.047555in}{0.763276in}}{\pgfqpoint{1.036956in}{0.767667in}}{\pgfqpoint{1.025906in}{0.767667in}}%
\pgfpathcurveto{\pgfqpoint{1.014856in}{0.767667in}}{\pgfqpoint{1.004257in}{0.763276in}}{\pgfqpoint{0.996443in}{0.755463in}}%
\pgfpathcurveto{\pgfqpoint{0.988630in}{0.747649in}}{\pgfqpoint{0.984239in}{0.737050in}}{\pgfqpoint{0.984239in}{0.726000in}}%
\pgfpathcurveto{\pgfqpoint{0.984239in}{0.714950in}}{\pgfqpoint{0.988630in}{0.704351in}}{\pgfqpoint{0.996443in}{0.696537in}}%
\pgfpathcurveto{\pgfqpoint{1.004257in}{0.688724in}}{\pgfqpoint{1.014856in}{0.684333in}}{\pgfqpoint{1.025906in}{0.684333in}}%
\pgfpathclose%
\pgfusepath{stroke,fill}%
\end{pgfscope}%
\begin{pgfscope}%
\pgfpathrectangle{\pgfqpoint{0.800000in}{0.528000in}}{\pgfqpoint{4.960000in}{3.696000in}}%
\pgfusepath{clip}%
\pgfsetbuttcap%
\pgfsetroundjoin%
\definecolor{currentfill}{rgb}{0.000000,0.000000,0.000000}%
\pgfsetfillcolor{currentfill}%
\pgfsetlinewidth{1.003750pt}%
\definecolor{currentstroke}{rgb}{0.000000,0.000000,0.000000}%
\pgfsetstrokecolor{currentstroke}%
\pgfsetdash{}{0pt}%
\pgfpathmoveto{\pgfqpoint{1.025906in}{0.684333in}}%
\pgfpathcurveto{\pgfqpoint{1.036956in}{0.684333in}}{\pgfqpoint{1.047555in}{0.688724in}}{\pgfqpoint{1.055369in}{0.696537in}}%
\pgfpathcurveto{\pgfqpoint{1.063182in}{0.704351in}}{\pgfqpoint{1.067573in}{0.714950in}}{\pgfqpoint{1.067573in}{0.726000in}}%
\pgfpathcurveto{\pgfqpoint{1.067573in}{0.737050in}}{\pgfqpoint{1.063182in}{0.747649in}}{\pgfqpoint{1.055369in}{0.755463in}}%
\pgfpathcurveto{\pgfqpoint{1.047555in}{0.763276in}}{\pgfqpoint{1.036956in}{0.767667in}}{\pgfqpoint{1.025906in}{0.767667in}}%
\pgfpathcurveto{\pgfqpoint{1.014856in}{0.767667in}}{\pgfqpoint{1.004257in}{0.763276in}}{\pgfqpoint{0.996443in}{0.755463in}}%
\pgfpathcurveto{\pgfqpoint{0.988630in}{0.747649in}}{\pgfqpoint{0.984239in}{0.737050in}}{\pgfqpoint{0.984239in}{0.726000in}}%
\pgfpathcurveto{\pgfqpoint{0.984239in}{0.714950in}}{\pgfqpoint{0.988630in}{0.704351in}}{\pgfqpoint{0.996443in}{0.696537in}}%
\pgfpathcurveto{\pgfqpoint{1.004257in}{0.688724in}}{\pgfqpoint{1.014856in}{0.684333in}}{\pgfqpoint{1.025906in}{0.684333in}}%
\pgfpathclose%
\pgfusepath{stroke,fill}%
\end{pgfscope}%
\begin{pgfscope}%
\pgfpathrectangle{\pgfqpoint{0.800000in}{0.528000in}}{\pgfqpoint{4.960000in}{3.696000in}}%
\pgfusepath{clip}%
\pgfsetbuttcap%
\pgfsetroundjoin%
\definecolor{currentfill}{rgb}{0.000000,0.000000,0.000000}%
\pgfsetfillcolor{currentfill}%
\pgfsetlinewidth{1.003750pt}%
\definecolor{currentstroke}{rgb}{0.000000,0.000000,0.000000}%
\pgfsetstrokecolor{currentstroke}%
\pgfsetdash{}{0pt}%
\pgfpathmoveto{\pgfqpoint{1.025906in}{0.684333in}}%
\pgfpathcurveto{\pgfqpoint{1.036956in}{0.684333in}}{\pgfqpoint{1.047555in}{0.688724in}}{\pgfqpoint{1.055369in}{0.696537in}}%
\pgfpathcurveto{\pgfqpoint{1.063182in}{0.704351in}}{\pgfqpoint{1.067573in}{0.714950in}}{\pgfqpoint{1.067573in}{0.726000in}}%
\pgfpathcurveto{\pgfqpoint{1.067573in}{0.737050in}}{\pgfqpoint{1.063182in}{0.747649in}}{\pgfqpoint{1.055369in}{0.755463in}}%
\pgfpathcurveto{\pgfqpoint{1.047555in}{0.763276in}}{\pgfqpoint{1.036956in}{0.767667in}}{\pgfqpoint{1.025906in}{0.767667in}}%
\pgfpathcurveto{\pgfqpoint{1.014856in}{0.767667in}}{\pgfqpoint{1.004257in}{0.763276in}}{\pgfqpoint{0.996443in}{0.755463in}}%
\pgfpathcurveto{\pgfqpoint{0.988630in}{0.747649in}}{\pgfqpoint{0.984239in}{0.737050in}}{\pgfqpoint{0.984239in}{0.726000in}}%
\pgfpathcurveto{\pgfqpoint{0.984239in}{0.714950in}}{\pgfqpoint{0.988630in}{0.704351in}}{\pgfqpoint{0.996443in}{0.696537in}}%
\pgfpathcurveto{\pgfqpoint{1.004257in}{0.688724in}}{\pgfqpoint{1.014856in}{0.684333in}}{\pgfqpoint{1.025906in}{0.684333in}}%
\pgfpathclose%
\pgfusepath{stroke,fill}%
\end{pgfscope}%
\begin{pgfscope}%
\pgfpathrectangle{\pgfqpoint{0.800000in}{0.528000in}}{\pgfqpoint{4.960000in}{3.696000in}}%
\pgfusepath{clip}%
\pgfsetbuttcap%
\pgfsetroundjoin%
\definecolor{currentfill}{rgb}{0.000000,0.000000,0.000000}%
\pgfsetfillcolor{currentfill}%
\pgfsetlinewidth{1.003750pt}%
\definecolor{currentstroke}{rgb}{0.000000,0.000000,0.000000}%
\pgfsetstrokecolor{currentstroke}%
\pgfsetdash{}{0pt}%
\pgfpathmoveto{\pgfqpoint{1.025906in}{0.684333in}}%
\pgfpathcurveto{\pgfqpoint{1.036956in}{0.684333in}}{\pgfqpoint{1.047555in}{0.688724in}}{\pgfqpoint{1.055369in}{0.696537in}}%
\pgfpathcurveto{\pgfqpoint{1.063182in}{0.704351in}}{\pgfqpoint{1.067573in}{0.714950in}}{\pgfqpoint{1.067573in}{0.726000in}}%
\pgfpathcurveto{\pgfqpoint{1.067573in}{0.737050in}}{\pgfqpoint{1.063182in}{0.747649in}}{\pgfqpoint{1.055369in}{0.755463in}}%
\pgfpathcurveto{\pgfqpoint{1.047555in}{0.763276in}}{\pgfqpoint{1.036956in}{0.767667in}}{\pgfqpoint{1.025906in}{0.767667in}}%
\pgfpathcurveto{\pgfqpoint{1.014856in}{0.767667in}}{\pgfqpoint{1.004257in}{0.763276in}}{\pgfqpoint{0.996443in}{0.755463in}}%
\pgfpathcurveto{\pgfqpoint{0.988630in}{0.747649in}}{\pgfqpoint{0.984239in}{0.737050in}}{\pgfqpoint{0.984239in}{0.726000in}}%
\pgfpathcurveto{\pgfqpoint{0.984239in}{0.714950in}}{\pgfqpoint{0.988630in}{0.704351in}}{\pgfqpoint{0.996443in}{0.696537in}}%
\pgfpathcurveto{\pgfqpoint{1.004257in}{0.688724in}}{\pgfqpoint{1.014856in}{0.684333in}}{\pgfqpoint{1.025906in}{0.684333in}}%
\pgfpathclose%
\pgfusepath{stroke,fill}%
\end{pgfscope}%
\begin{pgfscope}%
\pgfpathrectangle{\pgfqpoint{0.800000in}{0.528000in}}{\pgfqpoint{4.960000in}{3.696000in}}%
\pgfusepath{clip}%
\pgfsetbuttcap%
\pgfsetroundjoin%
\definecolor{currentfill}{rgb}{0.000000,0.000000,0.000000}%
\pgfsetfillcolor{currentfill}%
\pgfsetlinewidth{1.003750pt}%
\definecolor{currentstroke}{rgb}{0.000000,0.000000,0.000000}%
\pgfsetstrokecolor{currentstroke}%
\pgfsetdash{}{0pt}%
\pgfpathmoveto{\pgfqpoint{1.025906in}{0.684333in}}%
\pgfpathcurveto{\pgfqpoint{1.036956in}{0.684333in}}{\pgfqpoint{1.047555in}{0.688724in}}{\pgfqpoint{1.055369in}{0.696537in}}%
\pgfpathcurveto{\pgfqpoint{1.063182in}{0.704351in}}{\pgfqpoint{1.067573in}{0.714950in}}{\pgfqpoint{1.067573in}{0.726000in}}%
\pgfpathcurveto{\pgfqpoint{1.067573in}{0.737050in}}{\pgfqpoint{1.063182in}{0.747649in}}{\pgfqpoint{1.055369in}{0.755463in}}%
\pgfpathcurveto{\pgfqpoint{1.047555in}{0.763276in}}{\pgfqpoint{1.036956in}{0.767667in}}{\pgfqpoint{1.025906in}{0.767667in}}%
\pgfpathcurveto{\pgfqpoint{1.014856in}{0.767667in}}{\pgfqpoint{1.004257in}{0.763276in}}{\pgfqpoint{0.996443in}{0.755463in}}%
\pgfpathcurveto{\pgfqpoint{0.988630in}{0.747649in}}{\pgfqpoint{0.984239in}{0.737050in}}{\pgfqpoint{0.984239in}{0.726000in}}%
\pgfpathcurveto{\pgfqpoint{0.984239in}{0.714950in}}{\pgfqpoint{0.988630in}{0.704351in}}{\pgfqpoint{0.996443in}{0.696537in}}%
\pgfpathcurveto{\pgfqpoint{1.004257in}{0.688724in}}{\pgfqpoint{1.014856in}{0.684333in}}{\pgfqpoint{1.025906in}{0.684333in}}%
\pgfpathclose%
\pgfusepath{stroke,fill}%
\end{pgfscope}%
\begin{pgfscope}%
\pgfpathrectangle{\pgfqpoint{0.800000in}{0.528000in}}{\pgfqpoint{4.960000in}{3.696000in}}%
\pgfusepath{clip}%
\pgfsetbuttcap%
\pgfsetroundjoin%
\definecolor{currentfill}{rgb}{0.000000,0.000000,0.000000}%
\pgfsetfillcolor{currentfill}%
\pgfsetlinewidth{1.003750pt}%
\definecolor{currentstroke}{rgb}{0.000000,0.000000,0.000000}%
\pgfsetstrokecolor{currentstroke}%
\pgfsetdash{}{0pt}%
\pgfpathmoveto{\pgfqpoint{1.025906in}{0.684333in}}%
\pgfpathcurveto{\pgfqpoint{1.036956in}{0.684333in}}{\pgfqpoint{1.047555in}{0.688724in}}{\pgfqpoint{1.055369in}{0.696537in}}%
\pgfpathcurveto{\pgfqpoint{1.063182in}{0.704351in}}{\pgfqpoint{1.067573in}{0.714950in}}{\pgfqpoint{1.067573in}{0.726000in}}%
\pgfpathcurveto{\pgfqpoint{1.067573in}{0.737050in}}{\pgfqpoint{1.063182in}{0.747649in}}{\pgfqpoint{1.055369in}{0.755463in}}%
\pgfpathcurveto{\pgfqpoint{1.047555in}{0.763276in}}{\pgfqpoint{1.036956in}{0.767667in}}{\pgfqpoint{1.025906in}{0.767667in}}%
\pgfpathcurveto{\pgfqpoint{1.014856in}{0.767667in}}{\pgfqpoint{1.004257in}{0.763276in}}{\pgfqpoint{0.996443in}{0.755463in}}%
\pgfpathcurveto{\pgfqpoint{0.988630in}{0.747649in}}{\pgfqpoint{0.984239in}{0.737050in}}{\pgfqpoint{0.984239in}{0.726000in}}%
\pgfpathcurveto{\pgfqpoint{0.984239in}{0.714950in}}{\pgfqpoint{0.988630in}{0.704351in}}{\pgfqpoint{0.996443in}{0.696537in}}%
\pgfpathcurveto{\pgfqpoint{1.004257in}{0.688724in}}{\pgfqpoint{1.014856in}{0.684333in}}{\pgfqpoint{1.025906in}{0.684333in}}%
\pgfpathclose%
\pgfusepath{stroke,fill}%
\end{pgfscope}%
\begin{pgfscope}%
\pgfpathrectangle{\pgfqpoint{0.800000in}{0.528000in}}{\pgfqpoint{4.960000in}{3.696000in}}%
\pgfusepath{clip}%
\pgfsetbuttcap%
\pgfsetroundjoin%
\definecolor{currentfill}{rgb}{0.000000,0.000000,0.000000}%
\pgfsetfillcolor{currentfill}%
\pgfsetlinewidth{1.003750pt}%
\definecolor{currentstroke}{rgb}{0.000000,0.000000,0.000000}%
\pgfsetstrokecolor{currentstroke}%
\pgfsetdash{}{0pt}%
\pgfpathmoveto{\pgfqpoint{1.025906in}{0.684333in}}%
\pgfpathcurveto{\pgfqpoint{1.036956in}{0.684333in}}{\pgfqpoint{1.047555in}{0.688724in}}{\pgfqpoint{1.055369in}{0.696537in}}%
\pgfpathcurveto{\pgfqpoint{1.063182in}{0.704351in}}{\pgfqpoint{1.067573in}{0.714950in}}{\pgfqpoint{1.067573in}{0.726000in}}%
\pgfpathcurveto{\pgfqpoint{1.067573in}{0.737050in}}{\pgfqpoint{1.063182in}{0.747649in}}{\pgfqpoint{1.055369in}{0.755463in}}%
\pgfpathcurveto{\pgfqpoint{1.047555in}{0.763276in}}{\pgfqpoint{1.036956in}{0.767667in}}{\pgfqpoint{1.025906in}{0.767667in}}%
\pgfpathcurveto{\pgfqpoint{1.014856in}{0.767667in}}{\pgfqpoint{1.004257in}{0.763276in}}{\pgfqpoint{0.996443in}{0.755463in}}%
\pgfpathcurveto{\pgfqpoint{0.988630in}{0.747649in}}{\pgfqpoint{0.984239in}{0.737050in}}{\pgfqpoint{0.984239in}{0.726000in}}%
\pgfpathcurveto{\pgfqpoint{0.984239in}{0.714950in}}{\pgfqpoint{0.988630in}{0.704351in}}{\pgfqpoint{0.996443in}{0.696537in}}%
\pgfpathcurveto{\pgfqpoint{1.004257in}{0.688724in}}{\pgfqpoint{1.014856in}{0.684333in}}{\pgfqpoint{1.025906in}{0.684333in}}%
\pgfpathclose%
\pgfusepath{stroke,fill}%
\end{pgfscope}%
\begin{pgfscope}%
\pgfpathrectangle{\pgfqpoint{0.800000in}{0.528000in}}{\pgfqpoint{4.960000in}{3.696000in}}%
\pgfusepath{clip}%
\pgfsetbuttcap%
\pgfsetroundjoin%
\definecolor{currentfill}{rgb}{0.000000,0.000000,0.000000}%
\pgfsetfillcolor{currentfill}%
\pgfsetlinewidth{1.003750pt}%
\definecolor{currentstroke}{rgb}{0.000000,0.000000,0.000000}%
\pgfsetstrokecolor{currentstroke}%
\pgfsetdash{}{0pt}%
\pgfpathmoveto{\pgfqpoint{1.025906in}{0.684333in}}%
\pgfpathcurveto{\pgfqpoint{1.036956in}{0.684333in}}{\pgfqpoint{1.047555in}{0.688724in}}{\pgfqpoint{1.055369in}{0.696537in}}%
\pgfpathcurveto{\pgfqpoint{1.063182in}{0.704351in}}{\pgfqpoint{1.067573in}{0.714950in}}{\pgfqpoint{1.067573in}{0.726000in}}%
\pgfpathcurveto{\pgfqpoint{1.067573in}{0.737050in}}{\pgfqpoint{1.063182in}{0.747649in}}{\pgfqpoint{1.055369in}{0.755463in}}%
\pgfpathcurveto{\pgfqpoint{1.047555in}{0.763276in}}{\pgfqpoint{1.036956in}{0.767667in}}{\pgfqpoint{1.025906in}{0.767667in}}%
\pgfpathcurveto{\pgfqpoint{1.014856in}{0.767667in}}{\pgfqpoint{1.004257in}{0.763276in}}{\pgfqpoint{0.996443in}{0.755463in}}%
\pgfpathcurveto{\pgfqpoint{0.988630in}{0.747649in}}{\pgfqpoint{0.984239in}{0.737050in}}{\pgfqpoint{0.984239in}{0.726000in}}%
\pgfpathcurveto{\pgfqpoint{0.984239in}{0.714950in}}{\pgfqpoint{0.988630in}{0.704351in}}{\pgfqpoint{0.996443in}{0.696537in}}%
\pgfpathcurveto{\pgfqpoint{1.004257in}{0.688724in}}{\pgfqpoint{1.014856in}{0.684333in}}{\pgfqpoint{1.025906in}{0.684333in}}%
\pgfpathclose%
\pgfusepath{stroke,fill}%
\end{pgfscope}%
\begin{pgfscope}%
\pgfpathrectangle{\pgfqpoint{0.800000in}{0.528000in}}{\pgfqpoint{4.960000in}{3.696000in}}%
\pgfusepath{clip}%
\pgfsetbuttcap%
\pgfsetroundjoin%
\definecolor{currentfill}{rgb}{0.000000,0.000000,0.000000}%
\pgfsetfillcolor{currentfill}%
\pgfsetlinewidth{1.003750pt}%
\definecolor{currentstroke}{rgb}{0.000000,0.000000,0.000000}%
\pgfsetstrokecolor{currentstroke}%
\pgfsetdash{}{0pt}%
\pgfpathmoveto{\pgfqpoint{1.025906in}{0.684333in}}%
\pgfpathcurveto{\pgfqpoint{1.036956in}{0.684333in}}{\pgfqpoint{1.047555in}{0.688724in}}{\pgfqpoint{1.055369in}{0.696537in}}%
\pgfpathcurveto{\pgfqpoint{1.063182in}{0.704351in}}{\pgfqpoint{1.067573in}{0.714950in}}{\pgfqpoint{1.067573in}{0.726000in}}%
\pgfpathcurveto{\pgfqpoint{1.067573in}{0.737050in}}{\pgfqpoint{1.063182in}{0.747649in}}{\pgfqpoint{1.055369in}{0.755463in}}%
\pgfpathcurveto{\pgfqpoint{1.047555in}{0.763276in}}{\pgfqpoint{1.036956in}{0.767667in}}{\pgfqpoint{1.025906in}{0.767667in}}%
\pgfpathcurveto{\pgfqpoint{1.014856in}{0.767667in}}{\pgfqpoint{1.004257in}{0.763276in}}{\pgfqpoint{0.996443in}{0.755463in}}%
\pgfpathcurveto{\pgfqpoint{0.988630in}{0.747649in}}{\pgfqpoint{0.984239in}{0.737050in}}{\pgfqpoint{0.984239in}{0.726000in}}%
\pgfpathcurveto{\pgfqpoint{0.984239in}{0.714950in}}{\pgfqpoint{0.988630in}{0.704351in}}{\pgfqpoint{0.996443in}{0.696537in}}%
\pgfpathcurveto{\pgfqpoint{1.004257in}{0.688724in}}{\pgfqpoint{1.014856in}{0.684333in}}{\pgfqpoint{1.025906in}{0.684333in}}%
\pgfpathclose%
\pgfusepath{stroke,fill}%
\end{pgfscope}%
\begin{pgfscope}%
\pgfpathrectangle{\pgfqpoint{0.800000in}{0.528000in}}{\pgfqpoint{4.960000in}{3.696000in}}%
\pgfusepath{clip}%
\pgfsetbuttcap%
\pgfsetroundjoin%
\definecolor{currentfill}{rgb}{0.000000,0.000000,0.000000}%
\pgfsetfillcolor{currentfill}%
\pgfsetlinewidth{1.003750pt}%
\definecolor{currentstroke}{rgb}{0.000000,0.000000,0.000000}%
\pgfsetstrokecolor{currentstroke}%
\pgfsetdash{}{0pt}%
\pgfpathmoveto{\pgfqpoint{1.025906in}{0.684333in}}%
\pgfpathcurveto{\pgfqpoint{1.036956in}{0.684333in}}{\pgfqpoint{1.047555in}{0.688724in}}{\pgfqpoint{1.055369in}{0.696537in}}%
\pgfpathcurveto{\pgfqpoint{1.063182in}{0.704351in}}{\pgfqpoint{1.067573in}{0.714950in}}{\pgfqpoint{1.067573in}{0.726000in}}%
\pgfpathcurveto{\pgfqpoint{1.067573in}{0.737050in}}{\pgfqpoint{1.063182in}{0.747649in}}{\pgfqpoint{1.055369in}{0.755463in}}%
\pgfpathcurveto{\pgfqpoint{1.047555in}{0.763276in}}{\pgfqpoint{1.036956in}{0.767667in}}{\pgfqpoint{1.025906in}{0.767667in}}%
\pgfpathcurveto{\pgfqpoint{1.014856in}{0.767667in}}{\pgfqpoint{1.004257in}{0.763276in}}{\pgfqpoint{0.996443in}{0.755463in}}%
\pgfpathcurveto{\pgfqpoint{0.988630in}{0.747649in}}{\pgfqpoint{0.984239in}{0.737050in}}{\pgfqpoint{0.984239in}{0.726000in}}%
\pgfpathcurveto{\pgfqpoint{0.984239in}{0.714950in}}{\pgfqpoint{0.988630in}{0.704351in}}{\pgfqpoint{0.996443in}{0.696537in}}%
\pgfpathcurveto{\pgfqpoint{1.004257in}{0.688724in}}{\pgfqpoint{1.014856in}{0.684333in}}{\pgfqpoint{1.025906in}{0.684333in}}%
\pgfpathclose%
\pgfusepath{stroke,fill}%
\end{pgfscope}%
\begin{pgfscope}%
\pgfpathrectangle{\pgfqpoint{0.800000in}{0.528000in}}{\pgfqpoint{4.960000in}{3.696000in}}%
\pgfusepath{clip}%
\pgfsetbuttcap%
\pgfsetroundjoin%
\definecolor{currentfill}{rgb}{0.000000,0.000000,0.000000}%
\pgfsetfillcolor{currentfill}%
\pgfsetlinewidth{1.003750pt}%
\definecolor{currentstroke}{rgb}{0.000000,0.000000,0.000000}%
\pgfsetstrokecolor{currentstroke}%
\pgfsetdash{}{0pt}%
\pgfpathmoveto{\pgfqpoint{1.025906in}{0.684333in}}%
\pgfpathcurveto{\pgfqpoint{1.036956in}{0.684333in}}{\pgfqpoint{1.047555in}{0.688724in}}{\pgfqpoint{1.055369in}{0.696537in}}%
\pgfpathcurveto{\pgfqpoint{1.063182in}{0.704351in}}{\pgfqpoint{1.067573in}{0.714950in}}{\pgfqpoint{1.067573in}{0.726000in}}%
\pgfpathcurveto{\pgfqpoint{1.067573in}{0.737050in}}{\pgfqpoint{1.063182in}{0.747649in}}{\pgfqpoint{1.055369in}{0.755463in}}%
\pgfpathcurveto{\pgfqpoint{1.047555in}{0.763276in}}{\pgfqpoint{1.036956in}{0.767667in}}{\pgfqpoint{1.025906in}{0.767667in}}%
\pgfpathcurveto{\pgfqpoint{1.014856in}{0.767667in}}{\pgfqpoint{1.004257in}{0.763276in}}{\pgfqpoint{0.996443in}{0.755463in}}%
\pgfpathcurveto{\pgfqpoint{0.988630in}{0.747649in}}{\pgfqpoint{0.984239in}{0.737050in}}{\pgfqpoint{0.984239in}{0.726000in}}%
\pgfpathcurveto{\pgfqpoint{0.984239in}{0.714950in}}{\pgfqpoint{0.988630in}{0.704351in}}{\pgfqpoint{0.996443in}{0.696537in}}%
\pgfpathcurveto{\pgfqpoint{1.004257in}{0.688724in}}{\pgfqpoint{1.014856in}{0.684333in}}{\pgfqpoint{1.025906in}{0.684333in}}%
\pgfpathclose%
\pgfusepath{stroke,fill}%
\end{pgfscope}%
\begin{pgfscope}%
\pgfpathrectangle{\pgfqpoint{0.800000in}{0.528000in}}{\pgfqpoint{4.960000in}{3.696000in}}%
\pgfusepath{clip}%
\pgfsetbuttcap%
\pgfsetroundjoin%
\definecolor{currentfill}{rgb}{0.000000,0.000000,0.000000}%
\pgfsetfillcolor{currentfill}%
\pgfsetlinewidth{1.003750pt}%
\definecolor{currentstroke}{rgb}{0.000000,0.000000,0.000000}%
\pgfsetstrokecolor{currentstroke}%
\pgfsetdash{}{0pt}%
\pgfpathmoveto{\pgfqpoint{1.025906in}{0.684333in}}%
\pgfpathcurveto{\pgfqpoint{1.036956in}{0.684333in}}{\pgfqpoint{1.047555in}{0.688724in}}{\pgfqpoint{1.055369in}{0.696537in}}%
\pgfpathcurveto{\pgfqpoint{1.063182in}{0.704351in}}{\pgfqpoint{1.067573in}{0.714950in}}{\pgfqpoint{1.067573in}{0.726000in}}%
\pgfpathcurveto{\pgfqpoint{1.067573in}{0.737050in}}{\pgfqpoint{1.063182in}{0.747649in}}{\pgfqpoint{1.055369in}{0.755463in}}%
\pgfpathcurveto{\pgfqpoint{1.047555in}{0.763276in}}{\pgfqpoint{1.036956in}{0.767667in}}{\pgfqpoint{1.025906in}{0.767667in}}%
\pgfpathcurveto{\pgfqpoint{1.014856in}{0.767667in}}{\pgfqpoint{1.004257in}{0.763276in}}{\pgfqpoint{0.996443in}{0.755463in}}%
\pgfpathcurveto{\pgfqpoint{0.988630in}{0.747649in}}{\pgfqpoint{0.984239in}{0.737050in}}{\pgfqpoint{0.984239in}{0.726000in}}%
\pgfpathcurveto{\pgfqpoint{0.984239in}{0.714950in}}{\pgfqpoint{0.988630in}{0.704351in}}{\pgfqpoint{0.996443in}{0.696537in}}%
\pgfpathcurveto{\pgfqpoint{1.004257in}{0.688724in}}{\pgfqpoint{1.014856in}{0.684333in}}{\pgfqpoint{1.025906in}{0.684333in}}%
\pgfpathclose%
\pgfusepath{stroke,fill}%
\end{pgfscope}%
\begin{pgfscope}%
\pgfpathrectangle{\pgfqpoint{0.800000in}{0.528000in}}{\pgfqpoint{4.960000in}{3.696000in}}%
\pgfusepath{clip}%
\pgfsetbuttcap%
\pgfsetroundjoin%
\definecolor{currentfill}{rgb}{0.000000,0.000000,0.000000}%
\pgfsetfillcolor{currentfill}%
\pgfsetlinewidth{1.003750pt}%
\definecolor{currentstroke}{rgb}{0.000000,0.000000,0.000000}%
\pgfsetstrokecolor{currentstroke}%
\pgfsetdash{}{0pt}%
\pgfpathmoveto{\pgfqpoint{1.025906in}{0.684333in}}%
\pgfpathcurveto{\pgfqpoint{1.036956in}{0.684333in}}{\pgfqpoint{1.047555in}{0.688724in}}{\pgfqpoint{1.055369in}{0.696537in}}%
\pgfpathcurveto{\pgfqpoint{1.063182in}{0.704351in}}{\pgfqpoint{1.067573in}{0.714950in}}{\pgfqpoint{1.067573in}{0.726000in}}%
\pgfpathcurveto{\pgfqpoint{1.067573in}{0.737050in}}{\pgfqpoint{1.063182in}{0.747649in}}{\pgfqpoint{1.055369in}{0.755463in}}%
\pgfpathcurveto{\pgfqpoint{1.047555in}{0.763276in}}{\pgfqpoint{1.036956in}{0.767667in}}{\pgfqpoint{1.025906in}{0.767667in}}%
\pgfpathcurveto{\pgfqpoint{1.014856in}{0.767667in}}{\pgfqpoint{1.004257in}{0.763276in}}{\pgfqpoint{0.996443in}{0.755463in}}%
\pgfpathcurveto{\pgfqpoint{0.988630in}{0.747649in}}{\pgfqpoint{0.984239in}{0.737050in}}{\pgfqpoint{0.984239in}{0.726000in}}%
\pgfpathcurveto{\pgfqpoint{0.984239in}{0.714950in}}{\pgfqpoint{0.988630in}{0.704351in}}{\pgfqpoint{0.996443in}{0.696537in}}%
\pgfpathcurveto{\pgfqpoint{1.004257in}{0.688724in}}{\pgfqpoint{1.014856in}{0.684333in}}{\pgfqpoint{1.025906in}{0.684333in}}%
\pgfpathclose%
\pgfusepath{stroke,fill}%
\end{pgfscope}%
\begin{pgfscope}%
\pgfpathrectangle{\pgfqpoint{0.800000in}{0.528000in}}{\pgfqpoint{4.960000in}{3.696000in}}%
\pgfusepath{clip}%
\pgfsetbuttcap%
\pgfsetroundjoin%
\definecolor{currentfill}{rgb}{0.000000,0.000000,0.000000}%
\pgfsetfillcolor{currentfill}%
\pgfsetlinewidth{1.003750pt}%
\definecolor{currentstroke}{rgb}{0.000000,0.000000,0.000000}%
\pgfsetstrokecolor{currentstroke}%
\pgfsetdash{}{0pt}%
\pgfpathmoveto{\pgfqpoint{1.025906in}{0.684333in}}%
\pgfpathcurveto{\pgfqpoint{1.036956in}{0.684333in}}{\pgfqpoint{1.047555in}{0.688724in}}{\pgfqpoint{1.055369in}{0.696537in}}%
\pgfpathcurveto{\pgfqpoint{1.063182in}{0.704351in}}{\pgfqpoint{1.067573in}{0.714950in}}{\pgfqpoint{1.067573in}{0.726000in}}%
\pgfpathcurveto{\pgfqpoint{1.067573in}{0.737050in}}{\pgfqpoint{1.063182in}{0.747649in}}{\pgfqpoint{1.055369in}{0.755463in}}%
\pgfpathcurveto{\pgfqpoint{1.047555in}{0.763276in}}{\pgfqpoint{1.036956in}{0.767667in}}{\pgfqpoint{1.025906in}{0.767667in}}%
\pgfpathcurveto{\pgfqpoint{1.014856in}{0.767667in}}{\pgfqpoint{1.004257in}{0.763276in}}{\pgfqpoint{0.996443in}{0.755463in}}%
\pgfpathcurveto{\pgfqpoint{0.988630in}{0.747649in}}{\pgfqpoint{0.984239in}{0.737050in}}{\pgfqpoint{0.984239in}{0.726000in}}%
\pgfpathcurveto{\pgfqpoint{0.984239in}{0.714950in}}{\pgfqpoint{0.988630in}{0.704351in}}{\pgfqpoint{0.996443in}{0.696537in}}%
\pgfpathcurveto{\pgfqpoint{1.004257in}{0.688724in}}{\pgfqpoint{1.014856in}{0.684333in}}{\pgfqpoint{1.025906in}{0.684333in}}%
\pgfpathclose%
\pgfusepath{stroke,fill}%
\end{pgfscope}%
\begin{pgfscope}%
\pgfpathrectangle{\pgfqpoint{0.800000in}{0.528000in}}{\pgfqpoint{4.960000in}{3.696000in}}%
\pgfusepath{clip}%
\pgfsetbuttcap%
\pgfsetroundjoin%
\definecolor{currentfill}{rgb}{0.000000,0.000000,0.000000}%
\pgfsetfillcolor{currentfill}%
\pgfsetlinewidth{1.003750pt}%
\definecolor{currentstroke}{rgb}{0.000000,0.000000,0.000000}%
\pgfsetstrokecolor{currentstroke}%
\pgfsetdash{}{0pt}%
\pgfpathmoveto{\pgfqpoint{1.025906in}{0.684333in}}%
\pgfpathcurveto{\pgfqpoint{1.036956in}{0.684333in}}{\pgfqpoint{1.047555in}{0.688724in}}{\pgfqpoint{1.055369in}{0.696537in}}%
\pgfpathcurveto{\pgfqpoint{1.063182in}{0.704351in}}{\pgfqpoint{1.067573in}{0.714950in}}{\pgfqpoint{1.067573in}{0.726000in}}%
\pgfpathcurveto{\pgfqpoint{1.067573in}{0.737050in}}{\pgfqpoint{1.063182in}{0.747649in}}{\pgfqpoint{1.055369in}{0.755463in}}%
\pgfpathcurveto{\pgfqpoint{1.047555in}{0.763276in}}{\pgfqpoint{1.036956in}{0.767667in}}{\pgfqpoint{1.025906in}{0.767667in}}%
\pgfpathcurveto{\pgfqpoint{1.014856in}{0.767667in}}{\pgfqpoint{1.004257in}{0.763276in}}{\pgfqpoint{0.996443in}{0.755463in}}%
\pgfpathcurveto{\pgfqpoint{0.988630in}{0.747649in}}{\pgfqpoint{0.984239in}{0.737050in}}{\pgfqpoint{0.984239in}{0.726000in}}%
\pgfpathcurveto{\pgfqpoint{0.984239in}{0.714950in}}{\pgfqpoint{0.988630in}{0.704351in}}{\pgfqpoint{0.996443in}{0.696537in}}%
\pgfpathcurveto{\pgfqpoint{1.004257in}{0.688724in}}{\pgfqpoint{1.014856in}{0.684333in}}{\pgfqpoint{1.025906in}{0.684333in}}%
\pgfpathclose%
\pgfusepath{stroke,fill}%
\end{pgfscope}%
\begin{pgfscope}%
\pgfpathrectangle{\pgfqpoint{0.800000in}{0.528000in}}{\pgfqpoint{4.960000in}{3.696000in}}%
\pgfusepath{clip}%
\pgfsetbuttcap%
\pgfsetroundjoin%
\definecolor{currentfill}{rgb}{0.000000,0.000000,0.000000}%
\pgfsetfillcolor{currentfill}%
\pgfsetlinewidth{1.003750pt}%
\definecolor{currentstroke}{rgb}{0.000000,0.000000,0.000000}%
\pgfsetstrokecolor{currentstroke}%
\pgfsetdash{}{0pt}%
\pgfpathmoveto{\pgfqpoint{1.025906in}{0.684333in}}%
\pgfpathcurveto{\pgfqpoint{1.036956in}{0.684333in}}{\pgfqpoint{1.047555in}{0.688724in}}{\pgfqpoint{1.055369in}{0.696537in}}%
\pgfpathcurveto{\pgfqpoint{1.063182in}{0.704351in}}{\pgfqpoint{1.067573in}{0.714950in}}{\pgfqpoint{1.067573in}{0.726000in}}%
\pgfpathcurveto{\pgfqpoint{1.067573in}{0.737050in}}{\pgfqpoint{1.063182in}{0.747649in}}{\pgfqpoint{1.055369in}{0.755463in}}%
\pgfpathcurveto{\pgfqpoint{1.047555in}{0.763276in}}{\pgfqpoint{1.036956in}{0.767667in}}{\pgfqpoint{1.025906in}{0.767667in}}%
\pgfpathcurveto{\pgfqpoint{1.014856in}{0.767667in}}{\pgfqpoint{1.004257in}{0.763276in}}{\pgfqpoint{0.996443in}{0.755463in}}%
\pgfpathcurveto{\pgfqpoint{0.988630in}{0.747649in}}{\pgfqpoint{0.984239in}{0.737050in}}{\pgfqpoint{0.984239in}{0.726000in}}%
\pgfpathcurveto{\pgfqpoint{0.984239in}{0.714950in}}{\pgfqpoint{0.988630in}{0.704351in}}{\pgfqpoint{0.996443in}{0.696537in}}%
\pgfpathcurveto{\pgfqpoint{1.004257in}{0.688724in}}{\pgfqpoint{1.014856in}{0.684333in}}{\pgfqpoint{1.025906in}{0.684333in}}%
\pgfpathclose%
\pgfusepath{stroke,fill}%
\end{pgfscope}%
\begin{pgfscope}%
\pgfpathrectangle{\pgfqpoint{0.800000in}{0.528000in}}{\pgfqpoint{4.960000in}{3.696000in}}%
\pgfusepath{clip}%
\pgfsetbuttcap%
\pgfsetroundjoin%
\definecolor{currentfill}{rgb}{0.000000,0.000000,0.000000}%
\pgfsetfillcolor{currentfill}%
\pgfsetlinewidth{1.003750pt}%
\definecolor{currentstroke}{rgb}{0.000000,0.000000,0.000000}%
\pgfsetstrokecolor{currentstroke}%
\pgfsetdash{}{0pt}%
\pgfpathmoveto{\pgfqpoint{1.025906in}{0.684333in}}%
\pgfpathcurveto{\pgfqpoint{1.036956in}{0.684333in}}{\pgfqpoint{1.047555in}{0.688724in}}{\pgfqpoint{1.055369in}{0.696537in}}%
\pgfpathcurveto{\pgfqpoint{1.063182in}{0.704351in}}{\pgfqpoint{1.067573in}{0.714950in}}{\pgfqpoint{1.067573in}{0.726000in}}%
\pgfpathcurveto{\pgfqpoint{1.067573in}{0.737050in}}{\pgfqpoint{1.063182in}{0.747649in}}{\pgfqpoint{1.055369in}{0.755463in}}%
\pgfpathcurveto{\pgfqpoint{1.047555in}{0.763276in}}{\pgfqpoint{1.036956in}{0.767667in}}{\pgfqpoint{1.025906in}{0.767667in}}%
\pgfpathcurveto{\pgfqpoint{1.014856in}{0.767667in}}{\pgfqpoint{1.004257in}{0.763276in}}{\pgfqpoint{0.996443in}{0.755463in}}%
\pgfpathcurveto{\pgfqpoint{0.988630in}{0.747649in}}{\pgfqpoint{0.984239in}{0.737050in}}{\pgfqpoint{0.984239in}{0.726000in}}%
\pgfpathcurveto{\pgfqpoint{0.984239in}{0.714950in}}{\pgfqpoint{0.988630in}{0.704351in}}{\pgfqpoint{0.996443in}{0.696537in}}%
\pgfpathcurveto{\pgfqpoint{1.004257in}{0.688724in}}{\pgfqpoint{1.014856in}{0.684333in}}{\pgfqpoint{1.025906in}{0.684333in}}%
\pgfpathclose%
\pgfusepath{stroke,fill}%
\end{pgfscope}%
\begin{pgfscope}%
\pgfpathrectangle{\pgfqpoint{0.800000in}{0.528000in}}{\pgfqpoint{4.960000in}{3.696000in}}%
\pgfusepath{clip}%
\pgfsetbuttcap%
\pgfsetroundjoin%
\definecolor{currentfill}{rgb}{0.000000,0.000000,0.000000}%
\pgfsetfillcolor{currentfill}%
\pgfsetlinewidth{1.003750pt}%
\definecolor{currentstroke}{rgb}{0.000000,0.000000,0.000000}%
\pgfsetstrokecolor{currentstroke}%
\pgfsetdash{}{0pt}%
\pgfpathmoveto{\pgfqpoint{1.025906in}{0.684333in}}%
\pgfpathcurveto{\pgfqpoint{1.036956in}{0.684333in}}{\pgfqpoint{1.047555in}{0.688724in}}{\pgfqpoint{1.055369in}{0.696537in}}%
\pgfpathcurveto{\pgfqpoint{1.063182in}{0.704351in}}{\pgfqpoint{1.067573in}{0.714950in}}{\pgfqpoint{1.067573in}{0.726000in}}%
\pgfpathcurveto{\pgfqpoint{1.067573in}{0.737050in}}{\pgfqpoint{1.063182in}{0.747649in}}{\pgfqpoint{1.055369in}{0.755463in}}%
\pgfpathcurveto{\pgfqpoint{1.047555in}{0.763276in}}{\pgfqpoint{1.036956in}{0.767667in}}{\pgfqpoint{1.025906in}{0.767667in}}%
\pgfpathcurveto{\pgfqpoint{1.014856in}{0.767667in}}{\pgfqpoint{1.004257in}{0.763276in}}{\pgfqpoint{0.996443in}{0.755463in}}%
\pgfpathcurveto{\pgfqpoint{0.988630in}{0.747649in}}{\pgfqpoint{0.984239in}{0.737050in}}{\pgfqpoint{0.984239in}{0.726000in}}%
\pgfpathcurveto{\pgfqpoint{0.984239in}{0.714950in}}{\pgfqpoint{0.988630in}{0.704351in}}{\pgfqpoint{0.996443in}{0.696537in}}%
\pgfpathcurveto{\pgfqpoint{1.004257in}{0.688724in}}{\pgfqpoint{1.014856in}{0.684333in}}{\pgfqpoint{1.025906in}{0.684333in}}%
\pgfpathclose%
\pgfusepath{stroke,fill}%
\end{pgfscope}%
\begin{pgfscope}%
\pgfpathrectangle{\pgfqpoint{0.800000in}{0.528000in}}{\pgfqpoint{4.960000in}{3.696000in}}%
\pgfusepath{clip}%
\pgfsetbuttcap%
\pgfsetroundjoin%
\definecolor{currentfill}{rgb}{0.000000,0.000000,0.000000}%
\pgfsetfillcolor{currentfill}%
\pgfsetlinewidth{1.003750pt}%
\definecolor{currentstroke}{rgb}{0.000000,0.000000,0.000000}%
\pgfsetstrokecolor{currentstroke}%
\pgfsetdash{}{0pt}%
\pgfpathmoveto{\pgfqpoint{1.025906in}{0.684333in}}%
\pgfpathcurveto{\pgfqpoint{1.036956in}{0.684333in}}{\pgfqpoint{1.047555in}{0.688724in}}{\pgfqpoint{1.055369in}{0.696537in}}%
\pgfpathcurveto{\pgfqpoint{1.063182in}{0.704351in}}{\pgfqpoint{1.067573in}{0.714950in}}{\pgfqpoint{1.067573in}{0.726000in}}%
\pgfpathcurveto{\pgfqpoint{1.067573in}{0.737050in}}{\pgfqpoint{1.063182in}{0.747649in}}{\pgfqpoint{1.055369in}{0.755463in}}%
\pgfpathcurveto{\pgfqpoint{1.047555in}{0.763276in}}{\pgfqpoint{1.036956in}{0.767667in}}{\pgfqpoint{1.025906in}{0.767667in}}%
\pgfpathcurveto{\pgfqpoint{1.014856in}{0.767667in}}{\pgfqpoint{1.004257in}{0.763276in}}{\pgfqpoint{0.996443in}{0.755463in}}%
\pgfpathcurveto{\pgfqpoint{0.988630in}{0.747649in}}{\pgfqpoint{0.984239in}{0.737050in}}{\pgfqpoint{0.984239in}{0.726000in}}%
\pgfpathcurveto{\pgfqpoint{0.984239in}{0.714950in}}{\pgfqpoint{0.988630in}{0.704351in}}{\pgfqpoint{0.996443in}{0.696537in}}%
\pgfpathcurveto{\pgfqpoint{1.004257in}{0.688724in}}{\pgfqpoint{1.014856in}{0.684333in}}{\pgfqpoint{1.025906in}{0.684333in}}%
\pgfpathclose%
\pgfusepath{stroke,fill}%
\end{pgfscope}%
\begin{pgfscope}%
\pgfpathrectangle{\pgfqpoint{0.800000in}{0.528000in}}{\pgfqpoint{4.960000in}{3.696000in}}%
\pgfusepath{clip}%
\pgfsetbuttcap%
\pgfsetroundjoin%
\definecolor{currentfill}{rgb}{0.000000,0.000000,0.000000}%
\pgfsetfillcolor{currentfill}%
\pgfsetlinewidth{1.003750pt}%
\definecolor{currentstroke}{rgb}{0.000000,0.000000,0.000000}%
\pgfsetstrokecolor{currentstroke}%
\pgfsetdash{}{0pt}%
\pgfpathmoveto{\pgfqpoint{1.025906in}{0.684333in}}%
\pgfpathcurveto{\pgfqpoint{1.036956in}{0.684333in}}{\pgfqpoint{1.047555in}{0.688724in}}{\pgfqpoint{1.055369in}{0.696537in}}%
\pgfpathcurveto{\pgfqpoint{1.063182in}{0.704351in}}{\pgfqpoint{1.067573in}{0.714950in}}{\pgfqpoint{1.067573in}{0.726000in}}%
\pgfpathcurveto{\pgfqpoint{1.067573in}{0.737050in}}{\pgfqpoint{1.063182in}{0.747649in}}{\pgfqpoint{1.055369in}{0.755463in}}%
\pgfpathcurveto{\pgfqpoint{1.047555in}{0.763276in}}{\pgfqpoint{1.036956in}{0.767667in}}{\pgfqpoint{1.025906in}{0.767667in}}%
\pgfpathcurveto{\pgfqpoint{1.014856in}{0.767667in}}{\pgfqpoint{1.004257in}{0.763276in}}{\pgfqpoint{0.996443in}{0.755463in}}%
\pgfpathcurveto{\pgfqpoint{0.988630in}{0.747649in}}{\pgfqpoint{0.984239in}{0.737050in}}{\pgfqpoint{0.984239in}{0.726000in}}%
\pgfpathcurveto{\pgfqpoint{0.984239in}{0.714950in}}{\pgfqpoint{0.988630in}{0.704351in}}{\pgfqpoint{0.996443in}{0.696537in}}%
\pgfpathcurveto{\pgfqpoint{1.004257in}{0.688724in}}{\pgfqpoint{1.014856in}{0.684333in}}{\pgfqpoint{1.025906in}{0.684333in}}%
\pgfpathclose%
\pgfusepath{stroke,fill}%
\end{pgfscope}%
\begin{pgfscope}%
\pgfpathrectangle{\pgfqpoint{0.800000in}{0.528000in}}{\pgfqpoint{4.960000in}{3.696000in}}%
\pgfusepath{clip}%
\pgfsetbuttcap%
\pgfsetroundjoin%
\definecolor{currentfill}{rgb}{0.000000,0.000000,0.000000}%
\pgfsetfillcolor{currentfill}%
\pgfsetlinewidth{1.003750pt}%
\definecolor{currentstroke}{rgb}{0.000000,0.000000,0.000000}%
\pgfsetstrokecolor{currentstroke}%
\pgfsetdash{}{0pt}%
\pgfpathmoveto{\pgfqpoint{1.025906in}{0.684333in}}%
\pgfpathcurveto{\pgfqpoint{1.036956in}{0.684333in}}{\pgfqpoint{1.047555in}{0.688724in}}{\pgfqpoint{1.055369in}{0.696537in}}%
\pgfpathcurveto{\pgfqpoint{1.063182in}{0.704351in}}{\pgfqpoint{1.067573in}{0.714950in}}{\pgfqpoint{1.067573in}{0.726000in}}%
\pgfpathcurveto{\pgfqpoint{1.067573in}{0.737050in}}{\pgfqpoint{1.063182in}{0.747649in}}{\pgfqpoint{1.055369in}{0.755463in}}%
\pgfpathcurveto{\pgfqpoint{1.047555in}{0.763276in}}{\pgfqpoint{1.036956in}{0.767667in}}{\pgfqpoint{1.025906in}{0.767667in}}%
\pgfpathcurveto{\pgfqpoint{1.014856in}{0.767667in}}{\pgfqpoint{1.004257in}{0.763276in}}{\pgfqpoint{0.996443in}{0.755463in}}%
\pgfpathcurveto{\pgfqpoint{0.988630in}{0.747649in}}{\pgfqpoint{0.984239in}{0.737050in}}{\pgfqpoint{0.984239in}{0.726000in}}%
\pgfpathcurveto{\pgfqpoint{0.984239in}{0.714950in}}{\pgfqpoint{0.988630in}{0.704351in}}{\pgfqpoint{0.996443in}{0.696537in}}%
\pgfpathcurveto{\pgfqpoint{1.004257in}{0.688724in}}{\pgfqpoint{1.014856in}{0.684333in}}{\pgfqpoint{1.025906in}{0.684333in}}%
\pgfpathclose%
\pgfusepath{stroke,fill}%
\end{pgfscope}%
\begin{pgfscope}%
\pgfpathrectangle{\pgfqpoint{0.800000in}{0.528000in}}{\pgfqpoint{4.960000in}{3.696000in}}%
\pgfusepath{clip}%
\pgfsetbuttcap%
\pgfsetroundjoin%
\definecolor{currentfill}{rgb}{0.000000,0.000000,0.000000}%
\pgfsetfillcolor{currentfill}%
\pgfsetlinewidth{1.003750pt}%
\definecolor{currentstroke}{rgb}{0.000000,0.000000,0.000000}%
\pgfsetstrokecolor{currentstroke}%
\pgfsetdash{}{0pt}%
\pgfpathmoveto{\pgfqpoint{1.025906in}{0.684333in}}%
\pgfpathcurveto{\pgfqpoint{1.036956in}{0.684333in}}{\pgfqpoint{1.047555in}{0.688724in}}{\pgfqpoint{1.055369in}{0.696537in}}%
\pgfpathcurveto{\pgfqpoint{1.063182in}{0.704351in}}{\pgfqpoint{1.067573in}{0.714950in}}{\pgfqpoint{1.067573in}{0.726000in}}%
\pgfpathcurveto{\pgfqpoint{1.067573in}{0.737050in}}{\pgfqpoint{1.063182in}{0.747649in}}{\pgfqpoint{1.055369in}{0.755463in}}%
\pgfpathcurveto{\pgfqpoint{1.047555in}{0.763276in}}{\pgfqpoint{1.036956in}{0.767667in}}{\pgfqpoint{1.025906in}{0.767667in}}%
\pgfpathcurveto{\pgfqpoint{1.014856in}{0.767667in}}{\pgfqpoint{1.004257in}{0.763276in}}{\pgfqpoint{0.996443in}{0.755463in}}%
\pgfpathcurveto{\pgfqpoint{0.988630in}{0.747649in}}{\pgfqpoint{0.984239in}{0.737050in}}{\pgfqpoint{0.984239in}{0.726000in}}%
\pgfpathcurveto{\pgfqpoint{0.984239in}{0.714950in}}{\pgfqpoint{0.988630in}{0.704351in}}{\pgfqpoint{0.996443in}{0.696537in}}%
\pgfpathcurveto{\pgfqpoint{1.004257in}{0.688724in}}{\pgfqpoint{1.014856in}{0.684333in}}{\pgfqpoint{1.025906in}{0.684333in}}%
\pgfpathclose%
\pgfusepath{stroke,fill}%
\end{pgfscope}%
\begin{pgfscope}%
\pgfpathrectangle{\pgfqpoint{0.800000in}{0.528000in}}{\pgfqpoint{4.960000in}{3.696000in}}%
\pgfusepath{clip}%
\pgfsetbuttcap%
\pgfsetroundjoin%
\definecolor{currentfill}{rgb}{0.000000,0.000000,0.000000}%
\pgfsetfillcolor{currentfill}%
\pgfsetlinewidth{1.003750pt}%
\definecolor{currentstroke}{rgb}{0.000000,0.000000,0.000000}%
\pgfsetstrokecolor{currentstroke}%
\pgfsetdash{}{0pt}%
\pgfpathmoveto{\pgfqpoint{1.025906in}{0.684333in}}%
\pgfpathcurveto{\pgfqpoint{1.036956in}{0.684333in}}{\pgfqpoint{1.047555in}{0.688724in}}{\pgfqpoint{1.055369in}{0.696537in}}%
\pgfpathcurveto{\pgfqpoint{1.063182in}{0.704351in}}{\pgfqpoint{1.067573in}{0.714950in}}{\pgfqpoint{1.067573in}{0.726000in}}%
\pgfpathcurveto{\pgfqpoint{1.067573in}{0.737050in}}{\pgfqpoint{1.063182in}{0.747649in}}{\pgfqpoint{1.055369in}{0.755463in}}%
\pgfpathcurveto{\pgfqpoint{1.047555in}{0.763276in}}{\pgfqpoint{1.036956in}{0.767667in}}{\pgfqpoint{1.025906in}{0.767667in}}%
\pgfpathcurveto{\pgfqpoint{1.014856in}{0.767667in}}{\pgfqpoint{1.004257in}{0.763276in}}{\pgfqpoint{0.996443in}{0.755463in}}%
\pgfpathcurveto{\pgfqpoint{0.988630in}{0.747649in}}{\pgfqpoint{0.984239in}{0.737050in}}{\pgfqpoint{0.984239in}{0.726000in}}%
\pgfpathcurveto{\pgfqpoint{0.984239in}{0.714950in}}{\pgfqpoint{0.988630in}{0.704351in}}{\pgfqpoint{0.996443in}{0.696537in}}%
\pgfpathcurveto{\pgfqpoint{1.004257in}{0.688724in}}{\pgfqpoint{1.014856in}{0.684333in}}{\pgfqpoint{1.025906in}{0.684333in}}%
\pgfpathclose%
\pgfusepath{stroke,fill}%
\end{pgfscope}%
\begin{pgfscope}%
\pgfpathrectangle{\pgfqpoint{0.800000in}{0.528000in}}{\pgfqpoint{4.960000in}{3.696000in}}%
\pgfusepath{clip}%
\pgfsetbuttcap%
\pgfsetroundjoin%
\definecolor{currentfill}{rgb}{0.000000,0.000000,0.000000}%
\pgfsetfillcolor{currentfill}%
\pgfsetlinewidth{1.003750pt}%
\definecolor{currentstroke}{rgb}{0.000000,0.000000,0.000000}%
\pgfsetstrokecolor{currentstroke}%
\pgfsetdash{}{0pt}%
\pgfpathmoveto{\pgfqpoint{1.025906in}{0.684333in}}%
\pgfpathcurveto{\pgfqpoint{1.036956in}{0.684333in}}{\pgfqpoint{1.047555in}{0.688724in}}{\pgfqpoint{1.055369in}{0.696537in}}%
\pgfpathcurveto{\pgfqpoint{1.063182in}{0.704351in}}{\pgfqpoint{1.067573in}{0.714950in}}{\pgfqpoint{1.067573in}{0.726000in}}%
\pgfpathcurveto{\pgfqpoint{1.067573in}{0.737050in}}{\pgfqpoint{1.063182in}{0.747649in}}{\pgfqpoint{1.055369in}{0.755463in}}%
\pgfpathcurveto{\pgfqpoint{1.047555in}{0.763276in}}{\pgfqpoint{1.036956in}{0.767667in}}{\pgfqpoint{1.025906in}{0.767667in}}%
\pgfpathcurveto{\pgfqpoint{1.014856in}{0.767667in}}{\pgfqpoint{1.004257in}{0.763276in}}{\pgfqpoint{0.996443in}{0.755463in}}%
\pgfpathcurveto{\pgfqpoint{0.988630in}{0.747649in}}{\pgfqpoint{0.984239in}{0.737050in}}{\pgfqpoint{0.984239in}{0.726000in}}%
\pgfpathcurveto{\pgfqpoint{0.984239in}{0.714950in}}{\pgfqpoint{0.988630in}{0.704351in}}{\pgfqpoint{0.996443in}{0.696537in}}%
\pgfpathcurveto{\pgfqpoint{1.004257in}{0.688724in}}{\pgfqpoint{1.014856in}{0.684333in}}{\pgfqpoint{1.025906in}{0.684333in}}%
\pgfpathclose%
\pgfusepath{stroke,fill}%
\end{pgfscope}%
\begin{pgfscope}%
\pgfpathrectangle{\pgfqpoint{0.800000in}{0.528000in}}{\pgfqpoint{4.960000in}{3.696000in}}%
\pgfusepath{clip}%
\pgfsetbuttcap%
\pgfsetroundjoin%
\definecolor{currentfill}{rgb}{0.000000,0.000000,0.000000}%
\pgfsetfillcolor{currentfill}%
\pgfsetlinewidth{1.003750pt}%
\definecolor{currentstroke}{rgb}{0.000000,0.000000,0.000000}%
\pgfsetstrokecolor{currentstroke}%
\pgfsetdash{}{0pt}%
\pgfpathmoveto{\pgfqpoint{1.025906in}{0.684333in}}%
\pgfpathcurveto{\pgfqpoint{1.036956in}{0.684333in}}{\pgfqpoint{1.047555in}{0.688724in}}{\pgfqpoint{1.055369in}{0.696537in}}%
\pgfpathcurveto{\pgfqpoint{1.063182in}{0.704351in}}{\pgfqpoint{1.067573in}{0.714950in}}{\pgfqpoint{1.067573in}{0.726000in}}%
\pgfpathcurveto{\pgfqpoint{1.067573in}{0.737050in}}{\pgfqpoint{1.063182in}{0.747649in}}{\pgfqpoint{1.055369in}{0.755463in}}%
\pgfpathcurveto{\pgfqpoint{1.047555in}{0.763276in}}{\pgfqpoint{1.036956in}{0.767667in}}{\pgfqpoint{1.025906in}{0.767667in}}%
\pgfpathcurveto{\pgfqpoint{1.014856in}{0.767667in}}{\pgfqpoint{1.004257in}{0.763276in}}{\pgfqpoint{0.996443in}{0.755463in}}%
\pgfpathcurveto{\pgfqpoint{0.988630in}{0.747649in}}{\pgfqpoint{0.984239in}{0.737050in}}{\pgfqpoint{0.984239in}{0.726000in}}%
\pgfpathcurveto{\pgfqpoint{0.984239in}{0.714950in}}{\pgfqpoint{0.988630in}{0.704351in}}{\pgfqpoint{0.996443in}{0.696537in}}%
\pgfpathcurveto{\pgfqpoint{1.004257in}{0.688724in}}{\pgfqpoint{1.014856in}{0.684333in}}{\pgfqpoint{1.025906in}{0.684333in}}%
\pgfpathclose%
\pgfusepath{stroke,fill}%
\end{pgfscope}%
\begin{pgfscope}%
\pgfpathrectangle{\pgfqpoint{0.800000in}{0.528000in}}{\pgfqpoint{4.960000in}{3.696000in}}%
\pgfusepath{clip}%
\pgfsetbuttcap%
\pgfsetroundjoin%
\definecolor{currentfill}{rgb}{0.000000,0.000000,0.000000}%
\pgfsetfillcolor{currentfill}%
\pgfsetlinewidth{1.003750pt}%
\definecolor{currentstroke}{rgb}{0.000000,0.000000,0.000000}%
\pgfsetstrokecolor{currentstroke}%
\pgfsetdash{}{0pt}%
\pgfpathmoveto{\pgfqpoint{1.025906in}{0.684333in}}%
\pgfpathcurveto{\pgfqpoint{1.036956in}{0.684333in}}{\pgfqpoint{1.047555in}{0.688724in}}{\pgfqpoint{1.055369in}{0.696537in}}%
\pgfpathcurveto{\pgfqpoint{1.063182in}{0.704351in}}{\pgfqpoint{1.067573in}{0.714950in}}{\pgfqpoint{1.067573in}{0.726000in}}%
\pgfpathcurveto{\pgfqpoint{1.067573in}{0.737050in}}{\pgfqpoint{1.063182in}{0.747649in}}{\pgfqpoint{1.055369in}{0.755463in}}%
\pgfpathcurveto{\pgfqpoint{1.047555in}{0.763276in}}{\pgfqpoint{1.036956in}{0.767667in}}{\pgfqpoint{1.025906in}{0.767667in}}%
\pgfpathcurveto{\pgfqpoint{1.014856in}{0.767667in}}{\pgfqpoint{1.004257in}{0.763276in}}{\pgfqpoint{0.996443in}{0.755463in}}%
\pgfpathcurveto{\pgfqpoint{0.988630in}{0.747649in}}{\pgfqpoint{0.984239in}{0.737050in}}{\pgfqpoint{0.984239in}{0.726000in}}%
\pgfpathcurveto{\pgfqpoint{0.984239in}{0.714950in}}{\pgfqpoint{0.988630in}{0.704351in}}{\pgfqpoint{0.996443in}{0.696537in}}%
\pgfpathcurveto{\pgfqpoint{1.004257in}{0.688724in}}{\pgfqpoint{1.014856in}{0.684333in}}{\pgfqpoint{1.025906in}{0.684333in}}%
\pgfpathclose%
\pgfusepath{stroke,fill}%
\end{pgfscope}%
\begin{pgfscope}%
\pgfpathrectangle{\pgfqpoint{0.800000in}{0.528000in}}{\pgfqpoint{4.960000in}{3.696000in}}%
\pgfusepath{clip}%
\pgfsetbuttcap%
\pgfsetroundjoin%
\definecolor{currentfill}{rgb}{0.000000,0.000000,0.000000}%
\pgfsetfillcolor{currentfill}%
\pgfsetlinewidth{1.003750pt}%
\definecolor{currentstroke}{rgb}{0.000000,0.000000,0.000000}%
\pgfsetstrokecolor{currentstroke}%
\pgfsetdash{}{0pt}%
\pgfpathmoveto{\pgfqpoint{1.025906in}{0.684333in}}%
\pgfpathcurveto{\pgfqpoint{1.036956in}{0.684333in}}{\pgfqpoint{1.047555in}{0.688724in}}{\pgfqpoint{1.055369in}{0.696537in}}%
\pgfpathcurveto{\pgfqpoint{1.063182in}{0.704351in}}{\pgfqpoint{1.067573in}{0.714950in}}{\pgfqpoint{1.067573in}{0.726000in}}%
\pgfpathcurveto{\pgfqpoint{1.067573in}{0.737050in}}{\pgfqpoint{1.063182in}{0.747649in}}{\pgfqpoint{1.055369in}{0.755463in}}%
\pgfpathcurveto{\pgfqpoint{1.047555in}{0.763276in}}{\pgfqpoint{1.036956in}{0.767667in}}{\pgfqpoint{1.025906in}{0.767667in}}%
\pgfpathcurveto{\pgfqpoint{1.014856in}{0.767667in}}{\pgfqpoint{1.004257in}{0.763276in}}{\pgfqpoint{0.996443in}{0.755463in}}%
\pgfpathcurveto{\pgfqpoint{0.988630in}{0.747649in}}{\pgfqpoint{0.984239in}{0.737050in}}{\pgfqpoint{0.984239in}{0.726000in}}%
\pgfpathcurveto{\pgfqpoint{0.984239in}{0.714950in}}{\pgfqpoint{0.988630in}{0.704351in}}{\pgfqpoint{0.996443in}{0.696537in}}%
\pgfpathcurveto{\pgfqpoint{1.004257in}{0.688724in}}{\pgfqpoint{1.014856in}{0.684333in}}{\pgfqpoint{1.025906in}{0.684333in}}%
\pgfpathclose%
\pgfusepath{stroke,fill}%
\end{pgfscope}%
\begin{pgfscope}%
\pgfpathrectangle{\pgfqpoint{0.800000in}{0.528000in}}{\pgfqpoint{4.960000in}{3.696000in}}%
\pgfusepath{clip}%
\pgfsetbuttcap%
\pgfsetroundjoin%
\definecolor{currentfill}{rgb}{0.000000,0.000000,0.000000}%
\pgfsetfillcolor{currentfill}%
\pgfsetlinewidth{1.003750pt}%
\definecolor{currentstroke}{rgb}{0.000000,0.000000,0.000000}%
\pgfsetstrokecolor{currentstroke}%
\pgfsetdash{}{0pt}%
\pgfpathmoveto{\pgfqpoint{1.025906in}{0.684333in}}%
\pgfpathcurveto{\pgfqpoint{1.036956in}{0.684333in}}{\pgfqpoint{1.047555in}{0.688724in}}{\pgfqpoint{1.055369in}{0.696537in}}%
\pgfpathcurveto{\pgfqpoint{1.063182in}{0.704351in}}{\pgfqpoint{1.067573in}{0.714950in}}{\pgfqpoint{1.067573in}{0.726000in}}%
\pgfpathcurveto{\pgfqpoint{1.067573in}{0.737050in}}{\pgfqpoint{1.063182in}{0.747649in}}{\pgfqpoint{1.055369in}{0.755463in}}%
\pgfpathcurveto{\pgfqpoint{1.047555in}{0.763276in}}{\pgfqpoint{1.036956in}{0.767667in}}{\pgfqpoint{1.025906in}{0.767667in}}%
\pgfpathcurveto{\pgfqpoint{1.014856in}{0.767667in}}{\pgfqpoint{1.004257in}{0.763276in}}{\pgfqpoint{0.996443in}{0.755463in}}%
\pgfpathcurveto{\pgfqpoint{0.988630in}{0.747649in}}{\pgfqpoint{0.984239in}{0.737050in}}{\pgfqpoint{0.984239in}{0.726000in}}%
\pgfpathcurveto{\pgfqpoint{0.984239in}{0.714950in}}{\pgfqpoint{0.988630in}{0.704351in}}{\pgfqpoint{0.996443in}{0.696537in}}%
\pgfpathcurveto{\pgfqpoint{1.004257in}{0.688724in}}{\pgfqpoint{1.014856in}{0.684333in}}{\pgfqpoint{1.025906in}{0.684333in}}%
\pgfpathclose%
\pgfusepath{stroke,fill}%
\end{pgfscope}%
\begin{pgfscope}%
\pgfpathrectangle{\pgfqpoint{0.800000in}{0.528000in}}{\pgfqpoint{4.960000in}{3.696000in}}%
\pgfusepath{clip}%
\pgfsetbuttcap%
\pgfsetroundjoin%
\definecolor{currentfill}{rgb}{0.000000,0.000000,0.000000}%
\pgfsetfillcolor{currentfill}%
\pgfsetlinewidth{1.003750pt}%
\definecolor{currentstroke}{rgb}{0.000000,0.000000,0.000000}%
\pgfsetstrokecolor{currentstroke}%
\pgfsetdash{}{0pt}%
\pgfpathmoveto{\pgfqpoint{1.025906in}{0.684333in}}%
\pgfpathcurveto{\pgfqpoint{1.036956in}{0.684333in}}{\pgfqpoint{1.047555in}{0.688724in}}{\pgfqpoint{1.055369in}{0.696537in}}%
\pgfpathcurveto{\pgfqpoint{1.063182in}{0.704351in}}{\pgfqpoint{1.067573in}{0.714950in}}{\pgfqpoint{1.067573in}{0.726000in}}%
\pgfpathcurveto{\pgfqpoint{1.067573in}{0.737050in}}{\pgfqpoint{1.063182in}{0.747649in}}{\pgfqpoint{1.055369in}{0.755463in}}%
\pgfpathcurveto{\pgfqpoint{1.047555in}{0.763276in}}{\pgfqpoint{1.036956in}{0.767667in}}{\pgfqpoint{1.025906in}{0.767667in}}%
\pgfpathcurveto{\pgfqpoint{1.014856in}{0.767667in}}{\pgfqpoint{1.004257in}{0.763276in}}{\pgfqpoint{0.996443in}{0.755463in}}%
\pgfpathcurveto{\pgfqpoint{0.988630in}{0.747649in}}{\pgfqpoint{0.984239in}{0.737050in}}{\pgfqpoint{0.984239in}{0.726000in}}%
\pgfpathcurveto{\pgfqpoint{0.984239in}{0.714950in}}{\pgfqpoint{0.988630in}{0.704351in}}{\pgfqpoint{0.996443in}{0.696537in}}%
\pgfpathcurveto{\pgfqpoint{1.004257in}{0.688724in}}{\pgfqpoint{1.014856in}{0.684333in}}{\pgfqpoint{1.025906in}{0.684333in}}%
\pgfpathclose%
\pgfusepath{stroke,fill}%
\end{pgfscope}%
\begin{pgfscope}%
\pgfpathrectangle{\pgfqpoint{0.800000in}{0.528000in}}{\pgfqpoint{4.960000in}{3.696000in}}%
\pgfusepath{clip}%
\pgfsetbuttcap%
\pgfsetroundjoin%
\definecolor{currentfill}{rgb}{0.000000,0.000000,0.000000}%
\pgfsetfillcolor{currentfill}%
\pgfsetlinewidth{1.003750pt}%
\definecolor{currentstroke}{rgb}{0.000000,0.000000,0.000000}%
\pgfsetstrokecolor{currentstroke}%
\pgfsetdash{}{0pt}%
\pgfpathmoveto{\pgfqpoint{1.025906in}{0.684333in}}%
\pgfpathcurveto{\pgfqpoint{1.036956in}{0.684333in}}{\pgfqpoint{1.047555in}{0.688724in}}{\pgfqpoint{1.055369in}{0.696537in}}%
\pgfpathcurveto{\pgfqpoint{1.063182in}{0.704351in}}{\pgfqpoint{1.067573in}{0.714950in}}{\pgfqpoint{1.067573in}{0.726000in}}%
\pgfpathcurveto{\pgfqpoint{1.067573in}{0.737050in}}{\pgfqpoint{1.063182in}{0.747649in}}{\pgfqpoint{1.055369in}{0.755463in}}%
\pgfpathcurveto{\pgfqpoint{1.047555in}{0.763276in}}{\pgfqpoint{1.036956in}{0.767667in}}{\pgfqpoint{1.025906in}{0.767667in}}%
\pgfpathcurveto{\pgfqpoint{1.014856in}{0.767667in}}{\pgfqpoint{1.004257in}{0.763276in}}{\pgfqpoint{0.996443in}{0.755463in}}%
\pgfpathcurveto{\pgfqpoint{0.988630in}{0.747649in}}{\pgfqpoint{0.984239in}{0.737050in}}{\pgfqpoint{0.984239in}{0.726000in}}%
\pgfpathcurveto{\pgfqpoint{0.984239in}{0.714950in}}{\pgfqpoint{0.988630in}{0.704351in}}{\pgfqpoint{0.996443in}{0.696537in}}%
\pgfpathcurveto{\pgfqpoint{1.004257in}{0.688724in}}{\pgfqpoint{1.014856in}{0.684333in}}{\pgfqpoint{1.025906in}{0.684333in}}%
\pgfpathclose%
\pgfusepath{stroke,fill}%
\end{pgfscope}%
\begin{pgfscope}%
\pgfpathrectangle{\pgfqpoint{0.800000in}{0.528000in}}{\pgfqpoint{4.960000in}{3.696000in}}%
\pgfusepath{clip}%
\pgfsetbuttcap%
\pgfsetroundjoin%
\definecolor{currentfill}{rgb}{0.000000,0.000000,0.000000}%
\pgfsetfillcolor{currentfill}%
\pgfsetlinewidth{1.003750pt}%
\definecolor{currentstroke}{rgb}{0.000000,0.000000,0.000000}%
\pgfsetstrokecolor{currentstroke}%
\pgfsetdash{}{0pt}%
\pgfpathmoveto{\pgfqpoint{1.025906in}{0.684333in}}%
\pgfpathcurveto{\pgfqpoint{1.036956in}{0.684333in}}{\pgfqpoint{1.047555in}{0.688724in}}{\pgfqpoint{1.055369in}{0.696537in}}%
\pgfpathcurveto{\pgfqpoint{1.063182in}{0.704351in}}{\pgfqpoint{1.067573in}{0.714950in}}{\pgfqpoint{1.067573in}{0.726000in}}%
\pgfpathcurveto{\pgfqpoint{1.067573in}{0.737050in}}{\pgfqpoint{1.063182in}{0.747649in}}{\pgfqpoint{1.055369in}{0.755463in}}%
\pgfpathcurveto{\pgfqpoint{1.047555in}{0.763276in}}{\pgfqpoint{1.036956in}{0.767667in}}{\pgfqpoint{1.025906in}{0.767667in}}%
\pgfpathcurveto{\pgfqpoint{1.014856in}{0.767667in}}{\pgfqpoint{1.004257in}{0.763276in}}{\pgfqpoint{0.996443in}{0.755463in}}%
\pgfpathcurveto{\pgfqpoint{0.988630in}{0.747649in}}{\pgfqpoint{0.984239in}{0.737050in}}{\pgfqpoint{0.984239in}{0.726000in}}%
\pgfpathcurveto{\pgfqpoint{0.984239in}{0.714950in}}{\pgfqpoint{0.988630in}{0.704351in}}{\pgfqpoint{0.996443in}{0.696537in}}%
\pgfpathcurveto{\pgfqpoint{1.004257in}{0.688724in}}{\pgfqpoint{1.014856in}{0.684333in}}{\pgfqpoint{1.025906in}{0.684333in}}%
\pgfpathclose%
\pgfusepath{stroke,fill}%
\end{pgfscope}%
\begin{pgfscope}%
\pgfpathrectangle{\pgfqpoint{0.800000in}{0.528000in}}{\pgfqpoint{4.960000in}{3.696000in}}%
\pgfusepath{clip}%
\pgfsetbuttcap%
\pgfsetroundjoin%
\definecolor{currentfill}{rgb}{0.000000,0.000000,0.000000}%
\pgfsetfillcolor{currentfill}%
\pgfsetlinewidth{1.003750pt}%
\definecolor{currentstroke}{rgb}{0.000000,0.000000,0.000000}%
\pgfsetstrokecolor{currentstroke}%
\pgfsetdash{}{0pt}%
\pgfpathmoveto{\pgfqpoint{1.025906in}{0.684333in}}%
\pgfpathcurveto{\pgfqpoint{1.036956in}{0.684333in}}{\pgfqpoint{1.047555in}{0.688724in}}{\pgfqpoint{1.055369in}{0.696537in}}%
\pgfpathcurveto{\pgfqpoint{1.063182in}{0.704351in}}{\pgfqpoint{1.067573in}{0.714950in}}{\pgfqpoint{1.067573in}{0.726000in}}%
\pgfpathcurveto{\pgfqpoint{1.067573in}{0.737050in}}{\pgfqpoint{1.063182in}{0.747649in}}{\pgfqpoint{1.055369in}{0.755463in}}%
\pgfpathcurveto{\pgfqpoint{1.047555in}{0.763276in}}{\pgfqpoint{1.036956in}{0.767667in}}{\pgfqpoint{1.025906in}{0.767667in}}%
\pgfpathcurveto{\pgfqpoint{1.014856in}{0.767667in}}{\pgfqpoint{1.004257in}{0.763276in}}{\pgfqpoint{0.996443in}{0.755463in}}%
\pgfpathcurveto{\pgfqpoint{0.988630in}{0.747649in}}{\pgfqpoint{0.984239in}{0.737050in}}{\pgfqpoint{0.984239in}{0.726000in}}%
\pgfpathcurveto{\pgfqpoint{0.984239in}{0.714950in}}{\pgfqpoint{0.988630in}{0.704351in}}{\pgfqpoint{0.996443in}{0.696537in}}%
\pgfpathcurveto{\pgfqpoint{1.004257in}{0.688724in}}{\pgfqpoint{1.014856in}{0.684333in}}{\pgfqpoint{1.025906in}{0.684333in}}%
\pgfpathclose%
\pgfusepath{stroke,fill}%
\end{pgfscope}%
\begin{pgfscope}%
\pgfpathrectangle{\pgfqpoint{0.800000in}{0.528000in}}{\pgfqpoint{4.960000in}{3.696000in}}%
\pgfusepath{clip}%
\pgfsetbuttcap%
\pgfsetroundjoin%
\definecolor{currentfill}{rgb}{0.000000,0.000000,0.000000}%
\pgfsetfillcolor{currentfill}%
\pgfsetlinewidth{1.003750pt}%
\definecolor{currentstroke}{rgb}{0.000000,0.000000,0.000000}%
\pgfsetstrokecolor{currentstroke}%
\pgfsetdash{}{0pt}%
\pgfpathmoveto{\pgfqpoint{2.518786in}{0.684333in}}%
\pgfpathcurveto{\pgfqpoint{2.529836in}{0.684333in}}{\pgfqpoint{2.540435in}{0.688724in}}{\pgfqpoint{2.548249in}{0.696537in}}%
\pgfpathcurveto{\pgfqpoint{2.556062in}{0.704351in}}{\pgfqpoint{2.560452in}{0.714950in}}{\pgfqpoint{2.560452in}{0.726000in}}%
\pgfpathcurveto{\pgfqpoint{2.560452in}{0.737050in}}{\pgfqpoint{2.556062in}{0.747649in}}{\pgfqpoint{2.548249in}{0.755463in}}%
\pgfpathcurveto{\pgfqpoint{2.540435in}{0.763276in}}{\pgfqpoint{2.529836in}{0.767667in}}{\pgfqpoint{2.518786in}{0.767667in}}%
\pgfpathcurveto{\pgfqpoint{2.507736in}{0.767667in}}{\pgfqpoint{2.497137in}{0.763276in}}{\pgfqpoint{2.489323in}{0.755463in}}%
\pgfpathcurveto{\pgfqpoint{2.481509in}{0.747649in}}{\pgfqpoint{2.477119in}{0.737050in}}{\pgfqpoint{2.477119in}{0.726000in}}%
\pgfpathcurveto{\pgfqpoint{2.477119in}{0.714950in}}{\pgfqpoint{2.481509in}{0.704351in}}{\pgfqpoint{2.489323in}{0.696537in}}%
\pgfpathcurveto{\pgfqpoint{2.497137in}{0.688724in}}{\pgfqpoint{2.507736in}{0.684333in}}{\pgfqpoint{2.518786in}{0.684333in}}%
\pgfpathclose%
\pgfusepath{stroke,fill}%
\end{pgfscope}%
\begin{pgfscope}%
\pgfpathrectangle{\pgfqpoint{0.800000in}{0.528000in}}{\pgfqpoint{4.960000in}{3.696000in}}%
\pgfusepath{clip}%
\pgfsetbuttcap%
\pgfsetroundjoin%
\definecolor{currentfill}{rgb}{0.000000,0.000000,0.000000}%
\pgfsetfillcolor{currentfill}%
\pgfsetlinewidth{1.003750pt}%
\definecolor{currentstroke}{rgb}{0.000000,0.000000,0.000000}%
\pgfsetstrokecolor{currentstroke}%
\pgfsetdash{}{0pt}%
\pgfpathmoveto{\pgfqpoint{2.518786in}{0.684333in}}%
\pgfpathcurveto{\pgfqpoint{2.529836in}{0.684333in}}{\pgfqpoint{2.540435in}{0.688724in}}{\pgfqpoint{2.548249in}{0.696537in}}%
\pgfpathcurveto{\pgfqpoint{2.556062in}{0.704351in}}{\pgfqpoint{2.560452in}{0.714950in}}{\pgfqpoint{2.560452in}{0.726000in}}%
\pgfpathcurveto{\pgfqpoint{2.560452in}{0.737050in}}{\pgfqpoint{2.556062in}{0.747649in}}{\pgfqpoint{2.548249in}{0.755463in}}%
\pgfpathcurveto{\pgfqpoint{2.540435in}{0.763276in}}{\pgfqpoint{2.529836in}{0.767667in}}{\pgfqpoint{2.518786in}{0.767667in}}%
\pgfpathcurveto{\pgfqpoint{2.507736in}{0.767667in}}{\pgfqpoint{2.497137in}{0.763276in}}{\pgfqpoint{2.489323in}{0.755463in}}%
\pgfpathcurveto{\pgfqpoint{2.481509in}{0.747649in}}{\pgfqpoint{2.477119in}{0.737050in}}{\pgfqpoint{2.477119in}{0.726000in}}%
\pgfpathcurveto{\pgfqpoint{2.477119in}{0.714950in}}{\pgfqpoint{2.481509in}{0.704351in}}{\pgfqpoint{2.489323in}{0.696537in}}%
\pgfpathcurveto{\pgfqpoint{2.497137in}{0.688724in}}{\pgfqpoint{2.507736in}{0.684333in}}{\pgfqpoint{2.518786in}{0.684333in}}%
\pgfpathclose%
\pgfusepath{stroke,fill}%
\end{pgfscope}%
\begin{pgfscope}%
\pgfpathrectangle{\pgfqpoint{0.800000in}{0.528000in}}{\pgfqpoint{4.960000in}{3.696000in}}%
\pgfusepath{clip}%
\pgfsetbuttcap%
\pgfsetroundjoin%
\definecolor{currentfill}{rgb}{0.000000,0.000000,0.000000}%
\pgfsetfillcolor{currentfill}%
\pgfsetlinewidth{1.003750pt}%
\definecolor{currentstroke}{rgb}{0.000000,0.000000,0.000000}%
\pgfsetstrokecolor{currentstroke}%
\pgfsetdash{}{0pt}%
\pgfpathmoveto{\pgfqpoint{2.518786in}{0.684333in}}%
\pgfpathcurveto{\pgfqpoint{2.529836in}{0.684333in}}{\pgfqpoint{2.540435in}{0.688724in}}{\pgfqpoint{2.548249in}{0.696537in}}%
\pgfpathcurveto{\pgfqpoint{2.556062in}{0.704351in}}{\pgfqpoint{2.560452in}{0.714950in}}{\pgfqpoint{2.560452in}{0.726000in}}%
\pgfpathcurveto{\pgfqpoint{2.560452in}{0.737050in}}{\pgfqpoint{2.556062in}{0.747649in}}{\pgfqpoint{2.548249in}{0.755463in}}%
\pgfpathcurveto{\pgfqpoint{2.540435in}{0.763276in}}{\pgfqpoint{2.529836in}{0.767667in}}{\pgfqpoint{2.518786in}{0.767667in}}%
\pgfpathcurveto{\pgfqpoint{2.507736in}{0.767667in}}{\pgfqpoint{2.497137in}{0.763276in}}{\pgfqpoint{2.489323in}{0.755463in}}%
\pgfpathcurveto{\pgfqpoint{2.481509in}{0.747649in}}{\pgfqpoint{2.477119in}{0.737050in}}{\pgfqpoint{2.477119in}{0.726000in}}%
\pgfpathcurveto{\pgfqpoint{2.477119in}{0.714950in}}{\pgfqpoint{2.481509in}{0.704351in}}{\pgfqpoint{2.489323in}{0.696537in}}%
\pgfpathcurveto{\pgfqpoint{2.497137in}{0.688724in}}{\pgfqpoint{2.507736in}{0.684333in}}{\pgfqpoint{2.518786in}{0.684333in}}%
\pgfpathclose%
\pgfusepath{stroke,fill}%
\end{pgfscope}%
\begin{pgfscope}%
\pgfpathrectangle{\pgfqpoint{0.800000in}{0.528000in}}{\pgfqpoint{4.960000in}{3.696000in}}%
\pgfusepath{clip}%
\pgfsetbuttcap%
\pgfsetroundjoin%
\definecolor{currentfill}{rgb}{0.000000,0.000000,0.000000}%
\pgfsetfillcolor{currentfill}%
\pgfsetlinewidth{1.003750pt}%
\definecolor{currentstroke}{rgb}{0.000000,0.000000,0.000000}%
\pgfsetstrokecolor{currentstroke}%
\pgfsetdash{}{0pt}%
\pgfpathmoveto{\pgfqpoint{2.518786in}{0.684333in}}%
\pgfpathcurveto{\pgfqpoint{2.529836in}{0.684333in}}{\pgfqpoint{2.540435in}{0.688724in}}{\pgfqpoint{2.548249in}{0.696537in}}%
\pgfpathcurveto{\pgfqpoint{2.556062in}{0.704351in}}{\pgfqpoint{2.560452in}{0.714950in}}{\pgfqpoint{2.560452in}{0.726000in}}%
\pgfpathcurveto{\pgfqpoint{2.560452in}{0.737050in}}{\pgfqpoint{2.556062in}{0.747649in}}{\pgfqpoint{2.548249in}{0.755463in}}%
\pgfpathcurveto{\pgfqpoint{2.540435in}{0.763276in}}{\pgfqpoint{2.529836in}{0.767667in}}{\pgfqpoint{2.518786in}{0.767667in}}%
\pgfpathcurveto{\pgfqpoint{2.507736in}{0.767667in}}{\pgfqpoint{2.497137in}{0.763276in}}{\pgfqpoint{2.489323in}{0.755463in}}%
\pgfpathcurveto{\pgfqpoint{2.481509in}{0.747649in}}{\pgfqpoint{2.477119in}{0.737050in}}{\pgfqpoint{2.477119in}{0.726000in}}%
\pgfpathcurveto{\pgfqpoint{2.477119in}{0.714950in}}{\pgfqpoint{2.481509in}{0.704351in}}{\pgfqpoint{2.489323in}{0.696537in}}%
\pgfpathcurveto{\pgfqpoint{2.497137in}{0.688724in}}{\pgfqpoint{2.507736in}{0.684333in}}{\pgfqpoint{2.518786in}{0.684333in}}%
\pgfpathclose%
\pgfusepath{stroke,fill}%
\end{pgfscope}%
\begin{pgfscope}%
\pgfpathrectangle{\pgfqpoint{0.800000in}{0.528000in}}{\pgfqpoint{4.960000in}{3.696000in}}%
\pgfusepath{clip}%
\pgfsetbuttcap%
\pgfsetroundjoin%
\definecolor{currentfill}{rgb}{0.000000,0.000000,0.000000}%
\pgfsetfillcolor{currentfill}%
\pgfsetlinewidth{1.003750pt}%
\definecolor{currentstroke}{rgb}{0.000000,0.000000,0.000000}%
\pgfsetstrokecolor{currentstroke}%
\pgfsetdash{}{0pt}%
\pgfpathmoveto{\pgfqpoint{2.518786in}{0.684333in}}%
\pgfpathcurveto{\pgfqpoint{2.529836in}{0.684333in}}{\pgfqpoint{2.540435in}{0.688724in}}{\pgfqpoint{2.548249in}{0.696537in}}%
\pgfpathcurveto{\pgfqpoint{2.556062in}{0.704351in}}{\pgfqpoint{2.560452in}{0.714950in}}{\pgfqpoint{2.560452in}{0.726000in}}%
\pgfpathcurveto{\pgfqpoint{2.560452in}{0.737050in}}{\pgfqpoint{2.556062in}{0.747649in}}{\pgfqpoint{2.548249in}{0.755463in}}%
\pgfpathcurveto{\pgfqpoint{2.540435in}{0.763276in}}{\pgfqpoint{2.529836in}{0.767667in}}{\pgfqpoint{2.518786in}{0.767667in}}%
\pgfpathcurveto{\pgfqpoint{2.507736in}{0.767667in}}{\pgfqpoint{2.497137in}{0.763276in}}{\pgfqpoint{2.489323in}{0.755463in}}%
\pgfpathcurveto{\pgfqpoint{2.481509in}{0.747649in}}{\pgfqpoint{2.477119in}{0.737050in}}{\pgfqpoint{2.477119in}{0.726000in}}%
\pgfpathcurveto{\pgfqpoint{2.477119in}{0.714950in}}{\pgfqpoint{2.481509in}{0.704351in}}{\pgfqpoint{2.489323in}{0.696537in}}%
\pgfpathcurveto{\pgfqpoint{2.497137in}{0.688724in}}{\pgfqpoint{2.507736in}{0.684333in}}{\pgfqpoint{2.518786in}{0.684333in}}%
\pgfpathclose%
\pgfusepath{stroke,fill}%
\end{pgfscope}%
\begin{pgfscope}%
\pgfpathrectangle{\pgfqpoint{0.800000in}{0.528000in}}{\pgfqpoint{4.960000in}{3.696000in}}%
\pgfusepath{clip}%
\pgfsetbuttcap%
\pgfsetroundjoin%
\definecolor{currentfill}{rgb}{0.000000,0.000000,0.000000}%
\pgfsetfillcolor{currentfill}%
\pgfsetlinewidth{1.003750pt}%
\definecolor{currentstroke}{rgb}{0.000000,0.000000,0.000000}%
\pgfsetstrokecolor{currentstroke}%
\pgfsetdash{}{0pt}%
\pgfpathmoveto{\pgfqpoint{2.518786in}{0.684333in}}%
\pgfpathcurveto{\pgfqpoint{2.529836in}{0.684333in}}{\pgfqpoint{2.540435in}{0.688724in}}{\pgfqpoint{2.548249in}{0.696537in}}%
\pgfpathcurveto{\pgfqpoint{2.556062in}{0.704351in}}{\pgfqpoint{2.560452in}{0.714950in}}{\pgfqpoint{2.560452in}{0.726000in}}%
\pgfpathcurveto{\pgfqpoint{2.560452in}{0.737050in}}{\pgfqpoint{2.556062in}{0.747649in}}{\pgfqpoint{2.548249in}{0.755463in}}%
\pgfpathcurveto{\pgfqpoint{2.540435in}{0.763276in}}{\pgfqpoint{2.529836in}{0.767667in}}{\pgfqpoint{2.518786in}{0.767667in}}%
\pgfpathcurveto{\pgfqpoint{2.507736in}{0.767667in}}{\pgfqpoint{2.497137in}{0.763276in}}{\pgfqpoint{2.489323in}{0.755463in}}%
\pgfpathcurveto{\pgfqpoint{2.481509in}{0.747649in}}{\pgfqpoint{2.477119in}{0.737050in}}{\pgfqpoint{2.477119in}{0.726000in}}%
\pgfpathcurveto{\pgfqpoint{2.477119in}{0.714950in}}{\pgfqpoint{2.481509in}{0.704351in}}{\pgfqpoint{2.489323in}{0.696537in}}%
\pgfpathcurveto{\pgfqpoint{2.497137in}{0.688724in}}{\pgfqpoint{2.507736in}{0.684333in}}{\pgfqpoint{2.518786in}{0.684333in}}%
\pgfpathclose%
\pgfusepath{stroke,fill}%
\end{pgfscope}%
\begin{pgfscope}%
\pgfpathrectangle{\pgfqpoint{0.800000in}{0.528000in}}{\pgfqpoint{4.960000in}{3.696000in}}%
\pgfusepath{clip}%
\pgfsetbuttcap%
\pgfsetroundjoin%
\definecolor{currentfill}{rgb}{0.000000,0.000000,0.000000}%
\pgfsetfillcolor{currentfill}%
\pgfsetlinewidth{1.003750pt}%
\definecolor{currentstroke}{rgb}{0.000000,0.000000,0.000000}%
\pgfsetstrokecolor{currentstroke}%
\pgfsetdash{}{0pt}%
\pgfpathmoveto{\pgfqpoint{2.518786in}{0.684333in}}%
\pgfpathcurveto{\pgfqpoint{2.529836in}{0.684333in}}{\pgfqpoint{2.540435in}{0.688724in}}{\pgfqpoint{2.548249in}{0.696537in}}%
\pgfpathcurveto{\pgfqpoint{2.556062in}{0.704351in}}{\pgfqpoint{2.560452in}{0.714950in}}{\pgfqpoint{2.560452in}{0.726000in}}%
\pgfpathcurveto{\pgfqpoint{2.560452in}{0.737050in}}{\pgfqpoint{2.556062in}{0.747649in}}{\pgfqpoint{2.548249in}{0.755463in}}%
\pgfpathcurveto{\pgfqpoint{2.540435in}{0.763276in}}{\pgfqpoint{2.529836in}{0.767667in}}{\pgfqpoint{2.518786in}{0.767667in}}%
\pgfpathcurveto{\pgfqpoint{2.507736in}{0.767667in}}{\pgfqpoint{2.497137in}{0.763276in}}{\pgfqpoint{2.489323in}{0.755463in}}%
\pgfpathcurveto{\pgfqpoint{2.481509in}{0.747649in}}{\pgfqpoint{2.477119in}{0.737050in}}{\pgfqpoint{2.477119in}{0.726000in}}%
\pgfpathcurveto{\pgfqpoint{2.477119in}{0.714950in}}{\pgfqpoint{2.481509in}{0.704351in}}{\pgfqpoint{2.489323in}{0.696537in}}%
\pgfpathcurveto{\pgfqpoint{2.497137in}{0.688724in}}{\pgfqpoint{2.507736in}{0.684333in}}{\pgfqpoint{2.518786in}{0.684333in}}%
\pgfpathclose%
\pgfusepath{stroke,fill}%
\end{pgfscope}%
\begin{pgfscope}%
\pgfpathrectangle{\pgfqpoint{0.800000in}{0.528000in}}{\pgfqpoint{4.960000in}{3.696000in}}%
\pgfusepath{clip}%
\pgfsetbuttcap%
\pgfsetroundjoin%
\definecolor{currentfill}{rgb}{0.000000,0.000000,0.000000}%
\pgfsetfillcolor{currentfill}%
\pgfsetlinewidth{1.003750pt}%
\definecolor{currentstroke}{rgb}{0.000000,0.000000,0.000000}%
\pgfsetstrokecolor{currentstroke}%
\pgfsetdash{}{0pt}%
\pgfpathmoveto{\pgfqpoint{2.518786in}{0.684333in}}%
\pgfpathcurveto{\pgfqpoint{2.529836in}{0.684333in}}{\pgfqpoint{2.540435in}{0.688724in}}{\pgfqpoint{2.548249in}{0.696537in}}%
\pgfpathcurveto{\pgfqpoint{2.556062in}{0.704351in}}{\pgfqpoint{2.560452in}{0.714950in}}{\pgfqpoint{2.560452in}{0.726000in}}%
\pgfpathcurveto{\pgfqpoint{2.560452in}{0.737050in}}{\pgfqpoint{2.556062in}{0.747649in}}{\pgfqpoint{2.548249in}{0.755463in}}%
\pgfpathcurveto{\pgfqpoint{2.540435in}{0.763276in}}{\pgfqpoint{2.529836in}{0.767667in}}{\pgfqpoint{2.518786in}{0.767667in}}%
\pgfpathcurveto{\pgfqpoint{2.507736in}{0.767667in}}{\pgfqpoint{2.497137in}{0.763276in}}{\pgfqpoint{2.489323in}{0.755463in}}%
\pgfpathcurveto{\pgfqpoint{2.481509in}{0.747649in}}{\pgfqpoint{2.477119in}{0.737050in}}{\pgfqpoint{2.477119in}{0.726000in}}%
\pgfpathcurveto{\pgfqpoint{2.477119in}{0.714950in}}{\pgfqpoint{2.481509in}{0.704351in}}{\pgfqpoint{2.489323in}{0.696537in}}%
\pgfpathcurveto{\pgfqpoint{2.497137in}{0.688724in}}{\pgfqpoint{2.507736in}{0.684333in}}{\pgfqpoint{2.518786in}{0.684333in}}%
\pgfpathclose%
\pgfusepath{stroke,fill}%
\end{pgfscope}%
\begin{pgfscope}%
\pgfpathrectangle{\pgfqpoint{0.800000in}{0.528000in}}{\pgfqpoint{4.960000in}{3.696000in}}%
\pgfusepath{clip}%
\pgfsetbuttcap%
\pgfsetroundjoin%
\definecolor{currentfill}{rgb}{0.000000,0.000000,0.000000}%
\pgfsetfillcolor{currentfill}%
\pgfsetlinewidth{1.003750pt}%
\definecolor{currentstroke}{rgb}{0.000000,0.000000,0.000000}%
\pgfsetstrokecolor{currentstroke}%
\pgfsetdash{}{0pt}%
\pgfpathmoveto{\pgfqpoint{2.518786in}{0.684333in}}%
\pgfpathcurveto{\pgfqpoint{2.529836in}{0.684333in}}{\pgfqpoint{2.540435in}{0.688724in}}{\pgfqpoint{2.548249in}{0.696537in}}%
\pgfpathcurveto{\pgfqpoint{2.556062in}{0.704351in}}{\pgfqpoint{2.560452in}{0.714950in}}{\pgfqpoint{2.560452in}{0.726000in}}%
\pgfpathcurveto{\pgfqpoint{2.560452in}{0.737050in}}{\pgfqpoint{2.556062in}{0.747649in}}{\pgfqpoint{2.548249in}{0.755463in}}%
\pgfpathcurveto{\pgfqpoint{2.540435in}{0.763276in}}{\pgfqpoint{2.529836in}{0.767667in}}{\pgfqpoint{2.518786in}{0.767667in}}%
\pgfpathcurveto{\pgfqpoint{2.507736in}{0.767667in}}{\pgfqpoint{2.497137in}{0.763276in}}{\pgfqpoint{2.489323in}{0.755463in}}%
\pgfpathcurveto{\pgfqpoint{2.481509in}{0.747649in}}{\pgfqpoint{2.477119in}{0.737050in}}{\pgfqpoint{2.477119in}{0.726000in}}%
\pgfpathcurveto{\pgfqpoint{2.477119in}{0.714950in}}{\pgfqpoint{2.481509in}{0.704351in}}{\pgfqpoint{2.489323in}{0.696537in}}%
\pgfpathcurveto{\pgfqpoint{2.497137in}{0.688724in}}{\pgfqpoint{2.507736in}{0.684333in}}{\pgfqpoint{2.518786in}{0.684333in}}%
\pgfpathclose%
\pgfusepath{stroke,fill}%
\end{pgfscope}%
\begin{pgfscope}%
\pgfpathrectangle{\pgfqpoint{0.800000in}{0.528000in}}{\pgfqpoint{4.960000in}{3.696000in}}%
\pgfusepath{clip}%
\pgfsetbuttcap%
\pgfsetroundjoin%
\definecolor{currentfill}{rgb}{0.000000,0.000000,0.000000}%
\pgfsetfillcolor{currentfill}%
\pgfsetlinewidth{1.003750pt}%
\definecolor{currentstroke}{rgb}{0.000000,0.000000,0.000000}%
\pgfsetstrokecolor{currentstroke}%
\pgfsetdash{}{0pt}%
\pgfpathmoveto{\pgfqpoint{2.518786in}{0.684333in}}%
\pgfpathcurveto{\pgfqpoint{2.529836in}{0.684333in}}{\pgfqpoint{2.540435in}{0.688724in}}{\pgfqpoint{2.548249in}{0.696537in}}%
\pgfpathcurveto{\pgfqpoint{2.556062in}{0.704351in}}{\pgfqpoint{2.560452in}{0.714950in}}{\pgfqpoint{2.560452in}{0.726000in}}%
\pgfpathcurveto{\pgfqpoint{2.560452in}{0.737050in}}{\pgfqpoint{2.556062in}{0.747649in}}{\pgfqpoint{2.548249in}{0.755463in}}%
\pgfpathcurveto{\pgfqpoint{2.540435in}{0.763276in}}{\pgfqpoint{2.529836in}{0.767667in}}{\pgfqpoint{2.518786in}{0.767667in}}%
\pgfpathcurveto{\pgfqpoint{2.507736in}{0.767667in}}{\pgfqpoint{2.497137in}{0.763276in}}{\pgfqpoint{2.489323in}{0.755463in}}%
\pgfpathcurveto{\pgfqpoint{2.481509in}{0.747649in}}{\pgfqpoint{2.477119in}{0.737050in}}{\pgfqpoint{2.477119in}{0.726000in}}%
\pgfpathcurveto{\pgfqpoint{2.477119in}{0.714950in}}{\pgfqpoint{2.481509in}{0.704351in}}{\pgfqpoint{2.489323in}{0.696537in}}%
\pgfpathcurveto{\pgfqpoint{2.497137in}{0.688724in}}{\pgfqpoint{2.507736in}{0.684333in}}{\pgfqpoint{2.518786in}{0.684333in}}%
\pgfpathclose%
\pgfusepath{stroke,fill}%
\end{pgfscope}%
\begin{pgfscope}%
\pgfpathrectangle{\pgfqpoint{0.800000in}{0.528000in}}{\pgfqpoint{4.960000in}{3.696000in}}%
\pgfusepath{clip}%
\pgfsetbuttcap%
\pgfsetroundjoin%
\definecolor{currentfill}{rgb}{0.000000,0.000000,0.000000}%
\pgfsetfillcolor{currentfill}%
\pgfsetlinewidth{1.003750pt}%
\definecolor{currentstroke}{rgb}{0.000000,0.000000,0.000000}%
\pgfsetstrokecolor{currentstroke}%
\pgfsetdash{}{0pt}%
\pgfpathmoveto{\pgfqpoint{2.518786in}{0.684333in}}%
\pgfpathcurveto{\pgfqpoint{2.529836in}{0.684333in}}{\pgfqpoint{2.540435in}{0.688724in}}{\pgfqpoint{2.548249in}{0.696537in}}%
\pgfpathcurveto{\pgfqpoint{2.556062in}{0.704351in}}{\pgfqpoint{2.560452in}{0.714950in}}{\pgfqpoint{2.560452in}{0.726000in}}%
\pgfpathcurveto{\pgfqpoint{2.560452in}{0.737050in}}{\pgfqpoint{2.556062in}{0.747649in}}{\pgfqpoint{2.548249in}{0.755463in}}%
\pgfpathcurveto{\pgfqpoint{2.540435in}{0.763276in}}{\pgfqpoint{2.529836in}{0.767667in}}{\pgfqpoint{2.518786in}{0.767667in}}%
\pgfpathcurveto{\pgfqpoint{2.507736in}{0.767667in}}{\pgfqpoint{2.497137in}{0.763276in}}{\pgfqpoint{2.489323in}{0.755463in}}%
\pgfpathcurveto{\pgfqpoint{2.481509in}{0.747649in}}{\pgfqpoint{2.477119in}{0.737050in}}{\pgfqpoint{2.477119in}{0.726000in}}%
\pgfpathcurveto{\pgfqpoint{2.477119in}{0.714950in}}{\pgfqpoint{2.481509in}{0.704351in}}{\pgfqpoint{2.489323in}{0.696537in}}%
\pgfpathcurveto{\pgfqpoint{2.497137in}{0.688724in}}{\pgfqpoint{2.507736in}{0.684333in}}{\pgfqpoint{2.518786in}{0.684333in}}%
\pgfpathclose%
\pgfusepath{stroke,fill}%
\end{pgfscope}%
\begin{pgfscope}%
\pgfpathrectangle{\pgfqpoint{0.800000in}{0.528000in}}{\pgfqpoint{4.960000in}{3.696000in}}%
\pgfusepath{clip}%
\pgfsetbuttcap%
\pgfsetroundjoin%
\definecolor{currentfill}{rgb}{0.000000,0.000000,0.000000}%
\pgfsetfillcolor{currentfill}%
\pgfsetlinewidth{1.003750pt}%
\definecolor{currentstroke}{rgb}{0.000000,0.000000,0.000000}%
\pgfsetstrokecolor{currentstroke}%
\pgfsetdash{}{0pt}%
\pgfpathmoveto{\pgfqpoint{2.518786in}{0.684333in}}%
\pgfpathcurveto{\pgfqpoint{2.529836in}{0.684333in}}{\pgfqpoint{2.540435in}{0.688724in}}{\pgfqpoint{2.548249in}{0.696537in}}%
\pgfpathcurveto{\pgfqpoint{2.556062in}{0.704351in}}{\pgfqpoint{2.560452in}{0.714950in}}{\pgfqpoint{2.560452in}{0.726000in}}%
\pgfpathcurveto{\pgfqpoint{2.560452in}{0.737050in}}{\pgfqpoint{2.556062in}{0.747649in}}{\pgfqpoint{2.548249in}{0.755463in}}%
\pgfpathcurveto{\pgfqpoint{2.540435in}{0.763276in}}{\pgfqpoint{2.529836in}{0.767667in}}{\pgfqpoint{2.518786in}{0.767667in}}%
\pgfpathcurveto{\pgfqpoint{2.507736in}{0.767667in}}{\pgfqpoint{2.497137in}{0.763276in}}{\pgfqpoint{2.489323in}{0.755463in}}%
\pgfpathcurveto{\pgfqpoint{2.481509in}{0.747649in}}{\pgfqpoint{2.477119in}{0.737050in}}{\pgfqpoint{2.477119in}{0.726000in}}%
\pgfpathcurveto{\pgfqpoint{2.477119in}{0.714950in}}{\pgfqpoint{2.481509in}{0.704351in}}{\pgfqpoint{2.489323in}{0.696537in}}%
\pgfpathcurveto{\pgfqpoint{2.497137in}{0.688724in}}{\pgfqpoint{2.507736in}{0.684333in}}{\pgfqpoint{2.518786in}{0.684333in}}%
\pgfpathclose%
\pgfusepath{stroke,fill}%
\end{pgfscope}%
\begin{pgfscope}%
\pgfpathrectangle{\pgfqpoint{0.800000in}{0.528000in}}{\pgfqpoint{4.960000in}{3.696000in}}%
\pgfusepath{clip}%
\pgfsetbuttcap%
\pgfsetroundjoin%
\definecolor{currentfill}{rgb}{0.000000,0.000000,0.000000}%
\pgfsetfillcolor{currentfill}%
\pgfsetlinewidth{1.003750pt}%
\definecolor{currentstroke}{rgb}{0.000000,0.000000,0.000000}%
\pgfsetstrokecolor{currentstroke}%
\pgfsetdash{}{0pt}%
\pgfpathmoveto{\pgfqpoint{2.518786in}{0.684333in}}%
\pgfpathcurveto{\pgfqpoint{2.529836in}{0.684333in}}{\pgfqpoint{2.540435in}{0.688724in}}{\pgfqpoint{2.548249in}{0.696537in}}%
\pgfpathcurveto{\pgfqpoint{2.556062in}{0.704351in}}{\pgfqpoint{2.560452in}{0.714950in}}{\pgfqpoint{2.560452in}{0.726000in}}%
\pgfpathcurveto{\pgfqpoint{2.560452in}{0.737050in}}{\pgfqpoint{2.556062in}{0.747649in}}{\pgfqpoint{2.548249in}{0.755463in}}%
\pgfpathcurveto{\pgfqpoint{2.540435in}{0.763276in}}{\pgfqpoint{2.529836in}{0.767667in}}{\pgfqpoint{2.518786in}{0.767667in}}%
\pgfpathcurveto{\pgfqpoint{2.507736in}{0.767667in}}{\pgfqpoint{2.497137in}{0.763276in}}{\pgfqpoint{2.489323in}{0.755463in}}%
\pgfpathcurveto{\pgfqpoint{2.481509in}{0.747649in}}{\pgfqpoint{2.477119in}{0.737050in}}{\pgfqpoint{2.477119in}{0.726000in}}%
\pgfpathcurveto{\pgfqpoint{2.477119in}{0.714950in}}{\pgfqpoint{2.481509in}{0.704351in}}{\pgfqpoint{2.489323in}{0.696537in}}%
\pgfpathcurveto{\pgfqpoint{2.497137in}{0.688724in}}{\pgfqpoint{2.507736in}{0.684333in}}{\pgfqpoint{2.518786in}{0.684333in}}%
\pgfpathclose%
\pgfusepath{stroke,fill}%
\end{pgfscope}%
\begin{pgfscope}%
\pgfpathrectangle{\pgfqpoint{0.800000in}{0.528000in}}{\pgfqpoint{4.960000in}{3.696000in}}%
\pgfusepath{clip}%
\pgfsetbuttcap%
\pgfsetroundjoin%
\definecolor{currentfill}{rgb}{0.000000,0.000000,0.000000}%
\pgfsetfillcolor{currentfill}%
\pgfsetlinewidth{1.003750pt}%
\definecolor{currentstroke}{rgb}{0.000000,0.000000,0.000000}%
\pgfsetstrokecolor{currentstroke}%
\pgfsetdash{}{0pt}%
\pgfpathmoveto{\pgfqpoint{2.518786in}{0.684333in}}%
\pgfpathcurveto{\pgfqpoint{2.529836in}{0.684333in}}{\pgfqpoint{2.540435in}{0.688724in}}{\pgfqpoint{2.548249in}{0.696537in}}%
\pgfpathcurveto{\pgfqpoint{2.556062in}{0.704351in}}{\pgfqpoint{2.560452in}{0.714950in}}{\pgfqpoint{2.560452in}{0.726000in}}%
\pgfpathcurveto{\pgfqpoint{2.560452in}{0.737050in}}{\pgfqpoint{2.556062in}{0.747649in}}{\pgfqpoint{2.548249in}{0.755463in}}%
\pgfpathcurveto{\pgfqpoint{2.540435in}{0.763276in}}{\pgfqpoint{2.529836in}{0.767667in}}{\pgfqpoint{2.518786in}{0.767667in}}%
\pgfpathcurveto{\pgfqpoint{2.507736in}{0.767667in}}{\pgfqpoint{2.497137in}{0.763276in}}{\pgfqpoint{2.489323in}{0.755463in}}%
\pgfpathcurveto{\pgfqpoint{2.481509in}{0.747649in}}{\pgfqpoint{2.477119in}{0.737050in}}{\pgfqpoint{2.477119in}{0.726000in}}%
\pgfpathcurveto{\pgfqpoint{2.477119in}{0.714950in}}{\pgfqpoint{2.481509in}{0.704351in}}{\pgfqpoint{2.489323in}{0.696537in}}%
\pgfpathcurveto{\pgfqpoint{2.497137in}{0.688724in}}{\pgfqpoint{2.507736in}{0.684333in}}{\pgfqpoint{2.518786in}{0.684333in}}%
\pgfpathclose%
\pgfusepath{stroke,fill}%
\end{pgfscope}%
\begin{pgfscope}%
\pgfpathrectangle{\pgfqpoint{0.800000in}{0.528000in}}{\pgfqpoint{4.960000in}{3.696000in}}%
\pgfusepath{clip}%
\pgfsetbuttcap%
\pgfsetroundjoin%
\definecolor{currentfill}{rgb}{0.000000,0.000000,0.000000}%
\pgfsetfillcolor{currentfill}%
\pgfsetlinewidth{1.003750pt}%
\definecolor{currentstroke}{rgb}{0.000000,0.000000,0.000000}%
\pgfsetstrokecolor{currentstroke}%
\pgfsetdash{}{0pt}%
\pgfpathmoveto{\pgfqpoint{2.518786in}{0.684333in}}%
\pgfpathcurveto{\pgfqpoint{2.529836in}{0.684333in}}{\pgfqpoint{2.540435in}{0.688724in}}{\pgfqpoint{2.548249in}{0.696537in}}%
\pgfpathcurveto{\pgfqpoint{2.556062in}{0.704351in}}{\pgfqpoint{2.560452in}{0.714950in}}{\pgfqpoint{2.560452in}{0.726000in}}%
\pgfpathcurveto{\pgfqpoint{2.560452in}{0.737050in}}{\pgfqpoint{2.556062in}{0.747649in}}{\pgfqpoint{2.548249in}{0.755463in}}%
\pgfpathcurveto{\pgfqpoint{2.540435in}{0.763276in}}{\pgfqpoint{2.529836in}{0.767667in}}{\pgfqpoint{2.518786in}{0.767667in}}%
\pgfpathcurveto{\pgfqpoint{2.507736in}{0.767667in}}{\pgfqpoint{2.497137in}{0.763276in}}{\pgfqpoint{2.489323in}{0.755463in}}%
\pgfpathcurveto{\pgfqpoint{2.481509in}{0.747649in}}{\pgfqpoint{2.477119in}{0.737050in}}{\pgfqpoint{2.477119in}{0.726000in}}%
\pgfpathcurveto{\pgfqpoint{2.477119in}{0.714950in}}{\pgfqpoint{2.481509in}{0.704351in}}{\pgfqpoint{2.489323in}{0.696537in}}%
\pgfpathcurveto{\pgfqpoint{2.497137in}{0.688724in}}{\pgfqpoint{2.507736in}{0.684333in}}{\pgfqpoint{2.518786in}{0.684333in}}%
\pgfpathclose%
\pgfusepath{stroke,fill}%
\end{pgfscope}%
\begin{pgfscope}%
\pgfpathrectangle{\pgfqpoint{0.800000in}{0.528000in}}{\pgfqpoint{4.960000in}{3.696000in}}%
\pgfusepath{clip}%
\pgfsetbuttcap%
\pgfsetroundjoin%
\definecolor{currentfill}{rgb}{0.000000,0.000000,0.000000}%
\pgfsetfillcolor{currentfill}%
\pgfsetlinewidth{1.003750pt}%
\definecolor{currentstroke}{rgb}{0.000000,0.000000,0.000000}%
\pgfsetstrokecolor{currentstroke}%
\pgfsetdash{}{0pt}%
\pgfpathmoveto{\pgfqpoint{2.518786in}{0.684333in}}%
\pgfpathcurveto{\pgfqpoint{2.529836in}{0.684333in}}{\pgfqpoint{2.540435in}{0.688724in}}{\pgfqpoint{2.548249in}{0.696537in}}%
\pgfpathcurveto{\pgfqpoint{2.556062in}{0.704351in}}{\pgfqpoint{2.560452in}{0.714950in}}{\pgfqpoint{2.560452in}{0.726000in}}%
\pgfpathcurveto{\pgfqpoint{2.560452in}{0.737050in}}{\pgfqpoint{2.556062in}{0.747649in}}{\pgfqpoint{2.548249in}{0.755463in}}%
\pgfpathcurveto{\pgfqpoint{2.540435in}{0.763276in}}{\pgfqpoint{2.529836in}{0.767667in}}{\pgfqpoint{2.518786in}{0.767667in}}%
\pgfpathcurveto{\pgfqpoint{2.507736in}{0.767667in}}{\pgfqpoint{2.497137in}{0.763276in}}{\pgfqpoint{2.489323in}{0.755463in}}%
\pgfpathcurveto{\pgfqpoint{2.481509in}{0.747649in}}{\pgfqpoint{2.477119in}{0.737050in}}{\pgfqpoint{2.477119in}{0.726000in}}%
\pgfpathcurveto{\pgfqpoint{2.477119in}{0.714950in}}{\pgfqpoint{2.481509in}{0.704351in}}{\pgfqpoint{2.489323in}{0.696537in}}%
\pgfpathcurveto{\pgfqpoint{2.497137in}{0.688724in}}{\pgfqpoint{2.507736in}{0.684333in}}{\pgfqpoint{2.518786in}{0.684333in}}%
\pgfpathclose%
\pgfusepath{stroke,fill}%
\end{pgfscope}%
\begin{pgfscope}%
\pgfpathrectangle{\pgfqpoint{0.800000in}{0.528000in}}{\pgfqpoint{4.960000in}{3.696000in}}%
\pgfusepath{clip}%
\pgfsetbuttcap%
\pgfsetroundjoin%
\definecolor{currentfill}{rgb}{0.000000,0.000000,0.000000}%
\pgfsetfillcolor{currentfill}%
\pgfsetlinewidth{1.003750pt}%
\definecolor{currentstroke}{rgb}{0.000000,0.000000,0.000000}%
\pgfsetstrokecolor{currentstroke}%
\pgfsetdash{}{0pt}%
\pgfpathmoveto{\pgfqpoint{2.518786in}{0.684333in}}%
\pgfpathcurveto{\pgfqpoint{2.529836in}{0.684333in}}{\pgfqpoint{2.540435in}{0.688724in}}{\pgfqpoint{2.548249in}{0.696537in}}%
\pgfpathcurveto{\pgfqpoint{2.556062in}{0.704351in}}{\pgfqpoint{2.560452in}{0.714950in}}{\pgfqpoint{2.560452in}{0.726000in}}%
\pgfpathcurveto{\pgfqpoint{2.560452in}{0.737050in}}{\pgfqpoint{2.556062in}{0.747649in}}{\pgfqpoint{2.548249in}{0.755463in}}%
\pgfpathcurveto{\pgfqpoint{2.540435in}{0.763276in}}{\pgfqpoint{2.529836in}{0.767667in}}{\pgfqpoint{2.518786in}{0.767667in}}%
\pgfpathcurveto{\pgfqpoint{2.507736in}{0.767667in}}{\pgfqpoint{2.497137in}{0.763276in}}{\pgfqpoint{2.489323in}{0.755463in}}%
\pgfpathcurveto{\pgfqpoint{2.481509in}{0.747649in}}{\pgfqpoint{2.477119in}{0.737050in}}{\pgfqpoint{2.477119in}{0.726000in}}%
\pgfpathcurveto{\pgfqpoint{2.477119in}{0.714950in}}{\pgfqpoint{2.481509in}{0.704351in}}{\pgfqpoint{2.489323in}{0.696537in}}%
\pgfpathcurveto{\pgfqpoint{2.497137in}{0.688724in}}{\pgfqpoint{2.507736in}{0.684333in}}{\pgfqpoint{2.518786in}{0.684333in}}%
\pgfpathclose%
\pgfusepath{stroke,fill}%
\end{pgfscope}%
\begin{pgfscope}%
\pgfpathrectangle{\pgfqpoint{0.800000in}{0.528000in}}{\pgfqpoint{4.960000in}{3.696000in}}%
\pgfusepath{clip}%
\pgfsetbuttcap%
\pgfsetroundjoin%
\definecolor{currentfill}{rgb}{0.000000,0.000000,0.000000}%
\pgfsetfillcolor{currentfill}%
\pgfsetlinewidth{1.003750pt}%
\definecolor{currentstroke}{rgb}{0.000000,0.000000,0.000000}%
\pgfsetstrokecolor{currentstroke}%
\pgfsetdash{}{0pt}%
\pgfpathmoveto{\pgfqpoint{2.518786in}{0.684333in}}%
\pgfpathcurveto{\pgfqpoint{2.529836in}{0.684333in}}{\pgfqpoint{2.540435in}{0.688724in}}{\pgfqpoint{2.548249in}{0.696537in}}%
\pgfpathcurveto{\pgfqpoint{2.556062in}{0.704351in}}{\pgfqpoint{2.560452in}{0.714950in}}{\pgfqpoint{2.560452in}{0.726000in}}%
\pgfpathcurveto{\pgfqpoint{2.560452in}{0.737050in}}{\pgfqpoint{2.556062in}{0.747649in}}{\pgfqpoint{2.548249in}{0.755463in}}%
\pgfpathcurveto{\pgfqpoint{2.540435in}{0.763276in}}{\pgfqpoint{2.529836in}{0.767667in}}{\pgfqpoint{2.518786in}{0.767667in}}%
\pgfpathcurveto{\pgfqpoint{2.507736in}{0.767667in}}{\pgfqpoint{2.497137in}{0.763276in}}{\pgfqpoint{2.489323in}{0.755463in}}%
\pgfpathcurveto{\pgfqpoint{2.481509in}{0.747649in}}{\pgfqpoint{2.477119in}{0.737050in}}{\pgfqpoint{2.477119in}{0.726000in}}%
\pgfpathcurveto{\pgfqpoint{2.477119in}{0.714950in}}{\pgfqpoint{2.481509in}{0.704351in}}{\pgfqpoint{2.489323in}{0.696537in}}%
\pgfpathcurveto{\pgfqpoint{2.497137in}{0.688724in}}{\pgfqpoint{2.507736in}{0.684333in}}{\pgfqpoint{2.518786in}{0.684333in}}%
\pgfpathclose%
\pgfusepath{stroke,fill}%
\end{pgfscope}%
\begin{pgfscope}%
\pgfpathrectangle{\pgfqpoint{0.800000in}{0.528000in}}{\pgfqpoint{4.960000in}{3.696000in}}%
\pgfusepath{clip}%
\pgfsetbuttcap%
\pgfsetroundjoin%
\definecolor{currentfill}{rgb}{0.000000,0.000000,0.000000}%
\pgfsetfillcolor{currentfill}%
\pgfsetlinewidth{1.003750pt}%
\definecolor{currentstroke}{rgb}{0.000000,0.000000,0.000000}%
\pgfsetstrokecolor{currentstroke}%
\pgfsetdash{}{0pt}%
\pgfpathmoveto{\pgfqpoint{2.518786in}{0.684333in}}%
\pgfpathcurveto{\pgfqpoint{2.529836in}{0.684333in}}{\pgfqpoint{2.540435in}{0.688724in}}{\pgfqpoint{2.548249in}{0.696537in}}%
\pgfpathcurveto{\pgfqpoint{2.556062in}{0.704351in}}{\pgfqpoint{2.560452in}{0.714950in}}{\pgfqpoint{2.560452in}{0.726000in}}%
\pgfpathcurveto{\pgfqpoint{2.560452in}{0.737050in}}{\pgfqpoint{2.556062in}{0.747649in}}{\pgfqpoint{2.548249in}{0.755463in}}%
\pgfpathcurveto{\pgfqpoint{2.540435in}{0.763276in}}{\pgfqpoint{2.529836in}{0.767667in}}{\pgfqpoint{2.518786in}{0.767667in}}%
\pgfpathcurveto{\pgfqpoint{2.507736in}{0.767667in}}{\pgfqpoint{2.497137in}{0.763276in}}{\pgfqpoint{2.489323in}{0.755463in}}%
\pgfpathcurveto{\pgfqpoint{2.481509in}{0.747649in}}{\pgfqpoint{2.477119in}{0.737050in}}{\pgfqpoint{2.477119in}{0.726000in}}%
\pgfpathcurveto{\pgfqpoint{2.477119in}{0.714950in}}{\pgfqpoint{2.481509in}{0.704351in}}{\pgfqpoint{2.489323in}{0.696537in}}%
\pgfpathcurveto{\pgfqpoint{2.497137in}{0.688724in}}{\pgfqpoint{2.507736in}{0.684333in}}{\pgfqpoint{2.518786in}{0.684333in}}%
\pgfpathclose%
\pgfusepath{stroke,fill}%
\end{pgfscope}%
\begin{pgfscope}%
\pgfpathrectangle{\pgfqpoint{0.800000in}{0.528000in}}{\pgfqpoint{4.960000in}{3.696000in}}%
\pgfusepath{clip}%
\pgfsetbuttcap%
\pgfsetroundjoin%
\definecolor{currentfill}{rgb}{0.000000,0.000000,0.000000}%
\pgfsetfillcolor{currentfill}%
\pgfsetlinewidth{1.003750pt}%
\definecolor{currentstroke}{rgb}{0.000000,0.000000,0.000000}%
\pgfsetstrokecolor{currentstroke}%
\pgfsetdash{}{0pt}%
\pgfpathmoveto{\pgfqpoint{2.518786in}{0.684333in}}%
\pgfpathcurveto{\pgfqpoint{2.529836in}{0.684333in}}{\pgfqpoint{2.540435in}{0.688724in}}{\pgfqpoint{2.548249in}{0.696537in}}%
\pgfpathcurveto{\pgfqpoint{2.556062in}{0.704351in}}{\pgfqpoint{2.560452in}{0.714950in}}{\pgfqpoint{2.560452in}{0.726000in}}%
\pgfpathcurveto{\pgfqpoint{2.560452in}{0.737050in}}{\pgfqpoint{2.556062in}{0.747649in}}{\pgfqpoint{2.548249in}{0.755463in}}%
\pgfpathcurveto{\pgfqpoint{2.540435in}{0.763276in}}{\pgfqpoint{2.529836in}{0.767667in}}{\pgfqpoint{2.518786in}{0.767667in}}%
\pgfpathcurveto{\pgfqpoint{2.507736in}{0.767667in}}{\pgfqpoint{2.497137in}{0.763276in}}{\pgfqpoint{2.489323in}{0.755463in}}%
\pgfpathcurveto{\pgfqpoint{2.481509in}{0.747649in}}{\pgfqpoint{2.477119in}{0.737050in}}{\pgfqpoint{2.477119in}{0.726000in}}%
\pgfpathcurveto{\pgfqpoint{2.477119in}{0.714950in}}{\pgfqpoint{2.481509in}{0.704351in}}{\pgfqpoint{2.489323in}{0.696537in}}%
\pgfpathcurveto{\pgfqpoint{2.497137in}{0.688724in}}{\pgfqpoint{2.507736in}{0.684333in}}{\pgfqpoint{2.518786in}{0.684333in}}%
\pgfpathclose%
\pgfusepath{stroke,fill}%
\end{pgfscope}%
\begin{pgfscope}%
\pgfpathrectangle{\pgfqpoint{0.800000in}{0.528000in}}{\pgfqpoint{4.960000in}{3.696000in}}%
\pgfusepath{clip}%
\pgfsetbuttcap%
\pgfsetroundjoin%
\definecolor{currentfill}{rgb}{0.000000,0.000000,0.000000}%
\pgfsetfillcolor{currentfill}%
\pgfsetlinewidth{1.003750pt}%
\definecolor{currentstroke}{rgb}{0.000000,0.000000,0.000000}%
\pgfsetstrokecolor{currentstroke}%
\pgfsetdash{}{0pt}%
\pgfpathmoveto{\pgfqpoint{2.518786in}{0.684333in}}%
\pgfpathcurveto{\pgfqpoint{2.529836in}{0.684333in}}{\pgfqpoint{2.540435in}{0.688724in}}{\pgfqpoint{2.548249in}{0.696537in}}%
\pgfpathcurveto{\pgfqpoint{2.556062in}{0.704351in}}{\pgfqpoint{2.560452in}{0.714950in}}{\pgfqpoint{2.560452in}{0.726000in}}%
\pgfpathcurveto{\pgfqpoint{2.560452in}{0.737050in}}{\pgfqpoint{2.556062in}{0.747649in}}{\pgfqpoint{2.548249in}{0.755463in}}%
\pgfpathcurveto{\pgfqpoint{2.540435in}{0.763276in}}{\pgfqpoint{2.529836in}{0.767667in}}{\pgfqpoint{2.518786in}{0.767667in}}%
\pgfpathcurveto{\pgfqpoint{2.507736in}{0.767667in}}{\pgfqpoint{2.497137in}{0.763276in}}{\pgfqpoint{2.489323in}{0.755463in}}%
\pgfpathcurveto{\pgfqpoint{2.481509in}{0.747649in}}{\pgfqpoint{2.477119in}{0.737050in}}{\pgfqpoint{2.477119in}{0.726000in}}%
\pgfpathcurveto{\pgfqpoint{2.477119in}{0.714950in}}{\pgfqpoint{2.481509in}{0.704351in}}{\pgfqpoint{2.489323in}{0.696537in}}%
\pgfpathcurveto{\pgfqpoint{2.497137in}{0.688724in}}{\pgfqpoint{2.507736in}{0.684333in}}{\pgfqpoint{2.518786in}{0.684333in}}%
\pgfpathclose%
\pgfusepath{stroke,fill}%
\end{pgfscope}%
\begin{pgfscope}%
\pgfpathrectangle{\pgfqpoint{0.800000in}{0.528000in}}{\pgfqpoint{4.960000in}{3.696000in}}%
\pgfusepath{clip}%
\pgfsetbuttcap%
\pgfsetroundjoin%
\definecolor{currentfill}{rgb}{0.000000,0.000000,0.000000}%
\pgfsetfillcolor{currentfill}%
\pgfsetlinewidth{1.003750pt}%
\definecolor{currentstroke}{rgb}{0.000000,0.000000,0.000000}%
\pgfsetstrokecolor{currentstroke}%
\pgfsetdash{}{0pt}%
\pgfpathmoveto{\pgfqpoint{2.518786in}{0.684333in}}%
\pgfpathcurveto{\pgfqpoint{2.529836in}{0.684333in}}{\pgfqpoint{2.540435in}{0.688724in}}{\pgfqpoint{2.548249in}{0.696537in}}%
\pgfpathcurveto{\pgfqpoint{2.556062in}{0.704351in}}{\pgfqpoint{2.560452in}{0.714950in}}{\pgfqpoint{2.560452in}{0.726000in}}%
\pgfpathcurveto{\pgfqpoint{2.560452in}{0.737050in}}{\pgfqpoint{2.556062in}{0.747649in}}{\pgfqpoint{2.548249in}{0.755463in}}%
\pgfpathcurveto{\pgfqpoint{2.540435in}{0.763276in}}{\pgfqpoint{2.529836in}{0.767667in}}{\pgfqpoint{2.518786in}{0.767667in}}%
\pgfpathcurveto{\pgfqpoint{2.507736in}{0.767667in}}{\pgfqpoint{2.497137in}{0.763276in}}{\pgfqpoint{2.489323in}{0.755463in}}%
\pgfpathcurveto{\pgfqpoint{2.481509in}{0.747649in}}{\pgfqpoint{2.477119in}{0.737050in}}{\pgfqpoint{2.477119in}{0.726000in}}%
\pgfpathcurveto{\pgfqpoint{2.477119in}{0.714950in}}{\pgfqpoint{2.481509in}{0.704351in}}{\pgfqpoint{2.489323in}{0.696537in}}%
\pgfpathcurveto{\pgfqpoint{2.497137in}{0.688724in}}{\pgfqpoint{2.507736in}{0.684333in}}{\pgfqpoint{2.518786in}{0.684333in}}%
\pgfpathclose%
\pgfusepath{stroke,fill}%
\end{pgfscope}%
\begin{pgfscope}%
\pgfpathrectangle{\pgfqpoint{0.800000in}{0.528000in}}{\pgfqpoint{4.960000in}{3.696000in}}%
\pgfusepath{clip}%
\pgfsetbuttcap%
\pgfsetroundjoin%
\definecolor{currentfill}{rgb}{0.000000,0.000000,0.000000}%
\pgfsetfillcolor{currentfill}%
\pgfsetlinewidth{1.003750pt}%
\definecolor{currentstroke}{rgb}{0.000000,0.000000,0.000000}%
\pgfsetstrokecolor{currentstroke}%
\pgfsetdash{}{0pt}%
\pgfpathmoveto{\pgfqpoint{2.518786in}{0.684333in}}%
\pgfpathcurveto{\pgfqpoint{2.529836in}{0.684333in}}{\pgfqpoint{2.540435in}{0.688724in}}{\pgfqpoint{2.548249in}{0.696537in}}%
\pgfpathcurveto{\pgfqpoint{2.556062in}{0.704351in}}{\pgfqpoint{2.560452in}{0.714950in}}{\pgfqpoint{2.560452in}{0.726000in}}%
\pgfpathcurveto{\pgfqpoint{2.560452in}{0.737050in}}{\pgfqpoint{2.556062in}{0.747649in}}{\pgfqpoint{2.548249in}{0.755463in}}%
\pgfpathcurveto{\pgfqpoint{2.540435in}{0.763276in}}{\pgfqpoint{2.529836in}{0.767667in}}{\pgfqpoint{2.518786in}{0.767667in}}%
\pgfpathcurveto{\pgfqpoint{2.507736in}{0.767667in}}{\pgfqpoint{2.497137in}{0.763276in}}{\pgfqpoint{2.489323in}{0.755463in}}%
\pgfpathcurveto{\pgfqpoint{2.481509in}{0.747649in}}{\pgfqpoint{2.477119in}{0.737050in}}{\pgfqpoint{2.477119in}{0.726000in}}%
\pgfpathcurveto{\pgfqpoint{2.477119in}{0.714950in}}{\pgfqpoint{2.481509in}{0.704351in}}{\pgfqpoint{2.489323in}{0.696537in}}%
\pgfpathcurveto{\pgfqpoint{2.497137in}{0.688724in}}{\pgfqpoint{2.507736in}{0.684333in}}{\pgfqpoint{2.518786in}{0.684333in}}%
\pgfpathclose%
\pgfusepath{stroke,fill}%
\end{pgfscope}%
\begin{pgfscope}%
\pgfpathrectangle{\pgfqpoint{0.800000in}{0.528000in}}{\pgfqpoint{4.960000in}{3.696000in}}%
\pgfusepath{clip}%
\pgfsetbuttcap%
\pgfsetroundjoin%
\definecolor{currentfill}{rgb}{0.000000,0.000000,0.000000}%
\pgfsetfillcolor{currentfill}%
\pgfsetlinewidth{1.003750pt}%
\definecolor{currentstroke}{rgb}{0.000000,0.000000,0.000000}%
\pgfsetstrokecolor{currentstroke}%
\pgfsetdash{}{0pt}%
\pgfpathmoveto{\pgfqpoint{2.518786in}{0.684333in}}%
\pgfpathcurveto{\pgfqpoint{2.529836in}{0.684333in}}{\pgfqpoint{2.540435in}{0.688724in}}{\pgfqpoint{2.548249in}{0.696537in}}%
\pgfpathcurveto{\pgfqpoint{2.556062in}{0.704351in}}{\pgfqpoint{2.560452in}{0.714950in}}{\pgfqpoint{2.560452in}{0.726000in}}%
\pgfpathcurveto{\pgfqpoint{2.560452in}{0.737050in}}{\pgfqpoint{2.556062in}{0.747649in}}{\pgfqpoint{2.548249in}{0.755463in}}%
\pgfpathcurveto{\pgfqpoint{2.540435in}{0.763276in}}{\pgfqpoint{2.529836in}{0.767667in}}{\pgfqpoint{2.518786in}{0.767667in}}%
\pgfpathcurveto{\pgfqpoint{2.507736in}{0.767667in}}{\pgfqpoint{2.497137in}{0.763276in}}{\pgfqpoint{2.489323in}{0.755463in}}%
\pgfpathcurveto{\pgfqpoint{2.481509in}{0.747649in}}{\pgfqpoint{2.477119in}{0.737050in}}{\pgfqpoint{2.477119in}{0.726000in}}%
\pgfpathcurveto{\pgfqpoint{2.477119in}{0.714950in}}{\pgfqpoint{2.481509in}{0.704351in}}{\pgfqpoint{2.489323in}{0.696537in}}%
\pgfpathcurveto{\pgfqpoint{2.497137in}{0.688724in}}{\pgfqpoint{2.507736in}{0.684333in}}{\pgfqpoint{2.518786in}{0.684333in}}%
\pgfpathclose%
\pgfusepath{stroke,fill}%
\end{pgfscope}%
\begin{pgfscope}%
\pgfpathrectangle{\pgfqpoint{0.800000in}{0.528000in}}{\pgfqpoint{4.960000in}{3.696000in}}%
\pgfusepath{clip}%
\pgfsetbuttcap%
\pgfsetroundjoin%
\definecolor{currentfill}{rgb}{0.000000,0.000000,0.000000}%
\pgfsetfillcolor{currentfill}%
\pgfsetlinewidth{1.003750pt}%
\definecolor{currentstroke}{rgb}{0.000000,0.000000,0.000000}%
\pgfsetstrokecolor{currentstroke}%
\pgfsetdash{}{0pt}%
\pgfpathmoveto{\pgfqpoint{2.518786in}{0.684333in}}%
\pgfpathcurveto{\pgfqpoint{2.529836in}{0.684333in}}{\pgfqpoint{2.540435in}{0.688724in}}{\pgfqpoint{2.548249in}{0.696537in}}%
\pgfpathcurveto{\pgfqpoint{2.556062in}{0.704351in}}{\pgfqpoint{2.560452in}{0.714950in}}{\pgfqpoint{2.560452in}{0.726000in}}%
\pgfpathcurveto{\pgfqpoint{2.560452in}{0.737050in}}{\pgfqpoint{2.556062in}{0.747649in}}{\pgfqpoint{2.548249in}{0.755463in}}%
\pgfpathcurveto{\pgfqpoint{2.540435in}{0.763276in}}{\pgfqpoint{2.529836in}{0.767667in}}{\pgfqpoint{2.518786in}{0.767667in}}%
\pgfpathcurveto{\pgfqpoint{2.507736in}{0.767667in}}{\pgfqpoint{2.497137in}{0.763276in}}{\pgfqpoint{2.489323in}{0.755463in}}%
\pgfpathcurveto{\pgfqpoint{2.481509in}{0.747649in}}{\pgfqpoint{2.477119in}{0.737050in}}{\pgfqpoint{2.477119in}{0.726000in}}%
\pgfpathcurveto{\pgfqpoint{2.477119in}{0.714950in}}{\pgfqpoint{2.481509in}{0.704351in}}{\pgfqpoint{2.489323in}{0.696537in}}%
\pgfpathcurveto{\pgfqpoint{2.497137in}{0.688724in}}{\pgfqpoint{2.507736in}{0.684333in}}{\pgfqpoint{2.518786in}{0.684333in}}%
\pgfpathclose%
\pgfusepath{stroke,fill}%
\end{pgfscope}%
\begin{pgfscope}%
\pgfpathrectangle{\pgfqpoint{0.800000in}{0.528000in}}{\pgfqpoint{4.960000in}{3.696000in}}%
\pgfusepath{clip}%
\pgfsetbuttcap%
\pgfsetroundjoin%
\definecolor{currentfill}{rgb}{0.000000,0.000000,0.000000}%
\pgfsetfillcolor{currentfill}%
\pgfsetlinewidth{1.003750pt}%
\definecolor{currentstroke}{rgb}{0.000000,0.000000,0.000000}%
\pgfsetstrokecolor{currentstroke}%
\pgfsetdash{}{0pt}%
\pgfpathmoveto{\pgfqpoint{2.518786in}{0.684333in}}%
\pgfpathcurveto{\pgfqpoint{2.529836in}{0.684333in}}{\pgfqpoint{2.540435in}{0.688724in}}{\pgfqpoint{2.548249in}{0.696537in}}%
\pgfpathcurveto{\pgfqpoint{2.556062in}{0.704351in}}{\pgfqpoint{2.560452in}{0.714950in}}{\pgfqpoint{2.560452in}{0.726000in}}%
\pgfpathcurveto{\pgfqpoint{2.560452in}{0.737050in}}{\pgfqpoint{2.556062in}{0.747649in}}{\pgfqpoint{2.548249in}{0.755463in}}%
\pgfpathcurveto{\pgfqpoint{2.540435in}{0.763276in}}{\pgfqpoint{2.529836in}{0.767667in}}{\pgfqpoint{2.518786in}{0.767667in}}%
\pgfpathcurveto{\pgfqpoint{2.507736in}{0.767667in}}{\pgfqpoint{2.497137in}{0.763276in}}{\pgfqpoint{2.489323in}{0.755463in}}%
\pgfpathcurveto{\pgfqpoint{2.481509in}{0.747649in}}{\pgfqpoint{2.477119in}{0.737050in}}{\pgfqpoint{2.477119in}{0.726000in}}%
\pgfpathcurveto{\pgfqpoint{2.477119in}{0.714950in}}{\pgfqpoint{2.481509in}{0.704351in}}{\pgfqpoint{2.489323in}{0.696537in}}%
\pgfpathcurveto{\pgfqpoint{2.497137in}{0.688724in}}{\pgfqpoint{2.507736in}{0.684333in}}{\pgfqpoint{2.518786in}{0.684333in}}%
\pgfpathclose%
\pgfusepath{stroke,fill}%
\end{pgfscope}%
\begin{pgfscope}%
\pgfpathrectangle{\pgfqpoint{0.800000in}{0.528000in}}{\pgfqpoint{4.960000in}{3.696000in}}%
\pgfusepath{clip}%
\pgfsetbuttcap%
\pgfsetroundjoin%
\definecolor{currentfill}{rgb}{0.000000,0.000000,0.000000}%
\pgfsetfillcolor{currentfill}%
\pgfsetlinewidth{1.003750pt}%
\definecolor{currentstroke}{rgb}{0.000000,0.000000,0.000000}%
\pgfsetstrokecolor{currentstroke}%
\pgfsetdash{}{0pt}%
\pgfpathmoveto{\pgfqpoint{2.518786in}{0.684333in}}%
\pgfpathcurveto{\pgfqpoint{2.529836in}{0.684333in}}{\pgfqpoint{2.540435in}{0.688724in}}{\pgfqpoint{2.548249in}{0.696537in}}%
\pgfpathcurveto{\pgfqpoint{2.556062in}{0.704351in}}{\pgfqpoint{2.560452in}{0.714950in}}{\pgfqpoint{2.560452in}{0.726000in}}%
\pgfpathcurveto{\pgfqpoint{2.560452in}{0.737050in}}{\pgfqpoint{2.556062in}{0.747649in}}{\pgfqpoint{2.548249in}{0.755463in}}%
\pgfpathcurveto{\pgfqpoint{2.540435in}{0.763276in}}{\pgfqpoint{2.529836in}{0.767667in}}{\pgfqpoint{2.518786in}{0.767667in}}%
\pgfpathcurveto{\pgfqpoint{2.507736in}{0.767667in}}{\pgfqpoint{2.497137in}{0.763276in}}{\pgfqpoint{2.489323in}{0.755463in}}%
\pgfpathcurveto{\pgfqpoint{2.481509in}{0.747649in}}{\pgfqpoint{2.477119in}{0.737050in}}{\pgfqpoint{2.477119in}{0.726000in}}%
\pgfpathcurveto{\pgfqpoint{2.477119in}{0.714950in}}{\pgfqpoint{2.481509in}{0.704351in}}{\pgfqpoint{2.489323in}{0.696537in}}%
\pgfpathcurveto{\pgfqpoint{2.497137in}{0.688724in}}{\pgfqpoint{2.507736in}{0.684333in}}{\pgfqpoint{2.518786in}{0.684333in}}%
\pgfpathclose%
\pgfusepath{stroke,fill}%
\end{pgfscope}%
\begin{pgfscope}%
\pgfpathrectangle{\pgfqpoint{0.800000in}{0.528000in}}{\pgfqpoint{4.960000in}{3.696000in}}%
\pgfusepath{clip}%
\pgfsetbuttcap%
\pgfsetroundjoin%
\definecolor{currentfill}{rgb}{0.000000,0.000000,0.000000}%
\pgfsetfillcolor{currentfill}%
\pgfsetlinewidth{1.003750pt}%
\definecolor{currentstroke}{rgb}{0.000000,0.000000,0.000000}%
\pgfsetstrokecolor{currentstroke}%
\pgfsetdash{}{0pt}%
\pgfpathmoveto{\pgfqpoint{2.518786in}{0.684333in}}%
\pgfpathcurveto{\pgfqpoint{2.529836in}{0.684333in}}{\pgfqpoint{2.540435in}{0.688724in}}{\pgfqpoint{2.548249in}{0.696537in}}%
\pgfpathcurveto{\pgfqpoint{2.556062in}{0.704351in}}{\pgfqpoint{2.560452in}{0.714950in}}{\pgfqpoint{2.560452in}{0.726000in}}%
\pgfpathcurveto{\pgfqpoint{2.560452in}{0.737050in}}{\pgfqpoint{2.556062in}{0.747649in}}{\pgfqpoint{2.548249in}{0.755463in}}%
\pgfpathcurveto{\pgfqpoint{2.540435in}{0.763276in}}{\pgfqpoint{2.529836in}{0.767667in}}{\pgfqpoint{2.518786in}{0.767667in}}%
\pgfpathcurveto{\pgfqpoint{2.507736in}{0.767667in}}{\pgfqpoint{2.497137in}{0.763276in}}{\pgfqpoint{2.489323in}{0.755463in}}%
\pgfpathcurveto{\pgfqpoint{2.481509in}{0.747649in}}{\pgfqpoint{2.477119in}{0.737050in}}{\pgfqpoint{2.477119in}{0.726000in}}%
\pgfpathcurveto{\pgfqpoint{2.477119in}{0.714950in}}{\pgfqpoint{2.481509in}{0.704351in}}{\pgfqpoint{2.489323in}{0.696537in}}%
\pgfpathcurveto{\pgfqpoint{2.497137in}{0.688724in}}{\pgfqpoint{2.507736in}{0.684333in}}{\pgfqpoint{2.518786in}{0.684333in}}%
\pgfpathclose%
\pgfusepath{stroke,fill}%
\end{pgfscope}%
\begin{pgfscope}%
\pgfpathrectangle{\pgfqpoint{0.800000in}{0.528000in}}{\pgfqpoint{4.960000in}{3.696000in}}%
\pgfusepath{clip}%
\pgfsetbuttcap%
\pgfsetroundjoin%
\definecolor{currentfill}{rgb}{0.000000,0.000000,0.000000}%
\pgfsetfillcolor{currentfill}%
\pgfsetlinewidth{1.003750pt}%
\definecolor{currentstroke}{rgb}{0.000000,0.000000,0.000000}%
\pgfsetstrokecolor{currentstroke}%
\pgfsetdash{}{0pt}%
\pgfpathmoveto{\pgfqpoint{2.518786in}{0.684333in}}%
\pgfpathcurveto{\pgfqpoint{2.529836in}{0.684333in}}{\pgfqpoint{2.540435in}{0.688724in}}{\pgfqpoint{2.548249in}{0.696537in}}%
\pgfpathcurveto{\pgfqpoint{2.556062in}{0.704351in}}{\pgfqpoint{2.560452in}{0.714950in}}{\pgfqpoint{2.560452in}{0.726000in}}%
\pgfpathcurveto{\pgfqpoint{2.560452in}{0.737050in}}{\pgfqpoint{2.556062in}{0.747649in}}{\pgfqpoint{2.548249in}{0.755463in}}%
\pgfpathcurveto{\pgfqpoint{2.540435in}{0.763276in}}{\pgfqpoint{2.529836in}{0.767667in}}{\pgfqpoint{2.518786in}{0.767667in}}%
\pgfpathcurveto{\pgfqpoint{2.507736in}{0.767667in}}{\pgfqpoint{2.497137in}{0.763276in}}{\pgfqpoint{2.489323in}{0.755463in}}%
\pgfpathcurveto{\pgfqpoint{2.481509in}{0.747649in}}{\pgfqpoint{2.477119in}{0.737050in}}{\pgfqpoint{2.477119in}{0.726000in}}%
\pgfpathcurveto{\pgfqpoint{2.477119in}{0.714950in}}{\pgfqpoint{2.481509in}{0.704351in}}{\pgfqpoint{2.489323in}{0.696537in}}%
\pgfpathcurveto{\pgfqpoint{2.497137in}{0.688724in}}{\pgfqpoint{2.507736in}{0.684333in}}{\pgfqpoint{2.518786in}{0.684333in}}%
\pgfpathclose%
\pgfusepath{stroke,fill}%
\end{pgfscope}%
\begin{pgfscope}%
\pgfpathrectangle{\pgfqpoint{0.800000in}{0.528000in}}{\pgfqpoint{4.960000in}{3.696000in}}%
\pgfusepath{clip}%
\pgfsetbuttcap%
\pgfsetroundjoin%
\definecolor{currentfill}{rgb}{0.000000,0.000000,0.000000}%
\pgfsetfillcolor{currentfill}%
\pgfsetlinewidth{1.003750pt}%
\definecolor{currentstroke}{rgb}{0.000000,0.000000,0.000000}%
\pgfsetstrokecolor{currentstroke}%
\pgfsetdash{}{0pt}%
\pgfpathmoveto{\pgfqpoint{2.518786in}{0.684333in}}%
\pgfpathcurveto{\pgfqpoint{2.529836in}{0.684333in}}{\pgfqpoint{2.540435in}{0.688724in}}{\pgfqpoint{2.548249in}{0.696537in}}%
\pgfpathcurveto{\pgfqpoint{2.556062in}{0.704351in}}{\pgfqpoint{2.560452in}{0.714950in}}{\pgfqpoint{2.560452in}{0.726000in}}%
\pgfpathcurveto{\pgfqpoint{2.560452in}{0.737050in}}{\pgfqpoint{2.556062in}{0.747649in}}{\pgfqpoint{2.548249in}{0.755463in}}%
\pgfpathcurveto{\pgfqpoint{2.540435in}{0.763276in}}{\pgfqpoint{2.529836in}{0.767667in}}{\pgfqpoint{2.518786in}{0.767667in}}%
\pgfpathcurveto{\pgfqpoint{2.507736in}{0.767667in}}{\pgfqpoint{2.497137in}{0.763276in}}{\pgfqpoint{2.489323in}{0.755463in}}%
\pgfpathcurveto{\pgfqpoint{2.481509in}{0.747649in}}{\pgfqpoint{2.477119in}{0.737050in}}{\pgfqpoint{2.477119in}{0.726000in}}%
\pgfpathcurveto{\pgfqpoint{2.477119in}{0.714950in}}{\pgfqpoint{2.481509in}{0.704351in}}{\pgfqpoint{2.489323in}{0.696537in}}%
\pgfpathcurveto{\pgfqpoint{2.497137in}{0.688724in}}{\pgfqpoint{2.507736in}{0.684333in}}{\pgfqpoint{2.518786in}{0.684333in}}%
\pgfpathclose%
\pgfusepath{stroke,fill}%
\end{pgfscope}%
\begin{pgfscope}%
\pgfpathrectangle{\pgfqpoint{0.800000in}{0.528000in}}{\pgfqpoint{4.960000in}{3.696000in}}%
\pgfusepath{clip}%
\pgfsetbuttcap%
\pgfsetroundjoin%
\definecolor{currentfill}{rgb}{0.000000,0.000000,0.000000}%
\pgfsetfillcolor{currentfill}%
\pgfsetlinewidth{1.003750pt}%
\definecolor{currentstroke}{rgb}{0.000000,0.000000,0.000000}%
\pgfsetstrokecolor{currentstroke}%
\pgfsetdash{}{0pt}%
\pgfpathmoveto{\pgfqpoint{2.518786in}{0.684333in}}%
\pgfpathcurveto{\pgfqpoint{2.529836in}{0.684333in}}{\pgfqpoint{2.540435in}{0.688724in}}{\pgfqpoint{2.548249in}{0.696537in}}%
\pgfpathcurveto{\pgfqpoint{2.556062in}{0.704351in}}{\pgfqpoint{2.560452in}{0.714950in}}{\pgfqpoint{2.560452in}{0.726000in}}%
\pgfpathcurveto{\pgfqpoint{2.560452in}{0.737050in}}{\pgfqpoint{2.556062in}{0.747649in}}{\pgfqpoint{2.548249in}{0.755463in}}%
\pgfpathcurveto{\pgfqpoint{2.540435in}{0.763276in}}{\pgfqpoint{2.529836in}{0.767667in}}{\pgfqpoint{2.518786in}{0.767667in}}%
\pgfpathcurveto{\pgfqpoint{2.507736in}{0.767667in}}{\pgfqpoint{2.497137in}{0.763276in}}{\pgfqpoint{2.489323in}{0.755463in}}%
\pgfpathcurveto{\pgfqpoint{2.481509in}{0.747649in}}{\pgfqpoint{2.477119in}{0.737050in}}{\pgfqpoint{2.477119in}{0.726000in}}%
\pgfpathcurveto{\pgfqpoint{2.477119in}{0.714950in}}{\pgfqpoint{2.481509in}{0.704351in}}{\pgfqpoint{2.489323in}{0.696537in}}%
\pgfpathcurveto{\pgfqpoint{2.497137in}{0.688724in}}{\pgfqpoint{2.507736in}{0.684333in}}{\pgfqpoint{2.518786in}{0.684333in}}%
\pgfpathclose%
\pgfusepath{stroke,fill}%
\end{pgfscope}%
\begin{pgfscope}%
\pgfpathrectangle{\pgfqpoint{0.800000in}{0.528000in}}{\pgfqpoint{4.960000in}{3.696000in}}%
\pgfusepath{clip}%
\pgfsetbuttcap%
\pgfsetroundjoin%
\definecolor{currentfill}{rgb}{0.000000,0.000000,0.000000}%
\pgfsetfillcolor{currentfill}%
\pgfsetlinewidth{1.003750pt}%
\definecolor{currentstroke}{rgb}{0.000000,0.000000,0.000000}%
\pgfsetstrokecolor{currentstroke}%
\pgfsetdash{}{0pt}%
\pgfpathmoveto{\pgfqpoint{2.518786in}{0.684333in}}%
\pgfpathcurveto{\pgfqpoint{2.529836in}{0.684333in}}{\pgfqpoint{2.540435in}{0.688724in}}{\pgfqpoint{2.548249in}{0.696537in}}%
\pgfpathcurveto{\pgfqpoint{2.556062in}{0.704351in}}{\pgfqpoint{2.560452in}{0.714950in}}{\pgfqpoint{2.560452in}{0.726000in}}%
\pgfpathcurveto{\pgfqpoint{2.560452in}{0.737050in}}{\pgfqpoint{2.556062in}{0.747649in}}{\pgfqpoint{2.548249in}{0.755463in}}%
\pgfpathcurveto{\pgfqpoint{2.540435in}{0.763276in}}{\pgfqpoint{2.529836in}{0.767667in}}{\pgfqpoint{2.518786in}{0.767667in}}%
\pgfpathcurveto{\pgfqpoint{2.507736in}{0.767667in}}{\pgfqpoint{2.497137in}{0.763276in}}{\pgfqpoint{2.489323in}{0.755463in}}%
\pgfpathcurveto{\pgfqpoint{2.481509in}{0.747649in}}{\pgfqpoint{2.477119in}{0.737050in}}{\pgfqpoint{2.477119in}{0.726000in}}%
\pgfpathcurveto{\pgfqpoint{2.477119in}{0.714950in}}{\pgfqpoint{2.481509in}{0.704351in}}{\pgfqpoint{2.489323in}{0.696537in}}%
\pgfpathcurveto{\pgfqpoint{2.497137in}{0.688724in}}{\pgfqpoint{2.507736in}{0.684333in}}{\pgfqpoint{2.518786in}{0.684333in}}%
\pgfpathclose%
\pgfusepath{stroke,fill}%
\end{pgfscope}%
\begin{pgfscope}%
\pgfpathrectangle{\pgfqpoint{0.800000in}{0.528000in}}{\pgfqpoint{4.960000in}{3.696000in}}%
\pgfusepath{clip}%
\pgfsetbuttcap%
\pgfsetroundjoin%
\definecolor{currentfill}{rgb}{0.000000,0.000000,0.000000}%
\pgfsetfillcolor{currentfill}%
\pgfsetlinewidth{1.003750pt}%
\definecolor{currentstroke}{rgb}{0.000000,0.000000,0.000000}%
\pgfsetstrokecolor{currentstroke}%
\pgfsetdash{}{0pt}%
\pgfpathmoveto{\pgfqpoint{2.518786in}{0.684333in}}%
\pgfpathcurveto{\pgfqpoint{2.529836in}{0.684333in}}{\pgfqpoint{2.540435in}{0.688724in}}{\pgfqpoint{2.548249in}{0.696537in}}%
\pgfpathcurveto{\pgfqpoint{2.556062in}{0.704351in}}{\pgfqpoint{2.560452in}{0.714950in}}{\pgfqpoint{2.560452in}{0.726000in}}%
\pgfpathcurveto{\pgfqpoint{2.560452in}{0.737050in}}{\pgfqpoint{2.556062in}{0.747649in}}{\pgfqpoint{2.548249in}{0.755463in}}%
\pgfpathcurveto{\pgfqpoint{2.540435in}{0.763276in}}{\pgfqpoint{2.529836in}{0.767667in}}{\pgfqpoint{2.518786in}{0.767667in}}%
\pgfpathcurveto{\pgfqpoint{2.507736in}{0.767667in}}{\pgfqpoint{2.497137in}{0.763276in}}{\pgfqpoint{2.489323in}{0.755463in}}%
\pgfpathcurveto{\pgfqpoint{2.481509in}{0.747649in}}{\pgfqpoint{2.477119in}{0.737050in}}{\pgfqpoint{2.477119in}{0.726000in}}%
\pgfpathcurveto{\pgfqpoint{2.477119in}{0.714950in}}{\pgfqpoint{2.481509in}{0.704351in}}{\pgfqpoint{2.489323in}{0.696537in}}%
\pgfpathcurveto{\pgfqpoint{2.497137in}{0.688724in}}{\pgfqpoint{2.507736in}{0.684333in}}{\pgfqpoint{2.518786in}{0.684333in}}%
\pgfpathclose%
\pgfusepath{stroke,fill}%
\end{pgfscope}%
\begin{pgfscope}%
\pgfpathrectangle{\pgfqpoint{0.800000in}{0.528000in}}{\pgfqpoint{4.960000in}{3.696000in}}%
\pgfusepath{clip}%
\pgfsetbuttcap%
\pgfsetroundjoin%
\definecolor{currentfill}{rgb}{0.000000,0.000000,0.000000}%
\pgfsetfillcolor{currentfill}%
\pgfsetlinewidth{1.003750pt}%
\definecolor{currentstroke}{rgb}{0.000000,0.000000,0.000000}%
\pgfsetstrokecolor{currentstroke}%
\pgfsetdash{}{0pt}%
\pgfpathmoveto{\pgfqpoint{2.518786in}{0.684333in}}%
\pgfpathcurveto{\pgfqpoint{2.529836in}{0.684333in}}{\pgfqpoint{2.540435in}{0.688724in}}{\pgfqpoint{2.548249in}{0.696537in}}%
\pgfpathcurveto{\pgfqpoint{2.556062in}{0.704351in}}{\pgfqpoint{2.560452in}{0.714950in}}{\pgfqpoint{2.560452in}{0.726000in}}%
\pgfpathcurveto{\pgfqpoint{2.560452in}{0.737050in}}{\pgfqpoint{2.556062in}{0.747649in}}{\pgfqpoint{2.548249in}{0.755463in}}%
\pgfpathcurveto{\pgfqpoint{2.540435in}{0.763276in}}{\pgfqpoint{2.529836in}{0.767667in}}{\pgfqpoint{2.518786in}{0.767667in}}%
\pgfpathcurveto{\pgfqpoint{2.507736in}{0.767667in}}{\pgfqpoint{2.497137in}{0.763276in}}{\pgfqpoint{2.489323in}{0.755463in}}%
\pgfpathcurveto{\pgfqpoint{2.481509in}{0.747649in}}{\pgfqpoint{2.477119in}{0.737050in}}{\pgfqpoint{2.477119in}{0.726000in}}%
\pgfpathcurveto{\pgfqpoint{2.477119in}{0.714950in}}{\pgfqpoint{2.481509in}{0.704351in}}{\pgfqpoint{2.489323in}{0.696537in}}%
\pgfpathcurveto{\pgfqpoint{2.497137in}{0.688724in}}{\pgfqpoint{2.507736in}{0.684333in}}{\pgfqpoint{2.518786in}{0.684333in}}%
\pgfpathclose%
\pgfusepath{stroke,fill}%
\end{pgfscope}%
\begin{pgfscope}%
\pgfpathrectangle{\pgfqpoint{0.800000in}{0.528000in}}{\pgfqpoint{4.960000in}{3.696000in}}%
\pgfusepath{clip}%
\pgfsetbuttcap%
\pgfsetroundjoin%
\definecolor{currentfill}{rgb}{0.000000,0.000000,0.000000}%
\pgfsetfillcolor{currentfill}%
\pgfsetlinewidth{1.003750pt}%
\definecolor{currentstroke}{rgb}{0.000000,0.000000,0.000000}%
\pgfsetstrokecolor{currentstroke}%
\pgfsetdash{}{0pt}%
\pgfpathmoveto{\pgfqpoint{2.518786in}{0.684333in}}%
\pgfpathcurveto{\pgfqpoint{2.529836in}{0.684333in}}{\pgfqpoint{2.540435in}{0.688724in}}{\pgfqpoint{2.548249in}{0.696537in}}%
\pgfpathcurveto{\pgfqpoint{2.556062in}{0.704351in}}{\pgfqpoint{2.560452in}{0.714950in}}{\pgfqpoint{2.560452in}{0.726000in}}%
\pgfpathcurveto{\pgfqpoint{2.560452in}{0.737050in}}{\pgfqpoint{2.556062in}{0.747649in}}{\pgfqpoint{2.548249in}{0.755463in}}%
\pgfpathcurveto{\pgfqpoint{2.540435in}{0.763276in}}{\pgfqpoint{2.529836in}{0.767667in}}{\pgfqpoint{2.518786in}{0.767667in}}%
\pgfpathcurveto{\pgfqpoint{2.507736in}{0.767667in}}{\pgfqpoint{2.497137in}{0.763276in}}{\pgfqpoint{2.489323in}{0.755463in}}%
\pgfpathcurveto{\pgfqpoint{2.481509in}{0.747649in}}{\pgfqpoint{2.477119in}{0.737050in}}{\pgfqpoint{2.477119in}{0.726000in}}%
\pgfpathcurveto{\pgfqpoint{2.477119in}{0.714950in}}{\pgfqpoint{2.481509in}{0.704351in}}{\pgfqpoint{2.489323in}{0.696537in}}%
\pgfpathcurveto{\pgfqpoint{2.497137in}{0.688724in}}{\pgfqpoint{2.507736in}{0.684333in}}{\pgfqpoint{2.518786in}{0.684333in}}%
\pgfpathclose%
\pgfusepath{stroke,fill}%
\end{pgfscope}%
\begin{pgfscope}%
\pgfpathrectangle{\pgfqpoint{0.800000in}{0.528000in}}{\pgfqpoint{4.960000in}{3.696000in}}%
\pgfusepath{clip}%
\pgfsetbuttcap%
\pgfsetroundjoin%
\definecolor{currentfill}{rgb}{0.000000,0.000000,0.000000}%
\pgfsetfillcolor{currentfill}%
\pgfsetlinewidth{1.003750pt}%
\definecolor{currentstroke}{rgb}{0.000000,0.000000,0.000000}%
\pgfsetstrokecolor{currentstroke}%
\pgfsetdash{}{0pt}%
\pgfpathmoveto{\pgfqpoint{2.518786in}{0.684333in}}%
\pgfpathcurveto{\pgfqpoint{2.529836in}{0.684333in}}{\pgfqpoint{2.540435in}{0.688724in}}{\pgfqpoint{2.548249in}{0.696537in}}%
\pgfpathcurveto{\pgfqpoint{2.556062in}{0.704351in}}{\pgfqpoint{2.560452in}{0.714950in}}{\pgfqpoint{2.560452in}{0.726000in}}%
\pgfpathcurveto{\pgfqpoint{2.560452in}{0.737050in}}{\pgfqpoint{2.556062in}{0.747649in}}{\pgfqpoint{2.548249in}{0.755463in}}%
\pgfpathcurveto{\pgfqpoint{2.540435in}{0.763276in}}{\pgfqpoint{2.529836in}{0.767667in}}{\pgfqpoint{2.518786in}{0.767667in}}%
\pgfpathcurveto{\pgfqpoint{2.507736in}{0.767667in}}{\pgfqpoint{2.497137in}{0.763276in}}{\pgfqpoint{2.489323in}{0.755463in}}%
\pgfpathcurveto{\pgfqpoint{2.481509in}{0.747649in}}{\pgfqpoint{2.477119in}{0.737050in}}{\pgfqpoint{2.477119in}{0.726000in}}%
\pgfpathcurveto{\pgfqpoint{2.477119in}{0.714950in}}{\pgfqpoint{2.481509in}{0.704351in}}{\pgfqpoint{2.489323in}{0.696537in}}%
\pgfpathcurveto{\pgfqpoint{2.497137in}{0.688724in}}{\pgfqpoint{2.507736in}{0.684333in}}{\pgfqpoint{2.518786in}{0.684333in}}%
\pgfpathclose%
\pgfusepath{stroke,fill}%
\end{pgfscope}%
\begin{pgfscope}%
\pgfpathrectangle{\pgfqpoint{0.800000in}{0.528000in}}{\pgfqpoint{4.960000in}{3.696000in}}%
\pgfusepath{clip}%
\pgfsetbuttcap%
\pgfsetroundjoin%
\definecolor{currentfill}{rgb}{0.000000,0.000000,0.000000}%
\pgfsetfillcolor{currentfill}%
\pgfsetlinewidth{1.003750pt}%
\definecolor{currentstroke}{rgb}{0.000000,0.000000,0.000000}%
\pgfsetstrokecolor{currentstroke}%
\pgfsetdash{}{0pt}%
\pgfpathmoveto{\pgfqpoint{2.518786in}{0.684333in}}%
\pgfpathcurveto{\pgfqpoint{2.529836in}{0.684333in}}{\pgfqpoint{2.540435in}{0.688724in}}{\pgfqpoint{2.548249in}{0.696537in}}%
\pgfpathcurveto{\pgfqpoint{2.556062in}{0.704351in}}{\pgfqpoint{2.560452in}{0.714950in}}{\pgfqpoint{2.560452in}{0.726000in}}%
\pgfpathcurveto{\pgfqpoint{2.560452in}{0.737050in}}{\pgfqpoint{2.556062in}{0.747649in}}{\pgfqpoint{2.548249in}{0.755463in}}%
\pgfpathcurveto{\pgfqpoint{2.540435in}{0.763276in}}{\pgfqpoint{2.529836in}{0.767667in}}{\pgfqpoint{2.518786in}{0.767667in}}%
\pgfpathcurveto{\pgfqpoint{2.507736in}{0.767667in}}{\pgfqpoint{2.497137in}{0.763276in}}{\pgfqpoint{2.489323in}{0.755463in}}%
\pgfpathcurveto{\pgfqpoint{2.481509in}{0.747649in}}{\pgfqpoint{2.477119in}{0.737050in}}{\pgfqpoint{2.477119in}{0.726000in}}%
\pgfpathcurveto{\pgfqpoint{2.477119in}{0.714950in}}{\pgfqpoint{2.481509in}{0.704351in}}{\pgfqpoint{2.489323in}{0.696537in}}%
\pgfpathcurveto{\pgfqpoint{2.497137in}{0.688724in}}{\pgfqpoint{2.507736in}{0.684333in}}{\pgfqpoint{2.518786in}{0.684333in}}%
\pgfpathclose%
\pgfusepath{stroke,fill}%
\end{pgfscope}%
\begin{pgfscope}%
\pgfpathrectangle{\pgfqpoint{0.800000in}{0.528000in}}{\pgfqpoint{4.960000in}{3.696000in}}%
\pgfusepath{clip}%
\pgfsetbuttcap%
\pgfsetroundjoin%
\definecolor{currentfill}{rgb}{0.000000,0.000000,0.000000}%
\pgfsetfillcolor{currentfill}%
\pgfsetlinewidth{1.003750pt}%
\definecolor{currentstroke}{rgb}{0.000000,0.000000,0.000000}%
\pgfsetstrokecolor{currentstroke}%
\pgfsetdash{}{0pt}%
\pgfpathmoveto{\pgfqpoint{2.518786in}{0.684333in}}%
\pgfpathcurveto{\pgfqpoint{2.529836in}{0.684333in}}{\pgfqpoint{2.540435in}{0.688724in}}{\pgfqpoint{2.548249in}{0.696537in}}%
\pgfpathcurveto{\pgfqpoint{2.556062in}{0.704351in}}{\pgfqpoint{2.560452in}{0.714950in}}{\pgfqpoint{2.560452in}{0.726000in}}%
\pgfpathcurveto{\pgfqpoint{2.560452in}{0.737050in}}{\pgfqpoint{2.556062in}{0.747649in}}{\pgfqpoint{2.548249in}{0.755463in}}%
\pgfpathcurveto{\pgfqpoint{2.540435in}{0.763276in}}{\pgfqpoint{2.529836in}{0.767667in}}{\pgfqpoint{2.518786in}{0.767667in}}%
\pgfpathcurveto{\pgfqpoint{2.507736in}{0.767667in}}{\pgfqpoint{2.497137in}{0.763276in}}{\pgfqpoint{2.489323in}{0.755463in}}%
\pgfpathcurveto{\pgfqpoint{2.481509in}{0.747649in}}{\pgfqpoint{2.477119in}{0.737050in}}{\pgfqpoint{2.477119in}{0.726000in}}%
\pgfpathcurveto{\pgfqpoint{2.477119in}{0.714950in}}{\pgfqpoint{2.481509in}{0.704351in}}{\pgfqpoint{2.489323in}{0.696537in}}%
\pgfpathcurveto{\pgfqpoint{2.497137in}{0.688724in}}{\pgfqpoint{2.507736in}{0.684333in}}{\pgfqpoint{2.518786in}{0.684333in}}%
\pgfpathclose%
\pgfusepath{stroke,fill}%
\end{pgfscope}%
\begin{pgfscope}%
\pgfpathrectangle{\pgfqpoint{0.800000in}{0.528000in}}{\pgfqpoint{4.960000in}{3.696000in}}%
\pgfusepath{clip}%
\pgfsetbuttcap%
\pgfsetroundjoin%
\definecolor{currentfill}{rgb}{0.000000,0.000000,0.000000}%
\pgfsetfillcolor{currentfill}%
\pgfsetlinewidth{1.003750pt}%
\definecolor{currentstroke}{rgb}{0.000000,0.000000,0.000000}%
\pgfsetstrokecolor{currentstroke}%
\pgfsetdash{}{0pt}%
\pgfpathmoveto{\pgfqpoint{2.518786in}{0.684333in}}%
\pgfpathcurveto{\pgfqpoint{2.529836in}{0.684333in}}{\pgfqpoint{2.540435in}{0.688724in}}{\pgfqpoint{2.548249in}{0.696537in}}%
\pgfpathcurveto{\pgfqpoint{2.556062in}{0.704351in}}{\pgfqpoint{2.560452in}{0.714950in}}{\pgfqpoint{2.560452in}{0.726000in}}%
\pgfpathcurveto{\pgfqpoint{2.560452in}{0.737050in}}{\pgfqpoint{2.556062in}{0.747649in}}{\pgfqpoint{2.548249in}{0.755463in}}%
\pgfpathcurveto{\pgfqpoint{2.540435in}{0.763276in}}{\pgfqpoint{2.529836in}{0.767667in}}{\pgfqpoint{2.518786in}{0.767667in}}%
\pgfpathcurveto{\pgfqpoint{2.507736in}{0.767667in}}{\pgfqpoint{2.497137in}{0.763276in}}{\pgfqpoint{2.489323in}{0.755463in}}%
\pgfpathcurveto{\pgfqpoint{2.481509in}{0.747649in}}{\pgfqpoint{2.477119in}{0.737050in}}{\pgfqpoint{2.477119in}{0.726000in}}%
\pgfpathcurveto{\pgfqpoint{2.477119in}{0.714950in}}{\pgfqpoint{2.481509in}{0.704351in}}{\pgfqpoint{2.489323in}{0.696537in}}%
\pgfpathcurveto{\pgfqpoint{2.497137in}{0.688724in}}{\pgfqpoint{2.507736in}{0.684333in}}{\pgfqpoint{2.518786in}{0.684333in}}%
\pgfpathclose%
\pgfusepath{stroke,fill}%
\end{pgfscope}%
\begin{pgfscope}%
\pgfpathrectangle{\pgfqpoint{0.800000in}{0.528000in}}{\pgfqpoint{4.960000in}{3.696000in}}%
\pgfusepath{clip}%
\pgfsetbuttcap%
\pgfsetroundjoin%
\definecolor{currentfill}{rgb}{0.000000,0.000000,0.000000}%
\pgfsetfillcolor{currentfill}%
\pgfsetlinewidth{1.003750pt}%
\definecolor{currentstroke}{rgb}{0.000000,0.000000,0.000000}%
\pgfsetstrokecolor{currentstroke}%
\pgfsetdash{}{0pt}%
\pgfpathmoveto{\pgfqpoint{2.518786in}{0.684333in}}%
\pgfpathcurveto{\pgfqpoint{2.529836in}{0.684333in}}{\pgfqpoint{2.540435in}{0.688724in}}{\pgfqpoint{2.548249in}{0.696537in}}%
\pgfpathcurveto{\pgfqpoint{2.556062in}{0.704351in}}{\pgfqpoint{2.560452in}{0.714950in}}{\pgfqpoint{2.560452in}{0.726000in}}%
\pgfpathcurveto{\pgfqpoint{2.560452in}{0.737050in}}{\pgfqpoint{2.556062in}{0.747649in}}{\pgfqpoint{2.548249in}{0.755463in}}%
\pgfpathcurveto{\pgfqpoint{2.540435in}{0.763276in}}{\pgfqpoint{2.529836in}{0.767667in}}{\pgfqpoint{2.518786in}{0.767667in}}%
\pgfpathcurveto{\pgfqpoint{2.507736in}{0.767667in}}{\pgfqpoint{2.497137in}{0.763276in}}{\pgfqpoint{2.489323in}{0.755463in}}%
\pgfpathcurveto{\pgfqpoint{2.481509in}{0.747649in}}{\pgfqpoint{2.477119in}{0.737050in}}{\pgfqpoint{2.477119in}{0.726000in}}%
\pgfpathcurveto{\pgfqpoint{2.477119in}{0.714950in}}{\pgfqpoint{2.481509in}{0.704351in}}{\pgfqpoint{2.489323in}{0.696537in}}%
\pgfpathcurveto{\pgfqpoint{2.497137in}{0.688724in}}{\pgfqpoint{2.507736in}{0.684333in}}{\pgfqpoint{2.518786in}{0.684333in}}%
\pgfpathclose%
\pgfusepath{stroke,fill}%
\end{pgfscope}%
\begin{pgfscope}%
\pgfpathrectangle{\pgfqpoint{0.800000in}{0.528000in}}{\pgfqpoint{4.960000in}{3.696000in}}%
\pgfusepath{clip}%
\pgfsetbuttcap%
\pgfsetroundjoin%
\definecolor{currentfill}{rgb}{0.000000,0.000000,0.000000}%
\pgfsetfillcolor{currentfill}%
\pgfsetlinewidth{1.003750pt}%
\definecolor{currentstroke}{rgb}{0.000000,0.000000,0.000000}%
\pgfsetstrokecolor{currentstroke}%
\pgfsetdash{}{0pt}%
\pgfpathmoveto{\pgfqpoint{2.518786in}{0.684333in}}%
\pgfpathcurveto{\pgfqpoint{2.529836in}{0.684333in}}{\pgfqpoint{2.540435in}{0.688724in}}{\pgfqpoint{2.548249in}{0.696537in}}%
\pgfpathcurveto{\pgfqpoint{2.556062in}{0.704351in}}{\pgfqpoint{2.560452in}{0.714950in}}{\pgfqpoint{2.560452in}{0.726000in}}%
\pgfpathcurveto{\pgfqpoint{2.560452in}{0.737050in}}{\pgfqpoint{2.556062in}{0.747649in}}{\pgfqpoint{2.548249in}{0.755463in}}%
\pgfpathcurveto{\pgfqpoint{2.540435in}{0.763276in}}{\pgfqpoint{2.529836in}{0.767667in}}{\pgfqpoint{2.518786in}{0.767667in}}%
\pgfpathcurveto{\pgfqpoint{2.507736in}{0.767667in}}{\pgfqpoint{2.497137in}{0.763276in}}{\pgfqpoint{2.489323in}{0.755463in}}%
\pgfpathcurveto{\pgfqpoint{2.481509in}{0.747649in}}{\pgfqpoint{2.477119in}{0.737050in}}{\pgfqpoint{2.477119in}{0.726000in}}%
\pgfpathcurveto{\pgfqpoint{2.477119in}{0.714950in}}{\pgfqpoint{2.481509in}{0.704351in}}{\pgfqpoint{2.489323in}{0.696537in}}%
\pgfpathcurveto{\pgfqpoint{2.497137in}{0.688724in}}{\pgfqpoint{2.507736in}{0.684333in}}{\pgfqpoint{2.518786in}{0.684333in}}%
\pgfpathclose%
\pgfusepath{stroke,fill}%
\end{pgfscope}%
\begin{pgfscope}%
\pgfpathrectangle{\pgfqpoint{0.800000in}{0.528000in}}{\pgfqpoint{4.960000in}{3.696000in}}%
\pgfusepath{clip}%
\pgfsetbuttcap%
\pgfsetroundjoin%
\definecolor{currentfill}{rgb}{0.000000,0.000000,0.000000}%
\pgfsetfillcolor{currentfill}%
\pgfsetlinewidth{1.003750pt}%
\definecolor{currentstroke}{rgb}{0.000000,0.000000,0.000000}%
\pgfsetstrokecolor{currentstroke}%
\pgfsetdash{}{0pt}%
\pgfpathmoveto{\pgfqpoint{2.518786in}{0.684333in}}%
\pgfpathcurveto{\pgfqpoint{2.529836in}{0.684333in}}{\pgfqpoint{2.540435in}{0.688724in}}{\pgfqpoint{2.548249in}{0.696537in}}%
\pgfpathcurveto{\pgfqpoint{2.556062in}{0.704351in}}{\pgfqpoint{2.560452in}{0.714950in}}{\pgfqpoint{2.560452in}{0.726000in}}%
\pgfpathcurveto{\pgfqpoint{2.560452in}{0.737050in}}{\pgfqpoint{2.556062in}{0.747649in}}{\pgfqpoint{2.548249in}{0.755463in}}%
\pgfpathcurveto{\pgfqpoint{2.540435in}{0.763276in}}{\pgfqpoint{2.529836in}{0.767667in}}{\pgfqpoint{2.518786in}{0.767667in}}%
\pgfpathcurveto{\pgfqpoint{2.507736in}{0.767667in}}{\pgfqpoint{2.497137in}{0.763276in}}{\pgfqpoint{2.489323in}{0.755463in}}%
\pgfpathcurveto{\pgfqpoint{2.481509in}{0.747649in}}{\pgfqpoint{2.477119in}{0.737050in}}{\pgfqpoint{2.477119in}{0.726000in}}%
\pgfpathcurveto{\pgfqpoint{2.477119in}{0.714950in}}{\pgfqpoint{2.481509in}{0.704351in}}{\pgfqpoint{2.489323in}{0.696537in}}%
\pgfpathcurveto{\pgfqpoint{2.497137in}{0.688724in}}{\pgfqpoint{2.507736in}{0.684333in}}{\pgfqpoint{2.518786in}{0.684333in}}%
\pgfpathclose%
\pgfusepath{stroke,fill}%
\end{pgfscope}%
\begin{pgfscope}%
\pgfpathrectangle{\pgfqpoint{0.800000in}{0.528000in}}{\pgfqpoint{4.960000in}{3.696000in}}%
\pgfusepath{clip}%
\pgfsetbuttcap%
\pgfsetroundjoin%
\definecolor{currentfill}{rgb}{0.000000,0.000000,0.000000}%
\pgfsetfillcolor{currentfill}%
\pgfsetlinewidth{1.003750pt}%
\definecolor{currentstroke}{rgb}{0.000000,0.000000,0.000000}%
\pgfsetstrokecolor{currentstroke}%
\pgfsetdash{}{0pt}%
\pgfpathmoveto{\pgfqpoint{2.518786in}{0.684333in}}%
\pgfpathcurveto{\pgfqpoint{2.529836in}{0.684333in}}{\pgfqpoint{2.540435in}{0.688724in}}{\pgfqpoint{2.548249in}{0.696537in}}%
\pgfpathcurveto{\pgfqpoint{2.556062in}{0.704351in}}{\pgfqpoint{2.560452in}{0.714950in}}{\pgfqpoint{2.560452in}{0.726000in}}%
\pgfpathcurveto{\pgfqpoint{2.560452in}{0.737050in}}{\pgfqpoint{2.556062in}{0.747649in}}{\pgfqpoint{2.548249in}{0.755463in}}%
\pgfpathcurveto{\pgfqpoint{2.540435in}{0.763276in}}{\pgfqpoint{2.529836in}{0.767667in}}{\pgfqpoint{2.518786in}{0.767667in}}%
\pgfpathcurveto{\pgfqpoint{2.507736in}{0.767667in}}{\pgfqpoint{2.497137in}{0.763276in}}{\pgfqpoint{2.489323in}{0.755463in}}%
\pgfpathcurveto{\pgfqpoint{2.481509in}{0.747649in}}{\pgfqpoint{2.477119in}{0.737050in}}{\pgfqpoint{2.477119in}{0.726000in}}%
\pgfpathcurveto{\pgfqpoint{2.477119in}{0.714950in}}{\pgfqpoint{2.481509in}{0.704351in}}{\pgfqpoint{2.489323in}{0.696537in}}%
\pgfpathcurveto{\pgfqpoint{2.497137in}{0.688724in}}{\pgfqpoint{2.507736in}{0.684333in}}{\pgfqpoint{2.518786in}{0.684333in}}%
\pgfpathclose%
\pgfusepath{stroke,fill}%
\end{pgfscope}%
\begin{pgfscope}%
\pgfpathrectangle{\pgfqpoint{0.800000in}{0.528000in}}{\pgfqpoint{4.960000in}{3.696000in}}%
\pgfusepath{clip}%
\pgfsetbuttcap%
\pgfsetroundjoin%
\definecolor{currentfill}{rgb}{0.000000,0.000000,0.000000}%
\pgfsetfillcolor{currentfill}%
\pgfsetlinewidth{1.003750pt}%
\definecolor{currentstroke}{rgb}{0.000000,0.000000,0.000000}%
\pgfsetstrokecolor{currentstroke}%
\pgfsetdash{}{0pt}%
\pgfpathmoveto{\pgfqpoint{2.518786in}{0.684333in}}%
\pgfpathcurveto{\pgfqpoint{2.529836in}{0.684333in}}{\pgfqpoint{2.540435in}{0.688724in}}{\pgfqpoint{2.548249in}{0.696537in}}%
\pgfpathcurveto{\pgfqpoint{2.556062in}{0.704351in}}{\pgfqpoint{2.560452in}{0.714950in}}{\pgfqpoint{2.560452in}{0.726000in}}%
\pgfpathcurveto{\pgfqpoint{2.560452in}{0.737050in}}{\pgfqpoint{2.556062in}{0.747649in}}{\pgfqpoint{2.548249in}{0.755463in}}%
\pgfpathcurveto{\pgfqpoint{2.540435in}{0.763276in}}{\pgfqpoint{2.529836in}{0.767667in}}{\pgfqpoint{2.518786in}{0.767667in}}%
\pgfpathcurveto{\pgfqpoint{2.507736in}{0.767667in}}{\pgfqpoint{2.497137in}{0.763276in}}{\pgfqpoint{2.489323in}{0.755463in}}%
\pgfpathcurveto{\pgfqpoint{2.481509in}{0.747649in}}{\pgfqpoint{2.477119in}{0.737050in}}{\pgfqpoint{2.477119in}{0.726000in}}%
\pgfpathcurveto{\pgfqpoint{2.477119in}{0.714950in}}{\pgfqpoint{2.481509in}{0.704351in}}{\pgfqpoint{2.489323in}{0.696537in}}%
\pgfpathcurveto{\pgfqpoint{2.497137in}{0.688724in}}{\pgfqpoint{2.507736in}{0.684333in}}{\pgfqpoint{2.518786in}{0.684333in}}%
\pgfpathclose%
\pgfusepath{stroke,fill}%
\end{pgfscope}%
\begin{pgfscope}%
\pgfpathrectangle{\pgfqpoint{0.800000in}{0.528000in}}{\pgfqpoint{4.960000in}{3.696000in}}%
\pgfusepath{clip}%
\pgfsetbuttcap%
\pgfsetroundjoin%
\definecolor{currentfill}{rgb}{0.000000,0.000000,0.000000}%
\pgfsetfillcolor{currentfill}%
\pgfsetlinewidth{1.003750pt}%
\definecolor{currentstroke}{rgb}{0.000000,0.000000,0.000000}%
\pgfsetstrokecolor{currentstroke}%
\pgfsetdash{}{0pt}%
\pgfpathmoveto{\pgfqpoint{2.518786in}{0.684333in}}%
\pgfpathcurveto{\pgfqpoint{2.529836in}{0.684333in}}{\pgfqpoint{2.540435in}{0.688724in}}{\pgfqpoint{2.548249in}{0.696537in}}%
\pgfpathcurveto{\pgfqpoint{2.556062in}{0.704351in}}{\pgfqpoint{2.560452in}{0.714950in}}{\pgfqpoint{2.560452in}{0.726000in}}%
\pgfpathcurveto{\pgfqpoint{2.560452in}{0.737050in}}{\pgfqpoint{2.556062in}{0.747649in}}{\pgfqpoint{2.548249in}{0.755463in}}%
\pgfpathcurveto{\pgfqpoint{2.540435in}{0.763276in}}{\pgfqpoint{2.529836in}{0.767667in}}{\pgfqpoint{2.518786in}{0.767667in}}%
\pgfpathcurveto{\pgfqpoint{2.507736in}{0.767667in}}{\pgfqpoint{2.497137in}{0.763276in}}{\pgfqpoint{2.489323in}{0.755463in}}%
\pgfpathcurveto{\pgfqpoint{2.481509in}{0.747649in}}{\pgfqpoint{2.477119in}{0.737050in}}{\pgfqpoint{2.477119in}{0.726000in}}%
\pgfpathcurveto{\pgfqpoint{2.477119in}{0.714950in}}{\pgfqpoint{2.481509in}{0.704351in}}{\pgfqpoint{2.489323in}{0.696537in}}%
\pgfpathcurveto{\pgfqpoint{2.497137in}{0.688724in}}{\pgfqpoint{2.507736in}{0.684333in}}{\pgfqpoint{2.518786in}{0.684333in}}%
\pgfpathclose%
\pgfusepath{stroke,fill}%
\end{pgfscope}%
\begin{pgfscope}%
\pgfpathrectangle{\pgfqpoint{0.800000in}{0.528000in}}{\pgfqpoint{4.960000in}{3.696000in}}%
\pgfusepath{clip}%
\pgfsetbuttcap%
\pgfsetroundjoin%
\definecolor{currentfill}{rgb}{0.000000,0.000000,0.000000}%
\pgfsetfillcolor{currentfill}%
\pgfsetlinewidth{1.003750pt}%
\definecolor{currentstroke}{rgb}{0.000000,0.000000,0.000000}%
\pgfsetstrokecolor{currentstroke}%
\pgfsetdash{}{0pt}%
\pgfpathmoveto{\pgfqpoint{2.518786in}{0.684333in}}%
\pgfpathcurveto{\pgfqpoint{2.529836in}{0.684333in}}{\pgfqpoint{2.540435in}{0.688724in}}{\pgfqpoint{2.548249in}{0.696537in}}%
\pgfpathcurveto{\pgfqpoint{2.556062in}{0.704351in}}{\pgfqpoint{2.560452in}{0.714950in}}{\pgfqpoint{2.560452in}{0.726000in}}%
\pgfpathcurveto{\pgfqpoint{2.560452in}{0.737050in}}{\pgfqpoint{2.556062in}{0.747649in}}{\pgfqpoint{2.548249in}{0.755463in}}%
\pgfpathcurveto{\pgfqpoint{2.540435in}{0.763276in}}{\pgfqpoint{2.529836in}{0.767667in}}{\pgfqpoint{2.518786in}{0.767667in}}%
\pgfpathcurveto{\pgfqpoint{2.507736in}{0.767667in}}{\pgfqpoint{2.497137in}{0.763276in}}{\pgfqpoint{2.489323in}{0.755463in}}%
\pgfpathcurveto{\pgfqpoint{2.481509in}{0.747649in}}{\pgfqpoint{2.477119in}{0.737050in}}{\pgfqpoint{2.477119in}{0.726000in}}%
\pgfpathcurveto{\pgfqpoint{2.477119in}{0.714950in}}{\pgfqpoint{2.481509in}{0.704351in}}{\pgfqpoint{2.489323in}{0.696537in}}%
\pgfpathcurveto{\pgfqpoint{2.497137in}{0.688724in}}{\pgfqpoint{2.507736in}{0.684333in}}{\pgfqpoint{2.518786in}{0.684333in}}%
\pgfpathclose%
\pgfusepath{stroke,fill}%
\end{pgfscope}%
\begin{pgfscope}%
\pgfpathrectangle{\pgfqpoint{0.800000in}{0.528000in}}{\pgfqpoint{4.960000in}{3.696000in}}%
\pgfusepath{clip}%
\pgfsetbuttcap%
\pgfsetroundjoin%
\definecolor{currentfill}{rgb}{0.000000,0.000000,0.000000}%
\pgfsetfillcolor{currentfill}%
\pgfsetlinewidth{1.003750pt}%
\definecolor{currentstroke}{rgb}{0.000000,0.000000,0.000000}%
\pgfsetstrokecolor{currentstroke}%
\pgfsetdash{}{0pt}%
\pgfpathmoveto{\pgfqpoint{2.518786in}{0.684333in}}%
\pgfpathcurveto{\pgfqpoint{2.529836in}{0.684333in}}{\pgfqpoint{2.540435in}{0.688724in}}{\pgfqpoint{2.548249in}{0.696537in}}%
\pgfpathcurveto{\pgfqpoint{2.556062in}{0.704351in}}{\pgfqpoint{2.560452in}{0.714950in}}{\pgfqpoint{2.560452in}{0.726000in}}%
\pgfpathcurveto{\pgfqpoint{2.560452in}{0.737050in}}{\pgfqpoint{2.556062in}{0.747649in}}{\pgfqpoint{2.548249in}{0.755463in}}%
\pgfpathcurveto{\pgfqpoint{2.540435in}{0.763276in}}{\pgfqpoint{2.529836in}{0.767667in}}{\pgfqpoint{2.518786in}{0.767667in}}%
\pgfpathcurveto{\pgfqpoint{2.507736in}{0.767667in}}{\pgfqpoint{2.497137in}{0.763276in}}{\pgfqpoint{2.489323in}{0.755463in}}%
\pgfpathcurveto{\pgfqpoint{2.481509in}{0.747649in}}{\pgfqpoint{2.477119in}{0.737050in}}{\pgfqpoint{2.477119in}{0.726000in}}%
\pgfpathcurveto{\pgfqpoint{2.477119in}{0.714950in}}{\pgfqpoint{2.481509in}{0.704351in}}{\pgfqpoint{2.489323in}{0.696537in}}%
\pgfpathcurveto{\pgfqpoint{2.497137in}{0.688724in}}{\pgfqpoint{2.507736in}{0.684333in}}{\pgfqpoint{2.518786in}{0.684333in}}%
\pgfpathclose%
\pgfusepath{stroke,fill}%
\end{pgfscope}%
\begin{pgfscope}%
\pgfpathrectangle{\pgfqpoint{0.800000in}{0.528000in}}{\pgfqpoint{4.960000in}{3.696000in}}%
\pgfusepath{clip}%
\pgfsetbuttcap%
\pgfsetroundjoin%
\definecolor{currentfill}{rgb}{0.000000,0.000000,0.000000}%
\pgfsetfillcolor{currentfill}%
\pgfsetlinewidth{1.003750pt}%
\definecolor{currentstroke}{rgb}{0.000000,0.000000,0.000000}%
\pgfsetstrokecolor{currentstroke}%
\pgfsetdash{}{0pt}%
\pgfpathmoveto{\pgfqpoint{2.518786in}{0.684333in}}%
\pgfpathcurveto{\pgfqpoint{2.529836in}{0.684333in}}{\pgfqpoint{2.540435in}{0.688724in}}{\pgfqpoint{2.548249in}{0.696537in}}%
\pgfpathcurveto{\pgfqpoint{2.556062in}{0.704351in}}{\pgfqpoint{2.560452in}{0.714950in}}{\pgfqpoint{2.560452in}{0.726000in}}%
\pgfpathcurveto{\pgfqpoint{2.560452in}{0.737050in}}{\pgfqpoint{2.556062in}{0.747649in}}{\pgfqpoint{2.548249in}{0.755463in}}%
\pgfpathcurveto{\pgfqpoint{2.540435in}{0.763276in}}{\pgfqpoint{2.529836in}{0.767667in}}{\pgfqpoint{2.518786in}{0.767667in}}%
\pgfpathcurveto{\pgfqpoint{2.507736in}{0.767667in}}{\pgfqpoint{2.497137in}{0.763276in}}{\pgfqpoint{2.489323in}{0.755463in}}%
\pgfpathcurveto{\pgfqpoint{2.481509in}{0.747649in}}{\pgfqpoint{2.477119in}{0.737050in}}{\pgfqpoint{2.477119in}{0.726000in}}%
\pgfpathcurveto{\pgfqpoint{2.477119in}{0.714950in}}{\pgfqpoint{2.481509in}{0.704351in}}{\pgfqpoint{2.489323in}{0.696537in}}%
\pgfpathcurveto{\pgfqpoint{2.497137in}{0.688724in}}{\pgfqpoint{2.507736in}{0.684333in}}{\pgfqpoint{2.518786in}{0.684333in}}%
\pgfpathclose%
\pgfusepath{stroke,fill}%
\end{pgfscope}%
\begin{pgfscope}%
\pgfpathrectangle{\pgfqpoint{0.800000in}{0.528000in}}{\pgfqpoint{4.960000in}{3.696000in}}%
\pgfusepath{clip}%
\pgfsetbuttcap%
\pgfsetroundjoin%
\definecolor{currentfill}{rgb}{0.000000,0.000000,0.000000}%
\pgfsetfillcolor{currentfill}%
\pgfsetlinewidth{1.003750pt}%
\definecolor{currentstroke}{rgb}{0.000000,0.000000,0.000000}%
\pgfsetstrokecolor{currentstroke}%
\pgfsetdash{}{0pt}%
\pgfpathmoveto{\pgfqpoint{2.518786in}{0.684333in}}%
\pgfpathcurveto{\pgfqpoint{2.529836in}{0.684333in}}{\pgfqpoint{2.540435in}{0.688724in}}{\pgfqpoint{2.548249in}{0.696537in}}%
\pgfpathcurveto{\pgfqpoint{2.556062in}{0.704351in}}{\pgfqpoint{2.560452in}{0.714950in}}{\pgfqpoint{2.560452in}{0.726000in}}%
\pgfpathcurveto{\pgfqpoint{2.560452in}{0.737050in}}{\pgfqpoint{2.556062in}{0.747649in}}{\pgfqpoint{2.548249in}{0.755463in}}%
\pgfpathcurveto{\pgfqpoint{2.540435in}{0.763276in}}{\pgfqpoint{2.529836in}{0.767667in}}{\pgfqpoint{2.518786in}{0.767667in}}%
\pgfpathcurveto{\pgfqpoint{2.507736in}{0.767667in}}{\pgfqpoint{2.497137in}{0.763276in}}{\pgfqpoint{2.489323in}{0.755463in}}%
\pgfpathcurveto{\pgfqpoint{2.481509in}{0.747649in}}{\pgfqpoint{2.477119in}{0.737050in}}{\pgfqpoint{2.477119in}{0.726000in}}%
\pgfpathcurveto{\pgfqpoint{2.477119in}{0.714950in}}{\pgfqpoint{2.481509in}{0.704351in}}{\pgfqpoint{2.489323in}{0.696537in}}%
\pgfpathcurveto{\pgfqpoint{2.497137in}{0.688724in}}{\pgfqpoint{2.507736in}{0.684333in}}{\pgfqpoint{2.518786in}{0.684333in}}%
\pgfpathclose%
\pgfusepath{stroke,fill}%
\end{pgfscope}%
\begin{pgfscope}%
\pgfpathrectangle{\pgfqpoint{0.800000in}{0.528000in}}{\pgfqpoint{4.960000in}{3.696000in}}%
\pgfusepath{clip}%
\pgfsetbuttcap%
\pgfsetroundjoin%
\definecolor{currentfill}{rgb}{0.000000,0.000000,0.000000}%
\pgfsetfillcolor{currentfill}%
\pgfsetlinewidth{1.003750pt}%
\definecolor{currentstroke}{rgb}{0.000000,0.000000,0.000000}%
\pgfsetstrokecolor{currentstroke}%
\pgfsetdash{}{0pt}%
\pgfpathmoveto{\pgfqpoint{2.518786in}{0.684333in}}%
\pgfpathcurveto{\pgfqpoint{2.529836in}{0.684333in}}{\pgfqpoint{2.540435in}{0.688724in}}{\pgfqpoint{2.548249in}{0.696537in}}%
\pgfpathcurveto{\pgfqpoint{2.556062in}{0.704351in}}{\pgfqpoint{2.560452in}{0.714950in}}{\pgfqpoint{2.560452in}{0.726000in}}%
\pgfpathcurveto{\pgfqpoint{2.560452in}{0.737050in}}{\pgfqpoint{2.556062in}{0.747649in}}{\pgfqpoint{2.548249in}{0.755463in}}%
\pgfpathcurveto{\pgfqpoint{2.540435in}{0.763276in}}{\pgfqpoint{2.529836in}{0.767667in}}{\pgfqpoint{2.518786in}{0.767667in}}%
\pgfpathcurveto{\pgfqpoint{2.507736in}{0.767667in}}{\pgfqpoint{2.497137in}{0.763276in}}{\pgfqpoint{2.489323in}{0.755463in}}%
\pgfpathcurveto{\pgfqpoint{2.481509in}{0.747649in}}{\pgfqpoint{2.477119in}{0.737050in}}{\pgfqpoint{2.477119in}{0.726000in}}%
\pgfpathcurveto{\pgfqpoint{2.477119in}{0.714950in}}{\pgfqpoint{2.481509in}{0.704351in}}{\pgfqpoint{2.489323in}{0.696537in}}%
\pgfpathcurveto{\pgfqpoint{2.497137in}{0.688724in}}{\pgfqpoint{2.507736in}{0.684333in}}{\pgfqpoint{2.518786in}{0.684333in}}%
\pgfpathclose%
\pgfusepath{stroke,fill}%
\end{pgfscope}%
\begin{pgfscope}%
\pgfpathrectangle{\pgfqpoint{0.800000in}{0.528000in}}{\pgfqpoint{4.960000in}{3.696000in}}%
\pgfusepath{clip}%
\pgfsetbuttcap%
\pgfsetroundjoin%
\definecolor{currentfill}{rgb}{0.000000,0.000000,0.000000}%
\pgfsetfillcolor{currentfill}%
\pgfsetlinewidth{1.003750pt}%
\definecolor{currentstroke}{rgb}{0.000000,0.000000,0.000000}%
\pgfsetstrokecolor{currentstroke}%
\pgfsetdash{}{0pt}%
\pgfpathmoveto{\pgfqpoint{2.518786in}{0.684333in}}%
\pgfpathcurveto{\pgfqpoint{2.529836in}{0.684333in}}{\pgfqpoint{2.540435in}{0.688724in}}{\pgfqpoint{2.548249in}{0.696537in}}%
\pgfpathcurveto{\pgfqpoint{2.556062in}{0.704351in}}{\pgfqpoint{2.560452in}{0.714950in}}{\pgfqpoint{2.560452in}{0.726000in}}%
\pgfpathcurveto{\pgfqpoint{2.560452in}{0.737050in}}{\pgfqpoint{2.556062in}{0.747649in}}{\pgfqpoint{2.548249in}{0.755463in}}%
\pgfpathcurveto{\pgfqpoint{2.540435in}{0.763276in}}{\pgfqpoint{2.529836in}{0.767667in}}{\pgfqpoint{2.518786in}{0.767667in}}%
\pgfpathcurveto{\pgfqpoint{2.507736in}{0.767667in}}{\pgfqpoint{2.497137in}{0.763276in}}{\pgfqpoint{2.489323in}{0.755463in}}%
\pgfpathcurveto{\pgfqpoint{2.481509in}{0.747649in}}{\pgfqpoint{2.477119in}{0.737050in}}{\pgfqpoint{2.477119in}{0.726000in}}%
\pgfpathcurveto{\pgfqpoint{2.477119in}{0.714950in}}{\pgfqpoint{2.481509in}{0.704351in}}{\pgfqpoint{2.489323in}{0.696537in}}%
\pgfpathcurveto{\pgfqpoint{2.497137in}{0.688724in}}{\pgfqpoint{2.507736in}{0.684333in}}{\pgfqpoint{2.518786in}{0.684333in}}%
\pgfpathclose%
\pgfusepath{stroke,fill}%
\end{pgfscope}%
\begin{pgfscope}%
\pgfpathrectangle{\pgfqpoint{0.800000in}{0.528000in}}{\pgfqpoint{4.960000in}{3.696000in}}%
\pgfusepath{clip}%
\pgfsetbuttcap%
\pgfsetroundjoin%
\definecolor{currentfill}{rgb}{0.000000,0.000000,0.000000}%
\pgfsetfillcolor{currentfill}%
\pgfsetlinewidth{1.003750pt}%
\definecolor{currentstroke}{rgb}{0.000000,0.000000,0.000000}%
\pgfsetstrokecolor{currentstroke}%
\pgfsetdash{}{0pt}%
\pgfpathmoveto{\pgfqpoint{2.518786in}{0.684333in}}%
\pgfpathcurveto{\pgfqpoint{2.529836in}{0.684333in}}{\pgfqpoint{2.540435in}{0.688724in}}{\pgfqpoint{2.548249in}{0.696537in}}%
\pgfpathcurveto{\pgfqpoint{2.556062in}{0.704351in}}{\pgfqpoint{2.560452in}{0.714950in}}{\pgfqpoint{2.560452in}{0.726000in}}%
\pgfpathcurveto{\pgfqpoint{2.560452in}{0.737050in}}{\pgfqpoint{2.556062in}{0.747649in}}{\pgfqpoint{2.548249in}{0.755463in}}%
\pgfpathcurveto{\pgfqpoint{2.540435in}{0.763276in}}{\pgfqpoint{2.529836in}{0.767667in}}{\pgfqpoint{2.518786in}{0.767667in}}%
\pgfpathcurveto{\pgfqpoint{2.507736in}{0.767667in}}{\pgfqpoint{2.497137in}{0.763276in}}{\pgfqpoint{2.489323in}{0.755463in}}%
\pgfpathcurveto{\pgfqpoint{2.481509in}{0.747649in}}{\pgfqpoint{2.477119in}{0.737050in}}{\pgfqpoint{2.477119in}{0.726000in}}%
\pgfpathcurveto{\pgfqpoint{2.477119in}{0.714950in}}{\pgfqpoint{2.481509in}{0.704351in}}{\pgfqpoint{2.489323in}{0.696537in}}%
\pgfpathcurveto{\pgfqpoint{2.497137in}{0.688724in}}{\pgfqpoint{2.507736in}{0.684333in}}{\pgfqpoint{2.518786in}{0.684333in}}%
\pgfpathclose%
\pgfusepath{stroke,fill}%
\end{pgfscope}%
\begin{pgfscope}%
\pgfpathrectangle{\pgfqpoint{0.800000in}{0.528000in}}{\pgfqpoint{4.960000in}{3.696000in}}%
\pgfusepath{clip}%
\pgfsetbuttcap%
\pgfsetroundjoin%
\definecolor{currentfill}{rgb}{0.000000,0.000000,0.000000}%
\pgfsetfillcolor{currentfill}%
\pgfsetlinewidth{1.003750pt}%
\definecolor{currentstroke}{rgb}{0.000000,0.000000,0.000000}%
\pgfsetstrokecolor{currentstroke}%
\pgfsetdash{}{0pt}%
\pgfpathmoveto{\pgfqpoint{2.518786in}{0.684333in}}%
\pgfpathcurveto{\pgfqpoint{2.529836in}{0.684333in}}{\pgfqpoint{2.540435in}{0.688724in}}{\pgfqpoint{2.548249in}{0.696537in}}%
\pgfpathcurveto{\pgfqpoint{2.556062in}{0.704351in}}{\pgfqpoint{2.560452in}{0.714950in}}{\pgfqpoint{2.560452in}{0.726000in}}%
\pgfpathcurveto{\pgfqpoint{2.560452in}{0.737050in}}{\pgfqpoint{2.556062in}{0.747649in}}{\pgfqpoint{2.548249in}{0.755463in}}%
\pgfpathcurveto{\pgfqpoint{2.540435in}{0.763276in}}{\pgfqpoint{2.529836in}{0.767667in}}{\pgfqpoint{2.518786in}{0.767667in}}%
\pgfpathcurveto{\pgfqpoint{2.507736in}{0.767667in}}{\pgfqpoint{2.497137in}{0.763276in}}{\pgfqpoint{2.489323in}{0.755463in}}%
\pgfpathcurveto{\pgfqpoint{2.481509in}{0.747649in}}{\pgfqpoint{2.477119in}{0.737050in}}{\pgfqpoint{2.477119in}{0.726000in}}%
\pgfpathcurveto{\pgfqpoint{2.477119in}{0.714950in}}{\pgfqpoint{2.481509in}{0.704351in}}{\pgfqpoint{2.489323in}{0.696537in}}%
\pgfpathcurveto{\pgfqpoint{2.497137in}{0.688724in}}{\pgfqpoint{2.507736in}{0.684333in}}{\pgfqpoint{2.518786in}{0.684333in}}%
\pgfpathclose%
\pgfusepath{stroke,fill}%
\end{pgfscope}%
\begin{pgfscope}%
\pgfpathrectangle{\pgfqpoint{0.800000in}{0.528000in}}{\pgfqpoint{4.960000in}{3.696000in}}%
\pgfusepath{clip}%
\pgfsetbuttcap%
\pgfsetroundjoin%
\definecolor{currentfill}{rgb}{0.000000,0.000000,0.000000}%
\pgfsetfillcolor{currentfill}%
\pgfsetlinewidth{1.003750pt}%
\definecolor{currentstroke}{rgb}{0.000000,0.000000,0.000000}%
\pgfsetstrokecolor{currentstroke}%
\pgfsetdash{}{0pt}%
\pgfpathmoveto{\pgfqpoint{2.518786in}{0.684333in}}%
\pgfpathcurveto{\pgfqpoint{2.529836in}{0.684333in}}{\pgfqpoint{2.540435in}{0.688724in}}{\pgfqpoint{2.548249in}{0.696537in}}%
\pgfpathcurveto{\pgfqpoint{2.556062in}{0.704351in}}{\pgfqpoint{2.560452in}{0.714950in}}{\pgfqpoint{2.560452in}{0.726000in}}%
\pgfpathcurveto{\pgfqpoint{2.560452in}{0.737050in}}{\pgfqpoint{2.556062in}{0.747649in}}{\pgfqpoint{2.548249in}{0.755463in}}%
\pgfpathcurveto{\pgfqpoint{2.540435in}{0.763276in}}{\pgfqpoint{2.529836in}{0.767667in}}{\pgfqpoint{2.518786in}{0.767667in}}%
\pgfpathcurveto{\pgfqpoint{2.507736in}{0.767667in}}{\pgfqpoint{2.497137in}{0.763276in}}{\pgfqpoint{2.489323in}{0.755463in}}%
\pgfpathcurveto{\pgfqpoint{2.481509in}{0.747649in}}{\pgfqpoint{2.477119in}{0.737050in}}{\pgfqpoint{2.477119in}{0.726000in}}%
\pgfpathcurveto{\pgfqpoint{2.477119in}{0.714950in}}{\pgfqpoint{2.481509in}{0.704351in}}{\pgfqpoint{2.489323in}{0.696537in}}%
\pgfpathcurveto{\pgfqpoint{2.497137in}{0.688724in}}{\pgfqpoint{2.507736in}{0.684333in}}{\pgfqpoint{2.518786in}{0.684333in}}%
\pgfpathclose%
\pgfusepath{stroke,fill}%
\end{pgfscope}%
\begin{pgfscope}%
\pgfpathrectangle{\pgfqpoint{0.800000in}{0.528000in}}{\pgfqpoint{4.960000in}{3.696000in}}%
\pgfusepath{clip}%
\pgfsetbuttcap%
\pgfsetroundjoin%
\definecolor{currentfill}{rgb}{0.000000,0.000000,0.000000}%
\pgfsetfillcolor{currentfill}%
\pgfsetlinewidth{1.003750pt}%
\definecolor{currentstroke}{rgb}{0.000000,0.000000,0.000000}%
\pgfsetstrokecolor{currentstroke}%
\pgfsetdash{}{0pt}%
\pgfpathmoveto{\pgfqpoint{2.518786in}{0.684333in}}%
\pgfpathcurveto{\pgfqpoint{2.529836in}{0.684333in}}{\pgfqpoint{2.540435in}{0.688724in}}{\pgfqpoint{2.548249in}{0.696537in}}%
\pgfpathcurveto{\pgfqpoint{2.556062in}{0.704351in}}{\pgfqpoint{2.560452in}{0.714950in}}{\pgfqpoint{2.560452in}{0.726000in}}%
\pgfpathcurveto{\pgfqpoint{2.560452in}{0.737050in}}{\pgfqpoint{2.556062in}{0.747649in}}{\pgfqpoint{2.548249in}{0.755463in}}%
\pgfpathcurveto{\pgfqpoint{2.540435in}{0.763276in}}{\pgfqpoint{2.529836in}{0.767667in}}{\pgfqpoint{2.518786in}{0.767667in}}%
\pgfpathcurveto{\pgfqpoint{2.507736in}{0.767667in}}{\pgfqpoint{2.497137in}{0.763276in}}{\pgfqpoint{2.489323in}{0.755463in}}%
\pgfpathcurveto{\pgfqpoint{2.481509in}{0.747649in}}{\pgfqpoint{2.477119in}{0.737050in}}{\pgfqpoint{2.477119in}{0.726000in}}%
\pgfpathcurveto{\pgfqpoint{2.477119in}{0.714950in}}{\pgfqpoint{2.481509in}{0.704351in}}{\pgfqpoint{2.489323in}{0.696537in}}%
\pgfpathcurveto{\pgfqpoint{2.497137in}{0.688724in}}{\pgfqpoint{2.507736in}{0.684333in}}{\pgfqpoint{2.518786in}{0.684333in}}%
\pgfpathclose%
\pgfusepath{stroke,fill}%
\end{pgfscope}%
\begin{pgfscope}%
\pgfpathrectangle{\pgfqpoint{0.800000in}{0.528000in}}{\pgfqpoint{4.960000in}{3.696000in}}%
\pgfusepath{clip}%
\pgfsetbuttcap%
\pgfsetroundjoin%
\definecolor{currentfill}{rgb}{0.000000,0.000000,0.000000}%
\pgfsetfillcolor{currentfill}%
\pgfsetlinewidth{1.003750pt}%
\definecolor{currentstroke}{rgb}{0.000000,0.000000,0.000000}%
\pgfsetstrokecolor{currentstroke}%
\pgfsetdash{}{0pt}%
\pgfpathmoveto{\pgfqpoint{2.518786in}{0.684333in}}%
\pgfpathcurveto{\pgfqpoint{2.529836in}{0.684333in}}{\pgfqpoint{2.540435in}{0.688724in}}{\pgfqpoint{2.548249in}{0.696537in}}%
\pgfpathcurveto{\pgfqpoint{2.556062in}{0.704351in}}{\pgfqpoint{2.560452in}{0.714950in}}{\pgfqpoint{2.560452in}{0.726000in}}%
\pgfpathcurveto{\pgfqpoint{2.560452in}{0.737050in}}{\pgfqpoint{2.556062in}{0.747649in}}{\pgfqpoint{2.548249in}{0.755463in}}%
\pgfpathcurveto{\pgfqpoint{2.540435in}{0.763276in}}{\pgfqpoint{2.529836in}{0.767667in}}{\pgfqpoint{2.518786in}{0.767667in}}%
\pgfpathcurveto{\pgfqpoint{2.507736in}{0.767667in}}{\pgfqpoint{2.497137in}{0.763276in}}{\pgfqpoint{2.489323in}{0.755463in}}%
\pgfpathcurveto{\pgfqpoint{2.481509in}{0.747649in}}{\pgfqpoint{2.477119in}{0.737050in}}{\pgfqpoint{2.477119in}{0.726000in}}%
\pgfpathcurveto{\pgfqpoint{2.477119in}{0.714950in}}{\pgfqpoint{2.481509in}{0.704351in}}{\pgfqpoint{2.489323in}{0.696537in}}%
\pgfpathcurveto{\pgfqpoint{2.497137in}{0.688724in}}{\pgfqpoint{2.507736in}{0.684333in}}{\pgfqpoint{2.518786in}{0.684333in}}%
\pgfpathclose%
\pgfusepath{stroke,fill}%
\end{pgfscope}%
\begin{pgfscope}%
\pgfpathrectangle{\pgfqpoint{0.800000in}{0.528000in}}{\pgfqpoint{4.960000in}{3.696000in}}%
\pgfusepath{clip}%
\pgfsetbuttcap%
\pgfsetroundjoin%
\definecolor{currentfill}{rgb}{0.000000,0.000000,0.000000}%
\pgfsetfillcolor{currentfill}%
\pgfsetlinewidth{1.003750pt}%
\definecolor{currentstroke}{rgb}{0.000000,0.000000,0.000000}%
\pgfsetstrokecolor{currentstroke}%
\pgfsetdash{}{0pt}%
\pgfpathmoveto{\pgfqpoint{2.518786in}{0.684333in}}%
\pgfpathcurveto{\pgfqpoint{2.529836in}{0.684333in}}{\pgfqpoint{2.540435in}{0.688724in}}{\pgfqpoint{2.548249in}{0.696537in}}%
\pgfpathcurveto{\pgfqpoint{2.556062in}{0.704351in}}{\pgfqpoint{2.560452in}{0.714950in}}{\pgfqpoint{2.560452in}{0.726000in}}%
\pgfpathcurveto{\pgfqpoint{2.560452in}{0.737050in}}{\pgfqpoint{2.556062in}{0.747649in}}{\pgfqpoint{2.548249in}{0.755463in}}%
\pgfpathcurveto{\pgfqpoint{2.540435in}{0.763276in}}{\pgfqpoint{2.529836in}{0.767667in}}{\pgfqpoint{2.518786in}{0.767667in}}%
\pgfpathcurveto{\pgfqpoint{2.507736in}{0.767667in}}{\pgfqpoint{2.497137in}{0.763276in}}{\pgfqpoint{2.489323in}{0.755463in}}%
\pgfpathcurveto{\pgfqpoint{2.481509in}{0.747649in}}{\pgfqpoint{2.477119in}{0.737050in}}{\pgfqpoint{2.477119in}{0.726000in}}%
\pgfpathcurveto{\pgfqpoint{2.477119in}{0.714950in}}{\pgfqpoint{2.481509in}{0.704351in}}{\pgfqpoint{2.489323in}{0.696537in}}%
\pgfpathcurveto{\pgfqpoint{2.497137in}{0.688724in}}{\pgfqpoint{2.507736in}{0.684333in}}{\pgfqpoint{2.518786in}{0.684333in}}%
\pgfpathclose%
\pgfusepath{stroke,fill}%
\end{pgfscope}%
\begin{pgfscope}%
\pgfpathrectangle{\pgfqpoint{0.800000in}{0.528000in}}{\pgfqpoint{4.960000in}{3.696000in}}%
\pgfusepath{clip}%
\pgfsetbuttcap%
\pgfsetroundjoin%
\definecolor{currentfill}{rgb}{0.000000,0.000000,0.000000}%
\pgfsetfillcolor{currentfill}%
\pgfsetlinewidth{1.003750pt}%
\definecolor{currentstroke}{rgb}{0.000000,0.000000,0.000000}%
\pgfsetstrokecolor{currentstroke}%
\pgfsetdash{}{0pt}%
\pgfpathmoveto{\pgfqpoint{2.518786in}{0.684333in}}%
\pgfpathcurveto{\pgfqpoint{2.529836in}{0.684333in}}{\pgfqpoint{2.540435in}{0.688724in}}{\pgfqpoint{2.548249in}{0.696537in}}%
\pgfpathcurveto{\pgfqpoint{2.556062in}{0.704351in}}{\pgfqpoint{2.560452in}{0.714950in}}{\pgfqpoint{2.560452in}{0.726000in}}%
\pgfpathcurveto{\pgfqpoint{2.560452in}{0.737050in}}{\pgfqpoint{2.556062in}{0.747649in}}{\pgfqpoint{2.548249in}{0.755463in}}%
\pgfpathcurveto{\pgfqpoint{2.540435in}{0.763276in}}{\pgfqpoint{2.529836in}{0.767667in}}{\pgfqpoint{2.518786in}{0.767667in}}%
\pgfpathcurveto{\pgfqpoint{2.507736in}{0.767667in}}{\pgfqpoint{2.497137in}{0.763276in}}{\pgfqpoint{2.489323in}{0.755463in}}%
\pgfpathcurveto{\pgfqpoint{2.481509in}{0.747649in}}{\pgfqpoint{2.477119in}{0.737050in}}{\pgfqpoint{2.477119in}{0.726000in}}%
\pgfpathcurveto{\pgfqpoint{2.477119in}{0.714950in}}{\pgfqpoint{2.481509in}{0.704351in}}{\pgfqpoint{2.489323in}{0.696537in}}%
\pgfpathcurveto{\pgfqpoint{2.497137in}{0.688724in}}{\pgfqpoint{2.507736in}{0.684333in}}{\pgfqpoint{2.518786in}{0.684333in}}%
\pgfpathclose%
\pgfusepath{stroke,fill}%
\end{pgfscope}%
\begin{pgfscope}%
\pgfpathrectangle{\pgfqpoint{0.800000in}{0.528000in}}{\pgfqpoint{4.960000in}{3.696000in}}%
\pgfusepath{clip}%
\pgfsetbuttcap%
\pgfsetroundjoin%
\definecolor{currentfill}{rgb}{0.000000,0.000000,0.000000}%
\pgfsetfillcolor{currentfill}%
\pgfsetlinewidth{1.003750pt}%
\definecolor{currentstroke}{rgb}{0.000000,0.000000,0.000000}%
\pgfsetstrokecolor{currentstroke}%
\pgfsetdash{}{0pt}%
\pgfpathmoveto{\pgfqpoint{2.518786in}{0.684333in}}%
\pgfpathcurveto{\pgfqpoint{2.529836in}{0.684333in}}{\pgfqpoint{2.540435in}{0.688724in}}{\pgfqpoint{2.548249in}{0.696537in}}%
\pgfpathcurveto{\pgfqpoint{2.556062in}{0.704351in}}{\pgfqpoint{2.560452in}{0.714950in}}{\pgfqpoint{2.560452in}{0.726000in}}%
\pgfpathcurveto{\pgfqpoint{2.560452in}{0.737050in}}{\pgfqpoint{2.556062in}{0.747649in}}{\pgfqpoint{2.548249in}{0.755463in}}%
\pgfpathcurveto{\pgfqpoint{2.540435in}{0.763276in}}{\pgfqpoint{2.529836in}{0.767667in}}{\pgfqpoint{2.518786in}{0.767667in}}%
\pgfpathcurveto{\pgfqpoint{2.507736in}{0.767667in}}{\pgfqpoint{2.497137in}{0.763276in}}{\pgfqpoint{2.489323in}{0.755463in}}%
\pgfpathcurveto{\pgfqpoint{2.481509in}{0.747649in}}{\pgfqpoint{2.477119in}{0.737050in}}{\pgfqpoint{2.477119in}{0.726000in}}%
\pgfpathcurveto{\pgfqpoint{2.477119in}{0.714950in}}{\pgfqpoint{2.481509in}{0.704351in}}{\pgfqpoint{2.489323in}{0.696537in}}%
\pgfpathcurveto{\pgfqpoint{2.497137in}{0.688724in}}{\pgfqpoint{2.507736in}{0.684333in}}{\pgfqpoint{2.518786in}{0.684333in}}%
\pgfpathclose%
\pgfusepath{stroke,fill}%
\end{pgfscope}%
\begin{pgfscope}%
\pgfpathrectangle{\pgfqpoint{0.800000in}{0.528000in}}{\pgfqpoint{4.960000in}{3.696000in}}%
\pgfusepath{clip}%
\pgfsetbuttcap%
\pgfsetroundjoin%
\definecolor{currentfill}{rgb}{0.000000,0.000000,0.000000}%
\pgfsetfillcolor{currentfill}%
\pgfsetlinewidth{1.003750pt}%
\definecolor{currentstroke}{rgb}{0.000000,0.000000,0.000000}%
\pgfsetstrokecolor{currentstroke}%
\pgfsetdash{}{0pt}%
\pgfpathmoveto{\pgfqpoint{2.518786in}{0.684333in}}%
\pgfpathcurveto{\pgfqpoint{2.529836in}{0.684333in}}{\pgfqpoint{2.540435in}{0.688724in}}{\pgfqpoint{2.548249in}{0.696537in}}%
\pgfpathcurveto{\pgfqpoint{2.556062in}{0.704351in}}{\pgfqpoint{2.560452in}{0.714950in}}{\pgfqpoint{2.560452in}{0.726000in}}%
\pgfpathcurveto{\pgfqpoint{2.560452in}{0.737050in}}{\pgfqpoint{2.556062in}{0.747649in}}{\pgfqpoint{2.548249in}{0.755463in}}%
\pgfpathcurveto{\pgfqpoint{2.540435in}{0.763276in}}{\pgfqpoint{2.529836in}{0.767667in}}{\pgfqpoint{2.518786in}{0.767667in}}%
\pgfpathcurveto{\pgfqpoint{2.507736in}{0.767667in}}{\pgfqpoint{2.497137in}{0.763276in}}{\pgfqpoint{2.489323in}{0.755463in}}%
\pgfpathcurveto{\pgfqpoint{2.481509in}{0.747649in}}{\pgfqpoint{2.477119in}{0.737050in}}{\pgfqpoint{2.477119in}{0.726000in}}%
\pgfpathcurveto{\pgfqpoint{2.477119in}{0.714950in}}{\pgfqpoint{2.481509in}{0.704351in}}{\pgfqpoint{2.489323in}{0.696537in}}%
\pgfpathcurveto{\pgfqpoint{2.497137in}{0.688724in}}{\pgfqpoint{2.507736in}{0.684333in}}{\pgfqpoint{2.518786in}{0.684333in}}%
\pgfpathclose%
\pgfusepath{stroke,fill}%
\end{pgfscope}%
\begin{pgfscope}%
\pgfpathrectangle{\pgfqpoint{0.800000in}{0.528000in}}{\pgfqpoint{4.960000in}{3.696000in}}%
\pgfusepath{clip}%
\pgfsetbuttcap%
\pgfsetroundjoin%
\definecolor{currentfill}{rgb}{0.000000,0.000000,0.000000}%
\pgfsetfillcolor{currentfill}%
\pgfsetlinewidth{1.003750pt}%
\definecolor{currentstroke}{rgb}{0.000000,0.000000,0.000000}%
\pgfsetstrokecolor{currentstroke}%
\pgfsetdash{}{0pt}%
\pgfpathmoveto{\pgfqpoint{2.518786in}{0.684333in}}%
\pgfpathcurveto{\pgfqpoint{2.529836in}{0.684333in}}{\pgfqpoint{2.540435in}{0.688724in}}{\pgfqpoint{2.548249in}{0.696537in}}%
\pgfpathcurveto{\pgfqpoint{2.556062in}{0.704351in}}{\pgfqpoint{2.560452in}{0.714950in}}{\pgfqpoint{2.560452in}{0.726000in}}%
\pgfpathcurveto{\pgfqpoint{2.560452in}{0.737050in}}{\pgfqpoint{2.556062in}{0.747649in}}{\pgfqpoint{2.548249in}{0.755463in}}%
\pgfpathcurveto{\pgfqpoint{2.540435in}{0.763276in}}{\pgfqpoint{2.529836in}{0.767667in}}{\pgfqpoint{2.518786in}{0.767667in}}%
\pgfpathcurveto{\pgfqpoint{2.507736in}{0.767667in}}{\pgfqpoint{2.497137in}{0.763276in}}{\pgfqpoint{2.489323in}{0.755463in}}%
\pgfpathcurveto{\pgfqpoint{2.481509in}{0.747649in}}{\pgfqpoint{2.477119in}{0.737050in}}{\pgfqpoint{2.477119in}{0.726000in}}%
\pgfpathcurveto{\pgfqpoint{2.477119in}{0.714950in}}{\pgfqpoint{2.481509in}{0.704351in}}{\pgfqpoint{2.489323in}{0.696537in}}%
\pgfpathcurveto{\pgfqpoint{2.497137in}{0.688724in}}{\pgfqpoint{2.507736in}{0.684333in}}{\pgfqpoint{2.518786in}{0.684333in}}%
\pgfpathclose%
\pgfusepath{stroke,fill}%
\end{pgfscope}%
\begin{pgfscope}%
\pgfpathrectangle{\pgfqpoint{0.800000in}{0.528000in}}{\pgfqpoint{4.960000in}{3.696000in}}%
\pgfusepath{clip}%
\pgfsetbuttcap%
\pgfsetroundjoin%
\definecolor{currentfill}{rgb}{0.000000,0.000000,0.000000}%
\pgfsetfillcolor{currentfill}%
\pgfsetlinewidth{1.003750pt}%
\definecolor{currentstroke}{rgb}{0.000000,0.000000,0.000000}%
\pgfsetstrokecolor{currentstroke}%
\pgfsetdash{}{0pt}%
\pgfpathmoveto{\pgfqpoint{2.518786in}{0.684333in}}%
\pgfpathcurveto{\pgfqpoint{2.529836in}{0.684333in}}{\pgfqpoint{2.540435in}{0.688724in}}{\pgfqpoint{2.548249in}{0.696537in}}%
\pgfpathcurveto{\pgfqpoint{2.556062in}{0.704351in}}{\pgfqpoint{2.560452in}{0.714950in}}{\pgfqpoint{2.560452in}{0.726000in}}%
\pgfpathcurveto{\pgfqpoint{2.560452in}{0.737050in}}{\pgfqpoint{2.556062in}{0.747649in}}{\pgfqpoint{2.548249in}{0.755463in}}%
\pgfpathcurveto{\pgfqpoint{2.540435in}{0.763276in}}{\pgfqpoint{2.529836in}{0.767667in}}{\pgfqpoint{2.518786in}{0.767667in}}%
\pgfpathcurveto{\pgfqpoint{2.507736in}{0.767667in}}{\pgfqpoint{2.497137in}{0.763276in}}{\pgfqpoint{2.489323in}{0.755463in}}%
\pgfpathcurveto{\pgfqpoint{2.481509in}{0.747649in}}{\pgfqpoint{2.477119in}{0.737050in}}{\pgfqpoint{2.477119in}{0.726000in}}%
\pgfpathcurveto{\pgfqpoint{2.477119in}{0.714950in}}{\pgfqpoint{2.481509in}{0.704351in}}{\pgfqpoint{2.489323in}{0.696537in}}%
\pgfpathcurveto{\pgfqpoint{2.497137in}{0.688724in}}{\pgfqpoint{2.507736in}{0.684333in}}{\pgfqpoint{2.518786in}{0.684333in}}%
\pgfpathclose%
\pgfusepath{stroke,fill}%
\end{pgfscope}%
\begin{pgfscope}%
\pgfpathrectangle{\pgfqpoint{0.800000in}{0.528000in}}{\pgfqpoint{4.960000in}{3.696000in}}%
\pgfusepath{clip}%
\pgfsetbuttcap%
\pgfsetroundjoin%
\definecolor{currentfill}{rgb}{0.000000,0.000000,0.000000}%
\pgfsetfillcolor{currentfill}%
\pgfsetlinewidth{1.003750pt}%
\definecolor{currentstroke}{rgb}{0.000000,0.000000,0.000000}%
\pgfsetstrokecolor{currentstroke}%
\pgfsetdash{}{0pt}%
\pgfpathmoveto{\pgfqpoint{2.518786in}{0.684333in}}%
\pgfpathcurveto{\pgfqpoint{2.529836in}{0.684333in}}{\pgfqpoint{2.540435in}{0.688724in}}{\pgfqpoint{2.548249in}{0.696537in}}%
\pgfpathcurveto{\pgfqpoint{2.556062in}{0.704351in}}{\pgfqpoint{2.560452in}{0.714950in}}{\pgfqpoint{2.560452in}{0.726000in}}%
\pgfpathcurveto{\pgfqpoint{2.560452in}{0.737050in}}{\pgfqpoint{2.556062in}{0.747649in}}{\pgfqpoint{2.548249in}{0.755463in}}%
\pgfpathcurveto{\pgfqpoint{2.540435in}{0.763276in}}{\pgfqpoint{2.529836in}{0.767667in}}{\pgfqpoint{2.518786in}{0.767667in}}%
\pgfpathcurveto{\pgfqpoint{2.507736in}{0.767667in}}{\pgfqpoint{2.497137in}{0.763276in}}{\pgfqpoint{2.489323in}{0.755463in}}%
\pgfpathcurveto{\pgfqpoint{2.481509in}{0.747649in}}{\pgfqpoint{2.477119in}{0.737050in}}{\pgfqpoint{2.477119in}{0.726000in}}%
\pgfpathcurveto{\pgfqpoint{2.477119in}{0.714950in}}{\pgfqpoint{2.481509in}{0.704351in}}{\pgfqpoint{2.489323in}{0.696537in}}%
\pgfpathcurveto{\pgfqpoint{2.497137in}{0.688724in}}{\pgfqpoint{2.507736in}{0.684333in}}{\pgfqpoint{2.518786in}{0.684333in}}%
\pgfpathclose%
\pgfusepath{stroke,fill}%
\end{pgfscope}%
\begin{pgfscope}%
\pgfpathrectangle{\pgfqpoint{0.800000in}{0.528000in}}{\pgfqpoint{4.960000in}{3.696000in}}%
\pgfusepath{clip}%
\pgfsetbuttcap%
\pgfsetroundjoin%
\definecolor{currentfill}{rgb}{0.000000,0.000000,0.000000}%
\pgfsetfillcolor{currentfill}%
\pgfsetlinewidth{1.003750pt}%
\definecolor{currentstroke}{rgb}{0.000000,0.000000,0.000000}%
\pgfsetstrokecolor{currentstroke}%
\pgfsetdash{}{0pt}%
\pgfpathmoveto{\pgfqpoint{2.518786in}{0.684333in}}%
\pgfpathcurveto{\pgfqpoint{2.529836in}{0.684333in}}{\pgfqpoint{2.540435in}{0.688724in}}{\pgfqpoint{2.548249in}{0.696537in}}%
\pgfpathcurveto{\pgfqpoint{2.556062in}{0.704351in}}{\pgfqpoint{2.560452in}{0.714950in}}{\pgfqpoint{2.560452in}{0.726000in}}%
\pgfpathcurveto{\pgfqpoint{2.560452in}{0.737050in}}{\pgfqpoint{2.556062in}{0.747649in}}{\pgfqpoint{2.548249in}{0.755463in}}%
\pgfpathcurveto{\pgfqpoint{2.540435in}{0.763276in}}{\pgfqpoint{2.529836in}{0.767667in}}{\pgfqpoint{2.518786in}{0.767667in}}%
\pgfpathcurveto{\pgfqpoint{2.507736in}{0.767667in}}{\pgfqpoint{2.497137in}{0.763276in}}{\pgfqpoint{2.489323in}{0.755463in}}%
\pgfpathcurveto{\pgfqpoint{2.481509in}{0.747649in}}{\pgfqpoint{2.477119in}{0.737050in}}{\pgfqpoint{2.477119in}{0.726000in}}%
\pgfpathcurveto{\pgfqpoint{2.477119in}{0.714950in}}{\pgfqpoint{2.481509in}{0.704351in}}{\pgfqpoint{2.489323in}{0.696537in}}%
\pgfpathcurveto{\pgfqpoint{2.497137in}{0.688724in}}{\pgfqpoint{2.507736in}{0.684333in}}{\pgfqpoint{2.518786in}{0.684333in}}%
\pgfpathclose%
\pgfusepath{stroke,fill}%
\end{pgfscope}%
\begin{pgfscope}%
\pgfpathrectangle{\pgfqpoint{0.800000in}{0.528000in}}{\pgfqpoint{4.960000in}{3.696000in}}%
\pgfusepath{clip}%
\pgfsetbuttcap%
\pgfsetroundjoin%
\definecolor{currentfill}{rgb}{0.000000,0.000000,0.000000}%
\pgfsetfillcolor{currentfill}%
\pgfsetlinewidth{1.003750pt}%
\definecolor{currentstroke}{rgb}{0.000000,0.000000,0.000000}%
\pgfsetstrokecolor{currentstroke}%
\pgfsetdash{}{0pt}%
\pgfpathmoveto{\pgfqpoint{2.518786in}{0.684333in}}%
\pgfpathcurveto{\pgfqpoint{2.529836in}{0.684333in}}{\pgfqpoint{2.540435in}{0.688724in}}{\pgfqpoint{2.548249in}{0.696537in}}%
\pgfpathcurveto{\pgfqpoint{2.556062in}{0.704351in}}{\pgfqpoint{2.560452in}{0.714950in}}{\pgfqpoint{2.560452in}{0.726000in}}%
\pgfpathcurveto{\pgfqpoint{2.560452in}{0.737050in}}{\pgfqpoint{2.556062in}{0.747649in}}{\pgfqpoint{2.548249in}{0.755463in}}%
\pgfpathcurveto{\pgfqpoint{2.540435in}{0.763276in}}{\pgfqpoint{2.529836in}{0.767667in}}{\pgfqpoint{2.518786in}{0.767667in}}%
\pgfpathcurveto{\pgfqpoint{2.507736in}{0.767667in}}{\pgfqpoint{2.497137in}{0.763276in}}{\pgfqpoint{2.489323in}{0.755463in}}%
\pgfpathcurveto{\pgfqpoint{2.481509in}{0.747649in}}{\pgfqpoint{2.477119in}{0.737050in}}{\pgfqpoint{2.477119in}{0.726000in}}%
\pgfpathcurveto{\pgfqpoint{2.477119in}{0.714950in}}{\pgfqpoint{2.481509in}{0.704351in}}{\pgfqpoint{2.489323in}{0.696537in}}%
\pgfpathcurveto{\pgfqpoint{2.497137in}{0.688724in}}{\pgfqpoint{2.507736in}{0.684333in}}{\pgfqpoint{2.518786in}{0.684333in}}%
\pgfpathclose%
\pgfusepath{stroke,fill}%
\end{pgfscope}%
\begin{pgfscope}%
\pgfpathrectangle{\pgfqpoint{0.800000in}{0.528000in}}{\pgfqpoint{4.960000in}{3.696000in}}%
\pgfusepath{clip}%
\pgfsetbuttcap%
\pgfsetroundjoin%
\definecolor{currentfill}{rgb}{0.000000,0.000000,0.000000}%
\pgfsetfillcolor{currentfill}%
\pgfsetlinewidth{1.003750pt}%
\definecolor{currentstroke}{rgb}{0.000000,0.000000,0.000000}%
\pgfsetstrokecolor{currentstroke}%
\pgfsetdash{}{0pt}%
\pgfpathmoveto{\pgfqpoint{2.518786in}{0.684333in}}%
\pgfpathcurveto{\pgfqpoint{2.529836in}{0.684333in}}{\pgfqpoint{2.540435in}{0.688724in}}{\pgfqpoint{2.548249in}{0.696537in}}%
\pgfpathcurveto{\pgfqpoint{2.556062in}{0.704351in}}{\pgfqpoint{2.560452in}{0.714950in}}{\pgfqpoint{2.560452in}{0.726000in}}%
\pgfpathcurveto{\pgfqpoint{2.560452in}{0.737050in}}{\pgfqpoint{2.556062in}{0.747649in}}{\pgfqpoint{2.548249in}{0.755463in}}%
\pgfpathcurveto{\pgfqpoint{2.540435in}{0.763276in}}{\pgfqpoint{2.529836in}{0.767667in}}{\pgfqpoint{2.518786in}{0.767667in}}%
\pgfpathcurveto{\pgfqpoint{2.507736in}{0.767667in}}{\pgfqpoint{2.497137in}{0.763276in}}{\pgfqpoint{2.489323in}{0.755463in}}%
\pgfpathcurveto{\pgfqpoint{2.481509in}{0.747649in}}{\pgfqpoint{2.477119in}{0.737050in}}{\pgfqpoint{2.477119in}{0.726000in}}%
\pgfpathcurveto{\pgfqpoint{2.477119in}{0.714950in}}{\pgfqpoint{2.481509in}{0.704351in}}{\pgfqpoint{2.489323in}{0.696537in}}%
\pgfpathcurveto{\pgfqpoint{2.497137in}{0.688724in}}{\pgfqpoint{2.507736in}{0.684333in}}{\pgfqpoint{2.518786in}{0.684333in}}%
\pgfpathclose%
\pgfusepath{stroke,fill}%
\end{pgfscope}%
\begin{pgfscope}%
\pgfpathrectangle{\pgfqpoint{0.800000in}{0.528000in}}{\pgfqpoint{4.960000in}{3.696000in}}%
\pgfusepath{clip}%
\pgfsetbuttcap%
\pgfsetroundjoin%
\definecolor{currentfill}{rgb}{0.000000,0.000000,0.000000}%
\pgfsetfillcolor{currentfill}%
\pgfsetlinewidth{1.003750pt}%
\definecolor{currentstroke}{rgb}{0.000000,0.000000,0.000000}%
\pgfsetstrokecolor{currentstroke}%
\pgfsetdash{}{0pt}%
\pgfpathmoveto{\pgfqpoint{2.518786in}{0.684333in}}%
\pgfpathcurveto{\pgfqpoint{2.529836in}{0.684333in}}{\pgfqpoint{2.540435in}{0.688724in}}{\pgfqpoint{2.548249in}{0.696537in}}%
\pgfpathcurveto{\pgfqpoint{2.556062in}{0.704351in}}{\pgfqpoint{2.560452in}{0.714950in}}{\pgfqpoint{2.560452in}{0.726000in}}%
\pgfpathcurveto{\pgfqpoint{2.560452in}{0.737050in}}{\pgfqpoint{2.556062in}{0.747649in}}{\pgfqpoint{2.548249in}{0.755463in}}%
\pgfpathcurveto{\pgfqpoint{2.540435in}{0.763276in}}{\pgfqpoint{2.529836in}{0.767667in}}{\pgfqpoint{2.518786in}{0.767667in}}%
\pgfpathcurveto{\pgfqpoint{2.507736in}{0.767667in}}{\pgfqpoint{2.497137in}{0.763276in}}{\pgfqpoint{2.489323in}{0.755463in}}%
\pgfpathcurveto{\pgfqpoint{2.481509in}{0.747649in}}{\pgfqpoint{2.477119in}{0.737050in}}{\pgfqpoint{2.477119in}{0.726000in}}%
\pgfpathcurveto{\pgfqpoint{2.477119in}{0.714950in}}{\pgfqpoint{2.481509in}{0.704351in}}{\pgfqpoint{2.489323in}{0.696537in}}%
\pgfpathcurveto{\pgfqpoint{2.497137in}{0.688724in}}{\pgfqpoint{2.507736in}{0.684333in}}{\pgfqpoint{2.518786in}{0.684333in}}%
\pgfpathclose%
\pgfusepath{stroke,fill}%
\end{pgfscope}%
\begin{pgfscope}%
\pgfpathrectangle{\pgfqpoint{0.800000in}{0.528000in}}{\pgfqpoint{4.960000in}{3.696000in}}%
\pgfusepath{clip}%
\pgfsetbuttcap%
\pgfsetroundjoin%
\definecolor{currentfill}{rgb}{0.000000,0.000000,0.000000}%
\pgfsetfillcolor{currentfill}%
\pgfsetlinewidth{1.003750pt}%
\definecolor{currentstroke}{rgb}{0.000000,0.000000,0.000000}%
\pgfsetstrokecolor{currentstroke}%
\pgfsetdash{}{0pt}%
\pgfpathmoveto{\pgfqpoint{2.518786in}{0.684333in}}%
\pgfpathcurveto{\pgfqpoint{2.529836in}{0.684333in}}{\pgfqpoint{2.540435in}{0.688724in}}{\pgfqpoint{2.548249in}{0.696537in}}%
\pgfpathcurveto{\pgfqpoint{2.556062in}{0.704351in}}{\pgfqpoint{2.560452in}{0.714950in}}{\pgfqpoint{2.560452in}{0.726000in}}%
\pgfpathcurveto{\pgfqpoint{2.560452in}{0.737050in}}{\pgfqpoint{2.556062in}{0.747649in}}{\pgfqpoint{2.548249in}{0.755463in}}%
\pgfpathcurveto{\pgfqpoint{2.540435in}{0.763276in}}{\pgfqpoint{2.529836in}{0.767667in}}{\pgfqpoint{2.518786in}{0.767667in}}%
\pgfpathcurveto{\pgfqpoint{2.507736in}{0.767667in}}{\pgfqpoint{2.497137in}{0.763276in}}{\pgfqpoint{2.489323in}{0.755463in}}%
\pgfpathcurveto{\pgfqpoint{2.481509in}{0.747649in}}{\pgfqpoint{2.477119in}{0.737050in}}{\pgfqpoint{2.477119in}{0.726000in}}%
\pgfpathcurveto{\pgfqpoint{2.477119in}{0.714950in}}{\pgfqpoint{2.481509in}{0.704351in}}{\pgfqpoint{2.489323in}{0.696537in}}%
\pgfpathcurveto{\pgfqpoint{2.497137in}{0.688724in}}{\pgfqpoint{2.507736in}{0.684333in}}{\pgfqpoint{2.518786in}{0.684333in}}%
\pgfpathclose%
\pgfusepath{stroke,fill}%
\end{pgfscope}%
\begin{pgfscope}%
\pgfpathrectangle{\pgfqpoint{0.800000in}{0.528000in}}{\pgfqpoint{4.960000in}{3.696000in}}%
\pgfusepath{clip}%
\pgfsetbuttcap%
\pgfsetroundjoin%
\definecolor{currentfill}{rgb}{0.000000,0.000000,0.000000}%
\pgfsetfillcolor{currentfill}%
\pgfsetlinewidth{1.003750pt}%
\definecolor{currentstroke}{rgb}{0.000000,0.000000,0.000000}%
\pgfsetstrokecolor{currentstroke}%
\pgfsetdash{}{0pt}%
\pgfpathmoveto{\pgfqpoint{2.518786in}{0.684333in}}%
\pgfpathcurveto{\pgfqpoint{2.529836in}{0.684333in}}{\pgfqpoint{2.540435in}{0.688724in}}{\pgfqpoint{2.548249in}{0.696537in}}%
\pgfpathcurveto{\pgfqpoint{2.556062in}{0.704351in}}{\pgfqpoint{2.560452in}{0.714950in}}{\pgfqpoint{2.560452in}{0.726000in}}%
\pgfpathcurveto{\pgfqpoint{2.560452in}{0.737050in}}{\pgfqpoint{2.556062in}{0.747649in}}{\pgfqpoint{2.548249in}{0.755463in}}%
\pgfpathcurveto{\pgfqpoint{2.540435in}{0.763276in}}{\pgfqpoint{2.529836in}{0.767667in}}{\pgfqpoint{2.518786in}{0.767667in}}%
\pgfpathcurveto{\pgfqpoint{2.507736in}{0.767667in}}{\pgfqpoint{2.497137in}{0.763276in}}{\pgfqpoint{2.489323in}{0.755463in}}%
\pgfpathcurveto{\pgfqpoint{2.481509in}{0.747649in}}{\pgfqpoint{2.477119in}{0.737050in}}{\pgfqpoint{2.477119in}{0.726000in}}%
\pgfpathcurveto{\pgfqpoint{2.477119in}{0.714950in}}{\pgfqpoint{2.481509in}{0.704351in}}{\pgfqpoint{2.489323in}{0.696537in}}%
\pgfpathcurveto{\pgfqpoint{2.497137in}{0.688724in}}{\pgfqpoint{2.507736in}{0.684333in}}{\pgfqpoint{2.518786in}{0.684333in}}%
\pgfpathclose%
\pgfusepath{stroke,fill}%
\end{pgfscope}%
\begin{pgfscope}%
\pgfpathrectangle{\pgfqpoint{0.800000in}{0.528000in}}{\pgfqpoint{4.960000in}{3.696000in}}%
\pgfusepath{clip}%
\pgfsetbuttcap%
\pgfsetroundjoin%
\definecolor{currentfill}{rgb}{0.000000,0.000000,0.000000}%
\pgfsetfillcolor{currentfill}%
\pgfsetlinewidth{1.003750pt}%
\definecolor{currentstroke}{rgb}{0.000000,0.000000,0.000000}%
\pgfsetstrokecolor{currentstroke}%
\pgfsetdash{}{0pt}%
\pgfpathmoveto{\pgfqpoint{2.518786in}{0.684333in}}%
\pgfpathcurveto{\pgfqpoint{2.529836in}{0.684333in}}{\pgfqpoint{2.540435in}{0.688724in}}{\pgfqpoint{2.548249in}{0.696537in}}%
\pgfpathcurveto{\pgfqpoint{2.556062in}{0.704351in}}{\pgfqpoint{2.560452in}{0.714950in}}{\pgfqpoint{2.560452in}{0.726000in}}%
\pgfpathcurveto{\pgfqpoint{2.560452in}{0.737050in}}{\pgfqpoint{2.556062in}{0.747649in}}{\pgfqpoint{2.548249in}{0.755463in}}%
\pgfpathcurveto{\pgfqpoint{2.540435in}{0.763276in}}{\pgfqpoint{2.529836in}{0.767667in}}{\pgfqpoint{2.518786in}{0.767667in}}%
\pgfpathcurveto{\pgfqpoint{2.507736in}{0.767667in}}{\pgfqpoint{2.497137in}{0.763276in}}{\pgfqpoint{2.489323in}{0.755463in}}%
\pgfpathcurveto{\pgfqpoint{2.481509in}{0.747649in}}{\pgfqpoint{2.477119in}{0.737050in}}{\pgfqpoint{2.477119in}{0.726000in}}%
\pgfpathcurveto{\pgfqpoint{2.477119in}{0.714950in}}{\pgfqpoint{2.481509in}{0.704351in}}{\pgfqpoint{2.489323in}{0.696537in}}%
\pgfpathcurveto{\pgfqpoint{2.497137in}{0.688724in}}{\pgfqpoint{2.507736in}{0.684333in}}{\pgfqpoint{2.518786in}{0.684333in}}%
\pgfpathclose%
\pgfusepath{stroke,fill}%
\end{pgfscope}%
\begin{pgfscope}%
\pgfpathrectangle{\pgfqpoint{0.800000in}{0.528000in}}{\pgfqpoint{4.960000in}{3.696000in}}%
\pgfusepath{clip}%
\pgfsetbuttcap%
\pgfsetroundjoin%
\definecolor{currentfill}{rgb}{0.000000,0.000000,0.000000}%
\pgfsetfillcolor{currentfill}%
\pgfsetlinewidth{1.003750pt}%
\definecolor{currentstroke}{rgb}{0.000000,0.000000,0.000000}%
\pgfsetstrokecolor{currentstroke}%
\pgfsetdash{}{0pt}%
\pgfpathmoveto{\pgfqpoint{2.518786in}{0.684333in}}%
\pgfpathcurveto{\pgfqpoint{2.529836in}{0.684333in}}{\pgfqpoint{2.540435in}{0.688724in}}{\pgfqpoint{2.548249in}{0.696537in}}%
\pgfpathcurveto{\pgfqpoint{2.556062in}{0.704351in}}{\pgfqpoint{2.560452in}{0.714950in}}{\pgfqpoint{2.560452in}{0.726000in}}%
\pgfpathcurveto{\pgfqpoint{2.560452in}{0.737050in}}{\pgfqpoint{2.556062in}{0.747649in}}{\pgfqpoint{2.548249in}{0.755463in}}%
\pgfpathcurveto{\pgfqpoint{2.540435in}{0.763276in}}{\pgfqpoint{2.529836in}{0.767667in}}{\pgfqpoint{2.518786in}{0.767667in}}%
\pgfpathcurveto{\pgfqpoint{2.507736in}{0.767667in}}{\pgfqpoint{2.497137in}{0.763276in}}{\pgfqpoint{2.489323in}{0.755463in}}%
\pgfpathcurveto{\pgfqpoint{2.481509in}{0.747649in}}{\pgfqpoint{2.477119in}{0.737050in}}{\pgfqpoint{2.477119in}{0.726000in}}%
\pgfpathcurveto{\pgfqpoint{2.477119in}{0.714950in}}{\pgfqpoint{2.481509in}{0.704351in}}{\pgfqpoint{2.489323in}{0.696537in}}%
\pgfpathcurveto{\pgfqpoint{2.497137in}{0.688724in}}{\pgfqpoint{2.507736in}{0.684333in}}{\pgfqpoint{2.518786in}{0.684333in}}%
\pgfpathclose%
\pgfusepath{stroke,fill}%
\end{pgfscope}%
\begin{pgfscope}%
\pgfpathrectangle{\pgfqpoint{0.800000in}{0.528000in}}{\pgfqpoint{4.960000in}{3.696000in}}%
\pgfusepath{clip}%
\pgfsetbuttcap%
\pgfsetroundjoin%
\definecolor{currentfill}{rgb}{0.000000,0.000000,0.000000}%
\pgfsetfillcolor{currentfill}%
\pgfsetlinewidth{1.003750pt}%
\definecolor{currentstroke}{rgb}{0.000000,0.000000,0.000000}%
\pgfsetstrokecolor{currentstroke}%
\pgfsetdash{}{0pt}%
\pgfpathmoveto{\pgfqpoint{2.518786in}{0.684333in}}%
\pgfpathcurveto{\pgfqpoint{2.529836in}{0.684333in}}{\pgfqpoint{2.540435in}{0.688724in}}{\pgfqpoint{2.548249in}{0.696537in}}%
\pgfpathcurveto{\pgfqpoint{2.556062in}{0.704351in}}{\pgfqpoint{2.560452in}{0.714950in}}{\pgfqpoint{2.560452in}{0.726000in}}%
\pgfpathcurveto{\pgfqpoint{2.560452in}{0.737050in}}{\pgfqpoint{2.556062in}{0.747649in}}{\pgfqpoint{2.548249in}{0.755463in}}%
\pgfpathcurveto{\pgfqpoint{2.540435in}{0.763276in}}{\pgfqpoint{2.529836in}{0.767667in}}{\pgfqpoint{2.518786in}{0.767667in}}%
\pgfpathcurveto{\pgfqpoint{2.507736in}{0.767667in}}{\pgfqpoint{2.497137in}{0.763276in}}{\pgfqpoint{2.489323in}{0.755463in}}%
\pgfpathcurveto{\pgfqpoint{2.481509in}{0.747649in}}{\pgfqpoint{2.477119in}{0.737050in}}{\pgfqpoint{2.477119in}{0.726000in}}%
\pgfpathcurveto{\pgfqpoint{2.477119in}{0.714950in}}{\pgfqpoint{2.481509in}{0.704351in}}{\pgfqpoint{2.489323in}{0.696537in}}%
\pgfpathcurveto{\pgfqpoint{2.497137in}{0.688724in}}{\pgfqpoint{2.507736in}{0.684333in}}{\pgfqpoint{2.518786in}{0.684333in}}%
\pgfpathclose%
\pgfusepath{stroke,fill}%
\end{pgfscope}%
\begin{pgfscope}%
\pgfpathrectangle{\pgfqpoint{0.800000in}{0.528000in}}{\pgfqpoint{4.960000in}{3.696000in}}%
\pgfusepath{clip}%
\pgfsetbuttcap%
\pgfsetroundjoin%
\definecolor{currentfill}{rgb}{0.000000,0.000000,0.000000}%
\pgfsetfillcolor{currentfill}%
\pgfsetlinewidth{1.003750pt}%
\definecolor{currentstroke}{rgb}{0.000000,0.000000,0.000000}%
\pgfsetstrokecolor{currentstroke}%
\pgfsetdash{}{0pt}%
\pgfpathmoveto{\pgfqpoint{2.518786in}{0.684333in}}%
\pgfpathcurveto{\pgfqpoint{2.529836in}{0.684333in}}{\pgfqpoint{2.540435in}{0.688724in}}{\pgfqpoint{2.548249in}{0.696537in}}%
\pgfpathcurveto{\pgfqpoint{2.556062in}{0.704351in}}{\pgfqpoint{2.560452in}{0.714950in}}{\pgfqpoint{2.560452in}{0.726000in}}%
\pgfpathcurveto{\pgfqpoint{2.560452in}{0.737050in}}{\pgfqpoint{2.556062in}{0.747649in}}{\pgfqpoint{2.548249in}{0.755463in}}%
\pgfpathcurveto{\pgfqpoint{2.540435in}{0.763276in}}{\pgfqpoint{2.529836in}{0.767667in}}{\pgfqpoint{2.518786in}{0.767667in}}%
\pgfpathcurveto{\pgfqpoint{2.507736in}{0.767667in}}{\pgfqpoint{2.497137in}{0.763276in}}{\pgfqpoint{2.489323in}{0.755463in}}%
\pgfpathcurveto{\pgfqpoint{2.481509in}{0.747649in}}{\pgfqpoint{2.477119in}{0.737050in}}{\pgfqpoint{2.477119in}{0.726000in}}%
\pgfpathcurveto{\pgfqpoint{2.477119in}{0.714950in}}{\pgfqpoint{2.481509in}{0.704351in}}{\pgfqpoint{2.489323in}{0.696537in}}%
\pgfpathcurveto{\pgfqpoint{2.497137in}{0.688724in}}{\pgfqpoint{2.507736in}{0.684333in}}{\pgfqpoint{2.518786in}{0.684333in}}%
\pgfpathclose%
\pgfusepath{stroke,fill}%
\end{pgfscope}%
\begin{pgfscope}%
\pgfpathrectangle{\pgfqpoint{0.800000in}{0.528000in}}{\pgfqpoint{4.960000in}{3.696000in}}%
\pgfusepath{clip}%
\pgfsetbuttcap%
\pgfsetroundjoin%
\definecolor{currentfill}{rgb}{0.000000,0.000000,0.000000}%
\pgfsetfillcolor{currentfill}%
\pgfsetlinewidth{1.003750pt}%
\definecolor{currentstroke}{rgb}{0.000000,0.000000,0.000000}%
\pgfsetstrokecolor{currentstroke}%
\pgfsetdash{}{0pt}%
\pgfpathmoveto{\pgfqpoint{2.518786in}{0.684333in}}%
\pgfpathcurveto{\pgfqpoint{2.529836in}{0.684333in}}{\pgfqpoint{2.540435in}{0.688724in}}{\pgfqpoint{2.548249in}{0.696537in}}%
\pgfpathcurveto{\pgfqpoint{2.556062in}{0.704351in}}{\pgfqpoint{2.560452in}{0.714950in}}{\pgfqpoint{2.560452in}{0.726000in}}%
\pgfpathcurveto{\pgfqpoint{2.560452in}{0.737050in}}{\pgfqpoint{2.556062in}{0.747649in}}{\pgfqpoint{2.548249in}{0.755463in}}%
\pgfpathcurveto{\pgfqpoint{2.540435in}{0.763276in}}{\pgfqpoint{2.529836in}{0.767667in}}{\pgfqpoint{2.518786in}{0.767667in}}%
\pgfpathcurveto{\pgfqpoint{2.507736in}{0.767667in}}{\pgfqpoint{2.497137in}{0.763276in}}{\pgfqpoint{2.489323in}{0.755463in}}%
\pgfpathcurveto{\pgfqpoint{2.481509in}{0.747649in}}{\pgfqpoint{2.477119in}{0.737050in}}{\pgfqpoint{2.477119in}{0.726000in}}%
\pgfpathcurveto{\pgfqpoint{2.477119in}{0.714950in}}{\pgfqpoint{2.481509in}{0.704351in}}{\pgfqpoint{2.489323in}{0.696537in}}%
\pgfpathcurveto{\pgfqpoint{2.497137in}{0.688724in}}{\pgfqpoint{2.507736in}{0.684333in}}{\pgfqpoint{2.518786in}{0.684333in}}%
\pgfpathclose%
\pgfusepath{stroke,fill}%
\end{pgfscope}%
\begin{pgfscope}%
\pgfpathrectangle{\pgfqpoint{0.800000in}{0.528000in}}{\pgfqpoint{4.960000in}{3.696000in}}%
\pgfusepath{clip}%
\pgfsetbuttcap%
\pgfsetroundjoin%
\definecolor{currentfill}{rgb}{0.000000,0.000000,0.000000}%
\pgfsetfillcolor{currentfill}%
\pgfsetlinewidth{1.003750pt}%
\definecolor{currentstroke}{rgb}{0.000000,0.000000,0.000000}%
\pgfsetstrokecolor{currentstroke}%
\pgfsetdash{}{0pt}%
\pgfpathmoveto{\pgfqpoint{2.518786in}{0.684333in}}%
\pgfpathcurveto{\pgfqpoint{2.529836in}{0.684333in}}{\pgfqpoint{2.540435in}{0.688724in}}{\pgfqpoint{2.548249in}{0.696537in}}%
\pgfpathcurveto{\pgfqpoint{2.556062in}{0.704351in}}{\pgfqpoint{2.560452in}{0.714950in}}{\pgfqpoint{2.560452in}{0.726000in}}%
\pgfpathcurveto{\pgfqpoint{2.560452in}{0.737050in}}{\pgfqpoint{2.556062in}{0.747649in}}{\pgfqpoint{2.548249in}{0.755463in}}%
\pgfpathcurveto{\pgfqpoint{2.540435in}{0.763276in}}{\pgfqpoint{2.529836in}{0.767667in}}{\pgfqpoint{2.518786in}{0.767667in}}%
\pgfpathcurveto{\pgfqpoint{2.507736in}{0.767667in}}{\pgfqpoint{2.497137in}{0.763276in}}{\pgfqpoint{2.489323in}{0.755463in}}%
\pgfpathcurveto{\pgfqpoint{2.481509in}{0.747649in}}{\pgfqpoint{2.477119in}{0.737050in}}{\pgfqpoint{2.477119in}{0.726000in}}%
\pgfpathcurveto{\pgfqpoint{2.477119in}{0.714950in}}{\pgfqpoint{2.481509in}{0.704351in}}{\pgfqpoint{2.489323in}{0.696537in}}%
\pgfpathcurveto{\pgfqpoint{2.497137in}{0.688724in}}{\pgfqpoint{2.507736in}{0.684333in}}{\pgfqpoint{2.518786in}{0.684333in}}%
\pgfpathclose%
\pgfusepath{stroke,fill}%
\end{pgfscope}%
\begin{pgfscope}%
\pgfpathrectangle{\pgfqpoint{0.800000in}{0.528000in}}{\pgfqpoint{4.960000in}{3.696000in}}%
\pgfusepath{clip}%
\pgfsetbuttcap%
\pgfsetroundjoin%
\definecolor{currentfill}{rgb}{0.000000,0.000000,0.000000}%
\pgfsetfillcolor{currentfill}%
\pgfsetlinewidth{1.003750pt}%
\definecolor{currentstroke}{rgb}{0.000000,0.000000,0.000000}%
\pgfsetstrokecolor{currentstroke}%
\pgfsetdash{}{0pt}%
\pgfpathmoveto{\pgfqpoint{2.518786in}{0.684333in}}%
\pgfpathcurveto{\pgfqpoint{2.529836in}{0.684333in}}{\pgfqpoint{2.540435in}{0.688724in}}{\pgfqpoint{2.548249in}{0.696537in}}%
\pgfpathcurveto{\pgfqpoint{2.556062in}{0.704351in}}{\pgfqpoint{2.560452in}{0.714950in}}{\pgfqpoint{2.560452in}{0.726000in}}%
\pgfpathcurveto{\pgfqpoint{2.560452in}{0.737050in}}{\pgfqpoint{2.556062in}{0.747649in}}{\pgfqpoint{2.548249in}{0.755463in}}%
\pgfpathcurveto{\pgfqpoint{2.540435in}{0.763276in}}{\pgfqpoint{2.529836in}{0.767667in}}{\pgfqpoint{2.518786in}{0.767667in}}%
\pgfpathcurveto{\pgfqpoint{2.507736in}{0.767667in}}{\pgfqpoint{2.497137in}{0.763276in}}{\pgfqpoint{2.489323in}{0.755463in}}%
\pgfpathcurveto{\pgfqpoint{2.481509in}{0.747649in}}{\pgfqpoint{2.477119in}{0.737050in}}{\pgfqpoint{2.477119in}{0.726000in}}%
\pgfpathcurveto{\pgfqpoint{2.477119in}{0.714950in}}{\pgfqpoint{2.481509in}{0.704351in}}{\pgfqpoint{2.489323in}{0.696537in}}%
\pgfpathcurveto{\pgfqpoint{2.497137in}{0.688724in}}{\pgfqpoint{2.507736in}{0.684333in}}{\pgfqpoint{2.518786in}{0.684333in}}%
\pgfpathclose%
\pgfusepath{stroke,fill}%
\end{pgfscope}%
\begin{pgfscope}%
\pgfpathrectangle{\pgfqpoint{0.800000in}{0.528000in}}{\pgfqpoint{4.960000in}{3.696000in}}%
\pgfusepath{clip}%
\pgfsetbuttcap%
\pgfsetroundjoin%
\definecolor{currentfill}{rgb}{0.000000,0.000000,0.000000}%
\pgfsetfillcolor{currentfill}%
\pgfsetlinewidth{1.003750pt}%
\definecolor{currentstroke}{rgb}{0.000000,0.000000,0.000000}%
\pgfsetstrokecolor{currentstroke}%
\pgfsetdash{}{0pt}%
\pgfpathmoveto{\pgfqpoint{2.518786in}{0.684333in}}%
\pgfpathcurveto{\pgfqpoint{2.529836in}{0.684333in}}{\pgfqpoint{2.540435in}{0.688724in}}{\pgfqpoint{2.548249in}{0.696537in}}%
\pgfpathcurveto{\pgfqpoint{2.556062in}{0.704351in}}{\pgfqpoint{2.560452in}{0.714950in}}{\pgfqpoint{2.560452in}{0.726000in}}%
\pgfpathcurveto{\pgfqpoint{2.560452in}{0.737050in}}{\pgfqpoint{2.556062in}{0.747649in}}{\pgfqpoint{2.548249in}{0.755463in}}%
\pgfpathcurveto{\pgfqpoint{2.540435in}{0.763276in}}{\pgfqpoint{2.529836in}{0.767667in}}{\pgfqpoint{2.518786in}{0.767667in}}%
\pgfpathcurveto{\pgfqpoint{2.507736in}{0.767667in}}{\pgfqpoint{2.497137in}{0.763276in}}{\pgfqpoint{2.489323in}{0.755463in}}%
\pgfpathcurveto{\pgfqpoint{2.481509in}{0.747649in}}{\pgfqpoint{2.477119in}{0.737050in}}{\pgfqpoint{2.477119in}{0.726000in}}%
\pgfpathcurveto{\pgfqpoint{2.477119in}{0.714950in}}{\pgfqpoint{2.481509in}{0.704351in}}{\pgfqpoint{2.489323in}{0.696537in}}%
\pgfpathcurveto{\pgfqpoint{2.497137in}{0.688724in}}{\pgfqpoint{2.507736in}{0.684333in}}{\pgfqpoint{2.518786in}{0.684333in}}%
\pgfpathclose%
\pgfusepath{stroke,fill}%
\end{pgfscope}%
\begin{pgfscope}%
\pgfpathrectangle{\pgfqpoint{0.800000in}{0.528000in}}{\pgfqpoint{4.960000in}{3.696000in}}%
\pgfusepath{clip}%
\pgfsetbuttcap%
\pgfsetroundjoin%
\definecolor{currentfill}{rgb}{0.000000,0.000000,0.000000}%
\pgfsetfillcolor{currentfill}%
\pgfsetlinewidth{1.003750pt}%
\definecolor{currentstroke}{rgb}{0.000000,0.000000,0.000000}%
\pgfsetstrokecolor{currentstroke}%
\pgfsetdash{}{0pt}%
\pgfpathmoveto{\pgfqpoint{2.518786in}{0.684333in}}%
\pgfpathcurveto{\pgfqpoint{2.529836in}{0.684333in}}{\pgfqpoint{2.540435in}{0.688724in}}{\pgfqpoint{2.548249in}{0.696537in}}%
\pgfpathcurveto{\pgfqpoint{2.556062in}{0.704351in}}{\pgfqpoint{2.560452in}{0.714950in}}{\pgfqpoint{2.560452in}{0.726000in}}%
\pgfpathcurveto{\pgfqpoint{2.560452in}{0.737050in}}{\pgfqpoint{2.556062in}{0.747649in}}{\pgfqpoint{2.548249in}{0.755463in}}%
\pgfpathcurveto{\pgfqpoint{2.540435in}{0.763276in}}{\pgfqpoint{2.529836in}{0.767667in}}{\pgfqpoint{2.518786in}{0.767667in}}%
\pgfpathcurveto{\pgfqpoint{2.507736in}{0.767667in}}{\pgfqpoint{2.497137in}{0.763276in}}{\pgfqpoint{2.489323in}{0.755463in}}%
\pgfpathcurveto{\pgfqpoint{2.481509in}{0.747649in}}{\pgfqpoint{2.477119in}{0.737050in}}{\pgfqpoint{2.477119in}{0.726000in}}%
\pgfpathcurveto{\pgfqpoint{2.477119in}{0.714950in}}{\pgfqpoint{2.481509in}{0.704351in}}{\pgfqpoint{2.489323in}{0.696537in}}%
\pgfpathcurveto{\pgfqpoint{2.497137in}{0.688724in}}{\pgfqpoint{2.507736in}{0.684333in}}{\pgfqpoint{2.518786in}{0.684333in}}%
\pgfpathclose%
\pgfusepath{stroke,fill}%
\end{pgfscope}%
\begin{pgfscope}%
\pgfpathrectangle{\pgfqpoint{0.800000in}{0.528000in}}{\pgfqpoint{4.960000in}{3.696000in}}%
\pgfusepath{clip}%
\pgfsetbuttcap%
\pgfsetroundjoin%
\definecolor{currentfill}{rgb}{0.000000,0.000000,0.000000}%
\pgfsetfillcolor{currentfill}%
\pgfsetlinewidth{1.003750pt}%
\definecolor{currentstroke}{rgb}{0.000000,0.000000,0.000000}%
\pgfsetstrokecolor{currentstroke}%
\pgfsetdash{}{0pt}%
\pgfpathmoveto{\pgfqpoint{2.518786in}{0.684333in}}%
\pgfpathcurveto{\pgfqpoint{2.529836in}{0.684333in}}{\pgfqpoint{2.540435in}{0.688724in}}{\pgfqpoint{2.548249in}{0.696537in}}%
\pgfpathcurveto{\pgfqpoint{2.556062in}{0.704351in}}{\pgfqpoint{2.560452in}{0.714950in}}{\pgfqpoint{2.560452in}{0.726000in}}%
\pgfpathcurveto{\pgfqpoint{2.560452in}{0.737050in}}{\pgfqpoint{2.556062in}{0.747649in}}{\pgfqpoint{2.548249in}{0.755463in}}%
\pgfpathcurveto{\pgfqpoint{2.540435in}{0.763276in}}{\pgfqpoint{2.529836in}{0.767667in}}{\pgfqpoint{2.518786in}{0.767667in}}%
\pgfpathcurveto{\pgfqpoint{2.507736in}{0.767667in}}{\pgfqpoint{2.497137in}{0.763276in}}{\pgfqpoint{2.489323in}{0.755463in}}%
\pgfpathcurveto{\pgfqpoint{2.481509in}{0.747649in}}{\pgfqpoint{2.477119in}{0.737050in}}{\pgfqpoint{2.477119in}{0.726000in}}%
\pgfpathcurveto{\pgfqpoint{2.477119in}{0.714950in}}{\pgfqpoint{2.481509in}{0.704351in}}{\pgfqpoint{2.489323in}{0.696537in}}%
\pgfpathcurveto{\pgfqpoint{2.497137in}{0.688724in}}{\pgfqpoint{2.507736in}{0.684333in}}{\pgfqpoint{2.518786in}{0.684333in}}%
\pgfpathclose%
\pgfusepath{stroke,fill}%
\end{pgfscope}%
\begin{pgfscope}%
\pgfpathrectangle{\pgfqpoint{0.800000in}{0.528000in}}{\pgfqpoint{4.960000in}{3.696000in}}%
\pgfusepath{clip}%
\pgfsetbuttcap%
\pgfsetroundjoin%
\definecolor{currentfill}{rgb}{0.000000,0.000000,0.000000}%
\pgfsetfillcolor{currentfill}%
\pgfsetlinewidth{1.003750pt}%
\definecolor{currentstroke}{rgb}{0.000000,0.000000,0.000000}%
\pgfsetstrokecolor{currentstroke}%
\pgfsetdash{}{0pt}%
\pgfpathmoveto{\pgfqpoint{2.518786in}{0.684333in}}%
\pgfpathcurveto{\pgfqpoint{2.529836in}{0.684333in}}{\pgfqpoint{2.540435in}{0.688724in}}{\pgfqpoint{2.548249in}{0.696537in}}%
\pgfpathcurveto{\pgfqpoint{2.556062in}{0.704351in}}{\pgfqpoint{2.560452in}{0.714950in}}{\pgfqpoint{2.560452in}{0.726000in}}%
\pgfpathcurveto{\pgfqpoint{2.560452in}{0.737050in}}{\pgfqpoint{2.556062in}{0.747649in}}{\pgfqpoint{2.548249in}{0.755463in}}%
\pgfpathcurveto{\pgfqpoint{2.540435in}{0.763276in}}{\pgfqpoint{2.529836in}{0.767667in}}{\pgfqpoint{2.518786in}{0.767667in}}%
\pgfpathcurveto{\pgfqpoint{2.507736in}{0.767667in}}{\pgfqpoint{2.497137in}{0.763276in}}{\pgfqpoint{2.489323in}{0.755463in}}%
\pgfpathcurveto{\pgfqpoint{2.481509in}{0.747649in}}{\pgfqpoint{2.477119in}{0.737050in}}{\pgfqpoint{2.477119in}{0.726000in}}%
\pgfpathcurveto{\pgfqpoint{2.477119in}{0.714950in}}{\pgfqpoint{2.481509in}{0.704351in}}{\pgfqpoint{2.489323in}{0.696537in}}%
\pgfpathcurveto{\pgfqpoint{2.497137in}{0.688724in}}{\pgfqpoint{2.507736in}{0.684333in}}{\pgfqpoint{2.518786in}{0.684333in}}%
\pgfpathclose%
\pgfusepath{stroke,fill}%
\end{pgfscope}%
\begin{pgfscope}%
\pgfpathrectangle{\pgfqpoint{0.800000in}{0.528000in}}{\pgfqpoint{4.960000in}{3.696000in}}%
\pgfusepath{clip}%
\pgfsetbuttcap%
\pgfsetroundjoin%
\definecolor{currentfill}{rgb}{0.000000,0.000000,0.000000}%
\pgfsetfillcolor{currentfill}%
\pgfsetlinewidth{1.003750pt}%
\definecolor{currentstroke}{rgb}{0.000000,0.000000,0.000000}%
\pgfsetstrokecolor{currentstroke}%
\pgfsetdash{}{0pt}%
\pgfpathmoveto{\pgfqpoint{2.518786in}{0.684333in}}%
\pgfpathcurveto{\pgfqpoint{2.529836in}{0.684333in}}{\pgfqpoint{2.540435in}{0.688724in}}{\pgfqpoint{2.548249in}{0.696537in}}%
\pgfpathcurveto{\pgfqpoint{2.556062in}{0.704351in}}{\pgfqpoint{2.560452in}{0.714950in}}{\pgfqpoint{2.560452in}{0.726000in}}%
\pgfpathcurveto{\pgfqpoint{2.560452in}{0.737050in}}{\pgfqpoint{2.556062in}{0.747649in}}{\pgfqpoint{2.548249in}{0.755463in}}%
\pgfpathcurveto{\pgfqpoint{2.540435in}{0.763276in}}{\pgfqpoint{2.529836in}{0.767667in}}{\pgfqpoint{2.518786in}{0.767667in}}%
\pgfpathcurveto{\pgfqpoint{2.507736in}{0.767667in}}{\pgfqpoint{2.497137in}{0.763276in}}{\pgfqpoint{2.489323in}{0.755463in}}%
\pgfpathcurveto{\pgfqpoint{2.481509in}{0.747649in}}{\pgfqpoint{2.477119in}{0.737050in}}{\pgfqpoint{2.477119in}{0.726000in}}%
\pgfpathcurveto{\pgfqpoint{2.477119in}{0.714950in}}{\pgfqpoint{2.481509in}{0.704351in}}{\pgfqpoint{2.489323in}{0.696537in}}%
\pgfpathcurveto{\pgfqpoint{2.497137in}{0.688724in}}{\pgfqpoint{2.507736in}{0.684333in}}{\pgfqpoint{2.518786in}{0.684333in}}%
\pgfpathclose%
\pgfusepath{stroke,fill}%
\end{pgfscope}%
\begin{pgfscope}%
\pgfpathrectangle{\pgfqpoint{0.800000in}{0.528000in}}{\pgfqpoint{4.960000in}{3.696000in}}%
\pgfusepath{clip}%
\pgfsetbuttcap%
\pgfsetroundjoin%
\definecolor{currentfill}{rgb}{0.000000,0.000000,0.000000}%
\pgfsetfillcolor{currentfill}%
\pgfsetlinewidth{1.003750pt}%
\definecolor{currentstroke}{rgb}{0.000000,0.000000,0.000000}%
\pgfsetstrokecolor{currentstroke}%
\pgfsetdash{}{0pt}%
\pgfpathmoveto{\pgfqpoint{2.518786in}{0.684333in}}%
\pgfpathcurveto{\pgfqpoint{2.529836in}{0.684333in}}{\pgfqpoint{2.540435in}{0.688724in}}{\pgfqpoint{2.548249in}{0.696537in}}%
\pgfpathcurveto{\pgfqpoint{2.556062in}{0.704351in}}{\pgfqpoint{2.560452in}{0.714950in}}{\pgfqpoint{2.560452in}{0.726000in}}%
\pgfpathcurveto{\pgfqpoint{2.560452in}{0.737050in}}{\pgfqpoint{2.556062in}{0.747649in}}{\pgfqpoint{2.548249in}{0.755463in}}%
\pgfpathcurveto{\pgfqpoint{2.540435in}{0.763276in}}{\pgfqpoint{2.529836in}{0.767667in}}{\pgfqpoint{2.518786in}{0.767667in}}%
\pgfpathcurveto{\pgfqpoint{2.507736in}{0.767667in}}{\pgfqpoint{2.497137in}{0.763276in}}{\pgfqpoint{2.489323in}{0.755463in}}%
\pgfpathcurveto{\pgfqpoint{2.481509in}{0.747649in}}{\pgfqpoint{2.477119in}{0.737050in}}{\pgfqpoint{2.477119in}{0.726000in}}%
\pgfpathcurveto{\pgfqpoint{2.477119in}{0.714950in}}{\pgfqpoint{2.481509in}{0.704351in}}{\pgfqpoint{2.489323in}{0.696537in}}%
\pgfpathcurveto{\pgfqpoint{2.497137in}{0.688724in}}{\pgfqpoint{2.507736in}{0.684333in}}{\pgfqpoint{2.518786in}{0.684333in}}%
\pgfpathclose%
\pgfusepath{stroke,fill}%
\end{pgfscope}%
\begin{pgfscope}%
\pgfpathrectangle{\pgfqpoint{0.800000in}{0.528000in}}{\pgfqpoint{4.960000in}{3.696000in}}%
\pgfusepath{clip}%
\pgfsetbuttcap%
\pgfsetroundjoin%
\definecolor{currentfill}{rgb}{0.000000,0.000000,0.000000}%
\pgfsetfillcolor{currentfill}%
\pgfsetlinewidth{1.003750pt}%
\definecolor{currentstroke}{rgb}{0.000000,0.000000,0.000000}%
\pgfsetstrokecolor{currentstroke}%
\pgfsetdash{}{0pt}%
\pgfpathmoveto{\pgfqpoint{2.518786in}{0.684333in}}%
\pgfpathcurveto{\pgfqpoint{2.529836in}{0.684333in}}{\pgfqpoint{2.540435in}{0.688724in}}{\pgfqpoint{2.548249in}{0.696537in}}%
\pgfpathcurveto{\pgfqpoint{2.556062in}{0.704351in}}{\pgfqpoint{2.560452in}{0.714950in}}{\pgfqpoint{2.560452in}{0.726000in}}%
\pgfpathcurveto{\pgfqpoint{2.560452in}{0.737050in}}{\pgfqpoint{2.556062in}{0.747649in}}{\pgfqpoint{2.548249in}{0.755463in}}%
\pgfpathcurveto{\pgfqpoint{2.540435in}{0.763276in}}{\pgfqpoint{2.529836in}{0.767667in}}{\pgfqpoint{2.518786in}{0.767667in}}%
\pgfpathcurveto{\pgfqpoint{2.507736in}{0.767667in}}{\pgfqpoint{2.497137in}{0.763276in}}{\pgfqpoint{2.489323in}{0.755463in}}%
\pgfpathcurveto{\pgfqpoint{2.481509in}{0.747649in}}{\pgfqpoint{2.477119in}{0.737050in}}{\pgfqpoint{2.477119in}{0.726000in}}%
\pgfpathcurveto{\pgfqpoint{2.477119in}{0.714950in}}{\pgfqpoint{2.481509in}{0.704351in}}{\pgfqpoint{2.489323in}{0.696537in}}%
\pgfpathcurveto{\pgfqpoint{2.497137in}{0.688724in}}{\pgfqpoint{2.507736in}{0.684333in}}{\pgfqpoint{2.518786in}{0.684333in}}%
\pgfpathclose%
\pgfusepath{stroke,fill}%
\end{pgfscope}%
\begin{pgfscope}%
\pgfpathrectangle{\pgfqpoint{0.800000in}{0.528000in}}{\pgfqpoint{4.960000in}{3.696000in}}%
\pgfusepath{clip}%
\pgfsetbuttcap%
\pgfsetroundjoin%
\definecolor{currentfill}{rgb}{0.000000,0.000000,0.000000}%
\pgfsetfillcolor{currentfill}%
\pgfsetlinewidth{1.003750pt}%
\definecolor{currentstroke}{rgb}{0.000000,0.000000,0.000000}%
\pgfsetstrokecolor{currentstroke}%
\pgfsetdash{}{0pt}%
\pgfpathmoveto{\pgfqpoint{2.518786in}{0.684333in}}%
\pgfpathcurveto{\pgfqpoint{2.529836in}{0.684333in}}{\pgfqpoint{2.540435in}{0.688724in}}{\pgfqpoint{2.548249in}{0.696537in}}%
\pgfpathcurveto{\pgfqpoint{2.556062in}{0.704351in}}{\pgfqpoint{2.560452in}{0.714950in}}{\pgfqpoint{2.560452in}{0.726000in}}%
\pgfpathcurveto{\pgfqpoint{2.560452in}{0.737050in}}{\pgfqpoint{2.556062in}{0.747649in}}{\pgfqpoint{2.548249in}{0.755463in}}%
\pgfpathcurveto{\pgfqpoint{2.540435in}{0.763276in}}{\pgfqpoint{2.529836in}{0.767667in}}{\pgfqpoint{2.518786in}{0.767667in}}%
\pgfpathcurveto{\pgfqpoint{2.507736in}{0.767667in}}{\pgfqpoint{2.497137in}{0.763276in}}{\pgfqpoint{2.489323in}{0.755463in}}%
\pgfpathcurveto{\pgfqpoint{2.481509in}{0.747649in}}{\pgfqpoint{2.477119in}{0.737050in}}{\pgfqpoint{2.477119in}{0.726000in}}%
\pgfpathcurveto{\pgfqpoint{2.477119in}{0.714950in}}{\pgfqpoint{2.481509in}{0.704351in}}{\pgfqpoint{2.489323in}{0.696537in}}%
\pgfpathcurveto{\pgfqpoint{2.497137in}{0.688724in}}{\pgfqpoint{2.507736in}{0.684333in}}{\pgfqpoint{2.518786in}{0.684333in}}%
\pgfpathclose%
\pgfusepath{stroke,fill}%
\end{pgfscope}%
\begin{pgfscope}%
\pgfpathrectangle{\pgfqpoint{0.800000in}{0.528000in}}{\pgfqpoint{4.960000in}{3.696000in}}%
\pgfusepath{clip}%
\pgfsetbuttcap%
\pgfsetroundjoin%
\definecolor{currentfill}{rgb}{0.000000,0.000000,0.000000}%
\pgfsetfillcolor{currentfill}%
\pgfsetlinewidth{1.003750pt}%
\definecolor{currentstroke}{rgb}{0.000000,0.000000,0.000000}%
\pgfsetstrokecolor{currentstroke}%
\pgfsetdash{}{0pt}%
\pgfpathmoveto{\pgfqpoint{2.518786in}{0.684333in}}%
\pgfpathcurveto{\pgfqpoint{2.529836in}{0.684333in}}{\pgfqpoint{2.540435in}{0.688724in}}{\pgfqpoint{2.548249in}{0.696537in}}%
\pgfpathcurveto{\pgfqpoint{2.556062in}{0.704351in}}{\pgfqpoint{2.560452in}{0.714950in}}{\pgfqpoint{2.560452in}{0.726000in}}%
\pgfpathcurveto{\pgfqpoint{2.560452in}{0.737050in}}{\pgfqpoint{2.556062in}{0.747649in}}{\pgfqpoint{2.548249in}{0.755463in}}%
\pgfpathcurveto{\pgfqpoint{2.540435in}{0.763276in}}{\pgfqpoint{2.529836in}{0.767667in}}{\pgfqpoint{2.518786in}{0.767667in}}%
\pgfpathcurveto{\pgfqpoint{2.507736in}{0.767667in}}{\pgfqpoint{2.497137in}{0.763276in}}{\pgfqpoint{2.489323in}{0.755463in}}%
\pgfpathcurveto{\pgfqpoint{2.481509in}{0.747649in}}{\pgfqpoint{2.477119in}{0.737050in}}{\pgfqpoint{2.477119in}{0.726000in}}%
\pgfpathcurveto{\pgfqpoint{2.477119in}{0.714950in}}{\pgfqpoint{2.481509in}{0.704351in}}{\pgfqpoint{2.489323in}{0.696537in}}%
\pgfpathcurveto{\pgfqpoint{2.497137in}{0.688724in}}{\pgfqpoint{2.507736in}{0.684333in}}{\pgfqpoint{2.518786in}{0.684333in}}%
\pgfpathclose%
\pgfusepath{stroke,fill}%
\end{pgfscope}%
\begin{pgfscope}%
\pgfpathrectangle{\pgfqpoint{0.800000in}{0.528000in}}{\pgfqpoint{4.960000in}{3.696000in}}%
\pgfusepath{clip}%
\pgfsetbuttcap%
\pgfsetroundjoin%
\definecolor{currentfill}{rgb}{0.000000,0.000000,0.000000}%
\pgfsetfillcolor{currentfill}%
\pgfsetlinewidth{1.003750pt}%
\definecolor{currentstroke}{rgb}{0.000000,0.000000,0.000000}%
\pgfsetstrokecolor{currentstroke}%
\pgfsetdash{}{0pt}%
\pgfpathmoveto{\pgfqpoint{2.518786in}{0.684333in}}%
\pgfpathcurveto{\pgfqpoint{2.529836in}{0.684333in}}{\pgfqpoint{2.540435in}{0.688724in}}{\pgfqpoint{2.548249in}{0.696537in}}%
\pgfpathcurveto{\pgfqpoint{2.556062in}{0.704351in}}{\pgfqpoint{2.560452in}{0.714950in}}{\pgfqpoint{2.560452in}{0.726000in}}%
\pgfpathcurveto{\pgfqpoint{2.560452in}{0.737050in}}{\pgfqpoint{2.556062in}{0.747649in}}{\pgfqpoint{2.548249in}{0.755463in}}%
\pgfpathcurveto{\pgfqpoint{2.540435in}{0.763276in}}{\pgfqpoint{2.529836in}{0.767667in}}{\pgfqpoint{2.518786in}{0.767667in}}%
\pgfpathcurveto{\pgfqpoint{2.507736in}{0.767667in}}{\pgfqpoint{2.497137in}{0.763276in}}{\pgfqpoint{2.489323in}{0.755463in}}%
\pgfpathcurveto{\pgfqpoint{2.481509in}{0.747649in}}{\pgfqpoint{2.477119in}{0.737050in}}{\pgfqpoint{2.477119in}{0.726000in}}%
\pgfpathcurveto{\pgfqpoint{2.477119in}{0.714950in}}{\pgfqpoint{2.481509in}{0.704351in}}{\pgfqpoint{2.489323in}{0.696537in}}%
\pgfpathcurveto{\pgfqpoint{2.497137in}{0.688724in}}{\pgfqpoint{2.507736in}{0.684333in}}{\pgfqpoint{2.518786in}{0.684333in}}%
\pgfpathclose%
\pgfusepath{stroke,fill}%
\end{pgfscope}%
\begin{pgfscope}%
\pgfpathrectangle{\pgfqpoint{0.800000in}{0.528000in}}{\pgfqpoint{4.960000in}{3.696000in}}%
\pgfusepath{clip}%
\pgfsetbuttcap%
\pgfsetroundjoin%
\definecolor{currentfill}{rgb}{0.000000,0.000000,0.000000}%
\pgfsetfillcolor{currentfill}%
\pgfsetlinewidth{1.003750pt}%
\definecolor{currentstroke}{rgb}{0.000000,0.000000,0.000000}%
\pgfsetstrokecolor{currentstroke}%
\pgfsetdash{}{0pt}%
\pgfpathmoveto{\pgfqpoint{2.518786in}{0.684333in}}%
\pgfpathcurveto{\pgfqpoint{2.529836in}{0.684333in}}{\pgfqpoint{2.540435in}{0.688724in}}{\pgfqpoint{2.548249in}{0.696537in}}%
\pgfpathcurveto{\pgfqpoint{2.556062in}{0.704351in}}{\pgfqpoint{2.560452in}{0.714950in}}{\pgfqpoint{2.560452in}{0.726000in}}%
\pgfpathcurveto{\pgfqpoint{2.560452in}{0.737050in}}{\pgfqpoint{2.556062in}{0.747649in}}{\pgfqpoint{2.548249in}{0.755463in}}%
\pgfpathcurveto{\pgfqpoint{2.540435in}{0.763276in}}{\pgfqpoint{2.529836in}{0.767667in}}{\pgfqpoint{2.518786in}{0.767667in}}%
\pgfpathcurveto{\pgfqpoint{2.507736in}{0.767667in}}{\pgfqpoint{2.497137in}{0.763276in}}{\pgfqpoint{2.489323in}{0.755463in}}%
\pgfpathcurveto{\pgfqpoint{2.481509in}{0.747649in}}{\pgfqpoint{2.477119in}{0.737050in}}{\pgfqpoint{2.477119in}{0.726000in}}%
\pgfpathcurveto{\pgfqpoint{2.477119in}{0.714950in}}{\pgfqpoint{2.481509in}{0.704351in}}{\pgfqpoint{2.489323in}{0.696537in}}%
\pgfpathcurveto{\pgfqpoint{2.497137in}{0.688724in}}{\pgfqpoint{2.507736in}{0.684333in}}{\pgfqpoint{2.518786in}{0.684333in}}%
\pgfpathclose%
\pgfusepath{stroke,fill}%
\end{pgfscope}%
\begin{pgfscope}%
\pgfpathrectangle{\pgfqpoint{0.800000in}{0.528000in}}{\pgfqpoint{4.960000in}{3.696000in}}%
\pgfusepath{clip}%
\pgfsetbuttcap%
\pgfsetroundjoin%
\definecolor{currentfill}{rgb}{0.000000,0.000000,0.000000}%
\pgfsetfillcolor{currentfill}%
\pgfsetlinewidth{1.003750pt}%
\definecolor{currentstroke}{rgb}{0.000000,0.000000,0.000000}%
\pgfsetstrokecolor{currentstroke}%
\pgfsetdash{}{0pt}%
\pgfpathmoveto{\pgfqpoint{2.518786in}{0.684333in}}%
\pgfpathcurveto{\pgfqpoint{2.529836in}{0.684333in}}{\pgfqpoint{2.540435in}{0.688724in}}{\pgfqpoint{2.548249in}{0.696537in}}%
\pgfpathcurveto{\pgfqpoint{2.556062in}{0.704351in}}{\pgfqpoint{2.560452in}{0.714950in}}{\pgfqpoint{2.560452in}{0.726000in}}%
\pgfpathcurveto{\pgfqpoint{2.560452in}{0.737050in}}{\pgfqpoint{2.556062in}{0.747649in}}{\pgfqpoint{2.548249in}{0.755463in}}%
\pgfpathcurveto{\pgfqpoint{2.540435in}{0.763276in}}{\pgfqpoint{2.529836in}{0.767667in}}{\pgfqpoint{2.518786in}{0.767667in}}%
\pgfpathcurveto{\pgfqpoint{2.507736in}{0.767667in}}{\pgfqpoint{2.497137in}{0.763276in}}{\pgfqpoint{2.489323in}{0.755463in}}%
\pgfpathcurveto{\pgfqpoint{2.481509in}{0.747649in}}{\pgfqpoint{2.477119in}{0.737050in}}{\pgfqpoint{2.477119in}{0.726000in}}%
\pgfpathcurveto{\pgfqpoint{2.477119in}{0.714950in}}{\pgfqpoint{2.481509in}{0.704351in}}{\pgfqpoint{2.489323in}{0.696537in}}%
\pgfpathcurveto{\pgfqpoint{2.497137in}{0.688724in}}{\pgfqpoint{2.507736in}{0.684333in}}{\pgfqpoint{2.518786in}{0.684333in}}%
\pgfpathclose%
\pgfusepath{stroke,fill}%
\end{pgfscope}%
\begin{pgfscope}%
\pgfpathrectangle{\pgfqpoint{0.800000in}{0.528000in}}{\pgfqpoint{4.960000in}{3.696000in}}%
\pgfusepath{clip}%
\pgfsetbuttcap%
\pgfsetroundjoin%
\definecolor{currentfill}{rgb}{0.000000,0.000000,0.000000}%
\pgfsetfillcolor{currentfill}%
\pgfsetlinewidth{1.003750pt}%
\definecolor{currentstroke}{rgb}{0.000000,0.000000,0.000000}%
\pgfsetstrokecolor{currentstroke}%
\pgfsetdash{}{0pt}%
\pgfpathmoveto{\pgfqpoint{2.518786in}{0.684333in}}%
\pgfpathcurveto{\pgfqpoint{2.529836in}{0.684333in}}{\pgfqpoint{2.540435in}{0.688724in}}{\pgfqpoint{2.548249in}{0.696537in}}%
\pgfpathcurveto{\pgfqpoint{2.556062in}{0.704351in}}{\pgfqpoint{2.560452in}{0.714950in}}{\pgfqpoint{2.560452in}{0.726000in}}%
\pgfpathcurveto{\pgfqpoint{2.560452in}{0.737050in}}{\pgfqpoint{2.556062in}{0.747649in}}{\pgfqpoint{2.548249in}{0.755463in}}%
\pgfpathcurveto{\pgfqpoint{2.540435in}{0.763276in}}{\pgfqpoint{2.529836in}{0.767667in}}{\pgfqpoint{2.518786in}{0.767667in}}%
\pgfpathcurveto{\pgfqpoint{2.507736in}{0.767667in}}{\pgfqpoint{2.497137in}{0.763276in}}{\pgfqpoint{2.489323in}{0.755463in}}%
\pgfpathcurveto{\pgfqpoint{2.481509in}{0.747649in}}{\pgfqpoint{2.477119in}{0.737050in}}{\pgfqpoint{2.477119in}{0.726000in}}%
\pgfpathcurveto{\pgfqpoint{2.477119in}{0.714950in}}{\pgfqpoint{2.481509in}{0.704351in}}{\pgfqpoint{2.489323in}{0.696537in}}%
\pgfpathcurveto{\pgfqpoint{2.497137in}{0.688724in}}{\pgfqpoint{2.507736in}{0.684333in}}{\pgfqpoint{2.518786in}{0.684333in}}%
\pgfpathclose%
\pgfusepath{stroke,fill}%
\end{pgfscope}%
\begin{pgfscope}%
\pgfpathrectangle{\pgfqpoint{0.800000in}{0.528000in}}{\pgfqpoint{4.960000in}{3.696000in}}%
\pgfusepath{clip}%
\pgfsetbuttcap%
\pgfsetroundjoin%
\definecolor{currentfill}{rgb}{0.000000,0.000000,0.000000}%
\pgfsetfillcolor{currentfill}%
\pgfsetlinewidth{1.003750pt}%
\definecolor{currentstroke}{rgb}{0.000000,0.000000,0.000000}%
\pgfsetstrokecolor{currentstroke}%
\pgfsetdash{}{0pt}%
\pgfpathmoveto{\pgfqpoint{2.518786in}{0.684333in}}%
\pgfpathcurveto{\pgfqpoint{2.529836in}{0.684333in}}{\pgfqpoint{2.540435in}{0.688724in}}{\pgfqpoint{2.548249in}{0.696537in}}%
\pgfpathcurveto{\pgfqpoint{2.556062in}{0.704351in}}{\pgfqpoint{2.560452in}{0.714950in}}{\pgfqpoint{2.560452in}{0.726000in}}%
\pgfpathcurveto{\pgfqpoint{2.560452in}{0.737050in}}{\pgfqpoint{2.556062in}{0.747649in}}{\pgfqpoint{2.548249in}{0.755463in}}%
\pgfpathcurveto{\pgfqpoint{2.540435in}{0.763276in}}{\pgfqpoint{2.529836in}{0.767667in}}{\pgfqpoint{2.518786in}{0.767667in}}%
\pgfpathcurveto{\pgfqpoint{2.507736in}{0.767667in}}{\pgfqpoint{2.497137in}{0.763276in}}{\pgfqpoint{2.489323in}{0.755463in}}%
\pgfpathcurveto{\pgfqpoint{2.481509in}{0.747649in}}{\pgfqpoint{2.477119in}{0.737050in}}{\pgfqpoint{2.477119in}{0.726000in}}%
\pgfpathcurveto{\pgfqpoint{2.477119in}{0.714950in}}{\pgfqpoint{2.481509in}{0.704351in}}{\pgfqpoint{2.489323in}{0.696537in}}%
\pgfpathcurveto{\pgfqpoint{2.497137in}{0.688724in}}{\pgfqpoint{2.507736in}{0.684333in}}{\pgfqpoint{2.518786in}{0.684333in}}%
\pgfpathclose%
\pgfusepath{stroke,fill}%
\end{pgfscope}%
\begin{pgfscope}%
\pgfpathrectangle{\pgfqpoint{0.800000in}{0.528000in}}{\pgfqpoint{4.960000in}{3.696000in}}%
\pgfusepath{clip}%
\pgfsetbuttcap%
\pgfsetroundjoin%
\definecolor{currentfill}{rgb}{0.000000,0.000000,0.000000}%
\pgfsetfillcolor{currentfill}%
\pgfsetlinewidth{1.003750pt}%
\definecolor{currentstroke}{rgb}{0.000000,0.000000,0.000000}%
\pgfsetstrokecolor{currentstroke}%
\pgfsetdash{}{0pt}%
\pgfpathmoveto{\pgfqpoint{2.518786in}{0.684333in}}%
\pgfpathcurveto{\pgfqpoint{2.529836in}{0.684333in}}{\pgfqpoint{2.540435in}{0.688724in}}{\pgfqpoint{2.548249in}{0.696537in}}%
\pgfpathcurveto{\pgfqpoint{2.556062in}{0.704351in}}{\pgfqpoint{2.560452in}{0.714950in}}{\pgfqpoint{2.560452in}{0.726000in}}%
\pgfpathcurveto{\pgfqpoint{2.560452in}{0.737050in}}{\pgfqpoint{2.556062in}{0.747649in}}{\pgfqpoint{2.548249in}{0.755463in}}%
\pgfpathcurveto{\pgfqpoint{2.540435in}{0.763276in}}{\pgfqpoint{2.529836in}{0.767667in}}{\pgfqpoint{2.518786in}{0.767667in}}%
\pgfpathcurveto{\pgfqpoint{2.507736in}{0.767667in}}{\pgfqpoint{2.497137in}{0.763276in}}{\pgfqpoint{2.489323in}{0.755463in}}%
\pgfpathcurveto{\pgfqpoint{2.481509in}{0.747649in}}{\pgfqpoint{2.477119in}{0.737050in}}{\pgfqpoint{2.477119in}{0.726000in}}%
\pgfpathcurveto{\pgfqpoint{2.477119in}{0.714950in}}{\pgfqpoint{2.481509in}{0.704351in}}{\pgfqpoint{2.489323in}{0.696537in}}%
\pgfpathcurveto{\pgfqpoint{2.497137in}{0.688724in}}{\pgfqpoint{2.507736in}{0.684333in}}{\pgfqpoint{2.518786in}{0.684333in}}%
\pgfpathclose%
\pgfusepath{stroke,fill}%
\end{pgfscope}%
\begin{pgfscope}%
\pgfpathrectangle{\pgfqpoint{0.800000in}{0.528000in}}{\pgfqpoint{4.960000in}{3.696000in}}%
\pgfusepath{clip}%
\pgfsetbuttcap%
\pgfsetroundjoin%
\definecolor{currentfill}{rgb}{0.000000,0.000000,0.000000}%
\pgfsetfillcolor{currentfill}%
\pgfsetlinewidth{1.003750pt}%
\definecolor{currentstroke}{rgb}{0.000000,0.000000,0.000000}%
\pgfsetstrokecolor{currentstroke}%
\pgfsetdash{}{0pt}%
\pgfpathmoveto{\pgfqpoint{2.518786in}{0.684333in}}%
\pgfpathcurveto{\pgfqpoint{2.529836in}{0.684333in}}{\pgfqpoint{2.540435in}{0.688724in}}{\pgfqpoint{2.548249in}{0.696537in}}%
\pgfpathcurveto{\pgfqpoint{2.556062in}{0.704351in}}{\pgfqpoint{2.560452in}{0.714950in}}{\pgfqpoint{2.560452in}{0.726000in}}%
\pgfpathcurveto{\pgfqpoint{2.560452in}{0.737050in}}{\pgfqpoint{2.556062in}{0.747649in}}{\pgfqpoint{2.548249in}{0.755463in}}%
\pgfpathcurveto{\pgfqpoint{2.540435in}{0.763276in}}{\pgfqpoint{2.529836in}{0.767667in}}{\pgfqpoint{2.518786in}{0.767667in}}%
\pgfpathcurveto{\pgfqpoint{2.507736in}{0.767667in}}{\pgfqpoint{2.497137in}{0.763276in}}{\pgfqpoint{2.489323in}{0.755463in}}%
\pgfpathcurveto{\pgfqpoint{2.481509in}{0.747649in}}{\pgfqpoint{2.477119in}{0.737050in}}{\pgfqpoint{2.477119in}{0.726000in}}%
\pgfpathcurveto{\pgfqpoint{2.477119in}{0.714950in}}{\pgfqpoint{2.481509in}{0.704351in}}{\pgfqpoint{2.489323in}{0.696537in}}%
\pgfpathcurveto{\pgfqpoint{2.497137in}{0.688724in}}{\pgfqpoint{2.507736in}{0.684333in}}{\pgfqpoint{2.518786in}{0.684333in}}%
\pgfpathclose%
\pgfusepath{stroke,fill}%
\end{pgfscope}%
\begin{pgfscope}%
\pgfpathrectangle{\pgfqpoint{0.800000in}{0.528000in}}{\pgfqpoint{4.960000in}{3.696000in}}%
\pgfusepath{clip}%
\pgfsetbuttcap%
\pgfsetroundjoin%
\definecolor{currentfill}{rgb}{0.000000,0.000000,0.000000}%
\pgfsetfillcolor{currentfill}%
\pgfsetlinewidth{1.003750pt}%
\definecolor{currentstroke}{rgb}{0.000000,0.000000,0.000000}%
\pgfsetstrokecolor{currentstroke}%
\pgfsetdash{}{0pt}%
\pgfpathmoveto{\pgfqpoint{2.518786in}{0.684333in}}%
\pgfpathcurveto{\pgfqpoint{2.529836in}{0.684333in}}{\pgfqpoint{2.540435in}{0.688724in}}{\pgfqpoint{2.548249in}{0.696537in}}%
\pgfpathcurveto{\pgfqpoint{2.556062in}{0.704351in}}{\pgfqpoint{2.560452in}{0.714950in}}{\pgfqpoint{2.560452in}{0.726000in}}%
\pgfpathcurveto{\pgfqpoint{2.560452in}{0.737050in}}{\pgfqpoint{2.556062in}{0.747649in}}{\pgfqpoint{2.548249in}{0.755463in}}%
\pgfpathcurveto{\pgfqpoint{2.540435in}{0.763276in}}{\pgfqpoint{2.529836in}{0.767667in}}{\pgfqpoint{2.518786in}{0.767667in}}%
\pgfpathcurveto{\pgfqpoint{2.507736in}{0.767667in}}{\pgfqpoint{2.497137in}{0.763276in}}{\pgfqpoint{2.489323in}{0.755463in}}%
\pgfpathcurveto{\pgfqpoint{2.481509in}{0.747649in}}{\pgfqpoint{2.477119in}{0.737050in}}{\pgfqpoint{2.477119in}{0.726000in}}%
\pgfpathcurveto{\pgfqpoint{2.477119in}{0.714950in}}{\pgfqpoint{2.481509in}{0.704351in}}{\pgfqpoint{2.489323in}{0.696537in}}%
\pgfpathcurveto{\pgfqpoint{2.497137in}{0.688724in}}{\pgfqpoint{2.507736in}{0.684333in}}{\pgfqpoint{2.518786in}{0.684333in}}%
\pgfpathclose%
\pgfusepath{stroke,fill}%
\end{pgfscope}%
\begin{pgfscope}%
\pgfpathrectangle{\pgfqpoint{0.800000in}{0.528000in}}{\pgfqpoint{4.960000in}{3.696000in}}%
\pgfusepath{clip}%
\pgfsetbuttcap%
\pgfsetroundjoin%
\definecolor{currentfill}{rgb}{0.000000,0.000000,0.000000}%
\pgfsetfillcolor{currentfill}%
\pgfsetlinewidth{1.003750pt}%
\definecolor{currentstroke}{rgb}{0.000000,0.000000,0.000000}%
\pgfsetstrokecolor{currentstroke}%
\pgfsetdash{}{0pt}%
\pgfpathmoveto{\pgfqpoint{2.518786in}{0.684333in}}%
\pgfpathcurveto{\pgfqpoint{2.529836in}{0.684333in}}{\pgfqpoint{2.540435in}{0.688724in}}{\pgfqpoint{2.548249in}{0.696537in}}%
\pgfpathcurveto{\pgfqpoint{2.556062in}{0.704351in}}{\pgfqpoint{2.560452in}{0.714950in}}{\pgfqpoint{2.560452in}{0.726000in}}%
\pgfpathcurveto{\pgfqpoint{2.560452in}{0.737050in}}{\pgfqpoint{2.556062in}{0.747649in}}{\pgfqpoint{2.548249in}{0.755463in}}%
\pgfpathcurveto{\pgfqpoint{2.540435in}{0.763276in}}{\pgfqpoint{2.529836in}{0.767667in}}{\pgfqpoint{2.518786in}{0.767667in}}%
\pgfpathcurveto{\pgfqpoint{2.507736in}{0.767667in}}{\pgfqpoint{2.497137in}{0.763276in}}{\pgfqpoint{2.489323in}{0.755463in}}%
\pgfpathcurveto{\pgfqpoint{2.481509in}{0.747649in}}{\pgfqpoint{2.477119in}{0.737050in}}{\pgfqpoint{2.477119in}{0.726000in}}%
\pgfpathcurveto{\pgfqpoint{2.477119in}{0.714950in}}{\pgfqpoint{2.481509in}{0.704351in}}{\pgfqpoint{2.489323in}{0.696537in}}%
\pgfpathcurveto{\pgfqpoint{2.497137in}{0.688724in}}{\pgfqpoint{2.507736in}{0.684333in}}{\pgfqpoint{2.518786in}{0.684333in}}%
\pgfpathclose%
\pgfusepath{stroke,fill}%
\end{pgfscope}%
\begin{pgfscope}%
\pgfpathrectangle{\pgfqpoint{0.800000in}{0.528000in}}{\pgfqpoint{4.960000in}{3.696000in}}%
\pgfusepath{clip}%
\pgfsetbuttcap%
\pgfsetroundjoin%
\definecolor{currentfill}{rgb}{0.000000,0.000000,0.000000}%
\pgfsetfillcolor{currentfill}%
\pgfsetlinewidth{1.003750pt}%
\definecolor{currentstroke}{rgb}{0.000000,0.000000,0.000000}%
\pgfsetstrokecolor{currentstroke}%
\pgfsetdash{}{0pt}%
\pgfpathmoveto{\pgfqpoint{2.518786in}{0.684333in}}%
\pgfpathcurveto{\pgfqpoint{2.529836in}{0.684333in}}{\pgfqpoint{2.540435in}{0.688724in}}{\pgfqpoint{2.548249in}{0.696537in}}%
\pgfpathcurveto{\pgfqpoint{2.556062in}{0.704351in}}{\pgfqpoint{2.560452in}{0.714950in}}{\pgfqpoint{2.560452in}{0.726000in}}%
\pgfpathcurveto{\pgfqpoint{2.560452in}{0.737050in}}{\pgfqpoint{2.556062in}{0.747649in}}{\pgfqpoint{2.548249in}{0.755463in}}%
\pgfpathcurveto{\pgfqpoint{2.540435in}{0.763276in}}{\pgfqpoint{2.529836in}{0.767667in}}{\pgfqpoint{2.518786in}{0.767667in}}%
\pgfpathcurveto{\pgfqpoint{2.507736in}{0.767667in}}{\pgfqpoint{2.497137in}{0.763276in}}{\pgfqpoint{2.489323in}{0.755463in}}%
\pgfpathcurveto{\pgfqpoint{2.481509in}{0.747649in}}{\pgfqpoint{2.477119in}{0.737050in}}{\pgfqpoint{2.477119in}{0.726000in}}%
\pgfpathcurveto{\pgfqpoint{2.477119in}{0.714950in}}{\pgfqpoint{2.481509in}{0.704351in}}{\pgfqpoint{2.489323in}{0.696537in}}%
\pgfpathcurveto{\pgfqpoint{2.497137in}{0.688724in}}{\pgfqpoint{2.507736in}{0.684333in}}{\pgfqpoint{2.518786in}{0.684333in}}%
\pgfpathclose%
\pgfusepath{stroke,fill}%
\end{pgfscope}%
\begin{pgfscope}%
\pgfpathrectangle{\pgfqpoint{0.800000in}{0.528000in}}{\pgfqpoint{4.960000in}{3.696000in}}%
\pgfusepath{clip}%
\pgfsetbuttcap%
\pgfsetroundjoin%
\definecolor{currentfill}{rgb}{0.000000,0.000000,0.000000}%
\pgfsetfillcolor{currentfill}%
\pgfsetlinewidth{1.003750pt}%
\definecolor{currentstroke}{rgb}{0.000000,0.000000,0.000000}%
\pgfsetstrokecolor{currentstroke}%
\pgfsetdash{}{0pt}%
\pgfpathmoveto{\pgfqpoint{2.518786in}{0.684333in}}%
\pgfpathcurveto{\pgfqpoint{2.529836in}{0.684333in}}{\pgfqpoint{2.540435in}{0.688724in}}{\pgfqpoint{2.548249in}{0.696537in}}%
\pgfpathcurveto{\pgfqpoint{2.556062in}{0.704351in}}{\pgfqpoint{2.560452in}{0.714950in}}{\pgfqpoint{2.560452in}{0.726000in}}%
\pgfpathcurveto{\pgfqpoint{2.560452in}{0.737050in}}{\pgfqpoint{2.556062in}{0.747649in}}{\pgfqpoint{2.548249in}{0.755463in}}%
\pgfpathcurveto{\pgfqpoint{2.540435in}{0.763276in}}{\pgfqpoint{2.529836in}{0.767667in}}{\pgfqpoint{2.518786in}{0.767667in}}%
\pgfpathcurveto{\pgfqpoint{2.507736in}{0.767667in}}{\pgfqpoint{2.497137in}{0.763276in}}{\pgfqpoint{2.489323in}{0.755463in}}%
\pgfpathcurveto{\pgfqpoint{2.481509in}{0.747649in}}{\pgfqpoint{2.477119in}{0.737050in}}{\pgfqpoint{2.477119in}{0.726000in}}%
\pgfpathcurveto{\pgfqpoint{2.477119in}{0.714950in}}{\pgfqpoint{2.481509in}{0.704351in}}{\pgfqpoint{2.489323in}{0.696537in}}%
\pgfpathcurveto{\pgfqpoint{2.497137in}{0.688724in}}{\pgfqpoint{2.507736in}{0.684333in}}{\pgfqpoint{2.518786in}{0.684333in}}%
\pgfpathclose%
\pgfusepath{stroke,fill}%
\end{pgfscope}%
\begin{pgfscope}%
\pgfpathrectangle{\pgfqpoint{0.800000in}{0.528000in}}{\pgfqpoint{4.960000in}{3.696000in}}%
\pgfusepath{clip}%
\pgfsetbuttcap%
\pgfsetroundjoin%
\definecolor{currentfill}{rgb}{0.000000,0.000000,0.000000}%
\pgfsetfillcolor{currentfill}%
\pgfsetlinewidth{1.003750pt}%
\definecolor{currentstroke}{rgb}{0.000000,0.000000,0.000000}%
\pgfsetstrokecolor{currentstroke}%
\pgfsetdash{}{0pt}%
\pgfpathmoveto{\pgfqpoint{2.518786in}{0.684333in}}%
\pgfpathcurveto{\pgfqpoint{2.529836in}{0.684333in}}{\pgfqpoint{2.540435in}{0.688724in}}{\pgfqpoint{2.548249in}{0.696537in}}%
\pgfpathcurveto{\pgfqpoint{2.556062in}{0.704351in}}{\pgfqpoint{2.560452in}{0.714950in}}{\pgfqpoint{2.560452in}{0.726000in}}%
\pgfpathcurveto{\pgfqpoint{2.560452in}{0.737050in}}{\pgfqpoint{2.556062in}{0.747649in}}{\pgfqpoint{2.548249in}{0.755463in}}%
\pgfpathcurveto{\pgfqpoint{2.540435in}{0.763276in}}{\pgfqpoint{2.529836in}{0.767667in}}{\pgfqpoint{2.518786in}{0.767667in}}%
\pgfpathcurveto{\pgfqpoint{2.507736in}{0.767667in}}{\pgfqpoint{2.497137in}{0.763276in}}{\pgfqpoint{2.489323in}{0.755463in}}%
\pgfpathcurveto{\pgfqpoint{2.481509in}{0.747649in}}{\pgfqpoint{2.477119in}{0.737050in}}{\pgfqpoint{2.477119in}{0.726000in}}%
\pgfpathcurveto{\pgfqpoint{2.477119in}{0.714950in}}{\pgfqpoint{2.481509in}{0.704351in}}{\pgfqpoint{2.489323in}{0.696537in}}%
\pgfpathcurveto{\pgfqpoint{2.497137in}{0.688724in}}{\pgfqpoint{2.507736in}{0.684333in}}{\pgfqpoint{2.518786in}{0.684333in}}%
\pgfpathclose%
\pgfusepath{stroke,fill}%
\end{pgfscope}%
\begin{pgfscope}%
\pgfpathrectangle{\pgfqpoint{0.800000in}{0.528000in}}{\pgfqpoint{4.960000in}{3.696000in}}%
\pgfusepath{clip}%
\pgfsetbuttcap%
\pgfsetroundjoin%
\definecolor{currentfill}{rgb}{0.000000,0.000000,0.000000}%
\pgfsetfillcolor{currentfill}%
\pgfsetlinewidth{1.003750pt}%
\definecolor{currentstroke}{rgb}{0.000000,0.000000,0.000000}%
\pgfsetstrokecolor{currentstroke}%
\pgfsetdash{}{0pt}%
\pgfpathmoveto{\pgfqpoint{2.518786in}{0.684333in}}%
\pgfpathcurveto{\pgfqpoint{2.529836in}{0.684333in}}{\pgfqpoint{2.540435in}{0.688724in}}{\pgfqpoint{2.548249in}{0.696537in}}%
\pgfpathcurveto{\pgfqpoint{2.556062in}{0.704351in}}{\pgfqpoint{2.560452in}{0.714950in}}{\pgfqpoint{2.560452in}{0.726000in}}%
\pgfpathcurveto{\pgfqpoint{2.560452in}{0.737050in}}{\pgfqpoint{2.556062in}{0.747649in}}{\pgfqpoint{2.548249in}{0.755463in}}%
\pgfpathcurveto{\pgfqpoint{2.540435in}{0.763276in}}{\pgfqpoint{2.529836in}{0.767667in}}{\pgfqpoint{2.518786in}{0.767667in}}%
\pgfpathcurveto{\pgfqpoint{2.507736in}{0.767667in}}{\pgfqpoint{2.497137in}{0.763276in}}{\pgfqpoint{2.489323in}{0.755463in}}%
\pgfpathcurveto{\pgfqpoint{2.481509in}{0.747649in}}{\pgfqpoint{2.477119in}{0.737050in}}{\pgfqpoint{2.477119in}{0.726000in}}%
\pgfpathcurveto{\pgfqpoint{2.477119in}{0.714950in}}{\pgfqpoint{2.481509in}{0.704351in}}{\pgfqpoint{2.489323in}{0.696537in}}%
\pgfpathcurveto{\pgfqpoint{2.497137in}{0.688724in}}{\pgfqpoint{2.507736in}{0.684333in}}{\pgfqpoint{2.518786in}{0.684333in}}%
\pgfpathclose%
\pgfusepath{stroke,fill}%
\end{pgfscope}%
\begin{pgfscope}%
\pgfpathrectangle{\pgfqpoint{0.800000in}{0.528000in}}{\pgfqpoint{4.960000in}{3.696000in}}%
\pgfusepath{clip}%
\pgfsetbuttcap%
\pgfsetroundjoin%
\definecolor{currentfill}{rgb}{0.000000,0.000000,0.000000}%
\pgfsetfillcolor{currentfill}%
\pgfsetlinewidth{1.003750pt}%
\definecolor{currentstroke}{rgb}{0.000000,0.000000,0.000000}%
\pgfsetstrokecolor{currentstroke}%
\pgfsetdash{}{0pt}%
\pgfpathmoveto{\pgfqpoint{2.518786in}{0.684333in}}%
\pgfpathcurveto{\pgfqpoint{2.529836in}{0.684333in}}{\pgfqpoint{2.540435in}{0.688724in}}{\pgfqpoint{2.548249in}{0.696537in}}%
\pgfpathcurveto{\pgfqpoint{2.556062in}{0.704351in}}{\pgfqpoint{2.560452in}{0.714950in}}{\pgfqpoint{2.560452in}{0.726000in}}%
\pgfpathcurveto{\pgfqpoint{2.560452in}{0.737050in}}{\pgfqpoint{2.556062in}{0.747649in}}{\pgfqpoint{2.548249in}{0.755463in}}%
\pgfpathcurveto{\pgfqpoint{2.540435in}{0.763276in}}{\pgfqpoint{2.529836in}{0.767667in}}{\pgfqpoint{2.518786in}{0.767667in}}%
\pgfpathcurveto{\pgfqpoint{2.507736in}{0.767667in}}{\pgfqpoint{2.497137in}{0.763276in}}{\pgfqpoint{2.489323in}{0.755463in}}%
\pgfpathcurveto{\pgfqpoint{2.481509in}{0.747649in}}{\pgfqpoint{2.477119in}{0.737050in}}{\pgfqpoint{2.477119in}{0.726000in}}%
\pgfpathcurveto{\pgfqpoint{2.477119in}{0.714950in}}{\pgfqpoint{2.481509in}{0.704351in}}{\pgfqpoint{2.489323in}{0.696537in}}%
\pgfpathcurveto{\pgfqpoint{2.497137in}{0.688724in}}{\pgfqpoint{2.507736in}{0.684333in}}{\pgfqpoint{2.518786in}{0.684333in}}%
\pgfpathclose%
\pgfusepath{stroke,fill}%
\end{pgfscope}%
\begin{pgfscope}%
\pgfpathrectangle{\pgfqpoint{0.800000in}{0.528000in}}{\pgfqpoint{4.960000in}{3.696000in}}%
\pgfusepath{clip}%
\pgfsetbuttcap%
\pgfsetroundjoin%
\definecolor{currentfill}{rgb}{0.000000,0.000000,0.000000}%
\pgfsetfillcolor{currentfill}%
\pgfsetlinewidth{1.003750pt}%
\definecolor{currentstroke}{rgb}{0.000000,0.000000,0.000000}%
\pgfsetstrokecolor{currentstroke}%
\pgfsetdash{}{0pt}%
\pgfpathmoveto{\pgfqpoint{2.518786in}{0.684333in}}%
\pgfpathcurveto{\pgfqpoint{2.529836in}{0.684333in}}{\pgfqpoint{2.540435in}{0.688724in}}{\pgfqpoint{2.548249in}{0.696537in}}%
\pgfpathcurveto{\pgfqpoint{2.556062in}{0.704351in}}{\pgfqpoint{2.560452in}{0.714950in}}{\pgfqpoint{2.560452in}{0.726000in}}%
\pgfpathcurveto{\pgfqpoint{2.560452in}{0.737050in}}{\pgfqpoint{2.556062in}{0.747649in}}{\pgfqpoint{2.548249in}{0.755463in}}%
\pgfpathcurveto{\pgfqpoint{2.540435in}{0.763276in}}{\pgfqpoint{2.529836in}{0.767667in}}{\pgfqpoint{2.518786in}{0.767667in}}%
\pgfpathcurveto{\pgfqpoint{2.507736in}{0.767667in}}{\pgfqpoint{2.497137in}{0.763276in}}{\pgfqpoint{2.489323in}{0.755463in}}%
\pgfpathcurveto{\pgfqpoint{2.481509in}{0.747649in}}{\pgfqpoint{2.477119in}{0.737050in}}{\pgfqpoint{2.477119in}{0.726000in}}%
\pgfpathcurveto{\pgfqpoint{2.477119in}{0.714950in}}{\pgfqpoint{2.481509in}{0.704351in}}{\pgfqpoint{2.489323in}{0.696537in}}%
\pgfpathcurveto{\pgfqpoint{2.497137in}{0.688724in}}{\pgfqpoint{2.507736in}{0.684333in}}{\pgfqpoint{2.518786in}{0.684333in}}%
\pgfpathclose%
\pgfusepath{stroke,fill}%
\end{pgfscope}%
\begin{pgfscope}%
\pgfpathrectangle{\pgfqpoint{0.800000in}{0.528000in}}{\pgfqpoint{4.960000in}{3.696000in}}%
\pgfusepath{clip}%
\pgfsetbuttcap%
\pgfsetroundjoin%
\definecolor{currentfill}{rgb}{0.000000,0.000000,0.000000}%
\pgfsetfillcolor{currentfill}%
\pgfsetlinewidth{1.003750pt}%
\definecolor{currentstroke}{rgb}{0.000000,0.000000,0.000000}%
\pgfsetstrokecolor{currentstroke}%
\pgfsetdash{}{0pt}%
\pgfpathmoveto{\pgfqpoint{4.011666in}{0.684333in}}%
\pgfpathcurveto{\pgfqpoint{4.022716in}{0.684333in}}{\pgfqpoint{4.033315in}{0.688724in}}{\pgfqpoint{4.041128in}{0.696537in}}%
\pgfpathcurveto{\pgfqpoint{4.048942in}{0.704351in}}{\pgfqpoint{4.053332in}{0.714950in}}{\pgfqpoint{4.053332in}{0.726000in}}%
\pgfpathcurveto{\pgfqpoint{4.053332in}{0.737050in}}{\pgfqpoint{4.048942in}{0.747649in}}{\pgfqpoint{4.041128in}{0.755463in}}%
\pgfpathcurveto{\pgfqpoint{4.033315in}{0.763276in}}{\pgfqpoint{4.022716in}{0.767667in}}{\pgfqpoint{4.011666in}{0.767667in}}%
\pgfpathcurveto{\pgfqpoint{4.000616in}{0.767667in}}{\pgfqpoint{3.990016in}{0.763276in}}{\pgfqpoint{3.982203in}{0.755463in}}%
\pgfpathcurveto{\pgfqpoint{3.974389in}{0.747649in}}{\pgfqpoint{3.969999in}{0.737050in}}{\pgfqpoint{3.969999in}{0.726000in}}%
\pgfpathcurveto{\pgfqpoint{3.969999in}{0.714950in}}{\pgfqpoint{3.974389in}{0.704351in}}{\pgfqpoint{3.982203in}{0.696537in}}%
\pgfpathcurveto{\pgfqpoint{3.990016in}{0.688724in}}{\pgfqpoint{4.000616in}{0.684333in}}{\pgfqpoint{4.011666in}{0.684333in}}%
\pgfpathclose%
\pgfusepath{stroke,fill}%
\end{pgfscope}%
\begin{pgfscope}%
\pgfpathrectangle{\pgfqpoint{0.800000in}{0.528000in}}{\pgfqpoint{4.960000in}{3.696000in}}%
\pgfusepath{clip}%
\pgfsetbuttcap%
\pgfsetroundjoin%
\definecolor{currentfill}{rgb}{0.000000,0.000000,0.000000}%
\pgfsetfillcolor{currentfill}%
\pgfsetlinewidth{1.003750pt}%
\definecolor{currentstroke}{rgb}{0.000000,0.000000,0.000000}%
\pgfsetstrokecolor{currentstroke}%
\pgfsetdash{}{0pt}%
\pgfpathmoveto{\pgfqpoint{4.011666in}{0.684333in}}%
\pgfpathcurveto{\pgfqpoint{4.022716in}{0.684333in}}{\pgfqpoint{4.033315in}{0.688724in}}{\pgfqpoint{4.041128in}{0.696537in}}%
\pgfpathcurveto{\pgfqpoint{4.048942in}{0.704351in}}{\pgfqpoint{4.053332in}{0.714950in}}{\pgfqpoint{4.053332in}{0.726000in}}%
\pgfpathcurveto{\pgfqpoint{4.053332in}{0.737050in}}{\pgfqpoint{4.048942in}{0.747649in}}{\pgfqpoint{4.041128in}{0.755463in}}%
\pgfpathcurveto{\pgfqpoint{4.033315in}{0.763276in}}{\pgfqpoint{4.022716in}{0.767667in}}{\pgfqpoint{4.011666in}{0.767667in}}%
\pgfpathcurveto{\pgfqpoint{4.000616in}{0.767667in}}{\pgfqpoint{3.990016in}{0.763276in}}{\pgfqpoint{3.982203in}{0.755463in}}%
\pgfpathcurveto{\pgfqpoint{3.974389in}{0.747649in}}{\pgfqpoint{3.969999in}{0.737050in}}{\pgfqpoint{3.969999in}{0.726000in}}%
\pgfpathcurveto{\pgfqpoint{3.969999in}{0.714950in}}{\pgfqpoint{3.974389in}{0.704351in}}{\pgfqpoint{3.982203in}{0.696537in}}%
\pgfpathcurveto{\pgfqpoint{3.990016in}{0.688724in}}{\pgfqpoint{4.000616in}{0.684333in}}{\pgfqpoint{4.011666in}{0.684333in}}%
\pgfpathclose%
\pgfusepath{stroke,fill}%
\end{pgfscope}%
\begin{pgfscope}%
\pgfpathrectangle{\pgfqpoint{0.800000in}{0.528000in}}{\pgfqpoint{4.960000in}{3.696000in}}%
\pgfusepath{clip}%
\pgfsetbuttcap%
\pgfsetroundjoin%
\definecolor{currentfill}{rgb}{0.000000,0.000000,0.000000}%
\pgfsetfillcolor{currentfill}%
\pgfsetlinewidth{1.003750pt}%
\definecolor{currentstroke}{rgb}{0.000000,0.000000,0.000000}%
\pgfsetstrokecolor{currentstroke}%
\pgfsetdash{}{0pt}%
\pgfpathmoveto{\pgfqpoint{4.011666in}{0.684333in}}%
\pgfpathcurveto{\pgfqpoint{4.022716in}{0.684333in}}{\pgfqpoint{4.033315in}{0.688724in}}{\pgfqpoint{4.041128in}{0.696537in}}%
\pgfpathcurveto{\pgfqpoint{4.048942in}{0.704351in}}{\pgfqpoint{4.053332in}{0.714950in}}{\pgfqpoint{4.053332in}{0.726000in}}%
\pgfpathcurveto{\pgfqpoint{4.053332in}{0.737050in}}{\pgfqpoint{4.048942in}{0.747649in}}{\pgfqpoint{4.041128in}{0.755463in}}%
\pgfpathcurveto{\pgfqpoint{4.033315in}{0.763276in}}{\pgfqpoint{4.022716in}{0.767667in}}{\pgfqpoint{4.011666in}{0.767667in}}%
\pgfpathcurveto{\pgfqpoint{4.000616in}{0.767667in}}{\pgfqpoint{3.990016in}{0.763276in}}{\pgfqpoint{3.982203in}{0.755463in}}%
\pgfpathcurveto{\pgfqpoint{3.974389in}{0.747649in}}{\pgfqpoint{3.969999in}{0.737050in}}{\pgfqpoint{3.969999in}{0.726000in}}%
\pgfpathcurveto{\pgfqpoint{3.969999in}{0.714950in}}{\pgfqpoint{3.974389in}{0.704351in}}{\pgfqpoint{3.982203in}{0.696537in}}%
\pgfpathcurveto{\pgfqpoint{3.990016in}{0.688724in}}{\pgfqpoint{4.000616in}{0.684333in}}{\pgfqpoint{4.011666in}{0.684333in}}%
\pgfpathclose%
\pgfusepath{stroke,fill}%
\end{pgfscope}%
\begin{pgfscope}%
\pgfpathrectangle{\pgfqpoint{0.800000in}{0.528000in}}{\pgfqpoint{4.960000in}{3.696000in}}%
\pgfusepath{clip}%
\pgfsetbuttcap%
\pgfsetroundjoin%
\definecolor{currentfill}{rgb}{0.000000,0.000000,0.000000}%
\pgfsetfillcolor{currentfill}%
\pgfsetlinewidth{1.003750pt}%
\definecolor{currentstroke}{rgb}{0.000000,0.000000,0.000000}%
\pgfsetstrokecolor{currentstroke}%
\pgfsetdash{}{0pt}%
\pgfpathmoveto{\pgfqpoint{4.011666in}{0.684333in}}%
\pgfpathcurveto{\pgfqpoint{4.022716in}{0.684333in}}{\pgfqpoint{4.033315in}{0.688724in}}{\pgfqpoint{4.041128in}{0.696537in}}%
\pgfpathcurveto{\pgfqpoint{4.048942in}{0.704351in}}{\pgfqpoint{4.053332in}{0.714950in}}{\pgfqpoint{4.053332in}{0.726000in}}%
\pgfpathcurveto{\pgfqpoint{4.053332in}{0.737050in}}{\pgfqpoint{4.048942in}{0.747649in}}{\pgfqpoint{4.041128in}{0.755463in}}%
\pgfpathcurveto{\pgfqpoint{4.033315in}{0.763276in}}{\pgfqpoint{4.022716in}{0.767667in}}{\pgfqpoint{4.011666in}{0.767667in}}%
\pgfpathcurveto{\pgfqpoint{4.000616in}{0.767667in}}{\pgfqpoint{3.990016in}{0.763276in}}{\pgfqpoint{3.982203in}{0.755463in}}%
\pgfpathcurveto{\pgfqpoint{3.974389in}{0.747649in}}{\pgfqpoint{3.969999in}{0.737050in}}{\pgfqpoint{3.969999in}{0.726000in}}%
\pgfpathcurveto{\pgfqpoint{3.969999in}{0.714950in}}{\pgfqpoint{3.974389in}{0.704351in}}{\pgfqpoint{3.982203in}{0.696537in}}%
\pgfpathcurveto{\pgfqpoint{3.990016in}{0.688724in}}{\pgfqpoint{4.000616in}{0.684333in}}{\pgfqpoint{4.011666in}{0.684333in}}%
\pgfpathclose%
\pgfusepath{stroke,fill}%
\end{pgfscope}%
\begin{pgfscope}%
\pgfpathrectangle{\pgfqpoint{0.800000in}{0.528000in}}{\pgfqpoint{4.960000in}{3.696000in}}%
\pgfusepath{clip}%
\pgfsetbuttcap%
\pgfsetroundjoin%
\definecolor{currentfill}{rgb}{0.000000,0.000000,0.000000}%
\pgfsetfillcolor{currentfill}%
\pgfsetlinewidth{1.003750pt}%
\definecolor{currentstroke}{rgb}{0.000000,0.000000,0.000000}%
\pgfsetstrokecolor{currentstroke}%
\pgfsetdash{}{0pt}%
\pgfpathmoveto{\pgfqpoint{4.011666in}{0.684333in}}%
\pgfpathcurveto{\pgfqpoint{4.022716in}{0.684333in}}{\pgfqpoint{4.033315in}{0.688724in}}{\pgfqpoint{4.041128in}{0.696537in}}%
\pgfpathcurveto{\pgfqpoint{4.048942in}{0.704351in}}{\pgfqpoint{4.053332in}{0.714950in}}{\pgfqpoint{4.053332in}{0.726000in}}%
\pgfpathcurveto{\pgfqpoint{4.053332in}{0.737050in}}{\pgfqpoint{4.048942in}{0.747649in}}{\pgfqpoint{4.041128in}{0.755463in}}%
\pgfpathcurveto{\pgfqpoint{4.033315in}{0.763276in}}{\pgfqpoint{4.022716in}{0.767667in}}{\pgfqpoint{4.011666in}{0.767667in}}%
\pgfpathcurveto{\pgfqpoint{4.000616in}{0.767667in}}{\pgfqpoint{3.990016in}{0.763276in}}{\pgfqpoint{3.982203in}{0.755463in}}%
\pgfpathcurveto{\pgfqpoint{3.974389in}{0.747649in}}{\pgfqpoint{3.969999in}{0.737050in}}{\pgfqpoint{3.969999in}{0.726000in}}%
\pgfpathcurveto{\pgfqpoint{3.969999in}{0.714950in}}{\pgfqpoint{3.974389in}{0.704351in}}{\pgfqpoint{3.982203in}{0.696537in}}%
\pgfpathcurveto{\pgfqpoint{3.990016in}{0.688724in}}{\pgfqpoint{4.000616in}{0.684333in}}{\pgfqpoint{4.011666in}{0.684333in}}%
\pgfpathclose%
\pgfusepath{stroke,fill}%
\end{pgfscope}%
\begin{pgfscope}%
\pgfpathrectangle{\pgfqpoint{0.800000in}{0.528000in}}{\pgfqpoint{4.960000in}{3.696000in}}%
\pgfusepath{clip}%
\pgfsetbuttcap%
\pgfsetroundjoin%
\definecolor{currentfill}{rgb}{0.000000,0.000000,0.000000}%
\pgfsetfillcolor{currentfill}%
\pgfsetlinewidth{1.003750pt}%
\definecolor{currentstroke}{rgb}{0.000000,0.000000,0.000000}%
\pgfsetstrokecolor{currentstroke}%
\pgfsetdash{}{0pt}%
\pgfpathmoveto{\pgfqpoint{4.011666in}{0.684333in}}%
\pgfpathcurveto{\pgfqpoint{4.022716in}{0.684333in}}{\pgfqpoint{4.033315in}{0.688724in}}{\pgfqpoint{4.041128in}{0.696537in}}%
\pgfpathcurveto{\pgfqpoint{4.048942in}{0.704351in}}{\pgfqpoint{4.053332in}{0.714950in}}{\pgfqpoint{4.053332in}{0.726000in}}%
\pgfpathcurveto{\pgfqpoint{4.053332in}{0.737050in}}{\pgfqpoint{4.048942in}{0.747649in}}{\pgfqpoint{4.041128in}{0.755463in}}%
\pgfpathcurveto{\pgfqpoint{4.033315in}{0.763276in}}{\pgfqpoint{4.022716in}{0.767667in}}{\pgfqpoint{4.011666in}{0.767667in}}%
\pgfpathcurveto{\pgfqpoint{4.000616in}{0.767667in}}{\pgfqpoint{3.990016in}{0.763276in}}{\pgfqpoint{3.982203in}{0.755463in}}%
\pgfpathcurveto{\pgfqpoint{3.974389in}{0.747649in}}{\pgfqpoint{3.969999in}{0.737050in}}{\pgfqpoint{3.969999in}{0.726000in}}%
\pgfpathcurveto{\pgfqpoint{3.969999in}{0.714950in}}{\pgfqpoint{3.974389in}{0.704351in}}{\pgfqpoint{3.982203in}{0.696537in}}%
\pgfpathcurveto{\pgfqpoint{3.990016in}{0.688724in}}{\pgfqpoint{4.000616in}{0.684333in}}{\pgfqpoint{4.011666in}{0.684333in}}%
\pgfpathclose%
\pgfusepath{stroke,fill}%
\end{pgfscope}%
\begin{pgfscope}%
\pgfpathrectangle{\pgfqpoint{0.800000in}{0.528000in}}{\pgfqpoint{4.960000in}{3.696000in}}%
\pgfusepath{clip}%
\pgfsetbuttcap%
\pgfsetroundjoin%
\definecolor{currentfill}{rgb}{0.000000,0.000000,0.000000}%
\pgfsetfillcolor{currentfill}%
\pgfsetlinewidth{1.003750pt}%
\definecolor{currentstroke}{rgb}{0.000000,0.000000,0.000000}%
\pgfsetstrokecolor{currentstroke}%
\pgfsetdash{}{0pt}%
\pgfpathmoveto{\pgfqpoint{4.011666in}{0.684333in}}%
\pgfpathcurveto{\pgfqpoint{4.022716in}{0.684333in}}{\pgfqpoint{4.033315in}{0.688724in}}{\pgfqpoint{4.041128in}{0.696537in}}%
\pgfpathcurveto{\pgfqpoint{4.048942in}{0.704351in}}{\pgfqpoint{4.053332in}{0.714950in}}{\pgfqpoint{4.053332in}{0.726000in}}%
\pgfpathcurveto{\pgfqpoint{4.053332in}{0.737050in}}{\pgfqpoint{4.048942in}{0.747649in}}{\pgfqpoint{4.041128in}{0.755463in}}%
\pgfpathcurveto{\pgfqpoint{4.033315in}{0.763276in}}{\pgfqpoint{4.022716in}{0.767667in}}{\pgfqpoint{4.011666in}{0.767667in}}%
\pgfpathcurveto{\pgfqpoint{4.000616in}{0.767667in}}{\pgfqpoint{3.990016in}{0.763276in}}{\pgfqpoint{3.982203in}{0.755463in}}%
\pgfpathcurveto{\pgfqpoint{3.974389in}{0.747649in}}{\pgfqpoint{3.969999in}{0.737050in}}{\pgfqpoint{3.969999in}{0.726000in}}%
\pgfpathcurveto{\pgfqpoint{3.969999in}{0.714950in}}{\pgfqpoint{3.974389in}{0.704351in}}{\pgfqpoint{3.982203in}{0.696537in}}%
\pgfpathcurveto{\pgfqpoint{3.990016in}{0.688724in}}{\pgfqpoint{4.000616in}{0.684333in}}{\pgfqpoint{4.011666in}{0.684333in}}%
\pgfpathclose%
\pgfusepath{stroke,fill}%
\end{pgfscope}%
\begin{pgfscope}%
\pgfpathrectangle{\pgfqpoint{0.800000in}{0.528000in}}{\pgfqpoint{4.960000in}{3.696000in}}%
\pgfusepath{clip}%
\pgfsetbuttcap%
\pgfsetroundjoin%
\definecolor{currentfill}{rgb}{0.000000,0.000000,0.000000}%
\pgfsetfillcolor{currentfill}%
\pgfsetlinewidth{1.003750pt}%
\definecolor{currentstroke}{rgb}{0.000000,0.000000,0.000000}%
\pgfsetstrokecolor{currentstroke}%
\pgfsetdash{}{0pt}%
\pgfpathmoveto{\pgfqpoint{4.011666in}{0.684333in}}%
\pgfpathcurveto{\pgfqpoint{4.022716in}{0.684333in}}{\pgfqpoint{4.033315in}{0.688724in}}{\pgfqpoint{4.041128in}{0.696537in}}%
\pgfpathcurveto{\pgfqpoint{4.048942in}{0.704351in}}{\pgfqpoint{4.053332in}{0.714950in}}{\pgfqpoint{4.053332in}{0.726000in}}%
\pgfpathcurveto{\pgfqpoint{4.053332in}{0.737050in}}{\pgfqpoint{4.048942in}{0.747649in}}{\pgfqpoint{4.041128in}{0.755463in}}%
\pgfpathcurveto{\pgfqpoint{4.033315in}{0.763276in}}{\pgfqpoint{4.022716in}{0.767667in}}{\pgfqpoint{4.011666in}{0.767667in}}%
\pgfpathcurveto{\pgfqpoint{4.000616in}{0.767667in}}{\pgfqpoint{3.990016in}{0.763276in}}{\pgfqpoint{3.982203in}{0.755463in}}%
\pgfpathcurveto{\pgfqpoint{3.974389in}{0.747649in}}{\pgfqpoint{3.969999in}{0.737050in}}{\pgfqpoint{3.969999in}{0.726000in}}%
\pgfpathcurveto{\pgfqpoint{3.969999in}{0.714950in}}{\pgfqpoint{3.974389in}{0.704351in}}{\pgfqpoint{3.982203in}{0.696537in}}%
\pgfpathcurveto{\pgfqpoint{3.990016in}{0.688724in}}{\pgfqpoint{4.000616in}{0.684333in}}{\pgfqpoint{4.011666in}{0.684333in}}%
\pgfpathclose%
\pgfusepath{stroke,fill}%
\end{pgfscope}%
\begin{pgfscope}%
\pgfpathrectangle{\pgfqpoint{0.800000in}{0.528000in}}{\pgfqpoint{4.960000in}{3.696000in}}%
\pgfusepath{clip}%
\pgfsetbuttcap%
\pgfsetroundjoin%
\definecolor{currentfill}{rgb}{0.000000,0.000000,0.000000}%
\pgfsetfillcolor{currentfill}%
\pgfsetlinewidth{1.003750pt}%
\definecolor{currentstroke}{rgb}{0.000000,0.000000,0.000000}%
\pgfsetstrokecolor{currentstroke}%
\pgfsetdash{}{0pt}%
\pgfpathmoveto{\pgfqpoint{4.011666in}{0.684333in}}%
\pgfpathcurveto{\pgfqpoint{4.022716in}{0.684333in}}{\pgfqpoint{4.033315in}{0.688724in}}{\pgfqpoint{4.041128in}{0.696537in}}%
\pgfpathcurveto{\pgfqpoint{4.048942in}{0.704351in}}{\pgfqpoint{4.053332in}{0.714950in}}{\pgfqpoint{4.053332in}{0.726000in}}%
\pgfpathcurveto{\pgfqpoint{4.053332in}{0.737050in}}{\pgfqpoint{4.048942in}{0.747649in}}{\pgfqpoint{4.041128in}{0.755463in}}%
\pgfpathcurveto{\pgfqpoint{4.033315in}{0.763276in}}{\pgfqpoint{4.022716in}{0.767667in}}{\pgfqpoint{4.011666in}{0.767667in}}%
\pgfpathcurveto{\pgfqpoint{4.000616in}{0.767667in}}{\pgfqpoint{3.990016in}{0.763276in}}{\pgfqpoint{3.982203in}{0.755463in}}%
\pgfpathcurveto{\pgfqpoint{3.974389in}{0.747649in}}{\pgfqpoint{3.969999in}{0.737050in}}{\pgfqpoint{3.969999in}{0.726000in}}%
\pgfpathcurveto{\pgfqpoint{3.969999in}{0.714950in}}{\pgfqpoint{3.974389in}{0.704351in}}{\pgfqpoint{3.982203in}{0.696537in}}%
\pgfpathcurveto{\pgfqpoint{3.990016in}{0.688724in}}{\pgfqpoint{4.000616in}{0.684333in}}{\pgfqpoint{4.011666in}{0.684333in}}%
\pgfpathclose%
\pgfusepath{stroke,fill}%
\end{pgfscope}%
\begin{pgfscope}%
\pgfpathrectangle{\pgfqpoint{0.800000in}{0.528000in}}{\pgfqpoint{4.960000in}{3.696000in}}%
\pgfusepath{clip}%
\pgfsetbuttcap%
\pgfsetroundjoin%
\definecolor{currentfill}{rgb}{0.000000,0.000000,0.000000}%
\pgfsetfillcolor{currentfill}%
\pgfsetlinewidth{1.003750pt}%
\definecolor{currentstroke}{rgb}{0.000000,0.000000,0.000000}%
\pgfsetstrokecolor{currentstroke}%
\pgfsetdash{}{0pt}%
\pgfpathmoveto{\pgfqpoint{4.011666in}{0.684333in}}%
\pgfpathcurveto{\pgfqpoint{4.022716in}{0.684333in}}{\pgfqpoint{4.033315in}{0.688724in}}{\pgfqpoint{4.041128in}{0.696537in}}%
\pgfpathcurveto{\pgfqpoint{4.048942in}{0.704351in}}{\pgfqpoint{4.053332in}{0.714950in}}{\pgfqpoint{4.053332in}{0.726000in}}%
\pgfpathcurveto{\pgfqpoint{4.053332in}{0.737050in}}{\pgfqpoint{4.048942in}{0.747649in}}{\pgfqpoint{4.041128in}{0.755463in}}%
\pgfpathcurveto{\pgfqpoint{4.033315in}{0.763276in}}{\pgfqpoint{4.022716in}{0.767667in}}{\pgfqpoint{4.011666in}{0.767667in}}%
\pgfpathcurveto{\pgfqpoint{4.000616in}{0.767667in}}{\pgfqpoint{3.990016in}{0.763276in}}{\pgfqpoint{3.982203in}{0.755463in}}%
\pgfpathcurveto{\pgfqpoint{3.974389in}{0.747649in}}{\pgfqpoint{3.969999in}{0.737050in}}{\pgfqpoint{3.969999in}{0.726000in}}%
\pgfpathcurveto{\pgfqpoint{3.969999in}{0.714950in}}{\pgfqpoint{3.974389in}{0.704351in}}{\pgfqpoint{3.982203in}{0.696537in}}%
\pgfpathcurveto{\pgfqpoint{3.990016in}{0.688724in}}{\pgfqpoint{4.000616in}{0.684333in}}{\pgfqpoint{4.011666in}{0.684333in}}%
\pgfpathclose%
\pgfusepath{stroke,fill}%
\end{pgfscope}%
\begin{pgfscope}%
\pgfpathrectangle{\pgfqpoint{0.800000in}{0.528000in}}{\pgfqpoint{4.960000in}{3.696000in}}%
\pgfusepath{clip}%
\pgfsetbuttcap%
\pgfsetroundjoin%
\definecolor{currentfill}{rgb}{0.000000,0.000000,0.000000}%
\pgfsetfillcolor{currentfill}%
\pgfsetlinewidth{1.003750pt}%
\definecolor{currentstroke}{rgb}{0.000000,0.000000,0.000000}%
\pgfsetstrokecolor{currentstroke}%
\pgfsetdash{}{0pt}%
\pgfpathmoveto{\pgfqpoint{4.011666in}{0.684333in}}%
\pgfpathcurveto{\pgfqpoint{4.022716in}{0.684333in}}{\pgfqpoint{4.033315in}{0.688724in}}{\pgfqpoint{4.041128in}{0.696537in}}%
\pgfpathcurveto{\pgfqpoint{4.048942in}{0.704351in}}{\pgfqpoint{4.053332in}{0.714950in}}{\pgfqpoint{4.053332in}{0.726000in}}%
\pgfpathcurveto{\pgfqpoint{4.053332in}{0.737050in}}{\pgfqpoint{4.048942in}{0.747649in}}{\pgfqpoint{4.041128in}{0.755463in}}%
\pgfpathcurveto{\pgfqpoint{4.033315in}{0.763276in}}{\pgfqpoint{4.022716in}{0.767667in}}{\pgfqpoint{4.011666in}{0.767667in}}%
\pgfpathcurveto{\pgfqpoint{4.000616in}{0.767667in}}{\pgfqpoint{3.990016in}{0.763276in}}{\pgfqpoint{3.982203in}{0.755463in}}%
\pgfpathcurveto{\pgfqpoint{3.974389in}{0.747649in}}{\pgfqpoint{3.969999in}{0.737050in}}{\pgfqpoint{3.969999in}{0.726000in}}%
\pgfpathcurveto{\pgfqpoint{3.969999in}{0.714950in}}{\pgfqpoint{3.974389in}{0.704351in}}{\pgfqpoint{3.982203in}{0.696537in}}%
\pgfpathcurveto{\pgfqpoint{3.990016in}{0.688724in}}{\pgfqpoint{4.000616in}{0.684333in}}{\pgfqpoint{4.011666in}{0.684333in}}%
\pgfpathclose%
\pgfusepath{stroke,fill}%
\end{pgfscope}%
\begin{pgfscope}%
\pgfpathrectangle{\pgfqpoint{0.800000in}{0.528000in}}{\pgfqpoint{4.960000in}{3.696000in}}%
\pgfusepath{clip}%
\pgfsetbuttcap%
\pgfsetroundjoin%
\definecolor{currentfill}{rgb}{0.000000,0.000000,0.000000}%
\pgfsetfillcolor{currentfill}%
\pgfsetlinewidth{1.003750pt}%
\definecolor{currentstroke}{rgb}{0.000000,0.000000,0.000000}%
\pgfsetstrokecolor{currentstroke}%
\pgfsetdash{}{0pt}%
\pgfpathmoveto{\pgfqpoint{4.011666in}{0.684333in}}%
\pgfpathcurveto{\pgfqpoint{4.022716in}{0.684333in}}{\pgfqpoint{4.033315in}{0.688724in}}{\pgfqpoint{4.041128in}{0.696537in}}%
\pgfpathcurveto{\pgfqpoint{4.048942in}{0.704351in}}{\pgfqpoint{4.053332in}{0.714950in}}{\pgfqpoint{4.053332in}{0.726000in}}%
\pgfpathcurveto{\pgfqpoint{4.053332in}{0.737050in}}{\pgfqpoint{4.048942in}{0.747649in}}{\pgfqpoint{4.041128in}{0.755463in}}%
\pgfpathcurveto{\pgfqpoint{4.033315in}{0.763276in}}{\pgfqpoint{4.022716in}{0.767667in}}{\pgfqpoint{4.011666in}{0.767667in}}%
\pgfpathcurveto{\pgfqpoint{4.000616in}{0.767667in}}{\pgfqpoint{3.990016in}{0.763276in}}{\pgfqpoint{3.982203in}{0.755463in}}%
\pgfpathcurveto{\pgfqpoint{3.974389in}{0.747649in}}{\pgfqpoint{3.969999in}{0.737050in}}{\pgfqpoint{3.969999in}{0.726000in}}%
\pgfpathcurveto{\pgfqpoint{3.969999in}{0.714950in}}{\pgfqpoint{3.974389in}{0.704351in}}{\pgfqpoint{3.982203in}{0.696537in}}%
\pgfpathcurveto{\pgfqpoint{3.990016in}{0.688724in}}{\pgfqpoint{4.000616in}{0.684333in}}{\pgfqpoint{4.011666in}{0.684333in}}%
\pgfpathclose%
\pgfusepath{stroke,fill}%
\end{pgfscope}%
\begin{pgfscope}%
\pgfpathrectangle{\pgfqpoint{0.800000in}{0.528000in}}{\pgfqpoint{4.960000in}{3.696000in}}%
\pgfusepath{clip}%
\pgfsetbuttcap%
\pgfsetroundjoin%
\definecolor{currentfill}{rgb}{0.000000,0.000000,0.000000}%
\pgfsetfillcolor{currentfill}%
\pgfsetlinewidth{1.003750pt}%
\definecolor{currentstroke}{rgb}{0.000000,0.000000,0.000000}%
\pgfsetstrokecolor{currentstroke}%
\pgfsetdash{}{0pt}%
\pgfpathmoveto{\pgfqpoint{4.011666in}{0.684333in}}%
\pgfpathcurveto{\pgfqpoint{4.022716in}{0.684333in}}{\pgfqpoint{4.033315in}{0.688724in}}{\pgfqpoint{4.041128in}{0.696537in}}%
\pgfpathcurveto{\pgfqpoint{4.048942in}{0.704351in}}{\pgfqpoint{4.053332in}{0.714950in}}{\pgfqpoint{4.053332in}{0.726000in}}%
\pgfpathcurveto{\pgfqpoint{4.053332in}{0.737050in}}{\pgfqpoint{4.048942in}{0.747649in}}{\pgfqpoint{4.041128in}{0.755463in}}%
\pgfpathcurveto{\pgfqpoint{4.033315in}{0.763276in}}{\pgfqpoint{4.022716in}{0.767667in}}{\pgfqpoint{4.011666in}{0.767667in}}%
\pgfpathcurveto{\pgfqpoint{4.000616in}{0.767667in}}{\pgfqpoint{3.990016in}{0.763276in}}{\pgfqpoint{3.982203in}{0.755463in}}%
\pgfpathcurveto{\pgfqpoint{3.974389in}{0.747649in}}{\pgfqpoint{3.969999in}{0.737050in}}{\pgfqpoint{3.969999in}{0.726000in}}%
\pgfpathcurveto{\pgfqpoint{3.969999in}{0.714950in}}{\pgfqpoint{3.974389in}{0.704351in}}{\pgfqpoint{3.982203in}{0.696537in}}%
\pgfpathcurveto{\pgfqpoint{3.990016in}{0.688724in}}{\pgfqpoint{4.000616in}{0.684333in}}{\pgfqpoint{4.011666in}{0.684333in}}%
\pgfpathclose%
\pgfusepath{stroke,fill}%
\end{pgfscope}%
\begin{pgfscope}%
\pgfpathrectangle{\pgfqpoint{0.800000in}{0.528000in}}{\pgfqpoint{4.960000in}{3.696000in}}%
\pgfusepath{clip}%
\pgfsetbuttcap%
\pgfsetroundjoin%
\definecolor{currentfill}{rgb}{0.000000,0.000000,0.000000}%
\pgfsetfillcolor{currentfill}%
\pgfsetlinewidth{1.003750pt}%
\definecolor{currentstroke}{rgb}{0.000000,0.000000,0.000000}%
\pgfsetstrokecolor{currentstroke}%
\pgfsetdash{}{0pt}%
\pgfpathmoveto{\pgfqpoint{4.011666in}{0.684333in}}%
\pgfpathcurveto{\pgfqpoint{4.022716in}{0.684333in}}{\pgfqpoint{4.033315in}{0.688724in}}{\pgfqpoint{4.041128in}{0.696537in}}%
\pgfpathcurveto{\pgfqpoint{4.048942in}{0.704351in}}{\pgfqpoint{4.053332in}{0.714950in}}{\pgfqpoint{4.053332in}{0.726000in}}%
\pgfpathcurveto{\pgfqpoint{4.053332in}{0.737050in}}{\pgfqpoint{4.048942in}{0.747649in}}{\pgfqpoint{4.041128in}{0.755463in}}%
\pgfpathcurveto{\pgfqpoint{4.033315in}{0.763276in}}{\pgfqpoint{4.022716in}{0.767667in}}{\pgfqpoint{4.011666in}{0.767667in}}%
\pgfpathcurveto{\pgfqpoint{4.000616in}{0.767667in}}{\pgfqpoint{3.990016in}{0.763276in}}{\pgfqpoint{3.982203in}{0.755463in}}%
\pgfpathcurveto{\pgfqpoint{3.974389in}{0.747649in}}{\pgfqpoint{3.969999in}{0.737050in}}{\pgfqpoint{3.969999in}{0.726000in}}%
\pgfpathcurveto{\pgfqpoint{3.969999in}{0.714950in}}{\pgfqpoint{3.974389in}{0.704351in}}{\pgfqpoint{3.982203in}{0.696537in}}%
\pgfpathcurveto{\pgfqpoint{3.990016in}{0.688724in}}{\pgfqpoint{4.000616in}{0.684333in}}{\pgfqpoint{4.011666in}{0.684333in}}%
\pgfpathclose%
\pgfusepath{stroke,fill}%
\end{pgfscope}%
\begin{pgfscope}%
\pgfpathrectangle{\pgfqpoint{0.800000in}{0.528000in}}{\pgfqpoint{4.960000in}{3.696000in}}%
\pgfusepath{clip}%
\pgfsetbuttcap%
\pgfsetroundjoin%
\definecolor{currentfill}{rgb}{0.000000,0.000000,0.000000}%
\pgfsetfillcolor{currentfill}%
\pgfsetlinewidth{1.003750pt}%
\definecolor{currentstroke}{rgb}{0.000000,0.000000,0.000000}%
\pgfsetstrokecolor{currentstroke}%
\pgfsetdash{}{0pt}%
\pgfpathmoveto{\pgfqpoint{4.011666in}{0.684333in}}%
\pgfpathcurveto{\pgfqpoint{4.022716in}{0.684333in}}{\pgfqpoint{4.033315in}{0.688724in}}{\pgfqpoint{4.041128in}{0.696537in}}%
\pgfpathcurveto{\pgfqpoint{4.048942in}{0.704351in}}{\pgfqpoint{4.053332in}{0.714950in}}{\pgfqpoint{4.053332in}{0.726000in}}%
\pgfpathcurveto{\pgfqpoint{4.053332in}{0.737050in}}{\pgfqpoint{4.048942in}{0.747649in}}{\pgfqpoint{4.041128in}{0.755463in}}%
\pgfpathcurveto{\pgfqpoint{4.033315in}{0.763276in}}{\pgfqpoint{4.022716in}{0.767667in}}{\pgfqpoint{4.011666in}{0.767667in}}%
\pgfpathcurveto{\pgfqpoint{4.000616in}{0.767667in}}{\pgfqpoint{3.990016in}{0.763276in}}{\pgfqpoint{3.982203in}{0.755463in}}%
\pgfpathcurveto{\pgfqpoint{3.974389in}{0.747649in}}{\pgfqpoint{3.969999in}{0.737050in}}{\pgfqpoint{3.969999in}{0.726000in}}%
\pgfpathcurveto{\pgfqpoint{3.969999in}{0.714950in}}{\pgfqpoint{3.974389in}{0.704351in}}{\pgfqpoint{3.982203in}{0.696537in}}%
\pgfpathcurveto{\pgfqpoint{3.990016in}{0.688724in}}{\pgfqpoint{4.000616in}{0.684333in}}{\pgfqpoint{4.011666in}{0.684333in}}%
\pgfpathclose%
\pgfusepath{stroke,fill}%
\end{pgfscope}%
\begin{pgfscope}%
\pgfpathrectangle{\pgfqpoint{0.800000in}{0.528000in}}{\pgfqpoint{4.960000in}{3.696000in}}%
\pgfusepath{clip}%
\pgfsetbuttcap%
\pgfsetroundjoin%
\definecolor{currentfill}{rgb}{0.000000,0.000000,0.000000}%
\pgfsetfillcolor{currentfill}%
\pgfsetlinewidth{1.003750pt}%
\definecolor{currentstroke}{rgb}{0.000000,0.000000,0.000000}%
\pgfsetstrokecolor{currentstroke}%
\pgfsetdash{}{0pt}%
\pgfpathmoveto{\pgfqpoint{4.011666in}{0.684333in}}%
\pgfpathcurveto{\pgfqpoint{4.022716in}{0.684333in}}{\pgfqpoint{4.033315in}{0.688724in}}{\pgfqpoint{4.041128in}{0.696537in}}%
\pgfpathcurveto{\pgfqpoint{4.048942in}{0.704351in}}{\pgfqpoint{4.053332in}{0.714950in}}{\pgfqpoint{4.053332in}{0.726000in}}%
\pgfpathcurveto{\pgfqpoint{4.053332in}{0.737050in}}{\pgfqpoint{4.048942in}{0.747649in}}{\pgfqpoint{4.041128in}{0.755463in}}%
\pgfpathcurveto{\pgfqpoint{4.033315in}{0.763276in}}{\pgfqpoint{4.022716in}{0.767667in}}{\pgfqpoint{4.011666in}{0.767667in}}%
\pgfpathcurveto{\pgfqpoint{4.000616in}{0.767667in}}{\pgfqpoint{3.990016in}{0.763276in}}{\pgfqpoint{3.982203in}{0.755463in}}%
\pgfpathcurveto{\pgfqpoint{3.974389in}{0.747649in}}{\pgfqpoint{3.969999in}{0.737050in}}{\pgfqpoint{3.969999in}{0.726000in}}%
\pgfpathcurveto{\pgfqpoint{3.969999in}{0.714950in}}{\pgfqpoint{3.974389in}{0.704351in}}{\pgfqpoint{3.982203in}{0.696537in}}%
\pgfpathcurveto{\pgfqpoint{3.990016in}{0.688724in}}{\pgfqpoint{4.000616in}{0.684333in}}{\pgfqpoint{4.011666in}{0.684333in}}%
\pgfpathclose%
\pgfusepath{stroke,fill}%
\end{pgfscope}%
\begin{pgfscope}%
\pgfpathrectangle{\pgfqpoint{0.800000in}{0.528000in}}{\pgfqpoint{4.960000in}{3.696000in}}%
\pgfusepath{clip}%
\pgfsetbuttcap%
\pgfsetroundjoin%
\definecolor{currentfill}{rgb}{0.000000,0.000000,0.000000}%
\pgfsetfillcolor{currentfill}%
\pgfsetlinewidth{1.003750pt}%
\definecolor{currentstroke}{rgb}{0.000000,0.000000,0.000000}%
\pgfsetstrokecolor{currentstroke}%
\pgfsetdash{}{0pt}%
\pgfpathmoveto{\pgfqpoint{4.011666in}{0.684333in}}%
\pgfpathcurveto{\pgfqpoint{4.022716in}{0.684333in}}{\pgfqpoint{4.033315in}{0.688724in}}{\pgfqpoint{4.041128in}{0.696537in}}%
\pgfpathcurveto{\pgfqpoint{4.048942in}{0.704351in}}{\pgfqpoint{4.053332in}{0.714950in}}{\pgfqpoint{4.053332in}{0.726000in}}%
\pgfpathcurveto{\pgfqpoint{4.053332in}{0.737050in}}{\pgfqpoint{4.048942in}{0.747649in}}{\pgfqpoint{4.041128in}{0.755463in}}%
\pgfpathcurveto{\pgfqpoint{4.033315in}{0.763276in}}{\pgfqpoint{4.022716in}{0.767667in}}{\pgfqpoint{4.011666in}{0.767667in}}%
\pgfpathcurveto{\pgfqpoint{4.000616in}{0.767667in}}{\pgfqpoint{3.990016in}{0.763276in}}{\pgfqpoint{3.982203in}{0.755463in}}%
\pgfpathcurveto{\pgfqpoint{3.974389in}{0.747649in}}{\pgfqpoint{3.969999in}{0.737050in}}{\pgfqpoint{3.969999in}{0.726000in}}%
\pgfpathcurveto{\pgfqpoint{3.969999in}{0.714950in}}{\pgfqpoint{3.974389in}{0.704351in}}{\pgfqpoint{3.982203in}{0.696537in}}%
\pgfpathcurveto{\pgfqpoint{3.990016in}{0.688724in}}{\pgfqpoint{4.000616in}{0.684333in}}{\pgfqpoint{4.011666in}{0.684333in}}%
\pgfpathclose%
\pgfusepath{stroke,fill}%
\end{pgfscope}%
\begin{pgfscope}%
\pgfpathrectangle{\pgfqpoint{0.800000in}{0.528000in}}{\pgfqpoint{4.960000in}{3.696000in}}%
\pgfusepath{clip}%
\pgfsetbuttcap%
\pgfsetroundjoin%
\definecolor{currentfill}{rgb}{0.000000,0.000000,0.000000}%
\pgfsetfillcolor{currentfill}%
\pgfsetlinewidth{1.003750pt}%
\definecolor{currentstroke}{rgb}{0.000000,0.000000,0.000000}%
\pgfsetstrokecolor{currentstroke}%
\pgfsetdash{}{0pt}%
\pgfpathmoveto{\pgfqpoint{4.011666in}{0.684333in}}%
\pgfpathcurveto{\pgfqpoint{4.022716in}{0.684333in}}{\pgfqpoint{4.033315in}{0.688724in}}{\pgfqpoint{4.041128in}{0.696537in}}%
\pgfpathcurveto{\pgfqpoint{4.048942in}{0.704351in}}{\pgfqpoint{4.053332in}{0.714950in}}{\pgfqpoint{4.053332in}{0.726000in}}%
\pgfpathcurveto{\pgfqpoint{4.053332in}{0.737050in}}{\pgfqpoint{4.048942in}{0.747649in}}{\pgfqpoint{4.041128in}{0.755463in}}%
\pgfpathcurveto{\pgfqpoint{4.033315in}{0.763276in}}{\pgfqpoint{4.022716in}{0.767667in}}{\pgfqpoint{4.011666in}{0.767667in}}%
\pgfpathcurveto{\pgfqpoint{4.000616in}{0.767667in}}{\pgfqpoint{3.990016in}{0.763276in}}{\pgfqpoint{3.982203in}{0.755463in}}%
\pgfpathcurveto{\pgfqpoint{3.974389in}{0.747649in}}{\pgfqpoint{3.969999in}{0.737050in}}{\pgfqpoint{3.969999in}{0.726000in}}%
\pgfpathcurveto{\pgfqpoint{3.969999in}{0.714950in}}{\pgfqpoint{3.974389in}{0.704351in}}{\pgfqpoint{3.982203in}{0.696537in}}%
\pgfpathcurveto{\pgfqpoint{3.990016in}{0.688724in}}{\pgfqpoint{4.000616in}{0.684333in}}{\pgfqpoint{4.011666in}{0.684333in}}%
\pgfpathclose%
\pgfusepath{stroke,fill}%
\end{pgfscope}%
\begin{pgfscope}%
\pgfpathrectangle{\pgfqpoint{0.800000in}{0.528000in}}{\pgfqpoint{4.960000in}{3.696000in}}%
\pgfusepath{clip}%
\pgfsetbuttcap%
\pgfsetroundjoin%
\definecolor{currentfill}{rgb}{0.000000,0.000000,0.000000}%
\pgfsetfillcolor{currentfill}%
\pgfsetlinewidth{1.003750pt}%
\definecolor{currentstroke}{rgb}{0.000000,0.000000,0.000000}%
\pgfsetstrokecolor{currentstroke}%
\pgfsetdash{}{0pt}%
\pgfpathmoveto{\pgfqpoint{4.011666in}{0.684333in}}%
\pgfpathcurveto{\pgfqpoint{4.022716in}{0.684333in}}{\pgfqpoint{4.033315in}{0.688724in}}{\pgfqpoint{4.041128in}{0.696537in}}%
\pgfpathcurveto{\pgfqpoint{4.048942in}{0.704351in}}{\pgfqpoint{4.053332in}{0.714950in}}{\pgfqpoint{4.053332in}{0.726000in}}%
\pgfpathcurveto{\pgfqpoint{4.053332in}{0.737050in}}{\pgfqpoint{4.048942in}{0.747649in}}{\pgfqpoint{4.041128in}{0.755463in}}%
\pgfpathcurveto{\pgfqpoint{4.033315in}{0.763276in}}{\pgfqpoint{4.022716in}{0.767667in}}{\pgfqpoint{4.011666in}{0.767667in}}%
\pgfpathcurveto{\pgfqpoint{4.000616in}{0.767667in}}{\pgfqpoint{3.990016in}{0.763276in}}{\pgfqpoint{3.982203in}{0.755463in}}%
\pgfpathcurveto{\pgfqpoint{3.974389in}{0.747649in}}{\pgfqpoint{3.969999in}{0.737050in}}{\pgfqpoint{3.969999in}{0.726000in}}%
\pgfpathcurveto{\pgfqpoint{3.969999in}{0.714950in}}{\pgfqpoint{3.974389in}{0.704351in}}{\pgfqpoint{3.982203in}{0.696537in}}%
\pgfpathcurveto{\pgfqpoint{3.990016in}{0.688724in}}{\pgfqpoint{4.000616in}{0.684333in}}{\pgfqpoint{4.011666in}{0.684333in}}%
\pgfpathclose%
\pgfusepath{stroke,fill}%
\end{pgfscope}%
\begin{pgfscope}%
\pgfpathrectangle{\pgfqpoint{0.800000in}{0.528000in}}{\pgfqpoint{4.960000in}{3.696000in}}%
\pgfusepath{clip}%
\pgfsetbuttcap%
\pgfsetroundjoin%
\definecolor{currentfill}{rgb}{0.000000,0.000000,0.000000}%
\pgfsetfillcolor{currentfill}%
\pgfsetlinewidth{1.003750pt}%
\definecolor{currentstroke}{rgb}{0.000000,0.000000,0.000000}%
\pgfsetstrokecolor{currentstroke}%
\pgfsetdash{}{0pt}%
\pgfpathmoveto{\pgfqpoint{4.011666in}{0.684333in}}%
\pgfpathcurveto{\pgfqpoint{4.022716in}{0.684333in}}{\pgfqpoint{4.033315in}{0.688724in}}{\pgfqpoint{4.041128in}{0.696537in}}%
\pgfpathcurveto{\pgfqpoint{4.048942in}{0.704351in}}{\pgfqpoint{4.053332in}{0.714950in}}{\pgfqpoint{4.053332in}{0.726000in}}%
\pgfpathcurveto{\pgfqpoint{4.053332in}{0.737050in}}{\pgfqpoint{4.048942in}{0.747649in}}{\pgfqpoint{4.041128in}{0.755463in}}%
\pgfpathcurveto{\pgfqpoint{4.033315in}{0.763276in}}{\pgfqpoint{4.022716in}{0.767667in}}{\pgfqpoint{4.011666in}{0.767667in}}%
\pgfpathcurveto{\pgfqpoint{4.000616in}{0.767667in}}{\pgfqpoint{3.990016in}{0.763276in}}{\pgfqpoint{3.982203in}{0.755463in}}%
\pgfpathcurveto{\pgfqpoint{3.974389in}{0.747649in}}{\pgfqpoint{3.969999in}{0.737050in}}{\pgfqpoint{3.969999in}{0.726000in}}%
\pgfpathcurveto{\pgfqpoint{3.969999in}{0.714950in}}{\pgfqpoint{3.974389in}{0.704351in}}{\pgfqpoint{3.982203in}{0.696537in}}%
\pgfpathcurveto{\pgfqpoint{3.990016in}{0.688724in}}{\pgfqpoint{4.000616in}{0.684333in}}{\pgfqpoint{4.011666in}{0.684333in}}%
\pgfpathclose%
\pgfusepath{stroke,fill}%
\end{pgfscope}%
\begin{pgfscope}%
\pgfpathrectangle{\pgfqpoint{0.800000in}{0.528000in}}{\pgfqpoint{4.960000in}{3.696000in}}%
\pgfusepath{clip}%
\pgfsetbuttcap%
\pgfsetroundjoin%
\definecolor{currentfill}{rgb}{0.000000,0.000000,0.000000}%
\pgfsetfillcolor{currentfill}%
\pgfsetlinewidth{1.003750pt}%
\definecolor{currentstroke}{rgb}{0.000000,0.000000,0.000000}%
\pgfsetstrokecolor{currentstroke}%
\pgfsetdash{}{0pt}%
\pgfpathmoveto{\pgfqpoint{4.011666in}{0.684333in}}%
\pgfpathcurveto{\pgfqpoint{4.022716in}{0.684333in}}{\pgfqpoint{4.033315in}{0.688724in}}{\pgfqpoint{4.041128in}{0.696537in}}%
\pgfpathcurveto{\pgfqpoint{4.048942in}{0.704351in}}{\pgfqpoint{4.053332in}{0.714950in}}{\pgfqpoint{4.053332in}{0.726000in}}%
\pgfpathcurveto{\pgfqpoint{4.053332in}{0.737050in}}{\pgfqpoint{4.048942in}{0.747649in}}{\pgfqpoint{4.041128in}{0.755463in}}%
\pgfpathcurveto{\pgfqpoint{4.033315in}{0.763276in}}{\pgfqpoint{4.022716in}{0.767667in}}{\pgfqpoint{4.011666in}{0.767667in}}%
\pgfpathcurveto{\pgfqpoint{4.000616in}{0.767667in}}{\pgfqpoint{3.990016in}{0.763276in}}{\pgfqpoint{3.982203in}{0.755463in}}%
\pgfpathcurveto{\pgfqpoint{3.974389in}{0.747649in}}{\pgfqpoint{3.969999in}{0.737050in}}{\pgfqpoint{3.969999in}{0.726000in}}%
\pgfpathcurveto{\pgfqpoint{3.969999in}{0.714950in}}{\pgfqpoint{3.974389in}{0.704351in}}{\pgfqpoint{3.982203in}{0.696537in}}%
\pgfpathcurveto{\pgfqpoint{3.990016in}{0.688724in}}{\pgfqpoint{4.000616in}{0.684333in}}{\pgfqpoint{4.011666in}{0.684333in}}%
\pgfpathclose%
\pgfusepath{stroke,fill}%
\end{pgfscope}%
\begin{pgfscope}%
\pgfpathrectangle{\pgfqpoint{0.800000in}{0.528000in}}{\pgfqpoint{4.960000in}{3.696000in}}%
\pgfusepath{clip}%
\pgfsetbuttcap%
\pgfsetroundjoin%
\definecolor{currentfill}{rgb}{0.000000,0.000000,0.000000}%
\pgfsetfillcolor{currentfill}%
\pgfsetlinewidth{1.003750pt}%
\definecolor{currentstroke}{rgb}{0.000000,0.000000,0.000000}%
\pgfsetstrokecolor{currentstroke}%
\pgfsetdash{}{0pt}%
\pgfpathmoveto{\pgfqpoint{4.011666in}{0.684333in}}%
\pgfpathcurveto{\pgfqpoint{4.022716in}{0.684333in}}{\pgfqpoint{4.033315in}{0.688724in}}{\pgfqpoint{4.041128in}{0.696537in}}%
\pgfpathcurveto{\pgfqpoint{4.048942in}{0.704351in}}{\pgfqpoint{4.053332in}{0.714950in}}{\pgfqpoint{4.053332in}{0.726000in}}%
\pgfpathcurveto{\pgfqpoint{4.053332in}{0.737050in}}{\pgfqpoint{4.048942in}{0.747649in}}{\pgfqpoint{4.041128in}{0.755463in}}%
\pgfpathcurveto{\pgfqpoint{4.033315in}{0.763276in}}{\pgfqpoint{4.022716in}{0.767667in}}{\pgfqpoint{4.011666in}{0.767667in}}%
\pgfpathcurveto{\pgfqpoint{4.000616in}{0.767667in}}{\pgfqpoint{3.990016in}{0.763276in}}{\pgfqpoint{3.982203in}{0.755463in}}%
\pgfpathcurveto{\pgfqpoint{3.974389in}{0.747649in}}{\pgfqpoint{3.969999in}{0.737050in}}{\pgfqpoint{3.969999in}{0.726000in}}%
\pgfpathcurveto{\pgfqpoint{3.969999in}{0.714950in}}{\pgfqpoint{3.974389in}{0.704351in}}{\pgfqpoint{3.982203in}{0.696537in}}%
\pgfpathcurveto{\pgfqpoint{3.990016in}{0.688724in}}{\pgfqpoint{4.000616in}{0.684333in}}{\pgfqpoint{4.011666in}{0.684333in}}%
\pgfpathclose%
\pgfusepath{stroke,fill}%
\end{pgfscope}%
\begin{pgfscope}%
\pgfpathrectangle{\pgfqpoint{0.800000in}{0.528000in}}{\pgfqpoint{4.960000in}{3.696000in}}%
\pgfusepath{clip}%
\pgfsetbuttcap%
\pgfsetroundjoin%
\definecolor{currentfill}{rgb}{0.000000,0.000000,0.000000}%
\pgfsetfillcolor{currentfill}%
\pgfsetlinewidth{1.003750pt}%
\definecolor{currentstroke}{rgb}{0.000000,0.000000,0.000000}%
\pgfsetstrokecolor{currentstroke}%
\pgfsetdash{}{0pt}%
\pgfpathmoveto{\pgfqpoint{4.011666in}{0.684333in}}%
\pgfpathcurveto{\pgfqpoint{4.022716in}{0.684333in}}{\pgfqpoint{4.033315in}{0.688724in}}{\pgfqpoint{4.041128in}{0.696537in}}%
\pgfpathcurveto{\pgfqpoint{4.048942in}{0.704351in}}{\pgfqpoint{4.053332in}{0.714950in}}{\pgfqpoint{4.053332in}{0.726000in}}%
\pgfpathcurveto{\pgfqpoint{4.053332in}{0.737050in}}{\pgfqpoint{4.048942in}{0.747649in}}{\pgfqpoint{4.041128in}{0.755463in}}%
\pgfpathcurveto{\pgfqpoint{4.033315in}{0.763276in}}{\pgfqpoint{4.022716in}{0.767667in}}{\pgfqpoint{4.011666in}{0.767667in}}%
\pgfpathcurveto{\pgfqpoint{4.000616in}{0.767667in}}{\pgfqpoint{3.990016in}{0.763276in}}{\pgfqpoint{3.982203in}{0.755463in}}%
\pgfpathcurveto{\pgfqpoint{3.974389in}{0.747649in}}{\pgfqpoint{3.969999in}{0.737050in}}{\pgfqpoint{3.969999in}{0.726000in}}%
\pgfpathcurveto{\pgfqpoint{3.969999in}{0.714950in}}{\pgfqpoint{3.974389in}{0.704351in}}{\pgfqpoint{3.982203in}{0.696537in}}%
\pgfpathcurveto{\pgfqpoint{3.990016in}{0.688724in}}{\pgfqpoint{4.000616in}{0.684333in}}{\pgfqpoint{4.011666in}{0.684333in}}%
\pgfpathclose%
\pgfusepath{stroke,fill}%
\end{pgfscope}%
\begin{pgfscope}%
\pgfpathrectangle{\pgfqpoint{0.800000in}{0.528000in}}{\pgfqpoint{4.960000in}{3.696000in}}%
\pgfusepath{clip}%
\pgfsetbuttcap%
\pgfsetroundjoin%
\definecolor{currentfill}{rgb}{0.000000,0.000000,0.000000}%
\pgfsetfillcolor{currentfill}%
\pgfsetlinewidth{1.003750pt}%
\definecolor{currentstroke}{rgb}{0.000000,0.000000,0.000000}%
\pgfsetstrokecolor{currentstroke}%
\pgfsetdash{}{0pt}%
\pgfpathmoveto{\pgfqpoint{4.011666in}{0.684333in}}%
\pgfpathcurveto{\pgfqpoint{4.022716in}{0.684333in}}{\pgfqpoint{4.033315in}{0.688724in}}{\pgfqpoint{4.041128in}{0.696537in}}%
\pgfpathcurveto{\pgfqpoint{4.048942in}{0.704351in}}{\pgfqpoint{4.053332in}{0.714950in}}{\pgfqpoint{4.053332in}{0.726000in}}%
\pgfpathcurveto{\pgfqpoint{4.053332in}{0.737050in}}{\pgfqpoint{4.048942in}{0.747649in}}{\pgfqpoint{4.041128in}{0.755463in}}%
\pgfpathcurveto{\pgfqpoint{4.033315in}{0.763276in}}{\pgfqpoint{4.022716in}{0.767667in}}{\pgfqpoint{4.011666in}{0.767667in}}%
\pgfpathcurveto{\pgfqpoint{4.000616in}{0.767667in}}{\pgfqpoint{3.990016in}{0.763276in}}{\pgfqpoint{3.982203in}{0.755463in}}%
\pgfpathcurveto{\pgfqpoint{3.974389in}{0.747649in}}{\pgfqpoint{3.969999in}{0.737050in}}{\pgfqpoint{3.969999in}{0.726000in}}%
\pgfpathcurveto{\pgfqpoint{3.969999in}{0.714950in}}{\pgfqpoint{3.974389in}{0.704351in}}{\pgfqpoint{3.982203in}{0.696537in}}%
\pgfpathcurveto{\pgfqpoint{3.990016in}{0.688724in}}{\pgfqpoint{4.000616in}{0.684333in}}{\pgfqpoint{4.011666in}{0.684333in}}%
\pgfpathclose%
\pgfusepath{stroke,fill}%
\end{pgfscope}%
\begin{pgfscope}%
\pgfpathrectangle{\pgfqpoint{0.800000in}{0.528000in}}{\pgfqpoint{4.960000in}{3.696000in}}%
\pgfusepath{clip}%
\pgfsetbuttcap%
\pgfsetroundjoin%
\definecolor{currentfill}{rgb}{0.000000,0.000000,0.000000}%
\pgfsetfillcolor{currentfill}%
\pgfsetlinewidth{1.003750pt}%
\definecolor{currentstroke}{rgb}{0.000000,0.000000,0.000000}%
\pgfsetstrokecolor{currentstroke}%
\pgfsetdash{}{0pt}%
\pgfpathmoveto{\pgfqpoint{4.011666in}{0.684333in}}%
\pgfpathcurveto{\pgfqpoint{4.022716in}{0.684333in}}{\pgfqpoint{4.033315in}{0.688724in}}{\pgfqpoint{4.041128in}{0.696537in}}%
\pgfpathcurveto{\pgfqpoint{4.048942in}{0.704351in}}{\pgfqpoint{4.053332in}{0.714950in}}{\pgfqpoint{4.053332in}{0.726000in}}%
\pgfpathcurveto{\pgfqpoint{4.053332in}{0.737050in}}{\pgfqpoint{4.048942in}{0.747649in}}{\pgfqpoint{4.041128in}{0.755463in}}%
\pgfpathcurveto{\pgfqpoint{4.033315in}{0.763276in}}{\pgfqpoint{4.022716in}{0.767667in}}{\pgfqpoint{4.011666in}{0.767667in}}%
\pgfpathcurveto{\pgfqpoint{4.000616in}{0.767667in}}{\pgfqpoint{3.990016in}{0.763276in}}{\pgfqpoint{3.982203in}{0.755463in}}%
\pgfpathcurveto{\pgfqpoint{3.974389in}{0.747649in}}{\pgfqpoint{3.969999in}{0.737050in}}{\pgfqpoint{3.969999in}{0.726000in}}%
\pgfpathcurveto{\pgfqpoint{3.969999in}{0.714950in}}{\pgfqpoint{3.974389in}{0.704351in}}{\pgfqpoint{3.982203in}{0.696537in}}%
\pgfpathcurveto{\pgfqpoint{3.990016in}{0.688724in}}{\pgfqpoint{4.000616in}{0.684333in}}{\pgfqpoint{4.011666in}{0.684333in}}%
\pgfpathclose%
\pgfusepath{stroke,fill}%
\end{pgfscope}%
\begin{pgfscope}%
\pgfpathrectangle{\pgfqpoint{0.800000in}{0.528000in}}{\pgfqpoint{4.960000in}{3.696000in}}%
\pgfusepath{clip}%
\pgfsetbuttcap%
\pgfsetroundjoin%
\definecolor{currentfill}{rgb}{0.000000,0.000000,0.000000}%
\pgfsetfillcolor{currentfill}%
\pgfsetlinewidth{1.003750pt}%
\definecolor{currentstroke}{rgb}{0.000000,0.000000,0.000000}%
\pgfsetstrokecolor{currentstroke}%
\pgfsetdash{}{0pt}%
\pgfpathmoveto{\pgfqpoint{4.011666in}{0.684333in}}%
\pgfpathcurveto{\pgfqpoint{4.022716in}{0.684333in}}{\pgfqpoint{4.033315in}{0.688724in}}{\pgfqpoint{4.041128in}{0.696537in}}%
\pgfpathcurveto{\pgfqpoint{4.048942in}{0.704351in}}{\pgfqpoint{4.053332in}{0.714950in}}{\pgfqpoint{4.053332in}{0.726000in}}%
\pgfpathcurveto{\pgfqpoint{4.053332in}{0.737050in}}{\pgfqpoint{4.048942in}{0.747649in}}{\pgfqpoint{4.041128in}{0.755463in}}%
\pgfpathcurveto{\pgfqpoint{4.033315in}{0.763276in}}{\pgfqpoint{4.022716in}{0.767667in}}{\pgfqpoint{4.011666in}{0.767667in}}%
\pgfpathcurveto{\pgfqpoint{4.000616in}{0.767667in}}{\pgfqpoint{3.990016in}{0.763276in}}{\pgfqpoint{3.982203in}{0.755463in}}%
\pgfpathcurveto{\pgfqpoint{3.974389in}{0.747649in}}{\pgfqpoint{3.969999in}{0.737050in}}{\pgfqpoint{3.969999in}{0.726000in}}%
\pgfpathcurveto{\pgfqpoint{3.969999in}{0.714950in}}{\pgfqpoint{3.974389in}{0.704351in}}{\pgfqpoint{3.982203in}{0.696537in}}%
\pgfpathcurveto{\pgfqpoint{3.990016in}{0.688724in}}{\pgfqpoint{4.000616in}{0.684333in}}{\pgfqpoint{4.011666in}{0.684333in}}%
\pgfpathclose%
\pgfusepath{stroke,fill}%
\end{pgfscope}%
\begin{pgfscope}%
\pgfpathrectangle{\pgfqpoint{0.800000in}{0.528000in}}{\pgfqpoint{4.960000in}{3.696000in}}%
\pgfusepath{clip}%
\pgfsetbuttcap%
\pgfsetroundjoin%
\definecolor{currentfill}{rgb}{0.000000,0.000000,0.000000}%
\pgfsetfillcolor{currentfill}%
\pgfsetlinewidth{1.003750pt}%
\definecolor{currentstroke}{rgb}{0.000000,0.000000,0.000000}%
\pgfsetstrokecolor{currentstroke}%
\pgfsetdash{}{0pt}%
\pgfpathmoveto{\pgfqpoint{4.011666in}{0.684333in}}%
\pgfpathcurveto{\pgfqpoint{4.022716in}{0.684333in}}{\pgfqpoint{4.033315in}{0.688724in}}{\pgfqpoint{4.041128in}{0.696537in}}%
\pgfpathcurveto{\pgfqpoint{4.048942in}{0.704351in}}{\pgfqpoint{4.053332in}{0.714950in}}{\pgfqpoint{4.053332in}{0.726000in}}%
\pgfpathcurveto{\pgfqpoint{4.053332in}{0.737050in}}{\pgfqpoint{4.048942in}{0.747649in}}{\pgfqpoint{4.041128in}{0.755463in}}%
\pgfpathcurveto{\pgfqpoint{4.033315in}{0.763276in}}{\pgfqpoint{4.022716in}{0.767667in}}{\pgfqpoint{4.011666in}{0.767667in}}%
\pgfpathcurveto{\pgfqpoint{4.000616in}{0.767667in}}{\pgfqpoint{3.990016in}{0.763276in}}{\pgfqpoint{3.982203in}{0.755463in}}%
\pgfpathcurveto{\pgfqpoint{3.974389in}{0.747649in}}{\pgfqpoint{3.969999in}{0.737050in}}{\pgfqpoint{3.969999in}{0.726000in}}%
\pgfpathcurveto{\pgfqpoint{3.969999in}{0.714950in}}{\pgfqpoint{3.974389in}{0.704351in}}{\pgfqpoint{3.982203in}{0.696537in}}%
\pgfpathcurveto{\pgfqpoint{3.990016in}{0.688724in}}{\pgfqpoint{4.000616in}{0.684333in}}{\pgfqpoint{4.011666in}{0.684333in}}%
\pgfpathclose%
\pgfusepath{stroke,fill}%
\end{pgfscope}%
\begin{pgfscope}%
\pgfpathrectangle{\pgfqpoint{0.800000in}{0.528000in}}{\pgfqpoint{4.960000in}{3.696000in}}%
\pgfusepath{clip}%
\pgfsetbuttcap%
\pgfsetroundjoin%
\definecolor{currentfill}{rgb}{0.000000,0.000000,0.000000}%
\pgfsetfillcolor{currentfill}%
\pgfsetlinewidth{1.003750pt}%
\definecolor{currentstroke}{rgb}{0.000000,0.000000,0.000000}%
\pgfsetstrokecolor{currentstroke}%
\pgfsetdash{}{0pt}%
\pgfpathmoveto{\pgfqpoint{4.011666in}{0.684333in}}%
\pgfpathcurveto{\pgfqpoint{4.022716in}{0.684333in}}{\pgfqpoint{4.033315in}{0.688724in}}{\pgfqpoint{4.041128in}{0.696537in}}%
\pgfpathcurveto{\pgfqpoint{4.048942in}{0.704351in}}{\pgfqpoint{4.053332in}{0.714950in}}{\pgfqpoint{4.053332in}{0.726000in}}%
\pgfpathcurveto{\pgfqpoint{4.053332in}{0.737050in}}{\pgfqpoint{4.048942in}{0.747649in}}{\pgfqpoint{4.041128in}{0.755463in}}%
\pgfpathcurveto{\pgfqpoint{4.033315in}{0.763276in}}{\pgfqpoint{4.022716in}{0.767667in}}{\pgfqpoint{4.011666in}{0.767667in}}%
\pgfpathcurveto{\pgfqpoint{4.000616in}{0.767667in}}{\pgfqpoint{3.990016in}{0.763276in}}{\pgfqpoint{3.982203in}{0.755463in}}%
\pgfpathcurveto{\pgfqpoint{3.974389in}{0.747649in}}{\pgfqpoint{3.969999in}{0.737050in}}{\pgfqpoint{3.969999in}{0.726000in}}%
\pgfpathcurveto{\pgfqpoint{3.969999in}{0.714950in}}{\pgfqpoint{3.974389in}{0.704351in}}{\pgfqpoint{3.982203in}{0.696537in}}%
\pgfpathcurveto{\pgfqpoint{3.990016in}{0.688724in}}{\pgfqpoint{4.000616in}{0.684333in}}{\pgfqpoint{4.011666in}{0.684333in}}%
\pgfpathclose%
\pgfusepath{stroke,fill}%
\end{pgfscope}%
\begin{pgfscope}%
\pgfpathrectangle{\pgfqpoint{0.800000in}{0.528000in}}{\pgfqpoint{4.960000in}{3.696000in}}%
\pgfusepath{clip}%
\pgfsetbuttcap%
\pgfsetroundjoin%
\definecolor{currentfill}{rgb}{0.000000,0.000000,0.000000}%
\pgfsetfillcolor{currentfill}%
\pgfsetlinewidth{1.003750pt}%
\definecolor{currentstroke}{rgb}{0.000000,0.000000,0.000000}%
\pgfsetstrokecolor{currentstroke}%
\pgfsetdash{}{0pt}%
\pgfpathmoveto{\pgfqpoint{4.011666in}{0.684333in}}%
\pgfpathcurveto{\pgfqpoint{4.022716in}{0.684333in}}{\pgfqpoint{4.033315in}{0.688724in}}{\pgfqpoint{4.041128in}{0.696537in}}%
\pgfpathcurveto{\pgfqpoint{4.048942in}{0.704351in}}{\pgfqpoint{4.053332in}{0.714950in}}{\pgfqpoint{4.053332in}{0.726000in}}%
\pgfpathcurveto{\pgfqpoint{4.053332in}{0.737050in}}{\pgfqpoint{4.048942in}{0.747649in}}{\pgfqpoint{4.041128in}{0.755463in}}%
\pgfpathcurveto{\pgfqpoint{4.033315in}{0.763276in}}{\pgfqpoint{4.022716in}{0.767667in}}{\pgfqpoint{4.011666in}{0.767667in}}%
\pgfpathcurveto{\pgfqpoint{4.000616in}{0.767667in}}{\pgfqpoint{3.990016in}{0.763276in}}{\pgfqpoint{3.982203in}{0.755463in}}%
\pgfpathcurveto{\pgfqpoint{3.974389in}{0.747649in}}{\pgfqpoint{3.969999in}{0.737050in}}{\pgfqpoint{3.969999in}{0.726000in}}%
\pgfpathcurveto{\pgfqpoint{3.969999in}{0.714950in}}{\pgfqpoint{3.974389in}{0.704351in}}{\pgfqpoint{3.982203in}{0.696537in}}%
\pgfpathcurveto{\pgfqpoint{3.990016in}{0.688724in}}{\pgfqpoint{4.000616in}{0.684333in}}{\pgfqpoint{4.011666in}{0.684333in}}%
\pgfpathclose%
\pgfusepath{stroke,fill}%
\end{pgfscope}%
\begin{pgfscope}%
\pgfpathrectangle{\pgfqpoint{0.800000in}{0.528000in}}{\pgfqpoint{4.960000in}{3.696000in}}%
\pgfusepath{clip}%
\pgfsetbuttcap%
\pgfsetroundjoin%
\definecolor{currentfill}{rgb}{0.000000,0.000000,0.000000}%
\pgfsetfillcolor{currentfill}%
\pgfsetlinewidth{1.003750pt}%
\definecolor{currentstroke}{rgb}{0.000000,0.000000,0.000000}%
\pgfsetstrokecolor{currentstroke}%
\pgfsetdash{}{0pt}%
\pgfpathmoveto{\pgfqpoint{4.011666in}{0.684333in}}%
\pgfpathcurveto{\pgfqpoint{4.022716in}{0.684333in}}{\pgfqpoint{4.033315in}{0.688724in}}{\pgfqpoint{4.041128in}{0.696537in}}%
\pgfpathcurveto{\pgfqpoint{4.048942in}{0.704351in}}{\pgfqpoint{4.053332in}{0.714950in}}{\pgfqpoint{4.053332in}{0.726000in}}%
\pgfpathcurveto{\pgfqpoint{4.053332in}{0.737050in}}{\pgfqpoint{4.048942in}{0.747649in}}{\pgfqpoint{4.041128in}{0.755463in}}%
\pgfpathcurveto{\pgfqpoint{4.033315in}{0.763276in}}{\pgfqpoint{4.022716in}{0.767667in}}{\pgfqpoint{4.011666in}{0.767667in}}%
\pgfpathcurveto{\pgfqpoint{4.000616in}{0.767667in}}{\pgfqpoint{3.990016in}{0.763276in}}{\pgfqpoint{3.982203in}{0.755463in}}%
\pgfpathcurveto{\pgfqpoint{3.974389in}{0.747649in}}{\pgfqpoint{3.969999in}{0.737050in}}{\pgfqpoint{3.969999in}{0.726000in}}%
\pgfpathcurveto{\pgfqpoint{3.969999in}{0.714950in}}{\pgfqpoint{3.974389in}{0.704351in}}{\pgfqpoint{3.982203in}{0.696537in}}%
\pgfpathcurveto{\pgfqpoint{3.990016in}{0.688724in}}{\pgfqpoint{4.000616in}{0.684333in}}{\pgfqpoint{4.011666in}{0.684333in}}%
\pgfpathclose%
\pgfusepath{stroke,fill}%
\end{pgfscope}%
\begin{pgfscope}%
\pgfpathrectangle{\pgfqpoint{0.800000in}{0.528000in}}{\pgfqpoint{4.960000in}{3.696000in}}%
\pgfusepath{clip}%
\pgfsetbuttcap%
\pgfsetroundjoin%
\definecolor{currentfill}{rgb}{0.000000,0.000000,0.000000}%
\pgfsetfillcolor{currentfill}%
\pgfsetlinewidth{1.003750pt}%
\definecolor{currentstroke}{rgb}{0.000000,0.000000,0.000000}%
\pgfsetstrokecolor{currentstroke}%
\pgfsetdash{}{0pt}%
\pgfpathmoveto{\pgfqpoint{4.011666in}{0.684333in}}%
\pgfpathcurveto{\pgfqpoint{4.022716in}{0.684333in}}{\pgfqpoint{4.033315in}{0.688724in}}{\pgfqpoint{4.041128in}{0.696537in}}%
\pgfpathcurveto{\pgfqpoint{4.048942in}{0.704351in}}{\pgfqpoint{4.053332in}{0.714950in}}{\pgfqpoint{4.053332in}{0.726000in}}%
\pgfpathcurveto{\pgfqpoint{4.053332in}{0.737050in}}{\pgfqpoint{4.048942in}{0.747649in}}{\pgfqpoint{4.041128in}{0.755463in}}%
\pgfpathcurveto{\pgfqpoint{4.033315in}{0.763276in}}{\pgfqpoint{4.022716in}{0.767667in}}{\pgfqpoint{4.011666in}{0.767667in}}%
\pgfpathcurveto{\pgfqpoint{4.000616in}{0.767667in}}{\pgfqpoint{3.990016in}{0.763276in}}{\pgfqpoint{3.982203in}{0.755463in}}%
\pgfpathcurveto{\pgfqpoint{3.974389in}{0.747649in}}{\pgfqpoint{3.969999in}{0.737050in}}{\pgfqpoint{3.969999in}{0.726000in}}%
\pgfpathcurveto{\pgfqpoint{3.969999in}{0.714950in}}{\pgfqpoint{3.974389in}{0.704351in}}{\pgfqpoint{3.982203in}{0.696537in}}%
\pgfpathcurveto{\pgfqpoint{3.990016in}{0.688724in}}{\pgfqpoint{4.000616in}{0.684333in}}{\pgfqpoint{4.011666in}{0.684333in}}%
\pgfpathclose%
\pgfusepath{stroke,fill}%
\end{pgfscope}%
\begin{pgfscope}%
\pgfpathrectangle{\pgfqpoint{0.800000in}{0.528000in}}{\pgfqpoint{4.960000in}{3.696000in}}%
\pgfusepath{clip}%
\pgfsetbuttcap%
\pgfsetroundjoin%
\definecolor{currentfill}{rgb}{0.000000,0.000000,0.000000}%
\pgfsetfillcolor{currentfill}%
\pgfsetlinewidth{1.003750pt}%
\definecolor{currentstroke}{rgb}{0.000000,0.000000,0.000000}%
\pgfsetstrokecolor{currentstroke}%
\pgfsetdash{}{0pt}%
\pgfpathmoveto{\pgfqpoint{4.011666in}{0.684333in}}%
\pgfpathcurveto{\pgfqpoint{4.022716in}{0.684333in}}{\pgfqpoint{4.033315in}{0.688724in}}{\pgfqpoint{4.041128in}{0.696537in}}%
\pgfpathcurveto{\pgfqpoint{4.048942in}{0.704351in}}{\pgfqpoint{4.053332in}{0.714950in}}{\pgfqpoint{4.053332in}{0.726000in}}%
\pgfpathcurveto{\pgfqpoint{4.053332in}{0.737050in}}{\pgfqpoint{4.048942in}{0.747649in}}{\pgfqpoint{4.041128in}{0.755463in}}%
\pgfpathcurveto{\pgfqpoint{4.033315in}{0.763276in}}{\pgfqpoint{4.022716in}{0.767667in}}{\pgfqpoint{4.011666in}{0.767667in}}%
\pgfpathcurveto{\pgfqpoint{4.000616in}{0.767667in}}{\pgfqpoint{3.990016in}{0.763276in}}{\pgfqpoint{3.982203in}{0.755463in}}%
\pgfpathcurveto{\pgfqpoint{3.974389in}{0.747649in}}{\pgfqpoint{3.969999in}{0.737050in}}{\pgfqpoint{3.969999in}{0.726000in}}%
\pgfpathcurveto{\pgfqpoint{3.969999in}{0.714950in}}{\pgfqpoint{3.974389in}{0.704351in}}{\pgfqpoint{3.982203in}{0.696537in}}%
\pgfpathcurveto{\pgfqpoint{3.990016in}{0.688724in}}{\pgfqpoint{4.000616in}{0.684333in}}{\pgfqpoint{4.011666in}{0.684333in}}%
\pgfpathclose%
\pgfusepath{stroke,fill}%
\end{pgfscope}%
\begin{pgfscope}%
\pgfpathrectangle{\pgfqpoint{0.800000in}{0.528000in}}{\pgfqpoint{4.960000in}{3.696000in}}%
\pgfusepath{clip}%
\pgfsetbuttcap%
\pgfsetroundjoin%
\definecolor{currentfill}{rgb}{0.000000,0.000000,0.000000}%
\pgfsetfillcolor{currentfill}%
\pgfsetlinewidth{1.003750pt}%
\definecolor{currentstroke}{rgb}{0.000000,0.000000,0.000000}%
\pgfsetstrokecolor{currentstroke}%
\pgfsetdash{}{0pt}%
\pgfpathmoveto{\pgfqpoint{4.011666in}{0.684333in}}%
\pgfpathcurveto{\pgfqpoint{4.022716in}{0.684333in}}{\pgfqpoint{4.033315in}{0.688724in}}{\pgfqpoint{4.041128in}{0.696537in}}%
\pgfpathcurveto{\pgfqpoint{4.048942in}{0.704351in}}{\pgfqpoint{4.053332in}{0.714950in}}{\pgfqpoint{4.053332in}{0.726000in}}%
\pgfpathcurveto{\pgfqpoint{4.053332in}{0.737050in}}{\pgfqpoint{4.048942in}{0.747649in}}{\pgfqpoint{4.041128in}{0.755463in}}%
\pgfpathcurveto{\pgfqpoint{4.033315in}{0.763276in}}{\pgfqpoint{4.022716in}{0.767667in}}{\pgfqpoint{4.011666in}{0.767667in}}%
\pgfpathcurveto{\pgfqpoint{4.000616in}{0.767667in}}{\pgfqpoint{3.990016in}{0.763276in}}{\pgfqpoint{3.982203in}{0.755463in}}%
\pgfpathcurveto{\pgfqpoint{3.974389in}{0.747649in}}{\pgfqpoint{3.969999in}{0.737050in}}{\pgfqpoint{3.969999in}{0.726000in}}%
\pgfpathcurveto{\pgfqpoint{3.969999in}{0.714950in}}{\pgfqpoint{3.974389in}{0.704351in}}{\pgfqpoint{3.982203in}{0.696537in}}%
\pgfpathcurveto{\pgfqpoint{3.990016in}{0.688724in}}{\pgfqpoint{4.000616in}{0.684333in}}{\pgfqpoint{4.011666in}{0.684333in}}%
\pgfpathclose%
\pgfusepath{stroke,fill}%
\end{pgfscope}%
\begin{pgfscope}%
\pgfpathrectangle{\pgfqpoint{0.800000in}{0.528000in}}{\pgfqpoint{4.960000in}{3.696000in}}%
\pgfusepath{clip}%
\pgfsetbuttcap%
\pgfsetroundjoin%
\definecolor{currentfill}{rgb}{0.000000,0.000000,0.000000}%
\pgfsetfillcolor{currentfill}%
\pgfsetlinewidth{1.003750pt}%
\definecolor{currentstroke}{rgb}{0.000000,0.000000,0.000000}%
\pgfsetstrokecolor{currentstroke}%
\pgfsetdash{}{0pt}%
\pgfpathmoveto{\pgfqpoint{4.011666in}{0.684333in}}%
\pgfpathcurveto{\pgfqpoint{4.022716in}{0.684333in}}{\pgfqpoint{4.033315in}{0.688724in}}{\pgfqpoint{4.041128in}{0.696537in}}%
\pgfpathcurveto{\pgfqpoint{4.048942in}{0.704351in}}{\pgfqpoint{4.053332in}{0.714950in}}{\pgfqpoint{4.053332in}{0.726000in}}%
\pgfpathcurveto{\pgfqpoint{4.053332in}{0.737050in}}{\pgfqpoint{4.048942in}{0.747649in}}{\pgfqpoint{4.041128in}{0.755463in}}%
\pgfpathcurveto{\pgfqpoint{4.033315in}{0.763276in}}{\pgfqpoint{4.022716in}{0.767667in}}{\pgfqpoint{4.011666in}{0.767667in}}%
\pgfpathcurveto{\pgfqpoint{4.000616in}{0.767667in}}{\pgfqpoint{3.990016in}{0.763276in}}{\pgfqpoint{3.982203in}{0.755463in}}%
\pgfpathcurveto{\pgfqpoint{3.974389in}{0.747649in}}{\pgfqpoint{3.969999in}{0.737050in}}{\pgfqpoint{3.969999in}{0.726000in}}%
\pgfpathcurveto{\pgfqpoint{3.969999in}{0.714950in}}{\pgfqpoint{3.974389in}{0.704351in}}{\pgfqpoint{3.982203in}{0.696537in}}%
\pgfpathcurveto{\pgfqpoint{3.990016in}{0.688724in}}{\pgfqpoint{4.000616in}{0.684333in}}{\pgfqpoint{4.011666in}{0.684333in}}%
\pgfpathclose%
\pgfusepath{stroke,fill}%
\end{pgfscope}%
\begin{pgfscope}%
\pgfpathrectangle{\pgfqpoint{0.800000in}{0.528000in}}{\pgfqpoint{4.960000in}{3.696000in}}%
\pgfusepath{clip}%
\pgfsetbuttcap%
\pgfsetroundjoin%
\definecolor{currentfill}{rgb}{0.000000,0.000000,0.000000}%
\pgfsetfillcolor{currentfill}%
\pgfsetlinewidth{1.003750pt}%
\definecolor{currentstroke}{rgb}{0.000000,0.000000,0.000000}%
\pgfsetstrokecolor{currentstroke}%
\pgfsetdash{}{0pt}%
\pgfpathmoveto{\pgfqpoint{4.011666in}{0.684333in}}%
\pgfpathcurveto{\pgfqpoint{4.022716in}{0.684333in}}{\pgfqpoint{4.033315in}{0.688724in}}{\pgfqpoint{4.041128in}{0.696537in}}%
\pgfpathcurveto{\pgfqpoint{4.048942in}{0.704351in}}{\pgfqpoint{4.053332in}{0.714950in}}{\pgfqpoint{4.053332in}{0.726000in}}%
\pgfpathcurveto{\pgfqpoint{4.053332in}{0.737050in}}{\pgfqpoint{4.048942in}{0.747649in}}{\pgfqpoint{4.041128in}{0.755463in}}%
\pgfpathcurveto{\pgfqpoint{4.033315in}{0.763276in}}{\pgfqpoint{4.022716in}{0.767667in}}{\pgfqpoint{4.011666in}{0.767667in}}%
\pgfpathcurveto{\pgfqpoint{4.000616in}{0.767667in}}{\pgfqpoint{3.990016in}{0.763276in}}{\pgfqpoint{3.982203in}{0.755463in}}%
\pgfpathcurveto{\pgfqpoint{3.974389in}{0.747649in}}{\pgfqpoint{3.969999in}{0.737050in}}{\pgfqpoint{3.969999in}{0.726000in}}%
\pgfpathcurveto{\pgfqpoint{3.969999in}{0.714950in}}{\pgfqpoint{3.974389in}{0.704351in}}{\pgfqpoint{3.982203in}{0.696537in}}%
\pgfpathcurveto{\pgfqpoint{3.990016in}{0.688724in}}{\pgfqpoint{4.000616in}{0.684333in}}{\pgfqpoint{4.011666in}{0.684333in}}%
\pgfpathclose%
\pgfusepath{stroke,fill}%
\end{pgfscope}%
\begin{pgfscope}%
\pgfpathrectangle{\pgfqpoint{0.800000in}{0.528000in}}{\pgfqpoint{4.960000in}{3.696000in}}%
\pgfusepath{clip}%
\pgfsetbuttcap%
\pgfsetroundjoin%
\definecolor{currentfill}{rgb}{0.000000,0.000000,0.000000}%
\pgfsetfillcolor{currentfill}%
\pgfsetlinewidth{1.003750pt}%
\definecolor{currentstroke}{rgb}{0.000000,0.000000,0.000000}%
\pgfsetstrokecolor{currentstroke}%
\pgfsetdash{}{0pt}%
\pgfpathmoveto{\pgfqpoint{4.011666in}{0.684333in}}%
\pgfpathcurveto{\pgfqpoint{4.022716in}{0.684333in}}{\pgfqpoint{4.033315in}{0.688724in}}{\pgfqpoint{4.041128in}{0.696537in}}%
\pgfpathcurveto{\pgfqpoint{4.048942in}{0.704351in}}{\pgfqpoint{4.053332in}{0.714950in}}{\pgfqpoint{4.053332in}{0.726000in}}%
\pgfpathcurveto{\pgfqpoint{4.053332in}{0.737050in}}{\pgfqpoint{4.048942in}{0.747649in}}{\pgfqpoint{4.041128in}{0.755463in}}%
\pgfpathcurveto{\pgfqpoint{4.033315in}{0.763276in}}{\pgfqpoint{4.022716in}{0.767667in}}{\pgfqpoint{4.011666in}{0.767667in}}%
\pgfpathcurveto{\pgfqpoint{4.000616in}{0.767667in}}{\pgfqpoint{3.990016in}{0.763276in}}{\pgfqpoint{3.982203in}{0.755463in}}%
\pgfpathcurveto{\pgfqpoint{3.974389in}{0.747649in}}{\pgfqpoint{3.969999in}{0.737050in}}{\pgfqpoint{3.969999in}{0.726000in}}%
\pgfpathcurveto{\pgfqpoint{3.969999in}{0.714950in}}{\pgfqpoint{3.974389in}{0.704351in}}{\pgfqpoint{3.982203in}{0.696537in}}%
\pgfpathcurveto{\pgfqpoint{3.990016in}{0.688724in}}{\pgfqpoint{4.000616in}{0.684333in}}{\pgfqpoint{4.011666in}{0.684333in}}%
\pgfpathclose%
\pgfusepath{stroke,fill}%
\end{pgfscope}%
\begin{pgfscope}%
\pgfpathrectangle{\pgfqpoint{0.800000in}{0.528000in}}{\pgfqpoint{4.960000in}{3.696000in}}%
\pgfusepath{clip}%
\pgfsetbuttcap%
\pgfsetroundjoin%
\definecolor{currentfill}{rgb}{0.000000,0.000000,0.000000}%
\pgfsetfillcolor{currentfill}%
\pgfsetlinewidth{1.003750pt}%
\definecolor{currentstroke}{rgb}{0.000000,0.000000,0.000000}%
\pgfsetstrokecolor{currentstroke}%
\pgfsetdash{}{0pt}%
\pgfpathmoveto{\pgfqpoint{4.011666in}{0.684333in}}%
\pgfpathcurveto{\pgfqpoint{4.022716in}{0.684333in}}{\pgfqpoint{4.033315in}{0.688724in}}{\pgfqpoint{4.041128in}{0.696537in}}%
\pgfpathcurveto{\pgfqpoint{4.048942in}{0.704351in}}{\pgfqpoint{4.053332in}{0.714950in}}{\pgfqpoint{4.053332in}{0.726000in}}%
\pgfpathcurveto{\pgfqpoint{4.053332in}{0.737050in}}{\pgfqpoint{4.048942in}{0.747649in}}{\pgfqpoint{4.041128in}{0.755463in}}%
\pgfpathcurveto{\pgfqpoint{4.033315in}{0.763276in}}{\pgfqpoint{4.022716in}{0.767667in}}{\pgfqpoint{4.011666in}{0.767667in}}%
\pgfpathcurveto{\pgfqpoint{4.000616in}{0.767667in}}{\pgfqpoint{3.990016in}{0.763276in}}{\pgfqpoint{3.982203in}{0.755463in}}%
\pgfpathcurveto{\pgfqpoint{3.974389in}{0.747649in}}{\pgfqpoint{3.969999in}{0.737050in}}{\pgfqpoint{3.969999in}{0.726000in}}%
\pgfpathcurveto{\pgfqpoint{3.969999in}{0.714950in}}{\pgfqpoint{3.974389in}{0.704351in}}{\pgfqpoint{3.982203in}{0.696537in}}%
\pgfpathcurveto{\pgfqpoint{3.990016in}{0.688724in}}{\pgfqpoint{4.000616in}{0.684333in}}{\pgfqpoint{4.011666in}{0.684333in}}%
\pgfpathclose%
\pgfusepath{stroke,fill}%
\end{pgfscope}%
\begin{pgfscope}%
\pgfpathrectangle{\pgfqpoint{0.800000in}{0.528000in}}{\pgfqpoint{4.960000in}{3.696000in}}%
\pgfusepath{clip}%
\pgfsetbuttcap%
\pgfsetroundjoin%
\definecolor{currentfill}{rgb}{0.000000,0.000000,0.000000}%
\pgfsetfillcolor{currentfill}%
\pgfsetlinewidth{1.003750pt}%
\definecolor{currentstroke}{rgb}{0.000000,0.000000,0.000000}%
\pgfsetstrokecolor{currentstroke}%
\pgfsetdash{}{0pt}%
\pgfpathmoveto{\pgfqpoint{4.011666in}{0.684333in}}%
\pgfpathcurveto{\pgfqpoint{4.022716in}{0.684333in}}{\pgfqpoint{4.033315in}{0.688724in}}{\pgfqpoint{4.041128in}{0.696537in}}%
\pgfpathcurveto{\pgfqpoint{4.048942in}{0.704351in}}{\pgfqpoint{4.053332in}{0.714950in}}{\pgfqpoint{4.053332in}{0.726000in}}%
\pgfpathcurveto{\pgfqpoint{4.053332in}{0.737050in}}{\pgfqpoint{4.048942in}{0.747649in}}{\pgfqpoint{4.041128in}{0.755463in}}%
\pgfpathcurveto{\pgfqpoint{4.033315in}{0.763276in}}{\pgfqpoint{4.022716in}{0.767667in}}{\pgfqpoint{4.011666in}{0.767667in}}%
\pgfpathcurveto{\pgfqpoint{4.000616in}{0.767667in}}{\pgfqpoint{3.990016in}{0.763276in}}{\pgfqpoint{3.982203in}{0.755463in}}%
\pgfpathcurveto{\pgfqpoint{3.974389in}{0.747649in}}{\pgfqpoint{3.969999in}{0.737050in}}{\pgfqpoint{3.969999in}{0.726000in}}%
\pgfpathcurveto{\pgfqpoint{3.969999in}{0.714950in}}{\pgfqpoint{3.974389in}{0.704351in}}{\pgfqpoint{3.982203in}{0.696537in}}%
\pgfpathcurveto{\pgfqpoint{3.990016in}{0.688724in}}{\pgfqpoint{4.000616in}{0.684333in}}{\pgfqpoint{4.011666in}{0.684333in}}%
\pgfpathclose%
\pgfusepath{stroke,fill}%
\end{pgfscope}%
\begin{pgfscope}%
\pgfpathrectangle{\pgfqpoint{0.800000in}{0.528000in}}{\pgfqpoint{4.960000in}{3.696000in}}%
\pgfusepath{clip}%
\pgfsetbuttcap%
\pgfsetroundjoin%
\definecolor{currentfill}{rgb}{0.000000,0.000000,0.000000}%
\pgfsetfillcolor{currentfill}%
\pgfsetlinewidth{1.003750pt}%
\definecolor{currentstroke}{rgb}{0.000000,0.000000,0.000000}%
\pgfsetstrokecolor{currentstroke}%
\pgfsetdash{}{0pt}%
\pgfpathmoveto{\pgfqpoint{4.011666in}{0.684333in}}%
\pgfpathcurveto{\pgfqpoint{4.022716in}{0.684333in}}{\pgfqpoint{4.033315in}{0.688724in}}{\pgfqpoint{4.041128in}{0.696537in}}%
\pgfpathcurveto{\pgfqpoint{4.048942in}{0.704351in}}{\pgfqpoint{4.053332in}{0.714950in}}{\pgfqpoint{4.053332in}{0.726000in}}%
\pgfpathcurveto{\pgfqpoint{4.053332in}{0.737050in}}{\pgfqpoint{4.048942in}{0.747649in}}{\pgfqpoint{4.041128in}{0.755463in}}%
\pgfpathcurveto{\pgfqpoint{4.033315in}{0.763276in}}{\pgfqpoint{4.022716in}{0.767667in}}{\pgfqpoint{4.011666in}{0.767667in}}%
\pgfpathcurveto{\pgfqpoint{4.000616in}{0.767667in}}{\pgfqpoint{3.990016in}{0.763276in}}{\pgfqpoint{3.982203in}{0.755463in}}%
\pgfpathcurveto{\pgfqpoint{3.974389in}{0.747649in}}{\pgfqpoint{3.969999in}{0.737050in}}{\pgfqpoint{3.969999in}{0.726000in}}%
\pgfpathcurveto{\pgfqpoint{3.969999in}{0.714950in}}{\pgfqpoint{3.974389in}{0.704351in}}{\pgfqpoint{3.982203in}{0.696537in}}%
\pgfpathcurveto{\pgfqpoint{3.990016in}{0.688724in}}{\pgfqpoint{4.000616in}{0.684333in}}{\pgfqpoint{4.011666in}{0.684333in}}%
\pgfpathclose%
\pgfusepath{stroke,fill}%
\end{pgfscope}%
\begin{pgfscope}%
\pgfpathrectangle{\pgfqpoint{0.800000in}{0.528000in}}{\pgfqpoint{4.960000in}{3.696000in}}%
\pgfusepath{clip}%
\pgfsetbuttcap%
\pgfsetroundjoin%
\definecolor{currentfill}{rgb}{0.000000,0.000000,0.000000}%
\pgfsetfillcolor{currentfill}%
\pgfsetlinewidth{1.003750pt}%
\definecolor{currentstroke}{rgb}{0.000000,0.000000,0.000000}%
\pgfsetstrokecolor{currentstroke}%
\pgfsetdash{}{0pt}%
\pgfpathmoveto{\pgfqpoint{4.011666in}{0.684333in}}%
\pgfpathcurveto{\pgfqpoint{4.022716in}{0.684333in}}{\pgfqpoint{4.033315in}{0.688724in}}{\pgfqpoint{4.041128in}{0.696537in}}%
\pgfpathcurveto{\pgfqpoint{4.048942in}{0.704351in}}{\pgfqpoint{4.053332in}{0.714950in}}{\pgfqpoint{4.053332in}{0.726000in}}%
\pgfpathcurveto{\pgfqpoint{4.053332in}{0.737050in}}{\pgfqpoint{4.048942in}{0.747649in}}{\pgfqpoint{4.041128in}{0.755463in}}%
\pgfpathcurveto{\pgfqpoint{4.033315in}{0.763276in}}{\pgfqpoint{4.022716in}{0.767667in}}{\pgfqpoint{4.011666in}{0.767667in}}%
\pgfpathcurveto{\pgfqpoint{4.000616in}{0.767667in}}{\pgfqpoint{3.990016in}{0.763276in}}{\pgfqpoint{3.982203in}{0.755463in}}%
\pgfpathcurveto{\pgfqpoint{3.974389in}{0.747649in}}{\pgfqpoint{3.969999in}{0.737050in}}{\pgfqpoint{3.969999in}{0.726000in}}%
\pgfpathcurveto{\pgfqpoint{3.969999in}{0.714950in}}{\pgfqpoint{3.974389in}{0.704351in}}{\pgfqpoint{3.982203in}{0.696537in}}%
\pgfpathcurveto{\pgfqpoint{3.990016in}{0.688724in}}{\pgfqpoint{4.000616in}{0.684333in}}{\pgfqpoint{4.011666in}{0.684333in}}%
\pgfpathclose%
\pgfusepath{stroke,fill}%
\end{pgfscope}%
\begin{pgfscope}%
\pgfpathrectangle{\pgfqpoint{0.800000in}{0.528000in}}{\pgfqpoint{4.960000in}{3.696000in}}%
\pgfusepath{clip}%
\pgfsetbuttcap%
\pgfsetroundjoin%
\definecolor{currentfill}{rgb}{0.000000,0.000000,0.000000}%
\pgfsetfillcolor{currentfill}%
\pgfsetlinewidth{1.003750pt}%
\definecolor{currentstroke}{rgb}{0.000000,0.000000,0.000000}%
\pgfsetstrokecolor{currentstroke}%
\pgfsetdash{}{0pt}%
\pgfpathmoveto{\pgfqpoint{4.011666in}{0.684333in}}%
\pgfpathcurveto{\pgfqpoint{4.022716in}{0.684333in}}{\pgfqpoint{4.033315in}{0.688724in}}{\pgfqpoint{4.041128in}{0.696537in}}%
\pgfpathcurveto{\pgfqpoint{4.048942in}{0.704351in}}{\pgfqpoint{4.053332in}{0.714950in}}{\pgfqpoint{4.053332in}{0.726000in}}%
\pgfpathcurveto{\pgfqpoint{4.053332in}{0.737050in}}{\pgfqpoint{4.048942in}{0.747649in}}{\pgfqpoint{4.041128in}{0.755463in}}%
\pgfpathcurveto{\pgfqpoint{4.033315in}{0.763276in}}{\pgfqpoint{4.022716in}{0.767667in}}{\pgfqpoint{4.011666in}{0.767667in}}%
\pgfpathcurveto{\pgfqpoint{4.000616in}{0.767667in}}{\pgfqpoint{3.990016in}{0.763276in}}{\pgfqpoint{3.982203in}{0.755463in}}%
\pgfpathcurveto{\pgfqpoint{3.974389in}{0.747649in}}{\pgfqpoint{3.969999in}{0.737050in}}{\pgfqpoint{3.969999in}{0.726000in}}%
\pgfpathcurveto{\pgfqpoint{3.969999in}{0.714950in}}{\pgfqpoint{3.974389in}{0.704351in}}{\pgfqpoint{3.982203in}{0.696537in}}%
\pgfpathcurveto{\pgfqpoint{3.990016in}{0.688724in}}{\pgfqpoint{4.000616in}{0.684333in}}{\pgfqpoint{4.011666in}{0.684333in}}%
\pgfpathclose%
\pgfusepath{stroke,fill}%
\end{pgfscope}%
\begin{pgfscope}%
\pgfpathrectangle{\pgfqpoint{0.800000in}{0.528000in}}{\pgfqpoint{4.960000in}{3.696000in}}%
\pgfusepath{clip}%
\pgfsetbuttcap%
\pgfsetroundjoin%
\definecolor{currentfill}{rgb}{0.000000,0.000000,0.000000}%
\pgfsetfillcolor{currentfill}%
\pgfsetlinewidth{1.003750pt}%
\definecolor{currentstroke}{rgb}{0.000000,0.000000,0.000000}%
\pgfsetstrokecolor{currentstroke}%
\pgfsetdash{}{0pt}%
\pgfpathmoveto{\pgfqpoint{4.011666in}{0.684333in}}%
\pgfpathcurveto{\pgfqpoint{4.022716in}{0.684333in}}{\pgfqpoint{4.033315in}{0.688724in}}{\pgfqpoint{4.041128in}{0.696537in}}%
\pgfpathcurveto{\pgfqpoint{4.048942in}{0.704351in}}{\pgfqpoint{4.053332in}{0.714950in}}{\pgfqpoint{4.053332in}{0.726000in}}%
\pgfpathcurveto{\pgfqpoint{4.053332in}{0.737050in}}{\pgfqpoint{4.048942in}{0.747649in}}{\pgfqpoint{4.041128in}{0.755463in}}%
\pgfpathcurveto{\pgfqpoint{4.033315in}{0.763276in}}{\pgfqpoint{4.022716in}{0.767667in}}{\pgfqpoint{4.011666in}{0.767667in}}%
\pgfpathcurveto{\pgfqpoint{4.000616in}{0.767667in}}{\pgfqpoint{3.990016in}{0.763276in}}{\pgfqpoint{3.982203in}{0.755463in}}%
\pgfpathcurveto{\pgfqpoint{3.974389in}{0.747649in}}{\pgfqpoint{3.969999in}{0.737050in}}{\pgfqpoint{3.969999in}{0.726000in}}%
\pgfpathcurveto{\pgfqpoint{3.969999in}{0.714950in}}{\pgfqpoint{3.974389in}{0.704351in}}{\pgfqpoint{3.982203in}{0.696537in}}%
\pgfpathcurveto{\pgfqpoint{3.990016in}{0.688724in}}{\pgfqpoint{4.000616in}{0.684333in}}{\pgfqpoint{4.011666in}{0.684333in}}%
\pgfpathclose%
\pgfusepath{stroke,fill}%
\end{pgfscope}%
\begin{pgfscope}%
\pgfpathrectangle{\pgfqpoint{0.800000in}{0.528000in}}{\pgfqpoint{4.960000in}{3.696000in}}%
\pgfusepath{clip}%
\pgfsetbuttcap%
\pgfsetroundjoin%
\definecolor{currentfill}{rgb}{0.000000,0.000000,0.000000}%
\pgfsetfillcolor{currentfill}%
\pgfsetlinewidth{1.003750pt}%
\definecolor{currentstroke}{rgb}{0.000000,0.000000,0.000000}%
\pgfsetstrokecolor{currentstroke}%
\pgfsetdash{}{0pt}%
\pgfpathmoveto{\pgfqpoint{4.011666in}{0.684333in}}%
\pgfpathcurveto{\pgfqpoint{4.022716in}{0.684333in}}{\pgfqpoint{4.033315in}{0.688724in}}{\pgfqpoint{4.041128in}{0.696537in}}%
\pgfpathcurveto{\pgfqpoint{4.048942in}{0.704351in}}{\pgfqpoint{4.053332in}{0.714950in}}{\pgfqpoint{4.053332in}{0.726000in}}%
\pgfpathcurveto{\pgfqpoint{4.053332in}{0.737050in}}{\pgfqpoint{4.048942in}{0.747649in}}{\pgfqpoint{4.041128in}{0.755463in}}%
\pgfpathcurveto{\pgfqpoint{4.033315in}{0.763276in}}{\pgfqpoint{4.022716in}{0.767667in}}{\pgfqpoint{4.011666in}{0.767667in}}%
\pgfpathcurveto{\pgfqpoint{4.000616in}{0.767667in}}{\pgfqpoint{3.990016in}{0.763276in}}{\pgfqpoint{3.982203in}{0.755463in}}%
\pgfpathcurveto{\pgfqpoint{3.974389in}{0.747649in}}{\pgfqpoint{3.969999in}{0.737050in}}{\pgfqpoint{3.969999in}{0.726000in}}%
\pgfpathcurveto{\pgfqpoint{3.969999in}{0.714950in}}{\pgfqpoint{3.974389in}{0.704351in}}{\pgfqpoint{3.982203in}{0.696537in}}%
\pgfpathcurveto{\pgfqpoint{3.990016in}{0.688724in}}{\pgfqpoint{4.000616in}{0.684333in}}{\pgfqpoint{4.011666in}{0.684333in}}%
\pgfpathclose%
\pgfusepath{stroke,fill}%
\end{pgfscope}%
\begin{pgfscope}%
\pgfpathrectangle{\pgfqpoint{0.800000in}{0.528000in}}{\pgfqpoint{4.960000in}{3.696000in}}%
\pgfusepath{clip}%
\pgfsetbuttcap%
\pgfsetroundjoin%
\definecolor{currentfill}{rgb}{0.000000,0.000000,0.000000}%
\pgfsetfillcolor{currentfill}%
\pgfsetlinewidth{1.003750pt}%
\definecolor{currentstroke}{rgb}{0.000000,0.000000,0.000000}%
\pgfsetstrokecolor{currentstroke}%
\pgfsetdash{}{0pt}%
\pgfpathmoveto{\pgfqpoint{4.011666in}{0.684333in}}%
\pgfpathcurveto{\pgfqpoint{4.022716in}{0.684333in}}{\pgfqpoint{4.033315in}{0.688724in}}{\pgfqpoint{4.041128in}{0.696537in}}%
\pgfpathcurveto{\pgfqpoint{4.048942in}{0.704351in}}{\pgfqpoint{4.053332in}{0.714950in}}{\pgfqpoint{4.053332in}{0.726000in}}%
\pgfpathcurveto{\pgfqpoint{4.053332in}{0.737050in}}{\pgfqpoint{4.048942in}{0.747649in}}{\pgfqpoint{4.041128in}{0.755463in}}%
\pgfpathcurveto{\pgfqpoint{4.033315in}{0.763276in}}{\pgfqpoint{4.022716in}{0.767667in}}{\pgfqpoint{4.011666in}{0.767667in}}%
\pgfpathcurveto{\pgfqpoint{4.000616in}{0.767667in}}{\pgfqpoint{3.990016in}{0.763276in}}{\pgfqpoint{3.982203in}{0.755463in}}%
\pgfpathcurveto{\pgfqpoint{3.974389in}{0.747649in}}{\pgfqpoint{3.969999in}{0.737050in}}{\pgfqpoint{3.969999in}{0.726000in}}%
\pgfpathcurveto{\pgfqpoint{3.969999in}{0.714950in}}{\pgfqpoint{3.974389in}{0.704351in}}{\pgfqpoint{3.982203in}{0.696537in}}%
\pgfpathcurveto{\pgfqpoint{3.990016in}{0.688724in}}{\pgfqpoint{4.000616in}{0.684333in}}{\pgfqpoint{4.011666in}{0.684333in}}%
\pgfpathclose%
\pgfusepath{stroke,fill}%
\end{pgfscope}%
\begin{pgfscope}%
\pgfpathrectangle{\pgfqpoint{0.800000in}{0.528000in}}{\pgfqpoint{4.960000in}{3.696000in}}%
\pgfusepath{clip}%
\pgfsetbuttcap%
\pgfsetroundjoin%
\definecolor{currentfill}{rgb}{0.000000,0.000000,0.000000}%
\pgfsetfillcolor{currentfill}%
\pgfsetlinewidth{1.003750pt}%
\definecolor{currentstroke}{rgb}{0.000000,0.000000,0.000000}%
\pgfsetstrokecolor{currentstroke}%
\pgfsetdash{}{0pt}%
\pgfpathmoveto{\pgfqpoint{4.011666in}{0.684333in}}%
\pgfpathcurveto{\pgfqpoint{4.022716in}{0.684333in}}{\pgfqpoint{4.033315in}{0.688724in}}{\pgfqpoint{4.041128in}{0.696537in}}%
\pgfpathcurveto{\pgfqpoint{4.048942in}{0.704351in}}{\pgfqpoint{4.053332in}{0.714950in}}{\pgfqpoint{4.053332in}{0.726000in}}%
\pgfpathcurveto{\pgfqpoint{4.053332in}{0.737050in}}{\pgfqpoint{4.048942in}{0.747649in}}{\pgfqpoint{4.041128in}{0.755463in}}%
\pgfpathcurveto{\pgfqpoint{4.033315in}{0.763276in}}{\pgfqpoint{4.022716in}{0.767667in}}{\pgfqpoint{4.011666in}{0.767667in}}%
\pgfpathcurveto{\pgfqpoint{4.000616in}{0.767667in}}{\pgfqpoint{3.990016in}{0.763276in}}{\pgfqpoint{3.982203in}{0.755463in}}%
\pgfpathcurveto{\pgfqpoint{3.974389in}{0.747649in}}{\pgfqpoint{3.969999in}{0.737050in}}{\pgfqpoint{3.969999in}{0.726000in}}%
\pgfpathcurveto{\pgfqpoint{3.969999in}{0.714950in}}{\pgfqpoint{3.974389in}{0.704351in}}{\pgfqpoint{3.982203in}{0.696537in}}%
\pgfpathcurveto{\pgfqpoint{3.990016in}{0.688724in}}{\pgfqpoint{4.000616in}{0.684333in}}{\pgfqpoint{4.011666in}{0.684333in}}%
\pgfpathclose%
\pgfusepath{stroke,fill}%
\end{pgfscope}%
\begin{pgfscope}%
\pgfpathrectangle{\pgfqpoint{0.800000in}{0.528000in}}{\pgfqpoint{4.960000in}{3.696000in}}%
\pgfusepath{clip}%
\pgfsetbuttcap%
\pgfsetroundjoin%
\definecolor{currentfill}{rgb}{0.000000,0.000000,0.000000}%
\pgfsetfillcolor{currentfill}%
\pgfsetlinewidth{1.003750pt}%
\definecolor{currentstroke}{rgb}{0.000000,0.000000,0.000000}%
\pgfsetstrokecolor{currentstroke}%
\pgfsetdash{}{0pt}%
\pgfpathmoveto{\pgfqpoint{4.011666in}{0.684333in}}%
\pgfpathcurveto{\pgfqpoint{4.022716in}{0.684333in}}{\pgfqpoint{4.033315in}{0.688724in}}{\pgfqpoint{4.041128in}{0.696537in}}%
\pgfpathcurveto{\pgfqpoint{4.048942in}{0.704351in}}{\pgfqpoint{4.053332in}{0.714950in}}{\pgfqpoint{4.053332in}{0.726000in}}%
\pgfpathcurveto{\pgfqpoint{4.053332in}{0.737050in}}{\pgfqpoint{4.048942in}{0.747649in}}{\pgfqpoint{4.041128in}{0.755463in}}%
\pgfpathcurveto{\pgfqpoint{4.033315in}{0.763276in}}{\pgfqpoint{4.022716in}{0.767667in}}{\pgfqpoint{4.011666in}{0.767667in}}%
\pgfpathcurveto{\pgfqpoint{4.000616in}{0.767667in}}{\pgfqpoint{3.990016in}{0.763276in}}{\pgfqpoint{3.982203in}{0.755463in}}%
\pgfpathcurveto{\pgfqpoint{3.974389in}{0.747649in}}{\pgfqpoint{3.969999in}{0.737050in}}{\pgfqpoint{3.969999in}{0.726000in}}%
\pgfpathcurveto{\pgfqpoint{3.969999in}{0.714950in}}{\pgfqpoint{3.974389in}{0.704351in}}{\pgfqpoint{3.982203in}{0.696537in}}%
\pgfpathcurveto{\pgfqpoint{3.990016in}{0.688724in}}{\pgfqpoint{4.000616in}{0.684333in}}{\pgfqpoint{4.011666in}{0.684333in}}%
\pgfpathclose%
\pgfusepath{stroke,fill}%
\end{pgfscope}%
\begin{pgfscope}%
\pgfpathrectangle{\pgfqpoint{0.800000in}{0.528000in}}{\pgfqpoint{4.960000in}{3.696000in}}%
\pgfusepath{clip}%
\pgfsetbuttcap%
\pgfsetroundjoin%
\definecolor{currentfill}{rgb}{0.000000,0.000000,0.000000}%
\pgfsetfillcolor{currentfill}%
\pgfsetlinewidth{1.003750pt}%
\definecolor{currentstroke}{rgb}{0.000000,0.000000,0.000000}%
\pgfsetstrokecolor{currentstroke}%
\pgfsetdash{}{0pt}%
\pgfpathmoveto{\pgfqpoint{4.011666in}{0.684333in}}%
\pgfpathcurveto{\pgfqpoint{4.022716in}{0.684333in}}{\pgfqpoint{4.033315in}{0.688724in}}{\pgfqpoint{4.041128in}{0.696537in}}%
\pgfpathcurveto{\pgfqpoint{4.048942in}{0.704351in}}{\pgfqpoint{4.053332in}{0.714950in}}{\pgfqpoint{4.053332in}{0.726000in}}%
\pgfpathcurveto{\pgfqpoint{4.053332in}{0.737050in}}{\pgfqpoint{4.048942in}{0.747649in}}{\pgfqpoint{4.041128in}{0.755463in}}%
\pgfpathcurveto{\pgfqpoint{4.033315in}{0.763276in}}{\pgfqpoint{4.022716in}{0.767667in}}{\pgfqpoint{4.011666in}{0.767667in}}%
\pgfpathcurveto{\pgfqpoint{4.000616in}{0.767667in}}{\pgfqpoint{3.990016in}{0.763276in}}{\pgfqpoint{3.982203in}{0.755463in}}%
\pgfpathcurveto{\pgfqpoint{3.974389in}{0.747649in}}{\pgfqpoint{3.969999in}{0.737050in}}{\pgfqpoint{3.969999in}{0.726000in}}%
\pgfpathcurveto{\pgfqpoint{3.969999in}{0.714950in}}{\pgfqpoint{3.974389in}{0.704351in}}{\pgfqpoint{3.982203in}{0.696537in}}%
\pgfpathcurveto{\pgfqpoint{3.990016in}{0.688724in}}{\pgfqpoint{4.000616in}{0.684333in}}{\pgfqpoint{4.011666in}{0.684333in}}%
\pgfpathclose%
\pgfusepath{stroke,fill}%
\end{pgfscope}%
\begin{pgfscope}%
\pgfpathrectangle{\pgfqpoint{0.800000in}{0.528000in}}{\pgfqpoint{4.960000in}{3.696000in}}%
\pgfusepath{clip}%
\pgfsetbuttcap%
\pgfsetroundjoin%
\definecolor{currentfill}{rgb}{0.000000,0.000000,0.000000}%
\pgfsetfillcolor{currentfill}%
\pgfsetlinewidth{1.003750pt}%
\definecolor{currentstroke}{rgb}{0.000000,0.000000,0.000000}%
\pgfsetstrokecolor{currentstroke}%
\pgfsetdash{}{0pt}%
\pgfpathmoveto{\pgfqpoint{4.011666in}{0.684333in}}%
\pgfpathcurveto{\pgfqpoint{4.022716in}{0.684333in}}{\pgfqpoint{4.033315in}{0.688724in}}{\pgfqpoint{4.041128in}{0.696537in}}%
\pgfpathcurveto{\pgfqpoint{4.048942in}{0.704351in}}{\pgfqpoint{4.053332in}{0.714950in}}{\pgfqpoint{4.053332in}{0.726000in}}%
\pgfpathcurveto{\pgfqpoint{4.053332in}{0.737050in}}{\pgfqpoint{4.048942in}{0.747649in}}{\pgfqpoint{4.041128in}{0.755463in}}%
\pgfpathcurveto{\pgfqpoint{4.033315in}{0.763276in}}{\pgfqpoint{4.022716in}{0.767667in}}{\pgfqpoint{4.011666in}{0.767667in}}%
\pgfpathcurveto{\pgfqpoint{4.000616in}{0.767667in}}{\pgfqpoint{3.990016in}{0.763276in}}{\pgfqpoint{3.982203in}{0.755463in}}%
\pgfpathcurveto{\pgfqpoint{3.974389in}{0.747649in}}{\pgfqpoint{3.969999in}{0.737050in}}{\pgfqpoint{3.969999in}{0.726000in}}%
\pgfpathcurveto{\pgfqpoint{3.969999in}{0.714950in}}{\pgfqpoint{3.974389in}{0.704351in}}{\pgfqpoint{3.982203in}{0.696537in}}%
\pgfpathcurveto{\pgfqpoint{3.990016in}{0.688724in}}{\pgfqpoint{4.000616in}{0.684333in}}{\pgfqpoint{4.011666in}{0.684333in}}%
\pgfpathclose%
\pgfusepath{stroke,fill}%
\end{pgfscope}%
\begin{pgfscope}%
\pgfpathrectangle{\pgfqpoint{0.800000in}{0.528000in}}{\pgfqpoint{4.960000in}{3.696000in}}%
\pgfusepath{clip}%
\pgfsetbuttcap%
\pgfsetroundjoin%
\definecolor{currentfill}{rgb}{0.000000,0.000000,0.000000}%
\pgfsetfillcolor{currentfill}%
\pgfsetlinewidth{1.003750pt}%
\definecolor{currentstroke}{rgb}{0.000000,0.000000,0.000000}%
\pgfsetstrokecolor{currentstroke}%
\pgfsetdash{}{0pt}%
\pgfpathmoveto{\pgfqpoint{4.011666in}{0.684333in}}%
\pgfpathcurveto{\pgfqpoint{4.022716in}{0.684333in}}{\pgfqpoint{4.033315in}{0.688724in}}{\pgfqpoint{4.041128in}{0.696537in}}%
\pgfpathcurveto{\pgfqpoint{4.048942in}{0.704351in}}{\pgfqpoint{4.053332in}{0.714950in}}{\pgfqpoint{4.053332in}{0.726000in}}%
\pgfpathcurveto{\pgfqpoint{4.053332in}{0.737050in}}{\pgfqpoint{4.048942in}{0.747649in}}{\pgfqpoint{4.041128in}{0.755463in}}%
\pgfpathcurveto{\pgfqpoint{4.033315in}{0.763276in}}{\pgfqpoint{4.022716in}{0.767667in}}{\pgfqpoint{4.011666in}{0.767667in}}%
\pgfpathcurveto{\pgfqpoint{4.000616in}{0.767667in}}{\pgfqpoint{3.990016in}{0.763276in}}{\pgfqpoint{3.982203in}{0.755463in}}%
\pgfpathcurveto{\pgfqpoint{3.974389in}{0.747649in}}{\pgfqpoint{3.969999in}{0.737050in}}{\pgfqpoint{3.969999in}{0.726000in}}%
\pgfpathcurveto{\pgfqpoint{3.969999in}{0.714950in}}{\pgfqpoint{3.974389in}{0.704351in}}{\pgfqpoint{3.982203in}{0.696537in}}%
\pgfpathcurveto{\pgfqpoint{3.990016in}{0.688724in}}{\pgfqpoint{4.000616in}{0.684333in}}{\pgfqpoint{4.011666in}{0.684333in}}%
\pgfpathclose%
\pgfusepath{stroke,fill}%
\end{pgfscope}%
\begin{pgfscope}%
\pgfpathrectangle{\pgfqpoint{0.800000in}{0.528000in}}{\pgfqpoint{4.960000in}{3.696000in}}%
\pgfusepath{clip}%
\pgfsetbuttcap%
\pgfsetroundjoin%
\definecolor{currentfill}{rgb}{0.000000,0.000000,0.000000}%
\pgfsetfillcolor{currentfill}%
\pgfsetlinewidth{1.003750pt}%
\definecolor{currentstroke}{rgb}{0.000000,0.000000,0.000000}%
\pgfsetstrokecolor{currentstroke}%
\pgfsetdash{}{0pt}%
\pgfpathmoveto{\pgfqpoint{4.011666in}{0.684333in}}%
\pgfpathcurveto{\pgfqpoint{4.022716in}{0.684333in}}{\pgfqpoint{4.033315in}{0.688724in}}{\pgfqpoint{4.041128in}{0.696537in}}%
\pgfpathcurveto{\pgfqpoint{4.048942in}{0.704351in}}{\pgfqpoint{4.053332in}{0.714950in}}{\pgfqpoint{4.053332in}{0.726000in}}%
\pgfpathcurveto{\pgfqpoint{4.053332in}{0.737050in}}{\pgfqpoint{4.048942in}{0.747649in}}{\pgfqpoint{4.041128in}{0.755463in}}%
\pgfpathcurveto{\pgfqpoint{4.033315in}{0.763276in}}{\pgfqpoint{4.022716in}{0.767667in}}{\pgfqpoint{4.011666in}{0.767667in}}%
\pgfpathcurveto{\pgfqpoint{4.000616in}{0.767667in}}{\pgfqpoint{3.990016in}{0.763276in}}{\pgfqpoint{3.982203in}{0.755463in}}%
\pgfpathcurveto{\pgfqpoint{3.974389in}{0.747649in}}{\pgfqpoint{3.969999in}{0.737050in}}{\pgfqpoint{3.969999in}{0.726000in}}%
\pgfpathcurveto{\pgfqpoint{3.969999in}{0.714950in}}{\pgfqpoint{3.974389in}{0.704351in}}{\pgfqpoint{3.982203in}{0.696537in}}%
\pgfpathcurveto{\pgfqpoint{3.990016in}{0.688724in}}{\pgfqpoint{4.000616in}{0.684333in}}{\pgfqpoint{4.011666in}{0.684333in}}%
\pgfpathclose%
\pgfusepath{stroke,fill}%
\end{pgfscope}%
\begin{pgfscope}%
\pgfpathrectangle{\pgfqpoint{0.800000in}{0.528000in}}{\pgfqpoint{4.960000in}{3.696000in}}%
\pgfusepath{clip}%
\pgfsetbuttcap%
\pgfsetroundjoin%
\definecolor{currentfill}{rgb}{0.000000,0.000000,0.000000}%
\pgfsetfillcolor{currentfill}%
\pgfsetlinewidth{1.003750pt}%
\definecolor{currentstroke}{rgb}{0.000000,0.000000,0.000000}%
\pgfsetstrokecolor{currentstroke}%
\pgfsetdash{}{0pt}%
\pgfpathmoveto{\pgfqpoint{4.011666in}{0.684333in}}%
\pgfpathcurveto{\pgfqpoint{4.022716in}{0.684333in}}{\pgfqpoint{4.033315in}{0.688724in}}{\pgfqpoint{4.041128in}{0.696537in}}%
\pgfpathcurveto{\pgfqpoint{4.048942in}{0.704351in}}{\pgfqpoint{4.053332in}{0.714950in}}{\pgfqpoint{4.053332in}{0.726000in}}%
\pgfpathcurveto{\pgfqpoint{4.053332in}{0.737050in}}{\pgfqpoint{4.048942in}{0.747649in}}{\pgfqpoint{4.041128in}{0.755463in}}%
\pgfpathcurveto{\pgfqpoint{4.033315in}{0.763276in}}{\pgfqpoint{4.022716in}{0.767667in}}{\pgfqpoint{4.011666in}{0.767667in}}%
\pgfpathcurveto{\pgfqpoint{4.000616in}{0.767667in}}{\pgfqpoint{3.990016in}{0.763276in}}{\pgfqpoint{3.982203in}{0.755463in}}%
\pgfpathcurveto{\pgfqpoint{3.974389in}{0.747649in}}{\pgfqpoint{3.969999in}{0.737050in}}{\pgfqpoint{3.969999in}{0.726000in}}%
\pgfpathcurveto{\pgfqpoint{3.969999in}{0.714950in}}{\pgfqpoint{3.974389in}{0.704351in}}{\pgfqpoint{3.982203in}{0.696537in}}%
\pgfpathcurveto{\pgfqpoint{3.990016in}{0.688724in}}{\pgfqpoint{4.000616in}{0.684333in}}{\pgfqpoint{4.011666in}{0.684333in}}%
\pgfpathclose%
\pgfusepath{stroke,fill}%
\end{pgfscope}%
\begin{pgfscope}%
\pgfpathrectangle{\pgfqpoint{0.800000in}{0.528000in}}{\pgfqpoint{4.960000in}{3.696000in}}%
\pgfusepath{clip}%
\pgfsetbuttcap%
\pgfsetroundjoin%
\definecolor{currentfill}{rgb}{0.000000,0.000000,0.000000}%
\pgfsetfillcolor{currentfill}%
\pgfsetlinewidth{1.003750pt}%
\definecolor{currentstroke}{rgb}{0.000000,0.000000,0.000000}%
\pgfsetstrokecolor{currentstroke}%
\pgfsetdash{}{0pt}%
\pgfpathmoveto{\pgfqpoint{4.011666in}{0.684333in}}%
\pgfpathcurveto{\pgfqpoint{4.022716in}{0.684333in}}{\pgfqpoint{4.033315in}{0.688724in}}{\pgfqpoint{4.041128in}{0.696537in}}%
\pgfpathcurveto{\pgfqpoint{4.048942in}{0.704351in}}{\pgfqpoint{4.053332in}{0.714950in}}{\pgfqpoint{4.053332in}{0.726000in}}%
\pgfpathcurveto{\pgfqpoint{4.053332in}{0.737050in}}{\pgfqpoint{4.048942in}{0.747649in}}{\pgfqpoint{4.041128in}{0.755463in}}%
\pgfpathcurveto{\pgfqpoint{4.033315in}{0.763276in}}{\pgfqpoint{4.022716in}{0.767667in}}{\pgfqpoint{4.011666in}{0.767667in}}%
\pgfpathcurveto{\pgfqpoint{4.000616in}{0.767667in}}{\pgfqpoint{3.990016in}{0.763276in}}{\pgfqpoint{3.982203in}{0.755463in}}%
\pgfpathcurveto{\pgfqpoint{3.974389in}{0.747649in}}{\pgfqpoint{3.969999in}{0.737050in}}{\pgfqpoint{3.969999in}{0.726000in}}%
\pgfpathcurveto{\pgfqpoint{3.969999in}{0.714950in}}{\pgfqpoint{3.974389in}{0.704351in}}{\pgfqpoint{3.982203in}{0.696537in}}%
\pgfpathcurveto{\pgfqpoint{3.990016in}{0.688724in}}{\pgfqpoint{4.000616in}{0.684333in}}{\pgfqpoint{4.011666in}{0.684333in}}%
\pgfpathclose%
\pgfusepath{stroke,fill}%
\end{pgfscope}%
\begin{pgfscope}%
\pgfpathrectangle{\pgfqpoint{0.800000in}{0.528000in}}{\pgfqpoint{4.960000in}{3.696000in}}%
\pgfusepath{clip}%
\pgfsetbuttcap%
\pgfsetroundjoin%
\definecolor{currentfill}{rgb}{0.000000,0.000000,0.000000}%
\pgfsetfillcolor{currentfill}%
\pgfsetlinewidth{1.003750pt}%
\definecolor{currentstroke}{rgb}{0.000000,0.000000,0.000000}%
\pgfsetstrokecolor{currentstroke}%
\pgfsetdash{}{0pt}%
\pgfpathmoveto{\pgfqpoint{4.011666in}{0.684333in}}%
\pgfpathcurveto{\pgfqpoint{4.022716in}{0.684333in}}{\pgfqpoint{4.033315in}{0.688724in}}{\pgfqpoint{4.041128in}{0.696537in}}%
\pgfpathcurveto{\pgfqpoint{4.048942in}{0.704351in}}{\pgfqpoint{4.053332in}{0.714950in}}{\pgfqpoint{4.053332in}{0.726000in}}%
\pgfpathcurveto{\pgfqpoint{4.053332in}{0.737050in}}{\pgfqpoint{4.048942in}{0.747649in}}{\pgfqpoint{4.041128in}{0.755463in}}%
\pgfpathcurveto{\pgfqpoint{4.033315in}{0.763276in}}{\pgfqpoint{4.022716in}{0.767667in}}{\pgfqpoint{4.011666in}{0.767667in}}%
\pgfpathcurveto{\pgfqpoint{4.000616in}{0.767667in}}{\pgfqpoint{3.990016in}{0.763276in}}{\pgfqpoint{3.982203in}{0.755463in}}%
\pgfpathcurveto{\pgfqpoint{3.974389in}{0.747649in}}{\pgfqpoint{3.969999in}{0.737050in}}{\pgfqpoint{3.969999in}{0.726000in}}%
\pgfpathcurveto{\pgfqpoint{3.969999in}{0.714950in}}{\pgfqpoint{3.974389in}{0.704351in}}{\pgfqpoint{3.982203in}{0.696537in}}%
\pgfpathcurveto{\pgfqpoint{3.990016in}{0.688724in}}{\pgfqpoint{4.000616in}{0.684333in}}{\pgfqpoint{4.011666in}{0.684333in}}%
\pgfpathclose%
\pgfusepath{stroke,fill}%
\end{pgfscope}%
\begin{pgfscope}%
\pgfpathrectangle{\pgfqpoint{0.800000in}{0.528000in}}{\pgfqpoint{4.960000in}{3.696000in}}%
\pgfusepath{clip}%
\pgfsetbuttcap%
\pgfsetroundjoin%
\definecolor{currentfill}{rgb}{0.000000,0.000000,0.000000}%
\pgfsetfillcolor{currentfill}%
\pgfsetlinewidth{1.003750pt}%
\definecolor{currentstroke}{rgb}{0.000000,0.000000,0.000000}%
\pgfsetstrokecolor{currentstroke}%
\pgfsetdash{}{0pt}%
\pgfpathmoveto{\pgfqpoint{4.011666in}{0.684333in}}%
\pgfpathcurveto{\pgfqpoint{4.022716in}{0.684333in}}{\pgfqpoint{4.033315in}{0.688724in}}{\pgfqpoint{4.041128in}{0.696537in}}%
\pgfpathcurveto{\pgfqpoint{4.048942in}{0.704351in}}{\pgfqpoint{4.053332in}{0.714950in}}{\pgfqpoint{4.053332in}{0.726000in}}%
\pgfpathcurveto{\pgfqpoint{4.053332in}{0.737050in}}{\pgfqpoint{4.048942in}{0.747649in}}{\pgfqpoint{4.041128in}{0.755463in}}%
\pgfpathcurveto{\pgfqpoint{4.033315in}{0.763276in}}{\pgfqpoint{4.022716in}{0.767667in}}{\pgfqpoint{4.011666in}{0.767667in}}%
\pgfpathcurveto{\pgfqpoint{4.000616in}{0.767667in}}{\pgfqpoint{3.990016in}{0.763276in}}{\pgfqpoint{3.982203in}{0.755463in}}%
\pgfpathcurveto{\pgfqpoint{3.974389in}{0.747649in}}{\pgfqpoint{3.969999in}{0.737050in}}{\pgfqpoint{3.969999in}{0.726000in}}%
\pgfpathcurveto{\pgfqpoint{3.969999in}{0.714950in}}{\pgfqpoint{3.974389in}{0.704351in}}{\pgfqpoint{3.982203in}{0.696537in}}%
\pgfpathcurveto{\pgfqpoint{3.990016in}{0.688724in}}{\pgfqpoint{4.000616in}{0.684333in}}{\pgfqpoint{4.011666in}{0.684333in}}%
\pgfpathclose%
\pgfusepath{stroke,fill}%
\end{pgfscope}%
\begin{pgfscope}%
\pgfpathrectangle{\pgfqpoint{0.800000in}{0.528000in}}{\pgfqpoint{4.960000in}{3.696000in}}%
\pgfusepath{clip}%
\pgfsetbuttcap%
\pgfsetroundjoin%
\definecolor{currentfill}{rgb}{0.000000,0.000000,0.000000}%
\pgfsetfillcolor{currentfill}%
\pgfsetlinewidth{1.003750pt}%
\definecolor{currentstroke}{rgb}{0.000000,0.000000,0.000000}%
\pgfsetstrokecolor{currentstroke}%
\pgfsetdash{}{0pt}%
\pgfpathmoveto{\pgfqpoint{4.011666in}{0.684333in}}%
\pgfpathcurveto{\pgfqpoint{4.022716in}{0.684333in}}{\pgfqpoint{4.033315in}{0.688724in}}{\pgfqpoint{4.041128in}{0.696537in}}%
\pgfpathcurveto{\pgfqpoint{4.048942in}{0.704351in}}{\pgfqpoint{4.053332in}{0.714950in}}{\pgfqpoint{4.053332in}{0.726000in}}%
\pgfpathcurveto{\pgfqpoint{4.053332in}{0.737050in}}{\pgfqpoint{4.048942in}{0.747649in}}{\pgfqpoint{4.041128in}{0.755463in}}%
\pgfpathcurveto{\pgfqpoint{4.033315in}{0.763276in}}{\pgfqpoint{4.022716in}{0.767667in}}{\pgfqpoint{4.011666in}{0.767667in}}%
\pgfpathcurveto{\pgfqpoint{4.000616in}{0.767667in}}{\pgfqpoint{3.990016in}{0.763276in}}{\pgfqpoint{3.982203in}{0.755463in}}%
\pgfpathcurveto{\pgfqpoint{3.974389in}{0.747649in}}{\pgfqpoint{3.969999in}{0.737050in}}{\pgfqpoint{3.969999in}{0.726000in}}%
\pgfpathcurveto{\pgfqpoint{3.969999in}{0.714950in}}{\pgfqpoint{3.974389in}{0.704351in}}{\pgfqpoint{3.982203in}{0.696537in}}%
\pgfpathcurveto{\pgfqpoint{3.990016in}{0.688724in}}{\pgfqpoint{4.000616in}{0.684333in}}{\pgfqpoint{4.011666in}{0.684333in}}%
\pgfpathclose%
\pgfusepath{stroke,fill}%
\end{pgfscope}%
\begin{pgfscope}%
\pgfpathrectangle{\pgfqpoint{0.800000in}{0.528000in}}{\pgfqpoint{4.960000in}{3.696000in}}%
\pgfusepath{clip}%
\pgfsetbuttcap%
\pgfsetroundjoin%
\definecolor{currentfill}{rgb}{0.000000,0.000000,0.000000}%
\pgfsetfillcolor{currentfill}%
\pgfsetlinewidth{1.003750pt}%
\definecolor{currentstroke}{rgb}{0.000000,0.000000,0.000000}%
\pgfsetstrokecolor{currentstroke}%
\pgfsetdash{}{0pt}%
\pgfpathmoveto{\pgfqpoint{4.011666in}{0.684333in}}%
\pgfpathcurveto{\pgfqpoint{4.022716in}{0.684333in}}{\pgfqpoint{4.033315in}{0.688724in}}{\pgfqpoint{4.041128in}{0.696537in}}%
\pgfpathcurveto{\pgfqpoint{4.048942in}{0.704351in}}{\pgfqpoint{4.053332in}{0.714950in}}{\pgfqpoint{4.053332in}{0.726000in}}%
\pgfpathcurveto{\pgfqpoint{4.053332in}{0.737050in}}{\pgfqpoint{4.048942in}{0.747649in}}{\pgfqpoint{4.041128in}{0.755463in}}%
\pgfpathcurveto{\pgfqpoint{4.033315in}{0.763276in}}{\pgfqpoint{4.022716in}{0.767667in}}{\pgfqpoint{4.011666in}{0.767667in}}%
\pgfpathcurveto{\pgfqpoint{4.000616in}{0.767667in}}{\pgfqpoint{3.990016in}{0.763276in}}{\pgfqpoint{3.982203in}{0.755463in}}%
\pgfpathcurveto{\pgfqpoint{3.974389in}{0.747649in}}{\pgfqpoint{3.969999in}{0.737050in}}{\pgfqpoint{3.969999in}{0.726000in}}%
\pgfpathcurveto{\pgfqpoint{3.969999in}{0.714950in}}{\pgfqpoint{3.974389in}{0.704351in}}{\pgfqpoint{3.982203in}{0.696537in}}%
\pgfpathcurveto{\pgfqpoint{3.990016in}{0.688724in}}{\pgfqpoint{4.000616in}{0.684333in}}{\pgfqpoint{4.011666in}{0.684333in}}%
\pgfpathclose%
\pgfusepath{stroke,fill}%
\end{pgfscope}%
\begin{pgfscope}%
\pgfpathrectangle{\pgfqpoint{0.800000in}{0.528000in}}{\pgfqpoint{4.960000in}{3.696000in}}%
\pgfusepath{clip}%
\pgfsetbuttcap%
\pgfsetroundjoin%
\definecolor{currentfill}{rgb}{0.000000,0.000000,0.000000}%
\pgfsetfillcolor{currentfill}%
\pgfsetlinewidth{1.003750pt}%
\definecolor{currentstroke}{rgb}{0.000000,0.000000,0.000000}%
\pgfsetstrokecolor{currentstroke}%
\pgfsetdash{}{0pt}%
\pgfpathmoveto{\pgfqpoint{4.011666in}{0.684333in}}%
\pgfpathcurveto{\pgfqpoint{4.022716in}{0.684333in}}{\pgfqpoint{4.033315in}{0.688724in}}{\pgfqpoint{4.041128in}{0.696537in}}%
\pgfpathcurveto{\pgfqpoint{4.048942in}{0.704351in}}{\pgfqpoint{4.053332in}{0.714950in}}{\pgfqpoint{4.053332in}{0.726000in}}%
\pgfpathcurveto{\pgfqpoint{4.053332in}{0.737050in}}{\pgfqpoint{4.048942in}{0.747649in}}{\pgfqpoint{4.041128in}{0.755463in}}%
\pgfpathcurveto{\pgfqpoint{4.033315in}{0.763276in}}{\pgfqpoint{4.022716in}{0.767667in}}{\pgfqpoint{4.011666in}{0.767667in}}%
\pgfpathcurveto{\pgfqpoint{4.000616in}{0.767667in}}{\pgfqpoint{3.990016in}{0.763276in}}{\pgfqpoint{3.982203in}{0.755463in}}%
\pgfpathcurveto{\pgfqpoint{3.974389in}{0.747649in}}{\pgfqpoint{3.969999in}{0.737050in}}{\pgfqpoint{3.969999in}{0.726000in}}%
\pgfpathcurveto{\pgfqpoint{3.969999in}{0.714950in}}{\pgfqpoint{3.974389in}{0.704351in}}{\pgfqpoint{3.982203in}{0.696537in}}%
\pgfpathcurveto{\pgfqpoint{3.990016in}{0.688724in}}{\pgfqpoint{4.000616in}{0.684333in}}{\pgfqpoint{4.011666in}{0.684333in}}%
\pgfpathclose%
\pgfusepath{stroke,fill}%
\end{pgfscope}%
\begin{pgfscope}%
\pgfpathrectangle{\pgfqpoint{0.800000in}{0.528000in}}{\pgfqpoint{4.960000in}{3.696000in}}%
\pgfusepath{clip}%
\pgfsetbuttcap%
\pgfsetroundjoin%
\definecolor{currentfill}{rgb}{0.000000,0.000000,0.000000}%
\pgfsetfillcolor{currentfill}%
\pgfsetlinewidth{1.003750pt}%
\definecolor{currentstroke}{rgb}{0.000000,0.000000,0.000000}%
\pgfsetstrokecolor{currentstroke}%
\pgfsetdash{}{0pt}%
\pgfpathmoveto{\pgfqpoint{4.011666in}{0.684333in}}%
\pgfpathcurveto{\pgfqpoint{4.022716in}{0.684333in}}{\pgfqpoint{4.033315in}{0.688724in}}{\pgfqpoint{4.041128in}{0.696537in}}%
\pgfpathcurveto{\pgfqpoint{4.048942in}{0.704351in}}{\pgfqpoint{4.053332in}{0.714950in}}{\pgfqpoint{4.053332in}{0.726000in}}%
\pgfpathcurveto{\pgfqpoint{4.053332in}{0.737050in}}{\pgfqpoint{4.048942in}{0.747649in}}{\pgfqpoint{4.041128in}{0.755463in}}%
\pgfpathcurveto{\pgfqpoint{4.033315in}{0.763276in}}{\pgfqpoint{4.022716in}{0.767667in}}{\pgfqpoint{4.011666in}{0.767667in}}%
\pgfpathcurveto{\pgfqpoint{4.000616in}{0.767667in}}{\pgfqpoint{3.990016in}{0.763276in}}{\pgfqpoint{3.982203in}{0.755463in}}%
\pgfpathcurveto{\pgfqpoint{3.974389in}{0.747649in}}{\pgfqpoint{3.969999in}{0.737050in}}{\pgfqpoint{3.969999in}{0.726000in}}%
\pgfpathcurveto{\pgfqpoint{3.969999in}{0.714950in}}{\pgfqpoint{3.974389in}{0.704351in}}{\pgfqpoint{3.982203in}{0.696537in}}%
\pgfpathcurveto{\pgfqpoint{3.990016in}{0.688724in}}{\pgfqpoint{4.000616in}{0.684333in}}{\pgfqpoint{4.011666in}{0.684333in}}%
\pgfpathclose%
\pgfusepath{stroke,fill}%
\end{pgfscope}%
\begin{pgfscope}%
\pgfpathrectangle{\pgfqpoint{0.800000in}{0.528000in}}{\pgfqpoint{4.960000in}{3.696000in}}%
\pgfusepath{clip}%
\pgfsetbuttcap%
\pgfsetroundjoin%
\definecolor{currentfill}{rgb}{0.000000,0.000000,0.000000}%
\pgfsetfillcolor{currentfill}%
\pgfsetlinewidth{1.003750pt}%
\definecolor{currentstroke}{rgb}{0.000000,0.000000,0.000000}%
\pgfsetstrokecolor{currentstroke}%
\pgfsetdash{}{0pt}%
\pgfpathmoveto{\pgfqpoint{4.011666in}{0.684333in}}%
\pgfpathcurveto{\pgfqpoint{4.022716in}{0.684333in}}{\pgfqpoint{4.033315in}{0.688724in}}{\pgfqpoint{4.041128in}{0.696537in}}%
\pgfpathcurveto{\pgfqpoint{4.048942in}{0.704351in}}{\pgfqpoint{4.053332in}{0.714950in}}{\pgfqpoint{4.053332in}{0.726000in}}%
\pgfpathcurveto{\pgfqpoint{4.053332in}{0.737050in}}{\pgfqpoint{4.048942in}{0.747649in}}{\pgfqpoint{4.041128in}{0.755463in}}%
\pgfpathcurveto{\pgfqpoint{4.033315in}{0.763276in}}{\pgfqpoint{4.022716in}{0.767667in}}{\pgfqpoint{4.011666in}{0.767667in}}%
\pgfpathcurveto{\pgfqpoint{4.000616in}{0.767667in}}{\pgfqpoint{3.990016in}{0.763276in}}{\pgfqpoint{3.982203in}{0.755463in}}%
\pgfpathcurveto{\pgfqpoint{3.974389in}{0.747649in}}{\pgfqpoint{3.969999in}{0.737050in}}{\pgfqpoint{3.969999in}{0.726000in}}%
\pgfpathcurveto{\pgfqpoint{3.969999in}{0.714950in}}{\pgfqpoint{3.974389in}{0.704351in}}{\pgfqpoint{3.982203in}{0.696537in}}%
\pgfpathcurveto{\pgfqpoint{3.990016in}{0.688724in}}{\pgfqpoint{4.000616in}{0.684333in}}{\pgfqpoint{4.011666in}{0.684333in}}%
\pgfpathclose%
\pgfusepath{stroke,fill}%
\end{pgfscope}%
\begin{pgfscope}%
\pgfpathrectangle{\pgfqpoint{0.800000in}{0.528000in}}{\pgfqpoint{4.960000in}{3.696000in}}%
\pgfusepath{clip}%
\pgfsetbuttcap%
\pgfsetroundjoin%
\definecolor{currentfill}{rgb}{0.000000,0.000000,0.000000}%
\pgfsetfillcolor{currentfill}%
\pgfsetlinewidth{1.003750pt}%
\definecolor{currentstroke}{rgb}{0.000000,0.000000,0.000000}%
\pgfsetstrokecolor{currentstroke}%
\pgfsetdash{}{0pt}%
\pgfpathmoveto{\pgfqpoint{4.011666in}{0.684333in}}%
\pgfpathcurveto{\pgfqpoint{4.022716in}{0.684333in}}{\pgfqpoint{4.033315in}{0.688724in}}{\pgfqpoint{4.041128in}{0.696537in}}%
\pgfpathcurveto{\pgfqpoint{4.048942in}{0.704351in}}{\pgfqpoint{4.053332in}{0.714950in}}{\pgfqpoint{4.053332in}{0.726000in}}%
\pgfpathcurveto{\pgfqpoint{4.053332in}{0.737050in}}{\pgfqpoint{4.048942in}{0.747649in}}{\pgfqpoint{4.041128in}{0.755463in}}%
\pgfpathcurveto{\pgfqpoint{4.033315in}{0.763276in}}{\pgfqpoint{4.022716in}{0.767667in}}{\pgfqpoint{4.011666in}{0.767667in}}%
\pgfpathcurveto{\pgfqpoint{4.000616in}{0.767667in}}{\pgfqpoint{3.990016in}{0.763276in}}{\pgfqpoint{3.982203in}{0.755463in}}%
\pgfpathcurveto{\pgfqpoint{3.974389in}{0.747649in}}{\pgfqpoint{3.969999in}{0.737050in}}{\pgfqpoint{3.969999in}{0.726000in}}%
\pgfpathcurveto{\pgfqpoint{3.969999in}{0.714950in}}{\pgfqpoint{3.974389in}{0.704351in}}{\pgfqpoint{3.982203in}{0.696537in}}%
\pgfpathcurveto{\pgfqpoint{3.990016in}{0.688724in}}{\pgfqpoint{4.000616in}{0.684333in}}{\pgfqpoint{4.011666in}{0.684333in}}%
\pgfpathclose%
\pgfusepath{stroke,fill}%
\end{pgfscope}%
\begin{pgfscope}%
\pgfpathrectangle{\pgfqpoint{0.800000in}{0.528000in}}{\pgfqpoint{4.960000in}{3.696000in}}%
\pgfusepath{clip}%
\pgfsetbuttcap%
\pgfsetroundjoin%
\definecolor{currentfill}{rgb}{0.000000,0.000000,0.000000}%
\pgfsetfillcolor{currentfill}%
\pgfsetlinewidth{1.003750pt}%
\definecolor{currentstroke}{rgb}{0.000000,0.000000,0.000000}%
\pgfsetstrokecolor{currentstroke}%
\pgfsetdash{}{0pt}%
\pgfpathmoveto{\pgfqpoint{4.011666in}{0.684333in}}%
\pgfpathcurveto{\pgfqpoint{4.022716in}{0.684333in}}{\pgfqpoint{4.033315in}{0.688724in}}{\pgfqpoint{4.041128in}{0.696537in}}%
\pgfpathcurveto{\pgfqpoint{4.048942in}{0.704351in}}{\pgfqpoint{4.053332in}{0.714950in}}{\pgfqpoint{4.053332in}{0.726000in}}%
\pgfpathcurveto{\pgfqpoint{4.053332in}{0.737050in}}{\pgfqpoint{4.048942in}{0.747649in}}{\pgfqpoint{4.041128in}{0.755463in}}%
\pgfpathcurveto{\pgfqpoint{4.033315in}{0.763276in}}{\pgfqpoint{4.022716in}{0.767667in}}{\pgfqpoint{4.011666in}{0.767667in}}%
\pgfpathcurveto{\pgfqpoint{4.000616in}{0.767667in}}{\pgfqpoint{3.990016in}{0.763276in}}{\pgfqpoint{3.982203in}{0.755463in}}%
\pgfpathcurveto{\pgfqpoint{3.974389in}{0.747649in}}{\pgfqpoint{3.969999in}{0.737050in}}{\pgfqpoint{3.969999in}{0.726000in}}%
\pgfpathcurveto{\pgfqpoint{3.969999in}{0.714950in}}{\pgfqpoint{3.974389in}{0.704351in}}{\pgfqpoint{3.982203in}{0.696537in}}%
\pgfpathcurveto{\pgfqpoint{3.990016in}{0.688724in}}{\pgfqpoint{4.000616in}{0.684333in}}{\pgfqpoint{4.011666in}{0.684333in}}%
\pgfpathclose%
\pgfusepath{stroke,fill}%
\end{pgfscope}%
\begin{pgfscope}%
\pgfpathrectangle{\pgfqpoint{0.800000in}{0.528000in}}{\pgfqpoint{4.960000in}{3.696000in}}%
\pgfusepath{clip}%
\pgfsetbuttcap%
\pgfsetroundjoin%
\definecolor{currentfill}{rgb}{0.000000,0.000000,0.000000}%
\pgfsetfillcolor{currentfill}%
\pgfsetlinewidth{1.003750pt}%
\definecolor{currentstroke}{rgb}{0.000000,0.000000,0.000000}%
\pgfsetstrokecolor{currentstroke}%
\pgfsetdash{}{0pt}%
\pgfpathmoveto{\pgfqpoint{4.011666in}{0.684333in}}%
\pgfpathcurveto{\pgfqpoint{4.022716in}{0.684333in}}{\pgfqpoint{4.033315in}{0.688724in}}{\pgfqpoint{4.041128in}{0.696537in}}%
\pgfpathcurveto{\pgfqpoint{4.048942in}{0.704351in}}{\pgfqpoint{4.053332in}{0.714950in}}{\pgfqpoint{4.053332in}{0.726000in}}%
\pgfpathcurveto{\pgfqpoint{4.053332in}{0.737050in}}{\pgfqpoint{4.048942in}{0.747649in}}{\pgfqpoint{4.041128in}{0.755463in}}%
\pgfpathcurveto{\pgfqpoint{4.033315in}{0.763276in}}{\pgfqpoint{4.022716in}{0.767667in}}{\pgfqpoint{4.011666in}{0.767667in}}%
\pgfpathcurveto{\pgfqpoint{4.000616in}{0.767667in}}{\pgfqpoint{3.990016in}{0.763276in}}{\pgfqpoint{3.982203in}{0.755463in}}%
\pgfpathcurveto{\pgfqpoint{3.974389in}{0.747649in}}{\pgfqpoint{3.969999in}{0.737050in}}{\pgfqpoint{3.969999in}{0.726000in}}%
\pgfpathcurveto{\pgfqpoint{3.969999in}{0.714950in}}{\pgfqpoint{3.974389in}{0.704351in}}{\pgfqpoint{3.982203in}{0.696537in}}%
\pgfpathcurveto{\pgfqpoint{3.990016in}{0.688724in}}{\pgfqpoint{4.000616in}{0.684333in}}{\pgfqpoint{4.011666in}{0.684333in}}%
\pgfpathclose%
\pgfusepath{stroke,fill}%
\end{pgfscope}%
\begin{pgfscope}%
\pgfpathrectangle{\pgfqpoint{0.800000in}{0.528000in}}{\pgfqpoint{4.960000in}{3.696000in}}%
\pgfusepath{clip}%
\pgfsetbuttcap%
\pgfsetroundjoin%
\definecolor{currentfill}{rgb}{0.000000,0.000000,0.000000}%
\pgfsetfillcolor{currentfill}%
\pgfsetlinewidth{1.003750pt}%
\definecolor{currentstroke}{rgb}{0.000000,0.000000,0.000000}%
\pgfsetstrokecolor{currentstroke}%
\pgfsetdash{}{0pt}%
\pgfpathmoveto{\pgfqpoint{4.011666in}{0.684333in}}%
\pgfpathcurveto{\pgfqpoint{4.022716in}{0.684333in}}{\pgfqpoint{4.033315in}{0.688724in}}{\pgfqpoint{4.041128in}{0.696537in}}%
\pgfpathcurveto{\pgfqpoint{4.048942in}{0.704351in}}{\pgfqpoint{4.053332in}{0.714950in}}{\pgfqpoint{4.053332in}{0.726000in}}%
\pgfpathcurveto{\pgfqpoint{4.053332in}{0.737050in}}{\pgfqpoint{4.048942in}{0.747649in}}{\pgfqpoint{4.041128in}{0.755463in}}%
\pgfpathcurveto{\pgfqpoint{4.033315in}{0.763276in}}{\pgfqpoint{4.022716in}{0.767667in}}{\pgfqpoint{4.011666in}{0.767667in}}%
\pgfpathcurveto{\pgfqpoint{4.000616in}{0.767667in}}{\pgfqpoint{3.990016in}{0.763276in}}{\pgfqpoint{3.982203in}{0.755463in}}%
\pgfpathcurveto{\pgfqpoint{3.974389in}{0.747649in}}{\pgfqpoint{3.969999in}{0.737050in}}{\pgfqpoint{3.969999in}{0.726000in}}%
\pgfpathcurveto{\pgfqpoint{3.969999in}{0.714950in}}{\pgfqpoint{3.974389in}{0.704351in}}{\pgfqpoint{3.982203in}{0.696537in}}%
\pgfpathcurveto{\pgfqpoint{3.990016in}{0.688724in}}{\pgfqpoint{4.000616in}{0.684333in}}{\pgfqpoint{4.011666in}{0.684333in}}%
\pgfpathclose%
\pgfusepath{stroke,fill}%
\end{pgfscope}%
\begin{pgfscope}%
\pgfpathrectangle{\pgfqpoint{0.800000in}{0.528000in}}{\pgfqpoint{4.960000in}{3.696000in}}%
\pgfusepath{clip}%
\pgfsetbuttcap%
\pgfsetroundjoin%
\definecolor{currentfill}{rgb}{0.000000,0.000000,0.000000}%
\pgfsetfillcolor{currentfill}%
\pgfsetlinewidth{1.003750pt}%
\definecolor{currentstroke}{rgb}{0.000000,0.000000,0.000000}%
\pgfsetstrokecolor{currentstroke}%
\pgfsetdash{}{0pt}%
\pgfpathmoveto{\pgfqpoint{4.011666in}{0.684333in}}%
\pgfpathcurveto{\pgfqpoint{4.022716in}{0.684333in}}{\pgfqpoint{4.033315in}{0.688724in}}{\pgfqpoint{4.041128in}{0.696537in}}%
\pgfpathcurveto{\pgfqpoint{4.048942in}{0.704351in}}{\pgfqpoint{4.053332in}{0.714950in}}{\pgfqpoint{4.053332in}{0.726000in}}%
\pgfpathcurveto{\pgfqpoint{4.053332in}{0.737050in}}{\pgfqpoint{4.048942in}{0.747649in}}{\pgfqpoint{4.041128in}{0.755463in}}%
\pgfpathcurveto{\pgfqpoint{4.033315in}{0.763276in}}{\pgfqpoint{4.022716in}{0.767667in}}{\pgfqpoint{4.011666in}{0.767667in}}%
\pgfpathcurveto{\pgfqpoint{4.000616in}{0.767667in}}{\pgfqpoint{3.990016in}{0.763276in}}{\pgfqpoint{3.982203in}{0.755463in}}%
\pgfpathcurveto{\pgfqpoint{3.974389in}{0.747649in}}{\pgfqpoint{3.969999in}{0.737050in}}{\pgfqpoint{3.969999in}{0.726000in}}%
\pgfpathcurveto{\pgfqpoint{3.969999in}{0.714950in}}{\pgfqpoint{3.974389in}{0.704351in}}{\pgfqpoint{3.982203in}{0.696537in}}%
\pgfpathcurveto{\pgfqpoint{3.990016in}{0.688724in}}{\pgfqpoint{4.000616in}{0.684333in}}{\pgfqpoint{4.011666in}{0.684333in}}%
\pgfpathclose%
\pgfusepath{stroke,fill}%
\end{pgfscope}%
\begin{pgfscope}%
\pgfpathrectangle{\pgfqpoint{0.800000in}{0.528000in}}{\pgfqpoint{4.960000in}{3.696000in}}%
\pgfusepath{clip}%
\pgfsetbuttcap%
\pgfsetroundjoin%
\definecolor{currentfill}{rgb}{0.000000,0.000000,0.000000}%
\pgfsetfillcolor{currentfill}%
\pgfsetlinewidth{1.003750pt}%
\definecolor{currentstroke}{rgb}{0.000000,0.000000,0.000000}%
\pgfsetstrokecolor{currentstroke}%
\pgfsetdash{}{0pt}%
\pgfpathmoveto{\pgfqpoint{4.011666in}{0.684333in}}%
\pgfpathcurveto{\pgfqpoint{4.022716in}{0.684333in}}{\pgfqpoint{4.033315in}{0.688724in}}{\pgfqpoint{4.041128in}{0.696537in}}%
\pgfpathcurveto{\pgfqpoint{4.048942in}{0.704351in}}{\pgfqpoint{4.053332in}{0.714950in}}{\pgfqpoint{4.053332in}{0.726000in}}%
\pgfpathcurveto{\pgfqpoint{4.053332in}{0.737050in}}{\pgfqpoint{4.048942in}{0.747649in}}{\pgfqpoint{4.041128in}{0.755463in}}%
\pgfpathcurveto{\pgfqpoint{4.033315in}{0.763276in}}{\pgfqpoint{4.022716in}{0.767667in}}{\pgfqpoint{4.011666in}{0.767667in}}%
\pgfpathcurveto{\pgfqpoint{4.000616in}{0.767667in}}{\pgfqpoint{3.990016in}{0.763276in}}{\pgfqpoint{3.982203in}{0.755463in}}%
\pgfpathcurveto{\pgfqpoint{3.974389in}{0.747649in}}{\pgfqpoint{3.969999in}{0.737050in}}{\pgfqpoint{3.969999in}{0.726000in}}%
\pgfpathcurveto{\pgfqpoint{3.969999in}{0.714950in}}{\pgfqpoint{3.974389in}{0.704351in}}{\pgfqpoint{3.982203in}{0.696537in}}%
\pgfpathcurveto{\pgfqpoint{3.990016in}{0.688724in}}{\pgfqpoint{4.000616in}{0.684333in}}{\pgfqpoint{4.011666in}{0.684333in}}%
\pgfpathclose%
\pgfusepath{stroke,fill}%
\end{pgfscope}%
\begin{pgfscope}%
\pgfpathrectangle{\pgfqpoint{0.800000in}{0.528000in}}{\pgfqpoint{4.960000in}{3.696000in}}%
\pgfusepath{clip}%
\pgfsetbuttcap%
\pgfsetroundjoin%
\definecolor{currentfill}{rgb}{0.000000,0.000000,0.000000}%
\pgfsetfillcolor{currentfill}%
\pgfsetlinewidth{1.003750pt}%
\definecolor{currentstroke}{rgb}{0.000000,0.000000,0.000000}%
\pgfsetstrokecolor{currentstroke}%
\pgfsetdash{}{0pt}%
\pgfpathmoveto{\pgfqpoint{4.011666in}{0.684333in}}%
\pgfpathcurveto{\pgfqpoint{4.022716in}{0.684333in}}{\pgfqpoint{4.033315in}{0.688724in}}{\pgfqpoint{4.041128in}{0.696537in}}%
\pgfpathcurveto{\pgfqpoint{4.048942in}{0.704351in}}{\pgfqpoint{4.053332in}{0.714950in}}{\pgfqpoint{4.053332in}{0.726000in}}%
\pgfpathcurveto{\pgfqpoint{4.053332in}{0.737050in}}{\pgfqpoint{4.048942in}{0.747649in}}{\pgfqpoint{4.041128in}{0.755463in}}%
\pgfpathcurveto{\pgfqpoint{4.033315in}{0.763276in}}{\pgfqpoint{4.022716in}{0.767667in}}{\pgfqpoint{4.011666in}{0.767667in}}%
\pgfpathcurveto{\pgfqpoint{4.000616in}{0.767667in}}{\pgfqpoint{3.990016in}{0.763276in}}{\pgfqpoint{3.982203in}{0.755463in}}%
\pgfpathcurveto{\pgfqpoint{3.974389in}{0.747649in}}{\pgfqpoint{3.969999in}{0.737050in}}{\pgfqpoint{3.969999in}{0.726000in}}%
\pgfpathcurveto{\pgfqpoint{3.969999in}{0.714950in}}{\pgfqpoint{3.974389in}{0.704351in}}{\pgfqpoint{3.982203in}{0.696537in}}%
\pgfpathcurveto{\pgfqpoint{3.990016in}{0.688724in}}{\pgfqpoint{4.000616in}{0.684333in}}{\pgfqpoint{4.011666in}{0.684333in}}%
\pgfpathclose%
\pgfusepath{stroke,fill}%
\end{pgfscope}%
\begin{pgfscope}%
\pgfpathrectangle{\pgfqpoint{0.800000in}{0.528000in}}{\pgfqpoint{4.960000in}{3.696000in}}%
\pgfusepath{clip}%
\pgfsetbuttcap%
\pgfsetroundjoin%
\definecolor{currentfill}{rgb}{0.000000,0.000000,0.000000}%
\pgfsetfillcolor{currentfill}%
\pgfsetlinewidth{1.003750pt}%
\definecolor{currentstroke}{rgb}{0.000000,0.000000,0.000000}%
\pgfsetstrokecolor{currentstroke}%
\pgfsetdash{}{0pt}%
\pgfpathmoveto{\pgfqpoint{4.011666in}{0.684333in}}%
\pgfpathcurveto{\pgfqpoint{4.022716in}{0.684333in}}{\pgfqpoint{4.033315in}{0.688724in}}{\pgfqpoint{4.041128in}{0.696537in}}%
\pgfpathcurveto{\pgfqpoint{4.048942in}{0.704351in}}{\pgfqpoint{4.053332in}{0.714950in}}{\pgfqpoint{4.053332in}{0.726000in}}%
\pgfpathcurveto{\pgfqpoint{4.053332in}{0.737050in}}{\pgfqpoint{4.048942in}{0.747649in}}{\pgfqpoint{4.041128in}{0.755463in}}%
\pgfpathcurveto{\pgfqpoint{4.033315in}{0.763276in}}{\pgfqpoint{4.022716in}{0.767667in}}{\pgfqpoint{4.011666in}{0.767667in}}%
\pgfpathcurveto{\pgfqpoint{4.000616in}{0.767667in}}{\pgfqpoint{3.990016in}{0.763276in}}{\pgfqpoint{3.982203in}{0.755463in}}%
\pgfpathcurveto{\pgfqpoint{3.974389in}{0.747649in}}{\pgfqpoint{3.969999in}{0.737050in}}{\pgfqpoint{3.969999in}{0.726000in}}%
\pgfpathcurveto{\pgfqpoint{3.969999in}{0.714950in}}{\pgfqpoint{3.974389in}{0.704351in}}{\pgfqpoint{3.982203in}{0.696537in}}%
\pgfpathcurveto{\pgfqpoint{3.990016in}{0.688724in}}{\pgfqpoint{4.000616in}{0.684333in}}{\pgfqpoint{4.011666in}{0.684333in}}%
\pgfpathclose%
\pgfusepath{stroke,fill}%
\end{pgfscope}%
\begin{pgfscope}%
\pgfpathrectangle{\pgfqpoint{0.800000in}{0.528000in}}{\pgfqpoint{4.960000in}{3.696000in}}%
\pgfusepath{clip}%
\pgfsetbuttcap%
\pgfsetroundjoin%
\definecolor{currentfill}{rgb}{0.000000,0.000000,0.000000}%
\pgfsetfillcolor{currentfill}%
\pgfsetlinewidth{1.003750pt}%
\definecolor{currentstroke}{rgb}{0.000000,0.000000,0.000000}%
\pgfsetstrokecolor{currentstroke}%
\pgfsetdash{}{0pt}%
\pgfpathmoveto{\pgfqpoint{4.011666in}{0.684333in}}%
\pgfpathcurveto{\pgfqpoint{4.022716in}{0.684333in}}{\pgfqpoint{4.033315in}{0.688724in}}{\pgfqpoint{4.041128in}{0.696537in}}%
\pgfpathcurveto{\pgfqpoint{4.048942in}{0.704351in}}{\pgfqpoint{4.053332in}{0.714950in}}{\pgfqpoint{4.053332in}{0.726000in}}%
\pgfpathcurveto{\pgfqpoint{4.053332in}{0.737050in}}{\pgfqpoint{4.048942in}{0.747649in}}{\pgfqpoint{4.041128in}{0.755463in}}%
\pgfpathcurveto{\pgfqpoint{4.033315in}{0.763276in}}{\pgfqpoint{4.022716in}{0.767667in}}{\pgfqpoint{4.011666in}{0.767667in}}%
\pgfpathcurveto{\pgfqpoint{4.000616in}{0.767667in}}{\pgfqpoint{3.990016in}{0.763276in}}{\pgfqpoint{3.982203in}{0.755463in}}%
\pgfpathcurveto{\pgfqpoint{3.974389in}{0.747649in}}{\pgfqpoint{3.969999in}{0.737050in}}{\pgfqpoint{3.969999in}{0.726000in}}%
\pgfpathcurveto{\pgfqpoint{3.969999in}{0.714950in}}{\pgfqpoint{3.974389in}{0.704351in}}{\pgfqpoint{3.982203in}{0.696537in}}%
\pgfpathcurveto{\pgfqpoint{3.990016in}{0.688724in}}{\pgfqpoint{4.000616in}{0.684333in}}{\pgfqpoint{4.011666in}{0.684333in}}%
\pgfpathclose%
\pgfusepath{stroke,fill}%
\end{pgfscope}%
\begin{pgfscope}%
\pgfpathrectangle{\pgfqpoint{0.800000in}{0.528000in}}{\pgfqpoint{4.960000in}{3.696000in}}%
\pgfusepath{clip}%
\pgfsetbuttcap%
\pgfsetroundjoin%
\definecolor{currentfill}{rgb}{0.000000,0.000000,0.000000}%
\pgfsetfillcolor{currentfill}%
\pgfsetlinewidth{1.003750pt}%
\definecolor{currentstroke}{rgb}{0.000000,0.000000,0.000000}%
\pgfsetstrokecolor{currentstroke}%
\pgfsetdash{}{0pt}%
\pgfpathmoveto{\pgfqpoint{4.011666in}{0.684333in}}%
\pgfpathcurveto{\pgfqpoint{4.022716in}{0.684333in}}{\pgfqpoint{4.033315in}{0.688724in}}{\pgfqpoint{4.041128in}{0.696537in}}%
\pgfpathcurveto{\pgfqpoint{4.048942in}{0.704351in}}{\pgfqpoint{4.053332in}{0.714950in}}{\pgfqpoint{4.053332in}{0.726000in}}%
\pgfpathcurveto{\pgfqpoint{4.053332in}{0.737050in}}{\pgfqpoint{4.048942in}{0.747649in}}{\pgfqpoint{4.041128in}{0.755463in}}%
\pgfpathcurveto{\pgfqpoint{4.033315in}{0.763276in}}{\pgfqpoint{4.022716in}{0.767667in}}{\pgfqpoint{4.011666in}{0.767667in}}%
\pgfpathcurveto{\pgfqpoint{4.000616in}{0.767667in}}{\pgfqpoint{3.990016in}{0.763276in}}{\pgfqpoint{3.982203in}{0.755463in}}%
\pgfpathcurveto{\pgfqpoint{3.974389in}{0.747649in}}{\pgfqpoint{3.969999in}{0.737050in}}{\pgfqpoint{3.969999in}{0.726000in}}%
\pgfpathcurveto{\pgfqpoint{3.969999in}{0.714950in}}{\pgfqpoint{3.974389in}{0.704351in}}{\pgfqpoint{3.982203in}{0.696537in}}%
\pgfpathcurveto{\pgfqpoint{3.990016in}{0.688724in}}{\pgfqpoint{4.000616in}{0.684333in}}{\pgfqpoint{4.011666in}{0.684333in}}%
\pgfpathclose%
\pgfusepath{stroke,fill}%
\end{pgfscope}%
\begin{pgfscope}%
\pgfpathrectangle{\pgfqpoint{0.800000in}{0.528000in}}{\pgfqpoint{4.960000in}{3.696000in}}%
\pgfusepath{clip}%
\pgfsetbuttcap%
\pgfsetroundjoin%
\definecolor{currentfill}{rgb}{0.000000,0.000000,0.000000}%
\pgfsetfillcolor{currentfill}%
\pgfsetlinewidth{1.003750pt}%
\definecolor{currentstroke}{rgb}{0.000000,0.000000,0.000000}%
\pgfsetstrokecolor{currentstroke}%
\pgfsetdash{}{0pt}%
\pgfpathmoveto{\pgfqpoint{4.011666in}{0.684333in}}%
\pgfpathcurveto{\pgfqpoint{4.022716in}{0.684333in}}{\pgfqpoint{4.033315in}{0.688724in}}{\pgfqpoint{4.041128in}{0.696537in}}%
\pgfpathcurveto{\pgfqpoint{4.048942in}{0.704351in}}{\pgfqpoint{4.053332in}{0.714950in}}{\pgfqpoint{4.053332in}{0.726000in}}%
\pgfpathcurveto{\pgfqpoint{4.053332in}{0.737050in}}{\pgfqpoint{4.048942in}{0.747649in}}{\pgfqpoint{4.041128in}{0.755463in}}%
\pgfpathcurveto{\pgfqpoint{4.033315in}{0.763276in}}{\pgfqpoint{4.022716in}{0.767667in}}{\pgfqpoint{4.011666in}{0.767667in}}%
\pgfpathcurveto{\pgfqpoint{4.000616in}{0.767667in}}{\pgfqpoint{3.990016in}{0.763276in}}{\pgfqpoint{3.982203in}{0.755463in}}%
\pgfpathcurveto{\pgfqpoint{3.974389in}{0.747649in}}{\pgfqpoint{3.969999in}{0.737050in}}{\pgfqpoint{3.969999in}{0.726000in}}%
\pgfpathcurveto{\pgfqpoint{3.969999in}{0.714950in}}{\pgfqpoint{3.974389in}{0.704351in}}{\pgfqpoint{3.982203in}{0.696537in}}%
\pgfpathcurveto{\pgfqpoint{3.990016in}{0.688724in}}{\pgfqpoint{4.000616in}{0.684333in}}{\pgfqpoint{4.011666in}{0.684333in}}%
\pgfpathclose%
\pgfusepath{stroke,fill}%
\end{pgfscope}%
\begin{pgfscope}%
\pgfpathrectangle{\pgfqpoint{0.800000in}{0.528000in}}{\pgfqpoint{4.960000in}{3.696000in}}%
\pgfusepath{clip}%
\pgfsetbuttcap%
\pgfsetroundjoin%
\definecolor{currentfill}{rgb}{0.000000,0.000000,0.000000}%
\pgfsetfillcolor{currentfill}%
\pgfsetlinewidth{1.003750pt}%
\definecolor{currentstroke}{rgb}{0.000000,0.000000,0.000000}%
\pgfsetstrokecolor{currentstroke}%
\pgfsetdash{}{0pt}%
\pgfpathmoveto{\pgfqpoint{4.011666in}{0.684333in}}%
\pgfpathcurveto{\pgfqpoint{4.022716in}{0.684333in}}{\pgfqpoint{4.033315in}{0.688724in}}{\pgfqpoint{4.041128in}{0.696537in}}%
\pgfpathcurveto{\pgfqpoint{4.048942in}{0.704351in}}{\pgfqpoint{4.053332in}{0.714950in}}{\pgfqpoint{4.053332in}{0.726000in}}%
\pgfpathcurveto{\pgfqpoint{4.053332in}{0.737050in}}{\pgfqpoint{4.048942in}{0.747649in}}{\pgfqpoint{4.041128in}{0.755463in}}%
\pgfpathcurveto{\pgfqpoint{4.033315in}{0.763276in}}{\pgfqpoint{4.022716in}{0.767667in}}{\pgfqpoint{4.011666in}{0.767667in}}%
\pgfpathcurveto{\pgfqpoint{4.000616in}{0.767667in}}{\pgfqpoint{3.990016in}{0.763276in}}{\pgfqpoint{3.982203in}{0.755463in}}%
\pgfpathcurveto{\pgfqpoint{3.974389in}{0.747649in}}{\pgfqpoint{3.969999in}{0.737050in}}{\pgfqpoint{3.969999in}{0.726000in}}%
\pgfpathcurveto{\pgfqpoint{3.969999in}{0.714950in}}{\pgfqpoint{3.974389in}{0.704351in}}{\pgfqpoint{3.982203in}{0.696537in}}%
\pgfpathcurveto{\pgfqpoint{3.990016in}{0.688724in}}{\pgfqpoint{4.000616in}{0.684333in}}{\pgfqpoint{4.011666in}{0.684333in}}%
\pgfpathclose%
\pgfusepath{stroke,fill}%
\end{pgfscope}%
\begin{pgfscope}%
\pgfpathrectangle{\pgfqpoint{0.800000in}{0.528000in}}{\pgfqpoint{4.960000in}{3.696000in}}%
\pgfusepath{clip}%
\pgfsetbuttcap%
\pgfsetroundjoin%
\definecolor{currentfill}{rgb}{0.000000,0.000000,0.000000}%
\pgfsetfillcolor{currentfill}%
\pgfsetlinewidth{1.003750pt}%
\definecolor{currentstroke}{rgb}{0.000000,0.000000,0.000000}%
\pgfsetstrokecolor{currentstroke}%
\pgfsetdash{}{0pt}%
\pgfpathmoveto{\pgfqpoint{4.011666in}{0.684333in}}%
\pgfpathcurveto{\pgfqpoint{4.022716in}{0.684333in}}{\pgfqpoint{4.033315in}{0.688724in}}{\pgfqpoint{4.041128in}{0.696537in}}%
\pgfpathcurveto{\pgfqpoint{4.048942in}{0.704351in}}{\pgfqpoint{4.053332in}{0.714950in}}{\pgfqpoint{4.053332in}{0.726000in}}%
\pgfpathcurveto{\pgfqpoint{4.053332in}{0.737050in}}{\pgfqpoint{4.048942in}{0.747649in}}{\pgfqpoint{4.041128in}{0.755463in}}%
\pgfpathcurveto{\pgfqpoint{4.033315in}{0.763276in}}{\pgfqpoint{4.022716in}{0.767667in}}{\pgfqpoint{4.011666in}{0.767667in}}%
\pgfpathcurveto{\pgfqpoint{4.000616in}{0.767667in}}{\pgfqpoint{3.990016in}{0.763276in}}{\pgfqpoint{3.982203in}{0.755463in}}%
\pgfpathcurveto{\pgfqpoint{3.974389in}{0.747649in}}{\pgfqpoint{3.969999in}{0.737050in}}{\pgfqpoint{3.969999in}{0.726000in}}%
\pgfpathcurveto{\pgfqpoint{3.969999in}{0.714950in}}{\pgfqpoint{3.974389in}{0.704351in}}{\pgfqpoint{3.982203in}{0.696537in}}%
\pgfpathcurveto{\pgfqpoint{3.990016in}{0.688724in}}{\pgfqpoint{4.000616in}{0.684333in}}{\pgfqpoint{4.011666in}{0.684333in}}%
\pgfpathclose%
\pgfusepath{stroke,fill}%
\end{pgfscope}%
\begin{pgfscope}%
\pgfpathrectangle{\pgfqpoint{0.800000in}{0.528000in}}{\pgfqpoint{4.960000in}{3.696000in}}%
\pgfusepath{clip}%
\pgfsetbuttcap%
\pgfsetroundjoin%
\definecolor{currentfill}{rgb}{0.000000,0.000000,0.000000}%
\pgfsetfillcolor{currentfill}%
\pgfsetlinewidth{1.003750pt}%
\definecolor{currentstroke}{rgb}{0.000000,0.000000,0.000000}%
\pgfsetstrokecolor{currentstroke}%
\pgfsetdash{}{0pt}%
\pgfpathmoveto{\pgfqpoint{4.011666in}{0.684333in}}%
\pgfpathcurveto{\pgfqpoint{4.022716in}{0.684333in}}{\pgfqpoint{4.033315in}{0.688724in}}{\pgfqpoint{4.041128in}{0.696537in}}%
\pgfpathcurveto{\pgfqpoint{4.048942in}{0.704351in}}{\pgfqpoint{4.053332in}{0.714950in}}{\pgfqpoint{4.053332in}{0.726000in}}%
\pgfpathcurveto{\pgfqpoint{4.053332in}{0.737050in}}{\pgfqpoint{4.048942in}{0.747649in}}{\pgfqpoint{4.041128in}{0.755463in}}%
\pgfpathcurveto{\pgfqpoint{4.033315in}{0.763276in}}{\pgfqpoint{4.022716in}{0.767667in}}{\pgfqpoint{4.011666in}{0.767667in}}%
\pgfpathcurveto{\pgfqpoint{4.000616in}{0.767667in}}{\pgfqpoint{3.990016in}{0.763276in}}{\pgfqpoint{3.982203in}{0.755463in}}%
\pgfpathcurveto{\pgfqpoint{3.974389in}{0.747649in}}{\pgfqpoint{3.969999in}{0.737050in}}{\pgfqpoint{3.969999in}{0.726000in}}%
\pgfpathcurveto{\pgfqpoint{3.969999in}{0.714950in}}{\pgfqpoint{3.974389in}{0.704351in}}{\pgfqpoint{3.982203in}{0.696537in}}%
\pgfpathcurveto{\pgfqpoint{3.990016in}{0.688724in}}{\pgfqpoint{4.000616in}{0.684333in}}{\pgfqpoint{4.011666in}{0.684333in}}%
\pgfpathclose%
\pgfusepath{stroke,fill}%
\end{pgfscope}%
\begin{pgfscope}%
\pgfpathrectangle{\pgfqpoint{0.800000in}{0.528000in}}{\pgfqpoint{4.960000in}{3.696000in}}%
\pgfusepath{clip}%
\pgfsetbuttcap%
\pgfsetroundjoin%
\definecolor{currentfill}{rgb}{0.000000,0.000000,0.000000}%
\pgfsetfillcolor{currentfill}%
\pgfsetlinewidth{1.003750pt}%
\definecolor{currentstroke}{rgb}{0.000000,0.000000,0.000000}%
\pgfsetstrokecolor{currentstroke}%
\pgfsetdash{}{0pt}%
\pgfpathmoveto{\pgfqpoint{4.011666in}{0.684333in}}%
\pgfpathcurveto{\pgfqpoint{4.022716in}{0.684333in}}{\pgfqpoint{4.033315in}{0.688724in}}{\pgfqpoint{4.041128in}{0.696537in}}%
\pgfpathcurveto{\pgfqpoint{4.048942in}{0.704351in}}{\pgfqpoint{4.053332in}{0.714950in}}{\pgfqpoint{4.053332in}{0.726000in}}%
\pgfpathcurveto{\pgfqpoint{4.053332in}{0.737050in}}{\pgfqpoint{4.048942in}{0.747649in}}{\pgfqpoint{4.041128in}{0.755463in}}%
\pgfpathcurveto{\pgfqpoint{4.033315in}{0.763276in}}{\pgfqpoint{4.022716in}{0.767667in}}{\pgfqpoint{4.011666in}{0.767667in}}%
\pgfpathcurveto{\pgfqpoint{4.000616in}{0.767667in}}{\pgfqpoint{3.990016in}{0.763276in}}{\pgfqpoint{3.982203in}{0.755463in}}%
\pgfpathcurveto{\pgfqpoint{3.974389in}{0.747649in}}{\pgfqpoint{3.969999in}{0.737050in}}{\pgfqpoint{3.969999in}{0.726000in}}%
\pgfpathcurveto{\pgfqpoint{3.969999in}{0.714950in}}{\pgfqpoint{3.974389in}{0.704351in}}{\pgfqpoint{3.982203in}{0.696537in}}%
\pgfpathcurveto{\pgfqpoint{3.990016in}{0.688724in}}{\pgfqpoint{4.000616in}{0.684333in}}{\pgfqpoint{4.011666in}{0.684333in}}%
\pgfpathclose%
\pgfusepath{stroke,fill}%
\end{pgfscope}%
\begin{pgfscope}%
\pgfpathrectangle{\pgfqpoint{0.800000in}{0.528000in}}{\pgfqpoint{4.960000in}{3.696000in}}%
\pgfusepath{clip}%
\pgfsetbuttcap%
\pgfsetroundjoin%
\definecolor{currentfill}{rgb}{0.000000,0.000000,0.000000}%
\pgfsetfillcolor{currentfill}%
\pgfsetlinewidth{1.003750pt}%
\definecolor{currentstroke}{rgb}{0.000000,0.000000,0.000000}%
\pgfsetstrokecolor{currentstroke}%
\pgfsetdash{}{0pt}%
\pgfpathmoveto{\pgfqpoint{4.011666in}{0.684333in}}%
\pgfpathcurveto{\pgfqpoint{4.022716in}{0.684333in}}{\pgfqpoint{4.033315in}{0.688724in}}{\pgfqpoint{4.041128in}{0.696537in}}%
\pgfpathcurveto{\pgfqpoint{4.048942in}{0.704351in}}{\pgfqpoint{4.053332in}{0.714950in}}{\pgfqpoint{4.053332in}{0.726000in}}%
\pgfpathcurveto{\pgfqpoint{4.053332in}{0.737050in}}{\pgfqpoint{4.048942in}{0.747649in}}{\pgfqpoint{4.041128in}{0.755463in}}%
\pgfpathcurveto{\pgfqpoint{4.033315in}{0.763276in}}{\pgfqpoint{4.022716in}{0.767667in}}{\pgfqpoint{4.011666in}{0.767667in}}%
\pgfpathcurveto{\pgfqpoint{4.000616in}{0.767667in}}{\pgfqpoint{3.990016in}{0.763276in}}{\pgfqpoint{3.982203in}{0.755463in}}%
\pgfpathcurveto{\pgfqpoint{3.974389in}{0.747649in}}{\pgfqpoint{3.969999in}{0.737050in}}{\pgfqpoint{3.969999in}{0.726000in}}%
\pgfpathcurveto{\pgfqpoint{3.969999in}{0.714950in}}{\pgfqpoint{3.974389in}{0.704351in}}{\pgfqpoint{3.982203in}{0.696537in}}%
\pgfpathcurveto{\pgfqpoint{3.990016in}{0.688724in}}{\pgfqpoint{4.000616in}{0.684333in}}{\pgfqpoint{4.011666in}{0.684333in}}%
\pgfpathclose%
\pgfusepath{stroke,fill}%
\end{pgfscope}%
\begin{pgfscope}%
\pgfpathrectangle{\pgfqpoint{0.800000in}{0.528000in}}{\pgfqpoint{4.960000in}{3.696000in}}%
\pgfusepath{clip}%
\pgfsetbuttcap%
\pgfsetroundjoin%
\definecolor{currentfill}{rgb}{0.000000,0.000000,0.000000}%
\pgfsetfillcolor{currentfill}%
\pgfsetlinewidth{1.003750pt}%
\definecolor{currentstroke}{rgb}{0.000000,0.000000,0.000000}%
\pgfsetstrokecolor{currentstroke}%
\pgfsetdash{}{0pt}%
\pgfpathmoveto{\pgfqpoint{4.011666in}{0.684333in}}%
\pgfpathcurveto{\pgfqpoint{4.022716in}{0.684333in}}{\pgfqpoint{4.033315in}{0.688724in}}{\pgfqpoint{4.041128in}{0.696537in}}%
\pgfpathcurveto{\pgfqpoint{4.048942in}{0.704351in}}{\pgfqpoint{4.053332in}{0.714950in}}{\pgfqpoint{4.053332in}{0.726000in}}%
\pgfpathcurveto{\pgfqpoint{4.053332in}{0.737050in}}{\pgfqpoint{4.048942in}{0.747649in}}{\pgfqpoint{4.041128in}{0.755463in}}%
\pgfpathcurveto{\pgfqpoint{4.033315in}{0.763276in}}{\pgfqpoint{4.022716in}{0.767667in}}{\pgfqpoint{4.011666in}{0.767667in}}%
\pgfpathcurveto{\pgfqpoint{4.000616in}{0.767667in}}{\pgfqpoint{3.990016in}{0.763276in}}{\pgfqpoint{3.982203in}{0.755463in}}%
\pgfpathcurveto{\pgfqpoint{3.974389in}{0.747649in}}{\pgfqpoint{3.969999in}{0.737050in}}{\pgfqpoint{3.969999in}{0.726000in}}%
\pgfpathcurveto{\pgfqpoint{3.969999in}{0.714950in}}{\pgfqpoint{3.974389in}{0.704351in}}{\pgfqpoint{3.982203in}{0.696537in}}%
\pgfpathcurveto{\pgfqpoint{3.990016in}{0.688724in}}{\pgfqpoint{4.000616in}{0.684333in}}{\pgfqpoint{4.011666in}{0.684333in}}%
\pgfpathclose%
\pgfusepath{stroke,fill}%
\end{pgfscope}%
\begin{pgfscope}%
\pgfpathrectangle{\pgfqpoint{0.800000in}{0.528000in}}{\pgfqpoint{4.960000in}{3.696000in}}%
\pgfusepath{clip}%
\pgfsetbuttcap%
\pgfsetroundjoin%
\definecolor{currentfill}{rgb}{0.000000,0.000000,0.000000}%
\pgfsetfillcolor{currentfill}%
\pgfsetlinewidth{1.003750pt}%
\definecolor{currentstroke}{rgb}{0.000000,0.000000,0.000000}%
\pgfsetstrokecolor{currentstroke}%
\pgfsetdash{}{0pt}%
\pgfpathmoveto{\pgfqpoint{4.011666in}{0.684333in}}%
\pgfpathcurveto{\pgfqpoint{4.022716in}{0.684333in}}{\pgfqpoint{4.033315in}{0.688724in}}{\pgfqpoint{4.041128in}{0.696537in}}%
\pgfpathcurveto{\pgfqpoint{4.048942in}{0.704351in}}{\pgfqpoint{4.053332in}{0.714950in}}{\pgfqpoint{4.053332in}{0.726000in}}%
\pgfpathcurveto{\pgfqpoint{4.053332in}{0.737050in}}{\pgfqpoint{4.048942in}{0.747649in}}{\pgfqpoint{4.041128in}{0.755463in}}%
\pgfpathcurveto{\pgfqpoint{4.033315in}{0.763276in}}{\pgfqpoint{4.022716in}{0.767667in}}{\pgfqpoint{4.011666in}{0.767667in}}%
\pgfpathcurveto{\pgfqpoint{4.000616in}{0.767667in}}{\pgfqpoint{3.990016in}{0.763276in}}{\pgfqpoint{3.982203in}{0.755463in}}%
\pgfpathcurveto{\pgfqpoint{3.974389in}{0.747649in}}{\pgfqpoint{3.969999in}{0.737050in}}{\pgfqpoint{3.969999in}{0.726000in}}%
\pgfpathcurveto{\pgfqpoint{3.969999in}{0.714950in}}{\pgfqpoint{3.974389in}{0.704351in}}{\pgfqpoint{3.982203in}{0.696537in}}%
\pgfpathcurveto{\pgfqpoint{3.990016in}{0.688724in}}{\pgfqpoint{4.000616in}{0.684333in}}{\pgfqpoint{4.011666in}{0.684333in}}%
\pgfpathclose%
\pgfusepath{stroke,fill}%
\end{pgfscope}%
\begin{pgfscope}%
\pgfpathrectangle{\pgfqpoint{0.800000in}{0.528000in}}{\pgfqpoint{4.960000in}{3.696000in}}%
\pgfusepath{clip}%
\pgfsetbuttcap%
\pgfsetroundjoin%
\definecolor{currentfill}{rgb}{0.000000,0.000000,0.000000}%
\pgfsetfillcolor{currentfill}%
\pgfsetlinewidth{1.003750pt}%
\definecolor{currentstroke}{rgb}{0.000000,0.000000,0.000000}%
\pgfsetstrokecolor{currentstroke}%
\pgfsetdash{}{0pt}%
\pgfpathmoveto{\pgfqpoint{4.011666in}{0.684333in}}%
\pgfpathcurveto{\pgfqpoint{4.022716in}{0.684333in}}{\pgfqpoint{4.033315in}{0.688724in}}{\pgfqpoint{4.041128in}{0.696537in}}%
\pgfpathcurveto{\pgfqpoint{4.048942in}{0.704351in}}{\pgfqpoint{4.053332in}{0.714950in}}{\pgfqpoint{4.053332in}{0.726000in}}%
\pgfpathcurveto{\pgfqpoint{4.053332in}{0.737050in}}{\pgfqpoint{4.048942in}{0.747649in}}{\pgfqpoint{4.041128in}{0.755463in}}%
\pgfpathcurveto{\pgfqpoint{4.033315in}{0.763276in}}{\pgfqpoint{4.022716in}{0.767667in}}{\pgfqpoint{4.011666in}{0.767667in}}%
\pgfpathcurveto{\pgfqpoint{4.000616in}{0.767667in}}{\pgfqpoint{3.990016in}{0.763276in}}{\pgfqpoint{3.982203in}{0.755463in}}%
\pgfpathcurveto{\pgfqpoint{3.974389in}{0.747649in}}{\pgfqpoint{3.969999in}{0.737050in}}{\pgfqpoint{3.969999in}{0.726000in}}%
\pgfpathcurveto{\pgfqpoint{3.969999in}{0.714950in}}{\pgfqpoint{3.974389in}{0.704351in}}{\pgfqpoint{3.982203in}{0.696537in}}%
\pgfpathcurveto{\pgfqpoint{3.990016in}{0.688724in}}{\pgfqpoint{4.000616in}{0.684333in}}{\pgfqpoint{4.011666in}{0.684333in}}%
\pgfpathclose%
\pgfusepath{stroke,fill}%
\end{pgfscope}%
\begin{pgfscope}%
\pgfpathrectangle{\pgfqpoint{0.800000in}{0.528000in}}{\pgfqpoint{4.960000in}{3.696000in}}%
\pgfusepath{clip}%
\pgfsetbuttcap%
\pgfsetroundjoin%
\definecolor{currentfill}{rgb}{0.000000,0.000000,0.000000}%
\pgfsetfillcolor{currentfill}%
\pgfsetlinewidth{1.003750pt}%
\definecolor{currentstroke}{rgb}{0.000000,0.000000,0.000000}%
\pgfsetstrokecolor{currentstroke}%
\pgfsetdash{}{0pt}%
\pgfpathmoveto{\pgfqpoint{4.011666in}{0.684333in}}%
\pgfpathcurveto{\pgfqpoint{4.022716in}{0.684333in}}{\pgfqpoint{4.033315in}{0.688724in}}{\pgfqpoint{4.041128in}{0.696537in}}%
\pgfpathcurveto{\pgfqpoint{4.048942in}{0.704351in}}{\pgfqpoint{4.053332in}{0.714950in}}{\pgfqpoint{4.053332in}{0.726000in}}%
\pgfpathcurveto{\pgfqpoint{4.053332in}{0.737050in}}{\pgfqpoint{4.048942in}{0.747649in}}{\pgfqpoint{4.041128in}{0.755463in}}%
\pgfpathcurveto{\pgfqpoint{4.033315in}{0.763276in}}{\pgfqpoint{4.022716in}{0.767667in}}{\pgfqpoint{4.011666in}{0.767667in}}%
\pgfpathcurveto{\pgfqpoint{4.000616in}{0.767667in}}{\pgfqpoint{3.990016in}{0.763276in}}{\pgfqpoint{3.982203in}{0.755463in}}%
\pgfpathcurveto{\pgfqpoint{3.974389in}{0.747649in}}{\pgfqpoint{3.969999in}{0.737050in}}{\pgfqpoint{3.969999in}{0.726000in}}%
\pgfpathcurveto{\pgfqpoint{3.969999in}{0.714950in}}{\pgfqpoint{3.974389in}{0.704351in}}{\pgfqpoint{3.982203in}{0.696537in}}%
\pgfpathcurveto{\pgfqpoint{3.990016in}{0.688724in}}{\pgfqpoint{4.000616in}{0.684333in}}{\pgfqpoint{4.011666in}{0.684333in}}%
\pgfpathclose%
\pgfusepath{stroke,fill}%
\end{pgfscope}%
\begin{pgfscope}%
\pgfpathrectangle{\pgfqpoint{0.800000in}{0.528000in}}{\pgfqpoint{4.960000in}{3.696000in}}%
\pgfusepath{clip}%
\pgfsetbuttcap%
\pgfsetroundjoin%
\definecolor{currentfill}{rgb}{0.000000,0.000000,0.000000}%
\pgfsetfillcolor{currentfill}%
\pgfsetlinewidth{1.003750pt}%
\definecolor{currentstroke}{rgb}{0.000000,0.000000,0.000000}%
\pgfsetstrokecolor{currentstroke}%
\pgfsetdash{}{0pt}%
\pgfpathmoveto{\pgfqpoint{4.011666in}{0.684333in}}%
\pgfpathcurveto{\pgfqpoint{4.022716in}{0.684333in}}{\pgfqpoint{4.033315in}{0.688724in}}{\pgfqpoint{4.041128in}{0.696537in}}%
\pgfpathcurveto{\pgfqpoint{4.048942in}{0.704351in}}{\pgfqpoint{4.053332in}{0.714950in}}{\pgfqpoint{4.053332in}{0.726000in}}%
\pgfpathcurveto{\pgfqpoint{4.053332in}{0.737050in}}{\pgfqpoint{4.048942in}{0.747649in}}{\pgfqpoint{4.041128in}{0.755463in}}%
\pgfpathcurveto{\pgfqpoint{4.033315in}{0.763276in}}{\pgfqpoint{4.022716in}{0.767667in}}{\pgfqpoint{4.011666in}{0.767667in}}%
\pgfpathcurveto{\pgfqpoint{4.000616in}{0.767667in}}{\pgfqpoint{3.990016in}{0.763276in}}{\pgfqpoint{3.982203in}{0.755463in}}%
\pgfpathcurveto{\pgfqpoint{3.974389in}{0.747649in}}{\pgfqpoint{3.969999in}{0.737050in}}{\pgfqpoint{3.969999in}{0.726000in}}%
\pgfpathcurveto{\pgfqpoint{3.969999in}{0.714950in}}{\pgfqpoint{3.974389in}{0.704351in}}{\pgfqpoint{3.982203in}{0.696537in}}%
\pgfpathcurveto{\pgfqpoint{3.990016in}{0.688724in}}{\pgfqpoint{4.000616in}{0.684333in}}{\pgfqpoint{4.011666in}{0.684333in}}%
\pgfpathclose%
\pgfusepath{stroke,fill}%
\end{pgfscope}%
\begin{pgfscope}%
\pgfpathrectangle{\pgfqpoint{0.800000in}{0.528000in}}{\pgfqpoint{4.960000in}{3.696000in}}%
\pgfusepath{clip}%
\pgfsetbuttcap%
\pgfsetroundjoin%
\definecolor{currentfill}{rgb}{0.000000,0.000000,0.000000}%
\pgfsetfillcolor{currentfill}%
\pgfsetlinewidth{1.003750pt}%
\definecolor{currentstroke}{rgb}{0.000000,0.000000,0.000000}%
\pgfsetstrokecolor{currentstroke}%
\pgfsetdash{}{0pt}%
\pgfpathmoveto{\pgfqpoint{4.011666in}{0.684333in}}%
\pgfpathcurveto{\pgfqpoint{4.022716in}{0.684333in}}{\pgfqpoint{4.033315in}{0.688724in}}{\pgfqpoint{4.041128in}{0.696537in}}%
\pgfpathcurveto{\pgfqpoint{4.048942in}{0.704351in}}{\pgfqpoint{4.053332in}{0.714950in}}{\pgfqpoint{4.053332in}{0.726000in}}%
\pgfpathcurveto{\pgfqpoint{4.053332in}{0.737050in}}{\pgfqpoint{4.048942in}{0.747649in}}{\pgfqpoint{4.041128in}{0.755463in}}%
\pgfpathcurveto{\pgfqpoint{4.033315in}{0.763276in}}{\pgfqpoint{4.022716in}{0.767667in}}{\pgfqpoint{4.011666in}{0.767667in}}%
\pgfpathcurveto{\pgfqpoint{4.000616in}{0.767667in}}{\pgfqpoint{3.990016in}{0.763276in}}{\pgfqpoint{3.982203in}{0.755463in}}%
\pgfpathcurveto{\pgfqpoint{3.974389in}{0.747649in}}{\pgfqpoint{3.969999in}{0.737050in}}{\pgfqpoint{3.969999in}{0.726000in}}%
\pgfpathcurveto{\pgfqpoint{3.969999in}{0.714950in}}{\pgfqpoint{3.974389in}{0.704351in}}{\pgfqpoint{3.982203in}{0.696537in}}%
\pgfpathcurveto{\pgfqpoint{3.990016in}{0.688724in}}{\pgfqpoint{4.000616in}{0.684333in}}{\pgfqpoint{4.011666in}{0.684333in}}%
\pgfpathclose%
\pgfusepath{stroke,fill}%
\end{pgfscope}%
\begin{pgfscope}%
\pgfpathrectangle{\pgfqpoint{0.800000in}{0.528000in}}{\pgfqpoint{4.960000in}{3.696000in}}%
\pgfusepath{clip}%
\pgfsetbuttcap%
\pgfsetroundjoin%
\definecolor{currentfill}{rgb}{0.000000,0.000000,0.000000}%
\pgfsetfillcolor{currentfill}%
\pgfsetlinewidth{1.003750pt}%
\definecolor{currentstroke}{rgb}{0.000000,0.000000,0.000000}%
\pgfsetstrokecolor{currentstroke}%
\pgfsetdash{}{0pt}%
\pgfpathmoveto{\pgfqpoint{4.011666in}{0.684333in}}%
\pgfpathcurveto{\pgfqpoint{4.022716in}{0.684333in}}{\pgfqpoint{4.033315in}{0.688724in}}{\pgfqpoint{4.041128in}{0.696537in}}%
\pgfpathcurveto{\pgfqpoint{4.048942in}{0.704351in}}{\pgfqpoint{4.053332in}{0.714950in}}{\pgfqpoint{4.053332in}{0.726000in}}%
\pgfpathcurveto{\pgfqpoint{4.053332in}{0.737050in}}{\pgfqpoint{4.048942in}{0.747649in}}{\pgfqpoint{4.041128in}{0.755463in}}%
\pgfpathcurveto{\pgfqpoint{4.033315in}{0.763276in}}{\pgfqpoint{4.022716in}{0.767667in}}{\pgfqpoint{4.011666in}{0.767667in}}%
\pgfpathcurveto{\pgfqpoint{4.000616in}{0.767667in}}{\pgfqpoint{3.990016in}{0.763276in}}{\pgfqpoint{3.982203in}{0.755463in}}%
\pgfpathcurveto{\pgfqpoint{3.974389in}{0.747649in}}{\pgfqpoint{3.969999in}{0.737050in}}{\pgfqpoint{3.969999in}{0.726000in}}%
\pgfpathcurveto{\pgfqpoint{3.969999in}{0.714950in}}{\pgfqpoint{3.974389in}{0.704351in}}{\pgfqpoint{3.982203in}{0.696537in}}%
\pgfpathcurveto{\pgfqpoint{3.990016in}{0.688724in}}{\pgfqpoint{4.000616in}{0.684333in}}{\pgfqpoint{4.011666in}{0.684333in}}%
\pgfpathclose%
\pgfusepath{stroke,fill}%
\end{pgfscope}%
\begin{pgfscope}%
\pgfpathrectangle{\pgfqpoint{0.800000in}{0.528000in}}{\pgfqpoint{4.960000in}{3.696000in}}%
\pgfusepath{clip}%
\pgfsetbuttcap%
\pgfsetroundjoin%
\definecolor{currentfill}{rgb}{0.000000,0.000000,0.000000}%
\pgfsetfillcolor{currentfill}%
\pgfsetlinewidth{1.003750pt}%
\definecolor{currentstroke}{rgb}{0.000000,0.000000,0.000000}%
\pgfsetstrokecolor{currentstroke}%
\pgfsetdash{}{0pt}%
\pgfpathmoveto{\pgfqpoint{4.011666in}{0.684333in}}%
\pgfpathcurveto{\pgfqpoint{4.022716in}{0.684333in}}{\pgfqpoint{4.033315in}{0.688724in}}{\pgfqpoint{4.041128in}{0.696537in}}%
\pgfpathcurveto{\pgfqpoint{4.048942in}{0.704351in}}{\pgfqpoint{4.053332in}{0.714950in}}{\pgfqpoint{4.053332in}{0.726000in}}%
\pgfpathcurveto{\pgfqpoint{4.053332in}{0.737050in}}{\pgfqpoint{4.048942in}{0.747649in}}{\pgfqpoint{4.041128in}{0.755463in}}%
\pgfpathcurveto{\pgfqpoint{4.033315in}{0.763276in}}{\pgfqpoint{4.022716in}{0.767667in}}{\pgfqpoint{4.011666in}{0.767667in}}%
\pgfpathcurveto{\pgfqpoint{4.000616in}{0.767667in}}{\pgfqpoint{3.990016in}{0.763276in}}{\pgfqpoint{3.982203in}{0.755463in}}%
\pgfpathcurveto{\pgfqpoint{3.974389in}{0.747649in}}{\pgfqpoint{3.969999in}{0.737050in}}{\pgfqpoint{3.969999in}{0.726000in}}%
\pgfpathcurveto{\pgfqpoint{3.969999in}{0.714950in}}{\pgfqpoint{3.974389in}{0.704351in}}{\pgfqpoint{3.982203in}{0.696537in}}%
\pgfpathcurveto{\pgfqpoint{3.990016in}{0.688724in}}{\pgfqpoint{4.000616in}{0.684333in}}{\pgfqpoint{4.011666in}{0.684333in}}%
\pgfpathclose%
\pgfusepath{stroke,fill}%
\end{pgfscope}%
\begin{pgfscope}%
\pgfpathrectangle{\pgfqpoint{0.800000in}{0.528000in}}{\pgfqpoint{4.960000in}{3.696000in}}%
\pgfusepath{clip}%
\pgfsetbuttcap%
\pgfsetroundjoin%
\definecolor{currentfill}{rgb}{0.000000,0.000000,0.000000}%
\pgfsetfillcolor{currentfill}%
\pgfsetlinewidth{1.003750pt}%
\definecolor{currentstroke}{rgb}{0.000000,0.000000,0.000000}%
\pgfsetstrokecolor{currentstroke}%
\pgfsetdash{}{0pt}%
\pgfpathmoveto{\pgfqpoint{4.011666in}{0.684333in}}%
\pgfpathcurveto{\pgfqpoint{4.022716in}{0.684333in}}{\pgfqpoint{4.033315in}{0.688724in}}{\pgfqpoint{4.041128in}{0.696537in}}%
\pgfpathcurveto{\pgfqpoint{4.048942in}{0.704351in}}{\pgfqpoint{4.053332in}{0.714950in}}{\pgfqpoint{4.053332in}{0.726000in}}%
\pgfpathcurveto{\pgfqpoint{4.053332in}{0.737050in}}{\pgfqpoint{4.048942in}{0.747649in}}{\pgfqpoint{4.041128in}{0.755463in}}%
\pgfpathcurveto{\pgfqpoint{4.033315in}{0.763276in}}{\pgfqpoint{4.022716in}{0.767667in}}{\pgfqpoint{4.011666in}{0.767667in}}%
\pgfpathcurveto{\pgfqpoint{4.000616in}{0.767667in}}{\pgfqpoint{3.990016in}{0.763276in}}{\pgfqpoint{3.982203in}{0.755463in}}%
\pgfpathcurveto{\pgfqpoint{3.974389in}{0.747649in}}{\pgfqpoint{3.969999in}{0.737050in}}{\pgfqpoint{3.969999in}{0.726000in}}%
\pgfpathcurveto{\pgfqpoint{3.969999in}{0.714950in}}{\pgfqpoint{3.974389in}{0.704351in}}{\pgfqpoint{3.982203in}{0.696537in}}%
\pgfpathcurveto{\pgfqpoint{3.990016in}{0.688724in}}{\pgfqpoint{4.000616in}{0.684333in}}{\pgfqpoint{4.011666in}{0.684333in}}%
\pgfpathclose%
\pgfusepath{stroke,fill}%
\end{pgfscope}%
\begin{pgfscope}%
\pgfpathrectangle{\pgfqpoint{0.800000in}{0.528000in}}{\pgfqpoint{4.960000in}{3.696000in}}%
\pgfusepath{clip}%
\pgfsetbuttcap%
\pgfsetroundjoin%
\definecolor{currentfill}{rgb}{0.000000,0.000000,0.000000}%
\pgfsetfillcolor{currentfill}%
\pgfsetlinewidth{1.003750pt}%
\definecolor{currentstroke}{rgb}{0.000000,0.000000,0.000000}%
\pgfsetstrokecolor{currentstroke}%
\pgfsetdash{}{0pt}%
\pgfpathmoveto{\pgfqpoint{4.011666in}{0.684333in}}%
\pgfpathcurveto{\pgfqpoint{4.022716in}{0.684333in}}{\pgfqpoint{4.033315in}{0.688724in}}{\pgfqpoint{4.041128in}{0.696537in}}%
\pgfpathcurveto{\pgfqpoint{4.048942in}{0.704351in}}{\pgfqpoint{4.053332in}{0.714950in}}{\pgfqpoint{4.053332in}{0.726000in}}%
\pgfpathcurveto{\pgfqpoint{4.053332in}{0.737050in}}{\pgfqpoint{4.048942in}{0.747649in}}{\pgfqpoint{4.041128in}{0.755463in}}%
\pgfpathcurveto{\pgfqpoint{4.033315in}{0.763276in}}{\pgfqpoint{4.022716in}{0.767667in}}{\pgfqpoint{4.011666in}{0.767667in}}%
\pgfpathcurveto{\pgfqpoint{4.000616in}{0.767667in}}{\pgfqpoint{3.990016in}{0.763276in}}{\pgfqpoint{3.982203in}{0.755463in}}%
\pgfpathcurveto{\pgfqpoint{3.974389in}{0.747649in}}{\pgfqpoint{3.969999in}{0.737050in}}{\pgfqpoint{3.969999in}{0.726000in}}%
\pgfpathcurveto{\pgfqpoint{3.969999in}{0.714950in}}{\pgfqpoint{3.974389in}{0.704351in}}{\pgfqpoint{3.982203in}{0.696537in}}%
\pgfpathcurveto{\pgfqpoint{3.990016in}{0.688724in}}{\pgfqpoint{4.000616in}{0.684333in}}{\pgfqpoint{4.011666in}{0.684333in}}%
\pgfpathclose%
\pgfusepath{stroke,fill}%
\end{pgfscope}%
\begin{pgfscope}%
\pgfpathrectangle{\pgfqpoint{0.800000in}{0.528000in}}{\pgfqpoint{4.960000in}{3.696000in}}%
\pgfusepath{clip}%
\pgfsetbuttcap%
\pgfsetroundjoin%
\definecolor{currentfill}{rgb}{0.000000,0.000000,0.000000}%
\pgfsetfillcolor{currentfill}%
\pgfsetlinewidth{1.003750pt}%
\definecolor{currentstroke}{rgb}{0.000000,0.000000,0.000000}%
\pgfsetstrokecolor{currentstroke}%
\pgfsetdash{}{0pt}%
\pgfpathmoveto{\pgfqpoint{4.011666in}{0.684333in}}%
\pgfpathcurveto{\pgfqpoint{4.022716in}{0.684333in}}{\pgfqpoint{4.033315in}{0.688724in}}{\pgfqpoint{4.041128in}{0.696537in}}%
\pgfpathcurveto{\pgfqpoint{4.048942in}{0.704351in}}{\pgfqpoint{4.053332in}{0.714950in}}{\pgfqpoint{4.053332in}{0.726000in}}%
\pgfpathcurveto{\pgfqpoint{4.053332in}{0.737050in}}{\pgfqpoint{4.048942in}{0.747649in}}{\pgfqpoint{4.041128in}{0.755463in}}%
\pgfpathcurveto{\pgfqpoint{4.033315in}{0.763276in}}{\pgfqpoint{4.022716in}{0.767667in}}{\pgfqpoint{4.011666in}{0.767667in}}%
\pgfpathcurveto{\pgfqpoint{4.000616in}{0.767667in}}{\pgfqpoint{3.990016in}{0.763276in}}{\pgfqpoint{3.982203in}{0.755463in}}%
\pgfpathcurveto{\pgfqpoint{3.974389in}{0.747649in}}{\pgfqpoint{3.969999in}{0.737050in}}{\pgfqpoint{3.969999in}{0.726000in}}%
\pgfpathcurveto{\pgfqpoint{3.969999in}{0.714950in}}{\pgfqpoint{3.974389in}{0.704351in}}{\pgfqpoint{3.982203in}{0.696537in}}%
\pgfpathcurveto{\pgfqpoint{3.990016in}{0.688724in}}{\pgfqpoint{4.000616in}{0.684333in}}{\pgfqpoint{4.011666in}{0.684333in}}%
\pgfpathclose%
\pgfusepath{stroke,fill}%
\end{pgfscope}%
\begin{pgfscope}%
\pgfpathrectangle{\pgfqpoint{0.800000in}{0.528000in}}{\pgfqpoint{4.960000in}{3.696000in}}%
\pgfusepath{clip}%
\pgfsetbuttcap%
\pgfsetroundjoin%
\definecolor{currentfill}{rgb}{0.000000,0.000000,0.000000}%
\pgfsetfillcolor{currentfill}%
\pgfsetlinewidth{1.003750pt}%
\definecolor{currentstroke}{rgb}{0.000000,0.000000,0.000000}%
\pgfsetstrokecolor{currentstroke}%
\pgfsetdash{}{0pt}%
\pgfpathmoveto{\pgfqpoint{4.011666in}{0.684333in}}%
\pgfpathcurveto{\pgfqpoint{4.022716in}{0.684333in}}{\pgfqpoint{4.033315in}{0.688724in}}{\pgfqpoint{4.041128in}{0.696537in}}%
\pgfpathcurveto{\pgfqpoint{4.048942in}{0.704351in}}{\pgfqpoint{4.053332in}{0.714950in}}{\pgfqpoint{4.053332in}{0.726000in}}%
\pgfpathcurveto{\pgfqpoint{4.053332in}{0.737050in}}{\pgfqpoint{4.048942in}{0.747649in}}{\pgfqpoint{4.041128in}{0.755463in}}%
\pgfpathcurveto{\pgfqpoint{4.033315in}{0.763276in}}{\pgfqpoint{4.022716in}{0.767667in}}{\pgfqpoint{4.011666in}{0.767667in}}%
\pgfpathcurveto{\pgfqpoint{4.000616in}{0.767667in}}{\pgfqpoint{3.990016in}{0.763276in}}{\pgfqpoint{3.982203in}{0.755463in}}%
\pgfpathcurveto{\pgfqpoint{3.974389in}{0.747649in}}{\pgfqpoint{3.969999in}{0.737050in}}{\pgfqpoint{3.969999in}{0.726000in}}%
\pgfpathcurveto{\pgfqpoint{3.969999in}{0.714950in}}{\pgfqpoint{3.974389in}{0.704351in}}{\pgfqpoint{3.982203in}{0.696537in}}%
\pgfpathcurveto{\pgfqpoint{3.990016in}{0.688724in}}{\pgfqpoint{4.000616in}{0.684333in}}{\pgfqpoint{4.011666in}{0.684333in}}%
\pgfpathclose%
\pgfusepath{stroke,fill}%
\end{pgfscope}%
\begin{pgfscope}%
\pgfpathrectangle{\pgfqpoint{0.800000in}{0.528000in}}{\pgfqpoint{4.960000in}{3.696000in}}%
\pgfusepath{clip}%
\pgfsetbuttcap%
\pgfsetroundjoin%
\definecolor{currentfill}{rgb}{0.000000,0.000000,0.000000}%
\pgfsetfillcolor{currentfill}%
\pgfsetlinewidth{1.003750pt}%
\definecolor{currentstroke}{rgb}{0.000000,0.000000,0.000000}%
\pgfsetstrokecolor{currentstroke}%
\pgfsetdash{}{0pt}%
\pgfpathmoveto{\pgfqpoint{4.011666in}{0.684333in}}%
\pgfpathcurveto{\pgfqpoint{4.022716in}{0.684333in}}{\pgfqpoint{4.033315in}{0.688724in}}{\pgfqpoint{4.041128in}{0.696537in}}%
\pgfpathcurveto{\pgfqpoint{4.048942in}{0.704351in}}{\pgfqpoint{4.053332in}{0.714950in}}{\pgfqpoint{4.053332in}{0.726000in}}%
\pgfpathcurveto{\pgfqpoint{4.053332in}{0.737050in}}{\pgfqpoint{4.048942in}{0.747649in}}{\pgfqpoint{4.041128in}{0.755463in}}%
\pgfpathcurveto{\pgfqpoint{4.033315in}{0.763276in}}{\pgfqpoint{4.022716in}{0.767667in}}{\pgfqpoint{4.011666in}{0.767667in}}%
\pgfpathcurveto{\pgfqpoint{4.000616in}{0.767667in}}{\pgfqpoint{3.990016in}{0.763276in}}{\pgfqpoint{3.982203in}{0.755463in}}%
\pgfpathcurveto{\pgfqpoint{3.974389in}{0.747649in}}{\pgfqpoint{3.969999in}{0.737050in}}{\pgfqpoint{3.969999in}{0.726000in}}%
\pgfpathcurveto{\pgfqpoint{3.969999in}{0.714950in}}{\pgfqpoint{3.974389in}{0.704351in}}{\pgfqpoint{3.982203in}{0.696537in}}%
\pgfpathcurveto{\pgfqpoint{3.990016in}{0.688724in}}{\pgfqpoint{4.000616in}{0.684333in}}{\pgfqpoint{4.011666in}{0.684333in}}%
\pgfpathclose%
\pgfusepath{stroke,fill}%
\end{pgfscope}%
\begin{pgfscope}%
\pgfpathrectangle{\pgfqpoint{0.800000in}{0.528000in}}{\pgfqpoint{4.960000in}{3.696000in}}%
\pgfusepath{clip}%
\pgfsetbuttcap%
\pgfsetroundjoin%
\definecolor{currentfill}{rgb}{0.000000,0.000000,0.000000}%
\pgfsetfillcolor{currentfill}%
\pgfsetlinewidth{1.003750pt}%
\definecolor{currentstroke}{rgb}{0.000000,0.000000,0.000000}%
\pgfsetstrokecolor{currentstroke}%
\pgfsetdash{}{0pt}%
\pgfpathmoveto{\pgfqpoint{4.011666in}{0.684333in}}%
\pgfpathcurveto{\pgfqpoint{4.022716in}{0.684333in}}{\pgfqpoint{4.033315in}{0.688724in}}{\pgfqpoint{4.041128in}{0.696537in}}%
\pgfpathcurveto{\pgfqpoint{4.048942in}{0.704351in}}{\pgfqpoint{4.053332in}{0.714950in}}{\pgfqpoint{4.053332in}{0.726000in}}%
\pgfpathcurveto{\pgfqpoint{4.053332in}{0.737050in}}{\pgfqpoint{4.048942in}{0.747649in}}{\pgfqpoint{4.041128in}{0.755463in}}%
\pgfpathcurveto{\pgfqpoint{4.033315in}{0.763276in}}{\pgfqpoint{4.022716in}{0.767667in}}{\pgfqpoint{4.011666in}{0.767667in}}%
\pgfpathcurveto{\pgfqpoint{4.000616in}{0.767667in}}{\pgfqpoint{3.990016in}{0.763276in}}{\pgfqpoint{3.982203in}{0.755463in}}%
\pgfpathcurveto{\pgfqpoint{3.974389in}{0.747649in}}{\pgfqpoint{3.969999in}{0.737050in}}{\pgfqpoint{3.969999in}{0.726000in}}%
\pgfpathcurveto{\pgfqpoint{3.969999in}{0.714950in}}{\pgfqpoint{3.974389in}{0.704351in}}{\pgfqpoint{3.982203in}{0.696537in}}%
\pgfpathcurveto{\pgfqpoint{3.990016in}{0.688724in}}{\pgfqpoint{4.000616in}{0.684333in}}{\pgfqpoint{4.011666in}{0.684333in}}%
\pgfpathclose%
\pgfusepath{stroke,fill}%
\end{pgfscope}%
\begin{pgfscope}%
\pgfpathrectangle{\pgfqpoint{0.800000in}{0.528000in}}{\pgfqpoint{4.960000in}{3.696000in}}%
\pgfusepath{clip}%
\pgfsetbuttcap%
\pgfsetroundjoin%
\definecolor{currentfill}{rgb}{0.000000,0.000000,0.000000}%
\pgfsetfillcolor{currentfill}%
\pgfsetlinewidth{1.003750pt}%
\definecolor{currentstroke}{rgb}{0.000000,0.000000,0.000000}%
\pgfsetstrokecolor{currentstroke}%
\pgfsetdash{}{0pt}%
\pgfpathmoveto{\pgfqpoint{4.011666in}{0.684333in}}%
\pgfpathcurveto{\pgfqpoint{4.022716in}{0.684333in}}{\pgfqpoint{4.033315in}{0.688724in}}{\pgfqpoint{4.041128in}{0.696537in}}%
\pgfpathcurveto{\pgfqpoint{4.048942in}{0.704351in}}{\pgfqpoint{4.053332in}{0.714950in}}{\pgfqpoint{4.053332in}{0.726000in}}%
\pgfpathcurveto{\pgfqpoint{4.053332in}{0.737050in}}{\pgfqpoint{4.048942in}{0.747649in}}{\pgfqpoint{4.041128in}{0.755463in}}%
\pgfpathcurveto{\pgfqpoint{4.033315in}{0.763276in}}{\pgfqpoint{4.022716in}{0.767667in}}{\pgfqpoint{4.011666in}{0.767667in}}%
\pgfpathcurveto{\pgfqpoint{4.000616in}{0.767667in}}{\pgfqpoint{3.990016in}{0.763276in}}{\pgfqpoint{3.982203in}{0.755463in}}%
\pgfpathcurveto{\pgfqpoint{3.974389in}{0.747649in}}{\pgfqpoint{3.969999in}{0.737050in}}{\pgfqpoint{3.969999in}{0.726000in}}%
\pgfpathcurveto{\pgfqpoint{3.969999in}{0.714950in}}{\pgfqpoint{3.974389in}{0.704351in}}{\pgfqpoint{3.982203in}{0.696537in}}%
\pgfpathcurveto{\pgfqpoint{3.990016in}{0.688724in}}{\pgfqpoint{4.000616in}{0.684333in}}{\pgfqpoint{4.011666in}{0.684333in}}%
\pgfpathclose%
\pgfusepath{stroke,fill}%
\end{pgfscope}%
\begin{pgfscope}%
\pgfpathrectangle{\pgfqpoint{0.800000in}{0.528000in}}{\pgfqpoint{4.960000in}{3.696000in}}%
\pgfusepath{clip}%
\pgfsetbuttcap%
\pgfsetroundjoin%
\definecolor{currentfill}{rgb}{0.000000,0.000000,0.000000}%
\pgfsetfillcolor{currentfill}%
\pgfsetlinewidth{1.003750pt}%
\definecolor{currentstroke}{rgb}{0.000000,0.000000,0.000000}%
\pgfsetstrokecolor{currentstroke}%
\pgfsetdash{}{0pt}%
\pgfpathmoveto{\pgfqpoint{4.011666in}{0.684333in}}%
\pgfpathcurveto{\pgfqpoint{4.022716in}{0.684333in}}{\pgfqpoint{4.033315in}{0.688724in}}{\pgfqpoint{4.041128in}{0.696537in}}%
\pgfpathcurveto{\pgfqpoint{4.048942in}{0.704351in}}{\pgfqpoint{4.053332in}{0.714950in}}{\pgfqpoint{4.053332in}{0.726000in}}%
\pgfpathcurveto{\pgfqpoint{4.053332in}{0.737050in}}{\pgfqpoint{4.048942in}{0.747649in}}{\pgfqpoint{4.041128in}{0.755463in}}%
\pgfpathcurveto{\pgfqpoint{4.033315in}{0.763276in}}{\pgfqpoint{4.022716in}{0.767667in}}{\pgfqpoint{4.011666in}{0.767667in}}%
\pgfpathcurveto{\pgfqpoint{4.000616in}{0.767667in}}{\pgfqpoint{3.990016in}{0.763276in}}{\pgfqpoint{3.982203in}{0.755463in}}%
\pgfpathcurveto{\pgfqpoint{3.974389in}{0.747649in}}{\pgfqpoint{3.969999in}{0.737050in}}{\pgfqpoint{3.969999in}{0.726000in}}%
\pgfpathcurveto{\pgfqpoint{3.969999in}{0.714950in}}{\pgfqpoint{3.974389in}{0.704351in}}{\pgfqpoint{3.982203in}{0.696537in}}%
\pgfpathcurveto{\pgfqpoint{3.990016in}{0.688724in}}{\pgfqpoint{4.000616in}{0.684333in}}{\pgfqpoint{4.011666in}{0.684333in}}%
\pgfpathclose%
\pgfusepath{stroke,fill}%
\end{pgfscope}%
\begin{pgfscope}%
\pgfpathrectangle{\pgfqpoint{0.800000in}{0.528000in}}{\pgfqpoint{4.960000in}{3.696000in}}%
\pgfusepath{clip}%
\pgfsetbuttcap%
\pgfsetroundjoin%
\definecolor{currentfill}{rgb}{0.000000,0.000000,0.000000}%
\pgfsetfillcolor{currentfill}%
\pgfsetlinewidth{1.003750pt}%
\definecolor{currentstroke}{rgb}{0.000000,0.000000,0.000000}%
\pgfsetstrokecolor{currentstroke}%
\pgfsetdash{}{0pt}%
\pgfpathmoveto{\pgfqpoint{4.011666in}{0.684333in}}%
\pgfpathcurveto{\pgfqpoint{4.022716in}{0.684333in}}{\pgfqpoint{4.033315in}{0.688724in}}{\pgfqpoint{4.041128in}{0.696537in}}%
\pgfpathcurveto{\pgfqpoint{4.048942in}{0.704351in}}{\pgfqpoint{4.053332in}{0.714950in}}{\pgfqpoint{4.053332in}{0.726000in}}%
\pgfpathcurveto{\pgfqpoint{4.053332in}{0.737050in}}{\pgfqpoint{4.048942in}{0.747649in}}{\pgfqpoint{4.041128in}{0.755463in}}%
\pgfpathcurveto{\pgfqpoint{4.033315in}{0.763276in}}{\pgfqpoint{4.022716in}{0.767667in}}{\pgfqpoint{4.011666in}{0.767667in}}%
\pgfpathcurveto{\pgfqpoint{4.000616in}{0.767667in}}{\pgfqpoint{3.990016in}{0.763276in}}{\pgfqpoint{3.982203in}{0.755463in}}%
\pgfpathcurveto{\pgfqpoint{3.974389in}{0.747649in}}{\pgfqpoint{3.969999in}{0.737050in}}{\pgfqpoint{3.969999in}{0.726000in}}%
\pgfpathcurveto{\pgfqpoint{3.969999in}{0.714950in}}{\pgfqpoint{3.974389in}{0.704351in}}{\pgfqpoint{3.982203in}{0.696537in}}%
\pgfpathcurveto{\pgfqpoint{3.990016in}{0.688724in}}{\pgfqpoint{4.000616in}{0.684333in}}{\pgfqpoint{4.011666in}{0.684333in}}%
\pgfpathclose%
\pgfusepath{stroke,fill}%
\end{pgfscope}%
\begin{pgfscope}%
\pgfpathrectangle{\pgfqpoint{0.800000in}{0.528000in}}{\pgfqpoint{4.960000in}{3.696000in}}%
\pgfusepath{clip}%
\pgfsetbuttcap%
\pgfsetroundjoin%
\definecolor{currentfill}{rgb}{0.000000,0.000000,0.000000}%
\pgfsetfillcolor{currentfill}%
\pgfsetlinewidth{1.003750pt}%
\definecolor{currentstroke}{rgb}{0.000000,0.000000,0.000000}%
\pgfsetstrokecolor{currentstroke}%
\pgfsetdash{}{0pt}%
\pgfpathmoveto{\pgfqpoint{4.011666in}{0.684333in}}%
\pgfpathcurveto{\pgfqpoint{4.022716in}{0.684333in}}{\pgfqpoint{4.033315in}{0.688724in}}{\pgfqpoint{4.041128in}{0.696537in}}%
\pgfpathcurveto{\pgfqpoint{4.048942in}{0.704351in}}{\pgfqpoint{4.053332in}{0.714950in}}{\pgfqpoint{4.053332in}{0.726000in}}%
\pgfpathcurveto{\pgfqpoint{4.053332in}{0.737050in}}{\pgfqpoint{4.048942in}{0.747649in}}{\pgfqpoint{4.041128in}{0.755463in}}%
\pgfpathcurveto{\pgfqpoint{4.033315in}{0.763276in}}{\pgfqpoint{4.022716in}{0.767667in}}{\pgfqpoint{4.011666in}{0.767667in}}%
\pgfpathcurveto{\pgfqpoint{4.000616in}{0.767667in}}{\pgfqpoint{3.990016in}{0.763276in}}{\pgfqpoint{3.982203in}{0.755463in}}%
\pgfpathcurveto{\pgfqpoint{3.974389in}{0.747649in}}{\pgfqpoint{3.969999in}{0.737050in}}{\pgfqpoint{3.969999in}{0.726000in}}%
\pgfpathcurveto{\pgfqpoint{3.969999in}{0.714950in}}{\pgfqpoint{3.974389in}{0.704351in}}{\pgfqpoint{3.982203in}{0.696537in}}%
\pgfpathcurveto{\pgfqpoint{3.990016in}{0.688724in}}{\pgfqpoint{4.000616in}{0.684333in}}{\pgfqpoint{4.011666in}{0.684333in}}%
\pgfpathclose%
\pgfusepath{stroke,fill}%
\end{pgfscope}%
\begin{pgfscope}%
\pgfpathrectangle{\pgfqpoint{0.800000in}{0.528000in}}{\pgfqpoint{4.960000in}{3.696000in}}%
\pgfusepath{clip}%
\pgfsetbuttcap%
\pgfsetroundjoin%
\definecolor{currentfill}{rgb}{0.000000,0.000000,0.000000}%
\pgfsetfillcolor{currentfill}%
\pgfsetlinewidth{1.003750pt}%
\definecolor{currentstroke}{rgb}{0.000000,0.000000,0.000000}%
\pgfsetstrokecolor{currentstroke}%
\pgfsetdash{}{0pt}%
\pgfpathmoveto{\pgfqpoint{4.011666in}{0.684333in}}%
\pgfpathcurveto{\pgfqpoint{4.022716in}{0.684333in}}{\pgfqpoint{4.033315in}{0.688724in}}{\pgfqpoint{4.041128in}{0.696537in}}%
\pgfpathcurveto{\pgfqpoint{4.048942in}{0.704351in}}{\pgfqpoint{4.053332in}{0.714950in}}{\pgfqpoint{4.053332in}{0.726000in}}%
\pgfpathcurveto{\pgfqpoint{4.053332in}{0.737050in}}{\pgfqpoint{4.048942in}{0.747649in}}{\pgfqpoint{4.041128in}{0.755463in}}%
\pgfpathcurveto{\pgfqpoint{4.033315in}{0.763276in}}{\pgfqpoint{4.022716in}{0.767667in}}{\pgfqpoint{4.011666in}{0.767667in}}%
\pgfpathcurveto{\pgfqpoint{4.000616in}{0.767667in}}{\pgfqpoint{3.990016in}{0.763276in}}{\pgfqpoint{3.982203in}{0.755463in}}%
\pgfpathcurveto{\pgfqpoint{3.974389in}{0.747649in}}{\pgfqpoint{3.969999in}{0.737050in}}{\pgfqpoint{3.969999in}{0.726000in}}%
\pgfpathcurveto{\pgfqpoint{3.969999in}{0.714950in}}{\pgfqpoint{3.974389in}{0.704351in}}{\pgfqpoint{3.982203in}{0.696537in}}%
\pgfpathcurveto{\pgfqpoint{3.990016in}{0.688724in}}{\pgfqpoint{4.000616in}{0.684333in}}{\pgfqpoint{4.011666in}{0.684333in}}%
\pgfpathclose%
\pgfusepath{stroke,fill}%
\end{pgfscope}%
\begin{pgfscope}%
\pgfpathrectangle{\pgfqpoint{0.800000in}{0.528000in}}{\pgfqpoint{4.960000in}{3.696000in}}%
\pgfusepath{clip}%
\pgfsetbuttcap%
\pgfsetroundjoin%
\definecolor{currentfill}{rgb}{0.000000,0.000000,0.000000}%
\pgfsetfillcolor{currentfill}%
\pgfsetlinewidth{1.003750pt}%
\definecolor{currentstroke}{rgb}{0.000000,0.000000,0.000000}%
\pgfsetstrokecolor{currentstroke}%
\pgfsetdash{}{0pt}%
\pgfpathmoveto{\pgfqpoint{4.011666in}{0.684333in}}%
\pgfpathcurveto{\pgfqpoint{4.022716in}{0.684333in}}{\pgfqpoint{4.033315in}{0.688724in}}{\pgfqpoint{4.041128in}{0.696537in}}%
\pgfpathcurveto{\pgfqpoint{4.048942in}{0.704351in}}{\pgfqpoint{4.053332in}{0.714950in}}{\pgfqpoint{4.053332in}{0.726000in}}%
\pgfpathcurveto{\pgfqpoint{4.053332in}{0.737050in}}{\pgfqpoint{4.048942in}{0.747649in}}{\pgfqpoint{4.041128in}{0.755463in}}%
\pgfpathcurveto{\pgfqpoint{4.033315in}{0.763276in}}{\pgfqpoint{4.022716in}{0.767667in}}{\pgfqpoint{4.011666in}{0.767667in}}%
\pgfpathcurveto{\pgfqpoint{4.000616in}{0.767667in}}{\pgfqpoint{3.990016in}{0.763276in}}{\pgfqpoint{3.982203in}{0.755463in}}%
\pgfpathcurveto{\pgfqpoint{3.974389in}{0.747649in}}{\pgfqpoint{3.969999in}{0.737050in}}{\pgfqpoint{3.969999in}{0.726000in}}%
\pgfpathcurveto{\pgfqpoint{3.969999in}{0.714950in}}{\pgfqpoint{3.974389in}{0.704351in}}{\pgfqpoint{3.982203in}{0.696537in}}%
\pgfpathcurveto{\pgfqpoint{3.990016in}{0.688724in}}{\pgfqpoint{4.000616in}{0.684333in}}{\pgfqpoint{4.011666in}{0.684333in}}%
\pgfpathclose%
\pgfusepath{stroke,fill}%
\end{pgfscope}%
\begin{pgfscope}%
\pgfpathrectangle{\pgfqpoint{0.800000in}{0.528000in}}{\pgfqpoint{4.960000in}{3.696000in}}%
\pgfusepath{clip}%
\pgfsetbuttcap%
\pgfsetroundjoin%
\definecolor{currentfill}{rgb}{0.000000,0.000000,0.000000}%
\pgfsetfillcolor{currentfill}%
\pgfsetlinewidth{1.003750pt}%
\definecolor{currentstroke}{rgb}{0.000000,0.000000,0.000000}%
\pgfsetstrokecolor{currentstroke}%
\pgfsetdash{}{0pt}%
\pgfpathmoveto{\pgfqpoint{4.011666in}{0.684333in}}%
\pgfpathcurveto{\pgfqpoint{4.022716in}{0.684333in}}{\pgfqpoint{4.033315in}{0.688724in}}{\pgfqpoint{4.041128in}{0.696537in}}%
\pgfpathcurveto{\pgfqpoint{4.048942in}{0.704351in}}{\pgfqpoint{4.053332in}{0.714950in}}{\pgfqpoint{4.053332in}{0.726000in}}%
\pgfpathcurveto{\pgfqpoint{4.053332in}{0.737050in}}{\pgfqpoint{4.048942in}{0.747649in}}{\pgfqpoint{4.041128in}{0.755463in}}%
\pgfpathcurveto{\pgfqpoint{4.033315in}{0.763276in}}{\pgfqpoint{4.022716in}{0.767667in}}{\pgfqpoint{4.011666in}{0.767667in}}%
\pgfpathcurveto{\pgfqpoint{4.000616in}{0.767667in}}{\pgfqpoint{3.990016in}{0.763276in}}{\pgfqpoint{3.982203in}{0.755463in}}%
\pgfpathcurveto{\pgfqpoint{3.974389in}{0.747649in}}{\pgfqpoint{3.969999in}{0.737050in}}{\pgfqpoint{3.969999in}{0.726000in}}%
\pgfpathcurveto{\pgfqpoint{3.969999in}{0.714950in}}{\pgfqpoint{3.974389in}{0.704351in}}{\pgfqpoint{3.982203in}{0.696537in}}%
\pgfpathcurveto{\pgfqpoint{3.990016in}{0.688724in}}{\pgfqpoint{4.000616in}{0.684333in}}{\pgfqpoint{4.011666in}{0.684333in}}%
\pgfpathclose%
\pgfusepath{stroke,fill}%
\end{pgfscope}%
\begin{pgfscope}%
\pgfpathrectangle{\pgfqpoint{0.800000in}{0.528000in}}{\pgfqpoint{4.960000in}{3.696000in}}%
\pgfusepath{clip}%
\pgfsetbuttcap%
\pgfsetroundjoin%
\definecolor{currentfill}{rgb}{0.000000,0.000000,0.000000}%
\pgfsetfillcolor{currentfill}%
\pgfsetlinewidth{1.003750pt}%
\definecolor{currentstroke}{rgb}{0.000000,0.000000,0.000000}%
\pgfsetstrokecolor{currentstroke}%
\pgfsetdash{}{0pt}%
\pgfpathmoveto{\pgfqpoint{4.011666in}{0.684333in}}%
\pgfpathcurveto{\pgfqpoint{4.022716in}{0.684333in}}{\pgfqpoint{4.033315in}{0.688724in}}{\pgfqpoint{4.041128in}{0.696537in}}%
\pgfpathcurveto{\pgfqpoint{4.048942in}{0.704351in}}{\pgfqpoint{4.053332in}{0.714950in}}{\pgfqpoint{4.053332in}{0.726000in}}%
\pgfpathcurveto{\pgfqpoint{4.053332in}{0.737050in}}{\pgfqpoint{4.048942in}{0.747649in}}{\pgfqpoint{4.041128in}{0.755463in}}%
\pgfpathcurveto{\pgfqpoint{4.033315in}{0.763276in}}{\pgfqpoint{4.022716in}{0.767667in}}{\pgfqpoint{4.011666in}{0.767667in}}%
\pgfpathcurveto{\pgfqpoint{4.000616in}{0.767667in}}{\pgfqpoint{3.990016in}{0.763276in}}{\pgfqpoint{3.982203in}{0.755463in}}%
\pgfpathcurveto{\pgfqpoint{3.974389in}{0.747649in}}{\pgfqpoint{3.969999in}{0.737050in}}{\pgfqpoint{3.969999in}{0.726000in}}%
\pgfpathcurveto{\pgfqpoint{3.969999in}{0.714950in}}{\pgfqpoint{3.974389in}{0.704351in}}{\pgfqpoint{3.982203in}{0.696537in}}%
\pgfpathcurveto{\pgfqpoint{3.990016in}{0.688724in}}{\pgfqpoint{4.000616in}{0.684333in}}{\pgfqpoint{4.011666in}{0.684333in}}%
\pgfpathclose%
\pgfusepath{stroke,fill}%
\end{pgfscope}%
\begin{pgfscope}%
\pgfpathrectangle{\pgfqpoint{0.800000in}{0.528000in}}{\pgfqpoint{4.960000in}{3.696000in}}%
\pgfusepath{clip}%
\pgfsetbuttcap%
\pgfsetroundjoin%
\definecolor{currentfill}{rgb}{0.000000,0.000000,0.000000}%
\pgfsetfillcolor{currentfill}%
\pgfsetlinewidth{1.003750pt}%
\definecolor{currentstroke}{rgb}{0.000000,0.000000,0.000000}%
\pgfsetstrokecolor{currentstroke}%
\pgfsetdash{}{0pt}%
\pgfpathmoveto{\pgfqpoint{4.011666in}{0.684333in}}%
\pgfpathcurveto{\pgfqpoint{4.022716in}{0.684333in}}{\pgfqpoint{4.033315in}{0.688724in}}{\pgfqpoint{4.041128in}{0.696537in}}%
\pgfpathcurveto{\pgfqpoint{4.048942in}{0.704351in}}{\pgfqpoint{4.053332in}{0.714950in}}{\pgfqpoint{4.053332in}{0.726000in}}%
\pgfpathcurveto{\pgfqpoint{4.053332in}{0.737050in}}{\pgfqpoint{4.048942in}{0.747649in}}{\pgfqpoint{4.041128in}{0.755463in}}%
\pgfpathcurveto{\pgfqpoint{4.033315in}{0.763276in}}{\pgfqpoint{4.022716in}{0.767667in}}{\pgfqpoint{4.011666in}{0.767667in}}%
\pgfpathcurveto{\pgfqpoint{4.000616in}{0.767667in}}{\pgfqpoint{3.990016in}{0.763276in}}{\pgfqpoint{3.982203in}{0.755463in}}%
\pgfpathcurveto{\pgfqpoint{3.974389in}{0.747649in}}{\pgfqpoint{3.969999in}{0.737050in}}{\pgfqpoint{3.969999in}{0.726000in}}%
\pgfpathcurveto{\pgfqpoint{3.969999in}{0.714950in}}{\pgfqpoint{3.974389in}{0.704351in}}{\pgfqpoint{3.982203in}{0.696537in}}%
\pgfpathcurveto{\pgfqpoint{3.990016in}{0.688724in}}{\pgfqpoint{4.000616in}{0.684333in}}{\pgfqpoint{4.011666in}{0.684333in}}%
\pgfpathclose%
\pgfusepath{stroke,fill}%
\end{pgfscope}%
\begin{pgfscope}%
\pgfpathrectangle{\pgfqpoint{0.800000in}{0.528000in}}{\pgfqpoint{4.960000in}{3.696000in}}%
\pgfusepath{clip}%
\pgfsetbuttcap%
\pgfsetroundjoin%
\definecolor{currentfill}{rgb}{0.000000,0.000000,0.000000}%
\pgfsetfillcolor{currentfill}%
\pgfsetlinewidth{1.003750pt}%
\definecolor{currentstroke}{rgb}{0.000000,0.000000,0.000000}%
\pgfsetstrokecolor{currentstroke}%
\pgfsetdash{}{0pt}%
\pgfpathmoveto{\pgfqpoint{4.011666in}{0.684333in}}%
\pgfpathcurveto{\pgfqpoint{4.022716in}{0.684333in}}{\pgfqpoint{4.033315in}{0.688724in}}{\pgfqpoint{4.041128in}{0.696537in}}%
\pgfpathcurveto{\pgfqpoint{4.048942in}{0.704351in}}{\pgfqpoint{4.053332in}{0.714950in}}{\pgfqpoint{4.053332in}{0.726000in}}%
\pgfpathcurveto{\pgfqpoint{4.053332in}{0.737050in}}{\pgfqpoint{4.048942in}{0.747649in}}{\pgfqpoint{4.041128in}{0.755463in}}%
\pgfpathcurveto{\pgfqpoint{4.033315in}{0.763276in}}{\pgfqpoint{4.022716in}{0.767667in}}{\pgfqpoint{4.011666in}{0.767667in}}%
\pgfpathcurveto{\pgfqpoint{4.000616in}{0.767667in}}{\pgfqpoint{3.990016in}{0.763276in}}{\pgfqpoint{3.982203in}{0.755463in}}%
\pgfpathcurveto{\pgfqpoint{3.974389in}{0.747649in}}{\pgfqpoint{3.969999in}{0.737050in}}{\pgfqpoint{3.969999in}{0.726000in}}%
\pgfpathcurveto{\pgfqpoint{3.969999in}{0.714950in}}{\pgfqpoint{3.974389in}{0.704351in}}{\pgfqpoint{3.982203in}{0.696537in}}%
\pgfpathcurveto{\pgfqpoint{3.990016in}{0.688724in}}{\pgfqpoint{4.000616in}{0.684333in}}{\pgfqpoint{4.011666in}{0.684333in}}%
\pgfpathclose%
\pgfusepath{stroke,fill}%
\end{pgfscope}%
\begin{pgfscope}%
\pgfpathrectangle{\pgfqpoint{0.800000in}{0.528000in}}{\pgfqpoint{4.960000in}{3.696000in}}%
\pgfusepath{clip}%
\pgfsetbuttcap%
\pgfsetroundjoin%
\definecolor{currentfill}{rgb}{0.000000,0.000000,0.000000}%
\pgfsetfillcolor{currentfill}%
\pgfsetlinewidth{1.003750pt}%
\definecolor{currentstroke}{rgb}{0.000000,0.000000,0.000000}%
\pgfsetstrokecolor{currentstroke}%
\pgfsetdash{}{0pt}%
\pgfpathmoveto{\pgfqpoint{4.011666in}{0.684333in}}%
\pgfpathcurveto{\pgfqpoint{4.022716in}{0.684333in}}{\pgfqpoint{4.033315in}{0.688724in}}{\pgfqpoint{4.041128in}{0.696537in}}%
\pgfpathcurveto{\pgfqpoint{4.048942in}{0.704351in}}{\pgfqpoint{4.053332in}{0.714950in}}{\pgfqpoint{4.053332in}{0.726000in}}%
\pgfpathcurveto{\pgfqpoint{4.053332in}{0.737050in}}{\pgfqpoint{4.048942in}{0.747649in}}{\pgfqpoint{4.041128in}{0.755463in}}%
\pgfpathcurveto{\pgfqpoint{4.033315in}{0.763276in}}{\pgfqpoint{4.022716in}{0.767667in}}{\pgfqpoint{4.011666in}{0.767667in}}%
\pgfpathcurveto{\pgfqpoint{4.000616in}{0.767667in}}{\pgfqpoint{3.990016in}{0.763276in}}{\pgfqpoint{3.982203in}{0.755463in}}%
\pgfpathcurveto{\pgfqpoint{3.974389in}{0.747649in}}{\pgfqpoint{3.969999in}{0.737050in}}{\pgfqpoint{3.969999in}{0.726000in}}%
\pgfpathcurveto{\pgfqpoint{3.969999in}{0.714950in}}{\pgfqpoint{3.974389in}{0.704351in}}{\pgfqpoint{3.982203in}{0.696537in}}%
\pgfpathcurveto{\pgfqpoint{3.990016in}{0.688724in}}{\pgfqpoint{4.000616in}{0.684333in}}{\pgfqpoint{4.011666in}{0.684333in}}%
\pgfpathclose%
\pgfusepath{stroke,fill}%
\end{pgfscope}%
\begin{pgfscope}%
\pgfpathrectangle{\pgfqpoint{0.800000in}{0.528000in}}{\pgfqpoint{4.960000in}{3.696000in}}%
\pgfusepath{clip}%
\pgfsetbuttcap%
\pgfsetroundjoin%
\definecolor{currentfill}{rgb}{0.000000,0.000000,0.000000}%
\pgfsetfillcolor{currentfill}%
\pgfsetlinewidth{1.003750pt}%
\definecolor{currentstroke}{rgb}{0.000000,0.000000,0.000000}%
\pgfsetstrokecolor{currentstroke}%
\pgfsetdash{}{0pt}%
\pgfpathmoveto{\pgfqpoint{5.504545in}{0.684333in}}%
\pgfpathcurveto{\pgfqpoint{5.515596in}{0.684333in}}{\pgfqpoint{5.526195in}{0.688724in}}{\pgfqpoint{5.534008in}{0.696537in}}%
\pgfpathcurveto{\pgfqpoint{5.541822in}{0.704351in}}{\pgfqpoint{5.546212in}{0.714950in}}{\pgfqpoint{5.546212in}{0.726000in}}%
\pgfpathcurveto{\pgfqpoint{5.546212in}{0.737050in}}{\pgfqpoint{5.541822in}{0.747649in}}{\pgfqpoint{5.534008in}{0.755463in}}%
\pgfpathcurveto{\pgfqpoint{5.526195in}{0.763276in}}{\pgfqpoint{5.515596in}{0.767667in}}{\pgfqpoint{5.504545in}{0.767667in}}%
\pgfpathcurveto{\pgfqpoint{5.493495in}{0.767667in}}{\pgfqpoint{5.482896in}{0.763276in}}{\pgfqpoint{5.475083in}{0.755463in}}%
\pgfpathcurveto{\pgfqpoint{5.467269in}{0.747649in}}{\pgfqpoint{5.462879in}{0.737050in}}{\pgfqpoint{5.462879in}{0.726000in}}%
\pgfpathcurveto{\pgfqpoint{5.462879in}{0.714950in}}{\pgfqpoint{5.467269in}{0.704351in}}{\pgfqpoint{5.475083in}{0.696537in}}%
\pgfpathcurveto{\pgfqpoint{5.482896in}{0.688724in}}{\pgfqpoint{5.493495in}{0.684333in}}{\pgfqpoint{5.504545in}{0.684333in}}%
\pgfpathclose%
\pgfusepath{stroke,fill}%
\end{pgfscope}%
\begin{pgfscope}%
\pgfpathrectangle{\pgfqpoint{0.800000in}{0.528000in}}{\pgfqpoint{4.960000in}{3.696000in}}%
\pgfusepath{clip}%
\pgfsetbuttcap%
\pgfsetroundjoin%
\definecolor{currentfill}{rgb}{0.000000,0.000000,0.000000}%
\pgfsetfillcolor{currentfill}%
\pgfsetlinewidth{1.003750pt}%
\definecolor{currentstroke}{rgb}{0.000000,0.000000,0.000000}%
\pgfsetstrokecolor{currentstroke}%
\pgfsetdash{}{0pt}%
\pgfpathmoveto{\pgfqpoint{5.504545in}{0.684333in}}%
\pgfpathcurveto{\pgfqpoint{5.515596in}{0.684333in}}{\pgfqpoint{5.526195in}{0.688724in}}{\pgfqpoint{5.534008in}{0.696537in}}%
\pgfpathcurveto{\pgfqpoint{5.541822in}{0.704351in}}{\pgfqpoint{5.546212in}{0.714950in}}{\pgfqpoint{5.546212in}{0.726000in}}%
\pgfpathcurveto{\pgfqpoint{5.546212in}{0.737050in}}{\pgfqpoint{5.541822in}{0.747649in}}{\pgfqpoint{5.534008in}{0.755463in}}%
\pgfpathcurveto{\pgfqpoint{5.526195in}{0.763276in}}{\pgfqpoint{5.515596in}{0.767667in}}{\pgfqpoint{5.504545in}{0.767667in}}%
\pgfpathcurveto{\pgfqpoint{5.493495in}{0.767667in}}{\pgfqpoint{5.482896in}{0.763276in}}{\pgfqpoint{5.475083in}{0.755463in}}%
\pgfpathcurveto{\pgfqpoint{5.467269in}{0.747649in}}{\pgfqpoint{5.462879in}{0.737050in}}{\pgfqpoint{5.462879in}{0.726000in}}%
\pgfpathcurveto{\pgfqpoint{5.462879in}{0.714950in}}{\pgfqpoint{5.467269in}{0.704351in}}{\pgfqpoint{5.475083in}{0.696537in}}%
\pgfpathcurveto{\pgfqpoint{5.482896in}{0.688724in}}{\pgfqpoint{5.493495in}{0.684333in}}{\pgfqpoint{5.504545in}{0.684333in}}%
\pgfpathclose%
\pgfusepath{stroke,fill}%
\end{pgfscope}%
\begin{pgfscope}%
\pgfpathrectangle{\pgfqpoint{0.800000in}{0.528000in}}{\pgfqpoint{4.960000in}{3.696000in}}%
\pgfusepath{clip}%
\pgfsetbuttcap%
\pgfsetroundjoin%
\definecolor{currentfill}{rgb}{0.000000,0.000000,0.000000}%
\pgfsetfillcolor{currentfill}%
\pgfsetlinewidth{1.003750pt}%
\definecolor{currentstroke}{rgb}{0.000000,0.000000,0.000000}%
\pgfsetstrokecolor{currentstroke}%
\pgfsetdash{}{0pt}%
\pgfpathmoveto{\pgfqpoint{5.504545in}{0.684333in}}%
\pgfpathcurveto{\pgfqpoint{5.515596in}{0.684333in}}{\pgfqpoint{5.526195in}{0.688724in}}{\pgfqpoint{5.534008in}{0.696537in}}%
\pgfpathcurveto{\pgfqpoint{5.541822in}{0.704351in}}{\pgfqpoint{5.546212in}{0.714950in}}{\pgfqpoint{5.546212in}{0.726000in}}%
\pgfpathcurveto{\pgfqpoint{5.546212in}{0.737050in}}{\pgfqpoint{5.541822in}{0.747649in}}{\pgfqpoint{5.534008in}{0.755463in}}%
\pgfpathcurveto{\pgfqpoint{5.526195in}{0.763276in}}{\pgfqpoint{5.515596in}{0.767667in}}{\pgfqpoint{5.504545in}{0.767667in}}%
\pgfpathcurveto{\pgfqpoint{5.493495in}{0.767667in}}{\pgfqpoint{5.482896in}{0.763276in}}{\pgfqpoint{5.475083in}{0.755463in}}%
\pgfpathcurveto{\pgfqpoint{5.467269in}{0.747649in}}{\pgfqpoint{5.462879in}{0.737050in}}{\pgfqpoint{5.462879in}{0.726000in}}%
\pgfpathcurveto{\pgfqpoint{5.462879in}{0.714950in}}{\pgfqpoint{5.467269in}{0.704351in}}{\pgfqpoint{5.475083in}{0.696537in}}%
\pgfpathcurveto{\pgfqpoint{5.482896in}{0.688724in}}{\pgfqpoint{5.493495in}{0.684333in}}{\pgfqpoint{5.504545in}{0.684333in}}%
\pgfpathclose%
\pgfusepath{stroke,fill}%
\end{pgfscope}%
\begin{pgfscope}%
\pgfpathrectangle{\pgfqpoint{0.800000in}{0.528000in}}{\pgfqpoint{4.960000in}{3.696000in}}%
\pgfusepath{clip}%
\pgfsetbuttcap%
\pgfsetroundjoin%
\definecolor{currentfill}{rgb}{0.000000,0.000000,0.000000}%
\pgfsetfillcolor{currentfill}%
\pgfsetlinewidth{1.003750pt}%
\definecolor{currentstroke}{rgb}{0.000000,0.000000,0.000000}%
\pgfsetstrokecolor{currentstroke}%
\pgfsetdash{}{0pt}%
\pgfpathmoveto{\pgfqpoint{5.504545in}{0.684333in}}%
\pgfpathcurveto{\pgfqpoint{5.515596in}{0.684333in}}{\pgfqpoint{5.526195in}{0.688724in}}{\pgfqpoint{5.534008in}{0.696537in}}%
\pgfpathcurveto{\pgfqpoint{5.541822in}{0.704351in}}{\pgfqpoint{5.546212in}{0.714950in}}{\pgfqpoint{5.546212in}{0.726000in}}%
\pgfpathcurveto{\pgfqpoint{5.546212in}{0.737050in}}{\pgfqpoint{5.541822in}{0.747649in}}{\pgfqpoint{5.534008in}{0.755463in}}%
\pgfpathcurveto{\pgfqpoint{5.526195in}{0.763276in}}{\pgfqpoint{5.515596in}{0.767667in}}{\pgfqpoint{5.504545in}{0.767667in}}%
\pgfpathcurveto{\pgfqpoint{5.493495in}{0.767667in}}{\pgfqpoint{5.482896in}{0.763276in}}{\pgfqpoint{5.475083in}{0.755463in}}%
\pgfpathcurveto{\pgfqpoint{5.467269in}{0.747649in}}{\pgfqpoint{5.462879in}{0.737050in}}{\pgfqpoint{5.462879in}{0.726000in}}%
\pgfpathcurveto{\pgfqpoint{5.462879in}{0.714950in}}{\pgfqpoint{5.467269in}{0.704351in}}{\pgfqpoint{5.475083in}{0.696537in}}%
\pgfpathcurveto{\pgfqpoint{5.482896in}{0.688724in}}{\pgfqpoint{5.493495in}{0.684333in}}{\pgfqpoint{5.504545in}{0.684333in}}%
\pgfpathclose%
\pgfusepath{stroke,fill}%
\end{pgfscope}%
\begin{pgfscope}%
\pgfpathrectangle{\pgfqpoint{0.800000in}{0.528000in}}{\pgfqpoint{4.960000in}{3.696000in}}%
\pgfusepath{clip}%
\pgfsetbuttcap%
\pgfsetroundjoin%
\definecolor{currentfill}{rgb}{0.000000,0.000000,0.000000}%
\pgfsetfillcolor{currentfill}%
\pgfsetlinewidth{1.003750pt}%
\definecolor{currentstroke}{rgb}{0.000000,0.000000,0.000000}%
\pgfsetstrokecolor{currentstroke}%
\pgfsetdash{}{0pt}%
\pgfpathmoveto{\pgfqpoint{5.504545in}{0.684333in}}%
\pgfpathcurveto{\pgfqpoint{5.515596in}{0.684333in}}{\pgfqpoint{5.526195in}{0.688724in}}{\pgfqpoint{5.534008in}{0.696537in}}%
\pgfpathcurveto{\pgfqpoint{5.541822in}{0.704351in}}{\pgfqpoint{5.546212in}{0.714950in}}{\pgfqpoint{5.546212in}{0.726000in}}%
\pgfpathcurveto{\pgfqpoint{5.546212in}{0.737050in}}{\pgfqpoint{5.541822in}{0.747649in}}{\pgfqpoint{5.534008in}{0.755463in}}%
\pgfpathcurveto{\pgfqpoint{5.526195in}{0.763276in}}{\pgfqpoint{5.515596in}{0.767667in}}{\pgfqpoint{5.504545in}{0.767667in}}%
\pgfpathcurveto{\pgfqpoint{5.493495in}{0.767667in}}{\pgfqpoint{5.482896in}{0.763276in}}{\pgfqpoint{5.475083in}{0.755463in}}%
\pgfpathcurveto{\pgfqpoint{5.467269in}{0.747649in}}{\pgfqpoint{5.462879in}{0.737050in}}{\pgfqpoint{5.462879in}{0.726000in}}%
\pgfpathcurveto{\pgfqpoint{5.462879in}{0.714950in}}{\pgfqpoint{5.467269in}{0.704351in}}{\pgfqpoint{5.475083in}{0.696537in}}%
\pgfpathcurveto{\pgfqpoint{5.482896in}{0.688724in}}{\pgfqpoint{5.493495in}{0.684333in}}{\pgfqpoint{5.504545in}{0.684333in}}%
\pgfpathclose%
\pgfusepath{stroke,fill}%
\end{pgfscope}%
\begin{pgfscope}%
\pgfpathrectangle{\pgfqpoint{0.800000in}{0.528000in}}{\pgfqpoint{4.960000in}{3.696000in}}%
\pgfusepath{clip}%
\pgfsetbuttcap%
\pgfsetroundjoin%
\definecolor{currentfill}{rgb}{0.000000,0.000000,0.000000}%
\pgfsetfillcolor{currentfill}%
\pgfsetlinewidth{1.003750pt}%
\definecolor{currentstroke}{rgb}{0.000000,0.000000,0.000000}%
\pgfsetstrokecolor{currentstroke}%
\pgfsetdash{}{0pt}%
\pgfpathmoveto{\pgfqpoint{5.504545in}{0.684333in}}%
\pgfpathcurveto{\pgfqpoint{5.515596in}{0.684333in}}{\pgfqpoint{5.526195in}{0.688724in}}{\pgfqpoint{5.534008in}{0.696537in}}%
\pgfpathcurveto{\pgfqpoint{5.541822in}{0.704351in}}{\pgfqpoint{5.546212in}{0.714950in}}{\pgfqpoint{5.546212in}{0.726000in}}%
\pgfpathcurveto{\pgfqpoint{5.546212in}{0.737050in}}{\pgfqpoint{5.541822in}{0.747649in}}{\pgfqpoint{5.534008in}{0.755463in}}%
\pgfpathcurveto{\pgfqpoint{5.526195in}{0.763276in}}{\pgfqpoint{5.515596in}{0.767667in}}{\pgfqpoint{5.504545in}{0.767667in}}%
\pgfpathcurveto{\pgfqpoint{5.493495in}{0.767667in}}{\pgfqpoint{5.482896in}{0.763276in}}{\pgfqpoint{5.475083in}{0.755463in}}%
\pgfpathcurveto{\pgfqpoint{5.467269in}{0.747649in}}{\pgfqpoint{5.462879in}{0.737050in}}{\pgfqpoint{5.462879in}{0.726000in}}%
\pgfpathcurveto{\pgfqpoint{5.462879in}{0.714950in}}{\pgfqpoint{5.467269in}{0.704351in}}{\pgfqpoint{5.475083in}{0.696537in}}%
\pgfpathcurveto{\pgfqpoint{5.482896in}{0.688724in}}{\pgfqpoint{5.493495in}{0.684333in}}{\pgfqpoint{5.504545in}{0.684333in}}%
\pgfpathclose%
\pgfusepath{stroke,fill}%
\end{pgfscope}%
\begin{pgfscope}%
\pgfpathrectangle{\pgfqpoint{0.800000in}{0.528000in}}{\pgfqpoint{4.960000in}{3.696000in}}%
\pgfusepath{clip}%
\pgfsetbuttcap%
\pgfsetroundjoin%
\definecolor{currentfill}{rgb}{0.000000,0.000000,0.000000}%
\pgfsetfillcolor{currentfill}%
\pgfsetlinewidth{1.003750pt}%
\definecolor{currentstroke}{rgb}{0.000000,0.000000,0.000000}%
\pgfsetstrokecolor{currentstroke}%
\pgfsetdash{}{0pt}%
\pgfpathmoveto{\pgfqpoint{5.504545in}{0.684333in}}%
\pgfpathcurveto{\pgfqpoint{5.515596in}{0.684333in}}{\pgfqpoint{5.526195in}{0.688724in}}{\pgfqpoint{5.534008in}{0.696537in}}%
\pgfpathcurveto{\pgfqpoint{5.541822in}{0.704351in}}{\pgfqpoint{5.546212in}{0.714950in}}{\pgfqpoint{5.546212in}{0.726000in}}%
\pgfpathcurveto{\pgfqpoint{5.546212in}{0.737050in}}{\pgfqpoint{5.541822in}{0.747649in}}{\pgfqpoint{5.534008in}{0.755463in}}%
\pgfpathcurveto{\pgfqpoint{5.526195in}{0.763276in}}{\pgfqpoint{5.515596in}{0.767667in}}{\pgfqpoint{5.504545in}{0.767667in}}%
\pgfpathcurveto{\pgfqpoint{5.493495in}{0.767667in}}{\pgfqpoint{5.482896in}{0.763276in}}{\pgfqpoint{5.475083in}{0.755463in}}%
\pgfpathcurveto{\pgfqpoint{5.467269in}{0.747649in}}{\pgfqpoint{5.462879in}{0.737050in}}{\pgfqpoint{5.462879in}{0.726000in}}%
\pgfpathcurveto{\pgfqpoint{5.462879in}{0.714950in}}{\pgfqpoint{5.467269in}{0.704351in}}{\pgfqpoint{5.475083in}{0.696537in}}%
\pgfpathcurveto{\pgfqpoint{5.482896in}{0.688724in}}{\pgfqpoint{5.493495in}{0.684333in}}{\pgfqpoint{5.504545in}{0.684333in}}%
\pgfpathclose%
\pgfusepath{stroke,fill}%
\end{pgfscope}%
\begin{pgfscope}%
\pgfpathrectangle{\pgfqpoint{0.800000in}{0.528000in}}{\pgfqpoint{4.960000in}{3.696000in}}%
\pgfusepath{clip}%
\pgfsetbuttcap%
\pgfsetroundjoin%
\definecolor{currentfill}{rgb}{0.000000,0.000000,0.000000}%
\pgfsetfillcolor{currentfill}%
\pgfsetlinewidth{1.003750pt}%
\definecolor{currentstroke}{rgb}{0.000000,0.000000,0.000000}%
\pgfsetstrokecolor{currentstroke}%
\pgfsetdash{}{0pt}%
\pgfpathmoveto{\pgfqpoint{5.504545in}{0.684333in}}%
\pgfpathcurveto{\pgfqpoint{5.515596in}{0.684333in}}{\pgfqpoint{5.526195in}{0.688724in}}{\pgfqpoint{5.534008in}{0.696537in}}%
\pgfpathcurveto{\pgfqpoint{5.541822in}{0.704351in}}{\pgfqpoint{5.546212in}{0.714950in}}{\pgfqpoint{5.546212in}{0.726000in}}%
\pgfpathcurveto{\pgfqpoint{5.546212in}{0.737050in}}{\pgfqpoint{5.541822in}{0.747649in}}{\pgfqpoint{5.534008in}{0.755463in}}%
\pgfpathcurveto{\pgfqpoint{5.526195in}{0.763276in}}{\pgfqpoint{5.515596in}{0.767667in}}{\pgfqpoint{5.504545in}{0.767667in}}%
\pgfpathcurveto{\pgfqpoint{5.493495in}{0.767667in}}{\pgfqpoint{5.482896in}{0.763276in}}{\pgfqpoint{5.475083in}{0.755463in}}%
\pgfpathcurveto{\pgfqpoint{5.467269in}{0.747649in}}{\pgfqpoint{5.462879in}{0.737050in}}{\pgfqpoint{5.462879in}{0.726000in}}%
\pgfpathcurveto{\pgfqpoint{5.462879in}{0.714950in}}{\pgfqpoint{5.467269in}{0.704351in}}{\pgfqpoint{5.475083in}{0.696537in}}%
\pgfpathcurveto{\pgfqpoint{5.482896in}{0.688724in}}{\pgfqpoint{5.493495in}{0.684333in}}{\pgfqpoint{5.504545in}{0.684333in}}%
\pgfpathclose%
\pgfusepath{stroke,fill}%
\end{pgfscope}%
\begin{pgfscope}%
\pgfpathrectangle{\pgfqpoint{0.800000in}{0.528000in}}{\pgfqpoint{4.960000in}{3.696000in}}%
\pgfusepath{clip}%
\pgfsetbuttcap%
\pgfsetroundjoin%
\definecolor{currentfill}{rgb}{0.000000,0.000000,0.000000}%
\pgfsetfillcolor{currentfill}%
\pgfsetlinewidth{1.003750pt}%
\definecolor{currentstroke}{rgb}{0.000000,0.000000,0.000000}%
\pgfsetstrokecolor{currentstroke}%
\pgfsetdash{}{0pt}%
\pgfpathmoveto{\pgfqpoint{5.504545in}{0.684333in}}%
\pgfpathcurveto{\pgfqpoint{5.515596in}{0.684333in}}{\pgfqpoint{5.526195in}{0.688724in}}{\pgfqpoint{5.534008in}{0.696537in}}%
\pgfpathcurveto{\pgfqpoint{5.541822in}{0.704351in}}{\pgfqpoint{5.546212in}{0.714950in}}{\pgfqpoint{5.546212in}{0.726000in}}%
\pgfpathcurveto{\pgfqpoint{5.546212in}{0.737050in}}{\pgfqpoint{5.541822in}{0.747649in}}{\pgfqpoint{5.534008in}{0.755463in}}%
\pgfpathcurveto{\pgfqpoint{5.526195in}{0.763276in}}{\pgfqpoint{5.515596in}{0.767667in}}{\pgfqpoint{5.504545in}{0.767667in}}%
\pgfpathcurveto{\pgfqpoint{5.493495in}{0.767667in}}{\pgfqpoint{5.482896in}{0.763276in}}{\pgfqpoint{5.475083in}{0.755463in}}%
\pgfpathcurveto{\pgfqpoint{5.467269in}{0.747649in}}{\pgfqpoint{5.462879in}{0.737050in}}{\pgfqpoint{5.462879in}{0.726000in}}%
\pgfpathcurveto{\pgfqpoint{5.462879in}{0.714950in}}{\pgfqpoint{5.467269in}{0.704351in}}{\pgfqpoint{5.475083in}{0.696537in}}%
\pgfpathcurveto{\pgfqpoint{5.482896in}{0.688724in}}{\pgfqpoint{5.493495in}{0.684333in}}{\pgfqpoint{5.504545in}{0.684333in}}%
\pgfpathclose%
\pgfusepath{stroke,fill}%
\end{pgfscope}%
\begin{pgfscope}%
\pgfpathrectangle{\pgfqpoint{0.800000in}{0.528000in}}{\pgfqpoint{4.960000in}{3.696000in}}%
\pgfusepath{clip}%
\pgfsetbuttcap%
\pgfsetroundjoin%
\definecolor{currentfill}{rgb}{0.000000,0.000000,0.000000}%
\pgfsetfillcolor{currentfill}%
\pgfsetlinewidth{1.003750pt}%
\definecolor{currentstroke}{rgb}{0.000000,0.000000,0.000000}%
\pgfsetstrokecolor{currentstroke}%
\pgfsetdash{}{0pt}%
\pgfpathmoveto{\pgfqpoint{5.504545in}{0.684333in}}%
\pgfpathcurveto{\pgfqpoint{5.515596in}{0.684333in}}{\pgfqpoint{5.526195in}{0.688724in}}{\pgfqpoint{5.534008in}{0.696537in}}%
\pgfpathcurveto{\pgfqpoint{5.541822in}{0.704351in}}{\pgfqpoint{5.546212in}{0.714950in}}{\pgfqpoint{5.546212in}{0.726000in}}%
\pgfpathcurveto{\pgfqpoint{5.546212in}{0.737050in}}{\pgfqpoint{5.541822in}{0.747649in}}{\pgfqpoint{5.534008in}{0.755463in}}%
\pgfpathcurveto{\pgfqpoint{5.526195in}{0.763276in}}{\pgfqpoint{5.515596in}{0.767667in}}{\pgfqpoint{5.504545in}{0.767667in}}%
\pgfpathcurveto{\pgfqpoint{5.493495in}{0.767667in}}{\pgfqpoint{5.482896in}{0.763276in}}{\pgfqpoint{5.475083in}{0.755463in}}%
\pgfpathcurveto{\pgfqpoint{5.467269in}{0.747649in}}{\pgfqpoint{5.462879in}{0.737050in}}{\pgfqpoint{5.462879in}{0.726000in}}%
\pgfpathcurveto{\pgfqpoint{5.462879in}{0.714950in}}{\pgfqpoint{5.467269in}{0.704351in}}{\pgfqpoint{5.475083in}{0.696537in}}%
\pgfpathcurveto{\pgfqpoint{5.482896in}{0.688724in}}{\pgfqpoint{5.493495in}{0.684333in}}{\pgfqpoint{5.504545in}{0.684333in}}%
\pgfpathclose%
\pgfusepath{stroke,fill}%
\end{pgfscope}%
\begin{pgfscope}%
\pgfpathrectangle{\pgfqpoint{0.800000in}{0.528000in}}{\pgfqpoint{4.960000in}{3.696000in}}%
\pgfusepath{clip}%
\pgfsetbuttcap%
\pgfsetroundjoin%
\definecolor{currentfill}{rgb}{0.000000,0.000000,0.000000}%
\pgfsetfillcolor{currentfill}%
\pgfsetlinewidth{1.003750pt}%
\definecolor{currentstroke}{rgb}{0.000000,0.000000,0.000000}%
\pgfsetstrokecolor{currentstroke}%
\pgfsetdash{}{0pt}%
\pgfpathmoveto{\pgfqpoint{5.504545in}{0.684333in}}%
\pgfpathcurveto{\pgfqpoint{5.515596in}{0.684333in}}{\pgfqpoint{5.526195in}{0.688724in}}{\pgfqpoint{5.534008in}{0.696537in}}%
\pgfpathcurveto{\pgfqpoint{5.541822in}{0.704351in}}{\pgfqpoint{5.546212in}{0.714950in}}{\pgfqpoint{5.546212in}{0.726000in}}%
\pgfpathcurveto{\pgfqpoint{5.546212in}{0.737050in}}{\pgfqpoint{5.541822in}{0.747649in}}{\pgfqpoint{5.534008in}{0.755463in}}%
\pgfpathcurveto{\pgfqpoint{5.526195in}{0.763276in}}{\pgfqpoint{5.515596in}{0.767667in}}{\pgfqpoint{5.504545in}{0.767667in}}%
\pgfpathcurveto{\pgfqpoint{5.493495in}{0.767667in}}{\pgfqpoint{5.482896in}{0.763276in}}{\pgfqpoint{5.475083in}{0.755463in}}%
\pgfpathcurveto{\pgfqpoint{5.467269in}{0.747649in}}{\pgfqpoint{5.462879in}{0.737050in}}{\pgfqpoint{5.462879in}{0.726000in}}%
\pgfpathcurveto{\pgfqpoint{5.462879in}{0.714950in}}{\pgfqpoint{5.467269in}{0.704351in}}{\pgfqpoint{5.475083in}{0.696537in}}%
\pgfpathcurveto{\pgfqpoint{5.482896in}{0.688724in}}{\pgfqpoint{5.493495in}{0.684333in}}{\pgfqpoint{5.504545in}{0.684333in}}%
\pgfpathclose%
\pgfusepath{stroke,fill}%
\end{pgfscope}%
\begin{pgfscope}%
\pgfpathrectangle{\pgfqpoint{0.800000in}{0.528000in}}{\pgfqpoint{4.960000in}{3.696000in}}%
\pgfusepath{clip}%
\pgfsetbuttcap%
\pgfsetroundjoin%
\definecolor{currentfill}{rgb}{0.000000,0.000000,0.000000}%
\pgfsetfillcolor{currentfill}%
\pgfsetlinewidth{1.003750pt}%
\definecolor{currentstroke}{rgb}{0.000000,0.000000,0.000000}%
\pgfsetstrokecolor{currentstroke}%
\pgfsetdash{}{0pt}%
\pgfpathmoveto{\pgfqpoint{5.504545in}{0.684333in}}%
\pgfpathcurveto{\pgfqpoint{5.515596in}{0.684333in}}{\pgfqpoint{5.526195in}{0.688724in}}{\pgfqpoint{5.534008in}{0.696537in}}%
\pgfpathcurveto{\pgfqpoint{5.541822in}{0.704351in}}{\pgfqpoint{5.546212in}{0.714950in}}{\pgfqpoint{5.546212in}{0.726000in}}%
\pgfpathcurveto{\pgfqpoint{5.546212in}{0.737050in}}{\pgfqpoint{5.541822in}{0.747649in}}{\pgfqpoint{5.534008in}{0.755463in}}%
\pgfpathcurveto{\pgfqpoint{5.526195in}{0.763276in}}{\pgfqpoint{5.515596in}{0.767667in}}{\pgfqpoint{5.504545in}{0.767667in}}%
\pgfpathcurveto{\pgfqpoint{5.493495in}{0.767667in}}{\pgfqpoint{5.482896in}{0.763276in}}{\pgfqpoint{5.475083in}{0.755463in}}%
\pgfpathcurveto{\pgfqpoint{5.467269in}{0.747649in}}{\pgfqpoint{5.462879in}{0.737050in}}{\pgfqpoint{5.462879in}{0.726000in}}%
\pgfpathcurveto{\pgfqpoint{5.462879in}{0.714950in}}{\pgfqpoint{5.467269in}{0.704351in}}{\pgfqpoint{5.475083in}{0.696537in}}%
\pgfpathcurveto{\pgfqpoint{5.482896in}{0.688724in}}{\pgfqpoint{5.493495in}{0.684333in}}{\pgfqpoint{5.504545in}{0.684333in}}%
\pgfpathclose%
\pgfusepath{stroke,fill}%
\end{pgfscope}%
\begin{pgfscope}%
\pgfpathrectangle{\pgfqpoint{0.800000in}{0.528000in}}{\pgfqpoint{4.960000in}{3.696000in}}%
\pgfusepath{clip}%
\pgfsetbuttcap%
\pgfsetroundjoin%
\definecolor{currentfill}{rgb}{0.000000,0.000000,0.000000}%
\pgfsetfillcolor{currentfill}%
\pgfsetlinewidth{1.003750pt}%
\definecolor{currentstroke}{rgb}{0.000000,0.000000,0.000000}%
\pgfsetstrokecolor{currentstroke}%
\pgfsetdash{}{0pt}%
\pgfpathmoveto{\pgfqpoint{5.504545in}{0.684333in}}%
\pgfpathcurveto{\pgfqpoint{5.515596in}{0.684333in}}{\pgfqpoint{5.526195in}{0.688724in}}{\pgfqpoint{5.534008in}{0.696537in}}%
\pgfpathcurveto{\pgfqpoint{5.541822in}{0.704351in}}{\pgfqpoint{5.546212in}{0.714950in}}{\pgfqpoint{5.546212in}{0.726000in}}%
\pgfpathcurveto{\pgfqpoint{5.546212in}{0.737050in}}{\pgfqpoint{5.541822in}{0.747649in}}{\pgfqpoint{5.534008in}{0.755463in}}%
\pgfpathcurveto{\pgfqpoint{5.526195in}{0.763276in}}{\pgfqpoint{5.515596in}{0.767667in}}{\pgfqpoint{5.504545in}{0.767667in}}%
\pgfpathcurveto{\pgfqpoint{5.493495in}{0.767667in}}{\pgfqpoint{5.482896in}{0.763276in}}{\pgfqpoint{5.475083in}{0.755463in}}%
\pgfpathcurveto{\pgfqpoint{5.467269in}{0.747649in}}{\pgfqpoint{5.462879in}{0.737050in}}{\pgfqpoint{5.462879in}{0.726000in}}%
\pgfpathcurveto{\pgfqpoint{5.462879in}{0.714950in}}{\pgfqpoint{5.467269in}{0.704351in}}{\pgfqpoint{5.475083in}{0.696537in}}%
\pgfpathcurveto{\pgfqpoint{5.482896in}{0.688724in}}{\pgfqpoint{5.493495in}{0.684333in}}{\pgfqpoint{5.504545in}{0.684333in}}%
\pgfpathclose%
\pgfusepath{stroke,fill}%
\end{pgfscope}%
\begin{pgfscope}%
\pgfpathrectangle{\pgfqpoint{0.800000in}{0.528000in}}{\pgfqpoint{4.960000in}{3.696000in}}%
\pgfusepath{clip}%
\pgfsetbuttcap%
\pgfsetroundjoin%
\definecolor{currentfill}{rgb}{0.000000,0.000000,0.000000}%
\pgfsetfillcolor{currentfill}%
\pgfsetlinewidth{1.003750pt}%
\definecolor{currentstroke}{rgb}{0.000000,0.000000,0.000000}%
\pgfsetstrokecolor{currentstroke}%
\pgfsetdash{}{0pt}%
\pgfpathmoveto{\pgfqpoint{5.504545in}{0.684333in}}%
\pgfpathcurveto{\pgfqpoint{5.515596in}{0.684333in}}{\pgfqpoint{5.526195in}{0.688724in}}{\pgfqpoint{5.534008in}{0.696537in}}%
\pgfpathcurveto{\pgfqpoint{5.541822in}{0.704351in}}{\pgfqpoint{5.546212in}{0.714950in}}{\pgfqpoint{5.546212in}{0.726000in}}%
\pgfpathcurveto{\pgfqpoint{5.546212in}{0.737050in}}{\pgfqpoint{5.541822in}{0.747649in}}{\pgfqpoint{5.534008in}{0.755463in}}%
\pgfpathcurveto{\pgfqpoint{5.526195in}{0.763276in}}{\pgfqpoint{5.515596in}{0.767667in}}{\pgfqpoint{5.504545in}{0.767667in}}%
\pgfpathcurveto{\pgfqpoint{5.493495in}{0.767667in}}{\pgfqpoint{5.482896in}{0.763276in}}{\pgfqpoint{5.475083in}{0.755463in}}%
\pgfpathcurveto{\pgfqpoint{5.467269in}{0.747649in}}{\pgfqpoint{5.462879in}{0.737050in}}{\pgfqpoint{5.462879in}{0.726000in}}%
\pgfpathcurveto{\pgfqpoint{5.462879in}{0.714950in}}{\pgfqpoint{5.467269in}{0.704351in}}{\pgfqpoint{5.475083in}{0.696537in}}%
\pgfpathcurveto{\pgfqpoint{5.482896in}{0.688724in}}{\pgfqpoint{5.493495in}{0.684333in}}{\pgfqpoint{5.504545in}{0.684333in}}%
\pgfpathclose%
\pgfusepath{stroke,fill}%
\end{pgfscope}%
\begin{pgfscope}%
\pgfpathrectangle{\pgfqpoint{0.800000in}{0.528000in}}{\pgfqpoint{4.960000in}{3.696000in}}%
\pgfusepath{clip}%
\pgfsetbuttcap%
\pgfsetroundjoin%
\definecolor{currentfill}{rgb}{0.000000,0.000000,0.000000}%
\pgfsetfillcolor{currentfill}%
\pgfsetlinewidth{1.003750pt}%
\definecolor{currentstroke}{rgb}{0.000000,0.000000,0.000000}%
\pgfsetstrokecolor{currentstroke}%
\pgfsetdash{}{0pt}%
\pgfpathmoveto{\pgfqpoint{5.504545in}{0.684333in}}%
\pgfpathcurveto{\pgfqpoint{5.515596in}{0.684333in}}{\pgfqpoint{5.526195in}{0.688724in}}{\pgfqpoint{5.534008in}{0.696537in}}%
\pgfpathcurveto{\pgfqpoint{5.541822in}{0.704351in}}{\pgfqpoint{5.546212in}{0.714950in}}{\pgfqpoint{5.546212in}{0.726000in}}%
\pgfpathcurveto{\pgfqpoint{5.546212in}{0.737050in}}{\pgfqpoint{5.541822in}{0.747649in}}{\pgfqpoint{5.534008in}{0.755463in}}%
\pgfpathcurveto{\pgfqpoint{5.526195in}{0.763276in}}{\pgfqpoint{5.515596in}{0.767667in}}{\pgfqpoint{5.504545in}{0.767667in}}%
\pgfpathcurveto{\pgfqpoint{5.493495in}{0.767667in}}{\pgfqpoint{5.482896in}{0.763276in}}{\pgfqpoint{5.475083in}{0.755463in}}%
\pgfpathcurveto{\pgfqpoint{5.467269in}{0.747649in}}{\pgfqpoint{5.462879in}{0.737050in}}{\pgfqpoint{5.462879in}{0.726000in}}%
\pgfpathcurveto{\pgfqpoint{5.462879in}{0.714950in}}{\pgfqpoint{5.467269in}{0.704351in}}{\pgfqpoint{5.475083in}{0.696537in}}%
\pgfpathcurveto{\pgfqpoint{5.482896in}{0.688724in}}{\pgfqpoint{5.493495in}{0.684333in}}{\pgfqpoint{5.504545in}{0.684333in}}%
\pgfpathclose%
\pgfusepath{stroke,fill}%
\end{pgfscope}%
\begin{pgfscope}%
\pgfpathrectangle{\pgfqpoint{0.800000in}{0.528000in}}{\pgfqpoint{4.960000in}{3.696000in}}%
\pgfusepath{clip}%
\pgfsetbuttcap%
\pgfsetroundjoin%
\definecolor{currentfill}{rgb}{0.000000,0.000000,0.000000}%
\pgfsetfillcolor{currentfill}%
\pgfsetlinewidth{1.003750pt}%
\definecolor{currentstroke}{rgb}{0.000000,0.000000,0.000000}%
\pgfsetstrokecolor{currentstroke}%
\pgfsetdash{}{0pt}%
\pgfpathmoveto{\pgfqpoint{5.504545in}{0.684333in}}%
\pgfpathcurveto{\pgfqpoint{5.515596in}{0.684333in}}{\pgfqpoint{5.526195in}{0.688724in}}{\pgfqpoint{5.534008in}{0.696537in}}%
\pgfpathcurveto{\pgfqpoint{5.541822in}{0.704351in}}{\pgfqpoint{5.546212in}{0.714950in}}{\pgfqpoint{5.546212in}{0.726000in}}%
\pgfpathcurveto{\pgfqpoint{5.546212in}{0.737050in}}{\pgfqpoint{5.541822in}{0.747649in}}{\pgfqpoint{5.534008in}{0.755463in}}%
\pgfpathcurveto{\pgfqpoint{5.526195in}{0.763276in}}{\pgfqpoint{5.515596in}{0.767667in}}{\pgfqpoint{5.504545in}{0.767667in}}%
\pgfpathcurveto{\pgfqpoint{5.493495in}{0.767667in}}{\pgfqpoint{5.482896in}{0.763276in}}{\pgfqpoint{5.475083in}{0.755463in}}%
\pgfpathcurveto{\pgfqpoint{5.467269in}{0.747649in}}{\pgfqpoint{5.462879in}{0.737050in}}{\pgfqpoint{5.462879in}{0.726000in}}%
\pgfpathcurveto{\pgfqpoint{5.462879in}{0.714950in}}{\pgfqpoint{5.467269in}{0.704351in}}{\pgfqpoint{5.475083in}{0.696537in}}%
\pgfpathcurveto{\pgfqpoint{5.482896in}{0.688724in}}{\pgfqpoint{5.493495in}{0.684333in}}{\pgfqpoint{5.504545in}{0.684333in}}%
\pgfpathclose%
\pgfusepath{stroke,fill}%
\end{pgfscope}%
\begin{pgfscope}%
\pgfpathrectangle{\pgfqpoint{0.800000in}{0.528000in}}{\pgfqpoint{4.960000in}{3.696000in}}%
\pgfusepath{clip}%
\pgfsetbuttcap%
\pgfsetroundjoin%
\definecolor{currentfill}{rgb}{0.000000,0.000000,0.000000}%
\pgfsetfillcolor{currentfill}%
\pgfsetlinewidth{1.003750pt}%
\definecolor{currentstroke}{rgb}{0.000000,0.000000,0.000000}%
\pgfsetstrokecolor{currentstroke}%
\pgfsetdash{}{0pt}%
\pgfpathmoveto{\pgfqpoint{5.504545in}{0.684333in}}%
\pgfpathcurveto{\pgfqpoint{5.515596in}{0.684333in}}{\pgfqpoint{5.526195in}{0.688724in}}{\pgfqpoint{5.534008in}{0.696537in}}%
\pgfpathcurveto{\pgfqpoint{5.541822in}{0.704351in}}{\pgfqpoint{5.546212in}{0.714950in}}{\pgfqpoint{5.546212in}{0.726000in}}%
\pgfpathcurveto{\pgfqpoint{5.546212in}{0.737050in}}{\pgfqpoint{5.541822in}{0.747649in}}{\pgfqpoint{5.534008in}{0.755463in}}%
\pgfpathcurveto{\pgfqpoint{5.526195in}{0.763276in}}{\pgfqpoint{5.515596in}{0.767667in}}{\pgfqpoint{5.504545in}{0.767667in}}%
\pgfpathcurveto{\pgfqpoint{5.493495in}{0.767667in}}{\pgfqpoint{5.482896in}{0.763276in}}{\pgfqpoint{5.475083in}{0.755463in}}%
\pgfpathcurveto{\pgfqpoint{5.467269in}{0.747649in}}{\pgfqpoint{5.462879in}{0.737050in}}{\pgfqpoint{5.462879in}{0.726000in}}%
\pgfpathcurveto{\pgfqpoint{5.462879in}{0.714950in}}{\pgfqpoint{5.467269in}{0.704351in}}{\pgfqpoint{5.475083in}{0.696537in}}%
\pgfpathcurveto{\pgfqpoint{5.482896in}{0.688724in}}{\pgfqpoint{5.493495in}{0.684333in}}{\pgfqpoint{5.504545in}{0.684333in}}%
\pgfpathclose%
\pgfusepath{stroke,fill}%
\end{pgfscope}%
\begin{pgfscope}%
\pgfpathrectangle{\pgfqpoint{0.800000in}{0.528000in}}{\pgfqpoint{4.960000in}{3.696000in}}%
\pgfusepath{clip}%
\pgfsetbuttcap%
\pgfsetroundjoin%
\definecolor{currentfill}{rgb}{0.000000,0.000000,0.000000}%
\pgfsetfillcolor{currentfill}%
\pgfsetlinewidth{1.003750pt}%
\definecolor{currentstroke}{rgb}{0.000000,0.000000,0.000000}%
\pgfsetstrokecolor{currentstroke}%
\pgfsetdash{}{0pt}%
\pgfpathmoveto{\pgfqpoint{5.504545in}{0.684333in}}%
\pgfpathcurveto{\pgfqpoint{5.515596in}{0.684333in}}{\pgfqpoint{5.526195in}{0.688724in}}{\pgfqpoint{5.534008in}{0.696537in}}%
\pgfpathcurveto{\pgfqpoint{5.541822in}{0.704351in}}{\pgfqpoint{5.546212in}{0.714950in}}{\pgfqpoint{5.546212in}{0.726000in}}%
\pgfpathcurveto{\pgfqpoint{5.546212in}{0.737050in}}{\pgfqpoint{5.541822in}{0.747649in}}{\pgfqpoint{5.534008in}{0.755463in}}%
\pgfpathcurveto{\pgfqpoint{5.526195in}{0.763276in}}{\pgfqpoint{5.515596in}{0.767667in}}{\pgfqpoint{5.504545in}{0.767667in}}%
\pgfpathcurveto{\pgfqpoint{5.493495in}{0.767667in}}{\pgfqpoint{5.482896in}{0.763276in}}{\pgfqpoint{5.475083in}{0.755463in}}%
\pgfpathcurveto{\pgfqpoint{5.467269in}{0.747649in}}{\pgfqpoint{5.462879in}{0.737050in}}{\pgfqpoint{5.462879in}{0.726000in}}%
\pgfpathcurveto{\pgfqpoint{5.462879in}{0.714950in}}{\pgfqpoint{5.467269in}{0.704351in}}{\pgfqpoint{5.475083in}{0.696537in}}%
\pgfpathcurveto{\pgfqpoint{5.482896in}{0.688724in}}{\pgfqpoint{5.493495in}{0.684333in}}{\pgfqpoint{5.504545in}{0.684333in}}%
\pgfpathclose%
\pgfusepath{stroke,fill}%
\end{pgfscope}%
\begin{pgfscope}%
\pgfpathrectangle{\pgfqpoint{0.800000in}{0.528000in}}{\pgfqpoint{4.960000in}{3.696000in}}%
\pgfusepath{clip}%
\pgfsetbuttcap%
\pgfsetroundjoin%
\definecolor{currentfill}{rgb}{0.000000,0.000000,0.000000}%
\pgfsetfillcolor{currentfill}%
\pgfsetlinewidth{1.003750pt}%
\definecolor{currentstroke}{rgb}{0.000000,0.000000,0.000000}%
\pgfsetstrokecolor{currentstroke}%
\pgfsetdash{}{0pt}%
\pgfpathmoveto{\pgfqpoint{5.504545in}{0.684333in}}%
\pgfpathcurveto{\pgfqpoint{5.515596in}{0.684333in}}{\pgfqpoint{5.526195in}{0.688724in}}{\pgfqpoint{5.534008in}{0.696537in}}%
\pgfpathcurveto{\pgfqpoint{5.541822in}{0.704351in}}{\pgfqpoint{5.546212in}{0.714950in}}{\pgfqpoint{5.546212in}{0.726000in}}%
\pgfpathcurveto{\pgfqpoint{5.546212in}{0.737050in}}{\pgfqpoint{5.541822in}{0.747649in}}{\pgfqpoint{5.534008in}{0.755463in}}%
\pgfpathcurveto{\pgfqpoint{5.526195in}{0.763276in}}{\pgfqpoint{5.515596in}{0.767667in}}{\pgfqpoint{5.504545in}{0.767667in}}%
\pgfpathcurveto{\pgfqpoint{5.493495in}{0.767667in}}{\pgfqpoint{5.482896in}{0.763276in}}{\pgfqpoint{5.475083in}{0.755463in}}%
\pgfpathcurveto{\pgfqpoint{5.467269in}{0.747649in}}{\pgfqpoint{5.462879in}{0.737050in}}{\pgfqpoint{5.462879in}{0.726000in}}%
\pgfpathcurveto{\pgfqpoint{5.462879in}{0.714950in}}{\pgfqpoint{5.467269in}{0.704351in}}{\pgfqpoint{5.475083in}{0.696537in}}%
\pgfpathcurveto{\pgfqpoint{5.482896in}{0.688724in}}{\pgfqpoint{5.493495in}{0.684333in}}{\pgfqpoint{5.504545in}{0.684333in}}%
\pgfpathclose%
\pgfusepath{stroke,fill}%
\end{pgfscope}%
\begin{pgfscope}%
\pgfpathrectangle{\pgfqpoint{0.800000in}{0.528000in}}{\pgfqpoint{4.960000in}{3.696000in}}%
\pgfusepath{clip}%
\pgfsetbuttcap%
\pgfsetroundjoin%
\definecolor{currentfill}{rgb}{0.000000,0.000000,0.000000}%
\pgfsetfillcolor{currentfill}%
\pgfsetlinewidth{1.003750pt}%
\definecolor{currentstroke}{rgb}{0.000000,0.000000,0.000000}%
\pgfsetstrokecolor{currentstroke}%
\pgfsetdash{}{0pt}%
\pgfpathmoveto{\pgfqpoint{5.504545in}{0.684333in}}%
\pgfpathcurveto{\pgfqpoint{5.515596in}{0.684333in}}{\pgfqpoint{5.526195in}{0.688724in}}{\pgfqpoint{5.534008in}{0.696537in}}%
\pgfpathcurveto{\pgfqpoint{5.541822in}{0.704351in}}{\pgfqpoint{5.546212in}{0.714950in}}{\pgfqpoint{5.546212in}{0.726000in}}%
\pgfpathcurveto{\pgfqpoint{5.546212in}{0.737050in}}{\pgfqpoint{5.541822in}{0.747649in}}{\pgfqpoint{5.534008in}{0.755463in}}%
\pgfpathcurveto{\pgfqpoint{5.526195in}{0.763276in}}{\pgfqpoint{5.515596in}{0.767667in}}{\pgfqpoint{5.504545in}{0.767667in}}%
\pgfpathcurveto{\pgfqpoint{5.493495in}{0.767667in}}{\pgfqpoint{5.482896in}{0.763276in}}{\pgfqpoint{5.475083in}{0.755463in}}%
\pgfpathcurveto{\pgfqpoint{5.467269in}{0.747649in}}{\pgfqpoint{5.462879in}{0.737050in}}{\pgfqpoint{5.462879in}{0.726000in}}%
\pgfpathcurveto{\pgfqpoint{5.462879in}{0.714950in}}{\pgfqpoint{5.467269in}{0.704351in}}{\pgfqpoint{5.475083in}{0.696537in}}%
\pgfpathcurveto{\pgfqpoint{5.482896in}{0.688724in}}{\pgfqpoint{5.493495in}{0.684333in}}{\pgfqpoint{5.504545in}{0.684333in}}%
\pgfpathclose%
\pgfusepath{stroke,fill}%
\end{pgfscope}%
\begin{pgfscope}%
\pgfpathrectangle{\pgfqpoint{0.800000in}{0.528000in}}{\pgfqpoint{4.960000in}{3.696000in}}%
\pgfusepath{clip}%
\pgfsetbuttcap%
\pgfsetroundjoin%
\definecolor{currentfill}{rgb}{0.000000,0.000000,0.000000}%
\pgfsetfillcolor{currentfill}%
\pgfsetlinewidth{1.003750pt}%
\definecolor{currentstroke}{rgb}{0.000000,0.000000,0.000000}%
\pgfsetstrokecolor{currentstroke}%
\pgfsetdash{}{0pt}%
\pgfpathmoveto{\pgfqpoint{5.504545in}{0.684333in}}%
\pgfpathcurveto{\pgfqpoint{5.515596in}{0.684333in}}{\pgfqpoint{5.526195in}{0.688724in}}{\pgfqpoint{5.534008in}{0.696537in}}%
\pgfpathcurveto{\pgfqpoint{5.541822in}{0.704351in}}{\pgfqpoint{5.546212in}{0.714950in}}{\pgfqpoint{5.546212in}{0.726000in}}%
\pgfpathcurveto{\pgfqpoint{5.546212in}{0.737050in}}{\pgfqpoint{5.541822in}{0.747649in}}{\pgfqpoint{5.534008in}{0.755463in}}%
\pgfpathcurveto{\pgfqpoint{5.526195in}{0.763276in}}{\pgfqpoint{5.515596in}{0.767667in}}{\pgfqpoint{5.504545in}{0.767667in}}%
\pgfpathcurveto{\pgfqpoint{5.493495in}{0.767667in}}{\pgfqpoint{5.482896in}{0.763276in}}{\pgfqpoint{5.475083in}{0.755463in}}%
\pgfpathcurveto{\pgfqpoint{5.467269in}{0.747649in}}{\pgfqpoint{5.462879in}{0.737050in}}{\pgfqpoint{5.462879in}{0.726000in}}%
\pgfpathcurveto{\pgfqpoint{5.462879in}{0.714950in}}{\pgfqpoint{5.467269in}{0.704351in}}{\pgfqpoint{5.475083in}{0.696537in}}%
\pgfpathcurveto{\pgfqpoint{5.482896in}{0.688724in}}{\pgfqpoint{5.493495in}{0.684333in}}{\pgfqpoint{5.504545in}{0.684333in}}%
\pgfpathclose%
\pgfusepath{stroke,fill}%
\end{pgfscope}%
\begin{pgfscope}%
\pgfpathrectangle{\pgfqpoint{0.800000in}{0.528000in}}{\pgfqpoint{4.960000in}{3.696000in}}%
\pgfusepath{clip}%
\pgfsetbuttcap%
\pgfsetroundjoin%
\definecolor{currentfill}{rgb}{0.000000,0.000000,0.000000}%
\pgfsetfillcolor{currentfill}%
\pgfsetlinewidth{1.003750pt}%
\definecolor{currentstroke}{rgb}{0.000000,0.000000,0.000000}%
\pgfsetstrokecolor{currentstroke}%
\pgfsetdash{}{0pt}%
\pgfpathmoveto{\pgfqpoint{5.504545in}{0.684333in}}%
\pgfpathcurveto{\pgfqpoint{5.515596in}{0.684333in}}{\pgfqpoint{5.526195in}{0.688724in}}{\pgfqpoint{5.534008in}{0.696537in}}%
\pgfpathcurveto{\pgfqpoint{5.541822in}{0.704351in}}{\pgfqpoint{5.546212in}{0.714950in}}{\pgfqpoint{5.546212in}{0.726000in}}%
\pgfpathcurveto{\pgfqpoint{5.546212in}{0.737050in}}{\pgfqpoint{5.541822in}{0.747649in}}{\pgfqpoint{5.534008in}{0.755463in}}%
\pgfpathcurveto{\pgfqpoint{5.526195in}{0.763276in}}{\pgfqpoint{5.515596in}{0.767667in}}{\pgfqpoint{5.504545in}{0.767667in}}%
\pgfpathcurveto{\pgfqpoint{5.493495in}{0.767667in}}{\pgfqpoint{5.482896in}{0.763276in}}{\pgfqpoint{5.475083in}{0.755463in}}%
\pgfpathcurveto{\pgfqpoint{5.467269in}{0.747649in}}{\pgfqpoint{5.462879in}{0.737050in}}{\pgfqpoint{5.462879in}{0.726000in}}%
\pgfpathcurveto{\pgfqpoint{5.462879in}{0.714950in}}{\pgfqpoint{5.467269in}{0.704351in}}{\pgfqpoint{5.475083in}{0.696537in}}%
\pgfpathcurveto{\pgfqpoint{5.482896in}{0.688724in}}{\pgfqpoint{5.493495in}{0.684333in}}{\pgfqpoint{5.504545in}{0.684333in}}%
\pgfpathclose%
\pgfusepath{stroke,fill}%
\end{pgfscope}%
\begin{pgfscope}%
\pgfpathrectangle{\pgfqpoint{0.800000in}{0.528000in}}{\pgfqpoint{4.960000in}{3.696000in}}%
\pgfusepath{clip}%
\pgfsetbuttcap%
\pgfsetroundjoin%
\definecolor{currentfill}{rgb}{0.000000,0.000000,0.000000}%
\pgfsetfillcolor{currentfill}%
\pgfsetlinewidth{1.003750pt}%
\definecolor{currentstroke}{rgb}{0.000000,0.000000,0.000000}%
\pgfsetstrokecolor{currentstroke}%
\pgfsetdash{}{0pt}%
\pgfpathmoveto{\pgfqpoint{5.504545in}{0.684333in}}%
\pgfpathcurveto{\pgfqpoint{5.515596in}{0.684333in}}{\pgfqpoint{5.526195in}{0.688724in}}{\pgfqpoint{5.534008in}{0.696537in}}%
\pgfpathcurveto{\pgfqpoint{5.541822in}{0.704351in}}{\pgfqpoint{5.546212in}{0.714950in}}{\pgfqpoint{5.546212in}{0.726000in}}%
\pgfpathcurveto{\pgfqpoint{5.546212in}{0.737050in}}{\pgfqpoint{5.541822in}{0.747649in}}{\pgfqpoint{5.534008in}{0.755463in}}%
\pgfpathcurveto{\pgfqpoint{5.526195in}{0.763276in}}{\pgfqpoint{5.515596in}{0.767667in}}{\pgfqpoint{5.504545in}{0.767667in}}%
\pgfpathcurveto{\pgfqpoint{5.493495in}{0.767667in}}{\pgfqpoint{5.482896in}{0.763276in}}{\pgfqpoint{5.475083in}{0.755463in}}%
\pgfpathcurveto{\pgfqpoint{5.467269in}{0.747649in}}{\pgfqpoint{5.462879in}{0.737050in}}{\pgfqpoint{5.462879in}{0.726000in}}%
\pgfpathcurveto{\pgfqpoint{5.462879in}{0.714950in}}{\pgfqpoint{5.467269in}{0.704351in}}{\pgfqpoint{5.475083in}{0.696537in}}%
\pgfpathcurveto{\pgfqpoint{5.482896in}{0.688724in}}{\pgfqpoint{5.493495in}{0.684333in}}{\pgfqpoint{5.504545in}{0.684333in}}%
\pgfpathclose%
\pgfusepath{stroke,fill}%
\end{pgfscope}%
\begin{pgfscope}%
\pgfpathrectangle{\pgfqpoint{0.800000in}{0.528000in}}{\pgfqpoint{4.960000in}{3.696000in}}%
\pgfusepath{clip}%
\pgfsetbuttcap%
\pgfsetroundjoin%
\definecolor{currentfill}{rgb}{0.000000,0.000000,0.000000}%
\pgfsetfillcolor{currentfill}%
\pgfsetlinewidth{1.003750pt}%
\definecolor{currentstroke}{rgb}{0.000000,0.000000,0.000000}%
\pgfsetstrokecolor{currentstroke}%
\pgfsetdash{}{0pt}%
\pgfpathmoveto{\pgfqpoint{5.504545in}{0.684333in}}%
\pgfpathcurveto{\pgfqpoint{5.515596in}{0.684333in}}{\pgfqpoint{5.526195in}{0.688724in}}{\pgfqpoint{5.534008in}{0.696537in}}%
\pgfpathcurveto{\pgfqpoint{5.541822in}{0.704351in}}{\pgfqpoint{5.546212in}{0.714950in}}{\pgfqpoint{5.546212in}{0.726000in}}%
\pgfpathcurveto{\pgfqpoint{5.546212in}{0.737050in}}{\pgfqpoint{5.541822in}{0.747649in}}{\pgfqpoint{5.534008in}{0.755463in}}%
\pgfpathcurveto{\pgfqpoint{5.526195in}{0.763276in}}{\pgfqpoint{5.515596in}{0.767667in}}{\pgfqpoint{5.504545in}{0.767667in}}%
\pgfpathcurveto{\pgfqpoint{5.493495in}{0.767667in}}{\pgfqpoint{5.482896in}{0.763276in}}{\pgfqpoint{5.475083in}{0.755463in}}%
\pgfpathcurveto{\pgfqpoint{5.467269in}{0.747649in}}{\pgfqpoint{5.462879in}{0.737050in}}{\pgfqpoint{5.462879in}{0.726000in}}%
\pgfpathcurveto{\pgfqpoint{5.462879in}{0.714950in}}{\pgfqpoint{5.467269in}{0.704351in}}{\pgfqpoint{5.475083in}{0.696537in}}%
\pgfpathcurveto{\pgfqpoint{5.482896in}{0.688724in}}{\pgfqpoint{5.493495in}{0.684333in}}{\pgfqpoint{5.504545in}{0.684333in}}%
\pgfpathclose%
\pgfusepath{stroke,fill}%
\end{pgfscope}%
\begin{pgfscope}%
\pgfpathrectangle{\pgfqpoint{0.800000in}{0.528000in}}{\pgfqpoint{4.960000in}{3.696000in}}%
\pgfusepath{clip}%
\pgfsetbuttcap%
\pgfsetroundjoin%
\definecolor{currentfill}{rgb}{0.000000,0.000000,0.000000}%
\pgfsetfillcolor{currentfill}%
\pgfsetlinewidth{1.003750pt}%
\definecolor{currentstroke}{rgb}{0.000000,0.000000,0.000000}%
\pgfsetstrokecolor{currentstroke}%
\pgfsetdash{}{0pt}%
\pgfpathmoveto{\pgfqpoint{5.504545in}{0.684333in}}%
\pgfpathcurveto{\pgfqpoint{5.515596in}{0.684333in}}{\pgfqpoint{5.526195in}{0.688724in}}{\pgfqpoint{5.534008in}{0.696537in}}%
\pgfpathcurveto{\pgfqpoint{5.541822in}{0.704351in}}{\pgfqpoint{5.546212in}{0.714950in}}{\pgfqpoint{5.546212in}{0.726000in}}%
\pgfpathcurveto{\pgfqpoint{5.546212in}{0.737050in}}{\pgfqpoint{5.541822in}{0.747649in}}{\pgfqpoint{5.534008in}{0.755463in}}%
\pgfpathcurveto{\pgfqpoint{5.526195in}{0.763276in}}{\pgfqpoint{5.515596in}{0.767667in}}{\pgfqpoint{5.504545in}{0.767667in}}%
\pgfpathcurveto{\pgfqpoint{5.493495in}{0.767667in}}{\pgfqpoint{5.482896in}{0.763276in}}{\pgfqpoint{5.475083in}{0.755463in}}%
\pgfpathcurveto{\pgfqpoint{5.467269in}{0.747649in}}{\pgfqpoint{5.462879in}{0.737050in}}{\pgfqpoint{5.462879in}{0.726000in}}%
\pgfpathcurveto{\pgfqpoint{5.462879in}{0.714950in}}{\pgfqpoint{5.467269in}{0.704351in}}{\pgfqpoint{5.475083in}{0.696537in}}%
\pgfpathcurveto{\pgfqpoint{5.482896in}{0.688724in}}{\pgfqpoint{5.493495in}{0.684333in}}{\pgfqpoint{5.504545in}{0.684333in}}%
\pgfpathclose%
\pgfusepath{stroke,fill}%
\end{pgfscope}%
\begin{pgfscope}%
\pgfpathrectangle{\pgfqpoint{0.800000in}{0.528000in}}{\pgfqpoint{4.960000in}{3.696000in}}%
\pgfusepath{clip}%
\pgfsetbuttcap%
\pgfsetroundjoin%
\definecolor{currentfill}{rgb}{0.000000,0.000000,0.000000}%
\pgfsetfillcolor{currentfill}%
\pgfsetlinewidth{1.003750pt}%
\definecolor{currentstroke}{rgb}{0.000000,0.000000,0.000000}%
\pgfsetstrokecolor{currentstroke}%
\pgfsetdash{}{0pt}%
\pgfpathmoveto{\pgfqpoint{5.504545in}{0.684333in}}%
\pgfpathcurveto{\pgfqpoint{5.515596in}{0.684333in}}{\pgfqpoint{5.526195in}{0.688724in}}{\pgfqpoint{5.534008in}{0.696537in}}%
\pgfpathcurveto{\pgfqpoint{5.541822in}{0.704351in}}{\pgfqpoint{5.546212in}{0.714950in}}{\pgfqpoint{5.546212in}{0.726000in}}%
\pgfpathcurveto{\pgfqpoint{5.546212in}{0.737050in}}{\pgfqpoint{5.541822in}{0.747649in}}{\pgfqpoint{5.534008in}{0.755463in}}%
\pgfpathcurveto{\pgfqpoint{5.526195in}{0.763276in}}{\pgfqpoint{5.515596in}{0.767667in}}{\pgfqpoint{5.504545in}{0.767667in}}%
\pgfpathcurveto{\pgfqpoint{5.493495in}{0.767667in}}{\pgfqpoint{5.482896in}{0.763276in}}{\pgfqpoint{5.475083in}{0.755463in}}%
\pgfpathcurveto{\pgfqpoint{5.467269in}{0.747649in}}{\pgfqpoint{5.462879in}{0.737050in}}{\pgfqpoint{5.462879in}{0.726000in}}%
\pgfpathcurveto{\pgfqpoint{5.462879in}{0.714950in}}{\pgfqpoint{5.467269in}{0.704351in}}{\pgfqpoint{5.475083in}{0.696537in}}%
\pgfpathcurveto{\pgfqpoint{5.482896in}{0.688724in}}{\pgfqpoint{5.493495in}{0.684333in}}{\pgfqpoint{5.504545in}{0.684333in}}%
\pgfpathclose%
\pgfusepath{stroke,fill}%
\end{pgfscope}%
\begin{pgfscope}%
\pgfpathrectangle{\pgfqpoint{0.800000in}{0.528000in}}{\pgfqpoint{4.960000in}{3.696000in}}%
\pgfusepath{clip}%
\pgfsetbuttcap%
\pgfsetroundjoin%
\definecolor{currentfill}{rgb}{0.000000,0.000000,0.000000}%
\pgfsetfillcolor{currentfill}%
\pgfsetlinewidth{1.003750pt}%
\definecolor{currentstroke}{rgb}{0.000000,0.000000,0.000000}%
\pgfsetstrokecolor{currentstroke}%
\pgfsetdash{}{0pt}%
\pgfpathmoveto{\pgfqpoint{5.504545in}{0.684333in}}%
\pgfpathcurveto{\pgfqpoint{5.515596in}{0.684333in}}{\pgfqpoint{5.526195in}{0.688724in}}{\pgfqpoint{5.534008in}{0.696537in}}%
\pgfpathcurveto{\pgfqpoint{5.541822in}{0.704351in}}{\pgfqpoint{5.546212in}{0.714950in}}{\pgfqpoint{5.546212in}{0.726000in}}%
\pgfpathcurveto{\pgfqpoint{5.546212in}{0.737050in}}{\pgfqpoint{5.541822in}{0.747649in}}{\pgfqpoint{5.534008in}{0.755463in}}%
\pgfpathcurveto{\pgfqpoint{5.526195in}{0.763276in}}{\pgfqpoint{5.515596in}{0.767667in}}{\pgfqpoint{5.504545in}{0.767667in}}%
\pgfpathcurveto{\pgfqpoint{5.493495in}{0.767667in}}{\pgfqpoint{5.482896in}{0.763276in}}{\pgfqpoint{5.475083in}{0.755463in}}%
\pgfpathcurveto{\pgfqpoint{5.467269in}{0.747649in}}{\pgfqpoint{5.462879in}{0.737050in}}{\pgfqpoint{5.462879in}{0.726000in}}%
\pgfpathcurveto{\pgfqpoint{5.462879in}{0.714950in}}{\pgfqpoint{5.467269in}{0.704351in}}{\pgfqpoint{5.475083in}{0.696537in}}%
\pgfpathcurveto{\pgfqpoint{5.482896in}{0.688724in}}{\pgfqpoint{5.493495in}{0.684333in}}{\pgfqpoint{5.504545in}{0.684333in}}%
\pgfpathclose%
\pgfusepath{stroke,fill}%
\end{pgfscope}%
\begin{pgfscope}%
\pgfpathrectangle{\pgfqpoint{0.800000in}{0.528000in}}{\pgfqpoint{4.960000in}{3.696000in}}%
\pgfusepath{clip}%
\pgfsetbuttcap%
\pgfsetroundjoin%
\definecolor{currentfill}{rgb}{0.000000,0.000000,0.000000}%
\pgfsetfillcolor{currentfill}%
\pgfsetlinewidth{1.003750pt}%
\definecolor{currentstroke}{rgb}{0.000000,0.000000,0.000000}%
\pgfsetstrokecolor{currentstroke}%
\pgfsetdash{}{0pt}%
\pgfpathmoveto{\pgfqpoint{5.504545in}{0.684333in}}%
\pgfpathcurveto{\pgfqpoint{5.515596in}{0.684333in}}{\pgfqpoint{5.526195in}{0.688724in}}{\pgfqpoint{5.534008in}{0.696537in}}%
\pgfpathcurveto{\pgfqpoint{5.541822in}{0.704351in}}{\pgfqpoint{5.546212in}{0.714950in}}{\pgfqpoint{5.546212in}{0.726000in}}%
\pgfpathcurveto{\pgfqpoint{5.546212in}{0.737050in}}{\pgfqpoint{5.541822in}{0.747649in}}{\pgfqpoint{5.534008in}{0.755463in}}%
\pgfpathcurveto{\pgfqpoint{5.526195in}{0.763276in}}{\pgfqpoint{5.515596in}{0.767667in}}{\pgfqpoint{5.504545in}{0.767667in}}%
\pgfpathcurveto{\pgfqpoint{5.493495in}{0.767667in}}{\pgfqpoint{5.482896in}{0.763276in}}{\pgfqpoint{5.475083in}{0.755463in}}%
\pgfpathcurveto{\pgfqpoint{5.467269in}{0.747649in}}{\pgfqpoint{5.462879in}{0.737050in}}{\pgfqpoint{5.462879in}{0.726000in}}%
\pgfpathcurveto{\pgfqpoint{5.462879in}{0.714950in}}{\pgfqpoint{5.467269in}{0.704351in}}{\pgfqpoint{5.475083in}{0.696537in}}%
\pgfpathcurveto{\pgfqpoint{5.482896in}{0.688724in}}{\pgfqpoint{5.493495in}{0.684333in}}{\pgfqpoint{5.504545in}{0.684333in}}%
\pgfpathclose%
\pgfusepath{stroke,fill}%
\end{pgfscope}%
\begin{pgfscope}%
\pgfpathrectangle{\pgfqpoint{0.800000in}{0.528000in}}{\pgfqpoint{4.960000in}{3.696000in}}%
\pgfusepath{clip}%
\pgfsetbuttcap%
\pgfsetroundjoin%
\definecolor{currentfill}{rgb}{0.000000,0.000000,0.000000}%
\pgfsetfillcolor{currentfill}%
\pgfsetlinewidth{1.003750pt}%
\definecolor{currentstroke}{rgb}{0.000000,0.000000,0.000000}%
\pgfsetstrokecolor{currentstroke}%
\pgfsetdash{}{0pt}%
\pgfpathmoveto{\pgfqpoint{5.504545in}{0.684333in}}%
\pgfpathcurveto{\pgfqpoint{5.515596in}{0.684333in}}{\pgfqpoint{5.526195in}{0.688724in}}{\pgfqpoint{5.534008in}{0.696537in}}%
\pgfpathcurveto{\pgfqpoint{5.541822in}{0.704351in}}{\pgfqpoint{5.546212in}{0.714950in}}{\pgfqpoint{5.546212in}{0.726000in}}%
\pgfpathcurveto{\pgfqpoint{5.546212in}{0.737050in}}{\pgfqpoint{5.541822in}{0.747649in}}{\pgfqpoint{5.534008in}{0.755463in}}%
\pgfpathcurveto{\pgfqpoint{5.526195in}{0.763276in}}{\pgfqpoint{5.515596in}{0.767667in}}{\pgfqpoint{5.504545in}{0.767667in}}%
\pgfpathcurveto{\pgfqpoint{5.493495in}{0.767667in}}{\pgfqpoint{5.482896in}{0.763276in}}{\pgfqpoint{5.475083in}{0.755463in}}%
\pgfpathcurveto{\pgfqpoint{5.467269in}{0.747649in}}{\pgfqpoint{5.462879in}{0.737050in}}{\pgfqpoint{5.462879in}{0.726000in}}%
\pgfpathcurveto{\pgfqpoint{5.462879in}{0.714950in}}{\pgfqpoint{5.467269in}{0.704351in}}{\pgfqpoint{5.475083in}{0.696537in}}%
\pgfpathcurveto{\pgfqpoint{5.482896in}{0.688724in}}{\pgfqpoint{5.493495in}{0.684333in}}{\pgfqpoint{5.504545in}{0.684333in}}%
\pgfpathclose%
\pgfusepath{stroke,fill}%
\end{pgfscope}%
\begin{pgfscope}%
\pgfpathrectangle{\pgfqpoint{0.800000in}{0.528000in}}{\pgfqpoint{4.960000in}{3.696000in}}%
\pgfusepath{clip}%
\pgfsetbuttcap%
\pgfsetroundjoin%
\definecolor{currentfill}{rgb}{0.000000,0.000000,0.000000}%
\pgfsetfillcolor{currentfill}%
\pgfsetlinewidth{1.003750pt}%
\definecolor{currentstroke}{rgb}{0.000000,0.000000,0.000000}%
\pgfsetstrokecolor{currentstroke}%
\pgfsetdash{}{0pt}%
\pgfpathmoveto{\pgfqpoint{5.504545in}{0.684333in}}%
\pgfpathcurveto{\pgfqpoint{5.515596in}{0.684333in}}{\pgfqpoint{5.526195in}{0.688724in}}{\pgfqpoint{5.534008in}{0.696537in}}%
\pgfpathcurveto{\pgfqpoint{5.541822in}{0.704351in}}{\pgfqpoint{5.546212in}{0.714950in}}{\pgfqpoint{5.546212in}{0.726000in}}%
\pgfpathcurveto{\pgfqpoint{5.546212in}{0.737050in}}{\pgfqpoint{5.541822in}{0.747649in}}{\pgfqpoint{5.534008in}{0.755463in}}%
\pgfpathcurveto{\pgfqpoint{5.526195in}{0.763276in}}{\pgfqpoint{5.515596in}{0.767667in}}{\pgfqpoint{5.504545in}{0.767667in}}%
\pgfpathcurveto{\pgfqpoint{5.493495in}{0.767667in}}{\pgfqpoint{5.482896in}{0.763276in}}{\pgfqpoint{5.475083in}{0.755463in}}%
\pgfpathcurveto{\pgfqpoint{5.467269in}{0.747649in}}{\pgfqpoint{5.462879in}{0.737050in}}{\pgfqpoint{5.462879in}{0.726000in}}%
\pgfpathcurveto{\pgfqpoint{5.462879in}{0.714950in}}{\pgfqpoint{5.467269in}{0.704351in}}{\pgfqpoint{5.475083in}{0.696537in}}%
\pgfpathcurveto{\pgfqpoint{5.482896in}{0.688724in}}{\pgfqpoint{5.493495in}{0.684333in}}{\pgfqpoint{5.504545in}{0.684333in}}%
\pgfpathclose%
\pgfusepath{stroke,fill}%
\end{pgfscope}%
\begin{pgfscope}%
\pgfpathrectangle{\pgfqpoint{0.800000in}{0.528000in}}{\pgfqpoint{4.960000in}{3.696000in}}%
\pgfusepath{clip}%
\pgfsetbuttcap%
\pgfsetroundjoin%
\definecolor{currentfill}{rgb}{0.000000,0.000000,0.000000}%
\pgfsetfillcolor{currentfill}%
\pgfsetlinewidth{1.003750pt}%
\definecolor{currentstroke}{rgb}{0.000000,0.000000,0.000000}%
\pgfsetstrokecolor{currentstroke}%
\pgfsetdash{}{0pt}%
\pgfpathmoveto{\pgfqpoint{5.504545in}{0.684333in}}%
\pgfpathcurveto{\pgfqpoint{5.515596in}{0.684333in}}{\pgfqpoint{5.526195in}{0.688724in}}{\pgfqpoint{5.534008in}{0.696537in}}%
\pgfpathcurveto{\pgfqpoint{5.541822in}{0.704351in}}{\pgfqpoint{5.546212in}{0.714950in}}{\pgfqpoint{5.546212in}{0.726000in}}%
\pgfpathcurveto{\pgfqpoint{5.546212in}{0.737050in}}{\pgfqpoint{5.541822in}{0.747649in}}{\pgfqpoint{5.534008in}{0.755463in}}%
\pgfpathcurveto{\pgfqpoint{5.526195in}{0.763276in}}{\pgfqpoint{5.515596in}{0.767667in}}{\pgfqpoint{5.504545in}{0.767667in}}%
\pgfpathcurveto{\pgfqpoint{5.493495in}{0.767667in}}{\pgfqpoint{5.482896in}{0.763276in}}{\pgfqpoint{5.475083in}{0.755463in}}%
\pgfpathcurveto{\pgfqpoint{5.467269in}{0.747649in}}{\pgfqpoint{5.462879in}{0.737050in}}{\pgfqpoint{5.462879in}{0.726000in}}%
\pgfpathcurveto{\pgfqpoint{5.462879in}{0.714950in}}{\pgfqpoint{5.467269in}{0.704351in}}{\pgfqpoint{5.475083in}{0.696537in}}%
\pgfpathcurveto{\pgfqpoint{5.482896in}{0.688724in}}{\pgfqpoint{5.493495in}{0.684333in}}{\pgfqpoint{5.504545in}{0.684333in}}%
\pgfpathclose%
\pgfusepath{stroke,fill}%
\end{pgfscope}%
\begin{pgfscope}%
\pgfpathrectangle{\pgfqpoint{0.800000in}{0.528000in}}{\pgfqpoint{4.960000in}{3.696000in}}%
\pgfusepath{clip}%
\pgfsetbuttcap%
\pgfsetroundjoin%
\definecolor{currentfill}{rgb}{0.000000,0.000000,0.000000}%
\pgfsetfillcolor{currentfill}%
\pgfsetlinewidth{1.003750pt}%
\definecolor{currentstroke}{rgb}{0.000000,0.000000,0.000000}%
\pgfsetstrokecolor{currentstroke}%
\pgfsetdash{}{0pt}%
\pgfpathmoveto{\pgfqpoint{5.504545in}{0.684333in}}%
\pgfpathcurveto{\pgfqpoint{5.515596in}{0.684333in}}{\pgfqpoint{5.526195in}{0.688724in}}{\pgfqpoint{5.534008in}{0.696537in}}%
\pgfpathcurveto{\pgfqpoint{5.541822in}{0.704351in}}{\pgfqpoint{5.546212in}{0.714950in}}{\pgfqpoint{5.546212in}{0.726000in}}%
\pgfpathcurveto{\pgfqpoint{5.546212in}{0.737050in}}{\pgfqpoint{5.541822in}{0.747649in}}{\pgfqpoint{5.534008in}{0.755463in}}%
\pgfpathcurveto{\pgfqpoint{5.526195in}{0.763276in}}{\pgfqpoint{5.515596in}{0.767667in}}{\pgfqpoint{5.504545in}{0.767667in}}%
\pgfpathcurveto{\pgfqpoint{5.493495in}{0.767667in}}{\pgfqpoint{5.482896in}{0.763276in}}{\pgfqpoint{5.475083in}{0.755463in}}%
\pgfpathcurveto{\pgfqpoint{5.467269in}{0.747649in}}{\pgfqpoint{5.462879in}{0.737050in}}{\pgfqpoint{5.462879in}{0.726000in}}%
\pgfpathcurveto{\pgfqpoint{5.462879in}{0.714950in}}{\pgfqpoint{5.467269in}{0.704351in}}{\pgfqpoint{5.475083in}{0.696537in}}%
\pgfpathcurveto{\pgfqpoint{5.482896in}{0.688724in}}{\pgfqpoint{5.493495in}{0.684333in}}{\pgfqpoint{5.504545in}{0.684333in}}%
\pgfpathclose%
\pgfusepath{stroke,fill}%
\end{pgfscope}%
\begin{pgfscope}%
\pgfpathrectangle{\pgfqpoint{0.800000in}{0.528000in}}{\pgfqpoint{4.960000in}{3.696000in}}%
\pgfusepath{clip}%
\pgfsetbuttcap%
\pgfsetroundjoin%
\definecolor{currentfill}{rgb}{0.000000,0.000000,0.000000}%
\pgfsetfillcolor{currentfill}%
\pgfsetlinewidth{1.003750pt}%
\definecolor{currentstroke}{rgb}{0.000000,0.000000,0.000000}%
\pgfsetstrokecolor{currentstroke}%
\pgfsetdash{}{0pt}%
\pgfpathmoveto{\pgfqpoint{5.504545in}{0.684333in}}%
\pgfpathcurveto{\pgfqpoint{5.515596in}{0.684333in}}{\pgfqpoint{5.526195in}{0.688724in}}{\pgfqpoint{5.534008in}{0.696537in}}%
\pgfpathcurveto{\pgfqpoint{5.541822in}{0.704351in}}{\pgfqpoint{5.546212in}{0.714950in}}{\pgfqpoint{5.546212in}{0.726000in}}%
\pgfpathcurveto{\pgfqpoint{5.546212in}{0.737050in}}{\pgfqpoint{5.541822in}{0.747649in}}{\pgfqpoint{5.534008in}{0.755463in}}%
\pgfpathcurveto{\pgfqpoint{5.526195in}{0.763276in}}{\pgfqpoint{5.515596in}{0.767667in}}{\pgfqpoint{5.504545in}{0.767667in}}%
\pgfpathcurveto{\pgfqpoint{5.493495in}{0.767667in}}{\pgfqpoint{5.482896in}{0.763276in}}{\pgfqpoint{5.475083in}{0.755463in}}%
\pgfpathcurveto{\pgfqpoint{5.467269in}{0.747649in}}{\pgfqpoint{5.462879in}{0.737050in}}{\pgfqpoint{5.462879in}{0.726000in}}%
\pgfpathcurveto{\pgfqpoint{5.462879in}{0.714950in}}{\pgfqpoint{5.467269in}{0.704351in}}{\pgfqpoint{5.475083in}{0.696537in}}%
\pgfpathcurveto{\pgfqpoint{5.482896in}{0.688724in}}{\pgfqpoint{5.493495in}{0.684333in}}{\pgfqpoint{5.504545in}{0.684333in}}%
\pgfpathclose%
\pgfusepath{stroke,fill}%
\end{pgfscope}%
\begin{pgfscope}%
\pgfpathrectangle{\pgfqpoint{0.800000in}{0.528000in}}{\pgfqpoint{4.960000in}{3.696000in}}%
\pgfusepath{clip}%
\pgfsetbuttcap%
\pgfsetroundjoin%
\definecolor{currentfill}{rgb}{0.000000,0.000000,0.000000}%
\pgfsetfillcolor{currentfill}%
\pgfsetlinewidth{1.003750pt}%
\definecolor{currentstroke}{rgb}{0.000000,0.000000,0.000000}%
\pgfsetstrokecolor{currentstroke}%
\pgfsetdash{}{0pt}%
\pgfpathmoveto{\pgfqpoint{5.504545in}{0.684333in}}%
\pgfpathcurveto{\pgfqpoint{5.515596in}{0.684333in}}{\pgfqpoint{5.526195in}{0.688724in}}{\pgfqpoint{5.534008in}{0.696537in}}%
\pgfpathcurveto{\pgfqpoint{5.541822in}{0.704351in}}{\pgfqpoint{5.546212in}{0.714950in}}{\pgfqpoint{5.546212in}{0.726000in}}%
\pgfpathcurveto{\pgfqpoint{5.546212in}{0.737050in}}{\pgfqpoint{5.541822in}{0.747649in}}{\pgfqpoint{5.534008in}{0.755463in}}%
\pgfpathcurveto{\pgfqpoint{5.526195in}{0.763276in}}{\pgfqpoint{5.515596in}{0.767667in}}{\pgfqpoint{5.504545in}{0.767667in}}%
\pgfpathcurveto{\pgfqpoint{5.493495in}{0.767667in}}{\pgfqpoint{5.482896in}{0.763276in}}{\pgfqpoint{5.475083in}{0.755463in}}%
\pgfpathcurveto{\pgfqpoint{5.467269in}{0.747649in}}{\pgfqpoint{5.462879in}{0.737050in}}{\pgfqpoint{5.462879in}{0.726000in}}%
\pgfpathcurveto{\pgfqpoint{5.462879in}{0.714950in}}{\pgfqpoint{5.467269in}{0.704351in}}{\pgfqpoint{5.475083in}{0.696537in}}%
\pgfpathcurveto{\pgfqpoint{5.482896in}{0.688724in}}{\pgfqpoint{5.493495in}{0.684333in}}{\pgfqpoint{5.504545in}{0.684333in}}%
\pgfpathclose%
\pgfusepath{stroke,fill}%
\end{pgfscope}%
\begin{pgfscope}%
\pgfpathrectangle{\pgfqpoint{0.800000in}{0.528000in}}{\pgfqpoint{4.960000in}{3.696000in}}%
\pgfusepath{clip}%
\pgfsetbuttcap%
\pgfsetroundjoin%
\definecolor{currentfill}{rgb}{0.000000,0.000000,0.000000}%
\pgfsetfillcolor{currentfill}%
\pgfsetlinewidth{1.003750pt}%
\definecolor{currentstroke}{rgb}{0.000000,0.000000,0.000000}%
\pgfsetstrokecolor{currentstroke}%
\pgfsetdash{}{0pt}%
\pgfpathmoveto{\pgfqpoint{5.504545in}{0.684333in}}%
\pgfpathcurveto{\pgfqpoint{5.515596in}{0.684333in}}{\pgfqpoint{5.526195in}{0.688724in}}{\pgfqpoint{5.534008in}{0.696537in}}%
\pgfpathcurveto{\pgfqpoint{5.541822in}{0.704351in}}{\pgfqpoint{5.546212in}{0.714950in}}{\pgfqpoint{5.546212in}{0.726000in}}%
\pgfpathcurveto{\pgfqpoint{5.546212in}{0.737050in}}{\pgfqpoint{5.541822in}{0.747649in}}{\pgfqpoint{5.534008in}{0.755463in}}%
\pgfpathcurveto{\pgfqpoint{5.526195in}{0.763276in}}{\pgfqpoint{5.515596in}{0.767667in}}{\pgfqpoint{5.504545in}{0.767667in}}%
\pgfpathcurveto{\pgfqpoint{5.493495in}{0.767667in}}{\pgfqpoint{5.482896in}{0.763276in}}{\pgfqpoint{5.475083in}{0.755463in}}%
\pgfpathcurveto{\pgfqpoint{5.467269in}{0.747649in}}{\pgfqpoint{5.462879in}{0.737050in}}{\pgfqpoint{5.462879in}{0.726000in}}%
\pgfpathcurveto{\pgfqpoint{5.462879in}{0.714950in}}{\pgfqpoint{5.467269in}{0.704351in}}{\pgfqpoint{5.475083in}{0.696537in}}%
\pgfpathcurveto{\pgfqpoint{5.482896in}{0.688724in}}{\pgfqpoint{5.493495in}{0.684333in}}{\pgfqpoint{5.504545in}{0.684333in}}%
\pgfpathclose%
\pgfusepath{stroke,fill}%
\end{pgfscope}%
\begin{pgfscope}%
\pgfpathrectangle{\pgfqpoint{0.800000in}{0.528000in}}{\pgfqpoint{4.960000in}{3.696000in}}%
\pgfusepath{clip}%
\pgfsetbuttcap%
\pgfsetroundjoin%
\definecolor{currentfill}{rgb}{0.000000,0.000000,0.000000}%
\pgfsetfillcolor{currentfill}%
\pgfsetlinewidth{1.003750pt}%
\definecolor{currentstroke}{rgb}{0.000000,0.000000,0.000000}%
\pgfsetstrokecolor{currentstroke}%
\pgfsetdash{}{0pt}%
\pgfpathmoveto{\pgfqpoint{5.504545in}{0.684333in}}%
\pgfpathcurveto{\pgfqpoint{5.515596in}{0.684333in}}{\pgfqpoint{5.526195in}{0.688724in}}{\pgfqpoint{5.534008in}{0.696537in}}%
\pgfpathcurveto{\pgfqpoint{5.541822in}{0.704351in}}{\pgfqpoint{5.546212in}{0.714950in}}{\pgfqpoint{5.546212in}{0.726000in}}%
\pgfpathcurveto{\pgfqpoint{5.546212in}{0.737050in}}{\pgfqpoint{5.541822in}{0.747649in}}{\pgfqpoint{5.534008in}{0.755463in}}%
\pgfpathcurveto{\pgfqpoint{5.526195in}{0.763276in}}{\pgfqpoint{5.515596in}{0.767667in}}{\pgfqpoint{5.504545in}{0.767667in}}%
\pgfpathcurveto{\pgfqpoint{5.493495in}{0.767667in}}{\pgfqpoint{5.482896in}{0.763276in}}{\pgfqpoint{5.475083in}{0.755463in}}%
\pgfpathcurveto{\pgfqpoint{5.467269in}{0.747649in}}{\pgfqpoint{5.462879in}{0.737050in}}{\pgfqpoint{5.462879in}{0.726000in}}%
\pgfpathcurveto{\pgfqpoint{5.462879in}{0.714950in}}{\pgfqpoint{5.467269in}{0.704351in}}{\pgfqpoint{5.475083in}{0.696537in}}%
\pgfpathcurveto{\pgfqpoint{5.482896in}{0.688724in}}{\pgfqpoint{5.493495in}{0.684333in}}{\pgfqpoint{5.504545in}{0.684333in}}%
\pgfpathclose%
\pgfusepath{stroke,fill}%
\end{pgfscope}%
\begin{pgfscope}%
\pgfpathrectangle{\pgfqpoint{0.800000in}{0.528000in}}{\pgfqpoint{4.960000in}{3.696000in}}%
\pgfusepath{clip}%
\pgfsetbuttcap%
\pgfsetroundjoin%
\definecolor{currentfill}{rgb}{0.000000,0.000000,0.000000}%
\pgfsetfillcolor{currentfill}%
\pgfsetlinewidth{1.003750pt}%
\definecolor{currentstroke}{rgb}{0.000000,0.000000,0.000000}%
\pgfsetstrokecolor{currentstroke}%
\pgfsetdash{}{0pt}%
\pgfpathmoveto{\pgfqpoint{5.504545in}{0.684333in}}%
\pgfpathcurveto{\pgfqpoint{5.515596in}{0.684333in}}{\pgfqpoint{5.526195in}{0.688724in}}{\pgfqpoint{5.534008in}{0.696537in}}%
\pgfpathcurveto{\pgfqpoint{5.541822in}{0.704351in}}{\pgfqpoint{5.546212in}{0.714950in}}{\pgfqpoint{5.546212in}{0.726000in}}%
\pgfpathcurveto{\pgfqpoint{5.546212in}{0.737050in}}{\pgfqpoint{5.541822in}{0.747649in}}{\pgfqpoint{5.534008in}{0.755463in}}%
\pgfpathcurveto{\pgfqpoint{5.526195in}{0.763276in}}{\pgfqpoint{5.515596in}{0.767667in}}{\pgfqpoint{5.504545in}{0.767667in}}%
\pgfpathcurveto{\pgfqpoint{5.493495in}{0.767667in}}{\pgfqpoint{5.482896in}{0.763276in}}{\pgfqpoint{5.475083in}{0.755463in}}%
\pgfpathcurveto{\pgfqpoint{5.467269in}{0.747649in}}{\pgfqpoint{5.462879in}{0.737050in}}{\pgfqpoint{5.462879in}{0.726000in}}%
\pgfpathcurveto{\pgfqpoint{5.462879in}{0.714950in}}{\pgfqpoint{5.467269in}{0.704351in}}{\pgfqpoint{5.475083in}{0.696537in}}%
\pgfpathcurveto{\pgfqpoint{5.482896in}{0.688724in}}{\pgfqpoint{5.493495in}{0.684333in}}{\pgfqpoint{5.504545in}{0.684333in}}%
\pgfpathclose%
\pgfusepath{stroke,fill}%
\end{pgfscope}%
\begin{pgfscope}%
\pgfpathrectangle{\pgfqpoint{0.800000in}{0.528000in}}{\pgfqpoint{4.960000in}{3.696000in}}%
\pgfusepath{clip}%
\pgfsetbuttcap%
\pgfsetroundjoin%
\definecolor{currentfill}{rgb}{0.000000,0.000000,0.000000}%
\pgfsetfillcolor{currentfill}%
\pgfsetlinewidth{1.003750pt}%
\definecolor{currentstroke}{rgb}{0.000000,0.000000,0.000000}%
\pgfsetstrokecolor{currentstroke}%
\pgfsetdash{}{0pt}%
\pgfpathmoveto{\pgfqpoint{5.504545in}{0.684333in}}%
\pgfpathcurveto{\pgfqpoint{5.515596in}{0.684333in}}{\pgfqpoint{5.526195in}{0.688724in}}{\pgfqpoint{5.534008in}{0.696537in}}%
\pgfpathcurveto{\pgfqpoint{5.541822in}{0.704351in}}{\pgfqpoint{5.546212in}{0.714950in}}{\pgfqpoint{5.546212in}{0.726000in}}%
\pgfpathcurveto{\pgfqpoint{5.546212in}{0.737050in}}{\pgfqpoint{5.541822in}{0.747649in}}{\pgfqpoint{5.534008in}{0.755463in}}%
\pgfpathcurveto{\pgfqpoint{5.526195in}{0.763276in}}{\pgfqpoint{5.515596in}{0.767667in}}{\pgfqpoint{5.504545in}{0.767667in}}%
\pgfpathcurveto{\pgfqpoint{5.493495in}{0.767667in}}{\pgfqpoint{5.482896in}{0.763276in}}{\pgfqpoint{5.475083in}{0.755463in}}%
\pgfpathcurveto{\pgfqpoint{5.467269in}{0.747649in}}{\pgfqpoint{5.462879in}{0.737050in}}{\pgfqpoint{5.462879in}{0.726000in}}%
\pgfpathcurveto{\pgfqpoint{5.462879in}{0.714950in}}{\pgfqpoint{5.467269in}{0.704351in}}{\pgfqpoint{5.475083in}{0.696537in}}%
\pgfpathcurveto{\pgfqpoint{5.482896in}{0.688724in}}{\pgfqpoint{5.493495in}{0.684333in}}{\pgfqpoint{5.504545in}{0.684333in}}%
\pgfpathclose%
\pgfusepath{stroke,fill}%
\end{pgfscope}%
\begin{pgfscope}%
\pgfpathrectangle{\pgfqpoint{0.800000in}{0.528000in}}{\pgfqpoint{4.960000in}{3.696000in}}%
\pgfusepath{clip}%
\pgfsetbuttcap%
\pgfsetroundjoin%
\definecolor{currentfill}{rgb}{0.000000,0.000000,0.000000}%
\pgfsetfillcolor{currentfill}%
\pgfsetlinewidth{1.003750pt}%
\definecolor{currentstroke}{rgb}{0.000000,0.000000,0.000000}%
\pgfsetstrokecolor{currentstroke}%
\pgfsetdash{}{0pt}%
\pgfpathmoveto{\pgfqpoint{5.504545in}{0.684333in}}%
\pgfpathcurveto{\pgfqpoint{5.515596in}{0.684333in}}{\pgfqpoint{5.526195in}{0.688724in}}{\pgfqpoint{5.534008in}{0.696537in}}%
\pgfpathcurveto{\pgfqpoint{5.541822in}{0.704351in}}{\pgfqpoint{5.546212in}{0.714950in}}{\pgfqpoint{5.546212in}{0.726000in}}%
\pgfpathcurveto{\pgfqpoint{5.546212in}{0.737050in}}{\pgfqpoint{5.541822in}{0.747649in}}{\pgfqpoint{5.534008in}{0.755463in}}%
\pgfpathcurveto{\pgfqpoint{5.526195in}{0.763276in}}{\pgfqpoint{5.515596in}{0.767667in}}{\pgfqpoint{5.504545in}{0.767667in}}%
\pgfpathcurveto{\pgfqpoint{5.493495in}{0.767667in}}{\pgfqpoint{5.482896in}{0.763276in}}{\pgfqpoint{5.475083in}{0.755463in}}%
\pgfpathcurveto{\pgfqpoint{5.467269in}{0.747649in}}{\pgfqpoint{5.462879in}{0.737050in}}{\pgfqpoint{5.462879in}{0.726000in}}%
\pgfpathcurveto{\pgfqpoint{5.462879in}{0.714950in}}{\pgfqpoint{5.467269in}{0.704351in}}{\pgfqpoint{5.475083in}{0.696537in}}%
\pgfpathcurveto{\pgfqpoint{5.482896in}{0.688724in}}{\pgfqpoint{5.493495in}{0.684333in}}{\pgfqpoint{5.504545in}{0.684333in}}%
\pgfpathclose%
\pgfusepath{stroke,fill}%
\end{pgfscope}%
\begin{pgfscope}%
\pgfpathrectangle{\pgfqpoint{0.800000in}{0.528000in}}{\pgfqpoint{4.960000in}{3.696000in}}%
\pgfusepath{clip}%
\pgfsetbuttcap%
\pgfsetroundjoin%
\definecolor{currentfill}{rgb}{0.000000,0.000000,0.000000}%
\pgfsetfillcolor{currentfill}%
\pgfsetlinewidth{1.003750pt}%
\definecolor{currentstroke}{rgb}{0.000000,0.000000,0.000000}%
\pgfsetstrokecolor{currentstroke}%
\pgfsetdash{}{0pt}%
\pgfpathmoveto{\pgfqpoint{5.504545in}{0.684333in}}%
\pgfpathcurveto{\pgfqpoint{5.515596in}{0.684333in}}{\pgfqpoint{5.526195in}{0.688724in}}{\pgfqpoint{5.534008in}{0.696537in}}%
\pgfpathcurveto{\pgfqpoint{5.541822in}{0.704351in}}{\pgfqpoint{5.546212in}{0.714950in}}{\pgfqpoint{5.546212in}{0.726000in}}%
\pgfpathcurveto{\pgfqpoint{5.546212in}{0.737050in}}{\pgfqpoint{5.541822in}{0.747649in}}{\pgfqpoint{5.534008in}{0.755463in}}%
\pgfpathcurveto{\pgfqpoint{5.526195in}{0.763276in}}{\pgfqpoint{5.515596in}{0.767667in}}{\pgfqpoint{5.504545in}{0.767667in}}%
\pgfpathcurveto{\pgfqpoint{5.493495in}{0.767667in}}{\pgfqpoint{5.482896in}{0.763276in}}{\pgfqpoint{5.475083in}{0.755463in}}%
\pgfpathcurveto{\pgfqpoint{5.467269in}{0.747649in}}{\pgfqpoint{5.462879in}{0.737050in}}{\pgfqpoint{5.462879in}{0.726000in}}%
\pgfpathcurveto{\pgfqpoint{5.462879in}{0.714950in}}{\pgfqpoint{5.467269in}{0.704351in}}{\pgfqpoint{5.475083in}{0.696537in}}%
\pgfpathcurveto{\pgfqpoint{5.482896in}{0.688724in}}{\pgfqpoint{5.493495in}{0.684333in}}{\pgfqpoint{5.504545in}{0.684333in}}%
\pgfpathclose%
\pgfusepath{stroke,fill}%
\end{pgfscope}%
\begin{pgfscope}%
\pgfpathrectangle{\pgfqpoint{0.800000in}{0.528000in}}{\pgfqpoint{4.960000in}{3.696000in}}%
\pgfusepath{clip}%
\pgfsetbuttcap%
\pgfsetroundjoin%
\definecolor{currentfill}{rgb}{0.000000,0.000000,0.000000}%
\pgfsetfillcolor{currentfill}%
\pgfsetlinewidth{1.003750pt}%
\definecolor{currentstroke}{rgb}{0.000000,0.000000,0.000000}%
\pgfsetstrokecolor{currentstroke}%
\pgfsetdash{}{0pt}%
\pgfpathmoveto{\pgfqpoint{5.504545in}{0.684333in}}%
\pgfpathcurveto{\pgfqpoint{5.515596in}{0.684333in}}{\pgfqpoint{5.526195in}{0.688724in}}{\pgfqpoint{5.534008in}{0.696537in}}%
\pgfpathcurveto{\pgfqpoint{5.541822in}{0.704351in}}{\pgfqpoint{5.546212in}{0.714950in}}{\pgfqpoint{5.546212in}{0.726000in}}%
\pgfpathcurveto{\pgfqpoint{5.546212in}{0.737050in}}{\pgfqpoint{5.541822in}{0.747649in}}{\pgfqpoint{5.534008in}{0.755463in}}%
\pgfpathcurveto{\pgfqpoint{5.526195in}{0.763276in}}{\pgfqpoint{5.515596in}{0.767667in}}{\pgfqpoint{5.504545in}{0.767667in}}%
\pgfpathcurveto{\pgfqpoint{5.493495in}{0.767667in}}{\pgfqpoint{5.482896in}{0.763276in}}{\pgfqpoint{5.475083in}{0.755463in}}%
\pgfpathcurveto{\pgfqpoint{5.467269in}{0.747649in}}{\pgfqpoint{5.462879in}{0.737050in}}{\pgfqpoint{5.462879in}{0.726000in}}%
\pgfpathcurveto{\pgfqpoint{5.462879in}{0.714950in}}{\pgfqpoint{5.467269in}{0.704351in}}{\pgfqpoint{5.475083in}{0.696537in}}%
\pgfpathcurveto{\pgfqpoint{5.482896in}{0.688724in}}{\pgfqpoint{5.493495in}{0.684333in}}{\pgfqpoint{5.504545in}{0.684333in}}%
\pgfpathclose%
\pgfusepath{stroke,fill}%
\end{pgfscope}%
\begin{pgfscope}%
\pgfpathrectangle{\pgfqpoint{0.800000in}{0.528000in}}{\pgfqpoint{4.960000in}{3.696000in}}%
\pgfusepath{clip}%
\pgfsetbuttcap%
\pgfsetroundjoin%
\definecolor{currentfill}{rgb}{0.000000,0.000000,0.000000}%
\pgfsetfillcolor{currentfill}%
\pgfsetlinewidth{1.003750pt}%
\definecolor{currentstroke}{rgb}{0.000000,0.000000,0.000000}%
\pgfsetstrokecolor{currentstroke}%
\pgfsetdash{}{0pt}%
\pgfpathmoveto{\pgfqpoint{5.504545in}{0.684333in}}%
\pgfpathcurveto{\pgfqpoint{5.515596in}{0.684333in}}{\pgfqpoint{5.526195in}{0.688724in}}{\pgfqpoint{5.534008in}{0.696537in}}%
\pgfpathcurveto{\pgfqpoint{5.541822in}{0.704351in}}{\pgfqpoint{5.546212in}{0.714950in}}{\pgfqpoint{5.546212in}{0.726000in}}%
\pgfpathcurveto{\pgfqpoint{5.546212in}{0.737050in}}{\pgfqpoint{5.541822in}{0.747649in}}{\pgfqpoint{5.534008in}{0.755463in}}%
\pgfpathcurveto{\pgfqpoint{5.526195in}{0.763276in}}{\pgfqpoint{5.515596in}{0.767667in}}{\pgfqpoint{5.504545in}{0.767667in}}%
\pgfpathcurveto{\pgfqpoint{5.493495in}{0.767667in}}{\pgfqpoint{5.482896in}{0.763276in}}{\pgfqpoint{5.475083in}{0.755463in}}%
\pgfpathcurveto{\pgfqpoint{5.467269in}{0.747649in}}{\pgfqpoint{5.462879in}{0.737050in}}{\pgfqpoint{5.462879in}{0.726000in}}%
\pgfpathcurveto{\pgfqpoint{5.462879in}{0.714950in}}{\pgfqpoint{5.467269in}{0.704351in}}{\pgfqpoint{5.475083in}{0.696537in}}%
\pgfpathcurveto{\pgfqpoint{5.482896in}{0.688724in}}{\pgfqpoint{5.493495in}{0.684333in}}{\pgfqpoint{5.504545in}{0.684333in}}%
\pgfpathclose%
\pgfusepath{stroke,fill}%
\end{pgfscope}%
\begin{pgfscope}%
\pgfpathrectangle{\pgfqpoint{0.800000in}{0.528000in}}{\pgfqpoint{4.960000in}{3.696000in}}%
\pgfusepath{clip}%
\pgfsetbuttcap%
\pgfsetroundjoin%
\definecolor{currentfill}{rgb}{0.000000,0.000000,0.000000}%
\pgfsetfillcolor{currentfill}%
\pgfsetlinewidth{1.003750pt}%
\definecolor{currentstroke}{rgb}{0.000000,0.000000,0.000000}%
\pgfsetstrokecolor{currentstroke}%
\pgfsetdash{}{0pt}%
\pgfpathmoveto{\pgfqpoint{5.504545in}{0.684333in}}%
\pgfpathcurveto{\pgfqpoint{5.515596in}{0.684333in}}{\pgfqpoint{5.526195in}{0.688724in}}{\pgfqpoint{5.534008in}{0.696537in}}%
\pgfpathcurveto{\pgfqpoint{5.541822in}{0.704351in}}{\pgfqpoint{5.546212in}{0.714950in}}{\pgfqpoint{5.546212in}{0.726000in}}%
\pgfpathcurveto{\pgfqpoint{5.546212in}{0.737050in}}{\pgfqpoint{5.541822in}{0.747649in}}{\pgfqpoint{5.534008in}{0.755463in}}%
\pgfpathcurveto{\pgfqpoint{5.526195in}{0.763276in}}{\pgfqpoint{5.515596in}{0.767667in}}{\pgfqpoint{5.504545in}{0.767667in}}%
\pgfpathcurveto{\pgfqpoint{5.493495in}{0.767667in}}{\pgfqpoint{5.482896in}{0.763276in}}{\pgfqpoint{5.475083in}{0.755463in}}%
\pgfpathcurveto{\pgfqpoint{5.467269in}{0.747649in}}{\pgfqpoint{5.462879in}{0.737050in}}{\pgfqpoint{5.462879in}{0.726000in}}%
\pgfpathcurveto{\pgfqpoint{5.462879in}{0.714950in}}{\pgfqpoint{5.467269in}{0.704351in}}{\pgfqpoint{5.475083in}{0.696537in}}%
\pgfpathcurveto{\pgfqpoint{5.482896in}{0.688724in}}{\pgfqpoint{5.493495in}{0.684333in}}{\pgfqpoint{5.504545in}{0.684333in}}%
\pgfpathclose%
\pgfusepath{stroke,fill}%
\end{pgfscope}%
\begin{pgfscope}%
\pgfpathrectangle{\pgfqpoint{0.800000in}{0.528000in}}{\pgfqpoint{4.960000in}{3.696000in}}%
\pgfusepath{clip}%
\pgfsetbuttcap%
\pgfsetroundjoin%
\definecolor{currentfill}{rgb}{0.000000,0.000000,0.000000}%
\pgfsetfillcolor{currentfill}%
\pgfsetlinewidth{1.003750pt}%
\definecolor{currentstroke}{rgb}{0.000000,0.000000,0.000000}%
\pgfsetstrokecolor{currentstroke}%
\pgfsetdash{}{0pt}%
\pgfpathmoveto{\pgfqpoint{5.504545in}{0.684333in}}%
\pgfpathcurveto{\pgfqpoint{5.515596in}{0.684333in}}{\pgfqpoint{5.526195in}{0.688724in}}{\pgfqpoint{5.534008in}{0.696537in}}%
\pgfpathcurveto{\pgfqpoint{5.541822in}{0.704351in}}{\pgfqpoint{5.546212in}{0.714950in}}{\pgfqpoint{5.546212in}{0.726000in}}%
\pgfpathcurveto{\pgfqpoint{5.546212in}{0.737050in}}{\pgfqpoint{5.541822in}{0.747649in}}{\pgfqpoint{5.534008in}{0.755463in}}%
\pgfpathcurveto{\pgfqpoint{5.526195in}{0.763276in}}{\pgfqpoint{5.515596in}{0.767667in}}{\pgfqpoint{5.504545in}{0.767667in}}%
\pgfpathcurveto{\pgfqpoint{5.493495in}{0.767667in}}{\pgfqpoint{5.482896in}{0.763276in}}{\pgfqpoint{5.475083in}{0.755463in}}%
\pgfpathcurveto{\pgfqpoint{5.467269in}{0.747649in}}{\pgfqpoint{5.462879in}{0.737050in}}{\pgfqpoint{5.462879in}{0.726000in}}%
\pgfpathcurveto{\pgfqpoint{5.462879in}{0.714950in}}{\pgfqpoint{5.467269in}{0.704351in}}{\pgfqpoint{5.475083in}{0.696537in}}%
\pgfpathcurveto{\pgfqpoint{5.482896in}{0.688724in}}{\pgfqpoint{5.493495in}{0.684333in}}{\pgfqpoint{5.504545in}{0.684333in}}%
\pgfpathclose%
\pgfusepath{stroke,fill}%
\end{pgfscope}%
\begin{pgfscope}%
\pgfpathrectangle{\pgfqpoint{0.800000in}{0.528000in}}{\pgfqpoint{4.960000in}{3.696000in}}%
\pgfusepath{clip}%
\pgfsetbuttcap%
\pgfsetroundjoin%
\definecolor{currentfill}{rgb}{0.000000,0.000000,0.000000}%
\pgfsetfillcolor{currentfill}%
\pgfsetlinewidth{1.003750pt}%
\definecolor{currentstroke}{rgb}{0.000000,0.000000,0.000000}%
\pgfsetstrokecolor{currentstroke}%
\pgfsetdash{}{0pt}%
\pgfpathmoveto{\pgfqpoint{5.504545in}{0.684333in}}%
\pgfpathcurveto{\pgfqpoint{5.515596in}{0.684333in}}{\pgfqpoint{5.526195in}{0.688724in}}{\pgfqpoint{5.534008in}{0.696537in}}%
\pgfpathcurveto{\pgfqpoint{5.541822in}{0.704351in}}{\pgfqpoint{5.546212in}{0.714950in}}{\pgfqpoint{5.546212in}{0.726000in}}%
\pgfpathcurveto{\pgfqpoint{5.546212in}{0.737050in}}{\pgfqpoint{5.541822in}{0.747649in}}{\pgfqpoint{5.534008in}{0.755463in}}%
\pgfpathcurveto{\pgfqpoint{5.526195in}{0.763276in}}{\pgfqpoint{5.515596in}{0.767667in}}{\pgfqpoint{5.504545in}{0.767667in}}%
\pgfpathcurveto{\pgfqpoint{5.493495in}{0.767667in}}{\pgfqpoint{5.482896in}{0.763276in}}{\pgfqpoint{5.475083in}{0.755463in}}%
\pgfpathcurveto{\pgfqpoint{5.467269in}{0.747649in}}{\pgfqpoint{5.462879in}{0.737050in}}{\pgfqpoint{5.462879in}{0.726000in}}%
\pgfpathcurveto{\pgfqpoint{5.462879in}{0.714950in}}{\pgfqpoint{5.467269in}{0.704351in}}{\pgfqpoint{5.475083in}{0.696537in}}%
\pgfpathcurveto{\pgfqpoint{5.482896in}{0.688724in}}{\pgfqpoint{5.493495in}{0.684333in}}{\pgfqpoint{5.504545in}{0.684333in}}%
\pgfpathclose%
\pgfusepath{stroke,fill}%
\end{pgfscope}%
\begin{pgfscope}%
\pgfpathrectangle{\pgfqpoint{0.800000in}{0.528000in}}{\pgfqpoint{4.960000in}{3.696000in}}%
\pgfusepath{clip}%
\pgfsetbuttcap%
\pgfsetroundjoin%
\definecolor{currentfill}{rgb}{0.000000,0.000000,0.000000}%
\pgfsetfillcolor{currentfill}%
\pgfsetlinewidth{1.003750pt}%
\definecolor{currentstroke}{rgb}{0.000000,0.000000,0.000000}%
\pgfsetstrokecolor{currentstroke}%
\pgfsetdash{}{0pt}%
\pgfpathmoveto{\pgfqpoint{5.504545in}{0.684333in}}%
\pgfpathcurveto{\pgfqpoint{5.515596in}{0.684333in}}{\pgfqpoint{5.526195in}{0.688724in}}{\pgfqpoint{5.534008in}{0.696537in}}%
\pgfpathcurveto{\pgfqpoint{5.541822in}{0.704351in}}{\pgfqpoint{5.546212in}{0.714950in}}{\pgfqpoint{5.546212in}{0.726000in}}%
\pgfpathcurveto{\pgfqpoint{5.546212in}{0.737050in}}{\pgfqpoint{5.541822in}{0.747649in}}{\pgfqpoint{5.534008in}{0.755463in}}%
\pgfpathcurveto{\pgfqpoint{5.526195in}{0.763276in}}{\pgfqpoint{5.515596in}{0.767667in}}{\pgfqpoint{5.504545in}{0.767667in}}%
\pgfpathcurveto{\pgfqpoint{5.493495in}{0.767667in}}{\pgfqpoint{5.482896in}{0.763276in}}{\pgfqpoint{5.475083in}{0.755463in}}%
\pgfpathcurveto{\pgfqpoint{5.467269in}{0.747649in}}{\pgfqpoint{5.462879in}{0.737050in}}{\pgfqpoint{5.462879in}{0.726000in}}%
\pgfpathcurveto{\pgfqpoint{5.462879in}{0.714950in}}{\pgfqpoint{5.467269in}{0.704351in}}{\pgfqpoint{5.475083in}{0.696537in}}%
\pgfpathcurveto{\pgfqpoint{5.482896in}{0.688724in}}{\pgfqpoint{5.493495in}{0.684333in}}{\pgfqpoint{5.504545in}{0.684333in}}%
\pgfpathclose%
\pgfusepath{stroke,fill}%
\end{pgfscope}%
\begin{pgfscope}%
\pgfpathrectangle{\pgfqpoint{0.800000in}{0.528000in}}{\pgfqpoint{4.960000in}{3.696000in}}%
\pgfusepath{clip}%
\pgfsetbuttcap%
\pgfsetroundjoin%
\definecolor{currentfill}{rgb}{0.000000,0.000000,0.000000}%
\pgfsetfillcolor{currentfill}%
\pgfsetlinewidth{1.003750pt}%
\definecolor{currentstroke}{rgb}{0.000000,0.000000,0.000000}%
\pgfsetstrokecolor{currentstroke}%
\pgfsetdash{}{0pt}%
\pgfpathmoveto{\pgfqpoint{5.504545in}{0.684333in}}%
\pgfpathcurveto{\pgfqpoint{5.515596in}{0.684333in}}{\pgfqpoint{5.526195in}{0.688724in}}{\pgfqpoint{5.534008in}{0.696537in}}%
\pgfpathcurveto{\pgfqpoint{5.541822in}{0.704351in}}{\pgfqpoint{5.546212in}{0.714950in}}{\pgfqpoint{5.546212in}{0.726000in}}%
\pgfpathcurveto{\pgfqpoint{5.546212in}{0.737050in}}{\pgfqpoint{5.541822in}{0.747649in}}{\pgfqpoint{5.534008in}{0.755463in}}%
\pgfpathcurveto{\pgfqpoint{5.526195in}{0.763276in}}{\pgfqpoint{5.515596in}{0.767667in}}{\pgfqpoint{5.504545in}{0.767667in}}%
\pgfpathcurveto{\pgfqpoint{5.493495in}{0.767667in}}{\pgfqpoint{5.482896in}{0.763276in}}{\pgfqpoint{5.475083in}{0.755463in}}%
\pgfpathcurveto{\pgfqpoint{5.467269in}{0.747649in}}{\pgfqpoint{5.462879in}{0.737050in}}{\pgfqpoint{5.462879in}{0.726000in}}%
\pgfpathcurveto{\pgfqpoint{5.462879in}{0.714950in}}{\pgfqpoint{5.467269in}{0.704351in}}{\pgfqpoint{5.475083in}{0.696537in}}%
\pgfpathcurveto{\pgfqpoint{5.482896in}{0.688724in}}{\pgfqpoint{5.493495in}{0.684333in}}{\pgfqpoint{5.504545in}{0.684333in}}%
\pgfpathclose%
\pgfusepath{stroke,fill}%
\end{pgfscope}%
\begin{pgfscope}%
\pgfpathrectangle{\pgfqpoint{0.800000in}{0.528000in}}{\pgfqpoint{4.960000in}{3.696000in}}%
\pgfusepath{clip}%
\pgfsetbuttcap%
\pgfsetroundjoin%
\definecolor{currentfill}{rgb}{0.000000,0.000000,0.000000}%
\pgfsetfillcolor{currentfill}%
\pgfsetlinewidth{1.003750pt}%
\definecolor{currentstroke}{rgb}{0.000000,0.000000,0.000000}%
\pgfsetstrokecolor{currentstroke}%
\pgfsetdash{}{0pt}%
\pgfpathmoveto{\pgfqpoint{5.504545in}{0.684333in}}%
\pgfpathcurveto{\pgfqpoint{5.515596in}{0.684333in}}{\pgfqpoint{5.526195in}{0.688724in}}{\pgfqpoint{5.534008in}{0.696537in}}%
\pgfpathcurveto{\pgfqpoint{5.541822in}{0.704351in}}{\pgfqpoint{5.546212in}{0.714950in}}{\pgfqpoint{5.546212in}{0.726000in}}%
\pgfpathcurveto{\pgfqpoint{5.546212in}{0.737050in}}{\pgfqpoint{5.541822in}{0.747649in}}{\pgfqpoint{5.534008in}{0.755463in}}%
\pgfpathcurveto{\pgfqpoint{5.526195in}{0.763276in}}{\pgfqpoint{5.515596in}{0.767667in}}{\pgfqpoint{5.504545in}{0.767667in}}%
\pgfpathcurveto{\pgfqpoint{5.493495in}{0.767667in}}{\pgfqpoint{5.482896in}{0.763276in}}{\pgfqpoint{5.475083in}{0.755463in}}%
\pgfpathcurveto{\pgfqpoint{5.467269in}{0.747649in}}{\pgfqpoint{5.462879in}{0.737050in}}{\pgfqpoint{5.462879in}{0.726000in}}%
\pgfpathcurveto{\pgfqpoint{5.462879in}{0.714950in}}{\pgfqpoint{5.467269in}{0.704351in}}{\pgfqpoint{5.475083in}{0.696537in}}%
\pgfpathcurveto{\pgfqpoint{5.482896in}{0.688724in}}{\pgfqpoint{5.493495in}{0.684333in}}{\pgfqpoint{5.504545in}{0.684333in}}%
\pgfpathclose%
\pgfusepath{stroke,fill}%
\end{pgfscope}%
\begin{pgfscope}%
\pgfpathrectangle{\pgfqpoint{0.800000in}{0.528000in}}{\pgfqpoint{4.960000in}{3.696000in}}%
\pgfusepath{clip}%
\pgfsetbuttcap%
\pgfsetroundjoin%
\definecolor{currentfill}{rgb}{0.000000,0.000000,0.000000}%
\pgfsetfillcolor{currentfill}%
\pgfsetlinewidth{1.003750pt}%
\definecolor{currentstroke}{rgb}{0.000000,0.000000,0.000000}%
\pgfsetstrokecolor{currentstroke}%
\pgfsetdash{}{0pt}%
\pgfpathmoveto{\pgfqpoint{5.504545in}{0.684333in}}%
\pgfpathcurveto{\pgfqpoint{5.515596in}{0.684333in}}{\pgfqpoint{5.526195in}{0.688724in}}{\pgfqpoint{5.534008in}{0.696537in}}%
\pgfpathcurveto{\pgfqpoint{5.541822in}{0.704351in}}{\pgfqpoint{5.546212in}{0.714950in}}{\pgfqpoint{5.546212in}{0.726000in}}%
\pgfpathcurveto{\pgfqpoint{5.546212in}{0.737050in}}{\pgfqpoint{5.541822in}{0.747649in}}{\pgfqpoint{5.534008in}{0.755463in}}%
\pgfpathcurveto{\pgfqpoint{5.526195in}{0.763276in}}{\pgfqpoint{5.515596in}{0.767667in}}{\pgfqpoint{5.504545in}{0.767667in}}%
\pgfpathcurveto{\pgfqpoint{5.493495in}{0.767667in}}{\pgfqpoint{5.482896in}{0.763276in}}{\pgfqpoint{5.475083in}{0.755463in}}%
\pgfpathcurveto{\pgfqpoint{5.467269in}{0.747649in}}{\pgfqpoint{5.462879in}{0.737050in}}{\pgfqpoint{5.462879in}{0.726000in}}%
\pgfpathcurveto{\pgfqpoint{5.462879in}{0.714950in}}{\pgfqpoint{5.467269in}{0.704351in}}{\pgfqpoint{5.475083in}{0.696537in}}%
\pgfpathcurveto{\pgfqpoint{5.482896in}{0.688724in}}{\pgfqpoint{5.493495in}{0.684333in}}{\pgfqpoint{5.504545in}{0.684333in}}%
\pgfpathclose%
\pgfusepath{stroke,fill}%
\end{pgfscope}%
\begin{pgfscope}%
\pgfpathrectangle{\pgfqpoint{0.800000in}{0.528000in}}{\pgfqpoint{4.960000in}{3.696000in}}%
\pgfusepath{clip}%
\pgfsetbuttcap%
\pgfsetroundjoin%
\definecolor{currentfill}{rgb}{0.000000,0.000000,0.000000}%
\pgfsetfillcolor{currentfill}%
\pgfsetlinewidth{1.003750pt}%
\definecolor{currentstroke}{rgb}{0.000000,0.000000,0.000000}%
\pgfsetstrokecolor{currentstroke}%
\pgfsetdash{}{0pt}%
\pgfpathmoveto{\pgfqpoint{5.504545in}{0.684333in}}%
\pgfpathcurveto{\pgfqpoint{5.515596in}{0.684333in}}{\pgfqpoint{5.526195in}{0.688724in}}{\pgfqpoint{5.534008in}{0.696537in}}%
\pgfpathcurveto{\pgfqpoint{5.541822in}{0.704351in}}{\pgfqpoint{5.546212in}{0.714950in}}{\pgfqpoint{5.546212in}{0.726000in}}%
\pgfpathcurveto{\pgfqpoint{5.546212in}{0.737050in}}{\pgfqpoint{5.541822in}{0.747649in}}{\pgfqpoint{5.534008in}{0.755463in}}%
\pgfpathcurveto{\pgfqpoint{5.526195in}{0.763276in}}{\pgfqpoint{5.515596in}{0.767667in}}{\pgfqpoint{5.504545in}{0.767667in}}%
\pgfpathcurveto{\pgfqpoint{5.493495in}{0.767667in}}{\pgfqpoint{5.482896in}{0.763276in}}{\pgfqpoint{5.475083in}{0.755463in}}%
\pgfpathcurveto{\pgfqpoint{5.467269in}{0.747649in}}{\pgfqpoint{5.462879in}{0.737050in}}{\pgfqpoint{5.462879in}{0.726000in}}%
\pgfpathcurveto{\pgfqpoint{5.462879in}{0.714950in}}{\pgfqpoint{5.467269in}{0.704351in}}{\pgfqpoint{5.475083in}{0.696537in}}%
\pgfpathcurveto{\pgfqpoint{5.482896in}{0.688724in}}{\pgfqpoint{5.493495in}{0.684333in}}{\pgfqpoint{5.504545in}{0.684333in}}%
\pgfpathclose%
\pgfusepath{stroke,fill}%
\end{pgfscope}%
\begin{pgfscope}%
\pgfpathrectangle{\pgfqpoint{0.800000in}{0.528000in}}{\pgfqpoint{4.960000in}{3.696000in}}%
\pgfusepath{clip}%
\pgfsetbuttcap%
\pgfsetroundjoin%
\definecolor{currentfill}{rgb}{0.000000,0.000000,0.000000}%
\pgfsetfillcolor{currentfill}%
\pgfsetlinewidth{1.003750pt}%
\definecolor{currentstroke}{rgb}{0.000000,0.000000,0.000000}%
\pgfsetstrokecolor{currentstroke}%
\pgfsetdash{}{0pt}%
\pgfpathmoveto{\pgfqpoint{5.504545in}{0.684333in}}%
\pgfpathcurveto{\pgfqpoint{5.515596in}{0.684333in}}{\pgfqpoint{5.526195in}{0.688724in}}{\pgfqpoint{5.534008in}{0.696537in}}%
\pgfpathcurveto{\pgfqpoint{5.541822in}{0.704351in}}{\pgfqpoint{5.546212in}{0.714950in}}{\pgfqpoint{5.546212in}{0.726000in}}%
\pgfpathcurveto{\pgfqpoint{5.546212in}{0.737050in}}{\pgfqpoint{5.541822in}{0.747649in}}{\pgfqpoint{5.534008in}{0.755463in}}%
\pgfpathcurveto{\pgfqpoint{5.526195in}{0.763276in}}{\pgfqpoint{5.515596in}{0.767667in}}{\pgfqpoint{5.504545in}{0.767667in}}%
\pgfpathcurveto{\pgfqpoint{5.493495in}{0.767667in}}{\pgfqpoint{5.482896in}{0.763276in}}{\pgfqpoint{5.475083in}{0.755463in}}%
\pgfpathcurveto{\pgfqpoint{5.467269in}{0.747649in}}{\pgfqpoint{5.462879in}{0.737050in}}{\pgfqpoint{5.462879in}{0.726000in}}%
\pgfpathcurveto{\pgfqpoint{5.462879in}{0.714950in}}{\pgfqpoint{5.467269in}{0.704351in}}{\pgfqpoint{5.475083in}{0.696537in}}%
\pgfpathcurveto{\pgfqpoint{5.482896in}{0.688724in}}{\pgfqpoint{5.493495in}{0.684333in}}{\pgfqpoint{5.504545in}{0.684333in}}%
\pgfpathclose%
\pgfusepath{stroke,fill}%
\end{pgfscope}%
\begin{pgfscope}%
\pgfpathrectangle{\pgfqpoint{0.800000in}{0.528000in}}{\pgfqpoint{4.960000in}{3.696000in}}%
\pgfusepath{clip}%
\pgfsetbuttcap%
\pgfsetroundjoin%
\definecolor{currentfill}{rgb}{0.000000,0.000000,0.000000}%
\pgfsetfillcolor{currentfill}%
\pgfsetlinewidth{1.003750pt}%
\definecolor{currentstroke}{rgb}{0.000000,0.000000,0.000000}%
\pgfsetstrokecolor{currentstroke}%
\pgfsetdash{}{0pt}%
\pgfpathmoveto{\pgfqpoint{5.504545in}{0.684333in}}%
\pgfpathcurveto{\pgfqpoint{5.515596in}{0.684333in}}{\pgfqpoint{5.526195in}{0.688724in}}{\pgfqpoint{5.534008in}{0.696537in}}%
\pgfpathcurveto{\pgfqpoint{5.541822in}{0.704351in}}{\pgfqpoint{5.546212in}{0.714950in}}{\pgfqpoint{5.546212in}{0.726000in}}%
\pgfpathcurveto{\pgfqpoint{5.546212in}{0.737050in}}{\pgfqpoint{5.541822in}{0.747649in}}{\pgfqpoint{5.534008in}{0.755463in}}%
\pgfpathcurveto{\pgfqpoint{5.526195in}{0.763276in}}{\pgfqpoint{5.515596in}{0.767667in}}{\pgfqpoint{5.504545in}{0.767667in}}%
\pgfpathcurveto{\pgfqpoint{5.493495in}{0.767667in}}{\pgfqpoint{5.482896in}{0.763276in}}{\pgfqpoint{5.475083in}{0.755463in}}%
\pgfpathcurveto{\pgfqpoint{5.467269in}{0.747649in}}{\pgfqpoint{5.462879in}{0.737050in}}{\pgfqpoint{5.462879in}{0.726000in}}%
\pgfpathcurveto{\pgfqpoint{5.462879in}{0.714950in}}{\pgfqpoint{5.467269in}{0.704351in}}{\pgfqpoint{5.475083in}{0.696537in}}%
\pgfpathcurveto{\pgfqpoint{5.482896in}{0.688724in}}{\pgfqpoint{5.493495in}{0.684333in}}{\pgfqpoint{5.504545in}{0.684333in}}%
\pgfpathclose%
\pgfusepath{stroke,fill}%
\end{pgfscope}%
\begin{pgfscope}%
\pgfpathrectangle{\pgfqpoint{0.800000in}{0.528000in}}{\pgfqpoint{4.960000in}{3.696000in}}%
\pgfusepath{clip}%
\pgfsetbuttcap%
\pgfsetroundjoin%
\definecolor{currentfill}{rgb}{0.000000,0.000000,0.000000}%
\pgfsetfillcolor{currentfill}%
\pgfsetlinewidth{1.003750pt}%
\definecolor{currentstroke}{rgb}{0.000000,0.000000,0.000000}%
\pgfsetstrokecolor{currentstroke}%
\pgfsetdash{}{0pt}%
\pgfpathmoveto{\pgfqpoint{5.504545in}{0.684333in}}%
\pgfpathcurveto{\pgfqpoint{5.515596in}{0.684333in}}{\pgfqpoint{5.526195in}{0.688724in}}{\pgfqpoint{5.534008in}{0.696537in}}%
\pgfpathcurveto{\pgfqpoint{5.541822in}{0.704351in}}{\pgfqpoint{5.546212in}{0.714950in}}{\pgfqpoint{5.546212in}{0.726000in}}%
\pgfpathcurveto{\pgfqpoint{5.546212in}{0.737050in}}{\pgfqpoint{5.541822in}{0.747649in}}{\pgfqpoint{5.534008in}{0.755463in}}%
\pgfpathcurveto{\pgfqpoint{5.526195in}{0.763276in}}{\pgfqpoint{5.515596in}{0.767667in}}{\pgfqpoint{5.504545in}{0.767667in}}%
\pgfpathcurveto{\pgfqpoint{5.493495in}{0.767667in}}{\pgfqpoint{5.482896in}{0.763276in}}{\pgfqpoint{5.475083in}{0.755463in}}%
\pgfpathcurveto{\pgfqpoint{5.467269in}{0.747649in}}{\pgfqpoint{5.462879in}{0.737050in}}{\pgfqpoint{5.462879in}{0.726000in}}%
\pgfpathcurveto{\pgfqpoint{5.462879in}{0.714950in}}{\pgfqpoint{5.467269in}{0.704351in}}{\pgfqpoint{5.475083in}{0.696537in}}%
\pgfpathcurveto{\pgfqpoint{5.482896in}{0.688724in}}{\pgfqpoint{5.493495in}{0.684333in}}{\pgfqpoint{5.504545in}{0.684333in}}%
\pgfpathclose%
\pgfusepath{stroke,fill}%
\end{pgfscope}%
\begin{pgfscope}%
\pgfpathrectangle{\pgfqpoint{0.800000in}{0.528000in}}{\pgfqpoint{4.960000in}{3.696000in}}%
\pgfusepath{clip}%
\pgfsetbuttcap%
\pgfsetroundjoin%
\definecolor{currentfill}{rgb}{0.000000,0.000000,0.000000}%
\pgfsetfillcolor{currentfill}%
\pgfsetlinewidth{1.003750pt}%
\definecolor{currentstroke}{rgb}{0.000000,0.000000,0.000000}%
\pgfsetstrokecolor{currentstroke}%
\pgfsetdash{}{0pt}%
\pgfpathmoveto{\pgfqpoint{5.504545in}{0.684333in}}%
\pgfpathcurveto{\pgfqpoint{5.515596in}{0.684333in}}{\pgfqpoint{5.526195in}{0.688724in}}{\pgfqpoint{5.534008in}{0.696537in}}%
\pgfpathcurveto{\pgfqpoint{5.541822in}{0.704351in}}{\pgfqpoint{5.546212in}{0.714950in}}{\pgfqpoint{5.546212in}{0.726000in}}%
\pgfpathcurveto{\pgfqpoint{5.546212in}{0.737050in}}{\pgfqpoint{5.541822in}{0.747649in}}{\pgfqpoint{5.534008in}{0.755463in}}%
\pgfpathcurveto{\pgfqpoint{5.526195in}{0.763276in}}{\pgfqpoint{5.515596in}{0.767667in}}{\pgfqpoint{5.504545in}{0.767667in}}%
\pgfpathcurveto{\pgfqpoint{5.493495in}{0.767667in}}{\pgfqpoint{5.482896in}{0.763276in}}{\pgfqpoint{5.475083in}{0.755463in}}%
\pgfpathcurveto{\pgfqpoint{5.467269in}{0.747649in}}{\pgfqpoint{5.462879in}{0.737050in}}{\pgfqpoint{5.462879in}{0.726000in}}%
\pgfpathcurveto{\pgfqpoint{5.462879in}{0.714950in}}{\pgfqpoint{5.467269in}{0.704351in}}{\pgfqpoint{5.475083in}{0.696537in}}%
\pgfpathcurveto{\pgfqpoint{5.482896in}{0.688724in}}{\pgfqpoint{5.493495in}{0.684333in}}{\pgfqpoint{5.504545in}{0.684333in}}%
\pgfpathclose%
\pgfusepath{stroke,fill}%
\end{pgfscope}%
\begin{pgfscope}%
\pgfpathrectangle{\pgfqpoint{0.800000in}{0.528000in}}{\pgfqpoint{4.960000in}{3.696000in}}%
\pgfusepath{clip}%
\pgfsetbuttcap%
\pgfsetroundjoin%
\definecolor{currentfill}{rgb}{0.000000,0.000000,0.000000}%
\pgfsetfillcolor{currentfill}%
\pgfsetlinewidth{1.003750pt}%
\definecolor{currentstroke}{rgb}{0.000000,0.000000,0.000000}%
\pgfsetstrokecolor{currentstroke}%
\pgfsetdash{}{0pt}%
\pgfpathmoveto{\pgfqpoint{5.504545in}{0.684333in}}%
\pgfpathcurveto{\pgfqpoint{5.515596in}{0.684333in}}{\pgfqpoint{5.526195in}{0.688724in}}{\pgfqpoint{5.534008in}{0.696537in}}%
\pgfpathcurveto{\pgfqpoint{5.541822in}{0.704351in}}{\pgfqpoint{5.546212in}{0.714950in}}{\pgfqpoint{5.546212in}{0.726000in}}%
\pgfpathcurveto{\pgfqpoint{5.546212in}{0.737050in}}{\pgfqpoint{5.541822in}{0.747649in}}{\pgfqpoint{5.534008in}{0.755463in}}%
\pgfpathcurveto{\pgfqpoint{5.526195in}{0.763276in}}{\pgfqpoint{5.515596in}{0.767667in}}{\pgfqpoint{5.504545in}{0.767667in}}%
\pgfpathcurveto{\pgfqpoint{5.493495in}{0.767667in}}{\pgfqpoint{5.482896in}{0.763276in}}{\pgfqpoint{5.475083in}{0.755463in}}%
\pgfpathcurveto{\pgfqpoint{5.467269in}{0.747649in}}{\pgfqpoint{5.462879in}{0.737050in}}{\pgfqpoint{5.462879in}{0.726000in}}%
\pgfpathcurveto{\pgfqpoint{5.462879in}{0.714950in}}{\pgfqpoint{5.467269in}{0.704351in}}{\pgfqpoint{5.475083in}{0.696537in}}%
\pgfpathcurveto{\pgfqpoint{5.482896in}{0.688724in}}{\pgfqpoint{5.493495in}{0.684333in}}{\pgfqpoint{5.504545in}{0.684333in}}%
\pgfpathclose%
\pgfusepath{stroke,fill}%
\end{pgfscope}%
\begin{pgfscope}%
\pgfpathrectangle{\pgfqpoint{0.800000in}{0.528000in}}{\pgfqpoint{4.960000in}{3.696000in}}%
\pgfusepath{clip}%
\pgfsetbuttcap%
\pgfsetroundjoin%
\definecolor{currentfill}{rgb}{0.000000,0.000000,0.000000}%
\pgfsetfillcolor{currentfill}%
\pgfsetlinewidth{1.003750pt}%
\definecolor{currentstroke}{rgb}{0.000000,0.000000,0.000000}%
\pgfsetstrokecolor{currentstroke}%
\pgfsetdash{}{0pt}%
\pgfpathmoveto{\pgfqpoint{5.504545in}{0.684333in}}%
\pgfpathcurveto{\pgfqpoint{5.515596in}{0.684333in}}{\pgfqpoint{5.526195in}{0.688724in}}{\pgfqpoint{5.534008in}{0.696537in}}%
\pgfpathcurveto{\pgfqpoint{5.541822in}{0.704351in}}{\pgfqpoint{5.546212in}{0.714950in}}{\pgfqpoint{5.546212in}{0.726000in}}%
\pgfpathcurveto{\pgfqpoint{5.546212in}{0.737050in}}{\pgfqpoint{5.541822in}{0.747649in}}{\pgfqpoint{5.534008in}{0.755463in}}%
\pgfpathcurveto{\pgfqpoint{5.526195in}{0.763276in}}{\pgfqpoint{5.515596in}{0.767667in}}{\pgfqpoint{5.504545in}{0.767667in}}%
\pgfpathcurveto{\pgfqpoint{5.493495in}{0.767667in}}{\pgfqpoint{5.482896in}{0.763276in}}{\pgfqpoint{5.475083in}{0.755463in}}%
\pgfpathcurveto{\pgfqpoint{5.467269in}{0.747649in}}{\pgfqpoint{5.462879in}{0.737050in}}{\pgfqpoint{5.462879in}{0.726000in}}%
\pgfpathcurveto{\pgfqpoint{5.462879in}{0.714950in}}{\pgfqpoint{5.467269in}{0.704351in}}{\pgfqpoint{5.475083in}{0.696537in}}%
\pgfpathcurveto{\pgfqpoint{5.482896in}{0.688724in}}{\pgfqpoint{5.493495in}{0.684333in}}{\pgfqpoint{5.504545in}{0.684333in}}%
\pgfpathclose%
\pgfusepath{stroke,fill}%
\end{pgfscope}%
\begin{pgfscope}%
\pgfpathrectangle{\pgfqpoint{0.800000in}{0.528000in}}{\pgfqpoint{4.960000in}{3.696000in}}%
\pgfusepath{clip}%
\pgfsetbuttcap%
\pgfsetroundjoin%
\definecolor{currentfill}{rgb}{0.000000,0.000000,0.000000}%
\pgfsetfillcolor{currentfill}%
\pgfsetlinewidth{1.003750pt}%
\definecolor{currentstroke}{rgb}{0.000000,0.000000,0.000000}%
\pgfsetstrokecolor{currentstroke}%
\pgfsetdash{}{0pt}%
\pgfpathmoveto{\pgfqpoint{5.504545in}{0.684333in}}%
\pgfpathcurveto{\pgfqpoint{5.515596in}{0.684333in}}{\pgfqpoint{5.526195in}{0.688724in}}{\pgfqpoint{5.534008in}{0.696537in}}%
\pgfpathcurveto{\pgfqpoint{5.541822in}{0.704351in}}{\pgfqpoint{5.546212in}{0.714950in}}{\pgfqpoint{5.546212in}{0.726000in}}%
\pgfpathcurveto{\pgfqpoint{5.546212in}{0.737050in}}{\pgfqpoint{5.541822in}{0.747649in}}{\pgfqpoint{5.534008in}{0.755463in}}%
\pgfpathcurveto{\pgfqpoint{5.526195in}{0.763276in}}{\pgfqpoint{5.515596in}{0.767667in}}{\pgfqpoint{5.504545in}{0.767667in}}%
\pgfpathcurveto{\pgfqpoint{5.493495in}{0.767667in}}{\pgfqpoint{5.482896in}{0.763276in}}{\pgfqpoint{5.475083in}{0.755463in}}%
\pgfpathcurveto{\pgfqpoint{5.467269in}{0.747649in}}{\pgfqpoint{5.462879in}{0.737050in}}{\pgfqpoint{5.462879in}{0.726000in}}%
\pgfpathcurveto{\pgfqpoint{5.462879in}{0.714950in}}{\pgfqpoint{5.467269in}{0.704351in}}{\pgfqpoint{5.475083in}{0.696537in}}%
\pgfpathcurveto{\pgfqpoint{5.482896in}{0.688724in}}{\pgfqpoint{5.493495in}{0.684333in}}{\pgfqpoint{5.504545in}{0.684333in}}%
\pgfpathclose%
\pgfusepath{stroke,fill}%
\end{pgfscope}%
\begin{pgfscope}%
\pgfpathrectangle{\pgfqpoint{0.800000in}{0.528000in}}{\pgfqpoint{4.960000in}{3.696000in}}%
\pgfusepath{clip}%
\pgfsetbuttcap%
\pgfsetroundjoin%
\definecolor{currentfill}{rgb}{0.000000,0.000000,0.000000}%
\pgfsetfillcolor{currentfill}%
\pgfsetlinewidth{1.003750pt}%
\definecolor{currentstroke}{rgb}{0.000000,0.000000,0.000000}%
\pgfsetstrokecolor{currentstroke}%
\pgfsetdash{}{0pt}%
\pgfpathmoveto{\pgfqpoint{5.504545in}{0.684333in}}%
\pgfpathcurveto{\pgfqpoint{5.515596in}{0.684333in}}{\pgfqpoint{5.526195in}{0.688724in}}{\pgfqpoint{5.534008in}{0.696537in}}%
\pgfpathcurveto{\pgfqpoint{5.541822in}{0.704351in}}{\pgfqpoint{5.546212in}{0.714950in}}{\pgfqpoint{5.546212in}{0.726000in}}%
\pgfpathcurveto{\pgfqpoint{5.546212in}{0.737050in}}{\pgfqpoint{5.541822in}{0.747649in}}{\pgfqpoint{5.534008in}{0.755463in}}%
\pgfpathcurveto{\pgfqpoint{5.526195in}{0.763276in}}{\pgfqpoint{5.515596in}{0.767667in}}{\pgfqpoint{5.504545in}{0.767667in}}%
\pgfpathcurveto{\pgfqpoint{5.493495in}{0.767667in}}{\pgfqpoint{5.482896in}{0.763276in}}{\pgfqpoint{5.475083in}{0.755463in}}%
\pgfpathcurveto{\pgfqpoint{5.467269in}{0.747649in}}{\pgfqpoint{5.462879in}{0.737050in}}{\pgfqpoint{5.462879in}{0.726000in}}%
\pgfpathcurveto{\pgfqpoint{5.462879in}{0.714950in}}{\pgfqpoint{5.467269in}{0.704351in}}{\pgfqpoint{5.475083in}{0.696537in}}%
\pgfpathcurveto{\pgfqpoint{5.482896in}{0.688724in}}{\pgfqpoint{5.493495in}{0.684333in}}{\pgfqpoint{5.504545in}{0.684333in}}%
\pgfpathclose%
\pgfusepath{stroke,fill}%
\end{pgfscope}%
\begin{pgfscope}%
\pgfpathrectangle{\pgfqpoint{0.800000in}{0.528000in}}{\pgfqpoint{4.960000in}{3.696000in}}%
\pgfusepath{clip}%
\pgfsetbuttcap%
\pgfsetroundjoin%
\definecolor{currentfill}{rgb}{0.000000,0.000000,0.000000}%
\pgfsetfillcolor{currentfill}%
\pgfsetlinewidth{1.003750pt}%
\definecolor{currentstroke}{rgb}{0.000000,0.000000,0.000000}%
\pgfsetstrokecolor{currentstroke}%
\pgfsetdash{}{0pt}%
\pgfpathmoveto{\pgfqpoint{5.504545in}{0.684333in}}%
\pgfpathcurveto{\pgfqpoint{5.515596in}{0.684333in}}{\pgfqpoint{5.526195in}{0.688724in}}{\pgfqpoint{5.534008in}{0.696537in}}%
\pgfpathcurveto{\pgfqpoint{5.541822in}{0.704351in}}{\pgfqpoint{5.546212in}{0.714950in}}{\pgfqpoint{5.546212in}{0.726000in}}%
\pgfpathcurveto{\pgfqpoint{5.546212in}{0.737050in}}{\pgfqpoint{5.541822in}{0.747649in}}{\pgfqpoint{5.534008in}{0.755463in}}%
\pgfpathcurveto{\pgfqpoint{5.526195in}{0.763276in}}{\pgfqpoint{5.515596in}{0.767667in}}{\pgfqpoint{5.504545in}{0.767667in}}%
\pgfpathcurveto{\pgfqpoint{5.493495in}{0.767667in}}{\pgfqpoint{5.482896in}{0.763276in}}{\pgfqpoint{5.475083in}{0.755463in}}%
\pgfpathcurveto{\pgfqpoint{5.467269in}{0.747649in}}{\pgfqpoint{5.462879in}{0.737050in}}{\pgfqpoint{5.462879in}{0.726000in}}%
\pgfpathcurveto{\pgfqpoint{5.462879in}{0.714950in}}{\pgfqpoint{5.467269in}{0.704351in}}{\pgfqpoint{5.475083in}{0.696537in}}%
\pgfpathcurveto{\pgfqpoint{5.482896in}{0.688724in}}{\pgfqpoint{5.493495in}{0.684333in}}{\pgfqpoint{5.504545in}{0.684333in}}%
\pgfpathclose%
\pgfusepath{stroke,fill}%
\end{pgfscope}%
\begin{pgfscope}%
\pgfpathrectangle{\pgfqpoint{0.800000in}{0.528000in}}{\pgfqpoint{4.960000in}{3.696000in}}%
\pgfusepath{clip}%
\pgfsetbuttcap%
\pgfsetroundjoin%
\definecolor{currentfill}{rgb}{0.000000,0.000000,0.000000}%
\pgfsetfillcolor{currentfill}%
\pgfsetlinewidth{1.003750pt}%
\definecolor{currentstroke}{rgb}{0.000000,0.000000,0.000000}%
\pgfsetstrokecolor{currentstroke}%
\pgfsetdash{}{0pt}%
\pgfpathmoveto{\pgfqpoint{5.504545in}{0.684333in}}%
\pgfpathcurveto{\pgfqpoint{5.515596in}{0.684333in}}{\pgfqpoint{5.526195in}{0.688724in}}{\pgfqpoint{5.534008in}{0.696537in}}%
\pgfpathcurveto{\pgfqpoint{5.541822in}{0.704351in}}{\pgfqpoint{5.546212in}{0.714950in}}{\pgfqpoint{5.546212in}{0.726000in}}%
\pgfpathcurveto{\pgfqpoint{5.546212in}{0.737050in}}{\pgfqpoint{5.541822in}{0.747649in}}{\pgfqpoint{5.534008in}{0.755463in}}%
\pgfpathcurveto{\pgfqpoint{5.526195in}{0.763276in}}{\pgfqpoint{5.515596in}{0.767667in}}{\pgfqpoint{5.504545in}{0.767667in}}%
\pgfpathcurveto{\pgfqpoint{5.493495in}{0.767667in}}{\pgfqpoint{5.482896in}{0.763276in}}{\pgfqpoint{5.475083in}{0.755463in}}%
\pgfpathcurveto{\pgfqpoint{5.467269in}{0.747649in}}{\pgfqpoint{5.462879in}{0.737050in}}{\pgfqpoint{5.462879in}{0.726000in}}%
\pgfpathcurveto{\pgfqpoint{5.462879in}{0.714950in}}{\pgfqpoint{5.467269in}{0.704351in}}{\pgfqpoint{5.475083in}{0.696537in}}%
\pgfpathcurveto{\pgfqpoint{5.482896in}{0.688724in}}{\pgfqpoint{5.493495in}{0.684333in}}{\pgfqpoint{5.504545in}{0.684333in}}%
\pgfpathclose%
\pgfusepath{stroke,fill}%
\end{pgfscope}%
\begin{pgfscope}%
\pgfpathrectangle{\pgfqpoint{0.800000in}{0.528000in}}{\pgfqpoint{4.960000in}{3.696000in}}%
\pgfusepath{clip}%
\pgfsetbuttcap%
\pgfsetroundjoin%
\definecolor{currentfill}{rgb}{0.000000,0.000000,0.000000}%
\pgfsetfillcolor{currentfill}%
\pgfsetlinewidth{1.003750pt}%
\definecolor{currentstroke}{rgb}{0.000000,0.000000,0.000000}%
\pgfsetstrokecolor{currentstroke}%
\pgfsetdash{}{0pt}%
\pgfpathmoveto{\pgfqpoint{5.504545in}{0.684333in}}%
\pgfpathcurveto{\pgfqpoint{5.515596in}{0.684333in}}{\pgfqpoint{5.526195in}{0.688724in}}{\pgfqpoint{5.534008in}{0.696537in}}%
\pgfpathcurveto{\pgfqpoint{5.541822in}{0.704351in}}{\pgfqpoint{5.546212in}{0.714950in}}{\pgfqpoint{5.546212in}{0.726000in}}%
\pgfpathcurveto{\pgfqpoint{5.546212in}{0.737050in}}{\pgfqpoint{5.541822in}{0.747649in}}{\pgfqpoint{5.534008in}{0.755463in}}%
\pgfpathcurveto{\pgfqpoint{5.526195in}{0.763276in}}{\pgfqpoint{5.515596in}{0.767667in}}{\pgfqpoint{5.504545in}{0.767667in}}%
\pgfpathcurveto{\pgfqpoint{5.493495in}{0.767667in}}{\pgfqpoint{5.482896in}{0.763276in}}{\pgfqpoint{5.475083in}{0.755463in}}%
\pgfpathcurveto{\pgfqpoint{5.467269in}{0.747649in}}{\pgfqpoint{5.462879in}{0.737050in}}{\pgfqpoint{5.462879in}{0.726000in}}%
\pgfpathcurveto{\pgfqpoint{5.462879in}{0.714950in}}{\pgfqpoint{5.467269in}{0.704351in}}{\pgfqpoint{5.475083in}{0.696537in}}%
\pgfpathcurveto{\pgfqpoint{5.482896in}{0.688724in}}{\pgfqpoint{5.493495in}{0.684333in}}{\pgfqpoint{5.504545in}{0.684333in}}%
\pgfpathclose%
\pgfusepath{stroke,fill}%
\end{pgfscope}%
\begin{pgfscope}%
\pgfpathrectangle{\pgfqpoint{0.800000in}{0.528000in}}{\pgfqpoint{4.960000in}{3.696000in}}%
\pgfusepath{clip}%
\pgfsetbuttcap%
\pgfsetroundjoin%
\definecolor{currentfill}{rgb}{0.000000,0.000000,0.000000}%
\pgfsetfillcolor{currentfill}%
\pgfsetlinewidth{1.003750pt}%
\definecolor{currentstroke}{rgb}{0.000000,0.000000,0.000000}%
\pgfsetstrokecolor{currentstroke}%
\pgfsetdash{}{0pt}%
\pgfpathmoveto{\pgfqpoint{5.504545in}{0.684333in}}%
\pgfpathcurveto{\pgfqpoint{5.515596in}{0.684333in}}{\pgfqpoint{5.526195in}{0.688724in}}{\pgfqpoint{5.534008in}{0.696537in}}%
\pgfpathcurveto{\pgfqpoint{5.541822in}{0.704351in}}{\pgfqpoint{5.546212in}{0.714950in}}{\pgfqpoint{5.546212in}{0.726000in}}%
\pgfpathcurveto{\pgfqpoint{5.546212in}{0.737050in}}{\pgfqpoint{5.541822in}{0.747649in}}{\pgfqpoint{5.534008in}{0.755463in}}%
\pgfpathcurveto{\pgfqpoint{5.526195in}{0.763276in}}{\pgfqpoint{5.515596in}{0.767667in}}{\pgfqpoint{5.504545in}{0.767667in}}%
\pgfpathcurveto{\pgfqpoint{5.493495in}{0.767667in}}{\pgfqpoint{5.482896in}{0.763276in}}{\pgfqpoint{5.475083in}{0.755463in}}%
\pgfpathcurveto{\pgfqpoint{5.467269in}{0.747649in}}{\pgfqpoint{5.462879in}{0.737050in}}{\pgfqpoint{5.462879in}{0.726000in}}%
\pgfpathcurveto{\pgfqpoint{5.462879in}{0.714950in}}{\pgfqpoint{5.467269in}{0.704351in}}{\pgfqpoint{5.475083in}{0.696537in}}%
\pgfpathcurveto{\pgfqpoint{5.482896in}{0.688724in}}{\pgfqpoint{5.493495in}{0.684333in}}{\pgfqpoint{5.504545in}{0.684333in}}%
\pgfpathclose%
\pgfusepath{stroke,fill}%
\end{pgfscope}%
\begin{pgfscope}%
\pgfpathrectangle{\pgfqpoint{0.800000in}{0.528000in}}{\pgfqpoint{4.960000in}{3.696000in}}%
\pgfusepath{clip}%
\pgfsetbuttcap%
\pgfsetroundjoin%
\definecolor{currentfill}{rgb}{0.000000,0.000000,0.000000}%
\pgfsetfillcolor{currentfill}%
\pgfsetlinewidth{1.003750pt}%
\definecolor{currentstroke}{rgb}{0.000000,0.000000,0.000000}%
\pgfsetstrokecolor{currentstroke}%
\pgfsetdash{}{0pt}%
\pgfpathmoveto{\pgfqpoint{5.504545in}{0.684333in}}%
\pgfpathcurveto{\pgfqpoint{5.515596in}{0.684333in}}{\pgfqpoint{5.526195in}{0.688724in}}{\pgfqpoint{5.534008in}{0.696537in}}%
\pgfpathcurveto{\pgfqpoint{5.541822in}{0.704351in}}{\pgfqpoint{5.546212in}{0.714950in}}{\pgfqpoint{5.546212in}{0.726000in}}%
\pgfpathcurveto{\pgfqpoint{5.546212in}{0.737050in}}{\pgfqpoint{5.541822in}{0.747649in}}{\pgfqpoint{5.534008in}{0.755463in}}%
\pgfpathcurveto{\pgfqpoint{5.526195in}{0.763276in}}{\pgfqpoint{5.515596in}{0.767667in}}{\pgfqpoint{5.504545in}{0.767667in}}%
\pgfpathcurveto{\pgfqpoint{5.493495in}{0.767667in}}{\pgfqpoint{5.482896in}{0.763276in}}{\pgfqpoint{5.475083in}{0.755463in}}%
\pgfpathcurveto{\pgfqpoint{5.467269in}{0.747649in}}{\pgfqpoint{5.462879in}{0.737050in}}{\pgfqpoint{5.462879in}{0.726000in}}%
\pgfpathcurveto{\pgfqpoint{5.462879in}{0.714950in}}{\pgfqpoint{5.467269in}{0.704351in}}{\pgfqpoint{5.475083in}{0.696537in}}%
\pgfpathcurveto{\pgfqpoint{5.482896in}{0.688724in}}{\pgfqpoint{5.493495in}{0.684333in}}{\pgfqpoint{5.504545in}{0.684333in}}%
\pgfpathclose%
\pgfusepath{stroke,fill}%
\end{pgfscope}%
\begin{pgfscope}%
\pgfpathrectangle{\pgfqpoint{0.800000in}{0.528000in}}{\pgfqpoint{4.960000in}{3.696000in}}%
\pgfusepath{clip}%
\pgfsetbuttcap%
\pgfsetroundjoin%
\definecolor{currentfill}{rgb}{0.000000,0.000000,0.000000}%
\pgfsetfillcolor{currentfill}%
\pgfsetlinewidth{1.003750pt}%
\definecolor{currentstroke}{rgb}{0.000000,0.000000,0.000000}%
\pgfsetstrokecolor{currentstroke}%
\pgfsetdash{}{0pt}%
\pgfpathmoveto{\pgfqpoint{5.504545in}{0.684333in}}%
\pgfpathcurveto{\pgfqpoint{5.515596in}{0.684333in}}{\pgfqpoint{5.526195in}{0.688724in}}{\pgfqpoint{5.534008in}{0.696537in}}%
\pgfpathcurveto{\pgfqpoint{5.541822in}{0.704351in}}{\pgfqpoint{5.546212in}{0.714950in}}{\pgfqpoint{5.546212in}{0.726000in}}%
\pgfpathcurveto{\pgfqpoint{5.546212in}{0.737050in}}{\pgfqpoint{5.541822in}{0.747649in}}{\pgfqpoint{5.534008in}{0.755463in}}%
\pgfpathcurveto{\pgfqpoint{5.526195in}{0.763276in}}{\pgfqpoint{5.515596in}{0.767667in}}{\pgfqpoint{5.504545in}{0.767667in}}%
\pgfpathcurveto{\pgfqpoint{5.493495in}{0.767667in}}{\pgfqpoint{5.482896in}{0.763276in}}{\pgfqpoint{5.475083in}{0.755463in}}%
\pgfpathcurveto{\pgfqpoint{5.467269in}{0.747649in}}{\pgfqpoint{5.462879in}{0.737050in}}{\pgfqpoint{5.462879in}{0.726000in}}%
\pgfpathcurveto{\pgfqpoint{5.462879in}{0.714950in}}{\pgfqpoint{5.467269in}{0.704351in}}{\pgfqpoint{5.475083in}{0.696537in}}%
\pgfpathcurveto{\pgfqpoint{5.482896in}{0.688724in}}{\pgfqpoint{5.493495in}{0.684333in}}{\pgfqpoint{5.504545in}{0.684333in}}%
\pgfpathclose%
\pgfusepath{stroke,fill}%
\end{pgfscope}%
\begin{pgfscope}%
\pgfpathrectangle{\pgfqpoint{0.800000in}{0.528000in}}{\pgfqpoint{4.960000in}{3.696000in}}%
\pgfusepath{clip}%
\pgfsetbuttcap%
\pgfsetroundjoin%
\definecolor{currentfill}{rgb}{0.000000,0.000000,0.000000}%
\pgfsetfillcolor{currentfill}%
\pgfsetlinewidth{1.003750pt}%
\definecolor{currentstroke}{rgb}{0.000000,0.000000,0.000000}%
\pgfsetstrokecolor{currentstroke}%
\pgfsetdash{}{0pt}%
\pgfpathmoveto{\pgfqpoint{5.504545in}{0.684333in}}%
\pgfpathcurveto{\pgfqpoint{5.515596in}{0.684333in}}{\pgfqpoint{5.526195in}{0.688724in}}{\pgfqpoint{5.534008in}{0.696537in}}%
\pgfpathcurveto{\pgfqpoint{5.541822in}{0.704351in}}{\pgfqpoint{5.546212in}{0.714950in}}{\pgfqpoint{5.546212in}{0.726000in}}%
\pgfpathcurveto{\pgfqpoint{5.546212in}{0.737050in}}{\pgfqpoint{5.541822in}{0.747649in}}{\pgfqpoint{5.534008in}{0.755463in}}%
\pgfpathcurveto{\pgfqpoint{5.526195in}{0.763276in}}{\pgfqpoint{5.515596in}{0.767667in}}{\pgfqpoint{5.504545in}{0.767667in}}%
\pgfpathcurveto{\pgfqpoint{5.493495in}{0.767667in}}{\pgfqpoint{5.482896in}{0.763276in}}{\pgfqpoint{5.475083in}{0.755463in}}%
\pgfpathcurveto{\pgfqpoint{5.467269in}{0.747649in}}{\pgfqpoint{5.462879in}{0.737050in}}{\pgfqpoint{5.462879in}{0.726000in}}%
\pgfpathcurveto{\pgfqpoint{5.462879in}{0.714950in}}{\pgfqpoint{5.467269in}{0.704351in}}{\pgfqpoint{5.475083in}{0.696537in}}%
\pgfpathcurveto{\pgfqpoint{5.482896in}{0.688724in}}{\pgfqpoint{5.493495in}{0.684333in}}{\pgfqpoint{5.504545in}{0.684333in}}%
\pgfpathclose%
\pgfusepath{stroke,fill}%
\end{pgfscope}%
\begin{pgfscope}%
\pgfpathrectangle{\pgfqpoint{0.800000in}{0.528000in}}{\pgfqpoint{4.960000in}{3.696000in}}%
\pgfusepath{clip}%
\pgfsetbuttcap%
\pgfsetroundjoin%
\definecolor{currentfill}{rgb}{0.000000,0.000000,0.000000}%
\pgfsetfillcolor{currentfill}%
\pgfsetlinewidth{1.003750pt}%
\definecolor{currentstroke}{rgb}{0.000000,0.000000,0.000000}%
\pgfsetstrokecolor{currentstroke}%
\pgfsetdash{}{0pt}%
\pgfpathmoveto{\pgfqpoint{5.504545in}{0.684333in}}%
\pgfpathcurveto{\pgfqpoint{5.515596in}{0.684333in}}{\pgfqpoint{5.526195in}{0.688724in}}{\pgfqpoint{5.534008in}{0.696537in}}%
\pgfpathcurveto{\pgfqpoint{5.541822in}{0.704351in}}{\pgfqpoint{5.546212in}{0.714950in}}{\pgfqpoint{5.546212in}{0.726000in}}%
\pgfpathcurveto{\pgfqpoint{5.546212in}{0.737050in}}{\pgfqpoint{5.541822in}{0.747649in}}{\pgfqpoint{5.534008in}{0.755463in}}%
\pgfpathcurveto{\pgfqpoint{5.526195in}{0.763276in}}{\pgfqpoint{5.515596in}{0.767667in}}{\pgfqpoint{5.504545in}{0.767667in}}%
\pgfpathcurveto{\pgfqpoint{5.493495in}{0.767667in}}{\pgfqpoint{5.482896in}{0.763276in}}{\pgfqpoint{5.475083in}{0.755463in}}%
\pgfpathcurveto{\pgfqpoint{5.467269in}{0.747649in}}{\pgfqpoint{5.462879in}{0.737050in}}{\pgfqpoint{5.462879in}{0.726000in}}%
\pgfpathcurveto{\pgfqpoint{5.462879in}{0.714950in}}{\pgfqpoint{5.467269in}{0.704351in}}{\pgfqpoint{5.475083in}{0.696537in}}%
\pgfpathcurveto{\pgfqpoint{5.482896in}{0.688724in}}{\pgfqpoint{5.493495in}{0.684333in}}{\pgfqpoint{5.504545in}{0.684333in}}%
\pgfpathclose%
\pgfusepath{stroke,fill}%
\end{pgfscope}%
\begin{pgfscope}%
\pgfpathrectangle{\pgfqpoint{0.800000in}{0.528000in}}{\pgfqpoint{4.960000in}{3.696000in}}%
\pgfusepath{clip}%
\pgfsetbuttcap%
\pgfsetroundjoin%
\definecolor{currentfill}{rgb}{0.000000,0.000000,0.000000}%
\pgfsetfillcolor{currentfill}%
\pgfsetlinewidth{1.003750pt}%
\definecolor{currentstroke}{rgb}{0.000000,0.000000,0.000000}%
\pgfsetstrokecolor{currentstroke}%
\pgfsetdash{}{0pt}%
\pgfpathmoveto{\pgfqpoint{5.504545in}{0.684333in}}%
\pgfpathcurveto{\pgfqpoint{5.515596in}{0.684333in}}{\pgfqpoint{5.526195in}{0.688724in}}{\pgfqpoint{5.534008in}{0.696537in}}%
\pgfpathcurveto{\pgfqpoint{5.541822in}{0.704351in}}{\pgfqpoint{5.546212in}{0.714950in}}{\pgfqpoint{5.546212in}{0.726000in}}%
\pgfpathcurveto{\pgfqpoint{5.546212in}{0.737050in}}{\pgfqpoint{5.541822in}{0.747649in}}{\pgfqpoint{5.534008in}{0.755463in}}%
\pgfpathcurveto{\pgfqpoint{5.526195in}{0.763276in}}{\pgfqpoint{5.515596in}{0.767667in}}{\pgfqpoint{5.504545in}{0.767667in}}%
\pgfpathcurveto{\pgfqpoint{5.493495in}{0.767667in}}{\pgfqpoint{5.482896in}{0.763276in}}{\pgfqpoint{5.475083in}{0.755463in}}%
\pgfpathcurveto{\pgfqpoint{5.467269in}{0.747649in}}{\pgfqpoint{5.462879in}{0.737050in}}{\pgfqpoint{5.462879in}{0.726000in}}%
\pgfpathcurveto{\pgfqpoint{5.462879in}{0.714950in}}{\pgfqpoint{5.467269in}{0.704351in}}{\pgfqpoint{5.475083in}{0.696537in}}%
\pgfpathcurveto{\pgfqpoint{5.482896in}{0.688724in}}{\pgfqpoint{5.493495in}{0.684333in}}{\pgfqpoint{5.504545in}{0.684333in}}%
\pgfpathclose%
\pgfusepath{stroke,fill}%
\end{pgfscope}%
\begin{pgfscope}%
\pgfpathrectangle{\pgfqpoint{0.800000in}{0.528000in}}{\pgfqpoint{4.960000in}{3.696000in}}%
\pgfusepath{clip}%
\pgfsetbuttcap%
\pgfsetroundjoin%
\definecolor{currentfill}{rgb}{0.000000,0.000000,0.000000}%
\pgfsetfillcolor{currentfill}%
\pgfsetlinewidth{1.003750pt}%
\definecolor{currentstroke}{rgb}{0.000000,0.000000,0.000000}%
\pgfsetstrokecolor{currentstroke}%
\pgfsetdash{}{0pt}%
\pgfpathmoveto{\pgfqpoint{5.504545in}{0.684333in}}%
\pgfpathcurveto{\pgfqpoint{5.515596in}{0.684333in}}{\pgfqpoint{5.526195in}{0.688724in}}{\pgfqpoint{5.534008in}{0.696537in}}%
\pgfpathcurveto{\pgfqpoint{5.541822in}{0.704351in}}{\pgfqpoint{5.546212in}{0.714950in}}{\pgfqpoint{5.546212in}{0.726000in}}%
\pgfpathcurveto{\pgfqpoint{5.546212in}{0.737050in}}{\pgfqpoint{5.541822in}{0.747649in}}{\pgfqpoint{5.534008in}{0.755463in}}%
\pgfpathcurveto{\pgfqpoint{5.526195in}{0.763276in}}{\pgfqpoint{5.515596in}{0.767667in}}{\pgfqpoint{5.504545in}{0.767667in}}%
\pgfpathcurveto{\pgfqpoint{5.493495in}{0.767667in}}{\pgfqpoint{5.482896in}{0.763276in}}{\pgfqpoint{5.475083in}{0.755463in}}%
\pgfpathcurveto{\pgfqpoint{5.467269in}{0.747649in}}{\pgfqpoint{5.462879in}{0.737050in}}{\pgfqpoint{5.462879in}{0.726000in}}%
\pgfpathcurveto{\pgfqpoint{5.462879in}{0.714950in}}{\pgfqpoint{5.467269in}{0.704351in}}{\pgfqpoint{5.475083in}{0.696537in}}%
\pgfpathcurveto{\pgfqpoint{5.482896in}{0.688724in}}{\pgfqpoint{5.493495in}{0.684333in}}{\pgfqpoint{5.504545in}{0.684333in}}%
\pgfpathclose%
\pgfusepath{stroke,fill}%
\end{pgfscope}%
\begin{pgfscope}%
\pgfpathrectangle{\pgfqpoint{0.800000in}{0.528000in}}{\pgfqpoint{4.960000in}{3.696000in}}%
\pgfusepath{clip}%
\pgfsetbuttcap%
\pgfsetroundjoin%
\definecolor{currentfill}{rgb}{0.000000,0.000000,0.000000}%
\pgfsetfillcolor{currentfill}%
\pgfsetlinewidth{1.003750pt}%
\definecolor{currentstroke}{rgb}{0.000000,0.000000,0.000000}%
\pgfsetstrokecolor{currentstroke}%
\pgfsetdash{}{0pt}%
\pgfpathmoveto{\pgfqpoint{5.504545in}{0.684333in}}%
\pgfpathcurveto{\pgfqpoint{5.515596in}{0.684333in}}{\pgfqpoint{5.526195in}{0.688724in}}{\pgfqpoint{5.534008in}{0.696537in}}%
\pgfpathcurveto{\pgfqpoint{5.541822in}{0.704351in}}{\pgfqpoint{5.546212in}{0.714950in}}{\pgfqpoint{5.546212in}{0.726000in}}%
\pgfpathcurveto{\pgfqpoint{5.546212in}{0.737050in}}{\pgfqpoint{5.541822in}{0.747649in}}{\pgfqpoint{5.534008in}{0.755463in}}%
\pgfpathcurveto{\pgfqpoint{5.526195in}{0.763276in}}{\pgfqpoint{5.515596in}{0.767667in}}{\pgfqpoint{5.504545in}{0.767667in}}%
\pgfpathcurveto{\pgfqpoint{5.493495in}{0.767667in}}{\pgfqpoint{5.482896in}{0.763276in}}{\pgfqpoint{5.475083in}{0.755463in}}%
\pgfpathcurveto{\pgfqpoint{5.467269in}{0.747649in}}{\pgfqpoint{5.462879in}{0.737050in}}{\pgfqpoint{5.462879in}{0.726000in}}%
\pgfpathcurveto{\pgfqpoint{5.462879in}{0.714950in}}{\pgfqpoint{5.467269in}{0.704351in}}{\pgfqpoint{5.475083in}{0.696537in}}%
\pgfpathcurveto{\pgfqpoint{5.482896in}{0.688724in}}{\pgfqpoint{5.493495in}{0.684333in}}{\pgfqpoint{5.504545in}{0.684333in}}%
\pgfpathclose%
\pgfusepath{stroke,fill}%
\end{pgfscope}%
\begin{pgfscope}%
\pgfpathrectangle{\pgfqpoint{0.800000in}{0.528000in}}{\pgfqpoint{4.960000in}{3.696000in}}%
\pgfusepath{clip}%
\pgfsetbuttcap%
\pgfsetroundjoin%
\definecolor{currentfill}{rgb}{0.000000,0.000000,0.000000}%
\pgfsetfillcolor{currentfill}%
\pgfsetlinewidth{1.003750pt}%
\definecolor{currentstroke}{rgb}{0.000000,0.000000,0.000000}%
\pgfsetstrokecolor{currentstroke}%
\pgfsetdash{}{0pt}%
\pgfpathmoveto{\pgfqpoint{5.504545in}{0.684333in}}%
\pgfpathcurveto{\pgfqpoint{5.515596in}{0.684333in}}{\pgfqpoint{5.526195in}{0.688724in}}{\pgfqpoint{5.534008in}{0.696537in}}%
\pgfpathcurveto{\pgfqpoint{5.541822in}{0.704351in}}{\pgfqpoint{5.546212in}{0.714950in}}{\pgfqpoint{5.546212in}{0.726000in}}%
\pgfpathcurveto{\pgfqpoint{5.546212in}{0.737050in}}{\pgfqpoint{5.541822in}{0.747649in}}{\pgfqpoint{5.534008in}{0.755463in}}%
\pgfpathcurveto{\pgfqpoint{5.526195in}{0.763276in}}{\pgfqpoint{5.515596in}{0.767667in}}{\pgfqpoint{5.504545in}{0.767667in}}%
\pgfpathcurveto{\pgfqpoint{5.493495in}{0.767667in}}{\pgfqpoint{5.482896in}{0.763276in}}{\pgfqpoint{5.475083in}{0.755463in}}%
\pgfpathcurveto{\pgfqpoint{5.467269in}{0.747649in}}{\pgfqpoint{5.462879in}{0.737050in}}{\pgfqpoint{5.462879in}{0.726000in}}%
\pgfpathcurveto{\pgfqpoint{5.462879in}{0.714950in}}{\pgfqpoint{5.467269in}{0.704351in}}{\pgfqpoint{5.475083in}{0.696537in}}%
\pgfpathcurveto{\pgfqpoint{5.482896in}{0.688724in}}{\pgfqpoint{5.493495in}{0.684333in}}{\pgfqpoint{5.504545in}{0.684333in}}%
\pgfpathclose%
\pgfusepath{stroke,fill}%
\end{pgfscope}%
\begin{pgfscope}%
\pgfpathrectangle{\pgfqpoint{0.800000in}{0.528000in}}{\pgfqpoint{4.960000in}{3.696000in}}%
\pgfusepath{clip}%
\pgfsetbuttcap%
\pgfsetroundjoin%
\definecolor{currentfill}{rgb}{0.000000,0.000000,0.000000}%
\pgfsetfillcolor{currentfill}%
\pgfsetlinewidth{1.003750pt}%
\definecolor{currentstroke}{rgb}{0.000000,0.000000,0.000000}%
\pgfsetstrokecolor{currentstroke}%
\pgfsetdash{}{0pt}%
\pgfpathmoveto{\pgfqpoint{5.504545in}{0.684333in}}%
\pgfpathcurveto{\pgfqpoint{5.515596in}{0.684333in}}{\pgfqpoint{5.526195in}{0.688724in}}{\pgfqpoint{5.534008in}{0.696537in}}%
\pgfpathcurveto{\pgfqpoint{5.541822in}{0.704351in}}{\pgfqpoint{5.546212in}{0.714950in}}{\pgfqpoint{5.546212in}{0.726000in}}%
\pgfpathcurveto{\pgfqpoint{5.546212in}{0.737050in}}{\pgfqpoint{5.541822in}{0.747649in}}{\pgfqpoint{5.534008in}{0.755463in}}%
\pgfpathcurveto{\pgfqpoint{5.526195in}{0.763276in}}{\pgfqpoint{5.515596in}{0.767667in}}{\pgfqpoint{5.504545in}{0.767667in}}%
\pgfpathcurveto{\pgfqpoint{5.493495in}{0.767667in}}{\pgfqpoint{5.482896in}{0.763276in}}{\pgfqpoint{5.475083in}{0.755463in}}%
\pgfpathcurveto{\pgfqpoint{5.467269in}{0.747649in}}{\pgfqpoint{5.462879in}{0.737050in}}{\pgfqpoint{5.462879in}{0.726000in}}%
\pgfpathcurveto{\pgfqpoint{5.462879in}{0.714950in}}{\pgfqpoint{5.467269in}{0.704351in}}{\pgfqpoint{5.475083in}{0.696537in}}%
\pgfpathcurveto{\pgfqpoint{5.482896in}{0.688724in}}{\pgfqpoint{5.493495in}{0.684333in}}{\pgfqpoint{5.504545in}{0.684333in}}%
\pgfpathclose%
\pgfusepath{stroke,fill}%
\end{pgfscope}%
\begin{pgfscope}%
\pgfpathrectangle{\pgfqpoint{0.800000in}{0.528000in}}{\pgfqpoint{4.960000in}{3.696000in}}%
\pgfusepath{clip}%
\pgfsetbuttcap%
\pgfsetroundjoin%
\definecolor{currentfill}{rgb}{0.000000,0.000000,0.000000}%
\pgfsetfillcolor{currentfill}%
\pgfsetlinewidth{1.003750pt}%
\definecolor{currentstroke}{rgb}{0.000000,0.000000,0.000000}%
\pgfsetstrokecolor{currentstroke}%
\pgfsetdash{}{0pt}%
\pgfpathmoveto{\pgfqpoint{5.504545in}{0.684333in}}%
\pgfpathcurveto{\pgfqpoint{5.515596in}{0.684333in}}{\pgfqpoint{5.526195in}{0.688724in}}{\pgfqpoint{5.534008in}{0.696537in}}%
\pgfpathcurveto{\pgfqpoint{5.541822in}{0.704351in}}{\pgfqpoint{5.546212in}{0.714950in}}{\pgfqpoint{5.546212in}{0.726000in}}%
\pgfpathcurveto{\pgfqpoint{5.546212in}{0.737050in}}{\pgfqpoint{5.541822in}{0.747649in}}{\pgfqpoint{5.534008in}{0.755463in}}%
\pgfpathcurveto{\pgfqpoint{5.526195in}{0.763276in}}{\pgfqpoint{5.515596in}{0.767667in}}{\pgfqpoint{5.504545in}{0.767667in}}%
\pgfpathcurveto{\pgfqpoint{5.493495in}{0.767667in}}{\pgfqpoint{5.482896in}{0.763276in}}{\pgfqpoint{5.475083in}{0.755463in}}%
\pgfpathcurveto{\pgfqpoint{5.467269in}{0.747649in}}{\pgfqpoint{5.462879in}{0.737050in}}{\pgfqpoint{5.462879in}{0.726000in}}%
\pgfpathcurveto{\pgfqpoint{5.462879in}{0.714950in}}{\pgfqpoint{5.467269in}{0.704351in}}{\pgfqpoint{5.475083in}{0.696537in}}%
\pgfpathcurveto{\pgfqpoint{5.482896in}{0.688724in}}{\pgfqpoint{5.493495in}{0.684333in}}{\pgfqpoint{5.504545in}{0.684333in}}%
\pgfpathclose%
\pgfusepath{stroke,fill}%
\end{pgfscope}%
\begin{pgfscope}%
\pgfpathrectangle{\pgfqpoint{0.800000in}{0.528000in}}{\pgfqpoint{4.960000in}{3.696000in}}%
\pgfusepath{clip}%
\pgfsetbuttcap%
\pgfsetroundjoin%
\definecolor{currentfill}{rgb}{0.000000,0.000000,0.000000}%
\pgfsetfillcolor{currentfill}%
\pgfsetlinewidth{1.003750pt}%
\definecolor{currentstroke}{rgb}{0.000000,0.000000,0.000000}%
\pgfsetstrokecolor{currentstroke}%
\pgfsetdash{}{0pt}%
\pgfpathmoveto{\pgfqpoint{5.504545in}{0.684333in}}%
\pgfpathcurveto{\pgfqpoint{5.515596in}{0.684333in}}{\pgfqpoint{5.526195in}{0.688724in}}{\pgfqpoint{5.534008in}{0.696537in}}%
\pgfpathcurveto{\pgfqpoint{5.541822in}{0.704351in}}{\pgfqpoint{5.546212in}{0.714950in}}{\pgfqpoint{5.546212in}{0.726000in}}%
\pgfpathcurveto{\pgfqpoint{5.546212in}{0.737050in}}{\pgfqpoint{5.541822in}{0.747649in}}{\pgfqpoint{5.534008in}{0.755463in}}%
\pgfpathcurveto{\pgfqpoint{5.526195in}{0.763276in}}{\pgfqpoint{5.515596in}{0.767667in}}{\pgfqpoint{5.504545in}{0.767667in}}%
\pgfpathcurveto{\pgfqpoint{5.493495in}{0.767667in}}{\pgfqpoint{5.482896in}{0.763276in}}{\pgfqpoint{5.475083in}{0.755463in}}%
\pgfpathcurveto{\pgfqpoint{5.467269in}{0.747649in}}{\pgfqpoint{5.462879in}{0.737050in}}{\pgfqpoint{5.462879in}{0.726000in}}%
\pgfpathcurveto{\pgfqpoint{5.462879in}{0.714950in}}{\pgfqpoint{5.467269in}{0.704351in}}{\pgfqpoint{5.475083in}{0.696537in}}%
\pgfpathcurveto{\pgfqpoint{5.482896in}{0.688724in}}{\pgfqpoint{5.493495in}{0.684333in}}{\pgfqpoint{5.504545in}{0.684333in}}%
\pgfpathclose%
\pgfusepath{stroke,fill}%
\end{pgfscope}%
\begin{pgfscope}%
\pgfpathrectangle{\pgfqpoint{0.800000in}{0.528000in}}{\pgfqpoint{4.960000in}{3.696000in}}%
\pgfusepath{clip}%
\pgfsetbuttcap%
\pgfsetroundjoin%
\definecolor{currentfill}{rgb}{0.000000,0.000000,0.000000}%
\pgfsetfillcolor{currentfill}%
\pgfsetlinewidth{1.003750pt}%
\definecolor{currentstroke}{rgb}{0.000000,0.000000,0.000000}%
\pgfsetstrokecolor{currentstroke}%
\pgfsetdash{}{0pt}%
\pgfpathmoveto{\pgfqpoint{5.504545in}{0.684333in}}%
\pgfpathcurveto{\pgfqpoint{5.515596in}{0.684333in}}{\pgfqpoint{5.526195in}{0.688724in}}{\pgfqpoint{5.534008in}{0.696537in}}%
\pgfpathcurveto{\pgfqpoint{5.541822in}{0.704351in}}{\pgfqpoint{5.546212in}{0.714950in}}{\pgfqpoint{5.546212in}{0.726000in}}%
\pgfpathcurveto{\pgfqpoint{5.546212in}{0.737050in}}{\pgfqpoint{5.541822in}{0.747649in}}{\pgfqpoint{5.534008in}{0.755463in}}%
\pgfpathcurveto{\pgfqpoint{5.526195in}{0.763276in}}{\pgfqpoint{5.515596in}{0.767667in}}{\pgfqpoint{5.504545in}{0.767667in}}%
\pgfpathcurveto{\pgfqpoint{5.493495in}{0.767667in}}{\pgfqpoint{5.482896in}{0.763276in}}{\pgfqpoint{5.475083in}{0.755463in}}%
\pgfpathcurveto{\pgfqpoint{5.467269in}{0.747649in}}{\pgfqpoint{5.462879in}{0.737050in}}{\pgfqpoint{5.462879in}{0.726000in}}%
\pgfpathcurveto{\pgfqpoint{5.462879in}{0.714950in}}{\pgfqpoint{5.467269in}{0.704351in}}{\pgfqpoint{5.475083in}{0.696537in}}%
\pgfpathcurveto{\pgfqpoint{5.482896in}{0.688724in}}{\pgfqpoint{5.493495in}{0.684333in}}{\pgfqpoint{5.504545in}{0.684333in}}%
\pgfpathclose%
\pgfusepath{stroke,fill}%
\end{pgfscope}%
\begin{pgfscope}%
\pgfpathrectangle{\pgfqpoint{0.800000in}{0.528000in}}{\pgfqpoint{4.960000in}{3.696000in}}%
\pgfusepath{clip}%
\pgfsetbuttcap%
\pgfsetroundjoin%
\definecolor{currentfill}{rgb}{0.000000,0.000000,0.000000}%
\pgfsetfillcolor{currentfill}%
\pgfsetlinewidth{1.003750pt}%
\definecolor{currentstroke}{rgb}{0.000000,0.000000,0.000000}%
\pgfsetstrokecolor{currentstroke}%
\pgfsetdash{}{0pt}%
\pgfpathmoveto{\pgfqpoint{5.504545in}{0.684333in}}%
\pgfpathcurveto{\pgfqpoint{5.515596in}{0.684333in}}{\pgfqpoint{5.526195in}{0.688724in}}{\pgfqpoint{5.534008in}{0.696537in}}%
\pgfpathcurveto{\pgfqpoint{5.541822in}{0.704351in}}{\pgfqpoint{5.546212in}{0.714950in}}{\pgfqpoint{5.546212in}{0.726000in}}%
\pgfpathcurveto{\pgfqpoint{5.546212in}{0.737050in}}{\pgfqpoint{5.541822in}{0.747649in}}{\pgfqpoint{5.534008in}{0.755463in}}%
\pgfpathcurveto{\pgfqpoint{5.526195in}{0.763276in}}{\pgfqpoint{5.515596in}{0.767667in}}{\pgfqpoint{5.504545in}{0.767667in}}%
\pgfpathcurveto{\pgfqpoint{5.493495in}{0.767667in}}{\pgfqpoint{5.482896in}{0.763276in}}{\pgfqpoint{5.475083in}{0.755463in}}%
\pgfpathcurveto{\pgfqpoint{5.467269in}{0.747649in}}{\pgfqpoint{5.462879in}{0.737050in}}{\pgfqpoint{5.462879in}{0.726000in}}%
\pgfpathcurveto{\pgfqpoint{5.462879in}{0.714950in}}{\pgfqpoint{5.467269in}{0.704351in}}{\pgfqpoint{5.475083in}{0.696537in}}%
\pgfpathcurveto{\pgfqpoint{5.482896in}{0.688724in}}{\pgfqpoint{5.493495in}{0.684333in}}{\pgfqpoint{5.504545in}{0.684333in}}%
\pgfpathclose%
\pgfusepath{stroke,fill}%
\end{pgfscope}%
\begin{pgfscope}%
\pgfpathrectangle{\pgfqpoint{0.800000in}{0.528000in}}{\pgfqpoint{4.960000in}{3.696000in}}%
\pgfusepath{clip}%
\pgfsetbuttcap%
\pgfsetroundjoin%
\definecolor{currentfill}{rgb}{0.000000,0.000000,0.000000}%
\pgfsetfillcolor{currentfill}%
\pgfsetlinewidth{1.003750pt}%
\definecolor{currentstroke}{rgb}{0.000000,0.000000,0.000000}%
\pgfsetstrokecolor{currentstroke}%
\pgfsetdash{}{0pt}%
\pgfpathmoveto{\pgfqpoint{5.504545in}{0.684333in}}%
\pgfpathcurveto{\pgfqpoint{5.515596in}{0.684333in}}{\pgfqpoint{5.526195in}{0.688724in}}{\pgfqpoint{5.534008in}{0.696537in}}%
\pgfpathcurveto{\pgfqpoint{5.541822in}{0.704351in}}{\pgfqpoint{5.546212in}{0.714950in}}{\pgfqpoint{5.546212in}{0.726000in}}%
\pgfpathcurveto{\pgfqpoint{5.546212in}{0.737050in}}{\pgfqpoint{5.541822in}{0.747649in}}{\pgfqpoint{5.534008in}{0.755463in}}%
\pgfpathcurveto{\pgfqpoint{5.526195in}{0.763276in}}{\pgfqpoint{5.515596in}{0.767667in}}{\pgfqpoint{5.504545in}{0.767667in}}%
\pgfpathcurveto{\pgfqpoint{5.493495in}{0.767667in}}{\pgfqpoint{5.482896in}{0.763276in}}{\pgfqpoint{5.475083in}{0.755463in}}%
\pgfpathcurveto{\pgfqpoint{5.467269in}{0.747649in}}{\pgfqpoint{5.462879in}{0.737050in}}{\pgfqpoint{5.462879in}{0.726000in}}%
\pgfpathcurveto{\pgfqpoint{5.462879in}{0.714950in}}{\pgfqpoint{5.467269in}{0.704351in}}{\pgfqpoint{5.475083in}{0.696537in}}%
\pgfpathcurveto{\pgfqpoint{5.482896in}{0.688724in}}{\pgfqpoint{5.493495in}{0.684333in}}{\pgfqpoint{5.504545in}{0.684333in}}%
\pgfpathclose%
\pgfusepath{stroke,fill}%
\end{pgfscope}%
\begin{pgfscope}%
\pgfpathrectangle{\pgfqpoint{0.800000in}{0.528000in}}{\pgfqpoint{4.960000in}{3.696000in}}%
\pgfusepath{clip}%
\pgfsetbuttcap%
\pgfsetroundjoin%
\definecolor{currentfill}{rgb}{0.000000,0.000000,0.000000}%
\pgfsetfillcolor{currentfill}%
\pgfsetlinewidth{1.003750pt}%
\definecolor{currentstroke}{rgb}{0.000000,0.000000,0.000000}%
\pgfsetstrokecolor{currentstroke}%
\pgfsetdash{}{0pt}%
\pgfpathmoveto{\pgfqpoint{5.504545in}{0.684333in}}%
\pgfpathcurveto{\pgfqpoint{5.515596in}{0.684333in}}{\pgfqpoint{5.526195in}{0.688724in}}{\pgfqpoint{5.534008in}{0.696537in}}%
\pgfpathcurveto{\pgfqpoint{5.541822in}{0.704351in}}{\pgfqpoint{5.546212in}{0.714950in}}{\pgfqpoint{5.546212in}{0.726000in}}%
\pgfpathcurveto{\pgfqpoint{5.546212in}{0.737050in}}{\pgfqpoint{5.541822in}{0.747649in}}{\pgfqpoint{5.534008in}{0.755463in}}%
\pgfpathcurveto{\pgfqpoint{5.526195in}{0.763276in}}{\pgfqpoint{5.515596in}{0.767667in}}{\pgfqpoint{5.504545in}{0.767667in}}%
\pgfpathcurveto{\pgfqpoint{5.493495in}{0.767667in}}{\pgfqpoint{5.482896in}{0.763276in}}{\pgfqpoint{5.475083in}{0.755463in}}%
\pgfpathcurveto{\pgfqpoint{5.467269in}{0.747649in}}{\pgfqpoint{5.462879in}{0.737050in}}{\pgfqpoint{5.462879in}{0.726000in}}%
\pgfpathcurveto{\pgfqpoint{5.462879in}{0.714950in}}{\pgfqpoint{5.467269in}{0.704351in}}{\pgfqpoint{5.475083in}{0.696537in}}%
\pgfpathcurveto{\pgfqpoint{5.482896in}{0.688724in}}{\pgfqpoint{5.493495in}{0.684333in}}{\pgfqpoint{5.504545in}{0.684333in}}%
\pgfpathclose%
\pgfusepath{stroke,fill}%
\end{pgfscope}%
\begin{pgfscope}%
\pgfpathrectangle{\pgfqpoint{0.800000in}{0.528000in}}{\pgfqpoint{4.960000in}{3.696000in}}%
\pgfusepath{clip}%
\pgfsetbuttcap%
\pgfsetroundjoin%
\definecolor{currentfill}{rgb}{0.000000,0.000000,0.000000}%
\pgfsetfillcolor{currentfill}%
\pgfsetlinewidth{1.003750pt}%
\definecolor{currentstroke}{rgb}{0.000000,0.000000,0.000000}%
\pgfsetstrokecolor{currentstroke}%
\pgfsetdash{}{0pt}%
\pgfpathmoveto{\pgfqpoint{5.504545in}{0.684333in}}%
\pgfpathcurveto{\pgfqpoint{5.515596in}{0.684333in}}{\pgfqpoint{5.526195in}{0.688724in}}{\pgfqpoint{5.534008in}{0.696537in}}%
\pgfpathcurveto{\pgfqpoint{5.541822in}{0.704351in}}{\pgfqpoint{5.546212in}{0.714950in}}{\pgfqpoint{5.546212in}{0.726000in}}%
\pgfpathcurveto{\pgfqpoint{5.546212in}{0.737050in}}{\pgfqpoint{5.541822in}{0.747649in}}{\pgfqpoint{5.534008in}{0.755463in}}%
\pgfpathcurveto{\pgfqpoint{5.526195in}{0.763276in}}{\pgfqpoint{5.515596in}{0.767667in}}{\pgfqpoint{5.504545in}{0.767667in}}%
\pgfpathcurveto{\pgfqpoint{5.493495in}{0.767667in}}{\pgfqpoint{5.482896in}{0.763276in}}{\pgfqpoint{5.475083in}{0.755463in}}%
\pgfpathcurveto{\pgfqpoint{5.467269in}{0.747649in}}{\pgfqpoint{5.462879in}{0.737050in}}{\pgfqpoint{5.462879in}{0.726000in}}%
\pgfpathcurveto{\pgfqpoint{5.462879in}{0.714950in}}{\pgfqpoint{5.467269in}{0.704351in}}{\pgfqpoint{5.475083in}{0.696537in}}%
\pgfpathcurveto{\pgfqpoint{5.482896in}{0.688724in}}{\pgfqpoint{5.493495in}{0.684333in}}{\pgfqpoint{5.504545in}{0.684333in}}%
\pgfpathclose%
\pgfusepath{stroke,fill}%
\end{pgfscope}%
\begin{pgfscope}%
\pgfpathrectangle{\pgfqpoint{0.800000in}{0.528000in}}{\pgfqpoint{4.960000in}{3.696000in}}%
\pgfusepath{clip}%
\pgfsetbuttcap%
\pgfsetroundjoin%
\definecolor{currentfill}{rgb}{0.000000,0.000000,0.000000}%
\pgfsetfillcolor{currentfill}%
\pgfsetlinewidth{1.003750pt}%
\definecolor{currentstroke}{rgb}{0.000000,0.000000,0.000000}%
\pgfsetstrokecolor{currentstroke}%
\pgfsetdash{}{0pt}%
\pgfpathmoveto{\pgfqpoint{5.504545in}{0.684333in}}%
\pgfpathcurveto{\pgfqpoint{5.515596in}{0.684333in}}{\pgfqpoint{5.526195in}{0.688724in}}{\pgfqpoint{5.534008in}{0.696537in}}%
\pgfpathcurveto{\pgfqpoint{5.541822in}{0.704351in}}{\pgfqpoint{5.546212in}{0.714950in}}{\pgfqpoint{5.546212in}{0.726000in}}%
\pgfpathcurveto{\pgfqpoint{5.546212in}{0.737050in}}{\pgfqpoint{5.541822in}{0.747649in}}{\pgfqpoint{5.534008in}{0.755463in}}%
\pgfpathcurveto{\pgfqpoint{5.526195in}{0.763276in}}{\pgfqpoint{5.515596in}{0.767667in}}{\pgfqpoint{5.504545in}{0.767667in}}%
\pgfpathcurveto{\pgfqpoint{5.493495in}{0.767667in}}{\pgfqpoint{5.482896in}{0.763276in}}{\pgfqpoint{5.475083in}{0.755463in}}%
\pgfpathcurveto{\pgfqpoint{5.467269in}{0.747649in}}{\pgfqpoint{5.462879in}{0.737050in}}{\pgfqpoint{5.462879in}{0.726000in}}%
\pgfpathcurveto{\pgfqpoint{5.462879in}{0.714950in}}{\pgfqpoint{5.467269in}{0.704351in}}{\pgfqpoint{5.475083in}{0.696537in}}%
\pgfpathcurveto{\pgfqpoint{5.482896in}{0.688724in}}{\pgfqpoint{5.493495in}{0.684333in}}{\pgfqpoint{5.504545in}{0.684333in}}%
\pgfpathclose%
\pgfusepath{stroke,fill}%
\end{pgfscope}%
\begin{pgfscope}%
\pgfpathrectangle{\pgfqpoint{0.800000in}{0.528000in}}{\pgfqpoint{4.960000in}{3.696000in}}%
\pgfusepath{clip}%
\pgfsetbuttcap%
\pgfsetroundjoin%
\definecolor{currentfill}{rgb}{0.000000,0.000000,0.000000}%
\pgfsetfillcolor{currentfill}%
\pgfsetlinewidth{1.003750pt}%
\definecolor{currentstroke}{rgb}{0.000000,0.000000,0.000000}%
\pgfsetstrokecolor{currentstroke}%
\pgfsetdash{}{0pt}%
\pgfpathmoveto{\pgfqpoint{5.504545in}{0.684333in}}%
\pgfpathcurveto{\pgfqpoint{5.515596in}{0.684333in}}{\pgfqpoint{5.526195in}{0.688724in}}{\pgfqpoint{5.534008in}{0.696537in}}%
\pgfpathcurveto{\pgfqpoint{5.541822in}{0.704351in}}{\pgfqpoint{5.546212in}{0.714950in}}{\pgfqpoint{5.546212in}{0.726000in}}%
\pgfpathcurveto{\pgfqpoint{5.546212in}{0.737050in}}{\pgfqpoint{5.541822in}{0.747649in}}{\pgfqpoint{5.534008in}{0.755463in}}%
\pgfpathcurveto{\pgfqpoint{5.526195in}{0.763276in}}{\pgfqpoint{5.515596in}{0.767667in}}{\pgfqpoint{5.504545in}{0.767667in}}%
\pgfpathcurveto{\pgfqpoint{5.493495in}{0.767667in}}{\pgfqpoint{5.482896in}{0.763276in}}{\pgfqpoint{5.475083in}{0.755463in}}%
\pgfpathcurveto{\pgfqpoint{5.467269in}{0.747649in}}{\pgfqpoint{5.462879in}{0.737050in}}{\pgfqpoint{5.462879in}{0.726000in}}%
\pgfpathcurveto{\pgfqpoint{5.462879in}{0.714950in}}{\pgfqpoint{5.467269in}{0.704351in}}{\pgfqpoint{5.475083in}{0.696537in}}%
\pgfpathcurveto{\pgfqpoint{5.482896in}{0.688724in}}{\pgfqpoint{5.493495in}{0.684333in}}{\pgfqpoint{5.504545in}{0.684333in}}%
\pgfpathclose%
\pgfusepath{stroke,fill}%
\end{pgfscope}%
\begin{pgfscope}%
\pgfpathrectangle{\pgfqpoint{0.800000in}{0.528000in}}{\pgfqpoint{4.960000in}{3.696000in}}%
\pgfusepath{clip}%
\pgfsetbuttcap%
\pgfsetroundjoin%
\definecolor{currentfill}{rgb}{0.000000,0.000000,0.000000}%
\pgfsetfillcolor{currentfill}%
\pgfsetlinewidth{1.003750pt}%
\definecolor{currentstroke}{rgb}{0.000000,0.000000,0.000000}%
\pgfsetstrokecolor{currentstroke}%
\pgfsetdash{}{0pt}%
\pgfpathmoveto{\pgfqpoint{5.504545in}{0.684333in}}%
\pgfpathcurveto{\pgfqpoint{5.515596in}{0.684333in}}{\pgfqpoint{5.526195in}{0.688724in}}{\pgfqpoint{5.534008in}{0.696537in}}%
\pgfpathcurveto{\pgfqpoint{5.541822in}{0.704351in}}{\pgfqpoint{5.546212in}{0.714950in}}{\pgfqpoint{5.546212in}{0.726000in}}%
\pgfpathcurveto{\pgfqpoint{5.546212in}{0.737050in}}{\pgfqpoint{5.541822in}{0.747649in}}{\pgfqpoint{5.534008in}{0.755463in}}%
\pgfpathcurveto{\pgfqpoint{5.526195in}{0.763276in}}{\pgfqpoint{5.515596in}{0.767667in}}{\pgfqpoint{5.504545in}{0.767667in}}%
\pgfpathcurveto{\pgfqpoint{5.493495in}{0.767667in}}{\pgfqpoint{5.482896in}{0.763276in}}{\pgfqpoint{5.475083in}{0.755463in}}%
\pgfpathcurveto{\pgfqpoint{5.467269in}{0.747649in}}{\pgfqpoint{5.462879in}{0.737050in}}{\pgfqpoint{5.462879in}{0.726000in}}%
\pgfpathcurveto{\pgfqpoint{5.462879in}{0.714950in}}{\pgfqpoint{5.467269in}{0.704351in}}{\pgfqpoint{5.475083in}{0.696537in}}%
\pgfpathcurveto{\pgfqpoint{5.482896in}{0.688724in}}{\pgfqpoint{5.493495in}{0.684333in}}{\pgfqpoint{5.504545in}{0.684333in}}%
\pgfpathclose%
\pgfusepath{stroke,fill}%
\end{pgfscope}%
\begin{pgfscope}%
\pgfpathrectangle{\pgfqpoint{0.800000in}{0.528000in}}{\pgfqpoint{4.960000in}{3.696000in}}%
\pgfusepath{clip}%
\pgfsetbuttcap%
\pgfsetroundjoin%
\definecolor{currentfill}{rgb}{0.000000,0.000000,0.000000}%
\pgfsetfillcolor{currentfill}%
\pgfsetlinewidth{1.003750pt}%
\definecolor{currentstroke}{rgb}{0.000000,0.000000,0.000000}%
\pgfsetstrokecolor{currentstroke}%
\pgfsetdash{}{0pt}%
\pgfpathmoveto{\pgfqpoint{5.504545in}{0.684333in}}%
\pgfpathcurveto{\pgfqpoint{5.515596in}{0.684333in}}{\pgfqpoint{5.526195in}{0.688724in}}{\pgfqpoint{5.534008in}{0.696537in}}%
\pgfpathcurveto{\pgfqpoint{5.541822in}{0.704351in}}{\pgfqpoint{5.546212in}{0.714950in}}{\pgfqpoint{5.546212in}{0.726000in}}%
\pgfpathcurveto{\pgfqpoint{5.546212in}{0.737050in}}{\pgfqpoint{5.541822in}{0.747649in}}{\pgfqpoint{5.534008in}{0.755463in}}%
\pgfpathcurveto{\pgfqpoint{5.526195in}{0.763276in}}{\pgfqpoint{5.515596in}{0.767667in}}{\pgfqpoint{5.504545in}{0.767667in}}%
\pgfpathcurveto{\pgfqpoint{5.493495in}{0.767667in}}{\pgfqpoint{5.482896in}{0.763276in}}{\pgfqpoint{5.475083in}{0.755463in}}%
\pgfpathcurveto{\pgfqpoint{5.467269in}{0.747649in}}{\pgfqpoint{5.462879in}{0.737050in}}{\pgfqpoint{5.462879in}{0.726000in}}%
\pgfpathcurveto{\pgfqpoint{5.462879in}{0.714950in}}{\pgfqpoint{5.467269in}{0.704351in}}{\pgfqpoint{5.475083in}{0.696537in}}%
\pgfpathcurveto{\pgfqpoint{5.482896in}{0.688724in}}{\pgfqpoint{5.493495in}{0.684333in}}{\pgfqpoint{5.504545in}{0.684333in}}%
\pgfpathclose%
\pgfusepath{stroke,fill}%
\end{pgfscope}%
\begin{pgfscope}%
\pgfpathrectangle{\pgfqpoint{0.800000in}{0.528000in}}{\pgfqpoint{4.960000in}{3.696000in}}%
\pgfusepath{clip}%
\pgfsetbuttcap%
\pgfsetroundjoin%
\definecolor{currentfill}{rgb}{0.000000,0.000000,0.000000}%
\pgfsetfillcolor{currentfill}%
\pgfsetlinewidth{1.003750pt}%
\definecolor{currentstroke}{rgb}{0.000000,0.000000,0.000000}%
\pgfsetstrokecolor{currentstroke}%
\pgfsetdash{}{0pt}%
\pgfpathmoveto{\pgfqpoint{5.504545in}{0.684333in}}%
\pgfpathcurveto{\pgfqpoint{5.515596in}{0.684333in}}{\pgfqpoint{5.526195in}{0.688724in}}{\pgfqpoint{5.534008in}{0.696537in}}%
\pgfpathcurveto{\pgfqpoint{5.541822in}{0.704351in}}{\pgfqpoint{5.546212in}{0.714950in}}{\pgfqpoint{5.546212in}{0.726000in}}%
\pgfpathcurveto{\pgfqpoint{5.546212in}{0.737050in}}{\pgfqpoint{5.541822in}{0.747649in}}{\pgfqpoint{5.534008in}{0.755463in}}%
\pgfpathcurveto{\pgfqpoint{5.526195in}{0.763276in}}{\pgfqpoint{5.515596in}{0.767667in}}{\pgfqpoint{5.504545in}{0.767667in}}%
\pgfpathcurveto{\pgfqpoint{5.493495in}{0.767667in}}{\pgfqpoint{5.482896in}{0.763276in}}{\pgfqpoint{5.475083in}{0.755463in}}%
\pgfpathcurveto{\pgfqpoint{5.467269in}{0.747649in}}{\pgfqpoint{5.462879in}{0.737050in}}{\pgfqpoint{5.462879in}{0.726000in}}%
\pgfpathcurveto{\pgfqpoint{5.462879in}{0.714950in}}{\pgfqpoint{5.467269in}{0.704351in}}{\pgfqpoint{5.475083in}{0.696537in}}%
\pgfpathcurveto{\pgfqpoint{5.482896in}{0.688724in}}{\pgfqpoint{5.493495in}{0.684333in}}{\pgfqpoint{5.504545in}{0.684333in}}%
\pgfpathclose%
\pgfusepath{stroke,fill}%
\end{pgfscope}%
\begin{pgfscope}%
\pgfpathrectangle{\pgfqpoint{0.800000in}{0.528000in}}{\pgfqpoint{4.960000in}{3.696000in}}%
\pgfusepath{clip}%
\pgfsetbuttcap%
\pgfsetroundjoin%
\definecolor{currentfill}{rgb}{0.000000,0.000000,0.000000}%
\pgfsetfillcolor{currentfill}%
\pgfsetlinewidth{1.003750pt}%
\definecolor{currentstroke}{rgb}{0.000000,0.000000,0.000000}%
\pgfsetstrokecolor{currentstroke}%
\pgfsetdash{}{0pt}%
\pgfpathmoveto{\pgfqpoint{5.504545in}{0.684333in}}%
\pgfpathcurveto{\pgfqpoint{5.515596in}{0.684333in}}{\pgfqpoint{5.526195in}{0.688724in}}{\pgfqpoint{5.534008in}{0.696537in}}%
\pgfpathcurveto{\pgfqpoint{5.541822in}{0.704351in}}{\pgfqpoint{5.546212in}{0.714950in}}{\pgfqpoint{5.546212in}{0.726000in}}%
\pgfpathcurveto{\pgfqpoint{5.546212in}{0.737050in}}{\pgfqpoint{5.541822in}{0.747649in}}{\pgfqpoint{5.534008in}{0.755463in}}%
\pgfpathcurveto{\pgfqpoint{5.526195in}{0.763276in}}{\pgfqpoint{5.515596in}{0.767667in}}{\pgfqpoint{5.504545in}{0.767667in}}%
\pgfpathcurveto{\pgfqpoint{5.493495in}{0.767667in}}{\pgfqpoint{5.482896in}{0.763276in}}{\pgfqpoint{5.475083in}{0.755463in}}%
\pgfpathcurveto{\pgfqpoint{5.467269in}{0.747649in}}{\pgfqpoint{5.462879in}{0.737050in}}{\pgfqpoint{5.462879in}{0.726000in}}%
\pgfpathcurveto{\pgfqpoint{5.462879in}{0.714950in}}{\pgfqpoint{5.467269in}{0.704351in}}{\pgfqpoint{5.475083in}{0.696537in}}%
\pgfpathcurveto{\pgfqpoint{5.482896in}{0.688724in}}{\pgfqpoint{5.493495in}{0.684333in}}{\pgfqpoint{5.504545in}{0.684333in}}%
\pgfpathclose%
\pgfusepath{stroke,fill}%
\end{pgfscope}%
\begin{pgfscope}%
\pgfpathrectangle{\pgfqpoint{0.800000in}{0.528000in}}{\pgfqpoint{4.960000in}{3.696000in}}%
\pgfusepath{clip}%
\pgfsetbuttcap%
\pgfsetroundjoin%
\definecolor{currentfill}{rgb}{0.000000,0.000000,0.000000}%
\pgfsetfillcolor{currentfill}%
\pgfsetlinewidth{1.003750pt}%
\definecolor{currentstroke}{rgb}{0.000000,0.000000,0.000000}%
\pgfsetstrokecolor{currentstroke}%
\pgfsetdash{}{0pt}%
\pgfpathmoveto{\pgfqpoint{5.504545in}{0.684333in}}%
\pgfpathcurveto{\pgfqpoint{5.515596in}{0.684333in}}{\pgfqpoint{5.526195in}{0.688724in}}{\pgfqpoint{5.534008in}{0.696537in}}%
\pgfpathcurveto{\pgfqpoint{5.541822in}{0.704351in}}{\pgfqpoint{5.546212in}{0.714950in}}{\pgfqpoint{5.546212in}{0.726000in}}%
\pgfpathcurveto{\pgfqpoint{5.546212in}{0.737050in}}{\pgfqpoint{5.541822in}{0.747649in}}{\pgfqpoint{5.534008in}{0.755463in}}%
\pgfpathcurveto{\pgfqpoint{5.526195in}{0.763276in}}{\pgfqpoint{5.515596in}{0.767667in}}{\pgfqpoint{5.504545in}{0.767667in}}%
\pgfpathcurveto{\pgfqpoint{5.493495in}{0.767667in}}{\pgfqpoint{5.482896in}{0.763276in}}{\pgfqpoint{5.475083in}{0.755463in}}%
\pgfpathcurveto{\pgfqpoint{5.467269in}{0.747649in}}{\pgfqpoint{5.462879in}{0.737050in}}{\pgfqpoint{5.462879in}{0.726000in}}%
\pgfpathcurveto{\pgfqpoint{5.462879in}{0.714950in}}{\pgfqpoint{5.467269in}{0.704351in}}{\pgfqpoint{5.475083in}{0.696537in}}%
\pgfpathcurveto{\pgfqpoint{5.482896in}{0.688724in}}{\pgfqpoint{5.493495in}{0.684333in}}{\pgfqpoint{5.504545in}{0.684333in}}%
\pgfpathclose%
\pgfusepath{stroke,fill}%
\end{pgfscope}%
\begin{pgfscope}%
\pgfpathrectangle{\pgfqpoint{0.800000in}{0.528000in}}{\pgfqpoint{4.960000in}{3.696000in}}%
\pgfusepath{clip}%
\pgfsetbuttcap%
\pgfsetroundjoin%
\definecolor{currentfill}{rgb}{0.000000,0.000000,0.000000}%
\pgfsetfillcolor{currentfill}%
\pgfsetlinewidth{1.003750pt}%
\definecolor{currentstroke}{rgb}{0.000000,0.000000,0.000000}%
\pgfsetstrokecolor{currentstroke}%
\pgfsetdash{}{0pt}%
\pgfpathmoveto{\pgfqpoint{5.504545in}{0.684333in}}%
\pgfpathcurveto{\pgfqpoint{5.515596in}{0.684333in}}{\pgfqpoint{5.526195in}{0.688724in}}{\pgfqpoint{5.534008in}{0.696537in}}%
\pgfpathcurveto{\pgfqpoint{5.541822in}{0.704351in}}{\pgfqpoint{5.546212in}{0.714950in}}{\pgfqpoint{5.546212in}{0.726000in}}%
\pgfpathcurveto{\pgfqpoint{5.546212in}{0.737050in}}{\pgfqpoint{5.541822in}{0.747649in}}{\pgfqpoint{5.534008in}{0.755463in}}%
\pgfpathcurveto{\pgfqpoint{5.526195in}{0.763276in}}{\pgfqpoint{5.515596in}{0.767667in}}{\pgfqpoint{5.504545in}{0.767667in}}%
\pgfpathcurveto{\pgfqpoint{5.493495in}{0.767667in}}{\pgfqpoint{5.482896in}{0.763276in}}{\pgfqpoint{5.475083in}{0.755463in}}%
\pgfpathcurveto{\pgfqpoint{5.467269in}{0.747649in}}{\pgfqpoint{5.462879in}{0.737050in}}{\pgfqpoint{5.462879in}{0.726000in}}%
\pgfpathcurveto{\pgfqpoint{5.462879in}{0.714950in}}{\pgfqpoint{5.467269in}{0.704351in}}{\pgfqpoint{5.475083in}{0.696537in}}%
\pgfpathcurveto{\pgfqpoint{5.482896in}{0.688724in}}{\pgfqpoint{5.493495in}{0.684333in}}{\pgfqpoint{5.504545in}{0.684333in}}%
\pgfpathclose%
\pgfusepath{stroke,fill}%
\end{pgfscope}%
\begin{pgfscope}%
\pgfpathrectangle{\pgfqpoint{0.800000in}{0.528000in}}{\pgfqpoint{4.960000in}{3.696000in}}%
\pgfusepath{clip}%
\pgfsetbuttcap%
\pgfsetroundjoin%
\definecolor{currentfill}{rgb}{0.000000,0.000000,0.000000}%
\pgfsetfillcolor{currentfill}%
\pgfsetlinewidth{1.003750pt}%
\definecolor{currentstroke}{rgb}{0.000000,0.000000,0.000000}%
\pgfsetstrokecolor{currentstroke}%
\pgfsetdash{}{0pt}%
\pgfpathmoveto{\pgfqpoint{5.504545in}{0.684333in}}%
\pgfpathcurveto{\pgfqpoint{5.515596in}{0.684333in}}{\pgfqpoint{5.526195in}{0.688724in}}{\pgfqpoint{5.534008in}{0.696537in}}%
\pgfpathcurveto{\pgfqpoint{5.541822in}{0.704351in}}{\pgfqpoint{5.546212in}{0.714950in}}{\pgfqpoint{5.546212in}{0.726000in}}%
\pgfpathcurveto{\pgfqpoint{5.546212in}{0.737050in}}{\pgfqpoint{5.541822in}{0.747649in}}{\pgfqpoint{5.534008in}{0.755463in}}%
\pgfpathcurveto{\pgfqpoint{5.526195in}{0.763276in}}{\pgfqpoint{5.515596in}{0.767667in}}{\pgfqpoint{5.504545in}{0.767667in}}%
\pgfpathcurveto{\pgfqpoint{5.493495in}{0.767667in}}{\pgfqpoint{5.482896in}{0.763276in}}{\pgfqpoint{5.475083in}{0.755463in}}%
\pgfpathcurveto{\pgfqpoint{5.467269in}{0.747649in}}{\pgfqpoint{5.462879in}{0.737050in}}{\pgfqpoint{5.462879in}{0.726000in}}%
\pgfpathcurveto{\pgfqpoint{5.462879in}{0.714950in}}{\pgfqpoint{5.467269in}{0.704351in}}{\pgfqpoint{5.475083in}{0.696537in}}%
\pgfpathcurveto{\pgfqpoint{5.482896in}{0.688724in}}{\pgfqpoint{5.493495in}{0.684333in}}{\pgfqpoint{5.504545in}{0.684333in}}%
\pgfpathclose%
\pgfusepath{stroke,fill}%
\end{pgfscope}%
\begin{pgfscope}%
\pgfpathrectangle{\pgfqpoint{0.800000in}{0.528000in}}{\pgfqpoint{4.960000in}{3.696000in}}%
\pgfusepath{clip}%
\pgfsetbuttcap%
\pgfsetroundjoin%
\definecolor{currentfill}{rgb}{0.000000,0.000000,0.000000}%
\pgfsetfillcolor{currentfill}%
\pgfsetlinewidth{1.003750pt}%
\definecolor{currentstroke}{rgb}{0.000000,0.000000,0.000000}%
\pgfsetstrokecolor{currentstroke}%
\pgfsetdash{}{0pt}%
\pgfpathmoveto{\pgfqpoint{5.504545in}{0.684333in}}%
\pgfpathcurveto{\pgfqpoint{5.515596in}{0.684333in}}{\pgfqpoint{5.526195in}{0.688724in}}{\pgfqpoint{5.534008in}{0.696537in}}%
\pgfpathcurveto{\pgfqpoint{5.541822in}{0.704351in}}{\pgfqpoint{5.546212in}{0.714950in}}{\pgfqpoint{5.546212in}{0.726000in}}%
\pgfpathcurveto{\pgfqpoint{5.546212in}{0.737050in}}{\pgfqpoint{5.541822in}{0.747649in}}{\pgfqpoint{5.534008in}{0.755463in}}%
\pgfpathcurveto{\pgfqpoint{5.526195in}{0.763276in}}{\pgfqpoint{5.515596in}{0.767667in}}{\pgfqpoint{5.504545in}{0.767667in}}%
\pgfpathcurveto{\pgfqpoint{5.493495in}{0.767667in}}{\pgfqpoint{5.482896in}{0.763276in}}{\pgfqpoint{5.475083in}{0.755463in}}%
\pgfpathcurveto{\pgfqpoint{5.467269in}{0.747649in}}{\pgfqpoint{5.462879in}{0.737050in}}{\pgfqpoint{5.462879in}{0.726000in}}%
\pgfpathcurveto{\pgfqpoint{5.462879in}{0.714950in}}{\pgfqpoint{5.467269in}{0.704351in}}{\pgfqpoint{5.475083in}{0.696537in}}%
\pgfpathcurveto{\pgfqpoint{5.482896in}{0.688724in}}{\pgfqpoint{5.493495in}{0.684333in}}{\pgfqpoint{5.504545in}{0.684333in}}%
\pgfpathclose%
\pgfusepath{stroke,fill}%
\end{pgfscope}%
\begin{pgfscope}%
\pgfpathrectangle{\pgfqpoint{0.800000in}{0.528000in}}{\pgfqpoint{4.960000in}{3.696000in}}%
\pgfusepath{clip}%
\pgfsetbuttcap%
\pgfsetroundjoin%
\definecolor{currentfill}{rgb}{0.000000,0.000000,0.000000}%
\pgfsetfillcolor{currentfill}%
\pgfsetlinewidth{1.003750pt}%
\definecolor{currentstroke}{rgb}{0.000000,0.000000,0.000000}%
\pgfsetstrokecolor{currentstroke}%
\pgfsetdash{}{0pt}%
\pgfpathmoveto{\pgfqpoint{5.504545in}{0.684333in}}%
\pgfpathcurveto{\pgfqpoint{5.515596in}{0.684333in}}{\pgfqpoint{5.526195in}{0.688724in}}{\pgfqpoint{5.534008in}{0.696537in}}%
\pgfpathcurveto{\pgfqpoint{5.541822in}{0.704351in}}{\pgfqpoint{5.546212in}{0.714950in}}{\pgfqpoint{5.546212in}{0.726000in}}%
\pgfpathcurveto{\pgfqpoint{5.546212in}{0.737050in}}{\pgfqpoint{5.541822in}{0.747649in}}{\pgfqpoint{5.534008in}{0.755463in}}%
\pgfpathcurveto{\pgfqpoint{5.526195in}{0.763276in}}{\pgfqpoint{5.515596in}{0.767667in}}{\pgfqpoint{5.504545in}{0.767667in}}%
\pgfpathcurveto{\pgfqpoint{5.493495in}{0.767667in}}{\pgfqpoint{5.482896in}{0.763276in}}{\pgfqpoint{5.475083in}{0.755463in}}%
\pgfpathcurveto{\pgfqpoint{5.467269in}{0.747649in}}{\pgfqpoint{5.462879in}{0.737050in}}{\pgfqpoint{5.462879in}{0.726000in}}%
\pgfpathcurveto{\pgfqpoint{5.462879in}{0.714950in}}{\pgfqpoint{5.467269in}{0.704351in}}{\pgfqpoint{5.475083in}{0.696537in}}%
\pgfpathcurveto{\pgfqpoint{5.482896in}{0.688724in}}{\pgfqpoint{5.493495in}{0.684333in}}{\pgfqpoint{5.504545in}{0.684333in}}%
\pgfpathclose%
\pgfusepath{stroke,fill}%
\end{pgfscope}%
\begin{pgfscope}%
\pgfpathrectangle{\pgfqpoint{0.800000in}{0.528000in}}{\pgfqpoint{4.960000in}{3.696000in}}%
\pgfusepath{clip}%
\pgfsetbuttcap%
\pgfsetroundjoin%
\definecolor{currentfill}{rgb}{0.000000,0.000000,0.000000}%
\pgfsetfillcolor{currentfill}%
\pgfsetlinewidth{1.003750pt}%
\definecolor{currentstroke}{rgb}{0.000000,0.000000,0.000000}%
\pgfsetstrokecolor{currentstroke}%
\pgfsetdash{}{0pt}%
\pgfpathmoveto{\pgfqpoint{5.504545in}{0.684333in}}%
\pgfpathcurveto{\pgfqpoint{5.515596in}{0.684333in}}{\pgfqpoint{5.526195in}{0.688724in}}{\pgfqpoint{5.534008in}{0.696537in}}%
\pgfpathcurveto{\pgfqpoint{5.541822in}{0.704351in}}{\pgfqpoint{5.546212in}{0.714950in}}{\pgfqpoint{5.546212in}{0.726000in}}%
\pgfpathcurveto{\pgfqpoint{5.546212in}{0.737050in}}{\pgfqpoint{5.541822in}{0.747649in}}{\pgfqpoint{5.534008in}{0.755463in}}%
\pgfpathcurveto{\pgfqpoint{5.526195in}{0.763276in}}{\pgfqpoint{5.515596in}{0.767667in}}{\pgfqpoint{5.504545in}{0.767667in}}%
\pgfpathcurveto{\pgfqpoint{5.493495in}{0.767667in}}{\pgfqpoint{5.482896in}{0.763276in}}{\pgfqpoint{5.475083in}{0.755463in}}%
\pgfpathcurveto{\pgfqpoint{5.467269in}{0.747649in}}{\pgfqpoint{5.462879in}{0.737050in}}{\pgfqpoint{5.462879in}{0.726000in}}%
\pgfpathcurveto{\pgfqpoint{5.462879in}{0.714950in}}{\pgfqpoint{5.467269in}{0.704351in}}{\pgfqpoint{5.475083in}{0.696537in}}%
\pgfpathcurveto{\pgfqpoint{5.482896in}{0.688724in}}{\pgfqpoint{5.493495in}{0.684333in}}{\pgfqpoint{5.504545in}{0.684333in}}%
\pgfpathclose%
\pgfusepath{stroke,fill}%
\end{pgfscope}%
\begin{pgfscope}%
\pgfpathrectangle{\pgfqpoint{0.800000in}{0.528000in}}{\pgfqpoint{4.960000in}{3.696000in}}%
\pgfusepath{clip}%
\pgfsetbuttcap%
\pgfsetroundjoin%
\definecolor{currentfill}{rgb}{0.000000,0.000000,0.000000}%
\pgfsetfillcolor{currentfill}%
\pgfsetlinewidth{1.003750pt}%
\definecolor{currentstroke}{rgb}{0.000000,0.000000,0.000000}%
\pgfsetstrokecolor{currentstroke}%
\pgfsetdash{}{0pt}%
\pgfpathmoveto{\pgfqpoint{5.504545in}{0.684333in}}%
\pgfpathcurveto{\pgfqpoint{5.515596in}{0.684333in}}{\pgfqpoint{5.526195in}{0.688724in}}{\pgfqpoint{5.534008in}{0.696537in}}%
\pgfpathcurveto{\pgfqpoint{5.541822in}{0.704351in}}{\pgfqpoint{5.546212in}{0.714950in}}{\pgfqpoint{5.546212in}{0.726000in}}%
\pgfpathcurveto{\pgfqpoint{5.546212in}{0.737050in}}{\pgfqpoint{5.541822in}{0.747649in}}{\pgfqpoint{5.534008in}{0.755463in}}%
\pgfpathcurveto{\pgfqpoint{5.526195in}{0.763276in}}{\pgfqpoint{5.515596in}{0.767667in}}{\pgfqpoint{5.504545in}{0.767667in}}%
\pgfpathcurveto{\pgfqpoint{5.493495in}{0.767667in}}{\pgfqpoint{5.482896in}{0.763276in}}{\pgfqpoint{5.475083in}{0.755463in}}%
\pgfpathcurveto{\pgfqpoint{5.467269in}{0.747649in}}{\pgfqpoint{5.462879in}{0.737050in}}{\pgfqpoint{5.462879in}{0.726000in}}%
\pgfpathcurveto{\pgfqpoint{5.462879in}{0.714950in}}{\pgfqpoint{5.467269in}{0.704351in}}{\pgfqpoint{5.475083in}{0.696537in}}%
\pgfpathcurveto{\pgfqpoint{5.482896in}{0.688724in}}{\pgfqpoint{5.493495in}{0.684333in}}{\pgfqpoint{5.504545in}{0.684333in}}%
\pgfpathclose%
\pgfusepath{stroke,fill}%
\end{pgfscope}%
\begin{pgfscope}%
\pgfpathrectangle{\pgfqpoint{0.800000in}{0.528000in}}{\pgfqpoint{4.960000in}{3.696000in}}%
\pgfusepath{clip}%
\pgfsetbuttcap%
\pgfsetroundjoin%
\definecolor{currentfill}{rgb}{0.000000,0.000000,0.000000}%
\pgfsetfillcolor{currentfill}%
\pgfsetlinewidth{1.003750pt}%
\definecolor{currentstroke}{rgb}{0.000000,0.000000,0.000000}%
\pgfsetstrokecolor{currentstroke}%
\pgfsetdash{}{0pt}%
\pgfpathmoveto{\pgfqpoint{5.504545in}{0.684333in}}%
\pgfpathcurveto{\pgfqpoint{5.515596in}{0.684333in}}{\pgfqpoint{5.526195in}{0.688724in}}{\pgfqpoint{5.534008in}{0.696537in}}%
\pgfpathcurveto{\pgfqpoint{5.541822in}{0.704351in}}{\pgfqpoint{5.546212in}{0.714950in}}{\pgfqpoint{5.546212in}{0.726000in}}%
\pgfpathcurveto{\pgfqpoint{5.546212in}{0.737050in}}{\pgfqpoint{5.541822in}{0.747649in}}{\pgfqpoint{5.534008in}{0.755463in}}%
\pgfpathcurveto{\pgfqpoint{5.526195in}{0.763276in}}{\pgfqpoint{5.515596in}{0.767667in}}{\pgfqpoint{5.504545in}{0.767667in}}%
\pgfpathcurveto{\pgfqpoint{5.493495in}{0.767667in}}{\pgfqpoint{5.482896in}{0.763276in}}{\pgfqpoint{5.475083in}{0.755463in}}%
\pgfpathcurveto{\pgfqpoint{5.467269in}{0.747649in}}{\pgfqpoint{5.462879in}{0.737050in}}{\pgfqpoint{5.462879in}{0.726000in}}%
\pgfpathcurveto{\pgfqpoint{5.462879in}{0.714950in}}{\pgfqpoint{5.467269in}{0.704351in}}{\pgfqpoint{5.475083in}{0.696537in}}%
\pgfpathcurveto{\pgfqpoint{5.482896in}{0.688724in}}{\pgfqpoint{5.493495in}{0.684333in}}{\pgfqpoint{5.504545in}{0.684333in}}%
\pgfpathclose%
\pgfusepath{stroke,fill}%
\end{pgfscope}%
\begin{pgfscope}%
\pgfpathrectangle{\pgfqpoint{0.800000in}{0.528000in}}{\pgfqpoint{4.960000in}{3.696000in}}%
\pgfusepath{clip}%
\pgfsetbuttcap%
\pgfsetroundjoin%
\definecolor{currentfill}{rgb}{0.000000,0.000000,0.000000}%
\pgfsetfillcolor{currentfill}%
\pgfsetlinewidth{1.003750pt}%
\definecolor{currentstroke}{rgb}{0.000000,0.000000,0.000000}%
\pgfsetstrokecolor{currentstroke}%
\pgfsetdash{}{0pt}%
\pgfpathmoveto{\pgfqpoint{5.504545in}{0.684333in}}%
\pgfpathcurveto{\pgfqpoint{5.515596in}{0.684333in}}{\pgfqpoint{5.526195in}{0.688724in}}{\pgfqpoint{5.534008in}{0.696537in}}%
\pgfpathcurveto{\pgfqpoint{5.541822in}{0.704351in}}{\pgfqpoint{5.546212in}{0.714950in}}{\pgfqpoint{5.546212in}{0.726000in}}%
\pgfpathcurveto{\pgfqpoint{5.546212in}{0.737050in}}{\pgfqpoint{5.541822in}{0.747649in}}{\pgfqpoint{5.534008in}{0.755463in}}%
\pgfpathcurveto{\pgfqpoint{5.526195in}{0.763276in}}{\pgfqpoint{5.515596in}{0.767667in}}{\pgfqpoint{5.504545in}{0.767667in}}%
\pgfpathcurveto{\pgfqpoint{5.493495in}{0.767667in}}{\pgfqpoint{5.482896in}{0.763276in}}{\pgfqpoint{5.475083in}{0.755463in}}%
\pgfpathcurveto{\pgfqpoint{5.467269in}{0.747649in}}{\pgfqpoint{5.462879in}{0.737050in}}{\pgfqpoint{5.462879in}{0.726000in}}%
\pgfpathcurveto{\pgfqpoint{5.462879in}{0.714950in}}{\pgfqpoint{5.467269in}{0.704351in}}{\pgfqpoint{5.475083in}{0.696537in}}%
\pgfpathcurveto{\pgfqpoint{5.482896in}{0.688724in}}{\pgfqpoint{5.493495in}{0.684333in}}{\pgfqpoint{5.504545in}{0.684333in}}%
\pgfpathclose%
\pgfusepath{stroke,fill}%
\end{pgfscope}%
\begin{pgfscope}%
\pgfpathrectangle{\pgfqpoint{0.800000in}{0.528000in}}{\pgfqpoint{4.960000in}{3.696000in}}%
\pgfusepath{clip}%
\pgfsetbuttcap%
\pgfsetroundjoin%
\definecolor{currentfill}{rgb}{0.000000,0.000000,0.000000}%
\pgfsetfillcolor{currentfill}%
\pgfsetlinewidth{1.003750pt}%
\definecolor{currentstroke}{rgb}{0.000000,0.000000,0.000000}%
\pgfsetstrokecolor{currentstroke}%
\pgfsetdash{}{0pt}%
\pgfpathmoveto{\pgfqpoint{5.504545in}{0.684333in}}%
\pgfpathcurveto{\pgfqpoint{5.515596in}{0.684333in}}{\pgfqpoint{5.526195in}{0.688724in}}{\pgfqpoint{5.534008in}{0.696537in}}%
\pgfpathcurveto{\pgfqpoint{5.541822in}{0.704351in}}{\pgfqpoint{5.546212in}{0.714950in}}{\pgfqpoint{5.546212in}{0.726000in}}%
\pgfpathcurveto{\pgfqpoint{5.546212in}{0.737050in}}{\pgfqpoint{5.541822in}{0.747649in}}{\pgfqpoint{5.534008in}{0.755463in}}%
\pgfpathcurveto{\pgfqpoint{5.526195in}{0.763276in}}{\pgfqpoint{5.515596in}{0.767667in}}{\pgfqpoint{5.504545in}{0.767667in}}%
\pgfpathcurveto{\pgfqpoint{5.493495in}{0.767667in}}{\pgfqpoint{5.482896in}{0.763276in}}{\pgfqpoint{5.475083in}{0.755463in}}%
\pgfpathcurveto{\pgfqpoint{5.467269in}{0.747649in}}{\pgfqpoint{5.462879in}{0.737050in}}{\pgfqpoint{5.462879in}{0.726000in}}%
\pgfpathcurveto{\pgfqpoint{5.462879in}{0.714950in}}{\pgfqpoint{5.467269in}{0.704351in}}{\pgfqpoint{5.475083in}{0.696537in}}%
\pgfpathcurveto{\pgfqpoint{5.482896in}{0.688724in}}{\pgfqpoint{5.493495in}{0.684333in}}{\pgfqpoint{5.504545in}{0.684333in}}%
\pgfpathclose%
\pgfusepath{stroke,fill}%
\end{pgfscope}%
\begin{pgfscope}%
\pgfpathrectangle{\pgfqpoint{0.800000in}{0.528000in}}{\pgfqpoint{4.960000in}{3.696000in}}%
\pgfusepath{clip}%
\pgfsetbuttcap%
\pgfsetroundjoin%
\definecolor{currentfill}{rgb}{0.000000,0.000000,0.000000}%
\pgfsetfillcolor{currentfill}%
\pgfsetlinewidth{1.003750pt}%
\definecolor{currentstroke}{rgb}{0.000000,0.000000,0.000000}%
\pgfsetstrokecolor{currentstroke}%
\pgfsetdash{}{0pt}%
\pgfpathmoveto{\pgfqpoint{5.504545in}{0.684333in}}%
\pgfpathcurveto{\pgfqpoint{5.515596in}{0.684333in}}{\pgfqpoint{5.526195in}{0.688724in}}{\pgfqpoint{5.534008in}{0.696537in}}%
\pgfpathcurveto{\pgfqpoint{5.541822in}{0.704351in}}{\pgfqpoint{5.546212in}{0.714950in}}{\pgfqpoint{5.546212in}{0.726000in}}%
\pgfpathcurveto{\pgfqpoint{5.546212in}{0.737050in}}{\pgfqpoint{5.541822in}{0.747649in}}{\pgfqpoint{5.534008in}{0.755463in}}%
\pgfpathcurveto{\pgfqpoint{5.526195in}{0.763276in}}{\pgfqpoint{5.515596in}{0.767667in}}{\pgfqpoint{5.504545in}{0.767667in}}%
\pgfpathcurveto{\pgfqpoint{5.493495in}{0.767667in}}{\pgfqpoint{5.482896in}{0.763276in}}{\pgfqpoint{5.475083in}{0.755463in}}%
\pgfpathcurveto{\pgfqpoint{5.467269in}{0.747649in}}{\pgfqpoint{5.462879in}{0.737050in}}{\pgfqpoint{5.462879in}{0.726000in}}%
\pgfpathcurveto{\pgfqpoint{5.462879in}{0.714950in}}{\pgfqpoint{5.467269in}{0.704351in}}{\pgfqpoint{5.475083in}{0.696537in}}%
\pgfpathcurveto{\pgfqpoint{5.482896in}{0.688724in}}{\pgfqpoint{5.493495in}{0.684333in}}{\pgfqpoint{5.504545in}{0.684333in}}%
\pgfpathclose%
\pgfusepath{stroke,fill}%
\end{pgfscope}%
\begin{pgfscope}%
\pgfpathrectangle{\pgfqpoint{0.800000in}{0.528000in}}{\pgfqpoint{4.960000in}{3.696000in}}%
\pgfusepath{clip}%
\pgfsetbuttcap%
\pgfsetroundjoin%
\definecolor{currentfill}{rgb}{0.000000,0.000000,0.000000}%
\pgfsetfillcolor{currentfill}%
\pgfsetlinewidth{1.003750pt}%
\definecolor{currentstroke}{rgb}{0.000000,0.000000,0.000000}%
\pgfsetstrokecolor{currentstroke}%
\pgfsetdash{}{0pt}%
\pgfpathmoveto{\pgfqpoint{5.504545in}{0.684333in}}%
\pgfpathcurveto{\pgfqpoint{5.515596in}{0.684333in}}{\pgfqpoint{5.526195in}{0.688724in}}{\pgfqpoint{5.534008in}{0.696537in}}%
\pgfpathcurveto{\pgfqpoint{5.541822in}{0.704351in}}{\pgfqpoint{5.546212in}{0.714950in}}{\pgfqpoint{5.546212in}{0.726000in}}%
\pgfpathcurveto{\pgfqpoint{5.546212in}{0.737050in}}{\pgfqpoint{5.541822in}{0.747649in}}{\pgfqpoint{5.534008in}{0.755463in}}%
\pgfpathcurveto{\pgfqpoint{5.526195in}{0.763276in}}{\pgfqpoint{5.515596in}{0.767667in}}{\pgfqpoint{5.504545in}{0.767667in}}%
\pgfpathcurveto{\pgfqpoint{5.493495in}{0.767667in}}{\pgfqpoint{5.482896in}{0.763276in}}{\pgfqpoint{5.475083in}{0.755463in}}%
\pgfpathcurveto{\pgfqpoint{5.467269in}{0.747649in}}{\pgfqpoint{5.462879in}{0.737050in}}{\pgfqpoint{5.462879in}{0.726000in}}%
\pgfpathcurveto{\pgfqpoint{5.462879in}{0.714950in}}{\pgfqpoint{5.467269in}{0.704351in}}{\pgfqpoint{5.475083in}{0.696537in}}%
\pgfpathcurveto{\pgfqpoint{5.482896in}{0.688724in}}{\pgfqpoint{5.493495in}{0.684333in}}{\pgfqpoint{5.504545in}{0.684333in}}%
\pgfpathclose%
\pgfusepath{stroke,fill}%
\end{pgfscope}%
\begin{pgfscope}%
\pgfpathrectangle{\pgfqpoint{0.800000in}{0.528000in}}{\pgfqpoint{4.960000in}{3.696000in}}%
\pgfusepath{clip}%
\pgfsetbuttcap%
\pgfsetroundjoin%
\definecolor{currentfill}{rgb}{0.000000,0.000000,0.000000}%
\pgfsetfillcolor{currentfill}%
\pgfsetlinewidth{1.003750pt}%
\definecolor{currentstroke}{rgb}{0.000000,0.000000,0.000000}%
\pgfsetstrokecolor{currentstroke}%
\pgfsetdash{}{0pt}%
\pgfpathmoveto{\pgfqpoint{5.504545in}{0.684333in}}%
\pgfpathcurveto{\pgfqpoint{5.515596in}{0.684333in}}{\pgfqpoint{5.526195in}{0.688724in}}{\pgfqpoint{5.534008in}{0.696537in}}%
\pgfpathcurveto{\pgfqpoint{5.541822in}{0.704351in}}{\pgfqpoint{5.546212in}{0.714950in}}{\pgfqpoint{5.546212in}{0.726000in}}%
\pgfpathcurveto{\pgfqpoint{5.546212in}{0.737050in}}{\pgfqpoint{5.541822in}{0.747649in}}{\pgfqpoint{5.534008in}{0.755463in}}%
\pgfpathcurveto{\pgfqpoint{5.526195in}{0.763276in}}{\pgfqpoint{5.515596in}{0.767667in}}{\pgfqpoint{5.504545in}{0.767667in}}%
\pgfpathcurveto{\pgfqpoint{5.493495in}{0.767667in}}{\pgfqpoint{5.482896in}{0.763276in}}{\pgfqpoint{5.475083in}{0.755463in}}%
\pgfpathcurveto{\pgfqpoint{5.467269in}{0.747649in}}{\pgfqpoint{5.462879in}{0.737050in}}{\pgfqpoint{5.462879in}{0.726000in}}%
\pgfpathcurveto{\pgfqpoint{5.462879in}{0.714950in}}{\pgfqpoint{5.467269in}{0.704351in}}{\pgfqpoint{5.475083in}{0.696537in}}%
\pgfpathcurveto{\pgfqpoint{5.482896in}{0.688724in}}{\pgfqpoint{5.493495in}{0.684333in}}{\pgfqpoint{5.504545in}{0.684333in}}%
\pgfpathclose%
\pgfusepath{stroke,fill}%
\end{pgfscope}%
\begin{pgfscope}%
\pgfpathrectangle{\pgfqpoint{0.800000in}{0.528000in}}{\pgfqpoint{4.960000in}{3.696000in}}%
\pgfusepath{clip}%
\pgfsetbuttcap%
\pgfsetroundjoin%
\definecolor{currentfill}{rgb}{0.000000,0.000000,0.000000}%
\pgfsetfillcolor{currentfill}%
\pgfsetlinewidth{1.003750pt}%
\definecolor{currentstroke}{rgb}{0.000000,0.000000,0.000000}%
\pgfsetstrokecolor{currentstroke}%
\pgfsetdash{}{0pt}%
\pgfpathmoveto{\pgfqpoint{5.504545in}{0.684333in}}%
\pgfpathcurveto{\pgfqpoint{5.515596in}{0.684333in}}{\pgfqpoint{5.526195in}{0.688724in}}{\pgfqpoint{5.534008in}{0.696537in}}%
\pgfpathcurveto{\pgfqpoint{5.541822in}{0.704351in}}{\pgfqpoint{5.546212in}{0.714950in}}{\pgfqpoint{5.546212in}{0.726000in}}%
\pgfpathcurveto{\pgfqpoint{5.546212in}{0.737050in}}{\pgfqpoint{5.541822in}{0.747649in}}{\pgfqpoint{5.534008in}{0.755463in}}%
\pgfpathcurveto{\pgfqpoint{5.526195in}{0.763276in}}{\pgfqpoint{5.515596in}{0.767667in}}{\pgfqpoint{5.504545in}{0.767667in}}%
\pgfpathcurveto{\pgfqpoint{5.493495in}{0.767667in}}{\pgfqpoint{5.482896in}{0.763276in}}{\pgfqpoint{5.475083in}{0.755463in}}%
\pgfpathcurveto{\pgfqpoint{5.467269in}{0.747649in}}{\pgfqpoint{5.462879in}{0.737050in}}{\pgfqpoint{5.462879in}{0.726000in}}%
\pgfpathcurveto{\pgfqpoint{5.462879in}{0.714950in}}{\pgfqpoint{5.467269in}{0.704351in}}{\pgfqpoint{5.475083in}{0.696537in}}%
\pgfpathcurveto{\pgfqpoint{5.482896in}{0.688724in}}{\pgfqpoint{5.493495in}{0.684333in}}{\pgfqpoint{5.504545in}{0.684333in}}%
\pgfpathclose%
\pgfusepath{stroke,fill}%
\end{pgfscope}%
\begin{pgfscope}%
\pgfpathrectangle{\pgfqpoint{0.800000in}{0.528000in}}{\pgfqpoint{4.960000in}{3.696000in}}%
\pgfusepath{clip}%
\pgfsetbuttcap%
\pgfsetroundjoin%
\definecolor{currentfill}{rgb}{0.000000,0.000000,0.000000}%
\pgfsetfillcolor{currentfill}%
\pgfsetlinewidth{1.003750pt}%
\definecolor{currentstroke}{rgb}{0.000000,0.000000,0.000000}%
\pgfsetstrokecolor{currentstroke}%
\pgfsetdash{}{0pt}%
\pgfpathmoveto{\pgfqpoint{5.504545in}{0.684333in}}%
\pgfpathcurveto{\pgfqpoint{5.515596in}{0.684333in}}{\pgfqpoint{5.526195in}{0.688724in}}{\pgfqpoint{5.534008in}{0.696537in}}%
\pgfpathcurveto{\pgfqpoint{5.541822in}{0.704351in}}{\pgfqpoint{5.546212in}{0.714950in}}{\pgfqpoint{5.546212in}{0.726000in}}%
\pgfpathcurveto{\pgfqpoint{5.546212in}{0.737050in}}{\pgfqpoint{5.541822in}{0.747649in}}{\pgfqpoint{5.534008in}{0.755463in}}%
\pgfpathcurveto{\pgfqpoint{5.526195in}{0.763276in}}{\pgfqpoint{5.515596in}{0.767667in}}{\pgfqpoint{5.504545in}{0.767667in}}%
\pgfpathcurveto{\pgfqpoint{5.493495in}{0.767667in}}{\pgfqpoint{5.482896in}{0.763276in}}{\pgfqpoint{5.475083in}{0.755463in}}%
\pgfpathcurveto{\pgfqpoint{5.467269in}{0.747649in}}{\pgfqpoint{5.462879in}{0.737050in}}{\pgfqpoint{5.462879in}{0.726000in}}%
\pgfpathcurveto{\pgfqpoint{5.462879in}{0.714950in}}{\pgfqpoint{5.467269in}{0.704351in}}{\pgfqpoint{5.475083in}{0.696537in}}%
\pgfpathcurveto{\pgfqpoint{5.482896in}{0.688724in}}{\pgfqpoint{5.493495in}{0.684333in}}{\pgfqpoint{5.504545in}{0.684333in}}%
\pgfpathclose%
\pgfusepath{stroke,fill}%
\end{pgfscope}%
\begin{pgfscope}%
\pgfpathrectangle{\pgfqpoint{0.800000in}{0.528000in}}{\pgfqpoint{4.960000in}{3.696000in}}%
\pgfusepath{clip}%
\pgfsetbuttcap%
\pgfsetroundjoin%
\definecolor{currentfill}{rgb}{0.000000,0.000000,0.000000}%
\pgfsetfillcolor{currentfill}%
\pgfsetlinewidth{1.003750pt}%
\definecolor{currentstroke}{rgb}{0.000000,0.000000,0.000000}%
\pgfsetstrokecolor{currentstroke}%
\pgfsetdash{}{0pt}%
\pgfpathmoveto{\pgfqpoint{5.504545in}{0.684333in}}%
\pgfpathcurveto{\pgfqpoint{5.515596in}{0.684333in}}{\pgfqpoint{5.526195in}{0.688724in}}{\pgfqpoint{5.534008in}{0.696537in}}%
\pgfpathcurveto{\pgfqpoint{5.541822in}{0.704351in}}{\pgfqpoint{5.546212in}{0.714950in}}{\pgfqpoint{5.546212in}{0.726000in}}%
\pgfpathcurveto{\pgfqpoint{5.546212in}{0.737050in}}{\pgfqpoint{5.541822in}{0.747649in}}{\pgfqpoint{5.534008in}{0.755463in}}%
\pgfpathcurveto{\pgfqpoint{5.526195in}{0.763276in}}{\pgfqpoint{5.515596in}{0.767667in}}{\pgfqpoint{5.504545in}{0.767667in}}%
\pgfpathcurveto{\pgfqpoint{5.493495in}{0.767667in}}{\pgfqpoint{5.482896in}{0.763276in}}{\pgfqpoint{5.475083in}{0.755463in}}%
\pgfpathcurveto{\pgfqpoint{5.467269in}{0.747649in}}{\pgfqpoint{5.462879in}{0.737050in}}{\pgfqpoint{5.462879in}{0.726000in}}%
\pgfpathcurveto{\pgfqpoint{5.462879in}{0.714950in}}{\pgfqpoint{5.467269in}{0.704351in}}{\pgfqpoint{5.475083in}{0.696537in}}%
\pgfpathcurveto{\pgfqpoint{5.482896in}{0.688724in}}{\pgfqpoint{5.493495in}{0.684333in}}{\pgfqpoint{5.504545in}{0.684333in}}%
\pgfpathclose%
\pgfusepath{stroke,fill}%
\end{pgfscope}%
\begin{pgfscope}%
\pgfsetbuttcap%
\pgfsetroundjoin%
\definecolor{currentfill}{rgb}{0.000000,0.000000,0.000000}%
\pgfsetfillcolor{currentfill}%
\pgfsetlinewidth{0.803000pt}%
\definecolor{currentstroke}{rgb}{0.000000,0.000000,0.000000}%
\pgfsetstrokecolor{currentstroke}%
\pgfsetdash{}{0pt}%
\pgfsys@defobject{currentmarker}{\pgfqpoint{0.000000in}{-0.048611in}}{\pgfqpoint{0.000000in}{0.000000in}}{%
\pgfpathmoveto{\pgfqpoint{0.000000in}{0.000000in}}%
\pgfpathlineto{\pgfqpoint{0.000000in}{-0.048611in}}%
\pgfusepath{stroke,fill}%
}%
\begin{pgfscope}%
\pgfsys@transformshift{1.025906in}{0.528000in}%
\pgfsys@useobject{currentmarker}{}%
\end{pgfscope}%
\end{pgfscope}%
\begin{pgfscope}%
\definecolor{textcolor}{rgb}{0.000000,0.000000,0.000000}%
\pgfsetstrokecolor{textcolor}%
\pgfsetfillcolor{textcolor}%
\pgftext[x=1.025906in,y=0.430778in,,top]{\color{textcolor}\sffamily\fontsize{10.000000}{12.000000}\selectfont 20}%
\end{pgfscope}%
\begin{pgfscope}%
\pgfsetbuttcap%
\pgfsetroundjoin%
\definecolor{currentfill}{rgb}{0.000000,0.000000,0.000000}%
\pgfsetfillcolor{currentfill}%
\pgfsetlinewidth{0.803000pt}%
\definecolor{currentstroke}{rgb}{0.000000,0.000000,0.000000}%
\pgfsetstrokecolor{currentstroke}%
\pgfsetdash{}{0pt}%
\pgfsys@defobject{currentmarker}{\pgfqpoint{0.000000in}{-0.048611in}}{\pgfqpoint{0.000000in}{0.000000in}}{%
\pgfpathmoveto{\pgfqpoint{0.000000in}{0.000000in}}%
\pgfpathlineto{\pgfqpoint{0.000000in}{-0.048611in}}%
\pgfusepath{stroke,fill}%
}%
\begin{pgfscope}%
\pgfsys@transformshift{2.518786in}{0.528000in}%
\pgfsys@useobject{currentmarker}{}%
\end{pgfscope}%
\end{pgfscope}%
\begin{pgfscope}%
\definecolor{textcolor}{rgb}{0.000000,0.000000,0.000000}%
\pgfsetstrokecolor{textcolor}%
\pgfsetfillcolor{textcolor}%
\pgftext[x=2.518786in,y=0.430778in,,top]{\color{textcolor}\sffamily\fontsize{10.000000}{12.000000}\selectfont 40}%
\end{pgfscope}%
\begin{pgfscope}%
\pgfsetbuttcap%
\pgfsetroundjoin%
\definecolor{currentfill}{rgb}{0.000000,0.000000,0.000000}%
\pgfsetfillcolor{currentfill}%
\pgfsetlinewidth{0.803000pt}%
\definecolor{currentstroke}{rgb}{0.000000,0.000000,0.000000}%
\pgfsetstrokecolor{currentstroke}%
\pgfsetdash{}{0pt}%
\pgfsys@defobject{currentmarker}{\pgfqpoint{0.000000in}{-0.048611in}}{\pgfqpoint{0.000000in}{0.000000in}}{%
\pgfpathmoveto{\pgfqpoint{0.000000in}{0.000000in}}%
\pgfpathlineto{\pgfqpoint{0.000000in}{-0.048611in}}%
\pgfusepath{stroke,fill}%
}%
\begin{pgfscope}%
\pgfsys@transformshift{4.011666in}{0.528000in}%
\pgfsys@useobject{currentmarker}{}%
\end{pgfscope}%
\end{pgfscope}%
\begin{pgfscope}%
\definecolor{textcolor}{rgb}{0.000000,0.000000,0.000000}%
\pgfsetstrokecolor{textcolor}%
\pgfsetfillcolor{textcolor}%
\pgftext[x=4.011666in,y=0.430778in,,top]{\color{textcolor}\sffamily\fontsize{10.000000}{12.000000}\selectfont 60}%
\end{pgfscope}%
\begin{pgfscope}%
\pgfsetbuttcap%
\pgfsetroundjoin%
\definecolor{currentfill}{rgb}{0.000000,0.000000,0.000000}%
\pgfsetfillcolor{currentfill}%
\pgfsetlinewidth{0.803000pt}%
\definecolor{currentstroke}{rgb}{0.000000,0.000000,0.000000}%
\pgfsetstrokecolor{currentstroke}%
\pgfsetdash{}{0pt}%
\pgfsys@defobject{currentmarker}{\pgfqpoint{0.000000in}{-0.048611in}}{\pgfqpoint{0.000000in}{0.000000in}}{%
\pgfpathmoveto{\pgfqpoint{0.000000in}{0.000000in}}%
\pgfpathlineto{\pgfqpoint{0.000000in}{-0.048611in}}%
\pgfusepath{stroke,fill}%
}%
\begin{pgfscope}%
\pgfsys@transformshift{5.504545in}{0.528000in}%
\pgfsys@useobject{currentmarker}{}%
\end{pgfscope}%
\end{pgfscope}%
\begin{pgfscope}%
\definecolor{textcolor}{rgb}{0.000000,0.000000,0.000000}%
\pgfsetstrokecolor{textcolor}%
\pgfsetfillcolor{textcolor}%
\pgftext[x=5.504545in,y=0.430778in,,top]{\color{textcolor}\sffamily\fontsize{10.000000}{12.000000}\selectfont 80}%
\end{pgfscope}%
\begin{pgfscope}%
\definecolor{textcolor}{rgb}{0.000000,0.000000,0.000000}%
\pgfsetstrokecolor{textcolor}%
\pgfsetfillcolor{textcolor}%
\pgftext[x=3.280000in,y=0.240809in,,top]{\color{textcolor}\sffamily\fontsize{10.000000}{12.000000}\selectfont \(\displaystyle k\)}%
\end{pgfscope}%
\begin{pgfscope}%
\pgfsetbuttcap%
\pgfsetroundjoin%
\definecolor{currentfill}{rgb}{0.000000,0.000000,0.000000}%
\pgfsetfillcolor{currentfill}%
\pgfsetlinewidth{0.803000pt}%
\definecolor{currentstroke}{rgb}{0.000000,0.000000,0.000000}%
\pgfsetstrokecolor{currentstroke}%
\pgfsetdash{}{0pt}%
\pgfsys@defobject{currentmarker}{\pgfqpoint{-0.048611in}{0.000000in}}{\pgfqpoint{0.000000in}{0.000000in}}{%
\pgfpathmoveto{\pgfqpoint{0.000000in}{0.000000in}}%
\pgfpathlineto{\pgfqpoint{-0.048611in}{0.000000in}}%
\pgfusepath{stroke,fill}%
}%
\begin{pgfscope}%
\pgfsys@transformshift{0.800000in}{0.726000in}%
\pgfsys@useobject{currentmarker}{}%
\end{pgfscope}%
\end{pgfscope}%
\begin{pgfscope}%
\definecolor{textcolor}{rgb}{0.000000,0.000000,0.000000}%
\pgfsetstrokecolor{textcolor}%
\pgfsetfillcolor{textcolor}%
\pgftext[x=0.614413in,y=0.673238in,left,base]{\color{textcolor}\sffamily\fontsize{10.000000}{12.000000}\selectfont 5}%
\end{pgfscope}%
\begin{pgfscope}%
\pgfsetbuttcap%
\pgfsetroundjoin%
\definecolor{currentfill}{rgb}{0.000000,0.000000,0.000000}%
\pgfsetfillcolor{currentfill}%
\pgfsetlinewidth{0.803000pt}%
\definecolor{currentstroke}{rgb}{0.000000,0.000000,0.000000}%
\pgfsetstrokecolor{currentstroke}%
\pgfsetdash{}{0pt}%
\pgfsys@defobject{currentmarker}{\pgfqpoint{-0.048611in}{0.000000in}}{\pgfqpoint{0.000000in}{0.000000in}}{%
\pgfpathmoveto{\pgfqpoint{0.000000in}{0.000000in}}%
\pgfpathlineto{\pgfqpoint{-0.048611in}{0.000000in}}%
\pgfusepath{stroke,fill}%
}%
\begin{pgfscope}%
\pgfsys@transformshift{0.800000in}{4.026000in}%
\pgfsys@useobject{currentmarker}{}%
\end{pgfscope}%
\end{pgfscope}%
\begin{pgfscope}%
\definecolor{textcolor}{rgb}{0.000000,0.000000,0.000000}%
\pgfsetstrokecolor{textcolor}%
\pgfsetfillcolor{textcolor}%
\pgftext[x=0.614413in,y=3.973238in,left,base]{\color{textcolor}\sffamily\fontsize{10.000000}{12.000000}\selectfont 6}%
\end{pgfscope}%
\begin{pgfscope}%
\definecolor{textcolor}{rgb}{0.000000,0.000000,0.000000}%
\pgfsetstrokecolor{textcolor}%
\pgfsetfillcolor{textcolor}%
\pgftext[x=0.558857in,y=2.376000in,,bottom,rotate=90.000000]{\color{textcolor}\sffamily\fontsize{10.000000}{12.000000}\selectfont Number of GMRES Iterations}%
\end{pgfscope}%
\begin{pgfscope}%
\pgfsetrectcap%
\pgfsetmiterjoin%
\pgfsetlinewidth{0.803000pt}%
\definecolor{currentstroke}{rgb}{0.000000,0.000000,0.000000}%
\pgfsetstrokecolor{currentstroke}%
\pgfsetdash{}{0pt}%
\pgfpathmoveto{\pgfqpoint{0.800000in}{0.528000in}}%
\pgfpathlineto{\pgfqpoint{0.800000in}{4.224000in}}%
\pgfusepath{stroke}%
\end{pgfscope}%
\begin{pgfscope}%
\pgfsetrectcap%
\pgfsetmiterjoin%
\pgfsetlinewidth{0.803000pt}%
\definecolor{currentstroke}{rgb}{0.000000,0.000000,0.000000}%
\pgfsetstrokecolor{currentstroke}%
\pgfsetdash{}{0pt}%
\pgfpathmoveto{\pgfqpoint{5.760000in}{0.528000in}}%
\pgfpathlineto{\pgfqpoint{5.760000in}{4.224000in}}%
\pgfusepath{stroke}%
\end{pgfscope}%
\begin{pgfscope}%
\pgfsetrectcap%
\pgfsetmiterjoin%
\pgfsetlinewidth{0.803000pt}%
\definecolor{currentstroke}{rgb}{0.000000,0.000000,0.000000}%
\pgfsetstrokecolor{currentstroke}%
\pgfsetdash{}{0pt}%
\pgfpathmoveto{\pgfqpoint{0.800000in}{0.528000in}}%
\pgfpathlineto{\pgfqpoint{5.760000in}{0.528000in}}%
\pgfusepath{stroke}%
\end{pgfscope}%
\begin{pgfscope}%
\pgfsetrectcap%
\pgfsetmiterjoin%
\pgfsetlinewidth{0.803000pt}%
\definecolor{currentstroke}{rgb}{0.000000,0.000000,0.000000}%
\pgfsetstrokecolor{currentstroke}%
\pgfsetdash{}{0pt}%
\pgfpathmoveto{\pgfqpoint{0.800000in}{4.224000in}}%
\pgfpathlineto{\pgfqpoint{5.760000in}{4.224000in}}%
\pgfusepath{stroke}%
\end{pgfscope}%
\end{pgfpicture}%
\makeatother%
\endgroup%

  \caption{GMRES iteration counts for $\alpha = 0.5/k$}\label{fig:linfinityA2}
\end{subfigure}
\caption{GMRES iteration counts for $\AmatoI\Amatt$ where $\nso=\nst=1$ and $\NLiDRRRdtd{\Aso-\Ast} = \alpha$ as described in \cref{sec:num}.}
\end{figure}

  \begin{figure}
    \centering
    \begin{subfigure}{\textwidth}
      \centering
%% Creator: Matplotlib, PGF backend
%%
%% To include the figure in your LaTeX document, write
%%   \input{<filename>.pgf}
%%
%% Make sure the required packages are loaded in your preamble
%%   \usepackage{pgf}
%%
%% Figures using additional raster images can only be included by \input if
%% they are in the same directory as the main LaTeX file. For loading figures
%% from other directories you can use the `import` package
%%   \usepackage{import}
%% and then include the figures with
%%   \import{<path to file>}{<filename>.pgf}
%%
%% Matplotlib used the following preamble
%%   \usepackage{fontspec}
%%   \setmainfont{DejaVuSerif.ttf}[Path=/home/owen/progs/firedrake-complex/firedrake/lib/python3.5/site-packages/matplotlib/mpl-data/fonts/ttf/]
%%   \setsansfont{DejaVuSans.ttf}[Path=/home/owen/progs/firedrake-complex/firedrake/lib/python3.5/site-packages/matplotlib/mpl-data/fonts/ttf/]
%%   \setmonofont{DejaVuSansMono.ttf}[Path=/home/owen/progs/firedrake-complex/firedrake/lib/python3.5/site-packages/matplotlib/mpl-data/fonts/ttf/]
%%
\begingroup%
\makeatletter%
\begin{pgfpicture}%
\pgfpathrectangle{\pgfpointorigin}{\pgfqpoint{4.500000in}{2.500000in}}%
\pgfusepath{use as bounding box, clip}%
\begin{pgfscope}%
\pgfsetbuttcap%
\pgfsetmiterjoin%
\definecolor{currentfill}{rgb}{1.000000,1.000000,1.000000}%
\pgfsetfillcolor{currentfill}%
\pgfsetlinewidth{0.000000pt}%
\definecolor{currentstroke}{rgb}{1.000000,1.000000,1.000000}%
\pgfsetstrokecolor{currentstroke}%
\pgfsetdash{}{0pt}%
\pgfpathmoveto{\pgfqpoint{0.000000in}{0.000000in}}%
\pgfpathlineto{\pgfqpoint{4.500000in}{0.000000in}}%
\pgfpathlineto{\pgfqpoint{4.500000in}{2.500000in}}%
\pgfpathlineto{\pgfqpoint{0.000000in}{2.500000in}}%
\pgfpathclose%
\pgfusepath{fill}%
\end{pgfscope}%
\begin{pgfscope}%
\pgfsetbuttcap%
\pgfsetmiterjoin%
\definecolor{currentfill}{rgb}{1.000000,1.000000,1.000000}%
\pgfsetfillcolor{currentfill}%
\pgfsetlinewidth{0.000000pt}%
\definecolor{currentstroke}{rgb}{0.000000,0.000000,0.000000}%
\pgfsetstrokecolor{currentstroke}%
\pgfsetstrokeopacity{0.000000}%
\pgfsetdash{}{0pt}%
\pgfpathmoveto{\pgfqpoint{0.562500in}{0.275000in}}%
\pgfpathlineto{\pgfqpoint{4.050000in}{0.275000in}}%
\pgfpathlineto{\pgfqpoint{4.050000in}{2.200000in}}%
\pgfpathlineto{\pgfqpoint{0.562500in}{2.200000in}}%
\pgfpathclose%
\pgfusepath{fill}%
\end{pgfscope}%
\begin{pgfscope}%
\pgfpathrectangle{\pgfqpoint{0.562500in}{0.275000in}}{\pgfqpoint{3.487500in}{1.925000in}}%
\pgfusepath{clip}%
\pgfsetbuttcap%
\pgfsetroundjoin%
\definecolor{currentfill}{rgb}{0.000000,0.000000,0.000000}%
\pgfsetfillcolor{currentfill}%
\pgfsetlinewidth{1.003750pt}%
\definecolor{currentstroke}{rgb}{0.000000,0.000000,0.000000}%
\pgfsetstrokecolor{currentstroke}%
\pgfsetdash{}{0pt}%
\pgfpathmoveto{\pgfqpoint{0.721249in}{0.362164in}}%
\pgfpathcurveto{\pgfqpoint{0.726774in}{0.362164in}}{\pgfqpoint{0.732073in}{0.364359in}}{\pgfqpoint{0.735980in}{0.368266in}}%
\pgfpathcurveto{\pgfqpoint{0.739887in}{0.372173in}}{\pgfqpoint{0.742082in}{0.377472in}}{\pgfqpoint{0.742082in}{0.382997in}}%
\pgfpathcurveto{\pgfqpoint{0.742082in}{0.388522in}}{\pgfqpoint{0.739887in}{0.393822in}}{\pgfqpoint{0.735980in}{0.397729in}}%
\pgfpathcurveto{\pgfqpoint{0.732073in}{0.401636in}}{\pgfqpoint{0.726774in}{0.403831in}}{\pgfqpoint{0.721249in}{0.403831in}}%
\pgfpathcurveto{\pgfqpoint{0.715724in}{0.403831in}}{\pgfqpoint{0.710424in}{0.401636in}}{\pgfqpoint{0.706518in}{0.397729in}}%
\pgfpathcurveto{\pgfqpoint{0.702611in}{0.393822in}}{\pgfqpoint{0.700416in}{0.388522in}}{\pgfqpoint{0.700416in}{0.382997in}}%
\pgfpathcurveto{\pgfqpoint{0.700416in}{0.377472in}}{\pgfqpoint{0.702611in}{0.372173in}}{\pgfqpoint{0.706518in}{0.368266in}}%
\pgfpathcurveto{\pgfqpoint{0.710424in}{0.364359in}}{\pgfqpoint{0.715724in}{0.362164in}}{\pgfqpoint{0.721249in}{0.362164in}}%
\pgfpathclose%
\pgfusepath{stroke,fill}%
\end{pgfscope}%
\begin{pgfscope}%
\pgfpathrectangle{\pgfqpoint{0.562500in}{0.275000in}}{\pgfqpoint{3.487500in}{1.925000in}}%
\pgfusepath{clip}%
\pgfsetbuttcap%
\pgfsetroundjoin%
\definecolor{currentfill}{rgb}{0.000000,0.000000,0.000000}%
\pgfsetfillcolor{currentfill}%
\pgfsetlinewidth{1.003750pt}%
\definecolor{currentstroke}{rgb}{0.000000,0.000000,0.000000}%
\pgfsetstrokecolor{currentstroke}%
\pgfsetdash{}{0pt}%
\pgfpathmoveto{\pgfqpoint{0.721249in}{0.357106in}}%
\pgfpathcurveto{\pgfqpoint{0.726774in}{0.357106in}}{\pgfqpoint{0.732073in}{0.359301in}}{\pgfqpoint{0.735980in}{0.363207in}}%
\pgfpathcurveto{\pgfqpoint{0.739887in}{0.367114in}}{\pgfqpoint{0.742082in}{0.372414in}}{\pgfqpoint{0.742082in}{0.377939in}}%
\pgfpathcurveto{\pgfqpoint{0.742082in}{0.383464in}}{\pgfqpoint{0.739887in}{0.388763in}}{\pgfqpoint{0.735980in}{0.392670in}}%
\pgfpathcurveto{\pgfqpoint{0.732073in}{0.396577in}}{\pgfqpoint{0.726774in}{0.398772in}}{\pgfqpoint{0.721249in}{0.398772in}}%
\pgfpathcurveto{\pgfqpoint{0.715724in}{0.398772in}}{\pgfqpoint{0.710424in}{0.396577in}}{\pgfqpoint{0.706518in}{0.392670in}}%
\pgfpathcurveto{\pgfqpoint{0.702611in}{0.388763in}}{\pgfqpoint{0.700416in}{0.383464in}}{\pgfqpoint{0.700416in}{0.377939in}}%
\pgfpathcurveto{\pgfqpoint{0.700416in}{0.372414in}}{\pgfqpoint{0.702611in}{0.367114in}}{\pgfqpoint{0.706518in}{0.363207in}}%
\pgfpathcurveto{\pgfqpoint{0.710424in}{0.359301in}}{\pgfqpoint{0.715724in}{0.357106in}}{\pgfqpoint{0.721249in}{0.357106in}}%
\pgfpathclose%
\pgfusepath{stroke,fill}%
\end{pgfscope}%
\begin{pgfscope}%
\pgfpathrectangle{\pgfqpoint{0.562500in}{0.275000in}}{\pgfqpoint{3.487500in}{1.925000in}}%
\pgfusepath{clip}%
\pgfsetbuttcap%
\pgfsetroundjoin%
\definecolor{currentfill}{rgb}{0.000000,0.000000,0.000000}%
\pgfsetfillcolor{currentfill}%
\pgfsetlinewidth{1.003750pt}%
\definecolor{currentstroke}{rgb}{0.000000,0.000000,0.000000}%
\pgfsetstrokecolor{currentstroke}%
\pgfsetdash{}{0pt}%
\pgfpathmoveto{\pgfqpoint{0.721249in}{0.352047in}}%
\pgfpathcurveto{\pgfqpoint{0.726774in}{0.352047in}}{\pgfqpoint{0.732073in}{0.354242in}}{\pgfqpoint{0.735980in}{0.358149in}}%
\pgfpathcurveto{\pgfqpoint{0.739887in}{0.362056in}}{\pgfqpoint{0.742082in}{0.367355in}}{\pgfqpoint{0.742082in}{0.372880in}}%
\pgfpathcurveto{\pgfqpoint{0.742082in}{0.378405in}}{\pgfqpoint{0.739887in}{0.383705in}}{\pgfqpoint{0.735980in}{0.387612in}}%
\pgfpathcurveto{\pgfqpoint{0.732073in}{0.391519in}}{\pgfqpoint{0.726774in}{0.393714in}}{\pgfqpoint{0.721249in}{0.393714in}}%
\pgfpathcurveto{\pgfqpoint{0.715724in}{0.393714in}}{\pgfqpoint{0.710424in}{0.391519in}}{\pgfqpoint{0.706518in}{0.387612in}}%
\pgfpathcurveto{\pgfqpoint{0.702611in}{0.383705in}}{\pgfqpoint{0.700416in}{0.378405in}}{\pgfqpoint{0.700416in}{0.372880in}}%
\pgfpathcurveto{\pgfqpoint{0.700416in}{0.367355in}}{\pgfqpoint{0.702611in}{0.362056in}}{\pgfqpoint{0.706518in}{0.358149in}}%
\pgfpathcurveto{\pgfqpoint{0.710424in}{0.354242in}}{\pgfqpoint{0.715724in}{0.352047in}}{\pgfqpoint{0.721249in}{0.352047in}}%
\pgfpathclose%
\pgfusepath{stroke,fill}%
\end{pgfscope}%
\begin{pgfscope}%
\pgfpathrectangle{\pgfqpoint{0.562500in}{0.275000in}}{\pgfqpoint{3.487500in}{1.925000in}}%
\pgfusepath{clip}%
\pgfsetbuttcap%
\pgfsetroundjoin%
\definecolor{currentfill}{rgb}{0.000000,0.000000,0.000000}%
\pgfsetfillcolor{currentfill}%
\pgfsetlinewidth{1.003750pt}%
\definecolor{currentstroke}{rgb}{0.000000,0.000000,0.000000}%
\pgfsetstrokecolor{currentstroke}%
\pgfsetdash{}{0pt}%
\pgfpathmoveto{\pgfqpoint{0.721249in}{0.357106in}}%
\pgfpathcurveto{\pgfqpoint{0.726774in}{0.357106in}}{\pgfqpoint{0.732073in}{0.359301in}}{\pgfqpoint{0.735980in}{0.363207in}}%
\pgfpathcurveto{\pgfqpoint{0.739887in}{0.367114in}}{\pgfqpoint{0.742082in}{0.372414in}}{\pgfqpoint{0.742082in}{0.377939in}}%
\pgfpathcurveto{\pgfqpoint{0.742082in}{0.383464in}}{\pgfqpoint{0.739887in}{0.388763in}}{\pgfqpoint{0.735980in}{0.392670in}}%
\pgfpathcurveto{\pgfqpoint{0.732073in}{0.396577in}}{\pgfqpoint{0.726774in}{0.398772in}}{\pgfqpoint{0.721249in}{0.398772in}}%
\pgfpathcurveto{\pgfqpoint{0.715724in}{0.398772in}}{\pgfqpoint{0.710424in}{0.396577in}}{\pgfqpoint{0.706518in}{0.392670in}}%
\pgfpathcurveto{\pgfqpoint{0.702611in}{0.388763in}}{\pgfqpoint{0.700416in}{0.383464in}}{\pgfqpoint{0.700416in}{0.377939in}}%
\pgfpathcurveto{\pgfqpoint{0.700416in}{0.372414in}}{\pgfqpoint{0.702611in}{0.367114in}}{\pgfqpoint{0.706518in}{0.363207in}}%
\pgfpathcurveto{\pgfqpoint{0.710424in}{0.359301in}}{\pgfqpoint{0.715724in}{0.357106in}}{\pgfqpoint{0.721249in}{0.357106in}}%
\pgfpathclose%
\pgfusepath{stroke,fill}%
\end{pgfscope}%
\begin{pgfscope}%
\pgfpathrectangle{\pgfqpoint{0.562500in}{0.275000in}}{\pgfqpoint{3.487500in}{1.925000in}}%
\pgfusepath{clip}%
\pgfsetbuttcap%
\pgfsetroundjoin%
\definecolor{currentfill}{rgb}{0.000000,0.000000,0.000000}%
\pgfsetfillcolor{currentfill}%
\pgfsetlinewidth{1.003750pt}%
\definecolor{currentstroke}{rgb}{0.000000,0.000000,0.000000}%
\pgfsetstrokecolor{currentstroke}%
\pgfsetdash{}{0pt}%
\pgfpathmoveto{\pgfqpoint{0.721249in}{0.367223in}}%
\pgfpathcurveto{\pgfqpoint{0.726774in}{0.367223in}}{\pgfqpoint{0.732073in}{0.369418in}}{\pgfqpoint{0.735980in}{0.373325in}}%
\pgfpathcurveto{\pgfqpoint{0.739887in}{0.377231in}}{\pgfqpoint{0.742082in}{0.382531in}}{\pgfqpoint{0.742082in}{0.388056in}}%
\pgfpathcurveto{\pgfqpoint{0.742082in}{0.393581in}}{\pgfqpoint{0.739887in}{0.398881in}}{\pgfqpoint{0.735980in}{0.402787in}}%
\pgfpathcurveto{\pgfqpoint{0.732073in}{0.406694in}}{\pgfqpoint{0.726774in}{0.408889in}}{\pgfqpoint{0.721249in}{0.408889in}}%
\pgfpathcurveto{\pgfqpoint{0.715724in}{0.408889in}}{\pgfqpoint{0.710424in}{0.406694in}}{\pgfqpoint{0.706518in}{0.402787in}}%
\pgfpathcurveto{\pgfqpoint{0.702611in}{0.398881in}}{\pgfqpoint{0.700416in}{0.393581in}}{\pgfqpoint{0.700416in}{0.388056in}}%
\pgfpathcurveto{\pgfqpoint{0.700416in}{0.382531in}}{\pgfqpoint{0.702611in}{0.377231in}}{\pgfqpoint{0.706518in}{0.373325in}}%
\pgfpathcurveto{\pgfqpoint{0.710424in}{0.369418in}}{\pgfqpoint{0.715724in}{0.367223in}}{\pgfqpoint{0.721249in}{0.367223in}}%
\pgfpathclose%
\pgfusepath{stroke,fill}%
\end{pgfscope}%
\begin{pgfscope}%
\pgfpathrectangle{\pgfqpoint{0.562500in}{0.275000in}}{\pgfqpoint{3.487500in}{1.925000in}}%
\pgfusepath{clip}%
\pgfsetbuttcap%
\pgfsetroundjoin%
\definecolor{currentfill}{rgb}{0.000000,0.000000,0.000000}%
\pgfsetfillcolor{currentfill}%
\pgfsetlinewidth{1.003750pt}%
\definecolor{currentstroke}{rgb}{0.000000,0.000000,0.000000}%
\pgfsetstrokecolor{currentstroke}%
\pgfsetdash{}{0pt}%
\pgfpathmoveto{\pgfqpoint{0.721249in}{0.357106in}}%
\pgfpathcurveto{\pgfqpoint{0.726774in}{0.357106in}}{\pgfqpoint{0.732073in}{0.359301in}}{\pgfqpoint{0.735980in}{0.363207in}}%
\pgfpathcurveto{\pgfqpoint{0.739887in}{0.367114in}}{\pgfqpoint{0.742082in}{0.372414in}}{\pgfqpoint{0.742082in}{0.377939in}}%
\pgfpathcurveto{\pgfqpoint{0.742082in}{0.383464in}}{\pgfqpoint{0.739887in}{0.388763in}}{\pgfqpoint{0.735980in}{0.392670in}}%
\pgfpathcurveto{\pgfqpoint{0.732073in}{0.396577in}}{\pgfqpoint{0.726774in}{0.398772in}}{\pgfqpoint{0.721249in}{0.398772in}}%
\pgfpathcurveto{\pgfqpoint{0.715724in}{0.398772in}}{\pgfqpoint{0.710424in}{0.396577in}}{\pgfqpoint{0.706518in}{0.392670in}}%
\pgfpathcurveto{\pgfqpoint{0.702611in}{0.388763in}}{\pgfqpoint{0.700416in}{0.383464in}}{\pgfqpoint{0.700416in}{0.377939in}}%
\pgfpathcurveto{\pgfqpoint{0.700416in}{0.372414in}}{\pgfqpoint{0.702611in}{0.367114in}}{\pgfqpoint{0.706518in}{0.363207in}}%
\pgfpathcurveto{\pgfqpoint{0.710424in}{0.359301in}}{\pgfqpoint{0.715724in}{0.357106in}}{\pgfqpoint{0.721249in}{0.357106in}}%
\pgfpathclose%
\pgfusepath{stroke,fill}%
\end{pgfscope}%
\begin{pgfscope}%
\pgfpathrectangle{\pgfqpoint{0.562500in}{0.275000in}}{\pgfqpoint{3.487500in}{1.925000in}}%
\pgfusepath{clip}%
\pgfsetbuttcap%
\pgfsetroundjoin%
\definecolor{currentfill}{rgb}{0.000000,0.000000,0.000000}%
\pgfsetfillcolor{currentfill}%
\pgfsetlinewidth{1.003750pt}%
\definecolor{currentstroke}{rgb}{0.000000,0.000000,0.000000}%
\pgfsetstrokecolor{currentstroke}%
\pgfsetdash{}{0pt}%
\pgfpathmoveto{\pgfqpoint{0.721249in}{0.352047in}}%
\pgfpathcurveto{\pgfqpoint{0.726774in}{0.352047in}}{\pgfqpoint{0.732073in}{0.354242in}}{\pgfqpoint{0.735980in}{0.358149in}}%
\pgfpathcurveto{\pgfqpoint{0.739887in}{0.362056in}}{\pgfqpoint{0.742082in}{0.367355in}}{\pgfqpoint{0.742082in}{0.372880in}}%
\pgfpathcurveto{\pgfqpoint{0.742082in}{0.378405in}}{\pgfqpoint{0.739887in}{0.383705in}}{\pgfqpoint{0.735980in}{0.387612in}}%
\pgfpathcurveto{\pgfqpoint{0.732073in}{0.391519in}}{\pgfqpoint{0.726774in}{0.393714in}}{\pgfqpoint{0.721249in}{0.393714in}}%
\pgfpathcurveto{\pgfqpoint{0.715724in}{0.393714in}}{\pgfqpoint{0.710424in}{0.391519in}}{\pgfqpoint{0.706518in}{0.387612in}}%
\pgfpathcurveto{\pgfqpoint{0.702611in}{0.383705in}}{\pgfqpoint{0.700416in}{0.378405in}}{\pgfqpoint{0.700416in}{0.372880in}}%
\pgfpathcurveto{\pgfqpoint{0.700416in}{0.367355in}}{\pgfqpoint{0.702611in}{0.362056in}}{\pgfqpoint{0.706518in}{0.358149in}}%
\pgfpathcurveto{\pgfqpoint{0.710424in}{0.354242in}}{\pgfqpoint{0.715724in}{0.352047in}}{\pgfqpoint{0.721249in}{0.352047in}}%
\pgfpathclose%
\pgfusepath{stroke,fill}%
\end{pgfscope}%
\begin{pgfscope}%
\pgfpathrectangle{\pgfqpoint{0.562500in}{0.275000in}}{\pgfqpoint{3.487500in}{1.925000in}}%
\pgfusepath{clip}%
\pgfsetbuttcap%
\pgfsetroundjoin%
\definecolor{currentfill}{rgb}{0.000000,0.000000,0.000000}%
\pgfsetfillcolor{currentfill}%
\pgfsetlinewidth{1.003750pt}%
\definecolor{currentstroke}{rgb}{0.000000,0.000000,0.000000}%
\pgfsetstrokecolor{currentstroke}%
\pgfsetdash{}{0pt}%
\pgfpathmoveto{\pgfqpoint{0.721249in}{0.362164in}}%
\pgfpathcurveto{\pgfqpoint{0.726774in}{0.362164in}}{\pgfqpoint{0.732073in}{0.364359in}}{\pgfqpoint{0.735980in}{0.368266in}}%
\pgfpathcurveto{\pgfqpoint{0.739887in}{0.372173in}}{\pgfqpoint{0.742082in}{0.377472in}}{\pgfqpoint{0.742082in}{0.382997in}}%
\pgfpathcurveto{\pgfqpoint{0.742082in}{0.388522in}}{\pgfqpoint{0.739887in}{0.393822in}}{\pgfqpoint{0.735980in}{0.397729in}}%
\pgfpathcurveto{\pgfqpoint{0.732073in}{0.401636in}}{\pgfqpoint{0.726774in}{0.403831in}}{\pgfqpoint{0.721249in}{0.403831in}}%
\pgfpathcurveto{\pgfqpoint{0.715724in}{0.403831in}}{\pgfqpoint{0.710424in}{0.401636in}}{\pgfqpoint{0.706518in}{0.397729in}}%
\pgfpathcurveto{\pgfqpoint{0.702611in}{0.393822in}}{\pgfqpoint{0.700416in}{0.388522in}}{\pgfqpoint{0.700416in}{0.382997in}}%
\pgfpathcurveto{\pgfqpoint{0.700416in}{0.377472in}}{\pgfqpoint{0.702611in}{0.372173in}}{\pgfqpoint{0.706518in}{0.368266in}}%
\pgfpathcurveto{\pgfqpoint{0.710424in}{0.364359in}}{\pgfqpoint{0.715724in}{0.362164in}}{\pgfqpoint{0.721249in}{0.362164in}}%
\pgfpathclose%
\pgfusepath{stroke,fill}%
\end{pgfscope}%
\begin{pgfscope}%
\pgfpathrectangle{\pgfqpoint{0.562500in}{0.275000in}}{\pgfqpoint{3.487500in}{1.925000in}}%
\pgfusepath{clip}%
\pgfsetbuttcap%
\pgfsetroundjoin%
\definecolor{currentfill}{rgb}{0.000000,0.000000,0.000000}%
\pgfsetfillcolor{currentfill}%
\pgfsetlinewidth{1.003750pt}%
\definecolor{currentstroke}{rgb}{0.000000,0.000000,0.000000}%
\pgfsetstrokecolor{currentstroke}%
\pgfsetdash{}{0pt}%
\pgfpathmoveto{\pgfqpoint{0.721249in}{0.357106in}}%
\pgfpathcurveto{\pgfqpoint{0.726774in}{0.357106in}}{\pgfqpoint{0.732073in}{0.359301in}}{\pgfqpoint{0.735980in}{0.363207in}}%
\pgfpathcurveto{\pgfqpoint{0.739887in}{0.367114in}}{\pgfqpoint{0.742082in}{0.372414in}}{\pgfqpoint{0.742082in}{0.377939in}}%
\pgfpathcurveto{\pgfqpoint{0.742082in}{0.383464in}}{\pgfqpoint{0.739887in}{0.388763in}}{\pgfqpoint{0.735980in}{0.392670in}}%
\pgfpathcurveto{\pgfqpoint{0.732073in}{0.396577in}}{\pgfqpoint{0.726774in}{0.398772in}}{\pgfqpoint{0.721249in}{0.398772in}}%
\pgfpathcurveto{\pgfqpoint{0.715724in}{0.398772in}}{\pgfqpoint{0.710424in}{0.396577in}}{\pgfqpoint{0.706518in}{0.392670in}}%
\pgfpathcurveto{\pgfqpoint{0.702611in}{0.388763in}}{\pgfqpoint{0.700416in}{0.383464in}}{\pgfqpoint{0.700416in}{0.377939in}}%
\pgfpathcurveto{\pgfqpoint{0.700416in}{0.372414in}}{\pgfqpoint{0.702611in}{0.367114in}}{\pgfqpoint{0.706518in}{0.363207in}}%
\pgfpathcurveto{\pgfqpoint{0.710424in}{0.359301in}}{\pgfqpoint{0.715724in}{0.357106in}}{\pgfqpoint{0.721249in}{0.357106in}}%
\pgfpathclose%
\pgfusepath{stroke,fill}%
\end{pgfscope}%
\begin{pgfscope}%
\pgfpathrectangle{\pgfqpoint{0.562500in}{0.275000in}}{\pgfqpoint{3.487500in}{1.925000in}}%
\pgfusepath{clip}%
\pgfsetbuttcap%
\pgfsetroundjoin%
\definecolor{currentfill}{rgb}{0.000000,0.000000,0.000000}%
\pgfsetfillcolor{currentfill}%
\pgfsetlinewidth{1.003750pt}%
\definecolor{currentstroke}{rgb}{0.000000,0.000000,0.000000}%
\pgfsetstrokecolor{currentstroke}%
\pgfsetdash{}{0pt}%
\pgfpathmoveto{\pgfqpoint{0.721249in}{0.362164in}}%
\pgfpathcurveto{\pgfqpoint{0.726774in}{0.362164in}}{\pgfqpoint{0.732073in}{0.364359in}}{\pgfqpoint{0.735980in}{0.368266in}}%
\pgfpathcurveto{\pgfqpoint{0.739887in}{0.372173in}}{\pgfqpoint{0.742082in}{0.377472in}}{\pgfqpoint{0.742082in}{0.382997in}}%
\pgfpathcurveto{\pgfqpoint{0.742082in}{0.388522in}}{\pgfqpoint{0.739887in}{0.393822in}}{\pgfqpoint{0.735980in}{0.397729in}}%
\pgfpathcurveto{\pgfqpoint{0.732073in}{0.401636in}}{\pgfqpoint{0.726774in}{0.403831in}}{\pgfqpoint{0.721249in}{0.403831in}}%
\pgfpathcurveto{\pgfqpoint{0.715724in}{0.403831in}}{\pgfqpoint{0.710424in}{0.401636in}}{\pgfqpoint{0.706518in}{0.397729in}}%
\pgfpathcurveto{\pgfqpoint{0.702611in}{0.393822in}}{\pgfqpoint{0.700416in}{0.388522in}}{\pgfqpoint{0.700416in}{0.382997in}}%
\pgfpathcurveto{\pgfqpoint{0.700416in}{0.377472in}}{\pgfqpoint{0.702611in}{0.372173in}}{\pgfqpoint{0.706518in}{0.368266in}}%
\pgfpathcurveto{\pgfqpoint{0.710424in}{0.364359in}}{\pgfqpoint{0.715724in}{0.362164in}}{\pgfqpoint{0.721249in}{0.362164in}}%
\pgfpathclose%
\pgfusepath{stroke,fill}%
\end{pgfscope}%
\begin{pgfscope}%
\pgfpathrectangle{\pgfqpoint{0.562500in}{0.275000in}}{\pgfqpoint{3.487500in}{1.925000in}}%
\pgfusepath{clip}%
\pgfsetbuttcap%
\pgfsetroundjoin%
\definecolor{currentfill}{rgb}{0.000000,0.000000,0.000000}%
\pgfsetfillcolor{currentfill}%
\pgfsetlinewidth{1.003750pt}%
\definecolor{currentstroke}{rgb}{0.000000,0.000000,0.000000}%
\pgfsetstrokecolor{currentstroke}%
\pgfsetdash{}{0pt}%
\pgfpathmoveto{\pgfqpoint{0.721249in}{0.377340in}}%
\pgfpathcurveto{\pgfqpoint{0.726774in}{0.377340in}}{\pgfqpoint{0.732073in}{0.379535in}}{\pgfqpoint{0.735980in}{0.383442in}}%
\pgfpathcurveto{\pgfqpoint{0.739887in}{0.387349in}}{\pgfqpoint{0.742082in}{0.392648in}}{\pgfqpoint{0.742082in}{0.398173in}}%
\pgfpathcurveto{\pgfqpoint{0.742082in}{0.403698in}}{\pgfqpoint{0.739887in}{0.408998in}}{\pgfqpoint{0.735980in}{0.412905in}}%
\pgfpathcurveto{\pgfqpoint{0.732073in}{0.416811in}}{\pgfqpoint{0.726774in}{0.419006in}}{\pgfqpoint{0.721249in}{0.419006in}}%
\pgfpathcurveto{\pgfqpoint{0.715724in}{0.419006in}}{\pgfqpoint{0.710424in}{0.416811in}}{\pgfqpoint{0.706518in}{0.412905in}}%
\pgfpathcurveto{\pgfqpoint{0.702611in}{0.408998in}}{\pgfqpoint{0.700416in}{0.403698in}}{\pgfqpoint{0.700416in}{0.398173in}}%
\pgfpathcurveto{\pgfqpoint{0.700416in}{0.392648in}}{\pgfqpoint{0.702611in}{0.387349in}}{\pgfqpoint{0.706518in}{0.383442in}}%
\pgfpathcurveto{\pgfqpoint{0.710424in}{0.379535in}}{\pgfqpoint{0.715724in}{0.377340in}}{\pgfqpoint{0.721249in}{0.377340in}}%
\pgfpathclose%
\pgfusepath{stroke,fill}%
\end{pgfscope}%
\begin{pgfscope}%
\pgfpathrectangle{\pgfqpoint{0.562500in}{0.275000in}}{\pgfqpoint{3.487500in}{1.925000in}}%
\pgfusepath{clip}%
\pgfsetbuttcap%
\pgfsetroundjoin%
\definecolor{currentfill}{rgb}{0.000000,0.000000,0.000000}%
\pgfsetfillcolor{currentfill}%
\pgfsetlinewidth{1.003750pt}%
\definecolor{currentstroke}{rgb}{0.000000,0.000000,0.000000}%
\pgfsetstrokecolor{currentstroke}%
\pgfsetdash{}{0pt}%
\pgfpathmoveto{\pgfqpoint{0.721249in}{0.367223in}}%
\pgfpathcurveto{\pgfqpoint{0.726774in}{0.367223in}}{\pgfqpoint{0.732073in}{0.369418in}}{\pgfqpoint{0.735980in}{0.373325in}}%
\pgfpathcurveto{\pgfqpoint{0.739887in}{0.377231in}}{\pgfqpoint{0.742082in}{0.382531in}}{\pgfqpoint{0.742082in}{0.388056in}}%
\pgfpathcurveto{\pgfqpoint{0.742082in}{0.393581in}}{\pgfqpoint{0.739887in}{0.398881in}}{\pgfqpoint{0.735980in}{0.402787in}}%
\pgfpathcurveto{\pgfqpoint{0.732073in}{0.406694in}}{\pgfqpoint{0.726774in}{0.408889in}}{\pgfqpoint{0.721249in}{0.408889in}}%
\pgfpathcurveto{\pgfqpoint{0.715724in}{0.408889in}}{\pgfqpoint{0.710424in}{0.406694in}}{\pgfqpoint{0.706518in}{0.402787in}}%
\pgfpathcurveto{\pgfqpoint{0.702611in}{0.398881in}}{\pgfqpoint{0.700416in}{0.393581in}}{\pgfqpoint{0.700416in}{0.388056in}}%
\pgfpathcurveto{\pgfqpoint{0.700416in}{0.382531in}}{\pgfqpoint{0.702611in}{0.377231in}}{\pgfqpoint{0.706518in}{0.373325in}}%
\pgfpathcurveto{\pgfqpoint{0.710424in}{0.369418in}}{\pgfqpoint{0.715724in}{0.367223in}}{\pgfqpoint{0.721249in}{0.367223in}}%
\pgfpathclose%
\pgfusepath{stroke,fill}%
\end{pgfscope}%
\begin{pgfscope}%
\pgfpathrectangle{\pgfqpoint{0.562500in}{0.275000in}}{\pgfqpoint{3.487500in}{1.925000in}}%
\pgfusepath{clip}%
\pgfsetbuttcap%
\pgfsetroundjoin%
\definecolor{currentfill}{rgb}{0.000000,0.000000,0.000000}%
\pgfsetfillcolor{currentfill}%
\pgfsetlinewidth{1.003750pt}%
\definecolor{currentstroke}{rgb}{0.000000,0.000000,0.000000}%
\pgfsetstrokecolor{currentstroke}%
\pgfsetdash{}{0pt}%
\pgfpathmoveto{\pgfqpoint{0.721249in}{0.357106in}}%
\pgfpathcurveto{\pgfqpoint{0.726774in}{0.357106in}}{\pgfqpoint{0.732073in}{0.359301in}}{\pgfqpoint{0.735980in}{0.363207in}}%
\pgfpathcurveto{\pgfqpoint{0.739887in}{0.367114in}}{\pgfqpoint{0.742082in}{0.372414in}}{\pgfqpoint{0.742082in}{0.377939in}}%
\pgfpathcurveto{\pgfqpoint{0.742082in}{0.383464in}}{\pgfqpoint{0.739887in}{0.388763in}}{\pgfqpoint{0.735980in}{0.392670in}}%
\pgfpathcurveto{\pgfqpoint{0.732073in}{0.396577in}}{\pgfqpoint{0.726774in}{0.398772in}}{\pgfqpoint{0.721249in}{0.398772in}}%
\pgfpathcurveto{\pgfqpoint{0.715724in}{0.398772in}}{\pgfqpoint{0.710424in}{0.396577in}}{\pgfqpoint{0.706518in}{0.392670in}}%
\pgfpathcurveto{\pgfqpoint{0.702611in}{0.388763in}}{\pgfqpoint{0.700416in}{0.383464in}}{\pgfqpoint{0.700416in}{0.377939in}}%
\pgfpathcurveto{\pgfqpoint{0.700416in}{0.372414in}}{\pgfqpoint{0.702611in}{0.367114in}}{\pgfqpoint{0.706518in}{0.363207in}}%
\pgfpathcurveto{\pgfqpoint{0.710424in}{0.359301in}}{\pgfqpoint{0.715724in}{0.357106in}}{\pgfqpoint{0.721249in}{0.357106in}}%
\pgfpathclose%
\pgfusepath{stroke,fill}%
\end{pgfscope}%
\begin{pgfscope}%
\pgfpathrectangle{\pgfqpoint{0.562500in}{0.275000in}}{\pgfqpoint{3.487500in}{1.925000in}}%
\pgfusepath{clip}%
\pgfsetbuttcap%
\pgfsetroundjoin%
\definecolor{currentfill}{rgb}{0.000000,0.000000,0.000000}%
\pgfsetfillcolor{currentfill}%
\pgfsetlinewidth{1.003750pt}%
\definecolor{currentstroke}{rgb}{0.000000,0.000000,0.000000}%
\pgfsetstrokecolor{currentstroke}%
\pgfsetdash{}{0pt}%
\pgfpathmoveto{\pgfqpoint{0.721249in}{0.352047in}}%
\pgfpathcurveto{\pgfqpoint{0.726774in}{0.352047in}}{\pgfqpoint{0.732073in}{0.354242in}}{\pgfqpoint{0.735980in}{0.358149in}}%
\pgfpathcurveto{\pgfqpoint{0.739887in}{0.362056in}}{\pgfqpoint{0.742082in}{0.367355in}}{\pgfqpoint{0.742082in}{0.372880in}}%
\pgfpathcurveto{\pgfqpoint{0.742082in}{0.378405in}}{\pgfqpoint{0.739887in}{0.383705in}}{\pgfqpoint{0.735980in}{0.387612in}}%
\pgfpathcurveto{\pgfqpoint{0.732073in}{0.391519in}}{\pgfqpoint{0.726774in}{0.393714in}}{\pgfqpoint{0.721249in}{0.393714in}}%
\pgfpathcurveto{\pgfqpoint{0.715724in}{0.393714in}}{\pgfqpoint{0.710424in}{0.391519in}}{\pgfqpoint{0.706518in}{0.387612in}}%
\pgfpathcurveto{\pgfqpoint{0.702611in}{0.383705in}}{\pgfqpoint{0.700416in}{0.378405in}}{\pgfqpoint{0.700416in}{0.372880in}}%
\pgfpathcurveto{\pgfqpoint{0.700416in}{0.367355in}}{\pgfqpoint{0.702611in}{0.362056in}}{\pgfqpoint{0.706518in}{0.358149in}}%
\pgfpathcurveto{\pgfqpoint{0.710424in}{0.354242in}}{\pgfqpoint{0.715724in}{0.352047in}}{\pgfqpoint{0.721249in}{0.352047in}}%
\pgfpathclose%
\pgfusepath{stroke,fill}%
\end{pgfscope}%
\begin{pgfscope}%
\pgfpathrectangle{\pgfqpoint{0.562500in}{0.275000in}}{\pgfqpoint{3.487500in}{1.925000in}}%
\pgfusepath{clip}%
\pgfsetbuttcap%
\pgfsetroundjoin%
\definecolor{currentfill}{rgb}{0.000000,0.000000,0.000000}%
\pgfsetfillcolor{currentfill}%
\pgfsetlinewidth{1.003750pt}%
\definecolor{currentstroke}{rgb}{0.000000,0.000000,0.000000}%
\pgfsetstrokecolor{currentstroke}%
\pgfsetdash{}{0pt}%
\pgfpathmoveto{\pgfqpoint{0.721249in}{0.357106in}}%
\pgfpathcurveto{\pgfqpoint{0.726774in}{0.357106in}}{\pgfqpoint{0.732073in}{0.359301in}}{\pgfqpoint{0.735980in}{0.363207in}}%
\pgfpathcurveto{\pgfqpoint{0.739887in}{0.367114in}}{\pgfqpoint{0.742082in}{0.372414in}}{\pgfqpoint{0.742082in}{0.377939in}}%
\pgfpathcurveto{\pgfqpoint{0.742082in}{0.383464in}}{\pgfqpoint{0.739887in}{0.388763in}}{\pgfqpoint{0.735980in}{0.392670in}}%
\pgfpathcurveto{\pgfqpoint{0.732073in}{0.396577in}}{\pgfqpoint{0.726774in}{0.398772in}}{\pgfqpoint{0.721249in}{0.398772in}}%
\pgfpathcurveto{\pgfqpoint{0.715724in}{0.398772in}}{\pgfqpoint{0.710424in}{0.396577in}}{\pgfqpoint{0.706518in}{0.392670in}}%
\pgfpathcurveto{\pgfqpoint{0.702611in}{0.388763in}}{\pgfqpoint{0.700416in}{0.383464in}}{\pgfqpoint{0.700416in}{0.377939in}}%
\pgfpathcurveto{\pgfqpoint{0.700416in}{0.372414in}}{\pgfqpoint{0.702611in}{0.367114in}}{\pgfqpoint{0.706518in}{0.363207in}}%
\pgfpathcurveto{\pgfqpoint{0.710424in}{0.359301in}}{\pgfqpoint{0.715724in}{0.357106in}}{\pgfqpoint{0.721249in}{0.357106in}}%
\pgfpathclose%
\pgfusepath{stroke,fill}%
\end{pgfscope}%
\begin{pgfscope}%
\pgfpathrectangle{\pgfqpoint{0.562500in}{0.275000in}}{\pgfqpoint{3.487500in}{1.925000in}}%
\pgfusepath{clip}%
\pgfsetbuttcap%
\pgfsetroundjoin%
\definecolor{currentfill}{rgb}{0.000000,0.000000,0.000000}%
\pgfsetfillcolor{currentfill}%
\pgfsetlinewidth{1.003750pt}%
\definecolor{currentstroke}{rgb}{0.000000,0.000000,0.000000}%
\pgfsetstrokecolor{currentstroke}%
\pgfsetdash{}{0pt}%
\pgfpathmoveto{\pgfqpoint{0.721249in}{0.352047in}}%
\pgfpathcurveto{\pgfqpoint{0.726774in}{0.352047in}}{\pgfqpoint{0.732073in}{0.354242in}}{\pgfqpoint{0.735980in}{0.358149in}}%
\pgfpathcurveto{\pgfqpoint{0.739887in}{0.362056in}}{\pgfqpoint{0.742082in}{0.367355in}}{\pgfqpoint{0.742082in}{0.372880in}}%
\pgfpathcurveto{\pgfqpoint{0.742082in}{0.378405in}}{\pgfqpoint{0.739887in}{0.383705in}}{\pgfqpoint{0.735980in}{0.387612in}}%
\pgfpathcurveto{\pgfqpoint{0.732073in}{0.391519in}}{\pgfqpoint{0.726774in}{0.393714in}}{\pgfqpoint{0.721249in}{0.393714in}}%
\pgfpathcurveto{\pgfqpoint{0.715724in}{0.393714in}}{\pgfqpoint{0.710424in}{0.391519in}}{\pgfqpoint{0.706518in}{0.387612in}}%
\pgfpathcurveto{\pgfqpoint{0.702611in}{0.383705in}}{\pgfqpoint{0.700416in}{0.378405in}}{\pgfqpoint{0.700416in}{0.372880in}}%
\pgfpathcurveto{\pgfqpoint{0.700416in}{0.367355in}}{\pgfqpoint{0.702611in}{0.362056in}}{\pgfqpoint{0.706518in}{0.358149in}}%
\pgfpathcurveto{\pgfqpoint{0.710424in}{0.354242in}}{\pgfqpoint{0.715724in}{0.352047in}}{\pgfqpoint{0.721249in}{0.352047in}}%
\pgfpathclose%
\pgfusepath{stroke,fill}%
\end{pgfscope}%
\begin{pgfscope}%
\pgfpathrectangle{\pgfqpoint{0.562500in}{0.275000in}}{\pgfqpoint{3.487500in}{1.925000in}}%
\pgfusepath{clip}%
\pgfsetbuttcap%
\pgfsetroundjoin%
\definecolor{currentfill}{rgb}{0.000000,0.000000,0.000000}%
\pgfsetfillcolor{currentfill}%
\pgfsetlinewidth{1.003750pt}%
\definecolor{currentstroke}{rgb}{0.000000,0.000000,0.000000}%
\pgfsetstrokecolor{currentstroke}%
\pgfsetdash{}{0pt}%
\pgfpathmoveto{\pgfqpoint{0.721249in}{0.357106in}}%
\pgfpathcurveto{\pgfqpoint{0.726774in}{0.357106in}}{\pgfqpoint{0.732073in}{0.359301in}}{\pgfqpoint{0.735980in}{0.363207in}}%
\pgfpathcurveto{\pgfqpoint{0.739887in}{0.367114in}}{\pgfqpoint{0.742082in}{0.372414in}}{\pgfqpoint{0.742082in}{0.377939in}}%
\pgfpathcurveto{\pgfqpoint{0.742082in}{0.383464in}}{\pgfqpoint{0.739887in}{0.388763in}}{\pgfqpoint{0.735980in}{0.392670in}}%
\pgfpathcurveto{\pgfqpoint{0.732073in}{0.396577in}}{\pgfqpoint{0.726774in}{0.398772in}}{\pgfqpoint{0.721249in}{0.398772in}}%
\pgfpathcurveto{\pgfqpoint{0.715724in}{0.398772in}}{\pgfqpoint{0.710424in}{0.396577in}}{\pgfqpoint{0.706518in}{0.392670in}}%
\pgfpathcurveto{\pgfqpoint{0.702611in}{0.388763in}}{\pgfqpoint{0.700416in}{0.383464in}}{\pgfqpoint{0.700416in}{0.377939in}}%
\pgfpathcurveto{\pgfqpoint{0.700416in}{0.372414in}}{\pgfqpoint{0.702611in}{0.367114in}}{\pgfqpoint{0.706518in}{0.363207in}}%
\pgfpathcurveto{\pgfqpoint{0.710424in}{0.359301in}}{\pgfqpoint{0.715724in}{0.357106in}}{\pgfqpoint{0.721249in}{0.357106in}}%
\pgfpathclose%
\pgfusepath{stroke,fill}%
\end{pgfscope}%
\begin{pgfscope}%
\pgfpathrectangle{\pgfqpoint{0.562500in}{0.275000in}}{\pgfqpoint{3.487500in}{1.925000in}}%
\pgfusepath{clip}%
\pgfsetbuttcap%
\pgfsetroundjoin%
\definecolor{currentfill}{rgb}{0.000000,0.000000,0.000000}%
\pgfsetfillcolor{currentfill}%
\pgfsetlinewidth{1.003750pt}%
\definecolor{currentstroke}{rgb}{0.000000,0.000000,0.000000}%
\pgfsetstrokecolor{currentstroke}%
\pgfsetdash{}{0pt}%
\pgfpathmoveto{\pgfqpoint{0.721249in}{0.357106in}}%
\pgfpathcurveto{\pgfqpoint{0.726774in}{0.357106in}}{\pgfqpoint{0.732073in}{0.359301in}}{\pgfqpoint{0.735980in}{0.363207in}}%
\pgfpathcurveto{\pgfqpoint{0.739887in}{0.367114in}}{\pgfqpoint{0.742082in}{0.372414in}}{\pgfqpoint{0.742082in}{0.377939in}}%
\pgfpathcurveto{\pgfqpoint{0.742082in}{0.383464in}}{\pgfqpoint{0.739887in}{0.388763in}}{\pgfqpoint{0.735980in}{0.392670in}}%
\pgfpathcurveto{\pgfqpoint{0.732073in}{0.396577in}}{\pgfqpoint{0.726774in}{0.398772in}}{\pgfqpoint{0.721249in}{0.398772in}}%
\pgfpathcurveto{\pgfqpoint{0.715724in}{0.398772in}}{\pgfqpoint{0.710424in}{0.396577in}}{\pgfqpoint{0.706518in}{0.392670in}}%
\pgfpathcurveto{\pgfqpoint{0.702611in}{0.388763in}}{\pgfqpoint{0.700416in}{0.383464in}}{\pgfqpoint{0.700416in}{0.377939in}}%
\pgfpathcurveto{\pgfqpoint{0.700416in}{0.372414in}}{\pgfqpoint{0.702611in}{0.367114in}}{\pgfqpoint{0.706518in}{0.363207in}}%
\pgfpathcurveto{\pgfqpoint{0.710424in}{0.359301in}}{\pgfqpoint{0.715724in}{0.357106in}}{\pgfqpoint{0.721249in}{0.357106in}}%
\pgfpathclose%
\pgfusepath{stroke,fill}%
\end{pgfscope}%
\begin{pgfscope}%
\pgfpathrectangle{\pgfqpoint{0.562500in}{0.275000in}}{\pgfqpoint{3.487500in}{1.925000in}}%
\pgfusepath{clip}%
\pgfsetbuttcap%
\pgfsetroundjoin%
\definecolor{currentfill}{rgb}{0.000000,0.000000,0.000000}%
\pgfsetfillcolor{currentfill}%
\pgfsetlinewidth{1.003750pt}%
\definecolor{currentstroke}{rgb}{0.000000,0.000000,0.000000}%
\pgfsetstrokecolor{currentstroke}%
\pgfsetdash{}{0pt}%
\pgfpathmoveto{\pgfqpoint{0.721249in}{0.367223in}}%
\pgfpathcurveto{\pgfqpoint{0.726774in}{0.367223in}}{\pgfqpoint{0.732073in}{0.369418in}}{\pgfqpoint{0.735980in}{0.373325in}}%
\pgfpathcurveto{\pgfqpoint{0.739887in}{0.377231in}}{\pgfqpoint{0.742082in}{0.382531in}}{\pgfqpoint{0.742082in}{0.388056in}}%
\pgfpathcurveto{\pgfqpoint{0.742082in}{0.393581in}}{\pgfqpoint{0.739887in}{0.398881in}}{\pgfqpoint{0.735980in}{0.402787in}}%
\pgfpathcurveto{\pgfqpoint{0.732073in}{0.406694in}}{\pgfqpoint{0.726774in}{0.408889in}}{\pgfqpoint{0.721249in}{0.408889in}}%
\pgfpathcurveto{\pgfqpoint{0.715724in}{0.408889in}}{\pgfqpoint{0.710424in}{0.406694in}}{\pgfqpoint{0.706518in}{0.402787in}}%
\pgfpathcurveto{\pgfqpoint{0.702611in}{0.398881in}}{\pgfqpoint{0.700416in}{0.393581in}}{\pgfqpoint{0.700416in}{0.388056in}}%
\pgfpathcurveto{\pgfqpoint{0.700416in}{0.382531in}}{\pgfqpoint{0.702611in}{0.377231in}}{\pgfqpoint{0.706518in}{0.373325in}}%
\pgfpathcurveto{\pgfqpoint{0.710424in}{0.369418in}}{\pgfqpoint{0.715724in}{0.367223in}}{\pgfqpoint{0.721249in}{0.367223in}}%
\pgfpathclose%
\pgfusepath{stroke,fill}%
\end{pgfscope}%
\begin{pgfscope}%
\pgfpathrectangle{\pgfqpoint{0.562500in}{0.275000in}}{\pgfqpoint{3.487500in}{1.925000in}}%
\pgfusepath{clip}%
\pgfsetbuttcap%
\pgfsetroundjoin%
\definecolor{currentfill}{rgb}{0.000000,0.000000,0.000000}%
\pgfsetfillcolor{currentfill}%
\pgfsetlinewidth{1.003750pt}%
\definecolor{currentstroke}{rgb}{0.000000,0.000000,0.000000}%
\pgfsetstrokecolor{currentstroke}%
\pgfsetdash{}{0pt}%
\pgfpathmoveto{\pgfqpoint{0.721249in}{0.367223in}}%
\pgfpathcurveto{\pgfqpoint{0.726774in}{0.367223in}}{\pgfqpoint{0.732073in}{0.369418in}}{\pgfqpoint{0.735980in}{0.373325in}}%
\pgfpathcurveto{\pgfqpoint{0.739887in}{0.377231in}}{\pgfqpoint{0.742082in}{0.382531in}}{\pgfqpoint{0.742082in}{0.388056in}}%
\pgfpathcurveto{\pgfqpoint{0.742082in}{0.393581in}}{\pgfqpoint{0.739887in}{0.398881in}}{\pgfqpoint{0.735980in}{0.402787in}}%
\pgfpathcurveto{\pgfqpoint{0.732073in}{0.406694in}}{\pgfqpoint{0.726774in}{0.408889in}}{\pgfqpoint{0.721249in}{0.408889in}}%
\pgfpathcurveto{\pgfqpoint{0.715724in}{0.408889in}}{\pgfqpoint{0.710424in}{0.406694in}}{\pgfqpoint{0.706518in}{0.402787in}}%
\pgfpathcurveto{\pgfqpoint{0.702611in}{0.398881in}}{\pgfqpoint{0.700416in}{0.393581in}}{\pgfqpoint{0.700416in}{0.388056in}}%
\pgfpathcurveto{\pgfqpoint{0.700416in}{0.382531in}}{\pgfqpoint{0.702611in}{0.377231in}}{\pgfqpoint{0.706518in}{0.373325in}}%
\pgfpathcurveto{\pgfqpoint{0.710424in}{0.369418in}}{\pgfqpoint{0.715724in}{0.367223in}}{\pgfqpoint{0.721249in}{0.367223in}}%
\pgfpathclose%
\pgfusepath{stroke,fill}%
\end{pgfscope}%
\begin{pgfscope}%
\pgfpathrectangle{\pgfqpoint{0.562500in}{0.275000in}}{\pgfqpoint{3.487500in}{1.925000in}}%
\pgfusepath{clip}%
\pgfsetbuttcap%
\pgfsetroundjoin%
\definecolor{currentfill}{rgb}{0.000000,0.000000,0.000000}%
\pgfsetfillcolor{currentfill}%
\pgfsetlinewidth{1.003750pt}%
\definecolor{currentstroke}{rgb}{0.000000,0.000000,0.000000}%
\pgfsetstrokecolor{currentstroke}%
\pgfsetdash{}{0pt}%
\pgfpathmoveto{\pgfqpoint{0.721249in}{0.362164in}}%
\pgfpathcurveto{\pgfqpoint{0.726774in}{0.362164in}}{\pgfqpoint{0.732073in}{0.364359in}}{\pgfqpoint{0.735980in}{0.368266in}}%
\pgfpathcurveto{\pgfqpoint{0.739887in}{0.372173in}}{\pgfqpoint{0.742082in}{0.377472in}}{\pgfqpoint{0.742082in}{0.382997in}}%
\pgfpathcurveto{\pgfqpoint{0.742082in}{0.388522in}}{\pgfqpoint{0.739887in}{0.393822in}}{\pgfqpoint{0.735980in}{0.397729in}}%
\pgfpathcurveto{\pgfqpoint{0.732073in}{0.401636in}}{\pgfqpoint{0.726774in}{0.403831in}}{\pgfqpoint{0.721249in}{0.403831in}}%
\pgfpathcurveto{\pgfqpoint{0.715724in}{0.403831in}}{\pgfqpoint{0.710424in}{0.401636in}}{\pgfqpoint{0.706518in}{0.397729in}}%
\pgfpathcurveto{\pgfqpoint{0.702611in}{0.393822in}}{\pgfqpoint{0.700416in}{0.388522in}}{\pgfqpoint{0.700416in}{0.382997in}}%
\pgfpathcurveto{\pgfqpoint{0.700416in}{0.377472in}}{\pgfqpoint{0.702611in}{0.372173in}}{\pgfqpoint{0.706518in}{0.368266in}}%
\pgfpathcurveto{\pgfqpoint{0.710424in}{0.364359in}}{\pgfqpoint{0.715724in}{0.362164in}}{\pgfqpoint{0.721249in}{0.362164in}}%
\pgfpathclose%
\pgfusepath{stroke,fill}%
\end{pgfscope}%
\begin{pgfscope}%
\pgfpathrectangle{\pgfqpoint{0.562500in}{0.275000in}}{\pgfqpoint{3.487500in}{1.925000in}}%
\pgfusepath{clip}%
\pgfsetbuttcap%
\pgfsetroundjoin%
\definecolor{currentfill}{rgb}{0.000000,0.000000,0.000000}%
\pgfsetfillcolor{currentfill}%
\pgfsetlinewidth{1.003750pt}%
\definecolor{currentstroke}{rgb}{0.000000,0.000000,0.000000}%
\pgfsetstrokecolor{currentstroke}%
\pgfsetdash{}{0pt}%
\pgfpathmoveto{\pgfqpoint{0.721249in}{0.357106in}}%
\pgfpathcurveto{\pgfqpoint{0.726774in}{0.357106in}}{\pgfqpoint{0.732073in}{0.359301in}}{\pgfqpoint{0.735980in}{0.363207in}}%
\pgfpathcurveto{\pgfqpoint{0.739887in}{0.367114in}}{\pgfqpoint{0.742082in}{0.372414in}}{\pgfqpoint{0.742082in}{0.377939in}}%
\pgfpathcurveto{\pgfqpoint{0.742082in}{0.383464in}}{\pgfqpoint{0.739887in}{0.388763in}}{\pgfqpoint{0.735980in}{0.392670in}}%
\pgfpathcurveto{\pgfqpoint{0.732073in}{0.396577in}}{\pgfqpoint{0.726774in}{0.398772in}}{\pgfqpoint{0.721249in}{0.398772in}}%
\pgfpathcurveto{\pgfqpoint{0.715724in}{0.398772in}}{\pgfqpoint{0.710424in}{0.396577in}}{\pgfqpoint{0.706518in}{0.392670in}}%
\pgfpathcurveto{\pgfqpoint{0.702611in}{0.388763in}}{\pgfqpoint{0.700416in}{0.383464in}}{\pgfqpoint{0.700416in}{0.377939in}}%
\pgfpathcurveto{\pgfqpoint{0.700416in}{0.372414in}}{\pgfqpoint{0.702611in}{0.367114in}}{\pgfqpoint{0.706518in}{0.363207in}}%
\pgfpathcurveto{\pgfqpoint{0.710424in}{0.359301in}}{\pgfqpoint{0.715724in}{0.357106in}}{\pgfqpoint{0.721249in}{0.357106in}}%
\pgfpathclose%
\pgfusepath{stroke,fill}%
\end{pgfscope}%
\begin{pgfscope}%
\pgfpathrectangle{\pgfqpoint{0.562500in}{0.275000in}}{\pgfqpoint{3.487500in}{1.925000in}}%
\pgfusepath{clip}%
\pgfsetbuttcap%
\pgfsetroundjoin%
\definecolor{currentfill}{rgb}{0.000000,0.000000,0.000000}%
\pgfsetfillcolor{currentfill}%
\pgfsetlinewidth{1.003750pt}%
\definecolor{currentstroke}{rgb}{0.000000,0.000000,0.000000}%
\pgfsetstrokecolor{currentstroke}%
\pgfsetdash{}{0pt}%
\pgfpathmoveto{\pgfqpoint{0.721249in}{0.362164in}}%
\pgfpathcurveto{\pgfqpoint{0.726774in}{0.362164in}}{\pgfqpoint{0.732073in}{0.364359in}}{\pgfqpoint{0.735980in}{0.368266in}}%
\pgfpathcurveto{\pgfqpoint{0.739887in}{0.372173in}}{\pgfqpoint{0.742082in}{0.377472in}}{\pgfqpoint{0.742082in}{0.382997in}}%
\pgfpathcurveto{\pgfqpoint{0.742082in}{0.388522in}}{\pgfqpoint{0.739887in}{0.393822in}}{\pgfqpoint{0.735980in}{0.397729in}}%
\pgfpathcurveto{\pgfqpoint{0.732073in}{0.401636in}}{\pgfqpoint{0.726774in}{0.403831in}}{\pgfqpoint{0.721249in}{0.403831in}}%
\pgfpathcurveto{\pgfqpoint{0.715724in}{0.403831in}}{\pgfqpoint{0.710424in}{0.401636in}}{\pgfqpoint{0.706518in}{0.397729in}}%
\pgfpathcurveto{\pgfqpoint{0.702611in}{0.393822in}}{\pgfqpoint{0.700416in}{0.388522in}}{\pgfqpoint{0.700416in}{0.382997in}}%
\pgfpathcurveto{\pgfqpoint{0.700416in}{0.377472in}}{\pgfqpoint{0.702611in}{0.372173in}}{\pgfqpoint{0.706518in}{0.368266in}}%
\pgfpathcurveto{\pgfqpoint{0.710424in}{0.364359in}}{\pgfqpoint{0.715724in}{0.362164in}}{\pgfqpoint{0.721249in}{0.362164in}}%
\pgfpathclose%
\pgfusepath{stroke,fill}%
\end{pgfscope}%
\begin{pgfscope}%
\pgfpathrectangle{\pgfqpoint{0.562500in}{0.275000in}}{\pgfqpoint{3.487500in}{1.925000in}}%
\pgfusepath{clip}%
\pgfsetbuttcap%
\pgfsetroundjoin%
\definecolor{currentfill}{rgb}{0.000000,0.000000,0.000000}%
\pgfsetfillcolor{currentfill}%
\pgfsetlinewidth{1.003750pt}%
\definecolor{currentstroke}{rgb}{0.000000,0.000000,0.000000}%
\pgfsetstrokecolor{currentstroke}%
\pgfsetdash{}{0pt}%
\pgfpathmoveto{\pgfqpoint{0.721249in}{0.367223in}}%
\pgfpathcurveto{\pgfqpoint{0.726774in}{0.367223in}}{\pgfqpoint{0.732073in}{0.369418in}}{\pgfqpoint{0.735980in}{0.373325in}}%
\pgfpathcurveto{\pgfqpoint{0.739887in}{0.377231in}}{\pgfqpoint{0.742082in}{0.382531in}}{\pgfqpoint{0.742082in}{0.388056in}}%
\pgfpathcurveto{\pgfqpoint{0.742082in}{0.393581in}}{\pgfqpoint{0.739887in}{0.398881in}}{\pgfqpoint{0.735980in}{0.402787in}}%
\pgfpathcurveto{\pgfqpoint{0.732073in}{0.406694in}}{\pgfqpoint{0.726774in}{0.408889in}}{\pgfqpoint{0.721249in}{0.408889in}}%
\pgfpathcurveto{\pgfqpoint{0.715724in}{0.408889in}}{\pgfqpoint{0.710424in}{0.406694in}}{\pgfqpoint{0.706518in}{0.402787in}}%
\pgfpathcurveto{\pgfqpoint{0.702611in}{0.398881in}}{\pgfqpoint{0.700416in}{0.393581in}}{\pgfqpoint{0.700416in}{0.388056in}}%
\pgfpathcurveto{\pgfqpoint{0.700416in}{0.382531in}}{\pgfqpoint{0.702611in}{0.377231in}}{\pgfqpoint{0.706518in}{0.373325in}}%
\pgfpathcurveto{\pgfqpoint{0.710424in}{0.369418in}}{\pgfqpoint{0.715724in}{0.367223in}}{\pgfqpoint{0.721249in}{0.367223in}}%
\pgfpathclose%
\pgfusepath{stroke,fill}%
\end{pgfscope}%
\begin{pgfscope}%
\pgfpathrectangle{\pgfqpoint{0.562500in}{0.275000in}}{\pgfqpoint{3.487500in}{1.925000in}}%
\pgfusepath{clip}%
\pgfsetbuttcap%
\pgfsetroundjoin%
\definecolor{currentfill}{rgb}{0.000000,0.000000,0.000000}%
\pgfsetfillcolor{currentfill}%
\pgfsetlinewidth{1.003750pt}%
\definecolor{currentstroke}{rgb}{0.000000,0.000000,0.000000}%
\pgfsetstrokecolor{currentstroke}%
\pgfsetdash{}{0pt}%
\pgfpathmoveto{\pgfqpoint{0.721249in}{0.357106in}}%
\pgfpathcurveto{\pgfqpoint{0.726774in}{0.357106in}}{\pgfqpoint{0.732073in}{0.359301in}}{\pgfqpoint{0.735980in}{0.363207in}}%
\pgfpathcurveto{\pgfqpoint{0.739887in}{0.367114in}}{\pgfqpoint{0.742082in}{0.372414in}}{\pgfqpoint{0.742082in}{0.377939in}}%
\pgfpathcurveto{\pgfqpoint{0.742082in}{0.383464in}}{\pgfqpoint{0.739887in}{0.388763in}}{\pgfqpoint{0.735980in}{0.392670in}}%
\pgfpathcurveto{\pgfqpoint{0.732073in}{0.396577in}}{\pgfqpoint{0.726774in}{0.398772in}}{\pgfqpoint{0.721249in}{0.398772in}}%
\pgfpathcurveto{\pgfqpoint{0.715724in}{0.398772in}}{\pgfqpoint{0.710424in}{0.396577in}}{\pgfqpoint{0.706518in}{0.392670in}}%
\pgfpathcurveto{\pgfqpoint{0.702611in}{0.388763in}}{\pgfqpoint{0.700416in}{0.383464in}}{\pgfqpoint{0.700416in}{0.377939in}}%
\pgfpathcurveto{\pgfqpoint{0.700416in}{0.372414in}}{\pgfqpoint{0.702611in}{0.367114in}}{\pgfqpoint{0.706518in}{0.363207in}}%
\pgfpathcurveto{\pgfqpoint{0.710424in}{0.359301in}}{\pgfqpoint{0.715724in}{0.357106in}}{\pgfqpoint{0.721249in}{0.357106in}}%
\pgfpathclose%
\pgfusepath{stroke,fill}%
\end{pgfscope}%
\begin{pgfscope}%
\pgfpathrectangle{\pgfqpoint{0.562500in}{0.275000in}}{\pgfqpoint{3.487500in}{1.925000in}}%
\pgfusepath{clip}%
\pgfsetbuttcap%
\pgfsetroundjoin%
\definecolor{currentfill}{rgb}{0.000000,0.000000,0.000000}%
\pgfsetfillcolor{currentfill}%
\pgfsetlinewidth{1.003750pt}%
\definecolor{currentstroke}{rgb}{0.000000,0.000000,0.000000}%
\pgfsetstrokecolor{currentstroke}%
\pgfsetdash{}{0pt}%
\pgfpathmoveto{\pgfqpoint{0.721249in}{0.352047in}}%
\pgfpathcurveto{\pgfqpoint{0.726774in}{0.352047in}}{\pgfqpoint{0.732073in}{0.354242in}}{\pgfqpoint{0.735980in}{0.358149in}}%
\pgfpathcurveto{\pgfqpoint{0.739887in}{0.362056in}}{\pgfqpoint{0.742082in}{0.367355in}}{\pgfqpoint{0.742082in}{0.372880in}}%
\pgfpathcurveto{\pgfqpoint{0.742082in}{0.378405in}}{\pgfqpoint{0.739887in}{0.383705in}}{\pgfqpoint{0.735980in}{0.387612in}}%
\pgfpathcurveto{\pgfqpoint{0.732073in}{0.391519in}}{\pgfqpoint{0.726774in}{0.393714in}}{\pgfqpoint{0.721249in}{0.393714in}}%
\pgfpathcurveto{\pgfqpoint{0.715724in}{0.393714in}}{\pgfqpoint{0.710424in}{0.391519in}}{\pgfqpoint{0.706518in}{0.387612in}}%
\pgfpathcurveto{\pgfqpoint{0.702611in}{0.383705in}}{\pgfqpoint{0.700416in}{0.378405in}}{\pgfqpoint{0.700416in}{0.372880in}}%
\pgfpathcurveto{\pgfqpoint{0.700416in}{0.367355in}}{\pgfqpoint{0.702611in}{0.362056in}}{\pgfqpoint{0.706518in}{0.358149in}}%
\pgfpathcurveto{\pgfqpoint{0.710424in}{0.354242in}}{\pgfqpoint{0.715724in}{0.352047in}}{\pgfqpoint{0.721249in}{0.352047in}}%
\pgfpathclose%
\pgfusepath{stroke,fill}%
\end{pgfscope}%
\begin{pgfscope}%
\pgfpathrectangle{\pgfqpoint{0.562500in}{0.275000in}}{\pgfqpoint{3.487500in}{1.925000in}}%
\pgfusepath{clip}%
\pgfsetbuttcap%
\pgfsetroundjoin%
\definecolor{currentfill}{rgb}{0.000000,0.000000,0.000000}%
\pgfsetfillcolor{currentfill}%
\pgfsetlinewidth{1.003750pt}%
\definecolor{currentstroke}{rgb}{0.000000,0.000000,0.000000}%
\pgfsetstrokecolor{currentstroke}%
\pgfsetdash{}{0pt}%
\pgfpathmoveto{\pgfqpoint{0.721249in}{0.346988in}}%
\pgfpathcurveto{\pgfqpoint{0.726774in}{0.346988in}}{\pgfqpoint{0.732073in}{0.349184in}}{\pgfqpoint{0.735980in}{0.353090in}}%
\pgfpathcurveto{\pgfqpoint{0.739887in}{0.356997in}}{\pgfqpoint{0.742082in}{0.362297in}}{\pgfqpoint{0.742082in}{0.367822in}}%
\pgfpathcurveto{\pgfqpoint{0.742082in}{0.373347in}}{\pgfqpoint{0.739887in}{0.378646in}}{\pgfqpoint{0.735980in}{0.382553in}}%
\pgfpathcurveto{\pgfqpoint{0.732073in}{0.386460in}}{\pgfqpoint{0.726774in}{0.388655in}}{\pgfqpoint{0.721249in}{0.388655in}}%
\pgfpathcurveto{\pgfqpoint{0.715724in}{0.388655in}}{\pgfqpoint{0.710424in}{0.386460in}}{\pgfqpoint{0.706518in}{0.382553in}}%
\pgfpathcurveto{\pgfqpoint{0.702611in}{0.378646in}}{\pgfqpoint{0.700416in}{0.373347in}}{\pgfqpoint{0.700416in}{0.367822in}}%
\pgfpathcurveto{\pgfqpoint{0.700416in}{0.362297in}}{\pgfqpoint{0.702611in}{0.356997in}}{\pgfqpoint{0.706518in}{0.353090in}}%
\pgfpathcurveto{\pgfqpoint{0.710424in}{0.349184in}}{\pgfqpoint{0.715724in}{0.346988in}}{\pgfqpoint{0.721249in}{0.346988in}}%
\pgfpathclose%
\pgfusepath{stroke,fill}%
\end{pgfscope}%
\begin{pgfscope}%
\pgfpathrectangle{\pgfqpoint{0.562500in}{0.275000in}}{\pgfqpoint{3.487500in}{1.925000in}}%
\pgfusepath{clip}%
\pgfsetbuttcap%
\pgfsetroundjoin%
\definecolor{currentfill}{rgb}{0.000000,0.000000,0.000000}%
\pgfsetfillcolor{currentfill}%
\pgfsetlinewidth{1.003750pt}%
\definecolor{currentstroke}{rgb}{0.000000,0.000000,0.000000}%
\pgfsetstrokecolor{currentstroke}%
\pgfsetdash{}{0pt}%
\pgfpathmoveto{\pgfqpoint{0.721249in}{0.357106in}}%
\pgfpathcurveto{\pgfqpoint{0.726774in}{0.357106in}}{\pgfqpoint{0.732073in}{0.359301in}}{\pgfqpoint{0.735980in}{0.363207in}}%
\pgfpathcurveto{\pgfqpoint{0.739887in}{0.367114in}}{\pgfqpoint{0.742082in}{0.372414in}}{\pgfqpoint{0.742082in}{0.377939in}}%
\pgfpathcurveto{\pgfqpoint{0.742082in}{0.383464in}}{\pgfqpoint{0.739887in}{0.388763in}}{\pgfqpoint{0.735980in}{0.392670in}}%
\pgfpathcurveto{\pgfqpoint{0.732073in}{0.396577in}}{\pgfqpoint{0.726774in}{0.398772in}}{\pgfqpoint{0.721249in}{0.398772in}}%
\pgfpathcurveto{\pgfqpoint{0.715724in}{0.398772in}}{\pgfqpoint{0.710424in}{0.396577in}}{\pgfqpoint{0.706518in}{0.392670in}}%
\pgfpathcurveto{\pgfqpoint{0.702611in}{0.388763in}}{\pgfqpoint{0.700416in}{0.383464in}}{\pgfqpoint{0.700416in}{0.377939in}}%
\pgfpathcurveto{\pgfqpoint{0.700416in}{0.372414in}}{\pgfqpoint{0.702611in}{0.367114in}}{\pgfqpoint{0.706518in}{0.363207in}}%
\pgfpathcurveto{\pgfqpoint{0.710424in}{0.359301in}}{\pgfqpoint{0.715724in}{0.357106in}}{\pgfqpoint{0.721249in}{0.357106in}}%
\pgfpathclose%
\pgfusepath{stroke,fill}%
\end{pgfscope}%
\begin{pgfscope}%
\pgfpathrectangle{\pgfqpoint{0.562500in}{0.275000in}}{\pgfqpoint{3.487500in}{1.925000in}}%
\pgfusepath{clip}%
\pgfsetbuttcap%
\pgfsetroundjoin%
\definecolor{currentfill}{rgb}{0.000000,0.000000,0.000000}%
\pgfsetfillcolor{currentfill}%
\pgfsetlinewidth{1.003750pt}%
\definecolor{currentstroke}{rgb}{0.000000,0.000000,0.000000}%
\pgfsetstrokecolor{currentstroke}%
\pgfsetdash{}{0pt}%
\pgfpathmoveto{\pgfqpoint{0.721249in}{0.367223in}}%
\pgfpathcurveto{\pgfqpoint{0.726774in}{0.367223in}}{\pgfqpoint{0.732073in}{0.369418in}}{\pgfqpoint{0.735980in}{0.373325in}}%
\pgfpathcurveto{\pgfqpoint{0.739887in}{0.377231in}}{\pgfqpoint{0.742082in}{0.382531in}}{\pgfqpoint{0.742082in}{0.388056in}}%
\pgfpathcurveto{\pgfqpoint{0.742082in}{0.393581in}}{\pgfqpoint{0.739887in}{0.398881in}}{\pgfqpoint{0.735980in}{0.402787in}}%
\pgfpathcurveto{\pgfqpoint{0.732073in}{0.406694in}}{\pgfqpoint{0.726774in}{0.408889in}}{\pgfqpoint{0.721249in}{0.408889in}}%
\pgfpathcurveto{\pgfqpoint{0.715724in}{0.408889in}}{\pgfqpoint{0.710424in}{0.406694in}}{\pgfqpoint{0.706518in}{0.402787in}}%
\pgfpathcurveto{\pgfqpoint{0.702611in}{0.398881in}}{\pgfqpoint{0.700416in}{0.393581in}}{\pgfqpoint{0.700416in}{0.388056in}}%
\pgfpathcurveto{\pgfqpoint{0.700416in}{0.382531in}}{\pgfqpoint{0.702611in}{0.377231in}}{\pgfqpoint{0.706518in}{0.373325in}}%
\pgfpathcurveto{\pgfqpoint{0.710424in}{0.369418in}}{\pgfqpoint{0.715724in}{0.367223in}}{\pgfqpoint{0.721249in}{0.367223in}}%
\pgfpathclose%
\pgfusepath{stroke,fill}%
\end{pgfscope}%
\begin{pgfscope}%
\pgfpathrectangle{\pgfqpoint{0.562500in}{0.275000in}}{\pgfqpoint{3.487500in}{1.925000in}}%
\pgfusepath{clip}%
\pgfsetbuttcap%
\pgfsetroundjoin%
\definecolor{currentfill}{rgb}{0.000000,0.000000,0.000000}%
\pgfsetfillcolor{currentfill}%
\pgfsetlinewidth{1.003750pt}%
\definecolor{currentstroke}{rgb}{0.000000,0.000000,0.000000}%
\pgfsetstrokecolor{currentstroke}%
\pgfsetdash{}{0pt}%
\pgfpathmoveto{\pgfqpoint{0.721249in}{0.367223in}}%
\pgfpathcurveto{\pgfqpoint{0.726774in}{0.367223in}}{\pgfqpoint{0.732073in}{0.369418in}}{\pgfqpoint{0.735980in}{0.373325in}}%
\pgfpathcurveto{\pgfqpoint{0.739887in}{0.377231in}}{\pgfqpoint{0.742082in}{0.382531in}}{\pgfqpoint{0.742082in}{0.388056in}}%
\pgfpathcurveto{\pgfqpoint{0.742082in}{0.393581in}}{\pgfqpoint{0.739887in}{0.398881in}}{\pgfqpoint{0.735980in}{0.402787in}}%
\pgfpathcurveto{\pgfqpoint{0.732073in}{0.406694in}}{\pgfqpoint{0.726774in}{0.408889in}}{\pgfqpoint{0.721249in}{0.408889in}}%
\pgfpathcurveto{\pgfqpoint{0.715724in}{0.408889in}}{\pgfqpoint{0.710424in}{0.406694in}}{\pgfqpoint{0.706518in}{0.402787in}}%
\pgfpathcurveto{\pgfqpoint{0.702611in}{0.398881in}}{\pgfqpoint{0.700416in}{0.393581in}}{\pgfqpoint{0.700416in}{0.388056in}}%
\pgfpathcurveto{\pgfqpoint{0.700416in}{0.382531in}}{\pgfqpoint{0.702611in}{0.377231in}}{\pgfqpoint{0.706518in}{0.373325in}}%
\pgfpathcurveto{\pgfqpoint{0.710424in}{0.369418in}}{\pgfqpoint{0.715724in}{0.367223in}}{\pgfqpoint{0.721249in}{0.367223in}}%
\pgfpathclose%
\pgfusepath{stroke,fill}%
\end{pgfscope}%
\begin{pgfscope}%
\pgfpathrectangle{\pgfqpoint{0.562500in}{0.275000in}}{\pgfqpoint{3.487500in}{1.925000in}}%
\pgfusepath{clip}%
\pgfsetbuttcap%
\pgfsetroundjoin%
\definecolor{currentfill}{rgb}{0.000000,0.000000,0.000000}%
\pgfsetfillcolor{currentfill}%
\pgfsetlinewidth{1.003750pt}%
\definecolor{currentstroke}{rgb}{0.000000,0.000000,0.000000}%
\pgfsetstrokecolor{currentstroke}%
\pgfsetdash{}{0pt}%
\pgfpathmoveto{\pgfqpoint{0.721249in}{0.367223in}}%
\pgfpathcurveto{\pgfqpoint{0.726774in}{0.367223in}}{\pgfqpoint{0.732073in}{0.369418in}}{\pgfqpoint{0.735980in}{0.373325in}}%
\pgfpathcurveto{\pgfqpoint{0.739887in}{0.377231in}}{\pgfqpoint{0.742082in}{0.382531in}}{\pgfqpoint{0.742082in}{0.388056in}}%
\pgfpathcurveto{\pgfqpoint{0.742082in}{0.393581in}}{\pgfqpoint{0.739887in}{0.398881in}}{\pgfqpoint{0.735980in}{0.402787in}}%
\pgfpathcurveto{\pgfqpoint{0.732073in}{0.406694in}}{\pgfqpoint{0.726774in}{0.408889in}}{\pgfqpoint{0.721249in}{0.408889in}}%
\pgfpathcurveto{\pgfqpoint{0.715724in}{0.408889in}}{\pgfqpoint{0.710424in}{0.406694in}}{\pgfqpoint{0.706518in}{0.402787in}}%
\pgfpathcurveto{\pgfqpoint{0.702611in}{0.398881in}}{\pgfqpoint{0.700416in}{0.393581in}}{\pgfqpoint{0.700416in}{0.388056in}}%
\pgfpathcurveto{\pgfqpoint{0.700416in}{0.382531in}}{\pgfqpoint{0.702611in}{0.377231in}}{\pgfqpoint{0.706518in}{0.373325in}}%
\pgfpathcurveto{\pgfqpoint{0.710424in}{0.369418in}}{\pgfqpoint{0.715724in}{0.367223in}}{\pgfqpoint{0.721249in}{0.367223in}}%
\pgfpathclose%
\pgfusepath{stroke,fill}%
\end{pgfscope}%
\begin{pgfscope}%
\pgfpathrectangle{\pgfqpoint{0.562500in}{0.275000in}}{\pgfqpoint{3.487500in}{1.925000in}}%
\pgfusepath{clip}%
\pgfsetbuttcap%
\pgfsetroundjoin%
\definecolor{currentfill}{rgb}{0.000000,0.000000,0.000000}%
\pgfsetfillcolor{currentfill}%
\pgfsetlinewidth{1.003750pt}%
\definecolor{currentstroke}{rgb}{0.000000,0.000000,0.000000}%
\pgfsetstrokecolor{currentstroke}%
\pgfsetdash{}{0pt}%
\pgfpathmoveto{\pgfqpoint{0.721249in}{0.367223in}}%
\pgfpathcurveto{\pgfqpoint{0.726774in}{0.367223in}}{\pgfqpoint{0.732073in}{0.369418in}}{\pgfqpoint{0.735980in}{0.373325in}}%
\pgfpathcurveto{\pgfqpoint{0.739887in}{0.377231in}}{\pgfqpoint{0.742082in}{0.382531in}}{\pgfqpoint{0.742082in}{0.388056in}}%
\pgfpathcurveto{\pgfqpoint{0.742082in}{0.393581in}}{\pgfqpoint{0.739887in}{0.398881in}}{\pgfqpoint{0.735980in}{0.402787in}}%
\pgfpathcurveto{\pgfqpoint{0.732073in}{0.406694in}}{\pgfqpoint{0.726774in}{0.408889in}}{\pgfqpoint{0.721249in}{0.408889in}}%
\pgfpathcurveto{\pgfqpoint{0.715724in}{0.408889in}}{\pgfqpoint{0.710424in}{0.406694in}}{\pgfqpoint{0.706518in}{0.402787in}}%
\pgfpathcurveto{\pgfqpoint{0.702611in}{0.398881in}}{\pgfqpoint{0.700416in}{0.393581in}}{\pgfqpoint{0.700416in}{0.388056in}}%
\pgfpathcurveto{\pgfqpoint{0.700416in}{0.382531in}}{\pgfqpoint{0.702611in}{0.377231in}}{\pgfqpoint{0.706518in}{0.373325in}}%
\pgfpathcurveto{\pgfqpoint{0.710424in}{0.369418in}}{\pgfqpoint{0.715724in}{0.367223in}}{\pgfqpoint{0.721249in}{0.367223in}}%
\pgfpathclose%
\pgfusepath{stroke,fill}%
\end{pgfscope}%
\begin{pgfscope}%
\pgfpathrectangle{\pgfqpoint{0.562500in}{0.275000in}}{\pgfqpoint{3.487500in}{1.925000in}}%
\pgfusepath{clip}%
\pgfsetbuttcap%
\pgfsetroundjoin%
\definecolor{currentfill}{rgb}{0.000000,0.000000,0.000000}%
\pgfsetfillcolor{currentfill}%
\pgfsetlinewidth{1.003750pt}%
\definecolor{currentstroke}{rgb}{0.000000,0.000000,0.000000}%
\pgfsetstrokecolor{currentstroke}%
\pgfsetdash{}{0pt}%
\pgfpathmoveto{\pgfqpoint{0.721249in}{0.362164in}}%
\pgfpathcurveto{\pgfqpoint{0.726774in}{0.362164in}}{\pgfqpoint{0.732073in}{0.364359in}}{\pgfqpoint{0.735980in}{0.368266in}}%
\pgfpathcurveto{\pgfqpoint{0.739887in}{0.372173in}}{\pgfqpoint{0.742082in}{0.377472in}}{\pgfqpoint{0.742082in}{0.382997in}}%
\pgfpathcurveto{\pgfqpoint{0.742082in}{0.388522in}}{\pgfqpoint{0.739887in}{0.393822in}}{\pgfqpoint{0.735980in}{0.397729in}}%
\pgfpathcurveto{\pgfqpoint{0.732073in}{0.401636in}}{\pgfqpoint{0.726774in}{0.403831in}}{\pgfqpoint{0.721249in}{0.403831in}}%
\pgfpathcurveto{\pgfqpoint{0.715724in}{0.403831in}}{\pgfqpoint{0.710424in}{0.401636in}}{\pgfqpoint{0.706518in}{0.397729in}}%
\pgfpathcurveto{\pgfqpoint{0.702611in}{0.393822in}}{\pgfqpoint{0.700416in}{0.388522in}}{\pgfqpoint{0.700416in}{0.382997in}}%
\pgfpathcurveto{\pgfqpoint{0.700416in}{0.377472in}}{\pgfqpoint{0.702611in}{0.372173in}}{\pgfqpoint{0.706518in}{0.368266in}}%
\pgfpathcurveto{\pgfqpoint{0.710424in}{0.364359in}}{\pgfqpoint{0.715724in}{0.362164in}}{\pgfqpoint{0.721249in}{0.362164in}}%
\pgfpathclose%
\pgfusepath{stroke,fill}%
\end{pgfscope}%
\begin{pgfscope}%
\pgfpathrectangle{\pgfqpoint{0.562500in}{0.275000in}}{\pgfqpoint{3.487500in}{1.925000in}}%
\pgfusepath{clip}%
\pgfsetbuttcap%
\pgfsetroundjoin%
\definecolor{currentfill}{rgb}{0.000000,0.000000,0.000000}%
\pgfsetfillcolor{currentfill}%
\pgfsetlinewidth{1.003750pt}%
\definecolor{currentstroke}{rgb}{0.000000,0.000000,0.000000}%
\pgfsetstrokecolor{currentstroke}%
\pgfsetdash{}{0pt}%
\pgfpathmoveto{\pgfqpoint{0.721249in}{0.367223in}}%
\pgfpathcurveto{\pgfqpoint{0.726774in}{0.367223in}}{\pgfqpoint{0.732073in}{0.369418in}}{\pgfqpoint{0.735980in}{0.373325in}}%
\pgfpathcurveto{\pgfqpoint{0.739887in}{0.377231in}}{\pgfqpoint{0.742082in}{0.382531in}}{\pgfqpoint{0.742082in}{0.388056in}}%
\pgfpathcurveto{\pgfqpoint{0.742082in}{0.393581in}}{\pgfqpoint{0.739887in}{0.398881in}}{\pgfqpoint{0.735980in}{0.402787in}}%
\pgfpathcurveto{\pgfqpoint{0.732073in}{0.406694in}}{\pgfqpoint{0.726774in}{0.408889in}}{\pgfqpoint{0.721249in}{0.408889in}}%
\pgfpathcurveto{\pgfqpoint{0.715724in}{0.408889in}}{\pgfqpoint{0.710424in}{0.406694in}}{\pgfqpoint{0.706518in}{0.402787in}}%
\pgfpathcurveto{\pgfqpoint{0.702611in}{0.398881in}}{\pgfqpoint{0.700416in}{0.393581in}}{\pgfqpoint{0.700416in}{0.388056in}}%
\pgfpathcurveto{\pgfqpoint{0.700416in}{0.382531in}}{\pgfqpoint{0.702611in}{0.377231in}}{\pgfqpoint{0.706518in}{0.373325in}}%
\pgfpathcurveto{\pgfqpoint{0.710424in}{0.369418in}}{\pgfqpoint{0.715724in}{0.367223in}}{\pgfqpoint{0.721249in}{0.367223in}}%
\pgfpathclose%
\pgfusepath{stroke,fill}%
\end{pgfscope}%
\begin{pgfscope}%
\pgfpathrectangle{\pgfqpoint{0.562500in}{0.275000in}}{\pgfqpoint{3.487500in}{1.925000in}}%
\pgfusepath{clip}%
\pgfsetbuttcap%
\pgfsetroundjoin%
\definecolor{currentfill}{rgb}{0.000000,0.000000,0.000000}%
\pgfsetfillcolor{currentfill}%
\pgfsetlinewidth{1.003750pt}%
\definecolor{currentstroke}{rgb}{0.000000,0.000000,0.000000}%
\pgfsetstrokecolor{currentstroke}%
\pgfsetdash{}{0pt}%
\pgfpathmoveto{\pgfqpoint{0.721249in}{0.352047in}}%
\pgfpathcurveto{\pgfqpoint{0.726774in}{0.352047in}}{\pgfqpoint{0.732073in}{0.354242in}}{\pgfqpoint{0.735980in}{0.358149in}}%
\pgfpathcurveto{\pgfqpoint{0.739887in}{0.362056in}}{\pgfqpoint{0.742082in}{0.367355in}}{\pgfqpoint{0.742082in}{0.372880in}}%
\pgfpathcurveto{\pgfqpoint{0.742082in}{0.378405in}}{\pgfqpoint{0.739887in}{0.383705in}}{\pgfqpoint{0.735980in}{0.387612in}}%
\pgfpathcurveto{\pgfqpoint{0.732073in}{0.391519in}}{\pgfqpoint{0.726774in}{0.393714in}}{\pgfqpoint{0.721249in}{0.393714in}}%
\pgfpathcurveto{\pgfqpoint{0.715724in}{0.393714in}}{\pgfqpoint{0.710424in}{0.391519in}}{\pgfqpoint{0.706518in}{0.387612in}}%
\pgfpathcurveto{\pgfqpoint{0.702611in}{0.383705in}}{\pgfqpoint{0.700416in}{0.378405in}}{\pgfqpoint{0.700416in}{0.372880in}}%
\pgfpathcurveto{\pgfqpoint{0.700416in}{0.367355in}}{\pgfqpoint{0.702611in}{0.362056in}}{\pgfqpoint{0.706518in}{0.358149in}}%
\pgfpathcurveto{\pgfqpoint{0.710424in}{0.354242in}}{\pgfqpoint{0.715724in}{0.352047in}}{\pgfqpoint{0.721249in}{0.352047in}}%
\pgfpathclose%
\pgfusepath{stroke,fill}%
\end{pgfscope}%
\begin{pgfscope}%
\pgfpathrectangle{\pgfqpoint{0.562500in}{0.275000in}}{\pgfqpoint{3.487500in}{1.925000in}}%
\pgfusepath{clip}%
\pgfsetbuttcap%
\pgfsetroundjoin%
\definecolor{currentfill}{rgb}{0.000000,0.000000,0.000000}%
\pgfsetfillcolor{currentfill}%
\pgfsetlinewidth{1.003750pt}%
\definecolor{currentstroke}{rgb}{0.000000,0.000000,0.000000}%
\pgfsetstrokecolor{currentstroke}%
\pgfsetdash{}{0pt}%
\pgfpathmoveto{\pgfqpoint{0.721249in}{0.352047in}}%
\pgfpathcurveto{\pgfqpoint{0.726774in}{0.352047in}}{\pgfqpoint{0.732073in}{0.354242in}}{\pgfqpoint{0.735980in}{0.358149in}}%
\pgfpathcurveto{\pgfqpoint{0.739887in}{0.362056in}}{\pgfqpoint{0.742082in}{0.367355in}}{\pgfqpoint{0.742082in}{0.372880in}}%
\pgfpathcurveto{\pgfqpoint{0.742082in}{0.378405in}}{\pgfqpoint{0.739887in}{0.383705in}}{\pgfqpoint{0.735980in}{0.387612in}}%
\pgfpathcurveto{\pgfqpoint{0.732073in}{0.391519in}}{\pgfqpoint{0.726774in}{0.393714in}}{\pgfqpoint{0.721249in}{0.393714in}}%
\pgfpathcurveto{\pgfqpoint{0.715724in}{0.393714in}}{\pgfqpoint{0.710424in}{0.391519in}}{\pgfqpoint{0.706518in}{0.387612in}}%
\pgfpathcurveto{\pgfqpoint{0.702611in}{0.383705in}}{\pgfqpoint{0.700416in}{0.378405in}}{\pgfqpoint{0.700416in}{0.372880in}}%
\pgfpathcurveto{\pgfqpoint{0.700416in}{0.367355in}}{\pgfqpoint{0.702611in}{0.362056in}}{\pgfqpoint{0.706518in}{0.358149in}}%
\pgfpathcurveto{\pgfqpoint{0.710424in}{0.354242in}}{\pgfqpoint{0.715724in}{0.352047in}}{\pgfqpoint{0.721249in}{0.352047in}}%
\pgfpathclose%
\pgfusepath{stroke,fill}%
\end{pgfscope}%
\begin{pgfscope}%
\pgfpathrectangle{\pgfqpoint{0.562500in}{0.275000in}}{\pgfqpoint{3.487500in}{1.925000in}}%
\pgfusepath{clip}%
\pgfsetbuttcap%
\pgfsetroundjoin%
\definecolor{currentfill}{rgb}{0.000000,0.000000,0.000000}%
\pgfsetfillcolor{currentfill}%
\pgfsetlinewidth{1.003750pt}%
\definecolor{currentstroke}{rgb}{0.000000,0.000000,0.000000}%
\pgfsetstrokecolor{currentstroke}%
\pgfsetdash{}{0pt}%
\pgfpathmoveto{\pgfqpoint{0.721249in}{0.357106in}}%
\pgfpathcurveto{\pgfqpoint{0.726774in}{0.357106in}}{\pgfqpoint{0.732073in}{0.359301in}}{\pgfqpoint{0.735980in}{0.363207in}}%
\pgfpathcurveto{\pgfqpoint{0.739887in}{0.367114in}}{\pgfqpoint{0.742082in}{0.372414in}}{\pgfqpoint{0.742082in}{0.377939in}}%
\pgfpathcurveto{\pgfqpoint{0.742082in}{0.383464in}}{\pgfqpoint{0.739887in}{0.388763in}}{\pgfqpoint{0.735980in}{0.392670in}}%
\pgfpathcurveto{\pgfqpoint{0.732073in}{0.396577in}}{\pgfqpoint{0.726774in}{0.398772in}}{\pgfqpoint{0.721249in}{0.398772in}}%
\pgfpathcurveto{\pgfqpoint{0.715724in}{0.398772in}}{\pgfqpoint{0.710424in}{0.396577in}}{\pgfqpoint{0.706518in}{0.392670in}}%
\pgfpathcurveto{\pgfqpoint{0.702611in}{0.388763in}}{\pgfqpoint{0.700416in}{0.383464in}}{\pgfqpoint{0.700416in}{0.377939in}}%
\pgfpathcurveto{\pgfqpoint{0.700416in}{0.372414in}}{\pgfqpoint{0.702611in}{0.367114in}}{\pgfqpoint{0.706518in}{0.363207in}}%
\pgfpathcurveto{\pgfqpoint{0.710424in}{0.359301in}}{\pgfqpoint{0.715724in}{0.357106in}}{\pgfqpoint{0.721249in}{0.357106in}}%
\pgfpathclose%
\pgfusepath{stroke,fill}%
\end{pgfscope}%
\begin{pgfscope}%
\pgfpathrectangle{\pgfqpoint{0.562500in}{0.275000in}}{\pgfqpoint{3.487500in}{1.925000in}}%
\pgfusepath{clip}%
\pgfsetbuttcap%
\pgfsetroundjoin%
\definecolor{currentfill}{rgb}{0.000000,0.000000,0.000000}%
\pgfsetfillcolor{currentfill}%
\pgfsetlinewidth{1.003750pt}%
\definecolor{currentstroke}{rgb}{0.000000,0.000000,0.000000}%
\pgfsetstrokecolor{currentstroke}%
\pgfsetdash{}{0pt}%
\pgfpathmoveto{\pgfqpoint{0.721249in}{0.357106in}}%
\pgfpathcurveto{\pgfqpoint{0.726774in}{0.357106in}}{\pgfqpoint{0.732073in}{0.359301in}}{\pgfqpoint{0.735980in}{0.363207in}}%
\pgfpathcurveto{\pgfqpoint{0.739887in}{0.367114in}}{\pgfqpoint{0.742082in}{0.372414in}}{\pgfqpoint{0.742082in}{0.377939in}}%
\pgfpathcurveto{\pgfqpoint{0.742082in}{0.383464in}}{\pgfqpoint{0.739887in}{0.388763in}}{\pgfqpoint{0.735980in}{0.392670in}}%
\pgfpathcurveto{\pgfqpoint{0.732073in}{0.396577in}}{\pgfqpoint{0.726774in}{0.398772in}}{\pgfqpoint{0.721249in}{0.398772in}}%
\pgfpathcurveto{\pgfqpoint{0.715724in}{0.398772in}}{\pgfqpoint{0.710424in}{0.396577in}}{\pgfqpoint{0.706518in}{0.392670in}}%
\pgfpathcurveto{\pgfqpoint{0.702611in}{0.388763in}}{\pgfqpoint{0.700416in}{0.383464in}}{\pgfqpoint{0.700416in}{0.377939in}}%
\pgfpathcurveto{\pgfqpoint{0.700416in}{0.372414in}}{\pgfqpoint{0.702611in}{0.367114in}}{\pgfqpoint{0.706518in}{0.363207in}}%
\pgfpathcurveto{\pgfqpoint{0.710424in}{0.359301in}}{\pgfqpoint{0.715724in}{0.357106in}}{\pgfqpoint{0.721249in}{0.357106in}}%
\pgfpathclose%
\pgfusepath{stroke,fill}%
\end{pgfscope}%
\begin{pgfscope}%
\pgfpathrectangle{\pgfqpoint{0.562500in}{0.275000in}}{\pgfqpoint{3.487500in}{1.925000in}}%
\pgfusepath{clip}%
\pgfsetbuttcap%
\pgfsetroundjoin%
\definecolor{currentfill}{rgb}{0.000000,0.000000,0.000000}%
\pgfsetfillcolor{currentfill}%
\pgfsetlinewidth{1.003750pt}%
\definecolor{currentstroke}{rgb}{0.000000,0.000000,0.000000}%
\pgfsetstrokecolor{currentstroke}%
\pgfsetdash{}{0pt}%
\pgfpathmoveto{\pgfqpoint{0.721249in}{0.367223in}}%
\pgfpathcurveto{\pgfqpoint{0.726774in}{0.367223in}}{\pgfqpoint{0.732073in}{0.369418in}}{\pgfqpoint{0.735980in}{0.373325in}}%
\pgfpathcurveto{\pgfqpoint{0.739887in}{0.377231in}}{\pgfqpoint{0.742082in}{0.382531in}}{\pgfqpoint{0.742082in}{0.388056in}}%
\pgfpathcurveto{\pgfqpoint{0.742082in}{0.393581in}}{\pgfqpoint{0.739887in}{0.398881in}}{\pgfqpoint{0.735980in}{0.402787in}}%
\pgfpathcurveto{\pgfqpoint{0.732073in}{0.406694in}}{\pgfqpoint{0.726774in}{0.408889in}}{\pgfqpoint{0.721249in}{0.408889in}}%
\pgfpathcurveto{\pgfqpoint{0.715724in}{0.408889in}}{\pgfqpoint{0.710424in}{0.406694in}}{\pgfqpoint{0.706518in}{0.402787in}}%
\pgfpathcurveto{\pgfqpoint{0.702611in}{0.398881in}}{\pgfqpoint{0.700416in}{0.393581in}}{\pgfqpoint{0.700416in}{0.388056in}}%
\pgfpathcurveto{\pgfqpoint{0.700416in}{0.382531in}}{\pgfqpoint{0.702611in}{0.377231in}}{\pgfqpoint{0.706518in}{0.373325in}}%
\pgfpathcurveto{\pgfqpoint{0.710424in}{0.369418in}}{\pgfqpoint{0.715724in}{0.367223in}}{\pgfqpoint{0.721249in}{0.367223in}}%
\pgfpathclose%
\pgfusepath{stroke,fill}%
\end{pgfscope}%
\begin{pgfscope}%
\pgfpathrectangle{\pgfqpoint{0.562500in}{0.275000in}}{\pgfqpoint{3.487500in}{1.925000in}}%
\pgfusepath{clip}%
\pgfsetbuttcap%
\pgfsetroundjoin%
\definecolor{currentfill}{rgb}{0.000000,0.000000,0.000000}%
\pgfsetfillcolor{currentfill}%
\pgfsetlinewidth{1.003750pt}%
\definecolor{currentstroke}{rgb}{0.000000,0.000000,0.000000}%
\pgfsetstrokecolor{currentstroke}%
\pgfsetdash{}{0pt}%
\pgfpathmoveto{\pgfqpoint{0.721249in}{0.367223in}}%
\pgfpathcurveto{\pgfqpoint{0.726774in}{0.367223in}}{\pgfqpoint{0.732073in}{0.369418in}}{\pgfqpoint{0.735980in}{0.373325in}}%
\pgfpathcurveto{\pgfqpoint{0.739887in}{0.377231in}}{\pgfqpoint{0.742082in}{0.382531in}}{\pgfqpoint{0.742082in}{0.388056in}}%
\pgfpathcurveto{\pgfqpoint{0.742082in}{0.393581in}}{\pgfqpoint{0.739887in}{0.398881in}}{\pgfqpoint{0.735980in}{0.402787in}}%
\pgfpathcurveto{\pgfqpoint{0.732073in}{0.406694in}}{\pgfqpoint{0.726774in}{0.408889in}}{\pgfqpoint{0.721249in}{0.408889in}}%
\pgfpathcurveto{\pgfqpoint{0.715724in}{0.408889in}}{\pgfqpoint{0.710424in}{0.406694in}}{\pgfqpoint{0.706518in}{0.402787in}}%
\pgfpathcurveto{\pgfqpoint{0.702611in}{0.398881in}}{\pgfqpoint{0.700416in}{0.393581in}}{\pgfqpoint{0.700416in}{0.388056in}}%
\pgfpathcurveto{\pgfqpoint{0.700416in}{0.382531in}}{\pgfqpoint{0.702611in}{0.377231in}}{\pgfqpoint{0.706518in}{0.373325in}}%
\pgfpathcurveto{\pgfqpoint{0.710424in}{0.369418in}}{\pgfqpoint{0.715724in}{0.367223in}}{\pgfqpoint{0.721249in}{0.367223in}}%
\pgfpathclose%
\pgfusepath{stroke,fill}%
\end{pgfscope}%
\begin{pgfscope}%
\pgfpathrectangle{\pgfqpoint{0.562500in}{0.275000in}}{\pgfqpoint{3.487500in}{1.925000in}}%
\pgfusepath{clip}%
\pgfsetbuttcap%
\pgfsetroundjoin%
\definecolor{currentfill}{rgb}{0.000000,0.000000,0.000000}%
\pgfsetfillcolor{currentfill}%
\pgfsetlinewidth{1.003750pt}%
\definecolor{currentstroke}{rgb}{0.000000,0.000000,0.000000}%
\pgfsetstrokecolor{currentstroke}%
\pgfsetdash{}{0pt}%
\pgfpathmoveto{\pgfqpoint{0.721249in}{0.367223in}}%
\pgfpathcurveto{\pgfqpoint{0.726774in}{0.367223in}}{\pgfqpoint{0.732073in}{0.369418in}}{\pgfqpoint{0.735980in}{0.373325in}}%
\pgfpathcurveto{\pgfqpoint{0.739887in}{0.377231in}}{\pgfqpoint{0.742082in}{0.382531in}}{\pgfqpoint{0.742082in}{0.388056in}}%
\pgfpathcurveto{\pgfqpoint{0.742082in}{0.393581in}}{\pgfqpoint{0.739887in}{0.398881in}}{\pgfqpoint{0.735980in}{0.402787in}}%
\pgfpathcurveto{\pgfqpoint{0.732073in}{0.406694in}}{\pgfqpoint{0.726774in}{0.408889in}}{\pgfqpoint{0.721249in}{0.408889in}}%
\pgfpathcurveto{\pgfqpoint{0.715724in}{0.408889in}}{\pgfqpoint{0.710424in}{0.406694in}}{\pgfqpoint{0.706518in}{0.402787in}}%
\pgfpathcurveto{\pgfqpoint{0.702611in}{0.398881in}}{\pgfqpoint{0.700416in}{0.393581in}}{\pgfqpoint{0.700416in}{0.388056in}}%
\pgfpathcurveto{\pgfqpoint{0.700416in}{0.382531in}}{\pgfqpoint{0.702611in}{0.377231in}}{\pgfqpoint{0.706518in}{0.373325in}}%
\pgfpathcurveto{\pgfqpoint{0.710424in}{0.369418in}}{\pgfqpoint{0.715724in}{0.367223in}}{\pgfqpoint{0.721249in}{0.367223in}}%
\pgfpathclose%
\pgfusepath{stroke,fill}%
\end{pgfscope}%
\begin{pgfscope}%
\pgfpathrectangle{\pgfqpoint{0.562500in}{0.275000in}}{\pgfqpoint{3.487500in}{1.925000in}}%
\pgfusepath{clip}%
\pgfsetbuttcap%
\pgfsetroundjoin%
\definecolor{currentfill}{rgb}{0.000000,0.000000,0.000000}%
\pgfsetfillcolor{currentfill}%
\pgfsetlinewidth{1.003750pt}%
\definecolor{currentstroke}{rgb}{0.000000,0.000000,0.000000}%
\pgfsetstrokecolor{currentstroke}%
\pgfsetdash{}{0pt}%
\pgfpathmoveto{\pgfqpoint{0.721249in}{0.357106in}}%
\pgfpathcurveto{\pgfqpoint{0.726774in}{0.357106in}}{\pgfqpoint{0.732073in}{0.359301in}}{\pgfqpoint{0.735980in}{0.363207in}}%
\pgfpathcurveto{\pgfqpoint{0.739887in}{0.367114in}}{\pgfqpoint{0.742082in}{0.372414in}}{\pgfqpoint{0.742082in}{0.377939in}}%
\pgfpathcurveto{\pgfqpoint{0.742082in}{0.383464in}}{\pgfqpoint{0.739887in}{0.388763in}}{\pgfqpoint{0.735980in}{0.392670in}}%
\pgfpathcurveto{\pgfqpoint{0.732073in}{0.396577in}}{\pgfqpoint{0.726774in}{0.398772in}}{\pgfqpoint{0.721249in}{0.398772in}}%
\pgfpathcurveto{\pgfqpoint{0.715724in}{0.398772in}}{\pgfqpoint{0.710424in}{0.396577in}}{\pgfqpoint{0.706518in}{0.392670in}}%
\pgfpathcurveto{\pgfqpoint{0.702611in}{0.388763in}}{\pgfqpoint{0.700416in}{0.383464in}}{\pgfqpoint{0.700416in}{0.377939in}}%
\pgfpathcurveto{\pgfqpoint{0.700416in}{0.372414in}}{\pgfqpoint{0.702611in}{0.367114in}}{\pgfqpoint{0.706518in}{0.363207in}}%
\pgfpathcurveto{\pgfqpoint{0.710424in}{0.359301in}}{\pgfqpoint{0.715724in}{0.357106in}}{\pgfqpoint{0.721249in}{0.357106in}}%
\pgfpathclose%
\pgfusepath{stroke,fill}%
\end{pgfscope}%
\begin{pgfscope}%
\pgfpathrectangle{\pgfqpoint{0.562500in}{0.275000in}}{\pgfqpoint{3.487500in}{1.925000in}}%
\pgfusepath{clip}%
\pgfsetbuttcap%
\pgfsetroundjoin%
\definecolor{currentfill}{rgb}{0.000000,0.000000,0.000000}%
\pgfsetfillcolor{currentfill}%
\pgfsetlinewidth{1.003750pt}%
\definecolor{currentstroke}{rgb}{0.000000,0.000000,0.000000}%
\pgfsetstrokecolor{currentstroke}%
\pgfsetdash{}{0pt}%
\pgfpathmoveto{\pgfqpoint{0.721249in}{0.357106in}}%
\pgfpathcurveto{\pgfqpoint{0.726774in}{0.357106in}}{\pgfqpoint{0.732073in}{0.359301in}}{\pgfqpoint{0.735980in}{0.363207in}}%
\pgfpathcurveto{\pgfqpoint{0.739887in}{0.367114in}}{\pgfqpoint{0.742082in}{0.372414in}}{\pgfqpoint{0.742082in}{0.377939in}}%
\pgfpathcurveto{\pgfqpoint{0.742082in}{0.383464in}}{\pgfqpoint{0.739887in}{0.388763in}}{\pgfqpoint{0.735980in}{0.392670in}}%
\pgfpathcurveto{\pgfqpoint{0.732073in}{0.396577in}}{\pgfqpoint{0.726774in}{0.398772in}}{\pgfqpoint{0.721249in}{0.398772in}}%
\pgfpathcurveto{\pgfqpoint{0.715724in}{0.398772in}}{\pgfqpoint{0.710424in}{0.396577in}}{\pgfqpoint{0.706518in}{0.392670in}}%
\pgfpathcurveto{\pgfqpoint{0.702611in}{0.388763in}}{\pgfqpoint{0.700416in}{0.383464in}}{\pgfqpoint{0.700416in}{0.377939in}}%
\pgfpathcurveto{\pgfqpoint{0.700416in}{0.372414in}}{\pgfqpoint{0.702611in}{0.367114in}}{\pgfqpoint{0.706518in}{0.363207in}}%
\pgfpathcurveto{\pgfqpoint{0.710424in}{0.359301in}}{\pgfqpoint{0.715724in}{0.357106in}}{\pgfqpoint{0.721249in}{0.357106in}}%
\pgfpathclose%
\pgfusepath{stroke,fill}%
\end{pgfscope}%
\begin{pgfscope}%
\pgfpathrectangle{\pgfqpoint{0.562500in}{0.275000in}}{\pgfqpoint{3.487500in}{1.925000in}}%
\pgfusepath{clip}%
\pgfsetbuttcap%
\pgfsetroundjoin%
\definecolor{currentfill}{rgb}{0.000000,0.000000,0.000000}%
\pgfsetfillcolor{currentfill}%
\pgfsetlinewidth{1.003750pt}%
\definecolor{currentstroke}{rgb}{0.000000,0.000000,0.000000}%
\pgfsetstrokecolor{currentstroke}%
\pgfsetdash{}{0pt}%
\pgfpathmoveto{\pgfqpoint{0.721249in}{0.362164in}}%
\pgfpathcurveto{\pgfqpoint{0.726774in}{0.362164in}}{\pgfqpoint{0.732073in}{0.364359in}}{\pgfqpoint{0.735980in}{0.368266in}}%
\pgfpathcurveto{\pgfqpoint{0.739887in}{0.372173in}}{\pgfqpoint{0.742082in}{0.377472in}}{\pgfqpoint{0.742082in}{0.382997in}}%
\pgfpathcurveto{\pgfqpoint{0.742082in}{0.388522in}}{\pgfqpoint{0.739887in}{0.393822in}}{\pgfqpoint{0.735980in}{0.397729in}}%
\pgfpathcurveto{\pgfqpoint{0.732073in}{0.401636in}}{\pgfqpoint{0.726774in}{0.403831in}}{\pgfqpoint{0.721249in}{0.403831in}}%
\pgfpathcurveto{\pgfqpoint{0.715724in}{0.403831in}}{\pgfqpoint{0.710424in}{0.401636in}}{\pgfqpoint{0.706518in}{0.397729in}}%
\pgfpathcurveto{\pgfqpoint{0.702611in}{0.393822in}}{\pgfqpoint{0.700416in}{0.388522in}}{\pgfqpoint{0.700416in}{0.382997in}}%
\pgfpathcurveto{\pgfqpoint{0.700416in}{0.377472in}}{\pgfqpoint{0.702611in}{0.372173in}}{\pgfqpoint{0.706518in}{0.368266in}}%
\pgfpathcurveto{\pgfqpoint{0.710424in}{0.364359in}}{\pgfqpoint{0.715724in}{0.362164in}}{\pgfqpoint{0.721249in}{0.362164in}}%
\pgfpathclose%
\pgfusepath{stroke,fill}%
\end{pgfscope}%
\begin{pgfscope}%
\pgfpathrectangle{\pgfqpoint{0.562500in}{0.275000in}}{\pgfqpoint{3.487500in}{1.925000in}}%
\pgfusepath{clip}%
\pgfsetbuttcap%
\pgfsetroundjoin%
\definecolor{currentfill}{rgb}{0.000000,0.000000,0.000000}%
\pgfsetfillcolor{currentfill}%
\pgfsetlinewidth{1.003750pt}%
\definecolor{currentstroke}{rgb}{0.000000,0.000000,0.000000}%
\pgfsetstrokecolor{currentstroke}%
\pgfsetdash{}{0pt}%
\pgfpathmoveto{\pgfqpoint{0.721249in}{0.357106in}}%
\pgfpathcurveto{\pgfqpoint{0.726774in}{0.357106in}}{\pgfqpoint{0.732073in}{0.359301in}}{\pgfqpoint{0.735980in}{0.363207in}}%
\pgfpathcurveto{\pgfqpoint{0.739887in}{0.367114in}}{\pgfqpoint{0.742082in}{0.372414in}}{\pgfqpoint{0.742082in}{0.377939in}}%
\pgfpathcurveto{\pgfqpoint{0.742082in}{0.383464in}}{\pgfqpoint{0.739887in}{0.388763in}}{\pgfqpoint{0.735980in}{0.392670in}}%
\pgfpathcurveto{\pgfqpoint{0.732073in}{0.396577in}}{\pgfqpoint{0.726774in}{0.398772in}}{\pgfqpoint{0.721249in}{0.398772in}}%
\pgfpathcurveto{\pgfqpoint{0.715724in}{0.398772in}}{\pgfqpoint{0.710424in}{0.396577in}}{\pgfqpoint{0.706518in}{0.392670in}}%
\pgfpathcurveto{\pgfqpoint{0.702611in}{0.388763in}}{\pgfqpoint{0.700416in}{0.383464in}}{\pgfqpoint{0.700416in}{0.377939in}}%
\pgfpathcurveto{\pgfqpoint{0.700416in}{0.372414in}}{\pgfqpoint{0.702611in}{0.367114in}}{\pgfqpoint{0.706518in}{0.363207in}}%
\pgfpathcurveto{\pgfqpoint{0.710424in}{0.359301in}}{\pgfqpoint{0.715724in}{0.357106in}}{\pgfqpoint{0.721249in}{0.357106in}}%
\pgfpathclose%
\pgfusepath{stroke,fill}%
\end{pgfscope}%
\begin{pgfscope}%
\pgfpathrectangle{\pgfqpoint{0.562500in}{0.275000in}}{\pgfqpoint{3.487500in}{1.925000in}}%
\pgfusepath{clip}%
\pgfsetbuttcap%
\pgfsetroundjoin%
\definecolor{currentfill}{rgb}{0.000000,0.000000,0.000000}%
\pgfsetfillcolor{currentfill}%
\pgfsetlinewidth{1.003750pt}%
\definecolor{currentstroke}{rgb}{0.000000,0.000000,0.000000}%
\pgfsetstrokecolor{currentstroke}%
\pgfsetdash{}{0pt}%
\pgfpathmoveto{\pgfqpoint{0.721249in}{0.357106in}}%
\pgfpathcurveto{\pgfqpoint{0.726774in}{0.357106in}}{\pgfqpoint{0.732073in}{0.359301in}}{\pgfqpoint{0.735980in}{0.363207in}}%
\pgfpathcurveto{\pgfqpoint{0.739887in}{0.367114in}}{\pgfqpoint{0.742082in}{0.372414in}}{\pgfqpoint{0.742082in}{0.377939in}}%
\pgfpathcurveto{\pgfqpoint{0.742082in}{0.383464in}}{\pgfqpoint{0.739887in}{0.388763in}}{\pgfqpoint{0.735980in}{0.392670in}}%
\pgfpathcurveto{\pgfqpoint{0.732073in}{0.396577in}}{\pgfqpoint{0.726774in}{0.398772in}}{\pgfqpoint{0.721249in}{0.398772in}}%
\pgfpathcurveto{\pgfqpoint{0.715724in}{0.398772in}}{\pgfqpoint{0.710424in}{0.396577in}}{\pgfqpoint{0.706518in}{0.392670in}}%
\pgfpathcurveto{\pgfqpoint{0.702611in}{0.388763in}}{\pgfqpoint{0.700416in}{0.383464in}}{\pgfqpoint{0.700416in}{0.377939in}}%
\pgfpathcurveto{\pgfqpoint{0.700416in}{0.372414in}}{\pgfqpoint{0.702611in}{0.367114in}}{\pgfqpoint{0.706518in}{0.363207in}}%
\pgfpathcurveto{\pgfqpoint{0.710424in}{0.359301in}}{\pgfqpoint{0.715724in}{0.357106in}}{\pgfqpoint{0.721249in}{0.357106in}}%
\pgfpathclose%
\pgfusepath{stroke,fill}%
\end{pgfscope}%
\begin{pgfscope}%
\pgfpathrectangle{\pgfqpoint{0.562500in}{0.275000in}}{\pgfqpoint{3.487500in}{1.925000in}}%
\pgfusepath{clip}%
\pgfsetbuttcap%
\pgfsetroundjoin%
\definecolor{currentfill}{rgb}{0.000000,0.000000,0.000000}%
\pgfsetfillcolor{currentfill}%
\pgfsetlinewidth{1.003750pt}%
\definecolor{currentstroke}{rgb}{0.000000,0.000000,0.000000}%
\pgfsetstrokecolor{currentstroke}%
\pgfsetdash{}{0pt}%
\pgfpathmoveto{\pgfqpoint{0.721249in}{0.362164in}}%
\pgfpathcurveto{\pgfqpoint{0.726774in}{0.362164in}}{\pgfqpoint{0.732073in}{0.364359in}}{\pgfqpoint{0.735980in}{0.368266in}}%
\pgfpathcurveto{\pgfqpoint{0.739887in}{0.372173in}}{\pgfqpoint{0.742082in}{0.377472in}}{\pgfqpoint{0.742082in}{0.382997in}}%
\pgfpathcurveto{\pgfqpoint{0.742082in}{0.388522in}}{\pgfqpoint{0.739887in}{0.393822in}}{\pgfqpoint{0.735980in}{0.397729in}}%
\pgfpathcurveto{\pgfqpoint{0.732073in}{0.401636in}}{\pgfqpoint{0.726774in}{0.403831in}}{\pgfqpoint{0.721249in}{0.403831in}}%
\pgfpathcurveto{\pgfqpoint{0.715724in}{0.403831in}}{\pgfqpoint{0.710424in}{0.401636in}}{\pgfqpoint{0.706518in}{0.397729in}}%
\pgfpathcurveto{\pgfqpoint{0.702611in}{0.393822in}}{\pgfqpoint{0.700416in}{0.388522in}}{\pgfqpoint{0.700416in}{0.382997in}}%
\pgfpathcurveto{\pgfqpoint{0.700416in}{0.377472in}}{\pgfqpoint{0.702611in}{0.372173in}}{\pgfqpoint{0.706518in}{0.368266in}}%
\pgfpathcurveto{\pgfqpoint{0.710424in}{0.364359in}}{\pgfqpoint{0.715724in}{0.362164in}}{\pgfqpoint{0.721249in}{0.362164in}}%
\pgfpathclose%
\pgfusepath{stroke,fill}%
\end{pgfscope}%
\begin{pgfscope}%
\pgfpathrectangle{\pgfqpoint{0.562500in}{0.275000in}}{\pgfqpoint{3.487500in}{1.925000in}}%
\pgfusepath{clip}%
\pgfsetbuttcap%
\pgfsetroundjoin%
\definecolor{currentfill}{rgb}{0.000000,0.000000,0.000000}%
\pgfsetfillcolor{currentfill}%
\pgfsetlinewidth{1.003750pt}%
\definecolor{currentstroke}{rgb}{0.000000,0.000000,0.000000}%
\pgfsetstrokecolor{currentstroke}%
\pgfsetdash{}{0pt}%
\pgfpathmoveto{\pgfqpoint{0.721249in}{0.367223in}}%
\pgfpathcurveto{\pgfqpoint{0.726774in}{0.367223in}}{\pgfqpoint{0.732073in}{0.369418in}}{\pgfqpoint{0.735980in}{0.373325in}}%
\pgfpathcurveto{\pgfqpoint{0.739887in}{0.377231in}}{\pgfqpoint{0.742082in}{0.382531in}}{\pgfqpoint{0.742082in}{0.388056in}}%
\pgfpathcurveto{\pgfqpoint{0.742082in}{0.393581in}}{\pgfqpoint{0.739887in}{0.398881in}}{\pgfqpoint{0.735980in}{0.402787in}}%
\pgfpathcurveto{\pgfqpoint{0.732073in}{0.406694in}}{\pgfqpoint{0.726774in}{0.408889in}}{\pgfqpoint{0.721249in}{0.408889in}}%
\pgfpathcurveto{\pgfqpoint{0.715724in}{0.408889in}}{\pgfqpoint{0.710424in}{0.406694in}}{\pgfqpoint{0.706518in}{0.402787in}}%
\pgfpathcurveto{\pgfqpoint{0.702611in}{0.398881in}}{\pgfqpoint{0.700416in}{0.393581in}}{\pgfqpoint{0.700416in}{0.388056in}}%
\pgfpathcurveto{\pgfqpoint{0.700416in}{0.382531in}}{\pgfqpoint{0.702611in}{0.377231in}}{\pgfqpoint{0.706518in}{0.373325in}}%
\pgfpathcurveto{\pgfqpoint{0.710424in}{0.369418in}}{\pgfqpoint{0.715724in}{0.367223in}}{\pgfqpoint{0.721249in}{0.367223in}}%
\pgfpathclose%
\pgfusepath{stroke,fill}%
\end{pgfscope}%
\begin{pgfscope}%
\pgfpathrectangle{\pgfqpoint{0.562500in}{0.275000in}}{\pgfqpoint{3.487500in}{1.925000in}}%
\pgfusepath{clip}%
\pgfsetbuttcap%
\pgfsetroundjoin%
\definecolor{currentfill}{rgb}{0.000000,0.000000,0.000000}%
\pgfsetfillcolor{currentfill}%
\pgfsetlinewidth{1.003750pt}%
\definecolor{currentstroke}{rgb}{0.000000,0.000000,0.000000}%
\pgfsetstrokecolor{currentstroke}%
\pgfsetdash{}{0pt}%
\pgfpathmoveto{\pgfqpoint{0.721249in}{0.352047in}}%
\pgfpathcurveto{\pgfqpoint{0.726774in}{0.352047in}}{\pgfqpoint{0.732073in}{0.354242in}}{\pgfqpoint{0.735980in}{0.358149in}}%
\pgfpathcurveto{\pgfqpoint{0.739887in}{0.362056in}}{\pgfqpoint{0.742082in}{0.367355in}}{\pgfqpoint{0.742082in}{0.372880in}}%
\pgfpathcurveto{\pgfqpoint{0.742082in}{0.378405in}}{\pgfqpoint{0.739887in}{0.383705in}}{\pgfqpoint{0.735980in}{0.387612in}}%
\pgfpathcurveto{\pgfqpoint{0.732073in}{0.391519in}}{\pgfqpoint{0.726774in}{0.393714in}}{\pgfqpoint{0.721249in}{0.393714in}}%
\pgfpathcurveto{\pgfqpoint{0.715724in}{0.393714in}}{\pgfqpoint{0.710424in}{0.391519in}}{\pgfqpoint{0.706518in}{0.387612in}}%
\pgfpathcurveto{\pgfqpoint{0.702611in}{0.383705in}}{\pgfqpoint{0.700416in}{0.378405in}}{\pgfqpoint{0.700416in}{0.372880in}}%
\pgfpathcurveto{\pgfqpoint{0.700416in}{0.367355in}}{\pgfqpoint{0.702611in}{0.362056in}}{\pgfqpoint{0.706518in}{0.358149in}}%
\pgfpathcurveto{\pgfqpoint{0.710424in}{0.354242in}}{\pgfqpoint{0.715724in}{0.352047in}}{\pgfqpoint{0.721249in}{0.352047in}}%
\pgfpathclose%
\pgfusepath{stroke,fill}%
\end{pgfscope}%
\begin{pgfscope}%
\pgfpathrectangle{\pgfqpoint{0.562500in}{0.275000in}}{\pgfqpoint{3.487500in}{1.925000in}}%
\pgfusepath{clip}%
\pgfsetbuttcap%
\pgfsetroundjoin%
\definecolor{currentfill}{rgb}{0.000000,0.000000,0.000000}%
\pgfsetfillcolor{currentfill}%
\pgfsetlinewidth{1.003750pt}%
\definecolor{currentstroke}{rgb}{0.000000,0.000000,0.000000}%
\pgfsetstrokecolor{currentstroke}%
\pgfsetdash{}{0pt}%
\pgfpathmoveto{\pgfqpoint{0.721249in}{0.367223in}}%
\pgfpathcurveto{\pgfqpoint{0.726774in}{0.367223in}}{\pgfqpoint{0.732073in}{0.369418in}}{\pgfqpoint{0.735980in}{0.373325in}}%
\pgfpathcurveto{\pgfqpoint{0.739887in}{0.377231in}}{\pgfqpoint{0.742082in}{0.382531in}}{\pgfqpoint{0.742082in}{0.388056in}}%
\pgfpathcurveto{\pgfqpoint{0.742082in}{0.393581in}}{\pgfqpoint{0.739887in}{0.398881in}}{\pgfqpoint{0.735980in}{0.402787in}}%
\pgfpathcurveto{\pgfqpoint{0.732073in}{0.406694in}}{\pgfqpoint{0.726774in}{0.408889in}}{\pgfqpoint{0.721249in}{0.408889in}}%
\pgfpathcurveto{\pgfqpoint{0.715724in}{0.408889in}}{\pgfqpoint{0.710424in}{0.406694in}}{\pgfqpoint{0.706518in}{0.402787in}}%
\pgfpathcurveto{\pgfqpoint{0.702611in}{0.398881in}}{\pgfqpoint{0.700416in}{0.393581in}}{\pgfqpoint{0.700416in}{0.388056in}}%
\pgfpathcurveto{\pgfqpoint{0.700416in}{0.382531in}}{\pgfqpoint{0.702611in}{0.377231in}}{\pgfqpoint{0.706518in}{0.373325in}}%
\pgfpathcurveto{\pgfqpoint{0.710424in}{0.369418in}}{\pgfqpoint{0.715724in}{0.367223in}}{\pgfqpoint{0.721249in}{0.367223in}}%
\pgfpathclose%
\pgfusepath{stroke,fill}%
\end{pgfscope}%
\begin{pgfscope}%
\pgfpathrectangle{\pgfqpoint{0.562500in}{0.275000in}}{\pgfqpoint{3.487500in}{1.925000in}}%
\pgfusepath{clip}%
\pgfsetbuttcap%
\pgfsetroundjoin%
\definecolor{currentfill}{rgb}{0.000000,0.000000,0.000000}%
\pgfsetfillcolor{currentfill}%
\pgfsetlinewidth{1.003750pt}%
\definecolor{currentstroke}{rgb}{0.000000,0.000000,0.000000}%
\pgfsetstrokecolor{currentstroke}%
\pgfsetdash{}{0pt}%
\pgfpathmoveto{\pgfqpoint{0.721249in}{0.357106in}}%
\pgfpathcurveto{\pgfqpoint{0.726774in}{0.357106in}}{\pgfqpoint{0.732073in}{0.359301in}}{\pgfqpoint{0.735980in}{0.363207in}}%
\pgfpathcurveto{\pgfqpoint{0.739887in}{0.367114in}}{\pgfqpoint{0.742082in}{0.372414in}}{\pgfqpoint{0.742082in}{0.377939in}}%
\pgfpathcurveto{\pgfqpoint{0.742082in}{0.383464in}}{\pgfqpoint{0.739887in}{0.388763in}}{\pgfqpoint{0.735980in}{0.392670in}}%
\pgfpathcurveto{\pgfqpoint{0.732073in}{0.396577in}}{\pgfqpoint{0.726774in}{0.398772in}}{\pgfqpoint{0.721249in}{0.398772in}}%
\pgfpathcurveto{\pgfqpoint{0.715724in}{0.398772in}}{\pgfqpoint{0.710424in}{0.396577in}}{\pgfqpoint{0.706518in}{0.392670in}}%
\pgfpathcurveto{\pgfqpoint{0.702611in}{0.388763in}}{\pgfqpoint{0.700416in}{0.383464in}}{\pgfqpoint{0.700416in}{0.377939in}}%
\pgfpathcurveto{\pgfqpoint{0.700416in}{0.372414in}}{\pgfqpoint{0.702611in}{0.367114in}}{\pgfqpoint{0.706518in}{0.363207in}}%
\pgfpathcurveto{\pgfqpoint{0.710424in}{0.359301in}}{\pgfqpoint{0.715724in}{0.357106in}}{\pgfqpoint{0.721249in}{0.357106in}}%
\pgfpathclose%
\pgfusepath{stroke,fill}%
\end{pgfscope}%
\begin{pgfscope}%
\pgfpathrectangle{\pgfqpoint{0.562500in}{0.275000in}}{\pgfqpoint{3.487500in}{1.925000in}}%
\pgfusepath{clip}%
\pgfsetbuttcap%
\pgfsetroundjoin%
\definecolor{currentfill}{rgb}{0.000000,0.000000,0.000000}%
\pgfsetfillcolor{currentfill}%
\pgfsetlinewidth{1.003750pt}%
\definecolor{currentstroke}{rgb}{0.000000,0.000000,0.000000}%
\pgfsetstrokecolor{currentstroke}%
\pgfsetdash{}{0pt}%
\pgfpathmoveto{\pgfqpoint{0.721249in}{0.341930in}}%
\pgfpathcurveto{\pgfqpoint{0.726774in}{0.341930in}}{\pgfqpoint{0.732073in}{0.344125in}}{\pgfqpoint{0.735980in}{0.348032in}}%
\pgfpathcurveto{\pgfqpoint{0.739887in}{0.351939in}}{\pgfqpoint{0.742082in}{0.357238in}}{\pgfqpoint{0.742082in}{0.362763in}}%
\pgfpathcurveto{\pgfqpoint{0.742082in}{0.368288in}}{\pgfqpoint{0.739887in}{0.373588in}}{\pgfqpoint{0.735980in}{0.377495in}}%
\pgfpathcurveto{\pgfqpoint{0.732073in}{0.381401in}}{\pgfqpoint{0.726774in}{0.383597in}}{\pgfqpoint{0.721249in}{0.383597in}}%
\pgfpathcurveto{\pgfqpoint{0.715724in}{0.383597in}}{\pgfqpoint{0.710424in}{0.381401in}}{\pgfqpoint{0.706518in}{0.377495in}}%
\pgfpathcurveto{\pgfqpoint{0.702611in}{0.373588in}}{\pgfqpoint{0.700416in}{0.368288in}}{\pgfqpoint{0.700416in}{0.362763in}}%
\pgfpathcurveto{\pgfqpoint{0.700416in}{0.357238in}}{\pgfqpoint{0.702611in}{0.351939in}}{\pgfqpoint{0.706518in}{0.348032in}}%
\pgfpathcurveto{\pgfqpoint{0.710424in}{0.344125in}}{\pgfqpoint{0.715724in}{0.341930in}}{\pgfqpoint{0.721249in}{0.341930in}}%
\pgfpathclose%
\pgfusepath{stroke,fill}%
\end{pgfscope}%
\begin{pgfscope}%
\pgfpathrectangle{\pgfqpoint{0.562500in}{0.275000in}}{\pgfqpoint{3.487500in}{1.925000in}}%
\pgfusepath{clip}%
\pgfsetbuttcap%
\pgfsetroundjoin%
\definecolor{currentfill}{rgb}{0.000000,0.000000,0.000000}%
\pgfsetfillcolor{currentfill}%
\pgfsetlinewidth{1.003750pt}%
\definecolor{currentstroke}{rgb}{0.000000,0.000000,0.000000}%
\pgfsetstrokecolor{currentstroke}%
\pgfsetdash{}{0pt}%
\pgfpathmoveto{\pgfqpoint{0.721249in}{0.372281in}}%
\pgfpathcurveto{\pgfqpoint{0.726774in}{0.372281in}}{\pgfqpoint{0.732073in}{0.374476in}}{\pgfqpoint{0.735980in}{0.378383in}}%
\pgfpathcurveto{\pgfqpoint{0.739887in}{0.382290in}}{\pgfqpoint{0.742082in}{0.387590in}}{\pgfqpoint{0.742082in}{0.393115in}}%
\pgfpathcurveto{\pgfqpoint{0.742082in}{0.398640in}}{\pgfqpoint{0.739887in}{0.403939in}}{\pgfqpoint{0.735980in}{0.407846in}}%
\pgfpathcurveto{\pgfqpoint{0.732073in}{0.411753in}}{\pgfqpoint{0.726774in}{0.413948in}}{\pgfqpoint{0.721249in}{0.413948in}}%
\pgfpathcurveto{\pgfqpoint{0.715724in}{0.413948in}}{\pgfqpoint{0.710424in}{0.411753in}}{\pgfqpoint{0.706518in}{0.407846in}}%
\pgfpathcurveto{\pgfqpoint{0.702611in}{0.403939in}}{\pgfqpoint{0.700416in}{0.398640in}}{\pgfqpoint{0.700416in}{0.393115in}}%
\pgfpathcurveto{\pgfqpoint{0.700416in}{0.387590in}}{\pgfqpoint{0.702611in}{0.382290in}}{\pgfqpoint{0.706518in}{0.378383in}}%
\pgfpathcurveto{\pgfqpoint{0.710424in}{0.374476in}}{\pgfqpoint{0.715724in}{0.372281in}}{\pgfqpoint{0.721249in}{0.372281in}}%
\pgfpathclose%
\pgfusepath{stroke,fill}%
\end{pgfscope}%
\begin{pgfscope}%
\pgfpathrectangle{\pgfqpoint{0.562500in}{0.275000in}}{\pgfqpoint{3.487500in}{1.925000in}}%
\pgfusepath{clip}%
\pgfsetbuttcap%
\pgfsetroundjoin%
\definecolor{currentfill}{rgb}{0.000000,0.000000,0.000000}%
\pgfsetfillcolor{currentfill}%
\pgfsetlinewidth{1.003750pt}%
\definecolor{currentstroke}{rgb}{0.000000,0.000000,0.000000}%
\pgfsetstrokecolor{currentstroke}%
\pgfsetdash{}{0pt}%
\pgfpathmoveto{\pgfqpoint{0.721249in}{0.367223in}}%
\pgfpathcurveto{\pgfqpoint{0.726774in}{0.367223in}}{\pgfqpoint{0.732073in}{0.369418in}}{\pgfqpoint{0.735980in}{0.373325in}}%
\pgfpathcurveto{\pgfqpoint{0.739887in}{0.377231in}}{\pgfqpoint{0.742082in}{0.382531in}}{\pgfqpoint{0.742082in}{0.388056in}}%
\pgfpathcurveto{\pgfqpoint{0.742082in}{0.393581in}}{\pgfqpoint{0.739887in}{0.398881in}}{\pgfqpoint{0.735980in}{0.402787in}}%
\pgfpathcurveto{\pgfqpoint{0.732073in}{0.406694in}}{\pgfqpoint{0.726774in}{0.408889in}}{\pgfqpoint{0.721249in}{0.408889in}}%
\pgfpathcurveto{\pgfqpoint{0.715724in}{0.408889in}}{\pgfqpoint{0.710424in}{0.406694in}}{\pgfqpoint{0.706518in}{0.402787in}}%
\pgfpathcurveto{\pgfqpoint{0.702611in}{0.398881in}}{\pgfqpoint{0.700416in}{0.393581in}}{\pgfqpoint{0.700416in}{0.388056in}}%
\pgfpathcurveto{\pgfqpoint{0.700416in}{0.382531in}}{\pgfqpoint{0.702611in}{0.377231in}}{\pgfqpoint{0.706518in}{0.373325in}}%
\pgfpathcurveto{\pgfqpoint{0.710424in}{0.369418in}}{\pgfqpoint{0.715724in}{0.367223in}}{\pgfqpoint{0.721249in}{0.367223in}}%
\pgfpathclose%
\pgfusepath{stroke,fill}%
\end{pgfscope}%
\begin{pgfscope}%
\pgfpathrectangle{\pgfqpoint{0.562500in}{0.275000in}}{\pgfqpoint{3.487500in}{1.925000in}}%
\pgfusepath{clip}%
\pgfsetbuttcap%
\pgfsetroundjoin%
\definecolor{currentfill}{rgb}{0.000000,0.000000,0.000000}%
\pgfsetfillcolor{currentfill}%
\pgfsetlinewidth{1.003750pt}%
\definecolor{currentstroke}{rgb}{0.000000,0.000000,0.000000}%
\pgfsetstrokecolor{currentstroke}%
\pgfsetdash{}{0pt}%
\pgfpathmoveto{\pgfqpoint{0.721249in}{0.362164in}}%
\pgfpathcurveto{\pgfqpoint{0.726774in}{0.362164in}}{\pgfqpoint{0.732073in}{0.364359in}}{\pgfqpoint{0.735980in}{0.368266in}}%
\pgfpathcurveto{\pgfqpoint{0.739887in}{0.372173in}}{\pgfqpoint{0.742082in}{0.377472in}}{\pgfqpoint{0.742082in}{0.382997in}}%
\pgfpathcurveto{\pgfqpoint{0.742082in}{0.388522in}}{\pgfqpoint{0.739887in}{0.393822in}}{\pgfqpoint{0.735980in}{0.397729in}}%
\pgfpathcurveto{\pgfqpoint{0.732073in}{0.401636in}}{\pgfqpoint{0.726774in}{0.403831in}}{\pgfqpoint{0.721249in}{0.403831in}}%
\pgfpathcurveto{\pgfqpoint{0.715724in}{0.403831in}}{\pgfqpoint{0.710424in}{0.401636in}}{\pgfqpoint{0.706518in}{0.397729in}}%
\pgfpathcurveto{\pgfqpoint{0.702611in}{0.393822in}}{\pgfqpoint{0.700416in}{0.388522in}}{\pgfqpoint{0.700416in}{0.382997in}}%
\pgfpathcurveto{\pgfqpoint{0.700416in}{0.377472in}}{\pgfqpoint{0.702611in}{0.372173in}}{\pgfqpoint{0.706518in}{0.368266in}}%
\pgfpathcurveto{\pgfqpoint{0.710424in}{0.364359in}}{\pgfqpoint{0.715724in}{0.362164in}}{\pgfqpoint{0.721249in}{0.362164in}}%
\pgfpathclose%
\pgfusepath{stroke,fill}%
\end{pgfscope}%
\begin{pgfscope}%
\pgfpathrectangle{\pgfqpoint{0.562500in}{0.275000in}}{\pgfqpoint{3.487500in}{1.925000in}}%
\pgfusepath{clip}%
\pgfsetbuttcap%
\pgfsetroundjoin%
\definecolor{currentfill}{rgb}{0.000000,0.000000,0.000000}%
\pgfsetfillcolor{currentfill}%
\pgfsetlinewidth{1.003750pt}%
\definecolor{currentstroke}{rgb}{0.000000,0.000000,0.000000}%
\pgfsetstrokecolor{currentstroke}%
\pgfsetdash{}{0pt}%
\pgfpathmoveto{\pgfqpoint{0.721249in}{0.357106in}}%
\pgfpathcurveto{\pgfqpoint{0.726774in}{0.357106in}}{\pgfqpoint{0.732073in}{0.359301in}}{\pgfqpoint{0.735980in}{0.363207in}}%
\pgfpathcurveto{\pgfqpoint{0.739887in}{0.367114in}}{\pgfqpoint{0.742082in}{0.372414in}}{\pgfqpoint{0.742082in}{0.377939in}}%
\pgfpathcurveto{\pgfqpoint{0.742082in}{0.383464in}}{\pgfqpoint{0.739887in}{0.388763in}}{\pgfqpoint{0.735980in}{0.392670in}}%
\pgfpathcurveto{\pgfqpoint{0.732073in}{0.396577in}}{\pgfqpoint{0.726774in}{0.398772in}}{\pgfqpoint{0.721249in}{0.398772in}}%
\pgfpathcurveto{\pgfqpoint{0.715724in}{0.398772in}}{\pgfqpoint{0.710424in}{0.396577in}}{\pgfqpoint{0.706518in}{0.392670in}}%
\pgfpathcurveto{\pgfqpoint{0.702611in}{0.388763in}}{\pgfqpoint{0.700416in}{0.383464in}}{\pgfqpoint{0.700416in}{0.377939in}}%
\pgfpathcurveto{\pgfqpoint{0.700416in}{0.372414in}}{\pgfqpoint{0.702611in}{0.367114in}}{\pgfqpoint{0.706518in}{0.363207in}}%
\pgfpathcurveto{\pgfqpoint{0.710424in}{0.359301in}}{\pgfqpoint{0.715724in}{0.357106in}}{\pgfqpoint{0.721249in}{0.357106in}}%
\pgfpathclose%
\pgfusepath{stroke,fill}%
\end{pgfscope}%
\begin{pgfscope}%
\pgfpathrectangle{\pgfqpoint{0.562500in}{0.275000in}}{\pgfqpoint{3.487500in}{1.925000in}}%
\pgfusepath{clip}%
\pgfsetbuttcap%
\pgfsetroundjoin%
\definecolor{currentfill}{rgb}{0.000000,0.000000,0.000000}%
\pgfsetfillcolor{currentfill}%
\pgfsetlinewidth{1.003750pt}%
\definecolor{currentstroke}{rgb}{0.000000,0.000000,0.000000}%
\pgfsetstrokecolor{currentstroke}%
\pgfsetdash{}{0pt}%
\pgfpathmoveto{\pgfqpoint{0.721249in}{0.362164in}}%
\pgfpathcurveto{\pgfqpoint{0.726774in}{0.362164in}}{\pgfqpoint{0.732073in}{0.364359in}}{\pgfqpoint{0.735980in}{0.368266in}}%
\pgfpathcurveto{\pgfqpoint{0.739887in}{0.372173in}}{\pgfqpoint{0.742082in}{0.377472in}}{\pgfqpoint{0.742082in}{0.382997in}}%
\pgfpathcurveto{\pgfqpoint{0.742082in}{0.388522in}}{\pgfqpoint{0.739887in}{0.393822in}}{\pgfqpoint{0.735980in}{0.397729in}}%
\pgfpathcurveto{\pgfqpoint{0.732073in}{0.401636in}}{\pgfqpoint{0.726774in}{0.403831in}}{\pgfqpoint{0.721249in}{0.403831in}}%
\pgfpathcurveto{\pgfqpoint{0.715724in}{0.403831in}}{\pgfqpoint{0.710424in}{0.401636in}}{\pgfqpoint{0.706518in}{0.397729in}}%
\pgfpathcurveto{\pgfqpoint{0.702611in}{0.393822in}}{\pgfqpoint{0.700416in}{0.388522in}}{\pgfqpoint{0.700416in}{0.382997in}}%
\pgfpathcurveto{\pgfqpoint{0.700416in}{0.377472in}}{\pgfqpoint{0.702611in}{0.372173in}}{\pgfqpoint{0.706518in}{0.368266in}}%
\pgfpathcurveto{\pgfqpoint{0.710424in}{0.364359in}}{\pgfqpoint{0.715724in}{0.362164in}}{\pgfqpoint{0.721249in}{0.362164in}}%
\pgfpathclose%
\pgfusepath{stroke,fill}%
\end{pgfscope}%
\begin{pgfscope}%
\pgfpathrectangle{\pgfqpoint{0.562500in}{0.275000in}}{\pgfqpoint{3.487500in}{1.925000in}}%
\pgfusepath{clip}%
\pgfsetbuttcap%
\pgfsetroundjoin%
\definecolor{currentfill}{rgb}{0.000000,0.000000,0.000000}%
\pgfsetfillcolor{currentfill}%
\pgfsetlinewidth{1.003750pt}%
\definecolor{currentstroke}{rgb}{0.000000,0.000000,0.000000}%
\pgfsetstrokecolor{currentstroke}%
\pgfsetdash{}{0pt}%
\pgfpathmoveto{\pgfqpoint{0.721249in}{0.346988in}}%
\pgfpathcurveto{\pgfqpoint{0.726774in}{0.346988in}}{\pgfqpoint{0.732073in}{0.349184in}}{\pgfqpoint{0.735980in}{0.353090in}}%
\pgfpathcurveto{\pgfqpoint{0.739887in}{0.356997in}}{\pgfqpoint{0.742082in}{0.362297in}}{\pgfqpoint{0.742082in}{0.367822in}}%
\pgfpathcurveto{\pgfqpoint{0.742082in}{0.373347in}}{\pgfqpoint{0.739887in}{0.378646in}}{\pgfqpoint{0.735980in}{0.382553in}}%
\pgfpathcurveto{\pgfqpoint{0.732073in}{0.386460in}}{\pgfqpoint{0.726774in}{0.388655in}}{\pgfqpoint{0.721249in}{0.388655in}}%
\pgfpathcurveto{\pgfqpoint{0.715724in}{0.388655in}}{\pgfqpoint{0.710424in}{0.386460in}}{\pgfqpoint{0.706518in}{0.382553in}}%
\pgfpathcurveto{\pgfqpoint{0.702611in}{0.378646in}}{\pgfqpoint{0.700416in}{0.373347in}}{\pgfqpoint{0.700416in}{0.367822in}}%
\pgfpathcurveto{\pgfqpoint{0.700416in}{0.362297in}}{\pgfqpoint{0.702611in}{0.356997in}}{\pgfqpoint{0.706518in}{0.353090in}}%
\pgfpathcurveto{\pgfqpoint{0.710424in}{0.349184in}}{\pgfqpoint{0.715724in}{0.346988in}}{\pgfqpoint{0.721249in}{0.346988in}}%
\pgfpathclose%
\pgfusepath{stroke,fill}%
\end{pgfscope}%
\begin{pgfscope}%
\pgfpathrectangle{\pgfqpoint{0.562500in}{0.275000in}}{\pgfqpoint{3.487500in}{1.925000in}}%
\pgfusepath{clip}%
\pgfsetbuttcap%
\pgfsetroundjoin%
\definecolor{currentfill}{rgb}{0.000000,0.000000,0.000000}%
\pgfsetfillcolor{currentfill}%
\pgfsetlinewidth{1.003750pt}%
\definecolor{currentstroke}{rgb}{0.000000,0.000000,0.000000}%
\pgfsetstrokecolor{currentstroke}%
\pgfsetdash{}{0pt}%
\pgfpathmoveto{\pgfqpoint{0.721249in}{0.362164in}}%
\pgfpathcurveto{\pgfqpoint{0.726774in}{0.362164in}}{\pgfqpoint{0.732073in}{0.364359in}}{\pgfqpoint{0.735980in}{0.368266in}}%
\pgfpathcurveto{\pgfqpoint{0.739887in}{0.372173in}}{\pgfqpoint{0.742082in}{0.377472in}}{\pgfqpoint{0.742082in}{0.382997in}}%
\pgfpathcurveto{\pgfqpoint{0.742082in}{0.388522in}}{\pgfqpoint{0.739887in}{0.393822in}}{\pgfqpoint{0.735980in}{0.397729in}}%
\pgfpathcurveto{\pgfqpoint{0.732073in}{0.401636in}}{\pgfqpoint{0.726774in}{0.403831in}}{\pgfqpoint{0.721249in}{0.403831in}}%
\pgfpathcurveto{\pgfqpoint{0.715724in}{0.403831in}}{\pgfqpoint{0.710424in}{0.401636in}}{\pgfqpoint{0.706518in}{0.397729in}}%
\pgfpathcurveto{\pgfqpoint{0.702611in}{0.393822in}}{\pgfqpoint{0.700416in}{0.388522in}}{\pgfqpoint{0.700416in}{0.382997in}}%
\pgfpathcurveto{\pgfqpoint{0.700416in}{0.377472in}}{\pgfqpoint{0.702611in}{0.372173in}}{\pgfqpoint{0.706518in}{0.368266in}}%
\pgfpathcurveto{\pgfqpoint{0.710424in}{0.364359in}}{\pgfqpoint{0.715724in}{0.362164in}}{\pgfqpoint{0.721249in}{0.362164in}}%
\pgfpathclose%
\pgfusepath{stroke,fill}%
\end{pgfscope}%
\begin{pgfscope}%
\pgfpathrectangle{\pgfqpoint{0.562500in}{0.275000in}}{\pgfqpoint{3.487500in}{1.925000in}}%
\pgfusepath{clip}%
\pgfsetbuttcap%
\pgfsetroundjoin%
\definecolor{currentfill}{rgb}{0.000000,0.000000,0.000000}%
\pgfsetfillcolor{currentfill}%
\pgfsetlinewidth{1.003750pt}%
\definecolor{currentstroke}{rgb}{0.000000,0.000000,0.000000}%
\pgfsetstrokecolor{currentstroke}%
\pgfsetdash{}{0pt}%
\pgfpathmoveto{\pgfqpoint{0.721249in}{0.367223in}}%
\pgfpathcurveto{\pgfqpoint{0.726774in}{0.367223in}}{\pgfqpoint{0.732073in}{0.369418in}}{\pgfqpoint{0.735980in}{0.373325in}}%
\pgfpathcurveto{\pgfqpoint{0.739887in}{0.377231in}}{\pgfqpoint{0.742082in}{0.382531in}}{\pgfqpoint{0.742082in}{0.388056in}}%
\pgfpathcurveto{\pgfqpoint{0.742082in}{0.393581in}}{\pgfqpoint{0.739887in}{0.398881in}}{\pgfqpoint{0.735980in}{0.402787in}}%
\pgfpathcurveto{\pgfqpoint{0.732073in}{0.406694in}}{\pgfqpoint{0.726774in}{0.408889in}}{\pgfqpoint{0.721249in}{0.408889in}}%
\pgfpathcurveto{\pgfqpoint{0.715724in}{0.408889in}}{\pgfqpoint{0.710424in}{0.406694in}}{\pgfqpoint{0.706518in}{0.402787in}}%
\pgfpathcurveto{\pgfqpoint{0.702611in}{0.398881in}}{\pgfqpoint{0.700416in}{0.393581in}}{\pgfqpoint{0.700416in}{0.388056in}}%
\pgfpathcurveto{\pgfqpoint{0.700416in}{0.382531in}}{\pgfqpoint{0.702611in}{0.377231in}}{\pgfqpoint{0.706518in}{0.373325in}}%
\pgfpathcurveto{\pgfqpoint{0.710424in}{0.369418in}}{\pgfqpoint{0.715724in}{0.367223in}}{\pgfqpoint{0.721249in}{0.367223in}}%
\pgfpathclose%
\pgfusepath{stroke,fill}%
\end{pgfscope}%
\begin{pgfscope}%
\pgfpathrectangle{\pgfqpoint{0.562500in}{0.275000in}}{\pgfqpoint{3.487500in}{1.925000in}}%
\pgfusepath{clip}%
\pgfsetbuttcap%
\pgfsetroundjoin%
\definecolor{currentfill}{rgb}{0.000000,0.000000,0.000000}%
\pgfsetfillcolor{currentfill}%
\pgfsetlinewidth{1.003750pt}%
\definecolor{currentstroke}{rgb}{0.000000,0.000000,0.000000}%
\pgfsetstrokecolor{currentstroke}%
\pgfsetdash{}{0pt}%
\pgfpathmoveto{\pgfqpoint{0.721249in}{0.367223in}}%
\pgfpathcurveto{\pgfqpoint{0.726774in}{0.367223in}}{\pgfqpoint{0.732073in}{0.369418in}}{\pgfqpoint{0.735980in}{0.373325in}}%
\pgfpathcurveto{\pgfqpoint{0.739887in}{0.377231in}}{\pgfqpoint{0.742082in}{0.382531in}}{\pgfqpoint{0.742082in}{0.388056in}}%
\pgfpathcurveto{\pgfqpoint{0.742082in}{0.393581in}}{\pgfqpoint{0.739887in}{0.398881in}}{\pgfqpoint{0.735980in}{0.402787in}}%
\pgfpathcurveto{\pgfqpoint{0.732073in}{0.406694in}}{\pgfqpoint{0.726774in}{0.408889in}}{\pgfqpoint{0.721249in}{0.408889in}}%
\pgfpathcurveto{\pgfqpoint{0.715724in}{0.408889in}}{\pgfqpoint{0.710424in}{0.406694in}}{\pgfqpoint{0.706518in}{0.402787in}}%
\pgfpathcurveto{\pgfqpoint{0.702611in}{0.398881in}}{\pgfqpoint{0.700416in}{0.393581in}}{\pgfqpoint{0.700416in}{0.388056in}}%
\pgfpathcurveto{\pgfqpoint{0.700416in}{0.382531in}}{\pgfqpoint{0.702611in}{0.377231in}}{\pgfqpoint{0.706518in}{0.373325in}}%
\pgfpathcurveto{\pgfqpoint{0.710424in}{0.369418in}}{\pgfqpoint{0.715724in}{0.367223in}}{\pgfqpoint{0.721249in}{0.367223in}}%
\pgfpathclose%
\pgfusepath{stroke,fill}%
\end{pgfscope}%
\begin{pgfscope}%
\pgfpathrectangle{\pgfqpoint{0.562500in}{0.275000in}}{\pgfqpoint{3.487500in}{1.925000in}}%
\pgfusepath{clip}%
\pgfsetbuttcap%
\pgfsetroundjoin%
\definecolor{currentfill}{rgb}{0.000000,0.000000,0.000000}%
\pgfsetfillcolor{currentfill}%
\pgfsetlinewidth{1.003750pt}%
\definecolor{currentstroke}{rgb}{0.000000,0.000000,0.000000}%
\pgfsetstrokecolor{currentstroke}%
\pgfsetdash{}{0pt}%
\pgfpathmoveto{\pgfqpoint{0.721249in}{0.357106in}}%
\pgfpathcurveto{\pgfqpoint{0.726774in}{0.357106in}}{\pgfqpoint{0.732073in}{0.359301in}}{\pgfqpoint{0.735980in}{0.363207in}}%
\pgfpathcurveto{\pgfqpoint{0.739887in}{0.367114in}}{\pgfqpoint{0.742082in}{0.372414in}}{\pgfqpoint{0.742082in}{0.377939in}}%
\pgfpathcurveto{\pgfqpoint{0.742082in}{0.383464in}}{\pgfqpoint{0.739887in}{0.388763in}}{\pgfqpoint{0.735980in}{0.392670in}}%
\pgfpathcurveto{\pgfqpoint{0.732073in}{0.396577in}}{\pgfqpoint{0.726774in}{0.398772in}}{\pgfqpoint{0.721249in}{0.398772in}}%
\pgfpathcurveto{\pgfqpoint{0.715724in}{0.398772in}}{\pgfqpoint{0.710424in}{0.396577in}}{\pgfqpoint{0.706518in}{0.392670in}}%
\pgfpathcurveto{\pgfqpoint{0.702611in}{0.388763in}}{\pgfqpoint{0.700416in}{0.383464in}}{\pgfqpoint{0.700416in}{0.377939in}}%
\pgfpathcurveto{\pgfqpoint{0.700416in}{0.372414in}}{\pgfqpoint{0.702611in}{0.367114in}}{\pgfqpoint{0.706518in}{0.363207in}}%
\pgfpathcurveto{\pgfqpoint{0.710424in}{0.359301in}}{\pgfqpoint{0.715724in}{0.357106in}}{\pgfqpoint{0.721249in}{0.357106in}}%
\pgfpathclose%
\pgfusepath{stroke,fill}%
\end{pgfscope}%
\begin{pgfscope}%
\pgfpathrectangle{\pgfqpoint{0.562500in}{0.275000in}}{\pgfqpoint{3.487500in}{1.925000in}}%
\pgfusepath{clip}%
\pgfsetbuttcap%
\pgfsetroundjoin%
\definecolor{currentfill}{rgb}{0.000000,0.000000,0.000000}%
\pgfsetfillcolor{currentfill}%
\pgfsetlinewidth{1.003750pt}%
\definecolor{currentstroke}{rgb}{0.000000,0.000000,0.000000}%
\pgfsetstrokecolor{currentstroke}%
\pgfsetdash{}{0pt}%
\pgfpathmoveto{\pgfqpoint{0.721249in}{0.367223in}}%
\pgfpathcurveto{\pgfqpoint{0.726774in}{0.367223in}}{\pgfqpoint{0.732073in}{0.369418in}}{\pgfqpoint{0.735980in}{0.373325in}}%
\pgfpathcurveto{\pgfqpoint{0.739887in}{0.377231in}}{\pgfqpoint{0.742082in}{0.382531in}}{\pgfqpoint{0.742082in}{0.388056in}}%
\pgfpathcurveto{\pgfqpoint{0.742082in}{0.393581in}}{\pgfqpoint{0.739887in}{0.398881in}}{\pgfqpoint{0.735980in}{0.402787in}}%
\pgfpathcurveto{\pgfqpoint{0.732073in}{0.406694in}}{\pgfqpoint{0.726774in}{0.408889in}}{\pgfqpoint{0.721249in}{0.408889in}}%
\pgfpathcurveto{\pgfqpoint{0.715724in}{0.408889in}}{\pgfqpoint{0.710424in}{0.406694in}}{\pgfqpoint{0.706518in}{0.402787in}}%
\pgfpathcurveto{\pgfqpoint{0.702611in}{0.398881in}}{\pgfqpoint{0.700416in}{0.393581in}}{\pgfqpoint{0.700416in}{0.388056in}}%
\pgfpathcurveto{\pgfqpoint{0.700416in}{0.382531in}}{\pgfqpoint{0.702611in}{0.377231in}}{\pgfqpoint{0.706518in}{0.373325in}}%
\pgfpathcurveto{\pgfqpoint{0.710424in}{0.369418in}}{\pgfqpoint{0.715724in}{0.367223in}}{\pgfqpoint{0.721249in}{0.367223in}}%
\pgfpathclose%
\pgfusepath{stroke,fill}%
\end{pgfscope}%
\begin{pgfscope}%
\pgfpathrectangle{\pgfqpoint{0.562500in}{0.275000in}}{\pgfqpoint{3.487500in}{1.925000in}}%
\pgfusepath{clip}%
\pgfsetbuttcap%
\pgfsetroundjoin%
\definecolor{currentfill}{rgb}{0.000000,0.000000,0.000000}%
\pgfsetfillcolor{currentfill}%
\pgfsetlinewidth{1.003750pt}%
\definecolor{currentstroke}{rgb}{0.000000,0.000000,0.000000}%
\pgfsetstrokecolor{currentstroke}%
\pgfsetdash{}{0pt}%
\pgfpathmoveto{\pgfqpoint{0.721249in}{0.352047in}}%
\pgfpathcurveto{\pgfqpoint{0.726774in}{0.352047in}}{\pgfqpoint{0.732073in}{0.354242in}}{\pgfqpoint{0.735980in}{0.358149in}}%
\pgfpathcurveto{\pgfqpoint{0.739887in}{0.362056in}}{\pgfqpoint{0.742082in}{0.367355in}}{\pgfqpoint{0.742082in}{0.372880in}}%
\pgfpathcurveto{\pgfqpoint{0.742082in}{0.378405in}}{\pgfqpoint{0.739887in}{0.383705in}}{\pgfqpoint{0.735980in}{0.387612in}}%
\pgfpathcurveto{\pgfqpoint{0.732073in}{0.391519in}}{\pgfqpoint{0.726774in}{0.393714in}}{\pgfqpoint{0.721249in}{0.393714in}}%
\pgfpathcurveto{\pgfqpoint{0.715724in}{0.393714in}}{\pgfqpoint{0.710424in}{0.391519in}}{\pgfqpoint{0.706518in}{0.387612in}}%
\pgfpathcurveto{\pgfqpoint{0.702611in}{0.383705in}}{\pgfqpoint{0.700416in}{0.378405in}}{\pgfqpoint{0.700416in}{0.372880in}}%
\pgfpathcurveto{\pgfqpoint{0.700416in}{0.367355in}}{\pgfqpoint{0.702611in}{0.362056in}}{\pgfqpoint{0.706518in}{0.358149in}}%
\pgfpathcurveto{\pgfqpoint{0.710424in}{0.354242in}}{\pgfqpoint{0.715724in}{0.352047in}}{\pgfqpoint{0.721249in}{0.352047in}}%
\pgfpathclose%
\pgfusepath{stroke,fill}%
\end{pgfscope}%
\begin{pgfscope}%
\pgfpathrectangle{\pgfqpoint{0.562500in}{0.275000in}}{\pgfqpoint{3.487500in}{1.925000in}}%
\pgfusepath{clip}%
\pgfsetbuttcap%
\pgfsetroundjoin%
\definecolor{currentfill}{rgb}{0.000000,0.000000,0.000000}%
\pgfsetfillcolor{currentfill}%
\pgfsetlinewidth{1.003750pt}%
\definecolor{currentstroke}{rgb}{0.000000,0.000000,0.000000}%
\pgfsetstrokecolor{currentstroke}%
\pgfsetdash{}{0pt}%
\pgfpathmoveto{\pgfqpoint{0.721249in}{0.352047in}}%
\pgfpathcurveto{\pgfqpoint{0.726774in}{0.352047in}}{\pgfqpoint{0.732073in}{0.354242in}}{\pgfqpoint{0.735980in}{0.358149in}}%
\pgfpathcurveto{\pgfqpoint{0.739887in}{0.362056in}}{\pgfqpoint{0.742082in}{0.367355in}}{\pgfqpoint{0.742082in}{0.372880in}}%
\pgfpathcurveto{\pgfqpoint{0.742082in}{0.378405in}}{\pgfqpoint{0.739887in}{0.383705in}}{\pgfqpoint{0.735980in}{0.387612in}}%
\pgfpathcurveto{\pgfqpoint{0.732073in}{0.391519in}}{\pgfqpoint{0.726774in}{0.393714in}}{\pgfqpoint{0.721249in}{0.393714in}}%
\pgfpathcurveto{\pgfqpoint{0.715724in}{0.393714in}}{\pgfqpoint{0.710424in}{0.391519in}}{\pgfqpoint{0.706518in}{0.387612in}}%
\pgfpathcurveto{\pgfqpoint{0.702611in}{0.383705in}}{\pgfqpoint{0.700416in}{0.378405in}}{\pgfqpoint{0.700416in}{0.372880in}}%
\pgfpathcurveto{\pgfqpoint{0.700416in}{0.367355in}}{\pgfqpoint{0.702611in}{0.362056in}}{\pgfqpoint{0.706518in}{0.358149in}}%
\pgfpathcurveto{\pgfqpoint{0.710424in}{0.354242in}}{\pgfqpoint{0.715724in}{0.352047in}}{\pgfqpoint{0.721249in}{0.352047in}}%
\pgfpathclose%
\pgfusepath{stroke,fill}%
\end{pgfscope}%
\begin{pgfscope}%
\pgfpathrectangle{\pgfqpoint{0.562500in}{0.275000in}}{\pgfqpoint{3.487500in}{1.925000in}}%
\pgfusepath{clip}%
\pgfsetbuttcap%
\pgfsetroundjoin%
\definecolor{currentfill}{rgb}{0.000000,0.000000,0.000000}%
\pgfsetfillcolor{currentfill}%
\pgfsetlinewidth{1.003750pt}%
\definecolor{currentstroke}{rgb}{0.000000,0.000000,0.000000}%
\pgfsetstrokecolor{currentstroke}%
\pgfsetdash{}{0pt}%
\pgfpathmoveto{\pgfqpoint{0.721249in}{0.357106in}}%
\pgfpathcurveto{\pgfqpoint{0.726774in}{0.357106in}}{\pgfqpoint{0.732073in}{0.359301in}}{\pgfqpoint{0.735980in}{0.363207in}}%
\pgfpathcurveto{\pgfqpoint{0.739887in}{0.367114in}}{\pgfqpoint{0.742082in}{0.372414in}}{\pgfqpoint{0.742082in}{0.377939in}}%
\pgfpathcurveto{\pgfqpoint{0.742082in}{0.383464in}}{\pgfqpoint{0.739887in}{0.388763in}}{\pgfqpoint{0.735980in}{0.392670in}}%
\pgfpathcurveto{\pgfqpoint{0.732073in}{0.396577in}}{\pgfqpoint{0.726774in}{0.398772in}}{\pgfqpoint{0.721249in}{0.398772in}}%
\pgfpathcurveto{\pgfqpoint{0.715724in}{0.398772in}}{\pgfqpoint{0.710424in}{0.396577in}}{\pgfqpoint{0.706518in}{0.392670in}}%
\pgfpathcurveto{\pgfqpoint{0.702611in}{0.388763in}}{\pgfqpoint{0.700416in}{0.383464in}}{\pgfqpoint{0.700416in}{0.377939in}}%
\pgfpathcurveto{\pgfqpoint{0.700416in}{0.372414in}}{\pgfqpoint{0.702611in}{0.367114in}}{\pgfqpoint{0.706518in}{0.363207in}}%
\pgfpathcurveto{\pgfqpoint{0.710424in}{0.359301in}}{\pgfqpoint{0.715724in}{0.357106in}}{\pgfqpoint{0.721249in}{0.357106in}}%
\pgfpathclose%
\pgfusepath{stroke,fill}%
\end{pgfscope}%
\begin{pgfscope}%
\pgfpathrectangle{\pgfqpoint{0.562500in}{0.275000in}}{\pgfqpoint{3.487500in}{1.925000in}}%
\pgfusepath{clip}%
\pgfsetbuttcap%
\pgfsetroundjoin%
\definecolor{currentfill}{rgb}{0.000000,0.000000,0.000000}%
\pgfsetfillcolor{currentfill}%
\pgfsetlinewidth{1.003750pt}%
\definecolor{currentstroke}{rgb}{0.000000,0.000000,0.000000}%
\pgfsetstrokecolor{currentstroke}%
\pgfsetdash{}{0pt}%
\pgfpathmoveto{\pgfqpoint{0.721249in}{0.377340in}}%
\pgfpathcurveto{\pgfqpoint{0.726774in}{0.377340in}}{\pgfqpoint{0.732073in}{0.379535in}}{\pgfqpoint{0.735980in}{0.383442in}}%
\pgfpathcurveto{\pgfqpoint{0.739887in}{0.387349in}}{\pgfqpoint{0.742082in}{0.392648in}}{\pgfqpoint{0.742082in}{0.398173in}}%
\pgfpathcurveto{\pgfqpoint{0.742082in}{0.403698in}}{\pgfqpoint{0.739887in}{0.408998in}}{\pgfqpoint{0.735980in}{0.412905in}}%
\pgfpathcurveto{\pgfqpoint{0.732073in}{0.416811in}}{\pgfqpoint{0.726774in}{0.419006in}}{\pgfqpoint{0.721249in}{0.419006in}}%
\pgfpathcurveto{\pgfqpoint{0.715724in}{0.419006in}}{\pgfqpoint{0.710424in}{0.416811in}}{\pgfqpoint{0.706518in}{0.412905in}}%
\pgfpathcurveto{\pgfqpoint{0.702611in}{0.408998in}}{\pgfqpoint{0.700416in}{0.403698in}}{\pgfqpoint{0.700416in}{0.398173in}}%
\pgfpathcurveto{\pgfqpoint{0.700416in}{0.392648in}}{\pgfqpoint{0.702611in}{0.387349in}}{\pgfqpoint{0.706518in}{0.383442in}}%
\pgfpathcurveto{\pgfqpoint{0.710424in}{0.379535in}}{\pgfqpoint{0.715724in}{0.377340in}}{\pgfqpoint{0.721249in}{0.377340in}}%
\pgfpathclose%
\pgfusepath{stroke,fill}%
\end{pgfscope}%
\begin{pgfscope}%
\pgfpathrectangle{\pgfqpoint{0.562500in}{0.275000in}}{\pgfqpoint{3.487500in}{1.925000in}}%
\pgfusepath{clip}%
\pgfsetbuttcap%
\pgfsetroundjoin%
\definecolor{currentfill}{rgb}{0.000000,0.000000,0.000000}%
\pgfsetfillcolor{currentfill}%
\pgfsetlinewidth{1.003750pt}%
\definecolor{currentstroke}{rgb}{0.000000,0.000000,0.000000}%
\pgfsetstrokecolor{currentstroke}%
\pgfsetdash{}{0pt}%
\pgfpathmoveto{\pgfqpoint{0.721249in}{0.362164in}}%
\pgfpathcurveto{\pgfqpoint{0.726774in}{0.362164in}}{\pgfqpoint{0.732073in}{0.364359in}}{\pgfqpoint{0.735980in}{0.368266in}}%
\pgfpathcurveto{\pgfqpoint{0.739887in}{0.372173in}}{\pgfqpoint{0.742082in}{0.377472in}}{\pgfqpoint{0.742082in}{0.382997in}}%
\pgfpathcurveto{\pgfqpoint{0.742082in}{0.388522in}}{\pgfqpoint{0.739887in}{0.393822in}}{\pgfqpoint{0.735980in}{0.397729in}}%
\pgfpathcurveto{\pgfqpoint{0.732073in}{0.401636in}}{\pgfqpoint{0.726774in}{0.403831in}}{\pgfqpoint{0.721249in}{0.403831in}}%
\pgfpathcurveto{\pgfqpoint{0.715724in}{0.403831in}}{\pgfqpoint{0.710424in}{0.401636in}}{\pgfqpoint{0.706518in}{0.397729in}}%
\pgfpathcurveto{\pgfqpoint{0.702611in}{0.393822in}}{\pgfqpoint{0.700416in}{0.388522in}}{\pgfqpoint{0.700416in}{0.382997in}}%
\pgfpathcurveto{\pgfqpoint{0.700416in}{0.377472in}}{\pgfqpoint{0.702611in}{0.372173in}}{\pgfqpoint{0.706518in}{0.368266in}}%
\pgfpathcurveto{\pgfqpoint{0.710424in}{0.364359in}}{\pgfqpoint{0.715724in}{0.362164in}}{\pgfqpoint{0.721249in}{0.362164in}}%
\pgfpathclose%
\pgfusepath{stroke,fill}%
\end{pgfscope}%
\begin{pgfscope}%
\pgfpathrectangle{\pgfqpoint{0.562500in}{0.275000in}}{\pgfqpoint{3.487500in}{1.925000in}}%
\pgfusepath{clip}%
\pgfsetbuttcap%
\pgfsetroundjoin%
\definecolor{currentfill}{rgb}{0.000000,0.000000,0.000000}%
\pgfsetfillcolor{currentfill}%
\pgfsetlinewidth{1.003750pt}%
\definecolor{currentstroke}{rgb}{0.000000,0.000000,0.000000}%
\pgfsetstrokecolor{currentstroke}%
\pgfsetdash{}{0pt}%
\pgfpathmoveto{\pgfqpoint{0.721249in}{0.346988in}}%
\pgfpathcurveto{\pgfqpoint{0.726774in}{0.346988in}}{\pgfqpoint{0.732073in}{0.349184in}}{\pgfqpoint{0.735980in}{0.353090in}}%
\pgfpathcurveto{\pgfqpoint{0.739887in}{0.356997in}}{\pgfqpoint{0.742082in}{0.362297in}}{\pgfqpoint{0.742082in}{0.367822in}}%
\pgfpathcurveto{\pgfqpoint{0.742082in}{0.373347in}}{\pgfqpoint{0.739887in}{0.378646in}}{\pgfqpoint{0.735980in}{0.382553in}}%
\pgfpathcurveto{\pgfqpoint{0.732073in}{0.386460in}}{\pgfqpoint{0.726774in}{0.388655in}}{\pgfqpoint{0.721249in}{0.388655in}}%
\pgfpathcurveto{\pgfqpoint{0.715724in}{0.388655in}}{\pgfqpoint{0.710424in}{0.386460in}}{\pgfqpoint{0.706518in}{0.382553in}}%
\pgfpathcurveto{\pgfqpoint{0.702611in}{0.378646in}}{\pgfqpoint{0.700416in}{0.373347in}}{\pgfqpoint{0.700416in}{0.367822in}}%
\pgfpathcurveto{\pgfqpoint{0.700416in}{0.362297in}}{\pgfqpoint{0.702611in}{0.356997in}}{\pgfqpoint{0.706518in}{0.353090in}}%
\pgfpathcurveto{\pgfqpoint{0.710424in}{0.349184in}}{\pgfqpoint{0.715724in}{0.346988in}}{\pgfqpoint{0.721249in}{0.346988in}}%
\pgfpathclose%
\pgfusepath{stroke,fill}%
\end{pgfscope}%
\begin{pgfscope}%
\pgfpathrectangle{\pgfqpoint{0.562500in}{0.275000in}}{\pgfqpoint{3.487500in}{1.925000in}}%
\pgfusepath{clip}%
\pgfsetbuttcap%
\pgfsetroundjoin%
\definecolor{currentfill}{rgb}{0.000000,0.000000,0.000000}%
\pgfsetfillcolor{currentfill}%
\pgfsetlinewidth{1.003750pt}%
\definecolor{currentstroke}{rgb}{0.000000,0.000000,0.000000}%
\pgfsetstrokecolor{currentstroke}%
\pgfsetdash{}{0pt}%
\pgfpathmoveto{\pgfqpoint{0.721249in}{0.357106in}}%
\pgfpathcurveto{\pgfqpoint{0.726774in}{0.357106in}}{\pgfqpoint{0.732073in}{0.359301in}}{\pgfqpoint{0.735980in}{0.363207in}}%
\pgfpathcurveto{\pgfqpoint{0.739887in}{0.367114in}}{\pgfqpoint{0.742082in}{0.372414in}}{\pgfqpoint{0.742082in}{0.377939in}}%
\pgfpathcurveto{\pgfqpoint{0.742082in}{0.383464in}}{\pgfqpoint{0.739887in}{0.388763in}}{\pgfqpoint{0.735980in}{0.392670in}}%
\pgfpathcurveto{\pgfqpoint{0.732073in}{0.396577in}}{\pgfqpoint{0.726774in}{0.398772in}}{\pgfqpoint{0.721249in}{0.398772in}}%
\pgfpathcurveto{\pgfqpoint{0.715724in}{0.398772in}}{\pgfqpoint{0.710424in}{0.396577in}}{\pgfqpoint{0.706518in}{0.392670in}}%
\pgfpathcurveto{\pgfqpoint{0.702611in}{0.388763in}}{\pgfqpoint{0.700416in}{0.383464in}}{\pgfqpoint{0.700416in}{0.377939in}}%
\pgfpathcurveto{\pgfqpoint{0.700416in}{0.372414in}}{\pgfqpoint{0.702611in}{0.367114in}}{\pgfqpoint{0.706518in}{0.363207in}}%
\pgfpathcurveto{\pgfqpoint{0.710424in}{0.359301in}}{\pgfqpoint{0.715724in}{0.357106in}}{\pgfqpoint{0.721249in}{0.357106in}}%
\pgfpathclose%
\pgfusepath{stroke,fill}%
\end{pgfscope}%
\begin{pgfscope}%
\pgfpathrectangle{\pgfqpoint{0.562500in}{0.275000in}}{\pgfqpoint{3.487500in}{1.925000in}}%
\pgfusepath{clip}%
\pgfsetbuttcap%
\pgfsetroundjoin%
\definecolor{currentfill}{rgb}{0.000000,0.000000,0.000000}%
\pgfsetfillcolor{currentfill}%
\pgfsetlinewidth{1.003750pt}%
\definecolor{currentstroke}{rgb}{0.000000,0.000000,0.000000}%
\pgfsetstrokecolor{currentstroke}%
\pgfsetdash{}{0pt}%
\pgfpathmoveto{\pgfqpoint{0.721249in}{0.372281in}}%
\pgfpathcurveto{\pgfqpoint{0.726774in}{0.372281in}}{\pgfqpoint{0.732073in}{0.374476in}}{\pgfqpoint{0.735980in}{0.378383in}}%
\pgfpathcurveto{\pgfqpoint{0.739887in}{0.382290in}}{\pgfqpoint{0.742082in}{0.387590in}}{\pgfqpoint{0.742082in}{0.393115in}}%
\pgfpathcurveto{\pgfqpoint{0.742082in}{0.398640in}}{\pgfqpoint{0.739887in}{0.403939in}}{\pgfqpoint{0.735980in}{0.407846in}}%
\pgfpathcurveto{\pgfqpoint{0.732073in}{0.411753in}}{\pgfqpoint{0.726774in}{0.413948in}}{\pgfqpoint{0.721249in}{0.413948in}}%
\pgfpathcurveto{\pgfqpoint{0.715724in}{0.413948in}}{\pgfqpoint{0.710424in}{0.411753in}}{\pgfqpoint{0.706518in}{0.407846in}}%
\pgfpathcurveto{\pgfqpoint{0.702611in}{0.403939in}}{\pgfqpoint{0.700416in}{0.398640in}}{\pgfqpoint{0.700416in}{0.393115in}}%
\pgfpathcurveto{\pgfqpoint{0.700416in}{0.387590in}}{\pgfqpoint{0.702611in}{0.382290in}}{\pgfqpoint{0.706518in}{0.378383in}}%
\pgfpathcurveto{\pgfqpoint{0.710424in}{0.374476in}}{\pgfqpoint{0.715724in}{0.372281in}}{\pgfqpoint{0.721249in}{0.372281in}}%
\pgfpathclose%
\pgfusepath{stroke,fill}%
\end{pgfscope}%
\begin{pgfscope}%
\pgfpathrectangle{\pgfqpoint{0.562500in}{0.275000in}}{\pgfqpoint{3.487500in}{1.925000in}}%
\pgfusepath{clip}%
\pgfsetbuttcap%
\pgfsetroundjoin%
\definecolor{currentfill}{rgb}{0.000000,0.000000,0.000000}%
\pgfsetfillcolor{currentfill}%
\pgfsetlinewidth{1.003750pt}%
\definecolor{currentstroke}{rgb}{0.000000,0.000000,0.000000}%
\pgfsetstrokecolor{currentstroke}%
\pgfsetdash{}{0pt}%
\pgfpathmoveto{\pgfqpoint{0.721249in}{0.357106in}}%
\pgfpathcurveto{\pgfqpoint{0.726774in}{0.357106in}}{\pgfqpoint{0.732073in}{0.359301in}}{\pgfqpoint{0.735980in}{0.363207in}}%
\pgfpathcurveto{\pgfqpoint{0.739887in}{0.367114in}}{\pgfqpoint{0.742082in}{0.372414in}}{\pgfqpoint{0.742082in}{0.377939in}}%
\pgfpathcurveto{\pgfqpoint{0.742082in}{0.383464in}}{\pgfqpoint{0.739887in}{0.388763in}}{\pgfqpoint{0.735980in}{0.392670in}}%
\pgfpathcurveto{\pgfqpoint{0.732073in}{0.396577in}}{\pgfqpoint{0.726774in}{0.398772in}}{\pgfqpoint{0.721249in}{0.398772in}}%
\pgfpathcurveto{\pgfqpoint{0.715724in}{0.398772in}}{\pgfqpoint{0.710424in}{0.396577in}}{\pgfqpoint{0.706518in}{0.392670in}}%
\pgfpathcurveto{\pgfqpoint{0.702611in}{0.388763in}}{\pgfqpoint{0.700416in}{0.383464in}}{\pgfqpoint{0.700416in}{0.377939in}}%
\pgfpathcurveto{\pgfqpoint{0.700416in}{0.372414in}}{\pgfqpoint{0.702611in}{0.367114in}}{\pgfqpoint{0.706518in}{0.363207in}}%
\pgfpathcurveto{\pgfqpoint{0.710424in}{0.359301in}}{\pgfqpoint{0.715724in}{0.357106in}}{\pgfqpoint{0.721249in}{0.357106in}}%
\pgfpathclose%
\pgfusepath{stroke,fill}%
\end{pgfscope}%
\begin{pgfscope}%
\pgfpathrectangle{\pgfqpoint{0.562500in}{0.275000in}}{\pgfqpoint{3.487500in}{1.925000in}}%
\pgfusepath{clip}%
\pgfsetbuttcap%
\pgfsetroundjoin%
\definecolor{currentfill}{rgb}{0.000000,0.000000,0.000000}%
\pgfsetfillcolor{currentfill}%
\pgfsetlinewidth{1.003750pt}%
\definecolor{currentstroke}{rgb}{0.000000,0.000000,0.000000}%
\pgfsetstrokecolor{currentstroke}%
\pgfsetdash{}{0pt}%
\pgfpathmoveto{\pgfqpoint{0.721249in}{0.372281in}}%
\pgfpathcurveto{\pgfqpoint{0.726774in}{0.372281in}}{\pgfqpoint{0.732073in}{0.374476in}}{\pgfqpoint{0.735980in}{0.378383in}}%
\pgfpathcurveto{\pgfqpoint{0.739887in}{0.382290in}}{\pgfqpoint{0.742082in}{0.387590in}}{\pgfqpoint{0.742082in}{0.393115in}}%
\pgfpathcurveto{\pgfqpoint{0.742082in}{0.398640in}}{\pgfqpoint{0.739887in}{0.403939in}}{\pgfqpoint{0.735980in}{0.407846in}}%
\pgfpathcurveto{\pgfqpoint{0.732073in}{0.411753in}}{\pgfqpoint{0.726774in}{0.413948in}}{\pgfqpoint{0.721249in}{0.413948in}}%
\pgfpathcurveto{\pgfqpoint{0.715724in}{0.413948in}}{\pgfqpoint{0.710424in}{0.411753in}}{\pgfqpoint{0.706518in}{0.407846in}}%
\pgfpathcurveto{\pgfqpoint{0.702611in}{0.403939in}}{\pgfqpoint{0.700416in}{0.398640in}}{\pgfqpoint{0.700416in}{0.393115in}}%
\pgfpathcurveto{\pgfqpoint{0.700416in}{0.387590in}}{\pgfqpoint{0.702611in}{0.382290in}}{\pgfqpoint{0.706518in}{0.378383in}}%
\pgfpathcurveto{\pgfqpoint{0.710424in}{0.374476in}}{\pgfqpoint{0.715724in}{0.372281in}}{\pgfqpoint{0.721249in}{0.372281in}}%
\pgfpathclose%
\pgfusepath{stroke,fill}%
\end{pgfscope}%
\begin{pgfscope}%
\pgfpathrectangle{\pgfqpoint{0.562500in}{0.275000in}}{\pgfqpoint{3.487500in}{1.925000in}}%
\pgfusepath{clip}%
\pgfsetbuttcap%
\pgfsetroundjoin%
\definecolor{currentfill}{rgb}{0.000000,0.000000,0.000000}%
\pgfsetfillcolor{currentfill}%
\pgfsetlinewidth{1.003750pt}%
\definecolor{currentstroke}{rgb}{0.000000,0.000000,0.000000}%
\pgfsetstrokecolor{currentstroke}%
\pgfsetdash{}{0pt}%
\pgfpathmoveto{\pgfqpoint{0.721249in}{0.367223in}}%
\pgfpathcurveto{\pgfqpoint{0.726774in}{0.367223in}}{\pgfqpoint{0.732073in}{0.369418in}}{\pgfqpoint{0.735980in}{0.373325in}}%
\pgfpathcurveto{\pgfqpoint{0.739887in}{0.377231in}}{\pgfqpoint{0.742082in}{0.382531in}}{\pgfqpoint{0.742082in}{0.388056in}}%
\pgfpathcurveto{\pgfqpoint{0.742082in}{0.393581in}}{\pgfqpoint{0.739887in}{0.398881in}}{\pgfqpoint{0.735980in}{0.402787in}}%
\pgfpathcurveto{\pgfqpoint{0.732073in}{0.406694in}}{\pgfqpoint{0.726774in}{0.408889in}}{\pgfqpoint{0.721249in}{0.408889in}}%
\pgfpathcurveto{\pgfqpoint{0.715724in}{0.408889in}}{\pgfqpoint{0.710424in}{0.406694in}}{\pgfqpoint{0.706518in}{0.402787in}}%
\pgfpathcurveto{\pgfqpoint{0.702611in}{0.398881in}}{\pgfqpoint{0.700416in}{0.393581in}}{\pgfqpoint{0.700416in}{0.388056in}}%
\pgfpathcurveto{\pgfqpoint{0.700416in}{0.382531in}}{\pgfqpoint{0.702611in}{0.377231in}}{\pgfqpoint{0.706518in}{0.373325in}}%
\pgfpathcurveto{\pgfqpoint{0.710424in}{0.369418in}}{\pgfqpoint{0.715724in}{0.367223in}}{\pgfqpoint{0.721249in}{0.367223in}}%
\pgfpathclose%
\pgfusepath{stroke,fill}%
\end{pgfscope}%
\begin{pgfscope}%
\pgfpathrectangle{\pgfqpoint{0.562500in}{0.275000in}}{\pgfqpoint{3.487500in}{1.925000in}}%
\pgfusepath{clip}%
\pgfsetbuttcap%
\pgfsetroundjoin%
\definecolor{currentfill}{rgb}{0.000000,0.000000,0.000000}%
\pgfsetfillcolor{currentfill}%
\pgfsetlinewidth{1.003750pt}%
\definecolor{currentstroke}{rgb}{0.000000,0.000000,0.000000}%
\pgfsetstrokecolor{currentstroke}%
\pgfsetdash{}{0pt}%
\pgfpathmoveto{\pgfqpoint{0.721249in}{0.357106in}}%
\pgfpathcurveto{\pgfqpoint{0.726774in}{0.357106in}}{\pgfqpoint{0.732073in}{0.359301in}}{\pgfqpoint{0.735980in}{0.363207in}}%
\pgfpathcurveto{\pgfqpoint{0.739887in}{0.367114in}}{\pgfqpoint{0.742082in}{0.372414in}}{\pgfqpoint{0.742082in}{0.377939in}}%
\pgfpathcurveto{\pgfqpoint{0.742082in}{0.383464in}}{\pgfqpoint{0.739887in}{0.388763in}}{\pgfqpoint{0.735980in}{0.392670in}}%
\pgfpathcurveto{\pgfqpoint{0.732073in}{0.396577in}}{\pgfqpoint{0.726774in}{0.398772in}}{\pgfqpoint{0.721249in}{0.398772in}}%
\pgfpathcurveto{\pgfqpoint{0.715724in}{0.398772in}}{\pgfqpoint{0.710424in}{0.396577in}}{\pgfqpoint{0.706518in}{0.392670in}}%
\pgfpathcurveto{\pgfqpoint{0.702611in}{0.388763in}}{\pgfqpoint{0.700416in}{0.383464in}}{\pgfqpoint{0.700416in}{0.377939in}}%
\pgfpathcurveto{\pgfqpoint{0.700416in}{0.372414in}}{\pgfqpoint{0.702611in}{0.367114in}}{\pgfqpoint{0.706518in}{0.363207in}}%
\pgfpathcurveto{\pgfqpoint{0.710424in}{0.359301in}}{\pgfqpoint{0.715724in}{0.357106in}}{\pgfqpoint{0.721249in}{0.357106in}}%
\pgfpathclose%
\pgfusepath{stroke,fill}%
\end{pgfscope}%
\begin{pgfscope}%
\pgfpathrectangle{\pgfqpoint{0.562500in}{0.275000in}}{\pgfqpoint{3.487500in}{1.925000in}}%
\pgfusepath{clip}%
\pgfsetbuttcap%
\pgfsetroundjoin%
\definecolor{currentfill}{rgb}{0.000000,0.000000,0.000000}%
\pgfsetfillcolor{currentfill}%
\pgfsetlinewidth{1.003750pt}%
\definecolor{currentstroke}{rgb}{0.000000,0.000000,0.000000}%
\pgfsetstrokecolor{currentstroke}%
\pgfsetdash{}{0pt}%
\pgfpathmoveto{\pgfqpoint{0.721249in}{0.357106in}}%
\pgfpathcurveto{\pgfqpoint{0.726774in}{0.357106in}}{\pgfqpoint{0.732073in}{0.359301in}}{\pgfqpoint{0.735980in}{0.363207in}}%
\pgfpathcurveto{\pgfqpoint{0.739887in}{0.367114in}}{\pgfqpoint{0.742082in}{0.372414in}}{\pgfqpoint{0.742082in}{0.377939in}}%
\pgfpathcurveto{\pgfqpoint{0.742082in}{0.383464in}}{\pgfqpoint{0.739887in}{0.388763in}}{\pgfqpoint{0.735980in}{0.392670in}}%
\pgfpathcurveto{\pgfqpoint{0.732073in}{0.396577in}}{\pgfqpoint{0.726774in}{0.398772in}}{\pgfqpoint{0.721249in}{0.398772in}}%
\pgfpathcurveto{\pgfqpoint{0.715724in}{0.398772in}}{\pgfqpoint{0.710424in}{0.396577in}}{\pgfqpoint{0.706518in}{0.392670in}}%
\pgfpathcurveto{\pgfqpoint{0.702611in}{0.388763in}}{\pgfqpoint{0.700416in}{0.383464in}}{\pgfqpoint{0.700416in}{0.377939in}}%
\pgfpathcurveto{\pgfqpoint{0.700416in}{0.372414in}}{\pgfqpoint{0.702611in}{0.367114in}}{\pgfqpoint{0.706518in}{0.363207in}}%
\pgfpathcurveto{\pgfqpoint{0.710424in}{0.359301in}}{\pgfqpoint{0.715724in}{0.357106in}}{\pgfqpoint{0.721249in}{0.357106in}}%
\pgfpathclose%
\pgfusepath{stroke,fill}%
\end{pgfscope}%
\begin{pgfscope}%
\pgfpathrectangle{\pgfqpoint{0.562500in}{0.275000in}}{\pgfqpoint{3.487500in}{1.925000in}}%
\pgfusepath{clip}%
\pgfsetbuttcap%
\pgfsetroundjoin%
\definecolor{currentfill}{rgb}{0.000000,0.000000,0.000000}%
\pgfsetfillcolor{currentfill}%
\pgfsetlinewidth{1.003750pt}%
\definecolor{currentstroke}{rgb}{0.000000,0.000000,0.000000}%
\pgfsetstrokecolor{currentstroke}%
\pgfsetdash{}{0pt}%
\pgfpathmoveto{\pgfqpoint{0.721249in}{0.367223in}}%
\pgfpathcurveto{\pgfqpoint{0.726774in}{0.367223in}}{\pgfqpoint{0.732073in}{0.369418in}}{\pgfqpoint{0.735980in}{0.373325in}}%
\pgfpathcurveto{\pgfqpoint{0.739887in}{0.377231in}}{\pgfqpoint{0.742082in}{0.382531in}}{\pgfqpoint{0.742082in}{0.388056in}}%
\pgfpathcurveto{\pgfqpoint{0.742082in}{0.393581in}}{\pgfqpoint{0.739887in}{0.398881in}}{\pgfqpoint{0.735980in}{0.402787in}}%
\pgfpathcurveto{\pgfqpoint{0.732073in}{0.406694in}}{\pgfqpoint{0.726774in}{0.408889in}}{\pgfqpoint{0.721249in}{0.408889in}}%
\pgfpathcurveto{\pgfqpoint{0.715724in}{0.408889in}}{\pgfqpoint{0.710424in}{0.406694in}}{\pgfqpoint{0.706518in}{0.402787in}}%
\pgfpathcurveto{\pgfqpoint{0.702611in}{0.398881in}}{\pgfqpoint{0.700416in}{0.393581in}}{\pgfqpoint{0.700416in}{0.388056in}}%
\pgfpathcurveto{\pgfqpoint{0.700416in}{0.382531in}}{\pgfqpoint{0.702611in}{0.377231in}}{\pgfqpoint{0.706518in}{0.373325in}}%
\pgfpathcurveto{\pgfqpoint{0.710424in}{0.369418in}}{\pgfqpoint{0.715724in}{0.367223in}}{\pgfqpoint{0.721249in}{0.367223in}}%
\pgfpathclose%
\pgfusepath{stroke,fill}%
\end{pgfscope}%
\begin{pgfscope}%
\pgfpathrectangle{\pgfqpoint{0.562500in}{0.275000in}}{\pgfqpoint{3.487500in}{1.925000in}}%
\pgfusepath{clip}%
\pgfsetbuttcap%
\pgfsetroundjoin%
\definecolor{currentfill}{rgb}{0.000000,0.000000,0.000000}%
\pgfsetfillcolor{currentfill}%
\pgfsetlinewidth{1.003750pt}%
\definecolor{currentstroke}{rgb}{0.000000,0.000000,0.000000}%
\pgfsetstrokecolor{currentstroke}%
\pgfsetdash{}{0pt}%
\pgfpathmoveto{\pgfqpoint{0.721249in}{0.357106in}}%
\pgfpathcurveto{\pgfqpoint{0.726774in}{0.357106in}}{\pgfqpoint{0.732073in}{0.359301in}}{\pgfqpoint{0.735980in}{0.363207in}}%
\pgfpathcurveto{\pgfqpoint{0.739887in}{0.367114in}}{\pgfqpoint{0.742082in}{0.372414in}}{\pgfqpoint{0.742082in}{0.377939in}}%
\pgfpathcurveto{\pgfqpoint{0.742082in}{0.383464in}}{\pgfqpoint{0.739887in}{0.388763in}}{\pgfqpoint{0.735980in}{0.392670in}}%
\pgfpathcurveto{\pgfqpoint{0.732073in}{0.396577in}}{\pgfqpoint{0.726774in}{0.398772in}}{\pgfqpoint{0.721249in}{0.398772in}}%
\pgfpathcurveto{\pgfqpoint{0.715724in}{0.398772in}}{\pgfqpoint{0.710424in}{0.396577in}}{\pgfqpoint{0.706518in}{0.392670in}}%
\pgfpathcurveto{\pgfqpoint{0.702611in}{0.388763in}}{\pgfqpoint{0.700416in}{0.383464in}}{\pgfqpoint{0.700416in}{0.377939in}}%
\pgfpathcurveto{\pgfqpoint{0.700416in}{0.372414in}}{\pgfqpoint{0.702611in}{0.367114in}}{\pgfqpoint{0.706518in}{0.363207in}}%
\pgfpathcurveto{\pgfqpoint{0.710424in}{0.359301in}}{\pgfqpoint{0.715724in}{0.357106in}}{\pgfqpoint{0.721249in}{0.357106in}}%
\pgfpathclose%
\pgfusepath{stroke,fill}%
\end{pgfscope}%
\begin{pgfscope}%
\pgfpathrectangle{\pgfqpoint{0.562500in}{0.275000in}}{\pgfqpoint{3.487500in}{1.925000in}}%
\pgfusepath{clip}%
\pgfsetbuttcap%
\pgfsetroundjoin%
\definecolor{currentfill}{rgb}{0.000000,0.000000,0.000000}%
\pgfsetfillcolor{currentfill}%
\pgfsetlinewidth{1.003750pt}%
\definecolor{currentstroke}{rgb}{0.000000,0.000000,0.000000}%
\pgfsetstrokecolor{currentstroke}%
\pgfsetdash{}{0pt}%
\pgfpathmoveto{\pgfqpoint{0.721249in}{0.362164in}}%
\pgfpathcurveto{\pgfqpoint{0.726774in}{0.362164in}}{\pgfqpoint{0.732073in}{0.364359in}}{\pgfqpoint{0.735980in}{0.368266in}}%
\pgfpathcurveto{\pgfqpoint{0.739887in}{0.372173in}}{\pgfqpoint{0.742082in}{0.377472in}}{\pgfqpoint{0.742082in}{0.382997in}}%
\pgfpathcurveto{\pgfqpoint{0.742082in}{0.388522in}}{\pgfqpoint{0.739887in}{0.393822in}}{\pgfqpoint{0.735980in}{0.397729in}}%
\pgfpathcurveto{\pgfqpoint{0.732073in}{0.401636in}}{\pgfqpoint{0.726774in}{0.403831in}}{\pgfqpoint{0.721249in}{0.403831in}}%
\pgfpathcurveto{\pgfqpoint{0.715724in}{0.403831in}}{\pgfqpoint{0.710424in}{0.401636in}}{\pgfqpoint{0.706518in}{0.397729in}}%
\pgfpathcurveto{\pgfqpoint{0.702611in}{0.393822in}}{\pgfqpoint{0.700416in}{0.388522in}}{\pgfqpoint{0.700416in}{0.382997in}}%
\pgfpathcurveto{\pgfqpoint{0.700416in}{0.377472in}}{\pgfqpoint{0.702611in}{0.372173in}}{\pgfqpoint{0.706518in}{0.368266in}}%
\pgfpathcurveto{\pgfqpoint{0.710424in}{0.364359in}}{\pgfqpoint{0.715724in}{0.362164in}}{\pgfqpoint{0.721249in}{0.362164in}}%
\pgfpathclose%
\pgfusepath{stroke,fill}%
\end{pgfscope}%
\begin{pgfscope}%
\pgfpathrectangle{\pgfqpoint{0.562500in}{0.275000in}}{\pgfqpoint{3.487500in}{1.925000in}}%
\pgfusepath{clip}%
\pgfsetbuttcap%
\pgfsetroundjoin%
\definecolor{currentfill}{rgb}{0.000000,0.000000,0.000000}%
\pgfsetfillcolor{currentfill}%
\pgfsetlinewidth{1.003750pt}%
\definecolor{currentstroke}{rgb}{0.000000,0.000000,0.000000}%
\pgfsetstrokecolor{currentstroke}%
\pgfsetdash{}{0pt}%
\pgfpathmoveto{\pgfqpoint{0.721249in}{0.362164in}}%
\pgfpathcurveto{\pgfqpoint{0.726774in}{0.362164in}}{\pgfqpoint{0.732073in}{0.364359in}}{\pgfqpoint{0.735980in}{0.368266in}}%
\pgfpathcurveto{\pgfqpoint{0.739887in}{0.372173in}}{\pgfqpoint{0.742082in}{0.377472in}}{\pgfqpoint{0.742082in}{0.382997in}}%
\pgfpathcurveto{\pgfqpoint{0.742082in}{0.388522in}}{\pgfqpoint{0.739887in}{0.393822in}}{\pgfqpoint{0.735980in}{0.397729in}}%
\pgfpathcurveto{\pgfqpoint{0.732073in}{0.401636in}}{\pgfqpoint{0.726774in}{0.403831in}}{\pgfqpoint{0.721249in}{0.403831in}}%
\pgfpathcurveto{\pgfqpoint{0.715724in}{0.403831in}}{\pgfqpoint{0.710424in}{0.401636in}}{\pgfqpoint{0.706518in}{0.397729in}}%
\pgfpathcurveto{\pgfqpoint{0.702611in}{0.393822in}}{\pgfqpoint{0.700416in}{0.388522in}}{\pgfqpoint{0.700416in}{0.382997in}}%
\pgfpathcurveto{\pgfqpoint{0.700416in}{0.377472in}}{\pgfqpoint{0.702611in}{0.372173in}}{\pgfqpoint{0.706518in}{0.368266in}}%
\pgfpathcurveto{\pgfqpoint{0.710424in}{0.364359in}}{\pgfqpoint{0.715724in}{0.362164in}}{\pgfqpoint{0.721249in}{0.362164in}}%
\pgfpathclose%
\pgfusepath{stroke,fill}%
\end{pgfscope}%
\begin{pgfscope}%
\pgfpathrectangle{\pgfqpoint{0.562500in}{0.275000in}}{\pgfqpoint{3.487500in}{1.925000in}}%
\pgfusepath{clip}%
\pgfsetbuttcap%
\pgfsetroundjoin%
\definecolor{currentfill}{rgb}{0.000000,0.000000,0.000000}%
\pgfsetfillcolor{currentfill}%
\pgfsetlinewidth{1.003750pt}%
\definecolor{currentstroke}{rgb}{0.000000,0.000000,0.000000}%
\pgfsetstrokecolor{currentstroke}%
\pgfsetdash{}{0pt}%
\pgfpathmoveto{\pgfqpoint{0.721249in}{0.357106in}}%
\pgfpathcurveto{\pgfqpoint{0.726774in}{0.357106in}}{\pgfqpoint{0.732073in}{0.359301in}}{\pgfqpoint{0.735980in}{0.363207in}}%
\pgfpathcurveto{\pgfqpoint{0.739887in}{0.367114in}}{\pgfqpoint{0.742082in}{0.372414in}}{\pgfqpoint{0.742082in}{0.377939in}}%
\pgfpathcurveto{\pgfqpoint{0.742082in}{0.383464in}}{\pgfqpoint{0.739887in}{0.388763in}}{\pgfqpoint{0.735980in}{0.392670in}}%
\pgfpathcurveto{\pgfqpoint{0.732073in}{0.396577in}}{\pgfqpoint{0.726774in}{0.398772in}}{\pgfqpoint{0.721249in}{0.398772in}}%
\pgfpathcurveto{\pgfqpoint{0.715724in}{0.398772in}}{\pgfqpoint{0.710424in}{0.396577in}}{\pgfqpoint{0.706518in}{0.392670in}}%
\pgfpathcurveto{\pgfqpoint{0.702611in}{0.388763in}}{\pgfqpoint{0.700416in}{0.383464in}}{\pgfqpoint{0.700416in}{0.377939in}}%
\pgfpathcurveto{\pgfqpoint{0.700416in}{0.372414in}}{\pgfqpoint{0.702611in}{0.367114in}}{\pgfqpoint{0.706518in}{0.363207in}}%
\pgfpathcurveto{\pgfqpoint{0.710424in}{0.359301in}}{\pgfqpoint{0.715724in}{0.357106in}}{\pgfqpoint{0.721249in}{0.357106in}}%
\pgfpathclose%
\pgfusepath{stroke,fill}%
\end{pgfscope}%
\begin{pgfscope}%
\pgfpathrectangle{\pgfqpoint{0.562500in}{0.275000in}}{\pgfqpoint{3.487500in}{1.925000in}}%
\pgfusepath{clip}%
\pgfsetbuttcap%
\pgfsetroundjoin%
\definecolor{currentfill}{rgb}{0.000000,0.000000,0.000000}%
\pgfsetfillcolor{currentfill}%
\pgfsetlinewidth{1.003750pt}%
\definecolor{currentstroke}{rgb}{0.000000,0.000000,0.000000}%
\pgfsetstrokecolor{currentstroke}%
\pgfsetdash{}{0pt}%
\pgfpathmoveto{\pgfqpoint{0.721249in}{0.362164in}}%
\pgfpathcurveto{\pgfqpoint{0.726774in}{0.362164in}}{\pgfqpoint{0.732073in}{0.364359in}}{\pgfqpoint{0.735980in}{0.368266in}}%
\pgfpathcurveto{\pgfqpoint{0.739887in}{0.372173in}}{\pgfqpoint{0.742082in}{0.377472in}}{\pgfqpoint{0.742082in}{0.382997in}}%
\pgfpathcurveto{\pgfqpoint{0.742082in}{0.388522in}}{\pgfqpoint{0.739887in}{0.393822in}}{\pgfqpoint{0.735980in}{0.397729in}}%
\pgfpathcurveto{\pgfqpoint{0.732073in}{0.401636in}}{\pgfqpoint{0.726774in}{0.403831in}}{\pgfqpoint{0.721249in}{0.403831in}}%
\pgfpathcurveto{\pgfqpoint{0.715724in}{0.403831in}}{\pgfqpoint{0.710424in}{0.401636in}}{\pgfqpoint{0.706518in}{0.397729in}}%
\pgfpathcurveto{\pgfqpoint{0.702611in}{0.393822in}}{\pgfqpoint{0.700416in}{0.388522in}}{\pgfqpoint{0.700416in}{0.382997in}}%
\pgfpathcurveto{\pgfqpoint{0.700416in}{0.377472in}}{\pgfqpoint{0.702611in}{0.372173in}}{\pgfqpoint{0.706518in}{0.368266in}}%
\pgfpathcurveto{\pgfqpoint{0.710424in}{0.364359in}}{\pgfqpoint{0.715724in}{0.362164in}}{\pgfqpoint{0.721249in}{0.362164in}}%
\pgfpathclose%
\pgfusepath{stroke,fill}%
\end{pgfscope}%
\begin{pgfscope}%
\pgfpathrectangle{\pgfqpoint{0.562500in}{0.275000in}}{\pgfqpoint{3.487500in}{1.925000in}}%
\pgfusepath{clip}%
\pgfsetbuttcap%
\pgfsetroundjoin%
\definecolor{currentfill}{rgb}{0.000000,0.000000,0.000000}%
\pgfsetfillcolor{currentfill}%
\pgfsetlinewidth{1.003750pt}%
\definecolor{currentstroke}{rgb}{0.000000,0.000000,0.000000}%
\pgfsetstrokecolor{currentstroke}%
\pgfsetdash{}{0pt}%
\pgfpathmoveto{\pgfqpoint{0.721249in}{0.372281in}}%
\pgfpathcurveto{\pgfqpoint{0.726774in}{0.372281in}}{\pgfqpoint{0.732073in}{0.374476in}}{\pgfqpoint{0.735980in}{0.378383in}}%
\pgfpathcurveto{\pgfqpoint{0.739887in}{0.382290in}}{\pgfqpoint{0.742082in}{0.387590in}}{\pgfqpoint{0.742082in}{0.393115in}}%
\pgfpathcurveto{\pgfqpoint{0.742082in}{0.398640in}}{\pgfqpoint{0.739887in}{0.403939in}}{\pgfqpoint{0.735980in}{0.407846in}}%
\pgfpathcurveto{\pgfqpoint{0.732073in}{0.411753in}}{\pgfqpoint{0.726774in}{0.413948in}}{\pgfqpoint{0.721249in}{0.413948in}}%
\pgfpathcurveto{\pgfqpoint{0.715724in}{0.413948in}}{\pgfqpoint{0.710424in}{0.411753in}}{\pgfqpoint{0.706518in}{0.407846in}}%
\pgfpathcurveto{\pgfqpoint{0.702611in}{0.403939in}}{\pgfqpoint{0.700416in}{0.398640in}}{\pgfqpoint{0.700416in}{0.393115in}}%
\pgfpathcurveto{\pgfqpoint{0.700416in}{0.387590in}}{\pgfqpoint{0.702611in}{0.382290in}}{\pgfqpoint{0.706518in}{0.378383in}}%
\pgfpathcurveto{\pgfqpoint{0.710424in}{0.374476in}}{\pgfqpoint{0.715724in}{0.372281in}}{\pgfqpoint{0.721249in}{0.372281in}}%
\pgfpathclose%
\pgfusepath{stroke,fill}%
\end{pgfscope}%
\begin{pgfscope}%
\pgfpathrectangle{\pgfqpoint{0.562500in}{0.275000in}}{\pgfqpoint{3.487500in}{1.925000in}}%
\pgfusepath{clip}%
\pgfsetbuttcap%
\pgfsetroundjoin%
\definecolor{currentfill}{rgb}{0.000000,0.000000,0.000000}%
\pgfsetfillcolor{currentfill}%
\pgfsetlinewidth{1.003750pt}%
\definecolor{currentstroke}{rgb}{0.000000,0.000000,0.000000}%
\pgfsetstrokecolor{currentstroke}%
\pgfsetdash{}{0pt}%
\pgfpathmoveto{\pgfqpoint{0.721249in}{0.357106in}}%
\pgfpathcurveto{\pgfqpoint{0.726774in}{0.357106in}}{\pgfqpoint{0.732073in}{0.359301in}}{\pgfqpoint{0.735980in}{0.363207in}}%
\pgfpathcurveto{\pgfqpoint{0.739887in}{0.367114in}}{\pgfqpoint{0.742082in}{0.372414in}}{\pgfqpoint{0.742082in}{0.377939in}}%
\pgfpathcurveto{\pgfqpoint{0.742082in}{0.383464in}}{\pgfqpoint{0.739887in}{0.388763in}}{\pgfqpoint{0.735980in}{0.392670in}}%
\pgfpathcurveto{\pgfqpoint{0.732073in}{0.396577in}}{\pgfqpoint{0.726774in}{0.398772in}}{\pgfqpoint{0.721249in}{0.398772in}}%
\pgfpathcurveto{\pgfqpoint{0.715724in}{0.398772in}}{\pgfqpoint{0.710424in}{0.396577in}}{\pgfqpoint{0.706518in}{0.392670in}}%
\pgfpathcurveto{\pgfqpoint{0.702611in}{0.388763in}}{\pgfqpoint{0.700416in}{0.383464in}}{\pgfqpoint{0.700416in}{0.377939in}}%
\pgfpathcurveto{\pgfqpoint{0.700416in}{0.372414in}}{\pgfqpoint{0.702611in}{0.367114in}}{\pgfqpoint{0.706518in}{0.363207in}}%
\pgfpathcurveto{\pgfqpoint{0.710424in}{0.359301in}}{\pgfqpoint{0.715724in}{0.357106in}}{\pgfqpoint{0.721249in}{0.357106in}}%
\pgfpathclose%
\pgfusepath{stroke,fill}%
\end{pgfscope}%
\begin{pgfscope}%
\pgfpathrectangle{\pgfqpoint{0.562500in}{0.275000in}}{\pgfqpoint{3.487500in}{1.925000in}}%
\pgfusepath{clip}%
\pgfsetbuttcap%
\pgfsetroundjoin%
\definecolor{currentfill}{rgb}{0.000000,0.000000,0.000000}%
\pgfsetfillcolor{currentfill}%
\pgfsetlinewidth{1.003750pt}%
\definecolor{currentstroke}{rgb}{0.000000,0.000000,0.000000}%
\pgfsetstrokecolor{currentstroke}%
\pgfsetdash{}{0pt}%
\pgfpathmoveto{\pgfqpoint{0.721249in}{0.346988in}}%
\pgfpathcurveto{\pgfqpoint{0.726774in}{0.346988in}}{\pgfqpoint{0.732073in}{0.349184in}}{\pgfqpoint{0.735980in}{0.353090in}}%
\pgfpathcurveto{\pgfqpoint{0.739887in}{0.356997in}}{\pgfqpoint{0.742082in}{0.362297in}}{\pgfqpoint{0.742082in}{0.367822in}}%
\pgfpathcurveto{\pgfqpoint{0.742082in}{0.373347in}}{\pgfqpoint{0.739887in}{0.378646in}}{\pgfqpoint{0.735980in}{0.382553in}}%
\pgfpathcurveto{\pgfqpoint{0.732073in}{0.386460in}}{\pgfqpoint{0.726774in}{0.388655in}}{\pgfqpoint{0.721249in}{0.388655in}}%
\pgfpathcurveto{\pgfqpoint{0.715724in}{0.388655in}}{\pgfqpoint{0.710424in}{0.386460in}}{\pgfqpoint{0.706518in}{0.382553in}}%
\pgfpathcurveto{\pgfqpoint{0.702611in}{0.378646in}}{\pgfqpoint{0.700416in}{0.373347in}}{\pgfqpoint{0.700416in}{0.367822in}}%
\pgfpathcurveto{\pgfqpoint{0.700416in}{0.362297in}}{\pgfqpoint{0.702611in}{0.356997in}}{\pgfqpoint{0.706518in}{0.353090in}}%
\pgfpathcurveto{\pgfqpoint{0.710424in}{0.349184in}}{\pgfqpoint{0.715724in}{0.346988in}}{\pgfqpoint{0.721249in}{0.346988in}}%
\pgfpathclose%
\pgfusepath{stroke,fill}%
\end{pgfscope}%
\begin{pgfscope}%
\pgfpathrectangle{\pgfqpoint{0.562500in}{0.275000in}}{\pgfqpoint{3.487500in}{1.925000in}}%
\pgfusepath{clip}%
\pgfsetbuttcap%
\pgfsetroundjoin%
\definecolor{currentfill}{rgb}{0.000000,0.000000,0.000000}%
\pgfsetfillcolor{currentfill}%
\pgfsetlinewidth{1.003750pt}%
\definecolor{currentstroke}{rgb}{0.000000,0.000000,0.000000}%
\pgfsetstrokecolor{currentstroke}%
\pgfsetdash{}{0pt}%
\pgfpathmoveto{\pgfqpoint{0.721249in}{0.357106in}}%
\pgfpathcurveto{\pgfqpoint{0.726774in}{0.357106in}}{\pgfqpoint{0.732073in}{0.359301in}}{\pgfqpoint{0.735980in}{0.363207in}}%
\pgfpathcurveto{\pgfqpoint{0.739887in}{0.367114in}}{\pgfqpoint{0.742082in}{0.372414in}}{\pgfqpoint{0.742082in}{0.377939in}}%
\pgfpathcurveto{\pgfqpoint{0.742082in}{0.383464in}}{\pgfqpoint{0.739887in}{0.388763in}}{\pgfqpoint{0.735980in}{0.392670in}}%
\pgfpathcurveto{\pgfqpoint{0.732073in}{0.396577in}}{\pgfqpoint{0.726774in}{0.398772in}}{\pgfqpoint{0.721249in}{0.398772in}}%
\pgfpathcurveto{\pgfqpoint{0.715724in}{0.398772in}}{\pgfqpoint{0.710424in}{0.396577in}}{\pgfqpoint{0.706518in}{0.392670in}}%
\pgfpathcurveto{\pgfqpoint{0.702611in}{0.388763in}}{\pgfqpoint{0.700416in}{0.383464in}}{\pgfqpoint{0.700416in}{0.377939in}}%
\pgfpathcurveto{\pgfqpoint{0.700416in}{0.372414in}}{\pgfqpoint{0.702611in}{0.367114in}}{\pgfqpoint{0.706518in}{0.363207in}}%
\pgfpathcurveto{\pgfqpoint{0.710424in}{0.359301in}}{\pgfqpoint{0.715724in}{0.357106in}}{\pgfqpoint{0.721249in}{0.357106in}}%
\pgfpathclose%
\pgfusepath{stroke,fill}%
\end{pgfscope}%
\begin{pgfscope}%
\pgfpathrectangle{\pgfqpoint{0.562500in}{0.275000in}}{\pgfqpoint{3.487500in}{1.925000in}}%
\pgfusepath{clip}%
\pgfsetbuttcap%
\pgfsetroundjoin%
\definecolor{currentfill}{rgb}{0.000000,0.000000,0.000000}%
\pgfsetfillcolor{currentfill}%
\pgfsetlinewidth{1.003750pt}%
\definecolor{currentstroke}{rgb}{0.000000,0.000000,0.000000}%
\pgfsetstrokecolor{currentstroke}%
\pgfsetdash{}{0pt}%
\pgfpathmoveto{\pgfqpoint{0.721249in}{0.367223in}}%
\pgfpathcurveto{\pgfqpoint{0.726774in}{0.367223in}}{\pgfqpoint{0.732073in}{0.369418in}}{\pgfqpoint{0.735980in}{0.373325in}}%
\pgfpathcurveto{\pgfqpoint{0.739887in}{0.377231in}}{\pgfqpoint{0.742082in}{0.382531in}}{\pgfqpoint{0.742082in}{0.388056in}}%
\pgfpathcurveto{\pgfqpoint{0.742082in}{0.393581in}}{\pgfqpoint{0.739887in}{0.398881in}}{\pgfqpoint{0.735980in}{0.402787in}}%
\pgfpathcurveto{\pgfqpoint{0.732073in}{0.406694in}}{\pgfqpoint{0.726774in}{0.408889in}}{\pgfqpoint{0.721249in}{0.408889in}}%
\pgfpathcurveto{\pgfqpoint{0.715724in}{0.408889in}}{\pgfqpoint{0.710424in}{0.406694in}}{\pgfqpoint{0.706518in}{0.402787in}}%
\pgfpathcurveto{\pgfqpoint{0.702611in}{0.398881in}}{\pgfqpoint{0.700416in}{0.393581in}}{\pgfqpoint{0.700416in}{0.388056in}}%
\pgfpathcurveto{\pgfqpoint{0.700416in}{0.382531in}}{\pgfqpoint{0.702611in}{0.377231in}}{\pgfqpoint{0.706518in}{0.373325in}}%
\pgfpathcurveto{\pgfqpoint{0.710424in}{0.369418in}}{\pgfqpoint{0.715724in}{0.367223in}}{\pgfqpoint{0.721249in}{0.367223in}}%
\pgfpathclose%
\pgfusepath{stroke,fill}%
\end{pgfscope}%
\begin{pgfscope}%
\pgfpathrectangle{\pgfqpoint{0.562500in}{0.275000in}}{\pgfqpoint{3.487500in}{1.925000in}}%
\pgfusepath{clip}%
\pgfsetbuttcap%
\pgfsetroundjoin%
\definecolor{currentfill}{rgb}{0.000000,0.000000,0.000000}%
\pgfsetfillcolor{currentfill}%
\pgfsetlinewidth{1.003750pt}%
\definecolor{currentstroke}{rgb}{0.000000,0.000000,0.000000}%
\pgfsetstrokecolor{currentstroke}%
\pgfsetdash{}{0pt}%
\pgfpathmoveto{\pgfqpoint{0.721249in}{0.362164in}}%
\pgfpathcurveto{\pgfqpoint{0.726774in}{0.362164in}}{\pgfqpoint{0.732073in}{0.364359in}}{\pgfqpoint{0.735980in}{0.368266in}}%
\pgfpathcurveto{\pgfqpoint{0.739887in}{0.372173in}}{\pgfqpoint{0.742082in}{0.377472in}}{\pgfqpoint{0.742082in}{0.382997in}}%
\pgfpathcurveto{\pgfqpoint{0.742082in}{0.388522in}}{\pgfqpoint{0.739887in}{0.393822in}}{\pgfqpoint{0.735980in}{0.397729in}}%
\pgfpathcurveto{\pgfqpoint{0.732073in}{0.401636in}}{\pgfqpoint{0.726774in}{0.403831in}}{\pgfqpoint{0.721249in}{0.403831in}}%
\pgfpathcurveto{\pgfqpoint{0.715724in}{0.403831in}}{\pgfqpoint{0.710424in}{0.401636in}}{\pgfqpoint{0.706518in}{0.397729in}}%
\pgfpathcurveto{\pgfqpoint{0.702611in}{0.393822in}}{\pgfqpoint{0.700416in}{0.388522in}}{\pgfqpoint{0.700416in}{0.382997in}}%
\pgfpathcurveto{\pgfqpoint{0.700416in}{0.377472in}}{\pgfqpoint{0.702611in}{0.372173in}}{\pgfqpoint{0.706518in}{0.368266in}}%
\pgfpathcurveto{\pgfqpoint{0.710424in}{0.364359in}}{\pgfqpoint{0.715724in}{0.362164in}}{\pgfqpoint{0.721249in}{0.362164in}}%
\pgfpathclose%
\pgfusepath{stroke,fill}%
\end{pgfscope}%
\begin{pgfscope}%
\pgfpathrectangle{\pgfqpoint{0.562500in}{0.275000in}}{\pgfqpoint{3.487500in}{1.925000in}}%
\pgfusepath{clip}%
\pgfsetbuttcap%
\pgfsetroundjoin%
\definecolor{currentfill}{rgb}{0.000000,0.000000,0.000000}%
\pgfsetfillcolor{currentfill}%
\pgfsetlinewidth{1.003750pt}%
\definecolor{currentstroke}{rgb}{0.000000,0.000000,0.000000}%
\pgfsetstrokecolor{currentstroke}%
\pgfsetdash{}{0pt}%
\pgfpathmoveto{\pgfqpoint{0.721249in}{0.367223in}}%
\pgfpathcurveto{\pgfqpoint{0.726774in}{0.367223in}}{\pgfqpoint{0.732073in}{0.369418in}}{\pgfqpoint{0.735980in}{0.373325in}}%
\pgfpathcurveto{\pgfqpoint{0.739887in}{0.377231in}}{\pgfqpoint{0.742082in}{0.382531in}}{\pgfqpoint{0.742082in}{0.388056in}}%
\pgfpathcurveto{\pgfqpoint{0.742082in}{0.393581in}}{\pgfqpoint{0.739887in}{0.398881in}}{\pgfqpoint{0.735980in}{0.402787in}}%
\pgfpathcurveto{\pgfqpoint{0.732073in}{0.406694in}}{\pgfqpoint{0.726774in}{0.408889in}}{\pgfqpoint{0.721249in}{0.408889in}}%
\pgfpathcurveto{\pgfqpoint{0.715724in}{0.408889in}}{\pgfqpoint{0.710424in}{0.406694in}}{\pgfqpoint{0.706518in}{0.402787in}}%
\pgfpathcurveto{\pgfqpoint{0.702611in}{0.398881in}}{\pgfqpoint{0.700416in}{0.393581in}}{\pgfqpoint{0.700416in}{0.388056in}}%
\pgfpathcurveto{\pgfqpoint{0.700416in}{0.382531in}}{\pgfqpoint{0.702611in}{0.377231in}}{\pgfqpoint{0.706518in}{0.373325in}}%
\pgfpathcurveto{\pgfqpoint{0.710424in}{0.369418in}}{\pgfqpoint{0.715724in}{0.367223in}}{\pgfqpoint{0.721249in}{0.367223in}}%
\pgfpathclose%
\pgfusepath{stroke,fill}%
\end{pgfscope}%
\begin{pgfscope}%
\pgfpathrectangle{\pgfqpoint{0.562500in}{0.275000in}}{\pgfqpoint{3.487500in}{1.925000in}}%
\pgfusepath{clip}%
\pgfsetbuttcap%
\pgfsetroundjoin%
\definecolor{currentfill}{rgb}{0.000000,0.000000,0.000000}%
\pgfsetfillcolor{currentfill}%
\pgfsetlinewidth{1.003750pt}%
\definecolor{currentstroke}{rgb}{0.000000,0.000000,0.000000}%
\pgfsetstrokecolor{currentstroke}%
\pgfsetdash{}{0pt}%
\pgfpathmoveto{\pgfqpoint{0.721249in}{0.352047in}}%
\pgfpathcurveto{\pgfqpoint{0.726774in}{0.352047in}}{\pgfqpoint{0.732073in}{0.354242in}}{\pgfqpoint{0.735980in}{0.358149in}}%
\pgfpathcurveto{\pgfqpoint{0.739887in}{0.362056in}}{\pgfqpoint{0.742082in}{0.367355in}}{\pgfqpoint{0.742082in}{0.372880in}}%
\pgfpathcurveto{\pgfqpoint{0.742082in}{0.378405in}}{\pgfqpoint{0.739887in}{0.383705in}}{\pgfqpoint{0.735980in}{0.387612in}}%
\pgfpathcurveto{\pgfqpoint{0.732073in}{0.391519in}}{\pgfqpoint{0.726774in}{0.393714in}}{\pgfqpoint{0.721249in}{0.393714in}}%
\pgfpathcurveto{\pgfqpoint{0.715724in}{0.393714in}}{\pgfqpoint{0.710424in}{0.391519in}}{\pgfqpoint{0.706518in}{0.387612in}}%
\pgfpathcurveto{\pgfqpoint{0.702611in}{0.383705in}}{\pgfqpoint{0.700416in}{0.378405in}}{\pgfqpoint{0.700416in}{0.372880in}}%
\pgfpathcurveto{\pgfqpoint{0.700416in}{0.367355in}}{\pgfqpoint{0.702611in}{0.362056in}}{\pgfqpoint{0.706518in}{0.358149in}}%
\pgfpathcurveto{\pgfqpoint{0.710424in}{0.354242in}}{\pgfqpoint{0.715724in}{0.352047in}}{\pgfqpoint{0.721249in}{0.352047in}}%
\pgfpathclose%
\pgfusepath{stroke,fill}%
\end{pgfscope}%
\begin{pgfscope}%
\pgfpathrectangle{\pgfqpoint{0.562500in}{0.275000in}}{\pgfqpoint{3.487500in}{1.925000in}}%
\pgfusepath{clip}%
\pgfsetbuttcap%
\pgfsetroundjoin%
\definecolor{currentfill}{rgb}{0.000000,0.000000,0.000000}%
\pgfsetfillcolor{currentfill}%
\pgfsetlinewidth{1.003750pt}%
\definecolor{currentstroke}{rgb}{0.000000,0.000000,0.000000}%
\pgfsetstrokecolor{currentstroke}%
\pgfsetdash{}{0pt}%
\pgfpathmoveto{\pgfqpoint{0.721249in}{0.362164in}}%
\pgfpathcurveto{\pgfqpoint{0.726774in}{0.362164in}}{\pgfqpoint{0.732073in}{0.364359in}}{\pgfqpoint{0.735980in}{0.368266in}}%
\pgfpathcurveto{\pgfqpoint{0.739887in}{0.372173in}}{\pgfqpoint{0.742082in}{0.377472in}}{\pgfqpoint{0.742082in}{0.382997in}}%
\pgfpathcurveto{\pgfqpoint{0.742082in}{0.388522in}}{\pgfqpoint{0.739887in}{0.393822in}}{\pgfqpoint{0.735980in}{0.397729in}}%
\pgfpathcurveto{\pgfqpoint{0.732073in}{0.401636in}}{\pgfqpoint{0.726774in}{0.403831in}}{\pgfqpoint{0.721249in}{0.403831in}}%
\pgfpathcurveto{\pgfqpoint{0.715724in}{0.403831in}}{\pgfqpoint{0.710424in}{0.401636in}}{\pgfqpoint{0.706518in}{0.397729in}}%
\pgfpathcurveto{\pgfqpoint{0.702611in}{0.393822in}}{\pgfqpoint{0.700416in}{0.388522in}}{\pgfqpoint{0.700416in}{0.382997in}}%
\pgfpathcurveto{\pgfqpoint{0.700416in}{0.377472in}}{\pgfqpoint{0.702611in}{0.372173in}}{\pgfqpoint{0.706518in}{0.368266in}}%
\pgfpathcurveto{\pgfqpoint{0.710424in}{0.364359in}}{\pgfqpoint{0.715724in}{0.362164in}}{\pgfqpoint{0.721249in}{0.362164in}}%
\pgfpathclose%
\pgfusepath{stroke,fill}%
\end{pgfscope}%
\begin{pgfscope}%
\pgfpathrectangle{\pgfqpoint{0.562500in}{0.275000in}}{\pgfqpoint{3.487500in}{1.925000in}}%
\pgfusepath{clip}%
\pgfsetbuttcap%
\pgfsetroundjoin%
\definecolor{currentfill}{rgb}{0.000000,0.000000,0.000000}%
\pgfsetfillcolor{currentfill}%
\pgfsetlinewidth{1.003750pt}%
\definecolor{currentstroke}{rgb}{0.000000,0.000000,0.000000}%
\pgfsetstrokecolor{currentstroke}%
\pgfsetdash{}{0pt}%
\pgfpathmoveto{\pgfqpoint{0.721249in}{0.362164in}}%
\pgfpathcurveto{\pgfqpoint{0.726774in}{0.362164in}}{\pgfqpoint{0.732073in}{0.364359in}}{\pgfqpoint{0.735980in}{0.368266in}}%
\pgfpathcurveto{\pgfqpoint{0.739887in}{0.372173in}}{\pgfqpoint{0.742082in}{0.377472in}}{\pgfqpoint{0.742082in}{0.382997in}}%
\pgfpathcurveto{\pgfqpoint{0.742082in}{0.388522in}}{\pgfqpoint{0.739887in}{0.393822in}}{\pgfqpoint{0.735980in}{0.397729in}}%
\pgfpathcurveto{\pgfqpoint{0.732073in}{0.401636in}}{\pgfqpoint{0.726774in}{0.403831in}}{\pgfqpoint{0.721249in}{0.403831in}}%
\pgfpathcurveto{\pgfqpoint{0.715724in}{0.403831in}}{\pgfqpoint{0.710424in}{0.401636in}}{\pgfqpoint{0.706518in}{0.397729in}}%
\pgfpathcurveto{\pgfqpoint{0.702611in}{0.393822in}}{\pgfqpoint{0.700416in}{0.388522in}}{\pgfqpoint{0.700416in}{0.382997in}}%
\pgfpathcurveto{\pgfqpoint{0.700416in}{0.377472in}}{\pgfqpoint{0.702611in}{0.372173in}}{\pgfqpoint{0.706518in}{0.368266in}}%
\pgfpathcurveto{\pgfqpoint{0.710424in}{0.364359in}}{\pgfqpoint{0.715724in}{0.362164in}}{\pgfqpoint{0.721249in}{0.362164in}}%
\pgfpathclose%
\pgfusepath{stroke,fill}%
\end{pgfscope}%
\begin{pgfscope}%
\pgfpathrectangle{\pgfqpoint{0.562500in}{0.275000in}}{\pgfqpoint{3.487500in}{1.925000in}}%
\pgfusepath{clip}%
\pgfsetbuttcap%
\pgfsetroundjoin%
\definecolor{currentfill}{rgb}{0.000000,0.000000,0.000000}%
\pgfsetfillcolor{currentfill}%
\pgfsetlinewidth{1.003750pt}%
\definecolor{currentstroke}{rgb}{0.000000,0.000000,0.000000}%
\pgfsetstrokecolor{currentstroke}%
\pgfsetdash{}{0pt}%
\pgfpathmoveto{\pgfqpoint{0.721249in}{0.357106in}}%
\pgfpathcurveto{\pgfqpoint{0.726774in}{0.357106in}}{\pgfqpoint{0.732073in}{0.359301in}}{\pgfqpoint{0.735980in}{0.363207in}}%
\pgfpathcurveto{\pgfqpoint{0.739887in}{0.367114in}}{\pgfqpoint{0.742082in}{0.372414in}}{\pgfqpoint{0.742082in}{0.377939in}}%
\pgfpathcurveto{\pgfqpoint{0.742082in}{0.383464in}}{\pgfqpoint{0.739887in}{0.388763in}}{\pgfqpoint{0.735980in}{0.392670in}}%
\pgfpathcurveto{\pgfqpoint{0.732073in}{0.396577in}}{\pgfqpoint{0.726774in}{0.398772in}}{\pgfqpoint{0.721249in}{0.398772in}}%
\pgfpathcurveto{\pgfqpoint{0.715724in}{0.398772in}}{\pgfqpoint{0.710424in}{0.396577in}}{\pgfqpoint{0.706518in}{0.392670in}}%
\pgfpathcurveto{\pgfqpoint{0.702611in}{0.388763in}}{\pgfqpoint{0.700416in}{0.383464in}}{\pgfqpoint{0.700416in}{0.377939in}}%
\pgfpathcurveto{\pgfqpoint{0.700416in}{0.372414in}}{\pgfqpoint{0.702611in}{0.367114in}}{\pgfqpoint{0.706518in}{0.363207in}}%
\pgfpathcurveto{\pgfqpoint{0.710424in}{0.359301in}}{\pgfqpoint{0.715724in}{0.357106in}}{\pgfqpoint{0.721249in}{0.357106in}}%
\pgfpathclose%
\pgfusepath{stroke,fill}%
\end{pgfscope}%
\begin{pgfscope}%
\pgfpathrectangle{\pgfqpoint{0.562500in}{0.275000in}}{\pgfqpoint{3.487500in}{1.925000in}}%
\pgfusepath{clip}%
\pgfsetbuttcap%
\pgfsetroundjoin%
\definecolor{currentfill}{rgb}{0.000000,0.000000,0.000000}%
\pgfsetfillcolor{currentfill}%
\pgfsetlinewidth{1.003750pt}%
\definecolor{currentstroke}{rgb}{0.000000,0.000000,0.000000}%
\pgfsetstrokecolor{currentstroke}%
\pgfsetdash{}{0pt}%
\pgfpathmoveto{\pgfqpoint{0.721249in}{0.357106in}}%
\pgfpathcurveto{\pgfqpoint{0.726774in}{0.357106in}}{\pgfqpoint{0.732073in}{0.359301in}}{\pgfqpoint{0.735980in}{0.363207in}}%
\pgfpathcurveto{\pgfqpoint{0.739887in}{0.367114in}}{\pgfqpoint{0.742082in}{0.372414in}}{\pgfqpoint{0.742082in}{0.377939in}}%
\pgfpathcurveto{\pgfqpoint{0.742082in}{0.383464in}}{\pgfqpoint{0.739887in}{0.388763in}}{\pgfqpoint{0.735980in}{0.392670in}}%
\pgfpathcurveto{\pgfqpoint{0.732073in}{0.396577in}}{\pgfqpoint{0.726774in}{0.398772in}}{\pgfqpoint{0.721249in}{0.398772in}}%
\pgfpathcurveto{\pgfqpoint{0.715724in}{0.398772in}}{\pgfqpoint{0.710424in}{0.396577in}}{\pgfqpoint{0.706518in}{0.392670in}}%
\pgfpathcurveto{\pgfqpoint{0.702611in}{0.388763in}}{\pgfqpoint{0.700416in}{0.383464in}}{\pgfqpoint{0.700416in}{0.377939in}}%
\pgfpathcurveto{\pgfqpoint{0.700416in}{0.372414in}}{\pgfqpoint{0.702611in}{0.367114in}}{\pgfqpoint{0.706518in}{0.363207in}}%
\pgfpathcurveto{\pgfqpoint{0.710424in}{0.359301in}}{\pgfqpoint{0.715724in}{0.357106in}}{\pgfqpoint{0.721249in}{0.357106in}}%
\pgfpathclose%
\pgfusepath{stroke,fill}%
\end{pgfscope}%
\begin{pgfscope}%
\pgfpathrectangle{\pgfqpoint{0.562500in}{0.275000in}}{\pgfqpoint{3.487500in}{1.925000in}}%
\pgfusepath{clip}%
\pgfsetbuttcap%
\pgfsetroundjoin%
\definecolor{currentfill}{rgb}{0.000000,0.000000,0.000000}%
\pgfsetfillcolor{currentfill}%
\pgfsetlinewidth{1.003750pt}%
\definecolor{currentstroke}{rgb}{0.000000,0.000000,0.000000}%
\pgfsetstrokecolor{currentstroke}%
\pgfsetdash{}{0pt}%
\pgfpathmoveto{\pgfqpoint{0.721249in}{0.372281in}}%
\pgfpathcurveto{\pgfqpoint{0.726774in}{0.372281in}}{\pgfqpoint{0.732073in}{0.374476in}}{\pgfqpoint{0.735980in}{0.378383in}}%
\pgfpathcurveto{\pgfqpoint{0.739887in}{0.382290in}}{\pgfqpoint{0.742082in}{0.387590in}}{\pgfqpoint{0.742082in}{0.393115in}}%
\pgfpathcurveto{\pgfqpoint{0.742082in}{0.398640in}}{\pgfqpoint{0.739887in}{0.403939in}}{\pgfqpoint{0.735980in}{0.407846in}}%
\pgfpathcurveto{\pgfqpoint{0.732073in}{0.411753in}}{\pgfqpoint{0.726774in}{0.413948in}}{\pgfqpoint{0.721249in}{0.413948in}}%
\pgfpathcurveto{\pgfqpoint{0.715724in}{0.413948in}}{\pgfqpoint{0.710424in}{0.411753in}}{\pgfqpoint{0.706518in}{0.407846in}}%
\pgfpathcurveto{\pgfqpoint{0.702611in}{0.403939in}}{\pgfqpoint{0.700416in}{0.398640in}}{\pgfqpoint{0.700416in}{0.393115in}}%
\pgfpathcurveto{\pgfqpoint{0.700416in}{0.387590in}}{\pgfqpoint{0.702611in}{0.382290in}}{\pgfqpoint{0.706518in}{0.378383in}}%
\pgfpathcurveto{\pgfqpoint{0.710424in}{0.374476in}}{\pgfqpoint{0.715724in}{0.372281in}}{\pgfqpoint{0.721249in}{0.372281in}}%
\pgfpathclose%
\pgfusepath{stroke,fill}%
\end{pgfscope}%
\begin{pgfscope}%
\pgfpathrectangle{\pgfqpoint{0.562500in}{0.275000in}}{\pgfqpoint{3.487500in}{1.925000in}}%
\pgfusepath{clip}%
\pgfsetbuttcap%
\pgfsetroundjoin%
\definecolor{currentfill}{rgb}{0.000000,0.000000,0.000000}%
\pgfsetfillcolor{currentfill}%
\pgfsetlinewidth{1.003750pt}%
\definecolor{currentstroke}{rgb}{0.000000,0.000000,0.000000}%
\pgfsetstrokecolor{currentstroke}%
\pgfsetdash{}{0pt}%
\pgfpathmoveto{\pgfqpoint{0.721249in}{0.352047in}}%
\pgfpathcurveto{\pgfqpoint{0.726774in}{0.352047in}}{\pgfqpoint{0.732073in}{0.354242in}}{\pgfqpoint{0.735980in}{0.358149in}}%
\pgfpathcurveto{\pgfqpoint{0.739887in}{0.362056in}}{\pgfqpoint{0.742082in}{0.367355in}}{\pgfqpoint{0.742082in}{0.372880in}}%
\pgfpathcurveto{\pgfqpoint{0.742082in}{0.378405in}}{\pgfqpoint{0.739887in}{0.383705in}}{\pgfqpoint{0.735980in}{0.387612in}}%
\pgfpathcurveto{\pgfqpoint{0.732073in}{0.391519in}}{\pgfqpoint{0.726774in}{0.393714in}}{\pgfqpoint{0.721249in}{0.393714in}}%
\pgfpathcurveto{\pgfqpoint{0.715724in}{0.393714in}}{\pgfqpoint{0.710424in}{0.391519in}}{\pgfqpoint{0.706518in}{0.387612in}}%
\pgfpathcurveto{\pgfqpoint{0.702611in}{0.383705in}}{\pgfqpoint{0.700416in}{0.378405in}}{\pgfqpoint{0.700416in}{0.372880in}}%
\pgfpathcurveto{\pgfqpoint{0.700416in}{0.367355in}}{\pgfqpoint{0.702611in}{0.362056in}}{\pgfqpoint{0.706518in}{0.358149in}}%
\pgfpathcurveto{\pgfqpoint{0.710424in}{0.354242in}}{\pgfqpoint{0.715724in}{0.352047in}}{\pgfqpoint{0.721249in}{0.352047in}}%
\pgfpathclose%
\pgfusepath{stroke,fill}%
\end{pgfscope}%
\begin{pgfscope}%
\pgfpathrectangle{\pgfqpoint{0.562500in}{0.275000in}}{\pgfqpoint{3.487500in}{1.925000in}}%
\pgfusepath{clip}%
\pgfsetbuttcap%
\pgfsetroundjoin%
\definecolor{currentfill}{rgb}{0.000000,0.000000,0.000000}%
\pgfsetfillcolor{currentfill}%
\pgfsetlinewidth{1.003750pt}%
\definecolor{currentstroke}{rgb}{0.000000,0.000000,0.000000}%
\pgfsetstrokecolor{currentstroke}%
\pgfsetdash{}{0pt}%
\pgfpathmoveto{\pgfqpoint{0.721249in}{0.352047in}}%
\pgfpathcurveto{\pgfqpoint{0.726774in}{0.352047in}}{\pgfqpoint{0.732073in}{0.354242in}}{\pgfqpoint{0.735980in}{0.358149in}}%
\pgfpathcurveto{\pgfqpoint{0.739887in}{0.362056in}}{\pgfqpoint{0.742082in}{0.367355in}}{\pgfqpoint{0.742082in}{0.372880in}}%
\pgfpathcurveto{\pgfqpoint{0.742082in}{0.378405in}}{\pgfqpoint{0.739887in}{0.383705in}}{\pgfqpoint{0.735980in}{0.387612in}}%
\pgfpathcurveto{\pgfqpoint{0.732073in}{0.391519in}}{\pgfqpoint{0.726774in}{0.393714in}}{\pgfqpoint{0.721249in}{0.393714in}}%
\pgfpathcurveto{\pgfqpoint{0.715724in}{0.393714in}}{\pgfqpoint{0.710424in}{0.391519in}}{\pgfqpoint{0.706518in}{0.387612in}}%
\pgfpathcurveto{\pgfqpoint{0.702611in}{0.383705in}}{\pgfqpoint{0.700416in}{0.378405in}}{\pgfqpoint{0.700416in}{0.372880in}}%
\pgfpathcurveto{\pgfqpoint{0.700416in}{0.367355in}}{\pgfqpoint{0.702611in}{0.362056in}}{\pgfqpoint{0.706518in}{0.358149in}}%
\pgfpathcurveto{\pgfqpoint{0.710424in}{0.354242in}}{\pgfqpoint{0.715724in}{0.352047in}}{\pgfqpoint{0.721249in}{0.352047in}}%
\pgfpathclose%
\pgfusepath{stroke,fill}%
\end{pgfscope}%
\begin{pgfscope}%
\pgfpathrectangle{\pgfqpoint{0.562500in}{0.275000in}}{\pgfqpoint{3.487500in}{1.925000in}}%
\pgfusepath{clip}%
\pgfsetbuttcap%
\pgfsetroundjoin%
\definecolor{currentfill}{rgb}{0.000000,0.000000,0.000000}%
\pgfsetfillcolor{currentfill}%
\pgfsetlinewidth{1.003750pt}%
\definecolor{currentstroke}{rgb}{0.000000,0.000000,0.000000}%
\pgfsetstrokecolor{currentstroke}%
\pgfsetdash{}{0pt}%
\pgfpathmoveto{\pgfqpoint{0.721249in}{0.357106in}}%
\pgfpathcurveto{\pgfqpoint{0.726774in}{0.357106in}}{\pgfqpoint{0.732073in}{0.359301in}}{\pgfqpoint{0.735980in}{0.363207in}}%
\pgfpathcurveto{\pgfqpoint{0.739887in}{0.367114in}}{\pgfqpoint{0.742082in}{0.372414in}}{\pgfqpoint{0.742082in}{0.377939in}}%
\pgfpathcurveto{\pgfqpoint{0.742082in}{0.383464in}}{\pgfqpoint{0.739887in}{0.388763in}}{\pgfqpoint{0.735980in}{0.392670in}}%
\pgfpathcurveto{\pgfqpoint{0.732073in}{0.396577in}}{\pgfqpoint{0.726774in}{0.398772in}}{\pgfqpoint{0.721249in}{0.398772in}}%
\pgfpathcurveto{\pgfqpoint{0.715724in}{0.398772in}}{\pgfqpoint{0.710424in}{0.396577in}}{\pgfqpoint{0.706518in}{0.392670in}}%
\pgfpathcurveto{\pgfqpoint{0.702611in}{0.388763in}}{\pgfqpoint{0.700416in}{0.383464in}}{\pgfqpoint{0.700416in}{0.377939in}}%
\pgfpathcurveto{\pgfqpoint{0.700416in}{0.372414in}}{\pgfqpoint{0.702611in}{0.367114in}}{\pgfqpoint{0.706518in}{0.363207in}}%
\pgfpathcurveto{\pgfqpoint{0.710424in}{0.359301in}}{\pgfqpoint{0.715724in}{0.357106in}}{\pgfqpoint{0.721249in}{0.357106in}}%
\pgfpathclose%
\pgfusepath{stroke,fill}%
\end{pgfscope}%
\begin{pgfscope}%
\pgfpathrectangle{\pgfqpoint{0.562500in}{0.275000in}}{\pgfqpoint{3.487500in}{1.925000in}}%
\pgfusepath{clip}%
\pgfsetbuttcap%
\pgfsetroundjoin%
\definecolor{currentfill}{rgb}{0.000000,0.000000,0.000000}%
\pgfsetfillcolor{currentfill}%
\pgfsetlinewidth{1.003750pt}%
\definecolor{currentstroke}{rgb}{0.000000,0.000000,0.000000}%
\pgfsetstrokecolor{currentstroke}%
\pgfsetdash{}{0pt}%
\pgfpathmoveto{\pgfqpoint{0.721249in}{0.362164in}}%
\pgfpathcurveto{\pgfqpoint{0.726774in}{0.362164in}}{\pgfqpoint{0.732073in}{0.364359in}}{\pgfqpoint{0.735980in}{0.368266in}}%
\pgfpathcurveto{\pgfqpoint{0.739887in}{0.372173in}}{\pgfqpoint{0.742082in}{0.377472in}}{\pgfqpoint{0.742082in}{0.382997in}}%
\pgfpathcurveto{\pgfqpoint{0.742082in}{0.388522in}}{\pgfqpoint{0.739887in}{0.393822in}}{\pgfqpoint{0.735980in}{0.397729in}}%
\pgfpathcurveto{\pgfqpoint{0.732073in}{0.401636in}}{\pgfqpoint{0.726774in}{0.403831in}}{\pgfqpoint{0.721249in}{0.403831in}}%
\pgfpathcurveto{\pgfqpoint{0.715724in}{0.403831in}}{\pgfqpoint{0.710424in}{0.401636in}}{\pgfqpoint{0.706518in}{0.397729in}}%
\pgfpathcurveto{\pgfqpoint{0.702611in}{0.393822in}}{\pgfqpoint{0.700416in}{0.388522in}}{\pgfqpoint{0.700416in}{0.382997in}}%
\pgfpathcurveto{\pgfqpoint{0.700416in}{0.377472in}}{\pgfqpoint{0.702611in}{0.372173in}}{\pgfqpoint{0.706518in}{0.368266in}}%
\pgfpathcurveto{\pgfqpoint{0.710424in}{0.364359in}}{\pgfqpoint{0.715724in}{0.362164in}}{\pgfqpoint{0.721249in}{0.362164in}}%
\pgfpathclose%
\pgfusepath{stroke,fill}%
\end{pgfscope}%
\begin{pgfscope}%
\pgfpathrectangle{\pgfqpoint{0.562500in}{0.275000in}}{\pgfqpoint{3.487500in}{1.925000in}}%
\pgfusepath{clip}%
\pgfsetbuttcap%
\pgfsetroundjoin%
\definecolor{currentfill}{rgb}{0.000000,0.000000,0.000000}%
\pgfsetfillcolor{currentfill}%
\pgfsetlinewidth{1.003750pt}%
\definecolor{currentstroke}{rgb}{0.000000,0.000000,0.000000}%
\pgfsetstrokecolor{currentstroke}%
\pgfsetdash{}{0pt}%
\pgfpathmoveto{\pgfqpoint{0.721249in}{0.352047in}}%
\pgfpathcurveto{\pgfqpoint{0.726774in}{0.352047in}}{\pgfqpoint{0.732073in}{0.354242in}}{\pgfqpoint{0.735980in}{0.358149in}}%
\pgfpathcurveto{\pgfqpoint{0.739887in}{0.362056in}}{\pgfqpoint{0.742082in}{0.367355in}}{\pgfqpoint{0.742082in}{0.372880in}}%
\pgfpathcurveto{\pgfqpoint{0.742082in}{0.378405in}}{\pgfqpoint{0.739887in}{0.383705in}}{\pgfqpoint{0.735980in}{0.387612in}}%
\pgfpathcurveto{\pgfqpoint{0.732073in}{0.391519in}}{\pgfqpoint{0.726774in}{0.393714in}}{\pgfqpoint{0.721249in}{0.393714in}}%
\pgfpathcurveto{\pgfqpoint{0.715724in}{0.393714in}}{\pgfqpoint{0.710424in}{0.391519in}}{\pgfqpoint{0.706518in}{0.387612in}}%
\pgfpathcurveto{\pgfqpoint{0.702611in}{0.383705in}}{\pgfqpoint{0.700416in}{0.378405in}}{\pgfqpoint{0.700416in}{0.372880in}}%
\pgfpathcurveto{\pgfqpoint{0.700416in}{0.367355in}}{\pgfqpoint{0.702611in}{0.362056in}}{\pgfqpoint{0.706518in}{0.358149in}}%
\pgfpathcurveto{\pgfqpoint{0.710424in}{0.354242in}}{\pgfqpoint{0.715724in}{0.352047in}}{\pgfqpoint{0.721249in}{0.352047in}}%
\pgfpathclose%
\pgfusepath{stroke,fill}%
\end{pgfscope}%
\begin{pgfscope}%
\pgfpathrectangle{\pgfqpoint{0.562500in}{0.275000in}}{\pgfqpoint{3.487500in}{1.925000in}}%
\pgfusepath{clip}%
\pgfsetbuttcap%
\pgfsetroundjoin%
\definecolor{currentfill}{rgb}{0.000000,0.000000,0.000000}%
\pgfsetfillcolor{currentfill}%
\pgfsetlinewidth{1.003750pt}%
\definecolor{currentstroke}{rgb}{0.000000,0.000000,0.000000}%
\pgfsetstrokecolor{currentstroke}%
\pgfsetdash{}{0pt}%
\pgfpathmoveto{\pgfqpoint{1.772992in}{0.539214in}}%
\pgfpathcurveto{\pgfqpoint{1.778517in}{0.539214in}}{\pgfqpoint{1.783816in}{0.541409in}}{\pgfqpoint{1.787723in}{0.545316in}}%
\pgfpathcurveto{\pgfqpoint{1.791630in}{0.549223in}}{\pgfqpoint{1.793825in}{0.554522in}}{\pgfqpoint{1.793825in}{0.560047in}}%
\pgfpathcurveto{\pgfqpoint{1.793825in}{0.565572in}}{\pgfqpoint{1.791630in}{0.570872in}}{\pgfqpoint{1.787723in}{0.574779in}}%
\pgfpathcurveto{\pgfqpoint{1.783816in}{0.578685in}}{\pgfqpoint{1.778517in}{0.580881in}}{\pgfqpoint{1.772992in}{0.580881in}}%
\pgfpathcurveto{\pgfqpoint{1.767467in}{0.580881in}}{\pgfqpoint{1.762167in}{0.578685in}}{\pgfqpoint{1.758260in}{0.574779in}}%
\pgfpathcurveto{\pgfqpoint{1.754353in}{0.570872in}}{\pgfqpoint{1.752158in}{0.565572in}}{\pgfqpoint{1.752158in}{0.560047in}}%
\pgfpathcurveto{\pgfqpoint{1.752158in}{0.554522in}}{\pgfqpoint{1.754353in}{0.549223in}}{\pgfqpoint{1.758260in}{0.545316in}}%
\pgfpathcurveto{\pgfqpoint{1.762167in}{0.541409in}}{\pgfqpoint{1.767467in}{0.539214in}}{\pgfqpoint{1.772992in}{0.539214in}}%
\pgfpathclose%
\pgfusepath{stroke,fill}%
\end{pgfscope}%
\begin{pgfscope}%
\pgfpathrectangle{\pgfqpoint{0.562500in}{0.275000in}}{\pgfqpoint{3.487500in}{1.925000in}}%
\pgfusepath{clip}%
\pgfsetbuttcap%
\pgfsetroundjoin%
\definecolor{currentfill}{rgb}{0.000000,0.000000,0.000000}%
\pgfsetfillcolor{currentfill}%
\pgfsetlinewidth{1.003750pt}%
\definecolor{currentstroke}{rgb}{0.000000,0.000000,0.000000}%
\pgfsetstrokecolor{currentstroke}%
\pgfsetdash{}{0pt}%
\pgfpathmoveto{\pgfqpoint{1.772992in}{0.513921in}}%
\pgfpathcurveto{\pgfqpoint{1.778517in}{0.513921in}}{\pgfqpoint{1.783816in}{0.516116in}}{\pgfqpoint{1.787723in}{0.520023in}}%
\pgfpathcurveto{\pgfqpoint{1.791630in}{0.523930in}}{\pgfqpoint{1.793825in}{0.529229in}}{\pgfqpoint{1.793825in}{0.534754in}}%
\pgfpathcurveto{\pgfqpoint{1.793825in}{0.540279in}}{\pgfqpoint{1.791630in}{0.545579in}}{\pgfqpoint{1.787723in}{0.549486in}}%
\pgfpathcurveto{\pgfqpoint{1.783816in}{0.553393in}}{\pgfqpoint{1.778517in}{0.555588in}}{\pgfqpoint{1.772992in}{0.555588in}}%
\pgfpathcurveto{\pgfqpoint{1.767467in}{0.555588in}}{\pgfqpoint{1.762167in}{0.553393in}}{\pgfqpoint{1.758260in}{0.549486in}}%
\pgfpathcurveto{\pgfqpoint{1.754353in}{0.545579in}}{\pgfqpoint{1.752158in}{0.540279in}}{\pgfqpoint{1.752158in}{0.534754in}}%
\pgfpathcurveto{\pgfqpoint{1.752158in}{0.529229in}}{\pgfqpoint{1.754353in}{0.523930in}}{\pgfqpoint{1.758260in}{0.520023in}}%
\pgfpathcurveto{\pgfqpoint{1.762167in}{0.516116in}}{\pgfqpoint{1.767467in}{0.513921in}}{\pgfqpoint{1.772992in}{0.513921in}}%
\pgfpathclose%
\pgfusepath{stroke,fill}%
\end{pgfscope}%
\begin{pgfscope}%
\pgfpathrectangle{\pgfqpoint{0.562500in}{0.275000in}}{\pgfqpoint{3.487500in}{1.925000in}}%
\pgfusepath{clip}%
\pgfsetbuttcap%
\pgfsetroundjoin%
\definecolor{currentfill}{rgb}{0.000000,0.000000,0.000000}%
\pgfsetfillcolor{currentfill}%
\pgfsetlinewidth{1.003750pt}%
\definecolor{currentstroke}{rgb}{0.000000,0.000000,0.000000}%
\pgfsetstrokecolor{currentstroke}%
\pgfsetdash{}{0pt}%
\pgfpathmoveto{\pgfqpoint{1.772992in}{0.488628in}}%
\pgfpathcurveto{\pgfqpoint{1.778517in}{0.488628in}}{\pgfqpoint{1.783816in}{0.490823in}}{\pgfqpoint{1.787723in}{0.494730in}}%
\pgfpathcurveto{\pgfqpoint{1.791630in}{0.498637in}}{\pgfqpoint{1.793825in}{0.503937in}}{\pgfqpoint{1.793825in}{0.509462in}}%
\pgfpathcurveto{\pgfqpoint{1.793825in}{0.514987in}}{\pgfqpoint{1.791630in}{0.520286in}}{\pgfqpoint{1.787723in}{0.524193in}}%
\pgfpathcurveto{\pgfqpoint{1.783816in}{0.528100in}}{\pgfqpoint{1.778517in}{0.530295in}}{\pgfqpoint{1.772992in}{0.530295in}}%
\pgfpathcurveto{\pgfqpoint{1.767467in}{0.530295in}}{\pgfqpoint{1.762167in}{0.528100in}}{\pgfqpoint{1.758260in}{0.524193in}}%
\pgfpathcurveto{\pgfqpoint{1.754353in}{0.520286in}}{\pgfqpoint{1.752158in}{0.514987in}}{\pgfqpoint{1.752158in}{0.509462in}}%
\pgfpathcurveto{\pgfqpoint{1.752158in}{0.503937in}}{\pgfqpoint{1.754353in}{0.498637in}}{\pgfqpoint{1.758260in}{0.494730in}}%
\pgfpathcurveto{\pgfqpoint{1.762167in}{0.490823in}}{\pgfqpoint{1.767467in}{0.488628in}}{\pgfqpoint{1.772992in}{0.488628in}}%
\pgfpathclose%
\pgfusepath{stroke,fill}%
\end{pgfscope}%
\begin{pgfscope}%
\pgfpathrectangle{\pgfqpoint{0.562500in}{0.275000in}}{\pgfqpoint{3.487500in}{1.925000in}}%
\pgfusepath{clip}%
\pgfsetbuttcap%
\pgfsetroundjoin%
\definecolor{currentfill}{rgb}{0.000000,0.000000,0.000000}%
\pgfsetfillcolor{currentfill}%
\pgfsetlinewidth{1.003750pt}%
\definecolor{currentstroke}{rgb}{0.000000,0.000000,0.000000}%
\pgfsetstrokecolor{currentstroke}%
\pgfsetdash{}{0pt}%
\pgfpathmoveto{\pgfqpoint{1.772992in}{0.544272in}}%
\pgfpathcurveto{\pgfqpoint{1.778517in}{0.544272in}}{\pgfqpoint{1.783816in}{0.546468in}}{\pgfqpoint{1.787723in}{0.550374in}}%
\pgfpathcurveto{\pgfqpoint{1.791630in}{0.554281in}}{\pgfqpoint{1.793825in}{0.559581in}}{\pgfqpoint{1.793825in}{0.565106in}}%
\pgfpathcurveto{\pgfqpoint{1.793825in}{0.570631in}}{\pgfqpoint{1.791630in}{0.575930in}}{\pgfqpoint{1.787723in}{0.579837in}}%
\pgfpathcurveto{\pgfqpoint{1.783816in}{0.583744in}}{\pgfqpoint{1.778517in}{0.585939in}}{\pgfqpoint{1.772992in}{0.585939in}}%
\pgfpathcurveto{\pgfqpoint{1.767467in}{0.585939in}}{\pgfqpoint{1.762167in}{0.583744in}}{\pgfqpoint{1.758260in}{0.579837in}}%
\pgfpathcurveto{\pgfqpoint{1.754353in}{0.575930in}}{\pgfqpoint{1.752158in}{0.570631in}}{\pgfqpoint{1.752158in}{0.565106in}}%
\pgfpathcurveto{\pgfqpoint{1.752158in}{0.559581in}}{\pgfqpoint{1.754353in}{0.554281in}}{\pgfqpoint{1.758260in}{0.550374in}}%
\pgfpathcurveto{\pgfqpoint{1.762167in}{0.546468in}}{\pgfqpoint{1.767467in}{0.544272in}}{\pgfqpoint{1.772992in}{0.544272in}}%
\pgfpathclose%
\pgfusepath{stroke,fill}%
\end{pgfscope}%
\begin{pgfscope}%
\pgfpathrectangle{\pgfqpoint{0.562500in}{0.275000in}}{\pgfqpoint{3.487500in}{1.925000in}}%
\pgfusepath{clip}%
\pgfsetbuttcap%
\pgfsetroundjoin%
\definecolor{currentfill}{rgb}{0.000000,0.000000,0.000000}%
\pgfsetfillcolor{currentfill}%
\pgfsetlinewidth{1.003750pt}%
\definecolor{currentstroke}{rgb}{0.000000,0.000000,0.000000}%
\pgfsetstrokecolor{currentstroke}%
\pgfsetdash{}{0pt}%
\pgfpathmoveto{\pgfqpoint{1.772992in}{0.559448in}}%
\pgfpathcurveto{\pgfqpoint{1.778517in}{0.559448in}}{\pgfqpoint{1.783816in}{0.561643in}}{\pgfqpoint{1.787723in}{0.565550in}}%
\pgfpathcurveto{\pgfqpoint{1.791630in}{0.569457in}}{\pgfqpoint{1.793825in}{0.574756in}}{\pgfqpoint{1.793825in}{0.580282in}}%
\pgfpathcurveto{\pgfqpoint{1.793825in}{0.585807in}}{\pgfqpoint{1.791630in}{0.591106in}}{\pgfqpoint{1.787723in}{0.595013in}}%
\pgfpathcurveto{\pgfqpoint{1.783816in}{0.598920in}}{\pgfqpoint{1.778517in}{0.601115in}}{\pgfqpoint{1.772992in}{0.601115in}}%
\pgfpathcurveto{\pgfqpoint{1.767467in}{0.601115in}}{\pgfqpoint{1.762167in}{0.598920in}}{\pgfqpoint{1.758260in}{0.595013in}}%
\pgfpathcurveto{\pgfqpoint{1.754353in}{0.591106in}}{\pgfqpoint{1.752158in}{0.585807in}}{\pgfqpoint{1.752158in}{0.580282in}}%
\pgfpathcurveto{\pgfqpoint{1.752158in}{0.574756in}}{\pgfqpoint{1.754353in}{0.569457in}}{\pgfqpoint{1.758260in}{0.565550in}}%
\pgfpathcurveto{\pgfqpoint{1.762167in}{0.561643in}}{\pgfqpoint{1.767467in}{0.559448in}}{\pgfqpoint{1.772992in}{0.559448in}}%
\pgfpathclose%
\pgfusepath{stroke,fill}%
\end{pgfscope}%
\begin{pgfscope}%
\pgfpathrectangle{\pgfqpoint{0.562500in}{0.275000in}}{\pgfqpoint{3.487500in}{1.925000in}}%
\pgfusepath{clip}%
\pgfsetbuttcap%
\pgfsetroundjoin%
\definecolor{currentfill}{rgb}{0.000000,0.000000,0.000000}%
\pgfsetfillcolor{currentfill}%
\pgfsetlinewidth{1.003750pt}%
\definecolor{currentstroke}{rgb}{0.000000,0.000000,0.000000}%
\pgfsetstrokecolor{currentstroke}%
\pgfsetdash{}{0pt}%
\pgfpathmoveto{\pgfqpoint{1.772992in}{0.513921in}}%
\pgfpathcurveto{\pgfqpoint{1.778517in}{0.513921in}}{\pgfqpoint{1.783816in}{0.516116in}}{\pgfqpoint{1.787723in}{0.520023in}}%
\pgfpathcurveto{\pgfqpoint{1.791630in}{0.523930in}}{\pgfqpoint{1.793825in}{0.529229in}}{\pgfqpoint{1.793825in}{0.534754in}}%
\pgfpathcurveto{\pgfqpoint{1.793825in}{0.540279in}}{\pgfqpoint{1.791630in}{0.545579in}}{\pgfqpoint{1.787723in}{0.549486in}}%
\pgfpathcurveto{\pgfqpoint{1.783816in}{0.553393in}}{\pgfqpoint{1.778517in}{0.555588in}}{\pgfqpoint{1.772992in}{0.555588in}}%
\pgfpathcurveto{\pgfqpoint{1.767467in}{0.555588in}}{\pgfqpoint{1.762167in}{0.553393in}}{\pgfqpoint{1.758260in}{0.549486in}}%
\pgfpathcurveto{\pgfqpoint{1.754353in}{0.545579in}}{\pgfqpoint{1.752158in}{0.540279in}}{\pgfqpoint{1.752158in}{0.534754in}}%
\pgfpathcurveto{\pgfqpoint{1.752158in}{0.529229in}}{\pgfqpoint{1.754353in}{0.523930in}}{\pgfqpoint{1.758260in}{0.520023in}}%
\pgfpathcurveto{\pgfqpoint{1.762167in}{0.516116in}}{\pgfqpoint{1.767467in}{0.513921in}}{\pgfqpoint{1.772992in}{0.513921in}}%
\pgfpathclose%
\pgfusepath{stroke,fill}%
\end{pgfscope}%
\begin{pgfscope}%
\pgfpathrectangle{\pgfqpoint{0.562500in}{0.275000in}}{\pgfqpoint{3.487500in}{1.925000in}}%
\pgfusepath{clip}%
\pgfsetbuttcap%
\pgfsetroundjoin%
\definecolor{currentfill}{rgb}{0.000000,0.000000,0.000000}%
\pgfsetfillcolor{currentfill}%
\pgfsetlinewidth{1.003750pt}%
\definecolor{currentstroke}{rgb}{0.000000,0.000000,0.000000}%
\pgfsetstrokecolor{currentstroke}%
\pgfsetdash{}{0pt}%
\pgfpathmoveto{\pgfqpoint{1.772992in}{0.534155in}}%
\pgfpathcurveto{\pgfqpoint{1.778517in}{0.534155in}}{\pgfqpoint{1.783816in}{0.536350in}}{\pgfqpoint{1.787723in}{0.540257in}}%
\pgfpathcurveto{\pgfqpoint{1.791630in}{0.544164in}}{\pgfqpoint{1.793825in}{0.549464in}}{\pgfqpoint{1.793825in}{0.554989in}}%
\pgfpathcurveto{\pgfqpoint{1.793825in}{0.560514in}}{\pgfqpoint{1.791630in}{0.565813in}}{\pgfqpoint{1.787723in}{0.569720in}}%
\pgfpathcurveto{\pgfqpoint{1.783816in}{0.573627in}}{\pgfqpoint{1.778517in}{0.575822in}}{\pgfqpoint{1.772992in}{0.575822in}}%
\pgfpathcurveto{\pgfqpoint{1.767467in}{0.575822in}}{\pgfqpoint{1.762167in}{0.573627in}}{\pgfqpoint{1.758260in}{0.569720in}}%
\pgfpathcurveto{\pgfqpoint{1.754353in}{0.565813in}}{\pgfqpoint{1.752158in}{0.560514in}}{\pgfqpoint{1.752158in}{0.554989in}}%
\pgfpathcurveto{\pgfqpoint{1.752158in}{0.549464in}}{\pgfqpoint{1.754353in}{0.544164in}}{\pgfqpoint{1.758260in}{0.540257in}}%
\pgfpathcurveto{\pgfqpoint{1.762167in}{0.536350in}}{\pgfqpoint{1.767467in}{0.534155in}}{\pgfqpoint{1.772992in}{0.534155in}}%
\pgfpathclose%
\pgfusepath{stroke,fill}%
\end{pgfscope}%
\begin{pgfscope}%
\pgfpathrectangle{\pgfqpoint{0.562500in}{0.275000in}}{\pgfqpoint{3.487500in}{1.925000in}}%
\pgfusepath{clip}%
\pgfsetbuttcap%
\pgfsetroundjoin%
\definecolor{currentfill}{rgb}{0.000000,0.000000,0.000000}%
\pgfsetfillcolor{currentfill}%
\pgfsetlinewidth{1.003750pt}%
\definecolor{currentstroke}{rgb}{0.000000,0.000000,0.000000}%
\pgfsetstrokecolor{currentstroke}%
\pgfsetdash{}{0pt}%
\pgfpathmoveto{\pgfqpoint{1.772992in}{0.534155in}}%
\pgfpathcurveto{\pgfqpoint{1.778517in}{0.534155in}}{\pgfqpoint{1.783816in}{0.536350in}}{\pgfqpoint{1.787723in}{0.540257in}}%
\pgfpathcurveto{\pgfqpoint{1.791630in}{0.544164in}}{\pgfqpoint{1.793825in}{0.549464in}}{\pgfqpoint{1.793825in}{0.554989in}}%
\pgfpathcurveto{\pgfqpoint{1.793825in}{0.560514in}}{\pgfqpoint{1.791630in}{0.565813in}}{\pgfqpoint{1.787723in}{0.569720in}}%
\pgfpathcurveto{\pgfqpoint{1.783816in}{0.573627in}}{\pgfqpoint{1.778517in}{0.575822in}}{\pgfqpoint{1.772992in}{0.575822in}}%
\pgfpathcurveto{\pgfqpoint{1.767467in}{0.575822in}}{\pgfqpoint{1.762167in}{0.573627in}}{\pgfqpoint{1.758260in}{0.569720in}}%
\pgfpathcurveto{\pgfqpoint{1.754353in}{0.565813in}}{\pgfqpoint{1.752158in}{0.560514in}}{\pgfqpoint{1.752158in}{0.554989in}}%
\pgfpathcurveto{\pgfqpoint{1.752158in}{0.549464in}}{\pgfqpoint{1.754353in}{0.544164in}}{\pgfqpoint{1.758260in}{0.540257in}}%
\pgfpathcurveto{\pgfqpoint{1.762167in}{0.536350in}}{\pgfqpoint{1.767467in}{0.534155in}}{\pgfqpoint{1.772992in}{0.534155in}}%
\pgfpathclose%
\pgfusepath{stroke,fill}%
\end{pgfscope}%
\begin{pgfscope}%
\pgfpathrectangle{\pgfqpoint{0.562500in}{0.275000in}}{\pgfqpoint{3.487500in}{1.925000in}}%
\pgfusepath{clip}%
\pgfsetbuttcap%
\pgfsetroundjoin%
\definecolor{currentfill}{rgb}{0.000000,0.000000,0.000000}%
\pgfsetfillcolor{currentfill}%
\pgfsetlinewidth{1.003750pt}%
\definecolor{currentstroke}{rgb}{0.000000,0.000000,0.000000}%
\pgfsetstrokecolor{currentstroke}%
\pgfsetdash{}{0pt}%
\pgfpathmoveto{\pgfqpoint{1.772992in}{0.549331in}}%
\pgfpathcurveto{\pgfqpoint{1.778517in}{0.549331in}}{\pgfqpoint{1.783816in}{0.551526in}}{\pgfqpoint{1.787723in}{0.555433in}}%
\pgfpathcurveto{\pgfqpoint{1.791630in}{0.559340in}}{\pgfqpoint{1.793825in}{0.564639in}}{\pgfqpoint{1.793825in}{0.570164in}}%
\pgfpathcurveto{\pgfqpoint{1.793825in}{0.575689in}}{\pgfqpoint{1.791630in}{0.580989in}}{\pgfqpoint{1.787723in}{0.584896in}}%
\pgfpathcurveto{\pgfqpoint{1.783816in}{0.588803in}}{\pgfqpoint{1.778517in}{0.590998in}}{\pgfqpoint{1.772992in}{0.590998in}}%
\pgfpathcurveto{\pgfqpoint{1.767467in}{0.590998in}}{\pgfqpoint{1.762167in}{0.588803in}}{\pgfqpoint{1.758260in}{0.584896in}}%
\pgfpathcurveto{\pgfqpoint{1.754353in}{0.580989in}}{\pgfqpoint{1.752158in}{0.575689in}}{\pgfqpoint{1.752158in}{0.570164in}}%
\pgfpathcurveto{\pgfqpoint{1.752158in}{0.564639in}}{\pgfqpoint{1.754353in}{0.559340in}}{\pgfqpoint{1.758260in}{0.555433in}}%
\pgfpathcurveto{\pgfqpoint{1.762167in}{0.551526in}}{\pgfqpoint{1.767467in}{0.549331in}}{\pgfqpoint{1.772992in}{0.549331in}}%
\pgfpathclose%
\pgfusepath{stroke,fill}%
\end{pgfscope}%
\begin{pgfscope}%
\pgfpathrectangle{\pgfqpoint{0.562500in}{0.275000in}}{\pgfqpoint{3.487500in}{1.925000in}}%
\pgfusepath{clip}%
\pgfsetbuttcap%
\pgfsetroundjoin%
\definecolor{currentfill}{rgb}{0.000000,0.000000,0.000000}%
\pgfsetfillcolor{currentfill}%
\pgfsetlinewidth{1.003750pt}%
\definecolor{currentstroke}{rgb}{0.000000,0.000000,0.000000}%
\pgfsetstrokecolor{currentstroke}%
\pgfsetdash{}{0pt}%
\pgfpathmoveto{\pgfqpoint{1.772992in}{0.549331in}}%
\pgfpathcurveto{\pgfqpoint{1.778517in}{0.549331in}}{\pgfqpoint{1.783816in}{0.551526in}}{\pgfqpoint{1.787723in}{0.555433in}}%
\pgfpathcurveto{\pgfqpoint{1.791630in}{0.559340in}}{\pgfqpoint{1.793825in}{0.564639in}}{\pgfqpoint{1.793825in}{0.570164in}}%
\pgfpathcurveto{\pgfqpoint{1.793825in}{0.575689in}}{\pgfqpoint{1.791630in}{0.580989in}}{\pgfqpoint{1.787723in}{0.584896in}}%
\pgfpathcurveto{\pgfqpoint{1.783816in}{0.588803in}}{\pgfqpoint{1.778517in}{0.590998in}}{\pgfqpoint{1.772992in}{0.590998in}}%
\pgfpathcurveto{\pgfqpoint{1.767467in}{0.590998in}}{\pgfqpoint{1.762167in}{0.588803in}}{\pgfqpoint{1.758260in}{0.584896in}}%
\pgfpathcurveto{\pgfqpoint{1.754353in}{0.580989in}}{\pgfqpoint{1.752158in}{0.575689in}}{\pgfqpoint{1.752158in}{0.570164in}}%
\pgfpathcurveto{\pgfqpoint{1.752158in}{0.564639in}}{\pgfqpoint{1.754353in}{0.559340in}}{\pgfqpoint{1.758260in}{0.555433in}}%
\pgfpathcurveto{\pgfqpoint{1.762167in}{0.551526in}}{\pgfqpoint{1.767467in}{0.549331in}}{\pgfqpoint{1.772992in}{0.549331in}}%
\pgfpathclose%
\pgfusepath{stroke,fill}%
\end{pgfscope}%
\begin{pgfscope}%
\pgfpathrectangle{\pgfqpoint{0.562500in}{0.275000in}}{\pgfqpoint{3.487500in}{1.925000in}}%
\pgfusepath{clip}%
\pgfsetbuttcap%
\pgfsetroundjoin%
\definecolor{currentfill}{rgb}{0.000000,0.000000,0.000000}%
\pgfsetfillcolor{currentfill}%
\pgfsetlinewidth{1.003750pt}%
\definecolor{currentstroke}{rgb}{0.000000,0.000000,0.000000}%
\pgfsetstrokecolor{currentstroke}%
\pgfsetdash{}{0pt}%
\pgfpathmoveto{\pgfqpoint{1.772992in}{0.660620in}}%
\pgfpathcurveto{\pgfqpoint{1.778517in}{0.660620in}}{\pgfqpoint{1.783816in}{0.662815in}}{\pgfqpoint{1.787723in}{0.666721in}}%
\pgfpathcurveto{\pgfqpoint{1.791630in}{0.670628in}}{\pgfqpoint{1.793825in}{0.675928in}}{\pgfqpoint{1.793825in}{0.681453in}}%
\pgfpathcurveto{\pgfqpoint{1.793825in}{0.686978in}}{\pgfqpoint{1.791630in}{0.692277in}}{\pgfqpoint{1.787723in}{0.696184in}}%
\pgfpathcurveto{\pgfqpoint{1.783816in}{0.700091in}}{\pgfqpoint{1.778517in}{0.702286in}}{\pgfqpoint{1.772992in}{0.702286in}}%
\pgfpathcurveto{\pgfqpoint{1.767467in}{0.702286in}}{\pgfqpoint{1.762167in}{0.700091in}}{\pgfqpoint{1.758260in}{0.696184in}}%
\pgfpathcurveto{\pgfqpoint{1.754353in}{0.692277in}}{\pgfqpoint{1.752158in}{0.686978in}}{\pgfqpoint{1.752158in}{0.681453in}}%
\pgfpathcurveto{\pgfqpoint{1.752158in}{0.675928in}}{\pgfqpoint{1.754353in}{0.670628in}}{\pgfqpoint{1.758260in}{0.666721in}}%
\pgfpathcurveto{\pgfqpoint{1.762167in}{0.662815in}}{\pgfqpoint{1.767467in}{0.660620in}}{\pgfqpoint{1.772992in}{0.660620in}}%
\pgfpathclose%
\pgfusepath{stroke,fill}%
\end{pgfscope}%
\begin{pgfscope}%
\pgfpathrectangle{\pgfqpoint{0.562500in}{0.275000in}}{\pgfqpoint{3.487500in}{1.925000in}}%
\pgfusepath{clip}%
\pgfsetbuttcap%
\pgfsetroundjoin%
\definecolor{currentfill}{rgb}{0.000000,0.000000,0.000000}%
\pgfsetfillcolor{currentfill}%
\pgfsetlinewidth{1.003750pt}%
\definecolor{currentstroke}{rgb}{0.000000,0.000000,0.000000}%
\pgfsetstrokecolor{currentstroke}%
\pgfsetdash{}{0pt}%
\pgfpathmoveto{\pgfqpoint{1.772992in}{0.549331in}}%
\pgfpathcurveto{\pgfqpoint{1.778517in}{0.549331in}}{\pgfqpoint{1.783816in}{0.551526in}}{\pgfqpoint{1.787723in}{0.555433in}}%
\pgfpathcurveto{\pgfqpoint{1.791630in}{0.559340in}}{\pgfqpoint{1.793825in}{0.564639in}}{\pgfqpoint{1.793825in}{0.570164in}}%
\pgfpathcurveto{\pgfqpoint{1.793825in}{0.575689in}}{\pgfqpoint{1.791630in}{0.580989in}}{\pgfqpoint{1.787723in}{0.584896in}}%
\pgfpathcurveto{\pgfqpoint{1.783816in}{0.588803in}}{\pgfqpoint{1.778517in}{0.590998in}}{\pgfqpoint{1.772992in}{0.590998in}}%
\pgfpathcurveto{\pgfqpoint{1.767467in}{0.590998in}}{\pgfqpoint{1.762167in}{0.588803in}}{\pgfqpoint{1.758260in}{0.584896in}}%
\pgfpathcurveto{\pgfqpoint{1.754353in}{0.580989in}}{\pgfqpoint{1.752158in}{0.575689in}}{\pgfqpoint{1.752158in}{0.570164in}}%
\pgfpathcurveto{\pgfqpoint{1.752158in}{0.564639in}}{\pgfqpoint{1.754353in}{0.559340in}}{\pgfqpoint{1.758260in}{0.555433in}}%
\pgfpathcurveto{\pgfqpoint{1.762167in}{0.551526in}}{\pgfqpoint{1.767467in}{0.549331in}}{\pgfqpoint{1.772992in}{0.549331in}}%
\pgfpathclose%
\pgfusepath{stroke,fill}%
\end{pgfscope}%
\begin{pgfscope}%
\pgfpathrectangle{\pgfqpoint{0.562500in}{0.275000in}}{\pgfqpoint{3.487500in}{1.925000in}}%
\pgfusepath{clip}%
\pgfsetbuttcap%
\pgfsetroundjoin%
\definecolor{currentfill}{rgb}{0.000000,0.000000,0.000000}%
\pgfsetfillcolor{currentfill}%
\pgfsetlinewidth{1.003750pt}%
\definecolor{currentstroke}{rgb}{0.000000,0.000000,0.000000}%
\pgfsetstrokecolor{currentstroke}%
\pgfsetdash{}{0pt}%
\pgfpathmoveto{\pgfqpoint{1.772992in}{0.539214in}}%
\pgfpathcurveto{\pgfqpoint{1.778517in}{0.539214in}}{\pgfqpoint{1.783816in}{0.541409in}}{\pgfqpoint{1.787723in}{0.545316in}}%
\pgfpathcurveto{\pgfqpoint{1.791630in}{0.549223in}}{\pgfqpoint{1.793825in}{0.554522in}}{\pgfqpoint{1.793825in}{0.560047in}}%
\pgfpathcurveto{\pgfqpoint{1.793825in}{0.565572in}}{\pgfqpoint{1.791630in}{0.570872in}}{\pgfqpoint{1.787723in}{0.574779in}}%
\pgfpathcurveto{\pgfqpoint{1.783816in}{0.578685in}}{\pgfqpoint{1.778517in}{0.580881in}}{\pgfqpoint{1.772992in}{0.580881in}}%
\pgfpathcurveto{\pgfqpoint{1.767467in}{0.580881in}}{\pgfqpoint{1.762167in}{0.578685in}}{\pgfqpoint{1.758260in}{0.574779in}}%
\pgfpathcurveto{\pgfqpoint{1.754353in}{0.570872in}}{\pgfqpoint{1.752158in}{0.565572in}}{\pgfqpoint{1.752158in}{0.560047in}}%
\pgfpathcurveto{\pgfqpoint{1.752158in}{0.554522in}}{\pgfqpoint{1.754353in}{0.549223in}}{\pgfqpoint{1.758260in}{0.545316in}}%
\pgfpathcurveto{\pgfqpoint{1.762167in}{0.541409in}}{\pgfqpoint{1.767467in}{0.539214in}}{\pgfqpoint{1.772992in}{0.539214in}}%
\pgfpathclose%
\pgfusepath{stroke,fill}%
\end{pgfscope}%
\begin{pgfscope}%
\pgfpathrectangle{\pgfqpoint{0.562500in}{0.275000in}}{\pgfqpoint{3.487500in}{1.925000in}}%
\pgfusepath{clip}%
\pgfsetbuttcap%
\pgfsetroundjoin%
\definecolor{currentfill}{rgb}{0.000000,0.000000,0.000000}%
\pgfsetfillcolor{currentfill}%
\pgfsetlinewidth{1.003750pt}%
\definecolor{currentstroke}{rgb}{0.000000,0.000000,0.000000}%
\pgfsetstrokecolor{currentstroke}%
\pgfsetdash{}{0pt}%
\pgfpathmoveto{\pgfqpoint{1.772992in}{0.529097in}}%
\pgfpathcurveto{\pgfqpoint{1.778517in}{0.529097in}}{\pgfqpoint{1.783816in}{0.531292in}}{\pgfqpoint{1.787723in}{0.535199in}}%
\pgfpathcurveto{\pgfqpoint{1.791630in}{0.539106in}}{\pgfqpoint{1.793825in}{0.544405in}}{\pgfqpoint{1.793825in}{0.549930in}}%
\pgfpathcurveto{\pgfqpoint{1.793825in}{0.555455in}}{\pgfqpoint{1.791630in}{0.560755in}}{\pgfqpoint{1.787723in}{0.564662in}}%
\pgfpathcurveto{\pgfqpoint{1.783816in}{0.568568in}}{\pgfqpoint{1.778517in}{0.570763in}}{\pgfqpoint{1.772992in}{0.570763in}}%
\pgfpathcurveto{\pgfqpoint{1.767467in}{0.570763in}}{\pgfqpoint{1.762167in}{0.568568in}}{\pgfqpoint{1.758260in}{0.564662in}}%
\pgfpathcurveto{\pgfqpoint{1.754353in}{0.560755in}}{\pgfqpoint{1.752158in}{0.555455in}}{\pgfqpoint{1.752158in}{0.549930in}}%
\pgfpathcurveto{\pgfqpoint{1.752158in}{0.544405in}}{\pgfqpoint{1.754353in}{0.539106in}}{\pgfqpoint{1.758260in}{0.535199in}}%
\pgfpathcurveto{\pgfqpoint{1.762167in}{0.531292in}}{\pgfqpoint{1.767467in}{0.529097in}}{\pgfqpoint{1.772992in}{0.529097in}}%
\pgfpathclose%
\pgfusepath{stroke,fill}%
\end{pgfscope}%
\begin{pgfscope}%
\pgfpathrectangle{\pgfqpoint{0.562500in}{0.275000in}}{\pgfqpoint{3.487500in}{1.925000in}}%
\pgfusepath{clip}%
\pgfsetbuttcap%
\pgfsetroundjoin%
\definecolor{currentfill}{rgb}{0.000000,0.000000,0.000000}%
\pgfsetfillcolor{currentfill}%
\pgfsetlinewidth{1.003750pt}%
\definecolor{currentstroke}{rgb}{0.000000,0.000000,0.000000}%
\pgfsetstrokecolor{currentstroke}%
\pgfsetdash{}{0pt}%
\pgfpathmoveto{\pgfqpoint{1.772992in}{0.508863in}}%
\pgfpathcurveto{\pgfqpoint{1.778517in}{0.508863in}}{\pgfqpoint{1.783816in}{0.511058in}}{\pgfqpoint{1.787723in}{0.514964in}}%
\pgfpathcurveto{\pgfqpoint{1.791630in}{0.518871in}}{\pgfqpoint{1.793825in}{0.524171in}}{\pgfqpoint{1.793825in}{0.529696in}}%
\pgfpathcurveto{\pgfqpoint{1.793825in}{0.535221in}}{\pgfqpoint{1.791630in}{0.540520in}}{\pgfqpoint{1.787723in}{0.544427in}}%
\pgfpathcurveto{\pgfqpoint{1.783816in}{0.548334in}}{\pgfqpoint{1.778517in}{0.550529in}}{\pgfqpoint{1.772992in}{0.550529in}}%
\pgfpathcurveto{\pgfqpoint{1.767467in}{0.550529in}}{\pgfqpoint{1.762167in}{0.548334in}}{\pgfqpoint{1.758260in}{0.544427in}}%
\pgfpathcurveto{\pgfqpoint{1.754353in}{0.540520in}}{\pgfqpoint{1.752158in}{0.535221in}}{\pgfqpoint{1.752158in}{0.529696in}}%
\pgfpathcurveto{\pgfqpoint{1.752158in}{0.524171in}}{\pgfqpoint{1.754353in}{0.518871in}}{\pgfqpoint{1.758260in}{0.514964in}}%
\pgfpathcurveto{\pgfqpoint{1.762167in}{0.511058in}}{\pgfqpoint{1.767467in}{0.508863in}}{\pgfqpoint{1.772992in}{0.508863in}}%
\pgfpathclose%
\pgfusepath{stroke,fill}%
\end{pgfscope}%
\begin{pgfscope}%
\pgfpathrectangle{\pgfqpoint{0.562500in}{0.275000in}}{\pgfqpoint{3.487500in}{1.925000in}}%
\pgfusepath{clip}%
\pgfsetbuttcap%
\pgfsetroundjoin%
\definecolor{currentfill}{rgb}{0.000000,0.000000,0.000000}%
\pgfsetfillcolor{currentfill}%
\pgfsetlinewidth{1.003750pt}%
\definecolor{currentstroke}{rgb}{0.000000,0.000000,0.000000}%
\pgfsetstrokecolor{currentstroke}%
\pgfsetdash{}{0pt}%
\pgfpathmoveto{\pgfqpoint{1.772992in}{0.513921in}}%
\pgfpathcurveto{\pgfqpoint{1.778517in}{0.513921in}}{\pgfqpoint{1.783816in}{0.516116in}}{\pgfqpoint{1.787723in}{0.520023in}}%
\pgfpathcurveto{\pgfqpoint{1.791630in}{0.523930in}}{\pgfqpoint{1.793825in}{0.529229in}}{\pgfqpoint{1.793825in}{0.534754in}}%
\pgfpathcurveto{\pgfqpoint{1.793825in}{0.540279in}}{\pgfqpoint{1.791630in}{0.545579in}}{\pgfqpoint{1.787723in}{0.549486in}}%
\pgfpathcurveto{\pgfqpoint{1.783816in}{0.553393in}}{\pgfqpoint{1.778517in}{0.555588in}}{\pgfqpoint{1.772992in}{0.555588in}}%
\pgfpathcurveto{\pgfqpoint{1.767467in}{0.555588in}}{\pgfqpoint{1.762167in}{0.553393in}}{\pgfqpoint{1.758260in}{0.549486in}}%
\pgfpathcurveto{\pgfqpoint{1.754353in}{0.545579in}}{\pgfqpoint{1.752158in}{0.540279in}}{\pgfqpoint{1.752158in}{0.534754in}}%
\pgfpathcurveto{\pgfqpoint{1.752158in}{0.529229in}}{\pgfqpoint{1.754353in}{0.523930in}}{\pgfqpoint{1.758260in}{0.520023in}}%
\pgfpathcurveto{\pgfqpoint{1.762167in}{0.516116in}}{\pgfqpoint{1.767467in}{0.513921in}}{\pgfqpoint{1.772992in}{0.513921in}}%
\pgfpathclose%
\pgfusepath{stroke,fill}%
\end{pgfscope}%
\begin{pgfscope}%
\pgfpathrectangle{\pgfqpoint{0.562500in}{0.275000in}}{\pgfqpoint{3.487500in}{1.925000in}}%
\pgfusepath{clip}%
\pgfsetbuttcap%
\pgfsetroundjoin%
\definecolor{currentfill}{rgb}{0.000000,0.000000,0.000000}%
\pgfsetfillcolor{currentfill}%
\pgfsetlinewidth{1.003750pt}%
\definecolor{currentstroke}{rgb}{0.000000,0.000000,0.000000}%
\pgfsetstrokecolor{currentstroke}%
\pgfsetdash{}{0pt}%
\pgfpathmoveto{\pgfqpoint{1.772992in}{0.503804in}}%
\pgfpathcurveto{\pgfqpoint{1.778517in}{0.503804in}}{\pgfqpoint{1.783816in}{0.505999in}}{\pgfqpoint{1.787723in}{0.509906in}}%
\pgfpathcurveto{\pgfqpoint{1.791630in}{0.513813in}}{\pgfqpoint{1.793825in}{0.519112in}}{\pgfqpoint{1.793825in}{0.524637in}}%
\pgfpathcurveto{\pgfqpoint{1.793825in}{0.530162in}}{\pgfqpoint{1.791630in}{0.535462in}}{\pgfqpoint{1.787723in}{0.539369in}}%
\pgfpathcurveto{\pgfqpoint{1.783816in}{0.543275in}}{\pgfqpoint{1.778517in}{0.545471in}}{\pgfqpoint{1.772992in}{0.545471in}}%
\pgfpathcurveto{\pgfqpoint{1.767467in}{0.545471in}}{\pgfqpoint{1.762167in}{0.543275in}}{\pgfqpoint{1.758260in}{0.539369in}}%
\pgfpathcurveto{\pgfqpoint{1.754353in}{0.535462in}}{\pgfqpoint{1.752158in}{0.530162in}}{\pgfqpoint{1.752158in}{0.524637in}}%
\pgfpathcurveto{\pgfqpoint{1.752158in}{0.519112in}}{\pgfqpoint{1.754353in}{0.513813in}}{\pgfqpoint{1.758260in}{0.509906in}}%
\pgfpathcurveto{\pgfqpoint{1.762167in}{0.505999in}}{\pgfqpoint{1.767467in}{0.503804in}}{\pgfqpoint{1.772992in}{0.503804in}}%
\pgfpathclose%
\pgfusepath{stroke,fill}%
\end{pgfscope}%
\begin{pgfscope}%
\pgfpathrectangle{\pgfqpoint{0.562500in}{0.275000in}}{\pgfqpoint{3.487500in}{1.925000in}}%
\pgfusepath{clip}%
\pgfsetbuttcap%
\pgfsetroundjoin%
\definecolor{currentfill}{rgb}{0.000000,0.000000,0.000000}%
\pgfsetfillcolor{currentfill}%
\pgfsetlinewidth{1.003750pt}%
\definecolor{currentstroke}{rgb}{0.000000,0.000000,0.000000}%
\pgfsetstrokecolor{currentstroke}%
\pgfsetdash{}{0pt}%
\pgfpathmoveto{\pgfqpoint{1.772992in}{0.539214in}}%
\pgfpathcurveto{\pgfqpoint{1.778517in}{0.539214in}}{\pgfqpoint{1.783816in}{0.541409in}}{\pgfqpoint{1.787723in}{0.545316in}}%
\pgfpathcurveto{\pgfqpoint{1.791630in}{0.549223in}}{\pgfqpoint{1.793825in}{0.554522in}}{\pgfqpoint{1.793825in}{0.560047in}}%
\pgfpathcurveto{\pgfqpoint{1.793825in}{0.565572in}}{\pgfqpoint{1.791630in}{0.570872in}}{\pgfqpoint{1.787723in}{0.574779in}}%
\pgfpathcurveto{\pgfqpoint{1.783816in}{0.578685in}}{\pgfqpoint{1.778517in}{0.580881in}}{\pgfqpoint{1.772992in}{0.580881in}}%
\pgfpathcurveto{\pgfqpoint{1.767467in}{0.580881in}}{\pgfqpoint{1.762167in}{0.578685in}}{\pgfqpoint{1.758260in}{0.574779in}}%
\pgfpathcurveto{\pgfqpoint{1.754353in}{0.570872in}}{\pgfqpoint{1.752158in}{0.565572in}}{\pgfqpoint{1.752158in}{0.560047in}}%
\pgfpathcurveto{\pgfqpoint{1.752158in}{0.554522in}}{\pgfqpoint{1.754353in}{0.549223in}}{\pgfqpoint{1.758260in}{0.545316in}}%
\pgfpathcurveto{\pgfqpoint{1.762167in}{0.541409in}}{\pgfqpoint{1.767467in}{0.539214in}}{\pgfqpoint{1.772992in}{0.539214in}}%
\pgfpathclose%
\pgfusepath{stroke,fill}%
\end{pgfscope}%
\begin{pgfscope}%
\pgfpathrectangle{\pgfqpoint{0.562500in}{0.275000in}}{\pgfqpoint{3.487500in}{1.925000in}}%
\pgfusepath{clip}%
\pgfsetbuttcap%
\pgfsetroundjoin%
\definecolor{currentfill}{rgb}{0.000000,0.000000,0.000000}%
\pgfsetfillcolor{currentfill}%
\pgfsetlinewidth{1.003750pt}%
\definecolor{currentstroke}{rgb}{0.000000,0.000000,0.000000}%
\pgfsetstrokecolor{currentstroke}%
\pgfsetdash{}{0pt}%
\pgfpathmoveto{\pgfqpoint{1.772992in}{0.574624in}}%
\pgfpathcurveto{\pgfqpoint{1.778517in}{0.574624in}}{\pgfqpoint{1.783816in}{0.576819in}}{\pgfqpoint{1.787723in}{0.580726in}}%
\pgfpathcurveto{\pgfqpoint{1.791630in}{0.584633in}}{\pgfqpoint{1.793825in}{0.589932in}}{\pgfqpoint{1.793825in}{0.595457in}}%
\pgfpathcurveto{\pgfqpoint{1.793825in}{0.600982in}}{\pgfqpoint{1.791630in}{0.606282in}}{\pgfqpoint{1.787723in}{0.610189in}}%
\pgfpathcurveto{\pgfqpoint{1.783816in}{0.614095in}}{\pgfqpoint{1.778517in}{0.616291in}}{\pgfqpoint{1.772992in}{0.616291in}}%
\pgfpathcurveto{\pgfqpoint{1.767467in}{0.616291in}}{\pgfqpoint{1.762167in}{0.614095in}}{\pgfqpoint{1.758260in}{0.610189in}}%
\pgfpathcurveto{\pgfqpoint{1.754353in}{0.606282in}}{\pgfqpoint{1.752158in}{0.600982in}}{\pgfqpoint{1.752158in}{0.595457in}}%
\pgfpathcurveto{\pgfqpoint{1.752158in}{0.589932in}}{\pgfqpoint{1.754353in}{0.584633in}}{\pgfqpoint{1.758260in}{0.580726in}}%
\pgfpathcurveto{\pgfqpoint{1.762167in}{0.576819in}}{\pgfqpoint{1.767467in}{0.574624in}}{\pgfqpoint{1.772992in}{0.574624in}}%
\pgfpathclose%
\pgfusepath{stroke,fill}%
\end{pgfscope}%
\begin{pgfscope}%
\pgfpathrectangle{\pgfqpoint{0.562500in}{0.275000in}}{\pgfqpoint{3.487500in}{1.925000in}}%
\pgfusepath{clip}%
\pgfsetbuttcap%
\pgfsetroundjoin%
\definecolor{currentfill}{rgb}{0.000000,0.000000,0.000000}%
\pgfsetfillcolor{currentfill}%
\pgfsetlinewidth{1.003750pt}%
\definecolor{currentstroke}{rgb}{0.000000,0.000000,0.000000}%
\pgfsetstrokecolor{currentstroke}%
\pgfsetdash{}{0pt}%
\pgfpathmoveto{\pgfqpoint{1.772992in}{0.599917in}}%
\pgfpathcurveto{\pgfqpoint{1.778517in}{0.599917in}}{\pgfqpoint{1.783816in}{0.602112in}}{\pgfqpoint{1.787723in}{0.606019in}}%
\pgfpathcurveto{\pgfqpoint{1.791630in}{0.609925in}}{\pgfqpoint{1.793825in}{0.615225in}}{\pgfqpoint{1.793825in}{0.620750in}}%
\pgfpathcurveto{\pgfqpoint{1.793825in}{0.626275in}}{\pgfqpoint{1.791630in}{0.631575in}}{\pgfqpoint{1.787723in}{0.635481in}}%
\pgfpathcurveto{\pgfqpoint{1.783816in}{0.639388in}}{\pgfqpoint{1.778517in}{0.641583in}}{\pgfqpoint{1.772992in}{0.641583in}}%
\pgfpathcurveto{\pgfqpoint{1.767467in}{0.641583in}}{\pgfqpoint{1.762167in}{0.639388in}}{\pgfqpoint{1.758260in}{0.635481in}}%
\pgfpathcurveto{\pgfqpoint{1.754353in}{0.631575in}}{\pgfqpoint{1.752158in}{0.626275in}}{\pgfqpoint{1.752158in}{0.620750in}}%
\pgfpathcurveto{\pgfqpoint{1.752158in}{0.615225in}}{\pgfqpoint{1.754353in}{0.609925in}}{\pgfqpoint{1.758260in}{0.606019in}}%
\pgfpathcurveto{\pgfqpoint{1.762167in}{0.602112in}}{\pgfqpoint{1.767467in}{0.599917in}}{\pgfqpoint{1.772992in}{0.599917in}}%
\pgfpathclose%
\pgfusepath{stroke,fill}%
\end{pgfscope}%
\begin{pgfscope}%
\pgfpathrectangle{\pgfqpoint{0.562500in}{0.275000in}}{\pgfqpoint{3.487500in}{1.925000in}}%
\pgfusepath{clip}%
\pgfsetbuttcap%
\pgfsetroundjoin%
\definecolor{currentfill}{rgb}{0.000000,0.000000,0.000000}%
\pgfsetfillcolor{currentfill}%
\pgfsetlinewidth{1.003750pt}%
\definecolor{currentstroke}{rgb}{0.000000,0.000000,0.000000}%
\pgfsetstrokecolor{currentstroke}%
\pgfsetdash{}{0pt}%
\pgfpathmoveto{\pgfqpoint{1.772992in}{0.544272in}}%
\pgfpathcurveto{\pgfqpoint{1.778517in}{0.544272in}}{\pgfqpoint{1.783816in}{0.546468in}}{\pgfqpoint{1.787723in}{0.550374in}}%
\pgfpathcurveto{\pgfqpoint{1.791630in}{0.554281in}}{\pgfqpoint{1.793825in}{0.559581in}}{\pgfqpoint{1.793825in}{0.565106in}}%
\pgfpathcurveto{\pgfqpoint{1.793825in}{0.570631in}}{\pgfqpoint{1.791630in}{0.575930in}}{\pgfqpoint{1.787723in}{0.579837in}}%
\pgfpathcurveto{\pgfqpoint{1.783816in}{0.583744in}}{\pgfqpoint{1.778517in}{0.585939in}}{\pgfqpoint{1.772992in}{0.585939in}}%
\pgfpathcurveto{\pgfqpoint{1.767467in}{0.585939in}}{\pgfqpoint{1.762167in}{0.583744in}}{\pgfqpoint{1.758260in}{0.579837in}}%
\pgfpathcurveto{\pgfqpoint{1.754353in}{0.575930in}}{\pgfqpoint{1.752158in}{0.570631in}}{\pgfqpoint{1.752158in}{0.565106in}}%
\pgfpathcurveto{\pgfqpoint{1.752158in}{0.559581in}}{\pgfqpoint{1.754353in}{0.554281in}}{\pgfqpoint{1.758260in}{0.550374in}}%
\pgfpathcurveto{\pgfqpoint{1.762167in}{0.546468in}}{\pgfqpoint{1.767467in}{0.544272in}}{\pgfqpoint{1.772992in}{0.544272in}}%
\pgfpathclose%
\pgfusepath{stroke,fill}%
\end{pgfscope}%
\begin{pgfscope}%
\pgfpathrectangle{\pgfqpoint{0.562500in}{0.275000in}}{\pgfqpoint{3.487500in}{1.925000in}}%
\pgfusepath{clip}%
\pgfsetbuttcap%
\pgfsetroundjoin%
\definecolor{currentfill}{rgb}{0.000000,0.000000,0.000000}%
\pgfsetfillcolor{currentfill}%
\pgfsetlinewidth{1.003750pt}%
\definecolor{currentstroke}{rgb}{0.000000,0.000000,0.000000}%
\pgfsetstrokecolor{currentstroke}%
\pgfsetdash{}{0pt}%
\pgfpathmoveto{\pgfqpoint{1.772992in}{0.524038in}}%
\pgfpathcurveto{\pgfqpoint{1.778517in}{0.524038in}}{\pgfqpoint{1.783816in}{0.526233in}}{\pgfqpoint{1.787723in}{0.530140in}}%
\pgfpathcurveto{\pgfqpoint{1.791630in}{0.534047in}}{\pgfqpoint{1.793825in}{0.539346in}}{\pgfqpoint{1.793825in}{0.544872in}}%
\pgfpathcurveto{\pgfqpoint{1.793825in}{0.550397in}}{\pgfqpoint{1.791630in}{0.555696in}}{\pgfqpoint{1.787723in}{0.559603in}}%
\pgfpathcurveto{\pgfqpoint{1.783816in}{0.563510in}}{\pgfqpoint{1.778517in}{0.565705in}}{\pgfqpoint{1.772992in}{0.565705in}}%
\pgfpathcurveto{\pgfqpoint{1.767467in}{0.565705in}}{\pgfqpoint{1.762167in}{0.563510in}}{\pgfqpoint{1.758260in}{0.559603in}}%
\pgfpathcurveto{\pgfqpoint{1.754353in}{0.555696in}}{\pgfqpoint{1.752158in}{0.550397in}}{\pgfqpoint{1.752158in}{0.544872in}}%
\pgfpathcurveto{\pgfqpoint{1.752158in}{0.539346in}}{\pgfqpoint{1.754353in}{0.534047in}}{\pgfqpoint{1.758260in}{0.530140in}}%
\pgfpathcurveto{\pgfqpoint{1.762167in}{0.526233in}}{\pgfqpoint{1.767467in}{0.524038in}}{\pgfqpoint{1.772992in}{0.524038in}}%
\pgfpathclose%
\pgfusepath{stroke,fill}%
\end{pgfscope}%
\begin{pgfscope}%
\pgfpathrectangle{\pgfqpoint{0.562500in}{0.275000in}}{\pgfqpoint{3.487500in}{1.925000in}}%
\pgfusepath{clip}%
\pgfsetbuttcap%
\pgfsetroundjoin%
\definecolor{currentfill}{rgb}{0.000000,0.000000,0.000000}%
\pgfsetfillcolor{currentfill}%
\pgfsetlinewidth{1.003750pt}%
\definecolor{currentstroke}{rgb}{0.000000,0.000000,0.000000}%
\pgfsetstrokecolor{currentstroke}%
\pgfsetdash{}{0pt}%
\pgfpathmoveto{\pgfqpoint{1.772992in}{0.569565in}}%
\pgfpathcurveto{\pgfqpoint{1.778517in}{0.569565in}}{\pgfqpoint{1.783816in}{0.571760in}}{\pgfqpoint{1.787723in}{0.575667in}}%
\pgfpathcurveto{\pgfqpoint{1.791630in}{0.579574in}}{\pgfqpoint{1.793825in}{0.584874in}}{\pgfqpoint{1.793825in}{0.590399in}}%
\pgfpathcurveto{\pgfqpoint{1.793825in}{0.595924in}}{\pgfqpoint{1.791630in}{0.601223in}}{\pgfqpoint{1.787723in}{0.605130in}}%
\pgfpathcurveto{\pgfqpoint{1.783816in}{0.609037in}}{\pgfqpoint{1.778517in}{0.611232in}}{\pgfqpoint{1.772992in}{0.611232in}}%
\pgfpathcurveto{\pgfqpoint{1.767467in}{0.611232in}}{\pgfqpoint{1.762167in}{0.609037in}}{\pgfqpoint{1.758260in}{0.605130in}}%
\pgfpathcurveto{\pgfqpoint{1.754353in}{0.601223in}}{\pgfqpoint{1.752158in}{0.595924in}}{\pgfqpoint{1.752158in}{0.590399in}}%
\pgfpathcurveto{\pgfqpoint{1.752158in}{0.584874in}}{\pgfqpoint{1.754353in}{0.579574in}}{\pgfqpoint{1.758260in}{0.575667in}}%
\pgfpathcurveto{\pgfqpoint{1.762167in}{0.571760in}}{\pgfqpoint{1.767467in}{0.569565in}}{\pgfqpoint{1.772992in}{0.569565in}}%
\pgfpathclose%
\pgfusepath{stroke,fill}%
\end{pgfscope}%
\begin{pgfscope}%
\pgfpathrectangle{\pgfqpoint{0.562500in}{0.275000in}}{\pgfqpoint{3.487500in}{1.925000in}}%
\pgfusepath{clip}%
\pgfsetbuttcap%
\pgfsetroundjoin%
\definecolor{currentfill}{rgb}{0.000000,0.000000,0.000000}%
\pgfsetfillcolor{currentfill}%
\pgfsetlinewidth{1.003750pt}%
\definecolor{currentstroke}{rgb}{0.000000,0.000000,0.000000}%
\pgfsetstrokecolor{currentstroke}%
\pgfsetdash{}{0pt}%
\pgfpathmoveto{\pgfqpoint{1.772992in}{0.579682in}}%
\pgfpathcurveto{\pgfqpoint{1.778517in}{0.579682in}}{\pgfqpoint{1.783816in}{0.581878in}}{\pgfqpoint{1.787723in}{0.585784in}}%
\pgfpathcurveto{\pgfqpoint{1.791630in}{0.589691in}}{\pgfqpoint{1.793825in}{0.594991in}}{\pgfqpoint{1.793825in}{0.600516in}}%
\pgfpathcurveto{\pgfqpoint{1.793825in}{0.606041in}}{\pgfqpoint{1.791630in}{0.611340in}}{\pgfqpoint{1.787723in}{0.615247in}}%
\pgfpathcurveto{\pgfqpoint{1.783816in}{0.619154in}}{\pgfqpoint{1.778517in}{0.621349in}}{\pgfqpoint{1.772992in}{0.621349in}}%
\pgfpathcurveto{\pgfqpoint{1.767467in}{0.621349in}}{\pgfqpoint{1.762167in}{0.619154in}}{\pgfqpoint{1.758260in}{0.615247in}}%
\pgfpathcurveto{\pgfqpoint{1.754353in}{0.611340in}}{\pgfqpoint{1.752158in}{0.606041in}}{\pgfqpoint{1.752158in}{0.600516in}}%
\pgfpathcurveto{\pgfqpoint{1.752158in}{0.594991in}}{\pgfqpoint{1.754353in}{0.589691in}}{\pgfqpoint{1.758260in}{0.585784in}}%
\pgfpathcurveto{\pgfqpoint{1.762167in}{0.581878in}}{\pgfqpoint{1.767467in}{0.579682in}}{\pgfqpoint{1.772992in}{0.579682in}}%
\pgfpathclose%
\pgfusepath{stroke,fill}%
\end{pgfscope}%
\begin{pgfscope}%
\pgfpathrectangle{\pgfqpoint{0.562500in}{0.275000in}}{\pgfqpoint{3.487500in}{1.925000in}}%
\pgfusepath{clip}%
\pgfsetbuttcap%
\pgfsetroundjoin%
\definecolor{currentfill}{rgb}{0.000000,0.000000,0.000000}%
\pgfsetfillcolor{currentfill}%
\pgfsetlinewidth{1.003750pt}%
\definecolor{currentstroke}{rgb}{0.000000,0.000000,0.000000}%
\pgfsetstrokecolor{currentstroke}%
\pgfsetdash{}{0pt}%
\pgfpathmoveto{\pgfqpoint{1.772992in}{0.524038in}}%
\pgfpathcurveto{\pgfqpoint{1.778517in}{0.524038in}}{\pgfqpoint{1.783816in}{0.526233in}}{\pgfqpoint{1.787723in}{0.530140in}}%
\pgfpathcurveto{\pgfqpoint{1.791630in}{0.534047in}}{\pgfqpoint{1.793825in}{0.539346in}}{\pgfqpoint{1.793825in}{0.544872in}}%
\pgfpathcurveto{\pgfqpoint{1.793825in}{0.550397in}}{\pgfqpoint{1.791630in}{0.555696in}}{\pgfqpoint{1.787723in}{0.559603in}}%
\pgfpathcurveto{\pgfqpoint{1.783816in}{0.563510in}}{\pgfqpoint{1.778517in}{0.565705in}}{\pgfqpoint{1.772992in}{0.565705in}}%
\pgfpathcurveto{\pgfqpoint{1.767467in}{0.565705in}}{\pgfqpoint{1.762167in}{0.563510in}}{\pgfqpoint{1.758260in}{0.559603in}}%
\pgfpathcurveto{\pgfqpoint{1.754353in}{0.555696in}}{\pgfqpoint{1.752158in}{0.550397in}}{\pgfqpoint{1.752158in}{0.544872in}}%
\pgfpathcurveto{\pgfqpoint{1.752158in}{0.539346in}}{\pgfqpoint{1.754353in}{0.534047in}}{\pgfqpoint{1.758260in}{0.530140in}}%
\pgfpathcurveto{\pgfqpoint{1.762167in}{0.526233in}}{\pgfqpoint{1.767467in}{0.524038in}}{\pgfqpoint{1.772992in}{0.524038in}}%
\pgfpathclose%
\pgfusepath{stroke,fill}%
\end{pgfscope}%
\begin{pgfscope}%
\pgfpathrectangle{\pgfqpoint{0.562500in}{0.275000in}}{\pgfqpoint{3.487500in}{1.925000in}}%
\pgfusepath{clip}%
\pgfsetbuttcap%
\pgfsetroundjoin%
\definecolor{currentfill}{rgb}{0.000000,0.000000,0.000000}%
\pgfsetfillcolor{currentfill}%
\pgfsetlinewidth{1.003750pt}%
\definecolor{currentstroke}{rgb}{0.000000,0.000000,0.000000}%
\pgfsetstrokecolor{currentstroke}%
\pgfsetdash{}{0pt}%
\pgfpathmoveto{\pgfqpoint{1.772992in}{0.493687in}}%
\pgfpathcurveto{\pgfqpoint{1.778517in}{0.493687in}}{\pgfqpoint{1.783816in}{0.495882in}}{\pgfqpoint{1.787723in}{0.499789in}}%
\pgfpathcurveto{\pgfqpoint{1.791630in}{0.503696in}}{\pgfqpoint{1.793825in}{0.508995in}}{\pgfqpoint{1.793825in}{0.514520in}}%
\pgfpathcurveto{\pgfqpoint{1.793825in}{0.520045in}}{\pgfqpoint{1.791630in}{0.525345in}}{\pgfqpoint{1.787723in}{0.529252in}}%
\pgfpathcurveto{\pgfqpoint{1.783816in}{0.533158in}}{\pgfqpoint{1.778517in}{0.535353in}}{\pgfqpoint{1.772992in}{0.535353in}}%
\pgfpathcurveto{\pgfqpoint{1.767467in}{0.535353in}}{\pgfqpoint{1.762167in}{0.533158in}}{\pgfqpoint{1.758260in}{0.529252in}}%
\pgfpathcurveto{\pgfqpoint{1.754353in}{0.525345in}}{\pgfqpoint{1.752158in}{0.520045in}}{\pgfqpoint{1.752158in}{0.514520in}}%
\pgfpathcurveto{\pgfqpoint{1.752158in}{0.508995in}}{\pgfqpoint{1.754353in}{0.503696in}}{\pgfqpoint{1.758260in}{0.499789in}}%
\pgfpathcurveto{\pgfqpoint{1.762167in}{0.495882in}}{\pgfqpoint{1.767467in}{0.493687in}}{\pgfqpoint{1.772992in}{0.493687in}}%
\pgfpathclose%
\pgfusepath{stroke,fill}%
\end{pgfscope}%
\begin{pgfscope}%
\pgfpathrectangle{\pgfqpoint{0.562500in}{0.275000in}}{\pgfqpoint{3.487500in}{1.925000in}}%
\pgfusepath{clip}%
\pgfsetbuttcap%
\pgfsetroundjoin%
\definecolor{currentfill}{rgb}{0.000000,0.000000,0.000000}%
\pgfsetfillcolor{currentfill}%
\pgfsetlinewidth{1.003750pt}%
\definecolor{currentstroke}{rgb}{0.000000,0.000000,0.000000}%
\pgfsetstrokecolor{currentstroke}%
\pgfsetdash{}{0pt}%
\pgfpathmoveto{\pgfqpoint{1.772992in}{0.493687in}}%
\pgfpathcurveto{\pgfqpoint{1.778517in}{0.493687in}}{\pgfqpoint{1.783816in}{0.495882in}}{\pgfqpoint{1.787723in}{0.499789in}}%
\pgfpathcurveto{\pgfqpoint{1.791630in}{0.503696in}}{\pgfqpoint{1.793825in}{0.508995in}}{\pgfqpoint{1.793825in}{0.514520in}}%
\pgfpathcurveto{\pgfqpoint{1.793825in}{0.520045in}}{\pgfqpoint{1.791630in}{0.525345in}}{\pgfqpoint{1.787723in}{0.529252in}}%
\pgfpathcurveto{\pgfqpoint{1.783816in}{0.533158in}}{\pgfqpoint{1.778517in}{0.535353in}}{\pgfqpoint{1.772992in}{0.535353in}}%
\pgfpathcurveto{\pgfqpoint{1.767467in}{0.535353in}}{\pgfqpoint{1.762167in}{0.533158in}}{\pgfqpoint{1.758260in}{0.529252in}}%
\pgfpathcurveto{\pgfqpoint{1.754353in}{0.525345in}}{\pgfqpoint{1.752158in}{0.520045in}}{\pgfqpoint{1.752158in}{0.514520in}}%
\pgfpathcurveto{\pgfqpoint{1.752158in}{0.508995in}}{\pgfqpoint{1.754353in}{0.503696in}}{\pgfqpoint{1.758260in}{0.499789in}}%
\pgfpathcurveto{\pgfqpoint{1.762167in}{0.495882in}}{\pgfqpoint{1.767467in}{0.493687in}}{\pgfqpoint{1.772992in}{0.493687in}}%
\pgfpathclose%
\pgfusepath{stroke,fill}%
\end{pgfscope}%
\begin{pgfscope}%
\pgfpathrectangle{\pgfqpoint{0.562500in}{0.275000in}}{\pgfqpoint{3.487500in}{1.925000in}}%
\pgfusepath{clip}%
\pgfsetbuttcap%
\pgfsetroundjoin%
\definecolor{currentfill}{rgb}{0.000000,0.000000,0.000000}%
\pgfsetfillcolor{currentfill}%
\pgfsetlinewidth{1.003750pt}%
\definecolor{currentstroke}{rgb}{0.000000,0.000000,0.000000}%
\pgfsetstrokecolor{currentstroke}%
\pgfsetdash{}{0pt}%
\pgfpathmoveto{\pgfqpoint{1.772992in}{0.539214in}}%
\pgfpathcurveto{\pgfqpoint{1.778517in}{0.539214in}}{\pgfqpoint{1.783816in}{0.541409in}}{\pgfqpoint{1.787723in}{0.545316in}}%
\pgfpathcurveto{\pgfqpoint{1.791630in}{0.549223in}}{\pgfqpoint{1.793825in}{0.554522in}}{\pgfqpoint{1.793825in}{0.560047in}}%
\pgfpathcurveto{\pgfqpoint{1.793825in}{0.565572in}}{\pgfqpoint{1.791630in}{0.570872in}}{\pgfqpoint{1.787723in}{0.574779in}}%
\pgfpathcurveto{\pgfqpoint{1.783816in}{0.578685in}}{\pgfqpoint{1.778517in}{0.580881in}}{\pgfqpoint{1.772992in}{0.580881in}}%
\pgfpathcurveto{\pgfqpoint{1.767467in}{0.580881in}}{\pgfqpoint{1.762167in}{0.578685in}}{\pgfqpoint{1.758260in}{0.574779in}}%
\pgfpathcurveto{\pgfqpoint{1.754353in}{0.570872in}}{\pgfqpoint{1.752158in}{0.565572in}}{\pgfqpoint{1.752158in}{0.560047in}}%
\pgfpathcurveto{\pgfqpoint{1.752158in}{0.554522in}}{\pgfqpoint{1.754353in}{0.549223in}}{\pgfqpoint{1.758260in}{0.545316in}}%
\pgfpathcurveto{\pgfqpoint{1.762167in}{0.541409in}}{\pgfqpoint{1.767467in}{0.539214in}}{\pgfqpoint{1.772992in}{0.539214in}}%
\pgfpathclose%
\pgfusepath{stroke,fill}%
\end{pgfscope}%
\begin{pgfscope}%
\pgfpathrectangle{\pgfqpoint{0.562500in}{0.275000in}}{\pgfqpoint{3.487500in}{1.925000in}}%
\pgfusepath{clip}%
\pgfsetbuttcap%
\pgfsetroundjoin%
\definecolor{currentfill}{rgb}{0.000000,0.000000,0.000000}%
\pgfsetfillcolor{currentfill}%
\pgfsetlinewidth{1.003750pt}%
\definecolor{currentstroke}{rgb}{0.000000,0.000000,0.000000}%
\pgfsetstrokecolor{currentstroke}%
\pgfsetdash{}{0pt}%
\pgfpathmoveto{\pgfqpoint{1.772992in}{0.569565in}}%
\pgfpathcurveto{\pgfqpoint{1.778517in}{0.569565in}}{\pgfqpoint{1.783816in}{0.571760in}}{\pgfqpoint{1.787723in}{0.575667in}}%
\pgfpathcurveto{\pgfqpoint{1.791630in}{0.579574in}}{\pgfqpoint{1.793825in}{0.584874in}}{\pgfqpoint{1.793825in}{0.590399in}}%
\pgfpathcurveto{\pgfqpoint{1.793825in}{0.595924in}}{\pgfqpoint{1.791630in}{0.601223in}}{\pgfqpoint{1.787723in}{0.605130in}}%
\pgfpathcurveto{\pgfqpoint{1.783816in}{0.609037in}}{\pgfqpoint{1.778517in}{0.611232in}}{\pgfqpoint{1.772992in}{0.611232in}}%
\pgfpathcurveto{\pgfqpoint{1.767467in}{0.611232in}}{\pgfqpoint{1.762167in}{0.609037in}}{\pgfqpoint{1.758260in}{0.605130in}}%
\pgfpathcurveto{\pgfqpoint{1.754353in}{0.601223in}}{\pgfqpoint{1.752158in}{0.595924in}}{\pgfqpoint{1.752158in}{0.590399in}}%
\pgfpathcurveto{\pgfqpoint{1.752158in}{0.584874in}}{\pgfqpoint{1.754353in}{0.579574in}}{\pgfqpoint{1.758260in}{0.575667in}}%
\pgfpathcurveto{\pgfqpoint{1.762167in}{0.571760in}}{\pgfqpoint{1.767467in}{0.569565in}}{\pgfqpoint{1.772992in}{0.569565in}}%
\pgfpathclose%
\pgfusepath{stroke,fill}%
\end{pgfscope}%
\begin{pgfscope}%
\pgfpathrectangle{\pgfqpoint{0.562500in}{0.275000in}}{\pgfqpoint{3.487500in}{1.925000in}}%
\pgfusepath{clip}%
\pgfsetbuttcap%
\pgfsetroundjoin%
\definecolor{currentfill}{rgb}{0.000000,0.000000,0.000000}%
\pgfsetfillcolor{currentfill}%
\pgfsetlinewidth{1.003750pt}%
\definecolor{currentstroke}{rgb}{0.000000,0.000000,0.000000}%
\pgfsetstrokecolor{currentstroke}%
\pgfsetdash{}{0pt}%
\pgfpathmoveto{\pgfqpoint{1.772992in}{0.584741in}}%
\pgfpathcurveto{\pgfqpoint{1.778517in}{0.584741in}}{\pgfqpoint{1.783816in}{0.586936in}}{\pgfqpoint{1.787723in}{0.590843in}}%
\pgfpathcurveto{\pgfqpoint{1.791630in}{0.594750in}}{\pgfqpoint{1.793825in}{0.600049in}}{\pgfqpoint{1.793825in}{0.605574in}}%
\pgfpathcurveto{\pgfqpoint{1.793825in}{0.611099in}}{\pgfqpoint{1.791630in}{0.616399in}}{\pgfqpoint{1.787723in}{0.620306in}}%
\pgfpathcurveto{\pgfqpoint{1.783816in}{0.624213in}}{\pgfqpoint{1.778517in}{0.626408in}}{\pgfqpoint{1.772992in}{0.626408in}}%
\pgfpathcurveto{\pgfqpoint{1.767467in}{0.626408in}}{\pgfqpoint{1.762167in}{0.624213in}}{\pgfqpoint{1.758260in}{0.620306in}}%
\pgfpathcurveto{\pgfqpoint{1.754353in}{0.616399in}}{\pgfqpoint{1.752158in}{0.611099in}}{\pgfqpoint{1.752158in}{0.605574in}}%
\pgfpathcurveto{\pgfqpoint{1.752158in}{0.600049in}}{\pgfqpoint{1.754353in}{0.594750in}}{\pgfqpoint{1.758260in}{0.590843in}}%
\pgfpathcurveto{\pgfqpoint{1.762167in}{0.586936in}}{\pgfqpoint{1.767467in}{0.584741in}}{\pgfqpoint{1.772992in}{0.584741in}}%
\pgfpathclose%
\pgfusepath{stroke,fill}%
\end{pgfscope}%
\begin{pgfscope}%
\pgfpathrectangle{\pgfqpoint{0.562500in}{0.275000in}}{\pgfqpoint{3.487500in}{1.925000in}}%
\pgfusepath{clip}%
\pgfsetbuttcap%
\pgfsetroundjoin%
\definecolor{currentfill}{rgb}{0.000000,0.000000,0.000000}%
\pgfsetfillcolor{currentfill}%
\pgfsetlinewidth{1.003750pt}%
\definecolor{currentstroke}{rgb}{0.000000,0.000000,0.000000}%
\pgfsetstrokecolor{currentstroke}%
\pgfsetdash{}{0pt}%
\pgfpathmoveto{\pgfqpoint{1.772992in}{0.549331in}}%
\pgfpathcurveto{\pgfqpoint{1.778517in}{0.549331in}}{\pgfqpoint{1.783816in}{0.551526in}}{\pgfqpoint{1.787723in}{0.555433in}}%
\pgfpathcurveto{\pgfqpoint{1.791630in}{0.559340in}}{\pgfqpoint{1.793825in}{0.564639in}}{\pgfqpoint{1.793825in}{0.570164in}}%
\pgfpathcurveto{\pgfqpoint{1.793825in}{0.575689in}}{\pgfqpoint{1.791630in}{0.580989in}}{\pgfqpoint{1.787723in}{0.584896in}}%
\pgfpathcurveto{\pgfqpoint{1.783816in}{0.588803in}}{\pgfqpoint{1.778517in}{0.590998in}}{\pgfqpoint{1.772992in}{0.590998in}}%
\pgfpathcurveto{\pgfqpoint{1.767467in}{0.590998in}}{\pgfqpoint{1.762167in}{0.588803in}}{\pgfqpoint{1.758260in}{0.584896in}}%
\pgfpathcurveto{\pgfqpoint{1.754353in}{0.580989in}}{\pgfqpoint{1.752158in}{0.575689in}}{\pgfqpoint{1.752158in}{0.570164in}}%
\pgfpathcurveto{\pgfqpoint{1.752158in}{0.564639in}}{\pgfqpoint{1.754353in}{0.559340in}}{\pgfqpoint{1.758260in}{0.555433in}}%
\pgfpathcurveto{\pgfqpoint{1.762167in}{0.551526in}}{\pgfqpoint{1.767467in}{0.549331in}}{\pgfqpoint{1.772992in}{0.549331in}}%
\pgfpathclose%
\pgfusepath{stroke,fill}%
\end{pgfscope}%
\begin{pgfscope}%
\pgfpathrectangle{\pgfqpoint{0.562500in}{0.275000in}}{\pgfqpoint{3.487500in}{1.925000in}}%
\pgfusepath{clip}%
\pgfsetbuttcap%
\pgfsetroundjoin%
\definecolor{currentfill}{rgb}{0.000000,0.000000,0.000000}%
\pgfsetfillcolor{currentfill}%
\pgfsetlinewidth{1.003750pt}%
\definecolor{currentstroke}{rgb}{0.000000,0.000000,0.000000}%
\pgfsetstrokecolor{currentstroke}%
\pgfsetdash{}{0pt}%
\pgfpathmoveto{\pgfqpoint{1.772992in}{0.589800in}}%
\pgfpathcurveto{\pgfqpoint{1.778517in}{0.589800in}}{\pgfqpoint{1.783816in}{0.591995in}}{\pgfqpoint{1.787723in}{0.595902in}}%
\pgfpathcurveto{\pgfqpoint{1.791630in}{0.599808in}}{\pgfqpoint{1.793825in}{0.605108in}}{\pgfqpoint{1.793825in}{0.610633in}}%
\pgfpathcurveto{\pgfqpoint{1.793825in}{0.616158in}}{\pgfqpoint{1.791630in}{0.621457in}}{\pgfqpoint{1.787723in}{0.625364in}}%
\pgfpathcurveto{\pgfqpoint{1.783816in}{0.629271in}}{\pgfqpoint{1.778517in}{0.631466in}}{\pgfqpoint{1.772992in}{0.631466in}}%
\pgfpathcurveto{\pgfqpoint{1.767467in}{0.631466in}}{\pgfqpoint{1.762167in}{0.629271in}}{\pgfqpoint{1.758260in}{0.625364in}}%
\pgfpathcurveto{\pgfqpoint{1.754353in}{0.621457in}}{\pgfqpoint{1.752158in}{0.616158in}}{\pgfqpoint{1.752158in}{0.610633in}}%
\pgfpathcurveto{\pgfqpoint{1.752158in}{0.605108in}}{\pgfqpoint{1.754353in}{0.599808in}}{\pgfqpoint{1.758260in}{0.595902in}}%
\pgfpathcurveto{\pgfqpoint{1.762167in}{0.591995in}}{\pgfqpoint{1.767467in}{0.589800in}}{\pgfqpoint{1.772992in}{0.589800in}}%
\pgfpathclose%
\pgfusepath{stroke,fill}%
\end{pgfscope}%
\begin{pgfscope}%
\pgfpathrectangle{\pgfqpoint{0.562500in}{0.275000in}}{\pgfqpoint{3.487500in}{1.925000in}}%
\pgfusepath{clip}%
\pgfsetbuttcap%
\pgfsetroundjoin%
\definecolor{currentfill}{rgb}{0.000000,0.000000,0.000000}%
\pgfsetfillcolor{currentfill}%
\pgfsetlinewidth{1.003750pt}%
\definecolor{currentstroke}{rgb}{0.000000,0.000000,0.000000}%
\pgfsetstrokecolor{currentstroke}%
\pgfsetdash{}{0pt}%
\pgfpathmoveto{\pgfqpoint{1.772992in}{0.534155in}}%
\pgfpathcurveto{\pgfqpoint{1.778517in}{0.534155in}}{\pgfqpoint{1.783816in}{0.536350in}}{\pgfqpoint{1.787723in}{0.540257in}}%
\pgfpathcurveto{\pgfqpoint{1.791630in}{0.544164in}}{\pgfqpoint{1.793825in}{0.549464in}}{\pgfqpoint{1.793825in}{0.554989in}}%
\pgfpathcurveto{\pgfqpoint{1.793825in}{0.560514in}}{\pgfqpoint{1.791630in}{0.565813in}}{\pgfqpoint{1.787723in}{0.569720in}}%
\pgfpathcurveto{\pgfqpoint{1.783816in}{0.573627in}}{\pgfqpoint{1.778517in}{0.575822in}}{\pgfqpoint{1.772992in}{0.575822in}}%
\pgfpathcurveto{\pgfqpoint{1.767467in}{0.575822in}}{\pgfqpoint{1.762167in}{0.573627in}}{\pgfqpoint{1.758260in}{0.569720in}}%
\pgfpathcurveto{\pgfqpoint{1.754353in}{0.565813in}}{\pgfqpoint{1.752158in}{0.560514in}}{\pgfqpoint{1.752158in}{0.554989in}}%
\pgfpathcurveto{\pgfqpoint{1.752158in}{0.549464in}}{\pgfqpoint{1.754353in}{0.544164in}}{\pgfqpoint{1.758260in}{0.540257in}}%
\pgfpathcurveto{\pgfqpoint{1.762167in}{0.536350in}}{\pgfqpoint{1.767467in}{0.534155in}}{\pgfqpoint{1.772992in}{0.534155in}}%
\pgfpathclose%
\pgfusepath{stroke,fill}%
\end{pgfscope}%
\begin{pgfscope}%
\pgfpathrectangle{\pgfqpoint{0.562500in}{0.275000in}}{\pgfqpoint{3.487500in}{1.925000in}}%
\pgfusepath{clip}%
\pgfsetbuttcap%
\pgfsetroundjoin%
\definecolor{currentfill}{rgb}{0.000000,0.000000,0.000000}%
\pgfsetfillcolor{currentfill}%
\pgfsetlinewidth{1.003750pt}%
\definecolor{currentstroke}{rgb}{0.000000,0.000000,0.000000}%
\pgfsetstrokecolor{currentstroke}%
\pgfsetdash{}{0pt}%
\pgfpathmoveto{\pgfqpoint{1.772992in}{0.534155in}}%
\pgfpathcurveto{\pgfqpoint{1.778517in}{0.534155in}}{\pgfqpoint{1.783816in}{0.536350in}}{\pgfqpoint{1.787723in}{0.540257in}}%
\pgfpathcurveto{\pgfqpoint{1.791630in}{0.544164in}}{\pgfqpoint{1.793825in}{0.549464in}}{\pgfqpoint{1.793825in}{0.554989in}}%
\pgfpathcurveto{\pgfqpoint{1.793825in}{0.560514in}}{\pgfqpoint{1.791630in}{0.565813in}}{\pgfqpoint{1.787723in}{0.569720in}}%
\pgfpathcurveto{\pgfqpoint{1.783816in}{0.573627in}}{\pgfqpoint{1.778517in}{0.575822in}}{\pgfqpoint{1.772992in}{0.575822in}}%
\pgfpathcurveto{\pgfqpoint{1.767467in}{0.575822in}}{\pgfqpoint{1.762167in}{0.573627in}}{\pgfqpoint{1.758260in}{0.569720in}}%
\pgfpathcurveto{\pgfqpoint{1.754353in}{0.565813in}}{\pgfqpoint{1.752158in}{0.560514in}}{\pgfqpoint{1.752158in}{0.554989in}}%
\pgfpathcurveto{\pgfqpoint{1.752158in}{0.549464in}}{\pgfqpoint{1.754353in}{0.544164in}}{\pgfqpoint{1.758260in}{0.540257in}}%
\pgfpathcurveto{\pgfqpoint{1.762167in}{0.536350in}}{\pgfqpoint{1.767467in}{0.534155in}}{\pgfqpoint{1.772992in}{0.534155in}}%
\pgfpathclose%
\pgfusepath{stroke,fill}%
\end{pgfscope}%
\begin{pgfscope}%
\pgfpathrectangle{\pgfqpoint{0.562500in}{0.275000in}}{\pgfqpoint{3.487500in}{1.925000in}}%
\pgfusepath{clip}%
\pgfsetbuttcap%
\pgfsetroundjoin%
\definecolor{currentfill}{rgb}{0.000000,0.000000,0.000000}%
\pgfsetfillcolor{currentfill}%
\pgfsetlinewidth{1.003750pt}%
\definecolor{currentstroke}{rgb}{0.000000,0.000000,0.000000}%
\pgfsetstrokecolor{currentstroke}%
\pgfsetdash{}{0pt}%
\pgfpathmoveto{\pgfqpoint{1.772992in}{0.503804in}}%
\pgfpathcurveto{\pgfqpoint{1.778517in}{0.503804in}}{\pgfqpoint{1.783816in}{0.505999in}}{\pgfqpoint{1.787723in}{0.509906in}}%
\pgfpathcurveto{\pgfqpoint{1.791630in}{0.513813in}}{\pgfqpoint{1.793825in}{0.519112in}}{\pgfqpoint{1.793825in}{0.524637in}}%
\pgfpathcurveto{\pgfqpoint{1.793825in}{0.530162in}}{\pgfqpoint{1.791630in}{0.535462in}}{\pgfqpoint{1.787723in}{0.539369in}}%
\pgfpathcurveto{\pgfqpoint{1.783816in}{0.543275in}}{\pgfqpoint{1.778517in}{0.545471in}}{\pgfqpoint{1.772992in}{0.545471in}}%
\pgfpathcurveto{\pgfqpoint{1.767467in}{0.545471in}}{\pgfqpoint{1.762167in}{0.543275in}}{\pgfqpoint{1.758260in}{0.539369in}}%
\pgfpathcurveto{\pgfqpoint{1.754353in}{0.535462in}}{\pgfqpoint{1.752158in}{0.530162in}}{\pgfqpoint{1.752158in}{0.524637in}}%
\pgfpathcurveto{\pgfqpoint{1.752158in}{0.519112in}}{\pgfqpoint{1.754353in}{0.513813in}}{\pgfqpoint{1.758260in}{0.509906in}}%
\pgfpathcurveto{\pgfqpoint{1.762167in}{0.505999in}}{\pgfqpoint{1.767467in}{0.503804in}}{\pgfqpoint{1.772992in}{0.503804in}}%
\pgfpathclose%
\pgfusepath{stroke,fill}%
\end{pgfscope}%
\begin{pgfscope}%
\pgfpathrectangle{\pgfqpoint{0.562500in}{0.275000in}}{\pgfqpoint{3.487500in}{1.925000in}}%
\pgfusepath{clip}%
\pgfsetbuttcap%
\pgfsetroundjoin%
\definecolor{currentfill}{rgb}{0.000000,0.000000,0.000000}%
\pgfsetfillcolor{currentfill}%
\pgfsetlinewidth{1.003750pt}%
\definecolor{currentstroke}{rgb}{0.000000,0.000000,0.000000}%
\pgfsetstrokecolor{currentstroke}%
\pgfsetdash{}{0pt}%
\pgfpathmoveto{\pgfqpoint{1.772992in}{0.508863in}}%
\pgfpathcurveto{\pgfqpoint{1.778517in}{0.508863in}}{\pgfqpoint{1.783816in}{0.511058in}}{\pgfqpoint{1.787723in}{0.514964in}}%
\pgfpathcurveto{\pgfqpoint{1.791630in}{0.518871in}}{\pgfqpoint{1.793825in}{0.524171in}}{\pgfqpoint{1.793825in}{0.529696in}}%
\pgfpathcurveto{\pgfqpoint{1.793825in}{0.535221in}}{\pgfqpoint{1.791630in}{0.540520in}}{\pgfqpoint{1.787723in}{0.544427in}}%
\pgfpathcurveto{\pgfqpoint{1.783816in}{0.548334in}}{\pgfqpoint{1.778517in}{0.550529in}}{\pgfqpoint{1.772992in}{0.550529in}}%
\pgfpathcurveto{\pgfqpoint{1.767467in}{0.550529in}}{\pgfqpoint{1.762167in}{0.548334in}}{\pgfqpoint{1.758260in}{0.544427in}}%
\pgfpathcurveto{\pgfqpoint{1.754353in}{0.540520in}}{\pgfqpoint{1.752158in}{0.535221in}}{\pgfqpoint{1.752158in}{0.529696in}}%
\pgfpathcurveto{\pgfqpoint{1.752158in}{0.524171in}}{\pgfqpoint{1.754353in}{0.518871in}}{\pgfqpoint{1.758260in}{0.514964in}}%
\pgfpathcurveto{\pgfqpoint{1.762167in}{0.511058in}}{\pgfqpoint{1.767467in}{0.508863in}}{\pgfqpoint{1.772992in}{0.508863in}}%
\pgfpathclose%
\pgfusepath{stroke,fill}%
\end{pgfscope}%
\begin{pgfscope}%
\pgfpathrectangle{\pgfqpoint{0.562500in}{0.275000in}}{\pgfqpoint{3.487500in}{1.925000in}}%
\pgfusepath{clip}%
\pgfsetbuttcap%
\pgfsetroundjoin%
\definecolor{currentfill}{rgb}{0.000000,0.000000,0.000000}%
\pgfsetfillcolor{currentfill}%
\pgfsetlinewidth{1.003750pt}%
\definecolor{currentstroke}{rgb}{0.000000,0.000000,0.000000}%
\pgfsetstrokecolor{currentstroke}%
\pgfsetdash{}{0pt}%
\pgfpathmoveto{\pgfqpoint{1.772992in}{0.513921in}}%
\pgfpathcurveto{\pgfqpoint{1.778517in}{0.513921in}}{\pgfqpoint{1.783816in}{0.516116in}}{\pgfqpoint{1.787723in}{0.520023in}}%
\pgfpathcurveto{\pgfqpoint{1.791630in}{0.523930in}}{\pgfqpoint{1.793825in}{0.529229in}}{\pgfqpoint{1.793825in}{0.534754in}}%
\pgfpathcurveto{\pgfqpoint{1.793825in}{0.540279in}}{\pgfqpoint{1.791630in}{0.545579in}}{\pgfqpoint{1.787723in}{0.549486in}}%
\pgfpathcurveto{\pgfqpoint{1.783816in}{0.553393in}}{\pgfqpoint{1.778517in}{0.555588in}}{\pgfqpoint{1.772992in}{0.555588in}}%
\pgfpathcurveto{\pgfqpoint{1.767467in}{0.555588in}}{\pgfqpoint{1.762167in}{0.553393in}}{\pgfqpoint{1.758260in}{0.549486in}}%
\pgfpathcurveto{\pgfqpoint{1.754353in}{0.545579in}}{\pgfqpoint{1.752158in}{0.540279in}}{\pgfqpoint{1.752158in}{0.534754in}}%
\pgfpathcurveto{\pgfqpoint{1.752158in}{0.529229in}}{\pgfqpoint{1.754353in}{0.523930in}}{\pgfqpoint{1.758260in}{0.520023in}}%
\pgfpathcurveto{\pgfqpoint{1.762167in}{0.516116in}}{\pgfqpoint{1.767467in}{0.513921in}}{\pgfqpoint{1.772992in}{0.513921in}}%
\pgfpathclose%
\pgfusepath{stroke,fill}%
\end{pgfscope}%
\begin{pgfscope}%
\pgfpathrectangle{\pgfqpoint{0.562500in}{0.275000in}}{\pgfqpoint{3.487500in}{1.925000in}}%
\pgfusepath{clip}%
\pgfsetbuttcap%
\pgfsetroundjoin%
\definecolor{currentfill}{rgb}{0.000000,0.000000,0.000000}%
\pgfsetfillcolor{currentfill}%
\pgfsetlinewidth{1.003750pt}%
\definecolor{currentstroke}{rgb}{0.000000,0.000000,0.000000}%
\pgfsetstrokecolor{currentstroke}%
\pgfsetdash{}{0pt}%
\pgfpathmoveto{\pgfqpoint{1.772992in}{0.549331in}}%
\pgfpathcurveto{\pgfqpoint{1.778517in}{0.549331in}}{\pgfqpoint{1.783816in}{0.551526in}}{\pgfqpoint{1.787723in}{0.555433in}}%
\pgfpathcurveto{\pgfqpoint{1.791630in}{0.559340in}}{\pgfqpoint{1.793825in}{0.564639in}}{\pgfqpoint{1.793825in}{0.570164in}}%
\pgfpathcurveto{\pgfqpoint{1.793825in}{0.575689in}}{\pgfqpoint{1.791630in}{0.580989in}}{\pgfqpoint{1.787723in}{0.584896in}}%
\pgfpathcurveto{\pgfqpoint{1.783816in}{0.588803in}}{\pgfqpoint{1.778517in}{0.590998in}}{\pgfqpoint{1.772992in}{0.590998in}}%
\pgfpathcurveto{\pgfqpoint{1.767467in}{0.590998in}}{\pgfqpoint{1.762167in}{0.588803in}}{\pgfqpoint{1.758260in}{0.584896in}}%
\pgfpathcurveto{\pgfqpoint{1.754353in}{0.580989in}}{\pgfqpoint{1.752158in}{0.575689in}}{\pgfqpoint{1.752158in}{0.570164in}}%
\pgfpathcurveto{\pgfqpoint{1.752158in}{0.564639in}}{\pgfqpoint{1.754353in}{0.559340in}}{\pgfqpoint{1.758260in}{0.555433in}}%
\pgfpathcurveto{\pgfqpoint{1.762167in}{0.551526in}}{\pgfqpoint{1.767467in}{0.549331in}}{\pgfqpoint{1.772992in}{0.549331in}}%
\pgfpathclose%
\pgfusepath{stroke,fill}%
\end{pgfscope}%
\begin{pgfscope}%
\pgfpathrectangle{\pgfqpoint{0.562500in}{0.275000in}}{\pgfqpoint{3.487500in}{1.925000in}}%
\pgfusepath{clip}%
\pgfsetbuttcap%
\pgfsetroundjoin%
\definecolor{currentfill}{rgb}{0.000000,0.000000,0.000000}%
\pgfsetfillcolor{currentfill}%
\pgfsetlinewidth{1.003750pt}%
\definecolor{currentstroke}{rgb}{0.000000,0.000000,0.000000}%
\pgfsetstrokecolor{currentstroke}%
\pgfsetdash{}{0pt}%
\pgfpathmoveto{\pgfqpoint{1.772992in}{0.564507in}}%
\pgfpathcurveto{\pgfqpoint{1.778517in}{0.564507in}}{\pgfqpoint{1.783816in}{0.566702in}}{\pgfqpoint{1.787723in}{0.570609in}}%
\pgfpathcurveto{\pgfqpoint{1.791630in}{0.574515in}}{\pgfqpoint{1.793825in}{0.579815in}}{\pgfqpoint{1.793825in}{0.585340in}}%
\pgfpathcurveto{\pgfqpoint{1.793825in}{0.590865in}}{\pgfqpoint{1.791630in}{0.596165in}}{\pgfqpoint{1.787723in}{0.600071in}}%
\pgfpathcurveto{\pgfqpoint{1.783816in}{0.603978in}}{\pgfqpoint{1.778517in}{0.606173in}}{\pgfqpoint{1.772992in}{0.606173in}}%
\pgfpathcurveto{\pgfqpoint{1.767467in}{0.606173in}}{\pgfqpoint{1.762167in}{0.603978in}}{\pgfqpoint{1.758260in}{0.600071in}}%
\pgfpathcurveto{\pgfqpoint{1.754353in}{0.596165in}}{\pgfqpoint{1.752158in}{0.590865in}}{\pgfqpoint{1.752158in}{0.585340in}}%
\pgfpathcurveto{\pgfqpoint{1.752158in}{0.579815in}}{\pgfqpoint{1.754353in}{0.574515in}}{\pgfqpoint{1.758260in}{0.570609in}}%
\pgfpathcurveto{\pgfqpoint{1.762167in}{0.566702in}}{\pgfqpoint{1.767467in}{0.564507in}}{\pgfqpoint{1.772992in}{0.564507in}}%
\pgfpathclose%
\pgfusepath{stroke,fill}%
\end{pgfscope}%
\begin{pgfscope}%
\pgfpathrectangle{\pgfqpoint{0.562500in}{0.275000in}}{\pgfqpoint{3.487500in}{1.925000in}}%
\pgfusepath{clip}%
\pgfsetbuttcap%
\pgfsetroundjoin%
\definecolor{currentfill}{rgb}{0.000000,0.000000,0.000000}%
\pgfsetfillcolor{currentfill}%
\pgfsetlinewidth{1.003750pt}%
\definecolor{currentstroke}{rgb}{0.000000,0.000000,0.000000}%
\pgfsetstrokecolor{currentstroke}%
\pgfsetdash{}{0pt}%
\pgfpathmoveto{\pgfqpoint{1.772992in}{0.574624in}}%
\pgfpathcurveto{\pgfqpoint{1.778517in}{0.574624in}}{\pgfqpoint{1.783816in}{0.576819in}}{\pgfqpoint{1.787723in}{0.580726in}}%
\pgfpathcurveto{\pgfqpoint{1.791630in}{0.584633in}}{\pgfqpoint{1.793825in}{0.589932in}}{\pgfqpoint{1.793825in}{0.595457in}}%
\pgfpathcurveto{\pgfqpoint{1.793825in}{0.600982in}}{\pgfqpoint{1.791630in}{0.606282in}}{\pgfqpoint{1.787723in}{0.610189in}}%
\pgfpathcurveto{\pgfqpoint{1.783816in}{0.614095in}}{\pgfqpoint{1.778517in}{0.616291in}}{\pgfqpoint{1.772992in}{0.616291in}}%
\pgfpathcurveto{\pgfqpoint{1.767467in}{0.616291in}}{\pgfqpoint{1.762167in}{0.614095in}}{\pgfqpoint{1.758260in}{0.610189in}}%
\pgfpathcurveto{\pgfqpoint{1.754353in}{0.606282in}}{\pgfqpoint{1.752158in}{0.600982in}}{\pgfqpoint{1.752158in}{0.595457in}}%
\pgfpathcurveto{\pgfqpoint{1.752158in}{0.589932in}}{\pgfqpoint{1.754353in}{0.584633in}}{\pgfqpoint{1.758260in}{0.580726in}}%
\pgfpathcurveto{\pgfqpoint{1.762167in}{0.576819in}}{\pgfqpoint{1.767467in}{0.574624in}}{\pgfqpoint{1.772992in}{0.574624in}}%
\pgfpathclose%
\pgfusepath{stroke,fill}%
\end{pgfscope}%
\begin{pgfscope}%
\pgfpathrectangle{\pgfqpoint{0.562500in}{0.275000in}}{\pgfqpoint{3.487500in}{1.925000in}}%
\pgfusepath{clip}%
\pgfsetbuttcap%
\pgfsetroundjoin%
\definecolor{currentfill}{rgb}{0.000000,0.000000,0.000000}%
\pgfsetfillcolor{currentfill}%
\pgfsetlinewidth{1.003750pt}%
\definecolor{currentstroke}{rgb}{0.000000,0.000000,0.000000}%
\pgfsetstrokecolor{currentstroke}%
\pgfsetdash{}{0pt}%
\pgfpathmoveto{\pgfqpoint{1.772992in}{0.554390in}}%
\pgfpathcurveto{\pgfqpoint{1.778517in}{0.554390in}}{\pgfqpoint{1.783816in}{0.556585in}}{\pgfqpoint{1.787723in}{0.560492in}}%
\pgfpathcurveto{\pgfqpoint{1.791630in}{0.564398in}}{\pgfqpoint{1.793825in}{0.569698in}}{\pgfqpoint{1.793825in}{0.575223in}}%
\pgfpathcurveto{\pgfqpoint{1.793825in}{0.580748in}}{\pgfqpoint{1.791630in}{0.586048in}}{\pgfqpoint{1.787723in}{0.589954in}}%
\pgfpathcurveto{\pgfqpoint{1.783816in}{0.593861in}}{\pgfqpoint{1.778517in}{0.596056in}}{\pgfqpoint{1.772992in}{0.596056in}}%
\pgfpathcurveto{\pgfqpoint{1.767467in}{0.596056in}}{\pgfqpoint{1.762167in}{0.593861in}}{\pgfqpoint{1.758260in}{0.589954in}}%
\pgfpathcurveto{\pgfqpoint{1.754353in}{0.586048in}}{\pgfqpoint{1.752158in}{0.580748in}}{\pgfqpoint{1.752158in}{0.575223in}}%
\pgfpathcurveto{\pgfqpoint{1.752158in}{0.569698in}}{\pgfqpoint{1.754353in}{0.564398in}}{\pgfqpoint{1.758260in}{0.560492in}}%
\pgfpathcurveto{\pgfqpoint{1.762167in}{0.556585in}}{\pgfqpoint{1.767467in}{0.554390in}}{\pgfqpoint{1.772992in}{0.554390in}}%
\pgfpathclose%
\pgfusepath{stroke,fill}%
\end{pgfscope}%
\begin{pgfscope}%
\pgfpathrectangle{\pgfqpoint{0.562500in}{0.275000in}}{\pgfqpoint{3.487500in}{1.925000in}}%
\pgfusepath{clip}%
\pgfsetbuttcap%
\pgfsetroundjoin%
\definecolor{currentfill}{rgb}{0.000000,0.000000,0.000000}%
\pgfsetfillcolor{currentfill}%
\pgfsetlinewidth{1.003750pt}%
\definecolor{currentstroke}{rgb}{0.000000,0.000000,0.000000}%
\pgfsetstrokecolor{currentstroke}%
\pgfsetdash{}{0pt}%
\pgfpathmoveto{\pgfqpoint{1.772992in}{0.524038in}}%
\pgfpathcurveto{\pgfqpoint{1.778517in}{0.524038in}}{\pgfqpoint{1.783816in}{0.526233in}}{\pgfqpoint{1.787723in}{0.530140in}}%
\pgfpathcurveto{\pgfqpoint{1.791630in}{0.534047in}}{\pgfqpoint{1.793825in}{0.539346in}}{\pgfqpoint{1.793825in}{0.544872in}}%
\pgfpathcurveto{\pgfqpoint{1.793825in}{0.550397in}}{\pgfqpoint{1.791630in}{0.555696in}}{\pgfqpoint{1.787723in}{0.559603in}}%
\pgfpathcurveto{\pgfqpoint{1.783816in}{0.563510in}}{\pgfqpoint{1.778517in}{0.565705in}}{\pgfqpoint{1.772992in}{0.565705in}}%
\pgfpathcurveto{\pgfqpoint{1.767467in}{0.565705in}}{\pgfqpoint{1.762167in}{0.563510in}}{\pgfqpoint{1.758260in}{0.559603in}}%
\pgfpathcurveto{\pgfqpoint{1.754353in}{0.555696in}}{\pgfqpoint{1.752158in}{0.550397in}}{\pgfqpoint{1.752158in}{0.544872in}}%
\pgfpathcurveto{\pgfqpoint{1.752158in}{0.539346in}}{\pgfqpoint{1.754353in}{0.534047in}}{\pgfqpoint{1.758260in}{0.530140in}}%
\pgfpathcurveto{\pgfqpoint{1.762167in}{0.526233in}}{\pgfqpoint{1.767467in}{0.524038in}}{\pgfqpoint{1.772992in}{0.524038in}}%
\pgfpathclose%
\pgfusepath{stroke,fill}%
\end{pgfscope}%
\begin{pgfscope}%
\pgfpathrectangle{\pgfqpoint{0.562500in}{0.275000in}}{\pgfqpoint{3.487500in}{1.925000in}}%
\pgfusepath{clip}%
\pgfsetbuttcap%
\pgfsetroundjoin%
\definecolor{currentfill}{rgb}{0.000000,0.000000,0.000000}%
\pgfsetfillcolor{currentfill}%
\pgfsetlinewidth{1.003750pt}%
\definecolor{currentstroke}{rgb}{0.000000,0.000000,0.000000}%
\pgfsetstrokecolor{currentstroke}%
\pgfsetdash{}{0pt}%
\pgfpathmoveto{\pgfqpoint{1.772992in}{0.513921in}}%
\pgfpathcurveto{\pgfqpoint{1.778517in}{0.513921in}}{\pgfqpoint{1.783816in}{0.516116in}}{\pgfqpoint{1.787723in}{0.520023in}}%
\pgfpathcurveto{\pgfqpoint{1.791630in}{0.523930in}}{\pgfqpoint{1.793825in}{0.529229in}}{\pgfqpoint{1.793825in}{0.534754in}}%
\pgfpathcurveto{\pgfqpoint{1.793825in}{0.540279in}}{\pgfqpoint{1.791630in}{0.545579in}}{\pgfqpoint{1.787723in}{0.549486in}}%
\pgfpathcurveto{\pgfqpoint{1.783816in}{0.553393in}}{\pgfqpoint{1.778517in}{0.555588in}}{\pgfqpoint{1.772992in}{0.555588in}}%
\pgfpathcurveto{\pgfqpoint{1.767467in}{0.555588in}}{\pgfqpoint{1.762167in}{0.553393in}}{\pgfqpoint{1.758260in}{0.549486in}}%
\pgfpathcurveto{\pgfqpoint{1.754353in}{0.545579in}}{\pgfqpoint{1.752158in}{0.540279in}}{\pgfqpoint{1.752158in}{0.534754in}}%
\pgfpathcurveto{\pgfqpoint{1.752158in}{0.529229in}}{\pgfqpoint{1.754353in}{0.523930in}}{\pgfqpoint{1.758260in}{0.520023in}}%
\pgfpathcurveto{\pgfqpoint{1.762167in}{0.516116in}}{\pgfqpoint{1.767467in}{0.513921in}}{\pgfqpoint{1.772992in}{0.513921in}}%
\pgfpathclose%
\pgfusepath{stroke,fill}%
\end{pgfscope}%
\begin{pgfscope}%
\pgfpathrectangle{\pgfqpoint{0.562500in}{0.275000in}}{\pgfqpoint{3.487500in}{1.925000in}}%
\pgfusepath{clip}%
\pgfsetbuttcap%
\pgfsetroundjoin%
\definecolor{currentfill}{rgb}{0.000000,0.000000,0.000000}%
\pgfsetfillcolor{currentfill}%
\pgfsetlinewidth{1.003750pt}%
\definecolor{currentstroke}{rgb}{0.000000,0.000000,0.000000}%
\pgfsetstrokecolor{currentstroke}%
\pgfsetdash{}{0pt}%
\pgfpathmoveto{\pgfqpoint{1.772992in}{0.539214in}}%
\pgfpathcurveto{\pgfqpoint{1.778517in}{0.539214in}}{\pgfqpoint{1.783816in}{0.541409in}}{\pgfqpoint{1.787723in}{0.545316in}}%
\pgfpathcurveto{\pgfqpoint{1.791630in}{0.549223in}}{\pgfqpoint{1.793825in}{0.554522in}}{\pgfqpoint{1.793825in}{0.560047in}}%
\pgfpathcurveto{\pgfqpoint{1.793825in}{0.565572in}}{\pgfqpoint{1.791630in}{0.570872in}}{\pgfqpoint{1.787723in}{0.574779in}}%
\pgfpathcurveto{\pgfqpoint{1.783816in}{0.578685in}}{\pgfqpoint{1.778517in}{0.580881in}}{\pgfqpoint{1.772992in}{0.580881in}}%
\pgfpathcurveto{\pgfqpoint{1.767467in}{0.580881in}}{\pgfqpoint{1.762167in}{0.578685in}}{\pgfqpoint{1.758260in}{0.574779in}}%
\pgfpathcurveto{\pgfqpoint{1.754353in}{0.570872in}}{\pgfqpoint{1.752158in}{0.565572in}}{\pgfqpoint{1.752158in}{0.560047in}}%
\pgfpathcurveto{\pgfqpoint{1.752158in}{0.554522in}}{\pgfqpoint{1.754353in}{0.549223in}}{\pgfqpoint{1.758260in}{0.545316in}}%
\pgfpathcurveto{\pgfqpoint{1.762167in}{0.541409in}}{\pgfqpoint{1.767467in}{0.539214in}}{\pgfqpoint{1.772992in}{0.539214in}}%
\pgfpathclose%
\pgfusepath{stroke,fill}%
\end{pgfscope}%
\begin{pgfscope}%
\pgfpathrectangle{\pgfqpoint{0.562500in}{0.275000in}}{\pgfqpoint{3.487500in}{1.925000in}}%
\pgfusepath{clip}%
\pgfsetbuttcap%
\pgfsetroundjoin%
\definecolor{currentfill}{rgb}{0.000000,0.000000,0.000000}%
\pgfsetfillcolor{currentfill}%
\pgfsetlinewidth{1.003750pt}%
\definecolor{currentstroke}{rgb}{0.000000,0.000000,0.000000}%
\pgfsetstrokecolor{currentstroke}%
\pgfsetdash{}{0pt}%
\pgfpathmoveto{\pgfqpoint{1.772992in}{0.508863in}}%
\pgfpathcurveto{\pgfqpoint{1.778517in}{0.508863in}}{\pgfqpoint{1.783816in}{0.511058in}}{\pgfqpoint{1.787723in}{0.514964in}}%
\pgfpathcurveto{\pgfqpoint{1.791630in}{0.518871in}}{\pgfqpoint{1.793825in}{0.524171in}}{\pgfqpoint{1.793825in}{0.529696in}}%
\pgfpathcurveto{\pgfqpoint{1.793825in}{0.535221in}}{\pgfqpoint{1.791630in}{0.540520in}}{\pgfqpoint{1.787723in}{0.544427in}}%
\pgfpathcurveto{\pgfqpoint{1.783816in}{0.548334in}}{\pgfqpoint{1.778517in}{0.550529in}}{\pgfqpoint{1.772992in}{0.550529in}}%
\pgfpathcurveto{\pgfqpoint{1.767467in}{0.550529in}}{\pgfqpoint{1.762167in}{0.548334in}}{\pgfqpoint{1.758260in}{0.544427in}}%
\pgfpathcurveto{\pgfqpoint{1.754353in}{0.540520in}}{\pgfqpoint{1.752158in}{0.535221in}}{\pgfqpoint{1.752158in}{0.529696in}}%
\pgfpathcurveto{\pgfqpoint{1.752158in}{0.524171in}}{\pgfqpoint{1.754353in}{0.518871in}}{\pgfqpoint{1.758260in}{0.514964in}}%
\pgfpathcurveto{\pgfqpoint{1.762167in}{0.511058in}}{\pgfqpoint{1.767467in}{0.508863in}}{\pgfqpoint{1.772992in}{0.508863in}}%
\pgfpathclose%
\pgfusepath{stroke,fill}%
\end{pgfscope}%
\begin{pgfscope}%
\pgfpathrectangle{\pgfqpoint{0.562500in}{0.275000in}}{\pgfqpoint{3.487500in}{1.925000in}}%
\pgfusepath{clip}%
\pgfsetbuttcap%
\pgfsetroundjoin%
\definecolor{currentfill}{rgb}{0.000000,0.000000,0.000000}%
\pgfsetfillcolor{currentfill}%
\pgfsetlinewidth{1.003750pt}%
\definecolor{currentstroke}{rgb}{0.000000,0.000000,0.000000}%
\pgfsetstrokecolor{currentstroke}%
\pgfsetdash{}{0pt}%
\pgfpathmoveto{\pgfqpoint{1.772992in}{0.539214in}}%
\pgfpathcurveto{\pgfqpoint{1.778517in}{0.539214in}}{\pgfqpoint{1.783816in}{0.541409in}}{\pgfqpoint{1.787723in}{0.545316in}}%
\pgfpathcurveto{\pgfqpoint{1.791630in}{0.549223in}}{\pgfqpoint{1.793825in}{0.554522in}}{\pgfqpoint{1.793825in}{0.560047in}}%
\pgfpathcurveto{\pgfqpoint{1.793825in}{0.565572in}}{\pgfqpoint{1.791630in}{0.570872in}}{\pgfqpoint{1.787723in}{0.574779in}}%
\pgfpathcurveto{\pgfqpoint{1.783816in}{0.578685in}}{\pgfqpoint{1.778517in}{0.580881in}}{\pgfqpoint{1.772992in}{0.580881in}}%
\pgfpathcurveto{\pgfqpoint{1.767467in}{0.580881in}}{\pgfqpoint{1.762167in}{0.578685in}}{\pgfqpoint{1.758260in}{0.574779in}}%
\pgfpathcurveto{\pgfqpoint{1.754353in}{0.570872in}}{\pgfqpoint{1.752158in}{0.565572in}}{\pgfqpoint{1.752158in}{0.560047in}}%
\pgfpathcurveto{\pgfqpoint{1.752158in}{0.554522in}}{\pgfqpoint{1.754353in}{0.549223in}}{\pgfqpoint{1.758260in}{0.545316in}}%
\pgfpathcurveto{\pgfqpoint{1.762167in}{0.541409in}}{\pgfqpoint{1.767467in}{0.539214in}}{\pgfqpoint{1.772992in}{0.539214in}}%
\pgfpathclose%
\pgfusepath{stroke,fill}%
\end{pgfscope}%
\begin{pgfscope}%
\pgfpathrectangle{\pgfqpoint{0.562500in}{0.275000in}}{\pgfqpoint{3.487500in}{1.925000in}}%
\pgfusepath{clip}%
\pgfsetbuttcap%
\pgfsetroundjoin%
\definecolor{currentfill}{rgb}{0.000000,0.000000,0.000000}%
\pgfsetfillcolor{currentfill}%
\pgfsetlinewidth{1.003750pt}%
\definecolor{currentstroke}{rgb}{0.000000,0.000000,0.000000}%
\pgfsetstrokecolor{currentstroke}%
\pgfsetdash{}{0pt}%
\pgfpathmoveto{\pgfqpoint{1.772992in}{0.524038in}}%
\pgfpathcurveto{\pgfqpoint{1.778517in}{0.524038in}}{\pgfqpoint{1.783816in}{0.526233in}}{\pgfqpoint{1.787723in}{0.530140in}}%
\pgfpathcurveto{\pgfqpoint{1.791630in}{0.534047in}}{\pgfqpoint{1.793825in}{0.539346in}}{\pgfqpoint{1.793825in}{0.544872in}}%
\pgfpathcurveto{\pgfqpoint{1.793825in}{0.550397in}}{\pgfqpoint{1.791630in}{0.555696in}}{\pgfqpoint{1.787723in}{0.559603in}}%
\pgfpathcurveto{\pgfqpoint{1.783816in}{0.563510in}}{\pgfqpoint{1.778517in}{0.565705in}}{\pgfqpoint{1.772992in}{0.565705in}}%
\pgfpathcurveto{\pgfqpoint{1.767467in}{0.565705in}}{\pgfqpoint{1.762167in}{0.563510in}}{\pgfqpoint{1.758260in}{0.559603in}}%
\pgfpathcurveto{\pgfqpoint{1.754353in}{0.555696in}}{\pgfqpoint{1.752158in}{0.550397in}}{\pgfqpoint{1.752158in}{0.544872in}}%
\pgfpathcurveto{\pgfqpoint{1.752158in}{0.539346in}}{\pgfqpoint{1.754353in}{0.534047in}}{\pgfqpoint{1.758260in}{0.530140in}}%
\pgfpathcurveto{\pgfqpoint{1.762167in}{0.526233in}}{\pgfqpoint{1.767467in}{0.524038in}}{\pgfqpoint{1.772992in}{0.524038in}}%
\pgfpathclose%
\pgfusepath{stroke,fill}%
\end{pgfscope}%
\begin{pgfscope}%
\pgfpathrectangle{\pgfqpoint{0.562500in}{0.275000in}}{\pgfqpoint{3.487500in}{1.925000in}}%
\pgfusepath{clip}%
\pgfsetbuttcap%
\pgfsetroundjoin%
\definecolor{currentfill}{rgb}{0.000000,0.000000,0.000000}%
\pgfsetfillcolor{currentfill}%
\pgfsetlinewidth{1.003750pt}%
\definecolor{currentstroke}{rgb}{0.000000,0.000000,0.000000}%
\pgfsetstrokecolor{currentstroke}%
\pgfsetdash{}{0pt}%
\pgfpathmoveto{\pgfqpoint{1.772992in}{0.574624in}}%
\pgfpathcurveto{\pgfqpoint{1.778517in}{0.574624in}}{\pgfqpoint{1.783816in}{0.576819in}}{\pgfqpoint{1.787723in}{0.580726in}}%
\pgfpathcurveto{\pgfqpoint{1.791630in}{0.584633in}}{\pgfqpoint{1.793825in}{0.589932in}}{\pgfqpoint{1.793825in}{0.595457in}}%
\pgfpathcurveto{\pgfqpoint{1.793825in}{0.600982in}}{\pgfqpoint{1.791630in}{0.606282in}}{\pgfqpoint{1.787723in}{0.610189in}}%
\pgfpathcurveto{\pgfqpoint{1.783816in}{0.614095in}}{\pgfqpoint{1.778517in}{0.616291in}}{\pgfqpoint{1.772992in}{0.616291in}}%
\pgfpathcurveto{\pgfqpoint{1.767467in}{0.616291in}}{\pgfqpoint{1.762167in}{0.614095in}}{\pgfqpoint{1.758260in}{0.610189in}}%
\pgfpathcurveto{\pgfqpoint{1.754353in}{0.606282in}}{\pgfqpoint{1.752158in}{0.600982in}}{\pgfqpoint{1.752158in}{0.595457in}}%
\pgfpathcurveto{\pgfqpoint{1.752158in}{0.589932in}}{\pgfqpoint{1.754353in}{0.584633in}}{\pgfqpoint{1.758260in}{0.580726in}}%
\pgfpathcurveto{\pgfqpoint{1.762167in}{0.576819in}}{\pgfqpoint{1.767467in}{0.574624in}}{\pgfqpoint{1.772992in}{0.574624in}}%
\pgfpathclose%
\pgfusepath{stroke,fill}%
\end{pgfscope}%
\begin{pgfscope}%
\pgfpathrectangle{\pgfqpoint{0.562500in}{0.275000in}}{\pgfqpoint{3.487500in}{1.925000in}}%
\pgfusepath{clip}%
\pgfsetbuttcap%
\pgfsetroundjoin%
\definecolor{currentfill}{rgb}{0.000000,0.000000,0.000000}%
\pgfsetfillcolor{currentfill}%
\pgfsetlinewidth{1.003750pt}%
\definecolor{currentstroke}{rgb}{0.000000,0.000000,0.000000}%
\pgfsetstrokecolor{currentstroke}%
\pgfsetdash{}{0pt}%
\pgfpathmoveto{\pgfqpoint{1.772992in}{0.503804in}}%
\pgfpathcurveto{\pgfqpoint{1.778517in}{0.503804in}}{\pgfqpoint{1.783816in}{0.505999in}}{\pgfqpoint{1.787723in}{0.509906in}}%
\pgfpathcurveto{\pgfqpoint{1.791630in}{0.513813in}}{\pgfqpoint{1.793825in}{0.519112in}}{\pgfqpoint{1.793825in}{0.524637in}}%
\pgfpathcurveto{\pgfqpoint{1.793825in}{0.530162in}}{\pgfqpoint{1.791630in}{0.535462in}}{\pgfqpoint{1.787723in}{0.539369in}}%
\pgfpathcurveto{\pgfqpoint{1.783816in}{0.543275in}}{\pgfqpoint{1.778517in}{0.545471in}}{\pgfqpoint{1.772992in}{0.545471in}}%
\pgfpathcurveto{\pgfqpoint{1.767467in}{0.545471in}}{\pgfqpoint{1.762167in}{0.543275in}}{\pgfqpoint{1.758260in}{0.539369in}}%
\pgfpathcurveto{\pgfqpoint{1.754353in}{0.535462in}}{\pgfqpoint{1.752158in}{0.530162in}}{\pgfqpoint{1.752158in}{0.524637in}}%
\pgfpathcurveto{\pgfqpoint{1.752158in}{0.519112in}}{\pgfqpoint{1.754353in}{0.513813in}}{\pgfqpoint{1.758260in}{0.509906in}}%
\pgfpathcurveto{\pgfqpoint{1.762167in}{0.505999in}}{\pgfqpoint{1.767467in}{0.503804in}}{\pgfqpoint{1.772992in}{0.503804in}}%
\pgfpathclose%
\pgfusepath{stroke,fill}%
\end{pgfscope}%
\begin{pgfscope}%
\pgfpathrectangle{\pgfqpoint{0.562500in}{0.275000in}}{\pgfqpoint{3.487500in}{1.925000in}}%
\pgfusepath{clip}%
\pgfsetbuttcap%
\pgfsetroundjoin%
\definecolor{currentfill}{rgb}{0.000000,0.000000,0.000000}%
\pgfsetfillcolor{currentfill}%
\pgfsetlinewidth{1.003750pt}%
\definecolor{currentstroke}{rgb}{0.000000,0.000000,0.000000}%
\pgfsetstrokecolor{currentstroke}%
\pgfsetdash{}{0pt}%
\pgfpathmoveto{\pgfqpoint{1.772992in}{0.544272in}}%
\pgfpathcurveto{\pgfqpoint{1.778517in}{0.544272in}}{\pgfqpoint{1.783816in}{0.546468in}}{\pgfqpoint{1.787723in}{0.550374in}}%
\pgfpathcurveto{\pgfqpoint{1.791630in}{0.554281in}}{\pgfqpoint{1.793825in}{0.559581in}}{\pgfqpoint{1.793825in}{0.565106in}}%
\pgfpathcurveto{\pgfqpoint{1.793825in}{0.570631in}}{\pgfqpoint{1.791630in}{0.575930in}}{\pgfqpoint{1.787723in}{0.579837in}}%
\pgfpathcurveto{\pgfqpoint{1.783816in}{0.583744in}}{\pgfqpoint{1.778517in}{0.585939in}}{\pgfqpoint{1.772992in}{0.585939in}}%
\pgfpathcurveto{\pgfqpoint{1.767467in}{0.585939in}}{\pgfqpoint{1.762167in}{0.583744in}}{\pgfqpoint{1.758260in}{0.579837in}}%
\pgfpathcurveto{\pgfqpoint{1.754353in}{0.575930in}}{\pgfqpoint{1.752158in}{0.570631in}}{\pgfqpoint{1.752158in}{0.565106in}}%
\pgfpathcurveto{\pgfqpoint{1.752158in}{0.559581in}}{\pgfqpoint{1.754353in}{0.554281in}}{\pgfqpoint{1.758260in}{0.550374in}}%
\pgfpathcurveto{\pgfqpoint{1.762167in}{0.546468in}}{\pgfqpoint{1.767467in}{0.544272in}}{\pgfqpoint{1.772992in}{0.544272in}}%
\pgfpathclose%
\pgfusepath{stroke,fill}%
\end{pgfscope}%
\begin{pgfscope}%
\pgfpathrectangle{\pgfqpoint{0.562500in}{0.275000in}}{\pgfqpoint{3.487500in}{1.925000in}}%
\pgfusepath{clip}%
\pgfsetbuttcap%
\pgfsetroundjoin%
\definecolor{currentfill}{rgb}{0.000000,0.000000,0.000000}%
\pgfsetfillcolor{currentfill}%
\pgfsetlinewidth{1.003750pt}%
\definecolor{currentstroke}{rgb}{0.000000,0.000000,0.000000}%
\pgfsetstrokecolor{currentstroke}%
\pgfsetdash{}{0pt}%
\pgfpathmoveto{\pgfqpoint{1.772992in}{0.524038in}}%
\pgfpathcurveto{\pgfqpoint{1.778517in}{0.524038in}}{\pgfqpoint{1.783816in}{0.526233in}}{\pgfqpoint{1.787723in}{0.530140in}}%
\pgfpathcurveto{\pgfqpoint{1.791630in}{0.534047in}}{\pgfqpoint{1.793825in}{0.539346in}}{\pgfqpoint{1.793825in}{0.544872in}}%
\pgfpathcurveto{\pgfqpoint{1.793825in}{0.550397in}}{\pgfqpoint{1.791630in}{0.555696in}}{\pgfqpoint{1.787723in}{0.559603in}}%
\pgfpathcurveto{\pgfqpoint{1.783816in}{0.563510in}}{\pgfqpoint{1.778517in}{0.565705in}}{\pgfqpoint{1.772992in}{0.565705in}}%
\pgfpathcurveto{\pgfqpoint{1.767467in}{0.565705in}}{\pgfqpoint{1.762167in}{0.563510in}}{\pgfqpoint{1.758260in}{0.559603in}}%
\pgfpathcurveto{\pgfqpoint{1.754353in}{0.555696in}}{\pgfqpoint{1.752158in}{0.550397in}}{\pgfqpoint{1.752158in}{0.544872in}}%
\pgfpathcurveto{\pgfqpoint{1.752158in}{0.539346in}}{\pgfqpoint{1.754353in}{0.534047in}}{\pgfqpoint{1.758260in}{0.530140in}}%
\pgfpathcurveto{\pgfqpoint{1.762167in}{0.526233in}}{\pgfqpoint{1.767467in}{0.524038in}}{\pgfqpoint{1.772992in}{0.524038in}}%
\pgfpathclose%
\pgfusepath{stroke,fill}%
\end{pgfscope}%
\begin{pgfscope}%
\pgfpathrectangle{\pgfqpoint{0.562500in}{0.275000in}}{\pgfqpoint{3.487500in}{1.925000in}}%
\pgfusepath{clip}%
\pgfsetbuttcap%
\pgfsetroundjoin%
\definecolor{currentfill}{rgb}{0.000000,0.000000,0.000000}%
\pgfsetfillcolor{currentfill}%
\pgfsetlinewidth{1.003750pt}%
\definecolor{currentstroke}{rgb}{0.000000,0.000000,0.000000}%
\pgfsetstrokecolor{currentstroke}%
\pgfsetdash{}{0pt}%
\pgfpathmoveto{\pgfqpoint{1.772992in}{0.478511in}}%
\pgfpathcurveto{\pgfqpoint{1.778517in}{0.478511in}}{\pgfqpoint{1.783816in}{0.480706in}}{\pgfqpoint{1.787723in}{0.484613in}}%
\pgfpathcurveto{\pgfqpoint{1.791630in}{0.488520in}}{\pgfqpoint{1.793825in}{0.493819in}}{\pgfqpoint{1.793825in}{0.499344in}}%
\pgfpathcurveto{\pgfqpoint{1.793825in}{0.504870in}}{\pgfqpoint{1.791630in}{0.510169in}}{\pgfqpoint{1.787723in}{0.514076in}}%
\pgfpathcurveto{\pgfqpoint{1.783816in}{0.517983in}}{\pgfqpoint{1.778517in}{0.520178in}}{\pgfqpoint{1.772992in}{0.520178in}}%
\pgfpathcurveto{\pgfqpoint{1.767467in}{0.520178in}}{\pgfqpoint{1.762167in}{0.517983in}}{\pgfqpoint{1.758260in}{0.514076in}}%
\pgfpathcurveto{\pgfqpoint{1.754353in}{0.510169in}}{\pgfqpoint{1.752158in}{0.504870in}}{\pgfqpoint{1.752158in}{0.499344in}}%
\pgfpathcurveto{\pgfqpoint{1.752158in}{0.493819in}}{\pgfqpoint{1.754353in}{0.488520in}}{\pgfqpoint{1.758260in}{0.484613in}}%
\pgfpathcurveto{\pgfqpoint{1.762167in}{0.480706in}}{\pgfqpoint{1.767467in}{0.478511in}}{\pgfqpoint{1.772992in}{0.478511in}}%
\pgfpathclose%
\pgfusepath{stroke,fill}%
\end{pgfscope}%
\begin{pgfscope}%
\pgfpathrectangle{\pgfqpoint{0.562500in}{0.275000in}}{\pgfqpoint{3.487500in}{1.925000in}}%
\pgfusepath{clip}%
\pgfsetbuttcap%
\pgfsetroundjoin%
\definecolor{currentfill}{rgb}{0.000000,0.000000,0.000000}%
\pgfsetfillcolor{currentfill}%
\pgfsetlinewidth{1.003750pt}%
\definecolor{currentstroke}{rgb}{0.000000,0.000000,0.000000}%
\pgfsetstrokecolor{currentstroke}%
\pgfsetdash{}{0pt}%
\pgfpathmoveto{\pgfqpoint{1.772992in}{0.630268in}}%
\pgfpathcurveto{\pgfqpoint{1.778517in}{0.630268in}}{\pgfqpoint{1.783816in}{0.632463in}}{\pgfqpoint{1.787723in}{0.636370in}}%
\pgfpathcurveto{\pgfqpoint{1.791630in}{0.640277in}}{\pgfqpoint{1.793825in}{0.645576in}}{\pgfqpoint{1.793825in}{0.651101in}}%
\pgfpathcurveto{\pgfqpoint{1.793825in}{0.656627in}}{\pgfqpoint{1.791630in}{0.661926in}}{\pgfqpoint{1.787723in}{0.665833in}}%
\pgfpathcurveto{\pgfqpoint{1.783816in}{0.669740in}}{\pgfqpoint{1.778517in}{0.671935in}}{\pgfqpoint{1.772992in}{0.671935in}}%
\pgfpathcurveto{\pgfqpoint{1.767467in}{0.671935in}}{\pgfqpoint{1.762167in}{0.669740in}}{\pgfqpoint{1.758260in}{0.665833in}}%
\pgfpathcurveto{\pgfqpoint{1.754353in}{0.661926in}}{\pgfqpoint{1.752158in}{0.656627in}}{\pgfqpoint{1.752158in}{0.651101in}}%
\pgfpathcurveto{\pgfqpoint{1.752158in}{0.645576in}}{\pgfqpoint{1.754353in}{0.640277in}}{\pgfqpoint{1.758260in}{0.636370in}}%
\pgfpathcurveto{\pgfqpoint{1.762167in}{0.632463in}}{\pgfqpoint{1.767467in}{0.630268in}}{\pgfqpoint{1.772992in}{0.630268in}}%
\pgfpathclose%
\pgfusepath{stroke,fill}%
\end{pgfscope}%
\begin{pgfscope}%
\pgfpathrectangle{\pgfqpoint{0.562500in}{0.275000in}}{\pgfqpoint{3.487500in}{1.925000in}}%
\pgfusepath{clip}%
\pgfsetbuttcap%
\pgfsetroundjoin%
\definecolor{currentfill}{rgb}{0.000000,0.000000,0.000000}%
\pgfsetfillcolor{currentfill}%
\pgfsetlinewidth{1.003750pt}%
\definecolor{currentstroke}{rgb}{0.000000,0.000000,0.000000}%
\pgfsetstrokecolor{currentstroke}%
\pgfsetdash{}{0pt}%
\pgfpathmoveto{\pgfqpoint{1.772992in}{0.584741in}}%
\pgfpathcurveto{\pgfqpoint{1.778517in}{0.584741in}}{\pgfqpoint{1.783816in}{0.586936in}}{\pgfqpoint{1.787723in}{0.590843in}}%
\pgfpathcurveto{\pgfqpoint{1.791630in}{0.594750in}}{\pgfqpoint{1.793825in}{0.600049in}}{\pgfqpoint{1.793825in}{0.605574in}}%
\pgfpathcurveto{\pgfqpoint{1.793825in}{0.611099in}}{\pgfqpoint{1.791630in}{0.616399in}}{\pgfqpoint{1.787723in}{0.620306in}}%
\pgfpathcurveto{\pgfqpoint{1.783816in}{0.624213in}}{\pgfqpoint{1.778517in}{0.626408in}}{\pgfqpoint{1.772992in}{0.626408in}}%
\pgfpathcurveto{\pgfqpoint{1.767467in}{0.626408in}}{\pgfqpoint{1.762167in}{0.624213in}}{\pgfqpoint{1.758260in}{0.620306in}}%
\pgfpathcurveto{\pgfqpoint{1.754353in}{0.616399in}}{\pgfqpoint{1.752158in}{0.611099in}}{\pgfqpoint{1.752158in}{0.605574in}}%
\pgfpathcurveto{\pgfqpoint{1.752158in}{0.600049in}}{\pgfqpoint{1.754353in}{0.594750in}}{\pgfqpoint{1.758260in}{0.590843in}}%
\pgfpathcurveto{\pgfqpoint{1.762167in}{0.586936in}}{\pgfqpoint{1.767467in}{0.584741in}}{\pgfqpoint{1.772992in}{0.584741in}}%
\pgfpathclose%
\pgfusepath{stroke,fill}%
\end{pgfscope}%
\begin{pgfscope}%
\pgfpathrectangle{\pgfqpoint{0.562500in}{0.275000in}}{\pgfqpoint{3.487500in}{1.925000in}}%
\pgfusepath{clip}%
\pgfsetbuttcap%
\pgfsetroundjoin%
\definecolor{currentfill}{rgb}{0.000000,0.000000,0.000000}%
\pgfsetfillcolor{currentfill}%
\pgfsetlinewidth{1.003750pt}%
\definecolor{currentstroke}{rgb}{0.000000,0.000000,0.000000}%
\pgfsetstrokecolor{currentstroke}%
\pgfsetdash{}{0pt}%
\pgfpathmoveto{\pgfqpoint{1.772992in}{0.518980in}}%
\pgfpathcurveto{\pgfqpoint{1.778517in}{0.518980in}}{\pgfqpoint{1.783816in}{0.521175in}}{\pgfqpoint{1.787723in}{0.525082in}}%
\pgfpathcurveto{\pgfqpoint{1.791630in}{0.528988in}}{\pgfqpoint{1.793825in}{0.534288in}}{\pgfqpoint{1.793825in}{0.539813in}}%
\pgfpathcurveto{\pgfqpoint{1.793825in}{0.545338in}}{\pgfqpoint{1.791630in}{0.550638in}}{\pgfqpoint{1.787723in}{0.554544in}}%
\pgfpathcurveto{\pgfqpoint{1.783816in}{0.558451in}}{\pgfqpoint{1.778517in}{0.560646in}}{\pgfqpoint{1.772992in}{0.560646in}}%
\pgfpathcurveto{\pgfqpoint{1.767467in}{0.560646in}}{\pgfqpoint{1.762167in}{0.558451in}}{\pgfqpoint{1.758260in}{0.554544in}}%
\pgfpathcurveto{\pgfqpoint{1.754353in}{0.550638in}}{\pgfqpoint{1.752158in}{0.545338in}}{\pgfqpoint{1.752158in}{0.539813in}}%
\pgfpathcurveto{\pgfqpoint{1.752158in}{0.534288in}}{\pgfqpoint{1.754353in}{0.528988in}}{\pgfqpoint{1.758260in}{0.525082in}}%
\pgfpathcurveto{\pgfqpoint{1.762167in}{0.521175in}}{\pgfqpoint{1.767467in}{0.518980in}}{\pgfqpoint{1.772992in}{0.518980in}}%
\pgfpathclose%
\pgfusepath{stroke,fill}%
\end{pgfscope}%
\begin{pgfscope}%
\pgfpathrectangle{\pgfqpoint{0.562500in}{0.275000in}}{\pgfqpoint{3.487500in}{1.925000in}}%
\pgfusepath{clip}%
\pgfsetbuttcap%
\pgfsetroundjoin%
\definecolor{currentfill}{rgb}{0.000000,0.000000,0.000000}%
\pgfsetfillcolor{currentfill}%
\pgfsetlinewidth{1.003750pt}%
\definecolor{currentstroke}{rgb}{0.000000,0.000000,0.000000}%
\pgfsetstrokecolor{currentstroke}%
\pgfsetdash{}{0pt}%
\pgfpathmoveto{\pgfqpoint{1.772992in}{0.539214in}}%
\pgfpathcurveto{\pgfqpoint{1.778517in}{0.539214in}}{\pgfqpoint{1.783816in}{0.541409in}}{\pgfqpoint{1.787723in}{0.545316in}}%
\pgfpathcurveto{\pgfqpoint{1.791630in}{0.549223in}}{\pgfqpoint{1.793825in}{0.554522in}}{\pgfqpoint{1.793825in}{0.560047in}}%
\pgfpathcurveto{\pgfqpoint{1.793825in}{0.565572in}}{\pgfqpoint{1.791630in}{0.570872in}}{\pgfqpoint{1.787723in}{0.574779in}}%
\pgfpathcurveto{\pgfqpoint{1.783816in}{0.578685in}}{\pgfqpoint{1.778517in}{0.580881in}}{\pgfqpoint{1.772992in}{0.580881in}}%
\pgfpathcurveto{\pgfqpoint{1.767467in}{0.580881in}}{\pgfqpoint{1.762167in}{0.578685in}}{\pgfqpoint{1.758260in}{0.574779in}}%
\pgfpathcurveto{\pgfqpoint{1.754353in}{0.570872in}}{\pgfqpoint{1.752158in}{0.565572in}}{\pgfqpoint{1.752158in}{0.560047in}}%
\pgfpathcurveto{\pgfqpoint{1.752158in}{0.554522in}}{\pgfqpoint{1.754353in}{0.549223in}}{\pgfqpoint{1.758260in}{0.545316in}}%
\pgfpathcurveto{\pgfqpoint{1.762167in}{0.541409in}}{\pgfqpoint{1.767467in}{0.539214in}}{\pgfqpoint{1.772992in}{0.539214in}}%
\pgfpathclose%
\pgfusepath{stroke,fill}%
\end{pgfscope}%
\begin{pgfscope}%
\pgfpathrectangle{\pgfqpoint{0.562500in}{0.275000in}}{\pgfqpoint{3.487500in}{1.925000in}}%
\pgfusepath{clip}%
\pgfsetbuttcap%
\pgfsetroundjoin%
\definecolor{currentfill}{rgb}{0.000000,0.000000,0.000000}%
\pgfsetfillcolor{currentfill}%
\pgfsetlinewidth{1.003750pt}%
\definecolor{currentstroke}{rgb}{0.000000,0.000000,0.000000}%
\pgfsetstrokecolor{currentstroke}%
\pgfsetdash{}{0pt}%
\pgfpathmoveto{\pgfqpoint{1.772992in}{0.513921in}}%
\pgfpathcurveto{\pgfqpoint{1.778517in}{0.513921in}}{\pgfqpoint{1.783816in}{0.516116in}}{\pgfqpoint{1.787723in}{0.520023in}}%
\pgfpathcurveto{\pgfqpoint{1.791630in}{0.523930in}}{\pgfqpoint{1.793825in}{0.529229in}}{\pgfqpoint{1.793825in}{0.534754in}}%
\pgfpathcurveto{\pgfqpoint{1.793825in}{0.540279in}}{\pgfqpoint{1.791630in}{0.545579in}}{\pgfqpoint{1.787723in}{0.549486in}}%
\pgfpathcurveto{\pgfqpoint{1.783816in}{0.553393in}}{\pgfqpoint{1.778517in}{0.555588in}}{\pgfqpoint{1.772992in}{0.555588in}}%
\pgfpathcurveto{\pgfqpoint{1.767467in}{0.555588in}}{\pgfqpoint{1.762167in}{0.553393in}}{\pgfqpoint{1.758260in}{0.549486in}}%
\pgfpathcurveto{\pgfqpoint{1.754353in}{0.545579in}}{\pgfqpoint{1.752158in}{0.540279in}}{\pgfqpoint{1.752158in}{0.534754in}}%
\pgfpathcurveto{\pgfqpoint{1.752158in}{0.529229in}}{\pgfqpoint{1.754353in}{0.523930in}}{\pgfqpoint{1.758260in}{0.520023in}}%
\pgfpathcurveto{\pgfqpoint{1.762167in}{0.516116in}}{\pgfqpoint{1.767467in}{0.513921in}}{\pgfqpoint{1.772992in}{0.513921in}}%
\pgfpathclose%
\pgfusepath{stroke,fill}%
\end{pgfscope}%
\begin{pgfscope}%
\pgfpathrectangle{\pgfqpoint{0.562500in}{0.275000in}}{\pgfqpoint{3.487500in}{1.925000in}}%
\pgfusepath{clip}%
\pgfsetbuttcap%
\pgfsetroundjoin%
\definecolor{currentfill}{rgb}{0.000000,0.000000,0.000000}%
\pgfsetfillcolor{currentfill}%
\pgfsetlinewidth{1.003750pt}%
\definecolor{currentstroke}{rgb}{0.000000,0.000000,0.000000}%
\pgfsetstrokecolor{currentstroke}%
\pgfsetdash{}{0pt}%
\pgfpathmoveto{\pgfqpoint{1.772992in}{0.458277in}}%
\pgfpathcurveto{\pgfqpoint{1.778517in}{0.458277in}}{\pgfqpoint{1.783816in}{0.460472in}}{\pgfqpoint{1.787723in}{0.464379in}}%
\pgfpathcurveto{\pgfqpoint{1.791630in}{0.468286in}}{\pgfqpoint{1.793825in}{0.473585in}}{\pgfqpoint{1.793825in}{0.479110in}}%
\pgfpathcurveto{\pgfqpoint{1.793825in}{0.484635in}}{\pgfqpoint{1.791630in}{0.489935in}}{\pgfqpoint{1.787723in}{0.493842in}}%
\pgfpathcurveto{\pgfqpoint{1.783816in}{0.497748in}}{\pgfqpoint{1.778517in}{0.499944in}}{\pgfqpoint{1.772992in}{0.499944in}}%
\pgfpathcurveto{\pgfqpoint{1.767467in}{0.499944in}}{\pgfqpoint{1.762167in}{0.497748in}}{\pgfqpoint{1.758260in}{0.493842in}}%
\pgfpathcurveto{\pgfqpoint{1.754353in}{0.489935in}}{\pgfqpoint{1.752158in}{0.484635in}}{\pgfqpoint{1.752158in}{0.479110in}}%
\pgfpathcurveto{\pgfqpoint{1.752158in}{0.473585in}}{\pgfqpoint{1.754353in}{0.468286in}}{\pgfqpoint{1.758260in}{0.464379in}}%
\pgfpathcurveto{\pgfqpoint{1.762167in}{0.460472in}}{\pgfqpoint{1.767467in}{0.458277in}}{\pgfqpoint{1.772992in}{0.458277in}}%
\pgfpathclose%
\pgfusepath{stroke,fill}%
\end{pgfscope}%
\begin{pgfscope}%
\pgfpathrectangle{\pgfqpoint{0.562500in}{0.275000in}}{\pgfqpoint{3.487500in}{1.925000in}}%
\pgfusepath{clip}%
\pgfsetbuttcap%
\pgfsetroundjoin%
\definecolor{currentfill}{rgb}{0.000000,0.000000,0.000000}%
\pgfsetfillcolor{currentfill}%
\pgfsetlinewidth{1.003750pt}%
\definecolor{currentstroke}{rgb}{0.000000,0.000000,0.000000}%
\pgfsetstrokecolor{currentstroke}%
\pgfsetdash{}{0pt}%
\pgfpathmoveto{\pgfqpoint{1.772992in}{0.539214in}}%
\pgfpathcurveto{\pgfqpoint{1.778517in}{0.539214in}}{\pgfqpoint{1.783816in}{0.541409in}}{\pgfqpoint{1.787723in}{0.545316in}}%
\pgfpathcurveto{\pgfqpoint{1.791630in}{0.549223in}}{\pgfqpoint{1.793825in}{0.554522in}}{\pgfqpoint{1.793825in}{0.560047in}}%
\pgfpathcurveto{\pgfqpoint{1.793825in}{0.565572in}}{\pgfqpoint{1.791630in}{0.570872in}}{\pgfqpoint{1.787723in}{0.574779in}}%
\pgfpathcurveto{\pgfqpoint{1.783816in}{0.578685in}}{\pgfqpoint{1.778517in}{0.580881in}}{\pgfqpoint{1.772992in}{0.580881in}}%
\pgfpathcurveto{\pgfqpoint{1.767467in}{0.580881in}}{\pgfqpoint{1.762167in}{0.578685in}}{\pgfqpoint{1.758260in}{0.574779in}}%
\pgfpathcurveto{\pgfqpoint{1.754353in}{0.570872in}}{\pgfqpoint{1.752158in}{0.565572in}}{\pgfqpoint{1.752158in}{0.560047in}}%
\pgfpathcurveto{\pgfqpoint{1.752158in}{0.554522in}}{\pgfqpoint{1.754353in}{0.549223in}}{\pgfqpoint{1.758260in}{0.545316in}}%
\pgfpathcurveto{\pgfqpoint{1.762167in}{0.541409in}}{\pgfqpoint{1.767467in}{0.539214in}}{\pgfqpoint{1.772992in}{0.539214in}}%
\pgfpathclose%
\pgfusepath{stroke,fill}%
\end{pgfscope}%
\begin{pgfscope}%
\pgfpathrectangle{\pgfqpoint{0.562500in}{0.275000in}}{\pgfqpoint{3.487500in}{1.925000in}}%
\pgfusepath{clip}%
\pgfsetbuttcap%
\pgfsetroundjoin%
\definecolor{currentfill}{rgb}{0.000000,0.000000,0.000000}%
\pgfsetfillcolor{currentfill}%
\pgfsetlinewidth{1.003750pt}%
\definecolor{currentstroke}{rgb}{0.000000,0.000000,0.000000}%
\pgfsetstrokecolor{currentstroke}%
\pgfsetdash{}{0pt}%
\pgfpathmoveto{\pgfqpoint{1.772992in}{0.604975in}}%
\pgfpathcurveto{\pgfqpoint{1.778517in}{0.604975in}}{\pgfqpoint{1.783816in}{0.607170in}}{\pgfqpoint{1.787723in}{0.611077in}}%
\pgfpathcurveto{\pgfqpoint{1.791630in}{0.614984in}}{\pgfqpoint{1.793825in}{0.620284in}}{\pgfqpoint{1.793825in}{0.625809in}}%
\pgfpathcurveto{\pgfqpoint{1.793825in}{0.631334in}}{\pgfqpoint{1.791630in}{0.636633in}}{\pgfqpoint{1.787723in}{0.640540in}}%
\pgfpathcurveto{\pgfqpoint{1.783816in}{0.644447in}}{\pgfqpoint{1.778517in}{0.646642in}}{\pgfqpoint{1.772992in}{0.646642in}}%
\pgfpathcurveto{\pgfqpoint{1.767467in}{0.646642in}}{\pgfqpoint{1.762167in}{0.644447in}}{\pgfqpoint{1.758260in}{0.640540in}}%
\pgfpathcurveto{\pgfqpoint{1.754353in}{0.636633in}}{\pgfqpoint{1.752158in}{0.631334in}}{\pgfqpoint{1.752158in}{0.625809in}}%
\pgfpathcurveto{\pgfqpoint{1.752158in}{0.620284in}}{\pgfqpoint{1.754353in}{0.614984in}}{\pgfqpoint{1.758260in}{0.611077in}}%
\pgfpathcurveto{\pgfqpoint{1.762167in}{0.607170in}}{\pgfqpoint{1.767467in}{0.604975in}}{\pgfqpoint{1.772992in}{0.604975in}}%
\pgfpathclose%
\pgfusepath{stroke,fill}%
\end{pgfscope}%
\begin{pgfscope}%
\pgfpathrectangle{\pgfqpoint{0.562500in}{0.275000in}}{\pgfqpoint{3.487500in}{1.925000in}}%
\pgfusepath{clip}%
\pgfsetbuttcap%
\pgfsetroundjoin%
\definecolor{currentfill}{rgb}{0.000000,0.000000,0.000000}%
\pgfsetfillcolor{currentfill}%
\pgfsetlinewidth{1.003750pt}%
\definecolor{currentstroke}{rgb}{0.000000,0.000000,0.000000}%
\pgfsetstrokecolor{currentstroke}%
\pgfsetdash{}{0pt}%
\pgfpathmoveto{\pgfqpoint{1.772992in}{0.544272in}}%
\pgfpathcurveto{\pgfqpoint{1.778517in}{0.544272in}}{\pgfqpoint{1.783816in}{0.546468in}}{\pgfqpoint{1.787723in}{0.550374in}}%
\pgfpathcurveto{\pgfqpoint{1.791630in}{0.554281in}}{\pgfqpoint{1.793825in}{0.559581in}}{\pgfqpoint{1.793825in}{0.565106in}}%
\pgfpathcurveto{\pgfqpoint{1.793825in}{0.570631in}}{\pgfqpoint{1.791630in}{0.575930in}}{\pgfqpoint{1.787723in}{0.579837in}}%
\pgfpathcurveto{\pgfqpoint{1.783816in}{0.583744in}}{\pgfqpoint{1.778517in}{0.585939in}}{\pgfqpoint{1.772992in}{0.585939in}}%
\pgfpathcurveto{\pgfqpoint{1.767467in}{0.585939in}}{\pgfqpoint{1.762167in}{0.583744in}}{\pgfqpoint{1.758260in}{0.579837in}}%
\pgfpathcurveto{\pgfqpoint{1.754353in}{0.575930in}}{\pgfqpoint{1.752158in}{0.570631in}}{\pgfqpoint{1.752158in}{0.565106in}}%
\pgfpathcurveto{\pgfqpoint{1.752158in}{0.559581in}}{\pgfqpoint{1.754353in}{0.554281in}}{\pgfqpoint{1.758260in}{0.550374in}}%
\pgfpathcurveto{\pgfqpoint{1.762167in}{0.546468in}}{\pgfqpoint{1.767467in}{0.544272in}}{\pgfqpoint{1.772992in}{0.544272in}}%
\pgfpathclose%
\pgfusepath{stroke,fill}%
\end{pgfscope}%
\begin{pgfscope}%
\pgfpathrectangle{\pgfqpoint{0.562500in}{0.275000in}}{\pgfqpoint{3.487500in}{1.925000in}}%
\pgfusepath{clip}%
\pgfsetbuttcap%
\pgfsetroundjoin%
\definecolor{currentfill}{rgb}{0.000000,0.000000,0.000000}%
\pgfsetfillcolor{currentfill}%
\pgfsetlinewidth{1.003750pt}%
\definecolor{currentstroke}{rgb}{0.000000,0.000000,0.000000}%
\pgfsetstrokecolor{currentstroke}%
\pgfsetdash{}{0pt}%
\pgfpathmoveto{\pgfqpoint{1.772992in}{0.498745in}}%
\pgfpathcurveto{\pgfqpoint{1.778517in}{0.498745in}}{\pgfqpoint{1.783816in}{0.500941in}}{\pgfqpoint{1.787723in}{0.504847in}}%
\pgfpathcurveto{\pgfqpoint{1.791630in}{0.508754in}}{\pgfqpoint{1.793825in}{0.514054in}}{\pgfqpoint{1.793825in}{0.519579in}}%
\pgfpathcurveto{\pgfqpoint{1.793825in}{0.525104in}}{\pgfqpoint{1.791630in}{0.530403in}}{\pgfqpoint{1.787723in}{0.534310in}}%
\pgfpathcurveto{\pgfqpoint{1.783816in}{0.538217in}}{\pgfqpoint{1.778517in}{0.540412in}}{\pgfqpoint{1.772992in}{0.540412in}}%
\pgfpathcurveto{\pgfqpoint{1.767467in}{0.540412in}}{\pgfqpoint{1.762167in}{0.538217in}}{\pgfqpoint{1.758260in}{0.534310in}}%
\pgfpathcurveto{\pgfqpoint{1.754353in}{0.530403in}}{\pgfqpoint{1.752158in}{0.525104in}}{\pgfqpoint{1.752158in}{0.519579in}}%
\pgfpathcurveto{\pgfqpoint{1.752158in}{0.514054in}}{\pgfqpoint{1.754353in}{0.508754in}}{\pgfqpoint{1.758260in}{0.504847in}}%
\pgfpathcurveto{\pgfqpoint{1.762167in}{0.500941in}}{\pgfqpoint{1.767467in}{0.498745in}}{\pgfqpoint{1.772992in}{0.498745in}}%
\pgfpathclose%
\pgfusepath{stroke,fill}%
\end{pgfscope}%
\begin{pgfscope}%
\pgfpathrectangle{\pgfqpoint{0.562500in}{0.275000in}}{\pgfqpoint{3.487500in}{1.925000in}}%
\pgfusepath{clip}%
\pgfsetbuttcap%
\pgfsetroundjoin%
\definecolor{currentfill}{rgb}{0.000000,0.000000,0.000000}%
\pgfsetfillcolor{currentfill}%
\pgfsetlinewidth{1.003750pt}%
\definecolor{currentstroke}{rgb}{0.000000,0.000000,0.000000}%
\pgfsetstrokecolor{currentstroke}%
\pgfsetdash{}{0pt}%
\pgfpathmoveto{\pgfqpoint{1.772992in}{0.569565in}}%
\pgfpathcurveto{\pgfqpoint{1.778517in}{0.569565in}}{\pgfqpoint{1.783816in}{0.571760in}}{\pgfqpoint{1.787723in}{0.575667in}}%
\pgfpathcurveto{\pgfqpoint{1.791630in}{0.579574in}}{\pgfqpoint{1.793825in}{0.584874in}}{\pgfqpoint{1.793825in}{0.590399in}}%
\pgfpathcurveto{\pgfqpoint{1.793825in}{0.595924in}}{\pgfqpoint{1.791630in}{0.601223in}}{\pgfqpoint{1.787723in}{0.605130in}}%
\pgfpathcurveto{\pgfqpoint{1.783816in}{0.609037in}}{\pgfqpoint{1.778517in}{0.611232in}}{\pgfqpoint{1.772992in}{0.611232in}}%
\pgfpathcurveto{\pgfqpoint{1.767467in}{0.611232in}}{\pgfqpoint{1.762167in}{0.609037in}}{\pgfqpoint{1.758260in}{0.605130in}}%
\pgfpathcurveto{\pgfqpoint{1.754353in}{0.601223in}}{\pgfqpoint{1.752158in}{0.595924in}}{\pgfqpoint{1.752158in}{0.590399in}}%
\pgfpathcurveto{\pgfqpoint{1.752158in}{0.584874in}}{\pgfqpoint{1.754353in}{0.579574in}}{\pgfqpoint{1.758260in}{0.575667in}}%
\pgfpathcurveto{\pgfqpoint{1.762167in}{0.571760in}}{\pgfqpoint{1.767467in}{0.569565in}}{\pgfqpoint{1.772992in}{0.569565in}}%
\pgfpathclose%
\pgfusepath{stroke,fill}%
\end{pgfscope}%
\begin{pgfscope}%
\pgfpathrectangle{\pgfqpoint{0.562500in}{0.275000in}}{\pgfqpoint{3.487500in}{1.925000in}}%
\pgfusepath{clip}%
\pgfsetbuttcap%
\pgfsetroundjoin%
\definecolor{currentfill}{rgb}{0.000000,0.000000,0.000000}%
\pgfsetfillcolor{currentfill}%
\pgfsetlinewidth{1.003750pt}%
\definecolor{currentstroke}{rgb}{0.000000,0.000000,0.000000}%
\pgfsetstrokecolor{currentstroke}%
\pgfsetdash{}{0pt}%
\pgfpathmoveto{\pgfqpoint{1.772992in}{0.518980in}}%
\pgfpathcurveto{\pgfqpoint{1.778517in}{0.518980in}}{\pgfqpoint{1.783816in}{0.521175in}}{\pgfqpoint{1.787723in}{0.525082in}}%
\pgfpathcurveto{\pgfqpoint{1.791630in}{0.528988in}}{\pgfqpoint{1.793825in}{0.534288in}}{\pgfqpoint{1.793825in}{0.539813in}}%
\pgfpathcurveto{\pgfqpoint{1.793825in}{0.545338in}}{\pgfqpoint{1.791630in}{0.550638in}}{\pgfqpoint{1.787723in}{0.554544in}}%
\pgfpathcurveto{\pgfqpoint{1.783816in}{0.558451in}}{\pgfqpoint{1.778517in}{0.560646in}}{\pgfqpoint{1.772992in}{0.560646in}}%
\pgfpathcurveto{\pgfqpoint{1.767467in}{0.560646in}}{\pgfqpoint{1.762167in}{0.558451in}}{\pgfqpoint{1.758260in}{0.554544in}}%
\pgfpathcurveto{\pgfqpoint{1.754353in}{0.550638in}}{\pgfqpoint{1.752158in}{0.545338in}}{\pgfqpoint{1.752158in}{0.539813in}}%
\pgfpathcurveto{\pgfqpoint{1.752158in}{0.534288in}}{\pgfqpoint{1.754353in}{0.528988in}}{\pgfqpoint{1.758260in}{0.525082in}}%
\pgfpathcurveto{\pgfqpoint{1.762167in}{0.521175in}}{\pgfqpoint{1.767467in}{0.518980in}}{\pgfqpoint{1.772992in}{0.518980in}}%
\pgfpathclose%
\pgfusepath{stroke,fill}%
\end{pgfscope}%
\begin{pgfscope}%
\pgfpathrectangle{\pgfqpoint{0.562500in}{0.275000in}}{\pgfqpoint{3.487500in}{1.925000in}}%
\pgfusepath{clip}%
\pgfsetbuttcap%
\pgfsetroundjoin%
\definecolor{currentfill}{rgb}{0.000000,0.000000,0.000000}%
\pgfsetfillcolor{currentfill}%
\pgfsetlinewidth{1.003750pt}%
\definecolor{currentstroke}{rgb}{0.000000,0.000000,0.000000}%
\pgfsetstrokecolor{currentstroke}%
\pgfsetdash{}{0pt}%
\pgfpathmoveto{\pgfqpoint{1.772992in}{0.513921in}}%
\pgfpathcurveto{\pgfqpoint{1.778517in}{0.513921in}}{\pgfqpoint{1.783816in}{0.516116in}}{\pgfqpoint{1.787723in}{0.520023in}}%
\pgfpathcurveto{\pgfqpoint{1.791630in}{0.523930in}}{\pgfqpoint{1.793825in}{0.529229in}}{\pgfqpoint{1.793825in}{0.534754in}}%
\pgfpathcurveto{\pgfqpoint{1.793825in}{0.540279in}}{\pgfqpoint{1.791630in}{0.545579in}}{\pgfqpoint{1.787723in}{0.549486in}}%
\pgfpathcurveto{\pgfqpoint{1.783816in}{0.553393in}}{\pgfqpoint{1.778517in}{0.555588in}}{\pgfqpoint{1.772992in}{0.555588in}}%
\pgfpathcurveto{\pgfqpoint{1.767467in}{0.555588in}}{\pgfqpoint{1.762167in}{0.553393in}}{\pgfqpoint{1.758260in}{0.549486in}}%
\pgfpathcurveto{\pgfqpoint{1.754353in}{0.545579in}}{\pgfqpoint{1.752158in}{0.540279in}}{\pgfqpoint{1.752158in}{0.534754in}}%
\pgfpathcurveto{\pgfqpoint{1.752158in}{0.529229in}}{\pgfqpoint{1.754353in}{0.523930in}}{\pgfqpoint{1.758260in}{0.520023in}}%
\pgfpathcurveto{\pgfqpoint{1.762167in}{0.516116in}}{\pgfqpoint{1.767467in}{0.513921in}}{\pgfqpoint{1.772992in}{0.513921in}}%
\pgfpathclose%
\pgfusepath{stroke,fill}%
\end{pgfscope}%
\begin{pgfscope}%
\pgfpathrectangle{\pgfqpoint{0.562500in}{0.275000in}}{\pgfqpoint{3.487500in}{1.925000in}}%
\pgfusepath{clip}%
\pgfsetbuttcap%
\pgfsetroundjoin%
\definecolor{currentfill}{rgb}{0.000000,0.000000,0.000000}%
\pgfsetfillcolor{currentfill}%
\pgfsetlinewidth{1.003750pt}%
\definecolor{currentstroke}{rgb}{0.000000,0.000000,0.000000}%
\pgfsetstrokecolor{currentstroke}%
\pgfsetdash{}{0pt}%
\pgfpathmoveto{\pgfqpoint{1.772992in}{0.513921in}}%
\pgfpathcurveto{\pgfqpoint{1.778517in}{0.513921in}}{\pgfqpoint{1.783816in}{0.516116in}}{\pgfqpoint{1.787723in}{0.520023in}}%
\pgfpathcurveto{\pgfqpoint{1.791630in}{0.523930in}}{\pgfqpoint{1.793825in}{0.529229in}}{\pgfqpoint{1.793825in}{0.534754in}}%
\pgfpathcurveto{\pgfqpoint{1.793825in}{0.540279in}}{\pgfqpoint{1.791630in}{0.545579in}}{\pgfqpoint{1.787723in}{0.549486in}}%
\pgfpathcurveto{\pgfqpoint{1.783816in}{0.553393in}}{\pgfqpoint{1.778517in}{0.555588in}}{\pgfqpoint{1.772992in}{0.555588in}}%
\pgfpathcurveto{\pgfqpoint{1.767467in}{0.555588in}}{\pgfqpoint{1.762167in}{0.553393in}}{\pgfqpoint{1.758260in}{0.549486in}}%
\pgfpathcurveto{\pgfqpoint{1.754353in}{0.545579in}}{\pgfqpoint{1.752158in}{0.540279in}}{\pgfqpoint{1.752158in}{0.534754in}}%
\pgfpathcurveto{\pgfqpoint{1.752158in}{0.529229in}}{\pgfqpoint{1.754353in}{0.523930in}}{\pgfqpoint{1.758260in}{0.520023in}}%
\pgfpathcurveto{\pgfqpoint{1.762167in}{0.516116in}}{\pgfqpoint{1.767467in}{0.513921in}}{\pgfqpoint{1.772992in}{0.513921in}}%
\pgfpathclose%
\pgfusepath{stroke,fill}%
\end{pgfscope}%
\begin{pgfscope}%
\pgfpathrectangle{\pgfqpoint{0.562500in}{0.275000in}}{\pgfqpoint{3.487500in}{1.925000in}}%
\pgfusepath{clip}%
\pgfsetbuttcap%
\pgfsetroundjoin%
\definecolor{currentfill}{rgb}{0.000000,0.000000,0.000000}%
\pgfsetfillcolor{currentfill}%
\pgfsetlinewidth{1.003750pt}%
\definecolor{currentstroke}{rgb}{0.000000,0.000000,0.000000}%
\pgfsetstrokecolor{currentstroke}%
\pgfsetdash{}{0pt}%
\pgfpathmoveto{\pgfqpoint{1.772992in}{0.604975in}}%
\pgfpathcurveto{\pgfqpoint{1.778517in}{0.604975in}}{\pgfqpoint{1.783816in}{0.607170in}}{\pgfqpoint{1.787723in}{0.611077in}}%
\pgfpathcurveto{\pgfqpoint{1.791630in}{0.614984in}}{\pgfqpoint{1.793825in}{0.620284in}}{\pgfqpoint{1.793825in}{0.625809in}}%
\pgfpathcurveto{\pgfqpoint{1.793825in}{0.631334in}}{\pgfqpoint{1.791630in}{0.636633in}}{\pgfqpoint{1.787723in}{0.640540in}}%
\pgfpathcurveto{\pgfqpoint{1.783816in}{0.644447in}}{\pgfqpoint{1.778517in}{0.646642in}}{\pgfqpoint{1.772992in}{0.646642in}}%
\pgfpathcurveto{\pgfqpoint{1.767467in}{0.646642in}}{\pgfqpoint{1.762167in}{0.644447in}}{\pgfqpoint{1.758260in}{0.640540in}}%
\pgfpathcurveto{\pgfqpoint{1.754353in}{0.636633in}}{\pgfqpoint{1.752158in}{0.631334in}}{\pgfqpoint{1.752158in}{0.625809in}}%
\pgfpathcurveto{\pgfqpoint{1.752158in}{0.620284in}}{\pgfqpoint{1.754353in}{0.614984in}}{\pgfqpoint{1.758260in}{0.611077in}}%
\pgfpathcurveto{\pgfqpoint{1.762167in}{0.607170in}}{\pgfqpoint{1.767467in}{0.604975in}}{\pgfqpoint{1.772992in}{0.604975in}}%
\pgfpathclose%
\pgfusepath{stroke,fill}%
\end{pgfscope}%
\begin{pgfscope}%
\pgfpathrectangle{\pgfqpoint{0.562500in}{0.275000in}}{\pgfqpoint{3.487500in}{1.925000in}}%
\pgfusepath{clip}%
\pgfsetbuttcap%
\pgfsetroundjoin%
\definecolor{currentfill}{rgb}{0.000000,0.000000,0.000000}%
\pgfsetfillcolor{currentfill}%
\pgfsetlinewidth{1.003750pt}%
\definecolor{currentstroke}{rgb}{0.000000,0.000000,0.000000}%
\pgfsetstrokecolor{currentstroke}%
\pgfsetdash{}{0pt}%
\pgfpathmoveto{\pgfqpoint{1.772992in}{0.544272in}}%
\pgfpathcurveto{\pgfqpoint{1.778517in}{0.544272in}}{\pgfqpoint{1.783816in}{0.546468in}}{\pgfqpoint{1.787723in}{0.550374in}}%
\pgfpathcurveto{\pgfqpoint{1.791630in}{0.554281in}}{\pgfqpoint{1.793825in}{0.559581in}}{\pgfqpoint{1.793825in}{0.565106in}}%
\pgfpathcurveto{\pgfqpoint{1.793825in}{0.570631in}}{\pgfqpoint{1.791630in}{0.575930in}}{\pgfqpoint{1.787723in}{0.579837in}}%
\pgfpathcurveto{\pgfqpoint{1.783816in}{0.583744in}}{\pgfqpoint{1.778517in}{0.585939in}}{\pgfqpoint{1.772992in}{0.585939in}}%
\pgfpathcurveto{\pgfqpoint{1.767467in}{0.585939in}}{\pgfqpoint{1.762167in}{0.583744in}}{\pgfqpoint{1.758260in}{0.579837in}}%
\pgfpathcurveto{\pgfqpoint{1.754353in}{0.575930in}}{\pgfqpoint{1.752158in}{0.570631in}}{\pgfqpoint{1.752158in}{0.565106in}}%
\pgfpathcurveto{\pgfqpoint{1.752158in}{0.559581in}}{\pgfqpoint{1.754353in}{0.554281in}}{\pgfqpoint{1.758260in}{0.550374in}}%
\pgfpathcurveto{\pgfqpoint{1.762167in}{0.546468in}}{\pgfqpoint{1.767467in}{0.544272in}}{\pgfqpoint{1.772992in}{0.544272in}}%
\pgfpathclose%
\pgfusepath{stroke,fill}%
\end{pgfscope}%
\begin{pgfscope}%
\pgfpathrectangle{\pgfqpoint{0.562500in}{0.275000in}}{\pgfqpoint{3.487500in}{1.925000in}}%
\pgfusepath{clip}%
\pgfsetbuttcap%
\pgfsetroundjoin%
\definecolor{currentfill}{rgb}{0.000000,0.000000,0.000000}%
\pgfsetfillcolor{currentfill}%
\pgfsetlinewidth{1.003750pt}%
\definecolor{currentstroke}{rgb}{0.000000,0.000000,0.000000}%
\pgfsetstrokecolor{currentstroke}%
\pgfsetdash{}{0pt}%
\pgfpathmoveto{\pgfqpoint{1.772992in}{0.488628in}}%
\pgfpathcurveto{\pgfqpoint{1.778517in}{0.488628in}}{\pgfqpoint{1.783816in}{0.490823in}}{\pgfqpoint{1.787723in}{0.494730in}}%
\pgfpathcurveto{\pgfqpoint{1.791630in}{0.498637in}}{\pgfqpoint{1.793825in}{0.503937in}}{\pgfqpoint{1.793825in}{0.509462in}}%
\pgfpathcurveto{\pgfqpoint{1.793825in}{0.514987in}}{\pgfqpoint{1.791630in}{0.520286in}}{\pgfqpoint{1.787723in}{0.524193in}}%
\pgfpathcurveto{\pgfqpoint{1.783816in}{0.528100in}}{\pgfqpoint{1.778517in}{0.530295in}}{\pgfqpoint{1.772992in}{0.530295in}}%
\pgfpathcurveto{\pgfqpoint{1.767467in}{0.530295in}}{\pgfqpoint{1.762167in}{0.528100in}}{\pgfqpoint{1.758260in}{0.524193in}}%
\pgfpathcurveto{\pgfqpoint{1.754353in}{0.520286in}}{\pgfqpoint{1.752158in}{0.514987in}}{\pgfqpoint{1.752158in}{0.509462in}}%
\pgfpathcurveto{\pgfqpoint{1.752158in}{0.503937in}}{\pgfqpoint{1.754353in}{0.498637in}}{\pgfqpoint{1.758260in}{0.494730in}}%
\pgfpathcurveto{\pgfqpoint{1.762167in}{0.490823in}}{\pgfqpoint{1.767467in}{0.488628in}}{\pgfqpoint{1.772992in}{0.488628in}}%
\pgfpathclose%
\pgfusepath{stroke,fill}%
\end{pgfscope}%
\begin{pgfscope}%
\pgfpathrectangle{\pgfqpoint{0.562500in}{0.275000in}}{\pgfqpoint{3.487500in}{1.925000in}}%
\pgfusepath{clip}%
\pgfsetbuttcap%
\pgfsetroundjoin%
\definecolor{currentfill}{rgb}{0.000000,0.000000,0.000000}%
\pgfsetfillcolor{currentfill}%
\pgfsetlinewidth{1.003750pt}%
\definecolor{currentstroke}{rgb}{0.000000,0.000000,0.000000}%
\pgfsetstrokecolor{currentstroke}%
\pgfsetdash{}{0pt}%
\pgfpathmoveto{\pgfqpoint{1.772992in}{0.524038in}}%
\pgfpathcurveto{\pgfqpoint{1.778517in}{0.524038in}}{\pgfqpoint{1.783816in}{0.526233in}}{\pgfqpoint{1.787723in}{0.530140in}}%
\pgfpathcurveto{\pgfqpoint{1.791630in}{0.534047in}}{\pgfqpoint{1.793825in}{0.539346in}}{\pgfqpoint{1.793825in}{0.544872in}}%
\pgfpathcurveto{\pgfqpoint{1.793825in}{0.550397in}}{\pgfqpoint{1.791630in}{0.555696in}}{\pgfqpoint{1.787723in}{0.559603in}}%
\pgfpathcurveto{\pgfqpoint{1.783816in}{0.563510in}}{\pgfqpoint{1.778517in}{0.565705in}}{\pgfqpoint{1.772992in}{0.565705in}}%
\pgfpathcurveto{\pgfqpoint{1.767467in}{0.565705in}}{\pgfqpoint{1.762167in}{0.563510in}}{\pgfqpoint{1.758260in}{0.559603in}}%
\pgfpathcurveto{\pgfqpoint{1.754353in}{0.555696in}}{\pgfqpoint{1.752158in}{0.550397in}}{\pgfqpoint{1.752158in}{0.544872in}}%
\pgfpathcurveto{\pgfqpoint{1.752158in}{0.539346in}}{\pgfqpoint{1.754353in}{0.534047in}}{\pgfqpoint{1.758260in}{0.530140in}}%
\pgfpathcurveto{\pgfqpoint{1.762167in}{0.526233in}}{\pgfqpoint{1.767467in}{0.524038in}}{\pgfqpoint{1.772992in}{0.524038in}}%
\pgfpathclose%
\pgfusepath{stroke,fill}%
\end{pgfscope}%
\begin{pgfscope}%
\pgfpathrectangle{\pgfqpoint{0.562500in}{0.275000in}}{\pgfqpoint{3.487500in}{1.925000in}}%
\pgfusepath{clip}%
\pgfsetbuttcap%
\pgfsetroundjoin%
\definecolor{currentfill}{rgb}{0.000000,0.000000,0.000000}%
\pgfsetfillcolor{currentfill}%
\pgfsetlinewidth{1.003750pt}%
\definecolor{currentstroke}{rgb}{0.000000,0.000000,0.000000}%
\pgfsetstrokecolor{currentstroke}%
\pgfsetdash{}{0pt}%
\pgfpathmoveto{\pgfqpoint{1.772992in}{0.594858in}}%
\pgfpathcurveto{\pgfqpoint{1.778517in}{0.594858in}}{\pgfqpoint{1.783816in}{0.597053in}}{\pgfqpoint{1.787723in}{0.600960in}}%
\pgfpathcurveto{\pgfqpoint{1.791630in}{0.604867in}}{\pgfqpoint{1.793825in}{0.610166in}}{\pgfqpoint{1.793825in}{0.615691in}}%
\pgfpathcurveto{\pgfqpoint{1.793825in}{0.621217in}}{\pgfqpoint{1.791630in}{0.626516in}}{\pgfqpoint{1.787723in}{0.630423in}}%
\pgfpathcurveto{\pgfqpoint{1.783816in}{0.634330in}}{\pgfqpoint{1.778517in}{0.636525in}}{\pgfqpoint{1.772992in}{0.636525in}}%
\pgfpathcurveto{\pgfqpoint{1.767467in}{0.636525in}}{\pgfqpoint{1.762167in}{0.634330in}}{\pgfqpoint{1.758260in}{0.630423in}}%
\pgfpathcurveto{\pgfqpoint{1.754353in}{0.626516in}}{\pgfqpoint{1.752158in}{0.621217in}}{\pgfqpoint{1.752158in}{0.615691in}}%
\pgfpathcurveto{\pgfqpoint{1.752158in}{0.610166in}}{\pgfqpoint{1.754353in}{0.604867in}}{\pgfqpoint{1.758260in}{0.600960in}}%
\pgfpathcurveto{\pgfqpoint{1.762167in}{0.597053in}}{\pgfqpoint{1.767467in}{0.594858in}}{\pgfqpoint{1.772992in}{0.594858in}}%
\pgfpathclose%
\pgfusepath{stroke,fill}%
\end{pgfscope}%
\begin{pgfscope}%
\pgfpathrectangle{\pgfqpoint{0.562500in}{0.275000in}}{\pgfqpoint{3.487500in}{1.925000in}}%
\pgfusepath{clip}%
\pgfsetbuttcap%
\pgfsetroundjoin%
\definecolor{currentfill}{rgb}{0.000000,0.000000,0.000000}%
\pgfsetfillcolor{currentfill}%
\pgfsetlinewidth{1.003750pt}%
\definecolor{currentstroke}{rgb}{0.000000,0.000000,0.000000}%
\pgfsetstrokecolor{currentstroke}%
\pgfsetdash{}{0pt}%
\pgfpathmoveto{\pgfqpoint{1.772992in}{0.529097in}}%
\pgfpathcurveto{\pgfqpoint{1.778517in}{0.529097in}}{\pgfqpoint{1.783816in}{0.531292in}}{\pgfqpoint{1.787723in}{0.535199in}}%
\pgfpathcurveto{\pgfqpoint{1.791630in}{0.539106in}}{\pgfqpoint{1.793825in}{0.544405in}}{\pgfqpoint{1.793825in}{0.549930in}}%
\pgfpathcurveto{\pgfqpoint{1.793825in}{0.555455in}}{\pgfqpoint{1.791630in}{0.560755in}}{\pgfqpoint{1.787723in}{0.564662in}}%
\pgfpathcurveto{\pgfqpoint{1.783816in}{0.568568in}}{\pgfqpoint{1.778517in}{0.570763in}}{\pgfqpoint{1.772992in}{0.570763in}}%
\pgfpathcurveto{\pgfqpoint{1.767467in}{0.570763in}}{\pgfqpoint{1.762167in}{0.568568in}}{\pgfqpoint{1.758260in}{0.564662in}}%
\pgfpathcurveto{\pgfqpoint{1.754353in}{0.560755in}}{\pgfqpoint{1.752158in}{0.555455in}}{\pgfqpoint{1.752158in}{0.549930in}}%
\pgfpathcurveto{\pgfqpoint{1.752158in}{0.544405in}}{\pgfqpoint{1.754353in}{0.539106in}}{\pgfqpoint{1.758260in}{0.535199in}}%
\pgfpathcurveto{\pgfqpoint{1.762167in}{0.531292in}}{\pgfqpoint{1.767467in}{0.529097in}}{\pgfqpoint{1.772992in}{0.529097in}}%
\pgfpathclose%
\pgfusepath{stroke,fill}%
\end{pgfscope}%
\begin{pgfscope}%
\pgfpathrectangle{\pgfqpoint{0.562500in}{0.275000in}}{\pgfqpoint{3.487500in}{1.925000in}}%
\pgfusepath{clip}%
\pgfsetbuttcap%
\pgfsetroundjoin%
\definecolor{currentfill}{rgb}{0.000000,0.000000,0.000000}%
\pgfsetfillcolor{currentfill}%
\pgfsetlinewidth{1.003750pt}%
\definecolor{currentstroke}{rgb}{0.000000,0.000000,0.000000}%
\pgfsetstrokecolor{currentstroke}%
\pgfsetdash{}{0pt}%
\pgfpathmoveto{\pgfqpoint{1.772992in}{0.569565in}}%
\pgfpathcurveto{\pgfqpoint{1.778517in}{0.569565in}}{\pgfqpoint{1.783816in}{0.571760in}}{\pgfqpoint{1.787723in}{0.575667in}}%
\pgfpathcurveto{\pgfqpoint{1.791630in}{0.579574in}}{\pgfqpoint{1.793825in}{0.584874in}}{\pgfqpoint{1.793825in}{0.590399in}}%
\pgfpathcurveto{\pgfqpoint{1.793825in}{0.595924in}}{\pgfqpoint{1.791630in}{0.601223in}}{\pgfqpoint{1.787723in}{0.605130in}}%
\pgfpathcurveto{\pgfqpoint{1.783816in}{0.609037in}}{\pgfqpoint{1.778517in}{0.611232in}}{\pgfqpoint{1.772992in}{0.611232in}}%
\pgfpathcurveto{\pgfqpoint{1.767467in}{0.611232in}}{\pgfqpoint{1.762167in}{0.609037in}}{\pgfqpoint{1.758260in}{0.605130in}}%
\pgfpathcurveto{\pgfqpoint{1.754353in}{0.601223in}}{\pgfqpoint{1.752158in}{0.595924in}}{\pgfqpoint{1.752158in}{0.590399in}}%
\pgfpathcurveto{\pgfqpoint{1.752158in}{0.584874in}}{\pgfqpoint{1.754353in}{0.579574in}}{\pgfqpoint{1.758260in}{0.575667in}}%
\pgfpathcurveto{\pgfqpoint{1.762167in}{0.571760in}}{\pgfqpoint{1.767467in}{0.569565in}}{\pgfqpoint{1.772992in}{0.569565in}}%
\pgfpathclose%
\pgfusepath{stroke,fill}%
\end{pgfscope}%
\begin{pgfscope}%
\pgfpathrectangle{\pgfqpoint{0.562500in}{0.275000in}}{\pgfqpoint{3.487500in}{1.925000in}}%
\pgfusepath{clip}%
\pgfsetbuttcap%
\pgfsetroundjoin%
\definecolor{currentfill}{rgb}{0.000000,0.000000,0.000000}%
\pgfsetfillcolor{currentfill}%
\pgfsetlinewidth{1.003750pt}%
\definecolor{currentstroke}{rgb}{0.000000,0.000000,0.000000}%
\pgfsetstrokecolor{currentstroke}%
\pgfsetdash{}{0pt}%
\pgfpathmoveto{\pgfqpoint{1.772992in}{0.549331in}}%
\pgfpathcurveto{\pgfqpoint{1.778517in}{0.549331in}}{\pgfqpoint{1.783816in}{0.551526in}}{\pgfqpoint{1.787723in}{0.555433in}}%
\pgfpathcurveto{\pgfqpoint{1.791630in}{0.559340in}}{\pgfqpoint{1.793825in}{0.564639in}}{\pgfqpoint{1.793825in}{0.570164in}}%
\pgfpathcurveto{\pgfqpoint{1.793825in}{0.575689in}}{\pgfqpoint{1.791630in}{0.580989in}}{\pgfqpoint{1.787723in}{0.584896in}}%
\pgfpathcurveto{\pgfqpoint{1.783816in}{0.588803in}}{\pgfqpoint{1.778517in}{0.590998in}}{\pgfqpoint{1.772992in}{0.590998in}}%
\pgfpathcurveto{\pgfqpoint{1.767467in}{0.590998in}}{\pgfqpoint{1.762167in}{0.588803in}}{\pgfqpoint{1.758260in}{0.584896in}}%
\pgfpathcurveto{\pgfqpoint{1.754353in}{0.580989in}}{\pgfqpoint{1.752158in}{0.575689in}}{\pgfqpoint{1.752158in}{0.570164in}}%
\pgfpathcurveto{\pgfqpoint{1.752158in}{0.564639in}}{\pgfqpoint{1.754353in}{0.559340in}}{\pgfqpoint{1.758260in}{0.555433in}}%
\pgfpathcurveto{\pgfqpoint{1.762167in}{0.551526in}}{\pgfqpoint{1.767467in}{0.549331in}}{\pgfqpoint{1.772992in}{0.549331in}}%
\pgfpathclose%
\pgfusepath{stroke,fill}%
\end{pgfscope}%
\begin{pgfscope}%
\pgfpathrectangle{\pgfqpoint{0.562500in}{0.275000in}}{\pgfqpoint{3.487500in}{1.925000in}}%
\pgfusepath{clip}%
\pgfsetbuttcap%
\pgfsetroundjoin%
\definecolor{currentfill}{rgb}{0.000000,0.000000,0.000000}%
\pgfsetfillcolor{currentfill}%
\pgfsetlinewidth{1.003750pt}%
\definecolor{currentstroke}{rgb}{0.000000,0.000000,0.000000}%
\pgfsetstrokecolor{currentstroke}%
\pgfsetdash{}{0pt}%
\pgfpathmoveto{\pgfqpoint{1.772992in}{0.524038in}}%
\pgfpathcurveto{\pgfqpoint{1.778517in}{0.524038in}}{\pgfqpoint{1.783816in}{0.526233in}}{\pgfqpoint{1.787723in}{0.530140in}}%
\pgfpathcurveto{\pgfqpoint{1.791630in}{0.534047in}}{\pgfqpoint{1.793825in}{0.539346in}}{\pgfqpoint{1.793825in}{0.544872in}}%
\pgfpathcurveto{\pgfqpoint{1.793825in}{0.550397in}}{\pgfqpoint{1.791630in}{0.555696in}}{\pgfqpoint{1.787723in}{0.559603in}}%
\pgfpathcurveto{\pgfqpoint{1.783816in}{0.563510in}}{\pgfqpoint{1.778517in}{0.565705in}}{\pgfqpoint{1.772992in}{0.565705in}}%
\pgfpathcurveto{\pgfqpoint{1.767467in}{0.565705in}}{\pgfqpoint{1.762167in}{0.563510in}}{\pgfqpoint{1.758260in}{0.559603in}}%
\pgfpathcurveto{\pgfqpoint{1.754353in}{0.555696in}}{\pgfqpoint{1.752158in}{0.550397in}}{\pgfqpoint{1.752158in}{0.544872in}}%
\pgfpathcurveto{\pgfqpoint{1.752158in}{0.539346in}}{\pgfqpoint{1.754353in}{0.534047in}}{\pgfqpoint{1.758260in}{0.530140in}}%
\pgfpathcurveto{\pgfqpoint{1.762167in}{0.526233in}}{\pgfqpoint{1.767467in}{0.524038in}}{\pgfqpoint{1.772992in}{0.524038in}}%
\pgfpathclose%
\pgfusepath{stroke,fill}%
\end{pgfscope}%
\begin{pgfscope}%
\pgfpathrectangle{\pgfqpoint{0.562500in}{0.275000in}}{\pgfqpoint{3.487500in}{1.925000in}}%
\pgfusepath{clip}%
\pgfsetbuttcap%
\pgfsetroundjoin%
\definecolor{currentfill}{rgb}{0.000000,0.000000,0.000000}%
\pgfsetfillcolor{currentfill}%
\pgfsetlinewidth{1.003750pt}%
\definecolor{currentstroke}{rgb}{0.000000,0.000000,0.000000}%
\pgfsetstrokecolor{currentstroke}%
\pgfsetdash{}{0pt}%
\pgfpathmoveto{\pgfqpoint{1.772992in}{0.518980in}}%
\pgfpathcurveto{\pgfqpoint{1.778517in}{0.518980in}}{\pgfqpoint{1.783816in}{0.521175in}}{\pgfqpoint{1.787723in}{0.525082in}}%
\pgfpathcurveto{\pgfqpoint{1.791630in}{0.528988in}}{\pgfqpoint{1.793825in}{0.534288in}}{\pgfqpoint{1.793825in}{0.539813in}}%
\pgfpathcurveto{\pgfqpoint{1.793825in}{0.545338in}}{\pgfqpoint{1.791630in}{0.550638in}}{\pgfqpoint{1.787723in}{0.554544in}}%
\pgfpathcurveto{\pgfqpoint{1.783816in}{0.558451in}}{\pgfqpoint{1.778517in}{0.560646in}}{\pgfqpoint{1.772992in}{0.560646in}}%
\pgfpathcurveto{\pgfqpoint{1.767467in}{0.560646in}}{\pgfqpoint{1.762167in}{0.558451in}}{\pgfqpoint{1.758260in}{0.554544in}}%
\pgfpathcurveto{\pgfqpoint{1.754353in}{0.550638in}}{\pgfqpoint{1.752158in}{0.545338in}}{\pgfqpoint{1.752158in}{0.539813in}}%
\pgfpathcurveto{\pgfqpoint{1.752158in}{0.534288in}}{\pgfqpoint{1.754353in}{0.528988in}}{\pgfqpoint{1.758260in}{0.525082in}}%
\pgfpathcurveto{\pgfqpoint{1.762167in}{0.521175in}}{\pgfqpoint{1.767467in}{0.518980in}}{\pgfqpoint{1.772992in}{0.518980in}}%
\pgfpathclose%
\pgfusepath{stroke,fill}%
\end{pgfscope}%
\begin{pgfscope}%
\pgfpathrectangle{\pgfqpoint{0.562500in}{0.275000in}}{\pgfqpoint{3.487500in}{1.925000in}}%
\pgfusepath{clip}%
\pgfsetbuttcap%
\pgfsetroundjoin%
\definecolor{currentfill}{rgb}{0.000000,0.000000,0.000000}%
\pgfsetfillcolor{currentfill}%
\pgfsetlinewidth{1.003750pt}%
\definecolor{currentstroke}{rgb}{0.000000,0.000000,0.000000}%
\pgfsetstrokecolor{currentstroke}%
\pgfsetdash{}{0pt}%
\pgfpathmoveto{\pgfqpoint{1.772992in}{0.569565in}}%
\pgfpathcurveto{\pgfqpoint{1.778517in}{0.569565in}}{\pgfqpoint{1.783816in}{0.571760in}}{\pgfqpoint{1.787723in}{0.575667in}}%
\pgfpathcurveto{\pgfqpoint{1.791630in}{0.579574in}}{\pgfqpoint{1.793825in}{0.584874in}}{\pgfqpoint{1.793825in}{0.590399in}}%
\pgfpathcurveto{\pgfqpoint{1.793825in}{0.595924in}}{\pgfqpoint{1.791630in}{0.601223in}}{\pgfqpoint{1.787723in}{0.605130in}}%
\pgfpathcurveto{\pgfqpoint{1.783816in}{0.609037in}}{\pgfqpoint{1.778517in}{0.611232in}}{\pgfqpoint{1.772992in}{0.611232in}}%
\pgfpathcurveto{\pgfqpoint{1.767467in}{0.611232in}}{\pgfqpoint{1.762167in}{0.609037in}}{\pgfqpoint{1.758260in}{0.605130in}}%
\pgfpathcurveto{\pgfqpoint{1.754353in}{0.601223in}}{\pgfqpoint{1.752158in}{0.595924in}}{\pgfqpoint{1.752158in}{0.590399in}}%
\pgfpathcurveto{\pgfqpoint{1.752158in}{0.584874in}}{\pgfqpoint{1.754353in}{0.579574in}}{\pgfqpoint{1.758260in}{0.575667in}}%
\pgfpathcurveto{\pgfqpoint{1.762167in}{0.571760in}}{\pgfqpoint{1.767467in}{0.569565in}}{\pgfqpoint{1.772992in}{0.569565in}}%
\pgfpathclose%
\pgfusepath{stroke,fill}%
\end{pgfscope}%
\begin{pgfscope}%
\pgfpathrectangle{\pgfqpoint{0.562500in}{0.275000in}}{\pgfqpoint{3.487500in}{1.925000in}}%
\pgfusepath{clip}%
\pgfsetbuttcap%
\pgfsetroundjoin%
\definecolor{currentfill}{rgb}{0.000000,0.000000,0.000000}%
\pgfsetfillcolor{currentfill}%
\pgfsetlinewidth{1.003750pt}%
\definecolor{currentstroke}{rgb}{0.000000,0.000000,0.000000}%
\pgfsetstrokecolor{currentstroke}%
\pgfsetdash{}{0pt}%
\pgfpathmoveto{\pgfqpoint{1.772992in}{0.503804in}}%
\pgfpathcurveto{\pgfqpoint{1.778517in}{0.503804in}}{\pgfqpoint{1.783816in}{0.505999in}}{\pgfqpoint{1.787723in}{0.509906in}}%
\pgfpathcurveto{\pgfqpoint{1.791630in}{0.513813in}}{\pgfqpoint{1.793825in}{0.519112in}}{\pgfqpoint{1.793825in}{0.524637in}}%
\pgfpathcurveto{\pgfqpoint{1.793825in}{0.530162in}}{\pgfqpoint{1.791630in}{0.535462in}}{\pgfqpoint{1.787723in}{0.539369in}}%
\pgfpathcurveto{\pgfqpoint{1.783816in}{0.543275in}}{\pgfqpoint{1.778517in}{0.545471in}}{\pgfqpoint{1.772992in}{0.545471in}}%
\pgfpathcurveto{\pgfqpoint{1.767467in}{0.545471in}}{\pgfqpoint{1.762167in}{0.543275in}}{\pgfqpoint{1.758260in}{0.539369in}}%
\pgfpathcurveto{\pgfqpoint{1.754353in}{0.535462in}}{\pgfqpoint{1.752158in}{0.530162in}}{\pgfqpoint{1.752158in}{0.524637in}}%
\pgfpathcurveto{\pgfqpoint{1.752158in}{0.519112in}}{\pgfqpoint{1.754353in}{0.513813in}}{\pgfqpoint{1.758260in}{0.509906in}}%
\pgfpathcurveto{\pgfqpoint{1.762167in}{0.505999in}}{\pgfqpoint{1.767467in}{0.503804in}}{\pgfqpoint{1.772992in}{0.503804in}}%
\pgfpathclose%
\pgfusepath{stroke,fill}%
\end{pgfscope}%
\begin{pgfscope}%
\pgfpathrectangle{\pgfqpoint{0.562500in}{0.275000in}}{\pgfqpoint{3.487500in}{1.925000in}}%
\pgfusepath{clip}%
\pgfsetbuttcap%
\pgfsetroundjoin%
\definecolor{currentfill}{rgb}{0.000000,0.000000,0.000000}%
\pgfsetfillcolor{currentfill}%
\pgfsetlinewidth{1.003750pt}%
\definecolor{currentstroke}{rgb}{0.000000,0.000000,0.000000}%
\pgfsetstrokecolor{currentstroke}%
\pgfsetdash{}{0pt}%
\pgfpathmoveto{\pgfqpoint{1.772992in}{0.549331in}}%
\pgfpathcurveto{\pgfqpoint{1.778517in}{0.549331in}}{\pgfqpoint{1.783816in}{0.551526in}}{\pgfqpoint{1.787723in}{0.555433in}}%
\pgfpathcurveto{\pgfqpoint{1.791630in}{0.559340in}}{\pgfqpoint{1.793825in}{0.564639in}}{\pgfqpoint{1.793825in}{0.570164in}}%
\pgfpathcurveto{\pgfqpoint{1.793825in}{0.575689in}}{\pgfqpoint{1.791630in}{0.580989in}}{\pgfqpoint{1.787723in}{0.584896in}}%
\pgfpathcurveto{\pgfqpoint{1.783816in}{0.588803in}}{\pgfqpoint{1.778517in}{0.590998in}}{\pgfqpoint{1.772992in}{0.590998in}}%
\pgfpathcurveto{\pgfqpoint{1.767467in}{0.590998in}}{\pgfqpoint{1.762167in}{0.588803in}}{\pgfqpoint{1.758260in}{0.584896in}}%
\pgfpathcurveto{\pgfqpoint{1.754353in}{0.580989in}}{\pgfqpoint{1.752158in}{0.575689in}}{\pgfqpoint{1.752158in}{0.570164in}}%
\pgfpathcurveto{\pgfqpoint{1.752158in}{0.564639in}}{\pgfqpoint{1.754353in}{0.559340in}}{\pgfqpoint{1.758260in}{0.555433in}}%
\pgfpathcurveto{\pgfqpoint{1.762167in}{0.551526in}}{\pgfqpoint{1.767467in}{0.549331in}}{\pgfqpoint{1.772992in}{0.549331in}}%
\pgfpathclose%
\pgfusepath{stroke,fill}%
\end{pgfscope}%
\begin{pgfscope}%
\pgfpathrectangle{\pgfqpoint{0.562500in}{0.275000in}}{\pgfqpoint{3.487500in}{1.925000in}}%
\pgfusepath{clip}%
\pgfsetbuttcap%
\pgfsetroundjoin%
\definecolor{currentfill}{rgb}{0.000000,0.000000,0.000000}%
\pgfsetfillcolor{currentfill}%
\pgfsetlinewidth{1.003750pt}%
\definecolor{currentstroke}{rgb}{0.000000,0.000000,0.000000}%
\pgfsetstrokecolor{currentstroke}%
\pgfsetdash{}{0pt}%
\pgfpathmoveto{\pgfqpoint{1.772992in}{0.529097in}}%
\pgfpathcurveto{\pgfqpoint{1.778517in}{0.529097in}}{\pgfqpoint{1.783816in}{0.531292in}}{\pgfqpoint{1.787723in}{0.535199in}}%
\pgfpathcurveto{\pgfqpoint{1.791630in}{0.539106in}}{\pgfqpoint{1.793825in}{0.544405in}}{\pgfqpoint{1.793825in}{0.549930in}}%
\pgfpathcurveto{\pgfqpoint{1.793825in}{0.555455in}}{\pgfqpoint{1.791630in}{0.560755in}}{\pgfqpoint{1.787723in}{0.564662in}}%
\pgfpathcurveto{\pgfqpoint{1.783816in}{0.568568in}}{\pgfqpoint{1.778517in}{0.570763in}}{\pgfqpoint{1.772992in}{0.570763in}}%
\pgfpathcurveto{\pgfqpoint{1.767467in}{0.570763in}}{\pgfqpoint{1.762167in}{0.568568in}}{\pgfqpoint{1.758260in}{0.564662in}}%
\pgfpathcurveto{\pgfqpoint{1.754353in}{0.560755in}}{\pgfqpoint{1.752158in}{0.555455in}}{\pgfqpoint{1.752158in}{0.549930in}}%
\pgfpathcurveto{\pgfqpoint{1.752158in}{0.544405in}}{\pgfqpoint{1.754353in}{0.539106in}}{\pgfqpoint{1.758260in}{0.535199in}}%
\pgfpathcurveto{\pgfqpoint{1.762167in}{0.531292in}}{\pgfqpoint{1.767467in}{0.529097in}}{\pgfqpoint{1.772992in}{0.529097in}}%
\pgfpathclose%
\pgfusepath{stroke,fill}%
\end{pgfscope}%
\begin{pgfscope}%
\pgfpathrectangle{\pgfqpoint{0.562500in}{0.275000in}}{\pgfqpoint{3.487500in}{1.925000in}}%
\pgfusepath{clip}%
\pgfsetbuttcap%
\pgfsetroundjoin%
\definecolor{currentfill}{rgb}{0.000000,0.000000,0.000000}%
\pgfsetfillcolor{currentfill}%
\pgfsetlinewidth{1.003750pt}%
\definecolor{currentstroke}{rgb}{0.000000,0.000000,0.000000}%
\pgfsetstrokecolor{currentstroke}%
\pgfsetdash{}{0pt}%
\pgfpathmoveto{\pgfqpoint{1.772992in}{0.544272in}}%
\pgfpathcurveto{\pgfqpoint{1.778517in}{0.544272in}}{\pgfqpoint{1.783816in}{0.546468in}}{\pgfqpoint{1.787723in}{0.550374in}}%
\pgfpathcurveto{\pgfqpoint{1.791630in}{0.554281in}}{\pgfqpoint{1.793825in}{0.559581in}}{\pgfqpoint{1.793825in}{0.565106in}}%
\pgfpathcurveto{\pgfqpoint{1.793825in}{0.570631in}}{\pgfqpoint{1.791630in}{0.575930in}}{\pgfqpoint{1.787723in}{0.579837in}}%
\pgfpathcurveto{\pgfqpoint{1.783816in}{0.583744in}}{\pgfqpoint{1.778517in}{0.585939in}}{\pgfqpoint{1.772992in}{0.585939in}}%
\pgfpathcurveto{\pgfqpoint{1.767467in}{0.585939in}}{\pgfqpoint{1.762167in}{0.583744in}}{\pgfqpoint{1.758260in}{0.579837in}}%
\pgfpathcurveto{\pgfqpoint{1.754353in}{0.575930in}}{\pgfqpoint{1.752158in}{0.570631in}}{\pgfqpoint{1.752158in}{0.565106in}}%
\pgfpathcurveto{\pgfqpoint{1.752158in}{0.559581in}}{\pgfqpoint{1.754353in}{0.554281in}}{\pgfqpoint{1.758260in}{0.550374in}}%
\pgfpathcurveto{\pgfqpoint{1.762167in}{0.546468in}}{\pgfqpoint{1.767467in}{0.544272in}}{\pgfqpoint{1.772992in}{0.544272in}}%
\pgfpathclose%
\pgfusepath{stroke,fill}%
\end{pgfscope}%
\begin{pgfscope}%
\pgfpathrectangle{\pgfqpoint{0.562500in}{0.275000in}}{\pgfqpoint{3.487500in}{1.925000in}}%
\pgfusepath{clip}%
\pgfsetbuttcap%
\pgfsetroundjoin%
\definecolor{currentfill}{rgb}{0.000000,0.000000,0.000000}%
\pgfsetfillcolor{currentfill}%
\pgfsetlinewidth{1.003750pt}%
\definecolor{currentstroke}{rgb}{0.000000,0.000000,0.000000}%
\pgfsetstrokecolor{currentstroke}%
\pgfsetdash{}{0pt}%
\pgfpathmoveto{\pgfqpoint{1.772992in}{0.544272in}}%
\pgfpathcurveto{\pgfqpoint{1.778517in}{0.544272in}}{\pgfqpoint{1.783816in}{0.546468in}}{\pgfqpoint{1.787723in}{0.550374in}}%
\pgfpathcurveto{\pgfqpoint{1.791630in}{0.554281in}}{\pgfqpoint{1.793825in}{0.559581in}}{\pgfqpoint{1.793825in}{0.565106in}}%
\pgfpathcurveto{\pgfqpoint{1.793825in}{0.570631in}}{\pgfqpoint{1.791630in}{0.575930in}}{\pgfqpoint{1.787723in}{0.579837in}}%
\pgfpathcurveto{\pgfqpoint{1.783816in}{0.583744in}}{\pgfqpoint{1.778517in}{0.585939in}}{\pgfqpoint{1.772992in}{0.585939in}}%
\pgfpathcurveto{\pgfqpoint{1.767467in}{0.585939in}}{\pgfqpoint{1.762167in}{0.583744in}}{\pgfqpoint{1.758260in}{0.579837in}}%
\pgfpathcurveto{\pgfqpoint{1.754353in}{0.575930in}}{\pgfqpoint{1.752158in}{0.570631in}}{\pgfqpoint{1.752158in}{0.565106in}}%
\pgfpathcurveto{\pgfqpoint{1.752158in}{0.559581in}}{\pgfqpoint{1.754353in}{0.554281in}}{\pgfqpoint{1.758260in}{0.550374in}}%
\pgfpathcurveto{\pgfqpoint{1.762167in}{0.546468in}}{\pgfqpoint{1.767467in}{0.544272in}}{\pgfqpoint{1.772992in}{0.544272in}}%
\pgfpathclose%
\pgfusepath{stroke,fill}%
\end{pgfscope}%
\begin{pgfscope}%
\pgfpathrectangle{\pgfqpoint{0.562500in}{0.275000in}}{\pgfqpoint{3.487500in}{1.925000in}}%
\pgfusepath{clip}%
\pgfsetbuttcap%
\pgfsetroundjoin%
\definecolor{currentfill}{rgb}{0.000000,0.000000,0.000000}%
\pgfsetfillcolor{currentfill}%
\pgfsetlinewidth{1.003750pt}%
\definecolor{currentstroke}{rgb}{0.000000,0.000000,0.000000}%
\pgfsetstrokecolor{currentstroke}%
\pgfsetdash{}{0pt}%
\pgfpathmoveto{\pgfqpoint{1.772992in}{0.594858in}}%
\pgfpathcurveto{\pgfqpoint{1.778517in}{0.594858in}}{\pgfqpoint{1.783816in}{0.597053in}}{\pgfqpoint{1.787723in}{0.600960in}}%
\pgfpathcurveto{\pgfqpoint{1.791630in}{0.604867in}}{\pgfqpoint{1.793825in}{0.610166in}}{\pgfqpoint{1.793825in}{0.615691in}}%
\pgfpathcurveto{\pgfqpoint{1.793825in}{0.621217in}}{\pgfqpoint{1.791630in}{0.626516in}}{\pgfqpoint{1.787723in}{0.630423in}}%
\pgfpathcurveto{\pgfqpoint{1.783816in}{0.634330in}}{\pgfqpoint{1.778517in}{0.636525in}}{\pgfqpoint{1.772992in}{0.636525in}}%
\pgfpathcurveto{\pgfqpoint{1.767467in}{0.636525in}}{\pgfqpoint{1.762167in}{0.634330in}}{\pgfqpoint{1.758260in}{0.630423in}}%
\pgfpathcurveto{\pgfqpoint{1.754353in}{0.626516in}}{\pgfqpoint{1.752158in}{0.621217in}}{\pgfqpoint{1.752158in}{0.615691in}}%
\pgfpathcurveto{\pgfqpoint{1.752158in}{0.610166in}}{\pgfqpoint{1.754353in}{0.604867in}}{\pgfqpoint{1.758260in}{0.600960in}}%
\pgfpathcurveto{\pgfqpoint{1.762167in}{0.597053in}}{\pgfqpoint{1.767467in}{0.594858in}}{\pgfqpoint{1.772992in}{0.594858in}}%
\pgfpathclose%
\pgfusepath{stroke,fill}%
\end{pgfscope}%
\begin{pgfscope}%
\pgfpathrectangle{\pgfqpoint{0.562500in}{0.275000in}}{\pgfqpoint{3.487500in}{1.925000in}}%
\pgfusepath{clip}%
\pgfsetbuttcap%
\pgfsetroundjoin%
\definecolor{currentfill}{rgb}{0.000000,0.000000,0.000000}%
\pgfsetfillcolor{currentfill}%
\pgfsetlinewidth{1.003750pt}%
\definecolor{currentstroke}{rgb}{0.000000,0.000000,0.000000}%
\pgfsetstrokecolor{currentstroke}%
\pgfsetdash{}{0pt}%
\pgfpathmoveto{\pgfqpoint{1.772992in}{0.529097in}}%
\pgfpathcurveto{\pgfqpoint{1.778517in}{0.529097in}}{\pgfqpoint{1.783816in}{0.531292in}}{\pgfqpoint{1.787723in}{0.535199in}}%
\pgfpathcurveto{\pgfqpoint{1.791630in}{0.539106in}}{\pgfqpoint{1.793825in}{0.544405in}}{\pgfqpoint{1.793825in}{0.549930in}}%
\pgfpathcurveto{\pgfqpoint{1.793825in}{0.555455in}}{\pgfqpoint{1.791630in}{0.560755in}}{\pgfqpoint{1.787723in}{0.564662in}}%
\pgfpathcurveto{\pgfqpoint{1.783816in}{0.568568in}}{\pgfqpoint{1.778517in}{0.570763in}}{\pgfqpoint{1.772992in}{0.570763in}}%
\pgfpathcurveto{\pgfqpoint{1.767467in}{0.570763in}}{\pgfqpoint{1.762167in}{0.568568in}}{\pgfqpoint{1.758260in}{0.564662in}}%
\pgfpathcurveto{\pgfqpoint{1.754353in}{0.560755in}}{\pgfqpoint{1.752158in}{0.555455in}}{\pgfqpoint{1.752158in}{0.549930in}}%
\pgfpathcurveto{\pgfqpoint{1.752158in}{0.544405in}}{\pgfqpoint{1.754353in}{0.539106in}}{\pgfqpoint{1.758260in}{0.535199in}}%
\pgfpathcurveto{\pgfqpoint{1.762167in}{0.531292in}}{\pgfqpoint{1.767467in}{0.529097in}}{\pgfqpoint{1.772992in}{0.529097in}}%
\pgfpathclose%
\pgfusepath{stroke,fill}%
\end{pgfscope}%
\begin{pgfscope}%
\pgfpathrectangle{\pgfqpoint{0.562500in}{0.275000in}}{\pgfqpoint{3.487500in}{1.925000in}}%
\pgfusepath{clip}%
\pgfsetbuttcap%
\pgfsetroundjoin%
\definecolor{currentfill}{rgb}{0.000000,0.000000,0.000000}%
\pgfsetfillcolor{currentfill}%
\pgfsetlinewidth{1.003750pt}%
\definecolor{currentstroke}{rgb}{0.000000,0.000000,0.000000}%
\pgfsetstrokecolor{currentstroke}%
\pgfsetdash{}{0pt}%
\pgfpathmoveto{\pgfqpoint{1.772992in}{0.503804in}}%
\pgfpathcurveto{\pgfqpoint{1.778517in}{0.503804in}}{\pgfqpoint{1.783816in}{0.505999in}}{\pgfqpoint{1.787723in}{0.509906in}}%
\pgfpathcurveto{\pgfqpoint{1.791630in}{0.513813in}}{\pgfqpoint{1.793825in}{0.519112in}}{\pgfqpoint{1.793825in}{0.524637in}}%
\pgfpathcurveto{\pgfqpoint{1.793825in}{0.530162in}}{\pgfqpoint{1.791630in}{0.535462in}}{\pgfqpoint{1.787723in}{0.539369in}}%
\pgfpathcurveto{\pgfqpoint{1.783816in}{0.543275in}}{\pgfqpoint{1.778517in}{0.545471in}}{\pgfqpoint{1.772992in}{0.545471in}}%
\pgfpathcurveto{\pgfqpoint{1.767467in}{0.545471in}}{\pgfqpoint{1.762167in}{0.543275in}}{\pgfqpoint{1.758260in}{0.539369in}}%
\pgfpathcurveto{\pgfqpoint{1.754353in}{0.535462in}}{\pgfqpoint{1.752158in}{0.530162in}}{\pgfqpoint{1.752158in}{0.524637in}}%
\pgfpathcurveto{\pgfqpoint{1.752158in}{0.519112in}}{\pgfqpoint{1.754353in}{0.513813in}}{\pgfqpoint{1.758260in}{0.509906in}}%
\pgfpathcurveto{\pgfqpoint{1.762167in}{0.505999in}}{\pgfqpoint{1.767467in}{0.503804in}}{\pgfqpoint{1.772992in}{0.503804in}}%
\pgfpathclose%
\pgfusepath{stroke,fill}%
\end{pgfscope}%
\begin{pgfscope}%
\pgfpathrectangle{\pgfqpoint{0.562500in}{0.275000in}}{\pgfqpoint{3.487500in}{1.925000in}}%
\pgfusepath{clip}%
\pgfsetbuttcap%
\pgfsetroundjoin%
\definecolor{currentfill}{rgb}{0.000000,0.000000,0.000000}%
\pgfsetfillcolor{currentfill}%
\pgfsetlinewidth{1.003750pt}%
\definecolor{currentstroke}{rgb}{0.000000,0.000000,0.000000}%
\pgfsetstrokecolor{currentstroke}%
\pgfsetdash{}{0pt}%
\pgfpathmoveto{\pgfqpoint{1.772992in}{0.508863in}}%
\pgfpathcurveto{\pgfqpoint{1.778517in}{0.508863in}}{\pgfqpoint{1.783816in}{0.511058in}}{\pgfqpoint{1.787723in}{0.514964in}}%
\pgfpathcurveto{\pgfqpoint{1.791630in}{0.518871in}}{\pgfqpoint{1.793825in}{0.524171in}}{\pgfqpoint{1.793825in}{0.529696in}}%
\pgfpathcurveto{\pgfqpoint{1.793825in}{0.535221in}}{\pgfqpoint{1.791630in}{0.540520in}}{\pgfqpoint{1.787723in}{0.544427in}}%
\pgfpathcurveto{\pgfqpoint{1.783816in}{0.548334in}}{\pgfqpoint{1.778517in}{0.550529in}}{\pgfqpoint{1.772992in}{0.550529in}}%
\pgfpathcurveto{\pgfqpoint{1.767467in}{0.550529in}}{\pgfqpoint{1.762167in}{0.548334in}}{\pgfqpoint{1.758260in}{0.544427in}}%
\pgfpathcurveto{\pgfqpoint{1.754353in}{0.540520in}}{\pgfqpoint{1.752158in}{0.535221in}}{\pgfqpoint{1.752158in}{0.529696in}}%
\pgfpathcurveto{\pgfqpoint{1.752158in}{0.524171in}}{\pgfqpoint{1.754353in}{0.518871in}}{\pgfqpoint{1.758260in}{0.514964in}}%
\pgfpathcurveto{\pgfqpoint{1.762167in}{0.511058in}}{\pgfqpoint{1.767467in}{0.508863in}}{\pgfqpoint{1.772992in}{0.508863in}}%
\pgfpathclose%
\pgfusepath{stroke,fill}%
\end{pgfscope}%
\begin{pgfscope}%
\pgfpathrectangle{\pgfqpoint{0.562500in}{0.275000in}}{\pgfqpoint{3.487500in}{1.925000in}}%
\pgfusepath{clip}%
\pgfsetbuttcap%
\pgfsetroundjoin%
\definecolor{currentfill}{rgb}{0.000000,0.000000,0.000000}%
\pgfsetfillcolor{currentfill}%
\pgfsetlinewidth{1.003750pt}%
\definecolor{currentstroke}{rgb}{0.000000,0.000000,0.000000}%
\pgfsetstrokecolor{currentstroke}%
\pgfsetdash{}{0pt}%
\pgfpathmoveto{\pgfqpoint{1.772992in}{0.554390in}}%
\pgfpathcurveto{\pgfqpoint{1.778517in}{0.554390in}}{\pgfqpoint{1.783816in}{0.556585in}}{\pgfqpoint{1.787723in}{0.560492in}}%
\pgfpathcurveto{\pgfqpoint{1.791630in}{0.564398in}}{\pgfqpoint{1.793825in}{0.569698in}}{\pgfqpoint{1.793825in}{0.575223in}}%
\pgfpathcurveto{\pgfqpoint{1.793825in}{0.580748in}}{\pgfqpoint{1.791630in}{0.586048in}}{\pgfqpoint{1.787723in}{0.589954in}}%
\pgfpathcurveto{\pgfqpoint{1.783816in}{0.593861in}}{\pgfqpoint{1.778517in}{0.596056in}}{\pgfqpoint{1.772992in}{0.596056in}}%
\pgfpathcurveto{\pgfqpoint{1.767467in}{0.596056in}}{\pgfqpoint{1.762167in}{0.593861in}}{\pgfqpoint{1.758260in}{0.589954in}}%
\pgfpathcurveto{\pgfqpoint{1.754353in}{0.586048in}}{\pgfqpoint{1.752158in}{0.580748in}}{\pgfqpoint{1.752158in}{0.575223in}}%
\pgfpathcurveto{\pgfqpoint{1.752158in}{0.569698in}}{\pgfqpoint{1.754353in}{0.564398in}}{\pgfqpoint{1.758260in}{0.560492in}}%
\pgfpathcurveto{\pgfqpoint{1.762167in}{0.556585in}}{\pgfqpoint{1.767467in}{0.554390in}}{\pgfqpoint{1.772992in}{0.554390in}}%
\pgfpathclose%
\pgfusepath{stroke,fill}%
\end{pgfscope}%
\begin{pgfscope}%
\pgfpathrectangle{\pgfqpoint{0.562500in}{0.275000in}}{\pgfqpoint{3.487500in}{1.925000in}}%
\pgfusepath{clip}%
\pgfsetbuttcap%
\pgfsetroundjoin%
\definecolor{currentfill}{rgb}{0.000000,0.000000,0.000000}%
\pgfsetfillcolor{currentfill}%
\pgfsetlinewidth{1.003750pt}%
\definecolor{currentstroke}{rgb}{0.000000,0.000000,0.000000}%
\pgfsetstrokecolor{currentstroke}%
\pgfsetdash{}{0pt}%
\pgfpathmoveto{\pgfqpoint{1.772992in}{0.518980in}}%
\pgfpathcurveto{\pgfqpoint{1.778517in}{0.518980in}}{\pgfqpoint{1.783816in}{0.521175in}}{\pgfqpoint{1.787723in}{0.525082in}}%
\pgfpathcurveto{\pgfqpoint{1.791630in}{0.528988in}}{\pgfqpoint{1.793825in}{0.534288in}}{\pgfqpoint{1.793825in}{0.539813in}}%
\pgfpathcurveto{\pgfqpoint{1.793825in}{0.545338in}}{\pgfqpoint{1.791630in}{0.550638in}}{\pgfqpoint{1.787723in}{0.554544in}}%
\pgfpathcurveto{\pgfqpoint{1.783816in}{0.558451in}}{\pgfqpoint{1.778517in}{0.560646in}}{\pgfqpoint{1.772992in}{0.560646in}}%
\pgfpathcurveto{\pgfqpoint{1.767467in}{0.560646in}}{\pgfqpoint{1.762167in}{0.558451in}}{\pgfqpoint{1.758260in}{0.554544in}}%
\pgfpathcurveto{\pgfqpoint{1.754353in}{0.550638in}}{\pgfqpoint{1.752158in}{0.545338in}}{\pgfqpoint{1.752158in}{0.539813in}}%
\pgfpathcurveto{\pgfqpoint{1.752158in}{0.534288in}}{\pgfqpoint{1.754353in}{0.528988in}}{\pgfqpoint{1.758260in}{0.525082in}}%
\pgfpathcurveto{\pgfqpoint{1.762167in}{0.521175in}}{\pgfqpoint{1.767467in}{0.518980in}}{\pgfqpoint{1.772992in}{0.518980in}}%
\pgfpathclose%
\pgfusepath{stroke,fill}%
\end{pgfscope}%
\begin{pgfscope}%
\pgfpathrectangle{\pgfqpoint{0.562500in}{0.275000in}}{\pgfqpoint{3.487500in}{1.925000in}}%
\pgfusepath{clip}%
\pgfsetbuttcap%
\pgfsetroundjoin%
\definecolor{currentfill}{rgb}{0.000000,0.000000,0.000000}%
\pgfsetfillcolor{currentfill}%
\pgfsetlinewidth{1.003750pt}%
\definecolor{currentstroke}{rgb}{0.000000,0.000000,0.000000}%
\pgfsetstrokecolor{currentstroke}%
\pgfsetdash{}{0pt}%
\pgfpathmoveto{\pgfqpoint{1.772992in}{0.549331in}}%
\pgfpathcurveto{\pgfqpoint{1.778517in}{0.549331in}}{\pgfqpoint{1.783816in}{0.551526in}}{\pgfqpoint{1.787723in}{0.555433in}}%
\pgfpathcurveto{\pgfqpoint{1.791630in}{0.559340in}}{\pgfqpoint{1.793825in}{0.564639in}}{\pgfqpoint{1.793825in}{0.570164in}}%
\pgfpathcurveto{\pgfqpoint{1.793825in}{0.575689in}}{\pgfqpoint{1.791630in}{0.580989in}}{\pgfqpoint{1.787723in}{0.584896in}}%
\pgfpathcurveto{\pgfqpoint{1.783816in}{0.588803in}}{\pgfqpoint{1.778517in}{0.590998in}}{\pgfqpoint{1.772992in}{0.590998in}}%
\pgfpathcurveto{\pgfqpoint{1.767467in}{0.590998in}}{\pgfqpoint{1.762167in}{0.588803in}}{\pgfqpoint{1.758260in}{0.584896in}}%
\pgfpathcurveto{\pgfqpoint{1.754353in}{0.580989in}}{\pgfqpoint{1.752158in}{0.575689in}}{\pgfqpoint{1.752158in}{0.570164in}}%
\pgfpathcurveto{\pgfqpoint{1.752158in}{0.564639in}}{\pgfqpoint{1.754353in}{0.559340in}}{\pgfqpoint{1.758260in}{0.555433in}}%
\pgfpathcurveto{\pgfqpoint{1.762167in}{0.551526in}}{\pgfqpoint{1.767467in}{0.549331in}}{\pgfqpoint{1.772992in}{0.549331in}}%
\pgfpathclose%
\pgfusepath{stroke,fill}%
\end{pgfscope}%
\begin{pgfscope}%
\pgfpathrectangle{\pgfqpoint{0.562500in}{0.275000in}}{\pgfqpoint{3.487500in}{1.925000in}}%
\pgfusepath{clip}%
\pgfsetbuttcap%
\pgfsetroundjoin%
\definecolor{currentfill}{rgb}{0.000000,0.000000,0.000000}%
\pgfsetfillcolor{currentfill}%
\pgfsetlinewidth{1.003750pt}%
\definecolor{currentstroke}{rgb}{0.000000,0.000000,0.000000}%
\pgfsetstrokecolor{currentstroke}%
\pgfsetdash{}{0pt}%
\pgfpathmoveto{\pgfqpoint{1.772992in}{0.498745in}}%
\pgfpathcurveto{\pgfqpoint{1.778517in}{0.498745in}}{\pgfqpoint{1.783816in}{0.500941in}}{\pgfqpoint{1.787723in}{0.504847in}}%
\pgfpathcurveto{\pgfqpoint{1.791630in}{0.508754in}}{\pgfqpoint{1.793825in}{0.514054in}}{\pgfqpoint{1.793825in}{0.519579in}}%
\pgfpathcurveto{\pgfqpoint{1.793825in}{0.525104in}}{\pgfqpoint{1.791630in}{0.530403in}}{\pgfqpoint{1.787723in}{0.534310in}}%
\pgfpathcurveto{\pgfqpoint{1.783816in}{0.538217in}}{\pgfqpoint{1.778517in}{0.540412in}}{\pgfqpoint{1.772992in}{0.540412in}}%
\pgfpathcurveto{\pgfqpoint{1.767467in}{0.540412in}}{\pgfqpoint{1.762167in}{0.538217in}}{\pgfqpoint{1.758260in}{0.534310in}}%
\pgfpathcurveto{\pgfqpoint{1.754353in}{0.530403in}}{\pgfqpoint{1.752158in}{0.525104in}}{\pgfqpoint{1.752158in}{0.519579in}}%
\pgfpathcurveto{\pgfqpoint{1.752158in}{0.514054in}}{\pgfqpoint{1.754353in}{0.508754in}}{\pgfqpoint{1.758260in}{0.504847in}}%
\pgfpathcurveto{\pgfqpoint{1.762167in}{0.500941in}}{\pgfqpoint{1.767467in}{0.498745in}}{\pgfqpoint{1.772992in}{0.498745in}}%
\pgfpathclose%
\pgfusepath{stroke,fill}%
\end{pgfscope}%
\begin{pgfscope}%
\pgfpathrectangle{\pgfqpoint{0.562500in}{0.275000in}}{\pgfqpoint{3.487500in}{1.925000in}}%
\pgfusepath{clip}%
\pgfsetbuttcap%
\pgfsetroundjoin%
\definecolor{currentfill}{rgb}{0.000000,0.000000,0.000000}%
\pgfsetfillcolor{currentfill}%
\pgfsetlinewidth{1.003750pt}%
\definecolor{currentstroke}{rgb}{0.000000,0.000000,0.000000}%
\pgfsetstrokecolor{currentstroke}%
\pgfsetdash{}{0pt}%
\pgfpathmoveto{\pgfqpoint{1.772992in}{0.564507in}}%
\pgfpathcurveto{\pgfqpoint{1.778517in}{0.564507in}}{\pgfqpoint{1.783816in}{0.566702in}}{\pgfqpoint{1.787723in}{0.570609in}}%
\pgfpathcurveto{\pgfqpoint{1.791630in}{0.574515in}}{\pgfqpoint{1.793825in}{0.579815in}}{\pgfqpoint{1.793825in}{0.585340in}}%
\pgfpathcurveto{\pgfqpoint{1.793825in}{0.590865in}}{\pgfqpoint{1.791630in}{0.596165in}}{\pgfqpoint{1.787723in}{0.600071in}}%
\pgfpathcurveto{\pgfqpoint{1.783816in}{0.603978in}}{\pgfqpoint{1.778517in}{0.606173in}}{\pgfqpoint{1.772992in}{0.606173in}}%
\pgfpathcurveto{\pgfqpoint{1.767467in}{0.606173in}}{\pgfqpoint{1.762167in}{0.603978in}}{\pgfqpoint{1.758260in}{0.600071in}}%
\pgfpathcurveto{\pgfqpoint{1.754353in}{0.596165in}}{\pgfqpoint{1.752158in}{0.590865in}}{\pgfqpoint{1.752158in}{0.585340in}}%
\pgfpathcurveto{\pgfqpoint{1.752158in}{0.579815in}}{\pgfqpoint{1.754353in}{0.574515in}}{\pgfqpoint{1.758260in}{0.570609in}}%
\pgfpathcurveto{\pgfqpoint{1.762167in}{0.566702in}}{\pgfqpoint{1.767467in}{0.564507in}}{\pgfqpoint{1.772992in}{0.564507in}}%
\pgfpathclose%
\pgfusepath{stroke,fill}%
\end{pgfscope}%
\begin{pgfscope}%
\pgfpathrectangle{\pgfqpoint{0.562500in}{0.275000in}}{\pgfqpoint{3.487500in}{1.925000in}}%
\pgfusepath{clip}%
\pgfsetbuttcap%
\pgfsetroundjoin%
\definecolor{currentfill}{rgb}{0.000000,0.000000,0.000000}%
\pgfsetfillcolor{currentfill}%
\pgfsetlinewidth{1.003750pt}%
\definecolor{currentstroke}{rgb}{0.000000,0.000000,0.000000}%
\pgfsetstrokecolor{currentstroke}%
\pgfsetdash{}{0pt}%
\pgfpathmoveto{\pgfqpoint{1.772992in}{0.544272in}}%
\pgfpathcurveto{\pgfqpoint{1.778517in}{0.544272in}}{\pgfqpoint{1.783816in}{0.546468in}}{\pgfqpoint{1.787723in}{0.550374in}}%
\pgfpathcurveto{\pgfqpoint{1.791630in}{0.554281in}}{\pgfqpoint{1.793825in}{0.559581in}}{\pgfqpoint{1.793825in}{0.565106in}}%
\pgfpathcurveto{\pgfqpoint{1.793825in}{0.570631in}}{\pgfqpoint{1.791630in}{0.575930in}}{\pgfqpoint{1.787723in}{0.579837in}}%
\pgfpathcurveto{\pgfqpoint{1.783816in}{0.583744in}}{\pgfqpoint{1.778517in}{0.585939in}}{\pgfqpoint{1.772992in}{0.585939in}}%
\pgfpathcurveto{\pgfqpoint{1.767467in}{0.585939in}}{\pgfqpoint{1.762167in}{0.583744in}}{\pgfqpoint{1.758260in}{0.579837in}}%
\pgfpathcurveto{\pgfqpoint{1.754353in}{0.575930in}}{\pgfqpoint{1.752158in}{0.570631in}}{\pgfqpoint{1.752158in}{0.565106in}}%
\pgfpathcurveto{\pgfqpoint{1.752158in}{0.559581in}}{\pgfqpoint{1.754353in}{0.554281in}}{\pgfqpoint{1.758260in}{0.550374in}}%
\pgfpathcurveto{\pgfqpoint{1.762167in}{0.546468in}}{\pgfqpoint{1.767467in}{0.544272in}}{\pgfqpoint{1.772992in}{0.544272in}}%
\pgfpathclose%
\pgfusepath{stroke,fill}%
\end{pgfscope}%
\begin{pgfscope}%
\pgfpathrectangle{\pgfqpoint{0.562500in}{0.275000in}}{\pgfqpoint{3.487500in}{1.925000in}}%
\pgfusepath{clip}%
\pgfsetbuttcap%
\pgfsetroundjoin%
\definecolor{currentfill}{rgb}{0.000000,0.000000,0.000000}%
\pgfsetfillcolor{currentfill}%
\pgfsetlinewidth{1.003750pt}%
\definecolor{currentstroke}{rgb}{0.000000,0.000000,0.000000}%
\pgfsetstrokecolor{currentstroke}%
\pgfsetdash{}{0pt}%
\pgfpathmoveto{\pgfqpoint{1.772992in}{0.544272in}}%
\pgfpathcurveto{\pgfqpoint{1.778517in}{0.544272in}}{\pgfqpoint{1.783816in}{0.546468in}}{\pgfqpoint{1.787723in}{0.550374in}}%
\pgfpathcurveto{\pgfqpoint{1.791630in}{0.554281in}}{\pgfqpoint{1.793825in}{0.559581in}}{\pgfqpoint{1.793825in}{0.565106in}}%
\pgfpathcurveto{\pgfqpoint{1.793825in}{0.570631in}}{\pgfqpoint{1.791630in}{0.575930in}}{\pgfqpoint{1.787723in}{0.579837in}}%
\pgfpathcurveto{\pgfqpoint{1.783816in}{0.583744in}}{\pgfqpoint{1.778517in}{0.585939in}}{\pgfqpoint{1.772992in}{0.585939in}}%
\pgfpathcurveto{\pgfqpoint{1.767467in}{0.585939in}}{\pgfqpoint{1.762167in}{0.583744in}}{\pgfqpoint{1.758260in}{0.579837in}}%
\pgfpathcurveto{\pgfqpoint{1.754353in}{0.575930in}}{\pgfqpoint{1.752158in}{0.570631in}}{\pgfqpoint{1.752158in}{0.565106in}}%
\pgfpathcurveto{\pgfqpoint{1.752158in}{0.559581in}}{\pgfqpoint{1.754353in}{0.554281in}}{\pgfqpoint{1.758260in}{0.550374in}}%
\pgfpathcurveto{\pgfqpoint{1.762167in}{0.546468in}}{\pgfqpoint{1.767467in}{0.544272in}}{\pgfqpoint{1.772992in}{0.544272in}}%
\pgfpathclose%
\pgfusepath{stroke,fill}%
\end{pgfscope}%
\begin{pgfscope}%
\pgfpathrectangle{\pgfqpoint{0.562500in}{0.275000in}}{\pgfqpoint{3.487500in}{1.925000in}}%
\pgfusepath{clip}%
\pgfsetbuttcap%
\pgfsetroundjoin%
\definecolor{currentfill}{rgb}{0.000000,0.000000,0.000000}%
\pgfsetfillcolor{currentfill}%
\pgfsetlinewidth{1.003750pt}%
\definecolor{currentstroke}{rgb}{0.000000,0.000000,0.000000}%
\pgfsetstrokecolor{currentstroke}%
\pgfsetdash{}{0pt}%
\pgfpathmoveto{\pgfqpoint{1.772992in}{0.518980in}}%
\pgfpathcurveto{\pgfqpoint{1.778517in}{0.518980in}}{\pgfqpoint{1.783816in}{0.521175in}}{\pgfqpoint{1.787723in}{0.525082in}}%
\pgfpathcurveto{\pgfqpoint{1.791630in}{0.528988in}}{\pgfqpoint{1.793825in}{0.534288in}}{\pgfqpoint{1.793825in}{0.539813in}}%
\pgfpathcurveto{\pgfqpoint{1.793825in}{0.545338in}}{\pgfqpoint{1.791630in}{0.550638in}}{\pgfqpoint{1.787723in}{0.554544in}}%
\pgfpathcurveto{\pgfqpoint{1.783816in}{0.558451in}}{\pgfqpoint{1.778517in}{0.560646in}}{\pgfqpoint{1.772992in}{0.560646in}}%
\pgfpathcurveto{\pgfqpoint{1.767467in}{0.560646in}}{\pgfqpoint{1.762167in}{0.558451in}}{\pgfqpoint{1.758260in}{0.554544in}}%
\pgfpathcurveto{\pgfqpoint{1.754353in}{0.550638in}}{\pgfqpoint{1.752158in}{0.545338in}}{\pgfqpoint{1.752158in}{0.539813in}}%
\pgfpathcurveto{\pgfqpoint{1.752158in}{0.534288in}}{\pgfqpoint{1.754353in}{0.528988in}}{\pgfqpoint{1.758260in}{0.525082in}}%
\pgfpathcurveto{\pgfqpoint{1.762167in}{0.521175in}}{\pgfqpoint{1.767467in}{0.518980in}}{\pgfqpoint{1.772992in}{0.518980in}}%
\pgfpathclose%
\pgfusepath{stroke,fill}%
\end{pgfscope}%
\begin{pgfscope}%
\pgfpathrectangle{\pgfqpoint{0.562500in}{0.275000in}}{\pgfqpoint{3.487500in}{1.925000in}}%
\pgfusepath{clip}%
\pgfsetbuttcap%
\pgfsetroundjoin%
\definecolor{currentfill}{rgb}{0.000000,0.000000,0.000000}%
\pgfsetfillcolor{currentfill}%
\pgfsetlinewidth{1.003750pt}%
\definecolor{currentstroke}{rgb}{0.000000,0.000000,0.000000}%
\pgfsetstrokecolor{currentstroke}%
\pgfsetdash{}{0pt}%
\pgfpathmoveto{\pgfqpoint{1.772992in}{0.574624in}}%
\pgfpathcurveto{\pgfqpoint{1.778517in}{0.574624in}}{\pgfqpoint{1.783816in}{0.576819in}}{\pgfqpoint{1.787723in}{0.580726in}}%
\pgfpathcurveto{\pgfqpoint{1.791630in}{0.584633in}}{\pgfqpoint{1.793825in}{0.589932in}}{\pgfqpoint{1.793825in}{0.595457in}}%
\pgfpathcurveto{\pgfqpoint{1.793825in}{0.600982in}}{\pgfqpoint{1.791630in}{0.606282in}}{\pgfqpoint{1.787723in}{0.610189in}}%
\pgfpathcurveto{\pgfqpoint{1.783816in}{0.614095in}}{\pgfqpoint{1.778517in}{0.616291in}}{\pgfqpoint{1.772992in}{0.616291in}}%
\pgfpathcurveto{\pgfqpoint{1.767467in}{0.616291in}}{\pgfqpoint{1.762167in}{0.614095in}}{\pgfqpoint{1.758260in}{0.610189in}}%
\pgfpathcurveto{\pgfqpoint{1.754353in}{0.606282in}}{\pgfqpoint{1.752158in}{0.600982in}}{\pgfqpoint{1.752158in}{0.595457in}}%
\pgfpathcurveto{\pgfqpoint{1.752158in}{0.589932in}}{\pgfqpoint{1.754353in}{0.584633in}}{\pgfqpoint{1.758260in}{0.580726in}}%
\pgfpathcurveto{\pgfqpoint{1.762167in}{0.576819in}}{\pgfqpoint{1.767467in}{0.574624in}}{\pgfqpoint{1.772992in}{0.574624in}}%
\pgfpathclose%
\pgfusepath{stroke,fill}%
\end{pgfscope}%
\begin{pgfscope}%
\pgfpathrectangle{\pgfqpoint{0.562500in}{0.275000in}}{\pgfqpoint{3.487500in}{1.925000in}}%
\pgfusepath{clip}%
\pgfsetbuttcap%
\pgfsetroundjoin%
\definecolor{currentfill}{rgb}{0.000000,0.000000,0.000000}%
\pgfsetfillcolor{currentfill}%
\pgfsetlinewidth{1.003750pt}%
\definecolor{currentstroke}{rgb}{0.000000,0.000000,0.000000}%
\pgfsetstrokecolor{currentstroke}%
\pgfsetdash{}{0pt}%
\pgfpathmoveto{\pgfqpoint{1.772992in}{0.508863in}}%
\pgfpathcurveto{\pgfqpoint{1.778517in}{0.508863in}}{\pgfqpoint{1.783816in}{0.511058in}}{\pgfqpoint{1.787723in}{0.514964in}}%
\pgfpathcurveto{\pgfqpoint{1.791630in}{0.518871in}}{\pgfqpoint{1.793825in}{0.524171in}}{\pgfqpoint{1.793825in}{0.529696in}}%
\pgfpathcurveto{\pgfqpoint{1.793825in}{0.535221in}}{\pgfqpoint{1.791630in}{0.540520in}}{\pgfqpoint{1.787723in}{0.544427in}}%
\pgfpathcurveto{\pgfqpoint{1.783816in}{0.548334in}}{\pgfqpoint{1.778517in}{0.550529in}}{\pgfqpoint{1.772992in}{0.550529in}}%
\pgfpathcurveto{\pgfqpoint{1.767467in}{0.550529in}}{\pgfqpoint{1.762167in}{0.548334in}}{\pgfqpoint{1.758260in}{0.544427in}}%
\pgfpathcurveto{\pgfqpoint{1.754353in}{0.540520in}}{\pgfqpoint{1.752158in}{0.535221in}}{\pgfqpoint{1.752158in}{0.529696in}}%
\pgfpathcurveto{\pgfqpoint{1.752158in}{0.524171in}}{\pgfqpoint{1.754353in}{0.518871in}}{\pgfqpoint{1.758260in}{0.514964in}}%
\pgfpathcurveto{\pgfqpoint{1.762167in}{0.511058in}}{\pgfqpoint{1.767467in}{0.508863in}}{\pgfqpoint{1.772992in}{0.508863in}}%
\pgfpathclose%
\pgfusepath{stroke,fill}%
\end{pgfscope}%
\begin{pgfscope}%
\pgfpathrectangle{\pgfqpoint{0.562500in}{0.275000in}}{\pgfqpoint{3.487500in}{1.925000in}}%
\pgfusepath{clip}%
\pgfsetbuttcap%
\pgfsetroundjoin%
\definecolor{currentfill}{rgb}{0.000000,0.000000,0.000000}%
\pgfsetfillcolor{currentfill}%
\pgfsetlinewidth{1.003750pt}%
\definecolor{currentstroke}{rgb}{0.000000,0.000000,0.000000}%
\pgfsetstrokecolor{currentstroke}%
\pgfsetdash{}{0pt}%
\pgfpathmoveto{\pgfqpoint{1.772992in}{0.513921in}}%
\pgfpathcurveto{\pgfqpoint{1.778517in}{0.513921in}}{\pgfqpoint{1.783816in}{0.516116in}}{\pgfqpoint{1.787723in}{0.520023in}}%
\pgfpathcurveto{\pgfqpoint{1.791630in}{0.523930in}}{\pgfqpoint{1.793825in}{0.529229in}}{\pgfqpoint{1.793825in}{0.534754in}}%
\pgfpathcurveto{\pgfqpoint{1.793825in}{0.540279in}}{\pgfqpoint{1.791630in}{0.545579in}}{\pgfqpoint{1.787723in}{0.549486in}}%
\pgfpathcurveto{\pgfqpoint{1.783816in}{0.553393in}}{\pgfqpoint{1.778517in}{0.555588in}}{\pgfqpoint{1.772992in}{0.555588in}}%
\pgfpathcurveto{\pgfqpoint{1.767467in}{0.555588in}}{\pgfqpoint{1.762167in}{0.553393in}}{\pgfqpoint{1.758260in}{0.549486in}}%
\pgfpathcurveto{\pgfqpoint{1.754353in}{0.545579in}}{\pgfqpoint{1.752158in}{0.540279in}}{\pgfqpoint{1.752158in}{0.534754in}}%
\pgfpathcurveto{\pgfqpoint{1.752158in}{0.529229in}}{\pgfqpoint{1.754353in}{0.523930in}}{\pgfqpoint{1.758260in}{0.520023in}}%
\pgfpathcurveto{\pgfqpoint{1.762167in}{0.516116in}}{\pgfqpoint{1.767467in}{0.513921in}}{\pgfqpoint{1.772992in}{0.513921in}}%
\pgfpathclose%
\pgfusepath{stroke,fill}%
\end{pgfscope}%
\begin{pgfscope}%
\pgfpathrectangle{\pgfqpoint{0.562500in}{0.275000in}}{\pgfqpoint{3.487500in}{1.925000in}}%
\pgfusepath{clip}%
\pgfsetbuttcap%
\pgfsetroundjoin%
\definecolor{currentfill}{rgb}{0.000000,0.000000,0.000000}%
\pgfsetfillcolor{currentfill}%
\pgfsetlinewidth{1.003750pt}%
\definecolor{currentstroke}{rgb}{0.000000,0.000000,0.000000}%
\pgfsetstrokecolor{currentstroke}%
\pgfsetdash{}{0pt}%
\pgfpathmoveto{\pgfqpoint{1.772992in}{0.534155in}}%
\pgfpathcurveto{\pgfqpoint{1.778517in}{0.534155in}}{\pgfqpoint{1.783816in}{0.536350in}}{\pgfqpoint{1.787723in}{0.540257in}}%
\pgfpathcurveto{\pgfqpoint{1.791630in}{0.544164in}}{\pgfqpoint{1.793825in}{0.549464in}}{\pgfqpoint{1.793825in}{0.554989in}}%
\pgfpathcurveto{\pgfqpoint{1.793825in}{0.560514in}}{\pgfqpoint{1.791630in}{0.565813in}}{\pgfqpoint{1.787723in}{0.569720in}}%
\pgfpathcurveto{\pgfqpoint{1.783816in}{0.573627in}}{\pgfqpoint{1.778517in}{0.575822in}}{\pgfqpoint{1.772992in}{0.575822in}}%
\pgfpathcurveto{\pgfqpoint{1.767467in}{0.575822in}}{\pgfqpoint{1.762167in}{0.573627in}}{\pgfqpoint{1.758260in}{0.569720in}}%
\pgfpathcurveto{\pgfqpoint{1.754353in}{0.565813in}}{\pgfqpoint{1.752158in}{0.560514in}}{\pgfqpoint{1.752158in}{0.554989in}}%
\pgfpathcurveto{\pgfqpoint{1.752158in}{0.549464in}}{\pgfqpoint{1.754353in}{0.544164in}}{\pgfqpoint{1.758260in}{0.540257in}}%
\pgfpathcurveto{\pgfqpoint{1.762167in}{0.536350in}}{\pgfqpoint{1.767467in}{0.534155in}}{\pgfqpoint{1.772992in}{0.534155in}}%
\pgfpathclose%
\pgfusepath{stroke,fill}%
\end{pgfscope}%
\begin{pgfscope}%
\pgfpathrectangle{\pgfqpoint{0.562500in}{0.275000in}}{\pgfqpoint{3.487500in}{1.925000in}}%
\pgfusepath{clip}%
\pgfsetbuttcap%
\pgfsetroundjoin%
\definecolor{currentfill}{rgb}{0.000000,0.000000,0.000000}%
\pgfsetfillcolor{currentfill}%
\pgfsetlinewidth{1.003750pt}%
\definecolor{currentstroke}{rgb}{0.000000,0.000000,0.000000}%
\pgfsetstrokecolor{currentstroke}%
\pgfsetdash{}{0pt}%
\pgfpathmoveto{\pgfqpoint{1.772992in}{0.539214in}}%
\pgfpathcurveto{\pgfqpoint{1.778517in}{0.539214in}}{\pgfqpoint{1.783816in}{0.541409in}}{\pgfqpoint{1.787723in}{0.545316in}}%
\pgfpathcurveto{\pgfqpoint{1.791630in}{0.549223in}}{\pgfqpoint{1.793825in}{0.554522in}}{\pgfqpoint{1.793825in}{0.560047in}}%
\pgfpathcurveto{\pgfqpoint{1.793825in}{0.565572in}}{\pgfqpoint{1.791630in}{0.570872in}}{\pgfqpoint{1.787723in}{0.574779in}}%
\pgfpathcurveto{\pgfqpoint{1.783816in}{0.578685in}}{\pgfqpoint{1.778517in}{0.580881in}}{\pgfqpoint{1.772992in}{0.580881in}}%
\pgfpathcurveto{\pgfqpoint{1.767467in}{0.580881in}}{\pgfqpoint{1.762167in}{0.578685in}}{\pgfqpoint{1.758260in}{0.574779in}}%
\pgfpathcurveto{\pgfqpoint{1.754353in}{0.570872in}}{\pgfqpoint{1.752158in}{0.565572in}}{\pgfqpoint{1.752158in}{0.560047in}}%
\pgfpathcurveto{\pgfqpoint{1.752158in}{0.554522in}}{\pgfqpoint{1.754353in}{0.549223in}}{\pgfqpoint{1.758260in}{0.545316in}}%
\pgfpathcurveto{\pgfqpoint{1.762167in}{0.541409in}}{\pgfqpoint{1.767467in}{0.539214in}}{\pgfqpoint{1.772992in}{0.539214in}}%
\pgfpathclose%
\pgfusepath{stroke,fill}%
\end{pgfscope}%
\begin{pgfscope}%
\pgfpathrectangle{\pgfqpoint{0.562500in}{0.275000in}}{\pgfqpoint{3.487500in}{1.925000in}}%
\pgfusepath{clip}%
\pgfsetbuttcap%
\pgfsetroundjoin%
\definecolor{currentfill}{rgb}{0.000000,0.000000,0.000000}%
\pgfsetfillcolor{currentfill}%
\pgfsetlinewidth{1.003750pt}%
\definecolor{currentstroke}{rgb}{0.000000,0.000000,0.000000}%
\pgfsetstrokecolor{currentstroke}%
\pgfsetdash{}{0pt}%
\pgfpathmoveto{\pgfqpoint{1.772992in}{0.488628in}}%
\pgfpathcurveto{\pgfqpoint{1.778517in}{0.488628in}}{\pgfqpoint{1.783816in}{0.490823in}}{\pgfqpoint{1.787723in}{0.494730in}}%
\pgfpathcurveto{\pgfqpoint{1.791630in}{0.498637in}}{\pgfqpoint{1.793825in}{0.503937in}}{\pgfqpoint{1.793825in}{0.509462in}}%
\pgfpathcurveto{\pgfqpoint{1.793825in}{0.514987in}}{\pgfqpoint{1.791630in}{0.520286in}}{\pgfqpoint{1.787723in}{0.524193in}}%
\pgfpathcurveto{\pgfqpoint{1.783816in}{0.528100in}}{\pgfqpoint{1.778517in}{0.530295in}}{\pgfqpoint{1.772992in}{0.530295in}}%
\pgfpathcurveto{\pgfqpoint{1.767467in}{0.530295in}}{\pgfqpoint{1.762167in}{0.528100in}}{\pgfqpoint{1.758260in}{0.524193in}}%
\pgfpathcurveto{\pgfqpoint{1.754353in}{0.520286in}}{\pgfqpoint{1.752158in}{0.514987in}}{\pgfqpoint{1.752158in}{0.509462in}}%
\pgfpathcurveto{\pgfqpoint{1.752158in}{0.503937in}}{\pgfqpoint{1.754353in}{0.498637in}}{\pgfqpoint{1.758260in}{0.494730in}}%
\pgfpathcurveto{\pgfqpoint{1.762167in}{0.490823in}}{\pgfqpoint{1.767467in}{0.488628in}}{\pgfqpoint{1.772992in}{0.488628in}}%
\pgfpathclose%
\pgfusepath{stroke,fill}%
\end{pgfscope}%
\begin{pgfscope}%
\pgfpathrectangle{\pgfqpoint{0.562500in}{0.275000in}}{\pgfqpoint{3.487500in}{1.925000in}}%
\pgfusepath{clip}%
\pgfsetbuttcap%
\pgfsetroundjoin%
\definecolor{currentfill}{rgb}{0.000000,0.000000,0.000000}%
\pgfsetfillcolor{currentfill}%
\pgfsetlinewidth{1.003750pt}%
\definecolor{currentstroke}{rgb}{0.000000,0.000000,0.000000}%
\pgfsetstrokecolor{currentstroke}%
\pgfsetdash{}{0pt}%
\pgfpathmoveto{\pgfqpoint{2.824734in}{0.847786in}}%
\pgfpathcurveto{\pgfqpoint{2.830260in}{0.847786in}}{\pgfqpoint{2.835559in}{0.849982in}}{\pgfqpoint{2.839466in}{0.853888in}}%
\pgfpathcurveto{\pgfqpoint{2.843373in}{0.857795in}}{\pgfqpoint{2.845568in}{0.863095in}}{\pgfqpoint{2.845568in}{0.868620in}}%
\pgfpathcurveto{\pgfqpoint{2.845568in}{0.874145in}}{\pgfqpoint{2.843373in}{0.879444in}}{\pgfqpoint{2.839466in}{0.883351in}}%
\pgfpathcurveto{\pgfqpoint{2.835559in}{0.887258in}}{\pgfqpoint{2.830260in}{0.889453in}}{\pgfqpoint{2.824734in}{0.889453in}}%
\pgfpathcurveto{\pgfqpoint{2.819209in}{0.889453in}}{\pgfqpoint{2.813910in}{0.887258in}}{\pgfqpoint{2.810003in}{0.883351in}}%
\pgfpathcurveto{\pgfqpoint{2.806096in}{0.879444in}}{\pgfqpoint{2.803901in}{0.874145in}}{\pgfqpoint{2.803901in}{0.868620in}}%
\pgfpathcurveto{\pgfqpoint{2.803901in}{0.863095in}}{\pgfqpoint{2.806096in}{0.857795in}}{\pgfqpoint{2.810003in}{0.853888in}}%
\pgfpathcurveto{\pgfqpoint{2.813910in}{0.849982in}}{\pgfqpoint{2.819209in}{0.847786in}}{\pgfqpoint{2.824734in}{0.847786in}}%
\pgfpathclose%
\pgfusepath{stroke,fill}%
\end{pgfscope}%
\begin{pgfscope}%
\pgfpathrectangle{\pgfqpoint{0.562500in}{0.275000in}}{\pgfqpoint{3.487500in}{1.925000in}}%
\pgfusepath{clip}%
\pgfsetbuttcap%
\pgfsetroundjoin%
\definecolor{currentfill}{rgb}{0.000000,0.000000,0.000000}%
\pgfsetfillcolor{currentfill}%
\pgfsetlinewidth{1.003750pt}%
\definecolor{currentstroke}{rgb}{0.000000,0.000000,0.000000}%
\pgfsetstrokecolor{currentstroke}%
\pgfsetdash{}{0pt}%
\pgfpathmoveto{\pgfqpoint{2.824734in}{0.832611in}}%
\pgfpathcurveto{\pgfqpoint{2.830260in}{0.832611in}}{\pgfqpoint{2.835559in}{0.834806in}}{\pgfqpoint{2.839466in}{0.838713in}}%
\pgfpathcurveto{\pgfqpoint{2.843373in}{0.842620in}}{\pgfqpoint{2.845568in}{0.847919in}}{\pgfqpoint{2.845568in}{0.853444in}}%
\pgfpathcurveto{\pgfqpoint{2.845568in}{0.858969in}}{\pgfqpoint{2.843373in}{0.864269in}}{\pgfqpoint{2.839466in}{0.868175in}}%
\pgfpathcurveto{\pgfqpoint{2.835559in}{0.872082in}}{\pgfqpoint{2.830260in}{0.874277in}}{\pgfqpoint{2.824734in}{0.874277in}}%
\pgfpathcurveto{\pgfqpoint{2.819209in}{0.874277in}}{\pgfqpoint{2.813910in}{0.872082in}}{\pgfqpoint{2.810003in}{0.868175in}}%
\pgfpathcurveto{\pgfqpoint{2.806096in}{0.864269in}}{\pgfqpoint{2.803901in}{0.858969in}}{\pgfqpoint{2.803901in}{0.853444in}}%
\pgfpathcurveto{\pgfqpoint{2.803901in}{0.847919in}}{\pgfqpoint{2.806096in}{0.842620in}}{\pgfqpoint{2.810003in}{0.838713in}}%
\pgfpathcurveto{\pgfqpoint{2.813910in}{0.834806in}}{\pgfqpoint{2.819209in}{0.832611in}}{\pgfqpoint{2.824734in}{0.832611in}}%
\pgfpathclose%
\pgfusepath{stroke,fill}%
\end{pgfscope}%
\begin{pgfscope}%
\pgfpathrectangle{\pgfqpoint{0.562500in}{0.275000in}}{\pgfqpoint{3.487500in}{1.925000in}}%
\pgfusepath{clip}%
\pgfsetbuttcap%
\pgfsetroundjoin%
\definecolor{currentfill}{rgb}{0.000000,0.000000,0.000000}%
\pgfsetfillcolor{currentfill}%
\pgfsetlinewidth{1.003750pt}%
\definecolor{currentstroke}{rgb}{0.000000,0.000000,0.000000}%
\pgfsetstrokecolor{currentstroke}%
\pgfsetdash{}{0pt}%
\pgfpathmoveto{\pgfqpoint{2.824734in}{0.812376in}}%
\pgfpathcurveto{\pgfqpoint{2.830260in}{0.812376in}}{\pgfqpoint{2.835559in}{0.814572in}}{\pgfqpoint{2.839466in}{0.818478in}}%
\pgfpathcurveto{\pgfqpoint{2.843373in}{0.822385in}}{\pgfqpoint{2.845568in}{0.827685in}}{\pgfqpoint{2.845568in}{0.833210in}}%
\pgfpathcurveto{\pgfqpoint{2.845568in}{0.838735in}}{\pgfqpoint{2.843373in}{0.844034in}}{\pgfqpoint{2.839466in}{0.847941in}}%
\pgfpathcurveto{\pgfqpoint{2.835559in}{0.851848in}}{\pgfqpoint{2.830260in}{0.854043in}}{\pgfqpoint{2.824734in}{0.854043in}}%
\pgfpathcurveto{\pgfqpoint{2.819209in}{0.854043in}}{\pgfqpoint{2.813910in}{0.851848in}}{\pgfqpoint{2.810003in}{0.847941in}}%
\pgfpathcurveto{\pgfqpoint{2.806096in}{0.844034in}}{\pgfqpoint{2.803901in}{0.838735in}}{\pgfqpoint{2.803901in}{0.833210in}}%
\pgfpathcurveto{\pgfqpoint{2.803901in}{0.827685in}}{\pgfqpoint{2.806096in}{0.822385in}}{\pgfqpoint{2.810003in}{0.818478in}}%
\pgfpathcurveto{\pgfqpoint{2.813910in}{0.814572in}}{\pgfqpoint{2.819209in}{0.812376in}}{\pgfqpoint{2.824734in}{0.812376in}}%
\pgfpathclose%
\pgfusepath{stroke,fill}%
\end{pgfscope}%
\begin{pgfscope}%
\pgfpathrectangle{\pgfqpoint{0.562500in}{0.275000in}}{\pgfqpoint{3.487500in}{1.925000in}}%
\pgfusepath{clip}%
\pgfsetbuttcap%
\pgfsetroundjoin%
\definecolor{currentfill}{rgb}{0.000000,0.000000,0.000000}%
\pgfsetfillcolor{currentfill}%
\pgfsetlinewidth{1.003750pt}%
\definecolor{currentstroke}{rgb}{0.000000,0.000000,0.000000}%
\pgfsetstrokecolor{currentstroke}%
\pgfsetdash{}{0pt}%
\pgfpathmoveto{\pgfqpoint{2.824734in}{0.913548in}}%
\pgfpathcurveto{\pgfqpoint{2.830260in}{0.913548in}}{\pgfqpoint{2.835559in}{0.915743in}}{\pgfqpoint{2.839466in}{0.919650in}}%
\pgfpathcurveto{\pgfqpoint{2.843373in}{0.923557in}}{\pgfqpoint{2.845568in}{0.928856in}}{\pgfqpoint{2.845568in}{0.934381in}}%
\pgfpathcurveto{\pgfqpoint{2.845568in}{0.939906in}}{\pgfqpoint{2.843373in}{0.945206in}}{\pgfqpoint{2.839466in}{0.949113in}}%
\pgfpathcurveto{\pgfqpoint{2.835559in}{0.953019in}}{\pgfqpoint{2.830260in}{0.955214in}}{\pgfqpoint{2.824734in}{0.955214in}}%
\pgfpathcurveto{\pgfqpoint{2.819209in}{0.955214in}}{\pgfqpoint{2.813910in}{0.953019in}}{\pgfqpoint{2.810003in}{0.949113in}}%
\pgfpathcurveto{\pgfqpoint{2.806096in}{0.945206in}}{\pgfqpoint{2.803901in}{0.939906in}}{\pgfqpoint{2.803901in}{0.934381in}}%
\pgfpathcurveto{\pgfqpoint{2.803901in}{0.928856in}}{\pgfqpoint{2.806096in}{0.923557in}}{\pgfqpoint{2.810003in}{0.919650in}}%
\pgfpathcurveto{\pgfqpoint{2.813910in}{0.915743in}}{\pgfqpoint{2.819209in}{0.913548in}}{\pgfqpoint{2.824734in}{0.913548in}}%
\pgfpathclose%
\pgfusepath{stroke,fill}%
\end{pgfscope}%
\begin{pgfscope}%
\pgfpathrectangle{\pgfqpoint{0.562500in}{0.275000in}}{\pgfqpoint{3.487500in}{1.925000in}}%
\pgfusepath{clip}%
\pgfsetbuttcap%
\pgfsetroundjoin%
\definecolor{currentfill}{rgb}{0.000000,0.000000,0.000000}%
\pgfsetfillcolor{currentfill}%
\pgfsetlinewidth{1.003750pt}%
\definecolor{currentstroke}{rgb}{0.000000,0.000000,0.000000}%
\pgfsetstrokecolor{currentstroke}%
\pgfsetdash{}{0pt}%
\pgfpathmoveto{\pgfqpoint{2.824734in}{0.847786in}}%
\pgfpathcurveto{\pgfqpoint{2.830260in}{0.847786in}}{\pgfqpoint{2.835559in}{0.849982in}}{\pgfqpoint{2.839466in}{0.853888in}}%
\pgfpathcurveto{\pgfqpoint{2.843373in}{0.857795in}}{\pgfqpoint{2.845568in}{0.863095in}}{\pgfqpoint{2.845568in}{0.868620in}}%
\pgfpathcurveto{\pgfqpoint{2.845568in}{0.874145in}}{\pgfqpoint{2.843373in}{0.879444in}}{\pgfqpoint{2.839466in}{0.883351in}}%
\pgfpathcurveto{\pgfqpoint{2.835559in}{0.887258in}}{\pgfqpoint{2.830260in}{0.889453in}}{\pgfqpoint{2.824734in}{0.889453in}}%
\pgfpathcurveto{\pgfqpoint{2.819209in}{0.889453in}}{\pgfqpoint{2.813910in}{0.887258in}}{\pgfqpoint{2.810003in}{0.883351in}}%
\pgfpathcurveto{\pgfqpoint{2.806096in}{0.879444in}}{\pgfqpoint{2.803901in}{0.874145in}}{\pgfqpoint{2.803901in}{0.868620in}}%
\pgfpathcurveto{\pgfqpoint{2.803901in}{0.863095in}}{\pgfqpoint{2.806096in}{0.857795in}}{\pgfqpoint{2.810003in}{0.853888in}}%
\pgfpathcurveto{\pgfqpoint{2.813910in}{0.849982in}}{\pgfqpoint{2.819209in}{0.847786in}}{\pgfqpoint{2.824734in}{0.847786in}}%
\pgfpathclose%
\pgfusepath{stroke,fill}%
\end{pgfscope}%
\begin{pgfscope}%
\pgfpathrectangle{\pgfqpoint{0.562500in}{0.275000in}}{\pgfqpoint{3.487500in}{1.925000in}}%
\pgfusepath{clip}%
\pgfsetbuttcap%
\pgfsetroundjoin%
\definecolor{currentfill}{rgb}{0.000000,0.000000,0.000000}%
\pgfsetfillcolor{currentfill}%
\pgfsetlinewidth{1.003750pt}%
\definecolor{currentstroke}{rgb}{0.000000,0.000000,0.000000}%
\pgfsetstrokecolor{currentstroke}%
\pgfsetdash{}{0pt}%
\pgfpathmoveto{\pgfqpoint{2.824734in}{0.817435in}}%
\pgfpathcurveto{\pgfqpoint{2.830260in}{0.817435in}}{\pgfqpoint{2.835559in}{0.819630in}}{\pgfqpoint{2.839466in}{0.823537in}}%
\pgfpathcurveto{\pgfqpoint{2.843373in}{0.827444in}}{\pgfqpoint{2.845568in}{0.832743in}}{\pgfqpoint{2.845568in}{0.838268in}}%
\pgfpathcurveto{\pgfqpoint{2.845568in}{0.843793in}}{\pgfqpoint{2.843373in}{0.849093in}}{\pgfqpoint{2.839466in}{0.853000in}}%
\pgfpathcurveto{\pgfqpoint{2.835559in}{0.856907in}}{\pgfqpoint{2.830260in}{0.859102in}}{\pgfqpoint{2.824734in}{0.859102in}}%
\pgfpathcurveto{\pgfqpoint{2.819209in}{0.859102in}}{\pgfqpoint{2.813910in}{0.856907in}}{\pgfqpoint{2.810003in}{0.853000in}}%
\pgfpathcurveto{\pgfqpoint{2.806096in}{0.849093in}}{\pgfqpoint{2.803901in}{0.843793in}}{\pgfqpoint{2.803901in}{0.838268in}}%
\pgfpathcurveto{\pgfqpoint{2.803901in}{0.832743in}}{\pgfqpoint{2.806096in}{0.827444in}}{\pgfqpoint{2.810003in}{0.823537in}}%
\pgfpathcurveto{\pgfqpoint{2.813910in}{0.819630in}}{\pgfqpoint{2.819209in}{0.817435in}}{\pgfqpoint{2.824734in}{0.817435in}}%
\pgfpathclose%
\pgfusepath{stroke,fill}%
\end{pgfscope}%
\begin{pgfscope}%
\pgfpathrectangle{\pgfqpoint{0.562500in}{0.275000in}}{\pgfqpoint{3.487500in}{1.925000in}}%
\pgfusepath{clip}%
\pgfsetbuttcap%
\pgfsetroundjoin%
\definecolor{currentfill}{rgb}{0.000000,0.000000,0.000000}%
\pgfsetfillcolor{currentfill}%
\pgfsetlinewidth{1.003750pt}%
\definecolor{currentstroke}{rgb}{0.000000,0.000000,0.000000}%
\pgfsetstrokecolor{currentstroke}%
\pgfsetdash{}{0pt}%
\pgfpathmoveto{\pgfqpoint{2.824734in}{1.019778in}}%
\pgfpathcurveto{\pgfqpoint{2.830260in}{1.019778in}}{\pgfqpoint{2.835559in}{1.021973in}}{\pgfqpoint{2.839466in}{1.025880in}}%
\pgfpathcurveto{\pgfqpoint{2.843373in}{1.029786in}}{\pgfqpoint{2.845568in}{1.035086in}}{\pgfqpoint{2.845568in}{1.040611in}}%
\pgfpathcurveto{\pgfqpoint{2.845568in}{1.046136in}}{\pgfqpoint{2.843373in}{1.051436in}}{\pgfqpoint{2.839466in}{1.055342in}}%
\pgfpathcurveto{\pgfqpoint{2.835559in}{1.059249in}}{\pgfqpoint{2.830260in}{1.061444in}}{\pgfqpoint{2.824734in}{1.061444in}}%
\pgfpathcurveto{\pgfqpoint{2.819209in}{1.061444in}}{\pgfqpoint{2.813910in}{1.059249in}}{\pgfqpoint{2.810003in}{1.055342in}}%
\pgfpathcurveto{\pgfqpoint{2.806096in}{1.051436in}}{\pgfqpoint{2.803901in}{1.046136in}}{\pgfqpoint{2.803901in}{1.040611in}}%
\pgfpathcurveto{\pgfqpoint{2.803901in}{1.035086in}}{\pgfqpoint{2.806096in}{1.029786in}}{\pgfqpoint{2.810003in}{1.025880in}}%
\pgfpathcurveto{\pgfqpoint{2.813910in}{1.021973in}}{\pgfqpoint{2.819209in}{1.019778in}}{\pgfqpoint{2.824734in}{1.019778in}}%
\pgfpathclose%
\pgfusepath{stroke,fill}%
\end{pgfscope}%
\begin{pgfscope}%
\pgfpathrectangle{\pgfqpoint{0.562500in}{0.275000in}}{\pgfqpoint{3.487500in}{1.925000in}}%
\pgfusepath{clip}%
\pgfsetbuttcap%
\pgfsetroundjoin%
\definecolor{currentfill}{rgb}{0.000000,0.000000,0.000000}%
\pgfsetfillcolor{currentfill}%
\pgfsetlinewidth{1.003750pt}%
\definecolor{currentstroke}{rgb}{0.000000,0.000000,0.000000}%
\pgfsetstrokecolor{currentstroke}%
\pgfsetdash{}{0pt}%
\pgfpathmoveto{\pgfqpoint{2.824734in}{0.822494in}}%
\pgfpathcurveto{\pgfqpoint{2.830260in}{0.822494in}}{\pgfqpoint{2.835559in}{0.824689in}}{\pgfqpoint{2.839466in}{0.828596in}}%
\pgfpathcurveto{\pgfqpoint{2.843373in}{0.832502in}}{\pgfqpoint{2.845568in}{0.837802in}}{\pgfqpoint{2.845568in}{0.843327in}}%
\pgfpathcurveto{\pgfqpoint{2.845568in}{0.848852in}}{\pgfqpoint{2.843373in}{0.854152in}}{\pgfqpoint{2.839466in}{0.858058in}}%
\pgfpathcurveto{\pgfqpoint{2.835559in}{0.861965in}}{\pgfqpoint{2.830260in}{0.864160in}}{\pgfqpoint{2.824734in}{0.864160in}}%
\pgfpathcurveto{\pgfqpoint{2.819209in}{0.864160in}}{\pgfqpoint{2.813910in}{0.861965in}}{\pgfqpoint{2.810003in}{0.858058in}}%
\pgfpathcurveto{\pgfqpoint{2.806096in}{0.854152in}}{\pgfqpoint{2.803901in}{0.848852in}}{\pgfqpoint{2.803901in}{0.843327in}}%
\pgfpathcurveto{\pgfqpoint{2.803901in}{0.837802in}}{\pgfqpoint{2.806096in}{0.832502in}}{\pgfqpoint{2.810003in}{0.828596in}}%
\pgfpathcurveto{\pgfqpoint{2.813910in}{0.824689in}}{\pgfqpoint{2.819209in}{0.822494in}}{\pgfqpoint{2.824734in}{0.822494in}}%
\pgfpathclose%
\pgfusepath{stroke,fill}%
\end{pgfscope}%
\begin{pgfscope}%
\pgfpathrectangle{\pgfqpoint{0.562500in}{0.275000in}}{\pgfqpoint{3.487500in}{1.925000in}}%
\pgfusepath{clip}%
\pgfsetbuttcap%
\pgfsetroundjoin%
\definecolor{currentfill}{rgb}{0.000000,0.000000,0.000000}%
\pgfsetfillcolor{currentfill}%
\pgfsetlinewidth{1.003750pt}%
\definecolor{currentstroke}{rgb}{0.000000,0.000000,0.000000}%
\pgfsetstrokecolor{currentstroke}%
\pgfsetdash{}{0pt}%
\pgfpathmoveto{\pgfqpoint{2.824734in}{0.837669in}}%
\pgfpathcurveto{\pgfqpoint{2.830260in}{0.837669in}}{\pgfqpoint{2.835559in}{0.839864in}}{\pgfqpoint{2.839466in}{0.843771in}}%
\pgfpathcurveto{\pgfqpoint{2.843373in}{0.847678in}}{\pgfqpoint{2.845568in}{0.852978in}}{\pgfqpoint{2.845568in}{0.858503in}}%
\pgfpathcurveto{\pgfqpoint{2.845568in}{0.864028in}}{\pgfqpoint{2.843373in}{0.869327in}}{\pgfqpoint{2.839466in}{0.873234in}}%
\pgfpathcurveto{\pgfqpoint{2.835559in}{0.877141in}}{\pgfqpoint{2.830260in}{0.879336in}}{\pgfqpoint{2.824734in}{0.879336in}}%
\pgfpathcurveto{\pgfqpoint{2.819209in}{0.879336in}}{\pgfqpoint{2.813910in}{0.877141in}}{\pgfqpoint{2.810003in}{0.873234in}}%
\pgfpathcurveto{\pgfqpoint{2.806096in}{0.869327in}}{\pgfqpoint{2.803901in}{0.864028in}}{\pgfqpoint{2.803901in}{0.858503in}}%
\pgfpathcurveto{\pgfqpoint{2.803901in}{0.852978in}}{\pgfqpoint{2.806096in}{0.847678in}}{\pgfqpoint{2.810003in}{0.843771in}}%
\pgfpathcurveto{\pgfqpoint{2.813910in}{0.839864in}}{\pgfqpoint{2.819209in}{0.837669in}}{\pgfqpoint{2.824734in}{0.837669in}}%
\pgfpathclose%
\pgfusepath{stroke,fill}%
\end{pgfscope}%
\begin{pgfscope}%
\pgfpathrectangle{\pgfqpoint{0.562500in}{0.275000in}}{\pgfqpoint{3.487500in}{1.925000in}}%
\pgfusepath{clip}%
\pgfsetbuttcap%
\pgfsetroundjoin%
\definecolor{currentfill}{rgb}{0.000000,0.000000,0.000000}%
\pgfsetfillcolor{currentfill}%
\pgfsetlinewidth{1.003750pt}%
\definecolor{currentstroke}{rgb}{0.000000,0.000000,0.000000}%
\pgfsetstrokecolor{currentstroke}%
\pgfsetdash{}{0pt}%
\pgfpathmoveto{\pgfqpoint{2.824734in}{0.822494in}}%
\pgfpathcurveto{\pgfqpoint{2.830260in}{0.822494in}}{\pgfqpoint{2.835559in}{0.824689in}}{\pgfqpoint{2.839466in}{0.828596in}}%
\pgfpathcurveto{\pgfqpoint{2.843373in}{0.832502in}}{\pgfqpoint{2.845568in}{0.837802in}}{\pgfqpoint{2.845568in}{0.843327in}}%
\pgfpathcurveto{\pgfqpoint{2.845568in}{0.848852in}}{\pgfqpoint{2.843373in}{0.854152in}}{\pgfqpoint{2.839466in}{0.858058in}}%
\pgfpathcurveto{\pgfqpoint{2.835559in}{0.861965in}}{\pgfqpoint{2.830260in}{0.864160in}}{\pgfqpoint{2.824734in}{0.864160in}}%
\pgfpathcurveto{\pgfqpoint{2.819209in}{0.864160in}}{\pgfqpoint{2.813910in}{0.861965in}}{\pgfqpoint{2.810003in}{0.858058in}}%
\pgfpathcurveto{\pgfqpoint{2.806096in}{0.854152in}}{\pgfqpoint{2.803901in}{0.848852in}}{\pgfqpoint{2.803901in}{0.843327in}}%
\pgfpathcurveto{\pgfqpoint{2.803901in}{0.837802in}}{\pgfqpoint{2.806096in}{0.832502in}}{\pgfqpoint{2.810003in}{0.828596in}}%
\pgfpathcurveto{\pgfqpoint{2.813910in}{0.824689in}}{\pgfqpoint{2.819209in}{0.822494in}}{\pgfqpoint{2.824734in}{0.822494in}}%
\pgfpathclose%
\pgfusepath{stroke,fill}%
\end{pgfscope}%
\begin{pgfscope}%
\pgfpathrectangle{\pgfqpoint{0.562500in}{0.275000in}}{\pgfqpoint{3.487500in}{1.925000in}}%
\pgfusepath{clip}%
\pgfsetbuttcap%
\pgfsetroundjoin%
\definecolor{currentfill}{rgb}{0.000000,0.000000,0.000000}%
\pgfsetfillcolor{currentfill}%
\pgfsetlinewidth{1.003750pt}%
\definecolor{currentstroke}{rgb}{0.000000,0.000000,0.000000}%
\pgfsetstrokecolor{currentstroke}%
\pgfsetdash{}{0pt}%
\pgfpathmoveto{\pgfqpoint{2.824734in}{0.994485in}}%
\pgfpathcurveto{\pgfqpoint{2.830260in}{0.994485in}}{\pgfqpoint{2.835559in}{0.996680in}}{\pgfqpoint{2.839466in}{1.000587in}}%
\pgfpathcurveto{\pgfqpoint{2.843373in}{1.004494in}}{\pgfqpoint{2.845568in}{1.009793in}}{\pgfqpoint{2.845568in}{1.015318in}}%
\pgfpathcurveto{\pgfqpoint{2.845568in}{1.020843in}}{\pgfqpoint{2.843373in}{1.026143in}}{\pgfqpoint{2.839466in}{1.030050in}}%
\pgfpathcurveto{\pgfqpoint{2.835559in}{1.033956in}}{\pgfqpoint{2.830260in}{1.036152in}}{\pgfqpoint{2.824734in}{1.036152in}}%
\pgfpathcurveto{\pgfqpoint{2.819209in}{1.036152in}}{\pgfqpoint{2.813910in}{1.033956in}}{\pgfqpoint{2.810003in}{1.030050in}}%
\pgfpathcurveto{\pgfqpoint{2.806096in}{1.026143in}}{\pgfqpoint{2.803901in}{1.020843in}}{\pgfqpoint{2.803901in}{1.015318in}}%
\pgfpathcurveto{\pgfqpoint{2.803901in}{1.009793in}}{\pgfqpoint{2.806096in}{1.004494in}}{\pgfqpoint{2.810003in}{1.000587in}}%
\pgfpathcurveto{\pgfqpoint{2.813910in}{0.996680in}}{\pgfqpoint{2.819209in}{0.994485in}}{\pgfqpoint{2.824734in}{0.994485in}}%
\pgfpathclose%
\pgfusepath{stroke,fill}%
\end{pgfscope}%
\begin{pgfscope}%
\pgfpathrectangle{\pgfqpoint{0.562500in}{0.275000in}}{\pgfqpoint{3.487500in}{1.925000in}}%
\pgfusepath{clip}%
\pgfsetbuttcap%
\pgfsetroundjoin%
\definecolor{currentfill}{rgb}{0.000000,0.000000,0.000000}%
\pgfsetfillcolor{currentfill}%
\pgfsetlinewidth{1.003750pt}%
\definecolor{currentstroke}{rgb}{0.000000,0.000000,0.000000}%
\pgfsetstrokecolor{currentstroke}%
\pgfsetdash{}{0pt}%
\pgfpathmoveto{\pgfqpoint{2.824734in}{1.004602in}}%
\pgfpathcurveto{\pgfqpoint{2.830260in}{1.004602in}}{\pgfqpoint{2.835559in}{1.006797in}}{\pgfqpoint{2.839466in}{1.010704in}}%
\pgfpathcurveto{\pgfqpoint{2.843373in}{1.014611in}}{\pgfqpoint{2.845568in}{1.019910in}}{\pgfqpoint{2.845568in}{1.025435in}}%
\pgfpathcurveto{\pgfqpoint{2.845568in}{1.030960in}}{\pgfqpoint{2.843373in}{1.036260in}}{\pgfqpoint{2.839466in}{1.040167in}}%
\pgfpathcurveto{\pgfqpoint{2.835559in}{1.044074in}}{\pgfqpoint{2.830260in}{1.046269in}}{\pgfqpoint{2.824734in}{1.046269in}}%
\pgfpathcurveto{\pgfqpoint{2.819209in}{1.046269in}}{\pgfqpoint{2.813910in}{1.044074in}}{\pgfqpoint{2.810003in}{1.040167in}}%
\pgfpathcurveto{\pgfqpoint{2.806096in}{1.036260in}}{\pgfqpoint{2.803901in}{1.030960in}}{\pgfqpoint{2.803901in}{1.025435in}}%
\pgfpathcurveto{\pgfqpoint{2.803901in}{1.019910in}}{\pgfqpoint{2.806096in}{1.014611in}}{\pgfqpoint{2.810003in}{1.010704in}}%
\pgfpathcurveto{\pgfqpoint{2.813910in}{1.006797in}}{\pgfqpoint{2.819209in}{1.004602in}}{\pgfqpoint{2.824734in}{1.004602in}}%
\pgfpathclose%
\pgfusepath{stroke,fill}%
\end{pgfscope}%
\begin{pgfscope}%
\pgfpathrectangle{\pgfqpoint{0.562500in}{0.275000in}}{\pgfqpoint{3.487500in}{1.925000in}}%
\pgfusepath{clip}%
\pgfsetbuttcap%
\pgfsetroundjoin%
\definecolor{currentfill}{rgb}{0.000000,0.000000,0.000000}%
\pgfsetfillcolor{currentfill}%
\pgfsetlinewidth{1.003750pt}%
\definecolor{currentstroke}{rgb}{0.000000,0.000000,0.000000}%
\pgfsetstrokecolor{currentstroke}%
\pgfsetdash{}{0pt}%
\pgfpathmoveto{\pgfqpoint{2.824734in}{0.989426in}}%
\pgfpathcurveto{\pgfqpoint{2.830260in}{0.989426in}}{\pgfqpoint{2.835559in}{0.991621in}}{\pgfqpoint{2.839466in}{0.995528in}}%
\pgfpathcurveto{\pgfqpoint{2.843373in}{0.999435in}}{\pgfqpoint{2.845568in}{1.004735in}}{\pgfqpoint{2.845568in}{1.010260in}}%
\pgfpathcurveto{\pgfqpoint{2.845568in}{1.015785in}}{\pgfqpoint{2.843373in}{1.021084in}}{\pgfqpoint{2.839466in}{1.024991in}}%
\pgfpathcurveto{\pgfqpoint{2.835559in}{1.028898in}}{\pgfqpoint{2.830260in}{1.031093in}}{\pgfqpoint{2.824734in}{1.031093in}}%
\pgfpathcurveto{\pgfqpoint{2.819209in}{1.031093in}}{\pgfqpoint{2.813910in}{1.028898in}}{\pgfqpoint{2.810003in}{1.024991in}}%
\pgfpathcurveto{\pgfqpoint{2.806096in}{1.021084in}}{\pgfqpoint{2.803901in}{1.015785in}}{\pgfqpoint{2.803901in}{1.010260in}}%
\pgfpathcurveto{\pgfqpoint{2.803901in}{1.004735in}}{\pgfqpoint{2.806096in}{0.999435in}}{\pgfqpoint{2.810003in}{0.995528in}}%
\pgfpathcurveto{\pgfqpoint{2.813910in}{0.991621in}}{\pgfqpoint{2.819209in}{0.989426in}}{\pgfqpoint{2.824734in}{0.989426in}}%
\pgfpathclose%
\pgfusepath{stroke,fill}%
\end{pgfscope}%
\begin{pgfscope}%
\pgfpathrectangle{\pgfqpoint{0.562500in}{0.275000in}}{\pgfqpoint{3.487500in}{1.925000in}}%
\pgfusepath{clip}%
\pgfsetbuttcap%
\pgfsetroundjoin%
\definecolor{currentfill}{rgb}{0.000000,0.000000,0.000000}%
\pgfsetfillcolor{currentfill}%
\pgfsetlinewidth{1.003750pt}%
\definecolor{currentstroke}{rgb}{0.000000,0.000000,0.000000}%
\pgfsetstrokecolor{currentstroke}%
\pgfsetdash{}{0pt}%
\pgfpathmoveto{\pgfqpoint{2.824734in}{0.822494in}}%
\pgfpathcurveto{\pgfqpoint{2.830260in}{0.822494in}}{\pgfqpoint{2.835559in}{0.824689in}}{\pgfqpoint{2.839466in}{0.828596in}}%
\pgfpathcurveto{\pgfqpoint{2.843373in}{0.832502in}}{\pgfqpoint{2.845568in}{0.837802in}}{\pgfqpoint{2.845568in}{0.843327in}}%
\pgfpathcurveto{\pgfqpoint{2.845568in}{0.848852in}}{\pgfqpoint{2.843373in}{0.854152in}}{\pgfqpoint{2.839466in}{0.858058in}}%
\pgfpathcurveto{\pgfqpoint{2.835559in}{0.861965in}}{\pgfqpoint{2.830260in}{0.864160in}}{\pgfqpoint{2.824734in}{0.864160in}}%
\pgfpathcurveto{\pgfqpoint{2.819209in}{0.864160in}}{\pgfqpoint{2.813910in}{0.861965in}}{\pgfqpoint{2.810003in}{0.858058in}}%
\pgfpathcurveto{\pgfqpoint{2.806096in}{0.854152in}}{\pgfqpoint{2.803901in}{0.848852in}}{\pgfqpoint{2.803901in}{0.843327in}}%
\pgfpathcurveto{\pgfqpoint{2.803901in}{0.837802in}}{\pgfqpoint{2.806096in}{0.832502in}}{\pgfqpoint{2.810003in}{0.828596in}}%
\pgfpathcurveto{\pgfqpoint{2.813910in}{0.824689in}}{\pgfqpoint{2.819209in}{0.822494in}}{\pgfqpoint{2.824734in}{0.822494in}}%
\pgfpathclose%
\pgfusepath{stroke,fill}%
\end{pgfscope}%
\begin{pgfscope}%
\pgfpathrectangle{\pgfqpoint{0.562500in}{0.275000in}}{\pgfqpoint{3.487500in}{1.925000in}}%
\pgfusepath{clip}%
\pgfsetbuttcap%
\pgfsetroundjoin%
\definecolor{currentfill}{rgb}{0.000000,0.000000,0.000000}%
\pgfsetfillcolor{currentfill}%
\pgfsetlinewidth{1.003750pt}%
\definecolor{currentstroke}{rgb}{0.000000,0.000000,0.000000}%
\pgfsetstrokecolor{currentstroke}%
\pgfsetdash{}{0pt}%
\pgfpathmoveto{\pgfqpoint{2.824734in}{0.964133in}}%
\pgfpathcurveto{\pgfqpoint{2.830260in}{0.964133in}}{\pgfqpoint{2.835559in}{0.966329in}}{\pgfqpoint{2.839466in}{0.970235in}}%
\pgfpathcurveto{\pgfqpoint{2.843373in}{0.974142in}}{\pgfqpoint{2.845568in}{0.979442in}}{\pgfqpoint{2.845568in}{0.984967in}}%
\pgfpathcurveto{\pgfqpoint{2.845568in}{0.990492in}}{\pgfqpoint{2.843373in}{0.995791in}}{\pgfqpoint{2.839466in}{0.999698in}}%
\pgfpathcurveto{\pgfqpoint{2.835559in}{1.003605in}}{\pgfqpoint{2.830260in}{1.005800in}}{\pgfqpoint{2.824734in}{1.005800in}}%
\pgfpathcurveto{\pgfqpoint{2.819209in}{1.005800in}}{\pgfqpoint{2.813910in}{1.003605in}}{\pgfqpoint{2.810003in}{0.999698in}}%
\pgfpathcurveto{\pgfqpoint{2.806096in}{0.995791in}}{\pgfqpoint{2.803901in}{0.990492in}}{\pgfqpoint{2.803901in}{0.984967in}}%
\pgfpathcurveto{\pgfqpoint{2.803901in}{0.979442in}}{\pgfqpoint{2.806096in}{0.974142in}}{\pgfqpoint{2.810003in}{0.970235in}}%
\pgfpathcurveto{\pgfqpoint{2.813910in}{0.966329in}}{\pgfqpoint{2.819209in}{0.964133in}}{\pgfqpoint{2.824734in}{0.964133in}}%
\pgfpathclose%
\pgfusepath{stroke,fill}%
\end{pgfscope}%
\begin{pgfscope}%
\pgfpathrectangle{\pgfqpoint{0.562500in}{0.275000in}}{\pgfqpoint{3.487500in}{1.925000in}}%
\pgfusepath{clip}%
\pgfsetbuttcap%
\pgfsetroundjoin%
\definecolor{currentfill}{rgb}{0.000000,0.000000,0.000000}%
\pgfsetfillcolor{currentfill}%
\pgfsetlinewidth{1.003750pt}%
\definecolor{currentstroke}{rgb}{0.000000,0.000000,0.000000}%
\pgfsetstrokecolor{currentstroke}%
\pgfsetdash{}{0pt}%
\pgfpathmoveto{\pgfqpoint{2.824734in}{0.903431in}}%
\pgfpathcurveto{\pgfqpoint{2.830260in}{0.903431in}}{\pgfqpoint{2.835559in}{0.905626in}}{\pgfqpoint{2.839466in}{0.909533in}}%
\pgfpathcurveto{\pgfqpoint{2.843373in}{0.913439in}}{\pgfqpoint{2.845568in}{0.918739in}}{\pgfqpoint{2.845568in}{0.924264in}}%
\pgfpathcurveto{\pgfqpoint{2.845568in}{0.929789in}}{\pgfqpoint{2.843373in}{0.935089in}}{\pgfqpoint{2.839466in}{0.938995in}}%
\pgfpathcurveto{\pgfqpoint{2.835559in}{0.942902in}}{\pgfqpoint{2.830260in}{0.945097in}}{\pgfqpoint{2.824734in}{0.945097in}}%
\pgfpathcurveto{\pgfqpoint{2.819209in}{0.945097in}}{\pgfqpoint{2.813910in}{0.942902in}}{\pgfqpoint{2.810003in}{0.938995in}}%
\pgfpathcurveto{\pgfqpoint{2.806096in}{0.935089in}}{\pgfqpoint{2.803901in}{0.929789in}}{\pgfqpoint{2.803901in}{0.924264in}}%
\pgfpathcurveto{\pgfqpoint{2.803901in}{0.918739in}}{\pgfqpoint{2.806096in}{0.913439in}}{\pgfqpoint{2.810003in}{0.909533in}}%
\pgfpathcurveto{\pgfqpoint{2.813910in}{0.905626in}}{\pgfqpoint{2.819209in}{0.903431in}}{\pgfqpoint{2.824734in}{0.903431in}}%
\pgfpathclose%
\pgfusepath{stroke,fill}%
\end{pgfscope}%
\begin{pgfscope}%
\pgfpathrectangle{\pgfqpoint{0.562500in}{0.275000in}}{\pgfqpoint{3.487500in}{1.925000in}}%
\pgfusepath{clip}%
\pgfsetbuttcap%
\pgfsetroundjoin%
\definecolor{currentfill}{rgb}{0.000000,0.000000,0.000000}%
\pgfsetfillcolor{currentfill}%
\pgfsetlinewidth{1.003750pt}%
\definecolor{currentstroke}{rgb}{0.000000,0.000000,0.000000}%
\pgfsetstrokecolor{currentstroke}%
\pgfsetdash{}{0pt}%
\pgfpathmoveto{\pgfqpoint{2.824734in}{0.857904in}}%
\pgfpathcurveto{\pgfqpoint{2.830260in}{0.857904in}}{\pgfqpoint{2.835559in}{0.860099in}}{\pgfqpoint{2.839466in}{0.864006in}}%
\pgfpathcurveto{\pgfqpoint{2.843373in}{0.867912in}}{\pgfqpoint{2.845568in}{0.873212in}}{\pgfqpoint{2.845568in}{0.878737in}}%
\pgfpathcurveto{\pgfqpoint{2.845568in}{0.884262in}}{\pgfqpoint{2.843373in}{0.889561in}}{\pgfqpoint{2.839466in}{0.893468in}}%
\pgfpathcurveto{\pgfqpoint{2.835559in}{0.897375in}}{\pgfqpoint{2.830260in}{0.899570in}}{\pgfqpoint{2.824734in}{0.899570in}}%
\pgfpathcurveto{\pgfqpoint{2.819209in}{0.899570in}}{\pgfqpoint{2.813910in}{0.897375in}}{\pgfqpoint{2.810003in}{0.893468in}}%
\pgfpathcurveto{\pgfqpoint{2.806096in}{0.889561in}}{\pgfqpoint{2.803901in}{0.884262in}}{\pgfqpoint{2.803901in}{0.878737in}}%
\pgfpathcurveto{\pgfqpoint{2.803901in}{0.873212in}}{\pgfqpoint{2.806096in}{0.867912in}}{\pgfqpoint{2.810003in}{0.864006in}}%
\pgfpathcurveto{\pgfqpoint{2.813910in}{0.860099in}}{\pgfqpoint{2.819209in}{0.857904in}}{\pgfqpoint{2.824734in}{0.857904in}}%
\pgfpathclose%
\pgfusepath{stroke,fill}%
\end{pgfscope}%
\begin{pgfscope}%
\pgfpathrectangle{\pgfqpoint{0.562500in}{0.275000in}}{\pgfqpoint{3.487500in}{1.925000in}}%
\pgfusepath{clip}%
\pgfsetbuttcap%
\pgfsetroundjoin%
\definecolor{currentfill}{rgb}{0.000000,0.000000,0.000000}%
\pgfsetfillcolor{currentfill}%
\pgfsetlinewidth{1.003750pt}%
\definecolor{currentstroke}{rgb}{0.000000,0.000000,0.000000}%
\pgfsetstrokecolor{currentstroke}%
\pgfsetdash{}{0pt}%
\pgfpathmoveto{\pgfqpoint{2.824734in}{0.933782in}}%
\pgfpathcurveto{\pgfqpoint{2.830260in}{0.933782in}}{\pgfqpoint{2.835559in}{0.935977in}}{\pgfqpoint{2.839466in}{0.939884in}}%
\pgfpathcurveto{\pgfqpoint{2.843373in}{0.943791in}}{\pgfqpoint{2.845568in}{0.949090in}}{\pgfqpoint{2.845568in}{0.954615in}}%
\pgfpathcurveto{\pgfqpoint{2.845568in}{0.960140in}}{\pgfqpoint{2.843373in}{0.965440in}}{\pgfqpoint{2.839466in}{0.969347in}}%
\pgfpathcurveto{\pgfqpoint{2.835559in}{0.973254in}}{\pgfqpoint{2.830260in}{0.975449in}}{\pgfqpoint{2.824734in}{0.975449in}}%
\pgfpathcurveto{\pgfqpoint{2.819209in}{0.975449in}}{\pgfqpoint{2.813910in}{0.973254in}}{\pgfqpoint{2.810003in}{0.969347in}}%
\pgfpathcurveto{\pgfqpoint{2.806096in}{0.965440in}}{\pgfqpoint{2.803901in}{0.960140in}}{\pgfqpoint{2.803901in}{0.954615in}}%
\pgfpathcurveto{\pgfqpoint{2.803901in}{0.949090in}}{\pgfqpoint{2.806096in}{0.943791in}}{\pgfqpoint{2.810003in}{0.939884in}}%
\pgfpathcurveto{\pgfqpoint{2.813910in}{0.935977in}}{\pgfqpoint{2.819209in}{0.933782in}}{\pgfqpoint{2.824734in}{0.933782in}}%
\pgfpathclose%
\pgfusepath{stroke,fill}%
\end{pgfscope}%
\begin{pgfscope}%
\pgfpathrectangle{\pgfqpoint{0.562500in}{0.275000in}}{\pgfqpoint{3.487500in}{1.925000in}}%
\pgfusepath{clip}%
\pgfsetbuttcap%
\pgfsetroundjoin%
\definecolor{currentfill}{rgb}{0.000000,0.000000,0.000000}%
\pgfsetfillcolor{currentfill}%
\pgfsetlinewidth{1.003750pt}%
\definecolor{currentstroke}{rgb}{0.000000,0.000000,0.000000}%
\pgfsetstrokecolor{currentstroke}%
\pgfsetdash{}{0pt}%
\pgfpathmoveto{\pgfqpoint{2.824734in}{0.959075in}}%
\pgfpathcurveto{\pgfqpoint{2.830260in}{0.959075in}}{\pgfqpoint{2.835559in}{0.961270in}}{\pgfqpoint{2.839466in}{0.965177in}}%
\pgfpathcurveto{\pgfqpoint{2.843373in}{0.969084in}}{\pgfqpoint{2.845568in}{0.974383in}}{\pgfqpoint{2.845568in}{0.979908in}}%
\pgfpathcurveto{\pgfqpoint{2.845568in}{0.985433in}}{\pgfqpoint{2.843373in}{0.990733in}}{\pgfqpoint{2.839466in}{0.994640in}}%
\pgfpathcurveto{\pgfqpoint{2.835559in}{0.998546in}}{\pgfqpoint{2.830260in}{1.000742in}}{\pgfqpoint{2.824734in}{1.000742in}}%
\pgfpathcurveto{\pgfqpoint{2.819209in}{1.000742in}}{\pgfqpoint{2.813910in}{0.998546in}}{\pgfqpoint{2.810003in}{0.994640in}}%
\pgfpathcurveto{\pgfqpoint{2.806096in}{0.990733in}}{\pgfqpoint{2.803901in}{0.985433in}}{\pgfqpoint{2.803901in}{0.979908in}}%
\pgfpathcurveto{\pgfqpoint{2.803901in}{0.974383in}}{\pgfqpoint{2.806096in}{0.969084in}}{\pgfqpoint{2.810003in}{0.965177in}}%
\pgfpathcurveto{\pgfqpoint{2.813910in}{0.961270in}}{\pgfqpoint{2.819209in}{0.959075in}}{\pgfqpoint{2.824734in}{0.959075in}}%
\pgfpathclose%
\pgfusepath{stroke,fill}%
\end{pgfscope}%
\begin{pgfscope}%
\pgfpathrectangle{\pgfqpoint{0.562500in}{0.275000in}}{\pgfqpoint{3.487500in}{1.925000in}}%
\pgfusepath{clip}%
\pgfsetbuttcap%
\pgfsetroundjoin%
\definecolor{currentfill}{rgb}{0.000000,0.000000,0.000000}%
\pgfsetfillcolor{currentfill}%
\pgfsetlinewidth{1.003750pt}%
\definecolor{currentstroke}{rgb}{0.000000,0.000000,0.000000}%
\pgfsetstrokecolor{currentstroke}%
\pgfsetdash{}{0pt}%
\pgfpathmoveto{\pgfqpoint{2.824734in}{0.842728in}}%
\pgfpathcurveto{\pgfqpoint{2.830260in}{0.842728in}}{\pgfqpoint{2.835559in}{0.844923in}}{\pgfqpoint{2.839466in}{0.848830in}}%
\pgfpathcurveto{\pgfqpoint{2.843373in}{0.852737in}}{\pgfqpoint{2.845568in}{0.858036in}}{\pgfqpoint{2.845568in}{0.863561in}}%
\pgfpathcurveto{\pgfqpoint{2.845568in}{0.869086in}}{\pgfqpoint{2.843373in}{0.874386in}}{\pgfqpoint{2.839466in}{0.878293in}}%
\pgfpathcurveto{\pgfqpoint{2.835559in}{0.882199in}}{\pgfqpoint{2.830260in}{0.884395in}}{\pgfqpoint{2.824734in}{0.884395in}}%
\pgfpathcurveto{\pgfqpoint{2.819209in}{0.884395in}}{\pgfqpoint{2.813910in}{0.882199in}}{\pgfqpoint{2.810003in}{0.878293in}}%
\pgfpathcurveto{\pgfqpoint{2.806096in}{0.874386in}}{\pgfqpoint{2.803901in}{0.869086in}}{\pgfqpoint{2.803901in}{0.863561in}}%
\pgfpathcurveto{\pgfqpoint{2.803901in}{0.858036in}}{\pgfqpoint{2.806096in}{0.852737in}}{\pgfqpoint{2.810003in}{0.848830in}}%
\pgfpathcurveto{\pgfqpoint{2.813910in}{0.844923in}}{\pgfqpoint{2.819209in}{0.842728in}}{\pgfqpoint{2.824734in}{0.842728in}}%
\pgfpathclose%
\pgfusepath{stroke,fill}%
\end{pgfscope}%
\begin{pgfscope}%
\pgfpathrectangle{\pgfqpoint{0.562500in}{0.275000in}}{\pgfqpoint{3.487500in}{1.925000in}}%
\pgfusepath{clip}%
\pgfsetbuttcap%
\pgfsetroundjoin%
\definecolor{currentfill}{rgb}{0.000000,0.000000,0.000000}%
\pgfsetfillcolor{currentfill}%
\pgfsetlinewidth{1.003750pt}%
\definecolor{currentstroke}{rgb}{0.000000,0.000000,0.000000}%
\pgfsetstrokecolor{currentstroke}%
\pgfsetdash{}{0pt}%
\pgfpathmoveto{\pgfqpoint{2.824734in}{0.807318in}}%
\pgfpathcurveto{\pgfqpoint{2.830260in}{0.807318in}}{\pgfqpoint{2.835559in}{0.809513in}}{\pgfqpoint{2.839466in}{0.813420in}}%
\pgfpathcurveto{\pgfqpoint{2.843373in}{0.817327in}}{\pgfqpoint{2.845568in}{0.822626in}}{\pgfqpoint{2.845568in}{0.828151in}}%
\pgfpathcurveto{\pgfqpoint{2.845568in}{0.833676in}}{\pgfqpoint{2.843373in}{0.838976in}}{\pgfqpoint{2.839466in}{0.842883in}}%
\pgfpathcurveto{\pgfqpoint{2.835559in}{0.846789in}}{\pgfqpoint{2.830260in}{0.848985in}}{\pgfqpoint{2.824734in}{0.848985in}}%
\pgfpathcurveto{\pgfqpoint{2.819209in}{0.848985in}}{\pgfqpoint{2.813910in}{0.846789in}}{\pgfqpoint{2.810003in}{0.842883in}}%
\pgfpathcurveto{\pgfqpoint{2.806096in}{0.838976in}}{\pgfqpoint{2.803901in}{0.833676in}}{\pgfqpoint{2.803901in}{0.828151in}}%
\pgfpathcurveto{\pgfqpoint{2.803901in}{0.822626in}}{\pgfqpoint{2.806096in}{0.817327in}}{\pgfqpoint{2.810003in}{0.813420in}}%
\pgfpathcurveto{\pgfqpoint{2.813910in}{0.809513in}}{\pgfqpoint{2.819209in}{0.807318in}}{\pgfqpoint{2.824734in}{0.807318in}}%
\pgfpathclose%
\pgfusepath{stroke,fill}%
\end{pgfscope}%
\begin{pgfscope}%
\pgfpathrectangle{\pgfqpoint{0.562500in}{0.275000in}}{\pgfqpoint{3.487500in}{1.925000in}}%
\pgfusepath{clip}%
\pgfsetbuttcap%
\pgfsetroundjoin%
\definecolor{currentfill}{rgb}{0.000000,0.000000,0.000000}%
\pgfsetfillcolor{currentfill}%
\pgfsetlinewidth{1.003750pt}%
\definecolor{currentstroke}{rgb}{0.000000,0.000000,0.000000}%
\pgfsetstrokecolor{currentstroke}%
\pgfsetdash{}{0pt}%
\pgfpathmoveto{\pgfqpoint{2.824734in}{0.832611in}}%
\pgfpathcurveto{\pgfqpoint{2.830260in}{0.832611in}}{\pgfqpoint{2.835559in}{0.834806in}}{\pgfqpoint{2.839466in}{0.838713in}}%
\pgfpathcurveto{\pgfqpoint{2.843373in}{0.842620in}}{\pgfqpoint{2.845568in}{0.847919in}}{\pgfqpoint{2.845568in}{0.853444in}}%
\pgfpathcurveto{\pgfqpoint{2.845568in}{0.858969in}}{\pgfqpoint{2.843373in}{0.864269in}}{\pgfqpoint{2.839466in}{0.868175in}}%
\pgfpathcurveto{\pgfqpoint{2.835559in}{0.872082in}}{\pgfqpoint{2.830260in}{0.874277in}}{\pgfqpoint{2.824734in}{0.874277in}}%
\pgfpathcurveto{\pgfqpoint{2.819209in}{0.874277in}}{\pgfqpoint{2.813910in}{0.872082in}}{\pgfqpoint{2.810003in}{0.868175in}}%
\pgfpathcurveto{\pgfqpoint{2.806096in}{0.864269in}}{\pgfqpoint{2.803901in}{0.858969in}}{\pgfqpoint{2.803901in}{0.853444in}}%
\pgfpathcurveto{\pgfqpoint{2.803901in}{0.847919in}}{\pgfqpoint{2.806096in}{0.842620in}}{\pgfqpoint{2.810003in}{0.838713in}}%
\pgfpathcurveto{\pgfqpoint{2.813910in}{0.834806in}}{\pgfqpoint{2.819209in}{0.832611in}}{\pgfqpoint{2.824734in}{0.832611in}}%
\pgfpathclose%
\pgfusepath{stroke,fill}%
\end{pgfscope}%
\begin{pgfscope}%
\pgfpathrectangle{\pgfqpoint{0.562500in}{0.275000in}}{\pgfqpoint{3.487500in}{1.925000in}}%
\pgfusepath{clip}%
\pgfsetbuttcap%
\pgfsetroundjoin%
\definecolor{currentfill}{rgb}{0.000000,0.000000,0.000000}%
\pgfsetfillcolor{currentfill}%
\pgfsetlinewidth{1.003750pt}%
\definecolor{currentstroke}{rgb}{0.000000,0.000000,0.000000}%
\pgfsetstrokecolor{currentstroke}%
\pgfsetdash{}{0pt}%
\pgfpathmoveto{\pgfqpoint{2.824734in}{0.918606in}}%
\pgfpathcurveto{\pgfqpoint{2.830260in}{0.918606in}}{\pgfqpoint{2.835559in}{0.920802in}}{\pgfqpoint{2.839466in}{0.924708in}}%
\pgfpathcurveto{\pgfqpoint{2.843373in}{0.928615in}}{\pgfqpoint{2.845568in}{0.933915in}}{\pgfqpoint{2.845568in}{0.939440in}}%
\pgfpathcurveto{\pgfqpoint{2.845568in}{0.944965in}}{\pgfqpoint{2.843373in}{0.950264in}}{\pgfqpoint{2.839466in}{0.954171in}}%
\pgfpathcurveto{\pgfqpoint{2.835559in}{0.958078in}}{\pgfqpoint{2.830260in}{0.960273in}}{\pgfqpoint{2.824734in}{0.960273in}}%
\pgfpathcurveto{\pgfqpoint{2.819209in}{0.960273in}}{\pgfqpoint{2.813910in}{0.958078in}}{\pgfqpoint{2.810003in}{0.954171in}}%
\pgfpathcurveto{\pgfqpoint{2.806096in}{0.950264in}}{\pgfqpoint{2.803901in}{0.944965in}}{\pgfqpoint{2.803901in}{0.939440in}}%
\pgfpathcurveto{\pgfqpoint{2.803901in}{0.933915in}}{\pgfqpoint{2.806096in}{0.928615in}}{\pgfqpoint{2.810003in}{0.924708in}}%
\pgfpathcurveto{\pgfqpoint{2.813910in}{0.920802in}}{\pgfqpoint{2.819209in}{0.918606in}}{\pgfqpoint{2.824734in}{0.918606in}}%
\pgfpathclose%
\pgfusepath{stroke,fill}%
\end{pgfscope}%
\begin{pgfscope}%
\pgfpathrectangle{\pgfqpoint{0.562500in}{0.275000in}}{\pgfqpoint{3.487500in}{1.925000in}}%
\pgfusepath{clip}%
\pgfsetbuttcap%
\pgfsetroundjoin%
\definecolor{currentfill}{rgb}{0.000000,0.000000,0.000000}%
\pgfsetfillcolor{currentfill}%
\pgfsetlinewidth{1.003750pt}%
\definecolor{currentstroke}{rgb}{0.000000,0.000000,0.000000}%
\pgfsetstrokecolor{currentstroke}%
\pgfsetdash{}{0pt}%
\pgfpathmoveto{\pgfqpoint{2.824734in}{0.893314in}}%
\pgfpathcurveto{\pgfqpoint{2.830260in}{0.893314in}}{\pgfqpoint{2.835559in}{0.895509in}}{\pgfqpoint{2.839466in}{0.899415in}}%
\pgfpathcurveto{\pgfqpoint{2.843373in}{0.903322in}}{\pgfqpoint{2.845568in}{0.908622in}}{\pgfqpoint{2.845568in}{0.914147in}}%
\pgfpathcurveto{\pgfqpoint{2.845568in}{0.919672in}}{\pgfqpoint{2.843373in}{0.924971in}}{\pgfqpoint{2.839466in}{0.928878in}}%
\pgfpathcurveto{\pgfqpoint{2.835559in}{0.932785in}}{\pgfqpoint{2.830260in}{0.934980in}}{\pgfqpoint{2.824734in}{0.934980in}}%
\pgfpathcurveto{\pgfqpoint{2.819209in}{0.934980in}}{\pgfqpoint{2.813910in}{0.932785in}}{\pgfqpoint{2.810003in}{0.928878in}}%
\pgfpathcurveto{\pgfqpoint{2.806096in}{0.924971in}}{\pgfqpoint{2.803901in}{0.919672in}}{\pgfqpoint{2.803901in}{0.914147in}}%
\pgfpathcurveto{\pgfqpoint{2.803901in}{0.908622in}}{\pgfqpoint{2.806096in}{0.903322in}}{\pgfqpoint{2.810003in}{0.899415in}}%
\pgfpathcurveto{\pgfqpoint{2.813910in}{0.895509in}}{\pgfqpoint{2.819209in}{0.893314in}}{\pgfqpoint{2.824734in}{0.893314in}}%
\pgfpathclose%
\pgfusepath{stroke,fill}%
\end{pgfscope}%
\begin{pgfscope}%
\pgfpathrectangle{\pgfqpoint{0.562500in}{0.275000in}}{\pgfqpoint{3.487500in}{1.925000in}}%
\pgfusepath{clip}%
\pgfsetbuttcap%
\pgfsetroundjoin%
\definecolor{currentfill}{rgb}{0.000000,0.000000,0.000000}%
\pgfsetfillcolor{currentfill}%
\pgfsetlinewidth{1.003750pt}%
\definecolor{currentstroke}{rgb}{0.000000,0.000000,0.000000}%
\pgfsetstrokecolor{currentstroke}%
\pgfsetdash{}{0pt}%
\pgfpathmoveto{\pgfqpoint{2.824734in}{0.832611in}}%
\pgfpathcurveto{\pgfqpoint{2.830260in}{0.832611in}}{\pgfqpoint{2.835559in}{0.834806in}}{\pgfqpoint{2.839466in}{0.838713in}}%
\pgfpathcurveto{\pgfqpoint{2.843373in}{0.842620in}}{\pgfqpoint{2.845568in}{0.847919in}}{\pgfqpoint{2.845568in}{0.853444in}}%
\pgfpathcurveto{\pgfqpoint{2.845568in}{0.858969in}}{\pgfqpoint{2.843373in}{0.864269in}}{\pgfqpoint{2.839466in}{0.868175in}}%
\pgfpathcurveto{\pgfqpoint{2.835559in}{0.872082in}}{\pgfqpoint{2.830260in}{0.874277in}}{\pgfqpoint{2.824734in}{0.874277in}}%
\pgfpathcurveto{\pgfqpoint{2.819209in}{0.874277in}}{\pgfqpoint{2.813910in}{0.872082in}}{\pgfqpoint{2.810003in}{0.868175in}}%
\pgfpathcurveto{\pgfqpoint{2.806096in}{0.864269in}}{\pgfqpoint{2.803901in}{0.858969in}}{\pgfqpoint{2.803901in}{0.853444in}}%
\pgfpathcurveto{\pgfqpoint{2.803901in}{0.847919in}}{\pgfqpoint{2.806096in}{0.842620in}}{\pgfqpoint{2.810003in}{0.838713in}}%
\pgfpathcurveto{\pgfqpoint{2.813910in}{0.834806in}}{\pgfqpoint{2.819209in}{0.832611in}}{\pgfqpoint{2.824734in}{0.832611in}}%
\pgfpathclose%
\pgfusepath{stroke,fill}%
\end{pgfscope}%
\begin{pgfscope}%
\pgfpathrectangle{\pgfqpoint{0.562500in}{0.275000in}}{\pgfqpoint{3.487500in}{1.925000in}}%
\pgfusepath{clip}%
\pgfsetbuttcap%
\pgfsetroundjoin%
\definecolor{currentfill}{rgb}{0.000000,0.000000,0.000000}%
\pgfsetfillcolor{currentfill}%
\pgfsetlinewidth{1.003750pt}%
\definecolor{currentstroke}{rgb}{0.000000,0.000000,0.000000}%
\pgfsetstrokecolor{currentstroke}%
\pgfsetdash{}{0pt}%
\pgfpathmoveto{\pgfqpoint{2.824734in}{0.847786in}}%
\pgfpathcurveto{\pgfqpoint{2.830260in}{0.847786in}}{\pgfqpoint{2.835559in}{0.849982in}}{\pgfqpoint{2.839466in}{0.853888in}}%
\pgfpathcurveto{\pgfqpoint{2.843373in}{0.857795in}}{\pgfqpoint{2.845568in}{0.863095in}}{\pgfqpoint{2.845568in}{0.868620in}}%
\pgfpathcurveto{\pgfqpoint{2.845568in}{0.874145in}}{\pgfqpoint{2.843373in}{0.879444in}}{\pgfqpoint{2.839466in}{0.883351in}}%
\pgfpathcurveto{\pgfqpoint{2.835559in}{0.887258in}}{\pgfqpoint{2.830260in}{0.889453in}}{\pgfqpoint{2.824734in}{0.889453in}}%
\pgfpathcurveto{\pgfqpoint{2.819209in}{0.889453in}}{\pgfqpoint{2.813910in}{0.887258in}}{\pgfqpoint{2.810003in}{0.883351in}}%
\pgfpathcurveto{\pgfqpoint{2.806096in}{0.879444in}}{\pgfqpoint{2.803901in}{0.874145in}}{\pgfqpoint{2.803901in}{0.868620in}}%
\pgfpathcurveto{\pgfqpoint{2.803901in}{0.863095in}}{\pgfqpoint{2.806096in}{0.857795in}}{\pgfqpoint{2.810003in}{0.853888in}}%
\pgfpathcurveto{\pgfqpoint{2.813910in}{0.849982in}}{\pgfqpoint{2.819209in}{0.847786in}}{\pgfqpoint{2.824734in}{0.847786in}}%
\pgfpathclose%
\pgfusepath{stroke,fill}%
\end{pgfscope}%
\begin{pgfscope}%
\pgfpathrectangle{\pgfqpoint{0.562500in}{0.275000in}}{\pgfqpoint{3.487500in}{1.925000in}}%
\pgfusepath{clip}%
\pgfsetbuttcap%
\pgfsetroundjoin%
\definecolor{currentfill}{rgb}{0.000000,0.000000,0.000000}%
\pgfsetfillcolor{currentfill}%
\pgfsetlinewidth{1.003750pt}%
\definecolor{currentstroke}{rgb}{0.000000,0.000000,0.000000}%
\pgfsetstrokecolor{currentstroke}%
\pgfsetdash{}{0pt}%
\pgfpathmoveto{\pgfqpoint{2.824734in}{0.883196in}}%
\pgfpathcurveto{\pgfqpoint{2.830260in}{0.883196in}}{\pgfqpoint{2.835559in}{0.885392in}}{\pgfqpoint{2.839466in}{0.889298in}}%
\pgfpathcurveto{\pgfqpoint{2.843373in}{0.893205in}}{\pgfqpoint{2.845568in}{0.898505in}}{\pgfqpoint{2.845568in}{0.904030in}}%
\pgfpathcurveto{\pgfqpoint{2.845568in}{0.909555in}}{\pgfqpoint{2.843373in}{0.914854in}}{\pgfqpoint{2.839466in}{0.918761in}}%
\pgfpathcurveto{\pgfqpoint{2.835559in}{0.922668in}}{\pgfqpoint{2.830260in}{0.924863in}}{\pgfqpoint{2.824734in}{0.924863in}}%
\pgfpathcurveto{\pgfqpoint{2.819209in}{0.924863in}}{\pgfqpoint{2.813910in}{0.922668in}}{\pgfqpoint{2.810003in}{0.918761in}}%
\pgfpathcurveto{\pgfqpoint{2.806096in}{0.914854in}}{\pgfqpoint{2.803901in}{0.909555in}}{\pgfqpoint{2.803901in}{0.904030in}}%
\pgfpathcurveto{\pgfqpoint{2.803901in}{0.898505in}}{\pgfqpoint{2.806096in}{0.893205in}}{\pgfqpoint{2.810003in}{0.889298in}}%
\pgfpathcurveto{\pgfqpoint{2.813910in}{0.885392in}}{\pgfqpoint{2.819209in}{0.883196in}}{\pgfqpoint{2.824734in}{0.883196in}}%
\pgfpathclose%
\pgfusepath{stroke,fill}%
\end{pgfscope}%
\begin{pgfscope}%
\pgfpathrectangle{\pgfqpoint{0.562500in}{0.275000in}}{\pgfqpoint{3.487500in}{1.925000in}}%
\pgfusepath{clip}%
\pgfsetbuttcap%
\pgfsetroundjoin%
\definecolor{currentfill}{rgb}{0.000000,0.000000,0.000000}%
\pgfsetfillcolor{currentfill}%
\pgfsetlinewidth{1.003750pt}%
\definecolor{currentstroke}{rgb}{0.000000,0.000000,0.000000}%
\pgfsetstrokecolor{currentstroke}%
\pgfsetdash{}{0pt}%
\pgfpathmoveto{\pgfqpoint{2.824734in}{0.837669in}}%
\pgfpathcurveto{\pgfqpoint{2.830260in}{0.837669in}}{\pgfqpoint{2.835559in}{0.839864in}}{\pgfqpoint{2.839466in}{0.843771in}}%
\pgfpathcurveto{\pgfqpoint{2.843373in}{0.847678in}}{\pgfqpoint{2.845568in}{0.852978in}}{\pgfqpoint{2.845568in}{0.858503in}}%
\pgfpathcurveto{\pgfqpoint{2.845568in}{0.864028in}}{\pgfqpoint{2.843373in}{0.869327in}}{\pgfqpoint{2.839466in}{0.873234in}}%
\pgfpathcurveto{\pgfqpoint{2.835559in}{0.877141in}}{\pgfqpoint{2.830260in}{0.879336in}}{\pgfqpoint{2.824734in}{0.879336in}}%
\pgfpathcurveto{\pgfqpoint{2.819209in}{0.879336in}}{\pgfqpoint{2.813910in}{0.877141in}}{\pgfqpoint{2.810003in}{0.873234in}}%
\pgfpathcurveto{\pgfqpoint{2.806096in}{0.869327in}}{\pgfqpoint{2.803901in}{0.864028in}}{\pgfqpoint{2.803901in}{0.858503in}}%
\pgfpathcurveto{\pgfqpoint{2.803901in}{0.852978in}}{\pgfqpoint{2.806096in}{0.847678in}}{\pgfqpoint{2.810003in}{0.843771in}}%
\pgfpathcurveto{\pgfqpoint{2.813910in}{0.839864in}}{\pgfqpoint{2.819209in}{0.837669in}}{\pgfqpoint{2.824734in}{0.837669in}}%
\pgfpathclose%
\pgfusepath{stroke,fill}%
\end{pgfscope}%
\begin{pgfscope}%
\pgfpathrectangle{\pgfqpoint{0.562500in}{0.275000in}}{\pgfqpoint{3.487500in}{1.925000in}}%
\pgfusepath{clip}%
\pgfsetbuttcap%
\pgfsetroundjoin%
\definecolor{currentfill}{rgb}{0.000000,0.000000,0.000000}%
\pgfsetfillcolor{currentfill}%
\pgfsetlinewidth{1.003750pt}%
\definecolor{currentstroke}{rgb}{0.000000,0.000000,0.000000}%
\pgfsetstrokecolor{currentstroke}%
\pgfsetdash{}{0pt}%
\pgfpathmoveto{\pgfqpoint{2.824734in}{0.913548in}}%
\pgfpathcurveto{\pgfqpoint{2.830260in}{0.913548in}}{\pgfqpoint{2.835559in}{0.915743in}}{\pgfqpoint{2.839466in}{0.919650in}}%
\pgfpathcurveto{\pgfqpoint{2.843373in}{0.923557in}}{\pgfqpoint{2.845568in}{0.928856in}}{\pgfqpoint{2.845568in}{0.934381in}}%
\pgfpathcurveto{\pgfqpoint{2.845568in}{0.939906in}}{\pgfqpoint{2.843373in}{0.945206in}}{\pgfqpoint{2.839466in}{0.949113in}}%
\pgfpathcurveto{\pgfqpoint{2.835559in}{0.953019in}}{\pgfqpoint{2.830260in}{0.955214in}}{\pgfqpoint{2.824734in}{0.955214in}}%
\pgfpathcurveto{\pgfqpoint{2.819209in}{0.955214in}}{\pgfqpoint{2.813910in}{0.953019in}}{\pgfqpoint{2.810003in}{0.949113in}}%
\pgfpathcurveto{\pgfqpoint{2.806096in}{0.945206in}}{\pgfqpoint{2.803901in}{0.939906in}}{\pgfqpoint{2.803901in}{0.934381in}}%
\pgfpathcurveto{\pgfqpoint{2.803901in}{0.928856in}}{\pgfqpoint{2.806096in}{0.923557in}}{\pgfqpoint{2.810003in}{0.919650in}}%
\pgfpathcurveto{\pgfqpoint{2.813910in}{0.915743in}}{\pgfqpoint{2.819209in}{0.913548in}}{\pgfqpoint{2.824734in}{0.913548in}}%
\pgfpathclose%
\pgfusepath{stroke,fill}%
\end{pgfscope}%
\begin{pgfscope}%
\pgfpathrectangle{\pgfqpoint{0.562500in}{0.275000in}}{\pgfqpoint{3.487500in}{1.925000in}}%
\pgfusepath{clip}%
\pgfsetbuttcap%
\pgfsetroundjoin%
\definecolor{currentfill}{rgb}{0.000000,0.000000,0.000000}%
\pgfsetfillcolor{currentfill}%
\pgfsetlinewidth{1.003750pt}%
\definecolor{currentstroke}{rgb}{0.000000,0.000000,0.000000}%
\pgfsetstrokecolor{currentstroke}%
\pgfsetdash{}{0pt}%
\pgfpathmoveto{\pgfqpoint{2.824734in}{0.893314in}}%
\pgfpathcurveto{\pgfqpoint{2.830260in}{0.893314in}}{\pgfqpoint{2.835559in}{0.895509in}}{\pgfqpoint{2.839466in}{0.899415in}}%
\pgfpathcurveto{\pgfqpoint{2.843373in}{0.903322in}}{\pgfqpoint{2.845568in}{0.908622in}}{\pgfqpoint{2.845568in}{0.914147in}}%
\pgfpathcurveto{\pgfqpoint{2.845568in}{0.919672in}}{\pgfqpoint{2.843373in}{0.924971in}}{\pgfqpoint{2.839466in}{0.928878in}}%
\pgfpathcurveto{\pgfqpoint{2.835559in}{0.932785in}}{\pgfqpoint{2.830260in}{0.934980in}}{\pgfqpoint{2.824734in}{0.934980in}}%
\pgfpathcurveto{\pgfqpoint{2.819209in}{0.934980in}}{\pgfqpoint{2.813910in}{0.932785in}}{\pgfqpoint{2.810003in}{0.928878in}}%
\pgfpathcurveto{\pgfqpoint{2.806096in}{0.924971in}}{\pgfqpoint{2.803901in}{0.919672in}}{\pgfqpoint{2.803901in}{0.914147in}}%
\pgfpathcurveto{\pgfqpoint{2.803901in}{0.908622in}}{\pgfqpoint{2.806096in}{0.903322in}}{\pgfqpoint{2.810003in}{0.899415in}}%
\pgfpathcurveto{\pgfqpoint{2.813910in}{0.895509in}}{\pgfqpoint{2.819209in}{0.893314in}}{\pgfqpoint{2.824734in}{0.893314in}}%
\pgfpathclose%
\pgfusepath{stroke,fill}%
\end{pgfscope}%
\begin{pgfscope}%
\pgfpathrectangle{\pgfqpoint{0.562500in}{0.275000in}}{\pgfqpoint{3.487500in}{1.925000in}}%
\pgfusepath{clip}%
\pgfsetbuttcap%
\pgfsetroundjoin%
\definecolor{currentfill}{rgb}{0.000000,0.000000,0.000000}%
\pgfsetfillcolor{currentfill}%
\pgfsetlinewidth{1.003750pt}%
\definecolor{currentstroke}{rgb}{0.000000,0.000000,0.000000}%
\pgfsetstrokecolor{currentstroke}%
\pgfsetdash{}{0pt}%
\pgfpathmoveto{\pgfqpoint{2.824734in}{0.837669in}}%
\pgfpathcurveto{\pgfqpoint{2.830260in}{0.837669in}}{\pgfqpoint{2.835559in}{0.839864in}}{\pgfqpoint{2.839466in}{0.843771in}}%
\pgfpathcurveto{\pgfqpoint{2.843373in}{0.847678in}}{\pgfqpoint{2.845568in}{0.852978in}}{\pgfqpoint{2.845568in}{0.858503in}}%
\pgfpathcurveto{\pgfqpoint{2.845568in}{0.864028in}}{\pgfqpoint{2.843373in}{0.869327in}}{\pgfqpoint{2.839466in}{0.873234in}}%
\pgfpathcurveto{\pgfqpoint{2.835559in}{0.877141in}}{\pgfqpoint{2.830260in}{0.879336in}}{\pgfqpoint{2.824734in}{0.879336in}}%
\pgfpathcurveto{\pgfqpoint{2.819209in}{0.879336in}}{\pgfqpoint{2.813910in}{0.877141in}}{\pgfqpoint{2.810003in}{0.873234in}}%
\pgfpathcurveto{\pgfqpoint{2.806096in}{0.869327in}}{\pgfqpoint{2.803901in}{0.864028in}}{\pgfqpoint{2.803901in}{0.858503in}}%
\pgfpathcurveto{\pgfqpoint{2.803901in}{0.852978in}}{\pgfqpoint{2.806096in}{0.847678in}}{\pgfqpoint{2.810003in}{0.843771in}}%
\pgfpathcurveto{\pgfqpoint{2.813910in}{0.839864in}}{\pgfqpoint{2.819209in}{0.837669in}}{\pgfqpoint{2.824734in}{0.837669in}}%
\pgfpathclose%
\pgfusepath{stroke,fill}%
\end{pgfscope}%
\begin{pgfscope}%
\pgfpathrectangle{\pgfqpoint{0.562500in}{0.275000in}}{\pgfqpoint{3.487500in}{1.925000in}}%
\pgfusepath{clip}%
\pgfsetbuttcap%
\pgfsetroundjoin%
\definecolor{currentfill}{rgb}{0.000000,0.000000,0.000000}%
\pgfsetfillcolor{currentfill}%
\pgfsetlinewidth{1.003750pt}%
\definecolor{currentstroke}{rgb}{0.000000,0.000000,0.000000}%
\pgfsetstrokecolor{currentstroke}%
\pgfsetdash{}{0pt}%
\pgfpathmoveto{\pgfqpoint{2.824734in}{0.948958in}}%
\pgfpathcurveto{\pgfqpoint{2.830260in}{0.948958in}}{\pgfqpoint{2.835559in}{0.951153in}}{\pgfqpoint{2.839466in}{0.955060in}}%
\pgfpathcurveto{\pgfqpoint{2.843373in}{0.958967in}}{\pgfqpoint{2.845568in}{0.964266in}}{\pgfqpoint{2.845568in}{0.969791in}}%
\pgfpathcurveto{\pgfqpoint{2.845568in}{0.975316in}}{\pgfqpoint{2.843373in}{0.980616in}}{\pgfqpoint{2.839466in}{0.984522in}}%
\pgfpathcurveto{\pgfqpoint{2.835559in}{0.988429in}}{\pgfqpoint{2.830260in}{0.990624in}}{\pgfqpoint{2.824734in}{0.990624in}}%
\pgfpathcurveto{\pgfqpoint{2.819209in}{0.990624in}}{\pgfqpoint{2.813910in}{0.988429in}}{\pgfqpoint{2.810003in}{0.984522in}}%
\pgfpathcurveto{\pgfqpoint{2.806096in}{0.980616in}}{\pgfqpoint{2.803901in}{0.975316in}}{\pgfqpoint{2.803901in}{0.969791in}}%
\pgfpathcurveto{\pgfqpoint{2.803901in}{0.964266in}}{\pgfqpoint{2.806096in}{0.958967in}}{\pgfqpoint{2.810003in}{0.955060in}}%
\pgfpathcurveto{\pgfqpoint{2.813910in}{0.951153in}}{\pgfqpoint{2.819209in}{0.948958in}}{\pgfqpoint{2.824734in}{0.948958in}}%
\pgfpathclose%
\pgfusepath{stroke,fill}%
\end{pgfscope}%
\begin{pgfscope}%
\pgfpathrectangle{\pgfqpoint{0.562500in}{0.275000in}}{\pgfqpoint{3.487500in}{1.925000in}}%
\pgfusepath{clip}%
\pgfsetbuttcap%
\pgfsetroundjoin%
\definecolor{currentfill}{rgb}{0.000000,0.000000,0.000000}%
\pgfsetfillcolor{currentfill}%
\pgfsetlinewidth{1.003750pt}%
\definecolor{currentstroke}{rgb}{0.000000,0.000000,0.000000}%
\pgfsetstrokecolor{currentstroke}%
\pgfsetdash{}{0pt}%
\pgfpathmoveto{\pgfqpoint{2.824734in}{0.908489in}}%
\pgfpathcurveto{\pgfqpoint{2.830260in}{0.908489in}}{\pgfqpoint{2.835559in}{0.910684in}}{\pgfqpoint{2.839466in}{0.914591in}}%
\pgfpathcurveto{\pgfqpoint{2.843373in}{0.918498in}}{\pgfqpoint{2.845568in}{0.923798in}}{\pgfqpoint{2.845568in}{0.929323in}}%
\pgfpathcurveto{\pgfqpoint{2.845568in}{0.934848in}}{\pgfqpoint{2.843373in}{0.940147in}}{\pgfqpoint{2.839466in}{0.944054in}}%
\pgfpathcurveto{\pgfqpoint{2.835559in}{0.947961in}}{\pgfqpoint{2.830260in}{0.950156in}}{\pgfqpoint{2.824734in}{0.950156in}}%
\pgfpathcurveto{\pgfqpoint{2.819209in}{0.950156in}}{\pgfqpoint{2.813910in}{0.947961in}}{\pgfqpoint{2.810003in}{0.944054in}}%
\pgfpathcurveto{\pgfqpoint{2.806096in}{0.940147in}}{\pgfqpoint{2.803901in}{0.934848in}}{\pgfqpoint{2.803901in}{0.929323in}}%
\pgfpathcurveto{\pgfqpoint{2.803901in}{0.923798in}}{\pgfqpoint{2.806096in}{0.918498in}}{\pgfqpoint{2.810003in}{0.914591in}}%
\pgfpathcurveto{\pgfqpoint{2.813910in}{0.910684in}}{\pgfqpoint{2.819209in}{0.908489in}}{\pgfqpoint{2.824734in}{0.908489in}}%
\pgfpathclose%
\pgfusepath{stroke,fill}%
\end{pgfscope}%
\begin{pgfscope}%
\pgfpathrectangle{\pgfqpoint{0.562500in}{0.275000in}}{\pgfqpoint{3.487500in}{1.925000in}}%
\pgfusepath{clip}%
\pgfsetbuttcap%
\pgfsetroundjoin%
\definecolor{currentfill}{rgb}{0.000000,0.000000,0.000000}%
\pgfsetfillcolor{currentfill}%
\pgfsetlinewidth{1.003750pt}%
\definecolor{currentstroke}{rgb}{0.000000,0.000000,0.000000}%
\pgfsetstrokecolor{currentstroke}%
\pgfsetdash{}{0pt}%
\pgfpathmoveto{\pgfqpoint{2.824734in}{0.908489in}}%
\pgfpathcurveto{\pgfqpoint{2.830260in}{0.908489in}}{\pgfqpoint{2.835559in}{0.910684in}}{\pgfqpoint{2.839466in}{0.914591in}}%
\pgfpathcurveto{\pgfqpoint{2.843373in}{0.918498in}}{\pgfqpoint{2.845568in}{0.923798in}}{\pgfqpoint{2.845568in}{0.929323in}}%
\pgfpathcurveto{\pgfqpoint{2.845568in}{0.934848in}}{\pgfqpoint{2.843373in}{0.940147in}}{\pgfqpoint{2.839466in}{0.944054in}}%
\pgfpathcurveto{\pgfqpoint{2.835559in}{0.947961in}}{\pgfqpoint{2.830260in}{0.950156in}}{\pgfqpoint{2.824734in}{0.950156in}}%
\pgfpathcurveto{\pgfqpoint{2.819209in}{0.950156in}}{\pgfqpoint{2.813910in}{0.947961in}}{\pgfqpoint{2.810003in}{0.944054in}}%
\pgfpathcurveto{\pgfqpoint{2.806096in}{0.940147in}}{\pgfqpoint{2.803901in}{0.934848in}}{\pgfqpoint{2.803901in}{0.929323in}}%
\pgfpathcurveto{\pgfqpoint{2.803901in}{0.923798in}}{\pgfqpoint{2.806096in}{0.918498in}}{\pgfqpoint{2.810003in}{0.914591in}}%
\pgfpathcurveto{\pgfqpoint{2.813910in}{0.910684in}}{\pgfqpoint{2.819209in}{0.908489in}}{\pgfqpoint{2.824734in}{0.908489in}}%
\pgfpathclose%
\pgfusepath{stroke,fill}%
\end{pgfscope}%
\begin{pgfscope}%
\pgfpathrectangle{\pgfqpoint{0.562500in}{0.275000in}}{\pgfqpoint{3.487500in}{1.925000in}}%
\pgfusepath{clip}%
\pgfsetbuttcap%
\pgfsetroundjoin%
\definecolor{currentfill}{rgb}{0.000000,0.000000,0.000000}%
\pgfsetfillcolor{currentfill}%
\pgfsetlinewidth{1.003750pt}%
\definecolor{currentstroke}{rgb}{0.000000,0.000000,0.000000}%
\pgfsetstrokecolor{currentstroke}%
\pgfsetdash{}{0pt}%
\pgfpathmoveto{\pgfqpoint{2.824734in}{0.938841in}}%
\pgfpathcurveto{\pgfqpoint{2.830260in}{0.938841in}}{\pgfqpoint{2.835559in}{0.941036in}}{\pgfqpoint{2.839466in}{0.944943in}}%
\pgfpathcurveto{\pgfqpoint{2.843373in}{0.948849in}}{\pgfqpoint{2.845568in}{0.954149in}}{\pgfqpoint{2.845568in}{0.959674in}}%
\pgfpathcurveto{\pgfqpoint{2.845568in}{0.965199in}}{\pgfqpoint{2.843373in}{0.970499in}}{\pgfqpoint{2.839466in}{0.974405in}}%
\pgfpathcurveto{\pgfqpoint{2.835559in}{0.978312in}}{\pgfqpoint{2.830260in}{0.980507in}}{\pgfqpoint{2.824734in}{0.980507in}}%
\pgfpathcurveto{\pgfqpoint{2.819209in}{0.980507in}}{\pgfqpoint{2.813910in}{0.978312in}}{\pgfqpoint{2.810003in}{0.974405in}}%
\pgfpathcurveto{\pgfqpoint{2.806096in}{0.970499in}}{\pgfqpoint{2.803901in}{0.965199in}}{\pgfqpoint{2.803901in}{0.959674in}}%
\pgfpathcurveto{\pgfqpoint{2.803901in}{0.954149in}}{\pgfqpoint{2.806096in}{0.948849in}}{\pgfqpoint{2.810003in}{0.944943in}}%
\pgfpathcurveto{\pgfqpoint{2.813910in}{0.941036in}}{\pgfqpoint{2.819209in}{0.938841in}}{\pgfqpoint{2.824734in}{0.938841in}}%
\pgfpathclose%
\pgfusepath{stroke,fill}%
\end{pgfscope}%
\begin{pgfscope}%
\pgfpathrectangle{\pgfqpoint{0.562500in}{0.275000in}}{\pgfqpoint{3.487500in}{1.925000in}}%
\pgfusepath{clip}%
\pgfsetbuttcap%
\pgfsetroundjoin%
\definecolor{currentfill}{rgb}{0.000000,0.000000,0.000000}%
\pgfsetfillcolor{currentfill}%
\pgfsetlinewidth{1.003750pt}%
\definecolor{currentstroke}{rgb}{0.000000,0.000000,0.000000}%
\pgfsetstrokecolor{currentstroke}%
\pgfsetdash{}{0pt}%
\pgfpathmoveto{\pgfqpoint{2.824734in}{0.847786in}}%
\pgfpathcurveto{\pgfqpoint{2.830260in}{0.847786in}}{\pgfqpoint{2.835559in}{0.849982in}}{\pgfqpoint{2.839466in}{0.853888in}}%
\pgfpathcurveto{\pgfqpoint{2.843373in}{0.857795in}}{\pgfqpoint{2.845568in}{0.863095in}}{\pgfqpoint{2.845568in}{0.868620in}}%
\pgfpathcurveto{\pgfqpoint{2.845568in}{0.874145in}}{\pgfqpoint{2.843373in}{0.879444in}}{\pgfqpoint{2.839466in}{0.883351in}}%
\pgfpathcurveto{\pgfqpoint{2.835559in}{0.887258in}}{\pgfqpoint{2.830260in}{0.889453in}}{\pgfqpoint{2.824734in}{0.889453in}}%
\pgfpathcurveto{\pgfqpoint{2.819209in}{0.889453in}}{\pgfqpoint{2.813910in}{0.887258in}}{\pgfqpoint{2.810003in}{0.883351in}}%
\pgfpathcurveto{\pgfqpoint{2.806096in}{0.879444in}}{\pgfqpoint{2.803901in}{0.874145in}}{\pgfqpoint{2.803901in}{0.868620in}}%
\pgfpathcurveto{\pgfqpoint{2.803901in}{0.863095in}}{\pgfqpoint{2.806096in}{0.857795in}}{\pgfqpoint{2.810003in}{0.853888in}}%
\pgfpathcurveto{\pgfqpoint{2.813910in}{0.849982in}}{\pgfqpoint{2.819209in}{0.847786in}}{\pgfqpoint{2.824734in}{0.847786in}}%
\pgfpathclose%
\pgfusepath{stroke,fill}%
\end{pgfscope}%
\begin{pgfscope}%
\pgfpathrectangle{\pgfqpoint{0.562500in}{0.275000in}}{\pgfqpoint{3.487500in}{1.925000in}}%
\pgfusepath{clip}%
\pgfsetbuttcap%
\pgfsetroundjoin%
\definecolor{currentfill}{rgb}{0.000000,0.000000,0.000000}%
\pgfsetfillcolor{currentfill}%
\pgfsetlinewidth{1.003750pt}%
\definecolor{currentstroke}{rgb}{0.000000,0.000000,0.000000}%
\pgfsetstrokecolor{currentstroke}%
\pgfsetdash{}{0pt}%
\pgfpathmoveto{\pgfqpoint{2.824734in}{0.918606in}}%
\pgfpathcurveto{\pgfqpoint{2.830260in}{0.918606in}}{\pgfqpoint{2.835559in}{0.920802in}}{\pgfqpoint{2.839466in}{0.924708in}}%
\pgfpathcurveto{\pgfqpoint{2.843373in}{0.928615in}}{\pgfqpoint{2.845568in}{0.933915in}}{\pgfqpoint{2.845568in}{0.939440in}}%
\pgfpathcurveto{\pgfqpoint{2.845568in}{0.944965in}}{\pgfqpoint{2.843373in}{0.950264in}}{\pgfqpoint{2.839466in}{0.954171in}}%
\pgfpathcurveto{\pgfqpoint{2.835559in}{0.958078in}}{\pgfqpoint{2.830260in}{0.960273in}}{\pgfqpoint{2.824734in}{0.960273in}}%
\pgfpathcurveto{\pgfqpoint{2.819209in}{0.960273in}}{\pgfqpoint{2.813910in}{0.958078in}}{\pgfqpoint{2.810003in}{0.954171in}}%
\pgfpathcurveto{\pgfqpoint{2.806096in}{0.950264in}}{\pgfqpoint{2.803901in}{0.944965in}}{\pgfqpoint{2.803901in}{0.939440in}}%
\pgfpathcurveto{\pgfqpoint{2.803901in}{0.933915in}}{\pgfqpoint{2.806096in}{0.928615in}}{\pgfqpoint{2.810003in}{0.924708in}}%
\pgfpathcurveto{\pgfqpoint{2.813910in}{0.920802in}}{\pgfqpoint{2.819209in}{0.918606in}}{\pgfqpoint{2.824734in}{0.918606in}}%
\pgfpathclose%
\pgfusepath{stroke,fill}%
\end{pgfscope}%
\begin{pgfscope}%
\pgfpathrectangle{\pgfqpoint{0.562500in}{0.275000in}}{\pgfqpoint{3.487500in}{1.925000in}}%
\pgfusepath{clip}%
\pgfsetbuttcap%
\pgfsetroundjoin%
\definecolor{currentfill}{rgb}{0.000000,0.000000,0.000000}%
\pgfsetfillcolor{currentfill}%
\pgfsetlinewidth{1.003750pt}%
\definecolor{currentstroke}{rgb}{0.000000,0.000000,0.000000}%
\pgfsetstrokecolor{currentstroke}%
\pgfsetdash{}{0pt}%
\pgfpathmoveto{\pgfqpoint{2.824734in}{0.822494in}}%
\pgfpathcurveto{\pgfqpoint{2.830260in}{0.822494in}}{\pgfqpoint{2.835559in}{0.824689in}}{\pgfqpoint{2.839466in}{0.828596in}}%
\pgfpathcurveto{\pgfqpoint{2.843373in}{0.832502in}}{\pgfqpoint{2.845568in}{0.837802in}}{\pgfqpoint{2.845568in}{0.843327in}}%
\pgfpathcurveto{\pgfqpoint{2.845568in}{0.848852in}}{\pgfqpoint{2.843373in}{0.854152in}}{\pgfqpoint{2.839466in}{0.858058in}}%
\pgfpathcurveto{\pgfqpoint{2.835559in}{0.861965in}}{\pgfqpoint{2.830260in}{0.864160in}}{\pgfqpoint{2.824734in}{0.864160in}}%
\pgfpathcurveto{\pgfqpoint{2.819209in}{0.864160in}}{\pgfqpoint{2.813910in}{0.861965in}}{\pgfqpoint{2.810003in}{0.858058in}}%
\pgfpathcurveto{\pgfqpoint{2.806096in}{0.854152in}}{\pgfqpoint{2.803901in}{0.848852in}}{\pgfqpoint{2.803901in}{0.843327in}}%
\pgfpathcurveto{\pgfqpoint{2.803901in}{0.837802in}}{\pgfqpoint{2.806096in}{0.832502in}}{\pgfqpoint{2.810003in}{0.828596in}}%
\pgfpathcurveto{\pgfqpoint{2.813910in}{0.824689in}}{\pgfqpoint{2.819209in}{0.822494in}}{\pgfqpoint{2.824734in}{0.822494in}}%
\pgfpathclose%
\pgfusepath{stroke,fill}%
\end{pgfscope}%
\begin{pgfscope}%
\pgfpathrectangle{\pgfqpoint{0.562500in}{0.275000in}}{\pgfqpoint{3.487500in}{1.925000in}}%
\pgfusepath{clip}%
\pgfsetbuttcap%
\pgfsetroundjoin%
\definecolor{currentfill}{rgb}{0.000000,0.000000,0.000000}%
\pgfsetfillcolor{currentfill}%
\pgfsetlinewidth{1.003750pt}%
\definecolor{currentstroke}{rgb}{0.000000,0.000000,0.000000}%
\pgfsetstrokecolor{currentstroke}%
\pgfsetdash{}{0pt}%
\pgfpathmoveto{\pgfqpoint{2.824734in}{0.842728in}}%
\pgfpathcurveto{\pgfqpoint{2.830260in}{0.842728in}}{\pgfqpoint{2.835559in}{0.844923in}}{\pgfqpoint{2.839466in}{0.848830in}}%
\pgfpathcurveto{\pgfqpoint{2.843373in}{0.852737in}}{\pgfqpoint{2.845568in}{0.858036in}}{\pgfqpoint{2.845568in}{0.863561in}}%
\pgfpathcurveto{\pgfqpoint{2.845568in}{0.869086in}}{\pgfqpoint{2.843373in}{0.874386in}}{\pgfqpoint{2.839466in}{0.878293in}}%
\pgfpathcurveto{\pgfqpoint{2.835559in}{0.882199in}}{\pgfqpoint{2.830260in}{0.884395in}}{\pgfqpoint{2.824734in}{0.884395in}}%
\pgfpathcurveto{\pgfqpoint{2.819209in}{0.884395in}}{\pgfqpoint{2.813910in}{0.882199in}}{\pgfqpoint{2.810003in}{0.878293in}}%
\pgfpathcurveto{\pgfqpoint{2.806096in}{0.874386in}}{\pgfqpoint{2.803901in}{0.869086in}}{\pgfqpoint{2.803901in}{0.863561in}}%
\pgfpathcurveto{\pgfqpoint{2.803901in}{0.858036in}}{\pgfqpoint{2.806096in}{0.852737in}}{\pgfqpoint{2.810003in}{0.848830in}}%
\pgfpathcurveto{\pgfqpoint{2.813910in}{0.844923in}}{\pgfqpoint{2.819209in}{0.842728in}}{\pgfqpoint{2.824734in}{0.842728in}}%
\pgfpathclose%
\pgfusepath{stroke,fill}%
\end{pgfscope}%
\begin{pgfscope}%
\pgfpathrectangle{\pgfqpoint{0.562500in}{0.275000in}}{\pgfqpoint{3.487500in}{1.925000in}}%
\pgfusepath{clip}%
\pgfsetbuttcap%
\pgfsetroundjoin%
\definecolor{currentfill}{rgb}{0.000000,0.000000,0.000000}%
\pgfsetfillcolor{currentfill}%
\pgfsetlinewidth{1.003750pt}%
\definecolor{currentstroke}{rgb}{0.000000,0.000000,0.000000}%
\pgfsetstrokecolor{currentstroke}%
\pgfsetdash{}{0pt}%
\pgfpathmoveto{\pgfqpoint{2.824734in}{0.893314in}}%
\pgfpathcurveto{\pgfqpoint{2.830260in}{0.893314in}}{\pgfqpoint{2.835559in}{0.895509in}}{\pgfqpoint{2.839466in}{0.899415in}}%
\pgfpathcurveto{\pgfqpoint{2.843373in}{0.903322in}}{\pgfqpoint{2.845568in}{0.908622in}}{\pgfqpoint{2.845568in}{0.914147in}}%
\pgfpathcurveto{\pgfqpoint{2.845568in}{0.919672in}}{\pgfqpoint{2.843373in}{0.924971in}}{\pgfqpoint{2.839466in}{0.928878in}}%
\pgfpathcurveto{\pgfqpoint{2.835559in}{0.932785in}}{\pgfqpoint{2.830260in}{0.934980in}}{\pgfqpoint{2.824734in}{0.934980in}}%
\pgfpathcurveto{\pgfqpoint{2.819209in}{0.934980in}}{\pgfqpoint{2.813910in}{0.932785in}}{\pgfqpoint{2.810003in}{0.928878in}}%
\pgfpathcurveto{\pgfqpoint{2.806096in}{0.924971in}}{\pgfqpoint{2.803901in}{0.919672in}}{\pgfqpoint{2.803901in}{0.914147in}}%
\pgfpathcurveto{\pgfqpoint{2.803901in}{0.908622in}}{\pgfqpoint{2.806096in}{0.903322in}}{\pgfqpoint{2.810003in}{0.899415in}}%
\pgfpathcurveto{\pgfqpoint{2.813910in}{0.895509in}}{\pgfqpoint{2.819209in}{0.893314in}}{\pgfqpoint{2.824734in}{0.893314in}}%
\pgfpathclose%
\pgfusepath{stroke,fill}%
\end{pgfscope}%
\begin{pgfscope}%
\pgfpathrectangle{\pgfqpoint{0.562500in}{0.275000in}}{\pgfqpoint{3.487500in}{1.925000in}}%
\pgfusepath{clip}%
\pgfsetbuttcap%
\pgfsetroundjoin%
\definecolor{currentfill}{rgb}{0.000000,0.000000,0.000000}%
\pgfsetfillcolor{currentfill}%
\pgfsetlinewidth{1.003750pt}%
\definecolor{currentstroke}{rgb}{0.000000,0.000000,0.000000}%
\pgfsetstrokecolor{currentstroke}%
\pgfsetdash{}{0pt}%
\pgfpathmoveto{\pgfqpoint{2.824734in}{0.883196in}}%
\pgfpathcurveto{\pgfqpoint{2.830260in}{0.883196in}}{\pgfqpoint{2.835559in}{0.885392in}}{\pgfqpoint{2.839466in}{0.889298in}}%
\pgfpathcurveto{\pgfqpoint{2.843373in}{0.893205in}}{\pgfqpoint{2.845568in}{0.898505in}}{\pgfqpoint{2.845568in}{0.904030in}}%
\pgfpathcurveto{\pgfqpoint{2.845568in}{0.909555in}}{\pgfqpoint{2.843373in}{0.914854in}}{\pgfqpoint{2.839466in}{0.918761in}}%
\pgfpathcurveto{\pgfqpoint{2.835559in}{0.922668in}}{\pgfqpoint{2.830260in}{0.924863in}}{\pgfqpoint{2.824734in}{0.924863in}}%
\pgfpathcurveto{\pgfqpoint{2.819209in}{0.924863in}}{\pgfqpoint{2.813910in}{0.922668in}}{\pgfqpoint{2.810003in}{0.918761in}}%
\pgfpathcurveto{\pgfqpoint{2.806096in}{0.914854in}}{\pgfqpoint{2.803901in}{0.909555in}}{\pgfqpoint{2.803901in}{0.904030in}}%
\pgfpathcurveto{\pgfqpoint{2.803901in}{0.898505in}}{\pgfqpoint{2.806096in}{0.893205in}}{\pgfqpoint{2.810003in}{0.889298in}}%
\pgfpathcurveto{\pgfqpoint{2.813910in}{0.885392in}}{\pgfqpoint{2.819209in}{0.883196in}}{\pgfqpoint{2.824734in}{0.883196in}}%
\pgfpathclose%
\pgfusepath{stroke,fill}%
\end{pgfscope}%
\begin{pgfscope}%
\pgfpathrectangle{\pgfqpoint{0.562500in}{0.275000in}}{\pgfqpoint{3.487500in}{1.925000in}}%
\pgfusepath{clip}%
\pgfsetbuttcap%
\pgfsetroundjoin%
\definecolor{currentfill}{rgb}{0.000000,0.000000,0.000000}%
\pgfsetfillcolor{currentfill}%
\pgfsetlinewidth{1.003750pt}%
\definecolor{currentstroke}{rgb}{0.000000,0.000000,0.000000}%
\pgfsetstrokecolor{currentstroke}%
\pgfsetdash{}{0pt}%
\pgfpathmoveto{\pgfqpoint{2.824734in}{0.908489in}}%
\pgfpathcurveto{\pgfqpoint{2.830260in}{0.908489in}}{\pgfqpoint{2.835559in}{0.910684in}}{\pgfqpoint{2.839466in}{0.914591in}}%
\pgfpathcurveto{\pgfqpoint{2.843373in}{0.918498in}}{\pgfqpoint{2.845568in}{0.923798in}}{\pgfqpoint{2.845568in}{0.929323in}}%
\pgfpathcurveto{\pgfqpoint{2.845568in}{0.934848in}}{\pgfqpoint{2.843373in}{0.940147in}}{\pgfqpoint{2.839466in}{0.944054in}}%
\pgfpathcurveto{\pgfqpoint{2.835559in}{0.947961in}}{\pgfqpoint{2.830260in}{0.950156in}}{\pgfqpoint{2.824734in}{0.950156in}}%
\pgfpathcurveto{\pgfqpoint{2.819209in}{0.950156in}}{\pgfqpoint{2.813910in}{0.947961in}}{\pgfqpoint{2.810003in}{0.944054in}}%
\pgfpathcurveto{\pgfqpoint{2.806096in}{0.940147in}}{\pgfqpoint{2.803901in}{0.934848in}}{\pgfqpoint{2.803901in}{0.929323in}}%
\pgfpathcurveto{\pgfqpoint{2.803901in}{0.923798in}}{\pgfqpoint{2.806096in}{0.918498in}}{\pgfqpoint{2.810003in}{0.914591in}}%
\pgfpathcurveto{\pgfqpoint{2.813910in}{0.910684in}}{\pgfqpoint{2.819209in}{0.908489in}}{\pgfqpoint{2.824734in}{0.908489in}}%
\pgfpathclose%
\pgfusepath{stroke,fill}%
\end{pgfscope}%
\begin{pgfscope}%
\pgfpathrectangle{\pgfqpoint{0.562500in}{0.275000in}}{\pgfqpoint{3.487500in}{1.925000in}}%
\pgfusepath{clip}%
\pgfsetbuttcap%
\pgfsetroundjoin%
\definecolor{currentfill}{rgb}{0.000000,0.000000,0.000000}%
\pgfsetfillcolor{currentfill}%
\pgfsetlinewidth{1.003750pt}%
\definecolor{currentstroke}{rgb}{0.000000,0.000000,0.000000}%
\pgfsetstrokecolor{currentstroke}%
\pgfsetdash{}{0pt}%
\pgfpathmoveto{\pgfqpoint{2.824734in}{0.822494in}}%
\pgfpathcurveto{\pgfqpoint{2.830260in}{0.822494in}}{\pgfqpoint{2.835559in}{0.824689in}}{\pgfqpoint{2.839466in}{0.828596in}}%
\pgfpathcurveto{\pgfqpoint{2.843373in}{0.832502in}}{\pgfqpoint{2.845568in}{0.837802in}}{\pgfqpoint{2.845568in}{0.843327in}}%
\pgfpathcurveto{\pgfqpoint{2.845568in}{0.848852in}}{\pgfqpoint{2.843373in}{0.854152in}}{\pgfqpoint{2.839466in}{0.858058in}}%
\pgfpathcurveto{\pgfqpoint{2.835559in}{0.861965in}}{\pgfqpoint{2.830260in}{0.864160in}}{\pgfqpoint{2.824734in}{0.864160in}}%
\pgfpathcurveto{\pgfqpoint{2.819209in}{0.864160in}}{\pgfqpoint{2.813910in}{0.861965in}}{\pgfqpoint{2.810003in}{0.858058in}}%
\pgfpathcurveto{\pgfqpoint{2.806096in}{0.854152in}}{\pgfqpoint{2.803901in}{0.848852in}}{\pgfqpoint{2.803901in}{0.843327in}}%
\pgfpathcurveto{\pgfqpoint{2.803901in}{0.837802in}}{\pgfqpoint{2.806096in}{0.832502in}}{\pgfqpoint{2.810003in}{0.828596in}}%
\pgfpathcurveto{\pgfqpoint{2.813910in}{0.824689in}}{\pgfqpoint{2.819209in}{0.822494in}}{\pgfqpoint{2.824734in}{0.822494in}}%
\pgfpathclose%
\pgfusepath{stroke,fill}%
\end{pgfscope}%
\begin{pgfscope}%
\pgfpathrectangle{\pgfqpoint{0.562500in}{0.275000in}}{\pgfqpoint{3.487500in}{1.925000in}}%
\pgfusepath{clip}%
\pgfsetbuttcap%
\pgfsetroundjoin%
\definecolor{currentfill}{rgb}{0.000000,0.000000,0.000000}%
\pgfsetfillcolor{currentfill}%
\pgfsetlinewidth{1.003750pt}%
\definecolor{currentstroke}{rgb}{0.000000,0.000000,0.000000}%
\pgfsetstrokecolor{currentstroke}%
\pgfsetdash{}{0pt}%
\pgfpathmoveto{\pgfqpoint{2.824734in}{0.893314in}}%
\pgfpathcurveto{\pgfqpoint{2.830260in}{0.893314in}}{\pgfqpoint{2.835559in}{0.895509in}}{\pgfqpoint{2.839466in}{0.899415in}}%
\pgfpathcurveto{\pgfqpoint{2.843373in}{0.903322in}}{\pgfqpoint{2.845568in}{0.908622in}}{\pgfqpoint{2.845568in}{0.914147in}}%
\pgfpathcurveto{\pgfqpoint{2.845568in}{0.919672in}}{\pgfqpoint{2.843373in}{0.924971in}}{\pgfqpoint{2.839466in}{0.928878in}}%
\pgfpathcurveto{\pgfqpoint{2.835559in}{0.932785in}}{\pgfqpoint{2.830260in}{0.934980in}}{\pgfqpoint{2.824734in}{0.934980in}}%
\pgfpathcurveto{\pgfqpoint{2.819209in}{0.934980in}}{\pgfqpoint{2.813910in}{0.932785in}}{\pgfqpoint{2.810003in}{0.928878in}}%
\pgfpathcurveto{\pgfqpoint{2.806096in}{0.924971in}}{\pgfqpoint{2.803901in}{0.919672in}}{\pgfqpoint{2.803901in}{0.914147in}}%
\pgfpathcurveto{\pgfqpoint{2.803901in}{0.908622in}}{\pgfqpoint{2.806096in}{0.903322in}}{\pgfqpoint{2.810003in}{0.899415in}}%
\pgfpathcurveto{\pgfqpoint{2.813910in}{0.895509in}}{\pgfqpoint{2.819209in}{0.893314in}}{\pgfqpoint{2.824734in}{0.893314in}}%
\pgfpathclose%
\pgfusepath{stroke,fill}%
\end{pgfscope}%
\begin{pgfscope}%
\pgfpathrectangle{\pgfqpoint{0.562500in}{0.275000in}}{\pgfqpoint{3.487500in}{1.925000in}}%
\pgfusepath{clip}%
\pgfsetbuttcap%
\pgfsetroundjoin%
\definecolor{currentfill}{rgb}{0.000000,0.000000,0.000000}%
\pgfsetfillcolor{currentfill}%
\pgfsetlinewidth{1.003750pt}%
\definecolor{currentstroke}{rgb}{0.000000,0.000000,0.000000}%
\pgfsetstrokecolor{currentstroke}%
\pgfsetdash{}{0pt}%
\pgfpathmoveto{\pgfqpoint{2.824734in}{0.842728in}}%
\pgfpathcurveto{\pgfqpoint{2.830260in}{0.842728in}}{\pgfqpoint{2.835559in}{0.844923in}}{\pgfqpoint{2.839466in}{0.848830in}}%
\pgfpathcurveto{\pgfqpoint{2.843373in}{0.852737in}}{\pgfqpoint{2.845568in}{0.858036in}}{\pgfqpoint{2.845568in}{0.863561in}}%
\pgfpathcurveto{\pgfqpoint{2.845568in}{0.869086in}}{\pgfqpoint{2.843373in}{0.874386in}}{\pgfqpoint{2.839466in}{0.878293in}}%
\pgfpathcurveto{\pgfqpoint{2.835559in}{0.882199in}}{\pgfqpoint{2.830260in}{0.884395in}}{\pgfqpoint{2.824734in}{0.884395in}}%
\pgfpathcurveto{\pgfqpoint{2.819209in}{0.884395in}}{\pgfqpoint{2.813910in}{0.882199in}}{\pgfqpoint{2.810003in}{0.878293in}}%
\pgfpathcurveto{\pgfqpoint{2.806096in}{0.874386in}}{\pgfqpoint{2.803901in}{0.869086in}}{\pgfqpoint{2.803901in}{0.863561in}}%
\pgfpathcurveto{\pgfqpoint{2.803901in}{0.858036in}}{\pgfqpoint{2.806096in}{0.852737in}}{\pgfqpoint{2.810003in}{0.848830in}}%
\pgfpathcurveto{\pgfqpoint{2.813910in}{0.844923in}}{\pgfqpoint{2.819209in}{0.842728in}}{\pgfqpoint{2.824734in}{0.842728in}}%
\pgfpathclose%
\pgfusepath{stroke,fill}%
\end{pgfscope}%
\begin{pgfscope}%
\pgfpathrectangle{\pgfqpoint{0.562500in}{0.275000in}}{\pgfqpoint{3.487500in}{1.925000in}}%
\pgfusepath{clip}%
\pgfsetbuttcap%
\pgfsetroundjoin%
\definecolor{currentfill}{rgb}{0.000000,0.000000,0.000000}%
\pgfsetfillcolor{currentfill}%
\pgfsetlinewidth{1.003750pt}%
\definecolor{currentstroke}{rgb}{0.000000,0.000000,0.000000}%
\pgfsetstrokecolor{currentstroke}%
\pgfsetdash{}{0pt}%
\pgfpathmoveto{\pgfqpoint{2.824734in}{0.868021in}}%
\pgfpathcurveto{\pgfqpoint{2.830260in}{0.868021in}}{\pgfqpoint{2.835559in}{0.870216in}}{\pgfqpoint{2.839466in}{0.874123in}}%
\pgfpathcurveto{\pgfqpoint{2.843373in}{0.878029in}}{\pgfqpoint{2.845568in}{0.883329in}}{\pgfqpoint{2.845568in}{0.888854in}}%
\pgfpathcurveto{\pgfqpoint{2.845568in}{0.894379in}}{\pgfqpoint{2.843373in}{0.899679in}}{\pgfqpoint{2.839466in}{0.903585in}}%
\pgfpathcurveto{\pgfqpoint{2.835559in}{0.907492in}}{\pgfqpoint{2.830260in}{0.909687in}}{\pgfqpoint{2.824734in}{0.909687in}}%
\pgfpathcurveto{\pgfqpoint{2.819209in}{0.909687in}}{\pgfqpoint{2.813910in}{0.907492in}}{\pgfqpoint{2.810003in}{0.903585in}}%
\pgfpathcurveto{\pgfqpoint{2.806096in}{0.899679in}}{\pgfqpoint{2.803901in}{0.894379in}}{\pgfqpoint{2.803901in}{0.888854in}}%
\pgfpathcurveto{\pgfqpoint{2.803901in}{0.883329in}}{\pgfqpoint{2.806096in}{0.878029in}}{\pgfqpoint{2.810003in}{0.874123in}}%
\pgfpathcurveto{\pgfqpoint{2.813910in}{0.870216in}}{\pgfqpoint{2.819209in}{0.868021in}}{\pgfqpoint{2.824734in}{0.868021in}}%
\pgfpathclose%
\pgfusepath{stroke,fill}%
\end{pgfscope}%
\begin{pgfscope}%
\pgfpathrectangle{\pgfqpoint{0.562500in}{0.275000in}}{\pgfqpoint{3.487500in}{1.925000in}}%
\pgfusepath{clip}%
\pgfsetbuttcap%
\pgfsetroundjoin%
\definecolor{currentfill}{rgb}{0.000000,0.000000,0.000000}%
\pgfsetfillcolor{currentfill}%
\pgfsetlinewidth{1.003750pt}%
\definecolor{currentstroke}{rgb}{0.000000,0.000000,0.000000}%
\pgfsetstrokecolor{currentstroke}%
\pgfsetdash{}{0pt}%
\pgfpathmoveto{\pgfqpoint{2.824734in}{0.908489in}}%
\pgfpathcurveto{\pgfqpoint{2.830260in}{0.908489in}}{\pgfqpoint{2.835559in}{0.910684in}}{\pgfqpoint{2.839466in}{0.914591in}}%
\pgfpathcurveto{\pgfqpoint{2.843373in}{0.918498in}}{\pgfqpoint{2.845568in}{0.923798in}}{\pgfqpoint{2.845568in}{0.929323in}}%
\pgfpathcurveto{\pgfqpoint{2.845568in}{0.934848in}}{\pgfqpoint{2.843373in}{0.940147in}}{\pgfqpoint{2.839466in}{0.944054in}}%
\pgfpathcurveto{\pgfqpoint{2.835559in}{0.947961in}}{\pgfqpoint{2.830260in}{0.950156in}}{\pgfqpoint{2.824734in}{0.950156in}}%
\pgfpathcurveto{\pgfqpoint{2.819209in}{0.950156in}}{\pgfqpoint{2.813910in}{0.947961in}}{\pgfqpoint{2.810003in}{0.944054in}}%
\pgfpathcurveto{\pgfqpoint{2.806096in}{0.940147in}}{\pgfqpoint{2.803901in}{0.934848in}}{\pgfqpoint{2.803901in}{0.929323in}}%
\pgfpathcurveto{\pgfqpoint{2.803901in}{0.923798in}}{\pgfqpoint{2.806096in}{0.918498in}}{\pgfqpoint{2.810003in}{0.914591in}}%
\pgfpathcurveto{\pgfqpoint{2.813910in}{0.910684in}}{\pgfqpoint{2.819209in}{0.908489in}}{\pgfqpoint{2.824734in}{0.908489in}}%
\pgfpathclose%
\pgfusepath{stroke,fill}%
\end{pgfscope}%
\begin{pgfscope}%
\pgfpathrectangle{\pgfqpoint{0.562500in}{0.275000in}}{\pgfqpoint{3.487500in}{1.925000in}}%
\pgfusepath{clip}%
\pgfsetbuttcap%
\pgfsetroundjoin%
\definecolor{currentfill}{rgb}{0.000000,0.000000,0.000000}%
\pgfsetfillcolor{currentfill}%
\pgfsetlinewidth{1.003750pt}%
\definecolor{currentstroke}{rgb}{0.000000,0.000000,0.000000}%
\pgfsetstrokecolor{currentstroke}%
\pgfsetdash{}{0pt}%
\pgfpathmoveto{\pgfqpoint{2.824734in}{0.888255in}}%
\pgfpathcurveto{\pgfqpoint{2.830260in}{0.888255in}}{\pgfqpoint{2.835559in}{0.890450in}}{\pgfqpoint{2.839466in}{0.894357in}}%
\pgfpathcurveto{\pgfqpoint{2.843373in}{0.898264in}}{\pgfqpoint{2.845568in}{0.903563in}}{\pgfqpoint{2.845568in}{0.909088in}}%
\pgfpathcurveto{\pgfqpoint{2.845568in}{0.914613in}}{\pgfqpoint{2.843373in}{0.919913in}}{\pgfqpoint{2.839466in}{0.923820in}}%
\pgfpathcurveto{\pgfqpoint{2.835559in}{0.927727in}}{\pgfqpoint{2.830260in}{0.929922in}}{\pgfqpoint{2.824734in}{0.929922in}}%
\pgfpathcurveto{\pgfqpoint{2.819209in}{0.929922in}}{\pgfqpoint{2.813910in}{0.927727in}}{\pgfqpoint{2.810003in}{0.923820in}}%
\pgfpathcurveto{\pgfqpoint{2.806096in}{0.919913in}}{\pgfqpoint{2.803901in}{0.914613in}}{\pgfqpoint{2.803901in}{0.909088in}}%
\pgfpathcurveto{\pgfqpoint{2.803901in}{0.903563in}}{\pgfqpoint{2.806096in}{0.898264in}}{\pgfqpoint{2.810003in}{0.894357in}}%
\pgfpathcurveto{\pgfqpoint{2.813910in}{0.890450in}}{\pgfqpoint{2.819209in}{0.888255in}}{\pgfqpoint{2.824734in}{0.888255in}}%
\pgfpathclose%
\pgfusepath{stroke,fill}%
\end{pgfscope}%
\begin{pgfscope}%
\pgfpathrectangle{\pgfqpoint{0.562500in}{0.275000in}}{\pgfqpoint{3.487500in}{1.925000in}}%
\pgfusepath{clip}%
\pgfsetbuttcap%
\pgfsetroundjoin%
\definecolor{currentfill}{rgb}{0.000000,0.000000,0.000000}%
\pgfsetfillcolor{currentfill}%
\pgfsetlinewidth{1.003750pt}%
\definecolor{currentstroke}{rgb}{0.000000,0.000000,0.000000}%
\pgfsetstrokecolor{currentstroke}%
\pgfsetdash{}{0pt}%
\pgfpathmoveto{\pgfqpoint{2.824734in}{0.832611in}}%
\pgfpathcurveto{\pgfqpoint{2.830260in}{0.832611in}}{\pgfqpoint{2.835559in}{0.834806in}}{\pgfqpoint{2.839466in}{0.838713in}}%
\pgfpathcurveto{\pgfqpoint{2.843373in}{0.842620in}}{\pgfqpoint{2.845568in}{0.847919in}}{\pgfqpoint{2.845568in}{0.853444in}}%
\pgfpathcurveto{\pgfqpoint{2.845568in}{0.858969in}}{\pgfqpoint{2.843373in}{0.864269in}}{\pgfqpoint{2.839466in}{0.868175in}}%
\pgfpathcurveto{\pgfqpoint{2.835559in}{0.872082in}}{\pgfqpoint{2.830260in}{0.874277in}}{\pgfqpoint{2.824734in}{0.874277in}}%
\pgfpathcurveto{\pgfqpoint{2.819209in}{0.874277in}}{\pgfqpoint{2.813910in}{0.872082in}}{\pgfqpoint{2.810003in}{0.868175in}}%
\pgfpathcurveto{\pgfqpoint{2.806096in}{0.864269in}}{\pgfqpoint{2.803901in}{0.858969in}}{\pgfqpoint{2.803901in}{0.853444in}}%
\pgfpathcurveto{\pgfqpoint{2.803901in}{0.847919in}}{\pgfqpoint{2.806096in}{0.842620in}}{\pgfqpoint{2.810003in}{0.838713in}}%
\pgfpathcurveto{\pgfqpoint{2.813910in}{0.834806in}}{\pgfqpoint{2.819209in}{0.832611in}}{\pgfqpoint{2.824734in}{0.832611in}}%
\pgfpathclose%
\pgfusepath{stroke,fill}%
\end{pgfscope}%
\begin{pgfscope}%
\pgfpathrectangle{\pgfqpoint{0.562500in}{0.275000in}}{\pgfqpoint{3.487500in}{1.925000in}}%
\pgfusepath{clip}%
\pgfsetbuttcap%
\pgfsetroundjoin%
\definecolor{currentfill}{rgb}{0.000000,0.000000,0.000000}%
\pgfsetfillcolor{currentfill}%
\pgfsetlinewidth{1.003750pt}%
\definecolor{currentstroke}{rgb}{0.000000,0.000000,0.000000}%
\pgfsetstrokecolor{currentstroke}%
\pgfsetdash{}{0pt}%
\pgfpathmoveto{\pgfqpoint{2.824734in}{1.034953in}}%
\pgfpathcurveto{\pgfqpoint{2.830260in}{1.034953in}}{\pgfqpoint{2.835559in}{1.037149in}}{\pgfqpoint{2.839466in}{1.041055in}}%
\pgfpathcurveto{\pgfqpoint{2.843373in}{1.044962in}}{\pgfqpoint{2.845568in}{1.050262in}}{\pgfqpoint{2.845568in}{1.055787in}}%
\pgfpathcurveto{\pgfqpoint{2.845568in}{1.061312in}}{\pgfqpoint{2.843373in}{1.066611in}}{\pgfqpoint{2.839466in}{1.070518in}}%
\pgfpathcurveto{\pgfqpoint{2.835559in}{1.074425in}}{\pgfqpoint{2.830260in}{1.076620in}}{\pgfqpoint{2.824734in}{1.076620in}}%
\pgfpathcurveto{\pgfqpoint{2.819209in}{1.076620in}}{\pgfqpoint{2.813910in}{1.074425in}}{\pgfqpoint{2.810003in}{1.070518in}}%
\pgfpathcurveto{\pgfqpoint{2.806096in}{1.066611in}}{\pgfqpoint{2.803901in}{1.061312in}}{\pgfqpoint{2.803901in}{1.055787in}}%
\pgfpathcurveto{\pgfqpoint{2.803901in}{1.050262in}}{\pgfqpoint{2.806096in}{1.044962in}}{\pgfqpoint{2.810003in}{1.041055in}}%
\pgfpathcurveto{\pgfqpoint{2.813910in}{1.037149in}}{\pgfqpoint{2.819209in}{1.034953in}}{\pgfqpoint{2.824734in}{1.034953in}}%
\pgfpathclose%
\pgfusepath{stroke,fill}%
\end{pgfscope}%
\begin{pgfscope}%
\pgfpathrectangle{\pgfqpoint{0.562500in}{0.275000in}}{\pgfqpoint{3.487500in}{1.925000in}}%
\pgfusepath{clip}%
\pgfsetbuttcap%
\pgfsetroundjoin%
\definecolor{currentfill}{rgb}{0.000000,0.000000,0.000000}%
\pgfsetfillcolor{currentfill}%
\pgfsetlinewidth{1.003750pt}%
\definecolor{currentstroke}{rgb}{0.000000,0.000000,0.000000}%
\pgfsetstrokecolor{currentstroke}%
\pgfsetdash{}{0pt}%
\pgfpathmoveto{\pgfqpoint{2.824734in}{0.918606in}}%
\pgfpathcurveto{\pgfqpoint{2.830260in}{0.918606in}}{\pgfqpoint{2.835559in}{0.920802in}}{\pgfqpoint{2.839466in}{0.924708in}}%
\pgfpathcurveto{\pgfqpoint{2.843373in}{0.928615in}}{\pgfqpoint{2.845568in}{0.933915in}}{\pgfqpoint{2.845568in}{0.939440in}}%
\pgfpathcurveto{\pgfqpoint{2.845568in}{0.944965in}}{\pgfqpoint{2.843373in}{0.950264in}}{\pgfqpoint{2.839466in}{0.954171in}}%
\pgfpathcurveto{\pgfqpoint{2.835559in}{0.958078in}}{\pgfqpoint{2.830260in}{0.960273in}}{\pgfqpoint{2.824734in}{0.960273in}}%
\pgfpathcurveto{\pgfqpoint{2.819209in}{0.960273in}}{\pgfqpoint{2.813910in}{0.958078in}}{\pgfqpoint{2.810003in}{0.954171in}}%
\pgfpathcurveto{\pgfqpoint{2.806096in}{0.950264in}}{\pgfqpoint{2.803901in}{0.944965in}}{\pgfqpoint{2.803901in}{0.939440in}}%
\pgfpathcurveto{\pgfqpoint{2.803901in}{0.933915in}}{\pgfqpoint{2.806096in}{0.928615in}}{\pgfqpoint{2.810003in}{0.924708in}}%
\pgfpathcurveto{\pgfqpoint{2.813910in}{0.920802in}}{\pgfqpoint{2.819209in}{0.918606in}}{\pgfqpoint{2.824734in}{0.918606in}}%
\pgfpathclose%
\pgfusepath{stroke,fill}%
\end{pgfscope}%
\begin{pgfscope}%
\pgfpathrectangle{\pgfqpoint{0.562500in}{0.275000in}}{\pgfqpoint{3.487500in}{1.925000in}}%
\pgfusepath{clip}%
\pgfsetbuttcap%
\pgfsetroundjoin%
\definecolor{currentfill}{rgb}{0.000000,0.000000,0.000000}%
\pgfsetfillcolor{currentfill}%
\pgfsetlinewidth{1.003750pt}%
\definecolor{currentstroke}{rgb}{0.000000,0.000000,0.000000}%
\pgfsetstrokecolor{currentstroke}%
\pgfsetdash{}{0pt}%
\pgfpathmoveto{\pgfqpoint{2.824734in}{0.792142in}}%
\pgfpathcurveto{\pgfqpoint{2.830260in}{0.792142in}}{\pgfqpoint{2.835559in}{0.794337in}}{\pgfqpoint{2.839466in}{0.798244in}}%
\pgfpathcurveto{\pgfqpoint{2.843373in}{0.802151in}}{\pgfqpoint{2.845568in}{0.807450in}}{\pgfqpoint{2.845568in}{0.812976in}}%
\pgfpathcurveto{\pgfqpoint{2.845568in}{0.818501in}}{\pgfqpoint{2.843373in}{0.823800in}}{\pgfqpoint{2.839466in}{0.827707in}}%
\pgfpathcurveto{\pgfqpoint{2.835559in}{0.831614in}}{\pgfqpoint{2.830260in}{0.833809in}}{\pgfqpoint{2.824734in}{0.833809in}}%
\pgfpathcurveto{\pgfqpoint{2.819209in}{0.833809in}}{\pgfqpoint{2.813910in}{0.831614in}}{\pgfqpoint{2.810003in}{0.827707in}}%
\pgfpathcurveto{\pgfqpoint{2.806096in}{0.823800in}}{\pgfqpoint{2.803901in}{0.818501in}}{\pgfqpoint{2.803901in}{0.812976in}}%
\pgfpathcurveto{\pgfqpoint{2.803901in}{0.807450in}}{\pgfqpoint{2.806096in}{0.802151in}}{\pgfqpoint{2.810003in}{0.798244in}}%
\pgfpathcurveto{\pgfqpoint{2.813910in}{0.794337in}}{\pgfqpoint{2.819209in}{0.792142in}}{\pgfqpoint{2.824734in}{0.792142in}}%
\pgfpathclose%
\pgfusepath{stroke,fill}%
\end{pgfscope}%
\begin{pgfscope}%
\pgfpathrectangle{\pgfqpoint{0.562500in}{0.275000in}}{\pgfqpoint{3.487500in}{1.925000in}}%
\pgfusepath{clip}%
\pgfsetbuttcap%
\pgfsetroundjoin%
\definecolor{currentfill}{rgb}{0.000000,0.000000,0.000000}%
\pgfsetfillcolor{currentfill}%
\pgfsetlinewidth{1.003750pt}%
\definecolor{currentstroke}{rgb}{0.000000,0.000000,0.000000}%
\pgfsetstrokecolor{currentstroke}%
\pgfsetdash{}{0pt}%
\pgfpathmoveto{\pgfqpoint{2.824734in}{0.979309in}}%
\pgfpathcurveto{\pgfqpoint{2.830260in}{0.979309in}}{\pgfqpoint{2.835559in}{0.981504in}}{\pgfqpoint{2.839466in}{0.985411in}}%
\pgfpathcurveto{\pgfqpoint{2.843373in}{0.989318in}}{\pgfqpoint{2.845568in}{0.994617in}}{\pgfqpoint{2.845568in}{1.000142in}}%
\pgfpathcurveto{\pgfqpoint{2.845568in}{1.005668in}}{\pgfqpoint{2.843373in}{1.010967in}}{\pgfqpoint{2.839466in}{1.014874in}}%
\pgfpathcurveto{\pgfqpoint{2.835559in}{1.018781in}}{\pgfqpoint{2.830260in}{1.020976in}}{\pgfqpoint{2.824734in}{1.020976in}}%
\pgfpathcurveto{\pgfqpoint{2.819209in}{1.020976in}}{\pgfqpoint{2.813910in}{1.018781in}}{\pgfqpoint{2.810003in}{1.014874in}}%
\pgfpathcurveto{\pgfqpoint{2.806096in}{1.010967in}}{\pgfqpoint{2.803901in}{1.005668in}}{\pgfqpoint{2.803901in}{1.000142in}}%
\pgfpathcurveto{\pgfqpoint{2.803901in}{0.994617in}}{\pgfqpoint{2.806096in}{0.989318in}}{\pgfqpoint{2.810003in}{0.985411in}}%
\pgfpathcurveto{\pgfqpoint{2.813910in}{0.981504in}}{\pgfqpoint{2.819209in}{0.979309in}}{\pgfqpoint{2.824734in}{0.979309in}}%
\pgfpathclose%
\pgfusepath{stroke,fill}%
\end{pgfscope}%
\begin{pgfscope}%
\pgfpathrectangle{\pgfqpoint{0.562500in}{0.275000in}}{\pgfqpoint{3.487500in}{1.925000in}}%
\pgfusepath{clip}%
\pgfsetbuttcap%
\pgfsetroundjoin%
\definecolor{currentfill}{rgb}{0.000000,0.000000,0.000000}%
\pgfsetfillcolor{currentfill}%
\pgfsetlinewidth{1.003750pt}%
\definecolor{currentstroke}{rgb}{0.000000,0.000000,0.000000}%
\pgfsetstrokecolor{currentstroke}%
\pgfsetdash{}{0pt}%
\pgfpathmoveto{\pgfqpoint{2.824734in}{0.868021in}}%
\pgfpathcurveto{\pgfqpoint{2.830260in}{0.868021in}}{\pgfqpoint{2.835559in}{0.870216in}}{\pgfqpoint{2.839466in}{0.874123in}}%
\pgfpathcurveto{\pgfqpoint{2.843373in}{0.878029in}}{\pgfqpoint{2.845568in}{0.883329in}}{\pgfqpoint{2.845568in}{0.888854in}}%
\pgfpathcurveto{\pgfqpoint{2.845568in}{0.894379in}}{\pgfqpoint{2.843373in}{0.899679in}}{\pgfqpoint{2.839466in}{0.903585in}}%
\pgfpathcurveto{\pgfqpoint{2.835559in}{0.907492in}}{\pgfqpoint{2.830260in}{0.909687in}}{\pgfqpoint{2.824734in}{0.909687in}}%
\pgfpathcurveto{\pgfqpoint{2.819209in}{0.909687in}}{\pgfqpoint{2.813910in}{0.907492in}}{\pgfqpoint{2.810003in}{0.903585in}}%
\pgfpathcurveto{\pgfqpoint{2.806096in}{0.899679in}}{\pgfqpoint{2.803901in}{0.894379in}}{\pgfqpoint{2.803901in}{0.888854in}}%
\pgfpathcurveto{\pgfqpoint{2.803901in}{0.883329in}}{\pgfqpoint{2.806096in}{0.878029in}}{\pgfqpoint{2.810003in}{0.874123in}}%
\pgfpathcurveto{\pgfqpoint{2.813910in}{0.870216in}}{\pgfqpoint{2.819209in}{0.868021in}}{\pgfqpoint{2.824734in}{0.868021in}}%
\pgfpathclose%
\pgfusepath{stroke,fill}%
\end{pgfscope}%
\begin{pgfscope}%
\pgfpathrectangle{\pgfqpoint{0.562500in}{0.275000in}}{\pgfqpoint{3.487500in}{1.925000in}}%
\pgfusepath{clip}%
\pgfsetbuttcap%
\pgfsetroundjoin%
\definecolor{currentfill}{rgb}{0.000000,0.000000,0.000000}%
\pgfsetfillcolor{currentfill}%
\pgfsetlinewidth{1.003750pt}%
\definecolor{currentstroke}{rgb}{0.000000,0.000000,0.000000}%
\pgfsetstrokecolor{currentstroke}%
\pgfsetdash{}{0pt}%
\pgfpathmoveto{\pgfqpoint{2.824734in}{0.994485in}}%
\pgfpathcurveto{\pgfqpoint{2.830260in}{0.994485in}}{\pgfqpoint{2.835559in}{0.996680in}}{\pgfqpoint{2.839466in}{1.000587in}}%
\pgfpathcurveto{\pgfqpoint{2.843373in}{1.004494in}}{\pgfqpoint{2.845568in}{1.009793in}}{\pgfqpoint{2.845568in}{1.015318in}}%
\pgfpathcurveto{\pgfqpoint{2.845568in}{1.020843in}}{\pgfqpoint{2.843373in}{1.026143in}}{\pgfqpoint{2.839466in}{1.030050in}}%
\pgfpathcurveto{\pgfqpoint{2.835559in}{1.033956in}}{\pgfqpoint{2.830260in}{1.036152in}}{\pgfqpoint{2.824734in}{1.036152in}}%
\pgfpathcurveto{\pgfqpoint{2.819209in}{1.036152in}}{\pgfqpoint{2.813910in}{1.033956in}}{\pgfqpoint{2.810003in}{1.030050in}}%
\pgfpathcurveto{\pgfqpoint{2.806096in}{1.026143in}}{\pgfqpoint{2.803901in}{1.020843in}}{\pgfqpoint{2.803901in}{1.015318in}}%
\pgfpathcurveto{\pgfqpoint{2.803901in}{1.009793in}}{\pgfqpoint{2.806096in}{1.004494in}}{\pgfqpoint{2.810003in}{1.000587in}}%
\pgfpathcurveto{\pgfqpoint{2.813910in}{0.996680in}}{\pgfqpoint{2.819209in}{0.994485in}}{\pgfqpoint{2.824734in}{0.994485in}}%
\pgfpathclose%
\pgfusepath{stroke,fill}%
\end{pgfscope}%
\begin{pgfscope}%
\pgfpathrectangle{\pgfqpoint{0.562500in}{0.275000in}}{\pgfqpoint{3.487500in}{1.925000in}}%
\pgfusepath{clip}%
\pgfsetbuttcap%
\pgfsetroundjoin%
\definecolor{currentfill}{rgb}{0.000000,0.000000,0.000000}%
\pgfsetfillcolor{currentfill}%
\pgfsetlinewidth{1.003750pt}%
\definecolor{currentstroke}{rgb}{0.000000,0.000000,0.000000}%
\pgfsetstrokecolor{currentstroke}%
\pgfsetdash{}{0pt}%
\pgfpathmoveto{\pgfqpoint{2.824734in}{0.989426in}}%
\pgfpathcurveto{\pgfqpoint{2.830260in}{0.989426in}}{\pgfqpoint{2.835559in}{0.991621in}}{\pgfqpoint{2.839466in}{0.995528in}}%
\pgfpathcurveto{\pgfqpoint{2.843373in}{0.999435in}}{\pgfqpoint{2.845568in}{1.004735in}}{\pgfqpoint{2.845568in}{1.010260in}}%
\pgfpathcurveto{\pgfqpoint{2.845568in}{1.015785in}}{\pgfqpoint{2.843373in}{1.021084in}}{\pgfqpoint{2.839466in}{1.024991in}}%
\pgfpathcurveto{\pgfqpoint{2.835559in}{1.028898in}}{\pgfqpoint{2.830260in}{1.031093in}}{\pgfqpoint{2.824734in}{1.031093in}}%
\pgfpathcurveto{\pgfqpoint{2.819209in}{1.031093in}}{\pgfqpoint{2.813910in}{1.028898in}}{\pgfqpoint{2.810003in}{1.024991in}}%
\pgfpathcurveto{\pgfqpoint{2.806096in}{1.021084in}}{\pgfqpoint{2.803901in}{1.015785in}}{\pgfqpoint{2.803901in}{1.010260in}}%
\pgfpathcurveto{\pgfqpoint{2.803901in}{1.004735in}}{\pgfqpoint{2.806096in}{0.999435in}}{\pgfqpoint{2.810003in}{0.995528in}}%
\pgfpathcurveto{\pgfqpoint{2.813910in}{0.991621in}}{\pgfqpoint{2.819209in}{0.989426in}}{\pgfqpoint{2.824734in}{0.989426in}}%
\pgfpathclose%
\pgfusepath{stroke,fill}%
\end{pgfscope}%
\begin{pgfscope}%
\pgfpathrectangle{\pgfqpoint{0.562500in}{0.275000in}}{\pgfqpoint{3.487500in}{1.925000in}}%
\pgfusepath{clip}%
\pgfsetbuttcap%
\pgfsetroundjoin%
\definecolor{currentfill}{rgb}{0.000000,0.000000,0.000000}%
\pgfsetfillcolor{currentfill}%
\pgfsetlinewidth{1.003750pt}%
\definecolor{currentstroke}{rgb}{0.000000,0.000000,0.000000}%
\pgfsetstrokecolor{currentstroke}%
\pgfsetdash{}{0pt}%
\pgfpathmoveto{\pgfqpoint{2.824734in}{0.827552in}}%
\pgfpathcurveto{\pgfqpoint{2.830260in}{0.827552in}}{\pgfqpoint{2.835559in}{0.829747in}}{\pgfqpoint{2.839466in}{0.833654in}}%
\pgfpathcurveto{\pgfqpoint{2.843373in}{0.837561in}}{\pgfqpoint{2.845568in}{0.842860in}}{\pgfqpoint{2.845568in}{0.848386in}}%
\pgfpathcurveto{\pgfqpoint{2.845568in}{0.853911in}}{\pgfqpoint{2.843373in}{0.859210in}}{\pgfqpoint{2.839466in}{0.863117in}}%
\pgfpathcurveto{\pgfqpoint{2.835559in}{0.867024in}}{\pgfqpoint{2.830260in}{0.869219in}}{\pgfqpoint{2.824734in}{0.869219in}}%
\pgfpathcurveto{\pgfqpoint{2.819209in}{0.869219in}}{\pgfqpoint{2.813910in}{0.867024in}}{\pgfqpoint{2.810003in}{0.863117in}}%
\pgfpathcurveto{\pgfqpoint{2.806096in}{0.859210in}}{\pgfqpoint{2.803901in}{0.853911in}}{\pgfqpoint{2.803901in}{0.848386in}}%
\pgfpathcurveto{\pgfqpoint{2.803901in}{0.842860in}}{\pgfqpoint{2.806096in}{0.837561in}}{\pgfqpoint{2.810003in}{0.833654in}}%
\pgfpathcurveto{\pgfqpoint{2.813910in}{0.829747in}}{\pgfqpoint{2.819209in}{0.827552in}}{\pgfqpoint{2.824734in}{0.827552in}}%
\pgfpathclose%
\pgfusepath{stroke,fill}%
\end{pgfscope}%
\begin{pgfscope}%
\pgfpathrectangle{\pgfqpoint{0.562500in}{0.275000in}}{\pgfqpoint{3.487500in}{1.925000in}}%
\pgfusepath{clip}%
\pgfsetbuttcap%
\pgfsetroundjoin%
\definecolor{currentfill}{rgb}{0.000000,0.000000,0.000000}%
\pgfsetfillcolor{currentfill}%
\pgfsetlinewidth{1.003750pt}%
\definecolor{currentstroke}{rgb}{0.000000,0.000000,0.000000}%
\pgfsetstrokecolor{currentstroke}%
\pgfsetdash{}{0pt}%
\pgfpathmoveto{\pgfqpoint{2.824734in}{0.812376in}}%
\pgfpathcurveto{\pgfqpoint{2.830260in}{0.812376in}}{\pgfqpoint{2.835559in}{0.814572in}}{\pgfqpoint{2.839466in}{0.818478in}}%
\pgfpathcurveto{\pgfqpoint{2.843373in}{0.822385in}}{\pgfqpoint{2.845568in}{0.827685in}}{\pgfqpoint{2.845568in}{0.833210in}}%
\pgfpathcurveto{\pgfqpoint{2.845568in}{0.838735in}}{\pgfqpoint{2.843373in}{0.844034in}}{\pgfqpoint{2.839466in}{0.847941in}}%
\pgfpathcurveto{\pgfqpoint{2.835559in}{0.851848in}}{\pgfqpoint{2.830260in}{0.854043in}}{\pgfqpoint{2.824734in}{0.854043in}}%
\pgfpathcurveto{\pgfqpoint{2.819209in}{0.854043in}}{\pgfqpoint{2.813910in}{0.851848in}}{\pgfqpoint{2.810003in}{0.847941in}}%
\pgfpathcurveto{\pgfqpoint{2.806096in}{0.844034in}}{\pgfqpoint{2.803901in}{0.838735in}}{\pgfqpoint{2.803901in}{0.833210in}}%
\pgfpathcurveto{\pgfqpoint{2.803901in}{0.827685in}}{\pgfqpoint{2.806096in}{0.822385in}}{\pgfqpoint{2.810003in}{0.818478in}}%
\pgfpathcurveto{\pgfqpoint{2.813910in}{0.814572in}}{\pgfqpoint{2.819209in}{0.812376in}}{\pgfqpoint{2.824734in}{0.812376in}}%
\pgfpathclose%
\pgfusepath{stroke,fill}%
\end{pgfscope}%
\begin{pgfscope}%
\pgfpathrectangle{\pgfqpoint{0.562500in}{0.275000in}}{\pgfqpoint{3.487500in}{1.925000in}}%
\pgfusepath{clip}%
\pgfsetbuttcap%
\pgfsetroundjoin%
\definecolor{currentfill}{rgb}{0.000000,0.000000,0.000000}%
\pgfsetfillcolor{currentfill}%
\pgfsetlinewidth{1.003750pt}%
\definecolor{currentstroke}{rgb}{0.000000,0.000000,0.000000}%
\pgfsetstrokecolor{currentstroke}%
\pgfsetdash{}{0pt}%
\pgfpathmoveto{\pgfqpoint{2.824734in}{0.974251in}}%
\pgfpathcurveto{\pgfqpoint{2.830260in}{0.974251in}}{\pgfqpoint{2.835559in}{0.976446in}}{\pgfqpoint{2.839466in}{0.980353in}}%
\pgfpathcurveto{\pgfqpoint{2.843373in}{0.984259in}}{\pgfqpoint{2.845568in}{0.989559in}}{\pgfqpoint{2.845568in}{0.995084in}}%
\pgfpathcurveto{\pgfqpoint{2.845568in}{1.000609in}}{\pgfqpoint{2.843373in}{1.005909in}}{\pgfqpoint{2.839466in}{1.009815in}}%
\pgfpathcurveto{\pgfqpoint{2.835559in}{1.013722in}}{\pgfqpoint{2.830260in}{1.015917in}}{\pgfqpoint{2.824734in}{1.015917in}}%
\pgfpathcurveto{\pgfqpoint{2.819209in}{1.015917in}}{\pgfqpoint{2.813910in}{1.013722in}}{\pgfqpoint{2.810003in}{1.009815in}}%
\pgfpathcurveto{\pgfqpoint{2.806096in}{1.005909in}}{\pgfqpoint{2.803901in}{1.000609in}}{\pgfqpoint{2.803901in}{0.995084in}}%
\pgfpathcurveto{\pgfqpoint{2.803901in}{0.989559in}}{\pgfqpoint{2.806096in}{0.984259in}}{\pgfqpoint{2.810003in}{0.980353in}}%
\pgfpathcurveto{\pgfqpoint{2.813910in}{0.976446in}}{\pgfqpoint{2.819209in}{0.974251in}}{\pgfqpoint{2.824734in}{0.974251in}}%
\pgfpathclose%
\pgfusepath{stroke,fill}%
\end{pgfscope}%
\begin{pgfscope}%
\pgfpathrectangle{\pgfqpoint{0.562500in}{0.275000in}}{\pgfqpoint{3.487500in}{1.925000in}}%
\pgfusepath{clip}%
\pgfsetbuttcap%
\pgfsetroundjoin%
\definecolor{currentfill}{rgb}{0.000000,0.000000,0.000000}%
\pgfsetfillcolor{currentfill}%
\pgfsetlinewidth{1.003750pt}%
\definecolor{currentstroke}{rgb}{0.000000,0.000000,0.000000}%
\pgfsetstrokecolor{currentstroke}%
\pgfsetdash{}{0pt}%
\pgfpathmoveto{\pgfqpoint{2.824734in}{1.070363in}}%
\pgfpathcurveto{\pgfqpoint{2.830260in}{1.070363in}}{\pgfqpoint{2.835559in}{1.072558in}}{\pgfqpoint{2.839466in}{1.076465in}}%
\pgfpathcurveto{\pgfqpoint{2.843373in}{1.080372in}}{\pgfqpoint{2.845568in}{1.085672in}}{\pgfqpoint{2.845568in}{1.091197in}}%
\pgfpathcurveto{\pgfqpoint{2.845568in}{1.096722in}}{\pgfqpoint{2.843373in}{1.102021in}}{\pgfqpoint{2.839466in}{1.105928in}}%
\pgfpathcurveto{\pgfqpoint{2.835559in}{1.109835in}}{\pgfqpoint{2.830260in}{1.112030in}}{\pgfqpoint{2.824734in}{1.112030in}}%
\pgfpathcurveto{\pgfqpoint{2.819209in}{1.112030in}}{\pgfqpoint{2.813910in}{1.109835in}}{\pgfqpoint{2.810003in}{1.105928in}}%
\pgfpathcurveto{\pgfqpoint{2.806096in}{1.102021in}}{\pgfqpoint{2.803901in}{1.096722in}}{\pgfqpoint{2.803901in}{1.091197in}}%
\pgfpathcurveto{\pgfqpoint{2.803901in}{1.085672in}}{\pgfqpoint{2.806096in}{1.080372in}}{\pgfqpoint{2.810003in}{1.076465in}}%
\pgfpathcurveto{\pgfqpoint{2.813910in}{1.072558in}}{\pgfqpoint{2.819209in}{1.070363in}}{\pgfqpoint{2.824734in}{1.070363in}}%
\pgfpathclose%
\pgfusepath{stroke,fill}%
\end{pgfscope}%
\begin{pgfscope}%
\pgfpathrectangle{\pgfqpoint{0.562500in}{0.275000in}}{\pgfqpoint{3.487500in}{1.925000in}}%
\pgfusepath{clip}%
\pgfsetbuttcap%
\pgfsetroundjoin%
\definecolor{currentfill}{rgb}{0.000000,0.000000,0.000000}%
\pgfsetfillcolor{currentfill}%
\pgfsetlinewidth{1.003750pt}%
\definecolor{currentstroke}{rgb}{0.000000,0.000000,0.000000}%
\pgfsetstrokecolor{currentstroke}%
\pgfsetdash{}{0pt}%
\pgfpathmoveto{\pgfqpoint{2.824734in}{0.827552in}}%
\pgfpathcurveto{\pgfqpoint{2.830260in}{0.827552in}}{\pgfqpoint{2.835559in}{0.829747in}}{\pgfqpoint{2.839466in}{0.833654in}}%
\pgfpathcurveto{\pgfqpoint{2.843373in}{0.837561in}}{\pgfqpoint{2.845568in}{0.842860in}}{\pgfqpoint{2.845568in}{0.848386in}}%
\pgfpathcurveto{\pgfqpoint{2.845568in}{0.853911in}}{\pgfqpoint{2.843373in}{0.859210in}}{\pgfqpoint{2.839466in}{0.863117in}}%
\pgfpathcurveto{\pgfqpoint{2.835559in}{0.867024in}}{\pgfqpoint{2.830260in}{0.869219in}}{\pgfqpoint{2.824734in}{0.869219in}}%
\pgfpathcurveto{\pgfqpoint{2.819209in}{0.869219in}}{\pgfqpoint{2.813910in}{0.867024in}}{\pgfqpoint{2.810003in}{0.863117in}}%
\pgfpathcurveto{\pgfqpoint{2.806096in}{0.859210in}}{\pgfqpoint{2.803901in}{0.853911in}}{\pgfqpoint{2.803901in}{0.848386in}}%
\pgfpathcurveto{\pgfqpoint{2.803901in}{0.842860in}}{\pgfqpoint{2.806096in}{0.837561in}}{\pgfqpoint{2.810003in}{0.833654in}}%
\pgfpathcurveto{\pgfqpoint{2.813910in}{0.829747in}}{\pgfqpoint{2.819209in}{0.827552in}}{\pgfqpoint{2.824734in}{0.827552in}}%
\pgfpathclose%
\pgfusepath{stroke,fill}%
\end{pgfscope}%
\begin{pgfscope}%
\pgfpathrectangle{\pgfqpoint{0.562500in}{0.275000in}}{\pgfqpoint{3.487500in}{1.925000in}}%
\pgfusepath{clip}%
\pgfsetbuttcap%
\pgfsetroundjoin%
\definecolor{currentfill}{rgb}{0.000000,0.000000,0.000000}%
\pgfsetfillcolor{currentfill}%
\pgfsetlinewidth{1.003750pt}%
\definecolor{currentstroke}{rgb}{0.000000,0.000000,0.000000}%
\pgfsetstrokecolor{currentstroke}%
\pgfsetdash{}{0pt}%
\pgfpathmoveto{\pgfqpoint{2.824734in}{0.807318in}}%
\pgfpathcurveto{\pgfqpoint{2.830260in}{0.807318in}}{\pgfqpoint{2.835559in}{0.809513in}}{\pgfqpoint{2.839466in}{0.813420in}}%
\pgfpathcurveto{\pgfqpoint{2.843373in}{0.817327in}}{\pgfqpoint{2.845568in}{0.822626in}}{\pgfqpoint{2.845568in}{0.828151in}}%
\pgfpathcurveto{\pgfqpoint{2.845568in}{0.833676in}}{\pgfqpoint{2.843373in}{0.838976in}}{\pgfqpoint{2.839466in}{0.842883in}}%
\pgfpathcurveto{\pgfqpoint{2.835559in}{0.846789in}}{\pgfqpoint{2.830260in}{0.848985in}}{\pgfqpoint{2.824734in}{0.848985in}}%
\pgfpathcurveto{\pgfqpoint{2.819209in}{0.848985in}}{\pgfqpoint{2.813910in}{0.846789in}}{\pgfqpoint{2.810003in}{0.842883in}}%
\pgfpathcurveto{\pgfqpoint{2.806096in}{0.838976in}}{\pgfqpoint{2.803901in}{0.833676in}}{\pgfqpoint{2.803901in}{0.828151in}}%
\pgfpathcurveto{\pgfqpoint{2.803901in}{0.822626in}}{\pgfqpoint{2.806096in}{0.817327in}}{\pgfqpoint{2.810003in}{0.813420in}}%
\pgfpathcurveto{\pgfqpoint{2.813910in}{0.809513in}}{\pgfqpoint{2.819209in}{0.807318in}}{\pgfqpoint{2.824734in}{0.807318in}}%
\pgfpathclose%
\pgfusepath{stroke,fill}%
\end{pgfscope}%
\begin{pgfscope}%
\pgfpathrectangle{\pgfqpoint{0.562500in}{0.275000in}}{\pgfqpoint{3.487500in}{1.925000in}}%
\pgfusepath{clip}%
\pgfsetbuttcap%
\pgfsetroundjoin%
\definecolor{currentfill}{rgb}{0.000000,0.000000,0.000000}%
\pgfsetfillcolor{currentfill}%
\pgfsetlinewidth{1.003750pt}%
\definecolor{currentstroke}{rgb}{0.000000,0.000000,0.000000}%
\pgfsetstrokecolor{currentstroke}%
\pgfsetdash{}{0pt}%
\pgfpathmoveto{\pgfqpoint{2.824734in}{1.136125in}}%
\pgfpathcurveto{\pgfqpoint{2.830260in}{1.136125in}}{\pgfqpoint{2.835559in}{1.138320in}}{\pgfqpoint{2.839466in}{1.142227in}}%
\pgfpathcurveto{\pgfqpoint{2.843373in}{1.146133in}}{\pgfqpoint{2.845568in}{1.151433in}}{\pgfqpoint{2.845568in}{1.156958in}}%
\pgfpathcurveto{\pgfqpoint{2.845568in}{1.162483in}}{\pgfqpoint{2.843373in}{1.167783in}}{\pgfqpoint{2.839466in}{1.171689in}}%
\pgfpathcurveto{\pgfqpoint{2.835559in}{1.175596in}}{\pgfqpoint{2.830260in}{1.177791in}}{\pgfqpoint{2.824734in}{1.177791in}}%
\pgfpathcurveto{\pgfqpoint{2.819209in}{1.177791in}}{\pgfqpoint{2.813910in}{1.175596in}}{\pgfqpoint{2.810003in}{1.171689in}}%
\pgfpathcurveto{\pgfqpoint{2.806096in}{1.167783in}}{\pgfqpoint{2.803901in}{1.162483in}}{\pgfqpoint{2.803901in}{1.156958in}}%
\pgfpathcurveto{\pgfqpoint{2.803901in}{1.151433in}}{\pgfqpoint{2.806096in}{1.146133in}}{\pgfqpoint{2.810003in}{1.142227in}}%
\pgfpathcurveto{\pgfqpoint{2.813910in}{1.138320in}}{\pgfqpoint{2.819209in}{1.136125in}}{\pgfqpoint{2.824734in}{1.136125in}}%
\pgfpathclose%
\pgfusepath{stroke,fill}%
\end{pgfscope}%
\begin{pgfscope}%
\pgfpathrectangle{\pgfqpoint{0.562500in}{0.275000in}}{\pgfqpoint{3.487500in}{1.925000in}}%
\pgfusepath{clip}%
\pgfsetbuttcap%
\pgfsetroundjoin%
\definecolor{currentfill}{rgb}{0.000000,0.000000,0.000000}%
\pgfsetfillcolor{currentfill}%
\pgfsetlinewidth{1.003750pt}%
\definecolor{currentstroke}{rgb}{0.000000,0.000000,0.000000}%
\pgfsetstrokecolor{currentstroke}%
\pgfsetdash{}{0pt}%
\pgfpathmoveto{\pgfqpoint{2.824734in}{0.842728in}}%
\pgfpathcurveto{\pgfqpoint{2.830260in}{0.842728in}}{\pgfqpoint{2.835559in}{0.844923in}}{\pgfqpoint{2.839466in}{0.848830in}}%
\pgfpathcurveto{\pgfqpoint{2.843373in}{0.852737in}}{\pgfqpoint{2.845568in}{0.858036in}}{\pgfqpoint{2.845568in}{0.863561in}}%
\pgfpathcurveto{\pgfqpoint{2.845568in}{0.869086in}}{\pgfqpoint{2.843373in}{0.874386in}}{\pgfqpoint{2.839466in}{0.878293in}}%
\pgfpathcurveto{\pgfqpoint{2.835559in}{0.882199in}}{\pgfqpoint{2.830260in}{0.884395in}}{\pgfqpoint{2.824734in}{0.884395in}}%
\pgfpathcurveto{\pgfqpoint{2.819209in}{0.884395in}}{\pgfqpoint{2.813910in}{0.882199in}}{\pgfqpoint{2.810003in}{0.878293in}}%
\pgfpathcurveto{\pgfqpoint{2.806096in}{0.874386in}}{\pgfqpoint{2.803901in}{0.869086in}}{\pgfqpoint{2.803901in}{0.863561in}}%
\pgfpathcurveto{\pgfqpoint{2.803901in}{0.858036in}}{\pgfqpoint{2.806096in}{0.852737in}}{\pgfqpoint{2.810003in}{0.848830in}}%
\pgfpathcurveto{\pgfqpoint{2.813910in}{0.844923in}}{\pgfqpoint{2.819209in}{0.842728in}}{\pgfqpoint{2.824734in}{0.842728in}}%
\pgfpathclose%
\pgfusepath{stroke,fill}%
\end{pgfscope}%
\begin{pgfscope}%
\pgfpathrectangle{\pgfqpoint{0.562500in}{0.275000in}}{\pgfqpoint{3.487500in}{1.925000in}}%
\pgfusepath{clip}%
\pgfsetbuttcap%
\pgfsetroundjoin%
\definecolor{currentfill}{rgb}{0.000000,0.000000,0.000000}%
\pgfsetfillcolor{currentfill}%
\pgfsetlinewidth{1.003750pt}%
\definecolor{currentstroke}{rgb}{0.000000,0.000000,0.000000}%
\pgfsetstrokecolor{currentstroke}%
\pgfsetdash{}{0pt}%
\pgfpathmoveto{\pgfqpoint{2.824734in}{0.928724in}}%
\pgfpathcurveto{\pgfqpoint{2.830260in}{0.928724in}}{\pgfqpoint{2.835559in}{0.930919in}}{\pgfqpoint{2.839466in}{0.934825in}}%
\pgfpathcurveto{\pgfqpoint{2.843373in}{0.938732in}}{\pgfqpoint{2.845568in}{0.944032in}}{\pgfqpoint{2.845568in}{0.949557in}}%
\pgfpathcurveto{\pgfqpoint{2.845568in}{0.955082in}}{\pgfqpoint{2.843373in}{0.960381in}}{\pgfqpoint{2.839466in}{0.964288in}}%
\pgfpathcurveto{\pgfqpoint{2.835559in}{0.968195in}}{\pgfqpoint{2.830260in}{0.970390in}}{\pgfqpoint{2.824734in}{0.970390in}}%
\pgfpathcurveto{\pgfqpoint{2.819209in}{0.970390in}}{\pgfqpoint{2.813910in}{0.968195in}}{\pgfqpoint{2.810003in}{0.964288in}}%
\pgfpathcurveto{\pgfqpoint{2.806096in}{0.960381in}}{\pgfqpoint{2.803901in}{0.955082in}}{\pgfqpoint{2.803901in}{0.949557in}}%
\pgfpathcurveto{\pgfqpoint{2.803901in}{0.944032in}}{\pgfqpoint{2.806096in}{0.938732in}}{\pgfqpoint{2.810003in}{0.934825in}}%
\pgfpathcurveto{\pgfqpoint{2.813910in}{0.930919in}}{\pgfqpoint{2.819209in}{0.928724in}}{\pgfqpoint{2.824734in}{0.928724in}}%
\pgfpathclose%
\pgfusepath{stroke,fill}%
\end{pgfscope}%
\begin{pgfscope}%
\pgfpathrectangle{\pgfqpoint{0.562500in}{0.275000in}}{\pgfqpoint{3.487500in}{1.925000in}}%
\pgfusepath{clip}%
\pgfsetbuttcap%
\pgfsetroundjoin%
\definecolor{currentfill}{rgb}{0.000000,0.000000,0.000000}%
\pgfsetfillcolor{currentfill}%
\pgfsetlinewidth{1.003750pt}%
\definecolor{currentstroke}{rgb}{0.000000,0.000000,0.000000}%
\pgfsetstrokecolor{currentstroke}%
\pgfsetdash{}{0pt}%
\pgfpathmoveto{\pgfqpoint{2.824734in}{0.857904in}}%
\pgfpathcurveto{\pgfqpoint{2.830260in}{0.857904in}}{\pgfqpoint{2.835559in}{0.860099in}}{\pgfqpoint{2.839466in}{0.864006in}}%
\pgfpathcurveto{\pgfqpoint{2.843373in}{0.867912in}}{\pgfqpoint{2.845568in}{0.873212in}}{\pgfqpoint{2.845568in}{0.878737in}}%
\pgfpathcurveto{\pgfqpoint{2.845568in}{0.884262in}}{\pgfqpoint{2.843373in}{0.889561in}}{\pgfqpoint{2.839466in}{0.893468in}}%
\pgfpathcurveto{\pgfqpoint{2.835559in}{0.897375in}}{\pgfqpoint{2.830260in}{0.899570in}}{\pgfqpoint{2.824734in}{0.899570in}}%
\pgfpathcurveto{\pgfqpoint{2.819209in}{0.899570in}}{\pgfqpoint{2.813910in}{0.897375in}}{\pgfqpoint{2.810003in}{0.893468in}}%
\pgfpathcurveto{\pgfqpoint{2.806096in}{0.889561in}}{\pgfqpoint{2.803901in}{0.884262in}}{\pgfqpoint{2.803901in}{0.878737in}}%
\pgfpathcurveto{\pgfqpoint{2.803901in}{0.873212in}}{\pgfqpoint{2.806096in}{0.867912in}}{\pgfqpoint{2.810003in}{0.864006in}}%
\pgfpathcurveto{\pgfqpoint{2.813910in}{0.860099in}}{\pgfqpoint{2.819209in}{0.857904in}}{\pgfqpoint{2.824734in}{0.857904in}}%
\pgfpathclose%
\pgfusepath{stroke,fill}%
\end{pgfscope}%
\begin{pgfscope}%
\pgfpathrectangle{\pgfqpoint{0.562500in}{0.275000in}}{\pgfqpoint{3.487500in}{1.925000in}}%
\pgfusepath{clip}%
\pgfsetbuttcap%
\pgfsetroundjoin%
\definecolor{currentfill}{rgb}{0.000000,0.000000,0.000000}%
\pgfsetfillcolor{currentfill}%
\pgfsetlinewidth{1.003750pt}%
\definecolor{currentstroke}{rgb}{0.000000,0.000000,0.000000}%
\pgfsetstrokecolor{currentstroke}%
\pgfsetdash{}{0pt}%
\pgfpathmoveto{\pgfqpoint{2.824734in}{0.908489in}}%
\pgfpathcurveto{\pgfqpoint{2.830260in}{0.908489in}}{\pgfqpoint{2.835559in}{0.910684in}}{\pgfqpoint{2.839466in}{0.914591in}}%
\pgfpathcurveto{\pgfqpoint{2.843373in}{0.918498in}}{\pgfqpoint{2.845568in}{0.923798in}}{\pgfqpoint{2.845568in}{0.929323in}}%
\pgfpathcurveto{\pgfqpoint{2.845568in}{0.934848in}}{\pgfqpoint{2.843373in}{0.940147in}}{\pgfqpoint{2.839466in}{0.944054in}}%
\pgfpathcurveto{\pgfqpoint{2.835559in}{0.947961in}}{\pgfqpoint{2.830260in}{0.950156in}}{\pgfqpoint{2.824734in}{0.950156in}}%
\pgfpathcurveto{\pgfqpoint{2.819209in}{0.950156in}}{\pgfqpoint{2.813910in}{0.947961in}}{\pgfqpoint{2.810003in}{0.944054in}}%
\pgfpathcurveto{\pgfqpoint{2.806096in}{0.940147in}}{\pgfqpoint{2.803901in}{0.934848in}}{\pgfqpoint{2.803901in}{0.929323in}}%
\pgfpathcurveto{\pgfqpoint{2.803901in}{0.923798in}}{\pgfqpoint{2.806096in}{0.918498in}}{\pgfqpoint{2.810003in}{0.914591in}}%
\pgfpathcurveto{\pgfqpoint{2.813910in}{0.910684in}}{\pgfqpoint{2.819209in}{0.908489in}}{\pgfqpoint{2.824734in}{0.908489in}}%
\pgfpathclose%
\pgfusepath{stroke,fill}%
\end{pgfscope}%
\begin{pgfscope}%
\pgfpathrectangle{\pgfqpoint{0.562500in}{0.275000in}}{\pgfqpoint{3.487500in}{1.925000in}}%
\pgfusepath{clip}%
\pgfsetbuttcap%
\pgfsetroundjoin%
\definecolor{currentfill}{rgb}{0.000000,0.000000,0.000000}%
\pgfsetfillcolor{currentfill}%
\pgfsetlinewidth{1.003750pt}%
\definecolor{currentstroke}{rgb}{0.000000,0.000000,0.000000}%
\pgfsetstrokecolor{currentstroke}%
\pgfsetdash{}{0pt}%
\pgfpathmoveto{\pgfqpoint{2.824734in}{0.908489in}}%
\pgfpathcurveto{\pgfqpoint{2.830260in}{0.908489in}}{\pgfqpoint{2.835559in}{0.910684in}}{\pgfqpoint{2.839466in}{0.914591in}}%
\pgfpathcurveto{\pgfqpoint{2.843373in}{0.918498in}}{\pgfqpoint{2.845568in}{0.923798in}}{\pgfqpoint{2.845568in}{0.929323in}}%
\pgfpathcurveto{\pgfqpoint{2.845568in}{0.934848in}}{\pgfqpoint{2.843373in}{0.940147in}}{\pgfqpoint{2.839466in}{0.944054in}}%
\pgfpathcurveto{\pgfqpoint{2.835559in}{0.947961in}}{\pgfqpoint{2.830260in}{0.950156in}}{\pgfqpoint{2.824734in}{0.950156in}}%
\pgfpathcurveto{\pgfqpoint{2.819209in}{0.950156in}}{\pgfqpoint{2.813910in}{0.947961in}}{\pgfqpoint{2.810003in}{0.944054in}}%
\pgfpathcurveto{\pgfqpoint{2.806096in}{0.940147in}}{\pgfqpoint{2.803901in}{0.934848in}}{\pgfqpoint{2.803901in}{0.929323in}}%
\pgfpathcurveto{\pgfqpoint{2.803901in}{0.923798in}}{\pgfqpoint{2.806096in}{0.918498in}}{\pgfqpoint{2.810003in}{0.914591in}}%
\pgfpathcurveto{\pgfqpoint{2.813910in}{0.910684in}}{\pgfqpoint{2.819209in}{0.908489in}}{\pgfqpoint{2.824734in}{0.908489in}}%
\pgfpathclose%
\pgfusepath{stroke,fill}%
\end{pgfscope}%
\begin{pgfscope}%
\pgfpathrectangle{\pgfqpoint{0.562500in}{0.275000in}}{\pgfqpoint{3.487500in}{1.925000in}}%
\pgfusepath{clip}%
\pgfsetbuttcap%
\pgfsetroundjoin%
\definecolor{currentfill}{rgb}{0.000000,0.000000,0.000000}%
\pgfsetfillcolor{currentfill}%
\pgfsetlinewidth{1.003750pt}%
\definecolor{currentstroke}{rgb}{0.000000,0.000000,0.000000}%
\pgfsetstrokecolor{currentstroke}%
\pgfsetdash{}{0pt}%
\pgfpathmoveto{\pgfqpoint{2.824734in}{0.852845in}}%
\pgfpathcurveto{\pgfqpoint{2.830260in}{0.852845in}}{\pgfqpoint{2.835559in}{0.855040in}}{\pgfqpoint{2.839466in}{0.858947in}}%
\pgfpathcurveto{\pgfqpoint{2.843373in}{0.862854in}}{\pgfqpoint{2.845568in}{0.868153in}}{\pgfqpoint{2.845568in}{0.873678in}}%
\pgfpathcurveto{\pgfqpoint{2.845568in}{0.879203in}}{\pgfqpoint{2.843373in}{0.884503in}}{\pgfqpoint{2.839466in}{0.888410in}}%
\pgfpathcurveto{\pgfqpoint{2.835559in}{0.892317in}}{\pgfqpoint{2.830260in}{0.894512in}}{\pgfqpoint{2.824734in}{0.894512in}}%
\pgfpathcurveto{\pgfqpoint{2.819209in}{0.894512in}}{\pgfqpoint{2.813910in}{0.892317in}}{\pgfqpoint{2.810003in}{0.888410in}}%
\pgfpathcurveto{\pgfqpoint{2.806096in}{0.884503in}}{\pgfqpoint{2.803901in}{0.879203in}}{\pgfqpoint{2.803901in}{0.873678in}}%
\pgfpathcurveto{\pgfqpoint{2.803901in}{0.868153in}}{\pgfqpoint{2.806096in}{0.862854in}}{\pgfqpoint{2.810003in}{0.858947in}}%
\pgfpathcurveto{\pgfqpoint{2.813910in}{0.855040in}}{\pgfqpoint{2.819209in}{0.852845in}}{\pgfqpoint{2.824734in}{0.852845in}}%
\pgfpathclose%
\pgfusepath{stroke,fill}%
\end{pgfscope}%
\begin{pgfscope}%
\pgfpathrectangle{\pgfqpoint{0.562500in}{0.275000in}}{\pgfqpoint{3.487500in}{1.925000in}}%
\pgfusepath{clip}%
\pgfsetbuttcap%
\pgfsetroundjoin%
\definecolor{currentfill}{rgb}{0.000000,0.000000,0.000000}%
\pgfsetfillcolor{currentfill}%
\pgfsetlinewidth{1.003750pt}%
\definecolor{currentstroke}{rgb}{0.000000,0.000000,0.000000}%
\pgfsetstrokecolor{currentstroke}%
\pgfsetdash{}{0pt}%
\pgfpathmoveto{\pgfqpoint{2.824734in}{0.857904in}}%
\pgfpathcurveto{\pgfqpoint{2.830260in}{0.857904in}}{\pgfqpoint{2.835559in}{0.860099in}}{\pgfqpoint{2.839466in}{0.864006in}}%
\pgfpathcurveto{\pgfqpoint{2.843373in}{0.867912in}}{\pgfqpoint{2.845568in}{0.873212in}}{\pgfqpoint{2.845568in}{0.878737in}}%
\pgfpathcurveto{\pgfqpoint{2.845568in}{0.884262in}}{\pgfqpoint{2.843373in}{0.889561in}}{\pgfqpoint{2.839466in}{0.893468in}}%
\pgfpathcurveto{\pgfqpoint{2.835559in}{0.897375in}}{\pgfqpoint{2.830260in}{0.899570in}}{\pgfqpoint{2.824734in}{0.899570in}}%
\pgfpathcurveto{\pgfqpoint{2.819209in}{0.899570in}}{\pgfqpoint{2.813910in}{0.897375in}}{\pgfqpoint{2.810003in}{0.893468in}}%
\pgfpathcurveto{\pgfqpoint{2.806096in}{0.889561in}}{\pgfqpoint{2.803901in}{0.884262in}}{\pgfqpoint{2.803901in}{0.878737in}}%
\pgfpathcurveto{\pgfqpoint{2.803901in}{0.873212in}}{\pgfqpoint{2.806096in}{0.867912in}}{\pgfqpoint{2.810003in}{0.864006in}}%
\pgfpathcurveto{\pgfqpoint{2.813910in}{0.860099in}}{\pgfqpoint{2.819209in}{0.857904in}}{\pgfqpoint{2.824734in}{0.857904in}}%
\pgfpathclose%
\pgfusepath{stroke,fill}%
\end{pgfscope}%
\begin{pgfscope}%
\pgfpathrectangle{\pgfqpoint{0.562500in}{0.275000in}}{\pgfqpoint{3.487500in}{1.925000in}}%
\pgfusepath{clip}%
\pgfsetbuttcap%
\pgfsetroundjoin%
\definecolor{currentfill}{rgb}{0.000000,0.000000,0.000000}%
\pgfsetfillcolor{currentfill}%
\pgfsetlinewidth{1.003750pt}%
\definecolor{currentstroke}{rgb}{0.000000,0.000000,0.000000}%
\pgfsetstrokecolor{currentstroke}%
\pgfsetdash{}{0pt}%
\pgfpathmoveto{\pgfqpoint{2.824734in}{0.893314in}}%
\pgfpathcurveto{\pgfqpoint{2.830260in}{0.893314in}}{\pgfqpoint{2.835559in}{0.895509in}}{\pgfqpoint{2.839466in}{0.899415in}}%
\pgfpathcurveto{\pgfqpoint{2.843373in}{0.903322in}}{\pgfqpoint{2.845568in}{0.908622in}}{\pgfqpoint{2.845568in}{0.914147in}}%
\pgfpathcurveto{\pgfqpoint{2.845568in}{0.919672in}}{\pgfqpoint{2.843373in}{0.924971in}}{\pgfqpoint{2.839466in}{0.928878in}}%
\pgfpathcurveto{\pgfqpoint{2.835559in}{0.932785in}}{\pgfqpoint{2.830260in}{0.934980in}}{\pgfqpoint{2.824734in}{0.934980in}}%
\pgfpathcurveto{\pgfqpoint{2.819209in}{0.934980in}}{\pgfqpoint{2.813910in}{0.932785in}}{\pgfqpoint{2.810003in}{0.928878in}}%
\pgfpathcurveto{\pgfqpoint{2.806096in}{0.924971in}}{\pgfqpoint{2.803901in}{0.919672in}}{\pgfqpoint{2.803901in}{0.914147in}}%
\pgfpathcurveto{\pgfqpoint{2.803901in}{0.908622in}}{\pgfqpoint{2.806096in}{0.903322in}}{\pgfqpoint{2.810003in}{0.899415in}}%
\pgfpathcurveto{\pgfqpoint{2.813910in}{0.895509in}}{\pgfqpoint{2.819209in}{0.893314in}}{\pgfqpoint{2.824734in}{0.893314in}}%
\pgfpathclose%
\pgfusepath{stroke,fill}%
\end{pgfscope}%
\begin{pgfscope}%
\pgfpathrectangle{\pgfqpoint{0.562500in}{0.275000in}}{\pgfqpoint{3.487500in}{1.925000in}}%
\pgfusepath{clip}%
\pgfsetbuttcap%
\pgfsetroundjoin%
\definecolor{currentfill}{rgb}{0.000000,0.000000,0.000000}%
\pgfsetfillcolor{currentfill}%
\pgfsetlinewidth{1.003750pt}%
\definecolor{currentstroke}{rgb}{0.000000,0.000000,0.000000}%
\pgfsetstrokecolor{currentstroke}%
\pgfsetdash{}{0pt}%
\pgfpathmoveto{\pgfqpoint{2.824734in}{0.938841in}}%
\pgfpathcurveto{\pgfqpoint{2.830260in}{0.938841in}}{\pgfqpoint{2.835559in}{0.941036in}}{\pgfqpoint{2.839466in}{0.944943in}}%
\pgfpathcurveto{\pgfqpoint{2.843373in}{0.948849in}}{\pgfqpoint{2.845568in}{0.954149in}}{\pgfqpoint{2.845568in}{0.959674in}}%
\pgfpathcurveto{\pgfqpoint{2.845568in}{0.965199in}}{\pgfqpoint{2.843373in}{0.970499in}}{\pgfqpoint{2.839466in}{0.974405in}}%
\pgfpathcurveto{\pgfqpoint{2.835559in}{0.978312in}}{\pgfqpoint{2.830260in}{0.980507in}}{\pgfqpoint{2.824734in}{0.980507in}}%
\pgfpathcurveto{\pgfqpoint{2.819209in}{0.980507in}}{\pgfqpoint{2.813910in}{0.978312in}}{\pgfqpoint{2.810003in}{0.974405in}}%
\pgfpathcurveto{\pgfqpoint{2.806096in}{0.970499in}}{\pgfqpoint{2.803901in}{0.965199in}}{\pgfqpoint{2.803901in}{0.959674in}}%
\pgfpathcurveto{\pgfqpoint{2.803901in}{0.954149in}}{\pgfqpoint{2.806096in}{0.948849in}}{\pgfqpoint{2.810003in}{0.944943in}}%
\pgfpathcurveto{\pgfqpoint{2.813910in}{0.941036in}}{\pgfqpoint{2.819209in}{0.938841in}}{\pgfqpoint{2.824734in}{0.938841in}}%
\pgfpathclose%
\pgfusepath{stroke,fill}%
\end{pgfscope}%
\begin{pgfscope}%
\pgfpathrectangle{\pgfqpoint{0.562500in}{0.275000in}}{\pgfqpoint{3.487500in}{1.925000in}}%
\pgfusepath{clip}%
\pgfsetbuttcap%
\pgfsetroundjoin%
\definecolor{currentfill}{rgb}{0.000000,0.000000,0.000000}%
\pgfsetfillcolor{currentfill}%
\pgfsetlinewidth{1.003750pt}%
\definecolor{currentstroke}{rgb}{0.000000,0.000000,0.000000}%
\pgfsetstrokecolor{currentstroke}%
\pgfsetdash{}{0pt}%
\pgfpathmoveto{\pgfqpoint{2.824734in}{0.883196in}}%
\pgfpathcurveto{\pgfqpoint{2.830260in}{0.883196in}}{\pgfqpoint{2.835559in}{0.885392in}}{\pgfqpoint{2.839466in}{0.889298in}}%
\pgfpathcurveto{\pgfqpoint{2.843373in}{0.893205in}}{\pgfqpoint{2.845568in}{0.898505in}}{\pgfqpoint{2.845568in}{0.904030in}}%
\pgfpathcurveto{\pgfqpoint{2.845568in}{0.909555in}}{\pgfqpoint{2.843373in}{0.914854in}}{\pgfqpoint{2.839466in}{0.918761in}}%
\pgfpathcurveto{\pgfqpoint{2.835559in}{0.922668in}}{\pgfqpoint{2.830260in}{0.924863in}}{\pgfqpoint{2.824734in}{0.924863in}}%
\pgfpathcurveto{\pgfqpoint{2.819209in}{0.924863in}}{\pgfqpoint{2.813910in}{0.922668in}}{\pgfqpoint{2.810003in}{0.918761in}}%
\pgfpathcurveto{\pgfqpoint{2.806096in}{0.914854in}}{\pgfqpoint{2.803901in}{0.909555in}}{\pgfqpoint{2.803901in}{0.904030in}}%
\pgfpathcurveto{\pgfqpoint{2.803901in}{0.898505in}}{\pgfqpoint{2.806096in}{0.893205in}}{\pgfqpoint{2.810003in}{0.889298in}}%
\pgfpathcurveto{\pgfqpoint{2.813910in}{0.885392in}}{\pgfqpoint{2.819209in}{0.883196in}}{\pgfqpoint{2.824734in}{0.883196in}}%
\pgfpathclose%
\pgfusepath{stroke,fill}%
\end{pgfscope}%
\begin{pgfscope}%
\pgfpathrectangle{\pgfqpoint{0.562500in}{0.275000in}}{\pgfqpoint{3.487500in}{1.925000in}}%
\pgfusepath{clip}%
\pgfsetbuttcap%
\pgfsetroundjoin%
\definecolor{currentfill}{rgb}{0.000000,0.000000,0.000000}%
\pgfsetfillcolor{currentfill}%
\pgfsetlinewidth{1.003750pt}%
\definecolor{currentstroke}{rgb}{0.000000,0.000000,0.000000}%
\pgfsetstrokecolor{currentstroke}%
\pgfsetdash{}{0pt}%
\pgfpathmoveto{\pgfqpoint{2.824734in}{0.923665in}}%
\pgfpathcurveto{\pgfqpoint{2.830260in}{0.923665in}}{\pgfqpoint{2.835559in}{0.925860in}}{\pgfqpoint{2.839466in}{0.929767in}}%
\pgfpathcurveto{\pgfqpoint{2.843373in}{0.933674in}}{\pgfqpoint{2.845568in}{0.938973in}}{\pgfqpoint{2.845568in}{0.944498in}}%
\pgfpathcurveto{\pgfqpoint{2.845568in}{0.950023in}}{\pgfqpoint{2.843373in}{0.955323in}}{\pgfqpoint{2.839466in}{0.959230in}}%
\pgfpathcurveto{\pgfqpoint{2.835559in}{0.963136in}}{\pgfqpoint{2.830260in}{0.965332in}}{\pgfqpoint{2.824734in}{0.965332in}}%
\pgfpathcurveto{\pgfqpoint{2.819209in}{0.965332in}}{\pgfqpoint{2.813910in}{0.963136in}}{\pgfqpoint{2.810003in}{0.959230in}}%
\pgfpathcurveto{\pgfqpoint{2.806096in}{0.955323in}}{\pgfqpoint{2.803901in}{0.950023in}}{\pgfqpoint{2.803901in}{0.944498in}}%
\pgfpathcurveto{\pgfqpoint{2.803901in}{0.938973in}}{\pgfqpoint{2.806096in}{0.933674in}}{\pgfqpoint{2.810003in}{0.929767in}}%
\pgfpathcurveto{\pgfqpoint{2.813910in}{0.925860in}}{\pgfqpoint{2.819209in}{0.923665in}}{\pgfqpoint{2.824734in}{0.923665in}}%
\pgfpathclose%
\pgfusepath{stroke,fill}%
\end{pgfscope}%
\begin{pgfscope}%
\pgfpathrectangle{\pgfqpoint{0.562500in}{0.275000in}}{\pgfqpoint{3.487500in}{1.925000in}}%
\pgfusepath{clip}%
\pgfsetbuttcap%
\pgfsetroundjoin%
\definecolor{currentfill}{rgb}{0.000000,0.000000,0.000000}%
\pgfsetfillcolor{currentfill}%
\pgfsetlinewidth{1.003750pt}%
\definecolor{currentstroke}{rgb}{0.000000,0.000000,0.000000}%
\pgfsetstrokecolor{currentstroke}%
\pgfsetdash{}{0pt}%
\pgfpathmoveto{\pgfqpoint{2.824734in}{0.918606in}}%
\pgfpathcurveto{\pgfqpoint{2.830260in}{0.918606in}}{\pgfqpoint{2.835559in}{0.920802in}}{\pgfqpoint{2.839466in}{0.924708in}}%
\pgfpathcurveto{\pgfqpoint{2.843373in}{0.928615in}}{\pgfqpoint{2.845568in}{0.933915in}}{\pgfqpoint{2.845568in}{0.939440in}}%
\pgfpathcurveto{\pgfqpoint{2.845568in}{0.944965in}}{\pgfqpoint{2.843373in}{0.950264in}}{\pgfqpoint{2.839466in}{0.954171in}}%
\pgfpathcurveto{\pgfqpoint{2.835559in}{0.958078in}}{\pgfqpoint{2.830260in}{0.960273in}}{\pgfqpoint{2.824734in}{0.960273in}}%
\pgfpathcurveto{\pgfqpoint{2.819209in}{0.960273in}}{\pgfqpoint{2.813910in}{0.958078in}}{\pgfqpoint{2.810003in}{0.954171in}}%
\pgfpathcurveto{\pgfqpoint{2.806096in}{0.950264in}}{\pgfqpoint{2.803901in}{0.944965in}}{\pgfqpoint{2.803901in}{0.939440in}}%
\pgfpathcurveto{\pgfqpoint{2.803901in}{0.933915in}}{\pgfqpoint{2.806096in}{0.928615in}}{\pgfqpoint{2.810003in}{0.924708in}}%
\pgfpathcurveto{\pgfqpoint{2.813910in}{0.920802in}}{\pgfqpoint{2.819209in}{0.918606in}}{\pgfqpoint{2.824734in}{0.918606in}}%
\pgfpathclose%
\pgfusepath{stroke,fill}%
\end{pgfscope}%
\begin{pgfscope}%
\pgfpathrectangle{\pgfqpoint{0.562500in}{0.275000in}}{\pgfqpoint{3.487500in}{1.925000in}}%
\pgfusepath{clip}%
\pgfsetbuttcap%
\pgfsetroundjoin%
\definecolor{currentfill}{rgb}{0.000000,0.000000,0.000000}%
\pgfsetfillcolor{currentfill}%
\pgfsetlinewidth{1.003750pt}%
\definecolor{currentstroke}{rgb}{0.000000,0.000000,0.000000}%
\pgfsetstrokecolor{currentstroke}%
\pgfsetdash{}{0pt}%
\pgfpathmoveto{\pgfqpoint{2.824734in}{0.883196in}}%
\pgfpathcurveto{\pgfqpoint{2.830260in}{0.883196in}}{\pgfqpoint{2.835559in}{0.885392in}}{\pgfqpoint{2.839466in}{0.889298in}}%
\pgfpathcurveto{\pgfqpoint{2.843373in}{0.893205in}}{\pgfqpoint{2.845568in}{0.898505in}}{\pgfqpoint{2.845568in}{0.904030in}}%
\pgfpathcurveto{\pgfqpoint{2.845568in}{0.909555in}}{\pgfqpoint{2.843373in}{0.914854in}}{\pgfqpoint{2.839466in}{0.918761in}}%
\pgfpathcurveto{\pgfqpoint{2.835559in}{0.922668in}}{\pgfqpoint{2.830260in}{0.924863in}}{\pgfqpoint{2.824734in}{0.924863in}}%
\pgfpathcurveto{\pgfqpoint{2.819209in}{0.924863in}}{\pgfqpoint{2.813910in}{0.922668in}}{\pgfqpoint{2.810003in}{0.918761in}}%
\pgfpathcurveto{\pgfqpoint{2.806096in}{0.914854in}}{\pgfqpoint{2.803901in}{0.909555in}}{\pgfqpoint{2.803901in}{0.904030in}}%
\pgfpathcurveto{\pgfqpoint{2.803901in}{0.898505in}}{\pgfqpoint{2.806096in}{0.893205in}}{\pgfqpoint{2.810003in}{0.889298in}}%
\pgfpathcurveto{\pgfqpoint{2.813910in}{0.885392in}}{\pgfqpoint{2.819209in}{0.883196in}}{\pgfqpoint{2.824734in}{0.883196in}}%
\pgfpathclose%
\pgfusepath{stroke,fill}%
\end{pgfscope}%
\begin{pgfscope}%
\pgfpathrectangle{\pgfqpoint{0.562500in}{0.275000in}}{\pgfqpoint{3.487500in}{1.925000in}}%
\pgfusepath{clip}%
\pgfsetbuttcap%
\pgfsetroundjoin%
\definecolor{currentfill}{rgb}{0.000000,0.000000,0.000000}%
\pgfsetfillcolor{currentfill}%
\pgfsetlinewidth{1.003750pt}%
\definecolor{currentstroke}{rgb}{0.000000,0.000000,0.000000}%
\pgfsetstrokecolor{currentstroke}%
\pgfsetdash{}{0pt}%
\pgfpathmoveto{\pgfqpoint{2.824734in}{0.822494in}}%
\pgfpathcurveto{\pgfqpoint{2.830260in}{0.822494in}}{\pgfqpoint{2.835559in}{0.824689in}}{\pgfqpoint{2.839466in}{0.828596in}}%
\pgfpathcurveto{\pgfqpoint{2.843373in}{0.832502in}}{\pgfqpoint{2.845568in}{0.837802in}}{\pgfqpoint{2.845568in}{0.843327in}}%
\pgfpathcurveto{\pgfqpoint{2.845568in}{0.848852in}}{\pgfqpoint{2.843373in}{0.854152in}}{\pgfqpoint{2.839466in}{0.858058in}}%
\pgfpathcurveto{\pgfqpoint{2.835559in}{0.861965in}}{\pgfqpoint{2.830260in}{0.864160in}}{\pgfqpoint{2.824734in}{0.864160in}}%
\pgfpathcurveto{\pgfqpoint{2.819209in}{0.864160in}}{\pgfqpoint{2.813910in}{0.861965in}}{\pgfqpoint{2.810003in}{0.858058in}}%
\pgfpathcurveto{\pgfqpoint{2.806096in}{0.854152in}}{\pgfqpoint{2.803901in}{0.848852in}}{\pgfqpoint{2.803901in}{0.843327in}}%
\pgfpathcurveto{\pgfqpoint{2.803901in}{0.837802in}}{\pgfqpoint{2.806096in}{0.832502in}}{\pgfqpoint{2.810003in}{0.828596in}}%
\pgfpathcurveto{\pgfqpoint{2.813910in}{0.824689in}}{\pgfqpoint{2.819209in}{0.822494in}}{\pgfqpoint{2.824734in}{0.822494in}}%
\pgfpathclose%
\pgfusepath{stroke,fill}%
\end{pgfscope}%
\begin{pgfscope}%
\pgfpathrectangle{\pgfqpoint{0.562500in}{0.275000in}}{\pgfqpoint{3.487500in}{1.925000in}}%
\pgfusepath{clip}%
\pgfsetbuttcap%
\pgfsetroundjoin%
\definecolor{currentfill}{rgb}{0.000000,0.000000,0.000000}%
\pgfsetfillcolor{currentfill}%
\pgfsetlinewidth{1.003750pt}%
\definecolor{currentstroke}{rgb}{0.000000,0.000000,0.000000}%
\pgfsetstrokecolor{currentstroke}%
\pgfsetdash{}{0pt}%
\pgfpathmoveto{\pgfqpoint{2.824734in}{0.883196in}}%
\pgfpathcurveto{\pgfqpoint{2.830260in}{0.883196in}}{\pgfqpoint{2.835559in}{0.885392in}}{\pgfqpoint{2.839466in}{0.889298in}}%
\pgfpathcurveto{\pgfqpoint{2.843373in}{0.893205in}}{\pgfqpoint{2.845568in}{0.898505in}}{\pgfqpoint{2.845568in}{0.904030in}}%
\pgfpathcurveto{\pgfqpoint{2.845568in}{0.909555in}}{\pgfqpoint{2.843373in}{0.914854in}}{\pgfqpoint{2.839466in}{0.918761in}}%
\pgfpathcurveto{\pgfqpoint{2.835559in}{0.922668in}}{\pgfqpoint{2.830260in}{0.924863in}}{\pgfqpoint{2.824734in}{0.924863in}}%
\pgfpathcurveto{\pgfqpoint{2.819209in}{0.924863in}}{\pgfqpoint{2.813910in}{0.922668in}}{\pgfqpoint{2.810003in}{0.918761in}}%
\pgfpathcurveto{\pgfqpoint{2.806096in}{0.914854in}}{\pgfqpoint{2.803901in}{0.909555in}}{\pgfqpoint{2.803901in}{0.904030in}}%
\pgfpathcurveto{\pgfqpoint{2.803901in}{0.898505in}}{\pgfqpoint{2.806096in}{0.893205in}}{\pgfqpoint{2.810003in}{0.889298in}}%
\pgfpathcurveto{\pgfqpoint{2.813910in}{0.885392in}}{\pgfqpoint{2.819209in}{0.883196in}}{\pgfqpoint{2.824734in}{0.883196in}}%
\pgfpathclose%
\pgfusepath{stroke,fill}%
\end{pgfscope}%
\begin{pgfscope}%
\pgfpathrectangle{\pgfqpoint{0.562500in}{0.275000in}}{\pgfqpoint{3.487500in}{1.925000in}}%
\pgfusepath{clip}%
\pgfsetbuttcap%
\pgfsetroundjoin%
\definecolor{currentfill}{rgb}{0.000000,0.000000,0.000000}%
\pgfsetfillcolor{currentfill}%
\pgfsetlinewidth{1.003750pt}%
\definecolor{currentstroke}{rgb}{0.000000,0.000000,0.000000}%
\pgfsetstrokecolor{currentstroke}%
\pgfsetdash{}{0pt}%
\pgfpathmoveto{\pgfqpoint{2.824734in}{0.984368in}}%
\pgfpathcurveto{\pgfqpoint{2.830260in}{0.984368in}}{\pgfqpoint{2.835559in}{0.986563in}}{\pgfqpoint{2.839466in}{0.990470in}}%
\pgfpathcurveto{\pgfqpoint{2.843373in}{0.994376in}}{\pgfqpoint{2.845568in}{0.999676in}}{\pgfqpoint{2.845568in}{1.005201in}}%
\pgfpathcurveto{\pgfqpoint{2.845568in}{1.010726in}}{\pgfqpoint{2.843373in}{1.016026in}}{\pgfqpoint{2.839466in}{1.019932in}}%
\pgfpathcurveto{\pgfqpoint{2.835559in}{1.023839in}}{\pgfqpoint{2.830260in}{1.026034in}}{\pgfqpoint{2.824734in}{1.026034in}}%
\pgfpathcurveto{\pgfqpoint{2.819209in}{1.026034in}}{\pgfqpoint{2.813910in}{1.023839in}}{\pgfqpoint{2.810003in}{1.019932in}}%
\pgfpathcurveto{\pgfqpoint{2.806096in}{1.016026in}}{\pgfqpoint{2.803901in}{1.010726in}}{\pgfqpoint{2.803901in}{1.005201in}}%
\pgfpathcurveto{\pgfqpoint{2.803901in}{0.999676in}}{\pgfqpoint{2.806096in}{0.994376in}}{\pgfqpoint{2.810003in}{0.990470in}}%
\pgfpathcurveto{\pgfqpoint{2.813910in}{0.986563in}}{\pgfqpoint{2.819209in}{0.984368in}}{\pgfqpoint{2.824734in}{0.984368in}}%
\pgfpathclose%
\pgfusepath{stroke,fill}%
\end{pgfscope}%
\begin{pgfscope}%
\pgfpathrectangle{\pgfqpoint{0.562500in}{0.275000in}}{\pgfqpoint{3.487500in}{1.925000in}}%
\pgfusepath{clip}%
\pgfsetbuttcap%
\pgfsetroundjoin%
\definecolor{currentfill}{rgb}{0.000000,0.000000,0.000000}%
\pgfsetfillcolor{currentfill}%
\pgfsetlinewidth{1.003750pt}%
\definecolor{currentstroke}{rgb}{0.000000,0.000000,0.000000}%
\pgfsetstrokecolor{currentstroke}%
\pgfsetdash{}{0pt}%
\pgfpathmoveto{\pgfqpoint{2.824734in}{0.938841in}}%
\pgfpathcurveto{\pgfqpoint{2.830260in}{0.938841in}}{\pgfqpoint{2.835559in}{0.941036in}}{\pgfqpoint{2.839466in}{0.944943in}}%
\pgfpathcurveto{\pgfqpoint{2.843373in}{0.948849in}}{\pgfqpoint{2.845568in}{0.954149in}}{\pgfqpoint{2.845568in}{0.959674in}}%
\pgfpathcurveto{\pgfqpoint{2.845568in}{0.965199in}}{\pgfqpoint{2.843373in}{0.970499in}}{\pgfqpoint{2.839466in}{0.974405in}}%
\pgfpathcurveto{\pgfqpoint{2.835559in}{0.978312in}}{\pgfqpoint{2.830260in}{0.980507in}}{\pgfqpoint{2.824734in}{0.980507in}}%
\pgfpathcurveto{\pgfqpoint{2.819209in}{0.980507in}}{\pgfqpoint{2.813910in}{0.978312in}}{\pgfqpoint{2.810003in}{0.974405in}}%
\pgfpathcurveto{\pgfqpoint{2.806096in}{0.970499in}}{\pgfqpoint{2.803901in}{0.965199in}}{\pgfqpoint{2.803901in}{0.959674in}}%
\pgfpathcurveto{\pgfqpoint{2.803901in}{0.954149in}}{\pgfqpoint{2.806096in}{0.948849in}}{\pgfqpoint{2.810003in}{0.944943in}}%
\pgfpathcurveto{\pgfqpoint{2.813910in}{0.941036in}}{\pgfqpoint{2.819209in}{0.938841in}}{\pgfqpoint{2.824734in}{0.938841in}}%
\pgfpathclose%
\pgfusepath{stroke,fill}%
\end{pgfscope}%
\begin{pgfscope}%
\pgfpathrectangle{\pgfqpoint{0.562500in}{0.275000in}}{\pgfqpoint{3.487500in}{1.925000in}}%
\pgfusepath{clip}%
\pgfsetbuttcap%
\pgfsetroundjoin%
\definecolor{currentfill}{rgb}{0.000000,0.000000,0.000000}%
\pgfsetfillcolor{currentfill}%
\pgfsetlinewidth{1.003750pt}%
\definecolor{currentstroke}{rgb}{0.000000,0.000000,0.000000}%
\pgfsetstrokecolor{currentstroke}%
\pgfsetdash{}{0pt}%
\pgfpathmoveto{\pgfqpoint{2.824734in}{0.797201in}}%
\pgfpathcurveto{\pgfqpoint{2.830260in}{0.797201in}}{\pgfqpoint{2.835559in}{0.799396in}}{\pgfqpoint{2.839466in}{0.803303in}}%
\pgfpathcurveto{\pgfqpoint{2.843373in}{0.807210in}}{\pgfqpoint{2.845568in}{0.812509in}}{\pgfqpoint{2.845568in}{0.818034in}}%
\pgfpathcurveto{\pgfqpoint{2.845568in}{0.823559in}}{\pgfqpoint{2.843373in}{0.828859in}}{\pgfqpoint{2.839466in}{0.832766in}}%
\pgfpathcurveto{\pgfqpoint{2.835559in}{0.836672in}}{\pgfqpoint{2.830260in}{0.838867in}}{\pgfqpoint{2.824734in}{0.838867in}}%
\pgfpathcurveto{\pgfqpoint{2.819209in}{0.838867in}}{\pgfqpoint{2.813910in}{0.836672in}}{\pgfqpoint{2.810003in}{0.832766in}}%
\pgfpathcurveto{\pgfqpoint{2.806096in}{0.828859in}}{\pgfqpoint{2.803901in}{0.823559in}}{\pgfqpoint{2.803901in}{0.818034in}}%
\pgfpathcurveto{\pgfqpoint{2.803901in}{0.812509in}}{\pgfqpoint{2.806096in}{0.807210in}}{\pgfqpoint{2.810003in}{0.803303in}}%
\pgfpathcurveto{\pgfqpoint{2.813910in}{0.799396in}}{\pgfqpoint{2.819209in}{0.797201in}}{\pgfqpoint{2.824734in}{0.797201in}}%
\pgfpathclose%
\pgfusepath{stroke,fill}%
\end{pgfscope}%
\begin{pgfscope}%
\pgfpathrectangle{\pgfqpoint{0.562500in}{0.275000in}}{\pgfqpoint{3.487500in}{1.925000in}}%
\pgfusepath{clip}%
\pgfsetbuttcap%
\pgfsetroundjoin%
\definecolor{currentfill}{rgb}{0.000000,0.000000,0.000000}%
\pgfsetfillcolor{currentfill}%
\pgfsetlinewidth{1.003750pt}%
\definecolor{currentstroke}{rgb}{0.000000,0.000000,0.000000}%
\pgfsetstrokecolor{currentstroke}%
\pgfsetdash{}{0pt}%
\pgfpathmoveto{\pgfqpoint{2.824734in}{0.969192in}}%
\pgfpathcurveto{\pgfqpoint{2.830260in}{0.969192in}}{\pgfqpoint{2.835559in}{0.971387in}}{\pgfqpoint{2.839466in}{0.975294in}}%
\pgfpathcurveto{\pgfqpoint{2.843373in}{0.979201in}}{\pgfqpoint{2.845568in}{0.984500in}}{\pgfqpoint{2.845568in}{0.990025in}}%
\pgfpathcurveto{\pgfqpoint{2.845568in}{0.995550in}}{\pgfqpoint{2.843373in}{1.000850in}}{\pgfqpoint{2.839466in}{1.004757in}}%
\pgfpathcurveto{\pgfqpoint{2.835559in}{1.008664in}}{\pgfqpoint{2.830260in}{1.010859in}}{\pgfqpoint{2.824734in}{1.010859in}}%
\pgfpathcurveto{\pgfqpoint{2.819209in}{1.010859in}}{\pgfqpoint{2.813910in}{1.008664in}}{\pgfqpoint{2.810003in}{1.004757in}}%
\pgfpathcurveto{\pgfqpoint{2.806096in}{1.000850in}}{\pgfqpoint{2.803901in}{0.995550in}}{\pgfqpoint{2.803901in}{0.990025in}}%
\pgfpathcurveto{\pgfqpoint{2.803901in}{0.984500in}}{\pgfqpoint{2.806096in}{0.979201in}}{\pgfqpoint{2.810003in}{0.975294in}}%
\pgfpathcurveto{\pgfqpoint{2.813910in}{0.971387in}}{\pgfqpoint{2.819209in}{0.969192in}}{\pgfqpoint{2.824734in}{0.969192in}}%
\pgfpathclose%
\pgfusepath{stroke,fill}%
\end{pgfscope}%
\begin{pgfscope}%
\pgfpathrectangle{\pgfqpoint{0.562500in}{0.275000in}}{\pgfqpoint{3.487500in}{1.925000in}}%
\pgfusepath{clip}%
\pgfsetbuttcap%
\pgfsetroundjoin%
\definecolor{currentfill}{rgb}{0.000000,0.000000,0.000000}%
\pgfsetfillcolor{currentfill}%
\pgfsetlinewidth{1.003750pt}%
\definecolor{currentstroke}{rgb}{0.000000,0.000000,0.000000}%
\pgfsetstrokecolor{currentstroke}%
\pgfsetdash{}{0pt}%
\pgfpathmoveto{\pgfqpoint{2.824734in}{0.979309in}}%
\pgfpathcurveto{\pgfqpoint{2.830260in}{0.979309in}}{\pgfqpoint{2.835559in}{0.981504in}}{\pgfqpoint{2.839466in}{0.985411in}}%
\pgfpathcurveto{\pgfqpoint{2.843373in}{0.989318in}}{\pgfqpoint{2.845568in}{0.994617in}}{\pgfqpoint{2.845568in}{1.000142in}}%
\pgfpathcurveto{\pgfqpoint{2.845568in}{1.005668in}}{\pgfqpoint{2.843373in}{1.010967in}}{\pgfqpoint{2.839466in}{1.014874in}}%
\pgfpathcurveto{\pgfqpoint{2.835559in}{1.018781in}}{\pgfqpoint{2.830260in}{1.020976in}}{\pgfqpoint{2.824734in}{1.020976in}}%
\pgfpathcurveto{\pgfqpoint{2.819209in}{1.020976in}}{\pgfqpoint{2.813910in}{1.018781in}}{\pgfqpoint{2.810003in}{1.014874in}}%
\pgfpathcurveto{\pgfqpoint{2.806096in}{1.010967in}}{\pgfqpoint{2.803901in}{1.005668in}}{\pgfqpoint{2.803901in}{1.000142in}}%
\pgfpathcurveto{\pgfqpoint{2.803901in}{0.994617in}}{\pgfqpoint{2.806096in}{0.989318in}}{\pgfqpoint{2.810003in}{0.985411in}}%
\pgfpathcurveto{\pgfqpoint{2.813910in}{0.981504in}}{\pgfqpoint{2.819209in}{0.979309in}}{\pgfqpoint{2.824734in}{0.979309in}}%
\pgfpathclose%
\pgfusepath{stroke,fill}%
\end{pgfscope}%
\begin{pgfscope}%
\pgfpathrectangle{\pgfqpoint{0.562500in}{0.275000in}}{\pgfqpoint{3.487500in}{1.925000in}}%
\pgfusepath{clip}%
\pgfsetbuttcap%
\pgfsetroundjoin%
\definecolor{currentfill}{rgb}{0.000000,0.000000,0.000000}%
\pgfsetfillcolor{currentfill}%
\pgfsetlinewidth{1.003750pt}%
\definecolor{currentstroke}{rgb}{0.000000,0.000000,0.000000}%
\pgfsetstrokecolor{currentstroke}%
\pgfsetdash{}{0pt}%
\pgfpathmoveto{\pgfqpoint{2.824734in}{0.913548in}}%
\pgfpathcurveto{\pgfqpoint{2.830260in}{0.913548in}}{\pgfqpoint{2.835559in}{0.915743in}}{\pgfqpoint{2.839466in}{0.919650in}}%
\pgfpathcurveto{\pgfqpoint{2.843373in}{0.923557in}}{\pgfqpoint{2.845568in}{0.928856in}}{\pgfqpoint{2.845568in}{0.934381in}}%
\pgfpathcurveto{\pgfqpoint{2.845568in}{0.939906in}}{\pgfqpoint{2.843373in}{0.945206in}}{\pgfqpoint{2.839466in}{0.949113in}}%
\pgfpathcurveto{\pgfqpoint{2.835559in}{0.953019in}}{\pgfqpoint{2.830260in}{0.955214in}}{\pgfqpoint{2.824734in}{0.955214in}}%
\pgfpathcurveto{\pgfqpoint{2.819209in}{0.955214in}}{\pgfqpoint{2.813910in}{0.953019in}}{\pgfqpoint{2.810003in}{0.949113in}}%
\pgfpathcurveto{\pgfqpoint{2.806096in}{0.945206in}}{\pgfqpoint{2.803901in}{0.939906in}}{\pgfqpoint{2.803901in}{0.934381in}}%
\pgfpathcurveto{\pgfqpoint{2.803901in}{0.928856in}}{\pgfqpoint{2.806096in}{0.923557in}}{\pgfqpoint{2.810003in}{0.919650in}}%
\pgfpathcurveto{\pgfqpoint{2.813910in}{0.915743in}}{\pgfqpoint{2.819209in}{0.913548in}}{\pgfqpoint{2.824734in}{0.913548in}}%
\pgfpathclose%
\pgfusepath{stroke,fill}%
\end{pgfscope}%
\begin{pgfscope}%
\pgfpathrectangle{\pgfqpoint{0.562500in}{0.275000in}}{\pgfqpoint{3.487500in}{1.925000in}}%
\pgfusepath{clip}%
\pgfsetbuttcap%
\pgfsetroundjoin%
\definecolor{currentfill}{rgb}{0.000000,0.000000,0.000000}%
\pgfsetfillcolor{currentfill}%
\pgfsetlinewidth{1.003750pt}%
\definecolor{currentstroke}{rgb}{0.000000,0.000000,0.000000}%
\pgfsetstrokecolor{currentstroke}%
\pgfsetdash{}{0pt}%
\pgfpathmoveto{\pgfqpoint{2.824734in}{0.817435in}}%
\pgfpathcurveto{\pgfqpoint{2.830260in}{0.817435in}}{\pgfqpoint{2.835559in}{0.819630in}}{\pgfqpoint{2.839466in}{0.823537in}}%
\pgfpathcurveto{\pgfqpoint{2.843373in}{0.827444in}}{\pgfqpoint{2.845568in}{0.832743in}}{\pgfqpoint{2.845568in}{0.838268in}}%
\pgfpathcurveto{\pgfqpoint{2.845568in}{0.843793in}}{\pgfqpoint{2.843373in}{0.849093in}}{\pgfqpoint{2.839466in}{0.853000in}}%
\pgfpathcurveto{\pgfqpoint{2.835559in}{0.856907in}}{\pgfqpoint{2.830260in}{0.859102in}}{\pgfqpoint{2.824734in}{0.859102in}}%
\pgfpathcurveto{\pgfqpoint{2.819209in}{0.859102in}}{\pgfqpoint{2.813910in}{0.856907in}}{\pgfqpoint{2.810003in}{0.853000in}}%
\pgfpathcurveto{\pgfqpoint{2.806096in}{0.849093in}}{\pgfqpoint{2.803901in}{0.843793in}}{\pgfqpoint{2.803901in}{0.838268in}}%
\pgfpathcurveto{\pgfqpoint{2.803901in}{0.832743in}}{\pgfqpoint{2.806096in}{0.827444in}}{\pgfqpoint{2.810003in}{0.823537in}}%
\pgfpathcurveto{\pgfqpoint{2.813910in}{0.819630in}}{\pgfqpoint{2.819209in}{0.817435in}}{\pgfqpoint{2.824734in}{0.817435in}}%
\pgfpathclose%
\pgfusepath{stroke,fill}%
\end{pgfscope}%
\begin{pgfscope}%
\pgfpathrectangle{\pgfqpoint{0.562500in}{0.275000in}}{\pgfqpoint{3.487500in}{1.925000in}}%
\pgfusepath{clip}%
\pgfsetbuttcap%
\pgfsetroundjoin%
\definecolor{currentfill}{rgb}{0.000000,0.000000,0.000000}%
\pgfsetfillcolor{currentfill}%
\pgfsetlinewidth{1.003750pt}%
\definecolor{currentstroke}{rgb}{0.000000,0.000000,0.000000}%
\pgfsetstrokecolor{currentstroke}%
\pgfsetdash{}{0pt}%
\pgfpathmoveto{\pgfqpoint{2.824734in}{0.888255in}}%
\pgfpathcurveto{\pgfqpoint{2.830260in}{0.888255in}}{\pgfqpoint{2.835559in}{0.890450in}}{\pgfqpoint{2.839466in}{0.894357in}}%
\pgfpathcurveto{\pgfqpoint{2.843373in}{0.898264in}}{\pgfqpoint{2.845568in}{0.903563in}}{\pgfqpoint{2.845568in}{0.909088in}}%
\pgfpathcurveto{\pgfqpoint{2.845568in}{0.914613in}}{\pgfqpoint{2.843373in}{0.919913in}}{\pgfqpoint{2.839466in}{0.923820in}}%
\pgfpathcurveto{\pgfqpoint{2.835559in}{0.927727in}}{\pgfqpoint{2.830260in}{0.929922in}}{\pgfqpoint{2.824734in}{0.929922in}}%
\pgfpathcurveto{\pgfqpoint{2.819209in}{0.929922in}}{\pgfqpoint{2.813910in}{0.927727in}}{\pgfqpoint{2.810003in}{0.923820in}}%
\pgfpathcurveto{\pgfqpoint{2.806096in}{0.919913in}}{\pgfqpoint{2.803901in}{0.914613in}}{\pgfqpoint{2.803901in}{0.909088in}}%
\pgfpathcurveto{\pgfqpoint{2.803901in}{0.903563in}}{\pgfqpoint{2.806096in}{0.898264in}}{\pgfqpoint{2.810003in}{0.894357in}}%
\pgfpathcurveto{\pgfqpoint{2.813910in}{0.890450in}}{\pgfqpoint{2.819209in}{0.888255in}}{\pgfqpoint{2.824734in}{0.888255in}}%
\pgfpathclose%
\pgfusepath{stroke,fill}%
\end{pgfscope}%
\begin{pgfscope}%
\pgfpathrectangle{\pgfqpoint{0.562500in}{0.275000in}}{\pgfqpoint{3.487500in}{1.925000in}}%
\pgfusepath{clip}%
\pgfsetbuttcap%
\pgfsetroundjoin%
\definecolor{currentfill}{rgb}{0.000000,0.000000,0.000000}%
\pgfsetfillcolor{currentfill}%
\pgfsetlinewidth{1.003750pt}%
\definecolor{currentstroke}{rgb}{0.000000,0.000000,0.000000}%
\pgfsetstrokecolor{currentstroke}%
\pgfsetdash{}{0pt}%
\pgfpathmoveto{\pgfqpoint{2.824734in}{0.928724in}}%
\pgfpathcurveto{\pgfqpoint{2.830260in}{0.928724in}}{\pgfqpoint{2.835559in}{0.930919in}}{\pgfqpoint{2.839466in}{0.934825in}}%
\pgfpathcurveto{\pgfqpoint{2.843373in}{0.938732in}}{\pgfqpoint{2.845568in}{0.944032in}}{\pgfqpoint{2.845568in}{0.949557in}}%
\pgfpathcurveto{\pgfqpoint{2.845568in}{0.955082in}}{\pgfqpoint{2.843373in}{0.960381in}}{\pgfqpoint{2.839466in}{0.964288in}}%
\pgfpathcurveto{\pgfqpoint{2.835559in}{0.968195in}}{\pgfqpoint{2.830260in}{0.970390in}}{\pgfqpoint{2.824734in}{0.970390in}}%
\pgfpathcurveto{\pgfqpoint{2.819209in}{0.970390in}}{\pgfqpoint{2.813910in}{0.968195in}}{\pgfqpoint{2.810003in}{0.964288in}}%
\pgfpathcurveto{\pgfqpoint{2.806096in}{0.960381in}}{\pgfqpoint{2.803901in}{0.955082in}}{\pgfqpoint{2.803901in}{0.949557in}}%
\pgfpathcurveto{\pgfqpoint{2.803901in}{0.944032in}}{\pgfqpoint{2.806096in}{0.938732in}}{\pgfqpoint{2.810003in}{0.934825in}}%
\pgfpathcurveto{\pgfqpoint{2.813910in}{0.930919in}}{\pgfqpoint{2.819209in}{0.928724in}}{\pgfqpoint{2.824734in}{0.928724in}}%
\pgfpathclose%
\pgfusepath{stroke,fill}%
\end{pgfscope}%
\begin{pgfscope}%
\pgfpathrectangle{\pgfqpoint{0.562500in}{0.275000in}}{\pgfqpoint{3.487500in}{1.925000in}}%
\pgfusepath{clip}%
\pgfsetbuttcap%
\pgfsetroundjoin%
\definecolor{currentfill}{rgb}{0.000000,0.000000,0.000000}%
\pgfsetfillcolor{currentfill}%
\pgfsetlinewidth{1.003750pt}%
\definecolor{currentstroke}{rgb}{0.000000,0.000000,0.000000}%
\pgfsetstrokecolor{currentstroke}%
\pgfsetdash{}{0pt}%
\pgfpathmoveto{\pgfqpoint{2.824734in}{0.842728in}}%
\pgfpathcurveto{\pgfqpoint{2.830260in}{0.842728in}}{\pgfqpoint{2.835559in}{0.844923in}}{\pgfqpoint{2.839466in}{0.848830in}}%
\pgfpathcurveto{\pgfqpoint{2.843373in}{0.852737in}}{\pgfqpoint{2.845568in}{0.858036in}}{\pgfqpoint{2.845568in}{0.863561in}}%
\pgfpathcurveto{\pgfqpoint{2.845568in}{0.869086in}}{\pgfqpoint{2.843373in}{0.874386in}}{\pgfqpoint{2.839466in}{0.878293in}}%
\pgfpathcurveto{\pgfqpoint{2.835559in}{0.882199in}}{\pgfqpoint{2.830260in}{0.884395in}}{\pgfqpoint{2.824734in}{0.884395in}}%
\pgfpathcurveto{\pgfqpoint{2.819209in}{0.884395in}}{\pgfqpoint{2.813910in}{0.882199in}}{\pgfqpoint{2.810003in}{0.878293in}}%
\pgfpathcurveto{\pgfqpoint{2.806096in}{0.874386in}}{\pgfqpoint{2.803901in}{0.869086in}}{\pgfqpoint{2.803901in}{0.863561in}}%
\pgfpathcurveto{\pgfqpoint{2.803901in}{0.858036in}}{\pgfqpoint{2.806096in}{0.852737in}}{\pgfqpoint{2.810003in}{0.848830in}}%
\pgfpathcurveto{\pgfqpoint{2.813910in}{0.844923in}}{\pgfqpoint{2.819209in}{0.842728in}}{\pgfqpoint{2.824734in}{0.842728in}}%
\pgfpathclose%
\pgfusepath{stroke,fill}%
\end{pgfscope}%
\begin{pgfscope}%
\pgfpathrectangle{\pgfqpoint{0.562500in}{0.275000in}}{\pgfqpoint{3.487500in}{1.925000in}}%
\pgfusepath{clip}%
\pgfsetbuttcap%
\pgfsetroundjoin%
\definecolor{currentfill}{rgb}{0.000000,0.000000,0.000000}%
\pgfsetfillcolor{currentfill}%
\pgfsetlinewidth{1.003750pt}%
\definecolor{currentstroke}{rgb}{0.000000,0.000000,0.000000}%
\pgfsetstrokecolor{currentstroke}%
\pgfsetdash{}{0pt}%
\pgfpathmoveto{\pgfqpoint{2.824734in}{0.999543in}}%
\pgfpathcurveto{\pgfqpoint{2.830260in}{0.999543in}}{\pgfqpoint{2.835559in}{1.001739in}}{\pgfqpoint{2.839466in}{1.005645in}}%
\pgfpathcurveto{\pgfqpoint{2.843373in}{1.009552in}}{\pgfqpoint{2.845568in}{1.014852in}}{\pgfqpoint{2.845568in}{1.020377in}}%
\pgfpathcurveto{\pgfqpoint{2.845568in}{1.025902in}}{\pgfqpoint{2.843373in}{1.031201in}}{\pgfqpoint{2.839466in}{1.035108in}}%
\pgfpathcurveto{\pgfqpoint{2.835559in}{1.039015in}}{\pgfqpoint{2.830260in}{1.041210in}}{\pgfqpoint{2.824734in}{1.041210in}}%
\pgfpathcurveto{\pgfqpoint{2.819209in}{1.041210in}}{\pgfqpoint{2.813910in}{1.039015in}}{\pgfqpoint{2.810003in}{1.035108in}}%
\pgfpathcurveto{\pgfqpoint{2.806096in}{1.031201in}}{\pgfqpoint{2.803901in}{1.025902in}}{\pgfqpoint{2.803901in}{1.020377in}}%
\pgfpathcurveto{\pgfqpoint{2.803901in}{1.014852in}}{\pgfqpoint{2.806096in}{1.009552in}}{\pgfqpoint{2.810003in}{1.005645in}}%
\pgfpathcurveto{\pgfqpoint{2.813910in}{1.001739in}}{\pgfqpoint{2.819209in}{0.999543in}}{\pgfqpoint{2.824734in}{0.999543in}}%
\pgfpathclose%
\pgfusepath{stroke,fill}%
\end{pgfscope}%
\begin{pgfscope}%
\pgfpathrectangle{\pgfqpoint{0.562500in}{0.275000in}}{\pgfqpoint{3.487500in}{1.925000in}}%
\pgfusepath{clip}%
\pgfsetbuttcap%
\pgfsetroundjoin%
\definecolor{currentfill}{rgb}{0.000000,0.000000,0.000000}%
\pgfsetfillcolor{currentfill}%
\pgfsetlinewidth{1.003750pt}%
\definecolor{currentstroke}{rgb}{0.000000,0.000000,0.000000}%
\pgfsetstrokecolor{currentstroke}%
\pgfsetdash{}{0pt}%
\pgfpathmoveto{\pgfqpoint{2.824734in}{0.802259in}}%
\pgfpathcurveto{\pgfqpoint{2.830260in}{0.802259in}}{\pgfqpoint{2.835559in}{0.804454in}}{\pgfqpoint{2.839466in}{0.808361in}}%
\pgfpathcurveto{\pgfqpoint{2.843373in}{0.812268in}}{\pgfqpoint{2.845568in}{0.817568in}}{\pgfqpoint{2.845568in}{0.823093in}}%
\pgfpathcurveto{\pgfqpoint{2.845568in}{0.828618in}}{\pgfqpoint{2.843373in}{0.833917in}}{\pgfqpoint{2.839466in}{0.837824in}}%
\pgfpathcurveto{\pgfqpoint{2.835559in}{0.841731in}}{\pgfqpoint{2.830260in}{0.843926in}}{\pgfqpoint{2.824734in}{0.843926in}}%
\pgfpathcurveto{\pgfqpoint{2.819209in}{0.843926in}}{\pgfqpoint{2.813910in}{0.841731in}}{\pgfqpoint{2.810003in}{0.837824in}}%
\pgfpathcurveto{\pgfqpoint{2.806096in}{0.833917in}}{\pgfqpoint{2.803901in}{0.828618in}}{\pgfqpoint{2.803901in}{0.823093in}}%
\pgfpathcurveto{\pgfqpoint{2.803901in}{0.817568in}}{\pgfqpoint{2.806096in}{0.812268in}}{\pgfqpoint{2.810003in}{0.808361in}}%
\pgfpathcurveto{\pgfqpoint{2.813910in}{0.804454in}}{\pgfqpoint{2.819209in}{0.802259in}}{\pgfqpoint{2.824734in}{0.802259in}}%
\pgfpathclose%
\pgfusepath{stroke,fill}%
\end{pgfscope}%
\begin{pgfscope}%
\pgfpathrectangle{\pgfqpoint{0.562500in}{0.275000in}}{\pgfqpoint{3.487500in}{1.925000in}}%
\pgfusepath{clip}%
\pgfsetbuttcap%
\pgfsetroundjoin%
\definecolor{currentfill}{rgb}{0.000000,0.000000,0.000000}%
\pgfsetfillcolor{currentfill}%
\pgfsetlinewidth{1.003750pt}%
\definecolor{currentstroke}{rgb}{0.000000,0.000000,0.000000}%
\pgfsetstrokecolor{currentstroke}%
\pgfsetdash{}{0pt}%
\pgfpathmoveto{\pgfqpoint{2.824734in}{0.883196in}}%
\pgfpathcurveto{\pgfqpoint{2.830260in}{0.883196in}}{\pgfqpoint{2.835559in}{0.885392in}}{\pgfqpoint{2.839466in}{0.889298in}}%
\pgfpathcurveto{\pgfqpoint{2.843373in}{0.893205in}}{\pgfqpoint{2.845568in}{0.898505in}}{\pgfqpoint{2.845568in}{0.904030in}}%
\pgfpathcurveto{\pgfqpoint{2.845568in}{0.909555in}}{\pgfqpoint{2.843373in}{0.914854in}}{\pgfqpoint{2.839466in}{0.918761in}}%
\pgfpathcurveto{\pgfqpoint{2.835559in}{0.922668in}}{\pgfqpoint{2.830260in}{0.924863in}}{\pgfqpoint{2.824734in}{0.924863in}}%
\pgfpathcurveto{\pgfqpoint{2.819209in}{0.924863in}}{\pgfqpoint{2.813910in}{0.922668in}}{\pgfqpoint{2.810003in}{0.918761in}}%
\pgfpathcurveto{\pgfqpoint{2.806096in}{0.914854in}}{\pgfqpoint{2.803901in}{0.909555in}}{\pgfqpoint{2.803901in}{0.904030in}}%
\pgfpathcurveto{\pgfqpoint{2.803901in}{0.898505in}}{\pgfqpoint{2.806096in}{0.893205in}}{\pgfqpoint{2.810003in}{0.889298in}}%
\pgfpathcurveto{\pgfqpoint{2.813910in}{0.885392in}}{\pgfqpoint{2.819209in}{0.883196in}}{\pgfqpoint{2.824734in}{0.883196in}}%
\pgfpathclose%
\pgfusepath{stroke,fill}%
\end{pgfscope}%
\begin{pgfscope}%
\pgfpathrectangle{\pgfqpoint{0.562500in}{0.275000in}}{\pgfqpoint{3.487500in}{1.925000in}}%
\pgfusepath{clip}%
\pgfsetbuttcap%
\pgfsetroundjoin%
\definecolor{currentfill}{rgb}{0.000000,0.000000,0.000000}%
\pgfsetfillcolor{currentfill}%
\pgfsetlinewidth{1.003750pt}%
\definecolor{currentstroke}{rgb}{0.000000,0.000000,0.000000}%
\pgfsetstrokecolor{currentstroke}%
\pgfsetdash{}{0pt}%
\pgfpathmoveto{\pgfqpoint{2.824734in}{0.888255in}}%
\pgfpathcurveto{\pgfqpoint{2.830260in}{0.888255in}}{\pgfqpoint{2.835559in}{0.890450in}}{\pgfqpoint{2.839466in}{0.894357in}}%
\pgfpathcurveto{\pgfqpoint{2.843373in}{0.898264in}}{\pgfqpoint{2.845568in}{0.903563in}}{\pgfqpoint{2.845568in}{0.909088in}}%
\pgfpathcurveto{\pgfqpoint{2.845568in}{0.914613in}}{\pgfqpoint{2.843373in}{0.919913in}}{\pgfqpoint{2.839466in}{0.923820in}}%
\pgfpathcurveto{\pgfqpoint{2.835559in}{0.927727in}}{\pgfqpoint{2.830260in}{0.929922in}}{\pgfqpoint{2.824734in}{0.929922in}}%
\pgfpathcurveto{\pgfqpoint{2.819209in}{0.929922in}}{\pgfqpoint{2.813910in}{0.927727in}}{\pgfqpoint{2.810003in}{0.923820in}}%
\pgfpathcurveto{\pgfqpoint{2.806096in}{0.919913in}}{\pgfqpoint{2.803901in}{0.914613in}}{\pgfqpoint{2.803901in}{0.909088in}}%
\pgfpathcurveto{\pgfqpoint{2.803901in}{0.903563in}}{\pgfqpoint{2.806096in}{0.898264in}}{\pgfqpoint{2.810003in}{0.894357in}}%
\pgfpathcurveto{\pgfqpoint{2.813910in}{0.890450in}}{\pgfqpoint{2.819209in}{0.888255in}}{\pgfqpoint{2.824734in}{0.888255in}}%
\pgfpathclose%
\pgfusepath{stroke,fill}%
\end{pgfscope}%
\begin{pgfscope}%
\pgfpathrectangle{\pgfqpoint{0.562500in}{0.275000in}}{\pgfqpoint{3.487500in}{1.925000in}}%
\pgfusepath{clip}%
\pgfsetbuttcap%
\pgfsetroundjoin%
\definecolor{currentfill}{rgb}{0.000000,0.000000,0.000000}%
\pgfsetfillcolor{currentfill}%
\pgfsetlinewidth{1.003750pt}%
\definecolor{currentstroke}{rgb}{0.000000,0.000000,0.000000}%
\pgfsetstrokecolor{currentstroke}%
\pgfsetdash{}{0pt}%
\pgfpathmoveto{\pgfqpoint{2.824734in}{0.918606in}}%
\pgfpathcurveto{\pgfqpoint{2.830260in}{0.918606in}}{\pgfqpoint{2.835559in}{0.920802in}}{\pgfqpoint{2.839466in}{0.924708in}}%
\pgfpathcurveto{\pgfqpoint{2.843373in}{0.928615in}}{\pgfqpoint{2.845568in}{0.933915in}}{\pgfqpoint{2.845568in}{0.939440in}}%
\pgfpathcurveto{\pgfqpoint{2.845568in}{0.944965in}}{\pgfqpoint{2.843373in}{0.950264in}}{\pgfqpoint{2.839466in}{0.954171in}}%
\pgfpathcurveto{\pgfqpoint{2.835559in}{0.958078in}}{\pgfqpoint{2.830260in}{0.960273in}}{\pgfqpoint{2.824734in}{0.960273in}}%
\pgfpathcurveto{\pgfqpoint{2.819209in}{0.960273in}}{\pgfqpoint{2.813910in}{0.958078in}}{\pgfqpoint{2.810003in}{0.954171in}}%
\pgfpathcurveto{\pgfqpoint{2.806096in}{0.950264in}}{\pgfqpoint{2.803901in}{0.944965in}}{\pgfqpoint{2.803901in}{0.939440in}}%
\pgfpathcurveto{\pgfqpoint{2.803901in}{0.933915in}}{\pgfqpoint{2.806096in}{0.928615in}}{\pgfqpoint{2.810003in}{0.924708in}}%
\pgfpathcurveto{\pgfqpoint{2.813910in}{0.920802in}}{\pgfqpoint{2.819209in}{0.918606in}}{\pgfqpoint{2.824734in}{0.918606in}}%
\pgfpathclose%
\pgfusepath{stroke,fill}%
\end{pgfscope}%
\begin{pgfscope}%
\pgfpathrectangle{\pgfqpoint{0.562500in}{0.275000in}}{\pgfqpoint{3.487500in}{1.925000in}}%
\pgfusepath{clip}%
\pgfsetbuttcap%
\pgfsetroundjoin%
\definecolor{currentfill}{rgb}{0.000000,0.000000,0.000000}%
\pgfsetfillcolor{currentfill}%
\pgfsetlinewidth{1.003750pt}%
\definecolor{currentstroke}{rgb}{0.000000,0.000000,0.000000}%
\pgfsetstrokecolor{currentstroke}%
\pgfsetdash{}{0pt}%
\pgfpathmoveto{\pgfqpoint{2.824734in}{0.928724in}}%
\pgfpathcurveto{\pgfqpoint{2.830260in}{0.928724in}}{\pgfqpoint{2.835559in}{0.930919in}}{\pgfqpoint{2.839466in}{0.934825in}}%
\pgfpathcurveto{\pgfqpoint{2.843373in}{0.938732in}}{\pgfqpoint{2.845568in}{0.944032in}}{\pgfqpoint{2.845568in}{0.949557in}}%
\pgfpathcurveto{\pgfqpoint{2.845568in}{0.955082in}}{\pgfqpoint{2.843373in}{0.960381in}}{\pgfqpoint{2.839466in}{0.964288in}}%
\pgfpathcurveto{\pgfqpoint{2.835559in}{0.968195in}}{\pgfqpoint{2.830260in}{0.970390in}}{\pgfqpoint{2.824734in}{0.970390in}}%
\pgfpathcurveto{\pgfqpoint{2.819209in}{0.970390in}}{\pgfqpoint{2.813910in}{0.968195in}}{\pgfqpoint{2.810003in}{0.964288in}}%
\pgfpathcurveto{\pgfqpoint{2.806096in}{0.960381in}}{\pgfqpoint{2.803901in}{0.955082in}}{\pgfqpoint{2.803901in}{0.949557in}}%
\pgfpathcurveto{\pgfqpoint{2.803901in}{0.944032in}}{\pgfqpoint{2.806096in}{0.938732in}}{\pgfqpoint{2.810003in}{0.934825in}}%
\pgfpathcurveto{\pgfqpoint{2.813910in}{0.930919in}}{\pgfqpoint{2.819209in}{0.928724in}}{\pgfqpoint{2.824734in}{0.928724in}}%
\pgfpathclose%
\pgfusepath{stroke,fill}%
\end{pgfscope}%
\begin{pgfscope}%
\pgfpathrectangle{\pgfqpoint{0.562500in}{0.275000in}}{\pgfqpoint{3.487500in}{1.925000in}}%
\pgfusepath{clip}%
\pgfsetbuttcap%
\pgfsetroundjoin%
\definecolor{currentfill}{rgb}{0.000000,0.000000,0.000000}%
\pgfsetfillcolor{currentfill}%
\pgfsetlinewidth{1.003750pt}%
\definecolor{currentstroke}{rgb}{0.000000,0.000000,0.000000}%
\pgfsetstrokecolor{currentstroke}%
\pgfsetdash{}{0pt}%
\pgfpathmoveto{\pgfqpoint{2.824734in}{0.999543in}}%
\pgfpathcurveto{\pgfqpoint{2.830260in}{0.999543in}}{\pgfqpoint{2.835559in}{1.001739in}}{\pgfqpoint{2.839466in}{1.005645in}}%
\pgfpathcurveto{\pgfqpoint{2.843373in}{1.009552in}}{\pgfqpoint{2.845568in}{1.014852in}}{\pgfqpoint{2.845568in}{1.020377in}}%
\pgfpathcurveto{\pgfqpoint{2.845568in}{1.025902in}}{\pgfqpoint{2.843373in}{1.031201in}}{\pgfqpoint{2.839466in}{1.035108in}}%
\pgfpathcurveto{\pgfqpoint{2.835559in}{1.039015in}}{\pgfqpoint{2.830260in}{1.041210in}}{\pgfqpoint{2.824734in}{1.041210in}}%
\pgfpathcurveto{\pgfqpoint{2.819209in}{1.041210in}}{\pgfqpoint{2.813910in}{1.039015in}}{\pgfqpoint{2.810003in}{1.035108in}}%
\pgfpathcurveto{\pgfqpoint{2.806096in}{1.031201in}}{\pgfqpoint{2.803901in}{1.025902in}}{\pgfqpoint{2.803901in}{1.020377in}}%
\pgfpathcurveto{\pgfqpoint{2.803901in}{1.014852in}}{\pgfqpoint{2.806096in}{1.009552in}}{\pgfqpoint{2.810003in}{1.005645in}}%
\pgfpathcurveto{\pgfqpoint{2.813910in}{1.001739in}}{\pgfqpoint{2.819209in}{0.999543in}}{\pgfqpoint{2.824734in}{0.999543in}}%
\pgfpathclose%
\pgfusepath{stroke,fill}%
\end{pgfscope}%
\begin{pgfscope}%
\pgfpathrectangle{\pgfqpoint{0.562500in}{0.275000in}}{\pgfqpoint{3.487500in}{1.925000in}}%
\pgfusepath{clip}%
\pgfsetbuttcap%
\pgfsetroundjoin%
\definecolor{currentfill}{rgb}{0.000000,0.000000,0.000000}%
\pgfsetfillcolor{currentfill}%
\pgfsetlinewidth{1.003750pt}%
\definecolor{currentstroke}{rgb}{0.000000,0.000000,0.000000}%
\pgfsetstrokecolor{currentstroke}%
\pgfsetdash{}{0pt}%
\pgfpathmoveto{\pgfqpoint{2.824734in}{0.852845in}}%
\pgfpathcurveto{\pgfqpoint{2.830260in}{0.852845in}}{\pgfqpoint{2.835559in}{0.855040in}}{\pgfqpoint{2.839466in}{0.858947in}}%
\pgfpathcurveto{\pgfqpoint{2.843373in}{0.862854in}}{\pgfqpoint{2.845568in}{0.868153in}}{\pgfqpoint{2.845568in}{0.873678in}}%
\pgfpathcurveto{\pgfqpoint{2.845568in}{0.879203in}}{\pgfqpoint{2.843373in}{0.884503in}}{\pgfqpoint{2.839466in}{0.888410in}}%
\pgfpathcurveto{\pgfqpoint{2.835559in}{0.892317in}}{\pgfqpoint{2.830260in}{0.894512in}}{\pgfqpoint{2.824734in}{0.894512in}}%
\pgfpathcurveto{\pgfqpoint{2.819209in}{0.894512in}}{\pgfqpoint{2.813910in}{0.892317in}}{\pgfqpoint{2.810003in}{0.888410in}}%
\pgfpathcurveto{\pgfqpoint{2.806096in}{0.884503in}}{\pgfqpoint{2.803901in}{0.879203in}}{\pgfqpoint{2.803901in}{0.873678in}}%
\pgfpathcurveto{\pgfqpoint{2.803901in}{0.868153in}}{\pgfqpoint{2.806096in}{0.862854in}}{\pgfqpoint{2.810003in}{0.858947in}}%
\pgfpathcurveto{\pgfqpoint{2.813910in}{0.855040in}}{\pgfqpoint{2.819209in}{0.852845in}}{\pgfqpoint{2.824734in}{0.852845in}}%
\pgfpathclose%
\pgfusepath{stroke,fill}%
\end{pgfscope}%
\begin{pgfscope}%
\pgfpathrectangle{\pgfqpoint{0.562500in}{0.275000in}}{\pgfqpoint{3.487500in}{1.925000in}}%
\pgfusepath{clip}%
\pgfsetbuttcap%
\pgfsetroundjoin%
\definecolor{currentfill}{rgb}{0.000000,0.000000,0.000000}%
\pgfsetfillcolor{currentfill}%
\pgfsetlinewidth{1.003750pt}%
\definecolor{currentstroke}{rgb}{0.000000,0.000000,0.000000}%
\pgfsetstrokecolor{currentstroke}%
\pgfsetdash{}{0pt}%
\pgfpathmoveto{\pgfqpoint{2.824734in}{0.832611in}}%
\pgfpathcurveto{\pgfqpoint{2.830260in}{0.832611in}}{\pgfqpoint{2.835559in}{0.834806in}}{\pgfqpoint{2.839466in}{0.838713in}}%
\pgfpathcurveto{\pgfqpoint{2.843373in}{0.842620in}}{\pgfqpoint{2.845568in}{0.847919in}}{\pgfqpoint{2.845568in}{0.853444in}}%
\pgfpathcurveto{\pgfqpoint{2.845568in}{0.858969in}}{\pgfqpoint{2.843373in}{0.864269in}}{\pgfqpoint{2.839466in}{0.868175in}}%
\pgfpathcurveto{\pgfqpoint{2.835559in}{0.872082in}}{\pgfqpoint{2.830260in}{0.874277in}}{\pgfqpoint{2.824734in}{0.874277in}}%
\pgfpathcurveto{\pgfqpoint{2.819209in}{0.874277in}}{\pgfqpoint{2.813910in}{0.872082in}}{\pgfqpoint{2.810003in}{0.868175in}}%
\pgfpathcurveto{\pgfqpoint{2.806096in}{0.864269in}}{\pgfqpoint{2.803901in}{0.858969in}}{\pgfqpoint{2.803901in}{0.853444in}}%
\pgfpathcurveto{\pgfqpoint{2.803901in}{0.847919in}}{\pgfqpoint{2.806096in}{0.842620in}}{\pgfqpoint{2.810003in}{0.838713in}}%
\pgfpathcurveto{\pgfqpoint{2.813910in}{0.834806in}}{\pgfqpoint{2.819209in}{0.832611in}}{\pgfqpoint{2.824734in}{0.832611in}}%
\pgfpathclose%
\pgfusepath{stroke,fill}%
\end{pgfscope}%
\begin{pgfscope}%
\pgfpathrectangle{\pgfqpoint{0.562500in}{0.275000in}}{\pgfqpoint{3.487500in}{1.925000in}}%
\pgfusepath{clip}%
\pgfsetbuttcap%
\pgfsetroundjoin%
\definecolor{currentfill}{rgb}{0.000000,0.000000,0.000000}%
\pgfsetfillcolor{currentfill}%
\pgfsetlinewidth{1.003750pt}%
\definecolor{currentstroke}{rgb}{0.000000,0.000000,0.000000}%
\pgfsetstrokecolor{currentstroke}%
\pgfsetdash{}{0pt}%
\pgfpathmoveto{\pgfqpoint{2.824734in}{0.893314in}}%
\pgfpathcurveto{\pgfqpoint{2.830260in}{0.893314in}}{\pgfqpoint{2.835559in}{0.895509in}}{\pgfqpoint{2.839466in}{0.899415in}}%
\pgfpathcurveto{\pgfqpoint{2.843373in}{0.903322in}}{\pgfqpoint{2.845568in}{0.908622in}}{\pgfqpoint{2.845568in}{0.914147in}}%
\pgfpathcurveto{\pgfqpoint{2.845568in}{0.919672in}}{\pgfqpoint{2.843373in}{0.924971in}}{\pgfqpoint{2.839466in}{0.928878in}}%
\pgfpathcurveto{\pgfqpoint{2.835559in}{0.932785in}}{\pgfqpoint{2.830260in}{0.934980in}}{\pgfqpoint{2.824734in}{0.934980in}}%
\pgfpathcurveto{\pgfqpoint{2.819209in}{0.934980in}}{\pgfqpoint{2.813910in}{0.932785in}}{\pgfqpoint{2.810003in}{0.928878in}}%
\pgfpathcurveto{\pgfqpoint{2.806096in}{0.924971in}}{\pgfqpoint{2.803901in}{0.919672in}}{\pgfqpoint{2.803901in}{0.914147in}}%
\pgfpathcurveto{\pgfqpoint{2.803901in}{0.908622in}}{\pgfqpoint{2.806096in}{0.903322in}}{\pgfqpoint{2.810003in}{0.899415in}}%
\pgfpathcurveto{\pgfqpoint{2.813910in}{0.895509in}}{\pgfqpoint{2.819209in}{0.893314in}}{\pgfqpoint{2.824734in}{0.893314in}}%
\pgfpathclose%
\pgfusepath{stroke,fill}%
\end{pgfscope}%
\begin{pgfscope}%
\pgfpathrectangle{\pgfqpoint{0.562500in}{0.275000in}}{\pgfqpoint{3.487500in}{1.925000in}}%
\pgfusepath{clip}%
\pgfsetbuttcap%
\pgfsetroundjoin%
\definecolor{currentfill}{rgb}{0.000000,0.000000,0.000000}%
\pgfsetfillcolor{currentfill}%
\pgfsetlinewidth{1.003750pt}%
\definecolor{currentstroke}{rgb}{0.000000,0.000000,0.000000}%
\pgfsetstrokecolor{currentstroke}%
\pgfsetdash{}{0pt}%
\pgfpathmoveto{\pgfqpoint{2.824734in}{0.842728in}}%
\pgfpathcurveto{\pgfqpoint{2.830260in}{0.842728in}}{\pgfqpoint{2.835559in}{0.844923in}}{\pgfqpoint{2.839466in}{0.848830in}}%
\pgfpathcurveto{\pgfqpoint{2.843373in}{0.852737in}}{\pgfqpoint{2.845568in}{0.858036in}}{\pgfqpoint{2.845568in}{0.863561in}}%
\pgfpathcurveto{\pgfqpoint{2.845568in}{0.869086in}}{\pgfqpoint{2.843373in}{0.874386in}}{\pgfqpoint{2.839466in}{0.878293in}}%
\pgfpathcurveto{\pgfqpoint{2.835559in}{0.882199in}}{\pgfqpoint{2.830260in}{0.884395in}}{\pgfqpoint{2.824734in}{0.884395in}}%
\pgfpathcurveto{\pgfqpoint{2.819209in}{0.884395in}}{\pgfqpoint{2.813910in}{0.882199in}}{\pgfqpoint{2.810003in}{0.878293in}}%
\pgfpathcurveto{\pgfqpoint{2.806096in}{0.874386in}}{\pgfqpoint{2.803901in}{0.869086in}}{\pgfqpoint{2.803901in}{0.863561in}}%
\pgfpathcurveto{\pgfqpoint{2.803901in}{0.858036in}}{\pgfqpoint{2.806096in}{0.852737in}}{\pgfqpoint{2.810003in}{0.848830in}}%
\pgfpathcurveto{\pgfqpoint{2.813910in}{0.844923in}}{\pgfqpoint{2.819209in}{0.842728in}}{\pgfqpoint{2.824734in}{0.842728in}}%
\pgfpathclose%
\pgfusepath{stroke,fill}%
\end{pgfscope}%
\begin{pgfscope}%
\pgfpathrectangle{\pgfqpoint{0.562500in}{0.275000in}}{\pgfqpoint{3.487500in}{1.925000in}}%
\pgfusepath{clip}%
\pgfsetbuttcap%
\pgfsetroundjoin%
\definecolor{currentfill}{rgb}{0.000000,0.000000,0.000000}%
\pgfsetfillcolor{currentfill}%
\pgfsetlinewidth{1.003750pt}%
\definecolor{currentstroke}{rgb}{0.000000,0.000000,0.000000}%
\pgfsetstrokecolor{currentstroke}%
\pgfsetdash{}{0pt}%
\pgfpathmoveto{\pgfqpoint{2.824734in}{0.802259in}}%
\pgfpathcurveto{\pgfqpoint{2.830260in}{0.802259in}}{\pgfqpoint{2.835559in}{0.804454in}}{\pgfqpoint{2.839466in}{0.808361in}}%
\pgfpathcurveto{\pgfqpoint{2.843373in}{0.812268in}}{\pgfqpoint{2.845568in}{0.817568in}}{\pgfqpoint{2.845568in}{0.823093in}}%
\pgfpathcurveto{\pgfqpoint{2.845568in}{0.828618in}}{\pgfqpoint{2.843373in}{0.833917in}}{\pgfqpoint{2.839466in}{0.837824in}}%
\pgfpathcurveto{\pgfqpoint{2.835559in}{0.841731in}}{\pgfqpoint{2.830260in}{0.843926in}}{\pgfqpoint{2.824734in}{0.843926in}}%
\pgfpathcurveto{\pgfqpoint{2.819209in}{0.843926in}}{\pgfqpoint{2.813910in}{0.841731in}}{\pgfqpoint{2.810003in}{0.837824in}}%
\pgfpathcurveto{\pgfqpoint{2.806096in}{0.833917in}}{\pgfqpoint{2.803901in}{0.828618in}}{\pgfqpoint{2.803901in}{0.823093in}}%
\pgfpathcurveto{\pgfqpoint{2.803901in}{0.817568in}}{\pgfqpoint{2.806096in}{0.812268in}}{\pgfqpoint{2.810003in}{0.808361in}}%
\pgfpathcurveto{\pgfqpoint{2.813910in}{0.804454in}}{\pgfqpoint{2.819209in}{0.802259in}}{\pgfqpoint{2.824734in}{0.802259in}}%
\pgfpathclose%
\pgfusepath{stroke,fill}%
\end{pgfscope}%
\begin{pgfscope}%
\pgfpathrectangle{\pgfqpoint{0.562500in}{0.275000in}}{\pgfqpoint{3.487500in}{1.925000in}}%
\pgfusepath{clip}%
\pgfsetbuttcap%
\pgfsetroundjoin%
\definecolor{currentfill}{rgb}{0.000000,0.000000,0.000000}%
\pgfsetfillcolor{currentfill}%
\pgfsetlinewidth{1.003750pt}%
\definecolor{currentstroke}{rgb}{0.000000,0.000000,0.000000}%
\pgfsetstrokecolor{currentstroke}%
\pgfsetdash{}{0pt}%
\pgfpathmoveto{\pgfqpoint{3.876477in}{1.682450in}}%
\pgfpathcurveto{\pgfqpoint{3.882002in}{1.682450in}}{\pgfqpoint{3.887302in}{1.684645in}}{\pgfqpoint{3.891209in}{1.688552in}}%
\pgfpathcurveto{\pgfqpoint{3.895115in}{1.692459in}}{\pgfqpoint{3.897311in}{1.697758in}}{\pgfqpoint{3.897311in}{1.703283in}}%
\pgfpathcurveto{\pgfqpoint{3.897311in}{1.708808in}}{\pgfqpoint{3.895115in}{1.714108in}}{\pgfqpoint{3.891209in}{1.718015in}}%
\pgfpathcurveto{\pgfqpoint{3.887302in}{1.721921in}}{\pgfqpoint{3.882002in}{1.724117in}}{\pgfqpoint{3.876477in}{1.724117in}}%
\pgfpathcurveto{\pgfqpoint{3.870952in}{1.724117in}}{\pgfqpoint{3.865653in}{1.721921in}}{\pgfqpoint{3.861746in}{1.718015in}}%
\pgfpathcurveto{\pgfqpoint{3.857839in}{1.714108in}}{\pgfqpoint{3.855644in}{1.708808in}}{\pgfqpoint{3.855644in}{1.703283in}}%
\pgfpathcurveto{\pgfqpoint{3.855644in}{1.697758in}}{\pgfqpoint{3.857839in}{1.692459in}}{\pgfqpoint{3.861746in}{1.688552in}}%
\pgfpathcurveto{\pgfqpoint{3.865653in}{1.684645in}}{\pgfqpoint{3.870952in}{1.682450in}}{\pgfqpoint{3.876477in}{1.682450in}}%
\pgfpathclose%
\pgfusepath{stroke,fill}%
\end{pgfscope}%
\begin{pgfscope}%
\pgfpathrectangle{\pgfqpoint{0.562500in}{0.275000in}}{\pgfqpoint{3.487500in}{1.925000in}}%
\pgfusepath{clip}%
\pgfsetbuttcap%
\pgfsetroundjoin%
\definecolor{currentfill}{rgb}{0.000000,0.000000,0.000000}%
\pgfsetfillcolor{currentfill}%
\pgfsetlinewidth{1.003750pt}%
\definecolor{currentstroke}{rgb}{0.000000,0.000000,0.000000}%
\pgfsetstrokecolor{currentstroke}%
\pgfsetdash{}{0pt}%
\pgfpathmoveto{\pgfqpoint{3.876477in}{1.636923in}}%
\pgfpathcurveto{\pgfqpoint{3.882002in}{1.636923in}}{\pgfqpoint{3.887302in}{1.639118in}}{\pgfqpoint{3.891209in}{1.643025in}}%
\pgfpathcurveto{\pgfqpoint{3.895115in}{1.646932in}}{\pgfqpoint{3.897311in}{1.652231in}}{\pgfqpoint{3.897311in}{1.657756in}}%
\pgfpathcurveto{\pgfqpoint{3.897311in}{1.663281in}}{\pgfqpoint{3.895115in}{1.668581in}}{\pgfqpoint{3.891209in}{1.672487in}}%
\pgfpathcurveto{\pgfqpoint{3.887302in}{1.676394in}}{\pgfqpoint{3.882002in}{1.678589in}}{\pgfqpoint{3.876477in}{1.678589in}}%
\pgfpathcurveto{\pgfqpoint{3.870952in}{1.678589in}}{\pgfqpoint{3.865653in}{1.676394in}}{\pgfqpoint{3.861746in}{1.672487in}}%
\pgfpathcurveto{\pgfqpoint{3.857839in}{1.668581in}}{\pgfqpoint{3.855644in}{1.663281in}}{\pgfqpoint{3.855644in}{1.657756in}}%
\pgfpathcurveto{\pgfqpoint{3.855644in}{1.652231in}}{\pgfqpoint{3.857839in}{1.646932in}}{\pgfqpoint{3.861746in}{1.643025in}}%
\pgfpathcurveto{\pgfqpoint{3.865653in}{1.639118in}}{\pgfqpoint{3.870952in}{1.636923in}}{\pgfqpoint{3.876477in}{1.636923in}}%
\pgfpathclose%
\pgfusepath{stroke,fill}%
\end{pgfscope}%
\begin{pgfscope}%
\pgfpathrectangle{\pgfqpoint{0.562500in}{0.275000in}}{\pgfqpoint{3.487500in}{1.925000in}}%
\pgfusepath{clip}%
\pgfsetbuttcap%
\pgfsetroundjoin%
\definecolor{currentfill}{rgb}{0.000000,0.000000,0.000000}%
\pgfsetfillcolor{currentfill}%
\pgfsetlinewidth{1.003750pt}%
\definecolor{currentstroke}{rgb}{0.000000,0.000000,0.000000}%
\pgfsetstrokecolor{currentstroke}%
\pgfsetdash{}{0pt}%
\pgfpathmoveto{\pgfqpoint{3.876477in}{1.571161in}}%
\pgfpathcurveto{\pgfqpoint{3.882002in}{1.571161in}}{\pgfqpoint{3.887302in}{1.573357in}}{\pgfqpoint{3.891209in}{1.577263in}}%
\pgfpathcurveto{\pgfqpoint{3.895115in}{1.581170in}}{\pgfqpoint{3.897311in}{1.586470in}}{\pgfqpoint{3.897311in}{1.591995in}}%
\pgfpathcurveto{\pgfqpoint{3.897311in}{1.597520in}}{\pgfqpoint{3.895115in}{1.602819in}}{\pgfqpoint{3.891209in}{1.606726in}}%
\pgfpathcurveto{\pgfqpoint{3.887302in}{1.610633in}}{\pgfqpoint{3.882002in}{1.612828in}}{\pgfqpoint{3.876477in}{1.612828in}}%
\pgfpathcurveto{\pgfqpoint{3.870952in}{1.612828in}}{\pgfqpoint{3.865653in}{1.610633in}}{\pgfqpoint{3.861746in}{1.606726in}}%
\pgfpathcurveto{\pgfqpoint{3.857839in}{1.602819in}}{\pgfqpoint{3.855644in}{1.597520in}}{\pgfqpoint{3.855644in}{1.591995in}}%
\pgfpathcurveto{\pgfqpoint{3.855644in}{1.586470in}}{\pgfqpoint{3.857839in}{1.581170in}}{\pgfqpoint{3.861746in}{1.577263in}}%
\pgfpathcurveto{\pgfqpoint{3.865653in}{1.573357in}}{\pgfqpoint{3.870952in}{1.571161in}}{\pgfqpoint{3.876477in}{1.571161in}}%
\pgfpathclose%
\pgfusepath{stroke,fill}%
\end{pgfscope}%
\begin{pgfscope}%
\pgfpathrectangle{\pgfqpoint{0.562500in}{0.275000in}}{\pgfqpoint{3.487500in}{1.925000in}}%
\pgfusepath{clip}%
\pgfsetbuttcap%
\pgfsetroundjoin%
\definecolor{currentfill}{rgb}{0.000000,0.000000,0.000000}%
\pgfsetfillcolor{currentfill}%
\pgfsetlinewidth{1.003750pt}%
\definecolor{currentstroke}{rgb}{0.000000,0.000000,0.000000}%
\pgfsetstrokecolor{currentstroke}%
\pgfsetdash{}{0pt}%
\pgfpathmoveto{\pgfqpoint{3.876477in}{1.641981in}}%
\pgfpathcurveto{\pgfqpoint{3.882002in}{1.641981in}}{\pgfqpoint{3.887302in}{1.644176in}}{\pgfqpoint{3.891209in}{1.648083in}}%
\pgfpathcurveto{\pgfqpoint{3.895115in}{1.651990in}}{\pgfqpoint{3.897311in}{1.657290in}}{\pgfqpoint{3.897311in}{1.662815in}}%
\pgfpathcurveto{\pgfqpoint{3.897311in}{1.668340in}}{\pgfqpoint{3.895115in}{1.673639in}}{\pgfqpoint{3.891209in}{1.677546in}}%
\pgfpathcurveto{\pgfqpoint{3.887302in}{1.681453in}}{\pgfqpoint{3.882002in}{1.683648in}}{\pgfqpoint{3.876477in}{1.683648in}}%
\pgfpathcurveto{\pgfqpoint{3.870952in}{1.683648in}}{\pgfqpoint{3.865653in}{1.681453in}}{\pgfqpoint{3.861746in}{1.677546in}}%
\pgfpathcurveto{\pgfqpoint{3.857839in}{1.673639in}}{\pgfqpoint{3.855644in}{1.668340in}}{\pgfqpoint{3.855644in}{1.662815in}}%
\pgfpathcurveto{\pgfqpoint{3.855644in}{1.657290in}}{\pgfqpoint{3.857839in}{1.651990in}}{\pgfqpoint{3.861746in}{1.648083in}}%
\pgfpathcurveto{\pgfqpoint{3.865653in}{1.644176in}}{\pgfqpoint{3.870952in}{1.641981in}}{\pgfqpoint{3.876477in}{1.641981in}}%
\pgfpathclose%
\pgfusepath{stroke,fill}%
\end{pgfscope}%
\begin{pgfscope}%
\pgfpathrectangle{\pgfqpoint{0.562500in}{0.275000in}}{\pgfqpoint{3.487500in}{1.925000in}}%
\pgfusepath{clip}%
\pgfsetbuttcap%
\pgfsetroundjoin%
\definecolor{currentfill}{rgb}{0.000000,0.000000,0.000000}%
\pgfsetfillcolor{currentfill}%
\pgfsetlinewidth{1.003750pt}%
\definecolor{currentstroke}{rgb}{0.000000,0.000000,0.000000}%
\pgfsetstrokecolor{currentstroke}%
\pgfsetdash{}{0pt}%
\pgfpathmoveto{\pgfqpoint{3.876477in}{1.641981in}}%
\pgfpathcurveto{\pgfqpoint{3.882002in}{1.641981in}}{\pgfqpoint{3.887302in}{1.644176in}}{\pgfqpoint{3.891209in}{1.648083in}}%
\pgfpathcurveto{\pgfqpoint{3.895115in}{1.651990in}}{\pgfqpoint{3.897311in}{1.657290in}}{\pgfqpoint{3.897311in}{1.662815in}}%
\pgfpathcurveto{\pgfqpoint{3.897311in}{1.668340in}}{\pgfqpoint{3.895115in}{1.673639in}}{\pgfqpoint{3.891209in}{1.677546in}}%
\pgfpathcurveto{\pgfqpoint{3.887302in}{1.681453in}}{\pgfqpoint{3.882002in}{1.683648in}}{\pgfqpoint{3.876477in}{1.683648in}}%
\pgfpathcurveto{\pgfqpoint{3.870952in}{1.683648in}}{\pgfqpoint{3.865653in}{1.681453in}}{\pgfqpoint{3.861746in}{1.677546in}}%
\pgfpathcurveto{\pgfqpoint{3.857839in}{1.673639in}}{\pgfqpoint{3.855644in}{1.668340in}}{\pgfqpoint{3.855644in}{1.662815in}}%
\pgfpathcurveto{\pgfqpoint{3.855644in}{1.657290in}}{\pgfqpoint{3.857839in}{1.651990in}}{\pgfqpoint{3.861746in}{1.648083in}}%
\pgfpathcurveto{\pgfqpoint{3.865653in}{1.644176in}}{\pgfqpoint{3.870952in}{1.641981in}}{\pgfqpoint{3.876477in}{1.641981in}}%
\pgfpathclose%
\pgfusepath{stroke,fill}%
\end{pgfscope}%
\begin{pgfscope}%
\pgfpathrectangle{\pgfqpoint{0.562500in}{0.275000in}}{\pgfqpoint{3.487500in}{1.925000in}}%
\pgfusepath{clip}%
\pgfsetbuttcap%
\pgfsetroundjoin%
\definecolor{currentfill}{rgb}{0.000000,0.000000,0.000000}%
\pgfsetfillcolor{currentfill}%
\pgfsetlinewidth{1.003750pt}%
\definecolor{currentstroke}{rgb}{0.000000,0.000000,0.000000}%
\pgfsetstrokecolor{currentstroke}%
\pgfsetdash{}{0pt}%
\pgfpathmoveto{\pgfqpoint{3.876477in}{1.662216in}}%
\pgfpathcurveto{\pgfqpoint{3.882002in}{1.662216in}}{\pgfqpoint{3.887302in}{1.664411in}}{\pgfqpoint{3.891209in}{1.668318in}}%
\pgfpathcurveto{\pgfqpoint{3.895115in}{1.672224in}}{\pgfqpoint{3.897311in}{1.677524in}}{\pgfqpoint{3.897311in}{1.683049in}}%
\pgfpathcurveto{\pgfqpoint{3.897311in}{1.688574in}}{\pgfqpoint{3.895115in}{1.693874in}}{\pgfqpoint{3.891209in}{1.697780in}}%
\pgfpathcurveto{\pgfqpoint{3.887302in}{1.701687in}}{\pgfqpoint{3.882002in}{1.703882in}}{\pgfqpoint{3.876477in}{1.703882in}}%
\pgfpathcurveto{\pgfqpoint{3.870952in}{1.703882in}}{\pgfqpoint{3.865653in}{1.701687in}}{\pgfqpoint{3.861746in}{1.697780in}}%
\pgfpathcurveto{\pgfqpoint{3.857839in}{1.693874in}}{\pgfqpoint{3.855644in}{1.688574in}}{\pgfqpoint{3.855644in}{1.683049in}}%
\pgfpathcurveto{\pgfqpoint{3.855644in}{1.677524in}}{\pgfqpoint{3.857839in}{1.672224in}}{\pgfqpoint{3.861746in}{1.668318in}}%
\pgfpathcurveto{\pgfqpoint{3.865653in}{1.664411in}}{\pgfqpoint{3.870952in}{1.662216in}}{\pgfqpoint{3.876477in}{1.662216in}}%
\pgfpathclose%
\pgfusepath{stroke,fill}%
\end{pgfscope}%
\begin{pgfscope}%
\pgfpathrectangle{\pgfqpoint{0.562500in}{0.275000in}}{\pgfqpoint{3.487500in}{1.925000in}}%
\pgfusepath{clip}%
\pgfsetbuttcap%
\pgfsetroundjoin%
\definecolor{currentfill}{rgb}{0.000000,0.000000,0.000000}%
\pgfsetfillcolor{currentfill}%
\pgfsetlinewidth{1.003750pt}%
\definecolor{currentstroke}{rgb}{0.000000,0.000000,0.000000}%
\pgfsetstrokecolor{currentstroke}%
\pgfsetdash{}{0pt}%
\pgfpathmoveto{\pgfqpoint{3.876477in}{1.834207in}}%
\pgfpathcurveto{\pgfqpoint{3.882002in}{1.834207in}}{\pgfqpoint{3.887302in}{1.836402in}}{\pgfqpoint{3.891209in}{1.840309in}}%
\pgfpathcurveto{\pgfqpoint{3.895115in}{1.844216in}}{\pgfqpoint{3.897311in}{1.849515in}}{\pgfqpoint{3.897311in}{1.855040in}}%
\pgfpathcurveto{\pgfqpoint{3.897311in}{1.860565in}}{\pgfqpoint{3.895115in}{1.865865in}}{\pgfqpoint{3.891209in}{1.869772in}}%
\pgfpathcurveto{\pgfqpoint{3.887302in}{1.873678in}}{\pgfqpoint{3.882002in}{1.875874in}}{\pgfqpoint{3.876477in}{1.875874in}}%
\pgfpathcurveto{\pgfqpoint{3.870952in}{1.875874in}}{\pgfqpoint{3.865653in}{1.873678in}}{\pgfqpoint{3.861746in}{1.869772in}}%
\pgfpathcurveto{\pgfqpoint{3.857839in}{1.865865in}}{\pgfqpoint{3.855644in}{1.860565in}}{\pgfqpoint{3.855644in}{1.855040in}}%
\pgfpathcurveto{\pgfqpoint{3.855644in}{1.849515in}}{\pgfqpoint{3.857839in}{1.844216in}}{\pgfqpoint{3.861746in}{1.840309in}}%
\pgfpathcurveto{\pgfqpoint{3.865653in}{1.836402in}}{\pgfqpoint{3.870952in}{1.834207in}}{\pgfqpoint{3.876477in}{1.834207in}}%
\pgfpathclose%
\pgfusepath{stroke,fill}%
\end{pgfscope}%
\begin{pgfscope}%
\pgfpathrectangle{\pgfqpoint{0.562500in}{0.275000in}}{\pgfqpoint{3.487500in}{1.925000in}}%
\pgfusepath{clip}%
\pgfsetbuttcap%
\pgfsetroundjoin%
\definecolor{currentfill}{rgb}{0.000000,0.000000,0.000000}%
\pgfsetfillcolor{currentfill}%
\pgfsetlinewidth{1.003750pt}%
\definecolor{currentstroke}{rgb}{0.000000,0.000000,0.000000}%
\pgfsetstrokecolor{currentstroke}%
\pgfsetdash{}{0pt}%
\pgfpathmoveto{\pgfqpoint{3.876477in}{1.652098in}}%
\pgfpathcurveto{\pgfqpoint{3.882002in}{1.652098in}}{\pgfqpoint{3.887302in}{1.654294in}}{\pgfqpoint{3.891209in}{1.658200in}}%
\pgfpathcurveto{\pgfqpoint{3.895115in}{1.662107in}}{\pgfqpoint{3.897311in}{1.667407in}}{\pgfqpoint{3.897311in}{1.672932in}}%
\pgfpathcurveto{\pgfqpoint{3.897311in}{1.678457in}}{\pgfqpoint{3.895115in}{1.683756in}}{\pgfqpoint{3.891209in}{1.687663in}}%
\pgfpathcurveto{\pgfqpoint{3.887302in}{1.691570in}}{\pgfqpoint{3.882002in}{1.693765in}}{\pgfqpoint{3.876477in}{1.693765in}}%
\pgfpathcurveto{\pgfqpoint{3.870952in}{1.693765in}}{\pgfqpoint{3.865653in}{1.691570in}}{\pgfqpoint{3.861746in}{1.687663in}}%
\pgfpathcurveto{\pgfqpoint{3.857839in}{1.683756in}}{\pgfqpoint{3.855644in}{1.678457in}}{\pgfqpoint{3.855644in}{1.672932in}}%
\pgfpathcurveto{\pgfqpoint{3.855644in}{1.667407in}}{\pgfqpoint{3.857839in}{1.662107in}}{\pgfqpoint{3.861746in}{1.658200in}}%
\pgfpathcurveto{\pgfqpoint{3.865653in}{1.654294in}}{\pgfqpoint{3.870952in}{1.652098in}}{\pgfqpoint{3.876477in}{1.652098in}}%
\pgfpathclose%
\pgfusepath{stroke,fill}%
\end{pgfscope}%
\begin{pgfscope}%
\pgfpathrectangle{\pgfqpoint{0.562500in}{0.275000in}}{\pgfqpoint{3.487500in}{1.925000in}}%
\pgfusepath{clip}%
\pgfsetbuttcap%
\pgfsetroundjoin%
\definecolor{currentfill}{rgb}{0.000000,0.000000,0.000000}%
\pgfsetfillcolor{currentfill}%
\pgfsetlinewidth{1.003750pt}%
\definecolor{currentstroke}{rgb}{0.000000,0.000000,0.000000}%
\pgfsetstrokecolor{currentstroke}%
\pgfsetdash{}{0pt}%
\pgfpathmoveto{\pgfqpoint{3.876477in}{1.586337in}}%
\pgfpathcurveto{\pgfqpoint{3.882002in}{1.586337in}}{\pgfqpoint{3.887302in}{1.588532in}}{\pgfqpoint{3.891209in}{1.592439in}}%
\pgfpathcurveto{\pgfqpoint{3.895115in}{1.596346in}}{\pgfqpoint{3.897311in}{1.601645in}}{\pgfqpoint{3.897311in}{1.607170in}}%
\pgfpathcurveto{\pgfqpoint{3.897311in}{1.612695in}}{\pgfqpoint{3.895115in}{1.617995in}}{\pgfqpoint{3.891209in}{1.621902in}}%
\pgfpathcurveto{\pgfqpoint{3.887302in}{1.625809in}}{\pgfqpoint{3.882002in}{1.628004in}}{\pgfqpoint{3.876477in}{1.628004in}}%
\pgfpathcurveto{\pgfqpoint{3.870952in}{1.628004in}}{\pgfqpoint{3.865653in}{1.625809in}}{\pgfqpoint{3.861746in}{1.621902in}}%
\pgfpathcurveto{\pgfqpoint{3.857839in}{1.617995in}}{\pgfqpoint{3.855644in}{1.612695in}}{\pgfqpoint{3.855644in}{1.607170in}}%
\pgfpathcurveto{\pgfqpoint{3.855644in}{1.601645in}}{\pgfqpoint{3.857839in}{1.596346in}}{\pgfqpoint{3.861746in}{1.592439in}}%
\pgfpathcurveto{\pgfqpoint{3.865653in}{1.588532in}}{\pgfqpoint{3.870952in}{1.586337in}}{\pgfqpoint{3.876477in}{1.586337in}}%
\pgfpathclose%
\pgfusepath{stroke,fill}%
\end{pgfscope}%
\begin{pgfscope}%
\pgfpathrectangle{\pgfqpoint{0.562500in}{0.275000in}}{\pgfqpoint{3.487500in}{1.925000in}}%
\pgfusepath{clip}%
\pgfsetbuttcap%
\pgfsetroundjoin%
\definecolor{currentfill}{rgb}{0.000000,0.000000,0.000000}%
\pgfsetfillcolor{currentfill}%
\pgfsetlinewidth{1.003750pt}%
\definecolor{currentstroke}{rgb}{0.000000,0.000000,0.000000}%
\pgfsetstrokecolor{currentstroke}%
\pgfsetdash{}{0pt}%
\pgfpathmoveto{\pgfqpoint{3.876477in}{1.682450in}}%
\pgfpathcurveto{\pgfqpoint{3.882002in}{1.682450in}}{\pgfqpoint{3.887302in}{1.684645in}}{\pgfqpoint{3.891209in}{1.688552in}}%
\pgfpathcurveto{\pgfqpoint{3.895115in}{1.692459in}}{\pgfqpoint{3.897311in}{1.697758in}}{\pgfqpoint{3.897311in}{1.703283in}}%
\pgfpathcurveto{\pgfqpoint{3.897311in}{1.708808in}}{\pgfqpoint{3.895115in}{1.714108in}}{\pgfqpoint{3.891209in}{1.718015in}}%
\pgfpathcurveto{\pgfqpoint{3.887302in}{1.721921in}}{\pgfqpoint{3.882002in}{1.724117in}}{\pgfqpoint{3.876477in}{1.724117in}}%
\pgfpathcurveto{\pgfqpoint{3.870952in}{1.724117in}}{\pgfqpoint{3.865653in}{1.721921in}}{\pgfqpoint{3.861746in}{1.718015in}}%
\pgfpathcurveto{\pgfqpoint{3.857839in}{1.714108in}}{\pgfqpoint{3.855644in}{1.708808in}}{\pgfqpoint{3.855644in}{1.703283in}}%
\pgfpathcurveto{\pgfqpoint{3.855644in}{1.697758in}}{\pgfqpoint{3.857839in}{1.692459in}}{\pgfqpoint{3.861746in}{1.688552in}}%
\pgfpathcurveto{\pgfqpoint{3.865653in}{1.684645in}}{\pgfqpoint{3.870952in}{1.682450in}}{\pgfqpoint{3.876477in}{1.682450in}}%
\pgfpathclose%
\pgfusepath{stroke,fill}%
\end{pgfscope}%
\begin{pgfscope}%
\pgfpathrectangle{\pgfqpoint{0.562500in}{0.275000in}}{\pgfqpoint{3.487500in}{1.925000in}}%
\pgfusepath{clip}%
\pgfsetbuttcap%
\pgfsetroundjoin%
\definecolor{currentfill}{rgb}{0.000000,0.000000,0.000000}%
\pgfsetfillcolor{currentfill}%
\pgfsetlinewidth{1.003750pt}%
\definecolor{currentstroke}{rgb}{0.000000,0.000000,0.000000}%
\pgfsetstrokecolor{currentstroke}%
\pgfsetdash{}{0pt}%
\pgfpathmoveto{\pgfqpoint{3.876477in}{1.738094in}}%
\pgfpathcurveto{\pgfqpoint{3.882002in}{1.738094in}}{\pgfqpoint{3.887302in}{1.740289in}}{\pgfqpoint{3.891209in}{1.744196in}}%
\pgfpathcurveto{\pgfqpoint{3.895115in}{1.748103in}}{\pgfqpoint{3.897311in}{1.753402in}}{\pgfqpoint{3.897311in}{1.758927in}}%
\pgfpathcurveto{\pgfqpoint{3.897311in}{1.764452in}}{\pgfqpoint{3.895115in}{1.769752in}}{\pgfqpoint{3.891209in}{1.773659in}}%
\pgfpathcurveto{\pgfqpoint{3.887302in}{1.777566in}}{\pgfqpoint{3.882002in}{1.779761in}}{\pgfqpoint{3.876477in}{1.779761in}}%
\pgfpathcurveto{\pgfqpoint{3.870952in}{1.779761in}}{\pgfqpoint{3.865653in}{1.777566in}}{\pgfqpoint{3.861746in}{1.773659in}}%
\pgfpathcurveto{\pgfqpoint{3.857839in}{1.769752in}}{\pgfqpoint{3.855644in}{1.764452in}}{\pgfqpoint{3.855644in}{1.758927in}}%
\pgfpathcurveto{\pgfqpoint{3.855644in}{1.753402in}}{\pgfqpoint{3.857839in}{1.748103in}}{\pgfqpoint{3.861746in}{1.744196in}}%
\pgfpathcurveto{\pgfqpoint{3.865653in}{1.740289in}}{\pgfqpoint{3.870952in}{1.738094in}}{\pgfqpoint{3.876477in}{1.738094in}}%
\pgfpathclose%
\pgfusepath{stroke,fill}%
\end{pgfscope}%
\begin{pgfscope}%
\pgfpathrectangle{\pgfqpoint{0.562500in}{0.275000in}}{\pgfqpoint{3.487500in}{1.925000in}}%
\pgfusepath{clip}%
\pgfsetbuttcap%
\pgfsetroundjoin%
\definecolor{currentfill}{rgb}{0.000000,0.000000,0.000000}%
\pgfsetfillcolor{currentfill}%
\pgfsetlinewidth{1.003750pt}%
\definecolor{currentstroke}{rgb}{0.000000,0.000000,0.000000}%
\pgfsetstrokecolor{currentstroke}%
\pgfsetdash{}{0pt}%
\pgfpathmoveto{\pgfqpoint{3.876477in}{1.550927in}}%
\pgfpathcurveto{\pgfqpoint{3.882002in}{1.550927in}}{\pgfqpoint{3.887302in}{1.553122in}}{\pgfqpoint{3.891209in}{1.557029in}}%
\pgfpathcurveto{\pgfqpoint{3.895115in}{1.560936in}}{\pgfqpoint{3.897311in}{1.566235in}}{\pgfqpoint{3.897311in}{1.571760in}}%
\pgfpathcurveto{\pgfqpoint{3.897311in}{1.577286in}}{\pgfqpoint{3.895115in}{1.582585in}}{\pgfqpoint{3.891209in}{1.586492in}}%
\pgfpathcurveto{\pgfqpoint{3.887302in}{1.590399in}}{\pgfqpoint{3.882002in}{1.592594in}}{\pgfqpoint{3.876477in}{1.592594in}}%
\pgfpathcurveto{\pgfqpoint{3.870952in}{1.592594in}}{\pgfqpoint{3.865653in}{1.590399in}}{\pgfqpoint{3.861746in}{1.586492in}}%
\pgfpathcurveto{\pgfqpoint{3.857839in}{1.582585in}}{\pgfqpoint{3.855644in}{1.577286in}}{\pgfqpoint{3.855644in}{1.571760in}}%
\pgfpathcurveto{\pgfqpoint{3.855644in}{1.566235in}}{\pgfqpoint{3.857839in}{1.560936in}}{\pgfqpoint{3.861746in}{1.557029in}}%
\pgfpathcurveto{\pgfqpoint{3.865653in}{1.553122in}}{\pgfqpoint{3.870952in}{1.550927in}}{\pgfqpoint{3.876477in}{1.550927in}}%
\pgfpathclose%
\pgfusepath{stroke,fill}%
\end{pgfscope}%
\begin{pgfscope}%
\pgfpathrectangle{\pgfqpoint{0.562500in}{0.275000in}}{\pgfqpoint{3.487500in}{1.925000in}}%
\pgfusepath{clip}%
\pgfsetbuttcap%
\pgfsetroundjoin%
\definecolor{currentfill}{rgb}{0.000000,0.000000,0.000000}%
\pgfsetfillcolor{currentfill}%
\pgfsetlinewidth{1.003750pt}%
\definecolor{currentstroke}{rgb}{0.000000,0.000000,0.000000}%
\pgfsetstrokecolor{currentstroke}%
\pgfsetdash{}{0pt}%
\pgfpathmoveto{\pgfqpoint{3.876477in}{1.773504in}}%
\pgfpathcurveto{\pgfqpoint{3.882002in}{1.773504in}}{\pgfqpoint{3.887302in}{1.775699in}}{\pgfqpoint{3.891209in}{1.779606in}}%
\pgfpathcurveto{\pgfqpoint{3.895115in}{1.783513in}}{\pgfqpoint{3.897311in}{1.788812in}}{\pgfqpoint{3.897311in}{1.794337in}}%
\pgfpathcurveto{\pgfqpoint{3.897311in}{1.799862in}}{\pgfqpoint{3.895115in}{1.805162in}}{\pgfqpoint{3.891209in}{1.809069in}}%
\pgfpathcurveto{\pgfqpoint{3.887302in}{1.812976in}}{\pgfqpoint{3.882002in}{1.815171in}}{\pgfqpoint{3.876477in}{1.815171in}}%
\pgfpathcurveto{\pgfqpoint{3.870952in}{1.815171in}}{\pgfqpoint{3.865653in}{1.812976in}}{\pgfqpoint{3.861746in}{1.809069in}}%
\pgfpathcurveto{\pgfqpoint{3.857839in}{1.805162in}}{\pgfqpoint{3.855644in}{1.799862in}}{\pgfqpoint{3.855644in}{1.794337in}}%
\pgfpathcurveto{\pgfqpoint{3.855644in}{1.788812in}}{\pgfqpoint{3.857839in}{1.783513in}}{\pgfqpoint{3.861746in}{1.779606in}}%
\pgfpathcurveto{\pgfqpoint{3.865653in}{1.775699in}}{\pgfqpoint{3.870952in}{1.773504in}}{\pgfqpoint{3.876477in}{1.773504in}}%
\pgfpathclose%
\pgfusepath{stroke,fill}%
\end{pgfscope}%
\begin{pgfscope}%
\pgfpathrectangle{\pgfqpoint{0.562500in}{0.275000in}}{\pgfqpoint{3.487500in}{1.925000in}}%
\pgfusepath{clip}%
\pgfsetbuttcap%
\pgfsetroundjoin%
\definecolor{currentfill}{rgb}{0.000000,0.000000,0.000000}%
\pgfsetfillcolor{currentfill}%
\pgfsetlinewidth{1.003750pt}%
\definecolor{currentstroke}{rgb}{0.000000,0.000000,0.000000}%
\pgfsetstrokecolor{currentstroke}%
\pgfsetdash{}{0pt}%
\pgfpathmoveto{\pgfqpoint{3.876477in}{1.808914in}}%
\pgfpathcurveto{\pgfqpoint{3.882002in}{1.808914in}}{\pgfqpoint{3.887302in}{1.811109in}}{\pgfqpoint{3.891209in}{1.815016in}}%
\pgfpathcurveto{\pgfqpoint{3.895115in}{1.818923in}}{\pgfqpoint{3.897311in}{1.824222in}}{\pgfqpoint{3.897311in}{1.829747in}}%
\pgfpathcurveto{\pgfqpoint{3.897311in}{1.835272in}}{\pgfqpoint{3.895115in}{1.840572in}}{\pgfqpoint{3.891209in}{1.844479in}}%
\pgfpathcurveto{\pgfqpoint{3.887302in}{1.848386in}}{\pgfqpoint{3.882002in}{1.850581in}}{\pgfqpoint{3.876477in}{1.850581in}}%
\pgfpathcurveto{\pgfqpoint{3.870952in}{1.850581in}}{\pgfqpoint{3.865653in}{1.848386in}}{\pgfqpoint{3.861746in}{1.844479in}}%
\pgfpathcurveto{\pgfqpoint{3.857839in}{1.840572in}}{\pgfqpoint{3.855644in}{1.835272in}}{\pgfqpoint{3.855644in}{1.829747in}}%
\pgfpathcurveto{\pgfqpoint{3.855644in}{1.824222in}}{\pgfqpoint{3.857839in}{1.818923in}}{\pgfqpoint{3.861746in}{1.815016in}}%
\pgfpathcurveto{\pgfqpoint{3.865653in}{1.811109in}}{\pgfqpoint{3.870952in}{1.808914in}}{\pgfqpoint{3.876477in}{1.808914in}}%
\pgfpathclose%
\pgfusepath{stroke,fill}%
\end{pgfscope}%
\begin{pgfscope}%
\pgfpathrectangle{\pgfqpoint{0.562500in}{0.275000in}}{\pgfqpoint{3.487500in}{1.925000in}}%
\pgfusepath{clip}%
\pgfsetbuttcap%
\pgfsetroundjoin%
\definecolor{currentfill}{rgb}{0.000000,0.000000,0.000000}%
\pgfsetfillcolor{currentfill}%
\pgfsetlinewidth{1.003750pt}%
\definecolor{currentstroke}{rgb}{0.000000,0.000000,0.000000}%
\pgfsetstrokecolor{currentstroke}%
\pgfsetdash{}{0pt}%
\pgfpathmoveto{\pgfqpoint{3.876477in}{1.733036in}}%
\pgfpathcurveto{\pgfqpoint{3.882002in}{1.733036in}}{\pgfqpoint{3.887302in}{1.735231in}}{\pgfqpoint{3.891209in}{1.739137in}}%
\pgfpathcurveto{\pgfqpoint{3.895115in}{1.743044in}}{\pgfqpoint{3.897311in}{1.748344in}}{\pgfqpoint{3.897311in}{1.753869in}}%
\pgfpathcurveto{\pgfqpoint{3.897311in}{1.759394in}}{\pgfqpoint{3.895115in}{1.764693in}}{\pgfqpoint{3.891209in}{1.768600in}}%
\pgfpathcurveto{\pgfqpoint{3.887302in}{1.772507in}}{\pgfqpoint{3.882002in}{1.774702in}}{\pgfqpoint{3.876477in}{1.774702in}}%
\pgfpathcurveto{\pgfqpoint{3.870952in}{1.774702in}}{\pgfqpoint{3.865653in}{1.772507in}}{\pgfqpoint{3.861746in}{1.768600in}}%
\pgfpathcurveto{\pgfqpoint{3.857839in}{1.764693in}}{\pgfqpoint{3.855644in}{1.759394in}}{\pgfqpoint{3.855644in}{1.753869in}}%
\pgfpathcurveto{\pgfqpoint{3.855644in}{1.748344in}}{\pgfqpoint{3.857839in}{1.743044in}}{\pgfqpoint{3.861746in}{1.739137in}}%
\pgfpathcurveto{\pgfqpoint{3.865653in}{1.735231in}}{\pgfqpoint{3.870952in}{1.733036in}}{\pgfqpoint{3.876477in}{1.733036in}}%
\pgfpathclose%
\pgfusepath{stroke,fill}%
\end{pgfscope}%
\begin{pgfscope}%
\pgfpathrectangle{\pgfqpoint{0.562500in}{0.275000in}}{\pgfqpoint{3.487500in}{1.925000in}}%
\pgfusepath{clip}%
\pgfsetbuttcap%
\pgfsetroundjoin%
\definecolor{currentfill}{rgb}{0.000000,0.000000,0.000000}%
\pgfsetfillcolor{currentfill}%
\pgfsetlinewidth{1.003750pt}%
\definecolor{currentstroke}{rgb}{0.000000,0.000000,0.000000}%
\pgfsetstrokecolor{currentstroke}%
\pgfsetdash{}{0pt}%
\pgfpathmoveto{\pgfqpoint{3.876477in}{1.960671in}}%
\pgfpathcurveto{\pgfqpoint{3.882002in}{1.960671in}}{\pgfqpoint{3.887302in}{1.962866in}}{\pgfqpoint{3.891209in}{1.966773in}}%
\pgfpathcurveto{\pgfqpoint{3.895115in}{1.970680in}}{\pgfqpoint{3.897311in}{1.975979in}}{\pgfqpoint{3.897311in}{1.981504in}}%
\pgfpathcurveto{\pgfqpoint{3.897311in}{1.987029in}}{\pgfqpoint{3.895115in}{1.992329in}}{\pgfqpoint{3.891209in}{1.996236in}}%
\pgfpathcurveto{\pgfqpoint{3.887302in}{2.000143in}}{\pgfqpoint{3.882002in}{2.002338in}}{\pgfqpoint{3.876477in}{2.002338in}}%
\pgfpathcurveto{\pgfqpoint{3.870952in}{2.002338in}}{\pgfqpoint{3.865653in}{2.000143in}}{\pgfqpoint{3.861746in}{1.996236in}}%
\pgfpathcurveto{\pgfqpoint{3.857839in}{1.992329in}}{\pgfqpoint{3.855644in}{1.987029in}}{\pgfqpoint{3.855644in}{1.981504in}}%
\pgfpathcurveto{\pgfqpoint{3.855644in}{1.975979in}}{\pgfqpoint{3.857839in}{1.970680in}}{\pgfqpoint{3.861746in}{1.966773in}}%
\pgfpathcurveto{\pgfqpoint{3.865653in}{1.962866in}}{\pgfqpoint{3.870952in}{1.960671in}}{\pgfqpoint{3.876477in}{1.960671in}}%
\pgfpathclose%
\pgfusepath{stroke,fill}%
\end{pgfscope}%
\begin{pgfscope}%
\pgfpathrectangle{\pgfqpoint{0.562500in}{0.275000in}}{\pgfqpoint{3.487500in}{1.925000in}}%
\pgfusepath{clip}%
\pgfsetbuttcap%
\pgfsetroundjoin%
\definecolor{currentfill}{rgb}{0.000000,0.000000,0.000000}%
\pgfsetfillcolor{currentfill}%
\pgfsetlinewidth{1.003750pt}%
\definecolor{currentstroke}{rgb}{0.000000,0.000000,0.000000}%
\pgfsetstrokecolor{currentstroke}%
\pgfsetdash{}{0pt}%
\pgfpathmoveto{\pgfqpoint{3.876477in}{1.601513in}}%
\pgfpathcurveto{\pgfqpoint{3.882002in}{1.601513in}}{\pgfqpoint{3.887302in}{1.603708in}}{\pgfqpoint{3.891209in}{1.607615in}}%
\pgfpathcurveto{\pgfqpoint{3.895115in}{1.611522in}}{\pgfqpoint{3.897311in}{1.616821in}}{\pgfqpoint{3.897311in}{1.622346in}}%
\pgfpathcurveto{\pgfqpoint{3.897311in}{1.627871in}}{\pgfqpoint{3.895115in}{1.633171in}}{\pgfqpoint{3.891209in}{1.637078in}}%
\pgfpathcurveto{\pgfqpoint{3.887302in}{1.640984in}}{\pgfqpoint{3.882002in}{1.643179in}}{\pgfqpoint{3.876477in}{1.643179in}}%
\pgfpathcurveto{\pgfqpoint{3.870952in}{1.643179in}}{\pgfqpoint{3.865653in}{1.640984in}}{\pgfqpoint{3.861746in}{1.637078in}}%
\pgfpathcurveto{\pgfqpoint{3.857839in}{1.633171in}}{\pgfqpoint{3.855644in}{1.627871in}}{\pgfqpoint{3.855644in}{1.622346in}}%
\pgfpathcurveto{\pgfqpoint{3.855644in}{1.616821in}}{\pgfqpoint{3.857839in}{1.611522in}}{\pgfqpoint{3.861746in}{1.607615in}}%
\pgfpathcurveto{\pgfqpoint{3.865653in}{1.603708in}}{\pgfqpoint{3.870952in}{1.601513in}}{\pgfqpoint{3.876477in}{1.601513in}}%
\pgfpathclose%
\pgfusepath{stroke,fill}%
\end{pgfscope}%
\begin{pgfscope}%
\pgfpathrectangle{\pgfqpoint{0.562500in}{0.275000in}}{\pgfqpoint{3.487500in}{1.925000in}}%
\pgfusepath{clip}%
\pgfsetbuttcap%
\pgfsetroundjoin%
\definecolor{currentfill}{rgb}{0.000000,0.000000,0.000000}%
\pgfsetfillcolor{currentfill}%
\pgfsetlinewidth{1.003750pt}%
\definecolor{currentstroke}{rgb}{0.000000,0.000000,0.000000}%
\pgfsetstrokecolor{currentstroke}%
\pgfsetdash{}{0pt}%
\pgfpathmoveto{\pgfqpoint{3.876477in}{1.975847in}}%
\pgfpathcurveto{\pgfqpoint{3.882002in}{1.975847in}}{\pgfqpoint{3.887302in}{1.978042in}}{\pgfqpoint{3.891209in}{1.981949in}}%
\pgfpathcurveto{\pgfqpoint{3.895115in}{1.985855in}}{\pgfqpoint{3.897311in}{1.991155in}}{\pgfqpoint{3.897311in}{1.996680in}}%
\pgfpathcurveto{\pgfqpoint{3.897311in}{2.002205in}}{\pgfqpoint{3.895115in}{2.007505in}}{\pgfqpoint{3.891209in}{2.011411in}}%
\pgfpathcurveto{\pgfqpoint{3.887302in}{2.015318in}}{\pgfqpoint{3.882002in}{2.017513in}}{\pgfqpoint{3.876477in}{2.017513in}}%
\pgfpathcurveto{\pgfqpoint{3.870952in}{2.017513in}}{\pgfqpoint{3.865653in}{2.015318in}}{\pgfqpoint{3.861746in}{2.011411in}}%
\pgfpathcurveto{\pgfqpoint{3.857839in}{2.007505in}}{\pgfqpoint{3.855644in}{2.002205in}}{\pgfqpoint{3.855644in}{1.996680in}}%
\pgfpathcurveto{\pgfqpoint{3.855644in}{1.991155in}}{\pgfqpoint{3.857839in}{1.985855in}}{\pgfqpoint{3.861746in}{1.981949in}}%
\pgfpathcurveto{\pgfqpoint{3.865653in}{1.978042in}}{\pgfqpoint{3.870952in}{1.975847in}}{\pgfqpoint{3.876477in}{1.975847in}}%
\pgfpathclose%
\pgfusepath{stroke,fill}%
\end{pgfscope}%
\begin{pgfscope}%
\pgfpathrectangle{\pgfqpoint{0.562500in}{0.275000in}}{\pgfqpoint{3.487500in}{1.925000in}}%
\pgfusepath{clip}%
\pgfsetbuttcap%
\pgfsetroundjoin%
\definecolor{currentfill}{rgb}{0.000000,0.000000,0.000000}%
\pgfsetfillcolor{currentfill}%
\pgfsetlinewidth{1.003750pt}%
\definecolor{currentstroke}{rgb}{0.000000,0.000000,0.000000}%
\pgfsetstrokecolor{currentstroke}%
\pgfsetdash{}{0pt}%
\pgfpathmoveto{\pgfqpoint{3.876477in}{1.707743in}}%
\pgfpathcurveto{\pgfqpoint{3.882002in}{1.707743in}}{\pgfqpoint{3.887302in}{1.709938in}}{\pgfqpoint{3.891209in}{1.713845in}}%
\pgfpathcurveto{\pgfqpoint{3.895115in}{1.717751in}}{\pgfqpoint{3.897311in}{1.723051in}}{\pgfqpoint{3.897311in}{1.728576in}}%
\pgfpathcurveto{\pgfqpoint{3.897311in}{1.734101in}}{\pgfqpoint{3.895115in}{1.739401in}}{\pgfqpoint{3.891209in}{1.743307in}}%
\pgfpathcurveto{\pgfqpoint{3.887302in}{1.747214in}}{\pgfqpoint{3.882002in}{1.749409in}}{\pgfqpoint{3.876477in}{1.749409in}}%
\pgfpathcurveto{\pgfqpoint{3.870952in}{1.749409in}}{\pgfqpoint{3.865653in}{1.747214in}}{\pgfqpoint{3.861746in}{1.743307in}}%
\pgfpathcurveto{\pgfqpoint{3.857839in}{1.739401in}}{\pgfqpoint{3.855644in}{1.734101in}}{\pgfqpoint{3.855644in}{1.728576in}}%
\pgfpathcurveto{\pgfqpoint{3.855644in}{1.723051in}}{\pgfqpoint{3.857839in}{1.717751in}}{\pgfqpoint{3.861746in}{1.713845in}}%
\pgfpathcurveto{\pgfqpoint{3.865653in}{1.709938in}}{\pgfqpoint{3.870952in}{1.707743in}}{\pgfqpoint{3.876477in}{1.707743in}}%
\pgfpathclose%
\pgfusepath{stroke,fill}%
\end{pgfscope}%
\begin{pgfscope}%
\pgfpathrectangle{\pgfqpoint{0.562500in}{0.275000in}}{\pgfqpoint{3.487500in}{1.925000in}}%
\pgfusepath{clip}%
\pgfsetbuttcap%
\pgfsetroundjoin%
\definecolor{currentfill}{rgb}{0.000000,0.000000,0.000000}%
\pgfsetfillcolor{currentfill}%
\pgfsetlinewidth{1.003750pt}%
\definecolor{currentstroke}{rgb}{0.000000,0.000000,0.000000}%
\pgfsetstrokecolor{currentstroke}%
\pgfsetdash{}{0pt}%
\pgfpathmoveto{\pgfqpoint{3.876477in}{1.743153in}}%
\pgfpathcurveto{\pgfqpoint{3.882002in}{1.743153in}}{\pgfqpoint{3.887302in}{1.745348in}}{\pgfqpoint{3.891209in}{1.749255in}}%
\pgfpathcurveto{\pgfqpoint{3.895115in}{1.753161in}}{\pgfqpoint{3.897311in}{1.758461in}}{\pgfqpoint{3.897311in}{1.763986in}}%
\pgfpathcurveto{\pgfqpoint{3.897311in}{1.769511in}}{\pgfqpoint{3.895115in}{1.774811in}}{\pgfqpoint{3.891209in}{1.778717in}}%
\pgfpathcurveto{\pgfqpoint{3.887302in}{1.782624in}}{\pgfqpoint{3.882002in}{1.784819in}}{\pgfqpoint{3.876477in}{1.784819in}}%
\pgfpathcurveto{\pgfqpoint{3.870952in}{1.784819in}}{\pgfqpoint{3.865653in}{1.782624in}}{\pgfqpoint{3.861746in}{1.778717in}}%
\pgfpathcurveto{\pgfqpoint{3.857839in}{1.774811in}}{\pgfqpoint{3.855644in}{1.769511in}}{\pgfqpoint{3.855644in}{1.763986in}}%
\pgfpathcurveto{\pgfqpoint{3.855644in}{1.758461in}}{\pgfqpoint{3.857839in}{1.753161in}}{\pgfqpoint{3.861746in}{1.749255in}}%
\pgfpathcurveto{\pgfqpoint{3.865653in}{1.745348in}}{\pgfqpoint{3.870952in}{1.743153in}}{\pgfqpoint{3.876477in}{1.743153in}}%
\pgfpathclose%
\pgfusepath{stroke,fill}%
\end{pgfscope}%
\begin{pgfscope}%
\pgfpathrectangle{\pgfqpoint{0.562500in}{0.275000in}}{\pgfqpoint{3.487500in}{1.925000in}}%
\pgfusepath{clip}%
\pgfsetbuttcap%
\pgfsetroundjoin%
\definecolor{currentfill}{rgb}{0.000000,0.000000,0.000000}%
\pgfsetfillcolor{currentfill}%
\pgfsetlinewidth{1.003750pt}%
\definecolor{currentstroke}{rgb}{0.000000,0.000000,0.000000}%
\pgfsetstrokecolor{currentstroke}%
\pgfsetdash{}{0pt}%
\pgfpathmoveto{\pgfqpoint{3.876477in}{1.621747in}}%
\pgfpathcurveto{\pgfqpoint{3.882002in}{1.621747in}}{\pgfqpoint{3.887302in}{1.623942in}}{\pgfqpoint{3.891209in}{1.627849in}}%
\pgfpathcurveto{\pgfqpoint{3.895115in}{1.631756in}}{\pgfqpoint{3.897311in}{1.637055in}}{\pgfqpoint{3.897311in}{1.642580in}}%
\pgfpathcurveto{\pgfqpoint{3.897311in}{1.648105in}}{\pgfqpoint{3.895115in}{1.653405in}}{\pgfqpoint{3.891209in}{1.657312in}}%
\pgfpathcurveto{\pgfqpoint{3.887302in}{1.661219in}}{\pgfqpoint{3.882002in}{1.663414in}}{\pgfqpoint{3.876477in}{1.663414in}}%
\pgfpathcurveto{\pgfqpoint{3.870952in}{1.663414in}}{\pgfqpoint{3.865653in}{1.661219in}}{\pgfqpoint{3.861746in}{1.657312in}}%
\pgfpathcurveto{\pgfqpoint{3.857839in}{1.653405in}}{\pgfqpoint{3.855644in}{1.648105in}}{\pgfqpoint{3.855644in}{1.642580in}}%
\pgfpathcurveto{\pgfqpoint{3.855644in}{1.637055in}}{\pgfqpoint{3.857839in}{1.631756in}}{\pgfqpoint{3.861746in}{1.627849in}}%
\pgfpathcurveto{\pgfqpoint{3.865653in}{1.623942in}}{\pgfqpoint{3.870952in}{1.621747in}}{\pgfqpoint{3.876477in}{1.621747in}}%
\pgfpathclose%
\pgfusepath{stroke,fill}%
\end{pgfscope}%
\begin{pgfscope}%
\pgfpathrectangle{\pgfqpoint{0.562500in}{0.275000in}}{\pgfqpoint{3.487500in}{1.925000in}}%
\pgfusepath{clip}%
\pgfsetbuttcap%
\pgfsetroundjoin%
\definecolor{currentfill}{rgb}{0.000000,0.000000,0.000000}%
\pgfsetfillcolor{currentfill}%
\pgfsetlinewidth{1.003750pt}%
\definecolor{currentstroke}{rgb}{0.000000,0.000000,0.000000}%
\pgfsetstrokecolor{currentstroke}%
\pgfsetdash{}{0pt}%
\pgfpathmoveto{\pgfqpoint{3.876477in}{1.561044in}}%
\pgfpathcurveto{\pgfqpoint{3.882002in}{1.561044in}}{\pgfqpoint{3.887302in}{1.563239in}}{\pgfqpoint{3.891209in}{1.567146in}}%
\pgfpathcurveto{\pgfqpoint{3.895115in}{1.571053in}}{\pgfqpoint{3.897311in}{1.576353in}}{\pgfqpoint{3.897311in}{1.581878in}}%
\pgfpathcurveto{\pgfqpoint{3.897311in}{1.587403in}}{\pgfqpoint{3.895115in}{1.592702in}}{\pgfqpoint{3.891209in}{1.596609in}}%
\pgfpathcurveto{\pgfqpoint{3.887302in}{1.600516in}}{\pgfqpoint{3.882002in}{1.602711in}}{\pgfqpoint{3.876477in}{1.602711in}}%
\pgfpathcurveto{\pgfqpoint{3.870952in}{1.602711in}}{\pgfqpoint{3.865653in}{1.600516in}}{\pgfqpoint{3.861746in}{1.596609in}}%
\pgfpathcurveto{\pgfqpoint{3.857839in}{1.592702in}}{\pgfqpoint{3.855644in}{1.587403in}}{\pgfqpoint{3.855644in}{1.581878in}}%
\pgfpathcurveto{\pgfqpoint{3.855644in}{1.576353in}}{\pgfqpoint{3.857839in}{1.571053in}}{\pgfqpoint{3.861746in}{1.567146in}}%
\pgfpathcurveto{\pgfqpoint{3.865653in}{1.563239in}}{\pgfqpoint{3.870952in}{1.561044in}}{\pgfqpoint{3.876477in}{1.561044in}}%
\pgfpathclose%
\pgfusepath{stroke,fill}%
\end{pgfscope}%
\begin{pgfscope}%
\pgfpathrectangle{\pgfqpoint{0.562500in}{0.275000in}}{\pgfqpoint{3.487500in}{1.925000in}}%
\pgfusepath{clip}%
\pgfsetbuttcap%
\pgfsetroundjoin%
\definecolor{currentfill}{rgb}{0.000000,0.000000,0.000000}%
\pgfsetfillcolor{currentfill}%
\pgfsetlinewidth{1.003750pt}%
\definecolor{currentstroke}{rgb}{0.000000,0.000000,0.000000}%
\pgfsetstrokecolor{currentstroke}%
\pgfsetdash{}{0pt}%
\pgfpathmoveto{\pgfqpoint{3.876477in}{1.748211in}}%
\pgfpathcurveto{\pgfqpoint{3.882002in}{1.748211in}}{\pgfqpoint{3.887302in}{1.750406in}}{\pgfqpoint{3.891209in}{1.754313in}}%
\pgfpathcurveto{\pgfqpoint{3.895115in}{1.758220in}}{\pgfqpoint{3.897311in}{1.763519in}}{\pgfqpoint{3.897311in}{1.769045in}}%
\pgfpathcurveto{\pgfqpoint{3.897311in}{1.774570in}}{\pgfqpoint{3.895115in}{1.779869in}}{\pgfqpoint{3.891209in}{1.783776in}}%
\pgfpathcurveto{\pgfqpoint{3.887302in}{1.787683in}}{\pgfqpoint{3.882002in}{1.789878in}}{\pgfqpoint{3.876477in}{1.789878in}}%
\pgfpathcurveto{\pgfqpoint{3.870952in}{1.789878in}}{\pgfqpoint{3.865653in}{1.787683in}}{\pgfqpoint{3.861746in}{1.783776in}}%
\pgfpathcurveto{\pgfqpoint{3.857839in}{1.779869in}}{\pgfqpoint{3.855644in}{1.774570in}}{\pgfqpoint{3.855644in}{1.769045in}}%
\pgfpathcurveto{\pgfqpoint{3.855644in}{1.763519in}}{\pgfqpoint{3.857839in}{1.758220in}}{\pgfqpoint{3.861746in}{1.754313in}}%
\pgfpathcurveto{\pgfqpoint{3.865653in}{1.750406in}}{\pgfqpoint{3.870952in}{1.748211in}}{\pgfqpoint{3.876477in}{1.748211in}}%
\pgfpathclose%
\pgfusepath{stroke,fill}%
\end{pgfscope}%
\begin{pgfscope}%
\pgfpathrectangle{\pgfqpoint{0.562500in}{0.275000in}}{\pgfqpoint{3.487500in}{1.925000in}}%
\pgfusepath{clip}%
\pgfsetbuttcap%
\pgfsetroundjoin%
\definecolor{currentfill}{rgb}{0.000000,0.000000,0.000000}%
\pgfsetfillcolor{currentfill}%
\pgfsetlinewidth{1.003750pt}%
\definecolor{currentstroke}{rgb}{0.000000,0.000000,0.000000}%
\pgfsetstrokecolor{currentstroke}%
\pgfsetdash{}{0pt}%
\pgfpathmoveto{\pgfqpoint{3.876477in}{1.738094in}}%
\pgfpathcurveto{\pgfqpoint{3.882002in}{1.738094in}}{\pgfqpoint{3.887302in}{1.740289in}}{\pgfqpoint{3.891209in}{1.744196in}}%
\pgfpathcurveto{\pgfqpoint{3.895115in}{1.748103in}}{\pgfqpoint{3.897311in}{1.753402in}}{\pgfqpoint{3.897311in}{1.758927in}}%
\pgfpathcurveto{\pgfqpoint{3.897311in}{1.764452in}}{\pgfqpoint{3.895115in}{1.769752in}}{\pgfqpoint{3.891209in}{1.773659in}}%
\pgfpathcurveto{\pgfqpoint{3.887302in}{1.777566in}}{\pgfqpoint{3.882002in}{1.779761in}}{\pgfqpoint{3.876477in}{1.779761in}}%
\pgfpathcurveto{\pgfqpoint{3.870952in}{1.779761in}}{\pgfqpoint{3.865653in}{1.777566in}}{\pgfqpoint{3.861746in}{1.773659in}}%
\pgfpathcurveto{\pgfqpoint{3.857839in}{1.769752in}}{\pgfqpoint{3.855644in}{1.764452in}}{\pgfqpoint{3.855644in}{1.758927in}}%
\pgfpathcurveto{\pgfqpoint{3.855644in}{1.753402in}}{\pgfqpoint{3.857839in}{1.748103in}}{\pgfqpoint{3.861746in}{1.744196in}}%
\pgfpathcurveto{\pgfqpoint{3.865653in}{1.740289in}}{\pgfqpoint{3.870952in}{1.738094in}}{\pgfqpoint{3.876477in}{1.738094in}}%
\pgfpathclose%
\pgfusepath{stroke,fill}%
\end{pgfscope}%
\begin{pgfscope}%
\pgfpathrectangle{\pgfqpoint{0.562500in}{0.275000in}}{\pgfqpoint{3.487500in}{1.925000in}}%
\pgfusepath{clip}%
\pgfsetbuttcap%
\pgfsetroundjoin%
\definecolor{currentfill}{rgb}{0.000000,0.000000,0.000000}%
\pgfsetfillcolor{currentfill}%
\pgfsetlinewidth{1.003750pt}%
\definecolor{currentstroke}{rgb}{0.000000,0.000000,0.000000}%
\pgfsetstrokecolor{currentstroke}%
\pgfsetdash{}{0pt}%
\pgfpathmoveto{\pgfqpoint{3.876477in}{1.611630in}}%
\pgfpathcurveto{\pgfqpoint{3.882002in}{1.611630in}}{\pgfqpoint{3.887302in}{1.613825in}}{\pgfqpoint{3.891209in}{1.617732in}}%
\pgfpathcurveto{\pgfqpoint{3.895115in}{1.621639in}}{\pgfqpoint{3.897311in}{1.626938in}}{\pgfqpoint{3.897311in}{1.632463in}}%
\pgfpathcurveto{\pgfqpoint{3.897311in}{1.637988in}}{\pgfqpoint{3.895115in}{1.643288in}}{\pgfqpoint{3.891209in}{1.647195in}}%
\pgfpathcurveto{\pgfqpoint{3.887302in}{1.651101in}}{\pgfqpoint{3.882002in}{1.653297in}}{\pgfqpoint{3.876477in}{1.653297in}}%
\pgfpathcurveto{\pgfqpoint{3.870952in}{1.653297in}}{\pgfqpoint{3.865653in}{1.651101in}}{\pgfqpoint{3.861746in}{1.647195in}}%
\pgfpathcurveto{\pgfqpoint{3.857839in}{1.643288in}}{\pgfqpoint{3.855644in}{1.637988in}}{\pgfqpoint{3.855644in}{1.632463in}}%
\pgfpathcurveto{\pgfqpoint{3.855644in}{1.626938in}}{\pgfqpoint{3.857839in}{1.621639in}}{\pgfqpoint{3.861746in}{1.617732in}}%
\pgfpathcurveto{\pgfqpoint{3.865653in}{1.613825in}}{\pgfqpoint{3.870952in}{1.611630in}}{\pgfqpoint{3.876477in}{1.611630in}}%
\pgfpathclose%
\pgfusepath{stroke,fill}%
\end{pgfscope}%
\begin{pgfscope}%
\pgfpathrectangle{\pgfqpoint{0.562500in}{0.275000in}}{\pgfqpoint{3.487500in}{1.925000in}}%
\pgfusepath{clip}%
\pgfsetbuttcap%
\pgfsetroundjoin%
\definecolor{currentfill}{rgb}{0.000000,0.000000,0.000000}%
\pgfsetfillcolor{currentfill}%
\pgfsetlinewidth{1.003750pt}%
\definecolor{currentstroke}{rgb}{0.000000,0.000000,0.000000}%
\pgfsetstrokecolor{currentstroke}%
\pgfsetdash{}{0pt}%
\pgfpathmoveto{\pgfqpoint{3.876477in}{1.515517in}}%
\pgfpathcurveto{\pgfqpoint{3.882002in}{1.515517in}}{\pgfqpoint{3.887302in}{1.517712in}}{\pgfqpoint{3.891209in}{1.521619in}}%
\pgfpathcurveto{\pgfqpoint{3.895115in}{1.525526in}}{\pgfqpoint{3.897311in}{1.530825in}}{\pgfqpoint{3.897311in}{1.536351in}}%
\pgfpathcurveto{\pgfqpoint{3.897311in}{1.541876in}}{\pgfqpoint{3.895115in}{1.547175in}}{\pgfqpoint{3.891209in}{1.551082in}}%
\pgfpathcurveto{\pgfqpoint{3.887302in}{1.554989in}}{\pgfqpoint{3.882002in}{1.557184in}}{\pgfqpoint{3.876477in}{1.557184in}}%
\pgfpathcurveto{\pgfqpoint{3.870952in}{1.557184in}}{\pgfqpoint{3.865653in}{1.554989in}}{\pgfqpoint{3.861746in}{1.551082in}}%
\pgfpathcurveto{\pgfqpoint{3.857839in}{1.547175in}}{\pgfqpoint{3.855644in}{1.541876in}}{\pgfqpoint{3.855644in}{1.536351in}}%
\pgfpathcurveto{\pgfqpoint{3.855644in}{1.530825in}}{\pgfqpoint{3.857839in}{1.525526in}}{\pgfqpoint{3.861746in}{1.521619in}}%
\pgfpathcurveto{\pgfqpoint{3.865653in}{1.517712in}}{\pgfqpoint{3.870952in}{1.515517in}}{\pgfqpoint{3.876477in}{1.515517in}}%
\pgfpathclose%
\pgfusepath{stroke,fill}%
\end{pgfscope}%
\begin{pgfscope}%
\pgfpathrectangle{\pgfqpoint{0.562500in}{0.275000in}}{\pgfqpoint{3.487500in}{1.925000in}}%
\pgfusepath{clip}%
\pgfsetbuttcap%
\pgfsetroundjoin%
\definecolor{currentfill}{rgb}{0.000000,0.000000,0.000000}%
\pgfsetfillcolor{currentfill}%
\pgfsetlinewidth{1.003750pt}%
\definecolor{currentstroke}{rgb}{0.000000,0.000000,0.000000}%
\pgfsetstrokecolor{currentstroke}%
\pgfsetdash{}{0pt}%
\pgfpathmoveto{\pgfqpoint{3.876477in}{1.702684in}}%
\pgfpathcurveto{\pgfqpoint{3.882002in}{1.702684in}}{\pgfqpoint{3.887302in}{1.704879in}}{\pgfqpoint{3.891209in}{1.708786in}}%
\pgfpathcurveto{\pgfqpoint{3.895115in}{1.712693in}}{\pgfqpoint{3.897311in}{1.717992in}}{\pgfqpoint{3.897311in}{1.723517in}}%
\pgfpathcurveto{\pgfqpoint{3.897311in}{1.729043in}}{\pgfqpoint{3.895115in}{1.734342in}}{\pgfqpoint{3.891209in}{1.738249in}}%
\pgfpathcurveto{\pgfqpoint{3.887302in}{1.742156in}}{\pgfqpoint{3.882002in}{1.744351in}}{\pgfqpoint{3.876477in}{1.744351in}}%
\pgfpathcurveto{\pgfqpoint{3.870952in}{1.744351in}}{\pgfqpoint{3.865653in}{1.742156in}}{\pgfqpoint{3.861746in}{1.738249in}}%
\pgfpathcurveto{\pgfqpoint{3.857839in}{1.734342in}}{\pgfqpoint{3.855644in}{1.729043in}}{\pgfqpoint{3.855644in}{1.723517in}}%
\pgfpathcurveto{\pgfqpoint{3.855644in}{1.717992in}}{\pgfqpoint{3.857839in}{1.712693in}}{\pgfqpoint{3.861746in}{1.708786in}}%
\pgfpathcurveto{\pgfqpoint{3.865653in}{1.704879in}}{\pgfqpoint{3.870952in}{1.702684in}}{\pgfqpoint{3.876477in}{1.702684in}}%
\pgfpathclose%
\pgfusepath{stroke,fill}%
\end{pgfscope}%
\begin{pgfscope}%
\pgfpathrectangle{\pgfqpoint{0.562500in}{0.275000in}}{\pgfqpoint{3.487500in}{1.925000in}}%
\pgfusepath{clip}%
\pgfsetbuttcap%
\pgfsetroundjoin%
\definecolor{currentfill}{rgb}{0.000000,0.000000,0.000000}%
\pgfsetfillcolor{currentfill}%
\pgfsetlinewidth{1.003750pt}%
\definecolor{currentstroke}{rgb}{0.000000,0.000000,0.000000}%
\pgfsetstrokecolor{currentstroke}%
\pgfsetdash{}{0pt}%
\pgfpathmoveto{\pgfqpoint{3.876477in}{1.571161in}}%
\pgfpathcurveto{\pgfqpoint{3.882002in}{1.571161in}}{\pgfqpoint{3.887302in}{1.573357in}}{\pgfqpoint{3.891209in}{1.577263in}}%
\pgfpathcurveto{\pgfqpoint{3.895115in}{1.581170in}}{\pgfqpoint{3.897311in}{1.586470in}}{\pgfqpoint{3.897311in}{1.591995in}}%
\pgfpathcurveto{\pgfqpoint{3.897311in}{1.597520in}}{\pgfqpoint{3.895115in}{1.602819in}}{\pgfqpoint{3.891209in}{1.606726in}}%
\pgfpathcurveto{\pgfqpoint{3.887302in}{1.610633in}}{\pgfqpoint{3.882002in}{1.612828in}}{\pgfqpoint{3.876477in}{1.612828in}}%
\pgfpathcurveto{\pgfqpoint{3.870952in}{1.612828in}}{\pgfqpoint{3.865653in}{1.610633in}}{\pgfqpoint{3.861746in}{1.606726in}}%
\pgfpathcurveto{\pgfqpoint{3.857839in}{1.602819in}}{\pgfqpoint{3.855644in}{1.597520in}}{\pgfqpoint{3.855644in}{1.591995in}}%
\pgfpathcurveto{\pgfqpoint{3.855644in}{1.586470in}}{\pgfqpoint{3.857839in}{1.581170in}}{\pgfqpoint{3.861746in}{1.577263in}}%
\pgfpathcurveto{\pgfqpoint{3.865653in}{1.573357in}}{\pgfqpoint{3.870952in}{1.571161in}}{\pgfqpoint{3.876477in}{1.571161in}}%
\pgfpathclose%
\pgfusepath{stroke,fill}%
\end{pgfscope}%
\begin{pgfscope}%
\pgfpathrectangle{\pgfqpoint{0.562500in}{0.275000in}}{\pgfqpoint{3.487500in}{1.925000in}}%
\pgfusepath{clip}%
\pgfsetbuttcap%
\pgfsetroundjoin%
\definecolor{currentfill}{rgb}{0.000000,0.000000,0.000000}%
\pgfsetfillcolor{currentfill}%
\pgfsetlinewidth{1.003750pt}%
\definecolor{currentstroke}{rgb}{0.000000,0.000000,0.000000}%
\pgfsetstrokecolor{currentstroke}%
\pgfsetdash{}{0pt}%
\pgfpathmoveto{\pgfqpoint{3.876477in}{1.733036in}}%
\pgfpathcurveto{\pgfqpoint{3.882002in}{1.733036in}}{\pgfqpoint{3.887302in}{1.735231in}}{\pgfqpoint{3.891209in}{1.739137in}}%
\pgfpathcurveto{\pgfqpoint{3.895115in}{1.743044in}}{\pgfqpoint{3.897311in}{1.748344in}}{\pgfqpoint{3.897311in}{1.753869in}}%
\pgfpathcurveto{\pgfqpoint{3.897311in}{1.759394in}}{\pgfqpoint{3.895115in}{1.764693in}}{\pgfqpoint{3.891209in}{1.768600in}}%
\pgfpathcurveto{\pgfqpoint{3.887302in}{1.772507in}}{\pgfqpoint{3.882002in}{1.774702in}}{\pgfqpoint{3.876477in}{1.774702in}}%
\pgfpathcurveto{\pgfqpoint{3.870952in}{1.774702in}}{\pgfqpoint{3.865653in}{1.772507in}}{\pgfqpoint{3.861746in}{1.768600in}}%
\pgfpathcurveto{\pgfqpoint{3.857839in}{1.764693in}}{\pgfqpoint{3.855644in}{1.759394in}}{\pgfqpoint{3.855644in}{1.753869in}}%
\pgfpathcurveto{\pgfqpoint{3.855644in}{1.748344in}}{\pgfqpoint{3.857839in}{1.743044in}}{\pgfqpoint{3.861746in}{1.739137in}}%
\pgfpathcurveto{\pgfqpoint{3.865653in}{1.735231in}}{\pgfqpoint{3.870952in}{1.733036in}}{\pgfqpoint{3.876477in}{1.733036in}}%
\pgfpathclose%
\pgfusepath{stroke,fill}%
\end{pgfscope}%
\begin{pgfscope}%
\pgfpathrectangle{\pgfqpoint{0.562500in}{0.275000in}}{\pgfqpoint{3.487500in}{1.925000in}}%
\pgfusepath{clip}%
\pgfsetbuttcap%
\pgfsetroundjoin%
\definecolor{currentfill}{rgb}{0.000000,0.000000,0.000000}%
\pgfsetfillcolor{currentfill}%
\pgfsetlinewidth{1.003750pt}%
\definecolor{currentstroke}{rgb}{0.000000,0.000000,0.000000}%
\pgfsetstrokecolor{currentstroke}%
\pgfsetdash{}{0pt}%
\pgfpathmoveto{\pgfqpoint{3.876477in}{1.738094in}}%
\pgfpathcurveto{\pgfqpoint{3.882002in}{1.738094in}}{\pgfqpoint{3.887302in}{1.740289in}}{\pgfqpoint{3.891209in}{1.744196in}}%
\pgfpathcurveto{\pgfqpoint{3.895115in}{1.748103in}}{\pgfqpoint{3.897311in}{1.753402in}}{\pgfqpoint{3.897311in}{1.758927in}}%
\pgfpathcurveto{\pgfqpoint{3.897311in}{1.764452in}}{\pgfqpoint{3.895115in}{1.769752in}}{\pgfqpoint{3.891209in}{1.773659in}}%
\pgfpathcurveto{\pgfqpoint{3.887302in}{1.777566in}}{\pgfqpoint{3.882002in}{1.779761in}}{\pgfqpoint{3.876477in}{1.779761in}}%
\pgfpathcurveto{\pgfqpoint{3.870952in}{1.779761in}}{\pgfqpoint{3.865653in}{1.777566in}}{\pgfqpoint{3.861746in}{1.773659in}}%
\pgfpathcurveto{\pgfqpoint{3.857839in}{1.769752in}}{\pgfqpoint{3.855644in}{1.764452in}}{\pgfqpoint{3.855644in}{1.758927in}}%
\pgfpathcurveto{\pgfqpoint{3.855644in}{1.753402in}}{\pgfqpoint{3.857839in}{1.748103in}}{\pgfqpoint{3.861746in}{1.744196in}}%
\pgfpathcurveto{\pgfqpoint{3.865653in}{1.740289in}}{\pgfqpoint{3.870952in}{1.738094in}}{\pgfqpoint{3.876477in}{1.738094in}}%
\pgfpathclose%
\pgfusepath{stroke,fill}%
\end{pgfscope}%
\begin{pgfscope}%
\pgfpathrectangle{\pgfqpoint{0.562500in}{0.275000in}}{\pgfqpoint{3.487500in}{1.925000in}}%
\pgfusepath{clip}%
\pgfsetbuttcap%
\pgfsetroundjoin%
\definecolor{currentfill}{rgb}{0.000000,0.000000,0.000000}%
\pgfsetfillcolor{currentfill}%
\pgfsetlinewidth{1.003750pt}%
\definecolor{currentstroke}{rgb}{0.000000,0.000000,0.000000}%
\pgfsetstrokecolor{currentstroke}%
\pgfsetdash{}{0pt}%
\pgfpathmoveto{\pgfqpoint{3.876477in}{1.601513in}}%
\pgfpathcurveto{\pgfqpoint{3.882002in}{1.601513in}}{\pgfqpoint{3.887302in}{1.603708in}}{\pgfqpoint{3.891209in}{1.607615in}}%
\pgfpathcurveto{\pgfqpoint{3.895115in}{1.611522in}}{\pgfqpoint{3.897311in}{1.616821in}}{\pgfqpoint{3.897311in}{1.622346in}}%
\pgfpathcurveto{\pgfqpoint{3.897311in}{1.627871in}}{\pgfqpoint{3.895115in}{1.633171in}}{\pgfqpoint{3.891209in}{1.637078in}}%
\pgfpathcurveto{\pgfqpoint{3.887302in}{1.640984in}}{\pgfqpoint{3.882002in}{1.643179in}}{\pgfqpoint{3.876477in}{1.643179in}}%
\pgfpathcurveto{\pgfqpoint{3.870952in}{1.643179in}}{\pgfqpoint{3.865653in}{1.640984in}}{\pgfqpoint{3.861746in}{1.637078in}}%
\pgfpathcurveto{\pgfqpoint{3.857839in}{1.633171in}}{\pgfqpoint{3.855644in}{1.627871in}}{\pgfqpoint{3.855644in}{1.622346in}}%
\pgfpathcurveto{\pgfqpoint{3.855644in}{1.616821in}}{\pgfqpoint{3.857839in}{1.611522in}}{\pgfqpoint{3.861746in}{1.607615in}}%
\pgfpathcurveto{\pgfqpoint{3.865653in}{1.603708in}}{\pgfqpoint{3.870952in}{1.601513in}}{\pgfqpoint{3.876477in}{1.601513in}}%
\pgfpathclose%
\pgfusepath{stroke,fill}%
\end{pgfscope}%
\begin{pgfscope}%
\pgfpathrectangle{\pgfqpoint{0.562500in}{0.275000in}}{\pgfqpoint{3.487500in}{1.925000in}}%
\pgfusepath{clip}%
\pgfsetbuttcap%
\pgfsetroundjoin%
\definecolor{currentfill}{rgb}{0.000000,0.000000,0.000000}%
\pgfsetfillcolor{currentfill}%
\pgfsetlinewidth{1.003750pt}%
\definecolor{currentstroke}{rgb}{0.000000,0.000000,0.000000}%
\pgfsetstrokecolor{currentstroke}%
\pgfsetdash{}{0pt}%
\pgfpathmoveto{\pgfqpoint{3.876477in}{1.819031in}}%
\pgfpathcurveto{\pgfqpoint{3.882002in}{1.819031in}}{\pgfqpoint{3.887302in}{1.821226in}}{\pgfqpoint{3.891209in}{1.825133in}}%
\pgfpathcurveto{\pgfqpoint{3.895115in}{1.829040in}}{\pgfqpoint{3.897311in}{1.834339in}}{\pgfqpoint{3.897311in}{1.839864in}}%
\pgfpathcurveto{\pgfqpoint{3.897311in}{1.845390in}}{\pgfqpoint{3.895115in}{1.850689in}}{\pgfqpoint{3.891209in}{1.854596in}}%
\pgfpathcurveto{\pgfqpoint{3.887302in}{1.858503in}}{\pgfqpoint{3.882002in}{1.860698in}}{\pgfqpoint{3.876477in}{1.860698in}}%
\pgfpathcurveto{\pgfqpoint{3.870952in}{1.860698in}}{\pgfqpoint{3.865653in}{1.858503in}}{\pgfqpoint{3.861746in}{1.854596in}}%
\pgfpathcurveto{\pgfqpoint{3.857839in}{1.850689in}}{\pgfqpoint{3.855644in}{1.845390in}}{\pgfqpoint{3.855644in}{1.839864in}}%
\pgfpathcurveto{\pgfqpoint{3.855644in}{1.834339in}}{\pgfqpoint{3.857839in}{1.829040in}}{\pgfqpoint{3.861746in}{1.825133in}}%
\pgfpathcurveto{\pgfqpoint{3.865653in}{1.821226in}}{\pgfqpoint{3.870952in}{1.819031in}}{\pgfqpoint{3.876477in}{1.819031in}}%
\pgfpathclose%
\pgfusepath{stroke,fill}%
\end{pgfscope}%
\begin{pgfscope}%
\pgfpathrectangle{\pgfqpoint{0.562500in}{0.275000in}}{\pgfqpoint{3.487500in}{1.925000in}}%
\pgfusepath{clip}%
\pgfsetbuttcap%
\pgfsetroundjoin%
\definecolor{currentfill}{rgb}{0.000000,0.000000,0.000000}%
\pgfsetfillcolor{currentfill}%
\pgfsetlinewidth{1.003750pt}%
\definecolor{currentstroke}{rgb}{0.000000,0.000000,0.000000}%
\pgfsetstrokecolor{currentstroke}%
\pgfsetdash{}{0pt}%
\pgfpathmoveto{\pgfqpoint{3.876477in}{1.768445in}}%
\pgfpathcurveto{\pgfqpoint{3.882002in}{1.768445in}}{\pgfqpoint{3.887302in}{1.770641in}}{\pgfqpoint{3.891209in}{1.774547in}}%
\pgfpathcurveto{\pgfqpoint{3.895115in}{1.778454in}}{\pgfqpoint{3.897311in}{1.783754in}}{\pgfqpoint{3.897311in}{1.789279in}}%
\pgfpathcurveto{\pgfqpoint{3.897311in}{1.794804in}}{\pgfqpoint{3.895115in}{1.800103in}}{\pgfqpoint{3.891209in}{1.804010in}}%
\pgfpathcurveto{\pgfqpoint{3.887302in}{1.807917in}}{\pgfqpoint{3.882002in}{1.810112in}}{\pgfqpoint{3.876477in}{1.810112in}}%
\pgfpathcurveto{\pgfqpoint{3.870952in}{1.810112in}}{\pgfqpoint{3.865653in}{1.807917in}}{\pgfqpoint{3.861746in}{1.804010in}}%
\pgfpathcurveto{\pgfqpoint{3.857839in}{1.800103in}}{\pgfqpoint{3.855644in}{1.794804in}}{\pgfqpoint{3.855644in}{1.789279in}}%
\pgfpathcurveto{\pgfqpoint{3.855644in}{1.783754in}}{\pgfqpoint{3.857839in}{1.778454in}}{\pgfqpoint{3.861746in}{1.774547in}}%
\pgfpathcurveto{\pgfqpoint{3.865653in}{1.770641in}}{\pgfqpoint{3.870952in}{1.768445in}}{\pgfqpoint{3.876477in}{1.768445in}}%
\pgfpathclose%
\pgfusepath{stroke,fill}%
\end{pgfscope}%
\begin{pgfscope}%
\pgfpathrectangle{\pgfqpoint{0.562500in}{0.275000in}}{\pgfqpoint{3.487500in}{1.925000in}}%
\pgfusepath{clip}%
\pgfsetbuttcap%
\pgfsetroundjoin%
\definecolor{currentfill}{rgb}{0.000000,0.000000,0.000000}%
\pgfsetfillcolor{currentfill}%
\pgfsetlinewidth{1.003750pt}%
\definecolor{currentstroke}{rgb}{0.000000,0.000000,0.000000}%
\pgfsetstrokecolor{currentstroke}%
\pgfsetdash{}{0pt}%
\pgfpathmoveto{\pgfqpoint{3.876477in}{1.844324in}}%
\pgfpathcurveto{\pgfqpoint{3.882002in}{1.844324in}}{\pgfqpoint{3.887302in}{1.846519in}}{\pgfqpoint{3.891209in}{1.850426in}}%
\pgfpathcurveto{\pgfqpoint{3.895115in}{1.854333in}}{\pgfqpoint{3.897311in}{1.859632in}}{\pgfqpoint{3.897311in}{1.865157in}}%
\pgfpathcurveto{\pgfqpoint{3.897311in}{1.870682in}}{\pgfqpoint{3.895115in}{1.875982in}}{\pgfqpoint{3.891209in}{1.879889in}}%
\pgfpathcurveto{\pgfqpoint{3.887302in}{1.883796in}}{\pgfqpoint{3.882002in}{1.885991in}}{\pgfqpoint{3.876477in}{1.885991in}}%
\pgfpathcurveto{\pgfqpoint{3.870952in}{1.885991in}}{\pgfqpoint{3.865653in}{1.883796in}}{\pgfqpoint{3.861746in}{1.879889in}}%
\pgfpathcurveto{\pgfqpoint{3.857839in}{1.875982in}}{\pgfqpoint{3.855644in}{1.870682in}}{\pgfqpoint{3.855644in}{1.865157in}}%
\pgfpathcurveto{\pgfqpoint{3.855644in}{1.859632in}}{\pgfqpoint{3.857839in}{1.854333in}}{\pgfqpoint{3.861746in}{1.850426in}}%
\pgfpathcurveto{\pgfqpoint{3.865653in}{1.846519in}}{\pgfqpoint{3.870952in}{1.844324in}}{\pgfqpoint{3.876477in}{1.844324in}}%
\pgfpathclose%
\pgfusepath{stroke,fill}%
\end{pgfscope}%
\begin{pgfscope}%
\pgfpathrectangle{\pgfqpoint{0.562500in}{0.275000in}}{\pgfqpoint{3.487500in}{1.925000in}}%
\pgfusepath{clip}%
\pgfsetbuttcap%
\pgfsetroundjoin%
\definecolor{currentfill}{rgb}{0.000000,0.000000,0.000000}%
\pgfsetfillcolor{currentfill}%
\pgfsetlinewidth{1.003750pt}%
\definecolor{currentstroke}{rgb}{0.000000,0.000000,0.000000}%
\pgfsetstrokecolor{currentstroke}%
\pgfsetdash{}{0pt}%
\pgfpathmoveto{\pgfqpoint{3.876477in}{1.768445in}}%
\pgfpathcurveto{\pgfqpoint{3.882002in}{1.768445in}}{\pgfqpoint{3.887302in}{1.770641in}}{\pgfqpoint{3.891209in}{1.774547in}}%
\pgfpathcurveto{\pgfqpoint{3.895115in}{1.778454in}}{\pgfqpoint{3.897311in}{1.783754in}}{\pgfqpoint{3.897311in}{1.789279in}}%
\pgfpathcurveto{\pgfqpoint{3.897311in}{1.794804in}}{\pgfqpoint{3.895115in}{1.800103in}}{\pgfqpoint{3.891209in}{1.804010in}}%
\pgfpathcurveto{\pgfqpoint{3.887302in}{1.807917in}}{\pgfqpoint{3.882002in}{1.810112in}}{\pgfqpoint{3.876477in}{1.810112in}}%
\pgfpathcurveto{\pgfqpoint{3.870952in}{1.810112in}}{\pgfqpoint{3.865653in}{1.807917in}}{\pgfqpoint{3.861746in}{1.804010in}}%
\pgfpathcurveto{\pgfqpoint{3.857839in}{1.800103in}}{\pgfqpoint{3.855644in}{1.794804in}}{\pgfqpoint{3.855644in}{1.789279in}}%
\pgfpathcurveto{\pgfqpoint{3.855644in}{1.783754in}}{\pgfqpoint{3.857839in}{1.778454in}}{\pgfqpoint{3.861746in}{1.774547in}}%
\pgfpathcurveto{\pgfqpoint{3.865653in}{1.770641in}}{\pgfqpoint{3.870952in}{1.768445in}}{\pgfqpoint{3.876477in}{1.768445in}}%
\pgfpathclose%
\pgfusepath{stroke,fill}%
\end{pgfscope}%
\begin{pgfscope}%
\pgfpathrectangle{\pgfqpoint{0.562500in}{0.275000in}}{\pgfqpoint{3.487500in}{1.925000in}}%
\pgfusepath{clip}%
\pgfsetbuttcap%
\pgfsetroundjoin%
\definecolor{currentfill}{rgb}{0.000000,0.000000,0.000000}%
\pgfsetfillcolor{currentfill}%
\pgfsetlinewidth{1.003750pt}%
\definecolor{currentstroke}{rgb}{0.000000,0.000000,0.000000}%
\pgfsetstrokecolor{currentstroke}%
\pgfsetdash{}{0pt}%
\pgfpathmoveto{\pgfqpoint{3.876477in}{1.641981in}}%
\pgfpathcurveto{\pgfqpoint{3.882002in}{1.641981in}}{\pgfqpoint{3.887302in}{1.644176in}}{\pgfqpoint{3.891209in}{1.648083in}}%
\pgfpathcurveto{\pgfqpoint{3.895115in}{1.651990in}}{\pgfqpoint{3.897311in}{1.657290in}}{\pgfqpoint{3.897311in}{1.662815in}}%
\pgfpathcurveto{\pgfqpoint{3.897311in}{1.668340in}}{\pgfqpoint{3.895115in}{1.673639in}}{\pgfqpoint{3.891209in}{1.677546in}}%
\pgfpathcurveto{\pgfqpoint{3.887302in}{1.681453in}}{\pgfqpoint{3.882002in}{1.683648in}}{\pgfqpoint{3.876477in}{1.683648in}}%
\pgfpathcurveto{\pgfqpoint{3.870952in}{1.683648in}}{\pgfqpoint{3.865653in}{1.681453in}}{\pgfqpoint{3.861746in}{1.677546in}}%
\pgfpathcurveto{\pgfqpoint{3.857839in}{1.673639in}}{\pgfqpoint{3.855644in}{1.668340in}}{\pgfqpoint{3.855644in}{1.662815in}}%
\pgfpathcurveto{\pgfqpoint{3.855644in}{1.657290in}}{\pgfqpoint{3.857839in}{1.651990in}}{\pgfqpoint{3.861746in}{1.648083in}}%
\pgfpathcurveto{\pgfqpoint{3.865653in}{1.644176in}}{\pgfqpoint{3.870952in}{1.641981in}}{\pgfqpoint{3.876477in}{1.641981in}}%
\pgfpathclose%
\pgfusepath{stroke,fill}%
\end{pgfscope}%
\begin{pgfscope}%
\pgfpathrectangle{\pgfqpoint{0.562500in}{0.275000in}}{\pgfqpoint{3.487500in}{1.925000in}}%
\pgfusepath{clip}%
\pgfsetbuttcap%
\pgfsetroundjoin%
\definecolor{currentfill}{rgb}{0.000000,0.000000,0.000000}%
\pgfsetfillcolor{currentfill}%
\pgfsetlinewidth{1.003750pt}%
\definecolor{currentstroke}{rgb}{0.000000,0.000000,0.000000}%
\pgfsetstrokecolor{currentstroke}%
\pgfsetdash{}{0pt}%
\pgfpathmoveto{\pgfqpoint{3.876477in}{1.733036in}}%
\pgfpathcurveto{\pgfqpoint{3.882002in}{1.733036in}}{\pgfqpoint{3.887302in}{1.735231in}}{\pgfqpoint{3.891209in}{1.739137in}}%
\pgfpathcurveto{\pgfqpoint{3.895115in}{1.743044in}}{\pgfqpoint{3.897311in}{1.748344in}}{\pgfqpoint{3.897311in}{1.753869in}}%
\pgfpathcurveto{\pgfqpoint{3.897311in}{1.759394in}}{\pgfqpoint{3.895115in}{1.764693in}}{\pgfqpoint{3.891209in}{1.768600in}}%
\pgfpathcurveto{\pgfqpoint{3.887302in}{1.772507in}}{\pgfqpoint{3.882002in}{1.774702in}}{\pgfqpoint{3.876477in}{1.774702in}}%
\pgfpathcurveto{\pgfqpoint{3.870952in}{1.774702in}}{\pgfqpoint{3.865653in}{1.772507in}}{\pgfqpoint{3.861746in}{1.768600in}}%
\pgfpathcurveto{\pgfqpoint{3.857839in}{1.764693in}}{\pgfqpoint{3.855644in}{1.759394in}}{\pgfqpoint{3.855644in}{1.753869in}}%
\pgfpathcurveto{\pgfqpoint{3.855644in}{1.748344in}}{\pgfqpoint{3.857839in}{1.743044in}}{\pgfqpoint{3.861746in}{1.739137in}}%
\pgfpathcurveto{\pgfqpoint{3.865653in}{1.735231in}}{\pgfqpoint{3.870952in}{1.733036in}}{\pgfqpoint{3.876477in}{1.733036in}}%
\pgfpathclose%
\pgfusepath{stroke,fill}%
\end{pgfscope}%
\begin{pgfscope}%
\pgfpathrectangle{\pgfqpoint{0.562500in}{0.275000in}}{\pgfqpoint{3.487500in}{1.925000in}}%
\pgfusepath{clip}%
\pgfsetbuttcap%
\pgfsetroundjoin%
\definecolor{currentfill}{rgb}{0.000000,0.000000,0.000000}%
\pgfsetfillcolor{currentfill}%
\pgfsetlinewidth{1.003750pt}%
\definecolor{currentstroke}{rgb}{0.000000,0.000000,0.000000}%
\pgfsetstrokecolor{currentstroke}%
\pgfsetdash{}{0pt}%
\pgfpathmoveto{\pgfqpoint{3.876477in}{1.550927in}}%
\pgfpathcurveto{\pgfqpoint{3.882002in}{1.550927in}}{\pgfqpoint{3.887302in}{1.553122in}}{\pgfqpoint{3.891209in}{1.557029in}}%
\pgfpathcurveto{\pgfqpoint{3.895115in}{1.560936in}}{\pgfqpoint{3.897311in}{1.566235in}}{\pgfqpoint{3.897311in}{1.571760in}}%
\pgfpathcurveto{\pgfqpoint{3.897311in}{1.577286in}}{\pgfqpoint{3.895115in}{1.582585in}}{\pgfqpoint{3.891209in}{1.586492in}}%
\pgfpathcurveto{\pgfqpoint{3.887302in}{1.590399in}}{\pgfqpoint{3.882002in}{1.592594in}}{\pgfqpoint{3.876477in}{1.592594in}}%
\pgfpathcurveto{\pgfqpoint{3.870952in}{1.592594in}}{\pgfqpoint{3.865653in}{1.590399in}}{\pgfqpoint{3.861746in}{1.586492in}}%
\pgfpathcurveto{\pgfqpoint{3.857839in}{1.582585in}}{\pgfqpoint{3.855644in}{1.577286in}}{\pgfqpoint{3.855644in}{1.571760in}}%
\pgfpathcurveto{\pgfqpoint{3.855644in}{1.566235in}}{\pgfqpoint{3.857839in}{1.560936in}}{\pgfqpoint{3.861746in}{1.557029in}}%
\pgfpathcurveto{\pgfqpoint{3.865653in}{1.553122in}}{\pgfqpoint{3.870952in}{1.550927in}}{\pgfqpoint{3.876477in}{1.550927in}}%
\pgfpathclose%
\pgfusepath{stroke,fill}%
\end{pgfscope}%
\begin{pgfscope}%
\pgfpathrectangle{\pgfqpoint{0.562500in}{0.275000in}}{\pgfqpoint{3.487500in}{1.925000in}}%
\pgfusepath{clip}%
\pgfsetbuttcap%
\pgfsetroundjoin%
\definecolor{currentfill}{rgb}{0.000000,0.000000,0.000000}%
\pgfsetfillcolor{currentfill}%
\pgfsetlinewidth{1.003750pt}%
\definecolor{currentstroke}{rgb}{0.000000,0.000000,0.000000}%
\pgfsetstrokecolor{currentstroke}%
\pgfsetdash{}{0pt}%
\pgfpathmoveto{\pgfqpoint{3.876477in}{1.631864in}}%
\pgfpathcurveto{\pgfqpoint{3.882002in}{1.631864in}}{\pgfqpoint{3.887302in}{1.634059in}}{\pgfqpoint{3.891209in}{1.637966in}}%
\pgfpathcurveto{\pgfqpoint{3.895115in}{1.641873in}}{\pgfqpoint{3.897311in}{1.647172in}}{\pgfqpoint{3.897311in}{1.652698in}}%
\pgfpathcurveto{\pgfqpoint{3.897311in}{1.658223in}}{\pgfqpoint{3.895115in}{1.663522in}}{\pgfqpoint{3.891209in}{1.667429in}}%
\pgfpathcurveto{\pgfqpoint{3.887302in}{1.671336in}}{\pgfqpoint{3.882002in}{1.673531in}}{\pgfqpoint{3.876477in}{1.673531in}}%
\pgfpathcurveto{\pgfqpoint{3.870952in}{1.673531in}}{\pgfqpoint{3.865653in}{1.671336in}}{\pgfqpoint{3.861746in}{1.667429in}}%
\pgfpathcurveto{\pgfqpoint{3.857839in}{1.663522in}}{\pgfqpoint{3.855644in}{1.658223in}}{\pgfqpoint{3.855644in}{1.652698in}}%
\pgfpathcurveto{\pgfqpoint{3.855644in}{1.647172in}}{\pgfqpoint{3.857839in}{1.641873in}}{\pgfqpoint{3.861746in}{1.637966in}}%
\pgfpathcurveto{\pgfqpoint{3.865653in}{1.634059in}}{\pgfqpoint{3.870952in}{1.631864in}}{\pgfqpoint{3.876477in}{1.631864in}}%
\pgfpathclose%
\pgfusepath{stroke,fill}%
\end{pgfscope}%
\begin{pgfscope}%
\pgfpathrectangle{\pgfqpoint{0.562500in}{0.275000in}}{\pgfqpoint{3.487500in}{1.925000in}}%
\pgfusepath{clip}%
\pgfsetbuttcap%
\pgfsetroundjoin%
\definecolor{currentfill}{rgb}{0.000000,0.000000,0.000000}%
\pgfsetfillcolor{currentfill}%
\pgfsetlinewidth{1.003750pt}%
\definecolor{currentstroke}{rgb}{0.000000,0.000000,0.000000}%
\pgfsetstrokecolor{currentstroke}%
\pgfsetdash{}{0pt}%
\pgfpathmoveto{\pgfqpoint{3.876477in}{1.657157in}}%
\pgfpathcurveto{\pgfqpoint{3.882002in}{1.657157in}}{\pgfqpoint{3.887302in}{1.659352in}}{\pgfqpoint{3.891209in}{1.663259in}}%
\pgfpathcurveto{\pgfqpoint{3.895115in}{1.667166in}}{\pgfqpoint{3.897311in}{1.672465in}}{\pgfqpoint{3.897311in}{1.677990in}}%
\pgfpathcurveto{\pgfqpoint{3.897311in}{1.683515in}}{\pgfqpoint{3.895115in}{1.688815in}}{\pgfqpoint{3.891209in}{1.692722in}}%
\pgfpathcurveto{\pgfqpoint{3.887302in}{1.696629in}}{\pgfqpoint{3.882002in}{1.698824in}}{\pgfqpoint{3.876477in}{1.698824in}}%
\pgfpathcurveto{\pgfqpoint{3.870952in}{1.698824in}}{\pgfqpoint{3.865653in}{1.696629in}}{\pgfqpoint{3.861746in}{1.692722in}}%
\pgfpathcurveto{\pgfqpoint{3.857839in}{1.688815in}}{\pgfqpoint{3.855644in}{1.683515in}}{\pgfqpoint{3.855644in}{1.677990in}}%
\pgfpathcurveto{\pgfqpoint{3.855644in}{1.672465in}}{\pgfqpoint{3.857839in}{1.667166in}}{\pgfqpoint{3.861746in}{1.663259in}}%
\pgfpathcurveto{\pgfqpoint{3.865653in}{1.659352in}}{\pgfqpoint{3.870952in}{1.657157in}}{\pgfqpoint{3.876477in}{1.657157in}}%
\pgfpathclose%
\pgfusepath{stroke,fill}%
\end{pgfscope}%
\begin{pgfscope}%
\pgfpathrectangle{\pgfqpoint{0.562500in}{0.275000in}}{\pgfqpoint{3.487500in}{1.925000in}}%
\pgfusepath{clip}%
\pgfsetbuttcap%
\pgfsetroundjoin%
\definecolor{currentfill}{rgb}{0.000000,0.000000,0.000000}%
\pgfsetfillcolor{currentfill}%
\pgfsetlinewidth{1.003750pt}%
\definecolor{currentstroke}{rgb}{0.000000,0.000000,0.000000}%
\pgfsetstrokecolor{currentstroke}%
\pgfsetdash{}{0pt}%
\pgfpathmoveto{\pgfqpoint{3.876477in}{1.667274in}}%
\pgfpathcurveto{\pgfqpoint{3.882002in}{1.667274in}}{\pgfqpoint{3.887302in}{1.669469in}}{\pgfqpoint{3.891209in}{1.673376in}}%
\pgfpathcurveto{\pgfqpoint{3.895115in}{1.677283in}}{\pgfqpoint{3.897311in}{1.682582in}}{\pgfqpoint{3.897311in}{1.688107in}}%
\pgfpathcurveto{\pgfqpoint{3.897311in}{1.693633in}}{\pgfqpoint{3.895115in}{1.698932in}}{\pgfqpoint{3.891209in}{1.702839in}}%
\pgfpathcurveto{\pgfqpoint{3.887302in}{1.706746in}}{\pgfqpoint{3.882002in}{1.708941in}}{\pgfqpoint{3.876477in}{1.708941in}}%
\pgfpathcurveto{\pgfqpoint{3.870952in}{1.708941in}}{\pgfqpoint{3.865653in}{1.706746in}}{\pgfqpoint{3.861746in}{1.702839in}}%
\pgfpathcurveto{\pgfqpoint{3.857839in}{1.698932in}}{\pgfqpoint{3.855644in}{1.693633in}}{\pgfqpoint{3.855644in}{1.688107in}}%
\pgfpathcurveto{\pgfqpoint{3.855644in}{1.682582in}}{\pgfqpoint{3.857839in}{1.677283in}}{\pgfqpoint{3.861746in}{1.673376in}}%
\pgfpathcurveto{\pgfqpoint{3.865653in}{1.669469in}}{\pgfqpoint{3.870952in}{1.667274in}}{\pgfqpoint{3.876477in}{1.667274in}}%
\pgfpathclose%
\pgfusepath{stroke,fill}%
\end{pgfscope}%
\begin{pgfscope}%
\pgfpathrectangle{\pgfqpoint{0.562500in}{0.275000in}}{\pgfqpoint{3.487500in}{1.925000in}}%
\pgfusepath{clip}%
\pgfsetbuttcap%
\pgfsetroundjoin%
\definecolor{currentfill}{rgb}{0.000000,0.000000,0.000000}%
\pgfsetfillcolor{currentfill}%
\pgfsetlinewidth{1.003750pt}%
\definecolor{currentstroke}{rgb}{0.000000,0.000000,0.000000}%
\pgfsetstrokecolor{currentstroke}%
\pgfsetdash{}{0pt}%
\pgfpathmoveto{\pgfqpoint{3.876477in}{1.672333in}}%
\pgfpathcurveto{\pgfqpoint{3.882002in}{1.672333in}}{\pgfqpoint{3.887302in}{1.674528in}}{\pgfqpoint{3.891209in}{1.678435in}}%
\pgfpathcurveto{\pgfqpoint{3.895115in}{1.682341in}}{\pgfqpoint{3.897311in}{1.687641in}}{\pgfqpoint{3.897311in}{1.693166in}}%
\pgfpathcurveto{\pgfqpoint{3.897311in}{1.698691in}}{\pgfqpoint{3.895115in}{1.703991in}}{\pgfqpoint{3.891209in}{1.707897in}}%
\pgfpathcurveto{\pgfqpoint{3.887302in}{1.711804in}}{\pgfqpoint{3.882002in}{1.713999in}}{\pgfqpoint{3.876477in}{1.713999in}}%
\pgfpathcurveto{\pgfqpoint{3.870952in}{1.713999in}}{\pgfqpoint{3.865653in}{1.711804in}}{\pgfqpoint{3.861746in}{1.707897in}}%
\pgfpathcurveto{\pgfqpoint{3.857839in}{1.703991in}}{\pgfqpoint{3.855644in}{1.698691in}}{\pgfqpoint{3.855644in}{1.693166in}}%
\pgfpathcurveto{\pgfqpoint{3.855644in}{1.687641in}}{\pgfqpoint{3.857839in}{1.682341in}}{\pgfqpoint{3.861746in}{1.678435in}}%
\pgfpathcurveto{\pgfqpoint{3.865653in}{1.674528in}}{\pgfqpoint{3.870952in}{1.672333in}}{\pgfqpoint{3.876477in}{1.672333in}}%
\pgfpathclose%
\pgfusepath{stroke,fill}%
\end{pgfscope}%
\begin{pgfscope}%
\pgfpathrectangle{\pgfqpoint{0.562500in}{0.275000in}}{\pgfqpoint{3.487500in}{1.925000in}}%
\pgfusepath{clip}%
\pgfsetbuttcap%
\pgfsetroundjoin%
\definecolor{currentfill}{rgb}{0.000000,0.000000,0.000000}%
\pgfsetfillcolor{currentfill}%
\pgfsetlinewidth{1.003750pt}%
\definecolor{currentstroke}{rgb}{0.000000,0.000000,0.000000}%
\pgfsetstrokecolor{currentstroke}%
\pgfsetdash{}{0pt}%
\pgfpathmoveto{\pgfqpoint{3.876477in}{1.672333in}}%
\pgfpathcurveto{\pgfqpoint{3.882002in}{1.672333in}}{\pgfqpoint{3.887302in}{1.674528in}}{\pgfqpoint{3.891209in}{1.678435in}}%
\pgfpathcurveto{\pgfqpoint{3.895115in}{1.682341in}}{\pgfqpoint{3.897311in}{1.687641in}}{\pgfqpoint{3.897311in}{1.693166in}}%
\pgfpathcurveto{\pgfqpoint{3.897311in}{1.698691in}}{\pgfqpoint{3.895115in}{1.703991in}}{\pgfqpoint{3.891209in}{1.707897in}}%
\pgfpathcurveto{\pgfqpoint{3.887302in}{1.711804in}}{\pgfqpoint{3.882002in}{1.713999in}}{\pgfqpoint{3.876477in}{1.713999in}}%
\pgfpathcurveto{\pgfqpoint{3.870952in}{1.713999in}}{\pgfqpoint{3.865653in}{1.711804in}}{\pgfqpoint{3.861746in}{1.707897in}}%
\pgfpathcurveto{\pgfqpoint{3.857839in}{1.703991in}}{\pgfqpoint{3.855644in}{1.698691in}}{\pgfqpoint{3.855644in}{1.693166in}}%
\pgfpathcurveto{\pgfqpoint{3.855644in}{1.687641in}}{\pgfqpoint{3.857839in}{1.682341in}}{\pgfqpoint{3.861746in}{1.678435in}}%
\pgfpathcurveto{\pgfqpoint{3.865653in}{1.674528in}}{\pgfqpoint{3.870952in}{1.672333in}}{\pgfqpoint{3.876477in}{1.672333in}}%
\pgfpathclose%
\pgfusepath{stroke,fill}%
\end{pgfscope}%
\begin{pgfscope}%
\pgfpathrectangle{\pgfqpoint{0.562500in}{0.275000in}}{\pgfqpoint{3.487500in}{1.925000in}}%
\pgfusepath{clip}%
\pgfsetbuttcap%
\pgfsetroundjoin%
\definecolor{currentfill}{rgb}{0.000000,0.000000,0.000000}%
\pgfsetfillcolor{currentfill}%
\pgfsetlinewidth{1.003750pt}%
\definecolor{currentstroke}{rgb}{0.000000,0.000000,0.000000}%
\pgfsetstrokecolor{currentstroke}%
\pgfsetdash{}{0pt}%
\pgfpathmoveto{\pgfqpoint{3.876477in}{1.834207in}}%
\pgfpathcurveto{\pgfqpoint{3.882002in}{1.834207in}}{\pgfqpoint{3.887302in}{1.836402in}}{\pgfqpoint{3.891209in}{1.840309in}}%
\pgfpathcurveto{\pgfqpoint{3.895115in}{1.844216in}}{\pgfqpoint{3.897311in}{1.849515in}}{\pgfqpoint{3.897311in}{1.855040in}}%
\pgfpathcurveto{\pgfqpoint{3.897311in}{1.860565in}}{\pgfqpoint{3.895115in}{1.865865in}}{\pgfqpoint{3.891209in}{1.869772in}}%
\pgfpathcurveto{\pgfqpoint{3.887302in}{1.873678in}}{\pgfqpoint{3.882002in}{1.875874in}}{\pgfqpoint{3.876477in}{1.875874in}}%
\pgfpathcurveto{\pgfqpoint{3.870952in}{1.875874in}}{\pgfqpoint{3.865653in}{1.873678in}}{\pgfqpoint{3.861746in}{1.869772in}}%
\pgfpathcurveto{\pgfqpoint{3.857839in}{1.865865in}}{\pgfqpoint{3.855644in}{1.860565in}}{\pgfqpoint{3.855644in}{1.855040in}}%
\pgfpathcurveto{\pgfqpoint{3.855644in}{1.849515in}}{\pgfqpoint{3.857839in}{1.844216in}}{\pgfqpoint{3.861746in}{1.840309in}}%
\pgfpathcurveto{\pgfqpoint{3.865653in}{1.836402in}}{\pgfqpoint{3.870952in}{1.834207in}}{\pgfqpoint{3.876477in}{1.834207in}}%
\pgfpathclose%
\pgfusepath{stroke,fill}%
\end{pgfscope}%
\begin{pgfscope}%
\pgfpathrectangle{\pgfqpoint{0.562500in}{0.275000in}}{\pgfqpoint{3.487500in}{1.925000in}}%
\pgfusepath{clip}%
\pgfsetbuttcap%
\pgfsetroundjoin%
\definecolor{currentfill}{rgb}{0.000000,0.000000,0.000000}%
\pgfsetfillcolor{currentfill}%
\pgfsetlinewidth{1.003750pt}%
\definecolor{currentstroke}{rgb}{0.000000,0.000000,0.000000}%
\pgfsetstrokecolor{currentstroke}%
\pgfsetdash{}{0pt}%
\pgfpathmoveto{\pgfqpoint{3.876477in}{1.616688in}}%
\pgfpathcurveto{\pgfqpoint{3.882002in}{1.616688in}}{\pgfqpoint{3.887302in}{1.618884in}}{\pgfqpoint{3.891209in}{1.622790in}}%
\pgfpathcurveto{\pgfqpoint{3.895115in}{1.626697in}}{\pgfqpoint{3.897311in}{1.631997in}}{\pgfqpoint{3.897311in}{1.637522in}}%
\pgfpathcurveto{\pgfqpoint{3.897311in}{1.643047in}}{\pgfqpoint{3.895115in}{1.648346in}}{\pgfqpoint{3.891209in}{1.652253in}}%
\pgfpathcurveto{\pgfqpoint{3.887302in}{1.656160in}}{\pgfqpoint{3.882002in}{1.658355in}}{\pgfqpoint{3.876477in}{1.658355in}}%
\pgfpathcurveto{\pgfqpoint{3.870952in}{1.658355in}}{\pgfqpoint{3.865653in}{1.656160in}}{\pgfqpoint{3.861746in}{1.652253in}}%
\pgfpathcurveto{\pgfqpoint{3.857839in}{1.648346in}}{\pgfqpoint{3.855644in}{1.643047in}}{\pgfqpoint{3.855644in}{1.637522in}}%
\pgfpathcurveto{\pgfqpoint{3.855644in}{1.631997in}}{\pgfqpoint{3.857839in}{1.626697in}}{\pgfqpoint{3.861746in}{1.622790in}}%
\pgfpathcurveto{\pgfqpoint{3.865653in}{1.618884in}}{\pgfqpoint{3.870952in}{1.616688in}}{\pgfqpoint{3.876477in}{1.616688in}}%
\pgfpathclose%
\pgfusepath{stroke,fill}%
\end{pgfscope}%
\begin{pgfscope}%
\pgfpathrectangle{\pgfqpoint{0.562500in}{0.275000in}}{\pgfqpoint{3.487500in}{1.925000in}}%
\pgfusepath{clip}%
\pgfsetbuttcap%
\pgfsetroundjoin%
\definecolor{currentfill}{rgb}{0.000000,0.000000,0.000000}%
\pgfsetfillcolor{currentfill}%
\pgfsetlinewidth{1.003750pt}%
\definecolor{currentstroke}{rgb}{0.000000,0.000000,0.000000}%
\pgfsetstrokecolor{currentstroke}%
\pgfsetdash{}{0pt}%
\pgfpathmoveto{\pgfqpoint{3.876477in}{1.555986in}}%
\pgfpathcurveto{\pgfqpoint{3.882002in}{1.555986in}}{\pgfqpoint{3.887302in}{1.558181in}}{\pgfqpoint{3.891209in}{1.562088in}}%
\pgfpathcurveto{\pgfqpoint{3.895115in}{1.565994in}}{\pgfqpoint{3.897311in}{1.571294in}}{\pgfqpoint{3.897311in}{1.576819in}}%
\pgfpathcurveto{\pgfqpoint{3.897311in}{1.582344in}}{\pgfqpoint{3.895115in}{1.587644in}}{\pgfqpoint{3.891209in}{1.591550in}}%
\pgfpathcurveto{\pgfqpoint{3.887302in}{1.595457in}}{\pgfqpoint{3.882002in}{1.597652in}}{\pgfqpoint{3.876477in}{1.597652in}}%
\pgfpathcurveto{\pgfqpoint{3.870952in}{1.597652in}}{\pgfqpoint{3.865653in}{1.595457in}}{\pgfqpoint{3.861746in}{1.591550in}}%
\pgfpathcurveto{\pgfqpoint{3.857839in}{1.587644in}}{\pgfqpoint{3.855644in}{1.582344in}}{\pgfqpoint{3.855644in}{1.576819in}}%
\pgfpathcurveto{\pgfqpoint{3.855644in}{1.571294in}}{\pgfqpoint{3.857839in}{1.565994in}}{\pgfqpoint{3.861746in}{1.562088in}}%
\pgfpathcurveto{\pgfqpoint{3.865653in}{1.558181in}}{\pgfqpoint{3.870952in}{1.555986in}}{\pgfqpoint{3.876477in}{1.555986in}}%
\pgfpathclose%
\pgfusepath{stroke,fill}%
\end{pgfscope}%
\begin{pgfscope}%
\pgfpathrectangle{\pgfqpoint{0.562500in}{0.275000in}}{\pgfqpoint{3.487500in}{1.925000in}}%
\pgfusepath{clip}%
\pgfsetbuttcap%
\pgfsetroundjoin%
\definecolor{currentfill}{rgb}{0.000000,0.000000,0.000000}%
\pgfsetfillcolor{currentfill}%
\pgfsetlinewidth{1.003750pt}%
\definecolor{currentstroke}{rgb}{0.000000,0.000000,0.000000}%
\pgfsetstrokecolor{currentstroke}%
\pgfsetdash{}{0pt}%
\pgfpathmoveto{\pgfqpoint{3.876477in}{1.844324in}}%
\pgfpathcurveto{\pgfqpoint{3.882002in}{1.844324in}}{\pgfqpoint{3.887302in}{1.846519in}}{\pgfqpoint{3.891209in}{1.850426in}}%
\pgfpathcurveto{\pgfqpoint{3.895115in}{1.854333in}}{\pgfqpoint{3.897311in}{1.859632in}}{\pgfqpoint{3.897311in}{1.865157in}}%
\pgfpathcurveto{\pgfqpoint{3.897311in}{1.870682in}}{\pgfqpoint{3.895115in}{1.875982in}}{\pgfqpoint{3.891209in}{1.879889in}}%
\pgfpathcurveto{\pgfqpoint{3.887302in}{1.883796in}}{\pgfqpoint{3.882002in}{1.885991in}}{\pgfqpoint{3.876477in}{1.885991in}}%
\pgfpathcurveto{\pgfqpoint{3.870952in}{1.885991in}}{\pgfqpoint{3.865653in}{1.883796in}}{\pgfqpoint{3.861746in}{1.879889in}}%
\pgfpathcurveto{\pgfqpoint{3.857839in}{1.875982in}}{\pgfqpoint{3.855644in}{1.870682in}}{\pgfqpoint{3.855644in}{1.865157in}}%
\pgfpathcurveto{\pgfqpoint{3.855644in}{1.859632in}}{\pgfqpoint{3.857839in}{1.854333in}}{\pgfqpoint{3.861746in}{1.850426in}}%
\pgfpathcurveto{\pgfqpoint{3.865653in}{1.846519in}}{\pgfqpoint{3.870952in}{1.844324in}}{\pgfqpoint{3.876477in}{1.844324in}}%
\pgfpathclose%
\pgfusepath{stroke,fill}%
\end{pgfscope}%
\begin{pgfscope}%
\pgfpathrectangle{\pgfqpoint{0.562500in}{0.275000in}}{\pgfqpoint{3.487500in}{1.925000in}}%
\pgfusepath{clip}%
\pgfsetbuttcap%
\pgfsetroundjoin%
\definecolor{currentfill}{rgb}{0.000000,0.000000,0.000000}%
\pgfsetfillcolor{currentfill}%
\pgfsetlinewidth{1.003750pt}%
\definecolor{currentstroke}{rgb}{0.000000,0.000000,0.000000}%
\pgfsetstrokecolor{currentstroke}%
\pgfsetdash{}{0pt}%
\pgfpathmoveto{\pgfqpoint{3.876477in}{1.641981in}}%
\pgfpathcurveto{\pgfqpoint{3.882002in}{1.641981in}}{\pgfqpoint{3.887302in}{1.644176in}}{\pgfqpoint{3.891209in}{1.648083in}}%
\pgfpathcurveto{\pgfqpoint{3.895115in}{1.651990in}}{\pgfqpoint{3.897311in}{1.657290in}}{\pgfqpoint{3.897311in}{1.662815in}}%
\pgfpathcurveto{\pgfqpoint{3.897311in}{1.668340in}}{\pgfqpoint{3.895115in}{1.673639in}}{\pgfqpoint{3.891209in}{1.677546in}}%
\pgfpathcurveto{\pgfqpoint{3.887302in}{1.681453in}}{\pgfqpoint{3.882002in}{1.683648in}}{\pgfqpoint{3.876477in}{1.683648in}}%
\pgfpathcurveto{\pgfqpoint{3.870952in}{1.683648in}}{\pgfqpoint{3.865653in}{1.681453in}}{\pgfqpoint{3.861746in}{1.677546in}}%
\pgfpathcurveto{\pgfqpoint{3.857839in}{1.673639in}}{\pgfqpoint{3.855644in}{1.668340in}}{\pgfqpoint{3.855644in}{1.662815in}}%
\pgfpathcurveto{\pgfqpoint{3.855644in}{1.657290in}}{\pgfqpoint{3.857839in}{1.651990in}}{\pgfqpoint{3.861746in}{1.648083in}}%
\pgfpathcurveto{\pgfqpoint{3.865653in}{1.644176in}}{\pgfqpoint{3.870952in}{1.641981in}}{\pgfqpoint{3.876477in}{1.641981in}}%
\pgfpathclose%
\pgfusepath{stroke,fill}%
\end{pgfscope}%
\begin{pgfscope}%
\pgfpathrectangle{\pgfqpoint{0.562500in}{0.275000in}}{\pgfqpoint{3.487500in}{1.925000in}}%
\pgfusepath{clip}%
\pgfsetbuttcap%
\pgfsetroundjoin%
\definecolor{currentfill}{rgb}{0.000000,0.000000,0.000000}%
\pgfsetfillcolor{currentfill}%
\pgfsetlinewidth{1.003750pt}%
\definecolor{currentstroke}{rgb}{0.000000,0.000000,0.000000}%
\pgfsetstrokecolor{currentstroke}%
\pgfsetdash{}{0pt}%
\pgfpathmoveto{\pgfqpoint{3.876477in}{1.601513in}}%
\pgfpathcurveto{\pgfqpoint{3.882002in}{1.601513in}}{\pgfqpoint{3.887302in}{1.603708in}}{\pgfqpoint{3.891209in}{1.607615in}}%
\pgfpathcurveto{\pgfqpoint{3.895115in}{1.611522in}}{\pgfqpoint{3.897311in}{1.616821in}}{\pgfqpoint{3.897311in}{1.622346in}}%
\pgfpathcurveto{\pgfqpoint{3.897311in}{1.627871in}}{\pgfqpoint{3.895115in}{1.633171in}}{\pgfqpoint{3.891209in}{1.637078in}}%
\pgfpathcurveto{\pgfqpoint{3.887302in}{1.640984in}}{\pgfqpoint{3.882002in}{1.643179in}}{\pgfqpoint{3.876477in}{1.643179in}}%
\pgfpathcurveto{\pgfqpoint{3.870952in}{1.643179in}}{\pgfqpoint{3.865653in}{1.640984in}}{\pgfqpoint{3.861746in}{1.637078in}}%
\pgfpathcurveto{\pgfqpoint{3.857839in}{1.633171in}}{\pgfqpoint{3.855644in}{1.627871in}}{\pgfqpoint{3.855644in}{1.622346in}}%
\pgfpathcurveto{\pgfqpoint{3.855644in}{1.616821in}}{\pgfqpoint{3.857839in}{1.611522in}}{\pgfqpoint{3.861746in}{1.607615in}}%
\pgfpathcurveto{\pgfqpoint{3.865653in}{1.603708in}}{\pgfqpoint{3.870952in}{1.601513in}}{\pgfqpoint{3.876477in}{1.601513in}}%
\pgfpathclose%
\pgfusepath{stroke,fill}%
\end{pgfscope}%
\begin{pgfscope}%
\pgfpathrectangle{\pgfqpoint{0.562500in}{0.275000in}}{\pgfqpoint{3.487500in}{1.925000in}}%
\pgfusepath{clip}%
\pgfsetbuttcap%
\pgfsetroundjoin%
\definecolor{currentfill}{rgb}{0.000000,0.000000,0.000000}%
\pgfsetfillcolor{currentfill}%
\pgfsetlinewidth{1.003750pt}%
\definecolor{currentstroke}{rgb}{0.000000,0.000000,0.000000}%
\pgfsetstrokecolor{currentstroke}%
\pgfsetdash{}{0pt}%
\pgfpathmoveto{\pgfqpoint{3.876477in}{1.970788in}}%
\pgfpathcurveto{\pgfqpoint{3.882002in}{1.970788in}}{\pgfqpoint{3.887302in}{1.972983in}}{\pgfqpoint{3.891209in}{1.976890in}}%
\pgfpathcurveto{\pgfqpoint{3.895115in}{1.980797in}}{\pgfqpoint{3.897311in}{1.986096in}}{\pgfqpoint{3.897311in}{1.991621in}}%
\pgfpathcurveto{\pgfqpoint{3.897311in}{1.997147in}}{\pgfqpoint{3.895115in}{2.002446in}}{\pgfqpoint{3.891209in}{2.006353in}}%
\pgfpathcurveto{\pgfqpoint{3.887302in}{2.010260in}}{\pgfqpoint{3.882002in}{2.012455in}}{\pgfqpoint{3.876477in}{2.012455in}}%
\pgfpathcurveto{\pgfqpoint{3.870952in}{2.012455in}}{\pgfqpoint{3.865653in}{2.010260in}}{\pgfqpoint{3.861746in}{2.006353in}}%
\pgfpathcurveto{\pgfqpoint{3.857839in}{2.002446in}}{\pgfqpoint{3.855644in}{1.997147in}}{\pgfqpoint{3.855644in}{1.991621in}}%
\pgfpathcurveto{\pgfqpoint{3.855644in}{1.986096in}}{\pgfqpoint{3.857839in}{1.980797in}}{\pgfqpoint{3.861746in}{1.976890in}}%
\pgfpathcurveto{\pgfqpoint{3.865653in}{1.972983in}}{\pgfqpoint{3.870952in}{1.970788in}}{\pgfqpoint{3.876477in}{1.970788in}}%
\pgfpathclose%
\pgfusepath{stroke,fill}%
\end{pgfscope}%
\begin{pgfscope}%
\pgfpathrectangle{\pgfqpoint{0.562500in}{0.275000in}}{\pgfqpoint{3.487500in}{1.925000in}}%
\pgfusepath{clip}%
\pgfsetbuttcap%
\pgfsetroundjoin%
\definecolor{currentfill}{rgb}{0.000000,0.000000,0.000000}%
\pgfsetfillcolor{currentfill}%
\pgfsetlinewidth{1.003750pt}%
\definecolor{currentstroke}{rgb}{0.000000,0.000000,0.000000}%
\pgfsetstrokecolor{currentstroke}%
\pgfsetdash{}{0pt}%
\pgfpathmoveto{\pgfqpoint{3.876477in}{1.707743in}}%
\pgfpathcurveto{\pgfqpoint{3.882002in}{1.707743in}}{\pgfqpoint{3.887302in}{1.709938in}}{\pgfqpoint{3.891209in}{1.713845in}}%
\pgfpathcurveto{\pgfqpoint{3.895115in}{1.717751in}}{\pgfqpoint{3.897311in}{1.723051in}}{\pgfqpoint{3.897311in}{1.728576in}}%
\pgfpathcurveto{\pgfqpoint{3.897311in}{1.734101in}}{\pgfqpoint{3.895115in}{1.739401in}}{\pgfqpoint{3.891209in}{1.743307in}}%
\pgfpathcurveto{\pgfqpoint{3.887302in}{1.747214in}}{\pgfqpoint{3.882002in}{1.749409in}}{\pgfqpoint{3.876477in}{1.749409in}}%
\pgfpathcurveto{\pgfqpoint{3.870952in}{1.749409in}}{\pgfqpoint{3.865653in}{1.747214in}}{\pgfqpoint{3.861746in}{1.743307in}}%
\pgfpathcurveto{\pgfqpoint{3.857839in}{1.739401in}}{\pgfqpoint{3.855644in}{1.734101in}}{\pgfqpoint{3.855644in}{1.728576in}}%
\pgfpathcurveto{\pgfqpoint{3.855644in}{1.723051in}}{\pgfqpoint{3.857839in}{1.717751in}}{\pgfqpoint{3.861746in}{1.713845in}}%
\pgfpathcurveto{\pgfqpoint{3.865653in}{1.709938in}}{\pgfqpoint{3.870952in}{1.707743in}}{\pgfqpoint{3.876477in}{1.707743in}}%
\pgfpathclose%
\pgfusepath{stroke,fill}%
\end{pgfscope}%
\begin{pgfscope}%
\pgfpathrectangle{\pgfqpoint{0.562500in}{0.275000in}}{\pgfqpoint{3.487500in}{1.925000in}}%
\pgfusepath{clip}%
\pgfsetbuttcap%
\pgfsetroundjoin%
\definecolor{currentfill}{rgb}{0.000000,0.000000,0.000000}%
\pgfsetfillcolor{currentfill}%
\pgfsetlinewidth{1.003750pt}%
\definecolor{currentstroke}{rgb}{0.000000,0.000000,0.000000}%
\pgfsetstrokecolor{currentstroke}%
\pgfsetdash{}{0pt}%
\pgfpathmoveto{\pgfqpoint{3.876477in}{1.616688in}}%
\pgfpathcurveto{\pgfqpoint{3.882002in}{1.616688in}}{\pgfqpoint{3.887302in}{1.618884in}}{\pgfqpoint{3.891209in}{1.622790in}}%
\pgfpathcurveto{\pgfqpoint{3.895115in}{1.626697in}}{\pgfqpoint{3.897311in}{1.631997in}}{\pgfqpoint{3.897311in}{1.637522in}}%
\pgfpathcurveto{\pgfqpoint{3.897311in}{1.643047in}}{\pgfqpoint{3.895115in}{1.648346in}}{\pgfqpoint{3.891209in}{1.652253in}}%
\pgfpathcurveto{\pgfqpoint{3.887302in}{1.656160in}}{\pgfqpoint{3.882002in}{1.658355in}}{\pgfqpoint{3.876477in}{1.658355in}}%
\pgfpathcurveto{\pgfqpoint{3.870952in}{1.658355in}}{\pgfqpoint{3.865653in}{1.656160in}}{\pgfqpoint{3.861746in}{1.652253in}}%
\pgfpathcurveto{\pgfqpoint{3.857839in}{1.648346in}}{\pgfqpoint{3.855644in}{1.643047in}}{\pgfqpoint{3.855644in}{1.637522in}}%
\pgfpathcurveto{\pgfqpoint{3.855644in}{1.631997in}}{\pgfqpoint{3.857839in}{1.626697in}}{\pgfqpoint{3.861746in}{1.622790in}}%
\pgfpathcurveto{\pgfqpoint{3.865653in}{1.618884in}}{\pgfqpoint{3.870952in}{1.616688in}}{\pgfqpoint{3.876477in}{1.616688in}}%
\pgfpathclose%
\pgfusepath{stroke,fill}%
\end{pgfscope}%
\begin{pgfscope}%
\pgfpathrectangle{\pgfqpoint{0.562500in}{0.275000in}}{\pgfqpoint{3.487500in}{1.925000in}}%
\pgfusepath{clip}%
\pgfsetbuttcap%
\pgfsetroundjoin%
\definecolor{currentfill}{rgb}{0.000000,0.000000,0.000000}%
\pgfsetfillcolor{currentfill}%
\pgfsetlinewidth{1.003750pt}%
\definecolor{currentstroke}{rgb}{0.000000,0.000000,0.000000}%
\pgfsetstrokecolor{currentstroke}%
\pgfsetdash{}{0pt}%
\pgfpathmoveto{\pgfqpoint{3.876477in}{1.738094in}}%
\pgfpathcurveto{\pgfqpoint{3.882002in}{1.738094in}}{\pgfqpoint{3.887302in}{1.740289in}}{\pgfqpoint{3.891209in}{1.744196in}}%
\pgfpathcurveto{\pgfqpoint{3.895115in}{1.748103in}}{\pgfqpoint{3.897311in}{1.753402in}}{\pgfqpoint{3.897311in}{1.758927in}}%
\pgfpathcurveto{\pgfqpoint{3.897311in}{1.764452in}}{\pgfqpoint{3.895115in}{1.769752in}}{\pgfqpoint{3.891209in}{1.773659in}}%
\pgfpathcurveto{\pgfqpoint{3.887302in}{1.777566in}}{\pgfqpoint{3.882002in}{1.779761in}}{\pgfqpoint{3.876477in}{1.779761in}}%
\pgfpathcurveto{\pgfqpoint{3.870952in}{1.779761in}}{\pgfqpoint{3.865653in}{1.777566in}}{\pgfqpoint{3.861746in}{1.773659in}}%
\pgfpathcurveto{\pgfqpoint{3.857839in}{1.769752in}}{\pgfqpoint{3.855644in}{1.764452in}}{\pgfqpoint{3.855644in}{1.758927in}}%
\pgfpathcurveto{\pgfqpoint{3.855644in}{1.753402in}}{\pgfqpoint{3.857839in}{1.748103in}}{\pgfqpoint{3.861746in}{1.744196in}}%
\pgfpathcurveto{\pgfqpoint{3.865653in}{1.740289in}}{\pgfqpoint{3.870952in}{1.738094in}}{\pgfqpoint{3.876477in}{1.738094in}}%
\pgfpathclose%
\pgfusepath{stroke,fill}%
\end{pgfscope}%
\begin{pgfscope}%
\pgfpathrectangle{\pgfqpoint{0.562500in}{0.275000in}}{\pgfqpoint{3.487500in}{1.925000in}}%
\pgfusepath{clip}%
\pgfsetbuttcap%
\pgfsetroundjoin%
\definecolor{currentfill}{rgb}{0.000000,0.000000,0.000000}%
\pgfsetfillcolor{currentfill}%
\pgfsetlinewidth{1.003750pt}%
\definecolor{currentstroke}{rgb}{0.000000,0.000000,0.000000}%
\pgfsetstrokecolor{currentstroke}%
\pgfsetdash{}{0pt}%
\pgfpathmoveto{\pgfqpoint{3.876477in}{1.717860in}}%
\pgfpathcurveto{\pgfqpoint{3.882002in}{1.717860in}}{\pgfqpoint{3.887302in}{1.720055in}}{\pgfqpoint{3.891209in}{1.723962in}}%
\pgfpathcurveto{\pgfqpoint{3.895115in}{1.727869in}}{\pgfqpoint{3.897311in}{1.733168in}}{\pgfqpoint{3.897311in}{1.738693in}}%
\pgfpathcurveto{\pgfqpoint{3.897311in}{1.744218in}}{\pgfqpoint{3.895115in}{1.749518in}}{\pgfqpoint{3.891209in}{1.753425in}}%
\pgfpathcurveto{\pgfqpoint{3.887302in}{1.757331in}}{\pgfqpoint{3.882002in}{1.759526in}}{\pgfqpoint{3.876477in}{1.759526in}}%
\pgfpathcurveto{\pgfqpoint{3.870952in}{1.759526in}}{\pgfqpoint{3.865653in}{1.757331in}}{\pgfqpoint{3.861746in}{1.753425in}}%
\pgfpathcurveto{\pgfqpoint{3.857839in}{1.749518in}}{\pgfqpoint{3.855644in}{1.744218in}}{\pgfqpoint{3.855644in}{1.738693in}}%
\pgfpathcurveto{\pgfqpoint{3.855644in}{1.733168in}}{\pgfqpoint{3.857839in}{1.727869in}}{\pgfqpoint{3.861746in}{1.723962in}}%
\pgfpathcurveto{\pgfqpoint{3.865653in}{1.720055in}}{\pgfqpoint{3.870952in}{1.717860in}}{\pgfqpoint{3.876477in}{1.717860in}}%
\pgfpathclose%
\pgfusepath{stroke,fill}%
\end{pgfscope}%
\begin{pgfscope}%
\pgfpathrectangle{\pgfqpoint{0.562500in}{0.275000in}}{\pgfqpoint{3.487500in}{1.925000in}}%
\pgfusepath{clip}%
\pgfsetbuttcap%
\pgfsetroundjoin%
\definecolor{currentfill}{rgb}{0.000000,0.000000,0.000000}%
\pgfsetfillcolor{currentfill}%
\pgfsetlinewidth{1.003750pt}%
\definecolor{currentstroke}{rgb}{0.000000,0.000000,0.000000}%
\pgfsetstrokecolor{currentstroke}%
\pgfsetdash{}{0pt}%
\pgfpathmoveto{\pgfqpoint{3.876477in}{2.036549in}}%
\pgfpathcurveto{\pgfqpoint{3.882002in}{2.036549in}}{\pgfqpoint{3.887302in}{2.038745in}}{\pgfqpoint{3.891209in}{2.042651in}}%
\pgfpathcurveto{\pgfqpoint{3.895115in}{2.046558in}}{\pgfqpoint{3.897311in}{2.051858in}}{\pgfqpoint{3.897311in}{2.057383in}}%
\pgfpathcurveto{\pgfqpoint{3.897311in}{2.062908in}}{\pgfqpoint{3.895115in}{2.068207in}}{\pgfqpoint{3.891209in}{2.072114in}}%
\pgfpathcurveto{\pgfqpoint{3.887302in}{2.076021in}}{\pgfqpoint{3.882002in}{2.078216in}}{\pgfqpoint{3.876477in}{2.078216in}}%
\pgfpathcurveto{\pgfqpoint{3.870952in}{2.078216in}}{\pgfqpoint{3.865653in}{2.076021in}}{\pgfqpoint{3.861746in}{2.072114in}}%
\pgfpathcurveto{\pgfqpoint{3.857839in}{2.068207in}}{\pgfqpoint{3.855644in}{2.062908in}}{\pgfqpoint{3.855644in}{2.057383in}}%
\pgfpathcurveto{\pgfqpoint{3.855644in}{2.051858in}}{\pgfqpoint{3.857839in}{2.046558in}}{\pgfqpoint{3.861746in}{2.042651in}}%
\pgfpathcurveto{\pgfqpoint{3.865653in}{2.038745in}}{\pgfqpoint{3.870952in}{2.036549in}}{\pgfqpoint{3.876477in}{2.036549in}}%
\pgfpathclose%
\pgfusepath{stroke,fill}%
\end{pgfscope}%
\begin{pgfscope}%
\pgfpathrectangle{\pgfqpoint{0.562500in}{0.275000in}}{\pgfqpoint{3.487500in}{1.925000in}}%
\pgfusepath{clip}%
\pgfsetbuttcap%
\pgfsetroundjoin%
\definecolor{currentfill}{rgb}{0.000000,0.000000,0.000000}%
\pgfsetfillcolor{currentfill}%
\pgfsetlinewidth{1.003750pt}%
\definecolor{currentstroke}{rgb}{0.000000,0.000000,0.000000}%
\pgfsetstrokecolor{currentstroke}%
\pgfsetdash{}{0pt}%
\pgfpathmoveto{\pgfqpoint{3.876477in}{1.839265in}}%
\pgfpathcurveto{\pgfqpoint{3.882002in}{1.839265in}}{\pgfqpoint{3.887302in}{1.841461in}}{\pgfqpoint{3.891209in}{1.845367in}}%
\pgfpathcurveto{\pgfqpoint{3.895115in}{1.849274in}}{\pgfqpoint{3.897311in}{1.854574in}}{\pgfqpoint{3.897311in}{1.860099in}}%
\pgfpathcurveto{\pgfqpoint{3.897311in}{1.865624in}}{\pgfqpoint{3.895115in}{1.870923in}}{\pgfqpoint{3.891209in}{1.874830in}}%
\pgfpathcurveto{\pgfqpoint{3.887302in}{1.878737in}}{\pgfqpoint{3.882002in}{1.880932in}}{\pgfqpoint{3.876477in}{1.880932in}}%
\pgfpathcurveto{\pgfqpoint{3.870952in}{1.880932in}}{\pgfqpoint{3.865653in}{1.878737in}}{\pgfqpoint{3.861746in}{1.874830in}}%
\pgfpathcurveto{\pgfqpoint{3.857839in}{1.870923in}}{\pgfqpoint{3.855644in}{1.865624in}}{\pgfqpoint{3.855644in}{1.860099in}}%
\pgfpathcurveto{\pgfqpoint{3.855644in}{1.854574in}}{\pgfqpoint{3.857839in}{1.849274in}}{\pgfqpoint{3.861746in}{1.845367in}}%
\pgfpathcurveto{\pgfqpoint{3.865653in}{1.841461in}}{\pgfqpoint{3.870952in}{1.839265in}}{\pgfqpoint{3.876477in}{1.839265in}}%
\pgfpathclose%
\pgfusepath{stroke,fill}%
\end{pgfscope}%
\begin{pgfscope}%
\pgfpathrectangle{\pgfqpoint{0.562500in}{0.275000in}}{\pgfqpoint{3.487500in}{1.925000in}}%
\pgfusepath{clip}%
\pgfsetbuttcap%
\pgfsetroundjoin%
\definecolor{currentfill}{rgb}{0.000000,0.000000,0.000000}%
\pgfsetfillcolor{currentfill}%
\pgfsetlinewidth{1.003750pt}%
\definecolor{currentstroke}{rgb}{0.000000,0.000000,0.000000}%
\pgfsetstrokecolor{currentstroke}%
\pgfsetdash{}{0pt}%
\pgfpathmoveto{\pgfqpoint{3.876477in}{1.697626in}}%
\pgfpathcurveto{\pgfqpoint{3.882002in}{1.697626in}}{\pgfqpoint{3.887302in}{1.699821in}}{\pgfqpoint{3.891209in}{1.703727in}}%
\pgfpathcurveto{\pgfqpoint{3.895115in}{1.707634in}}{\pgfqpoint{3.897311in}{1.712934in}}{\pgfqpoint{3.897311in}{1.718459in}}%
\pgfpathcurveto{\pgfqpoint{3.897311in}{1.723984in}}{\pgfqpoint{3.895115in}{1.729283in}}{\pgfqpoint{3.891209in}{1.733190in}}%
\pgfpathcurveto{\pgfqpoint{3.887302in}{1.737097in}}{\pgfqpoint{3.882002in}{1.739292in}}{\pgfqpoint{3.876477in}{1.739292in}}%
\pgfpathcurveto{\pgfqpoint{3.870952in}{1.739292in}}{\pgfqpoint{3.865653in}{1.737097in}}{\pgfqpoint{3.861746in}{1.733190in}}%
\pgfpathcurveto{\pgfqpoint{3.857839in}{1.729283in}}{\pgfqpoint{3.855644in}{1.723984in}}{\pgfqpoint{3.855644in}{1.718459in}}%
\pgfpathcurveto{\pgfqpoint{3.855644in}{1.712934in}}{\pgfqpoint{3.857839in}{1.707634in}}{\pgfqpoint{3.861746in}{1.703727in}}%
\pgfpathcurveto{\pgfqpoint{3.865653in}{1.699821in}}{\pgfqpoint{3.870952in}{1.697626in}}{\pgfqpoint{3.876477in}{1.697626in}}%
\pgfpathclose%
\pgfusepath{stroke,fill}%
\end{pgfscope}%
\begin{pgfscope}%
\pgfpathrectangle{\pgfqpoint{0.562500in}{0.275000in}}{\pgfqpoint{3.487500in}{1.925000in}}%
\pgfusepath{clip}%
\pgfsetbuttcap%
\pgfsetroundjoin%
\definecolor{currentfill}{rgb}{0.000000,0.000000,0.000000}%
\pgfsetfillcolor{currentfill}%
\pgfsetlinewidth{1.003750pt}%
\definecolor{currentstroke}{rgb}{0.000000,0.000000,0.000000}%
\pgfsetstrokecolor{currentstroke}%
\pgfsetdash{}{0pt}%
\pgfpathmoveto{\pgfqpoint{3.876477in}{1.616688in}}%
\pgfpathcurveto{\pgfqpoint{3.882002in}{1.616688in}}{\pgfqpoint{3.887302in}{1.618884in}}{\pgfqpoint{3.891209in}{1.622790in}}%
\pgfpathcurveto{\pgfqpoint{3.895115in}{1.626697in}}{\pgfqpoint{3.897311in}{1.631997in}}{\pgfqpoint{3.897311in}{1.637522in}}%
\pgfpathcurveto{\pgfqpoint{3.897311in}{1.643047in}}{\pgfqpoint{3.895115in}{1.648346in}}{\pgfqpoint{3.891209in}{1.652253in}}%
\pgfpathcurveto{\pgfqpoint{3.887302in}{1.656160in}}{\pgfqpoint{3.882002in}{1.658355in}}{\pgfqpoint{3.876477in}{1.658355in}}%
\pgfpathcurveto{\pgfqpoint{3.870952in}{1.658355in}}{\pgfqpoint{3.865653in}{1.656160in}}{\pgfqpoint{3.861746in}{1.652253in}}%
\pgfpathcurveto{\pgfqpoint{3.857839in}{1.648346in}}{\pgfqpoint{3.855644in}{1.643047in}}{\pgfqpoint{3.855644in}{1.637522in}}%
\pgfpathcurveto{\pgfqpoint{3.855644in}{1.631997in}}{\pgfqpoint{3.857839in}{1.626697in}}{\pgfqpoint{3.861746in}{1.622790in}}%
\pgfpathcurveto{\pgfqpoint{3.865653in}{1.618884in}}{\pgfqpoint{3.870952in}{1.616688in}}{\pgfqpoint{3.876477in}{1.616688in}}%
\pgfpathclose%
\pgfusepath{stroke,fill}%
\end{pgfscope}%
\begin{pgfscope}%
\pgfpathrectangle{\pgfqpoint{0.562500in}{0.275000in}}{\pgfqpoint{3.487500in}{1.925000in}}%
\pgfusepath{clip}%
\pgfsetbuttcap%
\pgfsetroundjoin%
\definecolor{currentfill}{rgb}{0.000000,0.000000,0.000000}%
\pgfsetfillcolor{currentfill}%
\pgfsetlinewidth{1.003750pt}%
\definecolor{currentstroke}{rgb}{0.000000,0.000000,0.000000}%
\pgfsetstrokecolor{currentstroke}%
\pgfsetdash{}{0pt}%
\pgfpathmoveto{\pgfqpoint{3.876477in}{1.803855in}}%
\pgfpathcurveto{\pgfqpoint{3.882002in}{1.803855in}}{\pgfqpoint{3.887302in}{1.806051in}}{\pgfqpoint{3.891209in}{1.809957in}}%
\pgfpathcurveto{\pgfqpoint{3.895115in}{1.813864in}}{\pgfqpoint{3.897311in}{1.819164in}}{\pgfqpoint{3.897311in}{1.824689in}}%
\pgfpathcurveto{\pgfqpoint{3.897311in}{1.830214in}}{\pgfqpoint{3.895115in}{1.835513in}}{\pgfqpoint{3.891209in}{1.839420in}}%
\pgfpathcurveto{\pgfqpoint{3.887302in}{1.843327in}}{\pgfqpoint{3.882002in}{1.845522in}}{\pgfqpoint{3.876477in}{1.845522in}}%
\pgfpathcurveto{\pgfqpoint{3.870952in}{1.845522in}}{\pgfqpoint{3.865653in}{1.843327in}}{\pgfqpoint{3.861746in}{1.839420in}}%
\pgfpathcurveto{\pgfqpoint{3.857839in}{1.835513in}}{\pgfqpoint{3.855644in}{1.830214in}}{\pgfqpoint{3.855644in}{1.824689in}}%
\pgfpathcurveto{\pgfqpoint{3.855644in}{1.819164in}}{\pgfqpoint{3.857839in}{1.813864in}}{\pgfqpoint{3.861746in}{1.809957in}}%
\pgfpathcurveto{\pgfqpoint{3.865653in}{1.806051in}}{\pgfqpoint{3.870952in}{1.803855in}}{\pgfqpoint{3.876477in}{1.803855in}}%
\pgfpathclose%
\pgfusepath{stroke,fill}%
\end{pgfscope}%
\begin{pgfscope}%
\pgfpathrectangle{\pgfqpoint{0.562500in}{0.275000in}}{\pgfqpoint{3.487500in}{1.925000in}}%
\pgfusepath{clip}%
\pgfsetbuttcap%
\pgfsetroundjoin%
\definecolor{currentfill}{rgb}{0.000000,0.000000,0.000000}%
\pgfsetfillcolor{currentfill}%
\pgfsetlinewidth{1.003750pt}%
\definecolor{currentstroke}{rgb}{0.000000,0.000000,0.000000}%
\pgfsetstrokecolor{currentstroke}%
\pgfsetdash{}{0pt}%
\pgfpathmoveto{\pgfqpoint{3.876477in}{2.077018in}}%
\pgfpathcurveto{\pgfqpoint{3.882002in}{2.077018in}}{\pgfqpoint{3.887302in}{2.079213in}}{\pgfqpoint{3.891209in}{2.083120in}}%
\pgfpathcurveto{\pgfqpoint{3.895115in}{2.087027in}}{\pgfqpoint{3.897311in}{2.092326in}}{\pgfqpoint{3.897311in}{2.097851in}}%
\pgfpathcurveto{\pgfqpoint{3.897311in}{2.103376in}}{\pgfqpoint{3.895115in}{2.108676in}}{\pgfqpoint{3.891209in}{2.112583in}}%
\pgfpathcurveto{\pgfqpoint{3.887302in}{2.116490in}}{\pgfqpoint{3.882002in}{2.118685in}}{\pgfqpoint{3.876477in}{2.118685in}}%
\pgfpathcurveto{\pgfqpoint{3.870952in}{2.118685in}}{\pgfqpoint{3.865653in}{2.116490in}}{\pgfqpoint{3.861746in}{2.112583in}}%
\pgfpathcurveto{\pgfqpoint{3.857839in}{2.108676in}}{\pgfqpoint{3.855644in}{2.103376in}}{\pgfqpoint{3.855644in}{2.097851in}}%
\pgfpathcurveto{\pgfqpoint{3.855644in}{2.092326in}}{\pgfqpoint{3.857839in}{2.087027in}}{\pgfqpoint{3.861746in}{2.083120in}}%
\pgfpathcurveto{\pgfqpoint{3.865653in}{2.079213in}}{\pgfqpoint{3.870952in}{2.077018in}}{\pgfqpoint{3.876477in}{2.077018in}}%
\pgfpathclose%
\pgfusepath{stroke,fill}%
\end{pgfscope}%
\begin{pgfscope}%
\pgfpathrectangle{\pgfqpoint{0.562500in}{0.275000in}}{\pgfqpoint{3.487500in}{1.925000in}}%
\pgfusepath{clip}%
\pgfsetbuttcap%
\pgfsetroundjoin%
\definecolor{currentfill}{rgb}{0.000000,0.000000,0.000000}%
\pgfsetfillcolor{currentfill}%
\pgfsetlinewidth{1.003750pt}%
\definecolor{currentstroke}{rgb}{0.000000,0.000000,0.000000}%
\pgfsetstrokecolor{currentstroke}%
\pgfsetdash{}{0pt}%
\pgfpathmoveto{\pgfqpoint{3.876477in}{1.682450in}}%
\pgfpathcurveto{\pgfqpoint{3.882002in}{1.682450in}}{\pgfqpoint{3.887302in}{1.684645in}}{\pgfqpoint{3.891209in}{1.688552in}}%
\pgfpathcurveto{\pgfqpoint{3.895115in}{1.692459in}}{\pgfqpoint{3.897311in}{1.697758in}}{\pgfqpoint{3.897311in}{1.703283in}}%
\pgfpathcurveto{\pgfqpoint{3.897311in}{1.708808in}}{\pgfqpoint{3.895115in}{1.714108in}}{\pgfqpoint{3.891209in}{1.718015in}}%
\pgfpathcurveto{\pgfqpoint{3.887302in}{1.721921in}}{\pgfqpoint{3.882002in}{1.724117in}}{\pgfqpoint{3.876477in}{1.724117in}}%
\pgfpathcurveto{\pgfqpoint{3.870952in}{1.724117in}}{\pgfqpoint{3.865653in}{1.721921in}}{\pgfqpoint{3.861746in}{1.718015in}}%
\pgfpathcurveto{\pgfqpoint{3.857839in}{1.714108in}}{\pgfqpoint{3.855644in}{1.708808in}}{\pgfqpoint{3.855644in}{1.703283in}}%
\pgfpathcurveto{\pgfqpoint{3.855644in}{1.697758in}}{\pgfqpoint{3.857839in}{1.692459in}}{\pgfqpoint{3.861746in}{1.688552in}}%
\pgfpathcurveto{\pgfqpoint{3.865653in}{1.684645in}}{\pgfqpoint{3.870952in}{1.682450in}}{\pgfqpoint{3.876477in}{1.682450in}}%
\pgfpathclose%
\pgfusepath{stroke,fill}%
\end{pgfscope}%
\begin{pgfscope}%
\pgfpathrectangle{\pgfqpoint{0.562500in}{0.275000in}}{\pgfqpoint{3.487500in}{1.925000in}}%
\pgfusepath{clip}%
\pgfsetbuttcap%
\pgfsetroundjoin%
\definecolor{currentfill}{rgb}{0.000000,0.000000,0.000000}%
\pgfsetfillcolor{currentfill}%
\pgfsetlinewidth{1.003750pt}%
\definecolor{currentstroke}{rgb}{0.000000,0.000000,0.000000}%
\pgfsetstrokecolor{currentstroke}%
\pgfsetdash{}{0pt}%
\pgfpathmoveto{\pgfqpoint{3.876477in}{1.576220in}}%
\pgfpathcurveto{\pgfqpoint{3.882002in}{1.576220in}}{\pgfqpoint{3.887302in}{1.578415in}}{\pgfqpoint{3.891209in}{1.582322in}}%
\pgfpathcurveto{\pgfqpoint{3.895115in}{1.586229in}}{\pgfqpoint{3.897311in}{1.591528in}}{\pgfqpoint{3.897311in}{1.597053in}}%
\pgfpathcurveto{\pgfqpoint{3.897311in}{1.602578in}}{\pgfqpoint{3.895115in}{1.607878in}}{\pgfqpoint{3.891209in}{1.611785in}}%
\pgfpathcurveto{\pgfqpoint{3.887302in}{1.615692in}}{\pgfqpoint{3.882002in}{1.617887in}}{\pgfqpoint{3.876477in}{1.617887in}}%
\pgfpathcurveto{\pgfqpoint{3.870952in}{1.617887in}}{\pgfqpoint{3.865653in}{1.615692in}}{\pgfqpoint{3.861746in}{1.611785in}}%
\pgfpathcurveto{\pgfqpoint{3.857839in}{1.607878in}}{\pgfqpoint{3.855644in}{1.602578in}}{\pgfqpoint{3.855644in}{1.597053in}}%
\pgfpathcurveto{\pgfqpoint{3.855644in}{1.591528in}}{\pgfqpoint{3.857839in}{1.586229in}}{\pgfqpoint{3.861746in}{1.582322in}}%
\pgfpathcurveto{\pgfqpoint{3.865653in}{1.578415in}}{\pgfqpoint{3.870952in}{1.576220in}}{\pgfqpoint{3.876477in}{1.576220in}}%
\pgfpathclose%
\pgfusepath{stroke,fill}%
\end{pgfscope}%
\begin{pgfscope}%
\pgfpathrectangle{\pgfqpoint{0.562500in}{0.275000in}}{\pgfqpoint{3.487500in}{1.925000in}}%
\pgfusepath{clip}%
\pgfsetbuttcap%
\pgfsetroundjoin%
\definecolor{currentfill}{rgb}{0.000000,0.000000,0.000000}%
\pgfsetfillcolor{currentfill}%
\pgfsetlinewidth{1.003750pt}%
\definecolor{currentstroke}{rgb}{0.000000,0.000000,0.000000}%
\pgfsetstrokecolor{currentstroke}%
\pgfsetdash{}{0pt}%
\pgfpathmoveto{\pgfqpoint{3.876477in}{1.697626in}}%
\pgfpathcurveto{\pgfqpoint{3.882002in}{1.697626in}}{\pgfqpoint{3.887302in}{1.699821in}}{\pgfqpoint{3.891209in}{1.703727in}}%
\pgfpathcurveto{\pgfqpoint{3.895115in}{1.707634in}}{\pgfqpoint{3.897311in}{1.712934in}}{\pgfqpoint{3.897311in}{1.718459in}}%
\pgfpathcurveto{\pgfqpoint{3.897311in}{1.723984in}}{\pgfqpoint{3.895115in}{1.729283in}}{\pgfqpoint{3.891209in}{1.733190in}}%
\pgfpathcurveto{\pgfqpoint{3.887302in}{1.737097in}}{\pgfqpoint{3.882002in}{1.739292in}}{\pgfqpoint{3.876477in}{1.739292in}}%
\pgfpathcurveto{\pgfqpoint{3.870952in}{1.739292in}}{\pgfqpoint{3.865653in}{1.737097in}}{\pgfqpoint{3.861746in}{1.733190in}}%
\pgfpathcurveto{\pgfqpoint{3.857839in}{1.729283in}}{\pgfqpoint{3.855644in}{1.723984in}}{\pgfqpoint{3.855644in}{1.718459in}}%
\pgfpathcurveto{\pgfqpoint{3.855644in}{1.712934in}}{\pgfqpoint{3.857839in}{1.707634in}}{\pgfqpoint{3.861746in}{1.703727in}}%
\pgfpathcurveto{\pgfqpoint{3.865653in}{1.699821in}}{\pgfqpoint{3.870952in}{1.697626in}}{\pgfqpoint{3.876477in}{1.697626in}}%
\pgfpathclose%
\pgfusepath{stroke,fill}%
\end{pgfscope}%
\begin{pgfscope}%
\pgfpathrectangle{\pgfqpoint{0.562500in}{0.275000in}}{\pgfqpoint{3.487500in}{1.925000in}}%
\pgfusepath{clip}%
\pgfsetbuttcap%
\pgfsetroundjoin%
\definecolor{currentfill}{rgb}{0.000000,0.000000,0.000000}%
\pgfsetfillcolor{currentfill}%
\pgfsetlinewidth{1.003750pt}%
\definecolor{currentstroke}{rgb}{0.000000,0.000000,0.000000}%
\pgfsetstrokecolor{currentstroke}%
\pgfsetdash{}{0pt}%
\pgfpathmoveto{\pgfqpoint{3.876477in}{1.657157in}}%
\pgfpathcurveto{\pgfqpoint{3.882002in}{1.657157in}}{\pgfqpoint{3.887302in}{1.659352in}}{\pgfqpoint{3.891209in}{1.663259in}}%
\pgfpathcurveto{\pgfqpoint{3.895115in}{1.667166in}}{\pgfqpoint{3.897311in}{1.672465in}}{\pgfqpoint{3.897311in}{1.677990in}}%
\pgfpathcurveto{\pgfqpoint{3.897311in}{1.683515in}}{\pgfqpoint{3.895115in}{1.688815in}}{\pgfqpoint{3.891209in}{1.692722in}}%
\pgfpathcurveto{\pgfqpoint{3.887302in}{1.696629in}}{\pgfqpoint{3.882002in}{1.698824in}}{\pgfqpoint{3.876477in}{1.698824in}}%
\pgfpathcurveto{\pgfqpoint{3.870952in}{1.698824in}}{\pgfqpoint{3.865653in}{1.696629in}}{\pgfqpoint{3.861746in}{1.692722in}}%
\pgfpathcurveto{\pgfqpoint{3.857839in}{1.688815in}}{\pgfqpoint{3.855644in}{1.683515in}}{\pgfqpoint{3.855644in}{1.677990in}}%
\pgfpathcurveto{\pgfqpoint{3.855644in}{1.672465in}}{\pgfqpoint{3.857839in}{1.667166in}}{\pgfqpoint{3.861746in}{1.663259in}}%
\pgfpathcurveto{\pgfqpoint{3.865653in}{1.659352in}}{\pgfqpoint{3.870952in}{1.657157in}}{\pgfqpoint{3.876477in}{1.657157in}}%
\pgfpathclose%
\pgfusepath{stroke,fill}%
\end{pgfscope}%
\begin{pgfscope}%
\pgfpathrectangle{\pgfqpoint{0.562500in}{0.275000in}}{\pgfqpoint{3.487500in}{1.925000in}}%
\pgfusepath{clip}%
\pgfsetbuttcap%
\pgfsetroundjoin%
\definecolor{currentfill}{rgb}{0.000000,0.000000,0.000000}%
\pgfsetfillcolor{currentfill}%
\pgfsetlinewidth{1.003750pt}%
\definecolor{currentstroke}{rgb}{0.000000,0.000000,0.000000}%
\pgfsetstrokecolor{currentstroke}%
\pgfsetdash{}{0pt}%
\pgfpathmoveto{\pgfqpoint{3.876477in}{1.874675in}}%
\pgfpathcurveto{\pgfqpoint{3.882002in}{1.874675in}}{\pgfqpoint{3.887302in}{1.876870in}}{\pgfqpoint{3.891209in}{1.880777in}}%
\pgfpathcurveto{\pgfqpoint{3.895115in}{1.884684in}}{\pgfqpoint{3.897311in}{1.889984in}}{\pgfqpoint{3.897311in}{1.895509in}}%
\pgfpathcurveto{\pgfqpoint{3.897311in}{1.901034in}}{\pgfqpoint{3.895115in}{1.906333in}}{\pgfqpoint{3.891209in}{1.910240in}}%
\pgfpathcurveto{\pgfqpoint{3.887302in}{1.914147in}}{\pgfqpoint{3.882002in}{1.916342in}}{\pgfqpoint{3.876477in}{1.916342in}}%
\pgfpathcurveto{\pgfqpoint{3.870952in}{1.916342in}}{\pgfqpoint{3.865653in}{1.914147in}}{\pgfqpoint{3.861746in}{1.910240in}}%
\pgfpathcurveto{\pgfqpoint{3.857839in}{1.906333in}}{\pgfqpoint{3.855644in}{1.901034in}}{\pgfqpoint{3.855644in}{1.895509in}}%
\pgfpathcurveto{\pgfqpoint{3.855644in}{1.889984in}}{\pgfqpoint{3.857839in}{1.884684in}}{\pgfqpoint{3.861746in}{1.880777in}}%
\pgfpathcurveto{\pgfqpoint{3.865653in}{1.876870in}}{\pgfqpoint{3.870952in}{1.874675in}}{\pgfqpoint{3.876477in}{1.874675in}}%
\pgfpathclose%
\pgfusepath{stroke,fill}%
\end{pgfscope}%
\begin{pgfscope}%
\pgfpathrectangle{\pgfqpoint{0.562500in}{0.275000in}}{\pgfqpoint{3.487500in}{1.925000in}}%
\pgfusepath{clip}%
\pgfsetbuttcap%
\pgfsetroundjoin%
\definecolor{currentfill}{rgb}{0.000000,0.000000,0.000000}%
\pgfsetfillcolor{currentfill}%
\pgfsetlinewidth{1.003750pt}%
\definecolor{currentstroke}{rgb}{0.000000,0.000000,0.000000}%
\pgfsetstrokecolor{currentstroke}%
\pgfsetdash{}{0pt}%
\pgfpathmoveto{\pgfqpoint{3.876477in}{1.692567in}}%
\pgfpathcurveto{\pgfqpoint{3.882002in}{1.692567in}}{\pgfqpoint{3.887302in}{1.694762in}}{\pgfqpoint{3.891209in}{1.698669in}}%
\pgfpathcurveto{\pgfqpoint{3.895115in}{1.702576in}}{\pgfqpoint{3.897311in}{1.707875in}}{\pgfqpoint{3.897311in}{1.713400in}}%
\pgfpathcurveto{\pgfqpoint{3.897311in}{1.718925in}}{\pgfqpoint{3.895115in}{1.724225in}}{\pgfqpoint{3.891209in}{1.728132in}}%
\pgfpathcurveto{\pgfqpoint{3.887302in}{1.732039in}}{\pgfqpoint{3.882002in}{1.734234in}}{\pgfqpoint{3.876477in}{1.734234in}}%
\pgfpathcurveto{\pgfqpoint{3.870952in}{1.734234in}}{\pgfqpoint{3.865653in}{1.732039in}}{\pgfqpoint{3.861746in}{1.728132in}}%
\pgfpathcurveto{\pgfqpoint{3.857839in}{1.724225in}}{\pgfqpoint{3.855644in}{1.718925in}}{\pgfqpoint{3.855644in}{1.713400in}}%
\pgfpathcurveto{\pgfqpoint{3.855644in}{1.707875in}}{\pgfqpoint{3.857839in}{1.702576in}}{\pgfqpoint{3.861746in}{1.698669in}}%
\pgfpathcurveto{\pgfqpoint{3.865653in}{1.694762in}}{\pgfqpoint{3.870952in}{1.692567in}}{\pgfqpoint{3.876477in}{1.692567in}}%
\pgfpathclose%
\pgfusepath{stroke,fill}%
\end{pgfscope}%
\begin{pgfscope}%
\pgfpathrectangle{\pgfqpoint{0.562500in}{0.275000in}}{\pgfqpoint{3.487500in}{1.925000in}}%
\pgfusepath{clip}%
\pgfsetbuttcap%
\pgfsetroundjoin%
\definecolor{currentfill}{rgb}{0.000000,0.000000,0.000000}%
\pgfsetfillcolor{currentfill}%
\pgfsetlinewidth{1.003750pt}%
\definecolor{currentstroke}{rgb}{0.000000,0.000000,0.000000}%
\pgfsetstrokecolor{currentstroke}%
\pgfsetdash{}{0pt}%
\pgfpathmoveto{\pgfqpoint{3.876477in}{1.712801in}}%
\pgfpathcurveto{\pgfqpoint{3.882002in}{1.712801in}}{\pgfqpoint{3.887302in}{1.714996in}}{\pgfqpoint{3.891209in}{1.718903in}}%
\pgfpathcurveto{\pgfqpoint{3.895115in}{1.722810in}}{\pgfqpoint{3.897311in}{1.728110in}}{\pgfqpoint{3.897311in}{1.733635in}}%
\pgfpathcurveto{\pgfqpoint{3.897311in}{1.739160in}}{\pgfqpoint{3.895115in}{1.744459in}}{\pgfqpoint{3.891209in}{1.748366in}}%
\pgfpathcurveto{\pgfqpoint{3.887302in}{1.752273in}}{\pgfqpoint{3.882002in}{1.754468in}}{\pgfqpoint{3.876477in}{1.754468in}}%
\pgfpathcurveto{\pgfqpoint{3.870952in}{1.754468in}}{\pgfqpoint{3.865653in}{1.752273in}}{\pgfqpoint{3.861746in}{1.748366in}}%
\pgfpathcurveto{\pgfqpoint{3.857839in}{1.744459in}}{\pgfqpoint{3.855644in}{1.739160in}}{\pgfqpoint{3.855644in}{1.733635in}}%
\pgfpathcurveto{\pgfqpoint{3.855644in}{1.728110in}}{\pgfqpoint{3.857839in}{1.722810in}}{\pgfqpoint{3.861746in}{1.718903in}}%
\pgfpathcurveto{\pgfqpoint{3.865653in}{1.714996in}}{\pgfqpoint{3.870952in}{1.712801in}}{\pgfqpoint{3.876477in}{1.712801in}}%
\pgfpathclose%
\pgfusepath{stroke,fill}%
\end{pgfscope}%
\begin{pgfscope}%
\pgfpathrectangle{\pgfqpoint{0.562500in}{0.275000in}}{\pgfqpoint{3.487500in}{1.925000in}}%
\pgfusepath{clip}%
\pgfsetbuttcap%
\pgfsetroundjoin%
\definecolor{currentfill}{rgb}{0.000000,0.000000,0.000000}%
\pgfsetfillcolor{currentfill}%
\pgfsetlinewidth{1.003750pt}%
\definecolor{currentstroke}{rgb}{0.000000,0.000000,0.000000}%
\pgfsetstrokecolor{currentstroke}%
\pgfsetdash{}{0pt}%
\pgfpathmoveto{\pgfqpoint{3.876477in}{1.899968in}}%
\pgfpathcurveto{\pgfqpoint{3.882002in}{1.899968in}}{\pgfqpoint{3.887302in}{1.902163in}}{\pgfqpoint{3.891209in}{1.906070in}}%
\pgfpathcurveto{\pgfqpoint{3.895115in}{1.909977in}}{\pgfqpoint{3.897311in}{1.915276in}}{\pgfqpoint{3.897311in}{1.920802in}}%
\pgfpathcurveto{\pgfqpoint{3.897311in}{1.926327in}}{\pgfqpoint{3.895115in}{1.931626in}}{\pgfqpoint{3.891209in}{1.935533in}}%
\pgfpathcurveto{\pgfqpoint{3.887302in}{1.939440in}}{\pgfqpoint{3.882002in}{1.941635in}}{\pgfqpoint{3.876477in}{1.941635in}}%
\pgfpathcurveto{\pgfqpoint{3.870952in}{1.941635in}}{\pgfqpoint{3.865653in}{1.939440in}}{\pgfqpoint{3.861746in}{1.935533in}}%
\pgfpathcurveto{\pgfqpoint{3.857839in}{1.931626in}}{\pgfqpoint{3.855644in}{1.926327in}}{\pgfqpoint{3.855644in}{1.920802in}}%
\pgfpathcurveto{\pgfqpoint{3.855644in}{1.915276in}}{\pgfqpoint{3.857839in}{1.909977in}}{\pgfqpoint{3.861746in}{1.906070in}}%
\pgfpathcurveto{\pgfqpoint{3.865653in}{1.902163in}}{\pgfqpoint{3.870952in}{1.899968in}}{\pgfqpoint{3.876477in}{1.899968in}}%
\pgfpathclose%
\pgfusepath{stroke,fill}%
\end{pgfscope}%
\begin{pgfscope}%
\pgfpathrectangle{\pgfqpoint{0.562500in}{0.275000in}}{\pgfqpoint{3.487500in}{1.925000in}}%
\pgfusepath{clip}%
\pgfsetbuttcap%
\pgfsetroundjoin%
\definecolor{currentfill}{rgb}{0.000000,0.000000,0.000000}%
\pgfsetfillcolor{currentfill}%
\pgfsetlinewidth{1.003750pt}%
\definecolor{currentstroke}{rgb}{0.000000,0.000000,0.000000}%
\pgfsetstrokecolor{currentstroke}%
\pgfsetdash{}{0pt}%
\pgfpathmoveto{\pgfqpoint{3.876477in}{1.727977in}}%
\pgfpathcurveto{\pgfqpoint{3.882002in}{1.727977in}}{\pgfqpoint{3.887302in}{1.730172in}}{\pgfqpoint{3.891209in}{1.734079in}}%
\pgfpathcurveto{\pgfqpoint{3.895115in}{1.737986in}}{\pgfqpoint{3.897311in}{1.743285in}}{\pgfqpoint{3.897311in}{1.748810in}}%
\pgfpathcurveto{\pgfqpoint{3.897311in}{1.754335in}}{\pgfqpoint{3.895115in}{1.759635in}}{\pgfqpoint{3.891209in}{1.763542in}}%
\pgfpathcurveto{\pgfqpoint{3.887302in}{1.767448in}}{\pgfqpoint{3.882002in}{1.769644in}}{\pgfqpoint{3.876477in}{1.769644in}}%
\pgfpathcurveto{\pgfqpoint{3.870952in}{1.769644in}}{\pgfqpoint{3.865653in}{1.767448in}}{\pgfqpoint{3.861746in}{1.763542in}}%
\pgfpathcurveto{\pgfqpoint{3.857839in}{1.759635in}}{\pgfqpoint{3.855644in}{1.754335in}}{\pgfqpoint{3.855644in}{1.748810in}}%
\pgfpathcurveto{\pgfqpoint{3.855644in}{1.743285in}}{\pgfqpoint{3.857839in}{1.737986in}}{\pgfqpoint{3.861746in}{1.734079in}}%
\pgfpathcurveto{\pgfqpoint{3.865653in}{1.730172in}}{\pgfqpoint{3.870952in}{1.727977in}}{\pgfqpoint{3.876477in}{1.727977in}}%
\pgfpathclose%
\pgfusepath{stroke,fill}%
\end{pgfscope}%
\begin{pgfscope}%
\pgfpathrectangle{\pgfqpoint{0.562500in}{0.275000in}}{\pgfqpoint{3.487500in}{1.925000in}}%
\pgfusepath{clip}%
\pgfsetbuttcap%
\pgfsetroundjoin%
\definecolor{currentfill}{rgb}{0.000000,0.000000,0.000000}%
\pgfsetfillcolor{currentfill}%
\pgfsetlinewidth{1.003750pt}%
\definecolor{currentstroke}{rgb}{0.000000,0.000000,0.000000}%
\pgfsetstrokecolor{currentstroke}%
\pgfsetdash{}{0pt}%
\pgfpathmoveto{\pgfqpoint{3.876477in}{2.056784in}}%
\pgfpathcurveto{\pgfqpoint{3.882002in}{2.056784in}}{\pgfqpoint{3.887302in}{2.058979in}}{\pgfqpoint{3.891209in}{2.062886in}}%
\pgfpathcurveto{\pgfqpoint{3.895115in}{2.066792in}}{\pgfqpoint{3.897311in}{2.072092in}}{\pgfqpoint{3.897311in}{2.077617in}}%
\pgfpathcurveto{\pgfqpoint{3.897311in}{2.083142in}}{\pgfqpoint{3.895115in}{2.088442in}}{\pgfqpoint{3.891209in}{2.092348in}}%
\pgfpathcurveto{\pgfqpoint{3.887302in}{2.096255in}}{\pgfqpoint{3.882002in}{2.098450in}}{\pgfqpoint{3.876477in}{2.098450in}}%
\pgfpathcurveto{\pgfqpoint{3.870952in}{2.098450in}}{\pgfqpoint{3.865653in}{2.096255in}}{\pgfqpoint{3.861746in}{2.092348in}}%
\pgfpathcurveto{\pgfqpoint{3.857839in}{2.088442in}}{\pgfqpoint{3.855644in}{2.083142in}}{\pgfqpoint{3.855644in}{2.077617in}}%
\pgfpathcurveto{\pgfqpoint{3.855644in}{2.072092in}}{\pgfqpoint{3.857839in}{2.066792in}}{\pgfqpoint{3.861746in}{2.062886in}}%
\pgfpathcurveto{\pgfqpoint{3.865653in}{2.058979in}}{\pgfqpoint{3.870952in}{2.056784in}}{\pgfqpoint{3.876477in}{2.056784in}}%
\pgfpathclose%
\pgfusepath{stroke,fill}%
\end{pgfscope}%
\begin{pgfscope}%
\pgfpathrectangle{\pgfqpoint{0.562500in}{0.275000in}}{\pgfqpoint{3.487500in}{1.925000in}}%
\pgfusepath{clip}%
\pgfsetbuttcap%
\pgfsetroundjoin%
\definecolor{currentfill}{rgb}{0.000000,0.000000,0.000000}%
\pgfsetfillcolor{currentfill}%
\pgfsetlinewidth{1.003750pt}%
\definecolor{currentstroke}{rgb}{0.000000,0.000000,0.000000}%
\pgfsetstrokecolor{currentstroke}%
\pgfsetdash{}{0pt}%
\pgfpathmoveto{\pgfqpoint{3.876477in}{1.647040in}}%
\pgfpathcurveto{\pgfqpoint{3.882002in}{1.647040in}}{\pgfqpoint{3.887302in}{1.649235in}}{\pgfqpoint{3.891209in}{1.653142in}}%
\pgfpathcurveto{\pgfqpoint{3.895115in}{1.657049in}}{\pgfqpoint{3.897311in}{1.662348in}}{\pgfqpoint{3.897311in}{1.667873in}}%
\pgfpathcurveto{\pgfqpoint{3.897311in}{1.673398in}}{\pgfqpoint{3.895115in}{1.678698in}}{\pgfqpoint{3.891209in}{1.682605in}}%
\pgfpathcurveto{\pgfqpoint{3.887302in}{1.686511in}}{\pgfqpoint{3.882002in}{1.688707in}}{\pgfqpoint{3.876477in}{1.688707in}}%
\pgfpathcurveto{\pgfqpoint{3.870952in}{1.688707in}}{\pgfqpoint{3.865653in}{1.686511in}}{\pgfqpoint{3.861746in}{1.682605in}}%
\pgfpathcurveto{\pgfqpoint{3.857839in}{1.678698in}}{\pgfqpoint{3.855644in}{1.673398in}}{\pgfqpoint{3.855644in}{1.667873in}}%
\pgfpathcurveto{\pgfqpoint{3.855644in}{1.662348in}}{\pgfqpoint{3.857839in}{1.657049in}}{\pgfqpoint{3.861746in}{1.653142in}}%
\pgfpathcurveto{\pgfqpoint{3.865653in}{1.649235in}}{\pgfqpoint{3.870952in}{1.647040in}}{\pgfqpoint{3.876477in}{1.647040in}}%
\pgfpathclose%
\pgfusepath{stroke,fill}%
\end{pgfscope}%
\begin{pgfscope}%
\pgfpathrectangle{\pgfqpoint{0.562500in}{0.275000in}}{\pgfqpoint{3.487500in}{1.925000in}}%
\pgfusepath{clip}%
\pgfsetbuttcap%
\pgfsetroundjoin%
\definecolor{currentfill}{rgb}{0.000000,0.000000,0.000000}%
\pgfsetfillcolor{currentfill}%
\pgfsetlinewidth{1.003750pt}%
\definecolor{currentstroke}{rgb}{0.000000,0.000000,0.000000}%
\pgfsetstrokecolor{currentstroke}%
\pgfsetdash{}{0pt}%
\pgfpathmoveto{\pgfqpoint{3.876477in}{1.636923in}}%
\pgfpathcurveto{\pgfqpoint{3.882002in}{1.636923in}}{\pgfqpoint{3.887302in}{1.639118in}}{\pgfqpoint{3.891209in}{1.643025in}}%
\pgfpathcurveto{\pgfqpoint{3.895115in}{1.646932in}}{\pgfqpoint{3.897311in}{1.652231in}}{\pgfqpoint{3.897311in}{1.657756in}}%
\pgfpathcurveto{\pgfqpoint{3.897311in}{1.663281in}}{\pgfqpoint{3.895115in}{1.668581in}}{\pgfqpoint{3.891209in}{1.672487in}}%
\pgfpathcurveto{\pgfqpoint{3.887302in}{1.676394in}}{\pgfqpoint{3.882002in}{1.678589in}}{\pgfqpoint{3.876477in}{1.678589in}}%
\pgfpathcurveto{\pgfqpoint{3.870952in}{1.678589in}}{\pgfqpoint{3.865653in}{1.676394in}}{\pgfqpoint{3.861746in}{1.672487in}}%
\pgfpathcurveto{\pgfqpoint{3.857839in}{1.668581in}}{\pgfqpoint{3.855644in}{1.663281in}}{\pgfqpoint{3.855644in}{1.657756in}}%
\pgfpathcurveto{\pgfqpoint{3.855644in}{1.652231in}}{\pgfqpoint{3.857839in}{1.646932in}}{\pgfqpoint{3.861746in}{1.643025in}}%
\pgfpathcurveto{\pgfqpoint{3.865653in}{1.639118in}}{\pgfqpoint{3.870952in}{1.636923in}}{\pgfqpoint{3.876477in}{1.636923in}}%
\pgfpathclose%
\pgfusepath{stroke,fill}%
\end{pgfscope}%
\begin{pgfscope}%
\pgfpathrectangle{\pgfqpoint{0.562500in}{0.275000in}}{\pgfqpoint{3.487500in}{1.925000in}}%
\pgfusepath{clip}%
\pgfsetbuttcap%
\pgfsetroundjoin%
\definecolor{currentfill}{rgb}{0.000000,0.000000,0.000000}%
\pgfsetfillcolor{currentfill}%
\pgfsetlinewidth{1.003750pt}%
\definecolor{currentstroke}{rgb}{0.000000,0.000000,0.000000}%
\pgfsetstrokecolor{currentstroke}%
\pgfsetdash{}{0pt}%
\pgfpathmoveto{\pgfqpoint{3.876477in}{1.763387in}}%
\pgfpathcurveto{\pgfqpoint{3.882002in}{1.763387in}}{\pgfqpoint{3.887302in}{1.765582in}}{\pgfqpoint{3.891209in}{1.769489in}}%
\pgfpathcurveto{\pgfqpoint{3.895115in}{1.773396in}}{\pgfqpoint{3.897311in}{1.778695in}}{\pgfqpoint{3.897311in}{1.784220in}}%
\pgfpathcurveto{\pgfqpoint{3.897311in}{1.789745in}}{\pgfqpoint{3.895115in}{1.795045in}}{\pgfqpoint{3.891209in}{1.798952in}}%
\pgfpathcurveto{\pgfqpoint{3.887302in}{1.802858in}}{\pgfqpoint{3.882002in}{1.805054in}}{\pgfqpoint{3.876477in}{1.805054in}}%
\pgfpathcurveto{\pgfqpoint{3.870952in}{1.805054in}}{\pgfqpoint{3.865653in}{1.802858in}}{\pgfqpoint{3.861746in}{1.798952in}}%
\pgfpathcurveto{\pgfqpoint{3.857839in}{1.795045in}}{\pgfqpoint{3.855644in}{1.789745in}}{\pgfqpoint{3.855644in}{1.784220in}}%
\pgfpathcurveto{\pgfqpoint{3.855644in}{1.778695in}}{\pgfqpoint{3.857839in}{1.773396in}}{\pgfqpoint{3.861746in}{1.769489in}}%
\pgfpathcurveto{\pgfqpoint{3.865653in}{1.765582in}}{\pgfqpoint{3.870952in}{1.763387in}}{\pgfqpoint{3.876477in}{1.763387in}}%
\pgfpathclose%
\pgfusepath{stroke,fill}%
\end{pgfscope}%
\begin{pgfscope}%
\pgfpathrectangle{\pgfqpoint{0.562500in}{0.275000in}}{\pgfqpoint{3.487500in}{1.925000in}}%
\pgfusepath{clip}%
\pgfsetbuttcap%
\pgfsetroundjoin%
\definecolor{currentfill}{rgb}{0.000000,0.000000,0.000000}%
\pgfsetfillcolor{currentfill}%
\pgfsetlinewidth{1.003750pt}%
\definecolor{currentstroke}{rgb}{0.000000,0.000000,0.000000}%
\pgfsetstrokecolor{currentstroke}%
\pgfsetdash{}{0pt}%
\pgfpathmoveto{\pgfqpoint{3.876477in}{1.712801in}}%
\pgfpathcurveto{\pgfqpoint{3.882002in}{1.712801in}}{\pgfqpoint{3.887302in}{1.714996in}}{\pgfqpoint{3.891209in}{1.718903in}}%
\pgfpathcurveto{\pgfqpoint{3.895115in}{1.722810in}}{\pgfqpoint{3.897311in}{1.728110in}}{\pgfqpoint{3.897311in}{1.733635in}}%
\pgfpathcurveto{\pgfqpoint{3.897311in}{1.739160in}}{\pgfqpoint{3.895115in}{1.744459in}}{\pgfqpoint{3.891209in}{1.748366in}}%
\pgfpathcurveto{\pgfqpoint{3.887302in}{1.752273in}}{\pgfqpoint{3.882002in}{1.754468in}}{\pgfqpoint{3.876477in}{1.754468in}}%
\pgfpathcurveto{\pgfqpoint{3.870952in}{1.754468in}}{\pgfqpoint{3.865653in}{1.752273in}}{\pgfqpoint{3.861746in}{1.748366in}}%
\pgfpathcurveto{\pgfqpoint{3.857839in}{1.744459in}}{\pgfqpoint{3.855644in}{1.739160in}}{\pgfqpoint{3.855644in}{1.733635in}}%
\pgfpathcurveto{\pgfqpoint{3.855644in}{1.728110in}}{\pgfqpoint{3.857839in}{1.722810in}}{\pgfqpoint{3.861746in}{1.718903in}}%
\pgfpathcurveto{\pgfqpoint{3.865653in}{1.714996in}}{\pgfqpoint{3.870952in}{1.712801in}}{\pgfqpoint{3.876477in}{1.712801in}}%
\pgfpathclose%
\pgfusepath{stroke,fill}%
\end{pgfscope}%
\begin{pgfscope}%
\pgfpathrectangle{\pgfqpoint{0.562500in}{0.275000in}}{\pgfqpoint{3.487500in}{1.925000in}}%
\pgfusepath{clip}%
\pgfsetbuttcap%
\pgfsetroundjoin%
\definecolor{currentfill}{rgb}{0.000000,0.000000,0.000000}%
\pgfsetfillcolor{currentfill}%
\pgfsetlinewidth{1.003750pt}%
\definecolor{currentstroke}{rgb}{0.000000,0.000000,0.000000}%
\pgfsetstrokecolor{currentstroke}%
\pgfsetdash{}{0pt}%
\pgfpathmoveto{\pgfqpoint{3.876477in}{2.006198in}}%
\pgfpathcurveto{\pgfqpoint{3.882002in}{2.006198in}}{\pgfqpoint{3.887302in}{2.008393in}}{\pgfqpoint{3.891209in}{2.012300in}}%
\pgfpathcurveto{\pgfqpoint{3.895115in}{2.016207in}}{\pgfqpoint{3.897311in}{2.021506in}}{\pgfqpoint{3.897311in}{2.027031in}}%
\pgfpathcurveto{\pgfqpoint{3.897311in}{2.032556in}}{\pgfqpoint{3.895115in}{2.037856in}}{\pgfqpoint{3.891209in}{2.041763in}}%
\pgfpathcurveto{\pgfqpoint{3.887302in}{2.045670in}}{\pgfqpoint{3.882002in}{2.047865in}}{\pgfqpoint{3.876477in}{2.047865in}}%
\pgfpathcurveto{\pgfqpoint{3.870952in}{2.047865in}}{\pgfqpoint{3.865653in}{2.045670in}}{\pgfqpoint{3.861746in}{2.041763in}}%
\pgfpathcurveto{\pgfqpoint{3.857839in}{2.037856in}}{\pgfqpoint{3.855644in}{2.032556in}}{\pgfqpoint{3.855644in}{2.027031in}}%
\pgfpathcurveto{\pgfqpoint{3.855644in}{2.021506in}}{\pgfqpoint{3.857839in}{2.016207in}}{\pgfqpoint{3.861746in}{2.012300in}}%
\pgfpathcurveto{\pgfqpoint{3.865653in}{2.008393in}}{\pgfqpoint{3.870952in}{2.006198in}}{\pgfqpoint{3.876477in}{2.006198in}}%
\pgfpathclose%
\pgfusepath{stroke,fill}%
\end{pgfscope}%
\begin{pgfscope}%
\pgfpathrectangle{\pgfqpoint{0.562500in}{0.275000in}}{\pgfqpoint{3.487500in}{1.925000in}}%
\pgfusepath{clip}%
\pgfsetbuttcap%
\pgfsetroundjoin%
\definecolor{currentfill}{rgb}{0.000000,0.000000,0.000000}%
\pgfsetfillcolor{currentfill}%
\pgfsetlinewidth{1.003750pt}%
\definecolor{currentstroke}{rgb}{0.000000,0.000000,0.000000}%
\pgfsetstrokecolor{currentstroke}%
\pgfsetdash{}{0pt}%
\pgfpathmoveto{\pgfqpoint{3.876477in}{1.647040in}}%
\pgfpathcurveto{\pgfqpoint{3.882002in}{1.647040in}}{\pgfqpoint{3.887302in}{1.649235in}}{\pgfqpoint{3.891209in}{1.653142in}}%
\pgfpathcurveto{\pgfqpoint{3.895115in}{1.657049in}}{\pgfqpoint{3.897311in}{1.662348in}}{\pgfqpoint{3.897311in}{1.667873in}}%
\pgfpathcurveto{\pgfqpoint{3.897311in}{1.673398in}}{\pgfqpoint{3.895115in}{1.678698in}}{\pgfqpoint{3.891209in}{1.682605in}}%
\pgfpathcurveto{\pgfqpoint{3.887302in}{1.686511in}}{\pgfqpoint{3.882002in}{1.688707in}}{\pgfqpoint{3.876477in}{1.688707in}}%
\pgfpathcurveto{\pgfqpoint{3.870952in}{1.688707in}}{\pgfqpoint{3.865653in}{1.686511in}}{\pgfqpoint{3.861746in}{1.682605in}}%
\pgfpathcurveto{\pgfqpoint{3.857839in}{1.678698in}}{\pgfqpoint{3.855644in}{1.673398in}}{\pgfqpoint{3.855644in}{1.667873in}}%
\pgfpathcurveto{\pgfqpoint{3.855644in}{1.662348in}}{\pgfqpoint{3.857839in}{1.657049in}}{\pgfqpoint{3.861746in}{1.653142in}}%
\pgfpathcurveto{\pgfqpoint{3.865653in}{1.649235in}}{\pgfqpoint{3.870952in}{1.647040in}}{\pgfqpoint{3.876477in}{1.647040in}}%
\pgfpathclose%
\pgfusepath{stroke,fill}%
\end{pgfscope}%
\begin{pgfscope}%
\pgfpathrectangle{\pgfqpoint{0.562500in}{0.275000in}}{\pgfqpoint{3.487500in}{1.925000in}}%
\pgfusepath{clip}%
\pgfsetbuttcap%
\pgfsetroundjoin%
\definecolor{currentfill}{rgb}{0.000000,0.000000,0.000000}%
\pgfsetfillcolor{currentfill}%
\pgfsetlinewidth{1.003750pt}%
\definecolor{currentstroke}{rgb}{0.000000,0.000000,0.000000}%
\pgfsetstrokecolor{currentstroke}%
\pgfsetdash{}{0pt}%
\pgfpathmoveto{\pgfqpoint{3.876477in}{1.596454in}}%
\pgfpathcurveto{\pgfqpoint{3.882002in}{1.596454in}}{\pgfqpoint{3.887302in}{1.598649in}}{\pgfqpoint{3.891209in}{1.602556in}}%
\pgfpathcurveto{\pgfqpoint{3.895115in}{1.606463in}}{\pgfqpoint{3.897311in}{1.611762in}}{\pgfqpoint{3.897311in}{1.617288in}}%
\pgfpathcurveto{\pgfqpoint{3.897311in}{1.622813in}}{\pgfqpoint{3.895115in}{1.628112in}}{\pgfqpoint{3.891209in}{1.632019in}}%
\pgfpathcurveto{\pgfqpoint{3.887302in}{1.635926in}}{\pgfqpoint{3.882002in}{1.638121in}}{\pgfqpoint{3.876477in}{1.638121in}}%
\pgfpathcurveto{\pgfqpoint{3.870952in}{1.638121in}}{\pgfqpoint{3.865653in}{1.635926in}}{\pgfqpoint{3.861746in}{1.632019in}}%
\pgfpathcurveto{\pgfqpoint{3.857839in}{1.628112in}}{\pgfqpoint{3.855644in}{1.622813in}}{\pgfqpoint{3.855644in}{1.617288in}}%
\pgfpathcurveto{\pgfqpoint{3.855644in}{1.611762in}}{\pgfqpoint{3.857839in}{1.606463in}}{\pgfqpoint{3.861746in}{1.602556in}}%
\pgfpathcurveto{\pgfqpoint{3.865653in}{1.598649in}}{\pgfqpoint{3.870952in}{1.596454in}}{\pgfqpoint{3.876477in}{1.596454in}}%
\pgfpathclose%
\pgfusepath{stroke,fill}%
\end{pgfscope}%
\begin{pgfscope}%
\pgfpathrectangle{\pgfqpoint{0.562500in}{0.275000in}}{\pgfqpoint{3.487500in}{1.925000in}}%
\pgfusepath{clip}%
\pgfsetbuttcap%
\pgfsetroundjoin%
\definecolor{currentfill}{rgb}{0.000000,0.000000,0.000000}%
\pgfsetfillcolor{currentfill}%
\pgfsetlinewidth{1.003750pt}%
\definecolor{currentstroke}{rgb}{0.000000,0.000000,0.000000}%
\pgfsetstrokecolor{currentstroke}%
\pgfsetdash{}{0pt}%
\pgfpathmoveto{\pgfqpoint{3.876477in}{1.854441in}}%
\pgfpathcurveto{\pgfqpoint{3.882002in}{1.854441in}}{\pgfqpoint{3.887302in}{1.856636in}}{\pgfqpoint{3.891209in}{1.860543in}}%
\pgfpathcurveto{\pgfqpoint{3.895115in}{1.864450in}}{\pgfqpoint{3.897311in}{1.869749in}}{\pgfqpoint{3.897311in}{1.875274in}}%
\pgfpathcurveto{\pgfqpoint{3.897311in}{1.880799in}}{\pgfqpoint{3.895115in}{1.886099in}}{\pgfqpoint{3.891209in}{1.890006in}}%
\pgfpathcurveto{\pgfqpoint{3.887302in}{1.893913in}}{\pgfqpoint{3.882002in}{1.896108in}}{\pgfqpoint{3.876477in}{1.896108in}}%
\pgfpathcurveto{\pgfqpoint{3.870952in}{1.896108in}}{\pgfqpoint{3.865653in}{1.893913in}}{\pgfqpoint{3.861746in}{1.890006in}}%
\pgfpathcurveto{\pgfqpoint{3.857839in}{1.886099in}}{\pgfqpoint{3.855644in}{1.880799in}}{\pgfqpoint{3.855644in}{1.875274in}}%
\pgfpathcurveto{\pgfqpoint{3.855644in}{1.869749in}}{\pgfqpoint{3.857839in}{1.864450in}}{\pgfqpoint{3.861746in}{1.860543in}}%
\pgfpathcurveto{\pgfqpoint{3.865653in}{1.856636in}}{\pgfqpoint{3.870952in}{1.854441in}}{\pgfqpoint{3.876477in}{1.854441in}}%
\pgfpathclose%
\pgfusepath{stroke,fill}%
\end{pgfscope}%
\begin{pgfscope}%
\pgfpathrectangle{\pgfqpoint{0.562500in}{0.275000in}}{\pgfqpoint{3.487500in}{1.925000in}}%
\pgfusepath{clip}%
\pgfsetbuttcap%
\pgfsetroundjoin%
\definecolor{currentfill}{rgb}{0.000000,0.000000,0.000000}%
\pgfsetfillcolor{currentfill}%
\pgfsetlinewidth{1.003750pt}%
\definecolor{currentstroke}{rgb}{0.000000,0.000000,0.000000}%
\pgfsetstrokecolor{currentstroke}%
\pgfsetdash{}{0pt}%
\pgfpathmoveto{\pgfqpoint{3.876477in}{1.601513in}}%
\pgfpathcurveto{\pgfqpoint{3.882002in}{1.601513in}}{\pgfqpoint{3.887302in}{1.603708in}}{\pgfqpoint{3.891209in}{1.607615in}}%
\pgfpathcurveto{\pgfqpoint{3.895115in}{1.611522in}}{\pgfqpoint{3.897311in}{1.616821in}}{\pgfqpoint{3.897311in}{1.622346in}}%
\pgfpathcurveto{\pgfqpoint{3.897311in}{1.627871in}}{\pgfqpoint{3.895115in}{1.633171in}}{\pgfqpoint{3.891209in}{1.637078in}}%
\pgfpathcurveto{\pgfqpoint{3.887302in}{1.640984in}}{\pgfqpoint{3.882002in}{1.643179in}}{\pgfqpoint{3.876477in}{1.643179in}}%
\pgfpathcurveto{\pgfqpoint{3.870952in}{1.643179in}}{\pgfqpoint{3.865653in}{1.640984in}}{\pgfqpoint{3.861746in}{1.637078in}}%
\pgfpathcurveto{\pgfqpoint{3.857839in}{1.633171in}}{\pgfqpoint{3.855644in}{1.627871in}}{\pgfqpoint{3.855644in}{1.622346in}}%
\pgfpathcurveto{\pgfqpoint{3.855644in}{1.616821in}}{\pgfqpoint{3.857839in}{1.611522in}}{\pgfqpoint{3.861746in}{1.607615in}}%
\pgfpathcurveto{\pgfqpoint{3.865653in}{1.603708in}}{\pgfqpoint{3.870952in}{1.601513in}}{\pgfqpoint{3.876477in}{1.601513in}}%
\pgfpathclose%
\pgfusepath{stroke,fill}%
\end{pgfscope}%
\begin{pgfscope}%
\pgfpathrectangle{\pgfqpoint{0.562500in}{0.275000in}}{\pgfqpoint{3.487500in}{1.925000in}}%
\pgfusepath{clip}%
\pgfsetbuttcap%
\pgfsetroundjoin%
\definecolor{currentfill}{rgb}{0.000000,0.000000,0.000000}%
\pgfsetfillcolor{currentfill}%
\pgfsetlinewidth{1.003750pt}%
\definecolor{currentstroke}{rgb}{0.000000,0.000000,0.000000}%
\pgfsetstrokecolor{currentstroke}%
\pgfsetdash{}{0pt}%
\pgfpathmoveto{\pgfqpoint{3.876477in}{2.001140in}}%
\pgfpathcurveto{\pgfqpoint{3.882002in}{2.001140in}}{\pgfqpoint{3.887302in}{2.003335in}}{\pgfqpoint{3.891209in}{2.007241in}}%
\pgfpathcurveto{\pgfqpoint{3.895115in}{2.011148in}}{\pgfqpoint{3.897311in}{2.016448in}}{\pgfqpoint{3.897311in}{2.021973in}}%
\pgfpathcurveto{\pgfqpoint{3.897311in}{2.027498in}}{\pgfqpoint{3.895115in}{2.032797in}}{\pgfqpoint{3.891209in}{2.036704in}}%
\pgfpathcurveto{\pgfqpoint{3.887302in}{2.040611in}}{\pgfqpoint{3.882002in}{2.042806in}}{\pgfqpoint{3.876477in}{2.042806in}}%
\pgfpathcurveto{\pgfqpoint{3.870952in}{2.042806in}}{\pgfqpoint{3.865653in}{2.040611in}}{\pgfqpoint{3.861746in}{2.036704in}}%
\pgfpathcurveto{\pgfqpoint{3.857839in}{2.032797in}}{\pgfqpoint{3.855644in}{2.027498in}}{\pgfqpoint{3.855644in}{2.021973in}}%
\pgfpathcurveto{\pgfqpoint{3.855644in}{2.016448in}}{\pgfqpoint{3.857839in}{2.011148in}}{\pgfqpoint{3.861746in}{2.007241in}}%
\pgfpathcurveto{\pgfqpoint{3.865653in}{2.003335in}}{\pgfqpoint{3.870952in}{2.001140in}}{\pgfqpoint{3.876477in}{2.001140in}}%
\pgfpathclose%
\pgfusepath{stroke,fill}%
\end{pgfscope}%
\begin{pgfscope}%
\pgfpathrectangle{\pgfqpoint{0.562500in}{0.275000in}}{\pgfqpoint{3.487500in}{1.925000in}}%
\pgfusepath{clip}%
\pgfsetbuttcap%
\pgfsetroundjoin%
\definecolor{currentfill}{rgb}{0.000000,0.000000,0.000000}%
\pgfsetfillcolor{currentfill}%
\pgfsetlinewidth{1.003750pt}%
\definecolor{currentstroke}{rgb}{0.000000,0.000000,0.000000}%
\pgfsetstrokecolor{currentstroke}%
\pgfsetdash{}{0pt}%
\pgfpathmoveto{\pgfqpoint{3.876477in}{1.591396in}}%
\pgfpathcurveto{\pgfqpoint{3.882002in}{1.591396in}}{\pgfqpoint{3.887302in}{1.593591in}}{\pgfqpoint{3.891209in}{1.597498in}}%
\pgfpathcurveto{\pgfqpoint{3.895115in}{1.601404in}}{\pgfqpoint{3.897311in}{1.606704in}}{\pgfqpoint{3.897311in}{1.612229in}}%
\pgfpathcurveto{\pgfqpoint{3.897311in}{1.617754in}}{\pgfqpoint{3.895115in}{1.623054in}}{\pgfqpoint{3.891209in}{1.626960in}}%
\pgfpathcurveto{\pgfqpoint{3.887302in}{1.630867in}}{\pgfqpoint{3.882002in}{1.633062in}}{\pgfqpoint{3.876477in}{1.633062in}}%
\pgfpathcurveto{\pgfqpoint{3.870952in}{1.633062in}}{\pgfqpoint{3.865653in}{1.630867in}}{\pgfqpoint{3.861746in}{1.626960in}}%
\pgfpathcurveto{\pgfqpoint{3.857839in}{1.623054in}}{\pgfqpoint{3.855644in}{1.617754in}}{\pgfqpoint{3.855644in}{1.612229in}}%
\pgfpathcurveto{\pgfqpoint{3.855644in}{1.606704in}}{\pgfqpoint{3.857839in}{1.601404in}}{\pgfqpoint{3.861746in}{1.597498in}}%
\pgfpathcurveto{\pgfqpoint{3.865653in}{1.593591in}}{\pgfqpoint{3.870952in}{1.591396in}}{\pgfqpoint{3.876477in}{1.591396in}}%
\pgfpathclose%
\pgfusepath{stroke,fill}%
\end{pgfscope}%
\begin{pgfscope}%
\pgfpathrectangle{\pgfqpoint{0.562500in}{0.275000in}}{\pgfqpoint{3.487500in}{1.925000in}}%
\pgfusepath{clip}%
\pgfsetbuttcap%
\pgfsetroundjoin%
\definecolor{currentfill}{rgb}{0.000000,0.000000,0.000000}%
\pgfsetfillcolor{currentfill}%
\pgfsetlinewidth{1.003750pt}%
\definecolor{currentstroke}{rgb}{0.000000,0.000000,0.000000}%
\pgfsetstrokecolor{currentstroke}%
\pgfsetdash{}{0pt}%
\pgfpathmoveto{\pgfqpoint{3.876477in}{1.798797in}}%
\pgfpathcurveto{\pgfqpoint{3.882002in}{1.798797in}}{\pgfqpoint{3.887302in}{1.800992in}}{\pgfqpoint{3.891209in}{1.804899in}}%
\pgfpathcurveto{\pgfqpoint{3.895115in}{1.808806in}}{\pgfqpoint{3.897311in}{1.814105in}}{\pgfqpoint{3.897311in}{1.819630in}}%
\pgfpathcurveto{\pgfqpoint{3.897311in}{1.825155in}}{\pgfqpoint{3.895115in}{1.830455in}}{\pgfqpoint{3.891209in}{1.834362in}}%
\pgfpathcurveto{\pgfqpoint{3.887302in}{1.838268in}}{\pgfqpoint{3.882002in}{1.840464in}}{\pgfqpoint{3.876477in}{1.840464in}}%
\pgfpathcurveto{\pgfqpoint{3.870952in}{1.840464in}}{\pgfqpoint{3.865653in}{1.838268in}}{\pgfqpoint{3.861746in}{1.834362in}}%
\pgfpathcurveto{\pgfqpoint{3.857839in}{1.830455in}}{\pgfqpoint{3.855644in}{1.825155in}}{\pgfqpoint{3.855644in}{1.819630in}}%
\pgfpathcurveto{\pgfqpoint{3.855644in}{1.814105in}}{\pgfqpoint{3.857839in}{1.808806in}}{\pgfqpoint{3.861746in}{1.804899in}}%
\pgfpathcurveto{\pgfqpoint{3.865653in}{1.800992in}}{\pgfqpoint{3.870952in}{1.798797in}}{\pgfqpoint{3.876477in}{1.798797in}}%
\pgfpathclose%
\pgfusepath{stroke,fill}%
\end{pgfscope}%
\begin{pgfscope}%
\pgfpathrectangle{\pgfqpoint{0.562500in}{0.275000in}}{\pgfqpoint{3.487500in}{1.925000in}}%
\pgfusepath{clip}%
\pgfsetbuttcap%
\pgfsetroundjoin%
\definecolor{currentfill}{rgb}{0.000000,0.000000,0.000000}%
\pgfsetfillcolor{currentfill}%
\pgfsetlinewidth{1.003750pt}%
\definecolor{currentstroke}{rgb}{0.000000,0.000000,0.000000}%
\pgfsetstrokecolor{currentstroke}%
\pgfsetdash{}{0pt}%
\pgfpathmoveto{\pgfqpoint{3.876477in}{1.662216in}}%
\pgfpathcurveto{\pgfqpoint{3.882002in}{1.662216in}}{\pgfqpoint{3.887302in}{1.664411in}}{\pgfqpoint{3.891209in}{1.668318in}}%
\pgfpathcurveto{\pgfqpoint{3.895115in}{1.672224in}}{\pgfqpoint{3.897311in}{1.677524in}}{\pgfqpoint{3.897311in}{1.683049in}}%
\pgfpathcurveto{\pgfqpoint{3.897311in}{1.688574in}}{\pgfqpoint{3.895115in}{1.693874in}}{\pgfqpoint{3.891209in}{1.697780in}}%
\pgfpathcurveto{\pgfqpoint{3.887302in}{1.701687in}}{\pgfqpoint{3.882002in}{1.703882in}}{\pgfqpoint{3.876477in}{1.703882in}}%
\pgfpathcurveto{\pgfqpoint{3.870952in}{1.703882in}}{\pgfqpoint{3.865653in}{1.701687in}}{\pgfqpoint{3.861746in}{1.697780in}}%
\pgfpathcurveto{\pgfqpoint{3.857839in}{1.693874in}}{\pgfqpoint{3.855644in}{1.688574in}}{\pgfqpoint{3.855644in}{1.683049in}}%
\pgfpathcurveto{\pgfqpoint{3.855644in}{1.677524in}}{\pgfqpoint{3.857839in}{1.672224in}}{\pgfqpoint{3.861746in}{1.668318in}}%
\pgfpathcurveto{\pgfqpoint{3.865653in}{1.664411in}}{\pgfqpoint{3.870952in}{1.662216in}}{\pgfqpoint{3.876477in}{1.662216in}}%
\pgfpathclose%
\pgfusepath{stroke,fill}%
\end{pgfscope}%
\begin{pgfscope}%
\pgfpathrectangle{\pgfqpoint{0.562500in}{0.275000in}}{\pgfqpoint{3.487500in}{1.925000in}}%
\pgfusepath{clip}%
\pgfsetbuttcap%
\pgfsetroundjoin%
\definecolor{currentfill}{rgb}{0.000000,0.000000,0.000000}%
\pgfsetfillcolor{currentfill}%
\pgfsetlinewidth{1.003750pt}%
\definecolor{currentstroke}{rgb}{0.000000,0.000000,0.000000}%
\pgfsetstrokecolor{currentstroke}%
\pgfsetdash{}{0pt}%
\pgfpathmoveto{\pgfqpoint{3.876477in}{1.712801in}}%
\pgfpathcurveto{\pgfqpoint{3.882002in}{1.712801in}}{\pgfqpoint{3.887302in}{1.714996in}}{\pgfqpoint{3.891209in}{1.718903in}}%
\pgfpathcurveto{\pgfqpoint{3.895115in}{1.722810in}}{\pgfqpoint{3.897311in}{1.728110in}}{\pgfqpoint{3.897311in}{1.733635in}}%
\pgfpathcurveto{\pgfqpoint{3.897311in}{1.739160in}}{\pgfqpoint{3.895115in}{1.744459in}}{\pgfqpoint{3.891209in}{1.748366in}}%
\pgfpathcurveto{\pgfqpoint{3.887302in}{1.752273in}}{\pgfqpoint{3.882002in}{1.754468in}}{\pgfqpoint{3.876477in}{1.754468in}}%
\pgfpathcurveto{\pgfqpoint{3.870952in}{1.754468in}}{\pgfqpoint{3.865653in}{1.752273in}}{\pgfqpoint{3.861746in}{1.748366in}}%
\pgfpathcurveto{\pgfqpoint{3.857839in}{1.744459in}}{\pgfqpoint{3.855644in}{1.739160in}}{\pgfqpoint{3.855644in}{1.733635in}}%
\pgfpathcurveto{\pgfqpoint{3.855644in}{1.728110in}}{\pgfqpoint{3.857839in}{1.722810in}}{\pgfqpoint{3.861746in}{1.718903in}}%
\pgfpathcurveto{\pgfqpoint{3.865653in}{1.714996in}}{\pgfqpoint{3.870952in}{1.712801in}}{\pgfqpoint{3.876477in}{1.712801in}}%
\pgfpathclose%
\pgfusepath{stroke,fill}%
\end{pgfscope}%
\begin{pgfscope}%
\pgfpathrectangle{\pgfqpoint{0.562500in}{0.275000in}}{\pgfqpoint{3.487500in}{1.925000in}}%
\pgfusepath{clip}%
\pgfsetbuttcap%
\pgfsetroundjoin%
\definecolor{currentfill}{rgb}{0.000000,0.000000,0.000000}%
\pgfsetfillcolor{currentfill}%
\pgfsetlinewidth{1.003750pt}%
\definecolor{currentstroke}{rgb}{0.000000,0.000000,0.000000}%
\pgfsetstrokecolor{currentstroke}%
\pgfsetdash{}{0pt}%
\pgfpathmoveto{\pgfqpoint{3.876477in}{1.682450in}}%
\pgfpathcurveto{\pgfqpoint{3.882002in}{1.682450in}}{\pgfqpoint{3.887302in}{1.684645in}}{\pgfqpoint{3.891209in}{1.688552in}}%
\pgfpathcurveto{\pgfqpoint{3.895115in}{1.692459in}}{\pgfqpoint{3.897311in}{1.697758in}}{\pgfqpoint{3.897311in}{1.703283in}}%
\pgfpathcurveto{\pgfqpoint{3.897311in}{1.708808in}}{\pgfqpoint{3.895115in}{1.714108in}}{\pgfqpoint{3.891209in}{1.718015in}}%
\pgfpathcurveto{\pgfqpoint{3.887302in}{1.721921in}}{\pgfqpoint{3.882002in}{1.724117in}}{\pgfqpoint{3.876477in}{1.724117in}}%
\pgfpathcurveto{\pgfqpoint{3.870952in}{1.724117in}}{\pgfqpoint{3.865653in}{1.721921in}}{\pgfqpoint{3.861746in}{1.718015in}}%
\pgfpathcurveto{\pgfqpoint{3.857839in}{1.714108in}}{\pgfqpoint{3.855644in}{1.708808in}}{\pgfqpoint{3.855644in}{1.703283in}}%
\pgfpathcurveto{\pgfqpoint{3.855644in}{1.697758in}}{\pgfqpoint{3.857839in}{1.692459in}}{\pgfqpoint{3.861746in}{1.688552in}}%
\pgfpathcurveto{\pgfqpoint{3.865653in}{1.684645in}}{\pgfqpoint{3.870952in}{1.682450in}}{\pgfqpoint{3.876477in}{1.682450in}}%
\pgfpathclose%
\pgfusepath{stroke,fill}%
\end{pgfscope}%
\begin{pgfscope}%
\pgfpathrectangle{\pgfqpoint{0.562500in}{0.275000in}}{\pgfqpoint{3.487500in}{1.925000in}}%
\pgfusepath{clip}%
\pgfsetbuttcap%
\pgfsetroundjoin%
\definecolor{currentfill}{rgb}{0.000000,0.000000,0.000000}%
\pgfsetfillcolor{currentfill}%
\pgfsetlinewidth{1.003750pt}%
\definecolor{currentstroke}{rgb}{0.000000,0.000000,0.000000}%
\pgfsetstrokecolor{currentstroke}%
\pgfsetdash{}{0pt}%
\pgfpathmoveto{\pgfqpoint{3.876477in}{1.697626in}}%
\pgfpathcurveto{\pgfqpoint{3.882002in}{1.697626in}}{\pgfqpoint{3.887302in}{1.699821in}}{\pgfqpoint{3.891209in}{1.703727in}}%
\pgfpathcurveto{\pgfqpoint{3.895115in}{1.707634in}}{\pgfqpoint{3.897311in}{1.712934in}}{\pgfqpoint{3.897311in}{1.718459in}}%
\pgfpathcurveto{\pgfqpoint{3.897311in}{1.723984in}}{\pgfqpoint{3.895115in}{1.729283in}}{\pgfqpoint{3.891209in}{1.733190in}}%
\pgfpathcurveto{\pgfqpoint{3.887302in}{1.737097in}}{\pgfqpoint{3.882002in}{1.739292in}}{\pgfqpoint{3.876477in}{1.739292in}}%
\pgfpathcurveto{\pgfqpoint{3.870952in}{1.739292in}}{\pgfqpoint{3.865653in}{1.737097in}}{\pgfqpoint{3.861746in}{1.733190in}}%
\pgfpathcurveto{\pgfqpoint{3.857839in}{1.729283in}}{\pgfqpoint{3.855644in}{1.723984in}}{\pgfqpoint{3.855644in}{1.718459in}}%
\pgfpathcurveto{\pgfqpoint{3.855644in}{1.712934in}}{\pgfqpoint{3.857839in}{1.707634in}}{\pgfqpoint{3.861746in}{1.703727in}}%
\pgfpathcurveto{\pgfqpoint{3.865653in}{1.699821in}}{\pgfqpoint{3.870952in}{1.697626in}}{\pgfqpoint{3.876477in}{1.697626in}}%
\pgfpathclose%
\pgfusepath{stroke,fill}%
\end{pgfscope}%
\begin{pgfscope}%
\pgfpathrectangle{\pgfqpoint{0.562500in}{0.275000in}}{\pgfqpoint{3.487500in}{1.925000in}}%
\pgfusepath{clip}%
\pgfsetbuttcap%
\pgfsetroundjoin%
\definecolor{currentfill}{rgb}{0.000000,0.000000,0.000000}%
\pgfsetfillcolor{currentfill}%
\pgfsetlinewidth{1.003750pt}%
\definecolor{currentstroke}{rgb}{0.000000,0.000000,0.000000}%
\pgfsetstrokecolor{currentstroke}%
\pgfsetdash{}{0pt}%
\pgfpathmoveto{\pgfqpoint{3.876477in}{1.874675in}}%
\pgfpathcurveto{\pgfqpoint{3.882002in}{1.874675in}}{\pgfqpoint{3.887302in}{1.876870in}}{\pgfqpoint{3.891209in}{1.880777in}}%
\pgfpathcurveto{\pgfqpoint{3.895115in}{1.884684in}}{\pgfqpoint{3.897311in}{1.889984in}}{\pgfqpoint{3.897311in}{1.895509in}}%
\pgfpathcurveto{\pgfqpoint{3.897311in}{1.901034in}}{\pgfqpoint{3.895115in}{1.906333in}}{\pgfqpoint{3.891209in}{1.910240in}}%
\pgfpathcurveto{\pgfqpoint{3.887302in}{1.914147in}}{\pgfqpoint{3.882002in}{1.916342in}}{\pgfqpoint{3.876477in}{1.916342in}}%
\pgfpathcurveto{\pgfqpoint{3.870952in}{1.916342in}}{\pgfqpoint{3.865653in}{1.914147in}}{\pgfqpoint{3.861746in}{1.910240in}}%
\pgfpathcurveto{\pgfqpoint{3.857839in}{1.906333in}}{\pgfqpoint{3.855644in}{1.901034in}}{\pgfqpoint{3.855644in}{1.895509in}}%
\pgfpathcurveto{\pgfqpoint{3.855644in}{1.889984in}}{\pgfqpoint{3.857839in}{1.884684in}}{\pgfqpoint{3.861746in}{1.880777in}}%
\pgfpathcurveto{\pgfqpoint{3.865653in}{1.876870in}}{\pgfqpoint{3.870952in}{1.874675in}}{\pgfqpoint{3.876477in}{1.874675in}}%
\pgfpathclose%
\pgfusepath{stroke,fill}%
\end{pgfscope}%
\begin{pgfscope}%
\pgfpathrectangle{\pgfqpoint{0.562500in}{0.275000in}}{\pgfqpoint{3.487500in}{1.925000in}}%
\pgfusepath{clip}%
\pgfsetbuttcap%
\pgfsetroundjoin%
\definecolor{currentfill}{rgb}{0.000000,0.000000,0.000000}%
\pgfsetfillcolor{currentfill}%
\pgfsetlinewidth{1.003750pt}%
\definecolor{currentstroke}{rgb}{0.000000,0.000000,0.000000}%
\pgfsetstrokecolor{currentstroke}%
\pgfsetdash{}{0pt}%
\pgfpathmoveto{\pgfqpoint{3.876477in}{1.626806in}}%
\pgfpathcurveto{\pgfqpoint{3.882002in}{1.626806in}}{\pgfqpoint{3.887302in}{1.629001in}}{\pgfqpoint{3.891209in}{1.632908in}}%
\pgfpathcurveto{\pgfqpoint{3.895115in}{1.636814in}}{\pgfqpoint{3.897311in}{1.642114in}}{\pgfqpoint{3.897311in}{1.647639in}}%
\pgfpathcurveto{\pgfqpoint{3.897311in}{1.653164in}}{\pgfqpoint{3.895115in}{1.658464in}}{\pgfqpoint{3.891209in}{1.662370in}}%
\pgfpathcurveto{\pgfqpoint{3.887302in}{1.666277in}}{\pgfqpoint{3.882002in}{1.668472in}}{\pgfqpoint{3.876477in}{1.668472in}}%
\pgfpathcurveto{\pgfqpoint{3.870952in}{1.668472in}}{\pgfqpoint{3.865653in}{1.666277in}}{\pgfqpoint{3.861746in}{1.662370in}}%
\pgfpathcurveto{\pgfqpoint{3.857839in}{1.658464in}}{\pgfqpoint{3.855644in}{1.653164in}}{\pgfqpoint{3.855644in}{1.647639in}}%
\pgfpathcurveto{\pgfqpoint{3.855644in}{1.642114in}}{\pgfqpoint{3.857839in}{1.636814in}}{\pgfqpoint{3.861746in}{1.632908in}}%
\pgfpathcurveto{\pgfqpoint{3.865653in}{1.629001in}}{\pgfqpoint{3.870952in}{1.626806in}}{\pgfqpoint{3.876477in}{1.626806in}}%
\pgfpathclose%
\pgfusepath{stroke,fill}%
\end{pgfscope}%
\begin{pgfscope}%
\pgfpathrectangle{\pgfqpoint{0.562500in}{0.275000in}}{\pgfqpoint{3.487500in}{1.925000in}}%
\pgfusepath{clip}%
\pgfsetbuttcap%
\pgfsetroundjoin%
\definecolor{currentfill}{rgb}{0.000000,0.000000,0.000000}%
\pgfsetfillcolor{currentfill}%
\pgfsetlinewidth{1.003750pt}%
\definecolor{currentstroke}{rgb}{0.000000,0.000000,0.000000}%
\pgfsetstrokecolor{currentstroke}%
\pgfsetdash{}{0pt}%
\pgfpathmoveto{\pgfqpoint{3.876477in}{1.935378in}}%
\pgfpathcurveto{\pgfqpoint{3.882002in}{1.935378in}}{\pgfqpoint{3.887302in}{1.937573in}}{\pgfqpoint{3.891209in}{1.941480in}}%
\pgfpathcurveto{\pgfqpoint{3.895115in}{1.945387in}}{\pgfqpoint{3.897311in}{1.950686in}}{\pgfqpoint{3.897311in}{1.956211in}}%
\pgfpathcurveto{\pgfqpoint{3.897311in}{1.961737in}}{\pgfqpoint{3.895115in}{1.967036in}}{\pgfqpoint{3.891209in}{1.970943in}}%
\pgfpathcurveto{\pgfqpoint{3.887302in}{1.974850in}}{\pgfqpoint{3.882002in}{1.977045in}}{\pgfqpoint{3.876477in}{1.977045in}}%
\pgfpathcurveto{\pgfqpoint{3.870952in}{1.977045in}}{\pgfqpoint{3.865653in}{1.974850in}}{\pgfqpoint{3.861746in}{1.970943in}}%
\pgfpathcurveto{\pgfqpoint{3.857839in}{1.967036in}}{\pgfqpoint{3.855644in}{1.961737in}}{\pgfqpoint{3.855644in}{1.956211in}}%
\pgfpathcurveto{\pgfqpoint{3.855644in}{1.950686in}}{\pgfqpoint{3.857839in}{1.945387in}}{\pgfqpoint{3.861746in}{1.941480in}}%
\pgfpathcurveto{\pgfqpoint{3.865653in}{1.937573in}}{\pgfqpoint{3.870952in}{1.935378in}}{\pgfqpoint{3.876477in}{1.935378in}}%
\pgfpathclose%
\pgfusepath{stroke,fill}%
\end{pgfscope}%
\begin{pgfscope}%
\pgfpathrectangle{\pgfqpoint{0.562500in}{0.275000in}}{\pgfqpoint{3.487500in}{1.925000in}}%
\pgfusepath{clip}%
\pgfsetbuttcap%
\pgfsetroundjoin%
\definecolor{currentfill}{rgb}{0.000000,0.000000,0.000000}%
\pgfsetfillcolor{currentfill}%
\pgfsetlinewidth{1.003750pt}%
\definecolor{currentstroke}{rgb}{0.000000,0.000000,0.000000}%
\pgfsetstrokecolor{currentstroke}%
\pgfsetdash{}{0pt}%
\pgfpathmoveto{\pgfqpoint{3.876477in}{1.707743in}}%
\pgfpathcurveto{\pgfqpoint{3.882002in}{1.707743in}}{\pgfqpoint{3.887302in}{1.709938in}}{\pgfqpoint{3.891209in}{1.713845in}}%
\pgfpathcurveto{\pgfqpoint{3.895115in}{1.717751in}}{\pgfqpoint{3.897311in}{1.723051in}}{\pgfqpoint{3.897311in}{1.728576in}}%
\pgfpathcurveto{\pgfqpoint{3.897311in}{1.734101in}}{\pgfqpoint{3.895115in}{1.739401in}}{\pgfqpoint{3.891209in}{1.743307in}}%
\pgfpathcurveto{\pgfqpoint{3.887302in}{1.747214in}}{\pgfqpoint{3.882002in}{1.749409in}}{\pgfqpoint{3.876477in}{1.749409in}}%
\pgfpathcurveto{\pgfqpoint{3.870952in}{1.749409in}}{\pgfqpoint{3.865653in}{1.747214in}}{\pgfqpoint{3.861746in}{1.743307in}}%
\pgfpathcurveto{\pgfqpoint{3.857839in}{1.739401in}}{\pgfqpoint{3.855644in}{1.734101in}}{\pgfqpoint{3.855644in}{1.728576in}}%
\pgfpathcurveto{\pgfqpoint{3.855644in}{1.723051in}}{\pgfqpoint{3.857839in}{1.717751in}}{\pgfqpoint{3.861746in}{1.713845in}}%
\pgfpathcurveto{\pgfqpoint{3.865653in}{1.709938in}}{\pgfqpoint{3.870952in}{1.707743in}}{\pgfqpoint{3.876477in}{1.707743in}}%
\pgfpathclose%
\pgfusepath{stroke,fill}%
\end{pgfscope}%
\begin{pgfscope}%
\pgfpathrectangle{\pgfqpoint{0.562500in}{0.275000in}}{\pgfqpoint{3.487500in}{1.925000in}}%
\pgfusepath{clip}%
\pgfsetbuttcap%
\pgfsetroundjoin%
\definecolor{currentfill}{rgb}{0.000000,0.000000,0.000000}%
\pgfsetfillcolor{currentfill}%
\pgfsetlinewidth{1.003750pt}%
\definecolor{currentstroke}{rgb}{0.000000,0.000000,0.000000}%
\pgfsetstrokecolor{currentstroke}%
\pgfsetdash{}{0pt}%
\pgfpathmoveto{\pgfqpoint{3.876477in}{1.879734in}}%
\pgfpathcurveto{\pgfqpoint{3.882002in}{1.879734in}}{\pgfqpoint{3.887302in}{1.881929in}}{\pgfqpoint{3.891209in}{1.885836in}}%
\pgfpathcurveto{\pgfqpoint{3.895115in}{1.889743in}}{\pgfqpoint{3.897311in}{1.895042in}}{\pgfqpoint{3.897311in}{1.900567in}}%
\pgfpathcurveto{\pgfqpoint{3.897311in}{1.906092in}}{\pgfqpoint{3.895115in}{1.911392in}}{\pgfqpoint{3.891209in}{1.915299in}}%
\pgfpathcurveto{\pgfqpoint{3.887302in}{1.919205in}}{\pgfqpoint{3.882002in}{1.921401in}}{\pgfqpoint{3.876477in}{1.921401in}}%
\pgfpathcurveto{\pgfqpoint{3.870952in}{1.921401in}}{\pgfqpoint{3.865653in}{1.919205in}}{\pgfqpoint{3.861746in}{1.915299in}}%
\pgfpathcurveto{\pgfqpoint{3.857839in}{1.911392in}}{\pgfqpoint{3.855644in}{1.906092in}}{\pgfqpoint{3.855644in}{1.900567in}}%
\pgfpathcurveto{\pgfqpoint{3.855644in}{1.895042in}}{\pgfqpoint{3.857839in}{1.889743in}}{\pgfqpoint{3.861746in}{1.885836in}}%
\pgfpathcurveto{\pgfqpoint{3.865653in}{1.881929in}}{\pgfqpoint{3.870952in}{1.879734in}}{\pgfqpoint{3.876477in}{1.879734in}}%
\pgfpathclose%
\pgfusepath{stroke,fill}%
\end{pgfscope}%
\begin{pgfscope}%
\pgfpathrectangle{\pgfqpoint{0.562500in}{0.275000in}}{\pgfqpoint{3.487500in}{1.925000in}}%
\pgfusepath{clip}%
\pgfsetbuttcap%
\pgfsetroundjoin%
\definecolor{currentfill}{rgb}{0.000000,0.000000,0.000000}%
\pgfsetfillcolor{currentfill}%
\pgfsetlinewidth{1.003750pt}%
\definecolor{currentstroke}{rgb}{0.000000,0.000000,0.000000}%
\pgfsetstrokecolor{currentstroke}%
\pgfsetdash{}{0pt}%
\pgfpathmoveto{\pgfqpoint{3.876477in}{1.611630in}}%
\pgfpathcurveto{\pgfqpoint{3.882002in}{1.611630in}}{\pgfqpoint{3.887302in}{1.613825in}}{\pgfqpoint{3.891209in}{1.617732in}}%
\pgfpathcurveto{\pgfqpoint{3.895115in}{1.621639in}}{\pgfqpoint{3.897311in}{1.626938in}}{\pgfqpoint{3.897311in}{1.632463in}}%
\pgfpathcurveto{\pgfqpoint{3.897311in}{1.637988in}}{\pgfqpoint{3.895115in}{1.643288in}}{\pgfqpoint{3.891209in}{1.647195in}}%
\pgfpathcurveto{\pgfqpoint{3.887302in}{1.651101in}}{\pgfqpoint{3.882002in}{1.653297in}}{\pgfqpoint{3.876477in}{1.653297in}}%
\pgfpathcurveto{\pgfqpoint{3.870952in}{1.653297in}}{\pgfqpoint{3.865653in}{1.651101in}}{\pgfqpoint{3.861746in}{1.647195in}}%
\pgfpathcurveto{\pgfqpoint{3.857839in}{1.643288in}}{\pgfqpoint{3.855644in}{1.637988in}}{\pgfqpoint{3.855644in}{1.632463in}}%
\pgfpathcurveto{\pgfqpoint{3.855644in}{1.626938in}}{\pgfqpoint{3.857839in}{1.621639in}}{\pgfqpoint{3.861746in}{1.617732in}}%
\pgfpathcurveto{\pgfqpoint{3.865653in}{1.613825in}}{\pgfqpoint{3.870952in}{1.611630in}}{\pgfqpoint{3.876477in}{1.611630in}}%
\pgfpathclose%
\pgfusepath{stroke,fill}%
\end{pgfscope}%
\begin{pgfscope}%
\pgfpathrectangle{\pgfqpoint{0.562500in}{0.275000in}}{\pgfqpoint{3.487500in}{1.925000in}}%
\pgfusepath{clip}%
\pgfsetbuttcap%
\pgfsetroundjoin%
\definecolor{currentfill}{rgb}{0.000000,0.000000,0.000000}%
\pgfsetfillcolor{currentfill}%
\pgfsetlinewidth{1.003750pt}%
\definecolor{currentstroke}{rgb}{0.000000,0.000000,0.000000}%
\pgfsetstrokecolor{currentstroke}%
\pgfsetdash{}{0pt}%
\pgfpathmoveto{\pgfqpoint{3.876477in}{1.859500in}}%
\pgfpathcurveto{\pgfqpoint{3.882002in}{1.859500in}}{\pgfqpoint{3.887302in}{1.861695in}}{\pgfqpoint{3.891209in}{1.865602in}}%
\pgfpathcurveto{\pgfqpoint{3.895115in}{1.869508in}}{\pgfqpoint{3.897311in}{1.874808in}}{\pgfqpoint{3.897311in}{1.880333in}}%
\pgfpathcurveto{\pgfqpoint{3.897311in}{1.885858in}}{\pgfqpoint{3.895115in}{1.891158in}}{\pgfqpoint{3.891209in}{1.895064in}}%
\pgfpathcurveto{\pgfqpoint{3.887302in}{1.898971in}}{\pgfqpoint{3.882002in}{1.901166in}}{\pgfqpoint{3.876477in}{1.901166in}}%
\pgfpathcurveto{\pgfqpoint{3.870952in}{1.901166in}}{\pgfqpoint{3.865653in}{1.898971in}}{\pgfqpoint{3.861746in}{1.895064in}}%
\pgfpathcurveto{\pgfqpoint{3.857839in}{1.891158in}}{\pgfqpoint{3.855644in}{1.885858in}}{\pgfqpoint{3.855644in}{1.880333in}}%
\pgfpathcurveto{\pgfqpoint{3.855644in}{1.874808in}}{\pgfqpoint{3.857839in}{1.869508in}}{\pgfqpoint{3.861746in}{1.865602in}}%
\pgfpathcurveto{\pgfqpoint{3.865653in}{1.861695in}}{\pgfqpoint{3.870952in}{1.859500in}}{\pgfqpoint{3.876477in}{1.859500in}}%
\pgfpathclose%
\pgfusepath{stroke,fill}%
\end{pgfscope}%
\begin{pgfscope}%
\pgfpathrectangle{\pgfqpoint{0.562500in}{0.275000in}}{\pgfqpoint{3.487500in}{1.925000in}}%
\pgfusepath{clip}%
\pgfsetbuttcap%
\pgfsetroundjoin%
\definecolor{currentfill}{rgb}{0.000000,0.000000,0.000000}%
\pgfsetfillcolor{currentfill}%
\pgfsetlinewidth{1.003750pt}%
\definecolor{currentstroke}{rgb}{0.000000,0.000000,0.000000}%
\pgfsetstrokecolor{currentstroke}%
\pgfsetdash{}{0pt}%
\pgfpathmoveto{\pgfqpoint{3.876477in}{1.611630in}}%
\pgfpathcurveto{\pgfqpoint{3.882002in}{1.611630in}}{\pgfqpoint{3.887302in}{1.613825in}}{\pgfqpoint{3.891209in}{1.617732in}}%
\pgfpathcurveto{\pgfqpoint{3.895115in}{1.621639in}}{\pgfqpoint{3.897311in}{1.626938in}}{\pgfqpoint{3.897311in}{1.632463in}}%
\pgfpathcurveto{\pgfqpoint{3.897311in}{1.637988in}}{\pgfqpoint{3.895115in}{1.643288in}}{\pgfqpoint{3.891209in}{1.647195in}}%
\pgfpathcurveto{\pgfqpoint{3.887302in}{1.651101in}}{\pgfqpoint{3.882002in}{1.653297in}}{\pgfqpoint{3.876477in}{1.653297in}}%
\pgfpathcurveto{\pgfqpoint{3.870952in}{1.653297in}}{\pgfqpoint{3.865653in}{1.651101in}}{\pgfqpoint{3.861746in}{1.647195in}}%
\pgfpathcurveto{\pgfqpoint{3.857839in}{1.643288in}}{\pgfqpoint{3.855644in}{1.637988in}}{\pgfqpoint{3.855644in}{1.632463in}}%
\pgfpathcurveto{\pgfqpoint{3.855644in}{1.626938in}}{\pgfqpoint{3.857839in}{1.621639in}}{\pgfqpoint{3.861746in}{1.617732in}}%
\pgfpathcurveto{\pgfqpoint{3.865653in}{1.613825in}}{\pgfqpoint{3.870952in}{1.611630in}}{\pgfqpoint{3.876477in}{1.611630in}}%
\pgfpathclose%
\pgfusepath{stroke,fill}%
\end{pgfscope}%
\begin{pgfscope}%
\pgfpathrectangle{\pgfqpoint{0.562500in}{0.275000in}}{\pgfqpoint{3.487500in}{1.925000in}}%
\pgfusepath{clip}%
\pgfsetbuttcap%
\pgfsetroundjoin%
\definecolor{currentfill}{rgb}{0.000000,0.000000,0.000000}%
\pgfsetfillcolor{currentfill}%
\pgfsetlinewidth{1.003750pt}%
\definecolor{currentstroke}{rgb}{0.000000,0.000000,0.000000}%
\pgfsetstrokecolor{currentstroke}%
\pgfsetdash{}{0pt}%
\pgfpathmoveto{\pgfqpoint{3.876477in}{1.652098in}}%
\pgfpathcurveto{\pgfqpoint{3.882002in}{1.652098in}}{\pgfqpoint{3.887302in}{1.654294in}}{\pgfqpoint{3.891209in}{1.658200in}}%
\pgfpathcurveto{\pgfqpoint{3.895115in}{1.662107in}}{\pgfqpoint{3.897311in}{1.667407in}}{\pgfqpoint{3.897311in}{1.672932in}}%
\pgfpathcurveto{\pgfqpoint{3.897311in}{1.678457in}}{\pgfqpoint{3.895115in}{1.683756in}}{\pgfqpoint{3.891209in}{1.687663in}}%
\pgfpathcurveto{\pgfqpoint{3.887302in}{1.691570in}}{\pgfqpoint{3.882002in}{1.693765in}}{\pgfqpoint{3.876477in}{1.693765in}}%
\pgfpathcurveto{\pgfqpoint{3.870952in}{1.693765in}}{\pgfqpoint{3.865653in}{1.691570in}}{\pgfqpoint{3.861746in}{1.687663in}}%
\pgfpathcurveto{\pgfqpoint{3.857839in}{1.683756in}}{\pgfqpoint{3.855644in}{1.678457in}}{\pgfqpoint{3.855644in}{1.672932in}}%
\pgfpathcurveto{\pgfqpoint{3.855644in}{1.667407in}}{\pgfqpoint{3.857839in}{1.662107in}}{\pgfqpoint{3.861746in}{1.658200in}}%
\pgfpathcurveto{\pgfqpoint{3.865653in}{1.654294in}}{\pgfqpoint{3.870952in}{1.652098in}}{\pgfqpoint{3.876477in}{1.652098in}}%
\pgfpathclose%
\pgfusepath{stroke,fill}%
\end{pgfscope}%
\begin{pgfscope}%
\pgfpathrectangle{\pgfqpoint{0.562500in}{0.275000in}}{\pgfqpoint{3.487500in}{1.925000in}}%
\pgfusepath{clip}%
\pgfsetbuttcap%
\pgfsetroundjoin%
\definecolor{currentfill}{rgb}{0.000000,0.000000,0.000000}%
\pgfsetfillcolor{currentfill}%
\pgfsetlinewidth{1.003750pt}%
\definecolor{currentstroke}{rgb}{0.000000,0.000000,0.000000}%
\pgfsetstrokecolor{currentstroke}%
\pgfsetdash{}{0pt}%
\pgfpathmoveto{\pgfqpoint{3.876477in}{1.677391in}}%
\pgfpathcurveto{\pgfqpoint{3.882002in}{1.677391in}}{\pgfqpoint{3.887302in}{1.679586in}}{\pgfqpoint{3.891209in}{1.683493in}}%
\pgfpathcurveto{\pgfqpoint{3.895115in}{1.687400in}}{\pgfqpoint{3.897311in}{1.692700in}}{\pgfqpoint{3.897311in}{1.698225in}}%
\pgfpathcurveto{\pgfqpoint{3.897311in}{1.703750in}}{\pgfqpoint{3.895115in}{1.709049in}}{\pgfqpoint{3.891209in}{1.712956in}}%
\pgfpathcurveto{\pgfqpoint{3.887302in}{1.716863in}}{\pgfqpoint{3.882002in}{1.719058in}}{\pgfqpoint{3.876477in}{1.719058in}}%
\pgfpathcurveto{\pgfqpoint{3.870952in}{1.719058in}}{\pgfqpoint{3.865653in}{1.716863in}}{\pgfqpoint{3.861746in}{1.712956in}}%
\pgfpathcurveto{\pgfqpoint{3.857839in}{1.709049in}}{\pgfqpoint{3.855644in}{1.703750in}}{\pgfqpoint{3.855644in}{1.698225in}}%
\pgfpathcurveto{\pgfqpoint{3.855644in}{1.692700in}}{\pgfqpoint{3.857839in}{1.687400in}}{\pgfqpoint{3.861746in}{1.683493in}}%
\pgfpathcurveto{\pgfqpoint{3.865653in}{1.679586in}}{\pgfqpoint{3.870952in}{1.677391in}}{\pgfqpoint{3.876477in}{1.677391in}}%
\pgfpathclose%
\pgfusepath{stroke,fill}%
\end{pgfscope}%
\begin{pgfscope}%
\pgfpathrectangle{\pgfqpoint{0.562500in}{0.275000in}}{\pgfqpoint{3.487500in}{1.925000in}}%
\pgfusepath{clip}%
\pgfsetbuttcap%
\pgfsetroundjoin%
\definecolor{currentfill}{rgb}{0.000000,0.000000,0.000000}%
\pgfsetfillcolor{currentfill}%
\pgfsetlinewidth{1.003750pt}%
\definecolor{currentstroke}{rgb}{0.000000,0.000000,0.000000}%
\pgfsetstrokecolor{currentstroke}%
\pgfsetdash{}{0pt}%
\pgfpathmoveto{\pgfqpoint{3.876477in}{1.591396in}}%
\pgfpathcurveto{\pgfqpoint{3.882002in}{1.591396in}}{\pgfqpoint{3.887302in}{1.593591in}}{\pgfqpoint{3.891209in}{1.597498in}}%
\pgfpathcurveto{\pgfqpoint{3.895115in}{1.601404in}}{\pgfqpoint{3.897311in}{1.606704in}}{\pgfqpoint{3.897311in}{1.612229in}}%
\pgfpathcurveto{\pgfqpoint{3.897311in}{1.617754in}}{\pgfqpoint{3.895115in}{1.623054in}}{\pgfqpoint{3.891209in}{1.626960in}}%
\pgfpathcurveto{\pgfqpoint{3.887302in}{1.630867in}}{\pgfqpoint{3.882002in}{1.633062in}}{\pgfqpoint{3.876477in}{1.633062in}}%
\pgfpathcurveto{\pgfqpoint{3.870952in}{1.633062in}}{\pgfqpoint{3.865653in}{1.630867in}}{\pgfqpoint{3.861746in}{1.626960in}}%
\pgfpathcurveto{\pgfqpoint{3.857839in}{1.623054in}}{\pgfqpoint{3.855644in}{1.617754in}}{\pgfqpoint{3.855644in}{1.612229in}}%
\pgfpathcurveto{\pgfqpoint{3.855644in}{1.606704in}}{\pgfqpoint{3.857839in}{1.601404in}}{\pgfqpoint{3.861746in}{1.597498in}}%
\pgfpathcurveto{\pgfqpoint{3.865653in}{1.593591in}}{\pgfqpoint{3.870952in}{1.591396in}}{\pgfqpoint{3.876477in}{1.591396in}}%
\pgfpathclose%
\pgfusepath{stroke,fill}%
\end{pgfscope}%
\begin{pgfscope}%
\pgfpathrectangle{\pgfqpoint{0.562500in}{0.275000in}}{\pgfqpoint{3.487500in}{1.925000in}}%
\pgfusepath{clip}%
\pgfsetbuttcap%
\pgfsetroundjoin%
\definecolor{currentfill}{rgb}{0.000000,0.000000,0.000000}%
\pgfsetfillcolor{currentfill}%
\pgfsetlinewidth{1.003750pt}%
\definecolor{currentstroke}{rgb}{0.000000,0.000000,0.000000}%
\pgfsetstrokecolor{currentstroke}%
\pgfsetdash{}{0pt}%
\pgfpathmoveto{\pgfqpoint{3.876477in}{1.753270in}}%
\pgfpathcurveto{\pgfqpoint{3.882002in}{1.753270in}}{\pgfqpoint{3.887302in}{1.755465in}}{\pgfqpoint{3.891209in}{1.759372in}}%
\pgfpathcurveto{\pgfqpoint{3.895115in}{1.763279in}}{\pgfqpoint{3.897311in}{1.768578in}}{\pgfqpoint{3.897311in}{1.774103in}}%
\pgfpathcurveto{\pgfqpoint{3.897311in}{1.779628in}}{\pgfqpoint{3.895115in}{1.784928in}}{\pgfqpoint{3.891209in}{1.788835in}}%
\pgfpathcurveto{\pgfqpoint{3.887302in}{1.792741in}}{\pgfqpoint{3.882002in}{1.794936in}}{\pgfqpoint{3.876477in}{1.794936in}}%
\pgfpathcurveto{\pgfqpoint{3.870952in}{1.794936in}}{\pgfqpoint{3.865653in}{1.792741in}}{\pgfqpoint{3.861746in}{1.788835in}}%
\pgfpathcurveto{\pgfqpoint{3.857839in}{1.784928in}}{\pgfqpoint{3.855644in}{1.779628in}}{\pgfqpoint{3.855644in}{1.774103in}}%
\pgfpathcurveto{\pgfqpoint{3.855644in}{1.768578in}}{\pgfqpoint{3.857839in}{1.763279in}}{\pgfqpoint{3.861746in}{1.759372in}}%
\pgfpathcurveto{\pgfqpoint{3.865653in}{1.755465in}}{\pgfqpoint{3.870952in}{1.753270in}}{\pgfqpoint{3.876477in}{1.753270in}}%
\pgfpathclose%
\pgfusepath{stroke,fill}%
\end{pgfscope}%
\begin{pgfscope}%
\pgfpathrectangle{\pgfqpoint{0.562500in}{0.275000in}}{\pgfqpoint{3.487500in}{1.925000in}}%
\pgfusepath{clip}%
\pgfsetbuttcap%
\pgfsetroundjoin%
\definecolor{currentfill}{rgb}{0.000000,0.000000,0.000000}%
\pgfsetfillcolor{currentfill}%
\pgfsetlinewidth{1.003750pt}%
\definecolor{currentstroke}{rgb}{0.000000,0.000000,0.000000}%
\pgfsetstrokecolor{currentstroke}%
\pgfsetdash{}{0pt}%
\pgfpathmoveto{\pgfqpoint{3.876477in}{1.727977in}}%
\pgfpathcurveto{\pgfqpoint{3.882002in}{1.727977in}}{\pgfqpoint{3.887302in}{1.730172in}}{\pgfqpoint{3.891209in}{1.734079in}}%
\pgfpathcurveto{\pgfqpoint{3.895115in}{1.737986in}}{\pgfqpoint{3.897311in}{1.743285in}}{\pgfqpoint{3.897311in}{1.748810in}}%
\pgfpathcurveto{\pgfqpoint{3.897311in}{1.754335in}}{\pgfqpoint{3.895115in}{1.759635in}}{\pgfqpoint{3.891209in}{1.763542in}}%
\pgfpathcurveto{\pgfqpoint{3.887302in}{1.767448in}}{\pgfqpoint{3.882002in}{1.769644in}}{\pgfqpoint{3.876477in}{1.769644in}}%
\pgfpathcurveto{\pgfqpoint{3.870952in}{1.769644in}}{\pgfqpoint{3.865653in}{1.767448in}}{\pgfqpoint{3.861746in}{1.763542in}}%
\pgfpathcurveto{\pgfqpoint{3.857839in}{1.759635in}}{\pgfqpoint{3.855644in}{1.754335in}}{\pgfqpoint{3.855644in}{1.748810in}}%
\pgfpathcurveto{\pgfqpoint{3.855644in}{1.743285in}}{\pgfqpoint{3.857839in}{1.737986in}}{\pgfqpoint{3.861746in}{1.734079in}}%
\pgfpathcurveto{\pgfqpoint{3.865653in}{1.730172in}}{\pgfqpoint{3.870952in}{1.727977in}}{\pgfqpoint{3.876477in}{1.727977in}}%
\pgfpathclose%
\pgfusepath{stroke,fill}%
\end{pgfscope}%
\begin{pgfscope}%
\pgfpathrectangle{\pgfqpoint{0.562500in}{0.275000in}}{\pgfqpoint{3.487500in}{1.925000in}}%
\pgfusepath{clip}%
\pgfsetbuttcap%
\pgfsetroundjoin%
\definecolor{currentfill}{rgb}{0.000000,0.000000,0.000000}%
\pgfsetfillcolor{currentfill}%
\pgfsetlinewidth{1.003750pt}%
\definecolor{currentstroke}{rgb}{0.000000,0.000000,0.000000}%
\pgfsetstrokecolor{currentstroke}%
\pgfsetdash{}{0pt}%
\pgfpathmoveto{\pgfqpoint{3.876477in}{1.586337in}}%
\pgfpathcurveto{\pgfqpoint{3.882002in}{1.586337in}}{\pgfqpoint{3.887302in}{1.588532in}}{\pgfqpoint{3.891209in}{1.592439in}}%
\pgfpathcurveto{\pgfqpoint{3.895115in}{1.596346in}}{\pgfqpoint{3.897311in}{1.601645in}}{\pgfqpoint{3.897311in}{1.607170in}}%
\pgfpathcurveto{\pgfqpoint{3.897311in}{1.612695in}}{\pgfqpoint{3.895115in}{1.617995in}}{\pgfqpoint{3.891209in}{1.621902in}}%
\pgfpathcurveto{\pgfqpoint{3.887302in}{1.625809in}}{\pgfqpoint{3.882002in}{1.628004in}}{\pgfqpoint{3.876477in}{1.628004in}}%
\pgfpathcurveto{\pgfqpoint{3.870952in}{1.628004in}}{\pgfqpoint{3.865653in}{1.625809in}}{\pgfqpoint{3.861746in}{1.621902in}}%
\pgfpathcurveto{\pgfqpoint{3.857839in}{1.617995in}}{\pgfqpoint{3.855644in}{1.612695in}}{\pgfqpoint{3.855644in}{1.607170in}}%
\pgfpathcurveto{\pgfqpoint{3.855644in}{1.601645in}}{\pgfqpoint{3.857839in}{1.596346in}}{\pgfqpoint{3.861746in}{1.592439in}}%
\pgfpathcurveto{\pgfqpoint{3.865653in}{1.588532in}}{\pgfqpoint{3.870952in}{1.586337in}}{\pgfqpoint{3.876477in}{1.586337in}}%
\pgfpathclose%
\pgfusepath{stroke,fill}%
\end{pgfscope}%
\begin{pgfscope}%
\pgfsetbuttcap%
\pgfsetroundjoin%
\definecolor{currentfill}{rgb}{0.000000,0.000000,0.000000}%
\pgfsetfillcolor{currentfill}%
\pgfsetlinewidth{0.803000pt}%
\definecolor{currentstroke}{rgb}{0.000000,0.000000,0.000000}%
\pgfsetstrokecolor{currentstroke}%
\pgfsetdash{}{0pt}%
\pgfsys@defobject{currentmarker}{\pgfqpoint{0.000000in}{-0.048611in}}{\pgfqpoint{0.000000in}{0.000000in}}{%
\pgfpathmoveto{\pgfqpoint{0.000000in}{0.000000in}}%
\pgfpathlineto{\pgfqpoint{0.000000in}{-0.048611in}}%
\pgfusepath{stroke,fill}%
}%
\begin{pgfscope}%
\pgfsys@transformshift{0.721249in}{0.275000in}%
\pgfsys@useobject{currentmarker}{}%
\end{pgfscope}%
\end{pgfscope}%
\begin{pgfscope}%
\definecolor{textcolor}{rgb}{0.000000,0.000000,0.000000}%
\pgfsetstrokecolor{textcolor}%
\pgfsetfillcolor{textcolor}%
\pgftext[x=0.721249in,y=0.177778in,,top]{\color{textcolor}\sffamily\fontsize{10.000000}{12.000000}\selectfont 20}%
\end{pgfscope}%
\begin{pgfscope}%
\pgfsetbuttcap%
\pgfsetroundjoin%
\definecolor{currentfill}{rgb}{0.000000,0.000000,0.000000}%
\pgfsetfillcolor{currentfill}%
\pgfsetlinewidth{0.803000pt}%
\definecolor{currentstroke}{rgb}{0.000000,0.000000,0.000000}%
\pgfsetstrokecolor{currentstroke}%
\pgfsetdash{}{0pt}%
\pgfsys@defobject{currentmarker}{\pgfqpoint{0.000000in}{-0.048611in}}{\pgfqpoint{0.000000in}{0.000000in}}{%
\pgfpathmoveto{\pgfqpoint{0.000000in}{0.000000in}}%
\pgfpathlineto{\pgfqpoint{0.000000in}{-0.048611in}}%
\pgfusepath{stroke,fill}%
}%
\begin{pgfscope}%
\pgfsys@transformshift{1.772992in}{0.275000in}%
\pgfsys@useobject{currentmarker}{}%
\end{pgfscope}%
\end{pgfscope}%
\begin{pgfscope}%
\definecolor{textcolor}{rgb}{0.000000,0.000000,0.000000}%
\pgfsetstrokecolor{textcolor}%
\pgfsetfillcolor{textcolor}%
\pgftext[x=1.772992in,y=0.177778in,,top]{\color{textcolor}\sffamily\fontsize{10.000000}{12.000000}\selectfont 40}%
\end{pgfscope}%
\begin{pgfscope}%
\pgfsetbuttcap%
\pgfsetroundjoin%
\definecolor{currentfill}{rgb}{0.000000,0.000000,0.000000}%
\pgfsetfillcolor{currentfill}%
\pgfsetlinewidth{0.803000pt}%
\definecolor{currentstroke}{rgb}{0.000000,0.000000,0.000000}%
\pgfsetstrokecolor{currentstroke}%
\pgfsetdash{}{0pt}%
\pgfsys@defobject{currentmarker}{\pgfqpoint{0.000000in}{-0.048611in}}{\pgfqpoint{0.000000in}{0.000000in}}{%
\pgfpathmoveto{\pgfqpoint{0.000000in}{0.000000in}}%
\pgfpathlineto{\pgfqpoint{0.000000in}{-0.048611in}}%
\pgfusepath{stroke,fill}%
}%
\begin{pgfscope}%
\pgfsys@transformshift{2.824734in}{0.275000in}%
\pgfsys@useobject{currentmarker}{}%
\end{pgfscope}%
\end{pgfscope}%
\begin{pgfscope}%
\definecolor{textcolor}{rgb}{0.000000,0.000000,0.000000}%
\pgfsetstrokecolor{textcolor}%
\pgfsetfillcolor{textcolor}%
\pgftext[x=2.824734in,y=0.177778in,,top]{\color{textcolor}\sffamily\fontsize{10.000000}{12.000000}\selectfont 60}%
\end{pgfscope}%
\begin{pgfscope}%
\pgfsetbuttcap%
\pgfsetroundjoin%
\definecolor{currentfill}{rgb}{0.000000,0.000000,0.000000}%
\pgfsetfillcolor{currentfill}%
\pgfsetlinewidth{0.803000pt}%
\definecolor{currentstroke}{rgb}{0.000000,0.000000,0.000000}%
\pgfsetstrokecolor{currentstroke}%
\pgfsetdash{}{0pt}%
\pgfsys@defobject{currentmarker}{\pgfqpoint{0.000000in}{-0.048611in}}{\pgfqpoint{0.000000in}{0.000000in}}{%
\pgfpathmoveto{\pgfqpoint{0.000000in}{0.000000in}}%
\pgfpathlineto{\pgfqpoint{0.000000in}{-0.048611in}}%
\pgfusepath{stroke,fill}%
}%
\begin{pgfscope}%
\pgfsys@transformshift{3.876477in}{0.275000in}%
\pgfsys@useobject{currentmarker}{}%
\end{pgfscope}%
\end{pgfscope}%
\begin{pgfscope}%
\definecolor{textcolor}{rgb}{0.000000,0.000000,0.000000}%
\pgfsetstrokecolor{textcolor}%
\pgfsetfillcolor{textcolor}%
\pgftext[x=3.876477in,y=0.177778in,,top]{\color{textcolor}\sffamily\fontsize{10.000000}{12.000000}\selectfont 80}%
\end{pgfscope}%
\begin{pgfscope}%
\definecolor{textcolor}{rgb}{0.000000,0.000000,0.000000}%
\pgfsetstrokecolor{textcolor}%
\pgfsetfillcolor{textcolor}%
\pgftext[x=2.306250in,y=-0.012191in,,top]{\color{textcolor}\sffamily\fontsize{10.000000}{12.000000}\selectfont \(\displaystyle k\)}%
\end{pgfscope}%
\begin{pgfscope}%
\pgfsetbuttcap%
\pgfsetroundjoin%
\definecolor{currentfill}{rgb}{0.000000,0.000000,0.000000}%
\pgfsetfillcolor{currentfill}%
\pgfsetlinewidth{0.803000pt}%
\definecolor{currentstroke}{rgb}{0.000000,0.000000,0.000000}%
\pgfsetstrokecolor{currentstroke}%
\pgfsetdash{}{0pt}%
\pgfsys@defobject{currentmarker}{\pgfqpoint{-0.048611in}{0.000000in}}{\pgfqpoint{0.000000in}{0.000000in}}{%
\pgfpathmoveto{\pgfqpoint{0.000000in}{0.000000in}}%
\pgfpathlineto{\pgfqpoint{-0.048611in}{0.000000in}}%
\pgfusepath{stroke,fill}%
}%
\begin{pgfscope}%
\pgfsys@transformshift{0.562500in}{0.271709in}%
\pgfsys@useobject{currentmarker}{}%
\end{pgfscope}%
\end{pgfscope}%
\begin{pgfscope}%
\definecolor{textcolor}{rgb}{0.000000,0.000000,0.000000}%
\pgfsetstrokecolor{textcolor}%
\pgfsetfillcolor{textcolor}%
\pgftext[x=0.376912in,y=0.218947in,left,base]{\color{textcolor}\sffamily\fontsize{10.000000}{12.000000}\selectfont 0}%
\end{pgfscope}%
\begin{pgfscope}%
\pgfsetbuttcap%
\pgfsetroundjoin%
\definecolor{currentfill}{rgb}{0.000000,0.000000,0.000000}%
\pgfsetfillcolor{currentfill}%
\pgfsetlinewidth{0.803000pt}%
\definecolor{currentstroke}{rgb}{0.000000,0.000000,0.000000}%
\pgfsetstrokecolor{currentstroke}%
\pgfsetdash{}{0pt}%
\pgfsys@defobject{currentmarker}{\pgfqpoint{-0.048611in}{0.000000in}}{\pgfqpoint{0.000000in}{0.000000in}}{%
\pgfpathmoveto{\pgfqpoint{0.000000in}{0.000000in}}%
\pgfpathlineto{\pgfqpoint{-0.048611in}{0.000000in}}%
\pgfusepath{stroke,fill}%
}%
\begin{pgfscope}%
\pgfsys@transformshift{0.562500in}{0.474052in}%
\pgfsys@useobject{currentmarker}{}%
\end{pgfscope}%
\end{pgfscope}%
\begin{pgfscope}%
\definecolor{textcolor}{rgb}{0.000000,0.000000,0.000000}%
\pgfsetstrokecolor{textcolor}%
\pgfsetfillcolor{textcolor}%
\pgftext[x=0.288547in,y=0.421290in,left,base]{\color{textcolor}\sffamily\fontsize{10.000000}{12.000000}\selectfont 40}%
\end{pgfscope}%
\begin{pgfscope}%
\pgfsetbuttcap%
\pgfsetroundjoin%
\definecolor{currentfill}{rgb}{0.000000,0.000000,0.000000}%
\pgfsetfillcolor{currentfill}%
\pgfsetlinewidth{0.803000pt}%
\definecolor{currentstroke}{rgb}{0.000000,0.000000,0.000000}%
\pgfsetstrokecolor{currentstroke}%
\pgfsetdash{}{0pt}%
\pgfsys@defobject{currentmarker}{\pgfqpoint{-0.048611in}{0.000000in}}{\pgfqpoint{0.000000in}{0.000000in}}{%
\pgfpathmoveto{\pgfqpoint{0.000000in}{0.000000in}}%
\pgfpathlineto{\pgfqpoint{-0.048611in}{0.000000in}}%
\pgfusepath{stroke,fill}%
}%
\begin{pgfscope}%
\pgfsys@transformshift{0.562500in}{0.676394in}%
\pgfsys@useobject{currentmarker}{}%
\end{pgfscope}%
\end{pgfscope}%
\begin{pgfscope}%
\definecolor{textcolor}{rgb}{0.000000,0.000000,0.000000}%
\pgfsetstrokecolor{textcolor}%
\pgfsetfillcolor{textcolor}%
\pgftext[x=0.288547in,y=0.623633in,left,base]{\color{textcolor}\sffamily\fontsize{10.000000}{12.000000}\selectfont 80}%
\end{pgfscope}%
\begin{pgfscope}%
\pgfsetbuttcap%
\pgfsetroundjoin%
\definecolor{currentfill}{rgb}{0.000000,0.000000,0.000000}%
\pgfsetfillcolor{currentfill}%
\pgfsetlinewidth{0.803000pt}%
\definecolor{currentstroke}{rgb}{0.000000,0.000000,0.000000}%
\pgfsetstrokecolor{currentstroke}%
\pgfsetdash{}{0pt}%
\pgfsys@defobject{currentmarker}{\pgfqpoint{-0.048611in}{0.000000in}}{\pgfqpoint{0.000000in}{0.000000in}}{%
\pgfpathmoveto{\pgfqpoint{0.000000in}{0.000000in}}%
\pgfpathlineto{\pgfqpoint{-0.048611in}{0.000000in}}%
\pgfusepath{stroke,fill}%
}%
\begin{pgfscope}%
\pgfsys@transformshift{0.562500in}{0.878737in}%
\pgfsys@useobject{currentmarker}{}%
\end{pgfscope}%
\end{pgfscope}%
\begin{pgfscope}%
\definecolor{textcolor}{rgb}{0.000000,0.000000,0.000000}%
\pgfsetstrokecolor{textcolor}%
\pgfsetfillcolor{textcolor}%
\pgftext[x=0.200182in,y=0.825975in,left,base]{\color{textcolor}\sffamily\fontsize{10.000000}{12.000000}\selectfont 120}%
\end{pgfscope}%
\begin{pgfscope}%
\pgfsetbuttcap%
\pgfsetroundjoin%
\definecolor{currentfill}{rgb}{0.000000,0.000000,0.000000}%
\pgfsetfillcolor{currentfill}%
\pgfsetlinewidth{0.803000pt}%
\definecolor{currentstroke}{rgb}{0.000000,0.000000,0.000000}%
\pgfsetstrokecolor{currentstroke}%
\pgfsetdash{}{0pt}%
\pgfsys@defobject{currentmarker}{\pgfqpoint{-0.048611in}{0.000000in}}{\pgfqpoint{0.000000in}{0.000000in}}{%
\pgfpathmoveto{\pgfqpoint{0.000000in}{0.000000in}}%
\pgfpathlineto{\pgfqpoint{-0.048611in}{0.000000in}}%
\pgfusepath{stroke,fill}%
}%
\begin{pgfscope}%
\pgfsys@transformshift{0.562500in}{1.081080in}%
\pgfsys@useobject{currentmarker}{}%
\end{pgfscope}%
\end{pgfscope}%
\begin{pgfscope}%
\definecolor{textcolor}{rgb}{0.000000,0.000000,0.000000}%
\pgfsetstrokecolor{textcolor}%
\pgfsetfillcolor{textcolor}%
\pgftext[x=0.200182in,y=1.028318in,left,base]{\color{textcolor}\sffamily\fontsize{10.000000}{12.000000}\selectfont 160}%
\end{pgfscope}%
\begin{pgfscope}%
\pgfsetbuttcap%
\pgfsetroundjoin%
\definecolor{currentfill}{rgb}{0.000000,0.000000,0.000000}%
\pgfsetfillcolor{currentfill}%
\pgfsetlinewidth{0.803000pt}%
\definecolor{currentstroke}{rgb}{0.000000,0.000000,0.000000}%
\pgfsetstrokecolor{currentstroke}%
\pgfsetdash{}{0pt}%
\pgfsys@defobject{currentmarker}{\pgfqpoint{-0.048611in}{0.000000in}}{\pgfqpoint{0.000000in}{0.000000in}}{%
\pgfpathmoveto{\pgfqpoint{0.000000in}{0.000000in}}%
\pgfpathlineto{\pgfqpoint{-0.048611in}{0.000000in}}%
\pgfusepath{stroke,fill}%
}%
\begin{pgfscope}%
\pgfsys@transformshift{0.562500in}{1.283422in}%
\pgfsys@useobject{currentmarker}{}%
\end{pgfscope}%
\end{pgfscope}%
\begin{pgfscope}%
\definecolor{textcolor}{rgb}{0.000000,0.000000,0.000000}%
\pgfsetstrokecolor{textcolor}%
\pgfsetfillcolor{textcolor}%
\pgftext[x=0.200182in,y=1.230661in,left,base]{\color{textcolor}\sffamily\fontsize{10.000000}{12.000000}\selectfont 200}%
\end{pgfscope}%
\begin{pgfscope}%
\pgfsetbuttcap%
\pgfsetroundjoin%
\definecolor{currentfill}{rgb}{0.000000,0.000000,0.000000}%
\pgfsetfillcolor{currentfill}%
\pgfsetlinewidth{0.803000pt}%
\definecolor{currentstroke}{rgb}{0.000000,0.000000,0.000000}%
\pgfsetstrokecolor{currentstroke}%
\pgfsetdash{}{0pt}%
\pgfsys@defobject{currentmarker}{\pgfqpoint{-0.048611in}{0.000000in}}{\pgfqpoint{0.000000in}{0.000000in}}{%
\pgfpathmoveto{\pgfqpoint{0.000000in}{0.000000in}}%
\pgfpathlineto{\pgfqpoint{-0.048611in}{0.000000in}}%
\pgfusepath{stroke,fill}%
}%
\begin{pgfscope}%
\pgfsys@transformshift{0.562500in}{1.485765in}%
\pgfsys@useobject{currentmarker}{}%
\end{pgfscope}%
\end{pgfscope}%
\begin{pgfscope}%
\definecolor{textcolor}{rgb}{0.000000,0.000000,0.000000}%
\pgfsetstrokecolor{textcolor}%
\pgfsetfillcolor{textcolor}%
\pgftext[x=0.200182in,y=1.433003in,left,base]{\color{textcolor}\sffamily\fontsize{10.000000}{12.000000}\selectfont 240}%
\end{pgfscope}%
\begin{pgfscope}%
\pgfsetbuttcap%
\pgfsetroundjoin%
\definecolor{currentfill}{rgb}{0.000000,0.000000,0.000000}%
\pgfsetfillcolor{currentfill}%
\pgfsetlinewidth{0.803000pt}%
\definecolor{currentstroke}{rgb}{0.000000,0.000000,0.000000}%
\pgfsetstrokecolor{currentstroke}%
\pgfsetdash{}{0pt}%
\pgfsys@defobject{currentmarker}{\pgfqpoint{-0.048611in}{0.000000in}}{\pgfqpoint{0.000000in}{0.000000in}}{%
\pgfpathmoveto{\pgfqpoint{0.000000in}{0.000000in}}%
\pgfpathlineto{\pgfqpoint{-0.048611in}{0.000000in}}%
\pgfusepath{stroke,fill}%
}%
\begin{pgfscope}%
\pgfsys@transformshift{0.562500in}{1.688107in}%
\pgfsys@useobject{currentmarker}{}%
\end{pgfscope}%
\end{pgfscope}%
\begin{pgfscope}%
\definecolor{textcolor}{rgb}{0.000000,0.000000,0.000000}%
\pgfsetstrokecolor{textcolor}%
\pgfsetfillcolor{textcolor}%
\pgftext[x=0.200182in,y=1.635346in,left,base]{\color{textcolor}\sffamily\fontsize{10.000000}{12.000000}\selectfont 280}%
\end{pgfscope}%
\begin{pgfscope}%
\pgfsetbuttcap%
\pgfsetroundjoin%
\definecolor{currentfill}{rgb}{0.000000,0.000000,0.000000}%
\pgfsetfillcolor{currentfill}%
\pgfsetlinewidth{0.803000pt}%
\definecolor{currentstroke}{rgb}{0.000000,0.000000,0.000000}%
\pgfsetstrokecolor{currentstroke}%
\pgfsetdash{}{0pt}%
\pgfsys@defobject{currentmarker}{\pgfqpoint{-0.048611in}{0.000000in}}{\pgfqpoint{0.000000in}{0.000000in}}{%
\pgfpathmoveto{\pgfqpoint{0.000000in}{0.000000in}}%
\pgfpathlineto{\pgfqpoint{-0.048611in}{0.000000in}}%
\pgfusepath{stroke,fill}%
}%
\begin{pgfscope}%
\pgfsys@transformshift{0.562500in}{1.890450in}%
\pgfsys@useobject{currentmarker}{}%
\end{pgfscope}%
\end{pgfscope}%
\begin{pgfscope}%
\definecolor{textcolor}{rgb}{0.000000,0.000000,0.000000}%
\pgfsetstrokecolor{textcolor}%
\pgfsetfillcolor{textcolor}%
\pgftext[x=0.200182in,y=1.837689in,left,base]{\color{textcolor}\sffamily\fontsize{10.000000}{12.000000}\selectfont 320}%
\end{pgfscope}%
\begin{pgfscope}%
\pgfsetbuttcap%
\pgfsetroundjoin%
\definecolor{currentfill}{rgb}{0.000000,0.000000,0.000000}%
\pgfsetfillcolor{currentfill}%
\pgfsetlinewidth{0.803000pt}%
\definecolor{currentstroke}{rgb}{0.000000,0.000000,0.000000}%
\pgfsetstrokecolor{currentstroke}%
\pgfsetdash{}{0pt}%
\pgfsys@defobject{currentmarker}{\pgfqpoint{-0.048611in}{0.000000in}}{\pgfqpoint{0.000000in}{0.000000in}}{%
\pgfpathmoveto{\pgfqpoint{0.000000in}{0.000000in}}%
\pgfpathlineto{\pgfqpoint{-0.048611in}{0.000000in}}%
\pgfusepath{stroke,fill}%
}%
\begin{pgfscope}%
\pgfsys@transformshift{0.562500in}{2.092793in}%
\pgfsys@useobject{currentmarker}{}%
\end{pgfscope}%
\end{pgfscope}%
\begin{pgfscope}%
\definecolor{textcolor}{rgb}{0.000000,0.000000,0.000000}%
\pgfsetstrokecolor{textcolor}%
\pgfsetfillcolor{textcolor}%
\pgftext[x=0.200182in,y=2.040031in,left,base]{\color{textcolor}\sffamily\fontsize{10.000000}{12.000000}\selectfont 360}%
\end{pgfscope}%
\begin{pgfscope}%
\definecolor{textcolor}{rgb}{0.000000,0.000000,0.000000}%
\pgfsetstrokecolor{textcolor}%
\pgfsetfillcolor{textcolor}%
\pgftext[x=0.144626in,y=1.237500in,,bottom,rotate=90.000000]{\color{textcolor}\sffamily\fontsize{10.000000}{12.000000}\selectfont Number of GMRES Iterations}%
\end{pgfscope}%
\begin{pgfscope}%
\pgfsetrectcap%
\pgfsetmiterjoin%
\pgfsetlinewidth{0.803000pt}%
\definecolor{currentstroke}{rgb}{0.000000,0.000000,0.000000}%
\pgfsetstrokecolor{currentstroke}%
\pgfsetdash{}{0pt}%
\pgfpathmoveto{\pgfqpoint{0.562500in}{0.275000in}}%
\pgfpathlineto{\pgfqpoint{0.562500in}{2.200000in}}%
\pgfusepath{stroke}%
\end{pgfscope}%
\begin{pgfscope}%
\pgfsetrectcap%
\pgfsetmiterjoin%
\pgfsetlinewidth{0.803000pt}%
\definecolor{currentstroke}{rgb}{0.000000,0.000000,0.000000}%
\pgfsetstrokecolor{currentstroke}%
\pgfsetdash{}{0pt}%
\pgfpathmoveto{\pgfqpoint{4.050000in}{0.275000in}}%
\pgfpathlineto{\pgfqpoint{4.050000in}{2.200000in}}%
\pgfusepath{stroke}%
\end{pgfscope}%
\begin{pgfscope}%
\pgfsetrectcap%
\pgfsetmiterjoin%
\pgfsetlinewidth{0.803000pt}%
\definecolor{currentstroke}{rgb}{0.000000,0.000000,0.000000}%
\pgfsetstrokecolor{currentstroke}%
\pgfsetdash{}{0pt}%
\pgfpathmoveto{\pgfqpoint{0.562500in}{0.275000in}}%
\pgfpathlineto{\pgfqpoint{4.050000in}{0.275000in}}%
\pgfusepath{stroke}%
\end{pgfscope}%
\begin{pgfscope}%
\pgfsetrectcap%
\pgfsetmiterjoin%
\pgfsetlinewidth{0.803000pt}%
\definecolor{currentstroke}{rgb}{0.000000,0.000000,0.000000}%
\pgfsetstrokecolor{currentstroke}%
\pgfsetdash{}{0pt}%
\pgfpathmoveto{\pgfqpoint{0.562500in}{2.200000in}}%
\pgfpathlineto{\pgfqpoint{4.050000in}{2.200000in}}%
\pgfusepath{stroke}%
\end{pgfscope}%
\end{pgfpicture}%
\makeatother%
\endgroup%

\caption{GMRES iteration counts for $\alpha = 0.5$}\label{fig:linfinityn0}
    \end{subfigure}
    
    \begin{subfigure}{\textwidth}
      \centering
%% Creator: Matplotlib, PGF backend
%%
%% To include the figure in your LaTeX document, write
%%   \input{<filename>.pgf}
%%
%% Make sure the required packages are loaded in your preamble
%%   \usepackage{pgf}
%%
%% Figures using additional raster images can only be included by \input if
%% they are in the same directory as the main LaTeX file. For loading figures
%% from other directories you can use the `import` package
%%   \usepackage{import}
%% and then include the figures with
%%   \import{<path to file>}{<filename>.pgf}
%%
%% Matplotlib used the following preamble
%%   \usepackage{fontspec}
%%   \setmainfont{DejaVuSerif.ttf}[Path=/home/owen/progs/firedrake-complex/firedrake/lib/python3.5/site-packages/matplotlib/mpl-data/fonts/ttf/]
%%   \setsansfont{DejaVuSans.ttf}[Path=/home/owen/progs/firedrake-complex/firedrake/lib/python3.5/site-packages/matplotlib/mpl-data/fonts/ttf/]
%%   \setmonofont{DejaVuSansMono.ttf}[Path=/home/owen/progs/firedrake-complex/firedrake/lib/python3.5/site-packages/matplotlib/mpl-data/fonts/ttf/]
%%
\begingroup%
\makeatletter%
\begin{pgfpicture}%
\pgfpathrectangle{\pgfpointorigin}{\pgfqpoint{6.400000in}{4.800000in}}%
\pgfusepath{use as bounding box, clip}%
\begin{pgfscope}%
\pgfsetbuttcap%
\pgfsetmiterjoin%
\definecolor{currentfill}{rgb}{1.000000,1.000000,1.000000}%
\pgfsetfillcolor{currentfill}%
\pgfsetlinewidth{0.000000pt}%
\definecolor{currentstroke}{rgb}{1.000000,1.000000,1.000000}%
\pgfsetstrokecolor{currentstroke}%
\pgfsetdash{}{0pt}%
\pgfpathmoveto{\pgfqpoint{0.000000in}{0.000000in}}%
\pgfpathlineto{\pgfqpoint{6.400000in}{0.000000in}}%
\pgfpathlineto{\pgfqpoint{6.400000in}{4.800000in}}%
\pgfpathlineto{\pgfqpoint{0.000000in}{4.800000in}}%
\pgfpathclose%
\pgfusepath{fill}%
\end{pgfscope}%
\begin{pgfscope}%
\pgfsetbuttcap%
\pgfsetmiterjoin%
\definecolor{currentfill}{rgb}{1.000000,1.000000,1.000000}%
\pgfsetfillcolor{currentfill}%
\pgfsetlinewidth{0.000000pt}%
\definecolor{currentstroke}{rgb}{0.000000,0.000000,0.000000}%
\pgfsetstrokecolor{currentstroke}%
\pgfsetstrokeopacity{0.000000}%
\pgfsetdash{}{0pt}%
\pgfpathmoveto{\pgfqpoint{0.800000in}{0.528000in}}%
\pgfpathlineto{\pgfqpoint{5.760000in}{0.528000in}}%
\pgfpathlineto{\pgfqpoint{5.760000in}{4.224000in}}%
\pgfpathlineto{\pgfqpoint{0.800000in}{4.224000in}}%
\pgfpathclose%
\pgfusepath{fill}%
\end{pgfscope}%
\begin{pgfscope}%
\pgfpathrectangle{\pgfqpoint{0.800000in}{0.528000in}}{\pgfqpoint{4.960000in}{3.696000in}}%
\pgfusepath{clip}%
\pgfsetbuttcap%
\pgfsetroundjoin%
\definecolor{currentfill}{rgb}{0.000000,0.000000,0.000000}%
\pgfsetfillcolor{currentfill}%
\pgfsetlinewidth{1.003750pt}%
\definecolor{currentstroke}{rgb}{0.000000,0.000000,0.000000}%
\pgfsetstrokecolor{currentstroke}%
\pgfsetdash{}{0pt}%
\pgfpathmoveto{\pgfqpoint{1.025906in}{0.684199in}}%
\pgfpathcurveto{\pgfqpoint{1.036956in}{0.684199in}}{\pgfqpoint{1.047555in}{0.688590in}}{\pgfqpoint{1.055369in}{0.696403in}}%
\pgfpathcurveto{\pgfqpoint{1.063182in}{0.704217in}}{\pgfqpoint{1.067573in}{0.714816in}}{\pgfqpoint{1.067573in}{0.725866in}}%
\pgfpathcurveto{\pgfqpoint{1.067573in}{0.736916in}}{\pgfqpoint{1.063182in}{0.747515in}}{\pgfqpoint{1.055369in}{0.755329in}}%
\pgfpathcurveto{\pgfqpoint{1.047555in}{0.763142in}}{\pgfqpoint{1.036956in}{0.767533in}}{\pgfqpoint{1.025906in}{0.767533in}}%
\pgfpathcurveto{\pgfqpoint{1.014856in}{0.767533in}}{\pgfqpoint{1.004257in}{0.763142in}}{\pgfqpoint{0.996443in}{0.755329in}}%
\pgfpathcurveto{\pgfqpoint{0.988630in}{0.747515in}}{\pgfqpoint{0.984239in}{0.736916in}}{\pgfqpoint{0.984239in}{0.725866in}}%
\pgfpathcurveto{\pgfqpoint{0.984239in}{0.714816in}}{\pgfqpoint{0.988630in}{0.704217in}}{\pgfqpoint{0.996443in}{0.696403in}}%
\pgfpathcurveto{\pgfqpoint{1.004257in}{0.688590in}}{\pgfqpoint{1.014856in}{0.684199in}}{\pgfqpoint{1.025906in}{0.684199in}}%
\pgfpathclose%
\pgfusepath{stroke,fill}%
\end{pgfscope}%
\begin{pgfscope}%
\pgfpathrectangle{\pgfqpoint{0.800000in}{0.528000in}}{\pgfqpoint{4.960000in}{3.696000in}}%
\pgfusepath{clip}%
\pgfsetbuttcap%
\pgfsetroundjoin%
\definecolor{currentfill}{rgb}{0.000000,0.000000,0.000000}%
\pgfsetfillcolor{currentfill}%
\pgfsetlinewidth{1.003750pt}%
\definecolor{currentstroke}{rgb}{0.000000,0.000000,0.000000}%
\pgfsetstrokecolor{currentstroke}%
\pgfsetdash{}{0pt}%
\pgfpathmoveto{\pgfqpoint{1.025906in}{0.684199in}}%
\pgfpathcurveto{\pgfqpoint{1.036956in}{0.684199in}}{\pgfqpoint{1.047555in}{0.688590in}}{\pgfqpoint{1.055369in}{0.696403in}}%
\pgfpathcurveto{\pgfqpoint{1.063182in}{0.704217in}}{\pgfqpoint{1.067573in}{0.714816in}}{\pgfqpoint{1.067573in}{0.725866in}}%
\pgfpathcurveto{\pgfqpoint{1.067573in}{0.736916in}}{\pgfqpoint{1.063182in}{0.747515in}}{\pgfqpoint{1.055369in}{0.755329in}}%
\pgfpathcurveto{\pgfqpoint{1.047555in}{0.763142in}}{\pgfqpoint{1.036956in}{0.767533in}}{\pgfqpoint{1.025906in}{0.767533in}}%
\pgfpathcurveto{\pgfqpoint{1.014856in}{0.767533in}}{\pgfqpoint{1.004257in}{0.763142in}}{\pgfqpoint{0.996443in}{0.755329in}}%
\pgfpathcurveto{\pgfqpoint{0.988630in}{0.747515in}}{\pgfqpoint{0.984239in}{0.736916in}}{\pgfqpoint{0.984239in}{0.725866in}}%
\pgfpathcurveto{\pgfqpoint{0.984239in}{0.714816in}}{\pgfqpoint{0.988630in}{0.704217in}}{\pgfqpoint{0.996443in}{0.696403in}}%
\pgfpathcurveto{\pgfqpoint{1.004257in}{0.688590in}}{\pgfqpoint{1.014856in}{0.684199in}}{\pgfqpoint{1.025906in}{0.684199in}}%
\pgfpathclose%
\pgfusepath{stroke,fill}%
\end{pgfscope}%
\begin{pgfscope}%
\pgfpathrectangle{\pgfqpoint{0.800000in}{0.528000in}}{\pgfqpoint{4.960000in}{3.696000in}}%
\pgfusepath{clip}%
\pgfsetbuttcap%
\pgfsetroundjoin%
\definecolor{currentfill}{rgb}{0.000000,0.000000,0.000000}%
\pgfsetfillcolor{currentfill}%
\pgfsetlinewidth{1.003750pt}%
\definecolor{currentstroke}{rgb}{0.000000,0.000000,0.000000}%
\pgfsetstrokecolor{currentstroke}%
\pgfsetdash{}{0pt}%
\pgfpathmoveto{\pgfqpoint{1.025906in}{0.684199in}}%
\pgfpathcurveto{\pgfqpoint{1.036956in}{0.684199in}}{\pgfqpoint{1.047555in}{0.688590in}}{\pgfqpoint{1.055369in}{0.696403in}}%
\pgfpathcurveto{\pgfqpoint{1.063182in}{0.704217in}}{\pgfqpoint{1.067573in}{0.714816in}}{\pgfqpoint{1.067573in}{0.725866in}}%
\pgfpathcurveto{\pgfqpoint{1.067573in}{0.736916in}}{\pgfqpoint{1.063182in}{0.747515in}}{\pgfqpoint{1.055369in}{0.755329in}}%
\pgfpathcurveto{\pgfqpoint{1.047555in}{0.763142in}}{\pgfqpoint{1.036956in}{0.767533in}}{\pgfqpoint{1.025906in}{0.767533in}}%
\pgfpathcurveto{\pgfqpoint{1.014856in}{0.767533in}}{\pgfqpoint{1.004257in}{0.763142in}}{\pgfqpoint{0.996443in}{0.755329in}}%
\pgfpathcurveto{\pgfqpoint{0.988630in}{0.747515in}}{\pgfqpoint{0.984239in}{0.736916in}}{\pgfqpoint{0.984239in}{0.725866in}}%
\pgfpathcurveto{\pgfqpoint{0.984239in}{0.714816in}}{\pgfqpoint{0.988630in}{0.704217in}}{\pgfqpoint{0.996443in}{0.696403in}}%
\pgfpathcurveto{\pgfqpoint{1.004257in}{0.688590in}}{\pgfqpoint{1.014856in}{0.684199in}}{\pgfqpoint{1.025906in}{0.684199in}}%
\pgfpathclose%
\pgfusepath{stroke,fill}%
\end{pgfscope}%
\begin{pgfscope}%
\pgfpathrectangle{\pgfqpoint{0.800000in}{0.528000in}}{\pgfqpoint{4.960000in}{3.696000in}}%
\pgfusepath{clip}%
\pgfsetbuttcap%
\pgfsetroundjoin%
\definecolor{currentfill}{rgb}{0.000000,0.000000,0.000000}%
\pgfsetfillcolor{currentfill}%
\pgfsetlinewidth{1.003750pt}%
\definecolor{currentstroke}{rgb}{0.000000,0.000000,0.000000}%
\pgfsetstrokecolor{currentstroke}%
\pgfsetdash{}{0pt}%
\pgfpathmoveto{\pgfqpoint{1.025906in}{0.684199in}}%
\pgfpathcurveto{\pgfqpoint{1.036956in}{0.684199in}}{\pgfqpoint{1.047555in}{0.688590in}}{\pgfqpoint{1.055369in}{0.696403in}}%
\pgfpathcurveto{\pgfqpoint{1.063182in}{0.704217in}}{\pgfqpoint{1.067573in}{0.714816in}}{\pgfqpoint{1.067573in}{0.725866in}}%
\pgfpathcurveto{\pgfqpoint{1.067573in}{0.736916in}}{\pgfqpoint{1.063182in}{0.747515in}}{\pgfqpoint{1.055369in}{0.755329in}}%
\pgfpathcurveto{\pgfqpoint{1.047555in}{0.763142in}}{\pgfqpoint{1.036956in}{0.767533in}}{\pgfqpoint{1.025906in}{0.767533in}}%
\pgfpathcurveto{\pgfqpoint{1.014856in}{0.767533in}}{\pgfqpoint{1.004257in}{0.763142in}}{\pgfqpoint{0.996443in}{0.755329in}}%
\pgfpathcurveto{\pgfqpoint{0.988630in}{0.747515in}}{\pgfqpoint{0.984239in}{0.736916in}}{\pgfqpoint{0.984239in}{0.725866in}}%
\pgfpathcurveto{\pgfqpoint{0.984239in}{0.714816in}}{\pgfqpoint{0.988630in}{0.704217in}}{\pgfqpoint{0.996443in}{0.696403in}}%
\pgfpathcurveto{\pgfqpoint{1.004257in}{0.688590in}}{\pgfqpoint{1.014856in}{0.684199in}}{\pgfqpoint{1.025906in}{0.684199in}}%
\pgfpathclose%
\pgfusepath{stroke,fill}%
\end{pgfscope}%
\begin{pgfscope}%
\pgfpathrectangle{\pgfqpoint{0.800000in}{0.528000in}}{\pgfqpoint{4.960000in}{3.696000in}}%
\pgfusepath{clip}%
\pgfsetbuttcap%
\pgfsetroundjoin%
\definecolor{currentfill}{rgb}{0.000000,0.000000,0.000000}%
\pgfsetfillcolor{currentfill}%
\pgfsetlinewidth{1.003750pt}%
\definecolor{currentstroke}{rgb}{0.000000,0.000000,0.000000}%
\pgfsetstrokecolor{currentstroke}%
\pgfsetdash{}{0pt}%
\pgfpathmoveto{\pgfqpoint{1.025906in}{0.684199in}}%
\pgfpathcurveto{\pgfqpoint{1.036956in}{0.684199in}}{\pgfqpoint{1.047555in}{0.688590in}}{\pgfqpoint{1.055369in}{0.696403in}}%
\pgfpathcurveto{\pgfqpoint{1.063182in}{0.704217in}}{\pgfqpoint{1.067573in}{0.714816in}}{\pgfqpoint{1.067573in}{0.725866in}}%
\pgfpathcurveto{\pgfqpoint{1.067573in}{0.736916in}}{\pgfqpoint{1.063182in}{0.747515in}}{\pgfqpoint{1.055369in}{0.755329in}}%
\pgfpathcurveto{\pgfqpoint{1.047555in}{0.763142in}}{\pgfqpoint{1.036956in}{0.767533in}}{\pgfqpoint{1.025906in}{0.767533in}}%
\pgfpathcurveto{\pgfqpoint{1.014856in}{0.767533in}}{\pgfqpoint{1.004257in}{0.763142in}}{\pgfqpoint{0.996443in}{0.755329in}}%
\pgfpathcurveto{\pgfqpoint{0.988630in}{0.747515in}}{\pgfqpoint{0.984239in}{0.736916in}}{\pgfqpoint{0.984239in}{0.725866in}}%
\pgfpathcurveto{\pgfqpoint{0.984239in}{0.714816in}}{\pgfqpoint{0.988630in}{0.704217in}}{\pgfqpoint{0.996443in}{0.696403in}}%
\pgfpathcurveto{\pgfqpoint{1.004257in}{0.688590in}}{\pgfqpoint{1.014856in}{0.684199in}}{\pgfqpoint{1.025906in}{0.684199in}}%
\pgfpathclose%
\pgfusepath{stroke,fill}%
\end{pgfscope}%
\begin{pgfscope}%
\pgfpathrectangle{\pgfqpoint{0.800000in}{0.528000in}}{\pgfqpoint{4.960000in}{3.696000in}}%
\pgfusepath{clip}%
\pgfsetbuttcap%
\pgfsetroundjoin%
\definecolor{currentfill}{rgb}{0.000000,0.000000,0.000000}%
\pgfsetfillcolor{currentfill}%
\pgfsetlinewidth{1.003750pt}%
\definecolor{currentstroke}{rgb}{0.000000,0.000000,0.000000}%
\pgfsetstrokecolor{currentstroke}%
\pgfsetdash{}{0pt}%
\pgfpathmoveto{\pgfqpoint{1.025906in}{0.684199in}}%
\pgfpathcurveto{\pgfqpoint{1.036956in}{0.684199in}}{\pgfqpoint{1.047555in}{0.688590in}}{\pgfqpoint{1.055369in}{0.696403in}}%
\pgfpathcurveto{\pgfqpoint{1.063182in}{0.704217in}}{\pgfqpoint{1.067573in}{0.714816in}}{\pgfqpoint{1.067573in}{0.725866in}}%
\pgfpathcurveto{\pgfqpoint{1.067573in}{0.736916in}}{\pgfqpoint{1.063182in}{0.747515in}}{\pgfqpoint{1.055369in}{0.755329in}}%
\pgfpathcurveto{\pgfqpoint{1.047555in}{0.763142in}}{\pgfqpoint{1.036956in}{0.767533in}}{\pgfqpoint{1.025906in}{0.767533in}}%
\pgfpathcurveto{\pgfqpoint{1.014856in}{0.767533in}}{\pgfqpoint{1.004257in}{0.763142in}}{\pgfqpoint{0.996443in}{0.755329in}}%
\pgfpathcurveto{\pgfqpoint{0.988630in}{0.747515in}}{\pgfqpoint{0.984239in}{0.736916in}}{\pgfqpoint{0.984239in}{0.725866in}}%
\pgfpathcurveto{\pgfqpoint{0.984239in}{0.714816in}}{\pgfqpoint{0.988630in}{0.704217in}}{\pgfqpoint{0.996443in}{0.696403in}}%
\pgfpathcurveto{\pgfqpoint{1.004257in}{0.688590in}}{\pgfqpoint{1.014856in}{0.684199in}}{\pgfqpoint{1.025906in}{0.684199in}}%
\pgfpathclose%
\pgfusepath{stroke,fill}%
\end{pgfscope}%
\begin{pgfscope}%
\pgfpathrectangle{\pgfqpoint{0.800000in}{0.528000in}}{\pgfqpoint{4.960000in}{3.696000in}}%
\pgfusepath{clip}%
\pgfsetbuttcap%
\pgfsetroundjoin%
\definecolor{currentfill}{rgb}{0.000000,0.000000,0.000000}%
\pgfsetfillcolor{currentfill}%
\pgfsetlinewidth{1.003750pt}%
\definecolor{currentstroke}{rgb}{0.000000,0.000000,0.000000}%
\pgfsetstrokecolor{currentstroke}%
\pgfsetdash{}{0pt}%
\pgfpathmoveto{\pgfqpoint{1.025906in}{0.684199in}}%
\pgfpathcurveto{\pgfqpoint{1.036956in}{0.684199in}}{\pgfqpoint{1.047555in}{0.688590in}}{\pgfqpoint{1.055369in}{0.696403in}}%
\pgfpathcurveto{\pgfqpoint{1.063182in}{0.704217in}}{\pgfqpoint{1.067573in}{0.714816in}}{\pgfqpoint{1.067573in}{0.725866in}}%
\pgfpathcurveto{\pgfqpoint{1.067573in}{0.736916in}}{\pgfqpoint{1.063182in}{0.747515in}}{\pgfqpoint{1.055369in}{0.755329in}}%
\pgfpathcurveto{\pgfqpoint{1.047555in}{0.763142in}}{\pgfqpoint{1.036956in}{0.767533in}}{\pgfqpoint{1.025906in}{0.767533in}}%
\pgfpathcurveto{\pgfqpoint{1.014856in}{0.767533in}}{\pgfqpoint{1.004257in}{0.763142in}}{\pgfqpoint{0.996443in}{0.755329in}}%
\pgfpathcurveto{\pgfqpoint{0.988630in}{0.747515in}}{\pgfqpoint{0.984239in}{0.736916in}}{\pgfqpoint{0.984239in}{0.725866in}}%
\pgfpathcurveto{\pgfqpoint{0.984239in}{0.714816in}}{\pgfqpoint{0.988630in}{0.704217in}}{\pgfqpoint{0.996443in}{0.696403in}}%
\pgfpathcurveto{\pgfqpoint{1.004257in}{0.688590in}}{\pgfqpoint{1.014856in}{0.684199in}}{\pgfqpoint{1.025906in}{0.684199in}}%
\pgfpathclose%
\pgfusepath{stroke,fill}%
\end{pgfscope}%
\begin{pgfscope}%
\pgfpathrectangle{\pgfqpoint{0.800000in}{0.528000in}}{\pgfqpoint{4.960000in}{3.696000in}}%
\pgfusepath{clip}%
\pgfsetbuttcap%
\pgfsetroundjoin%
\definecolor{currentfill}{rgb}{0.000000,0.000000,0.000000}%
\pgfsetfillcolor{currentfill}%
\pgfsetlinewidth{1.003750pt}%
\definecolor{currentstroke}{rgb}{0.000000,0.000000,0.000000}%
\pgfsetstrokecolor{currentstroke}%
\pgfsetdash{}{0pt}%
\pgfpathmoveto{\pgfqpoint{1.025906in}{0.684199in}}%
\pgfpathcurveto{\pgfqpoint{1.036956in}{0.684199in}}{\pgfqpoint{1.047555in}{0.688590in}}{\pgfqpoint{1.055369in}{0.696403in}}%
\pgfpathcurveto{\pgfqpoint{1.063182in}{0.704217in}}{\pgfqpoint{1.067573in}{0.714816in}}{\pgfqpoint{1.067573in}{0.725866in}}%
\pgfpathcurveto{\pgfqpoint{1.067573in}{0.736916in}}{\pgfqpoint{1.063182in}{0.747515in}}{\pgfqpoint{1.055369in}{0.755329in}}%
\pgfpathcurveto{\pgfqpoint{1.047555in}{0.763142in}}{\pgfqpoint{1.036956in}{0.767533in}}{\pgfqpoint{1.025906in}{0.767533in}}%
\pgfpathcurveto{\pgfqpoint{1.014856in}{0.767533in}}{\pgfqpoint{1.004257in}{0.763142in}}{\pgfqpoint{0.996443in}{0.755329in}}%
\pgfpathcurveto{\pgfqpoint{0.988630in}{0.747515in}}{\pgfqpoint{0.984239in}{0.736916in}}{\pgfqpoint{0.984239in}{0.725866in}}%
\pgfpathcurveto{\pgfqpoint{0.984239in}{0.714816in}}{\pgfqpoint{0.988630in}{0.704217in}}{\pgfqpoint{0.996443in}{0.696403in}}%
\pgfpathcurveto{\pgfqpoint{1.004257in}{0.688590in}}{\pgfqpoint{1.014856in}{0.684199in}}{\pgfqpoint{1.025906in}{0.684199in}}%
\pgfpathclose%
\pgfusepath{stroke,fill}%
\end{pgfscope}%
\begin{pgfscope}%
\pgfpathrectangle{\pgfqpoint{0.800000in}{0.528000in}}{\pgfqpoint{4.960000in}{3.696000in}}%
\pgfusepath{clip}%
\pgfsetbuttcap%
\pgfsetroundjoin%
\definecolor{currentfill}{rgb}{0.000000,0.000000,0.000000}%
\pgfsetfillcolor{currentfill}%
\pgfsetlinewidth{1.003750pt}%
\definecolor{currentstroke}{rgb}{0.000000,0.000000,0.000000}%
\pgfsetstrokecolor{currentstroke}%
\pgfsetdash{}{0pt}%
\pgfpathmoveto{\pgfqpoint{1.025906in}{0.684199in}}%
\pgfpathcurveto{\pgfqpoint{1.036956in}{0.684199in}}{\pgfqpoint{1.047555in}{0.688590in}}{\pgfqpoint{1.055369in}{0.696403in}}%
\pgfpathcurveto{\pgfqpoint{1.063182in}{0.704217in}}{\pgfqpoint{1.067573in}{0.714816in}}{\pgfqpoint{1.067573in}{0.725866in}}%
\pgfpathcurveto{\pgfqpoint{1.067573in}{0.736916in}}{\pgfqpoint{1.063182in}{0.747515in}}{\pgfqpoint{1.055369in}{0.755329in}}%
\pgfpathcurveto{\pgfqpoint{1.047555in}{0.763142in}}{\pgfqpoint{1.036956in}{0.767533in}}{\pgfqpoint{1.025906in}{0.767533in}}%
\pgfpathcurveto{\pgfqpoint{1.014856in}{0.767533in}}{\pgfqpoint{1.004257in}{0.763142in}}{\pgfqpoint{0.996443in}{0.755329in}}%
\pgfpathcurveto{\pgfqpoint{0.988630in}{0.747515in}}{\pgfqpoint{0.984239in}{0.736916in}}{\pgfqpoint{0.984239in}{0.725866in}}%
\pgfpathcurveto{\pgfqpoint{0.984239in}{0.714816in}}{\pgfqpoint{0.988630in}{0.704217in}}{\pgfqpoint{0.996443in}{0.696403in}}%
\pgfpathcurveto{\pgfqpoint{1.004257in}{0.688590in}}{\pgfqpoint{1.014856in}{0.684199in}}{\pgfqpoint{1.025906in}{0.684199in}}%
\pgfpathclose%
\pgfusepath{stroke,fill}%
\end{pgfscope}%
\begin{pgfscope}%
\pgfpathrectangle{\pgfqpoint{0.800000in}{0.528000in}}{\pgfqpoint{4.960000in}{3.696000in}}%
\pgfusepath{clip}%
\pgfsetbuttcap%
\pgfsetroundjoin%
\definecolor{currentfill}{rgb}{0.000000,0.000000,0.000000}%
\pgfsetfillcolor{currentfill}%
\pgfsetlinewidth{1.003750pt}%
\definecolor{currentstroke}{rgb}{0.000000,0.000000,0.000000}%
\pgfsetstrokecolor{currentstroke}%
\pgfsetdash{}{0pt}%
\pgfpathmoveto{\pgfqpoint{1.025906in}{0.684199in}}%
\pgfpathcurveto{\pgfqpoint{1.036956in}{0.684199in}}{\pgfqpoint{1.047555in}{0.688590in}}{\pgfqpoint{1.055369in}{0.696403in}}%
\pgfpathcurveto{\pgfqpoint{1.063182in}{0.704217in}}{\pgfqpoint{1.067573in}{0.714816in}}{\pgfqpoint{1.067573in}{0.725866in}}%
\pgfpathcurveto{\pgfqpoint{1.067573in}{0.736916in}}{\pgfqpoint{1.063182in}{0.747515in}}{\pgfqpoint{1.055369in}{0.755329in}}%
\pgfpathcurveto{\pgfqpoint{1.047555in}{0.763142in}}{\pgfqpoint{1.036956in}{0.767533in}}{\pgfqpoint{1.025906in}{0.767533in}}%
\pgfpathcurveto{\pgfqpoint{1.014856in}{0.767533in}}{\pgfqpoint{1.004257in}{0.763142in}}{\pgfqpoint{0.996443in}{0.755329in}}%
\pgfpathcurveto{\pgfqpoint{0.988630in}{0.747515in}}{\pgfqpoint{0.984239in}{0.736916in}}{\pgfqpoint{0.984239in}{0.725866in}}%
\pgfpathcurveto{\pgfqpoint{0.984239in}{0.714816in}}{\pgfqpoint{0.988630in}{0.704217in}}{\pgfqpoint{0.996443in}{0.696403in}}%
\pgfpathcurveto{\pgfqpoint{1.004257in}{0.688590in}}{\pgfqpoint{1.014856in}{0.684199in}}{\pgfqpoint{1.025906in}{0.684199in}}%
\pgfpathclose%
\pgfusepath{stroke,fill}%
\end{pgfscope}%
\begin{pgfscope}%
\pgfpathrectangle{\pgfqpoint{0.800000in}{0.528000in}}{\pgfqpoint{4.960000in}{3.696000in}}%
\pgfusepath{clip}%
\pgfsetbuttcap%
\pgfsetroundjoin%
\definecolor{currentfill}{rgb}{0.000000,0.000000,0.000000}%
\pgfsetfillcolor{currentfill}%
\pgfsetlinewidth{1.003750pt}%
\definecolor{currentstroke}{rgb}{0.000000,0.000000,0.000000}%
\pgfsetstrokecolor{currentstroke}%
\pgfsetdash{}{0pt}%
\pgfpathmoveto{\pgfqpoint{1.025906in}{2.334266in}}%
\pgfpathcurveto{\pgfqpoint{1.036956in}{2.334266in}}{\pgfqpoint{1.047555in}{2.338657in}}{\pgfqpoint{1.055369in}{2.346470in}}%
\pgfpathcurveto{\pgfqpoint{1.063182in}{2.354284in}}{\pgfqpoint{1.067573in}{2.364883in}}{\pgfqpoint{1.067573in}{2.375933in}}%
\pgfpathcurveto{\pgfqpoint{1.067573in}{2.386983in}}{\pgfqpoint{1.063182in}{2.397582in}}{\pgfqpoint{1.055369in}{2.405396in}}%
\pgfpathcurveto{\pgfqpoint{1.047555in}{2.413209in}}{\pgfqpoint{1.036956in}{2.417600in}}{\pgfqpoint{1.025906in}{2.417600in}}%
\pgfpathcurveto{\pgfqpoint{1.014856in}{2.417600in}}{\pgfqpoint{1.004257in}{2.413209in}}{\pgfqpoint{0.996443in}{2.405396in}}%
\pgfpathcurveto{\pgfqpoint{0.988630in}{2.397582in}}{\pgfqpoint{0.984239in}{2.386983in}}{\pgfqpoint{0.984239in}{2.375933in}}%
\pgfpathcurveto{\pgfqpoint{0.984239in}{2.364883in}}{\pgfqpoint{0.988630in}{2.354284in}}{\pgfqpoint{0.996443in}{2.346470in}}%
\pgfpathcurveto{\pgfqpoint{1.004257in}{2.338657in}}{\pgfqpoint{1.014856in}{2.334266in}}{\pgfqpoint{1.025906in}{2.334266in}}%
\pgfpathclose%
\pgfusepath{stroke,fill}%
\end{pgfscope}%
\begin{pgfscope}%
\pgfpathrectangle{\pgfqpoint{0.800000in}{0.528000in}}{\pgfqpoint{4.960000in}{3.696000in}}%
\pgfusepath{clip}%
\pgfsetbuttcap%
\pgfsetroundjoin%
\definecolor{currentfill}{rgb}{0.000000,0.000000,0.000000}%
\pgfsetfillcolor{currentfill}%
\pgfsetlinewidth{1.003750pt}%
\definecolor{currentstroke}{rgb}{0.000000,0.000000,0.000000}%
\pgfsetstrokecolor{currentstroke}%
\pgfsetdash{}{0pt}%
\pgfpathmoveto{\pgfqpoint{1.025906in}{2.334266in}}%
\pgfpathcurveto{\pgfqpoint{1.036956in}{2.334266in}}{\pgfqpoint{1.047555in}{2.338657in}}{\pgfqpoint{1.055369in}{2.346470in}}%
\pgfpathcurveto{\pgfqpoint{1.063182in}{2.354284in}}{\pgfqpoint{1.067573in}{2.364883in}}{\pgfqpoint{1.067573in}{2.375933in}}%
\pgfpathcurveto{\pgfqpoint{1.067573in}{2.386983in}}{\pgfqpoint{1.063182in}{2.397582in}}{\pgfqpoint{1.055369in}{2.405396in}}%
\pgfpathcurveto{\pgfqpoint{1.047555in}{2.413209in}}{\pgfqpoint{1.036956in}{2.417600in}}{\pgfqpoint{1.025906in}{2.417600in}}%
\pgfpathcurveto{\pgfqpoint{1.014856in}{2.417600in}}{\pgfqpoint{1.004257in}{2.413209in}}{\pgfqpoint{0.996443in}{2.405396in}}%
\pgfpathcurveto{\pgfqpoint{0.988630in}{2.397582in}}{\pgfqpoint{0.984239in}{2.386983in}}{\pgfqpoint{0.984239in}{2.375933in}}%
\pgfpathcurveto{\pgfqpoint{0.984239in}{2.364883in}}{\pgfqpoint{0.988630in}{2.354284in}}{\pgfqpoint{0.996443in}{2.346470in}}%
\pgfpathcurveto{\pgfqpoint{1.004257in}{2.338657in}}{\pgfqpoint{1.014856in}{2.334266in}}{\pgfqpoint{1.025906in}{2.334266in}}%
\pgfpathclose%
\pgfusepath{stroke,fill}%
\end{pgfscope}%
\begin{pgfscope}%
\pgfpathrectangle{\pgfqpoint{0.800000in}{0.528000in}}{\pgfqpoint{4.960000in}{3.696000in}}%
\pgfusepath{clip}%
\pgfsetbuttcap%
\pgfsetroundjoin%
\definecolor{currentfill}{rgb}{0.000000,0.000000,0.000000}%
\pgfsetfillcolor{currentfill}%
\pgfsetlinewidth{1.003750pt}%
\definecolor{currentstroke}{rgb}{0.000000,0.000000,0.000000}%
\pgfsetstrokecolor{currentstroke}%
\pgfsetdash{}{0pt}%
\pgfpathmoveto{\pgfqpoint{1.025906in}{0.684199in}}%
\pgfpathcurveto{\pgfqpoint{1.036956in}{0.684199in}}{\pgfqpoint{1.047555in}{0.688590in}}{\pgfqpoint{1.055369in}{0.696403in}}%
\pgfpathcurveto{\pgfqpoint{1.063182in}{0.704217in}}{\pgfqpoint{1.067573in}{0.714816in}}{\pgfqpoint{1.067573in}{0.725866in}}%
\pgfpathcurveto{\pgfqpoint{1.067573in}{0.736916in}}{\pgfqpoint{1.063182in}{0.747515in}}{\pgfqpoint{1.055369in}{0.755329in}}%
\pgfpathcurveto{\pgfqpoint{1.047555in}{0.763142in}}{\pgfqpoint{1.036956in}{0.767533in}}{\pgfqpoint{1.025906in}{0.767533in}}%
\pgfpathcurveto{\pgfqpoint{1.014856in}{0.767533in}}{\pgfqpoint{1.004257in}{0.763142in}}{\pgfqpoint{0.996443in}{0.755329in}}%
\pgfpathcurveto{\pgfqpoint{0.988630in}{0.747515in}}{\pgfqpoint{0.984239in}{0.736916in}}{\pgfqpoint{0.984239in}{0.725866in}}%
\pgfpathcurveto{\pgfqpoint{0.984239in}{0.714816in}}{\pgfqpoint{0.988630in}{0.704217in}}{\pgfqpoint{0.996443in}{0.696403in}}%
\pgfpathcurveto{\pgfqpoint{1.004257in}{0.688590in}}{\pgfqpoint{1.014856in}{0.684199in}}{\pgfqpoint{1.025906in}{0.684199in}}%
\pgfpathclose%
\pgfusepath{stroke,fill}%
\end{pgfscope}%
\begin{pgfscope}%
\pgfpathrectangle{\pgfqpoint{0.800000in}{0.528000in}}{\pgfqpoint{4.960000in}{3.696000in}}%
\pgfusepath{clip}%
\pgfsetbuttcap%
\pgfsetroundjoin%
\definecolor{currentfill}{rgb}{0.000000,0.000000,0.000000}%
\pgfsetfillcolor{currentfill}%
\pgfsetlinewidth{1.003750pt}%
\definecolor{currentstroke}{rgb}{0.000000,0.000000,0.000000}%
\pgfsetstrokecolor{currentstroke}%
\pgfsetdash{}{0pt}%
\pgfpathmoveto{\pgfqpoint{1.025906in}{0.684199in}}%
\pgfpathcurveto{\pgfqpoint{1.036956in}{0.684199in}}{\pgfqpoint{1.047555in}{0.688590in}}{\pgfqpoint{1.055369in}{0.696403in}}%
\pgfpathcurveto{\pgfqpoint{1.063182in}{0.704217in}}{\pgfqpoint{1.067573in}{0.714816in}}{\pgfqpoint{1.067573in}{0.725866in}}%
\pgfpathcurveto{\pgfqpoint{1.067573in}{0.736916in}}{\pgfqpoint{1.063182in}{0.747515in}}{\pgfqpoint{1.055369in}{0.755329in}}%
\pgfpathcurveto{\pgfqpoint{1.047555in}{0.763142in}}{\pgfqpoint{1.036956in}{0.767533in}}{\pgfqpoint{1.025906in}{0.767533in}}%
\pgfpathcurveto{\pgfqpoint{1.014856in}{0.767533in}}{\pgfqpoint{1.004257in}{0.763142in}}{\pgfqpoint{0.996443in}{0.755329in}}%
\pgfpathcurveto{\pgfqpoint{0.988630in}{0.747515in}}{\pgfqpoint{0.984239in}{0.736916in}}{\pgfqpoint{0.984239in}{0.725866in}}%
\pgfpathcurveto{\pgfqpoint{0.984239in}{0.714816in}}{\pgfqpoint{0.988630in}{0.704217in}}{\pgfqpoint{0.996443in}{0.696403in}}%
\pgfpathcurveto{\pgfqpoint{1.004257in}{0.688590in}}{\pgfqpoint{1.014856in}{0.684199in}}{\pgfqpoint{1.025906in}{0.684199in}}%
\pgfpathclose%
\pgfusepath{stroke,fill}%
\end{pgfscope}%
\begin{pgfscope}%
\pgfpathrectangle{\pgfqpoint{0.800000in}{0.528000in}}{\pgfqpoint{4.960000in}{3.696000in}}%
\pgfusepath{clip}%
\pgfsetbuttcap%
\pgfsetroundjoin%
\definecolor{currentfill}{rgb}{0.000000,0.000000,0.000000}%
\pgfsetfillcolor{currentfill}%
\pgfsetlinewidth{1.003750pt}%
\definecolor{currentstroke}{rgb}{0.000000,0.000000,0.000000}%
\pgfsetstrokecolor{currentstroke}%
\pgfsetdash{}{0pt}%
\pgfpathmoveto{\pgfqpoint{1.025906in}{0.684199in}}%
\pgfpathcurveto{\pgfqpoint{1.036956in}{0.684199in}}{\pgfqpoint{1.047555in}{0.688590in}}{\pgfqpoint{1.055369in}{0.696403in}}%
\pgfpathcurveto{\pgfqpoint{1.063182in}{0.704217in}}{\pgfqpoint{1.067573in}{0.714816in}}{\pgfqpoint{1.067573in}{0.725866in}}%
\pgfpathcurveto{\pgfqpoint{1.067573in}{0.736916in}}{\pgfqpoint{1.063182in}{0.747515in}}{\pgfqpoint{1.055369in}{0.755329in}}%
\pgfpathcurveto{\pgfqpoint{1.047555in}{0.763142in}}{\pgfqpoint{1.036956in}{0.767533in}}{\pgfqpoint{1.025906in}{0.767533in}}%
\pgfpathcurveto{\pgfqpoint{1.014856in}{0.767533in}}{\pgfqpoint{1.004257in}{0.763142in}}{\pgfqpoint{0.996443in}{0.755329in}}%
\pgfpathcurveto{\pgfqpoint{0.988630in}{0.747515in}}{\pgfqpoint{0.984239in}{0.736916in}}{\pgfqpoint{0.984239in}{0.725866in}}%
\pgfpathcurveto{\pgfqpoint{0.984239in}{0.714816in}}{\pgfqpoint{0.988630in}{0.704217in}}{\pgfqpoint{0.996443in}{0.696403in}}%
\pgfpathcurveto{\pgfqpoint{1.004257in}{0.688590in}}{\pgfqpoint{1.014856in}{0.684199in}}{\pgfqpoint{1.025906in}{0.684199in}}%
\pgfpathclose%
\pgfusepath{stroke,fill}%
\end{pgfscope}%
\begin{pgfscope}%
\pgfpathrectangle{\pgfqpoint{0.800000in}{0.528000in}}{\pgfqpoint{4.960000in}{3.696000in}}%
\pgfusepath{clip}%
\pgfsetbuttcap%
\pgfsetroundjoin%
\definecolor{currentfill}{rgb}{0.000000,0.000000,0.000000}%
\pgfsetfillcolor{currentfill}%
\pgfsetlinewidth{1.003750pt}%
\definecolor{currentstroke}{rgb}{0.000000,0.000000,0.000000}%
\pgfsetstrokecolor{currentstroke}%
\pgfsetdash{}{0pt}%
\pgfpathmoveto{\pgfqpoint{1.025906in}{0.684199in}}%
\pgfpathcurveto{\pgfqpoint{1.036956in}{0.684199in}}{\pgfqpoint{1.047555in}{0.688590in}}{\pgfqpoint{1.055369in}{0.696403in}}%
\pgfpathcurveto{\pgfqpoint{1.063182in}{0.704217in}}{\pgfqpoint{1.067573in}{0.714816in}}{\pgfqpoint{1.067573in}{0.725866in}}%
\pgfpathcurveto{\pgfqpoint{1.067573in}{0.736916in}}{\pgfqpoint{1.063182in}{0.747515in}}{\pgfqpoint{1.055369in}{0.755329in}}%
\pgfpathcurveto{\pgfqpoint{1.047555in}{0.763142in}}{\pgfqpoint{1.036956in}{0.767533in}}{\pgfqpoint{1.025906in}{0.767533in}}%
\pgfpathcurveto{\pgfqpoint{1.014856in}{0.767533in}}{\pgfqpoint{1.004257in}{0.763142in}}{\pgfqpoint{0.996443in}{0.755329in}}%
\pgfpathcurveto{\pgfqpoint{0.988630in}{0.747515in}}{\pgfqpoint{0.984239in}{0.736916in}}{\pgfqpoint{0.984239in}{0.725866in}}%
\pgfpathcurveto{\pgfqpoint{0.984239in}{0.714816in}}{\pgfqpoint{0.988630in}{0.704217in}}{\pgfqpoint{0.996443in}{0.696403in}}%
\pgfpathcurveto{\pgfqpoint{1.004257in}{0.688590in}}{\pgfqpoint{1.014856in}{0.684199in}}{\pgfqpoint{1.025906in}{0.684199in}}%
\pgfpathclose%
\pgfusepath{stroke,fill}%
\end{pgfscope}%
\begin{pgfscope}%
\pgfpathrectangle{\pgfqpoint{0.800000in}{0.528000in}}{\pgfqpoint{4.960000in}{3.696000in}}%
\pgfusepath{clip}%
\pgfsetbuttcap%
\pgfsetroundjoin%
\definecolor{currentfill}{rgb}{0.000000,0.000000,0.000000}%
\pgfsetfillcolor{currentfill}%
\pgfsetlinewidth{1.003750pt}%
\definecolor{currentstroke}{rgb}{0.000000,0.000000,0.000000}%
\pgfsetstrokecolor{currentstroke}%
\pgfsetdash{}{0pt}%
\pgfpathmoveto{\pgfqpoint{1.025906in}{0.684199in}}%
\pgfpathcurveto{\pgfqpoint{1.036956in}{0.684199in}}{\pgfqpoint{1.047555in}{0.688590in}}{\pgfqpoint{1.055369in}{0.696403in}}%
\pgfpathcurveto{\pgfqpoint{1.063182in}{0.704217in}}{\pgfqpoint{1.067573in}{0.714816in}}{\pgfqpoint{1.067573in}{0.725866in}}%
\pgfpathcurveto{\pgfqpoint{1.067573in}{0.736916in}}{\pgfqpoint{1.063182in}{0.747515in}}{\pgfqpoint{1.055369in}{0.755329in}}%
\pgfpathcurveto{\pgfqpoint{1.047555in}{0.763142in}}{\pgfqpoint{1.036956in}{0.767533in}}{\pgfqpoint{1.025906in}{0.767533in}}%
\pgfpathcurveto{\pgfqpoint{1.014856in}{0.767533in}}{\pgfqpoint{1.004257in}{0.763142in}}{\pgfqpoint{0.996443in}{0.755329in}}%
\pgfpathcurveto{\pgfqpoint{0.988630in}{0.747515in}}{\pgfqpoint{0.984239in}{0.736916in}}{\pgfqpoint{0.984239in}{0.725866in}}%
\pgfpathcurveto{\pgfqpoint{0.984239in}{0.714816in}}{\pgfqpoint{0.988630in}{0.704217in}}{\pgfqpoint{0.996443in}{0.696403in}}%
\pgfpathcurveto{\pgfqpoint{1.004257in}{0.688590in}}{\pgfqpoint{1.014856in}{0.684199in}}{\pgfqpoint{1.025906in}{0.684199in}}%
\pgfpathclose%
\pgfusepath{stroke,fill}%
\end{pgfscope}%
\begin{pgfscope}%
\pgfpathrectangle{\pgfqpoint{0.800000in}{0.528000in}}{\pgfqpoint{4.960000in}{3.696000in}}%
\pgfusepath{clip}%
\pgfsetbuttcap%
\pgfsetroundjoin%
\definecolor{currentfill}{rgb}{0.000000,0.000000,0.000000}%
\pgfsetfillcolor{currentfill}%
\pgfsetlinewidth{1.003750pt}%
\definecolor{currentstroke}{rgb}{0.000000,0.000000,0.000000}%
\pgfsetstrokecolor{currentstroke}%
\pgfsetdash{}{0pt}%
\pgfpathmoveto{\pgfqpoint{1.025906in}{0.684199in}}%
\pgfpathcurveto{\pgfqpoint{1.036956in}{0.684199in}}{\pgfqpoint{1.047555in}{0.688590in}}{\pgfqpoint{1.055369in}{0.696403in}}%
\pgfpathcurveto{\pgfqpoint{1.063182in}{0.704217in}}{\pgfqpoint{1.067573in}{0.714816in}}{\pgfqpoint{1.067573in}{0.725866in}}%
\pgfpathcurveto{\pgfqpoint{1.067573in}{0.736916in}}{\pgfqpoint{1.063182in}{0.747515in}}{\pgfqpoint{1.055369in}{0.755329in}}%
\pgfpathcurveto{\pgfqpoint{1.047555in}{0.763142in}}{\pgfqpoint{1.036956in}{0.767533in}}{\pgfqpoint{1.025906in}{0.767533in}}%
\pgfpathcurveto{\pgfqpoint{1.014856in}{0.767533in}}{\pgfqpoint{1.004257in}{0.763142in}}{\pgfqpoint{0.996443in}{0.755329in}}%
\pgfpathcurveto{\pgfqpoint{0.988630in}{0.747515in}}{\pgfqpoint{0.984239in}{0.736916in}}{\pgfqpoint{0.984239in}{0.725866in}}%
\pgfpathcurveto{\pgfqpoint{0.984239in}{0.714816in}}{\pgfqpoint{0.988630in}{0.704217in}}{\pgfqpoint{0.996443in}{0.696403in}}%
\pgfpathcurveto{\pgfqpoint{1.004257in}{0.688590in}}{\pgfqpoint{1.014856in}{0.684199in}}{\pgfqpoint{1.025906in}{0.684199in}}%
\pgfpathclose%
\pgfusepath{stroke,fill}%
\end{pgfscope}%
\begin{pgfscope}%
\pgfpathrectangle{\pgfqpoint{0.800000in}{0.528000in}}{\pgfqpoint{4.960000in}{3.696000in}}%
\pgfusepath{clip}%
\pgfsetbuttcap%
\pgfsetroundjoin%
\definecolor{currentfill}{rgb}{0.000000,0.000000,0.000000}%
\pgfsetfillcolor{currentfill}%
\pgfsetlinewidth{1.003750pt}%
\definecolor{currentstroke}{rgb}{0.000000,0.000000,0.000000}%
\pgfsetstrokecolor{currentstroke}%
\pgfsetdash{}{0pt}%
\pgfpathmoveto{\pgfqpoint{1.025906in}{0.684199in}}%
\pgfpathcurveto{\pgfqpoint{1.036956in}{0.684199in}}{\pgfqpoint{1.047555in}{0.688590in}}{\pgfqpoint{1.055369in}{0.696403in}}%
\pgfpathcurveto{\pgfqpoint{1.063182in}{0.704217in}}{\pgfqpoint{1.067573in}{0.714816in}}{\pgfqpoint{1.067573in}{0.725866in}}%
\pgfpathcurveto{\pgfqpoint{1.067573in}{0.736916in}}{\pgfqpoint{1.063182in}{0.747515in}}{\pgfqpoint{1.055369in}{0.755329in}}%
\pgfpathcurveto{\pgfqpoint{1.047555in}{0.763142in}}{\pgfqpoint{1.036956in}{0.767533in}}{\pgfqpoint{1.025906in}{0.767533in}}%
\pgfpathcurveto{\pgfqpoint{1.014856in}{0.767533in}}{\pgfqpoint{1.004257in}{0.763142in}}{\pgfqpoint{0.996443in}{0.755329in}}%
\pgfpathcurveto{\pgfqpoint{0.988630in}{0.747515in}}{\pgfqpoint{0.984239in}{0.736916in}}{\pgfqpoint{0.984239in}{0.725866in}}%
\pgfpathcurveto{\pgfqpoint{0.984239in}{0.714816in}}{\pgfqpoint{0.988630in}{0.704217in}}{\pgfqpoint{0.996443in}{0.696403in}}%
\pgfpathcurveto{\pgfqpoint{1.004257in}{0.688590in}}{\pgfqpoint{1.014856in}{0.684199in}}{\pgfqpoint{1.025906in}{0.684199in}}%
\pgfpathclose%
\pgfusepath{stroke,fill}%
\end{pgfscope}%
\begin{pgfscope}%
\pgfpathrectangle{\pgfqpoint{0.800000in}{0.528000in}}{\pgfqpoint{4.960000in}{3.696000in}}%
\pgfusepath{clip}%
\pgfsetbuttcap%
\pgfsetroundjoin%
\definecolor{currentfill}{rgb}{0.000000,0.000000,0.000000}%
\pgfsetfillcolor{currentfill}%
\pgfsetlinewidth{1.003750pt}%
\definecolor{currentstroke}{rgb}{0.000000,0.000000,0.000000}%
\pgfsetstrokecolor{currentstroke}%
\pgfsetdash{}{0pt}%
\pgfpathmoveto{\pgfqpoint{1.025906in}{0.684199in}}%
\pgfpathcurveto{\pgfqpoint{1.036956in}{0.684199in}}{\pgfqpoint{1.047555in}{0.688590in}}{\pgfqpoint{1.055369in}{0.696403in}}%
\pgfpathcurveto{\pgfqpoint{1.063182in}{0.704217in}}{\pgfqpoint{1.067573in}{0.714816in}}{\pgfqpoint{1.067573in}{0.725866in}}%
\pgfpathcurveto{\pgfqpoint{1.067573in}{0.736916in}}{\pgfqpoint{1.063182in}{0.747515in}}{\pgfqpoint{1.055369in}{0.755329in}}%
\pgfpathcurveto{\pgfqpoint{1.047555in}{0.763142in}}{\pgfqpoint{1.036956in}{0.767533in}}{\pgfqpoint{1.025906in}{0.767533in}}%
\pgfpathcurveto{\pgfqpoint{1.014856in}{0.767533in}}{\pgfqpoint{1.004257in}{0.763142in}}{\pgfqpoint{0.996443in}{0.755329in}}%
\pgfpathcurveto{\pgfqpoint{0.988630in}{0.747515in}}{\pgfqpoint{0.984239in}{0.736916in}}{\pgfqpoint{0.984239in}{0.725866in}}%
\pgfpathcurveto{\pgfqpoint{0.984239in}{0.714816in}}{\pgfqpoint{0.988630in}{0.704217in}}{\pgfqpoint{0.996443in}{0.696403in}}%
\pgfpathcurveto{\pgfqpoint{1.004257in}{0.688590in}}{\pgfqpoint{1.014856in}{0.684199in}}{\pgfqpoint{1.025906in}{0.684199in}}%
\pgfpathclose%
\pgfusepath{stroke,fill}%
\end{pgfscope}%
\begin{pgfscope}%
\pgfpathrectangle{\pgfqpoint{0.800000in}{0.528000in}}{\pgfqpoint{4.960000in}{3.696000in}}%
\pgfusepath{clip}%
\pgfsetbuttcap%
\pgfsetroundjoin%
\definecolor{currentfill}{rgb}{0.000000,0.000000,0.000000}%
\pgfsetfillcolor{currentfill}%
\pgfsetlinewidth{1.003750pt}%
\definecolor{currentstroke}{rgb}{0.000000,0.000000,0.000000}%
\pgfsetstrokecolor{currentstroke}%
\pgfsetdash{}{0pt}%
\pgfpathmoveto{\pgfqpoint{1.025906in}{0.684199in}}%
\pgfpathcurveto{\pgfqpoint{1.036956in}{0.684199in}}{\pgfqpoint{1.047555in}{0.688590in}}{\pgfqpoint{1.055369in}{0.696403in}}%
\pgfpathcurveto{\pgfqpoint{1.063182in}{0.704217in}}{\pgfqpoint{1.067573in}{0.714816in}}{\pgfqpoint{1.067573in}{0.725866in}}%
\pgfpathcurveto{\pgfqpoint{1.067573in}{0.736916in}}{\pgfqpoint{1.063182in}{0.747515in}}{\pgfqpoint{1.055369in}{0.755329in}}%
\pgfpathcurveto{\pgfqpoint{1.047555in}{0.763142in}}{\pgfqpoint{1.036956in}{0.767533in}}{\pgfqpoint{1.025906in}{0.767533in}}%
\pgfpathcurveto{\pgfqpoint{1.014856in}{0.767533in}}{\pgfqpoint{1.004257in}{0.763142in}}{\pgfqpoint{0.996443in}{0.755329in}}%
\pgfpathcurveto{\pgfqpoint{0.988630in}{0.747515in}}{\pgfqpoint{0.984239in}{0.736916in}}{\pgfqpoint{0.984239in}{0.725866in}}%
\pgfpathcurveto{\pgfqpoint{0.984239in}{0.714816in}}{\pgfqpoint{0.988630in}{0.704217in}}{\pgfqpoint{0.996443in}{0.696403in}}%
\pgfpathcurveto{\pgfqpoint{1.004257in}{0.688590in}}{\pgfqpoint{1.014856in}{0.684199in}}{\pgfqpoint{1.025906in}{0.684199in}}%
\pgfpathclose%
\pgfusepath{stroke,fill}%
\end{pgfscope}%
\begin{pgfscope}%
\pgfpathrectangle{\pgfqpoint{0.800000in}{0.528000in}}{\pgfqpoint{4.960000in}{3.696000in}}%
\pgfusepath{clip}%
\pgfsetbuttcap%
\pgfsetroundjoin%
\definecolor{currentfill}{rgb}{0.000000,0.000000,0.000000}%
\pgfsetfillcolor{currentfill}%
\pgfsetlinewidth{1.003750pt}%
\definecolor{currentstroke}{rgb}{0.000000,0.000000,0.000000}%
\pgfsetstrokecolor{currentstroke}%
\pgfsetdash{}{0pt}%
\pgfpathmoveto{\pgfqpoint{1.025906in}{0.684199in}}%
\pgfpathcurveto{\pgfqpoint{1.036956in}{0.684199in}}{\pgfqpoint{1.047555in}{0.688590in}}{\pgfqpoint{1.055369in}{0.696403in}}%
\pgfpathcurveto{\pgfqpoint{1.063182in}{0.704217in}}{\pgfqpoint{1.067573in}{0.714816in}}{\pgfqpoint{1.067573in}{0.725866in}}%
\pgfpathcurveto{\pgfqpoint{1.067573in}{0.736916in}}{\pgfqpoint{1.063182in}{0.747515in}}{\pgfqpoint{1.055369in}{0.755329in}}%
\pgfpathcurveto{\pgfqpoint{1.047555in}{0.763142in}}{\pgfqpoint{1.036956in}{0.767533in}}{\pgfqpoint{1.025906in}{0.767533in}}%
\pgfpathcurveto{\pgfqpoint{1.014856in}{0.767533in}}{\pgfqpoint{1.004257in}{0.763142in}}{\pgfqpoint{0.996443in}{0.755329in}}%
\pgfpathcurveto{\pgfqpoint{0.988630in}{0.747515in}}{\pgfqpoint{0.984239in}{0.736916in}}{\pgfqpoint{0.984239in}{0.725866in}}%
\pgfpathcurveto{\pgfqpoint{0.984239in}{0.714816in}}{\pgfqpoint{0.988630in}{0.704217in}}{\pgfqpoint{0.996443in}{0.696403in}}%
\pgfpathcurveto{\pgfqpoint{1.004257in}{0.688590in}}{\pgfqpoint{1.014856in}{0.684199in}}{\pgfqpoint{1.025906in}{0.684199in}}%
\pgfpathclose%
\pgfusepath{stroke,fill}%
\end{pgfscope}%
\begin{pgfscope}%
\pgfpathrectangle{\pgfqpoint{0.800000in}{0.528000in}}{\pgfqpoint{4.960000in}{3.696000in}}%
\pgfusepath{clip}%
\pgfsetbuttcap%
\pgfsetroundjoin%
\definecolor{currentfill}{rgb}{0.000000,0.000000,0.000000}%
\pgfsetfillcolor{currentfill}%
\pgfsetlinewidth{1.003750pt}%
\definecolor{currentstroke}{rgb}{0.000000,0.000000,0.000000}%
\pgfsetstrokecolor{currentstroke}%
\pgfsetdash{}{0pt}%
\pgfpathmoveto{\pgfqpoint{1.025906in}{0.684199in}}%
\pgfpathcurveto{\pgfqpoint{1.036956in}{0.684199in}}{\pgfqpoint{1.047555in}{0.688590in}}{\pgfqpoint{1.055369in}{0.696403in}}%
\pgfpathcurveto{\pgfqpoint{1.063182in}{0.704217in}}{\pgfqpoint{1.067573in}{0.714816in}}{\pgfqpoint{1.067573in}{0.725866in}}%
\pgfpathcurveto{\pgfqpoint{1.067573in}{0.736916in}}{\pgfqpoint{1.063182in}{0.747515in}}{\pgfqpoint{1.055369in}{0.755329in}}%
\pgfpathcurveto{\pgfqpoint{1.047555in}{0.763142in}}{\pgfqpoint{1.036956in}{0.767533in}}{\pgfqpoint{1.025906in}{0.767533in}}%
\pgfpathcurveto{\pgfqpoint{1.014856in}{0.767533in}}{\pgfqpoint{1.004257in}{0.763142in}}{\pgfqpoint{0.996443in}{0.755329in}}%
\pgfpathcurveto{\pgfqpoint{0.988630in}{0.747515in}}{\pgfqpoint{0.984239in}{0.736916in}}{\pgfqpoint{0.984239in}{0.725866in}}%
\pgfpathcurveto{\pgfqpoint{0.984239in}{0.714816in}}{\pgfqpoint{0.988630in}{0.704217in}}{\pgfqpoint{0.996443in}{0.696403in}}%
\pgfpathcurveto{\pgfqpoint{1.004257in}{0.688590in}}{\pgfqpoint{1.014856in}{0.684199in}}{\pgfqpoint{1.025906in}{0.684199in}}%
\pgfpathclose%
\pgfusepath{stroke,fill}%
\end{pgfscope}%
\begin{pgfscope}%
\pgfpathrectangle{\pgfqpoint{0.800000in}{0.528000in}}{\pgfqpoint{4.960000in}{3.696000in}}%
\pgfusepath{clip}%
\pgfsetbuttcap%
\pgfsetroundjoin%
\definecolor{currentfill}{rgb}{0.000000,0.000000,0.000000}%
\pgfsetfillcolor{currentfill}%
\pgfsetlinewidth{1.003750pt}%
\definecolor{currentstroke}{rgb}{0.000000,0.000000,0.000000}%
\pgfsetstrokecolor{currentstroke}%
\pgfsetdash{}{0pt}%
\pgfpathmoveto{\pgfqpoint{1.025906in}{2.334266in}}%
\pgfpathcurveto{\pgfqpoint{1.036956in}{2.334266in}}{\pgfqpoint{1.047555in}{2.338657in}}{\pgfqpoint{1.055369in}{2.346470in}}%
\pgfpathcurveto{\pgfqpoint{1.063182in}{2.354284in}}{\pgfqpoint{1.067573in}{2.364883in}}{\pgfqpoint{1.067573in}{2.375933in}}%
\pgfpathcurveto{\pgfqpoint{1.067573in}{2.386983in}}{\pgfqpoint{1.063182in}{2.397582in}}{\pgfqpoint{1.055369in}{2.405396in}}%
\pgfpathcurveto{\pgfqpoint{1.047555in}{2.413209in}}{\pgfqpoint{1.036956in}{2.417600in}}{\pgfqpoint{1.025906in}{2.417600in}}%
\pgfpathcurveto{\pgfqpoint{1.014856in}{2.417600in}}{\pgfqpoint{1.004257in}{2.413209in}}{\pgfqpoint{0.996443in}{2.405396in}}%
\pgfpathcurveto{\pgfqpoint{0.988630in}{2.397582in}}{\pgfqpoint{0.984239in}{2.386983in}}{\pgfqpoint{0.984239in}{2.375933in}}%
\pgfpathcurveto{\pgfqpoint{0.984239in}{2.364883in}}{\pgfqpoint{0.988630in}{2.354284in}}{\pgfqpoint{0.996443in}{2.346470in}}%
\pgfpathcurveto{\pgfqpoint{1.004257in}{2.338657in}}{\pgfqpoint{1.014856in}{2.334266in}}{\pgfqpoint{1.025906in}{2.334266in}}%
\pgfpathclose%
\pgfusepath{stroke,fill}%
\end{pgfscope}%
\begin{pgfscope}%
\pgfpathrectangle{\pgfqpoint{0.800000in}{0.528000in}}{\pgfqpoint{4.960000in}{3.696000in}}%
\pgfusepath{clip}%
\pgfsetbuttcap%
\pgfsetroundjoin%
\definecolor{currentfill}{rgb}{0.000000,0.000000,0.000000}%
\pgfsetfillcolor{currentfill}%
\pgfsetlinewidth{1.003750pt}%
\definecolor{currentstroke}{rgb}{0.000000,0.000000,0.000000}%
\pgfsetstrokecolor{currentstroke}%
\pgfsetdash{}{0pt}%
\pgfpathmoveto{\pgfqpoint{1.025906in}{0.684199in}}%
\pgfpathcurveto{\pgfqpoint{1.036956in}{0.684199in}}{\pgfqpoint{1.047555in}{0.688590in}}{\pgfqpoint{1.055369in}{0.696403in}}%
\pgfpathcurveto{\pgfqpoint{1.063182in}{0.704217in}}{\pgfqpoint{1.067573in}{0.714816in}}{\pgfqpoint{1.067573in}{0.725866in}}%
\pgfpathcurveto{\pgfqpoint{1.067573in}{0.736916in}}{\pgfqpoint{1.063182in}{0.747515in}}{\pgfqpoint{1.055369in}{0.755329in}}%
\pgfpathcurveto{\pgfqpoint{1.047555in}{0.763142in}}{\pgfqpoint{1.036956in}{0.767533in}}{\pgfqpoint{1.025906in}{0.767533in}}%
\pgfpathcurveto{\pgfqpoint{1.014856in}{0.767533in}}{\pgfqpoint{1.004257in}{0.763142in}}{\pgfqpoint{0.996443in}{0.755329in}}%
\pgfpathcurveto{\pgfqpoint{0.988630in}{0.747515in}}{\pgfqpoint{0.984239in}{0.736916in}}{\pgfqpoint{0.984239in}{0.725866in}}%
\pgfpathcurveto{\pgfqpoint{0.984239in}{0.714816in}}{\pgfqpoint{0.988630in}{0.704217in}}{\pgfqpoint{0.996443in}{0.696403in}}%
\pgfpathcurveto{\pgfqpoint{1.004257in}{0.688590in}}{\pgfqpoint{1.014856in}{0.684199in}}{\pgfqpoint{1.025906in}{0.684199in}}%
\pgfpathclose%
\pgfusepath{stroke,fill}%
\end{pgfscope}%
\begin{pgfscope}%
\pgfpathrectangle{\pgfqpoint{0.800000in}{0.528000in}}{\pgfqpoint{4.960000in}{3.696000in}}%
\pgfusepath{clip}%
\pgfsetbuttcap%
\pgfsetroundjoin%
\definecolor{currentfill}{rgb}{0.000000,0.000000,0.000000}%
\pgfsetfillcolor{currentfill}%
\pgfsetlinewidth{1.003750pt}%
\definecolor{currentstroke}{rgb}{0.000000,0.000000,0.000000}%
\pgfsetstrokecolor{currentstroke}%
\pgfsetdash{}{0pt}%
\pgfpathmoveto{\pgfqpoint{1.025906in}{0.684199in}}%
\pgfpathcurveto{\pgfqpoint{1.036956in}{0.684199in}}{\pgfqpoint{1.047555in}{0.688590in}}{\pgfqpoint{1.055369in}{0.696403in}}%
\pgfpathcurveto{\pgfqpoint{1.063182in}{0.704217in}}{\pgfqpoint{1.067573in}{0.714816in}}{\pgfqpoint{1.067573in}{0.725866in}}%
\pgfpathcurveto{\pgfqpoint{1.067573in}{0.736916in}}{\pgfqpoint{1.063182in}{0.747515in}}{\pgfqpoint{1.055369in}{0.755329in}}%
\pgfpathcurveto{\pgfqpoint{1.047555in}{0.763142in}}{\pgfqpoint{1.036956in}{0.767533in}}{\pgfqpoint{1.025906in}{0.767533in}}%
\pgfpathcurveto{\pgfqpoint{1.014856in}{0.767533in}}{\pgfqpoint{1.004257in}{0.763142in}}{\pgfqpoint{0.996443in}{0.755329in}}%
\pgfpathcurveto{\pgfqpoint{0.988630in}{0.747515in}}{\pgfqpoint{0.984239in}{0.736916in}}{\pgfqpoint{0.984239in}{0.725866in}}%
\pgfpathcurveto{\pgfqpoint{0.984239in}{0.714816in}}{\pgfqpoint{0.988630in}{0.704217in}}{\pgfqpoint{0.996443in}{0.696403in}}%
\pgfpathcurveto{\pgfqpoint{1.004257in}{0.688590in}}{\pgfqpoint{1.014856in}{0.684199in}}{\pgfqpoint{1.025906in}{0.684199in}}%
\pgfpathclose%
\pgfusepath{stroke,fill}%
\end{pgfscope}%
\begin{pgfscope}%
\pgfpathrectangle{\pgfqpoint{0.800000in}{0.528000in}}{\pgfqpoint{4.960000in}{3.696000in}}%
\pgfusepath{clip}%
\pgfsetbuttcap%
\pgfsetroundjoin%
\definecolor{currentfill}{rgb}{0.000000,0.000000,0.000000}%
\pgfsetfillcolor{currentfill}%
\pgfsetlinewidth{1.003750pt}%
\definecolor{currentstroke}{rgb}{0.000000,0.000000,0.000000}%
\pgfsetstrokecolor{currentstroke}%
\pgfsetdash{}{0pt}%
\pgfpathmoveto{\pgfqpoint{1.025906in}{0.684199in}}%
\pgfpathcurveto{\pgfqpoint{1.036956in}{0.684199in}}{\pgfqpoint{1.047555in}{0.688590in}}{\pgfqpoint{1.055369in}{0.696403in}}%
\pgfpathcurveto{\pgfqpoint{1.063182in}{0.704217in}}{\pgfqpoint{1.067573in}{0.714816in}}{\pgfqpoint{1.067573in}{0.725866in}}%
\pgfpathcurveto{\pgfqpoint{1.067573in}{0.736916in}}{\pgfqpoint{1.063182in}{0.747515in}}{\pgfqpoint{1.055369in}{0.755329in}}%
\pgfpathcurveto{\pgfqpoint{1.047555in}{0.763142in}}{\pgfqpoint{1.036956in}{0.767533in}}{\pgfqpoint{1.025906in}{0.767533in}}%
\pgfpathcurveto{\pgfqpoint{1.014856in}{0.767533in}}{\pgfqpoint{1.004257in}{0.763142in}}{\pgfqpoint{0.996443in}{0.755329in}}%
\pgfpathcurveto{\pgfqpoint{0.988630in}{0.747515in}}{\pgfqpoint{0.984239in}{0.736916in}}{\pgfqpoint{0.984239in}{0.725866in}}%
\pgfpathcurveto{\pgfqpoint{0.984239in}{0.714816in}}{\pgfqpoint{0.988630in}{0.704217in}}{\pgfqpoint{0.996443in}{0.696403in}}%
\pgfpathcurveto{\pgfqpoint{1.004257in}{0.688590in}}{\pgfqpoint{1.014856in}{0.684199in}}{\pgfqpoint{1.025906in}{0.684199in}}%
\pgfpathclose%
\pgfusepath{stroke,fill}%
\end{pgfscope}%
\begin{pgfscope}%
\pgfpathrectangle{\pgfqpoint{0.800000in}{0.528000in}}{\pgfqpoint{4.960000in}{3.696000in}}%
\pgfusepath{clip}%
\pgfsetbuttcap%
\pgfsetroundjoin%
\definecolor{currentfill}{rgb}{0.000000,0.000000,0.000000}%
\pgfsetfillcolor{currentfill}%
\pgfsetlinewidth{1.003750pt}%
\definecolor{currentstroke}{rgb}{0.000000,0.000000,0.000000}%
\pgfsetstrokecolor{currentstroke}%
\pgfsetdash{}{0pt}%
\pgfpathmoveto{\pgfqpoint{1.025906in}{0.684199in}}%
\pgfpathcurveto{\pgfqpoint{1.036956in}{0.684199in}}{\pgfqpoint{1.047555in}{0.688590in}}{\pgfqpoint{1.055369in}{0.696403in}}%
\pgfpathcurveto{\pgfqpoint{1.063182in}{0.704217in}}{\pgfqpoint{1.067573in}{0.714816in}}{\pgfqpoint{1.067573in}{0.725866in}}%
\pgfpathcurveto{\pgfqpoint{1.067573in}{0.736916in}}{\pgfqpoint{1.063182in}{0.747515in}}{\pgfqpoint{1.055369in}{0.755329in}}%
\pgfpathcurveto{\pgfqpoint{1.047555in}{0.763142in}}{\pgfqpoint{1.036956in}{0.767533in}}{\pgfqpoint{1.025906in}{0.767533in}}%
\pgfpathcurveto{\pgfqpoint{1.014856in}{0.767533in}}{\pgfqpoint{1.004257in}{0.763142in}}{\pgfqpoint{0.996443in}{0.755329in}}%
\pgfpathcurveto{\pgfqpoint{0.988630in}{0.747515in}}{\pgfqpoint{0.984239in}{0.736916in}}{\pgfqpoint{0.984239in}{0.725866in}}%
\pgfpathcurveto{\pgfqpoint{0.984239in}{0.714816in}}{\pgfqpoint{0.988630in}{0.704217in}}{\pgfqpoint{0.996443in}{0.696403in}}%
\pgfpathcurveto{\pgfqpoint{1.004257in}{0.688590in}}{\pgfqpoint{1.014856in}{0.684199in}}{\pgfqpoint{1.025906in}{0.684199in}}%
\pgfpathclose%
\pgfusepath{stroke,fill}%
\end{pgfscope}%
\begin{pgfscope}%
\pgfpathrectangle{\pgfqpoint{0.800000in}{0.528000in}}{\pgfqpoint{4.960000in}{3.696000in}}%
\pgfusepath{clip}%
\pgfsetbuttcap%
\pgfsetroundjoin%
\definecolor{currentfill}{rgb}{0.000000,0.000000,0.000000}%
\pgfsetfillcolor{currentfill}%
\pgfsetlinewidth{1.003750pt}%
\definecolor{currentstroke}{rgb}{0.000000,0.000000,0.000000}%
\pgfsetstrokecolor{currentstroke}%
\pgfsetdash{}{0pt}%
\pgfpathmoveto{\pgfqpoint{1.025906in}{2.334266in}}%
\pgfpathcurveto{\pgfqpoint{1.036956in}{2.334266in}}{\pgfqpoint{1.047555in}{2.338657in}}{\pgfqpoint{1.055369in}{2.346470in}}%
\pgfpathcurveto{\pgfqpoint{1.063182in}{2.354284in}}{\pgfqpoint{1.067573in}{2.364883in}}{\pgfqpoint{1.067573in}{2.375933in}}%
\pgfpathcurveto{\pgfqpoint{1.067573in}{2.386983in}}{\pgfqpoint{1.063182in}{2.397582in}}{\pgfqpoint{1.055369in}{2.405396in}}%
\pgfpathcurveto{\pgfqpoint{1.047555in}{2.413209in}}{\pgfqpoint{1.036956in}{2.417600in}}{\pgfqpoint{1.025906in}{2.417600in}}%
\pgfpathcurveto{\pgfqpoint{1.014856in}{2.417600in}}{\pgfqpoint{1.004257in}{2.413209in}}{\pgfqpoint{0.996443in}{2.405396in}}%
\pgfpathcurveto{\pgfqpoint{0.988630in}{2.397582in}}{\pgfqpoint{0.984239in}{2.386983in}}{\pgfqpoint{0.984239in}{2.375933in}}%
\pgfpathcurveto{\pgfqpoint{0.984239in}{2.364883in}}{\pgfqpoint{0.988630in}{2.354284in}}{\pgfqpoint{0.996443in}{2.346470in}}%
\pgfpathcurveto{\pgfqpoint{1.004257in}{2.338657in}}{\pgfqpoint{1.014856in}{2.334266in}}{\pgfqpoint{1.025906in}{2.334266in}}%
\pgfpathclose%
\pgfusepath{stroke,fill}%
\end{pgfscope}%
\begin{pgfscope}%
\pgfpathrectangle{\pgfqpoint{0.800000in}{0.528000in}}{\pgfqpoint{4.960000in}{3.696000in}}%
\pgfusepath{clip}%
\pgfsetbuttcap%
\pgfsetroundjoin%
\definecolor{currentfill}{rgb}{0.000000,0.000000,0.000000}%
\pgfsetfillcolor{currentfill}%
\pgfsetlinewidth{1.003750pt}%
\definecolor{currentstroke}{rgb}{0.000000,0.000000,0.000000}%
\pgfsetstrokecolor{currentstroke}%
\pgfsetdash{}{0pt}%
\pgfpathmoveto{\pgfqpoint{1.025906in}{2.334266in}}%
\pgfpathcurveto{\pgfqpoint{1.036956in}{2.334266in}}{\pgfqpoint{1.047555in}{2.338657in}}{\pgfqpoint{1.055369in}{2.346470in}}%
\pgfpathcurveto{\pgfqpoint{1.063182in}{2.354284in}}{\pgfqpoint{1.067573in}{2.364883in}}{\pgfqpoint{1.067573in}{2.375933in}}%
\pgfpathcurveto{\pgfqpoint{1.067573in}{2.386983in}}{\pgfqpoint{1.063182in}{2.397582in}}{\pgfqpoint{1.055369in}{2.405396in}}%
\pgfpathcurveto{\pgfqpoint{1.047555in}{2.413209in}}{\pgfqpoint{1.036956in}{2.417600in}}{\pgfqpoint{1.025906in}{2.417600in}}%
\pgfpathcurveto{\pgfqpoint{1.014856in}{2.417600in}}{\pgfqpoint{1.004257in}{2.413209in}}{\pgfqpoint{0.996443in}{2.405396in}}%
\pgfpathcurveto{\pgfqpoint{0.988630in}{2.397582in}}{\pgfqpoint{0.984239in}{2.386983in}}{\pgfqpoint{0.984239in}{2.375933in}}%
\pgfpathcurveto{\pgfqpoint{0.984239in}{2.364883in}}{\pgfqpoint{0.988630in}{2.354284in}}{\pgfqpoint{0.996443in}{2.346470in}}%
\pgfpathcurveto{\pgfqpoint{1.004257in}{2.338657in}}{\pgfqpoint{1.014856in}{2.334266in}}{\pgfqpoint{1.025906in}{2.334266in}}%
\pgfpathclose%
\pgfusepath{stroke,fill}%
\end{pgfscope}%
\begin{pgfscope}%
\pgfpathrectangle{\pgfqpoint{0.800000in}{0.528000in}}{\pgfqpoint{4.960000in}{3.696000in}}%
\pgfusepath{clip}%
\pgfsetbuttcap%
\pgfsetroundjoin%
\definecolor{currentfill}{rgb}{0.000000,0.000000,0.000000}%
\pgfsetfillcolor{currentfill}%
\pgfsetlinewidth{1.003750pt}%
\definecolor{currentstroke}{rgb}{0.000000,0.000000,0.000000}%
\pgfsetstrokecolor{currentstroke}%
\pgfsetdash{}{0pt}%
\pgfpathmoveto{\pgfqpoint{1.025906in}{2.334266in}}%
\pgfpathcurveto{\pgfqpoint{1.036956in}{2.334266in}}{\pgfqpoint{1.047555in}{2.338657in}}{\pgfqpoint{1.055369in}{2.346470in}}%
\pgfpathcurveto{\pgfqpoint{1.063182in}{2.354284in}}{\pgfqpoint{1.067573in}{2.364883in}}{\pgfqpoint{1.067573in}{2.375933in}}%
\pgfpathcurveto{\pgfqpoint{1.067573in}{2.386983in}}{\pgfqpoint{1.063182in}{2.397582in}}{\pgfqpoint{1.055369in}{2.405396in}}%
\pgfpathcurveto{\pgfqpoint{1.047555in}{2.413209in}}{\pgfqpoint{1.036956in}{2.417600in}}{\pgfqpoint{1.025906in}{2.417600in}}%
\pgfpathcurveto{\pgfqpoint{1.014856in}{2.417600in}}{\pgfqpoint{1.004257in}{2.413209in}}{\pgfqpoint{0.996443in}{2.405396in}}%
\pgfpathcurveto{\pgfqpoint{0.988630in}{2.397582in}}{\pgfqpoint{0.984239in}{2.386983in}}{\pgfqpoint{0.984239in}{2.375933in}}%
\pgfpathcurveto{\pgfqpoint{0.984239in}{2.364883in}}{\pgfqpoint{0.988630in}{2.354284in}}{\pgfqpoint{0.996443in}{2.346470in}}%
\pgfpathcurveto{\pgfqpoint{1.004257in}{2.338657in}}{\pgfqpoint{1.014856in}{2.334266in}}{\pgfqpoint{1.025906in}{2.334266in}}%
\pgfpathclose%
\pgfusepath{stroke,fill}%
\end{pgfscope}%
\begin{pgfscope}%
\pgfpathrectangle{\pgfqpoint{0.800000in}{0.528000in}}{\pgfqpoint{4.960000in}{3.696000in}}%
\pgfusepath{clip}%
\pgfsetbuttcap%
\pgfsetroundjoin%
\definecolor{currentfill}{rgb}{0.000000,0.000000,0.000000}%
\pgfsetfillcolor{currentfill}%
\pgfsetlinewidth{1.003750pt}%
\definecolor{currentstroke}{rgb}{0.000000,0.000000,0.000000}%
\pgfsetstrokecolor{currentstroke}%
\pgfsetdash{}{0pt}%
\pgfpathmoveto{\pgfqpoint{1.025906in}{2.334266in}}%
\pgfpathcurveto{\pgfqpoint{1.036956in}{2.334266in}}{\pgfqpoint{1.047555in}{2.338657in}}{\pgfqpoint{1.055369in}{2.346470in}}%
\pgfpathcurveto{\pgfqpoint{1.063182in}{2.354284in}}{\pgfqpoint{1.067573in}{2.364883in}}{\pgfqpoint{1.067573in}{2.375933in}}%
\pgfpathcurveto{\pgfqpoint{1.067573in}{2.386983in}}{\pgfqpoint{1.063182in}{2.397582in}}{\pgfqpoint{1.055369in}{2.405396in}}%
\pgfpathcurveto{\pgfqpoint{1.047555in}{2.413209in}}{\pgfqpoint{1.036956in}{2.417600in}}{\pgfqpoint{1.025906in}{2.417600in}}%
\pgfpathcurveto{\pgfqpoint{1.014856in}{2.417600in}}{\pgfqpoint{1.004257in}{2.413209in}}{\pgfqpoint{0.996443in}{2.405396in}}%
\pgfpathcurveto{\pgfqpoint{0.988630in}{2.397582in}}{\pgfqpoint{0.984239in}{2.386983in}}{\pgfqpoint{0.984239in}{2.375933in}}%
\pgfpathcurveto{\pgfqpoint{0.984239in}{2.364883in}}{\pgfqpoint{0.988630in}{2.354284in}}{\pgfqpoint{0.996443in}{2.346470in}}%
\pgfpathcurveto{\pgfqpoint{1.004257in}{2.338657in}}{\pgfqpoint{1.014856in}{2.334266in}}{\pgfqpoint{1.025906in}{2.334266in}}%
\pgfpathclose%
\pgfusepath{stroke,fill}%
\end{pgfscope}%
\begin{pgfscope}%
\pgfpathrectangle{\pgfqpoint{0.800000in}{0.528000in}}{\pgfqpoint{4.960000in}{3.696000in}}%
\pgfusepath{clip}%
\pgfsetbuttcap%
\pgfsetroundjoin%
\definecolor{currentfill}{rgb}{0.000000,0.000000,0.000000}%
\pgfsetfillcolor{currentfill}%
\pgfsetlinewidth{1.003750pt}%
\definecolor{currentstroke}{rgb}{0.000000,0.000000,0.000000}%
\pgfsetstrokecolor{currentstroke}%
\pgfsetdash{}{0pt}%
\pgfpathmoveto{\pgfqpoint{1.025906in}{0.684199in}}%
\pgfpathcurveto{\pgfqpoint{1.036956in}{0.684199in}}{\pgfqpoint{1.047555in}{0.688590in}}{\pgfqpoint{1.055369in}{0.696403in}}%
\pgfpathcurveto{\pgfqpoint{1.063182in}{0.704217in}}{\pgfqpoint{1.067573in}{0.714816in}}{\pgfqpoint{1.067573in}{0.725866in}}%
\pgfpathcurveto{\pgfqpoint{1.067573in}{0.736916in}}{\pgfqpoint{1.063182in}{0.747515in}}{\pgfqpoint{1.055369in}{0.755329in}}%
\pgfpathcurveto{\pgfqpoint{1.047555in}{0.763142in}}{\pgfqpoint{1.036956in}{0.767533in}}{\pgfqpoint{1.025906in}{0.767533in}}%
\pgfpathcurveto{\pgfqpoint{1.014856in}{0.767533in}}{\pgfqpoint{1.004257in}{0.763142in}}{\pgfqpoint{0.996443in}{0.755329in}}%
\pgfpathcurveto{\pgfqpoint{0.988630in}{0.747515in}}{\pgfqpoint{0.984239in}{0.736916in}}{\pgfqpoint{0.984239in}{0.725866in}}%
\pgfpathcurveto{\pgfqpoint{0.984239in}{0.714816in}}{\pgfqpoint{0.988630in}{0.704217in}}{\pgfqpoint{0.996443in}{0.696403in}}%
\pgfpathcurveto{\pgfqpoint{1.004257in}{0.688590in}}{\pgfqpoint{1.014856in}{0.684199in}}{\pgfqpoint{1.025906in}{0.684199in}}%
\pgfpathclose%
\pgfusepath{stroke,fill}%
\end{pgfscope}%
\begin{pgfscope}%
\pgfpathrectangle{\pgfqpoint{0.800000in}{0.528000in}}{\pgfqpoint{4.960000in}{3.696000in}}%
\pgfusepath{clip}%
\pgfsetbuttcap%
\pgfsetroundjoin%
\definecolor{currentfill}{rgb}{0.000000,0.000000,0.000000}%
\pgfsetfillcolor{currentfill}%
\pgfsetlinewidth{1.003750pt}%
\definecolor{currentstroke}{rgb}{0.000000,0.000000,0.000000}%
\pgfsetstrokecolor{currentstroke}%
\pgfsetdash{}{0pt}%
\pgfpathmoveto{\pgfqpoint{1.025906in}{0.684199in}}%
\pgfpathcurveto{\pgfqpoint{1.036956in}{0.684199in}}{\pgfqpoint{1.047555in}{0.688590in}}{\pgfqpoint{1.055369in}{0.696403in}}%
\pgfpathcurveto{\pgfqpoint{1.063182in}{0.704217in}}{\pgfqpoint{1.067573in}{0.714816in}}{\pgfqpoint{1.067573in}{0.725866in}}%
\pgfpathcurveto{\pgfqpoint{1.067573in}{0.736916in}}{\pgfqpoint{1.063182in}{0.747515in}}{\pgfqpoint{1.055369in}{0.755329in}}%
\pgfpathcurveto{\pgfqpoint{1.047555in}{0.763142in}}{\pgfqpoint{1.036956in}{0.767533in}}{\pgfqpoint{1.025906in}{0.767533in}}%
\pgfpathcurveto{\pgfqpoint{1.014856in}{0.767533in}}{\pgfqpoint{1.004257in}{0.763142in}}{\pgfqpoint{0.996443in}{0.755329in}}%
\pgfpathcurveto{\pgfqpoint{0.988630in}{0.747515in}}{\pgfqpoint{0.984239in}{0.736916in}}{\pgfqpoint{0.984239in}{0.725866in}}%
\pgfpathcurveto{\pgfqpoint{0.984239in}{0.714816in}}{\pgfqpoint{0.988630in}{0.704217in}}{\pgfqpoint{0.996443in}{0.696403in}}%
\pgfpathcurveto{\pgfqpoint{1.004257in}{0.688590in}}{\pgfqpoint{1.014856in}{0.684199in}}{\pgfqpoint{1.025906in}{0.684199in}}%
\pgfpathclose%
\pgfusepath{stroke,fill}%
\end{pgfscope}%
\begin{pgfscope}%
\pgfpathrectangle{\pgfqpoint{0.800000in}{0.528000in}}{\pgfqpoint{4.960000in}{3.696000in}}%
\pgfusepath{clip}%
\pgfsetbuttcap%
\pgfsetroundjoin%
\definecolor{currentfill}{rgb}{0.000000,0.000000,0.000000}%
\pgfsetfillcolor{currentfill}%
\pgfsetlinewidth{1.003750pt}%
\definecolor{currentstroke}{rgb}{0.000000,0.000000,0.000000}%
\pgfsetstrokecolor{currentstroke}%
\pgfsetdash{}{0pt}%
\pgfpathmoveto{\pgfqpoint{1.025906in}{0.684199in}}%
\pgfpathcurveto{\pgfqpoint{1.036956in}{0.684199in}}{\pgfqpoint{1.047555in}{0.688590in}}{\pgfqpoint{1.055369in}{0.696403in}}%
\pgfpathcurveto{\pgfqpoint{1.063182in}{0.704217in}}{\pgfqpoint{1.067573in}{0.714816in}}{\pgfqpoint{1.067573in}{0.725866in}}%
\pgfpathcurveto{\pgfqpoint{1.067573in}{0.736916in}}{\pgfqpoint{1.063182in}{0.747515in}}{\pgfqpoint{1.055369in}{0.755329in}}%
\pgfpathcurveto{\pgfqpoint{1.047555in}{0.763142in}}{\pgfqpoint{1.036956in}{0.767533in}}{\pgfqpoint{1.025906in}{0.767533in}}%
\pgfpathcurveto{\pgfqpoint{1.014856in}{0.767533in}}{\pgfqpoint{1.004257in}{0.763142in}}{\pgfqpoint{0.996443in}{0.755329in}}%
\pgfpathcurveto{\pgfqpoint{0.988630in}{0.747515in}}{\pgfqpoint{0.984239in}{0.736916in}}{\pgfqpoint{0.984239in}{0.725866in}}%
\pgfpathcurveto{\pgfqpoint{0.984239in}{0.714816in}}{\pgfqpoint{0.988630in}{0.704217in}}{\pgfqpoint{0.996443in}{0.696403in}}%
\pgfpathcurveto{\pgfqpoint{1.004257in}{0.688590in}}{\pgfqpoint{1.014856in}{0.684199in}}{\pgfqpoint{1.025906in}{0.684199in}}%
\pgfpathclose%
\pgfusepath{stroke,fill}%
\end{pgfscope}%
\begin{pgfscope}%
\pgfpathrectangle{\pgfqpoint{0.800000in}{0.528000in}}{\pgfqpoint{4.960000in}{3.696000in}}%
\pgfusepath{clip}%
\pgfsetbuttcap%
\pgfsetroundjoin%
\definecolor{currentfill}{rgb}{0.000000,0.000000,0.000000}%
\pgfsetfillcolor{currentfill}%
\pgfsetlinewidth{1.003750pt}%
\definecolor{currentstroke}{rgb}{0.000000,0.000000,0.000000}%
\pgfsetstrokecolor{currentstroke}%
\pgfsetdash{}{0pt}%
\pgfpathmoveto{\pgfqpoint{1.025906in}{0.684199in}}%
\pgfpathcurveto{\pgfqpoint{1.036956in}{0.684199in}}{\pgfqpoint{1.047555in}{0.688590in}}{\pgfqpoint{1.055369in}{0.696403in}}%
\pgfpathcurveto{\pgfqpoint{1.063182in}{0.704217in}}{\pgfqpoint{1.067573in}{0.714816in}}{\pgfqpoint{1.067573in}{0.725866in}}%
\pgfpathcurveto{\pgfqpoint{1.067573in}{0.736916in}}{\pgfqpoint{1.063182in}{0.747515in}}{\pgfqpoint{1.055369in}{0.755329in}}%
\pgfpathcurveto{\pgfqpoint{1.047555in}{0.763142in}}{\pgfqpoint{1.036956in}{0.767533in}}{\pgfqpoint{1.025906in}{0.767533in}}%
\pgfpathcurveto{\pgfqpoint{1.014856in}{0.767533in}}{\pgfqpoint{1.004257in}{0.763142in}}{\pgfqpoint{0.996443in}{0.755329in}}%
\pgfpathcurveto{\pgfqpoint{0.988630in}{0.747515in}}{\pgfqpoint{0.984239in}{0.736916in}}{\pgfqpoint{0.984239in}{0.725866in}}%
\pgfpathcurveto{\pgfqpoint{0.984239in}{0.714816in}}{\pgfqpoint{0.988630in}{0.704217in}}{\pgfqpoint{0.996443in}{0.696403in}}%
\pgfpathcurveto{\pgfqpoint{1.004257in}{0.688590in}}{\pgfqpoint{1.014856in}{0.684199in}}{\pgfqpoint{1.025906in}{0.684199in}}%
\pgfpathclose%
\pgfusepath{stroke,fill}%
\end{pgfscope}%
\begin{pgfscope}%
\pgfpathrectangle{\pgfqpoint{0.800000in}{0.528000in}}{\pgfqpoint{4.960000in}{3.696000in}}%
\pgfusepath{clip}%
\pgfsetbuttcap%
\pgfsetroundjoin%
\definecolor{currentfill}{rgb}{0.000000,0.000000,0.000000}%
\pgfsetfillcolor{currentfill}%
\pgfsetlinewidth{1.003750pt}%
\definecolor{currentstroke}{rgb}{0.000000,0.000000,0.000000}%
\pgfsetstrokecolor{currentstroke}%
\pgfsetdash{}{0pt}%
\pgfpathmoveto{\pgfqpoint{1.025906in}{0.684199in}}%
\pgfpathcurveto{\pgfqpoint{1.036956in}{0.684199in}}{\pgfqpoint{1.047555in}{0.688590in}}{\pgfqpoint{1.055369in}{0.696403in}}%
\pgfpathcurveto{\pgfqpoint{1.063182in}{0.704217in}}{\pgfqpoint{1.067573in}{0.714816in}}{\pgfqpoint{1.067573in}{0.725866in}}%
\pgfpathcurveto{\pgfqpoint{1.067573in}{0.736916in}}{\pgfqpoint{1.063182in}{0.747515in}}{\pgfqpoint{1.055369in}{0.755329in}}%
\pgfpathcurveto{\pgfqpoint{1.047555in}{0.763142in}}{\pgfqpoint{1.036956in}{0.767533in}}{\pgfqpoint{1.025906in}{0.767533in}}%
\pgfpathcurveto{\pgfqpoint{1.014856in}{0.767533in}}{\pgfqpoint{1.004257in}{0.763142in}}{\pgfqpoint{0.996443in}{0.755329in}}%
\pgfpathcurveto{\pgfqpoint{0.988630in}{0.747515in}}{\pgfqpoint{0.984239in}{0.736916in}}{\pgfqpoint{0.984239in}{0.725866in}}%
\pgfpathcurveto{\pgfqpoint{0.984239in}{0.714816in}}{\pgfqpoint{0.988630in}{0.704217in}}{\pgfqpoint{0.996443in}{0.696403in}}%
\pgfpathcurveto{\pgfqpoint{1.004257in}{0.688590in}}{\pgfqpoint{1.014856in}{0.684199in}}{\pgfqpoint{1.025906in}{0.684199in}}%
\pgfpathclose%
\pgfusepath{stroke,fill}%
\end{pgfscope}%
\begin{pgfscope}%
\pgfpathrectangle{\pgfqpoint{0.800000in}{0.528000in}}{\pgfqpoint{4.960000in}{3.696000in}}%
\pgfusepath{clip}%
\pgfsetbuttcap%
\pgfsetroundjoin%
\definecolor{currentfill}{rgb}{0.000000,0.000000,0.000000}%
\pgfsetfillcolor{currentfill}%
\pgfsetlinewidth{1.003750pt}%
\definecolor{currentstroke}{rgb}{0.000000,0.000000,0.000000}%
\pgfsetstrokecolor{currentstroke}%
\pgfsetdash{}{0pt}%
\pgfpathmoveto{\pgfqpoint{1.025906in}{0.684199in}}%
\pgfpathcurveto{\pgfqpoint{1.036956in}{0.684199in}}{\pgfqpoint{1.047555in}{0.688590in}}{\pgfqpoint{1.055369in}{0.696403in}}%
\pgfpathcurveto{\pgfqpoint{1.063182in}{0.704217in}}{\pgfqpoint{1.067573in}{0.714816in}}{\pgfqpoint{1.067573in}{0.725866in}}%
\pgfpathcurveto{\pgfqpoint{1.067573in}{0.736916in}}{\pgfqpoint{1.063182in}{0.747515in}}{\pgfqpoint{1.055369in}{0.755329in}}%
\pgfpathcurveto{\pgfqpoint{1.047555in}{0.763142in}}{\pgfqpoint{1.036956in}{0.767533in}}{\pgfqpoint{1.025906in}{0.767533in}}%
\pgfpathcurveto{\pgfqpoint{1.014856in}{0.767533in}}{\pgfqpoint{1.004257in}{0.763142in}}{\pgfqpoint{0.996443in}{0.755329in}}%
\pgfpathcurveto{\pgfqpoint{0.988630in}{0.747515in}}{\pgfqpoint{0.984239in}{0.736916in}}{\pgfqpoint{0.984239in}{0.725866in}}%
\pgfpathcurveto{\pgfqpoint{0.984239in}{0.714816in}}{\pgfqpoint{0.988630in}{0.704217in}}{\pgfqpoint{0.996443in}{0.696403in}}%
\pgfpathcurveto{\pgfqpoint{1.004257in}{0.688590in}}{\pgfqpoint{1.014856in}{0.684199in}}{\pgfqpoint{1.025906in}{0.684199in}}%
\pgfpathclose%
\pgfusepath{stroke,fill}%
\end{pgfscope}%
\begin{pgfscope}%
\pgfpathrectangle{\pgfqpoint{0.800000in}{0.528000in}}{\pgfqpoint{4.960000in}{3.696000in}}%
\pgfusepath{clip}%
\pgfsetbuttcap%
\pgfsetroundjoin%
\definecolor{currentfill}{rgb}{0.000000,0.000000,0.000000}%
\pgfsetfillcolor{currentfill}%
\pgfsetlinewidth{1.003750pt}%
\definecolor{currentstroke}{rgb}{0.000000,0.000000,0.000000}%
\pgfsetstrokecolor{currentstroke}%
\pgfsetdash{}{0pt}%
\pgfpathmoveto{\pgfqpoint{1.025906in}{2.334266in}}%
\pgfpathcurveto{\pgfqpoint{1.036956in}{2.334266in}}{\pgfqpoint{1.047555in}{2.338657in}}{\pgfqpoint{1.055369in}{2.346470in}}%
\pgfpathcurveto{\pgfqpoint{1.063182in}{2.354284in}}{\pgfqpoint{1.067573in}{2.364883in}}{\pgfqpoint{1.067573in}{2.375933in}}%
\pgfpathcurveto{\pgfqpoint{1.067573in}{2.386983in}}{\pgfqpoint{1.063182in}{2.397582in}}{\pgfqpoint{1.055369in}{2.405396in}}%
\pgfpathcurveto{\pgfqpoint{1.047555in}{2.413209in}}{\pgfqpoint{1.036956in}{2.417600in}}{\pgfqpoint{1.025906in}{2.417600in}}%
\pgfpathcurveto{\pgfqpoint{1.014856in}{2.417600in}}{\pgfqpoint{1.004257in}{2.413209in}}{\pgfqpoint{0.996443in}{2.405396in}}%
\pgfpathcurveto{\pgfqpoint{0.988630in}{2.397582in}}{\pgfqpoint{0.984239in}{2.386983in}}{\pgfqpoint{0.984239in}{2.375933in}}%
\pgfpathcurveto{\pgfqpoint{0.984239in}{2.364883in}}{\pgfqpoint{0.988630in}{2.354284in}}{\pgfqpoint{0.996443in}{2.346470in}}%
\pgfpathcurveto{\pgfqpoint{1.004257in}{2.338657in}}{\pgfqpoint{1.014856in}{2.334266in}}{\pgfqpoint{1.025906in}{2.334266in}}%
\pgfpathclose%
\pgfusepath{stroke,fill}%
\end{pgfscope}%
\begin{pgfscope}%
\pgfpathrectangle{\pgfqpoint{0.800000in}{0.528000in}}{\pgfqpoint{4.960000in}{3.696000in}}%
\pgfusepath{clip}%
\pgfsetbuttcap%
\pgfsetroundjoin%
\definecolor{currentfill}{rgb}{0.000000,0.000000,0.000000}%
\pgfsetfillcolor{currentfill}%
\pgfsetlinewidth{1.003750pt}%
\definecolor{currentstroke}{rgb}{0.000000,0.000000,0.000000}%
\pgfsetstrokecolor{currentstroke}%
\pgfsetdash{}{0pt}%
\pgfpathmoveto{\pgfqpoint{1.025906in}{2.334266in}}%
\pgfpathcurveto{\pgfqpoint{1.036956in}{2.334266in}}{\pgfqpoint{1.047555in}{2.338657in}}{\pgfqpoint{1.055369in}{2.346470in}}%
\pgfpathcurveto{\pgfqpoint{1.063182in}{2.354284in}}{\pgfqpoint{1.067573in}{2.364883in}}{\pgfqpoint{1.067573in}{2.375933in}}%
\pgfpathcurveto{\pgfqpoint{1.067573in}{2.386983in}}{\pgfqpoint{1.063182in}{2.397582in}}{\pgfqpoint{1.055369in}{2.405396in}}%
\pgfpathcurveto{\pgfqpoint{1.047555in}{2.413209in}}{\pgfqpoint{1.036956in}{2.417600in}}{\pgfqpoint{1.025906in}{2.417600in}}%
\pgfpathcurveto{\pgfqpoint{1.014856in}{2.417600in}}{\pgfqpoint{1.004257in}{2.413209in}}{\pgfqpoint{0.996443in}{2.405396in}}%
\pgfpathcurveto{\pgfqpoint{0.988630in}{2.397582in}}{\pgfqpoint{0.984239in}{2.386983in}}{\pgfqpoint{0.984239in}{2.375933in}}%
\pgfpathcurveto{\pgfqpoint{0.984239in}{2.364883in}}{\pgfqpoint{0.988630in}{2.354284in}}{\pgfqpoint{0.996443in}{2.346470in}}%
\pgfpathcurveto{\pgfqpoint{1.004257in}{2.338657in}}{\pgfqpoint{1.014856in}{2.334266in}}{\pgfqpoint{1.025906in}{2.334266in}}%
\pgfpathclose%
\pgfusepath{stroke,fill}%
\end{pgfscope}%
\begin{pgfscope}%
\pgfpathrectangle{\pgfqpoint{0.800000in}{0.528000in}}{\pgfqpoint{4.960000in}{3.696000in}}%
\pgfusepath{clip}%
\pgfsetbuttcap%
\pgfsetroundjoin%
\definecolor{currentfill}{rgb}{0.000000,0.000000,0.000000}%
\pgfsetfillcolor{currentfill}%
\pgfsetlinewidth{1.003750pt}%
\definecolor{currentstroke}{rgb}{0.000000,0.000000,0.000000}%
\pgfsetstrokecolor{currentstroke}%
\pgfsetdash{}{0pt}%
\pgfpathmoveto{\pgfqpoint{1.025906in}{0.684199in}}%
\pgfpathcurveto{\pgfqpoint{1.036956in}{0.684199in}}{\pgfqpoint{1.047555in}{0.688590in}}{\pgfqpoint{1.055369in}{0.696403in}}%
\pgfpathcurveto{\pgfqpoint{1.063182in}{0.704217in}}{\pgfqpoint{1.067573in}{0.714816in}}{\pgfqpoint{1.067573in}{0.725866in}}%
\pgfpathcurveto{\pgfqpoint{1.067573in}{0.736916in}}{\pgfqpoint{1.063182in}{0.747515in}}{\pgfqpoint{1.055369in}{0.755329in}}%
\pgfpathcurveto{\pgfqpoint{1.047555in}{0.763142in}}{\pgfqpoint{1.036956in}{0.767533in}}{\pgfqpoint{1.025906in}{0.767533in}}%
\pgfpathcurveto{\pgfqpoint{1.014856in}{0.767533in}}{\pgfqpoint{1.004257in}{0.763142in}}{\pgfqpoint{0.996443in}{0.755329in}}%
\pgfpathcurveto{\pgfqpoint{0.988630in}{0.747515in}}{\pgfqpoint{0.984239in}{0.736916in}}{\pgfqpoint{0.984239in}{0.725866in}}%
\pgfpathcurveto{\pgfqpoint{0.984239in}{0.714816in}}{\pgfqpoint{0.988630in}{0.704217in}}{\pgfqpoint{0.996443in}{0.696403in}}%
\pgfpathcurveto{\pgfqpoint{1.004257in}{0.688590in}}{\pgfqpoint{1.014856in}{0.684199in}}{\pgfqpoint{1.025906in}{0.684199in}}%
\pgfpathclose%
\pgfusepath{stroke,fill}%
\end{pgfscope}%
\begin{pgfscope}%
\pgfpathrectangle{\pgfqpoint{0.800000in}{0.528000in}}{\pgfqpoint{4.960000in}{3.696000in}}%
\pgfusepath{clip}%
\pgfsetbuttcap%
\pgfsetroundjoin%
\definecolor{currentfill}{rgb}{0.000000,0.000000,0.000000}%
\pgfsetfillcolor{currentfill}%
\pgfsetlinewidth{1.003750pt}%
\definecolor{currentstroke}{rgb}{0.000000,0.000000,0.000000}%
\pgfsetstrokecolor{currentstroke}%
\pgfsetdash{}{0pt}%
\pgfpathmoveto{\pgfqpoint{1.025906in}{0.684199in}}%
\pgfpathcurveto{\pgfqpoint{1.036956in}{0.684199in}}{\pgfqpoint{1.047555in}{0.688590in}}{\pgfqpoint{1.055369in}{0.696403in}}%
\pgfpathcurveto{\pgfqpoint{1.063182in}{0.704217in}}{\pgfqpoint{1.067573in}{0.714816in}}{\pgfqpoint{1.067573in}{0.725866in}}%
\pgfpathcurveto{\pgfqpoint{1.067573in}{0.736916in}}{\pgfqpoint{1.063182in}{0.747515in}}{\pgfqpoint{1.055369in}{0.755329in}}%
\pgfpathcurveto{\pgfqpoint{1.047555in}{0.763142in}}{\pgfqpoint{1.036956in}{0.767533in}}{\pgfqpoint{1.025906in}{0.767533in}}%
\pgfpathcurveto{\pgfqpoint{1.014856in}{0.767533in}}{\pgfqpoint{1.004257in}{0.763142in}}{\pgfqpoint{0.996443in}{0.755329in}}%
\pgfpathcurveto{\pgfqpoint{0.988630in}{0.747515in}}{\pgfqpoint{0.984239in}{0.736916in}}{\pgfqpoint{0.984239in}{0.725866in}}%
\pgfpathcurveto{\pgfqpoint{0.984239in}{0.714816in}}{\pgfqpoint{0.988630in}{0.704217in}}{\pgfqpoint{0.996443in}{0.696403in}}%
\pgfpathcurveto{\pgfqpoint{1.004257in}{0.688590in}}{\pgfqpoint{1.014856in}{0.684199in}}{\pgfqpoint{1.025906in}{0.684199in}}%
\pgfpathclose%
\pgfusepath{stroke,fill}%
\end{pgfscope}%
\begin{pgfscope}%
\pgfpathrectangle{\pgfqpoint{0.800000in}{0.528000in}}{\pgfqpoint{4.960000in}{3.696000in}}%
\pgfusepath{clip}%
\pgfsetbuttcap%
\pgfsetroundjoin%
\definecolor{currentfill}{rgb}{0.000000,0.000000,0.000000}%
\pgfsetfillcolor{currentfill}%
\pgfsetlinewidth{1.003750pt}%
\definecolor{currentstroke}{rgb}{0.000000,0.000000,0.000000}%
\pgfsetstrokecolor{currentstroke}%
\pgfsetdash{}{0pt}%
\pgfpathmoveto{\pgfqpoint{1.025906in}{0.684199in}}%
\pgfpathcurveto{\pgfqpoint{1.036956in}{0.684199in}}{\pgfqpoint{1.047555in}{0.688590in}}{\pgfqpoint{1.055369in}{0.696403in}}%
\pgfpathcurveto{\pgfqpoint{1.063182in}{0.704217in}}{\pgfqpoint{1.067573in}{0.714816in}}{\pgfqpoint{1.067573in}{0.725866in}}%
\pgfpathcurveto{\pgfqpoint{1.067573in}{0.736916in}}{\pgfqpoint{1.063182in}{0.747515in}}{\pgfqpoint{1.055369in}{0.755329in}}%
\pgfpathcurveto{\pgfqpoint{1.047555in}{0.763142in}}{\pgfqpoint{1.036956in}{0.767533in}}{\pgfqpoint{1.025906in}{0.767533in}}%
\pgfpathcurveto{\pgfqpoint{1.014856in}{0.767533in}}{\pgfqpoint{1.004257in}{0.763142in}}{\pgfqpoint{0.996443in}{0.755329in}}%
\pgfpathcurveto{\pgfqpoint{0.988630in}{0.747515in}}{\pgfqpoint{0.984239in}{0.736916in}}{\pgfqpoint{0.984239in}{0.725866in}}%
\pgfpathcurveto{\pgfqpoint{0.984239in}{0.714816in}}{\pgfqpoint{0.988630in}{0.704217in}}{\pgfqpoint{0.996443in}{0.696403in}}%
\pgfpathcurveto{\pgfqpoint{1.004257in}{0.688590in}}{\pgfqpoint{1.014856in}{0.684199in}}{\pgfqpoint{1.025906in}{0.684199in}}%
\pgfpathclose%
\pgfusepath{stroke,fill}%
\end{pgfscope}%
\begin{pgfscope}%
\pgfpathrectangle{\pgfqpoint{0.800000in}{0.528000in}}{\pgfqpoint{4.960000in}{3.696000in}}%
\pgfusepath{clip}%
\pgfsetbuttcap%
\pgfsetroundjoin%
\definecolor{currentfill}{rgb}{0.000000,0.000000,0.000000}%
\pgfsetfillcolor{currentfill}%
\pgfsetlinewidth{1.003750pt}%
\definecolor{currentstroke}{rgb}{0.000000,0.000000,0.000000}%
\pgfsetstrokecolor{currentstroke}%
\pgfsetdash{}{0pt}%
\pgfpathmoveto{\pgfqpoint{1.025906in}{0.684199in}}%
\pgfpathcurveto{\pgfqpoint{1.036956in}{0.684199in}}{\pgfqpoint{1.047555in}{0.688590in}}{\pgfqpoint{1.055369in}{0.696403in}}%
\pgfpathcurveto{\pgfqpoint{1.063182in}{0.704217in}}{\pgfqpoint{1.067573in}{0.714816in}}{\pgfqpoint{1.067573in}{0.725866in}}%
\pgfpathcurveto{\pgfqpoint{1.067573in}{0.736916in}}{\pgfqpoint{1.063182in}{0.747515in}}{\pgfqpoint{1.055369in}{0.755329in}}%
\pgfpathcurveto{\pgfqpoint{1.047555in}{0.763142in}}{\pgfqpoint{1.036956in}{0.767533in}}{\pgfqpoint{1.025906in}{0.767533in}}%
\pgfpathcurveto{\pgfqpoint{1.014856in}{0.767533in}}{\pgfqpoint{1.004257in}{0.763142in}}{\pgfqpoint{0.996443in}{0.755329in}}%
\pgfpathcurveto{\pgfqpoint{0.988630in}{0.747515in}}{\pgfqpoint{0.984239in}{0.736916in}}{\pgfqpoint{0.984239in}{0.725866in}}%
\pgfpathcurveto{\pgfqpoint{0.984239in}{0.714816in}}{\pgfqpoint{0.988630in}{0.704217in}}{\pgfqpoint{0.996443in}{0.696403in}}%
\pgfpathcurveto{\pgfqpoint{1.004257in}{0.688590in}}{\pgfqpoint{1.014856in}{0.684199in}}{\pgfqpoint{1.025906in}{0.684199in}}%
\pgfpathclose%
\pgfusepath{stroke,fill}%
\end{pgfscope}%
\begin{pgfscope}%
\pgfpathrectangle{\pgfqpoint{0.800000in}{0.528000in}}{\pgfqpoint{4.960000in}{3.696000in}}%
\pgfusepath{clip}%
\pgfsetbuttcap%
\pgfsetroundjoin%
\definecolor{currentfill}{rgb}{0.000000,0.000000,0.000000}%
\pgfsetfillcolor{currentfill}%
\pgfsetlinewidth{1.003750pt}%
\definecolor{currentstroke}{rgb}{0.000000,0.000000,0.000000}%
\pgfsetstrokecolor{currentstroke}%
\pgfsetdash{}{0pt}%
\pgfpathmoveto{\pgfqpoint{1.025906in}{0.684199in}}%
\pgfpathcurveto{\pgfqpoint{1.036956in}{0.684199in}}{\pgfqpoint{1.047555in}{0.688590in}}{\pgfqpoint{1.055369in}{0.696403in}}%
\pgfpathcurveto{\pgfqpoint{1.063182in}{0.704217in}}{\pgfqpoint{1.067573in}{0.714816in}}{\pgfqpoint{1.067573in}{0.725866in}}%
\pgfpathcurveto{\pgfqpoint{1.067573in}{0.736916in}}{\pgfqpoint{1.063182in}{0.747515in}}{\pgfqpoint{1.055369in}{0.755329in}}%
\pgfpathcurveto{\pgfqpoint{1.047555in}{0.763142in}}{\pgfqpoint{1.036956in}{0.767533in}}{\pgfqpoint{1.025906in}{0.767533in}}%
\pgfpathcurveto{\pgfqpoint{1.014856in}{0.767533in}}{\pgfqpoint{1.004257in}{0.763142in}}{\pgfqpoint{0.996443in}{0.755329in}}%
\pgfpathcurveto{\pgfqpoint{0.988630in}{0.747515in}}{\pgfqpoint{0.984239in}{0.736916in}}{\pgfqpoint{0.984239in}{0.725866in}}%
\pgfpathcurveto{\pgfqpoint{0.984239in}{0.714816in}}{\pgfqpoint{0.988630in}{0.704217in}}{\pgfqpoint{0.996443in}{0.696403in}}%
\pgfpathcurveto{\pgfqpoint{1.004257in}{0.688590in}}{\pgfqpoint{1.014856in}{0.684199in}}{\pgfqpoint{1.025906in}{0.684199in}}%
\pgfpathclose%
\pgfusepath{stroke,fill}%
\end{pgfscope}%
\begin{pgfscope}%
\pgfpathrectangle{\pgfqpoint{0.800000in}{0.528000in}}{\pgfqpoint{4.960000in}{3.696000in}}%
\pgfusepath{clip}%
\pgfsetbuttcap%
\pgfsetroundjoin%
\definecolor{currentfill}{rgb}{0.000000,0.000000,0.000000}%
\pgfsetfillcolor{currentfill}%
\pgfsetlinewidth{1.003750pt}%
\definecolor{currentstroke}{rgb}{0.000000,0.000000,0.000000}%
\pgfsetstrokecolor{currentstroke}%
\pgfsetdash{}{0pt}%
\pgfpathmoveto{\pgfqpoint{1.025906in}{0.684199in}}%
\pgfpathcurveto{\pgfqpoint{1.036956in}{0.684199in}}{\pgfqpoint{1.047555in}{0.688590in}}{\pgfqpoint{1.055369in}{0.696403in}}%
\pgfpathcurveto{\pgfqpoint{1.063182in}{0.704217in}}{\pgfqpoint{1.067573in}{0.714816in}}{\pgfqpoint{1.067573in}{0.725866in}}%
\pgfpathcurveto{\pgfqpoint{1.067573in}{0.736916in}}{\pgfqpoint{1.063182in}{0.747515in}}{\pgfqpoint{1.055369in}{0.755329in}}%
\pgfpathcurveto{\pgfqpoint{1.047555in}{0.763142in}}{\pgfqpoint{1.036956in}{0.767533in}}{\pgfqpoint{1.025906in}{0.767533in}}%
\pgfpathcurveto{\pgfqpoint{1.014856in}{0.767533in}}{\pgfqpoint{1.004257in}{0.763142in}}{\pgfqpoint{0.996443in}{0.755329in}}%
\pgfpathcurveto{\pgfqpoint{0.988630in}{0.747515in}}{\pgfqpoint{0.984239in}{0.736916in}}{\pgfqpoint{0.984239in}{0.725866in}}%
\pgfpathcurveto{\pgfqpoint{0.984239in}{0.714816in}}{\pgfqpoint{0.988630in}{0.704217in}}{\pgfqpoint{0.996443in}{0.696403in}}%
\pgfpathcurveto{\pgfqpoint{1.004257in}{0.688590in}}{\pgfqpoint{1.014856in}{0.684199in}}{\pgfqpoint{1.025906in}{0.684199in}}%
\pgfpathclose%
\pgfusepath{stroke,fill}%
\end{pgfscope}%
\begin{pgfscope}%
\pgfpathrectangle{\pgfqpoint{0.800000in}{0.528000in}}{\pgfqpoint{4.960000in}{3.696000in}}%
\pgfusepath{clip}%
\pgfsetbuttcap%
\pgfsetroundjoin%
\definecolor{currentfill}{rgb}{0.000000,0.000000,0.000000}%
\pgfsetfillcolor{currentfill}%
\pgfsetlinewidth{1.003750pt}%
\definecolor{currentstroke}{rgb}{0.000000,0.000000,0.000000}%
\pgfsetstrokecolor{currentstroke}%
\pgfsetdash{}{0pt}%
\pgfpathmoveto{\pgfqpoint{1.025906in}{2.334266in}}%
\pgfpathcurveto{\pgfqpoint{1.036956in}{2.334266in}}{\pgfqpoint{1.047555in}{2.338657in}}{\pgfqpoint{1.055369in}{2.346470in}}%
\pgfpathcurveto{\pgfqpoint{1.063182in}{2.354284in}}{\pgfqpoint{1.067573in}{2.364883in}}{\pgfqpoint{1.067573in}{2.375933in}}%
\pgfpathcurveto{\pgfqpoint{1.067573in}{2.386983in}}{\pgfqpoint{1.063182in}{2.397582in}}{\pgfqpoint{1.055369in}{2.405396in}}%
\pgfpathcurveto{\pgfqpoint{1.047555in}{2.413209in}}{\pgfqpoint{1.036956in}{2.417600in}}{\pgfqpoint{1.025906in}{2.417600in}}%
\pgfpathcurveto{\pgfqpoint{1.014856in}{2.417600in}}{\pgfqpoint{1.004257in}{2.413209in}}{\pgfqpoint{0.996443in}{2.405396in}}%
\pgfpathcurveto{\pgfqpoint{0.988630in}{2.397582in}}{\pgfqpoint{0.984239in}{2.386983in}}{\pgfqpoint{0.984239in}{2.375933in}}%
\pgfpathcurveto{\pgfqpoint{0.984239in}{2.364883in}}{\pgfqpoint{0.988630in}{2.354284in}}{\pgfqpoint{0.996443in}{2.346470in}}%
\pgfpathcurveto{\pgfqpoint{1.004257in}{2.338657in}}{\pgfqpoint{1.014856in}{2.334266in}}{\pgfqpoint{1.025906in}{2.334266in}}%
\pgfpathclose%
\pgfusepath{stroke,fill}%
\end{pgfscope}%
\begin{pgfscope}%
\pgfpathrectangle{\pgfqpoint{0.800000in}{0.528000in}}{\pgfqpoint{4.960000in}{3.696000in}}%
\pgfusepath{clip}%
\pgfsetbuttcap%
\pgfsetroundjoin%
\definecolor{currentfill}{rgb}{0.000000,0.000000,0.000000}%
\pgfsetfillcolor{currentfill}%
\pgfsetlinewidth{1.003750pt}%
\definecolor{currentstroke}{rgb}{0.000000,0.000000,0.000000}%
\pgfsetstrokecolor{currentstroke}%
\pgfsetdash{}{0pt}%
\pgfpathmoveto{\pgfqpoint{1.025906in}{2.334266in}}%
\pgfpathcurveto{\pgfqpoint{1.036956in}{2.334266in}}{\pgfqpoint{1.047555in}{2.338657in}}{\pgfqpoint{1.055369in}{2.346470in}}%
\pgfpathcurveto{\pgfqpoint{1.063182in}{2.354284in}}{\pgfqpoint{1.067573in}{2.364883in}}{\pgfqpoint{1.067573in}{2.375933in}}%
\pgfpathcurveto{\pgfqpoint{1.067573in}{2.386983in}}{\pgfqpoint{1.063182in}{2.397582in}}{\pgfqpoint{1.055369in}{2.405396in}}%
\pgfpathcurveto{\pgfqpoint{1.047555in}{2.413209in}}{\pgfqpoint{1.036956in}{2.417600in}}{\pgfqpoint{1.025906in}{2.417600in}}%
\pgfpathcurveto{\pgfqpoint{1.014856in}{2.417600in}}{\pgfqpoint{1.004257in}{2.413209in}}{\pgfqpoint{0.996443in}{2.405396in}}%
\pgfpathcurveto{\pgfqpoint{0.988630in}{2.397582in}}{\pgfqpoint{0.984239in}{2.386983in}}{\pgfqpoint{0.984239in}{2.375933in}}%
\pgfpathcurveto{\pgfqpoint{0.984239in}{2.364883in}}{\pgfqpoint{0.988630in}{2.354284in}}{\pgfqpoint{0.996443in}{2.346470in}}%
\pgfpathcurveto{\pgfqpoint{1.004257in}{2.338657in}}{\pgfqpoint{1.014856in}{2.334266in}}{\pgfqpoint{1.025906in}{2.334266in}}%
\pgfpathclose%
\pgfusepath{stroke,fill}%
\end{pgfscope}%
\begin{pgfscope}%
\pgfpathrectangle{\pgfqpoint{0.800000in}{0.528000in}}{\pgfqpoint{4.960000in}{3.696000in}}%
\pgfusepath{clip}%
\pgfsetbuttcap%
\pgfsetroundjoin%
\definecolor{currentfill}{rgb}{0.000000,0.000000,0.000000}%
\pgfsetfillcolor{currentfill}%
\pgfsetlinewidth{1.003750pt}%
\definecolor{currentstroke}{rgb}{0.000000,0.000000,0.000000}%
\pgfsetstrokecolor{currentstroke}%
\pgfsetdash{}{0pt}%
\pgfpathmoveto{\pgfqpoint{1.025906in}{0.684199in}}%
\pgfpathcurveto{\pgfqpoint{1.036956in}{0.684199in}}{\pgfqpoint{1.047555in}{0.688590in}}{\pgfqpoint{1.055369in}{0.696403in}}%
\pgfpathcurveto{\pgfqpoint{1.063182in}{0.704217in}}{\pgfqpoint{1.067573in}{0.714816in}}{\pgfqpoint{1.067573in}{0.725866in}}%
\pgfpathcurveto{\pgfqpoint{1.067573in}{0.736916in}}{\pgfqpoint{1.063182in}{0.747515in}}{\pgfqpoint{1.055369in}{0.755329in}}%
\pgfpathcurveto{\pgfqpoint{1.047555in}{0.763142in}}{\pgfqpoint{1.036956in}{0.767533in}}{\pgfqpoint{1.025906in}{0.767533in}}%
\pgfpathcurveto{\pgfqpoint{1.014856in}{0.767533in}}{\pgfqpoint{1.004257in}{0.763142in}}{\pgfqpoint{0.996443in}{0.755329in}}%
\pgfpathcurveto{\pgfqpoint{0.988630in}{0.747515in}}{\pgfqpoint{0.984239in}{0.736916in}}{\pgfqpoint{0.984239in}{0.725866in}}%
\pgfpathcurveto{\pgfqpoint{0.984239in}{0.714816in}}{\pgfqpoint{0.988630in}{0.704217in}}{\pgfqpoint{0.996443in}{0.696403in}}%
\pgfpathcurveto{\pgfqpoint{1.004257in}{0.688590in}}{\pgfqpoint{1.014856in}{0.684199in}}{\pgfqpoint{1.025906in}{0.684199in}}%
\pgfpathclose%
\pgfusepath{stroke,fill}%
\end{pgfscope}%
\begin{pgfscope}%
\pgfpathrectangle{\pgfqpoint{0.800000in}{0.528000in}}{\pgfqpoint{4.960000in}{3.696000in}}%
\pgfusepath{clip}%
\pgfsetbuttcap%
\pgfsetroundjoin%
\definecolor{currentfill}{rgb}{0.000000,0.000000,0.000000}%
\pgfsetfillcolor{currentfill}%
\pgfsetlinewidth{1.003750pt}%
\definecolor{currentstroke}{rgb}{0.000000,0.000000,0.000000}%
\pgfsetstrokecolor{currentstroke}%
\pgfsetdash{}{0pt}%
\pgfpathmoveto{\pgfqpoint{1.025906in}{0.684199in}}%
\pgfpathcurveto{\pgfqpoint{1.036956in}{0.684199in}}{\pgfqpoint{1.047555in}{0.688590in}}{\pgfqpoint{1.055369in}{0.696403in}}%
\pgfpathcurveto{\pgfqpoint{1.063182in}{0.704217in}}{\pgfqpoint{1.067573in}{0.714816in}}{\pgfqpoint{1.067573in}{0.725866in}}%
\pgfpathcurveto{\pgfqpoint{1.067573in}{0.736916in}}{\pgfqpoint{1.063182in}{0.747515in}}{\pgfqpoint{1.055369in}{0.755329in}}%
\pgfpathcurveto{\pgfqpoint{1.047555in}{0.763142in}}{\pgfqpoint{1.036956in}{0.767533in}}{\pgfqpoint{1.025906in}{0.767533in}}%
\pgfpathcurveto{\pgfqpoint{1.014856in}{0.767533in}}{\pgfqpoint{1.004257in}{0.763142in}}{\pgfqpoint{0.996443in}{0.755329in}}%
\pgfpathcurveto{\pgfqpoint{0.988630in}{0.747515in}}{\pgfqpoint{0.984239in}{0.736916in}}{\pgfqpoint{0.984239in}{0.725866in}}%
\pgfpathcurveto{\pgfqpoint{0.984239in}{0.714816in}}{\pgfqpoint{0.988630in}{0.704217in}}{\pgfqpoint{0.996443in}{0.696403in}}%
\pgfpathcurveto{\pgfqpoint{1.004257in}{0.688590in}}{\pgfqpoint{1.014856in}{0.684199in}}{\pgfqpoint{1.025906in}{0.684199in}}%
\pgfpathclose%
\pgfusepath{stroke,fill}%
\end{pgfscope}%
\begin{pgfscope}%
\pgfpathrectangle{\pgfqpoint{0.800000in}{0.528000in}}{\pgfqpoint{4.960000in}{3.696000in}}%
\pgfusepath{clip}%
\pgfsetbuttcap%
\pgfsetroundjoin%
\definecolor{currentfill}{rgb}{0.000000,0.000000,0.000000}%
\pgfsetfillcolor{currentfill}%
\pgfsetlinewidth{1.003750pt}%
\definecolor{currentstroke}{rgb}{0.000000,0.000000,0.000000}%
\pgfsetstrokecolor{currentstroke}%
\pgfsetdash{}{0pt}%
\pgfpathmoveto{\pgfqpoint{1.025906in}{0.684199in}}%
\pgfpathcurveto{\pgfqpoint{1.036956in}{0.684199in}}{\pgfqpoint{1.047555in}{0.688590in}}{\pgfqpoint{1.055369in}{0.696403in}}%
\pgfpathcurveto{\pgfqpoint{1.063182in}{0.704217in}}{\pgfqpoint{1.067573in}{0.714816in}}{\pgfqpoint{1.067573in}{0.725866in}}%
\pgfpathcurveto{\pgfqpoint{1.067573in}{0.736916in}}{\pgfqpoint{1.063182in}{0.747515in}}{\pgfqpoint{1.055369in}{0.755329in}}%
\pgfpathcurveto{\pgfqpoint{1.047555in}{0.763142in}}{\pgfqpoint{1.036956in}{0.767533in}}{\pgfqpoint{1.025906in}{0.767533in}}%
\pgfpathcurveto{\pgfqpoint{1.014856in}{0.767533in}}{\pgfqpoint{1.004257in}{0.763142in}}{\pgfqpoint{0.996443in}{0.755329in}}%
\pgfpathcurveto{\pgfqpoint{0.988630in}{0.747515in}}{\pgfqpoint{0.984239in}{0.736916in}}{\pgfqpoint{0.984239in}{0.725866in}}%
\pgfpathcurveto{\pgfqpoint{0.984239in}{0.714816in}}{\pgfqpoint{0.988630in}{0.704217in}}{\pgfqpoint{0.996443in}{0.696403in}}%
\pgfpathcurveto{\pgfqpoint{1.004257in}{0.688590in}}{\pgfqpoint{1.014856in}{0.684199in}}{\pgfqpoint{1.025906in}{0.684199in}}%
\pgfpathclose%
\pgfusepath{stroke,fill}%
\end{pgfscope}%
\begin{pgfscope}%
\pgfpathrectangle{\pgfqpoint{0.800000in}{0.528000in}}{\pgfqpoint{4.960000in}{3.696000in}}%
\pgfusepath{clip}%
\pgfsetbuttcap%
\pgfsetroundjoin%
\definecolor{currentfill}{rgb}{0.000000,0.000000,0.000000}%
\pgfsetfillcolor{currentfill}%
\pgfsetlinewidth{1.003750pt}%
\definecolor{currentstroke}{rgb}{0.000000,0.000000,0.000000}%
\pgfsetstrokecolor{currentstroke}%
\pgfsetdash{}{0pt}%
\pgfpathmoveto{\pgfqpoint{1.025906in}{0.684199in}}%
\pgfpathcurveto{\pgfqpoint{1.036956in}{0.684199in}}{\pgfqpoint{1.047555in}{0.688590in}}{\pgfqpoint{1.055369in}{0.696403in}}%
\pgfpathcurveto{\pgfqpoint{1.063182in}{0.704217in}}{\pgfqpoint{1.067573in}{0.714816in}}{\pgfqpoint{1.067573in}{0.725866in}}%
\pgfpathcurveto{\pgfqpoint{1.067573in}{0.736916in}}{\pgfqpoint{1.063182in}{0.747515in}}{\pgfqpoint{1.055369in}{0.755329in}}%
\pgfpathcurveto{\pgfqpoint{1.047555in}{0.763142in}}{\pgfqpoint{1.036956in}{0.767533in}}{\pgfqpoint{1.025906in}{0.767533in}}%
\pgfpathcurveto{\pgfqpoint{1.014856in}{0.767533in}}{\pgfqpoint{1.004257in}{0.763142in}}{\pgfqpoint{0.996443in}{0.755329in}}%
\pgfpathcurveto{\pgfqpoint{0.988630in}{0.747515in}}{\pgfqpoint{0.984239in}{0.736916in}}{\pgfqpoint{0.984239in}{0.725866in}}%
\pgfpathcurveto{\pgfqpoint{0.984239in}{0.714816in}}{\pgfqpoint{0.988630in}{0.704217in}}{\pgfqpoint{0.996443in}{0.696403in}}%
\pgfpathcurveto{\pgfqpoint{1.004257in}{0.688590in}}{\pgfqpoint{1.014856in}{0.684199in}}{\pgfqpoint{1.025906in}{0.684199in}}%
\pgfpathclose%
\pgfusepath{stroke,fill}%
\end{pgfscope}%
\begin{pgfscope}%
\pgfpathrectangle{\pgfqpoint{0.800000in}{0.528000in}}{\pgfqpoint{4.960000in}{3.696000in}}%
\pgfusepath{clip}%
\pgfsetbuttcap%
\pgfsetroundjoin%
\definecolor{currentfill}{rgb}{0.000000,0.000000,0.000000}%
\pgfsetfillcolor{currentfill}%
\pgfsetlinewidth{1.003750pt}%
\definecolor{currentstroke}{rgb}{0.000000,0.000000,0.000000}%
\pgfsetstrokecolor{currentstroke}%
\pgfsetdash{}{0pt}%
\pgfpathmoveto{\pgfqpoint{1.025906in}{2.334266in}}%
\pgfpathcurveto{\pgfqpoint{1.036956in}{2.334266in}}{\pgfqpoint{1.047555in}{2.338657in}}{\pgfqpoint{1.055369in}{2.346470in}}%
\pgfpathcurveto{\pgfqpoint{1.063182in}{2.354284in}}{\pgfqpoint{1.067573in}{2.364883in}}{\pgfqpoint{1.067573in}{2.375933in}}%
\pgfpathcurveto{\pgfqpoint{1.067573in}{2.386983in}}{\pgfqpoint{1.063182in}{2.397582in}}{\pgfqpoint{1.055369in}{2.405396in}}%
\pgfpathcurveto{\pgfqpoint{1.047555in}{2.413209in}}{\pgfqpoint{1.036956in}{2.417600in}}{\pgfqpoint{1.025906in}{2.417600in}}%
\pgfpathcurveto{\pgfqpoint{1.014856in}{2.417600in}}{\pgfqpoint{1.004257in}{2.413209in}}{\pgfqpoint{0.996443in}{2.405396in}}%
\pgfpathcurveto{\pgfqpoint{0.988630in}{2.397582in}}{\pgfqpoint{0.984239in}{2.386983in}}{\pgfqpoint{0.984239in}{2.375933in}}%
\pgfpathcurveto{\pgfqpoint{0.984239in}{2.364883in}}{\pgfqpoint{0.988630in}{2.354284in}}{\pgfqpoint{0.996443in}{2.346470in}}%
\pgfpathcurveto{\pgfqpoint{1.004257in}{2.338657in}}{\pgfqpoint{1.014856in}{2.334266in}}{\pgfqpoint{1.025906in}{2.334266in}}%
\pgfpathclose%
\pgfusepath{stroke,fill}%
\end{pgfscope}%
\begin{pgfscope}%
\pgfpathrectangle{\pgfqpoint{0.800000in}{0.528000in}}{\pgfqpoint{4.960000in}{3.696000in}}%
\pgfusepath{clip}%
\pgfsetbuttcap%
\pgfsetroundjoin%
\definecolor{currentfill}{rgb}{0.000000,0.000000,0.000000}%
\pgfsetfillcolor{currentfill}%
\pgfsetlinewidth{1.003750pt}%
\definecolor{currentstroke}{rgb}{0.000000,0.000000,0.000000}%
\pgfsetstrokecolor{currentstroke}%
\pgfsetdash{}{0pt}%
\pgfpathmoveto{\pgfqpoint{1.025906in}{0.684199in}}%
\pgfpathcurveto{\pgfqpoint{1.036956in}{0.684199in}}{\pgfqpoint{1.047555in}{0.688590in}}{\pgfqpoint{1.055369in}{0.696403in}}%
\pgfpathcurveto{\pgfqpoint{1.063182in}{0.704217in}}{\pgfqpoint{1.067573in}{0.714816in}}{\pgfqpoint{1.067573in}{0.725866in}}%
\pgfpathcurveto{\pgfqpoint{1.067573in}{0.736916in}}{\pgfqpoint{1.063182in}{0.747515in}}{\pgfqpoint{1.055369in}{0.755329in}}%
\pgfpathcurveto{\pgfqpoint{1.047555in}{0.763142in}}{\pgfqpoint{1.036956in}{0.767533in}}{\pgfqpoint{1.025906in}{0.767533in}}%
\pgfpathcurveto{\pgfqpoint{1.014856in}{0.767533in}}{\pgfqpoint{1.004257in}{0.763142in}}{\pgfqpoint{0.996443in}{0.755329in}}%
\pgfpathcurveto{\pgfqpoint{0.988630in}{0.747515in}}{\pgfqpoint{0.984239in}{0.736916in}}{\pgfqpoint{0.984239in}{0.725866in}}%
\pgfpathcurveto{\pgfqpoint{0.984239in}{0.714816in}}{\pgfqpoint{0.988630in}{0.704217in}}{\pgfqpoint{0.996443in}{0.696403in}}%
\pgfpathcurveto{\pgfqpoint{1.004257in}{0.688590in}}{\pgfqpoint{1.014856in}{0.684199in}}{\pgfqpoint{1.025906in}{0.684199in}}%
\pgfpathclose%
\pgfusepath{stroke,fill}%
\end{pgfscope}%
\begin{pgfscope}%
\pgfpathrectangle{\pgfqpoint{0.800000in}{0.528000in}}{\pgfqpoint{4.960000in}{3.696000in}}%
\pgfusepath{clip}%
\pgfsetbuttcap%
\pgfsetroundjoin%
\definecolor{currentfill}{rgb}{0.000000,0.000000,0.000000}%
\pgfsetfillcolor{currentfill}%
\pgfsetlinewidth{1.003750pt}%
\definecolor{currentstroke}{rgb}{0.000000,0.000000,0.000000}%
\pgfsetstrokecolor{currentstroke}%
\pgfsetdash{}{0pt}%
\pgfpathmoveto{\pgfqpoint{1.025906in}{0.684199in}}%
\pgfpathcurveto{\pgfqpoint{1.036956in}{0.684199in}}{\pgfqpoint{1.047555in}{0.688590in}}{\pgfqpoint{1.055369in}{0.696403in}}%
\pgfpathcurveto{\pgfqpoint{1.063182in}{0.704217in}}{\pgfqpoint{1.067573in}{0.714816in}}{\pgfqpoint{1.067573in}{0.725866in}}%
\pgfpathcurveto{\pgfqpoint{1.067573in}{0.736916in}}{\pgfqpoint{1.063182in}{0.747515in}}{\pgfqpoint{1.055369in}{0.755329in}}%
\pgfpathcurveto{\pgfqpoint{1.047555in}{0.763142in}}{\pgfqpoint{1.036956in}{0.767533in}}{\pgfqpoint{1.025906in}{0.767533in}}%
\pgfpathcurveto{\pgfqpoint{1.014856in}{0.767533in}}{\pgfqpoint{1.004257in}{0.763142in}}{\pgfqpoint{0.996443in}{0.755329in}}%
\pgfpathcurveto{\pgfqpoint{0.988630in}{0.747515in}}{\pgfqpoint{0.984239in}{0.736916in}}{\pgfqpoint{0.984239in}{0.725866in}}%
\pgfpathcurveto{\pgfqpoint{0.984239in}{0.714816in}}{\pgfqpoint{0.988630in}{0.704217in}}{\pgfqpoint{0.996443in}{0.696403in}}%
\pgfpathcurveto{\pgfqpoint{1.004257in}{0.688590in}}{\pgfqpoint{1.014856in}{0.684199in}}{\pgfqpoint{1.025906in}{0.684199in}}%
\pgfpathclose%
\pgfusepath{stroke,fill}%
\end{pgfscope}%
\begin{pgfscope}%
\pgfpathrectangle{\pgfqpoint{0.800000in}{0.528000in}}{\pgfqpoint{4.960000in}{3.696000in}}%
\pgfusepath{clip}%
\pgfsetbuttcap%
\pgfsetroundjoin%
\definecolor{currentfill}{rgb}{0.000000,0.000000,0.000000}%
\pgfsetfillcolor{currentfill}%
\pgfsetlinewidth{1.003750pt}%
\definecolor{currentstroke}{rgb}{0.000000,0.000000,0.000000}%
\pgfsetstrokecolor{currentstroke}%
\pgfsetdash{}{0pt}%
\pgfpathmoveto{\pgfqpoint{1.025906in}{0.684199in}}%
\pgfpathcurveto{\pgfqpoint{1.036956in}{0.684199in}}{\pgfqpoint{1.047555in}{0.688590in}}{\pgfqpoint{1.055369in}{0.696403in}}%
\pgfpathcurveto{\pgfqpoint{1.063182in}{0.704217in}}{\pgfqpoint{1.067573in}{0.714816in}}{\pgfqpoint{1.067573in}{0.725866in}}%
\pgfpathcurveto{\pgfqpoint{1.067573in}{0.736916in}}{\pgfqpoint{1.063182in}{0.747515in}}{\pgfqpoint{1.055369in}{0.755329in}}%
\pgfpathcurveto{\pgfqpoint{1.047555in}{0.763142in}}{\pgfqpoint{1.036956in}{0.767533in}}{\pgfqpoint{1.025906in}{0.767533in}}%
\pgfpathcurveto{\pgfqpoint{1.014856in}{0.767533in}}{\pgfqpoint{1.004257in}{0.763142in}}{\pgfqpoint{0.996443in}{0.755329in}}%
\pgfpathcurveto{\pgfqpoint{0.988630in}{0.747515in}}{\pgfqpoint{0.984239in}{0.736916in}}{\pgfqpoint{0.984239in}{0.725866in}}%
\pgfpathcurveto{\pgfqpoint{0.984239in}{0.714816in}}{\pgfqpoint{0.988630in}{0.704217in}}{\pgfqpoint{0.996443in}{0.696403in}}%
\pgfpathcurveto{\pgfqpoint{1.004257in}{0.688590in}}{\pgfqpoint{1.014856in}{0.684199in}}{\pgfqpoint{1.025906in}{0.684199in}}%
\pgfpathclose%
\pgfusepath{stroke,fill}%
\end{pgfscope}%
\begin{pgfscope}%
\pgfpathrectangle{\pgfqpoint{0.800000in}{0.528000in}}{\pgfqpoint{4.960000in}{3.696000in}}%
\pgfusepath{clip}%
\pgfsetbuttcap%
\pgfsetroundjoin%
\definecolor{currentfill}{rgb}{0.000000,0.000000,0.000000}%
\pgfsetfillcolor{currentfill}%
\pgfsetlinewidth{1.003750pt}%
\definecolor{currentstroke}{rgb}{0.000000,0.000000,0.000000}%
\pgfsetstrokecolor{currentstroke}%
\pgfsetdash{}{0pt}%
\pgfpathmoveto{\pgfqpoint{1.025906in}{0.684199in}}%
\pgfpathcurveto{\pgfqpoint{1.036956in}{0.684199in}}{\pgfqpoint{1.047555in}{0.688590in}}{\pgfqpoint{1.055369in}{0.696403in}}%
\pgfpathcurveto{\pgfqpoint{1.063182in}{0.704217in}}{\pgfqpoint{1.067573in}{0.714816in}}{\pgfqpoint{1.067573in}{0.725866in}}%
\pgfpathcurveto{\pgfqpoint{1.067573in}{0.736916in}}{\pgfqpoint{1.063182in}{0.747515in}}{\pgfqpoint{1.055369in}{0.755329in}}%
\pgfpathcurveto{\pgfqpoint{1.047555in}{0.763142in}}{\pgfqpoint{1.036956in}{0.767533in}}{\pgfqpoint{1.025906in}{0.767533in}}%
\pgfpathcurveto{\pgfqpoint{1.014856in}{0.767533in}}{\pgfqpoint{1.004257in}{0.763142in}}{\pgfqpoint{0.996443in}{0.755329in}}%
\pgfpathcurveto{\pgfqpoint{0.988630in}{0.747515in}}{\pgfqpoint{0.984239in}{0.736916in}}{\pgfqpoint{0.984239in}{0.725866in}}%
\pgfpathcurveto{\pgfqpoint{0.984239in}{0.714816in}}{\pgfqpoint{0.988630in}{0.704217in}}{\pgfqpoint{0.996443in}{0.696403in}}%
\pgfpathcurveto{\pgfqpoint{1.004257in}{0.688590in}}{\pgfqpoint{1.014856in}{0.684199in}}{\pgfqpoint{1.025906in}{0.684199in}}%
\pgfpathclose%
\pgfusepath{stroke,fill}%
\end{pgfscope}%
\begin{pgfscope}%
\pgfpathrectangle{\pgfqpoint{0.800000in}{0.528000in}}{\pgfqpoint{4.960000in}{3.696000in}}%
\pgfusepath{clip}%
\pgfsetbuttcap%
\pgfsetroundjoin%
\definecolor{currentfill}{rgb}{0.000000,0.000000,0.000000}%
\pgfsetfillcolor{currentfill}%
\pgfsetlinewidth{1.003750pt}%
\definecolor{currentstroke}{rgb}{0.000000,0.000000,0.000000}%
\pgfsetstrokecolor{currentstroke}%
\pgfsetdash{}{0pt}%
\pgfpathmoveto{\pgfqpoint{1.025906in}{0.684199in}}%
\pgfpathcurveto{\pgfqpoint{1.036956in}{0.684199in}}{\pgfqpoint{1.047555in}{0.688590in}}{\pgfqpoint{1.055369in}{0.696403in}}%
\pgfpathcurveto{\pgfqpoint{1.063182in}{0.704217in}}{\pgfqpoint{1.067573in}{0.714816in}}{\pgfqpoint{1.067573in}{0.725866in}}%
\pgfpathcurveto{\pgfqpoint{1.067573in}{0.736916in}}{\pgfqpoint{1.063182in}{0.747515in}}{\pgfqpoint{1.055369in}{0.755329in}}%
\pgfpathcurveto{\pgfqpoint{1.047555in}{0.763142in}}{\pgfqpoint{1.036956in}{0.767533in}}{\pgfqpoint{1.025906in}{0.767533in}}%
\pgfpathcurveto{\pgfqpoint{1.014856in}{0.767533in}}{\pgfqpoint{1.004257in}{0.763142in}}{\pgfqpoint{0.996443in}{0.755329in}}%
\pgfpathcurveto{\pgfqpoint{0.988630in}{0.747515in}}{\pgfqpoint{0.984239in}{0.736916in}}{\pgfqpoint{0.984239in}{0.725866in}}%
\pgfpathcurveto{\pgfqpoint{0.984239in}{0.714816in}}{\pgfqpoint{0.988630in}{0.704217in}}{\pgfqpoint{0.996443in}{0.696403in}}%
\pgfpathcurveto{\pgfqpoint{1.004257in}{0.688590in}}{\pgfqpoint{1.014856in}{0.684199in}}{\pgfqpoint{1.025906in}{0.684199in}}%
\pgfpathclose%
\pgfusepath{stroke,fill}%
\end{pgfscope}%
\begin{pgfscope}%
\pgfpathrectangle{\pgfqpoint{0.800000in}{0.528000in}}{\pgfqpoint{4.960000in}{3.696000in}}%
\pgfusepath{clip}%
\pgfsetbuttcap%
\pgfsetroundjoin%
\definecolor{currentfill}{rgb}{0.000000,0.000000,0.000000}%
\pgfsetfillcolor{currentfill}%
\pgfsetlinewidth{1.003750pt}%
\definecolor{currentstroke}{rgb}{0.000000,0.000000,0.000000}%
\pgfsetstrokecolor{currentstroke}%
\pgfsetdash{}{0pt}%
\pgfpathmoveto{\pgfqpoint{1.025906in}{0.684199in}}%
\pgfpathcurveto{\pgfqpoint{1.036956in}{0.684199in}}{\pgfqpoint{1.047555in}{0.688590in}}{\pgfqpoint{1.055369in}{0.696403in}}%
\pgfpathcurveto{\pgfqpoint{1.063182in}{0.704217in}}{\pgfqpoint{1.067573in}{0.714816in}}{\pgfqpoint{1.067573in}{0.725866in}}%
\pgfpathcurveto{\pgfqpoint{1.067573in}{0.736916in}}{\pgfqpoint{1.063182in}{0.747515in}}{\pgfqpoint{1.055369in}{0.755329in}}%
\pgfpathcurveto{\pgfqpoint{1.047555in}{0.763142in}}{\pgfqpoint{1.036956in}{0.767533in}}{\pgfqpoint{1.025906in}{0.767533in}}%
\pgfpathcurveto{\pgfqpoint{1.014856in}{0.767533in}}{\pgfqpoint{1.004257in}{0.763142in}}{\pgfqpoint{0.996443in}{0.755329in}}%
\pgfpathcurveto{\pgfqpoint{0.988630in}{0.747515in}}{\pgfqpoint{0.984239in}{0.736916in}}{\pgfqpoint{0.984239in}{0.725866in}}%
\pgfpathcurveto{\pgfqpoint{0.984239in}{0.714816in}}{\pgfqpoint{0.988630in}{0.704217in}}{\pgfqpoint{0.996443in}{0.696403in}}%
\pgfpathcurveto{\pgfqpoint{1.004257in}{0.688590in}}{\pgfqpoint{1.014856in}{0.684199in}}{\pgfqpoint{1.025906in}{0.684199in}}%
\pgfpathclose%
\pgfusepath{stroke,fill}%
\end{pgfscope}%
\begin{pgfscope}%
\pgfpathrectangle{\pgfqpoint{0.800000in}{0.528000in}}{\pgfqpoint{4.960000in}{3.696000in}}%
\pgfusepath{clip}%
\pgfsetbuttcap%
\pgfsetroundjoin%
\definecolor{currentfill}{rgb}{0.000000,0.000000,0.000000}%
\pgfsetfillcolor{currentfill}%
\pgfsetlinewidth{1.003750pt}%
\definecolor{currentstroke}{rgb}{0.000000,0.000000,0.000000}%
\pgfsetstrokecolor{currentstroke}%
\pgfsetdash{}{0pt}%
\pgfpathmoveto{\pgfqpoint{1.025906in}{0.684199in}}%
\pgfpathcurveto{\pgfqpoint{1.036956in}{0.684199in}}{\pgfqpoint{1.047555in}{0.688590in}}{\pgfqpoint{1.055369in}{0.696403in}}%
\pgfpathcurveto{\pgfqpoint{1.063182in}{0.704217in}}{\pgfqpoint{1.067573in}{0.714816in}}{\pgfqpoint{1.067573in}{0.725866in}}%
\pgfpathcurveto{\pgfqpoint{1.067573in}{0.736916in}}{\pgfqpoint{1.063182in}{0.747515in}}{\pgfqpoint{1.055369in}{0.755329in}}%
\pgfpathcurveto{\pgfqpoint{1.047555in}{0.763142in}}{\pgfqpoint{1.036956in}{0.767533in}}{\pgfqpoint{1.025906in}{0.767533in}}%
\pgfpathcurveto{\pgfqpoint{1.014856in}{0.767533in}}{\pgfqpoint{1.004257in}{0.763142in}}{\pgfqpoint{0.996443in}{0.755329in}}%
\pgfpathcurveto{\pgfqpoint{0.988630in}{0.747515in}}{\pgfqpoint{0.984239in}{0.736916in}}{\pgfqpoint{0.984239in}{0.725866in}}%
\pgfpathcurveto{\pgfqpoint{0.984239in}{0.714816in}}{\pgfqpoint{0.988630in}{0.704217in}}{\pgfqpoint{0.996443in}{0.696403in}}%
\pgfpathcurveto{\pgfqpoint{1.004257in}{0.688590in}}{\pgfqpoint{1.014856in}{0.684199in}}{\pgfqpoint{1.025906in}{0.684199in}}%
\pgfpathclose%
\pgfusepath{stroke,fill}%
\end{pgfscope}%
\begin{pgfscope}%
\pgfpathrectangle{\pgfqpoint{0.800000in}{0.528000in}}{\pgfqpoint{4.960000in}{3.696000in}}%
\pgfusepath{clip}%
\pgfsetbuttcap%
\pgfsetroundjoin%
\definecolor{currentfill}{rgb}{0.000000,0.000000,0.000000}%
\pgfsetfillcolor{currentfill}%
\pgfsetlinewidth{1.003750pt}%
\definecolor{currentstroke}{rgb}{0.000000,0.000000,0.000000}%
\pgfsetstrokecolor{currentstroke}%
\pgfsetdash{}{0pt}%
\pgfpathmoveto{\pgfqpoint{1.025906in}{0.684199in}}%
\pgfpathcurveto{\pgfqpoint{1.036956in}{0.684199in}}{\pgfqpoint{1.047555in}{0.688590in}}{\pgfqpoint{1.055369in}{0.696403in}}%
\pgfpathcurveto{\pgfqpoint{1.063182in}{0.704217in}}{\pgfqpoint{1.067573in}{0.714816in}}{\pgfqpoint{1.067573in}{0.725866in}}%
\pgfpathcurveto{\pgfqpoint{1.067573in}{0.736916in}}{\pgfqpoint{1.063182in}{0.747515in}}{\pgfqpoint{1.055369in}{0.755329in}}%
\pgfpathcurveto{\pgfqpoint{1.047555in}{0.763142in}}{\pgfqpoint{1.036956in}{0.767533in}}{\pgfqpoint{1.025906in}{0.767533in}}%
\pgfpathcurveto{\pgfqpoint{1.014856in}{0.767533in}}{\pgfqpoint{1.004257in}{0.763142in}}{\pgfqpoint{0.996443in}{0.755329in}}%
\pgfpathcurveto{\pgfqpoint{0.988630in}{0.747515in}}{\pgfqpoint{0.984239in}{0.736916in}}{\pgfqpoint{0.984239in}{0.725866in}}%
\pgfpathcurveto{\pgfqpoint{0.984239in}{0.714816in}}{\pgfqpoint{0.988630in}{0.704217in}}{\pgfqpoint{0.996443in}{0.696403in}}%
\pgfpathcurveto{\pgfqpoint{1.004257in}{0.688590in}}{\pgfqpoint{1.014856in}{0.684199in}}{\pgfqpoint{1.025906in}{0.684199in}}%
\pgfpathclose%
\pgfusepath{stroke,fill}%
\end{pgfscope}%
\begin{pgfscope}%
\pgfpathrectangle{\pgfqpoint{0.800000in}{0.528000in}}{\pgfqpoint{4.960000in}{3.696000in}}%
\pgfusepath{clip}%
\pgfsetbuttcap%
\pgfsetroundjoin%
\definecolor{currentfill}{rgb}{0.000000,0.000000,0.000000}%
\pgfsetfillcolor{currentfill}%
\pgfsetlinewidth{1.003750pt}%
\definecolor{currentstroke}{rgb}{0.000000,0.000000,0.000000}%
\pgfsetstrokecolor{currentstroke}%
\pgfsetdash{}{0pt}%
\pgfpathmoveto{\pgfqpoint{1.025906in}{0.684199in}}%
\pgfpathcurveto{\pgfqpoint{1.036956in}{0.684199in}}{\pgfqpoint{1.047555in}{0.688590in}}{\pgfqpoint{1.055369in}{0.696403in}}%
\pgfpathcurveto{\pgfqpoint{1.063182in}{0.704217in}}{\pgfqpoint{1.067573in}{0.714816in}}{\pgfqpoint{1.067573in}{0.725866in}}%
\pgfpathcurveto{\pgfqpoint{1.067573in}{0.736916in}}{\pgfqpoint{1.063182in}{0.747515in}}{\pgfqpoint{1.055369in}{0.755329in}}%
\pgfpathcurveto{\pgfqpoint{1.047555in}{0.763142in}}{\pgfqpoint{1.036956in}{0.767533in}}{\pgfqpoint{1.025906in}{0.767533in}}%
\pgfpathcurveto{\pgfqpoint{1.014856in}{0.767533in}}{\pgfqpoint{1.004257in}{0.763142in}}{\pgfqpoint{0.996443in}{0.755329in}}%
\pgfpathcurveto{\pgfqpoint{0.988630in}{0.747515in}}{\pgfqpoint{0.984239in}{0.736916in}}{\pgfqpoint{0.984239in}{0.725866in}}%
\pgfpathcurveto{\pgfqpoint{0.984239in}{0.714816in}}{\pgfqpoint{0.988630in}{0.704217in}}{\pgfqpoint{0.996443in}{0.696403in}}%
\pgfpathcurveto{\pgfqpoint{1.004257in}{0.688590in}}{\pgfqpoint{1.014856in}{0.684199in}}{\pgfqpoint{1.025906in}{0.684199in}}%
\pgfpathclose%
\pgfusepath{stroke,fill}%
\end{pgfscope}%
\begin{pgfscope}%
\pgfpathrectangle{\pgfqpoint{0.800000in}{0.528000in}}{\pgfqpoint{4.960000in}{3.696000in}}%
\pgfusepath{clip}%
\pgfsetbuttcap%
\pgfsetroundjoin%
\definecolor{currentfill}{rgb}{0.000000,0.000000,0.000000}%
\pgfsetfillcolor{currentfill}%
\pgfsetlinewidth{1.003750pt}%
\definecolor{currentstroke}{rgb}{0.000000,0.000000,0.000000}%
\pgfsetstrokecolor{currentstroke}%
\pgfsetdash{}{0pt}%
\pgfpathmoveto{\pgfqpoint{1.025906in}{0.684199in}}%
\pgfpathcurveto{\pgfqpoint{1.036956in}{0.684199in}}{\pgfqpoint{1.047555in}{0.688590in}}{\pgfqpoint{1.055369in}{0.696403in}}%
\pgfpathcurveto{\pgfqpoint{1.063182in}{0.704217in}}{\pgfqpoint{1.067573in}{0.714816in}}{\pgfqpoint{1.067573in}{0.725866in}}%
\pgfpathcurveto{\pgfqpoint{1.067573in}{0.736916in}}{\pgfqpoint{1.063182in}{0.747515in}}{\pgfqpoint{1.055369in}{0.755329in}}%
\pgfpathcurveto{\pgfqpoint{1.047555in}{0.763142in}}{\pgfqpoint{1.036956in}{0.767533in}}{\pgfqpoint{1.025906in}{0.767533in}}%
\pgfpathcurveto{\pgfqpoint{1.014856in}{0.767533in}}{\pgfqpoint{1.004257in}{0.763142in}}{\pgfqpoint{0.996443in}{0.755329in}}%
\pgfpathcurveto{\pgfqpoint{0.988630in}{0.747515in}}{\pgfqpoint{0.984239in}{0.736916in}}{\pgfqpoint{0.984239in}{0.725866in}}%
\pgfpathcurveto{\pgfqpoint{0.984239in}{0.714816in}}{\pgfqpoint{0.988630in}{0.704217in}}{\pgfqpoint{0.996443in}{0.696403in}}%
\pgfpathcurveto{\pgfqpoint{1.004257in}{0.688590in}}{\pgfqpoint{1.014856in}{0.684199in}}{\pgfqpoint{1.025906in}{0.684199in}}%
\pgfpathclose%
\pgfusepath{stroke,fill}%
\end{pgfscope}%
\begin{pgfscope}%
\pgfpathrectangle{\pgfqpoint{0.800000in}{0.528000in}}{\pgfqpoint{4.960000in}{3.696000in}}%
\pgfusepath{clip}%
\pgfsetbuttcap%
\pgfsetroundjoin%
\definecolor{currentfill}{rgb}{0.000000,0.000000,0.000000}%
\pgfsetfillcolor{currentfill}%
\pgfsetlinewidth{1.003750pt}%
\definecolor{currentstroke}{rgb}{0.000000,0.000000,0.000000}%
\pgfsetstrokecolor{currentstroke}%
\pgfsetdash{}{0pt}%
\pgfpathmoveto{\pgfqpoint{1.025906in}{0.684199in}}%
\pgfpathcurveto{\pgfqpoint{1.036956in}{0.684199in}}{\pgfqpoint{1.047555in}{0.688590in}}{\pgfqpoint{1.055369in}{0.696403in}}%
\pgfpathcurveto{\pgfqpoint{1.063182in}{0.704217in}}{\pgfqpoint{1.067573in}{0.714816in}}{\pgfqpoint{1.067573in}{0.725866in}}%
\pgfpathcurveto{\pgfqpoint{1.067573in}{0.736916in}}{\pgfqpoint{1.063182in}{0.747515in}}{\pgfqpoint{1.055369in}{0.755329in}}%
\pgfpathcurveto{\pgfqpoint{1.047555in}{0.763142in}}{\pgfqpoint{1.036956in}{0.767533in}}{\pgfqpoint{1.025906in}{0.767533in}}%
\pgfpathcurveto{\pgfqpoint{1.014856in}{0.767533in}}{\pgfqpoint{1.004257in}{0.763142in}}{\pgfqpoint{0.996443in}{0.755329in}}%
\pgfpathcurveto{\pgfqpoint{0.988630in}{0.747515in}}{\pgfqpoint{0.984239in}{0.736916in}}{\pgfqpoint{0.984239in}{0.725866in}}%
\pgfpathcurveto{\pgfqpoint{0.984239in}{0.714816in}}{\pgfqpoint{0.988630in}{0.704217in}}{\pgfqpoint{0.996443in}{0.696403in}}%
\pgfpathcurveto{\pgfqpoint{1.004257in}{0.688590in}}{\pgfqpoint{1.014856in}{0.684199in}}{\pgfqpoint{1.025906in}{0.684199in}}%
\pgfpathclose%
\pgfusepath{stroke,fill}%
\end{pgfscope}%
\begin{pgfscope}%
\pgfpathrectangle{\pgfqpoint{0.800000in}{0.528000in}}{\pgfqpoint{4.960000in}{3.696000in}}%
\pgfusepath{clip}%
\pgfsetbuttcap%
\pgfsetroundjoin%
\definecolor{currentfill}{rgb}{0.000000,0.000000,0.000000}%
\pgfsetfillcolor{currentfill}%
\pgfsetlinewidth{1.003750pt}%
\definecolor{currentstroke}{rgb}{0.000000,0.000000,0.000000}%
\pgfsetstrokecolor{currentstroke}%
\pgfsetdash{}{0pt}%
\pgfpathmoveto{\pgfqpoint{1.025906in}{0.684199in}}%
\pgfpathcurveto{\pgfqpoint{1.036956in}{0.684199in}}{\pgfqpoint{1.047555in}{0.688590in}}{\pgfqpoint{1.055369in}{0.696403in}}%
\pgfpathcurveto{\pgfqpoint{1.063182in}{0.704217in}}{\pgfqpoint{1.067573in}{0.714816in}}{\pgfqpoint{1.067573in}{0.725866in}}%
\pgfpathcurveto{\pgfqpoint{1.067573in}{0.736916in}}{\pgfqpoint{1.063182in}{0.747515in}}{\pgfqpoint{1.055369in}{0.755329in}}%
\pgfpathcurveto{\pgfqpoint{1.047555in}{0.763142in}}{\pgfqpoint{1.036956in}{0.767533in}}{\pgfqpoint{1.025906in}{0.767533in}}%
\pgfpathcurveto{\pgfqpoint{1.014856in}{0.767533in}}{\pgfqpoint{1.004257in}{0.763142in}}{\pgfqpoint{0.996443in}{0.755329in}}%
\pgfpathcurveto{\pgfqpoint{0.988630in}{0.747515in}}{\pgfqpoint{0.984239in}{0.736916in}}{\pgfqpoint{0.984239in}{0.725866in}}%
\pgfpathcurveto{\pgfqpoint{0.984239in}{0.714816in}}{\pgfqpoint{0.988630in}{0.704217in}}{\pgfqpoint{0.996443in}{0.696403in}}%
\pgfpathcurveto{\pgfqpoint{1.004257in}{0.688590in}}{\pgfqpoint{1.014856in}{0.684199in}}{\pgfqpoint{1.025906in}{0.684199in}}%
\pgfpathclose%
\pgfusepath{stroke,fill}%
\end{pgfscope}%
\begin{pgfscope}%
\pgfpathrectangle{\pgfqpoint{0.800000in}{0.528000in}}{\pgfqpoint{4.960000in}{3.696000in}}%
\pgfusepath{clip}%
\pgfsetbuttcap%
\pgfsetroundjoin%
\definecolor{currentfill}{rgb}{0.000000,0.000000,0.000000}%
\pgfsetfillcolor{currentfill}%
\pgfsetlinewidth{1.003750pt}%
\definecolor{currentstroke}{rgb}{0.000000,0.000000,0.000000}%
\pgfsetstrokecolor{currentstroke}%
\pgfsetdash{}{0pt}%
\pgfpathmoveto{\pgfqpoint{1.025906in}{0.684199in}}%
\pgfpathcurveto{\pgfqpoint{1.036956in}{0.684199in}}{\pgfqpoint{1.047555in}{0.688590in}}{\pgfqpoint{1.055369in}{0.696403in}}%
\pgfpathcurveto{\pgfqpoint{1.063182in}{0.704217in}}{\pgfqpoint{1.067573in}{0.714816in}}{\pgfqpoint{1.067573in}{0.725866in}}%
\pgfpathcurveto{\pgfqpoint{1.067573in}{0.736916in}}{\pgfqpoint{1.063182in}{0.747515in}}{\pgfqpoint{1.055369in}{0.755329in}}%
\pgfpathcurveto{\pgfqpoint{1.047555in}{0.763142in}}{\pgfqpoint{1.036956in}{0.767533in}}{\pgfqpoint{1.025906in}{0.767533in}}%
\pgfpathcurveto{\pgfqpoint{1.014856in}{0.767533in}}{\pgfqpoint{1.004257in}{0.763142in}}{\pgfqpoint{0.996443in}{0.755329in}}%
\pgfpathcurveto{\pgfqpoint{0.988630in}{0.747515in}}{\pgfqpoint{0.984239in}{0.736916in}}{\pgfqpoint{0.984239in}{0.725866in}}%
\pgfpathcurveto{\pgfqpoint{0.984239in}{0.714816in}}{\pgfqpoint{0.988630in}{0.704217in}}{\pgfqpoint{0.996443in}{0.696403in}}%
\pgfpathcurveto{\pgfqpoint{1.004257in}{0.688590in}}{\pgfqpoint{1.014856in}{0.684199in}}{\pgfqpoint{1.025906in}{0.684199in}}%
\pgfpathclose%
\pgfusepath{stroke,fill}%
\end{pgfscope}%
\begin{pgfscope}%
\pgfpathrectangle{\pgfqpoint{0.800000in}{0.528000in}}{\pgfqpoint{4.960000in}{3.696000in}}%
\pgfusepath{clip}%
\pgfsetbuttcap%
\pgfsetroundjoin%
\definecolor{currentfill}{rgb}{0.000000,0.000000,0.000000}%
\pgfsetfillcolor{currentfill}%
\pgfsetlinewidth{1.003750pt}%
\definecolor{currentstroke}{rgb}{0.000000,0.000000,0.000000}%
\pgfsetstrokecolor{currentstroke}%
\pgfsetdash{}{0pt}%
\pgfpathmoveto{\pgfqpoint{1.025906in}{2.334266in}}%
\pgfpathcurveto{\pgfqpoint{1.036956in}{2.334266in}}{\pgfqpoint{1.047555in}{2.338657in}}{\pgfqpoint{1.055369in}{2.346470in}}%
\pgfpathcurveto{\pgfqpoint{1.063182in}{2.354284in}}{\pgfqpoint{1.067573in}{2.364883in}}{\pgfqpoint{1.067573in}{2.375933in}}%
\pgfpathcurveto{\pgfqpoint{1.067573in}{2.386983in}}{\pgfqpoint{1.063182in}{2.397582in}}{\pgfqpoint{1.055369in}{2.405396in}}%
\pgfpathcurveto{\pgfqpoint{1.047555in}{2.413209in}}{\pgfqpoint{1.036956in}{2.417600in}}{\pgfqpoint{1.025906in}{2.417600in}}%
\pgfpathcurveto{\pgfqpoint{1.014856in}{2.417600in}}{\pgfqpoint{1.004257in}{2.413209in}}{\pgfqpoint{0.996443in}{2.405396in}}%
\pgfpathcurveto{\pgfqpoint{0.988630in}{2.397582in}}{\pgfqpoint{0.984239in}{2.386983in}}{\pgfqpoint{0.984239in}{2.375933in}}%
\pgfpathcurveto{\pgfqpoint{0.984239in}{2.364883in}}{\pgfqpoint{0.988630in}{2.354284in}}{\pgfqpoint{0.996443in}{2.346470in}}%
\pgfpathcurveto{\pgfqpoint{1.004257in}{2.338657in}}{\pgfqpoint{1.014856in}{2.334266in}}{\pgfqpoint{1.025906in}{2.334266in}}%
\pgfpathclose%
\pgfusepath{stroke,fill}%
\end{pgfscope}%
\begin{pgfscope}%
\pgfpathrectangle{\pgfqpoint{0.800000in}{0.528000in}}{\pgfqpoint{4.960000in}{3.696000in}}%
\pgfusepath{clip}%
\pgfsetbuttcap%
\pgfsetroundjoin%
\definecolor{currentfill}{rgb}{0.000000,0.000000,0.000000}%
\pgfsetfillcolor{currentfill}%
\pgfsetlinewidth{1.003750pt}%
\definecolor{currentstroke}{rgb}{0.000000,0.000000,0.000000}%
\pgfsetstrokecolor{currentstroke}%
\pgfsetdash{}{0pt}%
\pgfpathmoveto{\pgfqpoint{1.025906in}{0.684199in}}%
\pgfpathcurveto{\pgfqpoint{1.036956in}{0.684199in}}{\pgfqpoint{1.047555in}{0.688590in}}{\pgfqpoint{1.055369in}{0.696403in}}%
\pgfpathcurveto{\pgfqpoint{1.063182in}{0.704217in}}{\pgfqpoint{1.067573in}{0.714816in}}{\pgfqpoint{1.067573in}{0.725866in}}%
\pgfpathcurveto{\pgfqpoint{1.067573in}{0.736916in}}{\pgfqpoint{1.063182in}{0.747515in}}{\pgfqpoint{1.055369in}{0.755329in}}%
\pgfpathcurveto{\pgfqpoint{1.047555in}{0.763142in}}{\pgfqpoint{1.036956in}{0.767533in}}{\pgfqpoint{1.025906in}{0.767533in}}%
\pgfpathcurveto{\pgfqpoint{1.014856in}{0.767533in}}{\pgfqpoint{1.004257in}{0.763142in}}{\pgfqpoint{0.996443in}{0.755329in}}%
\pgfpathcurveto{\pgfqpoint{0.988630in}{0.747515in}}{\pgfqpoint{0.984239in}{0.736916in}}{\pgfqpoint{0.984239in}{0.725866in}}%
\pgfpathcurveto{\pgfqpoint{0.984239in}{0.714816in}}{\pgfqpoint{0.988630in}{0.704217in}}{\pgfqpoint{0.996443in}{0.696403in}}%
\pgfpathcurveto{\pgfqpoint{1.004257in}{0.688590in}}{\pgfqpoint{1.014856in}{0.684199in}}{\pgfqpoint{1.025906in}{0.684199in}}%
\pgfpathclose%
\pgfusepath{stroke,fill}%
\end{pgfscope}%
\begin{pgfscope}%
\pgfpathrectangle{\pgfqpoint{0.800000in}{0.528000in}}{\pgfqpoint{4.960000in}{3.696000in}}%
\pgfusepath{clip}%
\pgfsetbuttcap%
\pgfsetroundjoin%
\definecolor{currentfill}{rgb}{0.000000,0.000000,0.000000}%
\pgfsetfillcolor{currentfill}%
\pgfsetlinewidth{1.003750pt}%
\definecolor{currentstroke}{rgb}{0.000000,0.000000,0.000000}%
\pgfsetstrokecolor{currentstroke}%
\pgfsetdash{}{0pt}%
\pgfpathmoveto{\pgfqpoint{1.025906in}{0.684199in}}%
\pgfpathcurveto{\pgfqpoint{1.036956in}{0.684199in}}{\pgfqpoint{1.047555in}{0.688590in}}{\pgfqpoint{1.055369in}{0.696403in}}%
\pgfpathcurveto{\pgfqpoint{1.063182in}{0.704217in}}{\pgfqpoint{1.067573in}{0.714816in}}{\pgfqpoint{1.067573in}{0.725866in}}%
\pgfpathcurveto{\pgfqpoint{1.067573in}{0.736916in}}{\pgfqpoint{1.063182in}{0.747515in}}{\pgfqpoint{1.055369in}{0.755329in}}%
\pgfpathcurveto{\pgfqpoint{1.047555in}{0.763142in}}{\pgfqpoint{1.036956in}{0.767533in}}{\pgfqpoint{1.025906in}{0.767533in}}%
\pgfpathcurveto{\pgfqpoint{1.014856in}{0.767533in}}{\pgfqpoint{1.004257in}{0.763142in}}{\pgfqpoint{0.996443in}{0.755329in}}%
\pgfpathcurveto{\pgfqpoint{0.988630in}{0.747515in}}{\pgfqpoint{0.984239in}{0.736916in}}{\pgfqpoint{0.984239in}{0.725866in}}%
\pgfpathcurveto{\pgfqpoint{0.984239in}{0.714816in}}{\pgfqpoint{0.988630in}{0.704217in}}{\pgfqpoint{0.996443in}{0.696403in}}%
\pgfpathcurveto{\pgfqpoint{1.004257in}{0.688590in}}{\pgfqpoint{1.014856in}{0.684199in}}{\pgfqpoint{1.025906in}{0.684199in}}%
\pgfpathclose%
\pgfusepath{stroke,fill}%
\end{pgfscope}%
\begin{pgfscope}%
\pgfpathrectangle{\pgfqpoint{0.800000in}{0.528000in}}{\pgfqpoint{4.960000in}{3.696000in}}%
\pgfusepath{clip}%
\pgfsetbuttcap%
\pgfsetroundjoin%
\definecolor{currentfill}{rgb}{0.000000,0.000000,0.000000}%
\pgfsetfillcolor{currentfill}%
\pgfsetlinewidth{1.003750pt}%
\definecolor{currentstroke}{rgb}{0.000000,0.000000,0.000000}%
\pgfsetstrokecolor{currentstroke}%
\pgfsetdash{}{0pt}%
\pgfpathmoveto{\pgfqpoint{1.025906in}{0.684199in}}%
\pgfpathcurveto{\pgfqpoint{1.036956in}{0.684199in}}{\pgfqpoint{1.047555in}{0.688590in}}{\pgfqpoint{1.055369in}{0.696403in}}%
\pgfpathcurveto{\pgfqpoint{1.063182in}{0.704217in}}{\pgfqpoint{1.067573in}{0.714816in}}{\pgfqpoint{1.067573in}{0.725866in}}%
\pgfpathcurveto{\pgfqpoint{1.067573in}{0.736916in}}{\pgfqpoint{1.063182in}{0.747515in}}{\pgfqpoint{1.055369in}{0.755329in}}%
\pgfpathcurveto{\pgfqpoint{1.047555in}{0.763142in}}{\pgfqpoint{1.036956in}{0.767533in}}{\pgfqpoint{1.025906in}{0.767533in}}%
\pgfpathcurveto{\pgfqpoint{1.014856in}{0.767533in}}{\pgfqpoint{1.004257in}{0.763142in}}{\pgfqpoint{0.996443in}{0.755329in}}%
\pgfpathcurveto{\pgfqpoint{0.988630in}{0.747515in}}{\pgfqpoint{0.984239in}{0.736916in}}{\pgfqpoint{0.984239in}{0.725866in}}%
\pgfpathcurveto{\pgfqpoint{0.984239in}{0.714816in}}{\pgfqpoint{0.988630in}{0.704217in}}{\pgfqpoint{0.996443in}{0.696403in}}%
\pgfpathcurveto{\pgfqpoint{1.004257in}{0.688590in}}{\pgfqpoint{1.014856in}{0.684199in}}{\pgfqpoint{1.025906in}{0.684199in}}%
\pgfpathclose%
\pgfusepath{stroke,fill}%
\end{pgfscope}%
\begin{pgfscope}%
\pgfpathrectangle{\pgfqpoint{0.800000in}{0.528000in}}{\pgfqpoint{4.960000in}{3.696000in}}%
\pgfusepath{clip}%
\pgfsetbuttcap%
\pgfsetroundjoin%
\definecolor{currentfill}{rgb}{0.000000,0.000000,0.000000}%
\pgfsetfillcolor{currentfill}%
\pgfsetlinewidth{1.003750pt}%
\definecolor{currentstroke}{rgb}{0.000000,0.000000,0.000000}%
\pgfsetstrokecolor{currentstroke}%
\pgfsetdash{}{0pt}%
\pgfpathmoveto{\pgfqpoint{1.025906in}{2.334266in}}%
\pgfpathcurveto{\pgfqpoint{1.036956in}{2.334266in}}{\pgfqpoint{1.047555in}{2.338657in}}{\pgfqpoint{1.055369in}{2.346470in}}%
\pgfpathcurveto{\pgfqpoint{1.063182in}{2.354284in}}{\pgfqpoint{1.067573in}{2.364883in}}{\pgfqpoint{1.067573in}{2.375933in}}%
\pgfpathcurveto{\pgfqpoint{1.067573in}{2.386983in}}{\pgfqpoint{1.063182in}{2.397582in}}{\pgfqpoint{1.055369in}{2.405396in}}%
\pgfpathcurveto{\pgfqpoint{1.047555in}{2.413209in}}{\pgfqpoint{1.036956in}{2.417600in}}{\pgfqpoint{1.025906in}{2.417600in}}%
\pgfpathcurveto{\pgfqpoint{1.014856in}{2.417600in}}{\pgfqpoint{1.004257in}{2.413209in}}{\pgfqpoint{0.996443in}{2.405396in}}%
\pgfpathcurveto{\pgfqpoint{0.988630in}{2.397582in}}{\pgfqpoint{0.984239in}{2.386983in}}{\pgfqpoint{0.984239in}{2.375933in}}%
\pgfpathcurveto{\pgfqpoint{0.984239in}{2.364883in}}{\pgfqpoint{0.988630in}{2.354284in}}{\pgfqpoint{0.996443in}{2.346470in}}%
\pgfpathcurveto{\pgfqpoint{1.004257in}{2.338657in}}{\pgfqpoint{1.014856in}{2.334266in}}{\pgfqpoint{1.025906in}{2.334266in}}%
\pgfpathclose%
\pgfusepath{stroke,fill}%
\end{pgfscope}%
\begin{pgfscope}%
\pgfpathrectangle{\pgfqpoint{0.800000in}{0.528000in}}{\pgfqpoint{4.960000in}{3.696000in}}%
\pgfusepath{clip}%
\pgfsetbuttcap%
\pgfsetroundjoin%
\definecolor{currentfill}{rgb}{0.000000,0.000000,0.000000}%
\pgfsetfillcolor{currentfill}%
\pgfsetlinewidth{1.003750pt}%
\definecolor{currentstroke}{rgb}{0.000000,0.000000,0.000000}%
\pgfsetstrokecolor{currentstroke}%
\pgfsetdash{}{0pt}%
\pgfpathmoveto{\pgfqpoint{1.025906in}{0.684199in}}%
\pgfpathcurveto{\pgfqpoint{1.036956in}{0.684199in}}{\pgfqpoint{1.047555in}{0.688590in}}{\pgfqpoint{1.055369in}{0.696403in}}%
\pgfpathcurveto{\pgfqpoint{1.063182in}{0.704217in}}{\pgfqpoint{1.067573in}{0.714816in}}{\pgfqpoint{1.067573in}{0.725866in}}%
\pgfpathcurveto{\pgfqpoint{1.067573in}{0.736916in}}{\pgfqpoint{1.063182in}{0.747515in}}{\pgfqpoint{1.055369in}{0.755329in}}%
\pgfpathcurveto{\pgfqpoint{1.047555in}{0.763142in}}{\pgfqpoint{1.036956in}{0.767533in}}{\pgfqpoint{1.025906in}{0.767533in}}%
\pgfpathcurveto{\pgfqpoint{1.014856in}{0.767533in}}{\pgfqpoint{1.004257in}{0.763142in}}{\pgfqpoint{0.996443in}{0.755329in}}%
\pgfpathcurveto{\pgfqpoint{0.988630in}{0.747515in}}{\pgfqpoint{0.984239in}{0.736916in}}{\pgfqpoint{0.984239in}{0.725866in}}%
\pgfpathcurveto{\pgfqpoint{0.984239in}{0.714816in}}{\pgfqpoint{0.988630in}{0.704217in}}{\pgfqpoint{0.996443in}{0.696403in}}%
\pgfpathcurveto{\pgfqpoint{1.004257in}{0.688590in}}{\pgfqpoint{1.014856in}{0.684199in}}{\pgfqpoint{1.025906in}{0.684199in}}%
\pgfpathclose%
\pgfusepath{stroke,fill}%
\end{pgfscope}%
\begin{pgfscope}%
\pgfpathrectangle{\pgfqpoint{0.800000in}{0.528000in}}{\pgfqpoint{4.960000in}{3.696000in}}%
\pgfusepath{clip}%
\pgfsetbuttcap%
\pgfsetroundjoin%
\definecolor{currentfill}{rgb}{0.000000,0.000000,0.000000}%
\pgfsetfillcolor{currentfill}%
\pgfsetlinewidth{1.003750pt}%
\definecolor{currentstroke}{rgb}{0.000000,0.000000,0.000000}%
\pgfsetstrokecolor{currentstroke}%
\pgfsetdash{}{0pt}%
\pgfpathmoveto{\pgfqpoint{1.025906in}{0.684199in}}%
\pgfpathcurveto{\pgfqpoint{1.036956in}{0.684199in}}{\pgfqpoint{1.047555in}{0.688590in}}{\pgfqpoint{1.055369in}{0.696403in}}%
\pgfpathcurveto{\pgfqpoint{1.063182in}{0.704217in}}{\pgfqpoint{1.067573in}{0.714816in}}{\pgfqpoint{1.067573in}{0.725866in}}%
\pgfpathcurveto{\pgfqpoint{1.067573in}{0.736916in}}{\pgfqpoint{1.063182in}{0.747515in}}{\pgfqpoint{1.055369in}{0.755329in}}%
\pgfpathcurveto{\pgfqpoint{1.047555in}{0.763142in}}{\pgfqpoint{1.036956in}{0.767533in}}{\pgfqpoint{1.025906in}{0.767533in}}%
\pgfpathcurveto{\pgfqpoint{1.014856in}{0.767533in}}{\pgfqpoint{1.004257in}{0.763142in}}{\pgfqpoint{0.996443in}{0.755329in}}%
\pgfpathcurveto{\pgfqpoint{0.988630in}{0.747515in}}{\pgfqpoint{0.984239in}{0.736916in}}{\pgfqpoint{0.984239in}{0.725866in}}%
\pgfpathcurveto{\pgfqpoint{0.984239in}{0.714816in}}{\pgfqpoint{0.988630in}{0.704217in}}{\pgfqpoint{0.996443in}{0.696403in}}%
\pgfpathcurveto{\pgfqpoint{1.004257in}{0.688590in}}{\pgfqpoint{1.014856in}{0.684199in}}{\pgfqpoint{1.025906in}{0.684199in}}%
\pgfpathclose%
\pgfusepath{stroke,fill}%
\end{pgfscope}%
\begin{pgfscope}%
\pgfpathrectangle{\pgfqpoint{0.800000in}{0.528000in}}{\pgfqpoint{4.960000in}{3.696000in}}%
\pgfusepath{clip}%
\pgfsetbuttcap%
\pgfsetroundjoin%
\definecolor{currentfill}{rgb}{0.000000,0.000000,0.000000}%
\pgfsetfillcolor{currentfill}%
\pgfsetlinewidth{1.003750pt}%
\definecolor{currentstroke}{rgb}{0.000000,0.000000,0.000000}%
\pgfsetstrokecolor{currentstroke}%
\pgfsetdash{}{0pt}%
\pgfpathmoveto{\pgfqpoint{1.025906in}{0.684199in}}%
\pgfpathcurveto{\pgfqpoint{1.036956in}{0.684199in}}{\pgfqpoint{1.047555in}{0.688590in}}{\pgfqpoint{1.055369in}{0.696403in}}%
\pgfpathcurveto{\pgfqpoint{1.063182in}{0.704217in}}{\pgfqpoint{1.067573in}{0.714816in}}{\pgfqpoint{1.067573in}{0.725866in}}%
\pgfpathcurveto{\pgfqpoint{1.067573in}{0.736916in}}{\pgfqpoint{1.063182in}{0.747515in}}{\pgfqpoint{1.055369in}{0.755329in}}%
\pgfpathcurveto{\pgfqpoint{1.047555in}{0.763142in}}{\pgfqpoint{1.036956in}{0.767533in}}{\pgfqpoint{1.025906in}{0.767533in}}%
\pgfpathcurveto{\pgfqpoint{1.014856in}{0.767533in}}{\pgfqpoint{1.004257in}{0.763142in}}{\pgfqpoint{0.996443in}{0.755329in}}%
\pgfpathcurveto{\pgfqpoint{0.988630in}{0.747515in}}{\pgfqpoint{0.984239in}{0.736916in}}{\pgfqpoint{0.984239in}{0.725866in}}%
\pgfpathcurveto{\pgfqpoint{0.984239in}{0.714816in}}{\pgfqpoint{0.988630in}{0.704217in}}{\pgfqpoint{0.996443in}{0.696403in}}%
\pgfpathcurveto{\pgfqpoint{1.004257in}{0.688590in}}{\pgfqpoint{1.014856in}{0.684199in}}{\pgfqpoint{1.025906in}{0.684199in}}%
\pgfpathclose%
\pgfusepath{stroke,fill}%
\end{pgfscope}%
\begin{pgfscope}%
\pgfpathrectangle{\pgfqpoint{0.800000in}{0.528000in}}{\pgfqpoint{4.960000in}{3.696000in}}%
\pgfusepath{clip}%
\pgfsetbuttcap%
\pgfsetroundjoin%
\definecolor{currentfill}{rgb}{0.000000,0.000000,0.000000}%
\pgfsetfillcolor{currentfill}%
\pgfsetlinewidth{1.003750pt}%
\definecolor{currentstroke}{rgb}{0.000000,0.000000,0.000000}%
\pgfsetstrokecolor{currentstroke}%
\pgfsetdash{}{0pt}%
\pgfpathmoveto{\pgfqpoint{1.025906in}{0.684199in}}%
\pgfpathcurveto{\pgfqpoint{1.036956in}{0.684199in}}{\pgfqpoint{1.047555in}{0.688590in}}{\pgfqpoint{1.055369in}{0.696403in}}%
\pgfpathcurveto{\pgfqpoint{1.063182in}{0.704217in}}{\pgfqpoint{1.067573in}{0.714816in}}{\pgfqpoint{1.067573in}{0.725866in}}%
\pgfpathcurveto{\pgfqpoint{1.067573in}{0.736916in}}{\pgfqpoint{1.063182in}{0.747515in}}{\pgfqpoint{1.055369in}{0.755329in}}%
\pgfpathcurveto{\pgfqpoint{1.047555in}{0.763142in}}{\pgfqpoint{1.036956in}{0.767533in}}{\pgfqpoint{1.025906in}{0.767533in}}%
\pgfpathcurveto{\pgfqpoint{1.014856in}{0.767533in}}{\pgfqpoint{1.004257in}{0.763142in}}{\pgfqpoint{0.996443in}{0.755329in}}%
\pgfpathcurveto{\pgfqpoint{0.988630in}{0.747515in}}{\pgfqpoint{0.984239in}{0.736916in}}{\pgfqpoint{0.984239in}{0.725866in}}%
\pgfpathcurveto{\pgfqpoint{0.984239in}{0.714816in}}{\pgfqpoint{0.988630in}{0.704217in}}{\pgfqpoint{0.996443in}{0.696403in}}%
\pgfpathcurveto{\pgfqpoint{1.004257in}{0.688590in}}{\pgfqpoint{1.014856in}{0.684199in}}{\pgfqpoint{1.025906in}{0.684199in}}%
\pgfpathclose%
\pgfusepath{stroke,fill}%
\end{pgfscope}%
\begin{pgfscope}%
\pgfpathrectangle{\pgfqpoint{0.800000in}{0.528000in}}{\pgfqpoint{4.960000in}{3.696000in}}%
\pgfusepath{clip}%
\pgfsetbuttcap%
\pgfsetroundjoin%
\definecolor{currentfill}{rgb}{0.000000,0.000000,0.000000}%
\pgfsetfillcolor{currentfill}%
\pgfsetlinewidth{1.003750pt}%
\definecolor{currentstroke}{rgb}{0.000000,0.000000,0.000000}%
\pgfsetstrokecolor{currentstroke}%
\pgfsetdash{}{0pt}%
\pgfpathmoveto{\pgfqpoint{1.025906in}{0.684199in}}%
\pgfpathcurveto{\pgfqpoint{1.036956in}{0.684199in}}{\pgfqpoint{1.047555in}{0.688590in}}{\pgfqpoint{1.055369in}{0.696403in}}%
\pgfpathcurveto{\pgfqpoint{1.063182in}{0.704217in}}{\pgfqpoint{1.067573in}{0.714816in}}{\pgfqpoint{1.067573in}{0.725866in}}%
\pgfpathcurveto{\pgfqpoint{1.067573in}{0.736916in}}{\pgfqpoint{1.063182in}{0.747515in}}{\pgfqpoint{1.055369in}{0.755329in}}%
\pgfpathcurveto{\pgfqpoint{1.047555in}{0.763142in}}{\pgfqpoint{1.036956in}{0.767533in}}{\pgfqpoint{1.025906in}{0.767533in}}%
\pgfpathcurveto{\pgfqpoint{1.014856in}{0.767533in}}{\pgfqpoint{1.004257in}{0.763142in}}{\pgfqpoint{0.996443in}{0.755329in}}%
\pgfpathcurveto{\pgfqpoint{0.988630in}{0.747515in}}{\pgfqpoint{0.984239in}{0.736916in}}{\pgfqpoint{0.984239in}{0.725866in}}%
\pgfpathcurveto{\pgfqpoint{0.984239in}{0.714816in}}{\pgfqpoint{0.988630in}{0.704217in}}{\pgfqpoint{0.996443in}{0.696403in}}%
\pgfpathcurveto{\pgfqpoint{1.004257in}{0.688590in}}{\pgfqpoint{1.014856in}{0.684199in}}{\pgfqpoint{1.025906in}{0.684199in}}%
\pgfpathclose%
\pgfusepath{stroke,fill}%
\end{pgfscope}%
\begin{pgfscope}%
\pgfpathrectangle{\pgfqpoint{0.800000in}{0.528000in}}{\pgfqpoint{4.960000in}{3.696000in}}%
\pgfusepath{clip}%
\pgfsetbuttcap%
\pgfsetroundjoin%
\definecolor{currentfill}{rgb}{0.000000,0.000000,0.000000}%
\pgfsetfillcolor{currentfill}%
\pgfsetlinewidth{1.003750pt}%
\definecolor{currentstroke}{rgb}{0.000000,0.000000,0.000000}%
\pgfsetstrokecolor{currentstroke}%
\pgfsetdash{}{0pt}%
\pgfpathmoveto{\pgfqpoint{1.025906in}{0.684199in}}%
\pgfpathcurveto{\pgfqpoint{1.036956in}{0.684199in}}{\pgfqpoint{1.047555in}{0.688590in}}{\pgfqpoint{1.055369in}{0.696403in}}%
\pgfpathcurveto{\pgfqpoint{1.063182in}{0.704217in}}{\pgfqpoint{1.067573in}{0.714816in}}{\pgfqpoint{1.067573in}{0.725866in}}%
\pgfpathcurveto{\pgfqpoint{1.067573in}{0.736916in}}{\pgfqpoint{1.063182in}{0.747515in}}{\pgfqpoint{1.055369in}{0.755329in}}%
\pgfpathcurveto{\pgfqpoint{1.047555in}{0.763142in}}{\pgfqpoint{1.036956in}{0.767533in}}{\pgfqpoint{1.025906in}{0.767533in}}%
\pgfpathcurveto{\pgfqpoint{1.014856in}{0.767533in}}{\pgfqpoint{1.004257in}{0.763142in}}{\pgfqpoint{0.996443in}{0.755329in}}%
\pgfpathcurveto{\pgfqpoint{0.988630in}{0.747515in}}{\pgfqpoint{0.984239in}{0.736916in}}{\pgfqpoint{0.984239in}{0.725866in}}%
\pgfpathcurveto{\pgfqpoint{0.984239in}{0.714816in}}{\pgfqpoint{0.988630in}{0.704217in}}{\pgfqpoint{0.996443in}{0.696403in}}%
\pgfpathcurveto{\pgfqpoint{1.004257in}{0.688590in}}{\pgfqpoint{1.014856in}{0.684199in}}{\pgfqpoint{1.025906in}{0.684199in}}%
\pgfpathclose%
\pgfusepath{stroke,fill}%
\end{pgfscope}%
\begin{pgfscope}%
\pgfpathrectangle{\pgfqpoint{0.800000in}{0.528000in}}{\pgfqpoint{4.960000in}{3.696000in}}%
\pgfusepath{clip}%
\pgfsetbuttcap%
\pgfsetroundjoin%
\definecolor{currentfill}{rgb}{0.000000,0.000000,0.000000}%
\pgfsetfillcolor{currentfill}%
\pgfsetlinewidth{1.003750pt}%
\definecolor{currentstroke}{rgb}{0.000000,0.000000,0.000000}%
\pgfsetstrokecolor{currentstroke}%
\pgfsetdash{}{0pt}%
\pgfpathmoveto{\pgfqpoint{1.025906in}{0.684199in}}%
\pgfpathcurveto{\pgfqpoint{1.036956in}{0.684199in}}{\pgfqpoint{1.047555in}{0.688590in}}{\pgfqpoint{1.055369in}{0.696403in}}%
\pgfpathcurveto{\pgfqpoint{1.063182in}{0.704217in}}{\pgfqpoint{1.067573in}{0.714816in}}{\pgfqpoint{1.067573in}{0.725866in}}%
\pgfpathcurveto{\pgfqpoint{1.067573in}{0.736916in}}{\pgfqpoint{1.063182in}{0.747515in}}{\pgfqpoint{1.055369in}{0.755329in}}%
\pgfpathcurveto{\pgfqpoint{1.047555in}{0.763142in}}{\pgfqpoint{1.036956in}{0.767533in}}{\pgfqpoint{1.025906in}{0.767533in}}%
\pgfpathcurveto{\pgfqpoint{1.014856in}{0.767533in}}{\pgfqpoint{1.004257in}{0.763142in}}{\pgfqpoint{0.996443in}{0.755329in}}%
\pgfpathcurveto{\pgfqpoint{0.988630in}{0.747515in}}{\pgfqpoint{0.984239in}{0.736916in}}{\pgfqpoint{0.984239in}{0.725866in}}%
\pgfpathcurveto{\pgfqpoint{0.984239in}{0.714816in}}{\pgfqpoint{0.988630in}{0.704217in}}{\pgfqpoint{0.996443in}{0.696403in}}%
\pgfpathcurveto{\pgfqpoint{1.004257in}{0.688590in}}{\pgfqpoint{1.014856in}{0.684199in}}{\pgfqpoint{1.025906in}{0.684199in}}%
\pgfpathclose%
\pgfusepath{stroke,fill}%
\end{pgfscope}%
\begin{pgfscope}%
\pgfpathrectangle{\pgfqpoint{0.800000in}{0.528000in}}{\pgfqpoint{4.960000in}{3.696000in}}%
\pgfusepath{clip}%
\pgfsetbuttcap%
\pgfsetroundjoin%
\definecolor{currentfill}{rgb}{0.000000,0.000000,0.000000}%
\pgfsetfillcolor{currentfill}%
\pgfsetlinewidth{1.003750pt}%
\definecolor{currentstroke}{rgb}{0.000000,0.000000,0.000000}%
\pgfsetstrokecolor{currentstroke}%
\pgfsetdash{}{0pt}%
\pgfpathmoveto{\pgfqpoint{1.025906in}{0.684199in}}%
\pgfpathcurveto{\pgfqpoint{1.036956in}{0.684199in}}{\pgfqpoint{1.047555in}{0.688590in}}{\pgfqpoint{1.055369in}{0.696403in}}%
\pgfpathcurveto{\pgfqpoint{1.063182in}{0.704217in}}{\pgfqpoint{1.067573in}{0.714816in}}{\pgfqpoint{1.067573in}{0.725866in}}%
\pgfpathcurveto{\pgfqpoint{1.067573in}{0.736916in}}{\pgfqpoint{1.063182in}{0.747515in}}{\pgfqpoint{1.055369in}{0.755329in}}%
\pgfpathcurveto{\pgfqpoint{1.047555in}{0.763142in}}{\pgfqpoint{1.036956in}{0.767533in}}{\pgfqpoint{1.025906in}{0.767533in}}%
\pgfpathcurveto{\pgfqpoint{1.014856in}{0.767533in}}{\pgfqpoint{1.004257in}{0.763142in}}{\pgfqpoint{0.996443in}{0.755329in}}%
\pgfpathcurveto{\pgfqpoint{0.988630in}{0.747515in}}{\pgfqpoint{0.984239in}{0.736916in}}{\pgfqpoint{0.984239in}{0.725866in}}%
\pgfpathcurveto{\pgfqpoint{0.984239in}{0.714816in}}{\pgfqpoint{0.988630in}{0.704217in}}{\pgfqpoint{0.996443in}{0.696403in}}%
\pgfpathcurveto{\pgfqpoint{1.004257in}{0.688590in}}{\pgfqpoint{1.014856in}{0.684199in}}{\pgfqpoint{1.025906in}{0.684199in}}%
\pgfpathclose%
\pgfusepath{stroke,fill}%
\end{pgfscope}%
\begin{pgfscope}%
\pgfpathrectangle{\pgfqpoint{0.800000in}{0.528000in}}{\pgfqpoint{4.960000in}{3.696000in}}%
\pgfusepath{clip}%
\pgfsetbuttcap%
\pgfsetroundjoin%
\definecolor{currentfill}{rgb}{0.000000,0.000000,0.000000}%
\pgfsetfillcolor{currentfill}%
\pgfsetlinewidth{1.003750pt}%
\definecolor{currentstroke}{rgb}{0.000000,0.000000,0.000000}%
\pgfsetstrokecolor{currentstroke}%
\pgfsetdash{}{0pt}%
\pgfpathmoveto{\pgfqpoint{1.025906in}{0.684199in}}%
\pgfpathcurveto{\pgfqpoint{1.036956in}{0.684199in}}{\pgfqpoint{1.047555in}{0.688590in}}{\pgfqpoint{1.055369in}{0.696403in}}%
\pgfpathcurveto{\pgfqpoint{1.063182in}{0.704217in}}{\pgfqpoint{1.067573in}{0.714816in}}{\pgfqpoint{1.067573in}{0.725866in}}%
\pgfpathcurveto{\pgfqpoint{1.067573in}{0.736916in}}{\pgfqpoint{1.063182in}{0.747515in}}{\pgfqpoint{1.055369in}{0.755329in}}%
\pgfpathcurveto{\pgfqpoint{1.047555in}{0.763142in}}{\pgfqpoint{1.036956in}{0.767533in}}{\pgfqpoint{1.025906in}{0.767533in}}%
\pgfpathcurveto{\pgfqpoint{1.014856in}{0.767533in}}{\pgfqpoint{1.004257in}{0.763142in}}{\pgfqpoint{0.996443in}{0.755329in}}%
\pgfpathcurveto{\pgfqpoint{0.988630in}{0.747515in}}{\pgfqpoint{0.984239in}{0.736916in}}{\pgfqpoint{0.984239in}{0.725866in}}%
\pgfpathcurveto{\pgfqpoint{0.984239in}{0.714816in}}{\pgfqpoint{0.988630in}{0.704217in}}{\pgfqpoint{0.996443in}{0.696403in}}%
\pgfpathcurveto{\pgfqpoint{1.004257in}{0.688590in}}{\pgfqpoint{1.014856in}{0.684199in}}{\pgfqpoint{1.025906in}{0.684199in}}%
\pgfpathclose%
\pgfusepath{stroke,fill}%
\end{pgfscope}%
\begin{pgfscope}%
\pgfpathrectangle{\pgfqpoint{0.800000in}{0.528000in}}{\pgfqpoint{4.960000in}{3.696000in}}%
\pgfusepath{clip}%
\pgfsetbuttcap%
\pgfsetroundjoin%
\definecolor{currentfill}{rgb}{0.000000,0.000000,0.000000}%
\pgfsetfillcolor{currentfill}%
\pgfsetlinewidth{1.003750pt}%
\definecolor{currentstroke}{rgb}{0.000000,0.000000,0.000000}%
\pgfsetstrokecolor{currentstroke}%
\pgfsetdash{}{0pt}%
\pgfpathmoveto{\pgfqpoint{1.025906in}{0.684199in}}%
\pgfpathcurveto{\pgfqpoint{1.036956in}{0.684199in}}{\pgfqpoint{1.047555in}{0.688590in}}{\pgfqpoint{1.055369in}{0.696403in}}%
\pgfpathcurveto{\pgfqpoint{1.063182in}{0.704217in}}{\pgfqpoint{1.067573in}{0.714816in}}{\pgfqpoint{1.067573in}{0.725866in}}%
\pgfpathcurveto{\pgfqpoint{1.067573in}{0.736916in}}{\pgfqpoint{1.063182in}{0.747515in}}{\pgfqpoint{1.055369in}{0.755329in}}%
\pgfpathcurveto{\pgfqpoint{1.047555in}{0.763142in}}{\pgfqpoint{1.036956in}{0.767533in}}{\pgfqpoint{1.025906in}{0.767533in}}%
\pgfpathcurveto{\pgfqpoint{1.014856in}{0.767533in}}{\pgfqpoint{1.004257in}{0.763142in}}{\pgfqpoint{0.996443in}{0.755329in}}%
\pgfpathcurveto{\pgfqpoint{0.988630in}{0.747515in}}{\pgfqpoint{0.984239in}{0.736916in}}{\pgfqpoint{0.984239in}{0.725866in}}%
\pgfpathcurveto{\pgfqpoint{0.984239in}{0.714816in}}{\pgfqpoint{0.988630in}{0.704217in}}{\pgfqpoint{0.996443in}{0.696403in}}%
\pgfpathcurveto{\pgfqpoint{1.004257in}{0.688590in}}{\pgfqpoint{1.014856in}{0.684199in}}{\pgfqpoint{1.025906in}{0.684199in}}%
\pgfpathclose%
\pgfusepath{stroke,fill}%
\end{pgfscope}%
\begin{pgfscope}%
\pgfpathrectangle{\pgfqpoint{0.800000in}{0.528000in}}{\pgfqpoint{4.960000in}{3.696000in}}%
\pgfusepath{clip}%
\pgfsetbuttcap%
\pgfsetroundjoin%
\definecolor{currentfill}{rgb}{0.000000,0.000000,0.000000}%
\pgfsetfillcolor{currentfill}%
\pgfsetlinewidth{1.003750pt}%
\definecolor{currentstroke}{rgb}{0.000000,0.000000,0.000000}%
\pgfsetstrokecolor{currentstroke}%
\pgfsetdash{}{0pt}%
\pgfpathmoveto{\pgfqpoint{1.025906in}{0.684199in}}%
\pgfpathcurveto{\pgfqpoint{1.036956in}{0.684199in}}{\pgfqpoint{1.047555in}{0.688590in}}{\pgfqpoint{1.055369in}{0.696403in}}%
\pgfpathcurveto{\pgfqpoint{1.063182in}{0.704217in}}{\pgfqpoint{1.067573in}{0.714816in}}{\pgfqpoint{1.067573in}{0.725866in}}%
\pgfpathcurveto{\pgfqpoint{1.067573in}{0.736916in}}{\pgfqpoint{1.063182in}{0.747515in}}{\pgfqpoint{1.055369in}{0.755329in}}%
\pgfpathcurveto{\pgfqpoint{1.047555in}{0.763142in}}{\pgfqpoint{1.036956in}{0.767533in}}{\pgfqpoint{1.025906in}{0.767533in}}%
\pgfpathcurveto{\pgfqpoint{1.014856in}{0.767533in}}{\pgfqpoint{1.004257in}{0.763142in}}{\pgfqpoint{0.996443in}{0.755329in}}%
\pgfpathcurveto{\pgfqpoint{0.988630in}{0.747515in}}{\pgfqpoint{0.984239in}{0.736916in}}{\pgfqpoint{0.984239in}{0.725866in}}%
\pgfpathcurveto{\pgfqpoint{0.984239in}{0.714816in}}{\pgfqpoint{0.988630in}{0.704217in}}{\pgfqpoint{0.996443in}{0.696403in}}%
\pgfpathcurveto{\pgfqpoint{1.004257in}{0.688590in}}{\pgfqpoint{1.014856in}{0.684199in}}{\pgfqpoint{1.025906in}{0.684199in}}%
\pgfpathclose%
\pgfusepath{stroke,fill}%
\end{pgfscope}%
\begin{pgfscope}%
\pgfpathrectangle{\pgfqpoint{0.800000in}{0.528000in}}{\pgfqpoint{4.960000in}{3.696000in}}%
\pgfusepath{clip}%
\pgfsetbuttcap%
\pgfsetroundjoin%
\definecolor{currentfill}{rgb}{0.000000,0.000000,0.000000}%
\pgfsetfillcolor{currentfill}%
\pgfsetlinewidth{1.003750pt}%
\definecolor{currentstroke}{rgb}{0.000000,0.000000,0.000000}%
\pgfsetstrokecolor{currentstroke}%
\pgfsetdash{}{0pt}%
\pgfpathmoveto{\pgfqpoint{1.025906in}{2.334266in}}%
\pgfpathcurveto{\pgfqpoint{1.036956in}{2.334266in}}{\pgfqpoint{1.047555in}{2.338657in}}{\pgfqpoint{1.055369in}{2.346470in}}%
\pgfpathcurveto{\pgfqpoint{1.063182in}{2.354284in}}{\pgfqpoint{1.067573in}{2.364883in}}{\pgfqpoint{1.067573in}{2.375933in}}%
\pgfpathcurveto{\pgfqpoint{1.067573in}{2.386983in}}{\pgfqpoint{1.063182in}{2.397582in}}{\pgfqpoint{1.055369in}{2.405396in}}%
\pgfpathcurveto{\pgfqpoint{1.047555in}{2.413209in}}{\pgfqpoint{1.036956in}{2.417600in}}{\pgfqpoint{1.025906in}{2.417600in}}%
\pgfpathcurveto{\pgfqpoint{1.014856in}{2.417600in}}{\pgfqpoint{1.004257in}{2.413209in}}{\pgfqpoint{0.996443in}{2.405396in}}%
\pgfpathcurveto{\pgfqpoint{0.988630in}{2.397582in}}{\pgfqpoint{0.984239in}{2.386983in}}{\pgfqpoint{0.984239in}{2.375933in}}%
\pgfpathcurveto{\pgfqpoint{0.984239in}{2.364883in}}{\pgfqpoint{0.988630in}{2.354284in}}{\pgfqpoint{0.996443in}{2.346470in}}%
\pgfpathcurveto{\pgfqpoint{1.004257in}{2.338657in}}{\pgfqpoint{1.014856in}{2.334266in}}{\pgfqpoint{1.025906in}{2.334266in}}%
\pgfpathclose%
\pgfusepath{stroke,fill}%
\end{pgfscope}%
\begin{pgfscope}%
\pgfpathrectangle{\pgfqpoint{0.800000in}{0.528000in}}{\pgfqpoint{4.960000in}{3.696000in}}%
\pgfusepath{clip}%
\pgfsetbuttcap%
\pgfsetroundjoin%
\definecolor{currentfill}{rgb}{0.000000,0.000000,0.000000}%
\pgfsetfillcolor{currentfill}%
\pgfsetlinewidth{1.003750pt}%
\definecolor{currentstroke}{rgb}{0.000000,0.000000,0.000000}%
\pgfsetstrokecolor{currentstroke}%
\pgfsetdash{}{0pt}%
\pgfpathmoveto{\pgfqpoint{1.025906in}{0.684199in}}%
\pgfpathcurveto{\pgfqpoint{1.036956in}{0.684199in}}{\pgfqpoint{1.047555in}{0.688590in}}{\pgfqpoint{1.055369in}{0.696403in}}%
\pgfpathcurveto{\pgfqpoint{1.063182in}{0.704217in}}{\pgfqpoint{1.067573in}{0.714816in}}{\pgfqpoint{1.067573in}{0.725866in}}%
\pgfpathcurveto{\pgfqpoint{1.067573in}{0.736916in}}{\pgfqpoint{1.063182in}{0.747515in}}{\pgfqpoint{1.055369in}{0.755329in}}%
\pgfpathcurveto{\pgfqpoint{1.047555in}{0.763142in}}{\pgfqpoint{1.036956in}{0.767533in}}{\pgfqpoint{1.025906in}{0.767533in}}%
\pgfpathcurveto{\pgfqpoint{1.014856in}{0.767533in}}{\pgfqpoint{1.004257in}{0.763142in}}{\pgfqpoint{0.996443in}{0.755329in}}%
\pgfpathcurveto{\pgfqpoint{0.988630in}{0.747515in}}{\pgfqpoint{0.984239in}{0.736916in}}{\pgfqpoint{0.984239in}{0.725866in}}%
\pgfpathcurveto{\pgfqpoint{0.984239in}{0.714816in}}{\pgfqpoint{0.988630in}{0.704217in}}{\pgfqpoint{0.996443in}{0.696403in}}%
\pgfpathcurveto{\pgfqpoint{1.004257in}{0.688590in}}{\pgfqpoint{1.014856in}{0.684199in}}{\pgfqpoint{1.025906in}{0.684199in}}%
\pgfpathclose%
\pgfusepath{stroke,fill}%
\end{pgfscope}%
\begin{pgfscope}%
\pgfpathrectangle{\pgfqpoint{0.800000in}{0.528000in}}{\pgfqpoint{4.960000in}{3.696000in}}%
\pgfusepath{clip}%
\pgfsetbuttcap%
\pgfsetroundjoin%
\definecolor{currentfill}{rgb}{0.000000,0.000000,0.000000}%
\pgfsetfillcolor{currentfill}%
\pgfsetlinewidth{1.003750pt}%
\definecolor{currentstroke}{rgb}{0.000000,0.000000,0.000000}%
\pgfsetstrokecolor{currentstroke}%
\pgfsetdash{}{0pt}%
\pgfpathmoveto{\pgfqpoint{1.025906in}{0.684199in}}%
\pgfpathcurveto{\pgfqpoint{1.036956in}{0.684199in}}{\pgfqpoint{1.047555in}{0.688590in}}{\pgfqpoint{1.055369in}{0.696403in}}%
\pgfpathcurveto{\pgfqpoint{1.063182in}{0.704217in}}{\pgfqpoint{1.067573in}{0.714816in}}{\pgfqpoint{1.067573in}{0.725866in}}%
\pgfpathcurveto{\pgfqpoint{1.067573in}{0.736916in}}{\pgfqpoint{1.063182in}{0.747515in}}{\pgfqpoint{1.055369in}{0.755329in}}%
\pgfpathcurveto{\pgfqpoint{1.047555in}{0.763142in}}{\pgfqpoint{1.036956in}{0.767533in}}{\pgfqpoint{1.025906in}{0.767533in}}%
\pgfpathcurveto{\pgfqpoint{1.014856in}{0.767533in}}{\pgfqpoint{1.004257in}{0.763142in}}{\pgfqpoint{0.996443in}{0.755329in}}%
\pgfpathcurveto{\pgfqpoint{0.988630in}{0.747515in}}{\pgfqpoint{0.984239in}{0.736916in}}{\pgfqpoint{0.984239in}{0.725866in}}%
\pgfpathcurveto{\pgfqpoint{0.984239in}{0.714816in}}{\pgfqpoint{0.988630in}{0.704217in}}{\pgfqpoint{0.996443in}{0.696403in}}%
\pgfpathcurveto{\pgfqpoint{1.004257in}{0.688590in}}{\pgfqpoint{1.014856in}{0.684199in}}{\pgfqpoint{1.025906in}{0.684199in}}%
\pgfpathclose%
\pgfusepath{stroke,fill}%
\end{pgfscope}%
\begin{pgfscope}%
\pgfpathrectangle{\pgfqpoint{0.800000in}{0.528000in}}{\pgfqpoint{4.960000in}{3.696000in}}%
\pgfusepath{clip}%
\pgfsetbuttcap%
\pgfsetroundjoin%
\definecolor{currentfill}{rgb}{0.000000,0.000000,0.000000}%
\pgfsetfillcolor{currentfill}%
\pgfsetlinewidth{1.003750pt}%
\definecolor{currentstroke}{rgb}{0.000000,0.000000,0.000000}%
\pgfsetstrokecolor{currentstroke}%
\pgfsetdash{}{0pt}%
\pgfpathmoveto{\pgfqpoint{1.025906in}{0.684199in}}%
\pgfpathcurveto{\pgfqpoint{1.036956in}{0.684199in}}{\pgfqpoint{1.047555in}{0.688590in}}{\pgfqpoint{1.055369in}{0.696403in}}%
\pgfpathcurveto{\pgfqpoint{1.063182in}{0.704217in}}{\pgfqpoint{1.067573in}{0.714816in}}{\pgfqpoint{1.067573in}{0.725866in}}%
\pgfpathcurveto{\pgfqpoint{1.067573in}{0.736916in}}{\pgfqpoint{1.063182in}{0.747515in}}{\pgfqpoint{1.055369in}{0.755329in}}%
\pgfpathcurveto{\pgfqpoint{1.047555in}{0.763142in}}{\pgfqpoint{1.036956in}{0.767533in}}{\pgfqpoint{1.025906in}{0.767533in}}%
\pgfpathcurveto{\pgfqpoint{1.014856in}{0.767533in}}{\pgfqpoint{1.004257in}{0.763142in}}{\pgfqpoint{0.996443in}{0.755329in}}%
\pgfpathcurveto{\pgfqpoint{0.988630in}{0.747515in}}{\pgfqpoint{0.984239in}{0.736916in}}{\pgfqpoint{0.984239in}{0.725866in}}%
\pgfpathcurveto{\pgfqpoint{0.984239in}{0.714816in}}{\pgfqpoint{0.988630in}{0.704217in}}{\pgfqpoint{0.996443in}{0.696403in}}%
\pgfpathcurveto{\pgfqpoint{1.004257in}{0.688590in}}{\pgfqpoint{1.014856in}{0.684199in}}{\pgfqpoint{1.025906in}{0.684199in}}%
\pgfpathclose%
\pgfusepath{stroke,fill}%
\end{pgfscope}%
\begin{pgfscope}%
\pgfpathrectangle{\pgfqpoint{0.800000in}{0.528000in}}{\pgfqpoint{4.960000in}{3.696000in}}%
\pgfusepath{clip}%
\pgfsetbuttcap%
\pgfsetroundjoin%
\definecolor{currentfill}{rgb}{0.000000,0.000000,0.000000}%
\pgfsetfillcolor{currentfill}%
\pgfsetlinewidth{1.003750pt}%
\definecolor{currentstroke}{rgb}{0.000000,0.000000,0.000000}%
\pgfsetstrokecolor{currentstroke}%
\pgfsetdash{}{0pt}%
\pgfpathmoveto{\pgfqpoint{1.025906in}{2.334266in}}%
\pgfpathcurveto{\pgfqpoint{1.036956in}{2.334266in}}{\pgfqpoint{1.047555in}{2.338657in}}{\pgfqpoint{1.055369in}{2.346470in}}%
\pgfpathcurveto{\pgfqpoint{1.063182in}{2.354284in}}{\pgfqpoint{1.067573in}{2.364883in}}{\pgfqpoint{1.067573in}{2.375933in}}%
\pgfpathcurveto{\pgfqpoint{1.067573in}{2.386983in}}{\pgfqpoint{1.063182in}{2.397582in}}{\pgfqpoint{1.055369in}{2.405396in}}%
\pgfpathcurveto{\pgfqpoint{1.047555in}{2.413209in}}{\pgfqpoint{1.036956in}{2.417600in}}{\pgfqpoint{1.025906in}{2.417600in}}%
\pgfpathcurveto{\pgfqpoint{1.014856in}{2.417600in}}{\pgfqpoint{1.004257in}{2.413209in}}{\pgfqpoint{0.996443in}{2.405396in}}%
\pgfpathcurveto{\pgfqpoint{0.988630in}{2.397582in}}{\pgfqpoint{0.984239in}{2.386983in}}{\pgfqpoint{0.984239in}{2.375933in}}%
\pgfpathcurveto{\pgfqpoint{0.984239in}{2.364883in}}{\pgfqpoint{0.988630in}{2.354284in}}{\pgfqpoint{0.996443in}{2.346470in}}%
\pgfpathcurveto{\pgfqpoint{1.004257in}{2.338657in}}{\pgfqpoint{1.014856in}{2.334266in}}{\pgfqpoint{1.025906in}{2.334266in}}%
\pgfpathclose%
\pgfusepath{stroke,fill}%
\end{pgfscope}%
\begin{pgfscope}%
\pgfpathrectangle{\pgfqpoint{0.800000in}{0.528000in}}{\pgfqpoint{4.960000in}{3.696000in}}%
\pgfusepath{clip}%
\pgfsetbuttcap%
\pgfsetroundjoin%
\definecolor{currentfill}{rgb}{0.000000,0.000000,0.000000}%
\pgfsetfillcolor{currentfill}%
\pgfsetlinewidth{1.003750pt}%
\definecolor{currentstroke}{rgb}{0.000000,0.000000,0.000000}%
\pgfsetstrokecolor{currentstroke}%
\pgfsetdash{}{0pt}%
\pgfpathmoveto{\pgfqpoint{1.025906in}{0.684199in}}%
\pgfpathcurveto{\pgfqpoint{1.036956in}{0.684199in}}{\pgfqpoint{1.047555in}{0.688590in}}{\pgfqpoint{1.055369in}{0.696403in}}%
\pgfpathcurveto{\pgfqpoint{1.063182in}{0.704217in}}{\pgfqpoint{1.067573in}{0.714816in}}{\pgfqpoint{1.067573in}{0.725866in}}%
\pgfpathcurveto{\pgfqpoint{1.067573in}{0.736916in}}{\pgfqpoint{1.063182in}{0.747515in}}{\pgfqpoint{1.055369in}{0.755329in}}%
\pgfpathcurveto{\pgfqpoint{1.047555in}{0.763142in}}{\pgfqpoint{1.036956in}{0.767533in}}{\pgfqpoint{1.025906in}{0.767533in}}%
\pgfpathcurveto{\pgfqpoint{1.014856in}{0.767533in}}{\pgfqpoint{1.004257in}{0.763142in}}{\pgfqpoint{0.996443in}{0.755329in}}%
\pgfpathcurveto{\pgfqpoint{0.988630in}{0.747515in}}{\pgfqpoint{0.984239in}{0.736916in}}{\pgfqpoint{0.984239in}{0.725866in}}%
\pgfpathcurveto{\pgfqpoint{0.984239in}{0.714816in}}{\pgfqpoint{0.988630in}{0.704217in}}{\pgfqpoint{0.996443in}{0.696403in}}%
\pgfpathcurveto{\pgfqpoint{1.004257in}{0.688590in}}{\pgfqpoint{1.014856in}{0.684199in}}{\pgfqpoint{1.025906in}{0.684199in}}%
\pgfpathclose%
\pgfusepath{stroke,fill}%
\end{pgfscope}%
\begin{pgfscope}%
\pgfpathrectangle{\pgfqpoint{0.800000in}{0.528000in}}{\pgfqpoint{4.960000in}{3.696000in}}%
\pgfusepath{clip}%
\pgfsetbuttcap%
\pgfsetroundjoin%
\definecolor{currentfill}{rgb}{0.000000,0.000000,0.000000}%
\pgfsetfillcolor{currentfill}%
\pgfsetlinewidth{1.003750pt}%
\definecolor{currentstroke}{rgb}{0.000000,0.000000,0.000000}%
\pgfsetstrokecolor{currentstroke}%
\pgfsetdash{}{0pt}%
\pgfpathmoveto{\pgfqpoint{1.025906in}{2.334266in}}%
\pgfpathcurveto{\pgfqpoint{1.036956in}{2.334266in}}{\pgfqpoint{1.047555in}{2.338657in}}{\pgfqpoint{1.055369in}{2.346470in}}%
\pgfpathcurveto{\pgfqpoint{1.063182in}{2.354284in}}{\pgfqpoint{1.067573in}{2.364883in}}{\pgfqpoint{1.067573in}{2.375933in}}%
\pgfpathcurveto{\pgfqpoint{1.067573in}{2.386983in}}{\pgfqpoint{1.063182in}{2.397582in}}{\pgfqpoint{1.055369in}{2.405396in}}%
\pgfpathcurveto{\pgfqpoint{1.047555in}{2.413209in}}{\pgfqpoint{1.036956in}{2.417600in}}{\pgfqpoint{1.025906in}{2.417600in}}%
\pgfpathcurveto{\pgfqpoint{1.014856in}{2.417600in}}{\pgfqpoint{1.004257in}{2.413209in}}{\pgfqpoint{0.996443in}{2.405396in}}%
\pgfpathcurveto{\pgfqpoint{0.988630in}{2.397582in}}{\pgfqpoint{0.984239in}{2.386983in}}{\pgfqpoint{0.984239in}{2.375933in}}%
\pgfpathcurveto{\pgfqpoint{0.984239in}{2.364883in}}{\pgfqpoint{0.988630in}{2.354284in}}{\pgfqpoint{0.996443in}{2.346470in}}%
\pgfpathcurveto{\pgfqpoint{1.004257in}{2.338657in}}{\pgfqpoint{1.014856in}{2.334266in}}{\pgfqpoint{1.025906in}{2.334266in}}%
\pgfpathclose%
\pgfusepath{stroke,fill}%
\end{pgfscope}%
\begin{pgfscope}%
\pgfpathrectangle{\pgfqpoint{0.800000in}{0.528000in}}{\pgfqpoint{4.960000in}{3.696000in}}%
\pgfusepath{clip}%
\pgfsetbuttcap%
\pgfsetroundjoin%
\definecolor{currentfill}{rgb}{0.000000,0.000000,0.000000}%
\pgfsetfillcolor{currentfill}%
\pgfsetlinewidth{1.003750pt}%
\definecolor{currentstroke}{rgb}{0.000000,0.000000,0.000000}%
\pgfsetstrokecolor{currentstroke}%
\pgfsetdash{}{0pt}%
\pgfpathmoveto{\pgfqpoint{1.025906in}{0.684199in}}%
\pgfpathcurveto{\pgfqpoint{1.036956in}{0.684199in}}{\pgfqpoint{1.047555in}{0.688590in}}{\pgfqpoint{1.055369in}{0.696403in}}%
\pgfpathcurveto{\pgfqpoint{1.063182in}{0.704217in}}{\pgfqpoint{1.067573in}{0.714816in}}{\pgfqpoint{1.067573in}{0.725866in}}%
\pgfpathcurveto{\pgfqpoint{1.067573in}{0.736916in}}{\pgfqpoint{1.063182in}{0.747515in}}{\pgfqpoint{1.055369in}{0.755329in}}%
\pgfpathcurveto{\pgfqpoint{1.047555in}{0.763142in}}{\pgfqpoint{1.036956in}{0.767533in}}{\pgfqpoint{1.025906in}{0.767533in}}%
\pgfpathcurveto{\pgfqpoint{1.014856in}{0.767533in}}{\pgfqpoint{1.004257in}{0.763142in}}{\pgfqpoint{0.996443in}{0.755329in}}%
\pgfpathcurveto{\pgfqpoint{0.988630in}{0.747515in}}{\pgfqpoint{0.984239in}{0.736916in}}{\pgfqpoint{0.984239in}{0.725866in}}%
\pgfpathcurveto{\pgfqpoint{0.984239in}{0.714816in}}{\pgfqpoint{0.988630in}{0.704217in}}{\pgfqpoint{0.996443in}{0.696403in}}%
\pgfpathcurveto{\pgfqpoint{1.004257in}{0.688590in}}{\pgfqpoint{1.014856in}{0.684199in}}{\pgfqpoint{1.025906in}{0.684199in}}%
\pgfpathclose%
\pgfusepath{stroke,fill}%
\end{pgfscope}%
\begin{pgfscope}%
\pgfpathrectangle{\pgfqpoint{0.800000in}{0.528000in}}{\pgfqpoint{4.960000in}{3.696000in}}%
\pgfusepath{clip}%
\pgfsetbuttcap%
\pgfsetroundjoin%
\definecolor{currentfill}{rgb}{0.000000,0.000000,0.000000}%
\pgfsetfillcolor{currentfill}%
\pgfsetlinewidth{1.003750pt}%
\definecolor{currentstroke}{rgb}{0.000000,0.000000,0.000000}%
\pgfsetstrokecolor{currentstroke}%
\pgfsetdash{}{0pt}%
\pgfpathmoveto{\pgfqpoint{1.025906in}{0.684199in}}%
\pgfpathcurveto{\pgfqpoint{1.036956in}{0.684199in}}{\pgfqpoint{1.047555in}{0.688590in}}{\pgfqpoint{1.055369in}{0.696403in}}%
\pgfpathcurveto{\pgfqpoint{1.063182in}{0.704217in}}{\pgfqpoint{1.067573in}{0.714816in}}{\pgfqpoint{1.067573in}{0.725866in}}%
\pgfpathcurveto{\pgfqpoint{1.067573in}{0.736916in}}{\pgfqpoint{1.063182in}{0.747515in}}{\pgfqpoint{1.055369in}{0.755329in}}%
\pgfpathcurveto{\pgfqpoint{1.047555in}{0.763142in}}{\pgfqpoint{1.036956in}{0.767533in}}{\pgfqpoint{1.025906in}{0.767533in}}%
\pgfpathcurveto{\pgfqpoint{1.014856in}{0.767533in}}{\pgfqpoint{1.004257in}{0.763142in}}{\pgfqpoint{0.996443in}{0.755329in}}%
\pgfpathcurveto{\pgfqpoint{0.988630in}{0.747515in}}{\pgfqpoint{0.984239in}{0.736916in}}{\pgfqpoint{0.984239in}{0.725866in}}%
\pgfpathcurveto{\pgfqpoint{0.984239in}{0.714816in}}{\pgfqpoint{0.988630in}{0.704217in}}{\pgfqpoint{0.996443in}{0.696403in}}%
\pgfpathcurveto{\pgfqpoint{1.004257in}{0.688590in}}{\pgfqpoint{1.014856in}{0.684199in}}{\pgfqpoint{1.025906in}{0.684199in}}%
\pgfpathclose%
\pgfusepath{stroke,fill}%
\end{pgfscope}%
\begin{pgfscope}%
\pgfpathrectangle{\pgfqpoint{0.800000in}{0.528000in}}{\pgfqpoint{4.960000in}{3.696000in}}%
\pgfusepath{clip}%
\pgfsetbuttcap%
\pgfsetroundjoin%
\definecolor{currentfill}{rgb}{0.000000,0.000000,0.000000}%
\pgfsetfillcolor{currentfill}%
\pgfsetlinewidth{1.003750pt}%
\definecolor{currentstroke}{rgb}{0.000000,0.000000,0.000000}%
\pgfsetstrokecolor{currentstroke}%
\pgfsetdash{}{0pt}%
\pgfpathmoveto{\pgfqpoint{1.025906in}{0.684199in}}%
\pgfpathcurveto{\pgfqpoint{1.036956in}{0.684199in}}{\pgfqpoint{1.047555in}{0.688590in}}{\pgfqpoint{1.055369in}{0.696403in}}%
\pgfpathcurveto{\pgfqpoint{1.063182in}{0.704217in}}{\pgfqpoint{1.067573in}{0.714816in}}{\pgfqpoint{1.067573in}{0.725866in}}%
\pgfpathcurveto{\pgfqpoint{1.067573in}{0.736916in}}{\pgfqpoint{1.063182in}{0.747515in}}{\pgfqpoint{1.055369in}{0.755329in}}%
\pgfpathcurveto{\pgfqpoint{1.047555in}{0.763142in}}{\pgfqpoint{1.036956in}{0.767533in}}{\pgfqpoint{1.025906in}{0.767533in}}%
\pgfpathcurveto{\pgfqpoint{1.014856in}{0.767533in}}{\pgfqpoint{1.004257in}{0.763142in}}{\pgfqpoint{0.996443in}{0.755329in}}%
\pgfpathcurveto{\pgfqpoint{0.988630in}{0.747515in}}{\pgfqpoint{0.984239in}{0.736916in}}{\pgfqpoint{0.984239in}{0.725866in}}%
\pgfpathcurveto{\pgfqpoint{0.984239in}{0.714816in}}{\pgfqpoint{0.988630in}{0.704217in}}{\pgfqpoint{0.996443in}{0.696403in}}%
\pgfpathcurveto{\pgfqpoint{1.004257in}{0.688590in}}{\pgfqpoint{1.014856in}{0.684199in}}{\pgfqpoint{1.025906in}{0.684199in}}%
\pgfpathclose%
\pgfusepath{stroke,fill}%
\end{pgfscope}%
\begin{pgfscope}%
\pgfpathrectangle{\pgfqpoint{0.800000in}{0.528000in}}{\pgfqpoint{4.960000in}{3.696000in}}%
\pgfusepath{clip}%
\pgfsetbuttcap%
\pgfsetroundjoin%
\definecolor{currentfill}{rgb}{0.000000,0.000000,0.000000}%
\pgfsetfillcolor{currentfill}%
\pgfsetlinewidth{1.003750pt}%
\definecolor{currentstroke}{rgb}{0.000000,0.000000,0.000000}%
\pgfsetstrokecolor{currentstroke}%
\pgfsetdash{}{0pt}%
\pgfpathmoveto{\pgfqpoint{1.025906in}{0.684199in}}%
\pgfpathcurveto{\pgfqpoint{1.036956in}{0.684199in}}{\pgfqpoint{1.047555in}{0.688590in}}{\pgfqpoint{1.055369in}{0.696403in}}%
\pgfpathcurveto{\pgfqpoint{1.063182in}{0.704217in}}{\pgfqpoint{1.067573in}{0.714816in}}{\pgfqpoint{1.067573in}{0.725866in}}%
\pgfpathcurveto{\pgfqpoint{1.067573in}{0.736916in}}{\pgfqpoint{1.063182in}{0.747515in}}{\pgfqpoint{1.055369in}{0.755329in}}%
\pgfpathcurveto{\pgfqpoint{1.047555in}{0.763142in}}{\pgfqpoint{1.036956in}{0.767533in}}{\pgfqpoint{1.025906in}{0.767533in}}%
\pgfpathcurveto{\pgfqpoint{1.014856in}{0.767533in}}{\pgfqpoint{1.004257in}{0.763142in}}{\pgfqpoint{0.996443in}{0.755329in}}%
\pgfpathcurveto{\pgfqpoint{0.988630in}{0.747515in}}{\pgfqpoint{0.984239in}{0.736916in}}{\pgfqpoint{0.984239in}{0.725866in}}%
\pgfpathcurveto{\pgfqpoint{0.984239in}{0.714816in}}{\pgfqpoint{0.988630in}{0.704217in}}{\pgfqpoint{0.996443in}{0.696403in}}%
\pgfpathcurveto{\pgfqpoint{1.004257in}{0.688590in}}{\pgfqpoint{1.014856in}{0.684199in}}{\pgfqpoint{1.025906in}{0.684199in}}%
\pgfpathclose%
\pgfusepath{stroke,fill}%
\end{pgfscope}%
\begin{pgfscope}%
\pgfpathrectangle{\pgfqpoint{0.800000in}{0.528000in}}{\pgfqpoint{4.960000in}{3.696000in}}%
\pgfusepath{clip}%
\pgfsetbuttcap%
\pgfsetroundjoin%
\definecolor{currentfill}{rgb}{0.000000,0.000000,0.000000}%
\pgfsetfillcolor{currentfill}%
\pgfsetlinewidth{1.003750pt}%
\definecolor{currentstroke}{rgb}{0.000000,0.000000,0.000000}%
\pgfsetstrokecolor{currentstroke}%
\pgfsetdash{}{0pt}%
\pgfpathmoveto{\pgfqpoint{1.025906in}{0.684199in}}%
\pgfpathcurveto{\pgfqpoint{1.036956in}{0.684199in}}{\pgfqpoint{1.047555in}{0.688590in}}{\pgfqpoint{1.055369in}{0.696403in}}%
\pgfpathcurveto{\pgfqpoint{1.063182in}{0.704217in}}{\pgfqpoint{1.067573in}{0.714816in}}{\pgfqpoint{1.067573in}{0.725866in}}%
\pgfpathcurveto{\pgfqpoint{1.067573in}{0.736916in}}{\pgfqpoint{1.063182in}{0.747515in}}{\pgfqpoint{1.055369in}{0.755329in}}%
\pgfpathcurveto{\pgfqpoint{1.047555in}{0.763142in}}{\pgfqpoint{1.036956in}{0.767533in}}{\pgfqpoint{1.025906in}{0.767533in}}%
\pgfpathcurveto{\pgfqpoint{1.014856in}{0.767533in}}{\pgfqpoint{1.004257in}{0.763142in}}{\pgfqpoint{0.996443in}{0.755329in}}%
\pgfpathcurveto{\pgfqpoint{0.988630in}{0.747515in}}{\pgfqpoint{0.984239in}{0.736916in}}{\pgfqpoint{0.984239in}{0.725866in}}%
\pgfpathcurveto{\pgfqpoint{0.984239in}{0.714816in}}{\pgfqpoint{0.988630in}{0.704217in}}{\pgfqpoint{0.996443in}{0.696403in}}%
\pgfpathcurveto{\pgfqpoint{1.004257in}{0.688590in}}{\pgfqpoint{1.014856in}{0.684199in}}{\pgfqpoint{1.025906in}{0.684199in}}%
\pgfpathclose%
\pgfusepath{stroke,fill}%
\end{pgfscope}%
\begin{pgfscope}%
\pgfpathrectangle{\pgfqpoint{0.800000in}{0.528000in}}{\pgfqpoint{4.960000in}{3.696000in}}%
\pgfusepath{clip}%
\pgfsetbuttcap%
\pgfsetroundjoin%
\definecolor{currentfill}{rgb}{0.000000,0.000000,0.000000}%
\pgfsetfillcolor{currentfill}%
\pgfsetlinewidth{1.003750pt}%
\definecolor{currentstroke}{rgb}{0.000000,0.000000,0.000000}%
\pgfsetstrokecolor{currentstroke}%
\pgfsetdash{}{0pt}%
\pgfpathmoveto{\pgfqpoint{1.025906in}{0.684199in}}%
\pgfpathcurveto{\pgfqpoint{1.036956in}{0.684199in}}{\pgfqpoint{1.047555in}{0.688590in}}{\pgfqpoint{1.055369in}{0.696403in}}%
\pgfpathcurveto{\pgfqpoint{1.063182in}{0.704217in}}{\pgfqpoint{1.067573in}{0.714816in}}{\pgfqpoint{1.067573in}{0.725866in}}%
\pgfpathcurveto{\pgfqpoint{1.067573in}{0.736916in}}{\pgfqpoint{1.063182in}{0.747515in}}{\pgfqpoint{1.055369in}{0.755329in}}%
\pgfpathcurveto{\pgfqpoint{1.047555in}{0.763142in}}{\pgfqpoint{1.036956in}{0.767533in}}{\pgfqpoint{1.025906in}{0.767533in}}%
\pgfpathcurveto{\pgfqpoint{1.014856in}{0.767533in}}{\pgfqpoint{1.004257in}{0.763142in}}{\pgfqpoint{0.996443in}{0.755329in}}%
\pgfpathcurveto{\pgfqpoint{0.988630in}{0.747515in}}{\pgfqpoint{0.984239in}{0.736916in}}{\pgfqpoint{0.984239in}{0.725866in}}%
\pgfpathcurveto{\pgfqpoint{0.984239in}{0.714816in}}{\pgfqpoint{0.988630in}{0.704217in}}{\pgfqpoint{0.996443in}{0.696403in}}%
\pgfpathcurveto{\pgfqpoint{1.004257in}{0.688590in}}{\pgfqpoint{1.014856in}{0.684199in}}{\pgfqpoint{1.025906in}{0.684199in}}%
\pgfpathclose%
\pgfusepath{stroke,fill}%
\end{pgfscope}%
\begin{pgfscope}%
\pgfpathrectangle{\pgfqpoint{0.800000in}{0.528000in}}{\pgfqpoint{4.960000in}{3.696000in}}%
\pgfusepath{clip}%
\pgfsetbuttcap%
\pgfsetroundjoin%
\definecolor{currentfill}{rgb}{0.000000,0.000000,0.000000}%
\pgfsetfillcolor{currentfill}%
\pgfsetlinewidth{1.003750pt}%
\definecolor{currentstroke}{rgb}{0.000000,0.000000,0.000000}%
\pgfsetstrokecolor{currentstroke}%
\pgfsetdash{}{0pt}%
\pgfpathmoveto{\pgfqpoint{1.025906in}{0.684199in}}%
\pgfpathcurveto{\pgfqpoint{1.036956in}{0.684199in}}{\pgfqpoint{1.047555in}{0.688590in}}{\pgfqpoint{1.055369in}{0.696403in}}%
\pgfpathcurveto{\pgfqpoint{1.063182in}{0.704217in}}{\pgfqpoint{1.067573in}{0.714816in}}{\pgfqpoint{1.067573in}{0.725866in}}%
\pgfpathcurveto{\pgfqpoint{1.067573in}{0.736916in}}{\pgfqpoint{1.063182in}{0.747515in}}{\pgfqpoint{1.055369in}{0.755329in}}%
\pgfpathcurveto{\pgfqpoint{1.047555in}{0.763142in}}{\pgfqpoint{1.036956in}{0.767533in}}{\pgfqpoint{1.025906in}{0.767533in}}%
\pgfpathcurveto{\pgfqpoint{1.014856in}{0.767533in}}{\pgfqpoint{1.004257in}{0.763142in}}{\pgfqpoint{0.996443in}{0.755329in}}%
\pgfpathcurveto{\pgfqpoint{0.988630in}{0.747515in}}{\pgfqpoint{0.984239in}{0.736916in}}{\pgfqpoint{0.984239in}{0.725866in}}%
\pgfpathcurveto{\pgfqpoint{0.984239in}{0.714816in}}{\pgfqpoint{0.988630in}{0.704217in}}{\pgfqpoint{0.996443in}{0.696403in}}%
\pgfpathcurveto{\pgfqpoint{1.004257in}{0.688590in}}{\pgfqpoint{1.014856in}{0.684199in}}{\pgfqpoint{1.025906in}{0.684199in}}%
\pgfpathclose%
\pgfusepath{stroke,fill}%
\end{pgfscope}%
\begin{pgfscope}%
\pgfpathrectangle{\pgfqpoint{0.800000in}{0.528000in}}{\pgfqpoint{4.960000in}{3.696000in}}%
\pgfusepath{clip}%
\pgfsetbuttcap%
\pgfsetroundjoin%
\definecolor{currentfill}{rgb}{0.000000,0.000000,0.000000}%
\pgfsetfillcolor{currentfill}%
\pgfsetlinewidth{1.003750pt}%
\definecolor{currentstroke}{rgb}{0.000000,0.000000,0.000000}%
\pgfsetstrokecolor{currentstroke}%
\pgfsetdash{}{0pt}%
\pgfpathmoveto{\pgfqpoint{1.025906in}{0.684199in}}%
\pgfpathcurveto{\pgfqpoint{1.036956in}{0.684199in}}{\pgfqpoint{1.047555in}{0.688590in}}{\pgfqpoint{1.055369in}{0.696403in}}%
\pgfpathcurveto{\pgfqpoint{1.063182in}{0.704217in}}{\pgfqpoint{1.067573in}{0.714816in}}{\pgfqpoint{1.067573in}{0.725866in}}%
\pgfpathcurveto{\pgfqpoint{1.067573in}{0.736916in}}{\pgfqpoint{1.063182in}{0.747515in}}{\pgfqpoint{1.055369in}{0.755329in}}%
\pgfpathcurveto{\pgfqpoint{1.047555in}{0.763142in}}{\pgfqpoint{1.036956in}{0.767533in}}{\pgfqpoint{1.025906in}{0.767533in}}%
\pgfpathcurveto{\pgfqpoint{1.014856in}{0.767533in}}{\pgfqpoint{1.004257in}{0.763142in}}{\pgfqpoint{0.996443in}{0.755329in}}%
\pgfpathcurveto{\pgfqpoint{0.988630in}{0.747515in}}{\pgfqpoint{0.984239in}{0.736916in}}{\pgfqpoint{0.984239in}{0.725866in}}%
\pgfpathcurveto{\pgfqpoint{0.984239in}{0.714816in}}{\pgfqpoint{0.988630in}{0.704217in}}{\pgfqpoint{0.996443in}{0.696403in}}%
\pgfpathcurveto{\pgfqpoint{1.004257in}{0.688590in}}{\pgfqpoint{1.014856in}{0.684199in}}{\pgfqpoint{1.025906in}{0.684199in}}%
\pgfpathclose%
\pgfusepath{stroke,fill}%
\end{pgfscope}%
\begin{pgfscope}%
\pgfpathrectangle{\pgfqpoint{0.800000in}{0.528000in}}{\pgfqpoint{4.960000in}{3.696000in}}%
\pgfusepath{clip}%
\pgfsetbuttcap%
\pgfsetroundjoin%
\definecolor{currentfill}{rgb}{0.000000,0.000000,0.000000}%
\pgfsetfillcolor{currentfill}%
\pgfsetlinewidth{1.003750pt}%
\definecolor{currentstroke}{rgb}{0.000000,0.000000,0.000000}%
\pgfsetstrokecolor{currentstroke}%
\pgfsetdash{}{0pt}%
\pgfpathmoveto{\pgfqpoint{1.025906in}{0.684199in}}%
\pgfpathcurveto{\pgfqpoint{1.036956in}{0.684199in}}{\pgfqpoint{1.047555in}{0.688590in}}{\pgfqpoint{1.055369in}{0.696403in}}%
\pgfpathcurveto{\pgfqpoint{1.063182in}{0.704217in}}{\pgfqpoint{1.067573in}{0.714816in}}{\pgfqpoint{1.067573in}{0.725866in}}%
\pgfpathcurveto{\pgfqpoint{1.067573in}{0.736916in}}{\pgfqpoint{1.063182in}{0.747515in}}{\pgfqpoint{1.055369in}{0.755329in}}%
\pgfpathcurveto{\pgfqpoint{1.047555in}{0.763142in}}{\pgfqpoint{1.036956in}{0.767533in}}{\pgfqpoint{1.025906in}{0.767533in}}%
\pgfpathcurveto{\pgfqpoint{1.014856in}{0.767533in}}{\pgfqpoint{1.004257in}{0.763142in}}{\pgfqpoint{0.996443in}{0.755329in}}%
\pgfpathcurveto{\pgfqpoint{0.988630in}{0.747515in}}{\pgfqpoint{0.984239in}{0.736916in}}{\pgfqpoint{0.984239in}{0.725866in}}%
\pgfpathcurveto{\pgfqpoint{0.984239in}{0.714816in}}{\pgfqpoint{0.988630in}{0.704217in}}{\pgfqpoint{0.996443in}{0.696403in}}%
\pgfpathcurveto{\pgfqpoint{1.004257in}{0.688590in}}{\pgfqpoint{1.014856in}{0.684199in}}{\pgfqpoint{1.025906in}{0.684199in}}%
\pgfpathclose%
\pgfusepath{stroke,fill}%
\end{pgfscope}%
\begin{pgfscope}%
\pgfpathrectangle{\pgfqpoint{0.800000in}{0.528000in}}{\pgfqpoint{4.960000in}{3.696000in}}%
\pgfusepath{clip}%
\pgfsetbuttcap%
\pgfsetroundjoin%
\definecolor{currentfill}{rgb}{0.000000,0.000000,0.000000}%
\pgfsetfillcolor{currentfill}%
\pgfsetlinewidth{1.003750pt}%
\definecolor{currentstroke}{rgb}{0.000000,0.000000,0.000000}%
\pgfsetstrokecolor{currentstroke}%
\pgfsetdash{}{0pt}%
\pgfpathmoveto{\pgfqpoint{1.025906in}{0.684199in}}%
\pgfpathcurveto{\pgfqpoint{1.036956in}{0.684199in}}{\pgfqpoint{1.047555in}{0.688590in}}{\pgfqpoint{1.055369in}{0.696403in}}%
\pgfpathcurveto{\pgfqpoint{1.063182in}{0.704217in}}{\pgfqpoint{1.067573in}{0.714816in}}{\pgfqpoint{1.067573in}{0.725866in}}%
\pgfpathcurveto{\pgfqpoint{1.067573in}{0.736916in}}{\pgfqpoint{1.063182in}{0.747515in}}{\pgfqpoint{1.055369in}{0.755329in}}%
\pgfpathcurveto{\pgfqpoint{1.047555in}{0.763142in}}{\pgfqpoint{1.036956in}{0.767533in}}{\pgfqpoint{1.025906in}{0.767533in}}%
\pgfpathcurveto{\pgfqpoint{1.014856in}{0.767533in}}{\pgfqpoint{1.004257in}{0.763142in}}{\pgfqpoint{0.996443in}{0.755329in}}%
\pgfpathcurveto{\pgfqpoint{0.988630in}{0.747515in}}{\pgfqpoint{0.984239in}{0.736916in}}{\pgfqpoint{0.984239in}{0.725866in}}%
\pgfpathcurveto{\pgfqpoint{0.984239in}{0.714816in}}{\pgfqpoint{0.988630in}{0.704217in}}{\pgfqpoint{0.996443in}{0.696403in}}%
\pgfpathcurveto{\pgfqpoint{1.004257in}{0.688590in}}{\pgfqpoint{1.014856in}{0.684199in}}{\pgfqpoint{1.025906in}{0.684199in}}%
\pgfpathclose%
\pgfusepath{stroke,fill}%
\end{pgfscope}%
\begin{pgfscope}%
\pgfpathrectangle{\pgfqpoint{0.800000in}{0.528000in}}{\pgfqpoint{4.960000in}{3.696000in}}%
\pgfusepath{clip}%
\pgfsetbuttcap%
\pgfsetroundjoin%
\definecolor{currentfill}{rgb}{0.000000,0.000000,0.000000}%
\pgfsetfillcolor{currentfill}%
\pgfsetlinewidth{1.003750pt}%
\definecolor{currentstroke}{rgb}{0.000000,0.000000,0.000000}%
\pgfsetstrokecolor{currentstroke}%
\pgfsetdash{}{0pt}%
\pgfpathmoveto{\pgfqpoint{1.025906in}{0.684199in}}%
\pgfpathcurveto{\pgfqpoint{1.036956in}{0.684199in}}{\pgfqpoint{1.047555in}{0.688590in}}{\pgfqpoint{1.055369in}{0.696403in}}%
\pgfpathcurveto{\pgfqpoint{1.063182in}{0.704217in}}{\pgfqpoint{1.067573in}{0.714816in}}{\pgfqpoint{1.067573in}{0.725866in}}%
\pgfpathcurveto{\pgfqpoint{1.067573in}{0.736916in}}{\pgfqpoint{1.063182in}{0.747515in}}{\pgfqpoint{1.055369in}{0.755329in}}%
\pgfpathcurveto{\pgfqpoint{1.047555in}{0.763142in}}{\pgfqpoint{1.036956in}{0.767533in}}{\pgfqpoint{1.025906in}{0.767533in}}%
\pgfpathcurveto{\pgfqpoint{1.014856in}{0.767533in}}{\pgfqpoint{1.004257in}{0.763142in}}{\pgfqpoint{0.996443in}{0.755329in}}%
\pgfpathcurveto{\pgfqpoint{0.988630in}{0.747515in}}{\pgfqpoint{0.984239in}{0.736916in}}{\pgfqpoint{0.984239in}{0.725866in}}%
\pgfpathcurveto{\pgfqpoint{0.984239in}{0.714816in}}{\pgfqpoint{0.988630in}{0.704217in}}{\pgfqpoint{0.996443in}{0.696403in}}%
\pgfpathcurveto{\pgfqpoint{1.004257in}{0.688590in}}{\pgfqpoint{1.014856in}{0.684199in}}{\pgfqpoint{1.025906in}{0.684199in}}%
\pgfpathclose%
\pgfusepath{stroke,fill}%
\end{pgfscope}%
\begin{pgfscope}%
\pgfpathrectangle{\pgfqpoint{0.800000in}{0.528000in}}{\pgfqpoint{4.960000in}{3.696000in}}%
\pgfusepath{clip}%
\pgfsetbuttcap%
\pgfsetroundjoin%
\definecolor{currentfill}{rgb}{0.000000,0.000000,0.000000}%
\pgfsetfillcolor{currentfill}%
\pgfsetlinewidth{1.003750pt}%
\definecolor{currentstroke}{rgb}{0.000000,0.000000,0.000000}%
\pgfsetstrokecolor{currentstroke}%
\pgfsetdash{}{0pt}%
\pgfpathmoveto{\pgfqpoint{2.518786in}{2.334266in}}%
\pgfpathcurveto{\pgfqpoint{2.529836in}{2.334266in}}{\pgfqpoint{2.540435in}{2.338657in}}{\pgfqpoint{2.548249in}{2.346470in}}%
\pgfpathcurveto{\pgfqpoint{2.556062in}{2.354284in}}{\pgfqpoint{2.560452in}{2.364883in}}{\pgfqpoint{2.560452in}{2.375933in}}%
\pgfpathcurveto{\pgfqpoint{2.560452in}{2.386983in}}{\pgfqpoint{2.556062in}{2.397582in}}{\pgfqpoint{2.548249in}{2.405396in}}%
\pgfpathcurveto{\pgfqpoint{2.540435in}{2.413209in}}{\pgfqpoint{2.529836in}{2.417600in}}{\pgfqpoint{2.518786in}{2.417600in}}%
\pgfpathcurveto{\pgfqpoint{2.507736in}{2.417600in}}{\pgfqpoint{2.497137in}{2.413209in}}{\pgfqpoint{2.489323in}{2.405396in}}%
\pgfpathcurveto{\pgfqpoint{2.481509in}{2.397582in}}{\pgfqpoint{2.477119in}{2.386983in}}{\pgfqpoint{2.477119in}{2.375933in}}%
\pgfpathcurveto{\pgfqpoint{2.477119in}{2.364883in}}{\pgfqpoint{2.481509in}{2.354284in}}{\pgfqpoint{2.489323in}{2.346470in}}%
\pgfpathcurveto{\pgfqpoint{2.497137in}{2.338657in}}{\pgfqpoint{2.507736in}{2.334266in}}{\pgfqpoint{2.518786in}{2.334266in}}%
\pgfpathclose%
\pgfusepath{stroke,fill}%
\end{pgfscope}%
\begin{pgfscope}%
\pgfpathrectangle{\pgfqpoint{0.800000in}{0.528000in}}{\pgfqpoint{4.960000in}{3.696000in}}%
\pgfusepath{clip}%
\pgfsetbuttcap%
\pgfsetroundjoin%
\definecolor{currentfill}{rgb}{0.000000,0.000000,0.000000}%
\pgfsetfillcolor{currentfill}%
\pgfsetlinewidth{1.003750pt}%
\definecolor{currentstroke}{rgb}{0.000000,0.000000,0.000000}%
\pgfsetstrokecolor{currentstroke}%
\pgfsetdash{}{0pt}%
\pgfpathmoveto{\pgfqpoint{2.518786in}{2.334266in}}%
\pgfpathcurveto{\pgfqpoint{2.529836in}{2.334266in}}{\pgfqpoint{2.540435in}{2.338657in}}{\pgfqpoint{2.548249in}{2.346470in}}%
\pgfpathcurveto{\pgfqpoint{2.556062in}{2.354284in}}{\pgfqpoint{2.560452in}{2.364883in}}{\pgfqpoint{2.560452in}{2.375933in}}%
\pgfpathcurveto{\pgfqpoint{2.560452in}{2.386983in}}{\pgfqpoint{2.556062in}{2.397582in}}{\pgfqpoint{2.548249in}{2.405396in}}%
\pgfpathcurveto{\pgfqpoint{2.540435in}{2.413209in}}{\pgfqpoint{2.529836in}{2.417600in}}{\pgfqpoint{2.518786in}{2.417600in}}%
\pgfpathcurveto{\pgfqpoint{2.507736in}{2.417600in}}{\pgfqpoint{2.497137in}{2.413209in}}{\pgfqpoint{2.489323in}{2.405396in}}%
\pgfpathcurveto{\pgfqpoint{2.481509in}{2.397582in}}{\pgfqpoint{2.477119in}{2.386983in}}{\pgfqpoint{2.477119in}{2.375933in}}%
\pgfpathcurveto{\pgfqpoint{2.477119in}{2.364883in}}{\pgfqpoint{2.481509in}{2.354284in}}{\pgfqpoint{2.489323in}{2.346470in}}%
\pgfpathcurveto{\pgfqpoint{2.497137in}{2.338657in}}{\pgfqpoint{2.507736in}{2.334266in}}{\pgfqpoint{2.518786in}{2.334266in}}%
\pgfpathclose%
\pgfusepath{stroke,fill}%
\end{pgfscope}%
\begin{pgfscope}%
\pgfpathrectangle{\pgfqpoint{0.800000in}{0.528000in}}{\pgfqpoint{4.960000in}{3.696000in}}%
\pgfusepath{clip}%
\pgfsetbuttcap%
\pgfsetroundjoin%
\definecolor{currentfill}{rgb}{0.000000,0.000000,0.000000}%
\pgfsetfillcolor{currentfill}%
\pgfsetlinewidth{1.003750pt}%
\definecolor{currentstroke}{rgb}{0.000000,0.000000,0.000000}%
\pgfsetstrokecolor{currentstroke}%
\pgfsetdash{}{0pt}%
\pgfpathmoveto{\pgfqpoint{2.518786in}{2.334266in}}%
\pgfpathcurveto{\pgfqpoint{2.529836in}{2.334266in}}{\pgfqpoint{2.540435in}{2.338657in}}{\pgfqpoint{2.548249in}{2.346470in}}%
\pgfpathcurveto{\pgfqpoint{2.556062in}{2.354284in}}{\pgfqpoint{2.560452in}{2.364883in}}{\pgfqpoint{2.560452in}{2.375933in}}%
\pgfpathcurveto{\pgfqpoint{2.560452in}{2.386983in}}{\pgfqpoint{2.556062in}{2.397582in}}{\pgfqpoint{2.548249in}{2.405396in}}%
\pgfpathcurveto{\pgfqpoint{2.540435in}{2.413209in}}{\pgfqpoint{2.529836in}{2.417600in}}{\pgfqpoint{2.518786in}{2.417600in}}%
\pgfpathcurveto{\pgfqpoint{2.507736in}{2.417600in}}{\pgfqpoint{2.497137in}{2.413209in}}{\pgfqpoint{2.489323in}{2.405396in}}%
\pgfpathcurveto{\pgfqpoint{2.481509in}{2.397582in}}{\pgfqpoint{2.477119in}{2.386983in}}{\pgfqpoint{2.477119in}{2.375933in}}%
\pgfpathcurveto{\pgfqpoint{2.477119in}{2.364883in}}{\pgfqpoint{2.481509in}{2.354284in}}{\pgfqpoint{2.489323in}{2.346470in}}%
\pgfpathcurveto{\pgfqpoint{2.497137in}{2.338657in}}{\pgfqpoint{2.507736in}{2.334266in}}{\pgfqpoint{2.518786in}{2.334266in}}%
\pgfpathclose%
\pgfusepath{stroke,fill}%
\end{pgfscope}%
\begin{pgfscope}%
\pgfpathrectangle{\pgfqpoint{0.800000in}{0.528000in}}{\pgfqpoint{4.960000in}{3.696000in}}%
\pgfusepath{clip}%
\pgfsetbuttcap%
\pgfsetroundjoin%
\definecolor{currentfill}{rgb}{0.000000,0.000000,0.000000}%
\pgfsetfillcolor{currentfill}%
\pgfsetlinewidth{1.003750pt}%
\definecolor{currentstroke}{rgb}{0.000000,0.000000,0.000000}%
\pgfsetstrokecolor{currentstroke}%
\pgfsetdash{}{0pt}%
\pgfpathmoveto{\pgfqpoint{2.518786in}{2.334266in}}%
\pgfpathcurveto{\pgfqpoint{2.529836in}{2.334266in}}{\pgfqpoint{2.540435in}{2.338657in}}{\pgfqpoint{2.548249in}{2.346470in}}%
\pgfpathcurveto{\pgfqpoint{2.556062in}{2.354284in}}{\pgfqpoint{2.560452in}{2.364883in}}{\pgfqpoint{2.560452in}{2.375933in}}%
\pgfpathcurveto{\pgfqpoint{2.560452in}{2.386983in}}{\pgfqpoint{2.556062in}{2.397582in}}{\pgfqpoint{2.548249in}{2.405396in}}%
\pgfpathcurveto{\pgfqpoint{2.540435in}{2.413209in}}{\pgfqpoint{2.529836in}{2.417600in}}{\pgfqpoint{2.518786in}{2.417600in}}%
\pgfpathcurveto{\pgfqpoint{2.507736in}{2.417600in}}{\pgfqpoint{2.497137in}{2.413209in}}{\pgfqpoint{2.489323in}{2.405396in}}%
\pgfpathcurveto{\pgfqpoint{2.481509in}{2.397582in}}{\pgfqpoint{2.477119in}{2.386983in}}{\pgfqpoint{2.477119in}{2.375933in}}%
\pgfpathcurveto{\pgfqpoint{2.477119in}{2.364883in}}{\pgfqpoint{2.481509in}{2.354284in}}{\pgfqpoint{2.489323in}{2.346470in}}%
\pgfpathcurveto{\pgfqpoint{2.497137in}{2.338657in}}{\pgfqpoint{2.507736in}{2.334266in}}{\pgfqpoint{2.518786in}{2.334266in}}%
\pgfpathclose%
\pgfusepath{stroke,fill}%
\end{pgfscope}%
\begin{pgfscope}%
\pgfpathrectangle{\pgfqpoint{0.800000in}{0.528000in}}{\pgfqpoint{4.960000in}{3.696000in}}%
\pgfusepath{clip}%
\pgfsetbuttcap%
\pgfsetroundjoin%
\definecolor{currentfill}{rgb}{0.000000,0.000000,0.000000}%
\pgfsetfillcolor{currentfill}%
\pgfsetlinewidth{1.003750pt}%
\definecolor{currentstroke}{rgb}{0.000000,0.000000,0.000000}%
\pgfsetstrokecolor{currentstroke}%
\pgfsetdash{}{0pt}%
\pgfpathmoveto{\pgfqpoint{2.518786in}{2.334266in}}%
\pgfpathcurveto{\pgfqpoint{2.529836in}{2.334266in}}{\pgfqpoint{2.540435in}{2.338657in}}{\pgfqpoint{2.548249in}{2.346470in}}%
\pgfpathcurveto{\pgfqpoint{2.556062in}{2.354284in}}{\pgfqpoint{2.560452in}{2.364883in}}{\pgfqpoint{2.560452in}{2.375933in}}%
\pgfpathcurveto{\pgfqpoint{2.560452in}{2.386983in}}{\pgfqpoint{2.556062in}{2.397582in}}{\pgfqpoint{2.548249in}{2.405396in}}%
\pgfpathcurveto{\pgfqpoint{2.540435in}{2.413209in}}{\pgfqpoint{2.529836in}{2.417600in}}{\pgfqpoint{2.518786in}{2.417600in}}%
\pgfpathcurveto{\pgfqpoint{2.507736in}{2.417600in}}{\pgfqpoint{2.497137in}{2.413209in}}{\pgfqpoint{2.489323in}{2.405396in}}%
\pgfpathcurveto{\pgfqpoint{2.481509in}{2.397582in}}{\pgfqpoint{2.477119in}{2.386983in}}{\pgfqpoint{2.477119in}{2.375933in}}%
\pgfpathcurveto{\pgfqpoint{2.477119in}{2.364883in}}{\pgfqpoint{2.481509in}{2.354284in}}{\pgfqpoint{2.489323in}{2.346470in}}%
\pgfpathcurveto{\pgfqpoint{2.497137in}{2.338657in}}{\pgfqpoint{2.507736in}{2.334266in}}{\pgfqpoint{2.518786in}{2.334266in}}%
\pgfpathclose%
\pgfusepath{stroke,fill}%
\end{pgfscope}%
\begin{pgfscope}%
\pgfpathrectangle{\pgfqpoint{0.800000in}{0.528000in}}{\pgfqpoint{4.960000in}{3.696000in}}%
\pgfusepath{clip}%
\pgfsetbuttcap%
\pgfsetroundjoin%
\definecolor{currentfill}{rgb}{0.000000,0.000000,0.000000}%
\pgfsetfillcolor{currentfill}%
\pgfsetlinewidth{1.003750pt}%
\definecolor{currentstroke}{rgb}{0.000000,0.000000,0.000000}%
\pgfsetstrokecolor{currentstroke}%
\pgfsetdash{}{0pt}%
\pgfpathmoveto{\pgfqpoint{2.518786in}{2.334266in}}%
\pgfpathcurveto{\pgfqpoint{2.529836in}{2.334266in}}{\pgfqpoint{2.540435in}{2.338657in}}{\pgfqpoint{2.548249in}{2.346470in}}%
\pgfpathcurveto{\pgfqpoint{2.556062in}{2.354284in}}{\pgfqpoint{2.560452in}{2.364883in}}{\pgfqpoint{2.560452in}{2.375933in}}%
\pgfpathcurveto{\pgfqpoint{2.560452in}{2.386983in}}{\pgfqpoint{2.556062in}{2.397582in}}{\pgfqpoint{2.548249in}{2.405396in}}%
\pgfpathcurveto{\pgfqpoint{2.540435in}{2.413209in}}{\pgfqpoint{2.529836in}{2.417600in}}{\pgfqpoint{2.518786in}{2.417600in}}%
\pgfpathcurveto{\pgfqpoint{2.507736in}{2.417600in}}{\pgfqpoint{2.497137in}{2.413209in}}{\pgfqpoint{2.489323in}{2.405396in}}%
\pgfpathcurveto{\pgfqpoint{2.481509in}{2.397582in}}{\pgfqpoint{2.477119in}{2.386983in}}{\pgfqpoint{2.477119in}{2.375933in}}%
\pgfpathcurveto{\pgfqpoint{2.477119in}{2.364883in}}{\pgfqpoint{2.481509in}{2.354284in}}{\pgfqpoint{2.489323in}{2.346470in}}%
\pgfpathcurveto{\pgfqpoint{2.497137in}{2.338657in}}{\pgfqpoint{2.507736in}{2.334266in}}{\pgfqpoint{2.518786in}{2.334266in}}%
\pgfpathclose%
\pgfusepath{stroke,fill}%
\end{pgfscope}%
\begin{pgfscope}%
\pgfpathrectangle{\pgfqpoint{0.800000in}{0.528000in}}{\pgfqpoint{4.960000in}{3.696000in}}%
\pgfusepath{clip}%
\pgfsetbuttcap%
\pgfsetroundjoin%
\definecolor{currentfill}{rgb}{0.000000,0.000000,0.000000}%
\pgfsetfillcolor{currentfill}%
\pgfsetlinewidth{1.003750pt}%
\definecolor{currentstroke}{rgb}{0.000000,0.000000,0.000000}%
\pgfsetstrokecolor{currentstroke}%
\pgfsetdash{}{0pt}%
\pgfpathmoveto{\pgfqpoint{2.518786in}{2.334266in}}%
\pgfpathcurveto{\pgfqpoint{2.529836in}{2.334266in}}{\pgfqpoint{2.540435in}{2.338657in}}{\pgfqpoint{2.548249in}{2.346470in}}%
\pgfpathcurveto{\pgfqpoint{2.556062in}{2.354284in}}{\pgfqpoint{2.560452in}{2.364883in}}{\pgfqpoint{2.560452in}{2.375933in}}%
\pgfpathcurveto{\pgfqpoint{2.560452in}{2.386983in}}{\pgfqpoint{2.556062in}{2.397582in}}{\pgfqpoint{2.548249in}{2.405396in}}%
\pgfpathcurveto{\pgfqpoint{2.540435in}{2.413209in}}{\pgfqpoint{2.529836in}{2.417600in}}{\pgfqpoint{2.518786in}{2.417600in}}%
\pgfpathcurveto{\pgfqpoint{2.507736in}{2.417600in}}{\pgfqpoint{2.497137in}{2.413209in}}{\pgfqpoint{2.489323in}{2.405396in}}%
\pgfpathcurveto{\pgfqpoint{2.481509in}{2.397582in}}{\pgfqpoint{2.477119in}{2.386983in}}{\pgfqpoint{2.477119in}{2.375933in}}%
\pgfpathcurveto{\pgfqpoint{2.477119in}{2.364883in}}{\pgfqpoint{2.481509in}{2.354284in}}{\pgfqpoint{2.489323in}{2.346470in}}%
\pgfpathcurveto{\pgfqpoint{2.497137in}{2.338657in}}{\pgfqpoint{2.507736in}{2.334266in}}{\pgfqpoint{2.518786in}{2.334266in}}%
\pgfpathclose%
\pgfusepath{stroke,fill}%
\end{pgfscope}%
\begin{pgfscope}%
\pgfpathrectangle{\pgfqpoint{0.800000in}{0.528000in}}{\pgfqpoint{4.960000in}{3.696000in}}%
\pgfusepath{clip}%
\pgfsetbuttcap%
\pgfsetroundjoin%
\definecolor{currentfill}{rgb}{0.000000,0.000000,0.000000}%
\pgfsetfillcolor{currentfill}%
\pgfsetlinewidth{1.003750pt}%
\definecolor{currentstroke}{rgb}{0.000000,0.000000,0.000000}%
\pgfsetstrokecolor{currentstroke}%
\pgfsetdash{}{0pt}%
\pgfpathmoveto{\pgfqpoint{2.518786in}{2.334266in}}%
\pgfpathcurveto{\pgfqpoint{2.529836in}{2.334266in}}{\pgfqpoint{2.540435in}{2.338657in}}{\pgfqpoint{2.548249in}{2.346470in}}%
\pgfpathcurveto{\pgfqpoint{2.556062in}{2.354284in}}{\pgfqpoint{2.560452in}{2.364883in}}{\pgfqpoint{2.560452in}{2.375933in}}%
\pgfpathcurveto{\pgfqpoint{2.560452in}{2.386983in}}{\pgfqpoint{2.556062in}{2.397582in}}{\pgfqpoint{2.548249in}{2.405396in}}%
\pgfpathcurveto{\pgfqpoint{2.540435in}{2.413209in}}{\pgfqpoint{2.529836in}{2.417600in}}{\pgfqpoint{2.518786in}{2.417600in}}%
\pgfpathcurveto{\pgfqpoint{2.507736in}{2.417600in}}{\pgfqpoint{2.497137in}{2.413209in}}{\pgfqpoint{2.489323in}{2.405396in}}%
\pgfpathcurveto{\pgfqpoint{2.481509in}{2.397582in}}{\pgfqpoint{2.477119in}{2.386983in}}{\pgfqpoint{2.477119in}{2.375933in}}%
\pgfpathcurveto{\pgfqpoint{2.477119in}{2.364883in}}{\pgfqpoint{2.481509in}{2.354284in}}{\pgfqpoint{2.489323in}{2.346470in}}%
\pgfpathcurveto{\pgfqpoint{2.497137in}{2.338657in}}{\pgfqpoint{2.507736in}{2.334266in}}{\pgfqpoint{2.518786in}{2.334266in}}%
\pgfpathclose%
\pgfusepath{stroke,fill}%
\end{pgfscope}%
\begin{pgfscope}%
\pgfpathrectangle{\pgfqpoint{0.800000in}{0.528000in}}{\pgfqpoint{4.960000in}{3.696000in}}%
\pgfusepath{clip}%
\pgfsetbuttcap%
\pgfsetroundjoin%
\definecolor{currentfill}{rgb}{0.000000,0.000000,0.000000}%
\pgfsetfillcolor{currentfill}%
\pgfsetlinewidth{1.003750pt}%
\definecolor{currentstroke}{rgb}{0.000000,0.000000,0.000000}%
\pgfsetstrokecolor{currentstroke}%
\pgfsetdash{}{0pt}%
\pgfpathmoveto{\pgfqpoint{2.518786in}{2.334266in}}%
\pgfpathcurveto{\pgfqpoint{2.529836in}{2.334266in}}{\pgfqpoint{2.540435in}{2.338657in}}{\pgfqpoint{2.548249in}{2.346470in}}%
\pgfpathcurveto{\pgfqpoint{2.556062in}{2.354284in}}{\pgfqpoint{2.560452in}{2.364883in}}{\pgfqpoint{2.560452in}{2.375933in}}%
\pgfpathcurveto{\pgfqpoint{2.560452in}{2.386983in}}{\pgfqpoint{2.556062in}{2.397582in}}{\pgfqpoint{2.548249in}{2.405396in}}%
\pgfpathcurveto{\pgfqpoint{2.540435in}{2.413209in}}{\pgfqpoint{2.529836in}{2.417600in}}{\pgfqpoint{2.518786in}{2.417600in}}%
\pgfpathcurveto{\pgfqpoint{2.507736in}{2.417600in}}{\pgfqpoint{2.497137in}{2.413209in}}{\pgfqpoint{2.489323in}{2.405396in}}%
\pgfpathcurveto{\pgfqpoint{2.481509in}{2.397582in}}{\pgfqpoint{2.477119in}{2.386983in}}{\pgfqpoint{2.477119in}{2.375933in}}%
\pgfpathcurveto{\pgfqpoint{2.477119in}{2.364883in}}{\pgfqpoint{2.481509in}{2.354284in}}{\pgfqpoint{2.489323in}{2.346470in}}%
\pgfpathcurveto{\pgfqpoint{2.497137in}{2.338657in}}{\pgfqpoint{2.507736in}{2.334266in}}{\pgfqpoint{2.518786in}{2.334266in}}%
\pgfpathclose%
\pgfusepath{stroke,fill}%
\end{pgfscope}%
\begin{pgfscope}%
\pgfpathrectangle{\pgfqpoint{0.800000in}{0.528000in}}{\pgfqpoint{4.960000in}{3.696000in}}%
\pgfusepath{clip}%
\pgfsetbuttcap%
\pgfsetroundjoin%
\definecolor{currentfill}{rgb}{0.000000,0.000000,0.000000}%
\pgfsetfillcolor{currentfill}%
\pgfsetlinewidth{1.003750pt}%
\definecolor{currentstroke}{rgb}{0.000000,0.000000,0.000000}%
\pgfsetstrokecolor{currentstroke}%
\pgfsetdash{}{0pt}%
\pgfpathmoveto{\pgfqpoint{2.518786in}{2.334266in}}%
\pgfpathcurveto{\pgfqpoint{2.529836in}{2.334266in}}{\pgfqpoint{2.540435in}{2.338657in}}{\pgfqpoint{2.548249in}{2.346470in}}%
\pgfpathcurveto{\pgfqpoint{2.556062in}{2.354284in}}{\pgfqpoint{2.560452in}{2.364883in}}{\pgfqpoint{2.560452in}{2.375933in}}%
\pgfpathcurveto{\pgfqpoint{2.560452in}{2.386983in}}{\pgfqpoint{2.556062in}{2.397582in}}{\pgfqpoint{2.548249in}{2.405396in}}%
\pgfpathcurveto{\pgfqpoint{2.540435in}{2.413209in}}{\pgfqpoint{2.529836in}{2.417600in}}{\pgfqpoint{2.518786in}{2.417600in}}%
\pgfpathcurveto{\pgfqpoint{2.507736in}{2.417600in}}{\pgfqpoint{2.497137in}{2.413209in}}{\pgfqpoint{2.489323in}{2.405396in}}%
\pgfpathcurveto{\pgfqpoint{2.481509in}{2.397582in}}{\pgfqpoint{2.477119in}{2.386983in}}{\pgfqpoint{2.477119in}{2.375933in}}%
\pgfpathcurveto{\pgfqpoint{2.477119in}{2.364883in}}{\pgfqpoint{2.481509in}{2.354284in}}{\pgfqpoint{2.489323in}{2.346470in}}%
\pgfpathcurveto{\pgfqpoint{2.497137in}{2.338657in}}{\pgfqpoint{2.507736in}{2.334266in}}{\pgfqpoint{2.518786in}{2.334266in}}%
\pgfpathclose%
\pgfusepath{stroke,fill}%
\end{pgfscope}%
\begin{pgfscope}%
\pgfpathrectangle{\pgfqpoint{0.800000in}{0.528000in}}{\pgfqpoint{4.960000in}{3.696000in}}%
\pgfusepath{clip}%
\pgfsetbuttcap%
\pgfsetroundjoin%
\definecolor{currentfill}{rgb}{0.000000,0.000000,0.000000}%
\pgfsetfillcolor{currentfill}%
\pgfsetlinewidth{1.003750pt}%
\definecolor{currentstroke}{rgb}{0.000000,0.000000,0.000000}%
\pgfsetstrokecolor{currentstroke}%
\pgfsetdash{}{0pt}%
\pgfpathmoveto{\pgfqpoint{2.518786in}{3.984333in}}%
\pgfpathcurveto{\pgfqpoint{2.529836in}{3.984333in}}{\pgfqpoint{2.540435in}{3.988724in}}{\pgfqpoint{2.548249in}{3.996537in}}%
\pgfpathcurveto{\pgfqpoint{2.556062in}{4.004351in}}{\pgfqpoint{2.560452in}{4.014950in}}{\pgfqpoint{2.560452in}{4.026000in}}%
\pgfpathcurveto{\pgfqpoint{2.560452in}{4.037050in}}{\pgfqpoint{2.556062in}{4.047649in}}{\pgfqpoint{2.548249in}{4.055463in}}%
\pgfpathcurveto{\pgfqpoint{2.540435in}{4.063276in}}{\pgfqpoint{2.529836in}{4.067667in}}{\pgfqpoint{2.518786in}{4.067667in}}%
\pgfpathcurveto{\pgfqpoint{2.507736in}{4.067667in}}{\pgfqpoint{2.497137in}{4.063276in}}{\pgfqpoint{2.489323in}{4.055463in}}%
\pgfpathcurveto{\pgfqpoint{2.481509in}{4.047649in}}{\pgfqpoint{2.477119in}{4.037050in}}{\pgfqpoint{2.477119in}{4.026000in}}%
\pgfpathcurveto{\pgfqpoint{2.477119in}{4.014950in}}{\pgfqpoint{2.481509in}{4.004351in}}{\pgfqpoint{2.489323in}{3.996537in}}%
\pgfpathcurveto{\pgfqpoint{2.497137in}{3.988724in}}{\pgfqpoint{2.507736in}{3.984333in}}{\pgfqpoint{2.518786in}{3.984333in}}%
\pgfpathclose%
\pgfusepath{stroke,fill}%
\end{pgfscope}%
\begin{pgfscope}%
\pgfpathrectangle{\pgfqpoint{0.800000in}{0.528000in}}{\pgfqpoint{4.960000in}{3.696000in}}%
\pgfusepath{clip}%
\pgfsetbuttcap%
\pgfsetroundjoin%
\definecolor{currentfill}{rgb}{0.000000,0.000000,0.000000}%
\pgfsetfillcolor{currentfill}%
\pgfsetlinewidth{1.003750pt}%
\definecolor{currentstroke}{rgb}{0.000000,0.000000,0.000000}%
\pgfsetstrokecolor{currentstroke}%
\pgfsetdash{}{0pt}%
\pgfpathmoveto{\pgfqpoint{2.518786in}{2.334266in}}%
\pgfpathcurveto{\pgfqpoint{2.529836in}{2.334266in}}{\pgfqpoint{2.540435in}{2.338657in}}{\pgfqpoint{2.548249in}{2.346470in}}%
\pgfpathcurveto{\pgfqpoint{2.556062in}{2.354284in}}{\pgfqpoint{2.560452in}{2.364883in}}{\pgfqpoint{2.560452in}{2.375933in}}%
\pgfpathcurveto{\pgfqpoint{2.560452in}{2.386983in}}{\pgfqpoint{2.556062in}{2.397582in}}{\pgfqpoint{2.548249in}{2.405396in}}%
\pgfpathcurveto{\pgfqpoint{2.540435in}{2.413209in}}{\pgfqpoint{2.529836in}{2.417600in}}{\pgfqpoint{2.518786in}{2.417600in}}%
\pgfpathcurveto{\pgfqpoint{2.507736in}{2.417600in}}{\pgfqpoint{2.497137in}{2.413209in}}{\pgfqpoint{2.489323in}{2.405396in}}%
\pgfpathcurveto{\pgfqpoint{2.481509in}{2.397582in}}{\pgfqpoint{2.477119in}{2.386983in}}{\pgfqpoint{2.477119in}{2.375933in}}%
\pgfpathcurveto{\pgfqpoint{2.477119in}{2.364883in}}{\pgfqpoint{2.481509in}{2.354284in}}{\pgfqpoint{2.489323in}{2.346470in}}%
\pgfpathcurveto{\pgfqpoint{2.497137in}{2.338657in}}{\pgfqpoint{2.507736in}{2.334266in}}{\pgfqpoint{2.518786in}{2.334266in}}%
\pgfpathclose%
\pgfusepath{stroke,fill}%
\end{pgfscope}%
\begin{pgfscope}%
\pgfpathrectangle{\pgfqpoint{0.800000in}{0.528000in}}{\pgfqpoint{4.960000in}{3.696000in}}%
\pgfusepath{clip}%
\pgfsetbuttcap%
\pgfsetroundjoin%
\definecolor{currentfill}{rgb}{0.000000,0.000000,0.000000}%
\pgfsetfillcolor{currentfill}%
\pgfsetlinewidth{1.003750pt}%
\definecolor{currentstroke}{rgb}{0.000000,0.000000,0.000000}%
\pgfsetstrokecolor{currentstroke}%
\pgfsetdash{}{0pt}%
\pgfpathmoveto{\pgfqpoint{2.518786in}{2.334266in}}%
\pgfpathcurveto{\pgfqpoint{2.529836in}{2.334266in}}{\pgfqpoint{2.540435in}{2.338657in}}{\pgfqpoint{2.548249in}{2.346470in}}%
\pgfpathcurveto{\pgfqpoint{2.556062in}{2.354284in}}{\pgfqpoint{2.560452in}{2.364883in}}{\pgfqpoint{2.560452in}{2.375933in}}%
\pgfpathcurveto{\pgfqpoint{2.560452in}{2.386983in}}{\pgfqpoint{2.556062in}{2.397582in}}{\pgfqpoint{2.548249in}{2.405396in}}%
\pgfpathcurveto{\pgfqpoint{2.540435in}{2.413209in}}{\pgfqpoint{2.529836in}{2.417600in}}{\pgfqpoint{2.518786in}{2.417600in}}%
\pgfpathcurveto{\pgfqpoint{2.507736in}{2.417600in}}{\pgfqpoint{2.497137in}{2.413209in}}{\pgfqpoint{2.489323in}{2.405396in}}%
\pgfpathcurveto{\pgfqpoint{2.481509in}{2.397582in}}{\pgfqpoint{2.477119in}{2.386983in}}{\pgfqpoint{2.477119in}{2.375933in}}%
\pgfpathcurveto{\pgfqpoint{2.477119in}{2.364883in}}{\pgfqpoint{2.481509in}{2.354284in}}{\pgfqpoint{2.489323in}{2.346470in}}%
\pgfpathcurveto{\pgfqpoint{2.497137in}{2.338657in}}{\pgfqpoint{2.507736in}{2.334266in}}{\pgfqpoint{2.518786in}{2.334266in}}%
\pgfpathclose%
\pgfusepath{stroke,fill}%
\end{pgfscope}%
\begin{pgfscope}%
\pgfpathrectangle{\pgfqpoint{0.800000in}{0.528000in}}{\pgfqpoint{4.960000in}{3.696000in}}%
\pgfusepath{clip}%
\pgfsetbuttcap%
\pgfsetroundjoin%
\definecolor{currentfill}{rgb}{0.000000,0.000000,0.000000}%
\pgfsetfillcolor{currentfill}%
\pgfsetlinewidth{1.003750pt}%
\definecolor{currentstroke}{rgb}{0.000000,0.000000,0.000000}%
\pgfsetstrokecolor{currentstroke}%
\pgfsetdash{}{0pt}%
\pgfpathmoveto{\pgfqpoint{2.518786in}{2.334266in}}%
\pgfpathcurveto{\pgfqpoint{2.529836in}{2.334266in}}{\pgfqpoint{2.540435in}{2.338657in}}{\pgfqpoint{2.548249in}{2.346470in}}%
\pgfpathcurveto{\pgfqpoint{2.556062in}{2.354284in}}{\pgfqpoint{2.560452in}{2.364883in}}{\pgfqpoint{2.560452in}{2.375933in}}%
\pgfpathcurveto{\pgfqpoint{2.560452in}{2.386983in}}{\pgfqpoint{2.556062in}{2.397582in}}{\pgfqpoint{2.548249in}{2.405396in}}%
\pgfpathcurveto{\pgfqpoint{2.540435in}{2.413209in}}{\pgfqpoint{2.529836in}{2.417600in}}{\pgfqpoint{2.518786in}{2.417600in}}%
\pgfpathcurveto{\pgfqpoint{2.507736in}{2.417600in}}{\pgfqpoint{2.497137in}{2.413209in}}{\pgfqpoint{2.489323in}{2.405396in}}%
\pgfpathcurveto{\pgfqpoint{2.481509in}{2.397582in}}{\pgfqpoint{2.477119in}{2.386983in}}{\pgfqpoint{2.477119in}{2.375933in}}%
\pgfpathcurveto{\pgfqpoint{2.477119in}{2.364883in}}{\pgfqpoint{2.481509in}{2.354284in}}{\pgfqpoint{2.489323in}{2.346470in}}%
\pgfpathcurveto{\pgfqpoint{2.497137in}{2.338657in}}{\pgfqpoint{2.507736in}{2.334266in}}{\pgfqpoint{2.518786in}{2.334266in}}%
\pgfpathclose%
\pgfusepath{stroke,fill}%
\end{pgfscope}%
\begin{pgfscope}%
\pgfpathrectangle{\pgfqpoint{0.800000in}{0.528000in}}{\pgfqpoint{4.960000in}{3.696000in}}%
\pgfusepath{clip}%
\pgfsetbuttcap%
\pgfsetroundjoin%
\definecolor{currentfill}{rgb}{0.000000,0.000000,0.000000}%
\pgfsetfillcolor{currentfill}%
\pgfsetlinewidth{1.003750pt}%
\definecolor{currentstroke}{rgb}{0.000000,0.000000,0.000000}%
\pgfsetstrokecolor{currentstroke}%
\pgfsetdash{}{0pt}%
\pgfpathmoveto{\pgfqpoint{2.518786in}{2.334266in}}%
\pgfpathcurveto{\pgfqpoint{2.529836in}{2.334266in}}{\pgfqpoint{2.540435in}{2.338657in}}{\pgfqpoint{2.548249in}{2.346470in}}%
\pgfpathcurveto{\pgfqpoint{2.556062in}{2.354284in}}{\pgfqpoint{2.560452in}{2.364883in}}{\pgfqpoint{2.560452in}{2.375933in}}%
\pgfpathcurveto{\pgfqpoint{2.560452in}{2.386983in}}{\pgfqpoint{2.556062in}{2.397582in}}{\pgfqpoint{2.548249in}{2.405396in}}%
\pgfpathcurveto{\pgfqpoint{2.540435in}{2.413209in}}{\pgfqpoint{2.529836in}{2.417600in}}{\pgfqpoint{2.518786in}{2.417600in}}%
\pgfpathcurveto{\pgfqpoint{2.507736in}{2.417600in}}{\pgfqpoint{2.497137in}{2.413209in}}{\pgfqpoint{2.489323in}{2.405396in}}%
\pgfpathcurveto{\pgfqpoint{2.481509in}{2.397582in}}{\pgfqpoint{2.477119in}{2.386983in}}{\pgfqpoint{2.477119in}{2.375933in}}%
\pgfpathcurveto{\pgfqpoint{2.477119in}{2.364883in}}{\pgfqpoint{2.481509in}{2.354284in}}{\pgfqpoint{2.489323in}{2.346470in}}%
\pgfpathcurveto{\pgfqpoint{2.497137in}{2.338657in}}{\pgfqpoint{2.507736in}{2.334266in}}{\pgfqpoint{2.518786in}{2.334266in}}%
\pgfpathclose%
\pgfusepath{stroke,fill}%
\end{pgfscope}%
\begin{pgfscope}%
\pgfpathrectangle{\pgfqpoint{0.800000in}{0.528000in}}{\pgfqpoint{4.960000in}{3.696000in}}%
\pgfusepath{clip}%
\pgfsetbuttcap%
\pgfsetroundjoin%
\definecolor{currentfill}{rgb}{0.000000,0.000000,0.000000}%
\pgfsetfillcolor{currentfill}%
\pgfsetlinewidth{1.003750pt}%
\definecolor{currentstroke}{rgb}{0.000000,0.000000,0.000000}%
\pgfsetstrokecolor{currentstroke}%
\pgfsetdash{}{0pt}%
\pgfpathmoveto{\pgfqpoint{2.518786in}{2.334266in}}%
\pgfpathcurveto{\pgfqpoint{2.529836in}{2.334266in}}{\pgfqpoint{2.540435in}{2.338657in}}{\pgfqpoint{2.548249in}{2.346470in}}%
\pgfpathcurveto{\pgfqpoint{2.556062in}{2.354284in}}{\pgfqpoint{2.560452in}{2.364883in}}{\pgfqpoint{2.560452in}{2.375933in}}%
\pgfpathcurveto{\pgfqpoint{2.560452in}{2.386983in}}{\pgfqpoint{2.556062in}{2.397582in}}{\pgfqpoint{2.548249in}{2.405396in}}%
\pgfpathcurveto{\pgfqpoint{2.540435in}{2.413209in}}{\pgfqpoint{2.529836in}{2.417600in}}{\pgfqpoint{2.518786in}{2.417600in}}%
\pgfpathcurveto{\pgfqpoint{2.507736in}{2.417600in}}{\pgfqpoint{2.497137in}{2.413209in}}{\pgfqpoint{2.489323in}{2.405396in}}%
\pgfpathcurveto{\pgfqpoint{2.481509in}{2.397582in}}{\pgfqpoint{2.477119in}{2.386983in}}{\pgfqpoint{2.477119in}{2.375933in}}%
\pgfpathcurveto{\pgfqpoint{2.477119in}{2.364883in}}{\pgfqpoint{2.481509in}{2.354284in}}{\pgfqpoint{2.489323in}{2.346470in}}%
\pgfpathcurveto{\pgfqpoint{2.497137in}{2.338657in}}{\pgfqpoint{2.507736in}{2.334266in}}{\pgfqpoint{2.518786in}{2.334266in}}%
\pgfpathclose%
\pgfusepath{stroke,fill}%
\end{pgfscope}%
\begin{pgfscope}%
\pgfpathrectangle{\pgfqpoint{0.800000in}{0.528000in}}{\pgfqpoint{4.960000in}{3.696000in}}%
\pgfusepath{clip}%
\pgfsetbuttcap%
\pgfsetroundjoin%
\definecolor{currentfill}{rgb}{0.000000,0.000000,0.000000}%
\pgfsetfillcolor{currentfill}%
\pgfsetlinewidth{1.003750pt}%
\definecolor{currentstroke}{rgb}{0.000000,0.000000,0.000000}%
\pgfsetstrokecolor{currentstroke}%
\pgfsetdash{}{0pt}%
\pgfpathmoveto{\pgfqpoint{2.518786in}{2.334266in}}%
\pgfpathcurveto{\pgfqpoint{2.529836in}{2.334266in}}{\pgfqpoint{2.540435in}{2.338657in}}{\pgfqpoint{2.548249in}{2.346470in}}%
\pgfpathcurveto{\pgfqpoint{2.556062in}{2.354284in}}{\pgfqpoint{2.560452in}{2.364883in}}{\pgfqpoint{2.560452in}{2.375933in}}%
\pgfpathcurveto{\pgfqpoint{2.560452in}{2.386983in}}{\pgfqpoint{2.556062in}{2.397582in}}{\pgfqpoint{2.548249in}{2.405396in}}%
\pgfpathcurveto{\pgfqpoint{2.540435in}{2.413209in}}{\pgfqpoint{2.529836in}{2.417600in}}{\pgfqpoint{2.518786in}{2.417600in}}%
\pgfpathcurveto{\pgfqpoint{2.507736in}{2.417600in}}{\pgfqpoint{2.497137in}{2.413209in}}{\pgfqpoint{2.489323in}{2.405396in}}%
\pgfpathcurveto{\pgfqpoint{2.481509in}{2.397582in}}{\pgfqpoint{2.477119in}{2.386983in}}{\pgfqpoint{2.477119in}{2.375933in}}%
\pgfpathcurveto{\pgfqpoint{2.477119in}{2.364883in}}{\pgfqpoint{2.481509in}{2.354284in}}{\pgfqpoint{2.489323in}{2.346470in}}%
\pgfpathcurveto{\pgfqpoint{2.497137in}{2.338657in}}{\pgfqpoint{2.507736in}{2.334266in}}{\pgfqpoint{2.518786in}{2.334266in}}%
\pgfpathclose%
\pgfusepath{stroke,fill}%
\end{pgfscope}%
\begin{pgfscope}%
\pgfpathrectangle{\pgfqpoint{0.800000in}{0.528000in}}{\pgfqpoint{4.960000in}{3.696000in}}%
\pgfusepath{clip}%
\pgfsetbuttcap%
\pgfsetroundjoin%
\definecolor{currentfill}{rgb}{0.000000,0.000000,0.000000}%
\pgfsetfillcolor{currentfill}%
\pgfsetlinewidth{1.003750pt}%
\definecolor{currentstroke}{rgb}{0.000000,0.000000,0.000000}%
\pgfsetstrokecolor{currentstroke}%
\pgfsetdash{}{0pt}%
\pgfpathmoveto{\pgfqpoint{2.518786in}{2.334266in}}%
\pgfpathcurveto{\pgfqpoint{2.529836in}{2.334266in}}{\pgfqpoint{2.540435in}{2.338657in}}{\pgfqpoint{2.548249in}{2.346470in}}%
\pgfpathcurveto{\pgfqpoint{2.556062in}{2.354284in}}{\pgfqpoint{2.560452in}{2.364883in}}{\pgfqpoint{2.560452in}{2.375933in}}%
\pgfpathcurveto{\pgfqpoint{2.560452in}{2.386983in}}{\pgfqpoint{2.556062in}{2.397582in}}{\pgfqpoint{2.548249in}{2.405396in}}%
\pgfpathcurveto{\pgfqpoint{2.540435in}{2.413209in}}{\pgfqpoint{2.529836in}{2.417600in}}{\pgfqpoint{2.518786in}{2.417600in}}%
\pgfpathcurveto{\pgfqpoint{2.507736in}{2.417600in}}{\pgfqpoint{2.497137in}{2.413209in}}{\pgfqpoint{2.489323in}{2.405396in}}%
\pgfpathcurveto{\pgfqpoint{2.481509in}{2.397582in}}{\pgfqpoint{2.477119in}{2.386983in}}{\pgfqpoint{2.477119in}{2.375933in}}%
\pgfpathcurveto{\pgfqpoint{2.477119in}{2.364883in}}{\pgfqpoint{2.481509in}{2.354284in}}{\pgfqpoint{2.489323in}{2.346470in}}%
\pgfpathcurveto{\pgfqpoint{2.497137in}{2.338657in}}{\pgfqpoint{2.507736in}{2.334266in}}{\pgfqpoint{2.518786in}{2.334266in}}%
\pgfpathclose%
\pgfusepath{stroke,fill}%
\end{pgfscope}%
\begin{pgfscope}%
\pgfpathrectangle{\pgfqpoint{0.800000in}{0.528000in}}{\pgfqpoint{4.960000in}{3.696000in}}%
\pgfusepath{clip}%
\pgfsetbuttcap%
\pgfsetroundjoin%
\definecolor{currentfill}{rgb}{0.000000,0.000000,0.000000}%
\pgfsetfillcolor{currentfill}%
\pgfsetlinewidth{1.003750pt}%
\definecolor{currentstroke}{rgb}{0.000000,0.000000,0.000000}%
\pgfsetstrokecolor{currentstroke}%
\pgfsetdash{}{0pt}%
\pgfpathmoveto{\pgfqpoint{2.518786in}{2.334266in}}%
\pgfpathcurveto{\pgfqpoint{2.529836in}{2.334266in}}{\pgfqpoint{2.540435in}{2.338657in}}{\pgfqpoint{2.548249in}{2.346470in}}%
\pgfpathcurveto{\pgfqpoint{2.556062in}{2.354284in}}{\pgfqpoint{2.560452in}{2.364883in}}{\pgfqpoint{2.560452in}{2.375933in}}%
\pgfpathcurveto{\pgfqpoint{2.560452in}{2.386983in}}{\pgfqpoint{2.556062in}{2.397582in}}{\pgfqpoint{2.548249in}{2.405396in}}%
\pgfpathcurveto{\pgfqpoint{2.540435in}{2.413209in}}{\pgfqpoint{2.529836in}{2.417600in}}{\pgfqpoint{2.518786in}{2.417600in}}%
\pgfpathcurveto{\pgfqpoint{2.507736in}{2.417600in}}{\pgfqpoint{2.497137in}{2.413209in}}{\pgfqpoint{2.489323in}{2.405396in}}%
\pgfpathcurveto{\pgfqpoint{2.481509in}{2.397582in}}{\pgfqpoint{2.477119in}{2.386983in}}{\pgfqpoint{2.477119in}{2.375933in}}%
\pgfpathcurveto{\pgfqpoint{2.477119in}{2.364883in}}{\pgfqpoint{2.481509in}{2.354284in}}{\pgfqpoint{2.489323in}{2.346470in}}%
\pgfpathcurveto{\pgfqpoint{2.497137in}{2.338657in}}{\pgfqpoint{2.507736in}{2.334266in}}{\pgfqpoint{2.518786in}{2.334266in}}%
\pgfpathclose%
\pgfusepath{stroke,fill}%
\end{pgfscope}%
\begin{pgfscope}%
\pgfpathrectangle{\pgfqpoint{0.800000in}{0.528000in}}{\pgfqpoint{4.960000in}{3.696000in}}%
\pgfusepath{clip}%
\pgfsetbuttcap%
\pgfsetroundjoin%
\definecolor{currentfill}{rgb}{0.000000,0.000000,0.000000}%
\pgfsetfillcolor{currentfill}%
\pgfsetlinewidth{1.003750pt}%
\definecolor{currentstroke}{rgb}{0.000000,0.000000,0.000000}%
\pgfsetstrokecolor{currentstroke}%
\pgfsetdash{}{0pt}%
\pgfpathmoveto{\pgfqpoint{2.518786in}{2.334266in}}%
\pgfpathcurveto{\pgfqpoint{2.529836in}{2.334266in}}{\pgfqpoint{2.540435in}{2.338657in}}{\pgfqpoint{2.548249in}{2.346470in}}%
\pgfpathcurveto{\pgfqpoint{2.556062in}{2.354284in}}{\pgfqpoint{2.560452in}{2.364883in}}{\pgfqpoint{2.560452in}{2.375933in}}%
\pgfpathcurveto{\pgfqpoint{2.560452in}{2.386983in}}{\pgfqpoint{2.556062in}{2.397582in}}{\pgfqpoint{2.548249in}{2.405396in}}%
\pgfpathcurveto{\pgfqpoint{2.540435in}{2.413209in}}{\pgfqpoint{2.529836in}{2.417600in}}{\pgfqpoint{2.518786in}{2.417600in}}%
\pgfpathcurveto{\pgfqpoint{2.507736in}{2.417600in}}{\pgfqpoint{2.497137in}{2.413209in}}{\pgfqpoint{2.489323in}{2.405396in}}%
\pgfpathcurveto{\pgfqpoint{2.481509in}{2.397582in}}{\pgfqpoint{2.477119in}{2.386983in}}{\pgfqpoint{2.477119in}{2.375933in}}%
\pgfpathcurveto{\pgfqpoint{2.477119in}{2.364883in}}{\pgfqpoint{2.481509in}{2.354284in}}{\pgfqpoint{2.489323in}{2.346470in}}%
\pgfpathcurveto{\pgfqpoint{2.497137in}{2.338657in}}{\pgfqpoint{2.507736in}{2.334266in}}{\pgfqpoint{2.518786in}{2.334266in}}%
\pgfpathclose%
\pgfusepath{stroke,fill}%
\end{pgfscope}%
\begin{pgfscope}%
\pgfpathrectangle{\pgfqpoint{0.800000in}{0.528000in}}{\pgfqpoint{4.960000in}{3.696000in}}%
\pgfusepath{clip}%
\pgfsetbuttcap%
\pgfsetroundjoin%
\definecolor{currentfill}{rgb}{0.000000,0.000000,0.000000}%
\pgfsetfillcolor{currentfill}%
\pgfsetlinewidth{1.003750pt}%
\definecolor{currentstroke}{rgb}{0.000000,0.000000,0.000000}%
\pgfsetstrokecolor{currentstroke}%
\pgfsetdash{}{0pt}%
\pgfpathmoveto{\pgfqpoint{2.518786in}{2.334266in}}%
\pgfpathcurveto{\pgfqpoint{2.529836in}{2.334266in}}{\pgfqpoint{2.540435in}{2.338657in}}{\pgfqpoint{2.548249in}{2.346470in}}%
\pgfpathcurveto{\pgfqpoint{2.556062in}{2.354284in}}{\pgfqpoint{2.560452in}{2.364883in}}{\pgfqpoint{2.560452in}{2.375933in}}%
\pgfpathcurveto{\pgfqpoint{2.560452in}{2.386983in}}{\pgfqpoint{2.556062in}{2.397582in}}{\pgfqpoint{2.548249in}{2.405396in}}%
\pgfpathcurveto{\pgfqpoint{2.540435in}{2.413209in}}{\pgfqpoint{2.529836in}{2.417600in}}{\pgfqpoint{2.518786in}{2.417600in}}%
\pgfpathcurveto{\pgfqpoint{2.507736in}{2.417600in}}{\pgfqpoint{2.497137in}{2.413209in}}{\pgfqpoint{2.489323in}{2.405396in}}%
\pgfpathcurveto{\pgfqpoint{2.481509in}{2.397582in}}{\pgfqpoint{2.477119in}{2.386983in}}{\pgfqpoint{2.477119in}{2.375933in}}%
\pgfpathcurveto{\pgfqpoint{2.477119in}{2.364883in}}{\pgfqpoint{2.481509in}{2.354284in}}{\pgfqpoint{2.489323in}{2.346470in}}%
\pgfpathcurveto{\pgfqpoint{2.497137in}{2.338657in}}{\pgfqpoint{2.507736in}{2.334266in}}{\pgfqpoint{2.518786in}{2.334266in}}%
\pgfpathclose%
\pgfusepath{stroke,fill}%
\end{pgfscope}%
\begin{pgfscope}%
\pgfpathrectangle{\pgfqpoint{0.800000in}{0.528000in}}{\pgfqpoint{4.960000in}{3.696000in}}%
\pgfusepath{clip}%
\pgfsetbuttcap%
\pgfsetroundjoin%
\definecolor{currentfill}{rgb}{0.000000,0.000000,0.000000}%
\pgfsetfillcolor{currentfill}%
\pgfsetlinewidth{1.003750pt}%
\definecolor{currentstroke}{rgb}{0.000000,0.000000,0.000000}%
\pgfsetstrokecolor{currentstroke}%
\pgfsetdash{}{0pt}%
\pgfpathmoveto{\pgfqpoint{2.518786in}{2.334266in}}%
\pgfpathcurveto{\pgfqpoint{2.529836in}{2.334266in}}{\pgfqpoint{2.540435in}{2.338657in}}{\pgfqpoint{2.548249in}{2.346470in}}%
\pgfpathcurveto{\pgfqpoint{2.556062in}{2.354284in}}{\pgfqpoint{2.560452in}{2.364883in}}{\pgfqpoint{2.560452in}{2.375933in}}%
\pgfpathcurveto{\pgfqpoint{2.560452in}{2.386983in}}{\pgfqpoint{2.556062in}{2.397582in}}{\pgfqpoint{2.548249in}{2.405396in}}%
\pgfpathcurveto{\pgfqpoint{2.540435in}{2.413209in}}{\pgfqpoint{2.529836in}{2.417600in}}{\pgfqpoint{2.518786in}{2.417600in}}%
\pgfpathcurveto{\pgfqpoint{2.507736in}{2.417600in}}{\pgfqpoint{2.497137in}{2.413209in}}{\pgfqpoint{2.489323in}{2.405396in}}%
\pgfpathcurveto{\pgfqpoint{2.481509in}{2.397582in}}{\pgfqpoint{2.477119in}{2.386983in}}{\pgfqpoint{2.477119in}{2.375933in}}%
\pgfpathcurveto{\pgfqpoint{2.477119in}{2.364883in}}{\pgfqpoint{2.481509in}{2.354284in}}{\pgfqpoint{2.489323in}{2.346470in}}%
\pgfpathcurveto{\pgfqpoint{2.497137in}{2.338657in}}{\pgfqpoint{2.507736in}{2.334266in}}{\pgfqpoint{2.518786in}{2.334266in}}%
\pgfpathclose%
\pgfusepath{stroke,fill}%
\end{pgfscope}%
\begin{pgfscope}%
\pgfpathrectangle{\pgfqpoint{0.800000in}{0.528000in}}{\pgfqpoint{4.960000in}{3.696000in}}%
\pgfusepath{clip}%
\pgfsetbuttcap%
\pgfsetroundjoin%
\definecolor{currentfill}{rgb}{0.000000,0.000000,0.000000}%
\pgfsetfillcolor{currentfill}%
\pgfsetlinewidth{1.003750pt}%
\definecolor{currentstroke}{rgb}{0.000000,0.000000,0.000000}%
\pgfsetstrokecolor{currentstroke}%
\pgfsetdash{}{0pt}%
\pgfpathmoveto{\pgfqpoint{2.518786in}{2.334266in}}%
\pgfpathcurveto{\pgfqpoint{2.529836in}{2.334266in}}{\pgfqpoint{2.540435in}{2.338657in}}{\pgfqpoint{2.548249in}{2.346470in}}%
\pgfpathcurveto{\pgfqpoint{2.556062in}{2.354284in}}{\pgfqpoint{2.560452in}{2.364883in}}{\pgfqpoint{2.560452in}{2.375933in}}%
\pgfpathcurveto{\pgfqpoint{2.560452in}{2.386983in}}{\pgfqpoint{2.556062in}{2.397582in}}{\pgfqpoint{2.548249in}{2.405396in}}%
\pgfpathcurveto{\pgfqpoint{2.540435in}{2.413209in}}{\pgfqpoint{2.529836in}{2.417600in}}{\pgfqpoint{2.518786in}{2.417600in}}%
\pgfpathcurveto{\pgfqpoint{2.507736in}{2.417600in}}{\pgfqpoint{2.497137in}{2.413209in}}{\pgfqpoint{2.489323in}{2.405396in}}%
\pgfpathcurveto{\pgfqpoint{2.481509in}{2.397582in}}{\pgfqpoint{2.477119in}{2.386983in}}{\pgfqpoint{2.477119in}{2.375933in}}%
\pgfpathcurveto{\pgfqpoint{2.477119in}{2.364883in}}{\pgfqpoint{2.481509in}{2.354284in}}{\pgfqpoint{2.489323in}{2.346470in}}%
\pgfpathcurveto{\pgfqpoint{2.497137in}{2.338657in}}{\pgfqpoint{2.507736in}{2.334266in}}{\pgfqpoint{2.518786in}{2.334266in}}%
\pgfpathclose%
\pgfusepath{stroke,fill}%
\end{pgfscope}%
\begin{pgfscope}%
\pgfpathrectangle{\pgfqpoint{0.800000in}{0.528000in}}{\pgfqpoint{4.960000in}{3.696000in}}%
\pgfusepath{clip}%
\pgfsetbuttcap%
\pgfsetroundjoin%
\definecolor{currentfill}{rgb}{0.000000,0.000000,0.000000}%
\pgfsetfillcolor{currentfill}%
\pgfsetlinewidth{1.003750pt}%
\definecolor{currentstroke}{rgb}{0.000000,0.000000,0.000000}%
\pgfsetstrokecolor{currentstroke}%
\pgfsetdash{}{0pt}%
\pgfpathmoveto{\pgfqpoint{2.518786in}{2.334266in}}%
\pgfpathcurveto{\pgfqpoint{2.529836in}{2.334266in}}{\pgfqpoint{2.540435in}{2.338657in}}{\pgfqpoint{2.548249in}{2.346470in}}%
\pgfpathcurveto{\pgfqpoint{2.556062in}{2.354284in}}{\pgfqpoint{2.560452in}{2.364883in}}{\pgfqpoint{2.560452in}{2.375933in}}%
\pgfpathcurveto{\pgfqpoint{2.560452in}{2.386983in}}{\pgfqpoint{2.556062in}{2.397582in}}{\pgfqpoint{2.548249in}{2.405396in}}%
\pgfpathcurveto{\pgfqpoint{2.540435in}{2.413209in}}{\pgfqpoint{2.529836in}{2.417600in}}{\pgfqpoint{2.518786in}{2.417600in}}%
\pgfpathcurveto{\pgfqpoint{2.507736in}{2.417600in}}{\pgfqpoint{2.497137in}{2.413209in}}{\pgfqpoint{2.489323in}{2.405396in}}%
\pgfpathcurveto{\pgfqpoint{2.481509in}{2.397582in}}{\pgfqpoint{2.477119in}{2.386983in}}{\pgfqpoint{2.477119in}{2.375933in}}%
\pgfpathcurveto{\pgfqpoint{2.477119in}{2.364883in}}{\pgfqpoint{2.481509in}{2.354284in}}{\pgfqpoint{2.489323in}{2.346470in}}%
\pgfpathcurveto{\pgfqpoint{2.497137in}{2.338657in}}{\pgfqpoint{2.507736in}{2.334266in}}{\pgfqpoint{2.518786in}{2.334266in}}%
\pgfpathclose%
\pgfusepath{stroke,fill}%
\end{pgfscope}%
\begin{pgfscope}%
\pgfpathrectangle{\pgfqpoint{0.800000in}{0.528000in}}{\pgfqpoint{4.960000in}{3.696000in}}%
\pgfusepath{clip}%
\pgfsetbuttcap%
\pgfsetroundjoin%
\definecolor{currentfill}{rgb}{0.000000,0.000000,0.000000}%
\pgfsetfillcolor{currentfill}%
\pgfsetlinewidth{1.003750pt}%
\definecolor{currentstroke}{rgb}{0.000000,0.000000,0.000000}%
\pgfsetstrokecolor{currentstroke}%
\pgfsetdash{}{0pt}%
\pgfpathmoveto{\pgfqpoint{2.518786in}{2.334266in}}%
\pgfpathcurveto{\pgfqpoint{2.529836in}{2.334266in}}{\pgfqpoint{2.540435in}{2.338657in}}{\pgfqpoint{2.548249in}{2.346470in}}%
\pgfpathcurveto{\pgfqpoint{2.556062in}{2.354284in}}{\pgfqpoint{2.560452in}{2.364883in}}{\pgfqpoint{2.560452in}{2.375933in}}%
\pgfpathcurveto{\pgfqpoint{2.560452in}{2.386983in}}{\pgfqpoint{2.556062in}{2.397582in}}{\pgfqpoint{2.548249in}{2.405396in}}%
\pgfpathcurveto{\pgfqpoint{2.540435in}{2.413209in}}{\pgfqpoint{2.529836in}{2.417600in}}{\pgfqpoint{2.518786in}{2.417600in}}%
\pgfpathcurveto{\pgfqpoint{2.507736in}{2.417600in}}{\pgfqpoint{2.497137in}{2.413209in}}{\pgfqpoint{2.489323in}{2.405396in}}%
\pgfpathcurveto{\pgfqpoint{2.481509in}{2.397582in}}{\pgfqpoint{2.477119in}{2.386983in}}{\pgfqpoint{2.477119in}{2.375933in}}%
\pgfpathcurveto{\pgfqpoint{2.477119in}{2.364883in}}{\pgfqpoint{2.481509in}{2.354284in}}{\pgfqpoint{2.489323in}{2.346470in}}%
\pgfpathcurveto{\pgfqpoint{2.497137in}{2.338657in}}{\pgfqpoint{2.507736in}{2.334266in}}{\pgfqpoint{2.518786in}{2.334266in}}%
\pgfpathclose%
\pgfusepath{stroke,fill}%
\end{pgfscope}%
\begin{pgfscope}%
\pgfpathrectangle{\pgfqpoint{0.800000in}{0.528000in}}{\pgfqpoint{4.960000in}{3.696000in}}%
\pgfusepath{clip}%
\pgfsetbuttcap%
\pgfsetroundjoin%
\definecolor{currentfill}{rgb}{0.000000,0.000000,0.000000}%
\pgfsetfillcolor{currentfill}%
\pgfsetlinewidth{1.003750pt}%
\definecolor{currentstroke}{rgb}{0.000000,0.000000,0.000000}%
\pgfsetstrokecolor{currentstroke}%
\pgfsetdash{}{0pt}%
\pgfpathmoveto{\pgfqpoint{2.518786in}{2.334266in}}%
\pgfpathcurveto{\pgfqpoint{2.529836in}{2.334266in}}{\pgfqpoint{2.540435in}{2.338657in}}{\pgfqpoint{2.548249in}{2.346470in}}%
\pgfpathcurveto{\pgfqpoint{2.556062in}{2.354284in}}{\pgfqpoint{2.560452in}{2.364883in}}{\pgfqpoint{2.560452in}{2.375933in}}%
\pgfpathcurveto{\pgfqpoint{2.560452in}{2.386983in}}{\pgfqpoint{2.556062in}{2.397582in}}{\pgfqpoint{2.548249in}{2.405396in}}%
\pgfpathcurveto{\pgfqpoint{2.540435in}{2.413209in}}{\pgfqpoint{2.529836in}{2.417600in}}{\pgfqpoint{2.518786in}{2.417600in}}%
\pgfpathcurveto{\pgfqpoint{2.507736in}{2.417600in}}{\pgfqpoint{2.497137in}{2.413209in}}{\pgfqpoint{2.489323in}{2.405396in}}%
\pgfpathcurveto{\pgfqpoint{2.481509in}{2.397582in}}{\pgfqpoint{2.477119in}{2.386983in}}{\pgfqpoint{2.477119in}{2.375933in}}%
\pgfpathcurveto{\pgfqpoint{2.477119in}{2.364883in}}{\pgfqpoint{2.481509in}{2.354284in}}{\pgfqpoint{2.489323in}{2.346470in}}%
\pgfpathcurveto{\pgfqpoint{2.497137in}{2.338657in}}{\pgfqpoint{2.507736in}{2.334266in}}{\pgfqpoint{2.518786in}{2.334266in}}%
\pgfpathclose%
\pgfusepath{stroke,fill}%
\end{pgfscope}%
\begin{pgfscope}%
\pgfpathrectangle{\pgfqpoint{0.800000in}{0.528000in}}{\pgfqpoint{4.960000in}{3.696000in}}%
\pgfusepath{clip}%
\pgfsetbuttcap%
\pgfsetroundjoin%
\definecolor{currentfill}{rgb}{0.000000,0.000000,0.000000}%
\pgfsetfillcolor{currentfill}%
\pgfsetlinewidth{1.003750pt}%
\definecolor{currentstroke}{rgb}{0.000000,0.000000,0.000000}%
\pgfsetstrokecolor{currentstroke}%
\pgfsetdash{}{0pt}%
\pgfpathmoveto{\pgfqpoint{2.518786in}{2.334266in}}%
\pgfpathcurveto{\pgfqpoint{2.529836in}{2.334266in}}{\pgfqpoint{2.540435in}{2.338657in}}{\pgfqpoint{2.548249in}{2.346470in}}%
\pgfpathcurveto{\pgfqpoint{2.556062in}{2.354284in}}{\pgfqpoint{2.560452in}{2.364883in}}{\pgfqpoint{2.560452in}{2.375933in}}%
\pgfpathcurveto{\pgfqpoint{2.560452in}{2.386983in}}{\pgfqpoint{2.556062in}{2.397582in}}{\pgfqpoint{2.548249in}{2.405396in}}%
\pgfpathcurveto{\pgfqpoint{2.540435in}{2.413209in}}{\pgfqpoint{2.529836in}{2.417600in}}{\pgfqpoint{2.518786in}{2.417600in}}%
\pgfpathcurveto{\pgfqpoint{2.507736in}{2.417600in}}{\pgfqpoint{2.497137in}{2.413209in}}{\pgfqpoint{2.489323in}{2.405396in}}%
\pgfpathcurveto{\pgfqpoint{2.481509in}{2.397582in}}{\pgfqpoint{2.477119in}{2.386983in}}{\pgfqpoint{2.477119in}{2.375933in}}%
\pgfpathcurveto{\pgfqpoint{2.477119in}{2.364883in}}{\pgfqpoint{2.481509in}{2.354284in}}{\pgfqpoint{2.489323in}{2.346470in}}%
\pgfpathcurveto{\pgfqpoint{2.497137in}{2.338657in}}{\pgfqpoint{2.507736in}{2.334266in}}{\pgfqpoint{2.518786in}{2.334266in}}%
\pgfpathclose%
\pgfusepath{stroke,fill}%
\end{pgfscope}%
\begin{pgfscope}%
\pgfpathrectangle{\pgfqpoint{0.800000in}{0.528000in}}{\pgfqpoint{4.960000in}{3.696000in}}%
\pgfusepath{clip}%
\pgfsetbuttcap%
\pgfsetroundjoin%
\definecolor{currentfill}{rgb}{0.000000,0.000000,0.000000}%
\pgfsetfillcolor{currentfill}%
\pgfsetlinewidth{1.003750pt}%
\definecolor{currentstroke}{rgb}{0.000000,0.000000,0.000000}%
\pgfsetstrokecolor{currentstroke}%
\pgfsetdash{}{0pt}%
\pgfpathmoveto{\pgfqpoint{2.518786in}{2.334266in}}%
\pgfpathcurveto{\pgfqpoint{2.529836in}{2.334266in}}{\pgfqpoint{2.540435in}{2.338657in}}{\pgfqpoint{2.548249in}{2.346470in}}%
\pgfpathcurveto{\pgfqpoint{2.556062in}{2.354284in}}{\pgfqpoint{2.560452in}{2.364883in}}{\pgfqpoint{2.560452in}{2.375933in}}%
\pgfpathcurveto{\pgfqpoint{2.560452in}{2.386983in}}{\pgfqpoint{2.556062in}{2.397582in}}{\pgfqpoint{2.548249in}{2.405396in}}%
\pgfpathcurveto{\pgfqpoint{2.540435in}{2.413209in}}{\pgfqpoint{2.529836in}{2.417600in}}{\pgfqpoint{2.518786in}{2.417600in}}%
\pgfpathcurveto{\pgfqpoint{2.507736in}{2.417600in}}{\pgfqpoint{2.497137in}{2.413209in}}{\pgfqpoint{2.489323in}{2.405396in}}%
\pgfpathcurveto{\pgfqpoint{2.481509in}{2.397582in}}{\pgfqpoint{2.477119in}{2.386983in}}{\pgfqpoint{2.477119in}{2.375933in}}%
\pgfpathcurveto{\pgfqpoint{2.477119in}{2.364883in}}{\pgfqpoint{2.481509in}{2.354284in}}{\pgfqpoint{2.489323in}{2.346470in}}%
\pgfpathcurveto{\pgfqpoint{2.497137in}{2.338657in}}{\pgfqpoint{2.507736in}{2.334266in}}{\pgfqpoint{2.518786in}{2.334266in}}%
\pgfpathclose%
\pgfusepath{stroke,fill}%
\end{pgfscope}%
\begin{pgfscope}%
\pgfpathrectangle{\pgfqpoint{0.800000in}{0.528000in}}{\pgfqpoint{4.960000in}{3.696000in}}%
\pgfusepath{clip}%
\pgfsetbuttcap%
\pgfsetroundjoin%
\definecolor{currentfill}{rgb}{0.000000,0.000000,0.000000}%
\pgfsetfillcolor{currentfill}%
\pgfsetlinewidth{1.003750pt}%
\definecolor{currentstroke}{rgb}{0.000000,0.000000,0.000000}%
\pgfsetstrokecolor{currentstroke}%
\pgfsetdash{}{0pt}%
\pgfpathmoveto{\pgfqpoint{2.518786in}{2.334266in}}%
\pgfpathcurveto{\pgfqpoint{2.529836in}{2.334266in}}{\pgfqpoint{2.540435in}{2.338657in}}{\pgfqpoint{2.548249in}{2.346470in}}%
\pgfpathcurveto{\pgfqpoint{2.556062in}{2.354284in}}{\pgfqpoint{2.560452in}{2.364883in}}{\pgfqpoint{2.560452in}{2.375933in}}%
\pgfpathcurveto{\pgfqpoint{2.560452in}{2.386983in}}{\pgfqpoint{2.556062in}{2.397582in}}{\pgfqpoint{2.548249in}{2.405396in}}%
\pgfpathcurveto{\pgfqpoint{2.540435in}{2.413209in}}{\pgfqpoint{2.529836in}{2.417600in}}{\pgfqpoint{2.518786in}{2.417600in}}%
\pgfpathcurveto{\pgfqpoint{2.507736in}{2.417600in}}{\pgfqpoint{2.497137in}{2.413209in}}{\pgfqpoint{2.489323in}{2.405396in}}%
\pgfpathcurveto{\pgfqpoint{2.481509in}{2.397582in}}{\pgfqpoint{2.477119in}{2.386983in}}{\pgfqpoint{2.477119in}{2.375933in}}%
\pgfpathcurveto{\pgfqpoint{2.477119in}{2.364883in}}{\pgfqpoint{2.481509in}{2.354284in}}{\pgfqpoint{2.489323in}{2.346470in}}%
\pgfpathcurveto{\pgfqpoint{2.497137in}{2.338657in}}{\pgfqpoint{2.507736in}{2.334266in}}{\pgfqpoint{2.518786in}{2.334266in}}%
\pgfpathclose%
\pgfusepath{stroke,fill}%
\end{pgfscope}%
\begin{pgfscope}%
\pgfpathrectangle{\pgfqpoint{0.800000in}{0.528000in}}{\pgfqpoint{4.960000in}{3.696000in}}%
\pgfusepath{clip}%
\pgfsetbuttcap%
\pgfsetroundjoin%
\definecolor{currentfill}{rgb}{0.000000,0.000000,0.000000}%
\pgfsetfillcolor{currentfill}%
\pgfsetlinewidth{1.003750pt}%
\definecolor{currentstroke}{rgb}{0.000000,0.000000,0.000000}%
\pgfsetstrokecolor{currentstroke}%
\pgfsetdash{}{0pt}%
\pgfpathmoveto{\pgfqpoint{2.518786in}{2.334266in}}%
\pgfpathcurveto{\pgfqpoint{2.529836in}{2.334266in}}{\pgfqpoint{2.540435in}{2.338657in}}{\pgfqpoint{2.548249in}{2.346470in}}%
\pgfpathcurveto{\pgfqpoint{2.556062in}{2.354284in}}{\pgfqpoint{2.560452in}{2.364883in}}{\pgfqpoint{2.560452in}{2.375933in}}%
\pgfpathcurveto{\pgfqpoint{2.560452in}{2.386983in}}{\pgfqpoint{2.556062in}{2.397582in}}{\pgfqpoint{2.548249in}{2.405396in}}%
\pgfpathcurveto{\pgfqpoint{2.540435in}{2.413209in}}{\pgfqpoint{2.529836in}{2.417600in}}{\pgfqpoint{2.518786in}{2.417600in}}%
\pgfpathcurveto{\pgfqpoint{2.507736in}{2.417600in}}{\pgfqpoint{2.497137in}{2.413209in}}{\pgfqpoint{2.489323in}{2.405396in}}%
\pgfpathcurveto{\pgfqpoint{2.481509in}{2.397582in}}{\pgfqpoint{2.477119in}{2.386983in}}{\pgfqpoint{2.477119in}{2.375933in}}%
\pgfpathcurveto{\pgfqpoint{2.477119in}{2.364883in}}{\pgfqpoint{2.481509in}{2.354284in}}{\pgfqpoint{2.489323in}{2.346470in}}%
\pgfpathcurveto{\pgfqpoint{2.497137in}{2.338657in}}{\pgfqpoint{2.507736in}{2.334266in}}{\pgfqpoint{2.518786in}{2.334266in}}%
\pgfpathclose%
\pgfusepath{stroke,fill}%
\end{pgfscope}%
\begin{pgfscope}%
\pgfpathrectangle{\pgfqpoint{0.800000in}{0.528000in}}{\pgfqpoint{4.960000in}{3.696000in}}%
\pgfusepath{clip}%
\pgfsetbuttcap%
\pgfsetroundjoin%
\definecolor{currentfill}{rgb}{0.000000,0.000000,0.000000}%
\pgfsetfillcolor{currentfill}%
\pgfsetlinewidth{1.003750pt}%
\definecolor{currentstroke}{rgb}{0.000000,0.000000,0.000000}%
\pgfsetstrokecolor{currentstroke}%
\pgfsetdash{}{0pt}%
\pgfpathmoveto{\pgfqpoint{2.518786in}{2.334266in}}%
\pgfpathcurveto{\pgfqpoint{2.529836in}{2.334266in}}{\pgfqpoint{2.540435in}{2.338657in}}{\pgfqpoint{2.548249in}{2.346470in}}%
\pgfpathcurveto{\pgfqpoint{2.556062in}{2.354284in}}{\pgfqpoint{2.560452in}{2.364883in}}{\pgfqpoint{2.560452in}{2.375933in}}%
\pgfpathcurveto{\pgfqpoint{2.560452in}{2.386983in}}{\pgfqpoint{2.556062in}{2.397582in}}{\pgfqpoint{2.548249in}{2.405396in}}%
\pgfpathcurveto{\pgfqpoint{2.540435in}{2.413209in}}{\pgfqpoint{2.529836in}{2.417600in}}{\pgfqpoint{2.518786in}{2.417600in}}%
\pgfpathcurveto{\pgfqpoint{2.507736in}{2.417600in}}{\pgfqpoint{2.497137in}{2.413209in}}{\pgfqpoint{2.489323in}{2.405396in}}%
\pgfpathcurveto{\pgfqpoint{2.481509in}{2.397582in}}{\pgfqpoint{2.477119in}{2.386983in}}{\pgfqpoint{2.477119in}{2.375933in}}%
\pgfpathcurveto{\pgfqpoint{2.477119in}{2.364883in}}{\pgfqpoint{2.481509in}{2.354284in}}{\pgfqpoint{2.489323in}{2.346470in}}%
\pgfpathcurveto{\pgfqpoint{2.497137in}{2.338657in}}{\pgfqpoint{2.507736in}{2.334266in}}{\pgfqpoint{2.518786in}{2.334266in}}%
\pgfpathclose%
\pgfusepath{stroke,fill}%
\end{pgfscope}%
\begin{pgfscope}%
\pgfpathrectangle{\pgfqpoint{0.800000in}{0.528000in}}{\pgfqpoint{4.960000in}{3.696000in}}%
\pgfusepath{clip}%
\pgfsetbuttcap%
\pgfsetroundjoin%
\definecolor{currentfill}{rgb}{0.000000,0.000000,0.000000}%
\pgfsetfillcolor{currentfill}%
\pgfsetlinewidth{1.003750pt}%
\definecolor{currentstroke}{rgb}{0.000000,0.000000,0.000000}%
\pgfsetstrokecolor{currentstroke}%
\pgfsetdash{}{0pt}%
\pgfpathmoveto{\pgfqpoint{2.518786in}{2.334266in}}%
\pgfpathcurveto{\pgfqpoint{2.529836in}{2.334266in}}{\pgfqpoint{2.540435in}{2.338657in}}{\pgfqpoint{2.548249in}{2.346470in}}%
\pgfpathcurveto{\pgfqpoint{2.556062in}{2.354284in}}{\pgfqpoint{2.560452in}{2.364883in}}{\pgfqpoint{2.560452in}{2.375933in}}%
\pgfpathcurveto{\pgfqpoint{2.560452in}{2.386983in}}{\pgfqpoint{2.556062in}{2.397582in}}{\pgfqpoint{2.548249in}{2.405396in}}%
\pgfpathcurveto{\pgfqpoint{2.540435in}{2.413209in}}{\pgfqpoint{2.529836in}{2.417600in}}{\pgfqpoint{2.518786in}{2.417600in}}%
\pgfpathcurveto{\pgfqpoint{2.507736in}{2.417600in}}{\pgfqpoint{2.497137in}{2.413209in}}{\pgfqpoint{2.489323in}{2.405396in}}%
\pgfpathcurveto{\pgfqpoint{2.481509in}{2.397582in}}{\pgfqpoint{2.477119in}{2.386983in}}{\pgfqpoint{2.477119in}{2.375933in}}%
\pgfpathcurveto{\pgfqpoint{2.477119in}{2.364883in}}{\pgfqpoint{2.481509in}{2.354284in}}{\pgfqpoint{2.489323in}{2.346470in}}%
\pgfpathcurveto{\pgfqpoint{2.497137in}{2.338657in}}{\pgfqpoint{2.507736in}{2.334266in}}{\pgfqpoint{2.518786in}{2.334266in}}%
\pgfpathclose%
\pgfusepath{stroke,fill}%
\end{pgfscope}%
\begin{pgfscope}%
\pgfpathrectangle{\pgfqpoint{0.800000in}{0.528000in}}{\pgfqpoint{4.960000in}{3.696000in}}%
\pgfusepath{clip}%
\pgfsetbuttcap%
\pgfsetroundjoin%
\definecolor{currentfill}{rgb}{0.000000,0.000000,0.000000}%
\pgfsetfillcolor{currentfill}%
\pgfsetlinewidth{1.003750pt}%
\definecolor{currentstroke}{rgb}{0.000000,0.000000,0.000000}%
\pgfsetstrokecolor{currentstroke}%
\pgfsetdash{}{0pt}%
\pgfpathmoveto{\pgfqpoint{2.518786in}{2.334266in}}%
\pgfpathcurveto{\pgfqpoint{2.529836in}{2.334266in}}{\pgfqpoint{2.540435in}{2.338657in}}{\pgfqpoint{2.548249in}{2.346470in}}%
\pgfpathcurveto{\pgfqpoint{2.556062in}{2.354284in}}{\pgfqpoint{2.560452in}{2.364883in}}{\pgfqpoint{2.560452in}{2.375933in}}%
\pgfpathcurveto{\pgfqpoint{2.560452in}{2.386983in}}{\pgfqpoint{2.556062in}{2.397582in}}{\pgfqpoint{2.548249in}{2.405396in}}%
\pgfpathcurveto{\pgfqpoint{2.540435in}{2.413209in}}{\pgfqpoint{2.529836in}{2.417600in}}{\pgfqpoint{2.518786in}{2.417600in}}%
\pgfpathcurveto{\pgfqpoint{2.507736in}{2.417600in}}{\pgfqpoint{2.497137in}{2.413209in}}{\pgfqpoint{2.489323in}{2.405396in}}%
\pgfpathcurveto{\pgfqpoint{2.481509in}{2.397582in}}{\pgfqpoint{2.477119in}{2.386983in}}{\pgfqpoint{2.477119in}{2.375933in}}%
\pgfpathcurveto{\pgfqpoint{2.477119in}{2.364883in}}{\pgfqpoint{2.481509in}{2.354284in}}{\pgfqpoint{2.489323in}{2.346470in}}%
\pgfpathcurveto{\pgfqpoint{2.497137in}{2.338657in}}{\pgfqpoint{2.507736in}{2.334266in}}{\pgfqpoint{2.518786in}{2.334266in}}%
\pgfpathclose%
\pgfusepath{stroke,fill}%
\end{pgfscope}%
\begin{pgfscope}%
\pgfpathrectangle{\pgfqpoint{0.800000in}{0.528000in}}{\pgfqpoint{4.960000in}{3.696000in}}%
\pgfusepath{clip}%
\pgfsetbuttcap%
\pgfsetroundjoin%
\definecolor{currentfill}{rgb}{0.000000,0.000000,0.000000}%
\pgfsetfillcolor{currentfill}%
\pgfsetlinewidth{1.003750pt}%
\definecolor{currentstroke}{rgb}{0.000000,0.000000,0.000000}%
\pgfsetstrokecolor{currentstroke}%
\pgfsetdash{}{0pt}%
\pgfpathmoveto{\pgfqpoint{2.518786in}{2.334266in}}%
\pgfpathcurveto{\pgfqpoint{2.529836in}{2.334266in}}{\pgfqpoint{2.540435in}{2.338657in}}{\pgfqpoint{2.548249in}{2.346470in}}%
\pgfpathcurveto{\pgfqpoint{2.556062in}{2.354284in}}{\pgfqpoint{2.560452in}{2.364883in}}{\pgfqpoint{2.560452in}{2.375933in}}%
\pgfpathcurveto{\pgfqpoint{2.560452in}{2.386983in}}{\pgfqpoint{2.556062in}{2.397582in}}{\pgfqpoint{2.548249in}{2.405396in}}%
\pgfpathcurveto{\pgfqpoint{2.540435in}{2.413209in}}{\pgfqpoint{2.529836in}{2.417600in}}{\pgfqpoint{2.518786in}{2.417600in}}%
\pgfpathcurveto{\pgfqpoint{2.507736in}{2.417600in}}{\pgfqpoint{2.497137in}{2.413209in}}{\pgfqpoint{2.489323in}{2.405396in}}%
\pgfpathcurveto{\pgfqpoint{2.481509in}{2.397582in}}{\pgfqpoint{2.477119in}{2.386983in}}{\pgfqpoint{2.477119in}{2.375933in}}%
\pgfpathcurveto{\pgfqpoint{2.477119in}{2.364883in}}{\pgfqpoint{2.481509in}{2.354284in}}{\pgfqpoint{2.489323in}{2.346470in}}%
\pgfpathcurveto{\pgfqpoint{2.497137in}{2.338657in}}{\pgfqpoint{2.507736in}{2.334266in}}{\pgfqpoint{2.518786in}{2.334266in}}%
\pgfpathclose%
\pgfusepath{stroke,fill}%
\end{pgfscope}%
\begin{pgfscope}%
\pgfpathrectangle{\pgfqpoint{0.800000in}{0.528000in}}{\pgfqpoint{4.960000in}{3.696000in}}%
\pgfusepath{clip}%
\pgfsetbuttcap%
\pgfsetroundjoin%
\definecolor{currentfill}{rgb}{0.000000,0.000000,0.000000}%
\pgfsetfillcolor{currentfill}%
\pgfsetlinewidth{1.003750pt}%
\definecolor{currentstroke}{rgb}{0.000000,0.000000,0.000000}%
\pgfsetstrokecolor{currentstroke}%
\pgfsetdash{}{0pt}%
\pgfpathmoveto{\pgfqpoint{2.518786in}{2.334266in}}%
\pgfpathcurveto{\pgfqpoint{2.529836in}{2.334266in}}{\pgfqpoint{2.540435in}{2.338657in}}{\pgfqpoint{2.548249in}{2.346470in}}%
\pgfpathcurveto{\pgfqpoint{2.556062in}{2.354284in}}{\pgfqpoint{2.560452in}{2.364883in}}{\pgfqpoint{2.560452in}{2.375933in}}%
\pgfpathcurveto{\pgfqpoint{2.560452in}{2.386983in}}{\pgfqpoint{2.556062in}{2.397582in}}{\pgfqpoint{2.548249in}{2.405396in}}%
\pgfpathcurveto{\pgfqpoint{2.540435in}{2.413209in}}{\pgfqpoint{2.529836in}{2.417600in}}{\pgfqpoint{2.518786in}{2.417600in}}%
\pgfpathcurveto{\pgfqpoint{2.507736in}{2.417600in}}{\pgfqpoint{2.497137in}{2.413209in}}{\pgfqpoint{2.489323in}{2.405396in}}%
\pgfpathcurveto{\pgfqpoint{2.481509in}{2.397582in}}{\pgfqpoint{2.477119in}{2.386983in}}{\pgfqpoint{2.477119in}{2.375933in}}%
\pgfpathcurveto{\pgfqpoint{2.477119in}{2.364883in}}{\pgfqpoint{2.481509in}{2.354284in}}{\pgfqpoint{2.489323in}{2.346470in}}%
\pgfpathcurveto{\pgfqpoint{2.497137in}{2.338657in}}{\pgfqpoint{2.507736in}{2.334266in}}{\pgfqpoint{2.518786in}{2.334266in}}%
\pgfpathclose%
\pgfusepath{stroke,fill}%
\end{pgfscope}%
\begin{pgfscope}%
\pgfpathrectangle{\pgfqpoint{0.800000in}{0.528000in}}{\pgfqpoint{4.960000in}{3.696000in}}%
\pgfusepath{clip}%
\pgfsetbuttcap%
\pgfsetroundjoin%
\definecolor{currentfill}{rgb}{0.000000,0.000000,0.000000}%
\pgfsetfillcolor{currentfill}%
\pgfsetlinewidth{1.003750pt}%
\definecolor{currentstroke}{rgb}{0.000000,0.000000,0.000000}%
\pgfsetstrokecolor{currentstroke}%
\pgfsetdash{}{0pt}%
\pgfpathmoveto{\pgfqpoint{2.518786in}{2.334266in}}%
\pgfpathcurveto{\pgfqpoint{2.529836in}{2.334266in}}{\pgfqpoint{2.540435in}{2.338657in}}{\pgfqpoint{2.548249in}{2.346470in}}%
\pgfpathcurveto{\pgfqpoint{2.556062in}{2.354284in}}{\pgfqpoint{2.560452in}{2.364883in}}{\pgfqpoint{2.560452in}{2.375933in}}%
\pgfpathcurveto{\pgfqpoint{2.560452in}{2.386983in}}{\pgfqpoint{2.556062in}{2.397582in}}{\pgfqpoint{2.548249in}{2.405396in}}%
\pgfpathcurveto{\pgfqpoint{2.540435in}{2.413209in}}{\pgfqpoint{2.529836in}{2.417600in}}{\pgfqpoint{2.518786in}{2.417600in}}%
\pgfpathcurveto{\pgfqpoint{2.507736in}{2.417600in}}{\pgfqpoint{2.497137in}{2.413209in}}{\pgfqpoint{2.489323in}{2.405396in}}%
\pgfpathcurveto{\pgfqpoint{2.481509in}{2.397582in}}{\pgfqpoint{2.477119in}{2.386983in}}{\pgfqpoint{2.477119in}{2.375933in}}%
\pgfpathcurveto{\pgfqpoint{2.477119in}{2.364883in}}{\pgfqpoint{2.481509in}{2.354284in}}{\pgfqpoint{2.489323in}{2.346470in}}%
\pgfpathcurveto{\pgfqpoint{2.497137in}{2.338657in}}{\pgfqpoint{2.507736in}{2.334266in}}{\pgfqpoint{2.518786in}{2.334266in}}%
\pgfpathclose%
\pgfusepath{stroke,fill}%
\end{pgfscope}%
\begin{pgfscope}%
\pgfpathrectangle{\pgfqpoint{0.800000in}{0.528000in}}{\pgfqpoint{4.960000in}{3.696000in}}%
\pgfusepath{clip}%
\pgfsetbuttcap%
\pgfsetroundjoin%
\definecolor{currentfill}{rgb}{0.000000,0.000000,0.000000}%
\pgfsetfillcolor{currentfill}%
\pgfsetlinewidth{1.003750pt}%
\definecolor{currentstroke}{rgb}{0.000000,0.000000,0.000000}%
\pgfsetstrokecolor{currentstroke}%
\pgfsetdash{}{0pt}%
\pgfpathmoveto{\pgfqpoint{2.518786in}{2.334266in}}%
\pgfpathcurveto{\pgfqpoint{2.529836in}{2.334266in}}{\pgfqpoint{2.540435in}{2.338657in}}{\pgfqpoint{2.548249in}{2.346470in}}%
\pgfpathcurveto{\pgfqpoint{2.556062in}{2.354284in}}{\pgfqpoint{2.560452in}{2.364883in}}{\pgfqpoint{2.560452in}{2.375933in}}%
\pgfpathcurveto{\pgfqpoint{2.560452in}{2.386983in}}{\pgfqpoint{2.556062in}{2.397582in}}{\pgfqpoint{2.548249in}{2.405396in}}%
\pgfpathcurveto{\pgfqpoint{2.540435in}{2.413209in}}{\pgfqpoint{2.529836in}{2.417600in}}{\pgfqpoint{2.518786in}{2.417600in}}%
\pgfpathcurveto{\pgfqpoint{2.507736in}{2.417600in}}{\pgfqpoint{2.497137in}{2.413209in}}{\pgfqpoint{2.489323in}{2.405396in}}%
\pgfpathcurveto{\pgfqpoint{2.481509in}{2.397582in}}{\pgfqpoint{2.477119in}{2.386983in}}{\pgfqpoint{2.477119in}{2.375933in}}%
\pgfpathcurveto{\pgfqpoint{2.477119in}{2.364883in}}{\pgfqpoint{2.481509in}{2.354284in}}{\pgfqpoint{2.489323in}{2.346470in}}%
\pgfpathcurveto{\pgfqpoint{2.497137in}{2.338657in}}{\pgfqpoint{2.507736in}{2.334266in}}{\pgfqpoint{2.518786in}{2.334266in}}%
\pgfpathclose%
\pgfusepath{stroke,fill}%
\end{pgfscope}%
\begin{pgfscope}%
\pgfpathrectangle{\pgfqpoint{0.800000in}{0.528000in}}{\pgfqpoint{4.960000in}{3.696000in}}%
\pgfusepath{clip}%
\pgfsetbuttcap%
\pgfsetroundjoin%
\definecolor{currentfill}{rgb}{0.000000,0.000000,0.000000}%
\pgfsetfillcolor{currentfill}%
\pgfsetlinewidth{1.003750pt}%
\definecolor{currentstroke}{rgb}{0.000000,0.000000,0.000000}%
\pgfsetstrokecolor{currentstroke}%
\pgfsetdash{}{0pt}%
\pgfpathmoveto{\pgfqpoint{2.518786in}{2.334266in}}%
\pgfpathcurveto{\pgfqpoint{2.529836in}{2.334266in}}{\pgfqpoint{2.540435in}{2.338657in}}{\pgfqpoint{2.548249in}{2.346470in}}%
\pgfpathcurveto{\pgfqpoint{2.556062in}{2.354284in}}{\pgfqpoint{2.560452in}{2.364883in}}{\pgfqpoint{2.560452in}{2.375933in}}%
\pgfpathcurveto{\pgfqpoint{2.560452in}{2.386983in}}{\pgfqpoint{2.556062in}{2.397582in}}{\pgfqpoint{2.548249in}{2.405396in}}%
\pgfpathcurveto{\pgfqpoint{2.540435in}{2.413209in}}{\pgfqpoint{2.529836in}{2.417600in}}{\pgfqpoint{2.518786in}{2.417600in}}%
\pgfpathcurveto{\pgfqpoint{2.507736in}{2.417600in}}{\pgfqpoint{2.497137in}{2.413209in}}{\pgfqpoint{2.489323in}{2.405396in}}%
\pgfpathcurveto{\pgfqpoint{2.481509in}{2.397582in}}{\pgfqpoint{2.477119in}{2.386983in}}{\pgfqpoint{2.477119in}{2.375933in}}%
\pgfpathcurveto{\pgfqpoint{2.477119in}{2.364883in}}{\pgfqpoint{2.481509in}{2.354284in}}{\pgfqpoint{2.489323in}{2.346470in}}%
\pgfpathcurveto{\pgfqpoint{2.497137in}{2.338657in}}{\pgfqpoint{2.507736in}{2.334266in}}{\pgfqpoint{2.518786in}{2.334266in}}%
\pgfpathclose%
\pgfusepath{stroke,fill}%
\end{pgfscope}%
\begin{pgfscope}%
\pgfpathrectangle{\pgfqpoint{0.800000in}{0.528000in}}{\pgfqpoint{4.960000in}{3.696000in}}%
\pgfusepath{clip}%
\pgfsetbuttcap%
\pgfsetroundjoin%
\definecolor{currentfill}{rgb}{0.000000,0.000000,0.000000}%
\pgfsetfillcolor{currentfill}%
\pgfsetlinewidth{1.003750pt}%
\definecolor{currentstroke}{rgb}{0.000000,0.000000,0.000000}%
\pgfsetstrokecolor{currentstroke}%
\pgfsetdash{}{0pt}%
\pgfpathmoveto{\pgfqpoint{2.518786in}{2.334266in}}%
\pgfpathcurveto{\pgfqpoint{2.529836in}{2.334266in}}{\pgfqpoint{2.540435in}{2.338657in}}{\pgfqpoint{2.548249in}{2.346470in}}%
\pgfpathcurveto{\pgfqpoint{2.556062in}{2.354284in}}{\pgfqpoint{2.560452in}{2.364883in}}{\pgfqpoint{2.560452in}{2.375933in}}%
\pgfpathcurveto{\pgfqpoint{2.560452in}{2.386983in}}{\pgfqpoint{2.556062in}{2.397582in}}{\pgfqpoint{2.548249in}{2.405396in}}%
\pgfpathcurveto{\pgfqpoint{2.540435in}{2.413209in}}{\pgfqpoint{2.529836in}{2.417600in}}{\pgfqpoint{2.518786in}{2.417600in}}%
\pgfpathcurveto{\pgfqpoint{2.507736in}{2.417600in}}{\pgfqpoint{2.497137in}{2.413209in}}{\pgfqpoint{2.489323in}{2.405396in}}%
\pgfpathcurveto{\pgfqpoint{2.481509in}{2.397582in}}{\pgfqpoint{2.477119in}{2.386983in}}{\pgfqpoint{2.477119in}{2.375933in}}%
\pgfpathcurveto{\pgfqpoint{2.477119in}{2.364883in}}{\pgfqpoint{2.481509in}{2.354284in}}{\pgfqpoint{2.489323in}{2.346470in}}%
\pgfpathcurveto{\pgfqpoint{2.497137in}{2.338657in}}{\pgfqpoint{2.507736in}{2.334266in}}{\pgfqpoint{2.518786in}{2.334266in}}%
\pgfpathclose%
\pgfusepath{stroke,fill}%
\end{pgfscope}%
\begin{pgfscope}%
\pgfpathrectangle{\pgfqpoint{0.800000in}{0.528000in}}{\pgfqpoint{4.960000in}{3.696000in}}%
\pgfusepath{clip}%
\pgfsetbuttcap%
\pgfsetroundjoin%
\definecolor{currentfill}{rgb}{0.000000,0.000000,0.000000}%
\pgfsetfillcolor{currentfill}%
\pgfsetlinewidth{1.003750pt}%
\definecolor{currentstroke}{rgb}{0.000000,0.000000,0.000000}%
\pgfsetstrokecolor{currentstroke}%
\pgfsetdash{}{0pt}%
\pgfpathmoveto{\pgfqpoint{2.518786in}{2.334266in}}%
\pgfpathcurveto{\pgfqpoint{2.529836in}{2.334266in}}{\pgfqpoint{2.540435in}{2.338657in}}{\pgfqpoint{2.548249in}{2.346470in}}%
\pgfpathcurveto{\pgfqpoint{2.556062in}{2.354284in}}{\pgfqpoint{2.560452in}{2.364883in}}{\pgfqpoint{2.560452in}{2.375933in}}%
\pgfpathcurveto{\pgfqpoint{2.560452in}{2.386983in}}{\pgfqpoint{2.556062in}{2.397582in}}{\pgfqpoint{2.548249in}{2.405396in}}%
\pgfpathcurveto{\pgfqpoint{2.540435in}{2.413209in}}{\pgfqpoint{2.529836in}{2.417600in}}{\pgfqpoint{2.518786in}{2.417600in}}%
\pgfpathcurveto{\pgfqpoint{2.507736in}{2.417600in}}{\pgfqpoint{2.497137in}{2.413209in}}{\pgfqpoint{2.489323in}{2.405396in}}%
\pgfpathcurveto{\pgfqpoint{2.481509in}{2.397582in}}{\pgfqpoint{2.477119in}{2.386983in}}{\pgfqpoint{2.477119in}{2.375933in}}%
\pgfpathcurveto{\pgfqpoint{2.477119in}{2.364883in}}{\pgfqpoint{2.481509in}{2.354284in}}{\pgfqpoint{2.489323in}{2.346470in}}%
\pgfpathcurveto{\pgfqpoint{2.497137in}{2.338657in}}{\pgfqpoint{2.507736in}{2.334266in}}{\pgfqpoint{2.518786in}{2.334266in}}%
\pgfpathclose%
\pgfusepath{stroke,fill}%
\end{pgfscope}%
\begin{pgfscope}%
\pgfpathrectangle{\pgfqpoint{0.800000in}{0.528000in}}{\pgfqpoint{4.960000in}{3.696000in}}%
\pgfusepath{clip}%
\pgfsetbuttcap%
\pgfsetroundjoin%
\definecolor{currentfill}{rgb}{0.000000,0.000000,0.000000}%
\pgfsetfillcolor{currentfill}%
\pgfsetlinewidth{1.003750pt}%
\definecolor{currentstroke}{rgb}{0.000000,0.000000,0.000000}%
\pgfsetstrokecolor{currentstroke}%
\pgfsetdash{}{0pt}%
\pgfpathmoveto{\pgfqpoint{2.518786in}{2.334266in}}%
\pgfpathcurveto{\pgfqpoint{2.529836in}{2.334266in}}{\pgfqpoint{2.540435in}{2.338657in}}{\pgfqpoint{2.548249in}{2.346470in}}%
\pgfpathcurveto{\pgfqpoint{2.556062in}{2.354284in}}{\pgfqpoint{2.560452in}{2.364883in}}{\pgfqpoint{2.560452in}{2.375933in}}%
\pgfpathcurveto{\pgfqpoint{2.560452in}{2.386983in}}{\pgfqpoint{2.556062in}{2.397582in}}{\pgfqpoint{2.548249in}{2.405396in}}%
\pgfpathcurveto{\pgfqpoint{2.540435in}{2.413209in}}{\pgfqpoint{2.529836in}{2.417600in}}{\pgfqpoint{2.518786in}{2.417600in}}%
\pgfpathcurveto{\pgfqpoint{2.507736in}{2.417600in}}{\pgfqpoint{2.497137in}{2.413209in}}{\pgfqpoint{2.489323in}{2.405396in}}%
\pgfpathcurveto{\pgfqpoint{2.481509in}{2.397582in}}{\pgfqpoint{2.477119in}{2.386983in}}{\pgfqpoint{2.477119in}{2.375933in}}%
\pgfpathcurveto{\pgfqpoint{2.477119in}{2.364883in}}{\pgfqpoint{2.481509in}{2.354284in}}{\pgfqpoint{2.489323in}{2.346470in}}%
\pgfpathcurveto{\pgfqpoint{2.497137in}{2.338657in}}{\pgfqpoint{2.507736in}{2.334266in}}{\pgfqpoint{2.518786in}{2.334266in}}%
\pgfpathclose%
\pgfusepath{stroke,fill}%
\end{pgfscope}%
\begin{pgfscope}%
\pgfpathrectangle{\pgfqpoint{0.800000in}{0.528000in}}{\pgfqpoint{4.960000in}{3.696000in}}%
\pgfusepath{clip}%
\pgfsetbuttcap%
\pgfsetroundjoin%
\definecolor{currentfill}{rgb}{0.000000,0.000000,0.000000}%
\pgfsetfillcolor{currentfill}%
\pgfsetlinewidth{1.003750pt}%
\definecolor{currentstroke}{rgb}{0.000000,0.000000,0.000000}%
\pgfsetstrokecolor{currentstroke}%
\pgfsetdash{}{0pt}%
\pgfpathmoveto{\pgfqpoint{2.518786in}{2.334266in}}%
\pgfpathcurveto{\pgfqpoint{2.529836in}{2.334266in}}{\pgfqpoint{2.540435in}{2.338657in}}{\pgfqpoint{2.548249in}{2.346470in}}%
\pgfpathcurveto{\pgfqpoint{2.556062in}{2.354284in}}{\pgfqpoint{2.560452in}{2.364883in}}{\pgfqpoint{2.560452in}{2.375933in}}%
\pgfpathcurveto{\pgfqpoint{2.560452in}{2.386983in}}{\pgfqpoint{2.556062in}{2.397582in}}{\pgfqpoint{2.548249in}{2.405396in}}%
\pgfpathcurveto{\pgfqpoint{2.540435in}{2.413209in}}{\pgfqpoint{2.529836in}{2.417600in}}{\pgfqpoint{2.518786in}{2.417600in}}%
\pgfpathcurveto{\pgfqpoint{2.507736in}{2.417600in}}{\pgfqpoint{2.497137in}{2.413209in}}{\pgfqpoint{2.489323in}{2.405396in}}%
\pgfpathcurveto{\pgfqpoint{2.481509in}{2.397582in}}{\pgfqpoint{2.477119in}{2.386983in}}{\pgfqpoint{2.477119in}{2.375933in}}%
\pgfpathcurveto{\pgfqpoint{2.477119in}{2.364883in}}{\pgfqpoint{2.481509in}{2.354284in}}{\pgfqpoint{2.489323in}{2.346470in}}%
\pgfpathcurveto{\pgfqpoint{2.497137in}{2.338657in}}{\pgfqpoint{2.507736in}{2.334266in}}{\pgfqpoint{2.518786in}{2.334266in}}%
\pgfpathclose%
\pgfusepath{stroke,fill}%
\end{pgfscope}%
\begin{pgfscope}%
\pgfpathrectangle{\pgfqpoint{0.800000in}{0.528000in}}{\pgfqpoint{4.960000in}{3.696000in}}%
\pgfusepath{clip}%
\pgfsetbuttcap%
\pgfsetroundjoin%
\definecolor{currentfill}{rgb}{0.000000,0.000000,0.000000}%
\pgfsetfillcolor{currentfill}%
\pgfsetlinewidth{1.003750pt}%
\definecolor{currentstroke}{rgb}{0.000000,0.000000,0.000000}%
\pgfsetstrokecolor{currentstroke}%
\pgfsetdash{}{0pt}%
\pgfpathmoveto{\pgfqpoint{2.518786in}{2.334266in}}%
\pgfpathcurveto{\pgfqpoint{2.529836in}{2.334266in}}{\pgfqpoint{2.540435in}{2.338657in}}{\pgfqpoint{2.548249in}{2.346470in}}%
\pgfpathcurveto{\pgfqpoint{2.556062in}{2.354284in}}{\pgfqpoint{2.560452in}{2.364883in}}{\pgfqpoint{2.560452in}{2.375933in}}%
\pgfpathcurveto{\pgfqpoint{2.560452in}{2.386983in}}{\pgfqpoint{2.556062in}{2.397582in}}{\pgfqpoint{2.548249in}{2.405396in}}%
\pgfpathcurveto{\pgfqpoint{2.540435in}{2.413209in}}{\pgfqpoint{2.529836in}{2.417600in}}{\pgfqpoint{2.518786in}{2.417600in}}%
\pgfpathcurveto{\pgfqpoint{2.507736in}{2.417600in}}{\pgfqpoint{2.497137in}{2.413209in}}{\pgfqpoint{2.489323in}{2.405396in}}%
\pgfpathcurveto{\pgfqpoint{2.481509in}{2.397582in}}{\pgfqpoint{2.477119in}{2.386983in}}{\pgfqpoint{2.477119in}{2.375933in}}%
\pgfpathcurveto{\pgfqpoint{2.477119in}{2.364883in}}{\pgfqpoint{2.481509in}{2.354284in}}{\pgfqpoint{2.489323in}{2.346470in}}%
\pgfpathcurveto{\pgfqpoint{2.497137in}{2.338657in}}{\pgfqpoint{2.507736in}{2.334266in}}{\pgfqpoint{2.518786in}{2.334266in}}%
\pgfpathclose%
\pgfusepath{stroke,fill}%
\end{pgfscope}%
\begin{pgfscope}%
\pgfpathrectangle{\pgfqpoint{0.800000in}{0.528000in}}{\pgfqpoint{4.960000in}{3.696000in}}%
\pgfusepath{clip}%
\pgfsetbuttcap%
\pgfsetroundjoin%
\definecolor{currentfill}{rgb}{0.000000,0.000000,0.000000}%
\pgfsetfillcolor{currentfill}%
\pgfsetlinewidth{1.003750pt}%
\definecolor{currentstroke}{rgb}{0.000000,0.000000,0.000000}%
\pgfsetstrokecolor{currentstroke}%
\pgfsetdash{}{0pt}%
\pgfpathmoveto{\pgfqpoint{2.518786in}{2.334266in}}%
\pgfpathcurveto{\pgfqpoint{2.529836in}{2.334266in}}{\pgfqpoint{2.540435in}{2.338657in}}{\pgfqpoint{2.548249in}{2.346470in}}%
\pgfpathcurveto{\pgfqpoint{2.556062in}{2.354284in}}{\pgfqpoint{2.560452in}{2.364883in}}{\pgfqpoint{2.560452in}{2.375933in}}%
\pgfpathcurveto{\pgfqpoint{2.560452in}{2.386983in}}{\pgfqpoint{2.556062in}{2.397582in}}{\pgfqpoint{2.548249in}{2.405396in}}%
\pgfpathcurveto{\pgfqpoint{2.540435in}{2.413209in}}{\pgfqpoint{2.529836in}{2.417600in}}{\pgfqpoint{2.518786in}{2.417600in}}%
\pgfpathcurveto{\pgfqpoint{2.507736in}{2.417600in}}{\pgfqpoint{2.497137in}{2.413209in}}{\pgfqpoint{2.489323in}{2.405396in}}%
\pgfpathcurveto{\pgfqpoint{2.481509in}{2.397582in}}{\pgfqpoint{2.477119in}{2.386983in}}{\pgfqpoint{2.477119in}{2.375933in}}%
\pgfpathcurveto{\pgfqpoint{2.477119in}{2.364883in}}{\pgfqpoint{2.481509in}{2.354284in}}{\pgfqpoint{2.489323in}{2.346470in}}%
\pgfpathcurveto{\pgfqpoint{2.497137in}{2.338657in}}{\pgfqpoint{2.507736in}{2.334266in}}{\pgfqpoint{2.518786in}{2.334266in}}%
\pgfpathclose%
\pgfusepath{stroke,fill}%
\end{pgfscope}%
\begin{pgfscope}%
\pgfpathrectangle{\pgfqpoint{0.800000in}{0.528000in}}{\pgfqpoint{4.960000in}{3.696000in}}%
\pgfusepath{clip}%
\pgfsetbuttcap%
\pgfsetroundjoin%
\definecolor{currentfill}{rgb}{0.000000,0.000000,0.000000}%
\pgfsetfillcolor{currentfill}%
\pgfsetlinewidth{1.003750pt}%
\definecolor{currentstroke}{rgb}{0.000000,0.000000,0.000000}%
\pgfsetstrokecolor{currentstroke}%
\pgfsetdash{}{0pt}%
\pgfpathmoveto{\pgfqpoint{2.518786in}{2.334266in}}%
\pgfpathcurveto{\pgfqpoint{2.529836in}{2.334266in}}{\pgfqpoint{2.540435in}{2.338657in}}{\pgfqpoint{2.548249in}{2.346470in}}%
\pgfpathcurveto{\pgfqpoint{2.556062in}{2.354284in}}{\pgfqpoint{2.560452in}{2.364883in}}{\pgfqpoint{2.560452in}{2.375933in}}%
\pgfpathcurveto{\pgfqpoint{2.560452in}{2.386983in}}{\pgfqpoint{2.556062in}{2.397582in}}{\pgfqpoint{2.548249in}{2.405396in}}%
\pgfpathcurveto{\pgfqpoint{2.540435in}{2.413209in}}{\pgfqpoint{2.529836in}{2.417600in}}{\pgfqpoint{2.518786in}{2.417600in}}%
\pgfpathcurveto{\pgfqpoint{2.507736in}{2.417600in}}{\pgfqpoint{2.497137in}{2.413209in}}{\pgfqpoint{2.489323in}{2.405396in}}%
\pgfpathcurveto{\pgfqpoint{2.481509in}{2.397582in}}{\pgfqpoint{2.477119in}{2.386983in}}{\pgfqpoint{2.477119in}{2.375933in}}%
\pgfpathcurveto{\pgfqpoint{2.477119in}{2.364883in}}{\pgfqpoint{2.481509in}{2.354284in}}{\pgfqpoint{2.489323in}{2.346470in}}%
\pgfpathcurveto{\pgfqpoint{2.497137in}{2.338657in}}{\pgfqpoint{2.507736in}{2.334266in}}{\pgfqpoint{2.518786in}{2.334266in}}%
\pgfpathclose%
\pgfusepath{stroke,fill}%
\end{pgfscope}%
\begin{pgfscope}%
\pgfpathrectangle{\pgfqpoint{0.800000in}{0.528000in}}{\pgfqpoint{4.960000in}{3.696000in}}%
\pgfusepath{clip}%
\pgfsetbuttcap%
\pgfsetroundjoin%
\definecolor{currentfill}{rgb}{0.000000,0.000000,0.000000}%
\pgfsetfillcolor{currentfill}%
\pgfsetlinewidth{1.003750pt}%
\definecolor{currentstroke}{rgb}{0.000000,0.000000,0.000000}%
\pgfsetstrokecolor{currentstroke}%
\pgfsetdash{}{0pt}%
\pgfpathmoveto{\pgfqpoint{2.518786in}{2.334266in}}%
\pgfpathcurveto{\pgfqpoint{2.529836in}{2.334266in}}{\pgfqpoint{2.540435in}{2.338657in}}{\pgfqpoint{2.548249in}{2.346470in}}%
\pgfpathcurveto{\pgfqpoint{2.556062in}{2.354284in}}{\pgfqpoint{2.560452in}{2.364883in}}{\pgfqpoint{2.560452in}{2.375933in}}%
\pgfpathcurveto{\pgfqpoint{2.560452in}{2.386983in}}{\pgfqpoint{2.556062in}{2.397582in}}{\pgfqpoint{2.548249in}{2.405396in}}%
\pgfpathcurveto{\pgfqpoint{2.540435in}{2.413209in}}{\pgfqpoint{2.529836in}{2.417600in}}{\pgfqpoint{2.518786in}{2.417600in}}%
\pgfpathcurveto{\pgfqpoint{2.507736in}{2.417600in}}{\pgfqpoint{2.497137in}{2.413209in}}{\pgfqpoint{2.489323in}{2.405396in}}%
\pgfpathcurveto{\pgfqpoint{2.481509in}{2.397582in}}{\pgfqpoint{2.477119in}{2.386983in}}{\pgfqpoint{2.477119in}{2.375933in}}%
\pgfpathcurveto{\pgfqpoint{2.477119in}{2.364883in}}{\pgfqpoint{2.481509in}{2.354284in}}{\pgfqpoint{2.489323in}{2.346470in}}%
\pgfpathcurveto{\pgfqpoint{2.497137in}{2.338657in}}{\pgfqpoint{2.507736in}{2.334266in}}{\pgfqpoint{2.518786in}{2.334266in}}%
\pgfpathclose%
\pgfusepath{stroke,fill}%
\end{pgfscope}%
\begin{pgfscope}%
\pgfpathrectangle{\pgfqpoint{0.800000in}{0.528000in}}{\pgfqpoint{4.960000in}{3.696000in}}%
\pgfusepath{clip}%
\pgfsetbuttcap%
\pgfsetroundjoin%
\definecolor{currentfill}{rgb}{0.000000,0.000000,0.000000}%
\pgfsetfillcolor{currentfill}%
\pgfsetlinewidth{1.003750pt}%
\definecolor{currentstroke}{rgb}{0.000000,0.000000,0.000000}%
\pgfsetstrokecolor{currentstroke}%
\pgfsetdash{}{0pt}%
\pgfpathmoveto{\pgfqpoint{2.518786in}{2.334266in}}%
\pgfpathcurveto{\pgfqpoint{2.529836in}{2.334266in}}{\pgfqpoint{2.540435in}{2.338657in}}{\pgfqpoint{2.548249in}{2.346470in}}%
\pgfpathcurveto{\pgfqpoint{2.556062in}{2.354284in}}{\pgfqpoint{2.560452in}{2.364883in}}{\pgfqpoint{2.560452in}{2.375933in}}%
\pgfpathcurveto{\pgfqpoint{2.560452in}{2.386983in}}{\pgfqpoint{2.556062in}{2.397582in}}{\pgfqpoint{2.548249in}{2.405396in}}%
\pgfpathcurveto{\pgfqpoint{2.540435in}{2.413209in}}{\pgfqpoint{2.529836in}{2.417600in}}{\pgfqpoint{2.518786in}{2.417600in}}%
\pgfpathcurveto{\pgfqpoint{2.507736in}{2.417600in}}{\pgfqpoint{2.497137in}{2.413209in}}{\pgfqpoint{2.489323in}{2.405396in}}%
\pgfpathcurveto{\pgfqpoint{2.481509in}{2.397582in}}{\pgfqpoint{2.477119in}{2.386983in}}{\pgfqpoint{2.477119in}{2.375933in}}%
\pgfpathcurveto{\pgfqpoint{2.477119in}{2.364883in}}{\pgfqpoint{2.481509in}{2.354284in}}{\pgfqpoint{2.489323in}{2.346470in}}%
\pgfpathcurveto{\pgfqpoint{2.497137in}{2.338657in}}{\pgfqpoint{2.507736in}{2.334266in}}{\pgfqpoint{2.518786in}{2.334266in}}%
\pgfpathclose%
\pgfusepath{stroke,fill}%
\end{pgfscope}%
\begin{pgfscope}%
\pgfpathrectangle{\pgfqpoint{0.800000in}{0.528000in}}{\pgfqpoint{4.960000in}{3.696000in}}%
\pgfusepath{clip}%
\pgfsetbuttcap%
\pgfsetroundjoin%
\definecolor{currentfill}{rgb}{0.000000,0.000000,0.000000}%
\pgfsetfillcolor{currentfill}%
\pgfsetlinewidth{1.003750pt}%
\definecolor{currentstroke}{rgb}{0.000000,0.000000,0.000000}%
\pgfsetstrokecolor{currentstroke}%
\pgfsetdash{}{0pt}%
\pgfpathmoveto{\pgfqpoint{2.518786in}{3.984333in}}%
\pgfpathcurveto{\pgfqpoint{2.529836in}{3.984333in}}{\pgfqpoint{2.540435in}{3.988724in}}{\pgfqpoint{2.548249in}{3.996537in}}%
\pgfpathcurveto{\pgfqpoint{2.556062in}{4.004351in}}{\pgfqpoint{2.560452in}{4.014950in}}{\pgfqpoint{2.560452in}{4.026000in}}%
\pgfpathcurveto{\pgfqpoint{2.560452in}{4.037050in}}{\pgfqpoint{2.556062in}{4.047649in}}{\pgfqpoint{2.548249in}{4.055463in}}%
\pgfpathcurveto{\pgfqpoint{2.540435in}{4.063276in}}{\pgfqpoint{2.529836in}{4.067667in}}{\pgfqpoint{2.518786in}{4.067667in}}%
\pgfpathcurveto{\pgfqpoint{2.507736in}{4.067667in}}{\pgfqpoint{2.497137in}{4.063276in}}{\pgfqpoint{2.489323in}{4.055463in}}%
\pgfpathcurveto{\pgfqpoint{2.481509in}{4.047649in}}{\pgfqpoint{2.477119in}{4.037050in}}{\pgfqpoint{2.477119in}{4.026000in}}%
\pgfpathcurveto{\pgfqpoint{2.477119in}{4.014950in}}{\pgfqpoint{2.481509in}{4.004351in}}{\pgfqpoint{2.489323in}{3.996537in}}%
\pgfpathcurveto{\pgfqpoint{2.497137in}{3.988724in}}{\pgfqpoint{2.507736in}{3.984333in}}{\pgfqpoint{2.518786in}{3.984333in}}%
\pgfpathclose%
\pgfusepath{stroke,fill}%
\end{pgfscope}%
\begin{pgfscope}%
\pgfpathrectangle{\pgfqpoint{0.800000in}{0.528000in}}{\pgfqpoint{4.960000in}{3.696000in}}%
\pgfusepath{clip}%
\pgfsetbuttcap%
\pgfsetroundjoin%
\definecolor{currentfill}{rgb}{0.000000,0.000000,0.000000}%
\pgfsetfillcolor{currentfill}%
\pgfsetlinewidth{1.003750pt}%
\definecolor{currentstroke}{rgb}{0.000000,0.000000,0.000000}%
\pgfsetstrokecolor{currentstroke}%
\pgfsetdash{}{0pt}%
\pgfpathmoveto{\pgfqpoint{2.518786in}{2.334266in}}%
\pgfpathcurveto{\pgfqpoint{2.529836in}{2.334266in}}{\pgfqpoint{2.540435in}{2.338657in}}{\pgfqpoint{2.548249in}{2.346470in}}%
\pgfpathcurveto{\pgfqpoint{2.556062in}{2.354284in}}{\pgfqpoint{2.560452in}{2.364883in}}{\pgfqpoint{2.560452in}{2.375933in}}%
\pgfpathcurveto{\pgfqpoint{2.560452in}{2.386983in}}{\pgfqpoint{2.556062in}{2.397582in}}{\pgfqpoint{2.548249in}{2.405396in}}%
\pgfpathcurveto{\pgfqpoint{2.540435in}{2.413209in}}{\pgfqpoint{2.529836in}{2.417600in}}{\pgfqpoint{2.518786in}{2.417600in}}%
\pgfpathcurveto{\pgfqpoint{2.507736in}{2.417600in}}{\pgfqpoint{2.497137in}{2.413209in}}{\pgfqpoint{2.489323in}{2.405396in}}%
\pgfpathcurveto{\pgfqpoint{2.481509in}{2.397582in}}{\pgfqpoint{2.477119in}{2.386983in}}{\pgfqpoint{2.477119in}{2.375933in}}%
\pgfpathcurveto{\pgfqpoint{2.477119in}{2.364883in}}{\pgfqpoint{2.481509in}{2.354284in}}{\pgfqpoint{2.489323in}{2.346470in}}%
\pgfpathcurveto{\pgfqpoint{2.497137in}{2.338657in}}{\pgfqpoint{2.507736in}{2.334266in}}{\pgfqpoint{2.518786in}{2.334266in}}%
\pgfpathclose%
\pgfusepath{stroke,fill}%
\end{pgfscope}%
\begin{pgfscope}%
\pgfpathrectangle{\pgfqpoint{0.800000in}{0.528000in}}{\pgfqpoint{4.960000in}{3.696000in}}%
\pgfusepath{clip}%
\pgfsetbuttcap%
\pgfsetroundjoin%
\definecolor{currentfill}{rgb}{0.000000,0.000000,0.000000}%
\pgfsetfillcolor{currentfill}%
\pgfsetlinewidth{1.003750pt}%
\definecolor{currentstroke}{rgb}{0.000000,0.000000,0.000000}%
\pgfsetstrokecolor{currentstroke}%
\pgfsetdash{}{0pt}%
\pgfpathmoveto{\pgfqpoint{2.518786in}{2.334266in}}%
\pgfpathcurveto{\pgfqpoint{2.529836in}{2.334266in}}{\pgfqpoint{2.540435in}{2.338657in}}{\pgfqpoint{2.548249in}{2.346470in}}%
\pgfpathcurveto{\pgfqpoint{2.556062in}{2.354284in}}{\pgfqpoint{2.560452in}{2.364883in}}{\pgfqpoint{2.560452in}{2.375933in}}%
\pgfpathcurveto{\pgfqpoint{2.560452in}{2.386983in}}{\pgfqpoint{2.556062in}{2.397582in}}{\pgfqpoint{2.548249in}{2.405396in}}%
\pgfpathcurveto{\pgfqpoint{2.540435in}{2.413209in}}{\pgfqpoint{2.529836in}{2.417600in}}{\pgfqpoint{2.518786in}{2.417600in}}%
\pgfpathcurveto{\pgfqpoint{2.507736in}{2.417600in}}{\pgfqpoint{2.497137in}{2.413209in}}{\pgfqpoint{2.489323in}{2.405396in}}%
\pgfpathcurveto{\pgfqpoint{2.481509in}{2.397582in}}{\pgfqpoint{2.477119in}{2.386983in}}{\pgfqpoint{2.477119in}{2.375933in}}%
\pgfpathcurveto{\pgfqpoint{2.477119in}{2.364883in}}{\pgfqpoint{2.481509in}{2.354284in}}{\pgfqpoint{2.489323in}{2.346470in}}%
\pgfpathcurveto{\pgfqpoint{2.497137in}{2.338657in}}{\pgfqpoint{2.507736in}{2.334266in}}{\pgfqpoint{2.518786in}{2.334266in}}%
\pgfpathclose%
\pgfusepath{stroke,fill}%
\end{pgfscope}%
\begin{pgfscope}%
\pgfpathrectangle{\pgfqpoint{0.800000in}{0.528000in}}{\pgfqpoint{4.960000in}{3.696000in}}%
\pgfusepath{clip}%
\pgfsetbuttcap%
\pgfsetroundjoin%
\definecolor{currentfill}{rgb}{0.000000,0.000000,0.000000}%
\pgfsetfillcolor{currentfill}%
\pgfsetlinewidth{1.003750pt}%
\definecolor{currentstroke}{rgb}{0.000000,0.000000,0.000000}%
\pgfsetstrokecolor{currentstroke}%
\pgfsetdash{}{0pt}%
\pgfpathmoveto{\pgfqpoint{2.518786in}{2.334266in}}%
\pgfpathcurveto{\pgfqpoint{2.529836in}{2.334266in}}{\pgfqpoint{2.540435in}{2.338657in}}{\pgfqpoint{2.548249in}{2.346470in}}%
\pgfpathcurveto{\pgfqpoint{2.556062in}{2.354284in}}{\pgfqpoint{2.560452in}{2.364883in}}{\pgfqpoint{2.560452in}{2.375933in}}%
\pgfpathcurveto{\pgfqpoint{2.560452in}{2.386983in}}{\pgfqpoint{2.556062in}{2.397582in}}{\pgfqpoint{2.548249in}{2.405396in}}%
\pgfpathcurveto{\pgfqpoint{2.540435in}{2.413209in}}{\pgfqpoint{2.529836in}{2.417600in}}{\pgfqpoint{2.518786in}{2.417600in}}%
\pgfpathcurveto{\pgfqpoint{2.507736in}{2.417600in}}{\pgfqpoint{2.497137in}{2.413209in}}{\pgfqpoint{2.489323in}{2.405396in}}%
\pgfpathcurveto{\pgfqpoint{2.481509in}{2.397582in}}{\pgfqpoint{2.477119in}{2.386983in}}{\pgfqpoint{2.477119in}{2.375933in}}%
\pgfpathcurveto{\pgfqpoint{2.477119in}{2.364883in}}{\pgfqpoint{2.481509in}{2.354284in}}{\pgfqpoint{2.489323in}{2.346470in}}%
\pgfpathcurveto{\pgfqpoint{2.497137in}{2.338657in}}{\pgfqpoint{2.507736in}{2.334266in}}{\pgfqpoint{2.518786in}{2.334266in}}%
\pgfpathclose%
\pgfusepath{stroke,fill}%
\end{pgfscope}%
\begin{pgfscope}%
\pgfpathrectangle{\pgfqpoint{0.800000in}{0.528000in}}{\pgfqpoint{4.960000in}{3.696000in}}%
\pgfusepath{clip}%
\pgfsetbuttcap%
\pgfsetroundjoin%
\definecolor{currentfill}{rgb}{0.000000,0.000000,0.000000}%
\pgfsetfillcolor{currentfill}%
\pgfsetlinewidth{1.003750pt}%
\definecolor{currentstroke}{rgb}{0.000000,0.000000,0.000000}%
\pgfsetstrokecolor{currentstroke}%
\pgfsetdash{}{0pt}%
\pgfpathmoveto{\pgfqpoint{2.518786in}{2.334266in}}%
\pgfpathcurveto{\pgfqpoint{2.529836in}{2.334266in}}{\pgfqpoint{2.540435in}{2.338657in}}{\pgfqpoint{2.548249in}{2.346470in}}%
\pgfpathcurveto{\pgfqpoint{2.556062in}{2.354284in}}{\pgfqpoint{2.560452in}{2.364883in}}{\pgfqpoint{2.560452in}{2.375933in}}%
\pgfpathcurveto{\pgfqpoint{2.560452in}{2.386983in}}{\pgfqpoint{2.556062in}{2.397582in}}{\pgfqpoint{2.548249in}{2.405396in}}%
\pgfpathcurveto{\pgfqpoint{2.540435in}{2.413209in}}{\pgfqpoint{2.529836in}{2.417600in}}{\pgfqpoint{2.518786in}{2.417600in}}%
\pgfpathcurveto{\pgfqpoint{2.507736in}{2.417600in}}{\pgfqpoint{2.497137in}{2.413209in}}{\pgfqpoint{2.489323in}{2.405396in}}%
\pgfpathcurveto{\pgfqpoint{2.481509in}{2.397582in}}{\pgfqpoint{2.477119in}{2.386983in}}{\pgfqpoint{2.477119in}{2.375933in}}%
\pgfpathcurveto{\pgfqpoint{2.477119in}{2.364883in}}{\pgfqpoint{2.481509in}{2.354284in}}{\pgfqpoint{2.489323in}{2.346470in}}%
\pgfpathcurveto{\pgfqpoint{2.497137in}{2.338657in}}{\pgfqpoint{2.507736in}{2.334266in}}{\pgfqpoint{2.518786in}{2.334266in}}%
\pgfpathclose%
\pgfusepath{stroke,fill}%
\end{pgfscope}%
\begin{pgfscope}%
\pgfpathrectangle{\pgfqpoint{0.800000in}{0.528000in}}{\pgfqpoint{4.960000in}{3.696000in}}%
\pgfusepath{clip}%
\pgfsetbuttcap%
\pgfsetroundjoin%
\definecolor{currentfill}{rgb}{0.000000,0.000000,0.000000}%
\pgfsetfillcolor{currentfill}%
\pgfsetlinewidth{1.003750pt}%
\definecolor{currentstroke}{rgb}{0.000000,0.000000,0.000000}%
\pgfsetstrokecolor{currentstroke}%
\pgfsetdash{}{0pt}%
\pgfpathmoveto{\pgfqpoint{2.518786in}{3.984333in}}%
\pgfpathcurveto{\pgfqpoint{2.529836in}{3.984333in}}{\pgfqpoint{2.540435in}{3.988724in}}{\pgfqpoint{2.548249in}{3.996537in}}%
\pgfpathcurveto{\pgfqpoint{2.556062in}{4.004351in}}{\pgfqpoint{2.560452in}{4.014950in}}{\pgfqpoint{2.560452in}{4.026000in}}%
\pgfpathcurveto{\pgfqpoint{2.560452in}{4.037050in}}{\pgfqpoint{2.556062in}{4.047649in}}{\pgfqpoint{2.548249in}{4.055463in}}%
\pgfpathcurveto{\pgfqpoint{2.540435in}{4.063276in}}{\pgfqpoint{2.529836in}{4.067667in}}{\pgfqpoint{2.518786in}{4.067667in}}%
\pgfpathcurveto{\pgfqpoint{2.507736in}{4.067667in}}{\pgfqpoint{2.497137in}{4.063276in}}{\pgfqpoint{2.489323in}{4.055463in}}%
\pgfpathcurveto{\pgfqpoint{2.481509in}{4.047649in}}{\pgfqpoint{2.477119in}{4.037050in}}{\pgfqpoint{2.477119in}{4.026000in}}%
\pgfpathcurveto{\pgfqpoint{2.477119in}{4.014950in}}{\pgfqpoint{2.481509in}{4.004351in}}{\pgfqpoint{2.489323in}{3.996537in}}%
\pgfpathcurveto{\pgfqpoint{2.497137in}{3.988724in}}{\pgfqpoint{2.507736in}{3.984333in}}{\pgfqpoint{2.518786in}{3.984333in}}%
\pgfpathclose%
\pgfusepath{stroke,fill}%
\end{pgfscope}%
\begin{pgfscope}%
\pgfpathrectangle{\pgfqpoint{0.800000in}{0.528000in}}{\pgfqpoint{4.960000in}{3.696000in}}%
\pgfusepath{clip}%
\pgfsetbuttcap%
\pgfsetroundjoin%
\definecolor{currentfill}{rgb}{0.000000,0.000000,0.000000}%
\pgfsetfillcolor{currentfill}%
\pgfsetlinewidth{1.003750pt}%
\definecolor{currentstroke}{rgb}{0.000000,0.000000,0.000000}%
\pgfsetstrokecolor{currentstroke}%
\pgfsetdash{}{0pt}%
\pgfpathmoveto{\pgfqpoint{2.518786in}{2.334266in}}%
\pgfpathcurveto{\pgfqpoint{2.529836in}{2.334266in}}{\pgfqpoint{2.540435in}{2.338657in}}{\pgfqpoint{2.548249in}{2.346470in}}%
\pgfpathcurveto{\pgfqpoint{2.556062in}{2.354284in}}{\pgfqpoint{2.560452in}{2.364883in}}{\pgfqpoint{2.560452in}{2.375933in}}%
\pgfpathcurveto{\pgfqpoint{2.560452in}{2.386983in}}{\pgfqpoint{2.556062in}{2.397582in}}{\pgfqpoint{2.548249in}{2.405396in}}%
\pgfpathcurveto{\pgfqpoint{2.540435in}{2.413209in}}{\pgfqpoint{2.529836in}{2.417600in}}{\pgfqpoint{2.518786in}{2.417600in}}%
\pgfpathcurveto{\pgfqpoint{2.507736in}{2.417600in}}{\pgfqpoint{2.497137in}{2.413209in}}{\pgfqpoint{2.489323in}{2.405396in}}%
\pgfpathcurveto{\pgfqpoint{2.481509in}{2.397582in}}{\pgfqpoint{2.477119in}{2.386983in}}{\pgfqpoint{2.477119in}{2.375933in}}%
\pgfpathcurveto{\pgfqpoint{2.477119in}{2.364883in}}{\pgfqpoint{2.481509in}{2.354284in}}{\pgfqpoint{2.489323in}{2.346470in}}%
\pgfpathcurveto{\pgfqpoint{2.497137in}{2.338657in}}{\pgfqpoint{2.507736in}{2.334266in}}{\pgfqpoint{2.518786in}{2.334266in}}%
\pgfpathclose%
\pgfusepath{stroke,fill}%
\end{pgfscope}%
\begin{pgfscope}%
\pgfpathrectangle{\pgfqpoint{0.800000in}{0.528000in}}{\pgfqpoint{4.960000in}{3.696000in}}%
\pgfusepath{clip}%
\pgfsetbuttcap%
\pgfsetroundjoin%
\definecolor{currentfill}{rgb}{0.000000,0.000000,0.000000}%
\pgfsetfillcolor{currentfill}%
\pgfsetlinewidth{1.003750pt}%
\definecolor{currentstroke}{rgb}{0.000000,0.000000,0.000000}%
\pgfsetstrokecolor{currentstroke}%
\pgfsetdash{}{0pt}%
\pgfpathmoveto{\pgfqpoint{2.518786in}{2.334266in}}%
\pgfpathcurveto{\pgfqpoint{2.529836in}{2.334266in}}{\pgfqpoint{2.540435in}{2.338657in}}{\pgfqpoint{2.548249in}{2.346470in}}%
\pgfpathcurveto{\pgfqpoint{2.556062in}{2.354284in}}{\pgfqpoint{2.560452in}{2.364883in}}{\pgfqpoint{2.560452in}{2.375933in}}%
\pgfpathcurveto{\pgfqpoint{2.560452in}{2.386983in}}{\pgfqpoint{2.556062in}{2.397582in}}{\pgfqpoint{2.548249in}{2.405396in}}%
\pgfpathcurveto{\pgfqpoint{2.540435in}{2.413209in}}{\pgfqpoint{2.529836in}{2.417600in}}{\pgfqpoint{2.518786in}{2.417600in}}%
\pgfpathcurveto{\pgfqpoint{2.507736in}{2.417600in}}{\pgfqpoint{2.497137in}{2.413209in}}{\pgfqpoint{2.489323in}{2.405396in}}%
\pgfpathcurveto{\pgfqpoint{2.481509in}{2.397582in}}{\pgfqpoint{2.477119in}{2.386983in}}{\pgfqpoint{2.477119in}{2.375933in}}%
\pgfpathcurveto{\pgfqpoint{2.477119in}{2.364883in}}{\pgfqpoint{2.481509in}{2.354284in}}{\pgfqpoint{2.489323in}{2.346470in}}%
\pgfpathcurveto{\pgfqpoint{2.497137in}{2.338657in}}{\pgfqpoint{2.507736in}{2.334266in}}{\pgfqpoint{2.518786in}{2.334266in}}%
\pgfpathclose%
\pgfusepath{stroke,fill}%
\end{pgfscope}%
\begin{pgfscope}%
\pgfpathrectangle{\pgfqpoint{0.800000in}{0.528000in}}{\pgfqpoint{4.960000in}{3.696000in}}%
\pgfusepath{clip}%
\pgfsetbuttcap%
\pgfsetroundjoin%
\definecolor{currentfill}{rgb}{0.000000,0.000000,0.000000}%
\pgfsetfillcolor{currentfill}%
\pgfsetlinewidth{1.003750pt}%
\definecolor{currentstroke}{rgb}{0.000000,0.000000,0.000000}%
\pgfsetstrokecolor{currentstroke}%
\pgfsetdash{}{0pt}%
\pgfpathmoveto{\pgfqpoint{2.518786in}{2.334266in}}%
\pgfpathcurveto{\pgfqpoint{2.529836in}{2.334266in}}{\pgfqpoint{2.540435in}{2.338657in}}{\pgfqpoint{2.548249in}{2.346470in}}%
\pgfpathcurveto{\pgfqpoint{2.556062in}{2.354284in}}{\pgfqpoint{2.560452in}{2.364883in}}{\pgfqpoint{2.560452in}{2.375933in}}%
\pgfpathcurveto{\pgfqpoint{2.560452in}{2.386983in}}{\pgfqpoint{2.556062in}{2.397582in}}{\pgfqpoint{2.548249in}{2.405396in}}%
\pgfpathcurveto{\pgfqpoint{2.540435in}{2.413209in}}{\pgfqpoint{2.529836in}{2.417600in}}{\pgfqpoint{2.518786in}{2.417600in}}%
\pgfpathcurveto{\pgfqpoint{2.507736in}{2.417600in}}{\pgfqpoint{2.497137in}{2.413209in}}{\pgfqpoint{2.489323in}{2.405396in}}%
\pgfpathcurveto{\pgfqpoint{2.481509in}{2.397582in}}{\pgfqpoint{2.477119in}{2.386983in}}{\pgfqpoint{2.477119in}{2.375933in}}%
\pgfpathcurveto{\pgfqpoint{2.477119in}{2.364883in}}{\pgfqpoint{2.481509in}{2.354284in}}{\pgfqpoint{2.489323in}{2.346470in}}%
\pgfpathcurveto{\pgfqpoint{2.497137in}{2.338657in}}{\pgfqpoint{2.507736in}{2.334266in}}{\pgfqpoint{2.518786in}{2.334266in}}%
\pgfpathclose%
\pgfusepath{stroke,fill}%
\end{pgfscope}%
\begin{pgfscope}%
\pgfpathrectangle{\pgfqpoint{0.800000in}{0.528000in}}{\pgfqpoint{4.960000in}{3.696000in}}%
\pgfusepath{clip}%
\pgfsetbuttcap%
\pgfsetroundjoin%
\definecolor{currentfill}{rgb}{0.000000,0.000000,0.000000}%
\pgfsetfillcolor{currentfill}%
\pgfsetlinewidth{1.003750pt}%
\definecolor{currentstroke}{rgb}{0.000000,0.000000,0.000000}%
\pgfsetstrokecolor{currentstroke}%
\pgfsetdash{}{0pt}%
\pgfpathmoveto{\pgfqpoint{2.518786in}{2.334266in}}%
\pgfpathcurveto{\pgfqpoint{2.529836in}{2.334266in}}{\pgfqpoint{2.540435in}{2.338657in}}{\pgfqpoint{2.548249in}{2.346470in}}%
\pgfpathcurveto{\pgfqpoint{2.556062in}{2.354284in}}{\pgfqpoint{2.560452in}{2.364883in}}{\pgfqpoint{2.560452in}{2.375933in}}%
\pgfpathcurveto{\pgfqpoint{2.560452in}{2.386983in}}{\pgfqpoint{2.556062in}{2.397582in}}{\pgfqpoint{2.548249in}{2.405396in}}%
\pgfpathcurveto{\pgfqpoint{2.540435in}{2.413209in}}{\pgfqpoint{2.529836in}{2.417600in}}{\pgfqpoint{2.518786in}{2.417600in}}%
\pgfpathcurveto{\pgfqpoint{2.507736in}{2.417600in}}{\pgfqpoint{2.497137in}{2.413209in}}{\pgfqpoint{2.489323in}{2.405396in}}%
\pgfpathcurveto{\pgfqpoint{2.481509in}{2.397582in}}{\pgfqpoint{2.477119in}{2.386983in}}{\pgfqpoint{2.477119in}{2.375933in}}%
\pgfpathcurveto{\pgfqpoint{2.477119in}{2.364883in}}{\pgfqpoint{2.481509in}{2.354284in}}{\pgfqpoint{2.489323in}{2.346470in}}%
\pgfpathcurveto{\pgfqpoint{2.497137in}{2.338657in}}{\pgfqpoint{2.507736in}{2.334266in}}{\pgfqpoint{2.518786in}{2.334266in}}%
\pgfpathclose%
\pgfusepath{stroke,fill}%
\end{pgfscope}%
\begin{pgfscope}%
\pgfpathrectangle{\pgfqpoint{0.800000in}{0.528000in}}{\pgfqpoint{4.960000in}{3.696000in}}%
\pgfusepath{clip}%
\pgfsetbuttcap%
\pgfsetroundjoin%
\definecolor{currentfill}{rgb}{0.000000,0.000000,0.000000}%
\pgfsetfillcolor{currentfill}%
\pgfsetlinewidth{1.003750pt}%
\definecolor{currentstroke}{rgb}{0.000000,0.000000,0.000000}%
\pgfsetstrokecolor{currentstroke}%
\pgfsetdash{}{0pt}%
\pgfpathmoveto{\pgfqpoint{2.518786in}{0.684199in}}%
\pgfpathcurveto{\pgfqpoint{2.529836in}{0.684199in}}{\pgfqpoint{2.540435in}{0.688590in}}{\pgfqpoint{2.548249in}{0.696403in}}%
\pgfpathcurveto{\pgfqpoint{2.556062in}{0.704217in}}{\pgfqpoint{2.560452in}{0.714816in}}{\pgfqpoint{2.560452in}{0.725866in}}%
\pgfpathcurveto{\pgfqpoint{2.560452in}{0.736916in}}{\pgfqpoint{2.556062in}{0.747515in}}{\pgfqpoint{2.548249in}{0.755329in}}%
\pgfpathcurveto{\pgfqpoint{2.540435in}{0.763142in}}{\pgfqpoint{2.529836in}{0.767533in}}{\pgfqpoint{2.518786in}{0.767533in}}%
\pgfpathcurveto{\pgfqpoint{2.507736in}{0.767533in}}{\pgfqpoint{2.497137in}{0.763142in}}{\pgfqpoint{2.489323in}{0.755329in}}%
\pgfpathcurveto{\pgfqpoint{2.481509in}{0.747515in}}{\pgfqpoint{2.477119in}{0.736916in}}{\pgfqpoint{2.477119in}{0.725866in}}%
\pgfpathcurveto{\pgfqpoint{2.477119in}{0.714816in}}{\pgfqpoint{2.481509in}{0.704217in}}{\pgfqpoint{2.489323in}{0.696403in}}%
\pgfpathcurveto{\pgfqpoint{2.497137in}{0.688590in}}{\pgfqpoint{2.507736in}{0.684199in}}{\pgfqpoint{2.518786in}{0.684199in}}%
\pgfpathclose%
\pgfusepath{stroke,fill}%
\end{pgfscope}%
\begin{pgfscope}%
\pgfpathrectangle{\pgfqpoint{0.800000in}{0.528000in}}{\pgfqpoint{4.960000in}{3.696000in}}%
\pgfusepath{clip}%
\pgfsetbuttcap%
\pgfsetroundjoin%
\definecolor{currentfill}{rgb}{0.000000,0.000000,0.000000}%
\pgfsetfillcolor{currentfill}%
\pgfsetlinewidth{1.003750pt}%
\definecolor{currentstroke}{rgb}{0.000000,0.000000,0.000000}%
\pgfsetstrokecolor{currentstroke}%
\pgfsetdash{}{0pt}%
\pgfpathmoveto{\pgfqpoint{2.518786in}{3.984333in}}%
\pgfpathcurveto{\pgfqpoint{2.529836in}{3.984333in}}{\pgfqpoint{2.540435in}{3.988724in}}{\pgfqpoint{2.548249in}{3.996537in}}%
\pgfpathcurveto{\pgfqpoint{2.556062in}{4.004351in}}{\pgfqpoint{2.560452in}{4.014950in}}{\pgfqpoint{2.560452in}{4.026000in}}%
\pgfpathcurveto{\pgfqpoint{2.560452in}{4.037050in}}{\pgfqpoint{2.556062in}{4.047649in}}{\pgfqpoint{2.548249in}{4.055463in}}%
\pgfpathcurveto{\pgfqpoint{2.540435in}{4.063276in}}{\pgfqpoint{2.529836in}{4.067667in}}{\pgfqpoint{2.518786in}{4.067667in}}%
\pgfpathcurveto{\pgfqpoint{2.507736in}{4.067667in}}{\pgfqpoint{2.497137in}{4.063276in}}{\pgfqpoint{2.489323in}{4.055463in}}%
\pgfpathcurveto{\pgfqpoint{2.481509in}{4.047649in}}{\pgfqpoint{2.477119in}{4.037050in}}{\pgfqpoint{2.477119in}{4.026000in}}%
\pgfpathcurveto{\pgfqpoint{2.477119in}{4.014950in}}{\pgfqpoint{2.481509in}{4.004351in}}{\pgfqpoint{2.489323in}{3.996537in}}%
\pgfpathcurveto{\pgfqpoint{2.497137in}{3.988724in}}{\pgfqpoint{2.507736in}{3.984333in}}{\pgfqpoint{2.518786in}{3.984333in}}%
\pgfpathclose%
\pgfusepath{stroke,fill}%
\end{pgfscope}%
\begin{pgfscope}%
\pgfpathrectangle{\pgfqpoint{0.800000in}{0.528000in}}{\pgfqpoint{4.960000in}{3.696000in}}%
\pgfusepath{clip}%
\pgfsetbuttcap%
\pgfsetroundjoin%
\definecolor{currentfill}{rgb}{0.000000,0.000000,0.000000}%
\pgfsetfillcolor{currentfill}%
\pgfsetlinewidth{1.003750pt}%
\definecolor{currentstroke}{rgb}{0.000000,0.000000,0.000000}%
\pgfsetstrokecolor{currentstroke}%
\pgfsetdash{}{0pt}%
\pgfpathmoveto{\pgfqpoint{2.518786in}{2.334266in}}%
\pgfpathcurveto{\pgfqpoint{2.529836in}{2.334266in}}{\pgfqpoint{2.540435in}{2.338657in}}{\pgfqpoint{2.548249in}{2.346470in}}%
\pgfpathcurveto{\pgfqpoint{2.556062in}{2.354284in}}{\pgfqpoint{2.560452in}{2.364883in}}{\pgfqpoint{2.560452in}{2.375933in}}%
\pgfpathcurveto{\pgfqpoint{2.560452in}{2.386983in}}{\pgfqpoint{2.556062in}{2.397582in}}{\pgfqpoint{2.548249in}{2.405396in}}%
\pgfpathcurveto{\pgfqpoint{2.540435in}{2.413209in}}{\pgfqpoint{2.529836in}{2.417600in}}{\pgfqpoint{2.518786in}{2.417600in}}%
\pgfpathcurveto{\pgfqpoint{2.507736in}{2.417600in}}{\pgfqpoint{2.497137in}{2.413209in}}{\pgfqpoint{2.489323in}{2.405396in}}%
\pgfpathcurveto{\pgfqpoint{2.481509in}{2.397582in}}{\pgfqpoint{2.477119in}{2.386983in}}{\pgfqpoint{2.477119in}{2.375933in}}%
\pgfpathcurveto{\pgfqpoint{2.477119in}{2.364883in}}{\pgfqpoint{2.481509in}{2.354284in}}{\pgfqpoint{2.489323in}{2.346470in}}%
\pgfpathcurveto{\pgfqpoint{2.497137in}{2.338657in}}{\pgfqpoint{2.507736in}{2.334266in}}{\pgfqpoint{2.518786in}{2.334266in}}%
\pgfpathclose%
\pgfusepath{stroke,fill}%
\end{pgfscope}%
\begin{pgfscope}%
\pgfpathrectangle{\pgfqpoint{0.800000in}{0.528000in}}{\pgfqpoint{4.960000in}{3.696000in}}%
\pgfusepath{clip}%
\pgfsetbuttcap%
\pgfsetroundjoin%
\definecolor{currentfill}{rgb}{0.000000,0.000000,0.000000}%
\pgfsetfillcolor{currentfill}%
\pgfsetlinewidth{1.003750pt}%
\definecolor{currentstroke}{rgb}{0.000000,0.000000,0.000000}%
\pgfsetstrokecolor{currentstroke}%
\pgfsetdash{}{0pt}%
\pgfpathmoveto{\pgfqpoint{2.518786in}{2.334266in}}%
\pgfpathcurveto{\pgfqpoint{2.529836in}{2.334266in}}{\pgfqpoint{2.540435in}{2.338657in}}{\pgfqpoint{2.548249in}{2.346470in}}%
\pgfpathcurveto{\pgfqpoint{2.556062in}{2.354284in}}{\pgfqpoint{2.560452in}{2.364883in}}{\pgfqpoint{2.560452in}{2.375933in}}%
\pgfpathcurveto{\pgfqpoint{2.560452in}{2.386983in}}{\pgfqpoint{2.556062in}{2.397582in}}{\pgfqpoint{2.548249in}{2.405396in}}%
\pgfpathcurveto{\pgfqpoint{2.540435in}{2.413209in}}{\pgfqpoint{2.529836in}{2.417600in}}{\pgfqpoint{2.518786in}{2.417600in}}%
\pgfpathcurveto{\pgfqpoint{2.507736in}{2.417600in}}{\pgfqpoint{2.497137in}{2.413209in}}{\pgfqpoint{2.489323in}{2.405396in}}%
\pgfpathcurveto{\pgfqpoint{2.481509in}{2.397582in}}{\pgfqpoint{2.477119in}{2.386983in}}{\pgfqpoint{2.477119in}{2.375933in}}%
\pgfpathcurveto{\pgfqpoint{2.477119in}{2.364883in}}{\pgfqpoint{2.481509in}{2.354284in}}{\pgfqpoint{2.489323in}{2.346470in}}%
\pgfpathcurveto{\pgfqpoint{2.497137in}{2.338657in}}{\pgfqpoint{2.507736in}{2.334266in}}{\pgfqpoint{2.518786in}{2.334266in}}%
\pgfpathclose%
\pgfusepath{stroke,fill}%
\end{pgfscope}%
\begin{pgfscope}%
\pgfpathrectangle{\pgfqpoint{0.800000in}{0.528000in}}{\pgfqpoint{4.960000in}{3.696000in}}%
\pgfusepath{clip}%
\pgfsetbuttcap%
\pgfsetroundjoin%
\definecolor{currentfill}{rgb}{0.000000,0.000000,0.000000}%
\pgfsetfillcolor{currentfill}%
\pgfsetlinewidth{1.003750pt}%
\definecolor{currentstroke}{rgb}{0.000000,0.000000,0.000000}%
\pgfsetstrokecolor{currentstroke}%
\pgfsetdash{}{0pt}%
\pgfpathmoveto{\pgfqpoint{2.518786in}{2.334266in}}%
\pgfpathcurveto{\pgfqpoint{2.529836in}{2.334266in}}{\pgfqpoint{2.540435in}{2.338657in}}{\pgfqpoint{2.548249in}{2.346470in}}%
\pgfpathcurveto{\pgfqpoint{2.556062in}{2.354284in}}{\pgfqpoint{2.560452in}{2.364883in}}{\pgfqpoint{2.560452in}{2.375933in}}%
\pgfpathcurveto{\pgfqpoint{2.560452in}{2.386983in}}{\pgfqpoint{2.556062in}{2.397582in}}{\pgfqpoint{2.548249in}{2.405396in}}%
\pgfpathcurveto{\pgfqpoint{2.540435in}{2.413209in}}{\pgfqpoint{2.529836in}{2.417600in}}{\pgfqpoint{2.518786in}{2.417600in}}%
\pgfpathcurveto{\pgfqpoint{2.507736in}{2.417600in}}{\pgfqpoint{2.497137in}{2.413209in}}{\pgfqpoint{2.489323in}{2.405396in}}%
\pgfpathcurveto{\pgfqpoint{2.481509in}{2.397582in}}{\pgfqpoint{2.477119in}{2.386983in}}{\pgfqpoint{2.477119in}{2.375933in}}%
\pgfpathcurveto{\pgfqpoint{2.477119in}{2.364883in}}{\pgfqpoint{2.481509in}{2.354284in}}{\pgfqpoint{2.489323in}{2.346470in}}%
\pgfpathcurveto{\pgfqpoint{2.497137in}{2.338657in}}{\pgfqpoint{2.507736in}{2.334266in}}{\pgfqpoint{2.518786in}{2.334266in}}%
\pgfpathclose%
\pgfusepath{stroke,fill}%
\end{pgfscope}%
\begin{pgfscope}%
\pgfpathrectangle{\pgfqpoint{0.800000in}{0.528000in}}{\pgfqpoint{4.960000in}{3.696000in}}%
\pgfusepath{clip}%
\pgfsetbuttcap%
\pgfsetroundjoin%
\definecolor{currentfill}{rgb}{0.000000,0.000000,0.000000}%
\pgfsetfillcolor{currentfill}%
\pgfsetlinewidth{1.003750pt}%
\definecolor{currentstroke}{rgb}{0.000000,0.000000,0.000000}%
\pgfsetstrokecolor{currentstroke}%
\pgfsetdash{}{0pt}%
\pgfpathmoveto{\pgfqpoint{2.518786in}{2.334266in}}%
\pgfpathcurveto{\pgfqpoint{2.529836in}{2.334266in}}{\pgfqpoint{2.540435in}{2.338657in}}{\pgfqpoint{2.548249in}{2.346470in}}%
\pgfpathcurveto{\pgfqpoint{2.556062in}{2.354284in}}{\pgfqpoint{2.560452in}{2.364883in}}{\pgfqpoint{2.560452in}{2.375933in}}%
\pgfpathcurveto{\pgfqpoint{2.560452in}{2.386983in}}{\pgfqpoint{2.556062in}{2.397582in}}{\pgfqpoint{2.548249in}{2.405396in}}%
\pgfpathcurveto{\pgfqpoint{2.540435in}{2.413209in}}{\pgfqpoint{2.529836in}{2.417600in}}{\pgfqpoint{2.518786in}{2.417600in}}%
\pgfpathcurveto{\pgfqpoint{2.507736in}{2.417600in}}{\pgfqpoint{2.497137in}{2.413209in}}{\pgfqpoint{2.489323in}{2.405396in}}%
\pgfpathcurveto{\pgfqpoint{2.481509in}{2.397582in}}{\pgfqpoint{2.477119in}{2.386983in}}{\pgfqpoint{2.477119in}{2.375933in}}%
\pgfpathcurveto{\pgfqpoint{2.477119in}{2.364883in}}{\pgfqpoint{2.481509in}{2.354284in}}{\pgfqpoint{2.489323in}{2.346470in}}%
\pgfpathcurveto{\pgfqpoint{2.497137in}{2.338657in}}{\pgfqpoint{2.507736in}{2.334266in}}{\pgfqpoint{2.518786in}{2.334266in}}%
\pgfpathclose%
\pgfusepath{stroke,fill}%
\end{pgfscope}%
\begin{pgfscope}%
\pgfpathrectangle{\pgfqpoint{0.800000in}{0.528000in}}{\pgfqpoint{4.960000in}{3.696000in}}%
\pgfusepath{clip}%
\pgfsetbuttcap%
\pgfsetroundjoin%
\definecolor{currentfill}{rgb}{0.000000,0.000000,0.000000}%
\pgfsetfillcolor{currentfill}%
\pgfsetlinewidth{1.003750pt}%
\definecolor{currentstroke}{rgb}{0.000000,0.000000,0.000000}%
\pgfsetstrokecolor{currentstroke}%
\pgfsetdash{}{0pt}%
\pgfpathmoveto{\pgfqpoint{2.518786in}{2.334266in}}%
\pgfpathcurveto{\pgfqpoint{2.529836in}{2.334266in}}{\pgfqpoint{2.540435in}{2.338657in}}{\pgfqpoint{2.548249in}{2.346470in}}%
\pgfpathcurveto{\pgfqpoint{2.556062in}{2.354284in}}{\pgfqpoint{2.560452in}{2.364883in}}{\pgfqpoint{2.560452in}{2.375933in}}%
\pgfpathcurveto{\pgfqpoint{2.560452in}{2.386983in}}{\pgfqpoint{2.556062in}{2.397582in}}{\pgfqpoint{2.548249in}{2.405396in}}%
\pgfpathcurveto{\pgfqpoint{2.540435in}{2.413209in}}{\pgfqpoint{2.529836in}{2.417600in}}{\pgfqpoint{2.518786in}{2.417600in}}%
\pgfpathcurveto{\pgfqpoint{2.507736in}{2.417600in}}{\pgfqpoint{2.497137in}{2.413209in}}{\pgfqpoint{2.489323in}{2.405396in}}%
\pgfpathcurveto{\pgfqpoint{2.481509in}{2.397582in}}{\pgfqpoint{2.477119in}{2.386983in}}{\pgfqpoint{2.477119in}{2.375933in}}%
\pgfpathcurveto{\pgfqpoint{2.477119in}{2.364883in}}{\pgfqpoint{2.481509in}{2.354284in}}{\pgfqpoint{2.489323in}{2.346470in}}%
\pgfpathcurveto{\pgfqpoint{2.497137in}{2.338657in}}{\pgfqpoint{2.507736in}{2.334266in}}{\pgfqpoint{2.518786in}{2.334266in}}%
\pgfpathclose%
\pgfusepath{stroke,fill}%
\end{pgfscope}%
\begin{pgfscope}%
\pgfpathrectangle{\pgfqpoint{0.800000in}{0.528000in}}{\pgfqpoint{4.960000in}{3.696000in}}%
\pgfusepath{clip}%
\pgfsetbuttcap%
\pgfsetroundjoin%
\definecolor{currentfill}{rgb}{0.000000,0.000000,0.000000}%
\pgfsetfillcolor{currentfill}%
\pgfsetlinewidth{1.003750pt}%
\definecolor{currentstroke}{rgb}{0.000000,0.000000,0.000000}%
\pgfsetstrokecolor{currentstroke}%
\pgfsetdash{}{0pt}%
\pgfpathmoveto{\pgfqpoint{2.518786in}{2.334266in}}%
\pgfpathcurveto{\pgfqpoint{2.529836in}{2.334266in}}{\pgfqpoint{2.540435in}{2.338657in}}{\pgfqpoint{2.548249in}{2.346470in}}%
\pgfpathcurveto{\pgfqpoint{2.556062in}{2.354284in}}{\pgfqpoint{2.560452in}{2.364883in}}{\pgfqpoint{2.560452in}{2.375933in}}%
\pgfpathcurveto{\pgfqpoint{2.560452in}{2.386983in}}{\pgfqpoint{2.556062in}{2.397582in}}{\pgfqpoint{2.548249in}{2.405396in}}%
\pgfpathcurveto{\pgfqpoint{2.540435in}{2.413209in}}{\pgfqpoint{2.529836in}{2.417600in}}{\pgfqpoint{2.518786in}{2.417600in}}%
\pgfpathcurveto{\pgfqpoint{2.507736in}{2.417600in}}{\pgfqpoint{2.497137in}{2.413209in}}{\pgfqpoint{2.489323in}{2.405396in}}%
\pgfpathcurveto{\pgfqpoint{2.481509in}{2.397582in}}{\pgfqpoint{2.477119in}{2.386983in}}{\pgfqpoint{2.477119in}{2.375933in}}%
\pgfpathcurveto{\pgfqpoint{2.477119in}{2.364883in}}{\pgfqpoint{2.481509in}{2.354284in}}{\pgfqpoint{2.489323in}{2.346470in}}%
\pgfpathcurveto{\pgfqpoint{2.497137in}{2.338657in}}{\pgfqpoint{2.507736in}{2.334266in}}{\pgfqpoint{2.518786in}{2.334266in}}%
\pgfpathclose%
\pgfusepath{stroke,fill}%
\end{pgfscope}%
\begin{pgfscope}%
\pgfpathrectangle{\pgfqpoint{0.800000in}{0.528000in}}{\pgfqpoint{4.960000in}{3.696000in}}%
\pgfusepath{clip}%
\pgfsetbuttcap%
\pgfsetroundjoin%
\definecolor{currentfill}{rgb}{0.000000,0.000000,0.000000}%
\pgfsetfillcolor{currentfill}%
\pgfsetlinewidth{1.003750pt}%
\definecolor{currentstroke}{rgb}{0.000000,0.000000,0.000000}%
\pgfsetstrokecolor{currentstroke}%
\pgfsetdash{}{0pt}%
\pgfpathmoveto{\pgfqpoint{2.518786in}{2.334266in}}%
\pgfpathcurveto{\pgfqpoint{2.529836in}{2.334266in}}{\pgfqpoint{2.540435in}{2.338657in}}{\pgfqpoint{2.548249in}{2.346470in}}%
\pgfpathcurveto{\pgfqpoint{2.556062in}{2.354284in}}{\pgfqpoint{2.560452in}{2.364883in}}{\pgfqpoint{2.560452in}{2.375933in}}%
\pgfpathcurveto{\pgfqpoint{2.560452in}{2.386983in}}{\pgfqpoint{2.556062in}{2.397582in}}{\pgfqpoint{2.548249in}{2.405396in}}%
\pgfpathcurveto{\pgfqpoint{2.540435in}{2.413209in}}{\pgfqpoint{2.529836in}{2.417600in}}{\pgfqpoint{2.518786in}{2.417600in}}%
\pgfpathcurveto{\pgfqpoint{2.507736in}{2.417600in}}{\pgfqpoint{2.497137in}{2.413209in}}{\pgfqpoint{2.489323in}{2.405396in}}%
\pgfpathcurveto{\pgfqpoint{2.481509in}{2.397582in}}{\pgfqpoint{2.477119in}{2.386983in}}{\pgfqpoint{2.477119in}{2.375933in}}%
\pgfpathcurveto{\pgfqpoint{2.477119in}{2.364883in}}{\pgfqpoint{2.481509in}{2.354284in}}{\pgfqpoint{2.489323in}{2.346470in}}%
\pgfpathcurveto{\pgfqpoint{2.497137in}{2.338657in}}{\pgfqpoint{2.507736in}{2.334266in}}{\pgfqpoint{2.518786in}{2.334266in}}%
\pgfpathclose%
\pgfusepath{stroke,fill}%
\end{pgfscope}%
\begin{pgfscope}%
\pgfpathrectangle{\pgfqpoint{0.800000in}{0.528000in}}{\pgfqpoint{4.960000in}{3.696000in}}%
\pgfusepath{clip}%
\pgfsetbuttcap%
\pgfsetroundjoin%
\definecolor{currentfill}{rgb}{0.000000,0.000000,0.000000}%
\pgfsetfillcolor{currentfill}%
\pgfsetlinewidth{1.003750pt}%
\definecolor{currentstroke}{rgb}{0.000000,0.000000,0.000000}%
\pgfsetstrokecolor{currentstroke}%
\pgfsetdash{}{0pt}%
\pgfpathmoveto{\pgfqpoint{2.518786in}{2.334266in}}%
\pgfpathcurveto{\pgfqpoint{2.529836in}{2.334266in}}{\pgfqpoint{2.540435in}{2.338657in}}{\pgfqpoint{2.548249in}{2.346470in}}%
\pgfpathcurveto{\pgfqpoint{2.556062in}{2.354284in}}{\pgfqpoint{2.560452in}{2.364883in}}{\pgfqpoint{2.560452in}{2.375933in}}%
\pgfpathcurveto{\pgfqpoint{2.560452in}{2.386983in}}{\pgfqpoint{2.556062in}{2.397582in}}{\pgfqpoint{2.548249in}{2.405396in}}%
\pgfpathcurveto{\pgfqpoint{2.540435in}{2.413209in}}{\pgfqpoint{2.529836in}{2.417600in}}{\pgfqpoint{2.518786in}{2.417600in}}%
\pgfpathcurveto{\pgfqpoint{2.507736in}{2.417600in}}{\pgfqpoint{2.497137in}{2.413209in}}{\pgfqpoint{2.489323in}{2.405396in}}%
\pgfpathcurveto{\pgfqpoint{2.481509in}{2.397582in}}{\pgfqpoint{2.477119in}{2.386983in}}{\pgfqpoint{2.477119in}{2.375933in}}%
\pgfpathcurveto{\pgfqpoint{2.477119in}{2.364883in}}{\pgfqpoint{2.481509in}{2.354284in}}{\pgfqpoint{2.489323in}{2.346470in}}%
\pgfpathcurveto{\pgfqpoint{2.497137in}{2.338657in}}{\pgfqpoint{2.507736in}{2.334266in}}{\pgfqpoint{2.518786in}{2.334266in}}%
\pgfpathclose%
\pgfusepath{stroke,fill}%
\end{pgfscope}%
\begin{pgfscope}%
\pgfpathrectangle{\pgfqpoint{0.800000in}{0.528000in}}{\pgfqpoint{4.960000in}{3.696000in}}%
\pgfusepath{clip}%
\pgfsetbuttcap%
\pgfsetroundjoin%
\definecolor{currentfill}{rgb}{0.000000,0.000000,0.000000}%
\pgfsetfillcolor{currentfill}%
\pgfsetlinewidth{1.003750pt}%
\definecolor{currentstroke}{rgb}{0.000000,0.000000,0.000000}%
\pgfsetstrokecolor{currentstroke}%
\pgfsetdash{}{0pt}%
\pgfpathmoveto{\pgfqpoint{2.518786in}{2.334266in}}%
\pgfpathcurveto{\pgfqpoint{2.529836in}{2.334266in}}{\pgfqpoint{2.540435in}{2.338657in}}{\pgfqpoint{2.548249in}{2.346470in}}%
\pgfpathcurveto{\pgfqpoint{2.556062in}{2.354284in}}{\pgfqpoint{2.560452in}{2.364883in}}{\pgfqpoint{2.560452in}{2.375933in}}%
\pgfpathcurveto{\pgfqpoint{2.560452in}{2.386983in}}{\pgfqpoint{2.556062in}{2.397582in}}{\pgfqpoint{2.548249in}{2.405396in}}%
\pgfpathcurveto{\pgfqpoint{2.540435in}{2.413209in}}{\pgfqpoint{2.529836in}{2.417600in}}{\pgfqpoint{2.518786in}{2.417600in}}%
\pgfpathcurveto{\pgfqpoint{2.507736in}{2.417600in}}{\pgfqpoint{2.497137in}{2.413209in}}{\pgfqpoint{2.489323in}{2.405396in}}%
\pgfpathcurveto{\pgfqpoint{2.481509in}{2.397582in}}{\pgfqpoint{2.477119in}{2.386983in}}{\pgfqpoint{2.477119in}{2.375933in}}%
\pgfpathcurveto{\pgfqpoint{2.477119in}{2.364883in}}{\pgfqpoint{2.481509in}{2.354284in}}{\pgfqpoint{2.489323in}{2.346470in}}%
\pgfpathcurveto{\pgfqpoint{2.497137in}{2.338657in}}{\pgfqpoint{2.507736in}{2.334266in}}{\pgfqpoint{2.518786in}{2.334266in}}%
\pgfpathclose%
\pgfusepath{stroke,fill}%
\end{pgfscope}%
\begin{pgfscope}%
\pgfpathrectangle{\pgfqpoint{0.800000in}{0.528000in}}{\pgfqpoint{4.960000in}{3.696000in}}%
\pgfusepath{clip}%
\pgfsetbuttcap%
\pgfsetroundjoin%
\definecolor{currentfill}{rgb}{0.000000,0.000000,0.000000}%
\pgfsetfillcolor{currentfill}%
\pgfsetlinewidth{1.003750pt}%
\definecolor{currentstroke}{rgb}{0.000000,0.000000,0.000000}%
\pgfsetstrokecolor{currentstroke}%
\pgfsetdash{}{0pt}%
\pgfpathmoveto{\pgfqpoint{2.518786in}{2.334266in}}%
\pgfpathcurveto{\pgfqpoint{2.529836in}{2.334266in}}{\pgfqpoint{2.540435in}{2.338657in}}{\pgfqpoint{2.548249in}{2.346470in}}%
\pgfpathcurveto{\pgfqpoint{2.556062in}{2.354284in}}{\pgfqpoint{2.560452in}{2.364883in}}{\pgfqpoint{2.560452in}{2.375933in}}%
\pgfpathcurveto{\pgfqpoint{2.560452in}{2.386983in}}{\pgfqpoint{2.556062in}{2.397582in}}{\pgfqpoint{2.548249in}{2.405396in}}%
\pgfpathcurveto{\pgfqpoint{2.540435in}{2.413209in}}{\pgfqpoint{2.529836in}{2.417600in}}{\pgfqpoint{2.518786in}{2.417600in}}%
\pgfpathcurveto{\pgfqpoint{2.507736in}{2.417600in}}{\pgfqpoint{2.497137in}{2.413209in}}{\pgfqpoint{2.489323in}{2.405396in}}%
\pgfpathcurveto{\pgfqpoint{2.481509in}{2.397582in}}{\pgfqpoint{2.477119in}{2.386983in}}{\pgfqpoint{2.477119in}{2.375933in}}%
\pgfpathcurveto{\pgfqpoint{2.477119in}{2.364883in}}{\pgfqpoint{2.481509in}{2.354284in}}{\pgfqpoint{2.489323in}{2.346470in}}%
\pgfpathcurveto{\pgfqpoint{2.497137in}{2.338657in}}{\pgfqpoint{2.507736in}{2.334266in}}{\pgfqpoint{2.518786in}{2.334266in}}%
\pgfpathclose%
\pgfusepath{stroke,fill}%
\end{pgfscope}%
\begin{pgfscope}%
\pgfpathrectangle{\pgfqpoint{0.800000in}{0.528000in}}{\pgfqpoint{4.960000in}{3.696000in}}%
\pgfusepath{clip}%
\pgfsetbuttcap%
\pgfsetroundjoin%
\definecolor{currentfill}{rgb}{0.000000,0.000000,0.000000}%
\pgfsetfillcolor{currentfill}%
\pgfsetlinewidth{1.003750pt}%
\definecolor{currentstroke}{rgb}{0.000000,0.000000,0.000000}%
\pgfsetstrokecolor{currentstroke}%
\pgfsetdash{}{0pt}%
\pgfpathmoveto{\pgfqpoint{2.518786in}{2.334266in}}%
\pgfpathcurveto{\pgfqpoint{2.529836in}{2.334266in}}{\pgfqpoint{2.540435in}{2.338657in}}{\pgfqpoint{2.548249in}{2.346470in}}%
\pgfpathcurveto{\pgfqpoint{2.556062in}{2.354284in}}{\pgfqpoint{2.560452in}{2.364883in}}{\pgfqpoint{2.560452in}{2.375933in}}%
\pgfpathcurveto{\pgfqpoint{2.560452in}{2.386983in}}{\pgfqpoint{2.556062in}{2.397582in}}{\pgfqpoint{2.548249in}{2.405396in}}%
\pgfpathcurveto{\pgfqpoint{2.540435in}{2.413209in}}{\pgfqpoint{2.529836in}{2.417600in}}{\pgfqpoint{2.518786in}{2.417600in}}%
\pgfpathcurveto{\pgfqpoint{2.507736in}{2.417600in}}{\pgfqpoint{2.497137in}{2.413209in}}{\pgfqpoint{2.489323in}{2.405396in}}%
\pgfpathcurveto{\pgfqpoint{2.481509in}{2.397582in}}{\pgfqpoint{2.477119in}{2.386983in}}{\pgfqpoint{2.477119in}{2.375933in}}%
\pgfpathcurveto{\pgfqpoint{2.477119in}{2.364883in}}{\pgfqpoint{2.481509in}{2.354284in}}{\pgfqpoint{2.489323in}{2.346470in}}%
\pgfpathcurveto{\pgfqpoint{2.497137in}{2.338657in}}{\pgfqpoint{2.507736in}{2.334266in}}{\pgfqpoint{2.518786in}{2.334266in}}%
\pgfpathclose%
\pgfusepath{stroke,fill}%
\end{pgfscope}%
\begin{pgfscope}%
\pgfpathrectangle{\pgfqpoint{0.800000in}{0.528000in}}{\pgfqpoint{4.960000in}{3.696000in}}%
\pgfusepath{clip}%
\pgfsetbuttcap%
\pgfsetroundjoin%
\definecolor{currentfill}{rgb}{0.000000,0.000000,0.000000}%
\pgfsetfillcolor{currentfill}%
\pgfsetlinewidth{1.003750pt}%
\definecolor{currentstroke}{rgb}{0.000000,0.000000,0.000000}%
\pgfsetstrokecolor{currentstroke}%
\pgfsetdash{}{0pt}%
\pgfpathmoveto{\pgfqpoint{2.518786in}{3.984333in}}%
\pgfpathcurveto{\pgfqpoint{2.529836in}{3.984333in}}{\pgfqpoint{2.540435in}{3.988724in}}{\pgfqpoint{2.548249in}{3.996537in}}%
\pgfpathcurveto{\pgfqpoint{2.556062in}{4.004351in}}{\pgfqpoint{2.560452in}{4.014950in}}{\pgfqpoint{2.560452in}{4.026000in}}%
\pgfpathcurveto{\pgfqpoint{2.560452in}{4.037050in}}{\pgfqpoint{2.556062in}{4.047649in}}{\pgfqpoint{2.548249in}{4.055463in}}%
\pgfpathcurveto{\pgfqpoint{2.540435in}{4.063276in}}{\pgfqpoint{2.529836in}{4.067667in}}{\pgfqpoint{2.518786in}{4.067667in}}%
\pgfpathcurveto{\pgfqpoint{2.507736in}{4.067667in}}{\pgfqpoint{2.497137in}{4.063276in}}{\pgfqpoint{2.489323in}{4.055463in}}%
\pgfpathcurveto{\pgfqpoint{2.481509in}{4.047649in}}{\pgfqpoint{2.477119in}{4.037050in}}{\pgfqpoint{2.477119in}{4.026000in}}%
\pgfpathcurveto{\pgfqpoint{2.477119in}{4.014950in}}{\pgfqpoint{2.481509in}{4.004351in}}{\pgfqpoint{2.489323in}{3.996537in}}%
\pgfpathcurveto{\pgfqpoint{2.497137in}{3.988724in}}{\pgfqpoint{2.507736in}{3.984333in}}{\pgfqpoint{2.518786in}{3.984333in}}%
\pgfpathclose%
\pgfusepath{stroke,fill}%
\end{pgfscope}%
\begin{pgfscope}%
\pgfpathrectangle{\pgfqpoint{0.800000in}{0.528000in}}{\pgfqpoint{4.960000in}{3.696000in}}%
\pgfusepath{clip}%
\pgfsetbuttcap%
\pgfsetroundjoin%
\definecolor{currentfill}{rgb}{0.000000,0.000000,0.000000}%
\pgfsetfillcolor{currentfill}%
\pgfsetlinewidth{1.003750pt}%
\definecolor{currentstroke}{rgb}{0.000000,0.000000,0.000000}%
\pgfsetstrokecolor{currentstroke}%
\pgfsetdash{}{0pt}%
\pgfpathmoveto{\pgfqpoint{2.518786in}{2.334266in}}%
\pgfpathcurveto{\pgfqpoint{2.529836in}{2.334266in}}{\pgfqpoint{2.540435in}{2.338657in}}{\pgfqpoint{2.548249in}{2.346470in}}%
\pgfpathcurveto{\pgfqpoint{2.556062in}{2.354284in}}{\pgfqpoint{2.560452in}{2.364883in}}{\pgfqpoint{2.560452in}{2.375933in}}%
\pgfpathcurveto{\pgfqpoint{2.560452in}{2.386983in}}{\pgfqpoint{2.556062in}{2.397582in}}{\pgfqpoint{2.548249in}{2.405396in}}%
\pgfpathcurveto{\pgfqpoint{2.540435in}{2.413209in}}{\pgfqpoint{2.529836in}{2.417600in}}{\pgfqpoint{2.518786in}{2.417600in}}%
\pgfpathcurveto{\pgfqpoint{2.507736in}{2.417600in}}{\pgfqpoint{2.497137in}{2.413209in}}{\pgfqpoint{2.489323in}{2.405396in}}%
\pgfpathcurveto{\pgfqpoint{2.481509in}{2.397582in}}{\pgfqpoint{2.477119in}{2.386983in}}{\pgfqpoint{2.477119in}{2.375933in}}%
\pgfpathcurveto{\pgfqpoint{2.477119in}{2.364883in}}{\pgfqpoint{2.481509in}{2.354284in}}{\pgfqpoint{2.489323in}{2.346470in}}%
\pgfpathcurveto{\pgfqpoint{2.497137in}{2.338657in}}{\pgfqpoint{2.507736in}{2.334266in}}{\pgfqpoint{2.518786in}{2.334266in}}%
\pgfpathclose%
\pgfusepath{stroke,fill}%
\end{pgfscope}%
\begin{pgfscope}%
\pgfpathrectangle{\pgfqpoint{0.800000in}{0.528000in}}{\pgfqpoint{4.960000in}{3.696000in}}%
\pgfusepath{clip}%
\pgfsetbuttcap%
\pgfsetroundjoin%
\definecolor{currentfill}{rgb}{0.000000,0.000000,0.000000}%
\pgfsetfillcolor{currentfill}%
\pgfsetlinewidth{1.003750pt}%
\definecolor{currentstroke}{rgb}{0.000000,0.000000,0.000000}%
\pgfsetstrokecolor{currentstroke}%
\pgfsetdash{}{0pt}%
\pgfpathmoveto{\pgfqpoint{2.518786in}{2.334266in}}%
\pgfpathcurveto{\pgfqpoint{2.529836in}{2.334266in}}{\pgfqpoint{2.540435in}{2.338657in}}{\pgfqpoint{2.548249in}{2.346470in}}%
\pgfpathcurveto{\pgfqpoint{2.556062in}{2.354284in}}{\pgfqpoint{2.560452in}{2.364883in}}{\pgfqpoint{2.560452in}{2.375933in}}%
\pgfpathcurveto{\pgfqpoint{2.560452in}{2.386983in}}{\pgfqpoint{2.556062in}{2.397582in}}{\pgfqpoint{2.548249in}{2.405396in}}%
\pgfpathcurveto{\pgfqpoint{2.540435in}{2.413209in}}{\pgfqpoint{2.529836in}{2.417600in}}{\pgfqpoint{2.518786in}{2.417600in}}%
\pgfpathcurveto{\pgfqpoint{2.507736in}{2.417600in}}{\pgfqpoint{2.497137in}{2.413209in}}{\pgfqpoint{2.489323in}{2.405396in}}%
\pgfpathcurveto{\pgfqpoint{2.481509in}{2.397582in}}{\pgfqpoint{2.477119in}{2.386983in}}{\pgfqpoint{2.477119in}{2.375933in}}%
\pgfpathcurveto{\pgfqpoint{2.477119in}{2.364883in}}{\pgfqpoint{2.481509in}{2.354284in}}{\pgfqpoint{2.489323in}{2.346470in}}%
\pgfpathcurveto{\pgfqpoint{2.497137in}{2.338657in}}{\pgfqpoint{2.507736in}{2.334266in}}{\pgfqpoint{2.518786in}{2.334266in}}%
\pgfpathclose%
\pgfusepath{stroke,fill}%
\end{pgfscope}%
\begin{pgfscope}%
\pgfpathrectangle{\pgfqpoint{0.800000in}{0.528000in}}{\pgfqpoint{4.960000in}{3.696000in}}%
\pgfusepath{clip}%
\pgfsetbuttcap%
\pgfsetroundjoin%
\definecolor{currentfill}{rgb}{0.000000,0.000000,0.000000}%
\pgfsetfillcolor{currentfill}%
\pgfsetlinewidth{1.003750pt}%
\definecolor{currentstroke}{rgb}{0.000000,0.000000,0.000000}%
\pgfsetstrokecolor{currentstroke}%
\pgfsetdash{}{0pt}%
\pgfpathmoveto{\pgfqpoint{2.518786in}{2.334266in}}%
\pgfpathcurveto{\pgfqpoint{2.529836in}{2.334266in}}{\pgfqpoint{2.540435in}{2.338657in}}{\pgfqpoint{2.548249in}{2.346470in}}%
\pgfpathcurveto{\pgfqpoint{2.556062in}{2.354284in}}{\pgfqpoint{2.560452in}{2.364883in}}{\pgfqpoint{2.560452in}{2.375933in}}%
\pgfpathcurveto{\pgfqpoint{2.560452in}{2.386983in}}{\pgfqpoint{2.556062in}{2.397582in}}{\pgfqpoint{2.548249in}{2.405396in}}%
\pgfpathcurveto{\pgfqpoint{2.540435in}{2.413209in}}{\pgfqpoint{2.529836in}{2.417600in}}{\pgfqpoint{2.518786in}{2.417600in}}%
\pgfpathcurveto{\pgfqpoint{2.507736in}{2.417600in}}{\pgfqpoint{2.497137in}{2.413209in}}{\pgfqpoint{2.489323in}{2.405396in}}%
\pgfpathcurveto{\pgfqpoint{2.481509in}{2.397582in}}{\pgfqpoint{2.477119in}{2.386983in}}{\pgfqpoint{2.477119in}{2.375933in}}%
\pgfpathcurveto{\pgfqpoint{2.477119in}{2.364883in}}{\pgfqpoint{2.481509in}{2.354284in}}{\pgfqpoint{2.489323in}{2.346470in}}%
\pgfpathcurveto{\pgfqpoint{2.497137in}{2.338657in}}{\pgfqpoint{2.507736in}{2.334266in}}{\pgfqpoint{2.518786in}{2.334266in}}%
\pgfpathclose%
\pgfusepath{stroke,fill}%
\end{pgfscope}%
\begin{pgfscope}%
\pgfpathrectangle{\pgfqpoint{0.800000in}{0.528000in}}{\pgfqpoint{4.960000in}{3.696000in}}%
\pgfusepath{clip}%
\pgfsetbuttcap%
\pgfsetroundjoin%
\definecolor{currentfill}{rgb}{0.000000,0.000000,0.000000}%
\pgfsetfillcolor{currentfill}%
\pgfsetlinewidth{1.003750pt}%
\definecolor{currentstroke}{rgb}{0.000000,0.000000,0.000000}%
\pgfsetstrokecolor{currentstroke}%
\pgfsetdash{}{0pt}%
\pgfpathmoveto{\pgfqpoint{2.518786in}{2.334266in}}%
\pgfpathcurveto{\pgfqpoint{2.529836in}{2.334266in}}{\pgfqpoint{2.540435in}{2.338657in}}{\pgfqpoint{2.548249in}{2.346470in}}%
\pgfpathcurveto{\pgfqpoint{2.556062in}{2.354284in}}{\pgfqpoint{2.560452in}{2.364883in}}{\pgfqpoint{2.560452in}{2.375933in}}%
\pgfpathcurveto{\pgfqpoint{2.560452in}{2.386983in}}{\pgfqpoint{2.556062in}{2.397582in}}{\pgfqpoint{2.548249in}{2.405396in}}%
\pgfpathcurveto{\pgfqpoint{2.540435in}{2.413209in}}{\pgfqpoint{2.529836in}{2.417600in}}{\pgfqpoint{2.518786in}{2.417600in}}%
\pgfpathcurveto{\pgfqpoint{2.507736in}{2.417600in}}{\pgfqpoint{2.497137in}{2.413209in}}{\pgfqpoint{2.489323in}{2.405396in}}%
\pgfpathcurveto{\pgfqpoint{2.481509in}{2.397582in}}{\pgfqpoint{2.477119in}{2.386983in}}{\pgfqpoint{2.477119in}{2.375933in}}%
\pgfpathcurveto{\pgfqpoint{2.477119in}{2.364883in}}{\pgfqpoint{2.481509in}{2.354284in}}{\pgfqpoint{2.489323in}{2.346470in}}%
\pgfpathcurveto{\pgfqpoint{2.497137in}{2.338657in}}{\pgfqpoint{2.507736in}{2.334266in}}{\pgfqpoint{2.518786in}{2.334266in}}%
\pgfpathclose%
\pgfusepath{stroke,fill}%
\end{pgfscope}%
\begin{pgfscope}%
\pgfpathrectangle{\pgfqpoint{0.800000in}{0.528000in}}{\pgfqpoint{4.960000in}{3.696000in}}%
\pgfusepath{clip}%
\pgfsetbuttcap%
\pgfsetroundjoin%
\definecolor{currentfill}{rgb}{0.000000,0.000000,0.000000}%
\pgfsetfillcolor{currentfill}%
\pgfsetlinewidth{1.003750pt}%
\definecolor{currentstroke}{rgb}{0.000000,0.000000,0.000000}%
\pgfsetstrokecolor{currentstroke}%
\pgfsetdash{}{0pt}%
\pgfpathmoveto{\pgfqpoint{2.518786in}{2.334266in}}%
\pgfpathcurveto{\pgfqpoint{2.529836in}{2.334266in}}{\pgfqpoint{2.540435in}{2.338657in}}{\pgfqpoint{2.548249in}{2.346470in}}%
\pgfpathcurveto{\pgfqpoint{2.556062in}{2.354284in}}{\pgfqpoint{2.560452in}{2.364883in}}{\pgfqpoint{2.560452in}{2.375933in}}%
\pgfpathcurveto{\pgfqpoint{2.560452in}{2.386983in}}{\pgfqpoint{2.556062in}{2.397582in}}{\pgfqpoint{2.548249in}{2.405396in}}%
\pgfpathcurveto{\pgfqpoint{2.540435in}{2.413209in}}{\pgfqpoint{2.529836in}{2.417600in}}{\pgfqpoint{2.518786in}{2.417600in}}%
\pgfpathcurveto{\pgfqpoint{2.507736in}{2.417600in}}{\pgfqpoint{2.497137in}{2.413209in}}{\pgfqpoint{2.489323in}{2.405396in}}%
\pgfpathcurveto{\pgfqpoint{2.481509in}{2.397582in}}{\pgfqpoint{2.477119in}{2.386983in}}{\pgfqpoint{2.477119in}{2.375933in}}%
\pgfpathcurveto{\pgfqpoint{2.477119in}{2.364883in}}{\pgfqpoint{2.481509in}{2.354284in}}{\pgfqpoint{2.489323in}{2.346470in}}%
\pgfpathcurveto{\pgfqpoint{2.497137in}{2.338657in}}{\pgfqpoint{2.507736in}{2.334266in}}{\pgfqpoint{2.518786in}{2.334266in}}%
\pgfpathclose%
\pgfusepath{stroke,fill}%
\end{pgfscope}%
\begin{pgfscope}%
\pgfpathrectangle{\pgfqpoint{0.800000in}{0.528000in}}{\pgfqpoint{4.960000in}{3.696000in}}%
\pgfusepath{clip}%
\pgfsetbuttcap%
\pgfsetroundjoin%
\definecolor{currentfill}{rgb}{0.000000,0.000000,0.000000}%
\pgfsetfillcolor{currentfill}%
\pgfsetlinewidth{1.003750pt}%
\definecolor{currentstroke}{rgb}{0.000000,0.000000,0.000000}%
\pgfsetstrokecolor{currentstroke}%
\pgfsetdash{}{0pt}%
\pgfpathmoveto{\pgfqpoint{2.518786in}{3.984333in}}%
\pgfpathcurveto{\pgfqpoint{2.529836in}{3.984333in}}{\pgfqpoint{2.540435in}{3.988724in}}{\pgfqpoint{2.548249in}{3.996537in}}%
\pgfpathcurveto{\pgfqpoint{2.556062in}{4.004351in}}{\pgfqpoint{2.560452in}{4.014950in}}{\pgfqpoint{2.560452in}{4.026000in}}%
\pgfpathcurveto{\pgfqpoint{2.560452in}{4.037050in}}{\pgfqpoint{2.556062in}{4.047649in}}{\pgfqpoint{2.548249in}{4.055463in}}%
\pgfpathcurveto{\pgfqpoint{2.540435in}{4.063276in}}{\pgfqpoint{2.529836in}{4.067667in}}{\pgfqpoint{2.518786in}{4.067667in}}%
\pgfpathcurveto{\pgfqpoint{2.507736in}{4.067667in}}{\pgfqpoint{2.497137in}{4.063276in}}{\pgfqpoint{2.489323in}{4.055463in}}%
\pgfpathcurveto{\pgfqpoint{2.481509in}{4.047649in}}{\pgfqpoint{2.477119in}{4.037050in}}{\pgfqpoint{2.477119in}{4.026000in}}%
\pgfpathcurveto{\pgfqpoint{2.477119in}{4.014950in}}{\pgfqpoint{2.481509in}{4.004351in}}{\pgfqpoint{2.489323in}{3.996537in}}%
\pgfpathcurveto{\pgfqpoint{2.497137in}{3.988724in}}{\pgfqpoint{2.507736in}{3.984333in}}{\pgfqpoint{2.518786in}{3.984333in}}%
\pgfpathclose%
\pgfusepath{stroke,fill}%
\end{pgfscope}%
\begin{pgfscope}%
\pgfpathrectangle{\pgfqpoint{0.800000in}{0.528000in}}{\pgfqpoint{4.960000in}{3.696000in}}%
\pgfusepath{clip}%
\pgfsetbuttcap%
\pgfsetroundjoin%
\definecolor{currentfill}{rgb}{0.000000,0.000000,0.000000}%
\pgfsetfillcolor{currentfill}%
\pgfsetlinewidth{1.003750pt}%
\definecolor{currentstroke}{rgb}{0.000000,0.000000,0.000000}%
\pgfsetstrokecolor{currentstroke}%
\pgfsetdash{}{0pt}%
\pgfpathmoveto{\pgfqpoint{2.518786in}{2.334266in}}%
\pgfpathcurveto{\pgfqpoint{2.529836in}{2.334266in}}{\pgfqpoint{2.540435in}{2.338657in}}{\pgfqpoint{2.548249in}{2.346470in}}%
\pgfpathcurveto{\pgfqpoint{2.556062in}{2.354284in}}{\pgfqpoint{2.560452in}{2.364883in}}{\pgfqpoint{2.560452in}{2.375933in}}%
\pgfpathcurveto{\pgfqpoint{2.560452in}{2.386983in}}{\pgfqpoint{2.556062in}{2.397582in}}{\pgfqpoint{2.548249in}{2.405396in}}%
\pgfpathcurveto{\pgfqpoint{2.540435in}{2.413209in}}{\pgfqpoint{2.529836in}{2.417600in}}{\pgfqpoint{2.518786in}{2.417600in}}%
\pgfpathcurveto{\pgfqpoint{2.507736in}{2.417600in}}{\pgfqpoint{2.497137in}{2.413209in}}{\pgfqpoint{2.489323in}{2.405396in}}%
\pgfpathcurveto{\pgfqpoint{2.481509in}{2.397582in}}{\pgfqpoint{2.477119in}{2.386983in}}{\pgfqpoint{2.477119in}{2.375933in}}%
\pgfpathcurveto{\pgfqpoint{2.477119in}{2.364883in}}{\pgfqpoint{2.481509in}{2.354284in}}{\pgfqpoint{2.489323in}{2.346470in}}%
\pgfpathcurveto{\pgfqpoint{2.497137in}{2.338657in}}{\pgfqpoint{2.507736in}{2.334266in}}{\pgfqpoint{2.518786in}{2.334266in}}%
\pgfpathclose%
\pgfusepath{stroke,fill}%
\end{pgfscope}%
\begin{pgfscope}%
\pgfpathrectangle{\pgfqpoint{0.800000in}{0.528000in}}{\pgfqpoint{4.960000in}{3.696000in}}%
\pgfusepath{clip}%
\pgfsetbuttcap%
\pgfsetroundjoin%
\definecolor{currentfill}{rgb}{0.000000,0.000000,0.000000}%
\pgfsetfillcolor{currentfill}%
\pgfsetlinewidth{1.003750pt}%
\definecolor{currentstroke}{rgb}{0.000000,0.000000,0.000000}%
\pgfsetstrokecolor{currentstroke}%
\pgfsetdash{}{0pt}%
\pgfpathmoveto{\pgfqpoint{2.518786in}{2.334266in}}%
\pgfpathcurveto{\pgfqpoint{2.529836in}{2.334266in}}{\pgfqpoint{2.540435in}{2.338657in}}{\pgfqpoint{2.548249in}{2.346470in}}%
\pgfpathcurveto{\pgfqpoint{2.556062in}{2.354284in}}{\pgfqpoint{2.560452in}{2.364883in}}{\pgfqpoint{2.560452in}{2.375933in}}%
\pgfpathcurveto{\pgfqpoint{2.560452in}{2.386983in}}{\pgfqpoint{2.556062in}{2.397582in}}{\pgfqpoint{2.548249in}{2.405396in}}%
\pgfpathcurveto{\pgfqpoint{2.540435in}{2.413209in}}{\pgfqpoint{2.529836in}{2.417600in}}{\pgfqpoint{2.518786in}{2.417600in}}%
\pgfpathcurveto{\pgfqpoint{2.507736in}{2.417600in}}{\pgfqpoint{2.497137in}{2.413209in}}{\pgfqpoint{2.489323in}{2.405396in}}%
\pgfpathcurveto{\pgfqpoint{2.481509in}{2.397582in}}{\pgfqpoint{2.477119in}{2.386983in}}{\pgfqpoint{2.477119in}{2.375933in}}%
\pgfpathcurveto{\pgfqpoint{2.477119in}{2.364883in}}{\pgfqpoint{2.481509in}{2.354284in}}{\pgfqpoint{2.489323in}{2.346470in}}%
\pgfpathcurveto{\pgfqpoint{2.497137in}{2.338657in}}{\pgfqpoint{2.507736in}{2.334266in}}{\pgfqpoint{2.518786in}{2.334266in}}%
\pgfpathclose%
\pgfusepath{stroke,fill}%
\end{pgfscope}%
\begin{pgfscope}%
\pgfpathrectangle{\pgfqpoint{0.800000in}{0.528000in}}{\pgfqpoint{4.960000in}{3.696000in}}%
\pgfusepath{clip}%
\pgfsetbuttcap%
\pgfsetroundjoin%
\definecolor{currentfill}{rgb}{0.000000,0.000000,0.000000}%
\pgfsetfillcolor{currentfill}%
\pgfsetlinewidth{1.003750pt}%
\definecolor{currentstroke}{rgb}{0.000000,0.000000,0.000000}%
\pgfsetstrokecolor{currentstroke}%
\pgfsetdash{}{0pt}%
\pgfpathmoveto{\pgfqpoint{2.518786in}{2.334266in}}%
\pgfpathcurveto{\pgfqpoint{2.529836in}{2.334266in}}{\pgfqpoint{2.540435in}{2.338657in}}{\pgfqpoint{2.548249in}{2.346470in}}%
\pgfpathcurveto{\pgfqpoint{2.556062in}{2.354284in}}{\pgfqpoint{2.560452in}{2.364883in}}{\pgfqpoint{2.560452in}{2.375933in}}%
\pgfpathcurveto{\pgfqpoint{2.560452in}{2.386983in}}{\pgfqpoint{2.556062in}{2.397582in}}{\pgfqpoint{2.548249in}{2.405396in}}%
\pgfpathcurveto{\pgfqpoint{2.540435in}{2.413209in}}{\pgfqpoint{2.529836in}{2.417600in}}{\pgfqpoint{2.518786in}{2.417600in}}%
\pgfpathcurveto{\pgfqpoint{2.507736in}{2.417600in}}{\pgfqpoint{2.497137in}{2.413209in}}{\pgfqpoint{2.489323in}{2.405396in}}%
\pgfpathcurveto{\pgfqpoint{2.481509in}{2.397582in}}{\pgfqpoint{2.477119in}{2.386983in}}{\pgfqpoint{2.477119in}{2.375933in}}%
\pgfpathcurveto{\pgfqpoint{2.477119in}{2.364883in}}{\pgfqpoint{2.481509in}{2.354284in}}{\pgfqpoint{2.489323in}{2.346470in}}%
\pgfpathcurveto{\pgfqpoint{2.497137in}{2.338657in}}{\pgfqpoint{2.507736in}{2.334266in}}{\pgfqpoint{2.518786in}{2.334266in}}%
\pgfpathclose%
\pgfusepath{stroke,fill}%
\end{pgfscope}%
\begin{pgfscope}%
\pgfpathrectangle{\pgfqpoint{0.800000in}{0.528000in}}{\pgfqpoint{4.960000in}{3.696000in}}%
\pgfusepath{clip}%
\pgfsetbuttcap%
\pgfsetroundjoin%
\definecolor{currentfill}{rgb}{0.000000,0.000000,0.000000}%
\pgfsetfillcolor{currentfill}%
\pgfsetlinewidth{1.003750pt}%
\definecolor{currentstroke}{rgb}{0.000000,0.000000,0.000000}%
\pgfsetstrokecolor{currentstroke}%
\pgfsetdash{}{0pt}%
\pgfpathmoveto{\pgfqpoint{2.518786in}{2.334266in}}%
\pgfpathcurveto{\pgfqpoint{2.529836in}{2.334266in}}{\pgfqpoint{2.540435in}{2.338657in}}{\pgfqpoint{2.548249in}{2.346470in}}%
\pgfpathcurveto{\pgfqpoint{2.556062in}{2.354284in}}{\pgfqpoint{2.560452in}{2.364883in}}{\pgfqpoint{2.560452in}{2.375933in}}%
\pgfpathcurveto{\pgfqpoint{2.560452in}{2.386983in}}{\pgfqpoint{2.556062in}{2.397582in}}{\pgfqpoint{2.548249in}{2.405396in}}%
\pgfpathcurveto{\pgfqpoint{2.540435in}{2.413209in}}{\pgfqpoint{2.529836in}{2.417600in}}{\pgfqpoint{2.518786in}{2.417600in}}%
\pgfpathcurveto{\pgfqpoint{2.507736in}{2.417600in}}{\pgfqpoint{2.497137in}{2.413209in}}{\pgfqpoint{2.489323in}{2.405396in}}%
\pgfpathcurveto{\pgfqpoint{2.481509in}{2.397582in}}{\pgfqpoint{2.477119in}{2.386983in}}{\pgfqpoint{2.477119in}{2.375933in}}%
\pgfpathcurveto{\pgfqpoint{2.477119in}{2.364883in}}{\pgfqpoint{2.481509in}{2.354284in}}{\pgfqpoint{2.489323in}{2.346470in}}%
\pgfpathcurveto{\pgfqpoint{2.497137in}{2.338657in}}{\pgfqpoint{2.507736in}{2.334266in}}{\pgfqpoint{2.518786in}{2.334266in}}%
\pgfpathclose%
\pgfusepath{stroke,fill}%
\end{pgfscope}%
\begin{pgfscope}%
\pgfpathrectangle{\pgfqpoint{0.800000in}{0.528000in}}{\pgfqpoint{4.960000in}{3.696000in}}%
\pgfusepath{clip}%
\pgfsetbuttcap%
\pgfsetroundjoin%
\definecolor{currentfill}{rgb}{0.000000,0.000000,0.000000}%
\pgfsetfillcolor{currentfill}%
\pgfsetlinewidth{1.003750pt}%
\definecolor{currentstroke}{rgb}{0.000000,0.000000,0.000000}%
\pgfsetstrokecolor{currentstroke}%
\pgfsetdash{}{0pt}%
\pgfpathmoveto{\pgfqpoint{2.518786in}{2.334266in}}%
\pgfpathcurveto{\pgfqpoint{2.529836in}{2.334266in}}{\pgfqpoint{2.540435in}{2.338657in}}{\pgfqpoint{2.548249in}{2.346470in}}%
\pgfpathcurveto{\pgfqpoint{2.556062in}{2.354284in}}{\pgfqpoint{2.560452in}{2.364883in}}{\pgfqpoint{2.560452in}{2.375933in}}%
\pgfpathcurveto{\pgfqpoint{2.560452in}{2.386983in}}{\pgfqpoint{2.556062in}{2.397582in}}{\pgfqpoint{2.548249in}{2.405396in}}%
\pgfpathcurveto{\pgfqpoint{2.540435in}{2.413209in}}{\pgfqpoint{2.529836in}{2.417600in}}{\pgfqpoint{2.518786in}{2.417600in}}%
\pgfpathcurveto{\pgfqpoint{2.507736in}{2.417600in}}{\pgfqpoint{2.497137in}{2.413209in}}{\pgfqpoint{2.489323in}{2.405396in}}%
\pgfpathcurveto{\pgfqpoint{2.481509in}{2.397582in}}{\pgfqpoint{2.477119in}{2.386983in}}{\pgfqpoint{2.477119in}{2.375933in}}%
\pgfpathcurveto{\pgfqpoint{2.477119in}{2.364883in}}{\pgfqpoint{2.481509in}{2.354284in}}{\pgfqpoint{2.489323in}{2.346470in}}%
\pgfpathcurveto{\pgfqpoint{2.497137in}{2.338657in}}{\pgfqpoint{2.507736in}{2.334266in}}{\pgfqpoint{2.518786in}{2.334266in}}%
\pgfpathclose%
\pgfusepath{stroke,fill}%
\end{pgfscope}%
\begin{pgfscope}%
\pgfpathrectangle{\pgfqpoint{0.800000in}{0.528000in}}{\pgfqpoint{4.960000in}{3.696000in}}%
\pgfusepath{clip}%
\pgfsetbuttcap%
\pgfsetroundjoin%
\definecolor{currentfill}{rgb}{0.000000,0.000000,0.000000}%
\pgfsetfillcolor{currentfill}%
\pgfsetlinewidth{1.003750pt}%
\definecolor{currentstroke}{rgb}{0.000000,0.000000,0.000000}%
\pgfsetstrokecolor{currentstroke}%
\pgfsetdash{}{0pt}%
\pgfpathmoveto{\pgfqpoint{2.518786in}{3.984333in}}%
\pgfpathcurveto{\pgfqpoint{2.529836in}{3.984333in}}{\pgfqpoint{2.540435in}{3.988724in}}{\pgfqpoint{2.548249in}{3.996537in}}%
\pgfpathcurveto{\pgfqpoint{2.556062in}{4.004351in}}{\pgfqpoint{2.560452in}{4.014950in}}{\pgfqpoint{2.560452in}{4.026000in}}%
\pgfpathcurveto{\pgfqpoint{2.560452in}{4.037050in}}{\pgfqpoint{2.556062in}{4.047649in}}{\pgfqpoint{2.548249in}{4.055463in}}%
\pgfpathcurveto{\pgfqpoint{2.540435in}{4.063276in}}{\pgfqpoint{2.529836in}{4.067667in}}{\pgfqpoint{2.518786in}{4.067667in}}%
\pgfpathcurveto{\pgfqpoint{2.507736in}{4.067667in}}{\pgfqpoint{2.497137in}{4.063276in}}{\pgfqpoint{2.489323in}{4.055463in}}%
\pgfpathcurveto{\pgfqpoint{2.481509in}{4.047649in}}{\pgfqpoint{2.477119in}{4.037050in}}{\pgfqpoint{2.477119in}{4.026000in}}%
\pgfpathcurveto{\pgfqpoint{2.477119in}{4.014950in}}{\pgfqpoint{2.481509in}{4.004351in}}{\pgfqpoint{2.489323in}{3.996537in}}%
\pgfpathcurveto{\pgfqpoint{2.497137in}{3.988724in}}{\pgfqpoint{2.507736in}{3.984333in}}{\pgfqpoint{2.518786in}{3.984333in}}%
\pgfpathclose%
\pgfusepath{stroke,fill}%
\end{pgfscope}%
\begin{pgfscope}%
\pgfpathrectangle{\pgfqpoint{0.800000in}{0.528000in}}{\pgfqpoint{4.960000in}{3.696000in}}%
\pgfusepath{clip}%
\pgfsetbuttcap%
\pgfsetroundjoin%
\definecolor{currentfill}{rgb}{0.000000,0.000000,0.000000}%
\pgfsetfillcolor{currentfill}%
\pgfsetlinewidth{1.003750pt}%
\definecolor{currentstroke}{rgb}{0.000000,0.000000,0.000000}%
\pgfsetstrokecolor{currentstroke}%
\pgfsetdash{}{0pt}%
\pgfpathmoveto{\pgfqpoint{2.518786in}{2.334266in}}%
\pgfpathcurveto{\pgfqpoint{2.529836in}{2.334266in}}{\pgfqpoint{2.540435in}{2.338657in}}{\pgfqpoint{2.548249in}{2.346470in}}%
\pgfpathcurveto{\pgfqpoint{2.556062in}{2.354284in}}{\pgfqpoint{2.560452in}{2.364883in}}{\pgfqpoint{2.560452in}{2.375933in}}%
\pgfpathcurveto{\pgfqpoint{2.560452in}{2.386983in}}{\pgfqpoint{2.556062in}{2.397582in}}{\pgfqpoint{2.548249in}{2.405396in}}%
\pgfpathcurveto{\pgfqpoint{2.540435in}{2.413209in}}{\pgfqpoint{2.529836in}{2.417600in}}{\pgfqpoint{2.518786in}{2.417600in}}%
\pgfpathcurveto{\pgfqpoint{2.507736in}{2.417600in}}{\pgfqpoint{2.497137in}{2.413209in}}{\pgfqpoint{2.489323in}{2.405396in}}%
\pgfpathcurveto{\pgfqpoint{2.481509in}{2.397582in}}{\pgfqpoint{2.477119in}{2.386983in}}{\pgfqpoint{2.477119in}{2.375933in}}%
\pgfpathcurveto{\pgfqpoint{2.477119in}{2.364883in}}{\pgfqpoint{2.481509in}{2.354284in}}{\pgfqpoint{2.489323in}{2.346470in}}%
\pgfpathcurveto{\pgfqpoint{2.497137in}{2.338657in}}{\pgfqpoint{2.507736in}{2.334266in}}{\pgfqpoint{2.518786in}{2.334266in}}%
\pgfpathclose%
\pgfusepath{stroke,fill}%
\end{pgfscope}%
\begin{pgfscope}%
\pgfpathrectangle{\pgfqpoint{0.800000in}{0.528000in}}{\pgfqpoint{4.960000in}{3.696000in}}%
\pgfusepath{clip}%
\pgfsetbuttcap%
\pgfsetroundjoin%
\definecolor{currentfill}{rgb}{0.000000,0.000000,0.000000}%
\pgfsetfillcolor{currentfill}%
\pgfsetlinewidth{1.003750pt}%
\definecolor{currentstroke}{rgb}{0.000000,0.000000,0.000000}%
\pgfsetstrokecolor{currentstroke}%
\pgfsetdash{}{0pt}%
\pgfpathmoveto{\pgfqpoint{2.518786in}{2.334266in}}%
\pgfpathcurveto{\pgfqpoint{2.529836in}{2.334266in}}{\pgfqpoint{2.540435in}{2.338657in}}{\pgfqpoint{2.548249in}{2.346470in}}%
\pgfpathcurveto{\pgfqpoint{2.556062in}{2.354284in}}{\pgfqpoint{2.560452in}{2.364883in}}{\pgfqpoint{2.560452in}{2.375933in}}%
\pgfpathcurveto{\pgfqpoint{2.560452in}{2.386983in}}{\pgfqpoint{2.556062in}{2.397582in}}{\pgfqpoint{2.548249in}{2.405396in}}%
\pgfpathcurveto{\pgfqpoint{2.540435in}{2.413209in}}{\pgfqpoint{2.529836in}{2.417600in}}{\pgfqpoint{2.518786in}{2.417600in}}%
\pgfpathcurveto{\pgfqpoint{2.507736in}{2.417600in}}{\pgfqpoint{2.497137in}{2.413209in}}{\pgfqpoint{2.489323in}{2.405396in}}%
\pgfpathcurveto{\pgfqpoint{2.481509in}{2.397582in}}{\pgfqpoint{2.477119in}{2.386983in}}{\pgfqpoint{2.477119in}{2.375933in}}%
\pgfpathcurveto{\pgfqpoint{2.477119in}{2.364883in}}{\pgfqpoint{2.481509in}{2.354284in}}{\pgfqpoint{2.489323in}{2.346470in}}%
\pgfpathcurveto{\pgfqpoint{2.497137in}{2.338657in}}{\pgfqpoint{2.507736in}{2.334266in}}{\pgfqpoint{2.518786in}{2.334266in}}%
\pgfpathclose%
\pgfusepath{stroke,fill}%
\end{pgfscope}%
\begin{pgfscope}%
\pgfpathrectangle{\pgfqpoint{0.800000in}{0.528000in}}{\pgfqpoint{4.960000in}{3.696000in}}%
\pgfusepath{clip}%
\pgfsetbuttcap%
\pgfsetroundjoin%
\definecolor{currentfill}{rgb}{0.000000,0.000000,0.000000}%
\pgfsetfillcolor{currentfill}%
\pgfsetlinewidth{1.003750pt}%
\definecolor{currentstroke}{rgb}{0.000000,0.000000,0.000000}%
\pgfsetstrokecolor{currentstroke}%
\pgfsetdash{}{0pt}%
\pgfpathmoveto{\pgfqpoint{2.518786in}{2.334266in}}%
\pgfpathcurveto{\pgfqpoint{2.529836in}{2.334266in}}{\pgfqpoint{2.540435in}{2.338657in}}{\pgfqpoint{2.548249in}{2.346470in}}%
\pgfpathcurveto{\pgfqpoint{2.556062in}{2.354284in}}{\pgfqpoint{2.560452in}{2.364883in}}{\pgfqpoint{2.560452in}{2.375933in}}%
\pgfpathcurveto{\pgfqpoint{2.560452in}{2.386983in}}{\pgfqpoint{2.556062in}{2.397582in}}{\pgfqpoint{2.548249in}{2.405396in}}%
\pgfpathcurveto{\pgfqpoint{2.540435in}{2.413209in}}{\pgfqpoint{2.529836in}{2.417600in}}{\pgfqpoint{2.518786in}{2.417600in}}%
\pgfpathcurveto{\pgfqpoint{2.507736in}{2.417600in}}{\pgfqpoint{2.497137in}{2.413209in}}{\pgfqpoint{2.489323in}{2.405396in}}%
\pgfpathcurveto{\pgfqpoint{2.481509in}{2.397582in}}{\pgfqpoint{2.477119in}{2.386983in}}{\pgfqpoint{2.477119in}{2.375933in}}%
\pgfpathcurveto{\pgfqpoint{2.477119in}{2.364883in}}{\pgfqpoint{2.481509in}{2.354284in}}{\pgfqpoint{2.489323in}{2.346470in}}%
\pgfpathcurveto{\pgfqpoint{2.497137in}{2.338657in}}{\pgfqpoint{2.507736in}{2.334266in}}{\pgfqpoint{2.518786in}{2.334266in}}%
\pgfpathclose%
\pgfusepath{stroke,fill}%
\end{pgfscope}%
\begin{pgfscope}%
\pgfpathrectangle{\pgfqpoint{0.800000in}{0.528000in}}{\pgfqpoint{4.960000in}{3.696000in}}%
\pgfusepath{clip}%
\pgfsetbuttcap%
\pgfsetroundjoin%
\definecolor{currentfill}{rgb}{0.000000,0.000000,0.000000}%
\pgfsetfillcolor{currentfill}%
\pgfsetlinewidth{1.003750pt}%
\definecolor{currentstroke}{rgb}{0.000000,0.000000,0.000000}%
\pgfsetstrokecolor{currentstroke}%
\pgfsetdash{}{0pt}%
\pgfpathmoveto{\pgfqpoint{2.518786in}{2.334266in}}%
\pgfpathcurveto{\pgfqpoint{2.529836in}{2.334266in}}{\pgfqpoint{2.540435in}{2.338657in}}{\pgfqpoint{2.548249in}{2.346470in}}%
\pgfpathcurveto{\pgfqpoint{2.556062in}{2.354284in}}{\pgfqpoint{2.560452in}{2.364883in}}{\pgfqpoint{2.560452in}{2.375933in}}%
\pgfpathcurveto{\pgfqpoint{2.560452in}{2.386983in}}{\pgfqpoint{2.556062in}{2.397582in}}{\pgfqpoint{2.548249in}{2.405396in}}%
\pgfpathcurveto{\pgfqpoint{2.540435in}{2.413209in}}{\pgfqpoint{2.529836in}{2.417600in}}{\pgfqpoint{2.518786in}{2.417600in}}%
\pgfpathcurveto{\pgfqpoint{2.507736in}{2.417600in}}{\pgfqpoint{2.497137in}{2.413209in}}{\pgfqpoint{2.489323in}{2.405396in}}%
\pgfpathcurveto{\pgfqpoint{2.481509in}{2.397582in}}{\pgfqpoint{2.477119in}{2.386983in}}{\pgfqpoint{2.477119in}{2.375933in}}%
\pgfpathcurveto{\pgfqpoint{2.477119in}{2.364883in}}{\pgfqpoint{2.481509in}{2.354284in}}{\pgfqpoint{2.489323in}{2.346470in}}%
\pgfpathcurveto{\pgfqpoint{2.497137in}{2.338657in}}{\pgfqpoint{2.507736in}{2.334266in}}{\pgfqpoint{2.518786in}{2.334266in}}%
\pgfpathclose%
\pgfusepath{stroke,fill}%
\end{pgfscope}%
\begin{pgfscope}%
\pgfpathrectangle{\pgfqpoint{0.800000in}{0.528000in}}{\pgfqpoint{4.960000in}{3.696000in}}%
\pgfusepath{clip}%
\pgfsetbuttcap%
\pgfsetroundjoin%
\definecolor{currentfill}{rgb}{0.000000,0.000000,0.000000}%
\pgfsetfillcolor{currentfill}%
\pgfsetlinewidth{1.003750pt}%
\definecolor{currentstroke}{rgb}{0.000000,0.000000,0.000000}%
\pgfsetstrokecolor{currentstroke}%
\pgfsetdash{}{0pt}%
\pgfpathmoveto{\pgfqpoint{2.518786in}{2.334266in}}%
\pgfpathcurveto{\pgfqpoint{2.529836in}{2.334266in}}{\pgfqpoint{2.540435in}{2.338657in}}{\pgfqpoint{2.548249in}{2.346470in}}%
\pgfpathcurveto{\pgfqpoint{2.556062in}{2.354284in}}{\pgfqpoint{2.560452in}{2.364883in}}{\pgfqpoint{2.560452in}{2.375933in}}%
\pgfpathcurveto{\pgfqpoint{2.560452in}{2.386983in}}{\pgfqpoint{2.556062in}{2.397582in}}{\pgfqpoint{2.548249in}{2.405396in}}%
\pgfpathcurveto{\pgfqpoint{2.540435in}{2.413209in}}{\pgfqpoint{2.529836in}{2.417600in}}{\pgfqpoint{2.518786in}{2.417600in}}%
\pgfpathcurveto{\pgfqpoint{2.507736in}{2.417600in}}{\pgfqpoint{2.497137in}{2.413209in}}{\pgfqpoint{2.489323in}{2.405396in}}%
\pgfpathcurveto{\pgfqpoint{2.481509in}{2.397582in}}{\pgfqpoint{2.477119in}{2.386983in}}{\pgfqpoint{2.477119in}{2.375933in}}%
\pgfpathcurveto{\pgfqpoint{2.477119in}{2.364883in}}{\pgfqpoint{2.481509in}{2.354284in}}{\pgfqpoint{2.489323in}{2.346470in}}%
\pgfpathcurveto{\pgfqpoint{2.497137in}{2.338657in}}{\pgfqpoint{2.507736in}{2.334266in}}{\pgfqpoint{2.518786in}{2.334266in}}%
\pgfpathclose%
\pgfusepath{stroke,fill}%
\end{pgfscope}%
\begin{pgfscope}%
\pgfpathrectangle{\pgfqpoint{0.800000in}{0.528000in}}{\pgfqpoint{4.960000in}{3.696000in}}%
\pgfusepath{clip}%
\pgfsetbuttcap%
\pgfsetroundjoin%
\definecolor{currentfill}{rgb}{0.000000,0.000000,0.000000}%
\pgfsetfillcolor{currentfill}%
\pgfsetlinewidth{1.003750pt}%
\definecolor{currentstroke}{rgb}{0.000000,0.000000,0.000000}%
\pgfsetstrokecolor{currentstroke}%
\pgfsetdash{}{0pt}%
\pgfpathmoveto{\pgfqpoint{2.518786in}{2.334266in}}%
\pgfpathcurveto{\pgfqpoint{2.529836in}{2.334266in}}{\pgfqpoint{2.540435in}{2.338657in}}{\pgfqpoint{2.548249in}{2.346470in}}%
\pgfpathcurveto{\pgfqpoint{2.556062in}{2.354284in}}{\pgfqpoint{2.560452in}{2.364883in}}{\pgfqpoint{2.560452in}{2.375933in}}%
\pgfpathcurveto{\pgfqpoint{2.560452in}{2.386983in}}{\pgfqpoint{2.556062in}{2.397582in}}{\pgfqpoint{2.548249in}{2.405396in}}%
\pgfpathcurveto{\pgfqpoint{2.540435in}{2.413209in}}{\pgfqpoint{2.529836in}{2.417600in}}{\pgfqpoint{2.518786in}{2.417600in}}%
\pgfpathcurveto{\pgfqpoint{2.507736in}{2.417600in}}{\pgfqpoint{2.497137in}{2.413209in}}{\pgfqpoint{2.489323in}{2.405396in}}%
\pgfpathcurveto{\pgfqpoint{2.481509in}{2.397582in}}{\pgfqpoint{2.477119in}{2.386983in}}{\pgfqpoint{2.477119in}{2.375933in}}%
\pgfpathcurveto{\pgfqpoint{2.477119in}{2.364883in}}{\pgfqpoint{2.481509in}{2.354284in}}{\pgfqpoint{2.489323in}{2.346470in}}%
\pgfpathcurveto{\pgfqpoint{2.497137in}{2.338657in}}{\pgfqpoint{2.507736in}{2.334266in}}{\pgfqpoint{2.518786in}{2.334266in}}%
\pgfpathclose%
\pgfusepath{stroke,fill}%
\end{pgfscope}%
\begin{pgfscope}%
\pgfpathrectangle{\pgfqpoint{0.800000in}{0.528000in}}{\pgfqpoint{4.960000in}{3.696000in}}%
\pgfusepath{clip}%
\pgfsetbuttcap%
\pgfsetroundjoin%
\definecolor{currentfill}{rgb}{0.000000,0.000000,0.000000}%
\pgfsetfillcolor{currentfill}%
\pgfsetlinewidth{1.003750pt}%
\definecolor{currentstroke}{rgb}{0.000000,0.000000,0.000000}%
\pgfsetstrokecolor{currentstroke}%
\pgfsetdash{}{0pt}%
\pgfpathmoveto{\pgfqpoint{2.518786in}{2.334266in}}%
\pgfpathcurveto{\pgfqpoint{2.529836in}{2.334266in}}{\pgfqpoint{2.540435in}{2.338657in}}{\pgfqpoint{2.548249in}{2.346470in}}%
\pgfpathcurveto{\pgfqpoint{2.556062in}{2.354284in}}{\pgfqpoint{2.560452in}{2.364883in}}{\pgfqpoint{2.560452in}{2.375933in}}%
\pgfpathcurveto{\pgfqpoint{2.560452in}{2.386983in}}{\pgfqpoint{2.556062in}{2.397582in}}{\pgfqpoint{2.548249in}{2.405396in}}%
\pgfpathcurveto{\pgfqpoint{2.540435in}{2.413209in}}{\pgfqpoint{2.529836in}{2.417600in}}{\pgfqpoint{2.518786in}{2.417600in}}%
\pgfpathcurveto{\pgfqpoint{2.507736in}{2.417600in}}{\pgfqpoint{2.497137in}{2.413209in}}{\pgfqpoint{2.489323in}{2.405396in}}%
\pgfpathcurveto{\pgfqpoint{2.481509in}{2.397582in}}{\pgfqpoint{2.477119in}{2.386983in}}{\pgfqpoint{2.477119in}{2.375933in}}%
\pgfpathcurveto{\pgfqpoint{2.477119in}{2.364883in}}{\pgfqpoint{2.481509in}{2.354284in}}{\pgfqpoint{2.489323in}{2.346470in}}%
\pgfpathcurveto{\pgfqpoint{2.497137in}{2.338657in}}{\pgfqpoint{2.507736in}{2.334266in}}{\pgfqpoint{2.518786in}{2.334266in}}%
\pgfpathclose%
\pgfusepath{stroke,fill}%
\end{pgfscope}%
\begin{pgfscope}%
\pgfpathrectangle{\pgfqpoint{0.800000in}{0.528000in}}{\pgfqpoint{4.960000in}{3.696000in}}%
\pgfusepath{clip}%
\pgfsetbuttcap%
\pgfsetroundjoin%
\definecolor{currentfill}{rgb}{0.000000,0.000000,0.000000}%
\pgfsetfillcolor{currentfill}%
\pgfsetlinewidth{1.003750pt}%
\definecolor{currentstroke}{rgb}{0.000000,0.000000,0.000000}%
\pgfsetstrokecolor{currentstroke}%
\pgfsetdash{}{0pt}%
\pgfpathmoveto{\pgfqpoint{2.518786in}{2.334266in}}%
\pgfpathcurveto{\pgfqpoint{2.529836in}{2.334266in}}{\pgfqpoint{2.540435in}{2.338657in}}{\pgfqpoint{2.548249in}{2.346470in}}%
\pgfpathcurveto{\pgfqpoint{2.556062in}{2.354284in}}{\pgfqpoint{2.560452in}{2.364883in}}{\pgfqpoint{2.560452in}{2.375933in}}%
\pgfpathcurveto{\pgfqpoint{2.560452in}{2.386983in}}{\pgfqpoint{2.556062in}{2.397582in}}{\pgfqpoint{2.548249in}{2.405396in}}%
\pgfpathcurveto{\pgfqpoint{2.540435in}{2.413209in}}{\pgfqpoint{2.529836in}{2.417600in}}{\pgfqpoint{2.518786in}{2.417600in}}%
\pgfpathcurveto{\pgfqpoint{2.507736in}{2.417600in}}{\pgfqpoint{2.497137in}{2.413209in}}{\pgfqpoint{2.489323in}{2.405396in}}%
\pgfpathcurveto{\pgfqpoint{2.481509in}{2.397582in}}{\pgfqpoint{2.477119in}{2.386983in}}{\pgfqpoint{2.477119in}{2.375933in}}%
\pgfpathcurveto{\pgfqpoint{2.477119in}{2.364883in}}{\pgfqpoint{2.481509in}{2.354284in}}{\pgfqpoint{2.489323in}{2.346470in}}%
\pgfpathcurveto{\pgfqpoint{2.497137in}{2.338657in}}{\pgfqpoint{2.507736in}{2.334266in}}{\pgfqpoint{2.518786in}{2.334266in}}%
\pgfpathclose%
\pgfusepath{stroke,fill}%
\end{pgfscope}%
\begin{pgfscope}%
\pgfpathrectangle{\pgfqpoint{0.800000in}{0.528000in}}{\pgfqpoint{4.960000in}{3.696000in}}%
\pgfusepath{clip}%
\pgfsetbuttcap%
\pgfsetroundjoin%
\definecolor{currentfill}{rgb}{0.000000,0.000000,0.000000}%
\pgfsetfillcolor{currentfill}%
\pgfsetlinewidth{1.003750pt}%
\definecolor{currentstroke}{rgb}{0.000000,0.000000,0.000000}%
\pgfsetstrokecolor{currentstroke}%
\pgfsetdash{}{0pt}%
\pgfpathmoveto{\pgfqpoint{2.518786in}{2.334266in}}%
\pgfpathcurveto{\pgfqpoint{2.529836in}{2.334266in}}{\pgfqpoint{2.540435in}{2.338657in}}{\pgfqpoint{2.548249in}{2.346470in}}%
\pgfpathcurveto{\pgfqpoint{2.556062in}{2.354284in}}{\pgfqpoint{2.560452in}{2.364883in}}{\pgfqpoint{2.560452in}{2.375933in}}%
\pgfpathcurveto{\pgfqpoint{2.560452in}{2.386983in}}{\pgfqpoint{2.556062in}{2.397582in}}{\pgfqpoint{2.548249in}{2.405396in}}%
\pgfpathcurveto{\pgfqpoint{2.540435in}{2.413209in}}{\pgfqpoint{2.529836in}{2.417600in}}{\pgfqpoint{2.518786in}{2.417600in}}%
\pgfpathcurveto{\pgfqpoint{2.507736in}{2.417600in}}{\pgfqpoint{2.497137in}{2.413209in}}{\pgfqpoint{2.489323in}{2.405396in}}%
\pgfpathcurveto{\pgfqpoint{2.481509in}{2.397582in}}{\pgfqpoint{2.477119in}{2.386983in}}{\pgfqpoint{2.477119in}{2.375933in}}%
\pgfpathcurveto{\pgfqpoint{2.477119in}{2.364883in}}{\pgfqpoint{2.481509in}{2.354284in}}{\pgfqpoint{2.489323in}{2.346470in}}%
\pgfpathcurveto{\pgfqpoint{2.497137in}{2.338657in}}{\pgfqpoint{2.507736in}{2.334266in}}{\pgfqpoint{2.518786in}{2.334266in}}%
\pgfpathclose%
\pgfusepath{stroke,fill}%
\end{pgfscope}%
\begin{pgfscope}%
\pgfpathrectangle{\pgfqpoint{0.800000in}{0.528000in}}{\pgfqpoint{4.960000in}{3.696000in}}%
\pgfusepath{clip}%
\pgfsetbuttcap%
\pgfsetroundjoin%
\definecolor{currentfill}{rgb}{0.000000,0.000000,0.000000}%
\pgfsetfillcolor{currentfill}%
\pgfsetlinewidth{1.003750pt}%
\definecolor{currentstroke}{rgb}{0.000000,0.000000,0.000000}%
\pgfsetstrokecolor{currentstroke}%
\pgfsetdash{}{0pt}%
\pgfpathmoveto{\pgfqpoint{2.518786in}{2.334266in}}%
\pgfpathcurveto{\pgfqpoint{2.529836in}{2.334266in}}{\pgfqpoint{2.540435in}{2.338657in}}{\pgfqpoint{2.548249in}{2.346470in}}%
\pgfpathcurveto{\pgfqpoint{2.556062in}{2.354284in}}{\pgfqpoint{2.560452in}{2.364883in}}{\pgfqpoint{2.560452in}{2.375933in}}%
\pgfpathcurveto{\pgfqpoint{2.560452in}{2.386983in}}{\pgfqpoint{2.556062in}{2.397582in}}{\pgfqpoint{2.548249in}{2.405396in}}%
\pgfpathcurveto{\pgfqpoint{2.540435in}{2.413209in}}{\pgfqpoint{2.529836in}{2.417600in}}{\pgfqpoint{2.518786in}{2.417600in}}%
\pgfpathcurveto{\pgfqpoint{2.507736in}{2.417600in}}{\pgfqpoint{2.497137in}{2.413209in}}{\pgfqpoint{2.489323in}{2.405396in}}%
\pgfpathcurveto{\pgfqpoint{2.481509in}{2.397582in}}{\pgfqpoint{2.477119in}{2.386983in}}{\pgfqpoint{2.477119in}{2.375933in}}%
\pgfpathcurveto{\pgfqpoint{2.477119in}{2.364883in}}{\pgfqpoint{2.481509in}{2.354284in}}{\pgfqpoint{2.489323in}{2.346470in}}%
\pgfpathcurveto{\pgfqpoint{2.497137in}{2.338657in}}{\pgfqpoint{2.507736in}{2.334266in}}{\pgfqpoint{2.518786in}{2.334266in}}%
\pgfpathclose%
\pgfusepath{stroke,fill}%
\end{pgfscope}%
\begin{pgfscope}%
\pgfpathrectangle{\pgfqpoint{0.800000in}{0.528000in}}{\pgfqpoint{4.960000in}{3.696000in}}%
\pgfusepath{clip}%
\pgfsetbuttcap%
\pgfsetroundjoin%
\definecolor{currentfill}{rgb}{0.000000,0.000000,0.000000}%
\pgfsetfillcolor{currentfill}%
\pgfsetlinewidth{1.003750pt}%
\definecolor{currentstroke}{rgb}{0.000000,0.000000,0.000000}%
\pgfsetstrokecolor{currentstroke}%
\pgfsetdash{}{0pt}%
\pgfpathmoveto{\pgfqpoint{2.518786in}{2.334266in}}%
\pgfpathcurveto{\pgfqpoint{2.529836in}{2.334266in}}{\pgfqpoint{2.540435in}{2.338657in}}{\pgfqpoint{2.548249in}{2.346470in}}%
\pgfpathcurveto{\pgfqpoint{2.556062in}{2.354284in}}{\pgfqpoint{2.560452in}{2.364883in}}{\pgfqpoint{2.560452in}{2.375933in}}%
\pgfpathcurveto{\pgfqpoint{2.560452in}{2.386983in}}{\pgfqpoint{2.556062in}{2.397582in}}{\pgfqpoint{2.548249in}{2.405396in}}%
\pgfpathcurveto{\pgfqpoint{2.540435in}{2.413209in}}{\pgfqpoint{2.529836in}{2.417600in}}{\pgfqpoint{2.518786in}{2.417600in}}%
\pgfpathcurveto{\pgfqpoint{2.507736in}{2.417600in}}{\pgfqpoint{2.497137in}{2.413209in}}{\pgfqpoint{2.489323in}{2.405396in}}%
\pgfpathcurveto{\pgfqpoint{2.481509in}{2.397582in}}{\pgfqpoint{2.477119in}{2.386983in}}{\pgfqpoint{2.477119in}{2.375933in}}%
\pgfpathcurveto{\pgfqpoint{2.477119in}{2.364883in}}{\pgfqpoint{2.481509in}{2.354284in}}{\pgfqpoint{2.489323in}{2.346470in}}%
\pgfpathcurveto{\pgfqpoint{2.497137in}{2.338657in}}{\pgfqpoint{2.507736in}{2.334266in}}{\pgfqpoint{2.518786in}{2.334266in}}%
\pgfpathclose%
\pgfusepath{stroke,fill}%
\end{pgfscope}%
\begin{pgfscope}%
\pgfpathrectangle{\pgfqpoint{0.800000in}{0.528000in}}{\pgfqpoint{4.960000in}{3.696000in}}%
\pgfusepath{clip}%
\pgfsetbuttcap%
\pgfsetroundjoin%
\definecolor{currentfill}{rgb}{0.000000,0.000000,0.000000}%
\pgfsetfillcolor{currentfill}%
\pgfsetlinewidth{1.003750pt}%
\definecolor{currentstroke}{rgb}{0.000000,0.000000,0.000000}%
\pgfsetstrokecolor{currentstroke}%
\pgfsetdash{}{0pt}%
\pgfpathmoveto{\pgfqpoint{2.518786in}{2.334266in}}%
\pgfpathcurveto{\pgfqpoint{2.529836in}{2.334266in}}{\pgfqpoint{2.540435in}{2.338657in}}{\pgfqpoint{2.548249in}{2.346470in}}%
\pgfpathcurveto{\pgfqpoint{2.556062in}{2.354284in}}{\pgfqpoint{2.560452in}{2.364883in}}{\pgfqpoint{2.560452in}{2.375933in}}%
\pgfpathcurveto{\pgfqpoint{2.560452in}{2.386983in}}{\pgfqpoint{2.556062in}{2.397582in}}{\pgfqpoint{2.548249in}{2.405396in}}%
\pgfpathcurveto{\pgfqpoint{2.540435in}{2.413209in}}{\pgfqpoint{2.529836in}{2.417600in}}{\pgfqpoint{2.518786in}{2.417600in}}%
\pgfpathcurveto{\pgfqpoint{2.507736in}{2.417600in}}{\pgfqpoint{2.497137in}{2.413209in}}{\pgfqpoint{2.489323in}{2.405396in}}%
\pgfpathcurveto{\pgfqpoint{2.481509in}{2.397582in}}{\pgfqpoint{2.477119in}{2.386983in}}{\pgfqpoint{2.477119in}{2.375933in}}%
\pgfpathcurveto{\pgfqpoint{2.477119in}{2.364883in}}{\pgfqpoint{2.481509in}{2.354284in}}{\pgfqpoint{2.489323in}{2.346470in}}%
\pgfpathcurveto{\pgfqpoint{2.497137in}{2.338657in}}{\pgfqpoint{2.507736in}{2.334266in}}{\pgfqpoint{2.518786in}{2.334266in}}%
\pgfpathclose%
\pgfusepath{stroke,fill}%
\end{pgfscope}%
\begin{pgfscope}%
\pgfpathrectangle{\pgfqpoint{0.800000in}{0.528000in}}{\pgfqpoint{4.960000in}{3.696000in}}%
\pgfusepath{clip}%
\pgfsetbuttcap%
\pgfsetroundjoin%
\definecolor{currentfill}{rgb}{0.000000,0.000000,0.000000}%
\pgfsetfillcolor{currentfill}%
\pgfsetlinewidth{1.003750pt}%
\definecolor{currentstroke}{rgb}{0.000000,0.000000,0.000000}%
\pgfsetstrokecolor{currentstroke}%
\pgfsetdash{}{0pt}%
\pgfpathmoveto{\pgfqpoint{2.518786in}{2.334266in}}%
\pgfpathcurveto{\pgfqpoint{2.529836in}{2.334266in}}{\pgfqpoint{2.540435in}{2.338657in}}{\pgfqpoint{2.548249in}{2.346470in}}%
\pgfpathcurveto{\pgfqpoint{2.556062in}{2.354284in}}{\pgfqpoint{2.560452in}{2.364883in}}{\pgfqpoint{2.560452in}{2.375933in}}%
\pgfpathcurveto{\pgfqpoint{2.560452in}{2.386983in}}{\pgfqpoint{2.556062in}{2.397582in}}{\pgfqpoint{2.548249in}{2.405396in}}%
\pgfpathcurveto{\pgfqpoint{2.540435in}{2.413209in}}{\pgfqpoint{2.529836in}{2.417600in}}{\pgfqpoint{2.518786in}{2.417600in}}%
\pgfpathcurveto{\pgfqpoint{2.507736in}{2.417600in}}{\pgfqpoint{2.497137in}{2.413209in}}{\pgfqpoint{2.489323in}{2.405396in}}%
\pgfpathcurveto{\pgfqpoint{2.481509in}{2.397582in}}{\pgfqpoint{2.477119in}{2.386983in}}{\pgfqpoint{2.477119in}{2.375933in}}%
\pgfpathcurveto{\pgfqpoint{2.477119in}{2.364883in}}{\pgfqpoint{2.481509in}{2.354284in}}{\pgfqpoint{2.489323in}{2.346470in}}%
\pgfpathcurveto{\pgfqpoint{2.497137in}{2.338657in}}{\pgfqpoint{2.507736in}{2.334266in}}{\pgfqpoint{2.518786in}{2.334266in}}%
\pgfpathclose%
\pgfusepath{stroke,fill}%
\end{pgfscope}%
\begin{pgfscope}%
\pgfpathrectangle{\pgfqpoint{0.800000in}{0.528000in}}{\pgfqpoint{4.960000in}{3.696000in}}%
\pgfusepath{clip}%
\pgfsetbuttcap%
\pgfsetroundjoin%
\definecolor{currentfill}{rgb}{0.000000,0.000000,0.000000}%
\pgfsetfillcolor{currentfill}%
\pgfsetlinewidth{1.003750pt}%
\definecolor{currentstroke}{rgb}{0.000000,0.000000,0.000000}%
\pgfsetstrokecolor{currentstroke}%
\pgfsetdash{}{0pt}%
\pgfpathmoveto{\pgfqpoint{2.518786in}{2.334266in}}%
\pgfpathcurveto{\pgfqpoint{2.529836in}{2.334266in}}{\pgfqpoint{2.540435in}{2.338657in}}{\pgfqpoint{2.548249in}{2.346470in}}%
\pgfpathcurveto{\pgfqpoint{2.556062in}{2.354284in}}{\pgfqpoint{2.560452in}{2.364883in}}{\pgfqpoint{2.560452in}{2.375933in}}%
\pgfpathcurveto{\pgfqpoint{2.560452in}{2.386983in}}{\pgfqpoint{2.556062in}{2.397582in}}{\pgfqpoint{2.548249in}{2.405396in}}%
\pgfpathcurveto{\pgfqpoint{2.540435in}{2.413209in}}{\pgfqpoint{2.529836in}{2.417600in}}{\pgfqpoint{2.518786in}{2.417600in}}%
\pgfpathcurveto{\pgfqpoint{2.507736in}{2.417600in}}{\pgfqpoint{2.497137in}{2.413209in}}{\pgfqpoint{2.489323in}{2.405396in}}%
\pgfpathcurveto{\pgfqpoint{2.481509in}{2.397582in}}{\pgfqpoint{2.477119in}{2.386983in}}{\pgfqpoint{2.477119in}{2.375933in}}%
\pgfpathcurveto{\pgfqpoint{2.477119in}{2.364883in}}{\pgfqpoint{2.481509in}{2.354284in}}{\pgfqpoint{2.489323in}{2.346470in}}%
\pgfpathcurveto{\pgfqpoint{2.497137in}{2.338657in}}{\pgfqpoint{2.507736in}{2.334266in}}{\pgfqpoint{2.518786in}{2.334266in}}%
\pgfpathclose%
\pgfusepath{stroke,fill}%
\end{pgfscope}%
\begin{pgfscope}%
\pgfpathrectangle{\pgfqpoint{0.800000in}{0.528000in}}{\pgfqpoint{4.960000in}{3.696000in}}%
\pgfusepath{clip}%
\pgfsetbuttcap%
\pgfsetroundjoin%
\definecolor{currentfill}{rgb}{0.000000,0.000000,0.000000}%
\pgfsetfillcolor{currentfill}%
\pgfsetlinewidth{1.003750pt}%
\definecolor{currentstroke}{rgb}{0.000000,0.000000,0.000000}%
\pgfsetstrokecolor{currentstroke}%
\pgfsetdash{}{0pt}%
\pgfpathmoveto{\pgfqpoint{2.518786in}{2.334266in}}%
\pgfpathcurveto{\pgfqpoint{2.529836in}{2.334266in}}{\pgfqpoint{2.540435in}{2.338657in}}{\pgfqpoint{2.548249in}{2.346470in}}%
\pgfpathcurveto{\pgfqpoint{2.556062in}{2.354284in}}{\pgfqpoint{2.560452in}{2.364883in}}{\pgfqpoint{2.560452in}{2.375933in}}%
\pgfpathcurveto{\pgfqpoint{2.560452in}{2.386983in}}{\pgfqpoint{2.556062in}{2.397582in}}{\pgfqpoint{2.548249in}{2.405396in}}%
\pgfpathcurveto{\pgfqpoint{2.540435in}{2.413209in}}{\pgfqpoint{2.529836in}{2.417600in}}{\pgfqpoint{2.518786in}{2.417600in}}%
\pgfpathcurveto{\pgfqpoint{2.507736in}{2.417600in}}{\pgfqpoint{2.497137in}{2.413209in}}{\pgfqpoint{2.489323in}{2.405396in}}%
\pgfpathcurveto{\pgfqpoint{2.481509in}{2.397582in}}{\pgfqpoint{2.477119in}{2.386983in}}{\pgfqpoint{2.477119in}{2.375933in}}%
\pgfpathcurveto{\pgfqpoint{2.477119in}{2.364883in}}{\pgfqpoint{2.481509in}{2.354284in}}{\pgfqpoint{2.489323in}{2.346470in}}%
\pgfpathcurveto{\pgfqpoint{2.497137in}{2.338657in}}{\pgfqpoint{2.507736in}{2.334266in}}{\pgfqpoint{2.518786in}{2.334266in}}%
\pgfpathclose%
\pgfusepath{stroke,fill}%
\end{pgfscope}%
\begin{pgfscope}%
\pgfpathrectangle{\pgfqpoint{0.800000in}{0.528000in}}{\pgfqpoint{4.960000in}{3.696000in}}%
\pgfusepath{clip}%
\pgfsetbuttcap%
\pgfsetroundjoin%
\definecolor{currentfill}{rgb}{0.000000,0.000000,0.000000}%
\pgfsetfillcolor{currentfill}%
\pgfsetlinewidth{1.003750pt}%
\definecolor{currentstroke}{rgb}{0.000000,0.000000,0.000000}%
\pgfsetstrokecolor{currentstroke}%
\pgfsetdash{}{0pt}%
\pgfpathmoveto{\pgfqpoint{2.518786in}{2.334266in}}%
\pgfpathcurveto{\pgfqpoint{2.529836in}{2.334266in}}{\pgfqpoint{2.540435in}{2.338657in}}{\pgfqpoint{2.548249in}{2.346470in}}%
\pgfpathcurveto{\pgfqpoint{2.556062in}{2.354284in}}{\pgfqpoint{2.560452in}{2.364883in}}{\pgfqpoint{2.560452in}{2.375933in}}%
\pgfpathcurveto{\pgfqpoint{2.560452in}{2.386983in}}{\pgfqpoint{2.556062in}{2.397582in}}{\pgfqpoint{2.548249in}{2.405396in}}%
\pgfpathcurveto{\pgfqpoint{2.540435in}{2.413209in}}{\pgfqpoint{2.529836in}{2.417600in}}{\pgfqpoint{2.518786in}{2.417600in}}%
\pgfpathcurveto{\pgfqpoint{2.507736in}{2.417600in}}{\pgfqpoint{2.497137in}{2.413209in}}{\pgfqpoint{2.489323in}{2.405396in}}%
\pgfpathcurveto{\pgfqpoint{2.481509in}{2.397582in}}{\pgfqpoint{2.477119in}{2.386983in}}{\pgfqpoint{2.477119in}{2.375933in}}%
\pgfpathcurveto{\pgfqpoint{2.477119in}{2.364883in}}{\pgfqpoint{2.481509in}{2.354284in}}{\pgfqpoint{2.489323in}{2.346470in}}%
\pgfpathcurveto{\pgfqpoint{2.497137in}{2.338657in}}{\pgfqpoint{2.507736in}{2.334266in}}{\pgfqpoint{2.518786in}{2.334266in}}%
\pgfpathclose%
\pgfusepath{stroke,fill}%
\end{pgfscope}%
\begin{pgfscope}%
\pgfpathrectangle{\pgfqpoint{0.800000in}{0.528000in}}{\pgfqpoint{4.960000in}{3.696000in}}%
\pgfusepath{clip}%
\pgfsetbuttcap%
\pgfsetroundjoin%
\definecolor{currentfill}{rgb}{0.000000,0.000000,0.000000}%
\pgfsetfillcolor{currentfill}%
\pgfsetlinewidth{1.003750pt}%
\definecolor{currentstroke}{rgb}{0.000000,0.000000,0.000000}%
\pgfsetstrokecolor{currentstroke}%
\pgfsetdash{}{0pt}%
\pgfpathmoveto{\pgfqpoint{2.518786in}{2.334266in}}%
\pgfpathcurveto{\pgfqpoint{2.529836in}{2.334266in}}{\pgfqpoint{2.540435in}{2.338657in}}{\pgfqpoint{2.548249in}{2.346470in}}%
\pgfpathcurveto{\pgfqpoint{2.556062in}{2.354284in}}{\pgfqpoint{2.560452in}{2.364883in}}{\pgfqpoint{2.560452in}{2.375933in}}%
\pgfpathcurveto{\pgfqpoint{2.560452in}{2.386983in}}{\pgfqpoint{2.556062in}{2.397582in}}{\pgfqpoint{2.548249in}{2.405396in}}%
\pgfpathcurveto{\pgfqpoint{2.540435in}{2.413209in}}{\pgfqpoint{2.529836in}{2.417600in}}{\pgfqpoint{2.518786in}{2.417600in}}%
\pgfpathcurveto{\pgfqpoint{2.507736in}{2.417600in}}{\pgfqpoint{2.497137in}{2.413209in}}{\pgfqpoint{2.489323in}{2.405396in}}%
\pgfpathcurveto{\pgfqpoint{2.481509in}{2.397582in}}{\pgfqpoint{2.477119in}{2.386983in}}{\pgfqpoint{2.477119in}{2.375933in}}%
\pgfpathcurveto{\pgfqpoint{2.477119in}{2.364883in}}{\pgfqpoint{2.481509in}{2.354284in}}{\pgfqpoint{2.489323in}{2.346470in}}%
\pgfpathcurveto{\pgfqpoint{2.497137in}{2.338657in}}{\pgfqpoint{2.507736in}{2.334266in}}{\pgfqpoint{2.518786in}{2.334266in}}%
\pgfpathclose%
\pgfusepath{stroke,fill}%
\end{pgfscope}%
\begin{pgfscope}%
\pgfpathrectangle{\pgfqpoint{0.800000in}{0.528000in}}{\pgfqpoint{4.960000in}{3.696000in}}%
\pgfusepath{clip}%
\pgfsetbuttcap%
\pgfsetroundjoin%
\definecolor{currentfill}{rgb}{0.000000,0.000000,0.000000}%
\pgfsetfillcolor{currentfill}%
\pgfsetlinewidth{1.003750pt}%
\definecolor{currentstroke}{rgb}{0.000000,0.000000,0.000000}%
\pgfsetstrokecolor{currentstroke}%
\pgfsetdash{}{0pt}%
\pgfpathmoveto{\pgfqpoint{4.011666in}{2.334266in}}%
\pgfpathcurveto{\pgfqpoint{4.022716in}{2.334266in}}{\pgfqpoint{4.033315in}{2.338657in}}{\pgfqpoint{4.041128in}{2.346470in}}%
\pgfpathcurveto{\pgfqpoint{4.048942in}{2.354284in}}{\pgfqpoint{4.053332in}{2.364883in}}{\pgfqpoint{4.053332in}{2.375933in}}%
\pgfpathcurveto{\pgfqpoint{4.053332in}{2.386983in}}{\pgfqpoint{4.048942in}{2.397582in}}{\pgfqpoint{4.041128in}{2.405396in}}%
\pgfpathcurveto{\pgfqpoint{4.033315in}{2.413209in}}{\pgfqpoint{4.022716in}{2.417600in}}{\pgfqpoint{4.011666in}{2.417600in}}%
\pgfpathcurveto{\pgfqpoint{4.000616in}{2.417600in}}{\pgfqpoint{3.990016in}{2.413209in}}{\pgfqpoint{3.982203in}{2.405396in}}%
\pgfpathcurveto{\pgfqpoint{3.974389in}{2.397582in}}{\pgfqpoint{3.969999in}{2.386983in}}{\pgfqpoint{3.969999in}{2.375933in}}%
\pgfpathcurveto{\pgfqpoint{3.969999in}{2.364883in}}{\pgfqpoint{3.974389in}{2.354284in}}{\pgfqpoint{3.982203in}{2.346470in}}%
\pgfpathcurveto{\pgfqpoint{3.990016in}{2.338657in}}{\pgfqpoint{4.000616in}{2.334266in}}{\pgfqpoint{4.011666in}{2.334266in}}%
\pgfpathclose%
\pgfusepath{stroke,fill}%
\end{pgfscope}%
\begin{pgfscope}%
\pgfpathrectangle{\pgfqpoint{0.800000in}{0.528000in}}{\pgfqpoint{4.960000in}{3.696000in}}%
\pgfusepath{clip}%
\pgfsetbuttcap%
\pgfsetroundjoin%
\definecolor{currentfill}{rgb}{0.000000,0.000000,0.000000}%
\pgfsetfillcolor{currentfill}%
\pgfsetlinewidth{1.003750pt}%
\definecolor{currentstroke}{rgb}{0.000000,0.000000,0.000000}%
\pgfsetstrokecolor{currentstroke}%
\pgfsetdash{}{0pt}%
\pgfpathmoveto{\pgfqpoint{4.011666in}{2.334266in}}%
\pgfpathcurveto{\pgfqpoint{4.022716in}{2.334266in}}{\pgfqpoint{4.033315in}{2.338657in}}{\pgfqpoint{4.041128in}{2.346470in}}%
\pgfpathcurveto{\pgfqpoint{4.048942in}{2.354284in}}{\pgfqpoint{4.053332in}{2.364883in}}{\pgfqpoint{4.053332in}{2.375933in}}%
\pgfpathcurveto{\pgfqpoint{4.053332in}{2.386983in}}{\pgfqpoint{4.048942in}{2.397582in}}{\pgfqpoint{4.041128in}{2.405396in}}%
\pgfpathcurveto{\pgfqpoint{4.033315in}{2.413209in}}{\pgfqpoint{4.022716in}{2.417600in}}{\pgfqpoint{4.011666in}{2.417600in}}%
\pgfpathcurveto{\pgfqpoint{4.000616in}{2.417600in}}{\pgfqpoint{3.990016in}{2.413209in}}{\pgfqpoint{3.982203in}{2.405396in}}%
\pgfpathcurveto{\pgfqpoint{3.974389in}{2.397582in}}{\pgfqpoint{3.969999in}{2.386983in}}{\pgfqpoint{3.969999in}{2.375933in}}%
\pgfpathcurveto{\pgfqpoint{3.969999in}{2.364883in}}{\pgfqpoint{3.974389in}{2.354284in}}{\pgfqpoint{3.982203in}{2.346470in}}%
\pgfpathcurveto{\pgfqpoint{3.990016in}{2.338657in}}{\pgfqpoint{4.000616in}{2.334266in}}{\pgfqpoint{4.011666in}{2.334266in}}%
\pgfpathclose%
\pgfusepath{stroke,fill}%
\end{pgfscope}%
\begin{pgfscope}%
\pgfpathrectangle{\pgfqpoint{0.800000in}{0.528000in}}{\pgfqpoint{4.960000in}{3.696000in}}%
\pgfusepath{clip}%
\pgfsetbuttcap%
\pgfsetroundjoin%
\definecolor{currentfill}{rgb}{0.000000,0.000000,0.000000}%
\pgfsetfillcolor{currentfill}%
\pgfsetlinewidth{1.003750pt}%
\definecolor{currentstroke}{rgb}{0.000000,0.000000,0.000000}%
\pgfsetstrokecolor{currentstroke}%
\pgfsetdash{}{0pt}%
\pgfpathmoveto{\pgfqpoint{4.011666in}{2.334266in}}%
\pgfpathcurveto{\pgfqpoint{4.022716in}{2.334266in}}{\pgfqpoint{4.033315in}{2.338657in}}{\pgfqpoint{4.041128in}{2.346470in}}%
\pgfpathcurveto{\pgfqpoint{4.048942in}{2.354284in}}{\pgfqpoint{4.053332in}{2.364883in}}{\pgfqpoint{4.053332in}{2.375933in}}%
\pgfpathcurveto{\pgfqpoint{4.053332in}{2.386983in}}{\pgfqpoint{4.048942in}{2.397582in}}{\pgfqpoint{4.041128in}{2.405396in}}%
\pgfpathcurveto{\pgfqpoint{4.033315in}{2.413209in}}{\pgfqpoint{4.022716in}{2.417600in}}{\pgfqpoint{4.011666in}{2.417600in}}%
\pgfpathcurveto{\pgfqpoint{4.000616in}{2.417600in}}{\pgfqpoint{3.990016in}{2.413209in}}{\pgfqpoint{3.982203in}{2.405396in}}%
\pgfpathcurveto{\pgfqpoint{3.974389in}{2.397582in}}{\pgfqpoint{3.969999in}{2.386983in}}{\pgfqpoint{3.969999in}{2.375933in}}%
\pgfpathcurveto{\pgfqpoint{3.969999in}{2.364883in}}{\pgfqpoint{3.974389in}{2.354284in}}{\pgfqpoint{3.982203in}{2.346470in}}%
\pgfpathcurveto{\pgfqpoint{3.990016in}{2.338657in}}{\pgfqpoint{4.000616in}{2.334266in}}{\pgfqpoint{4.011666in}{2.334266in}}%
\pgfpathclose%
\pgfusepath{stroke,fill}%
\end{pgfscope}%
\begin{pgfscope}%
\pgfpathrectangle{\pgfqpoint{0.800000in}{0.528000in}}{\pgfqpoint{4.960000in}{3.696000in}}%
\pgfusepath{clip}%
\pgfsetbuttcap%
\pgfsetroundjoin%
\definecolor{currentfill}{rgb}{0.000000,0.000000,0.000000}%
\pgfsetfillcolor{currentfill}%
\pgfsetlinewidth{1.003750pt}%
\definecolor{currentstroke}{rgb}{0.000000,0.000000,0.000000}%
\pgfsetstrokecolor{currentstroke}%
\pgfsetdash{}{0pt}%
\pgfpathmoveto{\pgfqpoint{4.011666in}{2.334266in}}%
\pgfpathcurveto{\pgfqpoint{4.022716in}{2.334266in}}{\pgfqpoint{4.033315in}{2.338657in}}{\pgfqpoint{4.041128in}{2.346470in}}%
\pgfpathcurveto{\pgfqpoint{4.048942in}{2.354284in}}{\pgfqpoint{4.053332in}{2.364883in}}{\pgfqpoint{4.053332in}{2.375933in}}%
\pgfpathcurveto{\pgfqpoint{4.053332in}{2.386983in}}{\pgfqpoint{4.048942in}{2.397582in}}{\pgfqpoint{4.041128in}{2.405396in}}%
\pgfpathcurveto{\pgfqpoint{4.033315in}{2.413209in}}{\pgfqpoint{4.022716in}{2.417600in}}{\pgfqpoint{4.011666in}{2.417600in}}%
\pgfpathcurveto{\pgfqpoint{4.000616in}{2.417600in}}{\pgfqpoint{3.990016in}{2.413209in}}{\pgfqpoint{3.982203in}{2.405396in}}%
\pgfpathcurveto{\pgfqpoint{3.974389in}{2.397582in}}{\pgfqpoint{3.969999in}{2.386983in}}{\pgfqpoint{3.969999in}{2.375933in}}%
\pgfpathcurveto{\pgfqpoint{3.969999in}{2.364883in}}{\pgfqpoint{3.974389in}{2.354284in}}{\pgfqpoint{3.982203in}{2.346470in}}%
\pgfpathcurveto{\pgfqpoint{3.990016in}{2.338657in}}{\pgfqpoint{4.000616in}{2.334266in}}{\pgfqpoint{4.011666in}{2.334266in}}%
\pgfpathclose%
\pgfusepath{stroke,fill}%
\end{pgfscope}%
\begin{pgfscope}%
\pgfpathrectangle{\pgfqpoint{0.800000in}{0.528000in}}{\pgfqpoint{4.960000in}{3.696000in}}%
\pgfusepath{clip}%
\pgfsetbuttcap%
\pgfsetroundjoin%
\definecolor{currentfill}{rgb}{0.000000,0.000000,0.000000}%
\pgfsetfillcolor{currentfill}%
\pgfsetlinewidth{1.003750pt}%
\definecolor{currentstroke}{rgb}{0.000000,0.000000,0.000000}%
\pgfsetstrokecolor{currentstroke}%
\pgfsetdash{}{0pt}%
\pgfpathmoveto{\pgfqpoint{4.011666in}{2.334266in}}%
\pgfpathcurveto{\pgfqpoint{4.022716in}{2.334266in}}{\pgfqpoint{4.033315in}{2.338657in}}{\pgfqpoint{4.041128in}{2.346470in}}%
\pgfpathcurveto{\pgfqpoint{4.048942in}{2.354284in}}{\pgfqpoint{4.053332in}{2.364883in}}{\pgfqpoint{4.053332in}{2.375933in}}%
\pgfpathcurveto{\pgfqpoint{4.053332in}{2.386983in}}{\pgfqpoint{4.048942in}{2.397582in}}{\pgfqpoint{4.041128in}{2.405396in}}%
\pgfpathcurveto{\pgfqpoint{4.033315in}{2.413209in}}{\pgfqpoint{4.022716in}{2.417600in}}{\pgfqpoint{4.011666in}{2.417600in}}%
\pgfpathcurveto{\pgfqpoint{4.000616in}{2.417600in}}{\pgfqpoint{3.990016in}{2.413209in}}{\pgfqpoint{3.982203in}{2.405396in}}%
\pgfpathcurveto{\pgfqpoint{3.974389in}{2.397582in}}{\pgfqpoint{3.969999in}{2.386983in}}{\pgfqpoint{3.969999in}{2.375933in}}%
\pgfpathcurveto{\pgfqpoint{3.969999in}{2.364883in}}{\pgfqpoint{3.974389in}{2.354284in}}{\pgfqpoint{3.982203in}{2.346470in}}%
\pgfpathcurveto{\pgfqpoint{3.990016in}{2.338657in}}{\pgfqpoint{4.000616in}{2.334266in}}{\pgfqpoint{4.011666in}{2.334266in}}%
\pgfpathclose%
\pgfusepath{stroke,fill}%
\end{pgfscope}%
\begin{pgfscope}%
\pgfpathrectangle{\pgfqpoint{0.800000in}{0.528000in}}{\pgfqpoint{4.960000in}{3.696000in}}%
\pgfusepath{clip}%
\pgfsetbuttcap%
\pgfsetroundjoin%
\definecolor{currentfill}{rgb}{0.000000,0.000000,0.000000}%
\pgfsetfillcolor{currentfill}%
\pgfsetlinewidth{1.003750pt}%
\definecolor{currentstroke}{rgb}{0.000000,0.000000,0.000000}%
\pgfsetstrokecolor{currentstroke}%
\pgfsetdash{}{0pt}%
\pgfpathmoveto{\pgfqpoint{4.011666in}{2.334266in}}%
\pgfpathcurveto{\pgfqpoint{4.022716in}{2.334266in}}{\pgfqpoint{4.033315in}{2.338657in}}{\pgfqpoint{4.041128in}{2.346470in}}%
\pgfpathcurveto{\pgfqpoint{4.048942in}{2.354284in}}{\pgfqpoint{4.053332in}{2.364883in}}{\pgfqpoint{4.053332in}{2.375933in}}%
\pgfpathcurveto{\pgfqpoint{4.053332in}{2.386983in}}{\pgfqpoint{4.048942in}{2.397582in}}{\pgfqpoint{4.041128in}{2.405396in}}%
\pgfpathcurveto{\pgfqpoint{4.033315in}{2.413209in}}{\pgfqpoint{4.022716in}{2.417600in}}{\pgfqpoint{4.011666in}{2.417600in}}%
\pgfpathcurveto{\pgfqpoint{4.000616in}{2.417600in}}{\pgfqpoint{3.990016in}{2.413209in}}{\pgfqpoint{3.982203in}{2.405396in}}%
\pgfpathcurveto{\pgfqpoint{3.974389in}{2.397582in}}{\pgfqpoint{3.969999in}{2.386983in}}{\pgfqpoint{3.969999in}{2.375933in}}%
\pgfpathcurveto{\pgfqpoint{3.969999in}{2.364883in}}{\pgfqpoint{3.974389in}{2.354284in}}{\pgfqpoint{3.982203in}{2.346470in}}%
\pgfpathcurveto{\pgfqpoint{3.990016in}{2.338657in}}{\pgfqpoint{4.000616in}{2.334266in}}{\pgfqpoint{4.011666in}{2.334266in}}%
\pgfpathclose%
\pgfusepath{stroke,fill}%
\end{pgfscope}%
\begin{pgfscope}%
\pgfpathrectangle{\pgfqpoint{0.800000in}{0.528000in}}{\pgfqpoint{4.960000in}{3.696000in}}%
\pgfusepath{clip}%
\pgfsetbuttcap%
\pgfsetroundjoin%
\definecolor{currentfill}{rgb}{0.000000,0.000000,0.000000}%
\pgfsetfillcolor{currentfill}%
\pgfsetlinewidth{1.003750pt}%
\definecolor{currentstroke}{rgb}{0.000000,0.000000,0.000000}%
\pgfsetstrokecolor{currentstroke}%
\pgfsetdash{}{0pt}%
\pgfpathmoveto{\pgfqpoint{4.011666in}{2.334266in}}%
\pgfpathcurveto{\pgfqpoint{4.022716in}{2.334266in}}{\pgfqpoint{4.033315in}{2.338657in}}{\pgfqpoint{4.041128in}{2.346470in}}%
\pgfpathcurveto{\pgfqpoint{4.048942in}{2.354284in}}{\pgfqpoint{4.053332in}{2.364883in}}{\pgfqpoint{4.053332in}{2.375933in}}%
\pgfpathcurveto{\pgfqpoint{4.053332in}{2.386983in}}{\pgfqpoint{4.048942in}{2.397582in}}{\pgfqpoint{4.041128in}{2.405396in}}%
\pgfpathcurveto{\pgfqpoint{4.033315in}{2.413209in}}{\pgfqpoint{4.022716in}{2.417600in}}{\pgfqpoint{4.011666in}{2.417600in}}%
\pgfpathcurveto{\pgfqpoint{4.000616in}{2.417600in}}{\pgfqpoint{3.990016in}{2.413209in}}{\pgfqpoint{3.982203in}{2.405396in}}%
\pgfpathcurveto{\pgfqpoint{3.974389in}{2.397582in}}{\pgfqpoint{3.969999in}{2.386983in}}{\pgfqpoint{3.969999in}{2.375933in}}%
\pgfpathcurveto{\pgfqpoint{3.969999in}{2.364883in}}{\pgfqpoint{3.974389in}{2.354284in}}{\pgfqpoint{3.982203in}{2.346470in}}%
\pgfpathcurveto{\pgfqpoint{3.990016in}{2.338657in}}{\pgfqpoint{4.000616in}{2.334266in}}{\pgfqpoint{4.011666in}{2.334266in}}%
\pgfpathclose%
\pgfusepath{stroke,fill}%
\end{pgfscope}%
\begin{pgfscope}%
\pgfpathrectangle{\pgfqpoint{0.800000in}{0.528000in}}{\pgfqpoint{4.960000in}{3.696000in}}%
\pgfusepath{clip}%
\pgfsetbuttcap%
\pgfsetroundjoin%
\definecolor{currentfill}{rgb}{0.000000,0.000000,0.000000}%
\pgfsetfillcolor{currentfill}%
\pgfsetlinewidth{1.003750pt}%
\definecolor{currentstroke}{rgb}{0.000000,0.000000,0.000000}%
\pgfsetstrokecolor{currentstroke}%
\pgfsetdash{}{0pt}%
\pgfpathmoveto{\pgfqpoint{4.011666in}{2.334266in}}%
\pgfpathcurveto{\pgfqpoint{4.022716in}{2.334266in}}{\pgfqpoint{4.033315in}{2.338657in}}{\pgfqpoint{4.041128in}{2.346470in}}%
\pgfpathcurveto{\pgfqpoint{4.048942in}{2.354284in}}{\pgfqpoint{4.053332in}{2.364883in}}{\pgfqpoint{4.053332in}{2.375933in}}%
\pgfpathcurveto{\pgfqpoint{4.053332in}{2.386983in}}{\pgfqpoint{4.048942in}{2.397582in}}{\pgfqpoint{4.041128in}{2.405396in}}%
\pgfpathcurveto{\pgfqpoint{4.033315in}{2.413209in}}{\pgfqpoint{4.022716in}{2.417600in}}{\pgfqpoint{4.011666in}{2.417600in}}%
\pgfpathcurveto{\pgfqpoint{4.000616in}{2.417600in}}{\pgfqpoint{3.990016in}{2.413209in}}{\pgfqpoint{3.982203in}{2.405396in}}%
\pgfpathcurveto{\pgfqpoint{3.974389in}{2.397582in}}{\pgfqpoint{3.969999in}{2.386983in}}{\pgfqpoint{3.969999in}{2.375933in}}%
\pgfpathcurveto{\pgfqpoint{3.969999in}{2.364883in}}{\pgfqpoint{3.974389in}{2.354284in}}{\pgfqpoint{3.982203in}{2.346470in}}%
\pgfpathcurveto{\pgfqpoint{3.990016in}{2.338657in}}{\pgfqpoint{4.000616in}{2.334266in}}{\pgfqpoint{4.011666in}{2.334266in}}%
\pgfpathclose%
\pgfusepath{stroke,fill}%
\end{pgfscope}%
\begin{pgfscope}%
\pgfpathrectangle{\pgfqpoint{0.800000in}{0.528000in}}{\pgfqpoint{4.960000in}{3.696000in}}%
\pgfusepath{clip}%
\pgfsetbuttcap%
\pgfsetroundjoin%
\definecolor{currentfill}{rgb}{0.000000,0.000000,0.000000}%
\pgfsetfillcolor{currentfill}%
\pgfsetlinewidth{1.003750pt}%
\definecolor{currentstroke}{rgb}{0.000000,0.000000,0.000000}%
\pgfsetstrokecolor{currentstroke}%
\pgfsetdash{}{0pt}%
\pgfpathmoveto{\pgfqpoint{4.011666in}{2.334266in}}%
\pgfpathcurveto{\pgfqpoint{4.022716in}{2.334266in}}{\pgfqpoint{4.033315in}{2.338657in}}{\pgfqpoint{4.041128in}{2.346470in}}%
\pgfpathcurveto{\pgfqpoint{4.048942in}{2.354284in}}{\pgfqpoint{4.053332in}{2.364883in}}{\pgfqpoint{4.053332in}{2.375933in}}%
\pgfpathcurveto{\pgfqpoint{4.053332in}{2.386983in}}{\pgfqpoint{4.048942in}{2.397582in}}{\pgfqpoint{4.041128in}{2.405396in}}%
\pgfpathcurveto{\pgfqpoint{4.033315in}{2.413209in}}{\pgfqpoint{4.022716in}{2.417600in}}{\pgfqpoint{4.011666in}{2.417600in}}%
\pgfpathcurveto{\pgfqpoint{4.000616in}{2.417600in}}{\pgfqpoint{3.990016in}{2.413209in}}{\pgfqpoint{3.982203in}{2.405396in}}%
\pgfpathcurveto{\pgfqpoint{3.974389in}{2.397582in}}{\pgfqpoint{3.969999in}{2.386983in}}{\pgfqpoint{3.969999in}{2.375933in}}%
\pgfpathcurveto{\pgfqpoint{3.969999in}{2.364883in}}{\pgfqpoint{3.974389in}{2.354284in}}{\pgfqpoint{3.982203in}{2.346470in}}%
\pgfpathcurveto{\pgfqpoint{3.990016in}{2.338657in}}{\pgfqpoint{4.000616in}{2.334266in}}{\pgfqpoint{4.011666in}{2.334266in}}%
\pgfpathclose%
\pgfusepath{stroke,fill}%
\end{pgfscope}%
\begin{pgfscope}%
\pgfpathrectangle{\pgfqpoint{0.800000in}{0.528000in}}{\pgfqpoint{4.960000in}{3.696000in}}%
\pgfusepath{clip}%
\pgfsetbuttcap%
\pgfsetroundjoin%
\definecolor{currentfill}{rgb}{0.000000,0.000000,0.000000}%
\pgfsetfillcolor{currentfill}%
\pgfsetlinewidth{1.003750pt}%
\definecolor{currentstroke}{rgb}{0.000000,0.000000,0.000000}%
\pgfsetstrokecolor{currentstroke}%
\pgfsetdash{}{0pt}%
\pgfpathmoveto{\pgfqpoint{4.011666in}{2.334266in}}%
\pgfpathcurveto{\pgfqpoint{4.022716in}{2.334266in}}{\pgfqpoint{4.033315in}{2.338657in}}{\pgfqpoint{4.041128in}{2.346470in}}%
\pgfpathcurveto{\pgfqpoint{4.048942in}{2.354284in}}{\pgfqpoint{4.053332in}{2.364883in}}{\pgfqpoint{4.053332in}{2.375933in}}%
\pgfpathcurveto{\pgfqpoint{4.053332in}{2.386983in}}{\pgfqpoint{4.048942in}{2.397582in}}{\pgfqpoint{4.041128in}{2.405396in}}%
\pgfpathcurveto{\pgfqpoint{4.033315in}{2.413209in}}{\pgfqpoint{4.022716in}{2.417600in}}{\pgfqpoint{4.011666in}{2.417600in}}%
\pgfpathcurveto{\pgfqpoint{4.000616in}{2.417600in}}{\pgfqpoint{3.990016in}{2.413209in}}{\pgfqpoint{3.982203in}{2.405396in}}%
\pgfpathcurveto{\pgfqpoint{3.974389in}{2.397582in}}{\pgfqpoint{3.969999in}{2.386983in}}{\pgfqpoint{3.969999in}{2.375933in}}%
\pgfpathcurveto{\pgfqpoint{3.969999in}{2.364883in}}{\pgfqpoint{3.974389in}{2.354284in}}{\pgfqpoint{3.982203in}{2.346470in}}%
\pgfpathcurveto{\pgfqpoint{3.990016in}{2.338657in}}{\pgfqpoint{4.000616in}{2.334266in}}{\pgfqpoint{4.011666in}{2.334266in}}%
\pgfpathclose%
\pgfusepath{stroke,fill}%
\end{pgfscope}%
\begin{pgfscope}%
\pgfpathrectangle{\pgfqpoint{0.800000in}{0.528000in}}{\pgfqpoint{4.960000in}{3.696000in}}%
\pgfusepath{clip}%
\pgfsetbuttcap%
\pgfsetroundjoin%
\definecolor{currentfill}{rgb}{0.000000,0.000000,0.000000}%
\pgfsetfillcolor{currentfill}%
\pgfsetlinewidth{1.003750pt}%
\definecolor{currentstroke}{rgb}{0.000000,0.000000,0.000000}%
\pgfsetstrokecolor{currentstroke}%
\pgfsetdash{}{0pt}%
\pgfpathmoveto{\pgfqpoint{4.011666in}{2.334266in}}%
\pgfpathcurveto{\pgfqpoint{4.022716in}{2.334266in}}{\pgfqpoint{4.033315in}{2.338657in}}{\pgfqpoint{4.041128in}{2.346470in}}%
\pgfpathcurveto{\pgfqpoint{4.048942in}{2.354284in}}{\pgfqpoint{4.053332in}{2.364883in}}{\pgfqpoint{4.053332in}{2.375933in}}%
\pgfpathcurveto{\pgfqpoint{4.053332in}{2.386983in}}{\pgfqpoint{4.048942in}{2.397582in}}{\pgfqpoint{4.041128in}{2.405396in}}%
\pgfpathcurveto{\pgfqpoint{4.033315in}{2.413209in}}{\pgfqpoint{4.022716in}{2.417600in}}{\pgfqpoint{4.011666in}{2.417600in}}%
\pgfpathcurveto{\pgfqpoint{4.000616in}{2.417600in}}{\pgfqpoint{3.990016in}{2.413209in}}{\pgfqpoint{3.982203in}{2.405396in}}%
\pgfpathcurveto{\pgfqpoint{3.974389in}{2.397582in}}{\pgfqpoint{3.969999in}{2.386983in}}{\pgfqpoint{3.969999in}{2.375933in}}%
\pgfpathcurveto{\pgfqpoint{3.969999in}{2.364883in}}{\pgfqpoint{3.974389in}{2.354284in}}{\pgfqpoint{3.982203in}{2.346470in}}%
\pgfpathcurveto{\pgfqpoint{3.990016in}{2.338657in}}{\pgfqpoint{4.000616in}{2.334266in}}{\pgfqpoint{4.011666in}{2.334266in}}%
\pgfpathclose%
\pgfusepath{stroke,fill}%
\end{pgfscope}%
\begin{pgfscope}%
\pgfpathrectangle{\pgfqpoint{0.800000in}{0.528000in}}{\pgfqpoint{4.960000in}{3.696000in}}%
\pgfusepath{clip}%
\pgfsetbuttcap%
\pgfsetroundjoin%
\definecolor{currentfill}{rgb}{0.000000,0.000000,0.000000}%
\pgfsetfillcolor{currentfill}%
\pgfsetlinewidth{1.003750pt}%
\definecolor{currentstroke}{rgb}{0.000000,0.000000,0.000000}%
\pgfsetstrokecolor{currentstroke}%
\pgfsetdash{}{0pt}%
\pgfpathmoveto{\pgfqpoint{4.011666in}{3.984333in}}%
\pgfpathcurveto{\pgfqpoint{4.022716in}{3.984333in}}{\pgfqpoint{4.033315in}{3.988724in}}{\pgfqpoint{4.041128in}{3.996537in}}%
\pgfpathcurveto{\pgfqpoint{4.048942in}{4.004351in}}{\pgfqpoint{4.053332in}{4.014950in}}{\pgfqpoint{4.053332in}{4.026000in}}%
\pgfpathcurveto{\pgfqpoint{4.053332in}{4.037050in}}{\pgfqpoint{4.048942in}{4.047649in}}{\pgfqpoint{4.041128in}{4.055463in}}%
\pgfpathcurveto{\pgfqpoint{4.033315in}{4.063276in}}{\pgfqpoint{4.022716in}{4.067667in}}{\pgfqpoint{4.011666in}{4.067667in}}%
\pgfpathcurveto{\pgfqpoint{4.000616in}{4.067667in}}{\pgfqpoint{3.990016in}{4.063276in}}{\pgfqpoint{3.982203in}{4.055463in}}%
\pgfpathcurveto{\pgfqpoint{3.974389in}{4.047649in}}{\pgfqpoint{3.969999in}{4.037050in}}{\pgfqpoint{3.969999in}{4.026000in}}%
\pgfpathcurveto{\pgfqpoint{3.969999in}{4.014950in}}{\pgfqpoint{3.974389in}{4.004351in}}{\pgfqpoint{3.982203in}{3.996537in}}%
\pgfpathcurveto{\pgfqpoint{3.990016in}{3.988724in}}{\pgfqpoint{4.000616in}{3.984333in}}{\pgfqpoint{4.011666in}{3.984333in}}%
\pgfpathclose%
\pgfusepath{stroke,fill}%
\end{pgfscope}%
\begin{pgfscope}%
\pgfpathrectangle{\pgfqpoint{0.800000in}{0.528000in}}{\pgfqpoint{4.960000in}{3.696000in}}%
\pgfusepath{clip}%
\pgfsetbuttcap%
\pgfsetroundjoin%
\definecolor{currentfill}{rgb}{0.000000,0.000000,0.000000}%
\pgfsetfillcolor{currentfill}%
\pgfsetlinewidth{1.003750pt}%
\definecolor{currentstroke}{rgb}{0.000000,0.000000,0.000000}%
\pgfsetstrokecolor{currentstroke}%
\pgfsetdash{}{0pt}%
\pgfpathmoveto{\pgfqpoint{4.011666in}{3.984333in}}%
\pgfpathcurveto{\pgfqpoint{4.022716in}{3.984333in}}{\pgfqpoint{4.033315in}{3.988724in}}{\pgfqpoint{4.041128in}{3.996537in}}%
\pgfpathcurveto{\pgfqpoint{4.048942in}{4.004351in}}{\pgfqpoint{4.053332in}{4.014950in}}{\pgfqpoint{4.053332in}{4.026000in}}%
\pgfpathcurveto{\pgfqpoint{4.053332in}{4.037050in}}{\pgfqpoint{4.048942in}{4.047649in}}{\pgfqpoint{4.041128in}{4.055463in}}%
\pgfpathcurveto{\pgfqpoint{4.033315in}{4.063276in}}{\pgfqpoint{4.022716in}{4.067667in}}{\pgfqpoint{4.011666in}{4.067667in}}%
\pgfpathcurveto{\pgfqpoint{4.000616in}{4.067667in}}{\pgfqpoint{3.990016in}{4.063276in}}{\pgfqpoint{3.982203in}{4.055463in}}%
\pgfpathcurveto{\pgfqpoint{3.974389in}{4.047649in}}{\pgfqpoint{3.969999in}{4.037050in}}{\pgfqpoint{3.969999in}{4.026000in}}%
\pgfpathcurveto{\pgfqpoint{3.969999in}{4.014950in}}{\pgfqpoint{3.974389in}{4.004351in}}{\pgfqpoint{3.982203in}{3.996537in}}%
\pgfpathcurveto{\pgfqpoint{3.990016in}{3.988724in}}{\pgfqpoint{4.000616in}{3.984333in}}{\pgfqpoint{4.011666in}{3.984333in}}%
\pgfpathclose%
\pgfusepath{stroke,fill}%
\end{pgfscope}%
\begin{pgfscope}%
\pgfpathrectangle{\pgfqpoint{0.800000in}{0.528000in}}{\pgfqpoint{4.960000in}{3.696000in}}%
\pgfusepath{clip}%
\pgfsetbuttcap%
\pgfsetroundjoin%
\definecolor{currentfill}{rgb}{0.000000,0.000000,0.000000}%
\pgfsetfillcolor{currentfill}%
\pgfsetlinewidth{1.003750pt}%
\definecolor{currentstroke}{rgb}{0.000000,0.000000,0.000000}%
\pgfsetstrokecolor{currentstroke}%
\pgfsetdash{}{0pt}%
\pgfpathmoveto{\pgfqpoint{4.011666in}{2.334266in}}%
\pgfpathcurveto{\pgfqpoint{4.022716in}{2.334266in}}{\pgfqpoint{4.033315in}{2.338657in}}{\pgfqpoint{4.041128in}{2.346470in}}%
\pgfpathcurveto{\pgfqpoint{4.048942in}{2.354284in}}{\pgfqpoint{4.053332in}{2.364883in}}{\pgfqpoint{4.053332in}{2.375933in}}%
\pgfpathcurveto{\pgfqpoint{4.053332in}{2.386983in}}{\pgfqpoint{4.048942in}{2.397582in}}{\pgfqpoint{4.041128in}{2.405396in}}%
\pgfpathcurveto{\pgfqpoint{4.033315in}{2.413209in}}{\pgfqpoint{4.022716in}{2.417600in}}{\pgfqpoint{4.011666in}{2.417600in}}%
\pgfpathcurveto{\pgfqpoint{4.000616in}{2.417600in}}{\pgfqpoint{3.990016in}{2.413209in}}{\pgfqpoint{3.982203in}{2.405396in}}%
\pgfpathcurveto{\pgfqpoint{3.974389in}{2.397582in}}{\pgfqpoint{3.969999in}{2.386983in}}{\pgfqpoint{3.969999in}{2.375933in}}%
\pgfpathcurveto{\pgfqpoint{3.969999in}{2.364883in}}{\pgfqpoint{3.974389in}{2.354284in}}{\pgfqpoint{3.982203in}{2.346470in}}%
\pgfpathcurveto{\pgfqpoint{3.990016in}{2.338657in}}{\pgfqpoint{4.000616in}{2.334266in}}{\pgfqpoint{4.011666in}{2.334266in}}%
\pgfpathclose%
\pgfusepath{stroke,fill}%
\end{pgfscope}%
\begin{pgfscope}%
\pgfpathrectangle{\pgfqpoint{0.800000in}{0.528000in}}{\pgfqpoint{4.960000in}{3.696000in}}%
\pgfusepath{clip}%
\pgfsetbuttcap%
\pgfsetroundjoin%
\definecolor{currentfill}{rgb}{0.000000,0.000000,0.000000}%
\pgfsetfillcolor{currentfill}%
\pgfsetlinewidth{1.003750pt}%
\definecolor{currentstroke}{rgb}{0.000000,0.000000,0.000000}%
\pgfsetstrokecolor{currentstroke}%
\pgfsetdash{}{0pt}%
\pgfpathmoveto{\pgfqpoint{4.011666in}{3.984333in}}%
\pgfpathcurveto{\pgfqpoint{4.022716in}{3.984333in}}{\pgfqpoint{4.033315in}{3.988724in}}{\pgfqpoint{4.041128in}{3.996537in}}%
\pgfpathcurveto{\pgfqpoint{4.048942in}{4.004351in}}{\pgfqpoint{4.053332in}{4.014950in}}{\pgfqpoint{4.053332in}{4.026000in}}%
\pgfpathcurveto{\pgfqpoint{4.053332in}{4.037050in}}{\pgfqpoint{4.048942in}{4.047649in}}{\pgfqpoint{4.041128in}{4.055463in}}%
\pgfpathcurveto{\pgfqpoint{4.033315in}{4.063276in}}{\pgfqpoint{4.022716in}{4.067667in}}{\pgfqpoint{4.011666in}{4.067667in}}%
\pgfpathcurveto{\pgfqpoint{4.000616in}{4.067667in}}{\pgfqpoint{3.990016in}{4.063276in}}{\pgfqpoint{3.982203in}{4.055463in}}%
\pgfpathcurveto{\pgfqpoint{3.974389in}{4.047649in}}{\pgfqpoint{3.969999in}{4.037050in}}{\pgfqpoint{3.969999in}{4.026000in}}%
\pgfpathcurveto{\pgfqpoint{3.969999in}{4.014950in}}{\pgfqpoint{3.974389in}{4.004351in}}{\pgfqpoint{3.982203in}{3.996537in}}%
\pgfpathcurveto{\pgfqpoint{3.990016in}{3.988724in}}{\pgfqpoint{4.000616in}{3.984333in}}{\pgfqpoint{4.011666in}{3.984333in}}%
\pgfpathclose%
\pgfusepath{stroke,fill}%
\end{pgfscope}%
\begin{pgfscope}%
\pgfpathrectangle{\pgfqpoint{0.800000in}{0.528000in}}{\pgfqpoint{4.960000in}{3.696000in}}%
\pgfusepath{clip}%
\pgfsetbuttcap%
\pgfsetroundjoin%
\definecolor{currentfill}{rgb}{0.000000,0.000000,0.000000}%
\pgfsetfillcolor{currentfill}%
\pgfsetlinewidth{1.003750pt}%
\definecolor{currentstroke}{rgb}{0.000000,0.000000,0.000000}%
\pgfsetstrokecolor{currentstroke}%
\pgfsetdash{}{0pt}%
\pgfpathmoveto{\pgfqpoint{4.011666in}{2.334266in}}%
\pgfpathcurveto{\pgfqpoint{4.022716in}{2.334266in}}{\pgfqpoint{4.033315in}{2.338657in}}{\pgfqpoint{4.041128in}{2.346470in}}%
\pgfpathcurveto{\pgfqpoint{4.048942in}{2.354284in}}{\pgfqpoint{4.053332in}{2.364883in}}{\pgfqpoint{4.053332in}{2.375933in}}%
\pgfpathcurveto{\pgfqpoint{4.053332in}{2.386983in}}{\pgfqpoint{4.048942in}{2.397582in}}{\pgfqpoint{4.041128in}{2.405396in}}%
\pgfpathcurveto{\pgfqpoint{4.033315in}{2.413209in}}{\pgfqpoint{4.022716in}{2.417600in}}{\pgfqpoint{4.011666in}{2.417600in}}%
\pgfpathcurveto{\pgfqpoint{4.000616in}{2.417600in}}{\pgfqpoint{3.990016in}{2.413209in}}{\pgfqpoint{3.982203in}{2.405396in}}%
\pgfpathcurveto{\pgfqpoint{3.974389in}{2.397582in}}{\pgfqpoint{3.969999in}{2.386983in}}{\pgfqpoint{3.969999in}{2.375933in}}%
\pgfpathcurveto{\pgfqpoint{3.969999in}{2.364883in}}{\pgfqpoint{3.974389in}{2.354284in}}{\pgfqpoint{3.982203in}{2.346470in}}%
\pgfpathcurveto{\pgfqpoint{3.990016in}{2.338657in}}{\pgfqpoint{4.000616in}{2.334266in}}{\pgfqpoint{4.011666in}{2.334266in}}%
\pgfpathclose%
\pgfusepath{stroke,fill}%
\end{pgfscope}%
\begin{pgfscope}%
\pgfpathrectangle{\pgfqpoint{0.800000in}{0.528000in}}{\pgfqpoint{4.960000in}{3.696000in}}%
\pgfusepath{clip}%
\pgfsetbuttcap%
\pgfsetroundjoin%
\definecolor{currentfill}{rgb}{0.000000,0.000000,0.000000}%
\pgfsetfillcolor{currentfill}%
\pgfsetlinewidth{1.003750pt}%
\definecolor{currentstroke}{rgb}{0.000000,0.000000,0.000000}%
\pgfsetstrokecolor{currentstroke}%
\pgfsetdash{}{0pt}%
\pgfpathmoveto{\pgfqpoint{4.011666in}{2.334266in}}%
\pgfpathcurveto{\pgfqpoint{4.022716in}{2.334266in}}{\pgfqpoint{4.033315in}{2.338657in}}{\pgfqpoint{4.041128in}{2.346470in}}%
\pgfpathcurveto{\pgfqpoint{4.048942in}{2.354284in}}{\pgfqpoint{4.053332in}{2.364883in}}{\pgfqpoint{4.053332in}{2.375933in}}%
\pgfpathcurveto{\pgfqpoint{4.053332in}{2.386983in}}{\pgfqpoint{4.048942in}{2.397582in}}{\pgfqpoint{4.041128in}{2.405396in}}%
\pgfpathcurveto{\pgfqpoint{4.033315in}{2.413209in}}{\pgfqpoint{4.022716in}{2.417600in}}{\pgfqpoint{4.011666in}{2.417600in}}%
\pgfpathcurveto{\pgfqpoint{4.000616in}{2.417600in}}{\pgfqpoint{3.990016in}{2.413209in}}{\pgfqpoint{3.982203in}{2.405396in}}%
\pgfpathcurveto{\pgfqpoint{3.974389in}{2.397582in}}{\pgfqpoint{3.969999in}{2.386983in}}{\pgfqpoint{3.969999in}{2.375933in}}%
\pgfpathcurveto{\pgfqpoint{3.969999in}{2.364883in}}{\pgfqpoint{3.974389in}{2.354284in}}{\pgfqpoint{3.982203in}{2.346470in}}%
\pgfpathcurveto{\pgfqpoint{3.990016in}{2.338657in}}{\pgfqpoint{4.000616in}{2.334266in}}{\pgfqpoint{4.011666in}{2.334266in}}%
\pgfpathclose%
\pgfusepath{stroke,fill}%
\end{pgfscope}%
\begin{pgfscope}%
\pgfpathrectangle{\pgfqpoint{0.800000in}{0.528000in}}{\pgfqpoint{4.960000in}{3.696000in}}%
\pgfusepath{clip}%
\pgfsetbuttcap%
\pgfsetroundjoin%
\definecolor{currentfill}{rgb}{0.000000,0.000000,0.000000}%
\pgfsetfillcolor{currentfill}%
\pgfsetlinewidth{1.003750pt}%
\definecolor{currentstroke}{rgb}{0.000000,0.000000,0.000000}%
\pgfsetstrokecolor{currentstroke}%
\pgfsetdash{}{0pt}%
\pgfpathmoveto{\pgfqpoint{4.011666in}{2.334266in}}%
\pgfpathcurveto{\pgfqpoint{4.022716in}{2.334266in}}{\pgfqpoint{4.033315in}{2.338657in}}{\pgfqpoint{4.041128in}{2.346470in}}%
\pgfpathcurveto{\pgfqpoint{4.048942in}{2.354284in}}{\pgfqpoint{4.053332in}{2.364883in}}{\pgfqpoint{4.053332in}{2.375933in}}%
\pgfpathcurveto{\pgfqpoint{4.053332in}{2.386983in}}{\pgfqpoint{4.048942in}{2.397582in}}{\pgfqpoint{4.041128in}{2.405396in}}%
\pgfpathcurveto{\pgfqpoint{4.033315in}{2.413209in}}{\pgfqpoint{4.022716in}{2.417600in}}{\pgfqpoint{4.011666in}{2.417600in}}%
\pgfpathcurveto{\pgfqpoint{4.000616in}{2.417600in}}{\pgfqpoint{3.990016in}{2.413209in}}{\pgfqpoint{3.982203in}{2.405396in}}%
\pgfpathcurveto{\pgfqpoint{3.974389in}{2.397582in}}{\pgfqpoint{3.969999in}{2.386983in}}{\pgfqpoint{3.969999in}{2.375933in}}%
\pgfpathcurveto{\pgfqpoint{3.969999in}{2.364883in}}{\pgfqpoint{3.974389in}{2.354284in}}{\pgfqpoint{3.982203in}{2.346470in}}%
\pgfpathcurveto{\pgfqpoint{3.990016in}{2.338657in}}{\pgfqpoint{4.000616in}{2.334266in}}{\pgfqpoint{4.011666in}{2.334266in}}%
\pgfpathclose%
\pgfusepath{stroke,fill}%
\end{pgfscope}%
\begin{pgfscope}%
\pgfpathrectangle{\pgfqpoint{0.800000in}{0.528000in}}{\pgfqpoint{4.960000in}{3.696000in}}%
\pgfusepath{clip}%
\pgfsetbuttcap%
\pgfsetroundjoin%
\definecolor{currentfill}{rgb}{0.000000,0.000000,0.000000}%
\pgfsetfillcolor{currentfill}%
\pgfsetlinewidth{1.003750pt}%
\definecolor{currentstroke}{rgb}{0.000000,0.000000,0.000000}%
\pgfsetstrokecolor{currentstroke}%
\pgfsetdash{}{0pt}%
\pgfpathmoveto{\pgfqpoint{4.011666in}{2.334266in}}%
\pgfpathcurveto{\pgfqpoint{4.022716in}{2.334266in}}{\pgfqpoint{4.033315in}{2.338657in}}{\pgfqpoint{4.041128in}{2.346470in}}%
\pgfpathcurveto{\pgfqpoint{4.048942in}{2.354284in}}{\pgfqpoint{4.053332in}{2.364883in}}{\pgfqpoint{4.053332in}{2.375933in}}%
\pgfpathcurveto{\pgfqpoint{4.053332in}{2.386983in}}{\pgfqpoint{4.048942in}{2.397582in}}{\pgfqpoint{4.041128in}{2.405396in}}%
\pgfpathcurveto{\pgfqpoint{4.033315in}{2.413209in}}{\pgfqpoint{4.022716in}{2.417600in}}{\pgfqpoint{4.011666in}{2.417600in}}%
\pgfpathcurveto{\pgfqpoint{4.000616in}{2.417600in}}{\pgfqpoint{3.990016in}{2.413209in}}{\pgfqpoint{3.982203in}{2.405396in}}%
\pgfpathcurveto{\pgfqpoint{3.974389in}{2.397582in}}{\pgfqpoint{3.969999in}{2.386983in}}{\pgfqpoint{3.969999in}{2.375933in}}%
\pgfpathcurveto{\pgfqpoint{3.969999in}{2.364883in}}{\pgfqpoint{3.974389in}{2.354284in}}{\pgfqpoint{3.982203in}{2.346470in}}%
\pgfpathcurveto{\pgfqpoint{3.990016in}{2.338657in}}{\pgfqpoint{4.000616in}{2.334266in}}{\pgfqpoint{4.011666in}{2.334266in}}%
\pgfpathclose%
\pgfusepath{stroke,fill}%
\end{pgfscope}%
\begin{pgfscope}%
\pgfpathrectangle{\pgfqpoint{0.800000in}{0.528000in}}{\pgfqpoint{4.960000in}{3.696000in}}%
\pgfusepath{clip}%
\pgfsetbuttcap%
\pgfsetroundjoin%
\definecolor{currentfill}{rgb}{0.000000,0.000000,0.000000}%
\pgfsetfillcolor{currentfill}%
\pgfsetlinewidth{1.003750pt}%
\definecolor{currentstroke}{rgb}{0.000000,0.000000,0.000000}%
\pgfsetstrokecolor{currentstroke}%
\pgfsetdash{}{0pt}%
\pgfpathmoveto{\pgfqpoint{4.011666in}{2.334266in}}%
\pgfpathcurveto{\pgfqpoint{4.022716in}{2.334266in}}{\pgfqpoint{4.033315in}{2.338657in}}{\pgfqpoint{4.041128in}{2.346470in}}%
\pgfpathcurveto{\pgfqpoint{4.048942in}{2.354284in}}{\pgfqpoint{4.053332in}{2.364883in}}{\pgfqpoint{4.053332in}{2.375933in}}%
\pgfpathcurveto{\pgfqpoint{4.053332in}{2.386983in}}{\pgfqpoint{4.048942in}{2.397582in}}{\pgfqpoint{4.041128in}{2.405396in}}%
\pgfpathcurveto{\pgfqpoint{4.033315in}{2.413209in}}{\pgfqpoint{4.022716in}{2.417600in}}{\pgfqpoint{4.011666in}{2.417600in}}%
\pgfpathcurveto{\pgfqpoint{4.000616in}{2.417600in}}{\pgfqpoint{3.990016in}{2.413209in}}{\pgfqpoint{3.982203in}{2.405396in}}%
\pgfpathcurveto{\pgfqpoint{3.974389in}{2.397582in}}{\pgfqpoint{3.969999in}{2.386983in}}{\pgfqpoint{3.969999in}{2.375933in}}%
\pgfpathcurveto{\pgfqpoint{3.969999in}{2.364883in}}{\pgfqpoint{3.974389in}{2.354284in}}{\pgfqpoint{3.982203in}{2.346470in}}%
\pgfpathcurveto{\pgfqpoint{3.990016in}{2.338657in}}{\pgfqpoint{4.000616in}{2.334266in}}{\pgfqpoint{4.011666in}{2.334266in}}%
\pgfpathclose%
\pgfusepath{stroke,fill}%
\end{pgfscope}%
\begin{pgfscope}%
\pgfpathrectangle{\pgfqpoint{0.800000in}{0.528000in}}{\pgfqpoint{4.960000in}{3.696000in}}%
\pgfusepath{clip}%
\pgfsetbuttcap%
\pgfsetroundjoin%
\definecolor{currentfill}{rgb}{0.000000,0.000000,0.000000}%
\pgfsetfillcolor{currentfill}%
\pgfsetlinewidth{1.003750pt}%
\definecolor{currentstroke}{rgb}{0.000000,0.000000,0.000000}%
\pgfsetstrokecolor{currentstroke}%
\pgfsetdash{}{0pt}%
\pgfpathmoveto{\pgfqpoint{4.011666in}{2.334266in}}%
\pgfpathcurveto{\pgfqpoint{4.022716in}{2.334266in}}{\pgfqpoint{4.033315in}{2.338657in}}{\pgfqpoint{4.041128in}{2.346470in}}%
\pgfpathcurveto{\pgfqpoint{4.048942in}{2.354284in}}{\pgfqpoint{4.053332in}{2.364883in}}{\pgfqpoint{4.053332in}{2.375933in}}%
\pgfpathcurveto{\pgfqpoint{4.053332in}{2.386983in}}{\pgfqpoint{4.048942in}{2.397582in}}{\pgfqpoint{4.041128in}{2.405396in}}%
\pgfpathcurveto{\pgfqpoint{4.033315in}{2.413209in}}{\pgfqpoint{4.022716in}{2.417600in}}{\pgfqpoint{4.011666in}{2.417600in}}%
\pgfpathcurveto{\pgfqpoint{4.000616in}{2.417600in}}{\pgfqpoint{3.990016in}{2.413209in}}{\pgfqpoint{3.982203in}{2.405396in}}%
\pgfpathcurveto{\pgfqpoint{3.974389in}{2.397582in}}{\pgfqpoint{3.969999in}{2.386983in}}{\pgfqpoint{3.969999in}{2.375933in}}%
\pgfpathcurveto{\pgfqpoint{3.969999in}{2.364883in}}{\pgfqpoint{3.974389in}{2.354284in}}{\pgfqpoint{3.982203in}{2.346470in}}%
\pgfpathcurveto{\pgfqpoint{3.990016in}{2.338657in}}{\pgfqpoint{4.000616in}{2.334266in}}{\pgfqpoint{4.011666in}{2.334266in}}%
\pgfpathclose%
\pgfusepath{stroke,fill}%
\end{pgfscope}%
\begin{pgfscope}%
\pgfpathrectangle{\pgfqpoint{0.800000in}{0.528000in}}{\pgfqpoint{4.960000in}{3.696000in}}%
\pgfusepath{clip}%
\pgfsetbuttcap%
\pgfsetroundjoin%
\definecolor{currentfill}{rgb}{0.000000,0.000000,0.000000}%
\pgfsetfillcolor{currentfill}%
\pgfsetlinewidth{1.003750pt}%
\definecolor{currentstroke}{rgb}{0.000000,0.000000,0.000000}%
\pgfsetstrokecolor{currentstroke}%
\pgfsetdash{}{0pt}%
\pgfpathmoveto{\pgfqpoint{4.011666in}{2.334266in}}%
\pgfpathcurveto{\pgfqpoint{4.022716in}{2.334266in}}{\pgfqpoint{4.033315in}{2.338657in}}{\pgfqpoint{4.041128in}{2.346470in}}%
\pgfpathcurveto{\pgfqpoint{4.048942in}{2.354284in}}{\pgfqpoint{4.053332in}{2.364883in}}{\pgfqpoint{4.053332in}{2.375933in}}%
\pgfpathcurveto{\pgfqpoint{4.053332in}{2.386983in}}{\pgfqpoint{4.048942in}{2.397582in}}{\pgfqpoint{4.041128in}{2.405396in}}%
\pgfpathcurveto{\pgfqpoint{4.033315in}{2.413209in}}{\pgfqpoint{4.022716in}{2.417600in}}{\pgfqpoint{4.011666in}{2.417600in}}%
\pgfpathcurveto{\pgfqpoint{4.000616in}{2.417600in}}{\pgfqpoint{3.990016in}{2.413209in}}{\pgfqpoint{3.982203in}{2.405396in}}%
\pgfpathcurveto{\pgfqpoint{3.974389in}{2.397582in}}{\pgfqpoint{3.969999in}{2.386983in}}{\pgfqpoint{3.969999in}{2.375933in}}%
\pgfpathcurveto{\pgfqpoint{3.969999in}{2.364883in}}{\pgfqpoint{3.974389in}{2.354284in}}{\pgfqpoint{3.982203in}{2.346470in}}%
\pgfpathcurveto{\pgfqpoint{3.990016in}{2.338657in}}{\pgfqpoint{4.000616in}{2.334266in}}{\pgfqpoint{4.011666in}{2.334266in}}%
\pgfpathclose%
\pgfusepath{stroke,fill}%
\end{pgfscope}%
\begin{pgfscope}%
\pgfpathrectangle{\pgfqpoint{0.800000in}{0.528000in}}{\pgfqpoint{4.960000in}{3.696000in}}%
\pgfusepath{clip}%
\pgfsetbuttcap%
\pgfsetroundjoin%
\definecolor{currentfill}{rgb}{0.000000,0.000000,0.000000}%
\pgfsetfillcolor{currentfill}%
\pgfsetlinewidth{1.003750pt}%
\definecolor{currentstroke}{rgb}{0.000000,0.000000,0.000000}%
\pgfsetstrokecolor{currentstroke}%
\pgfsetdash{}{0pt}%
\pgfpathmoveto{\pgfqpoint{4.011666in}{2.334266in}}%
\pgfpathcurveto{\pgfqpoint{4.022716in}{2.334266in}}{\pgfqpoint{4.033315in}{2.338657in}}{\pgfqpoint{4.041128in}{2.346470in}}%
\pgfpathcurveto{\pgfqpoint{4.048942in}{2.354284in}}{\pgfqpoint{4.053332in}{2.364883in}}{\pgfqpoint{4.053332in}{2.375933in}}%
\pgfpathcurveto{\pgfqpoint{4.053332in}{2.386983in}}{\pgfqpoint{4.048942in}{2.397582in}}{\pgfqpoint{4.041128in}{2.405396in}}%
\pgfpathcurveto{\pgfqpoint{4.033315in}{2.413209in}}{\pgfqpoint{4.022716in}{2.417600in}}{\pgfqpoint{4.011666in}{2.417600in}}%
\pgfpathcurveto{\pgfqpoint{4.000616in}{2.417600in}}{\pgfqpoint{3.990016in}{2.413209in}}{\pgfqpoint{3.982203in}{2.405396in}}%
\pgfpathcurveto{\pgfqpoint{3.974389in}{2.397582in}}{\pgfqpoint{3.969999in}{2.386983in}}{\pgfqpoint{3.969999in}{2.375933in}}%
\pgfpathcurveto{\pgfqpoint{3.969999in}{2.364883in}}{\pgfqpoint{3.974389in}{2.354284in}}{\pgfqpoint{3.982203in}{2.346470in}}%
\pgfpathcurveto{\pgfqpoint{3.990016in}{2.338657in}}{\pgfqpoint{4.000616in}{2.334266in}}{\pgfqpoint{4.011666in}{2.334266in}}%
\pgfpathclose%
\pgfusepath{stroke,fill}%
\end{pgfscope}%
\begin{pgfscope}%
\pgfpathrectangle{\pgfqpoint{0.800000in}{0.528000in}}{\pgfqpoint{4.960000in}{3.696000in}}%
\pgfusepath{clip}%
\pgfsetbuttcap%
\pgfsetroundjoin%
\definecolor{currentfill}{rgb}{0.000000,0.000000,0.000000}%
\pgfsetfillcolor{currentfill}%
\pgfsetlinewidth{1.003750pt}%
\definecolor{currentstroke}{rgb}{0.000000,0.000000,0.000000}%
\pgfsetstrokecolor{currentstroke}%
\pgfsetdash{}{0pt}%
\pgfpathmoveto{\pgfqpoint{4.011666in}{2.334266in}}%
\pgfpathcurveto{\pgfqpoint{4.022716in}{2.334266in}}{\pgfqpoint{4.033315in}{2.338657in}}{\pgfqpoint{4.041128in}{2.346470in}}%
\pgfpathcurveto{\pgfqpoint{4.048942in}{2.354284in}}{\pgfqpoint{4.053332in}{2.364883in}}{\pgfqpoint{4.053332in}{2.375933in}}%
\pgfpathcurveto{\pgfqpoint{4.053332in}{2.386983in}}{\pgfqpoint{4.048942in}{2.397582in}}{\pgfqpoint{4.041128in}{2.405396in}}%
\pgfpathcurveto{\pgfqpoint{4.033315in}{2.413209in}}{\pgfqpoint{4.022716in}{2.417600in}}{\pgfqpoint{4.011666in}{2.417600in}}%
\pgfpathcurveto{\pgfqpoint{4.000616in}{2.417600in}}{\pgfqpoint{3.990016in}{2.413209in}}{\pgfqpoint{3.982203in}{2.405396in}}%
\pgfpathcurveto{\pgfqpoint{3.974389in}{2.397582in}}{\pgfqpoint{3.969999in}{2.386983in}}{\pgfqpoint{3.969999in}{2.375933in}}%
\pgfpathcurveto{\pgfqpoint{3.969999in}{2.364883in}}{\pgfqpoint{3.974389in}{2.354284in}}{\pgfqpoint{3.982203in}{2.346470in}}%
\pgfpathcurveto{\pgfqpoint{3.990016in}{2.338657in}}{\pgfqpoint{4.000616in}{2.334266in}}{\pgfqpoint{4.011666in}{2.334266in}}%
\pgfpathclose%
\pgfusepath{stroke,fill}%
\end{pgfscope}%
\begin{pgfscope}%
\pgfpathrectangle{\pgfqpoint{0.800000in}{0.528000in}}{\pgfqpoint{4.960000in}{3.696000in}}%
\pgfusepath{clip}%
\pgfsetbuttcap%
\pgfsetroundjoin%
\definecolor{currentfill}{rgb}{0.000000,0.000000,0.000000}%
\pgfsetfillcolor{currentfill}%
\pgfsetlinewidth{1.003750pt}%
\definecolor{currentstroke}{rgb}{0.000000,0.000000,0.000000}%
\pgfsetstrokecolor{currentstroke}%
\pgfsetdash{}{0pt}%
\pgfpathmoveto{\pgfqpoint{4.011666in}{2.334266in}}%
\pgfpathcurveto{\pgfqpoint{4.022716in}{2.334266in}}{\pgfqpoint{4.033315in}{2.338657in}}{\pgfqpoint{4.041128in}{2.346470in}}%
\pgfpathcurveto{\pgfqpoint{4.048942in}{2.354284in}}{\pgfqpoint{4.053332in}{2.364883in}}{\pgfqpoint{4.053332in}{2.375933in}}%
\pgfpathcurveto{\pgfqpoint{4.053332in}{2.386983in}}{\pgfqpoint{4.048942in}{2.397582in}}{\pgfqpoint{4.041128in}{2.405396in}}%
\pgfpathcurveto{\pgfqpoint{4.033315in}{2.413209in}}{\pgfqpoint{4.022716in}{2.417600in}}{\pgfqpoint{4.011666in}{2.417600in}}%
\pgfpathcurveto{\pgfqpoint{4.000616in}{2.417600in}}{\pgfqpoint{3.990016in}{2.413209in}}{\pgfqpoint{3.982203in}{2.405396in}}%
\pgfpathcurveto{\pgfqpoint{3.974389in}{2.397582in}}{\pgfqpoint{3.969999in}{2.386983in}}{\pgfqpoint{3.969999in}{2.375933in}}%
\pgfpathcurveto{\pgfqpoint{3.969999in}{2.364883in}}{\pgfqpoint{3.974389in}{2.354284in}}{\pgfqpoint{3.982203in}{2.346470in}}%
\pgfpathcurveto{\pgfqpoint{3.990016in}{2.338657in}}{\pgfqpoint{4.000616in}{2.334266in}}{\pgfqpoint{4.011666in}{2.334266in}}%
\pgfpathclose%
\pgfusepath{stroke,fill}%
\end{pgfscope}%
\begin{pgfscope}%
\pgfpathrectangle{\pgfqpoint{0.800000in}{0.528000in}}{\pgfqpoint{4.960000in}{3.696000in}}%
\pgfusepath{clip}%
\pgfsetbuttcap%
\pgfsetroundjoin%
\definecolor{currentfill}{rgb}{0.000000,0.000000,0.000000}%
\pgfsetfillcolor{currentfill}%
\pgfsetlinewidth{1.003750pt}%
\definecolor{currentstroke}{rgb}{0.000000,0.000000,0.000000}%
\pgfsetstrokecolor{currentstroke}%
\pgfsetdash{}{0pt}%
\pgfpathmoveto{\pgfqpoint{4.011666in}{2.334266in}}%
\pgfpathcurveto{\pgfqpoint{4.022716in}{2.334266in}}{\pgfqpoint{4.033315in}{2.338657in}}{\pgfqpoint{4.041128in}{2.346470in}}%
\pgfpathcurveto{\pgfqpoint{4.048942in}{2.354284in}}{\pgfqpoint{4.053332in}{2.364883in}}{\pgfqpoint{4.053332in}{2.375933in}}%
\pgfpathcurveto{\pgfqpoint{4.053332in}{2.386983in}}{\pgfqpoint{4.048942in}{2.397582in}}{\pgfqpoint{4.041128in}{2.405396in}}%
\pgfpathcurveto{\pgfqpoint{4.033315in}{2.413209in}}{\pgfqpoint{4.022716in}{2.417600in}}{\pgfqpoint{4.011666in}{2.417600in}}%
\pgfpathcurveto{\pgfqpoint{4.000616in}{2.417600in}}{\pgfqpoint{3.990016in}{2.413209in}}{\pgfqpoint{3.982203in}{2.405396in}}%
\pgfpathcurveto{\pgfqpoint{3.974389in}{2.397582in}}{\pgfqpoint{3.969999in}{2.386983in}}{\pgfqpoint{3.969999in}{2.375933in}}%
\pgfpathcurveto{\pgfqpoint{3.969999in}{2.364883in}}{\pgfqpoint{3.974389in}{2.354284in}}{\pgfqpoint{3.982203in}{2.346470in}}%
\pgfpathcurveto{\pgfqpoint{3.990016in}{2.338657in}}{\pgfqpoint{4.000616in}{2.334266in}}{\pgfqpoint{4.011666in}{2.334266in}}%
\pgfpathclose%
\pgfusepath{stroke,fill}%
\end{pgfscope}%
\begin{pgfscope}%
\pgfpathrectangle{\pgfqpoint{0.800000in}{0.528000in}}{\pgfqpoint{4.960000in}{3.696000in}}%
\pgfusepath{clip}%
\pgfsetbuttcap%
\pgfsetroundjoin%
\definecolor{currentfill}{rgb}{0.000000,0.000000,0.000000}%
\pgfsetfillcolor{currentfill}%
\pgfsetlinewidth{1.003750pt}%
\definecolor{currentstroke}{rgb}{0.000000,0.000000,0.000000}%
\pgfsetstrokecolor{currentstroke}%
\pgfsetdash{}{0pt}%
\pgfpathmoveto{\pgfqpoint{4.011666in}{2.334266in}}%
\pgfpathcurveto{\pgfqpoint{4.022716in}{2.334266in}}{\pgfqpoint{4.033315in}{2.338657in}}{\pgfqpoint{4.041128in}{2.346470in}}%
\pgfpathcurveto{\pgfqpoint{4.048942in}{2.354284in}}{\pgfqpoint{4.053332in}{2.364883in}}{\pgfqpoint{4.053332in}{2.375933in}}%
\pgfpathcurveto{\pgfqpoint{4.053332in}{2.386983in}}{\pgfqpoint{4.048942in}{2.397582in}}{\pgfqpoint{4.041128in}{2.405396in}}%
\pgfpathcurveto{\pgfqpoint{4.033315in}{2.413209in}}{\pgfqpoint{4.022716in}{2.417600in}}{\pgfqpoint{4.011666in}{2.417600in}}%
\pgfpathcurveto{\pgfqpoint{4.000616in}{2.417600in}}{\pgfqpoint{3.990016in}{2.413209in}}{\pgfqpoint{3.982203in}{2.405396in}}%
\pgfpathcurveto{\pgfqpoint{3.974389in}{2.397582in}}{\pgfqpoint{3.969999in}{2.386983in}}{\pgfqpoint{3.969999in}{2.375933in}}%
\pgfpathcurveto{\pgfqpoint{3.969999in}{2.364883in}}{\pgfqpoint{3.974389in}{2.354284in}}{\pgfqpoint{3.982203in}{2.346470in}}%
\pgfpathcurveto{\pgfqpoint{3.990016in}{2.338657in}}{\pgfqpoint{4.000616in}{2.334266in}}{\pgfqpoint{4.011666in}{2.334266in}}%
\pgfpathclose%
\pgfusepath{stroke,fill}%
\end{pgfscope}%
\begin{pgfscope}%
\pgfpathrectangle{\pgfqpoint{0.800000in}{0.528000in}}{\pgfqpoint{4.960000in}{3.696000in}}%
\pgfusepath{clip}%
\pgfsetbuttcap%
\pgfsetroundjoin%
\definecolor{currentfill}{rgb}{0.000000,0.000000,0.000000}%
\pgfsetfillcolor{currentfill}%
\pgfsetlinewidth{1.003750pt}%
\definecolor{currentstroke}{rgb}{0.000000,0.000000,0.000000}%
\pgfsetstrokecolor{currentstroke}%
\pgfsetdash{}{0pt}%
\pgfpathmoveto{\pgfqpoint{4.011666in}{2.334266in}}%
\pgfpathcurveto{\pgfqpoint{4.022716in}{2.334266in}}{\pgfqpoint{4.033315in}{2.338657in}}{\pgfqpoint{4.041128in}{2.346470in}}%
\pgfpathcurveto{\pgfqpoint{4.048942in}{2.354284in}}{\pgfqpoint{4.053332in}{2.364883in}}{\pgfqpoint{4.053332in}{2.375933in}}%
\pgfpathcurveto{\pgfqpoint{4.053332in}{2.386983in}}{\pgfqpoint{4.048942in}{2.397582in}}{\pgfqpoint{4.041128in}{2.405396in}}%
\pgfpathcurveto{\pgfqpoint{4.033315in}{2.413209in}}{\pgfqpoint{4.022716in}{2.417600in}}{\pgfqpoint{4.011666in}{2.417600in}}%
\pgfpathcurveto{\pgfqpoint{4.000616in}{2.417600in}}{\pgfqpoint{3.990016in}{2.413209in}}{\pgfqpoint{3.982203in}{2.405396in}}%
\pgfpathcurveto{\pgfqpoint{3.974389in}{2.397582in}}{\pgfqpoint{3.969999in}{2.386983in}}{\pgfqpoint{3.969999in}{2.375933in}}%
\pgfpathcurveto{\pgfqpoint{3.969999in}{2.364883in}}{\pgfqpoint{3.974389in}{2.354284in}}{\pgfqpoint{3.982203in}{2.346470in}}%
\pgfpathcurveto{\pgfqpoint{3.990016in}{2.338657in}}{\pgfqpoint{4.000616in}{2.334266in}}{\pgfqpoint{4.011666in}{2.334266in}}%
\pgfpathclose%
\pgfusepath{stroke,fill}%
\end{pgfscope}%
\begin{pgfscope}%
\pgfpathrectangle{\pgfqpoint{0.800000in}{0.528000in}}{\pgfqpoint{4.960000in}{3.696000in}}%
\pgfusepath{clip}%
\pgfsetbuttcap%
\pgfsetroundjoin%
\definecolor{currentfill}{rgb}{0.000000,0.000000,0.000000}%
\pgfsetfillcolor{currentfill}%
\pgfsetlinewidth{1.003750pt}%
\definecolor{currentstroke}{rgb}{0.000000,0.000000,0.000000}%
\pgfsetstrokecolor{currentstroke}%
\pgfsetdash{}{0pt}%
\pgfpathmoveto{\pgfqpoint{4.011666in}{2.334266in}}%
\pgfpathcurveto{\pgfqpoint{4.022716in}{2.334266in}}{\pgfqpoint{4.033315in}{2.338657in}}{\pgfqpoint{4.041128in}{2.346470in}}%
\pgfpathcurveto{\pgfqpoint{4.048942in}{2.354284in}}{\pgfqpoint{4.053332in}{2.364883in}}{\pgfqpoint{4.053332in}{2.375933in}}%
\pgfpathcurveto{\pgfqpoint{4.053332in}{2.386983in}}{\pgfqpoint{4.048942in}{2.397582in}}{\pgfqpoint{4.041128in}{2.405396in}}%
\pgfpathcurveto{\pgfqpoint{4.033315in}{2.413209in}}{\pgfqpoint{4.022716in}{2.417600in}}{\pgfqpoint{4.011666in}{2.417600in}}%
\pgfpathcurveto{\pgfqpoint{4.000616in}{2.417600in}}{\pgfqpoint{3.990016in}{2.413209in}}{\pgfqpoint{3.982203in}{2.405396in}}%
\pgfpathcurveto{\pgfqpoint{3.974389in}{2.397582in}}{\pgfqpoint{3.969999in}{2.386983in}}{\pgfqpoint{3.969999in}{2.375933in}}%
\pgfpathcurveto{\pgfqpoint{3.969999in}{2.364883in}}{\pgfqpoint{3.974389in}{2.354284in}}{\pgfqpoint{3.982203in}{2.346470in}}%
\pgfpathcurveto{\pgfqpoint{3.990016in}{2.338657in}}{\pgfqpoint{4.000616in}{2.334266in}}{\pgfqpoint{4.011666in}{2.334266in}}%
\pgfpathclose%
\pgfusepath{stroke,fill}%
\end{pgfscope}%
\begin{pgfscope}%
\pgfpathrectangle{\pgfqpoint{0.800000in}{0.528000in}}{\pgfqpoint{4.960000in}{3.696000in}}%
\pgfusepath{clip}%
\pgfsetbuttcap%
\pgfsetroundjoin%
\definecolor{currentfill}{rgb}{0.000000,0.000000,0.000000}%
\pgfsetfillcolor{currentfill}%
\pgfsetlinewidth{1.003750pt}%
\definecolor{currentstroke}{rgb}{0.000000,0.000000,0.000000}%
\pgfsetstrokecolor{currentstroke}%
\pgfsetdash{}{0pt}%
\pgfpathmoveto{\pgfqpoint{4.011666in}{2.334266in}}%
\pgfpathcurveto{\pgfqpoint{4.022716in}{2.334266in}}{\pgfqpoint{4.033315in}{2.338657in}}{\pgfqpoint{4.041128in}{2.346470in}}%
\pgfpathcurveto{\pgfqpoint{4.048942in}{2.354284in}}{\pgfqpoint{4.053332in}{2.364883in}}{\pgfqpoint{4.053332in}{2.375933in}}%
\pgfpathcurveto{\pgfqpoint{4.053332in}{2.386983in}}{\pgfqpoint{4.048942in}{2.397582in}}{\pgfqpoint{4.041128in}{2.405396in}}%
\pgfpathcurveto{\pgfqpoint{4.033315in}{2.413209in}}{\pgfqpoint{4.022716in}{2.417600in}}{\pgfqpoint{4.011666in}{2.417600in}}%
\pgfpathcurveto{\pgfqpoint{4.000616in}{2.417600in}}{\pgfqpoint{3.990016in}{2.413209in}}{\pgfqpoint{3.982203in}{2.405396in}}%
\pgfpathcurveto{\pgfqpoint{3.974389in}{2.397582in}}{\pgfqpoint{3.969999in}{2.386983in}}{\pgfqpoint{3.969999in}{2.375933in}}%
\pgfpathcurveto{\pgfqpoint{3.969999in}{2.364883in}}{\pgfqpoint{3.974389in}{2.354284in}}{\pgfqpoint{3.982203in}{2.346470in}}%
\pgfpathcurveto{\pgfqpoint{3.990016in}{2.338657in}}{\pgfqpoint{4.000616in}{2.334266in}}{\pgfqpoint{4.011666in}{2.334266in}}%
\pgfpathclose%
\pgfusepath{stroke,fill}%
\end{pgfscope}%
\begin{pgfscope}%
\pgfpathrectangle{\pgfqpoint{0.800000in}{0.528000in}}{\pgfqpoint{4.960000in}{3.696000in}}%
\pgfusepath{clip}%
\pgfsetbuttcap%
\pgfsetroundjoin%
\definecolor{currentfill}{rgb}{0.000000,0.000000,0.000000}%
\pgfsetfillcolor{currentfill}%
\pgfsetlinewidth{1.003750pt}%
\definecolor{currentstroke}{rgb}{0.000000,0.000000,0.000000}%
\pgfsetstrokecolor{currentstroke}%
\pgfsetdash{}{0pt}%
\pgfpathmoveto{\pgfqpoint{4.011666in}{2.334266in}}%
\pgfpathcurveto{\pgfqpoint{4.022716in}{2.334266in}}{\pgfqpoint{4.033315in}{2.338657in}}{\pgfqpoint{4.041128in}{2.346470in}}%
\pgfpathcurveto{\pgfqpoint{4.048942in}{2.354284in}}{\pgfqpoint{4.053332in}{2.364883in}}{\pgfqpoint{4.053332in}{2.375933in}}%
\pgfpathcurveto{\pgfqpoint{4.053332in}{2.386983in}}{\pgfqpoint{4.048942in}{2.397582in}}{\pgfqpoint{4.041128in}{2.405396in}}%
\pgfpathcurveto{\pgfqpoint{4.033315in}{2.413209in}}{\pgfqpoint{4.022716in}{2.417600in}}{\pgfqpoint{4.011666in}{2.417600in}}%
\pgfpathcurveto{\pgfqpoint{4.000616in}{2.417600in}}{\pgfqpoint{3.990016in}{2.413209in}}{\pgfqpoint{3.982203in}{2.405396in}}%
\pgfpathcurveto{\pgfqpoint{3.974389in}{2.397582in}}{\pgfqpoint{3.969999in}{2.386983in}}{\pgfqpoint{3.969999in}{2.375933in}}%
\pgfpathcurveto{\pgfqpoint{3.969999in}{2.364883in}}{\pgfqpoint{3.974389in}{2.354284in}}{\pgfqpoint{3.982203in}{2.346470in}}%
\pgfpathcurveto{\pgfqpoint{3.990016in}{2.338657in}}{\pgfqpoint{4.000616in}{2.334266in}}{\pgfqpoint{4.011666in}{2.334266in}}%
\pgfpathclose%
\pgfusepath{stroke,fill}%
\end{pgfscope}%
\begin{pgfscope}%
\pgfpathrectangle{\pgfqpoint{0.800000in}{0.528000in}}{\pgfqpoint{4.960000in}{3.696000in}}%
\pgfusepath{clip}%
\pgfsetbuttcap%
\pgfsetroundjoin%
\definecolor{currentfill}{rgb}{0.000000,0.000000,0.000000}%
\pgfsetfillcolor{currentfill}%
\pgfsetlinewidth{1.003750pt}%
\definecolor{currentstroke}{rgb}{0.000000,0.000000,0.000000}%
\pgfsetstrokecolor{currentstroke}%
\pgfsetdash{}{0pt}%
\pgfpathmoveto{\pgfqpoint{4.011666in}{2.334266in}}%
\pgfpathcurveto{\pgfqpoint{4.022716in}{2.334266in}}{\pgfqpoint{4.033315in}{2.338657in}}{\pgfqpoint{4.041128in}{2.346470in}}%
\pgfpathcurveto{\pgfqpoint{4.048942in}{2.354284in}}{\pgfqpoint{4.053332in}{2.364883in}}{\pgfqpoint{4.053332in}{2.375933in}}%
\pgfpathcurveto{\pgfqpoint{4.053332in}{2.386983in}}{\pgfqpoint{4.048942in}{2.397582in}}{\pgfqpoint{4.041128in}{2.405396in}}%
\pgfpathcurveto{\pgfqpoint{4.033315in}{2.413209in}}{\pgfqpoint{4.022716in}{2.417600in}}{\pgfqpoint{4.011666in}{2.417600in}}%
\pgfpathcurveto{\pgfqpoint{4.000616in}{2.417600in}}{\pgfqpoint{3.990016in}{2.413209in}}{\pgfqpoint{3.982203in}{2.405396in}}%
\pgfpathcurveto{\pgfqpoint{3.974389in}{2.397582in}}{\pgfqpoint{3.969999in}{2.386983in}}{\pgfqpoint{3.969999in}{2.375933in}}%
\pgfpathcurveto{\pgfqpoint{3.969999in}{2.364883in}}{\pgfqpoint{3.974389in}{2.354284in}}{\pgfqpoint{3.982203in}{2.346470in}}%
\pgfpathcurveto{\pgfqpoint{3.990016in}{2.338657in}}{\pgfqpoint{4.000616in}{2.334266in}}{\pgfqpoint{4.011666in}{2.334266in}}%
\pgfpathclose%
\pgfusepath{stroke,fill}%
\end{pgfscope}%
\begin{pgfscope}%
\pgfpathrectangle{\pgfqpoint{0.800000in}{0.528000in}}{\pgfqpoint{4.960000in}{3.696000in}}%
\pgfusepath{clip}%
\pgfsetbuttcap%
\pgfsetroundjoin%
\definecolor{currentfill}{rgb}{0.000000,0.000000,0.000000}%
\pgfsetfillcolor{currentfill}%
\pgfsetlinewidth{1.003750pt}%
\definecolor{currentstroke}{rgb}{0.000000,0.000000,0.000000}%
\pgfsetstrokecolor{currentstroke}%
\pgfsetdash{}{0pt}%
\pgfpathmoveto{\pgfqpoint{4.011666in}{2.334266in}}%
\pgfpathcurveto{\pgfqpoint{4.022716in}{2.334266in}}{\pgfqpoint{4.033315in}{2.338657in}}{\pgfqpoint{4.041128in}{2.346470in}}%
\pgfpathcurveto{\pgfqpoint{4.048942in}{2.354284in}}{\pgfqpoint{4.053332in}{2.364883in}}{\pgfqpoint{4.053332in}{2.375933in}}%
\pgfpathcurveto{\pgfqpoint{4.053332in}{2.386983in}}{\pgfqpoint{4.048942in}{2.397582in}}{\pgfqpoint{4.041128in}{2.405396in}}%
\pgfpathcurveto{\pgfqpoint{4.033315in}{2.413209in}}{\pgfqpoint{4.022716in}{2.417600in}}{\pgfqpoint{4.011666in}{2.417600in}}%
\pgfpathcurveto{\pgfqpoint{4.000616in}{2.417600in}}{\pgfqpoint{3.990016in}{2.413209in}}{\pgfqpoint{3.982203in}{2.405396in}}%
\pgfpathcurveto{\pgfqpoint{3.974389in}{2.397582in}}{\pgfqpoint{3.969999in}{2.386983in}}{\pgfqpoint{3.969999in}{2.375933in}}%
\pgfpathcurveto{\pgfqpoint{3.969999in}{2.364883in}}{\pgfqpoint{3.974389in}{2.354284in}}{\pgfqpoint{3.982203in}{2.346470in}}%
\pgfpathcurveto{\pgfqpoint{3.990016in}{2.338657in}}{\pgfqpoint{4.000616in}{2.334266in}}{\pgfqpoint{4.011666in}{2.334266in}}%
\pgfpathclose%
\pgfusepath{stroke,fill}%
\end{pgfscope}%
\begin{pgfscope}%
\pgfpathrectangle{\pgfqpoint{0.800000in}{0.528000in}}{\pgfqpoint{4.960000in}{3.696000in}}%
\pgfusepath{clip}%
\pgfsetbuttcap%
\pgfsetroundjoin%
\definecolor{currentfill}{rgb}{0.000000,0.000000,0.000000}%
\pgfsetfillcolor{currentfill}%
\pgfsetlinewidth{1.003750pt}%
\definecolor{currentstroke}{rgb}{0.000000,0.000000,0.000000}%
\pgfsetstrokecolor{currentstroke}%
\pgfsetdash{}{0pt}%
\pgfpathmoveto{\pgfqpoint{4.011666in}{3.984333in}}%
\pgfpathcurveto{\pgfqpoint{4.022716in}{3.984333in}}{\pgfqpoint{4.033315in}{3.988724in}}{\pgfqpoint{4.041128in}{3.996537in}}%
\pgfpathcurveto{\pgfqpoint{4.048942in}{4.004351in}}{\pgfqpoint{4.053332in}{4.014950in}}{\pgfqpoint{4.053332in}{4.026000in}}%
\pgfpathcurveto{\pgfqpoint{4.053332in}{4.037050in}}{\pgfqpoint{4.048942in}{4.047649in}}{\pgfqpoint{4.041128in}{4.055463in}}%
\pgfpathcurveto{\pgfqpoint{4.033315in}{4.063276in}}{\pgfqpoint{4.022716in}{4.067667in}}{\pgfqpoint{4.011666in}{4.067667in}}%
\pgfpathcurveto{\pgfqpoint{4.000616in}{4.067667in}}{\pgfqpoint{3.990016in}{4.063276in}}{\pgfqpoint{3.982203in}{4.055463in}}%
\pgfpathcurveto{\pgfqpoint{3.974389in}{4.047649in}}{\pgfqpoint{3.969999in}{4.037050in}}{\pgfqpoint{3.969999in}{4.026000in}}%
\pgfpathcurveto{\pgfqpoint{3.969999in}{4.014950in}}{\pgfqpoint{3.974389in}{4.004351in}}{\pgfqpoint{3.982203in}{3.996537in}}%
\pgfpathcurveto{\pgfqpoint{3.990016in}{3.988724in}}{\pgfqpoint{4.000616in}{3.984333in}}{\pgfqpoint{4.011666in}{3.984333in}}%
\pgfpathclose%
\pgfusepath{stroke,fill}%
\end{pgfscope}%
\begin{pgfscope}%
\pgfpathrectangle{\pgfqpoint{0.800000in}{0.528000in}}{\pgfqpoint{4.960000in}{3.696000in}}%
\pgfusepath{clip}%
\pgfsetbuttcap%
\pgfsetroundjoin%
\definecolor{currentfill}{rgb}{0.000000,0.000000,0.000000}%
\pgfsetfillcolor{currentfill}%
\pgfsetlinewidth{1.003750pt}%
\definecolor{currentstroke}{rgb}{0.000000,0.000000,0.000000}%
\pgfsetstrokecolor{currentstroke}%
\pgfsetdash{}{0pt}%
\pgfpathmoveto{\pgfqpoint{4.011666in}{2.334266in}}%
\pgfpathcurveto{\pgfqpoint{4.022716in}{2.334266in}}{\pgfqpoint{4.033315in}{2.338657in}}{\pgfqpoint{4.041128in}{2.346470in}}%
\pgfpathcurveto{\pgfqpoint{4.048942in}{2.354284in}}{\pgfqpoint{4.053332in}{2.364883in}}{\pgfqpoint{4.053332in}{2.375933in}}%
\pgfpathcurveto{\pgfqpoint{4.053332in}{2.386983in}}{\pgfqpoint{4.048942in}{2.397582in}}{\pgfqpoint{4.041128in}{2.405396in}}%
\pgfpathcurveto{\pgfqpoint{4.033315in}{2.413209in}}{\pgfqpoint{4.022716in}{2.417600in}}{\pgfqpoint{4.011666in}{2.417600in}}%
\pgfpathcurveto{\pgfqpoint{4.000616in}{2.417600in}}{\pgfqpoint{3.990016in}{2.413209in}}{\pgfqpoint{3.982203in}{2.405396in}}%
\pgfpathcurveto{\pgfqpoint{3.974389in}{2.397582in}}{\pgfqpoint{3.969999in}{2.386983in}}{\pgfqpoint{3.969999in}{2.375933in}}%
\pgfpathcurveto{\pgfqpoint{3.969999in}{2.364883in}}{\pgfqpoint{3.974389in}{2.354284in}}{\pgfqpoint{3.982203in}{2.346470in}}%
\pgfpathcurveto{\pgfqpoint{3.990016in}{2.338657in}}{\pgfqpoint{4.000616in}{2.334266in}}{\pgfqpoint{4.011666in}{2.334266in}}%
\pgfpathclose%
\pgfusepath{stroke,fill}%
\end{pgfscope}%
\begin{pgfscope}%
\pgfpathrectangle{\pgfqpoint{0.800000in}{0.528000in}}{\pgfqpoint{4.960000in}{3.696000in}}%
\pgfusepath{clip}%
\pgfsetbuttcap%
\pgfsetroundjoin%
\definecolor{currentfill}{rgb}{0.000000,0.000000,0.000000}%
\pgfsetfillcolor{currentfill}%
\pgfsetlinewidth{1.003750pt}%
\definecolor{currentstroke}{rgb}{0.000000,0.000000,0.000000}%
\pgfsetstrokecolor{currentstroke}%
\pgfsetdash{}{0pt}%
\pgfpathmoveto{\pgfqpoint{4.011666in}{2.334266in}}%
\pgfpathcurveto{\pgfqpoint{4.022716in}{2.334266in}}{\pgfqpoint{4.033315in}{2.338657in}}{\pgfqpoint{4.041128in}{2.346470in}}%
\pgfpathcurveto{\pgfqpoint{4.048942in}{2.354284in}}{\pgfqpoint{4.053332in}{2.364883in}}{\pgfqpoint{4.053332in}{2.375933in}}%
\pgfpathcurveto{\pgfqpoint{4.053332in}{2.386983in}}{\pgfqpoint{4.048942in}{2.397582in}}{\pgfqpoint{4.041128in}{2.405396in}}%
\pgfpathcurveto{\pgfqpoint{4.033315in}{2.413209in}}{\pgfqpoint{4.022716in}{2.417600in}}{\pgfqpoint{4.011666in}{2.417600in}}%
\pgfpathcurveto{\pgfqpoint{4.000616in}{2.417600in}}{\pgfqpoint{3.990016in}{2.413209in}}{\pgfqpoint{3.982203in}{2.405396in}}%
\pgfpathcurveto{\pgfqpoint{3.974389in}{2.397582in}}{\pgfqpoint{3.969999in}{2.386983in}}{\pgfqpoint{3.969999in}{2.375933in}}%
\pgfpathcurveto{\pgfqpoint{3.969999in}{2.364883in}}{\pgfqpoint{3.974389in}{2.354284in}}{\pgfqpoint{3.982203in}{2.346470in}}%
\pgfpathcurveto{\pgfqpoint{3.990016in}{2.338657in}}{\pgfqpoint{4.000616in}{2.334266in}}{\pgfqpoint{4.011666in}{2.334266in}}%
\pgfpathclose%
\pgfusepath{stroke,fill}%
\end{pgfscope}%
\begin{pgfscope}%
\pgfpathrectangle{\pgfqpoint{0.800000in}{0.528000in}}{\pgfqpoint{4.960000in}{3.696000in}}%
\pgfusepath{clip}%
\pgfsetbuttcap%
\pgfsetroundjoin%
\definecolor{currentfill}{rgb}{0.000000,0.000000,0.000000}%
\pgfsetfillcolor{currentfill}%
\pgfsetlinewidth{1.003750pt}%
\definecolor{currentstroke}{rgb}{0.000000,0.000000,0.000000}%
\pgfsetstrokecolor{currentstroke}%
\pgfsetdash{}{0pt}%
\pgfpathmoveto{\pgfqpoint{4.011666in}{3.984333in}}%
\pgfpathcurveto{\pgfqpoint{4.022716in}{3.984333in}}{\pgfqpoint{4.033315in}{3.988724in}}{\pgfqpoint{4.041128in}{3.996537in}}%
\pgfpathcurveto{\pgfqpoint{4.048942in}{4.004351in}}{\pgfqpoint{4.053332in}{4.014950in}}{\pgfqpoint{4.053332in}{4.026000in}}%
\pgfpathcurveto{\pgfqpoint{4.053332in}{4.037050in}}{\pgfqpoint{4.048942in}{4.047649in}}{\pgfqpoint{4.041128in}{4.055463in}}%
\pgfpathcurveto{\pgfqpoint{4.033315in}{4.063276in}}{\pgfqpoint{4.022716in}{4.067667in}}{\pgfqpoint{4.011666in}{4.067667in}}%
\pgfpathcurveto{\pgfqpoint{4.000616in}{4.067667in}}{\pgfqpoint{3.990016in}{4.063276in}}{\pgfqpoint{3.982203in}{4.055463in}}%
\pgfpathcurveto{\pgfqpoint{3.974389in}{4.047649in}}{\pgfqpoint{3.969999in}{4.037050in}}{\pgfqpoint{3.969999in}{4.026000in}}%
\pgfpathcurveto{\pgfqpoint{3.969999in}{4.014950in}}{\pgfqpoint{3.974389in}{4.004351in}}{\pgfqpoint{3.982203in}{3.996537in}}%
\pgfpathcurveto{\pgfqpoint{3.990016in}{3.988724in}}{\pgfqpoint{4.000616in}{3.984333in}}{\pgfqpoint{4.011666in}{3.984333in}}%
\pgfpathclose%
\pgfusepath{stroke,fill}%
\end{pgfscope}%
\begin{pgfscope}%
\pgfpathrectangle{\pgfqpoint{0.800000in}{0.528000in}}{\pgfqpoint{4.960000in}{3.696000in}}%
\pgfusepath{clip}%
\pgfsetbuttcap%
\pgfsetroundjoin%
\definecolor{currentfill}{rgb}{0.000000,0.000000,0.000000}%
\pgfsetfillcolor{currentfill}%
\pgfsetlinewidth{1.003750pt}%
\definecolor{currentstroke}{rgb}{0.000000,0.000000,0.000000}%
\pgfsetstrokecolor{currentstroke}%
\pgfsetdash{}{0pt}%
\pgfpathmoveto{\pgfqpoint{4.011666in}{2.334266in}}%
\pgfpathcurveto{\pgfqpoint{4.022716in}{2.334266in}}{\pgfqpoint{4.033315in}{2.338657in}}{\pgfqpoint{4.041128in}{2.346470in}}%
\pgfpathcurveto{\pgfqpoint{4.048942in}{2.354284in}}{\pgfqpoint{4.053332in}{2.364883in}}{\pgfqpoint{4.053332in}{2.375933in}}%
\pgfpathcurveto{\pgfqpoint{4.053332in}{2.386983in}}{\pgfqpoint{4.048942in}{2.397582in}}{\pgfqpoint{4.041128in}{2.405396in}}%
\pgfpathcurveto{\pgfqpoint{4.033315in}{2.413209in}}{\pgfqpoint{4.022716in}{2.417600in}}{\pgfqpoint{4.011666in}{2.417600in}}%
\pgfpathcurveto{\pgfqpoint{4.000616in}{2.417600in}}{\pgfqpoint{3.990016in}{2.413209in}}{\pgfqpoint{3.982203in}{2.405396in}}%
\pgfpathcurveto{\pgfqpoint{3.974389in}{2.397582in}}{\pgfqpoint{3.969999in}{2.386983in}}{\pgfqpoint{3.969999in}{2.375933in}}%
\pgfpathcurveto{\pgfqpoint{3.969999in}{2.364883in}}{\pgfqpoint{3.974389in}{2.354284in}}{\pgfqpoint{3.982203in}{2.346470in}}%
\pgfpathcurveto{\pgfqpoint{3.990016in}{2.338657in}}{\pgfqpoint{4.000616in}{2.334266in}}{\pgfqpoint{4.011666in}{2.334266in}}%
\pgfpathclose%
\pgfusepath{stroke,fill}%
\end{pgfscope}%
\begin{pgfscope}%
\pgfpathrectangle{\pgfqpoint{0.800000in}{0.528000in}}{\pgfqpoint{4.960000in}{3.696000in}}%
\pgfusepath{clip}%
\pgfsetbuttcap%
\pgfsetroundjoin%
\definecolor{currentfill}{rgb}{0.000000,0.000000,0.000000}%
\pgfsetfillcolor{currentfill}%
\pgfsetlinewidth{1.003750pt}%
\definecolor{currentstroke}{rgb}{0.000000,0.000000,0.000000}%
\pgfsetstrokecolor{currentstroke}%
\pgfsetdash{}{0pt}%
\pgfpathmoveto{\pgfqpoint{4.011666in}{2.334266in}}%
\pgfpathcurveto{\pgfqpoint{4.022716in}{2.334266in}}{\pgfqpoint{4.033315in}{2.338657in}}{\pgfqpoint{4.041128in}{2.346470in}}%
\pgfpathcurveto{\pgfqpoint{4.048942in}{2.354284in}}{\pgfqpoint{4.053332in}{2.364883in}}{\pgfqpoint{4.053332in}{2.375933in}}%
\pgfpathcurveto{\pgfqpoint{4.053332in}{2.386983in}}{\pgfqpoint{4.048942in}{2.397582in}}{\pgfqpoint{4.041128in}{2.405396in}}%
\pgfpathcurveto{\pgfqpoint{4.033315in}{2.413209in}}{\pgfqpoint{4.022716in}{2.417600in}}{\pgfqpoint{4.011666in}{2.417600in}}%
\pgfpathcurveto{\pgfqpoint{4.000616in}{2.417600in}}{\pgfqpoint{3.990016in}{2.413209in}}{\pgfqpoint{3.982203in}{2.405396in}}%
\pgfpathcurveto{\pgfqpoint{3.974389in}{2.397582in}}{\pgfqpoint{3.969999in}{2.386983in}}{\pgfqpoint{3.969999in}{2.375933in}}%
\pgfpathcurveto{\pgfqpoint{3.969999in}{2.364883in}}{\pgfqpoint{3.974389in}{2.354284in}}{\pgfqpoint{3.982203in}{2.346470in}}%
\pgfpathcurveto{\pgfqpoint{3.990016in}{2.338657in}}{\pgfqpoint{4.000616in}{2.334266in}}{\pgfqpoint{4.011666in}{2.334266in}}%
\pgfpathclose%
\pgfusepath{stroke,fill}%
\end{pgfscope}%
\begin{pgfscope}%
\pgfpathrectangle{\pgfqpoint{0.800000in}{0.528000in}}{\pgfqpoint{4.960000in}{3.696000in}}%
\pgfusepath{clip}%
\pgfsetbuttcap%
\pgfsetroundjoin%
\definecolor{currentfill}{rgb}{0.000000,0.000000,0.000000}%
\pgfsetfillcolor{currentfill}%
\pgfsetlinewidth{1.003750pt}%
\definecolor{currentstroke}{rgb}{0.000000,0.000000,0.000000}%
\pgfsetstrokecolor{currentstroke}%
\pgfsetdash{}{0pt}%
\pgfpathmoveto{\pgfqpoint{4.011666in}{2.334266in}}%
\pgfpathcurveto{\pgfqpoint{4.022716in}{2.334266in}}{\pgfqpoint{4.033315in}{2.338657in}}{\pgfqpoint{4.041128in}{2.346470in}}%
\pgfpathcurveto{\pgfqpoint{4.048942in}{2.354284in}}{\pgfqpoint{4.053332in}{2.364883in}}{\pgfqpoint{4.053332in}{2.375933in}}%
\pgfpathcurveto{\pgfqpoint{4.053332in}{2.386983in}}{\pgfqpoint{4.048942in}{2.397582in}}{\pgfqpoint{4.041128in}{2.405396in}}%
\pgfpathcurveto{\pgfqpoint{4.033315in}{2.413209in}}{\pgfqpoint{4.022716in}{2.417600in}}{\pgfqpoint{4.011666in}{2.417600in}}%
\pgfpathcurveto{\pgfqpoint{4.000616in}{2.417600in}}{\pgfqpoint{3.990016in}{2.413209in}}{\pgfqpoint{3.982203in}{2.405396in}}%
\pgfpathcurveto{\pgfqpoint{3.974389in}{2.397582in}}{\pgfqpoint{3.969999in}{2.386983in}}{\pgfqpoint{3.969999in}{2.375933in}}%
\pgfpathcurveto{\pgfqpoint{3.969999in}{2.364883in}}{\pgfqpoint{3.974389in}{2.354284in}}{\pgfqpoint{3.982203in}{2.346470in}}%
\pgfpathcurveto{\pgfqpoint{3.990016in}{2.338657in}}{\pgfqpoint{4.000616in}{2.334266in}}{\pgfqpoint{4.011666in}{2.334266in}}%
\pgfpathclose%
\pgfusepath{stroke,fill}%
\end{pgfscope}%
\begin{pgfscope}%
\pgfpathrectangle{\pgfqpoint{0.800000in}{0.528000in}}{\pgfqpoint{4.960000in}{3.696000in}}%
\pgfusepath{clip}%
\pgfsetbuttcap%
\pgfsetroundjoin%
\definecolor{currentfill}{rgb}{0.000000,0.000000,0.000000}%
\pgfsetfillcolor{currentfill}%
\pgfsetlinewidth{1.003750pt}%
\definecolor{currentstroke}{rgb}{0.000000,0.000000,0.000000}%
\pgfsetstrokecolor{currentstroke}%
\pgfsetdash{}{0pt}%
\pgfpathmoveto{\pgfqpoint{4.011666in}{2.334266in}}%
\pgfpathcurveto{\pgfqpoint{4.022716in}{2.334266in}}{\pgfqpoint{4.033315in}{2.338657in}}{\pgfqpoint{4.041128in}{2.346470in}}%
\pgfpathcurveto{\pgfqpoint{4.048942in}{2.354284in}}{\pgfqpoint{4.053332in}{2.364883in}}{\pgfqpoint{4.053332in}{2.375933in}}%
\pgfpathcurveto{\pgfqpoint{4.053332in}{2.386983in}}{\pgfqpoint{4.048942in}{2.397582in}}{\pgfqpoint{4.041128in}{2.405396in}}%
\pgfpathcurveto{\pgfqpoint{4.033315in}{2.413209in}}{\pgfqpoint{4.022716in}{2.417600in}}{\pgfqpoint{4.011666in}{2.417600in}}%
\pgfpathcurveto{\pgfqpoint{4.000616in}{2.417600in}}{\pgfqpoint{3.990016in}{2.413209in}}{\pgfqpoint{3.982203in}{2.405396in}}%
\pgfpathcurveto{\pgfqpoint{3.974389in}{2.397582in}}{\pgfqpoint{3.969999in}{2.386983in}}{\pgfqpoint{3.969999in}{2.375933in}}%
\pgfpathcurveto{\pgfqpoint{3.969999in}{2.364883in}}{\pgfqpoint{3.974389in}{2.354284in}}{\pgfqpoint{3.982203in}{2.346470in}}%
\pgfpathcurveto{\pgfqpoint{3.990016in}{2.338657in}}{\pgfqpoint{4.000616in}{2.334266in}}{\pgfqpoint{4.011666in}{2.334266in}}%
\pgfpathclose%
\pgfusepath{stroke,fill}%
\end{pgfscope}%
\begin{pgfscope}%
\pgfpathrectangle{\pgfqpoint{0.800000in}{0.528000in}}{\pgfqpoint{4.960000in}{3.696000in}}%
\pgfusepath{clip}%
\pgfsetbuttcap%
\pgfsetroundjoin%
\definecolor{currentfill}{rgb}{0.000000,0.000000,0.000000}%
\pgfsetfillcolor{currentfill}%
\pgfsetlinewidth{1.003750pt}%
\definecolor{currentstroke}{rgb}{0.000000,0.000000,0.000000}%
\pgfsetstrokecolor{currentstroke}%
\pgfsetdash{}{0pt}%
\pgfpathmoveto{\pgfqpoint{4.011666in}{2.334266in}}%
\pgfpathcurveto{\pgfqpoint{4.022716in}{2.334266in}}{\pgfqpoint{4.033315in}{2.338657in}}{\pgfqpoint{4.041128in}{2.346470in}}%
\pgfpathcurveto{\pgfqpoint{4.048942in}{2.354284in}}{\pgfqpoint{4.053332in}{2.364883in}}{\pgfqpoint{4.053332in}{2.375933in}}%
\pgfpathcurveto{\pgfqpoint{4.053332in}{2.386983in}}{\pgfqpoint{4.048942in}{2.397582in}}{\pgfqpoint{4.041128in}{2.405396in}}%
\pgfpathcurveto{\pgfqpoint{4.033315in}{2.413209in}}{\pgfqpoint{4.022716in}{2.417600in}}{\pgfqpoint{4.011666in}{2.417600in}}%
\pgfpathcurveto{\pgfqpoint{4.000616in}{2.417600in}}{\pgfqpoint{3.990016in}{2.413209in}}{\pgfqpoint{3.982203in}{2.405396in}}%
\pgfpathcurveto{\pgfqpoint{3.974389in}{2.397582in}}{\pgfqpoint{3.969999in}{2.386983in}}{\pgfqpoint{3.969999in}{2.375933in}}%
\pgfpathcurveto{\pgfqpoint{3.969999in}{2.364883in}}{\pgfqpoint{3.974389in}{2.354284in}}{\pgfqpoint{3.982203in}{2.346470in}}%
\pgfpathcurveto{\pgfqpoint{3.990016in}{2.338657in}}{\pgfqpoint{4.000616in}{2.334266in}}{\pgfqpoint{4.011666in}{2.334266in}}%
\pgfpathclose%
\pgfusepath{stroke,fill}%
\end{pgfscope}%
\begin{pgfscope}%
\pgfpathrectangle{\pgfqpoint{0.800000in}{0.528000in}}{\pgfqpoint{4.960000in}{3.696000in}}%
\pgfusepath{clip}%
\pgfsetbuttcap%
\pgfsetroundjoin%
\definecolor{currentfill}{rgb}{0.000000,0.000000,0.000000}%
\pgfsetfillcolor{currentfill}%
\pgfsetlinewidth{1.003750pt}%
\definecolor{currentstroke}{rgb}{0.000000,0.000000,0.000000}%
\pgfsetstrokecolor{currentstroke}%
\pgfsetdash{}{0pt}%
\pgfpathmoveto{\pgfqpoint{4.011666in}{2.334266in}}%
\pgfpathcurveto{\pgfqpoint{4.022716in}{2.334266in}}{\pgfqpoint{4.033315in}{2.338657in}}{\pgfqpoint{4.041128in}{2.346470in}}%
\pgfpathcurveto{\pgfqpoint{4.048942in}{2.354284in}}{\pgfqpoint{4.053332in}{2.364883in}}{\pgfqpoint{4.053332in}{2.375933in}}%
\pgfpathcurveto{\pgfqpoint{4.053332in}{2.386983in}}{\pgfqpoint{4.048942in}{2.397582in}}{\pgfqpoint{4.041128in}{2.405396in}}%
\pgfpathcurveto{\pgfqpoint{4.033315in}{2.413209in}}{\pgfqpoint{4.022716in}{2.417600in}}{\pgfqpoint{4.011666in}{2.417600in}}%
\pgfpathcurveto{\pgfqpoint{4.000616in}{2.417600in}}{\pgfqpoint{3.990016in}{2.413209in}}{\pgfqpoint{3.982203in}{2.405396in}}%
\pgfpathcurveto{\pgfqpoint{3.974389in}{2.397582in}}{\pgfqpoint{3.969999in}{2.386983in}}{\pgfqpoint{3.969999in}{2.375933in}}%
\pgfpathcurveto{\pgfqpoint{3.969999in}{2.364883in}}{\pgfqpoint{3.974389in}{2.354284in}}{\pgfqpoint{3.982203in}{2.346470in}}%
\pgfpathcurveto{\pgfqpoint{3.990016in}{2.338657in}}{\pgfqpoint{4.000616in}{2.334266in}}{\pgfqpoint{4.011666in}{2.334266in}}%
\pgfpathclose%
\pgfusepath{stroke,fill}%
\end{pgfscope}%
\begin{pgfscope}%
\pgfpathrectangle{\pgfqpoint{0.800000in}{0.528000in}}{\pgfqpoint{4.960000in}{3.696000in}}%
\pgfusepath{clip}%
\pgfsetbuttcap%
\pgfsetroundjoin%
\definecolor{currentfill}{rgb}{0.000000,0.000000,0.000000}%
\pgfsetfillcolor{currentfill}%
\pgfsetlinewidth{1.003750pt}%
\definecolor{currentstroke}{rgb}{0.000000,0.000000,0.000000}%
\pgfsetstrokecolor{currentstroke}%
\pgfsetdash{}{0pt}%
\pgfpathmoveto{\pgfqpoint{4.011666in}{2.334266in}}%
\pgfpathcurveto{\pgfqpoint{4.022716in}{2.334266in}}{\pgfqpoint{4.033315in}{2.338657in}}{\pgfqpoint{4.041128in}{2.346470in}}%
\pgfpathcurveto{\pgfqpoint{4.048942in}{2.354284in}}{\pgfqpoint{4.053332in}{2.364883in}}{\pgfqpoint{4.053332in}{2.375933in}}%
\pgfpathcurveto{\pgfqpoint{4.053332in}{2.386983in}}{\pgfqpoint{4.048942in}{2.397582in}}{\pgfqpoint{4.041128in}{2.405396in}}%
\pgfpathcurveto{\pgfqpoint{4.033315in}{2.413209in}}{\pgfqpoint{4.022716in}{2.417600in}}{\pgfqpoint{4.011666in}{2.417600in}}%
\pgfpathcurveto{\pgfqpoint{4.000616in}{2.417600in}}{\pgfqpoint{3.990016in}{2.413209in}}{\pgfqpoint{3.982203in}{2.405396in}}%
\pgfpathcurveto{\pgfqpoint{3.974389in}{2.397582in}}{\pgfqpoint{3.969999in}{2.386983in}}{\pgfqpoint{3.969999in}{2.375933in}}%
\pgfpathcurveto{\pgfqpoint{3.969999in}{2.364883in}}{\pgfqpoint{3.974389in}{2.354284in}}{\pgfqpoint{3.982203in}{2.346470in}}%
\pgfpathcurveto{\pgfqpoint{3.990016in}{2.338657in}}{\pgfqpoint{4.000616in}{2.334266in}}{\pgfqpoint{4.011666in}{2.334266in}}%
\pgfpathclose%
\pgfusepath{stroke,fill}%
\end{pgfscope}%
\begin{pgfscope}%
\pgfpathrectangle{\pgfqpoint{0.800000in}{0.528000in}}{\pgfqpoint{4.960000in}{3.696000in}}%
\pgfusepath{clip}%
\pgfsetbuttcap%
\pgfsetroundjoin%
\definecolor{currentfill}{rgb}{0.000000,0.000000,0.000000}%
\pgfsetfillcolor{currentfill}%
\pgfsetlinewidth{1.003750pt}%
\definecolor{currentstroke}{rgb}{0.000000,0.000000,0.000000}%
\pgfsetstrokecolor{currentstroke}%
\pgfsetdash{}{0pt}%
\pgfpathmoveto{\pgfqpoint{4.011666in}{2.334266in}}%
\pgfpathcurveto{\pgfqpoint{4.022716in}{2.334266in}}{\pgfqpoint{4.033315in}{2.338657in}}{\pgfqpoint{4.041128in}{2.346470in}}%
\pgfpathcurveto{\pgfqpoint{4.048942in}{2.354284in}}{\pgfqpoint{4.053332in}{2.364883in}}{\pgfqpoint{4.053332in}{2.375933in}}%
\pgfpathcurveto{\pgfqpoint{4.053332in}{2.386983in}}{\pgfqpoint{4.048942in}{2.397582in}}{\pgfqpoint{4.041128in}{2.405396in}}%
\pgfpathcurveto{\pgfqpoint{4.033315in}{2.413209in}}{\pgfqpoint{4.022716in}{2.417600in}}{\pgfqpoint{4.011666in}{2.417600in}}%
\pgfpathcurveto{\pgfqpoint{4.000616in}{2.417600in}}{\pgfqpoint{3.990016in}{2.413209in}}{\pgfqpoint{3.982203in}{2.405396in}}%
\pgfpathcurveto{\pgfqpoint{3.974389in}{2.397582in}}{\pgfqpoint{3.969999in}{2.386983in}}{\pgfqpoint{3.969999in}{2.375933in}}%
\pgfpathcurveto{\pgfqpoint{3.969999in}{2.364883in}}{\pgfqpoint{3.974389in}{2.354284in}}{\pgfqpoint{3.982203in}{2.346470in}}%
\pgfpathcurveto{\pgfqpoint{3.990016in}{2.338657in}}{\pgfqpoint{4.000616in}{2.334266in}}{\pgfqpoint{4.011666in}{2.334266in}}%
\pgfpathclose%
\pgfusepath{stroke,fill}%
\end{pgfscope}%
\begin{pgfscope}%
\pgfpathrectangle{\pgfqpoint{0.800000in}{0.528000in}}{\pgfqpoint{4.960000in}{3.696000in}}%
\pgfusepath{clip}%
\pgfsetbuttcap%
\pgfsetroundjoin%
\definecolor{currentfill}{rgb}{0.000000,0.000000,0.000000}%
\pgfsetfillcolor{currentfill}%
\pgfsetlinewidth{1.003750pt}%
\definecolor{currentstroke}{rgb}{0.000000,0.000000,0.000000}%
\pgfsetstrokecolor{currentstroke}%
\pgfsetdash{}{0pt}%
\pgfpathmoveto{\pgfqpoint{4.011666in}{2.334266in}}%
\pgfpathcurveto{\pgfqpoint{4.022716in}{2.334266in}}{\pgfqpoint{4.033315in}{2.338657in}}{\pgfqpoint{4.041128in}{2.346470in}}%
\pgfpathcurveto{\pgfqpoint{4.048942in}{2.354284in}}{\pgfqpoint{4.053332in}{2.364883in}}{\pgfqpoint{4.053332in}{2.375933in}}%
\pgfpathcurveto{\pgfqpoint{4.053332in}{2.386983in}}{\pgfqpoint{4.048942in}{2.397582in}}{\pgfqpoint{4.041128in}{2.405396in}}%
\pgfpathcurveto{\pgfqpoint{4.033315in}{2.413209in}}{\pgfqpoint{4.022716in}{2.417600in}}{\pgfqpoint{4.011666in}{2.417600in}}%
\pgfpathcurveto{\pgfqpoint{4.000616in}{2.417600in}}{\pgfqpoint{3.990016in}{2.413209in}}{\pgfqpoint{3.982203in}{2.405396in}}%
\pgfpathcurveto{\pgfqpoint{3.974389in}{2.397582in}}{\pgfqpoint{3.969999in}{2.386983in}}{\pgfqpoint{3.969999in}{2.375933in}}%
\pgfpathcurveto{\pgfqpoint{3.969999in}{2.364883in}}{\pgfqpoint{3.974389in}{2.354284in}}{\pgfqpoint{3.982203in}{2.346470in}}%
\pgfpathcurveto{\pgfqpoint{3.990016in}{2.338657in}}{\pgfqpoint{4.000616in}{2.334266in}}{\pgfqpoint{4.011666in}{2.334266in}}%
\pgfpathclose%
\pgfusepath{stroke,fill}%
\end{pgfscope}%
\begin{pgfscope}%
\pgfpathrectangle{\pgfqpoint{0.800000in}{0.528000in}}{\pgfqpoint{4.960000in}{3.696000in}}%
\pgfusepath{clip}%
\pgfsetbuttcap%
\pgfsetroundjoin%
\definecolor{currentfill}{rgb}{0.000000,0.000000,0.000000}%
\pgfsetfillcolor{currentfill}%
\pgfsetlinewidth{1.003750pt}%
\definecolor{currentstroke}{rgb}{0.000000,0.000000,0.000000}%
\pgfsetstrokecolor{currentstroke}%
\pgfsetdash{}{0pt}%
\pgfpathmoveto{\pgfqpoint{4.011666in}{2.334266in}}%
\pgfpathcurveto{\pgfqpoint{4.022716in}{2.334266in}}{\pgfqpoint{4.033315in}{2.338657in}}{\pgfqpoint{4.041128in}{2.346470in}}%
\pgfpathcurveto{\pgfqpoint{4.048942in}{2.354284in}}{\pgfqpoint{4.053332in}{2.364883in}}{\pgfqpoint{4.053332in}{2.375933in}}%
\pgfpathcurveto{\pgfqpoint{4.053332in}{2.386983in}}{\pgfqpoint{4.048942in}{2.397582in}}{\pgfqpoint{4.041128in}{2.405396in}}%
\pgfpathcurveto{\pgfqpoint{4.033315in}{2.413209in}}{\pgfqpoint{4.022716in}{2.417600in}}{\pgfqpoint{4.011666in}{2.417600in}}%
\pgfpathcurveto{\pgfqpoint{4.000616in}{2.417600in}}{\pgfqpoint{3.990016in}{2.413209in}}{\pgfqpoint{3.982203in}{2.405396in}}%
\pgfpathcurveto{\pgfqpoint{3.974389in}{2.397582in}}{\pgfqpoint{3.969999in}{2.386983in}}{\pgfqpoint{3.969999in}{2.375933in}}%
\pgfpathcurveto{\pgfqpoint{3.969999in}{2.364883in}}{\pgfqpoint{3.974389in}{2.354284in}}{\pgfqpoint{3.982203in}{2.346470in}}%
\pgfpathcurveto{\pgfqpoint{3.990016in}{2.338657in}}{\pgfqpoint{4.000616in}{2.334266in}}{\pgfqpoint{4.011666in}{2.334266in}}%
\pgfpathclose%
\pgfusepath{stroke,fill}%
\end{pgfscope}%
\begin{pgfscope}%
\pgfpathrectangle{\pgfqpoint{0.800000in}{0.528000in}}{\pgfqpoint{4.960000in}{3.696000in}}%
\pgfusepath{clip}%
\pgfsetbuttcap%
\pgfsetroundjoin%
\definecolor{currentfill}{rgb}{0.000000,0.000000,0.000000}%
\pgfsetfillcolor{currentfill}%
\pgfsetlinewidth{1.003750pt}%
\definecolor{currentstroke}{rgb}{0.000000,0.000000,0.000000}%
\pgfsetstrokecolor{currentstroke}%
\pgfsetdash{}{0pt}%
\pgfpathmoveto{\pgfqpoint{4.011666in}{2.334266in}}%
\pgfpathcurveto{\pgfqpoint{4.022716in}{2.334266in}}{\pgfqpoint{4.033315in}{2.338657in}}{\pgfqpoint{4.041128in}{2.346470in}}%
\pgfpathcurveto{\pgfqpoint{4.048942in}{2.354284in}}{\pgfqpoint{4.053332in}{2.364883in}}{\pgfqpoint{4.053332in}{2.375933in}}%
\pgfpathcurveto{\pgfqpoint{4.053332in}{2.386983in}}{\pgfqpoint{4.048942in}{2.397582in}}{\pgfqpoint{4.041128in}{2.405396in}}%
\pgfpathcurveto{\pgfqpoint{4.033315in}{2.413209in}}{\pgfqpoint{4.022716in}{2.417600in}}{\pgfqpoint{4.011666in}{2.417600in}}%
\pgfpathcurveto{\pgfqpoint{4.000616in}{2.417600in}}{\pgfqpoint{3.990016in}{2.413209in}}{\pgfqpoint{3.982203in}{2.405396in}}%
\pgfpathcurveto{\pgfqpoint{3.974389in}{2.397582in}}{\pgfqpoint{3.969999in}{2.386983in}}{\pgfqpoint{3.969999in}{2.375933in}}%
\pgfpathcurveto{\pgfqpoint{3.969999in}{2.364883in}}{\pgfqpoint{3.974389in}{2.354284in}}{\pgfqpoint{3.982203in}{2.346470in}}%
\pgfpathcurveto{\pgfqpoint{3.990016in}{2.338657in}}{\pgfqpoint{4.000616in}{2.334266in}}{\pgfqpoint{4.011666in}{2.334266in}}%
\pgfpathclose%
\pgfusepath{stroke,fill}%
\end{pgfscope}%
\begin{pgfscope}%
\pgfpathrectangle{\pgfqpoint{0.800000in}{0.528000in}}{\pgfqpoint{4.960000in}{3.696000in}}%
\pgfusepath{clip}%
\pgfsetbuttcap%
\pgfsetroundjoin%
\definecolor{currentfill}{rgb}{0.000000,0.000000,0.000000}%
\pgfsetfillcolor{currentfill}%
\pgfsetlinewidth{1.003750pt}%
\definecolor{currentstroke}{rgb}{0.000000,0.000000,0.000000}%
\pgfsetstrokecolor{currentstroke}%
\pgfsetdash{}{0pt}%
\pgfpathmoveto{\pgfqpoint{4.011666in}{2.334266in}}%
\pgfpathcurveto{\pgfqpoint{4.022716in}{2.334266in}}{\pgfqpoint{4.033315in}{2.338657in}}{\pgfqpoint{4.041128in}{2.346470in}}%
\pgfpathcurveto{\pgfqpoint{4.048942in}{2.354284in}}{\pgfqpoint{4.053332in}{2.364883in}}{\pgfqpoint{4.053332in}{2.375933in}}%
\pgfpathcurveto{\pgfqpoint{4.053332in}{2.386983in}}{\pgfqpoint{4.048942in}{2.397582in}}{\pgfqpoint{4.041128in}{2.405396in}}%
\pgfpathcurveto{\pgfqpoint{4.033315in}{2.413209in}}{\pgfqpoint{4.022716in}{2.417600in}}{\pgfqpoint{4.011666in}{2.417600in}}%
\pgfpathcurveto{\pgfqpoint{4.000616in}{2.417600in}}{\pgfqpoint{3.990016in}{2.413209in}}{\pgfqpoint{3.982203in}{2.405396in}}%
\pgfpathcurveto{\pgfqpoint{3.974389in}{2.397582in}}{\pgfqpoint{3.969999in}{2.386983in}}{\pgfqpoint{3.969999in}{2.375933in}}%
\pgfpathcurveto{\pgfqpoint{3.969999in}{2.364883in}}{\pgfqpoint{3.974389in}{2.354284in}}{\pgfqpoint{3.982203in}{2.346470in}}%
\pgfpathcurveto{\pgfqpoint{3.990016in}{2.338657in}}{\pgfqpoint{4.000616in}{2.334266in}}{\pgfqpoint{4.011666in}{2.334266in}}%
\pgfpathclose%
\pgfusepath{stroke,fill}%
\end{pgfscope}%
\begin{pgfscope}%
\pgfpathrectangle{\pgfqpoint{0.800000in}{0.528000in}}{\pgfqpoint{4.960000in}{3.696000in}}%
\pgfusepath{clip}%
\pgfsetbuttcap%
\pgfsetroundjoin%
\definecolor{currentfill}{rgb}{0.000000,0.000000,0.000000}%
\pgfsetfillcolor{currentfill}%
\pgfsetlinewidth{1.003750pt}%
\definecolor{currentstroke}{rgb}{0.000000,0.000000,0.000000}%
\pgfsetstrokecolor{currentstroke}%
\pgfsetdash{}{0pt}%
\pgfpathmoveto{\pgfqpoint{4.011666in}{3.984333in}}%
\pgfpathcurveto{\pgfqpoint{4.022716in}{3.984333in}}{\pgfqpoint{4.033315in}{3.988724in}}{\pgfqpoint{4.041128in}{3.996537in}}%
\pgfpathcurveto{\pgfqpoint{4.048942in}{4.004351in}}{\pgfqpoint{4.053332in}{4.014950in}}{\pgfqpoint{4.053332in}{4.026000in}}%
\pgfpathcurveto{\pgfqpoint{4.053332in}{4.037050in}}{\pgfqpoint{4.048942in}{4.047649in}}{\pgfqpoint{4.041128in}{4.055463in}}%
\pgfpathcurveto{\pgfqpoint{4.033315in}{4.063276in}}{\pgfqpoint{4.022716in}{4.067667in}}{\pgfqpoint{4.011666in}{4.067667in}}%
\pgfpathcurveto{\pgfqpoint{4.000616in}{4.067667in}}{\pgfqpoint{3.990016in}{4.063276in}}{\pgfqpoint{3.982203in}{4.055463in}}%
\pgfpathcurveto{\pgfqpoint{3.974389in}{4.047649in}}{\pgfqpoint{3.969999in}{4.037050in}}{\pgfqpoint{3.969999in}{4.026000in}}%
\pgfpathcurveto{\pgfqpoint{3.969999in}{4.014950in}}{\pgfqpoint{3.974389in}{4.004351in}}{\pgfqpoint{3.982203in}{3.996537in}}%
\pgfpathcurveto{\pgfqpoint{3.990016in}{3.988724in}}{\pgfqpoint{4.000616in}{3.984333in}}{\pgfqpoint{4.011666in}{3.984333in}}%
\pgfpathclose%
\pgfusepath{stroke,fill}%
\end{pgfscope}%
\begin{pgfscope}%
\pgfpathrectangle{\pgfqpoint{0.800000in}{0.528000in}}{\pgfqpoint{4.960000in}{3.696000in}}%
\pgfusepath{clip}%
\pgfsetbuttcap%
\pgfsetroundjoin%
\definecolor{currentfill}{rgb}{0.000000,0.000000,0.000000}%
\pgfsetfillcolor{currentfill}%
\pgfsetlinewidth{1.003750pt}%
\definecolor{currentstroke}{rgb}{0.000000,0.000000,0.000000}%
\pgfsetstrokecolor{currentstroke}%
\pgfsetdash{}{0pt}%
\pgfpathmoveto{\pgfqpoint{4.011666in}{2.334266in}}%
\pgfpathcurveto{\pgfqpoint{4.022716in}{2.334266in}}{\pgfqpoint{4.033315in}{2.338657in}}{\pgfqpoint{4.041128in}{2.346470in}}%
\pgfpathcurveto{\pgfqpoint{4.048942in}{2.354284in}}{\pgfqpoint{4.053332in}{2.364883in}}{\pgfqpoint{4.053332in}{2.375933in}}%
\pgfpathcurveto{\pgfqpoint{4.053332in}{2.386983in}}{\pgfqpoint{4.048942in}{2.397582in}}{\pgfqpoint{4.041128in}{2.405396in}}%
\pgfpathcurveto{\pgfqpoint{4.033315in}{2.413209in}}{\pgfqpoint{4.022716in}{2.417600in}}{\pgfqpoint{4.011666in}{2.417600in}}%
\pgfpathcurveto{\pgfqpoint{4.000616in}{2.417600in}}{\pgfqpoint{3.990016in}{2.413209in}}{\pgfqpoint{3.982203in}{2.405396in}}%
\pgfpathcurveto{\pgfqpoint{3.974389in}{2.397582in}}{\pgfqpoint{3.969999in}{2.386983in}}{\pgfqpoint{3.969999in}{2.375933in}}%
\pgfpathcurveto{\pgfqpoint{3.969999in}{2.364883in}}{\pgfqpoint{3.974389in}{2.354284in}}{\pgfqpoint{3.982203in}{2.346470in}}%
\pgfpathcurveto{\pgfqpoint{3.990016in}{2.338657in}}{\pgfqpoint{4.000616in}{2.334266in}}{\pgfqpoint{4.011666in}{2.334266in}}%
\pgfpathclose%
\pgfusepath{stroke,fill}%
\end{pgfscope}%
\begin{pgfscope}%
\pgfpathrectangle{\pgfqpoint{0.800000in}{0.528000in}}{\pgfqpoint{4.960000in}{3.696000in}}%
\pgfusepath{clip}%
\pgfsetbuttcap%
\pgfsetroundjoin%
\definecolor{currentfill}{rgb}{0.000000,0.000000,0.000000}%
\pgfsetfillcolor{currentfill}%
\pgfsetlinewidth{1.003750pt}%
\definecolor{currentstroke}{rgb}{0.000000,0.000000,0.000000}%
\pgfsetstrokecolor{currentstroke}%
\pgfsetdash{}{0pt}%
\pgfpathmoveto{\pgfqpoint{4.011666in}{2.334266in}}%
\pgfpathcurveto{\pgfqpoint{4.022716in}{2.334266in}}{\pgfqpoint{4.033315in}{2.338657in}}{\pgfqpoint{4.041128in}{2.346470in}}%
\pgfpathcurveto{\pgfqpoint{4.048942in}{2.354284in}}{\pgfqpoint{4.053332in}{2.364883in}}{\pgfqpoint{4.053332in}{2.375933in}}%
\pgfpathcurveto{\pgfqpoint{4.053332in}{2.386983in}}{\pgfqpoint{4.048942in}{2.397582in}}{\pgfqpoint{4.041128in}{2.405396in}}%
\pgfpathcurveto{\pgfqpoint{4.033315in}{2.413209in}}{\pgfqpoint{4.022716in}{2.417600in}}{\pgfqpoint{4.011666in}{2.417600in}}%
\pgfpathcurveto{\pgfqpoint{4.000616in}{2.417600in}}{\pgfqpoint{3.990016in}{2.413209in}}{\pgfqpoint{3.982203in}{2.405396in}}%
\pgfpathcurveto{\pgfqpoint{3.974389in}{2.397582in}}{\pgfqpoint{3.969999in}{2.386983in}}{\pgfqpoint{3.969999in}{2.375933in}}%
\pgfpathcurveto{\pgfqpoint{3.969999in}{2.364883in}}{\pgfqpoint{3.974389in}{2.354284in}}{\pgfqpoint{3.982203in}{2.346470in}}%
\pgfpathcurveto{\pgfqpoint{3.990016in}{2.338657in}}{\pgfqpoint{4.000616in}{2.334266in}}{\pgfqpoint{4.011666in}{2.334266in}}%
\pgfpathclose%
\pgfusepath{stroke,fill}%
\end{pgfscope}%
\begin{pgfscope}%
\pgfpathrectangle{\pgfqpoint{0.800000in}{0.528000in}}{\pgfqpoint{4.960000in}{3.696000in}}%
\pgfusepath{clip}%
\pgfsetbuttcap%
\pgfsetroundjoin%
\definecolor{currentfill}{rgb}{0.000000,0.000000,0.000000}%
\pgfsetfillcolor{currentfill}%
\pgfsetlinewidth{1.003750pt}%
\definecolor{currentstroke}{rgb}{0.000000,0.000000,0.000000}%
\pgfsetstrokecolor{currentstroke}%
\pgfsetdash{}{0pt}%
\pgfpathmoveto{\pgfqpoint{4.011666in}{3.984333in}}%
\pgfpathcurveto{\pgfqpoint{4.022716in}{3.984333in}}{\pgfqpoint{4.033315in}{3.988724in}}{\pgfqpoint{4.041128in}{3.996537in}}%
\pgfpathcurveto{\pgfqpoint{4.048942in}{4.004351in}}{\pgfqpoint{4.053332in}{4.014950in}}{\pgfqpoint{4.053332in}{4.026000in}}%
\pgfpathcurveto{\pgfqpoint{4.053332in}{4.037050in}}{\pgfqpoint{4.048942in}{4.047649in}}{\pgfqpoint{4.041128in}{4.055463in}}%
\pgfpathcurveto{\pgfqpoint{4.033315in}{4.063276in}}{\pgfqpoint{4.022716in}{4.067667in}}{\pgfqpoint{4.011666in}{4.067667in}}%
\pgfpathcurveto{\pgfqpoint{4.000616in}{4.067667in}}{\pgfqpoint{3.990016in}{4.063276in}}{\pgfqpoint{3.982203in}{4.055463in}}%
\pgfpathcurveto{\pgfqpoint{3.974389in}{4.047649in}}{\pgfqpoint{3.969999in}{4.037050in}}{\pgfqpoint{3.969999in}{4.026000in}}%
\pgfpathcurveto{\pgfqpoint{3.969999in}{4.014950in}}{\pgfqpoint{3.974389in}{4.004351in}}{\pgfqpoint{3.982203in}{3.996537in}}%
\pgfpathcurveto{\pgfqpoint{3.990016in}{3.988724in}}{\pgfqpoint{4.000616in}{3.984333in}}{\pgfqpoint{4.011666in}{3.984333in}}%
\pgfpathclose%
\pgfusepath{stroke,fill}%
\end{pgfscope}%
\begin{pgfscope}%
\pgfpathrectangle{\pgfqpoint{0.800000in}{0.528000in}}{\pgfqpoint{4.960000in}{3.696000in}}%
\pgfusepath{clip}%
\pgfsetbuttcap%
\pgfsetroundjoin%
\definecolor{currentfill}{rgb}{0.000000,0.000000,0.000000}%
\pgfsetfillcolor{currentfill}%
\pgfsetlinewidth{1.003750pt}%
\definecolor{currentstroke}{rgb}{0.000000,0.000000,0.000000}%
\pgfsetstrokecolor{currentstroke}%
\pgfsetdash{}{0pt}%
\pgfpathmoveto{\pgfqpoint{4.011666in}{2.334266in}}%
\pgfpathcurveto{\pgfqpoint{4.022716in}{2.334266in}}{\pgfqpoint{4.033315in}{2.338657in}}{\pgfqpoint{4.041128in}{2.346470in}}%
\pgfpathcurveto{\pgfqpoint{4.048942in}{2.354284in}}{\pgfqpoint{4.053332in}{2.364883in}}{\pgfqpoint{4.053332in}{2.375933in}}%
\pgfpathcurveto{\pgfqpoint{4.053332in}{2.386983in}}{\pgfqpoint{4.048942in}{2.397582in}}{\pgfqpoint{4.041128in}{2.405396in}}%
\pgfpathcurveto{\pgfqpoint{4.033315in}{2.413209in}}{\pgfqpoint{4.022716in}{2.417600in}}{\pgfqpoint{4.011666in}{2.417600in}}%
\pgfpathcurveto{\pgfqpoint{4.000616in}{2.417600in}}{\pgfqpoint{3.990016in}{2.413209in}}{\pgfqpoint{3.982203in}{2.405396in}}%
\pgfpathcurveto{\pgfqpoint{3.974389in}{2.397582in}}{\pgfqpoint{3.969999in}{2.386983in}}{\pgfqpoint{3.969999in}{2.375933in}}%
\pgfpathcurveto{\pgfqpoint{3.969999in}{2.364883in}}{\pgfqpoint{3.974389in}{2.354284in}}{\pgfqpoint{3.982203in}{2.346470in}}%
\pgfpathcurveto{\pgfqpoint{3.990016in}{2.338657in}}{\pgfqpoint{4.000616in}{2.334266in}}{\pgfqpoint{4.011666in}{2.334266in}}%
\pgfpathclose%
\pgfusepath{stroke,fill}%
\end{pgfscope}%
\begin{pgfscope}%
\pgfpathrectangle{\pgfqpoint{0.800000in}{0.528000in}}{\pgfqpoint{4.960000in}{3.696000in}}%
\pgfusepath{clip}%
\pgfsetbuttcap%
\pgfsetroundjoin%
\definecolor{currentfill}{rgb}{0.000000,0.000000,0.000000}%
\pgfsetfillcolor{currentfill}%
\pgfsetlinewidth{1.003750pt}%
\definecolor{currentstroke}{rgb}{0.000000,0.000000,0.000000}%
\pgfsetstrokecolor{currentstroke}%
\pgfsetdash{}{0pt}%
\pgfpathmoveto{\pgfqpoint{4.011666in}{2.334266in}}%
\pgfpathcurveto{\pgfqpoint{4.022716in}{2.334266in}}{\pgfqpoint{4.033315in}{2.338657in}}{\pgfqpoint{4.041128in}{2.346470in}}%
\pgfpathcurveto{\pgfqpoint{4.048942in}{2.354284in}}{\pgfqpoint{4.053332in}{2.364883in}}{\pgfqpoint{4.053332in}{2.375933in}}%
\pgfpathcurveto{\pgfqpoint{4.053332in}{2.386983in}}{\pgfqpoint{4.048942in}{2.397582in}}{\pgfqpoint{4.041128in}{2.405396in}}%
\pgfpathcurveto{\pgfqpoint{4.033315in}{2.413209in}}{\pgfqpoint{4.022716in}{2.417600in}}{\pgfqpoint{4.011666in}{2.417600in}}%
\pgfpathcurveto{\pgfqpoint{4.000616in}{2.417600in}}{\pgfqpoint{3.990016in}{2.413209in}}{\pgfqpoint{3.982203in}{2.405396in}}%
\pgfpathcurveto{\pgfqpoint{3.974389in}{2.397582in}}{\pgfqpoint{3.969999in}{2.386983in}}{\pgfqpoint{3.969999in}{2.375933in}}%
\pgfpathcurveto{\pgfqpoint{3.969999in}{2.364883in}}{\pgfqpoint{3.974389in}{2.354284in}}{\pgfqpoint{3.982203in}{2.346470in}}%
\pgfpathcurveto{\pgfqpoint{3.990016in}{2.338657in}}{\pgfqpoint{4.000616in}{2.334266in}}{\pgfqpoint{4.011666in}{2.334266in}}%
\pgfpathclose%
\pgfusepath{stroke,fill}%
\end{pgfscope}%
\begin{pgfscope}%
\pgfpathrectangle{\pgfqpoint{0.800000in}{0.528000in}}{\pgfqpoint{4.960000in}{3.696000in}}%
\pgfusepath{clip}%
\pgfsetbuttcap%
\pgfsetroundjoin%
\definecolor{currentfill}{rgb}{0.000000,0.000000,0.000000}%
\pgfsetfillcolor{currentfill}%
\pgfsetlinewidth{1.003750pt}%
\definecolor{currentstroke}{rgb}{0.000000,0.000000,0.000000}%
\pgfsetstrokecolor{currentstroke}%
\pgfsetdash{}{0pt}%
\pgfpathmoveto{\pgfqpoint{4.011666in}{3.984333in}}%
\pgfpathcurveto{\pgfqpoint{4.022716in}{3.984333in}}{\pgfqpoint{4.033315in}{3.988724in}}{\pgfqpoint{4.041128in}{3.996537in}}%
\pgfpathcurveto{\pgfqpoint{4.048942in}{4.004351in}}{\pgfqpoint{4.053332in}{4.014950in}}{\pgfqpoint{4.053332in}{4.026000in}}%
\pgfpathcurveto{\pgfqpoint{4.053332in}{4.037050in}}{\pgfqpoint{4.048942in}{4.047649in}}{\pgfqpoint{4.041128in}{4.055463in}}%
\pgfpathcurveto{\pgfqpoint{4.033315in}{4.063276in}}{\pgfqpoint{4.022716in}{4.067667in}}{\pgfqpoint{4.011666in}{4.067667in}}%
\pgfpathcurveto{\pgfqpoint{4.000616in}{4.067667in}}{\pgfqpoint{3.990016in}{4.063276in}}{\pgfqpoint{3.982203in}{4.055463in}}%
\pgfpathcurveto{\pgfqpoint{3.974389in}{4.047649in}}{\pgfqpoint{3.969999in}{4.037050in}}{\pgfqpoint{3.969999in}{4.026000in}}%
\pgfpathcurveto{\pgfqpoint{3.969999in}{4.014950in}}{\pgfqpoint{3.974389in}{4.004351in}}{\pgfqpoint{3.982203in}{3.996537in}}%
\pgfpathcurveto{\pgfqpoint{3.990016in}{3.988724in}}{\pgfqpoint{4.000616in}{3.984333in}}{\pgfqpoint{4.011666in}{3.984333in}}%
\pgfpathclose%
\pgfusepath{stroke,fill}%
\end{pgfscope}%
\begin{pgfscope}%
\pgfpathrectangle{\pgfqpoint{0.800000in}{0.528000in}}{\pgfqpoint{4.960000in}{3.696000in}}%
\pgfusepath{clip}%
\pgfsetbuttcap%
\pgfsetroundjoin%
\definecolor{currentfill}{rgb}{0.000000,0.000000,0.000000}%
\pgfsetfillcolor{currentfill}%
\pgfsetlinewidth{1.003750pt}%
\definecolor{currentstroke}{rgb}{0.000000,0.000000,0.000000}%
\pgfsetstrokecolor{currentstroke}%
\pgfsetdash{}{0pt}%
\pgfpathmoveto{\pgfqpoint{4.011666in}{2.334266in}}%
\pgfpathcurveto{\pgfqpoint{4.022716in}{2.334266in}}{\pgfqpoint{4.033315in}{2.338657in}}{\pgfqpoint{4.041128in}{2.346470in}}%
\pgfpathcurveto{\pgfqpoint{4.048942in}{2.354284in}}{\pgfqpoint{4.053332in}{2.364883in}}{\pgfqpoint{4.053332in}{2.375933in}}%
\pgfpathcurveto{\pgfqpoint{4.053332in}{2.386983in}}{\pgfqpoint{4.048942in}{2.397582in}}{\pgfqpoint{4.041128in}{2.405396in}}%
\pgfpathcurveto{\pgfqpoint{4.033315in}{2.413209in}}{\pgfqpoint{4.022716in}{2.417600in}}{\pgfqpoint{4.011666in}{2.417600in}}%
\pgfpathcurveto{\pgfqpoint{4.000616in}{2.417600in}}{\pgfqpoint{3.990016in}{2.413209in}}{\pgfqpoint{3.982203in}{2.405396in}}%
\pgfpathcurveto{\pgfqpoint{3.974389in}{2.397582in}}{\pgfqpoint{3.969999in}{2.386983in}}{\pgfqpoint{3.969999in}{2.375933in}}%
\pgfpathcurveto{\pgfqpoint{3.969999in}{2.364883in}}{\pgfqpoint{3.974389in}{2.354284in}}{\pgfqpoint{3.982203in}{2.346470in}}%
\pgfpathcurveto{\pgfqpoint{3.990016in}{2.338657in}}{\pgfqpoint{4.000616in}{2.334266in}}{\pgfqpoint{4.011666in}{2.334266in}}%
\pgfpathclose%
\pgfusepath{stroke,fill}%
\end{pgfscope}%
\begin{pgfscope}%
\pgfpathrectangle{\pgfqpoint{0.800000in}{0.528000in}}{\pgfqpoint{4.960000in}{3.696000in}}%
\pgfusepath{clip}%
\pgfsetbuttcap%
\pgfsetroundjoin%
\definecolor{currentfill}{rgb}{0.000000,0.000000,0.000000}%
\pgfsetfillcolor{currentfill}%
\pgfsetlinewidth{1.003750pt}%
\definecolor{currentstroke}{rgb}{0.000000,0.000000,0.000000}%
\pgfsetstrokecolor{currentstroke}%
\pgfsetdash{}{0pt}%
\pgfpathmoveto{\pgfqpoint{4.011666in}{2.334266in}}%
\pgfpathcurveto{\pgfqpoint{4.022716in}{2.334266in}}{\pgfqpoint{4.033315in}{2.338657in}}{\pgfqpoint{4.041128in}{2.346470in}}%
\pgfpathcurveto{\pgfqpoint{4.048942in}{2.354284in}}{\pgfqpoint{4.053332in}{2.364883in}}{\pgfqpoint{4.053332in}{2.375933in}}%
\pgfpathcurveto{\pgfqpoint{4.053332in}{2.386983in}}{\pgfqpoint{4.048942in}{2.397582in}}{\pgfqpoint{4.041128in}{2.405396in}}%
\pgfpathcurveto{\pgfqpoint{4.033315in}{2.413209in}}{\pgfqpoint{4.022716in}{2.417600in}}{\pgfqpoint{4.011666in}{2.417600in}}%
\pgfpathcurveto{\pgfqpoint{4.000616in}{2.417600in}}{\pgfqpoint{3.990016in}{2.413209in}}{\pgfqpoint{3.982203in}{2.405396in}}%
\pgfpathcurveto{\pgfqpoint{3.974389in}{2.397582in}}{\pgfqpoint{3.969999in}{2.386983in}}{\pgfqpoint{3.969999in}{2.375933in}}%
\pgfpathcurveto{\pgfqpoint{3.969999in}{2.364883in}}{\pgfqpoint{3.974389in}{2.354284in}}{\pgfqpoint{3.982203in}{2.346470in}}%
\pgfpathcurveto{\pgfqpoint{3.990016in}{2.338657in}}{\pgfqpoint{4.000616in}{2.334266in}}{\pgfqpoint{4.011666in}{2.334266in}}%
\pgfpathclose%
\pgfusepath{stroke,fill}%
\end{pgfscope}%
\begin{pgfscope}%
\pgfpathrectangle{\pgfqpoint{0.800000in}{0.528000in}}{\pgfqpoint{4.960000in}{3.696000in}}%
\pgfusepath{clip}%
\pgfsetbuttcap%
\pgfsetroundjoin%
\definecolor{currentfill}{rgb}{0.000000,0.000000,0.000000}%
\pgfsetfillcolor{currentfill}%
\pgfsetlinewidth{1.003750pt}%
\definecolor{currentstroke}{rgb}{0.000000,0.000000,0.000000}%
\pgfsetstrokecolor{currentstroke}%
\pgfsetdash{}{0pt}%
\pgfpathmoveto{\pgfqpoint{4.011666in}{3.984333in}}%
\pgfpathcurveto{\pgfqpoint{4.022716in}{3.984333in}}{\pgfqpoint{4.033315in}{3.988724in}}{\pgfqpoint{4.041128in}{3.996537in}}%
\pgfpathcurveto{\pgfqpoint{4.048942in}{4.004351in}}{\pgfqpoint{4.053332in}{4.014950in}}{\pgfqpoint{4.053332in}{4.026000in}}%
\pgfpathcurveto{\pgfqpoint{4.053332in}{4.037050in}}{\pgfqpoint{4.048942in}{4.047649in}}{\pgfqpoint{4.041128in}{4.055463in}}%
\pgfpathcurveto{\pgfqpoint{4.033315in}{4.063276in}}{\pgfqpoint{4.022716in}{4.067667in}}{\pgfqpoint{4.011666in}{4.067667in}}%
\pgfpathcurveto{\pgfqpoint{4.000616in}{4.067667in}}{\pgfqpoint{3.990016in}{4.063276in}}{\pgfqpoint{3.982203in}{4.055463in}}%
\pgfpathcurveto{\pgfqpoint{3.974389in}{4.047649in}}{\pgfqpoint{3.969999in}{4.037050in}}{\pgfqpoint{3.969999in}{4.026000in}}%
\pgfpathcurveto{\pgfqpoint{3.969999in}{4.014950in}}{\pgfqpoint{3.974389in}{4.004351in}}{\pgfqpoint{3.982203in}{3.996537in}}%
\pgfpathcurveto{\pgfqpoint{3.990016in}{3.988724in}}{\pgfqpoint{4.000616in}{3.984333in}}{\pgfqpoint{4.011666in}{3.984333in}}%
\pgfpathclose%
\pgfusepath{stroke,fill}%
\end{pgfscope}%
\begin{pgfscope}%
\pgfpathrectangle{\pgfqpoint{0.800000in}{0.528000in}}{\pgfqpoint{4.960000in}{3.696000in}}%
\pgfusepath{clip}%
\pgfsetbuttcap%
\pgfsetroundjoin%
\definecolor{currentfill}{rgb}{0.000000,0.000000,0.000000}%
\pgfsetfillcolor{currentfill}%
\pgfsetlinewidth{1.003750pt}%
\definecolor{currentstroke}{rgb}{0.000000,0.000000,0.000000}%
\pgfsetstrokecolor{currentstroke}%
\pgfsetdash{}{0pt}%
\pgfpathmoveto{\pgfqpoint{4.011666in}{2.334266in}}%
\pgfpathcurveto{\pgfqpoint{4.022716in}{2.334266in}}{\pgfqpoint{4.033315in}{2.338657in}}{\pgfqpoint{4.041128in}{2.346470in}}%
\pgfpathcurveto{\pgfqpoint{4.048942in}{2.354284in}}{\pgfqpoint{4.053332in}{2.364883in}}{\pgfqpoint{4.053332in}{2.375933in}}%
\pgfpathcurveto{\pgfqpoint{4.053332in}{2.386983in}}{\pgfqpoint{4.048942in}{2.397582in}}{\pgfqpoint{4.041128in}{2.405396in}}%
\pgfpathcurveto{\pgfqpoint{4.033315in}{2.413209in}}{\pgfqpoint{4.022716in}{2.417600in}}{\pgfqpoint{4.011666in}{2.417600in}}%
\pgfpathcurveto{\pgfqpoint{4.000616in}{2.417600in}}{\pgfqpoint{3.990016in}{2.413209in}}{\pgfqpoint{3.982203in}{2.405396in}}%
\pgfpathcurveto{\pgfqpoint{3.974389in}{2.397582in}}{\pgfqpoint{3.969999in}{2.386983in}}{\pgfqpoint{3.969999in}{2.375933in}}%
\pgfpathcurveto{\pgfqpoint{3.969999in}{2.364883in}}{\pgfqpoint{3.974389in}{2.354284in}}{\pgfqpoint{3.982203in}{2.346470in}}%
\pgfpathcurveto{\pgfqpoint{3.990016in}{2.338657in}}{\pgfqpoint{4.000616in}{2.334266in}}{\pgfqpoint{4.011666in}{2.334266in}}%
\pgfpathclose%
\pgfusepath{stroke,fill}%
\end{pgfscope}%
\begin{pgfscope}%
\pgfpathrectangle{\pgfqpoint{0.800000in}{0.528000in}}{\pgfqpoint{4.960000in}{3.696000in}}%
\pgfusepath{clip}%
\pgfsetbuttcap%
\pgfsetroundjoin%
\definecolor{currentfill}{rgb}{0.000000,0.000000,0.000000}%
\pgfsetfillcolor{currentfill}%
\pgfsetlinewidth{1.003750pt}%
\definecolor{currentstroke}{rgb}{0.000000,0.000000,0.000000}%
\pgfsetstrokecolor{currentstroke}%
\pgfsetdash{}{0pt}%
\pgfpathmoveto{\pgfqpoint{4.011666in}{2.334266in}}%
\pgfpathcurveto{\pgfqpoint{4.022716in}{2.334266in}}{\pgfqpoint{4.033315in}{2.338657in}}{\pgfqpoint{4.041128in}{2.346470in}}%
\pgfpathcurveto{\pgfqpoint{4.048942in}{2.354284in}}{\pgfqpoint{4.053332in}{2.364883in}}{\pgfqpoint{4.053332in}{2.375933in}}%
\pgfpathcurveto{\pgfqpoint{4.053332in}{2.386983in}}{\pgfqpoint{4.048942in}{2.397582in}}{\pgfqpoint{4.041128in}{2.405396in}}%
\pgfpathcurveto{\pgfqpoint{4.033315in}{2.413209in}}{\pgfqpoint{4.022716in}{2.417600in}}{\pgfqpoint{4.011666in}{2.417600in}}%
\pgfpathcurveto{\pgfqpoint{4.000616in}{2.417600in}}{\pgfqpoint{3.990016in}{2.413209in}}{\pgfqpoint{3.982203in}{2.405396in}}%
\pgfpathcurveto{\pgfqpoint{3.974389in}{2.397582in}}{\pgfqpoint{3.969999in}{2.386983in}}{\pgfqpoint{3.969999in}{2.375933in}}%
\pgfpathcurveto{\pgfqpoint{3.969999in}{2.364883in}}{\pgfqpoint{3.974389in}{2.354284in}}{\pgfqpoint{3.982203in}{2.346470in}}%
\pgfpathcurveto{\pgfqpoint{3.990016in}{2.338657in}}{\pgfqpoint{4.000616in}{2.334266in}}{\pgfqpoint{4.011666in}{2.334266in}}%
\pgfpathclose%
\pgfusepath{stroke,fill}%
\end{pgfscope}%
\begin{pgfscope}%
\pgfpathrectangle{\pgfqpoint{0.800000in}{0.528000in}}{\pgfqpoint{4.960000in}{3.696000in}}%
\pgfusepath{clip}%
\pgfsetbuttcap%
\pgfsetroundjoin%
\definecolor{currentfill}{rgb}{0.000000,0.000000,0.000000}%
\pgfsetfillcolor{currentfill}%
\pgfsetlinewidth{1.003750pt}%
\definecolor{currentstroke}{rgb}{0.000000,0.000000,0.000000}%
\pgfsetstrokecolor{currentstroke}%
\pgfsetdash{}{0pt}%
\pgfpathmoveto{\pgfqpoint{4.011666in}{2.334266in}}%
\pgfpathcurveto{\pgfqpoint{4.022716in}{2.334266in}}{\pgfqpoint{4.033315in}{2.338657in}}{\pgfqpoint{4.041128in}{2.346470in}}%
\pgfpathcurveto{\pgfqpoint{4.048942in}{2.354284in}}{\pgfqpoint{4.053332in}{2.364883in}}{\pgfqpoint{4.053332in}{2.375933in}}%
\pgfpathcurveto{\pgfqpoint{4.053332in}{2.386983in}}{\pgfqpoint{4.048942in}{2.397582in}}{\pgfqpoint{4.041128in}{2.405396in}}%
\pgfpathcurveto{\pgfqpoint{4.033315in}{2.413209in}}{\pgfqpoint{4.022716in}{2.417600in}}{\pgfqpoint{4.011666in}{2.417600in}}%
\pgfpathcurveto{\pgfqpoint{4.000616in}{2.417600in}}{\pgfqpoint{3.990016in}{2.413209in}}{\pgfqpoint{3.982203in}{2.405396in}}%
\pgfpathcurveto{\pgfqpoint{3.974389in}{2.397582in}}{\pgfqpoint{3.969999in}{2.386983in}}{\pgfqpoint{3.969999in}{2.375933in}}%
\pgfpathcurveto{\pgfqpoint{3.969999in}{2.364883in}}{\pgfqpoint{3.974389in}{2.354284in}}{\pgfqpoint{3.982203in}{2.346470in}}%
\pgfpathcurveto{\pgfqpoint{3.990016in}{2.338657in}}{\pgfqpoint{4.000616in}{2.334266in}}{\pgfqpoint{4.011666in}{2.334266in}}%
\pgfpathclose%
\pgfusepath{stroke,fill}%
\end{pgfscope}%
\begin{pgfscope}%
\pgfpathrectangle{\pgfqpoint{0.800000in}{0.528000in}}{\pgfqpoint{4.960000in}{3.696000in}}%
\pgfusepath{clip}%
\pgfsetbuttcap%
\pgfsetroundjoin%
\definecolor{currentfill}{rgb}{0.000000,0.000000,0.000000}%
\pgfsetfillcolor{currentfill}%
\pgfsetlinewidth{1.003750pt}%
\definecolor{currentstroke}{rgb}{0.000000,0.000000,0.000000}%
\pgfsetstrokecolor{currentstroke}%
\pgfsetdash{}{0pt}%
\pgfpathmoveto{\pgfqpoint{4.011666in}{3.984333in}}%
\pgfpathcurveto{\pgfqpoint{4.022716in}{3.984333in}}{\pgfqpoint{4.033315in}{3.988724in}}{\pgfqpoint{4.041128in}{3.996537in}}%
\pgfpathcurveto{\pgfqpoint{4.048942in}{4.004351in}}{\pgfqpoint{4.053332in}{4.014950in}}{\pgfqpoint{4.053332in}{4.026000in}}%
\pgfpathcurveto{\pgfqpoint{4.053332in}{4.037050in}}{\pgfqpoint{4.048942in}{4.047649in}}{\pgfqpoint{4.041128in}{4.055463in}}%
\pgfpathcurveto{\pgfqpoint{4.033315in}{4.063276in}}{\pgfqpoint{4.022716in}{4.067667in}}{\pgfqpoint{4.011666in}{4.067667in}}%
\pgfpathcurveto{\pgfqpoint{4.000616in}{4.067667in}}{\pgfqpoint{3.990016in}{4.063276in}}{\pgfqpoint{3.982203in}{4.055463in}}%
\pgfpathcurveto{\pgfqpoint{3.974389in}{4.047649in}}{\pgfqpoint{3.969999in}{4.037050in}}{\pgfqpoint{3.969999in}{4.026000in}}%
\pgfpathcurveto{\pgfqpoint{3.969999in}{4.014950in}}{\pgfqpoint{3.974389in}{4.004351in}}{\pgfqpoint{3.982203in}{3.996537in}}%
\pgfpathcurveto{\pgfqpoint{3.990016in}{3.988724in}}{\pgfqpoint{4.000616in}{3.984333in}}{\pgfqpoint{4.011666in}{3.984333in}}%
\pgfpathclose%
\pgfusepath{stroke,fill}%
\end{pgfscope}%
\begin{pgfscope}%
\pgfpathrectangle{\pgfqpoint{0.800000in}{0.528000in}}{\pgfqpoint{4.960000in}{3.696000in}}%
\pgfusepath{clip}%
\pgfsetbuttcap%
\pgfsetroundjoin%
\definecolor{currentfill}{rgb}{0.000000,0.000000,0.000000}%
\pgfsetfillcolor{currentfill}%
\pgfsetlinewidth{1.003750pt}%
\definecolor{currentstroke}{rgb}{0.000000,0.000000,0.000000}%
\pgfsetstrokecolor{currentstroke}%
\pgfsetdash{}{0pt}%
\pgfpathmoveto{\pgfqpoint{4.011666in}{2.334266in}}%
\pgfpathcurveto{\pgfqpoint{4.022716in}{2.334266in}}{\pgfqpoint{4.033315in}{2.338657in}}{\pgfqpoint{4.041128in}{2.346470in}}%
\pgfpathcurveto{\pgfqpoint{4.048942in}{2.354284in}}{\pgfqpoint{4.053332in}{2.364883in}}{\pgfqpoint{4.053332in}{2.375933in}}%
\pgfpathcurveto{\pgfqpoint{4.053332in}{2.386983in}}{\pgfqpoint{4.048942in}{2.397582in}}{\pgfqpoint{4.041128in}{2.405396in}}%
\pgfpathcurveto{\pgfqpoint{4.033315in}{2.413209in}}{\pgfqpoint{4.022716in}{2.417600in}}{\pgfqpoint{4.011666in}{2.417600in}}%
\pgfpathcurveto{\pgfqpoint{4.000616in}{2.417600in}}{\pgfqpoint{3.990016in}{2.413209in}}{\pgfqpoint{3.982203in}{2.405396in}}%
\pgfpathcurveto{\pgfqpoint{3.974389in}{2.397582in}}{\pgfqpoint{3.969999in}{2.386983in}}{\pgfqpoint{3.969999in}{2.375933in}}%
\pgfpathcurveto{\pgfqpoint{3.969999in}{2.364883in}}{\pgfqpoint{3.974389in}{2.354284in}}{\pgfqpoint{3.982203in}{2.346470in}}%
\pgfpathcurveto{\pgfqpoint{3.990016in}{2.338657in}}{\pgfqpoint{4.000616in}{2.334266in}}{\pgfqpoint{4.011666in}{2.334266in}}%
\pgfpathclose%
\pgfusepath{stroke,fill}%
\end{pgfscope}%
\begin{pgfscope}%
\pgfpathrectangle{\pgfqpoint{0.800000in}{0.528000in}}{\pgfqpoint{4.960000in}{3.696000in}}%
\pgfusepath{clip}%
\pgfsetbuttcap%
\pgfsetroundjoin%
\definecolor{currentfill}{rgb}{0.000000,0.000000,0.000000}%
\pgfsetfillcolor{currentfill}%
\pgfsetlinewidth{1.003750pt}%
\definecolor{currentstroke}{rgb}{0.000000,0.000000,0.000000}%
\pgfsetstrokecolor{currentstroke}%
\pgfsetdash{}{0pt}%
\pgfpathmoveto{\pgfqpoint{4.011666in}{2.334266in}}%
\pgfpathcurveto{\pgfqpoint{4.022716in}{2.334266in}}{\pgfqpoint{4.033315in}{2.338657in}}{\pgfqpoint{4.041128in}{2.346470in}}%
\pgfpathcurveto{\pgfqpoint{4.048942in}{2.354284in}}{\pgfqpoint{4.053332in}{2.364883in}}{\pgfqpoint{4.053332in}{2.375933in}}%
\pgfpathcurveto{\pgfqpoint{4.053332in}{2.386983in}}{\pgfqpoint{4.048942in}{2.397582in}}{\pgfqpoint{4.041128in}{2.405396in}}%
\pgfpathcurveto{\pgfqpoint{4.033315in}{2.413209in}}{\pgfqpoint{4.022716in}{2.417600in}}{\pgfqpoint{4.011666in}{2.417600in}}%
\pgfpathcurveto{\pgfqpoint{4.000616in}{2.417600in}}{\pgfqpoint{3.990016in}{2.413209in}}{\pgfqpoint{3.982203in}{2.405396in}}%
\pgfpathcurveto{\pgfqpoint{3.974389in}{2.397582in}}{\pgfqpoint{3.969999in}{2.386983in}}{\pgfqpoint{3.969999in}{2.375933in}}%
\pgfpathcurveto{\pgfqpoint{3.969999in}{2.364883in}}{\pgfqpoint{3.974389in}{2.354284in}}{\pgfqpoint{3.982203in}{2.346470in}}%
\pgfpathcurveto{\pgfqpoint{3.990016in}{2.338657in}}{\pgfqpoint{4.000616in}{2.334266in}}{\pgfqpoint{4.011666in}{2.334266in}}%
\pgfpathclose%
\pgfusepath{stroke,fill}%
\end{pgfscope}%
\begin{pgfscope}%
\pgfpathrectangle{\pgfqpoint{0.800000in}{0.528000in}}{\pgfqpoint{4.960000in}{3.696000in}}%
\pgfusepath{clip}%
\pgfsetbuttcap%
\pgfsetroundjoin%
\definecolor{currentfill}{rgb}{0.000000,0.000000,0.000000}%
\pgfsetfillcolor{currentfill}%
\pgfsetlinewidth{1.003750pt}%
\definecolor{currentstroke}{rgb}{0.000000,0.000000,0.000000}%
\pgfsetstrokecolor{currentstroke}%
\pgfsetdash{}{0pt}%
\pgfpathmoveto{\pgfqpoint{4.011666in}{2.334266in}}%
\pgfpathcurveto{\pgfqpoint{4.022716in}{2.334266in}}{\pgfqpoint{4.033315in}{2.338657in}}{\pgfqpoint{4.041128in}{2.346470in}}%
\pgfpathcurveto{\pgfqpoint{4.048942in}{2.354284in}}{\pgfqpoint{4.053332in}{2.364883in}}{\pgfqpoint{4.053332in}{2.375933in}}%
\pgfpathcurveto{\pgfqpoint{4.053332in}{2.386983in}}{\pgfqpoint{4.048942in}{2.397582in}}{\pgfqpoint{4.041128in}{2.405396in}}%
\pgfpathcurveto{\pgfqpoint{4.033315in}{2.413209in}}{\pgfqpoint{4.022716in}{2.417600in}}{\pgfqpoint{4.011666in}{2.417600in}}%
\pgfpathcurveto{\pgfqpoint{4.000616in}{2.417600in}}{\pgfqpoint{3.990016in}{2.413209in}}{\pgfqpoint{3.982203in}{2.405396in}}%
\pgfpathcurveto{\pgfqpoint{3.974389in}{2.397582in}}{\pgfqpoint{3.969999in}{2.386983in}}{\pgfqpoint{3.969999in}{2.375933in}}%
\pgfpathcurveto{\pgfqpoint{3.969999in}{2.364883in}}{\pgfqpoint{3.974389in}{2.354284in}}{\pgfqpoint{3.982203in}{2.346470in}}%
\pgfpathcurveto{\pgfqpoint{3.990016in}{2.338657in}}{\pgfqpoint{4.000616in}{2.334266in}}{\pgfqpoint{4.011666in}{2.334266in}}%
\pgfpathclose%
\pgfusepath{stroke,fill}%
\end{pgfscope}%
\begin{pgfscope}%
\pgfpathrectangle{\pgfqpoint{0.800000in}{0.528000in}}{\pgfqpoint{4.960000in}{3.696000in}}%
\pgfusepath{clip}%
\pgfsetbuttcap%
\pgfsetroundjoin%
\definecolor{currentfill}{rgb}{0.000000,0.000000,0.000000}%
\pgfsetfillcolor{currentfill}%
\pgfsetlinewidth{1.003750pt}%
\definecolor{currentstroke}{rgb}{0.000000,0.000000,0.000000}%
\pgfsetstrokecolor{currentstroke}%
\pgfsetdash{}{0pt}%
\pgfpathmoveto{\pgfqpoint{4.011666in}{2.334266in}}%
\pgfpathcurveto{\pgfqpoint{4.022716in}{2.334266in}}{\pgfqpoint{4.033315in}{2.338657in}}{\pgfqpoint{4.041128in}{2.346470in}}%
\pgfpathcurveto{\pgfqpoint{4.048942in}{2.354284in}}{\pgfqpoint{4.053332in}{2.364883in}}{\pgfqpoint{4.053332in}{2.375933in}}%
\pgfpathcurveto{\pgfqpoint{4.053332in}{2.386983in}}{\pgfqpoint{4.048942in}{2.397582in}}{\pgfqpoint{4.041128in}{2.405396in}}%
\pgfpathcurveto{\pgfqpoint{4.033315in}{2.413209in}}{\pgfqpoint{4.022716in}{2.417600in}}{\pgfqpoint{4.011666in}{2.417600in}}%
\pgfpathcurveto{\pgfqpoint{4.000616in}{2.417600in}}{\pgfqpoint{3.990016in}{2.413209in}}{\pgfqpoint{3.982203in}{2.405396in}}%
\pgfpathcurveto{\pgfqpoint{3.974389in}{2.397582in}}{\pgfqpoint{3.969999in}{2.386983in}}{\pgfqpoint{3.969999in}{2.375933in}}%
\pgfpathcurveto{\pgfqpoint{3.969999in}{2.364883in}}{\pgfqpoint{3.974389in}{2.354284in}}{\pgfqpoint{3.982203in}{2.346470in}}%
\pgfpathcurveto{\pgfqpoint{3.990016in}{2.338657in}}{\pgfqpoint{4.000616in}{2.334266in}}{\pgfqpoint{4.011666in}{2.334266in}}%
\pgfpathclose%
\pgfusepath{stroke,fill}%
\end{pgfscope}%
\begin{pgfscope}%
\pgfpathrectangle{\pgfqpoint{0.800000in}{0.528000in}}{\pgfqpoint{4.960000in}{3.696000in}}%
\pgfusepath{clip}%
\pgfsetbuttcap%
\pgfsetroundjoin%
\definecolor{currentfill}{rgb}{0.000000,0.000000,0.000000}%
\pgfsetfillcolor{currentfill}%
\pgfsetlinewidth{1.003750pt}%
\definecolor{currentstroke}{rgb}{0.000000,0.000000,0.000000}%
\pgfsetstrokecolor{currentstroke}%
\pgfsetdash{}{0pt}%
\pgfpathmoveto{\pgfqpoint{4.011666in}{2.334266in}}%
\pgfpathcurveto{\pgfqpoint{4.022716in}{2.334266in}}{\pgfqpoint{4.033315in}{2.338657in}}{\pgfqpoint{4.041128in}{2.346470in}}%
\pgfpathcurveto{\pgfqpoint{4.048942in}{2.354284in}}{\pgfqpoint{4.053332in}{2.364883in}}{\pgfqpoint{4.053332in}{2.375933in}}%
\pgfpathcurveto{\pgfqpoint{4.053332in}{2.386983in}}{\pgfqpoint{4.048942in}{2.397582in}}{\pgfqpoint{4.041128in}{2.405396in}}%
\pgfpathcurveto{\pgfqpoint{4.033315in}{2.413209in}}{\pgfqpoint{4.022716in}{2.417600in}}{\pgfqpoint{4.011666in}{2.417600in}}%
\pgfpathcurveto{\pgfqpoint{4.000616in}{2.417600in}}{\pgfqpoint{3.990016in}{2.413209in}}{\pgfqpoint{3.982203in}{2.405396in}}%
\pgfpathcurveto{\pgfqpoint{3.974389in}{2.397582in}}{\pgfqpoint{3.969999in}{2.386983in}}{\pgfqpoint{3.969999in}{2.375933in}}%
\pgfpathcurveto{\pgfqpoint{3.969999in}{2.364883in}}{\pgfqpoint{3.974389in}{2.354284in}}{\pgfqpoint{3.982203in}{2.346470in}}%
\pgfpathcurveto{\pgfqpoint{3.990016in}{2.338657in}}{\pgfqpoint{4.000616in}{2.334266in}}{\pgfqpoint{4.011666in}{2.334266in}}%
\pgfpathclose%
\pgfusepath{stroke,fill}%
\end{pgfscope}%
\begin{pgfscope}%
\pgfpathrectangle{\pgfqpoint{0.800000in}{0.528000in}}{\pgfqpoint{4.960000in}{3.696000in}}%
\pgfusepath{clip}%
\pgfsetbuttcap%
\pgfsetroundjoin%
\definecolor{currentfill}{rgb}{0.000000,0.000000,0.000000}%
\pgfsetfillcolor{currentfill}%
\pgfsetlinewidth{1.003750pt}%
\definecolor{currentstroke}{rgb}{0.000000,0.000000,0.000000}%
\pgfsetstrokecolor{currentstroke}%
\pgfsetdash{}{0pt}%
\pgfpathmoveto{\pgfqpoint{4.011666in}{2.334266in}}%
\pgfpathcurveto{\pgfqpoint{4.022716in}{2.334266in}}{\pgfqpoint{4.033315in}{2.338657in}}{\pgfqpoint{4.041128in}{2.346470in}}%
\pgfpathcurveto{\pgfqpoint{4.048942in}{2.354284in}}{\pgfqpoint{4.053332in}{2.364883in}}{\pgfqpoint{4.053332in}{2.375933in}}%
\pgfpathcurveto{\pgfqpoint{4.053332in}{2.386983in}}{\pgfqpoint{4.048942in}{2.397582in}}{\pgfqpoint{4.041128in}{2.405396in}}%
\pgfpathcurveto{\pgfqpoint{4.033315in}{2.413209in}}{\pgfqpoint{4.022716in}{2.417600in}}{\pgfqpoint{4.011666in}{2.417600in}}%
\pgfpathcurveto{\pgfqpoint{4.000616in}{2.417600in}}{\pgfqpoint{3.990016in}{2.413209in}}{\pgfqpoint{3.982203in}{2.405396in}}%
\pgfpathcurveto{\pgfqpoint{3.974389in}{2.397582in}}{\pgfqpoint{3.969999in}{2.386983in}}{\pgfqpoint{3.969999in}{2.375933in}}%
\pgfpathcurveto{\pgfqpoint{3.969999in}{2.364883in}}{\pgfqpoint{3.974389in}{2.354284in}}{\pgfqpoint{3.982203in}{2.346470in}}%
\pgfpathcurveto{\pgfqpoint{3.990016in}{2.338657in}}{\pgfqpoint{4.000616in}{2.334266in}}{\pgfqpoint{4.011666in}{2.334266in}}%
\pgfpathclose%
\pgfusepath{stroke,fill}%
\end{pgfscope}%
\begin{pgfscope}%
\pgfpathrectangle{\pgfqpoint{0.800000in}{0.528000in}}{\pgfqpoint{4.960000in}{3.696000in}}%
\pgfusepath{clip}%
\pgfsetbuttcap%
\pgfsetroundjoin%
\definecolor{currentfill}{rgb}{0.000000,0.000000,0.000000}%
\pgfsetfillcolor{currentfill}%
\pgfsetlinewidth{1.003750pt}%
\definecolor{currentstroke}{rgb}{0.000000,0.000000,0.000000}%
\pgfsetstrokecolor{currentstroke}%
\pgfsetdash{}{0pt}%
\pgfpathmoveto{\pgfqpoint{4.011666in}{2.334266in}}%
\pgfpathcurveto{\pgfqpoint{4.022716in}{2.334266in}}{\pgfqpoint{4.033315in}{2.338657in}}{\pgfqpoint{4.041128in}{2.346470in}}%
\pgfpathcurveto{\pgfqpoint{4.048942in}{2.354284in}}{\pgfqpoint{4.053332in}{2.364883in}}{\pgfqpoint{4.053332in}{2.375933in}}%
\pgfpathcurveto{\pgfqpoint{4.053332in}{2.386983in}}{\pgfqpoint{4.048942in}{2.397582in}}{\pgfqpoint{4.041128in}{2.405396in}}%
\pgfpathcurveto{\pgfqpoint{4.033315in}{2.413209in}}{\pgfqpoint{4.022716in}{2.417600in}}{\pgfqpoint{4.011666in}{2.417600in}}%
\pgfpathcurveto{\pgfqpoint{4.000616in}{2.417600in}}{\pgfqpoint{3.990016in}{2.413209in}}{\pgfqpoint{3.982203in}{2.405396in}}%
\pgfpathcurveto{\pgfqpoint{3.974389in}{2.397582in}}{\pgfqpoint{3.969999in}{2.386983in}}{\pgfqpoint{3.969999in}{2.375933in}}%
\pgfpathcurveto{\pgfqpoint{3.969999in}{2.364883in}}{\pgfqpoint{3.974389in}{2.354284in}}{\pgfqpoint{3.982203in}{2.346470in}}%
\pgfpathcurveto{\pgfqpoint{3.990016in}{2.338657in}}{\pgfqpoint{4.000616in}{2.334266in}}{\pgfqpoint{4.011666in}{2.334266in}}%
\pgfpathclose%
\pgfusepath{stroke,fill}%
\end{pgfscope}%
\begin{pgfscope}%
\pgfpathrectangle{\pgfqpoint{0.800000in}{0.528000in}}{\pgfqpoint{4.960000in}{3.696000in}}%
\pgfusepath{clip}%
\pgfsetbuttcap%
\pgfsetroundjoin%
\definecolor{currentfill}{rgb}{0.000000,0.000000,0.000000}%
\pgfsetfillcolor{currentfill}%
\pgfsetlinewidth{1.003750pt}%
\definecolor{currentstroke}{rgb}{0.000000,0.000000,0.000000}%
\pgfsetstrokecolor{currentstroke}%
\pgfsetdash{}{0pt}%
\pgfpathmoveto{\pgfqpoint{4.011666in}{3.984333in}}%
\pgfpathcurveto{\pgfqpoint{4.022716in}{3.984333in}}{\pgfqpoint{4.033315in}{3.988724in}}{\pgfqpoint{4.041128in}{3.996537in}}%
\pgfpathcurveto{\pgfqpoint{4.048942in}{4.004351in}}{\pgfqpoint{4.053332in}{4.014950in}}{\pgfqpoint{4.053332in}{4.026000in}}%
\pgfpathcurveto{\pgfqpoint{4.053332in}{4.037050in}}{\pgfqpoint{4.048942in}{4.047649in}}{\pgfqpoint{4.041128in}{4.055463in}}%
\pgfpathcurveto{\pgfqpoint{4.033315in}{4.063276in}}{\pgfqpoint{4.022716in}{4.067667in}}{\pgfqpoint{4.011666in}{4.067667in}}%
\pgfpathcurveto{\pgfqpoint{4.000616in}{4.067667in}}{\pgfqpoint{3.990016in}{4.063276in}}{\pgfqpoint{3.982203in}{4.055463in}}%
\pgfpathcurveto{\pgfqpoint{3.974389in}{4.047649in}}{\pgfqpoint{3.969999in}{4.037050in}}{\pgfqpoint{3.969999in}{4.026000in}}%
\pgfpathcurveto{\pgfqpoint{3.969999in}{4.014950in}}{\pgfqpoint{3.974389in}{4.004351in}}{\pgfqpoint{3.982203in}{3.996537in}}%
\pgfpathcurveto{\pgfqpoint{3.990016in}{3.988724in}}{\pgfqpoint{4.000616in}{3.984333in}}{\pgfqpoint{4.011666in}{3.984333in}}%
\pgfpathclose%
\pgfusepath{stroke,fill}%
\end{pgfscope}%
\begin{pgfscope}%
\pgfpathrectangle{\pgfqpoint{0.800000in}{0.528000in}}{\pgfqpoint{4.960000in}{3.696000in}}%
\pgfusepath{clip}%
\pgfsetbuttcap%
\pgfsetroundjoin%
\definecolor{currentfill}{rgb}{0.000000,0.000000,0.000000}%
\pgfsetfillcolor{currentfill}%
\pgfsetlinewidth{1.003750pt}%
\definecolor{currentstroke}{rgb}{0.000000,0.000000,0.000000}%
\pgfsetstrokecolor{currentstroke}%
\pgfsetdash{}{0pt}%
\pgfpathmoveto{\pgfqpoint{4.011666in}{2.334266in}}%
\pgfpathcurveto{\pgfqpoint{4.022716in}{2.334266in}}{\pgfqpoint{4.033315in}{2.338657in}}{\pgfqpoint{4.041128in}{2.346470in}}%
\pgfpathcurveto{\pgfqpoint{4.048942in}{2.354284in}}{\pgfqpoint{4.053332in}{2.364883in}}{\pgfqpoint{4.053332in}{2.375933in}}%
\pgfpathcurveto{\pgfqpoint{4.053332in}{2.386983in}}{\pgfqpoint{4.048942in}{2.397582in}}{\pgfqpoint{4.041128in}{2.405396in}}%
\pgfpathcurveto{\pgfqpoint{4.033315in}{2.413209in}}{\pgfqpoint{4.022716in}{2.417600in}}{\pgfqpoint{4.011666in}{2.417600in}}%
\pgfpathcurveto{\pgfqpoint{4.000616in}{2.417600in}}{\pgfqpoint{3.990016in}{2.413209in}}{\pgfqpoint{3.982203in}{2.405396in}}%
\pgfpathcurveto{\pgfqpoint{3.974389in}{2.397582in}}{\pgfqpoint{3.969999in}{2.386983in}}{\pgfqpoint{3.969999in}{2.375933in}}%
\pgfpathcurveto{\pgfqpoint{3.969999in}{2.364883in}}{\pgfqpoint{3.974389in}{2.354284in}}{\pgfqpoint{3.982203in}{2.346470in}}%
\pgfpathcurveto{\pgfqpoint{3.990016in}{2.338657in}}{\pgfqpoint{4.000616in}{2.334266in}}{\pgfqpoint{4.011666in}{2.334266in}}%
\pgfpathclose%
\pgfusepath{stroke,fill}%
\end{pgfscope}%
\begin{pgfscope}%
\pgfpathrectangle{\pgfqpoint{0.800000in}{0.528000in}}{\pgfqpoint{4.960000in}{3.696000in}}%
\pgfusepath{clip}%
\pgfsetbuttcap%
\pgfsetroundjoin%
\definecolor{currentfill}{rgb}{0.000000,0.000000,0.000000}%
\pgfsetfillcolor{currentfill}%
\pgfsetlinewidth{1.003750pt}%
\definecolor{currentstroke}{rgb}{0.000000,0.000000,0.000000}%
\pgfsetstrokecolor{currentstroke}%
\pgfsetdash{}{0pt}%
\pgfpathmoveto{\pgfqpoint{4.011666in}{2.334266in}}%
\pgfpathcurveto{\pgfqpoint{4.022716in}{2.334266in}}{\pgfqpoint{4.033315in}{2.338657in}}{\pgfqpoint{4.041128in}{2.346470in}}%
\pgfpathcurveto{\pgfqpoint{4.048942in}{2.354284in}}{\pgfqpoint{4.053332in}{2.364883in}}{\pgfqpoint{4.053332in}{2.375933in}}%
\pgfpathcurveto{\pgfqpoint{4.053332in}{2.386983in}}{\pgfqpoint{4.048942in}{2.397582in}}{\pgfqpoint{4.041128in}{2.405396in}}%
\pgfpathcurveto{\pgfqpoint{4.033315in}{2.413209in}}{\pgfqpoint{4.022716in}{2.417600in}}{\pgfqpoint{4.011666in}{2.417600in}}%
\pgfpathcurveto{\pgfqpoint{4.000616in}{2.417600in}}{\pgfqpoint{3.990016in}{2.413209in}}{\pgfqpoint{3.982203in}{2.405396in}}%
\pgfpathcurveto{\pgfqpoint{3.974389in}{2.397582in}}{\pgfqpoint{3.969999in}{2.386983in}}{\pgfqpoint{3.969999in}{2.375933in}}%
\pgfpathcurveto{\pgfqpoint{3.969999in}{2.364883in}}{\pgfqpoint{3.974389in}{2.354284in}}{\pgfqpoint{3.982203in}{2.346470in}}%
\pgfpathcurveto{\pgfqpoint{3.990016in}{2.338657in}}{\pgfqpoint{4.000616in}{2.334266in}}{\pgfqpoint{4.011666in}{2.334266in}}%
\pgfpathclose%
\pgfusepath{stroke,fill}%
\end{pgfscope}%
\begin{pgfscope}%
\pgfpathrectangle{\pgfqpoint{0.800000in}{0.528000in}}{\pgfqpoint{4.960000in}{3.696000in}}%
\pgfusepath{clip}%
\pgfsetbuttcap%
\pgfsetroundjoin%
\definecolor{currentfill}{rgb}{0.000000,0.000000,0.000000}%
\pgfsetfillcolor{currentfill}%
\pgfsetlinewidth{1.003750pt}%
\definecolor{currentstroke}{rgb}{0.000000,0.000000,0.000000}%
\pgfsetstrokecolor{currentstroke}%
\pgfsetdash{}{0pt}%
\pgfpathmoveto{\pgfqpoint{4.011666in}{2.334266in}}%
\pgfpathcurveto{\pgfqpoint{4.022716in}{2.334266in}}{\pgfqpoint{4.033315in}{2.338657in}}{\pgfqpoint{4.041128in}{2.346470in}}%
\pgfpathcurveto{\pgfqpoint{4.048942in}{2.354284in}}{\pgfqpoint{4.053332in}{2.364883in}}{\pgfqpoint{4.053332in}{2.375933in}}%
\pgfpathcurveto{\pgfqpoint{4.053332in}{2.386983in}}{\pgfqpoint{4.048942in}{2.397582in}}{\pgfqpoint{4.041128in}{2.405396in}}%
\pgfpathcurveto{\pgfqpoint{4.033315in}{2.413209in}}{\pgfqpoint{4.022716in}{2.417600in}}{\pgfqpoint{4.011666in}{2.417600in}}%
\pgfpathcurveto{\pgfqpoint{4.000616in}{2.417600in}}{\pgfqpoint{3.990016in}{2.413209in}}{\pgfqpoint{3.982203in}{2.405396in}}%
\pgfpathcurveto{\pgfqpoint{3.974389in}{2.397582in}}{\pgfqpoint{3.969999in}{2.386983in}}{\pgfqpoint{3.969999in}{2.375933in}}%
\pgfpathcurveto{\pgfqpoint{3.969999in}{2.364883in}}{\pgfqpoint{3.974389in}{2.354284in}}{\pgfqpoint{3.982203in}{2.346470in}}%
\pgfpathcurveto{\pgfqpoint{3.990016in}{2.338657in}}{\pgfqpoint{4.000616in}{2.334266in}}{\pgfqpoint{4.011666in}{2.334266in}}%
\pgfpathclose%
\pgfusepath{stroke,fill}%
\end{pgfscope}%
\begin{pgfscope}%
\pgfpathrectangle{\pgfqpoint{0.800000in}{0.528000in}}{\pgfqpoint{4.960000in}{3.696000in}}%
\pgfusepath{clip}%
\pgfsetbuttcap%
\pgfsetroundjoin%
\definecolor{currentfill}{rgb}{0.000000,0.000000,0.000000}%
\pgfsetfillcolor{currentfill}%
\pgfsetlinewidth{1.003750pt}%
\definecolor{currentstroke}{rgb}{0.000000,0.000000,0.000000}%
\pgfsetstrokecolor{currentstroke}%
\pgfsetdash{}{0pt}%
\pgfpathmoveto{\pgfqpoint{4.011666in}{2.334266in}}%
\pgfpathcurveto{\pgfqpoint{4.022716in}{2.334266in}}{\pgfqpoint{4.033315in}{2.338657in}}{\pgfqpoint{4.041128in}{2.346470in}}%
\pgfpathcurveto{\pgfqpoint{4.048942in}{2.354284in}}{\pgfqpoint{4.053332in}{2.364883in}}{\pgfqpoint{4.053332in}{2.375933in}}%
\pgfpathcurveto{\pgfqpoint{4.053332in}{2.386983in}}{\pgfqpoint{4.048942in}{2.397582in}}{\pgfqpoint{4.041128in}{2.405396in}}%
\pgfpathcurveto{\pgfqpoint{4.033315in}{2.413209in}}{\pgfqpoint{4.022716in}{2.417600in}}{\pgfqpoint{4.011666in}{2.417600in}}%
\pgfpathcurveto{\pgfqpoint{4.000616in}{2.417600in}}{\pgfqpoint{3.990016in}{2.413209in}}{\pgfqpoint{3.982203in}{2.405396in}}%
\pgfpathcurveto{\pgfqpoint{3.974389in}{2.397582in}}{\pgfqpoint{3.969999in}{2.386983in}}{\pgfqpoint{3.969999in}{2.375933in}}%
\pgfpathcurveto{\pgfqpoint{3.969999in}{2.364883in}}{\pgfqpoint{3.974389in}{2.354284in}}{\pgfqpoint{3.982203in}{2.346470in}}%
\pgfpathcurveto{\pgfqpoint{3.990016in}{2.338657in}}{\pgfqpoint{4.000616in}{2.334266in}}{\pgfqpoint{4.011666in}{2.334266in}}%
\pgfpathclose%
\pgfusepath{stroke,fill}%
\end{pgfscope}%
\begin{pgfscope}%
\pgfpathrectangle{\pgfqpoint{0.800000in}{0.528000in}}{\pgfqpoint{4.960000in}{3.696000in}}%
\pgfusepath{clip}%
\pgfsetbuttcap%
\pgfsetroundjoin%
\definecolor{currentfill}{rgb}{0.000000,0.000000,0.000000}%
\pgfsetfillcolor{currentfill}%
\pgfsetlinewidth{1.003750pt}%
\definecolor{currentstroke}{rgb}{0.000000,0.000000,0.000000}%
\pgfsetstrokecolor{currentstroke}%
\pgfsetdash{}{0pt}%
\pgfpathmoveto{\pgfqpoint{4.011666in}{2.334266in}}%
\pgfpathcurveto{\pgfqpoint{4.022716in}{2.334266in}}{\pgfqpoint{4.033315in}{2.338657in}}{\pgfqpoint{4.041128in}{2.346470in}}%
\pgfpathcurveto{\pgfqpoint{4.048942in}{2.354284in}}{\pgfqpoint{4.053332in}{2.364883in}}{\pgfqpoint{4.053332in}{2.375933in}}%
\pgfpathcurveto{\pgfqpoint{4.053332in}{2.386983in}}{\pgfqpoint{4.048942in}{2.397582in}}{\pgfqpoint{4.041128in}{2.405396in}}%
\pgfpathcurveto{\pgfqpoint{4.033315in}{2.413209in}}{\pgfqpoint{4.022716in}{2.417600in}}{\pgfqpoint{4.011666in}{2.417600in}}%
\pgfpathcurveto{\pgfqpoint{4.000616in}{2.417600in}}{\pgfqpoint{3.990016in}{2.413209in}}{\pgfqpoint{3.982203in}{2.405396in}}%
\pgfpathcurveto{\pgfqpoint{3.974389in}{2.397582in}}{\pgfqpoint{3.969999in}{2.386983in}}{\pgfqpoint{3.969999in}{2.375933in}}%
\pgfpathcurveto{\pgfqpoint{3.969999in}{2.364883in}}{\pgfqpoint{3.974389in}{2.354284in}}{\pgfqpoint{3.982203in}{2.346470in}}%
\pgfpathcurveto{\pgfqpoint{3.990016in}{2.338657in}}{\pgfqpoint{4.000616in}{2.334266in}}{\pgfqpoint{4.011666in}{2.334266in}}%
\pgfpathclose%
\pgfusepath{stroke,fill}%
\end{pgfscope}%
\begin{pgfscope}%
\pgfpathrectangle{\pgfqpoint{0.800000in}{0.528000in}}{\pgfqpoint{4.960000in}{3.696000in}}%
\pgfusepath{clip}%
\pgfsetbuttcap%
\pgfsetroundjoin%
\definecolor{currentfill}{rgb}{0.000000,0.000000,0.000000}%
\pgfsetfillcolor{currentfill}%
\pgfsetlinewidth{1.003750pt}%
\definecolor{currentstroke}{rgb}{0.000000,0.000000,0.000000}%
\pgfsetstrokecolor{currentstroke}%
\pgfsetdash{}{0pt}%
\pgfpathmoveto{\pgfqpoint{4.011666in}{2.334266in}}%
\pgfpathcurveto{\pgfqpoint{4.022716in}{2.334266in}}{\pgfqpoint{4.033315in}{2.338657in}}{\pgfqpoint{4.041128in}{2.346470in}}%
\pgfpathcurveto{\pgfqpoint{4.048942in}{2.354284in}}{\pgfqpoint{4.053332in}{2.364883in}}{\pgfqpoint{4.053332in}{2.375933in}}%
\pgfpathcurveto{\pgfqpoint{4.053332in}{2.386983in}}{\pgfqpoint{4.048942in}{2.397582in}}{\pgfqpoint{4.041128in}{2.405396in}}%
\pgfpathcurveto{\pgfqpoint{4.033315in}{2.413209in}}{\pgfqpoint{4.022716in}{2.417600in}}{\pgfqpoint{4.011666in}{2.417600in}}%
\pgfpathcurveto{\pgfqpoint{4.000616in}{2.417600in}}{\pgfqpoint{3.990016in}{2.413209in}}{\pgfqpoint{3.982203in}{2.405396in}}%
\pgfpathcurveto{\pgfqpoint{3.974389in}{2.397582in}}{\pgfqpoint{3.969999in}{2.386983in}}{\pgfqpoint{3.969999in}{2.375933in}}%
\pgfpathcurveto{\pgfqpoint{3.969999in}{2.364883in}}{\pgfqpoint{3.974389in}{2.354284in}}{\pgfqpoint{3.982203in}{2.346470in}}%
\pgfpathcurveto{\pgfqpoint{3.990016in}{2.338657in}}{\pgfqpoint{4.000616in}{2.334266in}}{\pgfqpoint{4.011666in}{2.334266in}}%
\pgfpathclose%
\pgfusepath{stroke,fill}%
\end{pgfscope}%
\begin{pgfscope}%
\pgfpathrectangle{\pgfqpoint{0.800000in}{0.528000in}}{\pgfqpoint{4.960000in}{3.696000in}}%
\pgfusepath{clip}%
\pgfsetbuttcap%
\pgfsetroundjoin%
\definecolor{currentfill}{rgb}{0.000000,0.000000,0.000000}%
\pgfsetfillcolor{currentfill}%
\pgfsetlinewidth{1.003750pt}%
\definecolor{currentstroke}{rgb}{0.000000,0.000000,0.000000}%
\pgfsetstrokecolor{currentstroke}%
\pgfsetdash{}{0pt}%
\pgfpathmoveto{\pgfqpoint{4.011666in}{2.334266in}}%
\pgfpathcurveto{\pgfqpoint{4.022716in}{2.334266in}}{\pgfqpoint{4.033315in}{2.338657in}}{\pgfqpoint{4.041128in}{2.346470in}}%
\pgfpathcurveto{\pgfqpoint{4.048942in}{2.354284in}}{\pgfqpoint{4.053332in}{2.364883in}}{\pgfqpoint{4.053332in}{2.375933in}}%
\pgfpathcurveto{\pgfqpoint{4.053332in}{2.386983in}}{\pgfqpoint{4.048942in}{2.397582in}}{\pgfqpoint{4.041128in}{2.405396in}}%
\pgfpathcurveto{\pgfqpoint{4.033315in}{2.413209in}}{\pgfqpoint{4.022716in}{2.417600in}}{\pgfqpoint{4.011666in}{2.417600in}}%
\pgfpathcurveto{\pgfqpoint{4.000616in}{2.417600in}}{\pgfqpoint{3.990016in}{2.413209in}}{\pgfqpoint{3.982203in}{2.405396in}}%
\pgfpathcurveto{\pgfqpoint{3.974389in}{2.397582in}}{\pgfqpoint{3.969999in}{2.386983in}}{\pgfqpoint{3.969999in}{2.375933in}}%
\pgfpathcurveto{\pgfqpoint{3.969999in}{2.364883in}}{\pgfqpoint{3.974389in}{2.354284in}}{\pgfqpoint{3.982203in}{2.346470in}}%
\pgfpathcurveto{\pgfqpoint{3.990016in}{2.338657in}}{\pgfqpoint{4.000616in}{2.334266in}}{\pgfqpoint{4.011666in}{2.334266in}}%
\pgfpathclose%
\pgfusepath{stroke,fill}%
\end{pgfscope}%
\begin{pgfscope}%
\pgfpathrectangle{\pgfqpoint{0.800000in}{0.528000in}}{\pgfqpoint{4.960000in}{3.696000in}}%
\pgfusepath{clip}%
\pgfsetbuttcap%
\pgfsetroundjoin%
\definecolor{currentfill}{rgb}{0.000000,0.000000,0.000000}%
\pgfsetfillcolor{currentfill}%
\pgfsetlinewidth{1.003750pt}%
\definecolor{currentstroke}{rgb}{0.000000,0.000000,0.000000}%
\pgfsetstrokecolor{currentstroke}%
\pgfsetdash{}{0pt}%
\pgfpathmoveto{\pgfqpoint{4.011666in}{3.984333in}}%
\pgfpathcurveto{\pgfqpoint{4.022716in}{3.984333in}}{\pgfqpoint{4.033315in}{3.988724in}}{\pgfqpoint{4.041128in}{3.996537in}}%
\pgfpathcurveto{\pgfqpoint{4.048942in}{4.004351in}}{\pgfqpoint{4.053332in}{4.014950in}}{\pgfqpoint{4.053332in}{4.026000in}}%
\pgfpathcurveto{\pgfqpoint{4.053332in}{4.037050in}}{\pgfqpoint{4.048942in}{4.047649in}}{\pgfqpoint{4.041128in}{4.055463in}}%
\pgfpathcurveto{\pgfqpoint{4.033315in}{4.063276in}}{\pgfqpoint{4.022716in}{4.067667in}}{\pgfqpoint{4.011666in}{4.067667in}}%
\pgfpathcurveto{\pgfqpoint{4.000616in}{4.067667in}}{\pgfqpoint{3.990016in}{4.063276in}}{\pgfqpoint{3.982203in}{4.055463in}}%
\pgfpathcurveto{\pgfqpoint{3.974389in}{4.047649in}}{\pgfqpoint{3.969999in}{4.037050in}}{\pgfqpoint{3.969999in}{4.026000in}}%
\pgfpathcurveto{\pgfqpoint{3.969999in}{4.014950in}}{\pgfqpoint{3.974389in}{4.004351in}}{\pgfqpoint{3.982203in}{3.996537in}}%
\pgfpathcurveto{\pgfqpoint{3.990016in}{3.988724in}}{\pgfqpoint{4.000616in}{3.984333in}}{\pgfqpoint{4.011666in}{3.984333in}}%
\pgfpathclose%
\pgfusepath{stroke,fill}%
\end{pgfscope}%
\begin{pgfscope}%
\pgfpathrectangle{\pgfqpoint{0.800000in}{0.528000in}}{\pgfqpoint{4.960000in}{3.696000in}}%
\pgfusepath{clip}%
\pgfsetbuttcap%
\pgfsetroundjoin%
\definecolor{currentfill}{rgb}{0.000000,0.000000,0.000000}%
\pgfsetfillcolor{currentfill}%
\pgfsetlinewidth{1.003750pt}%
\definecolor{currentstroke}{rgb}{0.000000,0.000000,0.000000}%
\pgfsetstrokecolor{currentstroke}%
\pgfsetdash{}{0pt}%
\pgfpathmoveto{\pgfqpoint{4.011666in}{2.334266in}}%
\pgfpathcurveto{\pgfqpoint{4.022716in}{2.334266in}}{\pgfqpoint{4.033315in}{2.338657in}}{\pgfqpoint{4.041128in}{2.346470in}}%
\pgfpathcurveto{\pgfqpoint{4.048942in}{2.354284in}}{\pgfqpoint{4.053332in}{2.364883in}}{\pgfqpoint{4.053332in}{2.375933in}}%
\pgfpathcurveto{\pgfqpoint{4.053332in}{2.386983in}}{\pgfqpoint{4.048942in}{2.397582in}}{\pgfqpoint{4.041128in}{2.405396in}}%
\pgfpathcurveto{\pgfqpoint{4.033315in}{2.413209in}}{\pgfqpoint{4.022716in}{2.417600in}}{\pgfqpoint{4.011666in}{2.417600in}}%
\pgfpathcurveto{\pgfqpoint{4.000616in}{2.417600in}}{\pgfqpoint{3.990016in}{2.413209in}}{\pgfqpoint{3.982203in}{2.405396in}}%
\pgfpathcurveto{\pgfqpoint{3.974389in}{2.397582in}}{\pgfqpoint{3.969999in}{2.386983in}}{\pgfqpoint{3.969999in}{2.375933in}}%
\pgfpathcurveto{\pgfqpoint{3.969999in}{2.364883in}}{\pgfqpoint{3.974389in}{2.354284in}}{\pgfqpoint{3.982203in}{2.346470in}}%
\pgfpathcurveto{\pgfqpoint{3.990016in}{2.338657in}}{\pgfqpoint{4.000616in}{2.334266in}}{\pgfqpoint{4.011666in}{2.334266in}}%
\pgfpathclose%
\pgfusepath{stroke,fill}%
\end{pgfscope}%
\begin{pgfscope}%
\pgfpathrectangle{\pgfqpoint{0.800000in}{0.528000in}}{\pgfqpoint{4.960000in}{3.696000in}}%
\pgfusepath{clip}%
\pgfsetbuttcap%
\pgfsetroundjoin%
\definecolor{currentfill}{rgb}{0.000000,0.000000,0.000000}%
\pgfsetfillcolor{currentfill}%
\pgfsetlinewidth{1.003750pt}%
\definecolor{currentstroke}{rgb}{0.000000,0.000000,0.000000}%
\pgfsetstrokecolor{currentstroke}%
\pgfsetdash{}{0pt}%
\pgfpathmoveto{\pgfqpoint{4.011666in}{2.334266in}}%
\pgfpathcurveto{\pgfqpoint{4.022716in}{2.334266in}}{\pgfqpoint{4.033315in}{2.338657in}}{\pgfqpoint{4.041128in}{2.346470in}}%
\pgfpathcurveto{\pgfqpoint{4.048942in}{2.354284in}}{\pgfqpoint{4.053332in}{2.364883in}}{\pgfqpoint{4.053332in}{2.375933in}}%
\pgfpathcurveto{\pgfqpoint{4.053332in}{2.386983in}}{\pgfqpoint{4.048942in}{2.397582in}}{\pgfqpoint{4.041128in}{2.405396in}}%
\pgfpathcurveto{\pgfqpoint{4.033315in}{2.413209in}}{\pgfqpoint{4.022716in}{2.417600in}}{\pgfqpoint{4.011666in}{2.417600in}}%
\pgfpathcurveto{\pgfqpoint{4.000616in}{2.417600in}}{\pgfqpoint{3.990016in}{2.413209in}}{\pgfqpoint{3.982203in}{2.405396in}}%
\pgfpathcurveto{\pgfqpoint{3.974389in}{2.397582in}}{\pgfqpoint{3.969999in}{2.386983in}}{\pgfqpoint{3.969999in}{2.375933in}}%
\pgfpathcurveto{\pgfqpoint{3.969999in}{2.364883in}}{\pgfqpoint{3.974389in}{2.354284in}}{\pgfqpoint{3.982203in}{2.346470in}}%
\pgfpathcurveto{\pgfqpoint{3.990016in}{2.338657in}}{\pgfqpoint{4.000616in}{2.334266in}}{\pgfqpoint{4.011666in}{2.334266in}}%
\pgfpathclose%
\pgfusepath{stroke,fill}%
\end{pgfscope}%
\begin{pgfscope}%
\pgfpathrectangle{\pgfqpoint{0.800000in}{0.528000in}}{\pgfqpoint{4.960000in}{3.696000in}}%
\pgfusepath{clip}%
\pgfsetbuttcap%
\pgfsetroundjoin%
\definecolor{currentfill}{rgb}{0.000000,0.000000,0.000000}%
\pgfsetfillcolor{currentfill}%
\pgfsetlinewidth{1.003750pt}%
\definecolor{currentstroke}{rgb}{0.000000,0.000000,0.000000}%
\pgfsetstrokecolor{currentstroke}%
\pgfsetdash{}{0pt}%
\pgfpathmoveto{\pgfqpoint{4.011666in}{2.334266in}}%
\pgfpathcurveto{\pgfqpoint{4.022716in}{2.334266in}}{\pgfqpoint{4.033315in}{2.338657in}}{\pgfqpoint{4.041128in}{2.346470in}}%
\pgfpathcurveto{\pgfqpoint{4.048942in}{2.354284in}}{\pgfqpoint{4.053332in}{2.364883in}}{\pgfqpoint{4.053332in}{2.375933in}}%
\pgfpathcurveto{\pgfqpoint{4.053332in}{2.386983in}}{\pgfqpoint{4.048942in}{2.397582in}}{\pgfqpoint{4.041128in}{2.405396in}}%
\pgfpathcurveto{\pgfqpoint{4.033315in}{2.413209in}}{\pgfqpoint{4.022716in}{2.417600in}}{\pgfqpoint{4.011666in}{2.417600in}}%
\pgfpathcurveto{\pgfqpoint{4.000616in}{2.417600in}}{\pgfqpoint{3.990016in}{2.413209in}}{\pgfqpoint{3.982203in}{2.405396in}}%
\pgfpathcurveto{\pgfqpoint{3.974389in}{2.397582in}}{\pgfqpoint{3.969999in}{2.386983in}}{\pgfqpoint{3.969999in}{2.375933in}}%
\pgfpathcurveto{\pgfqpoint{3.969999in}{2.364883in}}{\pgfqpoint{3.974389in}{2.354284in}}{\pgfqpoint{3.982203in}{2.346470in}}%
\pgfpathcurveto{\pgfqpoint{3.990016in}{2.338657in}}{\pgfqpoint{4.000616in}{2.334266in}}{\pgfqpoint{4.011666in}{2.334266in}}%
\pgfpathclose%
\pgfusepath{stroke,fill}%
\end{pgfscope}%
\begin{pgfscope}%
\pgfpathrectangle{\pgfqpoint{0.800000in}{0.528000in}}{\pgfqpoint{4.960000in}{3.696000in}}%
\pgfusepath{clip}%
\pgfsetbuttcap%
\pgfsetroundjoin%
\definecolor{currentfill}{rgb}{0.000000,0.000000,0.000000}%
\pgfsetfillcolor{currentfill}%
\pgfsetlinewidth{1.003750pt}%
\definecolor{currentstroke}{rgb}{0.000000,0.000000,0.000000}%
\pgfsetstrokecolor{currentstroke}%
\pgfsetdash{}{0pt}%
\pgfpathmoveto{\pgfqpoint{4.011666in}{3.984333in}}%
\pgfpathcurveto{\pgfqpoint{4.022716in}{3.984333in}}{\pgfqpoint{4.033315in}{3.988724in}}{\pgfqpoint{4.041128in}{3.996537in}}%
\pgfpathcurveto{\pgfqpoint{4.048942in}{4.004351in}}{\pgfqpoint{4.053332in}{4.014950in}}{\pgfqpoint{4.053332in}{4.026000in}}%
\pgfpathcurveto{\pgfqpoint{4.053332in}{4.037050in}}{\pgfqpoint{4.048942in}{4.047649in}}{\pgfqpoint{4.041128in}{4.055463in}}%
\pgfpathcurveto{\pgfqpoint{4.033315in}{4.063276in}}{\pgfqpoint{4.022716in}{4.067667in}}{\pgfqpoint{4.011666in}{4.067667in}}%
\pgfpathcurveto{\pgfqpoint{4.000616in}{4.067667in}}{\pgfqpoint{3.990016in}{4.063276in}}{\pgfqpoint{3.982203in}{4.055463in}}%
\pgfpathcurveto{\pgfqpoint{3.974389in}{4.047649in}}{\pgfqpoint{3.969999in}{4.037050in}}{\pgfqpoint{3.969999in}{4.026000in}}%
\pgfpathcurveto{\pgfqpoint{3.969999in}{4.014950in}}{\pgfqpoint{3.974389in}{4.004351in}}{\pgfqpoint{3.982203in}{3.996537in}}%
\pgfpathcurveto{\pgfqpoint{3.990016in}{3.988724in}}{\pgfqpoint{4.000616in}{3.984333in}}{\pgfqpoint{4.011666in}{3.984333in}}%
\pgfpathclose%
\pgfusepath{stroke,fill}%
\end{pgfscope}%
\begin{pgfscope}%
\pgfpathrectangle{\pgfqpoint{0.800000in}{0.528000in}}{\pgfqpoint{4.960000in}{3.696000in}}%
\pgfusepath{clip}%
\pgfsetbuttcap%
\pgfsetroundjoin%
\definecolor{currentfill}{rgb}{0.000000,0.000000,0.000000}%
\pgfsetfillcolor{currentfill}%
\pgfsetlinewidth{1.003750pt}%
\definecolor{currentstroke}{rgb}{0.000000,0.000000,0.000000}%
\pgfsetstrokecolor{currentstroke}%
\pgfsetdash{}{0pt}%
\pgfpathmoveto{\pgfqpoint{4.011666in}{2.334266in}}%
\pgfpathcurveto{\pgfqpoint{4.022716in}{2.334266in}}{\pgfqpoint{4.033315in}{2.338657in}}{\pgfqpoint{4.041128in}{2.346470in}}%
\pgfpathcurveto{\pgfqpoint{4.048942in}{2.354284in}}{\pgfqpoint{4.053332in}{2.364883in}}{\pgfqpoint{4.053332in}{2.375933in}}%
\pgfpathcurveto{\pgfqpoint{4.053332in}{2.386983in}}{\pgfqpoint{4.048942in}{2.397582in}}{\pgfqpoint{4.041128in}{2.405396in}}%
\pgfpathcurveto{\pgfqpoint{4.033315in}{2.413209in}}{\pgfqpoint{4.022716in}{2.417600in}}{\pgfqpoint{4.011666in}{2.417600in}}%
\pgfpathcurveto{\pgfqpoint{4.000616in}{2.417600in}}{\pgfqpoint{3.990016in}{2.413209in}}{\pgfqpoint{3.982203in}{2.405396in}}%
\pgfpathcurveto{\pgfqpoint{3.974389in}{2.397582in}}{\pgfqpoint{3.969999in}{2.386983in}}{\pgfqpoint{3.969999in}{2.375933in}}%
\pgfpathcurveto{\pgfqpoint{3.969999in}{2.364883in}}{\pgfqpoint{3.974389in}{2.354284in}}{\pgfqpoint{3.982203in}{2.346470in}}%
\pgfpathcurveto{\pgfqpoint{3.990016in}{2.338657in}}{\pgfqpoint{4.000616in}{2.334266in}}{\pgfqpoint{4.011666in}{2.334266in}}%
\pgfpathclose%
\pgfusepath{stroke,fill}%
\end{pgfscope}%
\begin{pgfscope}%
\pgfpathrectangle{\pgfqpoint{0.800000in}{0.528000in}}{\pgfqpoint{4.960000in}{3.696000in}}%
\pgfusepath{clip}%
\pgfsetbuttcap%
\pgfsetroundjoin%
\definecolor{currentfill}{rgb}{0.000000,0.000000,0.000000}%
\pgfsetfillcolor{currentfill}%
\pgfsetlinewidth{1.003750pt}%
\definecolor{currentstroke}{rgb}{0.000000,0.000000,0.000000}%
\pgfsetstrokecolor{currentstroke}%
\pgfsetdash{}{0pt}%
\pgfpathmoveto{\pgfqpoint{4.011666in}{2.334266in}}%
\pgfpathcurveto{\pgfqpoint{4.022716in}{2.334266in}}{\pgfqpoint{4.033315in}{2.338657in}}{\pgfqpoint{4.041128in}{2.346470in}}%
\pgfpathcurveto{\pgfqpoint{4.048942in}{2.354284in}}{\pgfqpoint{4.053332in}{2.364883in}}{\pgfqpoint{4.053332in}{2.375933in}}%
\pgfpathcurveto{\pgfqpoint{4.053332in}{2.386983in}}{\pgfqpoint{4.048942in}{2.397582in}}{\pgfqpoint{4.041128in}{2.405396in}}%
\pgfpathcurveto{\pgfqpoint{4.033315in}{2.413209in}}{\pgfqpoint{4.022716in}{2.417600in}}{\pgfqpoint{4.011666in}{2.417600in}}%
\pgfpathcurveto{\pgfqpoint{4.000616in}{2.417600in}}{\pgfqpoint{3.990016in}{2.413209in}}{\pgfqpoint{3.982203in}{2.405396in}}%
\pgfpathcurveto{\pgfqpoint{3.974389in}{2.397582in}}{\pgfqpoint{3.969999in}{2.386983in}}{\pgfqpoint{3.969999in}{2.375933in}}%
\pgfpathcurveto{\pgfqpoint{3.969999in}{2.364883in}}{\pgfqpoint{3.974389in}{2.354284in}}{\pgfqpoint{3.982203in}{2.346470in}}%
\pgfpathcurveto{\pgfqpoint{3.990016in}{2.338657in}}{\pgfqpoint{4.000616in}{2.334266in}}{\pgfqpoint{4.011666in}{2.334266in}}%
\pgfpathclose%
\pgfusepath{stroke,fill}%
\end{pgfscope}%
\begin{pgfscope}%
\pgfpathrectangle{\pgfqpoint{0.800000in}{0.528000in}}{\pgfqpoint{4.960000in}{3.696000in}}%
\pgfusepath{clip}%
\pgfsetbuttcap%
\pgfsetroundjoin%
\definecolor{currentfill}{rgb}{0.000000,0.000000,0.000000}%
\pgfsetfillcolor{currentfill}%
\pgfsetlinewidth{1.003750pt}%
\definecolor{currentstroke}{rgb}{0.000000,0.000000,0.000000}%
\pgfsetstrokecolor{currentstroke}%
\pgfsetdash{}{0pt}%
\pgfpathmoveto{\pgfqpoint{4.011666in}{2.334266in}}%
\pgfpathcurveto{\pgfqpoint{4.022716in}{2.334266in}}{\pgfqpoint{4.033315in}{2.338657in}}{\pgfqpoint{4.041128in}{2.346470in}}%
\pgfpathcurveto{\pgfqpoint{4.048942in}{2.354284in}}{\pgfqpoint{4.053332in}{2.364883in}}{\pgfqpoint{4.053332in}{2.375933in}}%
\pgfpathcurveto{\pgfqpoint{4.053332in}{2.386983in}}{\pgfqpoint{4.048942in}{2.397582in}}{\pgfqpoint{4.041128in}{2.405396in}}%
\pgfpathcurveto{\pgfqpoint{4.033315in}{2.413209in}}{\pgfqpoint{4.022716in}{2.417600in}}{\pgfqpoint{4.011666in}{2.417600in}}%
\pgfpathcurveto{\pgfqpoint{4.000616in}{2.417600in}}{\pgfqpoint{3.990016in}{2.413209in}}{\pgfqpoint{3.982203in}{2.405396in}}%
\pgfpathcurveto{\pgfqpoint{3.974389in}{2.397582in}}{\pgfqpoint{3.969999in}{2.386983in}}{\pgfqpoint{3.969999in}{2.375933in}}%
\pgfpathcurveto{\pgfqpoint{3.969999in}{2.364883in}}{\pgfqpoint{3.974389in}{2.354284in}}{\pgfqpoint{3.982203in}{2.346470in}}%
\pgfpathcurveto{\pgfqpoint{3.990016in}{2.338657in}}{\pgfqpoint{4.000616in}{2.334266in}}{\pgfqpoint{4.011666in}{2.334266in}}%
\pgfpathclose%
\pgfusepath{stroke,fill}%
\end{pgfscope}%
\begin{pgfscope}%
\pgfpathrectangle{\pgfqpoint{0.800000in}{0.528000in}}{\pgfqpoint{4.960000in}{3.696000in}}%
\pgfusepath{clip}%
\pgfsetbuttcap%
\pgfsetroundjoin%
\definecolor{currentfill}{rgb}{0.000000,0.000000,0.000000}%
\pgfsetfillcolor{currentfill}%
\pgfsetlinewidth{1.003750pt}%
\definecolor{currentstroke}{rgb}{0.000000,0.000000,0.000000}%
\pgfsetstrokecolor{currentstroke}%
\pgfsetdash{}{0pt}%
\pgfpathmoveto{\pgfqpoint{4.011666in}{2.334266in}}%
\pgfpathcurveto{\pgfqpoint{4.022716in}{2.334266in}}{\pgfqpoint{4.033315in}{2.338657in}}{\pgfqpoint{4.041128in}{2.346470in}}%
\pgfpathcurveto{\pgfqpoint{4.048942in}{2.354284in}}{\pgfqpoint{4.053332in}{2.364883in}}{\pgfqpoint{4.053332in}{2.375933in}}%
\pgfpathcurveto{\pgfqpoint{4.053332in}{2.386983in}}{\pgfqpoint{4.048942in}{2.397582in}}{\pgfqpoint{4.041128in}{2.405396in}}%
\pgfpathcurveto{\pgfqpoint{4.033315in}{2.413209in}}{\pgfqpoint{4.022716in}{2.417600in}}{\pgfqpoint{4.011666in}{2.417600in}}%
\pgfpathcurveto{\pgfqpoint{4.000616in}{2.417600in}}{\pgfqpoint{3.990016in}{2.413209in}}{\pgfqpoint{3.982203in}{2.405396in}}%
\pgfpathcurveto{\pgfqpoint{3.974389in}{2.397582in}}{\pgfqpoint{3.969999in}{2.386983in}}{\pgfqpoint{3.969999in}{2.375933in}}%
\pgfpathcurveto{\pgfqpoint{3.969999in}{2.364883in}}{\pgfqpoint{3.974389in}{2.354284in}}{\pgfqpoint{3.982203in}{2.346470in}}%
\pgfpathcurveto{\pgfqpoint{3.990016in}{2.338657in}}{\pgfqpoint{4.000616in}{2.334266in}}{\pgfqpoint{4.011666in}{2.334266in}}%
\pgfpathclose%
\pgfusepath{stroke,fill}%
\end{pgfscope}%
\begin{pgfscope}%
\pgfpathrectangle{\pgfqpoint{0.800000in}{0.528000in}}{\pgfqpoint{4.960000in}{3.696000in}}%
\pgfusepath{clip}%
\pgfsetbuttcap%
\pgfsetroundjoin%
\definecolor{currentfill}{rgb}{0.000000,0.000000,0.000000}%
\pgfsetfillcolor{currentfill}%
\pgfsetlinewidth{1.003750pt}%
\definecolor{currentstroke}{rgb}{0.000000,0.000000,0.000000}%
\pgfsetstrokecolor{currentstroke}%
\pgfsetdash{}{0pt}%
\pgfpathmoveto{\pgfqpoint{4.011666in}{2.334266in}}%
\pgfpathcurveto{\pgfqpoint{4.022716in}{2.334266in}}{\pgfqpoint{4.033315in}{2.338657in}}{\pgfqpoint{4.041128in}{2.346470in}}%
\pgfpathcurveto{\pgfqpoint{4.048942in}{2.354284in}}{\pgfqpoint{4.053332in}{2.364883in}}{\pgfqpoint{4.053332in}{2.375933in}}%
\pgfpathcurveto{\pgfqpoint{4.053332in}{2.386983in}}{\pgfqpoint{4.048942in}{2.397582in}}{\pgfqpoint{4.041128in}{2.405396in}}%
\pgfpathcurveto{\pgfqpoint{4.033315in}{2.413209in}}{\pgfqpoint{4.022716in}{2.417600in}}{\pgfqpoint{4.011666in}{2.417600in}}%
\pgfpathcurveto{\pgfqpoint{4.000616in}{2.417600in}}{\pgfqpoint{3.990016in}{2.413209in}}{\pgfqpoint{3.982203in}{2.405396in}}%
\pgfpathcurveto{\pgfqpoint{3.974389in}{2.397582in}}{\pgfqpoint{3.969999in}{2.386983in}}{\pgfqpoint{3.969999in}{2.375933in}}%
\pgfpathcurveto{\pgfqpoint{3.969999in}{2.364883in}}{\pgfqpoint{3.974389in}{2.354284in}}{\pgfqpoint{3.982203in}{2.346470in}}%
\pgfpathcurveto{\pgfqpoint{3.990016in}{2.338657in}}{\pgfqpoint{4.000616in}{2.334266in}}{\pgfqpoint{4.011666in}{2.334266in}}%
\pgfpathclose%
\pgfusepath{stroke,fill}%
\end{pgfscope}%
\begin{pgfscope}%
\pgfpathrectangle{\pgfqpoint{0.800000in}{0.528000in}}{\pgfqpoint{4.960000in}{3.696000in}}%
\pgfusepath{clip}%
\pgfsetbuttcap%
\pgfsetroundjoin%
\definecolor{currentfill}{rgb}{0.000000,0.000000,0.000000}%
\pgfsetfillcolor{currentfill}%
\pgfsetlinewidth{1.003750pt}%
\definecolor{currentstroke}{rgb}{0.000000,0.000000,0.000000}%
\pgfsetstrokecolor{currentstroke}%
\pgfsetdash{}{0pt}%
\pgfpathmoveto{\pgfqpoint{4.011666in}{3.984333in}}%
\pgfpathcurveto{\pgfqpoint{4.022716in}{3.984333in}}{\pgfqpoint{4.033315in}{3.988724in}}{\pgfqpoint{4.041128in}{3.996537in}}%
\pgfpathcurveto{\pgfqpoint{4.048942in}{4.004351in}}{\pgfqpoint{4.053332in}{4.014950in}}{\pgfqpoint{4.053332in}{4.026000in}}%
\pgfpathcurveto{\pgfqpoint{4.053332in}{4.037050in}}{\pgfqpoint{4.048942in}{4.047649in}}{\pgfqpoint{4.041128in}{4.055463in}}%
\pgfpathcurveto{\pgfqpoint{4.033315in}{4.063276in}}{\pgfqpoint{4.022716in}{4.067667in}}{\pgfqpoint{4.011666in}{4.067667in}}%
\pgfpathcurveto{\pgfqpoint{4.000616in}{4.067667in}}{\pgfqpoint{3.990016in}{4.063276in}}{\pgfqpoint{3.982203in}{4.055463in}}%
\pgfpathcurveto{\pgfqpoint{3.974389in}{4.047649in}}{\pgfqpoint{3.969999in}{4.037050in}}{\pgfqpoint{3.969999in}{4.026000in}}%
\pgfpathcurveto{\pgfqpoint{3.969999in}{4.014950in}}{\pgfqpoint{3.974389in}{4.004351in}}{\pgfqpoint{3.982203in}{3.996537in}}%
\pgfpathcurveto{\pgfqpoint{3.990016in}{3.988724in}}{\pgfqpoint{4.000616in}{3.984333in}}{\pgfqpoint{4.011666in}{3.984333in}}%
\pgfpathclose%
\pgfusepath{stroke,fill}%
\end{pgfscope}%
\begin{pgfscope}%
\pgfpathrectangle{\pgfqpoint{0.800000in}{0.528000in}}{\pgfqpoint{4.960000in}{3.696000in}}%
\pgfusepath{clip}%
\pgfsetbuttcap%
\pgfsetroundjoin%
\definecolor{currentfill}{rgb}{0.000000,0.000000,0.000000}%
\pgfsetfillcolor{currentfill}%
\pgfsetlinewidth{1.003750pt}%
\definecolor{currentstroke}{rgb}{0.000000,0.000000,0.000000}%
\pgfsetstrokecolor{currentstroke}%
\pgfsetdash{}{0pt}%
\pgfpathmoveto{\pgfqpoint{4.011666in}{2.334266in}}%
\pgfpathcurveto{\pgfqpoint{4.022716in}{2.334266in}}{\pgfqpoint{4.033315in}{2.338657in}}{\pgfqpoint{4.041128in}{2.346470in}}%
\pgfpathcurveto{\pgfqpoint{4.048942in}{2.354284in}}{\pgfqpoint{4.053332in}{2.364883in}}{\pgfqpoint{4.053332in}{2.375933in}}%
\pgfpathcurveto{\pgfqpoint{4.053332in}{2.386983in}}{\pgfqpoint{4.048942in}{2.397582in}}{\pgfqpoint{4.041128in}{2.405396in}}%
\pgfpathcurveto{\pgfqpoint{4.033315in}{2.413209in}}{\pgfqpoint{4.022716in}{2.417600in}}{\pgfqpoint{4.011666in}{2.417600in}}%
\pgfpathcurveto{\pgfqpoint{4.000616in}{2.417600in}}{\pgfqpoint{3.990016in}{2.413209in}}{\pgfqpoint{3.982203in}{2.405396in}}%
\pgfpathcurveto{\pgfqpoint{3.974389in}{2.397582in}}{\pgfqpoint{3.969999in}{2.386983in}}{\pgfqpoint{3.969999in}{2.375933in}}%
\pgfpathcurveto{\pgfqpoint{3.969999in}{2.364883in}}{\pgfqpoint{3.974389in}{2.354284in}}{\pgfqpoint{3.982203in}{2.346470in}}%
\pgfpathcurveto{\pgfqpoint{3.990016in}{2.338657in}}{\pgfqpoint{4.000616in}{2.334266in}}{\pgfqpoint{4.011666in}{2.334266in}}%
\pgfpathclose%
\pgfusepath{stroke,fill}%
\end{pgfscope}%
\begin{pgfscope}%
\pgfpathrectangle{\pgfqpoint{0.800000in}{0.528000in}}{\pgfqpoint{4.960000in}{3.696000in}}%
\pgfusepath{clip}%
\pgfsetbuttcap%
\pgfsetroundjoin%
\definecolor{currentfill}{rgb}{0.000000,0.000000,0.000000}%
\pgfsetfillcolor{currentfill}%
\pgfsetlinewidth{1.003750pt}%
\definecolor{currentstroke}{rgb}{0.000000,0.000000,0.000000}%
\pgfsetstrokecolor{currentstroke}%
\pgfsetdash{}{0pt}%
\pgfpathmoveto{\pgfqpoint{4.011666in}{2.334266in}}%
\pgfpathcurveto{\pgfqpoint{4.022716in}{2.334266in}}{\pgfqpoint{4.033315in}{2.338657in}}{\pgfqpoint{4.041128in}{2.346470in}}%
\pgfpathcurveto{\pgfqpoint{4.048942in}{2.354284in}}{\pgfqpoint{4.053332in}{2.364883in}}{\pgfqpoint{4.053332in}{2.375933in}}%
\pgfpathcurveto{\pgfqpoint{4.053332in}{2.386983in}}{\pgfqpoint{4.048942in}{2.397582in}}{\pgfqpoint{4.041128in}{2.405396in}}%
\pgfpathcurveto{\pgfqpoint{4.033315in}{2.413209in}}{\pgfqpoint{4.022716in}{2.417600in}}{\pgfqpoint{4.011666in}{2.417600in}}%
\pgfpathcurveto{\pgfqpoint{4.000616in}{2.417600in}}{\pgfqpoint{3.990016in}{2.413209in}}{\pgfqpoint{3.982203in}{2.405396in}}%
\pgfpathcurveto{\pgfqpoint{3.974389in}{2.397582in}}{\pgfqpoint{3.969999in}{2.386983in}}{\pgfqpoint{3.969999in}{2.375933in}}%
\pgfpathcurveto{\pgfqpoint{3.969999in}{2.364883in}}{\pgfqpoint{3.974389in}{2.354284in}}{\pgfqpoint{3.982203in}{2.346470in}}%
\pgfpathcurveto{\pgfqpoint{3.990016in}{2.338657in}}{\pgfqpoint{4.000616in}{2.334266in}}{\pgfqpoint{4.011666in}{2.334266in}}%
\pgfpathclose%
\pgfusepath{stroke,fill}%
\end{pgfscope}%
\begin{pgfscope}%
\pgfpathrectangle{\pgfqpoint{0.800000in}{0.528000in}}{\pgfqpoint{4.960000in}{3.696000in}}%
\pgfusepath{clip}%
\pgfsetbuttcap%
\pgfsetroundjoin%
\definecolor{currentfill}{rgb}{0.000000,0.000000,0.000000}%
\pgfsetfillcolor{currentfill}%
\pgfsetlinewidth{1.003750pt}%
\definecolor{currentstroke}{rgb}{0.000000,0.000000,0.000000}%
\pgfsetstrokecolor{currentstroke}%
\pgfsetdash{}{0pt}%
\pgfpathmoveto{\pgfqpoint{4.011666in}{2.334266in}}%
\pgfpathcurveto{\pgfqpoint{4.022716in}{2.334266in}}{\pgfqpoint{4.033315in}{2.338657in}}{\pgfqpoint{4.041128in}{2.346470in}}%
\pgfpathcurveto{\pgfqpoint{4.048942in}{2.354284in}}{\pgfqpoint{4.053332in}{2.364883in}}{\pgfqpoint{4.053332in}{2.375933in}}%
\pgfpathcurveto{\pgfqpoint{4.053332in}{2.386983in}}{\pgfqpoint{4.048942in}{2.397582in}}{\pgfqpoint{4.041128in}{2.405396in}}%
\pgfpathcurveto{\pgfqpoint{4.033315in}{2.413209in}}{\pgfqpoint{4.022716in}{2.417600in}}{\pgfqpoint{4.011666in}{2.417600in}}%
\pgfpathcurveto{\pgfqpoint{4.000616in}{2.417600in}}{\pgfqpoint{3.990016in}{2.413209in}}{\pgfqpoint{3.982203in}{2.405396in}}%
\pgfpathcurveto{\pgfqpoint{3.974389in}{2.397582in}}{\pgfqpoint{3.969999in}{2.386983in}}{\pgfqpoint{3.969999in}{2.375933in}}%
\pgfpathcurveto{\pgfqpoint{3.969999in}{2.364883in}}{\pgfqpoint{3.974389in}{2.354284in}}{\pgfqpoint{3.982203in}{2.346470in}}%
\pgfpathcurveto{\pgfqpoint{3.990016in}{2.338657in}}{\pgfqpoint{4.000616in}{2.334266in}}{\pgfqpoint{4.011666in}{2.334266in}}%
\pgfpathclose%
\pgfusepath{stroke,fill}%
\end{pgfscope}%
\begin{pgfscope}%
\pgfpathrectangle{\pgfqpoint{0.800000in}{0.528000in}}{\pgfqpoint{4.960000in}{3.696000in}}%
\pgfusepath{clip}%
\pgfsetbuttcap%
\pgfsetroundjoin%
\definecolor{currentfill}{rgb}{0.000000,0.000000,0.000000}%
\pgfsetfillcolor{currentfill}%
\pgfsetlinewidth{1.003750pt}%
\definecolor{currentstroke}{rgb}{0.000000,0.000000,0.000000}%
\pgfsetstrokecolor{currentstroke}%
\pgfsetdash{}{0pt}%
\pgfpathmoveto{\pgfqpoint{4.011666in}{2.334266in}}%
\pgfpathcurveto{\pgfqpoint{4.022716in}{2.334266in}}{\pgfqpoint{4.033315in}{2.338657in}}{\pgfqpoint{4.041128in}{2.346470in}}%
\pgfpathcurveto{\pgfqpoint{4.048942in}{2.354284in}}{\pgfqpoint{4.053332in}{2.364883in}}{\pgfqpoint{4.053332in}{2.375933in}}%
\pgfpathcurveto{\pgfqpoint{4.053332in}{2.386983in}}{\pgfqpoint{4.048942in}{2.397582in}}{\pgfqpoint{4.041128in}{2.405396in}}%
\pgfpathcurveto{\pgfqpoint{4.033315in}{2.413209in}}{\pgfqpoint{4.022716in}{2.417600in}}{\pgfqpoint{4.011666in}{2.417600in}}%
\pgfpathcurveto{\pgfqpoint{4.000616in}{2.417600in}}{\pgfqpoint{3.990016in}{2.413209in}}{\pgfqpoint{3.982203in}{2.405396in}}%
\pgfpathcurveto{\pgfqpoint{3.974389in}{2.397582in}}{\pgfqpoint{3.969999in}{2.386983in}}{\pgfqpoint{3.969999in}{2.375933in}}%
\pgfpathcurveto{\pgfqpoint{3.969999in}{2.364883in}}{\pgfqpoint{3.974389in}{2.354284in}}{\pgfqpoint{3.982203in}{2.346470in}}%
\pgfpathcurveto{\pgfqpoint{3.990016in}{2.338657in}}{\pgfqpoint{4.000616in}{2.334266in}}{\pgfqpoint{4.011666in}{2.334266in}}%
\pgfpathclose%
\pgfusepath{stroke,fill}%
\end{pgfscope}%
\begin{pgfscope}%
\pgfpathrectangle{\pgfqpoint{0.800000in}{0.528000in}}{\pgfqpoint{4.960000in}{3.696000in}}%
\pgfusepath{clip}%
\pgfsetbuttcap%
\pgfsetroundjoin%
\definecolor{currentfill}{rgb}{0.000000,0.000000,0.000000}%
\pgfsetfillcolor{currentfill}%
\pgfsetlinewidth{1.003750pt}%
\definecolor{currentstroke}{rgb}{0.000000,0.000000,0.000000}%
\pgfsetstrokecolor{currentstroke}%
\pgfsetdash{}{0pt}%
\pgfpathmoveto{\pgfqpoint{4.011666in}{3.984333in}}%
\pgfpathcurveto{\pgfqpoint{4.022716in}{3.984333in}}{\pgfqpoint{4.033315in}{3.988724in}}{\pgfqpoint{4.041128in}{3.996537in}}%
\pgfpathcurveto{\pgfqpoint{4.048942in}{4.004351in}}{\pgfqpoint{4.053332in}{4.014950in}}{\pgfqpoint{4.053332in}{4.026000in}}%
\pgfpathcurveto{\pgfqpoint{4.053332in}{4.037050in}}{\pgfqpoint{4.048942in}{4.047649in}}{\pgfqpoint{4.041128in}{4.055463in}}%
\pgfpathcurveto{\pgfqpoint{4.033315in}{4.063276in}}{\pgfqpoint{4.022716in}{4.067667in}}{\pgfqpoint{4.011666in}{4.067667in}}%
\pgfpathcurveto{\pgfqpoint{4.000616in}{4.067667in}}{\pgfqpoint{3.990016in}{4.063276in}}{\pgfqpoint{3.982203in}{4.055463in}}%
\pgfpathcurveto{\pgfqpoint{3.974389in}{4.047649in}}{\pgfqpoint{3.969999in}{4.037050in}}{\pgfqpoint{3.969999in}{4.026000in}}%
\pgfpathcurveto{\pgfqpoint{3.969999in}{4.014950in}}{\pgfqpoint{3.974389in}{4.004351in}}{\pgfqpoint{3.982203in}{3.996537in}}%
\pgfpathcurveto{\pgfqpoint{3.990016in}{3.988724in}}{\pgfqpoint{4.000616in}{3.984333in}}{\pgfqpoint{4.011666in}{3.984333in}}%
\pgfpathclose%
\pgfusepath{stroke,fill}%
\end{pgfscope}%
\begin{pgfscope}%
\pgfpathrectangle{\pgfqpoint{0.800000in}{0.528000in}}{\pgfqpoint{4.960000in}{3.696000in}}%
\pgfusepath{clip}%
\pgfsetbuttcap%
\pgfsetroundjoin%
\definecolor{currentfill}{rgb}{0.000000,0.000000,0.000000}%
\pgfsetfillcolor{currentfill}%
\pgfsetlinewidth{1.003750pt}%
\definecolor{currentstroke}{rgb}{0.000000,0.000000,0.000000}%
\pgfsetstrokecolor{currentstroke}%
\pgfsetdash{}{0pt}%
\pgfpathmoveto{\pgfqpoint{4.011666in}{2.334266in}}%
\pgfpathcurveto{\pgfqpoint{4.022716in}{2.334266in}}{\pgfqpoint{4.033315in}{2.338657in}}{\pgfqpoint{4.041128in}{2.346470in}}%
\pgfpathcurveto{\pgfqpoint{4.048942in}{2.354284in}}{\pgfqpoint{4.053332in}{2.364883in}}{\pgfqpoint{4.053332in}{2.375933in}}%
\pgfpathcurveto{\pgfqpoint{4.053332in}{2.386983in}}{\pgfqpoint{4.048942in}{2.397582in}}{\pgfqpoint{4.041128in}{2.405396in}}%
\pgfpathcurveto{\pgfqpoint{4.033315in}{2.413209in}}{\pgfqpoint{4.022716in}{2.417600in}}{\pgfqpoint{4.011666in}{2.417600in}}%
\pgfpathcurveto{\pgfqpoint{4.000616in}{2.417600in}}{\pgfqpoint{3.990016in}{2.413209in}}{\pgfqpoint{3.982203in}{2.405396in}}%
\pgfpathcurveto{\pgfqpoint{3.974389in}{2.397582in}}{\pgfqpoint{3.969999in}{2.386983in}}{\pgfqpoint{3.969999in}{2.375933in}}%
\pgfpathcurveto{\pgfqpoint{3.969999in}{2.364883in}}{\pgfqpoint{3.974389in}{2.354284in}}{\pgfqpoint{3.982203in}{2.346470in}}%
\pgfpathcurveto{\pgfqpoint{3.990016in}{2.338657in}}{\pgfqpoint{4.000616in}{2.334266in}}{\pgfqpoint{4.011666in}{2.334266in}}%
\pgfpathclose%
\pgfusepath{stroke,fill}%
\end{pgfscope}%
\begin{pgfscope}%
\pgfpathrectangle{\pgfqpoint{0.800000in}{0.528000in}}{\pgfqpoint{4.960000in}{3.696000in}}%
\pgfusepath{clip}%
\pgfsetbuttcap%
\pgfsetroundjoin%
\definecolor{currentfill}{rgb}{0.000000,0.000000,0.000000}%
\pgfsetfillcolor{currentfill}%
\pgfsetlinewidth{1.003750pt}%
\definecolor{currentstroke}{rgb}{0.000000,0.000000,0.000000}%
\pgfsetstrokecolor{currentstroke}%
\pgfsetdash{}{0pt}%
\pgfpathmoveto{\pgfqpoint{4.011666in}{2.334266in}}%
\pgfpathcurveto{\pgfqpoint{4.022716in}{2.334266in}}{\pgfqpoint{4.033315in}{2.338657in}}{\pgfqpoint{4.041128in}{2.346470in}}%
\pgfpathcurveto{\pgfqpoint{4.048942in}{2.354284in}}{\pgfqpoint{4.053332in}{2.364883in}}{\pgfqpoint{4.053332in}{2.375933in}}%
\pgfpathcurveto{\pgfqpoint{4.053332in}{2.386983in}}{\pgfqpoint{4.048942in}{2.397582in}}{\pgfqpoint{4.041128in}{2.405396in}}%
\pgfpathcurveto{\pgfqpoint{4.033315in}{2.413209in}}{\pgfqpoint{4.022716in}{2.417600in}}{\pgfqpoint{4.011666in}{2.417600in}}%
\pgfpathcurveto{\pgfqpoint{4.000616in}{2.417600in}}{\pgfqpoint{3.990016in}{2.413209in}}{\pgfqpoint{3.982203in}{2.405396in}}%
\pgfpathcurveto{\pgfqpoint{3.974389in}{2.397582in}}{\pgfqpoint{3.969999in}{2.386983in}}{\pgfqpoint{3.969999in}{2.375933in}}%
\pgfpathcurveto{\pgfqpoint{3.969999in}{2.364883in}}{\pgfqpoint{3.974389in}{2.354284in}}{\pgfqpoint{3.982203in}{2.346470in}}%
\pgfpathcurveto{\pgfqpoint{3.990016in}{2.338657in}}{\pgfqpoint{4.000616in}{2.334266in}}{\pgfqpoint{4.011666in}{2.334266in}}%
\pgfpathclose%
\pgfusepath{stroke,fill}%
\end{pgfscope}%
\begin{pgfscope}%
\pgfpathrectangle{\pgfqpoint{0.800000in}{0.528000in}}{\pgfqpoint{4.960000in}{3.696000in}}%
\pgfusepath{clip}%
\pgfsetbuttcap%
\pgfsetroundjoin%
\definecolor{currentfill}{rgb}{0.000000,0.000000,0.000000}%
\pgfsetfillcolor{currentfill}%
\pgfsetlinewidth{1.003750pt}%
\definecolor{currentstroke}{rgb}{0.000000,0.000000,0.000000}%
\pgfsetstrokecolor{currentstroke}%
\pgfsetdash{}{0pt}%
\pgfpathmoveto{\pgfqpoint{4.011666in}{2.334266in}}%
\pgfpathcurveto{\pgfqpoint{4.022716in}{2.334266in}}{\pgfqpoint{4.033315in}{2.338657in}}{\pgfqpoint{4.041128in}{2.346470in}}%
\pgfpathcurveto{\pgfqpoint{4.048942in}{2.354284in}}{\pgfqpoint{4.053332in}{2.364883in}}{\pgfqpoint{4.053332in}{2.375933in}}%
\pgfpathcurveto{\pgfqpoint{4.053332in}{2.386983in}}{\pgfqpoint{4.048942in}{2.397582in}}{\pgfqpoint{4.041128in}{2.405396in}}%
\pgfpathcurveto{\pgfqpoint{4.033315in}{2.413209in}}{\pgfqpoint{4.022716in}{2.417600in}}{\pgfqpoint{4.011666in}{2.417600in}}%
\pgfpathcurveto{\pgfqpoint{4.000616in}{2.417600in}}{\pgfqpoint{3.990016in}{2.413209in}}{\pgfqpoint{3.982203in}{2.405396in}}%
\pgfpathcurveto{\pgfqpoint{3.974389in}{2.397582in}}{\pgfqpoint{3.969999in}{2.386983in}}{\pgfqpoint{3.969999in}{2.375933in}}%
\pgfpathcurveto{\pgfqpoint{3.969999in}{2.364883in}}{\pgfqpoint{3.974389in}{2.354284in}}{\pgfqpoint{3.982203in}{2.346470in}}%
\pgfpathcurveto{\pgfqpoint{3.990016in}{2.338657in}}{\pgfqpoint{4.000616in}{2.334266in}}{\pgfqpoint{4.011666in}{2.334266in}}%
\pgfpathclose%
\pgfusepath{stroke,fill}%
\end{pgfscope}%
\begin{pgfscope}%
\pgfpathrectangle{\pgfqpoint{0.800000in}{0.528000in}}{\pgfqpoint{4.960000in}{3.696000in}}%
\pgfusepath{clip}%
\pgfsetbuttcap%
\pgfsetroundjoin%
\definecolor{currentfill}{rgb}{0.000000,0.000000,0.000000}%
\pgfsetfillcolor{currentfill}%
\pgfsetlinewidth{1.003750pt}%
\definecolor{currentstroke}{rgb}{0.000000,0.000000,0.000000}%
\pgfsetstrokecolor{currentstroke}%
\pgfsetdash{}{0pt}%
\pgfpathmoveto{\pgfqpoint{4.011666in}{2.334266in}}%
\pgfpathcurveto{\pgfqpoint{4.022716in}{2.334266in}}{\pgfqpoint{4.033315in}{2.338657in}}{\pgfqpoint{4.041128in}{2.346470in}}%
\pgfpathcurveto{\pgfqpoint{4.048942in}{2.354284in}}{\pgfqpoint{4.053332in}{2.364883in}}{\pgfqpoint{4.053332in}{2.375933in}}%
\pgfpathcurveto{\pgfqpoint{4.053332in}{2.386983in}}{\pgfqpoint{4.048942in}{2.397582in}}{\pgfqpoint{4.041128in}{2.405396in}}%
\pgfpathcurveto{\pgfqpoint{4.033315in}{2.413209in}}{\pgfqpoint{4.022716in}{2.417600in}}{\pgfqpoint{4.011666in}{2.417600in}}%
\pgfpathcurveto{\pgfqpoint{4.000616in}{2.417600in}}{\pgfqpoint{3.990016in}{2.413209in}}{\pgfqpoint{3.982203in}{2.405396in}}%
\pgfpathcurveto{\pgfqpoint{3.974389in}{2.397582in}}{\pgfqpoint{3.969999in}{2.386983in}}{\pgfqpoint{3.969999in}{2.375933in}}%
\pgfpathcurveto{\pgfqpoint{3.969999in}{2.364883in}}{\pgfqpoint{3.974389in}{2.354284in}}{\pgfqpoint{3.982203in}{2.346470in}}%
\pgfpathcurveto{\pgfqpoint{3.990016in}{2.338657in}}{\pgfqpoint{4.000616in}{2.334266in}}{\pgfqpoint{4.011666in}{2.334266in}}%
\pgfpathclose%
\pgfusepath{stroke,fill}%
\end{pgfscope}%
\begin{pgfscope}%
\pgfpathrectangle{\pgfqpoint{0.800000in}{0.528000in}}{\pgfqpoint{4.960000in}{3.696000in}}%
\pgfusepath{clip}%
\pgfsetbuttcap%
\pgfsetroundjoin%
\definecolor{currentfill}{rgb}{0.000000,0.000000,0.000000}%
\pgfsetfillcolor{currentfill}%
\pgfsetlinewidth{1.003750pt}%
\definecolor{currentstroke}{rgb}{0.000000,0.000000,0.000000}%
\pgfsetstrokecolor{currentstroke}%
\pgfsetdash{}{0pt}%
\pgfpathmoveto{\pgfqpoint{4.011666in}{2.334266in}}%
\pgfpathcurveto{\pgfqpoint{4.022716in}{2.334266in}}{\pgfqpoint{4.033315in}{2.338657in}}{\pgfqpoint{4.041128in}{2.346470in}}%
\pgfpathcurveto{\pgfqpoint{4.048942in}{2.354284in}}{\pgfqpoint{4.053332in}{2.364883in}}{\pgfqpoint{4.053332in}{2.375933in}}%
\pgfpathcurveto{\pgfqpoint{4.053332in}{2.386983in}}{\pgfqpoint{4.048942in}{2.397582in}}{\pgfqpoint{4.041128in}{2.405396in}}%
\pgfpathcurveto{\pgfqpoint{4.033315in}{2.413209in}}{\pgfqpoint{4.022716in}{2.417600in}}{\pgfqpoint{4.011666in}{2.417600in}}%
\pgfpathcurveto{\pgfqpoint{4.000616in}{2.417600in}}{\pgfqpoint{3.990016in}{2.413209in}}{\pgfqpoint{3.982203in}{2.405396in}}%
\pgfpathcurveto{\pgfqpoint{3.974389in}{2.397582in}}{\pgfqpoint{3.969999in}{2.386983in}}{\pgfqpoint{3.969999in}{2.375933in}}%
\pgfpathcurveto{\pgfqpoint{3.969999in}{2.364883in}}{\pgfqpoint{3.974389in}{2.354284in}}{\pgfqpoint{3.982203in}{2.346470in}}%
\pgfpathcurveto{\pgfqpoint{3.990016in}{2.338657in}}{\pgfqpoint{4.000616in}{2.334266in}}{\pgfqpoint{4.011666in}{2.334266in}}%
\pgfpathclose%
\pgfusepath{stroke,fill}%
\end{pgfscope}%
\begin{pgfscope}%
\pgfpathrectangle{\pgfqpoint{0.800000in}{0.528000in}}{\pgfqpoint{4.960000in}{3.696000in}}%
\pgfusepath{clip}%
\pgfsetbuttcap%
\pgfsetroundjoin%
\definecolor{currentfill}{rgb}{0.000000,0.000000,0.000000}%
\pgfsetfillcolor{currentfill}%
\pgfsetlinewidth{1.003750pt}%
\definecolor{currentstroke}{rgb}{0.000000,0.000000,0.000000}%
\pgfsetstrokecolor{currentstroke}%
\pgfsetdash{}{0pt}%
\pgfpathmoveto{\pgfqpoint{4.011666in}{2.334266in}}%
\pgfpathcurveto{\pgfqpoint{4.022716in}{2.334266in}}{\pgfqpoint{4.033315in}{2.338657in}}{\pgfqpoint{4.041128in}{2.346470in}}%
\pgfpathcurveto{\pgfqpoint{4.048942in}{2.354284in}}{\pgfqpoint{4.053332in}{2.364883in}}{\pgfqpoint{4.053332in}{2.375933in}}%
\pgfpathcurveto{\pgfqpoint{4.053332in}{2.386983in}}{\pgfqpoint{4.048942in}{2.397582in}}{\pgfqpoint{4.041128in}{2.405396in}}%
\pgfpathcurveto{\pgfqpoint{4.033315in}{2.413209in}}{\pgfqpoint{4.022716in}{2.417600in}}{\pgfqpoint{4.011666in}{2.417600in}}%
\pgfpathcurveto{\pgfqpoint{4.000616in}{2.417600in}}{\pgfqpoint{3.990016in}{2.413209in}}{\pgfqpoint{3.982203in}{2.405396in}}%
\pgfpathcurveto{\pgfqpoint{3.974389in}{2.397582in}}{\pgfqpoint{3.969999in}{2.386983in}}{\pgfqpoint{3.969999in}{2.375933in}}%
\pgfpathcurveto{\pgfqpoint{3.969999in}{2.364883in}}{\pgfqpoint{3.974389in}{2.354284in}}{\pgfqpoint{3.982203in}{2.346470in}}%
\pgfpathcurveto{\pgfqpoint{3.990016in}{2.338657in}}{\pgfqpoint{4.000616in}{2.334266in}}{\pgfqpoint{4.011666in}{2.334266in}}%
\pgfpathclose%
\pgfusepath{stroke,fill}%
\end{pgfscope}%
\begin{pgfscope}%
\pgfpathrectangle{\pgfqpoint{0.800000in}{0.528000in}}{\pgfqpoint{4.960000in}{3.696000in}}%
\pgfusepath{clip}%
\pgfsetbuttcap%
\pgfsetroundjoin%
\definecolor{currentfill}{rgb}{0.000000,0.000000,0.000000}%
\pgfsetfillcolor{currentfill}%
\pgfsetlinewidth{1.003750pt}%
\definecolor{currentstroke}{rgb}{0.000000,0.000000,0.000000}%
\pgfsetstrokecolor{currentstroke}%
\pgfsetdash{}{0pt}%
\pgfpathmoveto{\pgfqpoint{5.504545in}{2.334266in}}%
\pgfpathcurveto{\pgfqpoint{5.515596in}{2.334266in}}{\pgfqpoint{5.526195in}{2.338657in}}{\pgfqpoint{5.534008in}{2.346470in}}%
\pgfpathcurveto{\pgfqpoint{5.541822in}{2.354284in}}{\pgfqpoint{5.546212in}{2.364883in}}{\pgfqpoint{5.546212in}{2.375933in}}%
\pgfpathcurveto{\pgfqpoint{5.546212in}{2.386983in}}{\pgfqpoint{5.541822in}{2.397582in}}{\pgfqpoint{5.534008in}{2.405396in}}%
\pgfpathcurveto{\pgfqpoint{5.526195in}{2.413209in}}{\pgfqpoint{5.515596in}{2.417600in}}{\pgfqpoint{5.504545in}{2.417600in}}%
\pgfpathcurveto{\pgfqpoint{5.493495in}{2.417600in}}{\pgfqpoint{5.482896in}{2.413209in}}{\pgfqpoint{5.475083in}{2.405396in}}%
\pgfpathcurveto{\pgfqpoint{5.467269in}{2.397582in}}{\pgfqpoint{5.462879in}{2.386983in}}{\pgfqpoint{5.462879in}{2.375933in}}%
\pgfpathcurveto{\pgfqpoint{5.462879in}{2.364883in}}{\pgfqpoint{5.467269in}{2.354284in}}{\pgfqpoint{5.475083in}{2.346470in}}%
\pgfpathcurveto{\pgfqpoint{5.482896in}{2.338657in}}{\pgfqpoint{5.493495in}{2.334266in}}{\pgfqpoint{5.504545in}{2.334266in}}%
\pgfpathclose%
\pgfusepath{stroke,fill}%
\end{pgfscope}%
\begin{pgfscope}%
\pgfpathrectangle{\pgfqpoint{0.800000in}{0.528000in}}{\pgfqpoint{4.960000in}{3.696000in}}%
\pgfusepath{clip}%
\pgfsetbuttcap%
\pgfsetroundjoin%
\definecolor{currentfill}{rgb}{0.000000,0.000000,0.000000}%
\pgfsetfillcolor{currentfill}%
\pgfsetlinewidth{1.003750pt}%
\definecolor{currentstroke}{rgb}{0.000000,0.000000,0.000000}%
\pgfsetstrokecolor{currentstroke}%
\pgfsetdash{}{0pt}%
\pgfpathmoveto{\pgfqpoint{5.504545in}{3.984333in}}%
\pgfpathcurveto{\pgfqpoint{5.515596in}{3.984333in}}{\pgfqpoint{5.526195in}{3.988724in}}{\pgfqpoint{5.534008in}{3.996537in}}%
\pgfpathcurveto{\pgfqpoint{5.541822in}{4.004351in}}{\pgfqpoint{5.546212in}{4.014950in}}{\pgfqpoint{5.546212in}{4.026000in}}%
\pgfpathcurveto{\pgfqpoint{5.546212in}{4.037050in}}{\pgfqpoint{5.541822in}{4.047649in}}{\pgfqpoint{5.534008in}{4.055463in}}%
\pgfpathcurveto{\pgfqpoint{5.526195in}{4.063276in}}{\pgfqpoint{5.515596in}{4.067667in}}{\pgfqpoint{5.504545in}{4.067667in}}%
\pgfpathcurveto{\pgfqpoint{5.493495in}{4.067667in}}{\pgfqpoint{5.482896in}{4.063276in}}{\pgfqpoint{5.475083in}{4.055463in}}%
\pgfpathcurveto{\pgfqpoint{5.467269in}{4.047649in}}{\pgfqpoint{5.462879in}{4.037050in}}{\pgfqpoint{5.462879in}{4.026000in}}%
\pgfpathcurveto{\pgfqpoint{5.462879in}{4.014950in}}{\pgfqpoint{5.467269in}{4.004351in}}{\pgfqpoint{5.475083in}{3.996537in}}%
\pgfpathcurveto{\pgfqpoint{5.482896in}{3.988724in}}{\pgfqpoint{5.493495in}{3.984333in}}{\pgfqpoint{5.504545in}{3.984333in}}%
\pgfpathclose%
\pgfusepath{stroke,fill}%
\end{pgfscope}%
\begin{pgfscope}%
\pgfpathrectangle{\pgfqpoint{0.800000in}{0.528000in}}{\pgfqpoint{4.960000in}{3.696000in}}%
\pgfusepath{clip}%
\pgfsetbuttcap%
\pgfsetroundjoin%
\definecolor{currentfill}{rgb}{0.000000,0.000000,0.000000}%
\pgfsetfillcolor{currentfill}%
\pgfsetlinewidth{1.003750pt}%
\definecolor{currentstroke}{rgb}{0.000000,0.000000,0.000000}%
\pgfsetstrokecolor{currentstroke}%
\pgfsetdash{}{0pt}%
\pgfpathmoveto{\pgfqpoint{5.504545in}{2.334266in}}%
\pgfpathcurveto{\pgfqpoint{5.515596in}{2.334266in}}{\pgfqpoint{5.526195in}{2.338657in}}{\pgfqpoint{5.534008in}{2.346470in}}%
\pgfpathcurveto{\pgfqpoint{5.541822in}{2.354284in}}{\pgfqpoint{5.546212in}{2.364883in}}{\pgfqpoint{5.546212in}{2.375933in}}%
\pgfpathcurveto{\pgfqpoint{5.546212in}{2.386983in}}{\pgfqpoint{5.541822in}{2.397582in}}{\pgfqpoint{5.534008in}{2.405396in}}%
\pgfpathcurveto{\pgfqpoint{5.526195in}{2.413209in}}{\pgfqpoint{5.515596in}{2.417600in}}{\pgfqpoint{5.504545in}{2.417600in}}%
\pgfpathcurveto{\pgfqpoint{5.493495in}{2.417600in}}{\pgfqpoint{5.482896in}{2.413209in}}{\pgfqpoint{5.475083in}{2.405396in}}%
\pgfpathcurveto{\pgfqpoint{5.467269in}{2.397582in}}{\pgfqpoint{5.462879in}{2.386983in}}{\pgfqpoint{5.462879in}{2.375933in}}%
\pgfpathcurveto{\pgfqpoint{5.462879in}{2.364883in}}{\pgfqpoint{5.467269in}{2.354284in}}{\pgfqpoint{5.475083in}{2.346470in}}%
\pgfpathcurveto{\pgfqpoint{5.482896in}{2.338657in}}{\pgfqpoint{5.493495in}{2.334266in}}{\pgfqpoint{5.504545in}{2.334266in}}%
\pgfpathclose%
\pgfusepath{stroke,fill}%
\end{pgfscope}%
\begin{pgfscope}%
\pgfpathrectangle{\pgfqpoint{0.800000in}{0.528000in}}{\pgfqpoint{4.960000in}{3.696000in}}%
\pgfusepath{clip}%
\pgfsetbuttcap%
\pgfsetroundjoin%
\definecolor{currentfill}{rgb}{0.000000,0.000000,0.000000}%
\pgfsetfillcolor{currentfill}%
\pgfsetlinewidth{1.003750pt}%
\definecolor{currentstroke}{rgb}{0.000000,0.000000,0.000000}%
\pgfsetstrokecolor{currentstroke}%
\pgfsetdash{}{0pt}%
\pgfpathmoveto{\pgfqpoint{5.504545in}{3.984333in}}%
\pgfpathcurveto{\pgfqpoint{5.515596in}{3.984333in}}{\pgfqpoint{5.526195in}{3.988724in}}{\pgfqpoint{5.534008in}{3.996537in}}%
\pgfpathcurveto{\pgfqpoint{5.541822in}{4.004351in}}{\pgfqpoint{5.546212in}{4.014950in}}{\pgfqpoint{5.546212in}{4.026000in}}%
\pgfpathcurveto{\pgfqpoint{5.546212in}{4.037050in}}{\pgfqpoint{5.541822in}{4.047649in}}{\pgfqpoint{5.534008in}{4.055463in}}%
\pgfpathcurveto{\pgfqpoint{5.526195in}{4.063276in}}{\pgfqpoint{5.515596in}{4.067667in}}{\pgfqpoint{5.504545in}{4.067667in}}%
\pgfpathcurveto{\pgfqpoint{5.493495in}{4.067667in}}{\pgfqpoint{5.482896in}{4.063276in}}{\pgfqpoint{5.475083in}{4.055463in}}%
\pgfpathcurveto{\pgfqpoint{5.467269in}{4.047649in}}{\pgfqpoint{5.462879in}{4.037050in}}{\pgfqpoint{5.462879in}{4.026000in}}%
\pgfpathcurveto{\pgfqpoint{5.462879in}{4.014950in}}{\pgfqpoint{5.467269in}{4.004351in}}{\pgfqpoint{5.475083in}{3.996537in}}%
\pgfpathcurveto{\pgfqpoint{5.482896in}{3.988724in}}{\pgfqpoint{5.493495in}{3.984333in}}{\pgfqpoint{5.504545in}{3.984333in}}%
\pgfpathclose%
\pgfusepath{stroke,fill}%
\end{pgfscope}%
\begin{pgfscope}%
\pgfpathrectangle{\pgfqpoint{0.800000in}{0.528000in}}{\pgfqpoint{4.960000in}{3.696000in}}%
\pgfusepath{clip}%
\pgfsetbuttcap%
\pgfsetroundjoin%
\definecolor{currentfill}{rgb}{0.000000,0.000000,0.000000}%
\pgfsetfillcolor{currentfill}%
\pgfsetlinewidth{1.003750pt}%
\definecolor{currentstroke}{rgb}{0.000000,0.000000,0.000000}%
\pgfsetstrokecolor{currentstroke}%
\pgfsetdash{}{0pt}%
\pgfpathmoveto{\pgfqpoint{5.504545in}{2.334266in}}%
\pgfpathcurveto{\pgfqpoint{5.515596in}{2.334266in}}{\pgfqpoint{5.526195in}{2.338657in}}{\pgfqpoint{5.534008in}{2.346470in}}%
\pgfpathcurveto{\pgfqpoint{5.541822in}{2.354284in}}{\pgfqpoint{5.546212in}{2.364883in}}{\pgfqpoint{5.546212in}{2.375933in}}%
\pgfpathcurveto{\pgfqpoint{5.546212in}{2.386983in}}{\pgfqpoint{5.541822in}{2.397582in}}{\pgfqpoint{5.534008in}{2.405396in}}%
\pgfpathcurveto{\pgfqpoint{5.526195in}{2.413209in}}{\pgfqpoint{5.515596in}{2.417600in}}{\pgfqpoint{5.504545in}{2.417600in}}%
\pgfpathcurveto{\pgfqpoint{5.493495in}{2.417600in}}{\pgfqpoint{5.482896in}{2.413209in}}{\pgfqpoint{5.475083in}{2.405396in}}%
\pgfpathcurveto{\pgfqpoint{5.467269in}{2.397582in}}{\pgfqpoint{5.462879in}{2.386983in}}{\pgfqpoint{5.462879in}{2.375933in}}%
\pgfpathcurveto{\pgfqpoint{5.462879in}{2.364883in}}{\pgfqpoint{5.467269in}{2.354284in}}{\pgfqpoint{5.475083in}{2.346470in}}%
\pgfpathcurveto{\pgfqpoint{5.482896in}{2.338657in}}{\pgfqpoint{5.493495in}{2.334266in}}{\pgfqpoint{5.504545in}{2.334266in}}%
\pgfpathclose%
\pgfusepath{stroke,fill}%
\end{pgfscope}%
\begin{pgfscope}%
\pgfpathrectangle{\pgfqpoint{0.800000in}{0.528000in}}{\pgfqpoint{4.960000in}{3.696000in}}%
\pgfusepath{clip}%
\pgfsetbuttcap%
\pgfsetroundjoin%
\definecolor{currentfill}{rgb}{0.000000,0.000000,0.000000}%
\pgfsetfillcolor{currentfill}%
\pgfsetlinewidth{1.003750pt}%
\definecolor{currentstroke}{rgb}{0.000000,0.000000,0.000000}%
\pgfsetstrokecolor{currentstroke}%
\pgfsetdash{}{0pt}%
\pgfpathmoveto{\pgfqpoint{5.504545in}{2.334266in}}%
\pgfpathcurveto{\pgfqpoint{5.515596in}{2.334266in}}{\pgfqpoint{5.526195in}{2.338657in}}{\pgfqpoint{5.534008in}{2.346470in}}%
\pgfpathcurveto{\pgfqpoint{5.541822in}{2.354284in}}{\pgfqpoint{5.546212in}{2.364883in}}{\pgfqpoint{5.546212in}{2.375933in}}%
\pgfpathcurveto{\pgfqpoint{5.546212in}{2.386983in}}{\pgfqpoint{5.541822in}{2.397582in}}{\pgfqpoint{5.534008in}{2.405396in}}%
\pgfpathcurveto{\pgfqpoint{5.526195in}{2.413209in}}{\pgfqpoint{5.515596in}{2.417600in}}{\pgfqpoint{5.504545in}{2.417600in}}%
\pgfpathcurveto{\pgfqpoint{5.493495in}{2.417600in}}{\pgfqpoint{5.482896in}{2.413209in}}{\pgfqpoint{5.475083in}{2.405396in}}%
\pgfpathcurveto{\pgfqpoint{5.467269in}{2.397582in}}{\pgfqpoint{5.462879in}{2.386983in}}{\pgfqpoint{5.462879in}{2.375933in}}%
\pgfpathcurveto{\pgfqpoint{5.462879in}{2.364883in}}{\pgfqpoint{5.467269in}{2.354284in}}{\pgfqpoint{5.475083in}{2.346470in}}%
\pgfpathcurveto{\pgfqpoint{5.482896in}{2.338657in}}{\pgfqpoint{5.493495in}{2.334266in}}{\pgfqpoint{5.504545in}{2.334266in}}%
\pgfpathclose%
\pgfusepath{stroke,fill}%
\end{pgfscope}%
\begin{pgfscope}%
\pgfpathrectangle{\pgfqpoint{0.800000in}{0.528000in}}{\pgfqpoint{4.960000in}{3.696000in}}%
\pgfusepath{clip}%
\pgfsetbuttcap%
\pgfsetroundjoin%
\definecolor{currentfill}{rgb}{0.000000,0.000000,0.000000}%
\pgfsetfillcolor{currentfill}%
\pgfsetlinewidth{1.003750pt}%
\definecolor{currentstroke}{rgb}{0.000000,0.000000,0.000000}%
\pgfsetstrokecolor{currentstroke}%
\pgfsetdash{}{0pt}%
\pgfpathmoveto{\pgfqpoint{5.504545in}{3.984333in}}%
\pgfpathcurveto{\pgfqpoint{5.515596in}{3.984333in}}{\pgfqpoint{5.526195in}{3.988724in}}{\pgfqpoint{5.534008in}{3.996537in}}%
\pgfpathcurveto{\pgfqpoint{5.541822in}{4.004351in}}{\pgfqpoint{5.546212in}{4.014950in}}{\pgfqpoint{5.546212in}{4.026000in}}%
\pgfpathcurveto{\pgfqpoint{5.546212in}{4.037050in}}{\pgfqpoint{5.541822in}{4.047649in}}{\pgfqpoint{5.534008in}{4.055463in}}%
\pgfpathcurveto{\pgfqpoint{5.526195in}{4.063276in}}{\pgfqpoint{5.515596in}{4.067667in}}{\pgfqpoint{5.504545in}{4.067667in}}%
\pgfpathcurveto{\pgfqpoint{5.493495in}{4.067667in}}{\pgfqpoint{5.482896in}{4.063276in}}{\pgfqpoint{5.475083in}{4.055463in}}%
\pgfpathcurveto{\pgfqpoint{5.467269in}{4.047649in}}{\pgfqpoint{5.462879in}{4.037050in}}{\pgfqpoint{5.462879in}{4.026000in}}%
\pgfpathcurveto{\pgfqpoint{5.462879in}{4.014950in}}{\pgfqpoint{5.467269in}{4.004351in}}{\pgfqpoint{5.475083in}{3.996537in}}%
\pgfpathcurveto{\pgfqpoint{5.482896in}{3.988724in}}{\pgfqpoint{5.493495in}{3.984333in}}{\pgfqpoint{5.504545in}{3.984333in}}%
\pgfpathclose%
\pgfusepath{stroke,fill}%
\end{pgfscope}%
\begin{pgfscope}%
\pgfpathrectangle{\pgfqpoint{0.800000in}{0.528000in}}{\pgfqpoint{4.960000in}{3.696000in}}%
\pgfusepath{clip}%
\pgfsetbuttcap%
\pgfsetroundjoin%
\definecolor{currentfill}{rgb}{0.000000,0.000000,0.000000}%
\pgfsetfillcolor{currentfill}%
\pgfsetlinewidth{1.003750pt}%
\definecolor{currentstroke}{rgb}{0.000000,0.000000,0.000000}%
\pgfsetstrokecolor{currentstroke}%
\pgfsetdash{}{0pt}%
\pgfpathmoveto{\pgfqpoint{5.504545in}{3.984333in}}%
\pgfpathcurveto{\pgfqpoint{5.515596in}{3.984333in}}{\pgfqpoint{5.526195in}{3.988724in}}{\pgfqpoint{5.534008in}{3.996537in}}%
\pgfpathcurveto{\pgfqpoint{5.541822in}{4.004351in}}{\pgfqpoint{5.546212in}{4.014950in}}{\pgfqpoint{5.546212in}{4.026000in}}%
\pgfpathcurveto{\pgfqpoint{5.546212in}{4.037050in}}{\pgfqpoint{5.541822in}{4.047649in}}{\pgfqpoint{5.534008in}{4.055463in}}%
\pgfpathcurveto{\pgfqpoint{5.526195in}{4.063276in}}{\pgfqpoint{5.515596in}{4.067667in}}{\pgfqpoint{5.504545in}{4.067667in}}%
\pgfpathcurveto{\pgfqpoint{5.493495in}{4.067667in}}{\pgfqpoint{5.482896in}{4.063276in}}{\pgfqpoint{5.475083in}{4.055463in}}%
\pgfpathcurveto{\pgfqpoint{5.467269in}{4.047649in}}{\pgfqpoint{5.462879in}{4.037050in}}{\pgfqpoint{5.462879in}{4.026000in}}%
\pgfpathcurveto{\pgfqpoint{5.462879in}{4.014950in}}{\pgfqpoint{5.467269in}{4.004351in}}{\pgfqpoint{5.475083in}{3.996537in}}%
\pgfpathcurveto{\pgfqpoint{5.482896in}{3.988724in}}{\pgfqpoint{5.493495in}{3.984333in}}{\pgfqpoint{5.504545in}{3.984333in}}%
\pgfpathclose%
\pgfusepath{stroke,fill}%
\end{pgfscope}%
\begin{pgfscope}%
\pgfpathrectangle{\pgfqpoint{0.800000in}{0.528000in}}{\pgfqpoint{4.960000in}{3.696000in}}%
\pgfusepath{clip}%
\pgfsetbuttcap%
\pgfsetroundjoin%
\definecolor{currentfill}{rgb}{0.000000,0.000000,0.000000}%
\pgfsetfillcolor{currentfill}%
\pgfsetlinewidth{1.003750pt}%
\definecolor{currentstroke}{rgb}{0.000000,0.000000,0.000000}%
\pgfsetstrokecolor{currentstroke}%
\pgfsetdash{}{0pt}%
\pgfpathmoveto{\pgfqpoint{5.504545in}{2.334266in}}%
\pgfpathcurveto{\pgfqpoint{5.515596in}{2.334266in}}{\pgfqpoint{5.526195in}{2.338657in}}{\pgfqpoint{5.534008in}{2.346470in}}%
\pgfpathcurveto{\pgfqpoint{5.541822in}{2.354284in}}{\pgfqpoint{5.546212in}{2.364883in}}{\pgfqpoint{5.546212in}{2.375933in}}%
\pgfpathcurveto{\pgfqpoint{5.546212in}{2.386983in}}{\pgfqpoint{5.541822in}{2.397582in}}{\pgfqpoint{5.534008in}{2.405396in}}%
\pgfpathcurveto{\pgfqpoint{5.526195in}{2.413209in}}{\pgfqpoint{5.515596in}{2.417600in}}{\pgfqpoint{5.504545in}{2.417600in}}%
\pgfpathcurveto{\pgfqpoint{5.493495in}{2.417600in}}{\pgfqpoint{5.482896in}{2.413209in}}{\pgfqpoint{5.475083in}{2.405396in}}%
\pgfpathcurveto{\pgfqpoint{5.467269in}{2.397582in}}{\pgfqpoint{5.462879in}{2.386983in}}{\pgfqpoint{5.462879in}{2.375933in}}%
\pgfpathcurveto{\pgfqpoint{5.462879in}{2.364883in}}{\pgfqpoint{5.467269in}{2.354284in}}{\pgfqpoint{5.475083in}{2.346470in}}%
\pgfpathcurveto{\pgfqpoint{5.482896in}{2.338657in}}{\pgfqpoint{5.493495in}{2.334266in}}{\pgfqpoint{5.504545in}{2.334266in}}%
\pgfpathclose%
\pgfusepath{stroke,fill}%
\end{pgfscope}%
\begin{pgfscope}%
\pgfpathrectangle{\pgfqpoint{0.800000in}{0.528000in}}{\pgfqpoint{4.960000in}{3.696000in}}%
\pgfusepath{clip}%
\pgfsetbuttcap%
\pgfsetroundjoin%
\definecolor{currentfill}{rgb}{0.000000,0.000000,0.000000}%
\pgfsetfillcolor{currentfill}%
\pgfsetlinewidth{1.003750pt}%
\definecolor{currentstroke}{rgb}{0.000000,0.000000,0.000000}%
\pgfsetstrokecolor{currentstroke}%
\pgfsetdash{}{0pt}%
\pgfpathmoveto{\pgfqpoint{5.504545in}{2.334266in}}%
\pgfpathcurveto{\pgfqpoint{5.515596in}{2.334266in}}{\pgfqpoint{5.526195in}{2.338657in}}{\pgfqpoint{5.534008in}{2.346470in}}%
\pgfpathcurveto{\pgfqpoint{5.541822in}{2.354284in}}{\pgfqpoint{5.546212in}{2.364883in}}{\pgfqpoint{5.546212in}{2.375933in}}%
\pgfpathcurveto{\pgfqpoint{5.546212in}{2.386983in}}{\pgfqpoint{5.541822in}{2.397582in}}{\pgfqpoint{5.534008in}{2.405396in}}%
\pgfpathcurveto{\pgfqpoint{5.526195in}{2.413209in}}{\pgfqpoint{5.515596in}{2.417600in}}{\pgfqpoint{5.504545in}{2.417600in}}%
\pgfpathcurveto{\pgfqpoint{5.493495in}{2.417600in}}{\pgfqpoint{5.482896in}{2.413209in}}{\pgfqpoint{5.475083in}{2.405396in}}%
\pgfpathcurveto{\pgfqpoint{5.467269in}{2.397582in}}{\pgfqpoint{5.462879in}{2.386983in}}{\pgfqpoint{5.462879in}{2.375933in}}%
\pgfpathcurveto{\pgfqpoint{5.462879in}{2.364883in}}{\pgfqpoint{5.467269in}{2.354284in}}{\pgfqpoint{5.475083in}{2.346470in}}%
\pgfpathcurveto{\pgfqpoint{5.482896in}{2.338657in}}{\pgfqpoint{5.493495in}{2.334266in}}{\pgfqpoint{5.504545in}{2.334266in}}%
\pgfpathclose%
\pgfusepath{stroke,fill}%
\end{pgfscope}%
\begin{pgfscope}%
\pgfpathrectangle{\pgfqpoint{0.800000in}{0.528000in}}{\pgfqpoint{4.960000in}{3.696000in}}%
\pgfusepath{clip}%
\pgfsetbuttcap%
\pgfsetroundjoin%
\definecolor{currentfill}{rgb}{0.000000,0.000000,0.000000}%
\pgfsetfillcolor{currentfill}%
\pgfsetlinewidth{1.003750pt}%
\definecolor{currentstroke}{rgb}{0.000000,0.000000,0.000000}%
\pgfsetstrokecolor{currentstroke}%
\pgfsetdash{}{0pt}%
\pgfpathmoveto{\pgfqpoint{5.504545in}{3.984333in}}%
\pgfpathcurveto{\pgfqpoint{5.515596in}{3.984333in}}{\pgfqpoint{5.526195in}{3.988724in}}{\pgfqpoint{5.534008in}{3.996537in}}%
\pgfpathcurveto{\pgfqpoint{5.541822in}{4.004351in}}{\pgfqpoint{5.546212in}{4.014950in}}{\pgfqpoint{5.546212in}{4.026000in}}%
\pgfpathcurveto{\pgfqpoint{5.546212in}{4.037050in}}{\pgfqpoint{5.541822in}{4.047649in}}{\pgfqpoint{5.534008in}{4.055463in}}%
\pgfpathcurveto{\pgfqpoint{5.526195in}{4.063276in}}{\pgfqpoint{5.515596in}{4.067667in}}{\pgfqpoint{5.504545in}{4.067667in}}%
\pgfpathcurveto{\pgfqpoint{5.493495in}{4.067667in}}{\pgfqpoint{5.482896in}{4.063276in}}{\pgfqpoint{5.475083in}{4.055463in}}%
\pgfpathcurveto{\pgfqpoint{5.467269in}{4.047649in}}{\pgfqpoint{5.462879in}{4.037050in}}{\pgfqpoint{5.462879in}{4.026000in}}%
\pgfpathcurveto{\pgfqpoint{5.462879in}{4.014950in}}{\pgfqpoint{5.467269in}{4.004351in}}{\pgfqpoint{5.475083in}{3.996537in}}%
\pgfpathcurveto{\pgfqpoint{5.482896in}{3.988724in}}{\pgfqpoint{5.493495in}{3.984333in}}{\pgfqpoint{5.504545in}{3.984333in}}%
\pgfpathclose%
\pgfusepath{stroke,fill}%
\end{pgfscope}%
\begin{pgfscope}%
\pgfpathrectangle{\pgfqpoint{0.800000in}{0.528000in}}{\pgfqpoint{4.960000in}{3.696000in}}%
\pgfusepath{clip}%
\pgfsetbuttcap%
\pgfsetroundjoin%
\definecolor{currentfill}{rgb}{0.000000,0.000000,0.000000}%
\pgfsetfillcolor{currentfill}%
\pgfsetlinewidth{1.003750pt}%
\definecolor{currentstroke}{rgb}{0.000000,0.000000,0.000000}%
\pgfsetstrokecolor{currentstroke}%
\pgfsetdash{}{0pt}%
\pgfpathmoveto{\pgfqpoint{5.504545in}{2.334266in}}%
\pgfpathcurveto{\pgfqpoint{5.515596in}{2.334266in}}{\pgfqpoint{5.526195in}{2.338657in}}{\pgfqpoint{5.534008in}{2.346470in}}%
\pgfpathcurveto{\pgfqpoint{5.541822in}{2.354284in}}{\pgfqpoint{5.546212in}{2.364883in}}{\pgfqpoint{5.546212in}{2.375933in}}%
\pgfpathcurveto{\pgfqpoint{5.546212in}{2.386983in}}{\pgfqpoint{5.541822in}{2.397582in}}{\pgfqpoint{5.534008in}{2.405396in}}%
\pgfpathcurveto{\pgfqpoint{5.526195in}{2.413209in}}{\pgfqpoint{5.515596in}{2.417600in}}{\pgfqpoint{5.504545in}{2.417600in}}%
\pgfpathcurveto{\pgfqpoint{5.493495in}{2.417600in}}{\pgfqpoint{5.482896in}{2.413209in}}{\pgfqpoint{5.475083in}{2.405396in}}%
\pgfpathcurveto{\pgfqpoint{5.467269in}{2.397582in}}{\pgfqpoint{5.462879in}{2.386983in}}{\pgfqpoint{5.462879in}{2.375933in}}%
\pgfpathcurveto{\pgfqpoint{5.462879in}{2.364883in}}{\pgfqpoint{5.467269in}{2.354284in}}{\pgfqpoint{5.475083in}{2.346470in}}%
\pgfpathcurveto{\pgfqpoint{5.482896in}{2.338657in}}{\pgfqpoint{5.493495in}{2.334266in}}{\pgfqpoint{5.504545in}{2.334266in}}%
\pgfpathclose%
\pgfusepath{stroke,fill}%
\end{pgfscope}%
\begin{pgfscope}%
\pgfpathrectangle{\pgfqpoint{0.800000in}{0.528000in}}{\pgfqpoint{4.960000in}{3.696000in}}%
\pgfusepath{clip}%
\pgfsetbuttcap%
\pgfsetroundjoin%
\definecolor{currentfill}{rgb}{0.000000,0.000000,0.000000}%
\pgfsetfillcolor{currentfill}%
\pgfsetlinewidth{1.003750pt}%
\definecolor{currentstroke}{rgb}{0.000000,0.000000,0.000000}%
\pgfsetstrokecolor{currentstroke}%
\pgfsetdash{}{0pt}%
\pgfpathmoveto{\pgfqpoint{5.504545in}{2.334266in}}%
\pgfpathcurveto{\pgfqpoint{5.515596in}{2.334266in}}{\pgfqpoint{5.526195in}{2.338657in}}{\pgfqpoint{5.534008in}{2.346470in}}%
\pgfpathcurveto{\pgfqpoint{5.541822in}{2.354284in}}{\pgfqpoint{5.546212in}{2.364883in}}{\pgfqpoint{5.546212in}{2.375933in}}%
\pgfpathcurveto{\pgfqpoint{5.546212in}{2.386983in}}{\pgfqpoint{5.541822in}{2.397582in}}{\pgfqpoint{5.534008in}{2.405396in}}%
\pgfpathcurveto{\pgfqpoint{5.526195in}{2.413209in}}{\pgfqpoint{5.515596in}{2.417600in}}{\pgfqpoint{5.504545in}{2.417600in}}%
\pgfpathcurveto{\pgfqpoint{5.493495in}{2.417600in}}{\pgfqpoint{5.482896in}{2.413209in}}{\pgfqpoint{5.475083in}{2.405396in}}%
\pgfpathcurveto{\pgfqpoint{5.467269in}{2.397582in}}{\pgfqpoint{5.462879in}{2.386983in}}{\pgfqpoint{5.462879in}{2.375933in}}%
\pgfpathcurveto{\pgfqpoint{5.462879in}{2.364883in}}{\pgfqpoint{5.467269in}{2.354284in}}{\pgfqpoint{5.475083in}{2.346470in}}%
\pgfpathcurveto{\pgfqpoint{5.482896in}{2.338657in}}{\pgfqpoint{5.493495in}{2.334266in}}{\pgfqpoint{5.504545in}{2.334266in}}%
\pgfpathclose%
\pgfusepath{stroke,fill}%
\end{pgfscope}%
\begin{pgfscope}%
\pgfpathrectangle{\pgfqpoint{0.800000in}{0.528000in}}{\pgfqpoint{4.960000in}{3.696000in}}%
\pgfusepath{clip}%
\pgfsetbuttcap%
\pgfsetroundjoin%
\definecolor{currentfill}{rgb}{0.000000,0.000000,0.000000}%
\pgfsetfillcolor{currentfill}%
\pgfsetlinewidth{1.003750pt}%
\definecolor{currentstroke}{rgb}{0.000000,0.000000,0.000000}%
\pgfsetstrokecolor{currentstroke}%
\pgfsetdash{}{0pt}%
\pgfpathmoveto{\pgfqpoint{5.504545in}{2.334266in}}%
\pgfpathcurveto{\pgfqpoint{5.515596in}{2.334266in}}{\pgfqpoint{5.526195in}{2.338657in}}{\pgfqpoint{5.534008in}{2.346470in}}%
\pgfpathcurveto{\pgfqpoint{5.541822in}{2.354284in}}{\pgfqpoint{5.546212in}{2.364883in}}{\pgfqpoint{5.546212in}{2.375933in}}%
\pgfpathcurveto{\pgfqpoint{5.546212in}{2.386983in}}{\pgfqpoint{5.541822in}{2.397582in}}{\pgfqpoint{5.534008in}{2.405396in}}%
\pgfpathcurveto{\pgfqpoint{5.526195in}{2.413209in}}{\pgfqpoint{5.515596in}{2.417600in}}{\pgfqpoint{5.504545in}{2.417600in}}%
\pgfpathcurveto{\pgfqpoint{5.493495in}{2.417600in}}{\pgfqpoint{5.482896in}{2.413209in}}{\pgfqpoint{5.475083in}{2.405396in}}%
\pgfpathcurveto{\pgfqpoint{5.467269in}{2.397582in}}{\pgfqpoint{5.462879in}{2.386983in}}{\pgfqpoint{5.462879in}{2.375933in}}%
\pgfpathcurveto{\pgfqpoint{5.462879in}{2.364883in}}{\pgfqpoint{5.467269in}{2.354284in}}{\pgfqpoint{5.475083in}{2.346470in}}%
\pgfpathcurveto{\pgfqpoint{5.482896in}{2.338657in}}{\pgfqpoint{5.493495in}{2.334266in}}{\pgfqpoint{5.504545in}{2.334266in}}%
\pgfpathclose%
\pgfusepath{stroke,fill}%
\end{pgfscope}%
\begin{pgfscope}%
\pgfpathrectangle{\pgfqpoint{0.800000in}{0.528000in}}{\pgfqpoint{4.960000in}{3.696000in}}%
\pgfusepath{clip}%
\pgfsetbuttcap%
\pgfsetroundjoin%
\definecolor{currentfill}{rgb}{0.000000,0.000000,0.000000}%
\pgfsetfillcolor{currentfill}%
\pgfsetlinewidth{1.003750pt}%
\definecolor{currentstroke}{rgb}{0.000000,0.000000,0.000000}%
\pgfsetstrokecolor{currentstroke}%
\pgfsetdash{}{0pt}%
\pgfpathmoveto{\pgfqpoint{5.504545in}{3.984333in}}%
\pgfpathcurveto{\pgfqpoint{5.515596in}{3.984333in}}{\pgfqpoint{5.526195in}{3.988724in}}{\pgfqpoint{5.534008in}{3.996537in}}%
\pgfpathcurveto{\pgfqpoint{5.541822in}{4.004351in}}{\pgfqpoint{5.546212in}{4.014950in}}{\pgfqpoint{5.546212in}{4.026000in}}%
\pgfpathcurveto{\pgfqpoint{5.546212in}{4.037050in}}{\pgfqpoint{5.541822in}{4.047649in}}{\pgfqpoint{5.534008in}{4.055463in}}%
\pgfpathcurveto{\pgfqpoint{5.526195in}{4.063276in}}{\pgfqpoint{5.515596in}{4.067667in}}{\pgfqpoint{5.504545in}{4.067667in}}%
\pgfpathcurveto{\pgfqpoint{5.493495in}{4.067667in}}{\pgfqpoint{5.482896in}{4.063276in}}{\pgfqpoint{5.475083in}{4.055463in}}%
\pgfpathcurveto{\pgfqpoint{5.467269in}{4.047649in}}{\pgfqpoint{5.462879in}{4.037050in}}{\pgfqpoint{5.462879in}{4.026000in}}%
\pgfpathcurveto{\pgfqpoint{5.462879in}{4.014950in}}{\pgfqpoint{5.467269in}{4.004351in}}{\pgfqpoint{5.475083in}{3.996537in}}%
\pgfpathcurveto{\pgfqpoint{5.482896in}{3.988724in}}{\pgfqpoint{5.493495in}{3.984333in}}{\pgfqpoint{5.504545in}{3.984333in}}%
\pgfpathclose%
\pgfusepath{stroke,fill}%
\end{pgfscope}%
\begin{pgfscope}%
\pgfpathrectangle{\pgfqpoint{0.800000in}{0.528000in}}{\pgfqpoint{4.960000in}{3.696000in}}%
\pgfusepath{clip}%
\pgfsetbuttcap%
\pgfsetroundjoin%
\definecolor{currentfill}{rgb}{0.000000,0.000000,0.000000}%
\pgfsetfillcolor{currentfill}%
\pgfsetlinewidth{1.003750pt}%
\definecolor{currentstroke}{rgb}{0.000000,0.000000,0.000000}%
\pgfsetstrokecolor{currentstroke}%
\pgfsetdash{}{0pt}%
\pgfpathmoveto{\pgfqpoint{5.504545in}{3.984333in}}%
\pgfpathcurveto{\pgfqpoint{5.515596in}{3.984333in}}{\pgfqpoint{5.526195in}{3.988724in}}{\pgfqpoint{5.534008in}{3.996537in}}%
\pgfpathcurveto{\pgfqpoint{5.541822in}{4.004351in}}{\pgfqpoint{5.546212in}{4.014950in}}{\pgfqpoint{5.546212in}{4.026000in}}%
\pgfpathcurveto{\pgfqpoint{5.546212in}{4.037050in}}{\pgfqpoint{5.541822in}{4.047649in}}{\pgfqpoint{5.534008in}{4.055463in}}%
\pgfpathcurveto{\pgfqpoint{5.526195in}{4.063276in}}{\pgfqpoint{5.515596in}{4.067667in}}{\pgfqpoint{5.504545in}{4.067667in}}%
\pgfpathcurveto{\pgfqpoint{5.493495in}{4.067667in}}{\pgfqpoint{5.482896in}{4.063276in}}{\pgfqpoint{5.475083in}{4.055463in}}%
\pgfpathcurveto{\pgfqpoint{5.467269in}{4.047649in}}{\pgfqpoint{5.462879in}{4.037050in}}{\pgfqpoint{5.462879in}{4.026000in}}%
\pgfpathcurveto{\pgfqpoint{5.462879in}{4.014950in}}{\pgfqpoint{5.467269in}{4.004351in}}{\pgfqpoint{5.475083in}{3.996537in}}%
\pgfpathcurveto{\pgfqpoint{5.482896in}{3.988724in}}{\pgfqpoint{5.493495in}{3.984333in}}{\pgfqpoint{5.504545in}{3.984333in}}%
\pgfpathclose%
\pgfusepath{stroke,fill}%
\end{pgfscope}%
\begin{pgfscope}%
\pgfpathrectangle{\pgfqpoint{0.800000in}{0.528000in}}{\pgfqpoint{4.960000in}{3.696000in}}%
\pgfusepath{clip}%
\pgfsetbuttcap%
\pgfsetroundjoin%
\definecolor{currentfill}{rgb}{0.000000,0.000000,0.000000}%
\pgfsetfillcolor{currentfill}%
\pgfsetlinewidth{1.003750pt}%
\definecolor{currentstroke}{rgb}{0.000000,0.000000,0.000000}%
\pgfsetstrokecolor{currentstroke}%
\pgfsetdash{}{0pt}%
\pgfpathmoveto{\pgfqpoint{5.504545in}{2.334266in}}%
\pgfpathcurveto{\pgfqpoint{5.515596in}{2.334266in}}{\pgfqpoint{5.526195in}{2.338657in}}{\pgfqpoint{5.534008in}{2.346470in}}%
\pgfpathcurveto{\pgfqpoint{5.541822in}{2.354284in}}{\pgfqpoint{5.546212in}{2.364883in}}{\pgfqpoint{5.546212in}{2.375933in}}%
\pgfpathcurveto{\pgfqpoint{5.546212in}{2.386983in}}{\pgfqpoint{5.541822in}{2.397582in}}{\pgfqpoint{5.534008in}{2.405396in}}%
\pgfpathcurveto{\pgfqpoint{5.526195in}{2.413209in}}{\pgfqpoint{5.515596in}{2.417600in}}{\pgfqpoint{5.504545in}{2.417600in}}%
\pgfpathcurveto{\pgfqpoint{5.493495in}{2.417600in}}{\pgfqpoint{5.482896in}{2.413209in}}{\pgfqpoint{5.475083in}{2.405396in}}%
\pgfpathcurveto{\pgfqpoint{5.467269in}{2.397582in}}{\pgfqpoint{5.462879in}{2.386983in}}{\pgfqpoint{5.462879in}{2.375933in}}%
\pgfpathcurveto{\pgfqpoint{5.462879in}{2.364883in}}{\pgfqpoint{5.467269in}{2.354284in}}{\pgfqpoint{5.475083in}{2.346470in}}%
\pgfpathcurveto{\pgfqpoint{5.482896in}{2.338657in}}{\pgfqpoint{5.493495in}{2.334266in}}{\pgfqpoint{5.504545in}{2.334266in}}%
\pgfpathclose%
\pgfusepath{stroke,fill}%
\end{pgfscope}%
\begin{pgfscope}%
\pgfpathrectangle{\pgfqpoint{0.800000in}{0.528000in}}{\pgfqpoint{4.960000in}{3.696000in}}%
\pgfusepath{clip}%
\pgfsetbuttcap%
\pgfsetroundjoin%
\definecolor{currentfill}{rgb}{0.000000,0.000000,0.000000}%
\pgfsetfillcolor{currentfill}%
\pgfsetlinewidth{1.003750pt}%
\definecolor{currentstroke}{rgb}{0.000000,0.000000,0.000000}%
\pgfsetstrokecolor{currentstroke}%
\pgfsetdash{}{0pt}%
\pgfpathmoveto{\pgfqpoint{5.504545in}{3.984333in}}%
\pgfpathcurveto{\pgfqpoint{5.515596in}{3.984333in}}{\pgfqpoint{5.526195in}{3.988724in}}{\pgfqpoint{5.534008in}{3.996537in}}%
\pgfpathcurveto{\pgfqpoint{5.541822in}{4.004351in}}{\pgfqpoint{5.546212in}{4.014950in}}{\pgfqpoint{5.546212in}{4.026000in}}%
\pgfpathcurveto{\pgfqpoint{5.546212in}{4.037050in}}{\pgfqpoint{5.541822in}{4.047649in}}{\pgfqpoint{5.534008in}{4.055463in}}%
\pgfpathcurveto{\pgfqpoint{5.526195in}{4.063276in}}{\pgfqpoint{5.515596in}{4.067667in}}{\pgfqpoint{5.504545in}{4.067667in}}%
\pgfpathcurveto{\pgfqpoint{5.493495in}{4.067667in}}{\pgfqpoint{5.482896in}{4.063276in}}{\pgfqpoint{5.475083in}{4.055463in}}%
\pgfpathcurveto{\pgfqpoint{5.467269in}{4.047649in}}{\pgfqpoint{5.462879in}{4.037050in}}{\pgfqpoint{5.462879in}{4.026000in}}%
\pgfpathcurveto{\pgfqpoint{5.462879in}{4.014950in}}{\pgfqpoint{5.467269in}{4.004351in}}{\pgfqpoint{5.475083in}{3.996537in}}%
\pgfpathcurveto{\pgfqpoint{5.482896in}{3.988724in}}{\pgfqpoint{5.493495in}{3.984333in}}{\pgfqpoint{5.504545in}{3.984333in}}%
\pgfpathclose%
\pgfusepath{stroke,fill}%
\end{pgfscope}%
\begin{pgfscope}%
\pgfpathrectangle{\pgfqpoint{0.800000in}{0.528000in}}{\pgfqpoint{4.960000in}{3.696000in}}%
\pgfusepath{clip}%
\pgfsetbuttcap%
\pgfsetroundjoin%
\definecolor{currentfill}{rgb}{0.000000,0.000000,0.000000}%
\pgfsetfillcolor{currentfill}%
\pgfsetlinewidth{1.003750pt}%
\definecolor{currentstroke}{rgb}{0.000000,0.000000,0.000000}%
\pgfsetstrokecolor{currentstroke}%
\pgfsetdash{}{0pt}%
\pgfpathmoveto{\pgfqpoint{5.504545in}{3.984333in}}%
\pgfpathcurveto{\pgfqpoint{5.515596in}{3.984333in}}{\pgfqpoint{5.526195in}{3.988724in}}{\pgfqpoint{5.534008in}{3.996537in}}%
\pgfpathcurveto{\pgfqpoint{5.541822in}{4.004351in}}{\pgfqpoint{5.546212in}{4.014950in}}{\pgfqpoint{5.546212in}{4.026000in}}%
\pgfpathcurveto{\pgfqpoint{5.546212in}{4.037050in}}{\pgfqpoint{5.541822in}{4.047649in}}{\pgfqpoint{5.534008in}{4.055463in}}%
\pgfpathcurveto{\pgfqpoint{5.526195in}{4.063276in}}{\pgfqpoint{5.515596in}{4.067667in}}{\pgfqpoint{5.504545in}{4.067667in}}%
\pgfpathcurveto{\pgfqpoint{5.493495in}{4.067667in}}{\pgfqpoint{5.482896in}{4.063276in}}{\pgfqpoint{5.475083in}{4.055463in}}%
\pgfpathcurveto{\pgfqpoint{5.467269in}{4.047649in}}{\pgfqpoint{5.462879in}{4.037050in}}{\pgfqpoint{5.462879in}{4.026000in}}%
\pgfpathcurveto{\pgfqpoint{5.462879in}{4.014950in}}{\pgfqpoint{5.467269in}{4.004351in}}{\pgfqpoint{5.475083in}{3.996537in}}%
\pgfpathcurveto{\pgfqpoint{5.482896in}{3.988724in}}{\pgfqpoint{5.493495in}{3.984333in}}{\pgfqpoint{5.504545in}{3.984333in}}%
\pgfpathclose%
\pgfusepath{stroke,fill}%
\end{pgfscope}%
\begin{pgfscope}%
\pgfpathrectangle{\pgfqpoint{0.800000in}{0.528000in}}{\pgfqpoint{4.960000in}{3.696000in}}%
\pgfusepath{clip}%
\pgfsetbuttcap%
\pgfsetroundjoin%
\definecolor{currentfill}{rgb}{0.000000,0.000000,0.000000}%
\pgfsetfillcolor{currentfill}%
\pgfsetlinewidth{1.003750pt}%
\definecolor{currentstroke}{rgb}{0.000000,0.000000,0.000000}%
\pgfsetstrokecolor{currentstroke}%
\pgfsetdash{}{0pt}%
\pgfpathmoveto{\pgfqpoint{5.504545in}{3.984333in}}%
\pgfpathcurveto{\pgfqpoint{5.515596in}{3.984333in}}{\pgfqpoint{5.526195in}{3.988724in}}{\pgfqpoint{5.534008in}{3.996537in}}%
\pgfpathcurveto{\pgfqpoint{5.541822in}{4.004351in}}{\pgfqpoint{5.546212in}{4.014950in}}{\pgfqpoint{5.546212in}{4.026000in}}%
\pgfpathcurveto{\pgfqpoint{5.546212in}{4.037050in}}{\pgfqpoint{5.541822in}{4.047649in}}{\pgfqpoint{5.534008in}{4.055463in}}%
\pgfpathcurveto{\pgfqpoint{5.526195in}{4.063276in}}{\pgfqpoint{5.515596in}{4.067667in}}{\pgfqpoint{5.504545in}{4.067667in}}%
\pgfpathcurveto{\pgfqpoint{5.493495in}{4.067667in}}{\pgfqpoint{5.482896in}{4.063276in}}{\pgfqpoint{5.475083in}{4.055463in}}%
\pgfpathcurveto{\pgfqpoint{5.467269in}{4.047649in}}{\pgfqpoint{5.462879in}{4.037050in}}{\pgfqpoint{5.462879in}{4.026000in}}%
\pgfpathcurveto{\pgfqpoint{5.462879in}{4.014950in}}{\pgfqpoint{5.467269in}{4.004351in}}{\pgfqpoint{5.475083in}{3.996537in}}%
\pgfpathcurveto{\pgfqpoint{5.482896in}{3.988724in}}{\pgfqpoint{5.493495in}{3.984333in}}{\pgfqpoint{5.504545in}{3.984333in}}%
\pgfpathclose%
\pgfusepath{stroke,fill}%
\end{pgfscope}%
\begin{pgfscope}%
\pgfpathrectangle{\pgfqpoint{0.800000in}{0.528000in}}{\pgfqpoint{4.960000in}{3.696000in}}%
\pgfusepath{clip}%
\pgfsetbuttcap%
\pgfsetroundjoin%
\definecolor{currentfill}{rgb}{0.000000,0.000000,0.000000}%
\pgfsetfillcolor{currentfill}%
\pgfsetlinewidth{1.003750pt}%
\definecolor{currentstroke}{rgb}{0.000000,0.000000,0.000000}%
\pgfsetstrokecolor{currentstroke}%
\pgfsetdash{}{0pt}%
\pgfpathmoveto{\pgfqpoint{5.504545in}{2.334266in}}%
\pgfpathcurveto{\pgfqpoint{5.515596in}{2.334266in}}{\pgfqpoint{5.526195in}{2.338657in}}{\pgfqpoint{5.534008in}{2.346470in}}%
\pgfpathcurveto{\pgfqpoint{5.541822in}{2.354284in}}{\pgfqpoint{5.546212in}{2.364883in}}{\pgfqpoint{5.546212in}{2.375933in}}%
\pgfpathcurveto{\pgfqpoint{5.546212in}{2.386983in}}{\pgfqpoint{5.541822in}{2.397582in}}{\pgfqpoint{5.534008in}{2.405396in}}%
\pgfpathcurveto{\pgfqpoint{5.526195in}{2.413209in}}{\pgfqpoint{5.515596in}{2.417600in}}{\pgfqpoint{5.504545in}{2.417600in}}%
\pgfpathcurveto{\pgfqpoint{5.493495in}{2.417600in}}{\pgfqpoint{5.482896in}{2.413209in}}{\pgfqpoint{5.475083in}{2.405396in}}%
\pgfpathcurveto{\pgfqpoint{5.467269in}{2.397582in}}{\pgfqpoint{5.462879in}{2.386983in}}{\pgfqpoint{5.462879in}{2.375933in}}%
\pgfpathcurveto{\pgfqpoint{5.462879in}{2.364883in}}{\pgfqpoint{5.467269in}{2.354284in}}{\pgfqpoint{5.475083in}{2.346470in}}%
\pgfpathcurveto{\pgfqpoint{5.482896in}{2.338657in}}{\pgfqpoint{5.493495in}{2.334266in}}{\pgfqpoint{5.504545in}{2.334266in}}%
\pgfpathclose%
\pgfusepath{stroke,fill}%
\end{pgfscope}%
\begin{pgfscope}%
\pgfpathrectangle{\pgfqpoint{0.800000in}{0.528000in}}{\pgfqpoint{4.960000in}{3.696000in}}%
\pgfusepath{clip}%
\pgfsetbuttcap%
\pgfsetroundjoin%
\definecolor{currentfill}{rgb}{0.000000,0.000000,0.000000}%
\pgfsetfillcolor{currentfill}%
\pgfsetlinewidth{1.003750pt}%
\definecolor{currentstroke}{rgb}{0.000000,0.000000,0.000000}%
\pgfsetstrokecolor{currentstroke}%
\pgfsetdash{}{0pt}%
\pgfpathmoveto{\pgfqpoint{5.504545in}{2.334266in}}%
\pgfpathcurveto{\pgfqpoint{5.515596in}{2.334266in}}{\pgfqpoint{5.526195in}{2.338657in}}{\pgfqpoint{5.534008in}{2.346470in}}%
\pgfpathcurveto{\pgfqpoint{5.541822in}{2.354284in}}{\pgfqpoint{5.546212in}{2.364883in}}{\pgfqpoint{5.546212in}{2.375933in}}%
\pgfpathcurveto{\pgfqpoint{5.546212in}{2.386983in}}{\pgfqpoint{5.541822in}{2.397582in}}{\pgfqpoint{5.534008in}{2.405396in}}%
\pgfpathcurveto{\pgfqpoint{5.526195in}{2.413209in}}{\pgfqpoint{5.515596in}{2.417600in}}{\pgfqpoint{5.504545in}{2.417600in}}%
\pgfpathcurveto{\pgfqpoint{5.493495in}{2.417600in}}{\pgfqpoint{5.482896in}{2.413209in}}{\pgfqpoint{5.475083in}{2.405396in}}%
\pgfpathcurveto{\pgfqpoint{5.467269in}{2.397582in}}{\pgfqpoint{5.462879in}{2.386983in}}{\pgfqpoint{5.462879in}{2.375933in}}%
\pgfpathcurveto{\pgfqpoint{5.462879in}{2.364883in}}{\pgfqpoint{5.467269in}{2.354284in}}{\pgfqpoint{5.475083in}{2.346470in}}%
\pgfpathcurveto{\pgfqpoint{5.482896in}{2.338657in}}{\pgfqpoint{5.493495in}{2.334266in}}{\pgfqpoint{5.504545in}{2.334266in}}%
\pgfpathclose%
\pgfusepath{stroke,fill}%
\end{pgfscope}%
\begin{pgfscope}%
\pgfpathrectangle{\pgfqpoint{0.800000in}{0.528000in}}{\pgfqpoint{4.960000in}{3.696000in}}%
\pgfusepath{clip}%
\pgfsetbuttcap%
\pgfsetroundjoin%
\definecolor{currentfill}{rgb}{0.000000,0.000000,0.000000}%
\pgfsetfillcolor{currentfill}%
\pgfsetlinewidth{1.003750pt}%
\definecolor{currentstroke}{rgb}{0.000000,0.000000,0.000000}%
\pgfsetstrokecolor{currentstroke}%
\pgfsetdash{}{0pt}%
\pgfpathmoveto{\pgfqpoint{5.504545in}{3.984333in}}%
\pgfpathcurveto{\pgfqpoint{5.515596in}{3.984333in}}{\pgfqpoint{5.526195in}{3.988724in}}{\pgfqpoint{5.534008in}{3.996537in}}%
\pgfpathcurveto{\pgfqpoint{5.541822in}{4.004351in}}{\pgfqpoint{5.546212in}{4.014950in}}{\pgfqpoint{5.546212in}{4.026000in}}%
\pgfpathcurveto{\pgfqpoint{5.546212in}{4.037050in}}{\pgfqpoint{5.541822in}{4.047649in}}{\pgfqpoint{5.534008in}{4.055463in}}%
\pgfpathcurveto{\pgfqpoint{5.526195in}{4.063276in}}{\pgfqpoint{5.515596in}{4.067667in}}{\pgfqpoint{5.504545in}{4.067667in}}%
\pgfpathcurveto{\pgfqpoint{5.493495in}{4.067667in}}{\pgfqpoint{5.482896in}{4.063276in}}{\pgfqpoint{5.475083in}{4.055463in}}%
\pgfpathcurveto{\pgfqpoint{5.467269in}{4.047649in}}{\pgfqpoint{5.462879in}{4.037050in}}{\pgfqpoint{5.462879in}{4.026000in}}%
\pgfpathcurveto{\pgfqpoint{5.462879in}{4.014950in}}{\pgfqpoint{5.467269in}{4.004351in}}{\pgfqpoint{5.475083in}{3.996537in}}%
\pgfpathcurveto{\pgfqpoint{5.482896in}{3.988724in}}{\pgfqpoint{5.493495in}{3.984333in}}{\pgfqpoint{5.504545in}{3.984333in}}%
\pgfpathclose%
\pgfusepath{stroke,fill}%
\end{pgfscope}%
\begin{pgfscope}%
\pgfpathrectangle{\pgfqpoint{0.800000in}{0.528000in}}{\pgfqpoint{4.960000in}{3.696000in}}%
\pgfusepath{clip}%
\pgfsetbuttcap%
\pgfsetroundjoin%
\definecolor{currentfill}{rgb}{0.000000,0.000000,0.000000}%
\pgfsetfillcolor{currentfill}%
\pgfsetlinewidth{1.003750pt}%
\definecolor{currentstroke}{rgb}{0.000000,0.000000,0.000000}%
\pgfsetstrokecolor{currentstroke}%
\pgfsetdash{}{0pt}%
\pgfpathmoveto{\pgfqpoint{5.504545in}{3.984333in}}%
\pgfpathcurveto{\pgfqpoint{5.515596in}{3.984333in}}{\pgfqpoint{5.526195in}{3.988724in}}{\pgfqpoint{5.534008in}{3.996537in}}%
\pgfpathcurveto{\pgfqpoint{5.541822in}{4.004351in}}{\pgfqpoint{5.546212in}{4.014950in}}{\pgfqpoint{5.546212in}{4.026000in}}%
\pgfpathcurveto{\pgfqpoint{5.546212in}{4.037050in}}{\pgfqpoint{5.541822in}{4.047649in}}{\pgfqpoint{5.534008in}{4.055463in}}%
\pgfpathcurveto{\pgfqpoint{5.526195in}{4.063276in}}{\pgfqpoint{5.515596in}{4.067667in}}{\pgfqpoint{5.504545in}{4.067667in}}%
\pgfpathcurveto{\pgfqpoint{5.493495in}{4.067667in}}{\pgfqpoint{5.482896in}{4.063276in}}{\pgfqpoint{5.475083in}{4.055463in}}%
\pgfpathcurveto{\pgfqpoint{5.467269in}{4.047649in}}{\pgfqpoint{5.462879in}{4.037050in}}{\pgfqpoint{5.462879in}{4.026000in}}%
\pgfpathcurveto{\pgfqpoint{5.462879in}{4.014950in}}{\pgfqpoint{5.467269in}{4.004351in}}{\pgfqpoint{5.475083in}{3.996537in}}%
\pgfpathcurveto{\pgfqpoint{5.482896in}{3.988724in}}{\pgfqpoint{5.493495in}{3.984333in}}{\pgfqpoint{5.504545in}{3.984333in}}%
\pgfpathclose%
\pgfusepath{stroke,fill}%
\end{pgfscope}%
\begin{pgfscope}%
\pgfpathrectangle{\pgfqpoint{0.800000in}{0.528000in}}{\pgfqpoint{4.960000in}{3.696000in}}%
\pgfusepath{clip}%
\pgfsetbuttcap%
\pgfsetroundjoin%
\definecolor{currentfill}{rgb}{0.000000,0.000000,0.000000}%
\pgfsetfillcolor{currentfill}%
\pgfsetlinewidth{1.003750pt}%
\definecolor{currentstroke}{rgb}{0.000000,0.000000,0.000000}%
\pgfsetstrokecolor{currentstroke}%
\pgfsetdash{}{0pt}%
\pgfpathmoveto{\pgfqpoint{5.504545in}{2.334266in}}%
\pgfpathcurveto{\pgfqpoint{5.515596in}{2.334266in}}{\pgfqpoint{5.526195in}{2.338657in}}{\pgfqpoint{5.534008in}{2.346470in}}%
\pgfpathcurveto{\pgfqpoint{5.541822in}{2.354284in}}{\pgfqpoint{5.546212in}{2.364883in}}{\pgfqpoint{5.546212in}{2.375933in}}%
\pgfpathcurveto{\pgfqpoint{5.546212in}{2.386983in}}{\pgfqpoint{5.541822in}{2.397582in}}{\pgfqpoint{5.534008in}{2.405396in}}%
\pgfpathcurveto{\pgfqpoint{5.526195in}{2.413209in}}{\pgfqpoint{5.515596in}{2.417600in}}{\pgfqpoint{5.504545in}{2.417600in}}%
\pgfpathcurveto{\pgfqpoint{5.493495in}{2.417600in}}{\pgfqpoint{5.482896in}{2.413209in}}{\pgfqpoint{5.475083in}{2.405396in}}%
\pgfpathcurveto{\pgfqpoint{5.467269in}{2.397582in}}{\pgfqpoint{5.462879in}{2.386983in}}{\pgfqpoint{5.462879in}{2.375933in}}%
\pgfpathcurveto{\pgfqpoint{5.462879in}{2.364883in}}{\pgfqpoint{5.467269in}{2.354284in}}{\pgfqpoint{5.475083in}{2.346470in}}%
\pgfpathcurveto{\pgfqpoint{5.482896in}{2.338657in}}{\pgfqpoint{5.493495in}{2.334266in}}{\pgfqpoint{5.504545in}{2.334266in}}%
\pgfpathclose%
\pgfusepath{stroke,fill}%
\end{pgfscope}%
\begin{pgfscope}%
\pgfpathrectangle{\pgfqpoint{0.800000in}{0.528000in}}{\pgfqpoint{4.960000in}{3.696000in}}%
\pgfusepath{clip}%
\pgfsetbuttcap%
\pgfsetroundjoin%
\definecolor{currentfill}{rgb}{0.000000,0.000000,0.000000}%
\pgfsetfillcolor{currentfill}%
\pgfsetlinewidth{1.003750pt}%
\definecolor{currentstroke}{rgb}{0.000000,0.000000,0.000000}%
\pgfsetstrokecolor{currentstroke}%
\pgfsetdash{}{0pt}%
\pgfpathmoveto{\pgfqpoint{5.504545in}{2.334266in}}%
\pgfpathcurveto{\pgfqpoint{5.515596in}{2.334266in}}{\pgfqpoint{5.526195in}{2.338657in}}{\pgfqpoint{5.534008in}{2.346470in}}%
\pgfpathcurveto{\pgfqpoint{5.541822in}{2.354284in}}{\pgfqpoint{5.546212in}{2.364883in}}{\pgfqpoint{5.546212in}{2.375933in}}%
\pgfpathcurveto{\pgfqpoint{5.546212in}{2.386983in}}{\pgfqpoint{5.541822in}{2.397582in}}{\pgfqpoint{5.534008in}{2.405396in}}%
\pgfpathcurveto{\pgfqpoint{5.526195in}{2.413209in}}{\pgfqpoint{5.515596in}{2.417600in}}{\pgfqpoint{5.504545in}{2.417600in}}%
\pgfpathcurveto{\pgfqpoint{5.493495in}{2.417600in}}{\pgfqpoint{5.482896in}{2.413209in}}{\pgfqpoint{5.475083in}{2.405396in}}%
\pgfpathcurveto{\pgfqpoint{5.467269in}{2.397582in}}{\pgfqpoint{5.462879in}{2.386983in}}{\pgfqpoint{5.462879in}{2.375933in}}%
\pgfpathcurveto{\pgfqpoint{5.462879in}{2.364883in}}{\pgfqpoint{5.467269in}{2.354284in}}{\pgfqpoint{5.475083in}{2.346470in}}%
\pgfpathcurveto{\pgfqpoint{5.482896in}{2.338657in}}{\pgfqpoint{5.493495in}{2.334266in}}{\pgfqpoint{5.504545in}{2.334266in}}%
\pgfpathclose%
\pgfusepath{stroke,fill}%
\end{pgfscope}%
\begin{pgfscope}%
\pgfpathrectangle{\pgfqpoint{0.800000in}{0.528000in}}{\pgfqpoint{4.960000in}{3.696000in}}%
\pgfusepath{clip}%
\pgfsetbuttcap%
\pgfsetroundjoin%
\definecolor{currentfill}{rgb}{0.000000,0.000000,0.000000}%
\pgfsetfillcolor{currentfill}%
\pgfsetlinewidth{1.003750pt}%
\definecolor{currentstroke}{rgb}{0.000000,0.000000,0.000000}%
\pgfsetstrokecolor{currentstroke}%
\pgfsetdash{}{0pt}%
\pgfpathmoveto{\pgfqpoint{5.504545in}{3.984333in}}%
\pgfpathcurveto{\pgfqpoint{5.515596in}{3.984333in}}{\pgfqpoint{5.526195in}{3.988724in}}{\pgfqpoint{5.534008in}{3.996537in}}%
\pgfpathcurveto{\pgfqpoint{5.541822in}{4.004351in}}{\pgfqpoint{5.546212in}{4.014950in}}{\pgfqpoint{5.546212in}{4.026000in}}%
\pgfpathcurveto{\pgfqpoint{5.546212in}{4.037050in}}{\pgfqpoint{5.541822in}{4.047649in}}{\pgfqpoint{5.534008in}{4.055463in}}%
\pgfpathcurveto{\pgfqpoint{5.526195in}{4.063276in}}{\pgfqpoint{5.515596in}{4.067667in}}{\pgfqpoint{5.504545in}{4.067667in}}%
\pgfpathcurveto{\pgfqpoint{5.493495in}{4.067667in}}{\pgfqpoint{5.482896in}{4.063276in}}{\pgfqpoint{5.475083in}{4.055463in}}%
\pgfpathcurveto{\pgfqpoint{5.467269in}{4.047649in}}{\pgfqpoint{5.462879in}{4.037050in}}{\pgfqpoint{5.462879in}{4.026000in}}%
\pgfpathcurveto{\pgfqpoint{5.462879in}{4.014950in}}{\pgfqpoint{5.467269in}{4.004351in}}{\pgfqpoint{5.475083in}{3.996537in}}%
\pgfpathcurveto{\pgfqpoint{5.482896in}{3.988724in}}{\pgfqpoint{5.493495in}{3.984333in}}{\pgfqpoint{5.504545in}{3.984333in}}%
\pgfpathclose%
\pgfusepath{stroke,fill}%
\end{pgfscope}%
\begin{pgfscope}%
\pgfpathrectangle{\pgfqpoint{0.800000in}{0.528000in}}{\pgfqpoint{4.960000in}{3.696000in}}%
\pgfusepath{clip}%
\pgfsetbuttcap%
\pgfsetroundjoin%
\definecolor{currentfill}{rgb}{0.000000,0.000000,0.000000}%
\pgfsetfillcolor{currentfill}%
\pgfsetlinewidth{1.003750pt}%
\definecolor{currentstroke}{rgb}{0.000000,0.000000,0.000000}%
\pgfsetstrokecolor{currentstroke}%
\pgfsetdash{}{0pt}%
\pgfpathmoveto{\pgfqpoint{5.504545in}{2.334266in}}%
\pgfpathcurveto{\pgfqpoint{5.515596in}{2.334266in}}{\pgfqpoint{5.526195in}{2.338657in}}{\pgfqpoint{5.534008in}{2.346470in}}%
\pgfpathcurveto{\pgfqpoint{5.541822in}{2.354284in}}{\pgfqpoint{5.546212in}{2.364883in}}{\pgfqpoint{5.546212in}{2.375933in}}%
\pgfpathcurveto{\pgfqpoint{5.546212in}{2.386983in}}{\pgfqpoint{5.541822in}{2.397582in}}{\pgfqpoint{5.534008in}{2.405396in}}%
\pgfpathcurveto{\pgfqpoint{5.526195in}{2.413209in}}{\pgfqpoint{5.515596in}{2.417600in}}{\pgfqpoint{5.504545in}{2.417600in}}%
\pgfpathcurveto{\pgfqpoint{5.493495in}{2.417600in}}{\pgfqpoint{5.482896in}{2.413209in}}{\pgfqpoint{5.475083in}{2.405396in}}%
\pgfpathcurveto{\pgfqpoint{5.467269in}{2.397582in}}{\pgfqpoint{5.462879in}{2.386983in}}{\pgfqpoint{5.462879in}{2.375933in}}%
\pgfpathcurveto{\pgfqpoint{5.462879in}{2.364883in}}{\pgfqpoint{5.467269in}{2.354284in}}{\pgfqpoint{5.475083in}{2.346470in}}%
\pgfpathcurveto{\pgfqpoint{5.482896in}{2.338657in}}{\pgfqpoint{5.493495in}{2.334266in}}{\pgfqpoint{5.504545in}{2.334266in}}%
\pgfpathclose%
\pgfusepath{stroke,fill}%
\end{pgfscope}%
\begin{pgfscope}%
\pgfpathrectangle{\pgfqpoint{0.800000in}{0.528000in}}{\pgfqpoint{4.960000in}{3.696000in}}%
\pgfusepath{clip}%
\pgfsetbuttcap%
\pgfsetroundjoin%
\definecolor{currentfill}{rgb}{0.000000,0.000000,0.000000}%
\pgfsetfillcolor{currentfill}%
\pgfsetlinewidth{1.003750pt}%
\definecolor{currentstroke}{rgb}{0.000000,0.000000,0.000000}%
\pgfsetstrokecolor{currentstroke}%
\pgfsetdash{}{0pt}%
\pgfpathmoveto{\pgfqpoint{5.504545in}{3.984333in}}%
\pgfpathcurveto{\pgfqpoint{5.515596in}{3.984333in}}{\pgfqpoint{5.526195in}{3.988724in}}{\pgfqpoint{5.534008in}{3.996537in}}%
\pgfpathcurveto{\pgfqpoint{5.541822in}{4.004351in}}{\pgfqpoint{5.546212in}{4.014950in}}{\pgfqpoint{5.546212in}{4.026000in}}%
\pgfpathcurveto{\pgfqpoint{5.546212in}{4.037050in}}{\pgfqpoint{5.541822in}{4.047649in}}{\pgfqpoint{5.534008in}{4.055463in}}%
\pgfpathcurveto{\pgfqpoint{5.526195in}{4.063276in}}{\pgfqpoint{5.515596in}{4.067667in}}{\pgfqpoint{5.504545in}{4.067667in}}%
\pgfpathcurveto{\pgfqpoint{5.493495in}{4.067667in}}{\pgfqpoint{5.482896in}{4.063276in}}{\pgfqpoint{5.475083in}{4.055463in}}%
\pgfpathcurveto{\pgfqpoint{5.467269in}{4.047649in}}{\pgfqpoint{5.462879in}{4.037050in}}{\pgfqpoint{5.462879in}{4.026000in}}%
\pgfpathcurveto{\pgfqpoint{5.462879in}{4.014950in}}{\pgfqpoint{5.467269in}{4.004351in}}{\pgfqpoint{5.475083in}{3.996537in}}%
\pgfpathcurveto{\pgfqpoint{5.482896in}{3.988724in}}{\pgfqpoint{5.493495in}{3.984333in}}{\pgfqpoint{5.504545in}{3.984333in}}%
\pgfpathclose%
\pgfusepath{stroke,fill}%
\end{pgfscope}%
\begin{pgfscope}%
\pgfpathrectangle{\pgfqpoint{0.800000in}{0.528000in}}{\pgfqpoint{4.960000in}{3.696000in}}%
\pgfusepath{clip}%
\pgfsetbuttcap%
\pgfsetroundjoin%
\definecolor{currentfill}{rgb}{0.000000,0.000000,0.000000}%
\pgfsetfillcolor{currentfill}%
\pgfsetlinewidth{1.003750pt}%
\definecolor{currentstroke}{rgb}{0.000000,0.000000,0.000000}%
\pgfsetstrokecolor{currentstroke}%
\pgfsetdash{}{0pt}%
\pgfpathmoveto{\pgfqpoint{5.504545in}{2.334266in}}%
\pgfpathcurveto{\pgfqpoint{5.515596in}{2.334266in}}{\pgfqpoint{5.526195in}{2.338657in}}{\pgfqpoint{5.534008in}{2.346470in}}%
\pgfpathcurveto{\pgfqpoint{5.541822in}{2.354284in}}{\pgfqpoint{5.546212in}{2.364883in}}{\pgfqpoint{5.546212in}{2.375933in}}%
\pgfpathcurveto{\pgfqpoint{5.546212in}{2.386983in}}{\pgfqpoint{5.541822in}{2.397582in}}{\pgfqpoint{5.534008in}{2.405396in}}%
\pgfpathcurveto{\pgfqpoint{5.526195in}{2.413209in}}{\pgfqpoint{5.515596in}{2.417600in}}{\pgfqpoint{5.504545in}{2.417600in}}%
\pgfpathcurveto{\pgfqpoint{5.493495in}{2.417600in}}{\pgfqpoint{5.482896in}{2.413209in}}{\pgfqpoint{5.475083in}{2.405396in}}%
\pgfpathcurveto{\pgfqpoint{5.467269in}{2.397582in}}{\pgfqpoint{5.462879in}{2.386983in}}{\pgfqpoint{5.462879in}{2.375933in}}%
\pgfpathcurveto{\pgfqpoint{5.462879in}{2.364883in}}{\pgfqpoint{5.467269in}{2.354284in}}{\pgfqpoint{5.475083in}{2.346470in}}%
\pgfpathcurveto{\pgfqpoint{5.482896in}{2.338657in}}{\pgfqpoint{5.493495in}{2.334266in}}{\pgfqpoint{5.504545in}{2.334266in}}%
\pgfpathclose%
\pgfusepath{stroke,fill}%
\end{pgfscope}%
\begin{pgfscope}%
\pgfpathrectangle{\pgfqpoint{0.800000in}{0.528000in}}{\pgfqpoint{4.960000in}{3.696000in}}%
\pgfusepath{clip}%
\pgfsetbuttcap%
\pgfsetroundjoin%
\definecolor{currentfill}{rgb}{0.000000,0.000000,0.000000}%
\pgfsetfillcolor{currentfill}%
\pgfsetlinewidth{1.003750pt}%
\definecolor{currentstroke}{rgb}{0.000000,0.000000,0.000000}%
\pgfsetstrokecolor{currentstroke}%
\pgfsetdash{}{0pt}%
\pgfpathmoveto{\pgfqpoint{5.504545in}{2.334266in}}%
\pgfpathcurveto{\pgfqpoint{5.515596in}{2.334266in}}{\pgfqpoint{5.526195in}{2.338657in}}{\pgfqpoint{5.534008in}{2.346470in}}%
\pgfpathcurveto{\pgfqpoint{5.541822in}{2.354284in}}{\pgfqpoint{5.546212in}{2.364883in}}{\pgfqpoint{5.546212in}{2.375933in}}%
\pgfpathcurveto{\pgfqpoint{5.546212in}{2.386983in}}{\pgfqpoint{5.541822in}{2.397582in}}{\pgfqpoint{5.534008in}{2.405396in}}%
\pgfpathcurveto{\pgfqpoint{5.526195in}{2.413209in}}{\pgfqpoint{5.515596in}{2.417600in}}{\pgfqpoint{5.504545in}{2.417600in}}%
\pgfpathcurveto{\pgfqpoint{5.493495in}{2.417600in}}{\pgfqpoint{5.482896in}{2.413209in}}{\pgfqpoint{5.475083in}{2.405396in}}%
\pgfpathcurveto{\pgfqpoint{5.467269in}{2.397582in}}{\pgfqpoint{5.462879in}{2.386983in}}{\pgfqpoint{5.462879in}{2.375933in}}%
\pgfpathcurveto{\pgfqpoint{5.462879in}{2.364883in}}{\pgfqpoint{5.467269in}{2.354284in}}{\pgfqpoint{5.475083in}{2.346470in}}%
\pgfpathcurveto{\pgfqpoint{5.482896in}{2.338657in}}{\pgfqpoint{5.493495in}{2.334266in}}{\pgfqpoint{5.504545in}{2.334266in}}%
\pgfpathclose%
\pgfusepath{stroke,fill}%
\end{pgfscope}%
\begin{pgfscope}%
\pgfpathrectangle{\pgfqpoint{0.800000in}{0.528000in}}{\pgfqpoint{4.960000in}{3.696000in}}%
\pgfusepath{clip}%
\pgfsetbuttcap%
\pgfsetroundjoin%
\definecolor{currentfill}{rgb}{0.000000,0.000000,0.000000}%
\pgfsetfillcolor{currentfill}%
\pgfsetlinewidth{1.003750pt}%
\definecolor{currentstroke}{rgb}{0.000000,0.000000,0.000000}%
\pgfsetstrokecolor{currentstroke}%
\pgfsetdash{}{0pt}%
\pgfpathmoveto{\pgfqpoint{5.504545in}{3.984333in}}%
\pgfpathcurveto{\pgfqpoint{5.515596in}{3.984333in}}{\pgfqpoint{5.526195in}{3.988724in}}{\pgfqpoint{5.534008in}{3.996537in}}%
\pgfpathcurveto{\pgfqpoint{5.541822in}{4.004351in}}{\pgfqpoint{5.546212in}{4.014950in}}{\pgfqpoint{5.546212in}{4.026000in}}%
\pgfpathcurveto{\pgfqpoint{5.546212in}{4.037050in}}{\pgfqpoint{5.541822in}{4.047649in}}{\pgfqpoint{5.534008in}{4.055463in}}%
\pgfpathcurveto{\pgfqpoint{5.526195in}{4.063276in}}{\pgfqpoint{5.515596in}{4.067667in}}{\pgfqpoint{5.504545in}{4.067667in}}%
\pgfpathcurveto{\pgfqpoint{5.493495in}{4.067667in}}{\pgfqpoint{5.482896in}{4.063276in}}{\pgfqpoint{5.475083in}{4.055463in}}%
\pgfpathcurveto{\pgfqpoint{5.467269in}{4.047649in}}{\pgfqpoint{5.462879in}{4.037050in}}{\pgfqpoint{5.462879in}{4.026000in}}%
\pgfpathcurveto{\pgfqpoint{5.462879in}{4.014950in}}{\pgfqpoint{5.467269in}{4.004351in}}{\pgfqpoint{5.475083in}{3.996537in}}%
\pgfpathcurveto{\pgfqpoint{5.482896in}{3.988724in}}{\pgfqpoint{5.493495in}{3.984333in}}{\pgfqpoint{5.504545in}{3.984333in}}%
\pgfpathclose%
\pgfusepath{stroke,fill}%
\end{pgfscope}%
\begin{pgfscope}%
\pgfpathrectangle{\pgfqpoint{0.800000in}{0.528000in}}{\pgfqpoint{4.960000in}{3.696000in}}%
\pgfusepath{clip}%
\pgfsetbuttcap%
\pgfsetroundjoin%
\definecolor{currentfill}{rgb}{0.000000,0.000000,0.000000}%
\pgfsetfillcolor{currentfill}%
\pgfsetlinewidth{1.003750pt}%
\definecolor{currentstroke}{rgb}{0.000000,0.000000,0.000000}%
\pgfsetstrokecolor{currentstroke}%
\pgfsetdash{}{0pt}%
\pgfpathmoveto{\pgfqpoint{5.504545in}{2.334266in}}%
\pgfpathcurveto{\pgfqpoint{5.515596in}{2.334266in}}{\pgfqpoint{5.526195in}{2.338657in}}{\pgfqpoint{5.534008in}{2.346470in}}%
\pgfpathcurveto{\pgfqpoint{5.541822in}{2.354284in}}{\pgfqpoint{5.546212in}{2.364883in}}{\pgfqpoint{5.546212in}{2.375933in}}%
\pgfpathcurveto{\pgfqpoint{5.546212in}{2.386983in}}{\pgfqpoint{5.541822in}{2.397582in}}{\pgfqpoint{5.534008in}{2.405396in}}%
\pgfpathcurveto{\pgfqpoint{5.526195in}{2.413209in}}{\pgfqpoint{5.515596in}{2.417600in}}{\pgfqpoint{5.504545in}{2.417600in}}%
\pgfpathcurveto{\pgfqpoint{5.493495in}{2.417600in}}{\pgfqpoint{5.482896in}{2.413209in}}{\pgfqpoint{5.475083in}{2.405396in}}%
\pgfpathcurveto{\pgfqpoint{5.467269in}{2.397582in}}{\pgfqpoint{5.462879in}{2.386983in}}{\pgfqpoint{5.462879in}{2.375933in}}%
\pgfpathcurveto{\pgfqpoint{5.462879in}{2.364883in}}{\pgfqpoint{5.467269in}{2.354284in}}{\pgfqpoint{5.475083in}{2.346470in}}%
\pgfpathcurveto{\pgfqpoint{5.482896in}{2.338657in}}{\pgfqpoint{5.493495in}{2.334266in}}{\pgfqpoint{5.504545in}{2.334266in}}%
\pgfpathclose%
\pgfusepath{stroke,fill}%
\end{pgfscope}%
\begin{pgfscope}%
\pgfpathrectangle{\pgfqpoint{0.800000in}{0.528000in}}{\pgfqpoint{4.960000in}{3.696000in}}%
\pgfusepath{clip}%
\pgfsetbuttcap%
\pgfsetroundjoin%
\definecolor{currentfill}{rgb}{0.000000,0.000000,0.000000}%
\pgfsetfillcolor{currentfill}%
\pgfsetlinewidth{1.003750pt}%
\definecolor{currentstroke}{rgb}{0.000000,0.000000,0.000000}%
\pgfsetstrokecolor{currentstroke}%
\pgfsetdash{}{0pt}%
\pgfpathmoveto{\pgfqpoint{5.504545in}{3.984333in}}%
\pgfpathcurveto{\pgfqpoint{5.515596in}{3.984333in}}{\pgfqpoint{5.526195in}{3.988724in}}{\pgfqpoint{5.534008in}{3.996537in}}%
\pgfpathcurveto{\pgfqpoint{5.541822in}{4.004351in}}{\pgfqpoint{5.546212in}{4.014950in}}{\pgfqpoint{5.546212in}{4.026000in}}%
\pgfpathcurveto{\pgfqpoint{5.546212in}{4.037050in}}{\pgfqpoint{5.541822in}{4.047649in}}{\pgfqpoint{5.534008in}{4.055463in}}%
\pgfpathcurveto{\pgfqpoint{5.526195in}{4.063276in}}{\pgfqpoint{5.515596in}{4.067667in}}{\pgfqpoint{5.504545in}{4.067667in}}%
\pgfpathcurveto{\pgfqpoint{5.493495in}{4.067667in}}{\pgfqpoint{5.482896in}{4.063276in}}{\pgfqpoint{5.475083in}{4.055463in}}%
\pgfpathcurveto{\pgfqpoint{5.467269in}{4.047649in}}{\pgfqpoint{5.462879in}{4.037050in}}{\pgfqpoint{5.462879in}{4.026000in}}%
\pgfpathcurveto{\pgfqpoint{5.462879in}{4.014950in}}{\pgfqpoint{5.467269in}{4.004351in}}{\pgfqpoint{5.475083in}{3.996537in}}%
\pgfpathcurveto{\pgfqpoint{5.482896in}{3.988724in}}{\pgfqpoint{5.493495in}{3.984333in}}{\pgfqpoint{5.504545in}{3.984333in}}%
\pgfpathclose%
\pgfusepath{stroke,fill}%
\end{pgfscope}%
\begin{pgfscope}%
\pgfpathrectangle{\pgfqpoint{0.800000in}{0.528000in}}{\pgfqpoint{4.960000in}{3.696000in}}%
\pgfusepath{clip}%
\pgfsetbuttcap%
\pgfsetroundjoin%
\definecolor{currentfill}{rgb}{0.000000,0.000000,0.000000}%
\pgfsetfillcolor{currentfill}%
\pgfsetlinewidth{1.003750pt}%
\definecolor{currentstroke}{rgb}{0.000000,0.000000,0.000000}%
\pgfsetstrokecolor{currentstroke}%
\pgfsetdash{}{0pt}%
\pgfpathmoveto{\pgfqpoint{5.504545in}{3.984333in}}%
\pgfpathcurveto{\pgfqpoint{5.515596in}{3.984333in}}{\pgfqpoint{5.526195in}{3.988724in}}{\pgfqpoint{5.534008in}{3.996537in}}%
\pgfpathcurveto{\pgfqpoint{5.541822in}{4.004351in}}{\pgfqpoint{5.546212in}{4.014950in}}{\pgfqpoint{5.546212in}{4.026000in}}%
\pgfpathcurveto{\pgfqpoint{5.546212in}{4.037050in}}{\pgfqpoint{5.541822in}{4.047649in}}{\pgfqpoint{5.534008in}{4.055463in}}%
\pgfpathcurveto{\pgfqpoint{5.526195in}{4.063276in}}{\pgfqpoint{5.515596in}{4.067667in}}{\pgfqpoint{5.504545in}{4.067667in}}%
\pgfpathcurveto{\pgfqpoint{5.493495in}{4.067667in}}{\pgfqpoint{5.482896in}{4.063276in}}{\pgfqpoint{5.475083in}{4.055463in}}%
\pgfpathcurveto{\pgfqpoint{5.467269in}{4.047649in}}{\pgfqpoint{5.462879in}{4.037050in}}{\pgfqpoint{5.462879in}{4.026000in}}%
\pgfpathcurveto{\pgfqpoint{5.462879in}{4.014950in}}{\pgfqpoint{5.467269in}{4.004351in}}{\pgfqpoint{5.475083in}{3.996537in}}%
\pgfpathcurveto{\pgfqpoint{5.482896in}{3.988724in}}{\pgfqpoint{5.493495in}{3.984333in}}{\pgfqpoint{5.504545in}{3.984333in}}%
\pgfpathclose%
\pgfusepath{stroke,fill}%
\end{pgfscope}%
\begin{pgfscope}%
\pgfpathrectangle{\pgfqpoint{0.800000in}{0.528000in}}{\pgfqpoint{4.960000in}{3.696000in}}%
\pgfusepath{clip}%
\pgfsetbuttcap%
\pgfsetroundjoin%
\definecolor{currentfill}{rgb}{0.000000,0.000000,0.000000}%
\pgfsetfillcolor{currentfill}%
\pgfsetlinewidth{1.003750pt}%
\definecolor{currentstroke}{rgb}{0.000000,0.000000,0.000000}%
\pgfsetstrokecolor{currentstroke}%
\pgfsetdash{}{0pt}%
\pgfpathmoveto{\pgfqpoint{5.504545in}{2.334266in}}%
\pgfpathcurveto{\pgfqpoint{5.515596in}{2.334266in}}{\pgfqpoint{5.526195in}{2.338657in}}{\pgfqpoint{5.534008in}{2.346470in}}%
\pgfpathcurveto{\pgfqpoint{5.541822in}{2.354284in}}{\pgfqpoint{5.546212in}{2.364883in}}{\pgfqpoint{5.546212in}{2.375933in}}%
\pgfpathcurveto{\pgfqpoint{5.546212in}{2.386983in}}{\pgfqpoint{5.541822in}{2.397582in}}{\pgfqpoint{5.534008in}{2.405396in}}%
\pgfpathcurveto{\pgfqpoint{5.526195in}{2.413209in}}{\pgfqpoint{5.515596in}{2.417600in}}{\pgfqpoint{5.504545in}{2.417600in}}%
\pgfpathcurveto{\pgfqpoint{5.493495in}{2.417600in}}{\pgfqpoint{5.482896in}{2.413209in}}{\pgfqpoint{5.475083in}{2.405396in}}%
\pgfpathcurveto{\pgfqpoint{5.467269in}{2.397582in}}{\pgfqpoint{5.462879in}{2.386983in}}{\pgfqpoint{5.462879in}{2.375933in}}%
\pgfpathcurveto{\pgfqpoint{5.462879in}{2.364883in}}{\pgfqpoint{5.467269in}{2.354284in}}{\pgfqpoint{5.475083in}{2.346470in}}%
\pgfpathcurveto{\pgfqpoint{5.482896in}{2.338657in}}{\pgfqpoint{5.493495in}{2.334266in}}{\pgfqpoint{5.504545in}{2.334266in}}%
\pgfpathclose%
\pgfusepath{stroke,fill}%
\end{pgfscope}%
\begin{pgfscope}%
\pgfpathrectangle{\pgfqpoint{0.800000in}{0.528000in}}{\pgfqpoint{4.960000in}{3.696000in}}%
\pgfusepath{clip}%
\pgfsetbuttcap%
\pgfsetroundjoin%
\definecolor{currentfill}{rgb}{0.000000,0.000000,0.000000}%
\pgfsetfillcolor{currentfill}%
\pgfsetlinewidth{1.003750pt}%
\definecolor{currentstroke}{rgb}{0.000000,0.000000,0.000000}%
\pgfsetstrokecolor{currentstroke}%
\pgfsetdash{}{0pt}%
\pgfpathmoveto{\pgfqpoint{5.504545in}{2.334266in}}%
\pgfpathcurveto{\pgfqpoint{5.515596in}{2.334266in}}{\pgfqpoint{5.526195in}{2.338657in}}{\pgfqpoint{5.534008in}{2.346470in}}%
\pgfpathcurveto{\pgfqpoint{5.541822in}{2.354284in}}{\pgfqpoint{5.546212in}{2.364883in}}{\pgfqpoint{5.546212in}{2.375933in}}%
\pgfpathcurveto{\pgfqpoint{5.546212in}{2.386983in}}{\pgfqpoint{5.541822in}{2.397582in}}{\pgfqpoint{5.534008in}{2.405396in}}%
\pgfpathcurveto{\pgfqpoint{5.526195in}{2.413209in}}{\pgfqpoint{5.515596in}{2.417600in}}{\pgfqpoint{5.504545in}{2.417600in}}%
\pgfpathcurveto{\pgfqpoint{5.493495in}{2.417600in}}{\pgfqpoint{5.482896in}{2.413209in}}{\pgfqpoint{5.475083in}{2.405396in}}%
\pgfpathcurveto{\pgfqpoint{5.467269in}{2.397582in}}{\pgfqpoint{5.462879in}{2.386983in}}{\pgfqpoint{5.462879in}{2.375933in}}%
\pgfpathcurveto{\pgfqpoint{5.462879in}{2.364883in}}{\pgfqpoint{5.467269in}{2.354284in}}{\pgfqpoint{5.475083in}{2.346470in}}%
\pgfpathcurveto{\pgfqpoint{5.482896in}{2.338657in}}{\pgfqpoint{5.493495in}{2.334266in}}{\pgfqpoint{5.504545in}{2.334266in}}%
\pgfpathclose%
\pgfusepath{stroke,fill}%
\end{pgfscope}%
\begin{pgfscope}%
\pgfpathrectangle{\pgfqpoint{0.800000in}{0.528000in}}{\pgfqpoint{4.960000in}{3.696000in}}%
\pgfusepath{clip}%
\pgfsetbuttcap%
\pgfsetroundjoin%
\definecolor{currentfill}{rgb}{0.000000,0.000000,0.000000}%
\pgfsetfillcolor{currentfill}%
\pgfsetlinewidth{1.003750pt}%
\definecolor{currentstroke}{rgb}{0.000000,0.000000,0.000000}%
\pgfsetstrokecolor{currentstroke}%
\pgfsetdash{}{0pt}%
\pgfpathmoveto{\pgfqpoint{5.504545in}{2.334266in}}%
\pgfpathcurveto{\pgfqpoint{5.515596in}{2.334266in}}{\pgfqpoint{5.526195in}{2.338657in}}{\pgfqpoint{5.534008in}{2.346470in}}%
\pgfpathcurveto{\pgfqpoint{5.541822in}{2.354284in}}{\pgfqpoint{5.546212in}{2.364883in}}{\pgfqpoint{5.546212in}{2.375933in}}%
\pgfpathcurveto{\pgfqpoint{5.546212in}{2.386983in}}{\pgfqpoint{5.541822in}{2.397582in}}{\pgfqpoint{5.534008in}{2.405396in}}%
\pgfpathcurveto{\pgfqpoint{5.526195in}{2.413209in}}{\pgfqpoint{5.515596in}{2.417600in}}{\pgfqpoint{5.504545in}{2.417600in}}%
\pgfpathcurveto{\pgfqpoint{5.493495in}{2.417600in}}{\pgfqpoint{5.482896in}{2.413209in}}{\pgfqpoint{5.475083in}{2.405396in}}%
\pgfpathcurveto{\pgfqpoint{5.467269in}{2.397582in}}{\pgfqpoint{5.462879in}{2.386983in}}{\pgfqpoint{5.462879in}{2.375933in}}%
\pgfpathcurveto{\pgfqpoint{5.462879in}{2.364883in}}{\pgfqpoint{5.467269in}{2.354284in}}{\pgfqpoint{5.475083in}{2.346470in}}%
\pgfpathcurveto{\pgfqpoint{5.482896in}{2.338657in}}{\pgfqpoint{5.493495in}{2.334266in}}{\pgfqpoint{5.504545in}{2.334266in}}%
\pgfpathclose%
\pgfusepath{stroke,fill}%
\end{pgfscope}%
\begin{pgfscope}%
\pgfpathrectangle{\pgfqpoint{0.800000in}{0.528000in}}{\pgfqpoint{4.960000in}{3.696000in}}%
\pgfusepath{clip}%
\pgfsetbuttcap%
\pgfsetroundjoin%
\definecolor{currentfill}{rgb}{0.000000,0.000000,0.000000}%
\pgfsetfillcolor{currentfill}%
\pgfsetlinewidth{1.003750pt}%
\definecolor{currentstroke}{rgb}{0.000000,0.000000,0.000000}%
\pgfsetstrokecolor{currentstroke}%
\pgfsetdash{}{0pt}%
\pgfpathmoveto{\pgfqpoint{5.504545in}{2.334266in}}%
\pgfpathcurveto{\pgfqpoint{5.515596in}{2.334266in}}{\pgfqpoint{5.526195in}{2.338657in}}{\pgfqpoint{5.534008in}{2.346470in}}%
\pgfpathcurveto{\pgfqpoint{5.541822in}{2.354284in}}{\pgfqpoint{5.546212in}{2.364883in}}{\pgfqpoint{5.546212in}{2.375933in}}%
\pgfpathcurveto{\pgfqpoint{5.546212in}{2.386983in}}{\pgfqpoint{5.541822in}{2.397582in}}{\pgfqpoint{5.534008in}{2.405396in}}%
\pgfpathcurveto{\pgfqpoint{5.526195in}{2.413209in}}{\pgfqpoint{5.515596in}{2.417600in}}{\pgfqpoint{5.504545in}{2.417600in}}%
\pgfpathcurveto{\pgfqpoint{5.493495in}{2.417600in}}{\pgfqpoint{5.482896in}{2.413209in}}{\pgfqpoint{5.475083in}{2.405396in}}%
\pgfpathcurveto{\pgfqpoint{5.467269in}{2.397582in}}{\pgfqpoint{5.462879in}{2.386983in}}{\pgfqpoint{5.462879in}{2.375933in}}%
\pgfpathcurveto{\pgfqpoint{5.462879in}{2.364883in}}{\pgfqpoint{5.467269in}{2.354284in}}{\pgfqpoint{5.475083in}{2.346470in}}%
\pgfpathcurveto{\pgfqpoint{5.482896in}{2.338657in}}{\pgfqpoint{5.493495in}{2.334266in}}{\pgfqpoint{5.504545in}{2.334266in}}%
\pgfpathclose%
\pgfusepath{stroke,fill}%
\end{pgfscope}%
\begin{pgfscope}%
\pgfpathrectangle{\pgfqpoint{0.800000in}{0.528000in}}{\pgfqpoint{4.960000in}{3.696000in}}%
\pgfusepath{clip}%
\pgfsetbuttcap%
\pgfsetroundjoin%
\definecolor{currentfill}{rgb}{0.000000,0.000000,0.000000}%
\pgfsetfillcolor{currentfill}%
\pgfsetlinewidth{1.003750pt}%
\definecolor{currentstroke}{rgb}{0.000000,0.000000,0.000000}%
\pgfsetstrokecolor{currentstroke}%
\pgfsetdash{}{0pt}%
\pgfpathmoveto{\pgfqpoint{5.504545in}{3.984333in}}%
\pgfpathcurveto{\pgfqpoint{5.515596in}{3.984333in}}{\pgfqpoint{5.526195in}{3.988724in}}{\pgfqpoint{5.534008in}{3.996537in}}%
\pgfpathcurveto{\pgfqpoint{5.541822in}{4.004351in}}{\pgfqpoint{5.546212in}{4.014950in}}{\pgfqpoint{5.546212in}{4.026000in}}%
\pgfpathcurveto{\pgfqpoint{5.546212in}{4.037050in}}{\pgfqpoint{5.541822in}{4.047649in}}{\pgfqpoint{5.534008in}{4.055463in}}%
\pgfpathcurveto{\pgfqpoint{5.526195in}{4.063276in}}{\pgfqpoint{5.515596in}{4.067667in}}{\pgfqpoint{5.504545in}{4.067667in}}%
\pgfpathcurveto{\pgfqpoint{5.493495in}{4.067667in}}{\pgfqpoint{5.482896in}{4.063276in}}{\pgfqpoint{5.475083in}{4.055463in}}%
\pgfpathcurveto{\pgfqpoint{5.467269in}{4.047649in}}{\pgfqpoint{5.462879in}{4.037050in}}{\pgfqpoint{5.462879in}{4.026000in}}%
\pgfpathcurveto{\pgfqpoint{5.462879in}{4.014950in}}{\pgfqpoint{5.467269in}{4.004351in}}{\pgfqpoint{5.475083in}{3.996537in}}%
\pgfpathcurveto{\pgfqpoint{5.482896in}{3.988724in}}{\pgfqpoint{5.493495in}{3.984333in}}{\pgfqpoint{5.504545in}{3.984333in}}%
\pgfpathclose%
\pgfusepath{stroke,fill}%
\end{pgfscope}%
\begin{pgfscope}%
\pgfpathrectangle{\pgfqpoint{0.800000in}{0.528000in}}{\pgfqpoint{4.960000in}{3.696000in}}%
\pgfusepath{clip}%
\pgfsetbuttcap%
\pgfsetroundjoin%
\definecolor{currentfill}{rgb}{0.000000,0.000000,0.000000}%
\pgfsetfillcolor{currentfill}%
\pgfsetlinewidth{1.003750pt}%
\definecolor{currentstroke}{rgb}{0.000000,0.000000,0.000000}%
\pgfsetstrokecolor{currentstroke}%
\pgfsetdash{}{0pt}%
\pgfpathmoveto{\pgfqpoint{5.504545in}{3.984333in}}%
\pgfpathcurveto{\pgfqpoint{5.515596in}{3.984333in}}{\pgfqpoint{5.526195in}{3.988724in}}{\pgfqpoint{5.534008in}{3.996537in}}%
\pgfpathcurveto{\pgfqpoint{5.541822in}{4.004351in}}{\pgfqpoint{5.546212in}{4.014950in}}{\pgfqpoint{5.546212in}{4.026000in}}%
\pgfpathcurveto{\pgfqpoint{5.546212in}{4.037050in}}{\pgfqpoint{5.541822in}{4.047649in}}{\pgfqpoint{5.534008in}{4.055463in}}%
\pgfpathcurveto{\pgfqpoint{5.526195in}{4.063276in}}{\pgfqpoint{5.515596in}{4.067667in}}{\pgfqpoint{5.504545in}{4.067667in}}%
\pgfpathcurveto{\pgfqpoint{5.493495in}{4.067667in}}{\pgfqpoint{5.482896in}{4.063276in}}{\pgfqpoint{5.475083in}{4.055463in}}%
\pgfpathcurveto{\pgfqpoint{5.467269in}{4.047649in}}{\pgfqpoint{5.462879in}{4.037050in}}{\pgfqpoint{5.462879in}{4.026000in}}%
\pgfpathcurveto{\pgfqpoint{5.462879in}{4.014950in}}{\pgfqpoint{5.467269in}{4.004351in}}{\pgfqpoint{5.475083in}{3.996537in}}%
\pgfpathcurveto{\pgfqpoint{5.482896in}{3.988724in}}{\pgfqpoint{5.493495in}{3.984333in}}{\pgfqpoint{5.504545in}{3.984333in}}%
\pgfpathclose%
\pgfusepath{stroke,fill}%
\end{pgfscope}%
\begin{pgfscope}%
\pgfpathrectangle{\pgfqpoint{0.800000in}{0.528000in}}{\pgfqpoint{4.960000in}{3.696000in}}%
\pgfusepath{clip}%
\pgfsetbuttcap%
\pgfsetroundjoin%
\definecolor{currentfill}{rgb}{0.000000,0.000000,0.000000}%
\pgfsetfillcolor{currentfill}%
\pgfsetlinewidth{1.003750pt}%
\definecolor{currentstroke}{rgb}{0.000000,0.000000,0.000000}%
\pgfsetstrokecolor{currentstroke}%
\pgfsetdash{}{0pt}%
\pgfpathmoveto{\pgfqpoint{5.504545in}{3.984333in}}%
\pgfpathcurveto{\pgfqpoint{5.515596in}{3.984333in}}{\pgfqpoint{5.526195in}{3.988724in}}{\pgfqpoint{5.534008in}{3.996537in}}%
\pgfpathcurveto{\pgfqpoint{5.541822in}{4.004351in}}{\pgfqpoint{5.546212in}{4.014950in}}{\pgfqpoint{5.546212in}{4.026000in}}%
\pgfpathcurveto{\pgfqpoint{5.546212in}{4.037050in}}{\pgfqpoint{5.541822in}{4.047649in}}{\pgfqpoint{5.534008in}{4.055463in}}%
\pgfpathcurveto{\pgfqpoint{5.526195in}{4.063276in}}{\pgfqpoint{5.515596in}{4.067667in}}{\pgfqpoint{5.504545in}{4.067667in}}%
\pgfpathcurveto{\pgfqpoint{5.493495in}{4.067667in}}{\pgfqpoint{5.482896in}{4.063276in}}{\pgfqpoint{5.475083in}{4.055463in}}%
\pgfpathcurveto{\pgfqpoint{5.467269in}{4.047649in}}{\pgfqpoint{5.462879in}{4.037050in}}{\pgfqpoint{5.462879in}{4.026000in}}%
\pgfpathcurveto{\pgfqpoint{5.462879in}{4.014950in}}{\pgfqpoint{5.467269in}{4.004351in}}{\pgfqpoint{5.475083in}{3.996537in}}%
\pgfpathcurveto{\pgfqpoint{5.482896in}{3.988724in}}{\pgfqpoint{5.493495in}{3.984333in}}{\pgfqpoint{5.504545in}{3.984333in}}%
\pgfpathclose%
\pgfusepath{stroke,fill}%
\end{pgfscope}%
\begin{pgfscope}%
\pgfpathrectangle{\pgfqpoint{0.800000in}{0.528000in}}{\pgfqpoint{4.960000in}{3.696000in}}%
\pgfusepath{clip}%
\pgfsetbuttcap%
\pgfsetroundjoin%
\definecolor{currentfill}{rgb}{0.000000,0.000000,0.000000}%
\pgfsetfillcolor{currentfill}%
\pgfsetlinewidth{1.003750pt}%
\definecolor{currentstroke}{rgb}{0.000000,0.000000,0.000000}%
\pgfsetstrokecolor{currentstroke}%
\pgfsetdash{}{0pt}%
\pgfpathmoveto{\pgfqpoint{5.504545in}{3.984333in}}%
\pgfpathcurveto{\pgfqpoint{5.515596in}{3.984333in}}{\pgfqpoint{5.526195in}{3.988724in}}{\pgfqpoint{5.534008in}{3.996537in}}%
\pgfpathcurveto{\pgfqpoint{5.541822in}{4.004351in}}{\pgfqpoint{5.546212in}{4.014950in}}{\pgfqpoint{5.546212in}{4.026000in}}%
\pgfpathcurveto{\pgfqpoint{5.546212in}{4.037050in}}{\pgfqpoint{5.541822in}{4.047649in}}{\pgfqpoint{5.534008in}{4.055463in}}%
\pgfpathcurveto{\pgfqpoint{5.526195in}{4.063276in}}{\pgfqpoint{5.515596in}{4.067667in}}{\pgfqpoint{5.504545in}{4.067667in}}%
\pgfpathcurveto{\pgfqpoint{5.493495in}{4.067667in}}{\pgfqpoint{5.482896in}{4.063276in}}{\pgfqpoint{5.475083in}{4.055463in}}%
\pgfpathcurveto{\pgfqpoint{5.467269in}{4.047649in}}{\pgfqpoint{5.462879in}{4.037050in}}{\pgfqpoint{5.462879in}{4.026000in}}%
\pgfpathcurveto{\pgfqpoint{5.462879in}{4.014950in}}{\pgfqpoint{5.467269in}{4.004351in}}{\pgfqpoint{5.475083in}{3.996537in}}%
\pgfpathcurveto{\pgfqpoint{5.482896in}{3.988724in}}{\pgfqpoint{5.493495in}{3.984333in}}{\pgfqpoint{5.504545in}{3.984333in}}%
\pgfpathclose%
\pgfusepath{stroke,fill}%
\end{pgfscope}%
\begin{pgfscope}%
\pgfpathrectangle{\pgfqpoint{0.800000in}{0.528000in}}{\pgfqpoint{4.960000in}{3.696000in}}%
\pgfusepath{clip}%
\pgfsetbuttcap%
\pgfsetroundjoin%
\definecolor{currentfill}{rgb}{0.000000,0.000000,0.000000}%
\pgfsetfillcolor{currentfill}%
\pgfsetlinewidth{1.003750pt}%
\definecolor{currentstroke}{rgb}{0.000000,0.000000,0.000000}%
\pgfsetstrokecolor{currentstroke}%
\pgfsetdash{}{0pt}%
\pgfpathmoveto{\pgfqpoint{5.504545in}{3.984333in}}%
\pgfpathcurveto{\pgfqpoint{5.515596in}{3.984333in}}{\pgfqpoint{5.526195in}{3.988724in}}{\pgfqpoint{5.534008in}{3.996537in}}%
\pgfpathcurveto{\pgfqpoint{5.541822in}{4.004351in}}{\pgfqpoint{5.546212in}{4.014950in}}{\pgfqpoint{5.546212in}{4.026000in}}%
\pgfpathcurveto{\pgfqpoint{5.546212in}{4.037050in}}{\pgfqpoint{5.541822in}{4.047649in}}{\pgfqpoint{5.534008in}{4.055463in}}%
\pgfpathcurveto{\pgfqpoint{5.526195in}{4.063276in}}{\pgfqpoint{5.515596in}{4.067667in}}{\pgfqpoint{5.504545in}{4.067667in}}%
\pgfpathcurveto{\pgfqpoint{5.493495in}{4.067667in}}{\pgfqpoint{5.482896in}{4.063276in}}{\pgfqpoint{5.475083in}{4.055463in}}%
\pgfpathcurveto{\pgfqpoint{5.467269in}{4.047649in}}{\pgfqpoint{5.462879in}{4.037050in}}{\pgfqpoint{5.462879in}{4.026000in}}%
\pgfpathcurveto{\pgfqpoint{5.462879in}{4.014950in}}{\pgfqpoint{5.467269in}{4.004351in}}{\pgfqpoint{5.475083in}{3.996537in}}%
\pgfpathcurveto{\pgfqpoint{5.482896in}{3.988724in}}{\pgfqpoint{5.493495in}{3.984333in}}{\pgfqpoint{5.504545in}{3.984333in}}%
\pgfpathclose%
\pgfusepath{stroke,fill}%
\end{pgfscope}%
\begin{pgfscope}%
\pgfpathrectangle{\pgfqpoint{0.800000in}{0.528000in}}{\pgfqpoint{4.960000in}{3.696000in}}%
\pgfusepath{clip}%
\pgfsetbuttcap%
\pgfsetroundjoin%
\definecolor{currentfill}{rgb}{0.000000,0.000000,0.000000}%
\pgfsetfillcolor{currentfill}%
\pgfsetlinewidth{1.003750pt}%
\definecolor{currentstroke}{rgb}{0.000000,0.000000,0.000000}%
\pgfsetstrokecolor{currentstroke}%
\pgfsetdash{}{0pt}%
\pgfpathmoveto{\pgfqpoint{5.504545in}{3.984333in}}%
\pgfpathcurveto{\pgfqpoint{5.515596in}{3.984333in}}{\pgfqpoint{5.526195in}{3.988724in}}{\pgfqpoint{5.534008in}{3.996537in}}%
\pgfpathcurveto{\pgfqpoint{5.541822in}{4.004351in}}{\pgfqpoint{5.546212in}{4.014950in}}{\pgfqpoint{5.546212in}{4.026000in}}%
\pgfpathcurveto{\pgfqpoint{5.546212in}{4.037050in}}{\pgfqpoint{5.541822in}{4.047649in}}{\pgfqpoint{5.534008in}{4.055463in}}%
\pgfpathcurveto{\pgfqpoint{5.526195in}{4.063276in}}{\pgfqpoint{5.515596in}{4.067667in}}{\pgfqpoint{5.504545in}{4.067667in}}%
\pgfpathcurveto{\pgfqpoint{5.493495in}{4.067667in}}{\pgfqpoint{5.482896in}{4.063276in}}{\pgfqpoint{5.475083in}{4.055463in}}%
\pgfpathcurveto{\pgfqpoint{5.467269in}{4.047649in}}{\pgfqpoint{5.462879in}{4.037050in}}{\pgfqpoint{5.462879in}{4.026000in}}%
\pgfpathcurveto{\pgfqpoint{5.462879in}{4.014950in}}{\pgfqpoint{5.467269in}{4.004351in}}{\pgfqpoint{5.475083in}{3.996537in}}%
\pgfpathcurveto{\pgfqpoint{5.482896in}{3.988724in}}{\pgfqpoint{5.493495in}{3.984333in}}{\pgfqpoint{5.504545in}{3.984333in}}%
\pgfpathclose%
\pgfusepath{stroke,fill}%
\end{pgfscope}%
\begin{pgfscope}%
\pgfpathrectangle{\pgfqpoint{0.800000in}{0.528000in}}{\pgfqpoint{4.960000in}{3.696000in}}%
\pgfusepath{clip}%
\pgfsetbuttcap%
\pgfsetroundjoin%
\definecolor{currentfill}{rgb}{0.000000,0.000000,0.000000}%
\pgfsetfillcolor{currentfill}%
\pgfsetlinewidth{1.003750pt}%
\definecolor{currentstroke}{rgb}{0.000000,0.000000,0.000000}%
\pgfsetstrokecolor{currentstroke}%
\pgfsetdash{}{0pt}%
\pgfpathmoveto{\pgfqpoint{5.504545in}{2.334266in}}%
\pgfpathcurveto{\pgfqpoint{5.515596in}{2.334266in}}{\pgfqpoint{5.526195in}{2.338657in}}{\pgfqpoint{5.534008in}{2.346470in}}%
\pgfpathcurveto{\pgfqpoint{5.541822in}{2.354284in}}{\pgfqpoint{5.546212in}{2.364883in}}{\pgfqpoint{5.546212in}{2.375933in}}%
\pgfpathcurveto{\pgfqpoint{5.546212in}{2.386983in}}{\pgfqpoint{5.541822in}{2.397582in}}{\pgfqpoint{5.534008in}{2.405396in}}%
\pgfpathcurveto{\pgfqpoint{5.526195in}{2.413209in}}{\pgfqpoint{5.515596in}{2.417600in}}{\pgfqpoint{5.504545in}{2.417600in}}%
\pgfpathcurveto{\pgfqpoint{5.493495in}{2.417600in}}{\pgfqpoint{5.482896in}{2.413209in}}{\pgfqpoint{5.475083in}{2.405396in}}%
\pgfpathcurveto{\pgfqpoint{5.467269in}{2.397582in}}{\pgfqpoint{5.462879in}{2.386983in}}{\pgfqpoint{5.462879in}{2.375933in}}%
\pgfpathcurveto{\pgfqpoint{5.462879in}{2.364883in}}{\pgfqpoint{5.467269in}{2.354284in}}{\pgfqpoint{5.475083in}{2.346470in}}%
\pgfpathcurveto{\pgfqpoint{5.482896in}{2.338657in}}{\pgfqpoint{5.493495in}{2.334266in}}{\pgfqpoint{5.504545in}{2.334266in}}%
\pgfpathclose%
\pgfusepath{stroke,fill}%
\end{pgfscope}%
\begin{pgfscope}%
\pgfpathrectangle{\pgfqpoint{0.800000in}{0.528000in}}{\pgfqpoint{4.960000in}{3.696000in}}%
\pgfusepath{clip}%
\pgfsetbuttcap%
\pgfsetroundjoin%
\definecolor{currentfill}{rgb}{0.000000,0.000000,0.000000}%
\pgfsetfillcolor{currentfill}%
\pgfsetlinewidth{1.003750pt}%
\definecolor{currentstroke}{rgb}{0.000000,0.000000,0.000000}%
\pgfsetstrokecolor{currentstroke}%
\pgfsetdash{}{0pt}%
\pgfpathmoveto{\pgfqpoint{5.504545in}{3.984333in}}%
\pgfpathcurveto{\pgfqpoint{5.515596in}{3.984333in}}{\pgfqpoint{5.526195in}{3.988724in}}{\pgfqpoint{5.534008in}{3.996537in}}%
\pgfpathcurveto{\pgfqpoint{5.541822in}{4.004351in}}{\pgfqpoint{5.546212in}{4.014950in}}{\pgfqpoint{5.546212in}{4.026000in}}%
\pgfpathcurveto{\pgfqpoint{5.546212in}{4.037050in}}{\pgfqpoint{5.541822in}{4.047649in}}{\pgfqpoint{5.534008in}{4.055463in}}%
\pgfpathcurveto{\pgfqpoint{5.526195in}{4.063276in}}{\pgfqpoint{5.515596in}{4.067667in}}{\pgfqpoint{5.504545in}{4.067667in}}%
\pgfpathcurveto{\pgfqpoint{5.493495in}{4.067667in}}{\pgfqpoint{5.482896in}{4.063276in}}{\pgfqpoint{5.475083in}{4.055463in}}%
\pgfpathcurveto{\pgfqpoint{5.467269in}{4.047649in}}{\pgfqpoint{5.462879in}{4.037050in}}{\pgfqpoint{5.462879in}{4.026000in}}%
\pgfpathcurveto{\pgfqpoint{5.462879in}{4.014950in}}{\pgfqpoint{5.467269in}{4.004351in}}{\pgfqpoint{5.475083in}{3.996537in}}%
\pgfpathcurveto{\pgfqpoint{5.482896in}{3.988724in}}{\pgfqpoint{5.493495in}{3.984333in}}{\pgfqpoint{5.504545in}{3.984333in}}%
\pgfpathclose%
\pgfusepath{stroke,fill}%
\end{pgfscope}%
\begin{pgfscope}%
\pgfpathrectangle{\pgfqpoint{0.800000in}{0.528000in}}{\pgfqpoint{4.960000in}{3.696000in}}%
\pgfusepath{clip}%
\pgfsetbuttcap%
\pgfsetroundjoin%
\definecolor{currentfill}{rgb}{0.000000,0.000000,0.000000}%
\pgfsetfillcolor{currentfill}%
\pgfsetlinewidth{1.003750pt}%
\definecolor{currentstroke}{rgb}{0.000000,0.000000,0.000000}%
\pgfsetstrokecolor{currentstroke}%
\pgfsetdash{}{0pt}%
\pgfpathmoveto{\pgfqpoint{5.504545in}{2.334266in}}%
\pgfpathcurveto{\pgfqpoint{5.515596in}{2.334266in}}{\pgfqpoint{5.526195in}{2.338657in}}{\pgfqpoint{5.534008in}{2.346470in}}%
\pgfpathcurveto{\pgfqpoint{5.541822in}{2.354284in}}{\pgfqpoint{5.546212in}{2.364883in}}{\pgfqpoint{5.546212in}{2.375933in}}%
\pgfpathcurveto{\pgfqpoint{5.546212in}{2.386983in}}{\pgfqpoint{5.541822in}{2.397582in}}{\pgfqpoint{5.534008in}{2.405396in}}%
\pgfpathcurveto{\pgfqpoint{5.526195in}{2.413209in}}{\pgfqpoint{5.515596in}{2.417600in}}{\pgfqpoint{5.504545in}{2.417600in}}%
\pgfpathcurveto{\pgfqpoint{5.493495in}{2.417600in}}{\pgfqpoint{5.482896in}{2.413209in}}{\pgfqpoint{5.475083in}{2.405396in}}%
\pgfpathcurveto{\pgfqpoint{5.467269in}{2.397582in}}{\pgfqpoint{5.462879in}{2.386983in}}{\pgfqpoint{5.462879in}{2.375933in}}%
\pgfpathcurveto{\pgfqpoint{5.462879in}{2.364883in}}{\pgfqpoint{5.467269in}{2.354284in}}{\pgfqpoint{5.475083in}{2.346470in}}%
\pgfpathcurveto{\pgfqpoint{5.482896in}{2.338657in}}{\pgfqpoint{5.493495in}{2.334266in}}{\pgfqpoint{5.504545in}{2.334266in}}%
\pgfpathclose%
\pgfusepath{stroke,fill}%
\end{pgfscope}%
\begin{pgfscope}%
\pgfpathrectangle{\pgfqpoint{0.800000in}{0.528000in}}{\pgfqpoint{4.960000in}{3.696000in}}%
\pgfusepath{clip}%
\pgfsetbuttcap%
\pgfsetroundjoin%
\definecolor{currentfill}{rgb}{0.000000,0.000000,0.000000}%
\pgfsetfillcolor{currentfill}%
\pgfsetlinewidth{1.003750pt}%
\definecolor{currentstroke}{rgb}{0.000000,0.000000,0.000000}%
\pgfsetstrokecolor{currentstroke}%
\pgfsetdash{}{0pt}%
\pgfpathmoveto{\pgfqpoint{5.504545in}{2.334266in}}%
\pgfpathcurveto{\pgfqpoint{5.515596in}{2.334266in}}{\pgfqpoint{5.526195in}{2.338657in}}{\pgfqpoint{5.534008in}{2.346470in}}%
\pgfpathcurveto{\pgfqpoint{5.541822in}{2.354284in}}{\pgfqpoint{5.546212in}{2.364883in}}{\pgfqpoint{5.546212in}{2.375933in}}%
\pgfpathcurveto{\pgfqpoint{5.546212in}{2.386983in}}{\pgfqpoint{5.541822in}{2.397582in}}{\pgfqpoint{5.534008in}{2.405396in}}%
\pgfpathcurveto{\pgfqpoint{5.526195in}{2.413209in}}{\pgfqpoint{5.515596in}{2.417600in}}{\pgfqpoint{5.504545in}{2.417600in}}%
\pgfpathcurveto{\pgfqpoint{5.493495in}{2.417600in}}{\pgfqpoint{5.482896in}{2.413209in}}{\pgfqpoint{5.475083in}{2.405396in}}%
\pgfpathcurveto{\pgfqpoint{5.467269in}{2.397582in}}{\pgfqpoint{5.462879in}{2.386983in}}{\pgfqpoint{5.462879in}{2.375933in}}%
\pgfpathcurveto{\pgfqpoint{5.462879in}{2.364883in}}{\pgfqpoint{5.467269in}{2.354284in}}{\pgfqpoint{5.475083in}{2.346470in}}%
\pgfpathcurveto{\pgfqpoint{5.482896in}{2.338657in}}{\pgfqpoint{5.493495in}{2.334266in}}{\pgfqpoint{5.504545in}{2.334266in}}%
\pgfpathclose%
\pgfusepath{stroke,fill}%
\end{pgfscope}%
\begin{pgfscope}%
\pgfpathrectangle{\pgfqpoint{0.800000in}{0.528000in}}{\pgfqpoint{4.960000in}{3.696000in}}%
\pgfusepath{clip}%
\pgfsetbuttcap%
\pgfsetroundjoin%
\definecolor{currentfill}{rgb}{0.000000,0.000000,0.000000}%
\pgfsetfillcolor{currentfill}%
\pgfsetlinewidth{1.003750pt}%
\definecolor{currentstroke}{rgb}{0.000000,0.000000,0.000000}%
\pgfsetstrokecolor{currentstroke}%
\pgfsetdash{}{0pt}%
\pgfpathmoveto{\pgfqpoint{5.504545in}{3.984333in}}%
\pgfpathcurveto{\pgfqpoint{5.515596in}{3.984333in}}{\pgfqpoint{5.526195in}{3.988724in}}{\pgfqpoint{5.534008in}{3.996537in}}%
\pgfpathcurveto{\pgfqpoint{5.541822in}{4.004351in}}{\pgfqpoint{5.546212in}{4.014950in}}{\pgfqpoint{5.546212in}{4.026000in}}%
\pgfpathcurveto{\pgfqpoint{5.546212in}{4.037050in}}{\pgfqpoint{5.541822in}{4.047649in}}{\pgfqpoint{5.534008in}{4.055463in}}%
\pgfpathcurveto{\pgfqpoint{5.526195in}{4.063276in}}{\pgfqpoint{5.515596in}{4.067667in}}{\pgfqpoint{5.504545in}{4.067667in}}%
\pgfpathcurveto{\pgfqpoint{5.493495in}{4.067667in}}{\pgfqpoint{5.482896in}{4.063276in}}{\pgfqpoint{5.475083in}{4.055463in}}%
\pgfpathcurveto{\pgfqpoint{5.467269in}{4.047649in}}{\pgfqpoint{5.462879in}{4.037050in}}{\pgfqpoint{5.462879in}{4.026000in}}%
\pgfpathcurveto{\pgfqpoint{5.462879in}{4.014950in}}{\pgfqpoint{5.467269in}{4.004351in}}{\pgfqpoint{5.475083in}{3.996537in}}%
\pgfpathcurveto{\pgfqpoint{5.482896in}{3.988724in}}{\pgfqpoint{5.493495in}{3.984333in}}{\pgfqpoint{5.504545in}{3.984333in}}%
\pgfpathclose%
\pgfusepath{stroke,fill}%
\end{pgfscope}%
\begin{pgfscope}%
\pgfpathrectangle{\pgfqpoint{0.800000in}{0.528000in}}{\pgfqpoint{4.960000in}{3.696000in}}%
\pgfusepath{clip}%
\pgfsetbuttcap%
\pgfsetroundjoin%
\definecolor{currentfill}{rgb}{0.000000,0.000000,0.000000}%
\pgfsetfillcolor{currentfill}%
\pgfsetlinewidth{1.003750pt}%
\definecolor{currentstroke}{rgb}{0.000000,0.000000,0.000000}%
\pgfsetstrokecolor{currentstroke}%
\pgfsetdash{}{0pt}%
\pgfpathmoveto{\pgfqpoint{5.504545in}{3.984333in}}%
\pgfpathcurveto{\pgfqpoint{5.515596in}{3.984333in}}{\pgfqpoint{5.526195in}{3.988724in}}{\pgfqpoint{5.534008in}{3.996537in}}%
\pgfpathcurveto{\pgfqpoint{5.541822in}{4.004351in}}{\pgfqpoint{5.546212in}{4.014950in}}{\pgfqpoint{5.546212in}{4.026000in}}%
\pgfpathcurveto{\pgfqpoint{5.546212in}{4.037050in}}{\pgfqpoint{5.541822in}{4.047649in}}{\pgfqpoint{5.534008in}{4.055463in}}%
\pgfpathcurveto{\pgfqpoint{5.526195in}{4.063276in}}{\pgfqpoint{5.515596in}{4.067667in}}{\pgfqpoint{5.504545in}{4.067667in}}%
\pgfpathcurveto{\pgfqpoint{5.493495in}{4.067667in}}{\pgfqpoint{5.482896in}{4.063276in}}{\pgfqpoint{5.475083in}{4.055463in}}%
\pgfpathcurveto{\pgfqpoint{5.467269in}{4.047649in}}{\pgfqpoint{5.462879in}{4.037050in}}{\pgfqpoint{5.462879in}{4.026000in}}%
\pgfpathcurveto{\pgfqpoint{5.462879in}{4.014950in}}{\pgfqpoint{5.467269in}{4.004351in}}{\pgfqpoint{5.475083in}{3.996537in}}%
\pgfpathcurveto{\pgfqpoint{5.482896in}{3.988724in}}{\pgfqpoint{5.493495in}{3.984333in}}{\pgfqpoint{5.504545in}{3.984333in}}%
\pgfpathclose%
\pgfusepath{stroke,fill}%
\end{pgfscope}%
\begin{pgfscope}%
\pgfpathrectangle{\pgfqpoint{0.800000in}{0.528000in}}{\pgfqpoint{4.960000in}{3.696000in}}%
\pgfusepath{clip}%
\pgfsetbuttcap%
\pgfsetroundjoin%
\definecolor{currentfill}{rgb}{0.000000,0.000000,0.000000}%
\pgfsetfillcolor{currentfill}%
\pgfsetlinewidth{1.003750pt}%
\definecolor{currentstroke}{rgb}{0.000000,0.000000,0.000000}%
\pgfsetstrokecolor{currentstroke}%
\pgfsetdash{}{0pt}%
\pgfpathmoveto{\pgfqpoint{5.504545in}{2.334266in}}%
\pgfpathcurveto{\pgfqpoint{5.515596in}{2.334266in}}{\pgfqpoint{5.526195in}{2.338657in}}{\pgfqpoint{5.534008in}{2.346470in}}%
\pgfpathcurveto{\pgfqpoint{5.541822in}{2.354284in}}{\pgfqpoint{5.546212in}{2.364883in}}{\pgfqpoint{5.546212in}{2.375933in}}%
\pgfpathcurveto{\pgfqpoint{5.546212in}{2.386983in}}{\pgfqpoint{5.541822in}{2.397582in}}{\pgfqpoint{5.534008in}{2.405396in}}%
\pgfpathcurveto{\pgfqpoint{5.526195in}{2.413209in}}{\pgfqpoint{5.515596in}{2.417600in}}{\pgfqpoint{5.504545in}{2.417600in}}%
\pgfpathcurveto{\pgfqpoint{5.493495in}{2.417600in}}{\pgfqpoint{5.482896in}{2.413209in}}{\pgfqpoint{5.475083in}{2.405396in}}%
\pgfpathcurveto{\pgfqpoint{5.467269in}{2.397582in}}{\pgfqpoint{5.462879in}{2.386983in}}{\pgfqpoint{5.462879in}{2.375933in}}%
\pgfpathcurveto{\pgfqpoint{5.462879in}{2.364883in}}{\pgfqpoint{5.467269in}{2.354284in}}{\pgfqpoint{5.475083in}{2.346470in}}%
\pgfpathcurveto{\pgfqpoint{5.482896in}{2.338657in}}{\pgfqpoint{5.493495in}{2.334266in}}{\pgfqpoint{5.504545in}{2.334266in}}%
\pgfpathclose%
\pgfusepath{stroke,fill}%
\end{pgfscope}%
\begin{pgfscope}%
\pgfpathrectangle{\pgfqpoint{0.800000in}{0.528000in}}{\pgfqpoint{4.960000in}{3.696000in}}%
\pgfusepath{clip}%
\pgfsetbuttcap%
\pgfsetroundjoin%
\definecolor{currentfill}{rgb}{0.000000,0.000000,0.000000}%
\pgfsetfillcolor{currentfill}%
\pgfsetlinewidth{1.003750pt}%
\definecolor{currentstroke}{rgb}{0.000000,0.000000,0.000000}%
\pgfsetstrokecolor{currentstroke}%
\pgfsetdash{}{0pt}%
\pgfpathmoveto{\pgfqpoint{5.504545in}{3.984333in}}%
\pgfpathcurveto{\pgfqpoint{5.515596in}{3.984333in}}{\pgfqpoint{5.526195in}{3.988724in}}{\pgfqpoint{5.534008in}{3.996537in}}%
\pgfpathcurveto{\pgfqpoint{5.541822in}{4.004351in}}{\pgfqpoint{5.546212in}{4.014950in}}{\pgfqpoint{5.546212in}{4.026000in}}%
\pgfpathcurveto{\pgfqpoint{5.546212in}{4.037050in}}{\pgfqpoint{5.541822in}{4.047649in}}{\pgfqpoint{5.534008in}{4.055463in}}%
\pgfpathcurveto{\pgfqpoint{5.526195in}{4.063276in}}{\pgfqpoint{5.515596in}{4.067667in}}{\pgfqpoint{5.504545in}{4.067667in}}%
\pgfpathcurveto{\pgfqpoint{5.493495in}{4.067667in}}{\pgfqpoint{5.482896in}{4.063276in}}{\pgfqpoint{5.475083in}{4.055463in}}%
\pgfpathcurveto{\pgfqpoint{5.467269in}{4.047649in}}{\pgfqpoint{5.462879in}{4.037050in}}{\pgfqpoint{5.462879in}{4.026000in}}%
\pgfpathcurveto{\pgfqpoint{5.462879in}{4.014950in}}{\pgfqpoint{5.467269in}{4.004351in}}{\pgfqpoint{5.475083in}{3.996537in}}%
\pgfpathcurveto{\pgfqpoint{5.482896in}{3.988724in}}{\pgfqpoint{5.493495in}{3.984333in}}{\pgfqpoint{5.504545in}{3.984333in}}%
\pgfpathclose%
\pgfusepath{stroke,fill}%
\end{pgfscope}%
\begin{pgfscope}%
\pgfpathrectangle{\pgfqpoint{0.800000in}{0.528000in}}{\pgfqpoint{4.960000in}{3.696000in}}%
\pgfusepath{clip}%
\pgfsetbuttcap%
\pgfsetroundjoin%
\definecolor{currentfill}{rgb}{0.000000,0.000000,0.000000}%
\pgfsetfillcolor{currentfill}%
\pgfsetlinewidth{1.003750pt}%
\definecolor{currentstroke}{rgb}{0.000000,0.000000,0.000000}%
\pgfsetstrokecolor{currentstroke}%
\pgfsetdash{}{0pt}%
\pgfpathmoveto{\pgfqpoint{5.504545in}{3.984333in}}%
\pgfpathcurveto{\pgfqpoint{5.515596in}{3.984333in}}{\pgfqpoint{5.526195in}{3.988724in}}{\pgfqpoint{5.534008in}{3.996537in}}%
\pgfpathcurveto{\pgfqpoint{5.541822in}{4.004351in}}{\pgfqpoint{5.546212in}{4.014950in}}{\pgfqpoint{5.546212in}{4.026000in}}%
\pgfpathcurveto{\pgfqpoint{5.546212in}{4.037050in}}{\pgfqpoint{5.541822in}{4.047649in}}{\pgfqpoint{5.534008in}{4.055463in}}%
\pgfpathcurveto{\pgfqpoint{5.526195in}{4.063276in}}{\pgfqpoint{5.515596in}{4.067667in}}{\pgfqpoint{5.504545in}{4.067667in}}%
\pgfpathcurveto{\pgfqpoint{5.493495in}{4.067667in}}{\pgfqpoint{5.482896in}{4.063276in}}{\pgfqpoint{5.475083in}{4.055463in}}%
\pgfpathcurveto{\pgfqpoint{5.467269in}{4.047649in}}{\pgfqpoint{5.462879in}{4.037050in}}{\pgfqpoint{5.462879in}{4.026000in}}%
\pgfpathcurveto{\pgfqpoint{5.462879in}{4.014950in}}{\pgfqpoint{5.467269in}{4.004351in}}{\pgfqpoint{5.475083in}{3.996537in}}%
\pgfpathcurveto{\pgfqpoint{5.482896in}{3.988724in}}{\pgfqpoint{5.493495in}{3.984333in}}{\pgfqpoint{5.504545in}{3.984333in}}%
\pgfpathclose%
\pgfusepath{stroke,fill}%
\end{pgfscope}%
\begin{pgfscope}%
\pgfpathrectangle{\pgfqpoint{0.800000in}{0.528000in}}{\pgfqpoint{4.960000in}{3.696000in}}%
\pgfusepath{clip}%
\pgfsetbuttcap%
\pgfsetroundjoin%
\definecolor{currentfill}{rgb}{0.000000,0.000000,0.000000}%
\pgfsetfillcolor{currentfill}%
\pgfsetlinewidth{1.003750pt}%
\definecolor{currentstroke}{rgb}{0.000000,0.000000,0.000000}%
\pgfsetstrokecolor{currentstroke}%
\pgfsetdash{}{0pt}%
\pgfpathmoveto{\pgfqpoint{5.504545in}{3.984333in}}%
\pgfpathcurveto{\pgfqpoint{5.515596in}{3.984333in}}{\pgfqpoint{5.526195in}{3.988724in}}{\pgfqpoint{5.534008in}{3.996537in}}%
\pgfpathcurveto{\pgfqpoint{5.541822in}{4.004351in}}{\pgfqpoint{5.546212in}{4.014950in}}{\pgfqpoint{5.546212in}{4.026000in}}%
\pgfpathcurveto{\pgfqpoint{5.546212in}{4.037050in}}{\pgfqpoint{5.541822in}{4.047649in}}{\pgfqpoint{5.534008in}{4.055463in}}%
\pgfpathcurveto{\pgfqpoint{5.526195in}{4.063276in}}{\pgfqpoint{5.515596in}{4.067667in}}{\pgfqpoint{5.504545in}{4.067667in}}%
\pgfpathcurveto{\pgfqpoint{5.493495in}{4.067667in}}{\pgfqpoint{5.482896in}{4.063276in}}{\pgfqpoint{5.475083in}{4.055463in}}%
\pgfpathcurveto{\pgfqpoint{5.467269in}{4.047649in}}{\pgfqpoint{5.462879in}{4.037050in}}{\pgfqpoint{5.462879in}{4.026000in}}%
\pgfpathcurveto{\pgfqpoint{5.462879in}{4.014950in}}{\pgfqpoint{5.467269in}{4.004351in}}{\pgfqpoint{5.475083in}{3.996537in}}%
\pgfpathcurveto{\pgfqpoint{5.482896in}{3.988724in}}{\pgfqpoint{5.493495in}{3.984333in}}{\pgfqpoint{5.504545in}{3.984333in}}%
\pgfpathclose%
\pgfusepath{stroke,fill}%
\end{pgfscope}%
\begin{pgfscope}%
\pgfpathrectangle{\pgfqpoint{0.800000in}{0.528000in}}{\pgfqpoint{4.960000in}{3.696000in}}%
\pgfusepath{clip}%
\pgfsetbuttcap%
\pgfsetroundjoin%
\definecolor{currentfill}{rgb}{0.000000,0.000000,0.000000}%
\pgfsetfillcolor{currentfill}%
\pgfsetlinewidth{1.003750pt}%
\definecolor{currentstroke}{rgb}{0.000000,0.000000,0.000000}%
\pgfsetstrokecolor{currentstroke}%
\pgfsetdash{}{0pt}%
\pgfpathmoveto{\pgfqpoint{5.504545in}{3.984333in}}%
\pgfpathcurveto{\pgfqpoint{5.515596in}{3.984333in}}{\pgfqpoint{5.526195in}{3.988724in}}{\pgfqpoint{5.534008in}{3.996537in}}%
\pgfpathcurveto{\pgfqpoint{5.541822in}{4.004351in}}{\pgfqpoint{5.546212in}{4.014950in}}{\pgfqpoint{5.546212in}{4.026000in}}%
\pgfpathcurveto{\pgfqpoint{5.546212in}{4.037050in}}{\pgfqpoint{5.541822in}{4.047649in}}{\pgfqpoint{5.534008in}{4.055463in}}%
\pgfpathcurveto{\pgfqpoint{5.526195in}{4.063276in}}{\pgfqpoint{5.515596in}{4.067667in}}{\pgfqpoint{5.504545in}{4.067667in}}%
\pgfpathcurveto{\pgfqpoint{5.493495in}{4.067667in}}{\pgfqpoint{5.482896in}{4.063276in}}{\pgfqpoint{5.475083in}{4.055463in}}%
\pgfpathcurveto{\pgfqpoint{5.467269in}{4.047649in}}{\pgfqpoint{5.462879in}{4.037050in}}{\pgfqpoint{5.462879in}{4.026000in}}%
\pgfpathcurveto{\pgfqpoint{5.462879in}{4.014950in}}{\pgfqpoint{5.467269in}{4.004351in}}{\pgfqpoint{5.475083in}{3.996537in}}%
\pgfpathcurveto{\pgfqpoint{5.482896in}{3.988724in}}{\pgfqpoint{5.493495in}{3.984333in}}{\pgfqpoint{5.504545in}{3.984333in}}%
\pgfpathclose%
\pgfusepath{stroke,fill}%
\end{pgfscope}%
\begin{pgfscope}%
\pgfpathrectangle{\pgfqpoint{0.800000in}{0.528000in}}{\pgfqpoint{4.960000in}{3.696000in}}%
\pgfusepath{clip}%
\pgfsetbuttcap%
\pgfsetroundjoin%
\definecolor{currentfill}{rgb}{0.000000,0.000000,0.000000}%
\pgfsetfillcolor{currentfill}%
\pgfsetlinewidth{1.003750pt}%
\definecolor{currentstroke}{rgb}{0.000000,0.000000,0.000000}%
\pgfsetstrokecolor{currentstroke}%
\pgfsetdash{}{0pt}%
\pgfpathmoveto{\pgfqpoint{5.504545in}{3.984333in}}%
\pgfpathcurveto{\pgfqpoint{5.515596in}{3.984333in}}{\pgfqpoint{5.526195in}{3.988724in}}{\pgfqpoint{5.534008in}{3.996537in}}%
\pgfpathcurveto{\pgfqpoint{5.541822in}{4.004351in}}{\pgfqpoint{5.546212in}{4.014950in}}{\pgfqpoint{5.546212in}{4.026000in}}%
\pgfpathcurveto{\pgfqpoint{5.546212in}{4.037050in}}{\pgfqpoint{5.541822in}{4.047649in}}{\pgfqpoint{5.534008in}{4.055463in}}%
\pgfpathcurveto{\pgfqpoint{5.526195in}{4.063276in}}{\pgfqpoint{5.515596in}{4.067667in}}{\pgfqpoint{5.504545in}{4.067667in}}%
\pgfpathcurveto{\pgfqpoint{5.493495in}{4.067667in}}{\pgfqpoint{5.482896in}{4.063276in}}{\pgfqpoint{5.475083in}{4.055463in}}%
\pgfpathcurveto{\pgfqpoint{5.467269in}{4.047649in}}{\pgfqpoint{5.462879in}{4.037050in}}{\pgfqpoint{5.462879in}{4.026000in}}%
\pgfpathcurveto{\pgfqpoint{5.462879in}{4.014950in}}{\pgfqpoint{5.467269in}{4.004351in}}{\pgfqpoint{5.475083in}{3.996537in}}%
\pgfpathcurveto{\pgfqpoint{5.482896in}{3.988724in}}{\pgfqpoint{5.493495in}{3.984333in}}{\pgfqpoint{5.504545in}{3.984333in}}%
\pgfpathclose%
\pgfusepath{stroke,fill}%
\end{pgfscope}%
\begin{pgfscope}%
\pgfpathrectangle{\pgfqpoint{0.800000in}{0.528000in}}{\pgfqpoint{4.960000in}{3.696000in}}%
\pgfusepath{clip}%
\pgfsetbuttcap%
\pgfsetroundjoin%
\definecolor{currentfill}{rgb}{0.000000,0.000000,0.000000}%
\pgfsetfillcolor{currentfill}%
\pgfsetlinewidth{1.003750pt}%
\definecolor{currentstroke}{rgb}{0.000000,0.000000,0.000000}%
\pgfsetstrokecolor{currentstroke}%
\pgfsetdash{}{0pt}%
\pgfpathmoveto{\pgfqpoint{5.504545in}{2.334266in}}%
\pgfpathcurveto{\pgfqpoint{5.515596in}{2.334266in}}{\pgfqpoint{5.526195in}{2.338657in}}{\pgfqpoint{5.534008in}{2.346470in}}%
\pgfpathcurveto{\pgfqpoint{5.541822in}{2.354284in}}{\pgfqpoint{5.546212in}{2.364883in}}{\pgfqpoint{5.546212in}{2.375933in}}%
\pgfpathcurveto{\pgfqpoint{5.546212in}{2.386983in}}{\pgfqpoint{5.541822in}{2.397582in}}{\pgfqpoint{5.534008in}{2.405396in}}%
\pgfpathcurveto{\pgfqpoint{5.526195in}{2.413209in}}{\pgfqpoint{5.515596in}{2.417600in}}{\pgfqpoint{5.504545in}{2.417600in}}%
\pgfpathcurveto{\pgfqpoint{5.493495in}{2.417600in}}{\pgfqpoint{5.482896in}{2.413209in}}{\pgfqpoint{5.475083in}{2.405396in}}%
\pgfpathcurveto{\pgfqpoint{5.467269in}{2.397582in}}{\pgfqpoint{5.462879in}{2.386983in}}{\pgfqpoint{5.462879in}{2.375933in}}%
\pgfpathcurveto{\pgfqpoint{5.462879in}{2.364883in}}{\pgfqpoint{5.467269in}{2.354284in}}{\pgfqpoint{5.475083in}{2.346470in}}%
\pgfpathcurveto{\pgfqpoint{5.482896in}{2.338657in}}{\pgfqpoint{5.493495in}{2.334266in}}{\pgfqpoint{5.504545in}{2.334266in}}%
\pgfpathclose%
\pgfusepath{stroke,fill}%
\end{pgfscope}%
\begin{pgfscope}%
\pgfpathrectangle{\pgfqpoint{0.800000in}{0.528000in}}{\pgfqpoint{4.960000in}{3.696000in}}%
\pgfusepath{clip}%
\pgfsetbuttcap%
\pgfsetroundjoin%
\definecolor{currentfill}{rgb}{0.000000,0.000000,0.000000}%
\pgfsetfillcolor{currentfill}%
\pgfsetlinewidth{1.003750pt}%
\definecolor{currentstroke}{rgb}{0.000000,0.000000,0.000000}%
\pgfsetstrokecolor{currentstroke}%
\pgfsetdash{}{0pt}%
\pgfpathmoveto{\pgfqpoint{5.504545in}{3.984333in}}%
\pgfpathcurveto{\pgfqpoint{5.515596in}{3.984333in}}{\pgfqpoint{5.526195in}{3.988724in}}{\pgfqpoint{5.534008in}{3.996537in}}%
\pgfpathcurveto{\pgfqpoint{5.541822in}{4.004351in}}{\pgfqpoint{5.546212in}{4.014950in}}{\pgfqpoint{5.546212in}{4.026000in}}%
\pgfpathcurveto{\pgfqpoint{5.546212in}{4.037050in}}{\pgfqpoint{5.541822in}{4.047649in}}{\pgfqpoint{5.534008in}{4.055463in}}%
\pgfpathcurveto{\pgfqpoint{5.526195in}{4.063276in}}{\pgfqpoint{5.515596in}{4.067667in}}{\pgfqpoint{5.504545in}{4.067667in}}%
\pgfpathcurveto{\pgfqpoint{5.493495in}{4.067667in}}{\pgfqpoint{5.482896in}{4.063276in}}{\pgfqpoint{5.475083in}{4.055463in}}%
\pgfpathcurveto{\pgfqpoint{5.467269in}{4.047649in}}{\pgfqpoint{5.462879in}{4.037050in}}{\pgfqpoint{5.462879in}{4.026000in}}%
\pgfpathcurveto{\pgfqpoint{5.462879in}{4.014950in}}{\pgfqpoint{5.467269in}{4.004351in}}{\pgfqpoint{5.475083in}{3.996537in}}%
\pgfpathcurveto{\pgfqpoint{5.482896in}{3.988724in}}{\pgfqpoint{5.493495in}{3.984333in}}{\pgfqpoint{5.504545in}{3.984333in}}%
\pgfpathclose%
\pgfusepath{stroke,fill}%
\end{pgfscope}%
\begin{pgfscope}%
\pgfpathrectangle{\pgfqpoint{0.800000in}{0.528000in}}{\pgfqpoint{4.960000in}{3.696000in}}%
\pgfusepath{clip}%
\pgfsetbuttcap%
\pgfsetroundjoin%
\definecolor{currentfill}{rgb}{0.000000,0.000000,0.000000}%
\pgfsetfillcolor{currentfill}%
\pgfsetlinewidth{1.003750pt}%
\definecolor{currentstroke}{rgb}{0.000000,0.000000,0.000000}%
\pgfsetstrokecolor{currentstroke}%
\pgfsetdash{}{0pt}%
\pgfpathmoveto{\pgfqpoint{5.504545in}{3.984333in}}%
\pgfpathcurveto{\pgfqpoint{5.515596in}{3.984333in}}{\pgfqpoint{5.526195in}{3.988724in}}{\pgfqpoint{5.534008in}{3.996537in}}%
\pgfpathcurveto{\pgfqpoint{5.541822in}{4.004351in}}{\pgfqpoint{5.546212in}{4.014950in}}{\pgfqpoint{5.546212in}{4.026000in}}%
\pgfpathcurveto{\pgfqpoint{5.546212in}{4.037050in}}{\pgfqpoint{5.541822in}{4.047649in}}{\pgfqpoint{5.534008in}{4.055463in}}%
\pgfpathcurveto{\pgfqpoint{5.526195in}{4.063276in}}{\pgfqpoint{5.515596in}{4.067667in}}{\pgfqpoint{5.504545in}{4.067667in}}%
\pgfpathcurveto{\pgfqpoint{5.493495in}{4.067667in}}{\pgfqpoint{5.482896in}{4.063276in}}{\pgfqpoint{5.475083in}{4.055463in}}%
\pgfpathcurveto{\pgfqpoint{5.467269in}{4.047649in}}{\pgfqpoint{5.462879in}{4.037050in}}{\pgfqpoint{5.462879in}{4.026000in}}%
\pgfpathcurveto{\pgfqpoint{5.462879in}{4.014950in}}{\pgfqpoint{5.467269in}{4.004351in}}{\pgfqpoint{5.475083in}{3.996537in}}%
\pgfpathcurveto{\pgfqpoint{5.482896in}{3.988724in}}{\pgfqpoint{5.493495in}{3.984333in}}{\pgfqpoint{5.504545in}{3.984333in}}%
\pgfpathclose%
\pgfusepath{stroke,fill}%
\end{pgfscope}%
\begin{pgfscope}%
\pgfpathrectangle{\pgfqpoint{0.800000in}{0.528000in}}{\pgfqpoint{4.960000in}{3.696000in}}%
\pgfusepath{clip}%
\pgfsetbuttcap%
\pgfsetroundjoin%
\definecolor{currentfill}{rgb}{0.000000,0.000000,0.000000}%
\pgfsetfillcolor{currentfill}%
\pgfsetlinewidth{1.003750pt}%
\definecolor{currentstroke}{rgb}{0.000000,0.000000,0.000000}%
\pgfsetstrokecolor{currentstroke}%
\pgfsetdash{}{0pt}%
\pgfpathmoveto{\pgfqpoint{5.504545in}{2.334266in}}%
\pgfpathcurveto{\pgfqpoint{5.515596in}{2.334266in}}{\pgfqpoint{5.526195in}{2.338657in}}{\pgfqpoint{5.534008in}{2.346470in}}%
\pgfpathcurveto{\pgfqpoint{5.541822in}{2.354284in}}{\pgfqpoint{5.546212in}{2.364883in}}{\pgfqpoint{5.546212in}{2.375933in}}%
\pgfpathcurveto{\pgfqpoint{5.546212in}{2.386983in}}{\pgfqpoint{5.541822in}{2.397582in}}{\pgfqpoint{5.534008in}{2.405396in}}%
\pgfpathcurveto{\pgfqpoint{5.526195in}{2.413209in}}{\pgfqpoint{5.515596in}{2.417600in}}{\pgfqpoint{5.504545in}{2.417600in}}%
\pgfpathcurveto{\pgfqpoint{5.493495in}{2.417600in}}{\pgfqpoint{5.482896in}{2.413209in}}{\pgfqpoint{5.475083in}{2.405396in}}%
\pgfpathcurveto{\pgfqpoint{5.467269in}{2.397582in}}{\pgfqpoint{5.462879in}{2.386983in}}{\pgfqpoint{5.462879in}{2.375933in}}%
\pgfpathcurveto{\pgfqpoint{5.462879in}{2.364883in}}{\pgfqpoint{5.467269in}{2.354284in}}{\pgfqpoint{5.475083in}{2.346470in}}%
\pgfpathcurveto{\pgfqpoint{5.482896in}{2.338657in}}{\pgfqpoint{5.493495in}{2.334266in}}{\pgfqpoint{5.504545in}{2.334266in}}%
\pgfpathclose%
\pgfusepath{stroke,fill}%
\end{pgfscope}%
\begin{pgfscope}%
\pgfpathrectangle{\pgfqpoint{0.800000in}{0.528000in}}{\pgfqpoint{4.960000in}{3.696000in}}%
\pgfusepath{clip}%
\pgfsetbuttcap%
\pgfsetroundjoin%
\definecolor{currentfill}{rgb}{0.000000,0.000000,0.000000}%
\pgfsetfillcolor{currentfill}%
\pgfsetlinewidth{1.003750pt}%
\definecolor{currentstroke}{rgb}{0.000000,0.000000,0.000000}%
\pgfsetstrokecolor{currentstroke}%
\pgfsetdash{}{0pt}%
\pgfpathmoveto{\pgfqpoint{5.504545in}{2.334266in}}%
\pgfpathcurveto{\pgfqpoint{5.515596in}{2.334266in}}{\pgfqpoint{5.526195in}{2.338657in}}{\pgfqpoint{5.534008in}{2.346470in}}%
\pgfpathcurveto{\pgfqpoint{5.541822in}{2.354284in}}{\pgfqpoint{5.546212in}{2.364883in}}{\pgfqpoint{5.546212in}{2.375933in}}%
\pgfpathcurveto{\pgfqpoint{5.546212in}{2.386983in}}{\pgfqpoint{5.541822in}{2.397582in}}{\pgfqpoint{5.534008in}{2.405396in}}%
\pgfpathcurveto{\pgfqpoint{5.526195in}{2.413209in}}{\pgfqpoint{5.515596in}{2.417600in}}{\pgfqpoint{5.504545in}{2.417600in}}%
\pgfpathcurveto{\pgfqpoint{5.493495in}{2.417600in}}{\pgfqpoint{5.482896in}{2.413209in}}{\pgfqpoint{5.475083in}{2.405396in}}%
\pgfpathcurveto{\pgfqpoint{5.467269in}{2.397582in}}{\pgfqpoint{5.462879in}{2.386983in}}{\pgfqpoint{5.462879in}{2.375933in}}%
\pgfpathcurveto{\pgfqpoint{5.462879in}{2.364883in}}{\pgfqpoint{5.467269in}{2.354284in}}{\pgfqpoint{5.475083in}{2.346470in}}%
\pgfpathcurveto{\pgfqpoint{5.482896in}{2.338657in}}{\pgfqpoint{5.493495in}{2.334266in}}{\pgfqpoint{5.504545in}{2.334266in}}%
\pgfpathclose%
\pgfusepath{stroke,fill}%
\end{pgfscope}%
\begin{pgfscope}%
\pgfpathrectangle{\pgfqpoint{0.800000in}{0.528000in}}{\pgfqpoint{4.960000in}{3.696000in}}%
\pgfusepath{clip}%
\pgfsetbuttcap%
\pgfsetroundjoin%
\definecolor{currentfill}{rgb}{0.000000,0.000000,0.000000}%
\pgfsetfillcolor{currentfill}%
\pgfsetlinewidth{1.003750pt}%
\definecolor{currentstroke}{rgb}{0.000000,0.000000,0.000000}%
\pgfsetstrokecolor{currentstroke}%
\pgfsetdash{}{0pt}%
\pgfpathmoveto{\pgfqpoint{5.504545in}{2.334266in}}%
\pgfpathcurveto{\pgfqpoint{5.515596in}{2.334266in}}{\pgfqpoint{5.526195in}{2.338657in}}{\pgfqpoint{5.534008in}{2.346470in}}%
\pgfpathcurveto{\pgfqpoint{5.541822in}{2.354284in}}{\pgfqpoint{5.546212in}{2.364883in}}{\pgfqpoint{5.546212in}{2.375933in}}%
\pgfpathcurveto{\pgfqpoint{5.546212in}{2.386983in}}{\pgfqpoint{5.541822in}{2.397582in}}{\pgfqpoint{5.534008in}{2.405396in}}%
\pgfpathcurveto{\pgfqpoint{5.526195in}{2.413209in}}{\pgfqpoint{5.515596in}{2.417600in}}{\pgfqpoint{5.504545in}{2.417600in}}%
\pgfpathcurveto{\pgfqpoint{5.493495in}{2.417600in}}{\pgfqpoint{5.482896in}{2.413209in}}{\pgfqpoint{5.475083in}{2.405396in}}%
\pgfpathcurveto{\pgfqpoint{5.467269in}{2.397582in}}{\pgfqpoint{5.462879in}{2.386983in}}{\pgfqpoint{5.462879in}{2.375933in}}%
\pgfpathcurveto{\pgfqpoint{5.462879in}{2.364883in}}{\pgfqpoint{5.467269in}{2.354284in}}{\pgfqpoint{5.475083in}{2.346470in}}%
\pgfpathcurveto{\pgfqpoint{5.482896in}{2.338657in}}{\pgfqpoint{5.493495in}{2.334266in}}{\pgfqpoint{5.504545in}{2.334266in}}%
\pgfpathclose%
\pgfusepath{stroke,fill}%
\end{pgfscope}%
\begin{pgfscope}%
\pgfpathrectangle{\pgfqpoint{0.800000in}{0.528000in}}{\pgfqpoint{4.960000in}{3.696000in}}%
\pgfusepath{clip}%
\pgfsetbuttcap%
\pgfsetroundjoin%
\definecolor{currentfill}{rgb}{0.000000,0.000000,0.000000}%
\pgfsetfillcolor{currentfill}%
\pgfsetlinewidth{1.003750pt}%
\definecolor{currentstroke}{rgb}{0.000000,0.000000,0.000000}%
\pgfsetstrokecolor{currentstroke}%
\pgfsetdash{}{0pt}%
\pgfpathmoveto{\pgfqpoint{5.504545in}{2.334266in}}%
\pgfpathcurveto{\pgfqpoint{5.515596in}{2.334266in}}{\pgfqpoint{5.526195in}{2.338657in}}{\pgfqpoint{5.534008in}{2.346470in}}%
\pgfpathcurveto{\pgfqpoint{5.541822in}{2.354284in}}{\pgfqpoint{5.546212in}{2.364883in}}{\pgfqpoint{5.546212in}{2.375933in}}%
\pgfpathcurveto{\pgfqpoint{5.546212in}{2.386983in}}{\pgfqpoint{5.541822in}{2.397582in}}{\pgfqpoint{5.534008in}{2.405396in}}%
\pgfpathcurveto{\pgfqpoint{5.526195in}{2.413209in}}{\pgfqpoint{5.515596in}{2.417600in}}{\pgfqpoint{5.504545in}{2.417600in}}%
\pgfpathcurveto{\pgfqpoint{5.493495in}{2.417600in}}{\pgfqpoint{5.482896in}{2.413209in}}{\pgfqpoint{5.475083in}{2.405396in}}%
\pgfpathcurveto{\pgfqpoint{5.467269in}{2.397582in}}{\pgfqpoint{5.462879in}{2.386983in}}{\pgfqpoint{5.462879in}{2.375933in}}%
\pgfpathcurveto{\pgfqpoint{5.462879in}{2.364883in}}{\pgfqpoint{5.467269in}{2.354284in}}{\pgfqpoint{5.475083in}{2.346470in}}%
\pgfpathcurveto{\pgfqpoint{5.482896in}{2.338657in}}{\pgfqpoint{5.493495in}{2.334266in}}{\pgfqpoint{5.504545in}{2.334266in}}%
\pgfpathclose%
\pgfusepath{stroke,fill}%
\end{pgfscope}%
\begin{pgfscope}%
\pgfpathrectangle{\pgfqpoint{0.800000in}{0.528000in}}{\pgfqpoint{4.960000in}{3.696000in}}%
\pgfusepath{clip}%
\pgfsetbuttcap%
\pgfsetroundjoin%
\definecolor{currentfill}{rgb}{0.000000,0.000000,0.000000}%
\pgfsetfillcolor{currentfill}%
\pgfsetlinewidth{1.003750pt}%
\definecolor{currentstroke}{rgb}{0.000000,0.000000,0.000000}%
\pgfsetstrokecolor{currentstroke}%
\pgfsetdash{}{0pt}%
\pgfpathmoveto{\pgfqpoint{5.504545in}{3.984333in}}%
\pgfpathcurveto{\pgfqpoint{5.515596in}{3.984333in}}{\pgfqpoint{5.526195in}{3.988724in}}{\pgfqpoint{5.534008in}{3.996537in}}%
\pgfpathcurveto{\pgfqpoint{5.541822in}{4.004351in}}{\pgfqpoint{5.546212in}{4.014950in}}{\pgfqpoint{5.546212in}{4.026000in}}%
\pgfpathcurveto{\pgfqpoint{5.546212in}{4.037050in}}{\pgfqpoint{5.541822in}{4.047649in}}{\pgfqpoint{5.534008in}{4.055463in}}%
\pgfpathcurveto{\pgfqpoint{5.526195in}{4.063276in}}{\pgfqpoint{5.515596in}{4.067667in}}{\pgfqpoint{5.504545in}{4.067667in}}%
\pgfpathcurveto{\pgfqpoint{5.493495in}{4.067667in}}{\pgfqpoint{5.482896in}{4.063276in}}{\pgfqpoint{5.475083in}{4.055463in}}%
\pgfpathcurveto{\pgfqpoint{5.467269in}{4.047649in}}{\pgfqpoint{5.462879in}{4.037050in}}{\pgfqpoint{5.462879in}{4.026000in}}%
\pgfpathcurveto{\pgfqpoint{5.462879in}{4.014950in}}{\pgfqpoint{5.467269in}{4.004351in}}{\pgfqpoint{5.475083in}{3.996537in}}%
\pgfpathcurveto{\pgfqpoint{5.482896in}{3.988724in}}{\pgfqpoint{5.493495in}{3.984333in}}{\pgfqpoint{5.504545in}{3.984333in}}%
\pgfpathclose%
\pgfusepath{stroke,fill}%
\end{pgfscope}%
\begin{pgfscope}%
\pgfpathrectangle{\pgfqpoint{0.800000in}{0.528000in}}{\pgfqpoint{4.960000in}{3.696000in}}%
\pgfusepath{clip}%
\pgfsetbuttcap%
\pgfsetroundjoin%
\definecolor{currentfill}{rgb}{0.000000,0.000000,0.000000}%
\pgfsetfillcolor{currentfill}%
\pgfsetlinewidth{1.003750pt}%
\definecolor{currentstroke}{rgb}{0.000000,0.000000,0.000000}%
\pgfsetstrokecolor{currentstroke}%
\pgfsetdash{}{0pt}%
\pgfpathmoveto{\pgfqpoint{5.504545in}{3.984333in}}%
\pgfpathcurveto{\pgfqpoint{5.515596in}{3.984333in}}{\pgfqpoint{5.526195in}{3.988724in}}{\pgfqpoint{5.534008in}{3.996537in}}%
\pgfpathcurveto{\pgfqpoint{5.541822in}{4.004351in}}{\pgfqpoint{5.546212in}{4.014950in}}{\pgfqpoint{5.546212in}{4.026000in}}%
\pgfpathcurveto{\pgfqpoint{5.546212in}{4.037050in}}{\pgfqpoint{5.541822in}{4.047649in}}{\pgfqpoint{5.534008in}{4.055463in}}%
\pgfpathcurveto{\pgfqpoint{5.526195in}{4.063276in}}{\pgfqpoint{5.515596in}{4.067667in}}{\pgfqpoint{5.504545in}{4.067667in}}%
\pgfpathcurveto{\pgfqpoint{5.493495in}{4.067667in}}{\pgfqpoint{5.482896in}{4.063276in}}{\pgfqpoint{5.475083in}{4.055463in}}%
\pgfpathcurveto{\pgfqpoint{5.467269in}{4.047649in}}{\pgfqpoint{5.462879in}{4.037050in}}{\pgfqpoint{5.462879in}{4.026000in}}%
\pgfpathcurveto{\pgfqpoint{5.462879in}{4.014950in}}{\pgfqpoint{5.467269in}{4.004351in}}{\pgfqpoint{5.475083in}{3.996537in}}%
\pgfpathcurveto{\pgfqpoint{5.482896in}{3.988724in}}{\pgfqpoint{5.493495in}{3.984333in}}{\pgfqpoint{5.504545in}{3.984333in}}%
\pgfpathclose%
\pgfusepath{stroke,fill}%
\end{pgfscope}%
\begin{pgfscope}%
\pgfpathrectangle{\pgfqpoint{0.800000in}{0.528000in}}{\pgfqpoint{4.960000in}{3.696000in}}%
\pgfusepath{clip}%
\pgfsetbuttcap%
\pgfsetroundjoin%
\definecolor{currentfill}{rgb}{0.000000,0.000000,0.000000}%
\pgfsetfillcolor{currentfill}%
\pgfsetlinewidth{1.003750pt}%
\definecolor{currentstroke}{rgb}{0.000000,0.000000,0.000000}%
\pgfsetstrokecolor{currentstroke}%
\pgfsetdash{}{0pt}%
\pgfpathmoveto{\pgfqpoint{5.504545in}{2.334266in}}%
\pgfpathcurveto{\pgfqpoint{5.515596in}{2.334266in}}{\pgfqpoint{5.526195in}{2.338657in}}{\pgfqpoint{5.534008in}{2.346470in}}%
\pgfpathcurveto{\pgfqpoint{5.541822in}{2.354284in}}{\pgfqpoint{5.546212in}{2.364883in}}{\pgfqpoint{5.546212in}{2.375933in}}%
\pgfpathcurveto{\pgfqpoint{5.546212in}{2.386983in}}{\pgfqpoint{5.541822in}{2.397582in}}{\pgfqpoint{5.534008in}{2.405396in}}%
\pgfpathcurveto{\pgfqpoint{5.526195in}{2.413209in}}{\pgfqpoint{5.515596in}{2.417600in}}{\pgfqpoint{5.504545in}{2.417600in}}%
\pgfpathcurveto{\pgfqpoint{5.493495in}{2.417600in}}{\pgfqpoint{5.482896in}{2.413209in}}{\pgfqpoint{5.475083in}{2.405396in}}%
\pgfpathcurveto{\pgfqpoint{5.467269in}{2.397582in}}{\pgfqpoint{5.462879in}{2.386983in}}{\pgfqpoint{5.462879in}{2.375933in}}%
\pgfpathcurveto{\pgfqpoint{5.462879in}{2.364883in}}{\pgfqpoint{5.467269in}{2.354284in}}{\pgfqpoint{5.475083in}{2.346470in}}%
\pgfpathcurveto{\pgfqpoint{5.482896in}{2.338657in}}{\pgfqpoint{5.493495in}{2.334266in}}{\pgfqpoint{5.504545in}{2.334266in}}%
\pgfpathclose%
\pgfusepath{stroke,fill}%
\end{pgfscope}%
\begin{pgfscope}%
\pgfpathrectangle{\pgfqpoint{0.800000in}{0.528000in}}{\pgfqpoint{4.960000in}{3.696000in}}%
\pgfusepath{clip}%
\pgfsetbuttcap%
\pgfsetroundjoin%
\definecolor{currentfill}{rgb}{0.000000,0.000000,0.000000}%
\pgfsetfillcolor{currentfill}%
\pgfsetlinewidth{1.003750pt}%
\definecolor{currentstroke}{rgb}{0.000000,0.000000,0.000000}%
\pgfsetstrokecolor{currentstroke}%
\pgfsetdash{}{0pt}%
\pgfpathmoveto{\pgfqpoint{5.504545in}{3.984333in}}%
\pgfpathcurveto{\pgfqpoint{5.515596in}{3.984333in}}{\pgfqpoint{5.526195in}{3.988724in}}{\pgfqpoint{5.534008in}{3.996537in}}%
\pgfpathcurveto{\pgfqpoint{5.541822in}{4.004351in}}{\pgfqpoint{5.546212in}{4.014950in}}{\pgfqpoint{5.546212in}{4.026000in}}%
\pgfpathcurveto{\pgfqpoint{5.546212in}{4.037050in}}{\pgfqpoint{5.541822in}{4.047649in}}{\pgfqpoint{5.534008in}{4.055463in}}%
\pgfpathcurveto{\pgfqpoint{5.526195in}{4.063276in}}{\pgfqpoint{5.515596in}{4.067667in}}{\pgfqpoint{5.504545in}{4.067667in}}%
\pgfpathcurveto{\pgfqpoint{5.493495in}{4.067667in}}{\pgfqpoint{5.482896in}{4.063276in}}{\pgfqpoint{5.475083in}{4.055463in}}%
\pgfpathcurveto{\pgfqpoint{5.467269in}{4.047649in}}{\pgfqpoint{5.462879in}{4.037050in}}{\pgfqpoint{5.462879in}{4.026000in}}%
\pgfpathcurveto{\pgfqpoint{5.462879in}{4.014950in}}{\pgfqpoint{5.467269in}{4.004351in}}{\pgfqpoint{5.475083in}{3.996537in}}%
\pgfpathcurveto{\pgfqpoint{5.482896in}{3.988724in}}{\pgfqpoint{5.493495in}{3.984333in}}{\pgfqpoint{5.504545in}{3.984333in}}%
\pgfpathclose%
\pgfusepath{stroke,fill}%
\end{pgfscope}%
\begin{pgfscope}%
\pgfpathrectangle{\pgfqpoint{0.800000in}{0.528000in}}{\pgfqpoint{4.960000in}{3.696000in}}%
\pgfusepath{clip}%
\pgfsetbuttcap%
\pgfsetroundjoin%
\definecolor{currentfill}{rgb}{0.000000,0.000000,0.000000}%
\pgfsetfillcolor{currentfill}%
\pgfsetlinewidth{1.003750pt}%
\definecolor{currentstroke}{rgb}{0.000000,0.000000,0.000000}%
\pgfsetstrokecolor{currentstroke}%
\pgfsetdash{}{0pt}%
\pgfpathmoveto{\pgfqpoint{5.504545in}{3.984333in}}%
\pgfpathcurveto{\pgfqpoint{5.515596in}{3.984333in}}{\pgfqpoint{5.526195in}{3.988724in}}{\pgfqpoint{5.534008in}{3.996537in}}%
\pgfpathcurveto{\pgfqpoint{5.541822in}{4.004351in}}{\pgfqpoint{5.546212in}{4.014950in}}{\pgfqpoint{5.546212in}{4.026000in}}%
\pgfpathcurveto{\pgfqpoint{5.546212in}{4.037050in}}{\pgfqpoint{5.541822in}{4.047649in}}{\pgfqpoint{5.534008in}{4.055463in}}%
\pgfpathcurveto{\pgfqpoint{5.526195in}{4.063276in}}{\pgfqpoint{5.515596in}{4.067667in}}{\pgfqpoint{5.504545in}{4.067667in}}%
\pgfpathcurveto{\pgfqpoint{5.493495in}{4.067667in}}{\pgfqpoint{5.482896in}{4.063276in}}{\pgfqpoint{5.475083in}{4.055463in}}%
\pgfpathcurveto{\pgfqpoint{5.467269in}{4.047649in}}{\pgfqpoint{5.462879in}{4.037050in}}{\pgfqpoint{5.462879in}{4.026000in}}%
\pgfpathcurveto{\pgfqpoint{5.462879in}{4.014950in}}{\pgfqpoint{5.467269in}{4.004351in}}{\pgfqpoint{5.475083in}{3.996537in}}%
\pgfpathcurveto{\pgfqpoint{5.482896in}{3.988724in}}{\pgfqpoint{5.493495in}{3.984333in}}{\pgfqpoint{5.504545in}{3.984333in}}%
\pgfpathclose%
\pgfusepath{stroke,fill}%
\end{pgfscope}%
\begin{pgfscope}%
\pgfpathrectangle{\pgfqpoint{0.800000in}{0.528000in}}{\pgfqpoint{4.960000in}{3.696000in}}%
\pgfusepath{clip}%
\pgfsetbuttcap%
\pgfsetroundjoin%
\definecolor{currentfill}{rgb}{0.000000,0.000000,0.000000}%
\pgfsetfillcolor{currentfill}%
\pgfsetlinewidth{1.003750pt}%
\definecolor{currentstroke}{rgb}{0.000000,0.000000,0.000000}%
\pgfsetstrokecolor{currentstroke}%
\pgfsetdash{}{0pt}%
\pgfpathmoveto{\pgfqpoint{5.504545in}{3.984333in}}%
\pgfpathcurveto{\pgfqpoint{5.515596in}{3.984333in}}{\pgfqpoint{5.526195in}{3.988724in}}{\pgfqpoint{5.534008in}{3.996537in}}%
\pgfpathcurveto{\pgfqpoint{5.541822in}{4.004351in}}{\pgfqpoint{5.546212in}{4.014950in}}{\pgfqpoint{5.546212in}{4.026000in}}%
\pgfpathcurveto{\pgfqpoint{5.546212in}{4.037050in}}{\pgfqpoint{5.541822in}{4.047649in}}{\pgfqpoint{5.534008in}{4.055463in}}%
\pgfpathcurveto{\pgfqpoint{5.526195in}{4.063276in}}{\pgfqpoint{5.515596in}{4.067667in}}{\pgfqpoint{5.504545in}{4.067667in}}%
\pgfpathcurveto{\pgfqpoint{5.493495in}{4.067667in}}{\pgfqpoint{5.482896in}{4.063276in}}{\pgfqpoint{5.475083in}{4.055463in}}%
\pgfpathcurveto{\pgfqpoint{5.467269in}{4.047649in}}{\pgfqpoint{5.462879in}{4.037050in}}{\pgfqpoint{5.462879in}{4.026000in}}%
\pgfpathcurveto{\pgfqpoint{5.462879in}{4.014950in}}{\pgfqpoint{5.467269in}{4.004351in}}{\pgfqpoint{5.475083in}{3.996537in}}%
\pgfpathcurveto{\pgfqpoint{5.482896in}{3.988724in}}{\pgfqpoint{5.493495in}{3.984333in}}{\pgfqpoint{5.504545in}{3.984333in}}%
\pgfpathclose%
\pgfusepath{stroke,fill}%
\end{pgfscope}%
\begin{pgfscope}%
\pgfpathrectangle{\pgfqpoint{0.800000in}{0.528000in}}{\pgfqpoint{4.960000in}{3.696000in}}%
\pgfusepath{clip}%
\pgfsetbuttcap%
\pgfsetroundjoin%
\definecolor{currentfill}{rgb}{0.000000,0.000000,0.000000}%
\pgfsetfillcolor{currentfill}%
\pgfsetlinewidth{1.003750pt}%
\definecolor{currentstroke}{rgb}{0.000000,0.000000,0.000000}%
\pgfsetstrokecolor{currentstroke}%
\pgfsetdash{}{0pt}%
\pgfpathmoveto{\pgfqpoint{5.504545in}{2.334266in}}%
\pgfpathcurveto{\pgfqpoint{5.515596in}{2.334266in}}{\pgfqpoint{5.526195in}{2.338657in}}{\pgfqpoint{5.534008in}{2.346470in}}%
\pgfpathcurveto{\pgfqpoint{5.541822in}{2.354284in}}{\pgfqpoint{5.546212in}{2.364883in}}{\pgfqpoint{5.546212in}{2.375933in}}%
\pgfpathcurveto{\pgfqpoint{5.546212in}{2.386983in}}{\pgfqpoint{5.541822in}{2.397582in}}{\pgfqpoint{5.534008in}{2.405396in}}%
\pgfpathcurveto{\pgfqpoint{5.526195in}{2.413209in}}{\pgfqpoint{5.515596in}{2.417600in}}{\pgfqpoint{5.504545in}{2.417600in}}%
\pgfpathcurveto{\pgfqpoint{5.493495in}{2.417600in}}{\pgfqpoint{5.482896in}{2.413209in}}{\pgfqpoint{5.475083in}{2.405396in}}%
\pgfpathcurveto{\pgfqpoint{5.467269in}{2.397582in}}{\pgfqpoint{5.462879in}{2.386983in}}{\pgfqpoint{5.462879in}{2.375933in}}%
\pgfpathcurveto{\pgfqpoint{5.462879in}{2.364883in}}{\pgfqpoint{5.467269in}{2.354284in}}{\pgfqpoint{5.475083in}{2.346470in}}%
\pgfpathcurveto{\pgfqpoint{5.482896in}{2.338657in}}{\pgfqpoint{5.493495in}{2.334266in}}{\pgfqpoint{5.504545in}{2.334266in}}%
\pgfpathclose%
\pgfusepath{stroke,fill}%
\end{pgfscope}%
\begin{pgfscope}%
\pgfpathrectangle{\pgfqpoint{0.800000in}{0.528000in}}{\pgfqpoint{4.960000in}{3.696000in}}%
\pgfusepath{clip}%
\pgfsetbuttcap%
\pgfsetroundjoin%
\definecolor{currentfill}{rgb}{0.000000,0.000000,0.000000}%
\pgfsetfillcolor{currentfill}%
\pgfsetlinewidth{1.003750pt}%
\definecolor{currentstroke}{rgb}{0.000000,0.000000,0.000000}%
\pgfsetstrokecolor{currentstroke}%
\pgfsetdash{}{0pt}%
\pgfpathmoveto{\pgfqpoint{5.504545in}{2.334266in}}%
\pgfpathcurveto{\pgfqpoint{5.515596in}{2.334266in}}{\pgfqpoint{5.526195in}{2.338657in}}{\pgfqpoint{5.534008in}{2.346470in}}%
\pgfpathcurveto{\pgfqpoint{5.541822in}{2.354284in}}{\pgfqpoint{5.546212in}{2.364883in}}{\pgfqpoint{5.546212in}{2.375933in}}%
\pgfpathcurveto{\pgfqpoint{5.546212in}{2.386983in}}{\pgfqpoint{5.541822in}{2.397582in}}{\pgfqpoint{5.534008in}{2.405396in}}%
\pgfpathcurveto{\pgfqpoint{5.526195in}{2.413209in}}{\pgfqpoint{5.515596in}{2.417600in}}{\pgfqpoint{5.504545in}{2.417600in}}%
\pgfpathcurveto{\pgfqpoint{5.493495in}{2.417600in}}{\pgfqpoint{5.482896in}{2.413209in}}{\pgfqpoint{5.475083in}{2.405396in}}%
\pgfpathcurveto{\pgfqpoint{5.467269in}{2.397582in}}{\pgfqpoint{5.462879in}{2.386983in}}{\pgfqpoint{5.462879in}{2.375933in}}%
\pgfpathcurveto{\pgfqpoint{5.462879in}{2.364883in}}{\pgfqpoint{5.467269in}{2.354284in}}{\pgfqpoint{5.475083in}{2.346470in}}%
\pgfpathcurveto{\pgfqpoint{5.482896in}{2.338657in}}{\pgfqpoint{5.493495in}{2.334266in}}{\pgfqpoint{5.504545in}{2.334266in}}%
\pgfpathclose%
\pgfusepath{stroke,fill}%
\end{pgfscope}%
\begin{pgfscope}%
\pgfpathrectangle{\pgfqpoint{0.800000in}{0.528000in}}{\pgfqpoint{4.960000in}{3.696000in}}%
\pgfusepath{clip}%
\pgfsetbuttcap%
\pgfsetroundjoin%
\definecolor{currentfill}{rgb}{0.000000,0.000000,0.000000}%
\pgfsetfillcolor{currentfill}%
\pgfsetlinewidth{1.003750pt}%
\definecolor{currentstroke}{rgb}{0.000000,0.000000,0.000000}%
\pgfsetstrokecolor{currentstroke}%
\pgfsetdash{}{0pt}%
\pgfpathmoveto{\pgfqpoint{5.504545in}{3.984333in}}%
\pgfpathcurveto{\pgfqpoint{5.515596in}{3.984333in}}{\pgfqpoint{5.526195in}{3.988724in}}{\pgfqpoint{5.534008in}{3.996537in}}%
\pgfpathcurveto{\pgfqpoint{5.541822in}{4.004351in}}{\pgfqpoint{5.546212in}{4.014950in}}{\pgfqpoint{5.546212in}{4.026000in}}%
\pgfpathcurveto{\pgfqpoint{5.546212in}{4.037050in}}{\pgfqpoint{5.541822in}{4.047649in}}{\pgfqpoint{5.534008in}{4.055463in}}%
\pgfpathcurveto{\pgfqpoint{5.526195in}{4.063276in}}{\pgfqpoint{5.515596in}{4.067667in}}{\pgfqpoint{5.504545in}{4.067667in}}%
\pgfpathcurveto{\pgfqpoint{5.493495in}{4.067667in}}{\pgfqpoint{5.482896in}{4.063276in}}{\pgfqpoint{5.475083in}{4.055463in}}%
\pgfpathcurveto{\pgfqpoint{5.467269in}{4.047649in}}{\pgfqpoint{5.462879in}{4.037050in}}{\pgfqpoint{5.462879in}{4.026000in}}%
\pgfpathcurveto{\pgfqpoint{5.462879in}{4.014950in}}{\pgfqpoint{5.467269in}{4.004351in}}{\pgfqpoint{5.475083in}{3.996537in}}%
\pgfpathcurveto{\pgfqpoint{5.482896in}{3.988724in}}{\pgfqpoint{5.493495in}{3.984333in}}{\pgfqpoint{5.504545in}{3.984333in}}%
\pgfpathclose%
\pgfusepath{stroke,fill}%
\end{pgfscope}%
\begin{pgfscope}%
\pgfpathrectangle{\pgfqpoint{0.800000in}{0.528000in}}{\pgfqpoint{4.960000in}{3.696000in}}%
\pgfusepath{clip}%
\pgfsetbuttcap%
\pgfsetroundjoin%
\definecolor{currentfill}{rgb}{0.000000,0.000000,0.000000}%
\pgfsetfillcolor{currentfill}%
\pgfsetlinewidth{1.003750pt}%
\definecolor{currentstroke}{rgb}{0.000000,0.000000,0.000000}%
\pgfsetstrokecolor{currentstroke}%
\pgfsetdash{}{0pt}%
\pgfpathmoveto{\pgfqpoint{5.504545in}{2.334266in}}%
\pgfpathcurveto{\pgfqpoint{5.515596in}{2.334266in}}{\pgfqpoint{5.526195in}{2.338657in}}{\pgfqpoint{5.534008in}{2.346470in}}%
\pgfpathcurveto{\pgfqpoint{5.541822in}{2.354284in}}{\pgfqpoint{5.546212in}{2.364883in}}{\pgfqpoint{5.546212in}{2.375933in}}%
\pgfpathcurveto{\pgfqpoint{5.546212in}{2.386983in}}{\pgfqpoint{5.541822in}{2.397582in}}{\pgfqpoint{5.534008in}{2.405396in}}%
\pgfpathcurveto{\pgfqpoint{5.526195in}{2.413209in}}{\pgfqpoint{5.515596in}{2.417600in}}{\pgfqpoint{5.504545in}{2.417600in}}%
\pgfpathcurveto{\pgfqpoint{5.493495in}{2.417600in}}{\pgfqpoint{5.482896in}{2.413209in}}{\pgfqpoint{5.475083in}{2.405396in}}%
\pgfpathcurveto{\pgfqpoint{5.467269in}{2.397582in}}{\pgfqpoint{5.462879in}{2.386983in}}{\pgfqpoint{5.462879in}{2.375933in}}%
\pgfpathcurveto{\pgfqpoint{5.462879in}{2.364883in}}{\pgfqpoint{5.467269in}{2.354284in}}{\pgfqpoint{5.475083in}{2.346470in}}%
\pgfpathcurveto{\pgfqpoint{5.482896in}{2.338657in}}{\pgfqpoint{5.493495in}{2.334266in}}{\pgfqpoint{5.504545in}{2.334266in}}%
\pgfpathclose%
\pgfusepath{stroke,fill}%
\end{pgfscope}%
\begin{pgfscope}%
\pgfpathrectangle{\pgfqpoint{0.800000in}{0.528000in}}{\pgfqpoint{4.960000in}{3.696000in}}%
\pgfusepath{clip}%
\pgfsetbuttcap%
\pgfsetroundjoin%
\definecolor{currentfill}{rgb}{0.000000,0.000000,0.000000}%
\pgfsetfillcolor{currentfill}%
\pgfsetlinewidth{1.003750pt}%
\definecolor{currentstroke}{rgb}{0.000000,0.000000,0.000000}%
\pgfsetstrokecolor{currentstroke}%
\pgfsetdash{}{0pt}%
\pgfpathmoveto{\pgfqpoint{5.504545in}{3.984333in}}%
\pgfpathcurveto{\pgfqpoint{5.515596in}{3.984333in}}{\pgfqpoint{5.526195in}{3.988724in}}{\pgfqpoint{5.534008in}{3.996537in}}%
\pgfpathcurveto{\pgfqpoint{5.541822in}{4.004351in}}{\pgfqpoint{5.546212in}{4.014950in}}{\pgfqpoint{5.546212in}{4.026000in}}%
\pgfpathcurveto{\pgfqpoint{5.546212in}{4.037050in}}{\pgfqpoint{5.541822in}{4.047649in}}{\pgfqpoint{5.534008in}{4.055463in}}%
\pgfpathcurveto{\pgfqpoint{5.526195in}{4.063276in}}{\pgfqpoint{5.515596in}{4.067667in}}{\pgfqpoint{5.504545in}{4.067667in}}%
\pgfpathcurveto{\pgfqpoint{5.493495in}{4.067667in}}{\pgfqpoint{5.482896in}{4.063276in}}{\pgfqpoint{5.475083in}{4.055463in}}%
\pgfpathcurveto{\pgfqpoint{5.467269in}{4.047649in}}{\pgfqpoint{5.462879in}{4.037050in}}{\pgfqpoint{5.462879in}{4.026000in}}%
\pgfpathcurveto{\pgfqpoint{5.462879in}{4.014950in}}{\pgfqpoint{5.467269in}{4.004351in}}{\pgfqpoint{5.475083in}{3.996537in}}%
\pgfpathcurveto{\pgfqpoint{5.482896in}{3.988724in}}{\pgfqpoint{5.493495in}{3.984333in}}{\pgfqpoint{5.504545in}{3.984333in}}%
\pgfpathclose%
\pgfusepath{stroke,fill}%
\end{pgfscope}%
\begin{pgfscope}%
\pgfpathrectangle{\pgfqpoint{0.800000in}{0.528000in}}{\pgfqpoint{4.960000in}{3.696000in}}%
\pgfusepath{clip}%
\pgfsetbuttcap%
\pgfsetroundjoin%
\definecolor{currentfill}{rgb}{0.000000,0.000000,0.000000}%
\pgfsetfillcolor{currentfill}%
\pgfsetlinewidth{1.003750pt}%
\definecolor{currentstroke}{rgb}{0.000000,0.000000,0.000000}%
\pgfsetstrokecolor{currentstroke}%
\pgfsetdash{}{0pt}%
\pgfpathmoveto{\pgfqpoint{5.504545in}{2.334266in}}%
\pgfpathcurveto{\pgfqpoint{5.515596in}{2.334266in}}{\pgfqpoint{5.526195in}{2.338657in}}{\pgfqpoint{5.534008in}{2.346470in}}%
\pgfpathcurveto{\pgfqpoint{5.541822in}{2.354284in}}{\pgfqpoint{5.546212in}{2.364883in}}{\pgfqpoint{5.546212in}{2.375933in}}%
\pgfpathcurveto{\pgfqpoint{5.546212in}{2.386983in}}{\pgfqpoint{5.541822in}{2.397582in}}{\pgfqpoint{5.534008in}{2.405396in}}%
\pgfpathcurveto{\pgfqpoint{5.526195in}{2.413209in}}{\pgfqpoint{5.515596in}{2.417600in}}{\pgfqpoint{5.504545in}{2.417600in}}%
\pgfpathcurveto{\pgfqpoint{5.493495in}{2.417600in}}{\pgfqpoint{5.482896in}{2.413209in}}{\pgfqpoint{5.475083in}{2.405396in}}%
\pgfpathcurveto{\pgfqpoint{5.467269in}{2.397582in}}{\pgfqpoint{5.462879in}{2.386983in}}{\pgfqpoint{5.462879in}{2.375933in}}%
\pgfpathcurveto{\pgfqpoint{5.462879in}{2.364883in}}{\pgfqpoint{5.467269in}{2.354284in}}{\pgfqpoint{5.475083in}{2.346470in}}%
\pgfpathcurveto{\pgfqpoint{5.482896in}{2.338657in}}{\pgfqpoint{5.493495in}{2.334266in}}{\pgfqpoint{5.504545in}{2.334266in}}%
\pgfpathclose%
\pgfusepath{stroke,fill}%
\end{pgfscope}%
\begin{pgfscope}%
\pgfpathrectangle{\pgfqpoint{0.800000in}{0.528000in}}{\pgfqpoint{4.960000in}{3.696000in}}%
\pgfusepath{clip}%
\pgfsetbuttcap%
\pgfsetroundjoin%
\definecolor{currentfill}{rgb}{0.000000,0.000000,0.000000}%
\pgfsetfillcolor{currentfill}%
\pgfsetlinewidth{1.003750pt}%
\definecolor{currentstroke}{rgb}{0.000000,0.000000,0.000000}%
\pgfsetstrokecolor{currentstroke}%
\pgfsetdash{}{0pt}%
\pgfpathmoveto{\pgfqpoint{5.504545in}{2.334266in}}%
\pgfpathcurveto{\pgfqpoint{5.515596in}{2.334266in}}{\pgfqpoint{5.526195in}{2.338657in}}{\pgfqpoint{5.534008in}{2.346470in}}%
\pgfpathcurveto{\pgfqpoint{5.541822in}{2.354284in}}{\pgfqpoint{5.546212in}{2.364883in}}{\pgfqpoint{5.546212in}{2.375933in}}%
\pgfpathcurveto{\pgfqpoint{5.546212in}{2.386983in}}{\pgfqpoint{5.541822in}{2.397582in}}{\pgfqpoint{5.534008in}{2.405396in}}%
\pgfpathcurveto{\pgfqpoint{5.526195in}{2.413209in}}{\pgfqpoint{5.515596in}{2.417600in}}{\pgfqpoint{5.504545in}{2.417600in}}%
\pgfpathcurveto{\pgfqpoint{5.493495in}{2.417600in}}{\pgfqpoint{5.482896in}{2.413209in}}{\pgfqpoint{5.475083in}{2.405396in}}%
\pgfpathcurveto{\pgfqpoint{5.467269in}{2.397582in}}{\pgfqpoint{5.462879in}{2.386983in}}{\pgfqpoint{5.462879in}{2.375933in}}%
\pgfpathcurveto{\pgfqpoint{5.462879in}{2.364883in}}{\pgfqpoint{5.467269in}{2.354284in}}{\pgfqpoint{5.475083in}{2.346470in}}%
\pgfpathcurveto{\pgfqpoint{5.482896in}{2.338657in}}{\pgfqpoint{5.493495in}{2.334266in}}{\pgfqpoint{5.504545in}{2.334266in}}%
\pgfpathclose%
\pgfusepath{stroke,fill}%
\end{pgfscope}%
\begin{pgfscope}%
\pgfpathrectangle{\pgfqpoint{0.800000in}{0.528000in}}{\pgfqpoint{4.960000in}{3.696000in}}%
\pgfusepath{clip}%
\pgfsetbuttcap%
\pgfsetroundjoin%
\definecolor{currentfill}{rgb}{0.000000,0.000000,0.000000}%
\pgfsetfillcolor{currentfill}%
\pgfsetlinewidth{1.003750pt}%
\definecolor{currentstroke}{rgb}{0.000000,0.000000,0.000000}%
\pgfsetstrokecolor{currentstroke}%
\pgfsetdash{}{0pt}%
\pgfpathmoveto{\pgfqpoint{5.504545in}{3.984333in}}%
\pgfpathcurveto{\pgfqpoint{5.515596in}{3.984333in}}{\pgfqpoint{5.526195in}{3.988724in}}{\pgfqpoint{5.534008in}{3.996537in}}%
\pgfpathcurveto{\pgfqpoint{5.541822in}{4.004351in}}{\pgfqpoint{5.546212in}{4.014950in}}{\pgfqpoint{5.546212in}{4.026000in}}%
\pgfpathcurveto{\pgfqpoint{5.546212in}{4.037050in}}{\pgfqpoint{5.541822in}{4.047649in}}{\pgfqpoint{5.534008in}{4.055463in}}%
\pgfpathcurveto{\pgfqpoint{5.526195in}{4.063276in}}{\pgfqpoint{5.515596in}{4.067667in}}{\pgfqpoint{5.504545in}{4.067667in}}%
\pgfpathcurveto{\pgfqpoint{5.493495in}{4.067667in}}{\pgfqpoint{5.482896in}{4.063276in}}{\pgfqpoint{5.475083in}{4.055463in}}%
\pgfpathcurveto{\pgfqpoint{5.467269in}{4.047649in}}{\pgfqpoint{5.462879in}{4.037050in}}{\pgfqpoint{5.462879in}{4.026000in}}%
\pgfpathcurveto{\pgfqpoint{5.462879in}{4.014950in}}{\pgfqpoint{5.467269in}{4.004351in}}{\pgfqpoint{5.475083in}{3.996537in}}%
\pgfpathcurveto{\pgfqpoint{5.482896in}{3.988724in}}{\pgfqpoint{5.493495in}{3.984333in}}{\pgfqpoint{5.504545in}{3.984333in}}%
\pgfpathclose%
\pgfusepath{stroke,fill}%
\end{pgfscope}%
\begin{pgfscope}%
\pgfpathrectangle{\pgfqpoint{0.800000in}{0.528000in}}{\pgfqpoint{4.960000in}{3.696000in}}%
\pgfusepath{clip}%
\pgfsetbuttcap%
\pgfsetroundjoin%
\definecolor{currentfill}{rgb}{0.000000,0.000000,0.000000}%
\pgfsetfillcolor{currentfill}%
\pgfsetlinewidth{1.003750pt}%
\definecolor{currentstroke}{rgb}{0.000000,0.000000,0.000000}%
\pgfsetstrokecolor{currentstroke}%
\pgfsetdash{}{0pt}%
\pgfpathmoveto{\pgfqpoint{5.504545in}{2.334266in}}%
\pgfpathcurveto{\pgfqpoint{5.515596in}{2.334266in}}{\pgfqpoint{5.526195in}{2.338657in}}{\pgfqpoint{5.534008in}{2.346470in}}%
\pgfpathcurveto{\pgfqpoint{5.541822in}{2.354284in}}{\pgfqpoint{5.546212in}{2.364883in}}{\pgfqpoint{5.546212in}{2.375933in}}%
\pgfpathcurveto{\pgfqpoint{5.546212in}{2.386983in}}{\pgfqpoint{5.541822in}{2.397582in}}{\pgfqpoint{5.534008in}{2.405396in}}%
\pgfpathcurveto{\pgfqpoint{5.526195in}{2.413209in}}{\pgfqpoint{5.515596in}{2.417600in}}{\pgfqpoint{5.504545in}{2.417600in}}%
\pgfpathcurveto{\pgfqpoint{5.493495in}{2.417600in}}{\pgfqpoint{5.482896in}{2.413209in}}{\pgfqpoint{5.475083in}{2.405396in}}%
\pgfpathcurveto{\pgfqpoint{5.467269in}{2.397582in}}{\pgfqpoint{5.462879in}{2.386983in}}{\pgfqpoint{5.462879in}{2.375933in}}%
\pgfpathcurveto{\pgfqpoint{5.462879in}{2.364883in}}{\pgfqpoint{5.467269in}{2.354284in}}{\pgfqpoint{5.475083in}{2.346470in}}%
\pgfpathcurveto{\pgfqpoint{5.482896in}{2.338657in}}{\pgfqpoint{5.493495in}{2.334266in}}{\pgfqpoint{5.504545in}{2.334266in}}%
\pgfpathclose%
\pgfusepath{stroke,fill}%
\end{pgfscope}%
\begin{pgfscope}%
\pgfpathrectangle{\pgfqpoint{0.800000in}{0.528000in}}{\pgfqpoint{4.960000in}{3.696000in}}%
\pgfusepath{clip}%
\pgfsetbuttcap%
\pgfsetroundjoin%
\definecolor{currentfill}{rgb}{0.000000,0.000000,0.000000}%
\pgfsetfillcolor{currentfill}%
\pgfsetlinewidth{1.003750pt}%
\definecolor{currentstroke}{rgb}{0.000000,0.000000,0.000000}%
\pgfsetstrokecolor{currentstroke}%
\pgfsetdash{}{0pt}%
\pgfpathmoveto{\pgfqpoint{5.504545in}{3.984333in}}%
\pgfpathcurveto{\pgfqpoint{5.515596in}{3.984333in}}{\pgfqpoint{5.526195in}{3.988724in}}{\pgfqpoint{5.534008in}{3.996537in}}%
\pgfpathcurveto{\pgfqpoint{5.541822in}{4.004351in}}{\pgfqpoint{5.546212in}{4.014950in}}{\pgfqpoint{5.546212in}{4.026000in}}%
\pgfpathcurveto{\pgfqpoint{5.546212in}{4.037050in}}{\pgfqpoint{5.541822in}{4.047649in}}{\pgfqpoint{5.534008in}{4.055463in}}%
\pgfpathcurveto{\pgfqpoint{5.526195in}{4.063276in}}{\pgfqpoint{5.515596in}{4.067667in}}{\pgfqpoint{5.504545in}{4.067667in}}%
\pgfpathcurveto{\pgfqpoint{5.493495in}{4.067667in}}{\pgfqpoint{5.482896in}{4.063276in}}{\pgfqpoint{5.475083in}{4.055463in}}%
\pgfpathcurveto{\pgfqpoint{5.467269in}{4.047649in}}{\pgfqpoint{5.462879in}{4.037050in}}{\pgfqpoint{5.462879in}{4.026000in}}%
\pgfpathcurveto{\pgfqpoint{5.462879in}{4.014950in}}{\pgfqpoint{5.467269in}{4.004351in}}{\pgfqpoint{5.475083in}{3.996537in}}%
\pgfpathcurveto{\pgfqpoint{5.482896in}{3.988724in}}{\pgfqpoint{5.493495in}{3.984333in}}{\pgfqpoint{5.504545in}{3.984333in}}%
\pgfpathclose%
\pgfusepath{stroke,fill}%
\end{pgfscope}%
\begin{pgfscope}%
\pgfpathrectangle{\pgfqpoint{0.800000in}{0.528000in}}{\pgfqpoint{4.960000in}{3.696000in}}%
\pgfusepath{clip}%
\pgfsetbuttcap%
\pgfsetroundjoin%
\definecolor{currentfill}{rgb}{0.000000,0.000000,0.000000}%
\pgfsetfillcolor{currentfill}%
\pgfsetlinewidth{1.003750pt}%
\definecolor{currentstroke}{rgb}{0.000000,0.000000,0.000000}%
\pgfsetstrokecolor{currentstroke}%
\pgfsetdash{}{0pt}%
\pgfpathmoveto{\pgfqpoint{5.504545in}{2.334266in}}%
\pgfpathcurveto{\pgfqpoint{5.515596in}{2.334266in}}{\pgfqpoint{5.526195in}{2.338657in}}{\pgfqpoint{5.534008in}{2.346470in}}%
\pgfpathcurveto{\pgfqpoint{5.541822in}{2.354284in}}{\pgfqpoint{5.546212in}{2.364883in}}{\pgfqpoint{5.546212in}{2.375933in}}%
\pgfpathcurveto{\pgfqpoint{5.546212in}{2.386983in}}{\pgfqpoint{5.541822in}{2.397582in}}{\pgfqpoint{5.534008in}{2.405396in}}%
\pgfpathcurveto{\pgfqpoint{5.526195in}{2.413209in}}{\pgfqpoint{5.515596in}{2.417600in}}{\pgfqpoint{5.504545in}{2.417600in}}%
\pgfpathcurveto{\pgfqpoint{5.493495in}{2.417600in}}{\pgfqpoint{5.482896in}{2.413209in}}{\pgfqpoint{5.475083in}{2.405396in}}%
\pgfpathcurveto{\pgfqpoint{5.467269in}{2.397582in}}{\pgfqpoint{5.462879in}{2.386983in}}{\pgfqpoint{5.462879in}{2.375933in}}%
\pgfpathcurveto{\pgfqpoint{5.462879in}{2.364883in}}{\pgfqpoint{5.467269in}{2.354284in}}{\pgfqpoint{5.475083in}{2.346470in}}%
\pgfpathcurveto{\pgfqpoint{5.482896in}{2.338657in}}{\pgfqpoint{5.493495in}{2.334266in}}{\pgfqpoint{5.504545in}{2.334266in}}%
\pgfpathclose%
\pgfusepath{stroke,fill}%
\end{pgfscope}%
\begin{pgfscope}%
\pgfpathrectangle{\pgfqpoint{0.800000in}{0.528000in}}{\pgfqpoint{4.960000in}{3.696000in}}%
\pgfusepath{clip}%
\pgfsetbuttcap%
\pgfsetroundjoin%
\definecolor{currentfill}{rgb}{0.000000,0.000000,0.000000}%
\pgfsetfillcolor{currentfill}%
\pgfsetlinewidth{1.003750pt}%
\definecolor{currentstroke}{rgb}{0.000000,0.000000,0.000000}%
\pgfsetstrokecolor{currentstroke}%
\pgfsetdash{}{0pt}%
\pgfpathmoveto{\pgfqpoint{5.504545in}{2.334266in}}%
\pgfpathcurveto{\pgfqpoint{5.515596in}{2.334266in}}{\pgfqpoint{5.526195in}{2.338657in}}{\pgfqpoint{5.534008in}{2.346470in}}%
\pgfpathcurveto{\pgfqpoint{5.541822in}{2.354284in}}{\pgfqpoint{5.546212in}{2.364883in}}{\pgfqpoint{5.546212in}{2.375933in}}%
\pgfpathcurveto{\pgfqpoint{5.546212in}{2.386983in}}{\pgfqpoint{5.541822in}{2.397582in}}{\pgfqpoint{5.534008in}{2.405396in}}%
\pgfpathcurveto{\pgfqpoint{5.526195in}{2.413209in}}{\pgfqpoint{5.515596in}{2.417600in}}{\pgfqpoint{5.504545in}{2.417600in}}%
\pgfpathcurveto{\pgfqpoint{5.493495in}{2.417600in}}{\pgfqpoint{5.482896in}{2.413209in}}{\pgfqpoint{5.475083in}{2.405396in}}%
\pgfpathcurveto{\pgfqpoint{5.467269in}{2.397582in}}{\pgfqpoint{5.462879in}{2.386983in}}{\pgfqpoint{5.462879in}{2.375933in}}%
\pgfpathcurveto{\pgfqpoint{5.462879in}{2.364883in}}{\pgfqpoint{5.467269in}{2.354284in}}{\pgfqpoint{5.475083in}{2.346470in}}%
\pgfpathcurveto{\pgfqpoint{5.482896in}{2.338657in}}{\pgfqpoint{5.493495in}{2.334266in}}{\pgfqpoint{5.504545in}{2.334266in}}%
\pgfpathclose%
\pgfusepath{stroke,fill}%
\end{pgfscope}%
\begin{pgfscope}%
\pgfpathrectangle{\pgfqpoint{0.800000in}{0.528000in}}{\pgfqpoint{4.960000in}{3.696000in}}%
\pgfusepath{clip}%
\pgfsetbuttcap%
\pgfsetroundjoin%
\definecolor{currentfill}{rgb}{0.000000,0.000000,0.000000}%
\pgfsetfillcolor{currentfill}%
\pgfsetlinewidth{1.003750pt}%
\definecolor{currentstroke}{rgb}{0.000000,0.000000,0.000000}%
\pgfsetstrokecolor{currentstroke}%
\pgfsetdash{}{0pt}%
\pgfpathmoveto{\pgfqpoint{5.504545in}{2.334266in}}%
\pgfpathcurveto{\pgfqpoint{5.515596in}{2.334266in}}{\pgfqpoint{5.526195in}{2.338657in}}{\pgfqpoint{5.534008in}{2.346470in}}%
\pgfpathcurveto{\pgfqpoint{5.541822in}{2.354284in}}{\pgfqpoint{5.546212in}{2.364883in}}{\pgfqpoint{5.546212in}{2.375933in}}%
\pgfpathcurveto{\pgfqpoint{5.546212in}{2.386983in}}{\pgfqpoint{5.541822in}{2.397582in}}{\pgfqpoint{5.534008in}{2.405396in}}%
\pgfpathcurveto{\pgfqpoint{5.526195in}{2.413209in}}{\pgfqpoint{5.515596in}{2.417600in}}{\pgfqpoint{5.504545in}{2.417600in}}%
\pgfpathcurveto{\pgfqpoint{5.493495in}{2.417600in}}{\pgfqpoint{5.482896in}{2.413209in}}{\pgfqpoint{5.475083in}{2.405396in}}%
\pgfpathcurveto{\pgfqpoint{5.467269in}{2.397582in}}{\pgfqpoint{5.462879in}{2.386983in}}{\pgfqpoint{5.462879in}{2.375933in}}%
\pgfpathcurveto{\pgfqpoint{5.462879in}{2.364883in}}{\pgfqpoint{5.467269in}{2.354284in}}{\pgfqpoint{5.475083in}{2.346470in}}%
\pgfpathcurveto{\pgfqpoint{5.482896in}{2.338657in}}{\pgfqpoint{5.493495in}{2.334266in}}{\pgfqpoint{5.504545in}{2.334266in}}%
\pgfpathclose%
\pgfusepath{stroke,fill}%
\end{pgfscope}%
\begin{pgfscope}%
\pgfpathrectangle{\pgfqpoint{0.800000in}{0.528000in}}{\pgfqpoint{4.960000in}{3.696000in}}%
\pgfusepath{clip}%
\pgfsetbuttcap%
\pgfsetroundjoin%
\definecolor{currentfill}{rgb}{0.000000,0.000000,0.000000}%
\pgfsetfillcolor{currentfill}%
\pgfsetlinewidth{1.003750pt}%
\definecolor{currentstroke}{rgb}{0.000000,0.000000,0.000000}%
\pgfsetstrokecolor{currentstroke}%
\pgfsetdash{}{0pt}%
\pgfpathmoveto{\pgfqpoint{5.504545in}{3.984333in}}%
\pgfpathcurveto{\pgfqpoint{5.515596in}{3.984333in}}{\pgfqpoint{5.526195in}{3.988724in}}{\pgfqpoint{5.534008in}{3.996537in}}%
\pgfpathcurveto{\pgfqpoint{5.541822in}{4.004351in}}{\pgfqpoint{5.546212in}{4.014950in}}{\pgfqpoint{5.546212in}{4.026000in}}%
\pgfpathcurveto{\pgfqpoint{5.546212in}{4.037050in}}{\pgfqpoint{5.541822in}{4.047649in}}{\pgfqpoint{5.534008in}{4.055463in}}%
\pgfpathcurveto{\pgfqpoint{5.526195in}{4.063276in}}{\pgfqpoint{5.515596in}{4.067667in}}{\pgfqpoint{5.504545in}{4.067667in}}%
\pgfpathcurveto{\pgfqpoint{5.493495in}{4.067667in}}{\pgfqpoint{5.482896in}{4.063276in}}{\pgfqpoint{5.475083in}{4.055463in}}%
\pgfpathcurveto{\pgfqpoint{5.467269in}{4.047649in}}{\pgfqpoint{5.462879in}{4.037050in}}{\pgfqpoint{5.462879in}{4.026000in}}%
\pgfpathcurveto{\pgfqpoint{5.462879in}{4.014950in}}{\pgfqpoint{5.467269in}{4.004351in}}{\pgfqpoint{5.475083in}{3.996537in}}%
\pgfpathcurveto{\pgfqpoint{5.482896in}{3.988724in}}{\pgfqpoint{5.493495in}{3.984333in}}{\pgfqpoint{5.504545in}{3.984333in}}%
\pgfpathclose%
\pgfusepath{stroke,fill}%
\end{pgfscope}%
\begin{pgfscope}%
\pgfpathrectangle{\pgfqpoint{0.800000in}{0.528000in}}{\pgfqpoint{4.960000in}{3.696000in}}%
\pgfusepath{clip}%
\pgfsetbuttcap%
\pgfsetroundjoin%
\definecolor{currentfill}{rgb}{0.000000,0.000000,0.000000}%
\pgfsetfillcolor{currentfill}%
\pgfsetlinewidth{1.003750pt}%
\definecolor{currentstroke}{rgb}{0.000000,0.000000,0.000000}%
\pgfsetstrokecolor{currentstroke}%
\pgfsetdash{}{0pt}%
\pgfpathmoveto{\pgfqpoint{5.504545in}{3.984333in}}%
\pgfpathcurveto{\pgfqpoint{5.515596in}{3.984333in}}{\pgfqpoint{5.526195in}{3.988724in}}{\pgfqpoint{5.534008in}{3.996537in}}%
\pgfpathcurveto{\pgfqpoint{5.541822in}{4.004351in}}{\pgfqpoint{5.546212in}{4.014950in}}{\pgfqpoint{5.546212in}{4.026000in}}%
\pgfpathcurveto{\pgfqpoint{5.546212in}{4.037050in}}{\pgfqpoint{5.541822in}{4.047649in}}{\pgfqpoint{5.534008in}{4.055463in}}%
\pgfpathcurveto{\pgfqpoint{5.526195in}{4.063276in}}{\pgfqpoint{5.515596in}{4.067667in}}{\pgfqpoint{5.504545in}{4.067667in}}%
\pgfpathcurveto{\pgfqpoint{5.493495in}{4.067667in}}{\pgfqpoint{5.482896in}{4.063276in}}{\pgfqpoint{5.475083in}{4.055463in}}%
\pgfpathcurveto{\pgfqpoint{5.467269in}{4.047649in}}{\pgfqpoint{5.462879in}{4.037050in}}{\pgfqpoint{5.462879in}{4.026000in}}%
\pgfpathcurveto{\pgfqpoint{5.462879in}{4.014950in}}{\pgfqpoint{5.467269in}{4.004351in}}{\pgfqpoint{5.475083in}{3.996537in}}%
\pgfpathcurveto{\pgfqpoint{5.482896in}{3.988724in}}{\pgfqpoint{5.493495in}{3.984333in}}{\pgfqpoint{5.504545in}{3.984333in}}%
\pgfpathclose%
\pgfusepath{stroke,fill}%
\end{pgfscope}%
\begin{pgfscope}%
\pgfpathrectangle{\pgfqpoint{0.800000in}{0.528000in}}{\pgfqpoint{4.960000in}{3.696000in}}%
\pgfusepath{clip}%
\pgfsetbuttcap%
\pgfsetroundjoin%
\definecolor{currentfill}{rgb}{0.000000,0.000000,0.000000}%
\pgfsetfillcolor{currentfill}%
\pgfsetlinewidth{1.003750pt}%
\definecolor{currentstroke}{rgb}{0.000000,0.000000,0.000000}%
\pgfsetstrokecolor{currentstroke}%
\pgfsetdash{}{0pt}%
\pgfpathmoveto{\pgfqpoint{5.504545in}{2.334266in}}%
\pgfpathcurveto{\pgfqpoint{5.515596in}{2.334266in}}{\pgfqpoint{5.526195in}{2.338657in}}{\pgfqpoint{5.534008in}{2.346470in}}%
\pgfpathcurveto{\pgfqpoint{5.541822in}{2.354284in}}{\pgfqpoint{5.546212in}{2.364883in}}{\pgfqpoint{5.546212in}{2.375933in}}%
\pgfpathcurveto{\pgfqpoint{5.546212in}{2.386983in}}{\pgfqpoint{5.541822in}{2.397582in}}{\pgfqpoint{5.534008in}{2.405396in}}%
\pgfpathcurveto{\pgfqpoint{5.526195in}{2.413209in}}{\pgfqpoint{5.515596in}{2.417600in}}{\pgfqpoint{5.504545in}{2.417600in}}%
\pgfpathcurveto{\pgfqpoint{5.493495in}{2.417600in}}{\pgfqpoint{5.482896in}{2.413209in}}{\pgfqpoint{5.475083in}{2.405396in}}%
\pgfpathcurveto{\pgfqpoint{5.467269in}{2.397582in}}{\pgfqpoint{5.462879in}{2.386983in}}{\pgfqpoint{5.462879in}{2.375933in}}%
\pgfpathcurveto{\pgfqpoint{5.462879in}{2.364883in}}{\pgfqpoint{5.467269in}{2.354284in}}{\pgfqpoint{5.475083in}{2.346470in}}%
\pgfpathcurveto{\pgfqpoint{5.482896in}{2.338657in}}{\pgfqpoint{5.493495in}{2.334266in}}{\pgfqpoint{5.504545in}{2.334266in}}%
\pgfpathclose%
\pgfusepath{stroke,fill}%
\end{pgfscope}%
\begin{pgfscope}%
\pgfpathrectangle{\pgfqpoint{0.800000in}{0.528000in}}{\pgfqpoint{4.960000in}{3.696000in}}%
\pgfusepath{clip}%
\pgfsetbuttcap%
\pgfsetroundjoin%
\definecolor{currentfill}{rgb}{0.000000,0.000000,0.000000}%
\pgfsetfillcolor{currentfill}%
\pgfsetlinewidth{1.003750pt}%
\definecolor{currentstroke}{rgb}{0.000000,0.000000,0.000000}%
\pgfsetstrokecolor{currentstroke}%
\pgfsetdash{}{0pt}%
\pgfpathmoveto{\pgfqpoint{5.504545in}{3.984333in}}%
\pgfpathcurveto{\pgfqpoint{5.515596in}{3.984333in}}{\pgfqpoint{5.526195in}{3.988724in}}{\pgfqpoint{5.534008in}{3.996537in}}%
\pgfpathcurveto{\pgfqpoint{5.541822in}{4.004351in}}{\pgfqpoint{5.546212in}{4.014950in}}{\pgfqpoint{5.546212in}{4.026000in}}%
\pgfpathcurveto{\pgfqpoint{5.546212in}{4.037050in}}{\pgfqpoint{5.541822in}{4.047649in}}{\pgfqpoint{5.534008in}{4.055463in}}%
\pgfpathcurveto{\pgfqpoint{5.526195in}{4.063276in}}{\pgfqpoint{5.515596in}{4.067667in}}{\pgfqpoint{5.504545in}{4.067667in}}%
\pgfpathcurveto{\pgfqpoint{5.493495in}{4.067667in}}{\pgfqpoint{5.482896in}{4.063276in}}{\pgfqpoint{5.475083in}{4.055463in}}%
\pgfpathcurveto{\pgfqpoint{5.467269in}{4.047649in}}{\pgfqpoint{5.462879in}{4.037050in}}{\pgfqpoint{5.462879in}{4.026000in}}%
\pgfpathcurveto{\pgfqpoint{5.462879in}{4.014950in}}{\pgfqpoint{5.467269in}{4.004351in}}{\pgfqpoint{5.475083in}{3.996537in}}%
\pgfpathcurveto{\pgfqpoint{5.482896in}{3.988724in}}{\pgfqpoint{5.493495in}{3.984333in}}{\pgfqpoint{5.504545in}{3.984333in}}%
\pgfpathclose%
\pgfusepath{stroke,fill}%
\end{pgfscope}%
\begin{pgfscope}%
\pgfpathrectangle{\pgfqpoint{0.800000in}{0.528000in}}{\pgfqpoint{4.960000in}{3.696000in}}%
\pgfusepath{clip}%
\pgfsetbuttcap%
\pgfsetroundjoin%
\definecolor{currentfill}{rgb}{0.000000,0.000000,0.000000}%
\pgfsetfillcolor{currentfill}%
\pgfsetlinewidth{1.003750pt}%
\definecolor{currentstroke}{rgb}{0.000000,0.000000,0.000000}%
\pgfsetstrokecolor{currentstroke}%
\pgfsetdash{}{0pt}%
\pgfpathmoveto{\pgfqpoint{5.504545in}{2.334266in}}%
\pgfpathcurveto{\pgfqpoint{5.515596in}{2.334266in}}{\pgfqpoint{5.526195in}{2.338657in}}{\pgfqpoint{5.534008in}{2.346470in}}%
\pgfpathcurveto{\pgfqpoint{5.541822in}{2.354284in}}{\pgfqpoint{5.546212in}{2.364883in}}{\pgfqpoint{5.546212in}{2.375933in}}%
\pgfpathcurveto{\pgfqpoint{5.546212in}{2.386983in}}{\pgfqpoint{5.541822in}{2.397582in}}{\pgfqpoint{5.534008in}{2.405396in}}%
\pgfpathcurveto{\pgfqpoint{5.526195in}{2.413209in}}{\pgfqpoint{5.515596in}{2.417600in}}{\pgfqpoint{5.504545in}{2.417600in}}%
\pgfpathcurveto{\pgfqpoint{5.493495in}{2.417600in}}{\pgfqpoint{5.482896in}{2.413209in}}{\pgfqpoint{5.475083in}{2.405396in}}%
\pgfpathcurveto{\pgfqpoint{5.467269in}{2.397582in}}{\pgfqpoint{5.462879in}{2.386983in}}{\pgfqpoint{5.462879in}{2.375933in}}%
\pgfpathcurveto{\pgfqpoint{5.462879in}{2.364883in}}{\pgfqpoint{5.467269in}{2.354284in}}{\pgfqpoint{5.475083in}{2.346470in}}%
\pgfpathcurveto{\pgfqpoint{5.482896in}{2.338657in}}{\pgfqpoint{5.493495in}{2.334266in}}{\pgfqpoint{5.504545in}{2.334266in}}%
\pgfpathclose%
\pgfusepath{stroke,fill}%
\end{pgfscope}%
\begin{pgfscope}%
\pgfpathrectangle{\pgfqpoint{0.800000in}{0.528000in}}{\pgfqpoint{4.960000in}{3.696000in}}%
\pgfusepath{clip}%
\pgfsetbuttcap%
\pgfsetroundjoin%
\definecolor{currentfill}{rgb}{0.000000,0.000000,0.000000}%
\pgfsetfillcolor{currentfill}%
\pgfsetlinewidth{1.003750pt}%
\definecolor{currentstroke}{rgb}{0.000000,0.000000,0.000000}%
\pgfsetstrokecolor{currentstroke}%
\pgfsetdash{}{0pt}%
\pgfpathmoveto{\pgfqpoint{5.504545in}{3.984333in}}%
\pgfpathcurveto{\pgfqpoint{5.515596in}{3.984333in}}{\pgfqpoint{5.526195in}{3.988724in}}{\pgfqpoint{5.534008in}{3.996537in}}%
\pgfpathcurveto{\pgfqpoint{5.541822in}{4.004351in}}{\pgfqpoint{5.546212in}{4.014950in}}{\pgfqpoint{5.546212in}{4.026000in}}%
\pgfpathcurveto{\pgfqpoint{5.546212in}{4.037050in}}{\pgfqpoint{5.541822in}{4.047649in}}{\pgfqpoint{5.534008in}{4.055463in}}%
\pgfpathcurveto{\pgfqpoint{5.526195in}{4.063276in}}{\pgfqpoint{5.515596in}{4.067667in}}{\pgfqpoint{5.504545in}{4.067667in}}%
\pgfpathcurveto{\pgfqpoint{5.493495in}{4.067667in}}{\pgfqpoint{5.482896in}{4.063276in}}{\pgfqpoint{5.475083in}{4.055463in}}%
\pgfpathcurveto{\pgfqpoint{5.467269in}{4.047649in}}{\pgfqpoint{5.462879in}{4.037050in}}{\pgfqpoint{5.462879in}{4.026000in}}%
\pgfpathcurveto{\pgfqpoint{5.462879in}{4.014950in}}{\pgfqpoint{5.467269in}{4.004351in}}{\pgfqpoint{5.475083in}{3.996537in}}%
\pgfpathcurveto{\pgfqpoint{5.482896in}{3.988724in}}{\pgfqpoint{5.493495in}{3.984333in}}{\pgfqpoint{5.504545in}{3.984333in}}%
\pgfpathclose%
\pgfusepath{stroke,fill}%
\end{pgfscope}%
\begin{pgfscope}%
\pgfpathrectangle{\pgfqpoint{0.800000in}{0.528000in}}{\pgfqpoint{4.960000in}{3.696000in}}%
\pgfusepath{clip}%
\pgfsetbuttcap%
\pgfsetroundjoin%
\definecolor{currentfill}{rgb}{0.000000,0.000000,0.000000}%
\pgfsetfillcolor{currentfill}%
\pgfsetlinewidth{1.003750pt}%
\definecolor{currentstroke}{rgb}{0.000000,0.000000,0.000000}%
\pgfsetstrokecolor{currentstroke}%
\pgfsetdash{}{0pt}%
\pgfpathmoveto{\pgfqpoint{5.504545in}{2.334266in}}%
\pgfpathcurveto{\pgfqpoint{5.515596in}{2.334266in}}{\pgfqpoint{5.526195in}{2.338657in}}{\pgfqpoint{5.534008in}{2.346470in}}%
\pgfpathcurveto{\pgfqpoint{5.541822in}{2.354284in}}{\pgfqpoint{5.546212in}{2.364883in}}{\pgfqpoint{5.546212in}{2.375933in}}%
\pgfpathcurveto{\pgfqpoint{5.546212in}{2.386983in}}{\pgfqpoint{5.541822in}{2.397582in}}{\pgfqpoint{5.534008in}{2.405396in}}%
\pgfpathcurveto{\pgfqpoint{5.526195in}{2.413209in}}{\pgfqpoint{5.515596in}{2.417600in}}{\pgfqpoint{5.504545in}{2.417600in}}%
\pgfpathcurveto{\pgfqpoint{5.493495in}{2.417600in}}{\pgfqpoint{5.482896in}{2.413209in}}{\pgfqpoint{5.475083in}{2.405396in}}%
\pgfpathcurveto{\pgfqpoint{5.467269in}{2.397582in}}{\pgfqpoint{5.462879in}{2.386983in}}{\pgfqpoint{5.462879in}{2.375933in}}%
\pgfpathcurveto{\pgfqpoint{5.462879in}{2.364883in}}{\pgfqpoint{5.467269in}{2.354284in}}{\pgfqpoint{5.475083in}{2.346470in}}%
\pgfpathcurveto{\pgfqpoint{5.482896in}{2.338657in}}{\pgfqpoint{5.493495in}{2.334266in}}{\pgfqpoint{5.504545in}{2.334266in}}%
\pgfpathclose%
\pgfusepath{stroke,fill}%
\end{pgfscope}%
\begin{pgfscope}%
\pgfpathrectangle{\pgfqpoint{0.800000in}{0.528000in}}{\pgfqpoint{4.960000in}{3.696000in}}%
\pgfusepath{clip}%
\pgfsetbuttcap%
\pgfsetroundjoin%
\definecolor{currentfill}{rgb}{0.000000,0.000000,0.000000}%
\pgfsetfillcolor{currentfill}%
\pgfsetlinewidth{1.003750pt}%
\definecolor{currentstroke}{rgb}{0.000000,0.000000,0.000000}%
\pgfsetstrokecolor{currentstroke}%
\pgfsetdash{}{0pt}%
\pgfpathmoveto{\pgfqpoint{5.504545in}{3.984333in}}%
\pgfpathcurveto{\pgfqpoint{5.515596in}{3.984333in}}{\pgfqpoint{5.526195in}{3.988724in}}{\pgfqpoint{5.534008in}{3.996537in}}%
\pgfpathcurveto{\pgfqpoint{5.541822in}{4.004351in}}{\pgfqpoint{5.546212in}{4.014950in}}{\pgfqpoint{5.546212in}{4.026000in}}%
\pgfpathcurveto{\pgfqpoint{5.546212in}{4.037050in}}{\pgfqpoint{5.541822in}{4.047649in}}{\pgfqpoint{5.534008in}{4.055463in}}%
\pgfpathcurveto{\pgfqpoint{5.526195in}{4.063276in}}{\pgfqpoint{5.515596in}{4.067667in}}{\pgfqpoint{5.504545in}{4.067667in}}%
\pgfpathcurveto{\pgfqpoint{5.493495in}{4.067667in}}{\pgfqpoint{5.482896in}{4.063276in}}{\pgfqpoint{5.475083in}{4.055463in}}%
\pgfpathcurveto{\pgfqpoint{5.467269in}{4.047649in}}{\pgfqpoint{5.462879in}{4.037050in}}{\pgfqpoint{5.462879in}{4.026000in}}%
\pgfpathcurveto{\pgfqpoint{5.462879in}{4.014950in}}{\pgfqpoint{5.467269in}{4.004351in}}{\pgfqpoint{5.475083in}{3.996537in}}%
\pgfpathcurveto{\pgfqpoint{5.482896in}{3.988724in}}{\pgfqpoint{5.493495in}{3.984333in}}{\pgfqpoint{5.504545in}{3.984333in}}%
\pgfpathclose%
\pgfusepath{stroke,fill}%
\end{pgfscope}%
\begin{pgfscope}%
\pgfpathrectangle{\pgfqpoint{0.800000in}{0.528000in}}{\pgfqpoint{4.960000in}{3.696000in}}%
\pgfusepath{clip}%
\pgfsetbuttcap%
\pgfsetroundjoin%
\definecolor{currentfill}{rgb}{0.000000,0.000000,0.000000}%
\pgfsetfillcolor{currentfill}%
\pgfsetlinewidth{1.003750pt}%
\definecolor{currentstroke}{rgb}{0.000000,0.000000,0.000000}%
\pgfsetstrokecolor{currentstroke}%
\pgfsetdash{}{0pt}%
\pgfpathmoveto{\pgfqpoint{5.504545in}{2.334266in}}%
\pgfpathcurveto{\pgfqpoint{5.515596in}{2.334266in}}{\pgfqpoint{5.526195in}{2.338657in}}{\pgfqpoint{5.534008in}{2.346470in}}%
\pgfpathcurveto{\pgfqpoint{5.541822in}{2.354284in}}{\pgfqpoint{5.546212in}{2.364883in}}{\pgfqpoint{5.546212in}{2.375933in}}%
\pgfpathcurveto{\pgfqpoint{5.546212in}{2.386983in}}{\pgfqpoint{5.541822in}{2.397582in}}{\pgfqpoint{5.534008in}{2.405396in}}%
\pgfpathcurveto{\pgfqpoint{5.526195in}{2.413209in}}{\pgfqpoint{5.515596in}{2.417600in}}{\pgfqpoint{5.504545in}{2.417600in}}%
\pgfpathcurveto{\pgfqpoint{5.493495in}{2.417600in}}{\pgfqpoint{5.482896in}{2.413209in}}{\pgfqpoint{5.475083in}{2.405396in}}%
\pgfpathcurveto{\pgfqpoint{5.467269in}{2.397582in}}{\pgfqpoint{5.462879in}{2.386983in}}{\pgfqpoint{5.462879in}{2.375933in}}%
\pgfpathcurveto{\pgfqpoint{5.462879in}{2.364883in}}{\pgfqpoint{5.467269in}{2.354284in}}{\pgfqpoint{5.475083in}{2.346470in}}%
\pgfpathcurveto{\pgfqpoint{5.482896in}{2.338657in}}{\pgfqpoint{5.493495in}{2.334266in}}{\pgfqpoint{5.504545in}{2.334266in}}%
\pgfpathclose%
\pgfusepath{stroke,fill}%
\end{pgfscope}%
\begin{pgfscope}%
\pgfpathrectangle{\pgfqpoint{0.800000in}{0.528000in}}{\pgfqpoint{4.960000in}{3.696000in}}%
\pgfusepath{clip}%
\pgfsetbuttcap%
\pgfsetroundjoin%
\definecolor{currentfill}{rgb}{0.000000,0.000000,0.000000}%
\pgfsetfillcolor{currentfill}%
\pgfsetlinewidth{1.003750pt}%
\definecolor{currentstroke}{rgb}{0.000000,0.000000,0.000000}%
\pgfsetstrokecolor{currentstroke}%
\pgfsetdash{}{0pt}%
\pgfpathmoveto{\pgfqpoint{5.504545in}{3.984333in}}%
\pgfpathcurveto{\pgfqpoint{5.515596in}{3.984333in}}{\pgfqpoint{5.526195in}{3.988724in}}{\pgfqpoint{5.534008in}{3.996537in}}%
\pgfpathcurveto{\pgfqpoint{5.541822in}{4.004351in}}{\pgfqpoint{5.546212in}{4.014950in}}{\pgfqpoint{5.546212in}{4.026000in}}%
\pgfpathcurveto{\pgfqpoint{5.546212in}{4.037050in}}{\pgfqpoint{5.541822in}{4.047649in}}{\pgfqpoint{5.534008in}{4.055463in}}%
\pgfpathcurveto{\pgfqpoint{5.526195in}{4.063276in}}{\pgfqpoint{5.515596in}{4.067667in}}{\pgfqpoint{5.504545in}{4.067667in}}%
\pgfpathcurveto{\pgfqpoint{5.493495in}{4.067667in}}{\pgfqpoint{5.482896in}{4.063276in}}{\pgfqpoint{5.475083in}{4.055463in}}%
\pgfpathcurveto{\pgfqpoint{5.467269in}{4.047649in}}{\pgfqpoint{5.462879in}{4.037050in}}{\pgfqpoint{5.462879in}{4.026000in}}%
\pgfpathcurveto{\pgfqpoint{5.462879in}{4.014950in}}{\pgfqpoint{5.467269in}{4.004351in}}{\pgfqpoint{5.475083in}{3.996537in}}%
\pgfpathcurveto{\pgfqpoint{5.482896in}{3.988724in}}{\pgfqpoint{5.493495in}{3.984333in}}{\pgfqpoint{5.504545in}{3.984333in}}%
\pgfpathclose%
\pgfusepath{stroke,fill}%
\end{pgfscope}%
\begin{pgfscope}%
\pgfpathrectangle{\pgfqpoint{0.800000in}{0.528000in}}{\pgfqpoint{4.960000in}{3.696000in}}%
\pgfusepath{clip}%
\pgfsetbuttcap%
\pgfsetroundjoin%
\definecolor{currentfill}{rgb}{0.000000,0.000000,0.000000}%
\pgfsetfillcolor{currentfill}%
\pgfsetlinewidth{1.003750pt}%
\definecolor{currentstroke}{rgb}{0.000000,0.000000,0.000000}%
\pgfsetstrokecolor{currentstroke}%
\pgfsetdash{}{0pt}%
\pgfpathmoveto{\pgfqpoint{5.504545in}{2.334266in}}%
\pgfpathcurveto{\pgfqpoint{5.515596in}{2.334266in}}{\pgfqpoint{5.526195in}{2.338657in}}{\pgfqpoint{5.534008in}{2.346470in}}%
\pgfpathcurveto{\pgfqpoint{5.541822in}{2.354284in}}{\pgfqpoint{5.546212in}{2.364883in}}{\pgfqpoint{5.546212in}{2.375933in}}%
\pgfpathcurveto{\pgfqpoint{5.546212in}{2.386983in}}{\pgfqpoint{5.541822in}{2.397582in}}{\pgfqpoint{5.534008in}{2.405396in}}%
\pgfpathcurveto{\pgfqpoint{5.526195in}{2.413209in}}{\pgfqpoint{5.515596in}{2.417600in}}{\pgfqpoint{5.504545in}{2.417600in}}%
\pgfpathcurveto{\pgfqpoint{5.493495in}{2.417600in}}{\pgfqpoint{5.482896in}{2.413209in}}{\pgfqpoint{5.475083in}{2.405396in}}%
\pgfpathcurveto{\pgfqpoint{5.467269in}{2.397582in}}{\pgfqpoint{5.462879in}{2.386983in}}{\pgfqpoint{5.462879in}{2.375933in}}%
\pgfpathcurveto{\pgfqpoint{5.462879in}{2.364883in}}{\pgfqpoint{5.467269in}{2.354284in}}{\pgfqpoint{5.475083in}{2.346470in}}%
\pgfpathcurveto{\pgfqpoint{5.482896in}{2.338657in}}{\pgfqpoint{5.493495in}{2.334266in}}{\pgfqpoint{5.504545in}{2.334266in}}%
\pgfpathclose%
\pgfusepath{stroke,fill}%
\end{pgfscope}%
\begin{pgfscope}%
\pgfpathrectangle{\pgfqpoint{0.800000in}{0.528000in}}{\pgfqpoint{4.960000in}{3.696000in}}%
\pgfusepath{clip}%
\pgfsetbuttcap%
\pgfsetroundjoin%
\definecolor{currentfill}{rgb}{0.000000,0.000000,0.000000}%
\pgfsetfillcolor{currentfill}%
\pgfsetlinewidth{1.003750pt}%
\definecolor{currentstroke}{rgb}{0.000000,0.000000,0.000000}%
\pgfsetstrokecolor{currentstroke}%
\pgfsetdash{}{0pt}%
\pgfpathmoveto{\pgfqpoint{5.504545in}{3.984333in}}%
\pgfpathcurveto{\pgfqpoint{5.515596in}{3.984333in}}{\pgfqpoint{5.526195in}{3.988724in}}{\pgfqpoint{5.534008in}{3.996537in}}%
\pgfpathcurveto{\pgfqpoint{5.541822in}{4.004351in}}{\pgfqpoint{5.546212in}{4.014950in}}{\pgfqpoint{5.546212in}{4.026000in}}%
\pgfpathcurveto{\pgfqpoint{5.546212in}{4.037050in}}{\pgfqpoint{5.541822in}{4.047649in}}{\pgfqpoint{5.534008in}{4.055463in}}%
\pgfpathcurveto{\pgfqpoint{5.526195in}{4.063276in}}{\pgfqpoint{5.515596in}{4.067667in}}{\pgfqpoint{5.504545in}{4.067667in}}%
\pgfpathcurveto{\pgfqpoint{5.493495in}{4.067667in}}{\pgfqpoint{5.482896in}{4.063276in}}{\pgfqpoint{5.475083in}{4.055463in}}%
\pgfpathcurveto{\pgfqpoint{5.467269in}{4.047649in}}{\pgfqpoint{5.462879in}{4.037050in}}{\pgfqpoint{5.462879in}{4.026000in}}%
\pgfpathcurveto{\pgfqpoint{5.462879in}{4.014950in}}{\pgfqpoint{5.467269in}{4.004351in}}{\pgfqpoint{5.475083in}{3.996537in}}%
\pgfpathcurveto{\pgfqpoint{5.482896in}{3.988724in}}{\pgfqpoint{5.493495in}{3.984333in}}{\pgfqpoint{5.504545in}{3.984333in}}%
\pgfpathclose%
\pgfusepath{stroke,fill}%
\end{pgfscope}%
\begin{pgfscope}%
\pgfpathrectangle{\pgfqpoint{0.800000in}{0.528000in}}{\pgfqpoint{4.960000in}{3.696000in}}%
\pgfusepath{clip}%
\pgfsetbuttcap%
\pgfsetroundjoin%
\definecolor{currentfill}{rgb}{0.000000,0.000000,0.000000}%
\pgfsetfillcolor{currentfill}%
\pgfsetlinewidth{1.003750pt}%
\definecolor{currentstroke}{rgb}{0.000000,0.000000,0.000000}%
\pgfsetstrokecolor{currentstroke}%
\pgfsetdash{}{0pt}%
\pgfpathmoveto{\pgfqpoint{5.504545in}{3.984333in}}%
\pgfpathcurveto{\pgfqpoint{5.515596in}{3.984333in}}{\pgfqpoint{5.526195in}{3.988724in}}{\pgfqpoint{5.534008in}{3.996537in}}%
\pgfpathcurveto{\pgfqpoint{5.541822in}{4.004351in}}{\pgfqpoint{5.546212in}{4.014950in}}{\pgfqpoint{5.546212in}{4.026000in}}%
\pgfpathcurveto{\pgfqpoint{5.546212in}{4.037050in}}{\pgfqpoint{5.541822in}{4.047649in}}{\pgfqpoint{5.534008in}{4.055463in}}%
\pgfpathcurveto{\pgfqpoint{5.526195in}{4.063276in}}{\pgfqpoint{5.515596in}{4.067667in}}{\pgfqpoint{5.504545in}{4.067667in}}%
\pgfpathcurveto{\pgfqpoint{5.493495in}{4.067667in}}{\pgfqpoint{5.482896in}{4.063276in}}{\pgfqpoint{5.475083in}{4.055463in}}%
\pgfpathcurveto{\pgfqpoint{5.467269in}{4.047649in}}{\pgfqpoint{5.462879in}{4.037050in}}{\pgfqpoint{5.462879in}{4.026000in}}%
\pgfpathcurveto{\pgfqpoint{5.462879in}{4.014950in}}{\pgfqpoint{5.467269in}{4.004351in}}{\pgfqpoint{5.475083in}{3.996537in}}%
\pgfpathcurveto{\pgfqpoint{5.482896in}{3.988724in}}{\pgfqpoint{5.493495in}{3.984333in}}{\pgfqpoint{5.504545in}{3.984333in}}%
\pgfpathclose%
\pgfusepath{stroke,fill}%
\end{pgfscope}%
\begin{pgfscope}%
\pgfpathrectangle{\pgfqpoint{0.800000in}{0.528000in}}{\pgfqpoint{4.960000in}{3.696000in}}%
\pgfusepath{clip}%
\pgfsetbuttcap%
\pgfsetroundjoin%
\definecolor{currentfill}{rgb}{0.000000,0.000000,0.000000}%
\pgfsetfillcolor{currentfill}%
\pgfsetlinewidth{1.003750pt}%
\definecolor{currentstroke}{rgb}{0.000000,0.000000,0.000000}%
\pgfsetstrokecolor{currentstroke}%
\pgfsetdash{}{0pt}%
\pgfpathmoveto{\pgfqpoint{5.504545in}{3.984333in}}%
\pgfpathcurveto{\pgfqpoint{5.515596in}{3.984333in}}{\pgfqpoint{5.526195in}{3.988724in}}{\pgfqpoint{5.534008in}{3.996537in}}%
\pgfpathcurveto{\pgfqpoint{5.541822in}{4.004351in}}{\pgfqpoint{5.546212in}{4.014950in}}{\pgfqpoint{5.546212in}{4.026000in}}%
\pgfpathcurveto{\pgfqpoint{5.546212in}{4.037050in}}{\pgfqpoint{5.541822in}{4.047649in}}{\pgfqpoint{5.534008in}{4.055463in}}%
\pgfpathcurveto{\pgfqpoint{5.526195in}{4.063276in}}{\pgfqpoint{5.515596in}{4.067667in}}{\pgfqpoint{5.504545in}{4.067667in}}%
\pgfpathcurveto{\pgfqpoint{5.493495in}{4.067667in}}{\pgfqpoint{5.482896in}{4.063276in}}{\pgfqpoint{5.475083in}{4.055463in}}%
\pgfpathcurveto{\pgfqpoint{5.467269in}{4.047649in}}{\pgfqpoint{5.462879in}{4.037050in}}{\pgfqpoint{5.462879in}{4.026000in}}%
\pgfpathcurveto{\pgfqpoint{5.462879in}{4.014950in}}{\pgfqpoint{5.467269in}{4.004351in}}{\pgfqpoint{5.475083in}{3.996537in}}%
\pgfpathcurveto{\pgfqpoint{5.482896in}{3.988724in}}{\pgfqpoint{5.493495in}{3.984333in}}{\pgfqpoint{5.504545in}{3.984333in}}%
\pgfpathclose%
\pgfusepath{stroke,fill}%
\end{pgfscope}%
\begin{pgfscope}%
\pgfpathrectangle{\pgfqpoint{0.800000in}{0.528000in}}{\pgfqpoint{4.960000in}{3.696000in}}%
\pgfusepath{clip}%
\pgfsetbuttcap%
\pgfsetroundjoin%
\definecolor{currentfill}{rgb}{0.000000,0.000000,0.000000}%
\pgfsetfillcolor{currentfill}%
\pgfsetlinewidth{1.003750pt}%
\definecolor{currentstroke}{rgb}{0.000000,0.000000,0.000000}%
\pgfsetstrokecolor{currentstroke}%
\pgfsetdash{}{0pt}%
\pgfpathmoveto{\pgfqpoint{5.504545in}{3.984333in}}%
\pgfpathcurveto{\pgfqpoint{5.515596in}{3.984333in}}{\pgfqpoint{5.526195in}{3.988724in}}{\pgfqpoint{5.534008in}{3.996537in}}%
\pgfpathcurveto{\pgfqpoint{5.541822in}{4.004351in}}{\pgfqpoint{5.546212in}{4.014950in}}{\pgfqpoint{5.546212in}{4.026000in}}%
\pgfpathcurveto{\pgfqpoint{5.546212in}{4.037050in}}{\pgfqpoint{5.541822in}{4.047649in}}{\pgfqpoint{5.534008in}{4.055463in}}%
\pgfpathcurveto{\pgfqpoint{5.526195in}{4.063276in}}{\pgfqpoint{5.515596in}{4.067667in}}{\pgfqpoint{5.504545in}{4.067667in}}%
\pgfpathcurveto{\pgfqpoint{5.493495in}{4.067667in}}{\pgfqpoint{5.482896in}{4.063276in}}{\pgfqpoint{5.475083in}{4.055463in}}%
\pgfpathcurveto{\pgfqpoint{5.467269in}{4.047649in}}{\pgfqpoint{5.462879in}{4.037050in}}{\pgfqpoint{5.462879in}{4.026000in}}%
\pgfpathcurveto{\pgfqpoint{5.462879in}{4.014950in}}{\pgfqpoint{5.467269in}{4.004351in}}{\pgfqpoint{5.475083in}{3.996537in}}%
\pgfpathcurveto{\pgfqpoint{5.482896in}{3.988724in}}{\pgfqpoint{5.493495in}{3.984333in}}{\pgfqpoint{5.504545in}{3.984333in}}%
\pgfpathclose%
\pgfusepath{stroke,fill}%
\end{pgfscope}%
\begin{pgfscope}%
\pgfpathrectangle{\pgfqpoint{0.800000in}{0.528000in}}{\pgfqpoint{4.960000in}{3.696000in}}%
\pgfusepath{clip}%
\pgfsetbuttcap%
\pgfsetroundjoin%
\definecolor{currentfill}{rgb}{0.000000,0.000000,0.000000}%
\pgfsetfillcolor{currentfill}%
\pgfsetlinewidth{1.003750pt}%
\definecolor{currentstroke}{rgb}{0.000000,0.000000,0.000000}%
\pgfsetstrokecolor{currentstroke}%
\pgfsetdash{}{0pt}%
\pgfpathmoveto{\pgfqpoint{5.504545in}{2.334266in}}%
\pgfpathcurveto{\pgfqpoint{5.515596in}{2.334266in}}{\pgfqpoint{5.526195in}{2.338657in}}{\pgfqpoint{5.534008in}{2.346470in}}%
\pgfpathcurveto{\pgfqpoint{5.541822in}{2.354284in}}{\pgfqpoint{5.546212in}{2.364883in}}{\pgfqpoint{5.546212in}{2.375933in}}%
\pgfpathcurveto{\pgfqpoint{5.546212in}{2.386983in}}{\pgfqpoint{5.541822in}{2.397582in}}{\pgfqpoint{5.534008in}{2.405396in}}%
\pgfpathcurveto{\pgfqpoint{5.526195in}{2.413209in}}{\pgfqpoint{5.515596in}{2.417600in}}{\pgfqpoint{5.504545in}{2.417600in}}%
\pgfpathcurveto{\pgfqpoint{5.493495in}{2.417600in}}{\pgfqpoint{5.482896in}{2.413209in}}{\pgfqpoint{5.475083in}{2.405396in}}%
\pgfpathcurveto{\pgfqpoint{5.467269in}{2.397582in}}{\pgfqpoint{5.462879in}{2.386983in}}{\pgfqpoint{5.462879in}{2.375933in}}%
\pgfpathcurveto{\pgfqpoint{5.462879in}{2.364883in}}{\pgfqpoint{5.467269in}{2.354284in}}{\pgfqpoint{5.475083in}{2.346470in}}%
\pgfpathcurveto{\pgfqpoint{5.482896in}{2.338657in}}{\pgfqpoint{5.493495in}{2.334266in}}{\pgfqpoint{5.504545in}{2.334266in}}%
\pgfpathclose%
\pgfusepath{stroke,fill}%
\end{pgfscope}%
\begin{pgfscope}%
\pgfpathrectangle{\pgfqpoint{0.800000in}{0.528000in}}{\pgfqpoint{4.960000in}{3.696000in}}%
\pgfusepath{clip}%
\pgfsetbuttcap%
\pgfsetroundjoin%
\definecolor{currentfill}{rgb}{0.000000,0.000000,0.000000}%
\pgfsetfillcolor{currentfill}%
\pgfsetlinewidth{1.003750pt}%
\definecolor{currentstroke}{rgb}{0.000000,0.000000,0.000000}%
\pgfsetstrokecolor{currentstroke}%
\pgfsetdash{}{0pt}%
\pgfpathmoveto{\pgfqpoint{5.504545in}{2.334266in}}%
\pgfpathcurveto{\pgfqpoint{5.515596in}{2.334266in}}{\pgfqpoint{5.526195in}{2.338657in}}{\pgfqpoint{5.534008in}{2.346470in}}%
\pgfpathcurveto{\pgfqpoint{5.541822in}{2.354284in}}{\pgfqpoint{5.546212in}{2.364883in}}{\pgfqpoint{5.546212in}{2.375933in}}%
\pgfpathcurveto{\pgfqpoint{5.546212in}{2.386983in}}{\pgfqpoint{5.541822in}{2.397582in}}{\pgfqpoint{5.534008in}{2.405396in}}%
\pgfpathcurveto{\pgfqpoint{5.526195in}{2.413209in}}{\pgfqpoint{5.515596in}{2.417600in}}{\pgfqpoint{5.504545in}{2.417600in}}%
\pgfpathcurveto{\pgfqpoint{5.493495in}{2.417600in}}{\pgfqpoint{5.482896in}{2.413209in}}{\pgfqpoint{5.475083in}{2.405396in}}%
\pgfpathcurveto{\pgfqpoint{5.467269in}{2.397582in}}{\pgfqpoint{5.462879in}{2.386983in}}{\pgfqpoint{5.462879in}{2.375933in}}%
\pgfpathcurveto{\pgfqpoint{5.462879in}{2.364883in}}{\pgfqpoint{5.467269in}{2.354284in}}{\pgfqpoint{5.475083in}{2.346470in}}%
\pgfpathcurveto{\pgfqpoint{5.482896in}{2.338657in}}{\pgfqpoint{5.493495in}{2.334266in}}{\pgfqpoint{5.504545in}{2.334266in}}%
\pgfpathclose%
\pgfusepath{stroke,fill}%
\end{pgfscope}%
\begin{pgfscope}%
\pgfsetbuttcap%
\pgfsetroundjoin%
\definecolor{currentfill}{rgb}{0.000000,0.000000,0.000000}%
\pgfsetfillcolor{currentfill}%
\pgfsetlinewidth{0.803000pt}%
\definecolor{currentstroke}{rgb}{0.000000,0.000000,0.000000}%
\pgfsetstrokecolor{currentstroke}%
\pgfsetdash{}{0pt}%
\pgfsys@defobject{currentmarker}{\pgfqpoint{0.000000in}{-0.048611in}}{\pgfqpoint{0.000000in}{0.000000in}}{%
\pgfpathmoveto{\pgfqpoint{0.000000in}{0.000000in}}%
\pgfpathlineto{\pgfqpoint{0.000000in}{-0.048611in}}%
\pgfusepath{stroke,fill}%
}%
\begin{pgfscope}%
\pgfsys@transformshift{1.025906in}{0.528000in}%
\pgfsys@useobject{currentmarker}{}%
\end{pgfscope}%
\end{pgfscope}%
\begin{pgfscope}%
\definecolor{textcolor}{rgb}{0.000000,0.000000,0.000000}%
\pgfsetstrokecolor{textcolor}%
\pgfsetfillcolor{textcolor}%
\pgftext[x=1.025906in,y=0.430778in,,top]{\color{textcolor}\sffamily\fontsize{10.000000}{12.000000}\selectfont 20}%
\end{pgfscope}%
\begin{pgfscope}%
\pgfsetbuttcap%
\pgfsetroundjoin%
\definecolor{currentfill}{rgb}{0.000000,0.000000,0.000000}%
\pgfsetfillcolor{currentfill}%
\pgfsetlinewidth{0.803000pt}%
\definecolor{currentstroke}{rgb}{0.000000,0.000000,0.000000}%
\pgfsetstrokecolor{currentstroke}%
\pgfsetdash{}{0pt}%
\pgfsys@defobject{currentmarker}{\pgfqpoint{0.000000in}{-0.048611in}}{\pgfqpoint{0.000000in}{0.000000in}}{%
\pgfpathmoveto{\pgfqpoint{0.000000in}{0.000000in}}%
\pgfpathlineto{\pgfqpoint{0.000000in}{-0.048611in}}%
\pgfusepath{stroke,fill}%
}%
\begin{pgfscope}%
\pgfsys@transformshift{2.518786in}{0.528000in}%
\pgfsys@useobject{currentmarker}{}%
\end{pgfscope}%
\end{pgfscope}%
\begin{pgfscope}%
\definecolor{textcolor}{rgb}{0.000000,0.000000,0.000000}%
\pgfsetstrokecolor{textcolor}%
\pgfsetfillcolor{textcolor}%
\pgftext[x=2.518786in,y=0.430778in,,top]{\color{textcolor}\sffamily\fontsize{10.000000}{12.000000}\selectfont 40}%
\end{pgfscope}%
\begin{pgfscope}%
\pgfsetbuttcap%
\pgfsetroundjoin%
\definecolor{currentfill}{rgb}{0.000000,0.000000,0.000000}%
\pgfsetfillcolor{currentfill}%
\pgfsetlinewidth{0.803000pt}%
\definecolor{currentstroke}{rgb}{0.000000,0.000000,0.000000}%
\pgfsetstrokecolor{currentstroke}%
\pgfsetdash{}{0pt}%
\pgfsys@defobject{currentmarker}{\pgfqpoint{0.000000in}{-0.048611in}}{\pgfqpoint{0.000000in}{0.000000in}}{%
\pgfpathmoveto{\pgfqpoint{0.000000in}{0.000000in}}%
\pgfpathlineto{\pgfqpoint{0.000000in}{-0.048611in}}%
\pgfusepath{stroke,fill}%
}%
\begin{pgfscope}%
\pgfsys@transformshift{4.011666in}{0.528000in}%
\pgfsys@useobject{currentmarker}{}%
\end{pgfscope}%
\end{pgfscope}%
\begin{pgfscope}%
\definecolor{textcolor}{rgb}{0.000000,0.000000,0.000000}%
\pgfsetstrokecolor{textcolor}%
\pgfsetfillcolor{textcolor}%
\pgftext[x=4.011666in,y=0.430778in,,top]{\color{textcolor}\sffamily\fontsize{10.000000}{12.000000}\selectfont 60}%
\end{pgfscope}%
\begin{pgfscope}%
\pgfsetbuttcap%
\pgfsetroundjoin%
\definecolor{currentfill}{rgb}{0.000000,0.000000,0.000000}%
\pgfsetfillcolor{currentfill}%
\pgfsetlinewidth{0.803000pt}%
\definecolor{currentstroke}{rgb}{0.000000,0.000000,0.000000}%
\pgfsetstrokecolor{currentstroke}%
\pgfsetdash{}{0pt}%
\pgfsys@defobject{currentmarker}{\pgfqpoint{0.000000in}{-0.048611in}}{\pgfqpoint{0.000000in}{0.000000in}}{%
\pgfpathmoveto{\pgfqpoint{0.000000in}{0.000000in}}%
\pgfpathlineto{\pgfqpoint{0.000000in}{-0.048611in}}%
\pgfusepath{stroke,fill}%
}%
\begin{pgfscope}%
\pgfsys@transformshift{5.504545in}{0.528000in}%
\pgfsys@useobject{currentmarker}{}%
\end{pgfscope}%
\end{pgfscope}%
\begin{pgfscope}%
\definecolor{textcolor}{rgb}{0.000000,0.000000,0.000000}%
\pgfsetstrokecolor{textcolor}%
\pgfsetfillcolor{textcolor}%
\pgftext[x=5.504545in,y=0.430778in,,top]{\color{textcolor}\sffamily\fontsize{10.000000}{12.000000}\selectfont 80}%
\end{pgfscope}%
\begin{pgfscope}%
\definecolor{textcolor}{rgb}{0.000000,0.000000,0.000000}%
\pgfsetstrokecolor{textcolor}%
\pgfsetfillcolor{textcolor}%
\pgftext[x=3.280000in,y=0.240809in,,top]{\color{textcolor}\sffamily\fontsize{10.000000}{12.000000}\selectfont \(\displaystyle k\)}%
\end{pgfscope}%
\begin{pgfscope}%
\pgfsetbuttcap%
\pgfsetroundjoin%
\definecolor{currentfill}{rgb}{0.000000,0.000000,0.000000}%
\pgfsetfillcolor{currentfill}%
\pgfsetlinewidth{0.803000pt}%
\definecolor{currentstroke}{rgb}{0.000000,0.000000,0.000000}%
\pgfsetstrokecolor{currentstroke}%
\pgfsetdash{}{0pt}%
\pgfsys@defobject{currentmarker}{\pgfqpoint{-0.048611in}{0.000000in}}{\pgfqpoint{0.000000in}{0.000000in}}{%
\pgfpathmoveto{\pgfqpoint{0.000000in}{0.000000in}}%
\pgfpathlineto{\pgfqpoint{-0.048611in}{0.000000in}}%
\pgfusepath{stroke,fill}%
}%
\begin{pgfscope}%
\pgfsys@transformshift{0.800000in}{0.725866in}%
\pgfsys@useobject{currentmarker}{}%
\end{pgfscope}%
\end{pgfscope}%
\begin{pgfscope}%
\definecolor{textcolor}{rgb}{0.000000,0.000000,0.000000}%
\pgfsetstrokecolor{textcolor}%
\pgfsetfillcolor{textcolor}%
\pgftext[x=0.614413in,y=0.673105in,left,base]{\color{textcolor}\sffamily\fontsize{10.000000}{12.000000}\selectfont 8}%
\end{pgfscope}%
\begin{pgfscope}%
\pgfsetbuttcap%
\pgfsetroundjoin%
\definecolor{currentfill}{rgb}{0.000000,0.000000,0.000000}%
\pgfsetfillcolor{currentfill}%
\pgfsetlinewidth{0.803000pt}%
\definecolor{currentstroke}{rgb}{0.000000,0.000000,0.000000}%
\pgfsetstrokecolor{currentstroke}%
\pgfsetdash{}{0pt}%
\pgfsys@defobject{currentmarker}{\pgfqpoint{-0.048611in}{0.000000in}}{\pgfqpoint{0.000000in}{0.000000in}}{%
\pgfpathmoveto{\pgfqpoint{0.000000in}{0.000000in}}%
\pgfpathlineto{\pgfqpoint{-0.048611in}{0.000000in}}%
\pgfusepath{stroke,fill}%
}%
\begin{pgfscope}%
\pgfsys@transformshift{0.800000in}{2.375933in}%
\pgfsys@useobject{currentmarker}{}%
\end{pgfscope}%
\end{pgfscope}%
\begin{pgfscope}%
\definecolor{textcolor}{rgb}{0.000000,0.000000,0.000000}%
\pgfsetstrokecolor{textcolor}%
\pgfsetfillcolor{textcolor}%
\pgftext[x=0.614413in,y=2.323172in,left,base]{\color{textcolor}\sffamily\fontsize{10.000000}{12.000000}\selectfont 9}%
\end{pgfscope}%
\begin{pgfscope}%
\pgfsetbuttcap%
\pgfsetroundjoin%
\definecolor{currentfill}{rgb}{0.000000,0.000000,0.000000}%
\pgfsetfillcolor{currentfill}%
\pgfsetlinewidth{0.803000pt}%
\definecolor{currentstroke}{rgb}{0.000000,0.000000,0.000000}%
\pgfsetstrokecolor{currentstroke}%
\pgfsetdash{}{0pt}%
\pgfsys@defobject{currentmarker}{\pgfqpoint{-0.048611in}{0.000000in}}{\pgfqpoint{0.000000in}{0.000000in}}{%
\pgfpathmoveto{\pgfqpoint{0.000000in}{0.000000in}}%
\pgfpathlineto{\pgfqpoint{-0.048611in}{0.000000in}}%
\pgfusepath{stroke,fill}%
}%
\begin{pgfscope}%
\pgfsys@transformshift{0.800000in}{4.026000in}%
\pgfsys@useobject{currentmarker}{}%
\end{pgfscope}%
\end{pgfscope}%
\begin{pgfscope}%
\definecolor{textcolor}{rgb}{0.000000,0.000000,0.000000}%
\pgfsetstrokecolor{textcolor}%
\pgfsetfillcolor{textcolor}%
\pgftext[x=0.526047in,y=3.973238in,left,base]{\color{textcolor}\sffamily\fontsize{10.000000}{12.000000}\selectfont 10}%
\end{pgfscope}%
\begin{pgfscope}%
\definecolor{textcolor}{rgb}{0.000000,0.000000,0.000000}%
\pgfsetstrokecolor{textcolor}%
\pgfsetfillcolor{textcolor}%
\pgftext[x=0.470492in,y=2.376000in,,bottom,rotate=90.000000]{\color{textcolor}\sffamily\fontsize{10.000000}{12.000000}\selectfont Number of GMRES Iterations}%
\end{pgfscope}%
\begin{pgfscope}%
\pgfsetrectcap%
\pgfsetmiterjoin%
\pgfsetlinewidth{0.803000pt}%
\definecolor{currentstroke}{rgb}{0.000000,0.000000,0.000000}%
\pgfsetstrokecolor{currentstroke}%
\pgfsetdash{}{0pt}%
\pgfpathmoveto{\pgfqpoint{0.800000in}{0.528000in}}%
\pgfpathlineto{\pgfqpoint{0.800000in}{4.224000in}}%
\pgfusepath{stroke}%
\end{pgfscope}%
\begin{pgfscope}%
\pgfsetrectcap%
\pgfsetmiterjoin%
\pgfsetlinewidth{0.803000pt}%
\definecolor{currentstroke}{rgb}{0.000000,0.000000,0.000000}%
\pgfsetstrokecolor{currentstroke}%
\pgfsetdash{}{0pt}%
\pgfpathmoveto{\pgfqpoint{5.760000in}{0.528000in}}%
\pgfpathlineto{\pgfqpoint{5.760000in}{4.224000in}}%
\pgfusepath{stroke}%
\end{pgfscope}%
\begin{pgfscope}%
\pgfsetrectcap%
\pgfsetmiterjoin%
\pgfsetlinewidth{0.803000pt}%
\definecolor{currentstroke}{rgb}{0.000000,0.000000,0.000000}%
\pgfsetstrokecolor{currentstroke}%
\pgfsetdash{}{0pt}%
\pgfpathmoveto{\pgfqpoint{0.800000in}{0.528000in}}%
\pgfpathlineto{\pgfqpoint{5.760000in}{0.528000in}}%
\pgfusepath{stroke}%
\end{pgfscope}%
\begin{pgfscope}%
\pgfsetrectcap%
\pgfsetmiterjoin%
\pgfsetlinewidth{0.803000pt}%
\definecolor{currentstroke}{rgb}{0.000000,0.000000,0.000000}%
\pgfsetstrokecolor{currentstroke}%
\pgfsetdash{}{0pt}%
\pgfpathmoveto{\pgfqpoint{0.800000in}{4.224000in}}%
\pgfpathlineto{\pgfqpoint{5.760000in}{4.224000in}}%
\pgfusepath{stroke}%
\end{pgfscope}%
\end{pgfpicture}%
\makeatother%
\endgroup%

  \caption{GMRES iteration counts for $\alpha = 0.5/k^{1/2}$}\label{fig:linfinityn1}
\end{subfigure}

    \begin{subfigure}{\textwidth}
      \centering
%% Creator: Matplotlib, PGF backend
%%
%% To include the figure in your LaTeX document, write
%%   \input{<filename>.pgf}
%%
%% Make sure the required packages are loaded in your preamble
%%   \usepackage{pgf}
%%
%% Figures using additional raster images can only be included by \input if
%% they are in the same directory as the main LaTeX file. For loading figures
%% from other directories you can use the `import` package
%%   \usepackage{import}
%% and then include the figures with
%%   \import{<path to file>}{<filename>.pgf}
%%
%% Matplotlib used the following preamble
%%   \usepackage{fontspec}
%%   \setmainfont{DejaVuSerif.ttf}[Path=/home/owen/progs/firedrake-complex/firedrake/lib/python3.5/site-packages/matplotlib/mpl-data/fonts/ttf/]
%%   \setsansfont{DejaVuSans.ttf}[Path=/home/owen/progs/firedrake-complex/firedrake/lib/python3.5/site-packages/matplotlib/mpl-data/fonts/ttf/]
%%   \setmonofont{DejaVuSansMono.ttf}[Path=/home/owen/progs/firedrake-complex/firedrake/lib/python3.5/site-packages/matplotlib/mpl-data/fonts/ttf/]
%%
\begingroup%
\makeatletter%
\begin{pgfpicture}%
\pgfpathrectangle{\pgfpointorigin}{\pgfqpoint{6.400000in}{4.800000in}}%
\pgfusepath{use as bounding box, clip}%
\begin{pgfscope}%
\pgfsetbuttcap%
\pgfsetmiterjoin%
\definecolor{currentfill}{rgb}{1.000000,1.000000,1.000000}%
\pgfsetfillcolor{currentfill}%
\pgfsetlinewidth{0.000000pt}%
\definecolor{currentstroke}{rgb}{1.000000,1.000000,1.000000}%
\pgfsetstrokecolor{currentstroke}%
\pgfsetdash{}{0pt}%
\pgfpathmoveto{\pgfqpoint{0.000000in}{0.000000in}}%
\pgfpathlineto{\pgfqpoint{6.400000in}{0.000000in}}%
\pgfpathlineto{\pgfqpoint{6.400000in}{4.800000in}}%
\pgfpathlineto{\pgfqpoint{0.000000in}{4.800000in}}%
\pgfpathclose%
\pgfusepath{fill}%
\end{pgfscope}%
\begin{pgfscope}%
\pgfsetbuttcap%
\pgfsetmiterjoin%
\definecolor{currentfill}{rgb}{1.000000,1.000000,1.000000}%
\pgfsetfillcolor{currentfill}%
\pgfsetlinewidth{0.000000pt}%
\definecolor{currentstroke}{rgb}{0.000000,0.000000,0.000000}%
\pgfsetstrokecolor{currentstroke}%
\pgfsetstrokeopacity{0.000000}%
\pgfsetdash{}{0pt}%
\pgfpathmoveto{\pgfqpoint{0.800000in}{0.528000in}}%
\pgfpathlineto{\pgfqpoint{5.760000in}{0.528000in}}%
\pgfpathlineto{\pgfqpoint{5.760000in}{4.224000in}}%
\pgfpathlineto{\pgfqpoint{0.800000in}{4.224000in}}%
\pgfpathclose%
\pgfusepath{fill}%
\end{pgfscope}%
\begin{pgfscope}%
\pgfpathrectangle{\pgfqpoint{0.800000in}{0.528000in}}{\pgfqpoint{4.960000in}{3.696000in}}%
\pgfusepath{clip}%
\pgfsetbuttcap%
\pgfsetroundjoin%
\definecolor{currentfill}{rgb}{0.000000,0.000000,0.000000}%
\pgfsetfillcolor{currentfill}%
\pgfsetlinewidth{1.003750pt}%
\definecolor{currentstroke}{rgb}{0.000000,0.000000,0.000000}%
\pgfsetstrokecolor{currentstroke}%
\pgfsetdash{}{0pt}%
\pgfpathmoveto{\pgfqpoint{1.025906in}{3.984333in}}%
\pgfpathcurveto{\pgfqpoint{1.036956in}{3.984333in}}{\pgfqpoint{1.047555in}{3.988724in}}{\pgfqpoint{1.055369in}{3.996537in}}%
\pgfpathcurveto{\pgfqpoint{1.063182in}{4.004351in}}{\pgfqpoint{1.067573in}{4.014950in}}{\pgfqpoint{1.067573in}{4.026000in}}%
\pgfpathcurveto{\pgfqpoint{1.067573in}{4.037050in}}{\pgfqpoint{1.063182in}{4.047649in}}{\pgfqpoint{1.055369in}{4.055463in}}%
\pgfpathcurveto{\pgfqpoint{1.047555in}{4.063276in}}{\pgfqpoint{1.036956in}{4.067667in}}{\pgfqpoint{1.025906in}{4.067667in}}%
\pgfpathcurveto{\pgfqpoint{1.014856in}{4.067667in}}{\pgfqpoint{1.004257in}{4.063276in}}{\pgfqpoint{0.996443in}{4.055463in}}%
\pgfpathcurveto{\pgfqpoint{0.988630in}{4.047649in}}{\pgfqpoint{0.984239in}{4.037050in}}{\pgfqpoint{0.984239in}{4.026000in}}%
\pgfpathcurveto{\pgfqpoint{0.984239in}{4.014950in}}{\pgfqpoint{0.988630in}{4.004351in}}{\pgfqpoint{0.996443in}{3.996537in}}%
\pgfpathcurveto{\pgfqpoint{1.004257in}{3.988724in}}{\pgfqpoint{1.014856in}{3.984333in}}{\pgfqpoint{1.025906in}{3.984333in}}%
\pgfpathclose%
\pgfusepath{stroke,fill}%
\end{pgfscope}%
\begin{pgfscope}%
\pgfpathrectangle{\pgfqpoint{0.800000in}{0.528000in}}{\pgfqpoint{4.960000in}{3.696000in}}%
\pgfusepath{clip}%
\pgfsetbuttcap%
\pgfsetroundjoin%
\definecolor{currentfill}{rgb}{0.000000,0.000000,0.000000}%
\pgfsetfillcolor{currentfill}%
\pgfsetlinewidth{1.003750pt}%
\definecolor{currentstroke}{rgb}{0.000000,0.000000,0.000000}%
\pgfsetstrokecolor{currentstroke}%
\pgfsetdash{}{0pt}%
\pgfpathmoveto{\pgfqpoint{1.025906in}{3.984333in}}%
\pgfpathcurveto{\pgfqpoint{1.036956in}{3.984333in}}{\pgfqpoint{1.047555in}{3.988724in}}{\pgfqpoint{1.055369in}{3.996537in}}%
\pgfpathcurveto{\pgfqpoint{1.063182in}{4.004351in}}{\pgfqpoint{1.067573in}{4.014950in}}{\pgfqpoint{1.067573in}{4.026000in}}%
\pgfpathcurveto{\pgfqpoint{1.067573in}{4.037050in}}{\pgfqpoint{1.063182in}{4.047649in}}{\pgfqpoint{1.055369in}{4.055463in}}%
\pgfpathcurveto{\pgfqpoint{1.047555in}{4.063276in}}{\pgfqpoint{1.036956in}{4.067667in}}{\pgfqpoint{1.025906in}{4.067667in}}%
\pgfpathcurveto{\pgfqpoint{1.014856in}{4.067667in}}{\pgfqpoint{1.004257in}{4.063276in}}{\pgfqpoint{0.996443in}{4.055463in}}%
\pgfpathcurveto{\pgfqpoint{0.988630in}{4.047649in}}{\pgfqpoint{0.984239in}{4.037050in}}{\pgfqpoint{0.984239in}{4.026000in}}%
\pgfpathcurveto{\pgfqpoint{0.984239in}{4.014950in}}{\pgfqpoint{0.988630in}{4.004351in}}{\pgfqpoint{0.996443in}{3.996537in}}%
\pgfpathcurveto{\pgfqpoint{1.004257in}{3.988724in}}{\pgfqpoint{1.014856in}{3.984333in}}{\pgfqpoint{1.025906in}{3.984333in}}%
\pgfpathclose%
\pgfusepath{stroke,fill}%
\end{pgfscope}%
\begin{pgfscope}%
\pgfpathrectangle{\pgfqpoint{0.800000in}{0.528000in}}{\pgfqpoint{4.960000in}{3.696000in}}%
\pgfusepath{clip}%
\pgfsetbuttcap%
\pgfsetroundjoin%
\definecolor{currentfill}{rgb}{0.000000,0.000000,0.000000}%
\pgfsetfillcolor{currentfill}%
\pgfsetlinewidth{1.003750pt}%
\definecolor{currentstroke}{rgb}{0.000000,0.000000,0.000000}%
\pgfsetstrokecolor{currentstroke}%
\pgfsetdash{}{0pt}%
\pgfpathmoveto{\pgfqpoint{1.025906in}{3.984333in}}%
\pgfpathcurveto{\pgfqpoint{1.036956in}{3.984333in}}{\pgfqpoint{1.047555in}{3.988724in}}{\pgfqpoint{1.055369in}{3.996537in}}%
\pgfpathcurveto{\pgfqpoint{1.063182in}{4.004351in}}{\pgfqpoint{1.067573in}{4.014950in}}{\pgfqpoint{1.067573in}{4.026000in}}%
\pgfpathcurveto{\pgfqpoint{1.067573in}{4.037050in}}{\pgfqpoint{1.063182in}{4.047649in}}{\pgfqpoint{1.055369in}{4.055463in}}%
\pgfpathcurveto{\pgfqpoint{1.047555in}{4.063276in}}{\pgfqpoint{1.036956in}{4.067667in}}{\pgfqpoint{1.025906in}{4.067667in}}%
\pgfpathcurveto{\pgfqpoint{1.014856in}{4.067667in}}{\pgfqpoint{1.004257in}{4.063276in}}{\pgfqpoint{0.996443in}{4.055463in}}%
\pgfpathcurveto{\pgfqpoint{0.988630in}{4.047649in}}{\pgfqpoint{0.984239in}{4.037050in}}{\pgfqpoint{0.984239in}{4.026000in}}%
\pgfpathcurveto{\pgfqpoint{0.984239in}{4.014950in}}{\pgfqpoint{0.988630in}{4.004351in}}{\pgfqpoint{0.996443in}{3.996537in}}%
\pgfpathcurveto{\pgfqpoint{1.004257in}{3.988724in}}{\pgfqpoint{1.014856in}{3.984333in}}{\pgfqpoint{1.025906in}{3.984333in}}%
\pgfpathclose%
\pgfusepath{stroke,fill}%
\end{pgfscope}%
\begin{pgfscope}%
\pgfpathrectangle{\pgfqpoint{0.800000in}{0.528000in}}{\pgfqpoint{4.960000in}{3.696000in}}%
\pgfusepath{clip}%
\pgfsetbuttcap%
\pgfsetroundjoin%
\definecolor{currentfill}{rgb}{0.000000,0.000000,0.000000}%
\pgfsetfillcolor{currentfill}%
\pgfsetlinewidth{1.003750pt}%
\definecolor{currentstroke}{rgb}{0.000000,0.000000,0.000000}%
\pgfsetstrokecolor{currentstroke}%
\pgfsetdash{}{0pt}%
\pgfpathmoveto{\pgfqpoint{1.025906in}{3.984333in}}%
\pgfpathcurveto{\pgfqpoint{1.036956in}{3.984333in}}{\pgfqpoint{1.047555in}{3.988724in}}{\pgfqpoint{1.055369in}{3.996537in}}%
\pgfpathcurveto{\pgfqpoint{1.063182in}{4.004351in}}{\pgfqpoint{1.067573in}{4.014950in}}{\pgfqpoint{1.067573in}{4.026000in}}%
\pgfpathcurveto{\pgfqpoint{1.067573in}{4.037050in}}{\pgfqpoint{1.063182in}{4.047649in}}{\pgfqpoint{1.055369in}{4.055463in}}%
\pgfpathcurveto{\pgfqpoint{1.047555in}{4.063276in}}{\pgfqpoint{1.036956in}{4.067667in}}{\pgfqpoint{1.025906in}{4.067667in}}%
\pgfpathcurveto{\pgfqpoint{1.014856in}{4.067667in}}{\pgfqpoint{1.004257in}{4.063276in}}{\pgfqpoint{0.996443in}{4.055463in}}%
\pgfpathcurveto{\pgfqpoint{0.988630in}{4.047649in}}{\pgfqpoint{0.984239in}{4.037050in}}{\pgfqpoint{0.984239in}{4.026000in}}%
\pgfpathcurveto{\pgfqpoint{0.984239in}{4.014950in}}{\pgfqpoint{0.988630in}{4.004351in}}{\pgfqpoint{0.996443in}{3.996537in}}%
\pgfpathcurveto{\pgfqpoint{1.004257in}{3.988724in}}{\pgfqpoint{1.014856in}{3.984333in}}{\pgfqpoint{1.025906in}{3.984333in}}%
\pgfpathclose%
\pgfusepath{stroke,fill}%
\end{pgfscope}%
\begin{pgfscope}%
\pgfpathrectangle{\pgfqpoint{0.800000in}{0.528000in}}{\pgfqpoint{4.960000in}{3.696000in}}%
\pgfusepath{clip}%
\pgfsetbuttcap%
\pgfsetroundjoin%
\definecolor{currentfill}{rgb}{0.000000,0.000000,0.000000}%
\pgfsetfillcolor{currentfill}%
\pgfsetlinewidth{1.003750pt}%
\definecolor{currentstroke}{rgb}{0.000000,0.000000,0.000000}%
\pgfsetstrokecolor{currentstroke}%
\pgfsetdash{}{0pt}%
\pgfpathmoveto{\pgfqpoint{1.025906in}{3.984333in}}%
\pgfpathcurveto{\pgfqpoint{1.036956in}{3.984333in}}{\pgfqpoint{1.047555in}{3.988724in}}{\pgfqpoint{1.055369in}{3.996537in}}%
\pgfpathcurveto{\pgfqpoint{1.063182in}{4.004351in}}{\pgfqpoint{1.067573in}{4.014950in}}{\pgfqpoint{1.067573in}{4.026000in}}%
\pgfpathcurveto{\pgfqpoint{1.067573in}{4.037050in}}{\pgfqpoint{1.063182in}{4.047649in}}{\pgfqpoint{1.055369in}{4.055463in}}%
\pgfpathcurveto{\pgfqpoint{1.047555in}{4.063276in}}{\pgfqpoint{1.036956in}{4.067667in}}{\pgfqpoint{1.025906in}{4.067667in}}%
\pgfpathcurveto{\pgfqpoint{1.014856in}{4.067667in}}{\pgfqpoint{1.004257in}{4.063276in}}{\pgfqpoint{0.996443in}{4.055463in}}%
\pgfpathcurveto{\pgfqpoint{0.988630in}{4.047649in}}{\pgfqpoint{0.984239in}{4.037050in}}{\pgfqpoint{0.984239in}{4.026000in}}%
\pgfpathcurveto{\pgfqpoint{0.984239in}{4.014950in}}{\pgfqpoint{0.988630in}{4.004351in}}{\pgfqpoint{0.996443in}{3.996537in}}%
\pgfpathcurveto{\pgfqpoint{1.004257in}{3.988724in}}{\pgfqpoint{1.014856in}{3.984333in}}{\pgfqpoint{1.025906in}{3.984333in}}%
\pgfpathclose%
\pgfusepath{stroke,fill}%
\end{pgfscope}%
\begin{pgfscope}%
\pgfpathrectangle{\pgfqpoint{0.800000in}{0.528000in}}{\pgfqpoint{4.960000in}{3.696000in}}%
\pgfusepath{clip}%
\pgfsetbuttcap%
\pgfsetroundjoin%
\definecolor{currentfill}{rgb}{0.000000,0.000000,0.000000}%
\pgfsetfillcolor{currentfill}%
\pgfsetlinewidth{1.003750pt}%
\definecolor{currentstroke}{rgb}{0.000000,0.000000,0.000000}%
\pgfsetstrokecolor{currentstroke}%
\pgfsetdash{}{0pt}%
\pgfpathmoveto{\pgfqpoint{1.025906in}{3.984333in}}%
\pgfpathcurveto{\pgfqpoint{1.036956in}{3.984333in}}{\pgfqpoint{1.047555in}{3.988724in}}{\pgfqpoint{1.055369in}{3.996537in}}%
\pgfpathcurveto{\pgfqpoint{1.063182in}{4.004351in}}{\pgfqpoint{1.067573in}{4.014950in}}{\pgfqpoint{1.067573in}{4.026000in}}%
\pgfpathcurveto{\pgfqpoint{1.067573in}{4.037050in}}{\pgfqpoint{1.063182in}{4.047649in}}{\pgfqpoint{1.055369in}{4.055463in}}%
\pgfpathcurveto{\pgfqpoint{1.047555in}{4.063276in}}{\pgfqpoint{1.036956in}{4.067667in}}{\pgfqpoint{1.025906in}{4.067667in}}%
\pgfpathcurveto{\pgfqpoint{1.014856in}{4.067667in}}{\pgfqpoint{1.004257in}{4.063276in}}{\pgfqpoint{0.996443in}{4.055463in}}%
\pgfpathcurveto{\pgfqpoint{0.988630in}{4.047649in}}{\pgfqpoint{0.984239in}{4.037050in}}{\pgfqpoint{0.984239in}{4.026000in}}%
\pgfpathcurveto{\pgfqpoint{0.984239in}{4.014950in}}{\pgfqpoint{0.988630in}{4.004351in}}{\pgfqpoint{0.996443in}{3.996537in}}%
\pgfpathcurveto{\pgfqpoint{1.004257in}{3.988724in}}{\pgfqpoint{1.014856in}{3.984333in}}{\pgfqpoint{1.025906in}{3.984333in}}%
\pgfpathclose%
\pgfusepath{stroke,fill}%
\end{pgfscope}%
\begin{pgfscope}%
\pgfpathrectangle{\pgfqpoint{0.800000in}{0.528000in}}{\pgfqpoint{4.960000in}{3.696000in}}%
\pgfusepath{clip}%
\pgfsetbuttcap%
\pgfsetroundjoin%
\definecolor{currentfill}{rgb}{0.000000,0.000000,0.000000}%
\pgfsetfillcolor{currentfill}%
\pgfsetlinewidth{1.003750pt}%
\definecolor{currentstroke}{rgb}{0.000000,0.000000,0.000000}%
\pgfsetstrokecolor{currentstroke}%
\pgfsetdash{}{0pt}%
\pgfpathmoveto{\pgfqpoint{1.025906in}{3.984333in}}%
\pgfpathcurveto{\pgfqpoint{1.036956in}{3.984333in}}{\pgfqpoint{1.047555in}{3.988724in}}{\pgfqpoint{1.055369in}{3.996537in}}%
\pgfpathcurveto{\pgfqpoint{1.063182in}{4.004351in}}{\pgfqpoint{1.067573in}{4.014950in}}{\pgfqpoint{1.067573in}{4.026000in}}%
\pgfpathcurveto{\pgfqpoint{1.067573in}{4.037050in}}{\pgfqpoint{1.063182in}{4.047649in}}{\pgfqpoint{1.055369in}{4.055463in}}%
\pgfpathcurveto{\pgfqpoint{1.047555in}{4.063276in}}{\pgfqpoint{1.036956in}{4.067667in}}{\pgfqpoint{1.025906in}{4.067667in}}%
\pgfpathcurveto{\pgfqpoint{1.014856in}{4.067667in}}{\pgfqpoint{1.004257in}{4.063276in}}{\pgfqpoint{0.996443in}{4.055463in}}%
\pgfpathcurveto{\pgfqpoint{0.988630in}{4.047649in}}{\pgfqpoint{0.984239in}{4.037050in}}{\pgfqpoint{0.984239in}{4.026000in}}%
\pgfpathcurveto{\pgfqpoint{0.984239in}{4.014950in}}{\pgfqpoint{0.988630in}{4.004351in}}{\pgfqpoint{0.996443in}{3.996537in}}%
\pgfpathcurveto{\pgfqpoint{1.004257in}{3.988724in}}{\pgfqpoint{1.014856in}{3.984333in}}{\pgfqpoint{1.025906in}{3.984333in}}%
\pgfpathclose%
\pgfusepath{stroke,fill}%
\end{pgfscope}%
\begin{pgfscope}%
\pgfpathrectangle{\pgfqpoint{0.800000in}{0.528000in}}{\pgfqpoint{4.960000in}{3.696000in}}%
\pgfusepath{clip}%
\pgfsetbuttcap%
\pgfsetroundjoin%
\definecolor{currentfill}{rgb}{0.000000,0.000000,0.000000}%
\pgfsetfillcolor{currentfill}%
\pgfsetlinewidth{1.003750pt}%
\definecolor{currentstroke}{rgb}{0.000000,0.000000,0.000000}%
\pgfsetstrokecolor{currentstroke}%
\pgfsetdash{}{0pt}%
\pgfpathmoveto{\pgfqpoint{1.025906in}{3.984333in}}%
\pgfpathcurveto{\pgfqpoint{1.036956in}{3.984333in}}{\pgfqpoint{1.047555in}{3.988724in}}{\pgfqpoint{1.055369in}{3.996537in}}%
\pgfpathcurveto{\pgfqpoint{1.063182in}{4.004351in}}{\pgfqpoint{1.067573in}{4.014950in}}{\pgfqpoint{1.067573in}{4.026000in}}%
\pgfpathcurveto{\pgfqpoint{1.067573in}{4.037050in}}{\pgfqpoint{1.063182in}{4.047649in}}{\pgfqpoint{1.055369in}{4.055463in}}%
\pgfpathcurveto{\pgfqpoint{1.047555in}{4.063276in}}{\pgfqpoint{1.036956in}{4.067667in}}{\pgfqpoint{1.025906in}{4.067667in}}%
\pgfpathcurveto{\pgfqpoint{1.014856in}{4.067667in}}{\pgfqpoint{1.004257in}{4.063276in}}{\pgfqpoint{0.996443in}{4.055463in}}%
\pgfpathcurveto{\pgfqpoint{0.988630in}{4.047649in}}{\pgfqpoint{0.984239in}{4.037050in}}{\pgfqpoint{0.984239in}{4.026000in}}%
\pgfpathcurveto{\pgfqpoint{0.984239in}{4.014950in}}{\pgfqpoint{0.988630in}{4.004351in}}{\pgfqpoint{0.996443in}{3.996537in}}%
\pgfpathcurveto{\pgfqpoint{1.004257in}{3.988724in}}{\pgfqpoint{1.014856in}{3.984333in}}{\pgfqpoint{1.025906in}{3.984333in}}%
\pgfpathclose%
\pgfusepath{stroke,fill}%
\end{pgfscope}%
\begin{pgfscope}%
\pgfpathrectangle{\pgfqpoint{0.800000in}{0.528000in}}{\pgfqpoint{4.960000in}{3.696000in}}%
\pgfusepath{clip}%
\pgfsetbuttcap%
\pgfsetroundjoin%
\definecolor{currentfill}{rgb}{0.000000,0.000000,0.000000}%
\pgfsetfillcolor{currentfill}%
\pgfsetlinewidth{1.003750pt}%
\definecolor{currentstroke}{rgb}{0.000000,0.000000,0.000000}%
\pgfsetstrokecolor{currentstroke}%
\pgfsetdash{}{0pt}%
\pgfpathmoveto{\pgfqpoint{1.025906in}{3.984333in}}%
\pgfpathcurveto{\pgfqpoint{1.036956in}{3.984333in}}{\pgfqpoint{1.047555in}{3.988724in}}{\pgfqpoint{1.055369in}{3.996537in}}%
\pgfpathcurveto{\pgfqpoint{1.063182in}{4.004351in}}{\pgfqpoint{1.067573in}{4.014950in}}{\pgfqpoint{1.067573in}{4.026000in}}%
\pgfpathcurveto{\pgfqpoint{1.067573in}{4.037050in}}{\pgfqpoint{1.063182in}{4.047649in}}{\pgfqpoint{1.055369in}{4.055463in}}%
\pgfpathcurveto{\pgfqpoint{1.047555in}{4.063276in}}{\pgfqpoint{1.036956in}{4.067667in}}{\pgfqpoint{1.025906in}{4.067667in}}%
\pgfpathcurveto{\pgfqpoint{1.014856in}{4.067667in}}{\pgfqpoint{1.004257in}{4.063276in}}{\pgfqpoint{0.996443in}{4.055463in}}%
\pgfpathcurveto{\pgfqpoint{0.988630in}{4.047649in}}{\pgfqpoint{0.984239in}{4.037050in}}{\pgfqpoint{0.984239in}{4.026000in}}%
\pgfpathcurveto{\pgfqpoint{0.984239in}{4.014950in}}{\pgfqpoint{0.988630in}{4.004351in}}{\pgfqpoint{0.996443in}{3.996537in}}%
\pgfpathcurveto{\pgfqpoint{1.004257in}{3.988724in}}{\pgfqpoint{1.014856in}{3.984333in}}{\pgfqpoint{1.025906in}{3.984333in}}%
\pgfpathclose%
\pgfusepath{stroke,fill}%
\end{pgfscope}%
\begin{pgfscope}%
\pgfpathrectangle{\pgfqpoint{0.800000in}{0.528000in}}{\pgfqpoint{4.960000in}{3.696000in}}%
\pgfusepath{clip}%
\pgfsetbuttcap%
\pgfsetroundjoin%
\definecolor{currentfill}{rgb}{0.000000,0.000000,0.000000}%
\pgfsetfillcolor{currentfill}%
\pgfsetlinewidth{1.003750pt}%
\definecolor{currentstroke}{rgb}{0.000000,0.000000,0.000000}%
\pgfsetstrokecolor{currentstroke}%
\pgfsetdash{}{0pt}%
\pgfpathmoveto{\pgfqpoint{1.025906in}{3.984333in}}%
\pgfpathcurveto{\pgfqpoint{1.036956in}{3.984333in}}{\pgfqpoint{1.047555in}{3.988724in}}{\pgfqpoint{1.055369in}{3.996537in}}%
\pgfpathcurveto{\pgfqpoint{1.063182in}{4.004351in}}{\pgfqpoint{1.067573in}{4.014950in}}{\pgfqpoint{1.067573in}{4.026000in}}%
\pgfpathcurveto{\pgfqpoint{1.067573in}{4.037050in}}{\pgfqpoint{1.063182in}{4.047649in}}{\pgfqpoint{1.055369in}{4.055463in}}%
\pgfpathcurveto{\pgfqpoint{1.047555in}{4.063276in}}{\pgfqpoint{1.036956in}{4.067667in}}{\pgfqpoint{1.025906in}{4.067667in}}%
\pgfpathcurveto{\pgfqpoint{1.014856in}{4.067667in}}{\pgfqpoint{1.004257in}{4.063276in}}{\pgfqpoint{0.996443in}{4.055463in}}%
\pgfpathcurveto{\pgfqpoint{0.988630in}{4.047649in}}{\pgfqpoint{0.984239in}{4.037050in}}{\pgfqpoint{0.984239in}{4.026000in}}%
\pgfpathcurveto{\pgfqpoint{0.984239in}{4.014950in}}{\pgfqpoint{0.988630in}{4.004351in}}{\pgfqpoint{0.996443in}{3.996537in}}%
\pgfpathcurveto{\pgfqpoint{1.004257in}{3.988724in}}{\pgfqpoint{1.014856in}{3.984333in}}{\pgfqpoint{1.025906in}{3.984333in}}%
\pgfpathclose%
\pgfusepath{stroke,fill}%
\end{pgfscope}%
\begin{pgfscope}%
\pgfpathrectangle{\pgfqpoint{0.800000in}{0.528000in}}{\pgfqpoint{4.960000in}{3.696000in}}%
\pgfusepath{clip}%
\pgfsetbuttcap%
\pgfsetroundjoin%
\definecolor{currentfill}{rgb}{0.000000,0.000000,0.000000}%
\pgfsetfillcolor{currentfill}%
\pgfsetlinewidth{1.003750pt}%
\definecolor{currentstroke}{rgb}{0.000000,0.000000,0.000000}%
\pgfsetstrokecolor{currentstroke}%
\pgfsetdash{}{0pt}%
\pgfpathmoveto{\pgfqpoint{1.025906in}{3.984333in}}%
\pgfpathcurveto{\pgfqpoint{1.036956in}{3.984333in}}{\pgfqpoint{1.047555in}{3.988724in}}{\pgfqpoint{1.055369in}{3.996537in}}%
\pgfpathcurveto{\pgfqpoint{1.063182in}{4.004351in}}{\pgfqpoint{1.067573in}{4.014950in}}{\pgfqpoint{1.067573in}{4.026000in}}%
\pgfpathcurveto{\pgfqpoint{1.067573in}{4.037050in}}{\pgfqpoint{1.063182in}{4.047649in}}{\pgfqpoint{1.055369in}{4.055463in}}%
\pgfpathcurveto{\pgfqpoint{1.047555in}{4.063276in}}{\pgfqpoint{1.036956in}{4.067667in}}{\pgfqpoint{1.025906in}{4.067667in}}%
\pgfpathcurveto{\pgfqpoint{1.014856in}{4.067667in}}{\pgfqpoint{1.004257in}{4.063276in}}{\pgfqpoint{0.996443in}{4.055463in}}%
\pgfpathcurveto{\pgfqpoint{0.988630in}{4.047649in}}{\pgfqpoint{0.984239in}{4.037050in}}{\pgfqpoint{0.984239in}{4.026000in}}%
\pgfpathcurveto{\pgfqpoint{0.984239in}{4.014950in}}{\pgfqpoint{0.988630in}{4.004351in}}{\pgfqpoint{0.996443in}{3.996537in}}%
\pgfpathcurveto{\pgfqpoint{1.004257in}{3.988724in}}{\pgfqpoint{1.014856in}{3.984333in}}{\pgfqpoint{1.025906in}{3.984333in}}%
\pgfpathclose%
\pgfusepath{stroke,fill}%
\end{pgfscope}%
\begin{pgfscope}%
\pgfpathrectangle{\pgfqpoint{0.800000in}{0.528000in}}{\pgfqpoint{4.960000in}{3.696000in}}%
\pgfusepath{clip}%
\pgfsetbuttcap%
\pgfsetroundjoin%
\definecolor{currentfill}{rgb}{0.000000,0.000000,0.000000}%
\pgfsetfillcolor{currentfill}%
\pgfsetlinewidth{1.003750pt}%
\definecolor{currentstroke}{rgb}{0.000000,0.000000,0.000000}%
\pgfsetstrokecolor{currentstroke}%
\pgfsetdash{}{0pt}%
\pgfpathmoveto{\pgfqpoint{1.025906in}{3.984333in}}%
\pgfpathcurveto{\pgfqpoint{1.036956in}{3.984333in}}{\pgfqpoint{1.047555in}{3.988724in}}{\pgfqpoint{1.055369in}{3.996537in}}%
\pgfpathcurveto{\pgfqpoint{1.063182in}{4.004351in}}{\pgfqpoint{1.067573in}{4.014950in}}{\pgfqpoint{1.067573in}{4.026000in}}%
\pgfpathcurveto{\pgfqpoint{1.067573in}{4.037050in}}{\pgfqpoint{1.063182in}{4.047649in}}{\pgfqpoint{1.055369in}{4.055463in}}%
\pgfpathcurveto{\pgfqpoint{1.047555in}{4.063276in}}{\pgfqpoint{1.036956in}{4.067667in}}{\pgfqpoint{1.025906in}{4.067667in}}%
\pgfpathcurveto{\pgfqpoint{1.014856in}{4.067667in}}{\pgfqpoint{1.004257in}{4.063276in}}{\pgfqpoint{0.996443in}{4.055463in}}%
\pgfpathcurveto{\pgfqpoint{0.988630in}{4.047649in}}{\pgfqpoint{0.984239in}{4.037050in}}{\pgfqpoint{0.984239in}{4.026000in}}%
\pgfpathcurveto{\pgfqpoint{0.984239in}{4.014950in}}{\pgfqpoint{0.988630in}{4.004351in}}{\pgfqpoint{0.996443in}{3.996537in}}%
\pgfpathcurveto{\pgfqpoint{1.004257in}{3.988724in}}{\pgfqpoint{1.014856in}{3.984333in}}{\pgfqpoint{1.025906in}{3.984333in}}%
\pgfpathclose%
\pgfusepath{stroke,fill}%
\end{pgfscope}%
\begin{pgfscope}%
\pgfpathrectangle{\pgfqpoint{0.800000in}{0.528000in}}{\pgfqpoint{4.960000in}{3.696000in}}%
\pgfusepath{clip}%
\pgfsetbuttcap%
\pgfsetroundjoin%
\definecolor{currentfill}{rgb}{0.000000,0.000000,0.000000}%
\pgfsetfillcolor{currentfill}%
\pgfsetlinewidth{1.003750pt}%
\definecolor{currentstroke}{rgb}{0.000000,0.000000,0.000000}%
\pgfsetstrokecolor{currentstroke}%
\pgfsetdash{}{0pt}%
\pgfpathmoveto{\pgfqpoint{1.025906in}{3.984333in}}%
\pgfpathcurveto{\pgfqpoint{1.036956in}{3.984333in}}{\pgfqpoint{1.047555in}{3.988724in}}{\pgfqpoint{1.055369in}{3.996537in}}%
\pgfpathcurveto{\pgfqpoint{1.063182in}{4.004351in}}{\pgfqpoint{1.067573in}{4.014950in}}{\pgfqpoint{1.067573in}{4.026000in}}%
\pgfpathcurveto{\pgfqpoint{1.067573in}{4.037050in}}{\pgfqpoint{1.063182in}{4.047649in}}{\pgfqpoint{1.055369in}{4.055463in}}%
\pgfpathcurveto{\pgfqpoint{1.047555in}{4.063276in}}{\pgfqpoint{1.036956in}{4.067667in}}{\pgfqpoint{1.025906in}{4.067667in}}%
\pgfpathcurveto{\pgfqpoint{1.014856in}{4.067667in}}{\pgfqpoint{1.004257in}{4.063276in}}{\pgfqpoint{0.996443in}{4.055463in}}%
\pgfpathcurveto{\pgfqpoint{0.988630in}{4.047649in}}{\pgfqpoint{0.984239in}{4.037050in}}{\pgfqpoint{0.984239in}{4.026000in}}%
\pgfpathcurveto{\pgfqpoint{0.984239in}{4.014950in}}{\pgfqpoint{0.988630in}{4.004351in}}{\pgfqpoint{0.996443in}{3.996537in}}%
\pgfpathcurveto{\pgfqpoint{1.004257in}{3.988724in}}{\pgfqpoint{1.014856in}{3.984333in}}{\pgfqpoint{1.025906in}{3.984333in}}%
\pgfpathclose%
\pgfusepath{stroke,fill}%
\end{pgfscope}%
\begin{pgfscope}%
\pgfpathrectangle{\pgfqpoint{0.800000in}{0.528000in}}{\pgfqpoint{4.960000in}{3.696000in}}%
\pgfusepath{clip}%
\pgfsetbuttcap%
\pgfsetroundjoin%
\definecolor{currentfill}{rgb}{0.000000,0.000000,0.000000}%
\pgfsetfillcolor{currentfill}%
\pgfsetlinewidth{1.003750pt}%
\definecolor{currentstroke}{rgb}{0.000000,0.000000,0.000000}%
\pgfsetstrokecolor{currentstroke}%
\pgfsetdash{}{0pt}%
\pgfpathmoveto{\pgfqpoint{1.025906in}{3.984333in}}%
\pgfpathcurveto{\pgfqpoint{1.036956in}{3.984333in}}{\pgfqpoint{1.047555in}{3.988724in}}{\pgfqpoint{1.055369in}{3.996537in}}%
\pgfpathcurveto{\pgfqpoint{1.063182in}{4.004351in}}{\pgfqpoint{1.067573in}{4.014950in}}{\pgfqpoint{1.067573in}{4.026000in}}%
\pgfpathcurveto{\pgfqpoint{1.067573in}{4.037050in}}{\pgfqpoint{1.063182in}{4.047649in}}{\pgfqpoint{1.055369in}{4.055463in}}%
\pgfpathcurveto{\pgfqpoint{1.047555in}{4.063276in}}{\pgfqpoint{1.036956in}{4.067667in}}{\pgfqpoint{1.025906in}{4.067667in}}%
\pgfpathcurveto{\pgfqpoint{1.014856in}{4.067667in}}{\pgfqpoint{1.004257in}{4.063276in}}{\pgfqpoint{0.996443in}{4.055463in}}%
\pgfpathcurveto{\pgfqpoint{0.988630in}{4.047649in}}{\pgfqpoint{0.984239in}{4.037050in}}{\pgfqpoint{0.984239in}{4.026000in}}%
\pgfpathcurveto{\pgfqpoint{0.984239in}{4.014950in}}{\pgfqpoint{0.988630in}{4.004351in}}{\pgfqpoint{0.996443in}{3.996537in}}%
\pgfpathcurveto{\pgfqpoint{1.004257in}{3.988724in}}{\pgfqpoint{1.014856in}{3.984333in}}{\pgfqpoint{1.025906in}{3.984333in}}%
\pgfpathclose%
\pgfusepath{stroke,fill}%
\end{pgfscope}%
\begin{pgfscope}%
\pgfpathrectangle{\pgfqpoint{0.800000in}{0.528000in}}{\pgfqpoint{4.960000in}{3.696000in}}%
\pgfusepath{clip}%
\pgfsetbuttcap%
\pgfsetroundjoin%
\definecolor{currentfill}{rgb}{0.000000,0.000000,0.000000}%
\pgfsetfillcolor{currentfill}%
\pgfsetlinewidth{1.003750pt}%
\definecolor{currentstroke}{rgb}{0.000000,0.000000,0.000000}%
\pgfsetstrokecolor{currentstroke}%
\pgfsetdash{}{0pt}%
\pgfpathmoveto{\pgfqpoint{1.025906in}{3.984333in}}%
\pgfpathcurveto{\pgfqpoint{1.036956in}{3.984333in}}{\pgfqpoint{1.047555in}{3.988724in}}{\pgfqpoint{1.055369in}{3.996537in}}%
\pgfpathcurveto{\pgfqpoint{1.063182in}{4.004351in}}{\pgfqpoint{1.067573in}{4.014950in}}{\pgfqpoint{1.067573in}{4.026000in}}%
\pgfpathcurveto{\pgfqpoint{1.067573in}{4.037050in}}{\pgfqpoint{1.063182in}{4.047649in}}{\pgfqpoint{1.055369in}{4.055463in}}%
\pgfpathcurveto{\pgfqpoint{1.047555in}{4.063276in}}{\pgfqpoint{1.036956in}{4.067667in}}{\pgfqpoint{1.025906in}{4.067667in}}%
\pgfpathcurveto{\pgfqpoint{1.014856in}{4.067667in}}{\pgfqpoint{1.004257in}{4.063276in}}{\pgfqpoint{0.996443in}{4.055463in}}%
\pgfpathcurveto{\pgfqpoint{0.988630in}{4.047649in}}{\pgfqpoint{0.984239in}{4.037050in}}{\pgfqpoint{0.984239in}{4.026000in}}%
\pgfpathcurveto{\pgfqpoint{0.984239in}{4.014950in}}{\pgfqpoint{0.988630in}{4.004351in}}{\pgfqpoint{0.996443in}{3.996537in}}%
\pgfpathcurveto{\pgfqpoint{1.004257in}{3.988724in}}{\pgfqpoint{1.014856in}{3.984333in}}{\pgfqpoint{1.025906in}{3.984333in}}%
\pgfpathclose%
\pgfusepath{stroke,fill}%
\end{pgfscope}%
\begin{pgfscope}%
\pgfpathrectangle{\pgfqpoint{0.800000in}{0.528000in}}{\pgfqpoint{4.960000in}{3.696000in}}%
\pgfusepath{clip}%
\pgfsetbuttcap%
\pgfsetroundjoin%
\definecolor{currentfill}{rgb}{0.000000,0.000000,0.000000}%
\pgfsetfillcolor{currentfill}%
\pgfsetlinewidth{1.003750pt}%
\definecolor{currentstroke}{rgb}{0.000000,0.000000,0.000000}%
\pgfsetstrokecolor{currentstroke}%
\pgfsetdash{}{0pt}%
\pgfpathmoveto{\pgfqpoint{1.025906in}{3.984333in}}%
\pgfpathcurveto{\pgfqpoint{1.036956in}{3.984333in}}{\pgfqpoint{1.047555in}{3.988724in}}{\pgfqpoint{1.055369in}{3.996537in}}%
\pgfpathcurveto{\pgfqpoint{1.063182in}{4.004351in}}{\pgfqpoint{1.067573in}{4.014950in}}{\pgfqpoint{1.067573in}{4.026000in}}%
\pgfpathcurveto{\pgfqpoint{1.067573in}{4.037050in}}{\pgfqpoint{1.063182in}{4.047649in}}{\pgfqpoint{1.055369in}{4.055463in}}%
\pgfpathcurveto{\pgfqpoint{1.047555in}{4.063276in}}{\pgfqpoint{1.036956in}{4.067667in}}{\pgfqpoint{1.025906in}{4.067667in}}%
\pgfpathcurveto{\pgfqpoint{1.014856in}{4.067667in}}{\pgfqpoint{1.004257in}{4.063276in}}{\pgfqpoint{0.996443in}{4.055463in}}%
\pgfpathcurveto{\pgfqpoint{0.988630in}{4.047649in}}{\pgfqpoint{0.984239in}{4.037050in}}{\pgfqpoint{0.984239in}{4.026000in}}%
\pgfpathcurveto{\pgfqpoint{0.984239in}{4.014950in}}{\pgfqpoint{0.988630in}{4.004351in}}{\pgfqpoint{0.996443in}{3.996537in}}%
\pgfpathcurveto{\pgfqpoint{1.004257in}{3.988724in}}{\pgfqpoint{1.014856in}{3.984333in}}{\pgfqpoint{1.025906in}{3.984333in}}%
\pgfpathclose%
\pgfusepath{stroke,fill}%
\end{pgfscope}%
\begin{pgfscope}%
\pgfpathrectangle{\pgfqpoint{0.800000in}{0.528000in}}{\pgfqpoint{4.960000in}{3.696000in}}%
\pgfusepath{clip}%
\pgfsetbuttcap%
\pgfsetroundjoin%
\definecolor{currentfill}{rgb}{0.000000,0.000000,0.000000}%
\pgfsetfillcolor{currentfill}%
\pgfsetlinewidth{1.003750pt}%
\definecolor{currentstroke}{rgb}{0.000000,0.000000,0.000000}%
\pgfsetstrokecolor{currentstroke}%
\pgfsetdash{}{0pt}%
\pgfpathmoveto{\pgfqpoint{1.025906in}{3.984333in}}%
\pgfpathcurveto{\pgfqpoint{1.036956in}{3.984333in}}{\pgfqpoint{1.047555in}{3.988724in}}{\pgfqpoint{1.055369in}{3.996537in}}%
\pgfpathcurveto{\pgfqpoint{1.063182in}{4.004351in}}{\pgfqpoint{1.067573in}{4.014950in}}{\pgfqpoint{1.067573in}{4.026000in}}%
\pgfpathcurveto{\pgfqpoint{1.067573in}{4.037050in}}{\pgfqpoint{1.063182in}{4.047649in}}{\pgfqpoint{1.055369in}{4.055463in}}%
\pgfpathcurveto{\pgfqpoint{1.047555in}{4.063276in}}{\pgfqpoint{1.036956in}{4.067667in}}{\pgfqpoint{1.025906in}{4.067667in}}%
\pgfpathcurveto{\pgfqpoint{1.014856in}{4.067667in}}{\pgfqpoint{1.004257in}{4.063276in}}{\pgfqpoint{0.996443in}{4.055463in}}%
\pgfpathcurveto{\pgfqpoint{0.988630in}{4.047649in}}{\pgfqpoint{0.984239in}{4.037050in}}{\pgfqpoint{0.984239in}{4.026000in}}%
\pgfpathcurveto{\pgfqpoint{0.984239in}{4.014950in}}{\pgfqpoint{0.988630in}{4.004351in}}{\pgfqpoint{0.996443in}{3.996537in}}%
\pgfpathcurveto{\pgfqpoint{1.004257in}{3.988724in}}{\pgfqpoint{1.014856in}{3.984333in}}{\pgfqpoint{1.025906in}{3.984333in}}%
\pgfpathclose%
\pgfusepath{stroke,fill}%
\end{pgfscope}%
\begin{pgfscope}%
\pgfpathrectangle{\pgfqpoint{0.800000in}{0.528000in}}{\pgfqpoint{4.960000in}{3.696000in}}%
\pgfusepath{clip}%
\pgfsetbuttcap%
\pgfsetroundjoin%
\definecolor{currentfill}{rgb}{0.000000,0.000000,0.000000}%
\pgfsetfillcolor{currentfill}%
\pgfsetlinewidth{1.003750pt}%
\definecolor{currentstroke}{rgb}{0.000000,0.000000,0.000000}%
\pgfsetstrokecolor{currentstroke}%
\pgfsetdash{}{0pt}%
\pgfpathmoveto{\pgfqpoint{1.025906in}{3.984333in}}%
\pgfpathcurveto{\pgfqpoint{1.036956in}{3.984333in}}{\pgfqpoint{1.047555in}{3.988724in}}{\pgfqpoint{1.055369in}{3.996537in}}%
\pgfpathcurveto{\pgfqpoint{1.063182in}{4.004351in}}{\pgfqpoint{1.067573in}{4.014950in}}{\pgfqpoint{1.067573in}{4.026000in}}%
\pgfpathcurveto{\pgfqpoint{1.067573in}{4.037050in}}{\pgfqpoint{1.063182in}{4.047649in}}{\pgfqpoint{1.055369in}{4.055463in}}%
\pgfpathcurveto{\pgfqpoint{1.047555in}{4.063276in}}{\pgfqpoint{1.036956in}{4.067667in}}{\pgfqpoint{1.025906in}{4.067667in}}%
\pgfpathcurveto{\pgfqpoint{1.014856in}{4.067667in}}{\pgfqpoint{1.004257in}{4.063276in}}{\pgfqpoint{0.996443in}{4.055463in}}%
\pgfpathcurveto{\pgfqpoint{0.988630in}{4.047649in}}{\pgfqpoint{0.984239in}{4.037050in}}{\pgfqpoint{0.984239in}{4.026000in}}%
\pgfpathcurveto{\pgfqpoint{0.984239in}{4.014950in}}{\pgfqpoint{0.988630in}{4.004351in}}{\pgfqpoint{0.996443in}{3.996537in}}%
\pgfpathcurveto{\pgfqpoint{1.004257in}{3.988724in}}{\pgfqpoint{1.014856in}{3.984333in}}{\pgfqpoint{1.025906in}{3.984333in}}%
\pgfpathclose%
\pgfusepath{stroke,fill}%
\end{pgfscope}%
\begin{pgfscope}%
\pgfpathrectangle{\pgfqpoint{0.800000in}{0.528000in}}{\pgfqpoint{4.960000in}{3.696000in}}%
\pgfusepath{clip}%
\pgfsetbuttcap%
\pgfsetroundjoin%
\definecolor{currentfill}{rgb}{0.000000,0.000000,0.000000}%
\pgfsetfillcolor{currentfill}%
\pgfsetlinewidth{1.003750pt}%
\definecolor{currentstroke}{rgb}{0.000000,0.000000,0.000000}%
\pgfsetstrokecolor{currentstroke}%
\pgfsetdash{}{0pt}%
\pgfpathmoveto{\pgfqpoint{1.025906in}{3.984333in}}%
\pgfpathcurveto{\pgfqpoint{1.036956in}{3.984333in}}{\pgfqpoint{1.047555in}{3.988724in}}{\pgfqpoint{1.055369in}{3.996537in}}%
\pgfpathcurveto{\pgfqpoint{1.063182in}{4.004351in}}{\pgfqpoint{1.067573in}{4.014950in}}{\pgfqpoint{1.067573in}{4.026000in}}%
\pgfpathcurveto{\pgfqpoint{1.067573in}{4.037050in}}{\pgfqpoint{1.063182in}{4.047649in}}{\pgfqpoint{1.055369in}{4.055463in}}%
\pgfpathcurveto{\pgfqpoint{1.047555in}{4.063276in}}{\pgfqpoint{1.036956in}{4.067667in}}{\pgfqpoint{1.025906in}{4.067667in}}%
\pgfpathcurveto{\pgfqpoint{1.014856in}{4.067667in}}{\pgfqpoint{1.004257in}{4.063276in}}{\pgfqpoint{0.996443in}{4.055463in}}%
\pgfpathcurveto{\pgfqpoint{0.988630in}{4.047649in}}{\pgfqpoint{0.984239in}{4.037050in}}{\pgfqpoint{0.984239in}{4.026000in}}%
\pgfpathcurveto{\pgfqpoint{0.984239in}{4.014950in}}{\pgfqpoint{0.988630in}{4.004351in}}{\pgfqpoint{0.996443in}{3.996537in}}%
\pgfpathcurveto{\pgfqpoint{1.004257in}{3.988724in}}{\pgfqpoint{1.014856in}{3.984333in}}{\pgfqpoint{1.025906in}{3.984333in}}%
\pgfpathclose%
\pgfusepath{stroke,fill}%
\end{pgfscope}%
\begin{pgfscope}%
\pgfpathrectangle{\pgfqpoint{0.800000in}{0.528000in}}{\pgfqpoint{4.960000in}{3.696000in}}%
\pgfusepath{clip}%
\pgfsetbuttcap%
\pgfsetroundjoin%
\definecolor{currentfill}{rgb}{0.000000,0.000000,0.000000}%
\pgfsetfillcolor{currentfill}%
\pgfsetlinewidth{1.003750pt}%
\definecolor{currentstroke}{rgb}{0.000000,0.000000,0.000000}%
\pgfsetstrokecolor{currentstroke}%
\pgfsetdash{}{0pt}%
\pgfpathmoveto{\pgfqpoint{1.025906in}{3.984333in}}%
\pgfpathcurveto{\pgfqpoint{1.036956in}{3.984333in}}{\pgfqpoint{1.047555in}{3.988724in}}{\pgfqpoint{1.055369in}{3.996537in}}%
\pgfpathcurveto{\pgfqpoint{1.063182in}{4.004351in}}{\pgfqpoint{1.067573in}{4.014950in}}{\pgfqpoint{1.067573in}{4.026000in}}%
\pgfpathcurveto{\pgfqpoint{1.067573in}{4.037050in}}{\pgfqpoint{1.063182in}{4.047649in}}{\pgfqpoint{1.055369in}{4.055463in}}%
\pgfpathcurveto{\pgfqpoint{1.047555in}{4.063276in}}{\pgfqpoint{1.036956in}{4.067667in}}{\pgfqpoint{1.025906in}{4.067667in}}%
\pgfpathcurveto{\pgfqpoint{1.014856in}{4.067667in}}{\pgfqpoint{1.004257in}{4.063276in}}{\pgfqpoint{0.996443in}{4.055463in}}%
\pgfpathcurveto{\pgfqpoint{0.988630in}{4.047649in}}{\pgfqpoint{0.984239in}{4.037050in}}{\pgfqpoint{0.984239in}{4.026000in}}%
\pgfpathcurveto{\pgfqpoint{0.984239in}{4.014950in}}{\pgfqpoint{0.988630in}{4.004351in}}{\pgfqpoint{0.996443in}{3.996537in}}%
\pgfpathcurveto{\pgfqpoint{1.004257in}{3.988724in}}{\pgfqpoint{1.014856in}{3.984333in}}{\pgfqpoint{1.025906in}{3.984333in}}%
\pgfpathclose%
\pgfusepath{stroke,fill}%
\end{pgfscope}%
\begin{pgfscope}%
\pgfpathrectangle{\pgfqpoint{0.800000in}{0.528000in}}{\pgfqpoint{4.960000in}{3.696000in}}%
\pgfusepath{clip}%
\pgfsetbuttcap%
\pgfsetroundjoin%
\definecolor{currentfill}{rgb}{0.000000,0.000000,0.000000}%
\pgfsetfillcolor{currentfill}%
\pgfsetlinewidth{1.003750pt}%
\definecolor{currentstroke}{rgb}{0.000000,0.000000,0.000000}%
\pgfsetstrokecolor{currentstroke}%
\pgfsetdash{}{0pt}%
\pgfpathmoveto{\pgfqpoint{1.025906in}{3.984333in}}%
\pgfpathcurveto{\pgfqpoint{1.036956in}{3.984333in}}{\pgfqpoint{1.047555in}{3.988724in}}{\pgfqpoint{1.055369in}{3.996537in}}%
\pgfpathcurveto{\pgfqpoint{1.063182in}{4.004351in}}{\pgfqpoint{1.067573in}{4.014950in}}{\pgfqpoint{1.067573in}{4.026000in}}%
\pgfpathcurveto{\pgfqpoint{1.067573in}{4.037050in}}{\pgfqpoint{1.063182in}{4.047649in}}{\pgfqpoint{1.055369in}{4.055463in}}%
\pgfpathcurveto{\pgfqpoint{1.047555in}{4.063276in}}{\pgfqpoint{1.036956in}{4.067667in}}{\pgfqpoint{1.025906in}{4.067667in}}%
\pgfpathcurveto{\pgfqpoint{1.014856in}{4.067667in}}{\pgfqpoint{1.004257in}{4.063276in}}{\pgfqpoint{0.996443in}{4.055463in}}%
\pgfpathcurveto{\pgfqpoint{0.988630in}{4.047649in}}{\pgfqpoint{0.984239in}{4.037050in}}{\pgfqpoint{0.984239in}{4.026000in}}%
\pgfpathcurveto{\pgfqpoint{0.984239in}{4.014950in}}{\pgfqpoint{0.988630in}{4.004351in}}{\pgfqpoint{0.996443in}{3.996537in}}%
\pgfpathcurveto{\pgfqpoint{1.004257in}{3.988724in}}{\pgfqpoint{1.014856in}{3.984333in}}{\pgfqpoint{1.025906in}{3.984333in}}%
\pgfpathclose%
\pgfusepath{stroke,fill}%
\end{pgfscope}%
\begin{pgfscope}%
\pgfpathrectangle{\pgfqpoint{0.800000in}{0.528000in}}{\pgfqpoint{4.960000in}{3.696000in}}%
\pgfusepath{clip}%
\pgfsetbuttcap%
\pgfsetroundjoin%
\definecolor{currentfill}{rgb}{0.000000,0.000000,0.000000}%
\pgfsetfillcolor{currentfill}%
\pgfsetlinewidth{1.003750pt}%
\definecolor{currentstroke}{rgb}{0.000000,0.000000,0.000000}%
\pgfsetstrokecolor{currentstroke}%
\pgfsetdash{}{0pt}%
\pgfpathmoveto{\pgfqpoint{1.025906in}{3.984333in}}%
\pgfpathcurveto{\pgfqpoint{1.036956in}{3.984333in}}{\pgfqpoint{1.047555in}{3.988724in}}{\pgfqpoint{1.055369in}{3.996537in}}%
\pgfpathcurveto{\pgfqpoint{1.063182in}{4.004351in}}{\pgfqpoint{1.067573in}{4.014950in}}{\pgfqpoint{1.067573in}{4.026000in}}%
\pgfpathcurveto{\pgfqpoint{1.067573in}{4.037050in}}{\pgfqpoint{1.063182in}{4.047649in}}{\pgfqpoint{1.055369in}{4.055463in}}%
\pgfpathcurveto{\pgfqpoint{1.047555in}{4.063276in}}{\pgfqpoint{1.036956in}{4.067667in}}{\pgfqpoint{1.025906in}{4.067667in}}%
\pgfpathcurveto{\pgfqpoint{1.014856in}{4.067667in}}{\pgfqpoint{1.004257in}{4.063276in}}{\pgfqpoint{0.996443in}{4.055463in}}%
\pgfpathcurveto{\pgfqpoint{0.988630in}{4.047649in}}{\pgfqpoint{0.984239in}{4.037050in}}{\pgfqpoint{0.984239in}{4.026000in}}%
\pgfpathcurveto{\pgfqpoint{0.984239in}{4.014950in}}{\pgfqpoint{0.988630in}{4.004351in}}{\pgfqpoint{0.996443in}{3.996537in}}%
\pgfpathcurveto{\pgfqpoint{1.004257in}{3.988724in}}{\pgfqpoint{1.014856in}{3.984333in}}{\pgfqpoint{1.025906in}{3.984333in}}%
\pgfpathclose%
\pgfusepath{stroke,fill}%
\end{pgfscope}%
\begin{pgfscope}%
\pgfpathrectangle{\pgfqpoint{0.800000in}{0.528000in}}{\pgfqpoint{4.960000in}{3.696000in}}%
\pgfusepath{clip}%
\pgfsetbuttcap%
\pgfsetroundjoin%
\definecolor{currentfill}{rgb}{0.000000,0.000000,0.000000}%
\pgfsetfillcolor{currentfill}%
\pgfsetlinewidth{1.003750pt}%
\definecolor{currentstroke}{rgb}{0.000000,0.000000,0.000000}%
\pgfsetstrokecolor{currentstroke}%
\pgfsetdash{}{0pt}%
\pgfpathmoveto{\pgfqpoint{1.025906in}{3.984333in}}%
\pgfpathcurveto{\pgfqpoint{1.036956in}{3.984333in}}{\pgfqpoint{1.047555in}{3.988724in}}{\pgfqpoint{1.055369in}{3.996537in}}%
\pgfpathcurveto{\pgfqpoint{1.063182in}{4.004351in}}{\pgfqpoint{1.067573in}{4.014950in}}{\pgfqpoint{1.067573in}{4.026000in}}%
\pgfpathcurveto{\pgfqpoint{1.067573in}{4.037050in}}{\pgfqpoint{1.063182in}{4.047649in}}{\pgfqpoint{1.055369in}{4.055463in}}%
\pgfpathcurveto{\pgfqpoint{1.047555in}{4.063276in}}{\pgfqpoint{1.036956in}{4.067667in}}{\pgfqpoint{1.025906in}{4.067667in}}%
\pgfpathcurveto{\pgfqpoint{1.014856in}{4.067667in}}{\pgfqpoint{1.004257in}{4.063276in}}{\pgfqpoint{0.996443in}{4.055463in}}%
\pgfpathcurveto{\pgfqpoint{0.988630in}{4.047649in}}{\pgfqpoint{0.984239in}{4.037050in}}{\pgfqpoint{0.984239in}{4.026000in}}%
\pgfpathcurveto{\pgfqpoint{0.984239in}{4.014950in}}{\pgfqpoint{0.988630in}{4.004351in}}{\pgfqpoint{0.996443in}{3.996537in}}%
\pgfpathcurveto{\pgfqpoint{1.004257in}{3.988724in}}{\pgfqpoint{1.014856in}{3.984333in}}{\pgfqpoint{1.025906in}{3.984333in}}%
\pgfpathclose%
\pgfusepath{stroke,fill}%
\end{pgfscope}%
\begin{pgfscope}%
\pgfpathrectangle{\pgfqpoint{0.800000in}{0.528000in}}{\pgfqpoint{4.960000in}{3.696000in}}%
\pgfusepath{clip}%
\pgfsetbuttcap%
\pgfsetroundjoin%
\definecolor{currentfill}{rgb}{0.000000,0.000000,0.000000}%
\pgfsetfillcolor{currentfill}%
\pgfsetlinewidth{1.003750pt}%
\definecolor{currentstroke}{rgb}{0.000000,0.000000,0.000000}%
\pgfsetstrokecolor{currentstroke}%
\pgfsetdash{}{0pt}%
\pgfpathmoveto{\pgfqpoint{1.025906in}{3.984333in}}%
\pgfpathcurveto{\pgfqpoint{1.036956in}{3.984333in}}{\pgfqpoint{1.047555in}{3.988724in}}{\pgfqpoint{1.055369in}{3.996537in}}%
\pgfpathcurveto{\pgfqpoint{1.063182in}{4.004351in}}{\pgfqpoint{1.067573in}{4.014950in}}{\pgfqpoint{1.067573in}{4.026000in}}%
\pgfpathcurveto{\pgfqpoint{1.067573in}{4.037050in}}{\pgfqpoint{1.063182in}{4.047649in}}{\pgfqpoint{1.055369in}{4.055463in}}%
\pgfpathcurveto{\pgfqpoint{1.047555in}{4.063276in}}{\pgfqpoint{1.036956in}{4.067667in}}{\pgfqpoint{1.025906in}{4.067667in}}%
\pgfpathcurveto{\pgfqpoint{1.014856in}{4.067667in}}{\pgfqpoint{1.004257in}{4.063276in}}{\pgfqpoint{0.996443in}{4.055463in}}%
\pgfpathcurveto{\pgfqpoint{0.988630in}{4.047649in}}{\pgfqpoint{0.984239in}{4.037050in}}{\pgfqpoint{0.984239in}{4.026000in}}%
\pgfpathcurveto{\pgfqpoint{0.984239in}{4.014950in}}{\pgfqpoint{0.988630in}{4.004351in}}{\pgfqpoint{0.996443in}{3.996537in}}%
\pgfpathcurveto{\pgfqpoint{1.004257in}{3.988724in}}{\pgfqpoint{1.014856in}{3.984333in}}{\pgfqpoint{1.025906in}{3.984333in}}%
\pgfpathclose%
\pgfusepath{stroke,fill}%
\end{pgfscope}%
\begin{pgfscope}%
\pgfpathrectangle{\pgfqpoint{0.800000in}{0.528000in}}{\pgfqpoint{4.960000in}{3.696000in}}%
\pgfusepath{clip}%
\pgfsetbuttcap%
\pgfsetroundjoin%
\definecolor{currentfill}{rgb}{0.000000,0.000000,0.000000}%
\pgfsetfillcolor{currentfill}%
\pgfsetlinewidth{1.003750pt}%
\definecolor{currentstroke}{rgb}{0.000000,0.000000,0.000000}%
\pgfsetstrokecolor{currentstroke}%
\pgfsetdash{}{0pt}%
\pgfpathmoveto{\pgfqpoint{1.025906in}{3.984333in}}%
\pgfpathcurveto{\pgfqpoint{1.036956in}{3.984333in}}{\pgfqpoint{1.047555in}{3.988724in}}{\pgfqpoint{1.055369in}{3.996537in}}%
\pgfpathcurveto{\pgfqpoint{1.063182in}{4.004351in}}{\pgfqpoint{1.067573in}{4.014950in}}{\pgfqpoint{1.067573in}{4.026000in}}%
\pgfpathcurveto{\pgfqpoint{1.067573in}{4.037050in}}{\pgfqpoint{1.063182in}{4.047649in}}{\pgfqpoint{1.055369in}{4.055463in}}%
\pgfpathcurveto{\pgfqpoint{1.047555in}{4.063276in}}{\pgfqpoint{1.036956in}{4.067667in}}{\pgfqpoint{1.025906in}{4.067667in}}%
\pgfpathcurveto{\pgfqpoint{1.014856in}{4.067667in}}{\pgfqpoint{1.004257in}{4.063276in}}{\pgfqpoint{0.996443in}{4.055463in}}%
\pgfpathcurveto{\pgfqpoint{0.988630in}{4.047649in}}{\pgfqpoint{0.984239in}{4.037050in}}{\pgfqpoint{0.984239in}{4.026000in}}%
\pgfpathcurveto{\pgfqpoint{0.984239in}{4.014950in}}{\pgfqpoint{0.988630in}{4.004351in}}{\pgfqpoint{0.996443in}{3.996537in}}%
\pgfpathcurveto{\pgfqpoint{1.004257in}{3.988724in}}{\pgfqpoint{1.014856in}{3.984333in}}{\pgfqpoint{1.025906in}{3.984333in}}%
\pgfpathclose%
\pgfusepath{stroke,fill}%
\end{pgfscope}%
\begin{pgfscope}%
\pgfpathrectangle{\pgfqpoint{0.800000in}{0.528000in}}{\pgfqpoint{4.960000in}{3.696000in}}%
\pgfusepath{clip}%
\pgfsetbuttcap%
\pgfsetroundjoin%
\definecolor{currentfill}{rgb}{0.000000,0.000000,0.000000}%
\pgfsetfillcolor{currentfill}%
\pgfsetlinewidth{1.003750pt}%
\definecolor{currentstroke}{rgb}{0.000000,0.000000,0.000000}%
\pgfsetstrokecolor{currentstroke}%
\pgfsetdash{}{0pt}%
\pgfpathmoveto{\pgfqpoint{1.025906in}{3.984333in}}%
\pgfpathcurveto{\pgfqpoint{1.036956in}{3.984333in}}{\pgfqpoint{1.047555in}{3.988724in}}{\pgfqpoint{1.055369in}{3.996537in}}%
\pgfpathcurveto{\pgfqpoint{1.063182in}{4.004351in}}{\pgfqpoint{1.067573in}{4.014950in}}{\pgfqpoint{1.067573in}{4.026000in}}%
\pgfpathcurveto{\pgfqpoint{1.067573in}{4.037050in}}{\pgfqpoint{1.063182in}{4.047649in}}{\pgfqpoint{1.055369in}{4.055463in}}%
\pgfpathcurveto{\pgfqpoint{1.047555in}{4.063276in}}{\pgfqpoint{1.036956in}{4.067667in}}{\pgfqpoint{1.025906in}{4.067667in}}%
\pgfpathcurveto{\pgfqpoint{1.014856in}{4.067667in}}{\pgfqpoint{1.004257in}{4.063276in}}{\pgfqpoint{0.996443in}{4.055463in}}%
\pgfpathcurveto{\pgfqpoint{0.988630in}{4.047649in}}{\pgfqpoint{0.984239in}{4.037050in}}{\pgfqpoint{0.984239in}{4.026000in}}%
\pgfpathcurveto{\pgfqpoint{0.984239in}{4.014950in}}{\pgfqpoint{0.988630in}{4.004351in}}{\pgfqpoint{0.996443in}{3.996537in}}%
\pgfpathcurveto{\pgfqpoint{1.004257in}{3.988724in}}{\pgfqpoint{1.014856in}{3.984333in}}{\pgfqpoint{1.025906in}{3.984333in}}%
\pgfpathclose%
\pgfusepath{stroke,fill}%
\end{pgfscope}%
\begin{pgfscope}%
\pgfpathrectangle{\pgfqpoint{0.800000in}{0.528000in}}{\pgfqpoint{4.960000in}{3.696000in}}%
\pgfusepath{clip}%
\pgfsetbuttcap%
\pgfsetroundjoin%
\definecolor{currentfill}{rgb}{0.000000,0.000000,0.000000}%
\pgfsetfillcolor{currentfill}%
\pgfsetlinewidth{1.003750pt}%
\definecolor{currentstroke}{rgb}{0.000000,0.000000,0.000000}%
\pgfsetstrokecolor{currentstroke}%
\pgfsetdash{}{0pt}%
\pgfpathmoveto{\pgfqpoint{1.025906in}{3.984333in}}%
\pgfpathcurveto{\pgfqpoint{1.036956in}{3.984333in}}{\pgfqpoint{1.047555in}{3.988724in}}{\pgfqpoint{1.055369in}{3.996537in}}%
\pgfpathcurveto{\pgfqpoint{1.063182in}{4.004351in}}{\pgfqpoint{1.067573in}{4.014950in}}{\pgfqpoint{1.067573in}{4.026000in}}%
\pgfpathcurveto{\pgfqpoint{1.067573in}{4.037050in}}{\pgfqpoint{1.063182in}{4.047649in}}{\pgfqpoint{1.055369in}{4.055463in}}%
\pgfpathcurveto{\pgfqpoint{1.047555in}{4.063276in}}{\pgfqpoint{1.036956in}{4.067667in}}{\pgfqpoint{1.025906in}{4.067667in}}%
\pgfpathcurveto{\pgfqpoint{1.014856in}{4.067667in}}{\pgfqpoint{1.004257in}{4.063276in}}{\pgfqpoint{0.996443in}{4.055463in}}%
\pgfpathcurveto{\pgfqpoint{0.988630in}{4.047649in}}{\pgfqpoint{0.984239in}{4.037050in}}{\pgfqpoint{0.984239in}{4.026000in}}%
\pgfpathcurveto{\pgfqpoint{0.984239in}{4.014950in}}{\pgfqpoint{0.988630in}{4.004351in}}{\pgfqpoint{0.996443in}{3.996537in}}%
\pgfpathcurveto{\pgfqpoint{1.004257in}{3.988724in}}{\pgfqpoint{1.014856in}{3.984333in}}{\pgfqpoint{1.025906in}{3.984333in}}%
\pgfpathclose%
\pgfusepath{stroke,fill}%
\end{pgfscope}%
\begin{pgfscope}%
\pgfpathrectangle{\pgfqpoint{0.800000in}{0.528000in}}{\pgfqpoint{4.960000in}{3.696000in}}%
\pgfusepath{clip}%
\pgfsetbuttcap%
\pgfsetroundjoin%
\definecolor{currentfill}{rgb}{0.000000,0.000000,0.000000}%
\pgfsetfillcolor{currentfill}%
\pgfsetlinewidth{1.003750pt}%
\definecolor{currentstroke}{rgb}{0.000000,0.000000,0.000000}%
\pgfsetstrokecolor{currentstroke}%
\pgfsetdash{}{0pt}%
\pgfpathmoveto{\pgfqpoint{1.025906in}{3.984333in}}%
\pgfpathcurveto{\pgfqpoint{1.036956in}{3.984333in}}{\pgfqpoint{1.047555in}{3.988724in}}{\pgfqpoint{1.055369in}{3.996537in}}%
\pgfpathcurveto{\pgfqpoint{1.063182in}{4.004351in}}{\pgfqpoint{1.067573in}{4.014950in}}{\pgfqpoint{1.067573in}{4.026000in}}%
\pgfpathcurveto{\pgfqpoint{1.067573in}{4.037050in}}{\pgfqpoint{1.063182in}{4.047649in}}{\pgfqpoint{1.055369in}{4.055463in}}%
\pgfpathcurveto{\pgfqpoint{1.047555in}{4.063276in}}{\pgfqpoint{1.036956in}{4.067667in}}{\pgfqpoint{1.025906in}{4.067667in}}%
\pgfpathcurveto{\pgfqpoint{1.014856in}{4.067667in}}{\pgfqpoint{1.004257in}{4.063276in}}{\pgfqpoint{0.996443in}{4.055463in}}%
\pgfpathcurveto{\pgfqpoint{0.988630in}{4.047649in}}{\pgfqpoint{0.984239in}{4.037050in}}{\pgfqpoint{0.984239in}{4.026000in}}%
\pgfpathcurveto{\pgfqpoint{0.984239in}{4.014950in}}{\pgfqpoint{0.988630in}{4.004351in}}{\pgfqpoint{0.996443in}{3.996537in}}%
\pgfpathcurveto{\pgfqpoint{1.004257in}{3.988724in}}{\pgfqpoint{1.014856in}{3.984333in}}{\pgfqpoint{1.025906in}{3.984333in}}%
\pgfpathclose%
\pgfusepath{stroke,fill}%
\end{pgfscope}%
\begin{pgfscope}%
\pgfpathrectangle{\pgfqpoint{0.800000in}{0.528000in}}{\pgfqpoint{4.960000in}{3.696000in}}%
\pgfusepath{clip}%
\pgfsetbuttcap%
\pgfsetroundjoin%
\definecolor{currentfill}{rgb}{0.000000,0.000000,0.000000}%
\pgfsetfillcolor{currentfill}%
\pgfsetlinewidth{1.003750pt}%
\definecolor{currentstroke}{rgb}{0.000000,0.000000,0.000000}%
\pgfsetstrokecolor{currentstroke}%
\pgfsetdash{}{0pt}%
\pgfpathmoveto{\pgfqpoint{1.025906in}{3.984333in}}%
\pgfpathcurveto{\pgfqpoint{1.036956in}{3.984333in}}{\pgfqpoint{1.047555in}{3.988724in}}{\pgfqpoint{1.055369in}{3.996537in}}%
\pgfpathcurveto{\pgfqpoint{1.063182in}{4.004351in}}{\pgfqpoint{1.067573in}{4.014950in}}{\pgfqpoint{1.067573in}{4.026000in}}%
\pgfpathcurveto{\pgfqpoint{1.067573in}{4.037050in}}{\pgfqpoint{1.063182in}{4.047649in}}{\pgfqpoint{1.055369in}{4.055463in}}%
\pgfpathcurveto{\pgfqpoint{1.047555in}{4.063276in}}{\pgfqpoint{1.036956in}{4.067667in}}{\pgfqpoint{1.025906in}{4.067667in}}%
\pgfpathcurveto{\pgfqpoint{1.014856in}{4.067667in}}{\pgfqpoint{1.004257in}{4.063276in}}{\pgfqpoint{0.996443in}{4.055463in}}%
\pgfpathcurveto{\pgfqpoint{0.988630in}{4.047649in}}{\pgfqpoint{0.984239in}{4.037050in}}{\pgfqpoint{0.984239in}{4.026000in}}%
\pgfpathcurveto{\pgfqpoint{0.984239in}{4.014950in}}{\pgfqpoint{0.988630in}{4.004351in}}{\pgfqpoint{0.996443in}{3.996537in}}%
\pgfpathcurveto{\pgfqpoint{1.004257in}{3.988724in}}{\pgfqpoint{1.014856in}{3.984333in}}{\pgfqpoint{1.025906in}{3.984333in}}%
\pgfpathclose%
\pgfusepath{stroke,fill}%
\end{pgfscope}%
\begin{pgfscope}%
\pgfpathrectangle{\pgfqpoint{0.800000in}{0.528000in}}{\pgfqpoint{4.960000in}{3.696000in}}%
\pgfusepath{clip}%
\pgfsetbuttcap%
\pgfsetroundjoin%
\definecolor{currentfill}{rgb}{0.000000,0.000000,0.000000}%
\pgfsetfillcolor{currentfill}%
\pgfsetlinewidth{1.003750pt}%
\definecolor{currentstroke}{rgb}{0.000000,0.000000,0.000000}%
\pgfsetstrokecolor{currentstroke}%
\pgfsetdash{}{0pt}%
\pgfpathmoveto{\pgfqpoint{1.025906in}{3.984333in}}%
\pgfpathcurveto{\pgfqpoint{1.036956in}{3.984333in}}{\pgfqpoint{1.047555in}{3.988724in}}{\pgfqpoint{1.055369in}{3.996537in}}%
\pgfpathcurveto{\pgfqpoint{1.063182in}{4.004351in}}{\pgfqpoint{1.067573in}{4.014950in}}{\pgfqpoint{1.067573in}{4.026000in}}%
\pgfpathcurveto{\pgfqpoint{1.067573in}{4.037050in}}{\pgfqpoint{1.063182in}{4.047649in}}{\pgfqpoint{1.055369in}{4.055463in}}%
\pgfpathcurveto{\pgfqpoint{1.047555in}{4.063276in}}{\pgfqpoint{1.036956in}{4.067667in}}{\pgfqpoint{1.025906in}{4.067667in}}%
\pgfpathcurveto{\pgfqpoint{1.014856in}{4.067667in}}{\pgfqpoint{1.004257in}{4.063276in}}{\pgfqpoint{0.996443in}{4.055463in}}%
\pgfpathcurveto{\pgfqpoint{0.988630in}{4.047649in}}{\pgfqpoint{0.984239in}{4.037050in}}{\pgfqpoint{0.984239in}{4.026000in}}%
\pgfpathcurveto{\pgfqpoint{0.984239in}{4.014950in}}{\pgfqpoint{0.988630in}{4.004351in}}{\pgfqpoint{0.996443in}{3.996537in}}%
\pgfpathcurveto{\pgfqpoint{1.004257in}{3.988724in}}{\pgfqpoint{1.014856in}{3.984333in}}{\pgfqpoint{1.025906in}{3.984333in}}%
\pgfpathclose%
\pgfusepath{stroke,fill}%
\end{pgfscope}%
\begin{pgfscope}%
\pgfpathrectangle{\pgfqpoint{0.800000in}{0.528000in}}{\pgfqpoint{4.960000in}{3.696000in}}%
\pgfusepath{clip}%
\pgfsetbuttcap%
\pgfsetroundjoin%
\definecolor{currentfill}{rgb}{0.000000,0.000000,0.000000}%
\pgfsetfillcolor{currentfill}%
\pgfsetlinewidth{1.003750pt}%
\definecolor{currentstroke}{rgb}{0.000000,0.000000,0.000000}%
\pgfsetstrokecolor{currentstroke}%
\pgfsetdash{}{0pt}%
\pgfpathmoveto{\pgfqpoint{1.025906in}{3.984333in}}%
\pgfpathcurveto{\pgfqpoint{1.036956in}{3.984333in}}{\pgfqpoint{1.047555in}{3.988724in}}{\pgfqpoint{1.055369in}{3.996537in}}%
\pgfpathcurveto{\pgfqpoint{1.063182in}{4.004351in}}{\pgfqpoint{1.067573in}{4.014950in}}{\pgfqpoint{1.067573in}{4.026000in}}%
\pgfpathcurveto{\pgfqpoint{1.067573in}{4.037050in}}{\pgfqpoint{1.063182in}{4.047649in}}{\pgfqpoint{1.055369in}{4.055463in}}%
\pgfpathcurveto{\pgfqpoint{1.047555in}{4.063276in}}{\pgfqpoint{1.036956in}{4.067667in}}{\pgfqpoint{1.025906in}{4.067667in}}%
\pgfpathcurveto{\pgfqpoint{1.014856in}{4.067667in}}{\pgfqpoint{1.004257in}{4.063276in}}{\pgfqpoint{0.996443in}{4.055463in}}%
\pgfpathcurveto{\pgfqpoint{0.988630in}{4.047649in}}{\pgfqpoint{0.984239in}{4.037050in}}{\pgfqpoint{0.984239in}{4.026000in}}%
\pgfpathcurveto{\pgfqpoint{0.984239in}{4.014950in}}{\pgfqpoint{0.988630in}{4.004351in}}{\pgfqpoint{0.996443in}{3.996537in}}%
\pgfpathcurveto{\pgfqpoint{1.004257in}{3.988724in}}{\pgfqpoint{1.014856in}{3.984333in}}{\pgfqpoint{1.025906in}{3.984333in}}%
\pgfpathclose%
\pgfusepath{stroke,fill}%
\end{pgfscope}%
\begin{pgfscope}%
\pgfpathrectangle{\pgfqpoint{0.800000in}{0.528000in}}{\pgfqpoint{4.960000in}{3.696000in}}%
\pgfusepath{clip}%
\pgfsetbuttcap%
\pgfsetroundjoin%
\definecolor{currentfill}{rgb}{0.000000,0.000000,0.000000}%
\pgfsetfillcolor{currentfill}%
\pgfsetlinewidth{1.003750pt}%
\definecolor{currentstroke}{rgb}{0.000000,0.000000,0.000000}%
\pgfsetstrokecolor{currentstroke}%
\pgfsetdash{}{0pt}%
\pgfpathmoveto{\pgfqpoint{1.025906in}{3.984333in}}%
\pgfpathcurveto{\pgfqpoint{1.036956in}{3.984333in}}{\pgfqpoint{1.047555in}{3.988724in}}{\pgfqpoint{1.055369in}{3.996537in}}%
\pgfpathcurveto{\pgfqpoint{1.063182in}{4.004351in}}{\pgfqpoint{1.067573in}{4.014950in}}{\pgfqpoint{1.067573in}{4.026000in}}%
\pgfpathcurveto{\pgfqpoint{1.067573in}{4.037050in}}{\pgfqpoint{1.063182in}{4.047649in}}{\pgfqpoint{1.055369in}{4.055463in}}%
\pgfpathcurveto{\pgfqpoint{1.047555in}{4.063276in}}{\pgfqpoint{1.036956in}{4.067667in}}{\pgfqpoint{1.025906in}{4.067667in}}%
\pgfpathcurveto{\pgfqpoint{1.014856in}{4.067667in}}{\pgfqpoint{1.004257in}{4.063276in}}{\pgfqpoint{0.996443in}{4.055463in}}%
\pgfpathcurveto{\pgfqpoint{0.988630in}{4.047649in}}{\pgfqpoint{0.984239in}{4.037050in}}{\pgfqpoint{0.984239in}{4.026000in}}%
\pgfpathcurveto{\pgfqpoint{0.984239in}{4.014950in}}{\pgfqpoint{0.988630in}{4.004351in}}{\pgfqpoint{0.996443in}{3.996537in}}%
\pgfpathcurveto{\pgfqpoint{1.004257in}{3.988724in}}{\pgfqpoint{1.014856in}{3.984333in}}{\pgfqpoint{1.025906in}{3.984333in}}%
\pgfpathclose%
\pgfusepath{stroke,fill}%
\end{pgfscope}%
\begin{pgfscope}%
\pgfpathrectangle{\pgfqpoint{0.800000in}{0.528000in}}{\pgfqpoint{4.960000in}{3.696000in}}%
\pgfusepath{clip}%
\pgfsetbuttcap%
\pgfsetroundjoin%
\definecolor{currentfill}{rgb}{0.000000,0.000000,0.000000}%
\pgfsetfillcolor{currentfill}%
\pgfsetlinewidth{1.003750pt}%
\definecolor{currentstroke}{rgb}{0.000000,0.000000,0.000000}%
\pgfsetstrokecolor{currentstroke}%
\pgfsetdash{}{0pt}%
\pgfpathmoveto{\pgfqpoint{1.025906in}{3.984333in}}%
\pgfpathcurveto{\pgfqpoint{1.036956in}{3.984333in}}{\pgfqpoint{1.047555in}{3.988724in}}{\pgfqpoint{1.055369in}{3.996537in}}%
\pgfpathcurveto{\pgfqpoint{1.063182in}{4.004351in}}{\pgfqpoint{1.067573in}{4.014950in}}{\pgfqpoint{1.067573in}{4.026000in}}%
\pgfpathcurveto{\pgfqpoint{1.067573in}{4.037050in}}{\pgfqpoint{1.063182in}{4.047649in}}{\pgfqpoint{1.055369in}{4.055463in}}%
\pgfpathcurveto{\pgfqpoint{1.047555in}{4.063276in}}{\pgfqpoint{1.036956in}{4.067667in}}{\pgfqpoint{1.025906in}{4.067667in}}%
\pgfpathcurveto{\pgfqpoint{1.014856in}{4.067667in}}{\pgfqpoint{1.004257in}{4.063276in}}{\pgfqpoint{0.996443in}{4.055463in}}%
\pgfpathcurveto{\pgfqpoint{0.988630in}{4.047649in}}{\pgfqpoint{0.984239in}{4.037050in}}{\pgfqpoint{0.984239in}{4.026000in}}%
\pgfpathcurveto{\pgfqpoint{0.984239in}{4.014950in}}{\pgfqpoint{0.988630in}{4.004351in}}{\pgfqpoint{0.996443in}{3.996537in}}%
\pgfpathcurveto{\pgfqpoint{1.004257in}{3.988724in}}{\pgfqpoint{1.014856in}{3.984333in}}{\pgfqpoint{1.025906in}{3.984333in}}%
\pgfpathclose%
\pgfusepath{stroke,fill}%
\end{pgfscope}%
\begin{pgfscope}%
\pgfpathrectangle{\pgfqpoint{0.800000in}{0.528000in}}{\pgfqpoint{4.960000in}{3.696000in}}%
\pgfusepath{clip}%
\pgfsetbuttcap%
\pgfsetroundjoin%
\definecolor{currentfill}{rgb}{0.000000,0.000000,0.000000}%
\pgfsetfillcolor{currentfill}%
\pgfsetlinewidth{1.003750pt}%
\definecolor{currentstroke}{rgb}{0.000000,0.000000,0.000000}%
\pgfsetstrokecolor{currentstroke}%
\pgfsetdash{}{0pt}%
\pgfpathmoveto{\pgfqpoint{1.025906in}{3.984333in}}%
\pgfpathcurveto{\pgfqpoint{1.036956in}{3.984333in}}{\pgfqpoint{1.047555in}{3.988724in}}{\pgfqpoint{1.055369in}{3.996537in}}%
\pgfpathcurveto{\pgfqpoint{1.063182in}{4.004351in}}{\pgfqpoint{1.067573in}{4.014950in}}{\pgfqpoint{1.067573in}{4.026000in}}%
\pgfpathcurveto{\pgfqpoint{1.067573in}{4.037050in}}{\pgfqpoint{1.063182in}{4.047649in}}{\pgfqpoint{1.055369in}{4.055463in}}%
\pgfpathcurveto{\pgfqpoint{1.047555in}{4.063276in}}{\pgfqpoint{1.036956in}{4.067667in}}{\pgfqpoint{1.025906in}{4.067667in}}%
\pgfpathcurveto{\pgfqpoint{1.014856in}{4.067667in}}{\pgfqpoint{1.004257in}{4.063276in}}{\pgfqpoint{0.996443in}{4.055463in}}%
\pgfpathcurveto{\pgfqpoint{0.988630in}{4.047649in}}{\pgfqpoint{0.984239in}{4.037050in}}{\pgfqpoint{0.984239in}{4.026000in}}%
\pgfpathcurveto{\pgfqpoint{0.984239in}{4.014950in}}{\pgfqpoint{0.988630in}{4.004351in}}{\pgfqpoint{0.996443in}{3.996537in}}%
\pgfpathcurveto{\pgfqpoint{1.004257in}{3.988724in}}{\pgfqpoint{1.014856in}{3.984333in}}{\pgfqpoint{1.025906in}{3.984333in}}%
\pgfpathclose%
\pgfusepath{stroke,fill}%
\end{pgfscope}%
\begin{pgfscope}%
\pgfpathrectangle{\pgfqpoint{0.800000in}{0.528000in}}{\pgfqpoint{4.960000in}{3.696000in}}%
\pgfusepath{clip}%
\pgfsetbuttcap%
\pgfsetroundjoin%
\definecolor{currentfill}{rgb}{0.000000,0.000000,0.000000}%
\pgfsetfillcolor{currentfill}%
\pgfsetlinewidth{1.003750pt}%
\definecolor{currentstroke}{rgb}{0.000000,0.000000,0.000000}%
\pgfsetstrokecolor{currentstroke}%
\pgfsetdash{}{0pt}%
\pgfpathmoveto{\pgfqpoint{1.025906in}{3.984333in}}%
\pgfpathcurveto{\pgfqpoint{1.036956in}{3.984333in}}{\pgfqpoint{1.047555in}{3.988724in}}{\pgfqpoint{1.055369in}{3.996537in}}%
\pgfpathcurveto{\pgfqpoint{1.063182in}{4.004351in}}{\pgfqpoint{1.067573in}{4.014950in}}{\pgfqpoint{1.067573in}{4.026000in}}%
\pgfpathcurveto{\pgfqpoint{1.067573in}{4.037050in}}{\pgfqpoint{1.063182in}{4.047649in}}{\pgfqpoint{1.055369in}{4.055463in}}%
\pgfpathcurveto{\pgfqpoint{1.047555in}{4.063276in}}{\pgfqpoint{1.036956in}{4.067667in}}{\pgfqpoint{1.025906in}{4.067667in}}%
\pgfpathcurveto{\pgfqpoint{1.014856in}{4.067667in}}{\pgfqpoint{1.004257in}{4.063276in}}{\pgfqpoint{0.996443in}{4.055463in}}%
\pgfpathcurveto{\pgfqpoint{0.988630in}{4.047649in}}{\pgfqpoint{0.984239in}{4.037050in}}{\pgfqpoint{0.984239in}{4.026000in}}%
\pgfpathcurveto{\pgfqpoint{0.984239in}{4.014950in}}{\pgfqpoint{0.988630in}{4.004351in}}{\pgfqpoint{0.996443in}{3.996537in}}%
\pgfpathcurveto{\pgfqpoint{1.004257in}{3.988724in}}{\pgfqpoint{1.014856in}{3.984333in}}{\pgfqpoint{1.025906in}{3.984333in}}%
\pgfpathclose%
\pgfusepath{stroke,fill}%
\end{pgfscope}%
\begin{pgfscope}%
\pgfpathrectangle{\pgfqpoint{0.800000in}{0.528000in}}{\pgfqpoint{4.960000in}{3.696000in}}%
\pgfusepath{clip}%
\pgfsetbuttcap%
\pgfsetroundjoin%
\definecolor{currentfill}{rgb}{0.000000,0.000000,0.000000}%
\pgfsetfillcolor{currentfill}%
\pgfsetlinewidth{1.003750pt}%
\definecolor{currentstroke}{rgb}{0.000000,0.000000,0.000000}%
\pgfsetstrokecolor{currentstroke}%
\pgfsetdash{}{0pt}%
\pgfpathmoveto{\pgfqpoint{1.025906in}{3.984333in}}%
\pgfpathcurveto{\pgfqpoint{1.036956in}{3.984333in}}{\pgfqpoint{1.047555in}{3.988724in}}{\pgfqpoint{1.055369in}{3.996537in}}%
\pgfpathcurveto{\pgfqpoint{1.063182in}{4.004351in}}{\pgfqpoint{1.067573in}{4.014950in}}{\pgfqpoint{1.067573in}{4.026000in}}%
\pgfpathcurveto{\pgfqpoint{1.067573in}{4.037050in}}{\pgfqpoint{1.063182in}{4.047649in}}{\pgfqpoint{1.055369in}{4.055463in}}%
\pgfpathcurveto{\pgfqpoint{1.047555in}{4.063276in}}{\pgfqpoint{1.036956in}{4.067667in}}{\pgfqpoint{1.025906in}{4.067667in}}%
\pgfpathcurveto{\pgfqpoint{1.014856in}{4.067667in}}{\pgfqpoint{1.004257in}{4.063276in}}{\pgfqpoint{0.996443in}{4.055463in}}%
\pgfpathcurveto{\pgfqpoint{0.988630in}{4.047649in}}{\pgfqpoint{0.984239in}{4.037050in}}{\pgfqpoint{0.984239in}{4.026000in}}%
\pgfpathcurveto{\pgfqpoint{0.984239in}{4.014950in}}{\pgfqpoint{0.988630in}{4.004351in}}{\pgfqpoint{0.996443in}{3.996537in}}%
\pgfpathcurveto{\pgfqpoint{1.004257in}{3.988724in}}{\pgfqpoint{1.014856in}{3.984333in}}{\pgfqpoint{1.025906in}{3.984333in}}%
\pgfpathclose%
\pgfusepath{stroke,fill}%
\end{pgfscope}%
\begin{pgfscope}%
\pgfpathrectangle{\pgfqpoint{0.800000in}{0.528000in}}{\pgfqpoint{4.960000in}{3.696000in}}%
\pgfusepath{clip}%
\pgfsetbuttcap%
\pgfsetroundjoin%
\definecolor{currentfill}{rgb}{0.000000,0.000000,0.000000}%
\pgfsetfillcolor{currentfill}%
\pgfsetlinewidth{1.003750pt}%
\definecolor{currentstroke}{rgb}{0.000000,0.000000,0.000000}%
\pgfsetstrokecolor{currentstroke}%
\pgfsetdash{}{0pt}%
\pgfpathmoveto{\pgfqpoint{1.025906in}{3.984333in}}%
\pgfpathcurveto{\pgfqpoint{1.036956in}{3.984333in}}{\pgfqpoint{1.047555in}{3.988724in}}{\pgfqpoint{1.055369in}{3.996537in}}%
\pgfpathcurveto{\pgfqpoint{1.063182in}{4.004351in}}{\pgfqpoint{1.067573in}{4.014950in}}{\pgfqpoint{1.067573in}{4.026000in}}%
\pgfpathcurveto{\pgfqpoint{1.067573in}{4.037050in}}{\pgfqpoint{1.063182in}{4.047649in}}{\pgfqpoint{1.055369in}{4.055463in}}%
\pgfpathcurveto{\pgfqpoint{1.047555in}{4.063276in}}{\pgfqpoint{1.036956in}{4.067667in}}{\pgfqpoint{1.025906in}{4.067667in}}%
\pgfpathcurveto{\pgfqpoint{1.014856in}{4.067667in}}{\pgfqpoint{1.004257in}{4.063276in}}{\pgfqpoint{0.996443in}{4.055463in}}%
\pgfpathcurveto{\pgfqpoint{0.988630in}{4.047649in}}{\pgfqpoint{0.984239in}{4.037050in}}{\pgfqpoint{0.984239in}{4.026000in}}%
\pgfpathcurveto{\pgfqpoint{0.984239in}{4.014950in}}{\pgfqpoint{0.988630in}{4.004351in}}{\pgfqpoint{0.996443in}{3.996537in}}%
\pgfpathcurveto{\pgfqpoint{1.004257in}{3.988724in}}{\pgfqpoint{1.014856in}{3.984333in}}{\pgfqpoint{1.025906in}{3.984333in}}%
\pgfpathclose%
\pgfusepath{stroke,fill}%
\end{pgfscope}%
\begin{pgfscope}%
\pgfpathrectangle{\pgfqpoint{0.800000in}{0.528000in}}{\pgfqpoint{4.960000in}{3.696000in}}%
\pgfusepath{clip}%
\pgfsetbuttcap%
\pgfsetroundjoin%
\definecolor{currentfill}{rgb}{0.000000,0.000000,0.000000}%
\pgfsetfillcolor{currentfill}%
\pgfsetlinewidth{1.003750pt}%
\definecolor{currentstroke}{rgb}{0.000000,0.000000,0.000000}%
\pgfsetstrokecolor{currentstroke}%
\pgfsetdash{}{0pt}%
\pgfpathmoveto{\pgfqpoint{1.025906in}{3.984333in}}%
\pgfpathcurveto{\pgfqpoint{1.036956in}{3.984333in}}{\pgfqpoint{1.047555in}{3.988724in}}{\pgfqpoint{1.055369in}{3.996537in}}%
\pgfpathcurveto{\pgfqpoint{1.063182in}{4.004351in}}{\pgfqpoint{1.067573in}{4.014950in}}{\pgfqpoint{1.067573in}{4.026000in}}%
\pgfpathcurveto{\pgfqpoint{1.067573in}{4.037050in}}{\pgfqpoint{1.063182in}{4.047649in}}{\pgfqpoint{1.055369in}{4.055463in}}%
\pgfpathcurveto{\pgfqpoint{1.047555in}{4.063276in}}{\pgfqpoint{1.036956in}{4.067667in}}{\pgfqpoint{1.025906in}{4.067667in}}%
\pgfpathcurveto{\pgfqpoint{1.014856in}{4.067667in}}{\pgfqpoint{1.004257in}{4.063276in}}{\pgfqpoint{0.996443in}{4.055463in}}%
\pgfpathcurveto{\pgfqpoint{0.988630in}{4.047649in}}{\pgfqpoint{0.984239in}{4.037050in}}{\pgfqpoint{0.984239in}{4.026000in}}%
\pgfpathcurveto{\pgfqpoint{0.984239in}{4.014950in}}{\pgfqpoint{0.988630in}{4.004351in}}{\pgfqpoint{0.996443in}{3.996537in}}%
\pgfpathcurveto{\pgfqpoint{1.004257in}{3.988724in}}{\pgfqpoint{1.014856in}{3.984333in}}{\pgfqpoint{1.025906in}{3.984333in}}%
\pgfpathclose%
\pgfusepath{stroke,fill}%
\end{pgfscope}%
\begin{pgfscope}%
\pgfpathrectangle{\pgfqpoint{0.800000in}{0.528000in}}{\pgfqpoint{4.960000in}{3.696000in}}%
\pgfusepath{clip}%
\pgfsetbuttcap%
\pgfsetroundjoin%
\definecolor{currentfill}{rgb}{0.000000,0.000000,0.000000}%
\pgfsetfillcolor{currentfill}%
\pgfsetlinewidth{1.003750pt}%
\definecolor{currentstroke}{rgb}{0.000000,0.000000,0.000000}%
\pgfsetstrokecolor{currentstroke}%
\pgfsetdash{}{0pt}%
\pgfpathmoveto{\pgfqpoint{1.025906in}{3.984333in}}%
\pgfpathcurveto{\pgfqpoint{1.036956in}{3.984333in}}{\pgfqpoint{1.047555in}{3.988724in}}{\pgfqpoint{1.055369in}{3.996537in}}%
\pgfpathcurveto{\pgfqpoint{1.063182in}{4.004351in}}{\pgfqpoint{1.067573in}{4.014950in}}{\pgfqpoint{1.067573in}{4.026000in}}%
\pgfpathcurveto{\pgfqpoint{1.067573in}{4.037050in}}{\pgfqpoint{1.063182in}{4.047649in}}{\pgfqpoint{1.055369in}{4.055463in}}%
\pgfpathcurveto{\pgfqpoint{1.047555in}{4.063276in}}{\pgfqpoint{1.036956in}{4.067667in}}{\pgfqpoint{1.025906in}{4.067667in}}%
\pgfpathcurveto{\pgfqpoint{1.014856in}{4.067667in}}{\pgfqpoint{1.004257in}{4.063276in}}{\pgfqpoint{0.996443in}{4.055463in}}%
\pgfpathcurveto{\pgfqpoint{0.988630in}{4.047649in}}{\pgfqpoint{0.984239in}{4.037050in}}{\pgfqpoint{0.984239in}{4.026000in}}%
\pgfpathcurveto{\pgfqpoint{0.984239in}{4.014950in}}{\pgfqpoint{0.988630in}{4.004351in}}{\pgfqpoint{0.996443in}{3.996537in}}%
\pgfpathcurveto{\pgfqpoint{1.004257in}{3.988724in}}{\pgfqpoint{1.014856in}{3.984333in}}{\pgfqpoint{1.025906in}{3.984333in}}%
\pgfpathclose%
\pgfusepath{stroke,fill}%
\end{pgfscope}%
\begin{pgfscope}%
\pgfpathrectangle{\pgfqpoint{0.800000in}{0.528000in}}{\pgfqpoint{4.960000in}{3.696000in}}%
\pgfusepath{clip}%
\pgfsetbuttcap%
\pgfsetroundjoin%
\definecolor{currentfill}{rgb}{0.000000,0.000000,0.000000}%
\pgfsetfillcolor{currentfill}%
\pgfsetlinewidth{1.003750pt}%
\definecolor{currentstroke}{rgb}{0.000000,0.000000,0.000000}%
\pgfsetstrokecolor{currentstroke}%
\pgfsetdash{}{0pt}%
\pgfpathmoveto{\pgfqpoint{1.025906in}{3.984333in}}%
\pgfpathcurveto{\pgfqpoint{1.036956in}{3.984333in}}{\pgfqpoint{1.047555in}{3.988724in}}{\pgfqpoint{1.055369in}{3.996537in}}%
\pgfpathcurveto{\pgfqpoint{1.063182in}{4.004351in}}{\pgfqpoint{1.067573in}{4.014950in}}{\pgfqpoint{1.067573in}{4.026000in}}%
\pgfpathcurveto{\pgfqpoint{1.067573in}{4.037050in}}{\pgfqpoint{1.063182in}{4.047649in}}{\pgfqpoint{1.055369in}{4.055463in}}%
\pgfpathcurveto{\pgfqpoint{1.047555in}{4.063276in}}{\pgfqpoint{1.036956in}{4.067667in}}{\pgfqpoint{1.025906in}{4.067667in}}%
\pgfpathcurveto{\pgfqpoint{1.014856in}{4.067667in}}{\pgfqpoint{1.004257in}{4.063276in}}{\pgfqpoint{0.996443in}{4.055463in}}%
\pgfpathcurveto{\pgfqpoint{0.988630in}{4.047649in}}{\pgfqpoint{0.984239in}{4.037050in}}{\pgfqpoint{0.984239in}{4.026000in}}%
\pgfpathcurveto{\pgfqpoint{0.984239in}{4.014950in}}{\pgfqpoint{0.988630in}{4.004351in}}{\pgfqpoint{0.996443in}{3.996537in}}%
\pgfpathcurveto{\pgfqpoint{1.004257in}{3.988724in}}{\pgfqpoint{1.014856in}{3.984333in}}{\pgfqpoint{1.025906in}{3.984333in}}%
\pgfpathclose%
\pgfusepath{stroke,fill}%
\end{pgfscope}%
\begin{pgfscope}%
\pgfpathrectangle{\pgfqpoint{0.800000in}{0.528000in}}{\pgfqpoint{4.960000in}{3.696000in}}%
\pgfusepath{clip}%
\pgfsetbuttcap%
\pgfsetroundjoin%
\definecolor{currentfill}{rgb}{0.000000,0.000000,0.000000}%
\pgfsetfillcolor{currentfill}%
\pgfsetlinewidth{1.003750pt}%
\definecolor{currentstroke}{rgb}{0.000000,0.000000,0.000000}%
\pgfsetstrokecolor{currentstroke}%
\pgfsetdash{}{0pt}%
\pgfpathmoveto{\pgfqpoint{1.025906in}{3.984333in}}%
\pgfpathcurveto{\pgfqpoint{1.036956in}{3.984333in}}{\pgfqpoint{1.047555in}{3.988724in}}{\pgfqpoint{1.055369in}{3.996537in}}%
\pgfpathcurveto{\pgfqpoint{1.063182in}{4.004351in}}{\pgfqpoint{1.067573in}{4.014950in}}{\pgfqpoint{1.067573in}{4.026000in}}%
\pgfpathcurveto{\pgfqpoint{1.067573in}{4.037050in}}{\pgfqpoint{1.063182in}{4.047649in}}{\pgfqpoint{1.055369in}{4.055463in}}%
\pgfpathcurveto{\pgfqpoint{1.047555in}{4.063276in}}{\pgfqpoint{1.036956in}{4.067667in}}{\pgfqpoint{1.025906in}{4.067667in}}%
\pgfpathcurveto{\pgfqpoint{1.014856in}{4.067667in}}{\pgfqpoint{1.004257in}{4.063276in}}{\pgfqpoint{0.996443in}{4.055463in}}%
\pgfpathcurveto{\pgfqpoint{0.988630in}{4.047649in}}{\pgfqpoint{0.984239in}{4.037050in}}{\pgfqpoint{0.984239in}{4.026000in}}%
\pgfpathcurveto{\pgfqpoint{0.984239in}{4.014950in}}{\pgfqpoint{0.988630in}{4.004351in}}{\pgfqpoint{0.996443in}{3.996537in}}%
\pgfpathcurveto{\pgfqpoint{1.004257in}{3.988724in}}{\pgfqpoint{1.014856in}{3.984333in}}{\pgfqpoint{1.025906in}{3.984333in}}%
\pgfpathclose%
\pgfusepath{stroke,fill}%
\end{pgfscope}%
\begin{pgfscope}%
\pgfpathrectangle{\pgfqpoint{0.800000in}{0.528000in}}{\pgfqpoint{4.960000in}{3.696000in}}%
\pgfusepath{clip}%
\pgfsetbuttcap%
\pgfsetroundjoin%
\definecolor{currentfill}{rgb}{0.000000,0.000000,0.000000}%
\pgfsetfillcolor{currentfill}%
\pgfsetlinewidth{1.003750pt}%
\definecolor{currentstroke}{rgb}{0.000000,0.000000,0.000000}%
\pgfsetstrokecolor{currentstroke}%
\pgfsetdash{}{0pt}%
\pgfpathmoveto{\pgfqpoint{1.025906in}{3.984333in}}%
\pgfpathcurveto{\pgfqpoint{1.036956in}{3.984333in}}{\pgfqpoint{1.047555in}{3.988724in}}{\pgfqpoint{1.055369in}{3.996537in}}%
\pgfpathcurveto{\pgfqpoint{1.063182in}{4.004351in}}{\pgfqpoint{1.067573in}{4.014950in}}{\pgfqpoint{1.067573in}{4.026000in}}%
\pgfpathcurveto{\pgfqpoint{1.067573in}{4.037050in}}{\pgfqpoint{1.063182in}{4.047649in}}{\pgfqpoint{1.055369in}{4.055463in}}%
\pgfpathcurveto{\pgfqpoint{1.047555in}{4.063276in}}{\pgfqpoint{1.036956in}{4.067667in}}{\pgfqpoint{1.025906in}{4.067667in}}%
\pgfpathcurveto{\pgfqpoint{1.014856in}{4.067667in}}{\pgfqpoint{1.004257in}{4.063276in}}{\pgfqpoint{0.996443in}{4.055463in}}%
\pgfpathcurveto{\pgfqpoint{0.988630in}{4.047649in}}{\pgfqpoint{0.984239in}{4.037050in}}{\pgfqpoint{0.984239in}{4.026000in}}%
\pgfpathcurveto{\pgfqpoint{0.984239in}{4.014950in}}{\pgfqpoint{0.988630in}{4.004351in}}{\pgfqpoint{0.996443in}{3.996537in}}%
\pgfpathcurveto{\pgfqpoint{1.004257in}{3.988724in}}{\pgfqpoint{1.014856in}{3.984333in}}{\pgfqpoint{1.025906in}{3.984333in}}%
\pgfpathclose%
\pgfusepath{stroke,fill}%
\end{pgfscope}%
\begin{pgfscope}%
\pgfpathrectangle{\pgfqpoint{0.800000in}{0.528000in}}{\pgfqpoint{4.960000in}{3.696000in}}%
\pgfusepath{clip}%
\pgfsetbuttcap%
\pgfsetroundjoin%
\definecolor{currentfill}{rgb}{0.000000,0.000000,0.000000}%
\pgfsetfillcolor{currentfill}%
\pgfsetlinewidth{1.003750pt}%
\definecolor{currentstroke}{rgb}{0.000000,0.000000,0.000000}%
\pgfsetstrokecolor{currentstroke}%
\pgfsetdash{}{0pt}%
\pgfpathmoveto{\pgfqpoint{1.025906in}{3.984333in}}%
\pgfpathcurveto{\pgfqpoint{1.036956in}{3.984333in}}{\pgfqpoint{1.047555in}{3.988724in}}{\pgfqpoint{1.055369in}{3.996537in}}%
\pgfpathcurveto{\pgfqpoint{1.063182in}{4.004351in}}{\pgfqpoint{1.067573in}{4.014950in}}{\pgfqpoint{1.067573in}{4.026000in}}%
\pgfpathcurveto{\pgfqpoint{1.067573in}{4.037050in}}{\pgfqpoint{1.063182in}{4.047649in}}{\pgfqpoint{1.055369in}{4.055463in}}%
\pgfpathcurveto{\pgfqpoint{1.047555in}{4.063276in}}{\pgfqpoint{1.036956in}{4.067667in}}{\pgfqpoint{1.025906in}{4.067667in}}%
\pgfpathcurveto{\pgfqpoint{1.014856in}{4.067667in}}{\pgfqpoint{1.004257in}{4.063276in}}{\pgfqpoint{0.996443in}{4.055463in}}%
\pgfpathcurveto{\pgfqpoint{0.988630in}{4.047649in}}{\pgfqpoint{0.984239in}{4.037050in}}{\pgfqpoint{0.984239in}{4.026000in}}%
\pgfpathcurveto{\pgfqpoint{0.984239in}{4.014950in}}{\pgfqpoint{0.988630in}{4.004351in}}{\pgfqpoint{0.996443in}{3.996537in}}%
\pgfpathcurveto{\pgfqpoint{1.004257in}{3.988724in}}{\pgfqpoint{1.014856in}{3.984333in}}{\pgfqpoint{1.025906in}{3.984333in}}%
\pgfpathclose%
\pgfusepath{stroke,fill}%
\end{pgfscope}%
\begin{pgfscope}%
\pgfpathrectangle{\pgfqpoint{0.800000in}{0.528000in}}{\pgfqpoint{4.960000in}{3.696000in}}%
\pgfusepath{clip}%
\pgfsetbuttcap%
\pgfsetroundjoin%
\definecolor{currentfill}{rgb}{0.000000,0.000000,0.000000}%
\pgfsetfillcolor{currentfill}%
\pgfsetlinewidth{1.003750pt}%
\definecolor{currentstroke}{rgb}{0.000000,0.000000,0.000000}%
\pgfsetstrokecolor{currentstroke}%
\pgfsetdash{}{0pt}%
\pgfpathmoveto{\pgfqpoint{1.025906in}{3.984333in}}%
\pgfpathcurveto{\pgfqpoint{1.036956in}{3.984333in}}{\pgfqpoint{1.047555in}{3.988724in}}{\pgfqpoint{1.055369in}{3.996537in}}%
\pgfpathcurveto{\pgfqpoint{1.063182in}{4.004351in}}{\pgfqpoint{1.067573in}{4.014950in}}{\pgfqpoint{1.067573in}{4.026000in}}%
\pgfpathcurveto{\pgfqpoint{1.067573in}{4.037050in}}{\pgfqpoint{1.063182in}{4.047649in}}{\pgfqpoint{1.055369in}{4.055463in}}%
\pgfpathcurveto{\pgfqpoint{1.047555in}{4.063276in}}{\pgfqpoint{1.036956in}{4.067667in}}{\pgfqpoint{1.025906in}{4.067667in}}%
\pgfpathcurveto{\pgfqpoint{1.014856in}{4.067667in}}{\pgfqpoint{1.004257in}{4.063276in}}{\pgfqpoint{0.996443in}{4.055463in}}%
\pgfpathcurveto{\pgfqpoint{0.988630in}{4.047649in}}{\pgfqpoint{0.984239in}{4.037050in}}{\pgfqpoint{0.984239in}{4.026000in}}%
\pgfpathcurveto{\pgfqpoint{0.984239in}{4.014950in}}{\pgfqpoint{0.988630in}{4.004351in}}{\pgfqpoint{0.996443in}{3.996537in}}%
\pgfpathcurveto{\pgfqpoint{1.004257in}{3.988724in}}{\pgfqpoint{1.014856in}{3.984333in}}{\pgfqpoint{1.025906in}{3.984333in}}%
\pgfpathclose%
\pgfusepath{stroke,fill}%
\end{pgfscope}%
\begin{pgfscope}%
\pgfpathrectangle{\pgfqpoint{0.800000in}{0.528000in}}{\pgfqpoint{4.960000in}{3.696000in}}%
\pgfusepath{clip}%
\pgfsetbuttcap%
\pgfsetroundjoin%
\definecolor{currentfill}{rgb}{0.000000,0.000000,0.000000}%
\pgfsetfillcolor{currentfill}%
\pgfsetlinewidth{1.003750pt}%
\definecolor{currentstroke}{rgb}{0.000000,0.000000,0.000000}%
\pgfsetstrokecolor{currentstroke}%
\pgfsetdash{}{0pt}%
\pgfpathmoveto{\pgfqpoint{1.025906in}{3.984333in}}%
\pgfpathcurveto{\pgfqpoint{1.036956in}{3.984333in}}{\pgfqpoint{1.047555in}{3.988724in}}{\pgfqpoint{1.055369in}{3.996537in}}%
\pgfpathcurveto{\pgfqpoint{1.063182in}{4.004351in}}{\pgfqpoint{1.067573in}{4.014950in}}{\pgfqpoint{1.067573in}{4.026000in}}%
\pgfpathcurveto{\pgfqpoint{1.067573in}{4.037050in}}{\pgfqpoint{1.063182in}{4.047649in}}{\pgfqpoint{1.055369in}{4.055463in}}%
\pgfpathcurveto{\pgfqpoint{1.047555in}{4.063276in}}{\pgfqpoint{1.036956in}{4.067667in}}{\pgfqpoint{1.025906in}{4.067667in}}%
\pgfpathcurveto{\pgfqpoint{1.014856in}{4.067667in}}{\pgfqpoint{1.004257in}{4.063276in}}{\pgfqpoint{0.996443in}{4.055463in}}%
\pgfpathcurveto{\pgfqpoint{0.988630in}{4.047649in}}{\pgfqpoint{0.984239in}{4.037050in}}{\pgfqpoint{0.984239in}{4.026000in}}%
\pgfpathcurveto{\pgfqpoint{0.984239in}{4.014950in}}{\pgfqpoint{0.988630in}{4.004351in}}{\pgfqpoint{0.996443in}{3.996537in}}%
\pgfpathcurveto{\pgfqpoint{1.004257in}{3.988724in}}{\pgfqpoint{1.014856in}{3.984333in}}{\pgfqpoint{1.025906in}{3.984333in}}%
\pgfpathclose%
\pgfusepath{stroke,fill}%
\end{pgfscope}%
\begin{pgfscope}%
\pgfpathrectangle{\pgfqpoint{0.800000in}{0.528000in}}{\pgfqpoint{4.960000in}{3.696000in}}%
\pgfusepath{clip}%
\pgfsetbuttcap%
\pgfsetroundjoin%
\definecolor{currentfill}{rgb}{0.000000,0.000000,0.000000}%
\pgfsetfillcolor{currentfill}%
\pgfsetlinewidth{1.003750pt}%
\definecolor{currentstroke}{rgb}{0.000000,0.000000,0.000000}%
\pgfsetstrokecolor{currentstroke}%
\pgfsetdash{}{0pt}%
\pgfpathmoveto{\pgfqpoint{1.025906in}{3.984333in}}%
\pgfpathcurveto{\pgfqpoint{1.036956in}{3.984333in}}{\pgfqpoint{1.047555in}{3.988724in}}{\pgfqpoint{1.055369in}{3.996537in}}%
\pgfpathcurveto{\pgfqpoint{1.063182in}{4.004351in}}{\pgfqpoint{1.067573in}{4.014950in}}{\pgfqpoint{1.067573in}{4.026000in}}%
\pgfpathcurveto{\pgfqpoint{1.067573in}{4.037050in}}{\pgfqpoint{1.063182in}{4.047649in}}{\pgfqpoint{1.055369in}{4.055463in}}%
\pgfpathcurveto{\pgfqpoint{1.047555in}{4.063276in}}{\pgfqpoint{1.036956in}{4.067667in}}{\pgfqpoint{1.025906in}{4.067667in}}%
\pgfpathcurveto{\pgfqpoint{1.014856in}{4.067667in}}{\pgfqpoint{1.004257in}{4.063276in}}{\pgfqpoint{0.996443in}{4.055463in}}%
\pgfpathcurveto{\pgfqpoint{0.988630in}{4.047649in}}{\pgfqpoint{0.984239in}{4.037050in}}{\pgfqpoint{0.984239in}{4.026000in}}%
\pgfpathcurveto{\pgfqpoint{0.984239in}{4.014950in}}{\pgfqpoint{0.988630in}{4.004351in}}{\pgfqpoint{0.996443in}{3.996537in}}%
\pgfpathcurveto{\pgfqpoint{1.004257in}{3.988724in}}{\pgfqpoint{1.014856in}{3.984333in}}{\pgfqpoint{1.025906in}{3.984333in}}%
\pgfpathclose%
\pgfusepath{stroke,fill}%
\end{pgfscope}%
\begin{pgfscope}%
\pgfpathrectangle{\pgfqpoint{0.800000in}{0.528000in}}{\pgfqpoint{4.960000in}{3.696000in}}%
\pgfusepath{clip}%
\pgfsetbuttcap%
\pgfsetroundjoin%
\definecolor{currentfill}{rgb}{0.000000,0.000000,0.000000}%
\pgfsetfillcolor{currentfill}%
\pgfsetlinewidth{1.003750pt}%
\definecolor{currentstroke}{rgb}{0.000000,0.000000,0.000000}%
\pgfsetstrokecolor{currentstroke}%
\pgfsetdash{}{0pt}%
\pgfpathmoveto{\pgfqpoint{1.025906in}{3.984333in}}%
\pgfpathcurveto{\pgfqpoint{1.036956in}{3.984333in}}{\pgfqpoint{1.047555in}{3.988724in}}{\pgfqpoint{1.055369in}{3.996537in}}%
\pgfpathcurveto{\pgfqpoint{1.063182in}{4.004351in}}{\pgfqpoint{1.067573in}{4.014950in}}{\pgfqpoint{1.067573in}{4.026000in}}%
\pgfpathcurveto{\pgfqpoint{1.067573in}{4.037050in}}{\pgfqpoint{1.063182in}{4.047649in}}{\pgfqpoint{1.055369in}{4.055463in}}%
\pgfpathcurveto{\pgfqpoint{1.047555in}{4.063276in}}{\pgfqpoint{1.036956in}{4.067667in}}{\pgfqpoint{1.025906in}{4.067667in}}%
\pgfpathcurveto{\pgfqpoint{1.014856in}{4.067667in}}{\pgfqpoint{1.004257in}{4.063276in}}{\pgfqpoint{0.996443in}{4.055463in}}%
\pgfpathcurveto{\pgfqpoint{0.988630in}{4.047649in}}{\pgfqpoint{0.984239in}{4.037050in}}{\pgfqpoint{0.984239in}{4.026000in}}%
\pgfpathcurveto{\pgfqpoint{0.984239in}{4.014950in}}{\pgfqpoint{0.988630in}{4.004351in}}{\pgfqpoint{0.996443in}{3.996537in}}%
\pgfpathcurveto{\pgfqpoint{1.004257in}{3.988724in}}{\pgfqpoint{1.014856in}{3.984333in}}{\pgfqpoint{1.025906in}{3.984333in}}%
\pgfpathclose%
\pgfusepath{stroke,fill}%
\end{pgfscope}%
\begin{pgfscope}%
\pgfpathrectangle{\pgfqpoint{0.800000in}{0.528000in}}{\pgfqpoint{4.960000in}{3.696000in}}%
\pgfusepath{clip}%
\pgfsetbuttcap%
\pgfsetroundjoin%
\definecolor{currentfill}{rgb}{0.000000,0.000000,0.000000}%
\pgfsetfillcolor{currentfill}%
\pgfsetlinewidth{1.003750pt}%
\definecolor{currentstroke}{rgb}{0.000000,0.000000,0.000000}%
\pgfsetstrokecolor{currentstroke}%
\pgfsetdash{}{0pt}%
\pgfpathmoveto{\pgfqpoint{1.025906in}{3.984333in}}%
\pgfpathcurveto{\pgfqpoint{1.036956in}{3.984333in}}{\pgfqpoint{1.047555in}{3.988724in}}{\pgfqpoint{1.055369in}{3.996537in}}%
\pgfpathcurveto{\pgfqpoint{1.063182in}{4.004351in}}{\pgfqpoint{1.067573in}{4.014950in}}{\pgfqpoint{1.067573in}{4.026000in}}%
\pgfpathcurveto{\pgfqpoint{1.067573in}{4.037050in}}{\pgfqpoint{1.063182in}{4.047649in}}{\pgfqpoint{1.055369in}{4.055463in}}%
\pgfpathcurveto{\pgfqpoint{1.047555in}{4.063276in}}{\pgfqpoint{1.036956in}{4.067667in}}{\pgfqpoint{1.025906in}{4.067667in}}%
\pgfpathcurveto{\pgfqpoint{1.014856in}{4.067667in}}{\pgfqpoint{1.004257in}{4.063276in}}{\pgfqpoint{0.996443in}{4.055463in}}%
\pgfpathcurveto{\pgfqpoint{0.988630in}{4.047649in}}{\pgfqpoint{0.984239in}{4.037050in}}{\pgfqpoint{0.984239in}{4.026000in}}%
\pgfpathcurveto{\pgfqpoint{0.984239in}{4.014950in}}{\pgfqpoint{0.988630in}{4.004351in}}{\pgfqpoint{0.996443in}{3.996537in}}%
\pgfpathcurveto{\pgfqpoint{1.004257in}{3.988724in}}{\pgfqpoint{1.014856in}{3.984333in}}{\pgfqpoint{1.025906in}{3.984333in}}%
\pgfpathclose%
\pgfusepath{stroke,fill}%
\end{pgfscope}%
\begin{pgfscope}%
\pgfpathrectangle{\pgfqpoint{0.800000in}{0.528000in}}{\pgfqpoint{4.960000in}{3.696000in}}%
\pgfusepath{clip}%
\pgfsetbuttcap%
\pgfsetroundjoin%
\definecolor{currentfill}{rgb}{0.000000,0.000000,0.000000}%
\pgfsetfillcolor{currentfill}%
\pgfsetlinewidth{1.003750pt}%
\definecolor{currentstroke}{rgb}{0.000000,0.000000,0.000000}%
\pgfsetstrokecolor{currentstroke}%
\pgfsetdash{}{0pt}%
\pgfpathmoveto{\pgfqpoint{1.025906in}{3.984333in}}%
\pgfpathcurveto{\pgfqpoint{1.036956in}{3.984333in}}{\pgfqpoint{1.047555in}{3.988724in}}{\pgfqpoint{1.055369in}{3.996537in}}%
\pgfpathcurveto{\pgfqpoint{1.063182in}{4.004351in}}{\pgfqpoint{1.067573in}{4.014950in}}{\pgfqpoint{1.067573in}{4.026000in}}%
\pgfpathcurveto{\pgfqpoint{1.067573in}{4.037050in}}{\pgfqpoint{1.063182in}{4.047649in}}{\pgfqpoint{1.055369in}{4.055463in}}%
\pgfpathcurveto{\pgfqpoint{1.047555in}{4.063276in}}{\pgfqpoint{1.036956in}{4.067667in}}{\pgfqpoint{1.025906in}{4.067667in}}%
\pgfpathcurveto{\pgfqpoint{1.014856in}{4.067667in}}{\pgfqpoint{1.004257in}{4.063276in}}{\pgfqpoint{0.996443in}{4.055463in}}%
\pgfpathcurveto{\pgfqpoint{0.988630in}{4.047649in}}{\pgfqpoint{0.984239in}{4.037050in}}{\pgfqpoint{0.984239in}{4.026000in}}%
\pgfpathcurveto{\pgfqpoint{0.984239in}{4.014950in}}{\pgfqpoint{0.988630in}{4.004351in}}{\pgfqpoint{0.996443in}{3.996537in}}%
\pgfpathcurveto{\pgfqpoint{1.004257in}{3.988724in}}{\pgfqpoint{1.014856in}{3.984333in}}{\pgfqpoint{1.025906in}{3.984333in}}%
\pgfpathclose%
\pgfusepath{stroke,fill}%
\end{pgfscope}%
\begin{pgfscope}%
\pgfpathrectangle{\pgfqpoint{0.800000in}{0.528000in}}{\pgfqpoint{4.960000in}{3.696000in}}%
\pgfusepath{clip}%
\pgfsetbuttcap%
\pgfsetroundjoin%
\definecolor{currentfill}{rgb}{0.000000,0.000000,0.000000}%
\pgfsetfillcolor{currentfill}%
\pgfsetlinewidth{1.003750pt}%
\definecolor{currentstroke}{rgb}{0.000000,0.000000,0.000000}%
\pgfsetstrokecolor{currentstroke}%
\pgfsetdash{}{0pt}%
\pgfpathmoveto{\pgfqpoint{1.025906in}{3.984333in}}%
\pgfpathcurveto{\pgfqpoint{1.036956in}{3.984333in}}{\pgfqpoint{1.047555in}{3.988724in}}{\pgfqpoint{1.055369in}{3.996537in}}%
\pgfpathcurveto{\pgfqpoint{1.063182in}{4.004351in}}{\pgfqpoint{1.067573in}{4.014950in}}{\pgfqpoint{1.067573in}{4.026000in}}%
\pgfpathcurveto{\pgfqpoint{1.067573in}{4.037050in}}{\pgfqpoint{1.063182in}{4.047649in}}{\pgfqpoint{1.055369in}{4.055463in}}%
\pgfpathcurveto{\pgfqpoint{1.047555in}{4.063276in}}{\pgfqpoint{1.036956in}{4.067667in}}{\pgfqpoint{1.025906in}{4.067667in}}%
\pgfpathcurveto{\pgfqpoint{1.014856in}{4.067667in}}{\pgfqpoint{1.004257in}{4.063276in}}{\pgfqpoint{0.996443in}{4.055463in}}%
\pgfpathcurveto{\pgfqpoint{0.988630in}{4.047649in}}{\pgfqpoint{0.984239in}{4.037050in}}{\pgfqpoint{0.984239in}{4.026000in}}%
\pgfpathcurveto{\pgfqpoint{0.984239in}{4.014950in}}{\pgfqpoint{0.988630in}{4.004351in}}{\pgfqpoint{0.996443in}{3.996537in}}%
\pgfpathcurveto{\pgfqpoint{1.004257in}{3.988724in}}{\pgfqpoint{1.014856in}{3.984333in}}{\pgfqpoint{1.025906in}{3.984333in}}%
\pgfpathclose%
\pgfusepath{stroke,fill}%
\end{pgfscope}%
\begin{pgfscope}%
\pgfpathrectangle{\pgfqpoint{0.800000in}{0.528000in}}{\pgfqpoint{4.960000in}{3.696000in}}%
\pgfusepath{clip}%
\pgfsetbuttcap%
\pgfsetroundjoin%
\definecolor{currentfill}{rgb}{0.000000,0.000000,0.000000}%
\pgfsetfillcolor{currentfill}%
\pgfsetlinewidth{1.003750pt}%
\definecolor{currentstroke}{rgb}{0.000000,0.000000,0.000000}%
\pgfsetstrokecolor{currentstroke}%
\pgfsetdash{}{0pt}%
\pgfpathmoveto{\pgfqpoint{1.025906in}{3.984333in}}%
\pgfpathcurveto{\pgfqpoint{1.036956in}{3.984333in}}{\pgfqpoint{1.047555in}{3.988724in}}{\pgfqpoint{1.055369in}{3.996537in}}%
\pgfpathcurveto{\pgfqpoint{1.063182in}{4.004351in}}{\pgfqpoint{1.067573in}{4.014950in}}{\pgfqpoint{1.067573in}{4.026000in}}%
\pgfpathcurveto{\pgfqpoint{1.067573in}{4.037050in}}{\pgfqpoint{1.063182in}{4.047649in}}{\pgfqpoint{1.055369in}{4.055463in}}%
\pgfpathcurveto{\pgfqpoint{1.047555in}{4.063276in}}{\pgfqpoint{1.036956in}{4.067667in}}{\pgfqpoint{1.025906in}{4.067667in}}%
\pgfpathcurveto{\pgfqpoint{1.014856in}{4.067667in}}{\pgfqpoint{1.004257in}{4.063276in}}{\pgfqpoint{0.996443in}{4.055463in}}%
\pgfpathcurveto{\pgfqpoint{0.988630in}{4.047649in}}{\pgfqpoint{0.984239in}{4.037050in}}{\pgfqpoint{0.984239in}{4.026000in}}%
\pgfpathcurveto{\pgfqpoint{0.984239in}{4.014950in}}{\pgfqpoint{0.988630in}{4.004351in}}{\pgfqpoint{0.996443in}{3.996537in}}%
\pgfpathcurveto{\pgfqpoint{1.004257in}{3.988724in}}{\pgfqpoint{1.014856in}{3.984333in}}{\pgfqpoint{1.025906in}{3.984333in}}%
\pgfpathclose%
\pgfusepath{stroke,fill}%
\end{pgfscope}%
\begin{pgfscope}%
\pgfpathrectangle{\pgfqpoint{0.800000in}{0.528000in}}{\pgfqpoint{4.960000in}{3.696000in}}%
\pgfusepath{clip}%
\pgfsetbuttcap%
\pgfsetroundjoin%
\definecolor{currentfill}{rgb}{0.000000,0.000000,0.000000}%
\pgfsetfillcolor{currentfill}%
\pgfsetlinewidth{1.003750pt}%
\definecolor{currentstroke}{rgb}{0.000000,0.000000,0.000000}%
\pgfsetstrokecolor{currentstroke}%
\pgfsetdash{}{0pt}%
\pgfpathmoveto{\pgfqpoint{1.025906in}{3.984333in}}%
\pgfpathcurveto{\pgfqpoint{1.036956in}{3.984333in}}{\pgfqpoint{1.047555in}{3.988724in}}{\pgfqpoint{1.055369in}{3.996537in}}%
\pgfpathcurveto{\pgfqpoint{1.063182in}{4.004351in}}{\pgfqpoint{1.067573in}{4.014950in}}{\pgfqpoint{1.067573in}{4.026000in}}%
\pgfpathcurveto{\pgfqpoint{1.067573in}{4.037050in}}{\pgfqpoint{1.063182in}{4.047649in}}{\pgfqpoint{1.055369in}{4.055463in}}%
\pgfpathcurveto{\pgfqpoint{1.047555in}{4.063276in}}{\pgfqpoint{1.036956in}{4.067667in}}{\pgfqpoint{1.025906in}{4.067667in}}%
\pgfpathcurveto{\pgfqpoint{1.014856in}{4.067667in}}{\pgfqpoint{1.004257in}{4.063276in}}{\pgfqpoint{0.996443in}{4.055463in}}%
\pgfpathcurveto{\pgfqpoint{0.988630in}{4.047649in}}{\pgfqpoint{0.984239in}{4.037050in}}{\pgfqpoint{0.984239in}{4.026000in}}%
\pgfpathcurveto{\pgfqpoint{0.984239in}{4.014950in}}{\pgfqpoint{0.988630in}{4.004351in}}{\pgfqpoint{0.996443in}{3.996537in}}%
\pgfpathcurveto{\pgfqpoint{1.004257in}{3.988724in}}{\pgfqpoint{1.014856in}{3.984333in}}{\pgfqpoint{1.025906in}{3.984333in}}%
\pgfpathclose%
\pgfusepath{stroke,fill}%
\end{pgfscope}%
\begin{pgfscope}%
\pgfpathrectangle{\pgfqpoint{0.800000in}{0.528000in}}{\pgfqpoint{4.960000in}{3.696000in}}%
\pgfusepath{clip}%
\pgfsetbuttcap%
\pgfsetroundjoin%
\definecolor{currentfill}{rgb}{0.000000,0.000000,0.000000}%
\pgfsetfillcolor{currentfill}%
\pgfsetlinewidth{1.003750pt}%
\definecolor{currentstroke}{rgb}{0.000000,0.000000,0.000000}%
\pgfsetstrokecolor{currentstroke}%
\pgfsetdash{}{0pt}%
\pgfpathmoveto{\pgfqpoint{1.025906in}{3.984333in}}%
\pgfpathcurveto{\pgfqpoint{1.036956in}{3.984333in}}{\pgfqpoint{1.047555in}{3.988724in}}{\pgfqpoint{1.055369in}{3.996537in}}%
\pgfpathcurveto{\pgfqpoint{1.063182in}{4.004351in}}{\pgfqpoint{1.067573in}{4.014950in}}{\pgfqpoint{1.067573in}{4.026000in}}%
\pgfpathcurveto{\pgfqpoint{1.067573in}{4.037050in}}{\pgfqpoint{1.063182in}{4.047649in}}{\pgfqpoint{1.055369in}{4.055463in}}%
\pgfpathcurveto{\pgfqpoint{1.047555in}{4.063276in}}{\pgfqpoint{1.036956in}{4.067667in}}{\pgfqpoint{1.025906in}{4.067667in}}%
\pgfpathcurveto{\pgfqpoint{1.014856in}{4.067667in}}{\pgfqpoint{1.004257in}{4.063276in}}{\pgfqpoint{0.996443in}{4.055463in}}%
\pgfpathcurveto{\pgfqpoint{0.988630in}{4.047649in}}{\pgfqpoint{0.984239in}{4.037050in}}{\pgfqpoint{0.984239in}{4.026000in}}%
\pgfpathcurveto{\pgfqpoint{0.984239in}{4.014950in}}{\pgfqpoint{0.988630in}{4.004351in}}{\pgfqpoint{0.996443in}{3.996537in}}%
\pgfpathcurveto{\pgfqpoint{1.004257in}{3.988724in}}{\pgfqpoint{1.014856in}{3.984333in}}{\pgfqpoint{1.025906in}{3.984333in}}%
\pgfpathclose%
\pgfusepath{stroke,fill}%
\end{pgfscope}%
\begin{pgfscope}%
\pgfpathrectangle{\pgfqpoint{0.800000in}{0.528000in}}{\pgfqpoint{4.960000in}{3.696000in}}%
\pgfusepath{clip}%
\pgfsetbuttcap%
\pgfsetroundjoin%
\definecolor{currentfill}{rgb}{0.000000,0.000000,0.000000}%
\pgfsetfillcolor{currentfill}%
\pgfsetlinewidth{1.003750pt}%
\definecolor{currentstroke}{rgb}{0.000000,0.000000,0.000000}%
\pgfsetstrokecolor{currentstroke}%
\pgfsetdash{}{0pt}%
\pgfpathmoveto{\pgfqpoint{1.025906in}{3.984333in}}%
\pgfpathcurveto{\pgfqpoint{1.036956in}{3.984333in}}{\pgfqpoint{1.047555in}{3.988724in}}{\pgfqpoint{1.055369in}{3.996537in}}%
\pgfpathcurveto{\pgfqpoint{1.063182in}{4.004351in}}{\pgfqpoint{1.067573in}{4.014950in}}{\pgfqpoint{1.067573in}{4.026000in}}%
\pgfpathcurveto{\pgfqpoint{1.067573in}{4.037050in}}{\pgfqpoint{1.063182in}{4.047649in}}{\pgfqpoint{1.055369in}{4.055463in}}%
\pgfpathcurveto{\pgfqpoint{1.047555in}{4.063276in}}{\pgfqpoint{1.036956in}{4.067667in}}{\pgfqpoint{1.025906in}{4.067667in}}%
\pgfpathcurveto{\pgfqpoint{1.014856in}{4.067667in}}{\pgfqpoint{1.004257in}{4.063276in}}{\pgfqpoint{0.996443in}{4.055463in}}%
\pgfpathcurveto{\pgfqpoint{0.988630in}{4.047649in}}{\pgfqpoint{0.984239in}{4.037050in}}{\pgfqpoint{0.984239in}{4.026000in}}%
\pgfpathcurveto{\pgfqpoint{0.984239in}{4.014950in}}{\pgfqpoint{0.988630in}{4.004351in}}{\pgfqpoint{0.996443in}{3.996537in}}%
\pgfpathcurveto{\pgfqpoint{1.004257in}{3.988724in}}{\pgfqpoint{1.014856in}{3.984333in}}{\pgfqpoint{1.025906in}{3.984333in}}%
\pgfpathclose%
\pgfusepath{stroke,fill}%
\end{pgfscope}%
\begin{pgfscope}%
\pgfpathrectangle{\pgfqpoint{0.800000in}{0.528000in}}{\pgfqpoint{4.960000in}{3.696000in}}%
\pgfusepath{clip}%
\pgfsetbuttcap%
\pgfsetroundjoin%
\definecolor{currentfill}{rgb}{0.000000,0.000000,0.000000}%
\pgfsetfillcolor{currentfill}%
\pgfsetlinewidth{1.003750pt}%
\definecolor{currentstroke}{rgb}{0.000000,0.000000,0.000000}%
\pgfsetstrokecolor{currentstroke}%
\pgfsetdash{}{0pt}%
\pgfpathmoveto{\pgfqpoint{1.025906in}{3.984333in}}%
\pgfpathcurveto{\pgfqpoint{1.036956in}{3.984333in}}{\pgfqpoint{1.047555in}{3.988724in}}{\pgfqpoint{1.055369in}{3.996537in}}%
\pgfpathcurveto{\pgfqpoint{1.063182in}{4.004351in}}{\pgfqpoint{1.067573in}{4.014950in}}{\pgfqpoint{1.067573in}{4.026000in}}%
\pgfpathcurveto{\pgfqpoint{1.067573in}{4.037050in}}{\pgfqpoint{1.063182in}{4.047649in}}{\pgfqpoint{1.055369in}{4.055463in}}%
\pgfpathcurveto{\pgfqpoint{1.047555in}{4.063276in}}{\pgfqpoint{1.036956in}{4.067667in}}{\pgfqpoint{1.025906in}{4.067667in}}%
\pgfpathcurveto{\pgfqpoint{1.014856in}{4.067667in}}{\pgfqpoint{1.004257in}{4.063276in}}{\pgfqpoint{0.996443in}{4.055463in}}%
\pgfpathcurveto{\pgfqpoint{0.988630in}{4.047649in}}{\pgfqpoint{0.984239in}{4.037050in}}{\pgfqpoint{0.984239in}{4.026000in}}%
\pgfpathcurveto{\pgfqpoint{0.984239in}{4.014950in}}{\pgfqpoint{0.988630in}{4.004351in}}{\pgfqpoint{0.996443in}{3.996537in}}%
\pgfpathcurveto{\pgfqpoint{1.004257in}{3.988724in}}{\pgfqpoint{1.014856in}{3.984333in}}{\pgfqpoint{1.025906in}{3.984333in}}%
\pgfpathclose%
\pgfusepath{stroke,fill}%
\end{pgfscope}%
\begin{pgfscope}%
\pgfpathrectangle{\pgfqpoint{0.800000in}{0.528000in}}{\pgfqpoint{4.960000in}{3.696000in}}%
\pgfusepath{clip}%
\pgfsetbuttcap%
\pgfsetroundjoin%
\definecolor{currentfill}{rgb}{0.000000,0.000000,0.000000}%
\pgfsetfillcolor{currentfill}%
\pgfsetlinewidth{1.003750pt}%
\definecolor{currentstroke}{rgb}{0.000000,0.000000,0.000000}%
\pgfsetstrokecolor{currentstroke}%
\pgfsetdash{}{0pt}%
\pgfpathmoveto{\pgfqpoint{1.025906in}{3.984333in}}%
\pgfpathcurveto{\pgfqpoint{1.036956in}{3.984333in}}{\pgfqpoint{1.047555in}{3.988724in}}{\pgfqpoint{1.055369in}{3.996537in}}%
\pgfpathcurveto{\pgfqpoint{1.063182in}{4.004351in}}{\pgfqpoint{1.067573in}{4.014950in}}{\pgfqpoint{1.067573in}{4.026000in}}%
\pgfpathcurveto{\pgfqpoint{1.067573in}{4.037050in}}{\pgfqpoint{1.063182in}{4.047649in}}{\pgfqpoint{1.055369in}{4.055463in}}%
\pgfpathcurveto{\pgfqpoint{1.047555in}{4.063276in}}{\pgfqpoint{1.036956in}{4.067667in}}{\pgfqpoint{1.025906in}{4.067667in}}%
\pgfpathcurveto{\pgfqpoint{1.014856in}{4.067667in}}{\pgfqpoint{1.004257in}{4.063276in}}{\pgfqpoint{0.996443in}{4.055463in}}%
\pgfpathcurveto{\pgfqpoint{0.988630in}{4.047649in}}{\pgfqpoint{0.984239in}{4.037050in}}{\pgfqpoint{0.984239in}{4.026000in}}%
\pgfpathcurveto{\pgfqpoint{0.984239in}{4.014950in}}{\pgfqpoint{0.988630in}{4.004351in}}{\pgfqpoint{0.996443in}{3.996537in}}%
\pgfpathcurveto{\pgfqpoint{1.004257in}{3.988724in}}{\pgfqpoint{1.014856in}{3.984333in}}{\pgfqpoint{1.025906in}{3.984333in}}%
\pgfpathclose%
\pgfusepath{stroke,fill}%
\end{pgfscope}%
\begin{pgfscope}%
\pgfpathrectangle{\pgfqpoint{0.800000in}{0.528000in}}{\pgfqpoint{4.960000in}{3.696000in}}%
\pgfusepath{clip}%
\pgfsetbuttcap%
\pgfsetroundjoin%
\definecolor{currentfill}{rgb}{0.000000,0.000000,0.000000}%
\pgfsetfillcolor{currentfill}%
\pgfsetlinewidth{1.003750pt}%
\definecolor{currentstroke}{rgb}{0.000000,0.000000,0.000000}%
\pgfsetstrokecolor{currentstroke}%
\pgfsetdash{}{0pt}%
\pgfpathmoveto{\pgfqpoint{1.025906in}{3.984333in}}%
\pgfpathcurveto{\pgfqpoint{1.036956in}{3.984333in}}{\pgfqpoint{1.047555in}{3.988724in}}{\pgfqpoint{1.055369in}{3.996537in}}%
\pgfpathcurveto{\pgfqpoint{1.063182in}{4.004351in}}{\pgfqpoint{1.067573in}{4.014950in}}{\pgfqpoint{1.067573in}{4.026000in}}%
\pgfpathcurveto{\pgfqpoint{1.067573in}{4.037050in}}{\pgfqpoint{1.063182in}{4.047649in}}{\pgfqpoint{1.055369in}{4.055463in}}%
\pgfpathcurveto{\pgfqpoint{1.047555in}{4.063276in}}{\pgfqpoint{1.036956in}{4.067667in}}{\pgfqpoint{1.025906in}{4.067667in}}%
\pgfpathcurveto{\pgfqpoint{1.014856in}{4.067667in}}{\pgfqpoint{1.004257in}{4.063276in}}{\pgfqpoint{0.996443in}{4.055463in}}%
\pgfpathcurveto{\pgfqpoint{0.988630in}{4.047649in}}{\pgfqpoint{0.984239in}{4.037050in}}{\pgfqpoint{0.984239in}{4.026000in}}%
\pgfpathcurveto{\pgfqpoint{0.984239in}{4.014950in}}{\pgfqpoint{0.988630in}{4.004351in}}{\pgfqpoint{0.996443in}{3.996537in}}%
\pgfpathcurveto{\pgfqpoint{1.004257in}{3.988724in}}{\pgfqpoint{1.014856in}{3.984333in}}{\pgfqpoint{1.025906in}{3.984333in}}%
\pgfpathclose%
\pgfusepath{stroke,fill}%
\end{pgfscope}%
\begin{pgfscope}%
\pgfpathrectangle{\pgfqpoint{0.800000in}{0.528000in}}{\pgfqpoint{4.960000in}{3.696000in}}%
\pgfusepath{clip}%
\pgfsetbuttcap%
\pgfsetroundjoin%
\definecolor{currentfill}{rgb}{0.000000,0.000000,0.000000}%
\pgfsetfillcolor{currentfill}%
\pgfsetlinewidth{1.003750pt}%
\definecolor{currentstroke}{rgb}{0.000000,0.000000,0.000000}%
\pgfsetstrokecolor{currentstroke}%
\pgfsetdash{}{0pt}%
\pgfpathmoveto{\pgfqpoint{1.025906in}{3.984333in}}%
\pgfpathcurveto{\pgfqpoint{1.036956in}{3.984333in}}{\pgfqpoint{1.047555in}{3.988724in}}{\pgfqpoint{1.055369in}{3.996537in}}%
\pgfpathcurveto{\pgfqpoint{1.063182in}{4.004351in}}{\pgfqpoint{1.067573in}{4.014950in}}{\pgfqpoint{1.067573in}{4.026000in}}%
\pgfpathcurveto{\pgfqpoint{1.067573in}{4.037050in}}{\pgfqpoint{1.063182in}{4.047649in}}{\pgfqpoint{1.055369in}{4.055463in}}%
\pgfpathcurveto{\pgfqpoint{1.047555in}{4.063276in}}{\pgfqpoint{1.036956in}{4.067667in}}{\pgfqpoint{1.025906in}{4.067667in}}%
\pgfpathcurveto{\pgfqpoint{1.014856in}{4.067667in}}{\pgfqpoint{1.004257in}{4.063276in}}{\pgfqpoint{0.996443in}{4.055463in}}%
\pgfpathcurveto{\pgfqpoint{0.988630in}{4.047649in}}{\pgfqpoint{0.984239in}{4.037050in}}{\pgfqpoint{0.984239in}{4.026000in}}%
\pgfpathcurveto{\pgfqpoint{0.984239in}{4.014950in}}{\pgfqpoint{0.988630in}{4.004351in}}{\pgfqpoint{0.996443in}{3.996537in}}%
\pgfpathcurveto{\pgfqpoint{1.004257in}{3.988724in}}{\pgfqpoint{1.014856in}{3.984333in}}{\pgfqpoint{1.025906in}{3.984333in}}%
\pgfpathclose%
\pgfusepath{stroke,fill}%
\end{pgfscope}%
\begin{pgfscope}%
\pgfpathrectangle{\pgfqpoint{0.800000in}{0.528000in}}{\pgfqpoint{4.960000in}{3.696000in}}%
\pgfusepath{clip}%
\pgfsetbuttcap%
\pgfsetroundjoin%
\definecolor{currentfill}{rgb}{0.000000,0.000000,0.000000}%
\pgfsetfillcolor{currentfill}%
\pgfsetlinewidth{1.003750pt}%
\definecolor{currentstroke}{rgb}{0.000000,0.000000,0.000000}%
\pgfsetstrokecolor{currentstroke}%
\pgfsetdash{}{0pt}%
\pgfpathmoveto{\pgfqpoint{1.025906in}{3.984333in}}%
\pgfpathcurveto{\pgfqpoint{1.036956in}{3.984333in}}{\pgfqpoint{1.047555in}{3.988724in}}{\pgfqpoint{1.055369in}{3.996537in}}%
\pgfpathcurveto{\pgfqpoint{1.063182in}{4.004351in}}{\pgfqpoint{1.067573in}{4.014950in}}{\pgfqpoint{1.067573in}{4.026000in}}%
\pgfpathcurveto{\pgfqpoint{1.067573in}{4.037050in}}{\pgfqpoint{1.063182in}{4.047649in}}{\pgfqpoint{1.055369in}{4.055463in}}%
\pgfpathcurveto{\pgfqpoint{1.047555in}{4.063276in}}{\pgfqpoint{1.036956in}{4.067667in}}{\pgfqpoint{1.025906in}{4.067667in}}%
\pgfpathcurveto{\pgfqpoint{1.014856in}{4.067667in}}{\pgfqpoint{1.004257in}{4.063276in}}{\pgfqpoint{0.996443in}{4.055463in}}%
\pgfpathcurveto{\pgfqpoint{0.988630in}{4.047649in}}{\pgfqpoint{0.984239in}{4.037050in}}{\pgfqpoint{0.984239in}{4.026000in}}%
\pgfpathcurveto{\pgfqpoint{0.984239in}{4.014950in}}{\pgfqpoint{0.988630in}{4.004351in}}{\pgfqpoint{0.996443in}{3.996537in}}%
\pgfpathcurveto{\pgfqpoint{1.004257in}{3.988724in}}{\pgfqpoint{1.014856in}{3.984333in}}{\pgfqpoint{1.025906in}{3.984333in}}%
\pgfpathclose%
\pgfusepath{stroke,fill}%
\end{pgfscope}%
\begin{pgfscope}%
\pgfpathrectangle{\pgfqpoint{0.800000in}{0.528000in}}{\pgfqpoint{4.960000in}{3.696000in}}%
\pgfusepath{clip}%
\pgfsetbuttcap%
\pgfsetroundjoin%
\definecolor{currentfill}{rgb}{0.000000,0.000000,0.000000}%
\pgfsetfillcolor{currentfill}%
\pgfsetlinewidth{1.003750pt}%
\definecolor{currentstroke}{rgb}{0.000000,0.000000,0.000000}%
\pgfsetstrokecolor{currentstroke}%
\pgfsetdash{}{0pt}%
\pgfpathmoveto{\pgfqpoint{1.025906in}{3.984333in}}%
\pgfpathcurveto{\pgfqpoint{1.036956in}{3.984333in}}{\pgfqpoint{1.047555in}{3.988724in}}{\pgfqpoint{1.055369in}{3.996537in}}%
\pgfpathcurveto{\pgfqpoint{1.063182in}{4.004351in}}{\pgfqpoint{1.067573in}{4.014950in}}{\pgfqpoint{1.067573in}{4.026000in}}%
\pgfpathcurveto{\pgfqpoint{1.067573in}{4.037050in}}{\pgfqpoint{1.063182in}{4.047649in}}{\pgfqpoint{1.055369in}{4.055463in}}%
\pgfpathcurveto{\pgfqpoint{1.047555in}{4.063276in}}{\pgfqpoint{1.036956in}{4.067667in}}{\pgfqpoint{1.025906in}{4.067667in}}%
\pgfpathcurveto{\pgfqpoint{1.014856in}{4.067667in}}{\pgfqpoint{1.004257in}{4.063276in}}{\pgfqpoint{0.996443in}{4.055463in}}%
\pgfpathcurveto{\pgfqpoint{0.988630in}{4.047649in}}{\pgfqpoint{0.984239in}{4.037050in}}{\pgfqpoint{0.984239in}{4.026000in}}%
\pgfpathcurveto{\pgfqpoint{0.984239in}{4.014950in}}{\pgfqpoint{0.988630in}{4.004351in}}{\pgfqpoint{0.996443in}{3.996537in}}%
\pgfpathcurveto{\pgfqpoint{1.004257in}{3.988724in}}{\pgfqpoint{1.014856in}{3.984333in}}{\pgfqpoint{1.025906in}{3.984333in}}%
\pgfpathclose%
\pgfusepath{stroke,fill}%
\end{pgfscope}%
\begin{pgfscope}%
\pgfpathrectangle{\pgfqpoint{0.800000in}{0.528000in}}{\pgfqpoint{4.960000in}{3.696000in}}%
\pgfusepath{clip}%
\pgfsetbuttcap%
\pgfsetroundjoin%
\definecolor{currentfill}{rgb}{0.000000,0.000000,0.000000}%
\pgfsetfillcolor{currentfill}%
\pgfsetlinewidth{1.003750pt}%
\definecolor{currentstroke}{rgb}{0.000000,0.000000,0.000000}%
\pgfsetstrokecolor{currentstroke}%
\pgfsetdash{}{0pt}%
\pgfpathmoveto{\pgfqpoint{1.025906in}{3.984333in}}%
\pgfpathcurveto{\pgfqpoint{1.036956in}{3.984333in}}{\pgfqpoint{1.047555in}{3.988724in}}{\pgfqpoint{1.055369in}{3.996537in}}%
\pgfpathcurveto{\pgfqpoint{1.063182in}{4.004351in}}{\pgfqpoint{1.067573in}{4.014950in}}{\pgfqpoint{1.067573in}{4.026000in}}%
\pgfpathcurveto{\pgfqpoint{1.067573in}{4.037050in}}{\pgfqpoint{1.063182in}{4.047649in}}{\pgfqpoint{1.055369in}{4.055463in}}%
\pgfpathcurveto{\pgfqpoint{1.047555in}{4.063276in}}{\pgfqpoint{1.036956in}{4.067667in}}{\pgfqpoint{1.025906in}{4.067667in}}%
\pgfpathcurveto{\pgfqpoint{1.014856in}{4.067667in}}{\pgfqpoint{1.004257in}{4.063276in}}{\pgfqpoint{0.996443in}{4.055463in}}%
\pgfpathcurveto{\pgfqpoint{0.988630in}{4.047649in}}{\pgfqpoint{0.984239in}{4.037050in}}{\pgfqpoint{0.984239in}{4.026000in}}%
\pgfpathcurveto{\pgfqpoint{0.984239in}{4.014950in}}{\pgfqpoint{0.988630in}{4.004351in}}{\pgfqpoint{0.996443in}{3.996537in}}%
\pgfpathcurveto{\pgfqpoint{1.004257in}{3.988724in}}{\pgfqpoint{1.014856in}{3.984333in}}{\pgfqpoint{1.025906in}{3.984333in}}%
\pgfpathclose%
\pgfusepath{stroke,fill}%
\end{pgfscope}%
\begin{pgfscope}%
\pgfpathrectangle{\pgfqpoint{0.800000in}{0.528000in}}{\pgfqpoint{4.960000in}{3.696000in}}%
\pgfusepath{clip}%
\pgfsetbuttcap%
\pgfsetroundjoin%
\definecolor{currentfill}{rgb}{0.000000,0.000000,0.000000}%
\pgfsetfillcolor{currentfill}%
\pgfsetlinewidth{1.003750pt}%
\definecolor{currentstroke}{rgb}{0.000000,0.000000,0.000000}%
\pgfsetstrokecolor{currentstroke}%
\pgfsetdash{}{0pt}%
\pgfpathmoveto{\pgfqpoint{1.025906in}{3.984333in}}%
\pgfpathcurveto{\pgfqpoint{1.036956in}{3.984333in}}{\pgfqpoint{1.047555in}{3.988724in}}{\pgfqpoint{1.055369in}{3.996537in}}%
\pgfpathcurveto{\pgfqpoint{1.063182in}{4.004351in}}{\pgfqpoint{1.067573in}{4.014950in}}{\pgfqpoint{1.067573in}{4.026000in}}%
\pgfpathcurveto{\pgfqpoint{1.067573in}{4.037050in}}{\pgfqpoint{1.063182in}{4.047649in}}{\pgfqpoint{1.055369in}{4.055463in}}%
\pgfpathcurveto{\pgfqpoint{1.047555in}{4.063276in}}{\pgfqpoint{1.036956in}{4.067667in}}{\pgfqpoint{1.025906in}{4.067667in}}%
\pgfpathcurveto{\pgfqpoint{1.014856in}{4.067667in}}{\pgfqpoint{1.004257in}{4.063276in}}{\pgfqpoint{0.996443in}{4.055463in}}%
\pgfpathcurveto{\pgfqpoint{0.988630in}{4.047649in}}{\pgfqpoint{0.984239in}{4.037050in}}{\pgfqpoint{0.984239in}{4.026000in}}%
\pgfpathcurveto{\pgfqpoint{0.984239in}{4.014950in}}{\pgfqpoint{0.988630in}{4.004351in}}{\pgfqpoint{0.996443in}{3.996537in}}%
\pgfpathcurveto{\pgfqpoint{1.004257in}{3.988724in}}{\pgfqpoint{1.014856in}{3.984333in}}{\pgfqpoint{1.025906in}{3.984333in}}%
\pgfpathclose%
\pgfusepath{stroke,fill}%
\end{pgfscope}%
\begin{pgfscope}%
\pgfpathrectangle{\pgfqpoint{0.800000in}{0.528000in}}{\pgfqpoint{4.960000in}{3.696000in}}%
\pgfusepath{clip}%
\pgfsetbuttcap%
\pgfsetroundjoin%
\definecolor{currentfill}{rgb}{0.000000,0.000000,0.000000}%
\pgfsetfillcolor{currentfill}%
\pgfsetlinewidth{1.003750pt}%
\definecolor{currentstroke}{rgb}{0.000000,0.000000,0.000000}%
\pgfsetstrokecolor{currentstroke}%
\pgfsetdash{}{0pt}%
\pgfpathmoveto{\pgfqpoint{1.025906in}{3.984333in}}%
\pgfpathcurveto{\pgfqpoint{1.036956in}{3.984333in}}{\pgfqpoint{1.047555in}{3.988724in}}{\pgfqpoint{1.055369in}{3.996537in}}%
\pgfpathcurveto{\pgfqpoint{1.063182in}{4.004351in}}{\pgfqpoint{1.067573in}{4.014950in}}{\pgfqpoint{1.067573in}{4.026000in}}%
\pgfpathcurveto{\pgfqpoint{1.067573in}{4.037050in}}{\pgfqpoint{1.063182in}{4.047649in}}{\pgfqpoint{1.055369in}{4.055463in}}%
\pgfpathcurveto{\pgfqpoint{1.047555in}{4.063276in}}{\pgfqpoint{1.036956in}{4.067667in}}{\pgfqpoint{1.025906in}{4.067667in}}%
\pgfpathcurveto{\pgfqpoint{1.014856in}{4.067667in}}{\pgfqpoint{1.004257in}{4.063276in}}{\pgfqpoint{0.996443in}{4.055463in}}%
\pgfpathcurveto{\pgfqpoint{0.988630in}{4.047649in}}{\pgfqpoint{0.984239in}{4.037050in}}{\pgfqpoint{0.984239in}{4.026000in}}%
\pgfpathcurveto{\pgfqpoint{0.984239in}{4.014950in}}{\pgfqpoint{0.988630in}{4.004351in}}{\pgfqpoint{0.996443in}{3.996537in}}%
\pgfpathcurveto{\pgfqpoint{1.004257in}{3.988724in}}{\pgfqpoint{1.014856in}{3.984333in}}{\pgfqpoint{1.025906in}{3.984333in}}%
\pgfpathclose%
\pgfusepath{stroke,fill}%
\end{pgfscope}%
\begin{pgfscope}%
\pgfpathrectangle{\pgfqpoint{0.800000in}{0.528000in}}{\pgfqpoint{4.960000in}{3.696000in}}%
\pgfusepath{clip}%
\pgfsetbuttcap%
\pgfsetroundjoin%
\definecolor{currentfill}{rgb}{0.000000,0.000000,0.000000}%
\pgfsetfillcolor{currentfill}%
\pgfsetlinewidth{1.003750pt}%
\definecolor{currentstroke}{rgb}{0.000000,0.000000,0.000000}%
\pgfsetstrokecolor{currentstroke}%
\pgfsetdash{}{0pt}%
\pgfpathmoveto{\pgfqpoint{1.025906in}{3.984333in}}%
\pgfpathcurveto{\pgfqpoint{1.036956in}{3.984333in}}{\pgfqpoint{1.047555in}{3.988724in}}{\pgfqpoint{1.055369in}{3.996537in}}%
\pgfpathcurveto{\pgfqpoint{1.063182in}{4.004351in}}{\pgfqpoint{1.067573in}{4.014950in}}{\pgfqpoint{1.067573in}{4.026000in}}%
\pgfpathcurveto{\pgfqpoint{1.067573in}{4.037050in}}{\pgfqpoint{1.063182in}{4.047649in}}{\pgfqpoint{1.055369in}{4.055463in}}%
\pgfpathcurveto{\pgfqpoint{1.047555in}{4.063276in}}{\pgfqpoint{1.036956in}{4.067667in}}{\pgfqpoint{1.025906in}{4.067667in}}%
\pgfpathcurveto{\pgfqpoint{1.014856in}{4.067667in}}{\pgfqpoint{1.004257in}{4.063276in}}{\pgfqpoint{0.996443in}{4.055463in}}%
\pgfpathcurveto{\pgfqpoint{0.988630in}{4.047649in}}{\pgfqpoint{0.984239in}{4.037050in}}{\pgfqpoint{0.984239in}{4.026000in}}%
\pgfpathcurveto{\pgfqpoint{0.984239in}{4.014950in}}{\pgfqpoint{0.988630in}{4.004351in}}{\pgfqpoint{0.996443in}{3.996537in}}%
\pgfpathcurveto{\pgfqpoint{1.004257in}{3.988724in}}{\pgfqpoint{1.014856in}{3.984333in}}{\pgfqpoint{1.025906in}{3.984333in}}%
\pgfpathclose%
\pgfusepath{stroke,fill}%
\end{pgfscope}%
\begin{pgfscope}%
\pgfpathrectangle{\pgfqpoint{0.800000in}{0.528000in}}{\pgfqpoint{4.960000in}{3.696000in}}%
\pgfusepath{clip}%
\pgfsetbuttcap%
\pgfsetroundjoin%
\definecolor{currentfill}{rgb}{0.000000,0.000000,0.000000}%
\pgfsetfillcolor{currentfill}%
\pgfsetlinewidth{1.003750pt}%
\definecolor{currentstroke}{rgb}{0.000000,0.000000,0.000000}%
\pgfsetstrokecolor{currentstroke}%
\pgfsetdash{}{0pt}%
\pgfpathmoveto{\pgfqpoint{1.025906in}{3.984333in}}%
\pgfpathcurveto{\pgfqpoint{1.036956in}{3.984333in}}{\pgfqpoint{1.047555in}{3.988724in}}{\pgfqpoint{1.055369in}{3.996537in}}%
\pgfpathcurveto{\pgfqpoint{1.063182in}{4.004351in}}{\pgfqpoint{1.067573in}{4.014950in}}{\pgfqpoint{1.067573in}{4.026000in}}%
\pgfpathcurveto{\pgfqpoint{1.067573in}{4.037050in}}{\pgfqpoint{1.063182in}{4.047649in}}{\pgfqpoint{1.055369in}{4.055463in}}%
\pgfpathcurveto{\pgfqpoint{1.047555in}{4.063276in}}{\pgfqpoint{1.036956in}{4.067667in}}{\pgfqpoint{1.025906in}{4.067667in}}%
\pgfpathcurveto{\pgfqpoint{1.014856in}{4.067667in}}{\pgfqpoint{1.004257in}{4.063276in}}{\pgfqpoint{0.996443in}{4.055463in}}%
\pgfpathcurveto{\pgfqpoint{0.988630in}{4.047649in}}{\pgfqpoint{0.984239in}{4.037050in}}{\pgfqpoint{0.984239in}{4.026000in}}%
\pgfpathcurveto{\pgfqpoint{0.984239in}{4.014950in}}{\pgfqpoint{0.988630in}{4.004351in}}{\pgfqpoint{0.996443in}{3.996537in}}%
\pgfpathcurveto{\pgfqpoint{1.004257in}{3.988724in}}{\pgfqpoint{1.014856in}{3.984333in}}{\pgfqpoint{1.025906in}{3.984333in}}%
\pgfpathclose%
\pgfusepath{stroke,fill}%
\end{pgfscope}%
\begin{pgfscope}%
\pgfpathrectangle{\pgfqpoint{0.800000in}{0.528000in}}{\pgfqpoint{4.960000in}{3.696000in}}%
\pgfusepath{clip}%
\pgfsetbuttcap%
\pgfsetroundjoin%
\definecolor{currentfill}{rgb}{0.000000,0.000000,0.000000}%
\pgfsetfillcolor{currentfill}%
\pgfsetlinewidth{1.003750pt}%
\definecolor{currentstroke}{rgb}{0.000000,0.000000,0.000000}%
\pgfsetstrokecolor{currentstroke}%
\pgfsetdash{}{0pt}%
\pgfpathmoveto{\pgfqpoint{1.025906in}{3.984333in}}%
\pgfpathcurveto{\pgfqpoint{1.036956in}{3.984333in}}{\pgfqpoint{1.047555in}{3.988724in}}{\pgfqpoint{1.055369in}{3.996537in}}%
\pgfpathcurveto{\pgfqpoint{1.063182in}{4.004351in}}{\pgfqpoint{1.067573in}{4.014950in}}{\pgfqpoint{1.067573in}{4.026000in}}%
\pgfpathcurveto{\pgfqpoint{1.067573in}{4.037050in}}{\pgfqpoint{1.063182in}{4.047649in}}{\pgfqpoint{1.055369in}{4.055463in}}%
\pgfpathcurveto{\pgfqpoint{1.047555in}{4.063276in}}{\pgfqpoint{1.036956in}{4.067667in}}{\pgfqpoint{1.025906in}{4.067667in}}%
\pgfpathcurveto{\pgfqpoint{1.014856in}{4.067667in}}{\pgfqpoint{1.004257in}{4.063276in}}{\pgfqpoint{0.996443in}{4.055463in}}%
\pgfpathcurveto{\pgfqpoint{0.988630in}{4.047649in}}{\pgfqpoint{0.984239in}{4.037050in}}{\pgfqpoint{0.984239in}{4.026000in}}%
\pgfpathcurveto{\pgfqpoint{0.984239in}{4.014950in}}{\pgfqpoint{0.988630in}{4.004351in}}{\pgfqpoint{0.996443in}{3.996537in}}%
\pgfpathcurveto{\pgfqpoint{1.004257in}{3.988724in}}{\pgfqpoint{1.014856in}{3.984333in}}{\pgfqpoint{1.025906in}{3.984333in}}%
\pgfpathclose%
\pgfusepath{stroke,fill}%
\end{pgfscope}%
\begin{pgfscope}%
\pgfpathrectangle{\pgfqpoint{0.800000in}{0.528000in}}{\pgfqpoint{4.960000in}{3.696000in}}%
\pgfusepath{clip}%
\pgfsetbuttcap%
\pgfsetroundjoin%
\definecolor{currentfill}{rgb}{0.000000,0.000000,0.000000}%
\pgfsetfillcolor{currentfill}%
\pgfsetlinewidth{1.003750pt}%
\definecolor{currentstroke}{rgb}{0.000000,0.000000,0.000000}%
\pgfsetstrokecolor{currentstroke}%
\pgfsetdash{}{0pt}%
\pgfpathmoveto{\pgfqpoint{1.025906in}{3.984333in}}%
\pgfpathcurveto{\pgfqpoint{1.036956in}{3.984333in}}{\pgfqpoint{1.047555in}{3.988724in}}{\pgfqpoint{1.055369in}{3.996537in}}%
\pgfpathcurveto{\pgfqpoint{1.063182in}{4.004351in}}{\pgfqpoint{1.067573in}{4.014950in}}{\pgfqpoint{1.067573in}{4.026000in}}%
\pgfpathcurveto{\pgfqpoint{1.067573in}{4.037050in}}{\pgfqpoint{1.063182in}{4.047649in}}{\pgfqpoint{1.055369in}{4.055463in}}%
\pgfpathcurveto{\pgfqpoint{1.047555in}{4.063276in}}{\pgfqpoint{1.036956in}{4.067667in}}{\pgfqpoint{1.025906in}{4.067667in}}%
\pgfpathcurveto{\pgfqpoint{1.014856in}{4.067667in}}{\pgfqpoint{1.004257in}{4.063276in}}{\pgfqpoint{0.996443in}{4.055463in}}%
\pgfpathcurveto{\pgfqpoint{0.988630in}{4.047649in}}{\pgfqpoint{0.984239in}{4.037050in}}{\pgfqpoint{0.984239in}{4.026000in}}%
\pgfpathcurveto{\pgfqpoint{0.984239in}{4.014950in}}{\pgfqpoint{0.988630in}{4.004351in}}{\pgfqpoint{0.996443in}{3.996537in}}%
\pgfpathcurveto{\pgfqpoint{1.004257in}{3.988724in}}{\pgfqpoint{1.014856in}{3.984333in}}{\pgfqpoint{1.025906in}{3.984333in}}%
\pgfpathclose%
\pgfusepath{stroke,fill}%
\end{pgfscope}%
\begin{pgfscope}%
\pgfpathrectangle{\pgfqpoint{0.800000in}{0.528000in}}{\pgfqpoint{4.960000in}{3.696000in}}%
\pgfusepath{clip}%
\pgfsetbuttcap%
\pgfsetroundjoin%
\definecolor{currentfill}{rgb}{0.000000,0.000000,0.000000}%
\pgfsetfillcolor{currentfill}%
\pgfsetlinewidth{1.003750pt}%
\definecolor{currentstroke}{rgb}{0.000000,0.000000,0.000000}%
\pgfsetstrokecolor{currentstroke}%
\pgfsetdash{}{0pt}%
\pgfpathmoveto{\pgfqpoint{1.025906in}{3.984333in}}%
\pgfpathcurveto{\pgfqpoint{1.036956in}{3.984333in}}{\pgfqpoint{1.047555in}{3.988724in}}{\pgfqpoint{1.055369in}{3.996537in}}%
\pgfpathcurveto{\pgfqpoint{1.063182in}{4.004351in}}{\pgfqpoint{1.067573in}{4.014950in}}{\pgfqpoint{1.067573in}{4.026000in}}%
\pgfpathcurveto{\pgfqpoint{1.067573in}{4.037050in}}{\pgfqpoint{1.063182in}{4.047649in}}{\pgfqpoint{1.055369in}{4.055463in}}%
\pgfpathcurveto{\pgfqpoint{1.047555in}{4.063276in}}{\pgfqpoint{1.036956in}{4.067667in}}{\pgfqpoint{1.025906in}{4.067667in}}%
\pgfpathcurveto{\pgfqpoint{1.014856in}{4.067667in}}{\pgfqpoint{1.004257in}{4.063276in}}{\pgfqpoint{0.996443in}{4.055463in}}%
\pgfpathcurveto{\pgfqpoint{0.988630in}{4.047649in}}{\pgfqpoint{0.984239in}{4.037050in}}{\pgfqpoint{0.984239in}{4.026000in}}%
\pgfpathcurveto{\pgfqpoint{0.984239in}{4.014950in}}{\pgfqpoint{0.988630in}{4.004351in}}{\pgfqpoint{0.996443in}{3.996537in}}%
\pgfpathcurveto{\pgfqpoint{1.004257in}{3.988724in}}{\pgfqpoint{1.014856in}{3.984333in}}{\pgfqpoint{1.025906in}{3.984333in}}%
\pgfpathclose%
\pgfusepath{stroke,fill}%
\end{pgfscope}%
\begin{pgfscope}%
\pgfpathrectangle{\pgfqpoint{0.800000in}{0.528000in}}{\pgfqpoint{4.960000in}{3.696000in}}%
\pgfusepath{clip}%
\pgfsetbuttcap%
\pgfsetroundjoin%
\definecolor{currentfill}{rgb}{0.000000,0.000000,0.000000}%
\pgfsetfillcolor{currentfill}%
\pgfsetlinewidth{1.003750pt}%
\definecolor{currentstroke}{rgb}{0.000000,0.000000,0.000000}%
\pgfsetstrokecolor{currentstroke}%
\pgfsetdash{}{0pt}%
\pgfpathmoveto{\pgfqpoint{1.025906in}{3.984333in}}%
\pgfpathcurveto{\pgfqpoint{1.036956in}{3.984333in}}{\pgfqpoint{1.047555in}{3.988724in}}{\pgfqpoint{1.055369in}{3.996537in}}%
\pgfpathcurveto{\pgfqpoint{1.063182in}{4.004351in}}{\pgfqpoint{1.067573in}{4.014950in}}{\pgfqpoint{1.067573in}{4.026000in}}%
\pgfpathcurveto{\pgfqpoint{1.067573in}{4.037050in}}{\pgfqpoint{1.063182in}{4.047649in}}{\pgfqpoint{1.055369in}{4.055463in}}%
\pgfpathcurveto{\pgfqpoint{1.047555in}{4.063276in}}{\pgfqpoint{1.036956in}{4.067667in}}{\pgfqpoint{1.025906in}{4.067667in}}%
\pgfpathcurveto{\pgfqpoint{1.014856in}{4.067667in}}{\pgfqpoint{1.004257in}{4.063276in}}{\pgfqpoint{0.996443in}{4.055463in}}%
\pgfpathcurveto{\pgfqpoint{0.988630in}{4.047649in}}{\pgfqpoint{0.984239in}{4.037050in}}{\pgfqpoint{0.984239in}{4.026000in}}%
\pgfpathcurveto{\pgfqpoint{0.984239in}{4.014950in}}{\pgfqpoint{0.988630in}{4.004351in}}{\pgfqpoint{0.996443in}{3.996537in}}%
\pgfpathcurveto{\pgfqpoint{1.004257in}{3.988724in}}{\pgfqpoint{1.014856in}{3.984333in}}{\pgfqpoint{1.025906in}{3.984333in}}%
\pgfpathclose%
\pgfusepath{stroke,fill}%
\end{pgfscope}%
\begin{pgfscope}%
\pgfpathrectangle{\pgfqpoint{0.800000in}{0.528000in}}{\pgfqpoint{4.960000in}{3.696000in}}%
\pgfusepath{clip}%
\pgfsetbuttcap%
\pgfsetroundjoin%
\definecolor{currentfill}{rgb}{0.000000,0.000000,0.000000}%
\pgfsetfillcolor{currentfill}%
\pgfsetlinewidth{1.003750pt}%
\definecolor{currentstroke}{rgb}{0.000000,0.000000,0.000000}%
\pgfsetstrokecolor{currentstroke}%
\pgfsetdash{}{0pt}%
\pgfpathmoveto{\pgfqpoint{1.025906in}{3.984333in}}%
\pgfpathcurveto{\pgfqpoint{1.036956in}{3.984333in}}{\pgfqpoint{1.047555in}{3.988724in}}{\pgfqpoint{1.055369in}{3.996537in}}%
\pgfpathcurveto{\pgfqpoint{1.063182in}{4.004351in}}{\pgfqpoint{1.067573in}{4.014950in}}{\pgfqpoint{1.067573in}{4.026000in}}%
\pgfpathcurveto{\pgfqpoint{1.067573in}{4.037050in}}{\pgfqpoint{1.063182in}{4.047649in}}{\pgfqpoint{1.055369in}{4.055463in}}%
\pgfpathcurveto{\pgfqpoint{1.047555in}{4.063276in}}{\pgfqpoint{1.036956in}{4.067667in}}{\pgfqpoint{1.025906in}{4.067667in}}%
\pgfpathcurveto{\pgfqpoint{1.014856in}{4.067667in}}{\pgfqpoint{1.004257in}{4.063276in}}{\pgfqpoint{0.996443in}{4.055463in}}%
\pgfpathcurveto{\pgfqpoint{0.988630in}{4.047649in}}{\pgfqpoint{0.984239in}{4.037050in}}{\pgfqpoint{0.984239in}{4.026000in}}%
\pgfpathcurveto{\pgfqpoint{0.984239in}{4.014950in}}{\pgfqpoint{0.988630in}{4.004351in}}{\pgfqpoint{0.996443in}{3.996537in}}%
\pgfpathcurveto{\pgfqpoint{1.004257in}{3.988724in}}{\pgfqpoint{1.014856in}{3.984333in}}{\pgfqpoint{1.025906in}{3.984333in}}%
\pgfpathclose%
\pgfusepath{stroke,fill}%
\end{pgfscope}%
\begin{pgfscope}%
\pgfpathrectangle{\pgfqpoint{0.800000in}{0.528000in}}{\pgfqpoint{4.960000in}{3.696000in}}%
\pgfusepath{clip}%
\pgfsetbuttcap%
\pgfsetroundjoin%
\definecolor{currentfill}{rgb}{0.000000,0.000000,0.000000}%
\pgfsetfillcolor{currentfill}%
\pgfsetlinewidth{1.003750pt}%
\definecolor{currentstroke}{rgb}{0.000000,0.000000,0.000000}%
\pgfsetstrokecolor{currentstroke}%
\pgfsetdash{}{0pt}%
\pgfpathmoveto{\pgfqpoint{1.025906in}{3.984333in}}%
\pgfpathcurveto{\pgfqpoint{1.036956in}{3.984333in}}{\pgfqpoint{1.047555in}{3.988724in}}{\pgfqpoint{1.055369in}{3.996537in}}%
\pgfpathcurveto{\pgfqpoint{1.063182in}{4.004351in}}{\pgfqpoint{1.067573in}{4.014950in}}{\pgfqpoint{1.067573in}{4.026000in}}%
\pgfpathcurveto{\pgfqpoint{1.067573in}{4.037050in}}{\pgfqpoint{1.063182in}{4.047649in}}{\pgfqpoint{1.055369in}{4.055463in}}%
\pgfpathcurveto{\pgfqpoint{1.047555in}{4.063276in}}{\pgfqpoint{1.036956in}{4.067667in}}{\pgfqpoint{1.025906in}{4.067667in}}%
\pgfpathcurveto{\pgfqpoint{1.014856in}{4.067667in}}{\pgfqpoint{1.004257in}{4.063276in}}{\pgfqpoint{0.996443in}{4.055463in}}%
\pgfpathcurveto{\pgfqpoint{0.988630in}{4.047649in}}{\pgfqpoint{0.984239in}{4.037050in}}{\pgfqpoint{0.984239in}{4.026000in}}%
\pgfpathcurveto{\pgfqpoint{0.984239in}{4.014950in}}{\pgfqpoint{0.988630in}{4.004351in}}{\pgfqpoint{0.996443in}{3.996537in}}%
\pgfpathcurveto{\pgfqpoint{1.004257in}{3.988724in}}{\pgfqpoint{1.014856in}{3.984333in}}{\pgfqpoint{1.025906in}{3.984333in}}%
\pgfpathclose%
\pgfusepath{stroke,fill}%
\end{pgfscope}%
\begin{pgfscope}%
\pgfpathrectangle{\pgfqpoint{0.800000in}{0.528000in}}{\pgfqpoint{4.960000in}{3.696000in}}%
\pgfusepath{clip}%
\pgfsetbuttcap%
\pgfsetroundjoin%
\definecolor{currentfill}{rgb}{0.000000,0.000000,0.000000}%
\pgfsetfillcolor{currentfill}%
\pgfsetlinewidth{1.003750pt}%
\definecolor{currentstroke}{rgb}{0.000000,0.000000,0.000000}%
\pgfsetstrokecolor{currentstroke}%
\pgfsetdash{}{0pt}%
\pgfpathmoveto{\pgfqpoint{1.025906in}{3.984333in}}%
\pgfpathcurveto{\pgfqpoint{1.036956in}{3.984333in}}{\pgfqpoint{1.047555in}{3.988724in}}{\pgfqpoint{1.055369in}{3.996537in}}%
\pgfpathcurveto{\pgfqpoint{1.063182in}{4.004351in}}{\pgfqpoint{1.067573in}{4.014950in}}{\pgfqpoint{1.067573in}{4.026000in}}%
\pgfpathcurveto{\pgfqpoint{1.067573in}{4.037050in}}{\pgfqpoint{1.063182in}{4.047649in}}{\pgfqpoint{1.055369in}{4.055463in}}%
\pgfpathcurveto{\pgfqpoint{1.047555in}{4.063276in}}{\pgfqpoint{1.036956in}{4.067667in}}{\pgfqpoint{1.025906in}{4.067667in}}%
\pgfpathcurveto{\pgfqpoint{1.014856in}{4.067667in}}{\pgfqpoint{1.004257in}{4.063276in}}{\pgfqpoint{0.996443in}{4.055463in}}%
\pgfpathcurveto{\pgfqpoint{0.988630in}{4.047649in}}{\pgfqpoint{0.984239in}{4.037050in}}{\pgfqpoint{0.984239in}{4.026000in}}%
\pgfpathcurveto{\pgfqpoint{0.984239in}{4.014950in}}{\pgfqpoint{0.988630in}{4.004351in}}{\pgfqpoint{0.996443in}{3.996537in}}%
\pgfpathcurveto{\pgfqpoint{1.004257in}{3.988724in}}{\pgfqpoint{1.014856in}{3.984333in}}{\pgfqpoint{1.025906in}{3.984333in}}%
\pgfpathclose%
\pgfusepath{stroke,fill}%
\end{pgfscope}%
\begin{pgfscope}%
\pgfpathrectangle{\pgfqpoint{0.800000in}{0.528000in}}{\pgfqpoint{4.960000in}{3.696000in}}%
\pgfusepath{clip}%
\pgfsetbuttcap%
\pgfsetroundjoin%
\definecolor{currentfill}{rgb}{0.000000,0.000000,0.000000}%
\pgfsetfillcolor{currentfill}%
\pgfsetlinewidth{1.003750pt}%
\definecolor{currentstroke}{rgb}{0.000000,0.000000,0.000000}%
\pgfsetstrokecolor{currentstroke}%
\pgfsetdash{}{0pt}%
\pgfpathmoveto{\pgfqpoint{1.025906in}{3.984333in}}%
\pgfpathcurveto{\pgfqpoint{1.036956in}{3.984333in}}{\pgfqpoint{1.047555in}{3.988724in}}{\pgfqpoint{1.055369in}{3.996537in}}%
\pgfpathcurveto{\pgfqpoint{1.063182in}{4.004351in}}{\pgfqpoint{1.067573in}{4.014950in}}{\pgfqpoint{1.067573in}{4.026000in}}%
\pgfpathcurveto{\pgfqpoint{1.067573in}{4.037050in}}{\pgfqpoint{1.063182in}{4.047649in}}{\pgfqpoint{1.055369in}{4.055463in}}%
\pgfpathcurveto{\pgfqpoint{1.047555in}{4.063276in}}{\pgfqpoint{1.036956in}{4.067667in}}{\pgfqpoint{1.025906in}{4.067667in}}%
\pgfpathcurveto{\pgfqpoint{1.014856in}{4.067667in}}{\pgfqpoint{1.004257in}{4.063276in}}{\pgfqpoint{0.996443in}{4.055463in}}%
\pgfpathcurveto{\pgfqpoint{0.988630in}{4.047649in}}{\pgfqpoint{0.984239in}{4.037050in}}{\pgfqpoint{0.984239in}{4.026000in}}%
\pgfpathcurveto{\pgfqpoint{0.984239in}{4.014950in}}{\pgfqpoint{0.988630in}{4.004351in}}{\pgfqpoint{0.996443in}{3.996537in}}%
\pgfpathcurveto{\pgfqpoint{1.004257in}{3.988724in}}{\pgfqpoint{1.014856in}{3.984333in}}{\pgfqpoint{1.025906in}{3.984333in}}%
\pgfpathclose%
\pgfusepath{stroke,fill}%
\end{pgfscope}%
\begin{pgfscope}%
\pgfpathrectangle{\pgfqpoint{0.800000in}{0.528000in}}{\pgfqpoint{4.960000in}{3.696000in}}%
\pgfusepath{clip}%
\pgfsetbuttcap%
\pgfsetroundjoin%
\definecolor{currentfill}{rgb}{0.000000,0.000000,0.000000}%
\pgfsetfillcolor{currentfill}%
\pgfsetlinewidth{1.003750pt}%
\definecolor{currentstroke}{rgb}{0.000000,0.000000,0.000000}%
\pgfsetstrokecolor{currentstroke}%
\pgfsetdash{}{0pt}%
\pgfpathmoveto{\pgfqpoint{1.025906in}{3.984333in}}%
\pgfpathcurveto{\pgfqpoint{1.036956in}{3.984333in}}{\pgfqpoint{1.047555in}{3.988724in}}{\pgfqpoint{1.055369in}{3.996537in}}%
\pgfpathcurveto{\pgfqpoint{1.063182in}{4.004351in}}{\pgfqpoint{1.067573in}{4.014950in}}{\pgfqpoint{1.067573in}{4.026000in}}%
\pgfpathcurveto{\pgfqpoint{1.067573in}{4.037050in}}{\pgfqpoint{1.063182in}{4.047649in}}{\pgfqpoint{1.055369in}{4.055463in}}%
\pgfpathcurveto{\pgfqpoint{1.047555in}{4.063276in}}{\pgfqpoint{1.036956in}{4.067667in}}{\pgfqpoint{1.025906in}{4.067667in}}%
\pgfpathcurveto{\pgfqpoint{1.014856in}{4.067667in}}{\pgfqpoint{1.004257in}{4.063276in}}{\pgfqpoint{0.996443in}{4.055463in}}%
\pgfpathcurveto{\pgfqpoint{0.988630in}{4.047649in}}{\pgfqpoint{0.984239in}{4.037050in}}{\pgfqpoint{0.984239in}{4.026000in}}%
\pgfpathcurveto{\pgfqpoint{0.984239in}{4.014950in}}{\pgfqpoint{0.988630in}{4.004351in}}{\pgfqpoint{0.996443in}{3.996537in}}%
\pgfpathcurveto{\pgfqpoint{1.004257in}{3.988724in}}{\pgfqpoint{1.014856in}{3.984333in}}{\pgfqpoint{1.025906in}{3.984333in}}%
\pgfpathclose%
\pgfusepath{stroke,fill}%
\end{pgfscope}%
\begin{pgfscope}%
\pgfpathrectangle{\pgfqpoint{0.800000in}{0.528000in}}{\pgfqpoint{4.960000in}{3.696000in}}%
\pgfusepath{clip}%
\pgfsetbuttcap%
\pgfsetroundjoin%
\definecolor{currentfill}{rgb}{0.000000,0.000000,0.000000}%
\pgfsetfillcolor{currentfill}%
\pgfsetlinewidth{1.003750pt}%
\definecolor{currentstroke}{rgb}{0.000000,0.000000,0.000000}%
\pgfsetstrokecolor{currentstroke}%
\pgfsetdash{}{0pt}%
\pgfpathmoveto{\pgfqpoint{1.025906in}{3.984333in}}%
\pgfpathcurveto{\pgfqpoint{1.036956in}{3.984333in}}{\pgfqpoint{1.047555in}{3.988724in}}{\pgfqpoint{1.055369in}{3.996537in}}%
\pgfpathcurveto{\pgfqpoint{1.063182in}{4.004351in}}{\pgfqpoint{1.067573in}{4.014950in}}{\pgfqpoint{1.067573in}{4.026000in}}%
\pgfpathcurveto{\pgfqpoint{1.067573in}{4.037050in}}{\pgfqpoint{1.063182in}{4.047649in}}{\pgfqpoint{1.055369in}{4.055463in}}%
\pgfpathcurveto{\pgfqpoint{1.047555in}{4.063276in}}{\pgfqpoint{1.036956in}{4.067667in}}{\pgfqpoint{1.025906in}{4.067667in}}%
\pgfpathcurveto{\pgfqpoint{1.014856in}{4.067667in}}{\pgfqpoint{1.004257in}{4.063276in}}{\pgfqpoint{0.996443in}{4.055463in}}%
\pgfpathcurveto{\pgfqpoint{0.988630in}{4.047649in}}{\pgfqpoint{0.984239in}{4.037050in}}{\pgfqpoint{0.984239in}{4.026000in}}%
\pgfpathcurveto{\pgfqpoint{0.984239in}{4.014950in}}{\pgfqpoint{0.988630in}{4.004351in}}{\pgfqpoint{0.996443in}{3.996537in}}%
\pgfpathcurveto{\pgfqpoint{1.004257in}{3.988724in}}{\pgfqpoint{1.014856in}{3.984333in}}{\pgfqpoint{1.025906in}{3.984333in}}%
\pgfpathclose%
\pgfusepath{stroke,fill}%
\end{pgfscope}%
\begin{pgfscope}%
\pgfpathrectangle{\pgfqpoint{0.800000in}{0.528000in}}{\pgfqpoint{4.960000in}{3.696000in}}%
\pgfusepath{clip}%
\pgfsetbuttcap%
\pgfsetroundjoin%
\definecolor{currentfill}{rgb}{0.000000,0.000000,0.000000}%
\pgfsetfillcolor{currentfill}%
\pgfsetlinewidth{1.003750pt}%
\definecolor{currentstroke}{rgb}{0.000000,0.000000,0.000000}%
\pgfsetstrokecolor{currentstroke}%
\pgfsetdash{}{0pt}%
\pgfpathmoveto{\pgfqpoint{1.025906in}{3.984333in}}%
\pgfpathcurveto{\pgfqpoint{1.036956in}{3.984333in}}{\pgfqpoint{1.047555in}{3.988724in}}{\pgfqpoint{1.055369in}{3.996537in}}%
\pgfpathcurveto{\pgfqpoint{1.063182in}{4.004351in}}{\pgfqpoint{1.067573in}{4.014950in}}{\pgfqpoint{1.067573in}{4.026000in}}%
\pgfpathcurveto{\pgfqpoint{1.067573in}{4.037050in}}{\pgfqpoint{1.063182in}{4.047649in}}{\pgfqpoint{1.055369in}{4.055463in}}%
\pgfpathcurveto{\pgfqpoint{1.047555in}{4.063276in}}{\pgfqpoint{1.036956in}{4.067667in}}{\pgfqpoint{1.025906in}{4.067667in}}%
\pgfpathcurveto{\pgfqpoint{1.014856in}{4.067667in}}{\pgfqpoint{1.004257in}{4.063276in}}{\pgfqpoint{0.996443in}{4.055463in}}%
\pgfpathcurveto{\pgfqpoint{0.988630in}{4.047649in}}{\pgfqpoint{0.984239in}{4.037050in}}{\pgfqpoint{0.984239in}{4.026000in}}%
\pgfpathcurveto{\pgfqpoint{0.984239in}{4.014950in}}{\pgfqpoint{0.988630in}{4.004351in}}{\pgfqpoint{0.996443in}{3.996537in}}%
\pgfpathcurveto{\pgfqpoint{1.004257in}{3.988724in}}{\pgfqpoint{1.014856in}{3.984333in}}{\pgfqpoint{1.025906in}{3.984333in}}%
\pgfpathclose%
\pgfusepath{stroke,fill}%
\end{pgfscope}%
\begin{pgfscope}%
\pgfpathrectangle{\pgfqpoint{0.800000in}{0.528000in}}{\pgfqpoint{4.960000in}{3.696000in}}%
\pgfusepath{clip}%
\pgfsetbuttcap%
\pgfsetroundjoin%
\definecolor{currentfill}{rgb}{0.000000,0.000000,0.000000}%
\pgfsetfillcolor{currentfill}%
\pgfsetlinewidth{1.003750pt}%
\definecolor{currentstroke}{rgb}{0.000000,0.000000,0.000000}%
\pgfsetstrokecolor{currentstroke}%
\pgfsetdash{}{0pt}%
\pgfpathmoveto{\pgfqpoint{1.025906in}{3.984333in}}%
\pgfpathcurveto{\pgfqpoint{1.036956in}{3.984333in}}{\pgfqpoint{1.047555in}{3.988724in}}{\pgfqpoint{1.055369in}{3.996537in}}%
\pgfpathcurveto{\pgfqpoint{1.063182in}{4.004351in}}{\pgfqpoint{1.067573in}{4.014950in}}{\pgfqpoint{1.067573in}{4.026000in}}%
\pgfpathcurveto{\pgfqpoint{1.067573in}{4.037050in}}{\pgfqpoint{1.063182in}{4.047649in}}{\pgfqpoint{1.055369in}{4.055463in}}%
\pgfpathcurveto{\pgfqpoint{1.047555in}{4.063276in}}{\pgfqpoint{1.036956in}{4.067667in}}{\pgfqpoint{1.025906in}{4.067667in}}%
\pgfpathcurveto{\pgfqpoint{1.014856in}{4.067667in}}{\pgfqpoint{1.004257in}{4.063276in}}{\pgfqpoint{0.996443in}{4.055463in}}%
\pgfpathcurveto{\pgfqpoint{0.988630in}{4.047649in}}{\pgfqpoint{0.984239in}{4.037050in}}{\pgfqpoint{0.984239in}{4.026000in}}%
\pgfpathcurveto{\pgfqpoint{0.984239in}{4.014950in}}{\pgfqpoint{0.988630in}{4.004351in}}{\pgfqpoint{0.996443in}{3.996537in}}%
\pgfpathcurveto{\pgfqpoint{1.004257in}{3.988724in}}{\pgfqpoint{1.014856in}{3.984333in}}{\pgfqpoint{1.025906in}{3.984333in}}%
\pgfpathclose%
\pgfusepath{stroke,fill}%
\end{pgfscope}%
\begin{pgfscope}%
\pgfpathrectangle{\pgfqpoint{0.800000in}{0.528000in}}{\pgfqpoint{4.960000in}{3.696000in}}%
\pgfusepath{clip}%
\pgfsetbuttcap%
\pgfsetroundjoin%
\definecolor{currentfill}{rgb}{0.000000,0.000000,0.000000}%
\pgfsetfillcolor{currentfill}%
\pgfsetlinewidth{1.003750pt}%
\definecolor{currentstroke}{rgb}{0.000000,0.000000,0.000000}%
\pgfsetstrokecolor{currentstroke}%
\pgfsetdash{}{0pt}%
\pgfpathmoveto{\pgfqpoint{1.025906in}{3.984333in}}%
\pgfpathcurveto{\pgfqpoint{1.036956in}{3.984333in}}{\pgfqpoint{1.047555in}{3.988724in}}{\pgfqpoint{1.055369in}{3.996537in}}%
\pgfpathcurveto{\pgfqpoint{1.063182in}{4.004351in}}{\pgfqpoint{1.067573in}{4.014950in}}{\pgfqpoint{1.067573in}{4.026000in}}%
\pgfpathcurveto{\pgfqpoint{1.067573in}{4.037050in}}{\pgfqpoint{1.063182in}{4.047649in}}{\pgfqpoint{1.055369in}{4.055463in}}%
\pgfpathcurveto{\pgfqpoint{1.047555in}{4.063276in}}{\pgfqpoint{1.036956in}{4.067667in}}{\pgfqpoint{1.025906in}{4.067667in}}%
\pgfpathcurveto{\pgfqpoint{1.014856in}{4.067667in}}{\pgfqpoint{1.004257in}{4.063276in}}{\pgfqpoint{0.996443in}{4.055463in}}%
\pgfpathcurveto{\pgfqpoint{0.988630in}{4.047649in}}{\pgfqpoint{0.984239in}{4.037050in}}{\pgfqpoint{0.984239in}{4.026000in}}%
\pgfpathcurveto{\pgfqpoint{0.984239in}{4.014950in}}{\pgfqpoint{0.988630in}{4.004351in}}{\pgfqpoint{0.996443in}{3.996537in}}%
\pgfpathcurveto{\pgfqpoint{1.004257in}{3.988724in}}{\pgfqpoint{1.014856in}{3.984333in}}{\pgfqpoint{1.025906in}{3.984333in}}%
\pgfpathclose%
\pgfusepath{stroke,fill}%
\end{pgfscope}%
\begin{pgfscope}%
\pgfpathrectangle{\pgfqpoint{0.800000in}{0.528000in}}{\pgfqpoint{4.960000in}{3.696000in}}%
\pgfusepath{clip}%
\pgfsetbuttcap%
\pgfsetroundjoin%
\definecolor{currentfill}{rgb}{0.000000,0.000000,0.000000}%
\pgfsetfillcolor{currentfill}%
\pgfsetlinewidth{1.003750pt}%
\definecolor{currentstroke}{rgb}{0.000000,0.000000,0.000000}%
\pgfsetstrokecolor{currentstroke}%
\pgfsetdash{}{0pt}%
\pgfpathmoveto{\pgfqpoint{1.025906in}{3.984333in}}%
\pgfpathcurveto{\pgfqpoint{1.036956in}{3.984333in}}{\pgfqpoint{1.047555in}{3.988724in}}{\pgfqpoint{1.055369in}{3.996537in}}%
\pgfpathcurveto{\pgfqpoint{1.063182in}{4.004351in}}{\pgfqpoint{1.067573in}{4.014950in}}{\pgfqpoint{1.067573in}{4.026000in}}%
\pgfpathcurveto{\pgfqpoint{1.067573in}{4.037050in}}{\pgfqpoint{1.063182in}{4.047649in}}{\pgfqpoint{1.055369in}{4.055463in}}%
\pgfpathcurveto{\pgfqpoint{1.047555in}{4.063276in}}{\pgfqpoint{1.036956in}{4.067667in}}{\pgfqpoint{1.025906in}{4.067667in}}%
\pgfpathcurveto{\pgfqpoint{1.014856in}{4.067667in}}{\pgfqpoint{1.004257in}{4.063276in}}{\pgfqpoint{0.996443in}{4.055463in}}%
\pgfpathcurveto{\pgfqpoint{0.988630in}{4.047649in}}{\pgfqpoint{0.984239in}{4.037050in}}{\pgfqpoint{0.984239in}{4.026000in}}%
\pgfpathcurveto{\pgfqpoint{0.984239in}{4.014950in}}{\pgfqpoint{0.988630in}{4.004351in}}{\pgfqpoint{0.996443in}{3.996537in}}%
\pgfpathcurveto{\pgfqpoint{1.004257in}{3.988724in}}{\pgfqpoint{1.014856in}{3.984333in}}{\pgfqpoint{1.025906in}{3.984333in}}%
\pgfpathclose%
\pgfusepath{stroke,fill}%
\end{pgfscope}%
\begin{pgfscope}%
\pgfpathrectangle{\pgfqpoint{0.800000in}{0.528000in}}{\pgfqpoint{4.960000in}{3.696000in}}%
\pgfusepath{clip}%
\pgfsetbuttcap%
\pgfsetroundjoin%
\definecolor{currentfill}{rgb}{0.000000,0.000000,0.000000}%
\pgfsetfillcolor{currentfill}%
\pgfsetlinewidth{1.003750pt}%
\definecolor{currentstroke}{rgb}{0.000000,0.000000,0.000000}%
\pgfsetstrokecolor{currentstroke}%
\pgfsetdash{}{0pt}%
\pgfpathmoveto{\pgfqpoint{1.025906in}{3.984333in}}%
\pgfpathcurveto{\pgfqpoint{1.036956in}{3.984333in}}{\pgfqpoint{1.047555in}{3.988724in}}{\pgfqpoint{1.055369in}{3.996537in}}%
\pgfpathcurveto{\pgfqpoint{1.063182in}{4.004351in}}{\pgfqpoint{1.067573in}{4.014950in}}{\pgfqpoint{1.067573in}{4.026000in}}%
\pgfpathcurveto{\pgfqpoint{1.067573in}{4.037050in}}{\pgfqpoint{1.063182in}{4.047649in}}{\pgfqpoint{1.055369in}{4.055463in}}%
\pgfpathcurveto{\pgfqpoint{1.047555in}{4.063276in}}{\pgfqpoint{1.036956in}{4.067667in}}{\pgfqpoint{1.025906in}{4.067667in}}%
\pgfpathcurveto{\pgfqpoint{1.014856in}{4.067667in}}{\pgfqpoint{1.004257in}{4.063276in}}{\pgfqpoint{0.996443in}{4.055463in}}%
\pgfpathcurveto{\pgfqpoint{0.988630in}{4.047649in}}{\pgfqpoint{0.984239in}{4.037050in}}{\pgfqpoint{0.984239in}{4.026000in}}%
\pgfpathcurveto{\pgfqpoint{0.984239in}{4.014950in}}{\pgfqpoint{0.988630in}{4.004351in}}{\pgfqpoint{0.996443in}{3.996537in}}%
\pgfpathcurveto{\pgfqpoint{1.004257in}{3.988724in}}{\pgfqpoint{1.014856in}{3.984333in}}{\pgfqpoint{1.025906in}{3.984333in}}%
\pgfpathclose%
\pgfusepath{stroke,fill}%
\end{pgfscope}%
\begin{pgfscope}%
\pgfpathrectangle{\pgfqpoint{0.800000in}{0.528000in}}{\pgfqpoint{4.960000in}{3.696000in}}%
\pgfusepath{clip}%
\pgfsetbuttcap%
\pgfsetroundjoin%
\definecolor{currentfill}{rgb}{0.000000,0.000000,0.000000}%
\pgfsetfillcolor{currentfill}%
\pgfsetlinewidth{1.003750pt}%
\definecolor{currentstroke}{rgb}{0.000000,0.000000,0.000000}%
\pgfsetstrokecolor{currentstroke}%
\pgfsetdash{}{0pt}%
\pgfpathmoveto{\pgfqpoint{1.025906in}{3.984333in}}%
\pgfpathcurveto{\pgfqpoint{1.036956in}{3.984333in}}{\pgfqpoint{1.047555in}{3.988724in}}{\pgfqpoint{1.055369in}{3.996537in}}%
\pgfpathcurveto{\pgfqpoint{1.063182in}{4.004351in}}{\pgfqpoint{1.067573in}{4.014950in}}{\pgfqpoint{1.067573in}{4.026000in}}%
\pgfpathcurveto{\pgfqpoint{1.067573in}{4.037050in}}{\pgfqpoint{1.063182in}{4.047649in}}{\pgfqpoint{1.055369in}{4.055463in}}%
\pgfpathcurveto{\pgfqpoint{1.047555in}{4.063276in}}{\pgfqpoint{1.036956in}{4.067667in}}{\pgfqpoint{1.025906in}{4.067667in}}%
\pgfpathcurveto{\pgfqpoint{1.014856in}{4.067667in}}{\pgfqpoint{1.004257in}{4.063276in}}{\pgfqpoint{0.996443in}{4.055463in}}%
\pgfpathcurveto{\pgfqpoint{0.988630in}{4.047649in}}{\pgfqpoint{0.984239in}{4.037050in}}{\pgfqpoint{0.984239in}{4.026000in}}%
\pgfpathcurveto{\pgfqpoint{0.984239in}{4.014950in}}{\pgfqpoint{0.988630in}{4.004351in}}{\pgfqpoint{0.996443in}{3.996537in}}%
\pgfpathcurveto{\pgfqpoint{1.004257in}{3.988724in}}{\pgfqpoint{1.014856in}{3.984333in}}{\pgfqpoint{1.025906in}{3.984333in}}%
\pgfpathclose%
\pgfusepath{stroke,fill}%
\end{pgfscope}%
\begin{pgfscope}%
\pgfpathrectangle{\pgfqpoint{0.800000in}{0.528000in}}{\pgfqpoint{4.960000in}{3.696000in}}%
\pgfusepath{clip}%
\pgfsetbuttcap%
\pgfsetroundjoin%
\definecolor{currentfill}{rgb}{0.000000,0.000000,0.000000}%
\pgfsetfillcolor{currentfill}%
\pgfsetlinewidth{1.003750pt}%
\definecolor{currentstroke}{rgb}{0.000000,0.000000,0.000000}%
\pgfsetstrokecolor{currentstroke}%
\pgfsetdash{}{0pt}%
\pgfpathmoveto{\pgfqpoint{1.025906in}{3.984333in}}%
\pgfpathcurveto{\pgfqpoint{1.036956in}{3.984333in}}{\pgfqpoint{1.047555in}{3.988724in}}{\pgfqpoint{1.055369in}{3.996537in}}%
\pgfpathcurveto{\pgfqpoint{1.063182in}{4.004351in}}{\pgfqpoint{1.067573in}{4.014950in}}{\pgfqpoint{1.067573in}{4.026000in}}%
\pgfpathcurveto{\pgfqpoint{1.067573in}{4.037050in}}{\pgfqpoint{1.063182in}{4.047649in}}{\pgfqpoint{1.055369in}{4.055463in}}%
\pgfpathcurveto{\pgfqpoint{1.047555in}{4.063276in}}{\pgfqpoint{1.036956in}{4.067667in}}{\pgfqpoint{1.025906in}{4.067667in}}%
\pgfpathcurveto{\pgfqpoint{1.014856in}{4.067667in}}{\pgfqpoint{1.004257in}{4.063276in}}{\pgfqpoint{0.996443in}{4.055463in}}%
\pgfpathcurveto{\pgfqpoint{0.988630in}{4.047649in}}{\pgfqpoint{0.984239in}{4.037050in}}{\pgfqpoint{0.984239in}{4.026000in}}%
\pgfpathcurveto{\pgfqpoint{0.984239in}{4.014950in}}{\pgfqpoint{0.988630in}{4.004351in}}{\pgfqpoint{0.996443in}{3.996537in}}%
\pgfpathcurveto{\pgfqpoint{1.004257in}{3.988724in}}{\pgfqpoint{1.014856in}{3.984333in}}{\pgfqpoint{1.025906in}{3.984333in}}%
\pgfpathclose%
\pgfusepath{stroke,fill}%
\end{pgfscope}%
\begin{pgfscope}%
\pgfpathrectangle{\pgfqpoint{0.800000in}{0.528000in}}{\pgfqpoint{4.960000in}{3.696000in}}%
\pgfusepath{clip}%
\pgfsetbuttcap%
\pgfsetroundjoin%
\definecolor{currentfill}{rgb}{0.000000,0.000000,0.000000}%
\pgfsetfillcolor{currentfill}%
\pgfsetlinewidth{1.003750pt}%
\definecolor{currentstroke}{rgb}{0.000000,0.000000,0.000000}%
\pgfsetstrokecolor{currentstroke}%
\pgfsetdash{}{0pt}%
\pgfpathmoveto{\pgfqpoint{1.025906in}{3.984333in}}%
\pgfpathcurveto{\pgfqpoint{1.036956in}{3.984333in}}{\pgfqpoint{1.047555in}{3.988724in}}{\pgfqpoint{1.055369in}{3.996537in}}%
\pgfpathcurveto{\pgfqpoint{1.063182in}{4.004351in}}{\pgfqpoint{1.067573in}{4.014950in}}{\pgfqpoint{1.067573in}{4.026000in}}%
\pgfpathcurveto{\pgfqpoint{1.067573in}{4.037050in}}{\pgfqpoint{1.063182in}{4.047649in}}{\pgfqpoint{1.055369in}{4.055463in}}%
\pgfpathcurveto{\pgfqpoint{1.047555in}{4.063276in}}{\pgfqpoint{1.036956in}{4.067667in}}{\pgfqpoint{1.025906in}{4.067667in}}%
\pgfpathcurveto{\pgfqpoint{1.014856in}{4.067667in}}{\pgfqpoint{1.004257in}{4.063276in}}{\pgfqpoint{0.996443in}{4.055463in}}%
\pgfpathcurveto{\pgfqpoint{0.988630in}{4.047649in}}{\pgfqpoint{0.984239in}{4.037050in}}{\pgfqpoint{0.984239in}{4.026000in}}%
\pgfpathcurveto{\pgfqpoint{0.984239in}{4.014950in}}{\pgfqpoint{0.988630in}{4.004351in}}{\pgfqpoint{0.996443in}{3.996537in}}%
\pgfpathcurveto{\pgfqpoint{1.004257in}{3.988724in}}{\pgfqpoint{1.014856in}{3.984333in}}{\pgfqpoint{1.025906in}{3.984333in}}%
\pgfpathclose%
\pgfusepath{stroke,fill}%
\end{pgfscope}%
\begin{pgfscope}%
\pgfpathrectangle{\pgfqpoint{0.800000in}{0.528000in}}{\pgfqpoint{4.960000in}{3.696000in}}%
\pgfusepath{clip}%
\pgfsetbuttcap%
\pgfsetroundjoin%
\definecolor{currentfill}{rgb}{0.000000,0.000000,0.000000}%
\pgfsetfillcolor{currentfill}%
\pgfsetlinewidth{1.003750pt}%
\definecolor{currentstroke}{rgb}{0.000000,0.000000,0.000000}%
\pgfsetstrokecolor{currentstroke}%
\pgfsetdash{}{0pt}%
\pgfpathmoveto{\pgfqpoint{1.025906in}{3.984333in}}%
\pgfpathcurveto{\pgfqpoint{1.036956in}{3.984333in}}{\pgfqpoint{1.047555in}{3.988724in}}{\pgfqpoint{1.055369in}{3.996537in}}%
\pgfpathcurveto{\pgfqpoint{1.063182in}{4.004351in}}{\pgfqpoint{1.067573in}{4.014950in}}{\pgfqpoint{1.067573in}{4.026000in}}%
\pgfpathcurveto{\pgfqpoint{1.067573in}{4.037050in}}{\pgfqpoint{1.063182in}{4.047649in}}{\pgfqpoint{1.055369in}{4.055463in}}%
\pgfpathcurveto{\pgfqpoint{1.047555in}{4.063276in}}{\pgfqpoint{1.036956in}{4.067667in}}{\pgfqpoint{1.025906in}{4.067667in}}%
\pgfpathcurveto{\pgfqpoint{1.014856in}{4.067667in}}{\pgfqpoint{1.004257in}{4.063276in}}{\pgfqpoint{0.996443in}{4.055463in}}%
\pgfpathcurveto{\pgfqpoint{0.988630in}{4.047649in}}{\pgfqpoint{0.984239in}{4.037050in}}{\pgfqpoint{0.984239in}{4.026000in}}%
\pgfpathcurveto{\pgfqpoint{0.984239in}{4.014950in}}{\pgfqpoint{0.988630in}{4.004351in}}{\pgfqpoint{0.996443in}{3.996537in}}%
\pgfpathcurveto{\pgfqpoint{1.004257in}{3.988724in}}{\pgfqpoint{1.014856in}{3.984333in}}{\pgfqpoint{1.025906in}{3.984333in}}%
\pgfpathclose%
\pgfusepath{stroke,fill}%
\end{pgfscope}%
\begin{pgfscope}%
\pgfpathrectangle{\pgfqpoint{0.800000in}{0.528000in}}{\pgfqpoint{4.960000in}{3.696000in}}%
\pgfusepath{clip}%
\pgfsetbuttcap%
\pgfsetroundjoin%
\definecolor{currentfill}{rgb}{0.000000,0.000000,0.000000}%
\pgfsetfillcolor{currentfill}%
\pgfsetlinewidth{1.003750pt}%
\definecolor{currentstroke}{rgb}{0.000000,0.000000,0.000000}%
\pgfsetstrokecolor{currentstroke}%
\pgfsetdash{}{0pt}%
\pgfpathmoveto{\pgfqpoint{1.025906in}{3.984333in}}%
\pgfpathcurveto{\pgfqpoint{1.036956in}{3.984333in}}{\pgfqpoint{1.047555in}{3.988724in}}{\pgfqpoint{1.055369in}{3.996537in}}%
\pgfpathcurveto{\pgfqpoint{1.063182in}{4.004351in}}{\pgfqpoint{1.067573in}{4.014950in}}{\pgfqpoint{1.067573in}{4.026000in}}%
\pgfpathcurveto{\pgfqpoint{1.067573in}{4.037050in}}{\pgfqpoint{1.063182in}{4.047649in}}{\pgfqpoint{1.055369in}{4.055463in}}%
\pgfpathcurveto{\pgfqpoint{1.047555in}{4.063276in}}{\pgfqpoint{1.036956in}{4.067667in}}{\pgfqpoint{1.025906in}{4.067667in}}%
\pgfpathcurveto{\pgfqpoint{1.014856in}{4.067667in}}{\pgfqpoint{1.004257in}{4.063276in}}{\pgfqpoint{0.996443in}{4.055463in}}%
\pgfpathcurveto{\pgfqpoint{0.988630in}{4.047649in}}{\pgfqpoint{0.984239in}{4.037050in}}{\pgfqpoint{0.984239in}{4.026000in}}%
\pgfpathcurveto{\pgfqpoint{0.984239in}{4.014950in}}{\pgfqpoint{0.988630in}{4.004351in}}{\pgfqpoint{0.996443in}{3.996537in}}%
\pgfpathcurveto{\pgfqpoint{1.004257in}{3.988724in}}{\pgfqpoint{1.014856in}{3.984333in}}{\pgfqpoint{1.025906in}{3.984333in}}%
\pgfpathclose%
\pgfusepath{stroke,fill}%
\end{pgfscope}%
\begin{pgfscope}%
\pgfpathrectangle{\pgfqpoint{0.800000in}{0.528000in}}{\pgfqpoint{4.960000in}{3.696000in}}%
\pgfusepath{clip}%
\pgfsetbuttcap%
\pgfsetroundjoin%
\definecolor{currentfill}{rgb}{0.000000,0.000000,0.000000}%
\pgfsetfillcolor{currentfill}%
\pgfsetlinewidth{1.003750pt}%
\definecolor{currentstroke}{rgb}{0.000000,0.000000,0.000000}%
\pgfsetstrokecolor{currentstroke}%
\pgfsetdash{}{0pt}%
\pgfpathmoveto{\pgfqpoint{1.025906in}{3.984333in}}%
\pgfpathcurveto{\pgfqpoint{1.036956in}{3.984333in}}{\pgfqpoint{1.047555in}{3.988724in}}{\pgfqpoint{1.055369in}{3.996537in}}%
\pgfpathcurveto{\pgfqpoint{1.063182in}{4.004351in}}{\pgfqpoint{1.067573in}{4.014950in}}{\pgfqpoint{1.067573in}{4.026000in}}%
\pgfpathcurveto{\pgfqpoint{1.067573in}{4.037050in}}{\pgfqpoint{1.063182in}{4.047649in}}{\pgfqpoint{1.055369in}{4.055463in}}%
\pgfpathcurveto{\pgfqpoint{1.047555in}{4.063276in}}{\pgfqpoint{1.036956in}{4.067667in}}{\pgfqpoint{1.025906in}{4.067667in}}%
\pgfpathcurveto{\pgfqpoint{1.014856in}{4.067667in}}{\pgfqpoint{1.004257in}{4.063276in}}{\pgfqpoint{0.996443in}{4.055463in}}%
\pgfpathcurveto{\pgfqpoint{0.988630in}{4.047649in}}{\pgfqpoint{0.984239in}{4.037050in}}{\pgfqpoint{0.984239in}{4.026000in}}%
\pgfpathcurveto{\pgfqpoint{0.984239in}{4.014950in}}{\pgfqpoint{0.988630in}{4.004351in}}{\pgfqpoint{0.996443in}{3.996537in}}%
\pgfpathcurveto{\pgfqpoint{1.004257in}{3.988724in}}{\pgfqpoint{1.014856in}{3.984333in}}{\pgfqpoint{1.025906in}{3.984333in}}%
\pgfpathclose%
\pgfusepath{stroke,fill}%
\end{pgfscope}%
\begin{pgfscope}%
\pgfpathrectangle{\pgfqpoint{0.800000in}{0.528000in}}{\pgfqpoint{4.960000in}{3.696000in}}%
\pgfusepath{clip}%
\pgfsetbuttcap%
\pgfsetroundjoin%
\definecolor{currentfill}{rgb}{0.000000,0.000000,0.000000}%
\pgfsetfillcolor{currentfill}%
\pgfsetlinewidth{1.003750pt}%
\definecolor{currentstroke}{rgb}{0.000000,0.000000,0.000000}%
\pgfsetstrokecolor{currentstroke}%
\pgfsetdash{}{0pt}%
\pgfpathmoveto{\pgfqpoint{1.025906in}{3.984333in}}%
\pgfpathcurveto{\pgfqpoint{1.036956in}{3.984333in}}{\pgfqpoint{1.047555in}{3.988724in}}{\pgfqpoint{1.055369in}{3.996537in}}%
\pgfpathcurveto{\pgfqpoint{1.063182in}{4.004351in}}{\pgfqpoint{1.067573in}{4.014950in}}{\pgfqpoint{1.067573in}{4.026000in}}%
\pgfpathcurveto{\pgfqpoint{1.067573in}{4.037050in}}{\pgfqpoint{1.063182in}{4.047649in}}{\pgfqpoint{1.055369in}{4.055463in}}%
\pgfpathcurveto{\pgfqpoint{1.047555in}{4.063276in}}{\pgfqpoint{1.036956in}{4.067667in}}{\pgfqpoint{1.025906in}{4.067667in}}%
\pgfpathcurveto{\pgfqpoint{1.014856in}{4.067667in}}{\pgfqpoint{1.004257in}{4.063276in}}{\pgfqpoint{0.996443in}{4.055463in}}%
\pgfpathcurveto{\pgfqpoint{0.988630in}{4.047649in}}{\pgfqpoint{0.984239in}{4.037050in}}{\pgfqpoint{0.984239in}{4.026000in}}%
\pgfpathcurveto{\pgfqpoint{0.984239in}{4.014950in}}{\pgfqpoint{0.988630in}{4.004351in}}{\pgfqpoint{0.996443in}{3.996537in}}%
\pgfpathcurveto{\pgfqpoint{1.004257in}{3.988724in}}{\pgfqpoint{1.014856in}{3.984333in}}{\pgfqpoint{1.025906in}{3.984333in}}%
\pgfpathclose%
\pgfusepath{stroke,fill}%
\end{pgfscope}%
\begin{pgfscope}%
\pgfpathrectangle{\pgfqpoint{0.800000in}{0.528000in}}{\pgfqpoint{4.960000in}{3.696000in}}%
\pgfusepath{clip}%
\pgfsetbuttcap%
\pgfsetroundjoin%
\definecolor{currentfill}{rgb}{0.000000,0.000000,0.000000}%
\pgfsetfillcolor{currentfill}%
\pgfsetlinewidth{1.003750pt}%
\definecolor{currentstroke}{rgb}{0.000000,0.000000,0.000000}%
\pgfsetstrokecolor{currentstroke}%
\pgfsetdash{}{0pt}%
\pgfpathmoveto{\pgfqpoint{1.025906in}{3.984333in}}%
\pgfpathcurveto{\pgfqpoint{1.036956in}{3.984333in}}{\pgfqpoint{1.047555in}{3.988724in}}{\pgfqpoint{1.055369in}{3.996537in}}%
\pgfpathcurveto{\pgfqpoint{1.063182in}{4.004351in}}{\pgfqpoint{1.067573in}{4.014950in}}{\pgfqpoint{1.067573in}{4.026000in}}%
\pgfpathcurveto{\pgfqpoint{1.067573in}{4.037050in}}{\pgfqpoint{1.063182in}{4.047649in}}{\pgfqpoint{1.055369in}{4.055463in}}%
\pgfpathcurveto{\pgfqpoint{1.047555in}{4.063276in}}{\pgfqpoint{1.036956in}{4.067667in}}{\pgfqpoint{1.025906in}{4.067667in}}%
\pgfpathcurveto{\pgfqpoint{1.014856in}{4.067667in}}{\pgfqpoint{1.004257in}{4.063276in}}{\pgfqpoint{0.996443in}{4.055463in}}%
\pgfpathcurveto{\pgfqpoint{0.988630in}{4.047649in}}{\pgfqpoint{0.984239in}{4.037050in}}{\pgfqpoint{0.984239in}{4.026000in}}%
\pgfpathcurveto{\pgfqpoint{0.984239in}{4.014950in}}{\pgfqpoint{0.988630in}{4.004351in}}{\pgfqpoint{0.996443in}{3.996537in}}%
\pgfpathcurveto{\pgfqpoint{1.004257in}{3.988724in}}{\pgfqpoint{1.014856in}{3.984333in}}{\pgfqpoint{1.025906in}{3.984333in}}%
\pgfpathclose%
\pgfusepath{stroke,fill}%
\end{pgfscope}%
\begin{pgfscope}%
\pgfpathrectangle{\pgfqpoint{0.800000in}{0.528000in}}{\pgfqpoint{4.960000in}{3.696000in}}%
\pgfusepath{clip}%
\pgfsetbuttcap%
\pgfsetroundjoin%
\definecolor{currentfill}{rgb}{0.000000,0.000000,0.000000}%
\pgfsetfillcolor{currentfill}%
\pgfsetlinewidth{1.003750pt}%
\definecolor{currentstroke}{rgb}{0.000000,0.000000,0.000000}%
\pgfsetstrokecolor{currentstroke}%
\pgfsetdash{}{0pt}%
\pgfpathmoveto{\pgfqpoint{1.025906in}{3.984333in}}%
\pgfpathcurveto{\pgfqpoint{1.036956in}{3.984333in}}{\pgfqpoint{1.047555in}{3.988724in}}{\pgfqpoint{1.055369in}{3.996537in}}%
\pgfpathcurveto{\pgfqpoint{1.063182in}{4.004351in}}{\pgfqpoint{1.067573in}{4.014950in}}{\pgfqpoint{1.067573in}{4.026000in}}%
\pgfpathcurveto{\pgfqpoint{1.067573in}{4.037050in}}{\pgfqpoint{1.063182in}{4.047649in}}{\pgfqpoint{1.055369in}{4.055463in}}%
\pgfpathcurveto{\pgfqpoint{1.047555in}{4.063276in}}{\pgfqpoint{1.036956in}{4.067667in}}{\pgfqpoint{1.025906in}{4.067667in}}%
\pgfpathcurveto{\pgfqpoint{1.014856in}{4.067667in}}{\pgfqpoint{1.004257in}{4.063276in}}{\pgfqpoint{0.996443in}{4.055463in}}%
\pgfpathcurveto{\pgfqpoint{0.988630in}{4.047649in}}{\pgfqpoint{0.984239in}{4.037050in}}{\pgfqpoint{0.984239in}{4.026000in}}%
\pgfpathcurveto{\pgfqpoint{0.984239in}{4.014950in}}{\pgfqpoint{0.988630in}{4.004351in}}{\pgfqpoint{0.996443in}{3.996537in}}%
\pgfpathcurveto{\pgfqpoint{1.004257in}{3.988724in}}{\pgfqpoint{1.014856in}{3.984333in}}{\pgfqpoint{1.025906in}{3.984333in}}%
\pgfpathclose%
\pgfusepath{stroke,fill}%
\end{pgfscope}%
\begin{pgfscope}%
\pgfpathrectangle{\pgfqpoint{0.800000in}{0.528000in}}{\pgfqpoint{4.960000in}{3.696000in}}%
\pgfusepath{clip}%
\pgfsetbuttcap%
\pgfsetroundjoin%
\definecolor{currentfill}{rgb}{0.000000,0.000000,0.000000}%
\pgfsetfillcolor{currentfill}%
\pgfsetlinewidth{1.003750pt}%
\definecolor{currentstroke}{rgb}{0.000000,0.000000,0.000000}%
\pgfsetstrokecolor{currentstroke}%
\pgfsetdash{}{0pt}%
\pgfpathmoveto{\pgfqpoint{1.025906in}{3.984333in}}%
\pgfpathcurveto{\pgfqpoint{1.036956in}{3.984333in}}{\pgfqpoint{1.047555in}{3.988724in}}{\pgfqpoint{1.055369in}{3.996537in}}%
\pgfpathcurveto{\pgfqpoint{1.063182in}{4.004351in}}{\pgfqpoint{1.067573in}{4.014950in}}{\pgfqpoint{1.067573in}{4.026000in}}%
\pgfpathcurveto{\pgfqpoint{1.067573in}{4.037050in}}{\pgfqpoint{1.063182in}{4.047649in}}{\pgfqpoint{1.055369in}{4.055463in}}%
\pgfpathcurveto{\pgfqpoint{1.047555in}{4.063276in}}{\pgfqpoint{1.036956in}{4.067667in}}{\pgfqpoint{1.025906in}{4.067667in}}%
\pgfpathcurveto{\pgfqpoint{1.014856in}{4.067667in}}{\pgfqpoint{1.004257in}{4.063276in}}{\pgfqpoint{0.996443in}{4.055463in}}%
\pgfpathcurveto{\pgfqpoint{0.988630in}{4.047649in}}{\pgfqpoint{0.984239in}{4.037050in}}{\pgfqpoint{0.984239in}{4.026000in}}%
\pgfpathcurveto{\pgfqpoint{0.984239in}{4.014950in}}{\pgfqpoint{0.988630in}{4.004351in}}{\pgfqpoint{0.996443in}{3.996537in}}%
\pgfpathcurveto{\pgfqpoint{1.004257in}{3.988724in}}{\pgfqpoint{1.014856in}{3.984333in}}{\pgfqpoint{1.025906in}{3.984333in}}%
\pgfpathclose%
\pgfusepath{stroke,fill}%
\end{pgfscope}%
\begin{pgfscope}%
\pgfpathrectangle{\pgfqpoint{0.800000in}{0.528000in}}{\pgfqpoint{4.960000in}{3.696000in}}%
\pgfusepath{clip}%
\pgfsetbuttcap%
\pgfsetroundjoin%
\definecolor{currentfill}{rgb}{0.000000,0.000000,0.000000}%
\pgfsetfillcolor{currentfill}%
\pgfsetlinewidth{1.003750pt}%
\definecolor{currentstroke}{rgb}{0.000000,0.000000,0.000000}%
\pgfsetstrokecolor{currentstroke}%
\pgfsetdash{}{0pt}%
\pgfpathmoveto{\pgfqpoint{1.025906in}{3.984333in}}%
\pgfpathcurveto{\pgfqpoint{1.036956in}{3.984333in}}{\pgfqpoint{1.047555in}{3.988724in}}{\pgfqpoint{1.055369in}{3.996537in}}%
\pgfpathcurveto{\pgfqpoint{1.063182in}{4.004351in}}{\pgfqpoint{1.067573in}{4.014950in}}{\pgfqpoint{1.067573in}{4.026000in}}%
\pgfpathcurveto{\pgfqpoint{1.067573in}{4.037050in}}{\pgfqpoint{1.063182in}{4.047649in}}{\pgfqpoint{1.055369in}{4.055463in}}%
\pgfpathcurveto{\pgfqpoint{1.047555in}{4.063276in}}{\pgfqpoint{1.036956in}{4.067667in}}{\pgfqpoint{1.025906in}{4.067667in}}%
\pgfpathcurveto{\pgfqpoint{1.014856in}{4.067667in}}{\pgfqpoint{1.004257in}{4.063276in}}{\pgfqpoint{0.996443in}{4.055463in}}%
\pgfpathcurveto{\pgfqpoint{0.988630in}{4.047649in}}{\pgfqpoint{0.984239in}{4.037050in}}{\pgfqpoint{0.984239in}{4.026000in}}%
\pgfpathcurveto{\pgfqpoint{0.984239in}{4.014950in}}{\pgfqpoint{0.988630in}{4.004351in}}{\pgfqpoint{0.996443in}{3.996537in}}%
\pgfpathcurveto{\pgfqpoint{1.004257in}{3.988724in}}{\pgfqpoint{1.014856in}{3.984333in}}{\pgfqpoint{1.025906in}{3.984333in}}%
\pgfpathclose%
\pgfusepath{stroke,fill}%
\end{pgfscope}%
\begin{pgfscope}%
\pgfpathrectangle{\pgfqpoint{0.800000in}{0.528000in}}{\pgfqpoint{4.960000in}{3.696000in}}%
\pgfusepath{clip}%
\pgfsetbuttcap%
\pgfsetroundjoin%
\definecolor{currentfill}{rgb}{0.000000,0.000000,0.000000}%
\pgfsetfillcolor{currentfill}%
\pgfsetlinewidth{1.003750pt}%
\definecolor{currentstroke}{rgb}{0.000000,0.000000,0.000000}%
\pgfsetstrokecolor{currentstroke}%
\pgfsetdash{}{0pt}%
\pgfpathmoveto{\pgfqpoint{1.025906in}{3.984333in}}%
\pgfpathcurveto{\pgfqpoint{1.036956in}{3.984333in}}{\pgfqpoint{1.047555in}{3.988724in}}{\pgfqpoint{1.055369in}{3.996537in}}%
\pgfpathcurveto{\pgfqpoint{1.063182in}{4.004351in}}{\pgfqpoint{1.067573in}{4.014950in}}{\pgfqpoint{1.067573in}{4.026000in}}%
\pgfpathcurveto{\pgfqpoint{1.067573in}{4.037050in}}{\pgfqpoint{1.063182in}{4.047649in}}{\pgfqpoint{1.055369in}{4.055463in}}%
\pgfpathcurveto{\pgfqpoint{1.047555in}{4.063276in}}{\pgfqpoint{1.036956in}{4.067667in}}{\pgfqpoint{1.025906in}{4.067667in}}%
\pgfpathcurveto{\pgfqpoint{1.014856in}{4.067667in}}{\pgfqpoint{1.004257in}{4.063276in}}{\pgfqpoint{0.996443in}{4.055463in}}%
\pgfpathcurveto{\pgfqpoint{0.988630in}{4.047649in}}{\pgfqpoint{0.984239in}{4.037050in}}{\pgfqpoint{0.984239in}{4.026000in}}%
\pgfpathcurveto{\pgfqpoint{0.984239in}{4.014950in}}{\pgfqpoint{0.988630in}{4.004351in}}{\pgfqpoint{0.996443in}{3.996537in}}%
\pgfpathcurveto{\pgfqpoint{1.004257in}{3.988724in}}{\pgfqpoint{1.014856in}{3.984333in}}{\pgfqpoint{1.025906in}{3.984333in}}%
\pgfpathclose%
\pgfusepath{stroke,fill}%
\end{pgfscope}%
\begin{pgfscope}%
\pgfpathrectangle{\pgfqpoint{0.800000in}{0.528000in}}{\pgfqpoint{4.960000in}{3.696000in}}%
\pgfusepath{clip}%
\pgfsetbuttcap%
\pgfsetroundjoin%
\definecolor{currentfill}{rgb}{0.000000,0.000000,0.000000}%
\pgfsetfillcolor{currentfill}%
\pgfsetlinewidth{1.003750pt}%
\definecolor{currentstroke}{rgb}{0.000000,0.000000,0.000000}%
\pgfsetstrokecolor{currentstroke}%
\pgfsetdash{}{0pt}%
\pgfpathmoveto{\pgfqpoint{1.025906in}{3.984333in}}%
\pgfpathcurveto{\pgfqpoint{1.036956in}{3.984333in}}{\pgfqpoint{1.047555in}{3.988724in}}{\pgfqpoint{1.055369in}{3.996537in}}%
\pgfpathcurveto{\pgfqpoint{1.063182in}{4.004351in}}{\pgfqpoint{1.067573in}{4.014950in}}{\pgfqpoint{1.067573in}{4.026000in}}%
\pgfpathcurveto{\pgfqpoint{1.067573in}{4.037050in}}{\pgfqpoint{1.063182in}{4.047649in}}{\pgfqpoint{1.055369in}{4.055463in}}%
\pgfpathcurveto{\pgfqpoint{1.047555in}{4.063276in}}{\pgfqpoint{1.036956in}{4.067667in}}{\pgfqpoint{1.025906in}{4.067667in}}%
\pgfpathcurveto{\pgfqpoint{1.014856in}{4.067667in}}{\pgfqpoint{1.004257in}{4.063276in}}{\pgfqpoint{0.996443in}{4.055463in}}%
\pgfpathcurveto{\pgfqpoint{0.988630in}{4.047649in}}{\pgfqpoint{0.984239in}{4.037050in}}{\pgfqpoint{0.984239in}{4.026000in}}%
\pgfpathcurveto{\pgfqpoint{0.984239in}{4.014950in}}{\pgfqpoint{0.988630in}{4.004351in}}{\pgfqpoint{0.996443in}{3.996537in}}%
\pgfpathcurveto{\pgfqpoint{1.004257in}{3.988724in}}{\pgfqpoint{1.014856in}{3.984333in}}{\pgfqpoint{1.025906in}{3.984333in}}%
\pgfpathclose%
\pgfusepath{stroke,fill}%
\end{pgfscope}%
\begin{pgfscope}%
\pgfpathrectangle{\pgfqpoint{0.800000in}{0.528000in}}{\pgfqpoint{4.960000in}{3.696000in}}%
\pgfusepath{clip}%
\pgfsetbuttcap%
\pgfsetroundjoin%
\definecolor{currentfill}{rgb}{0.000000,0.000000,0.000000}%
\pgfsetfillcolor{currentfill}%
\pgfsetlinewidth{1.003750pt}%
\definecolor{currentstroke}{rgb}{0.000000,0.000000,0.000000}%
\pgfsetstrokecolor{currentstroke}%
\pgfsetdash{}{0pt}%
\pgfpathmoveto{\pgfqpoint{1.025906in}{3.984333in}}%
\pgfpathcurveto{\pgfqpoint{1.036956in}{3.984333in}}{\pgfqpoint{1.047555in}{3.988724in}}{\pgfqpoint{1.055369in}{3.996537in}}%
\pgfpathcurveto{\pgfqpoint{1.063182in}{4.004351in}}{\pgfqpoint{1.067573in}{4.014950in}}{\pgfqpoint{1.067573in}{4.026000in}}%
\pgfpathcurveto{\pgfqpoint{1.067573in}{4.037050in}}{\pgfqpoint{1.063182in}{4.047649in}}{\pgfqpoint{1.055369in}{4.055463in}}%
\pgfpathcurveto{\pgfqpoint{1.047555in}{4.063276in}}{\pgfqpoint{1.036956in}{4.067667in}}{\pgfqpoint{1.025906in}{4.067667in}}%
\pgfpathcurveto{\pgfqpoint{1.014856in}{4.067667in}}{\pgfqpoint{1.004257in}{4.063276in}}{\pgfqpoint{0.996443in}{4.055463in}}%
\pgfpathcurveto{\pgfqpoint{0.988630in}{4.047649in}}{\pgfqpoint{0.984239in}{4.037050in}}{\pgfqpoint{0.984239in}{4.026000in}}%
\pgfpathcurveto{\pgfqpoint{0.984239in}{4.014950in}}{\pgfqpoint{0.988630in}{4.004351in}}{\pgfqpoint{0.996443in}{3.996537in}}%
\pgfpathcurveto{\pgfqpoint{1.004257in}{3.988724in}}{\pgfqpoint{1.014856in}{3.984333in}}{\pgfqpoint{1.025906in}{3.984333in}}%
\pgfpathclose%
\pgfusepath{stroke,fill}%
\end{pgfscope}%
\begin{pgfscope}%
\pgfpathrectangle{\pgfqpoint{0.800000in}{0.528000in}}{\pgfqpoint{4.960000in}{3.696000in}}%
\pgfusepath{clip}%
\pgfsetbuttcap%
\pgfsetroundjoin%
\definecolor{currentfill}{rgb}{0.000000,0.000000,0.000000}%
\pgfsetfillcolor{currentfill}%
\pgfsetlinewidth{1.003750pt}%
\definecolor{currentstroke}{rgb}{0.000000,0.000000,0.000000}%
\pgfsetstrokecolor{currentstroke}%
\pgfsetdash{}{0pt}%
\pgfpathmoveto{\pgfqpoint{1.025906in}{3.984333in}}%
\pgfpathcurveto{\pgfqpoint{1.036956in}{3.984333in}}{\pgfqpoint{1.047555in}{3.988724in}}{\pgfqpoint{1.055369in}{3.996537in}}%
\pgfpathcurveto{\pgfqpoint{1.063182in}{4.004351in}}{\pgfqpoint{1.067573in}{4.014950in}}{\pgfqpoint{1.067573in}{4.026000in}}%
\pgfpathcurveto{\pgfqpoint{1.067573in}{4.037050in}}{\pgfqpoint{1.063182in}{4.047649in}}{\pgfqpoint{1.055369in}{4.055463in}}%
\pgfpathcurveto{\pgfqpoint{1.047555in}{4.063276in}}{\pgfqpoint{1.036956in}{4.067667in}}{\pgfqpoint{1.025906in}{4.067667in}}%
\pgfpathcurveto{\pgfqpoint{1.014856in}{4.067667in}}{\pgfqpoint{1.004257in}{4.063276in}}{\pgfqpoint{0.996443in}{4.055463in}}%
\pgfpathcurveto{\pgfqpoint{0.988630in}{4.047649in}}{\pgfqpoint{0.984239in}{4.037050in}}{\pgfqpoint{0.984239in}{4.026000in}}%
\pgfpathcurveto{\pgfqpoint{0.984239in}{4.014950in}}{\pgfqpoint{0.988630in}{4.004351in}}{\pgfqpoint{0.996443in}{3.996537in}}%
\pgfpathcurveto{\pgfqpoint{1.004257in}{3.988724in}}{\pgfqpoint{1.014856in}{3.984333in}}{\pgfqpoint{1.025906in}{3.984333in}}%
\pgfpathclose%
\pgfusepath{stroke,fill}%
\end{pgfscope}%
\begin{pgfscope}%
\pgfpathrectangle{\pgfqpoint{0.800000in}{0.528000in}}{\pgfqpoint{4.960000in}{3.696000in}}%
\pgfusepath{clip}%
\pgfsetbuttcap%
\pgfsetroundjoin%
\definecolor{currentfill}{rgb}{0.000000,0.000000,0.000000}%
\pgfsetfillcolor{currentfill}%
\pgfsetlinewidth{1.003750pt}%
\definecolor{currentstroke}{rgb}{0.000000,0.000000,0.000000}%
\pgfsetstrokecolor{currentstroke}%
\pgfsetdash{}{0pt}%
\pgfpathmoveto{\pgfqpoint{1.025906in}{3.984333in}}%
\pgfpathcurveto{\pgfqpoint{1.036956in}{3.984333in}}{\pgfqpoint{1.047555in}{3.988724in}}{\pgfqpoint{1.055369in}{3.996537in}}%
\pgfpathcurveto{\pgfqpoint{1.063182in}{4.004351in}}{\pgfqpoint{1.067573in}{4.014950in}}{\pgfqpoint{1.067573in}{4.026000in}}%
\pgfpathcurveto{\pgfqpoint{1.067573in}{4.037050in}}{\pgfqpoint{1.063182in}{4.047649in}}{\pgfqpoint{1.055369in}{4.055463in}}%
\pgfpathcurveto{\pgfqpoint{1.047555in}{4.063276in}}{\pgfqpoint{1.036956in}{4.067667in}}{\pgfqpoint{1.025906in}{4.067667in}}%
\pgfpathcurveto{\pgfqpoint{1.014856in}{4.067667in}}{\pgfqpoint{1.004257in}{4.063276in}}{\pgfqpoint{0.996443in}{4.055463in}}%
\pgfpathcurveto{\pgfqpoint{0.988630in}{4.047649in}}{\pgfqpoint{0.984239in}{4.037050in}}{\pgfqpoint{0.984239in}{4.026000in}}%
\pgfpathcurveto{\pgfqpoint{0.984239in}{4.014950in}}{\pgfqpoint{0.988630in}{4.004351in}}{\pgfqpoint{0.996443in}{3.996537in}}%
\pgfpathcurveto{\pgfqpoint{1.004257in}{3.988724in}}{\pgfqpoint{1.014856in}{3.984333in}}{\pgfqpoint{1.025906in}{3.984333in}}%
\pgfpathclose%
\pgfusepath{stroke,fill}%
\end{pgfscope}%
\begin{pgfscope}%
\pgfpathrectangle{\pgfqpoint{0.800000in}{0.528000in}}{\pgfqpoint{4.960000in}{3.696000in}}%
\pgfusepath{clip}%
\pgfsetbuttcap%
\pgfsetroundjoin%
\definecolor{currentfill}{rgb}{0.000000,0.000000,0.000000}%
\pgfsetfillcolor{currentfill}%
\pgfsetlinewidth{1.003750pt}%
\definecolor{currentstroke}{rgb}{0.000000,0.000000,0.000000}%
\pgfsetstrokecolor{currentstroke}%
\pgfsetdash{}{0pt}%
\pgfpathmoveto{\pgfqpoint{1.025906in}{3.984333in}}%
\pgfpathcurveto{\pgfqpoint{1.036956in}{3.984333in}}{\pgfqpoint{1.047555in}{3.988724in}}{\pgfqpoint{1.055369in}{3.996537in}}%
\pgfpathcurveto{\pgfqpoint{1.063182in}{4.004351in}}{\pgfqpoint{1.067573in}{4.014950in}}{\pgfqpoint{1.067573in}{4.026000in}}%
\pgfpathcurveto{\pgfqpoint{1.067573in}{4.037050in}}{\pgfqpoint{1.063182in}{4.047649in}}{\pgfqpoint{1.055369in}{4.055463in}}%
\pgfpathcurveto{\pgfqpoint{1.047555in}{4.063276in}}{\pgfqpoint{1.036956in}{4.067667in}}{\pgfqpoint{1.025906in}{4.067667in}}%
\pgfpathcurveto{\pgfqpoint{1.014856in}{4.067667in}}{\pgfqpoint{1.004257in}{4.063276in}}{\pgfqpoint{0.996443in}{4.055463in}}%
\pgfpathcurveto{\pgfqpoint{0.988630in}{4.047649in}}{\pgfqpoint{0.984239in}{4.037050in}}{\pgfqpoint{0.984239in}{4.026000in}}%
\pgfpathcurveto{\pgfqpoint{0.984239in}{4.014950in}}{\pgfqpoint{0.988630in}{4.004351in}}{\pgfqpoint{0.996443in}{3.996537in}}%
\pgfpathcurveto{\pgfqpoint{1.004257in}{3.988724in}}{\pgfqpoint{1.014856in}{3.984333in}}{\pgfqpoint{1.025906in}{3.984333in}}%
\pgfpathclose%
\pgfusepath{stroke,fill}%
\end{pgfscope}%
\begin{pgfscope}%
\pgfpathrectangle{\pgfqpoint{0.800000in}{0.528000in}}{\pgfqpoint{4.960000in}{3.696000in}}%
\pgfusepath{clip}%
\pgfsetbuttcap%
\pgfsetroundjoin%
\definecolor{currentfill}{rgb}{0.000000,0.000000,0.000000}%
\pgfsetfillcolor{currentfill}%
\pgfsetlinewidth{1.003750pt}%
\definecolor{currentstroke}{rgb}{0.000000,0.000000,0.000000}%
\pgfsetstrokecolor{currentstroke}%
\pgfsetdash{}{0pt}%
\pgfpathmoveto{\pgfqpoint{1.025906in}{3.984333in}}%
\pgfpathcurveto{\pgfqpoint{1.036956in}{3.984333in}}{\pgfqpoint{1.047555in}{3.988724in}}{\pgfqpoint{1.055369in}{3.996537in}}%
\pgfpathcurveto{\pgfqpoint{1.063182in}{4.004351in}}{\pgfqpoint{1.067573in}{4.014950in}}{\pgfqpoint{1.067573in}{4.026000in}}%
\pgfpathcurveto{\pgfqpoint{1.067573in}{4.037050in}}{\pgfqpoint{1.063182in}{4.047649in}}{\pgfqpoint{1.055369in}{4.055463in}}%
\pgfpathcurveto{\pgfqpoint{1.047555in}{4.063276in}}{\pgfqpoint{1.036956in}{4.067667in}}{\pgfqpoint{1.025906in}{4.067667in}}%
\pgfpathcurveto{\pgfqpoint{1.014856in}{4.067667in}}{\pgfqpoint{1.004257in}{4.063276in}}{\pgfqpoint{0.996443in}{4.055463in}}%
\pgfpathcurveto{\pgfqpoint{0.988630in}{4.047649in}}{\pgfqpoint{0.984239in}{4.037050in}}{\pgfqpoint{0.984239in}{4.026000in}}%
\pgfpathcurveto{\pgfqpoint{0.984239in}{4.014950in}}{\pgfqpoint{0.988630in}{4.004351in}}{\pgfqpoint{0.996443in}{3.996537in}}%
\pgfpathcurveto{\pgfqpoint{1.004257in}{3.988724in}}{\pgfqpoint{1.014856in}{3.984333in}}{\pgfqpoint{1.025906in}{3.984333in}}%
\pgfpathclose%
\pgfusepath{stroke,fill}%
\end{pgfscope}%
\begin{pgfscope}%
\pgfpathrectangle{\pgfqpoint{0.800000in}{0.528000in}}{\pgfqpoint{4.960000in}{3.696000in}}%
\pgfusepath{clip}%
\pgfsetbuttcap%
\pgfsetroundjoin%
\definecolor{currentfill}{rgb}{0.000000,0.000000,0.000000}%
\pgfsetfillcolor{currentfill}%
\pgfsetlinewidth{1.003750pt}%
\definecolor{currentstroke}{rgb}{0.000000,0.000000,0.000000}%
\pgfsetstrokecolor{currentstroke}%
\pgfsetdash{}{0pt}%
\pgfpathmoveto{\pgfqpoint{1.025906in}{3.984333in}}%
\pgfpathcurveto{\pgfqpoint{1.036956in}{3.984333in}}{\pgfqpoint{1.047555in}{3.988724in}}{\pgfqpoint{1.055369in}{3.996537in}}%
\pgfpathcurveto{\pgfqpoint{1.063182in}{4.004351in}}{\pgfqpoint{1.067573in}{4.014950in}}{\pgfqpoint{1.067573in}{4.026000in}}%
\pgfpathcurveto{\pgfqpoint{1.067573in}{4.037050in}}{\pgfqpoint{1.063182in}{4.047649in}}{\pgfqpoint{1.055369in}{4.055463in}}%
\pgfpathcurveto{\pgfqpoint{1.047555in}{4.063276in}}{\pgfqpoint{1.036956in}{4.067667in}}{\pgfqpoint{1.025906in}{4.067667in}}%
\pgfpathcurveto{\pgfqpoint{1.014856in}{4.067667in}}{\pgfqpoint{1.004257in}{4.063276in}}{\pgfqpoint{0.996443in}{4.055463in}}%
\pgfpathcurveto{\pgfqpoint{0.988630in}{4.047649in}}{\pgfqpoint{0.984239in}{4.037050in}}{\pgfqpoint{0.984239in}{4.026000in}}%
\pgfpathcurveto{\pgfqpoint{0.984239in}{4.014950in}}{\pgfqpoint{0.988630in}{4.004351in}}{\pgfqpoint{0.996443in}{3.996537in}}%
\pgfpathcurveto{\pgfqpoint{1.004257in}{3.988724in}}{\pgfqpoint{1.014856in}{3.984333in}}{\pgfqpoint{1.025906in}{3.984333in}}%
\pgfpathclose%
\pgfusepath{stroke,fill}%
\end{pgfscope}%
\begin{pgfscope}%
\pgfpathrectangle{\pgfqpoint{0.800000in}{0.528000in}}{\pgfqpoint{4.960000in}{3.696000in}}%
\pgfusepath{clip}%
\pgfsetbuttcap%
\pgfsetroundjoin%
\definecolor{currentfill}{rgb}{0.000000,0.000000,0.000000}%
\pgfsetfillcolor{currentfill}%
\pgfsetlinewidth{1.003750pt}%
\definecolor{currentstroke}{rgb}{0.000000,0.000000,0.000000}%
\pgfsetstrokecolor{currentstroke}%
\pgfsetdash{}{0pt}%
\pgfpathmoveto{\pgfqpoint{1.025906in}{3.984333in}}%
\pgfpathcurveto{\pgfqpoint{1.036956in}{3.984333in}}{\pgfqpoint{1.047555in}{3.988724in}}{\pgfqpoint{1.055369in}{3.996537in}}%
\pgfpathcurveto{\pgfqpoint{1.063182in}{4.004351in}}{\pgfqpoint{1.067573in}{4.014950in}}{\pgfqpoint{1.067573in}{4.026000in}}%
\pgfpathcurveto{\pgfqpoint{1.067573in}{4.037050in}}{\pgfqpoint{1.063182in}{4.047649in}}{\pgfqpoint{1.055369in}{4.055463in}}%
\pgfpathcurveto{\pgfqpoint{1.047555in}{4.063276in}}{\pgfqpoint{1.036956in}{4.067667in}}{\pgfqpoint{1.025906in}{4.067667in}}%
\pgfpathcurveto{\pgfqpoint{1.014856in}{4.067667in}}{\pgfqpoint{1.004257in}{4.063276in}}{\pgfqpoint{0.996443in}{4.055463in}}%
\pgfpathcurveto{\pgfqpoint{0.988630in}{4.047649in}}{\pgfqpoint{0.984239in}{4.037050in}}{\pgfqpoint{0.984239in}{4.026000in}}%
\pgfpathcurveto{\pgfqpoint{0.984239in}{4.014950in}}{\pgfqpoint{0.988630in}{4.004351in}}{\pgfqpoint{0.996443in}{3.996537in}}%
\pgfpathcurveto{\pgfqpoint{1.004257in}{3.988724in}}{\pgfqpoint{1.014856in}{3.984333in}}{\pgfqpoint{1.025906in}{3.984333in}}%
\pgfpathclose%
\pgfusepath{stroke,fill}%
\end{pgfscope}%
\begin{pgfscope}%
\pgfpathrectangle{\pgfqpoint{0.800000in}{0.528000in}}{\pgfqpoint{4.960000in}{3.696000in}}%
\pgfusepath{clip}%
\pgfsetbuttcap%
\pgfsetroundjoin%
\definecolor{currentfill}{rgb}{0.000000,0.000000,0.000000}%
\pgfsetfillcolor{currentfill}%
\pgfsetlinewidth{1.003750pt}%
\definecolor{currentstroke}{rgb}{0.000000,0.000000,0.000000}%
\pgfsetstrokecolor{currentstroke}%
\pgfsetdash{}{0pt}%
\pgfpathmoveto{\pgfqpoint{2.518786in}{3.984333in}}%
\pgfpathcurveto{\pgfqpoint{2.529836in}{3.984333in}}{\pgfqpoint{2.540435in}{3.988724in}}{\pgfqpoint{2.548249in}{3.996537in}}%
\pgfpathcurveto{\pgfqpoint{2.556062in}{4.004351in}}{\pgfqpoint{2.560452in}{4.014950in}}{\pgfqpoint{2.560452in}{4.026000in}}%
\pgfpathcurveto{\pgfqpoint{2.560452in}{4.037050in}}{\pgfqpoint{2.556062in}{4.047649in}}{\pgfqpoint{2.548249in}{4.055463in}}%
\pgfpathcurveto{\pgfqpoint{2.540435in}{4.063276in}}{\pgfqpoint{2.529836in}{4.067667in}}{\pgfqpoint{2.518786in}{4.067667in}}%
\pgfpathcurveto{\pgfqpoint{2.507736in}{4.067667in}}{\pgfqpoint{2.497137in}{4.063276in}}{\pgfqpoint{2.489323in}{4.055463in}}%
\pgfpathcurveto{\pgfqpoint{2.481509in}{4.047649in}}{\pgfqpoint{2.477119in}{4.037050in}}{\pgfqpoint{2.477119in}{4.026000in}}%
\pgfpathcurveto{\pgfqpoint{2.477119in}{4.014950in}}{\pgfqpoint{2.481509in}{4.004351in}}{\pgfqpoint{2.489323in}{3.996537in}}%
\pgfpathcurveto{\pgfqpoint{2.497137in}{3.988724in}}{\pgfqpoint{2.507736in}{3.984333in}}{\pgfqpoint{2.518786in}{3.984333in}}%
\pgfpathclose%
\pgfusepath{stroke,fill}%
\end{pgfscope}%
\begin{pgfscope}%
\pgfpathrectangle{\pgfqpoint{0.800000in}{0.528000in}}{\pgfqpoint{4.960000in}{3.696000in}}%
\pgfusepath{clip}%
\pgfsetbuttcap%
\pgfsetroundjoin%
\definecolor{currentfill}{rgb}{0.000000,0.000000,0.000000}%
\pgfsetfillcolor{currentfill}%
\pgfsetlinewidth{1.003750pt}%
\definecolor{currentstroke}{rgb}{0.000000,0.000000,0.000000}%
\pgfsetstrokecolor{currentstroke}%
\pgfsetdash{}{0pt}%
\pgfpathmoveto{\pgfqpoint{2.518786in}{3.984333in}}%
\pgfpathcurveto{\pgfqpoint{2.529836in}{3.984333in}}{\pgfqpoint{2.540435in}{3.988724in}}{\pgfqpoint{2.548249in}{3.996537in}}%
\pgfpathcurveto{\pgfqpoint{2.556062in}{4.004351in}}{\pgfqpoint{2.560452in}{4.014950in}}{\pgfqpoint{2.560452in}{4.026000in}}%
\pgfpathcurveto{\pgfqpoint{2.560452in}{4.037050in}}{\pgfqpoint{2.556062in}{4.047649in}}{\pgfqpoint{2.548249in}{4.055463in}}%
\pgfpathcurveto{\pgfqpoint{2.540435in}{4.063276in}}{\pgfqpoint{2.529836in}{4.067667in}}{\pgfqpoint{2.518786in}{4.067667in}}%
\pgfpathcurveto{\pgfqpoint{2.507736in}{4.067667in}}{\pgfqpoint{2.497137in}{4.063276in}}{\pgfqpoint{2.489323in}{4.055463in}}%
\pgfpathcurveto{\pgfqpoint{2.481509in}{4.047649in}}{\pgfqpoint{2.477119in}{4.037050in}}{\pgfqpoint{2.477119in}{4.026000in}}%
\pgfpathcurveto{\pgfqpoint{2.477119in}{4.014950in}}{\pgfqpoint{2.481509in}{4.004351in}}{\pgfqpoint{2.489323in}{3.996537in}}%
\pgfpathcurveto{\pgfqpoint{2.497137in}{3.988724in}}{\pgfqpoint{2.507736in}{3.984333in}}{\pgfqpoint{2.518786in}{3.984333in}}%
\pgfpathclose%
\pgfusepath{stroke,fill}%
\end{pgfscope}%
\begin{pgfscope}%
\pgfpathrectangle{\pgfqpoint{0.800000in}{0.528000in}}{\pgfqpoint{4.960000in}{3.696000in}}%
\pgfusepath{clip}%
\pgfsetbuttcap%
\pgfsetroundjoin%
\definecolor{currentfill}{rgb}{0.000000,0.000000,0.000000}%
\pgfsetfillcolor{currentfill}%
\pgfsetlinewidth{1.003750pt}%
\definecolor{currentstroke}{rgb}{0.000000,0.000000,0.000000}%
\pgfsetstrokecolor{currentstroke}%
\pgfsetdash{}{0pt}%
\pgfpathmoveto{\pgfqpoint{2.518786in}{3.984333in}}%
\pgfpathcurveto{\pgfqpoint{2.529836in}{3.984333in}}{\pgfqpoint{2.540435in}{3.988724in}}{\pgfqpoint{2.548249in}{3.996537in}}%
\pgfpathcurveto{\pgfqpoint{2.556062in}{4.004351in}}{\pgfqpoint{2.560452in}{4.014950in}}{\pgfqpoint{2.560452in}{4.026000in}}%
\pgfpathcurveto{\pgfqpoint{2.560452in}{4.037050in}}{\pgfqpoint{2.556062in}{4.047649in}}{\pgfqpoint{2.548249in}{4.055463in}}%
\pgfpathcurveto{\pgfqpoint{2.540435in}{4.063276in}}{\pgfqpoint{2.529836in}{4.067667in}}{\pgfqpoint{2.518786in}{4.067667in}}%
\pgfpathcurveto{\pgfqpoint{2.507736in}{4.067667in}}{\pgfqpoint{2.497137in}{4.063276in}}{\pgfqpoint{2.489323in}{4.055463in}}%
\pgfpathcurveto{\pgfqpoint{2.481509in}{4.047649in}}{\pgfqpoint{2.477119in}{4.037050in}}{\pgfqpoint{2.477119in}{4.026000in}}%
\pgfpathcurveto{\pgfqpoint{2.477119in}{4.014950in}}{\pgfqpoint{2.481509in}{4.004351in}}{\pgfqpoint{2.489323in}{3.996537in}}%
\pgfpathcurveto{\pgfqpoint{2.497137in}{3.988724in}}{\pgfqpoint{2.507736in}{3.984333in}}{\pgfqpoint{2.518786in}{3.984333in}}%
\pgfpathclose%
\pgfusepath{stroke,fill}%
\end{pgfscope}%
\begin{pgfscope}%
\pgfpathrectangle{\pgfqpoint{0.800000in}{0.528000in}}{\pgfqpoint{4.960000in}{3.696000in}}%
\pgfusepath{clip}%
\pgfsetbuttcap%
\pgfsetroundjoin%
\definecolor{currentfill}{rgb}{0.000000,0.000000,0.000000}%
\pgfsetfillcolor{currentfill}%
\pgfsetlinewidth{1.003750pt}%
\definecolor{currentstroke}{rgb}{0.000000,0.000000,0.000000}%
\pgfsetstrokecolor{currentstroke}%
\pgfsetdash{}{0pt}%
\pgfpathmoveto{\pgfqpoint{2.518786in}{3.984333in}}%
\pgfpathcurveto{\pgfqpoint{2.529836in}{3.984333in}}{\pgfqpoint{2.540435in}{3.988724in}}{\pgfqpoint{2.548249in}{3.996537in}}%
\pgfpathcurveto{\pgfqpoint{2.556062in}{4.004351in}}{\pgfqpoint{2.560452in}{4.014950in}}{\pgfqpoint{2.560452in}{4.026000in}}%
\pgfpathcurveto{\pgfqpoint{2.560452in}{4.037050in}}{\pgfqpoint{2.556062in}{4.047649in}}{\pgfqpoint{2.548249in}{4.055463in}}%
\pgfpathcurveto{\pgfqpoint{2.540435in}{4.063276in}}{\pgfqpoint{2.529836in}{4.067667in}}{\pgfqpoint{2.518786in}{4.067667in}}%
\pgfpathcurveto{\pgfqpoint{2.507736in}{4.067667in}}{\pgfqpoint{2.497137in}{4.063276in}}{\pgfqpoint{2.489323in}{4.055463in}}%
\pgfpathcurveto{\pgfqpoint{2.481509in}{4.047649in}}{\pgfqpoint{2.477119in}{4.037050in}}{\pgfqpoint{2.477119in}{4.026000in}}%
\pgfpathcurveto{\pgfqpoint{2.477119in}{4.014950in}}{\pgfqpoint{2.481509in}{4.004351in}}{\pgfqpoint{2.489323in}{3.996537in}}%
\pgfpathcurveto{\pgfqpoint{2.497137in}{3.988724in}}{\pgfqpoint{2.507736in}{3.984333in}}{\pgfqpoint{2.518786in}{3.984333in}}%
\pgfpathclose%
\pgfusepath{stroke,fill}%
\end{pgfscope}%
\begin{pgfscope}%
\pgfpathrectangle{\pgfqpoint{0.800000in}{0.528000in}}{\pgfqpoint{4.960000in}{3.696000in}}%
\pgfusepath{clip}%
\pgfsetbuttcap%
\pgfsetroundjoin%
\definecolor{currentfill}{rgb}{0.000000,0.000000,0.000000}%
\pgfsetfillcolor{currentfill}%
\pgfsetlinewidth{1.003750pt}%
\definecolor{currentstroke}{rgb}{0.000000,0.000000,0.000000}%
\pgfsetstrokecolor{currentstroke}%
\pgfsetdash{}{0pt}%
\pgfpathmoveto{\pgfqpoint{2.518786in}{3.984333in}}%
\pgfpathcurveto{\pgfqpoint{2.529836in}{3.984333in}}{\pgfqpoint{2.540435in}{3.988724in}}{\pgfqpoint{2.548249in}{3.996537in}}%
\pgfpathcurveto{\pgfqpoint{2.556062in}{4.004351in}}{\pgfqpoint{2.560452in}{4.014950in}}{\pgfqpoint{2.560452in}{4.026000in}}%
\pgfpathcurveto{\pgfqpoint{2.560452in}{4.037050in}}{\pgfqpoint{2.556062in}{4.047649in}}{\pgfqpoint{2.548249in}{4.055463in}}%
\pgfpathcurveto{\pgfqpoint{2.540435in}{4.063276in}}{\pgfqpoint{2.529836in}{4.067667in}}{\pgfqpoint{2.518786in}{4.067667in}}%
\pgfpathcurveto{\pgfqpoint{2.507736in}{4.067667in}}{\pgfqpoint{2.497137in}{4.063276in}}{\pgfqpoint{2.489323in}{4.055463in}}%
\pgfpathcurveto{\pgfqpoint{2.481509in}{4.047649in}}{\pgfqpoint{2.477119in}{4.037050in}}{\pgfqpoint{2.477119in}{4.026000in}}%
\pgfpathcurveto{\pgfqpoint{2.477119in}{4.014950in}}{\pgfqpoint{2.481509in}{4.004351in}}{\pgfqpoint{2.489323in}{3.996537in}}%
\pgfpathcurveto{\pgfqpoint{2.497137in}{3.988724in}}{\pgfqpoint{2.507736in}{3.984333in}}{\pgfqpoint{2.518786in}{3.984333in}}%
\pgfpathclose%
\pgfusepath{stroke,fill}%
\end{pgfscope}%
\begin{pgfscope}%
\pgfpathrectangle{\pgfqpoint{0.800000in}{0.528000in}}{\pgfqpoint{4.960000in}{3.696000in}}%
\pgfusepath{clip}%
\pgfsetbuttcap%
\pgfsetroundjoin%
\definecolor{currentfill}{rgb}{0.000000,0.000000,0.000000}%
\pgfsetfillcolor{currentfill}%
\pgfsetlinewidth{1.003750pt}%
\definecolor{currentstroke}{rgb}{0.000000,0.000000,0.000000}%
\pgfsetstrokecolor{currentstroke}%
\pgfsetdash{}{0pt}%
\pgfpathmoveto{\pgfqpoint{2.518786in}{3.984333in}}%
\pgfpathcurveto{\pgfqpoint{2.529836in}{3.984333in}}{\pgfqpoint{2.540435in}{3.988724in}}{\pgfqpoint{2.548249in}{3.996537in}}%
\pgfpathcurveto{\pgfqpoint{2.556062in}{4.004351in}}{\pgfqpoint{2.560452in}{4.014950in}}{\pgfqpoint{2.560452in}{4.026000in}}%
\pgfpathcurveto{\pgfqpoint{2.560452in}{4.037050in}}{\pgfqpoint{2.556062in}{4.047649in}}{\pgfqpoint{2.548249in}{4.055463in}}%
\pgfpathcurveto{\pgfqpoint{2.540435in}{4.063276in}}{\pgfqpoint{2.529836in}{4.067667in}}{\pgfqpoint{2.518786in}{4.067667in}}%
\pgfpathcurveto{\pgfqpoint{2.507736in}{4.067667in}}{\pgfqpoint{2.497137in}{4.063276in}}{\pgfqpoint{2.489323in}{4.055463in}}%
\pgfpathcurveto{\pgfqpoint{2.481509in}{4.047649in}}{\pgfqpoint{2.477119in}{4.037050in}}{\pgfqpoint{2.477119in}{4.026000in}}%
\pgfpathcurveto{\pgfqpoint{2.477119in}{4.014950in}}{\pgfqpoint{2.481509in}{4.004351in}}{\pgfqpoint{2.489323in}{3.996537in}}%
\pgfpathcurveto{\pgfqpoint{2.497137in}{3.988724in}}{\pgfqpoint{2.507736in}{3.984333in}}{\pgfqpoint{2.518786in}{3.984333in}}%
\pgfpathclose%
\pgfusepath{stroke,fill}%
\end{pgfscope}%
\begin{pgfscope}%
\pgfpathrectangle{\pgfqpoint{0.800000in}{0.528000in}}{\pgfqpoint{4.960000in}{3.696000in}}%
\pgfusepath{clip}%
\pgfsetbuttcap%
\pgfsetroundjoin%
\definecolor{currentfill}{rgb}{0.000000,0.000000,0.000000}%
\pgfsetfillcolor{currentfill}%
\pgfsetlinewidth{1.003750pt}%
\definecolor{currentstroke}{rgb}{0.000000,0.000000,0.000000}%
\pgfsetstrokecolor{currentstroke}%
\pgfsetdash{}{0pt}%
\pgfpathmoveto{\pgfqpoint{2.518786in}{3.984333in}}%
\pgfpathcurveto{\pgfqpoint{2.529836in}{3.984333in}}{\pgfqpoint{2.540435in}{3.988724in}}{\pgfqpoint{2.548249in}{3.996537in}}%
\pgfpathcurveto{\pgfqpoint{2.556062in}{4.004351in}}{\pgfqpoint{2.560452in}{4.014950in}}{\pgfqpoint{2.560452in}{4.026000in}}%
\pgfpathcurveto{\pgfqpoint{2.560452in}{4.037050in}}{\pgfqpoint{2.556062in}{4.047649in}}{\pgfqpoint{2.548249in}{4.055463in}}%
\pgfpathcurveto{\pgfqpoint{2.540435in}{4.063276in}}{\pgfqpoint{2.529836in}{4.067667in}}{\pgfqpoint{2.518786in}{4.067667in}}%
\pgfpathcurveto{\pgfqpoint{2.507736in}{4.067667in}}{\pgfqpoint{2.497137in}{4.063276in}}{\pgfqpoint{2.489323in}{4.055463in}}%
\pgfpathcurveto{\pgfqpoint{2.481509in}{4.047649in}}{\pgfqpoint{2.477119in}{4.037050in}}{\pgfqpoint{2.477119in}{4.026000in}}%
\pgfpathcurveto{\pgfqpoint{2.477119in}{4.014950in}}{\pgfqpoint{2.481509in}{4.004351in}}{\pgfqpoint{2.489323in}{3.996537in}}%
\pgfpathcurveto{\pgfqpoint{2.497137in}{3.988724in}}{\pgfqpoint{2.507736in}{3.984333in}}{\pgfqpoint{2.518786in}{3.984333in}}%
\pgfpathclose%
\pgfusepath{stroke,fill}%
\end{pgfscope}%
\begin{pgfscope}%
\pgfpathrectangle{\pgfqpoint{0.800000in}{0.528000in}}{\pgfqpoint{4.960000in}{3.696000in}}%
\pgfusepath{clip}%
\pgfsetbuttcap%
\pgfsetroundjoin%
\definecolor{currentfill}{rgb}{0.000000,0.000000,0.000000}%
\pgfsetfillcolor{currentfill}%
\pgfsetlinewidth{1.003750pt}%
\definecolor{currentstroke}{rgb}{0.000000,0.000000,0.000000}%
\pgfsetstrokecolor{currentstroke}%
\pgfsetdash{}{0pt}%
\pgfpathmoveto{\pgfqpoint{2.518786in}{3.984333in}}%
\pgfpathcurveto{\pgfqpoint{2.529836in}{3.984333in}}{\pgfqpoint{2.540435in}{3.988724in}}{\pgfqpoint{2.548249in}{3.996537in}}%
\pgfpathcurveto{\pgfqpoint{2.556062in}{4.004351in}}{\pgfqpoint{2.560452in}{4.014950in}}{\pgfqpoint{2.560452in}{4.026000in}}%
\pgfpathcurveto{\pgfqpoint{2.560452in}{4.037050in}}{\pgfqpoint{2.556062in}{4.047649in}}{\pgfqpoint{2.548249in}{4.055463in}}%
\pgfpathcurveto{\pgfqpoint{2.540435in}{4.063276in}}{\pgfqpoint{2.529836in}{4.067667in}}{\pgfqpoint{2.518786in}{4.067667in}}%
\pgfpathcurveto{\pgfqpoint{2.507736in}{4.067667in}}{\pgfqpoint{2.497137in}{4.063276in}}{\pgfqpoint{2.489323in}{4.055463in}}%
\pgfpathcurveto{\pgfqpoint{2.481509in}{4.047649in}}{\pgfqpoint{2.477119in}{4.037050in}}{\pgfqpoint{2.477119in}{4.026000in}}%
\pgfpathcurveto{\pgfqpoint{2.477119in}{4.014950in}}{\pgfqpoint{2.481509in}{4.004351in}}{\pgfqpoint{2.489323in}{3.996537in}}%
\pgfpathcurveto{\pgfqpoint{2.497137in}{3.988724in}}{\pgfqpoint{2.507736in}{3.984333in}}{\pgfqpoint{2.518786in}{3.984333in}}%
\pgfpathclose%
\pgfusepath{stroke,fill}%
\end{pgfscope}%
\begin{pgfscope}%
\pgfpathrectangle{\pgfqpoint{0.800000in}{0.528000in}}{\pgfqpoint{4.960000in}{3.696000in}}%
\pgfusepath{clip}%
\pgfsetbuttcap%
\pgfsetroundjoin%
\definecolor{currentfill}{rgb}{0.000000,0.000000,0.000000}%
\pgfsetfillcolor{currentfill}%
\pgfsetlinewidth{1.003750pt}%
\definecolor{currentstroke}{rgb}{0.000000,0.000000,0.000000}%
\pgfsetstrokecolor{currentstroke}%
\pgfsetdash{}{0pt}%
\pgfpathmoveto{\pgfqpoint{2.518786in}{3.984333in}}%
\pgfpathcurveto{\pgfqpoint{2.529836in}{3.984333in}}{\pgfqpoint{2.540435in}{3.988724in}}{\pgfqpoint{2.548249in}{3.996537in}}%
\pgfpathcurveto{\pgfqpoint{2.556062in}{4.004351in}}{\pgfqpoint{2.560452in}{4.014950in}}{\pgfqpoint{2.560452in}{4.026000in}}%
\pgfpathcurveto{\pgfqpoint{2.560452in}{4.037050in}}{\pgfqpoint{2.556062in}{4.047649in}}{\pgfqpoint{2.548249in}{4.055463in}}%
\pgfpathcurveto{\pgfqpoint{2.540435in}{4.063276in}}{\pgfqpoint{2.529836in}{4.067667in}}{\pgfqpoint{2.518786in}{4.067667in}}%
\pgfpathcurveto{\pgfqpoint{2.507736in}{4.067667in}}{\pgfqpoint{2.497137in}{4.063276in}}{\pgfqpoint{2.489323in}{4.055463in}}%
\pgfpathcurveto{\pgfqpoint{2.481509in}{4.047649in}}{\pgfqpoint{2.477119in}{4.037050in}}{\pgfqpoint{2.477119in}{4.026000in}}%
\pgfpathcurveto{\pgfqpoint{2.477119in}{4.014950in}}{\pgfqpoint{2.481509in}{4.004351in}}{\pgfqpoint{2.489323in}{3.996537in}}%
\pgfpathcurveto{\pgfqpoint{2.497137in}{3.988724in}}{\pgfqpoint{2.507736in}{3.984333in}}{\pgfqpoint{2.518786in}{3.984333in}}%
\pgfpathclose%
\pgfusepath{stroke,fill}%
\end{pgfscope}%
\begin{pgfscope}%
\pgfpathrectangle{\pgfqpoint{0.800000in}{0.528000in}}{\pgfqpoint{4.960000in}{3.696000in}}%
\pgfusepath{clip}%
\pgfsetbuttcap%
\pgfsetroundjoin%
\definecolor{currentfill}{rgb}{0.000000,0.000000,0.000000}%
\pgfsetfillcolor{currentfill}%
\pgfsetlinewidth{1.003750pt}%
\definecolor{currentstroke}{rgb}{0.000000,0.000000,0.000000}%
\pgfsetstrokecolor{currentstroke}%
\pgfsetdash{}{0pt}%
\pgfpathmoveto{\pgfqpoint{2.518786in}{3.984333in}}%
\pgfpathcurveto{\pgfqpoint{2.529836in}{3.984333in}}{\pgfqpoint{2.540435in}{3.988724in}}{\pgfqpoint{2.548249in}{3.996537in}}%
\pgfpathcurveto{\pgfqpoint{2.556062in}{4.004351in}}{\pgfqpoint{2.560452in}{4.014950in}}{\pgfqpoint{2.560452in}{4.026000in}}%
\pgfpathcurveto{\pgfqpoint{2.560452in}{4.037050in}}{\pgfqpoint{2.556062in}{4.047649in}}{\pgfqpoint{2.548249in}{4.055463in}}%
\pgfpathcurveto{\pgfqpoint{2.540435in}{4.063276in}}{\pgfqpoint{2.529836in}{4.067667in}}{\pgfqpoint{2.518786in}{4.067667in}}%
\pgfpathcurveto{\pgfqpoint{2.507736in}{4.067667in}}{\pgfqpoint{2.497137in}{4.063276in}}{\pgfqpoint{2.489323in}{4.055463in}}%
\pgfpathcurveto{\pgfqpoint{2.481509in}{4.047649in}}{\pgfqpoint{2.477119in}{4.037050in}}{\pgfqpoint{2.477119in}{4.026000in}}%
\pgfpathcurveto{\pgfqpoint{2.477119in}{4.014950in}}{\pgfqpoint{2.481509in}{4.004351in}}{\pgfqpoint{2.489323in}{3.996537in}}%
\pgfpathcurveto{\pgfqpoint{2.497137in}{3.988724in}}{\pgfqpoint{2.507736in}{3.984333in}}{\pgfqpoint{2.518786in}{3.984333in}}%
\pgfpathclose%
\pgfusepath{stroke,fill}%
\end{pgfscope}%
\begin{pgfscope}%
\pgfpathrectangle{\pgfqpoint{0.800000in}{0.528000in}}{\pgfqpoint{4.960000in}{3.696000in}}%
\pgfusepath{clip}%
\pgfsetbuttcap%
\pgfsetroundjoin%
\definecolor{currentfill}{rgb}{0.000000,0.000000,0.000000}%
\pgfsetfillcolor{currentfill}%
\pgfsetlinewidth{1.003750pt}%
\definecolor{currentstroke}{rgb}{0.000000,0.000000,0.000000}%
\pgfsetstrokecolor{currentstroke}%
\pgfsetdash{}{0pt}%
\pgfpathmoveto{\pgfqpoint{2.518786in}{3.984333in}}%
\pgfpathcurveto{\pgfqpoint{2.529836in}{3.984333in}}{\pgfqpoint{2.540435in}{3.988724in}}{\pgfqpoint{2.548249in}{3.996537in}}%
\pgfpathcurveto{\pgfqpoint{2.556062in}{4.004351in}}{\pgfqpoint{2.560452in}{4.014950in}}{\pgfqpoint{2.560452in}{4.026000in}}%
\pgfpathcurveto{\pgfqpoint{2.560452in}{4.037050in}}{\pgfqpoint{2.556062in}{4.047649in}}{\pgfqpoint{2.548249in}{4.055463in}}%
\pgfpathcurveto{\pgfqpoint{2.540435in}{4.063276in}}{\pgfqpoint{2.529836in}{4.067667in}}{\pgfqpoint{2.518786in}{4.067667in}}%
\pgfpathcurveto{\pgfqpoint{2.507736in}{4.067667in}}{\pgfqpoint{2.497137in}{4.063276in}}{\pgfqpoint{2.489323in}{4.055463in}}%
\pgfpathcurveto{\pgfqpoint{2.481509in}{4.047649in}}{\pgfqpoint{2.477119in}{4.037050in}}{\pgfqpoint{2.477119in}{4.026000in}}%
\pgfpathcurveto{\pgfqpoint{2.477119in}{4.014950in}}{\pgfqpoint{2.481509in}{4.004351in}}{\pgfqpoint{2.489323in}{3.996537in}}%
\pgfpathcurveto{\pgfqpoint{2.497137in}{3.988724in}}{\pgfqpoint{2.507736in}{3.984333in}}{\pgfqpoint{2.518786in}{3.984333in}}%
\pgfpathclose%
\pgfusepath{stroke,fill}%
\end{pgfscope}%
\begin{pgfscope}%
\pgfpathrectangle{\pgfqpoint{0.800000in}{0.528000in}}{\pgfqpoint{4.960000in}{3.696000in}}%
\pgfusepath{clip}%
\pgfsetbuttcap%
\pgfsetroundjoin%
\definecolor{currentfill}{rgb}{0.000000,0.000000,0.000000}%
\pgfsetfillcolor{currentfill}%
\pgfsetlinewidth{1.003750pt}%
\definecolor{currentstroke}{rgb}{0.000000,0.000000,0.000000}%
\pgfsetstrokecolor{currentstroke}%
\pgfsetdash{}{0pt}%
\pgfpathmoveto{\pgfqpoint{2.518786in}{3.984333in}}%
\pgfpathcurveto{\pgfqpoint{2.529836in}{3.984333in}}{\pgfqpoint{2.540435in}{3.988724in}}{\pgfqpoint{2.548249in}{3.996537in}}%
\pgfpathcurveto{\pgfqpoint{2.556062in}{4.004351in}}{\pgfqpoint{2.560452in}{4.014950in}}{\pgfqpoint{2.560452in}{4.026000in}}%
\pgfpathcurveto{\pgfqpoint{2.560452in}{4.037050in}}{\pgfqpoint{2.556062in}{4.047649in}}{\pgfqpoint{2.548249in}{4.055463in}}%
\pgfpathcurveto{\pgfqpoint{2.540435in}{4.063276in}}{\pgfqpoint{2.529836in}{4.067667in}}{\pgfqpoint{2.518786in}{4.067667in}}%
\pgfpathcurveto{\pgfqpoint{2.507736in}{4.067667in}}{\pgfqpoint{2.497137in}{4.063276in}}{\pgfqpoint{2.489323in}{4.055463in}}%
\pgfpathcurveto{\pgfqpoint{2.481509in}{4.047649in}}{\pgfqpoint{2.477119in}{4.037050in}}{\pgfqpoint{2.477119in}{4.026000in}}%
\pgfpathcurveto{\pgfqpoint{2.477119in}{4.014950in}}{\pgfqpoint{2.481509in}{4.004351in}}{\pgfqpoint{2.489323in}{3.996537in}}%
\pgfpathcurveto{\pgfqpoint{2.497137in}{3.988724in}}{\pgfqpoint{2.507736in}{3.984333in}}{\pgfqpoint{2.518786in}{3.984333in}}%
\pgfpathclose%
\pgfusepath{stroke,fill}%
\end{pgfscope}%
\begin{pgfscope}%
\pgfpathrectangle{\pgfqpoint{0.800000in}{0.528000in}}{\pgfqpoint{4.960000in}{3.696000in}}%
\pgfusepath{clip}%
\pgfsetbuttcap%
\pgfsetroundjoin%
\definecolor{currentfill}{rgb}{0.000000,0.000000,0.000000}%
\pgfsetfillcolor{currentfill}%
\pgfsetlinewidth{1.003750pt}%
\definecolor{currentstroke}{rgb}{0.000000,0.000000,0.000000}%
\pgfsetstrokecolor{currentstroke}%
\pgfsetdash{}{0pt}%
\pgfpathmoveto{\pgfqpoint{2.518786in}{3.984333in}}%
\pgfpathcurveto{\pgfqpoint{2.529836in}{3.984333in}}{\pgfqpoint{2.540435in}{3.988724in}}{\pgfqpoint{2.548249in}{3.996537in}}%
\pgfpathcurveto{\pgfqpoint{2.556062in}{4.004351in}}{\pgfqpoint{2.560452in}{4.014950in}}{\pgfqpoint{2.560452in}{4.026000in}}%
\pgfpathcurveto{\pgfqpoint{2.560452in}{4.037050in}}{\pgfqpoint{2.556062in}{4.047649in}}{\pgfqpoint{2.548249in}{4.055463in}}%
\pgfpathcurveto{\pgfqpoint{2.540435in}{4.063276in}}{\pgfqpoint{2.529836in}{4.067667in}}{\pgfqpoint{2.518786in}{4.067667in}}%
\pgfpathcurveto{\pgfqpoint{2.507736in}{4.067667in}}{\pgfqpoint{2.497137in}{4.063276in}}{\pgfqpoint{2.489323in}{4.055463in}}%
\pgfpathcurveto{\pgfqpoint{2.481509in}{4.047649in}}{\pgfqpoint{2.477119in}{4.037050in}}{\pgfqpoint{2.477119in}{4.026000in}}%
\pgfpathcurveto{\pgfqpoint{2.477119in}{4.014950in}}{\pgfqpoint{2.481509in}{4.004351in}}{\pgfqpoint{2.489323in}{3.996537in}}%
\pgfpathcurveto{\pgfqpoint{2.497137in}{3.988724in}}{\pgfqpoint{2.507736in}{3.984333in}}{\pgfqpoint{2.518786in}{3.984333in}}%
\pgfpathclose%
\pgfusepath{stroke,fill}%
\end{pgfscope}%
\begin{pgfscope}%
\pgfpathrectangle{\pgfqpoint{0.800000in}{0.528000in}}{\pgfqpoint{4.960000in}{3.696000in}}%
\pgfusepath{clip}%
\pgfsetbuttcap%
\pgfsetroundjoin%
\definecolor{currentfill}{rgb}{0.000000,0.000000,0.000000}%
\pgfsetfillcolor{currentfill}%
\pgfsetlinewidth{1.003750pt}%
\definecolor{currentstroke}{rgb}{0.000000,0.000000,0.000000}%
\pgfsetstrokecolor{currentstroke}%
\pgfsetdash{}{0pt}%
\pgfpathmoveto{\pgfqpoint{2.518786in}{3.984333in}}%
\pgfpathcurveto{\pgfqpoint{2.529836in}{3.984333in}}{\pgfqpoint{2.540435in}{3.988724in}}{\pgfqpoint{2.548249in}{3.996537in}}%
\pgfpathcurveto{\pgfqpoint{2.556062in}{4.004351in}}{\pgfqpoint{2.560452in}{4.014950in}}{\pgfqpoint{2.560452in}{4.026000in}}%
\pgfpathcurveto{\pgfqpoint{2.560452in}{4.037050in}}{\pgfqpoint{2.556062in}{4.047649in}}{\pgfqpoint{2.548249in}{4.055463in}}%
\pgfpathcurveto{\pgfqpoint{2.540435in}{4.063276in}}{\pgfqpoint{2.529836in}{4.067667in}}{\pgfqpoint{2.518786in}{4.067667in}}%
\pgfpathcurveto{\pgfqpoint{2.507736in}{4.067667in}}{\pgfqpoint{2.497137in}{4.063276in}}{\pgfqpoint{2.489323in}{4.055463in}}%
\pgfpathcurveto{\pgfqpoint{2.481509in}{4.047649in}}{\pgfqpoint{2.477119in}{4.037050in}}{\pgfqpoint{2.477119in}{4.026000in}}%
\pgfpathcurveto{\pgfqpoint{2.477119in}{4.014950in}}{\pgfqpoint{2.481509in}{4.004351in}}{\pgfqpoint{2.489323in}{3.996537in}}%
\pgfpathcurveto{\pgfqpoint{2.497137in}{3.988724in}}{\pgfqpoint{2.507736in}{3.984333in}}{\pgfqpoint{2.518786in}{3.984333in}}%
\pgfpathclose%
\pgfusepath{stroke,fill}%
\end{pgfscope}%
\begin{pgfscope}%
\pgfpathrectangle{\pgfqpoint{0.800000in}{0.528000in}}{\pgfqpoint{4.960000in}{3.696000in}}%
\pgfusepath{clip}%
\pgfsetbuttcap%
\pgfsetroundjoin%
\definecolor{currentfill}{rgb}{0.000000,0.000000,0.000000}%
\pgfsetfillcolor{currentfill}%
\pgfsetlinewidth{1.003750pt}%
\definecolor{currentstroke}{rgb}{0.000000,0.000000,0.000000}%
\pgfsetstrokecolor{currentstroke}%
\pgfsetdash{}{0pt}%
\pgfpathmoveto{\pgfqpoint{2.518786in}{3.984333in}}%
\pgfpathcurveto{\pgfqpoint{2.529836in}{3.984333in}}{\pgfqpoint{2.540435in}{3.988724in}}{\pgfqpoint{2.548249in}{3.996537in}}%
\pgfpathcurveto{\pgfqpoint{2.556062in}{4.004351in}}{\pgfqpoint{2.560452in}{4.014950in}}{\pgfqpoint{2.560452in}{4.026000in}}%
\pgfpathcurveto{\pgfqpoint{2.560452in}{4.037050in}}{\pgfqpoint{2.556062in}{4.047649in}}{\pgfqpoint{2.548249in}{4.055463in}}%
\pgfpathcurveto{\pgfqpoint{2.540435in}{4.063276in}}{\pgfqpoint{2.529836in}{4.067667in}}{\pgfqpoint{2.518786in}{4.067667in}}%
\pgfpathcurveto{\pgfqpoint{2.507736in}{4.067667in}}{\pgfqpoint{2.497137in}{4.063276in}}{\pgfqpoint{2.489323in}{4.055463in}}%
\pgfpathcurveto{\pgfqpoint{2.481509in}{4.047649in}}{\pgfqpoint{2.477119in}{4.037050in}}{\pgfqpoint{2.477119in}{4.026000in}}%
\pgfpathcurveto{\pgfqpoint{2.477119in}{4.014950in}}{\pgfqpoint{2.481509in}{4.004351in}}{\pgfqpoint{2.489323in}{3.996537in}}%
\pgfpathcurveto{\pgfqpoint{2.497137in}{3.988724in}}{\pgfqpoint{2.507736in}{3.984333in}}{\pgfqpoint{2.518786in}{3.984333in}}%
\pgfpathclose%
\pgfusepath{stroke,fill}%
\end{pgfscope}%
\begin{pgfscope}%
\pgfpathrectangle{\pgfqpoint{0.800000in}{0.528000in}}{\pgfqpoint{4.960000in}{3.696000in}}%
\pgfusepath{clip}%
\pgfsetbuttcap%
\pgfsetroundjoin%
\definecolor{currentfill}{rgb}{0.000000,0.000000,0.000000}%
\pgfsetfillcolor{currentfill}%
\pgfsetlinewidth{1.003750pt}%
\definecolor{currentstroke}{rgb}{0.000000,0.000000,0.000000}%
\pgfsetstrokecolor{currentstroke}%
\pgfsetdash{}{0pt}%
\pgfpathmoveto{\pgfqpoint{2.518786in}{3.984333in}}%
\pgfpathcurveto{\pgfqpoint{2.529836in}{3.984333in}}{\pgfqpoint{2.540435in}{3.988724in}}{\pgfqpoint{2.548249in}{3.996537in}}%
\pgfpathcurveto{\pgfqpoint{2.556062in}{4.004351in}}{\pgfqpoint{2.560452in}{4.014950in}}{\pgfqpoint{2.560452in}{4.026000in}}%
\pgfpathcurveto{\pgfqpoint{2.560452in}{4.037050in}}{\pgfqpoint{2.556062in}{4.047649in}}{\pgfqpoint{2.548249in}{4.055463in}}%
\pgfpathcurveto{\pgfqpoint{2.540435in}{4.063276in}}{\pgfqpoint{2.529836in}{4.067667in}}{\pgfqpoint{2.518786in}{4.067667in}}%
\pgfpathcurveto{\pgfqpoint{2.507736in}{4.067667in}}{\pgfqpoint{2.497137in}{4.063276in}}{\pgfqpoint{2.489323in}{4.055463in}}%
\pgfpathcurveto{\pgfqpoint{2.481509in}{4.047649in}}{\pgfqpoint{2.477119in}{4.037050in}}{\pgfqpoint{2.477119in}{4.026000in}}%
\pgfpathcurveto{\pgfqpoint{2.477119in}{4.014950in}}{\pgfqpoint{2.481509in}{4.004351in}}{\pgfqpoint{2.489323in}{3.996537in}}%
\pgfpathcurveto{\pgfqpoint{2.497137in}{3.988724in}}{\pgfqpoint{2.507736in}{3.984333in}}{\pgfqpoint{2.518786in}{3.984333in}}%
\pgfpathclose%
\pgfusepath{stroke,fill}%
\end{pgfscope}%
\begin{pgfscope}%
\pgfpathrectangle{\pgfqpoint{0.800000in}{0.528000in}}{\pgfqpoint{4.960000in}{3.696000in}}%
\pgfusepath{clip}%
\pgfsetbuttcap%
\pgfsetroundjoin%
\definecolor{currentfill}{rgb}{0.000000,0.000000,0.000000}%
\pgfsetfillcolor{currentfill}%
\pgfsetlinewidth{1.003750pt}%
\definecolor{currentstroke}{rgb}{0.000000,0.000000,0.000000}%
\pgfsetstrokecolor{currentstroke}%
\pgfsetdash{}{0pt}%
\pgfpathmoveto{\pgfqpoint{2.518786in}{3.984333in}}%
\pgfpathcurveto{\pgfqpoint{2.529836in}{3.984333in}}{\pgfqpoint{2.540435in}{3.988724in}}{\pgfqpoint{2.548249in}{3.996537in}}%
\pgfpathcurveto{\pgfqpoint{2.556062in}{4.004351in}}{\pgfqpoint{2.560452in}{4.014950in}}{\pgfqpoint{2.560452in}{4.026000in}}%
\pgfpathcurveto{\pgfqpoint{2.560452in}{4.037050in}}{\pgfqpoint{2.556062in}{4.047649in}}{\pgfqpoint{2.548249in}{4.055463in}}%
\pgfpathcurveto{\pgfqpoint{2.540435in}{4.063276in}}{\pgfqpoint{2.529836in}{4.067667in}}{\pgfqpoint{2.518786in}{4.067667in}}%
\pgfpathcurveto{\pgfqpoint{2.507736in}{4.067667in}}{\pgfqpoint{2.497137in}{4.063276in}}{\pgfqpoint{2.489323in}{4.055463in}}%
\pgfpathcurveto{\pgfqpoint{2.481509in}{4.047649in}}{\pgfqpoint{2.477119in}{4.037050in}}{\pgfqpoint{2.477119in}{4.026000in}}%
\pgfpathcurveto{\pgfqpoint{2.477119in}{4.014950in}}{\pgfqpoint{2.481509in}{4.004351in}}{\pgfqpoint{2.489323in}{3.996537in}}%
\pgfpathcurveto{\pgfqpoint{2.497137in}{3.988724in}}{\pgfqpoint{2.507736in}{3.984333in}}{\pgfqpoint{2.518786in}{3.984333in}}%
\pgfpathclose%
\pgfusepath{stroke,fill}%
\end{pgfscope}%
\begin{pgfscope}%
\pgfpathrectangle{\pgfqpoint{0.800000in}{0.528000in}}{\pgfqpoint{4.960000in}{3.696000in}}%
\pgfusepath{clip}%
\pgfsetbuttcap%
\pgfsetroundjoin%
\definecolor{currentfill}{rgb}{0.000000,0.000000,0.000000}%
\pgfsetfillcolor{currentfill}%
\pgfsetlinewidth{1.003750pt}%
\definecolor{currentstroke}{rgb}{0.000000,0.000000,0.000000}%
\pgfsetstrokecolor{currentstroke}%
\pgfsetdash{}{0pt}%
\pgfpathmoveto{\pgfqpoint{2.518786in}{3.984333in}}%
\pgfpathcurveto{\pgfqpoint{2.529836in}{3.984333in}}{\pgfqpoint{2.540435in}{3.988724in}}{\pgfqpoint{2.548249in}{3.996537in}}%
\pgfpathcurveto{\pgfqpoint{2.556062in}{4.004351in}}{\pgfqpoint{2.560452in}{4.014950in}}{\pgfqpoint{2.560452in}{4.026000in}}%
\pgfpathcurveto{\pgfqpoint{2.560452in}{4.037050in}}{\pgfqpoint{2.556062in}{4.047649in}}{\pgfqpoint{2.548249in}{4.055463in}}%
\pgfpathcurveto{\pgfqpoint{2.540435in}{4.063276in}}{\pgfqpoint{2.529836in}{4.067667in}}{\pgfqpoint{2.518786in}{4.067667in}}%
\pgfpathcurveto{\pgfqpoint{2.507736in}{4.067667in}}{\pgfqpoint{2.497137in}{4.063276in}}{\pgfqpoint{2.489323in}{4.055463in}}%
\pgfpathcurveto{\pgfqpoint{2.481509in}{4.047649in}}{\pgfqpoint{2.477119in}{4.037050in}}{\pgfqpoint{2.477119in}{4.026000in}}%
\pgfpathcurveto{\pgfqpoint{2.477119in}{4.014950in}}{\pgfqpoint{2.481509in}{4.004351in}}{\pgfqpoint{2.489323in}{3.996537in}}%
\pgfpathcurveto{\pgfqpoint{2.497137in}{3.988724in}}{\pgfqpoint{2.507736in}{3.984333in}}{\pgfqpoint{2.518786in}{3.984333in}}%
\pgfpathclose%
\pgfusepath{stroke,fill}%
\end{pgfscope}%
\begin{pgfscope}%
\pgfpathrectangle{\pgfqpoint{0.800000in}{0.528000in}}{\pgfqpoint{4.960000in}{3.696000in}}%
\pgfusepath{clip}%
\pgfsetbuttcap%
\pgfsetroundjoin%
\definecolor{currentfill}{rgb}{0.000000,0.000000,0.000000}%
\pgfsetfillcolor{currentfill}%
\pgfsetlinewidth{1.003750pt}%
\definecolor{currentstroke}{rgb}{0.000000,0.000000,0.000000}%
\pgfsetstrokecolor{currentstroke}%
\pgfsetdash{}{0pt}%
\pgfpathmoveto{\pgfqpoint{2.518786in}{3.984333in}}%
\pgfpathcurveto{\pgfqpoint{2.529836in}{3.984333in}}{\pgfqpoint{2.540435in}{3.988724in}}{\pgfqpoint{2.548249in}{3.996537in}}%
\pgfpathcurveto{\pgfqpoint{2.556062in}{4.004351in}}{\pgfqpoint{2.560452in}{4.014950in}}{\pgfqpoint{2.560452in}{4.026000in}}%
\pgfpathcurveto{\pgfqpoint{2.560452in}{4.037050in}}{\pgfqpoint{2.556062in}{4.047649in}}{\pgfqpoint{2.548249in}{4.055463in}}%
\pgfpathcurveto{\pgfqpoint{2.540435in}{4.063276in}}{\pgfqpoint{2.529836in}{4.067667in}}{\pgfqpoint{2.518786in}{4.067667in}}%
\pgfpathcurveto{\pgfqpoint{2.507736in}{4.067667in}}{\pgfqpoint{2.497137in}{4.063276in}}{\pgfqpoint{2.489323in}{4.055463in}}%
\pgfpathcurveto{\pgfqpoint{2.481509in}{4.047649in}}{\pgfqpoint{2.477119in}{4.037050in}}{\pgfqpoint{2.477119in}{4.026000in}}%
\pgfpathcurveto{\pgfqpoint{2.477119in}{4.014950in}}{\pgfqpoint{2.481509in}{4.004351in}}{\pgfqpoint{2.489323in}{3.996537in}}%
\pgfpathcurveto{\pgfqpoint{2.497137in}{3.988724in}}{\pgfqpoint{2.507736in}{3.984333in}}{\pgfqpoint{2.518786in}{3.984333in}}%
\pgfpathclose%
\pgfusepath{stroke,fill}%
\end{pgfscope}%
\begin{pgfscope}%
\pgfpathrectangle{\pgfqpoint{0.800000in}{0.528000in}}{\pgfqpoint{4.960000in}{3.696000in}}%
\pgfusepath{clip}%
\pgfsetbuttcap%
\pgfsetroundjoin%
\definecolor{currentfill}{rgb}{0.000000,0.000000,0.000000}%
\pgfsetfillcolor{currentfill}%
\pgfsetlinewidth{1.003750pt}%
\definecolor{currentstroke}{rgb}{0.000000,0.000000,0.000000}%
\pgfsetstrokecolor{currentstroke}%
\pgfsetdash{}{0pt}%
\pgfpathmoveto{\pgfqpoint{2.518786in}{3.984333in}}%
\pgfpathcurveto{\pgfqpoint{2.529836in}{3.984333in}}{\pgfqpoint{2.540435in}{3.988724in}}{\pgfqpoint{2.548249in}{3.996537in}}%
\pgfpathcurveto{\pgfqpoint{2.556062in}{4.004351in}}{\pgfqpoint{2.560452in}{4.014950in}}{\pgfqpoint{2.560452in}{4.026000in}}%
\pgfpathcurveto{\pgfqpoint{2.560452in}{4.037050in}}{\pgfqpoint{2.556062in}{4.047649in}}{\pgfqpoint{2.548249in}{4.055463in}}%
\pgfpathcurveto{\pgfqpoint{2.540435in}{4.063276in}}{\pgfqpoint{2.529836in}{4.067667in}}{\pgfqpoint{2.518786in}{4.067667in}}%
\pgfpathcurveto{\pgfqpoint{2.507736in}{4.067667in}}{\pgfqpoint{2.497137in}{4.063276in}}{\pgfqpoint{2.489323in}{4.055463in}}%
\pgfpathcurveto{\pgfqpoint{2.481509in}{4.047649in}}{\pgfqpoint{2.477119in}{4.037050in}}{\pgfqpoint{2.477119in}{4.026000in}}%
\pgfpathcurveto{\pgfqpoint{2.477119in}{4.014950in}}{\pgfqpoint{2.481509in}{4.004351in}}{\pgfqpoint{2.489323in}{3.996537in}}%
\pgfpathcurveto{\pgfqpoint{2.497137in}{3.988724in}}{\pgfqpoint{2.507736in}{3.984333in}}{\pgfqpoint{2.518786in}{3.984333in}}%
\pgfpathclose%
\pgfusepath{stroke,fill}%
\end{pgfscope}%
\begin{pgfscope}%
\pgfpathrectangle{\pgfqpoint{0.800000in}{0.528000in}}{\pgfqpoint{4.960000in}{3.696000in}}%
\pgfusepath{clip}%
\pgfsetbuttcap%
\pgfsetroundjoin%
\definecolor{currentfill}{rgb}{0.000000,0.000000,0.000000}%
\pgfsetfillcolor{currentfill}%
\pgfsetlinewidth{1.003750pt}%
\definecolor{currentstroke}{rgb}{0.000000,0.000000,0.000000}%
\pgfsetstrokecolor{currentstroke}%
\pgfsetdash{}{0pt}%
\pgfpathmoveto{\pgfqpoint{2.518786in}{3.984333in}}%
\pgfpathcurveto{\pgfqpoint{2.529836in}{3.984333in}}{\pgfqpoint{2.540435in}{3.988724in}}{\pgfqpoint{2.548249in}{3.996537in}}%
\pgfpathcurveto{\pgfqpoint{2.556062in}{4.004351in}}{\pgfqpoint{2.560452in}{4.014950in}}{\pgfqpoint{2.560452in}{4.026000in}}%
\pgfpathcurveto{\pgfqpoint{2.560452in}{4.037050in}}{\pgfqpoint{2.556062in}{4.047649in}}{\pgfqpoint{2.548249in}{4.055463in}}%
\pgfpathcurveto{\pgfqpoint{2.540435in}{4.063276in}}{\pgfqpoint{2.529836in}{4.067667in}}{\pgfqpoint{2.518786in}{4.067667in}}%
\pgfpathcurveto{\pgfqpoint{2.507736in}{4.067667in}}{\pgfqpoint{2.497137in}{4.063276in}}{\pgfqpoint{2.489323in}{4.055463in}}%
\pgfpathcurveto{\pgfqpoint{2.481509in}{4.047649in}}{\pgfqpoint{2.477119in}{4.037050in}}{\pgfqpoint{2.477119in}{4.026000in}}%
\pgfpathcurveto{\pgfqpoint{2.477119in}{4.014950in}}{\pgfqpoint{2.481509in}{4.004351in}}{\pgfqpoint{2.489323in}{3.996537in}}%
\pgfpathcurveto{\pgfqpoint{2.497137in}{3.988724in}}{\pgfqpoint{2.507736in}{3.984333in}}{\pgfqpoint{2.518786in}{3.984333in}}%
\pgfpathclose%
\pgfusepath{stroke,fill}%
\end{pgfscope}%
\begin{pgfscope}%
\pgfpathrectangle{\pgfqpoint{0.800000in}{0.528000in}}{\pgfqpoint{4.960000in}{3.696000in}}%
\pgfusepath{clip}%
\pgfsetbuttcap%
\pgfsetroundjoin%
\definecolor{currentfill}{rgb}{0.000000,0.000000,0.000000}%
\pgfsetfillcolor{currentfill}%
\pgfsetlinewidth{1.003750pt}%
\definecolor{currentstroke}{rgb}{0.000000,0.000000,0.000000}%
\pgfsetstrokecolor{currentstroke}%
\pgfsetdash{}{0pt}%
\pgfpathmoveto{\pgfqpoint{2.518786in}{3.984333in}}%
\pgfpathcurveto{\pgfqpoint{2.529836in}{3.984333in}}{\pgfqpoint{2.540435in}{3.988724in}}{\pgfqpoint{2.548249in}{3.996537in}}%
\pgfpathcurveto{\pgfqpoint{2.556062in}{4.004351in}}{\pgfqpoint{2.560452in}{4.014950in}}{\pgfqpoint{2.560452in}{4.026000in}}%
\pgfpathcurveto{\pgfqpoint{2.560452in}{4.037050in}}{\pgfqpoint{2.556062in}{4.047649in}}{\pgfqpoint{2.548249in}{4.055463in}}%
\pgfpathcurveto{\pgfqpoint{2.540435in}{4.063276in}}{\pgfqpoint{2.529836in}{4.067667in}}{\pgfqpoint{2.518786in}{4.067667in}}%
\pgfpathcurveto{\pgfqpoint{2.507736in}{4.067667in}}{\pgfqpoint{2.497137in}{4.063276in}}{\pgfqpoint{2.489323in}{4.055463in}}%
\pgfpathcurveto{\pgfqpoint{2.481509in}{4.047649in}}{\pgfqpoint{2.477119in}{4.037050in}}{\pgfqpoint{2.477119in}{4.026000in}}%
\pgfpathcurveto{\pgfqpoint{2.477119in}{4.014950in}}{\pgfqpoint{2.481509in}{4.004351in}}{\pgfqpoint{2.489323in}{3.996537in}}%
\pgfpathcurveto{\pgfqpoint{2.497137in}{3.988724in}}{\pgfqpoint{2.507736in}{3.984333in}}{\pgfqpoint{2.518786in}{3.984333in}}%
\pgfpathclose%
\pgfusepath{stroke,fill}%
\end{pgfscope}%
\begin{pgfscope}%
\pgfpathrectangle{\pgfqpoint{0.800000in}{0.528000in}}{\pgfqpoint{4.960000in}{3.696000in}}%
\pgfusepath{clip}%
\pgfsetbuttcap%
\pgfsetroundjoin%
\definecolor{currentfill}{rgb}{0.000000,0.000000,0.000000}%
\pgfsetfillcolor{currentfill}%
\pgfsetlinewidth{1.003750pt}%
\definecolor{currentstroke}{rgb}{0.000000,0.000000,0.000000}%
\pgfsetstrokecolor{currentstroke}%
\pgfsetdash{}{0pt}%
\pgfpathmoveto{\pgfqpoint{2.518786in}{3.984333in}}%
\pgfpathcurveto{\pgfqpoint{2.529836in}{3.984333in}}{\pgfqpoint{2.540435in}{3.988724in}}{\pgfqpoint{2.548249in}{3.996537in}}%
\pgfpathcurveto{\pgfqpoint{2.556062in}{4.004351in}}{\pgfqpoint{2.560452in}{4.014950in}}{\pgfqpoint{2.560452in}{4.026000in}}%
\pgfpathcurveto{\pgfqpoint{2.560452in}{4.037050in}}{\pgfqpoint{2.556062in}{4.047649in}}{\pgfqpoint{2.548249in}{4.055463in}}%
\pgfpathcurveto{\pgfqpoint{2.540435in}{4.063276in}}{\pgfqpoint{2.529836in}{4.067667in}}{\pgfqpoint{2.518786in}{4.067667in}}%
\pgfpathcurveto{\pgfqpoint{2.507736in}{4.067667in}}{\pgfqpoint{2.497137in}{4.063276in}}{\pgfqpoint{2.489323in}{4.055463in}}%
\pgfpathcurveto{\pgfqpoint{2.481509in}{4.047649in}}{\pgfqpoint{2.477119in}{4.037050in}}{\pgfqpoint{2.477119in}{4.026000in}}%
\pgfpathcurveto{\pgfqpoint{2.477119in}{4.014950in}}{\pgfqpoint{2.481509in}{4.004351in}}{\pgfqpoint{2.489323in}{3.996537in}}%
\pgfpathcurveto{\pgfqpoint{2.497137in}{3.988724in}}{\pgfqpoint{2.507736in}{3.984333in}}{\pgfqpoint{2.518786in}{3.984333in}}%
\pgfpathclose%
\pgfusepath{stroke,fill}%
\end{pgfscope}%
\begin{pgfscope}%
\pgfpathrectangle{\pgfqpoint{0.800000in}{0.528000in}}{\pgfqpoint{4.960000in}{3.696000in}}%
\pgfusepath{clip}%
\pgfsetbuttcap%
\pgfsetroundjoin%
\definecolor{currentfill}{rgb}{0.000000,0.000000,0.000000}%
\pgfsetfillcolor{currentfill}%
\pgfsetlinewidth{1.003750pt}%
\definecolor{currentstroke}{rgb}{0.000000,0.000000,0.000000}%
\pgfsetstrokecolor{currentstroke}%
\pgfsetdash{}{0pt}%
\pgfpathmoveto{\pgfqpoint{2.518786in}{3.984333in}}%
\pgfpathcurveto{\pgfqpoint{2.529836in}{3.984333in}}{\pgfqpoint{2.540435in}{3.988724in}}{\pgfqpoint{2.548249in}{3.996537in}}%
\pgfpathcurveto{\pgfqpoint{2.556062in}{4.004351in}}{\pgfqpoint{2.560452in}{4.014950in}}{\pgfqpoint{2.560452in}{4.026000in}}%
\pgfpathcurveto{\pgfqpoint{2.560452in}{4.037050in}}{\pgfqpoint{2.556062in}{4.047649in}}{\pgfqpoint{2.548249in}{4.055463in}}%
\pgfpathcurveto{\pgfqpoint{2.540435in}{4.063276in}}{\pgfqpoint{2.529836in}{4.067667in}}{\pgfqpoint{2.518786in}{4.067667in}}%
\pgfpathcurveto{\pgfqpoint{2.507736in}{4.067667in}}{\pgfqpoint{2.497137in}{4.063276in}}{\pgfqpoint{2.489323in}{4.055463in}}%
\pgfpathcurveto{\pgfqpoint{2.481509in}{4.047649in}}{\pgfqpoint{2.477119in}{4.037050in}}{\pgfqpoint{2.477119in}{4.026000in}}%
\pgfpathcurveto{\pgfqpoint{2.477119in}{4.014950in}}{\pgfqpoint{2.481509in}{4.004351in}}{\pgfqpoint{2.489323in}{3.996537in}}%
\pgfpathcurveto{\pgfqpoint{2.497137in}{3.988724in}}{\pgfqpoint{2.507736in}{3.984333in}}{\pgfqpoint{2.518786in}{3.984333in}}%
\pgfpathclose%
\pgfusepath{stroke,fill}%
\end{pgfscope}%
\begin{pgfscope}%
\pgfpathrectangle{\pgfqpoint{0.800000in}{0.528000in}}{\pgfqpoint{4.960000in}{3.696000in}}%
\pgfusepath{clip}%
\pgfsetbuttcap%
\pgfsetroundjoin%
\definecolor{currentfill}{rgb}{0.000000,0.000000,0.000000}%
\pgfsetfillcolor{currentfill}%
\pgfsetlinewidth{1.003750pt}%
\definecolor{currentstroke}{rgb}{0.000000,0.000000,0.000000}%
\pgfsetstrokecolor{currentstroke}%
\pgfsetdash{}{0pt}%
\pgfpathmoveto{\pgfqpoint{2.518786in}{3.984333in}}%
\pgfpathcurveto{\pgfqpoint{2.529836in}{3.984333in}}{\pgfqpoint{2.540435in}{3.988724in}}{\pgfqpoint{2.548249in}{3.996537in}}%
\pgfpathcurveto{\pgfqpoint{2.556062in}{4.004351in}}{\pgfqpoint{2.560452in}{4.014950in}}{\pgfqpoint{2.560452in}{4.026000in}}%
\pgfpathcurveto{\pgfqpoint{2.560452in}{4.037050in}}{\pgfqpoint{2.556062in}{4.047649in}}{\pgfqpoint{2.548249in}{4.055463in}}%
\pgfpathcurveto{\pgfqpoint{2.540435in}{4.063276in}}{\pgfqpoint{2.529836in}{4.067667in}}{\pgfqpoint{2.518786in}{4.067667in}}%
\pgfpathcurveto{\pgfqpoint{2.507736in}{4.067667in}}{\pgfqpoint{2.497137in}{4.063276in}}{\pgfqpoint{2.489323in}{4.055463in}}%
\pgfpathcurveto{\pgfqpoint{2.481509in}{4.047649in}}{\pgfqpoint{2.477119in}{4.037050in}}{\pgfqpoint{2.477119in}{4.026000in}}%
\pgfpathcurveto{\pgfqpoint{2.477119in}{4.014950in}}{\pgfqpoint{2.481509in}{4.004351in}}{\pgfqpoint{2.489323in}{3.996537in}}%
\pgfpathcurveto{\pgfqpoint{2.497137in}{3.988724in}}{\pgfqpoint{2.507736in}{3.984333in}}{\pgfqpoint{2.518786in}{3.984333in}}%
\pgfpathclose%
\pgfusepath{stroke,fill}%
\end{pgfscope}%
\begin{pgfscope}%
\pgfpathrectangle{\pgfqpoint{0.800000in}{0.528000in}}{\pgfqpoint{4.960000in}{3.696000in}}%
\pgfusepath{clip}%
\pgfsetbuttcap%
\pgfsetroundjoin%
\definecolor{currentfill}{rgb}{0.000000,0.000000,0.000000}%
\pgfsetfillcolor{currentfill}%
\pgfsetlinewidth{1.003750pt}%
\definecolor{currentstroke}{rgb}{0.000000,0.000000,0.000000}%
\pgfsetstrokecolor{currentstroke}%
\pgfsetdash{}{0pt}%
\pgfpathmoveto{\pgfqpoint{2.518786in}{3.984333in}}%
\pgfpathcurveto{\pgfqpoint{2.529836in}{3.984333in}}{\pgfqpoint{2.540435in}{3.988724in}}{\pgfqpoint{2.548249in}{3.996537in}}%
\pgfpathcurveto{\pgfqpoint{2.556062in}{4.004351in}}{\pgfqpoint{2.560452in}{4.014950in}}{\pgfqpoint{2.560452in}{4.026000in}}%
\pgfpathcurveto{\pgfqpoint{2.560452in}{4.037050in}}{\pgfqpoint{2.556062in}{4.047649in}}{\pgfqpoint{2.548249in}{4.055463in}}%
\pgfpathcurveto{\pgfqpoint{2.540435in}{4.063276in}}{\pgfqpoint{2.529836in}{4.067667in}}{\pgfqpoint{2.518786in}{4.067667in}}%
\pgfpathcurveto{\pgfqpoint{2.507736in}{4.067667in}}{\pgfqpoint{2.497137in}{4.063276in}}{\pgfqpoint{2.489323in}{4.055463in}}%
\pgfpathcurveto{\pgfqpoint{2.481509in}{4.047649in}}{\pgfqpoint{2.477119in}{4.037050in}}{\pgfqpoint{2.477119in}{4.026000in}}%
\pgfpathcurveto{\pgfqpoint{2.477119in}{4.014950in}}{\pgfqpoint{2.481509in}{4.004351in}}{\pgfqpoint{2.489323in}{3.996537in}}%
\pgfpathcurveto{\pgfqpoint{2.497137in}{3.988724in}}{\pgfqpoint{2.507736in}{3.984333in}}{\pgfqpoint{2.518786in}{3.984333in}}%
\pgfpathclose%
\pgfusepath{stroke,fill}%
\end{pgfscope}%
\begin{pgfscope}%
\pgfpathrectangle{\pgfqpoint{0.800000in}{0.528000in}}{\pgfqpoint{4.960000in}{3.696000in}}%
\pgfusepath{clip}%
\pgfsetbuttcap%
\pgfsetroundjoin%
\definecolor{currentfill}{rgb}{0.000000,0.000000,0.000000}%
\pgfsetfillcolor{currentfill}%
\pgfsetlinewidth{1.003750pt}%
\definecolor{currentstroke}{rgb}{0.000000,0.000000,0.000000}%
\pgfsetstrokecolor{currentstroke}%
\pgfsetdash{}{0pt}%
\pgfpathmoveto{\pgfqpoint{2.518786in}{3.984333in}}%
\pgfpathcurveto{\pgfqpoint{2.529836in}{3.984333in}}{\pgfqpoint{2.540435in}{3.988724in}}{\pgfqpoint{2.548249in}{3.996537in}}%
\pgfpathcurveto{\pgfqpoint{2.556062in}{4.004351in}}{\pgfqpoint{2.560452in}{4.014950in}}{\pgfqpoint{2.560452in}{4.026000in}}%
\pgfpathcurveto{\pgfqpoint{2.560452in}{4.037050in}}{\pgfqpoint{2.556062in}{4.047649in}}{\pgfqpoint{2.548249in}{4.055463in}}%
\pgfpathcurveto{\pgfqpoint{2.540435in}{4.063276in}}{\pgfqpoint{2.529836in}{4.067667in}}{\pgfqpoint{2.518786in}{4.067667in}}%
\pgfpathcurveto{\pgfqpoint{2.507736in}{4.067667in}}{\pgfqpoint{2.497137in}{4.063276in}}{\pgfqpoint{2.489323in}{4.055463in}}%
\pgfpathcurveto{\pgfqpoint{2.481509in}{4.047649in}}{\pgfqpoint{2.477119in}{4.037050in}}{\pgfqpoint{2.477119in}{4.026000in}}%
\pgfpathcurveto{\pgfqpoint{2.477119in}{4.014950in}}{\pgfqpoint{2.481509in}{4.004351in}}{\pgfqpoint{2.489323in}{3.996537in}}%
\pgfpathcurveto{\pgfqpoint{2.497137in}{3.988724in}}{\pgfqpoint{2.507736in}{3.984333in}}{\pgfqpoint{2.518786in}{3.984333in}}%
\pgfpathclose%
\pgfusepath{stroke,fill}%
\end{pgfscope}%
\begin{pgfscope}%
\pgfpathrectangle{\pgfqpoint{0.800000in}{0.528000in}}{\pgfqpoint{4.960000in}{3.696000in}}%
\pgfusepath{clip}%
\pgfsetbuttcap%
\pgfsetroundjoin%
\definecolor{currentfill}{rgb}{0.000000,0.000000,0.000000}%
\pgfsetfillcolor{currentfill}%
\pgfsetlinewidth{1.003750pt}%
\definecolor{currentstroke}{rgb}{0.000000,0.000000,0.000000}%
\pgfsetstrokecolor{currentstroke}%
\pgfsetdash{}{0pt}%
\pgfpathmoveto{\pgfqpoint{2.518786in}{3.984333in}}%
\pgfpathcurveto{\pgfqpoint{2.529836in}{3.984333in}}{\pgfqpoint{2.540435in}{3.988724in}}{\pgfqpoint{2.548249in}{3.996537in}}%
\pgfpathcurveto{\pgfqpoint{2.556062in}{4.004351in}}{\pgfqpoint{2.560452in}{4.014950in}}{\pgfqpoint{2.560452in}{4.026000in}}%
\pgfpathcurveto{\pgfqpoint{2.560452in}{4.037050in}}{\pgfqpoint{2.556062in}{4.047649in}}{\pgfqpoint{2.548249in}{4.055463in}}%
\pgfpathcurveto{\pgfqpoint{2.540435in}{4.063276in}}{\pgfqpoint{2.529836in}{4.067667in}}{\pgfqpoint{2.518786in}{4.067667in}}%
\pgfpathcurveto{\pgfqpoint{2.507736in}{4.067667in}}{\pgfqpoint{2.497137in}{4.063276in}}{\pgfqpoint{2.489323in}{4.055463in}}%
\pgfpathcurveto{\pgfqpoint{2.481509in}{4.047649in}}{\pgfqpoint{2.477119in}{4.037050in}}{\pgfqpoint{2.477119in}{4.026000in}}%
\pgfpathcurveto{\pgfqpoint{2.477119in}{4.014950in}}{\pgfqpoint{2.481509in}{4.004351in}}{\pgfqpoint{2.489323in}{3.996537in}}%
\pgfpathcurveto{\pgfqpoint{2.497137in}{3.988724in}}{\pgfqpoint{2.507736in}{3.984333in}}{\pgfqpoint{2.518786in}{3.984333in}}%
\pgfpathclose%
\pgfusepath{stroke,fill}%
\end{pgfscope}%
\begin{pgfscope}%
\pgfpathrectangle{\pgfqpoint{0.800000in}{0.528000in}}{\pgfqpoint{4.960000in}{3.696000in}}%
\pgfusepath{clip}%
\pgfsetbuttcap%
\pgfsetroundjoin%
\definecolor{currentfill}{rgb}{0.000000,0.000000,0.000000}%
\pgfsetfillcolor{currentfill}%
\pgfsetlinewidth{1.003750pt}%
\definecolor{currentstroke}{rgb}{0.000000,0.000000,0.000000}%
\pgfsetstrokecolor{currentstroke}%
\pgfsetdash{}{0pt}%
\pgfpathmoveto{\pgfqpoint{2.518786in}{3.984333in}}%
\pgfpathcurveto{\pgfqpoint{2.529836in}{3.984333in}}{\pgfqpoint{2.540435in}{3.988724in}}{\pgfqpoint{2.548249in}{3.996537in}}%
\pgfpathcurveto{\pgfqpoint{2.556062in}{4.004351in}}{\pgfqpoint{2.560452in}{4.014950in}}{\pgfqpoint{2.560452in}{4.026000in}}%
\pgfpathcurveto{\pgfqpoint{2.560452in}{4.037050in}}{\pgfqpoint{2.556062in}{4.047649in}}{\pgfqpoint{2.548249in}{4.055463in}}%
\pgfpathcurveto{\pgfqpoint{2.540435in}{4.063276in}}{\pgfqpoint{2.529836in}{4.067667in}}{\pgfqpoint{2.518786in}{4.067667in}}%
\pgfpathcurveto{\pgfqpoint{2.507736in}{4.067667in}}{\pgfqpoint{2.497137in}{4.063276in}}{\pgfqpoint{2.489323in}{4.055463in}}%
\pgfpathcurveto{\pgfqpoint{2.481509in}{4.047649in}}{\pgfqpoint{2.477119in}{4.037050in}}{\pgfqpoint{2.477119in}{4.026000in}}%
\pgfpathcurveto{\pgfqpoint{2.477119in}{4.014950in}}{\pgfqpoint{2.481509in}{4.004351in}}{\pgfqpoint{2.489323in}{3.996537in}}%
\pgfpathcurveto{\pgfqpoint{2.497137in}{3.988724in}}{\pgfqpoint{2.507736in}{3.984333in}}{\pgfqpoint{2.518786in}{3.984333in}}%
\pgfpathclose%
\pgfusepath{stroke,fill}%
\end{pgfscope}%
\begin{pgfscope}%
\pgfpathrectangle{\pgfqpoint{0.800000in}{0.528000in}}{\pgfqpoint{4.960000in}{3.696000in}}%
\pgfusepath{clip}%
\pgfsetbuttcap%
\pgfsetroundjoin%
\definecolor{currentfill}{rgb}{0.000000,0.000000,0.000000}%
\pgfsetfillcolor{currentfill}%
\pgfsetlinewidth{1.003750pt}%
\definecolor{currentstroke}{rgb}{0.000000,0.000000,0.000000}%
\pgfsetstrokecolor{currentstroke}%
\pgfsetdash{}{0pt}%
\pgfpathmoveto{\pgfqpoint{2.518786in}{3.984333in}}%
\pgfpathcurveto{\pgfqpoint{2.529836in}{3.984333in}}{\pgfqpoint{2.540435in}{3.988724in}}{\pgfqpoint{2.548249in}{3.996537in}}%
\pgfpathcurveto{\pgfqpoint{2.556062in}{4.004351in}}{\pgfqpoint{2.560452in}{4.014950in}}{\pgfqpoint{2.560452in}{4.026000in}}%
\pgfpathcurveto{\pgfqpoint{2.560452in}{4.037050in}}{\pgfqpoint{2.556062in}{4.047649in}}{\pgfqpoint{2.548249in}{4.055463in}}%
\pgfpathcurveto{\pgfqpoint{2.540435in}{4.063276in}}{\pgfqpoint{2.529836in}{4.067667in}}{\pgfqpoint{2.518786in}{4.067667in}}%
\pgfpathcurveto{\pgfqpoint{2.507736in}{4.067667in}}{\pgfqpoint{2.497137in}{4.063276in}}{\pgfqpoint{2.489323in}{4.055463in}}%
\pgfpathcurveto{\pgfqpoint{2.481509in}{4.047649in}}{\pgfqpoint{2.477119in}{4.037050in}}{\pgfqpoint{2.477119in}{4.026000in}}%
\pgfpathcurveto{\pgfqpoint{2.477119in}{4.014950in}}{\pgfqpoint{2.481509in}{4.004351in}}{\pgfqpoint{2.489323in}{3.996537in}}%
\pgfpathcurveto{\pgfqpoint{2.497137in}{3.988724in}}{\pgfqpoint{2.507736in}{3.984333in}}{\pgfqpoint{2.518786in}{3.984333in}}%
\pgfpathclose%
\pgfusepath{stroke,fill}%
\end{pgfscope}%
\begin{pgfscope}%
\pgfpathrectangle{\pgfqpoint{0.800000in}{0.528000in}}{\pgfqpoint{4.960000in}{3.696000in}}%
\pgfusepath{clip}%
\pgfsetbuttcap%
\pgfsetroundjoin%
\definecolor{currentfill}{rgb}{0.000000,0.000000,0.000000}%
\pgfsetfillcolor{currentfill}%
\pgfsetlinewidth{1.003750pt}%
\definecolor{currentstroke}{rgb}{0.000000,0.000000,0.000000}%
\pgfsetstrokecolor{currentstroke}%
\pgfsetdash{}{0pt}%
\pgfpathmoveto{\pgfqpoint{2.518786in}{3.984333in}}%
\pgfpathcurveto{\pgfqpoint{2.529836in}{3.984333in}}{\pgfqpoint{2.540435in}{3.988724in}}{\pgfqpoint{2.548249in}{3.996537in}}%
\pgfpathcurveto{\pgfqpoint{2.556062in}{4.004351in}}{\pgfqpoint{2.560452in}{4.014950in}}{\pgfqpoint{2.560452in}{4.026000in}}%
\pgfpathcurveto{\pgfqpoint{2.560452in}{4.037050in}}{\pgfqpoint{2.556062in}{4.047649in}}{\pgfqpoint{2.548249in}{4.055463in}}%
\pgfpathcurveto{\pgfqpoint{2.540435in}{4.063276in}}{\pgfqpoint{2.529836in}{4.067667in}}{\pgfqpoint{2.518786in}{4.067667in}}%
\pgfpathcurveto{\pgfqpoint{2.507736in}{4.067667in}}{\pgfqpoint{2.497137in}{4.063276in}}{\pgfqpoint{2.489323in}{4.055463in}}%
\pgfpathcurveto{\pgfqpoint{2.481509in}{4.047649in}}{\pgfqpoint{2.477119in}{4.037050in}}{\pgfqpoint{2.477119in}{4.026000in}}%
\pgfpathcurveto{\pgfqpoint{2.477119in}{4.014950in}}{\pgfqpoint{2.481509in}{4.004351in}}{\pgfqpoint{2.489323in}{3.996537in}}%
\pgfpathcurveto{\pgfqpoint{2.497137in}{3.988724in}}{\pgfqpoint{2.507736in}{3.984333in}}{\pgfqpoint{2.518786in}{3.984333in}}%
\pgfpathclose%
\pgfusepath{stroke,fill}%
\end{pgfscope}%
\begin{pgfscope}%
\pgfpathrectangle{\pgfqpoint{0.800000in}{0.528000in}}{\pgfqpoint{4.960000in}{3.696000in}}%
\pgfusepath{clip}%
\pgfsetbuttcap%
\pgfsetroundjoin%
\definecolor{currentfill}{rgb}{0.000000,0.000000,0.000000}%
\pgfsetfillcolor{currentfill}%
\pgfsetlinewidth{1.003750pt}%
\definecolor{currentstroke}{rgb}{0.000000,0.000000,0.000000}%
\pgfsetstrokecolor{currentstroke}%
\pgfsetdash{}{0pt}%
\pgfpathmoveto{\pgfqpoint{2.518786in}{3.984333in}}%
\pgfpathcurveto{\pgfqpoint{2.529836in}{3.984333in}}{\pgfqpoint{2.540435in}{3.988724in}}{\pgfqpoint{2.548249in}{3.996537in}}%
\pgfpathcurveto{\pgfqpoint{2.556062in}{4.004351in}}{\pgfqpoint{2.560452in}{4.014950in}}{\pgfqpoint{2.560452in}{4.026000in}}%
\pgfpathcurveto{\pgfqpoint{2.560452in}{4.037050in}}{\pgfqpoint{2.556062in}{4.047649in}}{\pgfqpoint{2.548249in}{4.055463in}}%
\pgfpathcurveto{\pgfqpoint{2.540435in}{4.063276in}}{\pgfqpoint{2.529836in}{4.067667in}}{\pgfqpoint{2.518786in}{4.067667in}}%
\pgfpathcurveto{\pgfqpoint{2.507736in}{4.067667in}}{\pgfqpoint{2.497137in}{4.063276in}}{\pgfqpoint{2.489323in}{4.055463in}}%
\pgfpathcurveto{\pgfqpoint{2.481509in}{4.047649in}}{\pgfqpoint{2.477119in}{4.037050in}}{\pgfqpoint{2.477119in}{4.026000in}}%
\pgfpathcurveto{\pgfqpoint{2.477119in}{4.014950in}}{\pgfqpoint{2.481509in}{4.004351in}}{\pgfqpoint{2.489323in}{3.996537in}}%
\pgfpathcurveto{\pgfqpoint{2.497137in}{3.988724in}}{\pgfqpoint{2.507736in}{3.984333in}}{\pgfqpoint{2.518786in}{3.984333in}}%
\pgfpathclose%
\pgfusepath{stroke,fill}%
\end{pgfscope}%
\begin{pgfscope}%
\pgfpathrectangle{\pgfqpoint{0.800000in}{0.528000in}}{\pgfqpoint{4.960000in}{3.696000in}}%
\pgfusepath{clip}%
\pgfsetbuttcap%
\pgfsetroundjoin%
\definecolor{currentfill}{rgb}{0.000000,0.000000,0.000000}%
\pgfsetfillcolor{currentfill}%
\pgfsetlinewidth{1.003750pt}%
\definecolor{currentstroke}{rgb}{0.000000,0.000000,0.000000}%
\pgfsetstrokecolor{currentstroke}%
\pgfsetdash{}{0pt}%
\pgfpathmoveto{\pgfqpoint{2.518786in}{3.984333in}}%
\pgfpathcurveto{\pgfqpoint{2.529836in}{3.984333in}}{\pgfqpoint{2.540435in}{3.988724in}}{\pgfqpoint{2.548249in}{3.996537in}}%
\pgfpathcurveto{\pgfqpoint{2.556062in}{4.004351in}}{\pgfqpoint{2.560452in}{4.014950in}}{\pgfqpoint{2.560452in}{4.026000in}}%
\pgfpathcurveto{\pgfqpoint{2.560452in}{4.037050in}}{\pgfqpoint{2.556062in}{4.047649in}}{\pgfqpoint{2.548249in}{4.055463in}}%
\pgfpathcurveto{\pgfqpoint{2.540435in}{4.063276in}}{\pgfqpoint{2.529836in}{4.067667in}}{\pgfqpoint{2.518786in}{4.067667in}}%
\pgfpathcurveto{\pgfqpoint{2.507736in}{4.067667in}}{\pgfqpoint{2.497137in}{4.063276in}}{\pgfqpoint{2.489323in}{4.055463in}}%
\pgfpathcurveto{\pgfqpoint{2.481509in}{4.047649in}}{\pgfqpoint{2.477119in}{4.037050in}}{\pgfqpoint{2.477119in}{4.026000in}}%
\pgfpathcurveto{\pgfqpoint{2.477119in}{4.014950in}}{\pgfqpoint{2.481509in}{4.004351in}}{\pgfqpoint{2.489323in}{3.996537in}}%
\pgfpathcurveto{\pgfqpoint{2.497137in}{3.988724in}}{\pgfqpoint{2.507736in}{3.984333in}}{\pgfqpoint{2.518786in}{3.984333in}}%
\pgfpathclose%
\pgfusepath{stroke,fill}%
\end{pgfscope}%
\begin{pgfscope}%
\pgfpathrectangle{\pgfqpoint{0.800000in}{0.528000in}}{\pgfqpoint{4.960000in}{3.696000in}}%
\pgfusepath{clip}%
\pgfsetbuttcap%
\pgfsetroundjoin%
\definecolor{currentfill}{rgb}{0.000000,0.000000,0.000000}%
\pgfsetfillcolor{currentfill}%
\pgfsetlinewidth{1.003750pt}%
\definecolor{currentstroke}{rgb}{0.000000,0.000000,0.000000}%
\pgfsetstrokecolor{currentstroke}%
\pgfsetdash{}{0pt}%
\pgfpathmoveto{\pgfqpoint{2.518786in}{3.984333in}}%
\pgfpathcurveto{\pgfqpoint{2.529836in}{3.984333in}}{\pgfqpoint{2.540435in}{3.988724in}}{\pgfqpoint{2.548249in}{3.996537in}}%
\pgfpathcurveto{\pgfqpoint{2.556062in}{4.004351in}}{\pgfqpoint{2.560452in}{4.014950in}}{\pgfqpoint{2.560452in}{4.026000in}}%
\pgfpathcurveto{\pgfqpoint{2.560452in}{4.037050in}}{\pgfqpoint{2.556062in}{4.047649in}}{\pgfqpoint{2.548249in}{4.055463in}}%
\pgfpathcurveto{\pgfqpoint{2.540435in}{4.063276in}}{\pgfqpoint{2.529836in}{4.067667in}}{\pgfqpoint{2.518786in}{4.067667in}}%
\pgfpathcurveto{\pgfqpoint{2.507736in}{4.067667in}}{\pgfqpoint{2.497137in}{4.063276in}}{\pgfqpoint{2.489323in}{4.055463in}}%
\pgfpathcurveto{\pgfqpoint{2.481509in}{4.047649in}}{\pgfqpoint{2.477119in}{4.037050in}}{\pgfqpoint{2.477119in}{4.026000in}}%
\pgfpathcurveto{\pgfqpoint{2.477119in}{4.014950in}}{\pgfqpoint{2.481509in}{4.004351in}}{\pgfqpoint{2.489323in}{3.996537in}}%
\pgfpathcurveto{\pgfqpoint{2.497137in}{3.988724in}}{\pgfqpoint{2.507736in}{3.984333in}}{\pgfqpoint{2.518786in}{3.984333in}}%
\pgfpathclose%
\pgfusepath{stroke,fill}%
\end{pgfscope}%
\begin{pgfscope}%
\pgfpathrectangle{\pgfqpoint{0.800000in}{0.528000in}}{\pgfqpoint{4.960000in}{3.696000in}}%
\pgfusepath{clip}%
\pgfsetbuttcap%
\pgfsetroundjoin%
\definecolor{currentfill}{rgb}{0.000000,0.000000,0.000000}%
\pgfsetfillcolor{currentfill}%
\pgfsetlinewidth{1.003750pt}%
\definecolor{currentstroke}{rgb}{0.000000,0.000000,0.000000}%
\pgfsetstrokecolor{currentstroke}%
\pgfsetdash{}{0pt}%
\pgfpathmoveto{\pgfqpoint{2.518786in}{3.984333in}}%
\pgfpathcurveto{\pgfqpoint{2.529836in}{3.984333in}}{\pgfqpoint{2.540435in}{3.988724in}}{\pgfqpoint{2.548249in}{3.996537in}}%
\pgfpathcurveto{\pgfqpoint{2.556062in}{4.004351in}}{\pgfqpoint{2.560452in}{4.014950in}}{\pgfqpoint{2.560452in}{4.026000in}}%
\pgfpathcurveto{\pgfqpoint{2.560452in}{4.037050in}}{\pgfqpoint{2.556062in}{4.047649in}}{\pgfqpoint{2.548249in}{4.055463in}}%
\pgfpathcurveto{\pgfqpoint{2.540435in}{4.063276in}}{\pgfqpoint{2.529836in}{4.067667in}}{\pgfqpoint{2.518786in}{4.067667in}}%
\pgfpathcurveto{\pgfqpoint{2.507736in}{4.067667in}}{\pgfqpoint{2.497137in}{4.063276in}}{\pgfqpoint{2.489323in}{4.055463in}}%
\pgfpathcurveto{\pgfqpoint{2.481509in}{4.047649in}}{\pgfqpoint{2.477119in}{4.037050in}}{\pgfqpoint{2.477119in}{4.026000in}}%
\pgfpathcurveto{\pgfqpoint{2.477119in}{4.014950in}}{\pgfqpoint{2.481509in}{4.004351in}}{\pgfqpoint{2.489323in}{3.996537in}}%
\pgfpathcurveto{\pgfqpoint{2.497137in}{3.988724in}}{\pgfqpoint{2.507736in}{3.984333in}}{\pgfqpoint{2.518786in}{3.984333in}}%
\pgfpathclose%
\pgfusepath{stroke,fill}%
\end{pgfscope}%
\begin{pgfscope}%
\pgfpathrectangle{\pgfqpoint{0.800000in}{0.528000in}}{\pgfqpoint{4.960000in}{3.696000in}}%
\pgfusepath{clip}%
\pgfsetbuttcap%
\pgfsetroundjoin%
\definecolor{currentfill}{rgb}{0.000000,0.000000,0.000000}%
\pgfsetfillcolor{currentfill}%
\pgfsetlinewidth{1.003750pt}%
\definecolor{currentstroke}{rgb}{0.000000,0.000000,0.000000}%
\pgfsetstrokecolor{currentstroke}%
\pgfsetdash{}{0pt}%
\pgfpathmoveto{\pgfqpoint{2.518786in}{3.984333in}}%
\pgfpathcurveto{\pgfqpoint{2.529836in}{3.984333in}}{\pgfqpoint{2.540435in}{3.988724in}}{\pgfqpoint{2.548249in}{3.996537in}}%
\pgfpathcurveto{\pgfqpoint{2.556062in}{4.004351in}}{\pgfqpoint{2.560452in}{4.014950in}}{\pgfqpoint{2.560452in}{4.026000in}}%
\pgfpathcurveto{\pgfqpoint{2.560452in}{4.037050in}}{\pgfqpoint{2.556062in}{4.047649in}}{\pgfqpoint{2.548249in}{4.055463in}}%
\pgfpathcurveto{\pgfqpoint{2.540435in}{4.063276in}}{\pgfqpoint{2.529836in}{4.067667in}}{\pgfqpoint{2.518786in}{4.067667in}}%
\pgfpathcurveto{\pgfqpoint{2.507736in}{4.067667in}}{\pgfqpoint{2.497137in}{4.063276in}}{\pgfqpoint{2.489323in}{4.055463in}}%
\pgfpathcurveto{\pgfqpoint{2.481509in}{4.047649in}}{\pgfqpoint{2.477119in}{4.037050in}}{\pgfqpoint{2.477119in}{4.026000in}}%
\pgfpathcurveto{\pgfqpoint{2.477119in}{4.014950in}}{\pgfqpoint{2.481509in}{4.004351in}}{\pgfqpoint{2.489323in}{3.996537in}}%
\pgfpathcurveto{\pgfqpoint{2.497137in}{3.988724in}}{\pgfqpoint{2.507736in}{3.984333in}}{\pgfqpoint{2.518786in}{3.984333in}}%
\pgfpathclose%
\pgfusepath{stroke,fill}%
\end{pgfscope}%
\begin{pgfscope}%
\pgfpathrectangle{\pgfqpoint{0.800000in}{0.528000in}}{\pgfqpoint{4.960000in}{3.696000in}}%
\pgfusepath{clip}%
\pgfsetbuttcap%
\pgfsetroundjoin%
\definecolor{currentfill}{rgb}{0.000000,0.000000,0.000000}%
\pgfsetfillcolor{currentfill}%
\pgfsetlinewidth{1.003750pt}%
\definecolor{currentstroke}{rgb}{0.000000,0.000000,0.000000}%
\pgfsetstrokecolor{currentstroke}%
\pgfsetdash{}{0pt}%
\pgfpathmoveto{\pgfqpoint{2.518786in}{3.984333in}}%
\pgfpathcurveto{\pgfqpoint{2.529836in}{3.984333in}}{\pgfqpoint{2.540435in}{3.988724in}}{\pgfqpoint{2.548249in}{3.996537in}}%
\pgfpathcurveto{\pgfqpoint{2.556062in}{4.004351in}}{\pgfqpoint{2.560452in}{4.014950in}}{\pgfqpoint{2.560452in}{4.026000in}}%
\pgfpathcurveto{\pgfqpoint{2.560452in}{4.037050in}}{\pgfqpoint{2.556062in}{4.047649in}}{\pgfqpoint{2.548249in}{4.055463in}}%
\pgfpathcurveto{\pgfqpoint{2.540435in}{4.063276in}}{\pgfqpoint{2.529836in}{4.067667in}}{\pgfqpoint{2.518786in}{4.067667in}}%
\pgfpathcurveto{\pgfqpoint{2.507736in}{4.067667in}}{\pgfqpoint{2.497137in}{4.063276in}}{\pgfqpoint{2.489323in}{4.055463in}}%
\pgfpathcurveto{\pgfqpoint{2.481509in}{4.047649in}}{\pgfqpoint{2.477119in}{4.037050in}}{\pgfqpoint{2.477119in}{4.026000in}}%
\pgfpathcurveto{\pgfqpoint{2.477119in}{4.014950in}}{\pgfqpoint{2.481509in}{4.004351in}}{\pgfqpoint{2.489323in}{3.996537in}}%
\pgfpathcurveto{\pgfqpoint{2.497137in}{3.988724in}}{\pgfqpoint{2.507736in}{3.984333in}}{\pgfqpoint{2.518786in}{3.984333in}}%
\pgfpathclose%
\pgfusepath{stroke,fill}%
\end{pgfscope}%
\begin{pgfscope}%
\pgfpathrectangle{\pgfqpoint{0.800000in}{0.528000in}}{\pgfqpoint{4.960000in}{3.696000in}}%
\pgfusepath{clip}%
\pgfsetbuttcap%
\pgfsetroundjoin%
\definecolor{currentfill}{rgb}{0.000000,0.000000,0.000000}%
\pgfsetfillcolor{currentfill}%
\pgfsetlinewidth{1.003750pt}%
\definecolor{currentstroke}{rgb}{0.000000,0.000000,0.000000}%
\pgfsetstrokecolor{currentstroke}%
\pgfsetdash{}{0pt}%
\pgfpathmoveto{\pgfqpoint{2.518786in}{3.984333in}}%
\pgfpathcurveto{\pgfqpoint{2.529836in}{3.984333in}}{\pgfqpoint{2.540435in}{3.988724in}}{\pgfqpoint{2.548249in}{3.996537in}}%
\pgfpathcurveto{\pgfqpoint{2.556062in}{4.004351in}}{\pgfqpoint{2.560452in}{4.014950in}}{\pgfqpoint{2.560452in}{4.026000in}}%
\pgfpathcurveto{\pgfqpoint{2.560452in}{4.037050in}}{\pgfqpoint{2.556062in}{4.047649in}}{\pgfqpoint{2.548249in}{4.055463in}}%
\pgfpathcurveto{\pgfqpoint{2.540435in}{4.063276in}}{\pgfqpoint{2.529836in}{4.067667in}}{\pgfqpoint{2.518786in}{4.067667in}}%
\pgfpathcurveto{\pgfqpoint{2.507736in}{4.067667in}}{\pgfqpoint{2.497137in}{4.063276in}}{\pgfqpoint{2.489323in}{4.055463in}}%
\pgfpathcurveto{\pgfqpoint{2.481509in}{4.047649in}}{\pgfqpoint{2.477119in}{4.037050in}}{\pgfqpoint{2.477119in}{4.026000in}}%
\pgfpathcurveto{\pgfqpoint{2.477119in}{4.014950in}}{\pgfqpoint{2.481509in}{4.004351in}}{\pgfqpoint{2.489323in}{3.996537in}}%
\pgfpathcurveto{\pgfqpoint{2.497137in}{3.988724in}}{\pgfqpoint{2.507736in}{3.984333in}}{\pgfqpoint{2.518786in}{3.984333in}}%
\pgfpathclose%
\pgfusepath{stroke,fill}%
\end{pgfscope}%
\begin{pgfscope}%
\pgfpathrectangle{\pgfqpoint{0.800000in}{0.528000in}}{\pgfqpoint{4.960000in}{3.696000in}}%
\pgfusepath{clip}%
\pgfsetbuttcap%
\pgfsetroundjoin%
\definecolor{currentfill}{rgb}{0.000000,0.000000,0.000000}%
\pgfsetfillcolor{currentfill}%
\pgfsetlinewidth{1.003750pt}%
\definecolor{currentstroke}{rgb}{0.000000,0.000000,0.000000}%
\pgfsetstrokecolor{currentstroke}%
\pgfsetdash{}{0pt}%
\pgfpathmoveto{\pgfqpoint{2.518786in}{3.984333in}}%
\pgfpathcurveto{\pgfqpoint{2.529836in}{3.984333in}}{\pgfqpoint{2.540435in}{3.988724in}}{\pgfqpoint{2.548249in}{3.996537in}}%
\pgfpathcurveto{\pgfqpoint{2.556062in}{4.004351in}}{\pgfqpoint{2.560452in}{4.014950in}}{\pgfqpoint{2.560452in}{4.026000in}}%
\pgfpathcurveto{\pgfqpoint{2.560452in}{4.037050in}}{\pgfqpoint{2.556062in}{4.047649in}}{\pgfqpoint{2.548249in}{4.055463in}}%
\pgfpathcurveto{\pgfqpoint{2.540435in}{4.063276in}}{\pgfqpoint{2.529836in}{4.067667in}}{\pgfqpoint{2.518786in}{4.067667in}}%
\pgfpathcurveto{\pgfqpoint{2.507736in}{4.067667in}}{\pgfqpoint{2.497137in}{4.063276in}}{\pgfqpoint{2.489323in}{4.055463in}}%
\pgfpathcurveto{\pgfqpoint{2.481509in}{4.047649in}}{\pgfqpoint{2.477119in}{4.037050in}}{\pgfqpoint{2.477119in}{4.026000in}}%
\pgfpathcurveto{\pgfqpoint{2.477119in}{4.014950in}}{\pgfqpoint{2.481509in}{4.004351in}}{\pgfqpoint{2.489323in}{3.996537in}}%
\pgfpathcurveto{\pgfqpoint{2.497137in}{3.988724in}}{\pgfqpoint{2.507736in}{3.984333in}}{\pgfqpoint{2.518786in}{3.984333in}}%
\pgfpathclose%
\pgfusepath{stroke,fill}%
\end{pgfscope}%
\begin{pgfscope}%
\pgfpathrectangle{\pgfqpoint{0.800000in}{0.528000in}}{\pgfqpoint{4.960000in}{3.696000in}}%
\pgfusepath{clip}%
\pgfsetbuttcap%
\pgfsetroundjoin%
\definecolor{currentfill}{rgb}{0.000000,0.000000,0.000000}%
\pgfsetfillcolor{currentfill}%
\pgfsetlinewidth{1.003750pt}%
\definecolor{currentstroke}{rgb}{0.000000,0.000000,0.000000}%
\pgfsetstrokecolor{currentstroke}%
\pgfsetdash{}{0pt}%
\pgfpathmoveto{\pgfqpoint{2.518786in}{3.984333in}}%
\pgfpathcurveto{\pgfqpoint{2.529836in}{3.984333in}}{\pgfqpoint{2.540435in}{3.988724in}}{\pgfqpoint{2.548249in}{3.996537in}}%
\pgfpathcurveto{\pgfqpoint{2.556062in}{4.004351in}}{\pgfqpoint{2.560452in}{4.014950in}}{\pgfqpoint{2.560452in}{4.026000in}}%
\pgfpathcurveto{\pgfqpoint{2.560452in}{4.037050in}}{\pgfqpoint{2.556062in}{4.047649in}}{\pgfqpoint{2.548249in}{4.055463in}}%
\pgfpathcurveto{\pgfqpoint{2.540435in}{4.063276in}}{\pgfqpoint{2.529836in}{4.067667in}}{\pgfqpoint{2.518786in}{4.067667in}}%
\pgfpathcurveto{\pgfqpoint{2.507736in}{4.067667in}}{\pgfqpoint{2.497137in}{4.063276in}}{\pgfqpoint{2.489323in}{4.055463in}}%
\pgfpathcurveto{\pgfqpoint{2.481509in}{4.047649in}}{\pgfqpoint{2.477119in}{4.037050in}}{\pgfqpoint{2.477119in}{4.026000in}}%
\pgfpathcurveto{\pgfqpoint{2.477119in}{4.014950in}}{\pgfqpoint{2.481509in}{4.004351in}}{\pgfqpoint{2.489323in}{3.996537in}}%
\pgfpathcurveto{\pgfqpoint{2.497137in}{3.988724in}}{\pgfqpoint{2.507736in}{3.984333in}}{\pgfqpoint{2.518786in}{3.984333in}}%
\pgfpathclose%
\pgfusepath{stroke,fill}%
\end{pgfscope}%
\begin{pgfscope}%
\pgfpathrectangle{\pgfqpoint{0.800000in}{0.528000in}}{\pgfqpoint{4.960000in}{3.696000in}}%
\pgfusepath{clip}%
\pgfsetbuttcap%
\pgfsetroundjoin%
\definecolor{currentfill}{rgb}{0.000000,0.000000,0.000000}%
\pgfsetfillcolor{currentfill}%
\pgfsetlinewidth{1.003750pt}%
\definecolor{currentstroke}{rgb}{0.000000,0.000000,0.000000}%
\pgfsetstrokecolor{currentstroke}%
\pgfsetdash{}{0pt}%
\pgfpathmoveto{\pgfqpoint{2.518786in}{3.984333in}}%
\pgfpathcurveto{\pgfqpoint{2.529836in}{3.984333in}}{\pgfqpoint{2.540435in}{3.988724in}}{\pgfqpoint{2.548249in}{3.996537in}}%
\pgfpathcurveto{\pgfqpoint{2.556062in}{4.004351in}}{\pgfqpoint{2.560452in}{4.014950in}}{\pgfqpoint{2.560452in}{4.026000in}}%
\pgfpathcurveto{\pgfqpoint{2.560452in}{4.037050in}}{\pgfqpoint{2.556062in}{4.047649in}}{\pgfqpoint{2.548249in}{4.055463in}}%
\pgfpathcurveto{\pgfqpoint{2.540435in}{4.063276in}}{\pgfqpoint{2.529836in}{4.067667in}}{\pgfqpoint{2.518786in}{4.067667in}}%
\pgfpathcurveto{\pgfqpoint{2.507736in}{4.067667in}}{\pgfqpoint{2.497137in}{4.063276in}}{\pgfqpoint{2.489323in}{4.055463in}}%
\pgfpathcurveto{\pgfqpoint{2.481509in}{4.047649in}}{\pgfqpoint{2.477119in}{4.037050in}}{\pgfqpoint{2.477119in}{4.026000in}}%
\pgfpathcurveto{\pgfqpoint{2.477119in}{4.014950in}}{\pgfqpoint{2.481509in}{4.004351in}}{\pgfqpoint{2.489323in}{3.996537in}}%
\pgfpathcurveto{\pgfqpoint{2.497137in}{3.988724in}}{\pgfqpoint{2.507736in}{3.984333in}}{\pgfqpoint{2.518786in}{3.984333in}}%
\pgfpathclose%
\pgfusepath{stroke,fill}%
\end{pgfscope}%
\begin{pgfscope}%
\pgfpathrectangle{\pgfqpoint{0.800000in}{0.528000in}}{\pgfqpoint{4.960000in}{3.696000in}}%
\pgfusepath{clip}%
\pgfsetbuttcap%
\pgfsetroundjoin%
\definecolor{currentfill}{rgb}{0.000000,0.000000,0.000000}%
\pgfsetfillcolor{currentfill}%
\pgfsetlinewidth{1.003750pt}%
\definecolor{currentstroke}{rgb}{0.000000,0.000000,0.000000}%
\pgfsetstrokecolor{currentstroke}%
\pgfsetdash{}{0pt}%
\pgfpathmoveto{\pgfqpoint{2.518786in}{3.984333in}}%
\pgfpathcurveto{\pgfqpoint{2.529836in}{3.984333in}}{\pgfqpoint{2.540435in}{3.988724in}}{\pgfqpoint{2.548249in}{3.996537in}}%
\pgfpathcurveto{\pgfqpoint{2.556062in}{4.004351in}}{\pgfqpoint{2.560452in}{4.014950in}}{\pgfqpoint{2.560452in}{4.026000in}}%
\pgfpathcurveto{\pgfqpoint{2.560452in}{4.037050in}}{\pgfqpoint{2.556062in}{4.047649in}}{\pgfqpoint{2.548249in}{4.055463in}}%
\pgfpathcurveto{\pgfqpoint{2.540435in}{4.063276in}}{\pgfqpoint{2.529836in}{4.067667in}}{\pgfqpoint{2.518786in}{4.067667in}}%
\pgfpathcurveto{\pgfqpoint{2.507736in}{4.067667in}}{\pgfqpoint{2.497137in}{4.063276in}}{\pgfqpoint{2.489323in}{4.055463in}}%
\pgfpathcurveto{\pgfqpoint{2.481509in}{4.047649in}}{\pgfqpoint{2.477119in}{4.037050in}}{\pgfqpoint{2.477119in}{4.026000in}}%
\pgfpathcurveto{\pgfqpoint{2.477119in}{4.014950in}}{\pgfqpoint{2.481509in}{4.004351in}}{\pgfqpoint{2.489323in}{3.996537in}}%
\pgfpathcurveto{\pgfqpoint{2.497137in}{3.988724in}}{\pgfqpoint{2.507736in}{3.984333in}}{\pgfqpoint{2.518786in}{3.984333in}}%
\pgfpathclose%
\pgfusepath{stroke,fill}%
\end{pgfscope}%
\begin{pgfscope}%
\pgfpathrectangle{\pgfqpoint{0.800000in}{0.528000in}}{\pgfqpoint{4.960000in}{3.696000in}}%
\pgfusepath{clip}%
\pgfsetbuttcap%
\pgfsetroundjoin%
\definecolor{currentfill}{rgb}{0.000000,0.000000,0.000000}%
\pgfsetfillcolor{currentfill}%
\pgfsetlinewidth{1.003750pt}%
\definecolor{currentstroke}{rgb}{0.000000,0.000000,0.000000}%
\pgfsetstrokecolor{currentstroke}%
\pgfsetdash{}{0pt}%
\pgfpathmoveto{\pgfqpoint{2.518786in}{3.984333in}}%
\pgfpathcurveto{\pgfqpoint{2.529836in}{3.984333in}}{\pgfqpoint{2.540435in}{3.988724in}}{\pgfqpoint{2.548249in}{3.996537in}}%
\pgfpathcurveto{\pgfqpoint{2.556062in}{4.004351in}}{\pgfqpoint{2.560452in}{4.014950in}}{\pgfqpoint{2.560452in}{4.026000in}}%
\pgfpathcurveto{\pgfqpoint{2.560452in}{4.037050in}}{\pgfqpoint{2.556062in}{4.047649in}}{\pgfqpoint{2.548249in}{4.055463in}}%
\pgfpathcurveto{\pgfqpoint{2.540435in}{4.063276in}}{\pgfqpoint{2.529836in}{4.067667in}}{\pgfqpoint{2.518786in}{4.067667in}}%
\pgfpathcurveto{\pgfqpoint{2.507736in}{4.067667in}}{\pgfqpoint{2.497137in}{4.063276in}}{\pgfqpoint{2.489323in}{4.055463in}}%
\pgfpathcurveto{\pgfqpoint{2.481509in}{4.047649in}}{\pgfqpoint{2.477119in}{4.037050in}}{\pgfqpoint{2.477119in}{4.026000in}}%
\pgfpathcurveto{\pgfqpoint{2.477119in}{4.014950in}}{\pgfqpoint{2.481509in}{4.004351in}}{\pgfqpoint{2.489323in}{3.996537in}}%
\pgfpathcurveto{\pgfqpoint{2.497137in}{3.988724in}}{\pgfqpoint{2.507736in}{3.984333in}}{\pgfqpoint{2.518786in}{3.984333in}}%
\pgfpathclose%
\pgfusepath{stroke,fill}%
\end{pgfscope}%
\begin{pgfscope}%
\pgfpathrectangle{\pgfqpoint{0.800000in}{0.528000in}}{\pgfqpoint{4.960000in}{3.696000in}}%
\pgfusepath{clip}%
\pgfsetbuttcap%
\pgfsetroundjoin%
\definecolor{currentfill}{rgb}{0.000000,0.000000,0.000000}%
\pgfsetfillcolor{currentfill}%
\pgfsetlinewidth{1.003750pt}%
\definecolor{currentstroke}{rgb}{0.000000,0.000000,0.000000}%
\pgfsetstrokecolor{currentstroke}%
\pgfsetdash{}{0pt}%
\pgfpathmoveto{\pgfqpoint{2.518786in}{3.984333in}}%
\pgfpathcurveto{\pgfqpoint{2.529836in}{3.984333in}}{\pgfqpoint{2.540435in}{3.988724in}}{\pgfqpoint{2.548249in}{3.996537in}}%
\pgfpathcurveto{\pgfqpoint{2.556062in}{4.004351in}}{\pgfqpoint{2.560452in}{4.014950in}}{\pgfqpoint{2.560452in}{4.026000in}}%
\pgfpathcurveto{\pgfqpoint{2.560452in}{4.037050in}}{\pgfqpoint{2.556062in}{4.047649in}}{\pgfqpoint{2.548249in}{4.055463in}}%
\pgfpathcurveto{\pgfqpoint{2.540435in}{4.063276in}}{\pgfqpoint{2.529836in}{4.067667in}}{\pgfqpoint{2.518786in}{4.067667in}}%
\pgfpathcurveto{\pgfqpoint{2.507736in}{4.067667in}}{\pgfqpoint{2.497137in}{4.063276in}}{\pgfqpoint{2.489323in}{4.055463in}}%
\pgfpathcurveto{\pgfqpoint{2.481509in}{4.047649in}}{\pgfqpoint{2.477119in}{4.037050in}}{\pgfqpoint{2.477119in}{4.026000in}}%
\pgfpathcurveto{\pgfqpoint{2.477119in}{4.014950in}}{\pgfqpoint{2.481509in}{4.004351in}}{\pgfqpoint{2.489323in}{3.996537in}}%
\pgfpathcurveto{\pgfqpoint{2.497137in}{3.988724in}}{\pgfqpoint{2.507736in}{3.984333in}}{\pgfqpoint{2.518786in}{3.984333in}}%
\pgfpathclose%
\pgfusepath{stroke,fill}%
\end{pgfscope}%
\begin{pgfscope}%
\pgfpathrectangle{\pgfqpoint{0.800000in}{0.528000in}}{\pgfqpoint{4.960000in}{3.696000in}}%
\pgfusepath{clip}%
\pgfsetbuttcap%
\pgfsetroundjoin%
\definecolor{currentfill}{rgb}{0.000000,0.000000,0.000000}%
\pgfsetfillcolor{currentfill}%
\pgfsetlinewidth{1.003750pt}%
\definecolor{currentstroke}{rgb}{0.000000,0.000000,0.000000}%
\pgfsetstrokecolor{currentstroke}%
\pgfsetdash{}{0pt}%
\pgfpathmoveto{\pgfqpoint{2.518786in}{3.984333in}}%
\pgfpathcurveto{\pgfqpoint{2.529836in}{3.984333in}}{\pgfqpoint{2.540435in}{3.988724in}}{\pgfqpoint{2.548249in}{3.996537in}}%
\pgfpathcurveto{\pgfqpoint{2.556062in}{4.004351in}}{\pgfqpoint{2.560452in}{4.014950in}}{\pgfqpoint{2.560452in}{4.026000in}}%
\pgfpathcurveto{\pgfqpoint{2.560452in}{4.037050in}}{\pgfqpoint{2.556062in}{4.047649in}}{\pgfqpoint{2.548249in}{4.055463in}}%
\pgfpathcurveto{\pgfqpoint{2.540435in}{4.063276in}}{\pgfqpoint{2.529836in}{4.067667in}}{\pgfqpoint{2.518786in}{4.067667in}}%
\pgfpathcurveto{\pgfqpoint{2.507736in}{4.067667in}}{\pgfqpoint{2.497137in}{4.063276in}}{\pgfqpoint{2.489323in}{4.055463in}}%
\pgfpathcurveto{\pgfqpoint{2.481509in}{4.047649in}}{\pgfqpoint{2.477119in}{4.037050in}}{\pgfqpoint{2.477119in}{4.026000in}}%
\pgfpathcurveto{\pgfqpoint{2.477119in}{4.014950in}}{\pgfqpoint{2.481509in}{4.004351in}}{\pgfqpoint{2.489323in}{3.996537in}}%
\pgfpathcurveto{\pgfqpoint{2.497137in}{3.988724in}}{\pgfqpoint{2.507736in}{3.984333in}}{\pgfqpoint{2.518786in}{3.984333in}}%
\pgfpathclose%
\pgfusepath{stroke,fill}%
\end{pgfscope}%
\begin{pgfscope}%
\pgfpathrectangle{\pgfqpoint{0.800000in}{0.528000in}}{\pgfqpoint{4.960000in}{3.696000in}}%
\pgfusepath{clip}%
\pgfsetbuttcap%
\pgfsetroundjoin%
\definecolor{currentfill}{rgb}{0.000000,0.000000,0.000000}%
\pgfsetfillcolor{currentfill}%
\pgfsetlinewidth{1.003750pt}%
\definecolor{currentstroke}{rgb}{0.000000,0.000000,0.000000}%
\pgfsetstrokecolor{currentstroke}%
\pgfsetdash{}{0pt}%
\pgfpathmoveto{\pgfqpoint{2.518786in}{3.984333in}}%
\pgfpathcurveto{\pgfqpoint{2.529836in}{3.984333in}}{\pgfqpoint{2.540435in}{3.988724in}}{\pgfqpoint{2.548249in}{3.996537in}}%
\pgfpathcurveto{\pgfqpoint{2.556062in}{4.004351in}}{\pgfqpoint{2.560452in}{4.014950in}}{\pgfqpoint{2.560452in}{4.026000in}}%
\pgfpathcurveto{\pgfqpoint{2.560452in}{4.037050in}}{\pgfqpoint{2.556062in}{4.047649in}}{\pgfqpoint{2.548249in}{4.055463in}}%
\pgfpathcurveto{\pgfqpoint{2.540435in}{4.063276in}}{\pgfqpoint{2.529836in}{4.067667in}}{\pgfqpoint{2.518786in}{4.067667in}}%
\pgfpathcurveto{\pgfqpoint{2.507736in}{4.067667in}}{\pgfqpoint{2.497137in}{4.063276in}}{\pgfqpoint{2.489323in}{4.055463in}}%
\pgfpathcurveto{\pgfqpoint{2.481509in}{4.047649in}}{\pgfqpoint{2.477119in}{4.037050in}}{\pgfqpoint{2.477119in}{4.026000in}}%
\pgfpathcurveto{\pgfqpoint{2.477119in}{4.014950in}}{\pgfqpoint{2.481509in}{4.004351in}}{\pgfqpoint{2.489323in}{3.996537in}}%
\pgfpathcurveto{\pgfqpoint{2.497137in}{3.988724in}}{\pgfqpoint{2.507736in}{3.984333in}}{\pgfqpoint{2.518786in}{3.984333in}}%
\pgfpathclose%
\pgfusepath{stroke,fill}%
\end{pgfscope}%
\begin{pgfscope}%
\pgfpathrectangle{\pgfqpoint{0.800000in}{0.528000in}}{\pgfqpoint{4.960000in}{3.696000in}}%
\pgfusepath{clip}%
\pgfsetbuttcap%
\pgfsetroundjoin%
\definecolor{currentfill}{rgb}{0.000000,0.000000,0.000000}%
\pgfsetfillcolor{currentfill}%
\pgfsetlinewidth{1.003750pt}%
\definecolor{currentstroke}{rgb}{0.000000,0.000000,0.000000}%
\pgfsetstrokecolor{currentstroke}%
\pgfsetdash{}{0pt}%
\pgfpathmoveto{\pgfqpoint{2.518786in}{3.984333in}}%
\pgfpathcurveto{\pgfqpoint{2.529836in}{3.984333in}}{\pgfqpoint{2.540435in}{3.988724in}}{\pgfqpoint{2.548249in}{3.996537in}}%
\pgfpathcurveto{\pgfqpoint{2.556062in}{4.004351in}}{\pgfqpoint{2.560452in}{4.014950in}}{\pgfqpoint{2.560452in}{4.026000in}}%
\pgfpathcurveto{\pgfqpoint{2.560452in}{4.037050in}}{\pgfqpoint{2.556062in}{4.047649in}}{\pgfqpoint{2.548249in}{4.055463in}}%
\pgfpathcurveto{\pgfqpoint{2.540435in}{4.063276in}}{\pgfqpoint{2.529836in}{4.067667in}}{\pgfqpoint{2.518786in}{4.067667in}}%
\pgfpathcurveto{\pgfqpoint{2.507736in}{4.067667in}}{\pgfqpoint{2.497137in}{4.063276in}}{\pgfqpoint{2.489323in}{4.055463in}}%
\pgfpathcurveto{\pgfqpoint{2.481509in}{4.047649in}}{\pgfqpoint{2.477119in}{4.037050in}}{\pgfqpoint{2.477119in}{4.026000in}}%
\pgfpathcurveto{\pgfqpoint{2.477119in}{4.014950in}}{\pgfqpoint{2.481509in}{4.004351in}}{\pgfqpoint{2.489323in}{3.996537in}}%
\pgfpathcurveto{\pgfqpoint{2.497137in}{3.988724in}}{\pgfqpoint{2.507736in}{3.984333in}}{\pgfqpoint{2.518786in}{3.984333in}}%
\pgfpathclose%
\pgfusepath{stroke,fill}%
\end{pgfscope}%
\begin{pgfscope}%
\pgfpathrectangle{\pgfqpoint{0.800000in}{0.528000in}}{\pgfqpoint{4.960000in}{3.696000in}}%
\pgfusepath{clip}%
\pgfsetbuttcap%
\pgfsetroundjoin%
\definecolor{currentfill}{rgb}{0.000000,0.000000,0.000000}%
\pgfsetfillcolor{currentfill}%
\pgfsetlinewidth{1.003750pt}%
\definecolor{currentstroke}{rgb}{0.000000,0.000000,0.000000}%
\pgfsetstrokecolor{currentstroke}%
\pgfsetdash{}{0pt}%
\pgfpathmoveto{\pgfqpoint{2.518786in}{3.984333in}}%
\pgfpathcurveto{\pgfqpoint{2.529836in}{3.984333in}}{\pgfqpoint{2.540435in}{3.988724in}}{\pgfqpoint{2.548249in}{3.996537in}}%
\pgfpathcurveto{\pgfqpoint{2.556062in}{4.004351in}}{\pgfqpoint{2.560452in}{4.014950in}}{\pgfqpoint{2.560452in}{4.026000in}}%
\pgfpathcurveto{\pgfqpoint{2.560452in}{4.037050in}}{\pgfqpoint{2.556062in}{4.047649in}}{\pgfqpoint{2.548249in}{4.055463in}}%
\pgfpathcurveto{\pgfqpoint{2.540435in}{4.063276in}}{\pgfqpoint{2.529836in}{4.067667in}}{\pgfqpoint{2.518786in}{4.067667in}}%
\pgfpathcurveto{\pgfqpoint{2.507736in}{4.067667in}}{\pgfqpoint{2.497137in}{4.063276in}}{\pgfqpoint{2.489323in}{4.055463in}}%
\pgfpathcurveto{\pgfqpoint{2.481509in}{4.047649in}}{\pgfqpoint{2.477119in}{4.037050in}}{\pgfqpoint{2.477119in}{4.026000in}}%
\pgfpathcurveto{\pgfqpoint{2.477119in}{4.014950in}}{\pgfqpoint{2.481509in}{4.004351in}}{\pgfqpoint{2.489323in}{3.996537in}}%
\pgfpathcurveto{\pgfqpoint{2.497137in}{3.988724in}}{\pgfqpoint{2.507736in}{3.984333in}}{\pgfqpoint{2.518786in}{3.984333in}}%
\pgfpathclose%
\pgfusepath{stroke,fill}%
\end{pgfscope}%
\begin{pgfscope}%
\pgfpathrectangle{\pgfqpoint{0.800000in}{0.528000in}}{\pgfqpoint{4.960000in}{3.696000in}}%
\pgfusepath{clip}%
\pgfsetbuttcap%
\pgfsetroundjoin%
\definecolor{currentfill}{rgb}{0.000000,0.000000,0.000000}%
\pgfsetfillcolor{currentfill}%
\pgfsetlinewidth{1.003750pt}%
\definecolor{currentstroke}{rgb}{0.000000,0.000000,0.000000}%
\pgfsetstrokecolor{currentstroke}%
\pgfsetdash{}{0pt}%
\pgfpathmoveto{\pgfqpoint{2.518786in}{3.984333in}}%
\pgfpathcurveto{\pgfqpoint{2.529836in}{3.984333in}}{\pgfqpoint{2.540435in}{3.988724in}}{\pgfqpoint{2.548249in}{3.996537in}}%
\pgfpathcurveto{\pgfqpoint{2.556062in}{4.004351in}}{\pgfqpoint{2.560452in}{4.014950in}}{\pgfqpoint{2.560452in}{4.026000in}}%
\pgfpathcurveto{\pgfqpoint{2.560452in}{4.037050in}}{\pgfqpoint{2.556062in}{4.047649in}}{\pgfqpoint{2.548249in}{4.055463in}}%
\pgfpathcurveto{\pgfqpoint{2.540435in}{4.063276in}}{\pgfqpoint{2.529836in}{4.067667in}}{\pgfqpoint{2.518786in}{4.067667in}}%
\pgfpathcurveto{\pgfqpoint{2.507736in}{4.067667in}}{\pgfqpoint{2.497137in}{4.063276in}}{\pgfqpoint{2.489323in}{4.055463in}}%
\pgfpathcurveto{\pgfqpoint{2.481509in}{4.047649in}}{\pgfqpoint{2.477119in}{4.037050in}}{\pgfqpoint{2.477119in}{4.026000in}}%
\pgfpathcurveto{\pgfqpoint{2.477119in}{4.014950in}}{\pgfqpoint{2.481509in}{4.004351in}}{\pgfqpoint{2.489323in}{3.996537in}}%
\pgfpathcurveto{\pgfqpoint{2.497137in}{3.988724in}}{\pgfqpoint{2.507736in}{3.984333in}}{\pgfqpoint{2.518786in}{3.984333in}}%
\pgfpathclose%
\pgfusepath{stroke,fill}%
\end{pgfscope}%
\begin{pgfscope}%
\pgfpathrectangle{\pgfqpoint{0.800000in}{0.528000in}}{\pgfqpoint{4.960000in}{3.696000in}}%
\pgfusepath{clip}%
\pgfsetbuttcap%
\pgfsetroundjoin%
\definecolor{currentfill}{rgb}{0.000000,0.000000,0.000000}%
\pgfsetfillcolor{currentfill}%
\pgfsetlinewidth{1.003750pt}%
\definecolor{currentstroke}{rgb}{0.000000,0.000000,0.000000}%
\pgfsetstrokecolor{currentstroke}%
\pgfsetdash{}{0pt}%
\pgfpathmoveto{\pgfqpoint{2.518786in}{3.984333in}}%
\pgfpathcurveto{\pgfqpoint{2.529836in}{3.984333in}}{\pgfqpoint{2.540435in}{3.988724in}}{\pgfqpoint{2.548249in}{3.996537in}}%
\pgfpathcurveto{\pgfqpoint{2.556062in}{4.004351in}}{\pgfqpoint{2.560452in}{4.014950in}}{\pgfqpoint{2.560452in}{4.026000in}}%
\pgfpathcurveto{\pgfqpoint{2.560452in}{4.037050in}}{\pgfqpoint{2.556062in}{4.047649in}}{\pgfqpoint{2.548249in}{4.055463in}}%
\pgfpathcurveto{\pgfqpoint{2.540435in}{4.063276in}}{\pgfqpoint{2.529836in}{4.067667in}}{\pgfqpoint{2.518786in}{4.067667in}}%
\pgfpathcurveto{\pgfqpoint{2.507736in}{4.067667in}}{\pgfqpoint{2.497137in}{4.063276in}}{\pgfqpoint{2.489323in}{4.055463in}}%
\pgfpathcurveto{\pgfqpoint{2.481509in}{4.047649in}}{\pgfqpoint{2.477119in}{4.037050in}}{\pgfqpoint{2.477119in}{4.026000in}}%
\pgfpathcurveto{\pgfqpoint{2.477119in}{4.014950in}}{\pgfqpoint{2.481509in}{4.004351in}}{\pgfqpoint{2.489323in}{3.996537in}}%
\pgfpathcurveto{\pgfqpoint{2.497137in}{3.988724in}}{\pgfqpoint{2.507736in}{3.984333in}}{\pgfqpoint{2.518786in}{3.984333in}}%
\pgfpathclose%
\pgfusepath{stroke,fill}%
\end{pgfscope}%
\begin{pgfscope}%
\pgfpathrectangle{\pgfqpoint{0.800000in}{0.528000in}}{\pgfqpoint{4.960000in}{3.696000in}}%
\pgfusepath{clip}%
\pgfsetbuttcap%
\pgfsetroundjoin%
\definecolor{currentfill}{rgb}{0.000000,0.000000,0.000000}%
\pgfsetfillcolor{currentfill}%
\pgfsetlinewidth{1.003750pt}%
\definecolor{currentstroke}{rgb}{0.000000,0.000000,0.000000}%
\pgfsetstrokecolor{currentstroke}%
\pgfsetdash{}{0pt}%
\pgfpathmoveto{\pgfqpoint{2.518786in}{3.984333in}}%
\pgfpathcurveto{\pgfqpoint{2.529836in}{3.984333in}}{\pgfqpoint{2.540435in}{3.988724in}}{\pgfqpoint{2.548249in}{3.996537in}}%
\pgfpathcurveto{\pgfqpoint{2.556062in}{4.004351in}}{\pgfqpoint{2.560452in}{4.014950in}}{\pgfqpoint{2.560452in}{4.026000in}}%
\pgfpathcurveto{\pgfqpoint{2.560452in}{4.037050in}}{\pgfqpoint{2.556062in}{4.047649in}}{\pgfqpoint{2.548249in}{4.055463in}}%
\pgfpathcurveto{\pgfqpoint{2.540435in}{4.063276in}}{\pgfqpoint{2.529836in}{4.067667in}}{\pgfqpoint{2.518786in}{4.067667in}}%
\pgfpathcurveto{\pgfqpoint{2.507736in}{4.067667in}}{\pgfqpoint{2.497137in}{4.063276in}}{\pgfqpoint{2.489323in}{4.055463in}}%
\pgfpathcurveto{\pgfqpoint{2.481509in}{4.047649in}}{\pgfqpoint{2.477119in}{4.037050in}}{\pgfqpoint{2.477119in}{4.026000in}}%
\pgfpathcurveto{\pgfqpoint{2.477119in}{4.014950in}}{\pgfqpoint{2.481509in}{4.004351in}}{\pgfqpoint{2.489323in}{3.996537in}}%
\pgfpathcurveto{\pgfqpoint{2.497137in}{3.988724in}}{\pgfqpoint{2.507736in}{3.984333in}}{\pgfqpoint{2.518786in}{3.984333in}}%
\pgfpathclose%
\pgfusepath{stroke,fill}%
\end{pgfscope}%
\begin{pgfscope}%
\pgfpathrectangle{\pgfqpoint{0.800000in}{0.528000in}}{\pgfqpoint{4.960000in}{3.696000in}}%
\pgfusepath{clip}%
\pgfsetbuttcap%
\pgfsetroundjoin%
\definecolor{currentfill}{rgb}{0.000000,0.000000,0.000000}%
\pgfsetfillcolor{currentfill}%
\pgfsetlinewidth{1.003750pt}%
\definecolor{currentstroke}{rgb}{0.000000,0.000000,0.000000}%
\pgfsetstrokecolor{currentstroke}%
\pgfsetdash{}{0pt}%
\pgfpathmoveto{\pgfqpoint{2.518786in}{3.984333in}}%
\pgfpathcurveto{\pgfqpoint{2.529836in}{3.984333in}}{\pgfqpoint{2.540435in}{3.988724in}}{\pgfqpoint{2.548249in}{3.996537in}}%
\pgfpathcurveto{\pgfqpoint{2.556062in}{4.004351in}}{\pgfqpoint{2.560452in}{4.014950in}}{\pgfqpoint{2.560452in}{4.026000in}}%
\pgfpathcurveto{\pgfqpoint{2.560452in}{4.037050in}}{\pgfqpoint{2.556062in}{4.047649in}}{\pgfqpoint{2.548249in}{4.055463in}}%
\pgfpathcurveto{\pgfqpoint{2.540435in}{4.063276in}}{\pgfqpoint{2.529836in}{4.067667in}}{\pgfqpoint{2.518786in}{4.067667in}}%
\pgfpathcurveto{\pgfqpoint{2.507736in}{4.067667in}}{\pgfqpoint{2.497137in}{4.063276in}}{\pgfqpoint{2.489323in}{4.055463in}}%
\pgfpathcurveto{\pgfqpoint{2.481509in}{4.047649in}}{\pgfqpoint{2.477119in}{4.037050in}}{\pgfqpoint{2.477119in}{4.026000in}}%
\pgfpathcurveto{\pgfqpoint{2.477119in}{4.014950in}}{\pgfqpoint{2.481509in}{4.004351in}}{\pgfqpoint{2.489323in}{3.996537in}}%
\pgfpathcurveto{\pgfqpoint{2.497137in}{3.988724in}}{\pgfqpoint{2.507736in}{3.984333in}}{\pgfqpoint{2.518786in}{3.984333in}}%
\pgfpathclose%
\pgfusepath{stroke,fill}%
\end{pgfscope}%
\begin{pgfscope}%
\pgfpathrectangle{\pgfqpoint{0.800000in}{0.528000in}}{\pgfqpoint{4.960000in}{3.696000in}}%
\pgfusepath{clip}%
\pgfsetbuttcap%
\pgfsetroundjoin%
\definecolor{currentfill}{rgb}{0.000000,0.000000,0.000000}%
\pgfsetfillcolor{currentfill}%
\pgfsetlinewidth{1.003750pt}%
\definecolor{currentstroke}{rgb}{0.000000,0.000000,0.000000}%
\pgfsetstrokecolor{currentstroke}%
\pgfsetdash{}{0pt}%
\pgfpathmoveto{\pgfqpoint{2.518786in}{3.984333in}}%
\pgfpathcurveto{\pgfqpoint{2.529836in}{3.984333in}}{\pgfqpoint{2.540435in}{3.988724in}}{\pgfqpoint{2.548249in}{3.996537in}}%
\pgfpathcurveto{\pgfqpoint{2.556062in}{4.004351in}}{\pgfqpoint{2.560452in}{4.014950in}}{\pgfqpoint{2.560452in}{4.026000in}}%
\pgfpathcurveto{\pgfqpoint{2.560452in}{4.037050in}}{\pgfqpoint{2.556062in}{4.047649in}}{\pgfqpoint{2.548249in}{4.055463in}}%
\pgfpathcurveto{\pgfqpoint{2.540435in}{4.063276in}}{\pgfqpoint{2.529836in}{4.067667in}}{\pgfqpoint{2.518786in}{4.067667in}}%
\pgfpathcurveto{\pgfqpoint{2.507736in}{4.067667in}}{\pgfqpoint{2.497137in}{4.063276in}}{\pgfqpoint{2.489323in}{4.055463in}}%
\pgfpathcurveto{\pgfqpoint{2.481509in}{4.047649in}}{\pgfqpoint{2.477119in}{4.037050in}}{\pgfqpoint{2.477119in}{4.026000in}}%
\pgfpathcurveto{\pgfqpoint{2.477119in}{4.014950in}}{\pgfqpoint{2.481509in}{4.004351in}}{\pgfqpoint{2.489323in}{3.996537in}}%
\pgfpathcurveto{\pgfqpoint{2.497137in}{3.988724in}}{\pgfqpoint{2.507736in}{3.984333in}}{\pgfqpoint{2.518786in}{3.984333in}}%
\pgfpathclose%
\pgfusepath{stroke,fill}%
\end{pgfscope}%
\begin{pgfscope}%
\pgfpathrectangle{\pgfqpoint{0.800000in}{0.528000in}}{\pgfqpoint{4.960000in}{3.696000in}}%
\pgfusepath{clip}%
\pgfsetbuttcap%
\pgfsetroundjoin%
\definecolor{currentfill}{rgb}{0.000000,0.000000,0.000000}%
\pgfsetfillcolor{currentfill}%
\pgfsetlinewidth{1.003750pt}%
\definecolor{currentstroke}{rgb}{0.000000,0.000000,0.000000}%
\pgfsetstrokecolor{currentstroke}%
\pgfsetdash{}{0pt}%
\pgfpathmoveto{\pgfqpoint{2.518786in}{3.984333in}}%
\pgfpathcurveto{\pgfqpoint{2.529836in}{3.984333in}}{\pgfqpoint{2.540435in}{3.988724in}}{\pgfqpoint{2.548249in}{3.996537in}}%
\pgfpathcurveto{\pgfqpoint{2.556062in}{4.004351in}}{\pgfqpoint{2.560452in}{4.014950in}}{\pgfqpoint{2.560452in}{4.026000in}}%
\pgfpathcurveto{\pgfqpoint{2.560452in}{4.037050in}}{\pgfqpoint{2.556062in}{4.047649in}}{\pgfqpoint{2.548249in}{4.055463in}}%
\pgfpathcurveto{\pgfqpoint{2.540435in}{4.063276in}}{\pgfqpoint{2.529836in}{4.067667in}}{\pgfqpoint{2.518786in}{4.067667in}}%
\pgfpathcurveto{\pgfqpoint{2.507736in}{4.067667in}}{\pgfqpoint{2.497137in}{4.063276in}}{\pgfqpoint{2.489323in}{4.055463in}}%
\pgfpathcurveto{\pgfqpoint{2.481509in}{4.047649in}}{\pgfqpoint{2.477119in}{4.037050in}}{\pgfqpoint{2.477119in}{4.026000in}}%
\pgfpathcurveto{\pgfqpoint{2.477119in}{4.014950in}}{\pgfqpoint{2.481509in}{4.004351in}}{\pgfqpoint{2.489323in}{3.996537in}}%
\pgfpathcurveto{\pgfqpoint{2.497137in}{3.988724in}}{\pgfqpoint{2.507736in}{3.984333in}}{\pgfqpoint{2.518786in}{3.984333in}}%
\pgfpathclose%
\pgfusepath{stroke,fill}%
\end{pgfscope}%
\begin{pgfscope}%
\pgfpathrectangle{\pgfqpoint{0.800000in}{0.528000in}}{\pgfqpoint{4.960000in}{3.696000in}}%
\pgfusepath{clip}%
\pgfsetbuttcap%
\pgfsetroundjoin%
\definecolor{currentfill}{rgb}{0.000000,0.000000,0.000000}%
\pgfsetfillcolor{currentfill}%
\pgfsetlinewidth{1.003750pt}%
\definecolor{currentstroke}{rgb}{0.000000,0.000000,0.000000}%
\pgfsetstrokecolor{currentstroke}%
\pgfsetdash{}{0pt}%
\pgfpathmoveto{\pgfqpoint{2.518786in}{3.984333in}}%
\pgfpathcurveto{\pgfqpoint{2.529836in}{3.984333in}}{\pgfqpoint{2.540435in}{3.988724in}}{\pgfqpoint{2.548249in}{3.996537in}}%
\pgfpathcurveto{\pgfqpoint{2.556062in}{4.004351in}}{\pgfqpoint{2.560452in}{4.014950in}}{\pgfqpoint{2.560452in}{4.026000in}}%
\pgfpathcurveto{\pgfqpoint{2.560452in}{4.037050in}}{\pgfqpoint{2.556062in}{4.047649in}}{\pgfqpoint{2.548249in}{4.055463in}}%
\pgfpathcurveto{\pgfqpoint{2.540435in}{4.063276in}}{\pgfqpoint{2.529836in}{4.067667in}}{\pgfqpoint{2.518786in}{4.067667in}}%
\pgfpathcurveto{\pgfqpoint{2.507736in}{4.067667in}}{\pgfqpoint{2.497137in}{4.063276in}}{\pgfqpoint{2.489323in}{4.055463in}}%
\pgfpathcurveto{\pgfqpoint{2.481509in}{4.047649in}}{\pgfqpoint{2.477119in}{4.037050in}}{\pgfqpoint{2.477119in}{4.026000in}}%
\pgfpathcurveto{\pgfqpoint{2.477119in}{4.014950in}}{\pgfqpoint{2.481509in}{4.004351in}}{\pgfqpoint{2.489323in}{3.996537in}}%
\pgfpathcurveto{\pgfqpoint{2.497137in}{3.988724in}}{\pgfqpoint{2.507736in}{3.984333in}}{\pgfqpoint{2.518786in}{3.984333in}}%
\pgfpathclose%
\pgfusepath{stroke,fill}%
\end{pgfscope}%
\begin{pgfscope}%
\pgfpathrectangle{\pgfqpoint{0.800000in}{0.528000in}}{\pgfqpoint{4.960000in}{3.696000in}}%
\pgfusepath{clip}%
\pgfsetbuttcap%
\pgfsetroundjoin%
\definecolor{currentfill}{rgb}{0.000000,0.000000,0.000000}%
\pgfsetfillcolor{currentfill}%
\pgfsetlinewidth{1.003750pt}%
\definecolor{currentstroke}{rgb}{0.000000,0.000000,0.000000}%
\pgfsetstrokecolor{currentstroke}%
\pgfsetdash{}{0pt}%
\pgfpathmoveto{\pgfqpoint{2.518786in}{3.984333in}}%
\pgfpathcurveto{\pgfqpoint{2.529836in}{3.984333in}}{\pgfqpoint{2.540435in}{3.988724in}}{\pgfqpoint{2.548249in}{3.996537in}}%
\pgfpathcurveto{\pgfqpoint{2.556062in}{4.004351in}}{\pgfqpoint{2.560452in}{4.014950in}}{\pgfqpoint{2.560452in}{4.026000in}}%
\pgfpathcurveto{\pgfqpoint{2.560452in}{4.037050in}}{\pgfqpoint{2.556062in}{4.047649in}}{\pgfqpoint{2.548249in}{4.055463in}}%
\pgfpathcurveto{\pgfqpoint{2.540435in}{4.063276in}}{\pgfqpoint{2.529836in}{4.067667in}}{\pgfqpoint{2.518786in}{4.067667in}}%
\pgfpathcurveto{\pgfqpoint{2.507736in}{4.067667in}}{\pgfqpoint{2.497137in}{4.063276in}}{\pgfqpoint{2.489323in}{4.055463in}}%
\pgfpathcurveto{\pgfqpoint{2.481509in}{4.047649in}}{\pgfqpoint{2.477119in}{4.037050in}}{\pgfqpoint{2.477119in}{4.026000in}}%
\pgfpathcurveto{\pgfqpoint{2.477119in}{4.014950in}}{\pgfqpoint{2.481509in}{4.004351in}}{\pgfqpoint{2.489323in}{3.996537in}}%
\pgfpathcurveto{\pgfqpoint{2.497137in}{3.988724in}}{\pgfqpoint{2.507736in}{3.984333in}}{\pgfqpoint{2.518786in}{3.984333in}}%
\pgfpathclose%
\pgfusepath{stroke,fill}%
\end{pgfscope}%
\begin{pgfscope}%
\pgfpathrectangle{\pgfqpoint{0.800000in}{0.528000in}}{\pgfqpoint{4.960000in}{3.696000in}}%
\pgfusepath{clip}%
\pgfsetbuttcap%
\pgfsetroundjoin%
\definecolor{currentfill}{rgb}{0.000000,0.000000,0.000000}%
\pgfsetfillcolor{currentfill}%
\pgfsetlinewidth{1.003750pt}%
\definecolor{currentstroke}{rgb}{0.000000,0.000000,0.000000}%
\pgfsetstrokecolor{currentstroke}%
\pgfsetdash{}{0pt}%
\pgfpathmoveto{\pgfqpoint{2.518786in}{3.984333in}}%
\pgfpathcurveto{\pgfqpoint{2.529836in}{3.984333in}}{\pgfqpoint{2.540435in}{3.988724in}}{\pgfqpoint{2.548249in}{3.996537in}}%
\pgfpathcurveto{\pgfqpoint{2.556062in}{4.004351in}}{\pgfqpoint{2.560452in}{4.014950in}}{\pgfqpoint{2.560452in}{4.026000in}}%
\pgfpathcurveto{\pgfqpoint{2.560452in}{4.037050in}}{\pgfqpoint{2.556062in}{4.047649in}}{\pgfqpoint{2.548249in}{4.055463in}}%
\pgfpathcurveto{\pgfqpoint{2.540435in}{4.063276in}}{\pgfqpoint{2.529836in}{4.067667in}}{\pgfqpoint{2.518786in}{4.067667in}}%
\pgfpathcurveto{\pgfqpoint{2.507736in}{4.067667in}}{\pgfqpoint{2.497137in}{4.063276in}}{\pgfqpoint{2.489323in}{4.055463in}}%
\pgfpathcurveto{\pgfqpoint{2.481509in}{4.047649in}}{\pgfqpoint{2.477119in}{4.037050in}}{\pgfqpoint{2.477119in}{4.026000in}}%
\pgfpathcurveto{\pgfqpoint{2.477119in}{4.014950in}}{\pgfqpoint{2.481509in}{4.004351in}}{\pgfqpoint{2.489323in}{3.996537in}}%
\pgfpathcurveto{\pgfqpoint{2.497137in}{3.988724in}}{\pgfqpoint{2.507736in}{3.984333in}}{\pgfqpoint{2.518786in}{3.984333in}}%
\pgfpathclose%
\pgfusepath{stroke,fill}%
\end{pgfscope}%
\begin{pgfscope}%
\pgfpathrectangle{\pgfqpoint{0.800000in}{0.528000in}}{\pgfqpoint{4.960000in}{3.696000in}}%
\pgfusepath{clip}%
\pgfsetbuttcap%
\pgfsetroundjoin%
\definecolor{currentfill}{rgb}{0.000000,0.000000,0.000000}%
\pgfsetfillcolor{currentfill}%
\pgfsetlinewidth{1.003750pt}%
\definecolor{currentstroke}{rgb}{0.000000,0.000000,0.000000}%
\pgfsetstrokecolor{currentstroke}%
\pgfsetdash{}{0pt}%
\pgfpathmoveto{\pgfqpoint{2.518786in}{3.984333in}}%
\pgfpathcurveto{\pgfqpoint{2.529836in}{3.984333in}}{\pgfqpoint{2.540435in}{3.988724in}}{\pgfqpoint{2.548249in}{3.996537in}}%
\pgfpathcurveto{\pgfqpoint{2.556062in}{4.004351in}}{\pgfqpoint{2.560452in}{4.014950in}}{\pgfqpoint{2.560452in}{4.026000in}}%
\pgfpathcurveto{\pgfqpoint{2.560452in}{4.037050in}}{\pgfqpoint{2.556062in}{4.047649in}}{\pgfqpoint{2.548249in}{4.055463in}}%
\pgfpathcurveto{\pgfqpoint{2.540435in}{4.063276in}}{\pgfqpoint{2.529836in}{4.067667in}}{\pgfqpoint{2.518786in}{4.067667in}}%
\pgfpathcurveto{\pgfqpoint{2.507736in}{4.067667in}}{\pgfqpoint{2.497137in}{4.063276in}}{\pgfqpoint{2.489323in}{4.055463in}}%
\pgfpathcurveto{\pgfqpoint{2.481509in}{4.047649in}}{\pgfqpoint{2.477119in}{4.037050in}}{\pgfqpoint{2.477119in}{4.026000in}}%
\pgfpathcurveto{\pgfqpoint{2.477119in}{4.014950in}}{\pgfqpoint{2.481509in}{4.004351in}}{\pgfqpoint{2.489323in}{3.996537in}}%
\pgfpathcurveto{\pgfqpoint{2.497137in}{3.988724in}}{\pgfqpoint{2.507736in}{3.984333in}}{\pgfqpoint{2.518786in}{3.984333in}}%
\pgfpathclose%
\pgfusepath{stroke,fill}%
\end{pgfscope}%
\begin{pgfscope}%
\pgfpathrectangle{\pgfqpoint{0.800000in}{0.528000in}}{\pgfqpoint{4.960000in}{3.696000in}}%
\pgfusepath{clip}%
\pgfsetbuttcap%
\pgfsetroundjoin%
\definecolor{currentfill}{rgb}{0.000000,0.000000,0.000000}%
\pgfsetfillcolor{currentfill}%
\pgfsetlinewidth{1.003750pt}%
\definecolor{currentstroke}{rgb}{0.000000,0.000000,0.000000}%
\pgfsetstrokecolor{currentstroke}%
\pgfsetdash{}{0pt}%
\pgfpathmoveto{\pgfqpoint{2.518786in}{3.984333in}}%
\pgfpathcurveto{\pgfqpoint{2.529836in}{3.984333in}}{\pgfqpoint{2.540435in}{3.988724in}}{\pgfqpoint{2.548249in}{3.996537in}}%
\pgfpathcurveto{\pgfqpoint{2.556062in}{4.004351in}}{\pgfqpoint{2.560452in}{4.014950in}}{\pgfqpoint{2.560452in}{4.026000in}}%
\pgfpathcurveto{\pgfqpoint{2.560452in}{4.037050in}}{\pgfqpoint{2.556062in}{4.047649in}}{\pgfqpoint{2.548249in}{4.055463in}}%
\pgfpathcurveto{\pgfqpoint{2.540435in}{4.063276in}}{\pgfqpoint{2.529836in}{4.067667in}}{\pgfqpoint{2.518786in}{4.067667in}}%
\pgfpathcurveto{\pgfqpoint{2.507736in}{4.067667in}}{\pgfqpoint{2.497137in}{4.063276in}}{\pgfqpoint{2.489323in}{4.055463in}}%
\pgfpathcurveto{\pgfqpoint{2.481509in}{4.047649in}}{\pgfqpoint{2.477119in}{4.037050in}}{\pgfqpoint{2.477119in}{4.026000in}}%
\pgfpathcurveto{\pgfqpoint{2.477119in}{4.014950in}}{\pgfqpoint{2.481509in}{4.004351in}}{\pgfqpoint{2.489323in}{3.996537in}}%
\pgfpathcurveto{\pgfqpoint{2.497137in}{3.988724in}}{\pgfqpoint{2.507736in}{3.984333in}}{\pgfqpoint{2.518786in}{3.984333in}}%
\pgfpathclose%
\pgfusepath{stroke,fill}%
\end{pgfscope}%
\begin{pgfscope}%
\pgfpathrectangle{\pgfqpoint{0.800000in}{0.528000in}}{\pgfqpoint{4.960000in}{3.696000in}}%
\pgfusepath{clip}%
\pgfsetbuttcap%
\pgfsetroundjoin%
\definecolor{currentfill}{rgb}{0.000000,0.000000,0.000000}%
\pgfsetfillcolor{currentfill}%
\pgfsetlinewidth{1.003750pt}%
\definecolor{currentstroke}{rgb}{0.000000,0.000000,0.000000}%
\pgfsetstrokecolor{currentstroke}%
\pgfsetdash{}{0pt}%
\pgfpathmoveto{\pgfqpoint{2.518786in}{3.984333in}}%
\pgfpathcurveto{\pgfqpoint{2.529836in}{3.984333in}}{\pgfqpoint{2.540435in}{3.988724in}}{\pgfqpoint{2.548249in}{3.996537in}}%
\pgfpathcurveto{\pgfqpoint{2.556062in}{4.004351in}}{\pgfqpoint{2.560452in}{4.014950in}}{\pgfqpoint{2.560452in}{4.026000in}}%
\pgfpathcurveto{\pgfqpoint{2.560452in}{4.037050in}}{\pgfqpoint{2.556062in}{4.047649in}}{\pgfqpoint{2.548249in}{4.055463in}}%
\pgfpathcurveto{\pgfqpoint{2.540435in}{4.063276in}}{\pgfqpoint{2.529836in}{4.067667in}}{\pgfqpoint{2.518786in}{4.067667in}}%
\pgfpathcurveto{\pgfqpoint{2.507736in}{4.067667in}}{\pgfqpoint{2.497137in}{4.063276in}}{\pgfqpoint{2.489323in}{4.055463in}}%
\pgfpathcurveto{\pgfqpoint{2.481509in}{4.047649in}}{\pgfqpoint{2.477119in}{4.037050in}}{\pgfqpoint{2.477119in}{4.026000in}}%
\pgfpathcurveto{\pgfqpoint{2.477119in}{4.014950in}}{\pgfqpoint{2.481509in}{4.004351in}}{\pgfqpoint{2.489323in}{3.996537in}}%
\pgfpathcurveto{\pgfqpoint{2.497137in}{3.988724in}}{\pgfqpoint{2.507736in}{3.984333in}}{\pgfqpoint{2.518786in}{3.984333in}}%
\pgfpathclose%
\pgfusepath{stroke,fill}%
\end{pgfscope}%
\begin{pgfscope}%
\pgfpathrectangle{\pgfqpoint{0.800000in}{0.528000in}}{\pgfqpoint{4.960000in}{3.696000in}}%
\pgfusepath{clip}%
\pgfsetbuttcap%
\pgfsetroundjoin%
\definecolor{currentfill}{rgb}{0.000000,0.000000,0.000000}%
\pgfsetfillcolor{currentfill}%
\pgfsetlinewidth{1.003750pt}%
\definecolor{currentstroke}{rgb}{0.000000,0.000000,0.000000}%
\pgfsetstrokecolor{currentstroke}%
\pgfsetdash{}{0pt}%
\pgfpathmoveto{\pgfqpoint{2.518786in}{3.984333in}}%
\pgfpathcurveto{\pgfqpoint{2.529836in}{3.984333in}}{\pgfqpoint{2.540435in}{3.988724in}}{\pgfqpoint{2.548249in}{3.996537in}}%
\pgfpathcurveto{\pgfqpoint{2.556062in}{4.004351in}}{\pgfqpoint{2.560452in}{4.014950in}}{\pgfqpoint{2.560452in}{4.026000in}}%
\pgfpathcurveto{\pgfqpoint{2.560452in}{4.037050in}}{\pgfqpoint{2.556062in}{4.047649in}}{\pgfqpoint{2.548249in}{4.055463in}}%
\pgfpathcurveto{\pgfqpoint{2.540435in}{4.063276in}}{\pgfqpoint{2.529836in}{4.067667in}}{\pgfqpoint{2.518786in}{4.067667in}}%
\pgfpathcurveto{\pgfqpoint{2.507736in}{4.067667in}}{\pgfqpoint{2.497137in}{4.063276in}}{\pgfqpoint{2.489323in}{4.055463in}}%
\pgfpathcurveto{\pgfqpoint{2.481509in}{4.047649in}}{\pgfqpoint{2.477119in}{4.037050in}}{\pgfqpoint{2.477119in}{4.026000in}}%
\pgfpathcurveto{\pgfqpoint{2.477119in}{4.014950in}}{\pgfqpoint{2.481509in}{4.004351in}}{\pgfqpoint{2.489323in}{3.996537in}}%
\pgfpathcurveto{\pgfqpoint{2.497137in}{3.988724in}}{\pgfqpoint{2.507736in}{3.984333in}}{\pgfqpoint{2.518786in}{3.984333in}}%
\pgfpathclose%
\pgfusepath{stroke,fill}%
\end{pgfscope}%
\begin{pgfscope}%
\pgfpathrectangle{\pgfqpoint{0.800000in}{0.528000in}}{\pgfqpoint{4.960000in}{3.696000in}}%
\pgfusepath{clip}%
\pgfsetbuttcap%
\pgfsetroundjoin%
\definecolor{currentfill}{rgb}{0.000000,0.000000,0.000000}%
\pgfsetfillcolor{currentfill}%
\pgfsetlinewidth{1.003750pt}%
\definecolor{currentstroke}{rgb}{0.000000,0.000000,0.000000}%
\pgfsetstrokecolor{currentstroke}%
\pgfsetdash{}{0pt}%
\pgfpathmoveto{\pgfqpoint{2.518786in}{3.984333in}}%
\pgfpathcurveto{\pgfqpoint{2.529836in}{3.984333in}}{\pgfqpoint{2.540435in}{3.988724in}}{\pgfqpoint{2.548249in}{3.996537in}}%
\pgfpathcurveto{\pgfqpoint{2.556062in}{4.004351in}}{\pgfqpoint{2.560452in}{4.014950in}}{\pgfqpoint{2.560452in}{4.026000in}}%
\pgfpathcurveto{\pgfqpoint{2.560452in}{4.037050in}}{\pgfqpoint{2.556062in}{4.047649in}}{\pgfqpoint{2.548249in}{4.055463in}}%
\pgfpathcurveto{\pgfqpoint{2.540435in}{4.063276in}}{\pgfqpoint{2.529836in}{4.067667in}}{\pgfqpoint{2.518786in}{4.067667in}}%
\pgfpathcurveto{\pgfqpoint{2.507736in}{4.067667in}}{\pgfqpoint{2.497137in}{4.063276in}}{\pgfqpoint{2.489323in}{4.055463in}}%
\pgfpathcurveto{\pgfqpoint{2.481509in}{4.047649in}}{\pgfqpoint{2.477119in}{4.037050in}}{\pgfqpoint{2.477119in}{4.026000in}}%
\pgfpathcurveto{\pgfqpoint{2.477119in}{4.014950in}}{\pgfqpoint{2.481509in}{4.004351in}}{\pgfqpoint{2.489323in}{3.996537in}}%
\pgfpathcurveto{\pgfqpoint{2.497137in}{3.988724in}}{\pgfqpoint{2.507736in}{3.984333in}}{\pgfqpoint{2.518786in}{3.984333in}}%
\pgfpathclose%
\pgfusepath{stroke,fill}%
\end{pgfscope}%
\begin{pgfscope}%
\pgfpathrectangle{\pgfqpoint{0.800000in}{0.528000in}}{\pgfqpoint{4.960000in}{3.696000in}}%
\pgfusepath{clip}%
\pgfsetbuttcap%
\pgfsetroundjoin%
\definecolor{currentfill}{rgb}{0.000000,0.000000,0.000000}%
\pgfsetfillcolor{currentfill}%
\pgfsetlinewidth{1.003750pt}%
\definecolor{currentstroke}{rgb}{0.000000,0.000000,0.000000}%
\pgfsetstrokecolor{currentstroke}%
\pgfsetdash{}{0pt}%
\pgfpathmoveto{\pgfqpoint{2.518786in}{3.984333in}}%
\pgfpathcurveto{\pgfqpoint{2.529836in}{3.984333in}}{\pgfqpoint{2.540435in}{3.988724in}}{\pgfqpoint{2.548249in}{3.996537in}}%
\pgfpathcurveto{\pgfqpoint{2.556062in}{4.004351in}}{\pgfqpoint{2.560452in}{4.014950in}}{\pgfqpoint{2.560452in}{4.026000in}}%
\pgfpathcurveto{\pgfqpoint{2.560452in}{4.037050in}}{\pgfqpoint{2.556062in}{4.047649in}}{\pgfqpoint{2.548249in}{4.055463in}}%
\pgfpathcurveto{\pgfqpoint{2.540435in}{4.063276in}}{\pgfqpoint{2.529836in}{4.067667in}}{\pgfqpoint{2.518786in}{4.067667in}}%
\pgfpathcurveto{\pgfqpoint{2.507736in}{4.067667in}}{\pgfqpoint{2.497137in}{4.063276in}}{\pgfqpoint{2.489323in}{4.055463in}}%
\pgfpathcurveto{\pgfqpoint{2.481509in}{4.047649in}}{\pgfqpoint{2.477119in}{4.037050in}}{\pgfqpoint{2.477119in}{4.026000in}}%
\pgfpathcurveto{\pgfqpoint{2.477119in}{4.014950in}}{\pgfqpoint{2.481509in}{4.004351in}}{\pgfqpoint{2.489323in}{3.996537in}}%
\pgfpathcurveto{\pgfqpoint{2.497137in}{3.988724in}}{\pgfqpoint{2.507736in}{3.984333in}}{\pgfqpoint{2.518786in}{3.984333in}}%
\pgfpathclose%
\pgfusepath{stroke,fill}%
\end{pgfscope}%
\begin{pgfscope}%
\pgfpathrectangle{\pgfqpoint{0.800000in}{0.528000in}}{\pgfqpoint{4.960000in}{3.696000in}}%
\pgfusepath{clip}%
\pgfsetbuttcap%
\pgfsetroundjoin%
\definecolor{currentfill}{rgb}{0.000000,0.000000,0.000000}%
\pgfsetfillcolor{currentfill}%
\pgfsetlinewidth{1.003750pt}%
\definecolor{currentstroke}{rgb}{0.000000,0.000000,0.000000}%
\pgfsetstrokecolor{currentstroke}%
\pgfsetdash{}{0pt}%
\pgfpathmoveto{\pgfqpoint{2.518786in}{3.984333in}}%
\pgfpathcurveto{\pgfqpoint{2.529836in}{3.984333in}}{\pgfqpoint{2.540435in}{3.988724in}}{\pgfqpoint{2.548249in}{3.996537in}}%
\pgfpathcurveto{\pgfqpoint{2.556062in}{4.004351in}}{\pgfqpoint{2.560452in}{4.014950in}}{\pgfqpoint{2.560452in}{4.026000in}}%
\pgfpathcurveto{\pgfqpoint{2.560452in}{4.037050in}}{\pgfqpoint{2.556062in}{4.047649in}}{\pgfqpoint{2.548249in}{4.055463in}}%
\pgfpathcurveto{\pgfqpoint{2.540435in}{4.063276in}}{\pgfqpoint{2.529836in}{4.067667in}}{\pgfqpoint{2.518786in}{4.067667in}}%
\pgfpathcurveto{\pgfqpoint{2.507736in}{4.067667in}}{\pgfqpoint{2.497137in}{4.063276in}}{\pgfqpoint{2.489323in}{4.055463in}}%
\pgfpathcurveto{\pgfqpoint{2.481509in}{4.047649in}}{\pgfqpoint{2.477119in}{4.037050in}}{\pgfqpoint{2.477119in}{4.026000in}}%
\pgfpathcurveto{\pgfqpoint{2.477119in}{4.014950in}}{\pgfqpoint{2.481509in}{4.004351in}}{\pgfqpoint{2.489323in}{3.996537in}}%
\pgfpathcurveto{\pgfqpoint{2.497137in}{3.988724in}}{\pgfqpoint{2.507736in}{3.984333in}}{\pgfqpoint{2.518786in}{3.984333in}}%
\pgfpathclose%
\pgfusepath{stroke,fill}%
\end{pgfscope}%
\begin{pgfscope}%
\pgfpathrectangle{\pgfqpoint{0.800000in}{0.528000in}}{\pgfqpoint{4.960000in}{3.696000in}}%
\pgfusepath{clip}%
\pgfsetbuttcap%
\pgfsetroundjoin%
\definecolor{currentfill}{rgb}{0.000000,0.000000,0.000000}%
\pgfsetfillcolor{currentfill}%
\pgfsetlinewidth{1.003750pt}%
\definecolor{currentstroke}{rgb}{0.000000,0.000000,0.000000}%
\pgfsetstrokecolor{currentstroke}%
\pgfsetdash{}{0pt}%
\pgfpathmoveto{\pgfqpoint{2.518786in}{3.984333in}}%
\pgfpathcurveto{\pgfqpoint{2.529836in}{3.984333in}}{\pgfqpoint{2.540435in}{3.988724in}}{\pgfqpoint{2.548249in}{3.996537in}}%
\pgfpathcurveto{\pgfqpoint{2.556062in}{4.004351in}}{\pgfqpoint{2.560452in}{4.014950in}}{\pgfqpoint{2.560452in}{4.026000in}}%
\pgfpathcurveto{\pgfqpoint{2.560452in}{4.037050in}}{\pgfqpoint{2.556062in}{4.047649in}}{\pgfqpoint{2.548249in}{4.055463in}}%
\pgfpathcurveto{\pgfqpoint{2.540435in}{4.063276in}}{\pgfqpoint{2.529836in}{4.067667in}}{\pgfqpoint{2.518786in}{4.067667in}}%
\pgfpathcurveto{\pgfqpoint{2.507736in}{4.067667in}}{\pgfqpoint{2.497137in}{4.063276in}}{\pgfqpoint{2.489323in}{4.055463in}}%
\pgfpathcurveto{\pgfqpoint{2.481509in}{4.047649in}}{\pgfqpoint{2.477119in}{4.037050in}}{\pgfqpoint{2.477119in}{4.026000in}}%
\pgfpathcurveto{\pgfqpoint{2.477119in}{4.014950in}}{\pgfqpoint{2.481509in}{4.004351in}}{\pgfqpoint{2.489323in}{3.996537in}}%
\pgfpathcurveto{\pgfqpoint{2.497137in}{3.988724in}}{\pgfqpoint{2.507736in}{3.984333in}}{\pgfqpoint{2.518786in}{3.984333in}}%
\pgfpathclose%
\pgfusepath{stroke,fill}%
\end{pgfscope}%
\begin{pgfscope}%
\pgfpathrectangle{\pgfqpoint{0.800000in}{0.528000in}}{\pgfqpoint{4.960000in}{3.696000in}}%
\pgfusepath{clip}%
\pgfsetbuttcap%
\pgfsetroundjoin%
\definecolor{currentfill}{rgb}{0.000000,0.000000,0.000000}%
\pgfsetfillcolor{currentfill}%
\pgfsetlinewidth{1.003750pt}%
\definecolor{currentstroke}{rgb}{0.000000,0.000000,0.000000}%
\pgfsetstrokecolor{currentstroke}%
\pgfsetdash{}{0pt}%
\pgfpathmoveto{\pgfqpoint{2.518786in}{3.984333in}}%
\pgfpathcurveto{\pgfqpoint{2.529836in}{3.984333in}}{\pgfqpoint{2.540435in}{3.988724in}}{\pgfqpoint{2.548249in}{3.996537in}}%
\pgfpathcurveto{\pgfqpoint{2.556062in}{4.004351in}}{\pgfqpoint{2.560452in}{4.014950in}}{\pgfqpoint{2.560452in}{4.026000in}}%
\pgfpathcurveto{\pgfqpoint{2.560452in}{4.037050in}}{\pgfqpoint{2.556062in}{4.047649in}}{\pgfqpoint{2.548249in}{4.055463in}}%
\pgfpathcurveto{\pgfqpoint{2.540435in}{4.063276in}}{\pgfqpoint{2.529836in}{4.067667in}}{\pgfqpoint{2.518786in}{4.067667in}}%
\pgfpathcurveto{\pgfqpoint{2.507736in}{4.067667in}}{\pgfqpoint{2.497137in}{4.063276in}}{\pgfqpoint{2.489323in}{4.055463in}}%
\pgfpathcurveto{\pgfqpoint{2.481509in}{4.047649in}}{\pgfqpoint{2.477119in}{4.037050in}}{\pgfqpoint{2.477119in}{4.026000in}}%
\pgfpathcurveto{\pgfqpoint{2.477119in}{4.014950in}}{\pgfqpoint{2.481509in}{4.004351in}}{\pgfqpoint{2.489323in}{3.996537in}}%
\pgfpathcurveto{\pgfqpoint{2.497137in}{3.988724in}}{\pgfqpoint{2.507736in}{3.984333in}}{\pgfqpoint{2.518786in}{3.984333in}}%
\pgfpathclose%
\pgfusepath{stroke,fill}%
\end{pgfscope}%
\begin{pgfscope}%
\pgfpathrectangle{\pgfqpoint{0.800000in}{0.528000in}}{\pgfqpoint{4.960000in}{3.696000in}}%
\pgfusepath{clip}%
\pgfsetbuttcap%
\pgfsetroundjoin%
\definecolor{currentfill}{rgb}{0.000000,0.000000,0.000000}%
\pgfsetfillcolor{currentfill}%
\pgfsetlinewidth{1.003750pt}%
\definecolor{currentstroke}{rgb}{0.000000,0.000000,0.000000}%
\pgfsetstrokecolor{currentstroke}%
\pgfsetdash{}{0pt}%
\pgfpathmoveto{\pgfqpoint{2.518786in}{3.984333in}}%
\pgfpathcurveto{\pgfqpoint{2.529836in}{3.984333in}}{\pgfqpoint{2.540435in}{3.988724in}}{\pgfqpoint{2.548249in}{3.996537in}}%
\pgfpathcurveto{\pgfqpoint{2.556062in}{4.004351in}}{\pgfqpoint{2.560452in}{4.014950in}}{\pgfqpoint{2.560452in}{4.026000in}}%
\pgfpathcurveto{\pgfqpoint{2.560452in}{4.037050in}}{\pgfqpoint{2.556062in}{4.047649in}}{\pgfqpoint{2.548249in}{4.055463in}}%
\pgfpathcurveto{\pgfqpoint{2.540435in}{4.063276in}}{\pgfqpoint{2.529836in}{4.067667in}}{\pgfqpoint{2.518786in}{4.067667in}}%
\pgfpathcurveto{\pgfqpoint{2.507736in}{4.067667in}}{\pgfqpoint{2.497137in}{4.063276in}}{\pgfqpoint{2.489323in}{4.055463in}}%
\pgfpathcurveto{\pgfqpoint{2.481509in}{4.047649in}}{\pgfqpoint{2.477119in}{4.037050in}}{\pgfqpoint{2.477119in}{4.026000in}}%
\pgfpathcurveto{\pgfqpoint{2.477119in}{4.014950in}}{\pgfqpoint{2.481509in}{4.004351in}}{\pgfqpoint{2.489323in}{3.996537in}}%
\pgfpathcurveto{\pgfqpoint{2.497137in}{3.988724in}}{\pgfqpoint{2.507736in}{3.984333in}}{\pgfqpoint{2.518786in}{3.984333in}}%
\pgfpathclose%
\pgfusepath{stroke,fill}%
\end{pgfscope}%
\begin{pgfscope}%
\pgfpathrectangle{\pgfqpoint{0.800000in}{0.528000in}}{\pgfqpoint{4.960000in}{3.696000in}}%
\pgfusepath{clip}%
\pgfsetbuttcap%
\pgfsetroundjoin%
\definecolor{currentfill}{rgb}{0.000000,0.000000,0.000000}%
\pgfsetfillcolor{currentfill}%
\pgfsetlinewidth{1.003750pt}%
\definecolor{currentstroke}{rgb}{0.000000,0.000000,0.000000}%
\pgfsetstrokecolor{currentstroke}%
\pgfsetdash{}{0pt}%
\pgfpathmoveto{\pgfqpoint{2.518786in}{3.984333in}}%
\pgfpathcurveto{\pgfqpoint{2.529836in}{3.984333in}}{\pgfqpoint{2.540435in}{3.988724in}}{\pgfqpoint{2.548249in}{3.996537in}}%
\pgfpathcurveto{\pgfqpoint{2.556062in}{4.004351in}}{\pgfqpoint{2.560452in}{4.014950in}}{\pgfqpoint{2.560452in}{4.026000in}}%
\pgfpathcurveto{\pgfqpoint{2.560452in}{4.037050in}}{\pgfqpoint{2.556062in}{4.047649in}}{\pgfqpoint{2.548249in}{4.055463in}}%
\pgfpathcurveto{\pgfqpoint{2.540435in}{4.063276in}}{\pgfqpoint{2.529836in}{4.067667in}}{\pgfqpoint{2.518786in}{4.067667in}}%
\pgfpathcurveto{\pgfqpoint{2.507736in}{4.067667in}}{\pgfqpoint{2.497137in}{4.063276in}}{\pgfqpoint{2.489323in}{4.055463in}}%
\pgfpathcurveto{\pgfqpoint{2.481509in}{4.047649in}}{\pgfqpoint{2.477119in}{4.037050in}}{\pgfqpoint{2.477119in}{4.026000in}}%
\pgfpathcurveto{\pgfqpoint{2.477119in}{4.014950in}}{\pgfqpoint{2.481509in}{4.004351in}}{\pgfqpoint{2.489323in}{3.996537in}}%
\pgfpathcurveto{\pgfqpoint{2.497137in}{3.988724in}}{\pgfqpoint{2.507736in}{3.984333in}}{\pgfqpoint{2.518786in}{3.984333in}}%
\pgfpathclose%
\pgfusepath{stroke,fill}%
\end{pgfscope}%
\begin{pgfscope}%
\pgfpathrectangle{\pgfqpoint{0.800000in}{0.528000in}}{\pgfqpoint{4.960000in}{3.696000in}}%
\pgfusepath{clip}%
\pgfsetbuttcap%
\pgfsetroundjoin%
\definecolor{currentfill}{rgb}{0.000000,0.000000,0.000000}%
\pgfsetfillcolor{currentfill}%
\pgfsetlinewidth{1.003750pt}%
\definecolor{currentstroke}{rgb}{0.000000,0.000000,0.000000}%
\pgfsetstrokecolor{currentstroke}%
\pgfsetdash{}{0pt}%
\pgfpathmoveto{\pgfqpoint{2.518786in}{3.984333in}}%
\pgfpathcurveto{\pgfqpoint{2.529836in}{3.984333in}}{\pgfqpoint{2.540435in}{3.988724in}}{\pgfqpoint{2.548249in}{3.996537in}}%
\pgfpathcurveto{\pgfqpoint{2.556062in}{4.004351in}}{\pgfqpoint{2.560452in}{4.014950in}}{\pgfqpoint{2.560452in}{4.026000in}}%
\pgfpathcurveto{\pgfqpoint{2.560452in}{4.037050in}}{\pgfqpoint{2.556062in}{4.047649in}}{\pgfqpoint{2.548249in}{4.055463in}}%
\pgfpathcurveto{\pgfqpoint{2.540435in}{4.063276in}}{\pgfqpoint{2.529836in}{4.067667in}}{\pgfqpoint{2.518786in}{4.067667in}}%
\pgfpathcurveto{\pgfqpoint{2.507736in}{4.067667in}}{\pgfqpoint{2.497137in}{4.063276in}}{\pgfqpoint{2.489323in}{4.055463in}}%
\pgfpathcurveto{\pgfqpoint{2.481509in}{4.047649in}}{\pgfqpoint{2.477119in}{4.037050in}}{\pgfqpoint{2.477119in}{4.026000in}}%
\pgfpathcurveto{\pgfqpoint{2.477119in}{4.014950in}}{\pgfqpoint{2.481509in}{4.004351in}}{\pgfqpoint{2.489323in}{3.996537in}}%
\pgfpathcurveto{\pgfqpoint{2.497137in}{3.988724in}}{\pgfqpoint{2.507736in}{3.984333in}}{\pgfqpoint{2.518786in}{3.984333in}}%
\pgfpathclose%
\pgfusepath{stroke,fill}%
\end{pgfscope}%
\begin{pgfscope}%
\pgfpathrectangle{\pgfqpoint{0.800000in}{0.528000in}}{\pgfqpoint{4.960000in}{3.696000in}}%
\pgfusepath{clip}%
\pgfsetbuttcap%
\pgfsetroundjoin%
\definecolor{currentfill}{rgb}{0.000000,0.000000,0.000000}%
\pgfsetfillcolor{currentfill}%
\pgfsetlinewidth{1.003750pt}%
\definecolor{currentstroke}{rgb}{0.000000,0.000000,0.000000}%
\pgfsetstrokecolor{currentstroke}%
\pgfsetdash{}{0pt}%
\pgfpathmoveto{\pgfqpoint{2.518786in}{3.984333in}}%
\pgfpathcurveto{\pgfqpoint{2.529836in}{3.984333in}}{\pgfqpoint{2.540435in}{3.988724in}}{\pgfqpoint{2.548249in}{3.996537in}}%
\pgfpathcurveto{\pgfqpoint{2.556062in}{4.004351in}}{\pgfqpoint{2.560452in}{4.014950in}}{\pgfqpoint{2.560452in}{4.026000in}}%
\pgfpathcurveto{\pgfqpoint{2.560452in}{4.037050in}}{\pgfqpoint{2.556062in}{4.047649in}}{\pgfqpoint{2.548249in}{4.055463in}}%
\pgfpathcurveto{\pgfqpoint{2.540435in}{4.063276in}}{\pgfqpoint{2.529836in}{4.067667in}}{\pgfqpoint{2.518786in}{4.067667in}}%
\pgfpathcurveto{\pgfqpoint{2.507736in}{4.067667in}}{\pgfqpoint{2.497137in}{4.063276in}}{\pgfqpoint{2.489323in}{4.055463in}}%
\pgfpathcurveto{\pgfqpoint{2.481509in}{4.047649in}}{\pgfqpoint{2.477119in}{4.037050in}}{\pgfqpoint{2.477119in}{4.026000in}}%
\pgfpathcurveto{\pgfqpoint{2.477119in}{4.014950in}}{\pgfqpoint{2.481509in}{4.004351in}}{\pgfqpoint{2.489323in}{3.996537in}}%
\pgfpathcurveto{\pgfqpoint{2.497137in}{3.988724in}}{\pgfqpoint{2.507736in}{3.984333in}}{\pgfqpoint{2.518786in}{3.984333in}}%
\pgfpathclose%
\pgfusepath{stroke,fill}%
\end{pgfscope}%
\begin{pgfscope}%
\pgfpathrectangle{\pgfqpoint{0.800000in}{0.528000in}}{\pgfqpoint{4.960000in}{3.696000in}}%
\pgfusepath{clip}%
\pgfsetbuttcap%
\pgfsetroundjoin%
\definecolor{currentfill}{rgb}{0.000000,0.000000,0.000000}%
\pgfsetfillcolor{currentfill}%
\pgfsetlinewidth{1.003750pt}%
\definecolor{currentstroke}{rgb}{0.000000,0.000000,0.000000}%
\pgfsetstrokecolor{currentstroke}%
\pgfsetdash{}{0pt}%
\pgfpathmoveto{\pgfqpoint{2.518786in}{3.984333in}}%
\pgfpathcurveto{\pgfqpoint{2.529836in}{3.984333in}}{\pgfqpoint{2.540435in}{3.988724in}}{\pgfqpoint{2.548249in}{3.996537in}}%
\pgfpathcurveto{\pgfqpoint{2.556062in}{4.004351in}}{\pgfqpoint{2.560452in}{4.014950in}}{\pgfqpoint{2.560452in}{4.026000in}}%
\pgfpathcurveto{\pgfqpoint{2.560452in}{4.037050in}}{\pgfqpoint{2.556062in}{4.047649in}}{\pgfqpoint{2.548249in}{4.055463in}}%
\pgfpathcurveto{\pgfqpoint{2.540435in}{4.063276in}}{\pgfqpoint{2.529836in}{4.067667in}}{\pgfqpoint{2.518786in}{4.067667in}}%
\pgfpathcurveto{\pgfqpoint{2.507736in}{4.067667in}}{\pgfqpoint{2.497137in}{4.063276in}}{\pgfqpoint{2.489323in}{4.055463in}}%
\pgfpathcurveto{\pgfqpoint{2.481509in}{4.047649in}}{\pgfqpoint{2.477119in}{4.037050in}}{\pgfqpoint{2.477119in}{4.026000in}}%
\pgfpathcurveto{\pgfqpoint{2.477119in}{4.014950in}}{\pgfqpoint{2.481509in}{4.004351in}}{\pgfqpoint{2.489323in}{3.996537in}}%
\pgfpathcurveto{\pgfqpoint{2.497137in}{3.988724in}}{\pgfqpoint{2.507736in}{3.984333in}}{\pgfqpoint{2.518786in}{3.984333in}}%
\pgfpathclose%
\pgfusepath{stroke,fill}%
\end{pgfscope}%
\begin{pgfscope}%
\pgfpathrectangle{\pgfqpoint{0.800000in}{0.528000in}}{\pgfqpoint{4.960000in}{3.696000in}}%
\pgfusepath{clip}%
\pgfsetbuttcap%
\pgfsetroundjoin%
\definecolor{currentfill}{rgb}{0.000000,0.000000,0.000000}%
\pgfsetfillcolor{currentfill}%
\pgfsetlinewidth{1.003750pt}%
\definecolor{currentstroke}{rgb}{0.000000,0.000000,0.000000}%
\pgfsetstrokecolor{currentstroke}%
\pgfsetdash{}{0pt}%
\pgfpathmoveto{\pgfqpoint{2.518786in}{3.984333in}}%
\pgfpathcurveto{\pgfqpoint{2.529836in}{3.984333in}}{\pgfqpoint{2.540435in}{3.988724in}}{\pgfqpoint{2.548249in}{3.996537in}}%
\pgfpathcurveto{\pgfqpoint{2.556062in}{4.004351in}}{\pgfqpoint{2.560452in}{4.014950in}}{\pgfqpoint{2.560452in}{4.026000in}}%
\pgfpathcurveto{\pgfqpoint{2.560452in}{4.037050in}}{\pgfqpoint{2.556062in}{4.047649in}}{\pgfqpoint{2.548249in}{4.055463in}}%
\pgfpathcurveto{\pgfqpoint{2.540435in}{4.063276in}}{\pgfqpoint{2.529836in}{4.067667in}}{\pgfqpoint{2.518786in}{4.067667in}}%
\pgfpathcurveto{\pgfqpoint{2.507736in}{4.067667in}}{\pgfqpoint{2.497137in}{4.063276in}}{\pgfqpoint{2.489323in}{4.055463in}}%
\pgfpathcurveto{\pgfqpoint{2.481509in}{4.047649in}}{\pgfqpoint{2.477119in}{4.037050in}}{\pgfqpoint{2.477119in}{4.026000in}}%
\pgfpathcurveto{\pgfqpoint{2.477119in}{4.014950in}}{\pgfqpoint{2.481509in}{4.004351in}}{\pgfqpoint{2.489323in}{3.996537in}}%
\pgfpathcurveto{\pgfqpoint{2.497137in}{3.988724in}}{\pgfqpoint{2.507736in}{3.984333in}}{\pgfqpoint{2.518786in}{3.984333in}}%
\pgfpathclose%
\pgfusepath{stroke,fill}%
\end{pgfscope}%
\begin{pgfscope}%
\pgfpathrectangle{\pgfqpoint{0.800000in}{0.528000in}}{\pgfqpoint{4.960000in}{3.696000in}}%
\pgfusepath{clip}%
\pgfsetbuttcap%
\pgfsetroundjoin%
\definecolor{currentfill}{rgb}{0.000000,0.000000,0.000000}%
\pgfsetfillcolor{currentfill}%
\pgfsetlinewidth{1.003750pt}%
\definecolor{currentstroke}{rgb}{0.000000,0.000000,0.000000}%
\pgfsetstrokecolor{currentstroke}%
\pgfsetdash{}{0pt}%
\pgfpathmoveto{\pgfqpoint{2.518786in}{3.984333in}}%
\pgfpathcurveto{\pgfqpoint{2.529836in}{3.984333in}}{\pgfqpoint{2.540435in}{3.988724in}}{\pgfqpoint{2.548249in}{3.996537in}}%
\pgfpathcurveto{\pgfqpoint{2.556062in}{4.004351in}}{\pgfqpoint{2.560452in}{4.014950in}}{\pgfqpoint{2.560452in}{4.026000in}}%
\pgfpathcurveto{\pgfqpoint{2.560452in}{4.037050in}}{\pgfqpoint{2.556062in}{4.047649in}}{\pgfqpoint{2.548249in}{4.055463in}}%
\pgfpathcurveto{\pgfqpoint{2.540435in}{4.063276in}}{\pgfqpoint{2.529836in}{4.067667in}}{\pgfqpoint{2.518786in}{4.067667in}}%
\pgfpathcurveto{\pgfqpoint{2.507736in}{4.067667in}}{\pgfqpoint{2.497137in}{4.063276in}}{\pgfqpoint{2.489323in}{4.055463in}}%
\pgfpathcurveto{\pgfqpoint{2.481509in}{4.047649in}}{\pgfqpoint{2.477119in}{4.037050in}}{\pgfqpoint{2.477119in}{4.026000in}}%
\pgfpathcurveto{\pgfqpoint{2.477119in}{4.014950in}}{\pgfqpoint{2.481509in}{4.004351in}}{\pgfqpoint{2.489323in}{3.996537in}}%
\pgfpathcurveto{\pgfqpoint{2.497137in}{3.988724in}}{\pgfqpoint{2.507736in}{3.984333in}}{\pgfqpoint{2.518786in}{3.984333in}}%
\pgfpathclose%
\pgfusepath{stroke,fill}%
\end{pgfscope}%
\begin{pgfscope}%
\pgfpathrectangle{\pgfqpoint{0.800000in}{0.528000in}}{\pgfqpoint{4.960000in}{3.696000in}}%
\pgfusepath{clip}%
\pgfsetbuttcap%
\pgfsetroundjoin%
\definecolor{currentfill}{rgb}{0.000000,0.000000,0.000000}%
\pgfsetfillcolor{currentfill}%
\pgfsetlinewidth{1.003750pt}%
\definecolor{currentstroke}{rgb}{0.000000,0.000000,0.000000}%
\pgfsetstrokecolor{currentstroke}%
\pgfsetdash{}{0pt}%
\pgfpathmoveto{\pgfqpoint{2.518786in}{3.984333in}}%
\pgfpathcurveto{\pgfqpoint{2.529836in}{3.984333in}}{\pgfqpoint{2.540435in}{3.988724in}}{\pgfqpoint{2.548249in}{3.996537in}}%
\pgfpathcurveto{\pgfqpoint{2.556062in}{4.004351in}}{\pgfqpoint{2.560452in}{4.014950in}}{\pgfqpoint{2.560452in}{4.026000in}}%
\pgfpathcurveto{\pgfqpoint{2.560452in}{4.037050in}}{\pgfqpoint{2.556062in}{4.047649in}}{\pgfqpoint{2.548249in}{4.055463in}}%
\pgfpathcurveto{\pgfqpoint{2.540435in}{4.063276in}}{\pgfqpoint{2.529836in}{4.067667in}}{\pgfqpoint{2.518786in}{4.067667in}}%
\pgfpathcurveto{\pgfqpoint{2.507736in}{4.067667in}}{\pgfqpoint{2.497137in}{4.063276in}}{\pgfqpoint{2.489323in}{4.055463in}}%
\pgfpathcurveto{\pgfqpoint{2.481509in}{4.047649in}}{\pgfqpoint{2.477119in}{4.037050in}}{\pgfqpoint{2.477119in}{4.026000in}}%
\pgfpathcurveto{\pgfqpoint{2.477119in}{4.014950in}}{\pgfqpoint{2.481509in}{4.004351in}}{\pgfqpoint{2.489323in}{3.996537in}}%
\pgfpathcurveto{\pgfqpoint{2.497137in}{3.988724in}}{\pgfqpoint{2.507736in}{3.984333in}}{\pgfqpoint{2.518786in}{3.984333in}}%
\pgfpathclose%
\pgfusepath{stroke,fill}%
\end{pgfscope}%
\begin{pgfscope}%
\pgfpathrectangle{\pgfqpoint{0.800000in}{0.528000in}}{\pgfqpoint{4.960000in}{3.696000in}}%
\pgfusepath{clip}%
\pgfsetbuttcap%
\pgfsetroundjoin%
\definecolor{currentfill}{rgb}{0.000000,0.000000,0.000000}%
\pgfsetfillcolor{currentfill}%
\pgfsetlinewidth{1.003750pt}%
\definecolor{currentstroke}{rgb}{0.000000,0.000000,0.000000}%
\pgfsetstrokecolor{currentstroke}%
\pgfsetdash{}{0pt}%
\pgfpathmoveto{\pgfqpoint{2.518786in}{3.984333in}}%
\pgfpathcurveto{\pgfqpoint{2.529836in}{3.984333in}}{\pgfqpoint{2.540435in}{3.988724in}}{\pgfqpoint{2.548249in}{3.996537in}}%
\pgfpathcurveto{\pgfqpoint{2.556062in}{4.004351in}}{\pgfqpoint{2.560452in}{4.014950in}}{\pgfqpoint{2.560452in}{4.026000in}}%
\pgfpathcurveto{\pgfqpoint{2.560452in}{4.037050in}}{\pgfqpoint{2.556062in}{4.047649in}}{\pgfqpoint{2.548249in}{4.055463in}}%
\pgfpathcurveto{\pgfqpoint{2.540435in}{4.063276in}}{\pgfqpoint{2.529836in}{4.067667in}}{\pgfqpoint{2.518786in}{4.067667in}}%
\pgfpathcurveto{\pgfqpoint{2.507736in}{4.067667in}}{\pgfqpoint{2.497137in}{4.063276in}}{\pgfqpoint{2.489323in}{4.055463in}}%
\pgfpathcurveto{\pgfqpoint{2.481509in}{4.047649in}}{\pgfqpoint{2.477119in}{4.037050in}}{\pgfqpoint{2.477119in}{4.026000in}}%
\pgfpathcurveto{\pgfqpoint{2.477119in}{4.014950in}}{\pgfqpoint{2.481509in}{4.004351in}}{\pgfqpoint{2.489323in}{3.996537in}}%
\pgfpathcurveto{\pgfqpoint{2.497137in}{3.988724in}}{\pgfqpoint{2.507736in}{3.984333in}}{\pgfqpoint{2.518786in}{3.984333in}}%
\pgfpathclose%
\pgfusepath{stroke,fill}%
\end{pgfscope}%
\begin{pgfscope}%
\pgfpathrectangle{\pgfqpoint{0.800000in}{0.528000in}}{\pgfqpoint{4.960000in}{3.696000in}}%
\pgfusepath{clip}%
\pgfsetbuttcap%
\pgfsetroundjoin%
\definecolor{currentfill}{rgb}{0.000000,0.000000,0.000000}%
\pgfsetfillcolor{currentfill}%
\pgfsetlinewidth{1.003750pt}%
\definecolor{currentstroke}{rgb}{0.000000,0.000000,0.000000}%
\pgfsetstrokecolor{currentstroke}%
\pgfsetdash{}{0pt}%
\pgfpathmoveto{\pgfqpoint{2.518786in}{3.984333in}}%
\pgfpathcurveto{\pgfqpoint{2.529836in}{3.984333in}}{\pgfqpoint{2.540435in}{3.988724in}}{\pgfqpoint{2.548249in}{3.996537in}}%
\pgfpathcurveto{\pgfqpoint{2.556062in}{4.004351in}}{\pgfqpoint{2.560452in}{4.014950in}}{\pgfqpoint{2.560452in}{4.026000in}}%
\pgfpathcurveto{\pgfqpoint{2.560452in}{4.037050in}}{\pgfqpoint{2.556062in}{4.047649in}}{\pgfqpoint{2.548249in}{4.055463in}}%
\pgfpathcurveto{\pgfqpoint{2.540435in}{4.063276in}}{\pgfqpoint{2.529836in}{4.067667in}}{\pgfqpoint{2.518786in}{4.067667in}}%
\pgfpathcurveto{\pgfqpoint{2.507736in}{4.067667in}}{\pgfqpoint{2.497137in}{4.063276in}}{\pgfqpoint{2.489323in}{4.055463in}}%
\pgfpathcurveto{\pgfqpoint{2.481509in}{4.047649in}}{\pgfqpoint{2.477119in}{4.037050in}}{\pgfqpoint{2.477119in}{4.026000in}}%
\pgfpathcurveto{\pgfqpoint{2.477119in}{4.014950in}}{\pgfqpoint{2.481509in}{4.004351in}}{\pgfqpoint{2.489323in}{3.996537in}}%
\pgfpathcurveto{\pgfqpoint{2.497137in}{3.988724in}}{\pgfqpoint{2.507736in}{3.984333in}}{\pgfqpoint{2.518786in}{3.984333in}}%
\pgfpathclose%
\pgfusepath{stroke,fill}%
\end{pgfscope}%
\begin{pgfscope}%
\pgfpathrectangle{\pgfqpoint{0.800000in}{0.528000in}}{\pgfqpoint{4.960000in}{3.696000in}}%
\pgfusepath{clip}%
\pgfsetbuttcap%
\pgfsetroundjoin%
\definecolor{currentfill}{rgb}{0.000000,0.000000,0.000000}%
\pgfsetfillcolor{currentfill}%
\pgfsetlinewidth{1.003750pt}%
\definecolor{currentstroke}{rgb}{0.000000,0.000000,0.000000}%
\pgfsetstrokecolor{currentstroke}%
\pgfsetdash{}{0pt}%
\pgfpathmoveto{\pgfqpoint{2.518786in}{3.984333in}}%
\pgfpathcurveto{\pgfqpoint{2.529836in}{3.984333in}}{\pgfqpoint{2.540435in}{3.988724in}}{\pgfqpoint{2.548249in}{3.996537in}}%
\pgfpathcurveto{\pgfqpoint{2.556062in}{4.004351in}}{\pgfqpoint{2.560452in}{4.014950in}}{\pgfqpoint{2.560452in}{4.026000in}}%
\pgfpathcurveto{\pgfqpoint{2.560452in}{4.037050in}}{\pgfqpoint{2.556062in}{4.047649in}}{\pgfqpoint{2.548249in}{4.055463in}}%
\pgfpathcurveto{\pgfqpoint{2.540435in}{4.063276in}}{\pgfqpoint{2.529836in}{4.067667in}}{\pgfqpoint{2.518786in}{4.067667in}}%
\pgfpathcurveto{\pgfqpoint{2.507736in}{4.067667in}}{\pgfqpoint{2.497137in}{4.063276in}}{\pgfqpoint{2.489323in}{4.055463in}}%
\pgfpathcurveto{\pgfqpoint{2.481509in}{4.047649in}}{\pgfqpoint{2.477119in}{4.037050in}}{\pgfqpoint{2.477119in}{4.026000in}}%
\pgfpathcurveto{\pgfqpoint{2.477119in}{4.014950in}}{\pgfqpoint{2.481509in}{4.004351in}}{\pgfqpoint{2.489323in}{3.996537in}}%
\pgfpathcurveto{\pgfqpoint{2.497137in}{3.988724in}}{\pgfqpoint{2.507736in}{3.984333in}}{\pgfqpoint{2.518786in}{3.984333in}}%
\pgfpathclose%
\pgfusepath{stroke,fill}%
\end{pgfscope}%
\begin{pgfscope}%
\pgfpathrectangle{\pgfqpoint{0.800000in}{0.528000in}}{\pgfqpoint{4.960000in}{3.696000in}}%
\pgfusepath{clip}%
\pgfsetbuttcap%
\pgfsetroundjoin%
\definecolor{currentfill}{rgb}{0.000000,0.000000,0.000000}%
\pgfsetfillcolor{currentfill}%
\pgfsetlinewidth{1.003750pt}%
\definecolor{currentstroke}{rgb}{0.000000,0.000000,0.000000}%
\pgfsetstrokecolor{currentstroke}%
\pgfsetdash{}{0pt}%
\pgfpathmoveto{\pgfqpoint{2.518786in}{3.984333in}}%
\pgfpathcurveto{\pgfqpoint{2.529836in}{3.984333in}}{\pgfqpoint{2.540435in}{3.988724in}}{\pgfqpoint{2.548249in}{3.996537in}}%
\pgfpathcurveto{\pgfqpoint{2.556062in}{4.004351in}}{\pgfqpoint{2.560452in}{4.014950in}}{\pgfqpoint{2.560452in}{4.026000in}}%
\pgfpathcurveto{\pgfqpoint{2.560452in}{4.037050in}}{\pgfqpoint{2.556062in}{4.047649in}}{\pgfqpoint{2.548249in}{4.055463in}}%
\pgfpathcurveto{\pgfqpoint{2.540435in}{4.063276in}}{\pgfqpoint{2.529836in}{4.067667in}}{\pgfqpoint{2.518786in}{4.067667in}}%
\pgfpathcurveto{\pgfqpoint{2.507736in}{4.067667in}}{\pgfqpoint{2.497137in}{4.063276in}}{\pgfqpoint{2.489323in}{4.055463in}}%
\pgfpathcurveto{\pgfqpoint{2.481509in}{4.047649in}}{\pgfqpoint{2.477119in}{4.037050in}}{\pgfqpoint{2.477119in}{4.026000in}}%
\pgfpathcurveto{\pgfqpoint{2.477119in}{4.014950in}}{\pgfqpoint{2.481509in}{4.004351in}}{\pgfqpoint{2.489323in}{3.996537in}}%
\pgfpathcurveto{\pgfqpoint{2.497137in}{3.988724in}}{\pgfqpoint{2.507736in}{3.984333in}}{\pgfqpoint{2.518786in}{3.984333in}}%
\pgfpathclose%
\pgfusepath{stroke,fill}%
\end{pgfscope}%
\begin{pgfscope}%
\pgfpathrectangle{\pgfqpoint{0.800000in}{0.528000in}}{\pgfqpoint{4.960000in}{3.696000in}}%
\pgfusepath{clip}%
\pgfsetbuttcap%
\pgfsetroundjoin%
\definecolor{currentfill}{rgb}{0.000000,0.000000,0.000000}%
\pgfsetfillcolor{currentfill}%
\pgfsetlinewidth{1.003750pt}%
\definecolor{currentstroke}{rgb}{0.000000,0.000000,0.000000}%
\pgfsetstrokecolor{currentstroke}%
\pgfsetdash{}{0pt}%
\pgfpathmoveto{\pgfqpoint{2.518786in}{3.984333in}}%
\pgfpathcurveto{\pgfqpoint{2.529836in}{3.984333in}}{\pgfqpoint{2.540435in}{3.988724in}}{\pgfqpoint{2.548249in}{3.996537in}}%
\pgfpathcurveto{\pgfqpoint{2.556062in}{4.004351in}}{\pgfqpoint{2.560452in}{4.014950in}}{\pgfqpoint{2.560452in}{4.026000in}}%
\pgfpathcurveto{\pgfqpoint{2.560452in}{4.037050in}}{\pgfqpoint{2.556062in}{4.047649in}}{\pgfqpoint{2.548249in}{4.055463in}}%
\pgfpathcurveto{\pgfqpoint{2.540435in}{4.063276in}}{\pgfqpoint{2.529836in}{4.067667in}}{\pgfqpoint{2.518786in}{4.067667in}}%
\pgfpathcurveto{\pgfqpoint{2.507736in}{4.067667in}}{\pgfqpoint{2.497137in}{4.063276in}}{\pgfqpoint{2.489323in}{4.055463in}}%
\pgfpathcurveto{\pgfqpoint{2.481509in}{4.047649in}}{\pgfqpoint{2.477119in}{4.037050in}}{\pgfqpoint{2.477119in}{4.026000in}}%
\pgfpathcurveto{\pgfqpoint{2.477119in}{4.014950in}}{\pgfqpoint{2.481509in}{4.004351in}}{\pgfqpoint{2.489323in}{3.996537in}}%
\pgfpathcurveto{\pgfqpoint{2.497137in}{3.988724in}}{\pgfqpoint{2.507736in}{3.984333in}}{\pgfqpoint{2.518786in}{3.984333in}}%
\pgfpathclose%
\pgfusepath{stroke,fill}%
\end{pgfscope}%
\begin{pgfscope}%
\pgfpathrectangle{\pgfqpoint{0.800000in}{0.528000in}}{\pgfqpoint{4.960000in}{3.696000in}}%
\pgfusepath{clip}%
\pgfsetbuttcap%
\pgfsetroundjoin%
\definecolor{currentfill}{rgb}{0.000000,0.000000,0.000000}%
\pgfsetfillcolor{currentfill}%
\pgfsetlinewidth{1.003750pt}%
\definecolor{currentstroke}{rgb}{0.000000,0.000000,0.000000}%
\pgfsetstrokecolor{currentstroke}%
\pgfsetdash{}{0pt}%
\pgfpathmoveto{\pgfqpoint{2.518786in}{3.984333in}}%
\pgfpathcurveto{\pgfqpoint{2.529836in}{3.984333in}}{\pgfqpoint{2.540435in}{3.988724in}}{\pgfqpoint{2.548249in}{3.996537in}}%
\pgfpathcurveto{\pgfqpoint{2.556062in}{4.004351in}}{\pgfqpoint{2.560452in}{4.014950in}}{\pgfqpoint{2.560452in}{4.026000in}}%
\pgfpathcurveto{\pgfqpoint{2.560452in}{4.037050in}}{\pgfqpoint{2.556062in}{4.047649in}}{\pgfqpoint{2.548249in}{4.055463in}}%
\pgfpathcurveto{\pgfqpoint{2.540435in}{4.063276in}}{\pgfqpoint{2.529836in}{4.067667in}}{\pgfqpoint{2.518786in}{4.067667in}}%
\pgfpathcurveto{\pgfqpoint{2.507736in}{4.067667in}}{\pgfqpoint{2.497137in}{4.063276in}}{\pgfqpoint{2.489323in}{4.055463in}}%
\pgfpathcurveto{\pgfqpoint{2.481509in}{4.047649in}}{\pgfqpoint{2.477119in}{4.037050in}}{\pgfqpoint{2.477119in}{4.026000in}}%
\pgfpathcurveto{\pgfqpoint{2.477119in}{4.014950in}}{\pgfqpoint{2.481509in}{4.004351in}}{\pgfqpoint{2.489323in}{3.996537in}}%
\pgfpathcurveto{\pgfqpoint{2.497137in}{3.988724in}}{\pgfqpoint{2.507736in}{3.984333in}}{\pgfqpoint{2.518786in}{3.984333in}}%
\pgfpathclose%
\pgfusepath{stroke,fill}%
\end{pgfscope}%
\begin{pgfscope}%
\pgfpathrectangle{\pgfqpoint{0.800000in}{0.528000in}}{\pgfqpoint{4.960000in}{3.696000in}}%
\pgfusepath{clip}%
\pgfsetbuttcap%
\pgfsetroundjoin%
\definecolor{currentfill}{rgb}{0.000000,0.000000,0.000000}%
\pgfsetfillcolor{currentfill}%
\pgfsetlinewidth{1.003750pt}%
\definecolor{currentstroke}{rgb}{0.000000,0.000000,0.000000}%
\pgfsetstrokecolor{currentstroke}%
\pgfsetdash{}{0pt}%
\pgfpathmoveto{\pgfqpoint{2.518786in}{3.984333in}}%
\pgfpathcurveto{\pgfqpoint{2.529836in}{3.984333in}}{\pgfqpoint{2.540435in}{3.988724in}}{\pgfqpoint{2.548249in}{3.996537in}}%
\pgfpathcurveto{\pgfqpoint{2.556062in}{4.004351in}}{\pgfqpoint{2.560452in}{4.014950in}}{\pgfqpoint{2.560452in}{4.026000in}}%
\pgfpathcurveto{\pgfqpoint{2.560452in}{4.037050in}}{\pgfqpoint{2.556062in}{4.047649in}}{\pgfqpoint{2.548249in}{4.055463in}}%
\pgfpathcurveto{\pgfqpoint{2.540435in}{4.063276in}}{\pgfqpoint{2.529836in}{4.067667in}}{\pgfqpoint{2.518786in}{4.067667in}}%
\pgfpathcurveto{\pgfqpoint{2.507736in}{4.067667in}}{\pgfqpoint{2.497137in}{4.063276in}}{\pgfqpoint{2.489323in}{4.055463in}}%
\pgfpathcurveto{\pgfqpoint{2.481509in}{4.047649in}}{\pgfqpoint{2.477119in}{4.037050in}}{\pgfqpoint{2.477119in}{4.026000in}}%
\pgfpathcurveto{\pgfqpoint{2.477119in}{4.014950in}}{\pgfqpoint{2.481509in}{4.004351in}}{\pgfqpoint{2.489323in}{3.996537in}}%
\pgfpathcurveto{\pgfqpoint{2.497137in}{3.988724in}}{\pgfqpoint{2.507736in}{3.984333in}}{\pgfqpoint{2.518786in}{3.984333in}}%
\pgfpathclose%
\pgfusepath{stroke,fill}%
\end{pgfscope}%
\begin{pgfscope}%
\pgfpathrectangle{\pgfqpoint{0.800000in}{0.528000in}}{\pgfqpoint{4.960000in}{3.696000in}}%
\pgfusepath{clip}%
\pgfsetbuttcap%
\pgfsetroundjoin%
\definecolor{currentfill}{rgb}{0.000000,0.000000,0.000000}%
\pgfsetfillcolor{currentfill}%
\pgfsetlinewidth{1.003750pt}%
\definecolor{currentstroke}{rgb}{0.000000,0.000000,0.000000}%
\pgfsetstrokecolor{currentstroke}%
\pgfsetdash{}{0pt}%
\pgfpathmoveto{\pgfqpoint{2.518786in}{3.984333in}}%
\pgfpathcurveto{\pgfqpoint{2.529836in}{3.984333in}}{\pgfqpoint{2.540435in}{3.988724in}}{\pgfqpoint{2.548249in}{3.996537in}}%
\pgfpathcurveto{\pgfqpoint{2.556062in}{4.004351in}}{\pgfqpoint{2.560452in}{4.014950in}}{\pgfqpoint{2.560452in}{4.026000in}}%
\pgfpathcurveto{\pgfqpoint{2.560452in}{4.037050in}}{\pgfqpoint{2.556062in}{4.047649in}}{\pgfqpoint{2.548249in}{4.055463in}}%
\pgfpathcurveto{\pgfqpoint{2.540435in}{4.063276in}}{\pgfqpoint{2.529836in}{4.067667in}}{\pgfqpoint{2.518786in}{4.067667in}}%
\pgfpathcurveto{\pgfqpoint{2.507736in}{4.067667in}}{\pgfqpoint{2.497137in}{4.063276in}}{\pgfqpoint{2.489323in}{4.055463in}}%
\pgfpathcurveto{\pgfqpoint{2.481509in}{4.047649in}}{\pgfqpoint{2.477119in}{4.037050in}}{\pgfqpoint{2.477119in}{4.026000in}}%
\pgfpathcurveto{\pgfqpoint{2.477119in}{4.014950in}}{\pgfqpoint{2.481509in}{4.004351in}}{\pgfqpoint{2.489323in}{3.996537in}}%
\pgfpathcurveto{\pgfqpoint{2.497137in}{3.988724in}}{\pgfqpoint{2.507736in}{3.984333in}}{\pgfqpoint{2.518786in}{3.984333in}}%
\pgfpathclose%
\pgfusepath{stroke,fill}%
\end{pgfscope}%
\begin{pgfscope}%
\pgfpathrectangle{\pgfqpoint{0.800000in}{0.528000in}}{\pgfqpoint{4.960000in}{3.696000in}}%
\pgfusepath{clip}%
\pgfsetbuttcap%
\pgfsetroundjoin%
\definecolor{currentfill}{rgb}{0.000000,0.000000,0.000000}%
\pgfsetfillcolor{currentfill}%
\pgfsetlinewidth{1.003750pt}%
\definecolor{currentstroke}{rgb}{0.000000,0.000000,0.000000}%
\pgfsetstrokecolor{currentstroke}%
\pgfsetdash{}{0pt}%
\pgfpathmoveto{\pgfqpoint{2.518786in}{3.984333in}}%
\pgfpathcurveto{\pgfqpoint{2.529836in}{3.984333in}}{\pgfqpoint{2.540435in}{3.988724in}}{\pgfqpoint{2.548249in}{3.996537in}}%
\pgfpathcurveto{\pgfqpoint{2.556062in}{4.004351in}}{\pgfqpoint{2.560452in}{4.014950in}}{\pgfqpoint{2.560452in}{4.026000in}}%
\pgfpathcurveto{\pgfqpoint{2.560452in}{4.037050in}}{\pgfqpoint{2.556062in}{4.047649in}}{\pgfqpoint{2.548249in}{4.055463in}}%
\pgfpathcurveto{\pgfqpoint{2.540435in}{4.063276in}}{\pgfqpoint{2.529836in}{4.067667in}}{\pgfqpoint{2.518786in}{4.067667in}}%
\pgfpathcurveto{\pgfqpoint{2.507736in}{4.067667in}}{\pgfqpoint{2.497137in}{4.063276in}}{\pgfqpoint{2.489323in}{4.055463in}}%
\pgfpathcurveto{\pgfqpoint{2.481509in}{4.047649in}}{\pgfqpoint{2.477119in}{4.037050in}}{\pgfqpoint{2.477119in}{4.026000in}}%
\pgfpathcurveto{\pgfqpoint{2.477119in}{4.014950in}}{\pgfqpoint{2.481509in}{4.004351in}}{\pgfqpoint{2.489323in}{3.996537in}}%
\pgfpathcurveto{\pgfqpoint{2.497137in}{3.988724in}}{\pgfqpoint{2.507736in}{3.984333in}}{\pgfqpoint{2.518786in}{3.984333in}}%
\pgfpathclose%
\pgfusepath{stroke,fill}%
\end{pgfscope}%
\begin{pgfscope}%
\pgfpathrectangle{\pgfqpoint{0.800000in}{0.528000in}}{\pgfqpoint{4.960000in}{3.696000in}}%
\pgfusepath{clip}%
\pgfsetbuttcap%
\pgfsetroundjoin%
\definecolor{currentfill}{rgb}{0.000000,0.000000,0.000000}%
\pgfsetfillcolor{currentfill}%
\pgfsetlinewidth{1.003750pt}%
\definecolor{currentstroke}{rgb}{0.000000,0.000000,0.000000}%
\pgfsetstrokecolor{currentstroke}%
\pgfsetdash{}{0pt}%
\pgfpathmoveto{\pgfqpoint{2.518786in}{3.984333in}}%
\pgfpathcurveto{\pgfqpoint{2.529836in}{3.984333in}}{\pgfqpoint{2.540435in}{3.988724in}}{\pgfqpoint{2.548249in}{3.996537in}}%
\pgfpathcurveto{\pgfqpoint{2.556062in}{4.004351in}}{\pgfqpoint{2.560452in}{4.014950in}}{\pgfqpoint{2.560452in}{4.026000in}}%
\pgfpathcurveto{\pgfqpoint{2.560452in}{4.037050in}}{\pgfqpoint{2.556062in}{4.047649in}}{\pgfqpoint{2.548249in}{4.055463in}}%
\pgfpathcurveto{\pgfqpoint{2.540435in}{4.063276in}}{\pgfqpoint{2.529836in}{4.067667in}}{\pgfqpoint{2.518786in}{4.067667in}}%
\pgfpathcurveto{\pgfqpoint{2.507736in}{4.067667in}}{\pgfqpoint{2.497137in}{4.063276in}}{\pgfqpoint{2.489323in}{4.055463in}}%
\pgfpathcurveto{\pgfqpoint{2.481509in}{4.047649in}}{\pgfqpoint{2.477119in}{4.037050in}}{\pgfqpoint{2.477119in}{4.026000in}}%
\pgfpathcurveto{\pgfqpoint{2.477119in}{4.014950in}}{\pgfqpoint{2.481509in}{4.004351in}}{\pgfqpoint{2.489323in}{3.996537in}}%
\pgfpathcurveto{\pgfqpoint{2.497137in}{3.988724in}}{\pgfqpoint{2.507736in}{3.984333in}}{\pgfqpoint{2.518786in}{3.984333in}}%
\pgfpathclose%
\pgfusepath{stroke,fill}%
\end{pgfscope}%
\begin{pgfscope}%
\pgfpathrectangle{\pgfqpoint{0.800000in}{0.528000in}}{\pgfqpoint{4.960000in}{3.696000in}}%
\pgfusepath{clip}%
\pgfsetbuttcap%
\pgfsetroundjoin%
\definecolor{currentfill}{rgb}{0.000000,0.000000,0.000000}%
\pgfsetfillcolor{currentfill}%
\pgfsetlinewidth{1.003750pt}%
\definecolor{currentstroke}{rgb}{0.000000,0.000000,0.000000}%
\pgfsetstrokecolor{currentstroke}%
\pgfsetdash{}{0pt}%
\pgfpathmoveto{\pgfqpoint{2.518786in}{3.984333in}}%
\pgfpathcurveto{\pgfqpoint{2.529836in}{3.984333in}}{\pgfqpoint{2.540435in}{3.988724in}}{\pgfqpoint{2.548249in}{3.996537in}}%
\pgfpathcurveto{\pgfqpoint{2.556062in}{4.004351in}}{\pgfqpoint{2.560452in}{4.014950in}}{\pgfqpoint{2.560452in}{4.026000in}}%
\pgfpathcurveto{\pgfqpoint{2.560452in}{4.037050in}}{\pgfqpoint{2.556062in}{4.047649in}}{\pgfqpoint{2.548249in}{4.055463in}}%
\pgfpathcurveto{\pgfqpoint{2.540435in}{4.063276in}}{\pgfqpoint{2.529836in}{4.067667in}}{\pgfqpoint{2.518786in}{4.067667in}}%
\pgfpathcurveto{\pgfqpoint{2.507736in}{4.067667in}}{\pgfqpoint{2.497137in}{4.063276in}}{\pgfqpoint{2.489323in}{4.055463in}}%
\pgfpathcurveto{\pgfqpoint{2.481509in}{4.047649in}}{\pgfqpoint{2.477119in}{4.037050in}}{\pgfqpoint{2.477119in}{4.026000in}}%
\pgfpathcurveto{\pgfqpoint{2.477119in}{4.014950in}}{\pgfqpoint{2.481509in}{4.004351in}}{\pgfqpoint{2.489323in}{3.996537in}}%
\pgfpathcurveto{\pgfqpoint{2.497137in}{3.988724in}}{\pgfqpoint{2.507736in}{3.984333in}}{\pgfqpoint{2.518786in}{3.984333in}}%
\pgfpathclose%
\pgfusepath{stroke,fill}%
\end{pgfscope}%
\begin{pgfscope}%
\pgfpathrectangle{\pgfqpoint{0.800000in}{0.528000in}}{\pgfqpoint{4.960000in}{3.696000in}}%
\pgfusepath{clip}%
\pgfsetbuttcap%
\pgfsetroundjoin%
\definecolor{currentfill}{rgb}{0.000000,0.000000,0.000000}%
\pgfsetfillcolor{currentfill}%
\pgfsetlinewidth{1.003750pt}%
\definecolor{currentstroke}{rgb}{0.000000,0.000000,0.000000}%
\pgfsetstrokecolor{currentstroke}%
\pgfsetdash{}{0pt}%
\pgfpathmoveto{\pgfqpoint{2.518786in}{3.984333in}}%
\pgfpathcurveto{\pgfqpoint{2.529836in}{3.984333in}}{\pgfqpoint{2.540435in}{3.988724in}}{\pgfqpoint{2.548249in}{3.996537in}}%
\pgfpathcurveto{\pgfqpoint{2.556062in}{4.004351in}}{\pgfqpoint{2.560452in}{4.014950in}}{\pgfqpoint{2.560452in}{4.026000in}}%
\pgfpathcurveto{\pgfqpoint{2.560452in}{4.037050in}}{\pgfqpoint{2.556062in}{4.047649in}}{\pgfqpoint{2.548249in}{4.055463in}}%
\pgfpathcurveto{\pgfqpoint{2.540435in}{4.063276in}}{\pgfqpoint{2.529836in}{4.067667in}}{\pgfqpoint{2.518786in}{4.067667in}}%
\pgfpathcurveto{\pgfqpoint{2.507736in}{4.067667in}}{\pgfqpoint{2.497137in}{4.063276in}}{\pgfqpoint{2.489323in}{4.055463in}}%
\pgfpathcurveto{\pgfqpoint{2.481509in}{4.047649in}}{\pgfqpoint{2.477119in}{4.037050in}}{\pgfqpoint{2.477119in}{4.026000in}}%
\pgfpathcurveto{\pgfqpoint{2.477119in}{4.014950in}}{\pgfqpoint{2.481509in}{4.004351in}}{\pgfqpoint{2.489323in}{3.996537in}}%
\pgfpathcurveto{\pgfqpoint{2.497137in}{3.988724in}}{\pgfqpoint{2.507736in}{3.984333in}}{\pgfqpoint{2.518786in}{3.984333in}}%
\pgfpathclose%
\pgfusepath{stroke,fill}%
\end{pgfscope}%
\begin{pgfscope}%
\pgfpathrectangle{\pgfqpoint{0.800000in}{0.528000in}}{\pgfqpoint{4.960000in}{3.696000in}}%
\pgfusepath{clip}%
\pgfsetbuttcap%
\pgfsetroundjoin%
\definecolor{currentfill}{rgb}{0.000000,0.000000,0.000000}%
\pgfsetfillcolor{currentfill}%
\pgfsetlinewidth{1.003750pt}%
\definecolor{currentstroke}{rgb}{0.000000,0.000000,0.000000}%
\pgfsetstrokecolor{currentstroke}%
\pgfsetdash{}{0pt}%
\pgfpathmoveto{\pgfqpoint{2.518786in}{3.984333in}}%
\pgfpathcurveto{\pgfqpoint{2.529836in}{3.984333in}}{\pgfqpoint{2.540435in}{3.988724in}}{\pgfqpoint{2.548249in}{3.996537in}}%
\pgfpathcurveto{\pgfqpoint{2.556062in}{4.004351in}}{\pgfqpoint{2.560452in}{4.014950in}}{\pgfqpoint{2.560452in}{4.026000in}}%
\pgfpathcurveto{\pgfqpoint{2.560452in}{4.037050in}}{\pgfqpoint{2.556062in}{4.047649in}}{\pgfqpoint{2.548249in}{4.055463in}}%
\pgfpathcurveto{\pgfqpoint{2.540435in}{4.063276in}}{\pgfqpoint{2.529836in}{4.067667in}}{\pgfqpoint{2.518786in}{4.067667in}}%
\pgfpathcurveto{\pgfqpoint{2.507736in}{4.067667in}}{\pgfqpoint{2.497137in}{4.063276in}}{\pgfqpoint{2.489323in}{4.055463in}}%
\pgfpathcurveto{\pgfqpoint{2.481509in}{4.047649in}}{\pgfqpoint{2.477119in}{4.037050in}}{\pgfqpoint{2.477119in}{4.026000in}}%
\pgfpathcurveto{\pgfqpoint{2.477119in}{4.014950in}}{\pgfqpoint{2.481509in}{4.004351in}}{\pgfqpoint{2.489323in}{3.996537in}}%
\pgfpathcurveto{\pgfqpoint{2.497137in}{3.988724in}}{\pgfqpoint{2.507736in}{3.984333in}}{\pgfqpoint{2.518786in}{3.984333in}}%
\pgfpathclose%
\pgfusepath{stroke,fill}%
\end{pgfscope}%
\begin{pgfscope}%
\pgfpathrectangle{\pgfqpoint{0.800000in}{0.528000in}}{\pgfqpoint{4.960000in}{3.696000in}}%
\pgfusepath{clip}%
\pgfsetbuttcap%
\pgfsetroundjoin%
\definecolor{currentfill}{rgb}{0.000000,0.000000,0.000000}%
\pgfsetfillcolor{currentfill}%
\pgfsetlinewidth{1.003750pt}%
\definecolor{currentstroke}{rgb}{0.000000,0.000000,0.000000}%
\pgfsetstrokecolor{currentstroke}%
\pgfsetdash{}{0pt}%
\pgfpathmoveto{\pgfqpoint{2.518786in}{3.984333in}}%
\pgfpathcurveto{\pgfqpoint{2.529836in}{3.984333in}}{\pgfqpoint{2.540435in}{3.988724in}}{\pgfqpoint{2.548249in}{3.996537in}}%
\pgfpathcurveto{\pgfqpoint{2.556062in}{4.004351in}}{\pgfqpoint{2.560452in}{4.014950in}}{\pgfqpoint{2.560452in}{4.026000in}}%
\pgfpathcurveto{\pgfqpoint{2.560452in}{4.037050in}}{\pgfqpoint{2.556062in}{4.047649in}}{\pgfqpoint{2.548249in}{4.055463in}}%
\pgfpathcurveto{\pgfqpoint{2.540435in}{4.063276in}}{\pgfqpoint{2.529836in}{4.067667in}}{\pgfqpoint{2.518786in}{4.067667in}}%
\pgfpathcurveto{\pgfqpoint{2.507736in}{4.067667in}}{\pgfqpoint{2.497137in}{4.063276in}}{\pgfqpoint{2.489323in}{4.055463in}}%
\pgfpathcurveto{\pgfqpoint{2.481509in}{4.047649in}}{\pgfqpoint{2.477119in}{4.037050in}}{\pgfqpoint{2.477119in}{4.026000in}}%
\pgfpathcurveto{\pgfqpoint{2.477119in}{4.014950in}}{\pgfqpoint{2.481509in}{4.004351in}}{\pgfqpoint{2.489323in}{3.996537in}}%
\pgfpathcurveto{\pgfqpoint{2.497137in}{3.988724in}}{\pgfqpoint{2.507736in}{3.984333in}}{\pgfqpoint{2.518786in}{3.984333in}}%
\pgfpathclose%
\pgfusepath{stroke,fill}%
\end{pgfscope}%
\begin{pgfscope}%
\pgfpathrectangle{\pgfqpoint{0.800000in}{0.528000in}}{\pgfqpoint{4.960000in}{3.696000in}}%
\pgfusepath{clip}%
\pgfsetbuttcap%
\pgfsetroundjoin%
\definecolor{currentfill}{rgb}{0.000000,0.000000,0.000000}%
\pgfsetfillcolor{currentfill}%
\pgfsetlinewidth{1.003750pt}%
\definecolor{currentstroke}{rgb}{0.000000,0.000000,0.000000}%
\pgfsetstrokecolor{currentstroke}%
\pgfsetdash{}{0pt}%
\pgfpathmoveto{\pgfqpoint{2.518786in}{3.984333in}}%
\pgfpathcurveto{\pgfqpoint{2.529836in}{3.984333in}}{\pgfqpoint{2.540435in}{3.988724in}}{\pgfqpoint{2.548249in}{3.996537in}}%
\pgfpathcurveto{\pgfqpoint{2.556062in}{4.004351in}}{\pgfqpoint{2.560452in}{4.014950in}}{\pgfqpoint{2.560452in}{4.026000in}}%
\pgfpathcurveto{\pgfqpoint{2.560452in}{4.037050in}}{\pgfqpoint{2.556062in}{4.047649in}}{\pgfqpoint{2.548249in}{4.055463in}}%
\pgfpathcurveto{\pgfqpoint{2.540435in}{4.063276in}}{\pgfqpoint{2.529836in}{4.067667in}}{\pgfqpoint{2.518786in}{4.067667in}}%
\pgfpathcurveto{\pgfqpoint{2.507736in}{4.067667in}}{\pgfqpoint{2.497137in}{4.063276in}}{\pgfqpoint{2.489323in}{4.055463in}}%
\pgfpathcurveto{\pgfqpoint{2.481509in}{4.047649in}}{\pgfqpoint{2.477119in}{4.037050in}}{\pgfqpoint{2.477119in}{4.026000in}}%
\pgfpathcurveto{\pgfqpoint{2.477119in}{4.014950in}}{\pgfqpoint{2.481509in}{4.004351in}}{\pgfqpoint{2.489323in}{3.996537in}}%
\pgfpathcurveto{\pgfqpoint{2.497137in}{3.988724in}}{\pgfqpoint{2.507736in}{3.984333in}}{\pgfqpoint{2.518786in}{3.984333in}}%
\pgfpathclose%
\pgfusepath{stroke,fill}%
\end{pgfscope}%
\begin{pgfscope}%
\pgfpathrectangle{\pgfqpoint{0.800000in}{0.528000in}}{\pgfqpoint{4.960000in}{3.696000in}}%
\pgfusepath{clip}%
\pgfsetbuttcap%
\pgfsetroundjoin%
\definecolor{currentfill}{rgb}{0.000000,0.000000,0.000000}%
\pgfsetfillcolor{currentfill}%
\pgfsetlinewidth{1.003750pt}%
\definecolor{currentstroke}{rgb}{0.000000,0.000000,0.000000}%
\pgfsetstrokecolor{currentstroke}%
\pgfsetdash{}{0pt}%
\pgfpathmoveto{\pgfqpoint{2.518786in}{3.984333in}}%
\pgfpathcurveto{\pgfqpoint{2.529836in}{3.984333in}}{\pgfqpoint{2.540435in}{3.988724in}}{\pgfqpoint{2.548249in}{3.996537in}}%
\pgfpathcurveto{\pgfqpoint{2.556062in}{4.004351in}}{\pgfqpoint{2.560452in}{4.014950in}}{\pgfqpoint{2.560452in}{4.026000in}}%
\pgfpathcurveto{\pgfqpoint{2.560452in}{4.037050in}}{\pgfqpoint{2.556062in}{4.047649in}}{\pgfqpoint{2.548249in}{4.055463in}}%
\pgfpathcurveto{\pgfqpoint{2.540435in}{4.063276in}}{\pgfqpoint{2.529836in}{4.067667in}}{\pgfqpoint{2.518786in}{4.067667in}}%
\pgfpathcurveto{\pgfqpoint{2.507736in}{4.067667in}}{\pgfqpoint{2.497137in}{4.063276in}}{\pgfqpoint{2.489323in}{4.055463in}}%
\pgfpathcurveto{\pgfqpoint{2.481509in}{4.047649in}}{\pgfqpoint{2.477119in}{4.037050in}}{\pgfqpoint{2.477119in}{4.026000in}}%
\pgfpathcurveto{\pgfqpoint{2.477119in}{4.014950in}}{\pgfqpoint{2.481509in}{4.004351in}}{\pgfqpoint{2.489323in}{3.996537in}}%
\pgfpathcurveto{\pgfqpoint{2.497137in}{3.988724in}}{\pgfqpoint{2.507736in}{3.984333in}}{\pgfqpoint{2.518786in}{3.984333in}}%
\pgfpathclose%
\pgfusepath{stroke,fill}%
\end{pgfscope}%
\begin{pgfscope}%
\pgfpathrectangle{\pgfqpoint{0.800000in}{0.528000in}}{\pgfqpoint{4.960000in}{3.696000in}}%
\pgfusepath{clip}%
\pgfsetbuttcap%
\pgfsetroundjoin%
\definecolor{currentfill}{rgb}{0.000000,0.000000,0.000000}%
\pgfsetfillcolor{currentfill}%
\pgfsetlinewidth{1.003750pt}%
\definecolor{currentstroke}{rgb}{0.000000,0.000000,0.000000}%
\pgfsetstrokecolor{currentstroke}%
\pgfsetdash{}{0pt}%
\pgfpathmoveto{\pgfqpoint{2.518786in}{3.984333in}}%
\pgfpathcurveto{\pgfqpoint{2.529836in}{3.984333in}}{\pgfqpoint{2.540435in}{3.988724in}}{\pgfqpoint{2.548249in}{3.996537in}}%
\pgfpathcurveto{\pgfqpoint{2.556062in}{4.004351in}}{\pgfqpoint{2.560452in}{4.014950in}}{\pgfqpoint{2.560452in}{4.026000in}}%
\pgfpathcurveto{\pgfqpoint{2.560452in}{4.037050in}}{\pgfqpoint{2.556062in}{4.047649in}}{\pgfqpoint{2.548249in}{4.055463in}}%
\pgfpathcurveto{\pgfqpoint{2.540435in}{4.063276in}}{\pgfqpoint{2.529836in}{4.067667in}}{\pgfqpoint{2.518786in}{4.067667in}}%
\pgfpathcurveto{\pgfqpoint{2.507736in}{4.067667in}}{\pgfqpoint{2.497137in}{4.063276in}}{\pgfqpoint{2.489323in}{4.055463in}}%
\pgfpathcurveto{\pgfqpoint{2.481509in}{4.047649in}}{\pgfqpoint{2.477119in}{4.037050in}}{\pgfqpoint{2.477119in}{4.026000in}}%
\pgfpathcurveto{\pgfqpoint{2.477119in}{4.014950in}}{\pgfqpoint{2.481509in}{4.004351in}}{\pgfqpoint{2.489323in}{3.996537in}}%
\pgfpathcurveto{\pgfqpoint{2.497137in}{3.988724in}}{\pgfqpoint{2.507736in}{3.984333in}}{\pgfqpoint{2.518786in}{3.984333in}}%
\pgfpathclose%
\pgfusepath{stroke,fill}%
\end{pgfscope}%
\begin{pgfscope}%
\pgfpathrectangle{\pgfqpoint{0.800000in}{0.528000in}}{\pgfqpoint{4.960000in}{3.696000in}}%
\pgfusepath{clip}%
\pgfsetbuttcap%
\pgfsetroundjoin%
\definecolor{currentfill}{rgb}{0.000000,0.000000,0.000000}%
\pgfsetfillcolor{currentfill}%
\pgfsetlinewidth{1.003750pt}%
\definecolor{currentstroke}{rgb}{0.000000,0.000000,0.000000}%
\pgfsetstrokecolor{currentstroke}%
\pgfsetdash{}{0pt}%
\pgfpathmoveto{\pgfqpoint{2.518786in}{3.984333in}}%
\pgfpathcurveto{\pgfqpoint{2.529836in}{3.984333in}}{\pgfqpoint{2.540435in}{3.988724in}}{\pgfqpoint{2.548249in}{3.996537in}}%
\pgfpathcurveto{\pgfqpoint{2.556062in}{4.004351in}}{\pgfqpoint{2.560452in}{4.014950in}}{\pgfqpoint{2.560452in}{4.026000in}}%
\pgfpathcurveto{\pgfqpoint{2.560452in}{4.037050in}}{\pgfqpoint{2.556062in}{4.047649in}}{\pgfqpoint{2.548249in}{4.055463in}}%
\pgfpathcurveto{\pgfqpoint{2.540435in}{4.063276in}}{\pgfqpoint{2.529836in}{4.067667in}}{\pgfqpoint{2.518786in}{4.067667in}}%
\pgfpathcurveto{\pgfqpoint{2.507736in}{4.067667in}}{\pgfqpoint{2.497137in}{4.063276in}}{\pgfqpoint{2.489323in}{4.055463in}}%
\pgfpathcurveto{\pgfqpoint{2.481509in}{4.047649in}}{\pgfqpoint{2.477119in}{4.037050in}}{\pgfqpoint{2.477119in}{4.026000in}}%
\pgfpathcurveto{\pgfqpoint{2.477119in}{4.014950in}}{\pgfqpoint{2.481509in}{4.004351in}}{\pgfqpoint{2.489323in}{3.996537in}}%
\pgfpathcurveto{\pgfqpoint{2.497137in}{3.988724in}}{\pgfqpoint{2.507736in}{3.984333in}}{\pgfqpoint{2.518786in}{3.984333in}}%
\pgfpathclose%
\pgfusepath{stroke,fill}%
\end{pgfscope}%
\begin{pgfscope}%
\pgfpathrectangle{\pgfqpoint{0.800000in}{0.528000in}}{\pgfqpoint{4.960000in}{3.696000in}}%
\pgfusepath{clip}%
\pgfsetbuttcap%
\pgfsetroundjoin%
\definecolor{currentfill}{rgb}{0.000000,0.000000,0.000000}%
\pgfsetfillcolor{currentfill}%
\pgfsetlinewidth{1.003750pt}%
\definecolor{currentstroke}{rgb}{0.000000,0.000000,0.000000}%
\pgfsetstrokecolor{currentstroke}%
\pgfsetdash{}{0pt}%
\pgfpathmoveto{\pgfqpoint{2.518786in}{3.984333in}}%
\pgfpathcurveto{\pgfqpoint{2.529836in}{3.984333in}}{\pgfqpoint{2.540435in}{3.988724in}}{\pgfqpoint{2.548249in}{3.996537in}}%
\pgfpathcurveto{\pgfqpoint{2.556062in}{4.004351in}}{\pgfqpoint{2.560452in}{4.014950in}}{\pgfqpoint{2.560452in}{4.026000in}}%
\pgfpathcurveto{\pgfqpoint{2.560452in}{4.037050in}}{\pgfqpoint{2.556062in}{4.047649in}}{\pgfqpoint{2.548249in}{4.055463in}}%
\pgfpathcurveto{\pgfqpoint{2.540435in}{4.063276in}}{\pgfqpoint{2.529836in}{4.067667in}}{\pgfqpoint{2.518786in}{4.067667in}}%
\pgfpathcurveto{\pgfqpoint{2.507736in}{4.067667in}}{\pgfqpoint{2.497137in}{4.063276in}}{\pgfqpoint{2.489323in}{4.055463in}}%
\pgfpathcurveto{\pgfqpoint{2.481509in}{4.047649in}}{\pgfqpoint{2.477119in}{4.037050in}}{\pgfqpoint{2.477119in}{4.026000in}}%
\pgfpathcurveto{\pgfqpoint{2.477119in}{4.014950in}}{\pgfqpoint{2.481509in}{4.004351in}}{\pgfqpoint{2.489323in}{3.996537in}}%
\pgfpathcurveto{\pgfqpoint{2.497137in}{3.988724in}}{\pgfqpoint{2.507736in}{3.984333in}}{\pgfqpoint{2.518786in}{3.984333in}}%
\pgfpathclose%
\pgfusepath{stroke,fill}%
\end{pgfscope}%
\begin{pgfscope}%
\pgfpathrectangle{\pgfqpoint{0.800000in}{0.528000in}}{\pgfqpoint{4.960000in}{3.696000in}}%
\pgfusepath{clip}%
\pgfsetbuttcap%
\pgfsetroundjoin%
\definecolor{currentfill}{rgb}{0.000000,0.000000,0.000000}%
\pgfsetfillcolor{currentfill}%
\pgfsetlinewidth{1.003750pt}%
\definecolor{currentstroke}{rgb}{0.000000,0.000000,0.000000}%
\pgfsetstrokecolor{currentstroke}%
\pgfsetdash{}{0pt}%
\pgfpathmoveto{\pgfqpoint{2.518786in}{3.984333in}}%
\pgfpathcurveto{\pgfqpoint{2.529836in}{3.984333in}}{\pgfqpoint{2.540435in}{3.988724in}}{\pgfqpoint{2.548249in}{3.996537in}}%
\pgfpathcurveto{\pgfqpoint{2.556062in}{4.004351in}}{\pgfqpoint{2.560452in}{4.014950in}}{\pgfqpoint{2.560452in}{4.026000in}}%
\pgfpathcurveto{\pgfqpoint{2.560452in}{4.037050in}}{\pgfqpoint{2.556062in}{4.047649in}}{\pgfqpoint{2.548249in}{4.055463in}}%
\pgfpathcurveto{\pgfqpoint{2.540435in}{4.063276in}}{\pgfqpoint{2.529836in}{4.067667in}}{\pgfqpoint{2.518786in}{4.067667in}}%
\pgfpathcurveto{\pgfqpoint{2.507736in}{4.067667in}}{\pgfqpoint{2.497137in}{4.063276in}}{\pgfqpoint{2.489323in}{4.055463in}}%
\pgfpathcurveto{\pgfqpoint{2.481509in}{4.047649in}}{\pgfqpoint{2.477119in}{4.037050in}}{\pgfqpoint{2.477119in}{4.026000in}}%
\pgfpathcurveto{\pgfqpoint{2.477119in}{4.014950in}}{\pgfqpoint{2.481509in}{4.004351in}}{\pgfqpoint{2.489323in}{3.996537in}}%
\pgfpathcurveto{\pgfqpoint{2.497137in}{3.988724in}}{\pgfqpoint{2.507736in}{3.984333in}}{\pgfqpoint{2.518786in}{3.984333in}}%
\pgfpathclose%
\pgfusepath{stroke,fill}%
\end{pgfscope}%
\begin{pgfscope}%
\pgfpathrectangle{\pgfqpoint{0.800000in}{0.528000in}}{\pgfqpoint{4.960000in}{3.696000in}}%
\pgfusepath{clip}%
\pgfsetbuttcap%
\pgfsetroundjoin%
\definecolor{currentfill}{rgb}{0.000000,0.000000,0.000000}%
\pgfsetfillcolor{currentfill}%
\pgfsetlinewidth{1.003750pt}%
\definecolor{currentstroke}{rgb}{0.000000,0.000000,0.000000}%
\pgfsetstrokecolor{currentstroke}%
\pgfsetdash{}{0pt}%
\pgfpathmoveto{\pgfqpoint{2.518786in}{3.984333in}}%
\pgfpathcurveto{\pgfqpoint{2.529836in}{3.984333in}}{\pgfqpoint{2.540435in}{3.988724in}}{\pgfqpoint{2.548249in}{3.996537in}}%
\pgfpathcurveto{\pgfqpoint{2.556062in}{4.004351in}}{\pgfqpoint{2.560452in}{4.014950in}}{\pgfqpoint{2.560452in}{4.026000in}}%
\pgfpathcurveto{\pgfqpoint{2.560452in}{4.037050in}}{\pgfqpoint{2.556062in}{4.047649in}}{\pgfqpoint{2.548249in}{4.055463in}}%
\pgfpathcurveto{\pgfqpoint{2.540435in}{4.063276in}}{\pgfqpoint{2.529836in}{4.067667in}}{\pgfqpoint{2.518786in}{4.067667in}}%
\pgfpathcurveto{\pgfqpoint{2.507736in}{4.067667in}}{\pgfqpoint{2.497137in}{4.063276in}}{\pgfqpoint{2.489323in}{4.055463in}}%
\pgfpathcurveto{\pgfqpoint{2.481509in}{4.047649in}}{\pgfqpoint{2.477119in}{4.037050in}}{\pgfqpoint{2.477119in}{4.026000in}}%
\pgfpathcurveto{\pgfqpoint{2.477119in}{4.014950in}}{\pgfqpoint{2.481509in}{4.004351in}}{\pgfqpoint{2.489323in}{3.996537in}}%
\pgfpathcurveto{\pgfqpoint{2.497137in}{3.988724in}}{\pgfqpoint{2.507736in}{3.984333in}}{\pgfqpoint{2.518786in}{3.984333in}}%
\pgfpathclose%
\pgfusepath{stroke,fill}%
\end{pgfscope}%
\begin{pgfscope}%
\pgfpathrectangle{\pgfqpoint{0.800000in}{0.528000in}}{\pgfqpoint{4.960000in}{3.696000in}}%
\pgfusepath{clip}%
\pgfsetbuttcap%
\pgfsetroundjoin%
\definecolor{currentfill}{rgb}{0.000000,0.000000,0.000000}%
\pgfsetfillcolor{currentfill}%
\pgfsetlinewidth{1.003750pt}%
\definecolor{currentstroke}{rgb}{0.000000,0.000000,0.000000}%
\pgfsetstrokecolor{currentstroke}%
\pgfsetdash{}{0pt}%
\pgfpathmoveto{\pgfqpoint{2.518786in}{3.984333in}}%
\pgfpathcurveto{\pgfqpoint{2.529836in}{3.984333in}}{\pgfqpoint{2.540435in}{3.988724in}}{\pgfqpoint{2.548249in}{3.996537in}}%
\pgfpathcurveto{\pgfqpoint{2.556062in}{4.004351in}}{\pgfqpoint{2.560452in}{4.014950in}}{\pgfqpoint{2.560452in}{4.026000in}}%
\pgfpathcurveto{\pgfqpoint{2.560452in}{4.037050in}}{\pgfqpoint{2.556062in}{4.047649in}}{\pgfqpoint{2.548249in}{4.055463in}}%
\pgfpathcurveto{\pgfqpoint{2.540435in}{4.063276in}}{\pgfqpoint{2.529836in}{4.067667in}}{\pgfqpoint{2.518786in}{4.067667in}}%
\pgfpathcurveto{\pgfqpoint{2.507736in}{4.067667in}}{\pgfqpoint{2.497137in}{4.063276in}}{\pgfqpoint{2.489323in}{4.055463in}}%
\pgfpathcurveto{\pgfqpoint{2.481509in}{4.047649in}}{\pgfqpoint{2.477119in}{4.037050in}}{\pgfqpoint{2.477119in}{4.026000in}}%
\pgfpathcurveto{\pgfqpoint{2.477119in}{4.014950in}}{\pgfqpoint{2.481509in}{4.004351in}}{\pgfqpoint{2.489323in}{3.996537in}}%
\pgfpathcurveto{\pgfqpoint{2.497137in}{3.988724in}}{\pgfqpoint{2.507736in}{3.984333in}}{\pgfqpoint{2.518786in}{3.984333in}}%
\pgfpathclose%
\pgfusepath{stroke,fill}%
\end{pgfscope}%
\begin{pgfscope}%
\pgfpathrectangle{\pgfqpoint{0.800000in}{0.528000in}}{\pgfqpoint{4.960000in}{3.696000in}}%
\pgfusepath{clip}%
\pgfsetbuttcap%
\pgfsetroundjoin%
\definecolor{currentfill}{rgb}{0.000000,0.000000,0.000000}%
\pgfsetfillcolor{currentfill}%
\pgfsetlinewidth{1.003750pt}%
\definecolor{currentstroke}{rgb}{0.000000,0.000000,0.000000}%
\pgfsetstrokecolor{currentstroke}%
\pgfsetdash{}{0pt}%
\pgfpathmoveto{\pgfqpoint{2.518786in}{3.984333in}}%
\pgfpathcurveto{\pgfqpoint{2.529836in}{3.984333in}}{\pgfqpoint{2.540435in}{3.988724in}}{\pgfqpoint{2.548249in}{3.996537in}}%
\pgfpathcurveto{\pgfqpoint{2.556062in}{4.004351in}}{\pgfqpoint{2.560452in}{4.014950in}}{\pgfqpoint{2.560452in}{4.026000in}}%
\pgfpathcurveto{\pgfqpoint{2.560452in}{4.037050in}}{\pgfqpoint{2.556062in}{4.047649in}}{\pgfqpoint{2.548249in}{4.055463in}}%
\pgfpathcurveto{\pgfqpoint{2.540435in}{4.063276in}}{\pgfqpoint{2.529836in}{4.067667in}}{\pgfqpoint{2.518786in}{4.067667in}}%
\pgfpathcurveto{\pgfqpoint{2.507736in}{4.067667in}}{\pgfqpoint{2.497137in}{4.063276in}}{\pgfqpoint{2.489323in}{4.055463in}}%
\pgfpathcurveto{\pgfqpoint{2.481509in}{4.047649in}}{\pgfqpoint{2.477119in}{4.037050in}}{\pgfqpoint{2.477119in}{4.026000in}}%
\pgfpathcurveto{\pgfqpoint{2.477119in}{4.014950in}}{\pgfqpoint{2.481509in}{4.004351in}}{\pgfqpoint{2.489323in}{3.996537in}}%
\pgfpathcurveto{\pgfqpoint{2.497137in}{3.988724in}}{\pgfqpoint{2.507736in}{3.984333in}}{\pgfqpoint{2.518786in}{3.984333in}}%
\pgfpathclose%
\pgfusepath{stroke,fill}%
\end{pgfscope}%
\begin{pgfscope}%
\pgfpathrectangle{\pgfqpoint{0.800000in}{0.528000in}}{\pgfqpoint{4.960000in}{3.696000in}}%
\pgfusepath{clip}%
\pgfsetbuttcap%
\pgfsetroundjoin%
\definecolor{currentfill}{rgb}{0.000000,0.000000,0.000000}%
\pgfsetfillcolor{currentfill}%
\pgfsetlinewidth{1.003750pt}%
\definecolor{currentstroke}{rgb}{0.000000,0.000000,0.000000}%
\pgfsetstrokecolor{currentstroke}%
\pgfsetdash{}{0pt}%
\pgfpathmoveto{\pgfqpoint{2.518786in}{3.984333in}}%
\pgfpathcurveto{\pgfqpoint{2.529836in}{3.984333in}}{\pgfqpoint{2.540435in}{3.988724in}}{\pgfqpoint{2.548249in}{3.996537in}}%
\pgfpathcurveto{\pgfqpoint{2.556062in}{4.004351in}}{\pgfqpoint{2.560452in}{4.014950in}}{\pgfqpoint{2.560452in}{4.026000in}}%
\pgfpathcurveto{\pgfqpoint{2.560452in}{4.037050in}}{\pgfqpoint{2.556062in}{4.047649in}}{\pgfqpoint{2.548249in}{4.055463in}}%
\pgfpathcurveto{\pgfqpoint{2.540435in}{4.063276in}}{\pgfqpoint{2.529836in}{4.067667in}}{\pgfqpoint{2.518786in}{4.067667in}}%
\pgfpathcurveto{\pgfqpoint{2.507736in}{4.067667in}}{\pgfqpoint{2.497137in}{4.063276in}}{\pgfqpoint{2.489323in}{4.055463in}}%
\pgfpathcurveto{\pgfqpoint{2.481509in}{4.047649in}}{\pgfqpoint{2.477119in}{4.037050in}}{\pgfqpoint{2.477119in}{4.026000in}}%
\pgfpathcurveto{\pgfqpoint{2.477119in}{4.014950in}}{\pgfqpoint{2.481509in}{4.004351in}}{\pgfqpoint{2.489323in}{3.996537in}}%
\pgfpathcurveto{\pgfqpoint{2.497137in}{3.988724in}}{\pgfqpoint{2.507736in}{3.984333in}}{\pgfqpoint{2.518786in}{3.984333in}}%
\pgfpathclose%
\pgfusepath{stroke,fill}%
\end{pgfscope}%
\begin{pgfscope}%
\pgfpathrectangle{\pgfqpoint{0.800000in}{0.528000in}}{\pgfqpoint{4.960000in}{3.696000in}}%
\pgfusepath{clip}%
\pgfsetbuttcap%
\pgfsetroundjoin%
\definecolor{currentfill}{rgb}{0.000000,0.000000,0.000000}%
\pgfsetfillcolor{currentfill}%
\pgfsetlinewidth{1.003750pt}%
\definecolor{currentstroke}{rgb}{0.000000,0.000000,0.000000}%
\pgfsetstrokecolor{currentstroke}%
\pgfsetdash{}{0pt}%
\pgfpathmoveto{\pgfqpoint{2.518786in}{3.984333in}}%
\pgfpathcurveto{\pgfqpoint{2.529836in}{3.984333in}}{\pgfqpoint{2.540435in}{3.988724in}}{\pgfqpoint{2.548249in}{3.996537in}}%
\pgfpathcurveto{\pgfqpoint{2.556062in}{4.004351in}}{\pgfqpoint{2.560452in}{4.014950in}}{\pgfqpoint{2.560452in}{4.026000in}}%
\pgfpathcurveto{\pgfqpoint{2.560452in}{4.037050in}}{\pgfqpoint{2.556062in}{4.047649in}}{\pgfqpoint{2.548249in}{4.055463in}}%
\pgfpathcurveto{\pgfqpoint{2.540435in}{4.063276in}}{\pgfqpoint{2.529836in}{4.067667in}}{\pgfqpoint{2.518786in}{4.067667in}}%
\pgfpathcurveto{\pgfqpoint{2.507736in}{4.067667in}}{\pgfqpoint{2.497137in}{4.063276in}}{\pgfqpoint{2.489323in}{4.055463in}}%
\pgfpathcurveto{\pgfqpoint{2.481509in}{4.047649in}}{\pgfqpoint{2.477119in}{4.037050in}}{\pgfqpoint{2.477119in}{4.026000in}}%
\pgfpathcurveto{\pgfqpoint{2.477119in}{4.014950in}}{\pgfqpoint{2.481509in}{4.004351in}}{\pgfqpoint{2.489323in}{3.996537in}}%
\pgfpathcurveto{\pgfqpoint{2.497137in}{3.988724in}}{\pgfqpoint{2.507736in}{3.984333in}}{\pgfqpoint{2.518786in}{3.984333in}}%
\pgfpathclose%
\pgfusepath{stroke,fill}%
\end{pgfscope}%
\begin{pgfscope}%
\pgfpathrectangle{\pgfqpoint{0.800000in}{0.528000in}}{\pgfqpoint{4.960000in}{3.696000in}}%
\pgfusepath{clip}%
\pgfsetbuttcap%
\pgfsetroundjoin%
\definecolor{currentfill}{rgb}{0.000000,0.000000,0.000000}%
\pgfsetfillcolor{currentfill}%
\pgfsetlinewidth{1.003750pt}%
\definecolor{currentstroke}{rgb}{0.000000,0.000000,0.000000}%
\pgfsetstrokecolor{currentstroke}%
\pgfsetdash{}{0pt}%
\pgfpathmoveto{\pgfqpoint{2.518786in}{3.984333in}}%
\pgfpathcurveto{\pgfqpoint{2.529836in}{3.984333in}}{\pgfqpoint{2.540435in}{3.988724in}}{\pgfqpoint{2.548249in}{3.996537in}}%
\pgfpathcurveto{\pgfqpoint{2.556062in}{4.004351in}}{\pgfqpoint{2.560452in}{4.014950in}}{\pgfqpoint{2.560452in}{4.026000in}}%
\pgfpathcurveto{\pgfqpoint{2.560452in}{4.037050in}}{\pgfqpoint{2.556062in}{4.047649in}}{\pgfqpoint{2.548249in}{4.055463in}}%
\pgfpathcurveto{\pgfqpoint{2.540435in}{4.063276in}}{\pgfqpoint{2.529836in}{4.067667in}}{\pgfqpoint{2.518786in}{4.067667in}}%
\pgfpathcurveto{\pgfqpoint{2.507736in}{4.067667in}}{\pgfqpoint{2.497137in}{4.063276in}}{\pgfqpoint{2.489323in}{4.055463in}}%
\pgfpathcurveto{\pgfqpoint{2.481509in}{4.047649in}}{\pgfqpoint{2.477119in}{4.037050in}}{\pgfqpoint{2.477119in}{4.026000in}}%
\pgfpathcurveto{\pgfqpoint{2.477119in}{4.014950in}}{\pgfqpoint{2.481509in}{4.004351in}}{\pgfqpoint{2.489323in}{3.996537in}}%
\pgfpathcurveto{\pgfqpoint{2.497137in}{3.988724in}}{\pgfqpoint{2.507736in}{3.984333in}}{\pgfqpoint{2.518786in}{3.984333in}}%
\pgfpathclose%
\pgfusepath{stroke,fill}%
\end{pgfscope}%
\begin{pgfscope}%
\pgfpathrectangle{\pgfqpoint{0.800000in}{0.528000in}}{\pgfqpoint{4.960000in}{3.696000in}}%
\pgfusepath{clip}%
\pgfsetbuttcap%
\pgfsetroundjoin%
\definecolor{currentfill}{rgb}{0.000000,0.000000,0.000000}%
\pgfsetfillcolor{currentfill}%
\pgfsetlinewidth{1.003750pt}%
\definecolor{currentstroke}{rgb}{0.000000,0.000000,0.000000}%
\pgfsetstrokecolor{currentstroke}%
\pgfsetdash{}{0pt}%
\pgfpathmoveto{\pgfqpoint{4.011666in}{0.684333in}}%
\pgfpathcurveto{\pgfqpoint{4.022716in}{0.684333in}}{\pgfqpoint{4.033315in}{0.688724in}}{\pgfqpoint{4.041128in}{0.696537in}}%
\pgfpathcurveto{\pgfqpoint{4.048942in}{0.704351in}}{\pgfqpoint{4.053332in}{0.714950in}}{\pgfqpoint{4.053332in}{0.726000in}}%
\pgfpathcurveto{\pgfqpoint{4.053332in}{0.737050in}}{\pgfqpoint{4.048942in}{0.747649in}}{\pgfqpoint{4.041128in}{0.755463in}}%
\pgfpathcurveto{\pgfqpoint{4.033315in}{0.763276in}}{\pgfqpoint{4.022716in}{0.767667in}}{\pgfqpoint{4.011666in}{0.767667in}}%
\pgfpathcurveto{\pgfqpoint{4.000616in}{0.767667in}}{\pgfqpoint{3.990016in}{0.763276in}}{\pgfqpoint{3.982203in}{0.755463in}}%
\pgfpathcurveto{\pgfqpoint{3.974389in}{0.747649in}}{\pgfqpoint{3.969999in}{0.737050in}}{\pgfqpoint{3.969999in}{0.726000in}}%
\pgfpathcurveto{\pgfqpoint{3.969999in}{0.714950in}}{\pgfqpoint{3.974389in}{0.704351in}}{\pgfqpoint{3.982203in}{0.696537in}}%
\pgfpathcurveto{\pgfqpoint{3.990016in}{0.688724in}}{\pgfqpoint{4.000616in}{0.684333in}}{\pgfqpoint{4.011666in}{0.684333in}}%
\pgfpathclose%
\pgfusepath{stroke,fill}%
\end{pgfscope}%
\begin{pgfscope}%
\pgfpathrectangle{\pgfqpoint{0.800000in}{0.528000in}}{\pgfqpoint{4.960000in}{3.696000in}}%
\pgfusepath{clip}%
\pgfsetbuttcap%
\pgfsetroundjoin%
\definecolor{currentfill}{rgb}{0.000000,0.000000,0.000000}%
\pgfsetfillcolor{currentfill}%
\pgfsetlinewidth{1.003750pt}%
\definecolor{currentstroke}{rgb}{0.000000,0.000000,0.000000}%
\pgfsetstrokecolor{currentstroke}%
\pgfsetdash{}{0pt}%
\pgfpathmoveto{\pgfqpoint{4.011666in}{3.984333in}}%
\pgfpathcurveto{\pgfqpoint{4.022716in}{3.984333in}}{\pgfqpoint{4.033315in}{3.988724in}}{\pgfqpoint{4.041128in}{3.996537in}}%
\pgfpathcurveto{\pgfqpoint{4.048942in}{4.004351in}}{\pgfqpoint{4.053332in}{4.014950in}}{\pgfqpoint{4.053332in}{4.026000in}}%
\pgfpathcurveto{\pgfqpoint{4.053332in}{4.037050in}}{\pgfqpoint{4.048942in}{4.047649in}}{\pgfqpoint{4.041128in}{4.055463in}}%
\pgfpathcurveto{\pgfqpoint{4.033315in}{4.063276in}}{\pgfqpoint{4.022716in}{4.067667in}}{\pgfqpoint{4.011666in}{4.067667in}}%
\pgfpathcurveto{\pgfqpoint{4.000616in}{4.067667in}}{\pgfqpoint{3.990016in}{4.063276in}}{\pgfqpoint{3.982203in}{4.055463in}}%
\pgfpathcurveto{\pgfqpoint{3.974389in}{4.047649in}}{\pgfqpoint{3.969999in}{4.037050in}}{\pgfqpoint{3.969999in}{4.026000in}}%
\pgfpathcurveto{\pgfqpoint{3.969999in}{4.014950in}}{\pgfqpoint{3.974389in}{4.004351in}}{\pgfqpoint{3.982203in}{3.996537in}}%
\pgfpathcurveto{\pgfqpoint{3.990016in}{3.988724in}}{\pgfqpoint{4.000616in}{3.984333in}}{\pgfqpoint{4.011666in}{3.984333in}}%
\pgfpathclose%
\pgfusepath{stroke,fill}%
\end{pgfscope}%
\begin{pgfscope}%
\pgfpathrectangle{\pgfqpoint{0.800000in}{0.528000in}}{\pgfqpoint{4.960000in}{3.696000in}}%
\pgfusepath{clip}%
\pgfsetbuttcap%
\pgfsetroundjoin%
\definecolor{currentfill}{rgb}{0.000000,0.000000,0.000000}%
\pgfsetfillcolor{currentfill}%
\pgfsetlinewidth{1.003750pt}%
\definecolor{currentstroke}{rgb}{0.000000,0.000000,0.000000}%
\pgfsetstrokecolor{currentstroke}%
\pgfsetdash{}{0pt}%
\pgfpathmoveto{\pgfqpoint{4.011666in}{3.984333in}}%
\pgfpathcurveto{\pgfqpoint{4.022716in}{3.984333in}}{\pgfqpoint{4.033315in}{3.988724in}}{\pgfqpoint{4.041128in}{3.996537in}}%
\pgfpathcurveto{\pgfqpoint{4.048942in}{4.004351in}}{\pgfqpoint{4.053332in}{4.014950in}}{\pgfqpoint{4.053332in}{4.026000in}}%
\pgfpathcurveto{\pgfqpoint{4.053332in}{4.037050in}}{\pgfqpoint{4.048942in}{4.047649in}}{\pgfqpoint{4.041128in}{4.055463in}}%
\pgfpathcurveto{\pgfqpoint{4.033315in}{4.063276in}}{\pgfqpoint{4.022716in}{4.067667in}}{\pgfqpoint{4.011666in}{4.067667in}}%
\pgfpathcurveto{\pgfqpoint{4.000616in}{4.067667in}}{\pgfqpoint{3.990016in}{4.063276in}}{\pgfqpoint{3.982203in}{4.055463in}}%
\pgfpathcurveto{\pgfqpoint{3.974389in}{4.047649in}}{\pgfqpoint{3.969999in}{4.037050in}}{\pgfqpoint{3.969999in}{4.026000in}}%
\pgfpathcurveto{\pgfqpoint{3.969999in}{4.014950in}}{\pgfqpoint{3.974389in}{4.004351in}}{\pgfqpoint{3.982203in}{3.996537in}}%
\pgfpathcurveto{\pgfqpoint{3.990016in}{3.988724in}}{\pgfqpoint{4.000616in}{3.984333in}}{\pgfqpoint{4.011666in}{3.984333in}}%
\pgfpathclose%
\pgfusepath{stroke,fill}%
\end{pgfscope}%
\begin{pgfscope}%
\pgfpathrectangle{\pgfqpoint{0.800000in}{0.528000in}}{\pgfqpoint{4.960000in}{3.696000in}}%
\pgfusepath{clip}%
\pgfsetbuttcap%
\pgfsetroundjoin%
\definecolor{currentfill}{rgb}{0.000000,0.000000,0.000000}%
\pgfsetfillcolor{currentfill}%
\pgfsetlinewidth{1.003750pt}%
\definecolor{currentstroke}{rgb}{0.000000,0.000000,0.000000}%
\pgfsetstrokecolor{currentstroke}%
\pgfsetdash{}{0pt}%
\pgfpathmoveto{\pgfqpoint{4.011666in}{3.984333in}}%
\pgfpathcurveto{\pgfqpoint{4.022716in}{3.984333in}}{\pgfqpoint{4.033315in}{3.988724in}}{\pgfqpoint{4.041128in}{3.996537in}}%
\pgfpathcurveto{\pgfqpoint{4.048942in}{4.004351in}}{\pgfqpoint{4.053332in}{4.014950in}}{\pgfqpoint{4.053332in}{4.026000in}}%
\pgfpathcurveto{\pgfqpoint{4.053332in}{4.037050in}}{\pgfqpoint{4.048942in}{4.047649in}}{\pgfqpoint{4.041128in}{4.055463in}}%
\pgfpathcurveto{\pgfqpoint{4.033315in}{4.063276in}}{\pgfqpoint{4.022716in}{4.067667in}}{\pgfqpoint{4.011666in}{4.067667in}}%
\pgfpathcurveto{\pgfqpoint{4.000616in}{4.067667in}}{\pgfqpoint{3.990016in}{4.063276in}}{\pgfqpoint{3.982203in}{4.055463in}}%
\pgfpathcurveto{\pgfqpoint{3.974389in}{4.047649in}}{\pgfqpoint{3.969999in}{4.037050in}}{\pgfqpoint{3.969999in}{4.026000in}}%
\pgfpathcurveto{\pgfqpoint{3.969999in}{4.014950in}}{\pgfqpoint{3.974389in}{4.004351in}}{\pgfqpoint{3.982203in}{3.996537in}}%
\pgfpathcurveto{\pgfqpoint{3.990016in}{3.988724in}}{\pgfqpoint{4.000616in}{3.984333in}}{\pgfqpoint{4.011666in}{3.984333in}}%
\pgfpathclose%
\pgfusepath{stroke,fill}%
\end{pgfscope}%
\begin{pgfscope}%
\pgfpathrectangle{\pgfqpoint{0.800000in}{0.528000in}}{\pgfqpoint{4.960000in}{3.696000in}}%
\pgfusepath{clip}%
\pgfsetbuttcap%
\pgfsetroundjoin%
\definecolor{currentfill}{rgb}{0.000000,0.000000,0.000000}%
\pgfsetfillcolor{currentfill}%
\pgfsetlinewidth{1.003750pt}%
\definecolor{currentstroke}{rgb}{0.000000,0.000000,0.000000}%
\pgfsetstrokecolor{currentstroke}%
\pgfsetdash{}{0pt}%
\pgfpathmoveto{\pgfqpoint{4.011666in}{3.984333in}}%
\pgfpathcurveto{\pgfqpoint{4.022716in}{3.984333in}}{\pgfqpoint{4.033315in}{3.988724in}}{\pgfqpoint{4.041128in}{3.996537in}}%
\pgfpathcurveto{\pgfqpoint{4.048942in}{4.004351in}}{\pgfqpoint{4.053332in}{4.014950in}}{\pgfqpoint{4.053332in}{4.026000in}}%
\pgfpathcurveto{\pgfqpoint{4.053332in}{4.037050in}}{\pgfqpoint{4.048942in}{4.047649in}}{\pgfqpoint{4.041128in}{4.055463in}}%
\pgfpathcurveto{\pgfqpoint{4.033315in}{4.063276in}}{\pgfqpoint{4.022716in}{4.067667in}}{\pgfqpoint{4.011666in}{4.067667in}}%
\pgfpathcurveto{\pgfqpoint{4.000616in}{4.067667in}}{\pgfqpoint{3.990016in}{4.063276in}}{\pgfqpoint{3.982203in}{4.055463in}}%
\pgfpathcurveto{\pgfqpoint{3.974389in}{4.047649in}}{\pgfqpoint{3.969999in}{4.037050in}}{\pgfqpoint{3.969999in}{4.026000in}}%
\pgfpathcurveto{\pgfqpoint{3.969999in}{4.014950in}}{\pgfqpoint{3.974389in}{4.004351in}}{\pgfqpoint{3.982203in}{3.996537in}}%
\pgfpathcurveto{\pgfqpoint{3.990016in}{3.988724in}}{\pgfqpoint{4.000616in}{3.984333in}}{\pgfqpoint{4.011666in}{3.984333in}}%
\pgfpathclose%
\pgfusepath{stroke,fill}%
\end{pgfscope}%
\begin{pgfscope}%
\pgfpathrectangle{\pgfqpoint{0.800000in}{0.528000in}}{\pgfqpoint{4.960000in}{3.696000in}}%
\pgfusepath{clip}%
\pgfsetbuttcap%
\pgfsetroundjoin%
\definecolor{currentfill}{rgb}{0.000000,0.000000,0.000000}%
\pgfsetfillcolor{currentfill}%
\pgfsetlinewidth{1.003750pt}%
\definecolor{currentstroke}{rgb}{0.000000,0.000000,0.000000}%
\pgfsetstrokecolor{currentstroke}%
\pgfsetdash{}{0pt}%
\pgfpathmoveto{\pgfqpoint{4.011666in}{3.984333in}}%
\pgfpathcurveto{\pgfqpoint{4.022716in}{3.984333in}}{\pgfqpoint{4.033315in}{3.988724in}}{\pgfqpoint{4.041128in}{3.996537in}}%
\pgfpathcurveto{\pgfqpoint{4.048942in}{4.004351in}}{\pgfqpoint{4.053332in}{4.014950in}}{\pgfqpoint{4.053332in}{4.026000in}}%
\pgfpathcurveto{\pgfqpoint{4.053332in}{4.037050in}}{\pgfqpoint{4.048942in}{4.047649in}}{\pgfqpoint{4.041128in}{4.055463in}}%
\pgfpathcurveto{\pgfqpoint{4.033315in}{4.063276in}}{\pgfqpoint{4.022716in}{4.067667in}}{\pgfqpoint{4.011666in}{4.067667in}}%
\pgfpathcurveto{\pgfqpoint{4.000616in}{4.067667in}}{\pgfqpoint{3.990016in}{4.063276in}}{\pgfqpoint{3.982203in}{4.055463in}}%
\pgfpathcurveto{\pgfqpoint{3.974389in}{4.047649in}}{\pgfqpoint{3.969999in}{4.037050in}}{\pgfqpoint{3.969999in}{4.026000in}}%
\pgfpathcurveto{\pgfqpoint{3.969999in}{4.014950in}}{\pgfqpoint{3.974389in}{4.004351in}}{\pgfqpoint{3.982203in}{3.996537in}}%
\pgfpathcurveto{\pgfqpoint{3.990016in}{3.988724in}}{\pgfqpoint{4.000616in}{3.984333in}}{\pgfqpoint{4.011666in}{3.984333in}}%
\pgfpathclose%
\pgfusepath{stroke,fill}%
\end{pgfscope}%
\begin{pgfscope}%
\pgfpathrectangle{\pgfqpoint{0.800000in}{0.528000in}}{\pgfqpoint{4.960000in}{3.696000in}}%
\pgfusepath{clip}%
\pgfsetbuttcap%
\pgfsetroundjoin%
\definecolor{currentfill}{rgb}{0.000000,0.000000,0.000000}%
\pgfsetfillcolor{currentfill}%
\pgfsetlinewidth{1.003750pt}%
\definecolor{currentstroke}{rgb}{0.000000,0.000000,0.000000}%
\pgfsetstrokecolor{currentstroke}%
\pgfsetdash{}{0pt}%
\pgfpathmoveto{\pgfqpoint{4.011666in}{3.984333in}}%
\pgfpathcurveto{\pgfqpoint{4.022716in}{3.984333in}}{\pgfqpoint{4.033315in}{3.988724in}}{\pgfqpoint{4.041128in}{3.996537in}}%
\pgfpathcurveto{\pgfqpoint{4.048942in}{4.004351in}}{\pgfqpoint{4.053332in}{4.014950in}}{\pgfqpoint{4.053332in}{4.026000in}}%
\pgfpathcurveto{\pgfqpoint{4.053332in}{4.037050in}}{\pgfqpoint{4.048942in}{4.047649in}}{\pgfqpoint{4.041128in}{4.055463in}}%
\pgfpathcurveto{\pgfqpoint{4.033315in}{4.063276in}}{\pgfqpoint{4.022716in}{4.067667in}}{\pgfqpoint{4.011666in}{4.067667in}}%
\pgfpathcurveto{\pgfqpoint{4.000616in}{4.067667in}}{\pgfqpoint{3.990016in}{4.063276in}}{\pgfqpoint{3.982203in}{4.055463in}}%
\pgfpathcurveto{\pgfqpoint{3.974389in}{4.047649in}}{\pgfqpoint{3.969999in}{4.037050in}}{\pgfqpoint{3.969999in}{4.026000in}}%
\pgfpathcurveto{\pgfqpoint{3.969999in}{4.014950in}}{\pgfqpoint{3.974389in}{4.004351in}}{\pgfqpoint{3.982203in}{3.996537in}}%
\pgfpathcurveto{\pgfqpoint{3.990016in}{3.988724in}}{\pgfqpoint{4.000616in}{3.984333in}}{\pgfqpoint{4.011666in}{3.984333in}}%
\pgfpathclose%
\pgfusepath{stroke,fill}%
\end{pgfscope}%
\begin{pgfscope}%
\pgfpathrectangle{\pgfqpoint{0.800000in}{0.528000in}}{\pgfqpoint{4.960000in}{3.696000in}}%
\pgfusepath{clip}%
\pgfsetbuttcap%
\pgfsetroundjoin%
\definecolor{currentfill}{rgb}{0.000000,0.000000,0.000000}%
\pgfsetfillcolor{currentfill}%
\pgfsetlinewidth{1.003750pt}%
\definecolor{currentstroke}{rgb}{0.000000,0.000000,0.000000}%
\pgfsetstrokecolor{currentstroke}%
\pgfsetdash{}{0pt}%
\pgfpathmoveto{\pgfqpoint{4.011666in}{3.984333in}}%
\pgfpathcurveto{\pgfqpoint{4.022716in}{3.984333in}}{\pgfqpoint{4.033315in}{3.988724in}}{\pgfqpoint{4.041128in}{3.996537in}}%
\pgfpathcurveto{\pgfqpoint{4.048942in}{4.004351in}}{\pgfqpoint{4.053332in}{4.014950in}}{\pgfqpoint{4.053332in}{4.026000in}}%
\pgfpathcurveto{\pgfqpoint{4.053332in}{4.037050in}}{\pgfqpoint{4.048942in}{4.047649in}}{\pgfqpoint{4.041128in}{4.055463in}}%
\pgfpathcurveto{\pgfqpoint{4.033315in}{4.063276in}}{\pgfqpoint{4.022716in}{4.067667in}}{\pgfqpoint{4.011666in}{4.067667in}}%
\pgfpathcurveto{\pgfqpoint{4.000616in}{4.067667in}}{\pgfqpoint{3.990016in}{4.063276in}}{\pgfqpoint{3.982203in}{4.055463in}}%
\pgfpathcurveto{\pgfqpoint{3.974389in}{4.047649in}}{\pgfqpoint{3.969999in}{4.037050in}}{\pgfqpoint{3.969999in}{4.026000in}}%
\pgfpathcurveto{\pgfqpoint{3.969999in}{4.014950in}}{\pgfqpoint{3.974389in}{4.004351in}}{\pgfqpoint{3.982203in}{3.996537in}}%
\pgfpathcurveto{\pgfqpoint{3.990016in}{3.988724in}}{\pgfqpoint{4.000616in}{3.984333in}}{\pgfqpoint{4.011666in}{3.984333in}}%
\pgfpathclose%
\pgfusepath{stroke,fill}%
\end{pgfscope}%
\begin{pgfscope}%
\pgfpathrectangle{\pgfqpoint{0.800000in}{0.528000in}}{\pgfqpoint{4.960000in}{3.696000in}}%
\pgfusepath{clip}%
\pgfsetbuttcap%
\pgfsetroundjoin%
\definecolor{currentfill}{rgb}{0.000000,0.000000,0.000000}%
\pgfsetfillcolor{currentfill}%
\pgfsetlinewidth{1.003750pt}%
\definecolor{currentstroke}{rgb}{0.000000,0.000000,0.000000}%
\pgfsetstrokecolor{currentstroke}%
\pgfsetdash{}{0pt}%
\pgfpathmoveto{\pgfqpoint{4.011666in}{3.984333in}}%
\pgfpathcurveto{\pgfqpoint{4.022716in}{3.984333in}}{\pgfqpoint{4.033315in}{3.988724in}}{\pgfqpoint{4.041128in}{3.996537in}}%
\pgfpathcurveto{\pgfqpoint{4.048942in}{4.004351in}}{\pgfqpoint{4.053332in}{4.014950in}}{\pgfqpoint{4.053332in}{4.026000in}}%
\pgfpathcurveto{\pgfqpoint{4.053332in}{4.037050in}}{\pgfqpoint{4.048942in}{4.047649in}}{\pgfqpoint{4.041128in}{4.055463in}}%
\pgfpathcurveto{\pgfqpoint{4.033315in}{4.063276in}}{\pgfqpoint{4.022716in}{4.067667in}}{\pgfqpoint{4.011666in}{4.067667in}}%
\pgfpathcurveto{\pgfqpoint{4.000616in}{4.067667in}}{\pgfqpoint{3.990016in}{4.063276in}}{\pgfqpoint{3.982203in}{4.055463in}}%
\pgfpathcurveto{\pgfqpoint{3.974389in}{4.047649in}}{\pgfqpoint{3.969999in}{4.037050in}}{\pgfqpoint{3.969999in}{4.026000in}}%
\pgfpathcurveto{\pgfqpoint{3.969999in}{4.014950in}}{\pgfqpoint{3.974389in}{4.004351in}}{\pgfqpoint{3.982203in}{3.996537in}}%
\pgfpathcurveto{\pgfqpoint{3.990016in}{3.988724in}}{\pgfqpoint{4.000616in}{3.984333in}}{\pgfqpoint{4.011666in}{3.984333in}}%
\pgfpathclose%
\pgfusepath{stroke,fill}%
\end{pgfscope}%
\begin{pgfscope}%
\pgfpathrectangle{\pgfqpoint{0.800000in}{0.528000in}}{\pgfqpoint{4.960000in}{3.696000in}}%
\pgfusepath{clip}%
\pgfsetbuttcap%
\pgfsetroundjoin%
\definecolor{currentfill}{rgb}{0.000000,0.000000,0.000000}%
\pgfsetfillcolor{currentfill}%
\pgfsetlinewidth{1.003750pt}%
\definecolor{currentstroke}{rgb}{0.000000,0.000000,0.000000}%
\pgfsetstrokecolor{currentstroke}%
\pgfsetdash{}{0pt}%
\pgfpathmoveto{\pgfqpoint{4.011666in}{3.984333in}}%
\pgfpathcurveto{\pgfqpoint{4.022716in}{3.984333in}}{\pgfqpoint{4.033315in}{3.988724in}}{\pgfqpoint{4.041128in}{3.996537in}}%
\pgfpathcurveto{\pgfqpoint{4.048942in}{4.004351in}}{\pgfqpoint{4.053332in}{4.014950in}}{\pgfqpoint{4.053332in}{4.026000in}}%
\pgfpathcurveto{\pgfqpoint{4.053332in}{4.037050in}}{\pgfqpoint{4.048942in}{4.047649in}}{\pgfqpoint{4.041128in}{4.055463in}}%
\pgfpathcurveto{\pgfqpoint{4.033315in}{4.063276in}}{\pgfqpoint{4.022716in}{4.067667in}}{\pgfqpoint{4.011666in}{4.067667in}}%
\pgfpathcurveto{\pgfqpoint{4.000616in}{4.067667in}}{\pgfqpoint{3.990016in}{4.063276in}}{\pgfqpoint{3.982203in}{4.055463in}}%
\pgfpathcurveto{\pgfqpoint{3.974389in}{4.047649in}}{\pgfqpoint{3.969999in}{4.037050in}}{\pgfqpoint{3.969999in}{4.026000in}}%
\pgfpathcurveto{\pgfqpoint{3.969999in}{4.014950in}}{\pgfqpoint{3.974389in}{4.004351in}}{\pgfqpoint{3.982203in}{3.996537in}}%
\pgfpathcurveto{\pgfqpoint{3.990016in}{3.988724in}}{\pgfqpoint{4.000616in}{3.984333in}}{\pgfqpoint{4.011666in}{3.984333in}}%
\pgfpathclose%
\pgfusepath{stroke,fill}%
\end{pgfscope}%
\begin{pgfscope}%
\pgfpathrectangle{\pgfqpoint{0.800000in}{0.528000in}}{\pgfqpoint{4.960000in}{3.696000in}}%
\pgfusepath{clip}%
\pgfsetbuttcap%
\pgfsetroundjoin%
\definecolor{currentfill}{rgb}{0.000000,0.000000,0.000000}%
\pgfsetfillcolor{currentfill}%
\pgfsetlinewidth{1.003750pt}%
\definecolor{currentstroke}{rgb}{0.000000,0.000000,0.000000}%
\pgfsetstrokecolor{currentstroke}%
\pgfsetdash{}{0pt}%
\pgfpathmoveto{\pgfqpoint{4.011666in}{3.984333in}}%
\pgfpathcurveto{\pgfqpoint{4.022716in}{3.984333in}}{\pgfqpoint{4.033315in}{3.988724in}}{\pgfqpoint{4.041128in}{3.996537in}}%
\pgfpathcurveto{\pgfqpoint{4.048942in}{4.004351in}}{\pgfqpoint{4.053332in}{4.014950in}}{\pgfqpoint{4.053332in}{4.026000in}}%
\pgfpathcurveto{\pgfqpoint{4.053332in}{4.037050in}}{\pgfqpoint{4.048942in}{4.047649in}}{\pgfqpoint{4.041128in}{4.055463in}}%
\pgfpathcurveto{\pgfqpoint{4.033315in}{4.063276in}}{\pgfqpoint{4.022716in}{4.067667in}}{\pgfqpoint{4.011666in}{4.067667in}}%
\pgfpathcurveto{\pgfqpoint{4.000616in}{4.067667in}}{\pgfqpoint{3.990016in}{4.063276in}}{\pgfqpoint{3.982203in}{4.055463in}}%
\pgfpathcurveto{\pgfqpoint{3.974389in}{4.047649in}}{\pgfqpoint{3.969999in}{4.037050in}}{\pgfqpoint{3.969999in}{4.026000in}}%
\pgfpathcurveto{\pgfqpoint{3.969999in}{4.014950in}}{\pgfqpoint{3.974389in}{4.004351in}}{\pgfqpoint{3.982203in}{3.996537in}}%
\pgfpathcurveto{\pgfqpoint{3.990016in}{3.988724in}}{\pgfqpoint{4.000616in}{3.984333in}}{\pgfqpoint{4.011666in}{3.984333in}}%
\pgfpathclose%
\pgfusepath{stroke,fill}%
\end{pgfscope}%
\begin{pgfscope}%
\pgfpathrectangle{\pgfqpoint{0.800000in}{0.528000in}}{\pgfqpoint{4.960000in}{3.696000in}}%
\pgfusepath{clip}%
\pgfsetbuttcap%
\pgfsetroundjoin%
\definecolor{currentfill}{rgb}{0.000000,0.000000,0.000000}%
\pgfsetfillcolor{currentfill}%
\pgfsetlinewidth{1.003750pt}%
\definecolor{currentstroke}{rgb}{0.000000,0.000000,0.000000}%
\pgfsetstrokecolor{currentstroke}%
\pgfsetdash{}{0pt}%
\pgfpathmoveto{\pgfqpoint{4.011666in}{3.984333in}}%
\pgfpathcurveto{\pgfqpoint{4.022716in}{3.984333in}}{\pgfqpoint{4.033315in}{3.988724in}}{\pgfqpoint{4.041128in}{3.996537in}}%
\pgfpathcurveto{\pgfqpoint{4.048942in}{4.004351in}}{\pgfqpoint{4.053332in}{4.014950in}}{\pgfqpoint{4.053332in}{4.026000in}}%
\pgfpathcurveto{\pgfqpoint{4.053332in}{4.037050in}}{\pgfqpoint{4.048942in}{4.047649in}}{\pgfqpoint{4.041128in}{4.055463in}}%
\pgfpathcurveto{\pgfqpoint{4.033315in}{4.063276in}}{\pgfqpoint{4.022716in}{4.067667in}}{\pgfqpoint{4.011666in}{4.067667in}}%
\pgfpathcurveto{\pgfqpoint{4.000616in}{4.067667in}}{\pgfqpoint{3.990016in}{4.063276in}}{\pgfqpoint{3.982203in}{4.055463in}}%
\pgfpathcurveto{\pgfqpoint{3.974389in}{4.047649in}}{\pgfqpoint{3.969999in}{4.037050in}}{\pgfqpoint{3.969999in}{4.026000in}}%
\pgfpathcurveto{\pgfqpoint{3.969999in}{4.014950in}}{\pgfqpoint{3.974389in}{4.004351in}}{\pgfqpoint{3.982203in}{3.996537in}}%
\pgfpathcurveto{\pgfqpoint{3.990016in}{3.988724in}}{\pgfqpoint{4.000616in}{3.984333in}}{\pgfqpoint{4.011666in}{3.984333in}}%
\pgfpathclose%
\pgfusepath{stroke,fill}%
\end{pgfscope}%
\begin{pgfscope}%
\pgfpathrectangle{\pgfqpoint{0.800000in}{0.528000in}}{\pgfqpoint{4.960000in}{3.696000in}}%
\pgfusepath{clip}%
\pgfsetbuttcap%
\pgfsetroundjoin%
\definecolor{currentfill}{rgb}{0.000000,0.000000,0.000000}%
\pgfsetfillcolor{currentfill}%
\pgfsetlinewidth{1.003750pt}%
\definecolor{currentstroke}{rgb}{0.000000,0.000000,0.000000}%
\pgfsetstrokecolor{currentstroke}%
\pgfsetdash{}{0pt}%
\pgfpathmoveto{\pgfqpoint{4.011666in}{3.984333in}}%
\pgfpathcurveto{\pgfqpoint{4.022716in}{3.984333in}}{\pgfqpoint{4.033315in}{3.988724in}}{\pgfqpoint{4.041128in}{3.996537in}}%
\pgfpathcurveto{\pgfqpoint{4.048942in}{4.004351in}}{\pgfqpoint{4.053332in}{4.014950in}}{\pgfqpoint{4.053332in}{4.026000in}}%
\pgfpathcurveto{\pgfqpoint{4.053332in}{4.037050in}}{\pgfqpoint{4.048942in}{4.047649in}}{\pgfqpoint{4.041128in}{4.055463in}}%
\pgfpathcurveto{\pgfqpoint{4.033315in}{4.063276in}}{\pgfqpoint{4.022716in}{4.067667in}}{\pgfqpoint{4.011666in}{4.067667in}}%
\pgfpathcurveto{\pgfqpoint{4.000616in}{4.067667in}}{\pgfqpoint{3.990016in}{4.063276in}}{\pgfqpoint{3.982203in}{4.055463in}}%
\pgfpathcurveto{\pgfqpoint{3.974389in}{4.047649in}}{\pgfqpoint{3.969999in}{4.037050in}}{\pgfqpoint{3.969999in}{4.026000in}}%
\pgfpathcurveto{\pgfqpoint{3.969999in}{4.014950in}}{\pgfqpoint{3.974389in}{4.004351in}}{\pgfqpoint{3.982203in}{3.996537in}}%
\pgfpathcurveto{\pgfqpoint{3.990016in}{3.988724in}}{\pgfqpoint{4.000616in}{3.984333in}}{\pgfqpoint{4.011666in}{3.984333in}}%
\pgfpathclose%
\pgfusepath{stroke,fill}%
\end{pgfscope}%
\begin{pgfscope}%
\pgfpathrectangle{\pgfqpoint{0.800000in}{0.528000in}}{\pgfqpoint{4.960000in}{3.696000in}}%
\pgfusepath{clip}%
\pgfsetbuttcap%
\pgfsetroundjoin%
\definecolor{currentfill}{rgb}{0.000000,0.000000,0.000000}%
\pgfsetfillcolor{currentfill}%
\pgfsetlinewidth{1.003750pt}%
\definecolor{currentstroke}{rgb}{0.000000,0.000000,0.000000}%
\pgfsetstrokecolor{currentstroke}%
\pgfsetdash{}{0pt}%
\pgfpathmoveto{\pgfqpoint{4.011666in}{3.984333in}}%
\pgfpathcurveto{\pgfqpoint{4.022716in}{3.984333in}}{\pgfqpoint{4.033315in}{3.988724in}}{\pgfqpoint{4.041128in}{3.996537in}}%
\pgfpathcurveto{\pgfqpoint{4.048942in}{4.004351in}}{\pgfqpoint{4.053332in}{4.014950in}}{\pgfqpoint{4.053332in}{4.026000in}}%
\pgfpathcurveto{\pgfqpoint{4.053332in}{4.037050in}}{\pgfqpoint{4.048942in}{4.047649in}}{\pgfqpoint{4.041128in}{4.055463in}}%
\pgfpathcurveto{\pgfqpoint{4.033315in}{4.063276in}}{\pgfqpoint{4.022716in}{4.067667in}}{\pgfqpoint{4.011666in}{4.067667in}}%
\pgfpathcurveto{\pgfqpoint{4.000616in}{4.067667in}}{\pgfqpoint{3.990016in}{4.063276in}}{\pgfqpoint{3.982203in}{4.055463in}}%
\pgfpathcurveto{\pgfqpoint{3.974389in}{4.047649in}}{\pgfqpoint{3.969999in}{4.037050in}}{\pgfqpoint{3.969999in}{4.026000in}}%
\pgfpathcurveto{\pgfqpoint{3.969999in}{4.014950in}}{\pgfqpoint{3.974389in}{4.004351in}}{\pgfqpoint{3.982203in}{3.996537in}}%
\pgfpathcurveto{\pgfqpoint{3.990016in}{3.988724in}}{\pgfqpoint{4.000616in}{3.984333in}}{\pgfqpoint{4.011666in}{3.984333in}}%
\pgfpathclose%
\pgfusepath{stroke,fill}%
\end{pgfscope}%
\begin{pgfscope}%
\pgfpathrectangle{\pgfqpoint{0.800000in}{0.528000in}}{\pgfqpoint{4.960000in}{3.696000in}}%
\pgfusepath{clip}%
\pgfsetbuttcap%
\pgfsetroundjoin%
\definecolor{currentfill}{rgb}{0.000000,0.000000,0.000000}%
\pgfsetfillcolor{currentfill}%
\pgfsetlinewidth{1.003750pt}%
\definecolor{currentstroke}{rgb}{0.000000,0.000000,0.000000}%
\pgfsetstrokecolor{currentstroke}%
\pgfsetdash{}{0pt}%
\pgfpathmoveto{\pgfqpoint{4.011666in}{3.984333in}}%
\pgfpathcurveto{\pgfqpoint{4.022716in}{3.984333in}}{\pgfqpoint{4.033315in}{3.988724in}}{\pgfqpoint{4.041128in}{3.996537in}}%
\pgfpathcurveto{\pgfqpoint{4.048942in}{4.004351in}}{\pgfqpoint{4.053332in}{4.014950in}}{\pgfqpoint{4.053332in}{4.026000in}}%
\pgfpathcurveto{\pgfqpoint{4.053332in}{4.037050in}}{\pgfqpoint{4.048942in}{4.047649in}}{\pgfqpoint{4.041128in}{4.055463in}}%
\pgfpathcurveto{\pgfqpoint{4.033315in}{4.063276in}}{\pgfqpoint{4.022716in}{4.067667in}}{\pgfqpoint{4.011666in}{4.067667in}}%
\pgfpathcurveto{\pgfqpoint{4.000616in}{4.067667in}}{\pgfqpoint{3.990016in}{4.063276in}}{\pgfqpoint{3.982203in}{4.055463in}}%
\pgfpathcurveto{\pgfqpoint{3.974389in}{4.047649in}}{\pgfqpoint{3.969999in}{4.037050in}}{\pgfqpoint{3.969999in}{4.026000in}}%
\pgfpathcurveto{\pgfqpoint{3.969999in}{4.014950in}}{\pgfqpoint{3.974389in}{4.004351in}}{\pgfqpoint{3.982203in}{3.996537in}}%
\pgfpathcurveto{\pgfqpoint{3.990016in}{3.988724in}}{\pgfqpoint{4.000616in}{3.984333in}}{\pgfqpoint{4.011666in}{3.984333in}}%
\pgfpathclose%
\pgfusepath{stroke,fill}%
\end{pgfscope}%
\begin{pgfscope}%
\pgfpathrectangle{\pgfqpoint{0.800000in}{0.528000in}}{\pgfqpoint{4.960000in}{3.696000in}}%
\pgfusepath{clip}%
\pgfsetbuttcap%
\pgfsetroundjoin%
\definecolor{currentfill}{rgb}{0.000000,0.000000,0.000000}%
\pgfsetfillcolor{currentfill}%
\pgfsetlinewidth{1.003750pt}%
\definecolor{currentstroke}{rgb}{0.000000,0.000000,0.000000}%
\pgfsetstrokecolor{currentstroke}%
\pgfsetdash{}{0pt}%
\pgfpathmoveto{\pgfqpoint{4.011666in}{3.984333in}}%
\pgfpathcurveto{\pgfqpoint{4.022716in}{3.984333in}}{\pgfqpoint{4.033315in}{3.988724in}}{\pgfqpoint{4.041128in}{3.996537in}}%
\pgfpathcurveto{\pgfqpoint{4.048942in}{4.004351in}}{\pgfqpoint{4.053332in}{4.014950in}}{\pgfqpoint{4.053332in}{4.026000in}}%
\pgfpathcurveto{\pgfqpoint{4.053332in}{4.037050in}}{\pgfqpoint{4.048942in}{4.047649in}}{\pgfqpoint{4.041128in}{4.055463in}}%
\pgfpathcurveto{\pgfqpoint{4.033315in}{4.063276in}}{\pgfqpoint{4.022716in}{4.067667in}}{\pgfqpoint{4.011666in}{4.067667in}}%
\pgfpathcurveto{\pgfqpoint{4.000616in}{4.067667in}}{\pgfqpoint{3.990016in}{4.063276in}}{\pgfqpoint{3.982203in}{4.055463in}}%
\pgfpathcurveto{\pgfqpoint{3.974389in}{4.047649in}}{\pgfqpoint{3.969999in}{4.037050in}}{\pgfqpoint{3.969999in}{4.026000in}}%
\pgfpathcurveto{\pgfqpoint{3.969999in}{4.014950in}}{\pgfqpoint{3.974389in}{4.004351in}}{\pgfqpoint{3.982203in}{3.996537in}}%
\pgfpathcurveto{\pgfqpoint{3.990016in}{3.988724in}}{\pgfqpoint{4.000616in}{3.984333in}}{\pgfqpoint{4.011666in}{3.984333in}}%
\pgfpathclose%
\pgfusepath{stroke,fill}%
\end{pgfscope}%
\begin{pgfscope}%
\pgfpathrectangle{\pgfqpoint{0.800000in}{0.528000in}}{\pgfqpoint{4.960000in}{3.696000in}}%
\pgfusepath{clip}%
\pgfsetbuttcap%
\pgfsetroundjoin%
\definecolor{currentfill}{rgb}{0.000000,0.000000,0.000000}%
\pgfsetfillcolor{currentfill}%
\pgfsetlinewidth{1.003750pt}%
\definecolor{currentstroke}{rgb}{0.000000,0.000000,0.000000}%
\pgfsetstrokecolor{currentstroke}%
\pgfsetdash{}{0pt}%
\pgfpathmoveto{\pgfqpoint{4.011666in}{3.984333in}}%
\pgfpathcurveto{\pgfqpoint{4.022716in}{3.984333in}}{\pgfqpoint{4.033315in}{3.988724in}}{\pgfqpoint{4.041128in}{3.996537in}}%
\pgfpathcurveto{\pgfqpoint{4.048942in}{4.004351in}}{\pgfqpoint{4.053332in}{4.014950in}}{\pgfqpoint{4.053332in}{4.026000in}}%
\pgfpathcurveto{\pgfqpoint{4.053332in}{4.037050in}}{\pgfqpoint{4.048942in}{4.047649in}}{\pgfqpoint{4.041128in}{4.055463in}}%
\pgfpathcurveto{\pgfqpoint{4.033315in}{4.063276in}}{\pgfqpoint{4.022716in}{4.067667in}}{\pgfqpoint{4.011666in}{4.067667in}}%
\pgfpathcurveto{\pgfqpoint{4.000616in}{4.067667in}}{\pgfqpoint{3.990016in}{4.063276in}}{\pgfqpoint{3.982203in}{4.055463in}}%
\pgfpathcurveto{\pgfqpoint{3.974389in}{4.047649in}}{\pgfqpoint{3.969999in}{4.037050in}}{\pgfqpoint{3.969999in}{4.026000in}}%
\pgfpathcurveto{\pgfqpoint{3.969999in}{4.014950in}}{\pgfqpoint{3.974389in}{4.004351in}}{\pgfqpoint{3.982203in}{3.996537in}}%
\pgfpathcurveto{\pgfqpoint{3.990016in}{3.988724in}}{\pgfqpoint{4.000616in}{3.984333in}}{\pgfqpoint{4.011666in}{3.984333in}}%
\pgfpathclose%
\pgfusepath{stroke,fill}%
\end{pgfscope}%
\begin{pgfscope}%
\pgfpathrectangle{\pgfqpoint{0.800000in}{0.528000in}}{\pgfqpoint{4.960000in}{3.696000in}}%
\pgfusepath{clip}%
\pgfsetbuttcap%
\pgfsetroundjoin%
\definecolor{currentfill}{rgb}{0.000000,0.000000,0.000000}%
\pgfsetfillcolor{currentfill}%
\pgfsetlinewidth{1.003750pt}%
\definecolor{currentstroke}{rgb}{0.000000,0.000000,0.000000}%
\pgfsetstrokecolor{currentstroke}%
\pgfsetdash{}{0pt}%
\pgfpathmoveto{\pgfqpoint{4.011666in}{3.984333in}}%
\pgfpathcurveto{\pgfqpoint{4.022716in}{3.984333in}}{\pgfqpoint{4.033315in}{3.988724in}}{\pgfqpoint{4.041128in}{3.996537in}}%
\pgfpathcurveto{\pgfqpoint{4.048942in}{4.004351in}}{\pgfqpoint{4.053332in}{4.014950in}}{\pgfqpoint{4.053332in}{4.026000in}}%
\pgfpathcurveto{\pgfqpoint{4.053332in}{4.037050in}}{\pgfqpoint{4.048942in}{4.047649in}}{\pgfqpoint{4.041128in}{4.055463in}}%
\pgfpathcurveto{\pgfqpoint{4.033315in}{4.063276in}}{\pgfqpoint{4.022716in}{4.067667in}}{\pgfqpoint{4.011666in}{4.067667in}}%
\pgfpathcurveto{\pgfqpoint{4.000616in}{4.067667in}}{\pgfqpoint{3.990016in}{4.063276in}}{\pgfqpoint{3.982203in}{4.055463in}}%
\pgfpathcurveto{\pgfqpoint{3.974389in}{4.047649in}}{\pgfqpoint{3.969999in}{4.037050in}}{\pgfqpoint{3.969999in}{4.026000in}}%
\pgfpathcurveto{\pgfqpoint{3.969999in}{4.014950in}}{\pgfqpoint{3.974389in}{4.004351in}}{\pgfqpoint{3.982203in}{3.996537in}}%
\pgfpathcurveto{\pgfqpoint{3.990016in}{3.988724in}}{\pgfqpoint{4.000616in}{3.984333in}}{\pgfqpoint{4.011666in}{3.984333in}}%
\pgfpathclose%
\pgfusepath{stroke,fill}%
\end{pgfscope}%
\begin{pgfscope}%
\pgfpathrectangle{\pgfqpoint{0.800000in}{0.528000in}}{\pgfqpoint{4.960000in}{3.696000in}}%
\pgfusepath{clip}%
\pgfsetbuttcap%
\pgfsetroundjoin%
\definecolor{currentfill}{rgb}{0.000000,0.000000,0.000000}%
\pgfsetfillcolor{currentfill}%
\pgfsetlinewidth{1.003750pt}%
\definecolor{currentstroke}{rgb}{0.000000,0.000000,0.000000}%
\pgfsetstrokecolor{currentstroke}%
\pgfsetdash{}{0pt}%
\pgfpathmoveto{\pgfqpoint{4.011666in}{3.984333in}}%
\pgfpathcurveto{\pgfqpoint{4.022716in}{3.984333in}}{\pgfqpoint{4.033315in}{3.988724in}}{\pgfqpoint{4.041128in}{3.996537in}}%
\pgfpathcurveto{\pgfqpoint{4.048942in}{4.004351in}}{\pgfqpoint{4.053332in}{4.014950in}}{\pgfqpoint{4.053332in}{4.026000in}}%
\pgfpathcurveto{\pgfqpoint{4.053332in}{4.037050in}}{\pgfqpoint{4.048942in}{4.047649in}}{\pgfqpoint{4.041128in}{4.055463in}}%
\pgfpathcurveto{\pgfqpoint{4.033315in}{4.063276in}}{\pgfqpoint{4.022716in}{4.067667in}}{\pgfqpoint{4.011666in}{4.067667in}}%
\pgfpathcurveto{\pgfqpoint{4.000616in}{4.067667in}}{\pgfqpoint{3.990016in}{4.063276in}}{\pgfqpoint{3.982203in}{4.055463in}}%
\pgfpathcurveto{\pgfqpoint{3.974389in}{4.047649in}}{\pgfqpoint{3.969999in}{4.037050in}}{\pgfqpoint{3.969999in}{4.026000in}}%
\pgfpathcurveto{\pgfqpoint{3.969999in}{4.014950in}}{\pgfqpoint{3.974389in}{4.004351in}}{\pgfqpoint{3.982203in}{3.996537in}}%
\pgfpathcurveto{\pgfqpoint{3.990016in}{3.988724in}}{\pgfqpoint{4.000616in}{3.984333in}}{\pgfqpoint{4.011666in}{3.984333in}}%
\pgfpathclose%
\pgfusepath{stroke,fill}%
\end{pgfscope}%
\begin{pgfscope}%
\pgfpathrectangle{\pgfqpoint{0.800000in}{0.528000in}}{\pgfqpoint{4.960000in}{3.696000in}}%
\pgfusepath{clip}%
\pgfsetbuttcap%
\pgfsetroundjoin%
\definecolor{currentfill}{rgb}{0.000000,0.000000,0.000000}%
\pgfsetfillcolor{currentfill}%
\pgfsetlinewidth{1.003750pt}%
\definecolor{currentstroke}{rgb}{0.000000,0.000000,0.000000}%
\pgfsetstrokecolor{currentstroke}%
\pgfsetdash{}{0pt}%
\pgfpathmoveto{\pgfqpoint{4.011666in}{3.984333in}}%
\pgfpathcurveto{\pgfqpoint{4.022716in}{3.984333in}}{\pgfqpoint{4.033315in}{3.988724in}}{\pgfqpoint{4.041128in}{3.996537in}}%
\pgfpathcurveto{\pgfqpoint{4.048942in}{4.004351in}}{\pgfqpoint{4.053332in}{4.014950in}}{\pgfqpoint{4.053332in}{4.026000in}}%
\pgfpathcurveto{\pgfqpoint{4.053332in}{4.037050in}}{\pgfqpoint{4.048942in}{4.047649in}}{\pgfqpoint{4.041128in}{4.055463in}}%
\pgfpathcurveto{\pgfqpoint{4.033315in}{4.063276in}}{\pgfqpoint{4.022716in}{4.067667in}}{\pgfqpoint{4.011666in}{4.067667in}}%
\pgfpathcurveto{\pgfqpoint{4.000616in}{4.067667in}}{\pgfqpoint{3.990016in}{4.063276in}}{\pgfqpoint{3.982203in}{4.055463in}}%
\pgfpathcurveto{\pgfqpoint{3.974389in}{4.047649in}}{\pgfqpoint{3.969999in}{4.037050in}}{\pgfqpoint{3.969999in}{4.026000in}}%
\pgfpathcurveto{\pgfqpoint{3.969999in}{4.014950in}}{\pgfqpoint{3.974389in}{4.004351in}}{\pgfqpoint{3.982203in}{3.996537in}}%
\pgfpathcurveto{\pgfqpoint{3.990016in}{3.988724in}}{\pgfqpoint{4.000616in}{3.984333in}}{\pgfqpoint{4.011666in}{3.984333in}}%
\pgfpathclose%
\pgfusepath{stroke,fill}%
\end{pgfscope}%
\begin{pgfscope}%
\pgfpathrectangle{\pgfqpoint{0.800000in}{0.528000in}}{\pgfqpoint{4.960000in}{3.696000in}}%
\pgfusepath{clip}%
\pgfsetbuttcap%
\pgfsetroundjoin%
\definecolor{currentfill}{rgb}{0.000000,0.000000,0.000000}%
\pgfsetfillcolor{currentfill}%
\pgfsetlinewidth{1.003750pt}%
\definecolor{currentstroke}{rgb}{0.000000,0.000000,0.000000}%
\pgfsetstrokecolor{currentstroke}%
\pgfsetdash{}{0pt}%
\pgfpathmoveto{\pgfqpoint{4.011666in}{3.984333in}}%
\pgfpathcurveto{\pgfqpoint{4.022716in}{3.984333in}}{\pgfqpoint{4.033315in}{3.988724in}}{\pgfqpoint{4.041128in}{3.996537in}}%
\pgfpathcurveto{\pgfqpoint{4.048942in}{4.004351in}}{\pgfqpoint{4.053332in}{4.014950in}}{\pgfqpoint{4.053332in}{4.026000in}}%
\pgfpathcurveto{\pgfqpoint{4.053332in}{4.037050in}}{\pgfqpoint{4.048942in}{4.047649in}}{\pgfqpoint{4.041128in}{4.055463in}}%
\pgfpathcurveto{\pgfqpoint{4.033315in}{4.063276in}}{\pgfqpoint{4.022716in}{4.067667in}}{\pgfqpoint{4.011666in}{4.067667in}}%
\pgfpathcurveto{\pgfqpoint{4.000616in}{4.067667in}}{\pgfqpoint{3.990016in}{4.063276in}}{\pgfqpoint{3.982203in}{4.055463in}}%
\pgfpathcurveto{\pgfqpoint{3.974389in}{4.047649in}}{\pgfqpoint{3.969999in}{4.037050in}}{\pgfqpoint{3.969999in}{4.026000in}}%
\pgfpathcurveto{\pgfqpoint{3.969999in}{4.014950in}}{\pgfqpoint{3.974389in}{4.004351in}}{\pgfqpoint{3.982203in}{3.996537in}}%
\pgfpathcurveto{\pgfqpoint{3.990016in}{3.988724in}}{\pgfqpoint{4.000616in}{3.984333in}}{\pgfqpoint{4.011666in}{3.984333in}}%
\pgfpathclose%
\pgfusepath{stroke,fill}%
\end{pgfscope}%
\begin{pgfscope}%
\pgfpathrectangle{\pgfqpoint{0.800000in}{0.528000in}}{\pgfqpoint{4.960000in}{3.696000in}}%
\pgfusepath{clip}%
\pgfsetbuttcap%
\pgfsetroundjoin%
\definecolor{currentfill}{rgb}{0.000000,0.000000,0.000000}%
\pgfsetfillcolor{currentfill}%
\pgfsetlinewidth{1.003750pt}%
\definecolor{currentstroke}{rgb}{0.000000,0.000000,0.000000}%
\pgfsetstrokecolor{currentstroke}%
\pgfsetdash{}{0pt}%
\pgfpathmoveto{\pgfqpoint{4.011666in}{3.984333in}}%
\pgfpathcurveto{\pgfqpoint{4.022716in}{3.984333in}}{\pgfqpoint{4.033315in}{3.988724in}}{\pgfqpoint{4.041128in}{3.996537in}}%
\pgfpathcurveto{\pgfqpoint{4.048942in}{4.004351in}}{\pgfqpoint{4.053332in}{4.014950in}}{\pgfqpoint{4.053332in}{4.026000in}}%
\pgfpathcurveto{\pgfqpoint{4.053332in}{4.037050in}}{\pgfqpoint{4.048942in}{4.047649in}}{\pgfqpoint{4.041128in}{4.055463in}}%
\pgfpathcurveto{\pgfqpoint{4.033315in}{4.063276in}}{\pgfqpoint{4.022716in}{4.067667in}}{\pgfqpoint{4.011666in}{4.067667in}}%
\pgfpathcurveto{\pgfqpoint{4.000616in}{4.067667in}}{\pgfqpoint{3.990016in}{4.063276in}}{\pgfqpoint{3.982203in}{4.055463in}}%
\pgfpathcurveto{\pgfqpoint{3.974389in}{4.047649in}}{\pgfqpoint{3.969999in}{4.037050in}}{\pgfqpoint{3.969999in}{4.026000in}}%
\pgfpathcurveto{\pgfqpoint{3.969999in}{4.014950in}}{\pgfqpoint{3.974389in}{4.004351in}}{\pgfqpoint{3.982203in}{3.996537in}}%
\pgfpathcurveto{\pgfqpoint{3.990016in}{3.988724in}}{\pgfqpoint{4.000616in}{3.984333in}}{\pgfqpoint{4.011666in}{3.984333in}}%
\pgfpathclose%
\pgfusepath{stroke,fill}%
\end{pgfscope}%
\begin{pgfscope}%
\pgfpathrectangle{\pgfqpoint{0.800000in}{0.528000in}}{\pgfqpoint{4.960000in}{3.696000in}}%
\pgfusepath{clip}%
\pgfsetbuttcap%
\pgfsetroundjoin%
\definecolor{currentfill}{rgb}{0.000000,0.000000,0.000000}%
\pgfsetfillcolor{currentfill}%
\pgfsetlinewidth{1.003750pt}%
\definecolor{currentstroke}{rgb}{0.000000,0.000000,0.000000}%
\pgfsetstrokecolor{currentstroke}%
\pgfsetdash{}{0pt}%
\pgfpathmoveto{\pgfqpoint{4.011666in}{3.984333in}}%
\pgfpathcurveto{\pgfqpoint{4.022716in}{3.984333in}}{\pgfqpoint{4.033315in}{3.988724in}}{\pgfqpoint{4.041128in}{3.996537in}}%
\pgfpathcurveto{\pgfqpoint{4.048942in}{4.004351in}}{\pgfqpoint{4.053332in}{4.014950in}}{\pgfqpoint{4.053332in}{4.026000in}}%
\pgfpathcurveto{\pgfqpoint{4.053332in}{4.037050in}}{\pgfqpoint{4.048942in}{4.047649in}}{\pgfqpoint{4.041128in}{4.055463in}}%
\pgfpathcurveto{\pgfqpoint{4.033315in}{4.063276in}}{\pgfqpoint{4.022716in}{4.067667in}}{\pgfqpoint{4.011666in}{4.067667in}}%
\pgfpathcurveto{\pgfqpoint{4.000616in}{4.067667in}}{\pgfqpoint{3.990016in}{4.063276in}}{\pgfqpoint{3.982203in}{4.055463in}}%
\pgfpathcurveto{\pgfqpoint{3.974389in}{4.047649in}}{\pgfqpoint{3.969999in}{4.037050in}}{\pgfqpoint{3.969999in}{4.026000in}}%
\pgfpathcurveto{\pgfqpoint{3.969999in}{4.014950in}}{\pgfqpoint{3.974389in}{4.004351in}}{\pgfqpoint{3.982203in}{3.996537in}}%
\pgfpathcurveto{\pgfqpoint{3.990016in}{3.988724in}}{\pgfqpoint{4.000616in}{3.984333in}}{\pgfqpoint{4.011666in}{3.984333in}}%
\pgfpathclose%
\pgfusepath{stroke,fill}%
\end{pgfscope}%
\begin{pgfscope}%
\pgfpathrectangle{\pgfqpoint{0.800000in}{0.528000in}}{\pgfqpoint{4.960000in}{3.696000in}}%
\pgfusepath{clip}%
\pgfsetbuttcap%
\pgfsetroundjoin%
\definecolor{currentfill}{rgb}{0.000000,0.000000,0.000000}%
\pgfsetfillcolor{currentfill}%
\pgfsetlinewidth{1.003750pt}%
\definecolor{currentstroke}{rgb}{0.000000,0.000000,0.000000}%
\pgfsetstrokecolor{currentstroke}%
\pgfsetdash{}{0pt}%
\pgfpathmoveto{\pgfqpoint{4.011666in}{3.984333in}}%
\pgfpathcurveto{\pgfqpoint{4.022716in}{3.984333in}}{\pgfqpoint{4.033315in}{3.988724in}}{\pgfqpoint{4.041128in}{3.996537in}}%
\pgfpathcurveto{\pgfqpoint{4.048942in}{4.004351in}}{\pgfqpoint{4.053332in}{4.014950in}}{\pgfqpoint{4.053332in}{4.026000in}}%
\pgfpathcurveto{\pgfqpoint{4.053332in}{4.037050in}}{\pgfqpoint{4.048942in}{4.047649in}}{\pgfqpoint{4.041128in}{4.055463in}}%
\pgfpathcurveto{\pgfqpoint{4.033315in}{4.063276in}}{\pgfqpoint{4.022716in}{4.067667in}}{\pgfqpoint{4.011666in}{4.067667in}}%
\pgfpathcurveto{\pgfqpoint{4.000616in}{4.067667in}}{\pgfqpoint{3.990016in}{4.063276in}}{\pgfqpoint{3.982203in}{4.055463in}}%
\pgfpathcurveto{\pgfqpoint{3.974389in}{4.047649in}}{\pgfqpoint{3.969999in}{4.037050in}}{\pgfqpoint{3.969999in}{4.026000in}}%
\pgfpathcurveto{\pgfqpoint{3.969999in}{4.014950in}}{\pgfqpoint{3.974389in}{4.004351in}}{\pgfqpoint{3.982203in}{3.996537in}}%
\pgfpathcurveto{\pgfqpoint{3.990016in}{3.988724in}}{\pgfqpoint{4.000616in}{3.984333in}}{\pgfqpoint{4.011666in}{3.984333in}}%
\pgfpathclose%
\pgfusepath{stroke,fill}%
\end{pgfscope}%
\begin{pgfscope}%
\pgfpathrectangle{\pgfqpoint{0.800000in}{0.528000in}}{\pgfqpoint{4.960000in}{3.696000in}}%
\pgfusepath{clip}%
\pgfsetbuttcap%
\pgfsetroundjoin%
\definecolor{currentfill}{rgb}{0.000000,0.000000,0.000000}%
\pgfsetfillcolor{currentfill}%
\pgfsetlinewidth{1.003750pt}%
\definecolor{currentstroke}{rgb}{0.000000,0.000000,0.000000}%
\pgfsetstrokecolor{currentstroke}%
\pgfsetdash{}{0pt}%
\pgfpathmoveto{\pgfqpoint{4.011666in}{0.684333in}}%
\pgfpathcurveto{\pgfqpoint{4.022716in}{0.684333in}}{\pgfqpoint{4.033315in}{0.688724in}}{\pgfqpoint{4.041128in}{0.696537in}}%
\pgfpathcurveto{\pgfqpoint{4.048942in}{0.704351in}}{\pgfqpoint{4.053332in}{0.714950in}}{\pgfqpoint{4.053332in}{0.726000in}}%
\pgfpathcurveto{\pgfqpoint{4.053332in}{0.737050in}}{\pgfqpoint{4.048942in}{0.747649in}}{\pgfqpoint{4.041128in}{0.755463in}}%
\pgfpathcurveto{\pgfqpoint{4.033315in}{0.763276in}}{\pgfqpoint{4.022716in}{0.767667in}}{\pgfqpoint{4.011666in}{0.767667in}}%
\pgfpathcurveto{\pgfqpoint{4.000616in}{0.767667in}}{\pgfqpoint{3.990016in}{0.763276in}}{\pgfqpoint{3.982203in}{0.755463in}}%
\pgfpathcurveto{\pgfqpoint{3.974389in}{0.747649in}}{\pgfqpoint{3.969999in}{0.737050in}}{\pgfqpoint{3.969999in}{0.726000in}}%
\pgfpathcurveto{\pgfqpoint{3.969999in}{0.714950in}}{\pgfqpoint{3.974389in}{0.704351in}}{\pgfqpoint{3.982203in}{0.696537in}}%
\pgfpathcurveto{\pgfqpoint{3.990016in}{0.688724in}}{\pgfqpoint{4.000616in}{0.684333in}}{\pgfqpoint{4.011666in}{0.684333in}}%
\pgfpathclose%
\pgfusepath{stroke,fill}%
\end{pgfscope}%
\begin{pgfscope}%
\pgfpathrectangle{\pgfqpoint{0.800000in}{0.528000in}}{\pgfqpoint{4.960000in}{3.696000in}}%
\pgfusepath{clip}%
\pgfsetbuttcap%
\pgfsetroundjoin%
\definecolor{currentfill}{rgb}{0.000000,0.000000,0.000000}%
\pgfsetfillcolor{currentfill}%
\pgfsetlinewidth{1.003750pt}%
\definecolor{currentstroke}{rgb}{0.000000,0.000000,0.000000}%
\pgfsetstrokecolor{currentstroke}%
\pgfsetdash{}{0pt}%
\pgfpathmoveto{\pgfqpoint{4.011666in}{3.984333in}}%
\pgfpathcurveto{\pgfqpoint{4.022716in}{3.984333in}}{\pgfqpoint{4.033315in}{3.988724in}}{\pgfqpoint{4.041128in}{3.996537in}}%
\pgfpathcurveto{\pgfqpoint{4.048942in}{4.004351in}}{\pgfqpoint{4.053332in}{4.014950in}}{\pgfqpoint{4.053332in}{4.026000in}}%
\pgfpathcurveto{\pgfqpoint{4.053332in}{4.037050in}}{\pgfqpoint{4.048942in}{4.047649in}}{\pgfqpoint{4.041128in}{4.055463in}}%
\pgfpathcurveto{\pgfqpoint{4.033315in}{4.063276in}}{\pgfqpoint{4.022716in}{4.067667in}}{\pgfqpoint{4.011666in}{4.067667in}}%
\pgfpathcurveto{\pgfqpoint{4.000616in}{4.067667in}}{\pgfqpoint{3.990016in}{4.063276in}}{\pgfqpoint{3.982203in}{4.055463in}}%
\pgfpathcurveto{\pgfqpoint{3.974389in}{4.047649in}}{\pgfqpoint{3.969999in}{4.037050in}}{\pgfqpoint{3.969999in}{4.026000in}}%
\pgfpathcurveto{\pgfqpoint{3.969999in}{4.014950in}}{\pgfqpoint{3.974389in}{4.004351in}}{\pgfqpoint{3.982203in}{3.996537in}}%
\pgfpathcurveto{\pgfqpoint{3.990016in}{3.988724in}}{\pgfqpoint{4.000616in}{3.984333in}}{\pgfqpoint{4.011666in}{3.984333in}}%
\pgfpathclose%
\pgfusepath{stroke,fill}%
\end{pgfscope}%
\begin{pgfscope}%
\pgfpathrectangle{\pgfqpoint{0.800000in}{0.528000in}}{\pgfqpoint{4.960000in}{3.696000in}}%
\pgfusepath{clip}%
\pgfsetbuttcap%
\pgfsetroundjoin%
\definecolor{currentfill}{rgb}{0.000000,0.000000,0.000000}%
\pgfsetfillcolor{currentfill}%
\pgfsetlinewidth{1.003750pt}%
\definecolor{currentstroke}{rgb}{0.000000,0.000000,0.000000}%
\pgfsetstrokecolor{currentstroke}%
\pgfsetdash{}{0pt}%
\pgfpathmoveto{\pgfqpoint{4.011666in}{3.984333in}}%
\pgfpathcurveto{\pgfqpoint{4.022716in}{3.984333in}}{\pgfqpoint{4.033315in}{3.988724in}}{\pgfqpoint{4.041128in}{3.996537in}}%
\pgfpathcurveto{\pgfqpoint{4.048942in}{4.004351in}}{\pgfqpoint{4.053332in}{4.014950in}}{\pgfqpoint{4.053332in}{4.026000in}}%
\pgfpathcurveto{\pgfqpoint{4.053332in}{4.037050in}}{\pgfqpoint{4.048942in}{4.047649in}}{\pgfqpoint{4.041128in}{4.055463in}}%
\pgfpathcurveto{\pgfqpoint{4.033315in}{4.063276in}}{\pgfqpoint{4.022716in}{4.067667in}}{\pgfqpoint{4.011666in}{4.067667in}}%
\pgfpathcurveto{\pgfqpoint{4.000616in}{4.067667in}}{\pgfqpoint{3.990016in}{4.063276in}}{\pgfqpoint{3.982203in}{4.055463in}}%
\pgfpathcurveto{\pgfqpoint{3.974389in}{4.047649in}}{\pgfqpoint{3.969999in}{4.037050in}}{\pgfqpoint{3.969999in}{4.026000in}}%
\pgfpathcurveto{\pgfqpoint{3.969999in}{4.014950in}}{\pgfqpoint{3.974389in}{4.004351in}}{\pgfqpoint{3.982203in}{3.996537in}}%
\pgfpathcurveto{\pgfqpoint{3.990016in}{3.988724in}}{\pgfqpoint{4.000616in}{3.984333in}}{\pgfqpoint{4.011666in}{3.984333in}}%
\pgfpathclose%
\pgfusepath{stroke,fill}%
\end{pgfscope}%
\begin{pgfscope}%
\pgfpathrectangle{\pgfqpoint{0.800000in}{0.528000in}}{\pgfqpoint{4.960000in}{3.696000in}}%
\pgfusepath{clip}%
\pgfsetbuttcap%
\pgfsetroundjoin%
\definecolor{currentfill}{rgb}{0.000000,0.000000,0.000000}%
\pgfsetfillcolor{currentfill}%
\pgfsetlinewidth{1.003750pt}%
\definecolor{currentstroke}{rgb}{0.000000,0.000000,0.000000}%
\pgfsetstrokecolor{currentstroke}%
\pgfsetdash{}{0pt}%
\pgfpathmoveto{\pgfqpoint{4.011666in}{3.984333in}}%
\pgfpathcurveto{\pgfqpoint{4.022716in}{3.984333in}}{\pgfqpoint{4.033315in}{3.988724in}}{\pgfqpoint{4.041128in}{3.996537in}}%
\pgfpathcurveto{\pgfqpoint{4.048942in}{4.004351in}}{\pgfqpoint{4.053332in}{4.014950in}}{\pgfqpoint{4.053332in}{4.026000in}}%
\pgfpathcurveto{\pgfqpoint{4.053332in}{4.037050in}}{\pgfqpoint{4.048942in}{4.047649in}}{\pgfqpoint{4.041128in}{4.055463in}}%
\pgfpathcurveto{\pgfqpoint{4.033315in}{4.063276in}}{\pgfqpoint{4.022716in}{4.067667in}}{\pgfqpoint{4.011666in}{4.067667in}}%
\pgfpathcurveto{\pgfqpoint{4.000616in}{4.067667in}}{\pgfqpoint{3.990016in}{4.063276in}}{\pgfqpoint{3.982203in}{4.055463in}}%
\pgfpathcurveto{\pgfqpoint{3.974389in}{4.047649in}}{\pgfqpoint{3.969999in}{4.037050in}}{\pgfqpoint{3.969999in}{4.026000in}}%
\pgfpathcurveto{\pgfqpoint{3.969999in}{4.014950in}}{\pgfqpoint{3.974389in}{4.004351in}}{\pgfqpoint{3.982203in}{3.996537in}}%
\pgfpathcurveto{\pgfqpoint{3.990016in}{3.988724in}}{\pgfqpoint{4.000616in}{3.984333in}}{\pgfqpoint{4.011666in}{3.984333in}}%
\pgfpathclose%
\pgfusepath{stroke,fill}%
\end{pgfscope}%
\begin{pgfscope}%
\pgfpathrectangle{\pgfqpoint{0.800000in}{0.528000in}}{\pgfqpoint{4.960000in}{3.696000in}}%
\pgfusepath{clip}%
\pgfsetbuttcap%
\pgfsetroundjoin%
\definecolor{currentfill}{rgb}{0.000000,0.000000,0.000000}%
\pgfsetfillcolor{currentfill}%
\pgfsetlinewidth{1.003750pt}%
\definecolor{currentstroke}{rgb}{0.000000,0.000000,0.000000}%
\pgfsetstrokecolor{currentstroke}%
\pgfsetdash{}{0pt}%
\pgfpathmoveto{\pgfqpoint{4.011666in}{3.984333in}}%
\pgfpathcurveto{\pgfqpoint{4.022716in}{3.984333in}}{\pgfqpoint{4.033315in}{3.988724in}}{\pgfqpoint{4.041128in}{3.996537in}}%
\pgfpathcurveto{\pgfqpoint{4.048942in}{4.004351in}}{\pgfqpoint{4.053332in}{4.014950in}}{\pgfqpoint{4.053332in}{4.026000in}}%
\pgfpathcurveto{\pgfqpoint{4.053332in}{4.037050in}}{\pgfqpoint{4.048942in}{4.047649in}}{\pgfqpoint{4.041128in}{4.055463in}}%
\pgfpathcurveto{\pgfqpoint{4.033315in}{4.063276in}}{\pgfqpoint{4.022716in}{4.067667in}}{\pgfqpoint{4.011666in}{4.067667in}}%
\pgfpathcurveto{\pgfqpoint{4.000616in}{4.067667in}}{\pgfqpoint{3.990016in}{4.063276in}}{\pgfqpoint{3.982203in}{4.055463in}}%
\pgfpathcurveto{\pgfqpoint{3.974389in}{4.047649in}}{\pgfqpoint{3.969999in}{4.037050in}}{\pgfqpoint{3.969999in}{4.026000in}}%
\pgfpathcurveto{\pgfqpoint{3.969999in}{4.014950in}}{\pgfqpoint{3.974389in}{4.004351in}}{\pgfqpoint{3.982203in}{3.996537in}}%
\pgfpathcurveto{\pgfqpoint{3.990016in}{3.988724in}}{\pgfqpoint{4.000616in}{3.984333in}}{\pgfqpoint{4.011666in}{3.984333in}}%
\pgfpathclose%
\pgfusepath{stroke,fill}%
\end{pgfscope}%
\begin{pgfscope}%
\pgfpathrectangle{\pgfqpoint{0.800000in}{0.528000in}}{\pgfqpoint{4.960000in}{3.696000in}}%
\pgfusepath{clip}%
\pgfsetbuttcap%
\pgfsetroundjoin%
\definecolor{currentfill}{rgb}{0.000000,0.000000,0.000000}%
\pgfsetfillcolor{currentfill}%
\pgfsetlinewidth{1.003750pt}%
\definecolor{currentstroke}{rgb}{0.000000,0.000000,0.000000}%
\pgfsetstrokecolor{currentstroke}%
\pgfsetdash{}{0pt}%
\pgfpathmoveto{\pgfqpoint{4.011666in}{3.984333in}}%
\pgfpathcurveto{\pgfqpoint{4.022716in}{3.984333in}}{\pgfqpoint{4.033315in}{3.988724in}}{\pgfqpoint{4.041128in}{3.996537in}}%
\pgfpathcurveto{\pgfqpoint{4.048942in}{4.004351in}}{\pgfqpoint{4.053332in}{4.014950in}}{\pgfqpoint{4.053332in}{4.026000in}}%
\pgfpathcurveto{\pgfqpoint{4.053332in}{4.037050in}}{\pgfqpoint{4.048942in}{4.047649in}}{\pgfqpoint{4.041128in}{4.055463in}}%
\pgfpathcurveto{\pgfqpoint{4.033315in}{4.063276in}}{\pgfqpoint{4.022716in}{4.067667in}}{\pgfqpoint{4.011666in}{4.067667in}}%
\pgfpathcurveto{\pgfqpoint{4.000616in}{4.067667in}}{\pgfqpoint{3.990016in}{4.063276in}}{\pgfqpoint{3.982203in}{4.055463in}}%
\pgfpathcurveto{\pgfqpoint{3.974389in}{4.047649in}}{\pgfqpoint{3.969999in}{4.037050in}}{\pgfqpoint{3.969999in}{4.026000in}}%
\pgfpathcurveto{\pgfqpoint{3.969999in}{4.014950in}}{\pgfqpoint{3.974389in}{4.004351in}}{\pgfqpoint{3.982203in}{3.996537in}}%
\pgfpathcurveto{\pgfqpoint{3.990016in}{3.988724in}}{\pgfqpoint{4.000616in}{3.984333in}}{\pgfqpoint{4.011666in}{3.984333in}}%
\pgfpathclose%
\pgfusepath{stroke,fill}%
\end{pgfscope}%
\begin{pgfscope}%
\pgfpathrectangle{\pgfqpoint{0.800000in}{0.528000in}}{\pgfqpoint{4.960000in}{3.696000in}}%
\pgfusepath{clip}%
\pgfsetbuttcap%
\pgfsetroundjoin%
\definecolor{currentfill}{rgb}{0.000000,0.000000,0.000000}%
\pgfsetfillcolor{currentfill}%
\pgfsetlinewidth{1.003750pt}%
\definecolor{currentstroke}{rgb}{0.000000,0.000000,0.000000}%
\pgfsetstrokecolor{currentstroke}%
\pgfsetdash{}{0pt}%
\pgfpathmoveto{\pgfqpoint{4.011666in}{3.984333in}}%
\pgfpathcurveto{\pgfqpoint{4.022716in}{3.984333in}}{\pgfqpoint{4.033315in}{3.988724in}}{\pgfqpoint{4.041128in}{3.996537in}}%
\pgfpathcurveto{\pgfqpoint{4.048942in}{4.004351in}}{\pgfqpoint{4.053332in}{4.014950in}}{\pgfqpoint{4.053332in}{4.026000in}}%
\pgfpathcurveto{\pgfqpoint{4.053332in}{4.037050in}}{\pgfqpoint{4.048942in}{4.047649in}}{\pgfqpoint{4.041128in}{4.055463in}}%
\pgfpathcurveto{\pgfqpoint{4.033315in}{4.063276in}}{\pgfqpoint{4.022716in}{4.067667in}}{\pgfqpoint{4.011666in}{4.067667in}}%
\pgfpathcurveto{\pgfqpoint{4.000616in}{4.067667in}}{\pgfqpoint{3.990016in}{4.063276in}}{\pgfqpoint{3.982203in}{4.055463in}}%
\pgfpathcurveto{\pgfqpoint{3.974389in}{4.047649in}}{\pgfqpoint{3.969999in}{4.037050in}}{\pgfqpoint{3.969999in}{4.026000in}}%
\pgfpathcurveto{\pgfqpoint{3.969999in}{4.014950in}}{\pgfqpoint{3.974389in}{4.004351in}}{\pgfqpoint{3.982203in}{3.996537in}}%
\pgfpathcurveto{\pgfqpoint{3.990016in}{3.988724in}}{\pgfqpoint{4.000616in}{3.984333in}}{\pgfqpoint{4.011666in}{3.984333in}}%
\pgfpathclose%
\pgfusepath{stroke,fill}%
\end{pgfscope}%
\begin{pgfscope}%
\pgfpathrectangle{\pgfqpoint{0.800000in}{0.528000in}}{\pgfqpoint{4.960000in}{3.696000in}}%
\pgfusepath{clip}%
\pgfsetbuttcap%
\pgfsetroundjoin%
\definecolor{currentfill}{rgb}{0.000000,0.000000,0.000000}%
\pgfsetfillcolor{currentfill}%
\pgfsetlinewidth{1.003750pt}%
\definecolor{currentstroke}{rgb}{0.000000,0.000000,0.000000}%
\pgfsetstrokecolor{currentstroke}%
\pgfsetdash{}{0pt}%
\pgfpathmoveto{\pgfqpoint{4.011666in}{3.984333in}}%
\pgfpathcurveto{\pgfqpoint{4.022716in}{3.984333in}}{\pgfqpoint{4.033315in}{3.988724in}}{\pgfqpoint{4.041128in}{3.996537in}}%
\pgfpathcurveto{\pgfqpoint{4.048942in}{4.004351in}}{\pgfqpoint{4.053332in}{4.014950in}}{\pgfqpoint{4.053332in}{4.026000in}}%
\pgfpathcurveto{\pgfqpoint{4.053332in}{4.037050in}}{\pgfqpoint{4.048942in}{4.047649in}}{\pgfqpoint{4.041128in}{4.055463in}}%
\pgfpathcurveto{\pgfqpoint{4.033315in}{4.063276in}}{\pgfqpoint{4.022716in}{4.067667in}}{\pgfqpoint{4.011666in}{4.067667in}}%
\pgfpathcurveto{\pgfqpoint{4.000616in}{4.067667in}}{\pgfqpoint{3.990016in}{4.063276in}}{\pgfqpoint{3.982203in}{4.055463in}}%
\pgfpathcurveto{\pgfqpoint{3.974389in}{4.047649in}}{\pgfqpoint{3.969999in}{4.037050in}}{\pgfqpoint{3.969999in}{4.026000in}}%
\pgfpathcurveto{\pgfqpoint{3.969999in}{4.014950in}}{\pgfqpoint{3.974389in}{4.004351in}}{\pgfqpoint{3.982203in}{3.996537in}}%
\pgfpathcurveto{\pgfqpoint{3.990016in}{3.988724in}}{\pgfqpoint{4.000616in}{3.984333in}}{\pgfqpoint{4.011666in}{3.984333in}}%
\pgfpathclose%
\pgfusepath{stroke,fill}%
\end{pgfscope}%
\begin{pgfscope}%
\pgfpathrectangle{\pgfqpoint{0.800000in}{0.528000in}}{\pgfqpoint{4.960000in}{3.696000in}}%
\pgfusepath{clip}%
\pgfsetbuttcap%
\pgfsetroundjoin%
\definecolor{currentfill}{rgb}{0.000000,0.000000,0.000000}%
\pgfsetfillcolor{currentfill}%
\pgfsetlinewidth{1.003750pt}%
\definecolor{currentstroke}{rgb}{0.000000,0.000000,0.000000}%
\pgfsetstrokecolor{currentstroke}%
\pgfsetdash{}{0pt}%
\pgfpathmoveto{\pgfqpoint{4.011666in}{3.984333in}}%
\pgfpathcurveto{\pgfqpoint{4.022716in}{3.984333in}}{\pgfqpoint{4.033315in}{3.988724in}}{\pgfqpoint{4.041128in}{3.996537in}}%
\pgfpathcurveto{\pgfqpoint{4.048942in}{4.004351in}}{\pgfqpoint{4.053332in}{4.014950in}}{\pgfqpoint{4.053332in}{4.026000in}}%
\pgfpathcurveto{\pgfqpoint{4.053332in}{4.037050in}}{\pgfqpoint{4.048942in}{4.047649in}}{\pgfqpoint{4.041128in}{4.055463in}}%
\pgfpathcurveto{\pgfqpoint{4.033315in}{4.063276in}}{\pgfqpoint{4.022716in}{4.067667in}}{\pgfqpoint{4.011666in}{4.067667in}}%
\pgfpathcurveto{\pgfqpoint{4.000616in}{4.067667in}}{\pgfqpoint{3.990016in}{4.063276in}}{\pgfqpoint{3.982203in}{4.055463in}}%
\pgfpathcurveto{\pgfqpoint{3.974389in}{4.047649in}}{\pgfqpoint{3.969999in}{4.037050in}}{\pgfqpoint{3.969999in}{4.026000in}}%
\pgfpathcurveto{\pgfqpoint{3.969999in}{4.014950in}}{\pgfqpoint{3.974389in}{4.004351in}}{\pgfqpoint{3.982203in}{3.996537in}}%
\pgfpathcurveto{\pgfqpoint{3.990016in}{3.988724in}}{\pgfqpoint{4.000616in}{3.984333in}}{\pgfqpoint{4.011666in}{3.984333in}}%
\pgfpathclose%
\pgfusepath{stroke,fill}%
\end{pgfscope}%
\begin{pgfscope}%
\pgfpathrectangle{\pgfqpoint{0.800000in}{0.528000in}}{\pgfqpoint{4.960000in}{3.696000in}}%
\pgfusepath{clip}%
\pgfsetbuttcap%
\pgfsetroundjoin%
\definecolor{currentfill}{rgb}{0.000000,0.000000,0.000000}%
\pgfsetfillcolor{currentfill}%
\pgfsetlinewidth{1.003750pt}%
\definecolor{currentstroke}{rgb}{0.000000,0.000000,0.000000}%
\pgfsetstrokecolor{currentstroke}%
\pgfsetdash{}{0pt}%
\pgfpathmoveto{\pgfqpoint{4.011666in}{3.984333in}}%
\pgfpathcurveto{\pgfqpoint{4.022716in}{3.984333in}}{\pgfqpoint{4.033315in}{3.988724in}}{\pgfqpoint{4.041128in}{3.996537in}}%
\pgfpathcurveto{\pgfqpoint{4.048942in}{4.004351in}}{\pgfqpoint{4.053332in}{4.014950in}}{\pgfqpoint{4.053332in}{4.026000in}}%
\pgfpathcurveto{\pgfqpoint{4.053332in}{4.037050in}}{\pgfqpoint{4.048942in}{4.047649in}}{\pgfqpoint{4.041128in}{4.055463in}}%
\pgfpathcurveto{\pgfqpoint{4.033315in}{4.063276in}}{\pgfqpoint{4.022716in}{4.067667in}}{\pgfqpoint{4.011666in}{4.067667in}}%
\pgfpathcurveto{\pgfqpoint{4.000616in}{4.067667in}}{\pgfqpoint{3.990016in}{4.063276in}}{\pgfqpoint{3.982203in}{4.055463in}}%
\pgfpathcurveto{\pgfqpoint{3.974389in}{4.047649in}}{\pgfqpoint{3.969999in}{4.037050in}}{\pgfqpoint{3.969999in}{4.026000in}}%
\pgfpathcurveto{\pgfqpoint{3.969999in}{4.014950in}}{\pgfqpoint{3.974389in}{4.004351in}}{\pgfqpoint{3.982203in}{3.996537in}}%
\pgfpathcurveto{\pgfqpoint{3.990016in}{3.988724in}}{\pgfqpoint{4.000616in}{3.984333in}}{\pgfqpoint{4.011666in}{3.984333in}}%
\pgfpathclose%
\pgfusepath{stroke,fill}%
\end{pgfscope}%
\begin{pgfscope}%
\pgfpathrectangle{\pgfqpoint{0.800000in}{0.528000in}}{\pgfqpoint{4.960000in}{3.696000in}}%
\pgfusepath{clip}%
\pgfsetbuttcap%
\pgfsetroundjoin%
\definecolor{currentfill}{rgb}{0.000000,0.000000,0.000000}%
\pgfsetfillcolor{currentfill}%
\pgfsetlinewidth{1.003750pt}%
\definecolor{currentstroke}{rgb}{0.000000,0.000000,0.000000}%
\pgfsetstrokecolor{currentstroke}%
\pgfsetdash{}{0pt}%
\pgfpathmoveto{\pgfqpoint{4.011666in}{3.984333in}}%
\pgfpathcurveto{\pgfqpoint{4.022716in}{3.984333in}}{\pgfqpoint{4.033315in}{3.988724in}}{\pgfqpoint{4.041128in}{3.996537in}}%
\pgfpathcurveto{\pgfqpoint{4.048942in}{4.004351in}}{\pgfqpoint{4.053332in}{4.014950in}}{\pgfqpoint{4.053332in}{4.026000in}}%
\pgfpathcurveto{\pgfqpoint{4.053332in}{4.037050in}}{\pgfqpoint{4.048942in}{4.047649in}}{\pgfqpoint{4.041128in}{4.055463in}}%
\pgfpathcurveto{\pgfqpoint{4.033315in}{4.063276in}}{\pgfqpoint{4.022716in}{4.067667in}}{\pgfqpoint{4.011666in}{4.067667in}}%
\pgfpathcurveto{\pgfqpoint{4.000616in}{4.067667in}}{\pgfqpoint{3.990016in}{4.063276in}}{\pgfqpoint{3.982203in}{4.055463in}}%
\pgfpathcurveto{\pgfqpoint{3.974389in}{4.047649in}}{\pgfqpoint{3.969999in}{4.037050in}}{\pgfqpoint{3.969999in}{4.026000in}}%
\pgfpathcurveto{\pgfqpoint{3.969999in}{4.014950in}}{\pgfqpoint{3.974389in}{4.004351in}}{\pgfqpoint{3.982203in}{3.996537in}}%
\pgfpathcurveto{\pgfqpoint{3.990016in}{3.988724in}}{\pgfqpoint{4.000616in}{3.984333in}}{\pgfqpoint{4.011666in}{3.984333in}}%
\pgfpathclose%
\pgfusepath{stroke,fill}%
\end{pgfscope}%
\begin{pgfscope}%
\pgfpathrectangle{\pgfqpoint{0.800000in}{0.528000in}}{\pgfqpoint{4.960000in}{3.696000in}}%
\pgfusepath{clip}%
\pgfsetbuttcap%
\pgfsetroundjoin%
\definecolor{currentfill}{rgb}{0.000000,0.000000,0.000000}%
\pgfsetfillcolor{currentfill}%
\pgfsetlinewidth{1.003750pt}%
\definecolor{currentstroke}{rgb}{0.000000,0.000000,0.000000}%
\pgfsetstrokecolor{currentstroke}%
\pgfsetdash{}{0pt}%
\pgfpathmoveto{\pgfqpoint{4.011666in}{0.684333in}}%
\pgfpathcurveto{\pgfqpoint{4.022716in}{0.684333in}}{\pgfqpoint{4.033315in}{0.688724in}}{\pgfqpoint{4.041128in}{0.696537in}}%
\pgfpathcurveto{\pgfqpoint{4.048942in}{0.704351in}}{\pgfqpoint{4.053332in}{0.714950in}}{\pgfqpoint{4.053332in}{0.726000in}}%
\pgfpathcurveto{\pgfqpoint{4.053332in}{0.737050in}}{\pgfqpoint{4.048942in}{0.747649in}}{\pgfqpoint{4.041128in}{0.755463in}}%
\pgfpathcurveto{\pgfqpoint{4.033315in}{0.763276in}}{\pgfqpoint{4.022716in}{0.767667in}}{\pgfqpoint{4.011666in}{0.767667in}}%
\pgfpathcurveto{\pgfqpoint{4.000616in}{0.767667in}}{\pgfqpoint{3.990016in}{0.763276in}}{\pgfqpoint{3.982203in}{0.755463in}}%
\pgfpathcurveto{\pgfqpoint{3.974389in}{0.747649in}}{\pgfqpoint{3.969999in}{0.737050in}}{\pgfqpoint{3.969999in}{0.726000in}}%
\pgfpathcurveto{\pgfqpoint{3.969999in}{0.714950in}}{\pgfqpoint{3.974389in}{0.704351in}}{\pgfqpoint{3.982203in}{0.696537in}}%
\pgfpathcurveto{\pgfqpoint{3.990016in}{0.688724in}}{\pgfqpoint{4.000616in}{0.684333in}}{\pgfqpoint{4.011666in}{0.684333in}}%
\pgfpathclose%
\pgfusepath{stroke,fill}%
\end{pgfscope}%
\begin{pgfscope}%
\pgfpathrectangle{\pgfqpoint{0.800000in}{0.528000in}}{\pgfqpoint{4.960000in}{3.696000in}}%
\pgfusepath{clip}%
\pgfsetbuttcap%
\pgfsetroundjoin%
\definecolor{currentfill}{rgb}{0.000000,0.000000,0.000000}%
\pgfsetfillcolor{currentfill}%
\pgfsetlinewidth{1.003750pt}%
\definecolor{currentstroke}{rgb}{0.000000,0.000000,0.000000}%
\pgfsetstrokecolor{currentstroke}%
\pgfsetdash{}{0pt}%
\pgfpathmoveto{\pgfqpoint{4.011666in}{3.984333in}}%
\pgfpathcurveto{\pgfqpoint{4.022716in}{3.984333in}}{\pgfqpoint{4.033315in}{3.988724in}}{\pgfqpoint{4.041128in}{3.996537in}}%
\pgfpathcurveto{\pgfqpoint{4.048942in}{4.004351in}}{\pgfqpoint{4.053332in}{4.014950in}}{\pgfqpoint{4.053332in}{4.026000in}}%
\pgfpathcurveto{\pgfqpoint{4.053332in}{4.037050in}}{\pgfqpoint{4.048942in}{4.047649in}}{\pgfqpoint{4.041128in}{4.055463in}}%
\pgfpathcurveto{\pgfqpoint{4.033315in}{4.063276in}}{\pgfqpoint{4.022716in}{4.067667in}}{\pgfqpoint{4.011666in}{4.067667in}}%
\pgfpathcurveto{\pgfqpoint{4.000616in}{4.067667in}}{\pgfqpoint{3.990016in}{4.063276in}}{\pgfqpoint{3.982203in}{4.055463in}}%
\pgfpathcurveto{\pgfqpoint{3.974389in}{4.047649in}}{\pgfqpoint{3.969999in}{4.037050in}}{\pgfqpoint{3.969999in}{4.026000in}}%
\pgfpathcurveto{\pgfqpoint{3.969999in}{4.014950in}}{\pgfqpoint{3.974389in}{4.004351in}}{\pgfqpoint{3.982203in}{3.996537in}}%
\pgfpathcurveto{\pgfqpoint{3.990016in}{3.988724in}}{\pgfqpoint{4.000616in}{3.984333in}}{\pgfqpoint{4.011666in}{3.984333in}}%
\pgfpathclose%
\pgfusepath{stroke,fill}%
\end{pgfscope}%
\begin{pgfscope}%
\pgfpathrectangle{\pgfqpoint{0.800000in}{0.528000in}}{\pgfqpoint{4.960000in}{3.696000in}}%
\pgfusepath{clip}%
\pgfsetbuttcap%
\pgfsetroundjoin%
\definecolor{currentfill}{rgb}{0.000000,0.000000,0.000000}%
\pgfsetfillcolor{currentfill}%
\pgfsetlinewidth{1.003750pt}%
\definecolor{currentstroke}{rgb}{0.000000,0.000000,0.000000}%
\pgfsetstrokecolor{currentstroke}%
\pgfsetdash{}{0pt}%
\pgfpathmoveto{\pgfqpoint{4.011666in}{3.984333in}}%
\pgfpathcurveto{\pgfqpoint{4.022716in}{3.984333in}}{\pgfqpoint{4.033315in}{3.988724in}}{\pgfqpoint{4.041128in}{3.996537in}}%
\pgfpathcurveto{\pgfqpoint{4.048942in}{4.004351in}}{\pgfqpoint{4.053332in}{4.014950in}}{\pgfqpoint{4.053332in}{4.026000in}}%
\pgfpathcurveto{\pgfqpoint{4.053332in}{4.037050in}}{\pgfqpoint{4.048942in}{4.047649in}}{\pgfqpoint{4.041128in}{4.055463in}}%
\pgfpathcurveto{\pgfqpoint{4.033315in}{4.063276in}}{\pgfqpoint{4.022716in}{4.067667in}}{\pgfqpoint{4.011666in}{4.067667in}}%
\pgfpathcurveto{\pgfqpoint{4.000616in}{4.067667in}}{\pgfqpoint{3.990016in}{4.063276in}}{\pgfqpoint{3.982203in}{4.055463in}}%
\pgfpathcurveto{\pgfqpoint{3.974389in}{4.047649in}}{\pgfqpoint{3.969999in}{4.037050in}}{\pgfqpoint{3.969999in}{4.026000in}}%
\pgfpathcurveto{\pgfqpoint{3.969999in}{4.014950in}}{\pgfqpoint{3.974389in}{4.004351in}}{\pgfqpoint{3.982203in}{3.996537in}}%
\pgfpathcurveto{\pgfqpoint{3.990016in}{3.988724in}}{\pgfqpoint{4.000616in}{3.984333in}}{\pgfqpoint{4.011666in}{3.984333in}}%
\pgfpathclose%
\pgfusepath{stroke,fill}%
\end{pgfscope}%
\begin{pgfscope}%
\pgfpathrectangle{\pgfqpoint{0.800000in}{0.528000in}}{\pgfqpoint{4.960000in}{3.696000in}}%
\pgfusepath{clip}%
\pgfsetbuttcap%
\pgfsetroundjoin%
\definecolor{currentfill}{rgb}{0.000000,0.000000,0.000000}%
\pgfsetfillcolor{currentfill}%
\pgfsetlinewidth{1.003750pt}%
\definecolor{currentstroke}{rgb}{0.000000,0.000000,0.000000}%
\pgfsetstrokecolor{currentstroke}%
\pgfsetdash{}{0pt}%
\pgfpathmoveto{\pgfqpoint{4.011666in}{3.984333in}}%
\pgfpathcurveto{\pgfqpoint{4.022716in}{3.984333in}}{\pgfqpoint{4.033315in}{3.988724in}}{\pgfqpoint{4.041128in}{3.996537in}}%
\pgfpathcurveto{\pgfqpoint{4.048942in}{4.004351in}}{\pgfqpoint{4.053332in}{4.014950in}}{\pgfqpoint{4.053332in}{4.026000in}}%
\pgfpathcurveto{\pgfqpoint{4.053332in}{4.037050in}}{\pgfqpoint{4.048942in}{4.047649in}}{\pgfqpoint{4.041128in}{4.055463in}}%
\pgfpathcurveto{\pgfqpoint{4.033315in}{4.063276in}}{\pgfqpoint{4.022716in}{4.067667in}}{\pgfqpoint{4.011666in}{4.067667in}}%
\pgfpathcurveto{\pgfqpoint{4.000616in}{4.067667in}}{\pgfqpoint{3.990016in}{4.063276in}}{\pgfqpoint{3.982203in}{4.055463in}}%
\pgfpathcurveto{\pgfqpoint{3.974389in}{4.047649in}}{\pgfqpoint{3.969999in}{4.037050in}}{\pgfqpoint{3.969999in}{4.026000in}}%
\pgfpathcurveto{\pgfqpoint{3.969999in}{4.014950in}}{\pgfqpoint{3.974389in}{4.004351in}}{\pgfqpoint{3.982203in}{3.996537in}}%
\pgfpathcurveto{\pgfqpoint{3.990016in}{3.988724in}}{\pgfqpoint{4.000616in}{3.984333in}}{\pgfqpoint{4.011666in}{3.984333in}}%
\pgfpathclose%
\pgfusepath{stroke,fill}%
\end{pgfscope}%
\begin{pgfscope}%
\pgfpathrectangle{\pgfqpoint{0.800000in}{0.528000in}}{\pgfqpoint{4.960000in}{3.696000in}}%
\pgfusepath{clip}%
\pgfsetbuttcap%
\pgfsetroundjoin%
\definecolor{currentfill}{rgb}{0.000000,0.000000,0.000000}%
\pgfsetfillcolor{currentfill}%
\pgfsetlinewidth{1.003750pt}%
\definecolor{currentstroke}{rgb}{0.000000,0.000000,0.000000}%
\pgfsetstrokecolor{currentstroke}%
\pgfsetdash{}{0pt}%
\pgfpathmoveto{\pgfqpoint{4.011666in}{0.684333in}}%
\pgfpathcurveto{\pgfqpoint{4.022716in}{0.684333in}}{\pgfqpoint{4.033315in}{0.688724in}}{\pgfqpoint{4.041128in}{0.696537in}}%
\pgfpathcurveto{\pgfqpoint{4.048942in}{0.704351in}}{\pgfqpoint{4.053332in}{0.714950in}}{\pgfqpoint{4.053332in}{0.726000in}}%
\pgfpathcurveto{\pgfqpoint{4.053332in}{0.737050in}}{\pgfqpoint{4.048942in}{0.747649in}}{\pgfqpoint{4.041128in}{0.755463in}}%
\pgfpathcurveto{\pgfqpoint{4.033315in}{0.763276in}}{\pgfqpoint{4.022716in}{0.767667in}}{\pgfqpoint{4.011666in}{0.767667in}}%
\pgfpathcurveto{\pgfqpoint{4.000616in}{0.767667in}}{\pgfqpoint{3.990016in}{0.763276in}}{\pgfqpoint{3.982203in}{0.755463in}}%
\pgfpathcurveto{\pgfqpoint{3.974389in}{0.747649in}}{\pgfqpoint{3.969999in}{0.737050in}}{\pgfqpoint{3.969999in}{0.726000in}}%
\pgfpathcurveto{\pgfqpoint{3.969999in}{0.714950in}}{\pgfqpoint{3.974389in}{0.704351in}}{\pgfqpoint{3.982203in}{0.696537in}}%
\pgfpathcurveto{\pgfqpoint{3.990016in}{0.688724in}}{\pgfqpoint{4.000616in}{0.684333in}}{\pgfqpoint{4.011666in}{0.684333in}}%
\pgfpathclose%
\pgfusepath{stroke,fill}%
\end{pgfscope}%
\begin{pgfscope}%
\pgfpathrectangle{\pgfqpoint{0.800000in}{0.528000in}}{\pgfqpoint{4.960000in}{3.696000in}}%
\pgfusepath{clip}%
\pgfsetbuttcap%
\pgfsetroundjoin%
\definecolor{currentfill}{rgb}{0.000000,0.000000,0.000000}%
\pgfsetfillcolor{currentfill}%
\pgfsetlinewidth{1.003750pt}%
\definecolor{currentstroke}{rgb}{0.000000,0.000000,0.000000}%
\pgfsetstrokecolor{currentstroke}%
\pgfsetdash{}{0pt}%
\pgfpathmoveto{\pgfqpoint{4.011666in}{3.984333in}}%
\pgfpathcurveto{\pgfqpoint{4.022716in}{3.984333in}}{\pgfqpoint{4.033315in}{3.988724in}}{\pgfqpoint{4.041128in}{3.996537in}}%
\pgfpathcurveto{\pgfqpoint{4.048942in}{4.004351in}}{\pgfqpoint{4.053332in}{4.014950in}}{\pgfqpoint{4.053332in}{4.026000in}}%
\pgfpathcurveto{\pgfqpoint{4.053332in}{4.037050in}}{\pgfqpoint{4.048942in}{4.047649in}}{\pgfqpoint{4.041128in}{4.055463in}}%
\pgfpathcurveto{\pgfqpoint{4.033315in}{4.063276in}}{\pgfqpoint{4.022716in}{4.067667in}}{\pgfqpoint{4.011666in}{4.067667in}}%
\pgfpathcurveto{\pgfqpoint{4.000616in}{4.067667in}}{\pgfqpoint{3.990016in}{4.063276in}}{\pgfqpoint{3.982203in}{4.055463in}}%
\pgfpathcurveto{\pgfqpoint{3.974389in}{4.047649in}}{\pgfqpoint{3.969999in}{4.037050in}}{\pgfqpoint{3.969999in}{4.026000in}}%
\pgfpathcurveto{\pgfqpoint{3.969999in}{4.014950in}}{\pgfqpoint{3.974389in}{4.004351in}}{\pgfqpoint{3.982203in}{3.996537in}}%
\pgfpathcurveto{\pgfqpoint{3.990016in}{3.988724in}}{\pgfqpoint{4.000616in}{3.984333in}}{\pgfqpoint{4.011666in}{3.984333in}}%
\pgfpathclose%
\pgfusepath{stroke,fill}%
\end{pgfscope}%
\begin{pgfscope}%
\pgfpathrectangle{\pgfqpoint{0.800000in}{0.528000in}}{\pgfqpoint{4.960000in}{3.696000in}}%
\pgfusepath{clip}%
\pgfsetbuttcap%
\pgfsetroundjoin%
\definecolor{currentfill}{rgb}{0.000000,0.000000,0.000000}%
\pgfsetfillcolor{currentfill}%
\pgfsetlinewidth{1.003750pt}%
\definecolor{currentstroke}{rgb}{0.000000,0.000000,0.000000}%
\pgfsetstrokecolor{currentstroke}%
\pgfsetdash{}{0pt}%
\pgfpathmoveto{\pgfqpoint{4.011666in}{3.984333in}}%
\pgfpathcurveto{\pgfqpoint{4.022716in}{3.984333in}}{\pgfqpoint{4.033315in}{3.988724in}}{\pgfqpoint{4.041128in}{3.996537in}}%
\pgfpathcurveto{\pgfqpoint{4.048942in}{4.004351in}}{\pgfqpoint{4.053332in}{4.014950in}}{\pgfqpoint{4.053332in}{4.026000in}}%
\pgfpathcurveto{\pgfqpoint{4.053332in}{4.037050in}}{\pgfqpoint{4.048942in}{4.047649in}}{\pgfqpoint{4.041128in}{4.055463in}}%
\pgfpathcurveto{\pgfqpoint{4.033315in}{4.063276in}}{\pgfqpoint{4.022716in}{4.067667in}}{\pgfqpoint{4.011666in}{4.067667in}}%
\pgfpathcurveto{\pgfqpoint{4.000616in}{4.067667in}}{\pgfqpoint{3.990016in}{4.063276in}}{\pgfqpoint{3.982203in}{4.055463in}}%
\pgfpathcurveto{\pgfqpoint{3.974389in}{4.047649in}}{\pgfqpoint{3.969999in}{4.037050in}}{\pgfqpoint{3.969999in}{4.026000in}}%
\pgfpathcurveto{\pgfqpoint{3.969999in}{4.014950in}}{\pgfqpoint{3.974389in}{4.004351in}}{\pgfqpoint{3.982203in}{3.996537in}}%
\pgfpathcurveto{\pgfqpoint{3.990016in}{3.988724in}}{\pgfqpoint{4.000616in}{3.984333in}}{\pgfqpoint{4.011666in}{3.984333in}}%
\pgfpathclose%
\pgfusepath{stroke,fill}%
\end{pgfscope}%
\begin{pgfscope}%
\pgfpathrectangle{\pgfqpoint{0.800000in}{0.528000in}}{\pgfqpoint{4.960000in}{3.696000in}}%
\pgfusepath{clip}%
\pgfsetbuttcap%
\pgfsetroundjoin%
\definecolor{currentfill}{rgb}{0.000000,0.000000,0.000000}%
\pgfsetfillcolor{currentfill}%
\pgfsetlinewidth{1.003750pt}%
\definecolor{currentstroke}{rgb}{0.000000,0.000000,0.000000}%
\pgfsetstrokecolor{currentstroke}%
\pgfsetdash{}{0pt}%
\pgfpathmoveto{\pgfqpoint{4.011666in}{3.984333in}}%
\pgfpathcurveto{\pgfqpoint{4.022716in}{3.984333in}}{\pgfqpoint{4.033315in}{3.988724in}}{\pgfqpoint{4.041128in}{3.996537in}}%
\pgfpathcurveto{\pgfqpoint{4.048942in}{4.004351in}}{\pgfqpoint{4.053332in}{4.014950in}}{\pgfqpoint{4.053332in}{4.026000in}}%
\pgfpathcurveto{\pgfqpoint{4.053332in}{4.037050in}}{\pgfqpoint{4.048942in}{4.047649in}}{\pgfqpoint{4.041128in}{4.055463in}}%
\pgfpathcurveto{\pgfqpoint{4.033315in}{4.063276in}}{\pgfqpoint{4.022716in}{4.067667in}}{\pgfqpoint{4.011666in}{4.067667in}}%
\pgfpathcurveto{\pgfqpoint{4.000616in}{4.067667in}}{\pgfqpoint{3.990016in}{4.063276in}}{\pgfqpoint{3.982203in}{4.055463in}}%
\pgfpathcurveto{\pgfqpoint{3.974389in}{4.047649in}}{\pgfqpoint{3.969999in}{4.037050in}}{\pgfqpoint{3.969999in}{4.026000in}}%
\pgfpathcurveto{\pgfqpoint{3.969999in}{4.014950in}}{\pgfqpoint{3.974389in}{4.004351in}}{\pgfqpoint{3.982203in}{3.996537in}}%
\pgfpathcurveto{\pgfqpoint{3.990016in}{3.988724in}}{\pgfqpoint{4.000616in}{3.984333in}}{\pgfqpoint{4.011666in}{3.984333in}}%
\pgfpathclose%
\pgfusepath{stroke,fill}%
\end{pgfscope}%
\begin{pgfscope}%
\pgfpathrectangle{\pgfqpoint{0.800000in}{0.528000in}}{\pgfqpoint{4.960000in}{3.696000in}}%
\pgfusepath{clip}%
\pgfsetbuttcap%
\pgfsetroundjoin%
\definecolor{currentfill}{rgb}{0.000000,0.000000,0.000000}%
\pgfsetfillcolor{currentfill}%
\pgfsetlinewidth{1.003750pt}%
\definecolor{currentstroke}{rgb}{0.000000,0.000000,0.000000}%
\pgfsetstrokecolor{currentstroke}%
\pgfsetdash{}{0pt}%
\pgfpathmoveto{\pgfqpoint{4.011666in}{3.984333in}}%
\pgfpathcurveto{\pgfqpoint{4.022716in}{3.984333in}}{\pgfqpoint{4.033315in}{3.988724in}}{\pgfqpoint{4.041128in}{3.996537in}}%
\pgfpathcurveto{\pgfqpoint{4.048942in}{4.004351in}}{\pgfqpoint{4.053332in}{4.014950in}}{\pgfqpoint{4.053332in}{4.026000in}}%
\pgfpathcurveto{\pgfqpoint{4.053332in}{4.037050in}}{\pgfqpoint{4.048942in}{4.047649in}}{\pgfqpoint{4.041128in}{4.055463in}}%
\pgfpathcurveto{\pgfqpoint{4.033315in}{4.063276in}}{\pgfqpoint{4.022716in}{4.067667in}}{\pgfqpoint{4.011666in}{4.067667in}}%
\pgfpathcurveto{\pgfqpoint{4.000616in}{4.067667in}}{\pgfqpoint{3.990016in}{4.063276in}}{\pgfqpoint{3.982203in}{4.055463in}}%
\pgfpathcurveto{\pgfqpoint{3.974389in}{4.047649in}}{\pgfqpoint{3.969999in}{4.037050in}}{\pgfqpoint{3.969999in}{4.026000in}}%
\pgfpathcurveto{\pgfqpoint{3.969999in}{4.014950in}}{\pgfqpoint{3.974389in}{4.004351in}}{\pgfqpoint{3.982203in}{3.996537in}}%
\pgfpathcurveto{\pgfqpoint{3.990016in}{3.988724in}}{\pgfqpoint{4.000616in}{3.984333in}}{\pgfqpoint{4.011666in}{3.984333in}}%
\pgfpathclose%
\pgfusepath{stroke,fill}%
\end{pgfscope}%
\begin{pgfscope}%
\pgfpathrectangle{\pgfqpoint{0.800000in}{0.528000in}}{\pgfqpoint{4.960000in}{3.696000in}}%
\pgfusepath{clip}%
\pgfsetbuttcap%
\pgfsetroundjoin%
\definecolor{currentfill}{rgb}{0.000000,0.000000,0.000000}%
\pgfsetfillcolor{currentfill}%
\pgfsetlinewidth{1.003750pt}%
\definecolor{currentstroke}{rgb}{0.000000,0.000000,0.000000}%
\pgfsetstrokecolor{currentstroke}%
\pgfsetdash{}{0pt}%
\pgfpathmoveto{\pgfqpoint{4.011666in}{3.984333in}}%
\pgfpathcurveto{\pgfqpoint{4.022716in}{3.984333in}}{\pgfqpoint{4.033315in}{3.988724in}}{\pgfqpoint{4.041128in}{3.996537in}}%
\pgfpathcurveto{\pgfqpoint{4.048942in}{4.004351in}}{\pgfqpoint{4.053332in}{4.014950in}}{\pgfqpoint{4.053332in}{4.026000in}}%
\pgfpathcurveto{\pgfqpoint{4.053332in}{4.037050in}}{\pgfqpoint{4.048942in}{4.047649in}}{\pgfqpoint{4.041128in}{4.055463in}}%
\pgfpathcurveto{\pgfqpoint{4.033315in}{4.063276in}}{\pgfqpoint{4.022716in}{4.067667in}}{\pgfqpoint{4.011666in}{4.067667in}}%
\pgfpathcurveto{\pgfqpoint{4.000616in}{4.067667in}}{\pgfqpoint{3.990016in}{4.063276in}}{\pgfqpoint{3.982203in}{4.055463in}}%
\pgfpathcurveto{\pgfqpoint{3.974389in}{4.047649in}}{\pgfqpoint{3.969999in}{4.037050in}}{\pgfqpoint{3.969999in}{4.026000in}}%
\pgfpathcurveto{\pgfqpoint{3.969999in}{4.014950in}}{\pgfqpoint{3.974389in}{4.004351in}}{\pgfqpoint{3.982203in}{3.996537in}}%
\pgfpathcurveto{\pgfqpoint{3.990016in}{3.988724in}}{\pgfqpoint{4.000616in}{3.984333in}}{\pgfqpoint{4.011666in}{3.984333in}}%
\pgfpathclose%
\pgfusepath{stroke,fill}%
\end{pgfscope}%
\begin{pgfscope}%
\pgfpathrectangle{\pgfqpoint{0.800000in}{0.528000in}}{\pgfqpoint{4.960000in}{3.696000in}}%
\pgfusepath{clip}%
\pgfsetbuttcap%
\pgfsetroundjoin%
\definecolor{currentfill}{rgb}{0.000000,0.000000,0.000000}%
\pgfsetfillcolor{currentfill}%
\pgfsetlinewidth{1.003750pt}%
\definecolor{currentstroke}{rgb}{0.000000,0.000000,0.000000}%
\pgfsetstrokecolor{currentstroke}%
\pgfsetdash{}{0pt}%
\pgfpathmoveto{\pgfqpoint{4.011666in}{3.984333in}}%
\pgfpathcurveto{\pgfqpoint{4.022716in}{3.984333in}}{\pgfqpoint{4.033315in}{3.988724in}}{\pgfqpoint{4.041128in}{3.996537in}}%
\pgfpathcurveto{\pgfqpoint{4.048942in}{4.004351in}}{\pgfqpoint{4.053332in}{4.014950in}}{\pgfqpoint{4.053332in}{4.026000in}}%
\pgfpathcurveto{\pgfqpoint{4.053332in}{4.037050in}}{\pgfqpoint{4.048942in}{4.047649in}}{\pgfqpoint{4.041128in}{4.055463in}}%
\pgfpathcurveto{\pgfqpoint{4.033315in}{4.063276in}}{\pgfqpoint{4.022716in}{4.067667in}}{\pgfqpoint{4.011666in}{4.067667in}}%
\pgfpathcurveto{\pgfqpoint{4.000616in}{4.067667in}}{\pgfqpoint{3.990016in}{4.063276in}}{\pgfqpoint{3.982203in}{4.055463in}}%
\pgfpathcurveto{\pgfqpoint{3.974389in}{4.047649in}}{\pgfqpoint{3.969999in}{4.037050in}}{\pgfqpoint{3.969999in}{4.026000in}}%
\pgfpathcurveto{\pgfqpoint{3.969999in}{4.014950in}}{\pgfqpoint{3.974389in}{4.004351in}}{\pgfqpoint{3.982203in}{3.996537in}}%
\pgfpathcurveto{\pgfqpoint{3.990016in}{3.988724in}}{\pgfqpoint{4.000616in}{3.984333in}}{\pgfqpoint{4.011666in}{3.984333in}}%
\pgfpathclose%
\pgfusepath{stroke,fill}%
\end{pgfscope}%
\begin{pgfscope}%
\pgfpathrectangle{\pgfqpoint{0.800000in}{0.528000in}}{\pgfqpoint{4.960000in}{3.696000in}}%
\pgfusepath{clip}%
\pgfsetbuttcap%
\pgfsetroundjoin%
\definecolor{currentfill}{rgb}{0.000000,0.000000,0.000000}%
\pgfsetfillcolor{currentfill}%
\pgfsetlinewidth{1.003750pt}%
\definecolor{currentstroke}{rgb}{0.000000,0.000000,0.000000}%
\pgfsetstrokecolor{currentstroke}%
\pgfsetdash{}{0pt}%
\pgfpathmoveto{\pgfqpoint{4.011666in}{3.984333in}}%
\pgfpathcurveto{\pgfqpoint{4.022716in}{3.984333in}}{\pgfqpoint{4.033315in}{3.988724in}}{\pgfqpoint{4.041128in}{3.996537in}}%
\pgfpathcurveto{\pgfqpoint{4.048942in}{4.004351in}}{\pgfqpoint{4.053332in}{4.014950in}}{\pgfqpoint{4.053332in}{4.026000in}}%
\pgfpathcurveto{\pgfqpoint{4.053332in}{4.037050in}}{\pgfqpoint{4.048942in}{4.047649in}}{\pgfqpoint{4.041128in}{4.055463in}}%
\pgfpathcurveto{\pgfqpoint{4.033315in}{4.063276in}}{\pgfqpoint{4.022716in}{4.067667in}}{\pgfqpoint{4.011666in}{4.067667in}}%
\pgfpathcurveto{\pgfqpoint{4.000616in}{4.067667in}}{\pgfqpoint{3.990016in}{4.063276in}}{\pgfqpoint{3.982203in}{4.055463in}}%
\pgfpathcurveto{\pgfqpoint{3.974389in}{4.047649in}}{\pgfqpoint{3.969999in}{4.037050in}}{\pgfqpoint{3.969999in}{4.026000in}}%
\pgfpathcurveto{\pgfqpoint{3.969999in}{4.014950in}}{\pgfqpoint{3.974389in}{4.004351in}}{\pgfqpoint{3.982203in}{3.996537in}}%
\pgfpathcurveto{\pgfqpoint{3.990016in}{3.988724in}}{\pgfqpoint{4.000616in}{3.984333in}}{\pgfqpoint{4.011666in}{3.984333in}}%
\pgfpathclose%
\pgfusepath{stroke,fill}%
\end{pgfscope}%
\begin{pgfscope}%
\pgfpathrectangle{\pgfqpoint{0.800000in}{0.528000in}}{\pgfqpoint{4.960000in}{3.696000in}}%
\pgfusepath{clip}%
\pgfsetbuttcap%
\pgfsetroundjoin%
\definecolor{currentfill}{rgb}{0.000000,0.000000,0.000000}%
\pgfsetfillcolor{currentfill}%
\pgfsetlinewidth{1.003750pt}%
\definecolor{currentstroke}{rgb}{0.000000,0.000000,0.000000}%
\pgfsetstrokecolor{currentstroke}%
\pgfsetdash{}{0pt}%
\pgfpathmoveto{\pgfqpoint{4.011666in}{3.984333in}}%
\pgfpathcurveto{\pgfqpoint{4.022716in}{3.984333in}}{\pgfqpoint{4.033315in}{3.988724in}}{\pgfqpoint{4.041128in}{3.996537in}}%
\pgfpathcurveto{\pgfqpoint{4.048942in}{4.004351in}}{\pgfqpoint{4.053332in}{4.014950in}}{\pgfqpoint{4.053332in}{4.026000in}}%
\pgfpathcurveto{\pgfqpoint{4.053332in}{4.037050in}}{\pgfqpoint{4.048942in}{4.047649in}}{\pgfqpoint{4.041128in}{4.055463in}}%
\pgfpathcurveto{\pgfqpoint{4.033315in}{4.063276in}}{\pgfqpoint{4.022716in}{4.067667in}}{\pgfqpoint{4.011666in}{4.067667in}}%
\pgfpathcurveto{\pgfqpoint{4.000616in}{4.067667in}}{\pgfqpoint{3.990016in}{4.063276in}}{\pgfqpoint{3.982203in}{4.055463in}}%
\pgfpathcurveto{\pgfqpoint{3.974389in}{4.047649in}}{\pgfqpoint{3.969999in}{4.037050in}}{\pgfqpoint{3.969999in}{4.026000in}}%
\pgfpathcurveto{\pgfqpoint{3.969999in}{4.014950in}}{\pgfqpoint{3.974389in}{4.004351in}}{\pgfqpoint{3.982203in}{3.996537in}}%
\pgfpathcurveto{\pgfqpoint{3.990016in}{3.988724in}}{\pgfqpoint{4.000616in}{3.984333in}}{\pgfqpoint{4.011666in}{3.984333in}}%
\pgfpathclose%
\pgfusepath{stroke,fill}%
\end{pgfscope}%
\begin{pgfscope}%
\pgfpathrectangle{\pgfqpoint{0.800000in}{0.528000in}}{\pgfqpoint{4.960000in}{3.696000in}}%
\pgfusepath{clip}%
\pgfsetbuttcap%
\pgfsetroundjoin%
\definecolor{currentfill}{rgb}{0.000000,0.000000,0.000000}%
\pgfsetfillcolor{currentfill}%
\pgfsetlinewidth{1.003750pt}%
\definecolor{currentstroke}{rgb}{0.000000,0.000000,0.000000}%
\pgfsetstrokecolor{currentstroke}%
\pgfsetdash{}{0pt}%
\pgfpathmoveto{\pgfqpoint{4.011666in}{3.984333in}}%
\pgfpathcurveto{\pgfqpoint{4.022716in}{3.984333in}}{\pgfqpoint{4.033315in}{3.988724in}}{\pgfqpoint{4.041128in}{3.996537in}}%
\pgfpathcurveto{\pgfqpoint{4.048942in}{4.004351in}}{\pgfqpoint{4.053332in}{4.014950in}}{\pgfqpoint{4.053332in}{4.026000in}}%
\pgfpathcurveto{\pgfqpoint{4.053332in}{4.037050in}}{\pgfqpoint{4.048942in}{4.047649in}}{\pgfqpoint{4.041128in}{4.055463in}}%
\pgfpathcurveto{\pgfqpoint{4.033315in}{4.063276in}}{\pgfqpoint{4.022716in}{4.067667in}}{\pgfqpoint{4.011666in}{4.067667in}}%
\pgfpathcurveto{\pgfqpoint{4.000616in}{4.067667in}}{\pgfqpoint{3.990016in}{4.063276in}}{\pgfqpoint{3.982203in}{4.055463in}}%
\pgfpathcurveto{\pgfqpoint{3.974389in}{4.047649in}}{\pgfqpoint{3.969999in}{4.037050in}}{\pgfqpoint{3.969999in}{4.026000in}}%
\pgfpathcurveto{\pgfqpoint{3.969999in}{4.014950in}}{\pgfqpoint{3.974389in}{4.004351in}}{\pgfqpoint{3.982203in}{3.996537in}}%
\pgfpathcurveto{\pgfqpoint{3.990016in}{3.988724in}}{\pgfqpoint{4.000616in}{3.984333in}}{\pgfqpoint{4.011666in}{3.984333in}}%
\pgfpathclose%
\pgfusepath{stroke,fill}%
\end{pgfscope}%
\begin{pgfscope}%
\pgfpathrectangle{\pgfqpoint{0.800000in}{0.528000in}}{\pgfqpoint{4.960000in}{3.696000in}}%
\pgfusepath{clip}%
\pgfsetbuttcap%
\pgfsetroundjoin%
\definecolor{currentfill}{rgb}{0.000000,0.000000,0.000000}%
\pgfsetfillcolor{currentfill}%
\pgfsetlinewidth{1.003750pt}%
\definecolor{currentstroke}{rgb}{0.000000,0.000000,0.000000}%
\pgfsetstrokecolor{currentstroke}%
\pgfsetdash{}{0pt}%
\pgfpathmoveto{\pgfqpoint{4.011666in}{3.984333in}}%
\pgfpathcurveto{\pgfqpoint{4.022716in}{3.984333in}}{\pgfqpoint{4.033315in}{3.988724in}}{\pgfqpoint{4.041128in}{3.996537in}}%
\pgfpathcurveto{\pgfqpoint{4.048942in}{4.004351in}}{\pgfqpoint{4.053332in}{4.014950in}}{\pgfqpoint{4.053332in}{4.026000in}}%
\pgfpathcurveto{\pgfqpoint{4.053332in}{4.037050in}}{\pgfqpoint{4.048942in}{4.047649in}}{\pgfqpoint{4.041128in}{4.055463in}}%
\pgfpathcurveto{\pgfqpoint{4.033315in}{4.063276in}}{\pgfqpoint{4.022716in}{4.067667in}}{\pgfqpoint{4.011666in}{4.067667in}}%
\pgfpathcurveto{\pgfqpoint{4.000616in}{4.067667in}}{\pgfqpoint{3.990016in}{4.063276in}}{\pgfqpoint{3.982203in}{4.055463in}}%
\pgfpathcurveto{\pgfqpoint{3.974389in}{4.047649in}}{\pgfqpoint{3.969999in}{4.037050in}}{\pgfqpoint{3.969999in}{4.026000in}}%
\pgfpathcurveto{\pgfqpoint{3.969999in}{4.014950in}}{\pgfqpoint{3.974389in}{4.004351in}}{\pgfqpoint{3.982203in}{3.996537in}}%
\pgfpathcurveto{\pgfqpoint{3.990016in}{3.988724in}}{\pgfqpoint{4.000616in}{3.984333in}}{\pgfqpoint{4.011666in}{3.984333in}}%
\pgfpathclose%
\pgfusepath{stroke,fill}%
\end{pgfscope}%
\begin{pgfscope}%
\pgfpathrectangle{\pgfqpoint{0.800000in}{0.528000in}}{\pgfqpoint{4.960000in}{3.696000in}}%
\pgfusepath{clip}%
\pgfsetbuttcap%
\pgfsetroundjoin%
\definecolor{currentfill}{rgb}{0.000000,0.000000,0.000000}%
\pgfsetfillcolor{currentfill}%
\pgfsetlinewidth{1.003750pt}%
\definecolor{currentstroke}{rgb}{0.000000,0.000000,0.000000}%
\pgfsetstrokecolor{currentstroke}%
\pgfsetdash{}{0pt}%
\pgfpathmoveto{\pgfqpoint{4.011666in}{3.984333in}}%
\pgfpathcurveto{\pgfqpoint{4.022716in}{3.984333in}}{\pgfqpoint{4.033315in}{3.988724in}}{\pgfqpoint{4.041128in}{3.996537in}}%
\pgfpathcurveto{\pgfqpoint{4.048942in}{4.004351in}}{\pgfqpoint{4.053332in}{4.014950in}}{\pgfqpoint{4.053332in}{4.026000in}}%
\pgfpathcurveto{\pgfqpoint{4.053332in}{4.037050in}}{\pgfqpoint{4.048942in}{4.047649in}}{\pgfqpoint{4.041128in}{4.055463in}}%
\pgfpathcurveto{\pgfqpoint{4.033315in}{4.063276in}}{\pgfqpoint{4.022716in}{4.067667in}}{\pgfqpoint{4.011666in}{4.067667in}}%
\pgfpathcurveto{\pgfqpoint{4.000616in}{4.067667in}}{\pgfqpoint{3.990016in}{4.063276in}}{\pgfqpoint{3.982203in}{4.055463in}}%
\pgfpathcurveto{\pgfqpoint{3.974389in}{4.047649in}}{\pgfqpoint{3.969999in}{4.037050in}}{\pgfqpoint{3.969999in}{4.026000in}}%
\pgfpathcurveto{\pgfqpoint{3.969999in}{4.014950in}}{\pgfqpoint{3.974389in}{4.004351in}}{\pgfqpoint{3.982203in}{3.996537in}}%
\pgfpathcurveto{\pgfqpoint{3.990016in}{3.988724in}}{\pgfqpoint{4.000616in}{3.984333in}}{\pgfqpoint{4.011666in}{3.984333in}}%
\pgfpathclose%
\pgfusepath{stroke,fill}%
\end{pgfscope}%
\begin{pgfscope}%
\pgfpathrectangle{\pgfqpoint{0.800000in}{0.528000in}}{\pgfqpoint{4.960000in}{3.696000in}}%
\pgfusepath{clip}%
\pgfsetbuttcap%
\pgfsetroundjoin%
\definecolor{currentfill}{rgb}{0.000000,0.000000,0.000000}%
\pgfsetfillcolor{currentfill}%
\pgfsetlinewidth{1.003750pt}%
\definecolor{currentstroke}{rgb}{0.000000,0.000000,0.000000}%
\pgfsetstrokecolor{currentstroke}%
\pgfsetdash{}{0pt}%
\pgfpathmoveto{\pgfqpoint{4.011666in}{0.684333in}}%
\pgfpathcurveto{\pgfqpoint{4.022716in}{0.684333in}}{\pgfqpoint{4.033315in}{0.688724in}}{\pgfqpoint{4.041128in}{0.696537in}}%
\pgfpathcurveto{\pgfqpoint{4.048942in}{0.704351in}}{\pgfqpoint{4.053332in}{0.714950in}}{\pgfqpoint{4.053332in}{0.726000in}}%
\pgfpathcurveto{\pgfqpoint{4.053332in}{0.737050in}}{\pgfqpoint{4.048942in}{0.747649in}}{\pgfqpoint{4.041128in}{0.755463in}}%
\pgfpathcurveto{\pgfqpoint{4.033315in}{0.763276in}}{\pgfqpoint{4.022716in}{0.767667in}}{\pgfqpoint{4.011666in}{0.767667in}}%
\pgfpathcurveto{\pgfqpoint{4.000616in}{0.767667in}}{\pgfqpoint{3.990016in}{0.763276in}}{\pgfqpoint{3.982203in}{0.755463in}}%
\pgfpathcurveto{\pgfqpoint{3.974389in}{0.747649in}}{\pgfqpoint{3.969999in}{0.737050in}}{\pgfqpoint{3.969999in}{0.726000in}}%
\pgfpathcurveto{\pgfqpoint{3.969999in}{0.714950in}}{\pgfqpoint{3.974389in}{0.704351in}}{\pgfqpoint{3.982203in}{0.696537in}}%
\pgfpathcurveto{\pgfqpoint{3.990016in}{0.688724in}}{\pgfqpoint{4.000616in}{0.684333in}}{\pgfqpoint{4.011666in}{0.684333in}}%
\pgfpathclose%
\pgfusepath{stroke,fill}%
\end{pgfscope}%
\begin{pgfscope}%
\pgfpathrectangle{\pgfqpoint{0.800000in}{0.528000in}}{\pgfqpoint{4.960000in}{3.696000in}}%
\pgfusepath{clip}%
\pgfsetbuttcap%
\pgfsetroundjoin%
\definecolor{currentfill}{rgb}{0.000000,0.000000,0.000000}%
\pgfsetfillcolor{currentfill}%
\pgfsetlinewidth{1.003750pt}%
\definecolor{currentstroke}{rgb}{0.000000,0.000000,0.000000}%
\pgfsetstrokecolor{currentstroke}%
\pgfsetdash{}{0pt}%
\pgfpathmoveto{\pgfqpoint{4.011666in}{3.984333in}}%
\pgfpathcurveto{\pgfqpoint{4.022716in}{3.984333in}}{\pgfqpoint{4.033315in}{3.988724in}}{\pgfqpoint{4.041128in}{3.996537in}}%
\pgfpathcurveto{\pgfqpoint{4.048942in}{4.004351in}}{\pgfqpoint{4.053332in}{4.014950in}}{\pgfqpoint{4.053332in}{4.026000in}}%
\pgfpathcurveto{\pgfqpoint{4.053332in}{4.037050in}}{\pgfqpoint{4.048942in}{4.047649in}}{\pgfqpoint{4.041128in}{4.055463in}}%
\pgfpathcurveto{\pgfqpoint{4.033315in}{4.063276in}}{\pgfqpoint{4.022716in}{4.067667in}}{\pgfqpoint{4.011666in}{4.067667in}}%
\pgfpathcurveto{\pgfqpoint{4.000616in}{4.067667in}}{\pgfqpoint{3.990016in}{4.063276in}}{\pgfqpoint{3.982203in}{4.055463in}}%
\pgfpathcurveto{\pgfqpoint{3.974389in}{4.047649in}}{\pgfqpoint{3.969999in}{4.037050in}}{\pgfqpoint{3.969999in}{4.026000in}}%
\pgfpathcurveto{\pgfqpoint{3.969999in}{4.014950in}}{\pgfqpoint{3.974389in}{4.004351in}}{\pgfqpoint{3.982203in}{3.996537in}}%
\pgfpathcurveto{\pgfqpoint{3.990016in}{3.988724in}}{\pgfqpoint{4.000616in}{3.984333in}}{\pgfqpoint{4.011666in}{3.984333in}}%
\pgfpathclose%
\pgfusepath{stroke,fill}%
\end{pgfscope}%
\begin{pgfscope}%
\pgfpathrectangle{\pgfqpoint{0.800000in}{0.528000in}}{\pgfqpoint{4.960000in}{3.696000in}}%
\pgfusepath{clip}%
\pgfsetbuttcap%
\pgfsetroundjoin%
\definecolor{currentfill}{rgb}{0.000000,0.000000,0.000000}%
\pgfsetfillcolor{currentfill}%
\pgfsetlinewidth{1.003750pt}%
\definecolor{currentstroke}{rgb}{0.000000,0.000000,0.000000}%
\pgfsetstrokecolor{currentstroke}%
\pgfsetdash{}{0pt}%
\pgfpathmoveto{\pgfqpoint{4.011666in}{3.984333in}}%
\pgfpathcurveto{\pgfqpoint{4.022716in}{3.984333in}}{\pgfqpoint{4.033315in}{3.988724in}}{\pgfqpoint{4.041128in}{3.996537in}}%
\pgfpathcurveto{\pgfqpoint{4.048942in}{4.004351in}}{\pgfqpoint{4.053332in}{4.014950in}}{\pgfqpoint{4.053332in}{4.026000in}}%
\pgfpathcurveto{\pgfqpoint{4.053332in}{4.037050in}}{\pgfqpoint{4.048942in}{4.047649in}}{\pgfqpoint{4.041128in}{4.055463in}}%
\pgfpathcurveto{\pgfqpoint{4.033315in}{4.063276in}}{\pgfqpoint{4.022716in}{4.067667in}}{\pgfqpoint{4.011666in}{4.067667in}}%
\pgfpathcurveto{\pgfqpoint{4.000616in}{4.067667in}}{\pgfqpoint{3.990016in}{4.063276in}}{\pgfqpoint{3.982203in}{4.055463in}}%
\pgfpathcurveto{\pgfqpoint{3.974389in}{4.047649in}}{\pgfqpoint{3.969999in}{4.037050in}}{\pgfqpoint{3.969999in}{4.026000in}}%
\pgfpathcurveto{\pgfqpoint{3.969999in}{4.014950in}}{\pgfqpoint{3.974389in}{4.004351in}}{\pgfqpoint{3.982203in}{3.996537in}}%
\pgfpathcurveto{\pgfqpoint{3.990016in}{3.988724in}}{\pgfqpoint{4.000616in}{3.984333in}}{\pgfqpoint{4.011666in}{3.984333in}}%
\pgfpathclose%
\pgfusepath{stroke,fill}%
\end{pgfscope}%
\begin{pgfscope}%
\pgfpathrectangle{\pgfqpoint{0.800000in}{0.528000in}}{\pgfqpoint{4.960000in}{3.696000in}}%
\pgfusepath{clip}%
\pgfsetbuttcap%
\pgfsetroundjoin%
\definecolor{currentfill}{rgb}{0.000000,0.000000,0.000000}%
\pgfsetfillcolor{currentfill}%
\pgfsetlinewidth{1.003750pt}%
\definecolor{currentstroke}{rgb}{0.000000,0.000000,0.000000}%
\pgfsetstrokecolor{currentstroke}%
\pgfsetdash{}{0pt}%
\pgfpathmoveto{\pgfqpoint{4.011666in}{3.984333in}}%
\pgfpathcurveto{\pgfqpoint{4.022716in}{3.984333in}}{\pgfqpoint{4.033315in}{3.988724in}}{\pgfqpoint{4.041128in}{3.996537in}}%
\pgfpathcurveto{\pgfqpoint{4.048942in}{4.004351in}}{\pgfqpoint{4.053332in}{4.014950in}}{\pgfqpoint{4.053332in}{4.026000in}}%
\pgfpathcurveto{\pgfqpoint{4.053332in}{4.037050in}}{\pgfqpoint{4.048942in}{4.047649in}}{\pgfqpoint{4.041128in}{4.055463in}}%
\pgfpathcurveto{\pgfqpoint{4.033315in}{4.063276in}}{\pgfqpoint{4.022716in}{4.067667in}}{\pgfqpoint{4.011666in}{4.067667in}}%
\pgfpathcurveto{\pgfqpoint{4.000616in}{4.067667in}}{\pgfqpoint{3.990016in}{4.063276in}}{\pgfqpoint{3.982203in}{4.055463in}}%
\pgfpathcurveto{\pgfqpoint{3.974389in}{4.047649in}}{\pgfqpoint{3.969999in}{4.037050in}}{\pgfqpoint{3.969999in}{4.026000in}}%
\pgfpathcurveto{\pgfqpoint{3.969999in}{4.014950in}}{\pgfqpoint{3.974389in}{4.004351in}}{\pgfqpoint{3.982203in}{3.996537in}}%
\pgfpathcurveto{\pgfqpoint{3.990016in}{3.988724in}}{\pgfqpoint{4.000616in}{3.984333in}}{\pgfqpoint{4.011666in}{3.984333in}}%
\pgfpathclose%
\pgfusepath{stroke,fill}%
\end{pgfscope}%
\begin{pgfscope}%
\pgfpathrectangle{\pgfqpoint{0.800000in}{0.528000in}}{\pgfqpoint{4.960000in}{3.696000in}}%
\pgfusepath{clip}%
\pgfsetbuttcap%
\pgfsetroundjoin%
\definecolor{currentfill}{rgb}{0.000000,0.000000,0.000000}%
\pgfsetfillcolor{currentfill}%
\pgfsetlinewidth{1.003750pt}%
\definecolor{currentstroke}{rgb}{0.000000,0.000000,0.000000}%
\pgfsetstrokecolor{currentstroke}%
\pgfsetdash{}{0pt}%
\pgfpathmoveto{\pgfqpoint{4.011666in}{3.984333in}}%
\pgfpathcurveto{\pgfqpoint{4.022716in}{3.984333in}}{\pgfqpoint{4.033315in}{3.988724in}}{\pgfqpoint{4.041128in}{3.996537in}}%
\pgfpathcurveto{\pgfqpoint{4.048942in}{4.004351in}}{\pgfqpoint{4.053332in}{4.014950in}}{\pgfqpoint{4.053332in}{4.026000in}}%
\pgfpathcurveto{\pgfqpoint{4.053332in}{4.037050in}}{\pgfqpoint{4.048942in}{4.047649in}}{\pgfqpoint{4.041128in}{4.055463in}}%
\pgfpathcurveto{\pgfqpoint{4.033315in}{4.063276in}}{\pgfqpoint{4.022716in}{4.067667in}}{\pgfqpoint{4.011666in}{4.067667in}}%
\pgfpathcurveto{\pgfqpoint{4.000616in}{4.067667in}}{\pgfqpoint{3.990016in}{4.063276in}}{\pgfqpoint{3.982203in}{4.055463in}}%
\pgfpathcurveto{\pgfqpoint{3.974389in}{4.047649in}}{\pgfqpoint{3.969999in}{4.037050in}}{\pgfqpoint{3.969999in}{4.026000in}}%
\pgfpathcurveto{\pgfqpoint{3.969999in}{4.014950in}}{\pgfqpoint{3.974389in}{4.004351in}}{\pgfqpoint{3.982203in}{3.996537in}}%
\pgfpathcurveto{\pgfqpoint{3.990016in}{3.988724in}}{\pgfqpoint{4.000616in}{3.984333in}}{\pgfqpoint{4.011666in}{3.984333in}}%
\pgfpathclose%
\pgfusepath{stroke,fill}%
\end{pgfscope}%
\begin{pgfscope}%
\pgfpathrectangle{\pgfqpoint{0.800000in}{0.528000in}}{\pgfqpoint{4.960000in}{3.696000in}}%
\pgfusepath{clip}%
\pgfsetbuttcap%
\pgfsetroundjoin%
\definecolor{currentfill}{rgb}{0.000000,0.000000,0.000000}%
\pgfsetfillcolor{currentfill}%
\pgfsetlinewidth{1.003750pt}%
\definecolor{currentstroke}{rgb}{0.000000,0.000000,0.000000}%
\pgfsetstrokecolor{currentstroke}%
\pgfsetdash{}{0pt}%
\pgfpathmoveto{\pgfqpoint{4.011666in}{3.984333in}}%
\pgfpathcurveto{\pgfqpoint{4.022716in}{3.984333in}}{\pgfqpoint{4.033315in}{3.988724in}}{\pgfqpoint{4.041128in}{3.996537in}}%
\pgfpathcurveto{\pgfqpoint{4.048942in}{4.004351in}}{\pgfqpoint{4.053332in}{4.014950in}}{\pgfqpoint{4.053332in}{4.026000in}}%
\pgfpathcurveto{\pgfqpoint{4.053332in}{4.037050in}}{\pgfqpoint{4.048942in}{4.047649in}}{\pgfqpoint{4.041128in}{4.055463in}}%
\pgfpathcurveto{\pgfqpoint{4.033315in}{4.063276in}}{\pgfqpoint{4.022716in}{4.067667in}}{\pgfqpoint{4.011666in}{4.067667in}}%
\pgfpathcurveto{\pgfqpoint{4.000616in}{4.067667in}}{\pgfqpoint{3.990016in}{4.063276in}}{\pgfqpoint{3.982203in}{4.055463in}}%
\pgfpathcurveto{\pgfqpoint{3.974389in}{4.047649in}}{\pgfqpoint{3.969999in}{4.037050in}}{\pgfqpoint{3.969999in}{4.026000in}}%
\pgfpathcurveto{\pgfqpoint{3.969999in}{4.014950in}}{\pgfqpoint{3.974389in}{4.004351in}}{\pgfqpoint{3.982203in}{3.996537in}}%
\pgfpathcurveto{\pgfqpoint{3.990016in}{3.988724in}}{\pgfqpoint{4.000616in}{3.984333in}}{\pgfqpoint{4.011666in}{3.984333in}}%
\pgfpathclose%
\pgfusepath{stroke,fill}%
\end{pgfscope}%
\begin{pgfscope}%
\pgfpathrectangle{\pgfqpoint{0.800000in}{0.528000in}}{\pgfqpoint{4.960000in}{3.696000in}}%
\pgfusepath{clip}%
\pgfsetbuttcap%
\pgfsetroundjoin%
\definecolor{currentfill}{rgb}{0.000000,0.000000,0.000000}%
\pgfsetfillcolor{currentfill}%
\pgfsetlinewidth{1.003750pt}%
\definecolor{currentstroke}{rgb}{0.000000,0.000000,0.000000}%
\pgfsetstrokecolor{currentstroke}%
\pgfsetdash{}{0pt}%
\pgfpathmoveto{\pgfqpoint{4.011666in}{3.984333in}}%
\pgfpathcurveto{\pgfqpoint{4.022716in}{3.984333in}}{\pgfqpoint{4.033315in}{3.988724in}}{\pgfqpoint{4.041128in}{3.996537in}}%
\pgfpathcurveto{\pgfqpoint{4.048942in}{4.004351in}}{\pgfqpoint{4.053332in}{4.014950in}}{\pgfqpoint{4.053332in}{4.026000in}}%
\pgfpathcurveto{\pgfqpoint{4.053332in}{4.037050in}}{\pgfqpoint{4.048942in}{4.047649in}}{\pgfqpoint{4.041128in}{4.055463in}}%
\pgfpathcurveto{\pgfqpoint{4.033315in}{4.063276in}}{\pgfqpoint{4.022716in}{4.067667in}}{\pgfqpoint{4.011666in}{4.067667in}}%
\pgfpathcurveto{\pgfqpoint{4.000616in}{4.067667in}}{\pgfqpoint{3.990016in}{4.063276in}}{\pgfqpoint{3.982203in}{4.055463in}}%
\pgfpathcurveto{\pgfqpoint{3.974389in}{4.047649in}}{\pgfqpoint{3.969999in}{4.037050in}}{\pgfqpoint{3.969999in}{4.026000in}}%
\pgfpathcurveto{\pgfqpoint{3.969999in}{4.014950in}}{\pgfqpoint{3.974389in}{4.004351in}}{\pgfqpoint{3.982203in}{3.996537in}}%
\pgfpathcurveto{\pgfqpoint{3.990016in}{3.988724in}}{\pgfqpoint{4.000616in}{3.984333in}}{\pgfqpoint{4.011666in}{3.984333in}}%
\pgfpathclose%
\pgfusepath{stroke,fill}%
\end{pgfscope}%
\begin{pgfscope}%
\pgfpathrectangle{\pgfqpoint{0.800000in}{0.528000in}}{\pgfqpoint{4.960000in}{3.696000in}}%
\pgfusepath{clip}%
\pgfsetbuttcap%
\pgfsetroundjoin%
\definecolor{currentfill}{rgb}{0.000000,0.000000,0.000000}%
\pgfsetfillcolor{currentfill}%
\pgfsetlinewidth{1.003750pt}%
\definecolor{currentstroke}{rgb}{0.000000,0.000000,0.000000}%
\pgfsetstrokecolor{currentstroke}%
\pgfsetdash{}{0pt}%
\pgfpathmoveto{\pgfqpoint{4.011666in}{3.984333in}}%
\pgfpathcurveto{\pgfqpoint{4.022716in}{3.984333in}}{\pgfqpoint{4.033315in}{3.988724in}}{\pgfqpoint{4.041128in}{3.996537in}}%
\pgfpathcurveto{\pgfqpoint{4.048942in}{4.004351in}}{\pgfqpoint{4.053332in}{4.014950in}}{\pgfqpoint{4.053332in}{4.026000in}}%
\pgfpathcurveto{\pgfqpoint{4.053332in}{4.037050in}}{\pgfqpoint{4.048942in}{4.047649in}}{\pgfqpoint{4.041128in}{4.055463in}}%
\pgfpathcurveto{\pgfqpoint{4.033315in}{4.063276in}}{\pgfqpoint{4.022716in}{4.067667in}}{\pgfqpoint{4.011666in}{4.067667in}}%
\pgfpathcurveto{\pgfqpoint{4.000616in}{4.067667in}}{\pgfqpoint{3.990016in}{4.063276in}}{\pgfqpoint{3.982203in}{4.055463in}}%
\pgfpathcurveto{\pgfqpoint{3.974389in}{4.047649in}}{\pgfqpoint{3.969999in}{4.037050in}}{\pgfqpoint{3.969999in}{4.026000in}}%
\pgfpathcurveto{\pgfqpoint{3.969999in}{4.014950in}}{\pgfqpoint{3.974389in}{4.004351in}}{\pgfqpoint{3.982203in}{3.996537in}}%
\pgfpathcurveto{\pgfqpoint{3.990016in}{3.988724in}}{\pgfqpoint{4.000616in}{3.984333in}}{\pgfqpoint{4.011666in}{3.984333in}}%
\pgfpathclose%
\pgfusepath{stroke,fill}%
\end{pgfscope}%
\begin{pgfscope}%
\pgfpathrectangle{\pgfqpoint{0.800000in}{0.528000in}}{\pgfqpoint{4.960000in}{3.696000in}}%
\pgfusepath{clip}%
\pgfsetbuttcap%
\pgfsetroundjoin%
\definecolor{currentfill}{rgb}{0.000000,0.000000,0.000000}%
\pgfsetfillcolor{currentfill}%
\pgfsetlinewidth{1.003750pt}%
\definecolor{currentstroke}{rgb}{0.000000,0.000000,0.000000}%
\pgfsetstrokecolor{currentstroke}%
\pgfsetdash{}{0pt}%
\pgfpathmoveto{\pgfqpoint{4.011666in}{3.984333in}}%
\pgfpathcurveto{\pgfqpoint{4.022716in}{3.984333in}}{\pgfqpoint{4.033315in}{3.988724in}}{\pgfqpoint{4.041128in}{3.996537in}}%
\pgfpathcurveto{\pgfqpoint{4.048942in}{4.004351in}}{\pgfqpoint{4.053332in}{4.014950in}}{\pgfqpoint{4.053332in}{4.026000in}}%
\pgfpathcurveto{\pgfqpoint{4.053332in}{4.037050in}}{\pgfqpoint{4.048942in}{4.047649in}}{\pgfqpoint{4.041128in}{4.055463in}}%
\pgfpathcurveto{\pgfqpoint{4.033315in}{4.063276in}}{\pgfqpoint{4.022716in}{4.067667in}}{\pgfqpoint{4.011666in}{4.067667in}}%
\pgfpathcurveto{\pgfqpoint{4.000616in}{4.067667in}}{\pgfqpoint{3.990016in}{4.063276in}}{\pgfqpoint{3.982203in}{4.055463in}}%
\pgfpathcurveto{\pgfqpoint{3.974389in}{4.047649in}}{\pgfqpoint{3.969999in}{4.037050in}}{\pgfqpoint{3.969999in}{4.026000in}}%
\pgfpathcurveto{\pgfqpoint{3.969999in}{4.014950in}}{\pgfqpoint{3.974389in}{4.004351in}}{\pgfqpoint{3.982203in}{3.996537in}}%
\pgfpathcurveto{\pgfqpoint{3.990016in}{3.988724in}}{\pgfqpoint{4.000616in}{3.984333in}}{\pgfqpoint{4.011666in}{3.984333in}}%
\pgfpathclose%
\pgfusepath{stroke,fill}%
\end{pgfscope}%
\begin{pgfscope}%
\pgfpathrectangle{\pgfqpoint{0.800000in}{0.528000in}}{\pgfqpoint{4.960000in}{3.696000in}}%
\pgfusepath{clip}%
\pgfsetbuttcap%
\pgfsetroundjoin%
\definecolor{currentfill}{rgb}{0.000000,0.000000,0.000000}%
\pgfsetfillcolor{currentfill}%
\pgfsetlinewidth{1.003750pt}%
\definecolor{currentstroke}{rgb}{0.000000,0.000000,0.000000}%
\pgfsetstrokecolor{currentstroke}%
\pgfsetdash{}{0pt}%
\pgfpathmoveto{\pgfqpoint{4.011666in}{3.984333in}}%
\pgfpathcurveto{\pgfqpoint{4.022716in}{3.984333in}}{\pgfqpoint{4.033315in}{3.988724in}}{\pgfqpoint{4.041128in}{3.996537in}}%
\pgfpathcurveto{\pgfqpoint{4.048942in}{4.004351in}}{\pgfqpoint{4.053332in}{4.014950in}}{\pgfqpoint{4.053332in}{4.026000in}}%
\pgfpathcurveto{\pgfqpoint{4.053332in}{4.037050in}}{\pgfqpoint{4.048942in}{4.047649in}}{\pgfqpoint{4.041128in}{4.055463in}}%
\pgfpathcurveto{\pgfqpoint{4.033315in}{4.063276in}}{\pgfqpoint{4.022716in}{4.067667in}}{\pgfqpoint{4.011666in}{4.067667in}}%
\pgfpathcurveto{\pgfqpoint{4.000616in}{4.067667in}}{\pgfqpoint{3.990016in}{4.063276in}}{\pgfqpoint{3.982203in}{4.055463in}}%
\pgfpathcurveto{\pgfqpoint{3.974389in}{4.047649in}}{\pgfqpoint{3.969999in}{4.037050in}}{\pgfqpoint{3.969999in}{4.026000in}}%
\pgfpathcurveto{\pgfqpoint{3.969999in}{4.014950in}}{\pgfqpoint{3.974389in}{4.004351in}}{\pgfqpoint{3.982203in}{3.996537in}}%
\pgfpathcurveto{\pgfqpoint{3.990016in}{3.988724in}}{\pgfqpoint{4.000616in}{3.984333in}}{\pgfqpoint{4.011666in}{3.984333in}}%
\pgfpathclose%
\pgfusepath{stroke,fill}%
\end{pgfscope}%
\begin{pgfscope}%
\pgfpathrectangle{\pgfqpoint{0.800000in}{0.528000in}}{\pgfqpoint{4.960000in}{3.696000in}}%
\pgfusepath{clip}%
\pgfsetbuttcap%
\pgfsetroundjoin%
\definecolor{currentfill}{rgb}{0.000000,0.000000,0.000000}%
\pgfsetfillcolor{currentfill}%
\pgfsetlinewidth{1.003750pt}%
\definecolor{currentstroke}{rgb}{0.000000,0.000000,0.000000}%
\pgfsetstrokecolor{currentstroke}%
\pgfsetdash{}{0pt}%
\pgfpathmoveto{\pgfqpoint{4.011666in}{3.984333in}}%
\pgfpathcurveto{\pgfqpoint{4.022716in}{3.984333in}}{\pgfqpoint{4.033315in}{3.988724in}}{\pgfqpoint{4.041128in}{3.996537in}}%
\pgfpathcurveto{\pgfqpoint{4.048942in}{4.004351in}}{\pgfqpoint{4.053332in}{4.014950in}}{\pgfqpoint{4.053332in}{4.026000in}}%
\pgfpathcurveto{\pgfqpoint{4.053332in}{4.037050in}}{\pgfqpoint{4.048942in}{4.047649in}}{\pgfqpoint{4.041128in}{4.055463in}}%
\pgfpathcurveto{\pgfqpoint{4.033315in}{4.063276in}}{\pgfqpoint{4.022716in}{4.067667in}}{\pgfqpoint{4.011666in}{4.067667in}}%
\pgfpathcurveto{\pgfqpoint{4.000616in}{4.067667in}}{\pgfqpoint{3.990016in}{4.063276in}}{\pgfqpoint{3.982203in}{4.055463in}}%
\pgfpathcurveto{\pgfqpoint{3.974389in}{4.047649in}}{\pgfqpoint{3.969999in}{4.037050in}}{\pgfqpoint{3.969999in}{4.026000in}}%
\pgfpathcurveto{\pgfqpoint{3.969999in}{4.014950in}}{\pgfqpoint{3.974389in}{4.004351in}}{\pgfqpoint{3.982203in}{3.996537in}}%
\pgfpathcurveto{\pgfqpoint{3.990016in}{3.988724in}}{\pgfqpoint{4.000616in}{3.984333in}}{\pgfqpoint{4.011666in}{3.984333in}}%
\pgfpathclose%
\pgfusepath{stroke,fill}%
\end{pgfscope}%
\begin{pgfscope}%
\pgfpathrectangle{\pgfqpoint{0.800000in}{0.528000in}}{\pgfqpoint{4.960000in}{3.696000in}}%
\pgfusepath{clip}%
\pgfsetbuttcap%
\pgfsetroundjoin%
\definecolor{currentfill}{rgb}{0.000000,0.000000,0.000000}%
\pgfsetfillcolor{currentfill}%
\pgfsetlinewidth{1.003750pt}%
\definecolor{currentstroke}{rgb}{0.000000,0.000000,0.000000}%
\pgfsetstrokecolor{currentstroke}%
\pgfsetdash{}{0pt}%
\pgfpathmoveto{\pgfqpoint{4.011666in}{3.984333in}}%
\pgfpathcurveto{\pgfqpoint{4.022716in}{3.984333in}}{\pgfqpoint{4.033315in}{3.988724in}}{\pgfqpoint{4.041128in}{3.996537in}}%
\pgfpathcurveto{\pgfqpoint{4.048942in}{4.004351in}}{\pgfqpoint{4.053332in}{4.014950in}}{\pgfqpoint{4.053332in}{4.026000in}}%
\pgfpathcurveto{\pgfqpoint{4.053332in}{4.037050in}}{\pgfqpoint{4.048942in}{4.047649in}}{\pgfqpoint{4.041128in}{4.055463in}}%
\pgfpathcurveto{\pgfqpoint{4.033315in}{4.063276in}}{\pgfqpoint{4.022716in}{4.067667in}}{\pgfqpoint{4.011666in}{4.067667in}}%
\pgfpathcurveto{\pgfqpoint{4.000616in}{4.067667in}}{\pgfqpoint{3.990016in}{4.063276in}}{\pgfqpoint{3.982203in}{4.055463in}}%
\pgfpathcurveto{\pgfqpoint{3.974389in}{4.047649in}}{\pgfqpoint{3.969999in}{4.037050in}}{\pgfqpoint{3.969999in}{4.026000in}}%
\pgfpathcurveto{\pgfqpoint{3.969999in}{4.014950in}}{\pgfqpoint{3.974389in}{4.004351in}}{\pgfqpoint{3.982203in}{3.996537in}}%
\pgfpathcurveto{\pgfqpoint{3.990016in}{3.988724in}}{\pgfqpoint{4.000616in}{3.984333in}}{\pgfqpoint{4.011666in}{3.984333in}}%
\pgfpathclose%
\pgfusepath{stroke,fill}%
\end{pgfscope}%
\begin{pgfscope}%
\pgfpathrectangle{\pgfqpoint{0.800000in}{0.528000in}}{\pgfqpoint{4.960000in}{3.696000in}}%
\pgfusepath{clip}%
\pgfsetbuttcap%
\pgfsetroundjoin%
\definecolor{currentfill}{rgb}{0.000000,0.000000,0.000000}%
\pgfsetfillcolor{currentfill}%
\pgfsetlinewidth{1.003750pt}%
\definecolor{currentstroke}{rgb}{0.000000,0.000000,0.000000}%
\pgfsetstrokecolor{currentstroke}%
\pgfsetdash{}{0pt}%
\pgfpathmoveto{\pgfqpoint{4.011666in}{3.984333in}}%
\pgfpathcurveto{\pgfqpoint{4.022716in}{3.984333in}}{\pgfqpoint{4.033315in}{3.988724in}}{\pgfqpoint{4.041128in}{3.996537in}}%
\pgfpathcurveto{\pgfqpoint{4.048942in}{4.004351in}}{\pgfqpoint{4.053332in}{4.014950in}}{\pgfqpoint{4.053332in}{4.026000in}}%
\pgfpathcurveto{\pgfqpoint{4.053332in}{4.037050in}}{\pgfqpoint{4.048942in}{4.047649in}}{\pgfqpoint{4.041128in}{4.055463in}}%
\pgfpathcurveto{\pgfqpoint{4.033315in}{4.063276in}}{\pgfqpoint{4.022716in}{4.067667in}}{\pgfqpoint{4.011666in}{4.067667in}}%
\pgfpathcurveto{\pgfqpoint{4.000616in}{4.067667in}}{\pgfqpoint{3.990016in}{4.063276in}}{\pgfqpoint{3.982203in}{4.055463in}}%
\pgfpathcurveto{\pgfqpoint{3.974389in}{4.047649in}}{\pgfqpoint{3.969999in}{4.037050in}}{\pgfqpoint{3.969999in}{4.026000in}}%
\pgfpathcurveto{\pgfqpoint{3.969999in}{4.014950in}}{\pgfqpoint{3.974389in}{4.004351in}}{\pgfqpoint{3.982203in}{3.996537in}}%
\pgfpathcurveto{\pgfqpoint{3.990016in}{3.988724in}}{\pgfqpoint{4.000616in}{3.984333in}}{\pgfqpoint{4.011666in}{3.984333in}}%
\pgfpathclose%
\pgfusepath{stroke,fill}%
\end{pgfscope}%
\begin{pgfscope}%
\pgfpathrectangle{\pgfqpoint{0.800000in}{0.528000in}}{\pgfqpoint{4.960000in}{3.696000in}}%
\pgfusepath{clip}%
\pgfsetbuttcap%
\pgfsetroundjoin%
\definecolor{currentfill}{rgb}{0.000000,0.000000,0.000000}%
\pgfsetfillcolor{currentfill}%
\pgfsetlinewidth{1.003750pt}%
\definecolor{currentstroke}{rgb}{0.000000,0.000000,0.000000}%
\pgfsetstrokecolor{currentstroke}%
\pgfsetdash{}{0pt}%
\pgfpathmoveto{\pgfqpoint{4.011666in}{3.984333in}}%
\pgfpathcurveto{\pgfqpoint{4.022716in}{3.984333in}}{\pgfqpoint{4.033315in}{3.988724in}}{\pgfqpoint{4.041128in}{3.996537in}}%
\pgfpathcurveto{\pgfqpoint{4.048942in}{4.004351in}}{\pgfqpoint{4.053332in}{4.014950in}}{\pgfqpoint{4.053332in}{4.026000in}}%
\pgfpathcurveto{\pgfqpoint{4.053332in}{4.037050in}}{\pgfqpoint{4.048942in}{4.047649in}}{\pgfqpoint{4.041128in}{4.055463in}}%
\pgfpathcurveto{\pgfqpoint{4.033315in}{4.063276in}}{\pgfqpoint{4.022716in}{4.067667in}}{\pgfqpoint{4.011666in}{4.067667in}}%
\pgfpathcurveto{\pgfqpoint{4.000616in}{4.067667in}}{\pgfqpoint{3.990016in}{4.063276in}}{\pgfqpoint{3.982203in}{4.055463in}}%
\pgfpathcurveto{\pgfqpoint{3.974389in}{4.047649in}}{\pgfqpoint{3.969999in}{4.037050in}}{\pgfqpoint{3.969999in}{4.026000in}}%
\pgfpathcurveto{\pgfqpoint{3.969999in}{4.014950in}}{\pgfqpoint{3.974389in}{4.004351in}}{\pgfqpoint{3.982203in}{3.996537in}}%
\pgfpathcurveto{\pgfqpoint{3.990016in}{3.988724in}}{\pgfqpoint{4.000616in}{3.984333in}}{\pgfqpoint{4.011666in}{3.984333in}}%
\pgfpathclose%
\pgfusepath{stroke,fill}%
\end{pgfscope}%
\begin{pgfscope}%
\pgfpathrectangle{\pgfqpoint{0.800000in}{0.528000in}}{\pgfqpoint{4.960000in}{3.696000in}}%
\pgfusepath{clip}%
\pgfsetbuttcap%
\pgfsetroundjoin%
\definecolor{currentfill}{rgb}{0.000000,0.000000,0.000000}%
\pgfsetfillcolor{currentfill}%
\pgfsetlinewidth{1.003750pt}%
\definecolor{currentstroke}{rgb}{0.000000,0.000000,0.000000}%
\pgfsetstrokecolor{currentstroke}%
\pgfsetdash{}{0pt}%
\pgfpathmoveto{\pgfqpoint{4.011666in}{3.984333in}}%
\pgfpathcurveto{\pgfqpoint{4.022716in}{3.984333in}}{\pgfqpoint{4.033315in}{3.988724in}}{\pgfqpoint{4.041128in}{3.996537in}}%
\pgfpathcurveto{\pgfqpoint{4.048942in}{4.004351in}}{\pgfqpoint{4.053332in}{4.014950in}}{\pgfqpoint{4.053332in}{4.026000in}}%
\pgfpathcurveto{\pgfqpoint{4.053332in}{4.037050in}}{\pgfqpoint{4.048942in}{4.047649in}}{\pgfqpoint{4.041128in}{4.055463in}}%
\pgfpathcurveto{\pgfqpoint{4.033315in}{4.063276in}}{\pgfqpoint{4.022716in}{4.067667in}}{\pgfqpoint{4.011666in}{4.067667in}}%
\pgfpathcurveto{\pgfqpoint{4.000616in}{4.067667in}}{\pgfqpoint{3.990016in}{4.063276in}}{\pgfqpoint{3.982203in}{4.055463in}}%
\pgfpathcurveto{\pgfqpoint{3.974389in}{4.047649in}}{\pgfqpoint{3.969999in}{4.037050in}}{\pgfqpoint{3.969999in}{4.026000in}}%
\pgfpathcurveto{\pgfqpoint{3.969999in}{4.014950in}}{\pgfqpoint{3.974389in}{4.004351in}}{\pgfqpoint{3.982203in}{3.996537in}}%
\pgfpathcurveto{\pgfqpoint{3.990016in}{3.988724in}}{\pgfqpoint{4.000616in}{3.984333in}}{\pgfqpoint{4.011666in}{3.984333in}}%
\pgfpathclose%
\pgfusepath{stroke,fill}%
\end{pgfscope}%
\begin{pgfscope}%
\pgfpathrectangle{\pgfqpoint{0.800000in}{0.528000in}}{\pgfqpoint{4.960000in}{3.696000in}}%
\pgfusepath{clip}%
\pgfsetbuttcap%
\pgfsetroundjoin%
\definecolor{currentfill}{rgb}{0.000000,0.000000,0.000000}%
\pgfsetfillcolor{currentfill}%
\pgfsetlinewidth{1.003750pt}%
\definecolor{currentstroke}{rgb}{0.000000,0.000000,0.000000}%
\pgfsetstrokecolor{currentstroke}%
\pgfsetdash{}{0pt}%
\pgfpathmoveto{\pgfqpoint{4.011666in}{3.984333in}}%
\pgfpathcurveto{\pgfqpoint{4.022716in}{3.984333in}}{\pgfqpoint{4.033315in}{3.988724in}}{\pgfqpoint{4.041128in}{3.996537in}}%
\pgfpathcurveto{\pgfqpoint{4.048942in}{4.004351in}}{\pgfqpoint{4.053332in}{4.014950in}}{\pgfqpoint{4.053332in}{4.026000in}}%
\pgfpathcurveto{\pgfqpoint{4.053332in}{4.037050in}}{\pgfqpoint{4.048942in}{4.047649in}}{\pgfqpoint{4.041128in}{4.055463in}}%
\pgfpathcurveto{\pgfqpoint{4.033315in}{4.063276in}}{\pgfqpoint{4.022716in}{4.067667in}}{\pgfqpoint{4.011666in}{4.067667in}}%
\pgfpathcurveto{\pgfqpoint{4.000616in}{4.067667in}}{\pgfqpoint{3.990016in}{4.063276in}}{\pgfqpoint{3.982203in}{4.055463in}}%
\pgfpathcurveto{\pgfqpoint{3.974389in}{4.047649in}}{\pgfqpoint{3.969999in}{4.037050in}}{\pgfqpoint{3.969999in}{4.026000in}}%
\pgfpathcurveto{\pgfqpoint{3.969999in}{4.014950in}}{\pgfqpoint{3.974389in}{4.004351in}}{\pgfqpoint{3.982203in}{3.996537in}}%
\pgfpathcurveto{\pgfqpoint{3.990016in}{3.988724in}}{\pgfqpoint{4.000616in}{3.984333in}}{\pgfqpoint{4.011666in}{3.984333in}}%
\pgfpathclose%
\pgfusepath{stroke,fill}%
\end{pgfscope}%
\begin{pgfscope}%
\pgfpathrectangle{\pgfqpoint{0.800000in}{0.528000in}}{\pgfqpoint{4.960000in}{3.696000in}}%
\pgfusepath{clip}%
\pgfsetbuttcap%
\pgfsetroundjoin%
\definecolor{currentfill}{rgb}{0.000000,0.000000,0.000000}%
\pgfsetfillcolor{currentfill}%
\pgfsetlinewidth{1.003750pt}%
\definecolor{currentstroke}{rgb}{0.000000,0.000000,0.000000}%
\pgfsetstrokecolor{currentstroke}%
\pgfsetdash{}{0pt}%
\pgfpathmoveto{\pgfqpoint{4.011666in}{3.984333in}}%
\pgfpathcurveto{\pgfqpoint{4.022716in}{3.984333in}}{\pgfqpoint{4.033315in}{3.988724in}}{\pgfqpoint{4.041128in}{3.996537in}}%
\pgfpathcurveto{\pgfqpoint{4.048942in}{4.004351in}}{\pgfqpoint{4.053332in}{4.014950in}}{\pgfqpoint{4.053332in}{4.026000in}}%
\pgfpathcurveto{\pgfqpoint{4.053332in}{4.037050in}}{\pgfqpoint{4.048942in}{4.047649in}}{\pgfqpoint{4.041128in}{4.055463in}}%
\pgfpathcurveto{\pgfqpoint{4.033315in}{4.063276in}}{\pgfqpoint{4.022716in}{4.067667in}}{\pgfqpoint{4.011666in}{4.067667in}}%
\pgfpathcurveto{\pgfqpoint{4.000616in}{4.067667in}}{\pgfqpoint{3.990016in}{4.063276in}}{\pgfqpoint{3.982203in}{4.055463in}}%
\pgfpathcurveto{\pgfqpoint{3.974389in}{4.047649in}}{\pgfqpoint{3.969999in}{4.037050in}}{\pgfqpoint{3.969999in}{4.026000in}}%
\pgfpathcurveto{\pgfqpoint{3.969999in}{4.014950in}}{\pgfqpoint{3.974389in}{4.004351in}}{\pgfqpoint{3.982203in}{3.996537in}}%
\pgfpathcurveto{\pgfqpoint{3.990016in}{3.988724in}}{\pgfqpoint{4.000616in}{3.984333in}}{\pgfqpoint{4.011666in}{3.984333in}}%
\pgfpathclose%
\pgfusepath{stroke,fill}%
\end{pgfscope}%
\begin{pgfscope}%
\pgfpathrectangle{\pgfqpoint{0.800000in}{0.528000in}}{\pgfqpoint{4.960000in}{3.696000in}}%
\pgfusepath{clip}%
\pgfsetbuttcap%
\pgfsetroundjoin%
\definecolor{currentfill}{rgb}{0.000000,0.000000,0.000000}%
\pgfsetfillcolor{currentfill}%
\pgfsetlinewidth{1.003750pt}%
\definecolor{currentstroke}{rgb}{0.000000,0.000000,0.000000}%
\pgfsetstrokecolor{currentstroke}%
\pgfsetdash{}{0pt}%
\pgfpathmoveto{\pgfqpoint{4.011666in}{3.984333in}}%
\pgfpathcurveto{\pgfqpoint{4.022716in}{3.984333in}}{\pgfqpoint{4.033315in}{3.988724in}}{\pgfqpoint{4.041128in}{3.996537in}}%
\pgfpathcurveto{\pgfqpoint{4.048942in}{4.004351in}}{\pgfqpoint{4.053332in}{4.014950in}}{\pgfqpoint{4.053332in}{4.026000in}}%
\pgfpathcurveto{\pgfqpoint{4.053332in}{4.037050in}}{\pgfqpoint{4.048942in}{4.047649in}}{\pgfqpoint{4.041128in}{4.055463in}}%
\pgfpathcurveto{\pgfqpoint{4.033315in}{4.063276in}}{\pgfqpoint{4.022716in}{4.067667in}}{\pgfqpoint{4.011666in}{4.067667in}}%
\pgfpathcurveto{\pgfqpoint{4.000616in}{4.067667in}}{\pgfqpoint{3.990016in}{4.063276in}}{\pgfqpoint{3.982203in}{4.055463in}}%
\pgfpathcurveto{\pgfqpoint{3.974389in}{4.047649in}}{\pgfqpoint{3.969999in}{4.037050in}}{\pgfqpoint{3.969999in}{4.026000in}}%
\pgfpathcurveto{\pgfqpoint{3.969999in}{4.014950in}}{\pgfqpoint{3.974389in}{4.004351in}}{\pgfqpoint{3.982203in}{3.996537in}}%
\pgfpathcurveto{\pgfqpoint{3.990016in}{3.988724in}}{\pgfqpoint{4.000616in}{3.984333in}}{\pgfqpoint{4.011666in}{3.984333in}}%
\pgfpathclose%
\pgfusepath{stroke,fill}%
\end{pgfscope}%
\begin{pgfscope}%
\pgfpathrectangle{\pgfqpoint{0.800000in}{0.528000in}}{\pgfqpoint{4.960000in}{3.696000in}}%
\pgfusepath{clip}%
\pgfsetbuttcap%
\pgfsetroundjoin%
\definecolor{currentfill}{rgb}{0.000000,0.000000,0.000000}%
\pgfsetfillcolor{currentfill}%
\pgfsetlinewidth{1.003750pt}%
\definecolor{currentstroke}{rgb}{0.000000,0.000000,0.000000}%
\pgfsetstrokecolor{currentstroke}%
\pgfsetdash{}{0pt}%
\pgfpathmoveto{\pgfqpoint{4.011666in}{3.984333in}}%
\pgfpathcurveto{\pgfqpoint{4.022716in}{3.984333in}}{\pgfqpoint{4.033315in}{3.988724in}}{\pgfqpoint{4.041128in}{3.996537in}}%
\pgfpathcurveto{\pgfqpoint{4.048942in}{4.004351in}}{\pgfqpoint{4.053332in}{4.014950in}}{\pgfqpoint{4.053332in}{4.026000in}}%
\pgfpathcurveto{\pgfqpoint{4.053332in}{4.037050in}}{\pgfqpoint{4.048942in}{4.047649in}}{\pgfqpoint{4.041128in}{4.055463in}}%
\pgfpathcurveto{\pgfqpoint{4.033315in}{4.063276in}}{\pgfqpoint{4.022716in}{4.067667in}}{\pgfqpoint{4.011666in}{4.067667in}}%
\pgfpathcurveto{\pgfqpoint{4.000616in}{4.067667in}}{\pgfqpoint{3.990016in}{4.063276in}}{\pgfqpoint{3.982203in}{4.055463in}}%
\pgfpathcurveto{\pgfqpoint{3.974389in}{4.047649in}}{\pgfqpoint{3.969999in}{4.037050in}}{\pgfqpoint{3.969999in}{4.026000in}}%
\pgfpathcurveto{\pgfqpoint{3.969999in}{4.014950in}}{\pgfqpoint{3.974389in}{4.004351in}}{\pgfqpoint{3.982203in}{3.996537in}}%
\pgfpathcurveto{\pgfqpoint{3.990016in}{3.988724in}}{\pgfqpoint{4.000616in}{3.984333in}}{\pgfqpoint{4.011666in}{3.984333in}}%
\pgfpathclose%
\pgfusepath{stroke,fill}%
\end{pgfscope}%
\begin{pgfscope}%
\pgfpathrectangle{\pgfqpoint{0.800000in}{0.528000in}}{\pgfqpoint{4.960000in}{3.696000in}}%
\pgfusepath{clip}%
\pgfsetbuttcap%
\pgfsetroundjoin%
\definecolor{currentfill}{rgb}{0.000000,0.000000,0.000000}%
\pgfsetfillcolor{currentfill}%
\pgfsetlinewidth{1.003750pt}%
\definecolor{currentstroke}{rgb}{0.000000,0.000000,0.000000}%
\pgfsetstrokecolor{currentstroke}%
\pgfsetdash{}{0pt}%
\pgfpathmoveto{\pgfqpoint{4.011666in}{3.984333in}}%
\pgfpathcurveto{\pgfqpoint{4.022716in}{3.984333in}}{\pgfqpoint{4.033315in}{3.988724in}}{\pgfqpoint{4.041128in}{3.996537in}}%
\pgfpathcurveto{\pgfqpoint{4.048942in}{4.004351in}}{\pgfqpoint{4.053332in}{4.014950in}}{\pgfqpoint{4.053332in}{4.026000in}}%
\pgfpathcurveto{\pgfqpoint{4.053332in}{4.037050in}}{\pgfqpoint{4.048942in}{4.047649in}}{\pgfqpoint{4.041128in}{4.055463in}}%
\pgfpathcurveto{\pgfqpoint{4.033315in}{4.063276in}}{\pgfqpoint{4.022716in}{4.067667in}}{\pgfqpoint{4.011666in}{4.067667in}}%
\pgfpathcurveto{\pgfqpoint{4.000616in}{4.067667in}}{\pgfqpoint{3.990016in}{4.063276in}}{\pgfqpoint{3.982203in}{4.055463in}}%
\pgfpathcurveto{\pgfqpoint{3.974389in}{4.047649in}}{\pgfqpoint{3.969999in}{4.037050in}}{\pgfqpoint{3.969999in}{4.026000in}}%
\pgfpathcurveto{\pgfqpoint{3.969999in}{4.014950in}}{\pgfqpoint{3.974389in}{4.004351in}}{\pgfqpoint{3.982203in}{3.996537in}}%
\pgfpathcurveto{\pgfqpoint{3.990016in}{3.988724in}}{\pgfqpoint{4.000616in}{3.984333in}}{\pgfqpoint{4.011666in}{3.984333in}}%
\pgfpathclose%
\pgfusepath{stroke,fill}%
\end{pgfscope}%
\begin{pgfscope}%
\pgfpathrectangle{\pgfqpoint{0.800000in}{0.528000in}}{\pgfqpoint{4.960000in}{3.696000in}}%
\pgfusepath{clip}%
\pgfsetbuttcap%
\pgfsetroundjoin%
\definecolor{currentfill}{rgb}{0.000000,0.000000,0.000000}%
\pgfsetfillcolor{currentfill}%
\pgfsetlinewidth{1.003750pt}%
\definecolor{currentstroke}{rgb}{0.000000,0.000000,0.000000}%
\pgfsetstrokecolor{currentstroke}%
\pgfsetdash{}{0pt}%
\pgfpathmoveto{\pgfqpoint{4.011666in}{3.984333in}}%
\pgfpathcurveto{\pgfqpoint{4.022716in}{3.984333in}}{\pgfqpoint{4.033315in}{3.988724in}}{\pgfqpoint{4.041128in}{3.996537in}}%
\pgfpathcurveto{\pgfqpoint{4.048942in}{4.004351in}}{\pgfqpoint{4.053332in}{4.014950in}}{\pgfqpoint{4.053332in}{4.026000in}}%
\pgfpathcurveto{\pgfqpoint{4.053332in}{4.037050in}}{\pgfqpoint{4.048942in}{4.047649in}}{\pgfqpoint{4.041128in}{4.055463in}}%
\pgfpathcurveto{\pgfqpoint{4.033315in}{4.063276in}}{\pgfqpoint{4.022716in}{4.067667in}}{\pgfqpoint{4.011666in}{4.067667in}}%
\pgfpathcurveto{\pgfqpoint{4.000616in}{4.067667in}}{\pgfqpoint{3.990016in}{4.063276in}}{\pgfqpoint{3.982203in}{4.055463in}}%
\pgfpathcurveto{\pgfqpoint{3.974389in}{4.047649in}}{\pgfqpoint{3.969999in}{4.037050in}}{\pgfqpoint{3.969999in}{4.026000in}}%
\pgfpathcurveto{\pgfqpoint{3.969999in}{4.014950in}}{\pgfqpoint{3.974389in}{4.004351in}}{\pgfqpoint{3.982203in}{3.996537in}}%
\pgfpathcurveto{\pgfqpoint{3.990016in}{3.988724in}}{\pgfqpoint{4.000616in}{3.984333in}}{\pgfqpoint{4.011666in}{3.984333in}}%
\pgfpathclose%
\pgfusepath{stroke,fill}%
\end{pgfscope}%
\begin{pgfscope}%
\pgfpathrectangle{\pgfqpoint{0.800000in}{0.528000in}}{\pgfqpoint{4.960000in}{3.696000in}}%
\pgfusepath{clip}%
\pgfsetbuttcap%
\pgfsetroundjoin%
\definecolor{currentfill}{rgb}{0.000000,0.000000,0.000000}%
\pgfsetfillcolor{currentfill}%
\pgfsetlinewidth{1.003750pt}%
\definecolor{currentstroke}{rgb}{0.000000,0.000000,0.000000}%
\pgfsetstrokecolor{currentstroke}%
\pgfsetdash{}{0pt}%
\pgfpathmoveto{\pgfqpoint{4.011666in}{3.984333in}}%
\pgfpathcurveto{\pgfqpoint{4.022716in}{3.984333in}}{\pgfqpoint{4.033315in}{3.988724in}}{\pgfqpoint{4.041128in}{3.996537in}}%
\pgfpathcurveto{\pgfqpoint{4.048942in}{4.004351in}}{\pgfqpoint{4.053332in}{4.014950in}}{\pgfqpoint{4.053332in}{4.026000in}}%
\pgfpathcurveto{\pgfqpoint{4.053332in}{4.037050in}}{\pgfqpoint{4.048942in}{4.047649in}}{\pgfqpoint{4.041128in}{4.055463in}}%
\pgfpathcurveto{\pgfqpoint{4.033315in}{4.063276in}}{\pgfqpoint{4.022716in}{4.067667in}}{\pgfqpoint{4.011666in}{4.067667in}}%
\pgfpathcurveto{\pgfqpoint{4.000616in}{4.067667in}}{\pgfqpoint{3.990016in}{4.063276in}}{\pgfqpoint{3.982203in}{4.055463in}}%
\pgfpathcurveto{\pgfqpoint{3.974389in}{4.047649in}}{\pgfqpoint{3.969999in}{4.037050in}}{\pgfqpoint{3.969999in}{4.026000in}}%
\pgfpathcurveto{\pgfqpoint{3.969999in}{4.014950in}}{\pgfqpoint{3.974389in}{4.004351in}}{\pgfqpoint{3.982203in}{3.996537in}}%
\pgfpathcurveto{\pgfqpoint{3.990016in}{3.988724in}}{\pgfqpoint{4.000616in}{3.984333in}}{\pgfqpoint{4.011666in}{3.984333in}}%
\pgfpathclose%
\pgfusepath{stroke,fill}%
\end{pgfscope}%
\begin{pgfscope}%
\pgfpathrectangle{\pgfqpoint{0.800000in}{0.528000in}}{\pgfqpoint{4.960000in}{3.696000in}}%
\pgfusepath{clip}%
\pgfsetbuttcap%
\pgfsetroundjoin%
\definecolor{currentfill}{rgb}{0.000000,0.000000,0.000000}%
\pgfsetfillcolor{currentfill}%
\pgfsetlinewidth{1.003750pt}%
\definecolor{currentstroke}{rgb}{0.000000,0.000000,0.000000}%
\pgfsetstrokecolor{currentstroke}%
\pgfsetdash{}{0pt}%
\pgfpathmoveto{\pgfqpoint{4.011666in}{3.984333in}}%
\pgfpathcurveto{\pgfqpoint{4.022716in}{3.984333in}}{\pgfqpoint{4.033315in}{3.988724in}}{\pgfqpoint{4.041128in}{3.996537in}}%
\pgfpathcurveto{\pgfqpoint{4.048942in}{4.004351in}}{\pgfqpoint{4.053332in}{4.014950in}}{\pgfqpoint{4.053332in}{4.026000in}}%
\pgfpathcurveto{\pgfqpoint{4.053332in}{4.037050in}}{\pgfqpoint{4.048942in}{4.047649in}}{\pgfqpoint{4.041128in}{4.055463in}}%
\pgfpathcurveto{\pgfqpoint{4.033315in}{4.063276in}}{\pgfqpoint{4.022716in}{4.067667in}}{\pgfqpoint{4.011666in}{4.067667in}}%
\pgfpathcurveto{\pgfqpoint{4.000616in}{4.067667in}}{\pgfqpoint{3.990016in}{4.063276in}}{\pgfqpoint{3.982203in}{4.055463in}}%
\pgfpathcurveto{\pgfqpoint{3.974389in}{4.047649in}}{\pgfqpoint{3.969999in}{4.037050in}}{\pgfqpoint{3.969999in}{4.026000in}}%
\pgfpathcurveto{\pgfqpoint{3.969999in}{4.014950in}}{\pgfqpoint{3.974389in}{4.004351in}}{\pgfqpoint{3.982203in}{3.996537in}}%
\pgfpathcurveto{\pgfqpoint{3.990016in}{3.988724in}}{\pgfqpoint{4.000616in}{3.984333in}}{\pgfqpoint{4.011666in}{3.984333in}}%
\pgfpathclose%
\pgfusepath{stroke,fill}%
\end{pgfscope}%
\begin{pgfscope}%
\pgfpathrectangle{\pgfqpoint{0.800000in}{0.528000in}}{\pgfqpoint{4.960000in}{3.696000in}}%
\pgfusepath{clip}%
\pgfsetbuttcap%
\pgfsetroundjoin%
\definecolor{currentfill}{rgb}{0.000000,0.000000,0.000000}%
\pgfsetfillcolor{currentfill}%
\pgfsetlinewidth{1.003750pt}%
\definecolor{currentstroke}{rgb}{0.000000,0.000000,0.000000}%
\pgfsetstrokecolor{currentstroke}%
\pgfsetdash{}{0pt}%
\pgfpathmoveto{\pgfqpoint{4.011666in}{3.984333in}}%
\pgfpathcurveto{\pgfqpoint{4.022716in}{3.984333in}}{\pgfqpoint{4.033315in}{3.988724in}}{\pgfqpoint{4.041128in}{3.996537in}}%
\pgfpathcurveto{\pgfqpoint{4.048942in}{4.004351in}}{\pgfqpoint{4.053332in}{4.014950in}}{\pgfqpoint{4.053332in}{4.026000in}}%
\pgfpathcurveto{\pgfqpoint{4.053332in}{4.037050in}}{\pgfqpoint{4.048942in}{4.047649in}}{\pgfqpoint{4.041128in}{4.055463in}}%
\pgfpathcurveto{\pgfqpoint{4.033315in}{4.063276in}}{\pgfqpoint{4.022716in}{4.067667in}}{\pgfqpoint{4.011666in}{4.067667in}}%
\pgfpathcurveto{\pgfqpoint{4.000616in}{4.067667in}}{\pgfqpoint{3.990016in}{4.063276in}}{\pgfqpoint{3.982203in}{4.055463in}}%
\pgfpathcurveto{\pgfqpoint{3.974389in}{4.047649in}}{\pgfqpoint{3.969999in}{4.037050in}}{\pgfqpoint{3.969999in}{4.026000in}}%
\pgfpathcurveto{\pgfqpoint{3.969999in}{4.014950in}}{\pgfqpoint{3.974389in}{4.004351in}}{\pgfqpoint{3.982203in}{3.996537in}}%
\pgfpathcurveto{\pgfqpoint{3.990016in}{3.988724in}}{\pgfqpoint{4.000616in}{3.984333in}}{\pgfqpoint{4.011666in}{3.984333in}}%
\pgfpathclose%
\pgfusepath{stroke,fill}%
\end{pgfscope}%
\begin{pgfscope}%
\pgfpathrectangle{\pgfqpoint{0.800000in}{0.528000in}}{\pgfqpoint{4.960000in}{3.696000in}}%
\pgfusepath{clip}%
\pgfsetbuttcap%
\pgfsetroundjoin%
\definecolor{currentfill}{rgb}{0.000000,0.000000,0.000000}%
\pgfsetfillcolor{currentfill}%
\pgfsetlinewidth{1.003750pt}%
\definecolor{currentstroke}{rgb}{0.000000,0.000000,0.000000}%
\pgfsetstrokecolor{currentstroke}%
\pgfsetdash{}{0pt}%
\pgfpathmoveto{\pgfqpoint{4.011666in}{3.984333in}}%
\pgfpathcurveto{\pgfqpoint{4.022716in}{3.984333in}}{\pgfqpoint{4.033315in}{3.988724in}}{\pgfqpoint{4.041128in}{3.996537in}}%
\pgfpathcurveto{\pgfqpoint{4.048942in}{4.004351in}}{\pgfqpoint{4.053332in}{4.014950in}}{\pgfqpoint{4.053332in}{4.026000in}}%
\pgfpathcurveto{\pgfqpoint{4.053332in}{4.037050in}}{\pgfqpoint{4.048942in}{4.047649in}}{\pgfqpoint{4.041128in}{4.055463in}}%
\pgfpathcurveto{\pgfqpoint{4.033315in}{4.063276in}}{\pgfqpoint{4.022716in}{4.067667in}}{\pgfqpoint{4.011666in}{4.067667in}}%
\pgfpathcurveto{\pgfqpoint{4.000616in}{4.067667in}}{\pgfqpoint{3.990016in}{4.063276in}}{\pgfqpoint{3.982203in}{4.055463in}}%
\pgfpathcurveto{\pgfqpoint{3.974389in}{4.047649in}}{\pgfqpoint{3.969999in}{4.037050in}}{\pgfqpoint{3.969999in}{4.026000in}}%
\pgfpathcurveto{\pgfqpoint{3.969999in}{4.014950in}}{\pgfqpoint{3.974389in}{4.004351in}}{\pgfqpoint{3.982203in}{3.996537in}}%
\pgfpathcurveto{\pgfqpoint{3.990016in}{3.988724in}}{\pgfqpoint{4.000616in}{3.984333in}}{\pgfqpoint{4.011666in}{3.984333in}}%
\pgfpathclose%
\pgfusepath{stroke,fill}%
\end{pgfscope}%
\begin{pgfscope}%
\pgfpathrectangle{\pgfqpoint{0.800000in}{0.528000in}}{\pgfqpoint{4.960000in}{3.696000in}}%
\pgfusepath{clip}%
\pgfsetbuttcap%
\pgfsetroundjoin%
\definecolor{currentfill}{rgb}{0.000000,0.000000,0.000000}%
\pgfsetfillcolor{currentfill}%
\pgfsetlinewidth{1.003750pt}%
\definecolor{currentstroke}{rgb}{0.000000,0.000000,0.000000}%
\pgfsetstrokecolor{currentstroke}%
\pgfsetdash{}{0pt}%
\pgfpathmoveto{\pgfqpoint{4.011666in}{0.684333in}}%
\pgfpathcurveto{\pgfqpoint{4.022716in}{0.684333in}}{\pgfqpoint{4.033315in}{0.688724in}}{\pgfqpoint{4.041128in}{0.696537in}}%
\pgfpathcurveto{\pgfqpoint{4.048942in}{0.704351in}}{\pgfqpoint{4.053332in}{0.714950in}}{\pgfqpoint{4.053332in}{0.726000in}}%
\pgfpathcurveto{\pgfqpoint{4.053332in}{0.737050in}}{\pgfqpoint{4.048942in}{0.747649in}}{\pgfqpoint{4.041128in}{0.755463in}}%
\pgfpathcurveto{\pgfqpoint{4.033315in}{0.763276in}}{\pgfqpoint{4.022716in}{0.767667in}}{\pgfqpoint{4.011666in}{0.767667in}}%
\pgfpathcurveto{\pgfqpoint{4.000616in}{0.767667in}}{\pgfqpoint{3.990016in}{0.763276in}}{\pgfqpoint{3.982203in}{0.755463in}}%
\pgfpathcurveto{\pgfqpoint{3.974389in}{0.747649in}}{\pgfqpoint{3.969999in}{0.737050in}}{\pgfqpoint{3.969999in}{0.726000in}}%
\pgfpathcurveto{\pgfqpoint{3.969999in}{0.714950in}}{\pgfqpoint{3.974389in}{0.704351in}}{\pgfqpoint{3.982203in}{0.696537in}}%
\pgfpathcurveto{\pgfqpoint{3.990016in}{0.688724in}}{\pgfqpoint{4.000616in}{0.684333in}}{\pgfqpoint{4.011666in}{0.684333in}}%
\pgfpathclose%
\pgfusepath{stroke,fill}%
\end{pgfscope}%
\begin{pgfscope}%
\pgfpathrectangle{\pgfqpoint{0.800000in}{0.528000in}}{\pgfqpoint{4.960000in}{3.696000in}}%
\pgfusepath{clip}%
\pgfsetbuttcap%
\pgfsetroundjoin%
\definecolor{currentfill}{rgb}{0.000000,0.000000,0.000000}%
\pgfsetfillcolor{currentfill}%
\pgfsetlinewidth{1.003750pt}%
\definecolor{currentstroke}{rgb}{0.000000,0.000000,0.000000}%
\pgfsetstrokecolor{currentstroke}%
\pgfsetdash{}{0pt}%
\pgfpathmoveto{\pgfqpoint{4.011666in}{3.984333in}}%
\pgfpathcurveto{\pgfqpoint{4.022716in}{3.984333in}}{\pgfqpoint{4.033315in}{3.988724in}}{\pgfqpoint{4.041128in}{3.996537in}}%
\pgfpathcurveto{\pgfqpoint{4.048942in}{4.004351in}}{\pgfqpoint{4.053332in}{4.014950in}}{\pgfqpoint{4.053332in}{4.026000in}}%
\pgfpathcurveto{\pgfqpoint{4.053332in}{4.037050in}}{\pgfqpoint{4.048942in}{4.047649in}}{\pgfqpoint{4.041128in}{4.055463in}}%
\pgfpathcurveto{\pgfqpoint{4.033315in}{4.063276in}}{\pgfqpoint{4.022716in}{4.067667in}}{\pgfqpoint{4.011666in}{4.067667in}}%
\pgfpathcurveto{\pgfqpoint{4.000616in}{4.067667in}}{\pgfqpoint{3.990016in}{4.063276in}}{\pgfqpoint{3.982203in}{4.055463in}}%
\pgfpathcurveto{\pgfqpoint{3.974389in}{4.047649in}}{\pgfqpoint{3.969999in}{4.037050in}}{\pgfqpoint{3.969999in}{4.026000in}}%
\pgfpathcurveto{\pgfqpoint{3.969999in}{4.014950in}}{\pgfqpoint{3.974389in}{4.004351in}}{\pgfqpoint{3.982203in}{3.996537in}}%
\pgfpathcurveto{\pgfqpoint{3.990016in}{3.988724in}}{\pgfqpoint{4.000616in}{3.984333in}}{\pgfqpoint{4.011666in}{3.984333in}}%
\pgfpathclose%
\pgfusepath{stroke,fill}%
\end{pgfscope}%
\begin{pgfscope}%
\pgfpathrectangle{\pgfqpoint{0.800000in}{0.528000in}}{\pgfqpoint{4.960000in}{3.696000in}}%
\pgfusepath{clip}%
\pgfsetbuttcap%
\pgfsetroundjoin%
\definecolor{currentfill}{rgb}{0.000000,0.000000,0.000000}%
\pgfsetfillcolor{currentfill}%
\pgfsetlinewidth{1.003750pt}%
\definecolor{currentstroke}{rgb}{0.000000,0.000000,0.000000}%
\pgfsetstrokecolor{currentstroke}%
\pgfsetdash{}{0pt}%
\pgfpathmoveto{\pgfqpoint{4.011666in}{3.984333in}}%
\pgfpathcurveto{\pgfqpoint{4.022716in}{3.984333in}}{\pgfqpoint{4.033315in}{3.988724in}}{\pgfqpoint{4.041128in}{3.996537in}}%
\pgfpathcurveto{\pgfqpoint{4.048942in}{4.004351in}}{\pgfqpoint{4.053332in}{4.014950in}}{\pgfqpoint{4.053332in}{4.026000in}}%
\pgfpathcurveto{\pgfqpoint{4.053332in}{4.037050in}}{\pgfqpoint{4.048942in}{4.047649in}}{\pgfqpoint{4.041128in}{4.055463in}}%
\pgfpathcurveto{\pgfqpoint{4.033315in}{4.063276in}}{\pgfqpoint{4.022716in}{4.067667in}}{\pgfqpoint{4.011666in}{4.067667in}}%
\pgfpathcurveto{\pgfqpoint{4.000616in}{4.067667in}}{\pgfqpoint{3.990016in}{4.063276in}}{\pgfqpoint{3.982203in}{4.055463in}}%
\pgfpathcurveto{\pgfqpoint{3.974389in}{4.047649in}}{\pgfqpoint{3.969999in}{4.037050in}}{\pgfqpoint{3.969999in}{4.026000in}}%
\pgfpathcurveto{\pgfqpoint{3.969999in}{4.014950in}}{\pgfqpoint{3.974389in}{4.004351in}}{\pgfqpoint{3.982203in}{3.996537in}}%
\pgfpathcurveto{\pgfqpoint{3.990016in}{3.988724in}}{\pgfqpoint{4.000616in}{3.984333in}}{\pgfqpoint{4.011666in}{3.984333in}}%
\pgfpathclose%
\pgfusepath{stroke,fill}%
\end{pgfscope}%
\begin{pgfscope}%
\pgfpathrectangle{\pgfqpoint{0.800000in}{0.528000in}}{\pgfqpoint{4.960000in}{3.696000in}}%
\pgfusepath{clip}%
\pgfsetbuttcap%
\pgfsetroundjoin%
\definecolor{currentfill}{rgb}{0.000000,0.000000,0.000000}%
\pgfsetfillcolor{currentfill}%
\pgfsetlinewidth{1.003750pt}%
\definecolor{currentstroke}{rgb}{0.000000,0.000000,0.000000}%
\pgfsetstrokecolor{currentstroke}%
\pgfsetdash{}{0pt}%
\pgfpathmoveto{\pgfqpoint{4.011666in}{3.984333in}}%
\pgfpathcurveto{\pgfqpoint{4.022716in}{3.984333in}}{\pgfqpoint{4.033315in}{3.988724in}}{\pgfqpoint{4.041128in}{3.996537in}}%
\pgfpathcurveto{\pgfqpoint{4.048942in}{4.004351in}}{\pgfqpoint{4.053332in}{4.014950in}}{\pgfqpoint{4.053332in}{4.026000in}}%
\pgfpathcurveto{\pgfqpoint{4.053332in}{4.037050in}}{\pgfqpoint{4.048942in}{4.047649in}}{\pgfqpoint{4.041128in}{4.055463in}}%
\pgfpathcurveto{\pgfqpoint{4.033315in}{4.063276in}}{\pgfqpoint{4.022716in}{4.067667in}}{\pgfqpoint{4.011666in}{4.067667in}}%
\pgfpathcurveto{\pgfqpoint{4.000616in}{4.067667in}}{\pgfqpoint{3.990016in}{4.063276in}}{\pgfqpoint{3.982203in}{4.055463in}}%
\pgfpathcurveto{\pgfqpoint{3.974389in}{4.047649in}}{\pgfqpoint{3.969999in}{4.037050in}}{\pgfqpoint{3.969999in}{4.026000in}}%
\pgfpathcurveto{\pgfqpoint{3.969999in}{4.014950in}}{\pgfqpoint{3.974389in}{4.004351in}}{\pgfqpoint{3.982203in}{3.996537in}}%
\pgfpathcurveto{\pgfqpoint{3.990016in}{3.988724in}}{\pgfqpoint{4.000616in}{3.984333in}}{\pgfqpoint{4.011666in}{3.984333in}}%
\pgfpathclose%
\pgfusepath{stroke,fill}%
\end{pgfscope}%
\begin{pgfscope}%
\pgfpathrectangle{\pgfqpoint{0.800000in}{0.528000in}}{\pgfqpoint{4.960000in}{3.696000in}}%
\pgfusepath{clip}%
\pgfsetbuttcap%
\pgfsetroundjoin%
\definecolor{currentfill}{rgb}{0.000000,0.000000,0.000000}%
\pgfsetfillcolor{currentfill}%
\pgfsetlinewidth{1.003750pt}%
\definecolor{currentstroke}{rgb}{0.000000,0.000000,0.000000}%
\pgfsetstrokecolor{currentstroke}%
\pgfsetdash{}{0pt}%
\pgfpathmoveto{\pgfqpoint{4.011666in}{0.684333in}}%
\pgfpathcurveto{\pgfqpoint{4.022716in}{0.684333in}}{\pgfqpoint{4.033315in}{0.688724in}}{\pgfqpoint{4.041128in}{0.696537in}}%
\pgfpathcurveto{\pgfqpoint{4.048942in}{0.704351in}}{\pgfqpoint{4.053332in}{0.714950in}}{\pgfqpoint{4.053332in}{0.726000in}}%
\pgfpathcurveto{\pgfqpoint{4.053332in}{0.737050in}}{\pgfqpoint{4.048942in}{0.747649in}}{\pgfqpoint{4.041128in}{0.755463in}}%
\pgfpathcurveto{\pgfqpoint{4.033315in}{0.763276in}}{\pgfqpoint{4.022716in}{0.767667in}}{\pgfqpoint{4.011666in}{0.767667in}}%
\pgfpathcurveto{\pgfqpoint{4.000616in}{0.767667in}}{\pgfqpoint{3.990016in}{0.763276in}}{\pgfqpoint{3.982203in}{0.755463in}}%
\pgfpathcurveto{\pgfqpoint{3.974389in}{0.747649in}}{\pgfqpoint{3.969999in}{0.737050in}}{\pgfqpoint{3.969999in}{0.726000in}}%
\pgfpathcurveto{\pgfqpoint{3.969999in}{0.714950in}}{\pgfqpoint{3.974389in}{0.704351in}}{\pgfqpoint{3.982203in}{0.696537in}}%
\pgfpathcurveto{\pgfqpoint{3.990016in}{0.688724in}}{\pgfqpoint{4.000616in}{0.684333in}}{\pgfqpoint{4.011666in}{0.684333in}}%
\pgfpathclose%
\pgfusepath{stroke,fill}%
\end{pgfscope}%
\begin{pgfscope}%
\pgfpathrectangle{\pgfqpoint{0.800000in}{0.528000in}}{\pgfqpoint{4.960000in}{3.696000in}}%
\pgfusepath{clip}%
\pgfsetbuttcap%
\pgfsetroundjoin%
\definecolor{currentfill}{rgb}{0.000000,0.000000,0.000000}%
\pgfsetfillcolor{currentfill}%
\pgfsetlinewidth{1.003750pt}%
\definecolor{currentstroke}{rgb}{0.000000,0.000000,0.000000}%
\pgfsetstrokecolor{currentstroke}%
\pgfsetdash{}{0pt}%
\pgfpathmoveto{\pgfqpoint{4.011666in}{3.984333in}}%
\pgfpathcurveto{\pgfqpoint{4.022716in}{3.984333in}}{\pgfqpoint{4.033315in}{3.988724in}}{\pgfqpoint{4.041128in}{3.996537in}}%
\pgfpathcurveto{\pgfqpoint{4.048942in}{4.004351in}}{\pgfqpoint{4.053332in}{4.014950in}}{\pgfqpoint{4.053332in}{4.026000in}}%
\pgfpathcurveto{\pgfqpoint{4.053332in}{4.037050in}}{\pgfqpoint{4.048942in}{4.047649in}}{\pgfqpoint{4.041128in}{4.055463in}}%
\pgfpathcurveto{\pgfqpoint{4.033315in}{4.063276in}}{\pgfqpoint{4.022716in}{4.067667in}}{\pgfqpoint{4.011666in}{4.067667in}}%
\pgfpathcurveto{\pgfqpoint{4.000616in}{4.067667in}}{\pgfqpoint{3.990016in}{4.063276in}}{\pgfqpoint{3.982203in}{4.055463in}}%
\pgfpathcurveto{\pgfqpoint{3.974389in}{4.047649in}}{\pgfqpoint{3.969999in}{4.037050in}}{\pgfqpoint{3.969999in}{4.026000in}}%
\pgfpathcurveto{\pgfqpoint{3.969999in}{4.014950in}}{\pgfqpoint{3.974389in}{4.004351in}}{\pgfqpoint{3.982203in}{3.996537in}}%
\pgfpathcurveto{\pgfqpoint{3.990016in}{3.988724in}}{\pgfqpoint{4.000616in}{3.984333in}}{\pgfqpoint{4.011666in}{3.984333in}}%
\pgfpathclose%
\pgfusepath{stroke,fill}%
\end{pgfscope}%
\begin{pgfscope}%
\pgfpathrectangle{\pgfqpoint{0.800000in}{0.528000in}}{\pgfqpoint{4.960000in}{3.696000in}}%
\pgfusepath{clip}%
\pgfsetbuttcap%
\pgfsetroundjoin%
\definecolor{currentfill}{rgb}{0.000000,0.000000,0.000000}%
\pgfsetfillcolor{currentfill}%
\pgfsetlinewidth{1.003750pt}%
\definecolor{currentstroke}{rgb}{0.000000,0.000000,0.000000}%
\pgfsetstrokecolor{currentstroke}%
\pgfsetdash{}{0pt}%
\pgfpathmoveto{\pgfqpoint{4.011666in}{3.984333in}}%
\pgfpathcurveto{\pgfqpoint{4.022716in}{3.984333in}}{\pgfqpoint{4.033315in}{3.988724in}}{\pgfqpoint{4.041128in}{3.996537in}}%
\pgfpathcurveto{\pgfqpoint{4.048942in}{4.004351in}}{\pgfqpoint{4.053332in}{4.014950in}}{\pgfqpoint{4.053332in}{4.026000in}}%
\pgfpathcurveto{\pgfqpoint{4.053332in}{4.037050in}}{\pgfqpoint{4.048942in}{4.047649in}}{\pgfqpoint{4.041128in}{4.055463in}}%
\pgfpathcurveto{\pgfqpoint{4.033315in}{4.063276in}}{\pgfqpoint{4.022716in}{4.067667in}}{\pgfqpoint{4.011666in}{4.067667in}}%
\pgfpathcurveto{\pgfqpoint{4.000616in}{4.067667in}}{\pgfqpoint{3.990016in}{4.063276in}}{\pgfqpoint{3.982203in}{4.055463in}}%
\pgfpathcurveto{\pgfqpoint{3.974389in}{4.047649in}}{\pgfqpoint{3.969999in}{4.037050in}}{\pgfqpoint{3.969999in}{4.026000in}}%
\pgfpathcurveto{\pgfqpoint{3.969999in}{4.014950in}}{\pgfqpoint{3.974389in}{4.004351in}}{\pgfqpoint{3.982203in}{3.996537in}}%
\pgfpathcurveto{\pgfqpoint{3.990016in}{3.988724in}}{\pgfqpoint{4.000616in}{3.984333in}}{\pgfqpoint{4.011666in}{3.984333in}}%
\pgfpathclose%
\pgfusepath{stroke,fill}%
\end{pgfscope}%
\begin{pgfscope}%
\pgfpathrectangle{\pgfqpoint{0.800000in}{0.528000in}}{\pgfqpoint{4.960000in}{3.696000in}}%
\pgfusepath{clip}%
\pgfsetbuttcap%
\pgfsetroundjoin%
\definecolor{currentfill}{rgb}{0.000000,0.000000,0.000000}%
\pgfsetfillcolor{currentfill}%
\pgfsetlinewidth{1.003750pt}%
\definecolor{currentstroke}{rgb}{0.000000,0.000000,0.000000}%
\pgfsetstrokecolor{currentstroke}%
\pgfsetdash{}{0pt}%
\pgfpathmoveto{\pgfqpoint{4.011666in}{3.984333in}}%
\pgfpathcurveto{\pgfqpoint{4.022716in}{3.984333in}}{\pgfqpoint{4.033315in}{3.988724in}}{\pgfqpoint{4.041128in}{3.996537in}}%
\pgfpathcurveto{\pgfqpoint{4.048942in}{4.004351in}}{\pgfqpoint{4.053332in}{4.014950in}}{\pgfqpoint{4.053332in}{4.026000in}}%
\pgfpathcurveto{\pgfqpoint{4.053332in}{4.037050in}}{\pgfqpoint{4.048942in}{4.047649in}}{\pgfqpoint{4.041128in}{4.055463in}}%
\pgfpathcurveto{\pgfqpoint{4.033315in}{4.063276in}}{\pgfqpoint{4.022716in}{4.067667in}}{\pgfqpoint{4.011666in}{4.067667in}}%
\pgfpathcurveto{\pgfqpoint{4.000616in}{4.067667in}}{\pgfqpoint{3.990016in}{4.063276in}}{\pgfqpoint{3.982203in}{4.055463in}}%
\pgfpathcurveto{\pgfqpoint{3.974389in}{4.047649in}}{\pgfqpoint{3.969999in}{4.037050in}}{\pgfqpoint{3.969999in}{4.026000in}}%
\pgfpathcurveto{\pgfqpoint{3.969999in}{4.014950in}}{\pgfqpoint{3.974389in}{4.004351in}}{\pgfqpoint{3.982203in}{3.996537in}}%
\pgfpathcurveto{\pgfqpoint{3.990016in}{3.988724in}}{\pgfqpoint{4.000616in}{3.984333in}}{\pgfqpoint{4.011666in}{3.984333in}}%
\pgfpathclose%
\pgfusepath{stroke,fill}%
\end{pgfscope}%
\begin{pgfscope}%
\pgfpathrectangle{\pgfqpoint{0.800000in}{0.528000in}}{\pgfqpoint{4.960000in}{3.696000in}}%
\pgfusepath{clip}%
\pgfsetbuttcap%
\pgfsetroundjoin%
\definecolor{currentfill}{rgb}{0.000000,0.000000,0.000000}%
\pgfsetfillcolor{currentfill}%
\pgfsetlinewidth{1.003750pt}%
\definecolor{currentstroke}{rgb}{0.000000,0.000000,0.000000}%
\pgfsetstrokecolor{currentstroke}%
\pgfsetdash{}{0pt}%
\pgfpathmoveto{\pgfqpoint{4.011666in}{3.984333in}}%
\pgfpathcurveto{\pgfqpoint{4.022716in}{3.984333in}}{\pgfqpoint{4.033315in}{3.988724in}}{\pgfqpoint{4.041128in}{3.996537in}}%
\pgfpathcurveto{\pgfqpoint{4.048942in}{4.004351in}}{\pgfqpoint{4.053332in}{4.014950in}}{\pgfqpoint{4.053332in}{4.026000in}}%
\pgfpathcurveto{\pgfqpoint{4.053332in}{4.037050in}}{\pgfqpoint{4.048942in}{4.047649in}}{\pgfqpoint{4.041128in}{4.055463in}}%
\pgfpathcurveto{\pgfqpoint{4.033315in}{4.063276in}}{\pgfqpoint{4.022716in}{4.067667in}}{\pgfqpoint{4.011666in}{4.067667in}}%
\pgfpathcurveto{\pgfqpoint{4.000616in}{4.067667in}}{\pgfqpoint{3.990016in}{4.063276in}}{\pgfqpoint{3.982203in}{4.055463in}}%
\pgfpathcurveto{\pgfqpoint{3.974389in}{4.047649in}}{\pgfqpoint{3.969999in}{4.037050in}}{\pgfqpoint{3.969999in}{4.026000in}}%
\pgfpathcurveto{\pgfqpoint{3.969999in}{4.014950in}}{\pgfqpoint{3.974389in}{4.004351in}}{\pgfqpoint{3.982203in}{3.996537in}}%
\pgfpathcurveto{\pgfqpoint{3.990016in}{3.988724in}}{\pgfqpoint{4.000616in}{3.984333in}}{\pgfqpoint{4.011666in}{3.984333in}}%
\pgfpathclose%
\pgfusepath{stroke,fill}%
\end{pgfscope}%
\begin{pgfscope}%
\pgfpathrectangle{\pgfqpoint{0.800000in}{0.528000in}}{\pgfqpoint{4.960000in}{3.696000in}}%
\pgfusepath{clip}%
\pgfsetbuttcap%
\pgfsetroundjoin%
\definecolor{currentfill}{rgb}{0.000000,0.000000,0.000000}%
\pgfsetfillcolor{currentfill}%
\pgfsetlinewidth{1.003750pt}%
\definecolor{currentstroke}{rgb}{0.000000,0.000000,0.000000}%
\pgfsetstrokecolor{currentstroke}%
\pgfsetdash{}{0pt}%
\pgfpathmoveto{\pgfqpoint{4.011666in}{3.984333in}}%
\pgfpathcurveto{\pgfqpoint{4.022716in}{3.984333in}}{\pgfqpoint{4.033315in}{3.988724in}}{\pgfqpoint{4.041128in}{3.996537in}}%
\pgfpathcurveto{\pgfqpoint{4.048942in}{4.004351in}}{\pgfqpoint{4.053332in}{4.014950in}}{\pgfqpoint{4.053332in}{4.026000in}}%
\pgfpathcurveto{\pgfqpoint{4.053332in}{4.037050in}}{\pgfqpoint{4.048942in}{4.047649in}}{\pgfqpoint{4.041128in}{4.055463in}}%
\pgfpathcurveto{\pgfqpoint{4.033315in}{4.063276in}}{\pgfqpoint{4.022716in}{4.067667in}}{\pgfqpoint{4.011666in}{4.067667in}}%
\pgfpathcurveto{\pgfqpoint{4.000616in}{4.067667in}}{\pgfqpoint{3.990016in}{4.063276in}}{\pgfqpoint{3.982203in}{4.055463in}}%
\pgfpathcurveto{\pgfqpoint{3.974389in}{4.047649in}}{\pgfqpoint{3.969999in}{4.037050in}}{\pgfqpoint{3.969999in}{4.026000in}}%
\pgfpathcurveto{\pgfqpoint{3.969999in}{4.014950in}}{\pgfqpoint{3.974389in}{4.004351in}}{\pgfqpoint{3.982203in}{3.996537in}}%
\pgfpathcurveto{\pgfqpoint{3.990016in}{3.988724in}}{\pgfqpoint{4.000616in}{3.984333in}}{\pgfqpoint{4.011666in}{3.984333in}}%
\pgfpathclose%
\pgfusepath{stroke,fill}%
\end{pgfscope}%
\begin{pgfscope}%
\pgfpathrectangle{\pgfqpoint{0.800000in}{0.528000in}}{\pgfqpoint{4.960000in}{3.696000in}}%
\pgfusepath{clip}%
\pgfsetbuttcap%
\pgfsetroundjoin%
\definecolor{currentfill}{rgb}{0.000000,0.000000,0.000000}%
\pgfsetfillcolor{currentfill}%
\pgfsetlinewidth{1.003750pt}%
\definecolor{currentstroke}{rgb}{0.000000,0.000000,0.000000}%
\pgfsetstrokecolor{currentstroke}%
\pgfsetdash{}{0pt}%
\pgfpathmoveto{\pgfqpoint{4.011666in}{3.984333in}}%
\pgfpathcurveto{\pgfqpoint{4.022716in}{3.984333in}}{\pgfqpoint{4.033315in}{3.988724in}}{\pgfqpoint{4.041128in}{3.996537in}}%
\pgfpathcurveto{\pgfqpoint{4.048942in}{4.004351in}}{\pgfqpoint{4.053332in}{4.014950in}}{\pgfqpoint{4.053332in}{4.026000in}}%
\pgfpathcurveto{\pgfqpoint{4.053332in}{4.037050in}}{\pgfqpoint{4.048942in}{4.047649in}}{\pgfqpoint{4.041128in}{4.055463in}}%
\pgfpathcurveto{\pgfqpoint{4.033315in}{4.063276in}}{\pgfqpoint{4.022716in}{4.067667in}}{\pgfqpoint{4.011666in}{4.067667in}}%
\pgfpathcurveto{\pgfqpoint{4.000616in}{4.067667in}}{\pgfqpoint{3.990016in}{4.063276in}}{\pgfqpoint{3.982203in}{4.055463in}}%
\pgfpathcurveto{\pgfqpoint{3.974389in}{4.047649in}}{\pgfqpoint{3.969999in}{4.037050in}}{\pgfqpoint{3.969999in}{4.026000in}}%
\pgfpathcurveto{\pgfqpoint{3.969999in}{4.014950in}}{\pgfqpoint{3.974389in}{4.004351in}}{\pgfqpoint{3.982203in}{3.996537in}}%
\pgfpathcurveto{\pgfqpoint{3.990016in}{3.988724in}}{\pgfqpoint{4.000616in}{3.984333in}}{\pgfqpoint{4.011666in}{3.984333in}}%
\pgfpathclose%
\pgfusepath{stroke,fill}%
\end{pgfscope}%
\begin{pgfscope}%
\pgfpathrectangle{\pgfqpoint{0.800000in}{0.528000in}}{\pgfqpoint{4.960000in}{3.696000in}}%
\pgfusepath{clip}%
\pgfsetbuttcap%
\pgfsetroundjoin%
\definecolor{currentfill}{rgb}{0.000000,0.000000,0.000000}%
\pgfsetfillcolor{currentfill}%
\pgfsetlinewidth{1.003750pt}%
\definecolor{currentstroke}{rgb}{0.000000,0.000000,0.000000}%
\pgfsetstrokecolor{currentstroke}%
\pgfsetdash{}{0pt}%
\pgfpathmoveto{\pgfqpoint{4.011666in}{3.984333in}}%
\pgfpathcurveto{\pgfqpoint{4.022716in}{3.984333in}}{\pgfqpoint{4.033315in}{3.988724in}}{\pgfqpoint{4.041128in}{3.996537in}}%
\pgfpathcurveto{\pgfqpoint{4.048942in}{4.004351in}}{\pgfqpoint{4.053332in}{4.014950in}}{\pgfqpoint{4.053332in}{4.026000in}}%
\pgfpathcurveto{\pgfqpoint{4.053332in}{4.037050in}}{\pgfqpoint{4.048942in}{4.047649in}}{\pgfqpoint{4.041128in}{4.055463in}}%
\pgfpathcurveto{\pgfqpoint{4.033315in}{4.063276in}}{\pgfqpoint{4.022716in}{4.067667in}}{\pgfqpoint{4.011666in}{4.067667in}}%
\pgfpathcurveto{\pgfqpoint{4.000616in}{4.067667in}}{\pgfqpoint{3.990016in}{4.063276in}}{\pgfqpoint{3.982203in}{4.055463in}}%
\pgfpathcurveto{\pgfqpoint{3.974389in}{4.047649in}}{\pgfqpoint{3.969999in}{4.037050in}}{\pgfqpoint{3.969999in}{4.026000in}}%
\pgfpathcurveto{\pgfqpoint{3.969999in}{4.014950in}}{\pgfqpoint{3.974389in}{4.004351in}}{\pgfqpoint{3.982203in}{3.996537in}}%
\pgfpathcurveto{\pgfqpoint{3.990016in}{3.988724in}}{\pgfqpoint{4.000616in}{3.984333in}}{\pgfqpoint{4.011666in}{3.984333in}}%
\pgfpathclose%
\pgfusepath{stroke,fill}%
\end{pgfscope}%
\begin{pgfscope}%
\pgfpathrectangle{\pgfqpoint{0.800000in}{0.528000in}}{\pgfqpoint{4.960000in}{3.696000in}}%
\pgfusepath{clip}%
\pgfsetbuttcap%
\pgfsetroundjoin%
\definecolor{currentfill}{rgb}{0.000000,0.000000,0.000000}%
\pgfsetfillcolor{currentfill}%
\pgfsetlinewidth{1.003750pt}%
\definecolor{currentstroke}{rgb}{0.000000,0.000000,0.000000}%
\pgfsetstrokecolor{currentstroke}%
\pgfsetdash{}{0pt}%
\pgfpathmoveto{\pgfqpoint{4.011666in}{3.984333in}}%
\pgfpathcurveto{\pgfqpoint{4.022716in}{3.984333in}}{\pgfqpoint{4.033315in}{3.988724in}}{\pgfqpoint{4.041128in}{3.996537in}}%
\pgfpathcurveto{\pgfqpoint{4.048942in}{4.004351in}}{\pgfqpoint{4.053332in}{4.014950in}}{\pgfqpoint{4.053332in}{4.026000in}}%
\pgfpathcurveto{\pgfqpoint{4.053332in}{4.037050in}}{\pgfqpoint{4.048942in}{4.047649in}}{\pgfqpoint{4.041128in}{4.055463in}}%
\pgfpathcurveto{\pgfqpoint{4.033315in}{4.063276in}}{\pgfqpoint{4.022716in}{4.067667in}}{\pgfqpoint{4.011666in}{4.067667in}}%
\pgfpathcurveto{\pgfqpoint{4.000616in}{4.067667in}}{\pgfqpoint{3.990016in}{4.063276in}}{\pgfqpoint{3.982203in}{4.055463in}}%
\pgfpathcurveto{\pgfqpoint{3.974389in}{4.047649in}}{\pgfqpoint{3.969999in}{4.037050in}}{\pgfqpoint{3.969999in}{4.026000in}}%
\pgfpathcurveto{\pgfqpoint{3.969999in}{4.014950in}}{\pgfqpoint{3.974389in}{4.004351in}}{\pgfqpoint{3.982203in}{3.996537in}}%
\pgfpathcurveto{\pgfqpoint{3.990016in}{3.988724in}}{\pgfqpoint{4.000616in}{3.984333in}}{\pgfqpoint{4.011666in}{3.984333in}}%
\pgfpathclose%
\pgfusepath{stroke,fill}%
\end{pgfscope}%
\begin{pgfscope}%
\pgfpathrectangle{\pgfqpoint{0.800000in}{0.528000in}}{\pgfqpoint{4.960000in}{3.696000in}}%
\pgfusepath{clip}%
\pgfsetbuttcap%
\pgfsetroundjoin%
\definecolor{currentfill}{rgb}{0.000000,0.000000,0.000000}%
\pgfsetfillcolor{currentfill}%
\pgfsetlinewidth{1.003750pt}%
\definecolor{currentstroke}{rgb}{0.000000,0.000000,0.000000}%
\pgfsetstrokecolor{currentstroke}%
\pgfsetdash{}{0pt}%
\pgfpathmoveto{\pgfqpoint{4.011666in}{3.984333in}}%
\pgfpathcurveto{\pgfqpoint{4.022716in}{3.984333in}}{\pgfqpoint{4.033315in}{3.988724in}}{\pgfqpoint{4.041128in}{3.996537in}}%
\pgfpathcurveto{\pgfqpoint{4.048942in}{4.004351in}}{\pgfqpoint{4.053332in}{4.014950in}}{\pgfqpoint{4.053332in}{4.026000in}}%
\pgfpathcurveto{\pgfqpoint{4.053332in}{4.037050in}}{\pgfqpoint{4.048942in}{4.047649in}}{\pgfqpoint{4.041128in}{4.055463in}}%
\pgfpathcurveto{\pgfqpoint{4.033315in}{4.063276in}}{\pgfqpoint{4.022716in}{4.067667in}}{\pgfqpoint{4.011666in}{4.067667in}}%
\pgfpathcurveto{\pgfqpoint{4.000616in}{4.067667in}}{\pgfqpoint{3.990016in}{4.063276in}}{\pgfqpoint{3.982203in}{4.055463in}}%
\pgfpathcurveto{\pgfqpoint{3.974389in}{4.047649in}}{\pgfqpoint{3.969999in}{4.037050in}}{\pgfqpoint{3.969999in}{4.026000in}}%
\pgfpathcurveto{\pgfqpoint{3.969999in}{4.014950in}}{\pgfqpoint{3.974389in}{4.004351in}}{\pgfqpoint{3.982203in}{3.996537in}}%
\pgfpathcurveto{\pgfqpoint{3.990016in}{3.988724in}}{\pgfqpoint{4.000616in}{3.984333in}}{\pgfqpoint{4.011666in}{3.984333in}}%
\pgfpathclose%
\pgfusepath{stroke,fill}%
\end{pgfscope}%
\begin{pgfscope}%
\pgfpathrectangle{\pgfqpoint{0.800000in}{0.528000in}}{\pgfqpoint{4.960000in}{3.696000in}}%
\pgfusepath{clip}%
\pgfsetbuttcap%
\pgfsetroundjoin%
\definecolor{currentfill}{rgb}{0.000000,0.000000,0.000000}%
\pgfsetfillcolor{currentfill}%
\pgfsetlinewidth{1.003750pt}%
\definecolor{currentstroke}{rgb}{0.000000,0.000000,0.000000}%
\pgfsetstrokecolor{currentstroke}%
\pgfsetdash{}{0pt}%
\pgfpathmoveto{\pgfqpoint{4.011666in}{3.984333in}}%
\pgfpathcurveto{\pgfqpoint{4.022716in}{3.984333in}}{\pgfqpoint{4.033315in}{3.988724in}}{\pgfqpoint{4.041128in}{3.996537in}}%
\pgfpathcurveto{\pgfqpoint{4.048942in}{4.004351in}}{\pgfqpoint{4.053332in}{4.014950in}}{\pgfqpoint{4.053332in}{4.026000in}}%
\pgfpathcurveto{\pgfqpoint{4.053332in}{4.037050in}}{\pgfqpoint{4.048942in}{4.047649in}}{\pgfqpoint{4.041128in}{4.055463in}}%
\pgfpathcurveto{\pgfqpoint{4.033315in}{4.063276in}}{\pgfqpoint{4.022716in}{4.067667in}}{\pgfqpoint{4.011666in}{4.067667in}}%
\pgfpathcurveto{\pgfqpoint{4.000616in}{4.067667in}}{\pgfqpoint{3.990016in}{4.063276in}}{\pgfqpoint{3.982203in}{4.055463in}}%
\pgfpathcurveto{\pgfqpoint{3.974389in}{4.047649in}}{\pgfqpoint{3.969999in}{4.037050in}}{\pgfqpoint{3.969999in}{4.026000in}}%
\pgfpathcurveto{\pgfqpoint{3.969999in}{4.014950in}}{\pgfqpoint{3.974389in}{4.004351in}}{\pgfqpoint{3.982203in}{3.996537in}}%
\pgfpathcurveto{\pgfqpoint{3.990016in}{3.988724in}}{\pgfqpoint{4.000616in}{3.984333in}}{\pgfqpoint{4.011666in}{3.984333in}}%
\pgfpathclose%
\pgfusepath{stroke,fill}%
\end{pgfscope}%
\begin{pgfscope}%
\pgfpathrectangle{\pgfqpoint{0.800000in}{0.528000in}}{\pgfqpoint{4.960000in}{3.696000in}}%
\pgfusepath{clip}%
\pgfsetbuttcap%
\pgfsetroundjoin%
\definecolor{currentfill}{rgb}{0.000000,0.000000,0.000000}%
\pgfsetfillcolor{currentfill}%
\pgfsetlinewidth{1.003750pt}%
\definecolor{currentstroke}{rgb}{0.000000,0.000000,0.000000}%
\pgfsetstrokecolor{currentstroke}%
\pgfsetdash{}{0pt}%
\pgfpathmoveto{\pgfqpoint{4.011666in}{3.984333in}}%
\pgfpathcurveto{\pgfqpoint{4.022716in}{3.984333in}}{\pgfqpoint{4.033315in}{3.988724in}}{\pgfqpoint{4.041128in}{3.996537in}}%
\pgfpathcurveto{\pgfqpoint{4.048942in}{4.004351in}}{\pgfqpoint{4.053332in}{4.014950in}}{\pgfqpoint{4.053332in}{4.026000in}}%
\pgfpathcurveto{\pgfqpoint{4.053332in}{4.037050in}}{\pgfqpoint{4.048942in}{4.047649in}}{\pgfqpoint{4.041128in}{4.055463in}}%
\pgfpathcurveto{\pgfqpoint{4.033315in}{4.063276in}}{\pgfqpoint{4.022716in}{4.067667in}}{\pgfqpoint{4.011666in}{4.067667in}}%
\pgfpathcurveto{\pgfqpoint{4.000616in}{4.067667in}}{\pgfqpoint{3.990016in}{4.063276in}}{\pgfqpoint{3.982203in}{4.055463in}}%
\pgfpathcurveto{\pgfqpoint{3.974389in}{4.047649in}}{\pgfqpoint{3.969999in}{4.037050in}}{\pgfqpoint{3.969999in}{4.026000in}}%
\pgfpathcurveto{\pgfqpoint{3.969999in}{4.014950in}}{\pgfqpoint{3.974389in}{4.004351in}}{\pgfqpoint{3.982203in}{3.996537in}}%
\pgfpathcurveto{\pgfqpoint{3.990016in}{3.988724in}}{\pgfqpoint{4.000616in}{3.984333in}}{\pgfqpoint{4.011666in}{3.984333in}}%
\pgfpathclose%
\pgfusepath{stroke,fill}%
\end{pgfscope}%
\begin{pgfscope}%
\pgfpathrectangle{\pgfqpoint{0.800000in}{0.528000in}}{\pgfqpoint{4.960000in}{3.696000in}}%
\pgfusepath{clip}%
\pgfsetbuttcap%
\pgfsetroundjoin%
\definecolor{currentfill}{rgb}{0.000000,0.000000,0.000000}%
\pgfsetfillcolor{currentfill}%
\pgfsetlinewidth{1.003750pt}%
\definecolor{currentstroke}{rgb}{0.000000,0.000000,0.000000}%
\pgfsetstrokecolor{currentstroke}%
\pgfsetdash{}{0pt}%
\pgfpathmoveto{\pgfqpoint{4.011666in}{3.984333in}}%
\pgfpathcurveto{\pgfqpoint{4.022716in}{3.984333in}}{\pgfqpoint{4.033315in}{3.988724in}}{\pgfqpoint{4.041128in}{3.996537in}}%
\pgfpathcurveto{\pgfqpoint{4.048942in}{4.004351in}}{\pgfqpoint{4.053332in}{4.014950in}}{\pgfqpoint{4.053332in}{4.026000in}}%
\pgfpathcurveto{\pgfqpoint{4.053332in}{4.037050in}}{\pgfqpoint{4.048942in}{4.047649in}}{\pgfqpoint{4.041128in}{4.055463in}}%
\pgfpathcurveto{\pgfqpoint{4.033315in}{4.063276in}}{\pgfqpoint{4.022716in}{4.067667in}}{\pgfqpoint{4.011666in}{4.067667in}}%
\pgfpathcurveto{\pgfqpoint{4.000616in}{4.067667in}}{\pgfqpoint{3.990016in}{4.063276in}}{\pgfqpoint{3.982203in}{4.055463in}}%
\pgfpathcurveto{\pgfqpoint{3.974389in}{4.047649in}}{\pgfqpoint{3.969999in}{4.037050in}}{\pgfqpoint{3.969999in}{4.026000in}}%
\pgfpathcurveto{\pgfqpoint{3.969999in}{4.014950in}}{\pgfqpoint{3.974389in}{4.004351in}}{\pgfqpoint{3.982203in}{3.996537in}}%
\pgfpathcurveto{\pgfqpoint{3.990016in}{3.988724in}}{\pgfqpoint{4.000616in}{3.984333in}}{\pgfqpoint{4.011666in}{3.984333in}}%
\pgfpathclose%
\pgfusepath{stroke,fill}%
\end{pgfscope}%
\begin{pgfscope}%
\pgfpathrectangle{\pgfqpoint{0.800000in}{0.528000in}}{\pgfqpoint{4.960000in}{3.696000in}}%
\pgfusepath{clip}%
\pgfsetbuttcap%
\pgfsetroundjoin%
\definecolor{currentfill}{rgb}{0.000000,0.000000,0.000000}%
\pgfsetfillcolor{currentfill}%
\pgfsetlinewidth{1.003750pt}%
\definecolor{currentstroke}{rgb}{0.000000,0.000000,0.000000}%
\pgfsetstrokecolor{currentstroke}%
\pgfsetdash{}{0pt}%
\pgfpathmoveto{\pgfqpoint{4.011666in}{3.984333in}}%
\pgfpathcurveto{\pgfqpoint{4.022716in}{3.984333in}}{\pgfqpoint{4.033315in}{3.988724in}}{\pgfqpoint{4.041128in}{3.996537in}}%
\pgfpathcurveto{\pgfqpoint{4.048942in}{4.004351in}}{\pgfqpoint{4.053332in}{4.014950in}}{\pgfqpoint{4.053332in}{4.026000in}}%
\pgfpathcurveto{\pgfqpoint{4.053332in}{4.037050in}}{\pgfqpoint{4.048942in}{4.047649in}}{\pgfqpoint{4.041128in}{4.055463in}}%
\pgfpathcurveto{\pgfqpoint{4.033315in}{4.063276in}}{\pgfqpoint{4.022716in}{4.067667in}}{\pgfqpoint{4.011666in}{4.067667in}}%
\pgfpathcurveto{\pgfqpoint{4.000616in}{4.067667in}}{\pgfqpoint{3.990016in}{4.063276in}}{\pgfqpoint{3.982203in}{4.055463in}}%
\pgfpathcurveto{\pgfqpoint{3.974389in}{4.047649in}}{\pgfqpoint{3.969999in}{4.037050in}}{\pgfqpoint{3.969999in}{4.026000in}}%
\pgfpathcurveto{\pgfqpoint{3.969999in}{4.014950in}}{\pgfqpoint{3.974389in}{4.004351in}}{\pgfqpoint{3.982203in}{3.996537in}}%
\pgfpathcurveto{\pgfqpoint{3.990016in}{3.988724in}}{\pgfqpoint{4.000616in}{3.984333in}}{\pgfqpoint{4.011666in}{3.984333in}}%
\pgfpathclose%
\pgfusepath{stroke,fill}%
\end{pgfscope}%
\begin{pgfscope}%
\pgfpathrectangle{\pgfqpoint{0.800000in}{0.528000in}}{\pgfqpoint{4.960000in}{3.696000in}}%
\pgfusepath{clip}%
\pgfsetbuttcap%
\pgfsetroundjoin%
\definecolor{currentfill}{rgb}{0.000000,0.000000,0.000000}%
\pgfsetfillcolor{currentfill}%
\pgfsetlinewidth{1.003750pt}%
\definecolor{currentstroke}{rgb}{0.000000,0.000000,0.000000}%
\pgfsetstrokecolor{currentstroke}%
\pgfsetdash{}{0pt}%
\pgfpathmoveto{\pgfqpoint{4.011666in}{3.984333in}}%
\pgfpathcurveto{\pgfqpoint{4.022716in}{3.984333in}}{\pgfqpoint{4.033315in}{3.988724in}}{\pgfqpoint{4.041128in}{3.996537in}}%
\pgfpathcurveto{\pgfqpoint{4.048942in}{4.004351in}}{\pgfqpoint{4.053332in}{4.014950in}}{\pgfqpoint{4.053332in}{4.026000in}}%
\pgfpathcurveto{\pgfqpoint{4.053332in}{4.037050in}}{\pgfqpoint{4.048942in}{4.047649in}}{\pgfqpoint{4.041128in}{4.055463in}}%
\pgfpathcurveto{\pgfqpoint{4.033315in}{4.063276in}}{\pgfqpoint{4.022716in}{4.067667in}}{\pgfqpoint{4.011666in}{4.067667in}}%
\pgfpathcurveto{\pgfqpoint{4.000616in}{4.067667in}}{\pgfqpoint{3.990016in}{4.063276in}}{\pgfqpoint{3.982203in}{4.055463in}}%
\pgfpathcurveto{\pgfqpoint{3.974389in}{4.047649in}}{\pgfqpoint{3.969999in}{4.037050in}}{\pgfqpoint{3.969999in}{4.026000in}}%
\pgfpathcurveto{\pgfqpoint{3.969999in}{4.014950in}}{\pgfqpoint{3.974389in}{4.004351in}}{\pgfqpoint{3.982203in}{3.996537in}}%
\pgfpathcurveto{\pgfqpoint{3.990016in}{3.988724in}}{\pgfqpoint{4.000616in}{3.984333in}}{\pgfqpoint{4.011666in}{3.984333in}}%
\pgfpathclose%
\pgfusepath{stroke,fill}%
\end{pgfscope}%
\begin{pgfscope}%
\pgfpathrectangle{\pgfqpoint{0.800000in}{0.528000in}}{\pgfqpoint{4.960000in}{3.696000in}}%
\pgfusepath{clip}%
\pgfsetbuttcap%
\pgfsetroundjoin%
\definecolor{currentfill}{rgb}{0.000000,0.000000,0.000000}%
\pgfsetfillcolor{currentfill}%
\pgfsetlinewidth{1.003750pt}%
\definecolor{currentstroke}{rgb}{0.000000,0.000000,0.000000}%
\pgfsetstrokecolor{currentstroke}%
\pgfsetdash{}{0pt}%
\pgfpathmoveto{\pgfqpoint{4.011666in}{3.984333in}}%
\pgfpathcurveto{\pgfqpoint{4.022716in}{3.984333in}}{\pgfqpoint{4.033315in}{3.988724in}}{\pgfqpoint{4.041128in}{3.996537in}}%
\pgfpathcurveto{\pgfqpoint{4.048942in}{4.004351in}}{\pgfqpoint{4.053332in}{4.014950in}}{\pgfqpoint{4.053332in}{4.026000in}}%
\pgfpathcurveto{\pgfqpoint{4.053332in}{4.037050in}}{\pgfqpoint{4.048942in}{4.047649in}}{\pgfqpoint{4.041128in}{4.055463in}}%
\pgfpathcurveto{\pgfqpoint{4.033315in}{4.063276in}}{\pgfqpoint{4.022716in}{4.067667in}}{\pgfqpoint{4.011666in}{4.067667in}}%
\pgfpathcurveto{\pgfqpoint{4.000616in}{4.067667in}}{\pgfqpoint{3.990016in}{4.063276in}}{\pgfqpoint{3.982203in}{4.055463in}}%
\pgfpathcurveto{\pgfqpoint{3.974389in}{4.047649in}}{\pgfqpoint{3.969999in}{4.037050in}}{\pgfqpoint{3.969999in}{4.026000in}}%
\pgfpathcurveto{\pgfqpoint{3.969999in}{4.014950in}}{\pgfqpoint{3.974389in}{4.004351in}}{\pgfqpoint{3.982203in}{3.996537in}}%
\pgfpathcurveto{\pgfqpoint{3.990016in}{3.988724in}}{\pgfqpoint{4.000616in}{3.984333in}}{\pgfqpoint{4.011666in}{3.984333in}}%
\pgfpathclose%
\pgfusepath{stroke,fill}%
\end{pgfscope}%
\begin{pgfscope}%
\pgfpathrectangle{\pgfqpoint{0.800000in}{0.528000in}}{\pgfqpoint{4.960000in}{3.696000in}}%
\pgfusepath{clip}%
\pgfsetbuttcap%
\pgfsetroundjoin%
\definecolor{currentfill}{rgb}{0.000000,0.000000,0.000000}%
\pgfsetfillcolor{currentfill}%
\pgfsetlinewidth{1.003750pt}%
\definecolor{currentstroke}{rgb}{0.000000,0.000000,0.000000}%
\pgfsetstrokecolor{currentstroke}%
\pgfsetdash{}{0pt}%
\pgfpathmoveto{\pgfqpoint{4.011666in}{3.984333in}}%
\pgfpathcurveto{\pgfqpoint{4.022716in}{3.984333in}}{\pgfqpoint{4.033315in}{3.988724in}}{\pgfqpoint{4.041128in}{3.996537in}}%
\pgfpathcurveto{\pgfqpoint{4.048942in}{4.004351in}}{\pgfqpoint{4.053332in}{4.014950in}}{\pgfqpoint{4.053332in}{4.026000in}}%
\pgfpathcurveto{\pgfqpoint{4.053332in}{4.037050in}}{\pgfqpoint{4.048942in}{4.047649in}}{\pgfqpoint{4.041128in}{4.055463in}}%
\pgfpathcurveto{\pgfqpoint{4.033315in}{4.063276in}}{\pgfqpoint{4.022716in}{4.067667in}}{\pgfqpoint{4.011666in}{4.067667in}}%
\pgfpathcurveto{\pgfqpoint{4.000616in}{4.067667in}}{\pgfqpoint{3.990016in}{4.063276in}}{\pgfqpoint{3.982203in}{4.055463in}}%
\pgfpathcurveto{\pgfqpoint{3.974389in}{4.047649in}}{\pgfqpoint{3.969999in}{4.037050in}}{\pgfqpoint{3.969999in}{4.026000in}}%
\pgfpathcurveto{\pgfqpoint{3.969999in}{4.014950in}}{\pgfqpoint{3.974389in}{4.004351in}}{\pgfqpoint{3.982203in}{3.996537in}}%
\pgfpathcurveto{\pgfqpoint{3.990016in}{3.988724in}}{\pgfqpoint{4.000616in}{3.984333in}}{\pgfqpoint{4.011666in}{3.984333in}}%
\pgfpathclose%
\pgfusepath{stroke,fill}%
\end{pgfscope}%
\begin{pgfscope}%
\pgfpathrectangle{\pgfqpoint{0.800000in}{0.528000in}}{\pgfqpoint{4.960000in}{3.696000in}}%
\pgfusepath{clip}%
\pgfsetbuttcap%
\pgfsetroundjoin%
\definecolor{currentfill}{rgb}{0.000000,0.000000,0.000000}%
\pgfsetfillcolor{currentfill}%
\pgfsetlinewidth{1.003750pt}%
\definecolor{currentstroke}{rgb}{0.000000,0.000000,0.000000}%
\pgfsetstrokecolor{currentstroke}%
\pgfsetdash{}{0pt}%
\pgfpathmoveto{\pgfqpoint{4.011666in}{3.984333in}}%
\pgfpathcurveto{\pgfqpoint{4.022716in}{3.984333in}}{\pgfqpoint{4.033315in}{3.988724in}}{\pgfqpoint{4.041128in}{3.996537in}}%
\pgfpathcurveto{\pgfqpoint{4.048942in}{4.004351in}}{\pgfqpoint{4.053332in}{4.014950in}}{\pgfqpoint{4.053332in}{4.026000in}}%
\pgfpathcurveto{\pgfqpoint{4.053332in}{4.037050in}}{\pgfqpoint{4.048942in}{4.047649in}}{\pgfqpoint{4.041128in}{4.055463in}}%
\pgfpathcurveto{\pgfqpoint{4.033315in}{4.063276in}}{\pgfqpoint{4.022716in}{4.067667in}}{\pgfqpoint{4.011666in}{4.067667in}}%
\pgfpathcurveto{\pgfqpoint{4.000616in}{4.067667in}}{\pgfqpoint{3.990016in}{4.063276in}}{\pgfqpoint{3.982203in}{4.055463in}}%
\pgfpathcurveto{\pgfqpoint{3.974389in}{4.047649in}}{\pgfqpoint{3.969999in}{4.037050in}}{\pgfqpoint{3.969999in}{4.026000in}}%
\pgfpathcurveto{\pgfqpoint{3.969999in}{4.014950in}}{\pgfqpoint{3.974389in}{4.004351in}}{\pgfqpoint{3.982203in}{3.996537in}}%
\pgfpathcurveto{\pgfqpoint{3.990016in}{3.988724in}}{\pgfqpoint{4.000616in}{3.984333in}}{\pgfqpoint{4.011666in}{3.984333in}}%
\pgfpathclose%
\pgfusepath{stroke,fill}%
\end{pgfscope}%
\begin{pgfscope}%
\pgfpathrectangle{\pgfqpoint{0.800000in}{0.528000in}}{\pgfqpoint{4.960000in}{3.696000in}}%
\pgfusepath{clip}%
\pgfsetbuttcap%
\pgfsetroundjoin%
\definecolor{currentfill}{rgb}{0.000000,0.000000,0.000000}%
\pgfsetfillcolor{currentfill}%
\pgfsetlinewidth{1.003750pt}%
\definecolor{currentstroke}{rgb}{0.000000,0.000000,0.000000}%
\pgfsetstrokecolor{currentstroke}%
\pgfsetdash{}{0pt}%
\pgfpathmoveto{\pgfqpoint{4.011666in}{3.984333in}}%
\pgfpathcurveto{\pgfqpoint{4.022716in}{3.984333in}}{\pgfqpoint{4.033315in}{3.988724in}}{\pgfqpoint{4.041128in}{3.996537in}}%
\pgfpathcurveto{\pgfqpoint{4.048942in}{4.004351in}}{\pgfqpoint{4.053332in}{4.014950in}}{\pgfqpoint{4.053332in}{4.026000in}}%
\pgfpathcurveto{\pgfqpoint{4.053332in}{4.037050in}}{\pgfqpoint{4.048942in}{4.047649in}}{\pgfqpoint{4.041128in}{4.055463in}}%
\pgfpathcurveto{\pgfqpoint{4.033315in}{4.063276in}}{\pgfqpoint{4.022716in}{4.067667in}}{\pgfqpoint{4.011666in}{4.067667in}}%
\pgfpathcurveto{\pgfqpoint{4.000616in}{4.067667in}}{\pgfqpoint{3.990016in}{4.063276in}}{\pgfqpoint{3.982203in}{4.055463in}}%
\pgfpathcurveto{\pgfqpoint{3.974389in}{4.047649in}}{\pgfqpoint{3.969999in}{4.037050in}}{\pgfqpoint{3.969999in}{4.026000in}}%
\pgfpathcurveto{\pgfqpoint{3.969999in}{4.014950in}}{\pgfqpoint{3.974389in}{4.004351in}}{\pgfqpoint{3.982203in}{3.996537in}}%
\pgfpathcurveto{\pgfqpoint{3.990016in}{3.988724in}}{\pgfqpoint{4.000616in}{3.984333in}}{\pgfqpoint{4.011666in}{3.984333in}}%
\pgfpathclose%
\pgfusepath{stroke,fill}%
\end{pgfscope}%
\begin{pgfscope}%
\pgfpathrectangle{\pgfqpoint{0.800000in}{0.528000in}}{\pgfqpoint{4.960000in}{3.696000in}}%
\pgfusepath{clip}%
\pgfsetbuttcap%
\pgfsetroundjoin%
\definecolor{currentfill}{rgb}{0.000000,0.000000,0.000000}%
\pgfsetfillcolor{currentfill}%
\pgfsetlinewidth{1.003750pt}%
\definecolor{currentstroke}{rgb}{0.000000,0.000000,0.000000}%
\pgfsetstrokecolor{currentstroke}%
\pgfsetdash{}{0pt}%
\pgfpathmoveto{\pgfqpoint{4.011666in}{3.984333in}}%
\pgfpathcurveto{\pgfqpoint{4.022716in}{3.984333in}}{\pgfqpoint{4.033315in}{3.988724in}}{\pgfqpoint{4.041128in}{3.996537in}}%
\pgfpathcurveto{\pgfqpoint{4.048942in}{4.004351in}}{\pgfqpoint{4.053332in}{4.014950in}}{\pgfqpoint{4.053332in}{4.026000in}}%
\pgfpathcurveto{\pgfqpoint{4.053332in}{4.037050in}}{\pgfqpoint{4.048942in}{4.047649in}}{\pgfqpoint{4.041128in}{4.055463in}}%
\pgfpathcurveto{\pgfqpoint{4.033315in}{4.063276in}}{\pgfqpoint{4.022716in}{4.067667in}}{\pgfqpoint{4.011666in}{4.067667in}}%
\pgfpathcurveto{\pgfqpoint{4.000616in}{4.067667in}}{\pgfqpoint{3.990016in}{4.063276in}}{\pgfqpoint{3.982203in}{4.055463in}}%
\pgfpathcurveto{\pgfqpoint{3.974389in}{4.047649in}}{\pgfqpoint{3.969999in}{4.037050in}}{\pgfqpoint{3.969999in}{4.026000in}}%
\pgfpathcurveto{\pgfqpoint{3.969999in}{4.014950in}}{\pgfqpoint{3.974389in}{4.004351in}}{\pgfqpoint{3.982203in}{3.996537in}}%
\pgfpathcurveto{\pgfqpoint{3.990016in}{3.988724in}}{\pgfqpoint{4.000616in}{3.984333in}}{\pgfqpoint{4.011666in}{3.984333in}}%
\pgfpathclose%
\pgfusepath{stroke,fill}%
\end{pgfscope}%
\begin{pgfscope}%
\pgfpathrectangle{\pgfqpoint{0.800000in}{0.528000in}}{\pgfqpoint{4.960000in}{3.696000in}}%
\pgfusepath{clip}%
\pgfsetbuttcap%
\pgfsetroundjoin%
\definecolor{currentfill}{rgb}{0.000000,0.000000,0.000000}%
\pgfsetfillcolor{currentfill}%
\pgfsetlinewidth{1.003750pt}%
\definecolor{currentstroke}{rgb}{0.000000,0.000000,0.000000}%
\pgfsetstrokecolor{currentstroke}%
\pgfsetdash{}{0pt}%
\pgfpathmoveto{\pgfqpoint{5.504545in}{0.684333in}}%
\pgfpathcurveto{\pgfqpoint{5.515596in}{0.684333in}}{\pgfqpoint{5.526195in}{0.688724in}}{\pgfqpoint{5.534008in}{0.696537in}}%
\pgfpathcurveto{\pgfqpoint{5.541822in}{0.704351in}}{\pgfqpoint{5.546212in}{0.714950in}}{\pgfqpoint{5.546212in}{0.726000in}}%
\pgfpathcurveto{\pgfqpoint{5.546212in}{0.737050in}}{\pgfqpoint{5.541822in}{0.747649in}}{\pgfqpoint{5.534008in}{0.755463in}}%
\pgfpathcurveto{\pgfqpoint{5.526195in}{0.763276in}}{\pgfqpoint{5.515596in}{0.767667in}}{\pgfqpoint{5.504545in}{0.767667in}}%
\pgfpathcurveto{\pgfqpoint{5.493495in}{0.767667in}}{\pgfqpoint{5.482896in}{0.763276in}}{\pgfqpoint{5.475083in}{0.755463in}}%
\pgfpathcurveto{\pgfqpoint{5.467269in}{0.747649in}}{\pgfqpoint{5.462879in}{0.737050in}}{\pgfqpoint{5.462879in}{0.726000in}}%
\pgfpathcurveto{\pgfqpoint{5.462879in}{0.714950in}}{\pgfqpoint{5.467269in}{0.704351in}}{\pgfqpoint{5.475083in}{0.696537in}}%
\pgfpathcurveto{\pgfqpoint{5.482896in}{0.688724in}}{\pgfqpoint{5.493495in}{0.684333in}}{\pgfqpoint{5.504545in}{0.684333in}}%
\pgfpathclose%
\pgfusepath{stroke,fill}%
\end{pgfscope}%
\begin{pgfscope}%
\pgfpathrectangle{\pgfqpoint{0.800000in}{0.528000in}}{\pgfqpoint{4.960000in}{3.696000in}}%
\pgfusepath{clip}%
\pgfsetbuttcap%
\pgfsetroundjoin%
\definecolor{currentfill}{rgb}{0.000000,0.000000,0.000000}%
\pgfsetfillcolor{currentfill}%
\pgfsetlinewidth{1.003750pt}%
\definecolor{currentstroke}{rgb}{0.000000,0.000000,0.000000}%
\pgfsetstrokecolor{currentstroke}%
\pgfsetdash{}{0pt}%
\pgfpathmoveto{\pgfqpoint{5.504545in}{0.684333in}}%
\pgfpathcurveto{\pgfqpoint{5.515596in}{0.684333in}}{\pgfqpoint{5.526195in}{0.688724in}}{\pgfqpoint{5.534008in}{0.696537in}}%
\pgfpathcurveto{\pgfqpoint{5.541822in}{0.704351in}}{\pgfqpoint{5.546212in}{0.714950in}}{\pgfqpoint{5.546212in}{0.726000in}}%
\pgfpathcurveto{\pgfqpoint{5.546212in}{0.737050in}}{\pgfqpoint{5.541822in}{0.747649in}}{\pgfqpoint{5.534008in}{0.755463in}}%
\pgfpathcurveto{\pgfqpoint{5.526195in}{0.763276in}}{\pgfqpoint{5.515596in}{0.767667in}}{\pgfqpoint{5.504545in}{0.767667in}}%
\pgfpathcurveto{\pgfqpoint{5.493495in}{0.767667in}}{\pgfqpoint{5.482896in}{0.763276in}}{\pgfqpoint{5.475083in}{0.755463in}}%
\pgfpathcurveto{\pgfqpoint{5.467269in}{0.747649in}}{\pgfqpoint{5.462879in}{0.737050in}}{\pgfqpoint{5.462879in}{0.726000in}}%
\pgfpathcurveto{\pgfqpoint{5.462879in}{0.714950in}}{\pgfqpoint{5.467269in}{0.704351in}}{\pgfqpoint{5.475083in}{0.696537in}}%
\pgfpathcurveto{\pgfqpoint{5.482896in}{0.688724in}}{\pgfqpoint{5.493495in}{0.684333in}}{\pgfqpoint{5.504545in}{0.684333in}}%
\pgfpathclose%
\pgfusepath{stroke,fill}%
\end{pgfscope}%
\begin{pgfscope}%
\pgfpathrectangle{\pgfqpoint{0.800000in}{0.528000in}}{\pgfqpoint{4.960000in}{3.696000in}}%
\pgfusepath{clip}%
\pgfsetbuttcap%
\pgfsetroundjoin%
\definecolor{currentfill}{rgb}{0.000000,0.000000,0.000000}%
\pgfsetfillcolor{currentfill}%
\pgfsetlinewidth{1.003750pt}%
\definecolor{currentstroke}{rgb}{0.000000,0.000000,0.000000}%
\pgfsetstrokecolor{currentstroke}%
\pgfsetdash{}{0pt}%
\pgfpathmoveto{\pgfqpoint{5.504545in}{0.684333in}}%
\pgfpathcurveto{\pgfqpoint{5.515596in}{0.684333in}}{\pgfqpoint{5.526195in}{0.688724in}}{\pgfqpoint{5.534008in}{0.696537in}}%
\pgfpathcurveto{\pgfqpoint{5.541822in}{0.704351in}}{\pgfqpoint{5.546212in}{0.714950in}}{\pgfqpoint{5.546212in}{0.726000in}}%
\pgfpathcurveto{\pgfqpoint{5.546212in}{0.737050in}}{\pgfqpoint{5.541822in}{0.747649in}}{\pgfqpoint{5.534008in}{0.755463in}}%
\pgfpathcurveto{\pgfqpoint{5.526195in}{0.763276in}}{\pgfqpoint{5.515596in}{0.767667in}}{\pgfqpoint{5.504545in}{0.767667in}}%
\pgfpathcurveto{\pgfqpoint{5.493495in}{0.767667in}}{\pgfqpoint{5.482896in}{0.763276in}}{\pgfqpoint{5.475083in}{0.755463in}}%
\pgfpathcurveto{\pgfqpoint{5.467269in}{0.747649in}}{\pgfqpoint{5.462879in}{0.737050in}}{\pgfqpoint{5.462879in}{0.726000in}}%
\pgfpathcurveto{\pgfqpoint{5.462879in}{0.714950in}}{\pgfqpoint{5.467269in}{0.704351in}}{\pgfqpoint{5.475083in}{0.696537in}}%
\pgfpathcurveto{\pgfqpoint{5.482896in}{0.688724in}}{\pgfqpoint{5.493495in}{0.684333in}}{\pgfqpoint{5.504545in}{0.684333in}}%
\pgfpathclose%
\pgfusepath{stroke,fill}%
\end{pgfscope}%
\begin{pgfscope}%
\pgfpathrectangle{\pgfqpoint{0.800000in}{0.528000in}}{\pgfqpoint{4.960000in}{3.696000in}}%
\pgfusepath{clip}%
\pgfsetbuttcap%
\pgfsetroundjoin%
\definecolor{currentfill}{rgb}{0.000000,0.000000,0.000000}%
\pgfsetfillcolor{currentfill}%
\pgfsetlinewidth{1.003750pt}%
\definecolor{currentstroke}{rgb}{0.000000,0.000000,0.000000}%
\pgfsetstrokecolor{currentstroke}%
\pgfsetdash{}{0pt}%
\pgfpathmoveto{\pgfqpoint{5.504545in}{0.684333in}}%
\pgfpathcurveto{\pgfqpoint{5.515596in}{0.684333in}}{\pgfqpoint{5.526195in}{0.688724in}}{\pgfqpoint{5.534008in}{0.696537in}}%
\pgfpathcurveto{\pgfqpoint{5.541822in}{0.704351in}}{\pgfqpoint{5.546212in}{0.714950in}}{\pgfqpoint{5.546212in}{0.726000in}}%
\pgfpathcurveto{\pgfqpoint{5.546212in}{0.737050in}}{\pgfqpoint{5.541822in}{0.747649in}}{\pgfqpoint{5.534008in}{0.755463in}}%
\pgfpathcurveto{\pgfqpoint{5.526195in}{0.763276in}}{\pgfqpoint{5.515596in}{0.767667in}}{\pgfqpoint{5.504545in}{0.767667in}}%
\pgfpathcurveto{\pgfqpoint{5.493495in}{0.767667in}}{\pgfqpoint{5.482896in}{0.763276in}}{\pgfqpoint{5.475083in}{0.755463in}}%
\pgfpathcurveto{\pgfqpoint{5.467269in}{0.747649in}}{\pgfqpoint{5.462879in}{0.737050in}}{\pgfqpoint{5.462879in}{0.726000in}}%
\pgfpathcurveto{\pgfqpoint{5.462879in}{0.714950in}}{\pgfqpoint{5.467269in}{0.704351in}}{\pgfqpoint{5.475083in}{0.696537in}}%
\pgfpathcurveto{\pgfqpoint{5.482896in}{0.688724in}}{\pgfqpoint{5.493495in}{0.684333in}}{\pgfqpoint{5.504545in}{0.684333in}}%
\pgfpathclose%
\pgfusepath{stroke,fill}%
\end{pgfscope}%
\begin{pgfscope}%
\pgfpathrectangle{\pgfqpoint{0.800000in}{0.528000in}}{\pgfqpoint{4.960000in}{3.696000in}}%
\pgfusepath{clip}%
\pgfsetbuttcap%
\pgfsetroundjoin%
\definecolor{currentfill}{rgb}{0.000000,0.000000,0.000000}%
\pgfsetfillcolor{currentfill}%
\pgfsetlinewidth{1.003750pt}%
\definecolor{currentstroke}{rgb}{0.000000,0.000000,0.000000}%
\pgfsetstrokecolor{currentstroke}%
\pgfsetdash{}{0pt}%
\pgfpathmoveto{\pgfqpoint{5.504545in}{0.684333in}}%
\pgfpathcurveto{\pgfqpoint{5.515596in}{0.684333in}}{\pgfqpoint{5.526195in}{0.688724in}}{\pgfqpoint{5.534008in}{0.696537in}}%
\pgfpathcurveto{\pgfqpoint{5.541822in}{0.704351in}}{\pgfqpoint{5.546212in}{0.714950in}}{\pgfqpoint{5.546212in}{0.726000in}}%
\pgfpathcurveto{\pgfqpoint{5.546212in}{0.737050in}}{\pgfqpoint{5.541822in}{0.747649in}}{\pgfqpoint{5.534008in}{0.755463in}}%
\pgfpathcurveto{\pgfqpoint{5.526195in}{0.763276in}}{\pgfqpoint{5.515596in}{0.767667in}}{\pgfqpoint{5.504545in}{0.767667in}}%
\pgfpathcurveto{\pgfqpoint{5.493495in}{0.767667in}}{\pgfqpoint{5.482896in}{0.763276in}}{\pgfqpoint{5.475083in}{0.755463in}}%
\pgfpathcurveto{\pgfqpoint{5.467269in}{0.747649in}}{\pgfqpoint{5.462879in}{0.737050in}}{\pgfqpoint{5.462879in}{0.726000in}}%
\pgfpathcurveto{\pgfqpoint{5.462879in}{0.714950in}}{\pgfqpoint{5.467269in}{0.704351in}}{\pgfqpoint{5.475083in}{0.696537in}}%
\pgfpathcurveto{\pgfqpoint{5.482896in}{0.688724in}}{\pgfqpoint{5.493495in}{0.684333in}}{\pgfqpoint{5.504545in}{0.684333in}}%
\pgfpathclose%
\pgfusepath{stroke,fill}%
\end{pgfscope}%
\begin{pgfscope}%
\pgfpathrectangle{\pgfqpoint{0.800000in}{0.528000in}}{\pgfqpoint{4.960000in}{3.696000in}}%
\pgfusepath{clip}%
\pgfsetbuttcap%
\pgfsetroundjoin%
\definecolor{currentfill}{rgb}{0.000000,0.000000,0.000000}%
\pgfsetfillcolor{currentfill}%
\pgfsetlinewidth{1.003750pt}%
\definecolor{currentstroke}{rgb}{0.000000,0.000000,0.000000}%
\pgfsetstrokecolor{currentstroke}%
\pgfsetdash{}{0pt}%
\pgfpathmoveto{\pgfqpoint{5.504545in}{0.684333in}}%
\pgfpathcurveto{\pgfqpoint{5.515596in}{0.684333in}}{\pgfqpoint{5.526195in}{0.688724in}}{\pgfqpoint{5.534008in}{0.696537in}}%
\pgfpathcurveto{\pgfqpoint{5.541822in}{0.704351in}}{\pgfqpoint{5.546212in}{0.714950in}}{\pgfqpoint{5.546212in}{0.726000in}}%
\pgfpathcurveto{\pgfqpoint{5.546212in}{0.737050in}}{\pgfqpoint{5.541822in}{0.747649in}}{\pgfqpoint{5.534008in}{0.755463in}}%
\pgfpathcurveto{\pgfqpoint{5.526195in}{0.763276in}}{\pgfqpoint{5.515596in}{0.767667in}}{\pgfqpoint{5.504545in}{0.767667in}}%
\pgfpathcurveto{\pgfqpoint{5.493495in}{0.767667in}}{\pgfqpoint{5.482896in}{0.763276in}}{\pgfqpoint{5.475083in}{0.755463in}}%
\pgfpathcurveto{\pgfqpoint{5.467269in}{0.747649in}}{\pgfqpoint{5.462879in}{0.737050in}}{\pgfqpoint{5.462879in}{0.726000in}}%
\pgfpathcurveto{\pgfqpoint{5.462879in}{0.714950in}}{\pgfqpoint{5.467269in}{0.704351in}}{\pgfqpoint{5.475083in}{0.696537in}}%
\pgfpathcurveto{\pgfqpoint{5.482896in}{0.688724in}}{\pgfqpoint{5.493495in}{0.684333in}}{\pgfqpoint{5.504545in}{0.684333in}}%
\pgfpathclose%
\pgfusepath{stroke,fill}%
\end{pgfscope}%
\begin{pgfscope}%
\pgfpathrectangle{\pgfqpoint{0.800000in}{0.528000in}}{\pgfqpoint{4.960000in}{3.696000in}}%
\pgfusepath{clip}%
\pgfsetbuttcap%
\pgfsetroundjoin%
\definecolor{currentfill}{rgb}{0.000000,0.000000,0.000000}%
\pgfsetfillcolor{currentfill}%
\pgfsetlinewidth{1.003750pt}%
\definecolor{currentstroke}{rgb}{0.000000,0.000000,0.000000}%
\pgfsetstrokecolor{currentstroke}%
\pgfsetdash{}{0pt}%
\pgfpathmoveto{\pgfqpoint{5.504545in}{3.984333in}}%
\pgfpathcurveto{\pgfqpoint{5.515596in}{3.984333in}}{\pgfqpoint{5.526195in}{3.988724in}}{\pgfqpoint{5.534008in}{3.996537in}}%
\pgfpathcurveto{\pgfqpoint{5.541822in}{4.004351in}}{\pgfqpoint{5.546212in}{4.014950in}}{\pgfqpoint{5.546212in}{4.026000in}}%
\pgfpathcurveto{\pgfqpoint{5.546212in}{4.037050in}}{\pgfqpoint{5.541822in}{4.047649in}}{\pgfqpoint{5.534008in}{4.055463in}}%
\pgfpathcurveto{\pgfqpoint{5.526195in}{4.063276in}}{\pgfqpoint{5.515596in}{4.067667in}}{\pgfqpoint{5.504545in}{4.067667in}}%
\pgfpathcurveto{\pgfqpoint{5.493495in}{4.067667in}}{\pgfqpoint{5.482896in}{4.063276in}}{\pgfqpoint{5.475083in}{4.055463in}}%
\pgfpathcurveto{\pgfqpoint{5.467269in}{4.047649in}}{\pgfqpoint{5.462879in}{4.037050in}}{\pgfqpoint{5.462879in}{4.026000in}}%
\pgfpathcurveto{\pgfqpoint{5.462879in}{4.014950in}}{\pgfqpoint{5.467269in}{4.004351in}}{\pgfqpoint{5.475083in}{3.996537in}}%
\pgfpathcurveto{\pgfqpoint{5.482896in}{3.988724in}}{\pgfqpoint{5.493495in}{3.984333in}}{\pgfqpoint{5.504545in}{3.984333in}}%
\pgfpathclose%
\pgfusepath{stroke,fill}%
\end{pgfscope}%
\begin{pgfscope}%
\pgfpathrectangle{\pgfqpoint{0.800000in}{0.528000in}}{\pgfqpoint{4.960000in}{3.696000in}}%
\pgfusepath{clip}%
\pgfsetbuttcap%
\pgfsetroundjoin%
\definecolor{currentfill}{rgb}{0.000000,0.000000,0.000000}%
\pgfsetfillcolor{currentfill}%
\pgfsetlinewidth{1.003750pt}%
\definecolor{currentstroke}{rgb}{0.000000,0.000000,0.000000}%
\pgfsetstrokecolor{currentstroke}%
\pgfsetdash{}{0pt}%
\pgfpathmoveto{\pgfqpoint{5.504545in}{3.984333in}}%
\pgfpathcurveto{\pgfqpoint{5.515596in}{3.984333in}}{\pgfqpoint{5.526195in}{3.988724in}}{\pgfqpoint{5.534008in}{3.996537in}}%
\pgfpathcurveto{\pgfqpoint{5.541822in}{4.004351in}}{\pgfqpoint{5.546212in}{4.014950in}}{\pgfqpoint{5.546212in}{4.026000in}}%
\pgfpathcurveto{\pgfqpoint{5.546212in}{4.037050in}}{\pgfqpoint{5.541822in}{4.047649in}}{\pgfqpoint{5.534008in}{4.055463in}}%
\pgfpathcurveto{\pgfqpoint{5.526195in}{4.063276in}}{\pgfqpoint{5.515596in}{4.067667in}}{\pgfqpoint{5.504545in}{4.067667in}}%
\pgfpathcurveto{\pgfqpoint{5.493495in}{4.067667in}}{\pgfqpoint{5.482896in}{4.063276in}}{\pgfqpoint{5.475083in}{4.055463in}}%
\pgfpathcurveto{\pgfqpoint{5.467269in}{4.047649in}}{\pgfqpoint{5.462879in}{4.037050in}}{\pgfqpoint{5.462879in}{4.026000in}}%
\pgfpathcurveto{\pgfqpoint{5.462879in}{4.014950in}}{\pgfqpoint{5.467269in}{4.004351in}}{\pgfqpoint{5.475083in}{3.996537in}}%
\pgfpathcurveto{\pgfqpoint{5.482896in}{3.988724in}}{\pgfqpoint{5.493495in}{3.984333in}}{\pgfqpoint{5.504545in}{3.984333in}}%
\pgfpathclose%
\pgfusepath{stroke,fill}%
\end{pgfscope}%
\begin{pgfscope}%
\pgfpathrectangle{\pgfqpoint{0.800000in}{0.528000in}}{\pgfqpoint{4.960000in}{3.696000in}}%
\pgfusepath{clip}%
\pgfsetbuttcap%
\pgfsetroundjoin%
\definecolor{currentfill}{rgb}{0.000000,0.000000,0.000000}%
\pgfsetfillcolor{currentfill}%
\pgfsetlinewidth{1.003750pt}%
\definecolor{currentstroke}{rgb}{0.000000,0.000000,0.000000}%
\pgfsetstrokecolor{currentstroke}%
\pgfsetdash{}{0pt}%
\pgfpathmoveto{\pgfqpoint{5.504545in}{0.684333in}}%
\pgfpathcurveto{\pgfqpoint{5.515596in}{0.684333in}}{\pgfqpoint{5.526195in}{0.688724in}}{\pgfqpoint{5.534008in}{0.696537in}}%
\pgfpathcurveto{\pgfqpoint{5.541822in}{0.704351in}}{\pgfqpoint{5.546212in}{0.714950in}}{\pgfqpoint{5.546212in}{0.726000in}}%
\pgfpathcurveto{\pgfqpoint{5.546212in}{0.737050in}}{\pgfqpoint{5.541822in}{0.747649in}}{\pgfqpoint{5.534008in}{0.755463in}}%
\pgfpathcurveto{\pgfqpoint{5.526195in}{0.763276in}}{\pgfqpoint{5.515596in}{0.767667in}}{\pgfqpoint{5.504545in}{0.767667in}}%
\pgfpathcurveto{\pgfqpoint{5.493495in}{0.767667in}}{\pgfqpoint{5.482896in}{0.763276in}}{\pgfqpoint{5.475083in}{0.755463in}}%
\pgfpathcurveto{\pgfqpoint{5.467269in}{0.747649in}}{\pgfqpoint{5.462879in}{0.737050in}}{\pgfqpoint{5.462879in}{0.726000in}}%
\pgfpathcurveto{\pgfqpoint{5.462879in}{0.714950in}}{\pgfqpoint{5.467269in}{0.704351in}}{\pgfqpoint{5.475083in}{0.696537in}}%
\pgfpathcurveto{\pgfqpoint{5.482896in}{0.688724in}}{\pgfqpoint{5.493495in}{0.684333in}}{\pgfqpoint{5.504545in}{0.684333in}}%
\pgfpathclose%
\pgfusepath{stroke,fill}%
\end{pgfscope}%
\begin{pgfscope}%
\pgfpathrectangle{\pgfqpoint{0.800000in}{0.528000in}}{\pgfqpoint{4.960000in}{3.696000in}}%
\pgfusepath{clip}%
\pgfsetbuttcap%
\pgfsetroundjoin%
\definecolor{currentfill}{rgb}{0.000000,0.000000,0.000000}%
\pgfsetfillcolor{currentfill}%
\pgfsetlinewidth{1.003750pt}%
\definecolor{currentstroke}{rgb}{0.000000,0.000000,0.000000}%
\pgfsetstrokecolor{currentstroke}%
\pgfsetdash{}{0pt}%
\pgfpathmoveto{\pgfqpoint{5.504545in}{0.684333in}}%
\pgfpathcurveto{\pgfqpoint{5.515596in}{0.684333in}}{\pgfqpoint{5.526195in}{0.688724in}}{\pgfqpoint{5.534008in}{0.696537in}}%
\pgfpathcurveto{\pgfqpoint{5.541822in}{0.704351in}}{\pgfqpoint{5.546212in}{0.714950in}}{\pgfqpoint{5.546212in}{0.726000in}}%
\pgfpathcurveto{\pgfqpoint{5.546212in}{0.737050in}}{\pgfqpoint{5.541822in}{0.747649in}}{\pgfqpoint{5.534008in}{0.755463in}}%
\pgfpathcurveto{\pgfqpoint{5.526195in}{0.763276in}}{\pgfqpoint{5.515596in}{0.767667in}}{\pgfqpoint{5.504545in}{0.767667in}}%
\pgfpathcurveto{\pgfqpoint{5.493495in}{0.767667in}}{\pgfqpoint{5.482896in}{0.763276in}}{\pgfqpoint{5.475083in}{0.755463in}}%
\pgfpathcurveto{\pgfqpoint{5.467269in}{0.747649in}}{\pgfqpoint{5.462879in}{0.737050in}}{\pgfqpoint{5.462879in}{0.726000in}}%
\pgfpathcurveto{\pgfqpoint{5.462879in}{0.714950in}}{\pgfqpoint{5.467269in}{0.704351in}}{\pgfqpoint{5.475083in}{0.696537in}}%
\pgfpathcurveto{\pgfqpoint{5.482896in}{0.688724in}}{\pgfqpoint{5.493495in}{0.684333in}}{\pgfqpoint{5.504545in}{0.684333in}}%
\pgfpathclose%
\pgfusepath{stroke,fill}%
\end{pgfscope}%
\begin{pgfscope}%
\pgfpathrectangle{\pgfqpoint{0.800000in}{0.528000in}}{\pgfqpoint{4.960000in}{3.696000in}}%
\pgfusepath{clip}%
\pgfsetbuttcap%
\pgfsetroundjoin%
\definecolor{currentfill}{rgb}{0.000000,0.000000,0.000000}%
\pgfsetfillcolor{currentfill}%
\pgfsetlinewidth{1.003750pt}%
\definecolor{currentstroke}{rgb}{0.000000,0.000000,0.000000}%
\pgfsetstrokecolor{currentstroke}%
\pgfsetdash{}{0pt}%
\pgfpathmoveto{\pgfqpoint{5.504545in}{3.984333in}}%
\pgfpathcurveto{\pgfqpoint{5.515596in}{3.984333in}}{\pgfqpoint{5.526195in}{3.988724in}}{\pgfqpoint{5.534008in}{3.996537in}}%
\pgfpathcurveto{\pgfqpoint{5.541822in}{4.004351in}}{\pgfqpoint{5.546212in}{4.014950in}}{\pgfqpoint{5.546212in}{4.026000in}}%
\pgfpathcurveto{\pgfqpoint{5.546212in}{4.037050in}}{\pgfqpoint{5.541822in}{4.047649in}}{\pgfqpoint{5.534008in}{4.055463in}}%
\pgfpathcurveto{\pgfqpoint{5.526195in}{4.063276in}}{\pgfqpoint{5.515596in}{4.067667in}}{\pgfqpoint{5.504545in}{4.067667in}}%
\pgfpathcurveto{\pgfqpoint{5.493495in}{4.067667in}}{\pgfqpoint{5.482896in}{4.063276in}}{\pgfqpoint{5.475083in}{4.055463in}}%
\pgfpathcurveto{\pgfqpoint{5.467269in}{4.047649in}}{\pgfqpoint{5.462879in}{4.037050in}}{\pgfqpoint{5.462879in}{4.026000in}}%
\pgfpathcurveto{\pgfqpoint{5.462879in}{4.014950in}}{\pgfqpoint{5.467269in}{4.004351in}}{\pgfqpoint{5.475083in}{3.996537in}}%
\pgfpathcurveto{\pgfqpoint{5.482896in}{3.988724in}}{\pgfqpoint{5.493495in}{3.984333in}}{\pgfqpoint{5.504545in}{3.984333in}}%
\pgfpathclose%
\pgfusepath{stroke,fill}%
\end{pgfscope}%
\begin{pgfscope}%
\pgfpathrectangle{\pgfqpoint{0.800000in}{0.528000in}}{\pgfqpoint{4.960000in}{3.696000in}}%
\pgfusepath{clip}%
\pgfsetbuttcap%
\pgfsetroundjoin%
\definecolor{currentfill}{rgb}{0.000000,0.000000,0.000000}%
\pgfsetfillcolor{currentfill}%
\pgfsetlinewidth{1.003750pt}%
\definecolor{currentstroke}{rgb}{0.000000,0.000000,0.000000}%
\pgfsetstrokecolor{currentstroke}%
\pgfsetdash{}{0pt}%
\pgfpathmoveto{\pgfqpoint{5.504545in}{0.684333in}}%
\pgfpathcurveto{\pgfqpoint{5.515596in}{0.684333in}}{\pgfqpoint{5.526195in}{0.688724in}}{\pgfqpoint{5.534008in}{0.696537in}}%
\pgfpathcurveto{\pgfqpoint{5.541822in}{0.704351in}}{\pgfqpoint{5.546212in}{0.714950in}}{\pgfqpoint{5.546212in}{0.726000in}}%
\pgfpathcurveto{\pgfqpoint{5.546212in}{0.737050in}}{\pgfqpoint{5.541822in}{0.747649in}}{\pgfqpoint{5.534008in}{0.755463in}}%
\pgfpathcurveto{\pgfqpoint{5.526195in}{0.763276in}}{\pgfqpoint{5.515596in}{0.767667in}}{\pgfqpoint{5.504545in}{0.767667in}}%
\pgfpathcurveto{\pgfqpoint{5.493495in}{0.767667in}}{\pgfqpoint{5.482896in}{0.763276in}}{\pgfqpoint{5.475083in}{0.755463in}}%
\pgfpathcurveto{\pgfqpoint{5.467269in}{0.747649in}}{\pgfqpoint{5.462879in}{0.737050in}}{\pgfqpoint{5.462879in}{0.726000in}}%
\pgfpathcurveto{\pgfqpoint{5.462879in}{0.714950in}}{\pgfqpoint{5.467269in}{0.704351in}}{\pgfqpoint{5.475083in}{0.696537in}}%
\pgfpathcurveto{\pgfqpoint{5.482896in}{0.688724in}}{\pgfqpoint{5.493495in}{0.684333in}}{\pgfqpoint{5.504545in}{0.684333in}}%
\pgfpathclose%
\pgfusepath{stroke,fill}%
\end{pgfscope}%
\begin{pgfscope}%
\pgfpathrectangle{\pgfqpoint{0.800000in}{0.528000in}}{\pgfqpoint{4.960000in}{3.696000in}}%
\pgfusepath{clip}%
\pgfsetbuttcap%
\pgfsetroundjoin%
\definecolor{currentfill}{rgb}{0.000000,0.000000,0.000000}%
\pgfsetfillcolor{currentfill}%
\pgfsetlinewidth{1.003750pt}%
\definecolor{currentstroke}{rgb}{0.000000,0.000000,0.000000}%
\pgfsetstrokecolor{currentstroke}%
\pgfsetdash{}{0pt}%
\pgfpathmoveto{\pgfqpoint{5.504545in}{3.984333in}}%
\pgfpathcurveto{\pgfqpoint{5.515596in}{3.984333in}}{\pgfqpoint{5.526195in}{3.988724in}}{\pgfqpoint{5.534008in}{3.996537in}}%
\pgfpathcurveto{\pgfqpoint{5.541822in}{4.004351in}}{\pgfqpoint{5.546212in}{4.014950in}}{\pgfqpoint{5.546212in}{4.026000in}}%
\pgfpathcurveto{\pgfqpoint{5.546212in}{4.037050in}}{\pgfqpoint{5.541822in}{4.047649in}}{\pgfqpoint{5.534008in}{4.055463in}}%
\pgfpathcurveto{\pgfqpoint{5.526195in}{4.063276in}}{\pgfqpoint{5.515596in}{4.067667in}}{\pgfqpoint{5.504545in}{4.067667in}}%
\pgfpathcurveto{\pgfqpoint{5.493495in}{4.067667in}}{\pgfqpoint{5.482896in}{4.063276in}}{\pgfqpoint{5.475083in}{4.055463in}}%
\pgfpathcurveto{\pgfqpoint{5.467269in}{4.047649in}}{\pgfqpoint{5.462879in}{4.037050in}}{\pgfqpoint{5.462879in}{4.026000in}}%
\pgfpathcurveto{\pgfqpoint{5.462879in}{4.014950in}}{\pgfqpoint{5.467269in}{4.004351in}}{\pgfqpoint{5.475083in}{3.996537in}}%
\pgfpathcurveto{\pgfqpoint{5.482896in}{3.988724in}}{\pgfqpoint{5.493495in}{3.984333in}}{\pgfqpoint{5.504545in}{3.984333in}}%
\pgfpathclose%
\pgfusepath{stroke,fill}%
\end{pgfscope}%
\begin{pgfscope}%
\pgfpathrectangle{\pgfqpoint{0.800000in}{0.528000in}}{\pgfqpoint{4.960000in}{3.696000in}}%
\pgfusepath{clip}%
\pgfsetbuttcap%
\pgfsetroundjoin%
\definecolor{currentfill}{rgb}{0.000000,0.000000,0.000000}%
\pgfsetfillcolor{currentfill}%
\pgfsetlinewidth{1.003750pt}%
\definecolor{currentstroke}{rgb}{0.000000,0.000000,0.000000}%
\pgfsetstrokecolor{currentstroke}%
\pgfsetdash{}{0pt}%
\pgfpathmoveto{\pgfqpoint{5.504545in}{0.684333in}}%
\pgfpathcurveto{\pgfqpoint{5.515596in}{0.684333in}}{\pgfqpoint{5.526195in}{0.688724in}}{\pgfqpoint{5.534008in}{0.696537in}}%
\pgfpathcurveto{\pgfqpoint{5.541822in}{0.704351in}}{\pgfqpoint{5.546212in}{0.714950in}}{\pgfqpoint{5.546212in}{0.726000in}}%
\pgfpathcurveto{\pgfqpoint{5.546212in}{0.737050in}}{\pgfqpoint{5.541822in}{0.747649in}}{\pgfqpoint{5.534008in}{0.755463in}}%
\pgfpathcurveto{\pgfqpoint{5.526195in}{0.763276in}}{\pgfqpoint{5.515596in}{0.767667in}}{\pgfqpoint{5.504545in}{0.767667in}}%
\pgfpathcurveto{\pgfqpoint{5.493495in}{0.767667in}}{\pgfqpoint{5.482896in}{0.763276in}}{\pgfqpoint{5.475083in}{0.755463in}}%
\pgfpathcurveto{\pgfqpoint{5.467269in}{0.747649in}}{\pgfqpoint{5.462879in}{0.737050in}}{\pgfqpoint{5.462879in}{0.726000in}}%
\pgfpathcurveto{\pgfqpoint{5.462879in}{0.714950in}}{\pgfqpoint{5.467269in}{0.704351in}}{\pgfqpoint{5.475083in}{0.696537in}}%
\pgfpathcurveto{\pgfqpoint{5.482896in}{0.688724in}}{\pgfqpoint{5.493495in}{0.684333in}}{\pgfqpoint{5.504545in}{0.684333in}}%
\pgfpathclose%
\pgfusepath{stroke,fill}%
\end{pgfscope}%
\begin{pgfscope}%
\pgfpathrectangle{\pgfqpoint{0.800000in}{0.528000in}}{\pgfqpoint{4.960000in}{3.696000in}}%
\pgfusepath{clip}%
\pgfsetbuttcap%
\pgfsetroundjoin%
\definecolor{currentfill}{rgb}{0.000000,0.000000,0.000000}%
\pgfsetfillcolor{currentfill}%
\pgfsetlinewidth{1.003750pt}%
\definecolor{currentstroke}{rgb}{0.000000,0.000000,0.000000}%
\pgfsetstrokecolor{currentstroke}%
\pgfsetdash{}{0pt}%
\pgfpathmoveto{\pgfqpoint{5.504545in}{3.984333in}}%
\pgfpathcurveto{\pgfqpoint{5.515596in}{3.984333in}}{\pgfqpoint{5.526195in}{3.988724in}}{\pgfqpoint{5.534008in}{3.996537in}}%
\pgfpathcurveto{\pgfqpoint{5.541822in}{4.004351in}}{\pgfqpoint{5.546212in}{4.014950in}}{\pgfqpoint{5.546212in}{4.026000in}}%
\pgfpathcurveto{\pgfqpoint{5.546212in}{4.037050in}}{\pgfqpoint{5.541822in}{4.047649in}}{\pgfqpoint{5.534008in}{4.055463in}}%
\pgfpathcurveto{\pgfqpoint{5.526195in}{4.063276in}}{\pgfqpoint{5.515596in}{4.067667in}}{\pgfqpoint{5.504545in}{4.067667in}}%
\pgfpathcurveto{\pgfqpoint{5.493495in}{4.067667in}}{\pgfqpoint{5.482896in}{4.063276in}}{\pgfqpoint{5.475083in}{4.055463in}}%
\pgfpathcurveto{\pgfqpoint{5.467269in}{4.047649in}}{\pgfqpoint{5.462879in}{4.037050in}}{\pgfqpoint{5.462879in}{4.026000in}}%
\pgfpathcurveto{\pgfqpoint{5.462879in}{4.014950in}}{\pgfqpoint{5.467269in}{4.004351in}}{\pgfqpoint{5.475083in}{3.996537in}}%
\pgfpathcurveto{\pgfqpoint{5.482896in}{3.988724in}}{\pgfqpoint{5.493495in}{3.984333in}}{\pgfqpoint{5.504545in}{3.984333in}}%
\pgfpathclose%
\pgfusepath{stroke,fill}%
\end{pgfscope}%
\begin{pgfscope}%
\pgfpathrectangle{\pgfqpoint{0.800000in}{0.528000in}}{\pgfqpoint{4.960000in}{3.696000in}}%
\pgfusepath{clip}%
\pgfsetbuttcap%
\pgfsetroundjoin%
\definecolor{currentfill}{rgb}{0.000000,0.000000,0.000000}%
\pgfsetfillcolor{currentfill}%
\pgfsetlinewidth{1.003750pt}%
\definecolor{currentstroke}{rgb}{0.000000,0.000000,0.000000}%
\pgfsetstrokecolor{currentstroke}%
\pgfsetdash{}{0pt}%
\pgfpathmoveto{\pgfqpoint{5.504545in}{3.984333in}}%
\pgfpathcurveto{\pgfqpoint{5.515596in}{3.984333in}}{\pgfqpoint{5.526195in}{3.988724in}}{\pgfqpoint{5.534008in}{3.996537in}}%
\pgfpathcurveto{\pgfqpoint{5.541822in}{4.004351in}}{\pgfqpoint{5.546212in}{4.014950in}}{\pgfqpoint{5.546212in}{4.026000in}}%
\pgfpathcurveto{\pgfqpoint{5.546212in}{4.037050in}}{\pgfqpoint{5.541822in}{4.047649in}}{\pgfqpoint{5.534008in}{4.055463in}}%
\pgfpathcurveto{\pgfqpoint{5.526195in}{4.063276in}}{\pgfqpoint{5.515596in}{4.067667in}}{\pgfqpoint{5.504545in}{4.067667in}}%
\pgfpathcurveto{\pgfqpoint{5.493495in}{4.067667in}}{\pgfqpoint{5.482896in}{4.063276in}}{\pgfqpoint{5.475083in}{4.055463in}}%
\pgfpathcurveto{\pgfqpoint{5.467269in}{4.047649in}}{\pgfqpoint{5.462879in}{4.037050in}}{\pgfqpoint{5.462879in}{4.026000in}}%
\pgfpathcurveto{\pgfqpoint{5.462879in}{4.014950in}}{\pgfqpoint{5.467269in}{4.004351in}}{\pgfqpoint{5.475083in}{3.996537in}}%
\pgfpathcurveto{\pgfqpoint{5.482896in}{3.988724in}}{\pgfqpoint{5.493495in}{3.984333in}}{\pgfqpoint{5.504545in}{3.984333in}}%
\pgfpathclose%
\pgfusepath{stroke,fill}%
\end{pgfscope}%
\begin{pgfscope}%
\pgfpathrectangle{\pgfqpoint{0.800000in}{0.528000in}}{\pgfqpoint{4.960000in}{3.696000in}}%
\pgfusepath{clip}%
\pgfsetbuttcap%
\pgfsetroundjoin%
\definecolor{currentfill}{rgb}{0.000000,0.000000,0.000000}%
\pgfsetfillcolor{currentfill}%
\pgfsetlinewidth{1.003750pt}%
\definecolor{currentstroke}{rgb}{0.000000,0.000000,0.000000}%
\pgfsetstrokecolor{currentstroke}%
\pgfsetdash{}{0pt}%
\pgfpathmoveto{\pgfqpoint{5.504545in}{0.684333in}}%
\pgfpathcurveto{\pgfqpoint{5.515596in}{0.684333in}}{\pgfqpoint{5.526195in}{0.688724in}}{\pgfqpoint{5.534008in}{0.696537in}}%
\pgfpathcurveto{\pgfqpoint{5.541822in}{0.704351in}}{\pgfqpoint{5.546212in}{0.714950in}}{\pgfqpoint{5.546212in}{0.726000in}}%
\pgfpathcurveto{\pgfqpoint{5.546212in}{0.737050in}}{\pgfqpoint{5.541822in}{0.747649in}}{\pgfqpoint{5.534008in}{0.755463in}}%
\pgfpathcurveto{\pgfqpoint{5.526195in}{0.763276in}}{\pgfqpoint{5.515596in}{0.767667in}}{\pgfqpoint{5.504545in}{0.767667in}}%
\pgfpathcurveto{\pgfqpoint{5.493495in}{0.767667in}}{\pgfqpoint{5.482896in}{0.763276in}}{\pgfqpoint{5.475083in}{0.755463in}}%
\pgfpathcurveto{\pgfqpoint{5.467269in}{0.747649in}}{\pgfqpoint{5.462879in}{0.737050in}}{\pgfqpoint{5.462879in}{0.726000in}}%
\pgfpathcurveto{\pgfqpoint{5.462879in}{0.714950in}}{\pgfqpoint{5.467269in}{0.704351in}}{\pgfqpoint{5.475083in}{0.696537in}}%
\pgfpathcurveto{\pgfqpoint{5.482896in}{0.688724in}}{\pgfqpoint{5.493495in}{0.684333in}}{\pgfqpoint{5.504545in}{0.684333in}}%
\pgfpathclose%
\pgfusepath{stroke,fill}%
\end{pgfscope}%
\begin{pgfscope}%
\pgfpathrectangle{\pgfqpoint{0.800000in}{0.528000in}}{\pgfqpoint{4.960000in}{3.696000in}}%
\pgfusepath{clip}%
\pgfsetbuttcap%
\pgfsetroundjoin%
\definecolor{currentfill}{rgb}{0.000000,0.000000,0.000000}%
\pgfsetfillcolor{currentfill}%
\pgfsetlinewidth{1.003750pt}%
\definecolor{currentstroke}{rgb}{0.000000,0.000000,0.000000}%
\pgfsetstrokecolor{currentstroke}%
\pgfsetdash{}{0pt}%
\pgfpathmoveto{\pgfqpoint{5.504545in}{3.984333in}}%
\pgfpathcurveto{\pgfqpoint{5.515596in}{3.984333in}}{\pgfqpoint{5.526195in}{3.988724in}}{\pgfqpoint{5.534008in}{3.996537in}}%
\pgfpathcurveto{\pgfqpoint{5.541822in}{4.004351in}}{\pgfqpoint{5.546212in}{4.014950in}}{\pgfqpoint{5.546212in}{4.026000in}}%
\pgfpathcurveto{\pgfqpoint{5.546212in}{4.037050in}}{\pgfqpoint{5.541822in}{4.047649in}}{\pgfqpoint{5.534008in}{4.055463in}}%
\pgfpathcurveto{\pgfqpoint{5.526195in}{4.063276in}}{\pgfqpoint{5.515596in}{4.067667in}}{\pgfqpoint{5.504545in}{4.067667in}}%
\pgfpathcurveto{\pgfqpoint{5.493495in}{4.067667in}}{\pgfqpoint{5.482896in}{4.063276in}}{\pgfqpoint{5.475083in}{4.055463in}}%
\pgfpathcurveto{\pgfqpoint{5.467269in}{4.047649in}}{\pgfqpoint{5.462879in}{4.037050in}}{\pgfqpoint{5.462879in}{4.026000in}}%
\pgfpathcurveto{\pgfqpoint{5.462879in}{4.014950in}}{\pgfqpoint{5.467269in}{4.004351in}}{\pgfqpoint{5.475083in}{3.996537in}}%
\pgfpathcurveto{\pgfqpoint{5.482896in}{3.988724in}}{\pgfqpoint{5.493495in}{3.984333in}}{\pgfqpoint{5.504545in}{3.984333in}}%
\pgfpathclose%
\pgfusepath{stroke,fill}%
\end{pgfscope}%
\begin{pgfscope}%
\pgfpathrectangle{\pgfqpoint{0.800000in}{0.528000in}}{\pgfqpoint{4.960000in}{3.696000in}}%
\pgfusepath{clip}%
\pgfsetbuttcap%
\pgfsetroundjoin%
\definecolor{currentfill}{rgb}{0.000000,0.000000,0.000000}%
\pgfsetfillcolor{currentfill}%
\pgfsetlinewidth{1.003750pt}%
\definecolor{currentstroke}{rgb}{0.000000,0.000000,0.000000}%
\pgfsetstrokecolor{currentstroke}%
\pgfsetdash{}{0pt}%
\pgfpathmoveto{\pgfqpoint{5.504545in}{0.684333in}}%
\pgfpathcurveto{\pgfqpoint{5.515596in}{0.684333in}}{\pgfqpoint{5.526195in}{0.688724in}}{\pgfqpoint{5.534008in}{0.696537in}}%
\pgfpathcurveto{\pgfqpoint{5.541822in}{0.704351in}}{\pgfqpoint{5.546212in}{0.714950in}}{\pgfqpoint{5.546212in}{0.726000in}}%
\pgfpathcurveto{\pgfqpoint{5.546212in}{0.737050in}}{\pgfqpoint{5.541822in}{0.747649in}}{\pgfqpoint{5.534008in}{0.755463in}}%
\pgfpathcurveto{\pgfqpoint{5.526195in}{0.763276in}}{\pgfqpoint{5.515596in}{0.767667in}}{\pgfqpoint{5.504545in}{0.767667in}}%
\pgfpathcurveto{\pgfqpoint{5.493495in}{0.767667in}}{\pgfqpoint{5.482896in}{0.763276in}}{\pgfqpoint{5.475083in}{0.755463in}}%
\pgfpathcurveto{\pgfqpoint{5.467269in}{0.747649in}}{\pgfqpoint{5.462879in}{0.737050in}}{\pgfqpoint{5.462879in}{0.726000in}}%
\pgfpathcurveto{\pgfqpoint{5.462879in}{0.714950in}}{\pgfqpoint{5.467269in}{0.704351in}}{\pgfqpoint{5.475083in}{0.696537in}}%
\pgfpathcurveto{\pgfqpoint{5.482896in}{0.688724in}}{\pgfqpoint{5.493495in}{0.684333in}}{\pgfqpoint{5.504545in}{0.684333in}}%
\pgfpathclose%
\pgfusepath{stroke,fill}%
\end{pgfscope}%
\begin{pgfscope}%
\pgfpathrectangle{\pgfqpoint{0.800000in}{0.528000in}}{\pgfqpoint{4.960000in}{3.696000in}}%
\pgfusepath{clip}%
\pgfsetbuttcap%
\pgfsetroundjoin%
\definecolor{currentfill}{rgb}{0.000000,0.000000,0.000000}%
\pgfsetfillcolor{currentfill}%
\pgfsetlinewidth{1.003750pt}%
\definecolor{currentstroke}{rgb}{0.000000,0.000000,0.000000}%
\pgfsetstrokecolor{currentstroke}%
\pgfsetdash{}{0pt}%
\pgfpathmoveto{\pgfqpoint{5.504545in}{3.984333in}}%
\pgfpathcurveto{\pgfqpoint{5.515596in}{3.984333in}}{\pgfqpoint{5.526195in}{3.988724in}}{\pgfqpoint{5.534008in}{3.996537in}}%
\pgfpathcurveto{\pgfqpoint{5.541822in}{4.004351in}}{\pgfqpoint{5.546212in}{4.014950in}}{\pgfqpoint{5.546212in}{4.026000in}}%
\pgfpathcurveto{\pgfqpoint{5.546212in}{4.037050in}}{\pgfqpoint{5.541822in}{4.047649in}}{\pgfqpoint{5.534008in}{4.055463in}}%
\pgfpathcurveto{\pgfqpoint{5.526195in}{4.063276in}}{\pgfqpoint{5.515596in}{4.067667in}}{\pgfqpoint{5.504545in}{4.067667in}}%
\pgfpathcurveto{\pgfqpoint{5.493495in}{4.067667in}}{\pgfqpoint{5.482896in}{4.063276in}}{\pgfqpoint{5.475083in}{4.055463in}}%
\pgfpathcurveto{\pgfqpoint{5.467269in}{4.047649in}}{\pgfqpoint{5.462879in}{4.037050in}}{\pgfqpoint{5.462879in}{4.026000in}}%
\pgfpathcurveto{\pgfqpoint{5.462879in}{4.014950in}}{\pgfqpoint{5.467269in}{4.004351in}}{\pgfqpoint{5.475083in}{3.996537in}}%
\pgfpathcurveto{\pgfqpoint{5.482896in}{3.988724in}}{\pgfqpoint{5.493495in}{3.984333in}}{\pgfqpoint{5.504545in}{3.984333in}}%
\pgfpathclose%
\pgfusepath{stroke,fill}%
\end{pgfscope}%
\begin{pgfscope}%
\pgfpathrectangle{\pgfqpoint{0.800000in}{0.528000in}}{\pgfqpoint{4.960000in}{3.696000in}}%
\pgfusepath{clip}%
\pgfsetbuttcap%
\pgfsetroundjoin%
\definecolor{currentfill}{rgb}{0.000000,0.000000,0.000000}%
\pgfsetfillcolor{currentfill}%
\pgfsetlinewidth{1.003750pt}%
\definecolor{currentstroke}{rgb}{0.000000,0.000000,0.000000}%
\pgfsetstrokecolor{currentstroke}%
\pgfsetdash{}{0pt}%
\pgfpathmoveto{\pgfqpoint{5.504545in}{0.684333in}}%
\pgfpathcurveto{\pgfqpoint{5.515596in}{0.684333in}}{\pgfqpoint{5.526195in}{0.688724in}}{\pgfqpoint{5.534008in}{0.696537in}}%
\pgfpathcurveto{\pgfqpoint{5.541822in}{0.704351in}}{\pgfqpoint{5.546212in}{0.714950in}}{\pgfqpoint{5.546212in}{0.726000in}}%
\pgfpathcurveto{\pgfqpoint{5.546212in}{0.737050in}}{\pgfqpoint{5.541822in}{0.747649in}}{\pgfqpoint{5.534008in}{0.755463in}}%
\pgfpathcurveto{\pgfqpoint{5.526195in}{0.763276in}}{\pgfqpoint{5.515596in}{0.767667in}}{\pgfqpoint{5.504545in}{0.767667in}}%
\pgfpathcurveto{\pgfqpoint{5.493495in}{0.767667in}}{\pgfqpoint{5.482896in}{0.763276in}}{\pgfqpoint{5.475083in}{0.755463in}}%
\pgfpathcurveto{\pgfqpoint{5.467269in}{0.747649in}}{\pgfqpoint{5.462879in}{0.737050in}}{\pgfqpoint{5.462879in}{0.726000in}}%
\pgfpathcurveto{\pgfqpoint{5.462879in}{0.714950in}}{\pgfqpoint{5.467269in}{0.704351in}}{\pgfqpoint{5.475083in}{0.696537in}}%
\pgfpathcurveto{\pgfqpoint{5.482896in}{0.688724in}}{\pgfqpoint{5.493495in}{0.684333in}}{\pgfqpoint{5.504545in}{0.684333in}}%
\pgfpathclose%
\pgfusepath{stroke,fill}%
\end{pgfscope}%
\begin{pgfscope}%
\pgfpathrectangle{\pgfqpoint{0.800000in}{0.528000in}}{\pgfqpoint{4.960000in}{3.696000in}}%
\pgfusepath{clip}%
\pgfsetbuttcap%
\pgfsetroundjoin%
\definecolor{currentfill}{rgb}{0.000000,0.000000,0.000000}%
\pgfsetfillcolor{currentfill}%
\pgfsetlinewidth{1.003750pt}%
\definecolor{currentstroke}{rgb}{0.000000,0.000000,0.000000}%
\pgfsetstrokecolor{currentstroke}%
\pgfsetdash{}{0pt}%
\pgfpathmoveto{\pgfqpoint{5.504545in}{0.684333in}}%
\pgfpathcurveto{\pgfqpoint{5.515596in}{0.684333in}}{\pgfqpoint{5.526195in}{0.688724in}}{\pgfqpoint{5.534008in}{0.696537in}}%
\pgfpathcurveto{\pgfqpoint{5.541822in}{0.704351in}}{\pgfqpoint{5.546212in}{0.714950in}}{\pgfqpoint{5.546212in}{0.726000in}}%
\pgfpathcurveto{\pgfqpoint{5.546212in}{0.737050in}}{\pgfqpoint{5.541822in}{0.747649in}}{\pgfqpoint{5.534008in}{0.755463in}}%
\pgfpathcurveto{\pgfqpoint{5.526195in}{0.763276in}}{\pgfqpoint{5.515596in}{0.767667in}}{\pgfqpoint{5.504545in}{0.767667in}}%
\pgfpathcurveto{\pgfqpoint{5.493495in}{0.767667in}}{\pgfqpoint{5.482896in}{0.763276in}}{\pgfqpoint{5.475083in}{0.755463in}}%
\pgfpathcurveto{\pgfqpoint{5.467269in}{0.747649in}}{\pgfqpoint{5.462879in}{0.737050in}}{\pgfqpoint{5.462879in}{0.726000in}}%
\pgfpathcurveto{\pgfqpoint{5.462879in}{0.714950in}}{\pgfqpoint{5.467269in}{0.704351in}}{\pgfqpoint{5.475083in}{0.696537in}}%
\pgfpathcurveto{\pgfqpoint{5.482896in}{0.688724in}}{\pgfqpoint{5.493495in}{0.684333in}}{\pgfqpoint{5.504545in}{0.684333in}}%
\pgfpathclose%
\pgfusepath{stroke,fill}%
\end{pgfscope}%
\begin{pgfscope}%
\pgfpathrectangle{\pgfqpoint{0.800000in}{0.528000in}}{\pgfqpoint{4.960000in}{3.696000in}}%
\pgfusepath{clip}%
\pgfsetbuttcap%
\pgfsetroundjoin%
\definecolor{currentfill}{rgb}{0.000000,0.000000,0.000000}%
\pgfsetfillcolor{currentfill}%
\pgfsetlinewidth{1.003750pt}%
\definecolor{currentstroke}{rgb}{0.000000,0.000000,0.000000}%
\pgfsetstrokecolor{currentstroke}%
\pgfsetdash{}{0pt}%
\pgfpathmoveto{\pgfqpoint{5.504545in}{3.984333in}}%
\pgfpathcurveto{\pgfqpoint{5.515596in}{3.984333in}}{\pgfqpoint{5.526195in}{3.988724in}}{\pgfqpoint{5.534008in}{3.996537in}}%
\pgfpathcurveto{\pgfqpoint{5.541822in}{4.004351in}}{\pgfqpoint{5.546212in}{4.014950in}}{\pgfqpoint{5.546212in}{4.026000in}}%
\pgfpathcurveto{\pgfqpoint{5.546212in}{4.037050in}}{\pgfqpoint{5.541822in}{4.047649in}}{\pgfqpoint{5.534008in}{4.055463in}}%
\pgfpathcurveto{\pgfqpoint{5.526195in}{4.063276in}}{\pgfqpoint{5.515596in}{4.067667in}}{\pgfqpoint{5.504545in}{4.067667in}}%
\pgfpathcurveto{\pgfqpoint{5.493495in}{4.067667in}}{\pgfqpoint{5.482896in}{4.063276in}}{\pgfqpoint{5.475083in}{4.055463in}}%
\pgfpathcurveto{\pgfqpoint{5.467269in}{4.047649in}}{\pgfqpoint{5.462879in}{4.037050in}}{\pgfqpoint{5.462879in}{4.026000in}}%
\pgfpathcurveto{\pgfqpoint{5.462879in}{4.014950in}}{\pgfqpoint{5.467269in}{4.004351in}}{\pgfqpoint{5.475083in}{3.996537in}}%
\pgfpathcurveto{\pgfqpoint{5.482896in}{3.988724in}}{\pgfqpoint{5.493495in}{3.984333in}}{\pgfqpoint{5.504545in}{3.984333in}}%
\pgfpathclose%
\pgfusepath{stroke,fill}%
\end{pgfscope}%
\begin{pgfscope}%
\pgfpathrectangle{\pgfqpoint{0.800000in}{0.528000in}}{\pgfqpoint{4.960000in}{3.696000in}}%
\pgfusepath{clip}%
\pgfsetbuttcap%
\pgfsetroundjoin%
\definecolor{currentfill}{rgb}{0.000000,0.000000,0.000000}%
\pgfsetfillcolor{currentfill}%
\pgfsetlinewidth{1.003750pt}%
\definecolor{currentstroke}{rgb}{0.000000,0.000000,0.000000}%
\pgfsetstrokecolor{currentstroke}%
\pgfsetdash{}{0pt}%
\pgfpathmoveto{\pgfqpoint{5.504545in}{3.984333in}}%
\pgfpathcurveto{\pgfqpoint{5.515596in}{3.984333in}}{\pgfqpoint{5.526195in}{3.988724in}}{\pgfqpoint{5.534008in}{3.996537in}}%
\pgfpathcurveto{\pgfqpoint{5.541822in}{4.004351in}}{\pgfqpoint{5.546212in}{4.014950in}}{\pgfqpoint{5.546212in}{4.026000in}}%
\pgfpathcurveto{\pgfqpoint{5.546212in}{4.037050in}}{\pgfqpoint{5.541822in}{4.047649in}}{\pgfqpoint{5.534008in}{4.055463in}}%
\pgfpathcurveto{\pgfqpoint{5.526195in}{4.063276in}}{\pgfqpoint{5.515596in}{4.067667in}}{\pgfqpoint{5.504545in}{4.067667in}}%
\pgfpathcurveto{\pgfqpoint{5.493495in}{4.067667in}}{\pgfqpoint{5.482896in}{4.063276in}}{\pgfqpoint{5.475083in}{4.055463in}}%
\pgfpathcurveto{\pgfqpoint{5.467269in}{4.047649in}}{\pgfqpoint{5.462879in}{4.037050in}}{\pgfqpoint{5.462879in}{4.026000in}}%
\pgfpathcurveto{\pgfqpoint{5.462879in}{4.014950in}}{\pgfqpoint{5.467269in}{4.004351in}}{\pgfqpoint{5.475083in}{3.996537in}}%
\pgfpathcurveto{\pgfqpoint{5.482896in}{3.988724in}}{\pgfqpoint{5.493495in}{3.984333in}}{\pgfqpoint{5.504545in}{3.984333in}}%
\pgfpathclose%
\pgfusepath{stroke,fill}%
\end{pgfscope}%
\begin{pgfscope}%
\pgfpathrectangle{\pgfqpoint{0.800000in}{0.528000in}}{\pgfqpoint{4.960000in}{3.696000in}}%
\pgfusepath{clip}%
\pgfsetbuttcap%
\pgfsetroundjoin%
\definecolor{currentfill}{rgb}{0.000000,0.000000,0.000000}%
\pgfsetfillcolor{currentfill}%
\pgfsetlinewidth{1.003750pt}%
\definecolor{currentstroke}{rgb}{0.000000,0.000000,0.000000}%
\pgfsetstrokecolor{currentstroke}%
\pgfsetdash{}{0pt}%
\pgfpathmoveto{\pgfqpoint{5.504545in}{0.684333in}}%
\pgfpathcurveto{\pgfqpoint{5.515596in}{0.684333in}}{\pgfqpoint{5.526195in}{0.688724in}}{\pgfqpoint{5.534008in}{0.696537in}}%
\pgfpathcurveto{\pgfqpoint{5.541822in}{0.704351in}}{\pgfqpoint{5.546212in}{0.714950in}}{\pgfqpoint{5.546212in}{0.726000in}}%
\pgfpathcurveto{\pgfqpoint{5.546212in}{0.737050in}}{\pgfqpoint{5.541822in}{0.747649in}}{\pgfqpoint{5.534008in}{0.755463in}}%
\pgfpathcurveto{\pgfqpoint{5.526195in}{0.763276in}}{\pgfqpoint{5.515596in}{0.767667in}}{\pgfqpoint{5.504545in}{0.767667in}}%
\pgfpathcurveto{\pgfqpoint{5.493495in}{0.767667in}}{\pgfqpoint{5.482896in}{0.763276in}}{\pgfqpoint{5.475083in}{0.755463in}}%
\pgfpathcurveto{\pgfqpoint{5.467269in}{0.747649in}}{\pgfqpoint{5.462879in}{0.737050in}}{\pgfqpoint{5.462879in}{0.726000in}}%
\pgfpathcurveto{\pgfqpoint{5.462879in}{0.714950in}}{\pgfqpoint{5.467269in}{0.704351in}}{\pgfqpoint{5.475083in}{0.696537in}}%
\pgfpathcurveto{\pgfqpoint{5.482896in}{0.688724in}}{\pgfqpoint{5.493495in}{0.684333in}}{\pgfqpoint{5.504545in}{0.684333in}}%
\pgfpathclose%
\pgfusepath{stroke,fill}%
\end{pgfscope}%
\begin{pgfscope}%
\pgfpathrectangle{\pgfqpoint{0.800000in}{0.528000in}}{\pgfqpoint{4.960000in}{3.696000in}}%
\pgfusepath{clip}%
\pgfsetbuttcap%
\pgfsetroundjoin%
\definecolor{currentfill}{rgb}{0.000000,0.000000,0.000000}%
\pgfsetfillcolor{currentfill}%
\pgfsetlinewidth{1.003750pt}%
\definecolor{currentstroke}{rgb}{0.000000,0.000000,0.000000}%
\pgfsetstrokecolor{currentstroke}%
\pgfsetdash{}{0pt}%
\pgfpathmoveto{\pgfqpoint{5.504545in}{0.684333in}}%
\pgfpathcurveto{\pgfqpoint{5.515596in}{0.684333in}}{\pgfqpoint{5.526195in}{0.688724in}}{\pgfqpoint{5.534008in}{0.696537in}}%
\pgfpathcurveto{\pgfqpoint{5.541822in}{0.704351in}}{\pgfqpoint{5.546212in}{0.714950in}}{\pgfqpoint{5.546212in}{0.726000in}}%
\pgfpathcurveto{\pgfqpoint{5.546212in}{0.737050in}}{\pgfqpoint{5.541822in}{0.747649in}}{\pgfqpoint{5.534008in}{0.755463in}}%
\pgfpathcurveto{\pgfqpoint{5.526195in}{0.763276in}}{\pgfqpoint{5.515596in}{0.767667in}}{\pgfqpoint{5.504545in}{0.767667in}}%
\pgfpathcurveto{\pgfqpoint{5.493495in}{0.767667in}}{\pgfqpoint{5.482896in}{0.763276in}}{\pgfqpoint{5.475083in}{0.755463in}}%
\pgfpathcurveto{\pgfqpoint{5.467269in}{0.747649in}}{\pgfqpoint{5.462879in}{0.737050in}}{\pgfqpoint{5.462879in}{0.726000in}}%
\pgfpathcurveto{\pgfqpoint{5.462879in}{0.714950in}}{\pgfqpoint{5.467269in}{0.704351in}}{\pgfqpoint{5.475083in}{0.696537in}}%
\pgfpathcurveto{\pgfqpoint{5.482896in}{0.688724in}}{\pgfqpoint{5.493495in}{0.684333in}}{\pgfqpoint{5.504545in}{0.684333in}}%
\pgfpathclose%
\pgfusepath{stroke,fill}%
\end{pgfscope}%
\begin{pgfscope}%
\pgfpathrectangle{\pgfqpoint{0.800000in}{0.528000in}}{\pgfqpoint{4.960000in}{3.696000in}}%
\pgfusepath{clip}%
\pgfsetbuttcap%
\pgfsetroundjoin%
\definecolor{currentfill}{rgb}{0.000000,0.000000,0.000000}%
\pgfsetfillcolor{currentfill}%
\pgfsetlinewidth{1.003750pt}%
\definecolor{currentstroke}{rgb}{0.000000,0.000000,0.000000}%
\pgfsetstrokecolor{currentstroke}%
\pgfsetdash{}{0pt}%
\pgfpathmoveto{\pgfqpoint{5.504545in}{3.984333in}}%
\pgfpathcurveto{\pgfqpoint{5.515596in}{3.984333in}}{\pgfqpoint{5.526195in}{3.988724in}}{\pgfqpoint{5.534008in}{3.996537in}}%
\pgfpathcurveto{\pgfqpoint{5.541822in}{4.004351in}}{\pgfqpoint{5.546212in}{4.014950in}}{\pgfqpoint{5.546212in}{4.026000in}}%
\pgfpathcurveto{\pgfqpoint{5.546212in}{4.037050in}}{\pgfqpoint{5.541822in}{4.047649in}}{\pgfqpoint{5.534008in}{4.055463in}}%
\pgfpathcurveto{\pgfqpoint{5.526195in}{4.063276in}}{\pgfqpoint{5.515596in}{4.067667in}}{\pgfqpoint{5.504545in}{4.067667in}}%
\pgfpathcurveto{\pgfqpoint{5.493495in}{4.067667in}}{\pgfqpoint{5.482896in}{4.063276in}}{\pgfqpoint{5.475083in}{4.055463in}}%
\pgfpathcurveto{\pgfqpoint{5.467269in}{4.047649in}}{\pgfqpoint{5.462879in}{4.037050in}}{\pgfqpoint{5.462879in}{4.026000in}}%
\pgfpathcurveto{\pgfqpoint{5.462879in}{4.014950in}}{\pgfqpoint{5.467269in}{4.004351in}}{\pgfqpoint{5.475083in}{3.996537in}}%
\pgfpathcurveto{\pgfqpoint{5.482896in}{3.988724in}}{\pgfqpoint{5.493495in}{3.984333in}}{\pgfqpoint{5.504545in}{3.984333in}}%
\pgfpathclose%
\pgfusepath{stroke,fill}%
\end{pgfscope}%
\begin{pgfscope}%
\pgfpathrectangle{\pgfqpoint{0.800000in}{0.528000in}}{\pgfqpoint{4.960000in}{3.696000in}}%
\pgfusepath{clip}%
\pgfsetbuttcap%
\pgfsetroundjoin%
\definecolor{currentfill}{rgb}{0.000000,0.000000,0.000000}%
\pgfsetfillcolor{currentfill}%
\pgfsetlinewidth{1.003750pt}%
\definecolor{currentstroke}{rgb}{0.000000,0.000000,0.000000}%
\pgfsetstrokecolor{currentstroke}%
\pgfsetdash{}{0pt}%
\pgfpathmoveto{\pgfqpoint{5.504545in}{0.684333in}}%
\pgfpathcurveto{\pgfqpoint{5.515596in}{0.684333in}}{\pgfqpoint{5.526195in}{0.688724in}}{\pgfqpoint{5.534008in}{0.696537in}}%
\pgfpathcurveto{\pgfqpoint{5.541822in}{0.704351in}}{\pgfqpoint{5.546212in}{0.714950in}}{\pgfqpoint{5.546212in}{0.726000in}}%
\pgfpathcurveto{\pgfqpoint{5.546212in}{0.737050in}}{\pgfqpoint{5.541822in}{0.747649in}}{\pgfqpoint{5.534008in}{0.755463in}}%
\pgfpathcurveto{\pgfqpoint{5.526195in}{0.763276in}}{\pgfqpoint{5.515596in}{0.767667in}}{\pgfqpoint{5.504545in}{0.767667in}}%
\pgfpathcurveto{\pgfqpoint{5.493495in}{0.767667in}}{\pgfqpoint{5.482896in}{0.763276in}}{\pgfqpoint{5.475083in}{0.755463in}}%
\pgfpathcurveto{\pgfqpoint{5.467269in}{0.747649in}}{\pgfqpoint{5.462879in}{0.737050in}}{\pgfqpoint{5.462879in}{0.726000in}}%
\pgfpathcurveto{\pgfqpoint{5.462879in}{0.714950in}}{\pgfqpoint{5.467269in}{0.704351in}}{\pgfqpoint{5.475083in}{0.696537in}}%
\pgfpathcurveto{\pgfqpoint{5.482896in}{0.688724in}}{\pgfqpoint{5.493495in}{0.684333in}}{\pgfqpoint{5.504545in}{0.684333in}}%
\pgfpathclose%
\pgfusepath{stroke,fill}%
\end{pgfscope}%
\begin{pgfscope}%
\pgfpathrectangle{\pgfqpoint{0.800000in}{0.528000in}}{\pgfqpoint{4.960000in}{3.696000in}}%
\pgfusepath{clip}%
\pgfsetbuttcap%
\pgfsetroundjoin%
\definecolor{currentfill}{rgb}{0.000000,0.000000,0.000000}%
\pgfsetfillcolor{currentfill}%
\pgfsetlinewidth{1.003750pt}%
\definecolor{currentstroke}{rgb}{0.000000,0.000000,0.000000}%
\pgfsetstrokecolor{currentstroke}%
\pgfsetdash{}{0pt}%
\pgfpathmoveto{\pgfqpoint{5.504545in}{3.984333in}}%
\pgfpathcurveto{\pgfqpoint{5.515596in}{3.984333in}}{\pgfqpoint{5.526195in}{3.988724in}}{\pgfqpoint{5.534008in}{3.996537in}}%
\pgfpathcurveto{\pgfqpoint{5.541822in}{4.004351in}}{\pgfqpoint{5.546212in}{4.014950in}}{\pgfqpoint{5.546212in}{4.026000in}}%
\pgfpathcurveto{\pgfqpoint{5.546212in}{4.037050in}}{\pgfqpoint{5.541822in}{4.047649in}}{\pgfqpoint{5.534008in}{4.055463in}}%
\pgfpathcurveto{\pgfqpoint{5.526195in}{4.063276in}}{\pgfqpoint{5.515596in}{4.067667in}}{\pgfqpoint{5.504545in}{4.067667in}}%
\pgfpathcurveto{\pgfqpoint{5.493495in}{4.067667in}}{\pgfqpoint{5.482896in}{4.063276in}}{\pgfqpoint{5.475083in}{4.055463in}}%
\pgfpathcurveto{\pgfqpoint{5.467269in}{4.047649in}}{\pgfqpoint{5.462879in}{4.037050in}}{\pgfqpoint{5.462879in}{4.026000in}}%
\pgfpathcurveto{\pgfqpoint{5.462879in}{4.014950in}}{\pgfqpoint{5.467269in}{4.004351in}}{\pgfqpoint{5.475083in}{3.996537in}}%
\pgfpathcurveto{\pgfqpoint{5.482896in}{3.988724in}}{\pgfqpoint{5.493495in}{3.984333in}}{\pgfqpoint{5.504545in}{3.984333in}}%
\pgfpathclose%
\pgfusepath{stroke,fill}%
\end{pgfscope}%
\begin{pgfscope}%
\pgfpathrectangle{\pgfqpoint{0.800000in}{0.528000in}}{\pgfqpoint{4.960000in}{3.696000in}}%
\pgfusepath{clip}%
\pgfsetbuttcap%
\pgfsetroundjoin%
\definecolor{currentfill}{rgb}{0.000000,0.000000,0.000000}%
\pgfsetfillcolor{currentfill}%
\pgfsetlinewidth{1.003750pt}%
\definecolor{currentstroke}{rgb}{0.000000,0.000000,0.000000}%
\pgfsetstrokecolor{currentstroke}%
\pgfsetdash{}{0pt}%
\pgfpathmoveto{\pgfqpoint{5.504545in}{0.684333in}}%
\pgfpathcurveto{\pgfqpoint{5.515596in}{0.684333in}}{\pgfqpoint{5.526195in}{0.688724in}}{\pgfqpoint{5.534008in}{0.696537in}}%
\pgfpathcurveto{\pgfqpoint{5.541822in}{0.704351in}}{\pgfqpoint{5.546212in}{0.714950in}}{\pgfqpoint{5.546212in}{0.726000in}}%
\pgfpathcurveto{\pgfqpoint{5.546212in}{0.737050in}}{\pgfqpoint{5.541822in}{0.747649in}}{\pgfqpoint{5.534008in}{0.755463in}}%
\pgfpathcurveto{\pgfqpoint{5.526195in}{0.763276in}}{\pgfqpoint{5.515596in}{0.767667in}}{\pgfqpoint{5.504545in}{0.767667in}}%
\pgfpathcurveto{\pgfqpoint{5.493495in}{0.767667in}}{\pgfqpoint{5.482896in}{0.763276in}}{\pgfqpoint{5.475083in}{0.755463in}}%
\pgfpathcurveto{\pgfqpoint{5.467269in}{0.747649in}}{\pgfqpoint{5.462879in}{0.737050in}}{\pgfqpoint{5.462879in}{0.726000in}}%
\pgfpathcurveto{\pgfqpoint{5.462879in}{0.714950in}}{\pgfqpoint{5.467269in}{0.704351in}}{\pgfqpoint{5.475083in}{0.696537in}}%
\pgfpathcurveto{\pgfqpoint{5.482896in}{0.688724in}}{\pgfqpoint{5.493495in}{0.684333in}}{\pgfqpoint{5.504545in}{0.684333in}}%
\pgfpathclose%
\pgfusepath{stroke,fill}%
\end{pgfscope}%
\begin{pgfscope}%
\pgfpathrectangle{\pgfqpoint{0.800000in}{0.528000in}}{\pgfqpoint{4.960000in}{3.696000in}}%
\pgfusepath{clip}%
\pgfsetbuttcap%
\pgfsetroundjoin%
\definecolor{currentfill}{rgb}{0.000000,0.000000,0.000000}%
\pgfsetfillcolor{currentfill}%
\pgfsetlinewidth{1.003750pt}%
\definecolor{currentstroke}{rgb}{0.000000,0.000000,0.000000}%
\pgfsetstrokecolor{currentstroke}%
\pgfsetdash{}{0pt}%
\pgfpathmoveto{\pgfqpoint{5.504545in}{0.684333in}}%
\pgfpathcurveto{\pgfqpoint{5.515596in}{0.684333in}}{\pgfqpoint{5.526195in}{0.688724in}}{\pgfqpoint{5.534008in}{0.696537in}}%
\pgfpathcurveto{\pgfqpoint{5.541822in}{0.704351in}}{\pgfqpoint{5.546212in}{0.714950in}}{\pgfqpoint{5.546212in}{0.726000in}}%
\pgfpathcurveto{\pgfqpoint{5.546212in}{0.737050in}}{\pgfqpoint{5.541822in}{0.747649in}}{\pgfqpoint{5.534008in}{0.755463in}}%
\pgfpathcurveto{\pgfqpoint{5.526195in}{0.763276in}}{\pgfqpoint{5.515596in}{0.767667in}}{\pgfqpoint{5.504545in}{0.767667in}}%
\pgfpathcurveto{\pgfqpoint{5.493495in}{0.767667in}}{\pgfqpoint{5.482896in}{0.763276in}}{\pgfqpoint{5.475083in}{0.755463in}}%
\pgfpathcurveto{\pgfqpoint{5.467269in}{0.747649in}}{\pgfqpoint{5.462879in}{0.737050in}}{\pgfqpoint{5.462879in}{0.726000in}}%
\pgfpathcurveto{\pgfqpoint{5.462879in}{0.714950in}}{\pgfqpoint{5.467269in}{0.704351in}}{\pgfqpoint{5.475083in}{0.696537in}}%
\pgfpathcurveto{\pgfqpoint{5.482896in}{0.688724in}}{\pgfqpoint{5.493495in}{0.684333in}}{\pgfqpoint{5.504545in}{0.684333in}}%
\pgfpathclose%
\pgfusepath{stroke,fill}%
\end{pgfscope}%
\begin{pgfscope}%
\pgfpathrectangle{\pgfqpoint{0.800000in}{0.528000in}}{\pgfqpoint{4.960000in}{3.696000in}}%
\pgfusepath{clip}%
\pgfsetbuttcap%
\pgfsetroundjoin%
\definecolor{currentfill}{rgb}{0.000000,0.000000,0.000000}%
\pgfsetfillcolor{currentfill}%
\pgfsetlinewidth{1.003750pt}%
\definecolor{currentstroke}{rgb}{0.000000,0.000000,0.000000}%
\pgfsetstrokecolor{currentstroke}%
\pgfsetdash{}{0pt}%
\pgfpathmoveto{\pgfqpoint{5.504545in}{3.984333in}}%
\pgfpathcurveto{\pgfqpoint{5.515596in}{3.984333in}}{\pgfqpoint{5.526195in}{3.988724in}}{\pgfqpoint{5.534008in}{3.996537in}}%
\pgfpathcurveto{\pgfqpoint{5.541822in}{4.004351in}}{\pgfqpoint{5.546212in}{4.014950in}}{\pgfqpoint{5.546212in}{4.026000in}}%
\pgfpathcurveto{\pgfqpoint{5.546212in}{4.037050in}}{\pgfqpoint{5.541822in}{4.047649in}}{\pgfqpoint{5.534008in}{4.055463in}}%
\pgfpathcurveto{\pgfqpoint{5.526195in}{4.063276in}}{\pgfqpoint{5.515596in}{4.067667in}}{\pgfqpoint{5.504545in}{4.067667in}}%
\pgfpathcurveto{\pgfqpoint{5.493495in}{4.067667in}}{\pgfqpoint{5.482896in}{4.063276in}}{\pgfqpoint{5.475083in}{4.055463in}}%
\pgfpathcurveto{\pgfqpoint{5.467269in}{4.047649in}}{\pgfqpoint{5.462879in}{4.037050in}}{\pgfqpoint{5.462879in}{4.026000in}}%
\pgfpathcurveto{\pgfqpoint{5.462879in}{4.014950in}}{\pgfqpoint{5.467269in}{4.004351in}}{\pgfqpoint{5.475083in}{3.996537in}}%
\pgfpathcurveto{\pgfqpoint{5.482896in}{3.988724in}}{\pgfqpoint{5.493495in}{3.984333in}}{\pgfqpoint{5.504545in}{3.984333in}}%
\pgfpathclose%
\pgfusepath{stroke,fill}%
\end{pgfscope}%
\begin{pgfscope}%
\pgfpathrectangle{\pgfqpoint{0.800000in}{0.528000in}}{\pgfqpoint{4.960000in}{3.696000in}}%
\pgfusepath{clip}%
\pgfsetbuttcap%
\pgfsetroundjoin%
\definecolor{currentfill}{rgb}{0.000000,0.000000,0.000000}%
\pgfsetfillcolor{currentfill}%
\pgfsetlinewidth{1.003750pt}%
\definecolor{currentstroke}{rgb}{0.000000,0.000000,0.000000}%
\pgfsetstrokecolor{currentstroke}%
\pgfsetdash{}{0pt}%
\pgfpathmoveto{\pgfqpoint{5.504545in}{0.684333in}}%
\pgfpathcurveto{\pgfqpoint{5.515596in}{0.684333in}}{\pgfqpoint{5.526195in}{0.688724in}}{\pgfqpoint{5.534008in}{0.696537in}}%
\pgfpathcurveto{\pgfqpoint{5.541822in}{0.704351in}}{\pgfqpoint{5.546212in}{0.714950in}}{\pgfqpoint{5.546212in}{0.726000in}}%
\pgfpathcurveto{\pgfqpoint{5.546212in}{0.737050in}}{\pgfqpoint{5.541822in}{0.747649in}}{\pgfqpoint{5.534008in}{0.755463in}}%
\pgfpathcurveto{\pgfqpoint{5.526195in}{0.763276in}}{\pgfqpoint{5.515596in}{0.767667in}}{\pgfqpoint{5.504545in}{0.767667in}}%
\pgfpathcurveto{\pgfqpoint{5.493495in}{0.767667in}}{\pgfqpoint{5.482896in}{0.763276in}}{\pgfqpoint{5.475083in}{0.755463in}}%
\pgfpathcurveto{\pgfqpoint{5.467269in}{0.747649in}}{\pgfqpoint{5.462879in}{0.737050in}}{\pgfqpoint{5.462879in}{0.726000in}}%
\pgfpathcurveto{\pgfqpoint{5.462879in}{0.714950in}}{\pgfqpoint{5.467269in}{0.704351in}}{\pgfqpoint{5.475083in}{0.696537in}}%
\pgfpathcurveto{\pgfqpoint{5.482896in}{0.688724in}}{\pgfqpoint{5.493495in}{0.684333in}}{\pgfqpoint{5.504545in}{0.684333in}}%
\pgfpathclose%
\pgfusepath{stroke,fill}%
\end{pgfscope}%
\begin{pgfscope}%
\pgfpathrectangle{\pgfqpoint{0.800000in}{0.528000in}}{\pgfqpoint{4.960000in}{3.696000in}}%
\pgfusepath{clip}%
\pgfsetbuttcap%
\pgfsetroundjoin%
\definecolor{currentfill}{rgb}{0.000000,0.000000,0.000000}%
\pgfsetfillcolor{currentfill}%
\pgfsetlinewidth{1.003750pt}%
\definecolor{currentstroke}{rgb}{0.000000,0.000000,0.000000}%
\pgfsetstrokecolor{currentstroke}%
\pgfsetdash{}{0pt}%
\pgfpathmoveto{\pgfqpoint{5.504545in}{0.684333in}}%
\pgfpathcurveto{\pgfqpoint{5.515596in}{0.684333in}}{\pgfqpoint{5.526195in}{0.688724in}}{\pgfqpoint{5.534008in}{0.696537in}}%
\pgfpathcurveto{\pgfqpoint{5.541822in}{0.704351in}}{\pgfqpoint{5.546212in}{0.714950in}}{\pgfqpoint{5.546212in}{0.726000in}}%
\pgfpathcurveto{\pgfqpoint{5.546212in}{0.737050in}}{\pgfqpoint{5.541822in}{0.747649in}}{\pgfqpoint{5.534008in}{0.755463in}}%
\pgfpathcurveto{\pgfqpoint{5.526195in}{0.763276in}}{\pgfqpoint{5.515596in}{0.767667in}}{\pgfqpoint{5.504545in}{0.767667in}}%
\pgfpathcurveto{\pgfqpoint{5.493495in}{0.767667in}}{\pgfqpoint{5.482896in}{0.763276in}}{\pgfqpoint{5.475083in}{0.755463in}}%
\pgfpathcurveto{\pgfqpoint{5.467269in}{0.747649in}}{\pgfqpoint{5.462879in}{0.737050in}}{\pgfqpoint{5.462879in}{0.726000in}}%
\pgfpathcurveto{\pgfqpoint{5.462879in}{0.714950in}}{\pgfqpoint{5.467269in}{0.704351in}}{\pgfqpoint{5.475083in}{0.696537in}}%
\pgfpathcurveto{\pgfqpoint{5.482896in}{0.688724in}}{\pgfqpoint{5.493495in}{0.684333in}}{\pgfqpoint{5.504545in}{0.684333in}}%
\pgfpathclose%
\pgfusepath{stroke,fill}%
\end{pgfscope}%
\begin{pgfscope}%
\pgfpathrectangle{\pgfqpoint{0.800000in}{0.528000in}}{\pgfqpoint{4.960000in}{3.696000in}}%
\pgfusepath{clip}%
\pgfsetbuttcap%
\pgfsetroundjoin%
\definecolor{currentfill}{rgb}{0.000000,0.000000,0.000000}%
\pgfsetfillcolor{currentfill}%
\pgfsetlinewidth{1.003750pt}%
\definecolor{currentstroke}{rgb}{0.000000,0.000000,0.000000}%
\pgfsetstrokecolor{currentstroke}%
\pgfsetdash{}{0pt}%
\pgfpathmoveto{\pgfqpoint{5.504545in}{3.984333in}}%
\pgfpathcurveto{\pgfqpoint{5.515596in}{3.984333in}}{\pgfqpoint{5.526195in}{3.988724in}}{\pgfqpoint{5.534008in}{3.996537in}}%
\pgfpathcurveto{\pgfqpoint{5.541822in}{4.004351in}}{\pgfqpoint{5.546212in}{4.014950in}}{\pgfqpoint{5.546212in}{4.026000in}}%
\pgfpathcurveto{\pgfqpoint{5.546212in}{4.037050in}}{\pgfqpoint{5.541822in}{4.047649in}}{\pgfqpoint{5.534008in}{4.055463in}}%
\pgfpathcurveto{\pgfqpoint{5.526195in}{4.063276in}}{\pgfqpoint{5.515596in}{4.067667in}}{\pgfqpoint{5.504545in}{4.067667in}}%
\pgfpathcurveto{\pgfqpoint{5.493495in}{4.067667in}}{\pgfqpoint{5.482896in}{4.063276in}}{\pgfqpoint{5.475083in}{4.055463in}}%
\pgfpathcurveto{\pgfqpoint{5.467269in}{4.047649in}}{\pgfqpoint{5.462879in}{4.037050in}}{\pgfqpoint{5.462879in}{4.026000in}}%
\pgfpathcurveto{\pgfqpoint{5.462879in}{4.014950in}}{\pgfqpoint{5.467269in}{4.004351in}}{\pgfqpoint{5.475083in}{3.996537in}}%
\pgfpathcurveto{\pgfqpoint{5.482896in}{3.988724in}}{\pgfqpoint{5.493495in}{3.984333in}}{\pgfqpoint{5.504545in}{3.984333in}}%
\pgfpathclose%
\pgfusepath{stroke,fill}%
\end{pgfscope}%
\begin{pgfscope}%
\pgfpathrectangle{\pgfqpoint{0.800000in}{0.528000in}}{\pgfqpoint{4.960000in}{3.696000in}}%
\pgfusepath{clip}%
\pgfsetbuttcap%
\pgfsetroundjoin%
\definecolor{currentfill}{rgb}{0.000000,0.000000,0.000000}%
\pgfsetfillcolor{currentfill}%
\pgfsetlinewidth{1.003750pt}%
\definecolor{currentstroke}{rgb}{0.000000,0.000000,0.000000}%
\pgfsetstrokecolor{currentstroke}%
\pgfsetdash{}{0pt}%
\pgfpathmoveto{\pgfqpoint{5.504545in}{0.684333in}}%
\pgfpathcurveto{\pgfqpoint{5.515596in}{0.684333in}}{\pgfqpoint{5.526195in}{0.688724in}}{\pgfqpoint{5.534008in}{0.696537in}}%
\pgfpathcurveto{\pgfqpoint{5.541822in}{0.704351in}}{\pgfqpoint{5.546212in}{0.714950in}}{\pgfqpoint{5.546212in}{0.726000in}}%
\pgfpathcurveto{\pgfqpoint{5.546212in}{0.737050in}}{\pgfqpoint{5.541822in}{0.747649in}}{\pgfqpoint{5.534008in}{0.755463in}}%
\pgfpathcurveto{\pgfqpoint{5.526195in}{0.763276in}}{\pgfqpoint{5.515596in}{0.767667in}}{\pgfqpoint{5.504545in}{0.767667in}}%
\pgfpathcurveto{\pgfqpoint{5.493495in}{0.767667in}}{\pgfqpoint{5.482896in}{0.763276in}}{\pgfqpoint{5.475083in}{0.755463in}}%
\pgfpathcurveto{\pgfqpoint{5.467269in}{0.747649in}}{\pgfqpoint{5.462879in}{0.737050in}}{\pgfqpoint{5.462879in}{0.726000in}}%
\pgfpathcurveto{\pgfqpoint{5.462879in}{0.714950in}}{\pgfqpoint{5.467269in}{0.704351in}}{\pgfqpoint{5.475083in}{0.696537in}}%
\pgfpathcurveto{\pgfqpoint{5.482896in}{0.688724in}}{\pgfqpoint{5.493495in}{0.684333in}}{\pgfqpoint{5.504545in}{0.684333in}}%
\pgfpathclose%
\pgfusepath{stroke,fill}%
\end{pgfscope}%
\begin{pgfscope}%
\pgfpathrectangle{\pgfqpoint{0.800000in}{0.528000in}}{\pgfqpoint{4.960000in}{3.696000in}}%
\pgfusepath{clip}%
\pgfsetbuttcap%
\pgfsetroundjoin%
\definecolor{currentfill}{rgb}{0.000000,0.000000,0.000000}%
\pgfsetfillcolor{currentfill}%
\pgfsetlinewidth{1.003750pt}%
\definecolor{currentstroke}{rgb}{0.000000,0.000000,0.000000}%
\pgfsetstrokecolor{currentstroke}%
\pgfsetdash{}{0pt}%
\pgfpathmoveto{\pgfqpoint{5.504545in}{0.684333in}}%
\pgfpathcurveto{\pgfqpoint{5.515596in}{0.684333in}}{\pgfqpoint{5.526195in}{0.688724in}}{\pgfqpoint{5.534008in}{0.696537in}}%
\pgfpathcurveto{\pgfqpoint{5.541822in}{0.704351in}}{\pgfqpoint{5.546212in}{0.714950in}}{\pgfqpoint{5.546212in}{0.726000in}}%
\pgfpathcurveto{\pgfqpoint{5.546212in}{0.737050in}}{\pgfqpoint{5.541822in}{0.747649in}}{\pgfqpoint{5.534008in}{0.755463in}}%
\pgfpathcurveto{\pgfqpoint{5.526195in}{0.763276in}}{\pgfqpoint{5.515596in}{0.767667in}}{\pgfqpoint{5.504545in}{0.767667in}}%
\pgfpathcurveto{\pgfqpoint{5.493495in}{0.767667in}}{\pgfqpoint{5.482896in}{0.763276in}}{\pgfqpoint{5.475083in}{0.755463in}}%
\pgfpathcurveto{\pgfqpoint{5.467269in}{0.747649in}}{\pgfqpoint{5.462879in}{0.737050in}}{\pgfqpoint{5.462879in}{0.726000in}}%
\pgfpathcurveto{\pgfqpoint{5.462879in}{0.714950in}}{\pgfqpoint{5.467269in}{0.704351in}}{\pgfqpoint{5.475083in}{0.696537in}}%
\pgfpathcurveto{\pgfqpoint{5.482896in}{0.688724in}}{\pgfqpoint{5.493495in}{0.684333in}}{\pgfqpoint{5.504545in}{0.684333in}}%
\pgfpathclose%
\pgfusepath{stroke,fill}%
\end{pgfscope}%
\begin{pgfscope}%
\pgfpathrectangle{\pgfqpoint{0.800000in}{0.528000in}}{\pgfqpoint{4.960000in}{3.696000in}}%
\pgfusepath{clip}%
\pgfsetbuttcap%
\pgfsetroundjoin%
\definecolor{currentfill}{rgb}{0.000000,0.000000,0.000000}%
\pgfsetfillcolor{currentfill}%
\pgfsetlinewidth{1.003750pt}%
\definecolor{currentstroke}{rgb}{0.000000,0.000000,0.000000}%
\pgfsetstrokecolor{currentstroke}%
\pgfsetdash{}{0pt}%
\pgfpathmoveto{\pgfqpoint{5.504545in}{0.684333in}}%
\pgfpathcurveto{\pgfqpoint{5.515596in}{0.684333in}}{\pgfqpoint{5.526195in}{0.688724in}}{\pgfqpoint{5.534008in}{0.696537in}}%
\pgfpathcurveto{\pgfqpoint{5.541822in}{0.704351in}}{\pgfqpoint{5.546212in}{0.714950in}}{\pgfqpoint{5.546212in}{0.726000in}}%
\pgfpathcurveto{\pgfqpoint{5.546212in}{0.737050in}}{\pgfqpoint{5.541822in}{0.747649in}}{\pgfqpoint{5.534008in}{0.755463in}}%
\pgfpathcurveto{\pgfqpoint{5.526195in}{0.763276in}}{\pgfqpoint{5.515596in}{0.767667in}}{\pgfqpoint{5.504545in}{0.767667in}}%
\pgfpathcurveto{\pgfqpoint{5.493495in}{0.767667in}}{\pgfqpoint{5.482896in}{0.763276in}}{\pgfqpoint{5.475083in}{0.755463in}}%
\pgfpathcurveto{\pgfqpoint{5.467269in}{0.747649in}}{\pgfqpoint{5.462879in}{0.737050in}}{\pgfqpoint{5.462879in}{0.726000in}}%
\pgfpathcurveto{\pgfqpoint{5.462879in}{0.714950in}}{\pgfqpoint{5.467269in}{0.704351in}}{\pgfqpoint{5.475083in}{0.696537in}}%
\pgfpathcurveto{\pgfqpoint{5.482896in}{0.688724in}}{\pgfqpoint{5.493495in}{0.684333in}}{\pgfqpoint{5.504545in}{0.684333in}}%
\pgfpathclose%
\pgfusepath{stroke,fill}%
\end{pgfscope}%
\begin{pgfscope}%
\pgfpathrectangle{\pgfqpoint{0.800000in}{0.528000in}}{\pgfqpoint{4.960000in}{3.696000in}}%
\pgfusepath{clip}%
\pgfsetbuttcap%
\pgfsetroundjoin%
\definecolor{currentfill}{rgb}{0.000000,0.000000,0.000000}%
\pgfsetfillcolor{currentfill}%
\pgfsetlinewidth{1.003750pt}%
\definecolor{currentstroke}{rgb}{0.000000,0.000000,0.000000}%
\pgfsetstrokecolor{currentstroke}%
\pgfsetdash{}{0pt}%
\pgfpathmoveto{\pgfqpoint{5.504545in}{0.684333in}}%
\pgfpathcurveto{\pgfqpoint{5.515596in}{0.684333in}}{\pgfqpoint{5.526195in}{0.688724in}}{\pgfqpoint{5.534008in}{0.696537in}}%
\pgfpathcurveto{\pgfqpoint{5.541822in}{0.704351in}}{\pgfqpoint{5.546212in}{0.714950in}}{\pgfqpoint{5.546212in}{0.726000in}}%
\pgfpathcurveto{\pgfqpoint{5.546212in}{0.737050in}}{\pgfqpoint{5.541822in}{0.747649in}}{\pgfqpoint{5.534008in}{0.755463in}}%
\pgfpathcurveto{\pgfqpoint{5.526195in}{0.763276in}}{\pgfqpoint{5.515596in}{0.767667in}}{\pgfqpoint{5.504545in}{0.767667in}}%
\pgfpathcurveto{\pgfqpoint{5.493495in}{0.767667in}}{\pgfqpoint{5.482896in}{0.763276in}}{\pgfqpoint{5.475083in}{0.755463in}}%
\pgfpathcurveto{\pgfqpoint{5.467269in}{0.747649in}}{\pgfqpoint{5.462879in}{0.737050in}}{\pgfqpoint{5.462879in}{0.726000in}}%
\pgfpathcurveto{\pgfqpoint{5.462879in}{0.714950in}}{\pgfqpoint{5.467269in}{0.704351in}}{\pgfqpoint{5.475083in}{0.696537in}}%
\pgfpathcurveto{\pgfqpoint{5.482896in}{0.688724in}}{\pgfqpoint{5.493495in}{0.684333in}}{\pgfqpoint{5.504545in}{0.684333in}}%
\pgfpathclose%
\pgfusepath{stroke,fill}%
\end{pgfscope}%
\begin{pgfscope}%
\pgfpathrectangle{\pgfqpoint{0.800000in}{0.528000in}}{\pgfqpoint{4.960000in}{3.696000in}}%
\pgfusepath{clip}%
\pgfsetbuttcap%
\pgfsetroundjoin%
\definecolor{currentfill}{rgb}{0.000000,0.000000,0.000000}%
\pgfsetfillcolor{currentfill}%
\pgfsetlinewidth{1.003750pt}%
\definecolor{currentstroke}{rgb}{0.000000,0.000000,0.000000}%
\pgfsetstrokecolor{currentstroke}%
\pgfsetdash{}{0pt}%
\pgfpathmoveto{\pgfqpoint{5.504545in}{3.984333in}}%
\pgfpathcurveto{\pgfqpoint{5.515596in}{3.984333in}}{\pgfqpoint{5.526195in}{3.988724in}}{\pgfqpoint{5.534008in}{3.996537in}}%
\pgfpathcurveto{\pgfqpoint{5.541822in}{4.004351in}}{\pgfqpoint{5.546212in}{4.014950in}}{\pgfqpoint{5.546212in}{4.026000in}}%
\pgfpathcurveto{\pgfqpoint{5.546212in}{4.037050in}}{\pgfqpoint{5.541822in}{4.047649in}}{\pgfqpoint{5.534008in}{4.055463in}}%
\pgfpathcurveto{\pgfqpoint{5.526195in}{4.063276in}}{\pgfqpoint{5.515596in}{4.067667in}}{\pgfqpoint{5.504545in}{4.067667in}}%
\pgfpathcurveto{\pgfqpoint{5.493495in}{4.067667in}}{\pgfqpoint{5.482896in}{4.063276in}}{\pgfqpoint{5.475083in}{4.055463in}}%
\pgfpathcurveto{\pgfqpoint{5.467269in}{4.047649in}}{\pgfqpoint{5.462879in}{4.037050in}}{\pgfqpoint{5.462879in}{4.026000in}}%
\pgfpathcurveto{\pgfqpoint{5.462879in}{4.014950in}}{\pgfqpoint{5.467269in}{4.004351in}}{\pgfqpoint{5.475083in}{3.996537in}}%
\pgfpathcurveto{\pgfqpoint{5.482896in}{3.988724in}}{\pgfqpoint{5.493495in}{3.984333in}}{\pgfqpoint{5.504545in}{3.984333in}}%
\pgfpathclose%
\pgfusepath{stroke,fill}%
\end{pgfscope}%
\begin{pgfscope}%
\pgfpathrectangle{\pgfqpoint{0.800000in}{0.528000in}}{\pgfqpoint{4.960000in}{3.696000in}}%
\pgfusepath{clip}%
\pgfsetbuttcap%
\pgfsetroundjoin%
\definecolor{currentfill}{rgb}{0.000000,0.000000,0.000000}%
\pgfsetfillcolor{currentfill}%
\pgfsetlinewidth{1.003750pt}%
\definecolor{currentstroke}{rgb}{0.000000,0.000000,0.000000}%
\pgfsetstrokecolor{currentstroke}%
\pgfsetdash{}{0pt}%
\pgfpathmoveto{\pgfqpoint{5.504545in}{0.684333in}}%
\pgfpathcurveto{\pgfqpoint{5.515596in}{0.684333in}}{\pgfqpoint{5.526195in}{0.688724in}}{\pgfqpoint{5.534008in}{0.696537in}}%
\pgfpathcurveto{\pgfqpoint{5.541822in}{0.704351in}}{\pgfqpoint{5.546212in}{0.714950in}}{\pgfqpoint{5.546212in}{0.726000in}}%
\pgfpathcurveto{\pgfqpoint{5.546212in}{0.737050in}}{\pgfqpoint{5.541822in}{0.747649in}}{\pgfqpoint{5.534008in}{0.755463in}}%
\pgfpathcurveto{\pgfqpoint{5.526195in}{0.763276in}}{\pgfqpoint{5.515596in}{0.767667in}}{\pgfqpoint{5.504545in}{0.767667in}}%
\pgfpathcurveto{\pgfqpoint{5.493495in}{0.767667in}}{\pgfqpoint{5.482896in}{0.763276in}}{\pgfqpoint{5.475083in}{0.755463in}}%
\pgfpathcurveto{\pgfqpoint{5.467269in}{0.747649in}}{\pgfqpoint{5.462879in}{0.737050in}}{\pgfqpoint{5.462879in}{0.726000in}}%
\pgfpathcurveto{\pgfqpoint{5.462879in}{0.714950in}}{\pgfqpoint{5.467269in}{0.704351in}}{\pgfqpoint{5.475083in}{0.696537in}}%
\pgfpathcurveto{\pgfqpoint{5.482896in}{0.688724in}}{\pgfqpoint{5.493495in}{0.684333in}}{\pgfqpoint{5.504545in}{0.684333in}}%
\pgfpathclose%
\pgfusepath{stroke,fill}%
\end{pgfscope}%
\begin{pgfscope}%
\pgfpathrectangle{\pgfqpoint{0.800000in}{0.528000in}}{\pgfqpoint{4.960000in}{3.696000in}}%
\pgfusepath{clip}%
\pgfsetbuttcap%
\pgfsetroundjoin%
\definecolor{currentfill}{rgb}{0.000000,0.000000,0.000000}%
\pgfsetfillcolor{currentfill}%
\pgfsetlinewidth{1.003750pt}%
\definecolor{currentstroke}{rgb}{0.000000,0.000000,0.000000}%
\pgfsetstrokecolor{currentstroke}%
\pgfsetdash{}{0pt}%
\pgfpathmoveto{\pgfqpoint{5.504545in}{0.684333in}}%
\pgfpathcurveto{\pgfqpoint{5.515596in}{0.684333in}}{\pgfqpoint{5.526195in}{0.688724in}}{\pgfqpoint{5.534008in}{0.696537in}}%
\pgfpathcurveto{\pgfqpoint{5.541822in}{0.704351in}}{\pgfqpoint{5.546212in}{0.714950in}}{\pgfqpoint{5.546212in}{0.726000in}}%
\pgfpathcurveto{\pgfqpoint{5.546212in}{0.737050in}}{\pgfqpoint{5.541822in}{0.747649in}}{\pgfqpoint{5.534008in}{0.755463in}}%
\pgfpathcurveto{\pgfqpoint{5.526195in}{0.763276in}}{\pgfqpoint{5.515596in}{0.767667in}}{\pgfqpoint{5.504545in}{0.767667in}}%
\pgfpathcurveto{\pgfqpoint{5.493495in}{0.767667in}}{\pgfqpoint{5.482896in}{0.763276in}}{\pgfqpoint{5.475083in}{0.755463in}}%
\pgfpathcurveto{\pgfqpoint{5.467269in}{0.747649in}}{\pgfqpoint{5.462879in}{0.737050in}}{\pgfqpoint{5.462879in}{0.726000in}}%
\pgfpathcurveto{\pgfqpoint{5.462879in}{0.714950in}}{\pgfqpoint{5.467269in}{0.704351in}}{\pgfqpoint{5.475083in}{0.696537in}}%
\pgfpathcurveto{\pgfqpoint{5.482896in}{0.688724in}}{\pgfqpoint{5.493495in}{0.684333in}}{\pgfqpoint{5.504545in}{0.684333in}}%
\pgfpathclose%
\pgfusepath{stroke,fill}%
\end{pgfscope}%
\begin{pgfscope}%
\pgfpathrectangle{\pgfqpoint{0.800000in}{0.528000in}}{\pgfqpoint{4.960000in}{3.696000in}}%
\pgfusepath{clip}%
\pgfsetbuttcap%
\pgfsetroundjoin%
\definecolor{currentfill}{rgb}{0.000000,0.000000,0.000000}%
\pgfsetfillcolor{currentfill}%
\pgfsetlinewidth{1.003750pt}%
\definecolor{currentstroke}{rgb}{0.000000,0.000000,0.000000}%
\pgfsetstrokecolor{currentstroke}%
\pgfsetdash{}{0pt}%
\pgfpathmoveto{\pgfqpoint{5.504545in}{3.984333in}}%
\pgfpathcurveto{\pgfqpoint{5.515596in}{3.984333in}}{\pgfqpoint{5.526195in}{3.988724in}}{\pgfqpoint{5.534008in}{3.996537in}}%
\pgfpathcurveto{\pgfqpoint{5.541822in}{4.004351in}}{\pgfqpoint{5.546212in}{4.014950in}}{\pgfqpoint{5.546212in}{4.026000in}}%
\pgfpathcurveto{\pgfqpoint{5.546212in}{4.037050in}}{\pgfqpoint{5.541822in}{4.047649in}}{\pgfqpoint{5.534008in}{4.055463in}}%
\pgfpathcurveto{\pgfqpoint{5.526195in}{4.063276in}}{\pgfqpoint{5.515596in}{4.067667in}}{\pgfqpoint{5.504545in}{4.067667in}}%
\pgfpathcurveto{\pgfqpoint{5.493495in}{4.067667in}}{\pgfqpoint{5.482896in}{4.063276in}}{\pgfqpoint{5.475083in}{4.055463in}}%
\pgfpathcurveto{\pgfqpoint{5.467269in}{4.047649in}}{\pgfqpoint{5.462879in}{4.037050in}}{\pgfqpoint{5.462879in}{4.026000in}}%
\pgfpathcurveto{\pgfqpoint{5.462879in}{4.014950in}}{\pgfqpoint{5.467269in}{4.004351in}}{\pgfqpoint{5.475083in}{3.996537in}}%
\pgfpathcurveto{\pgfqpoint{5.482896in}{3.988724in}}{\pgfqpoint{5.493495in}{3.984333in}}{\pgfqpoint{5.504545in}{3.984333in}}%
\pgfpathclose%
\pgfusepath{stroke,fill}%
\end{pgfscope}%
\begin{pgfscope}%
\pgfpathrectangle{\pgfqpoint{0.800000in}{0.528000in}}{\pgfqpoint{4.960000in}{3.696000in}}%
\pgfusepath{clip}%
\pgfsetbuttcap%
\pgfsetroundjoin%
\definecolor{currentfill}{rgb}{0.000000,0.000000,0.000000}%
\pgfsetfillcolor{currentfill}%
\pgfsetlinewidth{1.003750pt}%
\definecolor{currentstroke}{rgb}{0.000000,0.000000,0.000000}%
\pgfsetstrokecolor{currentstroke}%
\pgfsetdash{}{0pt}%
\pgfpathmoveto{\pgfqpoint{5.504545in}{3.984333in}}%
\pgfpathcurveto{\pgfqpoint{5.515596in}{3.984333in}}{\pgfqpoint{5.526195in}{3.988724in}}{\pgfqpoint{5.534008in}{3.996537in}}%
\pgfpathcurveto{\pgfqpoint{5.541822in}{4.004351in}}{\pgfqpoint{5.546212in}{4.014950in}}{\pgfqpoint{5.546212in}{4.026000in}}%
\pgfpathcurveto{\pgfqpoint{5.546212in}{4.037050in}}{\pgfqpoint{5.541822in}{4.047649in}}{\pgfqpoint{5.534008in}{4.055463in}}%
\pgfpathcurveto{\pgfqpoint{5.526195in}{4.063276in}}{\pgfqpoint{5.515596in}{4.067667in}}{\pgfqpoint{5.504545in}{4.067667in}}%
\pgfpathcurveto{\pgfqpoint{5.493495in}{4.067667in}}{\pgfqpoint{5.482896in}{4.063276in}}{\pgfqpoint{5.475083in}{4.055463in}}%
\pgfpathcurveto{\pgfqpoint{5.467269in}{4.047649in}}{\pgfqpoint{5.462879in}{4.037050in}}{\pgfqpoint{5.462879in}{4.026000in}}%
\pgfpathcurveto{\pgfqpoint{5.462879in}{4.014950in}}{\pgfqpoint{5.467269in}{4.004351in}}{\pgfqpoint{5.475083in}{3.996537in}}%
\pgfpathcurveto{\pgfqpoint{5.482896in}{3.988724in}}{\pgfqpoint{5.493495in}{3.984333in}}{\pgfqpoint{5.504545in}{3.984333in}}%
\pgfpathclose%
\pgfusepath{stroke,fill}%
\end{pgfscope}%
\begin{pgfscope}%
\pgfpathrectangle{\pgfqpoint{0.800000in}{0.528000in}}{\pgfqpoint{4.960000in}{3.696000in}}%
\pgfusepath{clip}%
\pgfsetbuttcap%
\pgfsetroundjoin%
\definecolor{currentfill}{rgb}{0.000000,0.000000,0.000000}%
\pgfsetfillcolor{currentfill}%
\pgfsetlinewidth{1.003750pt}%
\definecolor{currentstroke}{rgb}{0.000000,0.000000,0.000000}%
\pgfsetstrokecolor{currentstroke}%
\pgfsetdash{}{0pt}%
\pgfpathmoveto{\pgfqpoint{5.504545in}{0.684333in}}%
\pgfpathcurveto{\pgfqpoint{5.515596in}{0.684333in}}{\pgfqpoint{5.526195in}{0.688724in}}{\pgfqpoint{5.534008in}{0.696537in}}%
\pgfpathcurveto{\pgfqpoint{5.541822in}{0.704351in}}{\pgfqpoint{5.546212in}{0.714950in}}{\pgfqpoint{5.546212in}{0.726000in}}%
\pgfpathcurveto{\pgfqpoint{5.546212in}{0.737050in}}{\pgfqpoint{5.541822in}{0.747649in}}{\pgfqpoint{5.534008in}{0.755463in}}%
\pgfpathcurveto{\pgfqpoint{5.526195in}{0.763276in}}{\pgfqpoint{5.515596in}{0.767667in}}{\pgfqpoint{5.504545in}{0.767667in}}%
\pgfpathcurveto{\pgfqpoint{5.493495in}{0.767667in}}{\pgfqpoint{5.482896in}{0.763276in}}{\pgfqpoint{5.475083in}{0.755463in}}%
\pgfpathcurveto{\pgfqpoint{5.467269in}{0.747649in}}{\pgfqpoint{5.462879in}{0.737050in}}{\pgfqpoint{5.462879in}{0.726000in}}%
\pgfpathcurveto{\pgfqpoint{5.462879in}{0.714950in}}{\pgfqpoint{5.467269in}{0.704351in}}{\pgfqpoint{5.475083in}{0.696537in}}%
\pgfpathcurveto{\pgfqpoint{5.482896in}{0.688724in}}{\pgfqpoint{5.493495in}{0.684333in}}{\pgfqpoint{5.504545in}{0.684333in}}%
\pgfpathclose%
\pgfusepath{stroke,fill}%
\end{pgfscope}%
\begin{pgfscope}%
\pgfpathrectangle{\pgfqpoint{0.800000in}{0.528000in}}{\pgfqpoint{4.960000in}{3.696000in}}%
\pgfusepath{clip}%
\pgfsetbuttcap%
\pgfsetroundjoin%
\definecolor{currentfill}{rgb}{0.000000,0.000000,0.000000}%
\pgfsetfillcolor{currentfill}%
\pgfsetlinewidth{1.003750pt}%
\definecolor{currentstroke}{rgb}{0.000000,0.000000,0.000000}%
\pgfsetstrokecolor{currentstroke}%
\pgfsetdash{}{0pt}%
\pgfpathmoveto{\pgfqpoint{5.504545in}{0.684333in}}%
\pgfpathcurveto{\pgfqpoint{5.515596in}{0.684333in}}{\pgfqpoint{5.526195in}{0.688724in}}{\pgfqpoint{5.534008in}{0.696537in}}%
\pgfpathcurveto{\pgfqpoint{5.541822in}{0.704351in}}{\pgfqpoint{5.546212in}{0.714950in}}{\pgfqpoint{5.546212in}{0.726000in}}%
\pgfpathcurveto{\pgfqpoint{5.546212in}{0.737050in}}{\pgfqpoint{5.541822in}{0.747649in}}{\pgfqpoint{5.534008in}{0.755463in}}%
\pgfpathcurveto{\pgfqpoint{5.526195in}{0.763276in}}{\pgfqpoint{5.515596in}{0.767667in}}{\pgfqpoint{5.504545in}{0.767667in}}%
\pgfpathcurveto{\pgfqpoint{5.493495in}{0.767667in}}{\pgfqpoint{5.482896in}{0.763276in}}{\pgfqpoint{5.475083in}{0.755463in}}%
\pgfpathcurveto{\pgfqpoint{5.467269in}{0.747649in}}{\pgfqpoint{5.462879in}{0.737050in}}{\pgfqpoint{5.462879in}{0.726000in}}%
\pgfpathcurveto{\pgfqpoint{5.462879in}{0.714950in}}{\pgfqpoint{5.467269in}{0.704351in}}{\pgfqpoint{5.475083in}{0.696537in}}%
\pgfpathcurveto{\pgfqpoint{5.482896in}{0.688724in}}{\pgfqpoint{5.493495in}{0.684333in}}{\pgfqpoint{5.504545in}{0.684333in}}%
\pgfpathclose%
\pgfusepath{stroke,fill}%
\end{pgfscope}%
\begin{pgfscope}%
\pgfpathrectangle{\pgfqpoint{0.800000in}{0.528000in}}{\pgfqpoint{4.960000in}{3.696000in}}%
\pgfusepath{clip}%
\pgfsetbuttcap%
\pgfsetroundjoin%
\definecolor{currentfill}{rgb}{0.000000,0.000000,0.000000}%
\pgfsetfillcolor{currentfill}%
\pgfsetlinewidth{1.003750pt}%
\definecolor{currentstroke}{rgb}{0.000000,0.000000,0.000000}%
\pgfsetstrokecolor{currentstroke}%
\pgfsetdash{}{0pt}%
\pgfpathmoveto{\pgfqpoint{5.504545in}{3.984333in}}%
\pgfpathcurveto{\pgfqpoint{5.515596in}{3.984333in}}{\pgfqpoint{5.526195in}{3.988724in}}{\pgfqpoint{5.534008in}{3.996537in}}%
\pgfpathcurveto{\pgfqpoint{5.541822in}{4.004351in}}{\pgfqpoint{5.546212in}{4.014950in}}{\pgfqpoint{5.546212in}{4.026000in}}%
\pgfpathcurveto{\pgfqpoint{5.546212in}{4.037050in}}{\pgfqpoint{5.541822in}{4.047649in}}{\pgfqpoint{5.534008in}{4.055463in}}%
\pgfpathcurveto{\pgfqpoint{5.526195in}{4.063276in}}{\pgfqpoint{5.515596in}{4.067667in}}{\pgfqpoint{5.504545in}{4.067667in}}%
\pgfpathcurveto{\pgfqpoint{5.493495in}{4.067667in}}{\pgfqpoint{5.482896in}{4.063276in}}{\pgfqpoint{5.475083in}{4.055463in}}%
\pgfpathcurveto{\pgfqpoint{5.467269in}{4.047649in}}{\pgfqpoint{5.462879in}{4.037050in}}{\pgfqpoint{5.462879in}{4.026000in}}%
\pgfpathcurveto{\pgfqpoint{5.462879in}{4.014950in}}{\pgfqpoint{5.467269in}{4.004351in}}{\pgfqpoint{5.475083in}{3.996537in}}%
\pgfpathcurveto{\pgfqpoint{5.482896in}{3.988724in}}{\pgfqpoint{5.493495in}{3.984333in}}{\pgfqpoint{5.504545in}{3.984333in}}%
\pgfpathclose%
\pgfusepath{stroke,fill}%
\end{pgfscope}%
\begin{pgfscope}%
\pgfpathrectangle{\pgfqpoint{0.800000in}{0.528000in}}{\pgfqpoint{4.960000in}{3.696000in}}%
\pgfusepath{clip}%
\pgfsetbuttcap%
\pgfsetroundjoin%
\definecolor{currentfill}{rgb}{0.000000,0.000000,0.000000}%
\pgfsetfillcolor{currentfill}%
\pgfsetlinewidth{1.003750pt}%
\definecolor{currentstroke}{rgb}{0.000000,0.000000,0.000000}%
\pgfsetstrokecolor{currentstroke}%
\pgfsetdash{}{0pt}%
\pgfpathmoveto{\pgfqpoint{5.504545in}{0.684333in}}%
\pgfpathcurveto{\pgfqpoint{5.515596in}{0.684333in}}{\pgfqpoint{5.526195in}{0.688724in}}{\pgfqpoint{5.534008in}{0.696537in}}%
\pgfpathcurveto{\pgfqpoint{5.541822in}{0.704351in}}{\pgfqpoint{5.546212in}{0.714950in}}{\pgfqpoint{5.546212in}{0.726000in}}%
\pgfpathcurveto{\pgfqpoint{5.546212in}{0.737050in}}{\pgfqpoint{5.541822in}{0.747649in}}{\pgfqpoint{5.534008in}{0.755463in}}%
\pgfpathcurveto{\pgfqpoint{5.526195in}{0.763276in}}{\pgfqpoint{5.515596in}{0.767667in}}{\pgfqpoint{5.504545in}{0.767667in}}%
\pgfpathcurveto{\pgfqpoint{5.493495in}{0.767667in}}{\pgfqpoint{5.482896in}{0.763276in}}{\pgfqpoint{5.475083in}{0.755463in}}%
\pgfpathcurveto{\pgfqpoint{5.467269in}{0.747649in}}{\pgfqpoint{5.462879in}{0.737050in}}{\pgfqpoint{5.462879in}{0.726000in}}%
\pgfpathcurveto{\pgfqpoint{5.462879in}{0.714950in}}{\pgfqpoint{5.467269in}{0.704351in}}{\pgfqpoint{5.475083in}{0.696537in}}%
\pgfpathcurveto{\pgfqpoint{5.482896in}{0.688724in}}{\pgfqpoint{5.493495in}{0.684333in}}{\pgfqpoint{5.504545in}{0.684333in}}%
\pgfpathclose%
\pgfusepath{stroke,fill}%
\end{pgfscope}%
\begin{pgfscope}%
\pgfpathrectangle{\pgfqpoint{0.800000in}{0.528000in}}{\pgfqpoint{4.960000in}{3.696000in}}%
\pgfusepath{clip}%
\pgfsetbuttcap%
\pgfsetroundjoin%
\definecolor{currentfill}{rgb}{0.000000,0.000000,0.000000}%
\pgfsetfillcolor{currentfill}%
\pgfsetlinewidth{1.003750pt}%
\definecolor{currentstroke}{rgb}{0.000000,0.000000,0.000000}%
\pgfsetstrokecolor{currentstroke}%
\pgfsetdash{}{0pt}%
\pgfpathmoveto{\pgfqpoint{5.504545in}{0.684333in}}%
\pgfpathcurveto{\pgfqpoint{5.515596in}{0.684333in}}{\pgfqpoint{5.526195in}{0.688724in}}{\pgfqpoint{5.534008in}{0.696537in}}%
\pgfpathcurveto{\pgfqpoint{5.541822in}{0.704351in}}{\pgfqpoint{5.546212in}{0.714950in}}{\pgfqpoint{5.546212in}{0.726000in}}%
\pgfpathcurveto{\pgfqpoint{5.546212in}{0.737050in}}{\pgfqpoint{5.541822in}{0.747649in}}{\pgfqpoint{5.534008in}{0.755463in}}%
\pgfpathcurveto{\pgfqpoint{5.526195in}{0.763276in}}{\pgfqpoint{5.515596in}{0.767667in}}{\pgfqpoint{5.504545in}{0.767667in}}%
\pgfpathcurveto{\pgfqpoint{5.493495in}{0.767667in}}{\pgfqpoint{5.482896in}{0.763276in}}{\pgfqpoint{5.475083in}{0.755463in}}%
\pgfpathcurveto{\pgfqpoint{5.467269in}{0.747649in}}{\pgfqpoint{5.462879in}{0.737050in}}{\pgfqpoint{5.462879in}{0.726000in}}%
\pgfpathcurveto{\pgfqpoint{5.462879in}{0.714950in}}{\pgfqpoint{5.467269in}{0.704351in}}{\pgfqpoint{5.475083in}{0.696537in}}%
\pgfpathcurveto{\pgfqpoint{5.482896in}{0.688724in}}{\pgfqpoint{5.493495in}{0.684333in}}{\pgfqpoint{5.504545in}{0.684333in}}%
\pgfpathclose%
\pgfusepath{stroke,fill}%
\end{pgfscope}%
\begin{pgfscope}%
\pgfpathrectangle{\pgfqpoint{0.800000in}{0.528000in}}{\pgfqpoint{4.960000in}{3.696000in}}%
\pgfusepath{clip}%
\pgfsetbuttcap%
\pgfsetroundjoin%
\definecolor{currentfill}{rgb}{0.000000,0.000000,0.000000}%
\pgfsetfillcolor{currentfill}%
\pgfsetlinewidth{1.003750pt}%
\definecolor{currentstroke}{rgb}{0.000000,0.000000,0.000000}%
\pgfsetstrokecolor{currentstroke}%
\pgfsetdash{}{0pt}%
\pgfpathmoveto{\pgfqpoint{5.504545in}{3.984333in}}%
\pgfpathcurveto{\pgfqpoint{5.515596in}{3.984333in}}{\pgfqpoint{5.526195in}{3.988724in}}{\pgfqpoint{5.534008in}{3.996537in}}%
\pgfpathcurveto{\pgfqpoint{5.541822in}{4.004351in}}{\pgfqpoint{5.546212in}{4.014950in}}{\pgfqpoint{5.546212in}{4.026000in}}%
\pgfpathcurveto{\pgfqpoint{5.546212in}{4.037050in}}{\pgfqpoint{5.541822in}{4.047649in}}{\pgfqpoint{5.534008in}{4.055463in}}%
\pgfpathcurveto{\pgfqpoint{5.526195in}{4.063276in}}{\pgfqpoint{5.515596in}{4.067667in}}{\pgfqpoint{5.504545in}{4.067667in}}%
\pgfpathcurveto{\pgfqpoint{5.493495in}{4.067667in}}{\pgfqpoint{5.482896in}{4.063276in}}{\pgfqpoint{5.475083in}{4.055463in}}%
\pgfpathcurveto{\pgfqpoint{5.467269in}{4.047649in}}{\pgfqpoint{5.462879in}{4.037050in}}{\pgfqpoint{5.462879in}{4.026000in}}%
\pgfpathcurveto{\pgfqpoint{5.462879in}{4.014950in}}{\pgfqpoint{5.467269in}{4.004351in}}{\pgfqpoint{5.475083in}{3.996537in}}%
\pgfpathcurveto{\pgfqpoint{5.482896in}{3.988724in}}{\pgfqpoint{5.493495in}{3.984333in}}{\pgfqpoint{5.504545in}{3.984333in}}%
\pgfpathclose%
\pgfusepath{stroke,fill}%
\end{pgfscope}%
\begin{pgfscope}%
\pgfpathrectangle{\pgfqpoint{0.800000in}{0.528000in}}{\pgfqpoint{4.960000in}{3.696000in}}%
\pgfusepath{clip}%
\pgfsetbuttcap%
\pgfsetroundjoin%
\definecolor{currentfill}{rgb}{0.000000,0.000000,0.000000}%
\pgfsetfillcolor{currentfill}%
\pgfsetlinewidth{1.003750pt}%
\definecolor{currentstroke}{rgb}{0.000000,0.000000,0.000000}%
\pgfsetstrokecolor{currentstroke}%
\pgfsetdash{}{0pt}%
\pgfpathmoveto{\pgfqpoint{5.504545in}{0.684333in}}%
\pgfpathcurveto{\pgfqpoint{5.515596in}{0.684333in}}{\pgfqpoint{5.526195in}{0.688724in}}{\pgfqpoint{5.534008in}{0.696537in}}%
\pgfpathcurveto{\pgfqpoint{5.541822in}{0.704351in}}{\pgfqpoint{5.546212in}{0.714950in}}{\pgfqpoint{5.546212in}{0.726000in}}%
\pgfpathcurveto{\pgfqpoint{5.546212in}{0.737050in}}{\pgfqpoint{5.541822in}{0.747649in}}{\pgfqpoint{5.534008in}{0.755463in}}%
\pgfpathcurveto{\pgfqpoint{5.526195in}{0.763276in}}{\pgfqpoint{5.515596in}{0.767667in}}{\pgfqpoint{5.504545in}{0.767667in}}%
\pgfpathcurveto{\pgfqpoint{5.493495in}{0.767667in}}{\pgfqpoint{5.482896in}{0.763276in}}{\pgfqpoint{5.475083in}{0.755463in}}%
\pgfpathcurveto{\pgfqpoint{5.467269in}{0.747649in}}{\pgfqpoint{5.462879in}{0.737050in}}{\pgfqpoint{5.462879in}{0.726000in}}%
\pgfpathcurveto{\pgfqpoint{5.462879in}{0.714950in}}{\pgfqpoint{5.467269in}{0.704351in}}{\pgfqpoint{5.475083in}{0.696537in}}%
\pgfpathcurveto{\pgfqpoint{5.482896in}{0.688724in}}{\pgfqpoint{5.493495in}{0.684333in}}{\pgfqpoint{5.504545in}{0.684333in}}%
\pgfpathclose%
\pgfusepath{stroke,fill}%
\end{pgfscope}%
\begin{pgfscope}%
\pgfpathrectangle{\pgfqpoint{0.800000in}{0.528000in}}{\pgfqpoint{4.960000in}{3.696000in}}%
\pgfusepath{clip}%
\pgfsetbuttcap%
\pgfsetroundjoin%
\definecolor{currentfill}{rgb}{0.000000,0.000000,0.000000}%
\pgfsetfillcolor{currentfill}%
\pgfsetlinewidth{1.003750pt}%
\definecolor{currentstroke}{rgb}{0.000000,0.000000,0.000000}%
\pgfsetstrokecolor{currentstroke}%
\pgfsetdash{}{0pt}%
\pgfpathmoveto{\pgfqpoint{5.504545in}{0.684333in}}%
\pgfpathcurveto{\pgfqpoint{5.515596in}{0.684333in}}{\pgfqpoint{5.526195in}{0.688724in}}{\pgfqpoint{5.534008in}{0.696537in}}%
\pgfpathcurveto{\pgfqpoint{5.541822in}{0.704351in}}{\pgfqpoint{5.546212in}{0.714950in}}{\pgfqpoint{5.546212in}{0.726000in}}%
\pgfpathcurveto{\pgfqpoint{5.546212in}{0.737050in}}{\pgfqpoint{5.541822in}{0.747649in}}{\pgfqpoint{5.534008in}{0.755463in}}%
\pgfpathcurveto{\pgfqpoint{5.526195in}{0.763276in}}{\pgfqpoint{5.515596in}{0.767667in}}{\pgfqpoint{5.504545in}{0.767667in}}%
\pgfpathcurveto{\pgfqpoint{5.493495in}{0.767667in}}{\pgfqpoint{5.482896in}{0.763276in}}{\pgfqpoint{5.475083in}{0.755463in}}%
\pgfpathcurveto{\pgfqpoint{5.467269in}{0.747649in}}{\pgfqpoint{5.462879in}{0.737050in}}{\pgfqpoint{5.462879in}{0.726000in}}%
\pgfpathcurveto{\pgfqpoint{5.462879in}{0.714950in}}{\pgfqpoint{5.467269in}{0.704351in}}{\pgfqpoint{5.475083in}{0.696537in}}%
\pgfpathcurveto{\pgfqpoint{5.482896in}{0.688724in}}{\pgfqpoint{5.493495in}{0.684333in}}{\pgfqpoint{5.504545in}{0.684333in}}%
\pgfpathclose%
\pgfusepath{stroke,fill}%
\end{pgfscope}%
\begin{pgfscope}%
\pgfpathrectangle{\pgfqpoint{0.800000in}{0.528000in}}{\pgfqpoint{4.960000in}{3.696000in}}%
\pgfusepath{clip}%
\pgfsetbuttcap%
\pgfsetroundjoin%
\definecolor{currentfill}{rgb}{0.000000,0.000000,0.000000}%
\pgfsetfillcolor{currentfill}%
\pgfsetlinewidth{1.003750pt}%
\definecolor{currentstroke}{rgb}{0.000000,0.000000,0.000000}%
\pgfsetstrokecolor{currentstroke}%
\pgfsetdash{}{0pt}%
\pgfpathmoveto{\pgfqpoint{5.504545in}{3.984333in}}%
\pgfpathcurveto{\pgfqpoint{5.515596in}{3.984333in}}{\pgfqpoint{5.526195in}{3.988724in}}{\pgfqpoint{5.534008in}{3.996537in}}%
\pgfpathcurveto{\pgfqpoint{5.541822in}{4.004351in}}{\pgfqpoint{5.546212in}{4.014950in}}{\pgfqpoint{5.546212in}{4.026000in}}%
\pgfpathcurveto{\pgfqpoint{5.546212in}{4.037050in}}{\pgfqpoint{5.541822in}{4.047649in}}{\pgfqpoint{5.534008in}{4.055463in}}%
\pgfpathcurveto{\pgfqpoint{5.526195in}{4.063276in}}{\pgfqpoint{5.515596in}{4.067667in}}{\pgfqpoint{5.504545in}{4.067667in}}%
\pgfpathcurveto{\pgfqpoint{5.493495in}{4.067667in}}{\pgfqpoint{5.482896in}{4.063276in}}{\pgfqpoint{5.475083in}{4.055463in}}%
\pgfpathcurveto{\pgfqpoint{5.467269in}{4.047649in}}{\pgfqpoint{5.462879in}{4.037050in}}{\pgfqpoint{5.462879in}{4.026000in}}%
\pgfpathcurveto{\pgfqpoint{5.462879in}{4.014950in}}{\pgfqpoint{5.467269in}{4.004351in}}{\pgfqpoint{5.475083in}{3.996537in}}%
\pgfpathcurveto{\pgfqpoint{5.482896in}{3.988724in}}{\pgfqpoint{5.493495in}{3.984333in}}{\pgfqpoint{5.504545in}{3.984333in}}%
\pgfpathclose%
\pgfusepath{stroke,fill}%
\end{pgfscope}%
\begin{pgfscope}%
\pgfpathrectangle{\pgfqpoint{0.800000in}{0.528000in}}{\pgfqpoint{4.960000in}{3.696000in}}%
\pgfusepath{clip}%
\pgfsetbuttcap%
\pgfsetroundjoin%
\definecolor{currentfill}{rgb}{0.000000,0.000000,0.000000}%
\pgfsetfillcolor{currentfill}%
\pgfsetlinewidth{1.003750pt}%
\definecolor{currentstroke}{rgb}{0.000000,0.000000,0.000000}%
\pgfsetstrokecolor{currentstroke}%
\pgfsetdash{}{0pt}%
\pgfpathmoveto{\pgfqpoint{5.504545in}{3.984333in}}%
\pgfpathcurveto{\pgfqpoint{5.515596in}{3.984333in}}{\pgfqpoint{5.526195in}{3.988724in}}{\pgfqpoint{5.534008in}{3.996537in}}%
\pgfpathcurveto{\pgfqpoint{5.541822in}{4.004351in}}{\pgfqpoint{5.546212in}{4.014950in}}{\pgfqpoint{5.546212in}{4.026000in}}%
\pgfpathcurveto{\pgfqpoint{5.546212in}{4.037050in}}{\pgfqpoint{5.541822in}{4.047649in}}{\pgfqpoint{5.534008in}{4.055463in}}%
\pgfpathcurveto{\pgfqpoint{5.526195in}{4.063276in}}{\pgfqpoint{5.515596in}{4.067667in}}{\pgfqpoint{5.504545in}{4.067667in}}%
\pgfpathcurveto{\pgfqpoint{5.493495in}{4.067667in}}{\pgfqpoint{5.482896in}{4.063276in}}{\pgfqpoint{5.475083in}{4.055463in}}%
\pgfpathcurveto{\pgfqpoint{5.467269in}{4.047649in}}{\pgfqpoint{5.462879in}{4.037050in}}{\pgfqpoint{5.462879in}{4.026000in}}%
\pgfpathcurveto{\pgfqpoint{5.462879in}{4.014950in}}{\pgfqpoint{5.467269in}{4.004351in}}{\pgfqpoint{5.475083in}{3.996537in}}%
\pgfpathcurveto{\pgfqpoint{5.482896in}{3.988724in}}{\pgfqpoint{5.493495in}{3.984333in}}{\pgfqpoint{5.504545in}{3.984333in}}%
\pgfpathclose%
\pgfusepath{stroke,fill}%
\end{pgfscope}%
\begin{pgfscope}%
\pgfpathrectangle{\pgfqpoint{0.800000in}{0.528000in}}{\pgfqpoint{4.960000in}{3.696000in}}%
\pgfusepath{clip}%
\pgfsetbuttcap%
\pgfsetroundjoin%
\definecolor{currentfill}{rgb}{0.000000,0.000000,0.000000}%
\pgfsetfillcolor{currentfill}%
\pgfsetlinewidth{1.003750pt}%
\definecolor{currentstroke}{rgb}{0.000000,0.000000,0.000000}%
\pgfsetstrokecolor{currentstroke}%
\pgfsetdash{}{0pt}%
\pgfpathmoveto{\pgfqpoint{5.504545in}{3.984333in}}%
\pgfpathcurveto{\pgfqpoint{5.515596in}{3.984333in}}{\pgfqpoint{5.526195in}{3.988724in}}{\pgfqpoint{5.534008in}{3.996537in}}%
\pgfpathcurveto{\pgfqpoint{5.541822in}{4.004351in}}{\pgfqpoint{5.546212in}{4.014950in}}{\pgfqpoint{5.546212in}{4.026000in}}%
\pgfpathcurveto{\pgfqpoint{5.546212in}{4.037050in}}{\pgfqpoint{5.541822in}{4.047649in}}{\pgfqpoint{5.534008in}{4.055463in}}%
\pgfpathcurveto{\pgfqpoint{5.526195in}{4.063276in}}{\pgfqpoint{5.515596in}{4.067667in}}{\pgfqpoint{5.504545in}{4.067667in}}%
\pgfpathcurveto{\pgfqpoint{5.493495in}{4.067667in}}{\pgfqpoint{5.482896in}{4.063276in}}{\pgfqpoint{5.475083in}{4.055463in}}%
\pgfpathcurveto{\pgfqpoint{5.467269in}{4.047649in}}{\pgfqpoint{5.462879in}{4.037050in}}{\pgfqpoint{5.462879in}{4.026000in}}%
\pgfpathcurveto{\pgfqpoint{5.462879in}{4.014950in}}{\pgfqpoint{5.467269in}{4.004351in}}{\pgfqpoint{5.475083in}{3.996537in}}%
\pgfpathcurveto{\pgfqpoint{5.482896in}{3.988724in}}{\pgfqpoint{5.493495in}{3.984333in}}{\pgfqpoint{5.504545in}{3.984333in}}%
\pgfpathclose%
\pgfusepath{stroke,fill}%
\end{pgfscope}%
\begin{pgfscope}%
\pgfpathrectangle{\pgfqpoint{0.800000in}{0.528000in}}{\pgfqpoint{4.960000in}{3.696000in}}%
\pgfusepath{clip}%
\pgfsetbuttcap%
\pgfsetroundjoin%
\definecolor{currentfill}{rgb}{0.000000,0.000000,0.000000}%
\pgfsetfillcolor{currentfill}%
\pgfsetlinewidth{1.003750pt}%
\definecolor{currentstroke}{rgb}{0.000000,0.000000,0.000000}%
\pgfsetstrokecolor{currentstroke}%
\pgfsetdash{}{0pt}%
\pgfpathmoveto{\pgfqpoint{5.504545in}{0.684333in}}%
\pgfpathcurveto{\pgfqpoint{5.515596in}{0.684333in}}{\pgfqpoint{5.526195in}{0.688724in}}{\pgfqpoint{5.534008in}{0.696537in}}%
\pgfpathcurveto{\pgfqpoint{5.541822in}{0.704351in}}{\pgfqpoint{5.546212in}{0.714950in}}{\pgfqpoint{5.546212in}{0.726000in}}%
\pgfpathcurveto{\pgfqpoint{5.546212in}{0.737050in}}{\pgfqpoint{5.541822in}{0.747649in}}{\pgfqpoint{5.534008in}{0.755463in}}%
\pgfpathcurveto{\pgfqpoint{5.526195in}{0.763276in}}{\pgfqpoint{5.515596in}{0.767667in}}{\pgfqpoint{5.504545in}{0.767667in}}%
\pgfpathcurveto{\pgfqpoint{5.493495in}{0.767667in}}{\pgfqpoint{5.482896in}{0.763276in}}{\pgfqpoint{5.475083in}{0.755463in}}%
\pgfpathcurveto{\pgfqpoint{5.467269in}{0.747649in}}{\pgfqpoint{5.462879in}{0.737050in}}{\pgfqpoint{5.462879in}{0.726000in}}%
\pgfpathcurveto{\pgfqpoint{5.462879in}{0.714950in}}{\pgfqpoint{5.467269in}{0.704351in}}{\pgfqpoint{5.475083in}{0.696537in}}%
\pgfpathcurveto{\pgfqpoint{5.482896in}{0.688724in}}{\pgfqpoint{5.493495in}{0.684333in}}{\pgfqpoint{5.504545in}{0.684333in}}%
\pgfpathclose%
\pgfusepath{stroke,fill}%
\end{pgfscope}%
\begin{pgfscope}%
\pgfpathrectangle{\pgfqpoint{0.800000in}{0.528000in}}{\pgfqpoint{4.960000in}{3.696000in}}%
\pgfusepath{clip}%
\pgfsetbuttcap%
\pgfsetroundjoin%
\definecolor{currentfill}{rgb}{0.000000,0.000000,0.000000}%
\pgfsetfillcolor{currentfill}%
\pgfsetlinewidth{1.003750pt}%
\definecolor{currentstroke}{rgb}{0.000000,0.000000,0.000000}%
\pgfsetstrokecolor{currentstroke}%
\pgfsetdash{}{0pt}%
\pgfpathmoveto{\pgfqpoint{5.504545in}{0.684333in}}%
\pgfpathcurveto{\pgfqpoint{5.515596in}{0.684333in}}{\pgfqpoint{5.526195in}{0.688724in}}{\pgfqpoint{5.534008in}{0.696537in}}%
\pgfpathcurveto{\pgfqpoint{5.541822in}{0.704351in}}{\pgfqpoint{5.546212in}{0.714950in}}{\pgfqpoint{5.546212in}{0.726000in}}%
\pgfpathcurveto{\pgfqpoint{5.546212in}{0.737050in}}{\pgfqpoint{5.541822in}{0.747649in}}{\pgfqpoint{5.534008in}{0.755463in}}%
\pgfpathcurveto{\pgfqpoint{5.526195in}{0.763276in}}{\pgfqpoint{5.515596in}{0.767667in}}{\pgfqpoint{5.504545in}{0.767667in}}%
\pgfpathcurveto{\pgfqpoint{5.493495in}{0.767667in}}{\pgfqpoint{5.482896in}{0.763276in}}{\pgfqpoint{5.475083in}{0.755463in}}%
\pgfpathcurveto{\pgfqpoint{5.467269in}{0.747649in}}{\pgfqpoint{5.462879in}{0.737050in}}{\pgfqpoint{5.462879in}{0.726000in}}%
\pgfpathcurveto{\pgfqpoint{5.462879in}{0.714950in}}{\pgfqpoint{5.467269in}{0.704351in}}{\pgfqpoint{5.475083in}{0.696537in}}%
\pgfpathcurveto{\pgfqpoint{5.482896in}{0.688724in}}{\pgfqpoint{5.493495in}{0.684333in}}{\pgfqpoint{5.504545in}{0.684333in}}%
\pgfpathclose%
\pgfusepath{stroke,fill}%
\end{pgfscope}%
\begin{pgfscope}%
\pgfpathrectangle{\pgfqpoint{0.800000in}{0.528000in}}{\pgfqpoint{4.960000in}{3.696000in}}%
\pgfusepath{clip}%
\pgfsetbuttcap%
\pgfsetroundjoin%
\definecolor{currentfill}{rgb}{0.000000,0.000000,0.000000}%
\pgfsetfillcolor{currentfill}%
\pgfsetlinewidth{1.003750pt}%
\definecolor{currentstroke}{rgb}{0.000000,0.000000,0.000000}%
\pgfsetstrokecolor{currentstroke}%
\pgfsetdash{}{0pt}%
\pgfpathmoveto{\pgfqpoint{5.504545in}{0.684333in}}%
\pgfpathcurveto{\pgfqpoint{5.515596in}{0.684333in}}{\pgfqpoint{5.526195in}{0.688724in}}{\pgfqpoint{5.534008in}{0.696537in}}%
\pgfpathcurveto{\pgfqpoint{5.541822in}{0.704351in}}{\pgfqpoint{5.546212in}{0.714950in}}{\pgfqpoint{5.546212in}{0.726000in}}%
\pgfpathcurveto{\pgfqpoint{5.546212in}{0.737050in}}{\pgfqpoint{5.541822in}{0.747649in}}{\pgfqpoint{5.534008in}{0.755463in}}%
\pgfpathcurveto{\pgfqpoint{5.526195in}{0.763276in}}{\pgfqpoint{5.515596in}{0.767667in}}{\pgfqpoint{5.504545in}{0.767667in}}%
\pgfpathcurveto{\pgfqpoint{5.493495in}{0.767667in}}{\pgfqpoint{5.482896in}{0.763276in}}{\pgfqpoint{5.475083in}{0.755463in}}%
\pgfpathcurveto{\pgfqpoint{5.467269in}{0.747649in}}{\pgfqpoint{5.462879in}{0.737050in}}{\pgfqpoint{5.462879in}{0.726000in}}%
\pgfpathcurveto{\pgfqpoint{5.462879in}{0.714950in}}{\pgfqpoint{5.467269in}{0.704351in}}{\pgfqpoint{5.475083in}{0.696537in}}%
\pgfpathcurveto{\pgfqpoint{5.482896in}{0.688724in}}{\pgfqpoint{5.493495in}{0.684333in}}{\pgfqpoint{5.504545in}{0.684333in}}%
\pgfpathclose%
\pgfusepath{stroke,fill}%
\end{pgfscope}%
\begin{pgfscope}%
\pgfpathrectangle{\pgfqpoint{0.800000in}{0.528000in}}{\pgfqpoint{4.960000in}{3.696000in}}%
\pgfusepath{clip}%
\pgfsetbuttcap%
\pgfsetroundjoin%
\definecolor{currentfill}{rgb}{0.000000,0.000000,0.000000}%
\pgfsetfillcolor{currentfill}%
\pgfsetlinewidth{1.003750pt}%
\definecolor{currentstroke}{rgb}{0.000000,0.000000,0.000000}%
\pgfsetstrokecolor{currentstroke}%
\pgfsetdash{}{0pt}%
\pgfpathmoveto{\pgfqpoint{5.504545in}{3.984333in}}%
\pgfpathcurveto{\pgfqpoint{5.515596in}{3.984333in}}{\pgfqpoint{5.526195in}{3.988724in}}{\pgfqpoint{5.534008in}{3.996537in}}%
\pgfpathcurveto{\pgfqpoint{5.541822in}{4.004351in}}{\pgfqpoint{5.546212in}{4.014950in}}{\pgfqpoint{5.546212in}{4.026000in}}%
\pgfpathcurveto{\pgfqpoint{5.546212in}{4.037050in}}{\pgfqpoint{5.541822in}{4.047649in}}{\pgfqpoint{5.534008in}{4.055463in}}%
\pgfpathcurveto{\pgfqpoint{5.526195in}{4.063276in}}{\pgfqpoint{5.515596in}{4.067667in}}{\pgfqpoint{5.504545in}{4.067667in}}%
\pgfpathcurveto{\pgfqpoint{5.493495in}{4.067667in}}{\pgfqpoint{5.482896in}{4.063276in}}{\pgfqpoint{5.475083in}{4.055463in}}%
\pgfpathcurveto{\pgfqpoint{5.467269in}{4.047649in}}{\pgfqpoint{5.462879in}{4.037050in}}{\pgfqpoint{5.462879in}{4.026000in}}%
\pgfpathcurveto{\pgfqpoint{5.462879in}{4.014950in}}{\pgfqpoint{5.467269in}{4.004351in}}{\pgfqpoint{5.475083in}{3.996537in}}%
\pgfpathcurveto{\pgfqpoint{5.482896in}{3.988724in}}{\pgfqpoint{5.493495in}{3.984333in}}{\pgfqpoint{5.504545in}{3.984333in}}%
\pgfpathclose%
\pgfusepath{stroke,fill}%
\end{pgfscope}%
\begin{pgfscope}%
\pgfpathrectangle{\pgfqpoint{0.800000in}{0.528000in}}{\pgfqpoint{4.960000in}{3.696000in}}%
\pgfusepath{clip}%
\pgfsetbuttcap%
\pgfsetroundjoin%
\definecolor{currentfill}{rgb}{0.000000,0.000000,0.000000}%
\pgfsetfillcolor{currentfill}%
\pgfsetlinewidth{1.003750pt}%
\definecolor{currentstroke}{rgb}{0.000000,0.000000,0.000000}%
\pgfsetstrokecolor{currentstroke}%
\pgfsetdash{}{0pt}%
\pgfpathmoveto{\pgfqpoint{5.504545in}{3.984333in}}%
\pgfpathcurveto{\pgfqpoint{5.515596in}{3.984333in}}{\pgfqpoint{5.526195in}{3.988724in}}{\pgfqpoint{5.534008in}{3.996537in}}%
\pgfpathcurveto{\pgfqpoint{5.541822in}{4.004351in}}{\pgfqpoint{5.546212in}{4.014950in}}{\pgfqpoint{5.546212in}{4.026000in}}%
\pgfpathcurveto{\pgfqpoint{5.546212in}{4.037050in}}{\pgfqpoint{5.541822in}{4.047649in}}{\pgfqpoint{5.534008in}{4.055463in}}%
\pgfpathcurveto{\pgfqpoint{5.526195in}{4.063276in}}{\pgfqpoint{5.515596in}{4.067667in}}{\pgfqpoint{5.504545in}{4.067667in}}%
\pgfpathcurveto{\pgfqpoint{5.493495in}{4.067667in}}{\pgfqpoint{5.482896in}{4.063276in}}{\pgfqpoint{5.475083in}{4.055463in}}%
\pgfpathcurveto{\pgfqpoint{5.467269in}{4.047649in}}{\pgfqpoint{5.462879in}{4.037050in}}{\pgfqpoint{5.462879in}{4.026000in}}%
\pgfpathcurveto{\pgfqpoint{5.462879in}{4.014950in}}{\pgfqpoint{5.467269in}{4.004351in}}{\pgfqpoint{5.475083in}{3.996537in}}%
\pgfpathcurveto{\pgfqpoint{5.482896in}{3.988724in}}{\pgfqpoint{5.493495in}{3.984333in}}{\pgfqpoint{5.504545in}{3.984333in}}%
\pgfpathclose%
\pgfusepath{stroke,fill}%
\end{pgfscope}%
\begin{pgfscope}%
\pgfpathrectangle{\pgfqpoint{0.800000in}{0.528000in}}{\pgfqpoint{4.960000in}{3.696000in}}%
\pgfusepath{clip}%
\pgfsetbuttcap%
\pgfsetroundjoin%
\definecolor{currentfill}{rgb}{0.000000,0.000000,0.000000}%
\pgfsetfillcolor{currentfill}%
\pgfsetlinewidth{1.003750pt}%
\definecolor{currentstroke}{rgb}{0.000000,0.000000,0.000000}%
\pgfsetstrokecolor{currentstroke}%
\pgfsetdash{}{0pt}%
\pgfpathmoveto{\pgfqpoint{5.504545in}{0.684333in}}%
\pgfpathcurveto{\pgfqpoint{5.515596in}{0.684333in}}{\pgfqpoint{5.526195in}{0.688724in}}{\pgfqpoint{5.534008in}{0.696537in}}%
\pgfpathcurveto{\pgfqpoint{5.541822in}{0.704351in}}{\pgfqpoint{5.546212in}{0.714950in}}{\pgfqpoint{5.546212in}{0.726000in}}%
\pgfpathcurveto{\pgfqpoint{5.546212in}{0.737050in}}{\pgfqpoint{5.541822in}{0.747649in}}{\pgfqpoint{5.534008in}{0.755463in}}%
\pgfpathcurveto{\pgfqpoint{5.526195in}{0.763276in}}{\pgfqpoint{5.515596in}{0.767667in}}{\pgfqpoint{5.504545in}{0.767667in}}%
\pgfpathcurveto{\pgfqpoint{5.493495in}{0.767667in}}{\pgfqpoint{5.482896in}{0.763276in}}{\pgfqpoint{5.475083in}{0.755463in}}%
\pgfpathcurveto{\pgfqpoint{5.467269in}{0.747649in}}{\pgfqpoint{5.462879in}{0.737050in}}{\pgfqpoint{5.462879in}{0.726000in}}%
\pgfpathcurveto{\pgfqpoint{5.462879in}{0.714950in}}{\pgfqpoint{5.467269in}{0.704351in}}{\pgfqpoint{5.475083in}{0.696537in}}%
\pgfpathcurveto{\pgfqpoint{5.482896in}{0.688724in}}{\pgfqpoint{5.493495in}{0.684333in}}{\pgfqpoint{5.504545in}{0.684333in}}%
\pgfpathclose%
\pgfusepath{stroke,fill}%
\end{pgfscope}%
\begin{pgfscope}%
\pgfpathrectangle{\pgfqpoint{0.800000in}{0.528000in}}{\pgfqpoint{4.960000in}{3.696000in}}%
\pgfusepath{clip}%
\pgfsetbuttcap%
\pgfsetroundjoin%
\definecolor{currentfill}{rgb}{0.000000,0.000000,0.000000}%
\pgfsetfillcolor{currentfill}%
\pgfsetlinewidth{1.003750pt}%
\definecolor{currentstroke}{rgb}{0.000000,0.000000,0.000000}%
\pgfsetstrokecolor{currentstroke}%
\pgfsetdash{}{0pt}%
\pgfpathmoveto{\pgfqpoint{5.504545in}{0.684333in}}%
\pgfpathcurveto{\pgfqpoint{5.515596in}{0.684333in}}{\pgfqpoint{5.526195in}{0.688724in}}{\pgfqpoint{5.534008in}{0.696537in}}%
\pgfpathcurveto{\pgfqpoint{5.541822in}{0.704351in}}{\pgfqpoint{5.546212in}{0.714950in}}{\pgfqpoint{5.546212in}{0.726000in}}%
\pgfpathcurveto{\pgfqpoint{5.546212in}{0.737050in}}{\pgfqpoint{5.541822in}{0.747649in}}{\pgfqpoint{5.534008in}{0.755463in}}%
\pgfpathcurveto{\pgfqpoint{5.526195in}{0.763276in}}{\pgfqpoint{5.515596in}{0.767667in}}{\pgfqpoint{5.504545in}{0.767667in}}%
\pgfpathcurveto{\pgfqpoint{5.493495in}{0.767667in}}{\pgfqpoint{5.482896in}{0.763276in}}{\pgfqpoint{5.475083in}{0.755463in}}%
\pgfpathcurveto{\pgfqpoint{5.467269in}{0.747649in}}{\pgfqpoint{5.462879in}{0.737050in}}{\pgfqpoint{5.462879in}{0.726000in}}%
\pgfpathcurveto{\pgfqpoint{5.462879in}{0.714950in}}{\pgfqpoint{5.467269in}{0.704351in}}{\pgfqpoint{5.475083in}{0.696537in}}%
\pgfpathcurveto{\pgfqpoint{5.482896in}{0.688724in}}{\pgfqpoint{5.493495in}{0.684333in}}{\pgfqpoint{5.504545in}{0.684333in}}%
\pgfpathclose%
\pgfusepath{stroke,fill}%
\end{pgfscope}%
\begin{pgfscope}%
\pgfpathrectangle{\pgfqpoint{0.800000in}{0.528000in}}{\pgfqpoint{4.960000in}{3.696000in}}%
\pgfusepath{clip}%
\pgfsetbuttcap%
\pgfsetroundjoin%
\definecolor{currentfill}{rgb}{0.000000,0.000000,0.000000}%
\pgfsetfillcolor{currentfill}%
\pgfsetlinewidth{1.003750pt}%
\definecolor{currentstroke}{rgb}{0.000000,0.000000,0.000000}%
\pgfsetstrokecolor{currentstroke}%
\pgfsetdash{}{0pt}%
\pgfpathmoveto{\pgfqpoint{5.504545in}{0.684333in}}%
\pgfpathcurveto{\pgfqpoint{5.515596in}{0.684333in}}{\pgfqpoint{5.526195in}{0.688724in}}{\pgfqpoint{5.534008in}{0.696537in}}%
\pgfpathcurveto{\pgfqpoint{5.541822in}{0.704351in}}{\pgfqpoint{5.546212in}{0.714950in}}{\pgfqpoint{5.546212in}{0.726000in}}%
\pgfpathcurveto{\pgfqpoint{5.546212in}{0.737050in}}{\pgfqpoint{5.541822in}{0.747649in}}{\pgfqpoint{5.534008in}{0.755463in}}%
\pgfpathcurveto{\pgfqpoint{5.526195in}{0.763276in}}{\pgfqpoint{5.515596in}{0.767667in}}{\pgfqpoint{5.504545in}{0.767667in}}%
\pgfpathcurveto{\pgfqpoint{5.493495in}{0.767667in}}{\pgfqpoint{5.482896in}{0.763276in}}{\pgfqpoint{5.475083in}{0.755463in}}%
\pgfpathcurveto{\pgfqpoint{5.467269in}{0.747649in}}{\pgfqpoint{5.462879in}{0.737050in}}{\pgfqpoint{5.462879in}{0.726000in}}%
\pgfpathcurveto{\pgfqpoint{5.462879in}{0.714950in}}{\pgfqpoint{5.467269in}{0.704351in}}{\pgfqpoint{5.475083in}{0.696537in}}%
\pgfpathcurveto{\pgfqpoint{5.482896in}{0.688724in}}{\pgfqpoint{5.493495in}{0.684333in}}{\pgfqpoint{5.504545in}{0.684333in}}%
\pgfpathclose%
\pgfusepath{stroke,fill}%
\end{pgfscope}%
\begin{pgfscope}%
\pgfpathrectangle{\pgfqpoint{0.800000in}{0.528000in}}{\pgfqpoint{4.960000in}{3.696000in}}%
\pgfusepath{clip}%
\pgfsetbuttcap%
\pgfsetroundjoin%
\definecolor{currentfill}{rgb}{0.000000,0.000000,0.000000}%
\pgfsetfillcolor{currentfill}%
\pgfsetlinewidth{1.003750pt}%
\definecolor{currentstroke}{rgb}{0.000000,0.000000,0.000000}%
\pgfsetstrokecolor{currentstroke}%
\pgfsetdash{}{0pt}%
\pgfpathmoveto{\pgfqpoint{5.504545in}{0.684333in}}%
\pgfpathcurveto{\pgfqpoint{5.515596in}{0.684333in}}{\pgfqpoint{5.526195in}{0.688724in}}{\pgfqpoint{5.534008in}{0.696537in}}%
\pgfpathcurveto{\pgfqpoint{5.541822in}{0.704351in}}{\pgfqpoint{5.546212in}{0.714950in}}{\pgfqpoint{5.546212in}{0.726000in}}%
\pgfpathcurveto{\pgfqpoint{5.546212in}{0.737050in}}{\pgfqpoint{5.541822in}{0.747649in}}{\pgfqpoint{5.534008in}{0.755463in}}%
\pgfpathcurveto{\pgfqpoint{5.526195in}{0.763276in}}{\pgfqpoint{5.515596in}{0.767667in}}{\pgfqpoint{5.504545in}{0.767667in}}%
\pgfpathcurveto{\pgfqpoint{5.493495in}{0.767667in}}{\pgfqpoint{5.482896in}{0.763276in}}{\pgfqpoint{5.475083in}{0.755463in}}%
\pgfpathcurveto{\pgfqpoint{5.467269in}{0.747649in}}{\pgfqpoint{5.462879in}{0.737050in}}{\pgfqpoint{5.462879in}{0.726000in}}%
\pgfpathcurveto{\pgfqpoint{5.462879in}{0.714950in}}{\pgfqpoint{5.467269in}{0.704351in}}{\pgfqpoint{5.475083in}{0.696537in}}%
\pgfpathcurveto{\pgfqpoint{5.482896in}{0.688724in}}{\pgfqpoint{5.493495in}{0.684333in}}{\pgfqpoint{5.504545in}{0.684333in}}%
\pgfpathclose%
\pgfusepath{stroke,fill}%
\end{pgfscope}%
\begin{pgfscope}%
\pgfpathrectangle{\pgfqpoint{0.800000in}{0.528000in}}{\pgfqpoint{4.960000in}{3.696000in}}%
\pgfusepath{clip}%
\pgfsetbuttcap%
\pgfsetroundjoin%
\definecolor{currentfill}{rgb}{0.000000,0.000000,0.000000}%
\pgfsetfillcolor{currentfill}%
\pgfsetlinewidth{1.003750pt}%
\definecolor{currentstroke}{rgb}{0.000000,0.000000,0.000000}%
\pgfsetstrokecolor{currentstroke}%
\pgfsetdash{}{0pt}%
\pgfpathmoveto{\pgfqpoint{5.504545in}{3.984333in}}%
\pgfpathcurveto{\pgfqpoint{5.515596in}{3.984333in}}{\pgfqpoint{5.526195in}{3.988724in}}{\pgfqpoint{5.534008in}{3.996537in}}%
\pgfpathcurveto{\pgfqpoint{5.541822in}{4.004351in}}{\pgfqpoint{5.546212in}{4.014950in}}{\pgfqpoint{5.546212in}{4.026000in}}%
\pgfpathcurveto{\pgfqpoint{5.546212in}{4.037050in}}{\pgfqpoint{5.541822in}{4.047649in}}{\pgfqpoint{5.534008in}{4.055463in}}%
\pgfpathcurveto{\pgfqpoint{5.526195in}{4.063276in}}{\pgfqpoint{5.515596in}{4.067667in}}{\pgfqpoint{5.504545in}{4.067667in}}%
\pgfpathcurveto{\pgfqpoint{5.493495in}{4.067667in}}{\pgfqpoint{5.482896in}{4.063276in}}{\pgfqpoint{5.475083in}{4.055463in}}%
\pgfpathcurveto{\pgfqpoint{5.467269in}{4.047649in}}{\pgfqpoint{5.462879in}{4.037050in}}{\pgfqpoint{5.462879in}{4.026000in}}%
\pgfpathcurveto{\pgfqpoint{5.462879in}{4.014950in}}{\pgfqpoint{5.467269in}{4.004351in}}{\pgfqpoint{5.475083in}{3.996537in}}%
\pgfpathcurveto{\pgfqpoint{5.482896in}{3.988724in}}{\pgfqpoint{5.493495in}{3.984333in}}{\pgfqpoint{5.504545in}{3.984333in}}%
\pgfpathclose%
\pgfusepath{stroke,fill}%
\end{pgfscope}%
\begin{pgfscope}%
\pgfpathrectangle{\pgfqpoint{0.800000in}{0.528000in}}{\pgfqpoint{4.960000in}{3.696000in}}%
\pgfusepath{clip}%
\pgfsetbuttcap%
\pgfsetroundjoin%
\definecolor{currentfill}{rgb}{0.000000,0.000000,0.000000}%
\pgfsetfillcolor{currentfill}%
\pgfsetlinewidth{1.003750pt}%
\definecolor{currentstroke}{rgb}{0.000000,0.000000,0.000000}%
\pgfsetstrokecolor{currentstroke}%
\pgfsetdash{}{0pt}%
\pgfpathmoveto{\pgfqpoint{5.504545in}{3.984333in}}%
\pgfpathcurveto{\pgfqpoint{5.515596in}{3.984333in}}{\pgfqpoint{5.526195in}{3.988724in}}{\pgfqpoint{5.534008in}{3.996537in}}%
\pgfpathcurveto{\pgfqpoint{5.541822in}{4.004351in}}{\pgfqpoint{5.546212in}{4.014950in}}{\pgfqpoint{5.546212in}{4.026000in}}%
\pgfpathcurveto{\pgfqpoint{5.546212in}{4.037050in}}{\pgfqpoint{5.541822in}{4.047649in}}{\pgfqpoint{5.534008in}{4.055463in}}%
\pgfpathcurveto{\pgfqpoint{5.526195in}{4.063276in}}{\pgfqpoint{5.515596in}{4.067667in}}{\pgfqpoint{5.504545in}{4.067667in}}%
\pgfpathcurveto{\pgfqpoint{5.493495in}{4.067667in}}{\pgfqpoint{5.482896in}{4.063276in}}{\pgfqpoint{5.475083in}{4.055463in}}%
\pgfpathcurveto{\pgfqpoint{5.467269in}{4.047649in}}{\pgfqpoint{5.462879in}{4.037050in}}{\pgfqpoint{5.462879in}{4.026000in}}%
\pgfpathcurveto{\pgfqpoint{5.462879in}{4.014950in}}{\pgfqpoint{5.467269in}{4.004351in}}{\pgfqpoint{5.475083in}{3.996537in}}%
\pgfpathcurveto{\pgfqpoint{5.482896in}{3.988724in}}{\pgfqpoint{5.493495in}{3.984333in}}{\pgfqpoint{5.504545in}{3.984333in}}%
\pgfpathclose%
\pgfusepath{stroke,fill}%
\end{pgfscope}%
\begin{pgfscope}%
\pgfpathrectangle{\pgfqpoint{0.800000in}{0.528000in}}{\pgfqpoint{4.960000in}{3.696000in}}%
\pgfusepath{clip}%
\pgfsetbuttcap%
\pgfsetroundjoin%
\definecolor{currentfill}{rgb}{0.000000,0.000000,0.000000}%
\pgfsetfillcolor{currentfill}%
\pgfsetlinewidth{1.003750pt}%
\definecolor{currentstroke}{rgb}{0.000000,0.000000,0.000000}%
\pgfsetstrokecolor{currentstroke}%
\pgfsetdash{}{0pt}%
\pgfpathmoveto{\pgfqpoint{5.504545in}{0.684333in}}%
\pgfpathcurveto{\pgfqpoint{5.515596in}{0.684333in}}{\pgfqpoint{5.526195in}{0.688724in}}{\pgfqpoint{5.534008in}{0.696537in}}%
\pgfpathcurveto{\pgfqpoint{5.541822in}{0.704351in}}{\pgfqpoint{5.546212in}{0.714950in}}{\pgfqpoint{5.546212in}{0.726000in}}%
\pgfpathcurveto{\pgfqpoint{5.546212in}{0.737050in}}{\pgfqpoint{5.541822in}{0.747649in}}{\pgfqpoint{5.534008in}{0.755463in}}%
\pgfpathcurveto{\pgfqpoint{5.526195in}{0.763276in}}{\pgfqpoint{5.515596in}{0.767667in}}{\pgfqpoint{5.504545in}{0.767667in}}%
\pgfpathcurveto{\pgfqpoint{5.493495in}{0.767667in}}{\pgfqpoint{5.482896in}{0.763276in}}{\pgfqpoint{5.475083in}{0.755463in}}%
\pgfpathcurveto{\pgfqpoint{5.467269in}{0.747649in}}{\pgfqpoint{5.462879in}{0.737050in}}{\pgfqpoint{5.462879in}{0.726000in}}%
\pgfpathcurveto{\pgfqpoint{5.462879in}{0.714950in}}{\pgfqpoint{5.467269in}{0.704351in}}{\pgfqpoint{5.475083in}{0.696537in}}%
\pgfpathcurveto{\pgfqpoint{5.482896in}{0.688724in}}{\pgfqpoint{5.493495in}{0.684333in}}{\pgfqpoint{5.504545in}{0.684333in}}%
\pgfpathclose%
\pgfusepath{stroke,fill}%
\end{pgfscope}%
\begin{pgfscope}%
\pgfpathrectangle{\pgfqpoint{0.800000in}{0.528000in}}{\pgfqpoint{4.960000in}{3.696000in}}%
\pgfusepath{clip}%
\pgfsetbuttcap%
\pgfsetroundjoin%
\definecolor{currentfill}{rgb}{0.000000,0.000000,0.000000}%
\pgfsetfillcolor{currentfill}%
\pgfsetlinewidth{1.003750pt}%
\definecolor{currentstroke}{rgb}{0.000000,0.000000,0.000000}%
\pgfsetstrokecolor{currentstroke}%
\pgfsetdash{}{0pt}%
\pgfpathmoveto{\pgfqpoint{5.504545in}{0.684333in}}%
\pgfpathcurveto{\pgfqpoint{5.515596in}{0.684333in}}{\pgfqpoint{5.526195in}{0.688724in}}{\pgfqpoint{5.534008in}{0.696537in}}%
\pgfpathcurveto{\pgfqpoint{5.541822in}{0.704351in}}{\pgfqpoint{5.546212in}{0.714950in}}{\pgfqpoint{5.546212in}{0.726000in}}%
\pgfpathcurveto{\pgfqpoint{5.546212in}{0.737050in}}{\pgfqpoint{5.541822in}{0.747649in}}{\pgfqpoint{5.534008in}{0.755463in}}%
\pgfpathcurveto{\pgfqpoint{5.526195in}{0.763276in}}{\pgfqpoint{5.515596in}{0.767667in}}{\pgfqpoint{5.504545in}{0.767667in}}%
\pgfpathcurveto{\pgfqpoint{5.493495in}{0.767667in}}{\pgfqpoint{5.482896in}{0.763276in}}{\pgfqpoint{5.475083in}{0.755463in}}%
\pgfpathcurveto{\pgfqpoint{5.467269in}{0.747649in}}{\pgfqpoint{5.462879in}{0.737050in}}{\pgfqpoint{5.462879in}{0.726000in}}%
\pgfpathcurveto{\pgfqpoint{5.462879in}{0.714950in}}{\pgfqpoint{5.467269in}{0.704351in}}{\pgfqpoint{5.475083in}{0.696537in}}%
\pgfpathcurveto{\pgfqpoint{5.482896in}{0.688724in}}{\pgfqpoint{5.493495in}{0.684333in}}{\pgfqpoint{5.504545in}{0.684333in}}%
\pgfpathclose%
\pgfusepath{stroke,fill}%
\end{pgfscope}%
\begin{pgfscope}%
\pgfpathrectangle{\pgfqpoint{0.800000in}{0.528000in}}{\pgfqpoint{4.960000in}{3.696000in}}%
\pgfusepath{clip}%
\pgfsetbuttcap%
\pgfsetroundjoin%
\definecolor{currentfill}{rgb}{0.000000,0.000000,0.000000}%
\pgfsetfillcolor{currentfill}%
\pgfsetlinewidth{1.003750pt}%
\definecolor{currentstroke}{rgb}{0.000000,0.000000,0.000000}%
\pgfsetstrokecolor{currentstroke}%
\pgfsetdash{}{0pt}%
\pgfpathmoveto{\pgfqpoint{5.504545in}{0.684333in}}%
\pgfpathcurveto{\pgfqpoint{5.515596in}{0.684333in}}{\pgfqpoint{5.526195in}{0.688724in}}{\pgfqpoint{5.534008in}{0.696537in}}%
\pgfpathcurveto{\pgfqpoint{5.541822in}{0.704351in}}{\pgfqpoint{5.546212in}{0.714950in}}{\pgfqpoint{5.546212in}{0.726000in}}%
\pgfpathcurveto{\pgfqpoint{5.546212in}{0.737050in}}{\pgfqpoint{5.541822in}{0.747649in}}{\pgfqpoint{5.534008in}{0.755463in}}%
\pgfpathcurveto{\pgfqpoint{5.526195in}{0.763276in}}{\pgfqpoint{5.515596in}{0.767667in}}{\pgfqpoint{5.504545in}{0.767667in}}%
\pgfpathcurveto{\pgfqpoint{5.493495in}{0.767667in}}{\pgfqpoint{5.482896in}{0.763276in}}{\pgfqpoint{5.475083in}{0.755463in}}%
\pgfpathcurveto{\pgfqpoint{5.467269in}{0.747649in}}{\pgfqpoint{5.462879in}{0.737050in}}{\pgfqpoint{5.462879in}{0.726000in}}%
\pgfpathcurveto{\pgfqpoint{5.462879in}{0.714950in}}{\pgfqpoint{5.467269in}{0.704351in}}{\pgfqpoint{5.475083in}{0.696537in}}%
\pgfpathcurveto{\pgfqpoint{5.482896in}{0.688724in}}{\pgfqpoint{5.493495in}{0.684333in}}{\pgfqpoint{5.504545in}{0.684333in}}%
\pgfpathclose%
\pgfusepath{stroke,fill}%
\end{pgfscope}%
\begin{pgfscope}%
\pgfpathrectangle{\pgfqpoint{0.800000in}{0.528000in}}{\pgfqpoint{4.960000in}{3.696000in}}%
\pgfusepath{clip}%
\pgfsetbuttcap%
\pgfsetroundjoin%
\definecolor{currentfill}{rgb}{0.000000,0.000000,0.000000}%
\pgfsetfillcolor{currentfill}%
\pgfsetlinewidth{1.003750pt}%
\definecolor{currentstroke}{rgb}{0.000000,0.000000,0.000000}%
\pgfsetstrokecolor{currentstroke}%
\pgfsetdash{}{0pt}%
\pgfpathmoveto{\pgfqpoint{5.504545in}{3.984333in}}%
\pgfpathcurveto{\pgfqpoint{5.515596in}{3.984333in}}{\pgfqpoint{5.526195in}{3.988724in}}{\pgfqpoint{5.534008in}{3.996537in}}%
\pgfpathcurveto{\pgfqpoint{5.541822in}{4.004351in}}{\pgfqpoint{5.546212in}{4.014950in}}{\pgfqpoint{5.546212in}{4.026000in}}%
\pgfpathcurveto{\pgfqpoint{5.546212in}{4.037050in}}{\pgfqpoint{5.541822in}{4.047649in}}{\pgfqpoint{5.534008in}{4.055463in}}%
\pgfpathcurveto{\pgfqpoint{5.526195in}{4.063276in}}{\pgfqpoint{5.515596in}{4.067667in}}{\pgfqpoint{5.504545in}{4.067667in}}%
\pgfpathcurveto{\pgfqpoint{5.493495in}{4.067667in}}{\pgfqpoint{5.482896in}{4.063276in}}{\pgfqpoint{5.475083in}{4.055463in}}%
\pgfpathcurveto{\pgfqpoint{5.467269in}{4.047649in}}{\pgfqpoint{5.462879in}{4.037050in}}{\pgfqpoint{5.462879in}{4.026000in}}%
\pgfpathcurveto{\pgfqpoint{5.462879in}{4.014950in}}{\pgfqpoint{5.467269in}{4.004351in}}{\pgfqpoint{5.475083in}{3.996537in}}%
\pgfpathcurveto{\pgfqpoint{5.482896in}{3.988724in}}{\pgfqpoint{5.493495in}{3.984333in}}{\pgfqpoint{5.504545in}{3.984333in}}%
\pgfpathclose%
\pgfusepath{stroke,fill}%
\end{pgfscope}%
\begin{pgfscope}%
\pgfpathrectangle{\pgfqpoint{0.800000in}{0.528000in}}{\pgfqpoint{4.960000in}{3.696000in}}%
\pgfusepath{clip}%
\pgfsetbuttcap%
\pgfsetroundjoin%
\definecolor{currentfill}{rgb}{0.000000,0.000000,0.000000}%
\pgfsetfillcolor{currentfill}%
\pgfsetlinewidth{1.003750pt}%
\definecolor{currentstroke}{rgb}{0.000000,0.000000,0.000000}%
\pgfsetstrokecolor{currentstroke}%
\pgfsetdash{}{0pt}%
\pgfpathmoveto{\pgfqpoint{5.504545in}{0.684333in}}%
\pgfpathcurveto{\pgfqpoint{5.515596in}{0.684333in}}{\pgfqpoint{5.526195in}{0.688724in}}{\pgfqpoint{5.534008in}{0.696537in}}%
\pgfpathcurveto{\pgfqpoint{5.541822in}{0.704351in}}{\pgfqpoint{5.546212in}{0.714950in}}{\pgfqpoint{5.546212in}{0.726000in}}%
\pgfpathcurveto{\pgfqpoint{5.546212in}{0.737050in}}{\pgfqpoint{5.541822in}{0.747649in}}{\pgfqpoint{5.534008in}{0.755463in}}%
\pgfpathcurveto{\pgfqpoint{5.526195in}{0.763276in}}{\pgfqpoint{5.515596in}{0.767667in}}{\pgfqpoint{5.504545in}{0.767667in}}%
\pgfpathcurveto{\pgfqpoint{5.493495in}{0.767667in}}{\pgfqpoint{5.482896in}{0.763276in}}{\pgfqpoint{5.475083in}{0.755463in}}%
\pgfpathcurveto{\pgfqpoint{5.467269in}{0.747649in}}{\pgfqpoint{5.462879in}{0.737050in}}{\pgfqpoint{5.462879in}{0.726000in}}%
\pgfpathcurveto{\pgfqpoint{5.462879in}{0.714950in}}{\pgfqpoint{5.467269in}{0.704351in}}{\pgfqpoint{5.475083in}{0.696537in}}%
\pgfpathcurveto{\pgfqpoint{5.482896in}{0.688724in}}{\pgfqpoint{5.493495in}{0.684333in}}{\pgfqpoint{5.504545in}{0.684333in}}%
\pgfpathclose%
\pgfusepath{stroke,fill}%
\end{pgfscope}%
\begin{pgfscope}%
\pgfpathrectangle{\pgfqpoint{0.800000in}{0.528000in}}{\pgfqpoint{4.960000in}{3.696000in}}%
\pgfusepath{clip}%
\pgfsetbuttcap%
\pgfsetroundjoin%
\definecolor{currentfill}{rgb}{0.000000,0.000000,0.000000}%
\pgfsetfillcolor{currentfill}%
\pgfsetlinewidth{1.003750pt}%
\definecolor{currentstroke}{rgb}{0.000000,0.000000,0.000000}%
\pgfsetstrokecolor{currentstroke}%
\pgfsetdash{}{0pt}%
\pgfpathmoveto{\pgfqpoint{5.504545in}{0.684333in}}%
\pgfpathcurveto{\pgfqpoint{5.515596in}{0.684333in}}{\pgfqpoint{5.526195in}{0.688724in}}{\pgfqpoint{5.534008in}{0.696537in}}%
\pgfpathcurveto{\pgfqpoint{5.541822in}{0.704351in}}{\pgfqpoint{5.546212in}{0.714950in}}{\pgfqpoint{5.546212in}{0.726000in}}%
\pgfpathcurveto{\pgfqpoint{5.546212in}{0.737050in}}{\pgfqpoint{5.541822in}{0.747649in}}{\pgfqpoint{5.534008in}{0.755463in}}%
\pgfpathcurveto{\pgfqpoint{5.526195in}{0.763276in}}{\pgfqpoint{5.515596in}{0.767667in}}{\pgfqpoint{5.504545in}{0.767667in}}%
\pgfpathcurveto{\pgfqpoint{5.493495in}{0.767667in}}{\pgfqpoint{5.482896in}{0.763276in}}{\pgfqpoint{5.475083in}{0.755463in}}%
\pgfpathcurveto{\pgfqpoint{5.467269in}{0.747649in}}{\pgfqpoint{5.462879in}{0.737050in}}{\pgfqpoint{5.462879in}{0.726000in}}%
\pgfpathcurveto{\pgfqpoint{5.462879in}{0.714950in}}{\pgfqpoint{5.467269in}{0.704351in}}{\pgfqpoint{5.475083in}{0.696537in}}%
\pgfpathcurveto{\pgfqpoint{5.482896in}{0.688724in}}{\pgfqpoint{5.493495in}{0.684333in}}{\pgfqpoint{5.504545in}{0.684333in}}%
\pgfpathclose%
\pgfusepath{stroke,fill}%
\end{pgfscope}%
\begin{pgfscope}%
\pgfpathrectangle{\pgfqpoint{0.800000in}{0.528000in}}{\pgfqpoint{4.960000in}{3.696000in}}%
\pgfusepath{clip}%
\pgfsetbuttcap%
\pgfsetroundjoin%
\definecolor{currentfill}{rgb}{0.000000,0.000000,0.000000}%
\pgfsetfillcolor{currentfill}%
\pgfsetlinewidth{1.003750pt}%
\definecolor{currentstroke}{rgb}{0.000000,0.000000,0.000000}%
\pgfsetstrokecolor{currentstroke}%
\pgfsetdash{}{0pt}%
\pgfpathmoveto{\pgfqpoint{5.504545in}{3.984333in}}%
\pgfpathcurveto{\pgfqpoint{5.515596in}{3.984333in}}{\pgfqpoint{5.526195in}{3.988724in}}{\pgfqpoint{5.534008in}{3.996537in}}%
\pgfpathcurveto{\pgfqpoint{5.541822in}{4.004351in}}{\pgfqpoint{5.546212in}{4.014950in}}{\pgfqpoint{5.546212in}{4.026000in}}%
\pgfpathcurveto{\pgfqpoint{5.546212in}{4.037050in}}{\pgfqpoint{5.541822in}{4.047649in}}{\pgfqpoint{5.534008in}{4.055463in}}%
\pgfpathcurveto{\pgfqpoint{5.526195in}{4.063276in}}{\pgfqpoint{5.515596in}{4.067667in}}{\pgfqpoint{5.504545in}{4.067667in}}%
\pgfpathcurveto{\pgfqpoint{5.493495in}{4.067667in}}{\pgfqpoint{5.482896in}{4.063276in}}{\pgfqpoint{5.475083in}{4.055463in}}%
\pgfpathcurveto{\pgfqpoint{5.467269in}{4.047649in}}{\pgfqpoint{5.462879in}{4.037050in}}{\pgfqpoint{5.462879in}{4.026000in}}%
\pgfpathcurveto{\pgfqpoint{5.462879in}{4.014950in}}{\pgfqpoint{5.467269in}{4.004351in}}{\pgfqpoint{5.475083in}{3.996537in}}%
\pgfpathcurveto{\pgfqpoint{5.482896in}{3.988724in}}{\pgfqpoint{5.493495in}{3.984333in}}{\pgfqpoint{5.504545in}{3.984333in}}%
\pgfpathclose%
\pgfusepath{stroke,fill}%
\end{pgfscope}%
\begin{pgfscope}%
\pgfpathrectangle{\pgfqpoint{0.800000in}{0.528000in}}{\pgfqpoint{4.960000in}{3.696000in}}%
\pgfusepath{clip}%
\pgfsetbuttcap%
\pgfsetroundjoin%
\definecolor{currentfill}{rgb}{0.000000,0.000000,0.000000}%
\pgfsetfillcolor{currentfill}%
\pgfsetlinewidth{1.003750pt}%
\definecolor{currentstroke}{rgb}{0.000000,0.000000,0.000000}%
\pgfsetstrokecolor{currentstroke}%
\pgfsetdash{}{0pt}%
\pgfpathmoveto{\pgfqpoint{5.504545in}{0.684333in}}%
\pgfpathcurveto{\pgfqpoint{5.515596in}{0.684333in}}{\pgfqpoint{5.526195in}{0.688724in}}{\pgfqpoint{5.534008in}{0.696537in}}%
\pgfpathcurveto{\pgfqpoint{5.541822in}{0.704351in}}{\pgfqpoint{5.546212in}{0.714950in}}{\pgfqpoint{5.546212in}{0.726000in}}%
\pgfpathcurveto{\pgfqpoint{5.546212in}{0.737050in}}{\pgfqpoint{5.541822in}{0.747649in}}{\pgfqpoint{5.534008in}{0.755463in}}%
\pgfpathcurveto{\pgfqpoint{5.526195in}{0.763276in}}{\pgfqpoint{5.515596in}{0.767667in}}{\pgfqpoint{5.504545in}{0.767667in}}%
\pgfpathcurveto{\pgfqpoint{5.493495in}{0.767667in}}{\pgfqpoint{5.482896in}{0.763276in}}{\pgfqpoint{5.475083in}{0.755463in}}%
\pgfpathcurveto{\pgfqpoint{5.467269in}{0.747649in}}{\pgfqpoint{5.462879in}{0.737050in}}{\pgfqpoint{5.462879in}{0.726000in}}%
\pgfpathcurveto{\pgfqpoint{5.462879in}{0.714950in}}{\pgfqpoint{5.467269in}{0.704351in}}{\pgfqpoint{5.475083in}{0.696537in}}%
\pgfpathcurveto{\pgfqpoint{5.482896in}{0.688724in}}{\pgfqpoint{5.493495in}{0.684333in}}{\pgfqpoint{5.504545in}{0.684333in}}%
\pgfpathclose%
\pgfusepath{stroke,fill}%
\end{pgfscope}%
\begin{pgfscope}%
\pgfpathrectangle{\pgfqpoint{0.800000in}{0.528000in}}{\pgfqpoint{4.960000in}{3.696000in}}%
\pgfusepath{clip}%
\pgfsetbuttcap%
\pgfsetroundjoin%
\definecolor{currentfill}{rgb}{0.000000,0.000000,0.000000}%
\pgfsetfillcolor{currentfill}%
\pgfsetlinewidth{1.003750pt}%
\definecolor{currentstroke}{rgb}{0.000000,0.000000,0.000000}%
\pgfsetstrokecolor{currentstroke}%
\pgfsetdash{}{0pt}%
\pgfpathmoveto{\pgfqpoint{5.504545in}{0.684333in}}%
\pgfpathcurveto{\pgfqpoint{5.515596in}{0.684333in}}{\pgfqpoint{5.526195in}{0.688724in}}{\pgfqpoint{5.534008in}{0.696537in}}%
\pgfpathcurveto{\pgfqpoint{5.541822in}{0.704351in}}{\pgfqpoint{5.546212in}{0.714950in}}{\pgfqpoint{5.546212in}{0.726000in}}%
\pgfpathcurveto{\pgfqpoint{5.546212in}{0.737050in}}{\pgfqpoint{5.541822in}{0.747649in}}{\pgfqpoint{5.534008in}{0.755463in}}%
\pgfpathcurveto{\pgfqpoint{5.526195in}{0.763276in}}{\pgfqpoint{5.515596in}{0.767667in}}{\pgfqpoint{5.504545in}{0.767667in}}%
\pgfpathcurveto{\pgfqpoint{5.493495in}{0.767667in}}{\pgfqpoint{5.482896in}{0.763276in}}{\pgfqpoint{5.475083in}{0.755463in}}%
\pgfpathcurveto{\pgfqpoint{5.467269in}{0.747649in}}{\pgfqpoint{5.462879in}{0.737050in}}{\pgfqpoint{5.462879in}{0.726000in}}%
\pgfpathcurveto{\pgfqpoint{5.462879in}{0.714950in}}{\pgfqpoint{5.467269in}{0.704351in}}{\pgfqpoint{5.475083in}{0.696537in}}%
\pgfpathcurveto{\pgfqpoint{5.482896in}{0.688724in}}{\pgfqpoint{5.493495in}{0.684333in}}{\pgfqpoint{5.504545in}{0.684333in}}%
\pgfpathclose%
\pgfusepath{stroke,fill}%
\end{pgfscope}%
\begin{pgfscope}%
\pgfpathrectangle{\pgfqpoint{0.800000in}{0.528000in}}{\pgfqpoint{4.960000in}{3.696000in}}%
\pgfusepath{clip}%
\pgfsetbuttcap%
\pgfsetroundjoin%
\definecolor{currentfill}{rgb}{0.000000,0.000000,0.000000}%
\pgfsetfillcolor{currentfill}%
\pgfsetlinewidth{1.003750pt}%
\definecolor{currentstroke}{rgb}{0.000000,0.000000,0.000000}%
\pgfsetstrokecolor{currentstroke}%
\pgfsetdash{}{0pt}%
\pgfpathmoveto{\pgfqpoint{5.504545in}{0.684333in}}%
\pgfpathcurveto{\pgfqpoint{5.515596in}{0.684333in}}{\pgfqpoint{5.526195in}{0.688724in}}{\pgfqpoint{5.534008in}{0.696537in}}%
\pgfpathcurveto{\pgfqpoint{5.541822in}{0.704351in}}{\pgfqpoint{5.546212in}{0.714950in}}{\pgfqpoint{5.546212in}{0.726000in}}%
\pgfpathcurveto{\pgfqpoint{5.546212in}{0.737050in}}{\pgfqpoint{5.541822in}{0.747649in}}{\pgfqpoint{5.534008in}{0.755463in}}%
\pgfpathcurveto{\pgfqpoint{5.526195in}{0.763276in}}{\pgfqpoint{5.515596in}{0.767667in}}{\pgfqpoint{5.504545in}{0.767667in}}%
\pgfpathcurveto{\pgfqpoint{5.493495in}{0.767667in}}{\pgfqpoint{5.482896in}{0.763276in}}{\pgfqpoint{5.475083in}{0.755463in}}%
\pgfpathcurveto{\pgfqpoint{5.467269in}{0.747649in}}{\pgfqpoint{5.462879in}{0.737050in}}{\pgfqpoint{5.462879in}{0.726000in}}%
\pgfpathcurveto{\pgfqpoint{5.462879in}{0.714950in}}{\pgfqpoint{5.467269in}{0.704351in}}{\pgfqpoint{5.475083in}{0.696537in}}%
\pgfpathcurveto{\pgfqpoint{5.482896in}{0.688724in}}{\pgfqpoint{5.493495in}{0.684333in}}{\pgfqpoint{5.504545in}{0.684333in}}%
\pgfpathclose%
\pgfusepath{stroke,fill}%
\end{pgfscope}%
\begin{pgfscope}%
\pgfpathrectangle{\pgfqpoint{0.800000in}{0.528000in}}{\pgfqpoint{4.960000in}{3.696000in}}%
\pgfusepath{clip}%
\pgfsetbuttcap%
\pgfsetroundjoin%
\definecolor{currentfill}{rgb}{0.000000,0.000000,0.000000}%
\pgfsetfillcolor{currentfill}%
\pgfsetlinewidth{1.003750pt}%
\definecolor{currentstroke}{rgb}{0.000000,0.000000,0.000000}%
\pgfsetstrokecolor{currentstroke}%
\pgfsetdash{}{0pt}%
\pgfpathmoveto{\pgfqpoint{5.504545in}{0.684333in}}%
\pgfpathcurveto{\pgfqpoint{5.515596in}{0.684333in}}{\pgfqpoint{5.526195in}{0.688724in}}{\pgfqpoint{5.534008in}{0.696537in}}%
\pgfpathcurveto{\pgfqpoint{5.541822in}{0.704351in}}{\pgfqpoint{5.546212in}{0.714950in}}{\pgfqpoint{5.546212in}{0.726000in}}%
\pgfpathcurveto{\pgfqpoint{5.546212in}{0.737050in}}{\pgfqpoint{5.541822in}{0.747649in}}{\pgfqpoint{5.534008in}{0.755463in}}%
\pgfpathcurveto{\pgfqpoint{5.526195in}{0.763276in}}{\pgfqpoint{5.515596in}{0.767667in}}{\pgfqpoint{5.504545in}{0.767667in}}%
\pgfpathcurveto{\pgfqpoint{5.493495in}{0.767667in}}{\pgfqpoint{5.482896in}{0.763276in}}{\pgfqpoint{5.475083in}{0.755463in}}%
\pgfpathcurveto{\pgfqpoint{5.467269in}{0.747649in}}{\pgfqpoint{5.462879in}{0.737050in}}{\pgfqpoint{5.462879in}{0.726000in}}%
\pgfpathcurveto{\pgfqpoint{5.462879in}{0.714950in}}{\pgfqpoint{5.467269in}{0.704351in}}{\pgfqpoint{5.475083in}{0.696537in}}%
\pgfpathcurveto{\pgfqpoint{5.482896in}{0.688724in}}{\pgfqpoint{5.493495in}{0.684333in}}{\pgfqpoint{5.504545in}{0.684333in}}%
\pgfpathclose%
\pgfusepath{stroke,fill}%
\end{pgfscope}%
\begin{pgfscope}%
\pgfpathrectangle{\pgfqpoint{0.800000in}{0.528000in}}{\pgfqpoint{4.960000in}{3.696000in}}%
\pgfusepath{clip}%
\pgfsetbuttcap%
\pgfsetroundjoin%
\definecolor{currentfill}{rgb}{0.000000,0.000000,0.000000}%
\pgfsetfillcolor{currentfill}%
\pgfsetlinewidth{1.003750pt}%
\definecolor{currentstroke}{rgb}{0.000000,0.000000,0.000000}%
\pgfsetstrokecolor{currentstroke}%
\pgfsetdash{}{0pt}%
\pgfpathmoveto{\pgfqpoint{5.504545in}{3.984333in}}%
\pgfpathcurveto{\pgfqpoint{5.515596in}{3.984333in}}{\pgfqpoint{5.526195in}{3.988724in}}{\pgfqpoint{5.534008in}{3.996537in}}%
\pgfpathcurveto{\pgfqpoint{5.541822in}{4.004351in}}{\pgfqpoint{5.546212in}{4.014950in}}{\pgfqpoint{5.546212in}{4.026000in}}%
\pgfpathcurveto{\pgfqpoint{5.546212in}{4.037050in}}{\pgfqpoint{5.541822in}{4.047649in}}{\pgfqpoint{5.534008in}{4.055463in}}%
\pgfpathcurveto{\pgfqpoint{5.526195in}{4.063276in}}{\pgfqpoint{5.515596in}{4.067667in}}{\pgfqpoint{5.504545in}{4.067667in}}%
\pgfpathcurveto{\pgfqpoint{5.493495in}{4.067667in}}{\pgfqpoint{5.482896in}{4.063276in}}{\pgfqpoint{5.475083in}{4.055463in}}%
\pgfpathcurveto{\pgfqpoint{5.467269in}{4.047649in}}{\pgfqpoint{5.462879in}{4.037050in}}{\pgfqpoint{5.462879in}{4.026000in}}%
\pgfpathcurveto{\pgfqpoint{5.462879in}{4.014950in}}{\pgfqpoint{5.467269in}{4.004351in}}{\pgfqpoint{5.475083in}{3.996537in}}%
\pgfpathcurveto{\pgfqpoint{5.482896in}{3.988724in}}{\pgfqpoint{5.493495in}{3.984333in}}{\pgfqpoint{5.504545in}{3.984333in}}%
\pgfpathclose%
\pgfusepath{stroke,fill}%
\end{pgfscope}%
\begin{pgfscope}%
\pgfpathrectangle{\pgfqpoint{0.800000in}{0.528000in}}{\pgfqpoint{4.960000in}{3.696000in}}%
\pgfusepath{clip}%
\pgfsetbuttcap%
\pgfsetroundjoin%
\definecolor{currentfill}{rgb}{0.000000,0.000000,0.000000}%
\pgfsetfillcolor{currentfill}%
\pgfsetlinewidth{1.003750pt}%
\definecolor{currentstroke}{rgb}{0.000000,0.000000,0.000000}%
\pgfsetstrokecolor{currentstroke}%
\pgfsetdash{}{0pt}%
\pgfpathmoveto{\pgfqpoint{5.504545in}{0.684333in}}%
\pgfpathcurveto{\pgfqpoint{5.515596in}{0.684333in}}{\pgfqpoint{5.526195in}{0.688724in}}{\pgfqpoint{5.534008in}{0.696537in}}%
\pgfpathcurveto{\pgfqpoint{5.541822in}{0.704351in}}{\pgfqpoint{5.546212in}{0.714950in}}{\pgfqpoint{5.546212in}{0.726000in}}%
\pgfpathcurveto{\pgfqpoint{5.546212in}{0.737050in}}{\pgfqpoint{5.541822in}{0.747649in}}{\pgfqpoint{5.534008in}{0.755463in}}%
\pgfpathcurveto{\pgfqpoint{5.526195in}{0.763276in}}{\pgfqpoint{5.515596in}{0.767667in}}{\pgfqpoint{5.504545in}{0.767667in}}%
\pgfpathcurveto{\pgfqpoint{5.493495in}{0.767667in}}{\pgfqpoint{5.482896in}{0.763276in}}{\pgfqpoint{5.475083in}{0.755463in}}%
\pgfpathcurveto{\pgfqpoint{5.467269in}{0.747649in}}{\pgfqpoint{5.462879in}{0.737050in}}{\pgfqpoint{5.462879in}{0.726000in}}%
\pgfpathcurveto{\pgfqpoint{5.462879in}{0.714950in}}{\pgfqpoint{5.467269in}{0.704351in}}{\pgfqpoint{5.475083in}{0.696537in}}%
\pgfpathcurveto{\pgfqpoint{5.482896in}{0.688724in}}{\pgfqpoint{5.493495in}{0.684333in}}{\pgfqpoint{5.504545in}{0.684333in}}%
\pgfpathclose%
\pgfusepath{stroke,fill}%
\end{pgfscope}%
\begin{pgfscope}%
\pgfpathrectangle{\pgfqpoint{0.800000in}{0.528000in}}{\pgfqpoint{4.960000in}{3.696000in}}%
\pgfusepath{clip}%
\pgfsetbuttcap%
\pgfsetroundjoin%
\definecolor{currentfill}{rgb}{0.000000,0.000000,0.000000}%
\pgfsetfillcolor{currentfill}%
\pgfsetlinewidth{1.003750pt}%
\definecolor{currentstroke}{rgb}{0.000000,0.000000,0.000000}%
\pgfsetstrokecolor{currentstroke}%
\pgfsetdash{}{0pt}%
\pgfpathmoveto{\pgfqpoint{5.504545in}{3.984333in}}%
\pgfpathcurveto{\pgfqpoint{5.515596in}{3.984333in}}{\pgfqpoint{5.526195in}{3.988724in}}{\pgfqpoint{5.534008in}{3.996537in}}%
\pgfpathcurveto{\pgfqpoint{5.541822in}{4.004351in}}{\pgfqpoint{5.546212in}{4.014950in}}{\pgfqpoint{5.546212in}{4.026000in}}%
\pgfpathcurveto{\pgfqpoint{5.546212in}{4.037050in}}{\pgfqpoint{5.541822in}{4.047649in}}{\pgfqpoint{5.534008in}{4.055463in}}%
\pgfpathcurveto{\pgfqpoint{5.526195in}{4.063276in}}{\pgfqpoint{5.515596in}{4.067667in}}{\pgfqpoint{5.504545in}{4.067667in}}%
\pgfpathcurveto{\pgfqpoint{5.493495in}{4.067667in}}{\pgfqpoint{5.482896in}{4.063276in}}{\pgfqpoint{5.475083in}{4.055463in}}%
\pgfpathcurveto{\pgfqpoint{5.467269in}{4.047649in}}{\pgfqpoint{5.462879in}{4.037050in}}{\pgfqpoint{5.462879in}{4.026000in}}%
\pgfpathcurveto{\pgfqpoint{5.462879in}{4.014950in}}{\pgfqpoint{5.467269in}{4.004351in}}{\pgfqpoint{5.475083in}{3.996537in}}%
\pgfpathcurveto{\pgfqpoint{5.482896in}{3.988724in}}{\pgfqpoint{5.493495in}{3.984333in}}{\pgfqpoint{5.504545in}{3.984333in}}%
\pgfpathclose%
\pgfusepath{stroke,fill}%
\end{pgfscope}%
\begin{pgfscope}%
\pgfpathrectangle{\pgfqpoint{0.800000in}{0.528000in}}{\pgfqpoint{4.960000in}{3.696000in}}%
\pgfusepath{clip}%
\pgfsetbuttcap%
\pgfsetroundjoin%
\definecolor{currentfill}{rgb}{0.000000,0.000000,0.000000}%
\pgfsetfillcolor{currentfill}%
\pgfsetlinewidth{1.003750pt}%
\definecolor{currentstroke}{rgb}{0.000000,0.000000,0.000000}%
\pgfsetstrokecolor{currentstroke}%
\pgfsetdash{}{0pt}%
\pgfpathmoveto{\pgfqpoint{5.504545in}{0.684333in}}%
\pgfpathcurveto{\pgfqpoint{5.515596in}{0.684333in}}{\pgfqpoint{5.526195in}{0.688724in}}{\pgfqpoint{5.534008in}{0.696537in}}%
\pgfpathcurveto{\pgfqpoint{5.541822in}{0.704351in}}{\pgfqpoint{5.546212in}{0.714950in}}{\pgfqpoint{5.546212in}{0.726000in}}%
\pgfpathcurveto{\pgfqpoint{5.546212in}{0.737050in}}{\pgfqpoint{5.541822in}{0.747649in}}{\pgfqpoint{5.534008in}{0.755463in}}%
\pgfpathcurveto{\pgfqpoint{5.526195in}{0.763276in}}{\pgfqpoint{5.515596in}{0.767667in}}{\pgfqpoint{5.504545in}{0.767667in}}%
\pgfpathcurveto{\pgfqpoint{5.493495in}{0.767667in}}{\pgfqpoint{5.482896in}{0.763276in}}{\pgfqpoint{5.475083in}{0.755463in}}%
\pgfpathcurveto{\pgfqpoint{5.467269in}{0.747649in}}{\pgfqpoint{5.462879in}{0.737050in}}{\pgfqpoint{5.462879in}{0.726000in}}%
\pgfpathcurveto{\pgfqpoint{5.462879in}{0.714950in}}{\pgfqpoint{5.467269in}{0.704351in}}{\pgfqpoint{5.475083in}{0.696537in}}%
\pgfpathcurveto{\pgfqpoint{5.482896in}{0.688724in}}{\pgfqpoint{5.493495in}{0.684333in}}{\pgfqpoint{5.504545in}{0.684333in}}%
\pgfpathclose%
\pgfusepath{stroke,fill}%
\end{pgfscope}%
\begin{pgfscope}%
\pgfpathrectangle{\pgfqpoint{0.800000in}{0.528000in}}{\pgfqpoint{4.960000in}{3.696000in}}%
\pgfusepath{clip}%
\pgfsetbuttcap%
\pgfsetroundjoin%
\definecolor{currentfill}{rgb}{0.000000,0.000000,0.000000}%
\pgfsetfillcolor{currentfill}%
\pgfsetlinewidth{1.003750pt}%
\definecolor{currentstroke}{rgb}{0.000000,0.000000,0.000000}%
\pgfsetstrokecolor{currentstroke}%
\pgfsetdash{}{0pt}%
\pgfpathmoveto{\pgfqpoint{5.504545in}{0.684333in}}%
\pgfpathcurveto{\pgfqpoint{5.515596in}{0.684333in}}{\pgfqpoint{5.526195in}{0.688724in}}{\pgfqpoint{5.534008in}{0.696537in}}%
\pgfpathcurveto{\pgfqpoint{5.541822in}{0.704351in}}{\pgfqpoint{5.546212in}{0.714950in}}{\pgfqpoint{5.546212in}{0.726000in}}%
\pgfpathcurveto{\pgfqpoint{5.546212in}{0.737050in}}{\pgfqpoint{5.541822in}{0.747649in}}{\pgfqpoint{5.534008in}{0.755463in}}%
\pgfpathcurveto{\pgfqpoint{5.526195in}{0.763276in}}{\pgfqpoint{5.515596in}{0.767667in}}{\pgfqpoint{5.504545in}{0.767667in}}%
\pgfpathcurveto{\pgfqpoint{5.493495in}{0.767667in}}{\pgfqpoint{5.482896in}{0.763276in}}{\pgfqpoint{5.475083in}{0.755463in}}%
\pgfpathcurveto{\pgfqpoint{5.467269in}{0.747649in}}{\pgfqpoint{5.462879in}{0.737050in}}{\pgfqpoint{5.462879in}{0.726000in}}%
\pgfpathcurveto{\pgfqpoint{5.462879in}{0.714950in}}{\pgfqpoint{5.467269in}{0.704351in}}{\pgfqpoint{5.475083in}{0.696537in}}%
\pgfpathcurveto{\pgfqpoint{5.482896in}{0.688724in}}{\pgfqpoint{5.493495in}{0.684333in}}{\pgfqpoint{5.504545in}{0.684333in}}%
\pgfpathclose%
\pgfusepath{stroke,fill}%
\end{pgfscope}%
\begin{pgfscope}%
\pgfpathrectangle{\pgfqpoint{0.800000in}{0.528000in}}{\pgfqpoint{4.960000in}{3.696000in}}%
\pgfusepath{clip}%
\pgfsetbuttcap%
\pgfsetroundjoin%
\definecolor{currentfill}{rgb}{0.000000,0.000000,0.000000}%
\pgfsetfillcolor{currentfill}%
\pgfsetlinewidth{1.003750pt}%
\definecolor{currentstroke}{rgb}{0.000000,0.000000,0.000000}%
\pgfsetstrokecolor{currentstroke}%
\pgfsetdash{}{0pt}%
\pgfpathmoveto{\pgfqpoint{5.504545in}{0.684333in}}%
\pgfpathcurveto{\pgfqpoint{5.515596in}{0.684333in}}{\pgfqpoint{5.526195in}{0.688724in}}{\pgfqpoint{5.534008in}{0.696537in}}%
\pgfpathcurveto{\pgfqpoint{5.541822in}{0.704351in}}{\pgfqpoint{5.546212in}{0.714950in}}{\pgfqpoint{5.546212in}{0.726000in}}%
\pgfpathcurveto{\pgfqpoint{5.546212in}{0.737050in}}{\pgfqpoint{5.541822in}{0.747649in}}{\pgfqpoint{5.534008in}{0.755463in}}%
\pgfpathcurveto{\pgfqpoint{5.526195in}{0.763276in}}{\pgfqpoint{5.515596in}{0.767667in}}{\pgfqpoint{5.504545in}{0.767667in}}%
\pgfpathcurveto{\pgfqpoint{5.493495in}{0.767667in}}{\pgfqpoint{5.482896in}{0.763276in}}{\pgfqpoint{5.475083in}{0.755463in}}%
\pgfpathcurveto{\pgfqpoint{5.467269in}{0.747649in}}{\pgfqpoint{5.462879in}{0.737050in}}{\pgfqpoint{5.462879in}{0.726000in}}%
\pgfpathcurveto{\pgfqpoint{5.462879in}{0.714950in}}{\pgfqpoint{5.467269in}{0.704351in}}{\pgfqpoint{5.475083in}{0.696537in}}%
\pgfpathcurveto{\pgfqpoint{5.482896in}{0.688724in}}{\pgfqpoint{5.493495in}{0.684333in}}{\pgfqpoint{5.504545in}{0.684333in}}%
\pgfpathclose%
\pgfusepath{stroke,fill}%
\end{pgfscope}%
\begin{pgfscope}%
\pgfpathrectangle{\pgfqpoint{0.800000in}{0.528000in}}{\pgfqpoint{4.960000in}{3.696000in}}%
\pgfusepath{clip}%
\pgfsetbuttcap%
\pgfsetroundjoin%
\definecolor{currentfill}{rgb}{0.000000,0.000000,0.000000}%
\pgfsetfillcolor{currentfill}%
\pgfsetlinewidth{1.003750pt}%
\definecolor{currentstroke}{rgb}{0.000000,0.000000,0.000000}%
\pgfsetstrokecolor{currentstroke}%
\pgfsetdash{}{0pt}%
\pgfpathmoveto{\pgfqpoint{5.504545in}{3.984333in}}%
\pgfpathcurveto{\pgfqpoint{5.515596in}{3.984333in}}{\pgfqpoint{5.526195in}{3.988724in}}{\pgfqpoint{5.534008in}{3.996537in}}%
\pgfpathcurveto{\pgfqpoint{5.541822in}{4.004351in}}{\pgfqpoint{5.546212in}{4.014950in}}{\pgfqpoint{5.546212in}{4.026000in}}%
\pgfpathcurveto{\pgfqpoint{5.546212in}{4.037050in}}{\pgfqpoint{5.541822in}{4.047649in}}{\pgfqpoint{5.534008in}{4.055463in}}%
\pgfpathcurveto{\pgfqpoint{5.526195in}{4.063276in}}{\pgfqpoint{5.515596in}{4.067667in}}{\pgfqpoint{5.504545in}{4.067667in}}%
\pgfpathcurveto{\pgfqpoint{5.493495in}{4.067667in}}{\pgfqpoint{5.482896in}{4.063276in}}{\pgfqpoint{5.475083in}{4.055463in}}%
\pgfpathcurveto{\pgfqpoint{5.467269in}{4.047649in}}{\pgfqpoint{5.462879in}{4.037050in}}{\pgfqpoint{5.462879in}{4.026000in}}%
\pgfpathcurveto{\pgfqpoint{5.462879in}{4.014950in}}{\pgfqpoint{5.467269in}{4.004351in}}{\pgfqpoint{5.475083in}{3.996537in}}%
\pgfpathcurveto{\pgfqpoint{5.482896in}{3.988724in}}{\pgfqpoint{5.493495in}{3.984333in}}{\pgfqpoint{5.504545in}{3.984333in}}%
\pgfpathclose%
\pgfusepath{stroke,fill}%
\end{pgfscope}%
\begin{pgfscope}%
\pgfpathrectangle{\pgfqpoint{0.800000in}{0.528000in}}{\pgfqpoint{4.960000in}{3.696000in}}%
\pgfusepath{clip}%
\pgfsetbuttcap%
\pgfsetroundjoin%
\definecolor{currentfill}{rgb}{0.000000,0.000000,0.000000}%
\pgfsetfillcolor{currentfill}%
\pgfsetlinewidth{1.003750pt}%
\definecolor{currentstroke}{rgb}{0.000000,0.000000,0.000000}%
\pgfsetstrokecolor{currentstroke}%
\pgfsetdash{}{0pt}%
\pgfpathmoveto{\pgfqpoint{5.504545in}{3.984333in}}%
\pgfpathcurveto{\pgfqpoint{5.515596in}{3.984333in}}{\pgfqpoint{5.526195in}{3.988724in}}{\pgfqpoint{5.534008in}{3.996537in}}%
\pgfpathcurveto{\pgfqpoint{5.541822in}{4.004351in}}{\pgfqpoint{5.546212in}{4.014950in}}{\pgfqpoint{5.546212in}{4.026000in}}%
\pgfpathcurveto{\pgfqpoint{5.546212in}{4.037050in}}{\pgfqpoint{5.541822in}{4.047649in}}{\pgfqpoint{5.534008in}{4.055463in}}%
\pgfpathcurveto{\pgfqpoint{5.526195in}{4.063276in}}{\pgfqpoint{5.515596in}{4.067667in}}{\pgfqpoint{5.504545in}{4.067667in}}%
\pgfpathcurveto{\pgfqpoint{5.493495in}{4.067667in}}{\pgfqpoint{5.482896in}{4.063276in}}{\pgfqpoint{5.475083in}{4.055463in}}%
\pgfpathcurveto{\pgfqpoint{5.467269in}{4.047649in}}{\pgfqpoint{5.462879in}{4.037050in}}{\pgfqpoint{5.462879in}{4.026000in}}%
\pgfpathcurveto{\pgfqpoint{5.462879in}{4.014950in}}{\pgfqpoint{5.467269in}{4.004351in}}{\pgfqpoint{5.475083in}{3.996537in}}%
\pgfpathcurveto{\pgfqpoint{5.482896in}{3.988724in}}{\pgfqpoint{5.493495in}{3.984333in}}{\pgfqpoint{5.504545in}{3.984333in}}%
\pgfpathclose%
\pgfusepath{stroke,fill}%
\end{pgfscope}%
\begin{pgfscope}%
\pgfpathrectangle{\pgfqpoint{0.800000in}{0.528000in}}{\pgfqpoint{4.960000in}{3.696000in}}%
\pgfusepath{clip}%
\pgfsetbuttcap%
\pgfsetroundjoin%
\definecolor{currentfill}{rgb}{0.000000,0.000000,0.000000}%
\pgfsetfillcolor{currentfill}%
\pgfsetlinewidth{1.003750pt}%
\definecolor{currentstroke}{rgb}{0.000000,0.000000,0.000000}%
\pgfsetstrokecolor{currentstroke}%
\pgfsetdash{}{0pt}%
\pgfpathmoveto{\pgfqpoint{5.504545in}{0.684333in}}%
\pgfpathcurveto{\pgfqpoint{5.515596in}{0.684333in}}{\pgfqpoint{5.526195in}{0.688724in}}{\pgfqpoint{5.534008in}{0.696537in}}%
\pgfpathcurveto{\pgfqpoint{5.541822in}{0.704351in}}{\pgfqpoint{5.546212in}{0.714950in}}{\pgfqpoint{5.546212in}{0.726000in}}%
\pgfpathcurveto{\pgfqpoint{5.546212in}{0.737050in}}{\pgfqpoint{5.541822in}{0.747649in}}{\pgfqpoint{5.534008in}{0.755463in}}%
\pgfpathcurveto{\pgfqpoint{5.526195in}{0.763276in}}{\pgfqpoint{5.515596in}{0.767667in}}{\pgfqpoint{5.504545in}{0.767667in}}%
\pgfpathcurveto{\pgfqpoint{5.493495in}{0.767667in}}{\pgfqpoint{5.482896in}{0.763276in}}{\pgfqpoint{5.475083in}{0.755463in}}%
\pgfpathcurveto{\pgfqpoint{5.467269in}{0.747649in}}{\pgfqpoint{5.462879in}{0.737050in}}{\pgfqpoint{5.462879in}{0.726000in}}%
\pgfpathcurveto{\pgfqpoint{5.462879in}{0.714950in}}{\pgfqpoint{5.467269in}{0.704351in}}{\pgfqpoint{5.475083in}{0.696537in}}%
\pgfpathcurveto{\pgfqpoint{5.482896in}{0.688724in}}{\pgfqpoint{5.493495in}{0.684333in}}{\pgfqpoint{5.504545in}{0.684333in}}%
\pgfpathclose%
\pgfusepath{stroke,fill}%
\end{pgfscope}%
\begin{pgfscope}%
\pgfpathrectangle{\pgfqpoint{0.800000in}{0.528000in}}{\pgfqpoint{4.960000in}{3.696000in}}%
\pgfusepath{clip}%
\pgfsetbuttcap%
\pgfsetroundjoin%
\definecolor{currentfill}{rgb}{0.000000,0.000000,0.000000}%
\pgfsetfillcolor{currentfill}%
\pgfsetlinewidth{1.003750pt}%
\definecolor{currentstroke}{rgb}{0.000000,0.000000,0.000000}%
\pgfsetstrokecolor{currentstroke}%
\pgfsetdash{}{0pt}%
\pgfpathmoveto{\pgfqpoint{5.504545in}{3.984333in}}%
\pgfpathcurveto{\pgfqpoint{5.515596in}{3.984333in}}{\pgfqpoint{5.526195in}{3.988724in}}{\pgfqpoint{5.534008in}{3.996537in}}%
\pgfpathcurveto{\pgfqpoint{5.541822in}{4.004351in}}{\pgfqpoint{5.546212in}{4.014950in}}{\pgfqpoint{5.546212in}{4.026000in}}%
\pgfpathcurveto{\pgfqpoint{5.546212in}{4.037050in}}{\pgfqpoint{5.541822in}{4.047649in}}{\pgfqpoint{5.534008in}{4.055463in}}%
\pgfpathcurveto{\pgfqpoint{5.526195in}{4.063276in}}{\pgfqpoint{5.515596in}{4.067667in}}{\pgfqpoint{5.504545in}{4.067667in}}%
\pgfpathcurveto{\pgfqpoint{5.493495in}{4.067667in}}{\pgfqpoint{5.482896in}{4.063276in}}{\pgfqpoint{5.475083in}{4.055463in}}%
\pgfpathcurveto{\pgfqpoint{5.467269in}{4.047649in}}{\pgfqpoint{5.462879in}{4.037050in}}{\pgfqpoint{5.462879in}{4.026000in}}%
\pgfpathcurveto{\pgfqpoint{5.462879in}{4.014950in}}{\pgfqpoint{5.467269in}{4.004351in}}{\pgfqpoint{5.475083in}{3.996537in}}%
\pgfpathcurveto{\pgfqpoint{5.482896in}{3.988724in}}{\pgfqpoint{5.493495in}{3.984333in}}{\pgfqpoint{5.504545in}{3.984333in}}%
\pgfpathclose%
\pgfusepath{stroke,fill}%
\end{pgfscope}%
\begin{pgfscope}%
\pgfpathrectangle{\pgfqpoint{0.800000in}{0.528000in}}{\pgfqpoint{4.960000in}{3.696000in}}%
\pgfusepath{clip}%
\pgfsetbuttcap%
\pgfsetroundjoin%
\definecolor{currentfill}{rgb}{0.000000,0.000000,0.000000}%
\pgfsetfillcolor{currentfill}%
\pgfsetlinewidth{1.003750pt}%
\definecolor{currentstroke}{rgb}{0.000000,0.000000,0.000000}%
\pgfsetstrokecolor{currentstroke}%
\pgfsetdash{}{0pt}%
\pgfpathmoveto{\pgfqpoint{5.504545in}{0.684333in}}%
\pgfpathcurveto{\pgfqpoint{5.515596in}{0.684333in}}{\pgfqpoint{5.526195in}{0.688724in}}{\pgfqpoint{5.534008in}{0.696537in}}%
\pgfpathcurveto{\pgfqpoint{5.541822in}{0.704351in}}{\pgfqpoint{5.546212in}{0.714950in}}{\pgfqpoint{5.546212in}{0.726000in}}%
\pgfpathcurveto{\pgfqpoint{5.546212in}{0.737050in}}{\pgfqpoint{5.541822in}{0.747649in}}{\pgfqpoint{5.534008in}{0.755463in}}%
\pgfpathcurveto{\pgfqpoint{5.526195in}{0.763276in}}{\pgfqpoint{5.515596in}{0.767667in}}{\pgfqpoint{5.504545in}{0.767667in}}%
\pgfpathcurveto{\pgfqpoint{5.493495in}{0.767667in}}{\pgfqpoint{5.482896in}{0.763276in}}{\pgfqpoint{5.475083in}{0.755463in}}%
\pgfpathcurveto{\pgfqpoint{5.467269in}{0.747649in}}{\pgfqpoint{5.462879in}{0.737050in}}{\pgfqpoint{5.462879in}{0.726000in}}%
\pgfpathcurveto{\pgfqpoint{5.462879in}{0.714950in}}{\pgfqpoint{5.467269in}{0.704351in}}{\pgfqpoint{5.475083in}{0.696537in}}%
\pgfpathcurveto{\pgfqpoint{5.482896in}{0.688724in}}{\pgfqpoint{5.493495in}{0.684333in}}{\pgfqpoint{5.504545in}{0.684333in}}%
\pgfpathclose%
\pgfusepath{stroke,fill}%
\end{pgfscope}%
\begin{pgfscope}%
\pgfpathrectangle{\pgfqpoint{0.800000in}{0.528000in}}{\pgfqpoint{4.960000in}{3.696000in}}%
\pgfusepath{clip}%
\pgfsetbuttcap%
\pgfsetroundjoin%
\definecolor{currentfill}{rgb}{0.000000,0.000000,0.000000}%
\pgfsetfillcolor{currentfill}%
\pgfsetlinewidth{1.003750pt}%
\definecolor{currentstroke}{rgb}{0.000000,0.000000,0.000000}%
\pgfsetstrokecolor{currentstroke}%
\pgfsetdash{}{0pt}%
\pgfpathmoveto{\pgfqpoint{5.504545in}{0.684333in}}%
\pgfpathcurveto{\pgfqpoint{5.515596in}{0.684333in}}{\pgfqpoint{5.526195in}{0.688724in}}{\pgfqpoint{5.534008in}{0.696537in}}%
\pgfpathcurveto{\pgfqpoint{5.541822in}{0.704351in}}{\pgfqpoint{5.546212in}{0.714950in}}{\pgfqpoint{5.546212in}{0.726000in}}%
\pgfpathcurveto{\pgfqpoint{5.546212in}{0.737050in}}{\pgfqpoint{5.541822in}{0.747649in}}{\pgfqpoint{5.534008in}{0.755463in}}%
\pgfpathcurveto{\pgfqpoint{5.526195in}{0.763276in}}{\pgfqpoint{5.515596in}{0.767667in}}{\pgfqpoint{5.504545in}{0.767667in}}%
\pgfpathcurveto{\pgfqpoint{5.493495in}{0.767667in}}{\pgfqpoint{5.482896in}{0.763276in}}{\pgfqpoint{5.475083in}{0.755463in}}%
\pgfpathcurveto{\pgfqpoint{5.467269in}{0.747649in}}{\pgfqpoint{5.462879in}{0.737050in}}{\pgfqpoint{5.462879in}{0.726000in}}%
\pgfpathcurveto{\pgfqpoint{5.462879in}{0.714950in}}{\pgfqpoint{5.467269in}{0.704351in}}{\pgfqpoint{5.475083in}{0.696537in}}%
\pgfpathcurveto{\pgfqpoint{5.482896in}{0.688724in}}{\pgfqpoint{5.493495in}{0.684333in}}{\pgfqpoint{5.504545in}{0.684333in}}%
\pgfpathclose%
\pgfusepath{stroke,fill}%
\end{pgfscope}%
\begin{pgfscope}%
\pgfpathrectangle{\pgfqpoint{0.800000in}{0.528000in}}{\pgfqpoint{4.960000in}{3.696000in}}%
\pgfusepath{clip}%
\pgfsetbuttcap%
\pgfsetroundjoin%
\definecolor{currentfill}{rgb}{0.000000,0.000000,0.000000}%
\pgfsetfillcolor{currentfill}%
\pgfsetlinewidth{1.003750pt}%
\definecolor{currentstroke}{rgb}{0.000000,0.000000,0.000000}%
\pgfsetstrokecolor{currentstroke}%
\pgfsetdash{}{0pt}%
\pgfpathmoveto{\pgfqpoint{5.504545in}{0.684333in}}%
\pgfpathcurveto{\pgfqpoint{5.515596in}{0.684333in}}{\pgfqpoint{5.526195in}{0.688724in}}{\pgfqpoint{5.534008in}{0.696537in}}%
\pgfpathcurveto{\pgfqpoint{5.541822in}{0.704351in}}{\pgfqpoint{5.546212in}{0.714950in}}{\pgfqpoint{5.546212in}{0.726000in}}%
\pgfpathcurveto{\pgfqpoint{5.546212in}{0.737050in}}{\pgfqpoint{5.541822in}{0.747649in}}{\pgfqpoint{5.534008in}{0.755463in}}%
\pgfpathcurveto{\pgfqpoint{5.526195in}{0.763276in}}{\pgfqpoint{5.515596in}{0.767667in}}{\pgfqpoint{5.504545in}{0.767667in}}%
\pgfpathcurveto{\pgfqpoint{5.493495in}{0.767667in}}{\pgfqpoint{5.482896in}{0.763276in}}{\pgfqpoint{5.475083in}{0.755463in}}%
\pgfpathcurveto{\pgfqpoint{5.467269in}{0.747649in}}{\pgfqpoint{5.462879in}{0.737050in}}{\pgfqpoint{5.462879in}{0.726000in}}%
\pgfpathcurveto{\pgfqpoint{5.462879in}{0.714950in}}{\pgfqpoint{5.467269in}{0.704351in}}{\pgfqpoint{5.475083in}{0.696537in}}%
\pgfpathcurveto{\pgfqpoint{5.482896in}{0.688724in}}{\pgfqpoint{5.493495in}{0.684333in}}{\pgfqpoint{5.504545in}{0.684333in}}%
\pgfpathclose%
\pgfusepath{stroke,fill}%
\end{pgfscope}%
\begin{pgfscope}%
\pgfpathrectangle{\pgfqpoint{0.800000in}{0.528000in}}{\pgfqpoint{4.960000in}{3.696000in}}%
\pgfusepath{clip}%
\pgfsetbuttcap%
\pgfsetroundjoin%
\definecolor{currentfill}{rgb}{0.000000,0.000000,0.000000}%
\pgfsetfillcolor{currentfill}%
\pgfsetlinewidth{1.003750pt}%
\definecolor{currentstroke}{rgb}{0.000000,0.000000,0.000000}%
\pgfsetstrokecolor{currentstroke}%
\pgfsetdash{}{0pt}%
\pgfpathmoveto{\pgfqpoint{5.504545in}{3.984333in}}%
\pgfpathcurveto{\pgfqpoint{5.515596in}{3.984333in}}{\pgfqpoint{5.526195in}{3.988724in}}{\pgfqpoint{5.534008in}{3.996537in}}%
\pgfpathcurveto{\pgfqpoint{5.541822in}{4.004351in}}{\pgfqpoint{5.546212in}{4.014950in}}{\pgfqpoint{5.546212in}{4.026000in}}%
\pgfpathcurveto{\pgfqpoint{5.546212in}{4.037050in}}{\pgfqpoint{5.541822in}{4.047649in}}{\pgfqpoint{5.534008in}{4.055463in}}%
\pgfpathcurveto{\pgfqpoint{5.526195in}{4.063276in}}{\pgfqpoint{5.515596in}{4.067667in}}{\pgfqpoint{5.504545in}{4.067667in}}%
\pgfpathcurveto{\pgfqpoint{5.493495in}{4.067667in}}{\pgfqpoint{5.482896in}{4.063276in}}{\pgfqpoint{5.475083in}{4.055463in}}%
\pgfpathcurveto{\pgfqpoint{5.467269in}{4.047649in}}{\pgfqpoint{5.462879in}{4.037050in}}{\pgfqpoint{5.462879in}{4.026000in}}%
\pgfpathcurveto{\pgfqpoint{5.462879in}{4.014950in}}{\pgfqpoint{5.467269in}{4.004351in}}{\pgfqpoint{5.475083in}{3.996537in}}%
\pgfpathcurveto{\pgfqpoint{5.482896in}{3.988724in}}{\pgfqpoint{5.493495in}{3.984333in}}{\pgfqpoint{5.504545in}{3.984333in}}%
\pgfpathclose%
\pgfusepath{stroke,fill}%
\end{pgfscope}%
\begin{pgfscope}%
\pgfpathrectangle{\pgfqpoint{0.800000in}{0.528000in}}{\pgfqpoint{4.960000in}{3.696000in}}%
\pgfusepath{clip}%
\pgfsetbuttcap%
\pgfsetroundjoin%
\definecolor{currentfill}{rgb}{0.000000,0.000000,0.000000}%
\pgfsetfillcolor{currentfill}%
\pgfsetlinewidth{1.003750pt}%
\definecolor{currentstroke}{rgb}{0.000000,0.000000,0.000000}%
\pgfsetstrokecolor{currentstroke}%
\pgfsetdash{}{0pt}%
\pgfpathmoveto{\pgfqpoint{5.504545in}{0.684333in}}%
\pgfpathcurveto{\pgfqpoint{5.515596in}{0.684333in}}{\pgfqpoint{5.526195in}{0.688724in}}{\pgfqpoint{5.534008in}{0.696537in}}%
\pgfpathcurveto{\pgfqpoint{5.541822in}{0.704351in}}{\pgfqpoint{5.546212in}{0.714950in}}{\pgfqpoint{5.546212in}{0.726000in}}%
\pgfpathcurveto{\pgfqpoint{5.546212in}{0.737050in}}{\pgfqpoint{5.541822in}{0.747649in}}{\pgfqpoint{5.534008in}{0.755463in}}%
\pgfpathcurveto{\pgfqpoint{5.526195in}{0.763276in}}{\pgfqpoint{5.515596in}{0.767667in}}{\pgfqpoint{5.504545in}{0.767667in}}%
\pgfpathcurveto{\pgfqpoint{5.493495in}{0.767667in}}{\pgfqpoint{5.482896in}{0.763276in}}{\pgfqpoint{5.475083in}{0.755463in}}%
\pgfpathcurveto{\pgfqpoint{5.467269in}{0.747649in}}{\pgfqpoint{5.462879in}{0.737050in}}{\pgfqpoint{5.462879in}{0.726000in}}%
\pgfpathcurveto{\pgfqpoint{5.462879in}{0.714950in}}{\pgfqpoint{5.467269in}{0.704351in}}{\pgfqpoint{5.475083in}{0.696537in}}%
\pgfpathcurveto{\pgfqpoint{5.482896in}{0.688724in}}{\pgfqpoint{5.493495in}{0.684333in}}{\pgfqpoint{5.504545in}{0.684333in}}%
\pgfpathclose%
\pgfusepath{stroke,fill}%
\end{pgfscope}%
\begin{pgfscope}%
\pgfpathrectangle{\pgfqpoint{0.800000in}{0.528000in}}{\pgfqpoint{4.960000in}{3.696000in}}%
\pgfusepath{clip}%
\pgfsetbuttcap%
\pgfsetroundjoin%
\definecolor{currentfill}{rgb}{0.000000,0.000000,0.000000}%
\pgfsetfillcolor{currentfill}%
\pgfsetlinewidth{1.003750pt}%
\definecolor{currentstroke}{rgb}{0.000000,0.000000,0.000000}%
\pgfsetstrokecolor{currentstroke}%
\pgfsetdash{}{0pt}%
\pgfpathmoveto{\pgfqpoint{5.504545in}{3.984333in}}%
\pgfpathcurveto{\pgfqpoint{5.515596in}{3.984333in}}{\pgfqpoint{5.526195in}{3.988724in}}{\pgfqpoint{5.534008in}{3.996537in}}%
\pgfpathcurveto{\pgfqpoint{5.541822in}{4.004351in}}{\pgfqpoint{5.546212in}{4.014950in}}{\pgfqpoint{5.546212in}{4.026000in}}%
\pgfpathcurveto{\pgfqpoint{5.546212in}{4.037050in}}{\pgfqpoint{5.541822in}{4.047649in}}{\pgfqpoint{5.534008in}{4.055463in}}%
\pgfpathcurveto{\pgfqpoint{5.526195in}{4.063276in}}{\pgfqpoint{5.515596in}{4.067667in}}{\pgfqpoint{5.504545in}{4.067667in}}%
\pgfpathcurveto{\pgfqpoint{5.493495in}{4.067667in}}{\pgfqpoint{5.482896in}{4.063276in}}{\pgfqpoint{5.475083in}{4.055463in}}%
\pgfpathcurveto{\pgfqpoint{5.467269in}{4.047649in}}{\pgfqpoint{5.462879in}{4.037050in}}{\pgfqpoint{5.462879in}{4.026000in}}%
\pgfpathcurveto{\pgfqpoint{5.462879in}{4.014950in}}{\pgfqpoint{5.467269in}{4.004351in}}{\pgfqpoint{5.475083in}{3.996537in}}%
\pgfpathcurveto{\pgfqpoint{5.482896in}{3.988724in}}{\pgfqpoint{5.493495in}{3.984333in}}{\pgfqpoint{5.504545in}{3.984333in}}%
\pgfpathclose%
\pgfusepath{stroke,fill}%
\end{pgfscope}%
\begin{pgfscope}%
\pgfpathrectangle{\pgfqpoint{0.800000in}{0.528000in}}{\pgfqpoint{4.960000in}{3.696000in}}%
\pgfusepath{clip}%
\pgfsetbuttcap%
\pgfsetroundjoin%
\definecolor{currentfill}{rgb}{0.000000,0.000000,0.000000}%
\pgfsetfillcolor{currentfill}%
\pgfsetlinewidth{1.003750pt}%
\definecolor{currentstroke}{rgb}{0.000000,0.000000,0.000000}%
\pgfsetstrokecolor{currentstroke}%
\pgfsetdash{}{0pt}%
\pgfpathmoveto{\pgfqpoint{5.504545in}{0.684333in}}%
\pgfpathcurveto{\pgfqpoint{5.515596in}{0.684333in}}{\pgfqpoint{5.526195in}{0.688724in}}{\pgfqpoint{5.534008in}{0.696537in}}%
\pgfpathcurveto{\pgfqpoint{5.541822in}{0.704351in}}{\pgfqpoint{5.546212in}{0.714950in}}{\pgfqpoint{5.546212in}{0.726000in}}%
\pgfpathcurveto{\pgfqpoint{5.546212in}{0.737050in}}{\pgfqpoint{5.541822in}{0.747649in}}{\pgfqpoint{5.534008in}{0.755463in}}%
\pgfpathcurveto{\pgfqpoint{5.526195in}{0.763276in}}{\pgfqpoint{5.515596in}{0.767667in}}{\pgfqpoint{5.504545in}{0.767667in}}%
\pgfpathcurveto{\pgfqpoint{5.493495in}{0.767667in}}{\pgfqpoint{5.482896in}{0.763276in}}{\pgfqpoint{5.475083in}{0.755463in}}%
\pgfpathcurveto{\pgfqpoint{5.467269in}{0.747649in}}{\pgfqpoint{5.462879in}{0.737050in}}{\pgfqpoint{5.462879in}{0.726000in}}%
\pgfpathcurveto{\pgfqpoint{5.462879in}{0.714950in}}{\pgfqpoint{5.467269in}{0.704351in}}{\pgfqpoint{5.475083in}{0.696537in}}%
\pgfpathcurveto{\pgfqpoint{5.482896in}{0.688724in}}{\pgfqpoint{5.493495in}{0.684333in}}{\pgfqpoint{5.504545in}{0.684333in}}%
\pgfpathclose%
\pgfusepath{stroke,fill}%
\end{pgfscope}%
\begin{pgfscope}%
\pgfpathrectangle{\pgfqpoint{0.800000in}{0.528000in}}{\pgfqpoint{4.960000in}{3.696000in}}%
\pgfusepath{clip}%
\pgfsetbuttcap%
\pgfsetroundjoin%
\definecolor{currentfill}{rgb}{0.000000,0.000000,0.000000}%
\pgfsetfillcolor{currentfill}%
\pgfsetlinewidth{1.003750pt}%
\definecolor{currentstroke}{rgb}{0.000000,0.000000,0.000000}%
\pgfsetstrokecolor{currentstroke}%
\pgfsetdash{}{0pt}%
\pgfpathmoveto{\pgfqpoint{5.504545in}{0.684333in}}%
\pgfpathcurveto{\pgfqpoint{5.515596in}{0.684333in}}{\pgfqpoint{5.526195in}{0.688724in}}{\pgfqpoint{5.534008in}{0.696537in}}%
\pgfpathcurveto{\pgfqpoint{5.541822in}{0.704351in}}{\pgfqpoint{5.546212in}{0.714950in}}{\pgfqpoint{5.546212in}{0.726000in}}%
\pgfpathcurveto{\pgfqpoint{5.546212in}{0.737050in}}{\pgfqpoint{5.541822in}{0.747649in}}{\pgfqpoint{5.534008in}{0.755463in}}%
\pgfpathcurveto{\pgfqpoint{5.526195in}{0.763276in}}{\pgfqpoint{5.515596in}{0.767667in}}{\pgfqpoint{5.504545in}{0.767667in}}%
\pgfpathcurveto{\pgfqpoint{5.493495in}{0.767667in}}{\pgfqpoint{5.482896in}{0.763276in}}{\pgfqpoint{5.475083in}{0.755463in}}%
\pgfpathcurveto{\pgfqpoint{5.467269in}{0.747649in}}{\pgfqpoint{5.462879in}{0.737050in}}{\pgfqpoint{5.462879in}{0.726000in}}%
\pgfpathcurveto{\pgfqpoint{5.462879in}{0.714950in}}{\pgfqpoint{5.467269in}{0.704351in}}{\pgfqpoint{5.475083in}{0.696537in}}%
\pgfpathcurveto{\pgfqpoint{5.482896in}{0.688724in}}{\pgfqpoint{5.493495in}{0.684333in}}{\pgfqpoint{5.504545in}{0.684333in}}%
\pgfpathclose%
\pgfusepath{stroke,fill}%
\end{pgfscope}%
\begin{pgfscope}%
\pgfpathrectangle{\pgfqpoint{0.800000in}{0.528000in}}{\pgfqpoint{4.960000in}{3.696000in}}%
\pgfusepath{clip}%
\pgfsetbuttcap%
\pgfsetroundjoin%
\definecolor{currentfill}{rgb}{0.000000,0.000000,0.000000}%
\pgfsetfillcolor{currentfill}%
\pgfsetlinewidth{1.003750pt}%
\definecolor{currentstroke}{rgb}{0.000000,0.000000,0.000000}%
\pgfsetstrokecolor{currentstroke}%
\pgfsetdash{}{0pt}%
\pgfpathmoveto{\pgfqpoint{5.504545in}{3.984333in}}%
\pgfpathcurveto{\pgfqpoint{5.515596in}{3.984333in}}{\pgfqpoint{5.526195in}{3.988724in}}{\pgfqpoint{5.534008in}{3.996537in}}%
\pgfpathcurveto{\pgfqpoint{5.541822in}{4.004351in}}{\pgfqpoint{5.546212in}{4.014950in}}{\pgfqpoint{5.546212in}{4.026000in}}%
\pgfpathcurveto{\pgfqpoint{5.546212in}{4.037050in}}{\pgfqpoint{5.541822in}{4.047649in}}{\pgfqpoint{5.534008in}{4.055463in}}%
\pgfpathcurveto{\pgfqpoint{5.526195in}{4.063276in}}{\pgfqpoint{5.515596in}{4.067667in}}{\pgfqpoint{5.504545in}{4.067667in}}%
\pgfpathcurveto{\pgfqpoint{5.493495in}{4.067667in}}{\pgfqpoint{5.482896in}{4.063276in}}{\pgfqpoint{5.475083in}{4.055463in}}%
\pgfpathcurveto{\pgfqpoint{5.467269in}{4.047649in}}{\pgfqpoint{5.462879in}{4.037050in}}{\pgfqpoint{5.462879in}{4.026000in}}%
\pgfpathcurveto{\pgfqpoint{5.462879in}{4.014950in}}{\pgfqpoint{5.467269in}{4.004351in}}{\pgfqpoint{5.475083in}{3.996537in}}%
\pgfpathcurveto{\pgfqpoint{5.482896in}{3.988724in}}{\pgfqpoint{5.493495in}{3.984333in}}{\pgfqpoint{5.504545in}{3.984333in}}%
\pgfpathclose%
\pgfusepath{stroke,fill}%
\end{pgfscope}%
\begin{pgfscope}%
\pgfpathrectangle{\pgfqpoint{0.800000in}{0.528000in}}{\pgfqpoint{4.960000in}{3.696000in}}%
\pgfusepath{clip}%
\pgfsetbuttcap%
\pgfsetroundjoin%
\definecolor{currentfill}{rgb}{0.000000,0.000000,0.000000}%
\pgfsetfillcolor{currentfill}%
\pgfsetlinewidth{1.003750pt}%
\definecolor{currentstroke}{rgb}{0.000000,0.000000,0.000000}%
\pgfsetstrokecolor{currentstroke}%
\pgfsetdash{}{0pt}%
\pgfpathmoveto{\pgfqpoint{5.504545in}{0.684333in}}%
\pgfpathcurveto{\pgfqpoint{5.515596in}{0.684333in}}{\pgfqpoint{5.526195in}{0.688724in}}{\pgfqpoint{5.534008in}{0.696537in}}%
\pgfpathcurveto{\pgfqpoint{5.541822in}{0.704351in}}{\pgfqpoint{5.546212in}{0.714950in}}{\pgfqpoint{5.546212in}{0.726000in}}%
\pgfpathcurveto{\pgfqpoint{5.546212in}{0.737050in}}{\pgfqpoint{5.541822in}{0.747649in}}{\pgfqpoint{5.534008in}{0.755463in}}%
\pgfpathcurveto{\pgfqpoint{5.526195in}{0.763276in}}{\pgfqpoint{5.515596in}{0.767667in}}{\pgfqpoint{5.504545in}{0.767667in}}%
\pgfpathcurveto{\pgfqpoint{5.493495in}{0.767667in}}{\pgfqpoint{5.482896in}{0.763276in}}{\pgfqpoint{5.475083in}{0.755463in}}%
\pgfpathcurveto{\pgfqpoint{5.467269in}{0.747649in}}{\pgfqpoint{5.462879in}{0.737050in}}{\pgfqpoint{5.462879in}{0.726000in}}%
\pgfpathcurveto{\pgfqpoint{5.462879in}{0.714950in}}{\pgfqpoint{5.467269in}{0.704351in}}{\pgfqpoint{5.475083in}{0.696537in}}%
\pgfpathcurveto{\pgfqpoint{5.482896in}{0.688724in}}{\pgfqpoint{5.493495in}{0.684333in}}{\pgfqpoint{5.504545in}{0.684333in}}%
\pgfpathclose%
\pgfusepath{stroke,fill}%
\end{pgfscope}%
\begin{pgfscope}%
\pgfpathrectangle{\pgfqpoint{0.800000in}{0.528000in}}{\pgfqpoint{4.960000in}{3.696000in}}%
\pgfusepath{clip}%
\pgfsetbuttcap%
\pgfsetroundjoin%
\definecolor{currentfill}{rgb}{0.000000,0.000000,0.000000}%
\pgfsetfillcolor{currentfill}%
\pgfsetlinewidth{1.003750pt}%
\definecolor{currentstroke}{rgb}{0.000000,0.000000,0.000000}%
\pgfsetstrokecolor{currentstroke}%
\pgfsetdash{}{0pt}%
\pgfpathmoveto{\pgfqpoint{5.504545in}{3.984333in}}%
\pgfpathcurveto{\pgfqpoint{5.515596in}{3.984333in}}{\pgfqpoint{5.526195in}{3.988724in}}{\pgfqpoint{5.534008in}{3.996537in}}%
\pgfpathcurveto{\pgfqpoint{5.541822in}{4.004351in}}{\pgfqpoint{5.546212in}{4.014950in}}{\pgfqpoint{5.546212in}{4.026000in}}%
\pgfpathcurveto{\pgfqpoint{5.546212in}{4.037050in}}{\pgfqpoint{5.541822in}{4.047649in}}{\pgfqpoint{5.534008in}{4.055463in}}%
\pgfpathcurveto{\pgfqpoint{5.526195in}{4.063276in}}{\pgfqpoint{5.515596in}{4.067667in}}{\pgfqpoint{5.504545in}{4.067667in}}%
\pgfpathcurveto{\pgfqpoint{5.493495in}{4.067667in}}{\pgfqpoint{5.482896in}{4.063276in}}{\pgfqpoint{5.475083in}{4.055463in}}%
\pgfpathcurveto{\pgfqpoint{5.467269in}{4.047649in}}{\pgfqpoint{5.462879in}{4.037050in}}{\pgfqpoint{5.462879in}{4.026000in}}%
\pgfpathcurveto{\pgfqpoint{5.462879in}{4.014950in}}{\pgfqpoint{5.467269in}{4.004351in}}{\pgfqpoint{5.475083in}{3.996537in}}%
\pgfpathcurveto{\pgfqpoint{5.482896in}{3.988724in}}{\pgfqpoint{5.493495in}{3.984333in}}{\pgfqpoint{5.504545in}{3.984333in}}%
\pgfpathclose%
\pgfusepath{stroke,fill}%
\end{pgfscope}%
\begin{pgfscope}%
\pgfpathrectangle{\pgfqpoint{0.800000in}{0.528000in}}{\pgfqpoint{4.960000in}{3.696000in}}%
\pgfusepath{clip}%
\pgfsetbuttcap%
\pgfsetroundjoin%
\definecolor{currentfill}{rgb}{0.000000,0.000000,0.000000}%
\pgfsetfillcolor{currentfill}%
\pgfsetlinewidth{1.003750pt}%
\definecolor{currentstroke}{rgb}{0.000000,0.000000,0.000000}%
\pgfsetstrokecolor{currentstroke}%
\pgfsetdash{}{0pt}%
\pgfpathmoveto{\pgfqpoint{5.504545in}{0.684333in}}%
\pgfpathcurveto{\pgfqpoint{5.515596in}{0.684333in}}{\pgfqpoint{5.526195in}{0.688724in}}{\pgfqpoint{5.534008in}{0.696537in}}%
\pgfpathcurveto{\pgfqpoint{5.541822in}{0.704351in}}{\pgfqpoint{5.546212in}{0.714950in}}{\pgfqpoint{5.546212in}{0.726000in}}%
\pgfpathcurveto{\pgfqpoint{5.546212in}{0.737050in}}{\pgfqpoint{5.541822in}{0.747649in}}{\pgfqpoint{5.534008in}{0.755463in}}%
\pgfpathcurveto{\pgfqpoint{5.526195in}{0.763276in}}{\pgfqpoint{5.515596in}{0.767667in}}{\pgfqpoint{5.504545in}{0.767667in}}%
\pgfpathcurveto{\pgfqpoint{5.493495in}{0.767667in}}{\pgfqpoint{5.482896in}{0.763276in}}{\pgfqpoint{5.475083in}{0.755463in}}%
\pgfpathcurveto{\pgfqpoint{5.467269in}{0.747649in}}{\pgfqpoint{5.462879in}{0.737050in}}{\pgfqpoint{5.462879in}{0.726000in}}%
\pgfpathcurveto{\pgfqpoint{5.462879in}{0.714950in}}{\pgfqpoint{5.467269in}{0.704351in}}{\pgfqpoint{5.475083in}{0.696537in}}%
\pgfpathcurveto{\pgfqpoint{5.482896in}{0.688724in}}{\pgfqpoint{5.493495in}{0.684333in}}{\pgfqpoint{5.504545in}{0.684333in}}%
\pgfpathclose%
\pgfusepath{stroke,fill}%
\end{pgfscope}%
\begin{pgfscope}%
\pgfpathrectangle{\pgfqpoint{0.800000in}{0.528000in}}{\pgfqpoint{4.960000in}{3.696000in}}%
\pgfusepath{clip}%
\pgfsetbuttcap%
\pgfsetroundjoin%
\definecolor{currentfill}{rgb}{0.000000,0.000000,0.000000}%
\pgfsetfillcolor{currentfill}%
\pgfsetlinewidth{1.003750pt}%
\definecolor{currentstroke}{rgb}{0.000000,0.000000,0.000000}%
\pgfsetstrokecolor{currentstroke}%
\pgfsetdash{}{0pt}%
\pgfpathmoveto{\pgfqpoint{5.504545in}{0.684333in}}%
\pgfpathcurveto{\pgfqpoint{5.515596in}{0.684333in}}{\pgfqpoint{5.526195in}{0.688724in}}{\pgfqpoint{5.534008in}{0.696537in}}%
\pgfpathcurveto{\pgfqpoint{5.541822in}{0.704351in}}{\pgfqpoint{5.546212in}{0.714950in}}{\pgfqpoint{5.546212in}{0.726000in}}%
\pgfpathcurveto{\pgfqpoint{5.546212in}{0.737050in}}{\pgfqpoint{5.541822in}{0.747649in}}{\pgfqpoint{5.534008in}{0.755463in}}%
\pgfpathcurveto{\pgfqpoint{5.526195in}{0.763276in}}{\pgfqpoint{5.515596in}{0.767667in}}{\pgfqpoint{5.504545in}{0.767667in}}%
\pgfpathcurveto{\pgfqpoint{5.493495in}{0.767667in}}{\pgfqpoint{5.482896in}{0.763276in}}{\pgfqpoint{5.475083in}{0.755463in}}%
\pgfpathcurveto{\pgfqpoint{5.467269in}{0.747649in}}{\pgfqpoint{5.462879in}{0.737050in}}{\pgfqpoint{5.462879in}{0.726000in}}%
\pgfpathcurveto{\pgfqpoint{5.462879in}{0.714950in}}{\pgfqpoint{5.467269in}{0.704351in}}{\pgfqpoint{5.475083in}{0.696537in}}%
\pgfpathcurveto{\pgfqpoint{5.482896in}{0.688724in}}{\pgfqpoint{5.493495in}{0.684333in}}{\pgfqpoint{5.504545in}{0.684333in}}%
\pgfpathclose%
\pgfusepath{stroke,fill}%
\end{pgfscope}%
\begin{pgfscope}%
\pgfsetbuttcap%
\pgfsetroundjoin%
\definecolor{currentfill}{rgb}{0.000000,0.000000,0.000000}%
\pgfsetfillcolor{currentfill}%
\pgfsetlinewidth{0.803000pt}%
\definecolor{currentstroke}{rgb}{0.000000,0.000000,0.000000}%
\pgfsetstrokecolor{currentstroke}%
\pgfsetdash{}{0pt}%
\pgfsys@defobject{currentmarker}{\pgfqpoint{0.000000in}{-0.048611in}}{\pgfqpoint{0.000000in}{0.000000in}}{%
\pgfpathmoveto{\pgfqpoint{0.000000in}{0.000000in}}%
\pgfpathlineto{\pgfqpoint{0.000000in}{-0.048611in}}%
\pgfusepath{stroke,fill}%
}%
\begin{pgfscope}%
\pgfsys@transformshift{1.025906in}{0.528000in}%
\pgfsys@useobject{currentmarker}{}%
\end{pgfscope}%
\end{pgfscope}%
\begin{pgfscope}%
\definecolor{textcolor}{rgb}{0.000000,0.000000,0.000000}%
\pgfsetstrokecolor{textcolor}%
\pgfsetfillcolor{textcolor}%
\pgftext[x=1.025906in,y=0.430778in,,top]{\color{textcolor}\sffamily\fontsize{10.000000}{12.000000}\selectfont 20}%
\end{pgfscope}%
\begin{pgfscope}%
\pgfsetbuttcap%
\pgfsetroundjoin%
\definecolor{currentfill}{rgb}{0.000000,0.000000,0.000000}%
\pgfsetfillcolor{currentfill}%
\pgfsetlinewidth{0.803000pt}%
\definecolor{currentstroke}{rgb}{0.000000,0.000000,0.000000}%
\pgfsetstrokecolor{currentstroke}%
\pgfsetdash{}{0pt}%
\pgfsys@defobject{currentmarker}{\pgfqpoint{0.000000in}{-0.048611in}}{\pgfqpoint{0.000000in}{0.000000in}}{%
\pgfpathmoveto{\pgfqpoint{0.000000in}{0.000000in}}%
\pgfpathlineto{\pgfqpoint{0.000000in}{-0.048611in}}%
\pgfusepath{stroke,fill}%
}%
\begin{pgfscope}%
\pgfsys@transformshift{2.518786in}{0.528000in}%
\pgfsys@useobject{currentmarker}{}%
\end{pgfscope}%
\end{pgfscope}%
\begin{pgfscope}%
\definecolor{textcolor}{rgb}{0.000000,0.000000,0.000000}%
\pgfsetstrokecolor{textcolor}%
\pgfsetfillcolor{textcolor}%
\pgftext[x=2.518786in,y=0.430778in,,top]{\color{textcolor}\sffamily\fontsize{10.000000}{12.000000}\selectfont 40}%
\end{pgfscope}%
\begin{pgfscope}%
\pgfsetbuttcap%
\pgfsetroundjoin%
\definecolor{currentfill}{rgb}{0.000000,0.000000,0.000000}%
\pgfsetfillcolor{currentfill}%
\pgfsetlinewidth{0.803000pt}%
\definecolor{currentstroke}{rgb}{0.000000,0.000000,0.000000}%
\pgfsetstrokecolor{currentstroke}%
\pgfsetdash{}{0pt}%
\pgfsys@defobject{currentmarker}{\pgfqpoint{0.000000in}{-0.048611in}}{\pgfqpoint{0.000000in}{0.000000in}}{%
\pgfpathmoveto{\pgfqpoint{0.000000in}{0.000000in}}%
\pgfpathlineto{\pgfqpoint{0.000000in}{-0.048611in}}%
\pgfusepath{stroke,fill}%
}%
\begin{pgfscope}%
\pgfsys@transformshift{4.011666in}{0.528000in}%
\pgfsys@useobject{currentmarker}{}%
\end{pgfscope}%
\end{pgfscope}%
\begin{pgfscope}%
\definecolor{textcolor}{rgb}{0.000000,0.000000,0.000000}%
\pgfsetstrokecolor{textcolor}%
\pgfsetfillcolor{textcolor}%
\pgftext[x=4.011666in,y=0.430778in,,top]{\color{textcolor}\sffamily\fontsize{10.000000}{12.000000}\selectfont 60}%
\end{pgfscope}%
\begin{pgfscope}%
\pgfsetbuttcap%
\pgfsetroundjoin%
\definecolor{currentfill}{rgb}{0.000000,0.000000,0.000000}%
\pgfsetfillcolor{currentfill}%
\pgfsetlinewidth{0.803000pt}%
\definecolor{currentstroke}{rgb}{0.000000,0.000000,0.000000}%
\pgfsetstrokecolor{currentstroke}%
\pgfsetdash{}{0pt}%
\pgfsys@defobject{currentmarker}{\pgfqpoint{0.000000in}{-0.048611in}}{\pgfqpoint{0.000000in}{0.000000in}}{%
\pgfpathmoveto{\pgfqpoint{0.000000in}{0.000000in}}%
\pgfpathlineto{\pgfqpoint{0.000000in}{-0.048611in}}%
\pgfusepath{stroke,fill}%
}%
\begin{pgfscope}%
\pgfsys@transformshift{5.504545in}{0.528000in}%
\pgfsys@useobject{currentmarker}{}%
\end{pgfscope}%
\end{pgfscope}%
\begin{pgfscope}%
\definecolor{textcolor}{rgb}{0.000000,0.000000,0.000000}%
\pgfsetstrokecolor{textcolor}%
\pgfsetfillcolor{textcolor}%
\pgftext[x=5.504545in,y=0.430778in,,top]{\color{textcolor}\sffamily\fontsize{10.000000}{12.000000}\selectfont 80}%
\end{pgfscope}%
\begin{pgfscope}%
\definecolor{textcolor}{rgb}{0.000000,0.000000,0.000000}%
\pgfsetstrokecolor{textcolor}%
\pgfsetfillcolor{textcolor}%
\pgftext[x=3.280000in,y=0.240809in,,top]{\color{textcolor}\sffamily\fontsize{10.000000}{12.000000}\selectfont \(\displaystyle k\)}%
\end{pgfscope}%
\begin{pgfscope}%
\pgfsetbuttcap%
\pgfsetroundjoin%
\definecolor{currentfill}{rgb}{0.000000,0.000000,0.000000}%
\pgfsetfillcolor{currentfill}%
\pgfsetlinewidth{0.803000pt}%
\definecolor{currentstroke}{rgb}{0.000000,0.000000,0.000000}%
\pgfsetstrokecolor{currentstroke}%
\pgfsetdash{}{0pt}%
\pgfsys@defobject{currentmarker}{\pgfqpoint{-0.048611in}{0.000000in}}{\pgfqpoint{0.000000in}{0.000000in}}{%
\pgfpathmoveto{\pgfqpoint{0.000000in}{0.000000in}}%
\pgfpathlineto{\pgfqpoint{-0.048611in}{0.000000in}}%
\pgfusepath{stroke,fill}%
}%
\begin{pgfscope}%
\pgfsys@transformshift{0.800000in}{0.726000in}%
\pgfsys@useobject{currentmarker}{}%
\end{pgfscope}%
\end{pgfscope}%
\begin{pgfscope}%
\definecolor{textcolor}{rgb}{0.000000,0.000000,0.000000}%
\pgfsetstrokecolor{textcolor}%
\pgfsetfillcolor{textcolor}%
\pgftext[x=0.614413in,y=0.673238in,left,base]{\color{textcolor}\sffamily\fontsize{10.000000}{12.000000}\selectfont 4}%
\end{pgfscope}%
\begin{pgfscope}%
\pgfsetbuttcap%
\pgfsetroundjoin%
\definecolor{currentfill}{rgb}{0.000000,0.000000,0.000000}%
\pgfsetfillcolor{currentfill}%
\pgfsetlinewidth{0.803000pt}%
\definecolor{currentstroke}{rgb}{0.000000,0.000000,0.000000}%
\pgfsetstrokecolor{currentstroke}%
\pgfsetdash{}{0pt}%
\pgfsys@defobject{currentmarker}{\pgfqpoint{-0.048611in}{0.000000in}}{\pgfqpoint{0.000000in}{0.000000in}}{%
\pgfpathmoveto{\pgfqpoint{0.000000in}{0.000000in}}%
\pgfpathlineto{\pgfqpoint{-0.048611in}{0.000000in}}%
\pgfusepath{stroke,fill}%
}%
\begin{pgfscope}%
\pgfsys@transformshift{0.800000in}{4.026000in}%
\pgfsys@useobject{currentmarker}{}%
\end{pgfscope}%
\end{pgfscope}%
\begin{pgfscope}%
\definecolor{textcolor}{rgb}{0.000000,0.000000,0.000000}%
\pgfsetstrokecolor{textcolor}%
\pgfsetfillcolor{textcolor}%
\pgftext[x=0.614413in,y=3.973238in,left,base]{\color{textcolor}\sffamily\fontsize{10.000000}{12.000000}\selectfont 5}%
\end{pgfscope}%
\begin{pgfscope}%
\definecolor{textcolor}{rgb}{0.000000,0.000000,0.000000}%
\pgfsetstrokecolor{textcolor}%
\pgfsetfillcolor{textcolor}%
\pgftext[x=0.558857in,y=2.376000in,,bottom,rotate=90.000000]{\color{textcolor}\sffamily\fontsize{10.000000}{12.000000}\selectfont Number of GMRES Iterations}%
\end{pgfscope}%
\begin{pgfscope}%
\pgfsetrectcap%
\pgfsetmiterjoin%
\pgfsetlinewidth{0.803000pt}%
\definecolor{currentstroke}{rgb}{0.000000,0.000000,0.000000}%
\pgfsetstrokecolor{currentstroke}%
\pgfsetdash{}{0pt}%
\pgfpathmoveto{\pgfqpoint{0.800000in}{0.528000in}}%
\pgfpathlineto{\pgfqpoint{0.800000in}{4.224000in}}%
\pgfusepath{stroke}%
\end{pgfscope}%
\begin{pgfscope}%
\pgfsetrectcap%
\pgfsetmiterjoin%
\pgfsetlinewidth{0.803000pt}%
\definecolor{currentstroke}{rgb}{0.000000,0.000000,0.000000}%
\pgfsetstrokecolor{currentstroke}%
\pgfsetdash{}{0pt}%
\pgfpathmoveto{\pgfqpoint{5.760000in}{0.528000in}}%
\pgfpathlineto{\pgfqpoint{5.760000in}{4.224000in}}%
\pgfusepath{stroke}%
\end{pgfscope}%
\begin{pgfscope}%
\pgfsetrectcap%
\pgfsetmiterjoin%
\pgfsetlinewidth{0.803000pt}%
\definecolor{currentstroke}{rgb}{0.000000,0.000000,0.000000}%
\pgfsetstrokecolor{currentstroke}%
\pgfsetdash{}{0pt}%
\pgfpathmoveto{\pgfqpoint{0.800000in}{0.528000in}}%
\pgfpathlineto{\pgfqpoint{5.760000in}{0.528000in}}%
\pgfusepath{stroke}%
\end{pgfscope}%
\begin{pgfscope}%
\pgfsetrectcap%
\pgfsetmiterjoin%
\pgfsetlinewidth{0.803000pt}%
\definecolor{currentstroke}{rgb}{0.000000,0.000000,0.000000}%
\pgfsetstrokecolor{currentstroke}%
\pgfsetdash{}{0pt}%
\pgfpathmoveto{\pgfqpoint{0.800000in}{4.224000in}}%
\pgfpathlineto{\pgfqpoint{5.760000in}{4.224000in}}%
\pgfusepath{stroke}%
\end{pgfscope}%
\end{pgfpicture}%
\makeatother%
\endgroup%

  \caption{GMRES iteration counts for $\alpha = 0.5/k$}\label{fig:linfinityn2}
\end{subfigure}
\caption{GMRES iteration counts for $\AmatoI\Amatt$ where $\Aso=\Ast=1$ and $\NLiDRRR{\nso-\nst} = \alpha$ as described in \cref{sec:num}.}
\end{figure}
\optodo{Need to figure out how to get computer modern consistently in axis labels. Some bookmarks saved that may help.}
  


%Say that these are for the TEDP defined in \cref{def:TEDP}.

\section{Definitions and conditions}\label{sec:3}

We now state the necessary technical definitions to prove \cref{cor:1,cor:1a} above.

\subsection{The variational problem and the Galerkin method}\label{sec:vpGm}
As this \lcnamecref{chap:nbpc} concerns finite-element discretisations of the Helmholtz equation, we will work with the variational formulation of \cref{prob:edp}, \cref{prob:vedp} above.

\bre[The EDP with data in $(\HokDR)'$]
In \cref{prob:edp} we defined the EDP with the antilinear functional $\LE$ arising from a function $f\in \LtDR$. In the rest of the \lcnamecref{chap:nbpc}, 
%\item In the rest of the paper, we usually consider the EDP with data given by $\LE$ defined in \cref{eq:EDPvar}, but sometimes 
we sometimes consider the EDP with general $\LE\in \HozDDRp$ and we indicate when this is the case.
In this latter situation, we define the dual norm by
\beq\label{eq:dualnorm}
\NHokDRp{L}= \sup_{v\in \HozDDR} \frac{\abs{\LE(v)}}{\NHokDR{v}}.
\eeq
%where $\|\cdot\|_{\HokDR}$ is defined by \cref{eq:1knorm}.
\ere

For the remainder of this \lcnamecref{chap:nbpc}, we let $(\Vhp)_{h>0}$ be a family of finite-dimensional, nested subspaces of $\HozDDR$, whose union is dense in $\HozDDR$. More specifically, we let $\Vhp$ consist of piecewise-polynomials on a simplicial mesh $\cTh$ with mesh-size $h$
%\ednote{Euan says: have problem that want to allow $C^{1,1}$ $\Dm$, so that statements later about $H^2$ regularity are covered, but easiest to define triangulation and hence subspaces on Lipschitz domains -- Euan to discuss with Ivan}
and fixed polynomial degree $p$. (Note that the dimension $N$ of $\Vhp$ then satisfies $N\sim h^{-d}$.) As in \cref{rem:crimes} above, we ignore any variational crimes resulting from this discretisation.

We now define the analogue of the finite-element approximation \cref{prob:fevtedp} for \cref{prob:vedp} with general data $L \in \HokDRp.$
\bprob[Finite-element approximation of \cref{prob:vedp} with general data]\label{prob:fevedpgen}
We say that $\uh \in \Vhp$ is the \defn{finite-element approximation of $u$} (the solution to \cref{prob:vedp} with general right-hand side $L \in \HokDRp$) if
\beq\label{eq:galerkin}
\aE(\uh,\vh) = \LE(\vh) \tforall \vh \in \Vhp.
\eeq
\eprob
Observe that implicit in our use of $\aE$ in \cref{eq:galerkin} is the fact that we are realising the Dirichlet-to-Neumann map $\TR$ exactly on $\GR.$
%Definition of Galerkin method

We now define the matrices associated with our finite-element discretisation. Let $\{\phi_i, i= 1, \ldots, N\}$ be a basis for $\Vhp$ with each $\phi_i$ \emph{real-valued}.
Let 
\beq\label{eq:matrixSjdef}
\big(\Smat_{A}\big)_{ij}\de \int_\Omega \big(A \nabla \phi_j)\cdot\nabla \phi_i, \quad
\big(\Mmat_{n}\big)_{ij}\de \int_\Omega n\,\phi_i\, \phi_j,
\quad\tand\quad
\big(\Nmat\big)_{ij}\de \int_{\GR} T_R (\gamma\phi_j) \,\gamma \phi_i
\eeq
be the stiffness, domain-mass, and boundary-mass matrices, respectively. Note that both $\Smat_A$ and $\Mmat_n$ are \emph{real-valued}, but $\Nmat$ is \emph{complex-valued} (because the DtN operator $T_R$ is complex-valued).
Let
\beq\label{eq:matrixAdef}
\Amat \de \Smat_{A} - k^2 \Mmat_{n} - \Nmat,
\eeq
and let $u_h\de \sum_j u_j \phi_j$. Then \cref{eq:galerkin} implies that
\beqs
\Amat \bu = \bff,
\eeqs
where $(\bff)_i \de \FE(\phi_i)$.
Similar to above we let 
\beq\label{eq:matrixAjdef}
\Amatj \de \Smat_{A^{(j)}} - k^2 \Mmat_{n^{(j)}} - \Nmat.
\eeq
Finally, let 
\beq\label{eq:Dk2}
\Dmat_k\de \Smat_I + k^2 \Mmat_1;
\eeq
so that \cref{eq:Dk} holds.
%then the weighted norm $\|\cdot\|_{\Dmat_k}$ is given by 
%\beq\label{eq:Dk3}
%\N{\bv}_{\Dmat_k}^2\de   \N{v_h}^2_{\HokDR}=\big( \Dmat_k \bv,\bv\big)_2,
%\eeq
%for
%$v_h =\sum_i v_i \phi_i$.

We recall the definition of quasi-uniformity (c.f. \cite[Definition 4.4.13]{BrSc:08}):

\bde[Quasi-uniform]\label{def:quasiuniform}
Let $\set{\cTh}_{h>0}$ be a set of triangulations of $\DR$ indexed by their mesh size $h.$ For each $T \in \cTh,$ let $\BT$ denote the largest ball contained in $T$.
If there exists $\rho > 0$ such that 
\beqs
\min\set{\diam \BT \st T \in \cT} \geq \rho h,
\eeqs
then $\set{\cTh}_{h>0}$ is said to be \defn{quasi-uniform}.
\ede

\ble[Norm equivalences of FE functions]\label{lem:normequiv}
If $\set{{\cTh}_{h>0}}$ is quasi-uniform with a nodal basis, then
there exist $m_\pm$ and $s_\pm$, independent of $h$ and $p$, such that
\beq\label{eq:normequiv1}
m_- h^{d/2} \N{\bv}_2 \leq \N{v_h}_{\LtDR} \leq m_+ h^{d/2} \N{\bv}_2,
\eeq
and
\beq\label{eq:normequiv2}
s_- h^{d/2} \N{\bv}_2 \leq \N{\nabla v_h}_{\LtDR} \leq s_+ h^{d/2-1} \N{\bv}_2,
\eeq
for all finite-element functions $v_h =\sum_i v_i \phi_i \in \Vhp$.
\ele

Written in terms of the matrices $\Mmat_1$ and $\Smat_I$ defined in \cref{eq:matrixSjdef}, the bounds \cref{eq:normequiv1} and \cref{eq:normequiv2} are, respectively, the familiar bounds
\beqs
(\Mmat_1 \bv,\bv)_2 \sim h^d \N{\bv}^2_2 \quad\tand\quad h^{d}\N{\bv}^2_2 \lesssim (\Smat_I \bv,\bv)_2 \lesssim h^{d-2} \N{\bv}^2_2.
\eeqs

For a proof of \cref{lem:normequiv}, see\footnote{In \cite[Chapter V, Lemma 2.5] the assumption is made that the meshes underlying $\Vhp$ are \emph{uniform}. However, the definition of uniformity in \cite[Chapter 2, Definition 5.1(4)]{Br:07} is the same as the more standard definition of quasi-uniformity in \cref{def:quasiuniform}.} \cite[Chapter V, Lemma 2.5]{Br:07}.\ednote{Euan---in \cite[Chapter V, Lemma 2.5]{Br:07} this result is written slightly differently (with scaled basis functions and a scaled 2-norm) but it's equivalent to our result. I presume I don't need to put anything 'translating' their result to ours? (Can download the 3rd edition from the CUP website if you use your Bath institutional login.)}

%% \bpf[Sketch proof of \cref{lem:normequiv}]\opntodo{can omit this if can find a good reference. One possibility . Need to check basis scaling business.}
%% The inequalities in \cref{eq:normequiv1} follow from writing $\|v_h\|_{\LtDR}$ as a sum of integrals over elements of the mesh, and then mapping to the reference element \ednote{Euan to discuss with Ivan}.
%% %\beqs
%% %\N{v_h}^2_{L^2(\O
%% %\eeqs
%% Then, \cref{eq:normequiv2} follows from \cref{eq:normequiv1} and the inequalities
%% \beqs
%% \N{v_h}_{L^2(\DR)}\lesssim \N{\nabla v_h}_{L^2(\DR)}\lesssim \frac{1}{h} \N{v_h}_{L^2(\DR)},
%% \eeqs
%% the first of which follows from the Poincar\'e inequality, since $v_h \in \HozDDR$
%% (see, e.g., \cite[Proposition 5.3.4]{BrSc:00}), the second of which follows from a standard inverse estimate (see, e.g., \cite[Theorem 4.5.11]{BrSc:00}).
%% \epf


Finally, we need the concept of the \emph{adjoint} sesquilinear form to $a(\cdot,\cdot)$.
\begin{definition}[The adjoint sesquilinear form $a^*(\cdot,\cdot)$]\label{def:adjoint}
Let $\Dm$, $n$, and $A$, be as in the definition of the EDP (\cref{prob:edp}). The adjoint sesquilinear form, $a^*(\cdot,\cdot)$, to $a(\cdot,\cdot)$ defined in \cref{eq:aedp} is given by
\beq\label{eq:EDPadjoint}
a^*(\vo,\vt) \de \int_{\DR} 
\Big((A \grad \vo)\cdot\grad \vtb
 - k^2 n \vo\vtb\Big) - \big\langle \gamma \vo,T_R(\gamma \vt)\big\rangle_{\GR}.
\eeq
\end{definition}

\noi It is then straightforward to check that
\beq\label{eq:A*}
\Amat^* \de \Smat_A -k^2 \Mmat_n - \Nmat^*
\eeq
(where $^*$ denotes conjugate transpose) is the Galerkin matrix for the sesquilinear form $a^*(\cdot,\cdot)$; i.e.~$(\Amat^*)_{ij} = a^*(\phi_j, \phi_i)$.

\ble[Link between variational problems involving $a(\cdot,\cdot)$ and $a^*(\cdot,\cdot)$]\label{lem:adjoint}
Given $\LE\in (\HozDDR)'$, if $u$ is the solution to the variational problem
\beq\label{eq:adjoint1}
a^*(u,v)= \LE(v) \quad\tfa v\in \HozDDR,
\eeq
then $\overline{u}$ satisfies
\beq\label{eq:adjoint2}
a(\overline{u},v)= \overline{\LE(\overline{v})} \quad\tfa v\in \HozDDR.
\eeq
\ele\optodo{Consider putting something like this in Chap 2.}

For the proof of \cref{lem:adjoint} we need the following property of the DtN map $T_R$:
\beq\label{eq:DtN}
\big\langle T_R\psi, \overline{\phi} \big\rangle_\Gamma = \big\langle T_R \phi, \overline{\psi}\big\rangle_\Gamma \quad\tfa \phi,\psi \in H^{1/2}(\GR).
\eeq
This property follows from the fact that, if $\uo$ and $\ut$ are solutions of the homogeneous Helmholtz equation $\Delta u +k^2 u=0$ in $\RRd\setminus \overline{\BR}$, both satisfying the Sommerfeld radiation condition \cref{eq:src}, then
\beqs
\int_{\GR} (\gamma \uo)\, \dn \ut = \int_{\GR} (\gamma \ut)\,\dn \uo;
\eeqs
see, e.g., \cite[Lemma 6.13]{Sp:15}.

\bpf[Proof of \cref{lem:adjoint}]
From \cref{eq:adjoint1} we have that 
\beqs
\overline{a^*(u,\overline{v})}= \overline{\LE(\overline{v})} \quad\tfa v\in \HozDDR.
\eeqs
Using the definition of $a^*(\cdot,\cdot)$ and the property \cref{eq:DtN} in the left-hand side of this last equation, we find \cref{eq:adjoint2}.
\epf



\section{Proofs of the main results}\label{sec:proofs}

The main part of the proofs of \cref{cor:1,cor:1a} is the following \lcnamecref{thm:1}.

\begin{theorem}[Main ingredient of the answer to \cref{it:nbpcq1}]\label{thm:1}
Assume that $\Dm$, $\Aso$, and $\nso$ satisfy \cref{cond:1nbpc}, and assume that $h$ and $p$ satisfy \cref{cond:2}. 
Let the $k$- and $h$-independent constants $\mpm$ and $\spm$ be given as in \cref{lem:normequiv}.
Then, given $\kz>0$, there exist $\Co, \Ct>0$, independent of $h$ and $k$ (but dependent on $\Dm, \Aso, \nso$, $p$, and $\kz$) such that
\begin{align}\nonumber
&\max\Big\{
\NDmatk{\Imat - (\Amat^{(1)})^{-1}\Amat^{(2)}}, 
\N{\Imat -\Amat^{(2)} (\Amat^{(1)})^{-1}}_{(\Dmat_k)^{-1}}
\Big\}\\
&\hspace{3cm} 
\leq C_1 \,k \,
\NLiDRRRdtd{\Aso-\Ast} + C_2 \, k \, \NLiDRRR{\nso-\nst}
\label{eq:main1}
\end{align}
and 
\begin{align}\nonumber
&\max\Big\{
\N{\Imat - (\Amat^{(1)})^{-1}\Amat^{(2)}}_2, 
\N{\Imat -\Amat^{(2)} (\Amat^{(1)})^{-1}}_2
\Big\}\\
&\hspace{0cm} 
\leq C_1 \,\left(\frac{s_+}{m_-}\right) \,\frac{1}{h} \,
\NLiDRRRdtd{\Aso-\Ast} + C_2 \, \left(\frac{m_+}{m_-} \right)k \, \NLiDRRR{\nso-\nst}
\label{eq:main1a}
\end{align}
for all $k\geq k_0$. 
\end{theorem}



\subsection{Proof of \cref{thm:1}} 

The following two lemmas are the heart of the proof of \cref{thm:1}.

\ble[Bounds on $(\Amato)^{-1} \Mmat_{n}$]\label{lem:keylemma1}
Assume that \cref{cond:1nbpc} holds, and assume that Part (i) of \cref{cond:2} holds. Then, for $n\in \LiDRRR$,
\beq\label{eq:keybound1}
\max\Big\{\big\| (\Amato)^{-1} \Mmat_{n} \big\|_{\Dmat_k}, \,\,
\big\|  \Mmat_{n}(\Amato)^{-1} \big\|_{(\Dmat_k)^{-1}}
\Big\}\leq 
C_2
%\frac{m_+}{m_-} \left[ C_{\rm FEM1}^{(1)} + C_{\rm bound}^{(1)}\right] 
\frac{\N{n}_{L^\infty(\DR)}}{k}
\eeq
and 
\beq\label{eq:keybound1a}
\max\Big\{\big\| (\Amato)^{-1} \Mmat_{n} \big\|_2, \,\,
\big\|  \Mmat_{n}(\Amato)^{-1} \big\|_2 
\Big\}\leq 
C_2 
%\frac{m_+}{m_-} \left[ C_{\rm FEM1}^{(1)} + C_{\rm bound}^{(1)}\right] 
\left(\frac{m_+}{m_-}\right) \frac{\N{n}_{L^\infty(\DR)}}{k}
\eeq
for all $k\geq k_0$,
where
\beq\label{eq:C2}
C_2\de%\frac{m_+}{m_-} 
%\left[ 
C_{\rm FEM1}^{(1)} + C_{\rm bound}^{(1)}.%\right].
\eeq
\ele

\ble[Bounds on $(\Amato)^{-1} \Smat_A$]\label{lem:keylemma2}
Assume that \cref{cond:1nbpc} holds, and assume that Part (ii) of \cref{cond:2} holds. Then, for $A\in L^\infty(\DR,\RR^{d\times d})$,
\beq\label{eq:keybound2}
\max\Big\{\big\| (\Amato)^{-1} \Smat_A \big\|_{(\Dmat_k)^{-1}}, \,\,
\big\| \Smat_A (\Amato)^{-1} \big\|_{\Dmat_k}\Big\} \leq C_1\, k\N{A}_{L^\infty(\DR)}
\eeq
and
\beq\label{eq:keybound2a}
\max\Big\{\big\| (\Amato)^{-1} \Smat_A \big\|_2, \,\,
\big\| \Smat_A (\Amato)^{-1} \big\|_2\Big\} \leq C_1\,\left(\frac{s_+}{m_-}\right) \frac{1}{h}\N{A}_{L^\infty(\DR)}
\eeq
%\begin{align}\nonumber
%&\max\Big\{\big\| (\Amato)^{-1} \Smat_A \big\|_2, \,\,
%\big\| \Smat_A (\Amato)^{-1} \big\|_2\Big\}\nonumber \\
%&\hspace{2cm}
% \leq \frac{s_+}{s_-} \left[ C_{\rm FEM2}^{(1)} + 
% \frac{1}{\min\big\{\Asomin,\nsomin\big\}}\left( \frac{1}{k_0} + 2 C^{(1)}_{\rm bound}\nsomax  \right) \right]k\N{A}_{L^\infty(\DR)}\label{eq:keybound2}
%% + C_{\rm bound}^{(1)}\right) \frac{\N{n}_{L^\infty(\DR)}}{k}.
%\end{align}
for all $k\geq k_0$, where
\beq\label{eq:C1nbpc}
C_1\de%\frac{s_+}{s_-} 
\left[ C_{\rm FEM2}^{(1)} + 
 \frac{1}{\min\big\{\Asomin,\nsomin\big\}}\left( \frac{1}{k_0} + 2 C^{(1)}_{\rm bound}\nsomax  \right) \right]
\eeq
\ele

\bpf[Proof of \cref{thm:1} using \cref{lem:keylemma1,lem:keylemma2}]
Using the definition of the matrices $\Amatj, \SmatA$, and $\Mmatn$ in \cref{eq:matrixAjdef} and \cref{eq:matrixSjdef}, we have
\begin{align}\nonumber
\Imat - (\Amato)^{-1}\Amatt = (\Amato)^{-1}\big(\Amato-\Amatt\big) &=  (\Amato)^{-1}\left( \Smat_{A^{(1)}} - \Smat_{A^{(2)}} - k^2 \big(\Mmat_{n^{(1)}}-\Mmat_{n^{(2)}}\big)\right)\\
&= (\Amato)^{-1}\left( \Smat_{A^{(1)}-A^{(2)}} - k^2 \Mmat_{n^{(1)}-n^{(2)}}\right),\label{eq:idea1}
\end{align}
and similarly 
\beq\label{eq:idea2}
\Imat -\Amatt  (\Amato)^{-1}= \left( \Smat_{A^{(1)}-A^{(2)}} - k^2 \Mmat_{n^{(1)}-n^{(2)}}\right)(\Amato)^{-1}.
\eeq
The bounds  \cref{eq:main1} on $\|\Imat - (\Amato)^{-1}\Amatt\|_2$ and  $\|\Imat - \Amatt(\Amato)^{-1}\|_2$ then follow from using the bounds \cref{eq:keybound1} and \cref{eq:keybound2} in \cref{eq:idea1} and \cref{eq:idea2}.
%
%, and $C_1$, $C_2$ in \cref{eq:main1} are given explicitly by
%\beq\label{eq:C1nbpc}
%C_1\de%\frac{s_+}{s_-} 
%\left[ C_{\rm FEM2}^{(1)} + 
% \frac{1}{\min\big\{\Asomin,\nsomin\big\}}\left( \frac{1}{k_0} + 2 C^{(1)}_{\rm bound}\nsomax  \right) \right] \,\,\tand\,\,
%\quad C_2\de  %+ \frac{m_+}{m_-} 
% \left[ C_{\rm FEM1}^{(1)} + C_{\rm bound}^{(1)}\right].
%\eeq
\epf

\

\bpf[Proof of \cref{lem:keylemma1}]
We first concentrate on proving \cref{eq:keybound1}.
Given $\bff \in \CC^N$ and $n\in \LiDRRR$, we create a variational problem whose Galerkin discretisation leads to the equation $\Amato \tbu = \Mmat_n\,\bff$.
Indeed, let $\widetilde{f} \de \sum_j f_j \phi_j\in \HozDDR$. Define $\widetilde{u}$ to be the solution of the variational problem 
\beq\label{eq:411}
a^{(1)}(\widetilde{u},v)= (n\widetilde{f},v)_{L^2(\Omega)} \quad\text{ for all } v\in H^1(\Omega),
\eeq
and let $\tu_h$ be the solution of the finite-element approximation of \cref{eq:411}, i.e.,
\beq\label{eq:41}
a^{(1)}(\tu_h,v_h)= (n\widetilde{f},v_h)_{L^2(\Omega)} \quad\text{ for all } v_h\in \Vhp,
\eeq
and let $\tbu$ be the vector of nodal values of $\tu_h$. The definition of $\widetilde{f}$ then implies that \cref{eq:41} is equivalent to the linear system $\Amato \tbu = \Mmat_{n}\,\bff$, and so to obtain a bound on $\|(\Amato)^{-1}\Mmat_n\|_{\Dmat_k}$ we need to bound $\|\tbu\|_{\Dmat_k}$ in terms of $\|\bff\|_{\Dmat_k}$. (Recall $\bff \in \CCN$ was arbitrary.) Because of the definition 
of $\|\cdot\|_{\Dmat_k}$ in \cref{eq:Dk}, this is bound equivalent to bounding $\|\tu_h\|_{\HokDR}$ in terms of $\|\widetilde{f}\|_{\HokDR}$.

%First observe that the bound \cref{eq:bound3} from Part (i) of \cref{cond:2} holds for the solution of the variational problem
%\beqs%\label{eq:411}
%a^{(1)}(u,v)= (n\phi_j,v)_{L^2(\Omega)} \quad\text{ for all } v\in H^1(\Omega),
%\eeqs
%and hence, by linearity, it also holds for the solution $\widetilde{u}$ of the variational problem \cref{eq:411}.

Using %the bounds in \cref{eq:normequiv1}, 
the triangle inequality and the bounds \cref{eq:bound3} and \cref{eq:bound1} from \cref{cond:2,cond:1nbpc} respectively, we find
%Note that the hypotheses imply that the bound on the solution operator 
%\cref{eq:bound_unif} holds (by \cref{cor:uniform}), and also that if $h k\sqrt{|k^2-\eps|} \leq C_1$ then quasi-optimality \cref{eq:qoeps_lemma} holds (by \cref{lem:qo}).
%Starting with \cref{eq:equiv} we then have 
\begin{align}
%m_- h^{d/2}k \N{\tbu}_2 \leq k\N{\tu_h}_{\LtDR}\leq  
\N{\tu_h}_{\HokDR} \leq
\N{\tu-\tu_h}_{\HokDR} + \N{\tu}_{\HokDR} \label{eq:mainevent1}
& \leq C^{(1)}_{\rm FEM1}\NLtDR{n\ftilde} + C^{(1)}_{\rm bound}\NLtDR{n\ftilde} \\ 
& \leq \mleft(C^{(1)}_{\rm FEM1} + C^{(1)}_{\rm bound}\mright)\NLiDRRR{n}\NLtDR{\ftilde} \label{eq:mainevent1a} \\
& \leq\big(C^{(1)}_{\rm FEM1}+  C^{(1)}_{\rm bound}\big)\NLiDRRR{n}\frac{\big\|\widetilde{f}\big\|_{\HokDR}}{k};\nonumber
%& \leq\big(C^{(1)}_{\rm FEM1}+  C^{(1)}_{\rm bound}\big)\N{n}_{L^\infty(\DR)} m_+ h^{d/2} \N{\bff}_2,
\end{align}
the bound on $\|(\Amato)^{-1}\Mmat_n\|_{\Dmat_k}$ in \cref{eq:keybound1} then follows from the definition of $\|\cdot\|_{\Dmat_k}$ in \cref{eq:Dk} and the definition of $C_2$ \cref{eq:C2}.

To prove the bound on $\|\Mmat_n(\Amato)^{-1}\|_{(\Dmat_k)^{-1}}$ in \cref{eq:keybound1}, first observe that the definitions of $\|\cdot\|_{\Dmat_k}$ and $\|\cdot\|_{(\Dmat_k)^{-1}}$ in \cref{eq:Dk} imply that, for any matrix $\Cmat \in \CCNtN$ and for any $\bv\in \CC^N$,
\beq\label{eq:A380-0}
\frac{
\big\|\matrixC \bv \big\|_{(\Dmat_k)^{-1}}
}{
\big\|\bv\|_{(\Dmat_k)^{-1}}
} = 
\frac{
\big\|\matrixC^* \bw \big\|_{\Dmat_k}
}{
\big\|\bw\|_{\Dmat_k}
}
\eeq
where $\bw \de (\Dmat_k)^{1/2}\bv$, and where $\matrixC^*$ is the conjugate transpose of $\matrixC$ (i.e.~the adjoint with respect to $(\cdot,\cdot)_2$).
Therefore, since $\Mmat_n$ is a real, symmetric matrix,
\beqs
\frac{
\big\|\Mmat_n (\Amato)^{-1}\bv\big\|_{(\Dmat_k)^{-1}}
}{
\N{\bv}_{(\Dmat_k)^{-1}}
}
=
\frac{\NDk{\mleft(\AmatoI\Mmatn\mright)^* \bw}}{\NDk{\bw}}
= 
\frac{
\big\|((\Amato)^*)^{-1}\Mmat_n\bw\big\|_{\Dmat_k}
}{
\N{\bw}_{\Dmat_k}
},
 \eeqs
 so that 
\beq\label{eq:A380} 
 \big\|\Mmat_n (\Amato)^{-1}\big\|_{(\Dmat_k)^{-1}}=\big\|((\Amato)^*)^{-1}\Mmat_n\big\|_{\Dmat_k}.
 \eeq 
Recall from the text below \cref{eq:A*} that $(\Amato)^*$ is the Galerkin matrix corresponding to the variational problem \cref{eq:adjoint1} -- the adjoint problem. \cref{lem:adjoint} implies that if the EDP %with coefficients $A^{(1)}$ and $n^{(1)}$ 
satisfies \cref{cond:1nbpc,cond:2}, then so does the adjoint problem. Therefore, the argument above leading to the bound on $\|(\Amato)^{-1}\Mmat_n\|_{\Dmat_k}$ under \cref{cond:1nbpc} and Part (i) of \cref{cond:2} proves the same bound on $\|((\Amato)^*)^{-1}\Mmat_n\|_{\Dmat_k}$, and then, using \cref{eq:A380}, also on $\big\|\Mmat_n(\Amato)^{-1}\big\|_{(\Dmat_k)^{-1}}$.

To prove the bound on  $\|(\Amato)^{-1}\Mmat_n\|_{2}$ in \cref{eq:keybound1a}, we use the bounds 
\beqs
m_- h^{d/2} k \N{\tbu}_2 \leq k \N{\widetilde{u}_h}_{\LtDR} \leq \N{\widetilde{u}_h}_{\HokDR}
\,\tand\,
\big\|\widetilde{f}\big\|_{\LtDR} \leq m_+ h^{d/2}\N{\bff}_2,
\eeqs
on either side of the inequality \cref{eq:mainevent1}, with these bounds coming from \cref{eq:normequiv1}. The proof of the bound on 
$\|\Mmat_n((\Amato)^*)^{-1}\|_{2}$ in \cref{eq:keybound1a} follows in a similar way to above, using the fact that 
$\|\Mmat_n (\Amato)^{-1}\|_2=\|((\Amato)^*)^{-1}\Mmat_n\|_2$ (compare to \cref{eq:A380}).
%, namely the variational problem \cref{eq:EDPvar} with the operator $T_R$ in $a^{(1)}(\cdot,\cdot)$ replaced by $\overline{T_R}$ (corresponding to the $-\ri k$ in the radiation condition \cref{eq:src} being changed to $+\ri k$).
%
%Now, if $u$ is the solution of the adjoint problem with data $\LE(v)$, then $\overline{u}$ is the solution of the original problem with data $\overline{\LE(\overline{v})}$; 
%
%in particular if $\LE(v)$ is as in \cref{eq:EDPa}, then the $L^2$ data of the adjoint problem is just $\overline{f}$. Therefore, if the EDP satisfies \cref{cond:1nbpc,cond:2}, then so does its adjoint, and
% the bound in \cref{eq:keybound1} on $\|(\Amato)^{-1}\Mmat_n\|_{2}$ also holds for $\|((\Amato)^*)^{-1}\Mmat_n\|_{2}$.
\epf

The proof of \cref{lem:keylemma2} uses the following \lcnamecref{lem:H1}, which one can prove sing the G\aa rding inequality \cref{eq:gardingbrief}; see \cite[Lemma 5.1]{GrPeSp:19}.

\ble[Bound for data in $\HozDDRs$]\label{lem:H1}
%With the sesquilinear form $a(\cdot,\cdot)$ defined by \cref{eq:EDPa} with $A=\Aso$ and $n=\nso$, 
Given $\widetilde{\LE}\in \HozDDRs$, let $\widetilde{u}$ be the solution of the variational problem
\beqs
\text{ find } \,\,\widetilde{u} \in H^1_{0,D}(\DR) \,\,\tst \,\,
a^{(1)}(\widetilde{u},v)=\widetilde{\LE}(v) \,\, \tfa v\in H^1_{0,D}(\DR).
\eeqs
If \cref{cond:1nbpc} holds, then $\widetilde{u}$ exists, is unique, and satisfies the bound
\beq\label{eq:bound2}
\N{\widetilde{u}}_{\HokDR} \leq \frac{1}{\min\{\Asomin,\nsomin\}}\left( 1 + 2 C^{(1)}_{\rm bound}\nsomax  k\right) \big\|\widetilde{\LE}\big\|_{(\HokDR)'}
\eeq
for all $k\geq k_0$.
\ele
Observe that, similar to \cref{rem:yesitis}, \cref{eq:bound2} is a $k$-independent bound, due to the norm $\NHokDRs{\LE}$ on the right-hand side.


\bpf[Proof of \cref{lem:keylemma2}]
In a similar way to the proof of \cref{lem:keylemma1}, given $\bff \in \CC^N$ and a symmetric $A\in L^\infty(\DR, \RR^{d\times d})$, let $\widetilde{f} \de \sum_j f_j \phi_j$ and observe that $\widetilde{f} \in \HozDDR$. Define $\widetilde{u}$ to be the solution of the variational problem 
\beq\label{eq:411a}
a^{(1)}(\widetilde{u},v)= \widetilde{\LE}(v) \quad\text{ for all } v\in H^1(\Omega),
\quad\text{ where } \quad
 \widetilde{\LE}(v) \de(A\nabla\widetilde{f},\nabla v)_{L^2(\Omega)}.
\eeq
Observe that the definition of the norms $\|\cdot\|_{(\HokDR)'}$ \cref{eq:dualnorm} and $\|\cdot\|_{\HokDR}$ \cref{eq:weightednorm} and the Cauchy-Schwarz inequality imply that
\begin{align}
\big\| \widetilde{\LE}\big\|_{(\HokDR)'}&\leq \big\|A\nabla \widetilde{f}\big\|_{\LtDR}\nonumber\\
&\leq \NLiDRRRdtd{A} \big\|\nabla \widetilde{f}\big\|_{\LtDR}\label{eq:Fbounda}\\
&\leq \NLiDRRRdtd{A} \big\| \widetilde{f}\big\|_{\HokDR}.\label{eq:Fbound}
\end{align}
Let $\tu_h$ be the solution of the finite element approximation of \cref{eq:411a}, i.e.,
\beq\label{eq:41a}
a^{(1)}(\tu_h,v_h)= \widetilde{\LE}(v_h) \quad\text{ for all } v_h\in \Vhp,
\eeq
and let $\tbu$ be the vector of nodal values of $\tu_h$. The definition of $\widetilde{f}$ then implies that \cref{eq:41a} is equivalent to $\Amato \tbu = \Smat_A\,\bff$. 

Similar to the proof of \cref{lem:keylemma1},
using the triangle inequality, the bound \cref{eq:bound4} from \cref{cond:2}, the bound \cref{eq:bound2} from \cref{lem:H1}, the bound \cref{eq:Fbound}, and the definition of $C_1$ \cref{eq:C1nbpc},
we find
%Note that the hypotheses imply that the bound on the solution operator 
%\cref{eq:bound_unif} holds (by \cref{cor:uniform}), and also that if $h k\sqrt{|k^2-\eps|} \leq C_1$ then quasi-optimality \cref{eq:qoeps_lemma} holds (by \cref{lem:qo}).
%Starting with \cref{eq:equiv} we then have 
\begin{align}\nonumber 
%s_- h^{(d-2)/2} \N{\tbu}_2 &\leq \N{\nabla \tu_h}_{\LtDR}\leq  
\N{\tu_h}_{\HokDR} &\leq
\N{\tu-\tu_h}_{\HokDR} + \N{\tu}_{\HokDR},\nonumber \\ \nonumber
& \leq \left[ C^{(1)}_{\rm FEM2} k + 
\frac{1}{\min\{\Asomin,\nsomin\}}\left( 1 + 2 C^{(1)}_{\rm bound}\nsomax k  \right) 
\right]\big\|\widetilde{\LE}\big\|_{(\HokDR)'},\\
&\leq C_1 \, k\, 
%\left[C^{(1)}_{\rm FEM2} k + \frac{1}{\min\{\Asomin,\nsomin\}}\left( 1 + 2 C^{(1)}_{\rm bound}\nsomaxk  \right) \right]
\NLiDRRRdtd{A} \big\|\nabla\widetilde{f}\big\|_{\LtDR},\label{eq:mainevent2}\\
&\leq C_1 \, k\, 
%\left[C^{(1)}_{\rm FEM2} k + \frac{1}{\min\{\Asomin,\nsomin\}}\left( 1 + 2 C^{(1)}_{\rm bound}\nsomaxk  \right) \right]
\NLiDRRRdtd{A} \big\|\widetilde{f}\big\|_{\HokDR},\nonumber
%&\leq \left[ C^{(1)}_{\rm FEM2} k + 
%\frac{1}{\min\{\Asomin,\nsomin\}}\left( 1 + 2 C^{(1)}_{\rm bound}\nsomaxk  \right) 
%\right]\big\|A\big\|_{L^\infty(\DR)}s_+ h^{(d-2)/2} \N{\bff}_2,
\end{align}
and the bound on $\|(\Amato)^{-1}\Smat_A\|_{\Dmat_k}$ in \cref{eq:keybound2} follows.

The bound on $\|\Smat_A(\Amato)^{-1}\|_{(\Dmat_k)^{-1}}$ follows in a similar way to how we obtained the 
bound on  $\|(\Amato)^{-1}\Mmat_n\|_{(\Dmat_k)^{-1}}$ from the bound on $\|\Mmat_n(\Amato)^{-1}\|_{\Dmat_k}$ in Part (i). Indeed, 
\cref{eq:A380-0} and the fact that $\Smat_A$ is a real, symmetric matrix imply that 
\beq\label{eq:A380-2} 
 \big\|\Smat_A (\Amato)^{-1}\big\|_{(\Dmat_k)^{-1}}=\big\|\big((\Amato)^*\big)^{-1}\Smat_A\big\|_{\Dmat_k}
 \eeq 
%since 
%\beqs
%\big\|\Smat_A(\Amato)^{-1}\big\|_{2}=\big\|(\Smat_A(\Amato)^{-1})^*\big\|_{2}=\big\|((\Amato)^*)^{-1}\Smat_A\big\|_{2},
%\eeqs
(c.f. \cref{eq:A380}),
and then the arguments in the proof of part (i) imply that 
the bound in \cref{eq:keybound2} on $\|(\Amato)^{-1}\Smat_A\|_{\Dmat_k}$ also holds for $\|((\Amato)^*)^{-1}\Smat_A\|_{\Dmat_k}$.

To prove the bound on  $\|(\Amato)^{-1}\Smat_A\|_{2}$ in \cref{eq:keybound2a}, we use the bounds 
\beqs
m_- h^{d/2} k \N{\tbu}_2 \leq k \N{\widetilde{u}_h}_{\LtDR} \leq \N{\widetilde{u}_h}_{\HokDR}
\,\tand\,
\big\|\nabla \widetilde{f}\big\|_{\LtDR} \leq s_+ h^{d/2-1}\N{\bff}_2,
\eeqs
on either side of the inequality \cref{eq:mainevent2}, with these bounds coming from \cref{eq:normequiv1}.and \cref{eq:normequiv2} respectively. The proof of the bound on 
$\|\Smat_A((\Amato)^*)^{-1}\|_{2}$ in \cref{eq:keybound2a} follows in a similar way to above, using \cref{eq:Fbound}.
\epf

\bre[Link to the results of \cite{GaGrSp:15}]
A result analogous to the Euclidean-norm bounds in \cref{thm:1} was proved in \cite{GaGrSp:15} for the case that $\Aso= \Ast= I$, $\nst= 1$, and $\nso = 1 + \ri\eps/k^2$, with the `absorption parameter' or `shift' $\eps$ satisfying $0<\eps\lesssim k^2$. The motivation for proving this result was that the so-called `shifted Laplacian preconditioning' of the Helmholtz equation is based on preconditioning (with these choices of parameters) $\Amatt$ with an approximation of $\Amato$. Similar to \cref{cor:1}, bounds on $\|\Imat -  (\Amato)^{-1}\Amatt \|_2$ and 
$\|\Imat - \Amatt  (\Amato)^{-1}\|_2$
 then give upper bounds on large the `shift' $\eps$ can be for GMRES to converge in a $k$-independent number of iterations in the case when the action of $(\Amato)^{-1}$ is computed exactly.

%\cite[Lemma 4.1]{GaGrSp:15}
The main differences between \cite{GaGrSp:15} and the present paper are that (i)  \cite{GaGrSp:15} focused on the TEDP, not the EDP,
(ii) \cite{GaGrSp:15} focused on the particular case that $\Dm$ is star-shaped with respect to a ball, finding a $k$- and $\eps$-explicit expression for $C^{(1)}_{\rm bound}$ in this case using Morawetz identities, whereas we assume the existence of $\Cboundo,$
(iii) \cite{GaGrSp:15} required a bound on 
$(\Amato)^{-1}\Mmat_{n}$, analogous to the bounds in \cref{lem:keylemma1} along with one on $(\Amato)^{-1}\Nmat$ (in the case that $T_R$ is approximated by $\ri k$), but \emph{not} on 
$(\Amato)^{-1}\Smat_{A}$, and (iv) \cite{GaGrSp:15} only proved bounds in the $\|\cdot\|_2$ norm.
%The result of \cref{thm:1}
\ere

%\bre[Analogue of \cref{thm:1} in a weighted norm]\label{rem:weight1}
%The PDE analysis of the Helmholtz equation naturally takes place in the weighted $H^1$ norm $\|\cdot\|_{\HokDR}$ defined by \cref{eq:1knorm}. The discrete analogue of this norm is the norm $\|\cdot\|_{\Dmat_k}$ defined by 
%\beq\label{eq:Dk}
%\N{\bv}_{\Dmat_k}^2\de \big( (\Smat_I + k^2 \Mmat_1)\bv,\bv\big)_2 = \N{v_h}^2_{\HokDR}
%\eeq
%for
%$v_h =\sum_i v_i \phi_i$. 
%This norm is used, e.g., in recent results about the convergence of domain-decomposition methods %in this norm are proved 
%for the Helmholtz equation \cite{GrSpVa:17}, \cite{GrSpZo:18}, and for the time-harmonic Maxwell equations \cite{BoDoGrSpTo:19}. 
%
%Inspecting the proof of \cref{lem:keylemma}, we see that the bounds \cref{eq:keybound1} and \cref{eq:keybound2} hold with the $\|\cdot\|_2$ norm replaced by the $\|\cdot\|_{\Dmat_k}$ norm and without the terms involving $m_\pm$ and $s_\pm$ on the right-hand side. \cref{thm:1} 
%%(and also \cref{cor:1}) 
%therefore also holds with the $\|\cdot\|_2$ norm replaced by the $\|\cdot\|_{\Dmat_k}$ norm and the constant $C_1$ modified appropriately.
%\ere

\subsection{Proofs of \cref{cor:1,cor:1a}}\label{sec:mainproofs}

We first give the set-up for weighted GMRES, first introduced in by Essai in \cite{Es:98}, largely following \cite[Section 5]{GrSpVa:17}. Consider the abstract  linear system 
% \begin{equation*}
$\matrixC \bx = \bd$
%\end{equation*}
in $\mathbb{C}^N$, where $\matrixC \in \CC^{N\times N}$ is invertible. Let $\bxsz$ be the initial guess, and define the initial residual $\br^0 \de \bd- C \bx^0$ and the standard Krylov spaces:  
\beqs  
\cK^m(\Cmat, \br^0) \de \mathrm{span}\big\{\matrixC^j \br^0 : j = 0, \ldots, m-1\big\}.
\eeqs
Analagously to the definition of $\NDk{\cdot}$ above, let $(\cdot , \cdot )_{\Dmat}$ denote the inner product on $\CC^n$ 
induced by some Hermitian positive-definite matrix $\Dmat$, i.e.~
%\begin{equation*}
$(\bv,\bw)_{\Dmat} \de (\Dmat \bv, \bw)_2,$
%\end{equation*} 
and let $\Vert \cdot \Vert_\Dmat$ be the induced norm. For $m \geq 1$, define the $m$th GMRES iterate $\bx^m$  to be  the unique element of $\cK^m$ satisfying  the  
 minimal residual  property: 
$$ \ \Vert \br^m \Vert_\Dmat \de \Vert \bd - \matrixC \bx^m \Vert_\Dmat \ = \ \min_{\bx \in \cK^m(C, \br^0)} \Vert {\bd} - {\matrixC} {\bx} \Vert_\Dmat. $$
Observe that when $\Dmat = \Imat$ this is the standard GMRES algorithm. We also note that in general, weighted GMRES requires the use of weighted Arnoldi process, also introduced by Essaie in \cite{Es:98}, see also the alternative implementations of the weighted Arnoldi process in \cite{GuPe:14}.


Let 
\beq\label{eq:fov}
W_\Dmat(\matrixC)\de \Big\{ (\matrixC \bx, \bx)_{\Dmat} : \bx \in \CCN, \|\bx\|_\Dmat=1\Big\};
\eeq
$W_\Dmat(\matrixC)$ is called the \emph{numerical range} or \emph{field of values} of $\matrixC$ (in the $(\cdot,\cdot)_\Dmat$ inner product).

%Recall the so-called ``Elman estimate" for GMRES

\begin{theorem}[Elman estimate for weighted GMRES]\label{thm:GMRES1_intro} 
Let $\matrixC$ be a matrix with $\bzero\notin W(\matrixC)$. Let $\beta\in[0,\pi/2)$ be defined such that
\beq\label{eq:cosbeta}
\cos \beta \de \frac{\mathrm{dist}\big(\bzero, W(\matrixC)\big)}{\N{\matrixC}_{2}}.
\eeq
If the matrix equation $\matrixC \bx = \by$ is solved using weighted GMRES then, 
for $m\in \mathbb{N}$, the GMRES residual $\br_m$ %\de \matrixC \bx_m - \by$ 
satisfies
\beq\label{eq:Elman}
\frac{\N{\bfr_m}_{\Dmat}}{\N{\bfr_0}_{\Dmat}} \leq \sin^m \beta. %, \quad \text{ where}\quad 
\eeq
\end{theorem}
The bound \cref{eq:Elman} with $\Dmat=\Imat$ was first proved in \cite[Theorem 6.3]{El:82} (see also \cite[Theorem 3.3]{EiElSc:83}) and was written in the above form in \cite[Equation 1.2]{BeGoTy:06}. The bound \cref{eq:Elman} (for arbitrary Hermitian positive-definite $\Dmat$) was stated (without proof) in \cite{CaWi:92}\ednote{Euan---where is it stated in this paper? I can't see it.} and proved in \cite[Theorem 5.1]{GrSpVa:17}. % (see also \cite[Remark 5.2]{GrSpVa:17}). 



\cref{thm:GMRES1_intro} has the following corollary, and the proofs of \cref{cor:1,cor:1a} follow from combining this with \cref{thm:1}.

\begin{corollary}
\label{cor:GMRES_intro} 
If $\|\Imat - \matrixC \|_\Dmat \leq \alpha < 1$, then, with $\beta$ defined as in \cref{eq:cosbeta},
\beqs
\cos \beta \geq \frac{1-\alpha}{1+\alpha}\eeqs
and
\beq\label{eq:gmressin}
\sin \beta \leq \frac{2 \sqrt{\alpha}}{(1+\alpha)^2}.
\eeq
\end{corollary}

\bpf[Proof of \cref{cor:1}]
This follows from \cref{thm:1} by applying \cref{cor:GMRES_intro} first with $\matrixC= (\Amato)^{-1} \Amatt$, $\Dmat=\Dmat_k$, and $\alpha=1/2$, and then with $\matrixC= \Amatt(\Amato)^{-1} $, $\Dmat=(\Dmat_k)^{-1}$, and $\alpha=1/2$.
\epf

\

\bpf[Proof of \cref{cor:1a}]
This follows from \cref{thm:1} by applying \cref{cor:GMRES_intro} first with $\matrixC= (\Amato)^{-1} \Amatt$, $\Dmat=\Imat$, and $\alpha=1/2$, and then with $\matrixC= \Amatt(\Amato)^{-1} $, $\Dmat=\Imat$, and $\alpha=1/2$.
\epf


\bre[The improvement of the Elman estimate \cref{eq:Elman} in \cite{BeGoTy:06}]
A stronger result than \cref{eq:Elman} is given for standard (unweighted) GMRES in \cite[Theorem 2.1]{BeGoTy:06}, and then converted to a result about weighted GMRES in \cite[Theorem 5.3]{BoDoGrSpTo:19}; indeed, the convergence factor $\sin \beta$ is replaced by a function of $\beta$ strictly less than $\sin\beta$ for $\beta\in (0,\pi/2)$. Using this stronger result, however, does not improve the $k$-dependence of \cref{cor:1}.
\ere


%\section{Proof of }\label{sec:proofPDE}

\subsection{Proofs of the PDE results \cref{thm:2,lem:sharp}}

\bpf[Proof of \cref{thm:2}]
%We first prove the upper bound \cref{eq:PDEbound}.
By the assumptions on $\Aso,\, \Ast,\, \nso,\,\st$, and $\Dm$, $u^{(1)}$ and $u^{(2)}$ exist, are unique, satisfy $a^{(1)}(u^{(1)}, v) = \LE(v)$ for all $v \in \HozDDR$ and $a^{(2)}(u^{(2)}, v) = \LE(v)$ respectively, where $\LE$ is given by \cref{eq:Ledp}. Subtracting these equations, we find that $u^{(1)}- u^{(2)}$ satisfies the variational problem
\beq\label{eq:vp1}
a^{(1)}(u^{(1)}-u^{(2)},v) = \widetilde{\LE}(v) \quad\tfa v\in H^1_{0,D}(\DR)
\eeq
where
\beqs
 \widetilde{\LE}(v)\de \int_{\DR} \left((\Ast-\Aso) \nabla u^{(2)}\right) \cdot\overline{\nabla v} + k^2 (\nso-\nst) u^{(2)}\overline{v}.
\eeqs
Now, by the Cauchy-Schwarz inequality and the definition of the norm $\|\cdot\|_{\HokDR}$ (see \cref{eq:weightednorm}), we have
\begin{align*}
| \widetilde{\LE}(v)| &\leq \NLiDRRRdtd{\Aso-\Ast} \big\|\nabla u^{(2)}\big\|_{L^2(\DR)}
\N{\nabla v}_{L^2(\DR)} 
\\& \hspace{5cm}+ k^2 
\NLiDRRR{\nso-\nst} \big\| u^{(2)}\big\|_{L^2(\DR)}
\N{v}_{L^2(\DR)}\\
&\leq\max\Big\{\NLiDRRRdtd{\Aso-\Ast}\,,\, \NLiDRRR{\nso-\nst}\Big\}
\big\| u^{(2)}\big\|_{\HokDR} \N{v}_{\HokDR}
\end{align*}
(by Cauchy--Schwarz in $\RR^2$). Therefore, by the definition of the norm $\|\cdot\|_{(\HokDR)'}$ \cref{eq:dualnorm},
\beqs
\big\|\widetilde{\LE}\big\|_{(\HokDR)'}\leq \max\Big\{\big\|\Aso-\Ast\big\|_{L^\infty(\DR)}\,,\, \big\|\nso-\nst\big\|_{L^\infty(\DR)}\Big\}
\big\| u^{(2)}\big\|_{\HokDR}.
\eeqs
Since \cref{cond:1nbpc} holds, we can then apply \cref{lem:H1}, i.e.~the bound \cref{eq:bound2}, to the solution of the variational problem \cref{eq:vp1}  to find that 
\begin{align*}
\frac{\big\| u^{(1)} - u^{(1)}\big\|_{\HokDR}}
{\big\| u^{(2)}\big\|_{\HokDR}, 
}
 \leq 
\,&\frac{1}{\min\big\{\Asomin,\nsomin\big\}}\left( 1 + 2 C^{(1)}_{\rm bound}\nsomax  k\right)
\\
&\quad \left(\max\Big\{\big\|\Aso-\Ast\big\|_{L^\infty(\DR)}\,,\, \big\|\nso-\nst\big\|_{L^\infty(\DR)}\Big\}\right),
\end{align*}
and then the result \cref{eq:PDEbound} follows with 
\beq\label{eq:C3}
C_3\de \frac{1}{\min\big\{\Asomin,\nsomin\big\}}\left( \frac{1}{k_0} + 2 C^{(1)}_{\rm bound}\nsomax  \right).
\eeq
\epf

\bpf[Proof of \cref{lem:sharp}]
We actually prove the stronger result that given any function $c(k)$ such that $c(k)>0$ for all $k>0$, there exist 
$f, \nso,$ and $ \nst$ (with $\nso\not= \nst$) with
\beqs
\big\|\nso-\nst\big\|_{L^\infty(\DR)} \sim c(k)
\eeqs
such that the corresponding solutions $u^{(1)}$ and $u^{(2)}$ of \cref{prob:vedp} with $\Aso = \Ast= I$ exist, are unique, and satisfy \cref{eq:sharp1}. 

The heart of the proof is the equation
\beq\label{eq:obs1}
(\Delta + k^2) \big(\re^{\ri k r}\chi(r)\big) =  \re^{\ri k r}\left(\ri k \frac{d-1}{r} \chi(r) + \Delta \chi(r)\right)=: -\widetilde{f}(r),
\eeq
where $\chi(r)$ is chosen to have $\supp \chi \subset \DR$. Observe that \cref{eq:obs1} is the Helmholtz equation for a circular wave $e^{ikr}$, with the added factor $\chi$ to ensure the function has compact support. Given plane-wave and circular-wave solutions to the Helmholtz equation usually scale optimally in terms of $k$, we expect that \cref{eq:obs1} will be $k$ to proving that \cref{eq:PDEbound} is sharp. The equation \cref{eq:obs1} can be proved using the formula for the Laplacian in $d$-dimensional spherical coordinates\ednote{Euan---do you know a good `standard' reference for something like this? Currently the best I've found is the Wikipedia article on the Laplacian....} and observing that $e^{ikr} \chi(r)$ has angular derivatives equal to zero.

Observe that \cref{eq:obs1} proves the sharpness of the nontrapping resolvent estimate \cref{eq:bound1}, since both the $L^2(\DR)$ norm of $\widetilde{f}$ and the $\HokDR$ norm of $\re^{\ri kr}\chi(r)$ are proportional to $k$, and hence each to other (see, e.g., \cite[Lemma 3.10]{ChMo:08},  \cite[Lemma 4.12]{Sp:14}). Also, recall that the proof of the PDE bound \cref{eq:PDEbound} used the nontrapping resolvent estimate \cref{eq:bound1} applied to $\uso-\ust.$ Therefore we expect that if we set things up so that
\beq\label{eq:sharpdiff}
\uso-\ust = e^{ikr}\chi(r),
\eeq
because \cref{eq:obs1} shows the sharpness of the resolvent estimate \cref{eq:bound1} we expect that combining \cref{eq:obs1} and \cref{eq:sharpdiff} will show the sharpness of the PDE bound \cref{eq:PDEbound}.

We now set things up so that \cref{eq:sharpdiff} holds. We define $\nso = 1$ and $\nst = \nso + c(k) \chitilde(r),$ for some function $\chitilde(r)$ such that $\widetilde{\chi}\in C^{\infty}(\DR)$, $\widetilde{\chi}\not = 1$ (so that $\nst\not = \nso$), $\supp \, \widetilde{\chi} \subset\DR$, and $\NLiDRRR{\chitilde} = 1$ (so that $\NLiDRRR{\nso-\nst} = c(k)$).   As above, let $\chi=\chi(r)$ with $\chi \in C^{\infty}(\DR)$ and $\supp \,\chi$ a compact subset of $\DR$. We will specify $\chitilde$ and $\chi$ in more detail later. Let $\ftilde(r)$ be as defined in \cref{eq:obs1}, and define
\beq\label{eq:obs3}
u^{(2)}(\bx)\de -\frac{1}{k^2 c(k)}\frac{\widetilde{f}(r)}{\widetilde{\chi}(r)}
\eeq
and
\beqs
f(\bx)\de -\big(\Delta +k^2 \nst(\bx)\big) u^{(2)}(\bx).
\eeqs
We will define $\chi$ and $\chitilde$ below in such a way that if $\chitilde=0,$ then $\chi = 0$. This relation means we understand the right-hand side of \cref{eq:obs3} to be zero if $\chitilde$ is zero. In addition, since $\widetilde{\chi}(r)$ has compact support and $\ftilde$ depends on $\chi,$ we need to tie both the support of $\widetilde{\chi}$ and how fast $\widetilde{\chi}$ vanishes in a neighbourhood of its support to the definition of $\chi$ for both $u^{(2)}$ and $f$ to be well defined. As the final part of the setup, let $\uso$ solve
\beqs
\mleft(\Delta + k^2\mright) \uso = -f,
\eeqs
i.e., \cref{prob:vedp} with coefficients $\Aso = I$ and $\nso =1$ and right-hand side $f$.

Now observe that by construction

\begin{align*}
  \mleft(\Delta + k^2\mright) \mleft(\uso-\ust\mright) &= -k^2 \mleft(1-\nst\mright)\ust\\
  &= -k^2 c(k) \chitilde \frac{-1}{k^2 c(k)} \frac{\ftilde}{\chitilde}\\
  &= \ftilde\\
  &= \mleft(\Delta + k^2 \mright)\mleft(e^{-ikr}\chi(r)\mright).
\end{align*}
Therefore, by uniqueness of the solution of \cref{prob:vedp} \beq\label{eq:obs4}
u^{(1)}(\bx)- u^{(2)}(\bx) = \re^{\ri k r}\chi(r),
\eeq
and therefore \cref{eq:obs4} we have
\beqs
\big\|u^{(1)}-u^{(2)}\big\|_{L^2(\DR)} \sim 1
\quad \tand \quad
\big\|u^{(1)}-u^{(2)}\big\|_{\HokDR} \sim k.
\eeqs
Furthermore, the definitions of $u^{(2)}$ and $\widetilde{f}$ imply that
\beqs
\big\| u^{(2)}\big\|_{L^2(\DR)} \sim \frac{1}{k\, c(k)} \quad\tand \quad 
\big\| u^{(2)}\big\|_{\HokDR} \sim \frac{1}{c(k)},
\eeqs 
and, since $\|\nso- \nst\|_{L^\infty(\DR)} = c(k)$, \cref{eq:sharp1} holds, as required.

Therefore, to complete the proof, we only need to show that there exists a choice of $\chi$ and $\widetilde{\chi}$ for which $u^{(2)}$ and $f$ defined by \cref{eq:obs3} are 
in $H^{1}(\DR)$ and $\LtDR$ respectively (in fact, we prove that they are in $W^{1,\infty}(\DR)$ and $L^\infty(\DR)$ respectively).
%well-defined. 
Because $\chi$ and $\widetilde{\chi}$ are in $C^\infty(\DR)$, the only issue is what happens at the boundary of the support of $\widetilde{\chi}$, where $u^{(2)}$ has the potential to be singular.
Since $\clos{\Dm} \subset \BR$, there exist $0<R_1<R_2<R$ such that $\overline{\Dm} \subset B_{R_2}\setminus B_{R_1} \subset \BR$. Let $\supp \chi = B_{R_2}\setminus B_{R_1}$ and let $\chi$ vanish to order $m$ at $r= R_1$ and $r=R_2$; i.e.~$\chi(r) \sim (r-R_1)^m$ as $r \downarrow R_1$ and 
$\chi(r) \sim (R_2-r)^m$ as $r \uparrow R_2$. The definition of $\widetilde{f}$ \cref{eq:obs3} then implies that $\widetilde{f}$ vanishes to order $m-2$. Let $\widetilde{\chi}(r)$ vanish to order $\mtilde$ at $r= R_1$ and $r=R_2$. 
We now claim that if $m >\mtilde+4$, then $u^{(2)}\in W^{1,\infty}(\DR)$ and $f$ $\in L^\infty(\DR)$. Indeed,  
straightforward calculation from \cref{eq:obs3} shows that  $u^{(2)}(r) \sim (r-R_1)^{m-\mtilde-2}$, $\nabla u^{(2)}(r) \sim (r-R_1)^{m-\mtilde-3}$, and $\Delta u^{(2)}(r) \sim (r-R_1)^{m-\mtilde-4}$ as $r \downarrow R_1$, with analogous behaviour at $r=R_2$.
The assumption 
$m >\mtilde+4$ therefore implies that $u^{(2)}$, $\nabla u ^{(2)}$, and $\Delta u^{(2)}$ vanish (and hence are finite) at $r=R_1$ and $r=R_2$.
\epf

\bre[Why doesn't \cref{lem:sharp} cover the case $\Aso\neq  \Ast$?]
When $\nj\de1$, $j=1,2,$, $\Aso\de I$, and $\Ast\de I + c(k)\widetilde{\chi}$, the variational problem \cref{eq:vp1} implies that 
\beqs%\label{eq:obs2}
\Delta \big( u^{(1)} - u^{(2)}\big) + k^2 \big( u^{(1)} - u^{(2)}\big) = c(k)\nabla\cdot \big(\widetilde{\chi}\nabla u^{(2)}\big).
\eeqs
It is now much harder than in \cref{eq:obs2} to set things up so that $ u^{(1)}(\bx) - u^{(2)}(\bx)=\re^{\ri kr}\chi(r)$ (so that one can then use \cref{eq:obs1}).
\ere

%\section{Proof of \cref{lem:2}}

\section{Extension of nearby preconditioning results to weaker norms}\label{sec:weaknorm}
Recall from \cref{sec:num,sec:main} that GMRES applied to $\AmatoI \Amatt$ converges in a $k$-independent number of iterations if $\NLiDRRR{\nso-\nst} \lesssim 1/k$ (with an analagous result for $\Aso-\Ast$). This result (and the related numerics) shows that $1/k$ is a sharp threshold when we consider the \emph{magnitude}  of the difference between $\nso$ and $\nst$. However, this result does not say anything about the \emph{spatial} variability between $\nso$ and $\nst$. For example if $\nso$ and $\nst$ (defined on the unit square) are given by
\beq\label{eq:noweak}
\nso(\bx) =
\begin{dcases}
  1 &\tif \bx_1 \leq \half\\
  2  &\tif \bx_1 > \half
  \end{dcases}
\eeq
and
\beq\label{eq:ntweak}
\nst(\bx) =
\begin{dcases}
  1 &\tif \bx_1 \leq \half+\alpha\\
  2  &\tif \bx_1 > \half+\alpha
  \end{dcases}
\eeq
for some $0 < \alpha < 1/2,$ then $\NLiDRRR{\nso-\nst} = 1$ for all $\alpha$, but one would expect that for small $\alpha$ the corresponding solutions of \cref{prob:edp} satisfy $\uso \approx \ust.$ In addition, one might expect that GMRES applied to $\AmatoI\Amatt$ would converge in a $k$-independent number of iterations. Therefore, in this \lcnamecref{sec:weaknorm} we seek to obtain analogues of \cref{thm:1,cor:1,cor:1a} with the difference in $\nso-\nst$ and $\Aso-\Ast$ measured in weaker norms than the $L^\infty$ norm.

The (realistic) best-case result we could obtain would be that GMRES applied to $\AmatoI\Amatt$ converges in a $k$-independent number of iterations if $\NLoDRRR{\nso-\nst} \lesssim 1/k$. This result is `best' in the sense that it depends optimally on $k$; recall the discussion in \cref{rem:physical1k} that the length scale $1/k$ is the length scale governing the behaviour of Helmholtz problems. It is also `best' with regards to the norm used to measure $\nso-\nst$. When we measure $\nso-\nst$ in the $L^\infty$ norm as above, we are able to control the magnitude of $\nso-\nst$, but not the spatial variability; if $\nso-\nst \neq 0$ only on a set of small (but nonzero) measure, and $\nso-\nst=1$ on this small set, then $\NLiDRRR{\nso-\nst} = 1$, regardless of the measure of the set. In contrast, the $L^1$ norm allows us to control both the magnitude of $\nso-\nst$ and the measure of the sets on which it is nonzero.

We will give numerical results indicating that this best-case result is sharp (our results actually indicate that we obtain $k$-independent convergence if $\NLqDRRR{\nso-\nst}\sim 1/k$ for any $q>0$). We will also provide theory results that are, to our knowledge, the best one can prove, although they are sub-optimal in both $q$ and the dependence on $k.$


\subsection{Theory in weaker norms}\label{sec:weakertheory}
Before we prove results analogous to \cref{thm:1} in weaker norms (so that we can conclude results analogous to \cref{cor:1,cor:1a} in weaker norms), we first recap why the terms  $\NLiDRRRdtd{\Aso-\Ast}$ and $\NLiDRRR{\nso-\nst}$ appear in \cref{thm:1}. These terms appear in cref{thm:1} because the terms $\NLiDRRR{n}$ and $\NLiDRRRdtd{A}$ appear in \cref{lem:keylemma1,lem:keylemma2}, respectively. These terms appear because in \cref{eq:mainevent1a,eq:Fbounda} we use the bounds
\beq\label{eq:keynbound}
\NLtDR{n\ftilde} \leq \NLiDRRR{n}\NLtDR{\ftilde}
\eeq
and
\beq\label{eq:keyAbound}
\NLtDR{A \grad \ftilde} \leq \NLiDRRRdtd{A}\NLtDR{\grad \ftilde}
\eeq
respectively, for some function $\ftilde,$ and these bounds are carried through the rest of the proof.

However, we observe that we have the following generalisation of H\"older's inequality: If $q,s > 2$ such that $1/2 = 1/q+1/s,$ then
\beq\label{eq:genholder}
\NLtDR{\vo\vt} \leq \NLqDR{\vo}\NLsDR{\vt}.
\eeq

If we instead use \cref{eq:genholder} to bound \cref{eq:keynbound,eq:keyAbound} we instead obtain
\beq\label{eq:keynbound2}
\NLtDR{n\ftilde} \leq \NLqDRRR{n}\NLsDR{\ftilde}
\eeq
and
\beq\label{eq:keyAbound2}
\NLtDR{A\grad\ftilde} \leq \NLqDRRRdtd{A}\NLsDR{\ftilde}.
\eeq

As $\ftilde \in \Vhp$, we can apply an inverse inequality to bound $\NLsDR{\ftilde}$ by $\NLtDR{\ftilde}$. The required inverse inequality is (see \cite[Theorem 4.5.11 and Remark 4.5.20]{BrSc:08}
\beq\label{eq:inverses}
\NLsDR{\ftilde} \leq \Cinvs h^{d\mleft(\frac1{s} - \half\mright)} \NLtDR{\ftilde}.
\eeq
If we then apply \cref{eq:inverses} to \cref{eq:keynbound2,eq:keyAbound2} we obtain
\beq\label{eq:keynboundfinal}
\NLtDR{n\ftilde} \leq \Cinvs \NLqDRRR{n} h^{d\mleft(\frac1{s} - \half\mright)} \NLtDR{\ftilde}
\eeq
and
\beq\label{eq:keyAboundfinal}
\NLtDR{A\grad\ftilde} \leq \Cinvs \NLqDRRRdtd{A} h^{d\mleft(\frac1{s} - \half\mright)} \NLtDR{\grad\ftilde}.
\eeq

Replacing \cref{eq:mainevent1a,eq:Fbounda} with \cref{eq:keynboundfinal,eq:keyAboundfinal}, as in the proofs of \cref{lem:keylemma1,lem:keylemma2}, and proceeding as in those proofs, we obtain \cref{lem:keylemma1a,lem:keylemma2a} below, the analogues of \cref{lem:keylemma1,lem:keylemma2}.

\ble[Alternative bounds on $(\Amato)^{-1} \Mmat_{n}$]\label{lem:keylemma1a}
Under the assumptions of \cref{lem:keylemma1}, for $n\in \LiDRRR$ and for any $s,q > 2$ such that $1/s + 1/q = \half$,
\beq\label{eq:keybound12}
\max\set{\NDk{\AmatoI \Mmatn},\NDkI{\Mmatn\AmatoI}} \leq \Cttilde h^{-\frac{d}{q}} \frac{\NLqDRRR{n}}k
\eeq
and 
\beq\label{eq:keybound1a2}
\max\set{\Nt{\AmatoI \Mmatn},\Nt{\Mmatn\AmatoI}} \leq \Cttilde\mleft(\frac{\mplus}{\mminus}\mright) h^{-\frac{d}q} \frac{\NLqDRRR{n}}k
\eeq
for all $k\geq \kz$,
where
\beq\label{eq:C2tilde}
\Cttilde\de%\frac{m_+}{m_-} 
%\left[ 
\Cinvs\Ct,
\eeq
where $\Ct$ is defined by \cref{eq:C2} and $1/s = 1/2 - 1/q.$
\ele

\ble[Alternative bounds on $(\Amato)^{-1} \Smat_A$]\label{lem:keylemma2a}
Under the assumptions of \cref{lem:keylemma2}, for $A\in L^\infty(\DR,\RR^{d\times d})$ and for any $s,q > 2$ such that $1/s + 1/q = \half$,
\beq\label{eq:keybound22}
\max\set{\NDk{\AmatoI \SmatA},\NDkI{\SmatA\AmatoI}} \leq \Cotilde h^{-\frac{d}q}k \NLqDRRRdtd{A}
\eeq
and
\beq\label{eq:keybound2a2}
\max\set{\Nt{\AmatoI \SmatA},\Nt{\SmatA\AmatoI}} \leq \Cotilde\mleft(\frac{\splus}{\mminus}\mright) h^{-\frac{d}q-1} \NLqDRRR{A}
\eeq
%\begin{align}\nonumber
%&\max\Big\{\big\| (\Amato)^{-1} \Smat_A \big\|_2, \,\,
%\big\| \Smat_A (\Amato)^{-1} \big\|_2\Big\}\nonumber \\
%&\hspace{2cm}
% \leq \frac{s_+}{s_-} \left[ C_{\rm FEM2}^{(1)} + 
% \frac{1}{\min\big\{\Asomin,\nsomin\big\}}\left( \frac{1}{k_0} + 2 C^{(1)}_{\rm bound}\nsomax  \right) \right]k\N{A}_{L^\infty(\DR)}\label{eq:keybound2}
%% + C_{\rm bound}^{(1)}\right) \frac{\N{n}_{L^\infty(\DR)}}{k}.
%\end{align}
for all $k\geq k_0$, where
\beq\label{eq:C1tildenbpc}
\Cotilde \de \Cinvs\Co,
\eeq
where $\Co$ is given by \cref{eq:C1nbpc} and $1/s = 1/2 - 1/q.$
\ele

We can use \cref{lem:keylemma1a,lem:keylemma2a} in palce of \cref{lem:keylemma1,lem:keylemma2} to obtain the following analogue of \cref{thm:1} in weaker norms.

\begin{theorem}[Alternative main ingredient to answer to \cref{it:nbpcq1}]\label{thm:1alt}
Let the assumptions of \cref{thm:1} hold.   Then, given $\kz>0$ and $q >2$, there exist $\Cotilde, \Cttilde>0$, independent of $h$ and $k$ (but dependent on $\Dm, \Aso, \nso$, $p$, $q$, and $\kz$) such that
\begin{align}\nonumber
&\max\set{\NDk{\Imat - \AmatoI\Amatt},\NDkI{\Imat -\Amatt\AmatoI}}\\
&\hspace{3cm} 
\leq \Cotilde kh^{-\frac{s}q} \NLqDRRRdtd{\Aso-\Ast} + \Cttilde  kh^{-\frac{s}q}  \NLqDRRR{\nso-\nst}
\label{eq:main1alt}
\end{align}
and 
\begin{align}\nonumber
&\max\set{\Nt{\Imat - \AmatoI\Amatt}, \Nt{\Imat -\Amatt\AmatoI}}\\
&\hspace{0cm}
\leq \Cotilde \mleft(\frac{\splus}{\mminus}\mright) h^{-\frac{s}q-1}\NLqDRRRdtd{\Aso-\Ast} + \Cttilde \mleft(\frac{\mplus}{\mminus}\mright) kh^{-\frac{s}q}\NLqDRRR{\nso-\nst}
\label{eq:main1aalt}
\end{align}
for all $k\geq k_0$. 
\end{theorem}

We can now use \cref{thm:1alt} to obtain the following analogues to \cref{cor:1,cor:1a} in weaker norms.

\bth[Alternative answer to \cref{it:nbpcq1}: $k$-independent weighted GMRES iterations]\label{cor:1alt}
Let the assumptions of \cref{cor:1a} hold.  Given $k_0>0$ and $q >2$, let $\Cotilde$ and $\Cttilde$ be as in \cref{thm:1alt}. Then if 
% there exists $C_2>0$, independent of $h$ and $k$ (but dependent on $\Dm, \Aso, \nso$, $p$, and $k_0$) and given explicitly in \cref{eq:C2} below,
% such that if 
\beq\label{eq:condalt}
\Cotilde kh^{-\frac{d}{q}} \NLqDRRRdtd{\Aso-\Ast} +\Cttilde  kh^{-\frac{d}{q}} \NLqDRRR{\nso-\nst}
\leq \frac{1}{2}
\eeq
for all $k\geq k_0$, then \emph{both} weighted GMRES working in $\|\cdot\|_{\Dmat_k}$ (and the associated inner product) applied to \cref{eq:pcsystem1} \emph{and} weighted GMRES working in $\|\cdot\|_{(\Dmat_k)^{-1}}$ (and the associated inner product) applied to \cref{eq:pcsystem2}  converge in a $k$-independent number of iterations for all $k\geq k_0$.
\enth

\bth[Alternative answer to \cref{it:nbpcq1}: $k$-independent (unweighted) GMRES iterations]\label{cor:1aalt}
Let the assumptions of \cref{cor:1a} hold.  Given $k_0>0$, and $q >2$, let $\Cotilde$ and $\Cttilde$ be as in \cref{thm:1alt}. Then if 
% there exists $C_2>0$, independent of $h$ and $k$ (but dependent on $\Dm, \Aso, \nso$, $p$, and $k_0$) and given explicitly in \cref{eq:C2} below,
% such that if 
\beq\label{eq:condaalt}
\Cotilde \mleft(\frac{\splus}{\mminus}\mright) h^{-\frac{d}{q}-1} \NLqDRRRdtd{\Aso-\Ast} + \Cttilde \mleft(\frac{\mplus}{\mminus}\mright) kh^{-\frac{d}{q}} \NLqDRRR{\nso-\nst} \leq \half
\eeq
for all $k\geq k_0$, then standard GMRES (working in the Euclidean norm and inner product) applied to either of the equations \cref{eq:pcsystem1} or \cref{eq:pcsystem2}
%\beqs
%(\Amat^{(1)})^{-1}\Amat^{(2)}\bu = \bff\quad\text{ or } \quad\Amat^{(2)}(\Amat^{(1)})^{-1}\bv = \bff
%\eeqs
 converges in a $k$-independent number of iterations for all $k\geq k_0$.
\enth

\bre[Trade of between norm and restriction on the norm]
Observe that in \cref{cor:1alt,cor:1aalt} there is a trade-off between the norm that one uses to measure $\no-\nt$ and the restriction on the magnitude of this norm. E.g., the condition on $\no-\nt$ in both \cref{cor:1alt,cor:1aalt} can be summarised as
\beq\label{eq:altsufficientlysmall}
\NLqDRRR{\no-\nt} k h^{-\frac{d}{q}} \text{ is sufficiently small}.
\eeq
Observe that as $q \downarrow 2,$ we measure $\no-\nt$ in a weaker norm, but the condition \cref{eq:altsufficientlysmall} becomes more restrictive; the power of $h$ increases. Conversely, as $q \uparrow \infty,$ we measure $\no-\nt$ in a stronger norm, but the condition \cref{eq:altsufficientlysmall} becomes less restrictive; the power of $h$ decreases. (Also observe that in the $q\uparrow\infty$ limit we recover the condition \cref{eq:sufficientlysmall} we previously proved for $\NLiDRRR{\no-\nt}.$
\ere

The numerics in \cref{sec:weakernumerics} below suggest that in certain cases, a sufficient condition for nearby preconditioning to be effective is
\beq\label{eq:experimentalsufficientlysmall}
\NLqDRRR{\no-\nt} k \quad\text{is sufficiently small},
\eeq
for \emph{any} $q \geq 1$, and moreover \cref{eq:experimentalsufficientlysmall} may be sharp in its $k$-dependence. (This requirement would fit with our previous observation about $1/k$ being the length scale below which perturbations cannot be seen---see \cref{rem:physical1k} above.) However, we do not say that \cref{eq:experimentalsufficientlysmall} is sufficient for all cases; recall that for transmission problems, very small perturbations in $n$ can lead to very different behaviour in the solution $u$ if $k$ is a quasi-resonance for $\no$ or $\nt$; see the discussion at the end of \cref{sec:wpdisc} above.


\subsection{Numerics in weaker norms}\label{sec:weakernumerics}
For our computations, we use the computational setup as in \cref{app:compsetup}, with $f$ and $\gI$ corresponding to a plane wave passing through homogeneous media. We emphasise that whilst the preceeding theory is for \cref{prob:vedp} and our computations are for \cref{prob:vtedp}, we expect one could prove analagous results to those above for \cref{probvtedp}, see \cref{sec:TEDP} below for an outline of the major steps needed to extend the theory to \cref{prob:vtedp}. We let $\Aso=\Ast=I,$ and we define $\nso$ and $\nst$ by \cref{eq:noweak,eq:ntweak}. For $\alpha = 0.2,0.2/k^{0.1},0.2/k^{0.2},\ldots,0.2/k$ and for $k=10,20,\ldots,100$ we used GMRES to solve $\AmatoI\Amatt = \AmatoI \bff$ (for $\bff$ given by the Helmholtz problem), and we record the number of GMRES iterations taken to achieve convergence.

Our results in \cref{fig:l1low,fig:l1med,fig:l1high} (also displayed in \cref{tab:l1}) indicate the following conclusions for $\NLqDRR{\nso-\nst} \sim 0.1/k^{\beta}$ :
\bit
\item For $\beta \in (0,0.6)$ there is clear growth of the number of GMRES iterations with $k$,
\item For $\beta = 1$ there is clear boundedness of the number of GMRES iterations with $k$, and
  \item for $\beta \in (0.7,0.9)$ it is unclear if the number of GMRES iterations grows with $k.$
\eit

If we compare our numerical results with the theory results in \cref{cor:1aalt}, we see that the theory (if $h \sim k^{-3/2}$ and $d=2$, as in our computational experiments) predicts that the number of iterations will remain bounded if $\NLqDRR{\nso-\nst} k^{1+3/q}$ is sufficiently small, for any $q > 2.$ Our computed results indicate that this result is not sharp. The computed results indicate that if $\NLqDRR{\nso-\nst} \sim k^{-1}$ for any $q \geq 1,$ then the number of GMRES iterations is bounded as $k$ increases. Observe again that the `best case' $1/k$ condition is only predicted by the theory in the $q\rightarrow \infty$ limit.

\begin{figure}
%% Creator: Matplotlib, PGF backend
%%
%% To include the figure in your LaTeX document, write
%%   \input{<filename>.pgf}
%%
%% Make sure the required packages are loaded in your preamble
%%   \usepackage{pgf}
%%
%% Figures using additional raster images can only be included by \input if
%% they are in the same directory as the main LaTeX file. For loading figures
%% from other directories you can use the `import` package
%%   \usepackage{import}
%% and then include the figures with
%%   \import{<path to file>}{<filename>.pgf}
%%
%% Matplotlib used the following preamble
%%   \usepackage{fontspec}
%%   \setmainfont{DejaVuSerif.ttf}[Path=/home/owen/progs/firedrake-complex/firedrake/lib/python3.5/site-packages/matplotlib/mpl-data/fonts/ttf/]
%%   \setsansfont{DejaVuSans.ttf}[Path=/home/owen/progs/firedrake-complex/firedrake/lib/python3.5/site-packages/matplotlib/mpl-data/fonts/ttf/]
%%   \setmonofont{DejaVuSansMono.ttf}[Path=/home/owen/progs/firedrake-complex/firedrake/lib/python3.5/site-packages/matplotlib/mpl-data/fonts/ttf/]
%%
\begingroup%
\makeatletter%
\begin{pgfpicture}%
\pgfpathrectangle{\pgfpointorigin}{\pgfqpoint{6.400000in}{4.800000in}}%
\pgfusepath{use as bounding box, clip}%
\begin{pgfscope}%
\pgfsetbuttcap%
\pgfsetmiterjoin%
\definecolor{currentfill}{rgb}{1.000000,1.000000,1.000000}%
\pgfsetfillcolor{currentfill}%
\pgfsetlinewidth{0.000000pt}%
\definecolor{currentstroke}{rgb}{1.000000,1.000000,1.000000}%
\pgfsetstrokecolor{currentstroke}%
\pgfsetdash{}{0pt}%
\pgfpathmoveto{\pgfqpoint{0.000000in}{0.000000in}}%
\pgfpathlineto{\pgfqpoint{6.400000in}{0.000000in}}%
\pgfpathlineto{\pgfqpoint{6.400000in}{4.800000in}}%
\pgfpathlineto{\pgfqpoint{0.000000in}{4.800000in}}%
\pgfpathclose%
\pgfusepath{fill}%
\end{pgfscope}%
\begin{pgfscope}%
\pgfsetbuttcap%
\pgfsetmiterjoin%
\definecolor{currentfill}{rgb}{1.000000,1.000000,1.000000}%
\pgfsetfillcolor{currentfill}%
\pgfsetlinewidth{0.000000pt}%
\definecolor{currentstroke}{rgb}{0.000000,0.000000,0.000000}%
\pgfsetstrokecolor{currentstroke}%
\pgfsetstrokeopacity{0.000000}%
\pgfsetdash{}{0pt}%
\pgfpathmoveto{\pgfqpoint{0.800000in}{0.528000in}}%
\pgfpathlineto{\pgfqpoint{5.760000in}{0.528000in}}%
\pgfpathlineto{\pgfqpoint{5.760000in}{4.224000in}}%
\pgfpathlineto{\pgfqpoint{0.800000in}{4.224000in}}%
\pgfpathclose%
\pgfusepath{fill}%
\end{pgfscope}%
\begin{pgfscope}%
\pgfsetbuttcap%
\pgfsetroundjoin%
\definecolor{currentfill}{rgb}{0.000000,0.000000,0.000000}%
\pgfsetfillcolor{currentfill}%
\pgfsetlinewidth{0.803000pt}%
\definecolor{currentstroke}{rgb}{0.000000,0.000000,0.000000}%
\pgfsetstrokecolor{currentstroke}%
\pgfsetdash{}{0pt}%
\pgfsys@defobject{currentmarker}{\pgfqpoint{0.000000in}{-0.048611in}}{\pgfqpoint{0.000000in}{0.000000in}}{%
\pgfpathmoveto{\pgfqpoint{0.000000in}{0.000000in}}%
\pgfpathlineto{\pgfqpoint{0.000000in}{-0.048611in}}%
\pgfusepath{stroke,fill}%
}%
\begin{pgfscope}%
\pgfsys@transformshift{1.250909in}{0.528000in}%
\pgfsys@useobject{currentmarker}{}%
\end{pgfscope}%
\end{pgfscope}%
\begin{pgfscope}%
\definecolor{textcolor}{rgb}{0.000000,0.000000,0.000000}%
\pgfsetstrokecolor{textcolor}%
\pgfsetfillcolor{textcolor}%
\pgftext[x=1.250909in,y=0.430778in,,top]{\color{textcolor}\sffamily\fontsize{10.000000}{12.000000}\selectfont \(\displaystyle 10\)}%
\end{pgfscope}%
\begin{pgfscope}%
\pgfsetbuttcap%
\pgfsetroundjoin%
\definecolor{currentfill}{rgb}{0.000000,0.000000,0.000000}%
\pgfsetfillcolor{currentfill}%
\pgfsetlinewidth{0.803000pt}%
\definecolor{currentstroke}{rgb}{0.000000,0.000000,0.000000}%
\pgfsetstrokecolor{currentstroke}%
\pgfsetdash{}{0pt}%
\pgfsys@defobject{currentmarker}{\pgfqpoint{0.000000in}{-0.048611in}}{\pgfqpoint{0.000000in}{0.000000in}}{%
\pgfpathmoveto{\pgfqpoint{0.000000in}{0.000000in}}%
\pgfpathlineto{\pgfqpoint{0.000000in}{-0.048611in}}%
\pgfusepath{stroke,fill}%
}%
\begin{pgfscope}%
\pgfsys@transformshift{1.701818in}{0.528000in}%
\pgfsys@useobject{currentmarker}{}%
\end{pgfscope}%
\end{pgfscope}%
\begin{pgfscope}%
\definecolor{textcolor}{rgb}{0.000000,0.000000,0.000000}%
\pgfsetstrokecolor{textcolor}%
\pgfsetfillcolor{textcolor}%
\pgftext[x=1.701818in,y=0.430778in,,top]{\color{textcolor}\sffamily\fontsize{10.000000}{12.000000}\selectfont \(\displaystyle 20\)}%
\end{pgfscope}%
\begin{pgfscope}%
\pgfsetbuttcap%
\pgfsetroundjoin%
\definecolor{currentfill}{rgb}{0.000000,0.000000,0.000000}%
\pgfsetfillcolor{currentfill}%
\pgfsetlinewidth{0.803000pt}%
\definecolor{currentstroke}{rgb}{0.000000,0.000000,0.000000}%
\pgfsetstrokecolor{currentstroke}%
\pgfsetdash{}{0pt}%
\pgfsys@defobject{currentmarker}{\pgfqpoint{0.000000in}{-0.048611in}}{\pgfqpoint{0.000000in}{0.000000in}}{%
\pgfpathmoveto{\pgfqpoint{0.000000in}{0.000000in}}%
\pgfpathlineto{\pgfqpoint{0.000000in}{-0.048611in}}%
\pgfusepath{stroke,fill}%
}%
\begin{pgfscope}%
\pgfsys@transformshift{2.152727in}{0.528000in}%
\pgfsys@useobject{currentmarker}{}%
\end{pgfscope}%
\end{pgfscope}%
\begin{pgfscope}%
\definecolor{textcolor}{rgb}{0.000000,0.000000,0.000000}%
\pgfsetstrokecolor{textcolor}%
\pgfsetfillcolor{textcolor}%
\pgftext[x=2.152727in,y=0.430778in,,top]{\color{textcolor}\sffamily\fontsize{10.000000}{12.000000}\selectfont \(\displaystyle 30\)}%
\end{pgfscope}%
\begin{pgfscope}%
\pgfsetbuttcap%
\pgfsetroundjoin%
\definecolor{currentfill}{rgb}{0.000000,0.000000,0.000000}%
\pgfsetfillcolor{currentfill}%
\pgfsetlinewidth{0.803000pt}%
\definecolor{currentstroke}{rgb}{0.000000,0.000000,0.000000}%
\pgfsetstrokecolor{currentstroke}%
\pgfsetdash{}{0pt}%
\pgfsys@defobject{currentmarker}{\pgfqpoint{0.000000in}{-0.048611in}}{\pgfqpoint{0.000000in}{0.000000in}}{%
\pgfpathmoveto{\pgfqpoint{0.000000in}{0.000000in}}%
\pgfpathlineto{\pgfqpoint{0.000000in}{-0.048611in}}%
\pgfusepath{stroke,fill}%
}%
\begin{pgfscope}%
\pgfsys@transformshift{2.603636in}{0.528000in}%
\pgfsys@useobject{currentmarker}{}%
\end{pgfscope}%
\end{pgfscope}%
\begin{pgfscope}%
\definecolor{textcolor}{rgb}{0.000000,0.000000,0.000000}%
\pgfsetstrokecolor{textcolor}%
\pgfsetfillcolor{textcolor}%
\pgftext[x=2.603636in,y=0.430778in,,top]{\color{textcolor}\sffamily\fontsize{10.000000}{12.000000}\selectfont \(\displaystyle 40\)}%
\end{pgfscope}%
\begin{pgfscope}%
\pgfsetbuttcap%
\pgfsetroundjoin%
\definecolor{currentfill}{rgb}{0.000000,0.000000,0.000000}%
\pgfsetfillcolor{currentfill}%
\pgfsetlinewidth{0.803000pt}%
\definecolor{currentstroke}{rgb}{0.000000,0.000000,0.000000}%
\pgfsetstrokecolor{currentstroke}%
\pgfsetdash{}{0pt}%
\pgfsys@defobject{currentmarker}{\pgfqpoint{0.000000in}{-0.048611in}}{\pgfqpoint{0.000000in}{0.000000in}}{%
\pgfpathmoveto{\pgfqpoint{0.000000in}{0.000000in}}%
\pgfpathlineto{\pgfqpoint{0.000000in}{-0.048611in}}%
\pgfusepath{stroke,fill}%
}%
\begin{pgfscope}%
\pgfsys@transformshift{3.054545in}{0.528000in}%
\pgfsys@useobject{currentmarker}{}%
\end{pgfscope}%
\end{pgfscope}%
\begin{pgfscope}%
\definecolor{textcolor}{rgb}{0.000000,0.000000,0.000000}%
\pgfsetstrokecolor{textcolor}%
\pgfsetfillcolor{textcolor}%
\pgftext[x=3.054545in,y=0.430778in,,top]{\color{textcolor}\sffamily\fontsize{10.000000}{12.000000}\selectfont \(\displaystyle 50\)}%
\end{pgfscope}%
\begin{pgfscope}%
\pgfsetbuttcap%
\pgfsetroundjoin%
\definecolor{currentfill}{rgb}{0.000000,0.000000,0.000000}%
\pgfsetfillcolor{currentfill}%
\pgfsetlinewidth{0.803000pt}%
\definecolor{currentstroke}{rgb}{0.000000,0.000000,0.000000}%
\pgfsetstrokecolor{currentstroke}%
\pgfsetdash{}{0pt}%
\pgfsys@defobject{currentmarker}{\pgfqpoint{0.000000in}{-0.048611in}}{\pgfqpoint{0.000000in}{0.000000in}}{%
\pgfpathmoveto{\pgfqpoint{0.000000in}{0.000000in}}%
\pgfpathlineto{\pgfqpoint{0.000000in}{-0.048611in}}%
\pgfusepath{stroke,fill}%
}%
\begin{pgfscope}%
\pgfsys@transformshift{3.505455in}{0.528000in}%
\pgfsys@useobject{currentmarker}{}%
\end{pgfscope}%
\end{pgfscope}%
\begin{pgfscope}%
\definecolor{textcolor}{rgb}{0.000000,0.000000,0.000000}%
\pgfsetstrokecolor{textcolor}%
\pgfsetfillcolor{textcolor}%
\pgftext[x=3.505455in,y=0.430778in,,top]{\color{textcolor}\sffamily\fontsize{10.000000}{12.000000}\selectfont \(\displaystyle 60\)}%
\end{pgfscope}%
\begin{pgfscope}%
\pgfsetbuttcap%
\pgfsetroundjoin%
\definecolor{currentfill}{rgb}{0.000000,0.000000,0.000000}%
\pgfsetfillcolor{currentfill}%
\pgfsetlinewidth{0.803000pt}%
\definecolor{currentstroke}{rgb}{0.000000,0.000000,0.000000}%
\pgfsetstrokecolor{currentstroke}%
\pgfsetdash{}{0pt}%
\pgfsys@defobject{currentmarker}{\pgfqpoint{0.000000in}{-0.048611in}}{\pgfqpoint{0.000000in}{0.000000in}}{%
\pgfpathmoveto{\pgfqpoint{0.000000in}{0.000000in}}%
\pgfpathlineto{\pgfqpoint{0.000000in}{-0.048611in}}%
\pgfusepath{stroke,fill}%
}%
\begin{pgfscope}%
\pgfsys@transformshift{3.956364in}{0.528000in}%
\pgfsys@useobject{currentmarker}{}%
\end{pgfscope}%
\end{pgfscope}%
\begin{pgfscope}%
\definecolor{textcolor}{rgb}{0.000000,0.000000,0.000000}%
\pgfsetstrokecolor{textcolor}%
\pgfsetfillcolor{textcolor}%
\pgftext[x=3.956364in,y=0.430778in,,top]{\color{textcolor}\sffamily\fontsize{10.000000}{12.000000}\selectfont \(\displaystyle 70\)}%
\end{pgfscope}%
\begin{pgfscope}%
\pgfsetbuttcap%
\pgfsetroundjoin%
\definecolor{currentfill}{rgb}{0.000000,0.000000,0.000000}%
\pgfsetfillcolor{currentfill}%
\pgfsetlinewidth{0.803000pt}%
\definecolor{currentstroke}{rgb}{0.000000,0.000000,0.000000}%
\pgfsetstrokecolor{currentstroke}%
\pgfsetdash{}{0pt}%
\pgfsys@defobject{currentmarker}{\pgfqpoint{0.000000in}{-0.048611in}}{\pgfqpoint{0.000000in}{0.000000in}}{%
\pgfpathmoveto{\pgfqpoint{0.000000in}{0.000000in}}%
\pgfpathlineto{\pgfqpoint{0.000000in}{-0.048611in}}%
\pgfusepath{stroke,fill}%
}%
\begin{pgfscope}%
\pgfsys@transformshift{4.407273in}{0.528000in}%
\pgfsys@useobject{currentmarker}{}%
\end{pgfscope}%
\end{pgfscope}%
\begin{pgfscope}%
\definecolor{textcolor}{rgb}{0.000000,0.000000,0.000000}%
\pgfsetstrokecolor{textcolor}%
\pgfsetfillcolor{textcolor}%
\pgftext[x=4.407273in,y=0.430778in,,top]{\color{textcolor}\sffamily\fontsize{10.000000}{12.000000}\selectfont \(\displaystyle 80\)}%
\end{pgfscope}%
\begin{pgfscope}%
\pgfsetbuttcap%
\pgfsetroundjoin%
\definecolor{currentfill}{rgb}{0.000000,0.000000,0.000000}%
\pgfsetfillcolor{currentfill}%
\pgfsetlinewidth{0.803000pt}%
\definecolor{currentstroke}{rgb}{0.000000,0.000000,0.000000}%
\pgfsetstrokecolor{currentstroke}%
\pgfsetdash{}{0pt}%
\pgfsys@defobject{currentmarker}{\pgfqpoint{0.000000in}{-0.048611in}}{\pgfqpoint{0.000000in}{0.000000in}}{%
\pgfpathmoveto{\pgfqpoint{0.000000in}{0.000000in}}%
\pgfpathlineto{\pgfqpoint{0.000000in}{-0.048611in}}%
\pgfusepath{stroke,fill}%
}%
\begin{pgfscope}%
\pgfsys@transformshift{4.858182in}{0.528000in}%
\pgfsys@useobject{currentmarker}{}%
\end{pgfscope}%
\end{pgfscope}%
\begin{pgfscope}%
\definecolor{textcolor}{rgb}{0.000000,0.000000,0.000000}%
\pgfsetstrokecolor{textcolor}%
\pgfsetfillcolor{textcolor}%
\pgftext[x=4.858182in,y=0.430778in,,top]{\color{textcolor}\sffamily\fontsize{10.000000}{12.000000}\selectfont \(\displaystyle 90\)}%
\end{pgfscope}%
\begin{pgfscope}%
\pgfsetbuttcap%
\pgfsetroundjoin%
\definecolor{currentfill}{rgb}{0.000000,0.000000,0.000000}%
\pgfsetfillcolor{currentfill}%
\pgfsetlinewidth{0.803000pt}%
\definecolor{currentstroke}{rgb}{0.000000,0.000000,0.000000}%
\pgfsetstrokecolor{currentstroke}%
\pgfsetdash{}{0pt}%
\pgfsys@defobject{currentmarker}{\pgfqpoint{0.000000in}{-0.048611in}}{\pgfqpoint{0.000000in}{0.000000in}}{%
\pgfpathmoveto{\pgfqpoint{0.000000in}{0.000000in}}%
\pgfpathlineto{\pgfqpoint{0.000000in}{-0.048611in}}%
\pgfusepath{stroke,fill}%
}%
\begin{pgfscope}%
\pgfsys@transformshift{5.309091in}{0.528000in}%
\pgfsys@useobject{currentmarker}{}%
\end{pgfscope}%
\end{pgfscope}%
\begin{pgfscope}%
\definecolor{textcolor}{rgb}{0.000000,0.000000,0.000000}%
\pgfsetstrokecolor{textcolor}%
\pgfsetfillcolor{textcolor}%
\pgftext[x=5.309091in,y=0.430778in,,top]{\color{textcolor}\sffamily\fontsize{10.000000}{12.000000}\selectfont \(\displaystyle 100\)}%
\end{pgfscope}%
\begin{pgfscope}%
\definecolor{textcolor}{rgb}{0.000000,0.000000,0.000000}%
\pgfsetstrokecolor{textcolor}%
\pgfsetfillcolor{textcolor}%
\pgftext[x=3.280000in,y=0.240809in,,top]{\color{textcolor}\sffamily\fontsize{10.000000}{12.000000}\selectfont \(\displaystyle k\)}%
\end{pgfscope}%
\begin{pgfscope}%
\pgfsetbuttcap%
\pgfsetroundjoin%
\definecolor{currentfill}{rgb}{0.000000,0.000000,0.000000}%
\pgfsetfillcolor{currentfill}%
\pgfsetlinewidth{0.803000pt}%
\definecolor{currentstroke}{rgb}{0.000000,0.000000,0.000000}%
\pgfsetstrokecolor{currentstroke}%
\pgfsetdash{}{0pt}%
\pgfsys@defobject{currentmarker}{\pgfqpoint{-0.048611in}{0.000000in}}{\pgfqpoint{0.000000in}{0.000000in}}{%
\pgfpathmoveto{\pgfqpoint{0.000000in}{0.000000in}}%
\pgfpathlineto{\pgfqpoint{-0.048611in}{0.000000in}}%
\pgfusepath{stroke,fill}%
}%
\begin{pgfscope}%
\pgfsys@transformshift{0.800000in}{0.678442in}%
\pgfsys@useobject{currentmarker}{}%
\end{pgfscope}%
\end{pgfscope}%
\begin{pgfscope}%
\definecolor{textcolor}{rgb}{0.000000,0.000000,0.000000}%
\pgfsetstrokecolor{textcolor}%
\pgfsetfillcolor{textcolor}%
\pgftext[x=0.633333in,y=0.625680in,left,base]{\color{textcolor}\sffamily\fontsize{10.000000}{12.000000}\selectfont \(\displaystyle 0\)}%
\end{pgfscope}%
\begin{pgfscope}%
\pgfsetbuttcap%
\pgfsetroundjoin%
\definecolor{currentfill}{rgb}{0.000000,0.000000,0.000000}%
\pgfsetfillcolor{currentfill}%
\pgfsetlinewidth{0.803000pt}%
\definecolor{currentstroke}{rgb}{0.000000,0.000000,0.000000}%
\pgfsetstrokecolor{currentstroke}%
\pgfsetdash{}{0pt}%
\pgfsys@defobject{currentmarker}{\pgfqpoint{-0.048611in}{0.000000in}}{\pgfqpoint{0.000000in}{0.000000in}}{%
\pgfpathmoveto{\pgfqpoint{0.000000in}{0.000000in}}%
\pgfpathlineto{\pgfqpoint{-0.048611in}{0.000000in}}%
\pgfusepath{stroke,fill}%
}%
\begin{pgfscope}%
\pgfsys@transformshift{0.800000in}{1.077492in}%
\pgfsys@useobject{currentmarker}{}%
\end{pgfscope}%
\end{pgfscope}%
\begin{pgfscope}%
\definecolor{textcolor}{rgb}{0.000000,0.000000,0.000000}%
\pgfsetstrokecolor{textcolor}%
\pgfsetfillcolor{textcolor}%
\pgftext[x=0.494444in,y=1.024730in,left,base]{\color{textcolor}\sffamily\fontsize{10.000000}{12.000000}\selectfont \(\displaystyle 250\)}%
\end{pgfscope}%
\begin{pgfscope}%
\pgfsetbuttcap%
\pgfsetroundjoin%
\definecolor{currentfill}{rgb}{0.000000,0.000000,0.000000}%
\pgfsetfillcolor{currentfill}%
\pgfsetlinewidth{0.803000pt}%
\definecolor{currentstroke}{rgb}{0.000000,0.000000,0.000000}%
\pgfsetstrokecolor{currentstroke}%
\pgfsetdash{}{0pt}%
\pgfsys@defobject{currentmarker}{\pgfqpoint{-0.048611in}{0.000000in}}{\pgfqpoint{0.000000in}{0.000000in}}{%
\pgfpathmoveto{\pgfqpoint{0.000000in}{0.000000in}}%
\pgfpathlineto{\pgfqpoint{-0.048611in}{0.000000in}}%
\pgfusepath{stroke,fill}%
}%
\begin{pgfscope}%
\pgfsys@transformshift{0.800000in}{1.476542in}%
\pgfsys@useobject{currentmarker}{}%
\end{pgfscope}%
\end{pgfscope}%
\begin{pgfscope}%
\definecolor{textcolor}{rgb}{0.000000,0.000000,0.000000}%
\pgfsetstrokecolor{textcolor}%
\pgfsetfillcolor{textcolor}%
\pgftext[x=0.494444in,y=1.423780in,left,base]{\color{textcolor}\sffamily\fontsize{10.000000}{12.000000}\selectfont \(\displaystyle 500\)}%
\end{pgfscope}%
\begin{pgfscope}%
\pgfsetbuttcap%
\pgfsetroundjoin%
\definecolor{currentfill}{rgb}{0.000000,0.000000,0.000000}%
\pgfsetfillcolor{currentfill}%
\pgfsetlinewidth{0.803000pt}%
\definecolor{currentstroke}{rgb}{0.000000,0.000000,0.000000}%
\pgfsetstrokecolor{currentstroke}%
\pgfsetdash{}{0pt}%
\pgfsys@defobject{currentmarker}{\pgfqpoint{-0.048611in}{0.000000in}}{\pgfqpoint{0.000000in}{0.000000in}}{%
\pgfpathmoveto{\pgfqpoint{0.000000in}{0.000000in}}%
\pgfpathlineto{\pgfqpoint{-0.048611in}{0.000000in}}%
\pgfusepath{stroke,fill}%
}%
\begin{pgfscope}%
\pgfsys@transformshift{0.800000in}{1.875591in}%
\pgfsys@useobject{currentmarker}{}%
\end{pgfscope}%
\end{pgfscope}%
\begin{pgfscope}%
\definecolor{textcolor}{rgb}{0.000000,0.000000,0.000000}%
\pgfsetstrokecolor{textcolor}%
\pgfsetfillcolor{textcolor}%
\pgftext[x=0.494444in,y=1.822830in,left,base]{\color{textcolor}\sffamily\fontsize{10.000000}{12.000000}\selectfont \(\displaystyle 750\)}%
\end{pgfscope}%
\begin{pgfscope}%
\pgfsetbuttcap%
\pgfsetroundjoin%
\definecolor{currentfill}{rgb}{0.000000,0.000000,0.000000}%
\pgfsetfillcolor{currentfill}%
\pgfsetlinewidth{0.803000pt}%
\definecolor{currentstroke}{rgb}{0.000000,0.000000,0.000000}%
\pgfsetstrokecolor{currentstroke}%
\pgfsetdash{}{0pt}%
\pgfsys@defobject{currentmarker}{\pgfqpoint{-0.048611in}{0.000000in}}{\pgfqpoint{0.000000in}{0.000000in}}{%
\pgfpathmoveto{\pgfqpoint{0.000000in}{0.000000in}}%
\pgfpathlineto{\pgfqpoint{-0.048611in}{0.000000in}}%
\pgfusepath{stroke,fill}%
}%
\begin{pgfscope}%
\pgfsys@transformshift{0.800000in}{2.274641in}%
\pgfsys@useobject{currentmarker}{}%
\end{pgfscope}%
\end{pgfscope}%
\begin{pgfscope}%
\definecolor{textcolor}{rgb}{0.000000,0.000000,0.000000}%
\pgfsetstrokecolor{textcolor}%
\pgfsetfillcolor{textcolor}%
\pgftext[x=0.424999in,y=2.221880in,left,base]{\color{textcolor}\sffamily\fontsize{10.000000}{12.000000}\selectfont \(\displaystyle 1000\)}%
\end{pgfscope}%
\begin{pgfscope}%
\pgfsetbuttcap%
\pgfsetroundjoin%
\definecolor{currentfill}{rgb}{0.000000,0.000000,0.000000}%
\pgfsetfillcolor{currentfill}%
\pgfsetlinewidth{0.803000pt}%
\definecolor{currentstroke}{rgb}{0.000000,0.000000,0.000000}%
\pgfsetstrokecolor{currentstroke}%
\pgfsetdash{}{0pt}%
\pgfsys@defobject{currentmarker}{\pgfqpoint{-0.048611in}{0.000000in}}{\pgfqpoint{0.000000in}{0.000000in}}{%
\pgfpathmoveto{\pgfqpoint{0.000000in}{0.000000in}}%
\pgfpathlineto{\pgfqpoint{-0.048611in}{0.000000in}}%
\pgfusepath{stroke,fill}%
}%
\begin{pgfscope}%
\pgfsys@transformshift{0.800000in}{2.673691in}%
\pgfsys@useobject{currentmarker}{}%
\end{pgfscope}%
\end{pgfscope}%
\begin{pgfscope}%
\definecolor{textcolor}{rgb}{0.000000,0.000000,0.000000}%
\pgfsetstrokecolor{textcolor}%
\pgfsetfillcolor{textcolor}%
\pgftext[x=0.424999in,y=2.620930in,left,base]{\color{textcolor}\sffamily\fontsize{10.000000}{12.000000}\selectfont \(\displaystyle 1250\)}%
\end{pgfscope}%
\begin{pgfscope}%
\pgfsetbuttcap%
\pgfsetroundjoin%
\definecolor{currentfill}{rgb}{0.000000,0.000000,0.000000}%
\pgfsetfillcolor{currentfill}%
\pgfsetlinewidth{0.803000pt}%
\definecolor{currentstroke}{rgb}{0.000000,0.000000,0.000000}%
\pgfsetstrokecolor{currentstroke}%
\pgfsetdash{}{0pt}%
\pgfsys@defobject{currentmarker}{\pgfqpoint{-0.048611in}{0.000000in}}{\pgfqpoint{0.000000in}{0.000000in}}{%
\pgfpathmoveto{\pgfqpoint{0.000000in}{0.000000in}}%
\pgfpathlineto{\pgfqpoint{-0.048611in}{0.000000in}}%
\pgfusepath{stroke,fill}%
}%
\begin{pgfscope}%
\pgfsys@transformshift{0.800000in}{3.072741in}%
\pgfsys@useobject{currentmarker}{}%
\end{pgfscope}%
\end{pgfscope}%
\begin{pgfscope}%
\definecolor{textcolor}{rgb}{0.000000,0.000000,0.000000}%
\pgfsetstrokecolor{textcolor}%
\pgfsetfillcolor{textcolor}%
\pgftext[x=0.424999in,y=3.019980in,left,base]{\color{textcolor}\sffamily\fontsize{10.000000}{12.000000}\selectfont \(\displaystyle 1500\)}%
\end{pgfscope}%
\begin{pgfscope}%
\pgfsetbuttcap%
\pgfsetroundjoin%
\definecolor{currentfill}{rgb}{0.000000,0.000000,0.000000}%
\pgfsetfillcolor{currentfill}%
\pgfsetlinewidth{0.803000pt}%
\definecolor{currentstroke}{rgb}{0.000000,0.000000,0.000000}%
\pgfsetstrokecolor{currentstroke}%
\pgfsetdash{}{0pt}%
\pgfsys@defobject{currentmarker}{\pgfqpoint{-0.048611in}{0.000000in}}{\pgfqpoint{0.000000in}{0.000000in}}{%
\pgfpathmoveto{\pgfqpoint{0.000000in}{0.000000in}}%
\pgfpathlineto{\pgfqpoint{-0.048611in}{0.000000in}}%
\pgfusepath{stroke,fill}%
}%
\begin{pgfscope}%
\pgfsys@transformshift{0.800000in}{3.471791in}%
\pgfsys@useobject{currentmarker}{}%
\end{pgfscope}%
\end{pgfscope}%
\begin{pgfscope}%
\definecolor{textcolor}{rgb}{0.000000,0.000000,0.000000}%
\pgfsetstrokecolor{textcolor}%
\pgfsetfillcolor{textcolor}%
\pgftext[x=0.424999in,y=3.419029in,left,base]{\color{textcolor}\sffamily\fontsize{10.000000}{12.000000}\selectfont \(\displaystyle 1750\)}%
\end{pgfscope}%
\begin{pgfscope}%
\pgfsetbuttcap%
\pgfsetroundjoin%
\definecolor{currentfill}{rgb}{0.000000,0.000000,0.000000}%
\pgfsetfillcolor{currentfill}%
\pgfsetlinewidth{0.803000pt}%
\definecolor{currentstroke}{rgb}{0.000000,0.000000,0.000000}%
\pgfsetstrokecolor{currentstroke}%
\pgfsetdash{}{0pt}%
\pgfsys@defobject{currentmarker}{\pgfqpoint{-0.048611in}{0.000000in}}{\pgfqpoint{0.000000in}{0.000000in}}{%
\pgfpathmoveto{\pgfqpoint{0.000000in}{0.000000in}}%
\pgfpathlineto{\pgfqpoint{-0.048611in}{0.000000in}}%
\pgfusepath{stroke,fill}%
}%
\begin{pgfscope}%
\pgfsys@transformshift{0.800000in}{3.870841in}%
\pgfsys@useobject{currentmarker}{}%
\end{pgfscope}%
\end{pgfscope}%
\begin{pgfscope}%
\definecolor{textcolor}{rgb}{0.000000,0.000000,0.000000}%
\pgfsetstrokecolor{textcolor}%
\pgfsetfillcolor{textcolor}%
\pgftext[x=0.424999in,y=3.818079in,left,base]{\color{textcolor}\sffamily\fontsize{10.000000}{12.000000}\selectfont \(\displaystyle 2000\)}%
\end{pgfscope}%
\begin{pgfscope}%
\definecolor{textcolor}{rgb}{0.000000,0.000000,0.000000}%
\pgfsetstrokecolor{textcolor}%
\pgfsetfillcolor{textcolor}%
\pgftext[x=0.369444in,y=2.376000in,,bottom,rotate=90.000000]{\color{textcolor}\sffamily\fontsize{10.000000}{12.000000}\selectfont Number of GMRES iterations}%
\end{pgfscope}%
\begin{pgfscope}%
\pgfpathrectangle{\pgfqpoint{0.800000in}{0.528000in}}{\pgfqpoint{4.960000in}{3.696000in}}%
\pgfusepath{clip}%
\pgfsetbuttcap%
\pgfsetroundjoin%
\pgfsetlinewidth{1.505625pt}%
\definecolor{currentstroke}{rgb}{0.843137,0.000000,0.000000}%
\pgfsetstrokecolor{currentstroke}%
\pgfsetdash{{5.550000pt}{2.400000pt}}{0.000000pt}%
\pgfpathmoveto{\pgfqpoint{1.250909in}{0.700789in}}%
\pgfpathlineto{\pgfqpoint{1.701818in}{0.742290in}}%
\pgfpathlineto{\pgfqpoint{2.152727in}{0.868390in}}%
\pgfpathlineto{\pgfqpoint{2.603636in}{1.090261in}}%
\pgfpathlineto{\pgfqpoint{3.054545in}{1.360019in}}%
\pgfpathlineto{\pgfqpoint{3.505455in}{1.679259in}}%
\pgfpathlineto{\pgfqpoint{3.956364in}{2.178869in}}%
\pgfpathlineto{\pgfqpoint{4.407273in}{2.712000in}}%
\pgfpathlineto{\pgfqpoint{4.858182in}{3.384000in}}%
\pgfpathlineto{\pgfqpoint{5.309091in}{4.056000in}}%
\pgfusepath{stroke}%
\end{pgfscope}%
\begin{pgfscope}%
\pgfpathrectangle{\pgfqpoint{0.800000in}{0.528000in}}{\pgfqpoint{4.960000in}{3.696000in}}%
\pgfusepath{clip}%
\pgfsetbuttcap%
\pgfsetroundjoin%
\definecolor{currentfill}{rgb}{0.843137,0.000000,0.000000}%
\pgfsetfillcolor{currentfill}%
\pgfsetlinewidth{1.003750pt}%
\definecolor{currentstroke}{rgb}{0.843137,0.000000,0.000000}%
\pgfsetstrokecolor{currentstroke}%
\pgfsetdash{}{0pt}%
\pgfsys@defobject{currentmarker}{\pgfqpoint{-0.041667in}{-0.041667in}}{\pgfqpoint{0.041667in}{0.041667in}}{%
\pgfpathmoveto{\pgfqpoint{0.000000in}{-0.041667in}}%
\pgfpathcurveto{\pgfqpoint{0.011050in}{-0.041667in}}{\pgfqpoint{0.021649in}{-0.037276in}}{\pgfqpoint{0.029463in}{-0.029463in}}%
\pgfpathcurveto{\pgfqpoint{0.037276in}{-0.021649in}}{\pgfqpoint{0.041667in}{-0.011050in}}{\pgfqpoint{0.041667in}{0.000000in}}%
\pgfpathcurveto{\pgfqpoint{0.041667in}{0.011050in}}{\pgfqpoint{0.037276in}{0.021649in}}{\pgfqpoint{0.029463in}{0.029463in}}%
\pgfpathcurveto{\pgfqpoint{0.021649in}{0.037276in}}{\pgfqpoint{0.011050in}{0.041667in}}{\pgfqpoint{0.000000in}{0.041667in}}%
\pgfpathcurveto{\pgfqpoint{-0.011050in}{0.041667in}}{\pgfqpoint{-0.021649in}{0.037276in}}{\pgfqpoint{-0.029463in}{0.029463in}}%
\pgfpathcurveto{\pgfqpoint{-0.037276in}{0.021649in}}{\pgfqpoint{-0.041667in}{0.011050in}}{\pgfqpoint{-0.041667in}{0.000000in}}%
\pgfpathcurveto{\pgfqpoint{-0.041667in}{-0.011050in}}{\pgfqpoint{-0.037276in}{-0.021649in}}{\pgfqpoint{-0.029463in}{-0.029463in}}%
\pgfpathcurveto{\pgfqpoint{-0.021649in}{-0.037276in}}{\pgfqpoint{-0.011050in}{-0.041667in}}{\pgfqpoint{0.000000in}{-0.041667in}}%
\pgfpathclose%
\pgfusepath{stroke,fill}%
}%
\begin{pgfscope}%
\pgfsys@transformshift{1.250909in}{0.700789in}%
\pgfsys@useobject{currentmarker}{}%
\end{pgfscope}%
\begin{pgfscope}%
\pgfsys@transformshift{1.701818in}{0.742290in}%
\pgfsys@useobject{currentmarker}{}%
\end{pgfscope}%
\begin{pgfscope}%
\pgfsys@transformshift{2.152727in}{0.868390in}%
\pgfsys@useobject{currentmarker}{}%
\end{pgfscope}%
\begin{pgfscope}%
\pgfsys@transformshift{2.603636in}{1.090261in}%
\pgfsys@useobject{currentmarker}{}%
\end{pgfscope}%
\begin{pgfscope}%
\pgfsys@transformshift{3.054545in}{1.360019in}%
\pgfsys@useobject{currentmarker}{}%
\end{pgfscope}%
\begin{pgfscope}%
\pgfsys@transformshift{3.505455in}{1.679259in}%
\pgfsys@useobject{currentmarker}{}%
\end{pgfscope}%
\begin{pgfscope}%
\pgfsys@transformshift{3.956364in}{2.178869in}%
\pgfsys@useobject{currentmarker}{}%
\end{pgfscope}%
\begin{pgfscope}%
\pgfsys@transformshift{4.407273in}{2.712000in}%
\pgfsys@useobject{currentmarker}{}%
\end{pgfscope}%
\begin{pgfscope}%
\pgfsys@transformshift{4.858182in}{3.384000in}%
\pgfsys@useobject{currentmarker}{}%
\end{pgfscope}%
\begin{pgfscope}%
\pgfsys@transformshift{5.309091in}{4.056000in}%
\pgfsys@useobject{currentmarker}{}%
\end{pgfscope}%
\end{pgfscope}%
\begin{pgfscope}%
\pgfpathrectangle{\pgfqpoint{0.800000in}{0.528000in}}{\pgfqpoint{4.960000in}{3.696000in}}%
\pgfusepath{clip}%
\pgfsetbuttcap%
\pgfsetroundjoin%
\pgfsetlinewidth{1.505625pt}%
\definecolor{currentstroke}{rgb}{0.549020,0.235294,1.000000}%
\pgfsetstrokecolor{currentstroke}%
\pgfsetdash{{5.550000pt}{2.400000pt}}{0.000000pt}%
\pgfpathmoveto{\pgfqpoint{1.250909in}{0.699192in}}%
\pgfpathlineto{\pgfqpoint{1.701818in}{0.721539in}}%
\pgfpathlineto{\pgfqpoint{2.152727in}{0.790176in}}%
\pgfpathlineto{\pgfqpoint{2.603636in}{0.913083in}}%
\pgfpathlineto{\pgfqpoint{3.054545in}{1.096646in}}%
\pgfpathlineto{\pgfqpoint{3.505455in}{1.307344in}}%
\pgfpathlineto{\pgfqpoint{3.956364in}{1.620200in}}%
\pgfpathlineto{\pgfqpoint{4.407273in}{1.995306in}}%
\pgfpathlineto{\pgfqpoint{4.858182in}{2.478955in}}%
\pgfpathlineto{\pgfqpoint{5.309091in}{2.901948in}}%
\pgfusepath{stroke}%
\end{pgfscope}%
\begin{pgfscope}%
\pgfpathrectangle{\pgfqpoint{0.800000in}{0.528000in}}{\pgfqpoint{4.960000in}{3.696000in}}%
\pgfusepath{clip}%
\pgfsetbuttcap%
\pgfsetmiterjoin%
\definecolor{currentfill}{rgb}{0.549020,0.235294,1.000000}%
\pgfsetfillcolor{currentfill}%
\pgfsetlinewidth{1.003750pt}%
\definecolor{currentstroke}{rgb}{0.549020,0.235294,1.000000}%
\pgfsetstrokecolor{currentstroke}%
\pgfsetdash{}{0pt}%
\pgfsys@defobject{currentmarker}{\pgfqpoint{-0.041667in}{-0.041667in}}{\pgfqpoint{0.041667in}{0.041667in}}{%
\pgfpathmoveto{\pgfqpoint{0.000000in}{0.041667in}}%
\pgfpathlineto{\pgfqpoint{-0.041667in}{-0.041667in}}%
\pgfpathlineto{\pgfqpoint{0.041667in}{-0.041667in}}%
\pgfpathclose%
\pgfusepath{stroke,fill}%
}%
\begin{pgfscope}%
\pgfsys@transformshift{1.250909in}{0.699192in}%
\pgfsys@useobject{currentmarker}{}%
\end{pgfscope}%
\begin{pgfscope}%
\pgfsys@transformshift{1.701818in}{0.721539in}%
\pgfsys@useobject{currentmarker}{}%
\end{pgfscope}%
\begin{pgfscope}%
\pgfsys@transformshift{2.152727in}{0.790176in}%
\pgfsys@useobject{currentmarker}{}%
\end{pgfscope}%
\begin{pgfscope}%
\pgfsys@transformshift{2.603636in}{0.913083in}%
\pgfsys@useobject{currentmarker}{}%
\end{pgfscope}%
\begin{pgfscope}%
\pgfsys@transformshift{3.054545in}{1.096646in}%
\pgfsys@useobject{currentmarker}{}%
\end{pgfscope}%
\begin{pgfscope}%
\pgfsys@transformshift{3.505455in}{1.307344in}%
\pgfsys@useobject{currentmarker}{}%
\end{pgfscope}%
\begin{pgfscope}%
\pgfsys@transformshift{3.956364in}{1.620200in}%
\pgfsys@useobject{currentmarker}{}%
\end{pgfscope}%
\begin{pgfscope}%
\pgfsys@transformshift{4.407273in}{1.995306in}%
\pgfsys@useobject{currentmarker}{}%
\end{pgfscope}%
\begin{pgfscope}%
\pgfsys@transformshift{4.858182in}{2.478955in}%
\pgfsys@useobject{currentmarker}{}%
\end{pgfscope}%
\begin{pgfscope}%
\pgfsys@transformshift{5.309091in}{2.901948in}%
\pgfsys@useobject{currentmarker}{}%
\end{pgfscope}%
\end{pgfscope}%
\begin{pgfscope}%
\pgfpathrectangle{\pgfqpoint{0.800000in}{0.528000in}}{\pgfqpoint{4.960000in}{3.696000in}}%
\pgfusepath{clip}%
\pgfsetbuttcap%
\pgfsetroundjoin%
\pgfsetlinewidth{1.505625pt}%
\definecolor{currentstroke}{rgb}{0.007843,0.533333,0.000000}%
\pgfsetstrokecolor{currentstroke}%
\pgfsetdash{{5.550000pt}{2.400000pt}}{0.000000pt}%
\pgfpathmoveto{\pgfqpoint{1.250909in}{0.697596in}}%
\pgfpathlineto{\pgfqpoint{1.701818in}{0.713558in}}%
\pgfpathlineto{\pgfqpoint{2.152727in}{0.742290in}}%
\pgfpathlineto{\pgfqpoint{2.603636in}{0.801349in}}%
\pgfpathlineto{\pgfqpoint{3.054545in}{0.892333in}}%
\pgfpathlineto{\pgfqpoint{3.505455in}{0.996086in}}%
\pgfpathlineto{\pgfqpoint{3.956364in}{1.144532in}}%
\pgfpathlineto{\pgfqpoint{4.407273in}{1.347249in}}%
\pgfpathlineto{\pgfqpoint{4.858182in}{1.557948in}}%
\pgfpathlineto{\pgfqpoint{5.309091in}{1.837283in}}%
\pgfusepath{stroke}%
\end{pgfscope}%
\begin{pgfscope}%
\pgfpathrectangle{\pgfqpoint{0.800000in}{0.528000in}}{\pgfqpoint{4.960000in}{3.696000in}}%
\pgfusepath{clip}%
\pgfsetbuttcap%
\pgfsetmiterjoin%
\definecolor{currentfill}{rgb}{0.007843,0.533333,0.000000}%
\pgfsetfillcolor{currentfill}%
\pgfsetlinewidth{1.003750pt}%
\definecolor{currentstroke}{rgb}{0.007843,0.533333,0.000000}%
\pgfsetstrokecolor{currentstroke}%
\pgfsetdash{}{0pt}%
\pgfsys@defobject{currentmarker}{\pgfqpoint{-0.041667in}{-0.041667in}}{\pgfqpoint{0.041667in}{0.041667in}}{%
\pgfpathmoveto{\pgfqpoint{-0.000000in}{-0.041667in}}%
\pgfpathlineto{\pgfqpoint{0.041667in}{0.041667in}}%
\pgfpathlineto{\pgfqpoint{-0.041667in}{0.041667in}}%
\pgfpathclose%
\pgfusepath{stroke,fill}%
}%
\begin{pgfscope}%
\pgfsys@transformshift{1.250909in}{0.697596in}%
\pgfsys@useobject{currentmarker}{}%
\end{pgfscope}%
\begin{pgfscope}%
\pgfsys@transformshift{1.701818in}{0.713558in}%
\pgfsys@useobject{currentmarker}{}%
\end{pgfscope}%
\begin{pgfscope}%
\pgfsys@transformshift{2.152727in}{0.742290in}%
\pgfsys@useobject{currentmarker}{}%
\end{pgfscope}%
\begin{pgfscope}%
\pgfsys@transformshift{2.603636in}{0.801349in}%
\pgfsys@useobject{currentmarker}{}%
\end{pgfscope}%
\begin{pgfscope}%
\pgfsys@transformshift{3.054545in}{0.892333in}%
\pgfsys@useobject{currentmarker}{}%
\end{pgfscope}%
\begin{pgfscope}%
\pgfsys@transformshift{3.505455in}{0.996086in}%
\pgfsys@useobject{currentmarker}{}%
\end{pgfscope}%
\begin{pgfscope}%
\pgfsys@transformshift{3.956364in}{1.144532in}%
\pgfsys@useobject{currentmarker}{}%
\end{pgfscope}%
\begin{pgfscope}%
\pgfsys@transformshift{4.407273in}{1.347249in}%
\pgfsys@useobject{currentmarker}{}%
\end{pgfscope}%
\begin{pgfscope}%
\pgfsys@transformshift{4.858182in}{1.557948in}%
\pgfsys@useobject{currentmarker}{}%
\end{pgfscope}%
\begin{pgfscope}%
\pgfsys@transformshift{5.309091in}{1.837283in}%
\pgfsys@useobject{currentmarker}{}%
\end{pgfscope}%
\end{pgfscope}%
\begin{pgfscope}%
\pgfpathrectangle{\pgfqpoint{0.800000in}{0.528000in}}{\pgfqpoint{4.960000in}{3.696000in}}%
\pgfusepath{clip}%
\pgfsetbuttcap%
\pgfsetroundjoin%
\pgfsetlinewidth{1.505625pt}%
\definecolor{currentstroke}{rgb}{0.000000,0.674510,0.780392}%
\pgfsetstrokecolor{currentstroke}%
\pgfsetdash{{5.550000pt}{2.400000pt}}{0.000000pt}%
\pgfpathmoveto{\pgfqpoint{1.250909in}{0.696000in}}%
\pgfpathlineto{\pgfqpoint{1.701818in}{0.707173in}}%
\pgfpathlineto{\pgfqpoint{2.152727in}{0.718347in}}%
\pgfpathlineto{\pgfqpoint{2.603636in}{0.742290in}}%
\pgfpathlineto{\pgfqpoint{3.054545in}{0.771021in}}%
\pgfpathlineto{\pgfqpoint{3.505455in}{0.815715in}}%
\pgfpathlineto{\pgfqpoint{3.956364in}{0.868390in}}%
\pgfpathlineto{\pgfqpoint{4.407273in}{0.938622in}}%
\pgfpathlineto{\pgfqpoint{4.858182in}{1.012048in}}%
\pgfpathlineto{\pgfqpoint{5.309091in}{1.109416in}}%
\pgfusepath{stroke}%
\end{pgfscope}%
\begin{pgfscope}%
\pgfpathrectangle{\pgfqpoint{0.800000in}{0.528000in}}{\pgfqpoint{4.960000in}{3.696000in}}%
\pgfusepath{clip}%
\pgfsetbuttcap%
\pgfsetmiterjoin%
\definecolor{currentfill}{rgb}{0.000000,0.674510,0.780392}%
\pgfsetfillcolor{currentfill}%
\pgfsetlinewidth{1.003750pt}%
\definecolor{currentstroke}{rgb}{0.000000,0.674510,0.780392}%
\pgfsetstrokecolor{currentstroke}%
\pgfsetdash{}{0pt}%
\pgfsys@defobject{currentmarker}{\pgfqpoint{-0.041667in}{-0.041667in}}{\pgfqpoint{0.041667in}{0.041667in}}{%
\pgfpathmoveto{\pgfqpoint{0.041667in}{-0.000000in}}%
\pgfpathlineto{\pgfqpoint{-0.041667in}{0.041667in}}%
\pgfpathlineto{\pgfqpoint{-0.041667in}{-0.041667in}}%
\pgfpathclose%
\pgfusepath{stroke,fill}%
}%
\begin{pgfscope}%
\pgfsys@transformshift{1.250909in}{0.696000in}%
\pgfsys@useobject{currentmarker}{}%
\end{pgfscope}%
\begin{pgfscope}%
\pgfsys@transformshift{1.701818in}{0.707173in}%
\pgfsys@useobject{currentmarker}{}%
\end{pgfscope}%
\begin{pgfscope}%
\pgfsys@transformshift{2.152727in}{0.718347in}%
\pgfsys@useobject{currentmarker}{}%
\end{pgfscope}%
\begin{pgfscope}%
\pgfsys@transformshift{2.603636in}{0.742290in}%
\pgfsys@useobject{currentmarker}{}%
\end{pgfscope}%
\begin{pgfscope}%
\pgfsys@transformshift{3.054545in}{0.771021in}%
\pgfsys@useobject{currentmarker}{}%
\end{pgfscope}%
\begin{pgfscope}%
\pgfsys@transformshift{3.505455in}{0.815715in}%
\pgfsys@useobject{currentmarker}{}%
\end{pgfscope}%
\begin{pgfscope}%
\pgfsys@transformshift{3.956364in}{0.868390in}%
\pgfsys@useobject{currentmarker}{}%
\end{pgfscope}%
\begin{pgfscope}%
\pgfsys@transformshift{4.407273in}{0.938622in}%
\pgfsys@useobject{currentmarker}{}%
\end{pgfscope}%
\begin{pgfscope}%
\pgfsys@transformshift{4.858182in}{1.012048in}%
\pgfsys@useobject{currentmarker}{}%
\end{pgfscope}%
\begin{pgfscope}%
\pgfsys@transformshift{5.309091in}{1.109416in}%
\pgfsys@useobject{currentmarker}{}%
\end{pgfscope}%
\end{pgfscope}%
\begin{pgfscope}%
\pgfsetrectcap%
\pgfsetmiterjoin%
\pgfsetlinewidth{0.803000pt}%
\definecolor{currentstroke}{rgb}{0.000000,0.000000,0.000000}%
\pgfsetstrokecolor{currentstroke}%
\pgfsetdash{}{0pt}%
\pgfpathmoveto{\pgfqpoint{0.800000in}{0.528000in}}%
\pgfpathlineto{\pgfqpoint{0.800000in}{4.224000in}}%
\pgfusepath{stroke}%
\end{pgfscope}%
\begin{pgfscope}%
\pgfsetrectcap%
\pgfsetmiterjoin%
\pgfsetlinewidth{0.000000pt}%
\definecolor{currentstroke}{rgb}{0.000000,0.000000,0.000000}%
\pgfsetstrokecolor{currentstroke}%
\pgfsetstrokeopacity{0.000000}%
\pgfsetdash{}{0pt}%
\pgfpathmoveto{\pgfqpoint{5.760000in}{0.528000in}}%
\pgfpathlineto{\pgfqpoint{5.760000in}{4.224000in}}%
\pgfusepath{}%
\end{pgfscope}%
\begin{pgfscope}%
\pgfsetrectcap%
\pgfsetmiterjoin%
\pgfsetlinewidth{0.803000pt}%
\definecolor{currentstroke}{rgb}{0.000000,0.000000,0.000000}%
\pgfsetstrokecolor{currentstroke}%
\pgfsetdash{}{0pt}%
\pgfpathmoveto{\pgfqpoint{0.800000in}{0.528000in}}%
\pgfpathlineto{\pgfqpoint{5.760000in}{0.528000in}}%
\pgfusepath{stroke}%
\end{pgfscope}%
\begin{pgfscope}%
\pgfsetrectcap%
\pgfsetmiterjoin%
\pgfsetlinewidth{0.000000pt}%
\definecolor{currentstroke}{rgb}{0.000000,0.000000,0.000000}%
\pgfsetstrokecolor{currentstroke}%
\pgfsetstrokeopacity{0.000000}%
\pgfsetdash{}{0pt}%
\pgfpathmoveto{\pgfqpoint{0.800000in}{4.224000in}}%
\pgfpathlineto{\pgfqpoint{5.760000in}{4.224000in}}%
\pgfusepath{}%
\end{pgfscope}%
\begin{pgfscope}%
\pgfsetbuttcap%
\pgfsetmiterjoin%
\definecolor{currentfill}{rgb}{1.000000,1.000000,1.000000}%
\pgfsetfillcolor{currentfill}%
\pgfsetfillopacity{0.800000}%
\pgfsetlinewidth{1.003750pt}%
\definecolor{currentstroke}{rgb}{0.800000,0.800000,0.800000}%
\pgfsetstrokecolor{currentstroke}%
\pgfsetstrokeopacity{0.800000}%
\pgfsetdash{}{0pt}%
\pgfpathmoveto{\pgfqpoint{0.897222in}{3.297460in}}%
\pgfpathlineto{\pgfqpoint{1.790209in}{3.297460in}}%
\pgfpathquadraticcurveto{\pgfqpoint{1.817987in}{3.297460in}}{\pgfqpoint{1.817987in}{3.325238in}}%
\pgfpathlineto{\pgfqpoint{1.817987in}{4.126778in}}%
\pgfpathquadraticcurveto{\pgfqpoint{1.817987in}{4.154556in}}{\pgfqpoint{1.790209in}{4.154556in}}%
\pgfpathlineto{\pgfqpoint{0.897222in}{4.154556in}}%
\pgfpathquadraticcurveto{\pgfqpoint{0.869444in}{4.154556in}}{\pgfqpoint{0.869444in}{4.126778in}}%
\pgfpathlineto{\pgfqpoint{0.869444in}{3.325238in}}%
\pgfpathquadraticcurveto{\pgfqpoint{0.869444in}{3.297460in}}{\pgfqpoint{0.897222in}{3.297460in}}%
\pgfpathclose%
\pgfusepath{stroke,fill}%
\end{pgfscope}%
\begin{pgfscope}%
\pgfsetbuttcap%
\pgfsetroundjoin%
\pgfsetlinewidth{1.505625pt}%
\definecolor{currentstroke}{rgb}{0.843137,0.000000,0.000000}%
\pgfsetstrokecolor{currentstroke}%
\pgfsetdash{{5.550000pt}{2.400000pt}}{0.000000pt}%
\pgfpathmoveto{\pgfqpoint{0.925000in}{4.042088in}}%
\pgfpathlineto{\pgfqpoint{1.202778in}{4.042088in}}%
\pgfusepath{stroke}%
\end{pgfscope}%
\begin{pgfscope}%
\pgfsetbuttcap%
\pgfsetroundjoin%
\definecolor{currentfill}{rgb}{0.843137,0.000000,0.000000}%
\pgfsetfillcolor{currentfill}%
\pgfsetlinewidth{1.003750pt}%
\definecolor{currentstroke}{rgb}{0.843137,0.000000,0.000000}%
\pgfsetstrokecolor{currentstroke}%
\pgfsetdash{}{0pt}%
\pgfsys@defobject{currentmarker}{\pgfqpoint{-0.041667in}{-0.041667in}}{\pgfqpoint{0.041667in}{0.041667in}}{%
\pgfpathmoveto{\pgfqpoint{0.000000in}{-0.041667in}}%
\pgfpathcurveto{\pgfqpoint{0.011050in}{-0.041667in}}{\pgfqpoint{0.021649in}{-0.037276in}}{\pgfqpoint{0.029463in}{-0.029463in}}%
\pgfpathcurveto{\pgfqpoint{0.037276in}{-0.021649in}}{\pgfqpoint{0.041667in}{-0.011050in}}{\pgfqpoint{0.041667in}{0.000000in}}%
\pgfpathcurveto{\pgfqpoint{0.041667in}{0.011050in}}{\pgfqpoint{0.037276in}{0.021649in}}{\pgfqpoint{0.029463in}{0.029463in}}%
\pgfpathcurveto{\pgfqpoint{0.021649in}{0.037276in}}{\pgfqpoint{0.011050in}{0.041667in}}{\pgfqpoint{0.000000in}{0.041667in}}%
\pgfpathcurveto{\pgfqpoint{-0.011050in}{0.041667in}}{\pgfqpoint{-0.021649in}{0.037276in}}{\pgfqpoint{-0.029463in}{0.029463in}}%
\pgfpathcurveto{\pgfqpoint{-0.037276in}{0.021649in}}{\pgfqpoint{-0.041667in}{0.011050in}}{\pgfqpoint{-0.041667in}{0.000000in}}%
\pgfpathcurveto{\pgfqpoint{-0.041667in}{-0.011050in}}{\pgfqpoint{-0.037276in}{-0.021649in}}{\pgfqpoint{-0.029463in}{-0.029463in}}%
\pgfpathcurveto{\pgfqpoint{-0.021649in}{-0.037276in}}{\pgfqpoint{-0.011050in}{-0.041667in}}{\pgfqpoint{0.000000in}{-0.041667in}}%
\pgfpathclose%
\pgfusepath{stroke,fill}%
}%
\begin{pgfscope}%
\pgfsys@transformshift{1.063889in}{4.042088in}%
\pgfsys@useobject{currentmarker}{}%
\end{pgfscope}%
\end{pgfscope}%
\begin{pgfscope}%
\definecolor{textcolor}{rgb}{0.000000,0.000000,0.000000}%
\pgfsetstrokecolor{textcolor}%
\pgfsetfillcolor{textcolor}%
\pgftext[x=1.313889in,y=3.993477in,left,base]{\color{textcolor}\sffamily\fontsize{10.000000}{12.000000}\selectfont \(\displaystyle \beta = 0\)}%
\end{pgfscope}%
\begin{pgfscope}%
\pgfsetbuttcap%
\pgfsetroundjoin%
\pgfsetlinewidth{1.505625pt}%
\definecolor{currentstroke}{rgb}{0.549020,0.235294,1.000000}%
\pgfsetstrokecolor{currentstroke}%
\pgfsetdash{{5.550000pt}{2.400000pt}}{0.000000pt}%
\pgfpathmoveto{\pgfqpoint{0.925000in}{3.838231in}}%
\pgfpathlineto{\pgfqpoint{1.202778in}{3.838231in}}%
\pgfusepath{stroke}%
\end{pgfscope}%
\begin{pgfscope}%
\pgfsetbuttcap%
\pgfsetmiterjoin%
\definecolor{currentfill}{rgb}{0.549020,0.235294,1.000000}%
\pgfsetfillcolor{currentfill}%
\pgfsetlinewidth{1.003750pt}%
\definecolor{currentstroke}{rgb}{0.549020,0.235294,1.000000}%
\pgfsetstrokecolor{currentstroke}%
\pgfsetdash{}{0pt}%
\pgfsys@defobject{currentmarker}{\pgfqpoint{-0.041667in}{-0.041667in}}{\pgfqpoint{0.041667in}{0.041667in}}{%
\pgfpathmoveto{\pgfqpoint{0.000000in}{0.041667in}}%
\pgfpathlineto{\pgfqpoint{-0.041667in}{-0.041667in}}%
\pgfpathlineto{\pgfqpoint{0.041667in}{-0.041667in}}%
\pgfpathclose%
\pgfusepath{stroke,fill}%
}%
\begin{pgfscope}%
\pgfsys@transformshift{1.063889in}{3.838231in}%
\pgfsys@useobject{currentmarker}{}%
\end{pgfscope}%
\end{pgfscope}%
\begin{pgfscope}%
\definecolor{textcolor}{rgb}{0.000000,0.000000,0.000000}%
\pgfsetstrokecolor{textcolor}%
\pgfsetfillcolor{textcolor}%
\pgftext[x=1.313889in,y=3.789620in,left,base]{\color{textcolor}\sffamily\fontsize{10.000000}{12.000000}\selectfont \(\displaystyle \beta = 0.1\)}%
\end{pgfscope}%
\begin{pgfscope}%
\pgfsetbuttcap%
\pgfsetroundjoin%
\pgfsetlinewidth{1.505625pt}%
\definecolor{currentstroke}{rgb}{0.007843,0.533333,0.000000}%
\pgfsetstrokecolor{currentstroke}%
\pgfsetdash{{5.550000pt}{2.400000pt}}{0.000000pt}%
\pgfpathmoveto{\pgfqpoint{0.925000in}{3.634374in}}%
\pgfpathlineto{\pgfqpoint{1.202778in}{3.634374in}}%
\pgfusepath{stroke}%
\end{pgfscope}%
\begin{pgfscope}%
\pgfsetbuttcap%
\pgfsetmiterjoin%
\definecolor{currentfill}{rgb}{0.007843,0.533333,0.000000}%
\pgfsetfillcolor{currentfill}%
\pgfsetlinewidth{1.003750pt}%
\definecolor{currentstroke}{rgb}{0.007843,0.533333,0.000000}%
\pgfsetstrokecolor{currentstroke}%
\pgfsetdash{}{0pt}%
\pgfsys@defobject{currentmarker}{\pgfqpoint{-0.041667in}{-0.041667in}}{\pgfqpoint{0.041667in}{0.041667in}}{%
\pgfpathmoveto{\pgfqpoint{-0.000000in}{-0.041667in}}%
\pgfpathlineto{\pgfqpoint{0.041667in}{0.041667in}}%
\pgfpathlineto{\pgfqpoint{-0.041667in}{0.041667in}}%
\pgfpathclose%
\pgfusepath{stroke,fill}%
}%
\begin{pgfscope}%
\pgfsys@transformshift{1.063889in}{3.634374in}%
\pgfsys@useobject{currentmarker}{}%
\end{pgfscope}%
\end{pgfscope}%
\begin{pgfscope}%
\definecolor{textcolor}{rgb}{0.000000,0.000000,0.000000}%
\pgfsetstrokecolor{textcolor}%
\pgfsetfillcolor{textcolor}%
\pgftext[x=1.313889in,y=3.585762in,left,base]{\color{textcolor}\sffamily\fontsize{10.000000}{12.000000}\selectfont \(\displaystyle \beta = 0.2\)}%
\end{pgfscope}%
\begin{pgfscope}%
\pgfsetbuttcap%
\pgfsetroundjoin%
\pgfsetlinewidth{1.505625pt}%
\definecolor{currentstroke}{rgb}{0.000000,0.674510,0.780392}%
\pgfsetstrokecolor{currentstroke}%
\pgfsetdash{{5.550000pt}{2.400000pt}}{0.000000pt}%
\pgfpathmoveto{\pgfqpoint{0.925000in}{3.430516in}}%
\pgfpathlineto{\pgfqpoint{1.202778in}{3.430516in}}%
\pgfusepath{stroke}%
\end{pgfscope}%
\begin{pgfscope}%
\pgfsetbuttcap%
\pgfsetmiterjoin%
\definecolor{currentfill}{rgb}{0.000000,0.674510,0.780392}%
\pgfsetfillcolor{currentfill}%
\pgfsetlinewidth{1.003750pt}%
\definecolor{currentstroke}{rgb}{0.000000,0.674510,0.780392}%
\pgfsetstrokecolor{currentstroke}%
\pgfsetdash{}{0pt}%
\pgfsys@defobject{currentmarker}{\pgfqpoint{-0.041667in}{-0.041667in}}{\pgfqpoint{0.041667in}{0.041667in}}{%
\pgfpathmoveto{\pgfqpoint{0.041667in}{-0.000000in}}%
\pgfpathlineto{\pgfqpoint{-0.041667in}{0.041667in}}%
\pgfpathlineto{\pgfqpoint{-0.041667in}{-0.041667in}}%
\pgfpathclose%
\pgfusepath{stroke,fill}%
}%
\begin{pgfscope}%
\pgfsys@transformshift{1.063889in}{3.430516in}%
\pgfsys@useobject{currentmarker}{}%
\end{pgfscope}%
\end{pgfscope}%
\begin{pgfscope}%
\definecolor{textcolor}{rgb}{0.000000,0.000000,0.000000}%
\pgfsetstrokecolor{textcolor}%
\pgfsetfillcolor{textcolor}%
\pgftext[x=1.313889in,y=3.381905in,left,base]{\color{textcolor}\sffamily\fontsize{10.000000}{12.000000}\selectfont \(\displaystyle \beta = 0.3\)}%
\end{pgfscope}%
\end{pgfpicture}%
\makeatother%
\endgroup%

  \caption{GMRES iteration counts for $\AmatoI\Amatt$ given by \cref{eq:noweak,eq:ntweak}, where $\alpha = 0.2/k^\beta,$ for $\beta = 0,0.1,0.2,0.3.$}\label{fig:l1low}
\end{figure}

\begin{figure}
  %% Creator: Matplotlib, PGF backend
%%
%% To include the figure in your LaTeX document, write
%%   \input{<filename>.pgf}
%%
%% Make sure the required packages are loaded in your preamble
%%   \usepackage{pgf}
%%
%% Figures using additional raster images can only be included by \input if
%% they are in the same directory as the main LaTeX file. For loading figures
%% from other directories you can use the `import` package
%%   \usepackage{import}
%% and then include the figures with
%%   \import{<path to file>}{<filename>.pgf}
%%
%% Matplotlib used the following preamble
%%   \usepackage{fontspec}
%%   \setmainfont{DejaVuSerif.ttf}[Path=/home/owen/progs/firedrake-complex/firedrake/lib/python3.5/site-packages/matplotlib/mpl-data/fonts/ttf/]
%%   \setsansfont{DejaVuSans.ttf}[Path=/home/owen/progs/firedrake-complex/firedrake/lib/python3.5/site-packages/matplotlib/mpl-data/fonts/ttf/]
%%   \setmonofont{DejaVuSansMono.ttf}[Path=/home/owen/progs/firedrake-complex/firedrake/lib/python3.5/site-packages/matplotlib/mpl-data/fonts/ttf/]
%%
\begingroup%
\makeatletter%
\begin{pgfpicture}%
\pgfpathrectangle{\pgfpointorigin}{\pgfqpoint{6.400000in}{4.800000in}}%
\pgfusepath{use as bounding box, clip}%
\begin{pgfscope}%
\pgfsetbuttcap%
\pgfsetmiterjoin%
\definecolor{currentfill}{rgb}{1.000000,1.000000,1.000000}%
\pgfsetfillcolor{currentfill}%
\pgfsetlinewidth{0.000000pt}%
\definecolor{currentstroke}{rgb}{1.000000,1.000000,1.000000}%
\pgfsetstrokecolor{currentstroke}%
\pgfsetdash{}{0pt}%
\pgfpathmoveto{\pgfqpoint{0.000000in}{0.000000in}}%
\pgfpathlineto{\pgfqpoint{6.400000in}{0.000000in}}%
\pgfpathlineto{\pgfqpoint{6.400000in}{4.800000in}}%
\pgfpathlineto{\pgfqpoint{0.000000in}{4.800000in}}%
\pgfpathclose%
\pgfusepath{fill}%
\end{pgfscope}%
\begin{pgfscope}%
\pgfsetbuttcap%
\pgfsetmiterjoin%
\definecolor{currentfill}{rgb}{1.000000,1.000000,1.000000}%
\pgfsetfillcolor{currentfill}%
\pgfsetlinewidth{0.000000pt}%
\definecolor{currentstroke}{rgb}{0.000000,0.000000,0.000000}%
\pgfsetstrokecolor{currentstroke}%
\pgfsetstrokeopacity{0.000000}%
\pgfsetdash{}{0pt}%
\pgfpathmoveto{\pgfqpoint{0.800000in}{0.528000in}}%
\pgfpathlineto{\pgfqpoint{5.760000in}{0.528000in}}%
\pgfpathlineto{\pgfqpoint{5.760000in}{4.224000in}}%
\pgfpathlineto{\pgfqpoint{0.800000in}{4.224000in}}%
\pgfpathclose%
\pgfusepath{fill}%
\end{pgfscope}%
\begin{pgfscope}%
\pgfsetbuttcap%
\pgfsetroundjoin%
\definecolor{currentfill}{rgb}{0.000000,0.000000,0.000000}%
\pgfsetfillcolor{currentfill}%
\pgfsetlinewidth{0.803000pt}%
\definecolor{currentstroke}{rgb}{0.000000,0.000000,0.000000}%
\pgfsetstrokecolor{currentstroke}%
\pgfsetdash{}{0pt}%
\pgfsys@defobject{currentmarker}{\pgfqpoint{0.000000in}{-0.048611in}}{\pgfqpoint{0.000000in}{0.000000in}}{%
\pgfpathmoveto{\pgfqpoint{0.000000in}{0.000000in}}%
\pgfpathlineto{\pgfqpoint{0.000000in}{-0.048611in}}%
\pgfusepath{stroke,fill}%
}%
\begin{pgfscope}%
\pgfsys@transformshift{1.250909in}{0.528000in}%
\pgfsys@useobject{currentmarker}{}%
\end{pgfscope}%
\end{pgfscope}%
\begin{pgfscope}%
\definecolor{textcolor}{rgb}{0.000000,0.000000,0.000000}%
\pgfsetstrokecolor{textcolor}%
\pgfsetfillcolor{textcolor}%
\pgftext[x=1.250909in,y=0.430778in,,top]{\color{textcolor}\sffamily\fontsize{10.000000}{12.000000}\selectfont 10}%
\end{pgfscope}%
\begin{pgfscope}%
\pgfsetbuttcap%
\pgfsetroundjoin%
\definecolor{currentfill}{rgb}{0.000000,0.000000,0.000000}%
\pgfsetfillcolor{currentfill}%
\pgfsetlinewidth{0.803000pt}%
\definecolor{currentstroke}{rgb}{0.000000,0.000000,0.000000}%
\pgfsetstrokecolor{currentstroke}%
\pgfsetdash{}{0pt}%
\pgfsys@defobject{currentmarker}{\pgfqpoint{0.000000in}{-0.048611in}}{\pgfqpoint{0.000000in}{0.000000in}}{%
\pgfpathmoveto{\pgfqpoint{0.000000in}{0.000000in}}%
\pgfpathlineto{\pgfqpoint{0.000000in}{-0.048611in}}%
\pgfusepath{stroke,fill}%
}%
\begin{pgfscope}%
\pgfsys@transformshift{1.701818in}{0.528000in}%
\pgfsys@useobject{currentmarker}{}%
\end{pgfscope}%
\end{pgfscope}%
\begin{pgfscope}%
\definecolor{textcolor}{rgb}{0.000000,0.000000,0.000000}%
\pgfsetstrokecolor{textcolor}%
\pgfsetfillcolor{textcolor}%
\pgftext[x=1.701818in,y=0.430778in,,top]{\color{textcolor}\sffamily\fontsize{10.000000}{12.000000}\selectfont 20}%
\end{pgfscope}%
\begin{pgfscope}%
\pgfsetbuttcap%
\pgfsetroundjoin%
\definecolor{currentfill}{rgb}{0.000000,0.000000,0.000000}%
\pgfsetfillcolor{currentfill}%
\pgfsetlinewidth{0.803000pt}%
\definecolor{currentstroke}{rgb}{0.000000,0.000000,0.000000}%
\pgfsetstrokecolor{currentstroke}%
\pgfsetdash{}{0pt}%
\pgfsys@defobject{currentmarker}{\pgfqpoint{0.000000in}{-0.048611in}}{\pgfqpoint{0.000000in}{0.000000in}}{%
\pgfpathmoveto{\pgfqpoint{0.000000in}{0.000000in}}%
\pgfpathlineto{\pgfqpoint{0.000000in}{-0.048611in}}%
\pgfusepath{stroke,fill}%
}%
\begin{pgfscope}%
\pgfsys@transformshift{2.152727in}{0.528000in}%
\pgfsys@useobject{currentmarker}{}%
\end{pgfscope}%
\end{pgfscope}%
\begin{pgfscope}%
\definecolor{textcolor}{rgb}{0.000000,0.000000,0.000000}%
\pgfsetstrokecolor{textcolor}%
\pgfsetfillcolor{textcolor}%
\pgftext[x=2.152727in,y=0.430778in,,top]{\color{textcolor}\sffamily\fontsize{10.000000}{12.000000}\selectfont 30}%
\end{pgfscope}%
\begin{pgfscope}%
\pgfsetbuttcap%
\pgfsetroundjoin%
\definecolor{currentfill}{rgb}{0.000000,0.000000,0.000000}%
\pgfsetfillcolor{currentfill}%
\pgfsetlinewidth{0.803000pt}%
\definecolor{currentstroke}{rgb}{0.000000,0.000000,0.000000}%
\pgfsetstrokecolor{currentstroke}%
\pgfsetdash{}{0pt}%
\pgfsys@defobject{currentmarker}{\pgfqpoint{0.000000in}{-0.048611in}}{\pgfqpoint{0.000000in}{0.000000in}}{%
\pgfpathmoveto{\pgfqpoint{0.000000in}{0.000000in}}%
\pgfpathlineto{\pgfqpoint{0.000000in}{-0.048611in}}%
\pgfusepath{stroke,fill}%
}%
\begin{pgfscope}%
\pgfsys@transformshift{2.603636in}{0.528000in}%
\pgfsys@useobject{currentmarker}{}%
\end{pgfscope}%
\end{pgfscope}%
\begin{pgfscope}%
\definecolor{textcolor}{rgb}{0.000000,0.000000,0.000000}%
\pgfsetstrokecolor{textcolor}%
\pgfsetfillcolor{textcolor}%
\pgftext[x=2.603636in,y=0.430778in,,top]{\color{textcolor}\sffamily\fontsize{10.000000}{12.000000}\selectfont 40}%
\end{pgfscope}%
\begin{pgfscope}%
\pgfsetbuttcap%
\pgfsetroundjoin%
\definecolor{currentfill}{rgb}{0.000000,0.000000,0.000000}%
\pgfsetfillcolor{currentfill}%
\pgfsetlinewidth{0.803000pt}%
\definecolor{currentstroke}{rgb}{0.000000,0.000000,0.000000}%
\pgfsetstrokecolor{currentstroke}%
\pgfsetdash{}{0pt}%
\pgfsys@defobject{currentmarker}{\pgfqpoint{0.000000in}{-0.048611in}}{\pgfqpoint{0.000000in}{0.000000in}}{%
\pgfpathmoveto{\pgfqpoint{0.000000in}{0.000000in}}%
\pgfpathlineto{\pgfqpoint{0.000000in}{-0.048611in}}%
\pgfusepath{stroke,fill}%
}%
\begin{pgfscope}%
\pgfsys@transformshift{3.054545in}{0.528000in}%
\pgfsys@useobject{currentmarker}{}%
\end{pgfscope}%
\end{pgfscope}%
\begin{pgfscope}%
\definecolor{textcolor}{rgb}{0.000000,0.000000,0.000000}%
\pgfsetstrokecolor{textcolor}%
\pgfsetfillcolor{textcolor}%
\pgftext[x=3.054545in,y=0.430778in,,top]{\color{textcolor}\sffamily\fontsize{10.000000}{12.000000}\selectfont 50}%
\end{pgfscope}%
\begin{pgfscope}%
\pgfsetbuttcap%
\pgfsetroundjoin%
\definecolor{currentfill}{rgb}{0.000000,0.000000,0.000000}%
\pgfsetfillcolor{currentfill}%
\pgfsetlinewidth{0.803000pt}%
\definecolor{currentstroke}{rgb}{0.000000,0.000000,0.000000}%
\pgfsetstrokecolor{currentstroke}%
\pgfsetdash{}{0pt}%
\pgfsys@defobject{currentmarker}{\pgfqpoint{0.000000in}{-0.048611in}}{\pgfqpoint{0.000000in}{0.000000in}}{%
\pgfpathmoveto{\pgfqpoint{0.000000in}{0.000000in}}%
\pgfpathlineto{\pgfqpoint{0.000000in}{-0.048611in}}%
\pgfusepath{stroke,fill}%
}%
\begin{pgfscope}%
\pgfsys@transformshift{3.505455in}{0.528000in}%
\pgfsys@useobject{currentmarker}{}%
\end{pgfscope}%
\end{pgfscope}%
\begin{pgfscope}%
\definecolor{textcolor}{rgb}{0.000000,0.000000,0.000000}%
\pgfsetstrokecolor{textcolor}%
\pgfsetfillcolor{textcolor}%
\pgftext[x=3.505455in,y=0.430778in,,top]{\color{textcolor}\sffamily\fontsize{10.000000}{12.000000}\selectfont 60}%
\end{pgfscope}%
\begin{pgfscope}%
\pgfsetbuttcap%
\pgfsetroundjoin%
\definecolor{currentfill}{rgb}{0.000000,0.000000,0.000000}%
\pgfsetfillcolor{currentfill}%
\pgfsetlinewidth{0.803000pt}%
\definecolor{currentstroke}{rgb}{0.000000,0.000000,0.000000}%
\pgfsetstrokecolor{currentstroke}%
\pgfsetdash{}{0pt}%
\pgfsys@defobject{currentmarker}{\pgfqpoint{0.000000in}{-0.048611in}}{\pgfqpoint{0.000000in}{0.000000in}}{%
\pgfpathmoveto{\pgfqpoint{0.000000in}{0.000000in}}%
\pgfpathlineto{\pgfqpoint{0.000000in}{-0.048611in}}%
\pgfusepath{stroke,fill}%
}%
\begin{pgfscope}%
\pgfsys@transformshift{3.956364in}{0.528000in}%
\pgfsys@useobject{currentmarker}{}%
\end{pgfscope}%
\end{pgfscope}%
\begin{pgfscope}%
\definecolor{textcolor}{rgb}{0.000000,0.000000,0.000000}%
\pgfsetstrokecolor{textcolor}%
\pgfsetfillcolor{textcolor}%
\pgftext[x=3.956364in,y=0.430778in,,top]{\color{textcolor}\sffamily\fontsize{10.000000}{12.000000}\selectfont 70}%
\end{pgfscope}%
\begin{pgfscope}%
\pgfsetbuttcap%
\pgfsetroundjoin%
\definecolor{currentfill}{rgb}{0.000000,0.000000,0.000000}%
\pgfsetfillcolor{currentfill}%
\pgfsetlinewidth{0.803000pt}%
\definecolor{currentstroke}{rgb}{0.000000,0.000000,0.000000}%
\pgfsetstrokecolor{currentstroke}%
\pgfsetdash{}{0pt}%
\pgfsys@defobject{currentmarker}{\pgfqpoint{0.000000in}{-0.048611in}}{\pgfqpoint{0.000000in}{0.000000in}}{%
\pgfpathmoveto{\pgfqpoint{0.000000in}{0.000000in}}%
\pgfpathlineto{\pgfqpoint{0.000000in}{-0.048611in}}%
\pgfusepath{stroke,fill}%
}%
\begin{pgfscope}%
\pgfsys@transformshift{4.407273in}{0.528000in}%
\pgfsys@useobject{currentmarker}{}%
\end{pgfscope}%
\end{pgfscope}%
\begin{pgfscope}%
\definecolor{textcolor}{rgb}{0.000000,0.000000,0.000000}%
\pgfsetstrokecolor{textcolor}%
\pgfsetfillcolor{textcolor}%
\pgftext[x=4.407273in,y=0.430778in,,top]{\color{textcolor}\sffamily\fontsize{10.000000}{12.000000}\selectfont 80}%
\end{pgfscope}%
\begin{pgfscope}%
\pgfsetbuttcap%
\pgfsetroundjoin%
\definecolor{currentfill}{rgb}{0.000000,0.000000,0.000000}%
\pgfsetfillcolor{currentfill}%
\pgfsetlinewidth{0.803000pt}%
\definecolor{currentstroke}{rgb}{0.000000,0.000000,0.000000}%
\pgfsetstrokecolor{currentstroke}%
\pgfsetdash{}{0pt}%
\pgfsys@defobject{currentmarker}{\pgfqpoint{0.000000in}{-0.048611in}}{\pgfqpoint{0.000000in}{0.000000in}}{%
\pgfpathmoveto{\pgfqpoint{0.000000in}{0.000000in}}%
\pgfpathlineto{\pgfqpoint{0.000000in}{-0.048611in}}%
\pgfusepath{stroke,fill}%
}%
\begin{pgfscope}%
\pgfsys@transformshift{4.858182in}{0.528000in}%
\pgfsys@useobject{currentmarker}{}%
\end{pgfscope}%
\end{pgfscope}%
\begin{pgfscope}%
\definecolor{textcolor}{rgb}{0.000000,0.000000,0.000000}%
\pgfsetstrokecolor{textcolor}%
\pgfsetfillcolor{textcolor}%
\pgftext[x=4.858182in,y=0.430778in,,top]{\color{textcolor}\sffamily\fontsize{10.000000}{12.000000}\selectfont 90}%
\end{pgfscope}%
\begin{pgfscope}%
\pgfsetbuttcap%
\pgfsetroundjoin%
\definecolor{currentfill}{rgb}{0.000000,0.000000,0.000000}%
\pgfsetfillcolor{currentfill}%
\pgfsetlinewidth{0.803000pt}%
\definecolor{currentstroke}{rgb}{0.000000,0.000000,0.000000}%
\pgfsetstrokecolor{currentstroke}%
\pgfsetdash{}{0pt}%
\pgfsys@defobject{currentmarker}{\pgfqpoint{0.000000in}{-0.048611in}}{\pgfqpoint{0.000000in}{0.000000in}}{%
\pgfpathmoveto{\pgfqpoint{0.000000in}{0.000000in}}%
\pgfpathlineto{\pgfqpoint{0.000000in}{-0.048611in}}%
\pgfusepath{stroke,fill}%
}%
\begin{pgfscope}%
\pgfsys@transformshift{5.309091in}{0.528000in}%
\pgfsys@useobject{currentmarker}{}%
\end{pgfscope}%
\end{pgfscope}%
\begin{pgfscope}%
\definecolor{textcolor}{rgb}{0.000000,0.000000,0.000000}%
\pgfsetstrokecolor{textcolor}%
\pgfsetfillcolor{textcolor}%
\pgftext[x=5.309091in,y=0.430778in,,top]{\color{textcolor}\sffamily\fontsize{10.000000}{12.000000}\selectfont 100}%
\end{pgfscope}%
\begin{pgfscope}%
\definecolor{textcolor}{rgb}{0.000000,0.000000,0.000000}%
\pgfsetstrokecolor{textcolor}%
\pgfsetfillcolor{textcolor}%
\pgftext[x=3.280000in,y=0.240809in,,top]{\color{textcolor}\sffamily\fontsize{10.000000}{12.000000}\selectfont \(\displaystyle k\)}%
\end{pgfscope}%
\begin{pgfscope}%
\pgfsetbuttcap%
\pgfsetroundjoin%
\definecolor{currentfill}{rgb}{0.000000,0.000000,0.000000}%
\pgfsetfillcolor{currentfill}%
\pgfsetlinewidth{0.803000pt}%
\definecolor{currentstroke}{rgb}{0.000000,0.000000,0.000000}%
\pgfsetstrokecolor{currentstroke}%
\pgfsetdash{}{0pt}%
\pgfsys@defobject{currentmarker}{\pgfqpoint{-0.048611in}{0.000000in}}{\pgfqpoint{0.000000in}{0.000000in}}{%
\pgfpathmoveto{\pgfqpoint{0.000000in}{0.000000in}}%
\pgfpathlineto{\pgfqpoint{-0.048611in}{0.000000in}}%
\pgfusepath{stroke,fill}%
}%
\begin{pgfscope}%
\pgfsys@transformshift{0.800000in}{0.770667in}%
\pgfsys@useobject{currentmarker}{}%
\end{pgfscope}%
\end{pgfscope}%
\begin{pgfscope}%
\definecolor{textcolor}{rgb}{0.000000,0.000000,0.000000}%
\pgfsetstrokecolor{textcolor}%
\pgfsetfillcolor{textcolor}%
\pgftext[x=0.526047in,y=0.717905in,left,base]{\color{textcolor}\sffamily\fontsize{10.000000}{12.000000}\selectfont 10}%
\end{pgfscope}%
\begin{pgfscope}%
\pgfsetbuttcap%
\pgfsetroundjoin%
\definecolor{currentfill}{rgb}{0.000000,0.000000,0.000000}%
\pgfsetfillcolor{currentfill}%
\pgfsetlinewidth{0.803000pt}%
\definecolor{currentstroke}{rgb}{0.000000,0.000000,0.000000}%
\pgfsetstrokecolor{currentstroke}%
\pgfsetdash{}{0pt}%
\pgfsys@defobject{currentmarker}{\pgfqpoint{-0.048611in}{0.000000in}}{\pgfqpoint{0.000000in}{0.000000in}}{%
\pgfpathmoveto{\pgfqpoint{0.000000in}{0.000000in}}%
\pgfpathlineto{\pgfqpoint{-0.048611in}{0.000000in}}%
\pgfusepath{stroke,fill}%
}%
\begin{pgfscope}%
\pgfsys@transformshift{0.800000in}{1.144000in}%
\pgfsys@useobject{currentmarker}{}%
\end{pgfscope}%
\end{pgfscope}%
\begin{pgfscope}%
\definecolor{textcolor}{rgb}{0.000000,0.000000,0.000000}%
\pgfsetstrokecolor{textcolor}%
\pgfsetfillcolor{textcolor}%
\pgftext[x=0.526047in,y=1.091238in,left,base]{\color{textcolor}\sffamily\fontsize{10.000000}{12.000000}\selectfont 20}%
\end{pgfscope}%
\begin{pgfscope}%
\pgfsetbuttcap%
\pgfsetroundjoin%
\definecolor{currentfill}{rgb}{0.000000,0.000000,0.000000}%
\pgfsetfillcolor{currentfill}%
\pgfsetlinewidth{0.803000pt}%
\definecolor{currentstroke}{rgb}{0.000000,0.000000,0.000000}%
\pgfsetstrokecolor{currentstroke}%
\pgfsetdash{}{0pt}%
\pgfsys@defobject{currentmarker}{\pgfqpoint{-0.048611in}{0.000000in}}{\pgfqpoint{0.000000in}{0.000000in}}{%
\pgfpathmoveto{\pgfqpoint{0.000000in}{0.000000in}}%
\pgfpathlineto{\pgfqpoint{-0.048611in}{0.000000in}}%
\pgfusepath{stroke,fill}%
}%
\begin{pgfscope}%
\pgfsys@transformshift{0.800000in}{1.517333in}%
\pgfsys@useobject{currentmarker}{}%
\end{pgfscope}%
\end{pgfscope}%
\begin{pgfscope}%
\definecolor{textcolor}{rgb}{0.000000,0.000000,0.000000}%
\pgfsetstrokecolor{textcolor}%
\pgfsetfillcolor{textcolor}%
\pgftext[x=0.526047in,y=1.464572in,left,base]{\color{textcolor}\sffamily\fontsize{10.000000}{12.000000}\selectfont 30}%
\end{pgfscope}%
\begin{pgfscope}%
\pgfsetbuttcap%
\pgfsetroundjoin%
\definecolor{currentfill}{rgb}{0.000000,0.000000,0.000000}%
\pgfsetfillcolor{currentfill}%
\pgfsetlinewidth{0.803000pt}%
\definecolor{currentstroke}{rgb}{0.000000,0.000000,0.000000}%
\pgfsetstrokecolor{currentstroke}%
\pgfsetdash{}{0pt}%
\pgfsys@defobject{currentmarker}{\pgfqpoint{-0.048611in}{0.000000in}}{\pgfqpoint{0.000000in}{0.000000in}}{%
\pgfpathmoveto{\pgfqpoint{0.000000in}{0.000000in}}%
\pgfpathlineto{\pgfqpoint{-0.048611in}{0.000000in}}%
\pgfusepath{stroke,fill}%
}%
\begin{pgfscope}%
\pgfsys@transformshift{0.800000in}{1.890667in}%
\pgfsys@useobject{currentmarker}{}%
\end{pgfscope}%
\end{pgfscope}%
\begin{pgfscope}%
\definecolor{textcolor}{rgb}{0.000000,0.000000,0.000000}%
\pgfsetstrokecolor{textcolor}%
\pgfsetfillcolor{textcolor}%
\pgftext[x=0.526047in,y=1.837905in,left,base]{\color{textcolor}\sffamily\fontsize{10.000000}{12.000000}\selectfont 40}%
\end{pgfscope}%
\begin{pgfscope}%
\pgfsetbuttcap%
\pgfsetroundjoin%
\definecolor{currentfill}{rgb}{0.000000,0.000000,0.000000}%
\pgfsetfillcolor{currentfill}%
\pgfsetlinewidth{0.803000pt}%
\definecolor{currentstroke}{rgb}{0.000000,0.000000,0.000000}%
\pgfsetstrokecolor{currentstroke}%
\pgfsetdash{}{0pt}%
\pgfsys@defobject{currentmarker}{\pgfqpoint{-0.048611in}{0.000000in}}{\pgfqpoint{0.000000in}{0.000000in}}{%
\pgfpathmoveto{\pgfqpoint{0.000000in}{0.000000in}}%
\pgfpathlineto{\pgfqpoint{-0.048611in}{0.000000in}}%
\pgfusepath{stroke,fill}%
}%
\begin{pgfscope}%
\pgfsys@transformshift{0.800000in}{2.264000in}%
\pgfsys@useobject{currentmarker}{}%
\end{pgfscope}%
\end{pgfscope}%
\begin{pgfscope}%
\definecolor{textcolor}{rgb}{0.000000,0.000000,0.000000}%
\pgfsetstrokecolor{textcolor}%
\pgfsetfillcolor{textcolor}%
\pgftext[x=0.526047in,y=2.211238in,left,base]{\color{textcolor}\sffamily\fontsize{10.000000}{12.000000}\selectfont 50}%
\end{pgfscope}%
\begin{pgfscope}%
\pgfsetbuttcap%
\pgfsetroundjoin%
\definecolor{currentfill}{rgb}{0.000000,0.000000,0.000000}%
\pgfsetfillcolor{currentfill}%
\pgfsetlinewidth{0.803000pt}%
\definecolor{currentstroke}{rgb}{0.000000,0.000000,0.000000}%
\pgfsetstrokecolor{currentstroke}%
\pgfsetdash{}{0pt}%
\pgfsys@defobject{currentmarker}{\pgfqpoint{-0.048611in}{0.000000in}}{\pgfqpoint{0.000000in}{0.000000in}}{%
\pgfpathmoveto{\pgfqpoint{0.000000in}{0.000000in}}%
\pgfpathlineto{\pgfqpoint{-0.048611in}{0.000000in}}%
\pgfusepath{stroke,fill}%
}%
\begin{pgfscope}%
\pgfsys@transformshift{0.800000in}{2.637333in}%
\pgfsys@useobject{currentmarker}{}%
\end{pgfscope}%
\end{pgfscope}%
\begin{pgfscope}%
\definecolor{textcolor}{rgb}{0.000000,0.000000,0.000000}%
\pgfsetstrokecolor{textcolor}%
\pgfsetfillcolor{textcolor}%
\pgftext[x=0.526047in,y=2.584572in,left,base]{\color{textcolor}\sffamily\fontsize{10.000000}{12.000000}\selectfont 60}%
\end{pgfscope}%
\begin{pgfscope}%
\pgfsetbuttcap%
\pgfsetroundjoin%
\definecolor{currentfill}{rgb}{0.000000,0.000000,0.000000}%
\pgfsetfillcolor{currentfill}%
\pgfsetlinewidth{0.803000pt}%
\definecolor{currentstroke}{rgb}{0.000000,0.000000,0.000000}%
\pgfsetstrokecolor{currentstroke}%
\pgfsetdash{}{0pt}%
\pgfsys@defobject{currentmarker}{\pgfqpoint{-0.048611in}{0.000000in}}{\pgfqpoint{0.000000in}{0.000000in}}{%
\pgfpathmoveto{\pgfqpoint{0.000000in}{0.000000in}}%
\pgfpathlineto{\pgfqpoint{-0.048611in}{0.000000in}}%
\pgfusepath{stroke,fill}%
}%
\begin{pgfscope}%
\pgfsys@transformshift{0.800000in}{3.010667in}%
\pgfsys@useobject{currentmarker}{}%
\end{pgfscope}%
\end{pgfscope}%
\begin{pgfscope}%
\definecolor{textcolor}{rgb}{0.000000,0.000000,0.000000}%
\pgfsetstrokecolor{textcolor}%
\pgfsetfillcolor{textcolor}%
\pgftext[x=0.526047in,y=2.957905in,left,base]{\color{textcolor}\sffamily\fontsize{10.000000}{12.000000}\selectfont 70}%
\end{pgfscope}%
\begin{pgfscope}%
\pgfsetbuttcap%
\pgfsetroundjoin%
\definecolor{currentfill}{rgb}{0.000000,0.000000,0.000000}%
\pgfsetfillcolor{currentfill}%
\pgfsetlinewidth{0.803000pt}%
\definecolor{currentstroke}{rgb}{0.000000,0.000000,0.000000}%
\pgfsetstrokecolor{currentstroke}%
\pgfsetdash{}{0pt}%
\pgfsys@defobject{currentmarker}{\pgfqpoint{-0.048611in}{0.000000in}}{\pgfqpoint{0.000000in}{0.000000in}}{%
\pgfpathmoveto{\pgfqpoint{0.000000in}{0.000000in}}%
\pgfpathlineto{\pgfqpoint{-0.048611in}{0.000000in}}%
\pgfusepath{stroke,fill}%
}%
\begin{pgfscope}%
\pgfsys@transformshift{0.800000in}{3.384000in}%
\pgfsys@useobject{currentmarker}{}%
\end{pgfscope}%
\end{pgfscope}%
\begin{pgfscope}%
\definecolor{textcolor}{rgb}{0.000000,0.000000,0.000000}%
\pgfsetstrokecolor{textcolor}%
\pgfsetfillcolor{textcolor}%
\pgftext[x=0.526047in,y=3.331238in,left,base]{\color{textcolor}\sffamily\fontsize{10.000000}{12.000000}\selectfont 80}%
\end{pgfscope}%
\begin{pgfscope}%
\pgfsetbuttcap%
\pgfsetroundjoin%
\definecolor{currentfill}{rgb}{0.000000,0.000000,0.000000}%
\pgfsetfillcolor{currentfill}%
\pgfsetlinewidth{0.803000pt}%
\definecolor{currentstroke}{rgb}{0.000000,0.000000,0.000000}%
\pgfsetstrokecolor{currentstroke}%
\pgfsetdash{}{0pt}%
\pgfsys@defobject{currentmarker}{\pgfqpoint{-0.048611in}{0.000000in}}{\pgfqpoint{0.000000in}{0.000000in}}{%
\pgfpathmoveto{\pgfqpoint{0.000000in}{0.000000in}}%
\pgfpathlineto{\pgfqpoint{-0.048611in}{0.000000in}}%
\pgfusepath{stroke,fill}%
}%
\begin{pgfscope}%
\pgfsys@transformshift{0.800000in}{3.757333in}%
\pgfsys@useobject{currentmarker}{}%
\end{pgfscope}%
\end{pgfscope}%
\begin{pgfscope}%
\definecolor{textcolor}{rgb}{0.000000,0.000000,0.000000}%
\pgfsetstrokecolor{textcolor}%
\pgfsetfillcolor{textcolor}%
\pgftext[x=0.526047in,y=3.704572in,left,base]{\color{textcolor}\sffamily\fontsize{10.000000}{12.000000}\selectfont 90}%
\end{pgfscope}%
\begin{pgfscope}%
\pgfsetbuttcap%
\pgfsetroundjoin%
\definecolor{currentfill}{rgb}{0.000000,0.000000,0.000000}%
\pgfsetfillcolor{currentfill}%
\pgfsetlinewidth{0.803000pt}%
\definecolor{currentstroke}{rgb}{0.000000,0.000000,0.000000}%
\pgfsetstrokecolor{currentstroke}%
\pgfsetdash{}{0pt}%
\pgfsys@defobject{currentmarker}{\pgfqpoint{-0.048611in}{0.000000in}}{\pgfqpoint{0.000000in}{0.000000in}}{%
\pgfpathmoveto{\pgfqpoint{0.000000in}{0.000000in}}%
\pgfpathlineto{\pgfqpoint{-0.048611in}{0.000000in}}%
\pgfusepath{stroke,fill}%
}%
\begin{pgfscope}%
\pgfsys@transformshift{0.800000in}{4.130667in}%
\pgfsys@useobject{currentmarker}{}%
\end{pgfscope}%
\end{pgfscope}%
\begin{pgfscope}%
\definecolor{textcolor}{rgb}{0.000000,0.000000,0.000000}%
\pgfsetstrokecolor{textcolor}%
\pgfsetfillcolor{textcolor}%
\pgftext[x=0.437682in,y=4.077905in,left,base]{\color{textcolor}\sffamily\fontsize{10.000000}{12.000000}\selectfont 100}%
\end{pgfscope}%
\begin{pgfscope}%
\definecolor{textcolor}{rgb}{0.000000,0.000000,0.000000}%
\pgfsetstrokecolor{textcolor}%
\pgfsetfillcolor{textcolor}%
\pgftext[x=0.382126in,y=2.376000in,,bottom,rotate=90.000000]{\color{textcolor}\sffamily\fontsize{10.000000}{12.000000}\selectfont Number of GMRES iterations}%
\end{pgfscope}%
\begin{pgfscope}%
\pgfpathrectangle{\pgfqpoint{0.800000in}{0.528000in}}{\pgfqpoint{4.960000in}{3.696000in}}%
\pgfusepath{clip}%
\pgfsetbuttcap%
\pgfsetroundjoin%
\pgfsetlinewidth{1.505625pt}%
\definecolor{currentstroke}{rgb}{0.000000,0.000000,0.000000}%
\pgfsetstrokecolor{currentstroke}%
\pgfsetdash{{5.550000pt}{2.400000pt}}{0.000000pt}%
\pgfpathmoveto{\pgfqpoint{1.250909in}{0.770667in}}%
\pgfpathlineto{\pgfqpoint{1.701818in}{0.957333in}}%
\pgfpathlineto{\pgfqpoint{2.152727in}{1.144000in}}%
\pgfpathlineto{\pgfqpoint{2.603636in}{1.330667in}}%
\pgfpathlineto{\pgfqpoint{3.054545in}{1.517333in}}%
\pgfpathlineto{\pgfqpoint{3.505455in}{1.965333in}}%
\pgfpathlineto{\pgfqpoint{3.956364in}{2.376000in}}%
\pgfpathlineto{\pgfqpoint{4.407273in}{2.786667in}}%
\pgfpathlineto{\pgfqpoint{4.858182in}{3.421333in}}%
\pgfpathlineto{\pgfqpoint{5.309091in}{4.056000in}}%
\pgfusepath{stroke}%
\end{pgfscope}%
\begin{pgfscope}%
\pgfpathrectangle{\pgfqpoint{0.800000in}{0.528000in}}{\pgfqpoint{4.960000in}{3.696000in}}%
\pgfusepath{clip}%
\pgfsetbuttcap%
\pgfsetroundjoin%
\definecolor{currentfill}{rgb}{0.000000,0.000000,0.000000}%
\pgfsetfillcolor{currentfill}%
\pgfsetlinewidth{1.003750pt}%
\definecolor{currentstroke}{rgb}{0.000000,0.000000,0.000000}%
\pgfsetstrokecolor{currentstroke}%
\pgfsetdash{}{0pt}%
\pgfsys@defobject{currentmarker}{\pgfqpoint{-0.041667in}{-0.041667in}}{\pgfqpoint{0.041667in}{0.041667in}}{%
\pgfpathmoveto{\pgfqpoint{0.000000in}{-0.041667in}}%
\pgfpathcurveto{\pgfqpoint{0.011050in}{-0.041667in}}{\pgfqpoint{0.021649in}{-0.037276in}}{\pgfqpoint{0.029463in}{-0.029463in}}%
\pgfpathcurveto{\pgfqpoint{0.037276in}{-0.021649in}}{\pgfqpoint{0.041667in}{-0.011050in}}{\pgfqpoint{0.041667in}{0.000000in}}%
\pgfpathcurveto{\pgfqpoint{0.041667in}{0.011050in}}{\pgfqpoint{0.037276in}{0.021649in}}{\pgfqpoint{0.029463in}{0.029463in}}%
\pgfpathcurveto{\pgfqpoint{0.021649in}{0.037276in}}{\pgfqpoint{0.011050in}{0.041667in}}{\pgfqpoint{0.000000in}{0.041667in}}%
\pgfpathcurveto{\pgfqpoint{-0.011050in}{0.041667in}}{\pgfqpoint{-0.021649in}{0.037276in}}{\pgfqpoint{-0.029463in}{0.029463in}}%
\pgfpathcurveto{\pgfqpoint{-0.037276in}{0.021649in}}{\pgfqpoint{-0.041667in}{0.011050in}}{\pgfqpoint{-0.041667in}{0.000000in}}%
\pgfpathcurveto{\pgfqpoint{-0.041667in}{-0.011050in}}{\pgfqpoint{-0.037276in}{-0.021649in}}{\pgfqpoint{-0.029463in}{-0.029463in}}%
\pgfpathcurveto{\pgfqpoint{-0.021649in}{-0.037276in}}{\pgfqpoint{-0.011050in}{-0.041667in}}{\pgfqpoint{0.000000in}{-0.041667in}}%
\pgfpathclose%
\pgfusepath{stroke,fill}%
}%
\begin{pgfscope}%
\pgfsys@transformshift{1.250909in}{0.770667in}%
\pgfsys@useobject{currentmarker}{}%
\end{pgfscope}%
\begin{pgfscope}%
\pgfsys@transformshift{1.701818in}{0.957333in}%
\pgfsys@useobject{currentmarker}{}%
\end{pgfscope}%
\begin{pgfscope}%
\pgfsys@transformshift{2.152727in}{1.144000in}%
\pgfsys@useobject{currentmarker}{}%
\end{pgfscope}%
\begin{pgfscope}%
\pgfsys@transformshift{2.603636in}{1.330667in}%
\pgfsys@useobject{currentmarker}{}%
\end{pgfscope}%
\begin{pgfscope}%
\pgfsys@transformshift{3.054545in}{1.517333in}%
\pgfsys@useobject{currentmarker}{}%
\end{pgfscope}%
\begin{pgfscope}%
\pgfsys@transformshift{3.505455in}{1.965333in}%
\pgfsys@useobject{currentmarker}{}%
\end{pgfscope}%
\begin{pgfscope}%
\pgfsys@transformshift{3.956364in}{2.376000in}%
\pgfsys@useobject{currentmarker}{}%
\end{pgfscope}%
\begin{pgfscope}%
\pgfsys@transformshift{4.407273in}{2.786667in}%
\pgfsys@useobject{currentmarker}{}%
\end{pgfscope}%
\begin{pgfscope}%
\pgfsys@transformshift{4.858182in}{3.421333in}%
\pgfsys@useobject{currentmarker}{}%
\end{pgfscope}%
\begin{pgfscope}%
\pgfsys@transformshift{5.309091in}{4.056000in}%
\pgfsys@useobject{currentmarker}{}%
\end{pgfscope}%
\end{pgfscope}%
\begin{pgfscope}%
\pgfpathrectangle{\pgfqpoint{0.800000in}{0.528000in}}{\pgfqpoint{4.960000in}{3.696000in}}%
\pgfusepath{clip}%
\pgfsetbuttcap%
\pgfsetroundjoin%
\pgfsetlinewidth{1.505625pt}%
\definecolor{currentstroke}{rgb}{0.000000,0.000000,0.000000}%
\pgfsetstrokecolor{currentstroke}%
\pgfsetdash{{5.550000pt}{2.400000pt}}{0.000000pt}%
\pgfpathmoveto{\pgfqpoint{1.250909in}{0.770667in}}%
\pgfpathlineto{\pgfqpoint{1.701818in}{0.882667in}}%
\pgfpathlineto{\pgfqpoint{2.152727in}{0.994667in}}%
\pgfpathlineto{\pgfqpoint{2.603636in}{1.106667in}}%
\pgfpathlineto{\pgfqpoint{3.054545in}{1.218667in}}%
\pgfpathlineto{\pgfqpoint{3.505455in}{1.330667in}}%
\pgfpathlineto{\pgfqpoint{3.956364in}{1.442667in}}%
\pgfpathlineto{\pgfqpoint{4.407273in}{1.554667in}}%
\pgfpathlineto{\pgfqpoint{4.858182in}{1.778667in}}%
\pgfpathlineto{\pgfqpoint{5.309091in}{1.928000in}}%
\pgfusepath{stroke}%
\end{pgfscope}%
\begin{pgfscope}%
\pgfpathrectangle{\pgfqpoint{0.800000in}{0.528000in}}{\pgfqpoint{4.960000in}{3.696000in}}%
\pgfusepath{clip}%
\pgfsetbuttcap%
\pgfsetmiterjoin%
\definecolor{currentfill}{rgb}{0.000000,0.000000,0.000000}%
\pgfsetfillcolor{currentfill}%
\pgfsetlinewidth{1.003750pt}%
\definecolor{currentstroke}{rgb}{0.000000,0.000000,0.000000}%
\pgfsetstrokecolor{currentstroke}%
\pgfsetdash{}{0pt}%
\pgfsys@defobject{currentmarker}{\pgfqpoint{-0.041667in}{-0.041667in}}{\pgfqpoint{0.041667in}{0.041667in}}{%
\pgfpathmoveto{\pgfqpoint{-0.000000in}{-0.041667in}}%
\pgfpathlineto{\pgfqpoint{0.041667in}{0.041667in}}%
\pgfpathlineto{\pgfqpoint{-0.041667in}{0.041667in}}%
\pgfpathclose%
\pgfusepath{stroke,fill}%
}%
\begin{pgfscope}%
\pgfsys@transformshift{1.250909in}{0.770667in}%
\pgfsys@useobject{currentmarker}{}%
\end{pgfscope}%
\begin{pgfscope}%
\pgfsys@transformshift{1.701818in}{0.882667in}%
\pgfsys@useobject{currentmarker}{}%
\end{pgfscope}%
\begin{pgfscope}%
\pgfsys@transformshift{2.152727in}{0.994667in}%
\pgfsys@useobject{currentmarker}{}%
\end{pgfscope}%
\begin{pgfscope}%
\pgfsys@transformshift{2.603636in}{1.106667in}%
\pgfsys@useobject{currentmarker}{}%
\end{pgfscope}%
\begin{pgfscope}%
\pgfsys@transformshift{3.054545in}{1.218667in}%
\pgfsys@useobject{currentmarker}{}%
\end{pgfscope}%
\begin{pgfscope}%
\pgfsys@transformshift{3.505455in}{1.330667in}%
\pgfsys@useobject{currentmarker}{}%
\end{pgfscope}%
\begin{pgfscope}%
\pgfsys@transformshift{3.956364in}{1.442667in}%
\pgfsys@useobject{currentmarker}{}%
\end{pgfscope}%
\begin{pgfscope}%
\pgfsys@transformshift{4.407273in}{1.554667in}%
\pgfsys@useobject{currentmarker}{}%
\end{pgfscope}%
\begin{pgfscope}%
\pgfsys@transformshift{4.858182in}{1.778667in}%
\pgfsys@useobject{currentmarker}{}%
\end{pgfscope}%
\begin{pgfscope}%
\pgfsys@transformshift{5.309091in}{1.928000in}%
\pgfsys@useobject{currentmarker}{}%
\end{pgfscope}%
\end{pgfscope}%
\begin{pgfscope}%
\pgfpathrectangle{\pgfqpoint{0.800000in}{0.528000in}}{\pgfqpoint{4.960000in}{3.696000in}}%
\pgfusepath{clip}%
\pgfsetbuttcap%
\pgfsetroundjoin%
\pgfsetlinewidth{1.505625pt}%
\definecolor{currentstroke}{rgb}{0.000000,0.000000,0.000000}%
\pgfsetstrokecolor{currentstroke}%
\pgfsetdash{{5.550000pt}{2.400000pt}}{0.000000pt}%
\pgfpathmoveto{\pgfqpoint{1.250909in}{0.733333in}}%
\pgfpathlineto{\pgfqpoint{1.701818in}{0.808000in}}%
\pgfpathlineto{\pgfqpoint{2.152727in}{0.882667in}}%
\pgfpathlineto{\pgfqpoint{2.603636in}{0.920000in}}%
\pgfpathlineto{\pgfqpoint{3.054545in}{0.994667in}}%
\pgfpathlineto{\pgfqpoint{3.505455in}{1.032000in}}%
\pgfpathlineto{\pgfqpoint{3.956364in}{1.106667in}}%
\pgfpathlineto{\pgfqpoint{4.407273in}{1.106667in}}%
\pgfpathlineto{\pgfqpoint{4.858182in}{1.181333in}}%
\pgfpathlineto{\pgfqpoint{5.309091in}{1.218667in}}%
\pgfusepath{stroke}%
\end{pgfscope}%
\begin{pgfscope}%
\pgfpathrectangle{\pgfqpoint{0.800000in}{0.528000in}}{\pgfqpoint{4.960000in}{3.696000in}}%
\pgfusepath{clip}%
\pgfsetbuttcap%
\pgfsetmiterjoin%
\definecolor{currentfill}{rgb}{0.000000,0.000000,0.000000}%
\pgfsetfillcolor{currentfill}%
\pgfsetlinewidth{1.003750pt}%
\definecolor{currentstroke}{rgb}{0.000000,0.000000,0.000000}%
\pgfsetstrokecolor{currentstroke}%
\pgfsetdash{}{0pt}%
\pgfsys@defobject{currentmarker}{\pgfqpoint{-0.041667in}{-0.041667in}}{\pgfqpoint{0.041667in}{0.041667in}}{%
\pgfpathmoveto{\pgfqpoint{-0.020833in}{-0.041667in}}%
\pgfpathlineto{\pgfqpoint{0.000000in}{-0.020833in}}%
\pgfpathlineto{\pgfqpoint{0.020833in}{-0.041667in}}%
\pgfpathlineto{\pgfqpoint{0.041667in}{-0.020833in}}%
\pgfpathlineto{\pgfqpoint{0.020833in}{0.000000in}}%
\pgfpathlineto{\pgfqpoint{0.041667in}{0.020833in}}%
\pgfpathlineto{\pgfqpoint{0.020833in}{0.041667in}}%
\pgfpathlineto{\pgfqpoint{0.000000in}{0.020833in}}%
\pgfpathlineto{\pgfqpoint{-0.020833in}{0.041667in}}%
\pgfpathlineto{\pgfqpoint{-0.041667in}{0.020833in}}%
\pgfpathlineto{\pgfqpoint{-0.020833in}{0.000000in}}%
\pgfpathlineto{\pgfqpoint{-0.041667in}{-0.020833in}}%
\pgfpathclose%
\pgfusepath{stroke,fill}%
}%
\begin{pgfscope}%
\pgfsys@transformshift{1.250909in}{0.733333in}%
\pgfsys@useobject{currentmarker}{}%
\end{pgfscope}%
\begin{pgfscope}%
\pgfsys@transformshift{1.701818in}{0.808000in}%
\pgfsys@useobject{currentmarker}{}%
\end{pgfscope}%
\begin{pgfscope}%
\pgfsys@transformshift{2.152727in}{0.882667in}%
\pgfsys@useobject{currentmarker}{}%
\end{pgfscope}%
\begin{pgfscope}%
\pgfsys@transformshift{2.603636in}{0.920000in}%
\pgfsys@useobject{currentmarker}{}%
\end{pgfscope}%
\begin{pgfscope}%
\pgfsys@transformshift{3.054545in}{0.994667in}%
\pgfsys@useobject{currentmarker}{}%
\end{pgfscope}%
\begin{pgfscope}%
\pgfsys@transformshift{3.505455in}{1.032000in}%
\pgfsys@useobject{currentmarker}{}%
\end{pgfscope}%
\begin{pgfscope}%
\pgfsys@transformshift{3.956364in}{1.106667in}%
\pgfsys@useobject{currentmarker}{}%
\end{pgfscope}%
\begin{pgfscope}%
\pgfsys@transformshift{4.407273in}{1.106667in}%
\pgfsys@useobject{currentmarker}{}%
\end{pgfscope}%
\begin{pgfscope}%
\pgfsys@transformshift{4.858182in}{1.181333in}%
\pgfsys@useobject{currentmarker}{}%
\end{pgfscope}%
\begin{pgfscope}%
\pgfsys@transformshift{5.309091in}{1.218667in}%
\pgfsys@useobject{currentmarker}{}%
\end{pgfscope}%
\end{pgfscope}%
\begin{pgfscope}%
\pgfpathrectangle{\pgfqpoint{0.800000in}{0.528000in}}{\pgfqpoint{4.960000in}{3.696000in}}%
\pgfusepath{clip}%
\pgfsetbuttcap%
\pgfsetroundjoin%
\pgfsetlinewidth{1.505625pt}%
\definecolor{currentstroke}{rgb}{0.000000,0.000000,0.000000}%
\pgfsetstrokecolor{currentstroke}%
\pgfsetdash{{5.550000pt}{2.400000pt}}{0.000000pt}%
\pgfpathmoveto{\pgfqpoint{1.250909in}{0.696000in}}%
\pgfpathlineto{\pgfqpoint{1.701818in}{0.733333in}}%
\pgfpathlineto{\pgfqpoint{2.152727in}{0.770667in}}%
\pgfpathlineto{\pgfqpoint{2.603636in}{0.808000in}}%
\pgfpathlineto{\pgfqpoint{3.054545in}{0.845333in}}%
\pgfpathlineto{\pgfqpoint{3.505455in}{0.882667in}}%
\pgfpathlineto{\pgfqpoint{3.956364in}{0.882667in}}%
\pgfpathlineto{\pgfqpoint{4.407273in}{0.920000in}}%
\pgfpathlineto{\pgfqpoint{4.858182in}{0.920000in}}%
\pgfpathlineto{\pgfqpoint{5.309091in}{0.920000in}}%
\pgfusepath{stroke}%
\end{pgfscope}%
\begin{pgfscope}%
\pgfpathrectangle{\pgfqpoint{0.800000in}{0.528000in}}{\pgfqpoint{4.960000in}{3.696000in}}%
\pgfusepath{clip}%
\pgfsetbuttcap%
\pgfsetmiterjoin%
\definecolor{currentfill}{rgb}{0.000000,0.000000,0.000000}%
\pgfsetfillcolor{currentfill}%
\pgfsetlinewidth{1.003750pt}%
\definecolor{currentstroke}{rgb}{0.000000,0.000000,0.000000}%
\pgfsetstrokecolor{currentstroke}%
\pgfsetdash{}{0pt}%
\pgfsys@defobject{currentmarker}{\pgfqpoint{-0.035355in}{-0.058926in}}{\pgfqpoint{0.035355in}{0.058926in}}{%
\pgfpathmoveto{\pgfqpoint{-0.000000in}{-0.058926in}}%
\pgfpathlineto{\pgfqpoint{0.035355in}{0.000000in}}%
\pgfpathlineto{\pgfqpoint{0.000000in}{0.058926in}}%
\pgfpathlineto{\pgfqpoint{-0.035355in}{0.000000in}}%
\pgfpathclose%
\pgfusepath{stroke,fill}%
}%
\begin{pgfscope}%
\pgfsys@transformshift{1.250909in}{0.696000in}%
\pgfsys@useobject{currentmarker}{}%
\end{pgfscope}%
\begin{pgfscope}%
\pgfsys@transformshift{1.701818in}{0.733333in}%
\pgfsys@useobject{currentmarker}{}%
\end{pgfscope}%
\begin{pgfscope}%
\pgfsys@transformshift{2.152727in}{0.770667in}%
\pgfsys@useobject{currentmarker}{}%
\end{pgfscope}%
\begin{pgfscope}%
\pgfsys@transformshift{2.603636in}{0.808000in}%
\pgfsys@useobject{currentmarker}{}%
\end{pgfscope}%
\begin{pgfscope}%
\pgfsys@transformshift{3.054545in}{0.845333in}%
\pgfsys@useobject{currentmarker}{}%
\end{pgfscope}%
\begin{pgfscope}%
\pgfsys@transformshift{3.505455in}{0.882667in}%
\pgfsys@useobject{currentmarker}{}%
\end{pgfscope}%
\begin{pgfscope}%
\pgfsys@transformshift{3.956364in}{0.882667in}%
\pgfsys@useobject{currentmarker}{}%
\end{pgfscope}%
\begin{pgfscope}%
\pgfsys@transformshift{4.407273in}{0.920000in}%
\pgfsys@useobject{currentmarker}{}%
\end{pgfscope}%
\begin{pgfscope}%
\pgfsys@transformshift{4.858182in}{0.920000in}%
\pgfsys@useobject{currentmarker}{}%
\end{pgfscope}%
\begin{pgfscope}%
\pgfsys@transformshift{5.309091in}{0.920000in}%
\pgfsys@useobject{currentmarker}{}%
\end{pgfscope}%
\end{pgfscope}%
\begin{pgfscope}%
\pgfsetrectcap%
\pgfsetmiterjoin%
\pgfsetlinewidth{0.803000pt}%
\definecolor{currentstroke}{rgb}{0.000000,0.000000,0.000000}%
\pgfsetstrokecolor{currentstroke}%
\pgfsetdash{}{0pt}%
\pgfpathmoveto{\pgfqpoint{0.800000in}{0.528000in}}%
\pgfpathlineto{\pgfqpoint{0.800000in}{4.224000in}}%
\pgfusepath{stroke}%
\end{pgfscope}%
\begin{pgfscope}%
\pgfsetrectcap%
\pgfsetmiterjoin%
\pgfsetlinewidth{0.803000pt}%
\definecolor{currentstroke}{rgb}{0.000000,0.000000,0.000000}%
\pgfsetstrokecolor{currentstroke}%
\pgfsetdash{}{0pt}%
\pgfpathmoveto{\pgfqpoint{5.760000in}{0.528000in}}%
\pgfpathlineto{\pgfqpoint{5.760000in}{4.224000in}}%
\pgfusepath{stroke}%
\end{pgfscope}%
\begin{pgfscope}%
\pgfsetrectcap%
\pgfsetmiterjoin%
\pgfsetlinewidth{0.803000pt}%
\definecolor{currentstroke}{rgb}{0.000000,0.000000,0.000000}%
\pgfsetstrokecolor{currentstroke}%
\pgfsetdash{}{0pt}%
\pgfpathmoveto{\pgfqpoint{0.800000in}{0.528000in}}%
\pgfpathlineto{\pgfqpoint{5.760000in}{0.528000in}}%
\pgfusepath{stroke}%
\end{pgfscope}%
\begin{pgfscope}%
\pgfsetrectcap%
\pgfsetmiterjoin%
\pgfsetlinewidth{0.803000pt}%
\definecolor{currentstroke}{rgb}{0.000000,0.000000,0.000000}%
\pgfsetstrokecolor{currentstroke}%
\pgfsetdash{}{0pt}%
\pgfpathmoveto{\pgfqpoint{0.800000in}{4.224000in}}%
\pgfpathlineto{\pgfqpoint{5.760000in}{4.224000in}}%
\pgfusepath{stroke}%
\end{pgfscope}%
\begin{pgfscope}%
\pgfsetbuttcap%
\pgfsetmiterjoin%
\definecolor{currentfill}{rgb}{1.000000,1.000000,1.000000}%
\pgfsetfillcolor{currentfill}%
\pgfsetfillopacity{0.800000}%
\pgfsetlinewidth{1.003750pt}%
\definecolor{currentstroke}{rgb}{0.800000,0.800000,0.800000}%
\pgfsetstrokecolor{currentstroke}%
\pgfsetstrokeopacity{0.800000}%
\pgfsetdash{}{0pt}%
\pgfpathmoveto{\pgfqpoint{0.897222in}{3.297460in}}%
\pgfpathlineto{\pgfqpoint{1.790209in}{3.297460in}}%
\pgfpathquadraticcurveto{\pgfqpoint{1.817987in}{3.297460in}}{\pgfqpoint{1.817987in}{3.325238in}}%
\pgfpathlineto{\pgfqpoint{1.817987in}{4.126778in}}%
\pgfpathquadraticcurveto{\pgfqpoint{1.817987in}{4.154556in}}{\pgfqpoint{1.790209in}{4.154556in}}%
\pgfpathlineto{\pgfqpoint{0.897222in}{4.154556in}}%
\pgfpathquadraticcurveto{\pgfqpoint{0.869444in}{4.154556in}}{\pgfqpoint{0.869444in}{4.126778in}}%
\pgfpathlineto{\pgfqpoint{0.869444in}{3.325238in}}%
\pgfpathquadraticcurveto{\pgfqpoint{0.869444in}{3.297460in}}{\pgfqpoint{0.897222in}{3.297460in}}%
\pgfpathclose%
\pgfusepath{stroke,fill}%
\end{pgfscope}%
\begin{pgfscope}%
\pgfsetbuttcap%
\pgfsetroundjoin%
\pgfsetlinewidth{1.505625pt}%
\definecolor{currentstroke}{rgb}{0.000000,0.000000,0.000000}%
\pgfsetstrokecolor{currentstroke}%
\pgfsetdash{{5.550000pt}{2.400000pt}}{0.000000pt}%
\pgfpathmoveto{\pgfqpoint{0.925000in}{4.042088in}}%
\pgfpathlineto{\pgfqpoint{1.202778in}{4.042088in}}%
\pgfusepath{stroke}%
\end{pgfscope}%
\begin{pgfscope}%
\pgfsetbuttcap%
\pgfsetroundjoin%
\definecolor{currentfill}{rgb}{0.000000,0.000000,0.000000}%
\pgfsetfillcolor{currentfill}%
\pgfsetlinewidth{1.003750pt}%
\definecolor{currentstroke}{rgb}{0.000000,0.000000,0.000000}%
\pgfsetstrokecolor{currentstroke}%
\pgfsetdash{}{0pt}%
\pgfsys@defobject{currentmarker}{\pgfqpoint{-0.041667in}{-0.041667in}}{\pgfqpoint{0.041667in}{0.041667in}}{%
\pgfpathmoveto{\pgfqpoint{0.000000in}{-0.041667in}}%
\pgfpathcurveto{\pgfqpoint{0.011050in}{-0.041667in}}{\pgfqpoint{0.021649in}{-0.037276in}}{\pgfqpoint{0.029463in}{-0.029463in}}%
\pgfpathcurveto{\pgfqpoint{0.037276in}{-0.021649in}}{\pgfqpoint{0.041667in}{-0.011050in}}{\pgfqpoint{0.041667in}{0.000000in}}%
\pgfpathcurveto{\pgfqpoint{0.041667in}{0.011050in}}{\pgfqpoint{0.037276in}{0.021649in}}{\pgfqpoint{0.029463in}{0.029463in}}%
\pgfpathcurveto{\pgfqpoint{0.021649in}{0.037276in}}{\pgfqpoint{0.011050in}{0.041667in}}{\pgfqpoint{0.000000in}{0.041667in}}%
\pgfpathcurveto{\pgfqpoint{-0.011050in}{0.041667in}}{\pgfqpoint{-0.021649in}{0.037276in}}{\pgfqpoint{-0.029463in}{0.029463in}}%
\pgfpathcurveto{\pgfqpoint{-0.037276in}{0.021649in}}{\pgfqpoint{-0.041667in}{0.011050in}}{\pgfqpoint{-0.041667in}{0.000000in}}%
\pgfpathcurveto{\pgfqpoint{-0.041667in}{-0.011050in}}{\pgfqpoint{-0.037276in}{-0.021649in}}{\pgfqpoint{-0.029463in}{-0.029463in}}%
\pgfpathcurveto{\pgfqpoint{-0.021649in}{-0.037276in}}{\pgfqpoint{-0.011050in}{-0.041667in}}{\pgfqpoint{0.000000in}{-0.041667in}}%
\pgfpathclose%
\pgfusepath{stroke,fill}%
}%
\begin{pgfscope}%
\pgfsys@transformshift{1.063889in}{4.042088in}%
\pgfsys@useobject{currentmarker}{}%
\end{pgfscope}%
\end{pgfscope}%
\begin{pgfscope}%
\definecolor{textcolor}{rgb}{0.000000,0.000000,0.000000}%
\pgfsetstrokecolor{textcolor}%
\pgfsetfillcolor{textcolor}%
\pgftext[x=1.313889in,y=3.993477in,left,base]{\color{textcolor}\sffamily\fontsize{10.000000}{12.000000}\selectfont \(\displaystyle \beta = 0.4\)}%
\end{pgfscope}%
\begin{pgfscope}%
\pgfsetbuttcap%
\pgfsetroundjoin%
\pgfsetlinewidth{1.505625pt}%
\definecolor{currentstroke}{rgb}{0.000000,0.000000,0.000000}%
\pgfsetstrokecolor{currentstroke}%
\pgfsetdash{{5.550000pt}{2.400000pt}}{0.000000pt}%
\pgfpathmoveto{\pgfqpoint{0.925000in}{3.838231in}}%
\pgfpathlineto{\pgfqpoint{1.202778in}{3.838231in}}%
\pgfusepath{stroke}%
\end{pgfscope}%
\begin{pgfscope}%
\pgfsetbuttcap%
\pgfsetmiterjoin%
\definecolor{currentfill}{rgb}{0.000000,0.000000,0.000000}%
\pgfsetfillcolor{currentfill}%
\pgfsetlinewidth{1.003750pt}%
\definecolor{currentstroke}{rgb}{0.000000,0.000000,0.000000}%
\pgfsetstrokecolor{currentstroke}%
\pgfsetdash{}{0pt}%
\pgfsys@defobject{currentmarker}{\pgfqpoint{-0.041667in}{-0.041667in}}{\pgfqpoint{0.041667in}{0.041667in}}{%
\pgfpathmoveto{\pgfqpoint{-0.000000in}{-0.041667in}}%
\pgfpathlineto{\pgfqpoint{0.041667in}{0.041667in}}%
\pgfpathlineto{\pgfqpoint{-0.041667in}{0.041667in}}%
\pgfpathclose%
\pgfusepath{stroke,fill}%
}%
\begin{pgfscope}%
\pgfsys@transformshift{1.063889in}{3.838231in}%
\pgfsys@useobject{currentmarker}{}%
\end{pgfscope}%
\end{pgfscope}%
\begin{pgfscope}%
\definecolor{textcolor}{rgb}{0.000000,0.000000,0.000000}%
\pgfsetstrokecolor{textcolor}%
\pgfsetfillcolor{textcolor}%
\pgftext[x=1.313889in,y=3.789620in,left,base]{\color{textcolor}\sffamily\fontsize{10.000000}{12.000000}\selectfont \(\displaystyle \beta = 0.5\)}%
\end{pgfscope}%
\begin{pgfscope}%
\pgfsetbuttcap%
\pgfsetroundjoin%
\pgfsetlinewidth{1.505625pt}%
\definecolor{currentstroke}{rgb}{0.000000,0.000000,0.000000}%
\pgfsetstrokecolor{currentstroke}%
\pgfsetdash{{5.550000pt}{2.400000pt}}{0.000000pt}%
\pgfpathmoveto{\pgfqpoint{0.925000in}{3.634374in}}%
\pgfpathlineto{\pgfqpoint{1.202778in}{3.634374in}}%
\pgfusepath{stroke}%
\end{pgfscope}%
\begin{pgfscope}%
\pgfsetbuttcap%
\pgfsetmiterjoin%
\definecolor{currentfill}{rgb}{0.000000,0.000000,0.000000}%
\pgfsetfillcolor{currentfill}%
\pgfsetlinewidth{1.003750pt}%
\definecolor{currentstroke}{rgb}{0.000000,0.000000,0.000000}%
\pgfsetstrokecolor{currentstroke}%
\pgfsetdash{}{0pt}%
\pgfsys@defobject{currentmarker}{\pgfqpoint{-0.041667in}{-0.041667in}}{\pgfqpoint{0.041667in}{0.041667in}}{%
\pgfpathmoveto{\pgfqpoint{-0.020833in}{-0.041667in}}%
\pgfpathlineto{\pgfqpoint{0.000000in}{-0.020833in}}%
\pgfpathlineto{\pgfqpoint{0.020833in}{-0.041667in}}%
\pgfpathlineto{\pgfqpoint{0.041667in}{-0.020833in}}%
\pgfpathlineto{\pgfqpoint{0.020833in}{0.000000in}}%
\pgfpathlineto{\pgfqpoint{0.041667in}{0.020833in}}%
\pgfpathlineto{\pgfqpoint{0.020833in}{0.041667in}}%
\pgfpathlineto{\pgfqpoint{0.000000in}{0.020833in}}%
\pgfpathlineto{\pgfqpoint{-0.020833in}{0.041667in}}%
\pgfpathlineto{\pgfqpoint{-0.041667in}{0.020833in}}%
\pgfpathlineto{\pgfqpoint{-0.020833in}{0.000000in}}%
\pgfpathlineto{\pgfqpoint{-0.041667in}{-0.020833in}}%
\pgfpathclose%
\pgfusepath{stroke,fill}%
}%
\begin{pgfscope}%
\pgfsys@transformshift{1.063889in}{3.634374in}%
\pgfsys@useobject{currentmarker}{}%
\end{pgfscope}%
\end{pgfscope}%
\begin{pgfscope}%
\definecolor{textcolor}{rgb}{0.000000,0.000000,0.000000}%
\pgfsetstrokecolor{textcolor}%
\pgfsetfillcolor{textcolor}%
\pgftext[x=1.313889in,y=3.585762in,left,base]{\color{textcolor}\sffamily\fontsize{10.000000}{12.000000}\selectfont \(\displaystyle \beta = 0.6\)}%
\end{pgfscope}%
\begin{pgfscope}%
\pgfsetbuttcap%
\pgfsetroundjoin%
\pgfsetlinewidth{1.505625pt}%
\definecolor{currentstroke}{rgb}{0.000000,0.000000,0.000000}%
\pgfsetstrokecolor{currentstroke}%
\pgfsetdash{{5.550000pt}{2.400000pt}}{0.000000pt}%
\pgfpathmoveto{\pgfqpoint{0.925000in}{3.430516in}}%
\pgfpathlineto{\pgfqpoint{1.202778in}{3.430516in}}%
\pgfusepath{stroke}%
\end{pgfscope}%
\begin{pgfscope}%
\pgfsetbuttcap%
\pgfsetmiterjoin%
\definecolor{currentfill}{rgb}{0.000000,0.000000,0.000000}%
\pgfsetfillcolor{currentfill}%
\pgfsetlinewidth{1.003750pt}%
\definecolor{currentstroke}{rgb}{0.000000,0.000000,0.000000}%
\pgfsetstrokecolor{currentstroke}%
\pgfsetdash{}{0pt}%
\pgfsys@defobject{currentmarker}{\pgfqpoint{-0.035355in}{-0.058926in}}{\pgfqpoint{0.035355in}{0.058926in}}{%
\pgfpathmoveto{\pgfqpoint{-0.000000in}{-0.058926in}}%
\pgfpathlineto{\pgfqpoint{0.035355in}{0.000000in}}%
\pgfpathlineto{\pgfqpoint{0.000000in}{0.058926in}}%
\pgfpathlineto{\pgfqpoint{-0.035355in}{0.000000in}}%
\pgfpathclose%
\pgfusepath{stroke,fill}%
}%
\begin{pgfscope}%
\pgfsys@transformshift{1.063889in}{3.430516in}%
\pgfsys@useobject{currentmarker}{}%
\end{pgfscope}%
\end{pgfscope}%
\begin{pgfscope}%
\definecolor{textcolor}{rgb}{0.000000,0.000000,0.000000}%
\pgfsetstrokecolor{textcolor}%
\pgfsetfillcolor{textcolor}%
\pgftext[x=1.313889in,y=3.381905in,left,base]{\color{textcolor}\sffamily\fontsize{10.000000}{12.000000}\selectfont \(\displaystyle \beta = 0.7\)}%
\end{pgfscope}%
\end{pgfpicture}%
\makeatother%
\endgroup%

    \caption{GMRES iteration counts for $\AmatoI\Amatt$ given by \cref{eq:noweak,eq:ntweak}, where $\alpha = 0.2/k^\beta,$ for $\beta = 0.4,0.5,0.6,0.7.$}\label{fig:l1med}
\end{figure}
    
    \begin{figure}
    %% Creator: Matplotlib, PGF backend
%%
%% To include the figure in your LaTeX document, write
%%   \input{<filename>.pgf}
%%
%% Make sure the required packages are loaded in your preamble
%%   \usepackage{pgf}
%%
%% Figures using additional raster images can only be included by \input if
%% they are in the same directory as the main LaTeX file. For loading figures
%% from other directories you can use the `import` package
%%   \usepackage{import}
%% and then include the figures with
%%   \import{<path to file>}{<filename>.pgf}
%%
%% Matplotlib used the following preamble
%%   \usepackage{fontspec}
%%   \setmainfont{DejaVuSerif.ttf}[Path=/home/owen/progs/firedrake-complex/firedrake/lib/python3.5/site-packages/matplotlib/mpl-data/fonts/ttf/]
%%   \setsansfont{DejaVuSans.ttf}[Path=/home/owen/progs/firedrake-complex/firedrake/lib/python3.5/site-packages/matplotlib/mpl-data/fonts/ttf/]
%%   \setmonofont{DejaVuSansMono.ttf}[Path=/home/owen/progs/firedrake-complex/firedrake/lib/python3.5/site-packages/matplotlib/mpl-data/fonts/ttf/]
%%
\begingroup%
\makeatletter%
\begin{pgfpicture}%
\pgfpathrectangle{\pgfpointorigin}{\pgfqpoint{6.400000in}{4.800000in}}%
\pgfusepath{use as bounding box, clip}%
\begin{pgfscope}%
\pgfsetbuttcap%
\pgfsetmiterjoin%
\definecolor{currentfill}{rgb}{1.000000,1.000000,1.000000}%
\pgfsetfillcolor{currentfill}%
\pgfsetlinewidth{0.000000pt}%
\definecolor{currentstroke}{rgb}{1.000000,1.000000,1.000000}%
\pgfsetstrokecolor{currentstroke}%
\pgfsetdash{}{0pt}%
\pgfpathmoveto{\pgfqpoint{0.000000in}{0.000000in}}%
\pgfpathlineto{\pgfqpoint{6.400000in}{0.000000in}}%
\pgfpathlineto{\pgfqpoint{6.400000in}{4.800000in}}%
\pgfpathlineto{\pgfqpoint{0.000000in}{4.800000in}}%
\pgfpathclose%
\pgfusepath{fill}%
\end{pgfscope}%
\begin{pgfscope}%
\pgfsetbuttcap%
\pgfsetmiterjoin%
\definecolor{currentfill}{rgb}{1.000000,1.000000,1.000000}%
\pgfsetfillcolor{currentfill}%
\pgfsetlinewidth{0.000000pt}%
\definecolor{currentstroke}{rgb}{0.000000,0.000000,0.000000}%
\pgfsetstrokecolor{currentstroke}%
\pgfsetstrokeopacity{0.000000}%
\pgfsetdash{}{0pt}%
\pgfpathmoveto{\pgfqpoint{0.800000in}{0.528000in}}%
\pgfpathlineto{\pgfqpoint{5.760000in}{0.528000in}}%
\pgfpathlineto{\pgfqpoint{5.760000in}{4.224000in}}%
\pgfpathlineto{\pgfqpoint{0.800000in}{4.224000in}}%
\pgfpathclose%
\pgfusepath{fill}%
\end{pgfscope}%
\begin{pgfscope}%
\pgfsetbuttcap%
\pgfsetroundjoin%
\definecolor{currentfill}{rgb}{0.000000,0.000000,0.000000}%
\pgfsetfillcolor{currentfill}%
\pgfsetlinewidth{0.803000pt}%
\definecolor{currentstroke}{rgb}{0.000000,0.000000,0.000000}%
\pgfsetstrokecolor{currentstroke}%
\pgfsetdash{}{0pt}%
\pgfsys@defobject{currentmarker}{\pgfqpoint{0.000000in}{-0.048611in}}{\pgfqpoint{0.000000in}{0.000000in}}{%
\pgfpathmoveto{\pgfqpoint{0.000000in}{0.000000in}}%
\pgfpathlineto{\pgfqpoint{0.000000in}{-0.048611in}}%
\pgfusepath{stroke,fill}%
}%
\begin{pgfscope}%
\pgfsys@transformshift{1.250909in}{0.528000in}%
\pgfsys@useobject{currentmarker}{}%
\end{pgfscope}%
\end{pgfscope}%
\begin{pgfscope}%
\definecolor{textcolor}{rgb}{0.000000,0.000000,0.000000}%
\pgfsetstrokecolor{textcolor}%
\pgfsetfillcolor{textcolor}%
\pgftext[x=1.250909in,y=0.430778in,,top]{\color{textcolor}\sffamily\fontsize{10.000000}{12.000000}\selectfont 10}%
\end{pgfscope}%
\begin{pgfscope}%
\pgfsetbuttcap%
\pgfsetroundjoin%
\definecolor{currentfill}{rgb}{0.000000,0.000000,0.000000}%
\pgfsetfillcolor{currentfill}%
\pgfsetlinewidth{0.803000pt}%
\definecolor{currentstroke}{rgb}{0.000000,0.000000,0.000000}%
\pgfsetstrokecolor{currentstroke}%
\pgfsetdash{}{0pt}%
\pgfsys@defobject{currentmarker}{\pgfqpoint{0.000000in}{-0.048611in}}{\pgfqpoint{0.000000in}{0.000000in}}{%
\pgfpathmoveto{\pgfqpoint{0.000000in}{0.000000in}}%
\pgfpathlineto{\pgfqpoint{0.000000in}{-0.048611in}}%
\pgfusepath{stroke,fill}%
}%
\begin{pgfscope}%
\pgfsys@transformshift{1.701818in}{0.528000in}%
\pgfsys@useobject{currentmarker}{}%
\end{pgfscope}%
\end{pgfscope}%
\begin{pgfscope}%
\definecolor{textcolor}{rgb}{0.000000,0.000000,0.000000}%
\pgfsetstrokecolor{textcolor}%
\pgfsetfillcolor{textcolor}%
\pgftext[x=1.701818in,y=0.430778in,,top]{\color{textcolor}\sffamily\fontsize{10.000000}{12.000000}\selectfont 20}%
\end{pgfscope}%
\begin{pgfscope}%
\pgfsetbuttcap%
\pgfsetroundjoin%
\definecolor{currentfill}{rgb}{0.000000,0.000000,0.000000}%
\pgfsetfillcolor{currentfill}%
\pgfsetlinewidth{0.803000pt}%
\definecolor{currentstroke}{rgb}{0.000000,0.000000,0.000000}%
\pgfsetstrokecolor{currentstroke}%
\pgfsetdash{}{0pt}%
\pgfsys@defobject{currentmarker}{\pgfqpoint{0.000000in}{-0.048611in}}{\pgfqpoint{0.000000in}{0.000000in}}{%
\pgfpathmoveto{\pgfqpoint{0.000000in}{0.000000in}}%
\pgfpathlineto{\pgfqpoint{0.000000in}{-0.048611in}}%
\pgfusepath{stroke,fill}%
}%
\begin{pgfscope}%
\pgfsys@transformshift{2.152727in}{0.528000in}%
\pgfsys@useobject{currentmarker}{}%
\end{pgfscope}%
\end{pgfscope}%
\begin{pgfscope}%
\definecolor{textcolor}{rgb}{0.000000,0.000000,0.000000}%
\pgfsetstrokecolor{textcolor}%
\pgfsetfillcolor{textcolor}%
\pgftext[x=2.152727in,y=0.430778in,,top]{\color{textcolor}\sffamily\fontsize{10.000000}{12.000000}\selectfont 30}%
\end{pgfscope}%
\begin{pgfscope}%
\pgfsetbuttcap%
\pgfsetroundjoin%
\definecolor{currentfill}{rgb}{0.000000,0.000000,0.000000}%
\pgfsetfillcolor{currentfill}%
\pgfsetlinewidth{0.803000pt}%
\definecolor{currentstroke}{rgb}{0.000000,0.000000,0.000000}%
\pgfsetstrokecolor{currentstroke}%
\pgfsetdash{}{0pt}%
\pgfsys@defobject{currentmarker}{\pgfqpoint{0.000000in}{-0.048611in}}{\pgfqpoint{0.000000in}{0.000000in}}{%
\pgfpathmoveto{\pgfqpoint{0.000000in}{0.000000in}}%
\pgfpathlineto{\pgfqpoint{0.000000in}{-0.048611in}}%
\pgfusepath{stroke,fill}%
}%
\begin{pgfscope}%
\pgfsys@transformshift{2.603636in}{0.528000in}%
\pgfsys@useobject{currentmarker}{}%
\end{pgfscope}%
\end{pgfscope}%
\begin{pgfscope}%
\definecolor{textcolor}{rgb}{0.000000,0.000000,0.000000}%
\pgfsetstrokecolor{textcolor}%
\pgfsetfillcolor{textcolor}%
\pgftext[x=2.603636in,y=0.430778in,,top]{\color{textcolor}\sffamily\fontsize{10.000000}{12.000000}\selectfont 40}%
\end{pgfscope}%
\begin{pgfscope}%
\pgfsetbuttcap%
\pgfsetroundjoin%
\definecolor{currentfill}{rgb}{0.000000,0.000000,0.000000}%
\pgfsetfillcolor{currentfill}%
\pgfsetlinewidth{0.803000pt}%
\definecolor{currentstroke}{rgb}{0.000000,0.000000,0.000000}%
\pgfsetstrokecolor{currentstroke}%
\pgfsetdash{}{0pt}%
\pgfsys@defobject{currentmarker}{\pgfqpoint{0.000000in}{-0.048611in}}{\pgfqpoint{0.000000in}{0.000000in}}{%
\pgfpathmoveto{\pgfqpoint{0.000000in}{0.000000in}}%
\pgfpathlineto{\pgfqpoint{0.000000in}{-0.048611in}}%
\pgfusepath{stroke,fill}%
}%
\begin{pgfscope}%
\pgfsys@transformshift{3.054545in}{0.528000in}%
\pgfsys@useobject{currentmarker}{}%
\end{pgfscope}%
\end{pgfscope}%
\begin{pgfscope}%
\definecolor{textcolor}{rgb}{0.000000,0.000000,0.000000}%
\pgfsetstrokecolor{textcolor}%
\pgfsetfillcolor{textcolor}%
\pgftext[x=3.054545in,y=0.430778in,,top]{\color{textcolor}\sffamily\fontsize{10.000000}{12.000000}\selectfont 50}%
\end{pgfscope}%
\begin{pgfscope}%
\pgfsetbuttcap%
\pgfsetroundjoin%
\definecolor{currentfill}{rgb}{0.000000,0.000000,0.000000}%
\pgfsetfillcolor{currentfill}%
\pgfsetlinewidth{0.803000pt}%
\definecolor{currentstroke}{rgb}{0.000000,0.000000,0.000000}%
\pgfsetstrokecolor{currentstroke}%
\pgfsetdash{}{0pt}%
\pgfsys@defobject{currentmarker}{\pgfqpoint{0.000000in}{-0.048611in}}{\pgfqpoint{0.000000in}{0.000000in}}{%
\pgfpathmoveto{\pgfqpoint{0.000000in}{0.000000in}}%
\pgfpathlineto{\pgfqpoint{0.000000in}{-0.048611in}}%
\pgfusepath{stroke,fill}%
}%
\begin{pgfscope}%
\pgfsys@transformshift{3.505455in}{0.528000in}%
\pgfsys@useobject{currentmarker}{}%
\end{pgfscope}%
\end{pgfscope}%
\begin{pgfscope}%
\definecolor{textcolor}{rgb}{0.000000,0.000000,0.000000}%
\pgfsetstrokecolor{textcolor}%
\pgfsetfillcolor{textcolor}%
\pgftext[x=3.505455in,y=0.430778in,,top]{\color{textcolor}\sffamily\fontsize{10.000000}{12.000000}\selectfont 60}%
\end{pgfscope}%
\begin{pgfscope}%
\pgfsetbuttcap%
\pgfsetroundjoin%
\definecolor{currentfill}{rgb}{0.000000,0.000000,0.000000}%
\pgfsetfillcolor{currentfill}%
\pgfsetlinewidth{0.803000pt}%
\definecolor{currentstroke}{rgb}{0.000000,0.000000,0.000000}%
\pgfsetstrokecolor{currentstroke}%
\pgfsetdash{}{0pt}%
\pgfsys@defobject{currentmarker}{\pgfqpoint{0.000000in}{-0.048611in}}{\pgfqpoint{0.000000in}{0.000000in}}{%
\pgfpathmoveto{\pgfqpoint{0.000000in}{0.000000in}}%
\pgfpathlineto{\pgfqpoint{0.000000in}{-0.048611in}}%
\pgfusepath{stroke,fill}%
}%
\begin{pgfscope}%
\pgfsys@transformshift{3.956364in}{0.528000in}%
\pgfsys@useobject{currentmarker}{}%
\end{pgfscope}%
\end{pgfscope}%
\begin{pgfscope}%
\definecolor{textcolor}{rgb}{0.000000,0.000000,0.000000}%
\pgfsetstrokecolor{textcolor}%
\pgfsetfillcolor{textcolor}%
\pgftext[x=3.956364in,y=0.430778in,,top]{\color{textcolor}\sffamily\fontsize{10.000000}{12.000000}\selectfont 70}%
\end{pgfscope}%
\begin{pgfscope}%
\pgfsetbuttcap%
\pgfsetroundjoin%
\definecolor{currentfill}{rgb}{0.000000,0.000000,0.000000}%
\pgfsetfillcolor{currentfill}%
\pgfsetlinewidth{0.803000pt}%
\definecolor{currentstroke}{rgb}{0.000000,0.000000,0.000000}%
\pgfsetstrokecolor{currentstroke}%
\pgfsetdash{}{0pt}%
\pgfsys@defobject{currentmarker}{\pgfqpoint{0.000000in}{-0.048611in}}{\pgfqpoint{0.000000in}{0.000000in}}{%
\pgfpathmoveto{\pgfqpoint{0.000000in}{0.000000in}}%
\pgfpathlineto{\pgfqpoint{0.000000in}{-0.048611in}}%
\pgfusepath{stroke,fill}%
}%
\begin{pgfscope}%
\pgfsys@transformshift{4.407273in}{0.528000in}%
\pgfsys@useobject{currentmarker}{}%
\end{pgfscope}%
\end{pgfscope}%
\begin{pgfscope}%
\definecolor{textcolor}{rgb}{0.000000,0.000000,0.000000}%
\pgfsetstrokecolor{textcolor}%
\pgfsetfillcolor{textcolor}%
\pgftext[x=4.407273in,y=0.430778in,,top]{\color{textcolor}\sffamily\fontsize{10.000000}{12.000000}\selectfont 80}%
\end{pgfscope}%
\begin{pgfscope}%
\pgfsetbuttcap%
\pgfsetroundjoin%
\definecolor{currentfill}{rgb}{0.000000,0.000000,0.000000}%
\pgfsetfillcolor{currentfill}%
\pgfsetlinewidth{0.803000pt}%
\definecolor{currentstroke}{rgb}{0.000000,0.000000,0.000000}%
\pgfsetstrokecolor{currentstroke}%
\pgfsetdash{}{0pt}%
\pgfsys@defobject{currentmarker}{\pgfqpoint{0.000000in}{-0.048611in}}{\pgfqpoint{0.000000in}{0.000000in}}{%
\pgfpathmoveto{\pgfqpoint{0.000000in}{0.000000in}}%
\pgfpathlineto{\pgfqpoint{0.000000in}{-0.048611in}}%
\pgfusepath{stroke,fill}%
}%
\begin{pgfscope}%
\pgfsys@transformshift{4.858182in}{0.528000in}%
\pgfsys@useobject{currentmarker}{}%
\end{pgfscope}%
\end{pgfscope}%
\begin{pgfscope}%
\definecolor{textcolor}{rgb}{0.000000,0.000000,0.000000}%
\pgfsetstrokecolor{textcolor}%
\pgfsetfillcolor{textcolor}%
\pgftext[x=4.858182in,y=0.430778in,,top]{\color{textcolor}\sffamily\fontsize{10.000000}{12.000000}\selectfont 90}%
\end{pgfscope}%
\begin{pgfscope}%
\pgfsetbuttcap%
\pgfsetroundjoin%
\definecolor{currentfill}{rgb}{0.000000,0.000000,0.000000}%
\pgfsetfillcolor{currentfill}%
\pgfsetlinewidth{0.803000pt}%
\definecolor{currentstroke}{rgb}{0.000000,0.000000,0.000000}%
\pgfsetstrokecolor{currentstroke}%
\pgfsetdash{}{0pt}%
\pgfsys@defobject{currentmarker}{\pgfqpoint{0.000000in}{-0.048611in}}{\pgfqpoint{0.000000in}{0.000000in}}{%
\pgfpathmoveto{\pgfqpoint{0.000000in}{0.000000in}}%
\pgfpathlineto{\pgfqpoint{0.000000in}{-0.048611in}}%
\pgfusepath{stroke,fill}%
}%
\begin{pgfscope}%
\pgfsys@transformshift{5.309091in}{0.528000in}%
\pgfsys@useobject{currentmarker}{}%
\end{pgfscope}%
\end{pgfscope}%
\begin{pgfscope}%
\definecolor{textcolor}{rgb}{0.000000,0.000000,0.000000}%
\pgfsetstrokecolor{textcolor}%
\pgfsetfillcolor{textcolor}%
\pgftext[x=5.309091in,y=0.430778in,,top]{\color{textcolor}\sffamily\fontsize{10.000000}{12.000000}\selectfont 100}%
\end{pgfscope}%
\begin{pgfscope}%
\definecolor{textcolor}{rgb}{0.000000,0.000000,0.000000}%
\pgfsetstrokecolor{textcolor}%
\pgfsetfillcolor{textcolor}%
\pgftext[x=3.280000in,y=0.240809in,,top]{\color{textcolor}\sffamily\fontsize{10.000000}{12.000000}\selectfont \(\displaystyle k\)}%
\end{pgfscope}%
\begin{pgfscope}%
\pgfsetbuttcap%
\pgfsetroundjoin%
\definecolor{currentfill}{rgb}{0.000000,0.000000,0.000000}%
\pgfsetfillcolor{currentfill}%
\pgfsetlinewidth{0.803000pt}%
\definecolor{currentstroke}{rgb}{0.000000,0.000000,0.000000}%
\pgfsetstrokecolor{currentstroke}%
\pgfsetdash{}{0pt}%
\pgfsys@defobject{currentmarker}{\pgfqpoint{-0.048611in}{0.000000in}}{\pgfqpoint{0.000000in}{0.000000in}}{%
\pgfpathmoveto{\pgfqpoint{0.000000in}{0.000000in}}%
\pgfpathlineto{\pgfqpoint{-0.048611in}{0.000000in}}%
\pgfusepath{stroke,fill}%
}%
\begin{pgfscope}%
\pgfsys@transformshift{0.800000in}{0.696000in}%
\pgfsys@useobject{currentmarker}{}%
\end{pgfscope}%
\end{pgfscope}%
\begin{pgfscope}%
\definecolor{textcolor}{rgb}{0.000000,0.000000,0.000000}%
\pgfsetstrokecolor{textcolor}%
\pgfsetfillcolor{textcolor}%
\pgftext[x=0.614413in,y=0.643238in,left,base]{\color{textcolor}\sffamily\fontsize{10.000000}{12.000000}\selectfont 6}%
\end{pgfscope}%
\begin{pgfscope}%
\pgfsetbuttcap%
\pgfsetroundjoin%
\definecolor{currentfill}{rgb}{0.000000,0.000000,0.000000}%
\pgfsetfillcolor{currentfill}%
\pgfsetlinewidth{0.803000pt}%
\definecolor{currentstroke}{rgb}{0.000000,0.000000,0.000000}%
\pgfsetstrokecolor{currentstroke}%
\pgfsetdash{}{0pt}%
\pgfsys@defobject{currentmarker}{\pgfqpoint{-0.048611in}{0.000000in}}{\pgfqpoint{0.000000in}{0.000000in}}{%
\pgfpathmoveto{\pgfqpoint{0.000000in}{0.000000in}}%
\pgfpathlineto{\pgfqpoint{-0.048611in}{0.000000in}}%
\pgfusepath{stroke,fill}%
}%
\begin{pgfscope}%
\pgfsys@transformshift{0.800000in}{1.536000in}%
\pgfsys@useobject{currentmarker}{}%
\end{pgfscope}%
\end{pgfscope}%
\begin{pgfscope}%
\definecolor{textcolor}{rgb}{0.000000,0.000000,0.000000}%
\pgfsetstrokecolor{textcolor}%
\pgfsetfillcolor{textcolor}%
\pgftext[x=0.614413in,y=1.483238in,left,base]{\color{textcolor}\sffamily\fontsize{10.000000}{12.000000}\selectfont 7}%
\end{pgfscope}%
\begin{pgfscope}%
\pgfsetbuttcap%
\pgfsetroundjoin%
\definecolor{currentfill}{rgb}{0.000000,0.000000,0.000000}%
\pgfsetfillcolor{currentfill}%
\pgfsetlinewidth{0.803000pt}%
\definecolor{currentstroke}{rgb}{0.000000,0.000000,0.000000}%
\pgfsetstrokecolor{currentstroke}%
\pgfsetdash{}{0pt}%
\pgfsys@defobject{currentmarker}{\pgfqpoint{-0.048611in}{0.000000in}}{\pgfqpoint{0.000000in}{0.000000in}}{%
\pgfpathmoveto{\pgfqpoint{0.000000in}{0.000000in}}%
\pgfpathlineto{\pgfqpoint{-0.048611in}{0.000000in}}%
\pgfusepath{stroke,fill}%
}%
\begin{pgfscope}%
\pgfsys@transformshift{0.800000in}{2.376000in}%
\pgfsys@useobject{currentmarker}{}%
\end{pgfscope}%
\end{pgfscope}%
\begin{pgfscope}%
\definecolor{textcolor}{rgb}{0.000000,0.000000,0.000000}%
\pgfsetstrokecolor{textcolor}%
\pgfsetfillcolor{textcolor}%
\pgftext[x=0.614413in,y=2.323238in,left,base]{\color{textcolor}\sffamily\fontsize{10.000000}{12.000000}\selectfont 8}%
\end{pgfscope}%
\begin{pgfscope}%
\pgfsetbuttcap%
\pgfsetroundjoin%
\definecolor{currentfill}{rgb}{0.000000,0.000000,0.000000}%
\pgfsetfillcolor{currentfill}%
\pgfsetlinewidth{0.803000pt}%
\definecolor{currentstroke}{rgb}{0.000000,0.000000,0.000000}%
\pgfsetstrokecolor{currentstroke}%
\pgfsetdash{}{0pt}%
\pgfsys@defobject{currentmarker}{\pgfqpoint{-0.048611in}{0.000000in}}{\pgfqpoint{0.000000in}{0.000000in}}{%
\pgfpathmoveto{\pgfqpoint{0.000000in}{0.000000in}}%
\pgfpathlineto{\pgfqpoint{-0.048611in}{0.000000in}}%
\pgfusepath{stroke,fill}%
}%
\begin{pgfscope}%
\pgfsys@transformshift{0.800000in}{3.216000in}%
\pgfsys@useobject{currentmarker}{}%
\end{pgfscope}%
\end{pgfscope}%
\begin{pgfscope}%
\definecolor{textcolor}{rgb}{0.000000,0.000000,0.000000}%
\pgfsetstrokecolor{textcolor}%
\pgfsetfillcolor{textcolor}%
\pgftext[x=0.614413in,y=3.163238in,left,base]{\color{textcolor}\sffamily\fontsize{10.000000}{12.000000}\selectfont 9}%
\end{pgfscope}%
\begin{pgfscope}%
\pgfsetbuttcap%
\pgfsetroundjoin%
\definecolor{currentfill}{rgb}{0.000000,0.000000,0.000000}%
\pgfsetfillcolor{currentfill}%
\pgfsetlinewidth{0.803000pt}%
\definecolor{currentstroke}{rgb}{0.000000,0.000000,0.000000}%
\pgfsetstrokecolor{currentstroke}%
\pgfsetdash{}{0pt}%
\pgfsys@defobject{currentmarker}{\pgfqpoint{-0.048611in}{0.000000in}}{\pgfqpoint{0.000000in}{0.000000in}}{%
\pgfpathmoveto{\pgfqpoint{0.000000in}{0.000000in}}%
\pgfpathlineto{\pgfqpoint{-0.048611in}{0.000000in}}%
\pgfusepath{stroke,fill}%
}%
\begin{pgfscope}%
\pgfsys@transformshift{0.800000in}{4.056000in}%
\pgfsys@useobject{currentmarker}{}%
\end{pgfscope}%
\end{pgfscope}%
\begin{pgfscope}%
\definecolor{textcolor}{rgb}{0.000000,0.000000,0.000000}%
\pgfsetstrokecolor{textcolor}%
\pgfsetfillcolor{textcolor}%
\pgftext[x=0.526047in,y=4.003238in,left,base]{\color{textcolor}\sffamily\fontsize{10.000000}{12.000000}\selectfont 10}%
\end{pgfscope}%
\begin{pgfscope}%
\definecolor{textcolor}{rgb}{0.000000,0.000000,0.000000}%
\pgfsetstrokecolor{textcolor}%
\pgfsetfillcolor{textcolor}%
\pgftext[x=0.470492in,y=2.376000in,,bottom,rotate=90.000000]{\color{textcolor}\sffamily\fontsize{10.000000}{12.000000}\selectfont Number of GMRES iterations}%
\end{pgfscope}%
\begin{pgfscope}%
\pgfpathrectangle{\pgfqpoint{0.800000in}{0.528000in}}{\pgfqpoint{4.960000in}{3.696000in}}%
\pgfusepath{clip}%
\pgfsetbuttcap%
\pgfsetroundjoin%
\pgfsetlinewidth{1.505625pt}%
\definecolor{currentstroke}{rgb}{0.843137,0.000000,0.000000}%
\pgfsetstrokecolor{currentstroke}%
\pgfsetdash{{5.550000pt}{2.400000pt}}{0.000000pt}%
\pgfpathmoveto{\pgfqpoint{1.250909in}{2.376000in}}%
\pgfpathlineto{\pgfqpoint{1.701818in}{2.376000in}}%
\pgfpathlineto{\pgfqpoint{2.152727in}{3.216000in}}%
\pgfpathlineto{\pgfqpoint{2.603636in}{3.216000in}}%
\pgfpathlineto{\pgfqpoint{3.054545in}{4.056000in}}%
\pgfpathlineto{\pgfqpoint{3.505455in}{4.056000in}}%
\pgfpathlineto{\pgfqpoint{3.956364in}{4.056000in}}%
\pgfpathlineto{\pgfqpoint{4.407273in}{4.056000in}}%
\pgfpathlineto{\pgfqpoint{4.858182in}{4.056000in}}%
\pgfpathlineto{\pgfqpoint{5.309091in}{4.056000in}}%
\pgfusepath{stroke}%
\end{pgfscope}%
\begin{pgfscope}%
\pgfpathrectangle{\pgfqpoint{0.800000in}{0.528000in}}{\pgfqpoint{4.960000in}{3.696000in}}%
\pgfusepath{clip}%
\pgfsetbuttcap%
\pgfsetroundjoin%
\definecolor{currentfill}{rgb}{0.843137,0.000000,0.000000}%
\pgfsetfillcolor{currentfill}%
\pgfsetlinewidth{1.003750pt}%
\definecolor{currentstroke}{rgb}{0.843137,0.000000,0.000000}%
\pgfsetstrokecolor{currentstroke}%
\pgfsetdash{}{0pt}%
\pgfsys@defobject{currentmarker}{\pgfqpoint{-0.041667in}{-0.041667in}}{\pgfqpoint{0.041667in}{0.041667in}}{%
\pgfpathmoveto{\pgfqpoint{0.000000in}{-0.041667in}}%
\pgfpathcurveto{\pgfqpoint{0.011050in}{-0.041667in}}{\pgfqpoint{0.021649in}{-0.037276in}}{\pgfqpoint{0.029463in}{-0.029463in}}%
\pgfpathcurveto{\pgfqpoint{0.037276in}{-0.021649in}}{\pgfqpoint{0.041667in}{-0.011050in}}{\pgfqpoint{0.041667in}{0.000000in}}%
\pgfpathcurveto{\pgfqpoint{0.041667in}{0.011050in}}{\pgfqpoint{0.037276in}{0.021649in}}{\pgfqpoint{0.029463in}{0.029463in}}%
\pgfpathcurveto{\pgfqpoint{0.021649in}{0.037276in}}{\pgfqpoint{0.011050in}{0.041667in}}{\pgfqpoint{0.000000in}{0.041667in}}%
\pgfpathcurveto{\pgfqpoint{-0.011050in}{0.041667in}}{\pgfqpoint{-0.021649in}{0.037276in}}{\pgfqpoint{-0.029463in}{0.029463in}}%
\pgfpathcurveto{\pgfqpoint{-0.037276in}{0.021649in}}{\pgfqpoint{-0.041667in}{0.011050in}}{\pgfqpoint{-0.041667in}{0.000000in}}%
\pgfpathcurveto{\pgfqpoint{-0.041667in}{-0.011050in}}{\pgfqpoint{-0.037276in}{-0.021649in}}{\pgfqpoint{-0.029463in}{-0.029463in}}%
\pgfpathcurveto{\pgfqpoint{-0.021649in}{-0.037276in}}{\pgfqpoint{-0.011050in}{-0.041667in}}{\pgfqpoint{0.000000in}{-0.041667in}}%
\pgfpathclose%
\pgfusepath{stroke,fill}%
}%
\begin{pgfscope}%
\pgfsys@transformshift{1.250909in}{2.376000in}%
\pgfsys@useobject{currentmarker}{}%
\end{pgfscope}%
\begin{pgfscope}%
\pgfsys@transformshift{1.701818in}{2.376000in}%
\pgfsys@useobject{currentmarker}{}%
\end{pgfscope}%
\begin{pgfscope}%
\pgfsys@transformshift{2.152727in}{3.216000in}%
\pgfsys@useobject{currentmarker}{}%
\end{pgfscope}%
\begin{pgfscope}%
\pgfsys@transformshift{2.603636in}{3.216000in}%
\pgfsys@useobject{currentmarker}{}%
\end{pgfscope}%
\begin{pgfscope}%
\pgfsys@transformshift{3.054545in}{4.056000in}%
\pgfsys@useobject{currentmarker}{}%
\end{pgfscope}%
\begin{pgfscope}%
\pgfsys@transformshift{3.505455in}{4.056000in}%
\pgfsys@useobject{currentmarker}{}%
\end{pgfscope}%
\begin{pgfscope}%
\pgfsys@transformshift{3.956364in}{4.056000in}%
\pgfsys@useobject{currentmarker}{}%
\end{pgfscope}%
\begin{pgfscope}%
\pgfsys@transformshift{4.407273in}{4.056000in}%
\pgfsys@useobject{currentmarker}{}%
\end{pgfscope}%
\begin{pgfscope}%
\pgfsys@transformshift{4.858182in}{4.056000in}%
\pgfsys@useobject{currentmarker}{}%
\end{pgfscope}%
\begin{pgfscope}%
\pgfsys@transformshift{5.309091in}{4.056000in}%
\pgfsys@useobject{currentmarker}{}%
\end{pgfscope}%
\end{pgfscope}%
\begin{pgfscope}%
\pgfpathrectangle{\pgfqpoint{0.800000in}{0.528000in}}{\pgfqpoint{4.960000in}{3.696000in}}%
\pgfusepath{clip}%
\pgfsetbuttcap%
\pgfsetroundjoin%
\pgfsetlinewidth{1.505625pt}%
\definecolor{currentstroke}{rgb}{0.549020,0.235294,1.000000}%
\pgfsetstrokecolor{currentstroke}%
\pgfsetdash{{5.550000pt}{2.400000pt}}{0.000000pt}%
\pgfpathmoveto{\pgfqpoint{1.250909in}{1.536000in}}%
\pgfpathlineto{\pgfqpoint{1.701818in}{1.536000in}}%
\pgfpathlineto{\pgfqpoint{2.152727in}{2.376000in}}%
\pgfpathlineto{\pgfqpoint{2.603636in}{2.376000in}}%
\pgfpathlineto{\pgfqpoint{3.054545in}{2.376000in}}%
\pgfpathlineto{\pgfqpoint{3.505455in}{2.376000in}}%
\pgfpathlineto{\pgfqpoint{3.956364in}{2.376000in}}%
\pgfpathlineto{\pgfqpoint{4.407273in}{2.376000in}}%
\pgfpathlineto{\pgfqpoint{4.858182in}{2.376000in}}%
\pgfpathlineto{\pgfqpoint{5.309091in}{2.376000in}}%
\pgfusepath{stroke}%
\end{pgfscope}%
\begin{pgfscope}%
\pgfpathrectangle{\pgfqpoint{0.800000in}{0.528000in}}{\pgfqpoint{4.960000in}{3.696000in}}%
\pgfusepath{clip}%
\pgfsetbuttcap%
\pgfsetmiterjoin%
\definecolor{currentfill}{rgb}{0.549020,0.235294,1.000000}%
\pgfsetfillcolor{currentfill}%
\pgfsetlinewidth{1.003750pt}%
\definecolor{currentstroke}{rgb}{0.549020,0.235294,1.000000}%
\pgfsetstrokecolor{currentstroke}%
\pgfsetdash{}{0pt}%
\pgfsys@defobject{currentmarker}{\pgfqpoint{-0.041667in}{-0.041667in}}{\pgfqpoint{0.041667in}{0.041667in}}{%
\pgfpathmoveto{\pgfqpoint{0.000000in}{0.041667in}}%
\pgfpathlineto{\pgfqpoint{-0.041667in}{-0.041667in}}%
\pgfpathlineto{\pgfqpoint{0.041667in}{-0.041667in}}%
\pgfpathclose%
\pgfusepath{stroke,fill}%
}%
\begin{pgfscope}%
\pgfsys@transformshift{1.250909in}{1.536000in}%
\pgfsys@useobject{currentmarker}{}%
\end{pgfscope}%
\begin{pgfscope}%
\pgfsys@transformshift{1.701818in}{1.536000in}%
\pgfsys@useobject{currentmarker}{}%
\end{pgfscope}%
\begin{pgfscope}%
\pgfsys@transformshift{2.152727in}{2.376000in}%
\pgfsys@useobject{currentmarker}{}%
\end{pgfscope}%
\begin{pgfscope}%
\pgfsys@transformshift{2.603636in}{2.376000in}%
\pgfsys@useobject{currentmarker}{}%
\end{pgfscope}%
\begin{pgfscope}%
\pgfsys@transformshift{3.054545in}{2.376000in}%
\pgfsys@useobject{currentmarker}{}%
\end{pgfscope}%
\begin{pgfscope}%
\pgfsys@transformshift{3.505455in}{2.376000in}%
\pgfsys@useobject{currentmarker}{}%
\end{pgfscope}%
\begin{pgfscope}%
\pgfsys@transformshift{3.956364in}{2.376000in}%
\pgfsys@useobject{currentmarker}{}%
\end{pgfscope}%
\begin{pgfscope}%
\pgfsys@transformshift{4.407273in}{2.376000in}%
\pgfsys@useobject{currentmarker}{}%
\end{pgfscope}%
\begin{pgfscope}%
\pgfsys@transformshift{4.858182in}{2.376000in}%
\pgfsys@useobject{currentmarker}{}%
\end{pgfscope}%
\begin{pgfscope}%
\pgfsys@transformshift{5.309091in}{2.376000in}%
\pgfsys@useobject{currentmarker}{}%
\end{pgfscope}%
\end{pgfscope}%
\begin{pgfscope}%
\pgfpathrectangle{\pgfqpoint{0.800000in}{0.528000in}}{\pgfqpoint{4.960000in}{3.696000in}}%
\pgfusepath{clip}%
\pgfsetbuttcap%
\pgfsetroundjoin%
\pgfsetlinewidth{1.505625pt}%
\definecolor{currentstroke}{rgb}{0.007843,0.533333,0.000000}%
\pgfsetstrokecolor{currentstroke}%
\pgfsetdash{{5.550000pt}{2.400000pt}}{0.000000pt}%
\pgfpathmoveto{\pgfqpoint{1.250909in}{1.536000in}}%
\pgfpathlineto{\pgfqpoint{1.701818in}{0.696000in}}%
\pgfpathlineto{\pgfqpoint{2.152727in}{1.536000in}}%
\pgfpathlineto{\pgfqpoint{2.603636in}{1.536000in}}%
\pgfpathlineto{\pgfqpoint{3.054545in}{1.536000in}}%
\pgfpathlineto{\pgfqpoint{3.505455in}{1.536000in}}%
\pgfpathlineto{\pgfqpoint{3.956364in}{1.536000in}}%
\pgfpathlineto{\pgfqpoint{4.407273in}{1.536000in}}%
\pgfpathlineto{\pgfqpoint{4.858182in}{1.536000in}}%
\pgfpathlineto{\pgfqpoint{5.309091in}{1.536000in}}%
\pgfusepath{stroke}%
\end{pgfscope}%
\begin{pgfscope}%
\pgfpathrectangle{\pgfqpoint{0.800000in}{0.528000in}}{\pgfqpoint{4.960000in}{3.696000in}}%
\pgfusepath{clip}%
\pgfsetbuttcap%
\pgfsetmiterjoin%
\definecolor{currentfill}{rgb}{0.007843,0.533333,0.000000}%
\pgfsetfillcolor{currentfill}%
\pgfsetlinewidth{1.003750pt}%
\definecolor{currentstroke}{rgb}{0.007843,0.533333,0.000000}%
\pgfsetstrokecolor{currentstroke}%
\pgfsetdash{}{0pt}%
\pgfsys@defobject{currentmarker}{\pgfqpoint{-0.041667in}{-0.041667in}}{\pgfqpoint{0.041667in}{0.041667in}}{%
\pgfpathmoveto{\pgfqpoint{-0.000000in}{-0.041667in}}%
\pgfpathlineto{\pgfqpoint{0.041667in}{0.041667in}}%
\pgfpathlineto{\pgfqpoint{-0.041667in}{0.041667in}}%
\pgfpathclose%
\pgfusepath{stroke,fill}%
}%
\begin{pgfscope}%
\pgfsys@transformshift{1.250909in}{1.536000in}%
\pgfsys@useobject{currentmarker}{}%
\end{pgfscope}%
\begin{pgfscope}%
\pgfsys@transformshift{1.701818in}{0.696000in}%
\pgfsys@useobject{currentmarker}{}%
\end{pgfscope}%
\begin{pgfscope}%
\pgfsys@transformshift{2.152727in}{1.536000in}%
\pgfsys@useobject{currentmarker}{}%
\end{pgfscope}%
\begin{pgfscope}%
\pgfsys@transformshift{2.603636in}{1.536000in}%
\pgfsys@useobject{currentmarker}{}%
\end{pgfscope}%
\begin{pgfscope}%
\pgfsys@transformshift{3.054545in}{1.536000in}%
\pgfsys@useobject{currentmarker}{}%
\end{pgfscope}%
\begin{pgfscope}%
\pgfsys@transformshift{3.505455in}{1.536000in}%
\pgfsys@useobject{currentmarker}{}%
\end{pgfscope}%
\begin{pgfscope}%
\pgfsys@transformshift{3.956364in}{1.536000in}%
\pgfsys@useobject{currentmarker}{}%
\end{pgfscope}%
\begin{pgfscope}%
\pgfsys@transformshift{4.407273in}{1.536000in}%
\pgfsys@useobject{currentmarker}{}%
\end{pgfscope}%
\begin{pgfscope}%
\pgfsys@transformshift{4.858182in}{1.536000in}%
\pgfsys@useobject{currentmarker}{}%
\end{pgfscope}%
\begin{pgfscope}%
\pgfsys@transformshift{5.309091in}{1.536000in}%
\pgfsys@useobject{currentmarker}{}%
\end{pgfscope}%
\end{pgfscope}%
\begin{pgfscope}%
\pgfsetrectcap%
\pgfsetmiterjoin%
\pgfsetlinewidth{0.803000pt}%
\definecolor{currentstroke}{rgb}{0.000000,0.000000,0.000000}%
\pgfsetstrokecolor{currentstroke}%
\pgfsetdash{}{0pt}%
\pgfpathmoveto{\pgfqpoint{0.800000in}{0.528000in}}%
\pgfpathlineto{\pgfqpoint{0.800000in}{4.224000in}}%
\pgfusepath{stroke}%
\end{pgfscope}%
\begin{pgfscope}%
\pgfsetrectcap%
\pgfsetmiterjoin%
\pgfsetlinewidth{0.803000pt}%
\definecolor{currentstroke}{rgb}{0.000000,0.000000,0.000000}%
\pgfsetstrokecolor{currentstroke}%
\pgfsetdash{}{0pt}%
\pgfpathmoveto{\pgfqpoint{5.760000in}{0.528000in}}%
\pgfpathlineto{\pgfqpoint{5.760000in}{4.224000in}}%
\pgfusepath{stroke}%
\end{pgfscope}%
\begin{pgfscope}%
\pgfsetrectcap%
\pgfsetmiterjoin%
\pgfsetlinewidth{0.803000pt}%
\definecolor{currentstroke}{rgb}{0.000000,0.000000,0.000000}%
\pgfsetstrokecolor{currentstroke}%
\pgfsetdash{}{0pt}%
\pgfpathmoveto{\pgfqpoint{0.800000in}{0.528000in}}%
\pgfpathlineto{\pgfqpoint{5.760000in}{0.528000in}}%
\pgfusepath{stroke}%
\end{pgfscope}%
\begin{pgfscope}%
\pgfsetrectcap%
\pgfsetmiterjoin%
\pgfsetlinewidth{0.803000pt}%
\definecolor{currentstroke}{rgb}{0.000000,0.000000,0.000000}%
\pgfsetstrokecolor{currentstroke}%
\pgfsetdash{}{0pt}%
\pgfpathmoveto{\pgfqpoint{0.800000in}{4.224000in}}%
\pgfpathlineto{\pgfqpoint{5.760000in}{4.224000in}}%
\pgfusepath{stroke}%
\end{pgfscope}%
\begin{pgfscope}%
\pgfsetbuttcap%
\pgfsetmiterjoin%
\definecolor{currentfill}{rgb}{1.000000,1.000000,1.000000}%
\pgfsetfillcolor{currentfill}%
\pgfsetfillopacity{0.800000}%
\pgfsetlinewidth{1.003750pt}%
\definecolor{currentstroke}{rgb}{0.800000,0.800000,0.800000}%
\pgfsetstrokecolor{currentstroke}%
\pgfsetstrokeopacity{0.800000}%
\pgfsetdash{}{0pt}%
\pgfpathmoveto{\pgfqpoint{0.897222in}{3.501317in}}%
\pgfpathlineto{\pgfqpoint{1.790209in}{3.501317in}}%
\pgfpathquadraticcurveto{\pgfqpoint{1.817987in}{3.501317in}}{\pgfqpoint{1.817987in}{3.529095in}}%
\pgfpathlineto{\pgfqpoint{1.817987in}{4.126778in}}%
\pgfpathquadraticcurveto{\pgfqpoint{1.817987in}{4.154556in}}{\pgfqpoint{1.790209in}{4.154556in}}%
\pgfpathlineto{\pgfqpoint{0.897222in}{4.154556in}}%
\pgfpathquadraticcurveto{\pgfqpoint{0.869444in}{4.154556in}}{\pgfqpoint{0.869444in}{4.126778in}}%
\pgfpathlineto{\pgfqpoint{0.869444in}{3.529095in}}%
\pgfpathquadraticcurveto{\pgfqpoint{0.869444in}{3.501317in}}{\pgfqpoint{0.897222in}{3.501317in}}%
\pgfpathclose%
\pgfusepath{stroke,fill}%
\end{pgfscope}%
\begin{pgfscope}%
\pgfsetbuttcap%
\pgfsetroundjoin%
\pgfsetlinewidth{1.505625pt}%
\definecolor{currentstroke}{rgb}{0.843137,0.000000,0.000000}%
\pgfsetstrokecolor{currentstroke}%
\pgfsetdash{{5.550000pt}{2.400000pt}}{0.000000pt}%
\pgfpathmoveto{\pgfqpoint{0.925000in}{4.042088in}}%
\pgfpathlineto{\pgfqpoint{1.202778in}{4.042088in}}%
\pgfusepath{stroke}%
\end{pgfscope}%
\begin{pgfscope}%
\pgfsetbuttcap%
\pgfsetroundjoin%
\definecolor{currentfill}{rgb}{0.843137,0.000000,0.000000}%
\pgfsetfillcolor{currentfill}%
\pgfsetlinewidth{1.003750pt}%
\definecolor{currentstroke}{rgb}{0.843137,0.000000,0.000000}%
\pgfsetstrokecolor{currentstroke}%
\pgfsetdash{}{0pt}%
\pgfsys@defobject{currentmarker}{\pgfqpoint{-0.041667in}{-0.041667in}}{\pgfqpoint{0.041667in}{0.041667in}}{%
\pgfpathmoveto{\pgfqpoint{0.000000in}{-0.041667in}}%
\pgfpathcurveto{\pgfqpoint{0.011050in}{-0.041667in}}{\pgfqpoint{0.021649in}{-0.037276in}}{\pgfqpoint{0.029463in}{-0.029463in}}%
\pgfpathcurveto{\pgfqpoint{0.037276in}{-0.021649in}}{\pgfqpoint{0.041667in}{-0.011050in}}{\pgfqpoint{0.041667in}{0.000000in}}%
\pgfpathcurveto{\pgfqpoint{0.041667in}{0.011050in}}{\pgfqpoint{0.037276in}{0.021649in}}{\pgfqpoint{0.029463in}{0.029463in}}%
\pgfpathcurveto{\pgfqpoint{0.021649in}{0.037276in}}{\pgfqpoint{0.011050in}{0.041667in}}{\pgfqpoint{0.000000in}{0.041667in}}%
\pgfpathcurveto{\pgfqpoint{-0.011050in}{0.041667in}}{\pgfqpoint{-0.021649in}{0.037276in}}{\pgfqpoint{-0.029463in}{0.029463in}}%
\pgfpathcurveto{\pgfqpoint{-0.037276in}{0.021649in}}{\pgfqpoint{-0.041667in}{0.011050in}}{\pgfqpoint{-0.041667in}{0.000000in}}%
\pgfpathcurveto{\pgfqpoint{-0.041667in}{-0.011050in}}{\pgfqpoint{-0.037276in}{-0.021649in}}{\pgfqpoint{-0.029463in}{-0.029463in}}%
\pgfpathcurveto{\pgfqpoint{-0.021649in}{-0.037276in}}{\pgfqpoint{-0.011050in}{-0.041667in}}{\pgfqpoint{0.000000in}{-0.041667in}}%
\pgfpathclose%
\pgfusepath{stroke,fill}%
}%
\begin{pgfscope}%
\pgfsys@transformshift{1.063889in}{4.042088in}%
\pgfsys@useobject{currentmarker}{}%
\end{pgfscope}%
\end{pgfscope}%
\begin{pgfscope}%
\definecolor{textcolor}{rgb}{0.000000,0.000000,0.000000}%
\pgfsetstrokecolor{textcolor}%
\pgfsetfillcolor{textcolor}%
\pgftext[x=1.313889in,y=3.993477in,left,base]{\color{textcolor}\sffamily\fontsize{10.000000}{12.000000}\selectfont \(\displaystyle \beta = 0.8\)}%
\end{pgfscope}%
\begin{pgfscope}%
\pgfsetbuttcap%
\pgfsetroundjoin%
\pgfsetlinewidth{1.505625pt}%
\definecolor{currentstroke}{rgb}{0.549020,0.235294,1.000000}%
\pgfsetstrokecolor{currentstroke}%
\pgfsetdash{{5.550000pt}{2.400000pt}}{0.000000pt}%
\pgfpathmoveto{\pgfqpoint{0.925000in}{3.838231in}}%
\pgfpathlineto{\pgfqpoint{1.202778in}{3.838231in}}%
\pgfusepath{stroke}%
\end{pgfscope}%
\begin{pgfscope}%
\pgfsetbuttcap%
\pgfsetmiterjoin%
\definecolor{currentfill}{rgb}{0.549020,0.235294,1.000000}%
\pgfsetfillcolor{currentfill}%
\pgfsetlinewidth{1.003750pt}%
\definecolor{currentstroke}{rgb}{0.549020,0.235294,1.000000}%
\pgfsetstrokecolor{currentstroke}%
\pgfsetdash{}{0pt}%
\pgfsys@defobject{currentmarker}{\pgfqpoint{-0.041667in}{-0.041667in}}{\pgfqpoint{0.041667in}{0.041667in}}{%
\pgfpathmoveto{\pgfqpoint{0.000000in}{0.041667in}}%
\pgfpathlineto{\pgfqpoint{-0.041667in}{-0.041667in}}%
\pgfpathlineto{\pgfqpoint{0.041667in}{-0.041667in}}%
\pgfpathclose%
\pgfusepath{stroke,fill}%
}%
\begin{pgfscope}%
\pgfsys@transformshift{1.063889in}{3.838231in}%
\pgfsys@useobject{currentmarker}{}%
\end{pgfscope}%
\end{pgfscope}%
\begin{pgfscope}%
\definecolor{textcolor}{rgb}{0.000000,0.000000,0.000000}%
\pgfsetstrokecolor{textcolor}%
\pgfsetfillcolor{textcolor}%
\pgftext[x=1.313889in,y=3.789620in,left,base]{\color{textcolor}\sffamily\fontsize{10.000000}{12.000000}\selectfont \(\displaystyle \beta = 0.9\)}%
\end{pgfscope}%
\begin{pgfscope}%
\pgfsetbuttcap%
\pgfsetroundjoin%
\pgfsetlinewidth{1.505625pt}%
\definecolor{currentstroke}{rgb}{0.007843,0.533333,0.000000}%
\pgfsetstrokecolor{currentstroke}%
\pgfsetdash{{5.550000pt}{2.400000pt}}{0.000000pt}%
\pgfpathmoveto{\pgfqpoint{0.925000in}{3.634374in}}%
\pgfpathlineto{\pgfqpoint{1.202778in}{3.634374in}}%
\pgfusepath{stroke}%
\end{pgfscope}%
\begin{pgfscope}%
\pgfsetbuttcap%
\pgfsetmiterjoin%
\definecolor{currentfill}{rgb}{0.007843,0.533333,0.000000}%
\pgfsetfillcolor{currentfill}%
\pgfsetlinewidth{1.003750pt}%
\definecolor{currentstroke}{rgb}{0.007843,0.533333,0.000000}%
\pgfsetstrokecolor{currentstroke}%
\pgfsetdash{}{0pt}%
\pgfsys@defobject{currentmarker}{\pgfqpoint{-0.041667in}{-0.041667in}}{\pgfqpoint{0.041667in}{0.041667in}}{%
\pgfpathmoveto{\pgfqpoint{-0.000000in}{-0.041667in}}%
\pgfpathlineto{\pgfqpoint{0.041667in}{0.041667in}}%
\pgfpathlineto{\pgfqpoint{-0.041667in}{0.041667in}}%
\pgfpathclose%
\pgfusepath{stroke,fill}%
}%
\begin{pgfscope}%
\pgfsys@transformshift{1.063889in}{3.634374in}%
\pgfsys@useobject{currentmarker}{}%
\end{pgfscope}%
\end{pgfscope}%
\begin{pgfscope}%
\definecolor{textcolor}{rgb}{0.000000,0.000000,0.000000}%
\pgfsetstrokecolor{textcolor}%
\pgfsetfillcolor{textcolor}%
\pgftext[x=1.313889in,y=3.585763in,left,base]{\color{textcolor}\sffamily\fontsize{10.000000}{12.000000}\selectfont \(\displaystyle \beta = 1\)}%
\end{pgfscope}%
\end{pgfpicture}%
\makeatother%
\endgroup%

      \caption{GMRES iteration counts for $\AmatoI\Amatt$ given by \cref{eq:noweak,eq:ntweak}, where $\alpha = 0.2/k^\beta,$ for $\beta = 0.8,0.9,1.$}\label{fig:l1high}
\end{figure}
\optodo{Remove border on plots - frameon argument to fig?}      
\begin{table}
  \centering
  \begin{tabular}{Sc Sc Sc Sc Sc Sc Sc Sc Sc ScSc }
\toprule

$\eps$\textbackslash$k$ &  10.0  &  20.0  &  30.0  &  40.0  &  50.0  &  60.0  &  70.0  &  80.0  &  90.0  &  100.0 \\

\midrule

0.0 &     14 &     40 &    119 &    258 &    427 &    627 &    940 &   1274 &   1695 &   2116 \\

0.1 &     13 &     27 &     70 &    147 &    262 &    394 &    590 &    825 &   1128 &   1393 \\

0.2 &     12 &     22 &     40 &     77 &    134 &    199 &    292 &    419 &    551 &    726 \\

0.3 &     11 &     18 &     25 &     40 &     58 &     86 &    119 &    163 &    209 &    270 \\

0.4 &     10 &     15 &     20 &     25 &     30 &     42 &     53 &     64 &     81 &     98 \\

0.5 &     10 &     13 &     16 &     19 &     22 &     25 &     28 &     31 &     37 &     41 \\

0.6 &      9 &     11 &     13 &     14 &     16 &     17 &     19 &     19 &     21 &     22 \\

0.7 &      8 &      9 &     10 &     11 &     12 &     13 &     13 &     14 &     14 &     14 \\

0.8 &      8 &      8 &      9 &      9 &     10 &     10 &     10 &     10 &     10 &     10 \\

0.9 &      7 &      7 &      8 &      8 &      8 &      8 &      8 &      8 &      8 &      8 \\

1.0 &      7 &      6 &      7 &      7 &      7 &      7 &      7 &      7 &      7 &      7 \\

\bottomrule

\end{tabular}


  \caption{GMRES iteration counts for $\AmatoI\Amatt$ given by \cref{eq:noweak,eq:ntweak}, where $\alpha = 0.2/k^\beta.$}\label{tab:l1}
  \end{table}

\section{Probabilistic nearby preconditioning results}\label{sec:nbpcstochastic}

Recall from \cref{sec:intronbpc} that our motivation for developing nearby preconditioning was to speed up UQ calculations for the Helmholtz equation. However, all of the results above have been deterministic (as opposed to stochastic) results. Therefore we now turn our attention to obtaining results on the effectiveness of nearby preconditioning for stochastic Helmholtz problems, i.e., \cref{prob:msedp,somsedp,svsedp} from \cref{chap:stochastic}. Firstly, in \cref{cor:stonbpcas} below, prove an `essentially deterministic' result on the effectiveness of nearby preconditioning, before proving probabilistic results on the effectiveness of nearby preconditioning applied to stochastic problems.

Throughout this \cref{sec:nbpcstochastic} we consider \cref{prob:msedp} from \cref{chap:stochastic} but with $A=I$. That is, for simplicity's sake we only consider the case of random $n$, although everything we say could be easily extended to random $A$. To maintain consistent notation with the rest of this \lcnamecref{chap:nbpc} we will use a superscript $(2)$ to refer to the stochastic problem (e.g., the random coefficient will be $\nst(\omega)$, the solution will be $\ust(\omega)$, the matrices arising from the finite-element discretiation will be $\Amatt(\omega),$ etc.). We let $\nso \in \LiDRRR$ define a \emph{deterministic} Helmholtz problem. We will use the discretisation of this deterministic Helmholtz problem to precondition the discretisations of the realisations of the stochastic Helmholtz problem. I.e., we will consider the performance of GMRES applied to
\beq\label{eq:stopc}
\AmatoI\Amatt(\omega)\bu = \AmatoI \bff.
\eeq
For simplicity's sake, in all that follows we will measure $\no-\nt$ in the $L^{\infty}$ norm, or though one could use any of the weaker norms discussed in \cref{sec:weaknorm} above, and obtain analogous results.

\subsection{Probabilistic theory for nearby preconditioning}
If we apply \cref{cor:1a} to the above set-up we can straightforwardly conclude the following \lcnamecref{cor:stonbpcas}

\bco[Almost-sure nearby preconditioning]\label{cor:stonbpcas}
Let the assumptions of \cref{cor:1a} hold, and let $\nt$ satisfy the assumptions at the start of \cref{sec:hh-results}. Then the number of GMRES iterations needed to solve \cref{eq:stopc} is bounded independently of $k$ almost surely if
\beq\label{eq:nbpcas}
\NLiDRRR{\nso-\nst(\omega)} \leq \frac1{2\Ct k}
\eeq
almost surely.
\eco
The numerical results in \cref{sec:num} above can be seen as confirming \cref{cor:stonbpcas}.

\bre[\Cref{cor:stonbpcas} is not ideal]\label{rem:notideal}
\Cref{cor:stonbpcas} is not ideal for two main reasons:
\ben
\item\label[itemreason]{it:notideal1} The condition \cref{eq:nbpcas} must hold almost surely, and
  \item\label[itemreason]{it:notideal2} \Cref{cor:stonbpcas} does not give any explicit information on the distribution of the number of GMRES iterations depends on the distribution of $\NLiDRRR{\nso-\nst}.$
    \een
    \Cref{it:notideal1} is not ideal because in many physically realistic problems $\NLiDRRR{\no-\nt(\omega)}$ may be unbounded (e.g., if $\nt$ is a lognormal random field) or even if bounded may not satisfy the condition \cref{eq:nbpcas} almost surely. \Cref{it:notideal2} is not ideal because it means one cannot infer information about the distribution of the number of GMRES iterations from the distribution of$\NLiDRRR{\nso-\nst(\omega)}.$
    \ere
    In order to correct the deficiencies described in \cref{rem:notideal} we will first prove a bound on the number of GMRES iterations depending explicitly on $\NLiDRRR{\nso-\nst(\omega)}$, and then use this bound to prove a probabalistic estimate (\cref{lem:probgmres1} below) for the number of GMRES iterations. We first define notation for the number of GMRES iterations required for convergence. Because \cref{lem:probgmres1} below is a \emph{deterministic} result, i.e., it does not require $\nst$ to be a random field. Therefore for this \lcnamecref{lem:probgmres1} only, we assume $\nst$ is as given at the beginning of this \lcnamecref{chap:nbpc}.

\bde[Number of GMRES iterations required for convergence]
Let $\GMRES{\eps}{\nso}{\nst}$ denote the number of iterations required for GMRES in the unweighted norm $\Nt{\cdot}$ with $\Nt{\brz} = 1,$ applied to
\beqs
\AmatoI\Amatt  \bu = \AmatoI \bff
\eeqs
to converge to within a tolerance $\eps,$ i.e., to achieve
\beqs
\frac{\Nt{\brm}}{\Nt{\bff}} < \eps.
\eeqs
    \ede
    
\ble[Maximum number of GMRES iterations]\label{lem:probgmres1}
Let $0 < \eps < 1,$ $\nst:\Omega \rightarrow \LiDRRR$ be a random field, and $\no, \,\Dm,$ and $f$ be as in \cref{sec:problem}, and let $\dofs$ denote the number of degrees of freedom, i.e. the size of the matrices $\Amato$ and $\Amatt$. Then there exists a function $\Gfnname:\RRp\rightarrow [0,\dofs]$ such that
\beqs
\GMRES{\eps}{\nso}{\nst} \leq \Gfn{\nso-\nst}.
\eeqs

Moreover, $\Gfnname$ is given by
\beq\label{eq:gdef}
\Gfn{\NLiDRRR{\nso-\nst}} =
\begin{dcases}
\min\set{N,\frac{\loge{\eps}}{\loge\mleft(\frac{2\alpha^{1/2}}{\mleft(1+\alpha\mright)^2}\mright)}+1} & \tif \alpha < 1\\
N & \tif \alpha \geq 1,
\end{dcases}
\eeq

where $\alpha = \Ct k \NLiDRRR{\nso-\nst},$ where $\Ct$ is given by \cref{eq:C2}.
\ele

See \cref{fig:G} for some examples of the function $\Gfnname$.

The proof of \cref{lem:probgmres1} uses the following corollary \cite[Corollary 3]{SaSc:86} of \cite[Proposition 2]{SaSc:86} on the `lucky breakdown' of GMRES.
\bco[Guaranteed convergence of GMRES]\label{cor:gmresguaranteed}
For an $N \times N$ problem GMRES converges in at most $N$ iterations.
\eco

\bpf[Proof of \cref{lem:probgmres1}]
For $\alpha \geq 1$, the result is immediate from \cref{cor:gmresguaranteed}. For $\alpha < 1,$ if we insert \cref{eq:gmressin} (the corollary of the ELman estimate) into the Elman estimate \cref{eq:Elman} (with $\Dmat=\Imat,$ so $\NDmat{\cdot} = \Nt{\cdot}$), we obtain, for $m \in \NN$
\beq\label{eq:gmressub}
\frac{\Nt{\brm}}{\Nt{\brz}} \leq \mleft(\frac{2\sqrt{\alpha}}{\mleft(1+\alpha\mright)^2}\mright)^m.
\eeq
To obtain a bound on the number of iterations needed to obtain the solution to within a tolerance $\eps,$ we set the right-hand side of \cref{eq:gmressub} to be less than $\eps$ and solve for $m$ to obtain that the GMRES residual is less than $\eps$ (recall we assume $\Nt{\brz} = 1$) if
\beq\label{eq:mlower}
m \geq \frac{\loge{\eps}}{\loge\mleft(\frac{2\alpha^{1/2}}{\mleft(1+\alpha\mright)^2}\mright)}.
\eeq
Hence, if $\ms$ is the smallest integer satisfying \cref{eq:mlower}, then
\beq\label{eq:gmressub2}
\ms  \leq\frac{\loge{\eps}}{\loge\mleft(\frac{2\alpha^{1/2}}{\mleft(1+\alpha\mright)^2}\mright)}+1.
\eeq
The result for $\alpha < 1$ therefore follows from \cref{eq:gmressub2,cor:gmresguaranteed} as GMRES will have converged to within a tolerance $\eps$ within $\ms$ iterations.
\epf

\bre[Why not use the ceiling in \cref{eq:gmressub2}?]
One could replace the bound \cref{eq:gmressub2} by the equality
\beqs
\ms  =\ceil{\frac{\loge{\eps}}{\loge\mleft(\frac{2\alpha^{1/2}}{\mleft(1+\alpha\mright)^2}\mright)}}.
\eeqs
However, the change in the definition of $\Gfnname$ \cref{eq:gdef} would mean $\Gfnname$ would no longer be continuous. As we must use numerical methods to calculate probabilities associated with $\Gfnname$ (see \cref{thm:probgmres,rem:computable} below), it is convenient if $\Gfnname$ is continuous, and so we use \cref{eq:gmressub2}.
\ere

\bre[Why the dependence on $\alpha$ in \cref{lem:probgmres1}?]
The reason that \cref{eq:gdef} has two cases depending on $\alpha = \Ct k \NLiDRRR{\nso-\nst}$ is because \cref{cor:GMRES_intro} only holds if $\alpha < 1$. Therefore if $\alpha \geq 1$ the only result available to us is \cref{cor:gmresguaranteed}.
\ere

\begin{figure}
  \centering
  %% Creator: Matplotlib, PGF backend
%%
%% To include the figure in your LaTeX document, write
%%   \input{<filename>.pgf}
%%
%% Make sure the required packages are loaded in your preamble
%%   \usepackage{pgf}
%%
%% Figures using additional raster images can only be included by \input if
%% they are in the same directory as the main LaTeX file. For loading figures
%% from other directories you can use the `import` package
%%   \usepackage{import}
%% and then include the figures with
%%   \import{<path to file>}{<filename>.pgf}
%%
%% Matplotlib used the following preamble
%%   \usepackage{amssymb}
%%   \usepackage{mleftright}
%%   \usepackage{fontspec}
%%   \setmainfont{DejaVuSerif.ttf}[Path=/home/owen/progs/firedrake-complex/firedrake/lib/python3.5/site-packages/matplotlib/mpl-data/fonts/ttf/]
%%   \setsansfont{DejaVuSans.ttf}[Path=/home/owen/progs/firedrake-complex/firedrake/lib/python3.5/site-packages/matplotlib/mpl-data/fonts/ttf/]
%%   \setmonofont{DejaVuSansMono.ttf}[Path=/home/owen/progs/firedrake-complex/firedrake/lib/python3.5/site-packages/matplotlib/mpl-data/fonts/ttf/]
%%
\begingroup%
\makeatletter%
\begin{pgfpicture}%
\pgfpathrectangle{\pgfpointorigin}{\pgfqpoint{6.400000in}{4.800000in}}%
\pgfusepath{use as bounding box, clip}%
\begin{pgfscope}%
\pgfsetbuttcap%
\pgfsetmiterjoin%
\definecolor{currentfill}{rgb}{1.000000,1.000000,1.000000}%
\pgfsetfillcolor{currentfill}%
\pgfsetlinewidth{0.000000pt}%
\definecolor{currentstroke}{rgb}{1.000000,1.000000,1.000000}%
\pgfsetstrokecolor{currentstroke}%
\pgfsetdash{}{0pt}%
\pgfpathmoveto{\pgfqpoint{0.000000in}{0.000000in}}%
\pgfpathlineto{\pgfqpoint{6.400000in}{0.000000in}}%
\pgfpathlineto{\pgfqpoint{6.400000in}{4.800000in}}%
\pgfpathlineto{\pgfqpoint{0.000000in}{4.800000in}}%
\pgfpathclose%
\pgfusepath{fill}%
\end{pgfscope}%
\begin{pgfscope}%
\pgfsetbuttcap%
\pgfsetmiterjoin%
\definecolor{currentfill}{rgb}{1.000000,1.000000,1.000000}%
\pgfsetfillcolor{currentfill}%
\pgfsetlinewidth{0.000000pt}%
\definecolor{currentstroke}{rgb}{0.000000,0.000000,0.000000}%
\pgfsetstrokecolor{currentstroke}%
\pgfsetstrokeopacity{0.000000}%
\pgfsetdash{}{0pt}%
\pgfpathmoveto{\pgfqpoint{0.800000in}{0.528000in}}%
\pgfpathlineto{\pgfqpoint{5.760000in}{0.528000in}}%
\pgfpathlineto{\pgfqpoint{5.760000in}{4.224000in}}%
\pgfpathlineto{\pgfqpoint{0.800000in}{4.224000in}}%
\pgfpathclose%
\pgfusepath{fill}%
\end{pgfscope}%
\begin{pgfscope}%
\pgfsetbuttcap%
\pgfsetroundjoin%
\definecolor{currentfill}{rgb}{0.000000,0.000000,0.000000}%
\pgfsetfillcolor{currentfill}%
\pgfsetlinewidth{0.803000pt}%
\definecolor{currentstroke}{rgb}{0.000000,0.000000,0.000000}%
\pgfsetstrokecolor{currentstroke}%
\pgfsetdash{}{0pt}%
\pgfsys@defobject{currentmarker}{\pgfqpoint{0.000000in}{-0.048611in}}{\pgfqpoint{0.000000in}{0.000000in}}{%
\pgfpathmoveto{\pgfqpoint{0.000000in}{0.000000in}}%
\pgfpathlineto{\pgfqpoint{0.000000in}{-0.048611in}}%
\pgfusepath{stroke,fill}%
}%
\begin{pgfscope}%
\pgfsys@transformshift{0.979908in}{0.528000in}%
\pgfsys@useobject{currentmarker}{}%
\end{pgfscope}%
\end{pgfscope}%
\begin{pgfscope}%
\definecolor{textcolor}{rgb}{0.000000,0.000000,0.000000}%
\pgfsetstrokecolor{textcolor}%
\pgfsetfillcolor{textcolor}%
\pgftext[x=0.979908in,y=0.430778in,,top]{\color{textcolor}\sffamily\fontsize{10.000000}{12.000000}\selectfont 0.0}%
\end{pgfscope}%
\begin{pgfscope}%
\pgfsetbuttcap%
\pgfsetroundjoin%
\definecolor{currentfill}{rgb}{0.000000,0.000000,0.000000}%
\pgfsetfillcolor{currentfill}%
\pgfsetlinewidth{0.803000pt}%
\definecolor{currentstroke}{rgb}{0.000000,0.000000,0.000000}%
\pgfsetstrokecolor{currentstroke}%
\pgfsetdash{}{0pt}%
\pgfsys@defobject{currentmarker}{\pgfqpoint{0.000000in}{-0.048611in}}{\pgfqpoint{0.000000in}{0.000000in}}{%
\pgfpathmoveto{\pgfqpoint{0.000000in}{0.000000in}}%
\pgfpathlineto{\pgfqpoint{0.000000in}{-0.048611in}}%
\pgfusepath{stroke,fill}%
}%
\begin{pgfscope}%
\pgfsys@transformshift{1.890836in}{0.528000in}%
\pgfsys@useobject{currentmarker}{}%
\end{pgfscope}%
\end{pgfscope}%
\begin{pgfscope}%
\definecolor{textcolor}{rgb}{0.000000,0.000000,0.000000}%
\pgfsetstrokecolor{textcolor}%
\pgfsetfillcolor{textcolor}%
\pgftext[x=1.890836in,y=0.430778in,,top]{\color{textcolor}\sffamily\fontsize{10.000000}{12.000000}\selectfont 0.2}%
\end{pgfscope}%
\begin{pgfscope}%
\pgfsetbuttcap%
\pgfsetroundjoin%
\definecolor{currentfill}{rgb}{0.000000,0.000000,0.000000}%
\pgfsetfillcolor{currentfill}%
\pgfsetlinewidth{0.803000pt}%
\definecolor{currentstroke}{rgb}{0.000000,0.000000,0.000000}%
\pgfsetstrokecolor{currentstroke}%
\pgfsetdash{}{0pt}%
\pgfsys@defobject{currentmarker}{\pgfqpoint{0.000000in}{-0.048611in}}{\pgfqpoint{0.000000in}{0.000000in}}{%
\pgfpathmoveto{\pgfqpoint{0.000000in}{0.000000in}}%
\pgfpathlineto{\pgfqpoint{0.000000in}{-0.048611in}}%
\pgfusepath{stroke,fill}%
}%
\begin{pgfscope}%
\pgfsys@transformshift{2.801763in}{0.528000in}%
\pgfsys@useobject{currentmarker}{}%
\end{pgfscope}%
\end{pgfscope}%
\begin{pgfscope}%
\definecolor{textcolor}{rgb}{0.000000,0.000000,0.000000}%
\pgfsetstrokecolor{textcolor}%
\pgfsetfillcolor{textcolor}%
\pgftext[x=2.801763in,y=0.430778in,,top]{\color{textcolor}\sffamily\fontsize{10.000000}{12.000000}\selectfont 0.4}%
\end{pgfscope}%
\begin{pgfscope}%
\pgfsetbuttcap%
\pgfsetroundjoin%
\definecolor{currentfill}{rgb}{0.000000,0.000000,0.000000}%
\pgfsetfillcolor{currentfill}%
\pgfsetlinewidth{0.803000pt}%
\definecolor{currentstroke}{rgb}{0.000000,0.000000,0.000000}%
\pgfsetstrokecolor{currentstroke}%
\pgfsetdash{}{0pt}%
\pgfsys@defobject{currentmarker}{\pgfqpoint{0.000000in}{-0.048611in}}{\pgfqpoint{0.000000in}{0.000000in}}{%
\pgfpathmoveto{\pgfqpoint{0.000000in}{0.000000in}}%
\pgfpathlineto{\pgfqpoint{0.000000in}{-0.048611in}}%
\pgfusepath{stroke,fill}%
}%
\begin{pgfscope}%
\pgfsys@transformshift{3.712691in}{0.528000in}%
\pgfsys@useobject{currentmarker}{}%
\end{pgfscope}%
\end{pgfscope}%
\begin{pgfscope}%
\definecolor{textcolor}{rgb}{0.000000,0.000000,0.000000}%
\pgfsetstrokecolor{textcolor}%
\pgfsetfillcolor{textcolor}%
\pgftext[x=3.712691in,y=0.430778in,,top]{\color{textcolor}\sffamily\fontsize{10.000000}{12.000000}\selectfont 0.6}%
\end{pgfscope}%
\begin{pgfscope}%
\pgfsetbuttcap%
\pgfsetroundjoin%
\definecolor{currentfill}{rgb}{0.000000,0.000000,0.000000}%
\pgfsetfillcolor{currentfill}%
\pgfsetlinewidth{0.803000pt}%
\definecolor{currentstroke}{rgb}{0.000000,0.000000,0.000000}%
\pgfsetstrokecolor{currentstroke}%
\pgfsetdash{}{0pt}%
\pgfsys@defobject{currentmarker}{\pgfqpoint{0.000000in}{-0.048611in}}{\pgfqpoint{0.000000in}{0.000000in}}{%
\pgfpathmoveto{\pgfqpoint{0.000000in}{0.000000in}}%
\pgfpathlineto{\pgfqpoint{0.000000in}{-0.048611in}}%
\pgfusepath{stroke,fill}%
}%
\begin{pgfscope}%
\pgfsys@transformshift{4.623618in}{0.528000in}%
\pgfsys@useobject{currentmarker}{}%
\end{pgfscope}%
\end{pgfscope}%
\begin{pgfscope}%
\definecolor{textcolor}{rgb}{0.000000,0.000000,0.000000}%
\pgfsetstrokecolor{textcolor}%
\pgfsetfillcolor{textcolor}%
\pgftext[x=4.623618in,y=0.430778in,,top]{\color{textcolor}\sffamily\fontsize{10.000000}{12.000000}\selectfont 0.8}%
\end{pgfscope}%
\begin{pgfscope}%
\pgfsetbuttcap%
\pgfsetroundjoin%
\definecolor{currentfill}{rgb}{0.000000,0.000000,0.000000}%
\pgfsetfillcolor{currentfill}%
\pgfsetlinewidth{0.803000pt}%
\definecolor{currentstroke}{rgb}{0.000000,0.000000,0.000000}%
\pgfsetstrokecolor{currentstroke}%
\pgfsetdash{}{0pt}%
\pgfsys@defobject{currentmarker}{\pgfqpoint{0.000000in}{-0.048611in}}{\pgfqpoint{0.000000in}{0.000000in}}{%
\pgfpathmoveto{\pgfqpoint{0.000000in}{0.000000in}}%
\pgfpathlineto{\pgfqpoint{0.000000in}{-0.048611in}}%
\pgfusepath{stroke,fill}%
}%
\begin{pgfscope}%
\pgfsys@transformshift{5.534545in}{0.528000in}%
\pgfsys@useobject{currentmarker}{}%
\end{pgfscope}%
\end{pgfscope}%
\begin{pgfscope}%
\definecolor{textcolor}{rgb}{0.000000,0.000000,0.000000}%
\pgfsetstrokecolor{textcolor}%
\pgfsetfillcolor{textcolor}%
\pgftext[x=5.534545in,y=0.430778in,,top]{\color{textcolor}\sffamily\fontsize{10.000000}{12.000000}\selectfont 1.0}%
\end{pgfscope}%
\begin{pgfscope}%
\definecolor{textcolor}{rgb}{0.000000,0.000000,0.000000}%
\pgfsetstrokecolor{textcolor}%
\pgfsetfillcolor{textcolor}%
\pgftext[x=3.280000in,y=0.240809in,,top]{\color{textcolor}\sffamily\fontsize{10.000000}{12.000000}\selectfont \(\displaystyle \|n_{1} - n_{2}\|_{L^{\infty}\mleft(D_{R},\mathbb{R}\mright)}\)}%
\end{pgfscope}%
\begin{pgfscope}%
\pgfsetbuttcap%
\pgfsetroundjoin%
\definecolor{currentfill}{rgb}{0.000000,0.000000,0.000000}%
\pgfsetfillcolor{currentfill}%
\pgfsetlinewidth{0.803000pt}%
\definecolor{currentstroke}{rgb}{0.000000,0.000000,0.000000}%
\pgfsetstrokecolor{currentstroke}%
\pgfsetdash{}{0pt}%
\pgfsys@defobject{currentmarker}{\pgfqpoint{-0.048611in}{0.000000in}}{\pgfqpoint{0.000000in}{0.000000in}}{%
\pgfpathmoveto{\pgfqpoint{0.000000in}{0.000000in}}%
\pgfpathlineto{\pgfqpoint{-0.048611in}{0.000000in}}%
\pgfusepath{stroke,fill}%
}%
\begin{pgfscope}%
\pgfsys@transformshift{0.800000in}{0.700721in}%
\pgfsys@useobject{currentmarker}{}%
\end{pgfscope}%
\end{pgfscope}%
\begin{pgfscope}%
\definecolor{textcolor}{rgb}{0.000000,0.000000,0.000000}%
\pgfsetstrokecolor{textcolor}%
\pgfsetfillcolor{textcolor}%
\pgftext[x=0.501581in,y=0.647960in,left,base]{\color{textcolor}\sffamily\fontsize{10.000000}{12.000000}\selectfont \(\displaystyle {10^{1}}\)}%
\end{pgfscope}%
\begin{pgfscope}%
\pgfsetbuttcap%
\pgfsetroundjoin%
\definecolor{currentfill}{rgb}{0.000000,0.000000,0.000000}%
\pgfsetfillcolor{currentfill}%
\pgfsetlinewidth{0.803000pt}%
\definecolor{currentstroke}{rgb}{0.000000,0.000000,0.000000}%
\pgfsetstrokecolor{currentstroke}%
\pgfsetdash{}{0pt}%
\pgfsys@defobject{currentmarker}{\pgfqpoint{-0.048611in}{0.000000in}}{\pgfqpoint{0.000000in}{0.000000in}}{%
\pgfpathmoveto{\pgfqpoint{0.000000in}{0.000000in}}%
\pgfpathlineto{\pgfqpoint{-0.048611in}{0.000000in}}%
\pgfusepath{stroke,fill}%
}%
\begin{pgfscope}%
\pgfsys@transformshift{0.800000in}{1.371777in}%
\pgfsys@useobject{currentmarker}{}%
\end{pgfscope}%
\end{pgfscope}%
\begin{pgfscope}%
\definecolor{textcolor}{rgb}{0.000000,0.000000,0.000000}%
\pgfsetstrokecolor{textcolor}%
\pgfsetfillcolor{textcolor}%
\pgftext[x=0.501581in,y=1.319015in,left,base]{\color{textcolor}\sffamily\fontsize{10.000000}{12.000000}\selectfont \(\displaystyle {10^{2}}\)}%
\end{pgfscope}%
\begin{pgfscope}%
\pgfsetbuttcap%
\pgfsetroundjoin%
\definecolor{currentfill}{rgb}{0.000000,0.000000,0.000000}%
\pgfsetfillcolor{currentfill}%
\pgfsetlinewidth{0.803000pt}%
\definecolor{currentstroke}{rgb}{0.000000,0.000000,0.000000}%
\pgfsetstrokecolor{currentstroke}%
\pgfsetdash{}{0pt}%
\pgfsys@defobject{currentmarker}{\pgfqpoint{-0.048611in}{0.000000in}}{\pgfqpoint{0.000000in}{0.000000in}}{%
\pgfpathmoveto{\pgfqpoint{0.000000in}{0.000000in}}%
\pgfpathlineto{\pgfqpoint{-0.048611in}{0.000000in}}%
\pgfusepath{stroke,fill}%
}%
\begin{pgfscope}%
\pgfsys@transformshift{0.800000in}{2.042833in}%
\pgfsys@useobject{currentmarker}{}%
\end{pgfscope}%
\end{pgfscope}%
\begin{pgfscope}%
\definecolor{textcolor}{rgb}{0.000000,0.000000,0.000000}%
\pgfsetstrokecolor{textcolor}%
\pgfsetfillcolor{textcolor}%
\pgftext[x=0.501581in,y=1.990071in,left,base]{\color{textcolor}\sffamily\fontsize{10.000000}{12.000000}\selectfont \(\displaystyle {10^{3}}\)}%
\end{pgfscope}%
\begin{pgfscope}%
\pgfsetbuttcap%
\pgfsetroundjoin%
\definecolor{currentfill}{rgb}{0.000000,0.000000,0.000000}%
\pgfsetfillcolor{currentfill}%
\pgfsetlinewidth{0.803000pt}%
\definecolor{currentstroke}{rgb}{0.000000,0.000000,0.000000}%
\pgfsetstrokecolor{currentstroke}%
\pgfsetdash{}{0pt}%
\pgfsys@defobject{currentmarker}{\pgfqpoint{-0.048611in}{0.000000in}}{\pgfqpoint{0.000000in}{0.000000in}}{%
\pgfpathmoveto{\pgfqpoint{0.000000in}{0.000000in}}%
\pgfpathlineto{\pgfqpoint{-0.048611in}{0.000000in}}%
\pgfusepath{stroke,fill}%
}%
\begin{pgfscope}%
\pgfsys@transformshift{0.800000in}{2.713889in}%
\pgfsys@useobject{currentmarker}{}%
\end{pgfscope}%
\end{pgfscope}%
\begin{pgfscope}%
\definecolor{textcolor}{rgb}{0.000000,0.000000,0.000000}%
\pgfsetstrokecolor{textcolor}%
\pgfsetfillcolor{textcolor}%
\pgftext[x=0.501581in,y=2.661127in,left,base]{\color{textcolor}\sffamily\fontsize{10.000000}{12.000000}\selectfont \(\displaystyle {10^{4}}\)}%
\end{pgfscope}%
\begin{pgfscope}%
\pgfsetbuttcap%
\pgfsetroundjoin%
\definecolor{currentfill}{rgb}{0.000000,0.000000,0.000000}%
\pgfsetfillcolor{currentfill}%
\pgfsetlinewidth{0.803000pt}%
\definecolor{currentstroke}{rgb}{0.000000,0.000000,0.000000}%
\pgfsetstrokecolor{currentstroke}%
\pgfsetdash{}{0pt}%
\pgfsys@defobject{currentmarker}{\pgfqpoint{-0.048611in}{0.000000in}}{\pgfqpoint{0.000000in}{0.000000in}}{%
\pgfpathmoveto{\pgfqpoint{0.000000in}{0.000000in}}%
\pgfpathlineto{\pgfqpoint{-0.048611in}{0.000000in}}%
\pgfusepath{stroke,fill}%
}%
\begin{pgfscope}%
\pgfsys@transformshift{0.800000in}{3.384944in}%
\pgfsys@useobject{currentmarker}{}%
\end{pgfscope}%
\end{pgfscope}%
\begin{pgfscope}%
\definecolor{textcolor}{rgb}{0.000000,0.000000,0.000000}%
\pgfsetstrokecolor{textcolor}%
\pgfsetfillcolor{textcolor}%
\pgftext[x=0.501581in,y=3.332183in,left,base]{\color{textcolor}\sffamily\fontsize{10.000000}{12.000000}\selectfont \(\displaystyle {10^{5}}\)}%
\end{pgfscope}%
\begin{pgfscope}%
\pgfsetbuttcap%
\pgfsetroundjoin%
\definecolor{currentfill}{rgb}{0.000000,0.000000,0.000000}%
\pgfsetfillcolor{currentfill}%
\pgfsetlinewidth{0.803000pt}%
\definecolor{currentstroke}{rgb}{0.000000,0.000000,0.000000}%
\pgfsetstrokecolor{currentstroke}%
\pgfsetdash{}{0pt}%
\pgfsys@defobject{currentmarker}{\pgfqpoint{-0.048611in}{0.000000in}}{\pgfqpoint{0.000000in}{0.000000in}}{%
\pgfpathmoveto{\pgfqpoint{0.000000in}{0.000000in}}%
\pgfpathlineto{\pgfqpoint{-0.048611in}{0.000000in}}%
\pgfusepath{stroke,fill}%
}%
\begin{pgfscope}%
\pgfsys@transformshift{0.800000in}{4.056000in}%
\pgfsys@useobject{currentmarker}{}%
\end{pgfscope}%
\end{pgfscope}%
\begin{pgfscope}%
\definecolor{textcolor}{rgb}{0.000000,0.000000,0.000000}%
\pgfsetstrokecolor{textcolor}%
\pgfsetfillcolor{textcolor}%
\pgftext[x=0.501581in,y=4.003238in,left,base]{\color{textcolor}\sffamily\fontsize{10.000000}{12.000000}\selectfont \(\displaystyle {10^{6}}\)}%
\end{pgfscope}%
\begin{pgfscope}%
\pgfsetbuttcap%
\pgfsetroundjoin%
\definecolor{currentfill}{rgb}{0.000000,0.000000,0.000000}%
\pgfsetfillcolor{currentfill}%
\pgfsetlinewidth{0.602250pt}%
\definecolor{currentstroke}{rgb}{0.000000,0.000000,0.000000}%
\pgfsetstrokecolor{currentstroke}%
\pgfsetdash{}{0pt}%
\pgfsys@defobject{currentmarker}{\pgfqpoint{-0.027778in}{0.000000in}}{\pgfqpoint{0.000000in}{0.000000in}}{%
\pgfpathmoveto{\pgfqpoint{0.000000in}{0.000000in}}%
\pgfpathlineto{\pgfqpoint{-0.027778in}{0.000000in}}%
\pgfusepath{stroke,fill}%
}%
\begin{pgfscope}%
\pgfsys@transformshift{0.800000in}{0.551848in}%
\pgfsys@useobject{currentmarker}{}%
\end{pgfscope}%
\end{pgfscope}%
\begin{pgfscope}%
\pgfsetbuttcap%
\pgfsetroundjoin%
\definecolor{currentfill}{rgb}{0.000000,0.000000,0.000000}%
\pgfsetfillcolor{currentfill}%
\pgfsetlinewidth{0.602250pt}%
\definecolor{currentstroke}{rgb}{0.000000,0.000000,0.000000}%
\pgfsetstrokecolor{currentstroke}%
\pgfsetdash{}{0pt}%
\pgfsys@defobject{currentmarker}{\pgfqpoint{-0.027778in}{0.000000in}}{\pgfqpoint{0.000000in}{0.000000in}}{%
\pgfpathmoveto{\pgfqpoint{0.000000in}{0.000000in}}%
\pgfpathlineto{\pgfqpoint{-0.027778in}{0.000000in}}%
\pgfusepath{stroke,fill}%
}%
\begin{pgfscope}%
\pgfsys@transformshift{0.800000in}{0.596773in}%
\pgfsys@useobject{currentmarker}{}%
\end{pgfscope}%
\end{pgfscope}%
\begin{pgfscope}%
\pgfsetbuttcap%
\pgfsetroundjoin%
\definecolor{currentfill}{rgb}{0.000000,0.000000,0.000000}%
\pgfsetfillcolor{currentfill}%
\pgfsetlinewidth{0.602250pt}%
\definecolor{currentstroke}{rgb}{0.000000,0.000000,0.000000}%
\pgfsetstrokecolor{currentstroke}%
\pgfsetdash{}{0pt}%
\pgfsys@defobject{currentmarker}{\pgfqpoint{-0.027778in}{0.000000in}}{\pgfqpoint{0.000000in}{0.000000in}}{%
\pgfpathmoveto{\pgfqpoint{0.000000in}{0.000000in}}%
\pgfpathlineto{\pgfqpoint{-0.027778in}{0.000000in}}%
\pgfusepath{stroke,fill}%
}%
\begin{pgfscope}%
\pgfsys@transformshift{0.800000in}{0.635689in}%
\pgfsys@useobject{currentmarker}{}%
\end{pgfscope}%
\end{pgfscope}%
\begin{pgfscope}%
\pgfsetbuttcap%
\pgfsetroundjoin%
\definecolor{currentfill}{rgb}{0.000000,0.000000,0.000000}%
\pgfsetfillcolor{currentfill}%
\pgfsetlinewidth{0.602250pt}%
\definecolor{currentstroke}{rgb}{0.000000,0.000000,0.000000}%
\pgfsetstrokecolor{currentstroke}%
\pgfsetdash{}{0pt}%
\pgfsys@defobject{currentmarker}{\pgfqpoint{-0.027778in}{0.000000in}}{\pgfqpoint{0.000000in}{0.000000in}}{%
\pgfpathmoveto{\pgfqpoint{0.000000in}{0.000000in}}%
\pgfpathlineto{\pgfqpoint{-0.027778in}{0.000000in}}%
\pgfusepath{stroke,fill}%
}%
\begin{pgfscope}%
\pgfsys@transformshift{0.800000in}{0.670015in}%
\pgfsys@useobject{currentmarker}{}%
\end{pgfscope}%
\end{pgfscope}%
\begin{pgfscope}%
\pgfsetbuttcap%
\pgfsetroundjoin%
\definecolor{currentfill}{rgb}{0.000000,0.000000,0.000000}%
\pgfsetfillcolor{currentfill}%
\pgfsetlinewidth{0.602250pt}%
\definecolor{currentstroke}{rgb}{0.000000,0.000000,0.000000}%
\pgfsetstrokecolor{currentstroke}%
\pgfsetdash{}{0pt}%
\pgfsys@defobject{currentmarker}{\pgfqpoint{-0.027778in}{0.000000in}}{\pgfqpoint{0.000000in}{0.000000in}}{%
\pgfpathmoveto{\pgfqpoint{0.000000in}{0.000000in}}%
\pgfpathlineto{\pgfqpoint{-0.027778in}{0.000000in}}%
\pgfusepath{stroke,fill}%
}%
\begin{pgfscope}%
\pgfsys@transformshift{0.800000in}{0.902729in}%
\pgfsys@useobject{currentmarker}{}%
\end{pgfscope}%
\end{pgfscope}%
\begin{pgfscope}%
\pgfsetbuttcap%
\pgfsetroundjoin%
\definecolor{currentfill}{rgb}{0.000000,0.000000,0.000000}%
\pgfsetfillcolor{currentfill}%
\pgfsetlinewidth{0.602250pt}%
\definecolor{currentstroke}{rgb}{0.000000,0.000000,0.000000}%
\pgfsetstrokecolor{currentstroke}%
\pgfsetdash{}{0pt}%
\pgfsys@defobject{currentmarker}{\pgfqpoint{-0.027778in}{0.000000in}}{\pgfqpoint{0.000000in}{0.000000in}}{%
\pgfpathmoveto{\pgfqpoint{0.000000in}{0.000000in}}%
\pgfpathlineto{\pgfqpoint{-0.027778in}{0.000000in}}%
\pgfusepath{stroke,fill}%
}%
\begin{pgfscope}%
\pgfsys@transformshift{0.800000in}{1.020896in}%
\pgfsys@useobject{currentmarker}{}%
\end{pgfscope}%
\end{pgfscope}%
\begin{pgfscope}%
\pgfsetbuttcap%
\pgfsetroundjoin%
\definecolor{currentfill}{rgb}{0.000000,0.000000,0.000000}%
\pgfsetfillcolor{currentfill}%
\pgfsetlinewidth{0.602250pt}%
\definecolor{currentstroke}{rgb}{0.000000,0.000000,0.000000}%
\pgfsetstrokecolor{currentstroke}%
\pgfsetdash{}{0pt}%
\pgfsys@defobject{currentmarker}{\pgfqpoint{-0.027778in}{0.000000in}}{\pgfqpoint{0.000000in}{0.000000in}}{%
\pgfpathmoveto{\pgfqpoint{0.000000in}{0.000000in}}%
\pgfpathlineto{\pgfqpoint{-0.027778in}{0.000000in}}%
\pgfusepath{stroke,fill}%
}%
\begin{pgfscope}%
\pgfsys@transformshift{0.800000in}{1.104737in}%
\pgfsys@useobject{currentmarker}{}%
\end{pgfscope}%
\end{pgfscope}%
\begin{pgfscope}%
\pgfsetbuttcap%
\pgfsetroundjoin%
\definecolor{currentfill}{rgb}{0.000000,0.000000,0.000000}%
\pgfsetfillcolor{currentfill}%
\pgfsetlinewidth{0.602250pt}%
\definecolor{currentstroke}{rgb}{0.000000,0.000000,0.000000}%
\pgfsetstrokecolor{currentstroke}%
\pgfsetdash{}{0pt}%
\pgfsys@defobject{currentmarker}{\pgfqpoint{-0.027778in}{0.000000in}}{\pgfqpoint{0.000000in}{0.000000in}}{%
\pgfpathmoveto{\pgfqpoint{0.000000in}{0.000000in}}%
\pgfpathlineto{\pgfqpoint{-0.027778in}{0.000000in}}%
\pgfusepath{stroke,fill}%
}%
\begin{pgfscope}%
\pgfsys@transformshift{0.800000in}{1.169769in}%
\pgfsys@useobject{currentmarker}{}%
\end{pgfscope}%
\end{pgfscope}%
\begin{pgfscope}%
\pgfsetbuttcap%
\pgfsetroundjoin%
\definecolor{currentfill}{rgb}{0.000000,0.000000,0.000000}%
\pgfsetfillcolor{currentfill}%
\pgfsetlinewidth{0.602250pt}%
\definecolor{currentstroke}{rgb}{0.000000,0.000000,0.000000}%
\pgfsetstrokecolor{currentstroke}%
\pgfsetdash{}{0pt}%
\pgfsys@defobject{currentmarker}{\pgfqpoint{-0.027778in}{0.000000in}}{\pgfqpoint{0.000000in}{0.000000in}}{%
\pgfpathmoveto{\pgfqpoint{0.000000in}{0.000000in}}%
\pgfpathlineto{\pgfqpoint{-0.027778in}{0.000000in}}%
\pgfusepath{stroke,fill}%
}%
\begin{pgfscope}%
\pgfsys@transformshift{0.800000in}{1.222904in}%
\pgfsys@useobject{currentmarker}{}%
\end{pgfscope}%
\end{pgfscope}%
\begin{pgfscope}%
\pgfsetbuttcap%
\pgfsetroundjoin%
\definecolor{currentfill}{rgb}{0.000000,0.000000,0.000000}%
\pgfsetfillcolor{currentfill}%
\pgfsetlinewidth{0.602250pt}%
\definecolor{currentstroke}{rgb}{0.000000,0.000000,0.000000}%
\pgfsetstrokecolor{currentstroke}%
\pgfsetdash{}{0pt}%
\pgfsys@defobject{currentmarker}{\pgfqpoint{-0.027778in}{0.000000in}}{\pgfqpoint{0.000000in}{0.000000in}}{%
\pgfpathmoveto{\pgfqpoint{0.000000in}{0.000000in}}%
\pgfpathlineto{\pgfqpoint{-0.027778in}{0.000000in}}%
\pgfusepath{stroke,fill}%
}%
\begin{pgfscope}%
\pgfsys@transformshift{0.800000in}{1.267829in}%
\pgfsys@useobject{currentmarker}{}%
\end{pgfscope}%
\end{pgfscope}%
\begin{pgfscope}%
\pgfsetbuttcap%
\pgfsetroundjoin%
\definecolor{currentfill}{rgb}{0.000000,0.000000,0.000000}%
\pgfsetfillcolor{currentfill}%
\pgfsetlinewidth{0.602250pt}%
\definecolor{currentstroke}{rgb}{0.000000,0.000000,0.000000}%
\pgfsetstrokecolor{currentstroke}%
\pgfsetdash{}{0pt}%
\pgfsys@defobject{currentmarker}{\pgfqpoint{-0.027778in}{0.000000in}}{\pgfqpoint{0.000000in}{0.000000in}}{%
\pgfpathmoveto{\pgfqpoint{0.000000in}{0.000000in}}%
\pgfpathlineto{\pgfqpoint{-0.027778in}{0.000000in}}%
\pgfusepath{stroke,fill}%
}%
\begin{pgfscope}%
\pgfsys@transformshift{0.800000in}{1.306745in}%
\pgfsys@useobject{currentmarker}{}%
\end{pgfscope}%
\end{pgfscope}%
\begin{pgfscope}%
\pgfsetbuttcap%
\pgfsetroundjoin%
\definecolor{currentfill}{rgb}{0.000000,0.000000,0.000000}%
\pgfsetfillcolor{currentfill}%
\pgfsetlinewidth{0.602250pt}%
\definecolor{currentstroke}{rgb}{0.000000,0.000000,0.000000}%
\pgfsetstrokecolor{currentstroke}%
\pgfsetdash{}{0pt}%
\pgfsys@defobject{currentmarker}{\pgfqpoint{-0.027778in}{0.000000in}}{\pgfqpoint{0.000000in}{0.000000in}}{%
\pgfpathmoveto{\pgfqpoint{0.000000in}{0.000000in}}%
\pgfpathlineto{\pgfqpoint{-0.027778in}{0.000000in}}%
\pgfusepath{stroke,fill}%
}%
\begin{pgfscope}%
\pgfsys@transformshift{0.800000in}{1.341071in}%
\pgfsys@useobject{currentmarker}{}%
\end{pgfscope}%
\end{pgfscope}%
\begin{pgfscope}%
\pgfsetbuttcap%
\pgfsetroundjoin%
\definecolor{currentfill}{rgb}{0.000000,0.000000,0.000000}%
\pgfsetfillcolor{currentfill}%
\pgfsetlinewidth{0.602250pt}%
\definecolor{currentstroke}{rgb}{0.000000,0.000000,0.000000}%
\pgfsetstrokecolor{currentstroke}%
\pgfsetdash{}{0pt}%
\pgfsys@defobject{currentmarker}{\pgfqpoint{-0.027778in}{0.000000in}}{\pgfqpoint{0.000000in}{0.000000in}}{%
\pgfpathmoveto{\pgfqpoint{0.000000in}{0.000000in}}%
\pgfpathlineto{\pgfqpoint{-0.027778in}{0.000000in}}%
\pgfusepath{stroke,fill}%
}%
\begin{pgfscope}%
\pgfsys@transformshift{0.800000in}{1.573785in}%
\pgfsys@useobject{currentmarker}{}%
\end{pgfscope}%
\end{pgfscope}%
\begin{pgfscope}%
\pgfsetbuttcap%
\pgfsetroundjoin%
\definecolor{currentfill}{rgb}{0.000000,0.000000,0.000000}%
\pgfsetfillcolor{currentfill}%
\pgfsetlinewidth{0.602250pt}%
\definecolor{currentstroke}{rgb}{0.000000,0.000000,0.000000}%
\pgfsetstrokecolor{currentstroke}%
\pgfsetdash{}{0pt}%
\pgfsys@defobject{currentmarker}{\pgfqpoint{-0.027778in}{0.000000in}}{\pgfqpoint{0.000000in}{0.000000in}}{%
\pgfpathmoveto{\pgfqpoint{0.000000in}{0.000000in}}%
\pgfpathlineto{\pgfqpoint{-0.027778in}{0.000000in}}%
\pgfusepath{stroke,fill}%
}%
\begin{pgfscope}%
\pgfsys@transformshift{0.800000in}{1.691952in}%
\pgfsys@useobject{currentmarker}{}%
\end{pgfscope}%
\end{pgfscope}%
\begin{pgfscope}%
\pgfsetbuttcap%
\pgfsetroundjoin%
\definecolor{currentfill}{rgb}{0.000000,0.000000,0.000000}%
\pgfsetfillcolor{currentfill}%
\pgfsetlinewidth{0.602250pt}%
\definecolor{currentstroke}{rgb}{0.000000,0.000000,0.000000}%
\pgfsetstrokecolor{currentstroke}%
\pgfsetdash{}{0pt}%
\pgfsys@defobject{currentmarker}{\pgfqpoint{-0.027778in}{0.000000in}}{\pgfqpoint{0.000000in}{0.000000in}}{%
\pgfpathmoveto{\pgfqpoint{0.000000in}{0.000000in}}%
\pgfpathlineto{\pgfqpoint{-0.027778in}{0.000000in}}%
\pgfusepath{stroke,fill}%
}%
\begin{pgfscope}%
\pgfsys@transformshift{0.800000in}{1.775793in}%
\pgfsys@useobject{currentmarker}{}%
\end{pgfscope}%
\end{pgfscope}%
\begin{pgfscope}%
\pgfsetbuttcap%
\pgfsetroundjoin%
\definecolor{currentfill}{rgb}{0.000000,0.000000,0.000000}%
\pgfsetfillcolor{currentfill}%
\pgfsetlinewidth{0.602250pt}%
\definecolor{currentstroke}{rgb}{0.000000,0.000000,0.000000}%
\pgfsetstrokecolor{currentstroke}%
\pgfsetdash{}{0pt}%
\pgfsys@defobject{currentmarker}{\pgfqpoint{-0.027778in}{0.000000in}}{\pgfqpoint{0.000000in}{0.000000in}}{%
\pgfpathmoveto{\pgfqpoint{0.000000in}{0.000000in}}%
\pgfpathlineto{\pgfqpoint{-0.027778in}{0.000000in}}%
\pgfusepath{stroke,fill}%
}%
\begin{pgfscope}%
\pgfsys@transformshift{0.800000in}{1.840825in}%
\pgfsys@useobject{currentmarker}{}%
\end{pgfscope}%
\end{pgfscope}%
\begin{pgfscope}%
\pgfsetbuttcap%
\pgfsetroundjoin%
\definecolor{currentfill}{rgb}{0.000000,0.000000,0.000000}%
\pgfsetfillcolor{currentfill}%
\pgfsetlinewidth{0.602250pt}%
\definecolor{currentstroke}{rgb}{0.000000,0.000000,0.000000}%
\pgfsetstrokecolor{currentstroke}%
\pgfsetdash{}{0pt}%
\pgfsys@defobject{currentmarker}{\pgfqpoint{-0.027778in}{0.000000in}}{\pgfqpoint{0.000000in}{0.000000in}}{%
\pgfpathmoveto{\pgfqpoint{0.000000in}{0.000000in}}%
\pgfpathlineto{\pgfqpoint{-0.027778in}{0.000000in}}%
\pgfusepath{stroke,fill}%
}%
\begin{pgfscope}%
\pgfsys@transformshift{0.800000in}{1.893960in}%
\pgfsys@useobject{currentmarker}{}%
\end{pgfscope}%
\end{pgfscope}%
\begin{pgfscope}%
\pgfsetbuttcap%
\pgfsetroundjoin%
\definecolor{currentfill}{rgb}{0.000000,0.000000,0.000000}%
\pgfsetfillcolor{currentfill}%
\pgfsetlinewidth{0.602250pt}%
\definecolor{currentstroke}{rgb}{0.000000,0.000000,0.000000}%
\pgfsetstrokecolor{currentstroke}%
\pgfsetdash{}{0pt}%
\pgfsys@defobject{currentmarker}{\pgfqpoint{-0.027778in}{0.000000in}}{\pgfqpoint{0.000000in}{0.000000in}}{%
\pgfpathmoveto{\pgfqpoint{0.000000in}{0.000000in}}%
\pgfpathlineto{\pgfqpoint{-0.027778in}{0.000000in}}%
\pgfusepath{stroke,fill}%
}%
\begin{pgfscope}%
\pgfsys@transformshift{0.800000in}{1.938885in}%
\pgfsys@useobject{currentmarker}{}%
\end{pgfscope}%
\end{pgfscope}%
\begin{pgfscope}%
\pgfsetbuttcap%
\pgfsetroundjoin%
\definecolor{currentfill}{rgb}{0.000000,0.000000,0.000000}%
\pgfsetfillcolor{currentfill}%
\pgfsetlinewidth{0.602250pt}%
\definecolor{currentstroke}{rgb}{0.000000,0.000000,0.000000}%
\pgfsetstrokecolor{currentstroke}%
\pgfsetdash{}{0pt}%
\pgfsys@defobject{currentmarker}{\pgfqpoint{-0.027778in}{0.000000in}}{\pgfqpoint{0.000000in}{0.000000in}}{%
\pgfpathmoveto{\pgfqpoint{0.000000in}{0.000000in}}%
\pgfpathlineto{\pgfqpoint{-0.027778in}{0.000000in}}%
\pgfusepath{stroke,fill}%
}%
\begin{pgfscope}%
\pgfsys@transformshift{0.800000in}{1.977801in}%
\pgfsys@useobject{currentmarker}{}%
\end{pgfscope}%
\end{pgfscope}%
\begin{pgfscope}%
\pgfsetbuttcap%
\pgfsetroundjoin%
\definecolor{currentfill}{rgb}{0.000000,0.000000,0.000000}%
\pgfsetfillcolor{currentfill}%
\pgfsetlinewidth{0.602250pt}%
\definecolor{currentstroke}{rgb}{0.000000,0.000000,0.000000}%
\pgfsetstrokecolor{currentstroke}%
\pgfsetdash{}{0pt}%
\pgfsys@defobject{currentmarker}{\pgfqpoint{-0.027778in}{0.000000in}}{\pgfqpoint{0.000000in}{0.000000in}}{%
\pgfpathmoveto{\pgfqpoint{0.000000in}{0.000000in}}%
\pgfpathlineto{\pgfqpoint{-0.027778in}{0.000000in}}%
\pgfusepath{stroke,fill}%
}%
\begin{pgfscope}%
\pgfsys@transformshift{0.800000in}{2.012127in}%
\pgfsys@useobject{currentmarker}{}%
\end{pgfscope}%
\end{pgfscope}%
\begin{pgfscope}%
\pgfsetbuttcap%
\pgfsetroundjoin%
\definecolor{currentfill}{rgb}{0.000000,0.000000,0.000000}%
\pgfsetfillcolor{currentfill}%
\pgfsetlinewidth{0.602250pt}%
\definecolor{currentstroke}{rgb}{0.000000,0.000000,0.000000}%
\pgfsetstrokecolor{currentstroke}%
\pgfsetdash{}{0pt}%
\pgfsys@defobject{currentmarker}{\pgfqpoint{-0.027778in}{0.000000in}}{\pgfqpoint{0.000000in}{0.000000in}}{%
\pgfpathmoveto{\pgfqpoint{0.000000in}{0.000000in}}%
\pgfpathlineto{\pgfqpoint{-0.027778in}{0.000000in}}%
\pgfusepath{stroke,fill}%
}%
\begin{pgfscope}%
\pgfsys@transformshift{0.800000in}{2.244841in}%
\pgfsys@useobject{currentmarker}{}%
\end{pgfscope}%
\end{pgfscope}%
\begin{pgfscope}%
\pgfsetbuttcap%
\pgfsetroundjoin%
\definecolor{currentfill}{rgb}{0.000000,0.000000,0.000000}%
\pgfsetfillcolor{currentfill}%
\pgfsetlinewidth{0.602250pt}%
\definecolor{currentstroke}{rgb}{0.000000,0.000000,0.000000}%
\pgfsetstrokecolor{currentstroke}%
\pgfsetdash{}{0pt}%
\pgfsys@defobject{currentmarker}{\pgfqpoint{-0.027778in}{0.000000in}}{\pgfqpoint{0.000000in}{0.000000in}}{%
\pgfpathmoveto{\pgfqpoint{0.000000in}{0.000000in}}%
\pgfpathlineto{\pgfqpoint{-0.027778in}{0.000000in}}%
\pgfusepath{stroke,fill}%
}%
\begin{pgfscope}%
\pgfsys@transformshift{0.800000in}{2.363008in}%
\pgfsys@useobject{currentmarker}{}%
\end{pgfscope}%
\end{pgfscope}%
\begin{pgfscope}%
\pgfsetbuttcap%
\pgfsetroundjoin%
\definecolor{currentfill}{rgb}{0.000000,0.000000,0.000000}%
\pgfsetfillcolor{currentfill}%
\pgfsetlinewidth{0.602250pt}%
\definecolor{currentstroke}{rgb}{0.000000,0.000000,0.000000}%
\pgfsetstrokecolor{currentstroke}%
\pgfsetdash{}{0pt}%
\pgfsys@defobject{currentmarker}{\pgfqpoint{-0.027778in}{0.000000in}}{\pgfqpoint{0.000000in}{0.000000in}}{%
\pgfpathmoveto{\pgfqpoint{0.000000in}{0.000000in}}%
\pgfpathlineto{\pgfqpoint{-0.027778in}{0.000000in}}%
\pgfusepath{stroke,fill}%
}%
\begin{pgfscope}%
\pgfsys@transformshift{0.800000in}{2.446849in}%
\pgfsys@useobject{currentmarker}{}%
\end{pgfscope}%
\end{pgfscope}%
\begin{pgfscope}%
\pgfsetbuttcap%
\pgfsetroundjoin%
\definecolor{currentfill}{rgb}{0.000000,0.000000,0.000000}%
\pgfsetfillcolor{currentfill}%
\pgfsetlinewidth{0.602250pt}%
\definecolor{currentstroke}{rgb}{0.000000,0.000000,0.000000}%
\pgfsetstrokecolor{currentstroke}%
\pgfsetdash{}{0pt}%
\pgfsys@defobject{currentmarker}{\pgfqpoint{-0.027778in}{0.000000in}}{\pgfqpoint{0.000000in}{0.000000in}}{%
\pgfpathmoveto{\pgfqpoint{0.000000in}{0.000000in}}%
\pgfpathlineto{\pgfqpoint{-0.027778in}{0.000000in}}%
\pgfusepath{stroke,fill}%
}%
\begin{pgfscope}%
\pgfsys@transformshift{0.800000in}{2.511881in}%
\pgfsys@useobject{currentmarker}{}%
\end{pgfscope}%
\end{pgfscope}%
\begin{pgfscope}%
\pgfsetbuttcap%
\pgfsetroundjoin%
\definecolor{currentfill}{rgb}{0.000000,0.000000,0.000000}%
\pgfsetfillcolor{currentfill}%
\pgfsetlinewidth{0.602250pt}%
\definecolor{currentstroke}{rgb}{0.000000,0.000000,0.000000}%
\pgfsetstrokecolor{currentstroke}%
\pgfsetdash{}{0pt}%
\pgfsys@defobject{currentmarker}{\pgfqpoint{-0.027778in}{0.000000in}}{\pgfqpoint{0.000000in}{0.000000in}}{%
\pgfpathmoveto{\pgfqpoint{0.000000in}{0.000000in}}%
\pgfpathlineto{\pgfqpoint{-0.027778in}{0.000000in}}%
\pgfusepath{stroke,fill}%
}%
\begin{pgfscope}%
\pgfsys@transformshift{0.800000in}{2.565016in}%
\pgfsys@useobject{currentmarker}{}%
\end{pgfscope}%
\end{pgfscope}%
\begin{pgfscope}%
\pgfsetbuttcap%
\pgfsetroundjoin%
\definecolor{currentfill}{rgb}{0.000000,0.000000,0.000000}%
\pgfsetfillcolor{currentfill}%
\pgfsetlinewidth{0.602250pt}%
\definecolor{currentstroke}{rgb}{0.000000,0.000000,0.000000}%
\pgfsetstrokecolor{currentstroke}%
\pgfsetdash{}{0pt}%
\pgfsys@defobject{currentmarker}{\pgfqpoint{-0.027778in}{0.000000in}}{\pgfqpoint{0.000000in}{0.000000in}}{%
\pgfpathmoveto{\pgfqpoint{0.000000in}{0.000000in}}%
\pgfpathlineto{\pgfqpoint{-0.027778in}{0.000000in}}%
\pgfusepath{stroke,fill}%
}%
\begin{pgfscope}%
\pgfsys@transformshift{0.800000in}{2.609941in}%
\pgfsys@useobject{currentmarker}{}%
\end{pgfscope}%
\end{pgfscope}%
\begin{pgfscope}%
\pgfsetbuttcap%
\pgfsetroundjoin%
\definecolor{currentfill}{rgb}{0.000000,0.000000,0.000000}%
\pgfsetfillcolor{currentfill}%
\pgfsetlinewidth{0.602250pt}%
\definecolor{currentstroke}{rgb}{0.000000,0.000000,0.000000}%
\pgfsetstrokecolor{currentstroke}%
\pgfsetdash{}{0pt}%
\pgfsys@defobject{currentmarker}{\pgfqpoint{-0.027778in}{0.000000in}}{\pgfqpoint{0.000000in}{0.000000in}}{%
\pgfpathmoveto{\pgfqpoint{0.000000in}{0.000000in}}%
\pgfpathlineto{\pgfqpoint{-0.027778in}{0.000000in}}%
\pgfusepath{stroke,fill}%
}%
\begin{pgfscope}%
\pgfsys@transformshift{0.800000in}{2.648856in}%
\pgfsys@useobject{currentmarker}{}%
\end{pgfscope}%
\end{pgfscope}%
\begin{pgfscope}%
\pgfsetbuttcap%
\pgfsetroundjoin%
\definecolor{currentfill}{rgb}{0.000000,0.000000,0.000000}%
\pgfsetfillcolor{currentfill}%
\pgfsetlinewidth{0.602250pt}%
\definecolor{currentstroke}{rgb}{0.000000,0.000000,0.000000}%
\pgfsetstrokecolor{currentstroke}%
\pgfsetdash{}{0pt}%
\pgfsys@defobject{currentmarker}{\pgfqpoint{-0.027778in}{0.000000in}}{\pgfqpoint{0.000000in}{0.000000in}}{%
\pgfpathmoveto{\pgfqpoint{0.000000in}{0.000000in}}%
\pgfpathlineto{\pgfqpoint{-0.027778in}{0.000000in}}%
\pgfusepath{stroke,fill}%
}%
\begin{pgfscope}%
\pgfsys@transformshift{0.800000in}{2.683183in}%
\pgfsys@useobject{currentmarker}{}%
\end{pgfscope}%
\end{pgfscope}%
\begin{pgfscope}%
\pgfsetbuttcap%
\pgfsetroundjoin%
\definecolor{currentfill}{rgb}{0.000000,0.000000,0.000000}%
\pgfsetfillcolor{currentfill}%
\pgfsetlinewidth{0.602250pt}%
\definecolor{currentstroke}{rgb}{0.000000,0.000000,0.000000}%
\pgfsetstrokecolor{currentstroke}%
\pgfsetdash{}{0pt}%
\pgfsys@defobject{currentmarker}{\pgfqpoint{-0.027778in}{0.000000in}}{\pgfqpoint{0.000000in}{0.000000in}}{%
\pgfpathmoveto{\pgfqpoint{0.000000in}{0.000000in}}%
\pgfpathlineto{\pgfqpoint{-0.027778in}{0.000000in}}%
\pgfusepath{stroke,fill}%
}%
\begin{pgfscope}%
\pgfsys@transformshift{0.800000in}{2.915896in}%
\pgfsys@useobject{currentmarker}{}%
\end{pgfscope}%
\end{pgfscope}%
\begin{pgfscope}%
\pgfsetbuttcap%
\pgfsetroundjoin%
\definecolor{currentfill}{rgb}{0.000000,0.000000,0.000000}%
\pgfsetfillcolor{currentfill}%
\pgfsetlinewidth{0.602250pt}%
\definecolor{currentstroke}{rgb}{0.000000,0.000000,0.000000}%
\pgfsetstrokecolor{currentstroke}%
\pgfsetdash{}{0pt}%
\pgfsys@defobject{currentmarker}{\pgfqpoint{-0.027778in}{0.000000in}}{\pgfqpoint{0.000000in}{0.000000in}}{%
\pgfpathmoveto{\pgfqpoint{0.000000in}{0.000000in}}%
\pgfpathlineto{\pgfqpoint{-0.027778in}{0.000000in}}%
\pgfusepath{stroke,fill}%
}%
\begin{pgfscope}%
\pgfsys@transformshift{0.800000in}{3.034063in}%
\pgfsys@useobject{currentmarker}{}%
\end{pgfscope}%
\end{pgfscope}%
\begin{pgfscope}%
\pgfsetbuttcap%
\pgfsetroundjoin%
\definecolor{currentfill}{rgb}{0.000000,0.000000,0.000000}%
\pgfsetfillcolor{currentfill}%
\pgfsetlinewidth{0.602250pt}%
\definecolor{currentstroke}{rgb}{0.000000,0.000000,0.000000}%
\pgfsetstrokecolor{currentstroke}%
\pgfsetdash{}{0pt}%
\pgfsys@defobject{currentmarker}{\pgfqpoint{-0.027778in}{0.000000in}}{\pgfqpoint{0.000000in}{0.000000in}}{%
\pgfpathmoveto{\pgfqpoint{0.000000in}{0.000000in}}%
\pgfpathlineto{\pgfqpoint{-0.027778in}{0.000000in}}%
\pgfusepath{stroke,fill}%
}%
\begin{pgfscope}%
\pgfsys@transformshift{0.800000in}{3.117904in}%
\pgfsys@useobject{currentmarker}{}%
\end{pgfscope}%
\end{pgfscope}%
\begin{pgfscope}%
\pgfsetbuttcap%
\pgfsetroundjoin%
\definecolor{currentfill}{rgb}{0.000000,0.000000,0.000000}%
\pgfsetfillcolor{currentfill}%
\pgfsetlinewidth{0.602250pt}%
\definecolor{currentstroke}{rgb}{0.000000,0.000000,0.000000}%
\pgfsetstrokecolor{currentstroke}%
\pgfsetdash{}{0pt}%
\pgfsys@defobject{currentmarker}{\pgfqpoint{-0.027778in}{0.000000in}}{\pgfqpoint{0.000000in}{0.000000in}}{%
\pgfpathmoveto{\pgfqpoint{0.000000in}{0.000000in}}%
\pgfpathlineto{\pgfqpoint{-0.027778in}{0.000000in}}%
\pgfusepath{stroke,fill}%
}%
\begin{pgfscope}%
\pgfsys@transformshift{0.800000in}{3.182936in}%
\pgfsys@useobject{currentmarker}{}%
\end{pgfscope}%
\end{pgfscope}%
\begin{pgfscope}%
\pgfsetbuttcap%
\pgfsetroundjoin%
\definecolor{currentfill}{rgb}{0.000000,0.000000,0.000000}%
\pgfsetfillcolor{currentfill}%
\pgfsetlinewidth{0.602250pt}%
\definecolor{currentstroke}{rgb}{0.000000,0.000000,0.000000}%
\pgfsetstrokecolor{currentstroke}%
\pgfsetdash{}{0pt}%
\pgfsys@defobject{currentmarker}{\pgfqpoint{-0.027778in}{0.000000in}}{\pgfqpoint{0.000000in}{0.000000in}}{%
\pgfpathmoveto{\pgfqpoint{0.000000in}{0.000000in}}%
\pgfpathlineto{\pgfqpoint{-0.027778in}{0.000000in}}%
\pgfusepath{stroke,fill}%
}%
\begin{pgfscope}%
\pgfsys@transformshift{0.800000in}{3.236071in}%
\pgfsys@useobject{currentmarker}{}%
\end{pgfscope}%
\end{pgfscope}%
\begin{pgfscope}%
\pgfsetbuttcap%
\pgfsetroundjoin%
\definecolor{currentfill}{rgb}{0.000000,0.000000,0.000000}%
\pgfsetfillcolor{currentfill}%
\pgfsetlinewidth{0.602250pt}%
\definecolor{currentstroke}{rgb}{0.000000,0.000000,0.000000}%
\pgfsetstrokecolor{currentstroke}%
\pgfsetdash{}{0pt}%
\pgfsys@defobject{currentmarker}{\pgfqpoint{-0.027778in}{0.000000in}}{\pgfqpoint{0.000000in}{0.000000in}}{%
\pgfpathmoveto{\pgfqpoint{0.000000in}{0.000000in}}%
\pgfpathlineto{\pgfqpoint{-0.027778in}{0.000000in}}%
\pgfusepath{stroke,fill}%
}%
\begin{pgfscope}%
\pgfsys@transformshift{0.800000in}{3.280996in}%
\pgfsys@useobject{currentmarker}{}%
\end{pgfscope}%
\end{pgfscope}%
\begin{pgfscope}%
\pgfsetbuttcap%
\pgfsetroundjoin%
\definecolor{currentfill}{rgb}{0.000000,0.000000,0.000000}%
\pgfsetfillcolor{currentfill}%
\pgfsetlinewidth{0.602250pt}%
\definecolor{currentstroke}{rgb}{0.000000,0.000000,0.000000}%
\pgfsetstrokecolor{currentstroke}%
\pgfsetdash{}{0pt}%
\pgfsys@defobject{currentmarker}{\pgfqpoint{-0.027778in}{0.000000in}}{\pgfqpoint{0.000000in}{0.000000in}}{%
\pgfpathmoveto{\pgfqpoint{0.000000in}{0.000000in}}%
\pgfpathlineto{\pgfqpoint{-0.027778in}{0.000000in}}%
\pgfusepath{stroke,fill}%
}%
\begin{pgfscope}%
\pgfsys@transformshift{0.800000in}{3.319912in}%
\pgfsys@useobject{currentmarker}{}%
\end{pgfscope}%
\end{pgfscope}%
\begin{pgfscope}%
\pgfsetbuttcap%
\pgfsetroundjoin%
\definecolor{currentfill}{rgb}{0.000000,0.000000,0.000000}%
\pgfsetfillcolor{currentfill}%
\pgfsetlinewidth{0.602250pt}%
\definecolor{currentstroke}{rgb}{0.000000,0.000000,0.000000}%
\pgfsetstrokecolor{currentstroke}%
\pgfsetdash{}{0pt}%
\pgfsys@defobject{currentmarker}{\pgfqpoint{-0.027778in}{0.000000in}}{\pgfqpoint{0.000000in}{0.000000in}}{%
\pgfpathmoveto{\pgfqpoint{0.000000in}{0.000000in}}%
\pgfpathlineto{\pgfqpoint{-0.027778in}{0.000000in}}%
\pgfusepath{stroke,fill}%
}%
\begin{pgfscope}%
\pgfsys@transformshift{0.800000in}{3.354238in}%
\pgfsys@useobject{currentmarker}{}%
\end{pgfscope}%
\end{pgfscope}%
\begin{pgfscope}%
\pgfsetbuttcap%
\pgfsetroundjoin%
\definecolor{currentfill}{rgb}{0.000000,0.000000,0.000000}%
\pgfsetfillcolor{currentfill}%
\pgfsetlinewidth{0.602250pt}%
\definecolor{currentstroke}{rgb}{0.000000,0.000000,0.000000}%
\pgfsetstrokecolor{currentstroke}%
\pgfsetdash{}{0pt}%
\pgfsys@defobject{currentmarker}{\pgfqpoint{-0.027778in}{0.000000in}}{\pgfqpoint{0.000000in}{0.000000in}}{%
\pgfpathmoveto{\pgfqpoint{0.000000in}{0.000000in}}%
\pgfpathlineto{\pgfqpoint{-0.027778in}{0.000000in}}%
\pgfusepath{stroke,fill}%
}%
\begin{pgfscope}%
\pgfsys@transformshift{0.800000in}{3.586952in}%
\pgfsys@useobject{currentmarker}{}%
\end{pgfscope}%
\end{pgfscope}%
\begin{pgfscope}%
\pgfsetbuttcap%
\pgfsetroundjoin%
\definecolor{currentfill}{rgb}{0.000000,0.000000,0.000000}%
\pgfsetfillcolor{currentfill}%
\pgfsetlinewidth{0.602250pt}%
\definecolor{currentstroke}{rgb}{0.000000,0.000000,0.000000}%
\pgfsetstrokecolor{currentstroke}%
\pgfsetdash{}{0pt}%
\pgfsys@defobject{currentmarker}{\pgfqpoint{-0.027778in}{0.000000in}}{\pgfqpoint{0.000000in}{0.000000in}}{%
\pgfpathmoveto{\pgfqpoint{0.000000in}{0.000000in}}%
\pgfpathlineto{\pgfqpoint{-0.027778in}{0.000000in}}%
\pgfusepath{stroke,fill}%
}%
\begin{pgfscope}%
\pgfsys@transformshift{0.800000in}{3.705119in}%
\pgfsys@useobject{currentmarker}{}%
\end{pgfscope}%
\end{pgfscope}%
\begin{pgfscope}%
\pgfsetbuttcap%
\pgfsetroundjoin%
\definecolor{currentfill}{rgb}{0.000000,0.000000,0.000000}%
\pgfsetfillcolor{currentfill}%
\pgfsetlinewidth{0.602250pt}%
\definecolor{currentstroke}{rgb}{0.000000,0.000000,0.000000}%
\pgfsetstrokecolor{currentstroke}%
\pgfsetdash{}{0pt}%
\pgfsys@defobject{currentmarker}{\pgfqpoint{-0.027778in}{0.000000in}}{\pgfqpoint{0.000000in}{0.000000in}}{%
\pgfpathmoveto{\pgfqpoint{0.000000in}{0.000000in}}%
\pgfpathlineto{\pgfqpoint{-0.027778in}{0.000000in}}%
\pgfusepath{stroke,fill}%
}%
\begin{pgfscope}%
\pgfsys@transformshift{0.800000in}{3.788960in}%
\pgfsys@useobject{currentmarker}{}%
\end{pgfscope}%
\end{pgfscope}%
\begin{pgfscope}%
\pgfsetbuttcap%
\pgfsetroundjoin%
\definecolor{currentfill}{rgb}{0.000000,0.000000,0.000000}%
\pgfsetfillcolor{currentfill}%
\pgfsetlinewidth{0.602250pt}%
\definecolor{currentstroke}{rgb}{0.000000,0.000000,0.000000}%
\pgfsetstrokecolor{currentstroke}%
\pgfsetdash{}{0pt}%
\pgfsys@defobject{currentmarker}{\pgfqpoint{-0.027778in}{0.000000in}}{\pgfqpoint{0.000000in}{0.000000in}}{%
\pgfpathmoveto{\pgfqpoint{0.000000in}{0.000000in}}%
\pgfpathlineto{\pgfqpoint{-0.027778in}{0.000000in}}%
\pgfusepath{stroke,fill}%
}%
\begin{pgfscope}%
\pgfsys@transformshift{0.800000in}{3.853992in}%
\pgfsys@useobject{currentmarker}{}%
\end{pgfscope}%
\end{pgfscope}%
\begin{pgfscope}%
\pgfsetbuttcap%
\pgfsetroundjoin%
\definecolor{currentfill}{rgb}{0.000000,0.000000,0.000000}%
\pgfsetfillcolor{currentfill}%
\pgfsetlinewidth{0.602250pt}%
\definecolor{currentstroke}{rgb}{0.000000,0.000000,0.000000}%
\pgfsetstrokecolor{currentstroke}%
\pgfsetdash{}{0pt}%
\pgfsys@defobject{currentmarker}{\pgfqpoint{-0.027778in}{0.000000in}}{\pgfqpoint{0.000000in}{0.000000in}}{%
\pgfpathmoveto{\pgfqpoint{0.000000in}{0.000000in}}%
\pgfpathlineto{\pgfqpoint{-0.027778in}{0.000000in}}%
\pgfusepath{stroke,fill}%
}%
\begin{pgfscope}%
\pgfsys@transformshift{0.800000in}{3.907127in}%
\pgfsys@useobject{currentmarker}{}%
\end{pgfscope}%
\end{pgfscope}%
\begin{pgfscope}%
\pgfsetbuttcap%
\pgfsetroundjoin%
\definecolor{currentfill}{rgb}{0.000000,0.000000,0.000000}%
\pgfsetfillcolor{currentfill}%
\pgfsetlinewidth{0.602250pt}%
\definecolor{currentstroke}{rgb}{0.000000,0.000000,0.000000}%
\pgfsetstrokecolor{currentstroke}%
\pgfsetdash{}{0pt}%
\pgfsys@defobject{currentmarker}{\pgfqpoint{-0.027778in}{0.000000in}}{\pgfqpoint{0.000000in}{0.000000in}}{%
\pgfpathmoveto{\pgfqpoint{0.000000in}{0.000000in}}%
\pgfpathlineto{\pgfqpoint{-0.027778in}{0.000000in}}%
\pgfusepath{stroke,fill}%
}%
\begin{pgfscope}%
\pgfsys@transformshift{0.800000in}{3.952052in}%
\pgfsys@useobject{currentmarker}{}%
\end{pgfscope}%
\end{pgfscope}%
\begin{pgfscope}%
\pgfsetbuttcap%
\pgfsetroundjoin%
\definecolor{currentfill}{rgb}{0.000000,0.000000,0.000000}%
\pgfsetfillcolor{currentfill}%
\pgfsetlinewidth{0.602250pt}%
\definecolor{currentstroke}{rgb}{0.000000,0.000000,0.000000}%
\pgfsetstrokecolor{currentstroke}%
\pgfsetdash{}{0pt}%
\pgfsys@defobject{currentmarker}{\pgfqpoint{-0.027778in}{0.000000in}}{\pgfqpoint{0.000000in}{0.000000in}}{%
\pgfpathmoveto{\pgfqpoint{0.000000in}{0.000000in}}%
\pgfpathlineto{\pgfqpoint{-0.027778in}{0.000000in}}%
\pgfusepath{stroke,fill}%
}%
\begin{pgfscope}%
\pgfsys@transformshift{0.800000in}{3.990968in}%
\pgfsys@useobject{currentmarker}{}%
\end{pgfscope}%
\end{pgfscope}%
\begin{pgfscope}%
\pgfsetbuttcap%
\pgfsetroundjoin%
\definecolor{currentfill}{rgb}{0.000000,0.000000,0.000000}%
\pgfsetfillcolor{currentfill}%
\pgfsetlinewidth{0.602250pt}%
\definecolor{currentstroke}{rgb}{0.000000,0.000000,0.000000}%
\pgfsetstrokecolor{currentstroke}%
\pgfsetdash{}{0pt}%
\pgfsys@defobject{currentmarker}{\pgfqpoint{-0.027778in}{0.000000in}}{\pgfqpoint{0.000000in}{0.000000in}}{%
\pgfpathmoveto{\pgfqpoint{0.000000in}{0.000000in}}%
\pgfpathlineto{\pgfqpoint{-0.027778in}{0.000000in}}%
\pgfusepath{stroke,fill}%
}%
\begin{pgfscope}%
\pgfsys@transformshift{0.800000in}{4.025294in}%
\pgfsys@useobject{currentmarker}{}%
\end{pgfscope}%
\end{pgfscope}%
\begin{pgfscope}%
\definecolor{textcolor}{rgb}{0.000000,0.000000,0.000000}%
\pgfsetstrokecolor{textcolor}%
\pgfsetfillcolor{textcolor}%
\pgftext[x=0.446026in,y=2.376000in,,bottom,rotate=90.000000]{\color{textcolor}\sffamily\fontsize{10.000000}{12.000000}\selectfont \(\displaystyle G_{10^{-5}}\)}%
\end{pgfscope}%
\begin{pgfscope}%
\pgfpathrectangle{\pgfqpoint{0.800000in}{0.528000in}}{\pgfqpoint{4.960000in}{3.696000in}}%
\pgfusepath{clip}%
\pgfsetbuttcap%
\pgfsetroundjoin%
\pgfsetlinewidth{1.505625pt}%
\definecolor{currentstroke}{rgb}{0.000000,0.000000,0.000000}%
\pgfsetstrokecolor{currentstroke}%
\pgfsetdash{{5.550000pt}{2.400000pt}}{0.000000pt}%
\pgfpathmoveto{\pgfqpoint{1.025455in}{0.696000in}}%
\pgfpathlineto{\pgfqpoint{1.039434in}{0.721839in}}%
\pgfpathlineto{\pgfqpoint{1.055218in}{0.746349in}}%
\pgfpathlineto{\pgfqpoint{1.073256in}{0.770271in}}%
\pgfpathlineto{\pgfqpoint{1.093549in}{0.793492in}}%
\pgfpathlineto{\pgfqpoint{1.116096in}{0.815921in}}%
\pgfpathlineto{\pgfqpoint{1.140899in}{0.837469in}}%
\pgfpathlineto{\pgfqpoint{1.167956in}{0.858045in}}%
\pgfpathlineto{\pgfqpoint{1.197268in}{0.877550in}}%
\pgfpathlineto{\pgfqpoint{1.228835in}{0.895884in}}%
\pgfpathlineto{\pgfqpoint{1.262656in}{0.912941in}}%
\pgfpathlineto{\pgfqpoint{1.298733in}{0.928619in}}%
\pgfpathlineto{\pgfqpoint{1.337064in}{0.942823in}}%
\pgfpathlineto{\pgfqpoint{1.377650in}{0.955467in}}%
\pgfpathlineto{\pgfqpoint{1.420490in}{0.966482in}}%
\pgfpathlineto{\pgfqpoint{1.466488in}{0.975986in}}%
\pgfpathlineto{\pgfqpoint{1.515642in}{0.983833in}}%
\pgfpathlineto{\pgfqpoint{1.568403in}{0.989970in}}%
\pgfpathlineto{\pgfqpoint{1.625675in}{0.994350in}}%
\pgfpathlineto{\pgfqpoint{1.688357in}{0.996862in}}%
\pgfpathlineto{\pgfqpoint{1.757804in}{0.997363in}}%
\pgfpathlineto{\pgfqpoint{1.835819in}{0.995649in}}%
\pgfpathlineto{\pgfqpoint{1.925559in}{0.991399in}}%
\pgfpathlineto{\pgfqpoint{2.032886in}{0.984011in}}%
\pgfpathlineto{\pgfqpoint{2.168623in}{0.972329in}}%
\pgfpathlineto{\pgfqpoint{2.366592in}{0.952842in}}%
\pgfpathlineto{\pgfqpoint{3.026789in}{0.886470in}}%
\pgfpathlineto{\pgfqpoint{3.256776in}{0.865673in}}%
\pgfpathlineto{\pgfqpoint{3.258129in}{2.648856in}}%
\pgfpathlineto{\pgfqpoint{5.534545in}{2.648856in}}%
\pgfpathlineto{\pgfqpoint{5.534545in}{2.648856in}}%
\pgfusepath{stroke}%
\end{pgfscope}%
\begin{pgfscope}%
\pgfpathrectangle{\pgfqpoint{0.800000in}{0.528000in}}{\pgfqpoint{4.960000in}{3.696000in}}%
\pgfusepath{clip}%
\pgfsetbuttcap%
\pgfsetroundjoin%
\pgfsetlinewidth{1.505625pt}%
\definecolor{currentstroke}{rgb}{0.000000,0.000000,0.000000}%
\pgfsetstrokecolor{currentstroke}%
\pgfsetdash{{9.600000pt}{2.400000pt}{1.500000pt}{2.400000pt}}{0.000000pt}%
\pgfpathmoveto{\pgfqpoint{1.025455in}{0.767471in}}%
\pgfpathlineto{\pgfqpoint{1.038532in}{0.797294in}}%
\pgfpathlineto{\pgfqpoint{1.053414in}{0.825680in}}%
\pgfpathlineto{\pgfqpoint{1.069648in}{0.851821in}}%
\pgfpathlineto{\pgfqpoint{1.086784in}{0.875288in}}%
\pgfpathlineto{\pgfqpoint{1.105273in}{0.896857in}}%
\pgfpathlineto{\pgfqpoint{1.124664in}{0.916067in}}%
\pgfpathlineto{\pgfqpoint{1.144957in}{0.933044in}}%
\pgfpathlineto{\pgfqpoint{1.166152in}{0.947871in}}%
\pgfpathlineto{\pgfqpoint{1.188249in}{0.960613in}}%
\pgfpathlineto{\pgfqpoint{1.211248in}{0.971328in}}%
\pgfpathlineto{\pgfqpoint{1.235599in}{0.980223in}}%
\pgfpathlineto{\pgfqpoint{1.261304in}{0.987254in}}%
\pgfpathlineto{\pgfqpoint{1.288812in}{0.992484in}}%
\pgfpathlineto{\pgfqpoint{1.319026in}{0.995921in}}%
\pgfpathlineto{\pgfqpoint{1.351945in}{0.997388in}}%
\pgfpathlineto{\pgfqpoint{1.388924in}{0.996747in}}%
\pgfpathlineto{\pgfqpoint{1.430862in}{0.993733in}}%
\pgfpathlineto{\pgfqpoint{1.480016in}{0.987903in}}%
\pgfpathlineto{\pgfqpoint{1.540444in}{0.978402in}}%
\pgfpathlineto{\pgfqpoint{1.621616in}{0.963238in}}%
\pgfpathlineto{\pgfqpoint{1.774940in}{0.931834in}}%
\pgfpathlineto{\pgfqpoint{1.960734in}{0.894560in}}%
\pgfpathlineto{\pgfqpoint{2.102333in}{0.868476in}}%
\pgfpathlineto{\pgfqpoint{2.118117in}{0.865712in}}%
\pgfpathlineto{\pgfqpoint{2.119469in}{3.254880in}}%
\pgfpathlineto{\pgfqpoint{5.534545in}{3.254880in}}%
\pgfpathlineto{\pgfqpoint{5.534545in}{3.254880in}}%
\pgfusepath{stroke}%
\end{pgfscope}%
\begin{pgfscope}%
\pgfpathrectangle{\pgfqpoint{0.800000in}{0.528000in}}{\pgfqpoint{4.960000in}{3.696000in}}%
\pgfusepath{clip}%
\pgfsetbuttcap%
\pgfsetroundjoin%
\pgfsetlinewidth{1.505625pt}%
\definecolor{currentstroke}{rgb}{0.000000,0.000000,0.000000}%
\pgfsetstrokecolor{currentstroke}%
\pgfsetdash{{1.500000pt}{2.475000pt}}{0.000000pt}%
\pgfpathmoveto{\pgfqpoint{1.025455in}{0.883856in}}%
\pgfpathlineto{\pgfqpoint{1.036277in}{0.912517in}}%
\pgfpathlineto{\pgfqpoint{1.047551in}{0.936099in}}%
\pgfpathlineto{\pgfqpoint{1.058825in}{0.954564in}}%
\pgfpathlineto{\pgfqpoint{1.070099in}{0.968824in}}%
\pgfpathlineto{\pgfqpoint{1.081373in}{0.979569in}}%
\pgfpathlineto{\pgfqpoint{1.092647in}{0.987358in}}%
\pgfpathlineto{\pgfqpoint{1.104372in}{0.992819in}}%
\pgfpathlineto{\pgfqpoint{1.116998in}{0.996190in}}%
\pgfpathlineto{\pgfqpoint{1.130527in}{0.997429in}}%
\pgfpathlineto{\pgfqpoint{1.145408in}{0.996517in}}%
\pgfpathlineto{\pgfqpoint{1.162545in}{0.993176in}}%
\pgfpathlineto{\pgfqpoint{1.182838in}{0.986884in}}%
\pgfpathlineto{\pgfqpoint{1.208542in}{0.976466in}}%
\pgfpathlineto{\pgfqpoint{1.244618in}{0.959242in}}%
\pgfpathlineto{\pgfqpoint{1.420039in}{0.872394in}}%
\pgfpathlineto{\pgfqpoint{1.435372in}{0.865634in}}%
\pgfpathlineto{\pgfqpoint{1.436725in}{4.056000in}}%
\pgfpathlineto{\pgfqpoint{5.534545in}{4.056000in}}%
\pgfpathlineto{\pgfqpoint{5.534545in}{4.056000in}}%
\pgfusepath{stroke}%
\end{pgfscope}%
\begin{pgfscope}%
\pgfsetrectcap%
\pgfsetmiterjoin%
\pgfsetlinewidth{0.803000pt}%
\definecolor{currentstroke}{rgb}{0.000000,0.000000,0.000000}%
\pgfsetstrokecolor{currentstroke}%
\pgfsetdash{}{0pt}%
\pgfpathmoveto{\pgfqpoint{0.800000in}{0.528000in}}%
\pgfpathlineto{\pgfqpoint{0.800000in}{4.224000in}}%
\pgfusepath{stroke}%
\end{pgfscope}%
\begin{pgfscope}%
\pgfsetrectcap%
\pgfsetmiterjoin%
\pgfsetlinewidth{0.803000pt}%
\definecolor{currentstroke}{rgb}{0.000000,0.000000,0.000000}%
\pgfsetstrokecolor{currentstroke}%
\pgfsetdash{}{0pt}%
\pgfpathmoveto{\pgfqpoint{5.760000in}{0.528000in}}%
\pgfpathlineto{\pgfqpoint{5.760000in}{4.224000in}}%
\pgfusepath{stroke}%
\end{pgfscope}%
\begin{pgfscope}%
\pgfsetrectcap%
\pgfsetmiterjoin%
\pgfsetlinewidth{0.803000pt}%
\definecolor{currentstroke}{rgb}{0.000000,0.000000,0.000000}%
\pgfsetstrokecolor{currentstroke}%
\pgfsetdash{}{0pt}%
\pgfpathmoveto{\pgfqpoint{0.800000in}{0.528000in}}%
\pgfpathlineto{\pgfqpoint{5.760000in}{0.528000in}}%
\pgfusepath{stroke}%
\end{pgfscope}%
\begin{pgfscope}%
\pgfsetrectcap%
\pgfsetmiterjoin%
\pgfsetlinewidth{0.803000pt}%
\definecolor{currentstroke}{rgb}{0.000000,0.000000,0.000000}%
\pgfsetstrokecolor{currentstroke}%
\pgfsetdash{}{0pt}%
\pgfpathmoveto{\pgfqpoint{0.800000in}{4.224000in}}%
\pgfpathlineto{\pgfqpoint{5.760000in}{4.224000in}}%
\pgfusepath{stroke}%
\end{pgfscope}%
\begin{pgfscope}%
\pgfsetbuttcap%
\pgfsetmiterjoin%
\definecolor{currentfill}{rgb}{1.000000,1.000000,1.000000}%
\pgfsetfillcolor{currentfill}%
\pgfsetfillopacity{0.800000}%
\pgfsetlinewidth{1.003750pt}%
\definecolor{currentstroke}{rgb}{0.800000,0.800000,0.800000}%
\pgfsetstrokecolor{currentstroke}%
\pgfsetstrokeopacity{0.800000}%
\pgfsetdash{}{0pt}%
\pgfpathmoveto{\pgfqpoint{4.748137in}{0.597444in}}%
\pgfpathlineto{\pgfqpoint{5.662778in}{0.597444in}}%
\pgfpathquadraticcurveto{\pgfqpoint{5.690556in}{0.597444in}}{\pgfqpoint{5.690556in}{0.625222in}}%
\pgfpathlineto{\pgfqpoint{5.690556in}{1.222905in}}%
\pgfpathquadraticcurveto{\pgfqpoint{5.690556in}{1.250683in}}{\pgfqpoint{5.662778in}{1.250683in}}%
\pgfpathlineto{\pgfqpoint{4.748137in}{1.250683in}}%
\pgfpathquadraticcurveto{\pgfqpoint{4.720360in}{1.250683in}}{\pgfqpoint{4.720360in}{1.222905in}}%
\pgfpathlineto{\pgfqpoint{4.720360in}{0.625222in}}%
\pgfpathquadraticcurveto{\pgfqpoint{4.720360in}{0.597444in}}{\pgfqpoint{4.748137in}{0.597444in}}%
\pgfpathclose%
\pgfusepath{stroke,fill}%
\end{pgfscope}%
\begin{pgfscope}%
\pgfsetbuttcap%
\pgfsetroundjoin%
\pgfsetlinewidth{1.505625pt}%
\definecolor{currentstroke}{rgb}{0.000000,0.000000,0.000000}%
\pgfsetstrokecolor{currentstroke}%
\pgfsetdash{{5.550000pt}{2.400000pt}}{0.000000pt}%
\pgfpathmoveto{\pgfqpoint{4.775915in}{1.138215in}}%
\pgfpathlineto{\pgfqpoint{5.053693in}{1.138215in}}%
\pgfusepath{stroke}%
\end{pgfscope}%
\begin{pgfscope}%
\definecolor{textcolor}{rgb}{0.000000,0.000000,0.000000}%
\pgfsetstrokecolor{textcolor}%
\pgfsetfillcolor{textcolor}%
\pgftext[x=5.164804in,y=1.089604in,left,base]{\color{textcolor}\sffamily\fontsize{10.000000}{12.000000}\selectfont \(\displaystyle k = 20\)}%
\end{pgfscope}%
\begin{pgfscope}%
\pgfsetbuttcap%
\pgfsetroundjoin%
\pgfsetlinewidth{1.505625pt}%
\definecolor{currentstroke}{rgb}{0.000000,0.000000,0.000000}%
\pgfsetstrokecolor{currentstroke}%
\pgfsetdash{{9.600000pt}{2.400000pt}{1.500000pt}{2.400000pt}}{0.000000pt}%
\pgfpathmoveto{\pgfqpoint{4.775915in}{0.934358in}}%
\pgfpathlineto{\pgfqpoint{5.053693in}{0.934358in}}%
\pgfusepath{stroke}%
\end{pgfscope}%
\begin{pgfscope}%
\definecolor{textcolor}{rgb}{0.000000,0.000000,0.000000}%
\pgfsetstrokecolor{textcolor}%
\pgfsetfillcolor{textcolor}%
\pgftext[x=5.164804in,y=0.885747in,left,base]{\color{textcolor}\sffamily\fontsize{10.000000}{12.000000}\selectfont \(\displaystyle k = 40\)}%
\end{pgfscope}%
\begin{pgfscope}%
\pgfsetbuttcap%
\pgfsetroundjoin%
\pgfsetlinewidth{1.505625pt}%
\definecolor{currentstroke}{rgb}{0.000000,0.000000,0.000000}%
\pgfsetstrokecolor{currentstroke}%
\pgfsetdash{{1.500000pt}{2.475000pt}}{0.000000pt}%
\pgfpathmoveto{\pgfqpoint{4.775915in}{0.730501in}}%
\pgfpathlineto{\pgfqpoint{5.053693in}{0.730501in}}%
\pgfusepath{stroke}%
\end{pgfscope}%
\begin{pgfscope}%
\definecolor{textcolor}{rgb}{0.000000,0.000000,0.000000}%
\pgfsetstrokecolor{textcolor}%
\pgfsetfillcolor{textcolor}%
\pgftext[x=5.164804in,y=0.681890in,left,base]{\color{textcolor}\sffamily\fontsize{10.000000}{12.000000}\selectfont \(\displaystyle k = 100\)}%
\end{pgfscope}%
\end{pgfpicture}%
\makeatother%
\endgroup%

  \caption{Plot of $\Gfnname$ for $\NLiDRRR{\no-\nt} \in \mleft(0.01,1.0\mright)$, for $k=20,40,100$ $\Ct = 0.1,$ $N=\ceil{k^3},$ and $\eps = 10^{-5}$.\label{fig:G}}
    \end{figure}


\Cref{lem:probgmres1} gives us the relationship between the (bound on the) number of GMRES iterations required for convergence and $\NLiDRRR{\nso-\nst(\omega)}.$ We can use this relationshi infer probabalistic properties of the number of GMRES iterations required for convergence from the probability distribution of $\NLiDRRR{\nso-\nst(\omega)}.$ (For the probabalistic notation, we refer the reader to \cref{chap:stochastic}.)

\bth[Probabilistic GMRES convergence]\label{thm:probgmres}
Let $\nso \in \LiDRRR$ be fixed, and let $\nst:\Omega\rightarrow\LiDRRR$ be a random field. Let $\eps$ and $\dofs$ be as in \cref{lem:probgmres1}, and let $\Aso = \Ast = I.$ Fix $ R \in \NN.$ Then
\beq\label{eq:GMRESprob}
\PP\mleft(\Gfn{\NLiDRRR{\nso-\nst}} \leq R\mright)\leq\PP\mleft(\GMRES{\eps}{\nso}{\nst} \leq R\mright) .
\eeq
\enth

\bpf[Proof of \cref{thm:probgmres}]
By \cref{lem:probgmres1} we have the implication: if $\Gfn{\nso-\nst(\omega)} \leq R$, then $\GMRES{\eps}{\nso}{\nst(\omega)} \leq R.$ Therefore we have the set inclusion
\beqs
\set{\omega \in \Omega \st \Gfn{\nso-\nst(\omega)} \leq R} \subseteq \set{\omega \in \Omega \st \GMRES{\eps}{\nso}{\nst(\omega)} \leq R}.
\eeqs
The result immediately follows.
\epf

\ble[$\GMRES{\eps}{\no}{\nt}$ is a random variable]\label{lem:randomvariable}
Under the assumptions of \cref{lem:probgmres1}, $\GMRES{\eps}{\nso}{\nst}$ is a random variable, i.e., $\GMRES{\eps}{\nso}{\nst}:\Omega\rightarrow \RR$ is measurable.
\ele

\bpf[Sketch Proof of \cref{lem:randomvariable}]
All of the operations used in constructing the vectors $\bxm$ in the GMRES algorithm are measurable functions of $\bxmmo$ and $\AmatoI\Amatt$ (see, e.g., \cite[Algorithms 11.4.2 and 5.1.3]{GoVa:13}), therefore $\mleft(\brm\mright)_{m=1}^N$ is a sequence of random variables, i.e., a stochastic process (see, e.g., \cite[Definition 2.1.4]{Ok:13}). The stopping criterion $\Nt{\brm}/\Nt{\bff} < \eps$ is an exit time for the stochastic process $\bxm$ from the set $\CCN \setminus \ball{\eps\Nt{\bff}}{\bxs},$ where $\bxs$ is the true solution. Therefore, because we assume $\OFP$ is a complete probability space, it follows from, e.g.,  \cite[Example 7.2.2]{Ok:13} that $\GMRES{\eps}{\no}{\nt}$ is a stopping time (see \cite[Definition 7.2.1]{Ok:13}). Because $\GMRES{\eps}{\no}{\nt}$ is a stopping time, it is measurable with respect to the associated filtration (see, e.g., \cite[Definition 3.2.2]{Ok:13}), and so is measurable with respect to $\cF$. I.e., $\GMRES{\eps}{\no}{\nt}$ is a random variable.
\epf

\bre[The expression \cref{eq:GMRESprob} is computable]\label{rem:computable}
Because the function $\Gfnname$ is not invertible (as is clear from \cref{fig:G}), one cannot write the left-hand side of \cref{eq:GMRESprob} as
\beqs
\PP\mleft(\NLiDRRR{\nso-\nst} \leq \GfnnameI\mleft(\mleft[0,R\mright]\mright)\mright).
\eeqs
However, one can still compute the set
\beq\label{eq:inverseset}
\GfnnameI\mleft(\mleft[0,R\mright]\mright) = \set{\alpha \st \Gfn{\alpha} \in \mleft[0,R\mright]}
\eeq
(where $\GfnnameI$ in \cref{eq:inverseset} denotes the pullback), and therefore one can compute the probabilities in \cref{eq:GMRESprob}. The main effort in computing $\GfnI{\mleft[0,R\mright]}$ is finding if there are any values of $\alpha < 1$ such that $\Gfn{\alpha} = R$, as the existence, or not, of such values determines the range of $\alpha$ over which we must integrate. However, these values can be computed numerically using standard root-finding algorithms.
\ere

\bre[\Cref{thm:probgmres} is pessimistic]\label{rem:pessimistic}
Observe that we expect the bound in \cref{thm:probgmres} to be pessimistic, i.e., we expect that \cref{eq:GMRESprob} is not sharp in its dependence on $\alpha$. In particular, we expect \cref{eq:GMRESprob} is not sharp for large values of $R.$ We now show that in most cases, for large values of $R$ the left-hand side of \cref{eq:GMRESprob} is independent of $R$.

One can show via elementary calculus that for $\alpha < 1$ $\Gfnname$ achieves its maximum when $\alpha = 1/3$ (assuming that for $\alpha < 1$ the expression involving $\alpha$ in \cref{eq:gdef} is always  at most $N$). Also observe that $\Gfnname$ over the range $\alpha  \in (0,1)$ only depends on $k$ through the dependence of $\alpha$ on $k.$ Therefore, the maximum of $\Gfnname$ over $\alpha \in (0,1)$ is independent of $k.$ Let $\Gfnmaxlo$ denote the value of this maximum. Then, for any $R \in \mleft(\Gfnmaxlo,N\mright)$, the estimate $\PP\mleft(\Gfn{\nso-\nst} \leq R\mright)$ is equal to $\PP\mleft(\alpha<1\mright) = \PP\mleft(\NLiDRRR{\nso-\nst} < 1/\mleft(\Ct k\mright)\mright),$ i.e., the lower-bound in \cref{eq:GMRESprob} is \emph{independent} of $R$, for $R \in \mleft(\Gfnmaxlo,N\mright)$  This is almost certainly not sharp - we would expect $\PP\mleft(\GMRES{\eps}{\nso}{\nst} \leq R\mright)$ to increase with $R$. However, because the only rigorous result we have available if $\alpha \geq 1$ is \cref{cor:gmresguaranteed}, we cannot prove a better bound.
\ere

\subsection{Qualitative predictions from \cref{thm:probgmres}}\label{sec:qualgmres}

Notwithstanding the comments in \cref{rem:pessimistic}, we wil show in this \lcnamecref{sec:qualgmres} that \cref{thm:probgmres} can give \emph{qualitatively} correct predictions. To illustrate this correctness, we take $\Ao=\At=I,$ $\no =1$, and $\nt = \no + \eta,$ where $\eta$ is an $\Exp{\sigma}$ random variable, where $\sigma$ is the shape parameter. By construction, $\NLiDRRR{\no-\nt} = \eta$. Recall that the standard deviation of $\eta$ is $\sigma$, and therefore one might expect (based on \cref{cor:1,cor:1a}) that if $\sigma \sim 1/k,$ then the number of GMRES iterations may be controllable (in some sense) as $k \rightarrow \infty.$

In our numerical experiments we use the computational setup described in \cref{app:compsetup}, with $f=1$ and $\gI=0.$ we consider three cases:
\ben
\item\label[itemcase]{it:sigma1} $\displaystyle \sigma  = 1,$
\item\label[itemcase]{it:sigma2} $\displaystyle \sigma  = \frac{1}k,$ and
  \item\label[itemcase]{it:sigma3} $\displaystyle \sigma  = \frac{1}{k^2}.$
    \een

    For each of these cases we calculate
    \beq\label{eq:gmresprob}
    \PP\mleft(\GMRES{\eps}{\no}{\nt} \leq 12\mright),
    \eeq
     and compare these experimental results to the lower bounds given by \cref{thm:probgmres}.

    Qualitatively, from \cref{fig:prob-theory-plot-0.0,fig:prob-theory-plot-1.0,fig:prob-theory-plot-2.0} we expect that in \cref{it:sigma1} the probability \cref{eq:gmresprob} \emph{decreases} as $k$ increases, in \cref{it:sigma2} the probability \cref{eq:gmresprob} \emph{is constant} as $k$ increases, and in \cref{it:sigma3} the probability \cref{eq:gmresprob} \emph{increases} as $k$ increases. This qualitative behaviour is what we see experimentally in \cref{fig:prob-plot-0.0,fig:prob-plot-1.0,fig:prob-plot-2.0}, although the values for the probabilities are higher, as is expected, because \cref{thm:probgmres} is a lower bound.

\begin{figure}[p]
  \centering
  \begin{subfigure}{\textwidth}
    \centering
%% Creator: Matplotlib, PGF backend
%%
%% To include the figure in your LaTeX document, write
%%   \input{<filename>.pgf}
%%
%% Make sure the required packages are loaded in your preamble
%%   \usepackage{pgf}
%%
%% Figures using additional raster images can only be included by \input if
%% they are in the same directory as the main LaTeX file. For loading figures
%% from other directories you can use the `import` package
%%   \usepackage{import}
%% and then include the figures with
%%   \import{<path to file>}{<filename>.pgf}
%%
%% Matplotlib used the following preamble
%%   \usepackage{fontspec}
%%   \setmainfont{DejaVuSerif.ttf}[Path=/home/owen/progs/firedrake-complex/firedrake/lib/python3.5/site-packages/matplotlib/mpl-data/fonts/ttf/]
%%   \setsansfont{DejaVuSans.ttf}[Path=/home/owen/progs/firedrake-complex/firedrake/lib/python3.5/site-packages/matplotlib/mpl-data/fonts/ttf/]
%%   \setmonofont{DejaVuSansMono.ttf}[Path=/home/owen/progs/firedrake-complex/firedrake/lib/python3.5/site-packages/matplotlib/mpl-data/fonts/ttf/]
%%
\begingroup%
\makeatletter%
\begin{pgfpicture}%
\pgfpathrectangle{\pgfpointorigin}{\pgfqpoint{6.400000in}{4.800000in}}%
\pgfusepath{use as bounding box, clip}%
\begin{pgfscope}%
\pgfsetbuttcap%
\pgfsetmiterjoin%
\definecolor{currentfill}{rgb}{1.000000,1.000000,1.000000}%
\pgfsetfillcolor{currentfill}%
\pgfsetlinewidth{0.000000pt}%
\definecolor{currentstroke}{rgb}{1.000000,1.000000,1.000000}%
\pgfsetstrokecolor{currentstroke}%
\pgfsetdash{}{0pt}%
\pgfpathmoveto{\pgfqpoint{0.000000in}{0.000000in}}%
\pgfpathlineto{\pgfqpoint{6.400000in}{0.000000in}}%
\pgfpathlineto{\pgfqpoint{6.400000in}{4.800000in}}%
\pgfpathlineto{\pgfqpoint{0.000000in}{4.800000in}}%
\pgfpathclose%
\pgfusepath{fill}%
\end{pgfscope}%
\begin{pgfscope}%
\pgfsetbuttcap%
\pgfsetmiterjoin%
\definecolor{currentfill}{rgb}{1.000000,1.000000,1.000000}%
\pgfsetfillcolor{currentfill}%
\pgfsetlinewidth{0.000000pt}%
\definecolor{currentstroke}{rgb}{0.000000,0.000000,0.000000}%
\pgfsetstrokecolor{currentstroke}%
\pgfsetstrokeopacity{0.000000}%
\pgfsetdash{}{0pt}%
\pgfpathmoveto{\pgfqpoint{0.800000in}{0.528000in}}%
\pgfpathlineto{\pgfqpoint{5.760000in}{0.528000in}}%
\pgfpathlineto{\pgfqpoint{5.760000in}{4.224000in}}%
\pgfpathlineto{\pgfqpoint{0.800000in}{4.224000in}}%
\pgfpathclose%
\pgfusepath{fill}%
\end{pgfscope}%
\begin{pgfscope}%
\pgfsetbuttcap%
\pgfsetroundjoin%
\definecolor{currentfill}{rgb}{0.000000,0.000000,0.000000}%
\pgfsetfillcolor{currentfill}%
\pgfsetlinewidth{0.803000pt}%
\definecolor{currentstroke}{rgb}{0.000000,0.000000,0.000000}%
\pgfsetstrokecolor{currentstroke}%
\pgfsetdash{}{0pt}%
\pgfsys@defobject{currentmarker}{\pgfqpoint{0.000000in}{-0.048611in}}{\pgfqpoint{0.000000in}{0.000000in}}{%
\pgfpathmoveto{\pgfqpoint{0.000000in}{0.000000in}}%
\pgfpathlineto{\pgfqpoint{0.000000in}{-0.048611in}}%
\pgfusepath{stroke,fill}%
}%
\begin{pgfscope}%
\pgfsys@transformshift{1.025455in}{0.528000in}%
\pgfsys@useobject{currentmarker}{}%
\end{pgfscope}%
\end{pgfscope}%
\begin{pgfscope}%
\definecolor{textcolor}{rgb}{0.000000,0.000000,0.000000}%
\pgfsetstrokecolor{textcolor}%
\pgfsetfillcolor{textcolor}%
\pgftext[x=1.025455in,y=0.430778in,,top]{\color{textcolor}\sffamily\fontsize{10.000000}{12.000000}\selectfont 10}%
\end{pgfscope}%
\begin{pgfscope}%
\pgfsetbuttcap%
\pgfsetroundjoin%
\definecolor{currentfill}{rgb}{0.000000,0.000000,0.000000}%
\pgfsetfillcolor{currentfill}%
\pgfsetlinewidth{0.803000pt}%
\definecolor{currentstroke}{rgb}{0.000000,0.000000,0.000000}%
\pgfsetstrokecolor{currentstroke}%
\pgfsetdash{}{0pt}%
\pgfsys@defobject{currentmarker}{\pgfqpoint{0.000000in}{-0.048611in}}{\pgfqpoint{0.000000in}{0.000000in}}{%
\pgfpathmoveto{\pgfqpoint{0.000000in}{0.000000in}}%
\pgfpathlineto{\pgfqpoint{0.000000in}{-0.048611in}}%
\pgfusepath{stroke,fill}%
}%
\begin{pgfscope}%
\pgfsys@transformshift{1.776970in}{0.528000in}%
\pgfsys@useobject{currentmarker}{}%
\end{pgfscope}%
\end{pgfscope}%
\begin{pgfscope}%
\definecolor{textcolor}{rgb}{0.000000,0.000000,0.000000}%
\pgfsetstrokecolor{textcolor}%
\pgfsetfillcolor{textcolor}%
\pgftext[x=1.776970in,y=0.430778in,,top]{\color{textcolor}\sffamily\fontsize{10.000000}{12.000000}\selectfont 15}%
\end{pgfscope}%
\begin{pgfscope}%
\pgfsetbuttcap%
\pgfsetroundjoin%
\definecolor{currentfill}{rgb}{0.000000,0.000000,0.000000}%
\pgfsetfillcolor{currentfill}%
\pgfsetlinewidth{0.803000pt}%
\definecolor{currentstroke}{rgb}{0.000000,0.000000,0.000000}%
\pgfsetstrokecolor{currentstroke}%
\pgfsetdash{}{0pt}%
\pgfsys@defobject{currentmarker}{\pgfqpoint{0.000000in}{-0.048611in}}{\pgfqpoint{0.000000in}{0.000000in}}{%
\pgfpathmoveto{\pgfqpoint{0.000000in}{0.000000in}}%
\pgfpathlineto{\pgfqpoint{0.000000in}{-0.048611in}}%
\pgfusepath{stroke,fill}%
}%
\begin{pgfscope}%
\pgfsys@transformshift{2.528485in}{0.528000in}%
\pgfsys@useobject{currentmarker}{}%
\end{pgfscope}%
\end{pgfscope}%
\begin{pgfscope}%
\definecolor{textcolor}{rgb}{0.000000,0.000000,0.000000}%
\pgfsetstrokecolor{textcolor}%
\pgfsetfillcolor{textcolor}%
\pgftext[x=2.528485in,y=0.430778in,,top]{\color{textcolor}\sffamily\fontsize{10.000000}{12.000000}\selectfont 20}%
\end{pgfscope}%
\begin{pgfscope}%
\pgfsetbuttcap%
\pgfsetroundjoin%
\definecolor{currentfill}{rgb}{0.000000,0.000000,0.000000}%
\pgfsetfillcolor{currentfill}%
\pgfsetlinewidth{0.803000pt}%
\definecolor{currentstroke}{rgb}{0.000000,0.000000,0.000000}%
\pgfsetstrokecolor{currentstroke}%
\pgfsetdash{}{0pt}%
\pgfsys@defobject{currentmarker}{\pgfqpoint{0.000000in}{-0.048611in}}{\pgfqpoint{0.000000in}{0.000000in}}{%
\pgfpathmoveto{\pgfqpoint{0.000000in}{0.000000in}}%
\pgfpathlineto{\pgfqpoint{0.000000in}{-0.048611in}}%
\pgfusepath{stroke,fill}%
}%
\begin{pgfscope}%
\pgfsys@transformshift{3.280000in}{0.528000in}%
\pgfsys@useobject{currentmarker}{}%
\end{pgfscope}%
\end{pgfscope}%
\begin{pgfscope}%
\definecolor{textcolor}{rgb}{0.000000,0.000000,0.000000}%
\pgfsetstrokecolor{textcolor}%
\pgfsetfillcolor{textcolor}%
\pgftext[x=3.280000in,y=0.430778in,,top]{\color{textcolor}\sffamily\fontsize{10.000000}{12.000000}\selectfont 25}%
\end{pgfscope}%
\begin{pgfscope}%
\pgfsetbuttcap%
\pgfsetroundjoin%
\definecolor{currentfill}{rgb}{0.000000,0.000000,0.000000}%
\pgfsetfillcolor{currentfill}%
\pgfsetlinewidth{0.803000pt}%
\definecolor{currentstroke}{rgb}{0.000000,0.000000,0.000000}%
\pgfsetstrokecolor{currentstroke}%
\pgfsetdash{}{0pt}%
\pgfsys@defobject{currentmarker}{\pgfqpoint{0.000000in}{-0.048611in}}{\pgfqpoint{0.000000in}{0.000000in}}{%
\pgfpathmoveto{\pgfqpoint{0.000000in}{0.000000in}}%
\pgfpathlineto{\pgfqpoint{0.000000in}{-0.048611in}}%
\pgfusepath{stroke,fill}%
}%
\begin{pgfscope}%
\pgfsys@transformshift{4.031515in}{0.528000in}%
\pgfsys@useobject{currentmarker}{}%
\end{pgfscope}%
\end{pgfscope}%
\begin{pgfscope}%
\definecolor{textcolor}{rgb}{0.000000,0.000000,0.000000}%
\pgfsetstrokecolor{textcolor}%
\pgfsetfillcolor{textcolor}%
\pgftext[x=4.031515in,y=0.430778in,,top]{\color{textcolor}\sffamily\fontsize{10.000000}{12.000000}\selectfont 30}%
\end{pgfscope}%
\begin{pgfscope}%
\pgfsetbuttcap%
\pgfsetroundjoin%
\definecolor{currentfill}{rgb}{0.000000,0.000000,0.000000}%
\pgfsetfillcolor{currentfill}%
\pgfsetlinewidth{0.803000pt}%
\definecolor{currentstroke}{rgb}{0.000000,0.000000,0.000000}%
\pgfsetstrokecolor{currentstroke}%
\pgfsetdash{}{0pt}%
\pgfsys@defobject{currentmarker}{\pgfqpoint{0.000000in}{-0.048611in}}{\pgfqpoint{0.000000in}{0.000000in}}{%
\pgfpathmoveto{\pgfqpoint{0.000000in}{0.000000in}}%
\pgfpathlineto{\pgfqpoint{0.000000in}{-0.048611in}}%
\pgfusepath{stroke,fill}%
}%
\begin{pgfscope}%
\pgfsys@transformshift{4.783030in}{0.528000in}%
\pgfsys@useobject{currentmarker}{}%
\end{pgfscope}%
\end{pgfscope}%
\begin{pgfscope}%
\definecolor{textcolor}{rgb}{0.000000,0.000000,0.000000}%
\pgfsetstrokecolor{textcolor}%
\pgfsetfillcolor{textcolor}%
\pgftext[x=4.783030in,y=0.430778in,,top]{\color{textcolor}\sffamily\fontsize{10.000000}{12.000000}\selectfont 35}%
\end{pgfscope}%
\begin{pgfscope}%
\pgfsetbuttcap%
\pgfsetroundjoin%
\definecolor{currentfill}{rgb}{0.000000,0.000000,0.000000}%
\pgfsetfillcolor{currentfill}%
\pgfsetlinewidth{0.803000pt}%
\definecolor{currentstroke}{rgb}{0.000000,0.000000,0.000000}%
\pgfsetstrokecolor{currentstroke}%
\pgfsetdash{}{0pt}%
\pgfsys@defobject{currentmarker}{\pgfqpoint{0.000000in}{-0.048611in}}{\pgfqpoint{0.000000in}{0.000000in}}{%
\pgfpathmoveto{\pgfqpoint{0.000000in}{0.000000in}}%
\pgfpathlineto{\pgfqpoint{0.000000in}{-0.048611in}}%
\pgfusepath{stroke,fill}%
}%
\begin{pgfscope}%
\pgfsys@transformshift{5.534545in}{0.528000in}%
\pgfsys@useobject{currentmarker}{}%
\end{pgfscope}%
\end{pgfscope}%
\begin{pgfscope}%
\definecolor{textcolor}{rgb}{0.000000,0.000000,0.000000}%
\pgfsetstrokecolor{textcolor}%
\pgfsetfillcolor{textcolor}%
\pgftext[x=5.534545in,y=0.430778in,,top]{\color{textcolor}\sffamily\fontsize{10.000000}{12.000000}\selectfont 40}%
\end{pgfscope}%
\begin{pgfscope}%
\definecolor{textcolor}{rgb}{0.000000,0.000000,0.000000}%
\pgfsetstrokecolor{textcolor}%
\pgfsetfillcolor{textcolor}%
\pgftext[x=3.280000in,y=0.240809in,,top]{\color{textcolor}\sffamily\fontsize{10.000000}{12.000000}\selectfont \(\displaystyle k\)}%
\end{pgfscope}%
\begin{pgfscope}%
\pgfsetbuttcap%
\pgfsetroundjoin%
\definecolor{currentfill}{rgb}{0.000000,0.000000,0.000000}%
\pgfsetfillcolor{currentfill}%
\pgfsetlinewidth{0.803000pt}%
\definecolor{currentstroke}{rgb}{0.000000,0.000000,0.000000}%
\pgfsetstrokecolor{currentstroke}%
\pgfsetdash{}{0pt}%
\pgfsys@defobject{currentmarker}{\pgfqpoint{-0.048611in}{0.000000in}}{\pgfqpoint{0.000000in}{0.000000in}}{%
\pgfpathmoveto{\pgfqpoint{0.000000in}{0.000000in}}%
\pgfpathlineto{\pgfqpoint{-0.048611in}{0.000000in}}%
\pgfusepath{stroke,fill}%
}%
\begin{pgfscope}%
\pgfsys@transformshift{0.800000in}{0.848886in}%
\pgfsys@useobject{currentmarker}{}%
\end{pgfscope}%
\end{pgfscope}%
\begin{pgfscope}%
\definecolor{textcolor}{rgb}{0.000000,0.000000,0.000000}%
\pgfsetstrokecolor{textcolor}%
\pgfsetfillcolor{textcolor}%
\pgftext[x=0.393533in,y=0.796124in,left,base]{\color{textcolor}\sffamily\fontsize{10.000000}{12.000000}\selectfont 0.01}%
\end{pgfscope}%
\begin{pgfscope}%
\pgfsetbuttcap%
\pgfsetroundjoin%
\definecolor{currentfill}{rgb}{0.000000,0.000000,0.000000}%
\pgfsetfillcolor{currentfill}%
\pgfsetlinewidth{0.803000pt}%
\definecolor{currentstroke}{rgb}{0.000000,0.000000,0.000000}%
\pgfsetstrokecolor{currentstroke}%
\pgfsetdash{}{0pt}%
\pgfsys@defobject{currentmarker}{\pgfqpoint{-0.048611in}{0.000000in}}{\pgfqpoint{0.000000in}{0.000000in}}{%
\pgfpathmoveto{\pgfqpoint{0.000000in}{0.000000in}}%
\pgfpathlineto{\pgfqpoint{-0.048611in}{0.000000in}}%
\pgfusepath{stroke,fill}%
}%
\begin{pgfscope}%
\pgfsys@transformshift{0.800000in}{2.141757in}%
\pgfsys@useobject{currentmarker}{}%
\end{pgfscope}%
\end{pgfscope}%
\begin{pgfscope}%
\definecolor{textcolor}{rgb}{0.000000,0.000000,0.000000}%
\pgfsetstrokecolor{textcolor}%
\pgfsetfillcolor{textcolor}%
\pgftext[x=0.393533in,y=2.088995in,left,base]{\color{textcolor}\sffamily\fontsize{10.000000}{12.000000}\selectfont 0.02}%
\end{pgfscope}%
\begin{pgfscope}%
\pgfsetbuttcap%
\pgfsetroundjoin%
\definecolor{currentfill}{rgb}{0.000000,0.000000,0.000000}%
\pgfsetfillcolor{currentfill}%
\pgfsetlinewidth{0.803000pt}%
\definecolor{currentstroke}{rgb}{0.000000,0.000000,0.000000}%
\pgfsetstrokecolor{currentstroke}%
\pgfsetdash{}{0pt}%
\pgfsys@defobject{currentmarker}{\pgfqpoint{-0.048611in}{0.000000in}}{\pgfqpoint{0.000000in}{0.000000in}}{%
\pgfpathmoveto{\pgfqpoint{0.000000in}{0.000000in}}%
\pgfpathlineto{\pgfqpoint{-0.048611in}{0.000000in}}%
\pgfusepath{stroke,fill}%
}%
\begin{pgfscope}%
\pgfsys@transformshift{0.800000in}{3.434628in}%
\pgfsys@useobject{currentmarker}{}%
\end{pgfscope}%
\end{pgfscope}%
\begin{pgfscope}%
\definecolor{textcolor}{rgb}{0.000000,0.000000,0.000000}%
\pgfsetstrokecolor{textcolor}%
\pgfsetfillcolor{textcolor}%
\pgftext[x=0.393533in,y=3.381866in,left,base]{\color{textcolor}\sffamily\fontsize{10.000000}{12.000000}\selectfont 0.03}%
\end{pgfscope}%
\begin{pgfscope}%
\definecolor{textcolor}{rgb}{0.000000,0.000000,0.000000}%
\pgfsetstrokecolor{textcolor}%
\pgfsetfillcolor{textcolor}%
\pgftext[x=0.337977in,y=2.376000in,,bottom,rotate=90.000000]{\color{textcolor}\sffamily\fontsize{10.000000}{12.000000}\selectfont Probability Number of GMRES iterations is at most 12}%
\end{pgfscope}%
\begin{pgfscope}%
\pgfpathrectangle{\pgfqpoint{0.800000in}{0.528000in}}{\pgfqpoint{4.960000in}{3.696000in}}%
\pgfusepath{clip}%
\pgfsetbuttcap%
\pgfsetroundjoin%
\definecolor{currentfill}{rgb}{0.121569,0.466667,0.705882}%
\pgfsetfillcolor{currentfill}%
\pgfsetlinewidth{1.003750pt}%
\definecolor{currentstroke}{rgb}{0.121569,0.466667,0.705882}%
\pgfsetstrokecolor{currentstroke}%
\pgfsetdash{}{0pt}%
\pgfsys@defobject{currentmarker}{\pgfqpoint{-0.020833in}{-0.020833in}}{\pgfqpoint{0.020833in}{0.020833in}}{%
\pgfpathmoveto{\pgfqpoint{0.000000in}{-0.020833in}}%
\pgfpathcurveto{\pgfqpoint{0.005525in}{-0.020833in}}{\pgfqpoint{0.010825in}{-0.018638in}}{\pgfqpoint{0.014731in}{-0.014731in}}%
\pgfpathcurveto{\pgfqpoint{0.018638in}{-0.010825in}}{\pgfqpoint{0.020833in}{-0.005525in}}{\pgfqpoint{0.020833in}{0.000000in}}%
\pgfpathcurveto{\pgfqpoint{0.020833in}{0.005525in}}{\pgfqpoint{0.018638in}{0.010825in}}{\pgfqpoint{0.014731in}{0.014731in}}%
\pgfpathcurveto{\pgfqpoint{0.010825in}{0.018638in}}{\pgfqpoint{0.005525in}{0.020833in}}{\pgfqpoint{0.000000in}{0.020833in}}%
\pgfpathcurveto{\pgfqpoint{-0.005525in}{0.020833in}}{\pgfqpoint{-0.010825in}{0.018638in}}{\pgfqpoint{-0.014731in}{0.014731in}}%
\pgfpathcurveto{\pgfqpoint{-0.018638in}{0.010825in}}{\pgfqpoint{-0.020833in}{0.005525in}}{\pgfqpoint{-0.020833in}{0.000000in}}%
\pgfpathcurveto{\pgfqpoint{-0.020833in}{-0.005525in}}{\pgfqpoint{-0.018638in}{-0.010825in}}{\pgfqpoint{-0.014731in}{-0.014731in}}%
\pgfpathcurveto{\pgfqpoint{-0.010825in}{-0.018638in}}{\pgfqpoint{-0.005525in}{-0.020833in}}{\pgfqpoint{0.000000in}{-0.020833in}}%
\pgfpathclose%
\pgfusepath{stroke,fill}%
}%
\begin{pgfscope}%
\pgfsys@transformshift{1.025455in}{4.056000in}%
\pgfsys@useobject{currentmarker}{}%
\end{pgfscope}%
\begin{pgfscope}%
\pgfsys@transformshift{1.070545in}{3.927174in}%
\pgfsys@useobject{currentmarker}{}%
\end{pgfscope}%
\begin{pgfscope}%
\pgfsys@transformshift{1.115636in}{3.805518in}%
\pgfsys@useobject{currentmarker}{}%
\end{pgfscope}%
\begin{pgfscope}%
\pgfsys@transformshift{1.160727in}{3.690450in}%
\pgfsys@useobject{currentmarker}{}%
\end{pgfscope}%
\begin{pgfscope}%
\pgfsys@transformshift{1.205818in}{3.581448in}%
\pgfsys@useobject{currentmarker}{}%
\end{pgfscope}%
\begin{pgfscope}%
\pgfsys@transformshift{1.250909in}{3.478046in}%
\pgfsys@useobject{currentmarker}{}%
\end{pgfscope}%
\begin{pgfscope}%
\pgfsys@transformshift{1.296000in}{3.379823in}%
\pgfsys@useobject{currentmarker}{}%
\end{pgfscope}%
\begin{pgfscope}%
\pgfsys@transformshift{1.341091in}{3.286399in}%
\pgfsys@useobject{currentmarker}{}%
\end{pgfscope}%
\begin{pgfscope}%
\pgfsys@transformshift{1.386182in}{3.197431in}%
\pgfsys@useobject{currentmarker}{}%
\end{pgfscope}%
\begin{pgfscope}%
\pgfsys@transformshift{1.431273in}{3.112608in}%
\pgfsys@useobject{currentmarker}{}%
\end{pgfscope}%
\begin{pgfscope}%
\pgfsys@transformshift{1.476364in}{3.031646in}%
\pgfsys@useobject{currentmarker}{}%
\end{pgfscope}%
\begin{pgfscope}%
\pgfsys@transformshift{1.521455in}{2.954288in}%
\pgfsys@useobject{currentmarker}{}%
\end{pgfscope}%
\begin{pgfscope}%
\pgfsys@transformshift{1.566545in}{2.880298in}%
\pgfsys@useobject{currentmarker}{}%
\end{pgfscope}%
\begin{pgfscope}%
\pgfsys@transformshift{1.611636in}{2.809462in}%
\pgfsys@useobject{currentmarker}{}%
\end{pgfscope}%
\begin{pgfscope}%
\pgfsys@transformshift{1.656727in}{2.741581in}%
\pgfsys@useobject{currentmarker}{}%
\end{pgfscope}%
\begin{pgfscope}%
\pgfsys@transformshift{1.701818in}{2.676474in}%
\pgfsys@useobject{currentmarker}{}%
\end{pgfscope}%
\begin{pgfscope}%
\pgfsys@transformshift{1.746909in}{2.613976in}%
\pgfsys@useobject{currentmarker}{}%
\end{pgfscope}%
\begin{pgfscope}%
\pgfsys@transformshift{1.792000in}{2.553932in}%
\pgfsys@useobject{currentmarker}{}%
\end{pgfscope}%
\begin{pgfscope}%
\pgfsys@transformshift{1.837091in}{2.496200in}%
\pgfsys@useobject{currentmarker}{}%
\end{pgfscope}%
\begin{pgfscope}%
\pgfsys@transformshift{1.882182in}{2.440649in}%
\pgfsys@useobject{currentmarker}{}%
\end{pgfscope}%
\begin{pgfscope}%
\pgfsys@transformshift{1.927273in}{2.387159in}%
\pgfsys@useobject{currentmarker}{}%
\end{pgfscope}%
\begin{pgfscope}%
\pgfsys@transformshift{1.972364in}{2.335616in}%
\pgfsys@useobject{currentmarker}{}%
\end{pgfscope}%
\begin{pgfscope}%
\pgfsys@transformshift{2.017455in}{2.285917in}%
\pgfsys@useobject{currentmarker}{}%
\end{pgfscope}%
\begin{pgfscope}%
\pgfsys@transformshift{2.062545in}{2.237963in}%
\pgfsys@useobject{currentmarker}{}%
\end{pgfscope}%
\begin{pgfscope}%
\pgfsys@transformshift{2.107636in}{2.191665in}%
\pgfsys@useobject{currentmarker}{}%
\end{pgfscope}%
\begin{pgfscope}%
\pgfsys@transformshift{2.152727in}{2.146938in}%
\pgfsys@useobject{currentmarker}{}%
\end{pgfscope}%
\begin{pgfscope}%
\pgfsys@transformshift{2.197818in}{2.103704in}%
\pgfsys@useobject{currentmarker}{}%
\end{pgfscope}%
\begin{pgfscope}%
\pgfsys@transformshift{2.242909in}{2.061889in}%
\pgfsys@useobject{currentmarker}{}%
\end{pgfscope}%
\begin{pgfscope}%
\pgfsys@transformshift{2.288000in}{2.021425in}%
\pgfsys@useobject{currentmarker}{}%
\end{pgfscope}%
\begin{pgfscope}%
\pgfsys@transformshift{2.333091in}{1.982246in}%
\pgfsys@useobject{currentmarker}{}%
\end{pgfscope}%
\begin{pgfscope}%
\pgfsys@transformshift{2.378182in}{1.944293in}%
\pgfsys@useobject{currentmarker}{}%
\end{pgfscope}%
\begin{pgfscope}%
\pgfsys@transformshift{2.423273in}{1.907509in}%
\pgfsys@useobject{currentmarker}{}%
\end{pgfscope}%
\begin{pgfscope}%
\pgfsys@transformshift{2.468364in}{1.871841in}%
\pgfsys@useobject{currentmarker}{}%
\end{pgfscope}%
\begin{pgfscope}%
\pgfsys@transformshift{2.513455in}{1.837239in}%
\pgfsys@useobject{currentmarker}{}%
\end{pgfscope}%
\begin{pgfscope}%
\pgfsys@transformshift{2.558545in}{1.803656in}%
\pgfsys@useobject{currentmarker}{}%
\end{pgfscope}%
\begin{pgfscope}%
\pgfsys@transformshift{2.603636in}{1.771047in}%
\pgfsys@useobject{currentmarker}{}%
\end{pgfscope}%
\begin{pgfscope}%
\pgfsys@transformshift{2.648727in}{1.739371in}%
\pgfsys@useobject{currentmarker}{}%
\end{pgfscope}%
\begin{pgfscope}%
\pgfsys@transformshift{2.693818in}{1.708588in}%
\pgfsys@useobject{currentmarker}{}%
\end{pgfscope}%
\begin{pgfscope}%
\pgfsys@transformshift{2.738909in}{1.678660in}%
\pgfsys@useobject{currentmarker}{}%
\end{pgfscope}%
\begin{pgfscope}%
\pgfsys@transformshift{2.784000in}{1.649554in}%
\pgfsys@useobject{currentmarker}{}%
\end{pgfscope}%
\begin{pgfscope}%
\pgfsys@transformshift{2.829091in}{1.621235in}%
\pgfsys@useobject{currentmarker}{}%
\end{pgfscope}%
\begin{pgfscope}%
\pgfsys@transformshift{2.874182in}{1.593672in}%
\pgfsys@useobject{currentmarker}{}%
\end{pgfscope}%
\begin{pgfscope}%
\pgfsys@transformshift{2.919273in}{1.566835in}%
\pgfsys@useobject{currentmarker}{}%
\end{pgfscope}%
\begin{pgfscope}%
\pgfsys@transformshift{2.964364in}{1.540695in}%
\pgfsys@useobject{currentmarker}{}%
\end{pgfscope}%
\begin{pgfscope}%
\pgfsys@transformshift{3.009455in}{1.515227in}%
\pgfsys@useobject{currentmarker}{}%
\end{pgfscope}%
\begin{pgfscope}%
\pgfsys@transformshift{3.054545in}{1.490404in}%
\pgfsys@useobject{currentmarker}{}%
\end{pgfscope}%
\begin{pgfscope}%
\pgfsys@transformshift{3.099636in}{1.466202in}%
\pgfsys@useobject{currentmarker}{}%
\end{pgfscope}%
\begin{pgfscope}%
\pgfsys@transformshift{3.144727in}{1.442598in}%
\pgfsys@useobject{currentmarker}{}%
\end{pgfscope}%
\begin{pgfscope}%
\pgfsys@transformshift{3.189818in}{1.419570in}%
\pgfsys@useobject{currentmarker}{}%
\end{pgfscope}%
\begin{pgfscope}%
\pgfsys@transformshift{3.234909in}{1.397098in}%
\pgfsys@useobject{currentmarker}{}%
\end{pgfscope}%
\begin{pgfscope}%
\pgfsys@transformshift{3.280000in}{1.375161in}%
\pgfsys@useobject{currentmarker}{}%
\end{pgfscope}%
\begin{pgfscope}%
\pgfsys@transformshift{3.325091in}{1.353741in}%
\pgfsys@useobject{currentmarker}{}%
\end{pgfscope}%
\begin{pgfscope}%
\pgfsys@transformshift{3.370182in}{1.332819in}%
\pgfsys@useobject{currentmarker}{}%
\end{pgfscope}%
\begin{pgfscope}%
\pgfsys@transformshift{3.415273in}{1.312379in}%
\pgfsys@useobject{currentmarker}{}%
\end{pgfscope}%
\begin{pgfscope}%
\pgfsys@transformshift{3.460364in}{1.292403in}%
\pgfsys@useobject{currentmarker}{}%
\end{pgfscope}%
\begin{pgfscope}%
\pgfsys@transformshift{3.505455in}{1.272877in}%
\pgfsys@useobject{currentmarker}{}%
\end{pgfscope}%
\begin{pgfscope}%
\pgfsys@transformshift{3.550545in}{1.253785in}%
\pgfsys@useobject{currentmarker}{}%
\end{pgfscope}%
\begin{pgfscope}%
\pgfsys@transformshift{3.595636in}{1.235113in}%
\pgfsys@useobject{currentmarker}{}%
\end{pgfscope}%
\begin{pgfscope}%
\pgfsys@transformshift{3.640727in}{1.216848in}%
\pgfsys@useobject{currentmarker}{}%
\end{pgfscope}%
\begin{pgfscope}%
\pgfsys@transformshift{3.685818in}{1.198975in}%
\pgfsys@useobject{currentmarker}{}%
\end{pgfscope}%
\begin{pgfscope}%
\pgfsys@transformshift{3.730909in}{1.181483in}%
\pgfsys@useobject{currentmarker}{}%
\end{pgfscope}%
\begin{pgfscope}%
\pgfsys@transformshift{3.776000in}{1.164360in}%
\pgfsys@useobject{currentmarker}{}%
\end{pgfscope}%
\begin{pgfscope}%
\pgfsys@transformshift{3.821091in}{1.147593in}%
\pgfsys@useobject{currentmarker}{}%
\end{pgfscope}%
\begin{pgfscope}%
\pgfsys@transformshift{3.866182in}{1.131173in}%
\pgfsys@useobject{currentmarker}{}%
\end{pgfscope}%
\begin{pgfscope}%
\pgfsys@transformshift{3.911273in}{1.115087in}%
\pgfsys@useobject{currentmarker}{}%
\end{pgfscope}%
\begin{pgfscope}%
\pgfsys@transformshift{3.956364in}{1.099327in}%
\pgfsys@useobject{currentmarker}{}%
\end{pgfscope}%
\begin{pgfscope}%
\pgfsys@transformshift{4.001455in}{1.083883in}%
\pgfsys@useobject{currentmarker}{}%
\end{pgfscope}%
\begin{pgfscope}%
\pgfsys@transformshift{4.046545in}{1.068744in}%
\pgfsys@useobject{currentmarker}{}%
\end{pgfscope}%
\begin{pgfscope}%
\pgfsys@transformshift{4.091636in}{1.053903in}%
\pgfsys@useobject{currentmarker}{}%
\end{pgfscope}%
\begin{pgfscope}%
\pgfsys@transformshift{4.136727in}{1.039350in}%
\pgfsys@useobject{currentmarker}{}%
\end{pgfscope}%
\begin{pgfscope}%
\pgfsys@transformshift{4.181818in}{1.025077in}%
\pgfsys@useobject{currentmarker}{}%
\end{pgfscope}%
\begin{pgfscope}%
\pgfsys@transformshift{4.226909in}{1.011076in}%
\pgfsys@useobject{currentmarker}{}%
\end{pgfscope}%
\begin{pgfscope}%
\pgfsys@transformshift{4.272000in}{0.997339in}%
\pgfsys@useobject{currentmarker}{}%
\end{pgfscope}%
\begin{pgfscope}%
\pgfsys@transformshift{4.317091in}{0.983860in}%
\pgfsys@useobject{currentmarker}{}%
\end{pgfscope}%
\begin{pgfscope}%
\pgfsys@transformshift{4.362182in}{0.970630in}%
\pgfsys@useobject{currentmarker}{}%
\end{pgfscope}%
\begin{pgfscope}%
\pgfsys@transformshift{4.407273in}{0.957643in}%
\pgfsys@useobject{currentmarker}{}%
\end{pgfscope}%
\begin{pgfscope}%
\pgfsys@transformshift{4.452364in}{0.944892in}%
\pgfsys@useobject{currentmarker}{}%
\end{pgfscope}%
\begin{pgfscope}%
\pgfsys@transformshift{4.497455in}{0.932371in}%
\pgfsys@useobject{currentmarker}{}%
\end{pgfscope}%
\begin{pgfscope}%
\pgfsys@transformshift{4.542545in}{0.920074in}%
\pgfsys@useobject{currentmarker}{}%
\end{pgfscope}%
\begin{pgfscope}%
\pgfsys@transformshift{4.587636in}{0.907995in}%
\pgfsys@useobject{currentmarker}{}%
\end{pgfscope}%
\begin{pgfscope}%
\pgfsys@transformshift{4.632727in}{0.896128in}%
\pgfsys@useobject{currentmarker}{}%
\end{pgfscope}%
\begin{pgfscope}%
\pgfsys@transformshift{4.677818in}{0.884467in}%
\pgfsys@useobject{currentmarker}{}%
\end{pgfscope}%
\begin{pgfscope}%
\pgfsys@transformshift{4.722909in}{0.873008in}%
\pgfsys@useobject{currentmarker}{}%
\end{pgfscope}%
\begin{pgfscope}%
\pgfsys@transformshift{4.768000in}{0.861744in}%
\pgfsys@useobject{currentmarker}{}%
\end{pgfscope}%
\begin{pgfscope}%
\pgfsys@transformshift{4.813091in}{0.850672in}%
\pgfsys@useobject{currentmarker}{}%
\end{pgfscope}%
\begin{pgfscope}%
\pgfsys@transformshift{4.858182in}{0.839786in}%
\pgfsys@useobject{currentmarker}{}%
\end{pgfscope}%
\begin{pgfscope}%
\pgfsys@transformshift{4.903273in}{0.829081in}%
\pgfsys@useobject{currentmarker}{}%
\end{pgfscope}%
\begin{pgfscope}%
\pgfsys@transformshift{4.948364in}{0.818553in}%
\pgfsys@useobject{currentmarker}{}%
\end{pgfscope}%
\begin{pgfscope}%
\pgfsys@transformshift{4.993455in}{0.808198in}%
\pgfsys@useobject{currentmarker}{}%
\end{pgfscope}%
\begin{pgfscope}%
\pgfsys@transformshift{5.038545in}{0.798012in}%
\pgfsys@useobject{currentmarker}{}%
\end{pgfscope}%
\begin{pgfscope}%
\pgfsys@transformshift{5.083636in}{0.787990in}%
\pgfsys@useobject{currentmarker}{}%
\end{pgfscope}%
\begin{pgfscope}%
\pgfsys@transformshift{5.128727in}{0.778128in}%
\pgfsys@useobject{currentmarker}{}%
\end{pgfscope}%
\begin{pgfscope}%
\pgfsys@transformshift{5.173818in}{0.768423in}%
\pgfsys@useobject{currentmarker}{}%
\end{pgfscope}%
\begin{pgfscope}%
\pgfsys@transformshift{5.218909in}{0.758871in}%
\pgfsys@useobject{currentmarker}{}%
\end{pgfscope}%
\begin{pgfscope}%
\pgfsys@transformshift{5.264000in}{0.749468in}%
\pgfsys@useobject{currentmarker}{}%
\end{pgfscope}%
\begin{pgfscope}%
\pgfsys@transformshift{5.309091in}{0.740211in}%
\pgfsys@useobject{currentmarker}{}%
\end{pgfscope}%
\begin{pgfscope}%
\pgfsys@transformshift{5.354182in}{0.731097in}%
\pgfsys@useobject{currentmarker}{}%
\end{pgfscope}%
\begin{pgfscope}%
\pgfsys@transformshift{5.399273in}{0.722121in}%
\pgfsys@useobject{currentmarker}{}%
\end{pgfscope}%
\begin{pgfscope}%
\pgfsys@transformshift{5.444364in}{0.713282in}%
\pgfsys@useobject{currentmarker}{}%
\end{pgfscope}%
\begin{pgfscope}%
\pgfsys@transformshift{5.489455in}{0.704576in}%
\pgfsys@useobject{currentmarker}{}%
\end{pgfscope}%
\begin{pgfscope}%
\pgfsys@transformshift{5.534545in}{0.696000in}%
\pgfsys@useobject{currentmarker}{}%
\end{pgfscope}%
\end{pgfscope}%
\begin{pgfscope}%
\pgfsetrectcap%
\pgfsetmiterjoin%
\pgfsetlinewidth{0.803000pt}%
\definecolor{currentstroke}{rgb}{0.000000,0.000000,0.000000}%
\pgfsetstrokecolor{currentstroke}%
\pgfsetdash{}{0pt}%
\pgfpathmoveto{\pgfqpoint{0.800000in}{0.528000in}}%
\pgfpathlineto{\pgfqpoint{0.800000in}{4.224000in}}%
\pgfusepath{stroke}%
\end{pgfscope}%
\begin{pgfscope}%
\pgfsetrectcap%
\pgfsetmiterjoin%
\pgfsetlinewidth{0.803000pt}%
\definecolor{currentstroke}{rgb}{0.000000,0.000000,0.000000}%
\pgfsetstrokecolor{currentstroke}%
\pgfsetdash{}{0pt}%
\pgfpathmoveto{\pgfqpoint{5.760000in}{0.528000in}}%
\pgfpathlineto{\pgfqpoint{5.760000in}{4.224000in}}%
\pgfusepath{stroke}%
\end{pgfscope}%
\begin{pgfscope}%
\pgfsetrectcap%
\pgfsetmiterjoin%
\pgfsetlinewidth{0.803000pt}%
\definecolor{currentstroke}{rgb}{0.000000,0.000000,0.000000}%
\pgfsetstrokecolor{currentstroke}%
\pgfsetdash{}{0pt}%
\pgfpathmoveto{\pgfqpoint{0.800000in}{0.528000in}}%
\pgfpathlineto{\pgfqpoint{5.760000in}{0.528000in}}%
\pgfusepath{stroke}%
\end{pgfscope}%
\begin{pgfscope}%
\pgfsetrectcap%
\pgfsetmiterjoin%
\pgfsetlinewidth{0.803000pt}%
\definecolor{currentstroke}{rgb}{0.000000,0.000000,0.000000}%
\pgfsetstrokecolor{currentstroke}%
\pgfsetdash{}{0pt}%
\pgfpathmoveto{\pgfqpoint{0.800000in}{4.224000in}}%
\pgfpathlineto{\pgfqpoint{5.760000in}{4.224000in}}%
\pgfusepath{stroke}%
\end{pgfscope}%
\end{pgfpicture}%
\makeatother%
\endgroup%

\caption{The lower bound in \cref{eq:GMRESprob} with $\NLiDRR{\no-\nt} \sim \Exp{\sigma}$ with $\sigma = 1.$\label{fig:prob-theory-plot-0.0}}
\end{subfigure}

\begin{subfigure}{\textwidth}
    \centering
%% Creator: Matplotlib, PGF backend
%%
%% To include the figure in your LaTeX document, write
%%   \input{<filename>.pgf}
%%
%% Make sure the required packages are loaded in your preamble
%%   \usepackage{pgf}
%%
%% Figures using additional raster images can only be included by \input if
%% they are in the same directory as the main LaTeX file. For loading figures
%% from other directories you can use the `import` package
%%   \usepackage{import}
%% and then include the figures with
%%   \import{<path to file>}{<filename>.pgf}
%%
%% Matplotlib used the following preamble
%%   \usepackage{fontspec}
%%   \setmainfont{DejaVuSerif.ttf}[Path=/home/owen/progs/firedrake-complex/firedrake/lib/python3.5/site-packages/matplotlib/mpl-data/fonts/ttf/]
%%   \setsansfont{DejaVuSans.ttf}[Path=/home/owen/progs/firedrake-complex/firedrake/lib/python3.5/site-packages/matplotlib/mpl-data/fonts/ttf/]
%%   \setmonofont{DejaVuSansMono.ttf}[Path=/home/owen/progs/firedrake-complex/firedrake/lib/python3.5/site-packages/matplotlib/mpl-data/fonts/ttf/]
%%
\begingroup%
\makeatletter%
\begin{pgfpicture}%
\pgfpathrectangle{\pgfpointorigin}{\pgfqpoint{6.000000in}{2.500000in}}%
\pgfusepath{use as bounding box, clip}%
\begin{pgfscope}%
\pgfsetbuttcap%
\pgfsetmiterjoin%
\definecolor{currentfill}{rgb}{1.000000,1.000000,1.000000}%
\pgfsetfillcolor{currentfill}%
\pgfsetlinewidth{0.000000pt}%
\definecolor{currentstroke}{rgb}{1.000000,1.000000,1.000000}%
\pgfsetstrokecolor{currentstroke}%
\pgfsetdash{}{0pt}%
\pgfpathmoveto{\pgfqpoint{0.000000in}{0.000000in}}%
\pgfpathlineto{\pgfqpoint{6.000000in}{0.000000in}}%
\pgfpathlineto{\pgfqpoint{6.000000in}{2.500000in}}%
\pgfpathlineto{\pgfqpoint{0.000000in}{2.500000in}}%
\pgfpathclose%
\pgfusepath{fill}%
\end{pgfscope}%
\begin{pgfscope}%
\pgfsetbuttcap%
\pgfsetmiterjoin%
\definecolor{currentfill}{rgb}{1.000000,1.000000,1.000000}%
\pgfsetfillcolor{currentfill}%
\pgfsetlinewidth{0.000000pt}%
\definecolor{currentstroke}{rgb}{0.000000,0.000000,0.000000}%
\pgfsetstrokecolor{currentstroke}%
\pgfsetstrokeopacity{0.000000}%
\pgfsetdash{}{0pt}%
\pgfpathmoveto{\pgfqpoint{0.750000in}{0.275000in}}%
\pgfpathlineto{\pgfqpoint{5.400000in}{0.275000in}}%
\pgfpathlineto{\pgfqpoint{5.400000in}{2.200000in}}%
\pgfpathlineto{\pgfqpoint{0.750000in}{2.200000in}}%
\pgfpathclose%
\pgfusepath{fill}%
\end{pgfscope}%
\begin{pgfscope}%
\pgfsetbuttcap%
\pgfsetroundjoin%
\definecolor{currentfill}{rgb}{0.000000,0.000000,0.000000}%
\pgfsetfillcolor{currentfill}%
\pgfsetlinewidth{0.803000pt}%
\definecolor{currentstroke}{rgb}{0.000000,0.000000,0.000000}%
\pgfsetstrokecolor{currentstroke}%
\pgfsetdash{}{0pt}%
\pgfsys@defobject{currentmarker}{\pgfqpoint{0.000000in}{-0.048611in}}{\pgfqpoint{0.000000in}{0.000000in}}{%
\pgfpathmoveto{\pgfqpoint{0.000000in}{0.000000in}}%
\pgfpathlineto{\pgfqpoint{0.000000in}{-0.048611in}}%
\pgfusepath{stroke,fill}%
}%
\begin{pgfscope}%
\pgfsys@transformshift{0.961364in}{0.275000in}%
\pgfsys@useobject{currentmarker}{}%
\end{pgfscope}%
\end{pgfscope}%
\begin{pgfscope}%
\definecolor{textcolor}{rgb}{0.000000,0.000000,0.000000}%
\pgfsetstrokecolor{textcolor}%
\pgfsetfillcolor{textcolor}%
\pgftext[x=0.961364in,y=0.177778in,,top]{\color{textcolor}\sffamily\fontsize{10.000000}{12.000000}\selectfont 10}%
\end{pgfscope}%
\begin{pgfscope}%
\pgfsetbuttcap%
\pgfsetroundjoin%
\definecolor{currentfill}{rgb}{0.000000,0.000000,0.000000}%
\pgfsetfillcolor{currentfill}%
\pgfsetlinewidth{0.803000pt}%
\definecolor{currentstroke}{rgb}{0.000000,0.000000,0.000000}%
\pgfsetstrokecolor{currentstroke}%
\pgfsetdash{}{0pt}%
\pgfsys@defobject{currentmarker}{\pgfqpoint{0.000000in}{-0.048611in}}{\pgfqpoint{0.000000in}{0.000000in}}{%
\pgfpathmoveto{\pgfqpoint{0.000000in}{0.000000in}}%
\pgfpathlineto{\pgfqpoint{0.000000in}{-0.048611in}}%
\pgfusepath{stroke,fill}%
}%
\begin{pgfscope}%
\pgfsys@transformshift{1.665909in}{0.275000in}%
\pgfsys@useobject{currentmarker}{}%
\end{pgfscope}%
\end{pgfscope}%
\begin{pgfscope}%
\definecolor{textcolor}{rgb}{0.000000,0.000000,0.000000}%
\pgfsetstrokecolor{textcolor}%
\pgfsetfillcolor{textcolor}%
\pgftext[x=1.665909in,y=0.177778in,,top]{\color{textcolor}\sffamily\fontsize{10.000000}{12.000000}\selectfont 15}%
\end{pgfscope}%
\begin{pgfscope}%
\pgfsetbuttcap%
\pgfsetroundjoin%
\definecolor{currentfill}{rgb}{0.000000,0.000000,0.000000}%
\pgfsetfillcolor{currentfill}%
\pgfsetlinewidth{0.803000pt}%
\definecolor{currentstroke}{rgb}{0.000000,0.000000,0.000000}%
\pgfsetstrokecolor{currentstroke}%
\pgfsetdash{}{0pt}%
\pgfsys@defobject{currentmarker}{\pgfqpoint{0.000000in}{-0.048611in}}{\pgfqpoint{0.000000in}{0.000000in}}{%
\pgfpathmoveto{\pgfqpoint{0.000000in}{0.000000in}}%
\pgfpathlineto{\pgfqpoint{0.000000in}{-0.048611in}}%
\pgfusepath{stroke,fill}%
}%
\begin{pgfscope}%
\pgfsys@transformshift{2.370455in}{0.275000in}%
\pgfsys@useobject{currentmarker}{}%
\end{pgfscope}%
\end{pgfscope}%
\begin{pgfscope}%
\definecolor{textcolor}{rgb}{0.000000,0.000000,0.000000}%
\pgfsetstrokecolor{textcolor}%
\pgfsetfillcolor{textcolor}%
\pgftext[x=2.370455in,y=0.177778in,,top]{\color{textcolor}\sffamily\fontsize{10.000000}{12.000000}\selectfont 20}%
\end{pgfscope}%
\begin{pgfscope}%
\pgfsetbuttcap%
\pgfsetroundjoin%
\definecolor{currentfill}{rgb}{0.000000,0.000000,0.000000}%
\pgfsetfillcolor{currentfill}%
\pgfsetlinewidth{0.803000pt}%
\definecolor{currentstroke}{rgb}{0.000000,0.000000,0.000000}%
\pgfsetstrokecolor{currentstroke}%
\pgfsetdash{}{0pt}%
\pgfsys@defobject{currentmarker}{\pgfqpoint{0.000000in}{-0.048611in}}{\pgfqpoint{0.000000in}{0.000000in}}{%
\pgfpathmoveto{\pgfqpoint{0.000000in}{0.000000in}}%
\pgfpathlineto{\pgfqpoint{0.000000in}{-0.048611in}}%
\pgfusepath{stroke,fill}%
}%
\begin{pgfscope}%
\pgfsys@transformshift{3.075000in}{0.275000in}%
\pgfsys@useobject{currentmarker}{}%
\end{pgfscope}%
\end{pgfscope}%
\begin{pgfscope}%
\definecolor{textcolor}{rgb}{0.000000,0.000000,0.000000}%
\pgfsetstrokecolor{textcolor}%
\pgfsetfillcolor{textcolor}%
\pgftext[x=3.075000in,y=0.177778in,,top]{\color{textcolor}\sffamily\fontsize{10.000000}{12.000000}\selectfont 25}%
\end{pgfscope}%
\begin{pgfscope}%
\pgfsetbuttcap%
\pgfsetroundjoin%
\definecolor{currentfill}{rgb}{0.000000,0.000000,0.000000}%
\pgfsetfillcolor{currentfill}%
\pgfsetlinewidth{0.803000pt}%
\definecolor{currentstroke}{rgb}{0.000000,0.000000,0.000000}%
\pgfsetstrokecolor{currentstroke}%
\pgfsetdash{}{0pt}%
\pgfsys@defobject{currentmarker}{\pgfqpoint{0.000000in}{-0.048611in}}{\pgfqpoint{0.000000in}{0.000000in}}{%
\pgfpathmoveto{\pgfqpoint{0.000000in}{0.000000in}}%
\pgfpathlineto{\pgfqpoint{0.000000in}{-0.048611in}}%
\pgfusepath{stroke,fill}%
}%
\begin{pgfscope}%
\pgfsys@transformshift{3.779545in}{0.275000in}%
\pgfsys@useobject{currentmarker}{}%
\end{pgfscope}%
\end{pgfscope}%
\begin{pgfscope}%
\definecolor{textcolor}{rgb}{0.000000,0.000000,0.000000}%
\pgfsetstrokecolor{textcolor}%
\pgfsetfillcolor{textcolor}%
\pgftext[x=3.779545in,y=0.177778in,,top]{\color{textcolor}\sffamily\fontsize{10.000000}{12.000000}\selectfont 30}%
\end{pgfscope}%
\begin{pgfscope}%
\pgfsetbuttcap%
\pgfsetroundjoin%
\definecolor{currentfill}{rgb}{0.000000,0.000000,0.000000}%
\pgfsetfillcolor{currentfill}%
\pgfsetlinewidth{0.803000pt}%
\definecolor{currentstroke}{rgb}{0.000000,0.000000,0.000000}%
\pgfsetstrokecolor{currentstroke}%
\pgfsetdash{}{0pt}%
\pgfsys@defobject{currentmarker}{\pgfqpoint{0.000000in}{-0.048611in}}{\pgfqpoint{0.000000in}{0.000000in}}{%
\pgfpathmoveto{\pgfqpoint{0.000000in}{0.000000in}}%
\pgfpathlineto{\pgfqpoint{0.000000in}{-0.048611in}}%
\pgfusepath{stroke,fill}%
}%
\begin{pgfscope}%
\pgfsys@transformshift{4.484091in}{0.275000in}%
\pgfsys@useobject{currentmarker}{}%
\end{pgfscope}%
\end{pgfscope}%
\begin{pgfscope}%
\definecolor{textcolor}{rgb}{0.000000,0.000000,0.000000}%
\pgfsetstrokecolor{textcolor}%
\pgfsetfillcolor{textcolor}%
\pgftext[x=4.484091in,y=0.177778in,,top]{\color{textcolor}\sffamily\fontsize{10.000000}{12.000000}\selectfont 35}%
\end{pgfscope}%
\begin{pgfscope}%
\pgfsetbuttcap%
\pgfsetroundjoin%
\definecolor{currentfill}{rgb}{0.000000,0.000000,0.000000}%
\pgfsetfillcolor{currentfill}%
\pgfsetlinewidth{0.803000pt}%
\definecolor{currentstroke}{rgb}{0.000000,0.000000,0.000000}%
\pgfsetstrokecolor{currentstroke}%
\pgfsetdash{}{0pt}%
\pgfsys@defobject{currentmarker}{\pgfqpoint{0.000000in}{-0.048611in}}{\pgfqpoint{0.000000in}{0.000000in}}{%
\pgfpathmoveto{\pgfqpoint{0.000000in}{0.000000in}}%
\pgfpathlineto{\pgfqpoint{0.000000in}{-0.048611in}}%
\pgfusepath{stroke,fill}%
}%
\begin{pgfscope}%
\pgfsys@transformshift{5.188636in}{0.275000in}%
\pgfsys@useobject{currentmarker}{}%
\end{pgfscope}%
\end{pgfscope}%
\begin{pgfscope}%
\definecolor{textcolor}{rgb}{0.000000,0.000000,0.000000}%
\pgfsetstrokecolor{textcolor}%
\pgfsetfillcolor{textcolor}%
\pgftext[x=5.188636in,y=0.177778in,,top]{\color{textcolor}\sffamily\fontsize{10.000000}{12.000000}\selectfont 40}%
\end{pgfscope}%
\begin{pgfscope}%
\definecolor{textcolor}{rgb}{0.000000,0.000000,0.000000}%
\pgfsetstrokecolor{textcolor}%
\pgfsetfillcolor{textcolor}%
\pgftext[x=3.075000in,y=-0.012191in,,top]{\color{textcolor}\sffamily\fontsize{10.000000}{12.000000}\selectfont \(\displaystyle k\)}%
\end{pgfscope}%
\begin{pgfscope}%
\pgfsetbuttcap%
\pgfsetroundjoin%
\definecolor{currentfill}{rgb}{0.000000,0.000000,0.000000}%
\pgfsetfillcolor{currentfill}%
\pgfsetlinewidth{0.803000pt}%
\definecolor{currentstroke}{rgb}{0.000000,0.000000,0.000000}%
\pgfsetstrokecolor{currentstroke}%
\pgfsetdash{}{0pt}%
\pgfsys@defobject{currentmarker}{\pgfqpoint{-0.048611in}{0.000000in}}{\pgfqpoint{0.000000in}{0.000000in}}{%
\pgfpathmoveto{\pgfqpoint{0.000000in}{0.000000in}}%
\pgfpathlineto{\pgfqpoint{-0.048611in}{0.000000in}}%
\pgfusepath{stroke,fill}%
}%
\begin{pgfscope}%
\pgfsys@transformshift{0.750000in}{0.552046in}%
\pgfsys@useobject{currentmarker}{}%
\end{pgfscope}%
\end{pgfscope}%
\begin{pgfscope}%
\definecolor{textcolor}{rgb}{0.000000,0.000000,0.000000}%
\pgfsetstrokecolor{textcolor}%
\pgfsetfillcolor{textcolor}%
\pgftext[x=0.431898in,y=0.499284in,left,base]{\color{textcolor}\sffamily\fontsize{10.000000}{12.000000}\selectfont 6.5}%
\end{pgfscope}%
\begin{pgfscope}%
\pgfsetbuttcap%
\pgfsetroundjoin%
\definecolor{currentfill}{rgb}{0.000000,0.000000,0.000000}%
\pgfsetfillcolor{currentfill}%
\pgfsetlinewidth{0.803000pt}%
\definecolor{currentstroke}{rgb}{0.000000,0.000000,0.000000}%
\pgfsetstrokecolor{currentstroke}%
\pgfsetdash{}{0pt}%
\pgfsys@defobject{currentmarker}{\pgfqpoint{-0.048611in}{0.000000in}}{\pgfqpoint{0.000000in}{0.000000in}}{%
\pgfpathmoveto{\pgfqpoint{0.000000in}{0.000000in}}%
\pgfpathlineto{\pgfqpoint{-0.048611in}{0.000000in}}%
\pgfusepath{stroke,fill}%
}%
\begin{pgfscope}%
\pgfsys@transformshift{0.750000in}{1.066191in}%
\pgfsys@useobject{currentmarker}{}%
\end{pgfscope}%
\end{pgfscope}%
\begin{pgfscope}%
\definecolor{textcolor}{rgb}{0.000000,0.000000,0.000000}%
\pgfsetstrokecolor{textcolor}%
\pgfsetfillcolor{textcolor}%
\pgftext[x=0.431898in,y=1.013430in,left,base]{\color{textcolor}\sffamily\fontsize{10.000000}{12.000000}\selectfont 7.0}%
\end{pgfscope}%
\begin{pgfscope}%
\pgfsetbuttcap%
\pgfsetroundjoin%
\definecolor{currentfill}{rgb}{0.000000,0.000000,0.000000}%
\pgfsetfillcolor{currentfill}%
\pgfsetlinewidth{0.803000pt}%
\definecolor{currentstroke}{rgb}{0.000000,0.000000,0.000000}%
\pgfsetstrokecolor{currentstroke}%
\pgfsetdash{}{0pt}%
\pgfsys@defobject{currentmarker}{\pgfqpoint{-0.048611in}{0.000000in}}{\pgfqpoint{0.000000in}{0.000000in}}{%
\pgfpathmoveto{\pgfqpoint{0.000000in}{0.000000in}}%
\pgfpathlineto{\pgfqpoint{-0.048611in}{0.000000in}}%
\pgfusepath{stroke,fill}%
}%
\begin{pgfscope}%
\pgfsys@transformshift{0.750000in}{1.580342in}%
\pgfsys@useobject{currentmarker}{}%
\end{pgfscope}%
\end{pgfscope}%
\begin{pgfscope}%
\definecolor{textcolor}{rgb}{0.000000,0.000000,0.000000}%
\pgfsetstrokecolor{textcolor}%
\pgfsetfillcolor{textcolor}%
\pgftext[x=0.431898in,y=1.527580in,left,base]{\color{textcolor}\sffamily\fontsize{10.000000}{12.000000}\selectfont 7.5}%
\end{pgfscope}%
\begin{pgfscope}%
\pgfsetbuttcap%
\pgfsetroundjoin%
\definecolor{currentfill}{rgb}{0.000000,0.000000,0.000000}%
\pgfsetfillcolor{currentfill}%
\pgfsetlinewidth{0.803000pt}%
\definecolor{currentstroke}{rgb}{0.000000,0.000000,0.000000}%
\pgfsetstrokecolor{currentstroke}%
\pgfsetdash{}{0pt}%
\pgfsys@defobject{currentmarker}{\pgfqpoint{-0.048611in}{0.000000in}}{\pgfqpoint{0.000000in}{0.000000in}}{%
\pgfpathmoveto{\pgfqpoint{0.000000in}{0.000000in}}%
\pgfpathlineto{\pgfqpoint{-0.048611in}{0.000000in}}%
\pgfusepath{stroke,fill}%
}%
\begin{pgfscope}%
\pgfsys@transformshift{0.750000in}{2.094492in}%
\pgfsys@useobject{currentmarker}{}%
\end{pgfscope}%
\end{pgfscope}%
\begin{pgfscope}%
\definecolor{textcolor}{rgb}{0.000000,0.000000,0.000000}%
\pgfsetstrokecolor{textcolor}%
\pgfsetfillcolor{textcolor}%
\pgftext[x=0.431898in,y=2.041731in,left,base]{\color{textcolor}\sffamily\fontsize{10.000000}{12.000000}\selectfont 8.0}%
\end{pgfscope}%
\begin{pgfscope}%
\definecolor{textcolor}{rgb}{0.000000,0.000000,0.000000}%
\pgfsetstrokecolor{textcolor}%
\pgfsetfillcolor{textcolor}%
\pgftext[x=0.191936in,y=0.325194in,left,base,rotate=90.000000]{\color{textcolor}\sffamily\fontsize{10.000000}{12.000000}\selectfont Probability that number of}%
\end{pgfscope}%
\begin{pgfscope}%
\definecolor{textcolor}{rgb}{0.000000,0.000000,0.000000}%
\pgfsetstrokecolor{textcolor}%
\pgfsetfillcolor{textcolor}%
\pgftext[x=0.347453in,y=0.160467in,left,base,rotate=90.000000]{\color{textcolor}\sffamily\fontsize{10.000000}{12.000000}\selectfont GMRES iterations is at most 12}%
\end{pgfscope}%
\begin{pgfscope}%
\definecolor{textcolor}{rgb}{0.000000,0.000000,0.000000}%
\pgfsetstrokecolor{textcolor}%
\pgfsetfillcolor{textcolor}%
\pgftext[x=0.750000in,y=2.241667in,left,base]{\color{textcolor}\sffamily\fontsize{10.000000}{12.000000}\selectfont 1e−11+2.983096487e−1}%
\end{pgfscope}%
\begin{pgfscope}%
\pgfpathrectangle{\pgfqpoint{0.750000in}{0.275000in}}{\pgfqpoint{4.650000in}{1.925000in}}%
\pgfusepath{clip}%
\pgfsetbuttcap%
\pgfsetroundjoin%
\definecolor{currentfill}{rgb}{0.000000,0.000000,0.000000}%
\pgfsetfillcolor{currentfill}%
\pgfsetlinewidth{1.003750pt}%
\definecolor{currentstroke}{rgb}{0.000000,0.000000,0.000000}%
\pgfsetstrokecolor{currentstroke}%
\pgfsetdash{}{0pt}%
\pgfsys@defobject{currentmarker}{\pgfqpoint{-0.020833in}{-0.020833in}}{\pgfqpoint{0.020833in}{0.020833in}}{%
\pgfpathmoveto{\pgfqpoint{0.000000in}{-0.020833in}}%
\pgfpathcurveto{\pgfqpoint{0.005525in}{-0.020833in}}{\pgfqpoint{0.010825in}{-0.018638in}}{\pgfqpoint{0.014731in}{-0.014731in}}%
\pgfpathcurveto{\pgfqpoint{0.018638in}{-0.010825in}}{\pgfqpoint{0.020833in}{-0.005525in}}{\pgfqpoint{0.020833in}{0.000000in}}%
\pgfpathcurveto{\pgfqpoint{0.020833in}{0.005525in}}{\pgfqpoint{0.018638in}{0.010825in}}{\pgfqpoint{0.014731in}{0.014731in}}%
\pgfpathcurveto{\pgfqpoint{0.010825in}{0.018638in}}{\pgfqpoint{0.005525in}{0.020833in}}{\pgfqpoint{0.000000in}{0.020833in}}%
\pgfpathcurveto{\pgfqpoint{-0.005525in}{0.020833in}}{\pgfqpoint{-0.010825in}{0.018638in}}{\pgfqpoint{-0.014731in}{0.014731in}}%
\pgfpathcurveto{\pgfqpoint{-0.018638in}{0.010825in}}{\pgfqpoint{-0.020833in}{0.005525in}}{\pgfqpoint{-0.020833in}{0.000000in}}%
\pgfpathcurveto{\pgfqpoint{-0.020833in}{-0.005525in}}{\pgfqpoint{-0.018638in}{-0.010825in}}{\pgfqpoint{-0.014731in}{-0.014731in}}%
\pgfpathcurveto{\pgfqpoint{-0.010825in}{-0.018638in}}{\pgfqpoint{-0.005525in}{-0.020833in}}{\pgfqpoint{0.000000in}{-0.020833in}}%
\pgfpathclose%
\pgfusepath{stroke,fill}%
}%
\begin{pgfscope}%
\pgfsys@transformshift{0.961364in}{1.237490in}%
\pgfsys@useobject{currentmarker}{}%
\end{pgfscope}%
\begin{pgfscope}%
\pgfsys@transformshift{1.003636in}{1.237485in}%
\pgfsys@useobject{currentmarker}{}%
\end{pgfscope}%
\begin{pgfscope}%
\pgfsys@transformshift{1.045909in}{1.237490in}%
\pgfsys@useobject{currentmarker}{}%
\end{pgfscope}%
\begin{pgfscope}%
\pgfsys@transformshift{1.088182in}{1.237495in}%
\pgfsys@useobject{currentmarker}{}%
\end{pgfscope}%
\begin{pgfscope}%
\pgfsys@transformshift{1.130455in}{1.237485in}%
\pgfsys@useobject{currentmarker}{}%
\end{pgfscope}%
\begin{pgfscope}%
\pgfsys@transformshift{1.172727in}{1.237480in}%
\pgfsys@useobject{currentmarker}{}%
\end{pgfscope}%
\begin{pgfscope}%
\pgfsys@transformshift{1.215000in}{1.237490in}%
\pgfsys@useobject{currentmarker}{}%
\end{pgfscope}%
\begin{pgfscope}%
\pgfsys@transformshift{1.257273in}{1.237485in}%
\pgfsys@useobject{currentmarker}{}%
\end{pgfscope}%
\begin{pgfscope}%
\pgfsys@transformshift{1.299545in}{2.112480in}%
\pgfsys@useobject{currentmarker}{}%
\end{pgfscope}%
\begin{pgfscope}%
\pgfsys@transformshift{1.341818in}{2.112480in}%
\pgfsys@useobject{currentmarker}{}%
\end{pgfscope}%
\begin{pgfscope}%
\pgfsys@transformshift{1.384091in}{2.112495in}%
\pgfsys@useobject{currentmarker}{}%
\end{pgfscope}%
\begin{pgfscope}%
\pgfsys@transformshift{1.426364in}{2.112490in}%
\pgfsys@useobject{currentmarker}{}%
\end{pgfscope}%
\begin{pgfscope}%
\pgfsys@transformshift{1.468636in}{2.112490in}%
\pgfsys@useobject{currentmarker}{}%
\end{pgfscope}%
\begin{pgfscope}%
\pgfsys@transformshift{1.510909in}{2.112485in}%
\pgfsys@useobject{currentmarker}{}%
\end{pgfscope}%
\begin{pgfscope}%
\pgfsys@transformshift{1.553182in}{2.112485in}%
\pgfsys@useobject{currentmarker}{}%
\end{pgfscope}%
\begin{pgfscope}%
\pgfsys@transformshift{1.595455in}{2.112495in}%
\pgfsys@useobject{currentmarker}{}%
\end{pgfscope}%
\begin{pgfscope}%
\pgfsys@transformshift{1.637727in}{2.112495in}%
\pgfsys@useobject{currentmarker}{}%
\end{pgfscope}%
\begin{pgfscope}%
\pgfsys@transformshift{1.680000in}{2.112480in}%
\pgfsys@useobject{currentmarker}{}%
\end{pgfscope}%
\begin{pgfscope}%
\pgfsys@transformshift{1.722273in}{2.112490in}%
\pgfsys@useobject{currentmarker}{}%
\end{pgfscope}%
\begin{pgfscope}%
\pgfsys@transformshift{1.764545in}{2.112485in}%
\pgfsys@useobject{currentmarker}{}%
\end{pgfscope}%
\begin{pgfscope}%
\pgfsys@transformshift{1.806818in}{2.112490in}%
\pgfsys@useobject{currentmarker}{}%
\end{pgfscope}%
\begin{pgfscope}%
\pgfsys@transformshift{1.849091in}{2.112480in}%
\pgfsys@useobject{currentmarker}{}%
\end{pgfscope}%
\begin{pgfscope}%
\pgfsys@transformshift{1.891364in}{2.112485in}%
\pgfsys@useobject{currentmarker}{}%
\end{pgfscope}%
\begin{pgfscope}%
\pgfsys@transformshift{1.933636in}{2.112485in}%
\pgfsys@useobject{currentmarker}{}%
\end{pgfscope}%
\begin{pgfscope}%
\pgfsys@transformshift{1.975909in}{2.112490in}%
\pgfsys@useobject{currentmarker}{}%
\end{pgfscope}%
\begin{pgfscope}%
\pgfsys@transformshift{2.018182in}{2.112485in}%
\pgfsys@useobject{currentmarker}{}%
\end{pgfscope}%
\begin{pgfscope}%
\pgfsys@transformshift{2.060455in}{2.112485in}%
\pgfsys@useobject{currentmarker}{}%
\end{pgfscope}%
\begin{pgfscope}%
\pgfsys@transformshift{2.102727in}{2.112480in}%
\pgfsys@useobject{currentmarker}{}%
\end{pgfscope}%
\begin{pgfscope}%
\pgfsys@transformshift{2.145000in}{2.112490in}%
\pgfsys@useobject{currentmarker}{}%
\end{pgfscope}%
\begin{pgfscope}%
\pgfsys@transformshift{2.187273in}{2.112495in}%
\pgfsys@useobject{currentmarker}{}%
\end{pgfscope}%
\begin{pgfscope}%
\pgfsys@transformshift{2.229545in}{2.112480in}%
\pgfsys@useobject{currentmarker}{}%
\end{pgfscope}%
\begin{pgfscope}%
\pgfsys@transformshift{2.271818in}{2.112485in}%
\pgfsys@useobject{currentmarker}{}%
\end{pgfscope}%
\begin{pgfscope}%
\pgfsys@transformshift{2.314091in}{2.112471in}%
\pgfsys@useobject{currentmarker}{}%
\end{pgfscope}%
\begin{pgfscope}%
\pgfsys@transformshift{2.356364in}{2.112485in}%
\pgfsys@useobject{currentmarker}{}%
\end{pgfscope}%
\begin{pgfscope}%
\pgfsys@transformshift{2.398636in}{2.112495in}%
\pgfsys@useobject{currentmarker}{}%
\end{pgfscope}%
\begin{pgfscope}%
\pgfsys@transformshift{2.440909in}{2.112490in}%
\pgfsys@useobject{currentmarker}{}%
\end{pgfscope}%
\begin{pgfscope}%
\pgfsys@transformshift{2.483182in}{2.112485in}%
\pgfsys@useobject{currentmarker}{}%
\end{pgfscope}%
\begin{pgfscope}%
\pgfsys@transformshift{2.525455in}{2.112485in}%
\pgfsys@useobject{currentmarker}{}%
\end{pgfscope}%
\begin{pgfscope}%
\pgfsys@transformshift{2.567727in}{2.112471in}%
\pgfsys@useobject{currentmarker}{}%
\end{pgfscope}%
\begin{pgfscope}%
\pgfsys@transformshift{2.610000in}{2.112490in}%
\pgfsys@useobject{currentmarker}{}%
\end{pgfscope}%
\begin{pgfscope}%
\pgfsys@transformshift{2.652273in}{2.112480in}%
\pgfsys@useobject{currentmarker}{}%
\end{pgfscope}%
\begin{pgfscope}%
\pgfsys@transformshift{2.694545in}{2.112485in}%
\pgfsys@useobject{currentmarker}{}%
\end{pgfscope}%
\begin{pgfscope}%
\pgfsys@transformshift{2.736818in}{2.112485in}%
\pgfsys@useobject{currentmarker}{}%
\end{pgfscope}%
\begin{pgfscope}%
\pgfsys@transformshift{2.779091in}{2.112495in}%
\pgfsys@useobject{currentmarker}{}%
\end{pgfscope}%
\begin{pgfscope}%
\pgfsys@transformshift{2.821364in}{2.112490in}%
\pgfsys@useobject{currentmarker}{}%
\end{pgfscope}%
\begin{pgfscope}%
\pgfsys@transformshift{2.863636in}{2.112490in}%
\pgfsys@useobject{currentmarker}{}%
\end{pgfscope}%
\begin{pgfscope}%
\pgfsys@transformshift{2.905909in}{2.112490in}%
\pgfsys@useobject{currentmarker}{}%
\end{pgfscope}%
\begin{pgfscope}%
\pgfsys@transformshift{2.948182in}{2.112490in}%
\pgfsys@useobject{currentmarker}{}%
\end{pgfscope}%
\begin{pgfscope}%
\pgfsys@transformshift{2.990455in}{0.362505in}%
\pgfsys@useobject{currentmarker}{}%
\end{pgfscope}%
\begin{pgfscope}%
\pgfsys@transformshift{3.032727in}{0.362510in}%
\pgfsys@useobject{currentmarker}{}%
\end{pgfscope}%
\begin{pgfscope}%
\pgfsys@transformshift{3.075000in}{0.362505in}%
\pgfsys@useobject{currentmarker}{}%
\end{pgfscope}%
\begin{pgfscope}%
\pgfsys@transformshift{3.117273in}{0.362510in}%
\pgfsys@useobject{currentmarker}{}%
\end{pgfscope}%
\begin{pgfscope}%
\pgfsys@transformshift{3.159545in}{0.362510in}%
\pgfsys@useobject{currentmarker}{}%
\end{pgfscope}%
\begin{pgfscope}%
\pgfsys@transformshift{3.201818in}{0.362510in}%
\pgfsys@useobject{currentmarker}{}%
\end{pgfscope}%
\begin{pgfscope}%
\pgfsys@transformshift{3.244091in}{0.362505in}%
\pgfsys@useobject{currentmarker}{}%
\end{pgfscope}%
\begin{pgfscope}%
\pgfsys@transformshift{3.286364in}{0.362505in}%
\pgfsys@useobject{currentmarker}{}%
\end{pgfscope}%
\begin{pgfscope}%
\pgfsys@transformshift{3.328636in}{0.362510in}%
\pgfsys@useobject{currentmarker}{}%
\end{pgfscope}%
\begin{pgfscope}%
\pgfsys@transformshift{3.370909in}{0.362510in}%
\pgfsys@useobject{currentmarker}{}%
\end{pgfscope}%
\begin{pgfscope}%
\pgfsys@transformshift{3.413182in}{0.362510in}%
\pgfsys@useobject{currentmarker}{}%
\end{pgfscope}%
\begin{pgfscope}%
\pgfsys@transformshift{3.455455in}{0.362510in}%
\pgfsys@useobject{currentmarker}{}%
\end{pgfscope}%
\begin{pgfscope}%
\pgfsys@transformshift{3.497727in}{0.362505in}%
\pgfsys@useobject{currentmarker}{}%
\end{pgfscope}%
\begin{pgfscope}%
\pgfsys@transformshift{3.540000in}{0.362505in}%
\pgfsys@useobject{currentmarker}{}%
\end{pgfscope}%
\begin{pgfscope}%
\pgfsys@transformshift{3.582273in}{0.362505in}%
\pgfsys@useobject{currentmarker}{}%
\end{pgfscope}%
\begin{pgfscope}%
\pgfsys@transformshift{3.624545in}{0.362510in}%
\pgfsys@useobject{currentmarker}{}%
\end{pgfscope}%
\begin{pgfscope}%
\pgfsys@transformshift{3.666818in}{0.362510in}%
\pgfsys@useobject{currentmarker}{}%
\end{pgfscope}%
\begin{pgfscope}%
\pgfsys@transformshift{3.709091in}{0.362505in}%
\pgfsys@useobject{currentmarker}{}%
\end{pgfscope}%
\begin{pgfscope}%
\pgfsys@transformshift{3.751364in}{0.362510in}%
\pgfsys@useobject{currentmarker}{}%
\end{pgfscope}%
\begin{pgfscope}%
\pgfsys@transformshift{3.793636in}{0.362510in}%
\pgfsys@useobject{currentmarker}{}%
\end{pgfscope}%
\begin{pgfscope}%
\pgfsys@transformshift{3.835909in}{0.362510in}%
\pgfsys@useobject{currentmarker}{}%
\end{pgfscope}%
\begin{pgfscope}%
\pgfsys@transformshift{3.878182in}{0.362510in}%
\pgfsys@useobject{currentmarker}{}%
\end{pgfscope}%
\begin{pgfscope}%
\pgfsys@transformshift{3.920455in}{0.362510in}%
\pgfsys@useobject{currentmarker}{}%
\end{pgfscope}%
\begin{pgfscope}%
\pgfsys@transformshift{3.962727in}{0.362510in}%
\pgfsys@useobject{currentmarker}{}%
\end{pgfscope}%
\begin{pgfscope}%
\pgfsys@transformshift{4.005000in}{0.362505in}%
\pgfsys@useobject{currentmarker}{}%
\end{pgfscope}%
\begin{pgfscope}%
\pgfsys@transformshift{4.047273in}{0.362510in}%
\pgfsys@useobject{currentmarker}{}%
\end{pgfscope}%
\begin{pgfscope}%
\pgfsys@transformshift{4.089545in}{0.362505in}%
\pgfsys@useobject{currentmarker}{}%
\end{pgfscope}%
\begin{pgfscope}%
\pgfsys@transformshift{4.131818in}{0.362510in}%
\pgfsys@useobject{currentmarker}{}%
\end{pgfscope}%
\begin{pgfscope}%
\pgfsys@transformshift{4.174091in}{0.362510in}%
\pgfsys@useobject{currentmarker}{}%
\end{pgfscope}%
\begin{pgfscope}%
\pgfsys@transformshift{4.216364in}{0.362510in}%
\pgfsys@useobject{currentmarker}{}%
\end{pgfscope}%
\begin{pgfscope}%
\pgfsys@transformshift{4.258636in}{0.362510in}%
\pgfsys@useobject{currentmarker}{}%
\end{pgfscope}%
\begin{pgfscope}%
\pgfsys@transformshift{4.300909in}{0.362505in}%
\pgfsys@useobject{currentmarker}{}%
\end{pgfscope}%
\begin{pgfscope}%
\pgfsys@transformshift{4.343182in}{0.362510in}%
\pgfsys@useobject{currentmarker}{}%
\end{pgfscope}%
\begin{pgfscope}%
\pgfsys@transformshift{4.385455in}{0.362505in}%
\pgfsys@useobject{currentmarker}{}%
\end{pgfscope}%
\begin{pgfscope}%
\pgfsys@transformshift{4.427727in}{0.362505in}%
\pgfsys@useobject{currentmarker}{}%
\end{pgfscope}%
\begin{pgfscope}%
\pgfsys@transformshift{4.470000in}{0.362505in}%
\pgfsys@useobject{currentmarker}{}%
\end{pgfscope}%
\begin{pgfscope}%
\pgfsys@transformshift{4.512273in}{0.362505in}%
\pgfsys@useobject{currentmarker}{}%
\end{pgfscope}%
\begin{pgfscope}%
\pgfsys@transformshift{4.554545in}{0.362505in}%
\pgfsys@useobject{currentmarker}{}%
\end{pgfscope}%
\begin{pgfscope}%
\pgfsys@transformshift{4.596818in}{0.362505in}%
\pgfsys@useobject{currentmarker}{}%
\end{pgfscope}%
\begin{pgfscope}%
\pgfsys@transformshift{4.639091in}{0.362505in}%
\pgfsys@useobject{currentmarker}{}%
\end{pgfscope}%
\begin{pgfscope}%
\pgfsys@transformshift{4.681364in}{0.362505in}%
\pgfsys@useobject{currentmarker}{}%
\end{pgfscope}%
\begin{pgfscope}%
\pgfsys@transformshift{4.723636in}{0.362510in}%
\pgfsys@useobject{currentmarker}{}%
\end{pgfscope}%
\begin{pgfscope}%
\pgfsys@transformshift{4.765909in}{0.362510in}%
\pgfsys@useobject{currentmarker}{}%
\end{pgfscope}%
\begin{pgfscope}%
\pgfsys@transformshift{4.808182in}{0.362510in}%
\pgfsys@useobject{currentmarker}{}%
\end{pgfscope}%
\begin{pgfscope}%
\pgfsys@transformshift{4.850455in}{0.362510in}%
\pgfsys@useobject{currentmarker}{}%
\end{pgfscope}%
\begin{pgfscope}%
\pgfsys@transformshift{4.892727in}{0.362505in}%
\pgfsys@useobject{currentmarker}{}%
\end{pgfscope}%
\begin{pgfscope}%
\pgfsys@transformshift{4.935000in}{0.362510in}%
\pgfsys@useobject{currentmarker}{}%
\end{pgfscope}%
\begin{pgfscope}%
\pgfsys@transformshift{4.977273in}{0.362500in}%
\pgfsys@useobject{currentmarker}{}%
\end{pgfscope}%
\begin{pgfscope}%
\pgfsys@transformshift{5.019545in}{0.362505in}%
\pgfsys@useobject{currentmarker}{}%
\end{pgfscope}%
\begin{pgfscope}%
\pgfsys@transformshift{5.061818in}{0.362510in}%
\pgfsys@useobject{currentmarker}{}%
\end{pgfscope}%
\begin{pgfscope}%
\pgfsys@transformshift{5.104091in}{0.362505in}%
\pgfsys@useobject{currentmarker}{}%
\end{pgfscope}%
\begin{pgfscope}%
\pgfsys@transformshift{5.146364in}{0.362505in}%
\pgfsys@useobject{currentmarker}{}%
\end{pgfscope}%
\begin{pgfscope}%
\pgfsys@transformshift{5.188636in}{0.362510in}%
\pgfsys@useobject{currentmarker}{}%
\end{pgfscope}%
\end{pgfscope}%
\begin{pgfscope}%
\pgfsetrectcap%
\pgfsetmiterjoin%
\pgfsetlinewidth{0.803000pt}%
\definecolor{currentstroke}{rgb}{0.000000,0.000000,0.000000}%
\pgfsetstrokecolor{currentstroke}%
\pgfsetdash{}{0pt}%
\pgfpathmoveto{\pgfqpoint{0.750000in}{0.274995in}}%
\pgfpathlineto{\pgfqpoint{0.750000in}{2.199995in}}%
\pgfusepath{stroke}%
\end{pgfscope}%
\begin{pgfscope}%
\pgfsetrectcap%
\pgfsetmiterjoin%
\pgfsetlinewidth{0.803000pt}%
\definecolor{currentstroke}{rgb}{0.000000,0.000000,0.000000}%
\pgfsetstrokecolor{currentstroke}%
\pgfsetdash{}{0pt}%
\pgfpathmoveto{\pgfqpoint{5.400000in}{0.274995in}}%
\pgfpathlineto{\pgfqpoint{5.400000in}{2.199995in}}%
\pgfusepath{stroke}%
\end{pgfscope}%
\begin{pgfscope}%
\pgfsetrectcap%
\pgfsetmiterjoin%
\pgfsetlinewidth{0.803000pt}%
\definecolor{currentstroke}{rgb}{0.000000,0.000000,0.000000}%
\pgfsetstrokecolor{currentstroke}%
\pgfsetdash{}{0pt}%
\pgfpathmoveto{\pgfqpoint{0.750000in}{0.275000in}}%
\pgfpathlineto{\pgfqpoint{5.400000in}{0.275000in}}%
\pgfusepath{stroke}%
\end{pgfscope}%
\begin{pgfscope}%
\pgfsetrectcap%
\pgfsetmiterjoin%
\pgfsetlinewidth{0.803000pt}%
\definecolor{currentstroke}{rgb}{0.000000,0.000000,0.000000}%
\pgfsetstrokecolor{currentstroke}%
\pgfsetdash{}{0pt}%
\pgfpathmoveto{\pgfqpoint{0.750000in}{2.200000in}}%
\pgfpathlineto{\pgfqpoint{5.400000in}{2.200000in}}%
\pgfusepath{stroke}%
\end{pgfscope}%
\end{pgfpicture}%
\makeatother%
\endgroup%

\caption{The lower bound in \cref{eq:GMRESprob} with $\NLiDRR{\no-\nt} \sim \Exp{\sigma}$ with $\sigma = 1/k.$\label{fig:prob-theory-plot-1.0}}
\end{subfigure}

\begin{subfigure}{\textwidth}
    \centering
    %% Creator: Matplotlib, PGF backend
%%
%% To include the figure in your LaTeX document, write
%%   \input{<filename>.pgf}
%%
%% Make sure the required packages are loaded in your preamble
%%   \usepackage{pgf}
%%
%% Figures using additional raster images can only be included by \input if
%% they are in the same directory as the main LaTeX file. For loading figures
%% from other directories you can use the `import` package
%%   \usepackage{import}
%% and then include the figures with
%%   \import{<path to file>}{<filename>.pgf}
%%
%% Matplotlib used the following preamble
%%   \usepackage{fontspec}
%%   \setmainfont{DejaVuSerif.ttf}[Path=/home/owen/progs/firedrake-complex/firedrake/lib/python3.5/site-packages/matplotlib/mpl-data/fonts/ttf/]
%%   \setsansfont{DejaVuSans.ttf}[Path=/home/owen/progs/firedrake-complex/firedrake/lib/python3.5/site-packages/matplotlib/mpl-data/fonts/ttf/]
%%   \setmonofont{DejaVuSansMono.ttf}[Path=/home/owen/progs/firedrake-complex/firedrake/lib/python3.5/site-packages/matplotlib/mpl-data/fonts/ttf/]
%%
\begingroup%
\makeatletter%
\begin{pgfpicture}%
\pgfpathrectangle{\pgfpointorigin}{\pgfqpoint{6.000000in}{2.500000in}}%
\pgfusepath{use as bounding box, clip}%
\begin{pgfscope}%
\pgfsetbuttcap%
\pgfsetmiterjoin%
\definecolor{currentfill}{rgb}{1.000000,1.000000,1.000000}%
\pgfsetfillcolor{currentfill}%
\pgfsetlinewidth{0.000000pt}%
\definecolor{currentstroke}{rgb}{1.000000,1.000000,1.000000}%
\pgfsetstrokecolor{currentstroke}%
\pgfsetdash{}{0pt}%
\pgfpathmoveto{\pgfqpoint{0.000000in}{0.000000in}}%
\pgfpathlineto{\pgfqpoint{6.000000in}{0.000000in}}%
\pgfpathlineto{\pgfqpoint{6.000000in}{2.500000in}}%
\pgfpathlineto{\pgfqpoint{0.000000in}{2.500000in}}%
\pgfpathclose%
\pgfusepath{fill}%
\end{pgfscope}%
\begin{pgfscope}%
\pgfsetbuttcap%
\pgfsetmiterjoin%
\definecolor{currentfill}{rgb}{1.000000,1.000000,1.000000}%
\pgfsetfillcolor{currentfill}%
\pgfsetlinewidth{0.000000pt}%
\definecolor{currentstroke}{rgb}{0.000000,0.000000,0.000000}%
\pgfsetstrokecolor{currentstroke}%
\pgfsetstrokeopacity{0.000000}%
\pgfsetdash{}{0pt}%
\pgfpathmoveto{\pgfqpoint{0.750000in}{0.275000in}}%
\pgfpathlineto{\pgfqpoint{5.400000in}{0.275000in}}%
\pgfpathlineto{\pgfqpoint{5.400000in}{2.200000in}}%
\pgfpathlineto{\pgfqpoint{0.750000in}{2.200000in}}%
\pgfpathclose%
\pgfusepath{fill}%
\end{pgfscope}%
\begin{pgfscope}%
\pgfsetbuttcap%
\pgfsetroundjoin%
\definecolor{currentfill}{rgb}{0.000000,0.000000,0.000000}%
\pgfsetfillcolor{currentfill}%
\pgfsetlinewidth{0.803000pt}%
\definecolor{currentstroke}{rgb}{0.000000,0.000000,0.000000}%
\pgfsetstrokecolor{currentstroke}%
\pgfsetdash{}{0pt}%
\pgfsys@defobject{currentmarker}{\pgfqpoint{0.000000in}{-0.048611in}}{\pgfqpoint{0.000000in}{0.000000in}}{%
\pgfpathmoveto{\pgfqpoint{0.000000in}{0.000000in}}%
\pgfpathlineto{\pgfqpoint{0.000000in}{-0.048611in}}%
\pgfusepath{stroke,fill}%
}%
\begin{pgfscope}%
\pgfsys@transformshift{0.961364in}{0.275000in}%
\pgfsys@useobject{currentmarker}{}%
\end{pgfscope}%
\end{pgfscope}%
\begin{pgfscope}%
\definecolor{textcolor}{rgb}{0.000000,0.000000,0.000000}%
\pgfsetstrokecolor{textcolor}%
\pgfsetfillcolor{textcolor}%
\pgftext[x=0.961364in,y=0.177778in,,top]{\color{textcolor}\sffamily\fontsize{10.000000}{12.000000}\selectfont 10}%
\end{pgfscope}%
\begin{pgfscope}%
\pgfsetbuttcap%
\pgfsetroundjoin%
\definecolor{currentfill}{rgb}{0.000000,0.000000,0.000000}%
\pgfsetfillcolor{currentfill}%
\pgfsetlinewidth{0.803000pt}%
\definecolor{currentstroke}{rgb}{0.000000,0.000000,0.000000}%
\pgfsetstrokecolor{currentstroke}%
\pgfsetdash{}{0pt}%
\pgfsys@defobject{currentmarker}{\pgfqpoint{0.000000in}{-0.048611in}}{\pgfqpoint{0.000000in}{0.000000in}}{%
\pgfpathmoveto{\pgfqpoint{0.000000in}{0.000000in}}%
\pgfpathlineto{\pgfqpoint{0.000000in}{-0.048611in}}%
\pgfusepath{stroke,fill}%
}%
\begin{pgfscope}%
\pgfsys@transformshift{1.665909in}{0.275000in}%
\pgfsys@useobject{currentmarker}{}%
\end{pgfscope}%
\end{pgfscope}%
\begin{pgfscope}%
\definecolor{textcolor}{rgb}{0.000000,0.000000,0.000000}%
\pgfsetstrokecolor{textcolor}%
\pgfsetfillcolor{textcolor}%
\pgftext[x=1.665909in,y=0.177778in,,top]{\color{textcolor}\sffamily\fontsize{10.000000}{12.000000}\selectfont 15}%
\end{pgfscope}%
\begin{pgfscope}%
\pgfsetbuttcap%
\pgfsetroundjoin%
\definecolor{currentfill}{rgb}{0.000000,0.000000,0.000000}%
\pgfsetfillcolor{currentfill}%
\pgfsetlinewidth{0.803000pt}%
\definecolor{currentstroke}{rgb}{0.000000,0.000000,0.000000}%
\pgfsetstrokecolor{currentstroke}%
\pgfsetdash{}{0pt}%
\pgfsys@defobject{currentmarker}{\pgfqpoint{0.000000in}{-0.048611in}}{\pgfqpoint{0.000000in}{0.000000in}}{%
\pgfpathmoveto{\pgfqpoint{0.000000in}{0.000000in}}%
\pgfpathlineto{\pgfqpoint{0.000000in}{-0.048611in}}%
\pgfusepath{stroke,fill}%
}%
\begin{pgfscope}%
\pgfsys@transformshift{2.370455in}{0.275000in}%
\pgfsys@useobject{currentmarker}{}%
\end{pgfscope}%
\end{pgfscope}%
\begin{pgfscope}%
\definecolor{textcolor}{rgb}{0.000000,0.000000,0.000000}%
\pgfsetstrokecolor{textcolor}%
\pgfsetfillcolor{textcolor}%
\pgftext[x=2.370455in,y=0.177778in,,top]{\color{textcolor}\sffamily\fontsize{10.000000}{12.000000}\selectfont 20}%
\end{pgfscope}%
\begin{pgfscope}%
\pgfsetbuttcap%
\pgfsetroundjoin%
\definecolor{currentfill}{rgb}{0.000000,0.000000,0.000000}%
\pgfsetfillcolor{currentfill}%
\pgfsetlinewidth{0.803000pt}%
\definecolor{currentstroke}{rgb}{0.000000,0.000000,0.000000}%
\pgfsetstrokecolor{currentstroke}%
\pgfsetdash{}{0pt}%
\pgfsys@defobject{currentmarker}{\pgfqpoint{0.000000in}{-0.048611in}}{\pgfqpoint{0.000000in}{0.000000in}}{%
\pgfpathmoveto{\pgfqpoint{0.000000in}{0.000000in}}%
\pgfpathlineto{\pgfqpoint{0.000000in}{-0.048611in}}%
\pgfusepath{stroke,fill}%
}%
\begin{pgfscope}%
\pgfsys@transformshift{3.075000in}{0.275000in}%
\pgfsys@useobject{currentmarker}{}%
\end{pgfscope}%
\end{pgfscope}%
\begin{pgfscope}%
\definecolor{textcolor}{rgb}{0.000000,0.000000,0.000000}%
\pgfsetstrokecolor{textcolor}%
\pgfsetfillcolor{textcolor}%
\pgftext[x=3.075000in,y=0.177778in,,top]{\color{textcolor}\sffamily\fontsize{10.000000}{12.000000}\selectfont 25}%
\end{pgfscope}%
\begin{pgfscope}%
\pgfsetbuttcap%
\pgfsetroundjoin%
\definecolor{currentfill}{rgb}{0.000000,0.000000,0.000000}%
\pgfsetfillcolor{currentfill}%
\pgfsetlinewidth{0.803000pt}%
\definecolor{currentstroke}{rgb}{0.000000,0.000000,0.000000}%
\pgfsetstrokecolor{currentstroke}%
\pgfsetdash{}{0pt}%
\pgfsys@defobject{currentmarker}{\pgfqpoint{0.000000in}{-0.048611in}}{\pgfqpoint{0.000000in}{0.000000in}}{%
\pgfpathmoveto{\pgfqpoint{0.000000in}{0.000000in}}%
\pgfpathlineto{\pgfqpoint{0.000000in}{-0.048611in}}%
\pgfusepath{stroke,fill}%
}%
\begin{pgfscope}%
\pgfsys@transformshift{3.779545in}{0.275000in}%
\pgfsys@useobject{currentmarker}{}%
\end{pgfscope}%
\end{pgfscope}%
\begin{pgfscope}%
\definecolor{textcolor}{rgb}{0.000000,0.000000,0.000000}%
\pgfsetstrokecolor{textcolor}%
\pgfsetfillcolor{textcolor}%
\pgftext[x=3.779545in,y=0.177778in,,top]{\color{textcolor}\sffamily\fontsize{10.000000}{12.000000}\selectfont 30}%
\end{pgfscope}%
\begin{pgfscope}%
\pgfsetbuttcap%
\pgfsetroundjoin%
\definecolor{currentfill}{rgb}{0.000000,0.000000,0.000000}%
\pgfsetfillcolor{currentfill}%
\pgfsetlinewidth{0.803000pt}%
\definecolor{currentstroke}{rgb}{0.000000,0.000000,0.000000}%
\pgfsetstrokecolor{currentstroke}%
\pgfsetdash{}{0pt}%
\pgfsys@defobject{currentmarker}{\pgfqpoint{0.000000in}{-0.048611in}}{\pgfqpoint{0.000000in}{0.000000in}}{%
\pgfpathmoveto{\pgfqpoint{0.000000in}{0.000000in}}%
\pgfpathlineto{\pgfqpoint{0.000000in}{-0.048611in}}%
\pgfusepath{stroke,fill}%
}%
\begin{pgfscope}%
\pgfsys@transformshift{4.484091in}{0.275000in}%
\pgfsys@useobject{currentmarker}{}%
\end{pgfscope}%
\end{pgfscope}%
\begin{pgfscope}%
\definecolor{textcolor}{rgb}{0.000000,0.000000,0.000000}%
\pgfsetstrokecolor{textcolor}%
\pgfsetfillcolor{textcolor}%
\pgftext[x=4.484091in,y=0.177778in,,top]{\color{textcolor}\sffamily\fontsize{10.000000}{12.000000}\selectfont 35}%
\end{pgfscope}%
\begin{pgfscope}%
\pgfsetbuttcap%
\pgfsetroundjoin%
\definecolor{currentfill}{rgb}{0.000000,0.000000,0.000000}%
\pgfsetfillcolor{currentfill}%
\pgfsetlinewidth{0.803000pt}%
\definecolor{currentstroke}{rgb}{0.000000,0.000000,0.000000}%
\pgfsetstrokecolor{currentstroke}%
\pgfsetdash{}{0pt}%
\pgfsys@defobject{currentmarker}{\pgfqpoint{0.000000in}{-0.048611in}}{\pgfqpoint{0.000000in}{0.000000in}}{%
\pgfpathmoveto{\pgfqpoint{0.000000in}{0.000000in}}%
\pgfpathlineto{\pgfqpoint{0.000000in}{-0.048611in}}%
\pgfusepath{stroke,fill}%
}%
\begin{pgfscope}%
\pgfsys@transformshift{5.188636in}{0.275000in}%
\pgfsys@useobject{currentmarker}{}%
\end{pgfscope}%
\end{pgfscope}%
\begin{pgfscope}%
\definecolor{textcolor}{rgb}{0.000000,0.000000,0.000000}%
\pgfsetstrokecolor{textcolor}%
\pgfsetfillcolor{textcolor}%
\pgftext[x=5.188636in,y=0.177778in,,top]{\color{textcolor}\sffamily\fontsize{10.000000}{12.000000}\selectfont 40}%
\end{pgfscope}%
\begin{pgfscope}%
\definecolor{textcolor}{rgb}{0.000000,0.000000,0.000000}%
\pgfsetstrokecolor{textcolor}%
\pgfsetfillcolor{textcolor}%
\pgftext[x=3.075000in,y=-0.012191in,,top]{\color{textcolor}\sffamily\fontsize{10.000000}{12.000000}\selectfont \(\displaystyle k\)}%
\end{pgfscope}%
\begin{pgfscope}%
\pgfsetbuttcap%
\pgfsetroundjoin%
\definecolor{currentfill}{rgb}{0.000000,0.000000,0.000000}%
\pgfsetfillcolor{currentfill}%
\pgfsetlinewidth{0.803000pt}%
\definecolor{currentstroke}{rgb}{0.000000,0.000000,0.000000}%
\pgfsetstrokecolor{currentstroke}%
\pgfsetdash{}{0pt}%
\pgfsys@defobject{currentmarker}{\pgfqpoint{-0.048611in}{0.000000in}}{\pgfqpoint{0.000000in}{0.000000in}}{%
\pgfpathmoveto{\pgfqpoint{0.000000in}{0.000000in}}%
\pgfpathlineto{\pgfqpoint{-0.048611in}{0.000000in}}%
\pgfusepath{stroke,fill}%
}%
\begin{pgfscope}%
\pgfsys@transformshift{0.750000in}{0.298219in}%
\pgfsys@useobject{currentmarker}{}%
\end{pgfscope}%
\end{pgfscope}%
\begin{pgfscope}%
\definecolor{textcolor}{rgb}{0.000000,0.000000,0.000000}%
\pgfsetstrokecolor{textcolor}%
\pgfsetfillcolor{textcolor}%
\pgftext[x=0.343533in,y=0.245457in,left,base]{\color{textcolor}\sffamily\fontsize{10.000000}{12.000000}\selectfont 0.97}%
\end{pgfscope}%
\begin{pgfscope}%
\pgfsetbuttcap%
\pgfsetroundjoin%
\definecolor{currentfill}{rgb}{0.000000,0.000000,0.000000}%
\pgfsetfillcolor{currentfill}%
\pgfsetlinewidth{0.803000pt}%
\definecolor{currentstroke}{rgb}{0.000000,0.000000,0.000000}%
\pgfsetstrokecolor{currentstroke}%
\pgfsetdash{}{0pt}%
\pgfsys@defobject{currentmarker}{\pgfqpoint{-0.048611in}{0.000000in}}{\pgfqpoint{0.000000in}{0.000000in}}{%
\pgfpathmoveto{\pgfqpoint{0.000000in}{0.000000in}}%
\pgfpathlineto{\pgfqpoint{-0.048611in}{0.000000in}}%
\pgfusepath{stroke,fill}%
}%
\begin{pgfscope}%
\pgfsys@transformshift{0.750000in}{0.902993in}%
\pgfsys@useobject{currentmarker}{}%
\end{pgfscope}%
\end{pgfscope}%
\begin{pgfscope}%
\definecolor{textcolor}{rgb}{0.000000,0.000000,0.000000}%
\pgfsetstrokecolor{textcolor}%
\pgfsetfillcolor{textcolor}%
\pgftext[x=0.343533in,y=0.850232in,left,base]{\color{textcolor}\sffamily\fontsize{10.000000}{12.000000}\selectfont 0.98}%
\end{pgfscope}%
\begin{pgfscope}%
\pgfsetbuttcap%
\pgfsetroundjoin%
\definecolor{currentfill}{rgb}{0.000000,0.000000,0.000000}%
\pgfsetfillcolor{currentfill}%
\pgfsetlinewidth{0.803000pt}%
\definecolor{currentstroke}{rgb}{0.000000,0.000000,0.000000}%
\pgfsetstrokecolor{currentstroke}%
\pgfsetdash{}{0pt}%
\pgfsys@defobject{currentmarker}{\pgfqpoint{-0.048611in}{0.000000in}}{\pgfqpoint{0.000000in}{0.000000in}}{%
\pgfpathmoveto{\pgfqpoint{0.000000in}{0.000000in}}%
\pgfpathlineto{\pgfqpoint{-0.048611in}{0.000000in}}%
\pgfusepath{stroke,fill}%
}%
\begin{pgfscope}%
\pgfsys@transformshift{0.750000in}{1.507768in}%
\pgfsys@useobject{currentmarker}{}%
\end{pgfscope}%
\end{pgfscope}%
\begin{pgfscope}%
\definecolor{textcolor}{rgb}{0.000000,0.000000,0.000000}%
\pgfsetstrokecolor{textcolor}%
\pgfsetfillcolor{textcolor}%
\pgftext[x=0.343533in,y=1.455006in,left,base]{\color{textcolor}\sffamily\fontsize{10.000000}{12.000000}\selectfont 0.99}%
\end{pgfscope}%
\begin{pgfscope}%
\pgfsetbuttcap%
\pgfsetroundjoin%
\definecolor{currentfill}{rgb}{0.000000,0.000000,0.000000}%
\pgfsetfillcolor{currentfill}%
\pgfsetlinewidth{0.803000pt}%
\definecolor{currentstroke}{rgb}{0.000000,0.000000,0.000000}%
\pgfsetstrokecolor{currentstroke}%
\pgfsetdash{}{0pt}%
\pgfsys@defobject{currentmarker}{\pgfqpoint{-0.048611in}{0.000000in}}{\pgfqpoint{0.000000in}{0.000000in}}{%
\pgfpathmoveto{\pgfqpoint{0.000000in}{0.000000in}}%
\pgfpathlineto{\pgfqpoint{-0.048611in}{0.000000in}}%
\pgfusepath{stroke,fill}%
}%
\begin{pgfscope}%
\pgfsys@transformshift{0.750000in}{2.112542in}%
\pgfsys@useobject{currentmarker}{}%
\end{pgfscope}%
\end{pgfscope}%
\begin{pgfscope}%
\definecolor{textcolor}{rgb}{0.000000,0.000000,0.000000}%
\pgfsetstrokecolor{textcolor}%
\pgfsetfillcolor{textcolor}%
\pgftext[x=0.343533in,y=2.059781in,left,base]{\color{textcolor}\sffamily\fontsize{10.000000}{12.000000}\selectfont 1.00}%
\end{pgfscope}%
\begin{pgfscope}%
\definecolor{textcolor}{rgb}{0.000000,0.000000,0.000000}%
\pgfsetstrokecolor{textcolor}%
\pgfsetfillcolor{textcolor}%
\pgftext[x=0.103570in,y=0.325194in,left,base,rotate=90.000000]{\color{textcolor}\sffamily\fontsize{10.000000}{12.000000}\selectfont Probability that number of}%
\end{pgfscope}%
\begin{pgfscope}%
\definecolor{textcolor}{rgb}{0.000000,0.000000,0.000000}%
\pgfsetstrokecolor{textcolor}%
\pgfsetfillcolor{textcolor}%
\pgftext[x=0.259088in,y=0.160467in,left,base,rotate=90.000000]{\color{textcolor}\sffamily\fontsize{10.000000}{12.000000}\selectfont GMRES iterations is at most 12}%
\end{pgfscope}%
\begin{pgfscope}%
\pgfpathrectangle{\pgfqpoint{0.750000in}{0.275000in}}{\pgfqpoint{4.650000in}{1.925000in}}%
\pgfusepath{clip}%
\pgfsetbuttcap%
\pgfsetroundjoin%
\definecolor{currentfill}{rgb}{0.000000,0.000000,0.000000}%
\pgfsetfillcolor{currentfill}%
\pgfsetlinewidth{1.003750pt}%
\definecolor{currentstroke}{rgb}{0.000000,0.000000,0.000000}%
\pgfsetstrokecolor{currentstroke}%
\pgfsetdash{}{0pt}%
\pgfsys@defobject{currentmarker}{\pgfqpoint{-0.020833in}{-0.020833in}}{\pgfqpoint{0.020833in}{0.020833in}}{%
\pgfpathmoveto{\pgfqpoint{0.000000in}{-0.020833in}}%
\pgfpathcurveto{\pgfqpoint{0.005525in}{-0.020833in}}{\pgfqpoint{0.010825in}{-0.018638in}}{\pgfqpoint{0.014731in}{-0.014731in}}%
\pgfpathcurveto{\pgfqpoint{0.018638in}{-0.010825in}}{\pgfqpoint{0.020833in}{-0.005525in}}{\pgfqpoint{0.020833in}{0.000000in}}%
\pgfpathcurveto{\pgfqpoint{0.020833in}{0.005525in}}{\pgfqpoint{0.018638in}{0.010825in}}{\pgfqpoint{0.014731in}{0.014731in}}%
\pgfpathcurveto{\pgfqpoint{0.010825in}{0.018638in}}{\pgfqpoint{0.005525in}{0.020833in}}{\pgfqpoint{0.000000in}{0.020833in}}%
\pgfpathcurveto{\pgfqpoint{-0.005525in}{0.020833in}}{\pgfqpoint{-0.010825in}{0.018638in}}{\pgfqpoint{-0.014731in}{0.014731in}}%
\pgfpathcurveto{\pgfqpoint{-0.018638in}{0.010825in}}{\pgfqpoint{-0.020833in}{0.005525in}}{\pgfqpoint{-0.020833in}{0.000000in}}%
\pgfpathcurveto{\pgfqpoint{-0.020833in}{-0.005525in}}{\pgfqpoint{-0.018638in}{-0.010825in}}{\pgfqpoint{-0.014731in}{-0.014731in}}%
\pgfpathcurveto{\pgfqpoint{-0.010825in}{-0.018638in}}{\pgfqpoint{-0.005525in}{-0.020833in}}{\pgfqpoint{0.000000in}{-0.020833in}}%
\pgfpathclose%
\pgfusepath{stroke,fill}%
}%
\begin{pgfscope}%
\pgfsys@transformshift{0.961364in}{0.362500in}%
\pgfsys@useobject{currentmarker}{}%
\end{pgfscope}%
\begin{pgfscope}%
\pgfsys@transformshift{1.003636in}{0.538950in}%
\pgfsys@useobject{currentmarker}{}%
\end{pgfscope}%
\begin{pgfscope}%
\pgfsys@transformshift{1.045909in}{0.697609in}%
\pgfsys@useobject{currentmarker}{}%
\end{pgfscope}%
\begin{pgfscope}%
\pgfsys@transformshift{1.088182in}{0.840272in}%
\pgfsys@useobject{currentmarker}{}%
\end{pgfscope}%
\begin{pgfscope}%
\pgfsys@transformshift{1.130455in}{0.968550in}%
\pgfsys@useobject{currentmarker}{}%
\end{pgfscope}%
\begin{pgfscope}%
\pgfsys@transformshift{1.172727in}{1.083894in}%
\pgfsys@useobject{currentmarker}{}%
\end{pgfscope}%
\begin{pgfscope}%
\pgfsys@transformshift{1.215000in}{1.187609in}%
\pgfsys@useobject{currentmarker}{}%
\end{pgfscope}%
\begin{pgfscope}%
\pgfsys@transformshift{1.257273in}{1.280866in}%
\pgfsys@useobject{currentmarker}{}%
\end{pgfscope}%
\begin{pgfscope}%
\pgfsys@transformshift{1.299545in}{1.364721in}%
\pgfsys@useobject{currentmarker}{}%
\end{pgfscope}%
\begin{pgfscope}%
\pgfsys@transformshift{1.341818in}{1.440121in}%
\pgfsys@useobject{currentmarker}{}%
\end{pgfscope}%
\begin{pgfscope}%
\pgfsys@transformshift{1.384091in}{1.507919in}%
\pgfsys@useobject{currentmarker}{}%
\end{pgfscope}%
\begin{pgfscope}%
\pgfsys@transformshift{1.426364in}{1.568881in}%
\pgfsys@useobject{currentmarker}{}%
\end{pgfscope}%
\begin{pgfscope}%
\pgfsys@transformshift{1.468636in}{1.623696in}%
\pgfsys@useobject{currentmarker}{}%
\end{pgfscope}%
\begin{pgfscope}%
\pgfsys@transformshift{1.510909in}{1.672985in}%
\pgfsys@useobject{currentmarker}{}%
\end{pgfscope}%
\begin{pgfscope}%
\pgfsys@transformshift{1.553182in}{1.717303in}%
\pgfsys@useobject{currentmarker}{}%
\end{pgfscope}%
\begin{pgfscope}%
\pgfsys@transformshift{1.595455in}{1.757154in}%
\pgfsys@useobject{currentmarker}{}%
\end{pgfscope}%
\begin{pgfscope}%
\pgfsys@transformshift{1.637727in}{1.792986in}%
\pgfsys@useobject{currentmarker}{}%
\end{pgfscope}%
\begin{pgfscope}%
\pgfsys@transformshift{1.680000in}{1.825206in}%
\pgfsys@useobject{currentmarker}{}%
\end{pgfscope}%
\begin{pgfscope}%
\pgfsys@transformshift{1.722273in}{1.854177in}%
\pgfsys@useobject{currentmarker}{}%
\end{pgfscope}%
\begin{pgfscope}%
\pgfsys@transformshift{1.764545in}{1.880227in}%
\pgfsys@useobject{currentmarker}{}%
\end{pgfscope}%
\begin{pgfscope}%
\pgfsys@transformshift{1.806818in}{1.903651in}%
\pgfsys@useobject{currentmarker}{}%
\end{pgfscope}%
\begin{pgfscope}%
\pgfsys@transformshift{1.849091in}{1.924712in}%
\pgfsys@useobject{currentmarker}{}%
\end{pgfscope}%
\begin{pgfscope}%
\pgfsys@transformshift{1.891364in}{1.943650in}%
\pgfsys@useobject{currentmarker}{}%
\end{pgfscope}%
\begin{pgfscope}%
\pgfsys@transformshift{1.933636in}{1.960679in}%
\pgfsys@useobject{currentmarker}{}%
\end{pgfscope}%
\begin{pgfscope}%
\pgfsys@transformshift{1.975909in}{1.975991in}%
\pgfsys@useobject{currentmarker}{}%
\end{pgfscope}%
\begin{pgfscope}%
\pgfsys@transformshift{2.018182in}{1.989759in}%
\pgfsys@useobject{currentmarker}{}%
\end{pgfscope}%
\begin{pgfscope}%
\pgfsys@transformshift{2.060455in}{2.002139in}%
\pgfsys@useobject{currentmarker}{}%
\end{pgfscope}%
\begin{pgfscope}%
\pgfsys@transformshift{2.102727in}{2.013270in}%
\pgfsys@useobject{currentmarker}{}%
\end{pgfscope}%
\begin{pgfscope}%
\pgfsys@transformshift{2.145000in}{2.023280in}%
\pgfsys@useobject{currentmarker}{}%
\end{pgfscope}%
\begin{pgfscope}%
\pgfsys@transformshift{2.187273in}{2.032280in}%
\pgfsys@useobject{currentmarker}{}%
\end{pgfscope}%
\begin{pgfscope}%
\pgfsys@transformshift{2.229545in}{2.040372in}%
\pgfsys@useobject{currentmarker}{}%
\end{pgfscope}%
\begin{pgfscope}%
\pgfsys@transformshift{2.271818in}{2.047649in}%
\pgfsys@useobject{currentmarker}{}%
\end{pgfscope}%
\begin{pgfscope}%
\pgfsys@transformshift{2.314091in}{2.054192in}%
\pgfsys@useobject{currentmarker}{}%
\end{pgfscope}%
\begin{pgfscope}%
\pgfsys@transformshift{2.356364in}{2.060075in}%
\pgfsys@useobject{currentmarker}{}%
\end{pgfscope}%
\begin{pgfscope}%
\pgfsys@transformshift{2.398636in}{2.065365in}%
\pgfsys@useobject{currentmarker}{}%
\end{pgfscope}%
\begin{pgfscope}%
\pgfsys@transformshift{2.440909in}{2.070122in}%
\pgfsys@useobject{currentmarker}{}%
\end{pgfscope}%
\begin{pgfscope}%
\pgfsys@transformshift{2.483182in}{2.074399in}%
\pgfsys@useobject{currentmarker}{}%
\end{pgfscope}%
\begin{pgfscope}%
\pgfsys@transformshift{2.525455in}{2.078245in}%
\pgfsys@useobject{currentmarker}{}%
\end{pgfscope}%
\begin{pgfscope}%
\pgfsys@transformshift{2.567727in}{2.081703in}%
\pgfsys@useobject{currentmarker}{}%
\end{pgfscope}%
\begin{pgfscope}%
\pgfsys@transformshift{2.610000in}{2.084812in}%
\pgfsys@useobject{currentmarker}{}%
\end{pgfscope}%
\begin{pgfscope}%
\pgfsys@transformshift{2.652273in}{2.087608in}%
\pgfsys@useobject{currentmarker}{}%
\end{pgfscope}%
\begin{pgfscope}%
\pgfsys@transformshift{2.694545in}{2.090122in}%
\pgfsys@useobject{currentmarker}{}%
\end{pgfscope}%
\begin{pgfscope}%
\pgfsys@transformshift{2.736818in}{2.092383in}%
\pgfsys@useobject{currentmarker}{}%
\end{pgfscope}%
\begin{pgfscope}%
\pgfsys@transformshift{2.779091in}{2.094415in}%
\pgfsys@useobject{currentmarker}{}%
\end{pgfscope}%
\begin{pgfscope}%
\pgfsys@transformshift{2.821364in}{2.096243in}%
\pgfsys@useobject{currentmarker}{}%
\end{pgfscope}%
\begin{pgfscope}%
\pgfsys@transformshift{2.863636in}{2.097886in}%
\pgfsys@useobject{currentmarker}{}%
\end{pgfscope}%
\begin{pgfscope}%
\pgfsys@transformshift{2.905909in}{2.099364in}%
\pgfsys@useobject{currentmarker}{}%
\end{pgfscope}%
\begin{pgfscope}%
\pgfsys@transformshift{2.948182in}{2.100693in}%
\pgfsys@useobject{currentmarker}{}%
\end{pgfscope}%
\begin{pgfscope}%
\pgfsys@transformshift{2.990455in}{2.101888in}%
\pgfsys@useobject{currentmarker}{}%
\end{pgfscope}%
\begin{pgfscope}%
\pgfsys@transformshift{3.032727in}{2.102962in}%
\pgfsys@useobject{currentmarker}{}%
\end{pgfscope}%
\begin{pgfscope}%
\pgfsys@transformshift{3.075000in}{2.103928in}%
\pgfsys@useobject{currentmarker}{}%
\end{pgfscope}%
\begin{pgfscope}%
\pgfsys@transformshift{3.117273in}{2.104796in}%
\pgfsys@useobject{currentmarker}{}%
\end{pgfscope}%
\begin{pgfscope}%
\pgfsys@transformshift{3.159545in}{2.105577in}%
\pgfsys@useobject{currentmarker}{}%
\end{pgfscope}%
\begin{pgfscope}%
\pgfsys@transformshift{3.201818in}{2.106280in}%
\pgfsys@useobject{currentmarker}{}%
\end{pgfscope}%
\begin{pgfscope}%
\pgfsys@transformshift{3.244091in}{2.106911in}%
\pgfsys@useobject{currentmarker}{}%
\end{pgfscope}%
\begin{pgfscope}%
\pgfsys@transformshift{3.286364in}{2.107479in}%
\pgfsys@useobject{currentmarker}{}%
\end{pgfscope}%
\begin{pgfscope}%
\pgfsys@transformshift{3.328636in}{2.107989in}%
\pgfsys@useobject{currentmarker}{}%
\end{pgfscope}%
\begin{pgfscope}%
\pgfsys@transformshift{3.370909in}{2.108449in}%
\pgfsys@useobject{currentmarker}{}%
\end{pgfscope}%
\begin{pgfscope}%
\pgfsys@transformshift{3.413182in}{2.108861in}%
\pgfsys@useobject{currentmarker}{}%
\end{pgfscope}%
\begin{pgfscope}%
\pgfsys@transformshift{3.455455in}{2.109232in}%
\pgfsys@useobject{currentmarker}{}%
\end{pgfscope}%
\begin{pgfscope}%
\pgfsys@transformshift{3.497727in}{2.109566in}%
\pgfsys@useobject{currentmarker}{}%
\end{pgfscope}%
\begin{pgfscope}%
\pgfsys@transformshift{3.540000in}{2.109866in}%
\pgfsys@useobject{currentmarker}{}%
\end{pgfscope}%
\begin{pgfscope}%
\pgfsys@transformshift{3.582273in}{2.110136in}%
\pgfsys@useobject{currentmarker}{}%
\end{pgfscope}%
\begin{pgfscope}%
\pgfsys@transformshift{3.624545in}{2.110379in}%
\pgfsys@useobject{currentmarker}{}%
\end{pgfscope}%
\begin{pgfscope}%
\pgfsys@transformshift{3.666818in}{2.110597in}%
\pgfsys@useobject{currentmarker}{}%
\end{pgfscope}%
\begin{pgfscope}%
\pgfsys@transformshift{3.709091in}{2.110793in}%
\pgfsys@useobject{currentmarker}{}%
\end{pgfscope}%
\begin{pgfscope}%
\pgfsys@transformshift{3.751364in}{2.110969in}%
\pgfsys@useobject{currentmarker}{}%
\end{pgfscope}%
\begin{pgfscope}%
\pgfsys@transformshift{3.793636in}{2.111128in}%
\pgfsys@useobject{currentmarker}{}%
\end{pgfscope}%
\begin{pgfscope}%
\pgfsys@transformshift{3.835909in}{2.111271in}%
\pgfsys@useobject{currentmarker}{}%
\end{pgfscope}%
\begin{pgfscope}%
\pgfsys@transformshift{3.878182in}{2.111399in}%
\pgfsys@useobject{currentmarker}{}%
\end{pgfscope}%
\begin{pgfscope}%
\pgfsys@transformshift{3.920455in}{2.111514in}%
\pgfsys@useobject{currentmarker}{}%
\end{pgfscope}%
\begin{pgfscope}%
\pgfsys@transformshift{3.962727in}{2.111618in}%
\pgfsys@useobject{currentmarker}{}%
\end{pgfscope}%
\begin{pgfscope}%
\pgfsys@transformshift{4.005000in}{2.111711in}%
\pgfsys@useobject{currentmarker}{}%
\end{pgfscope}%
\begin{pgfscope}%
\pgfsys@transformshift{4.047273in}{2.111795in}%
\pgfsys@useobject{currentmarker}{}%
\end{pgfscope}%
\begin{pgfscope}%
\pgfsys@transformshift{4.089545in}{2.111870in}%
\pgfsys@useobject{currentmarker}{}%
\end{pgfscope}%
\begin{pgfscope}%
\pgfsys@transformshift{4.131818in}{2.111938in}%
\pgfsys@useobject{currentmarker}{}%
\end{pgfscope}%
\begin{pgfscope}%
\pgfsys@transformshift{4.174091in}{2.111999in}%
\pgfsys@useobject{currentmarker}{}%
\end{pgfscope}%
\begin{pgfscope}%
\pgfsys@transformshift{4.216364in}{2.112054in}%
\pgfsys@useobject{currentmarker}{}%
\end{pgfscope}%
\begin{pgfscope}%
\pgfsys@transformshift{4.258636in}{2.112103in}%
\pgfsys@useobject{currentmarker}{}%
\end{pgfscope}%
\begin{pgfscope}%
\pgfsys@transformshift{4.300909in}{2.112147in}%
\pgfsys@useobject{currentmarker}{}%
\end{pgfscope}%
\begin{pgfscope}%
\pgfsys@transformshift{4.343182in}{2.112187in}%
\pgfsys@useobject{currentmarker}{}%
\end{pgfscope}%
\begin{pgfscope}%
\pgfsys@transformshift{4.385455in}{2.112223in}%
\pgfsys@useobject{currentmarker}{}%
\end{pgfscope}%
\begin{pgfscope}%
\pgfsys@transformshift{4.427727in}{2.112255in}%
\pgfsys@useobject{currentmarker}{}%
\end{pgfscope}%
\begin{pgfscope}%
\pgfsys@transformshift{4.470000in}{2.112284in}%
\pgfsys@useobject{currentmarker}{}%
\end{pgfscope}%
\begin{pgfscope}%
\pgfsys@transformshift{4.512273in}{2.112310in}%
\pgfsys@useobject{currentmarker}{}%
\end{pgfscope}%
\begin{pgfscope}%
\pgfsys@transformshift{4.554545in}{2.112334in}%
\pgfsys@useobject{currentmarker}{}%
\end{pgfscope}%
\begin{pgfscope}%
\pgfsys@transformshift{4.596818in}{2.112355in}%
\pgfsys@useobject{currentmarker}{}%
\end{pgfscope}%
\begin{pgfscope}%
\pgfsys@transformshift{4.639091in}{2.112374in}%
\pgfsys@useobject{currentmarker}{}%
\end{pgfscope}%
\begin{pgfscope}%
\pgfsys@transformshift{4.681364in}{2.112391in}%
\pgfsys@useobject{currentmarker}{}%
\end{pgfscope}%
\begin{pgfscope}%
\pgfsys@transformshift{4.723636in}{2.112406in}%
\pgfsys@useobject{currentmarker}{}%
\end{pgfscope}%
\begin{pgfscope}%
\pgfsys@transformshift{4.765909in}{2.112420in}%
\pgfsys@useobject{currentmarker}{}%
\end{pgfscope}%
\begin{pgfscope}%
\pgfsys@transformshift{4.808182in}{2.112432in}%
\pgfsys@useobject{currentmarker}{}%
\end{pgfscope}%
\begin{pgfscope}%
\pgfsys@transformshift{4.850455in}{2.112443in}%
\pgfsys@useobject{currentmarker}{}%
\end{pgfscope}%
\begin{pgfscope}%
\pgfsys@transformshift{4.892727in}{2.112453in}%
\pgfsys@useobject{currentmarker}{}%
\end{pgfscope}%
\begin{pgfscope}%
\pgfsys@transformshift{4.935000in}{2.112462in}%
\pgfsys@useobject{currentmarker}{}%
\end{pgfscope}%
\begin{pgfscope}%
\pgfsys@transformshift{4.977273in}{2.112470in}%
\pgfsys@useobject{currentmarker}{}%
\end{pgfscope}%
\begin{pgfscope}%
\pgfsys@transformshift{5.019545in}{2.112478in}%
\pgfsys@useobject{currentmarker}{}%
\end{pgfscope}%
\begin{pgfscope}%
\pgfsys@transformshift{5.061818in}{2.112484in}%
\pgfsys@useobject{currentmarker}{}%
\end{pgfscope}%
\begin{pgfscope}%
\pgfsys@transformshift{5.104091in}{2.112490in}%
\pgfsys@useobject{currentmarker}{}%
\end{pgfscope}%
\begin{pgfscope}%
\pgfsys@transformshift{5.146364in}{2.112495in}%
\pgfsys@useobject{currentmarker}{}%
\end{pgfscope}%
\begin{pgfscope}%
\pgfsys@transformshift{5.188636in}{2.112500in}%
\pgfsys@useobject{currentmarker}{}%
\end{pgfscope}%
\end{pgfscope}%
\begin{pgfscope}%
\pgfsetrectcap%
\pgfsetmiterjoin%
\pgfsetlinewidth{0.803000pt}%
\definecolor{currentstroke}{rgb}{0.000000,0.000000,0.000000}%
\pgfsetstrokecolor{currentstroke}%
\pgfsetdash{}{0pt}%
\pgfpathmoveto{\pgfqpoint{0.750000in}{0.275000in}}%
\pgfpathlineto{\pgfqpoint{0.750000in}{2.200000in}}%
\pgfusepath{stroke}%
\end{pgfscope}%
\begin{pgfscope}%
\pgfsetrectcap%
\pgfsetmiterjoin%
\pgfsetlinewidth{0.803000pt}%
\definecolor{currentstroke}{rgb}{0.000000,0.000000,0.000000}%
\pgfsetstrokecolor{currentstroke}%
\pgfsetdash{}{0pt}%
\pgfpathmoveto{\pgfqpoint{5.400000in}{0.275000in}}%
\pgfpathlineto{\pgfqpoint{5.400000in}{2.200000in}}%
\pgfusepath{stroke}%
\end{pgfscope}%
\begin{pgfscope}%
\pgfsetrectcap%
\pgfsetmiterjoin%
\pgfsetlinewidth{0.803000pt}%
\definecolor{currentstroke}{rgb}{0.000000,0.000000,0.000000}%
\pgfsetstrokecolor{currentstroke}%
\pgfsetdash{}{0pt}%
\pgfpathmoveto{\pgfqpoint{0.750000in}{0.275000in}}%
\pgfpathlineto{\pgfqpoint{5.400000in}{0.275000in}}%
\pgfusepath{stroke}%
\end{pgfscope}%
\begin{pgfscope}%
\pgfsetrectcap%
\pgfsetmiterjoin%
\pgfsetlinewidth{0.803000pt}%
\definecolor{currentstroke}{rgb}{0.000000,0.000000,0.000000}%
\pgfsetstrokecolor{currentstroke}%
\pgfsetdash{}{0pt}%
\pgfpathmoveto{\pgfqpoint{0.750000in}{2.200000in}}%
\pgfpathlineto{\pgfqpoint{5.400000in}{2.200000in}}%
\pgfusepath{stroke}%
\end{pgfscope}%
\end{pgfpicture}%
\makeatother%
\endgroup%

    \caption{The lower bound in \cref{eq:GMRESprob} with $\NLiDRR{\no-\nt} \sim \Exp{\sigma}$ with $\sigma = 1/k^2.$\label{fig:prob-theory-plot-2.0}}
\end{subfigure}
\caption{The lower bound in \cref{eq:GMRESprob} for $R=12$, $\eps = 10^{-5}$, $N = \ceil{k^{3}}$, and $\Ct=0.1,$ for different functional forms of $\NLiDRR{\no-\nt}$.}
\end{figure}
\ednote{Euan---the y axis on \cref{fig:prob-theory-plot-1.0} isn't correct. But basically, the values are all around 0.29, to within $10^{-11}$.}
\optodo{Fix the plot.}
\begin{figure}[p]
  \centering
  \begin{subfigure}{\textwidth}
    \centering
%% Creator: Matplotlib, PGF backend
%%
%% To include the figure in your LaTeX document, write
%%   \input{<filename>.pgf}
%%
%% Make sure the required packages are loaded in your preamble
%%   \usepackage{pgf}
%%
%% Figures using additional raster images can only be included by \input if
%% they are in the same directory as the main LaTeX file. For loading figures
%% from other directories you can use the `import` package
%%   \usepackage{import}
%% and then include the figures with
%%   \import{<path to file>}{<filename>.pgf}
%%
%% Matplotlib used the following preamble
%%   \usepackage{fontspec}
%%   \setmainfont{DejaVuSerif.ttf}[Path=/home/owen/progs/firedrake-complex/firedrake/lib/python3.5/site-packages/matplotlib/mpl-data/fonts/ttf/]
%%   \setsansfont{DejaVuSans.ttf}[Path=/home/owen/progs/firedrake-complex/firedrake/lib/python3.5/site-packages/matplotlib/mpl-data/fonts/ttf/]
%%   \setmonofont{DejaVuSansMono.ttf}[Path=/home/owen/progs/firedrake-complex/firedrake/lib/python3.5/site-packages/matplotlib/mpl-data/fonts/ttf/]
%%
\begingroup%
\makeatletter%
\begin{pgfpicture}%
\pgfpathrectangle{\pgfpointorigin}{\pgfqpoint{6.400000in}{4.800000in}}%
\pgfusepath{use as bounding box, clip}%
\begin{pgfscope}%
\pgfsetbuttcap%
\pgfsetmiterjoin%
\definecolor{currentfill}{rgb}{1.000000,1.000000,1.000000}%
\pgfsetfillcolor{currentfill}%
\pgfsetlinewidth{0.000000pt}%
\definecolor{currentstroke}{rgb}{1.000000,1.000000,1.000000}%
\pgfsetstrokecolor{currentstroke}%
\pgfsetdash{}{0pt}%
\pgfpathmoveto{\pgfqpoint{0.000000in}{0.000000in}}%
\pgfpathlineto{\pgfqpoint{6.400000in}{0.000000in}}%
\pgfpathlineto{\pgfqpoint{6.400000in}{4.800000in}}%
\pgfpathlineto{\pgfqpoint{0.000000in}{4.800000in}}%
\pgfpathclose%
\pgfusepath{fill}%
\end{pgfscope}%
\begin{pgfscope}%
\pgfsetbuttcap%
\pgfsetmiterjoin%
\definecolor{currentfill}{rgb}{1.000000,1.000000,1.000000}%
\pgfsetfillcolor{currentfill}%
\pgfsetlinewidth{0.000000pt}%
\definecolor{currentstroke}{rgb}{0.000000,0.000000,0.000000}%
\pgfsetstrokecolor{currentstroke}%
\pgfsetstrokeopacity{0.000000}%
\pgfsetdash{}{0pt}%
\pgfpathmoveto{\pgfqpoint{0.800000in}{0.528000in}}%
\pgfpathlineto{\pgfqpoint{5.760000in}{0.528000in}}%
\pgfpathlineto{\pgfqpoint{5.760000in}{4.224000in}}%
\pgfpathlineto{\pgfqpoint{0.800000in}{4.224000in}}%
\pgfpathclose%
\pgfusepath{fill}%
\end{pgfscope}%
\begin{pgfscope}%
\pgfsetbuttcap%
\pgfsetroundjoin%
\definecolor{currentfill}{rgb}{0.000000,0.000000,0.000000}%
\pgfsetfillcolor{currentfill}%
\pgfsetlinewidth{0.803000pt}%
\definecolor{currentstroke}{rgb}{0.000000,0.000000,0.000000}%
\pgfsetstrokecolor{currentstroke}%
\pgfsetdash{}{0pt}%
\pgfsys@defobject{currentmarker}{\pgfqpoint{0.000000in}{-0.048611in}}{\pgfqpoint{0.000000in}{0.000000in}}{%
\pgfpathmoveto{\pgfqpoint{0.000000in}{0.000000in}}%
\pgfpathlineto{\pgfqpoint{0.000000in}{-0.048611in}}%
\pgfusepath{stroke,fill}%
}%
\begin{pgfscope}%
\pgfsys@transformshift{1.025455in}{0.528000in}%
\pgfsys@useobject{currentmarker}{}%
\end{pgfscope}%
\end{pgfscope}%
\begin{pgfscope}%
\definecolor{textcolor}{rgb}{0.000000,0.000000,0.000000}%
\pgfsetstrokecolor{textcolor}%
\pgfsetfillcolor{textcolor}%
\pgftext[x=1.025455in,y=0.430778in,,top]{\color{textcolor}\sffamily\fontsize{10.000000}{12.000000}\selectfont 10}%
\end{pgfscope}%
\begin{pgfscope}%
\pgfsetbuttcap%
\pgfsetroundjoin%
\definecolor{currentfill}{rgb}{0.000000,0.000000,0.000000}%
\pgfsetfillcolor{currentfill}%
\pgfsetlinewidth{0.803000pt}%
\definecolor{currentstroke}{rgb}{0.000000,0.000000,0.000000}%
\pgfsetstrokecolor{currentstroke}%
\pgfsetdash{}{0pt}%
\pgfsys@defobject{currentmarker}{\pgfqpoint{0.000000in}{-0.048611in}}{\pgfqpoint{0.000000in}{0.000000in}}{%
\pgfpathmoveto{\pgfqpoint{0.000000in}{0.000000in}}%
\pgfpathlineto{\pgfqpoint{0.000000in}{-0.048611in}}%
\pgfusepath{stroke,fill}%
}%
\begin{pgfscope}%
\pgfsys@transformshift{2.528485in}{0.528000in}%
\pgfsys@useobject{currentmarker}{}%
\end{pgfscope}%
\end{pgfscope}%
\begin{pgfscope}%
\definecolor{textcolor}{rgb}{0.000000,0.000000,0.000000}%
\pgfsetstrokecolor{textcolor}%
\pgfsetfillcolor{textcolor}%
\pgftext[x=2.528485in,y=0.430778in,,top]{\color{textcolor}\sffamily\fontsize{10.000000}{12.000000}\selectfont 20}%
\end{pgfscope}%
\begin{pgfscope}%
\pgfsetbuttcap%
\pgfsetroundjoin%
\definecolor{currentfill}{rgb}{0.000000,0.000000,0.000000}%
\pgfsetfillcolor{currentfill}%
\pgfsetlinewidth{0.803000pt}%
\definecolor{currentstroke}{rgb}{0.000000,0.000000,0.000000}%
\pgfsetstrokecolor{currentstroke}%
\pgfsetdash{}{0pt}%
\pgfsys@defobject{currentmarker}{\pgfqpoint{0.000000in}{-0.048611in}}{\pgfqpoint{0.000000in}{0.000000in}}{%
\pgfpathmoveto{\pgfqpoint{0.000000in}{0.000000in}}%
\pgfpathlineto{\pgfqpoint{0.000000in}{-0.048611in}}%
\pgfusepath{stroke,fill}%
}%
\begin{pgfscope}%
\pgfsys@transformshift{4.031515in}{0.528000in}%
\pgfsys@useobject{currentmarker}{}%
\end{pgfscope}%
\end{pgfscope}%
\begin{pgfscope}%
\definecolor{textcolor}{rgb}{0.000000,0.000000,0.000000}%
\pgfsetstrokecolor{textcolor}%
\pgfsetfillcolor{textcolor}%
\pgftext[x=4.031515in,y=0.430778in,,top]{\color{textcolor}\sffamily\fontsize{10.000000}{12.000000}\selectfont 30}%
\end{pgfscope}%
\begin{pgfscope}%
\pgfsetbuttcap%
\pgfsetroundjoin%
\definecolor{currentfill}{rgb}{0.000000,0.000000,0.000000}%
\pgfsetfillcolor{currentfill}%
\pgfsetlinewidth{0.803000pt}%
\definecolor{currentstroke}{rgb}{0.000000,0.000000,0.000000}%
\pgfsetstrokecolor{currentstroke}%
\pgfsetdash{}{0pt}%
\pgfsys@defobject{currentmarker}{\pgfqpoint{0.000000in}{-0.048611in}}{\pgfqpoint{0.000000in}{0.000000in}}{%
\pgfpathmoveto{\pgfqpoint{0.000000in}{0.000000in}}%
\pgfpathlineto{\pgfqpoint{0.000000in}{-0.048611in}}%
\pgfusepath{stroke,fill}%
}%
\begin{pgfscope}%
\pgfsys@transformshift{5.534545in}{0.528000in}%
\pgfsys@useobject{currentmarker}{}%
\end{pgfscope}%
\end{pgfscope}%
\begin{pgfscope}%
\definecolor{textcolor}{rgb}{0.000000,0.000000,0.000000}%
\pgfsetstrokecolor{textcolor}%
\pgfsetfillcolor{textcolor}%
\pgftext[x=5.534545in,y=0.430778in,,top]{\color{textcolor}\sffamily\fontsize{10.000000}{12.000000}\selectfont 40}%
\end{pgfscope}%
\begin{pgfscope}%
\definecolor{textcolor}{rgb}{0.000000,0.000000,0.000000}%
\pgfsetstrokecolor{textcolor}%
\pgfsetfillcolor{textcolor}%
\pgftext[x=3.280000in,y=0.240809in,,top]{\color{textcolor}\sffamily\fontsize{10.000000}{12.000000}\selectfont \(\displaystyle k\)}%
\end{pgfscope}%
\begin{pgfscope}%
\pgfsetbuttcap%
\pgfsetroundjoin%
\definecolor{currentfill}{rgb}{0.000000,0.000000,0.000000}%
\pgfsetfillcolor{currentfill}%
\pgfsetlinewidth{0.803000pt}%
\definecolor{currentstroke}{rgb}{0.000000,0.000000,0.000000}%
\pgfsetstrokecolor{currentstroke}%
\pgfsetdash{}{0pt}%
\pgfsys@defobject{currentmarker}{\pgfqpoint{-0.048611in}{0.000000in}}{\pgfqpoint{0.000000in}{0.000000in}}{%
\pgfpathmoveto{\pgfqpoint{0.000000in}{0.000000in}}%
\pgfpathlineto{\pgfqpoint{-0.048611in}{0.000000in}}%
\pgfusepath{stroke,fill}%
}%
\begin{pgfscope}%
\pgfsys@transformshift{0.800000in}{0.696000in}%
\pgfsys@useobject{currentmarker}{}%
\end{pgfscope}%
\end{pgfscope}%
\begin{pgfscope}%
\definecolor{textcolor}{rgb}{0.000000,0.000000,0.000000}%
\pgfsetstrokecolor{textcolor}%
\pgfsetfillcolor{textcolor}%
\pgftext[x=0.481898in,y=0.643238in,left,base]{\color{textcolor}\sffamily\fontsize{10.000000}{12.000000}\selectfont 0.0}%
\end{pgfscope}%
\begin{pgfscope}%
\pgfsetbuttcap%
\pgfsetroundjoin%
\definecolor{currentfill}{rgb}{0.000000,0.000000,0.000000}%
\pgfsetfillcolor{currentfill}%
\pgfsetlinewidth{0.803000pt}%
\definecolor{currentstroke}{rgb}{0.000000,0.000000,0.000000}%
\pgfsetstrokecolor{currentstroke}%
\pgfsetdash{}{0pt}%
\pgfsys@defobject{currentmarker}{\pgfqpoint{-0.048611in}{0.000000in}}{\pgfqpoint{0.000000in}{0.000000in}}{%
\pgfpathmoveto{\pgfqpoint{0.000000in}{0.000000in}}%
\pgfpathlineto{\pgfqpoint{-0.048611in}{0.000000in}}%
\pgfusepath{stroke,fill}%
}%
\begin{pgfscope}%
\pgfsys@transformshift{0.800000in}{1.515512in}%
\pgfsys@useobject{currentmarker}{}%
\end{pgfscope}%
\end{pgfscope}%
\begin{pgfscope}%
\definecolor{textcolor}{rgb}{0.000000,0.000000,0.000000}%
\pgfsetstrokecolor{textcolor}%
\pgfsetfillcolor{textcolor}%
\pgftext[x=0.481898in,y=1.462751in,left,base]{\color{textcolor}\sffamily\fontsize{10.000000}{12.000000}\selectfont 0.1}%
\end{pgfscope}%
\begin{pgfscope}%
\pgfsetbuttcap%
\pgfsetroundjoin%
\definecolor{currentfill}{rgb}{0.000000,0.000000,0.000000}%
\pgfsetfillcolor{currentfill}%
\pgfsetlinewidth{0.803000pt}%
\definecolor{currentstroke}{rgb}{0.000000,0.000000,0.000000}%
\pgfsetstrokecolor{currentstroke}%
\pgfsetdash{}{0pt}%
\pgfsys@defobject{currentmarker}{\pgfqpoint{-0.048611in}{0.000000in}}{\pgfqpoint{0.000000in}{0.000000in}}{%
\pgfpathmoveto{\pgfqpoint{0.000000in}{0.000000in}}%
\pgfpathlineto{\pgfqpoint{-0.048611in}{0.000000in}}%
\pgfusepath{stroke,fill}%
}%
\begin{pgfscope}%
\pgfsys@transformshift{0.800000in}{2.335024in}%
\pgfsys@useobject{currentmarker}{}%
\end{pgfscope}%
\end{pgfscope}%
\begin{pgfscope}%
\definecolor{textcolor}{rgb}{0.000000,0.000000,0.000000}%
\pgfsetstrokecolor{textcolor}%
\pgfsetfillcolor{textcolor}%
\pgftext[x=0.481898in,y=2.282263in,left,base]{\color{textcolor}\sffamily\fontsize{10.000000}{12.000000}\selectfont 0.2}%
\end{pgfscope}%
\begin{pgfscope}%
\pgfsetbuttcap%
\pgfsetroundjoin%
\definecolor{currentfill}{rgb}{0.000000,0.000000,0.000000}%
\pgfsetfillcolor{currentfill}%
\pgfsetlinewidth{0.803000pt}%
\definecolor{currentstroke}{rgb}{0.000000,0.000000,0.000000}%
\pgfsetstrokecolor{currentstroke}%
\pgfsetdash{}{0pt}%
\pgfsys@defobject{currentmarker}{\pgfqpoint{-0.048611in}{0.000000in}}{\pgfqpoint{0.000000in}{0.000000in}}{%
\pgfpathmoveto{\pgfqpoint{0.000000in}{0.000000in}}%
\pgfpathlineto{\pgfqpoint{-0.048611in}{0.000000in}}%
\pgfusepath{stroke,fill}%
}%
\begin{pgfscope}%
\pgfsys@transformshift{0.800000in}{3.154537in}%
\pgfsys@useobject{currentmarker}{}%
\end{pgfscope}%
\end{pgfscope}%
\begin{pgfscope}%
\definecolor{textcolor}{rgb}{0.000000,0.000000,0.000000}%
\pgfsetstrokecolor{textcolor}%
\pgfsetfillcolor{textcolor}%
\pgftext[x=0.481898in,y=3.101775in,left,base]{\color{textcolor}\sffamily\fontsize{10.000000}{12.000000}\selectfont 0.3}%
\end{pgfscope}%
\begin{pgfscope}%
\pgfsetbuttcap%
\pgfsetroundjoin%
\definecolor{currentfill}{rgb}{0.000000,0.000000,0.000000}%
\pgfsetfillcolor{currentfill}%
\pgfsetlinewidth{0.803000pt}%
\definecolor{currentstroke}{rgb}{0.000000,0.000000,0.000000}%
\pgfsetstrokecolor{currentstroke}%
\pgfsetdash{}{0pt}%
\pgfsys@defobject{currentmarker}{\pgfqpoint{-0.048611in}{0.000000in}}{\pgfqpoint{0.000000in}{0.000000in}}{%
\pgfpathmoveto{\pgfqpoint{0.000000in}{0.000000in}}%
\pgfpathlineto{\pgfqpoint{-0.048611in}{0.000000in}}%
\pgfusepath{stroke,fill}%
}%
\begin{pgfscope}%
\pgfsys@transformshift{0.800000in}{3.974049in}%
\pgfsys@useobject{currentmarker}{}%
\end{pgfscope}%
\end{pgfscope}%
\begin{pgfscope}%
\definecolor{textcolor}{rgb}{0.000000,0.000000,0.000000}%
\pgfsetstrokecolor{textcolor}%
\pgfsetfillcolor{textcolor}%
\pgftext[x=0.481898in,y=3.921287in,left,base]{\color{textcolor}\sffamily\fontsize{10.000000}{12.000000}\selectfont 0.4}%
\end{pgfscope}%
\begin{pgfscope}%
\definecolor{textcolor}{rgb}{0.000000,0.000000,0.000000}%
\pgfsetstrokecolor{textcolor}%
\pgfsetfillcolor{textcolor}%
\pgftext[x=0.426343in,y=2.376000in,,bottom,rotate=90.000000]{\color{textcolor}\sffamily\fontsize{10.000000}{12.000000}\selectfont Number of GMRES iterations}%
\end{pgfscope}%
\begin{pgfscope}%
\pgfpathrectangle{\pgfqpoint{0.800000in}{0.528000in}}{\pgfqpoint{4.960000in}{3.696000in}}%
\pgfusepath{clip}%
\pgfsetbuttcap%
\pgfsetroundjoin%
\definecolor{currentfill}{rgb}{0.000000,0.000000,0.000000}%
\pgfsetfillcolor{currentfill}%
\pgfsetlinewidth{1.003750pt}%
\definecolor{currentstroke}{rgb}{0.000000,0.000000,0.000000}%
\pgfsetstrokecolor{currentstroke}%
\pgfsetdash{}{0pt}%
\pgfsys@defobject{currentmarker}{\pgfqpoint{-0.041667in}{-0.041667in}}{\pgfqpoint{0.041667in}{0.041667in}}{%
\pgfpathmoveto{\pgfqpoint{0.000000in}{-0.041667in}}%
\pgfpathcurveto{\pgfqpoint{0.011050in}{-0.041667in}}{\pgfqpoint{0.021649in}{-0.037276in}}{\pgfqpoint{0.029463in}{-0.029463in}}%
\pgfpathcurveto{\pgfqpoint{0.037276in}{-0.021649in}}{\pgfqpoint{0.041667in}{-0.011050in}}{\pgfqpoint{0.041667in}{0.000000in}}%
\pgfpathcurveto{\pgfqpoint{0.041667in}{0.011050in}}{\pgfqpoint{0.037276in}{0.021649in}}{\pgfqpoint{0.029463in}{0.029463in}}%
\pgfpathcurveto{\pgfqpoint{0.021649in}{0.037276in}}{\pgfqpoint{0.011050in}{0.041667in}}{\pgfqpoint{0.000000in}{0.041667in}}%
\pgfpathcurveto{\pgfqpoint{-0.011050in}{0.041667in}}{\pgfqpoint{-0.021649in}{0.037276in}}{\pgfqpoint{-0.029463in}{0.029463in}}%
\pgfpathcurveto{\pgfqpoint{-0.037276in}{0.021649in}}{\pgfqpoint{-0.041667in}{0.011050in}}{\pgfqpoint{-0.041667in}{0.000000in}}%
\pgfpathcurveto{\pgfqpoint{-0.041667in}{-0.011050in}}{\pgfqpoint{-0.037276in}{-0.021649in}}{\pgfqpoint{-0.029463in}{-0.029463in}}%
\pgfpathcurveto{\pgfqpoint{-0.021649in}{-0.037276in}}{\pgfqpoint{-0.011050in}{-0.041667in}}{\pgfqpoint{0.000000in}{-0.041667in}}%
\pgfpathclose%
\pgfusepath{stroke,fill}%
}%
\begin{pgfscope}%
\pgfsys@transformshift{1.025455in}{4.056000in}%
\pgfsys@useobject{currentmarker}{}%
\end{pgfscope}%
\end{pgfscope}%
\begin{pgfscope}%
\pgfpathrectangle{\pgfqpoint{0.800000in}{0.528000in}}{\pgfqpoint{4.960000in}{3.696000in}}%
\pgfusepath{clip}%
\pgfsetbuttcap%
\pgfsetroundjoin%
\definecolor{currentfill}{rgb}{0.000000,0.000000,0.000000}%
\pgfsetfillcolor{currentfill}%
\pgfsetlinewidth{1.003750pt}%
\definecolor{currentstroke}{rgb}{0.000000,0.000000,0.000000}%
\pgfsetstrokecolor{currentstroke}%
\pgfsetdash{}{0pt}%
\pgfsys@defobject{currentmarker}{\pgfqpoint{-0.041667in}{-0.041667in}}{\pgfqpoint{0.041667in}{0.041667in}}{%
\pgfpathmoveto{\pgfqpoint{0.000000in}{-0.041667in}}%
\pgfpathcurveto{\pgfqpoint{0.011050in}{-0.041667in}}{\pgfqpoint{0.021649in}{-0.037276in}}{\pgfqpoint{0.029463in}{-0.029463in}}%
\pgfpathcurveto{\pgfqpoint{0.037276in}{-0.021649in}}{\pgfqpoint{0.041667in}{-0.011050in}}{\pgfqpoint{0.041667in}{0.000000in}}%
\pgfpathcurveto{\pgfqpoint{0.041667in}{0.011050in}}{\pgfqpoint{0.037276in}{0.021649in}}{\pgfqpoint{0.029463in}{0.029463in}}%
\pgfpathcurveto{\pgfqpoint{0.021649in}{0.037276in}}{\pgfqpoint{0.011050in}{0.041667in}}{\pgfqpoint{0.000000in}{0.041667in}}%
\pgfpathcurveto{\pgfqpoint{-0.011050in}{0.041667in}}{\pgfqpoint{-0.021649in}{0.037276in}}{\pgfqpoint{-0.029463in}{0.029463in}}%
\pgfpathcurveto{\pgfqpoint{-0.037276in}{0.021649in}}{\pgfqpoint{-0.041667in}{0.011050in}}{\pgfqpoint{-0.041667in}{0.000000in}}%
\pgfpathcurveto{\pgfqpoint{-0.041667in}{-0.011050in}}{\pgfqpoint{-0.037276in}{-0.021649in}}{\pgfqpoint{-0.029463in}{-0.029463in}}%
\pgfpathcurveto{\pgfqpoint{-0.021649in}{-0.037276in}}{\pgfqpoint{-0.011050in}{-0.041667in}}{\pgfqpoint{0.000000in}{-0.041667in}}%
\pgfpathclose%
\pgfusepath{stroke,fill}%
}%
\begin{pgfscope}%
\pgfsys@transformshift{2.528485in}{0.933659in}%
\pgfsys@useobject{currentmarker}{}%
\end{pgfscope}%
\end{pgfscope}%
\begin{pgfscope}%
\pgfpathrectangle{\pgfqpoint{0.800000in}{0.528000in}}{\pgfqpoint{4.960000in}{3.696000in}}%
\pgfusepath{clip}%
\pgfsetbuttcap%
\pgfsetroundjoin%
\definecolor{currentfill}{rgb}{0.000000,0.000000,0.000000}%
\pgfsetfillcolor{currentfill}%
\pgfsetlinewidth{1.003750pt}%
\definecolor{currentstroke}{rgb}{0.000000,0.000000,0.000000}%
\pgfsetstrokecolor{currentstroke}%
\pgfsetdash{}{0pt}%
\pgfsys@defobject{currentmarker}{\pgfqpoint{-0.041667in}{-0.041667in}}{\pgfqpoint{0.041667in}{0.041667in}}{%
\pgfpathmoveto{\pgfqpoint{0.000000in}{-0.041667in}}%
\pgfpathcurveto{\pgfqpoint{0.011050in}{-0.041667in}}{\pgfqpoint{0.021649in}{-0.037276in}}{\pgfqpoint{0.029463in}{-0.029463in}}%
\pgfpathcurveto{\pgfqpoint{0.037276in}{-0.021649in}}{\pgfqpoint{0.041667in}{-0.011050in}}{\pgfqpoint{0.041667in}{0.000000in}}%
\pgfpathcurveto{\pgfqpoint{0.041667in}{0.011050in}}{\pgfqpoint{0.037276in}{0.021649in}}{\pgfqpoint{0.029463in}{0.029463in}}%
\pgfpathcurveto{\pgfqpoint{0.021649in}{0.037276in}}{\pgfqpoint{0.011050in}{0.041667in}}{\pgfqpoint{0.000000in}{0.041667in}}%
\pgfpathcurveto{\pgfqpoint{-0.011050in}{0.041667in}}{\pgfqpoint{-0.021649in}{0.037276in}}{\pgfqpoint{-0.029463in}{0.029463in}}%
\pgfpathcurveto{\pgfqpoint{-0.037276in}{0.021649in}}{\pgfqpoint{-0.041667in}{0.011050in}}{\pgfqpoint{-0.041667in}{0.000000in}}%
\pgfpathcurveto{\pgfqpoint{-0.041667in}{-0.011050in}}{\pgfqpoint{-0.037276in}{-0.021649in}}{\pgfqpoint{-0.029463in}{-0.029463in}}%
\pgfpathcurveto{\pgfqpoint{-0.021649in}{-0.037276in}}{\pgfqpoint{-0.011050in}{-0.041667in}}{\pgfqpoint{0.000000in}{-0.041667in}}%
\pgfpathclose%
\pgfusepath{stroke,fill}%
}%
\begin{pgfscope}%
\pgfsys@transformshift{4.031515in}{0.696000in}%
\pgfsys@useobject{currentmarker}{}%
\end{pgfscope}%
\end{pgfscope}%
\begin{pgfscope}%
\pgfpathrectangle{\pgfqpoint{0.800000in}{0.528000in}}{\pgfqpoint{4.960000in}{3.696000in}}%
\pgfusepath{clip}%
\pgfsetbuttcap%
\pgfsetroundjoin%
\definecolor{currentfill}{rgb}{0.000000,0.000000,0.000000}%
\pgfsetfillcolor{currentfill}%
\pgfsetlinewidth{1.003750pt}%
\definecolor{currentstroke}{rgb}{0.000000,0.000000,0.000000}%
\pgfsetstrokecolor{currentstroke}%
\pgfsetdash{}{0pt}%
\pgfsys@defobject{currentmarker}{\pgfqpoint{-0.041667in}{-0.041667in}}{\pgfqpoint{0.041667in}{0.041667in}}{%
\pgfpathmoveto{\pgfqpoint{0.000000in}{-0.041667in}}%
\pgfpathcurveto{\pgfqpoint{0.011050in}{-0.041667in}}{\pgfqpoint{0.021649in}{-0.037276in}}{\pgfqpoint{0.029463in}{-0.029463in}}%
\pgfpathcurveto{\pgfqpoint{0.037276in}{-0.021649in}}{\pgfqpoint{0.041667in}{-0.011050in}}{\pgfqpoint{0.041667in}{0.000000in}}%
\pgfpathcurveto{\pgfqpoint{0.041667in}{0.011050in}}{\pgfqpoint{0.037276in}{0.021649in}}{\pgfqpoint{0.029463in}{0.029463in}}%
\pgfpathcurveto{\pgfqpoint{0.021649in}{0.037276in}}{\pgfqpoint{0.011050in}{0.041667in}}{\pgfqpoint{0.000000in}{0.041667in}}%
\pgfpathcurveto{\pgfqpoint{-0.011050in}{0.041667in}}{\pgfqpoint{-0.021649in}{0.037276in}}{\pgfqpoint{-0.029463in}{0.029463in}}%
\pgfpathcurveto{\pgfqpoint{-0.037276in}{0.021649in}}{\pgfqpoint{-0.041667in}{0.011050in}}{\pgfqpoint{-0.041667in}{0.000000in}}%
\pgfpathcurveto{\pgfqpoint{-0.041667in}{-0.011050in}}{\pgfqpoint{-0.037276in}{-0.021649in}}{\pgfqpoint{-0.029463in}{-0.029463in}}%
\pgfpathcurveto{\pgfqpoint{-0.021649in}{-0.037276in}}{\pgfqpoint{-0.011050in}{-0.041667in}}{\pgfqpoint{0.000000in}{-0.041667in}}%
\pgfpathclose%
\pgfusepath{stroke,fill}%
}%
\begin{pgfscope}%
\pgfsys@transformshift{5.534545in}{0.696000in}%
\pgfsys@useobject{currentmarker}{}%
\end{pgfscope}%
\end{pgfscope}%
\begin{pgfscope}%
\pgfsetrectcap%
\pgfsetmiterjoin%
\pgfsetlinewidth{0.803000pt}%
\definecolor{currentstroke}{rgb}{0.000000,0.000000,0.000000}%
\pgfsetstrokecolor{currentstroke}%
\pgfsetdash{}{0pt}%
\pgfpathmoveto{\pgfqpoint{0.800000in}{0.528000in}}%
\pgfpathlineto{\pgfqpoint{0.800000in}{4.224000in}}%
\pgfusepath{stroke}%
\end{pgfscope}%
\begin{pgfscope}%
\pgfsetrectcap%
\pgfsetmiterjoin%
\pgfsetlinewidth{0.803000pt}%
\definecolor{currentstroke}{rgb}{0.000000,0.000000,0.000000}%
\pgfsetstrokecolor{currentstroke}%
\pgfsetdash{}{0pt}%
\pgfpathmoveto{\pgfqpoint{5.760000in}{0.528000in}}%
\pgfpathlineto{\pgfqpoint{5.760000in}{4.224000in}}%
\pgfusepath{stroke}%
\end{pgfscope}%
\begin{pgfscope}%
\pgfsetrectcap%
\pgfsetmiterjoin%
\pgfsetlinewidth{0.803000pt}%
\definecolor{currentstroke}{rgb}{0.000000,0.000000,0.000000}%
\pgfsetstrokecolor{currentstroke}%
\pgfsetdash{}{0pt}%
\pgfpathmoveto{\pgfqpoint{0.800000in}{0.528000in}}%
\pgfpathlineto{\pgfqpoint{5.760000in}{0.528000in}}%
\pgfusepath{stroke}%
\end{pgfscope}%
\begin{pgfscope}%
\pgfsetrectcap%
\pgfsetmiterjoin%
\pgfsetlinewidth{0.803000pt}%
\definecolor{currentstroke}{rgb}{0.000000,0.000000,0.000000}%
\pgfsetstrokecolor{currentstroke}%
\pgfsetdash{}{0pt}%
\pgfpathmoveto{\pgfqpoint{0.800000in}{4.224000in}}%
\pgfpathlineto{\pgfqpoint{5.760000in}{4.224000in}}%
\pgfusepath{stroke}%
\end{pgfscope}%
\end{pgfpicture}%
\makeatother%
\endgroup%

\caption{Empirical probability that $\GMRES{\eps}{\no}{\nt}\leq 12$ for $\sigma = 1.$\label{fig:prob-plot-0.0}}
\end{subfigure}

\begin{subfigure}{\textwidth}
    \centering
%% Creator: Matplotlib, PGF backend
%%
%% To include the figure in your LaTeX document, write
%%   \input{<filename>.pgf}
%%
%% Make sure the required packages are loaded in your preamble
%%   \usepackage{pgf}
%%
%% Figures using additional raster images can only be included by \input if
%% they are in the same directory as the main LaTeX file. For loading figures
%% from other directories you can use the `import` package
%%   \usepackage{import}
%% and then include the figures with
%%   \import{<path to file>}{<filename>.pgf}
%%
%% Matplotlib used the following preamble
%%   \usepackage{fontspec}
%%   \setmainfont{DejaVuSerif.ttf}[Path=/home/owen/progs/firedrake-complex/firedrake/lib/python3.5/site-packages/matplotlib/mpl-data/fonts/ttf/]
%%   \setsansfont{DejaVuSans.ttf}[Path=/home/owen/progs/firedrake-complex/firedrake/lib/python3.5/site-packages/matplotlib/mpl-data/fonts/ttf/]
%%   \setmonofont{DejaVuSansMono.ttf}[Path=/home/owen/progs/firedrake-complex/firedrake/lib/python3.5/site-packages/matplotlib/mpl-data/fonts/ttf/]
%%
\begingroup%
\makeatletter%
\begin{pgfpicture}%
\pgfpathrectangle{\pgfpointorigin}{\pgfqpoint{6.400000in}{4.800000in}}%
\pgfusepath{use as bounding box, clip}%
\begin{pgfscope}%
\pgfsetbuttcap%
\pgfsetmiterjoin%
\definecolor{currentfill}{rgb}{1.000000,1.000000,1.000000}%
\pgfsetfillcolor{currentfill}%
\pgfsetlinewidth{0.000000pt}%
\definecolor{currentstroke}{rgb}{1.000000,1.000000,1.000000}%
\pgfsetstrokecolor{currentstroke}%
\pgfsetdash{}{0pt}%
\pgfpathmoveto{\pgfqpoint{0.000000in}{0.000000in}}%
\pgfpathlineto{\pgfqpoint{6.400000in}{0.000000in}}%
\pgfpathlineto{\pgfqpoint{6.400000in}{4.800000in}}%
\pgfpathlineto{\pgfqpoint{0.000000in}{4.800000in}}%
\pgfpathclose%
\pgfusepath{fill}%
\end{pgfscope}%
\begin{pgfscope}%
\pgfsetbuttcap%
\pgfsetmiterjoin%
\definecolor{currentfill}{rgb}{1.000000,1.000000,1.000000}%
\pgfsetfillcolor{currentfill}%
\pgfsetlinewidth{0.000000pt}%
\definecolor{currentstroke}{rgb}{0.000000,0.000000,0.000000}%
\pgfsetstrokecolor{currentstroke}%
\pgfsetstrokeopacity{0.000000}%
\pgfsetdash{}{0pt}%
\pgfpathmoveto{\pgfqpoint{0.800000in}{0.528000in}}%
\pgfpathlineto{\pgfqpoint{5.760000in}{0.528000in}}%
\pgfpathlineto{\pgfqpoint{5.760000in}{4.224000in}}%
\pgfpathlineto{\pgfqpoint{0.800000in}{4.224000in}}%
\pgfpathclose%
\pgfusepath{fill}%
\end{pgfscope}%
\begin{pgfscope}%
\pgfsetbuttcap%
\pgfsetroundjoin%
\definecolor{currentfill}{rgb}{0.000000,0.000000,0.000000}%
\pgfsetfillcolor{currentfill}%
\pgfsetlinewidth{0.803000pt}%
\definecolor{currentstroke}{rgb}{0.000000,0.000000,0.000000}%
\pgfsetstrokecolor{currentstroke}%
\pgfsetdash{}{0pt}%
\pgfsys@defobject{currentmarker}{\pgfqpoint{0.000000in}{-0.048611in}}{\pgfqpoint{0.000000in}{0.000000in}}{%
\pgfpathmoveto{\pgfqpoint{0.000000in}{0.000000in}}%
\pgfpathlineto{\pgfqpoint{0.000000in}{-0.048611in}}%
\pgfusepath{stroke,fill}%
}%
\begin{pgfscope}%
\pgfsys@transformshift{1.025455in}{0.528000in}%
\pgfsys@useobject{currentmarker}{}%
\end{pgfscope}%
\end{pgfscope}%
\begin{pgfscope}%
\definecolor{textcolor}{rgb}{0.000000,0.000000,0.000000}%
\pgfsetstrokecolor{textcolor}%
\pgfsetfillcolor{textcolor}%
\pgftext[x=1.025455in,y=0.430778in,,top]{\color{textcolor}\sffamily\fontsize{10.000000}{12.000000}\selectfont 10}%
\end{pgfscope}%
\begin{pgfscope}%
\pgfsetbuttcap%
\pgfsetroundjoin%
\definecolor{currentfill}{rgb}{0.000000,0.000000,0.000000}%
\pgfsetfillcolor{currentfill}%
\pgfsetlinewidth{0.803000pt}%
\definecolor{currentstroke}{rgb}{0.000000,0.000000,0.000000}%
\pgfsetstrokecolor{currentstroke}%
\pgfsetdash{}{0pt}%
\pgfsys@defobject{currentmarker}{\pgfqpoint{0.000000in}{-0.048611in}}{\pgfqpoint{0.000000in}{0.000000in}}{%
\pgfpathmoveto{\pgfqpoint{0.000000in}{0.000000in}}%
\pgfpathlineto{\pgfqpoint{0.000000in}{-0.048611in}}%
\pgfusepath{stroke,fill}%
}%
\begin{pgfscope}%
\pgfsys@transformshift{2.528485in}{0.528000in}%
\pgfsys@useobject{currentmarker}{}%
\end{pgfscope}%
\end{pgfscope}%
\begin{pgfscope}%
\definecolor{textcolor}{rgb}{0.000000,0.000000,0.000000}%
\pgfsetstrokecolor{textcolor}%
\pgfsetfillcolor{textcolor}%
\pgftext[x=2.528485in,y=0.430778in,,top]{\color{textcolor}\sffamily\fontsize{10.000000}{12.000000}\selectfont 20}%
\end{pgfscope}%
\begin{pgfscope}%
\pgfsetbuttcap%
\pgfsetroundjoin%
\definecolor{currentfill}{rgb}{0.000000,0.000000,0.000000}%
\pgfsetfillcolor{currentfill}%
\pgfsetlinewidth{0.803000pt}%
\definecolor{currentstroke}{rgb}{0.000000,0.000000,0.000000}%
\pgfsetstrokecolor{currentstroke}%
\pgfsetdash{}{0pt}%
\pgfsys@defobject{currentmarker}{\pgfqpoint{0.000000in}{-0.048611in}}{\pgfqpoint{0.000000in}{0.000000in}}{%
\pgfpathmoveto{\pgfqpoint{0.000000in}{0.000000in}}%
\pgfpathlineto{\pgfqpoint{0.000000in}{-0.048611in}}%
\pgfusepath{stroke,fill}%
}%
\begin{pgfscope}%
\pgfsys@transformshift{4.031515in}{0.528000in}%
\pgfsys@useobject{currentmarker}{}%
\end{pgfscope}%
\end{pgfscope}%
\begin{pgfscope}%
\definecolor{textcolor}{rgb}{0.000000,0.000000,0.000000}%
\pgfsetstrokecolor{textcolor}%
\pgfsetfillcolor{textcolor}%
\pgftext[x=4.031515in,y=0.430778in,,top]{\color{textcolor}\sffamily\fontsize{10.000000}{12.000000}\selectfont 30}%
\end{pgfscope}%
\begin{pgfscope}%
\pgfsetbuttcap%
\pgfsetroundjoin%
\definecolor{currentfill}{rgb}{0.000000,0.000000,0.000000}%
\pgfsetfillcolor{currentfill}%
\pgfsetlinewidth{0.803000pt}%
\definecolor{currentstroke}{rgb}{0.000000,0.000000,0.000000}%
\pgfsetstrokecolor{currentstroke}%
\pgfsetdash{}{0pt}%
\pgfsys@defobject{currentmarker}{\pgfqpoint{0.000000in}{-0.048611in}}{\pgfqpoint{0.000000in}{0.000000in}}{%
\pgfpathmoveto{\pgfqpoint{0.000000in}{0.000000in}}%
\pgfpathlineto{\pgfqpoint{0.000000in}{-0.048611in}}%
\pgfusepath{stroke,fill}%
}%
\begin{pgfscope}%
\pgfsys@transformshift{5.534545in}{0.528000in}%
\pgfsys@useobject{currentmarker}{}%
\end{pgfscope}%
\end{pgfscope}%
\begin{pgfscope}%
\definecolor{textcolor}{rgb}{0.000000,0.000000,0.000000}%
\pgfsetstrokecolor{textcolor}%
\pgfsetfillcolor{textcolor}%
\pgftext[x=5.534545in,y=0.430778in,,top]{\color{textcolor}\sffamily\fontsize{10.000000}{12.000000}\selectfont 40}%
\end{pgfscope}%
\begin{pgfscope}%
\definecolor{textcolor}{rgb}{0.000000,0.000000,0.000000}%
\pgfsetstrokecolor{textcolor}%
\pgfsetfillcolor{textcolor}%
\pgftext[x=3.280000in,y=0.240809in,,top]{\color{textcolor}\sffamily\fontsize{10.000000}{12.000000}\selectfont \(\displaystyle k\)}%
\end{pgfscope}%
\begin{pgfscope}%
\pgfsetbuttcap%
\pgfsetroundjoin%
\definecolor{currentfill}{rgb}{0.000000,0.000000,0.000000}%
\pgfsetfillcolor{currentfill}%
\pgfsetlinewidth{0.803000pt}%
\definecolor{currentstroke}{rgb}{0.000000,0.000000,0.000000}%
\pgfsetstrokecolor{currentstroke}%
\pgfsetdash{}{0pt}%
\pgfsys@defobject{currentmarker}{\pgfqpoint{-0.048611in}{0.000000in}}{\pgfqpoint{0.000000in}{0.000000in}}{%
\pgfpathmoveto{\pgfqpoint{0.000000in}{0.000000in}}%
\pgfpathlineto{\pgfqpoint{-0.048611in}{0.000000in}}%
\pgfusepath{stroke,fill}%
}%
\begin{pgfscope}%
\pgfsys@transformshift{0.800000in}{0.682922in}%
\pgfsys@useobject{currentmarker}{}%
\end{pgfscope}%
\end{pgfscope}%
\begin{pgfscope}%
\definecolor{textcolor}{rgb}{0.000000,0.000000,0.000000}%
\pgfsetstrokecolor{textcolor}%
\pgfsetfillcolor{textcolor}%
\pgftext[x=0.305168in,y=0.630161in,left,base]{\color{textcolor}\sffamily\fontsize{10.000000}{12.000000}\selectfont 0.992}%
\end{pgfscope}%
\begin{pgfscope}%
\pgfsetbuttcap%
\pgfsetroundjoin%
\definecolor{currentfill}{rgb}{0.000000,0.000000,0.000000}%
\pgfsetfillcolor{currentfill}%
\pgfsetlinewidth{0.803000pt}%
\definecolor{currentstroke}{rgb}{0.000000,0.000000,0.000000}%
\pgfsetstrokecolor{currentstroke}%
\pgfsetdash{}{0pt}%
\pgfsys@defobject{currentmarker}{\pgfqpoint{-0.048611in}{0.000000in}}{\pgfqpoint{0.000000in}{0.000000in}}{%
\pgfpathmoveto{\pgfqpoint{0.000000in}{0.000000in}}%
\pgfpathlineto{\pgfqpoint{-0.048611in}{0.000000in}}%
\pgfusepath{stroke,fill}%
}%
\begin{pgfscope}%
\pgfsys@transformshift{0.800000in}{1.568192in}%
\pgfsys@useobject{currentmarker}{}%
\end{pgfscope}%
\end{pgfscope}%
\begin{pgfscope}%
\definecolor{textcolor}{rgb}{0.000000,0.000000,0.000000}%
\pgfsetstrokecolor{textcolor}%
\pgfsetfillcolor{textcolor}%
\pgftext[x=0.305168in,y=1.515430in,left,base]{\color{textcolor}\sffamily\fontsize{10.000000}{12.000000}\selectfont 0.994}%
\end{pgfscope}%
\begin{pgfscope}%
\pgfsetbuttcap%
\pgfsetroundjoin%
\definecolor{currentfill}{rgb}{0.000000,0.000000,0.000000}%
\pgfsetfillcolor{currentfill}%
\pgfsetlinewidth{0.803000pt}%
\definecolor{currentstroke}{rgb}{0.000000,0.000000,0.000000}%
\pgfsetstrokecolor{currentstroke}%
\pgfsetdash{}{0pt}%
\pgfsys@defobject{currentmarker}{\pgfqpoint{-0.048611in}{0.000000in}}{\pgfqpoint{0.000000in}{0.000000in}}{%
\pgfpathmoveto{\pgfqpoint{0.000000in}{0.000000in}}%
\pgfpathlineto{\pgfqpoint{-0.048611in}{0.000000in}}%
\pgfusepath{stroke,fill}%
}%
\begin{pgfscope}%
\pgfsys@transformshift{0.800000in}{2.453461in}%
\pgfsys@useobject{currentmarker}{}%
\end{pgfscope}%
\end{pgfscope}%
\begin{pgfscope}%
\definecolor{textcolor}{rgb}{0.000000,0.000000,0.000000}%
\pgfsetstrokecolor{textcolor}%
\pgfsetfillcolor{textcolor}%
\pgftext[x=0.305168in,y=2.400700in,left,base]{\color{textcolor}\sffamily\fontsize{10.000000}{12.000000}\selectfont 0.996}%
\end{pgfscope}%
\begin{pgfscope}%
\pgfsetbuttcap%
\pgfsetroundjoin%
\definecolor{currentfill}{rgb}{0.000000,0.000000,0.000000}%
\pgfsetfillcolor{currentfill}%
\pgfsetlinewidth{0.803000pt}%
\definecolor{currentstroke}{rgb}{0.000000,0.000000,0.000000}%
\pgfsetstrokecolor{currentstroke}%
\pgfsetdash{}{0pt}%
\pgfsys@defobject{currentmarker}{\pgfqpoint{-0.048611in}{0.000000in}}{\pgfqpoint{0.000000in}{0.000000in}}{%
\pgfpathmoveto{\pgfqpoint{0.000000in}{0.000000in}}%
\pgfpathlineto{\pgfqpoint{-0.048611in}{0.000000in}}%
\pgfusepath{stroke,fill}%
}%
\begin{pgfscope}%
\pgfsys@transformshift{0.800000in}{3.338731in}%
\pgfsys@useobject{currentmarker}{}%
\end{pgfscope}%
\end{pgfscope}%
\begin{pgfscope}%
\definecolor{textcolor}{rgb}{0.000000,0.000000,0.000000}%
\pgfsetstrokecolor{textcolor}%
\pgfsetfillcolor{textcolor}%
\pgftext[x=0.305168in,y=3.285969in,left,base]{\color{textcolor}\sffamily\fontsize{10.000000}{12.000000}\selectfont 0.998}%
\end{pgfscope}%
\begin{pgfscope}%
\pgfsetbuttcap%
\pgfsetroundjoin%
\definecolor{currentfill}{rgb}{0.000000,0.000000,0.000000}%
\pgfsetfillcolor{currentfill}%
\pgfsetlinewidth{0.803000pt}%
\definecolor{currentstroke}{rgb}{0.000000,0.000000,0.000000}%
\pgfsetstrokecolor{currentstroke}%
\pgfsetdash{}{0pt}%
\pgfsys@defobject{currentmarker}{\pgfqpoint{-0.048611in}{0.000000in}}{\pgfqpoint{0.000000in}{0.000000in}}{%
\pgfpathmoveto{\pgfqpoint{0.000000in}{0.000000in}}%
\pgfpathlineto{\pgfqpoint{-0.048611in}{0.000000in}}%
\pgfusepath{stroke,fill}%
}%
\begin{pgfscope}%
\pgfsys@transformshift{0.800000in}{4.224000in}%
\pgfsys@useobject{currentmarker}{}%
\end{pgfscope}%
\end{pgfscope}%
\begin{pgfscope}%
\definecolor{textcolor}{rgb}{0.000000,0.000000,0.000000}%
\pgfsetstrokecolor{textcolor}%
\pgfsetfillcolor{textcolor}%
\pgftext[x=0.305168in,y=4.171238in,left,base]{\color{textcolor}\sffamily\fontsize{10.000000}{12.000000}\selectfont 1.000}%
\end{pgfscope}%
\begin{pgfscope}%
\definecolor{textcolor}{rgb}{0.000000,0.000000,0.000000}%
\pgfsetstrokecolor{textcolor}%
\pgfsetfillcolor{textcolor}%
\pgftext[x=0.249612in,y=2.376000in,,bottom,rotate=90.000000]{\color{textcolor}\sffamily\fontsize{10.000000}{12.000000}\selectfont Number of GMRES iterations}%
\end{pgfscope}%
\begin{pgfscope}%
\pgfpathrectangle{\pgfqpoint{0.800000in}{0.528000in}}{\pgfqpoint{4.960000in}{3.696000in}}%
\pgfusepath{clip}%
\pgfsetbuttcap%
\pgfsetroundjoin%
\definecolor{currentfill}{rgb}{0.000000,0.000000,0.000000}%
\pgfsetfillcolor{currentfill}%
\pgfsetlinewidth{1.003750pt}%
\definecolor{currentstroke}{rgb}{0.000000,0.000000,0.000000}%
\pgfsetstrokecolor{currentstroke}%
\pgfsetdash{}{0pt}%
\pgfsys@defobject{currentmarker}{\pgfqpoint{-0.041667in}{-0.041667in}}{\pgfqpoint{0.041667in}{0.041667in}}{%
\pgfpathmoveto{\pgfqpoint{0.000000in}{-0.041667in}}%
\pgfpathcurveto{\pgfqpoint{0.011050in}{-0.041667in}}{\pgfqpoint{0.021649in}{-0.037276in}}{\pgfqpoint{0.029463in}{-0.029463in}}%
\pgfpathcurveto{\pgfqpoint{0.037276in}{-0.021649in}}{\pgfqpoint{0.041667in}{-0.011050in}}{\pgfqpoint{0.041667in}{0.000000in}}%
\pgfpathcurveto{\pgfqpoint{0.041667in}{0.011050in}}{\pgfqpoint{0.037276in}{0.021649in}}{\pgfqpoint{0.029463in}{0.029463in}}%
\pgfpathcurveto{\pgfqpoint{0.021649in}{0.037276in}}{\pgfqpoint{0.011050in}{0.041667in}}{\pgfqpoint{0.000000in}{0.041667in}}%
\pgfpathcurveto{\pgfqpoint{-0.011050in}{0.041667in}}{\pgfqpoint{-0.021649in}{0.037276in}}{\pgfqpoint{-0.029463in}{0.029463in}}%
\pgfpathcurveto{\pgfqpoint{-0.037276in}{0.021649in}}{\pgfqpoint{-0.041667in}{0.011050in}}{\pgfqpoint{-0.041667in}{0.000000in}}%
\pgfpathcurveto{\pgfqpoint{-0.041667in}{-0.011050in}}{\pgfqpoint{-0.037276in}{-0.021649in}}{\pgfqpoint{-0.029463in}{-0.029463in}}%
\pgfpathcurveto{\pgfqpoint{-0.021649in}{-0.037276in}}{\pgfqpoint{-0.011050in}{-0.041667in}}{\pgfqpoint{0.000000in}{-0.041667in}}%
\pgfpathclose%
\pgfusepath{stroke,fill}%
}%
\begin{pgfscope}%
\pgfsys@transformshift{1.025455in}{3.781365in}%
\pgfsys@useobject{currentmarker}{}%
\end{pgfscope}%
\end{pgfscope}%
\begin{pgfscope}%
\pgfpathrectangle{\pgfqpoint{0.800000in}{0.528000in}}{\pgfqpoint{4.960000in}{3.696000in}}%
\pgfusepath{clip}%
\pgfsetbuttcap%
\pgfsetroundjoin%
\definecolor{currentfill}{rgb}{0.000000,0.000000,0.000000}%
\pgfsetfillcolor{currentfill}%
\pgfsetlinewidth{1.003750pt}%
\definecolor{currentstroke}{rgb}{0.000000,0.000000,0.000000}%
\pgfsetstrokecolor{currentstroke}%
\pgfsetdash{}{0pt}%
\pgfsys@defobject{currentmarker}{\pgfqpoint{-0.041667in}{-0.041667in}}{\pgfqpoint{0.041667in}{0.041667in}}{%
\pgfpathmoveto{\pgfqpoint{0.000000in}{-0.041667in}}%
\pgfpathcurveto{\pgfqpoint{0.011050in}{-0.041667in}}{\pgfqpoint{0.021649in}{-0.037276in}}{\pgfqpoint{0.029463in}{-0.029463in}}%
\pgfpathcurveto{\pgfqpoint{0.037276in}{-0.021649in}}{\pgfqpoint{0.041667in}{-0.011050in}}{\pgfqpoint{0.041667in}{0.000000in}}%
\pgfpathcurveto{\pgfqpoint{0.041667in}{0.011050in}}{\pgfqpoint{0.037276in}{0.021649in}}{\pgfqpoint{0.029463in}{0.029463in}}%
\pgfpathcurveto{\pgfqpoint{0.021649in}{0.037276in}}{\pgfqpoint{0.011050in}{0.041667in}}{\pgfqpoint{0.000000in}{0.041667in}}%
\pgfpathcurveto{\pgfqpoint{-0.011050in}{0.041667in}}{\pgfqpoint{-0.021649in}{0.037276in}}{\pgfqpoint{-0.029463in}{0.029463in}}%
\pgfpathcurveto{\pgfqpoint{-0.037276in}{0.021649in}}{\pgfqpoint{-0.041667in}{0.011050in}}{\pgfqpoint{-0.041667in}{0.000000in}}%
\pgfpathcurveto{\pgfqpoint{-0.041667in}{-0.011050in}}{\pgfqpoint{-0.037276in}{-0.021649in}}{\pgfqpoint{-0.029463in}{-0.029463in}}%
\pgfpathcurveto{\pgfqpoint{-0.021649in}{-0.037276in}}{\pgfqpoint{-0.011050in}{-0.041667in}}{\pgfqpoint{0.000000in}{-0.041667in}}%
\pgfpathclose%
\pgfusepath{stroke,fill}%
}%
\begin{pgfscope}%
\pgfsys@transformshift{2.528485in}{0.682922in}%
\pgfsys@useobject{currentmarker}{}%
\end{pgfscope}%
\end{pgfscope}%
\begin{pgfscope}%
\pgfpathrectangle{\pgfqpoint{0.800000in}{0.528000in}}{\pgfqpoint{4.960000in}{3.696000in}}%
\pgfusepath{clip}%
\pgfsetbuttcap%
\pgfsetroundjoin%
\definecolor{currentfill}{rgb}{0.000000,0.000000,0.000000}%
\pgfsetfillcolor{currentfill}%
\pgfsetlinewidth{1.003750pt}%
\definecolor{currentstroke}{rgb}{0.000000,0.000000,0.000000}%
\pgfsetstrokecolor{currentstroke}%
\pgfsetdash{}{0pt}%
\pgfsys@defobject{currentmarker}{\pgfqpoint{-0.041667in}{-0.041667in}}{\pgfqpoint{0.041667in}{0.041667in}}{%
\pgfpathmoveto{\pgfqpoint{0.000000in}{-0.041667in}}%
\pgfpathcurveto{\pgfqpoint{0.011050in}{-0.041667in}}{\pgfqpoint{0.021649in}{-0.037276in}}{\pgfqpoint{0.029463in}{-0.029463in}}%
\pgfpathcurveto{\pgfqpoint{0.037276in}{-0.021649in}}{\pgfqpoint{0.041667in}{-0.011050in}}{\pgfqpoint{0.041667in}{0.000000in}}%
\pgfpathcurveto{\pgfqpoint{0.041667in}{0.011050in}}{\pgfqpoint{0.037276in}{0.021649in}}{\pgfqpoint{0.029463in}{0.029463in}}%
\pgfpathcurveto{\pgfqpoint{0.021649in}{0.037276in}}{\pgfqpoint{0.011050in}{0.041667in}}{\pgfqpoint{0.000000in}{0.041667in}}%
\pgfpathcurveto{\pgfqpoint{-0.011050in}{0.041667in}}{\pgfqpoint{-0.021649in}{0.037276in}}{\pgfqpoint{-0.029463in}{0.029463in}}%
\pgfpathcurveto{\pgfqpoint{-0.037276in}{0.021649in}}{\pgfqpoint{-0.041667in}{0.011050in}}{\pgfqpoint{-0.041667in}{0.000000in}}%
\pgfpathcurveto{\pgfqpoint{-0.041667in}{-0.011050in}}{\pgfqpoint{-0.037276in}{-0.021649in}}{\pgfqpoint{-0.029463in}{-0.029463in}}%
\pgfpathcurveto{\pgfqpoint{-0.021649in}{-0.037276in}}{\pgfqpoint{-0.011050in}{-0.041667in}}{\pgfqpoint{0.000000in}{-0.041667in}}%
\pgfpathclose%
\pgfusepath{stroke,fill}%
}%
\begin{pgfscope}%
\pgfsys@transformshift{4.031515in}{0.682922in}%
\pgfsys@useobject{currentmarker}{}%
\end{pgfscope}%
\end{pgfscope}%
\begin{pgfscope}%
\pgfpathrectangle{\pgfqpoint{0.800000in}{0.528000in}}{\pgfqpoint{4.960000in}{3.696000in}}%
\pgfusepath{clip}%
\pgfsetbuttcap%
\pgfsetroundjoin%
\definecolor{currentfill}{rgb}{0.000000,0.000000,0.000000}%
\pgfsetfillcolor{currentfill}%
\pgfsetlinewidth{1.003750pt}%
\definecolor{currentstroke}{rgb}{0.000000,0.000000,0.000000}%
\pgfsetstrokecolor{currentstroke}%
\pgfsetdash{}{0pt}%
\pgfsys@defobject{currentmarker}{\pgfqpoint{-0.041667in}{-0.041667in}}{\pgfqpoint{0.041667in}{0.041667in}}{%
\pgfpathmoveto{\pgfqpoint{0.000000in}{-0.041667in}}%
\pgfpathcurveto{\pgfqpoint{0.011050in}{-0.041667in}}{\pgfqpoint{0.021649in}{-0.037276in}}{\pgfqpoint{0.029463in}{-0.029463in}}%
\pgfpathcurveto{\pgfqpoint{0.037276in}{-0.021649in}}{\pgfqpoint{0.041667in}{-0.011050in}}{\pgfqpoint{0.041667in}{0.000000in}}%
\pgfpathcurveto{\pgfqpoint{0.041667in}{0.011050in}}{\pgfqpoint{0.037276in}{0.021649in}}{\pgfqpoint{0.029463in}{0.029463in}}%
\pgfpathcurveto{\pgfqpoint{0.021649in}{0.037276in}}{\pgfqpoint{0.011050in}{0.041667in}}{\pgfqpoint{0.000000in}{0.041667in}}%
\pgfpathcurveto{\pgfqpoint{-0.011050in}{0.041667in}}{\pgfqpoint{-0.021649in}{0.037276in}}{\pgfqpoint{-0.029463in}{0.029463in}}%
\pgfpathcurveto{\pgfqpoint{-0.037276in}{0.021649in}}{\pgfqpoint{-0.041667in}{0.011050in}}{\pgfqpoint{-0.041667in}{0.000000in}}%
\pgfpathcurveto{\pgfqpoint{-0.041667in}{-0.011050in}}{\pgfqpoint{-0.037276in}{-0.021649in}}{\pgfqpoint{-0.029463in}{-0.029463in}}%
\pgfpathcurveto{\pgfqpoint{-0.021649in}{-0.037276in}}{\pgfqpoint{-0.011050in}{-0.041667in}}{\pgfqpoint{0.000000in}{-0.041667in}}%
\pgfpathclose%
\pgfusepath{stroke,fill}%
}%
\begin{pgfscope}%
\pgfsys@transformshift{5.534545in}{1.568192in}%
\pgfsys@useobject{currentmarker}{}%
\end{pgfscope}%
\end{pgfscope}%
\begin{pgfscope}%
\pgfsetrectcap%
\pgfsetmiterjoin%
\pgfsetlinewidth{0.803000pt}%
\definecolor{currentstroke}{rgb}{0.000000,0.000000,0.000000}%
\pgfsetstrokecolor{currentstroke}%
\pgfsetdash{}{0pt}%
\pgfpathmoveto{\pgfqpoint{0.800000in}{0.528000in}}%
\pgfpathlineto{\pgfqpoint{0.800000in}{4.224000in}}%
\pgfusepath{stroke}%
\end{pgfscope}%
\begin{pgfscope}%
\pgfsetrectcap%
\pgfsetmiterjoin%
\pgfsetlinewidth{0.803000pt}%
\definecolor{currentstroke}{rgb}{0.000000,0.000000,0.000000}%
\pgfsetstrokecolor{currentstroke}%
\pgfsetdash{}{0pt}%
\pgfpathmoveto{\pgfqpoint{5.760000in}{0.528000in}}%
\pgfpathlineto{\pgfqpoint{5.760000in}{4.224000in}}%
\pgfusepath{stroke}%
\end{pgfscope}%
\begin{pgfscope}%
\pgfsetrectcap%
\pgfsetmiterjoin%
\pgfsetlinewidth{0.803000pt}%
\definecolor{currentstroke}{rgb}{0.000000,0.000000,0.000000}%
\pgfsetstrokecolor{currentstroke}%
\pgfsetdash{}{0pt}%
\pgfpathmoveto{\pgfqpoint{0.800000in}{0.528000in}}%
\pgfpathlineto{\pgfqpoint{5.760000in}{0.528000in}}%
\pgfusepath{stroke}%
\end{pgfscope}%
\begin{pgfscope}%
\pgfsetrectcap%
\pgfsetmiterjoin%
\pgfsetlinewidth{0.803000pt}%
\definecolor{currentstroke}{rgb}{0.000000,0.000000,0.000000}%
\pgfsetstrokecolor{currentstroke}%
\pgfsetdash{}{0pt}%
\pgfpathmoveto{\pgfqpoint{0.800000in}{4.224000in}}%
\pgfpathlineto{\pgfqpoint{5.760000in}{4.224000in}}%
\pgfusepath{stroke}%
\end{pgfscope}%
\end{pgfpicture}%
\makeatother%
\endgroup%

\caption{Empirical probability that $\GMRES{\eps}{\no}{\nt}\leq 12$ for $\sigma = 1/k$\label{fig:prob-plot-1.0}}
\end{subfigure}

\begin{subfigure}{\textwidth}
    \centering
%% Creator: Matplotlib, PGF backend
%%
%% To include the figure in your LaTeX document, write
%%   \input{<filename>.pgf}
%%
%% Make sure the required packages are loaded in your preamble
%%   \usepackage{pgf}
%%
%% Figures using additional raster images can only be included by \input if
%% they are in the same directory as the main LaTeX file. For loading figures
%% from other directories you can use the `import` package
%%   \usepackage{import}
%% and then include the figures with
%%   \import{<path to file>}{<filename>.pgf}
%%
%% Matplotlib used the following preamble
%%   \usepackage{fontspec}
%%   \setmainfont{DejaVuSerif.ttf}[Path=/home/owen/progs/firedrake-complex/firedrake/lib/python3.5/site-packages/matplotlib/mpl-data/fonts/ttf/]
%%   \setsansfont{DejaVuSans.ttf}[Path=/home/owen/progs/firedrake-complex/firedrake/lib/python3.5/site-packages/matplotlib/mpl-data/fonts/ttf/]
%%   \setmonofont{DejaVuSansMono.ttf}[Path=/home/owen/progs/firedrake-complex/firedrake/lib/python3.5/site-packages/matplotlib/mpl-data/fonts/ttf/]
%%
\begingroup%
\makeatletter%
\begin{pgfpicture}%
\pgfpathrectangle{\pgfpointorigin}{\pgfqpoint{6.400000in}{4.800000in}}%
\pgfusepath{use as bounding box, clip}%
\begin{pgfscope}%
\pgfsetbuttcap%
\pgfsetmiterjoin%
\definecolor{currentfill}{rgb}{1.000000,1.000000,1.000000}%
\pgfsetfillcolor{currentfill}%
\pgfsetlinewidth{0.000000pt}%
\definecolor{currentstroke}{rgb}{1.000000,1.000000,1.000000}%
\pgfsetstrokecolor{currentstroke}%
\pgfsetdash{}{0pt}%
\pgfpathmoveto{\pgfqpoint{0.000000in}{0.000000in}}%
\pgfpathlineto{\pgfqpoint{6.400000in}{0.000000in}}%
\pgfpathlineto{\pgfqpoint{6.400000in}{4.800000in}}%
\pgfpathlineto{\pgfqpoint{0.000000in}{4.800000in}}%
\pgfpathclose%
\pgfusepath{fill}%
\end{pgfscope}%
\begin{pgfscope}%
\pgfsetbuttcap%
\pgfsetmiterjoin%
\definecolor{currentfill}{rgb}{1.000000,1.000000,1.000000}%
\pgfsetfillcolor{currentfill}%
\pgfsetlinewidth{0.000000pt}%
\definecolor{currentstroke}{rgb}{0.000000,0.000000,0.000000}%
\pgfsetstrokecolor{currentstroke}%
\pgfsetstrokeopacity{0.000000}%
\pgfsetdash{}{0pt}%
\pgfpathmoveto{\pgfqpoint{0.800000in}{0.528000in}}%
\pgfpathlineto{\pgfqpoint{5.760000in}{0.528000in}}%
\pgfpathlineto{\pgfqpoint{5.760000in}{4.224000in}}%
\pgfpathlineto{\pgfqpoint{0.800000in}{4.224000in}}%
\pgfpathclose%
\pgfusepath{fill}%
\end{pgfscope}%
\begin{pgfscope}%
\pgfsetbuttcap%
\pgfsetroundjoin%
\definecolor{currentfill}{rgb}{0.000000,0.000000,0.000000}%
\pgfsetfillcolor{currentfill}%
\pgfsetlinewidth{0.803000pt}%
\definecolor{currentstroke}{rgb}{0.000000,0.000000,0.000000}%
\pgfsetstrokecolor{currentstroke}%
\pgfsetdash{}{0pt}%
\pgfsys@defobject{currentmarker}{\pgfqpoint{0.000000in}{-0.048611in}}{\pgfqpoint{0.000000in}{0.000000in}}{%
\pgfpathmoveto{\pgfqpoint{0.000000in}{0.000000in}}%
\pgfpathlineto{\pgfqpoint{0.000000in}{-0.048611in}}%
\pgfusepath{stroke,fill}%
}%
\begin{pgfscope}%
\pgfsys@transformshift{1.025455in}{0.528000in}%
\pgfsys@useobject{currentmarker}{}%
\end{pgfscope}%
\end{pgfscope}%
\begin{pgfscope}%
\definecolor{textcolor}{rgb}{0.000000,0.000000,0.000000}%
\pgfsetstrokecolor{textcolor}%
\pgfsetfillcolor{textcolor}%
\pgftext[x=1.025455in,y=0.430778in,,top]{\color{textcolor}\sffamily\fontsize{10.000000}{12.000000}\selectfont 10}%
\end{pgfscope}%
\begin{pgfscope}%
\pgfsetbuttcap%
\pgfsetroundjoin%
\definecolor{currentfill}{rgb}{0.000000,0.000000,0.000000}%
\pgfsetfillcolor{currentfill}%
\pgfsetlinewidth{0.803000pt}%
\definecolor{currentstroke}{rgb}{0.000000,0.000000,0.000000}%
\pgfsetstrokecolor{currentstroke}%
\pgfsetdash{}{0pt}%
\pgfsys@defobject{currentmarker}{\pgfqpoint{0.000000in}{-0.048611in}}{\pgfqpoint{0.000000in}{0.000000in}}{%
\pgfpathmoveto{\pgfqpoint{0.000000in}{0.000000in}}%
\pgfpathlineto{\pgfqpoint{0.000000in}{-0.048611in}}%
\pgfusepath{stroke,fill}%
}%
\begin{pgfscope}%
\pgfsys@transformshift{2.528485in}{0.528000in}%
\pgfsys@useobject{currentmarker}{}%
\end{pgfscope}%
\end{pgfscope}%
\begin{pgfscope}%
\definecolor{textcolor}{rgb}{0.000000,0.000000,0.000000}%
\pgfsetstrokecolor{textcolor}%
\pgfsetfillcolor{textcolor}%
\pgftext[x=2.528485in,y=0.430778in,,top]{\color{textcolor}\sffamily\fontsize{10.000000}{12.000000}\selectfont 20}%
\end{pgfscope}%
\begin{pgfscope}%
\pgfsetbuttcap%
\pgfsetroundjoin%
\definecolor{currentfill}{rgb}{0.000000,0.000000,0.000000}%
\pgfsetfillcolor{currentfill}%
\pgfsetlinewidth{0.803000pt}%
\definecolor{currentstroke}{rgb}{0.000000,0.000000,0.000000}%
\pgfsetstrokecolor{currentstroke}%
\pgfsetdash{}{0pt}%
\pgfsys@defobject{currentmarker}{\pgfqpoint{0.000000in}{-0.048611in}}{\pgfqpoint{0.000000in}{0.000000in}}{%
\pgfpathmoveto{\pgfqpoint{0.000000in}{0.000000in}}%
\pgfpathlineto{\pgfqpoint{0.000000in}{-0.048611in}}%
\pgfusepath{stroke,fill}%
}%
\begin{pgfscope}%
\pgfsys@transformshift{4.031515in}{0.528000in}%
\pgfsys@useobject{currentmarker}{}%
\end{pgfscope}%
\end{pgfscope}%
\begin{pgfscope}%
\definecolor{textcolor}{rgb}{0.000000,0.000000,0.000000}%
\pgfsetstrokecolor{textcolor}%
\pgfsetfillcolor{textcolor}%
\pgftext[x=4.031515in,y=0.430778in,,top]{\color{textcolor}\sffamily\fontsize{10.000000}{12.000000}\selectfont 30}%
\end{pgfscope}%
\begin{pgfscope}%
\pgfsetbuttcap%
\pgfsetroundjoin%
\definecolor{currentfill}{rgb}{0.000000,0.000000,0.000000}%
\pgfsetfillcolor{currentfill}%
\pgfsetlinewidth{0.803000pt}%
\definecolor{currentstroke}{rgb}{0.000000,0.000000,0.000000}%
\pgfsetstrokecolor{currentstroke}%
\pgfsetdash{}{0pt}%
\pgfsys@defobject{currentmarker}{\pgfqpoint{0.000000in}{-0.048611in}}{\pgfqpoint{0.000000in}{0.000000in}}{%
\pgfpathmoveto{\pgfqpoint{0.000000in}{0.000000in}}%
\pgfpathlineto{\pgfqpoint{0.000000in}{-0.048611in}}%
\pgfusepath{stroke,fill}%
}%
\begin{pgfscope}%
\pgfsys@transformshift{5.534545in}{0.528000in}%
\pgfsys@useobject{currentmarker}{}%
\end{pgfscope}%
\end{pgfscope}%
\begin{pgfscope}%
\definecolor{textcolor}{rgb}{0.000000,0.000000,0.000000}%
\pgfsetstrokecolor{textcolor}%
\pgfsetfillcolor{textcolor}%
\pgftext[x=5.534545in,y=0.430778in,,top]{\color{textcolor}\sffamily\fontsize{10.000000}{12.000000}\selectfont 40}%
\end{pgfscope}%
\begin{pgfscope}%
\definecolor{textcolor}{rgb}{0.000000,0.000000,0.000000}%
\pgfsetstrokecolor{textcolor}%
\pgfsetfillcolor{textcolor}%
\pgftext[x=3.280000in,y=0.240809in,,top]{\color{textcolor}\sffamily\fontsize{10.000000}{12.000000}\selectfont \(\displaystyle k\)}%
\end{pgfscope}%
\begin{pgfscope}%
\pgfsetbuttcap%
\pgfsetroundjoin%
\definecolor{currentfill}{rgb}{0.000000,0.000000,0.000000}%
\pgfsetfillcolor{currentfill}%
\pgfsetlinewidth{0.803000pt}%
\definecolor{currentstroke}{rgb}{0.000000,0.000000,0.000000}%
\pgfsetstrokecolor{currentstroke}%
\pgfsetdash{}{0pt}%
\pgfsys@defobject{currentmarker}{\pgfqpoint{-0.048611in}{0.000000in}}{\pgfqpoint{0.000000in}{0.000000in}}{%
\pgfpathmoveto{\pgfqpoint{0.000000in}{0.000000in}}%
\pgfpathlineto{\pgfqpoint{-0.048611in}{0.000000in}}%
\pgfusepath{stroke,fill}%
}%
\begin{pgfscope}%
\pgfsys@transformshift{0.800000in}{2.376000in}%
\pgfsys@useobject{currentmarker}{}%
\end{pgfscope}%
\end{pgfscope}%
\begin{pgfscope}%
\definecolor{textcolor}{rgb}{0.000000,0.000000,0.000000}%
\pgfsetstrokecolor{textcolor}%
\pgfsetfillcolor{textcolor}%
\pgftext[x=0.614413in,y=2.323238in,left,base]{\color{textcolor}\sffamily\fontsize{10.000000}{12.000000}\selectfont 1}%
\end{pgfscope}%
\begin{pgfscope}%
\definecolor{textcolor}{rgb}{0.000000,0.000000,0.000000}%
\pgfsetstrokecolor{textcolor}%
\pgfsetfillcolor{textcolor}%
\pgftext[x=0.558857in,y=2.376000in,,bottom,rotate=90.000000]{\color{textcolor}\sffamily\fontsize{10.000000}{12.000000}\selectfont Number of GMRES iterations}%
\end{pgfscope}%
\begin{pgfscope}%
\pgfpathrectangle{\pgfqpoint{0.800000in}{0.528000in}}{\pgfqpoint{4.960000in}{3.696000in}}%
\pgfusepath{clip}%
\pgfsetbuttcap%
\pgfsetroundjoin%
\definecolor{currentfill}{rgb}{0.000000,0.000000,0.000000}%
\pgfsetfillcolor{currentfill}%
\pgfsetlinewidth{1.003750pt}%
\definecolor{currentstroke}{rgb}{0.000000,0.000000,0.000000}%
\pgfsetstrokecolor{currentstroke}%
\pgfsetdash{}{0pt}%
\pgfsys@defobject{currentmarker}{\pgfqpoint{-0.041667in}{-0.041667in}}{\pgfqpoint{0.041667in}{0.041667in}}{%
\pgfpathmoveto{\pgfqpoint{0.000000in}{-0.041667in}}%
\pgfpathcurveto{\pgfqpoint{0.011050in}{-0.041667in}}{\pgfqpoint{0.021649in}{-0.037276in}}{\pgfqpoint{0.029463in}{-0.029463in}}%
\pgfpathcurveto{\pgfqpoint{0.037276in}{-0.021649in}}{\pgfqpoint{0.041667in}{-0.011050in}}{\pgfqpoint{0.041667in}{0.000000in}}%
\pgfpathcurveto{\pgfqpoint{0.041667in}{0.011050in}}{\pgfqpoint{0.037276in}{0.021649in}}{\pgfqpoint{0.029463in}{0.029463in}}%
\pgfpathcurveto{\pgfqpoint{0.021649in}{0.037276in}}{\pgfqpoint{0.011050in}{0.041667in}}{\pgfqpoint{0.000000in}{0.041667in}}%
\pgfpathcurveto{\pgfqpoint{-0.011050in}{0.041667in}}{\pgfqpoint{-0.021649in}{0.037276in}}{\pgfqpoint{-0.029463in}{0.029463in}}%
\pgfpathcurveto{\pgfqpoint{-0.037276in}{0.021649in}}{\pgfqpoint{-0.041667in}{0.011050in}}{\pgfqpoint{-0.041667in}{0.000000in}}%
\pgfpathcurveto{\pgfqpoint{-0.041667in}{-0.011050in}}{\pgfqpoint{-0.037276in}{-0.021649in}}{\pgfqpoint{-0.029463in}{-0.029463in}}%
\pgfpathcurveto{\pgfqpoint{-0.021649in}{-0.037276in}}{\pgfqpoint{-0.011050in}{-0.041667in}}{\pgfqpoint{0.000000in}{-0.041667in}}%
\pgfpathclose%
\pgfusepath{stroke,fill}%
}%
\begin{pgfscope}%
\pgfsys@transformshift{1.025455in}{2.376000in}%
\pgfsys@useobject{currentmarker}{}%
\end{pgfscope}%
\end{pgfscope}%
\begin{pgfscope}%
\pgfpathrectangle{\pgfqpoint{0.800000in}{0.528000in}}{\pgfqpoint{4.960000in}{3.696000in}}%
\pgfusepath{clip}%
\pgfsetbuttcap%
\pgfsetroundjoin%
\definecolor{currentfill}{rgb}{0.000000,0.000000,0.000000}%
\pgfsetfillcolor{currentfill}%
\pgfsetlinewidth{1.003750pt}%
\definecolor{currentstroke}{rgb}{0.000000,0.000000,0.000000}%
\pgfsetstrokecolor{currentstroke}%
\pgfsetdash{}{0pt}%
\pgfsys@defobject{currentmarker}{\pgfqpoint{-0.041667in}{-0.041667in}}{\pgfqpoint{0.041667in}{0.041667in}}{%
\pgfpathmoveto{\pgfqpoint{0.000000in}{-0.041667in}}%
\pgfpathcurveto{\pgfqpoint{0.011050in}{-0.041667in}}{\pgfqpoint{0.021649in}{-0.037276in}}{\pgfqpoint{0.029463in}{-0.029463in}}%
\pgfpathcurveto{\pgfqpoint{0.037276in}{-0.021649in}}{\pgfqpoint{0.041667in}{-0.011050in}}{\pgfqpoint{0.041667in}{0.000000in}}%
\pgfpathcurveto{\pgfqpoint{0.041667in}{0.011050in}}{\pgfqpoint{0.037276in}{0.021649in}}{\pgfqpoint{0.029463in}{0.029463in}}%
\pgfpathcurveto{\pgfqpoint{0.021649in}{0.037276in}}{\pgfqpoint{0.011050in}{0.041667in}}{\pgfqpoint{0.000000in}{0.041667in}}%
\pgfpathcurveto{\pgfqpoint{-0.011050in}{0.041667in}}{\pgfqpoint{-0.021649in}{0.037276in}}{\pgfqpoint{-0.029463in}{0.029463in}}%
\pgfpathcurveto{\pgfqpoint{-0.037276in}{0.021649in}}{\pgfqpoint{-0.041667in}{0.011050in}}{\pgfqpoint{-0.041667in}{0.000000in}}%
\pgfpathcurveto{\pgfqpoint{-0.041667in}{-0.011050in}}{\pgfqpoint{-0.037276in}{-0.021649in}}{\pgfqpoint{-0.029463in}{-0.029463in}}%
\pgfpathcurveto{\pgfqpoint{-0.021649in}{-0.037276in}}{\pgfqpoint{-0.011050in}{-0.041667in}}{\pgfqpoint{0.000000in}{-0.041667in}}%
\pgfpathclose%
\pgfusepath{stroke,fill}%
}%
\begin{pgfscope}%
\pgfsys@transformshift{2.528485in}{2.376000in}%
\pgfsys@useobject{currentmarker}{}%
\end{pgfscope}%
\end{pgfscope}%
\begin{pgfscope}%
\pgfpathrectangle{\pgfqpoint{0.800000in}{0.528000in}}{\pgfqpoint{4.960000in}{3.696000in}}%
\pgfusepath{clip}%
\pgfsetbuttcap%
\pgfsetroundjoin%
\definecolor{currentfill}{rgb}{0.000000,0.000000,0.000000}%
\pgfsetfillcolor{currentfill}%
\pgfsetlinewidth{1.003750pt}%
\definecolor{currentstroke}{rgb}{0.000000,0.000000,0.000000}%
\pgfsetstrokecolor{currentstroke}%
\pgfsetdash{}{0pt}%
\pgfsys@defobject{currentmarker}{\pgfqpoint{-0.041667in}{-0.041667in}}{\pgfqpoint{0.041667in}{0.041667in}}{%
\pgfpathmoveto{\pgfqpoint{0.000000in}{-0.041667in}}%
\pgfpathcurveto{\pgfqpoint{0.011050in}{-0.041667in}}{\pgfqpoint{0.021649in}{-0.037276in}}{\pgfqpoint{0.029463in}{-0.029463in}}%
\pgfpathcurveto{\pgfqpoint{0.037276in}{-0.021649in}}{\pgfqpoint{0.041667in}{-0.011050in}}{\pgfqpoint{0.041667in}{0.000000in}}%
\pgfpathcurveto{\pgfqpoint{0.041667in}{0.011050in}}{\pgfqpoint{0.037276in}{0.021649in}}{\pgfqpoint{0.029463in}{0.029463in}}%
\pgfpathcurveto{\pgfqpoint{0.021649in}{0.037276in}}{\pgfqpoint{0.011050in}{0.041667in}}{\pgfqpoint{0.000000in}{0.041667in}}%
\pgfpathcurveto{\pgfqpoint{-0.011050in}{0.041667in}}{\pgfqpoint{-0.021649in}{0.037276in}}{\pgfqpoint{-0.029463in}{0.029463in}}%
\pgfpathcurveto{\pgfqpoint{-0.037276in}{0.021649in}}{\pgfqpoint{-0.041667in}{0.011050in}}{\pgfqpoint{-0.041667in}{0.000000in}}%
\pgfpathcurveto{\pgfqpoint{-0.041667in}{-0.011050in}}{\pgfqpoint{-0.037276in}{-0.021649in}}{\pgfqpoint{-0.029463in}{-0.029463in}}%
\pgfpathcurveto{\pgfqpoint{-0.021649in}{-0.037276in}}{\pgfqpoint{-0.011050in}{-0.041667in}}{\pgfqpoint{0.000000in}{-0.041667in}}%
\pgfpathclose%
\pgfusepath{stroke,fill}%
}%
\begin{pgfscope}%
\pgfsys@transformshift{4.031515in}{2.376000in}%
\pgfsys@useobject{currentmarker}{}%
\end{pgfscope}%
\end{pgfscope}%
\begin{pgfscope}%
\pgfpathrectangle{\pgfqpoint{0.800000in}{0.528000in}}{\pgfqpoint{4.960000in}{3.696000in}}%
\pgfusepath{clip}%
\pgfsetbuttcap%
\pgfsetroundjoin%
\definecolor{currentfill}{rgb}{0.000000,0.000000,0.000000}%
\pgfsetfillcolor{currentfill}%
\pgfsetlinewidth{1.003750pt}%
\definecolor{currentstroke}{rgb}{0.000000,0.000000,0.000000}%
\pgfsetstrokecolor{currentstroke}%
\pgfsetdash{}{0pt}%
\pgfsys@defobject{currentmarker}{\pgfqpoint{-0.041667in}{-0.041667in}}{\pgfqpoint{0.041667in}{0.041667in}}{%
\pgfpathmoveto{\pgfqpoint{0.000000in}{-0.041667in}}%
\pgfpathcurveto{\pgfqpoint{0.011050in}{-0.041667in}}{\pgfqpoint{0.021649in}{-0.037276in}}{\pgfqpoint{0.029463in}{-0.029463in}}%
\pgfpathcurveto{\pgfqpoint{0.037276in}{-0.021649in}}{\pgfqpoint{0.041667in}{-0.011050in}}{\pgfqpoint{0.041667in}{0.000000in}}%
\pgfpathcurveto{\pgfqpoint{0.041667in}{0.011050in}}{\pgfqpoint{0.037276in}{0.021649in}}{\pgfqpoint{0.029463in}{0.029463in}}%
\pgfpathcurveto{\pgfqpoint{0.021649in}{0.037276in}}{\pgfqpoint{0.011050in}{0.041667in}}{\pgfqpoint{0.000000in}{0.041667in}}%
\pgfpathcurveto{\pgfqpoint{-0.011050in}{0.041667in}}{\pgfqpoint{-0.021649in}{0.037276in}}{\pgfqpoint{-0.029463in}{0.029463in}}%
\pgfpathcurveto{\pgfqpoint{-0.037276in}{0.021649in}}{\pgfqpoint{-0.041667in}{0.011050in}}{\pgfqpoint{-0.041667in}{0.000000in}}%
\pgfpathcurveto{\pgfqpoint{-0.041667in}{-0.011050in}}{\pgfqpoint{-0.037276in}{-0.021649in}}{\pgfqpoint{-0.029463in}{-0.029463in}}%
\pgfpathcurveto{\pgfqpoint{-0.021649in}{-0.037276in}}{\pgfqpoint{-0.011050in}{-0.041667in}}{\pgfqpoint{0.000000in}{-0.041667in}}%
\pgfpathclose%
\pgfusepath{stroke,fill}%
}%
\begin{pgfscope}%
\pgfsys@transformshift{5.534545in}{2.376000in}%
\pgfsys@useobject{currentmarker}{}%
\end{pgfscope}%
\end{pgfscope}%
\begin{pgfscope}%
\pgfsetrectcap%
\pgfsetmiterjoin%
\pgfsetlinewidth{0.803000pt}%
\definecolor{currentstroke}{rgb}{0.000000,0.000000,0.000000}%
\pgfsetstrokecolor{currentstroke}%
\pgfsetdash{}{0pt}%
\pgfpathmoveto{\pgfqpoint{0.800000in}{0.528000in}}%
\pgfpathlineto{\pgfqpoint{0.800000in}{4.224000in}}%
\pgfusepath{stroke}%
\end{pgfscope}%
\begin{pgfscope}%
\pgfsetrectcap%
\pgfsetmiterjoin%
\pgfsetlinewidth{0.803000pt}%
\definecolor{currentstroke}{rgb}{0.000000,0.000000,0.000000}%
\pgfsetstrokecolor{currentstroke}%
\pgfsetdash{}{0pt}%
\pgfpathmoveto{\pgfqpoint{5.760000in}{0.528000in}}%
\pgfpathlineto{\pgfqpoint{5.760000in}{4.224000in}}%
\pgfusepath{stroke}%
\end{pgfscope}%
\begin{pgfscope}%
\pgfsetrectcap%
\pgfsetmiterjoin%
\pgfsetlinewidth{0.803000pt}%
\definecolor{currentstroke}{rgb}{0.000000,0.000000,0.000000}%
\pgfsetstrokecolor{currentstroke}%
\pgfsetdash{}{0pt}%
\pgfpathmoveto{\pgfqpoint{0.800000in}{0.528000in}}%
\pgfpathlineto{\pgfqpoint{5.760000in}{0.528000in}}%
\pgfusepath{stroke}%
\end{pgfscope}%
\begin{pgfscope}%
\pgfsetrectcap%
\pgfsetmiterjoin%
\pgfsetlinewidth{0.803000pt}%
\definecolor{currentstroke}{rgb}{0.000000,0.000000,0.000000}%
\pgfsetstrokecolor{currentstroke}%
\pgfsetdash{}{0pt}%
\pgfpathmoveto{\pgfqpoint{0.800000in}{4.224000in}}%
\pgfpathlineto{\pgfqpoint{5.760000in}{4.224000in}}%
\pgfusepath{stroke}%
\end{pgfscope}%
\end{pgfpicture}%
\makeatother%
\endgroup%

\caption{Empirical probability that $\GMRES{\eps}{\no}{\nt}\leq 12$ for $\sigma = 1.$\label{fig:prob-plot-2.0}}
\end{subfigure}
\caption{Empirical probability (calculated from 1000 realisations) that $\GMRES{\eps}{\no}{\nt}\leq 12$ for $k = 10, 20, 30, 40,$ where $R=12$, $\eps = 10^{-5}$, $N = \ceil{k^{3}}$, and $\Ct=0.1,$ with $\NLiDRR{\no-\nt} \sim \Exp{\sigma}$ for different functional forms of $\sigma.$}
\end{figure}


\section{Applying nearby preconditioning to Quasi-Monte-Carlo methods for the Helmholtz equation}\label{sec:nbpcqmc}

We now apply nearby preconditioning to a Quasi-Monte-Carlo (QMC) method for the Helmholtz equation. We begin with a brief description of QMC methods, before detailing two ways we apply nearby preconditioning to these methods. Finally, we give computational results for applying nearby preconditioning to QMC methods for the Helmholtz equation.

\subsection{Brief description of QMC}

QMC methods (or rules) are high-dimensional quadrature rules, designed to yield superior rates of convergence (with respect to the number of integration points) compared to Monte-Carlo methods. Suppose one wants to approximate $\EXP{Q},$ where $Q$ is some random variable (later in this \lcnamecref{sec:nbpcqmc}, $Q$ will be a function of the solution $u(\omega)$ of a stochastic Helmholtz equation). By definition, the expectation is
\beq\label{eq:qmcexpdef}
\EXP{Q} = \int_\Omega Q(\omega)\ \ddPPomega.
\eeq

If we now suppose $Q$ depends on the sample space $\Omega$ via a finite set of random variables $\Uo,\ldots,\UJ$, then we can rewrite \cref{eq:qmcexpdef} as
\beq\label{eq:qmcexp2}
\EXP{Q} = \int_\Omega Q\mleft((\Uo(\omega),\ldots,\UJ(\omega)\mright)\, \ddPPomega.
\eeq
If, for example, the $\Uj$ are all Uniform random variables on $\mleft[-1/2,1/2\mright]$, then \cref{eq:qmcexp2} can be rewritten as
\beq\label{eq:qmcexp3}
\EXP{Q} = \int_{\cube{J}} Q\mleft(\by\mright)\, \dd\lambda(\by),
\eeq
where $\lambda$ denotes Lebesgue measure.

Any quadrature rule, or method for approximating $\EXP{Q}$, can then be seen as a method for approximating the $J$-dimensional integral on the right-hand side of \cref{eq:qmcexp3} and vice-versa. Equal-weight quadrature rules choose points $\byo,\ldots,\byNpoints \in \cube{J}$ and use the approximation
\beqs
\EXP{Q} \approx \frac1{\Npoints}\sum_{l=1}^{\Npoints} Q\mleft(\byl\mright).
\eeqs
Monte-Carlo and Quasi-Monte-Carlo rules are methods for choosing the points $\byl$. In a Monte-Carlo rule the points are chosen at random in accordance with the associated probability distribution. For example, in the case that the $\Uj$ are $\Unif(-1/2,1/2)$ random variables, the points $\byl$ are chosen according to the Uniform distribution on $\cube{J}$. Observe that Monte-Carlo rules do not need the dependence of $Q$ on $\omega$ to take the form prescribed above, they apply to any random variable.

Quasi-Monte-Carlo rules, in contrast to Monte-Carlo rules, do require the dependence on $\omega$ to be via finitely- or countable-many random variables, as QMC rules are high-dimensional quadrature rules (performing quadrature on the high-dimensional cube $\cube{J}).$ In QMC rules the points $\byl$ are not chosen completely at random, unlike Monte-Carlo rules.

The main advantage of QMC rules is that they can exhibit higher rates of convergence compared to Monte-Carlo rules; Monte Carlo rules typically converge with rate $\Npoints^{-1/2}$ (see\ednote{I'll show this in the MLMC chapter, and add in a reference once it's done.}), whereas QMC rules can converge with rates up to $\Npoints^{-1}$  or with even higher rates for higher-order QMC rules, see, e.g., \cite[Penultimate paragraph of Section 1.2]{KuNu:16}.

In applying QMC rules to stochastic PDEs, we assume that the random coefficient ($n$ in our case) is dependent on finitely many (or countably many) random variables, as in \cref{eq:qmcexp2} above, and we then use QMC rules to estimate expectations of quantities of interest of the solution $u$, i.e., $Q = Q(u).$ We note that applying QMC rules to stochastic PDEs is a vibrant and active research area. For recent overviews of this field, see \cite{KuNu:16,KuNu:18b} (and the associated tutorial \cite{KuNu:18a}). We note that there is currently no rigorous study of how QMC methods behave for the Helmholtz equation, although we understand such work is currently underway by Ganesh, Kuo, and Sloan \cite{GaKuSl}.

\subsection{Methods for applying nearby preconditioning to QMC}\label{sec:nbpcqmcnum}
In all of our previous uses of nearby preconditioning, we have fixed $\nso$, the value for which we calculate the preconditioner, and have then used $\Amato$ to precondition $\Amatt$ for different values of $\nst.$ However, the key idea for applying nearby preconditioning to QMC methods for the Helmholtz equation is to choose \emph{a number} of different realisations of $\nso$ and use each realisation of $\nso$ as a preconditioner only for those  realisations of $\nst$ for which $\Amato$ is a good preconditioner for $\Amatt.$ We adopt this approach because it is highly unlikely that a single realisation of $\nso$ will be a good preconditioner for every realisation of $\nst.$

Therefore, the algorithms presented in this \lcnamecref{sec:nbpcqmcnum} seek to answer the two questions:
\ben
\item For which realisations of $n$ should a preconditioner be calculated?
  \item To which realisations of $n$ should each preconditioner be applied?
\een

We now detail two methods for using nearby preconditioning to speed up QMC methods for the Helmholtz equation. To apply these methods, we use the following model problem: We consider the Interior Impedance Problem in 2-d with $f=1$ and $\gI=0$, $A = I$, and $n$ given by
\beq\label{eq:artificialkl}
n(\omega,\bx) = 1 + \sum_{j=1}^{10} \Uj(\omega) \sqrt{\lambdaj} \psij(\bx),
\eeq
where
\beq\label{eq:artificialkllambdas}
\sqrt{\lambdaj} = j^{-2}
\eeq
and
\beq\label{eq:artificialklfuns}
\psij(\bx) = \cos\mleft(\frac{j\pi}4 x\mright)\cos\mleft(\frac{\mleft(j+1\mright)\pi}4 y\mright).
\eeq
Observe that $\NLiDRR{\psij}=1$ for all $j,$ and $\sqrt{\lambdaj} \rightarrow 0$ as $j \rightarrow \infty.$ This expansion is based on the random-field expansion used in \cite[Section 5.1]{GiGrKuScSl:19}, although the main change we make from \cite{GiGrKuScSl:19} is to introduce the factors $1/4$ in \cref{eq:artificialklfuns}. We introduce this factor to ensure that the oscillations in the medium $n$ are `low frequency' compared to the frequency $k$ of the waves passing through the medium. Expansions similar to \cref{eq:artificialkl} are often decribed as `artificial Karhunen--Lo\`eve expansions' due to their similarity with the Karhunen--Lo\`eve expansion of a random field. In a Karhunen--Lo\`eve expansion the $\Uj$ are independent random variables whose distribution is determined by the distribution of the random field, and the $\lambdaj$ and $\psij$ are the eigenvalues and eigenvectors of the covariance operator, see, e.g., \cite[Section 7.4]{LoPoSh:14}. In view of the fact that we will be using QMC methods to approximate $\EXP{Q(u)}$ (for some quantities of interest $Q$) we will sometimes instead write $n(\by)$ for $\by \in \cube{10}$, by which we mean
\beqs
n(\by) = 1 + \sum_{j=1}^{10} \by_{j} \sqrt{\lambdaj} \psij.
\eeqs
There is no a priori reason that one must have such an affine dependence of the random field on the randomness in order to apply nearby preconditioning to QMC methods. One could, for example, take $n$ to be a lognormal random field, in which case $n$ would take the form $n(\by) = \exp\mleft(\nz + \sum_j \Nj \sqrt{\lambdaj} \psij\mright)$ where the $\Nj$ are Normal$(0,1)$ random variables.. However, affine dependence will allow us to easily define the so-called `parallelisable' nearby-preconditioning-QMC algorithm below.

We stress that the results in this \lcnamecref{sec:nbpcqmcnum} are strictly numerical; there is no current theory to support these calculations. In particular, we show in \cref{sec:nbpcqmcnumerics} below that it appears that for the QMC error for Helmholtz problems to remain bounded as $k$ increases, one must increase the number of QMC points with $k.$ We again remark that there is currently no theoretical justification for this behaviour.


\paragraph{Terminology} Before we describe the nearby-preconditioning-QMC algorithms in detail we establish two pieces of terminology that will be of use in describing these algorithms. Firstly, we will use the word `point' to refer to a point in the parameter space $\cube{J}$, and use phrases such as `calculate a preconditioner at the point $\by$' as shorthand for `calculate the preconditioner corresponding to the finite-element discretisation of the Helmholtz IIP (as described above) with coefficient $n(\by)$'.

We we alse use the words `nearby' and `nearest' (when referring to QMC points) to mean: nearest in the metric
\beq\label{eq:dmqc}
\dQMC(\byo,\byt) = \NLiDRRR{n(\byo)-n(\byt)},
\eeq
i.e., the metric on $\cube{J}$ induced by the spatial $L^\infty$ norm. We use this metric to describe the geometry of our QMC points as our results in \cref{sec:intronbpc} above indicate that it is the $L^\infty$-norm of the different in the coefficients that dictates the effectiveness of nearby preconditioning\footnote{Although, in line with the results in \cref{sec:weaknorm}, we could instead use a spatial $L^q$ norm, for some $q \geq 1$ in \cref{eq:dqmc}.}. Therefore, when considering which QMC points will yield preconditioners suitable for use with other QMC points, this metric is a natural metric to use.


\subsubsection{A sequential algorithm}
We first describe a straightforward algorithm that uses nearby preconditioning to speed up QMC calculation for the Helmholtz equation. We call this a `sequential' algorithm because, unlike the `parallel' algorithm we describe below, it is intrinsically sequential and cannot be parallelised, i.e., finite-element solves for different realisations of the random field $n$ cannot be treated in parallel. One can, of course, use parallelisation for inidividual finite-element solves, if the linear systems $\Amat$ are large enough to warrant this.

An overview of the algorithm is:
\ben
\item Choose a point $\by$ for which to calculate a preconditioner
\item\label[itemstep]{it:nearest} Find the nearest non-computed point and attempt a preconditioned GMRES solve at that point.
    \item If GMRES converges quickly (i.e., in fewer than a preset number of iterations), return to \cref{it:nearest}.
\item If GMRES takes too long to converge, recalculate the preconditioner at the current point, and return to \cref{it:nearest}.
  \een
  The algorithm is written in more formal pseudocode in \cref{alg:seq}.
\begin{algorithm}[h]
\DontPrintSemicolon
\SetKwInOut{Input}{input}\SetKwInOut{Output}{output}
\SetKwFunction{Nearest}{nearest}

\Input{$\maxGMRES$,$\SQMC$}
\BlankLine
Choose starting point $\bystart$\;
$\bypre \defined \bystart$\;
$\Sremaining \defined \SQMC\setminus\set{\bypre}$\;
Calculate preconditioner $\Lmat\Umat = \AmatpreI$\;
$\bycurrent \defined$ \Nearest{$\bypre,\Sremaining$}\;
\While{$\Sremaining \neq \emptyset$}{
\eIf{GMRES applied to $\Lmat\Umat\Amat\bycurrent = \Lmat\Umat \bb$ converges in fewer than $\maxGMRES$ iterations}{
$\Sremaining \defined \Sremaining\setminus\set{\bycurrent}$\;
$\bycurrent \defined$ \Nearest{$\bypre,\Sremaining$}\;
}{
$\bypre \defined \bycurrent$\;
Calculate preconditioner $\Lmat\Umat = \AmatpreI$\;
}
}
\caption{Algorithm to perform all solves in a QMC method using nearby preconditioning\label{alg:seq}. $\maxGMRES$ is the maximum allowed number of GMRES iterations and $\SQMC$ is the set of all QMC points. $\nearest(\bypre,\Sremaining)$ denotes the point in $\SQMC$ nearest to $\bypre$ in the $\dQMC$ metric.}
\end{algorithm}
\subsubsection{A parallel algorithm}

The main disadvantage of the `sequential' algorithm described above is that the points at which preconditioners are calculated are identified as the algorithm progresses. Therefore the algorithm cannot be parallelised by sending different collections of QMC points to different processors (as then one would need to know which preconditioner to use for each point at the start of the algorithm). Therefore, we now suggest an alternative algorithm that allows one to specify the number of preconditioning points \emph{before} the algorithm begins. The algorithm then calculates which points to use as preconditioning points, before performing the linear solves. Because the preconditioners are known in advance, the solves can be computed in parallel if required. The most complicated part of the algorithm is deciding at which points to calculate the preconditioners, and so we describe this part of the algorithm in more detail here. A more formal pseudocode description of the algorithm is given in \cref{alg:par}.

Suppose we are given a set $\SQMC = \set{\byo,\ldots,\byNQMC}$ of QMC points and a number $\Npretarget$; the target number of preconditioners to compute. The aim of this algorithm is to select (approximately) $\Npretarget$ QMC points that are (approximately) equally spaced with respect to the $\dQMC$ metric defined above. If such a goal is acheived, then one expects that the preconditioning points are best located to minimise the total number of GMRES iterations across the solves for all of the QMC points.

The algorithm contains two key ideas:
\ben
  \item Use a surrogate metric in place of $\dQMC$, and
\item Locate the preconditioning points according to a tensor-product rule.
  \een
  We now describe each of these two ideas in turn.

\paragraph{Surrogate metric} Whilst the metric $\dQMC$ is the metric related to the performance of nearby preconditioning (as described in \cref{sec:intronbpc} above), in practice $\dQMC$ is difficult to work with; it is not obvious what geometry it induces on $\cube{J}$. Therefore, we work in an alternative, although related metric
  \beqs
\dapprox(\byo,\byt) \de \sum_{j=1}^{J} \sqrt{\lambdaj} \abs{{\byo}_{j} - {\byt}_{j}}.
\eeqs
Observe that $\dapprox$ is a weighted $L^1$ metric on $\cube{J}$, with the weights corresponding to the terms in \cref{eq:artificialkl}. Recall that $\sqrt{\lambdaj} \rightarrow 0$ as $j \rightarrow \infty$; therefore the higher dimensions contribute less to the value of $\dapprox$ (or, informally, points are `closer' in higher dimensions, or higher dimensions are `smaller' than lower dimensions). However, it is easy to compute with $\dapprox,$ and it is obvious that it enables one to think of $\cube{J}$ as the high-dimensional rectangle $\mleft[0,\sqrt{\lambdao}\mright]\times\cdots\times\mleft[0,\sqrt{\lambdaJ}\mright]$ equipped with the standard $L^1$ metric. Also observe that the structure of $\dapprox$ is similar to that of $\dQMC$ and for all $\byo,\byt \in \cube{J},$ the inequality $\dQMC\mleft(\byo,\byt\mright) \leq \dapprox\mleft(\byo,\byt\mright)$ holds. Therefore we expect $\dapprox$ is a reasonable approximation of $\dQMC.$

\paragraph{Tensor-product rule} To understand why we use locate the preconditioning points using a tensor-product rule, we first decribe the heuristic we use. Let us assume we want to cover $\cube{J}$ with balls of radius $r$ (where these balls are measured in the $\dapprox$ metric). Therefore, given the centres of two adjacent balls $\bcone$ and $\bct$, then we will have
\beq\label{eq:centres2r}
\dapprox(\bcone,\bct) = 2r.
\eeq
The question now arises of how we choose $\bcone$ and $\bct$ so that \cref{eq:centres2r} holds. We observe that, by the definition of $\dapprox$, if we choose $\bcone$ and $\bct$ such that
\beqs
\sqrt{\lambdaj}\abs{{\bcone}_{j}-{\bct}_j} = \frac{2r}{J},
\eeqs
then we will have \cref{eq:centres2r} by construction, because
\beqs
\dapprox(\bcone,\bct) = \sum_{j=1}^J \frac{2r}J = 2r.
\eeqs
Therefore, in dimension $j$ we choose the centres of the balls to be spaced
\beqs
\min\set{\frac{2r}{J\sqrt{\lambdaj}},1}
\eeqs
apart (where we include the minimum so that, for high dimensions, we include at least one centre). That is, in dimension $j$, we take
\beqs
\Nj \de \max\set{1,\frac{J\sqrt{\lambdaj}}{2r}}
\eeqs
equally spaced points in the sets $\centresj = \set{c_{j,1},\ldots,c_{j,\Nj}},$ and then we form the centres $\bcone,\ldots,\bcNpre$ by taking tensor products of the points in $\centreso,\ldots,\centresJ,$ giving a total of $\Npre = \No \times \cdots \times \NJ$ preconditioning points.

However, there are three immediate objections to the above approach:
\ben
\item The above procedure assumes we know the radius of the balls we wish to construct, and then returns the total number of preconditioning points, and their locations. However, we only know in advance the ideal total number of preconditioning points.
\item There is no guarantee that the numbers of points $\Nj$ calculated above are integers.
  \item There is no guarantee the preconditioning points given by the above procedure are QMC points.
    \een
    These questions are all completely valid, and so we slightly modify the above procedure to deal with them.

    Recall that we assume that we are given a target number of preconditioners $\Npretarget$. The above procedure (amongst other things) defines a map $\Npreideal:\RRp \rightarrow \RRp$ given by $r \mapsto \Npre.$ Therefore we can calculate numerically the value $\rideal$ such that $\Npreideal(\rideal) = \Npretarget.$ (In our computations, we do this via interval bisection.)

    Once we know $\rideal,$ we can calculate the numbers of centres in each dimension $\No(\rideal),\ldots,\NJ(\rideal)$ as above (recalling that the $\Nj(\rideal)$ are not necessarily integers). We then obtain integers $\Npreactualj = \round{\Nj(\rideal)}$, where $\round{\cdot}$ denotes rounding to the nearest integer. (Recall $\Nj(\rideal) \geq 1$ for all $j$ by construction, so $\Npreactualj$ will be a positive integer for all $j.$)

We then take $\Npreactualj$ centres in each dimension, as described above. We then obtain a total of $\Npreactual = \Npreactualo \times \cdots \times \NpreactualJ$ preconditioning points.

These points may not be QMC points. We could simply calculate the preconditioners at these non-QMC points. However we instead replace each calculated centre with its nearest QMC point and calculate the preconditioners at these QMC points.

    This algorithm is summarised more formally in \cref{alg:par}.
    
    

%% Define
%% \beqs
%% \Npreidealj(r) = \max\set{\frac{J \sqrt{\lambdaj}}{2r},1}.
%% \eeqs
%% The `ideal' total number of QMC points is
%% \beqs
%% \Npreideal(r)=\prod_{j=1}^J  \Npreidealj(r)
%% \eeqs

%% Want to calculate the number of preconditioners $\Npre$, the set
%% \beqs
%% \Spre=\set{\ypreo,\ldots,\ypreNpre}
%% \eeqs
%% of QMC points at which to calculate the preconditioner and the map
%% \beqs
%% \nearestpre:\SQMC\rightarrow\Spre
%% \eeqs
%% taking each QMC point to its nearest (in the induced spatial $L^\infty$ norm) preconditioner, where $\SQMC$ is the set of QMC points.

\begin{algorithm}[h]
\DontPrintSemicolon
\SetKwInOut{Input}{input}\SetKwInOut{Output}{output}
\SetKwFunction{Round}{round}

\Input{$\Npretarget \in \NN$}
\Output{$\Spre$, $\nearestpre$}
\BlankLine
Solve (numerically) $\Npreideal(\rideal) = \Npretarget$ for $\rideal$\;
\For{j $= 1$ \KwTo $J$}{
Calculate $\Npreactualj =$ \Round{$\Npreidealj(\rideal)$}\;
Define $\Sprej$ to be set of $\Npreactualj$ equally spaced points in $\mleft[-1/2,1/2\mright]$\;
}
Define $\displaystyle\Npre = \prod_{j=1}^J \Npreactualj$\;
Define $\Spre$ by taking all possible tensor products of points in $\Sprej$\;
\For{l $=1$ \KwTo $\NQMC$}{
Calculate $\nearestpre\mleft(\by^{(l)}\mright)$ by brute force\;
}
\caption{Algorithm to determine $\Spre$ and $\nearestpre$\label{alg:par}. $\Spre$ is the set of preconditioning points, and $\nearestpre:\SQMC\rightarrow\Spre$ maps each QMC points to its nearest preconditioning point.}
\end{algorithm}

\subsubsection{Advantages and disadvantages of each method}

The advantages of the sequential algorithm are:
\bit
\item Its simplicity; the algorithm is simple and intuitive to describe, and
\item Its lack of heuristics - one only needs to specify the maximum number of GMRES iterations; this could be determined, for example by the memory constraints of the machine one is using.
    \eit
    However, the disadvantages of the sequential algorithm are:
    \bit
  \item The algorithm is inherently serial; one must see whether a given solve converges in the required number of GMRES iterations before knowing whether we must recalculate the preconditioner for subsequent solves. (In principle one could parallelise the algorithm by splitting the QMC points up onto different groups of processors, and then use the sequential algorithm on each group of processors. However, there is no guarantee one would split the QMC points up in a way that grouped nearby points, therefore this approach could lead to a substantial increase in computational work.)
    \item There is no guarantee that this method for exploring the sample space and choosing the preconditioning points will yield an optimal collection of preconditioning points (optimal in the sense of the minimal number of preconditioning points needed).
    \eit

    The advantages and disadvantages of the parallel algorithm are, by and large, the reverse of those for the sequential algorithm. The advantages of the parallel algorithm are:
    \bit
  \item The algorithm is fully parallelisable; once the preconditioning points and the map $\nearestpre$ from the QMC points to the preconditioning points have been calculated, one can send different linear solves to different groups of processors as one chooses. (Although note that, unless one sends all of the QMC points corresponding to a single preconditioner to the \emph{same} group of processors, one may need to calculate the same preconditioner multiple times, on different groups of processes\footnote{In our code, we split up the points with respect to the order they are generated by the QMC code. This was purely to make the code simpler.} However, the decrease in computational time gained from parallelisation should more than offset this increase in computational effort.)
    \item The preconditioning points should fill the parameter space `well'. Given the points are chosen a priori to be well spaced according to the $\dapprox$ metric, one expects they will be a close to optimal collection (in the sense described above).
      \eit
      The disadvantages of the parallel algorithm are:
      \bit
    \item One needs a heuristic for how many preconditioning points to choose, as this is not given by the algorithm. (In our numerical experiments below, we obtain this heuristic by using the sequential algorithm for low $k$, and then extrapolating the proportion of preconditioning points used for low values of $k$ to larger values of $k.$
      \item The number of preconditioning points generated is not exactly $\Npretarget$ due to rounding the `ideal' number of centres in each dimension to the nearest integer. However, we expect the number of generated points will be close to $\Npretarget$.
      \eit


\subsection{Numerical Experiments}\label{sec:nbpcqmcnumerics}
We now describe numerical experiments demonstrating the effectiveness of nearby preconditioning for tackling Helmholtz problems.

As described above, before we perform our numerical experiments, we need to determine:
\bit
\item How the number of QMC points should scale with $k$, and
  \item How many preconditioners we should choose.
    \eit
    We tackle each of these in turn, before using the parallel algorithm detailed above to apply nearby preconditioning to QMC methods for the Helmholtz equation. Throughout this \lcnamecref{sec:nbpcqmcnumerics} we use the model problem detailed in \cref{eq:artificialkl,eq:artificialkllambdas,eq:artificialklfuns} above.

\subsubsection{QMC error estimators}
    
    To determine how the number of QMC points should scale with $k$, we first estimate the QMC error for increasing $k.$ The QMC rules we use is a a randomly shifted QMC rule, we use such a rule because there exists an error estimator for this rule, see \cref{eq:errest} below. Our exposition below follows that in \cite[Section 4.2]{GrKuNuScSl:11}.

    Suppose our QMC points are $\byo,\ldots,\byNQMC$, and the resulting QMC rule is
    \beqs
\QMC{\NQMC}{Q} = \frac1{\NQMC}\sum_{l=1}^{\NQMC} Q\mleft(u\mleft(\byl\mright)\mright).
\eeqs
For a `shift' $\shift \in \cube{J}$ we define the shifted QMC rule
\beqs
\QMCshift{Q}{\shift} = \frac1{\NQMC}\sum_{l=1}^{\NQMC} Q\mleft(u\mleft(\byl\oplus\shift\mright)\mright),
\eeqs
where $\by \oplus \shift$ denotes $\by + \shift$ `wrapped around' onto the hypercube $\cube{J}$. (Formally $\by \oplus \shift = \fracoperator{\mleft(\by + \bhalf\mright)+\shift} - \bhalf,$ where $\fracoperator{\cdot}$ denotes the fractional part and $\bhalf$ denotes the $J$-dimensional vector with every entry $1/2.$)

We then define the randomly shifted QMC rule (with multiple randomly chosen shifts $\shifto,\ldots,\shiftNshifts$)
\beqs
\QMCrandshift{Q}{\Nshifts} = \frac1{\Nshifts}\sum_{s=1}^{\Nshifts} \QMCshift{Q}{\shifts} = \frac1{\NQMC\Nshifts}\sum_{s=1}^{\Nshifts}\sum_{l=1}^{\NQMC} Q\mleft(u\mleft(\byl\oplus \shifts\mright)\mright).
\eeqs

Having defined the randomly shifted QMC rule, one can use the standard statistical estimator of the standard deviation of the error in $\QMCrandshift{Q}{\Nshifts}$ \cite[Equation (4.6)]{GrKuNuScSl:11}
\beq\label{eq:errest}
\QMCerror{\NQMC}{\Nshifts} = \mleft(\frac1{\Nshifts\mleft(\Nshifts-1\mright)}\sum_{s=1}^{\Nshifts} \mleft(\QMCshift{Q}{\shifts} - \QMCrandshift{Q}{\Nshifts}\mright)^2\mright)^{\half}.
\eeq
(See \cref{app:complexerror} for proof that $\QMCerror{\NQMC}{\Nshifts}^2$ is an unbiased estimator of the variance of $\QMCrandshift{Q}{\Nshifts}$; recall that it does \emph{not} follow that $\QMCerror{\NQMC}{\Nshifts}$ is an \emph{unbiased} estimator of the standard deviation of $\QMCrandshift{Q}{\Nshifts}$.)

\subsubsection{$k$-dependence of the number of QMC points}

We first seek to determine how $\QMCerror{\NQMC}{\Nshifts}$ depends on $k.$ We estimated $\QMCerror{\NQMC}{\Nshifts}$ for the setup described in \cref{app:compsetup} with $\NQMC = 2048$ and $\Nshifts=20$ (i.e., 40,960 PDE solves in total) for $k = 10,20,30,40,50,60$. We set $h = 0.002$ for all of the computations (as $0.002 \approx 60^{-3/2}$) to avoid having to consider the effect of finite-element error. The quantities of interest (QoIs) we considered were:
\bit
\item The integral of the solution over the whole domain,
\item The value of the solution at the origin,
\item The value of the solution at the top-right corner of the domain, and
\item The $x$-component of the gradient of the solution at the top-right corner of the domain.
  \eit
  Observe that these QoIs require increasing regularity of the solution. (The integral is defined for functions in $\LoD$, point evaluation for functions in $\Hfn{}{3/2 + \eps}{D}$ and point evaluation of the gradient for functions in $\Hfn{}{5/2+\eps}{D}$ (in 3-d - the corresponding function spaces are $\Hfn{}{1+\eps}{D}$ and $\Hfn{}{2+\eps}{D}$ in 2-d) for any $\eps > 0.$) Therefore computing for this range of QoIs will give a good insight into the behaviour of QMC applied to the Helmholtz equation\footnote{We can evaluate point values of $\uh$ because $\uh$ is continuous, and we can use the value of $\grad \uh$ on the upper-rightmost mesh element as a proxy for $\grad \uh((1,1))$; such an use is possible due the structure of our mesh; see \cref{fig:grid}.}.
%%     That is, we randomly choose $\shifto,\ldots,\shiftNshifts$ points in $\cube{J}$ (the `shifts')rause the standard error estimator
%%     \beqs
%%     \mleft(\frac1{\nu\mleft(\nu-1\mright)} \sum_{s=1}^{\Nshifts} 
%%     \eeqs$h = 0.002$ (relation to $k=60$ - maximum?)

%In \cref{fig:integralCalpha,fig:originCalpha,fig:toprightCalpha,fig:gradienttoprightCalpha} we plot how $C$ and $\alpha$ depend on $k$, for the plots of the QMC error with increasing $\NQMC$ for each value of $k,$ see \cref{app:hhqmcconv}.

\begin{figure}[h]
    \centering
    \begin{subfigure}{\textwidth}
      \centering
%% Creator: Matplotlib, PGF backend
%%
%% To include the figure in your LaTeX document, write
%%   \input{<filename>.pgf}
%%
%% Make sure the required packages are loaded in your preamble
%%   \usepackage{pgf}
%%
%% Figures using additional raster images can only be included by \input if
%% they are in the same directory as the main LaTeX file. For loading figures
%% from other directories you can use the `import` package
%%   \usepackage{import}
%% and then include the figures with
%%   \import{<path to file>}{<filename>.pgf}
%%
%% Matplotlib used the following preamble
%%   \usepackage{fontspec}
%%   \setmainfont{DejaVuSerif.ttf}[Path=/home/owen/progs/firedrake-complex/firedrake/lib/python3.5/site-packages/matplotlib/mpl-data/fonts/ttf/]
%%   \setsansfont{DejaVuSans.ttf}[Path=/home/owen/progs/firedrake-complex/firedrake/lib/python3.5/site-packages/matplotlib/mpl-data/fonts/ttf/]
%%   \setmonofont{DejaVuSansMono.ttf}[Path=/home/owen/progs/firedrake-complex/firedrake/lib/python3.5/site-packages/matplotlib/mpl-data/fonts/ttf/]
%%
\begingroup%
\makeatletter%
\begin{pgfpicture}%
\pgfpathrectangle{\pgfpointorigin}{\pgfqpoint{5.000000in}{4.000000in}}%
\pgfusepath{use as bounding box, clip}%
\begin{pgfscope}%
\pgfsetbuttcap%
\pgfsetmiterjoin%
\definecolor{currentfill}{rgb}{1.000000,1.000000,1.000000}%
\pgfsetfillcolor{currentfill}%
\pgfsetlinewidth{0.000000pt}%
\definecolor{currentstroke}{rgb}{1.000000,1.000000,1.000000}%
\pgfsetstrokecolor{currentstroke}%
\pgfsetdash{}{0pt}%
\pgfpathmoveto{\pgfqpoint{0.000000in}{0.000000in}}%
\pgfpathlineto{\pgfqpoint{5.000000in}{0.000000in}}%
\pgfpathlineto{\pgfqpoint{5.000000in}{4.000000in}}%
\pgfpathlineto{\pgfqpoint{0.000000in}{4.000000in}}%
\pgfpathclose%
\pgfusepath{fill}%
\end{pgfscope}%
\begin{pgfscope}%
\pgfsetbuttcap%
\pgfsetmiterjoin%
\definecolor{currentfill}{rgb}{1.000000,1.000000,1.000000}%
\pgfsetfillcolor{currentfill}%
\pgfsetlinewidth{0.000000pt}%
\definecolor{currentstroke}{rgb}{0.000000,0.000000,0.000000}%
\pgfsetstrokecolor{currentstroke}%
\pgfsetstrokeopacity{0.000000}%
\pgfsetdash{}{0pt}%
\pgfpathmoveto{\pgfqpoint{0.625000in}{0.440000in}}%
\pgfpathlineto{\pgfqpoint{4.500000in}{0.440000in}}%
\pgfpathlineto{\pgfqpoint{4.500000in}{3.520000in}}%
\pgfpathlineto{\pgfqpoint{0.625000in}{3.520000in}}%
\pgfpathclose%
\pgfusepath{fill}%
\end{pgfscope}%
\begin{pgfscope}%
\pgfsetbuttcap%
\pgfsetroundjoin%
\definecolor{currentfill}{rgb}{0.000000,0.000000,0.000000}%
\pgfsetfillcolor{currentfill}%
\pgfsetlinewidth{0.803000pt}%
\definecolor{currentstroke}{rgb}{0.000000,0.000000,0.000000}%
\pgfsetstrokecolor{currentstroke}%
\pgfsetdash{}{0pt}%
\pgfsys@defobject{currentmarker}{\pgfqpoint{0.000000in}{-0.048611in}}{\pgfqpoint{0.000000in}{0.000000in}}{%
\pgfpathmoveto{\pgfqpoint{0.000000in}{0.000000in}}%
\pgfpathlineto{\pgfqpoint{0.000000in}{-0.048611in}}%
\pgfusepath{stroke,fill}%
}%
\begin{pgfscope}%
\pgfsys@transformshift{0.801136in}{0.440000in}%
\pgfsys@useobject{currentmarker}{}%
\end{pgfscope}%
\end{pgfscope}%
\begin{pgfscope}%
\definecolor{textcolor}{rgb}{0.000000,0.000000,0.000000}%
\pgfsetstrokecolor{textcolor}%
\pgfsetfillcolor{textcolor}%
\pgftext[x=0.801136in,y=0.342778in,,top]{\color{textcolor}\sffamily\fontsize{10.000000}{12.000000}\selectfont \(\displaystyle {10^{1}}\)}%
\end{pgfscope}%
\begin{pgfscope}%
\pgfsetbuttcap%
\pgfsetroundjoin%
\definecolor{currentfill}{rgb}{0.000000,0.000000,0.000000}%
\pgfsetfillcolor{currentfill}%
\pgfsetlinewidth{0.602250pt}%
\definecolor{currentstroke}{rgb}{0.000000,0.000000,0.000000}%
\pgfsetstrokecolor{currentstroke}%
\pgfsetdash{}{0pt}%
\pgfsys@defobject{currentmarker}{\pgfqpoint{0.000000in}{-0.027778in}}{\pgfqpoint{0.000000in}{0.000000in}}{%
\pgfpathmoveto{\pgfqpoint{0.000000in}{0.000000in}}%
\pgfpathlineto{\pgfqpoint{0.000000in}{-0.027778in}}%
\pgfusepath{stroke,fill}%
}%
\begin{pgfscope}%
\pgfsys@transformshift{2.163913in}{0.440000in}%
\pgfsys@useobject{currentmarker}{}%
\end{pgfscope}%
\end{pgfscope}%
\begin{pgfscope}%
\definecolor{textcolor}{rgb}{0.000000,0.000000,0.000000}%
\pgfsetstrokecolor{textcolor}%
\pgfsetfillcolor{textcolor}%
\pgftext[x=2.163913in,y=0.365000in,,top]{\color{textcolor}\sffamily\fontsize{10.000000}{12.000000}\selectfont \(\displaystyle {2\times10^{1}}\)}%
\end{pgfscope}%
\begin{pgfscope}%
\pgfsetbuttcap%
\pgfsetroundjoin%
\definecolor{currentfill}{rgb}{0.000000,0.000000,0.000000}%
\pgfsetfillcolor{currentfill}%
\pgfsetlinewidth{0.602250pt}%
\definecolor{currentstroke}{rgb}{0.000000,0.000000,0.000000}%
\pgfsetstrokecolor{currentstroke}%
\pgfsetdash{}{0pt}%
\pgfsys@defobject{currentmarker}{\pgfqpoint{0.000000in}{-0.027778in}}{\pgfqpoint{0.000000in}{0.000000in}}{%
\pgfpathmoveto{\pgfqpoint{0.000000in}{0.000000in}}%
\pgfpathlineto{\pgfqpoint{0.000000in}{-0.027778in}}%
\pgfusepath{stroke,fill}%
}%
\begin{pgfscope}%
\pgfsys@transformshift{2.961087in}{0.440000in}%
\pgfsys@useobject{currentmarker}{}%
\end{pgfscope}%
\end{pgfscope}%
\begin{pgfscope}%
\definecolor{textcolor}{rgb}{0.000000,0.000000,0.000000}%
\pgfsetstrokecolor{textcolor}%
\pgfsetfillcolor{textcolor}%
\pgftext[x=2.961087in,y=0.365000in,,top]{\color{textcolor}\sffamily\fontsize{10.000000}{12.000000}\selectfont \(\displaystyle {3\times10^{1}}\)}%
\end{pgfscope}%
\begin{pgfscope}%
\pgfsetbuttcap%
\pgfsetroundjoin%
\definecolor{currentfill}{rgb}{0.000000,0.000000,0.000000}%
\pgfsetfillcolor{currentfill}%
\pgfsetlinewidth{0.602250pt}%
\definecolor{currentstroke}{rgb}{0.000000,0.000000,0.000000}%
\pgfsetstrokecolor{currentstroke}%
\pgfsetdash{}{0pt}%
\pgfsys@defobject{currentmarker}{\pgfqpoint{0.000000in}{-0.027778in}}{\pgfqpoint{0.000000in}{0.000000in}}{%
\pgfpathmoveto{\pgfqpoint{0.000000in}{0.000000in}}%
\pgfpathlineto{\pgfqpoint{0.000000in}{-0.027778in}}%
\pgfusepath{stroke,fill}%
}%
\begin{pgfscope}%
\pgfsys@transformshift{3.526690in}{0.440000in}%
\pgfsys@useobject{currentmarker}{}%
\end{pgfscope}%
\end{pgfscope}%
\begin{pgfscope}%
\definecolor{textcolor}{rgb}{0.000000,0.000000,0.000000}%
\pgfsetstrokecolor{textcolor}%
\pgfsetfillcolor{textcolor}%
\pgftext[x=3.526690in,y=0.365000in,,top]{\color{textcolor}\sffamily\fontsize{10.000000}{12.000000}\selectfont \(\displaystyle {4\times10^{1}}\)}%
\end{pgfscope}%
\begin{pgfscope}%
\pgfsetbuttcap%
\pgfsetroundjoin%
\definecolor{currentfill}{rgb}{0.000000,0.000000,0.000000}%
\pgfsetfillcolor{currentfill}%
\pgfsetlinewidth{0.602250pt}%
\definecolor{currentstroke}{rgb}{0.000000,0.000000,0.000000}%
\pgfsetstrokecolor{currentstroke}%
\pgfsetdash{}{0pt}%
\pgfsys@defobject{currentmarker}{\pgfqpoint{0.000000in}{-0.027778in}}{\pgfqpoint{0.000000in}{0.000000in}}{%
\pgfpathmoveto{\pgfqpoint{0.000000in}{0.000000in}}%
\pgfpathlineto{\pgfqpoint{0.000000in}{-0.027778in}}%
\pgfusepath{stroke,fill}%
}%
\begin{pgfscope}%
\pgfsys@transformshift{3.965406in}{0.440000in}%
\pgfsys@useobject{currentmarker}{}%
\end{pgfscope}%
\end{pgfscope}%
\begin{pgfscope}%
\pgfsetbuttcap%
\pgfsetroundjoin%
\definecolor{currentfill}{rgb}{0.000000,0.000000,0.000000}%
\pgfsetfillcolor{currentfill}%
\pgfsetlinewidth{0.602250pt}%
\definecolor{currentstroke}{rgb}{0.000000,0.000000,0.000000}%
\pgfsetstrokecolor{currentstroke}%
\pgfsetdash{}{0pt}%
\pgfsys@defobject{currentmarker}{\pgfqpoint{0.000000in}{-0.027778in}}{\pgfqpoint{0.000000in}{0.000000in}}{%
\pgfpathmoveto{\pgfqpoint{0.000000in}{0.000000in}}%
\pgfpathlineto{\pgfqpoint{0.000000in}{-0.027778in}}%
\pgfusepath{stroke,fill}%
}%
\begin{pgfscope}%
\pgfsys@transformshift{4.323864in}{0.440000in}%
\pgfsys@useobject{currentmarker}{}%
\end{pgfscope}%
\end{pgfscope}%
\begin{pgfscope}%
\definecolor{textcolor}{rgb}{0.000000,0.000000,0.000000}%
\pgfsetstrokecolor{textcolor}%
\pgfsetfillcolor{textcolor}%
\pgftext[x=4.323864in,y=0.365000in,,top]{\color{textcolor}\sffamily\fontsize{10.000000}{12.000000}\selectfont \(\displaystyle {6\times10^{1}}\)}%
\end{pgfscope}%
\begin{pgfscope}%
\definecolor{textcolor}{rgb}{0.000000,0.000000,0.000000}%
\pgfsetstrokecolor{textcolor}%
\pgfsetfillcolor{textcolor}%
\pgftext[x=2.562500in,y=0.152809in,,top]{\color{textcolor}\sffamily\fontsize{10.000000}{12.000000}\selectfont \(\displaystyle k\)}%
\end{pgfscope}%
\begin{pgfscope}%
\pgfsetbuttcap%
\pgfsetroundjoin%
\definecolor{currentfill}{rgb}{0.000000,0.000000,0.000000}%
\pgfsetfillcolor{currentfill}%
\pgfsetlinewidth{0.803000pt}%
\definecolor{currentstroke}{rgb}{0.000000,0.000000,0.000000}%
\pgfsetstrokecolor{currentstroke}%
\pgfsetdash{}{0pt}%
\pgfsys@defobject{currentmarker}{\pgfqpoint{-0.048611in}{0.000000in}}{\pgfqpoint{0.000000in}{0.000000in}}{%
\pgfpathmoveto{\pgfqpoint{0.000000in}{0.000000in}}%
\pgfpathlineto{\pgfqpoint{-0.048611in}{0.000000in}}%
\pgfusepath{stroke,fill}%
}%
\begin{pgfscope}%
\pgfsys@transformshift{0.625000in}{1.908201in}%
\pgfsys@useobject{currentmarker}{}%
\end{pgfscope}%
\end{pgfscope}%
\begin{pgfscope}%
\definecolor{textcolor}{rgb}{0.000000,0.000000,0.000000}%
\pgfsetstrokecolor{textcolor}%
\pgfsetfillcolor{textcolor}%
\pgftext[x=0.239775in,y=1.855439in,left,base]{\color{textcolor}\sffamily\fontsize{10.000000}{12.000000}\selectfont \(\displaystyle {10^{-3}}\)}%
\end{pgfscope}%
\begin{pgfscope}%
\pgfsetbuttcap%
\pgfsetroundjoin%
\definecolor{currentfill}{rgb}{0.000000,0.000000,0.000000}%
\pgfsetfillcolor{currentfill}%
\pgfsetlinewidth{0.602250pt}%
\definecolor{currentstroke}{rgb}{0.000000,0.000000,0.000000}%
\pgfsetstrokecolor{currentstroke}%
\pgfsetdash{}{0pt}%
\pgfsys@defobject{currentmarker}{\pgfqpoint{-0.027778in}{0.000000in}}{\pgfqpoint{0.000000in}{0.000000in}}{%
\pgfpathmoveto{\pgfqpoint{0.000000in}{0.000000in}}%
\pgfpathlineto{\pgfqpoint{-0.027778in}{0.000000in}}%
\pgfusepath{stroke,fill}%
}%
\begin{pgfscope}%
\pgfsys@transformshift{0.625000in}{0.494966in}%
\pgfsys@useobject{currentmarker}{}%
\end{pgfscope}%
\end{pgfscope}%
\begin{pgfscope}%
\pgfsetbuttcap%
\pgfsetroundjoin%
\definecolor{currentfill}{rgb}{0.000000,0.000000,0.000000}%
\pgfsetfillcolor{currentfill}%
\pgfsetlinewidth{0.602250pt}%
\definecolor{currentstroke}{rgb}{0.000000,0.000000,0.000000}%
\pgfsetstrokecolor{currentstroke}%
\pgfsetdash{}{0pt}%
\pgfsys@defobject{currentmarker}{\pgfqpoint{-0.027778in}{0.000000in}}{\pgfqpoint{0.000000in}{0.000000in}}{%
\pgfpathmoveto{\pgfqpoint{0.000000in}{0.000000in}}%
\pgfpathlineto{\pgfqpoint{-0.027778in}{0.000000in}}%
\pgfusepath{stroke,fill}%
}%
\begin{pgfscope}%
\pgfsys@transformshift{0.625000in}{0.851002in}%
\pgfsys@useobject{currentmarker}{}%
\end{pgfscope}%
\end{pgfscope}%
\begin{pgfscope}%
\pgfsetbuttcap%
\pgfsetroundjoin%
\definecolor{currentfill}{rgb}{0.000000,0.000000,0.000000}%
\pgfsetfillcolor{currentfill}%
\pgfsetlinewidth{0.602250pt}%
\definecolor{currentstroke}{rgb}{0.000000,0.000000,0.000000}%
\pgfsetstrokecolor{currentstroke}%
\pgfsetdash{}{0pt}%
\pgfsys@defobject{currentmarker}{\pgfqpoint{-0.027778in}{0.000000in}}{\pgfqpoint{0.000000in}{0.000000in}}{%
\pgfpathmoveto{\pgfqpoint{0.000000in}{0.000000in}}%
\pgfpathlineto{\pgfqpoint{-0.027778in}{0.000000in}}%
\pgfusepath{stroke,fill}%
}%
\begin{pgfscope}%
\pgfsys@transformshift{0.625000in}{1.103613in}%
\pgfsys@useobject{currentmarker}{}%
\end{pgfscope}%
\end{pgfscope}%
\begin{pgfscope}%
\pgfsetbuttcap%
\pgfsetroundjoin%
\definecolor{currentfill}{rgb}{0.000000,0.000000,0.000000}%
\pgfsetfillcolor{currentfill}%
\pgfsetlinewidth{0.602250pt}%
\definecolor{currentstroke}{rgb}{0.000000,0.000000,0.000000}%
\pgfsetstrokecolor{currentstroke}%
\pgfsetdash{}{0pt}%
\pgfsys@defobject{currentmarker}{\pgfqpoint{-0.027778in}{0.000000in}}{\pgfqpoint{0.000000in}{0.000000in}}{%
\pgfpathmoveto{\pgfqpoint{0.000000in}{0.000000in}}%
\pgfpathlineto{\pgfqpoint{-0.027778in}{0.000000in}}%
\pgfusepath{stroke,fill}%
}%
\begin{pgfscope}%
\pgfsys@transformshift{0.625000in}{1.299554in}%
\pgfsys@useobject{currentmarker}{}%
\end{pgfscope}%
\end{pgfscope}%
\begin{pgfscope}%
\pgfsetbuttcap%
\pgfsetroundjoin%
\definecolor{currentfill}{rgb}{0.000000,0.000000,0.000000}%
\pgfsetfillcolor{currentfill}%
\pgfsetlinewidth{0.602250pt}%
\definecolor{currentstroke}{rgb}{0.000000,0.000000,0.000000}%
\pgfsetstrokecolor{currentstroke}%
\pgfsetdash{}{0pt}%
\pgfsys@defobject{currentmarker}{\pgfqpoint{-0.027778in}{0.000000in}}{\pgfqpoint{0.000000in}{0.000000in}}{%
\pgfpathmoveto{\pgfqpoint{0.000000in}{0.000000in}}%
\pgfpathlineto{\pgfqpoint{-0.027778in}{0.000000in}}%
\pgfusepath{stroke,fill}%
}%
\begin{pgfscope}%
\pgfsys@transformshift{0.625000in}{1.459649in}%
\pgfsys@useobject{currentmarker}{}%
\end{pgfscope}%
\end{pgfscope}%
\begin{pgfscope}%
\pgfsetbuttcap%
\pgfsetroundjoin%
\definecolor{currentfill}{rgb}{0.000000,0.000000,0.000000}%
\pgfsetfillcolor{currentfill}%
\pgfsetlinewidth{0.602250pt}%
\definecolor{currentstroke}{rgb}{0.000000,0.000000,0.000000}%
\pgfsetstrokecolor{currentstroke}%
\pgfsetdash{}{0pt}%
\pgfsys@defobject{currentmarker}{\pgfqpoint{-0.027778in}{0.000000in}}{\pgfqpoint{0.000000in}{0.000000in}}{%
\pgfpathmoveto{\pgfqpoint{0.000000in}{0.000000in}}%
\pgfpathlineto{\pgfqpoint{-0.027778in}{0.000000in}}%
\pgfusepath{stroke,fill}%
}%
\begin{pgfscope}%
\pgfsys@transformshift{0.625000in}{1.595007in}%
\pgfsys@useobject{currentmarker}{}%
\end{pgfscope}%
\end{pgfscope}%
\begin{pgfscope}%
\pgfsetbuttcap%
\pgfsetroundjoin%
\definecolor{currentfill}{rgb}{0.000000,0.000000,0.000000}%
\pgfsetfillcolor{currentfill}%
\pgfsetlinewidth{0.602250pt}%
\definecolor{currentstroke}{rgb}{0.000000,0.000000,0.000000}%
\pgfsetstrokecolor{currentstroke}%
\pgfsetdash{}{0pt}%
\pgfsys@defobject{currentmarker}{\pgfqpoint{-0.027778in}{0.000000in}}{\pgfqpoint{0.000000in}{0.000000in}}{%
\pgfpathmoveto{\pgfqpoint{0.000000in}{0.000000in}}%
\pgfpathlineto{\pgfqpoint{-0.027778in}{0.000000in}}%
\pgfusepath{stroke,fill}%
}%
\begin{pgfscope}%
\pgfsys@transformshift{0.625000in}{1.712260in}%
\pgfsys@useobject{currentmarker}{}%
\end{pgfscope}%
\end{pgfscope}%
\begin{pgfscope}%
\pgfsetbuttcap%
\pgfsetroundjoin%
\definecolor{currentfill}{rgb}{0.000000,0.000000,0.000000}%
\pgfsetfillcolor{currentfill}%
\pgfsetlinewidth{0.602250pt}%
\definecolor{currentstroke}{rgb}{0.000000,0.000000,0.000000}%
\pgfsetstrokecolor{currentstroke}%
\pgfsetdash{}{0pt}%
\pgfsys@defobject{currentmarker}{\pgfqpoint{-0.027778in}{0.000000in}}{\pgfqpoint{0.000000in}{0.000000in}}{%
\pgfpathmoveto{\pgfqpoint{0.000000in}{0.000000in}}%
\pgfpathlineto{\pgfqpoint{-0.027778in}{0.000000in}}%
\pgfusepath{stroke,fill}%
}%
\begin{pgfscope}%
\pgfsys@transformshift{0.625000in}{1.815684in}%
\pgfsys@useobject{currentmarker}{}%
\end{pgfscope}%
\end{pgfscope}%
\begin{pgfscope}%
\pgfsetbuttcap%
\pgfsetroundjoin%
\definecolor{currentfill}{rgb}{0.000000,0.000000,0.000000}%
\pgfsetfillcolor{currentfill}%
\pgfsetlinewidth{0.602250pt}%
\definecolor{currentstroke}{rgb}{0.000000,0.000000,0.000000}%
\pgfsetstrokecolor{currentstroke}%
\pgfsetdash{}{0pt}%
\pgfsys@defobject{currentmarker}{\pgfqpoint{-0.027778in}{0.000000in}}{\pgfqpoint{0.000000in}{0.000000in}}{%
\pgfpathmoveto{\pgfqpoint{0.000000in}{0.000000in}}%
\pgfpathlineto{\pgfqpoint{-0.027778in}{0.000000in}}%
\pgfusepath{stroke,fill}%
}%
\begin{pgfscope}%
\pgfsys@transformshift{0.625000in}{2.516848in}%
\pgfsys@useobject{currentmarker}{}%
\end{pgfscope}%
\end{pgfscope}%
\begin{pgfscope}%
\pgfsetbuttcap%
\pgfsetroundjoin%
\definecolor{currentfill}{rgb}{0.000000,0.000000,0.000000}%
\pgfsetfillcolor{currentfill}%
\pgfsetlinewidth{0.602250pt}%
\definecolor{currentstroke}{rgb}{0.000000,0.000000,0.000000}%
\pgfsetstrokecolor{currentstroke}%
\pgfsetdash{}{0pt}%
\pgfsys@defobject{currentmarker}{\pgfqpoint{-0.027778in}{0.000000in}}{\pgfqpoint{0.000000in}{0.000000in}}{%
\pgfpathmoveto{\pgfqpoint{0.000000in}{0.000000in}}%
\pgfpathlineto{\pgfqpoint{-0.027778in}{0.000000in}}%
\pgfusepath{stroke,fill}%
}%
\begin{pgfscope}%
\pgfsys@transformshift{0.625000in}{2.872884in}%
\pgfsys@useobject{currentmarker}{}%
\end{pgfscope}%
\end{pgfscope}%
\begin{pgfscope}%
\pgfsetbuttcap%
\pgfsetroundjoin%
\definecolor{currentfill}{rgb}{0.000000,0.000000,0.000000}%
\pgfsetfillcolor{currentfill}%
\pgfsetlinewidth{0.602250pt}%
\definecolor{currentstroke}{rgb}{0.000000,0.000000,0.000000}%
\pgfsetstrokecolor{currentstroke}%
\pgfsetdash{}{0pt}%
\pgfsys@defobject{currentmarker}{\pgfqpoint{-0.027778in}{0.000000in}}{\pgfqpoint{0.000000in}{0.000000in}}{%
\pgfpathmoveto{\pgfqpoint{0.000000in}{0.000000in}}%
\pgfpathlineto{\pgfqpoint{-0.027778in}{0.000000in}}%
\pgfusepath{stroke,fill}%
}%
\begin{pgfscope}%
\pgfsys@transformshift{0.625000in}{3.125495in}%
\pgfsys@useobject{currentmarker}{}%
\end{pgfscope}%
\end{pgfscope}%
\begin{pgfscope}%
\pgfsetbuttcap%
\pgfsetroundjoin%
\definecolor{currentfill}{rgb}{0.000000,0.000000,0.000000}%
\pgfsetfillcolor{currentfill}%
\pgfsetlinewidth{0.602250pt}%
\definecolor{currentstroke}{rgb}{0.000000,0.000000,0.000000}%
\pgfsetstrokecolor{currentstroke}%
\pgfsetdash{}{0pt}%
\pgfsys@defobject{currentmarker}{\pgfqpoint{-0.027778in}{0.000000in}}{\pgfqpoint{0.000000in}{0.000000in}}{%
\pgfpathmoveto{\pgfqpoint{0.000000in}{0.000000in}}%
\pgfpathlineto{\pgfqpoint{-0.027778in}{0.000000in}}%
\pgfusepath{stroke,fill}%
}%
\begin{pgfscope}%
\pgfsys@transformshift{0.625000in}{3.321436in}%
\pgfsys@useobject{currentmarker}{}%
\end{pgfscope}%
\end{pgfscope}%
\begin{pgfscope}%
\pgfsetbuttcap%
\pgfsetroundjoin%
\definecolor{currentfill}{rgb}{0.000000,0.000000,0.000000}%
\pgfsetfillcolor{currentfill}%
\pgfsetlinewidth{0.602250pt}%
\definecolor{currentstroke}{rgb}{0.000000,0.000000,0.000000}%
\pgfsetstrokecolor{currentstroke}%
\pgfsetdash{}{0pt}%
\pgfsys@defobject{currentmarker}{\pgfqpoint{-0.027778in}{0.000000in}}{\pgfqpoint{0.000000in}{0.000000in}}{%
\pgfpathmoveto{\pgfqpoint{0.000000in}{0.000000in}}%
\pgfpathlineto{\pgfqpoint{-0.027778in}{0.000000in}}%
\pgfusepath{stroke,fill}%
}%
\begin{pgfscope}%
\pgfsys@transformshift{0.625000in}{3.481531in}%
\pgfsys@useobject{currentmarker}{}%
\end{pgfscope}%
\end{pgfscope}%
\begin{pgfscope}%
\definecolor{textcolor}{rgb}{0.000000,0.000000,0.000000}%
\pgfsetstrokecolor{textcolor}%
\pgfsetfillcolor{textcolor}%
\pgftext[x=0.184220in,y=1.980000in,,bottom,rotate=90.000000]{\color{textcolor}\sffamily\fontsize{10.000000}{12.000000}\selectfont \(\displaystyle C\)}%
\end{pgfscope}%
\begin{pgfscope}%
\pgfpathrectangle{\pgfqpoint{0.625000in}{0.440000in}}{\pgfqpoint{3.875000in}{3.080000in}}%
\pgfusepath{clip}%
\pgfsetbuttcap%
\pgfsetroundjoin%
\definecolor{currentfill}{rgb}{0.000000,0.000000,0.000000}%
\pgfsetfillcolor{currentfill}%
\pgfsetlinewidth{1.003750pt}%
\definecolor{currentstroke}{rgb}{0.000000,0.000000,0.000000}%
\pgfsetstrokecolor{currentstroke}%
\pgfsetdash{}{0pt}%
\pgfsys@defobject{currentmarker}{\pgfqpoint{-0.041667in}{-0.041667in}}{\pgfqpoint{0.041667in}{0.041667in}}{%
\pgfpathmoveto{\pgfqpoint{0.000000in}{-0.041667in}}%
\pgfpathcurveto{\pgfqpoint{0.011050in}{-0.041667in}}{\pgfqpoint{0.021649in}{-0.037276in}}{\pgfqpoint{0.029463in}{-0.029463in}}%
\pgfpathcurveto{\pgfqpoint{0.037276in}{-0.021649in}}{\pgfqpoint{0.041667in}{-0.011050in}}{\pgfqpoint{0.041667in}{0.000000in}}%
\pgfpathcurveto{\pgfqpoint{0.041667in}{0.011050in}}{\pgfqpoint{0.037276in}{0.021649in}}{\pgfqpoint{0.029463in}{0.029463in}}%
\pgfpathcurveto{\pgfqpoint{0.021649in}{0.037276in}}{\pgfqpoint{0.011050in}{0.041667in}}{\pgfqpoint{0.000000in}{0.041667in}}%
\pgfpathcurveto{\pgfqpoint{-0.011050in}{0.041667in}}{\pgfqpoint{-0.021649in}{0.037276in}}{\pgfqpoint{-0.029463in}{0.029463in}}%
\pgfpathcurveto{\pgfqpoint{-0.037276in}{0.021649in}}{\pgfqpoint{-0.041667in}{0.011050in}}{\pgfqpoint{-0.041667in}{0.000000in}}%
\pgfpathcurveto{\pgfqpoint{-0.041667in}{-0.011050in}}{\pgfqpoint{-0.037276in}{-0.021649in}}{\pgfqpoint{-0.029463in}{-0.029463in}}%
\pgfpathcurveto{\pgfqpoint{-0.021649in}{-0.037276in}}{\pgfqpoint{-0.011050in}{-0.041667in}}{\pgfqpoint{0.000000in}{-0.041667in}}%
\pgfpathclose%
\pgfusepath{stroke,fill}%
}%
\begin{pgfscope}%
\pgfsys@transformshift{0.801136in}{3.380000in}%
\pgfsys@useobject{currentmarker}{}%
\end{pgfscope}%
\begin{pgfscope}%
\pgfsys@transformshift{2.163913in}{2.625858in}%
\pgfsys@useobject{currentmarker}{}%
\end{pgfscope}%
\begin{pgfscope}%
\pgfsys@transformshift{2.961087in}{2.076879in}%
\pgfsys@useobject{currentmarker}{}%
\end{pgfscope}%
\begin{pgfscope}%
\pgfsys@transformshift{3.526690in}{1.198547in}%
\pgfsys@useobject{currentmarker}{}%
\end{pgfscope}%
\begin{pgfscope}%
\pgfsys@transformshift{3.965406in}{0.972270in}%
\pgfsys@useobject{currentmarker}{}%
\end{pgfscope}%
\begin{pgfscope}%
\pgfsys@transformshift{4.323864in}{0.580000in}%
\pgfsys@useobject{currentmarker}{}%
\end{pgfscope}%
\end{pgfscope}%
\begin{pgfscope}%
\pgfsetrectcap%
\pgfsetmiterjoin%
\pgfsetlinewidth{0.803000pt}%
\definecolor{currentstroke}{rgb}{0.000000,0.000000,0.000000}%
\pgfsetstrokecolor{currentstroke}%
\pgfsetdash{}{0pt}%
\pgfpathmoveto{\pgfqpoint{0.625000in}{0.440000in}}%
\pgfpathlineto{\pgfqpoint{0.625000in}{3.520000in}}%
\pgfusepath{stroke}%
\end{pgfscope}%
\begin{pgfscope}%
\pgfsetrectcap%
\pgfsetmiterjoin%
\pgfsetlinewidth{0.803000pt}%
\definecolor{currentstroke}{rgb}{0.000000,0.000000,0.000000}%
\pgfsetstrokecolor{currentstroke}%
\pgfsetdash{}{0pt}%
\pgfpathmoveto{\pgfqpoint{4.500000in}{0.440000in}}%
\pgfpathlineto{\pgfqpoint{4.500000in}{3.520000in}}%
\pgfusepath{stroke}%
\end{pgfscope}%
\begin{pgfscope}%
\pgfsetrectcap%
\pgfsetmiterjoin%
\pgfsetlinewidth{0.803000pt}%
\definecolor{currentstroke}{rgb}{0.000000,0.000000,0.000000}%
\pgfsetstrokecolor{currentstroke}%
\pgfsetdash{}{0pt}%
\pgfpathmoveto{\pgfqpoint{0.625000in}{0.440000in}}%
\pgfpathlineto{\pgfqpoint{4.500000in}{0.440000in}}%
\pgfusepath{stroke}%
\end{pgfscope}%
\begin{pgfscope}%
\pgfsetrectcap%
\pgfsetmiterjoin%
\pgfsetlinewidth{0.803000pt}%
\definecolor{currentstroke}{rgb}{0.000000,0.000000,0.000000}%
\pgfsetstrokecolor{currentstroke}%
\pgfsetdash{}{0pt}%
\pgfpathmoveto{\pgfqpoint{0.625000in}{3.520000in}}%
\pgfpathlineto{\pgfqpoint{4.500000in}{3.520000in}}%
\pgfusepath{stroke}%
\end{pgfscope}%
\end{pgfpicture}%
\makeatother%
\endgroup%

    \end{subfigure}
    \begin{subfigure}{\textwidth}
                \centering
      %% Creator: Matplotlib, PGF backend
%%
%% To include the figure in your LaTeX document, write
%%   \input{<filename>.pgf}
%%
%% Make sure the required packages are loaded in your preamble
%%   \usepackage{pgf}
%%
%% Figures using additional raster images can only be included by \input if
%% they are in the same directory as the main LaTeX file. For loading figures
%% from other directories you can use the `import` package
%%   \usepackage{import}
%% and then include the figures with
%%   \import{<path to file>}{<filename>.pgf}
%%
%% Matplotlib used the following preamble
%%   \usepackage{fontspec}
%%   \setmainfont{DejaVuSerif.ttf}[Path=/home/owen/progs/firedrake-complex/firedrake/lib/python3.5/site-packages/matplotlib/mpl-data/fonts/ttf/]
%%   \setsansfont{DejaVuSans.ttf}[Path=/home/owen/progs/firedrake-complex/firedrake/lib/python3.5/site-packages/matplotlib/mpl-data/fonts/ttf/]
%%   \setmonofont{DejaVuSansMono.ttf}[Path=/home/owen/progs/firedrake-complex/firedrake/lib/python3.5/site-packages/matplotlib/mpl-data/fonts/ttf/]
%%
\begingroup%
\makeatletter%
\begin{pgfpicture}%
\pgfpathrectangle{\pgfpointorigin}{\pgfqpoint{5.000000in}{4.000000in}}%
\pgfusepath{use as bounding box, clip}%
\begin{pgfscope}%
\pgfsetbuttcap%
\pgfsetmiterjoin%
\definecolor{currentfill}{rgb}{1.000000,1.000000,1.000000}%
\pgfsetfillcolor{currentfill}%
\pgfsetlinewidth{0.000000pt}%
\definecolor{currentstroke}{rgb}{1.000000,1.000000,1.000000}%
\pgfsetstrokecolor{currentstroke}%
\pgfsetdash{}{0pt}%
\pgfpathmoveto{\pgfqpoint{0.000000in}{0.000000in}}%
\pgfpathlineto{\pgfqpoint{5.000000in}{0.000000in}}%
\pgfpathlineto{\pgfqpoint{5.000000in}{4.000000in}}%
\pgfpathlineto{\pgfqpoint{0.000000in}{4.000000in}}%
\pgfpathclose%
\pgfusepath{fill}%
\end{pgfscope}%
\begin{pgfscope}%
\pgfsetbuttcap%
\pgfsetmiterjoin%
\definecolor{currentfill}{rgb}{1.000000,1.000000,1.000000}%
\pgfsetfillcolor{currentfill}%
\pgfsetlinewidth{0.000000pt}%
\definecolor{currentstroke}{rgb}{0.000000,0.000000,0.000000}%
\pgfsetstrokecolor{currentstroke}%
\pgfsetstrokeopacity{0.000000}%
\pgfsetdash{}{0pt}%
\pgfpathmoveto{\pgfqpoint{0.625000in}{0.440000in}}%
\pgfpathlineto{\pgfqpoint{4.500000in}{0.440000in}}%
\pgfpathlineto{\pgfqpoint{4.500000in}{3.520000in}}%
\pgfpathlineto{\pgfqpoint{0.625000in}{3.520000in}}%
\pgfpathclose%
\pgfusepath{fill}%
\end{pgfscope}%
\begin{pgfscope}%
\pgfsetbuttcap%
\pgfsetroundjoin%
\definecolor{currentfill}{rgb}{0.000000,0.000000,0.000000}%
\pgfsetfillcolor{currentfill}%
\pgfsetlinewidth{0.803000pt}%
\definecolor{currentstroke}{rgb}{0.000000,0.000000,0.000000}%
\pgfsetstrokecolor{currentstroke}%
\pgfsetdash{}{0pt}%
\pgfsys@defobject{currentmarker}{\pgfqpoint{0.000000in}{-0.048611in}}{\pgfqpoint{0.000000in}{0.000000in}}{%
\pgfpathmoveto{\pgfqpoint{0.000000in}{0.000000in}}%
\pgfpathlineto{\pgfqpoint{0.000000in}{-0.048611in}}%
\pgfusepath{stroke,fill}%
}%
\begin{pgfscope}%
\pgfsys@transformshift{0.801136in}{0.440000in}%
\pgfsys@useobject{currentmarker}{}%
\end{pgfscope}%
\end{pgfscope}%
\begin{pgfscope}%
\definecolor{textcolor}{rgb}{0.000000,0.000000,0.000000}%
\pgfsetstrokecolor{textcolor}%
\pgfsetfillcolor{textcolor}%
\pgftext[x=0.801136in,y=0.342778in,,top]{\color{textcolor}\sffamily\fontsize{10.000000}{12.000000}\selectfont \(\displaystyle 10^{1}\)}%
\end{pgfscope}%
\begin{pgfscope}%
\pgfsetbuttcap%
\pgfsetroundjoin%
\definecolor{currentfill}{rgb}{0.000000,0.000000,0.000000}%
\pgfsetfillcolor{currentfill}%
\pgfsetlinewidth{0.602250pt}%
\definecolor{currentstroke}{rgb}{0.000000,0.000000,0.000000}%
\pgfsetstrokecolor{currentstroke}%
\pgfsetdash{}{0pt}%
\pgfsys@defobject{currentmarker}{\pgfqpoint{0.000000in}{-0.027778in}}{\pgfqpoint{0.000000in}{0.000000in}}{%
\pgfpathmoveto{\pgfqpoint{0.000000in}{0.000000in}}%
\pgfpathlineto{\pgfqpoint{0.000000in}{-0.027778in}}%
\pgfusepath{stroke,fill}%
}%
\begin{pgfscope}%
\pgfsys@transformshift{2.163913in}{0.440000in}%
\pgfsys@useobject{currentmarker}{}%
\end{pgfscope}%
\end{pgfscope}%
\begin{pgfscope}%
\definecolor{textcolor}{rgb}{0.000000,0.000000,0.000000}%
\pgfsetstrokecolor{textcolor}%
\pgfsetfillcolor{textcolor}%
\pgftext[x=2.163913in,y=0.365000in,,top]{\color{textcolor}\sffamily\fontsize{10.000000}{12.000000}\selectfont \(\displaystyle 2\times10^{1}\)}%
\end{pgfscope}%
\begin{pgfscope}%
\pgfsetbuttcap%
\pgfsetroundjoin%
\definecolor{currentfill}{rgb}{0.000000,0.000000,0.000000}%
\pgfsetfillcolor{currentfill}%
\pgfsetlinewidth{0.602250pt}%
\definecolor{currentstroke}{rgb}{0.000000,0.000000,0.000000}%
\pgfsetstrokecolor{currentstroke}%
\pgfsetdash{}{0pt}%
\pgfsys@defobject{currentmarker}{\pgfqpoint{0.000000in}{-0.027778in}}{\pgfqpoint{0.000000in}{0.000000in}}{%
\pgfpathmoveto{\pgfqpoint{0.000000in}{0.000000in}}%
\pgfpathlineto{\pgfqpoint{0.000000in}{-0.027778in}}%
\pgfusepath{stroke,fill}%
}%
\begin{pgfscope}%
\pgfsys@transformshift{2.961087in}{0.440000in}%
\pgfsys@useobject{currentmarker}{}%
\end{pgfscope}%
\end{pgfscope}%
\begin{pgfscope}%
\definecolor{textcolor}{rgb}{0.000000,0.000000,0.000000}%
\pgfsetstrokecolor{textcolor}%
\pgfsetfillcolor{textcolor}%
\pgftext[x=2.961087in,y=0.365000in,,top]{\color{textcolor}\sffamily\fontsize{10.000000}{12.000000}\selectfont \(\displaystyle 3\times10^{1}\)}%
\end{pgfscope}%
\begin{pgfscope}%
\pgfsetbuttcap%
\pgfsetroundjoin%
\definecolor{currentfill}{rgb}{0.000000,0.000000,0.000000}%
\pgfsetfillcolor{currentfill}%
\pgfsetlinewidth{0.602250pt}%
\definecolor{currentstroke}{rgb}{0.000000,0.000000,0.000000}%
\pgfsetstrokecolor{currentstroke}%
\pgfsetdash{}{0pt}%
\pgfsys@defobject{currentmarker}{\pgfqpoint{0.000000in}{-0.027778in}}{\pgfqpoint{0.000000in}{0.000000in}}{%
\pgfpathmoveto{\pgfqpoint{0.000000in}{0.000000in}}%
\pgfpathlineto{\pgfqpoint{0.000000in}{-0.027778in}}%
\pgfusepath{stroke,fill}%
}%
\begin{pgfscope}%
\pgfsys@transformshift{3.526690in}{0.440000in}%
\pgfsys@useobject{currentmarker}{}%
\end{pgfscope}%
\end{pgfscope}%
\begin{pgfscope}%
\definecolor{textcolor}{rgb}{0.000000,0.000000,0.000000}%
\pgfsetstrokecolor{textcolor}%
\pgfsetfillcolor{textcolor}%
\pgftext[x=3.526690in,y=0.365000in,,top]{\color{textcolor}\sffamily\fontsize{10.000000}{12.000000}\selectfont \(\displaystyle 4\times10^{1}\)}%
\end{pgfscope}%
\begin{pgfscope}%
\pgfsetbuttcap%
\pgfsetroundjoin%
\definecolor{currentfill}{rgb}{0.000000,0.000000,0.000000}%
\pgfsetfillcolor{currentfill}%
\pgfsetlinewidth{0.602250pt}%
\definecolor{currentstroke}{rgb}{0.000000,0.000000,0.000000}%
\pgfsetstrokecolor{currentstroke}%
\pgfsetdash{}{0pt}%
\pgfsys@defobject{currentmarker}{\pgfqpoint{0.000000in}{-0.027778in}}{\pgfqpoint{0.000000in}{0.000000in}}{%
\pgfpathmoveto{\pgfqpoint{0.000000in}{0.000000in}}%
\pgfpathlineto{\pgfqpoint{0.000000in}{-0.027778in}}%
\pgfusepath{stroke,fill}%
}%
\begin{pgfscope}%
\pgfsys@transformshift{3.965406in}{0.440000in}%
\pgfsys@useobject{currentmarker}{}%
\end{pgfscope}%
\end{pgfscope}%
\begin{pgfscope}%
\pgfsetbuttcap%
\pgfsetroundjoin%
\definecolor{currentfill}{rgb}{0.000000,0.000000,0.000000}%
\pgfsetfillcolor{currentfill}%
\pgfsetlinewidth{0.602250pt}%
\definecolor{currentstroke}{rgb}{0.000000,0.000000,0.000000}%
\pgfsetstrokecolor{currentstroke}%
\pgfsetdash{}{0pt}%
\pgfsys@defobject{currentmarker}{\pgfqpoint{0.000000in}{-0.027778in}}{\pgfqpoint{0.000000in}{0.000000in}}{%
\pgfpathmoveto{\pgfqpoint{0.000000in}{0.000000in}}%
\pgfpathlineto{\pgfqpoint{0.000000in}{-0.027778in}}%
\pgfusepath{stroke,fill}%
}%
\begin{pgfscope}%
\pgfsys@transformshift{4.323864in}{0.440000in}%
\pgfsys@useobject{currentmarker}{}%
\end{pgfscope}%
\end{pgfscope}%
\begin{pgfscope}%
\definecolor{textcolor}{rgb}{0.000000,0.000000,0.000000}%
\pgfsetstrokecolor{textcolor}%
\pgfsetfillcolor{textcolor}%
\pgftext[x=4.323864in,y=0.365000in,,top]{\color{textcolor}\sffamily\fontsize{10.000000}{12.000000}\selectfont \(\displaystyle 6\times10^{1}\)}%
\end{pgfscope}%
\begin{pgfscope}%
\definecolor{textcolor}{rgb}{0.000000,0.000000,0.000000}%
\pgfsetstrokecolor{textcolor}%
\pgfsetfillcolor{textcolor}%
\pgftext[x=2.562500in,y=0.152809in,,top]{\color{textcolor}\sffamily\fontsize{10.000000}{12.000000}\selectfont \(\displaystyle k\)}%
\end{pgfscope}%
\begin{pgfscope}%
\pgfsetbuttcap%
\pgfsetroundjoin%
\definecolor{currentfill}{rgb}{0.000000,0.000000,0.000000}%
\pgfsetfillcolor{currentfill}%
\pgfsetlinewidth{0.803000pt}%
\definecolor{currentstroke}{rgb}{0.000000,0.000000,0.000000}%
\pgfsetstrokecolor{currentstroke}%
\pgfsetdash{}{0pt}%
\pgfsys@defobject{currentmarker}{\pgfqpoint{-0.048611in}{0.000000in}}{\pgfqpoint{0.000000in}{0.000000in}}{%
\pgfpathmoveto{\pgfqpoint{0.000000in}{0.000000in}}%
\pgfpathlineto{\pgfqpoint{-0.048611in}{0.000000in}}%
\pgfusepath{stroke,fill}%
}%
\begin{pgfscope}%
\pgfsys@transformshift{0.625000in}{0.623639in}%
\pgfsys@useobject{currentmarker}{}%
\end{pgfscope}%
\end{pgfscope}%
\begin{pgfscope}%
\definecolor{textcolor}{rgb}{0.000000,0.000000,0.000000}%
\pgfsetstrokecolor{textcolor}%
\pgfsetfillcolor{textcolor}%
\pgftext[x=0.280863in,y=0.570877in,left,base]{\color{textcolor}\sffamily\fontsize{10.000000}{12.000000}\selectfont \(\displaystyle 0.60\)}%
\end{pgfscope}%
\begin{pgfscope}%
\pgfsetbuttcap%
\pgfsetroundjoin%
\definecolor{currentfill}{rgb}{0.000000,0.000000,0.000000}%
\pgfsetfillcolor{currentfill}%
\pgfsetlinewidth{0.803000pt}%
\definecolor{currentstroke}{rgb}{0.000000,0.000000,0.000000}%
\pgfsetstrokecolor{currentstroke}%
\pgfsetdash{}{0pt}%
\pgfsys@defobject{currentmarker}{\pgfqpoint{-0.048611in}{0.000000in}}{\pgfqpoint{0.000000in}{0.000000in}}{%
\pgfpathmoveto{\pgfqpoint{0.000000in}{0.000000in}}%
\pgfpathlineto{\pgfqpoint{-0.048611in}{0.000000in}}%
\pgfusepath{stroke,fill}%
}%
\begin{pgfscope}%
\pgfsys@transformshift{0.625000in}{0.990129in}%
\pgfsys@useobject{currentmarker}{}%
\end{pgfscope}%
\end{pgfscope}%
\begin{pgfscope}%
\definecolor{textcolor}{rgb}{0.000000,0.000000,0.000000}%
\pgfsetstrokecolor{textcolor}%
\pgfsetfillcolor{textcolor}%
\pgftext[x=0.280863in,y=0.937367in,left,base]{\color{textcolor}\sffamily\fontsize{10.000000}{12.000000}\selectfont \(\displaystyle 0.65\)}%
\end{pgfscope}%
\begin{pgfscope}%
\pgfsetbuttcap%
\pgfsetroundjoin%
\definecolor{currentfill}{rgb}{0.000000,0.000000,0.000000}%
\pgfsetfillcolor{currentfill}%
\pgfsetlinewidth{0.803000pt}%
\definecolor{currentstroke}{rgb}{0.000000,0.000000,0.000000}%
\pgfsetstrokecolor{currentstroke}%
\pgfsetdash{}{0pt}%
\pgfsys@defobject{currentmarker}{\pgfqpoint{-0.048611in}{0.000000in}}{\pgfqpoint{0.000000in}{0.000000in}}{%
\pgfpathmoveto{\pgfqpoint{0.000000in}{0.000000in}}%
\pgfpathlineto{\pgfqpoint{-0.048611in}{0.000000in}}%
\pgfusepath{stroke,fill}%
}%
\begin{pgfscope}%
\pgfsys@transformshift{0.625000in}{1.356618in}%
\pgfsys@useobject{currentmarker}{}%
\end{pgfscope}%
\end{pgfscope}%
\begin{pgfscope}%
\definecolor{textcolor}{rgb}{0.000000,0.000000,0.000000}%
\pgfsetstrokecolor{textcolor}%
\pgfsetfillcolor{textcolor}%
\pgftext[x=0.280863in,y=1.303857in,left,base]{\color{textcolor}\sffamily\fontsize{10.000000}{12.000000}\selectfont \(\displaystyle 0.70\)}%
\end{pgfscope}%
\begin{pgfscope}%
\pgfsetbuttcap%
\pgfsetroundjoin%
\definecolor{currentfill}{rgb}{0.000000,0.000000,0.000000}%
\pgfsetfillcolor{currentfill}%
\pgfsetlinewidth{0.803000pt}%
\definecolor{currentstroke}{rgb}{0.000000,0.000000,0.000000}%
\pgfsetstrokecolor{currentstroke}%
\pgfsetdash{}{0pt}%
\pgfsys@defobject{currentmarker}{\pgfqpoint{-0.048611in}{0.000000in}}{\pgfqpoint{0.000000in}{0.000000in}}{%
\pgfpathmoveto{\pgfqpoint{0.000000in}{0.000000in}}%
\pgfpathlineto{\pgfqpoint{-0.048611in}{0.000000in}}%
\pgfusepath{stroke,fill}%
}%
\begin{pgfscope}%
\pgfsys@transformshift{0.625000in}{1.723108in}%
\pgfsys@useobject{currentmarker}{}%
\end{pgfscope}%
\end{pgfscope}%
\begin{pgfscope}%
\definecolor{textcolor}{rgb}{0.000000,0.000000,0.000000}%
\pgfsetstrokecolor{textcolor}%
\pgfsetfillcolor{textcolor}%
\pgftext[x=0.280863in,y=1.670347in,left,base]{\color{textcolor}\sffamily\fontsize{10.000000}{12.000000}\selectfont \(\displaystyle 0.75\)}%
\end{pgfscope}%
\begin{pgfscope}%
\pgfsetbuttcap%
\pgfsetroundjoin%
\definecolor{currentfill}{rgb}{0.000000,0.000000,0.000000}%
\pgfsetfillcolor{currentfill}%
\pgfsetlinewidth{0.803000pt}%
\definecolor{currentstroke}{rgb}{0.000000,0.000000,0.000000}%
\pgfsetstrokecolor{currentstroke}%
\pgfsetdash{}{0pt}%
\pgfsys@defobject{currentmarker}{\pgfqpoint{-0.048611in}{0.000000in}}{\pgfqpoint{0.000000in}{0.000000in}}{%
\pgfpathmoveto{\pgfqpoint{0.000000in}{0.000000in}}%
\pgfpathlineto{\pgfqpoint{-0.048611in}{0.000000in}}%
\pgfusepath{stroke,fill}%
}%
\begin{pgfscope}%
\pgfsys@transformshift{0.625000in}{2.089598in}%
\pgfsys@useobject{currentmarker}{}%
\end{pgfscope}%
\end{pgfscope}%
\begin{pgfscope}%
\definecolor{textcolor}{rgb}{0.000000,0.000000,0.000000}%
\pgfsetstrokecolor{textcolor}%
\pgfsetfillcolor{textcolor}%
\pgftext[x=0.280863in,y=2.036837in,left,base]{\color{textcolor}\sffamily\fontsize{10.000000}{12.000000}\selectfont \(\displaystyle 0.80\)}%
\end{pgfscope}%
\begin{pgfscope}%
\pgfsetbuttcap%
\pgfsetroundjoin%
\definecolor{currentfill}{rgb}{0.000000,0.000000,0.000000}%
\pgfsetfillcolor{currentfill}%
\pgfsetlinewidth{0.803000pt}%
\definecolor{currentstroke}{rgb}{0.000000,0.000000,0.000000}%
\pgfsetstrokecolor{currentstroke}%
\pgfsetdash{}{0pt}%
\pgfsys@defobject{currentmarker}{\pgfqpoint{-0.048611in}{0.000000in}}{\pgfqpoint{0.000000in}{0.000000in}}{%
\pgfpathmoveto{\pgfqpoint{0.000000in}{0.000000in}}%
\pgfpathlineto{\pgfqpoint{-0.048611in}{0.000000in}}%
\pgfusepath{stroke,fill}%
}%
\begin{pgfscope}%
\pgfsys@transformshift{0.625000in}{2.456088in}%
\pgfsys@useobject{currentmarker}{}%
\end{pgfscope}%
\end{pgfscope}%
\begin{pgfscope}%
\definecolor{textcolor}{rgb}{0.000000,0.000000,0.000000}%
\pgfsetstrokecolor{textcolor}%
\pgfsetfillcolor{textcolor}%
\pgftext[x=0.280863in,y=2.403327in,left,base]{\color{textcolor}\sffamily\fontsize{10.000000}{12.000000}\selectfont \(\displaystyle 0.85\)}%
\end{pgfscope}%
\begin{pgfscope}%
\pgfsetbuttcap%
\pgfsetroundjoin%
\definecolor{currentfill}{rgb}{0.000000,0.000000,0.000000}%
\pgfsetfillcolor{currentfill}%
\pgfsetlinewidth{0.803000pt}%
\definecolor{currentstroke}{rgb}{0.000000,0.000000,0.000000}%
\pgfsetstrokecolor{currentstroke}%
\pgfsetdash{}{0pt}%
\pgfsys@defobject{currentmarker}{\pgfqpoint{-0.048611in}{0.000000in}}{\pgfqpoint{0.000000in}{0.000000in}}{%
\pgfpathmoveto{\pgfqpoint{0.000000in}{0.000000in}}%
\pgfpathlineto{\pgfqpoint{-0.048611in}{0.000000in}}%
\pgfusepath{stroke,fill}%
}%
\begin{pgfscope}%
\pgfsys@transformshift{0.625000in}{2.822578in}%
\pgfsys@useobject{currentmarker}{}%
\end{pgfscope}%
\end{pgfscope}%
\begin{pgfscope}%
\definecolor{textcolor}{rgb}{0.000000,0.000000,0.000000}%
\pgfsetstrokecolor{textcolor}%
\pgfsetfillcolor{textcolor}%
\pgftext[x=0.280863in,y=2.769817in,left,base]{\color{textcolor}\sffamily\fontsize{10.000000}{12.000000}\selectfont \(\displaystyle 0.90\)}%
\end{pgfscope}%
\begin{pgfscope}%
\pgfsetbuttcap%
\pgfsetroundjoin%
\definecolor{currentfill}{rgb}{0.000000,0.000000,0.000000}%
\pgfsetfillcolor{currentfill}%
\pgfsetlinewidth{0.803000pt}%
\definecolor{currentstroke}{rgb}{0.000000,0.000000,0.000000}%
\pgfsetstrokecolor{currentstroke}%
\pgfsetdash{}{0pt}%
\pgfsys@defobject{currentmarker}{\pgfqpoint{-0.048611in}{0.000000in}}{\pgfqpoint{0.000000in}{0.000000in}}{%
\pgfpathmoveto{\pgfqpoint{0.000000in}{0.000000in}}%
\pgfpathlineto{\pgfqpoint{-0.048611in}{0.000000in}}%
\pgfusepath{stroke,fill}%
}%
\begin{pgfscope}%
\pgfsys@transformshift{0.625000in}{3.189068in}%
\pgfsys@useobject{currentmarker}{}%
\end{pgfscope}%
\end{pgfscope}%
\begin{pgfscope}%
\definecolor{textcolor}{rgb}{0.000000,0.000000,0.000000}%
\pgfsetstrokecolor{textcolor}%
\pgfsetfillcolor{textcolor}%
\pgftext[x=0.280863in,y=3.136307in,left,base]{\color{textcolor}\sffamily\fontsize{10.000000}{12.000000}\selectfont \(\displaystyle 0.95\)}%
\end{pgfscope}%
\begin{pgfscope}%
\definecolor{textcolor}{rgb}{0.000000,0.000000,0.000000}%
\pgfsetstrokecolor{textcolor}%
\pgfsetfillcolor{textcolor}%
\pgftext[x=0.225308in,y=1.980000in,,bottom,rotate=90.000000]{\color{textcolor}\sffamily\fontsize{10.000000}{12.000000}\selectfont \(\displaystyle \alpha\)}%
\end{pgfscope}%
\begin{pgfscope}%
\pgfpathrectangle{\pgfqpoint{0.625000in}{0.440000in}}{\pgfqpoint{3.875000in}{3.080000in}}%
\pgfusepath{clip}%
\pgfsetbuttcap%
\pgfsetroundjoin%
\pgfsetlinewidth{1.505625pt}%
\definecolor{currentstroke}{rgb}{0.000000,0.000000,0.000000}%
\pgfsetstrokecolor{currentstroke}%
\pgfsetdash{{5.550000pt}{2.400000pt}}{0.000000pt}%
\pgfpathmoveto{\pgfqpoint{0.801136in}{3.380000in}}%
\pgfpathlineto{\pgfqpoint{2.163913in}{2.582426in}}%
\pgfpathlineto{\pgfqpoint{2.961087in}{2.115875in}}%
\pgfpathlineto{\pgfqpoint{3.526690in}{1.784851in}}%
\pgfpathlineto{\pgfqpoint{3.965406in}{1.528090in}}%
\pgfpathlineto{\pgfqpoint{4.323864in}{1.318300in}}%
\pgfusepath{stroke}%
\end{pgfscope}%
\begin{pgfscope}%
\pgfpathrectangle{\pgfqpoint{0.625000in}{0.440000in}}{\pgfqpoint{3.875000in}{3.080000in}}%
\pgfusepath{clip}%
\pgfsetbuttcap%
\pgfsetroundjoin%
\definecolor{currentfill}{rgb}{0.000000,0.000000,0.000000}%
\pgfsetfillcolor{currentfill}%
\pgfsetlinewidth{1.003750pt}%
\definecolor{currentstroke}{rgb}{0.000000,0.000000,0.000000}%
\pgfsetstrokecolor{currentstroke}%
\pgfsetdash{}{0pt}%
\pgfsys@defobject{currentmarker}{\pgfqpoint{-0.041667in}{-0.041667in}}{\pgfqpoint{0.041667in}{0.041667in}}{%
\pgfpathmoveto{\pgfqpoint{0.000000in}{-0.041667in}}%
\pgfpathcurveto{\pgfqpoint{0.011050in}{-0.041667in}}{\pgfqpoint{0.021649in}{-0.037276in}}{\pgfqpoint{0.029463in}{-0.029463in}}%
\pgfpathcurveto{\pgfqpoint{0.037276in}{-0.021649in}}{\pgfqpoint{0.041667in}{-0.011050in}}{\pgfqpoint{0.041667in}{0.000000in}}%
\pgfpathcurveto{\pgfqpoint{0.041667in}{0.011050in}}{\pgfqpoint{0.037276in}{0.021649in}}{\pgfqpoint{0.029463in}{0.029463in}}%
\pgfpathcurveto{\pgfqpoint{0.021649in}{0.037276in}}{\pgfqpoint{0.011050in}{0.041667in}}{\pgfqpoint{0.000000in}{0.041667in}}%
\pgfpathcurveto{\pgfqpoint{-0.011050in}{0.041667in}}{\pgfqpoint{-0.021649in}{0.037276in}}{\pgfqpoint{-0.029463in}{0.029463in}}%
\pgfpathcurveto{\pgfqpoint{-0.037276in}{0.021649in}}{\pgfqpoint{-0.041667in}{0.011050in}}{\pgfqpoint{-0.041667in}{0.000000in}}%
\pgfpathcurveto{\pgfqpoint{-0.041667in}{-0.011050in}}{\pgfqpoint{-0.037276in}{-0.021649in}}{\pgfqpoint{-0.029463in}{-0.029463in}}%
\pgfpathcurveto{\pgfqpoint{-0.021649in}{-0.037276in}}{\pgfqpoint{-0.011050in}{-0.041667in}}{\pgfqpoint{0.000000in}{-0.041667in}}%
\pgfpathclose%
\pgfusepath{stroke,fill}%
}%
\begin{pgfscope}%
\pgfsys@transformshift{0.801136in}{2.792835in}%
\pgfsys@useobject{currentmarker}{}%
\end{pgfscope}%
\begin{pgfscope}%
\pgfsys@transformshift{2.163913in}{2.980508in}%
\pgfsys@useobject{currentmarker}{}%
\end{pgfscope}%
\begin{pgfscope}%
\pgfsys@transformshift{2.961087in}{2.884867in}%
\pgfsys@useobject{currentmarker}{}%
\end{pgfscope}%
\begin{pgfscope}%
\pgfsys@transformshift{3.526690in}{2.021546in}%
\pgfsys@useobject{currentmarker}{}%
\end{pgfscope}%
\begin{pgfscope}%
\pgfsys@transformshift{3.965406in}{1.449785in}%
\pgfsys@useobject{currentmarker}{}%
\end{pgfscope}%
\begin{pgfscope}%
\pgfsys@transformshift{4.323864in}{0.580000in}%
\pgfsys@useobject{currentmarker}{}%
\end{pgfscope}%
\end{pgfscope}%
\begin{pgfscope}%
\pgfsetrectcap%
\pgfsetmiterjoin%
\pgfsetlinewidth{0.803000pt}%
\definecolor{currentstroke}{rgb}{0.000000,0.000000,0.000000}%
\pgfsetstrokecolor{currentstroke}%
\pgfsetdash{}{0pt}%
\pgfpathmoveto{\pgfqpoint{0.625000in}{0.440000in}}%
\pgfpathlineto{\pgfqpoint{0.625000in}{3.520000in}}%
\pgfusepath{stroke}%
\end{pgfscope}%
\begin{pgfscope}%
\pgfsetrectcap%
\pgfsetmiterjoin%
\pgfsetlinewidth{0.803000pt}%
\definecolor{currentstroke}{rgb}{0.000000,0.000000,0.000000}%
\pgfsetstrokecolor{currentstroke}%
\pgfsetdash{}{0pt}%
\pgfpathmoveto{\pgfqpoint{4.500000in}{0.440000in}}%
\pgfpathlineto{\pgfqpoint{4.500000in}{3.520000in}}%
\pgfusepath{stroke}%
\end{pgfscope}%
\begin{pgfscope}%
\pgfsetrectcap%
\pgfsetmiterjoin%
\pgfsetlinewidth{0.803000pt}%
\definecolor{currentstroke}{rgb}{0.000000,0.000000,0.000000}%
\pgfsetstrokecolor{currentstroke}%
\pgfsetdash{}{0pt}%
\pgfpathmoveto{\pgfqpoint{0.625000in}{0.440000in}}%
\pgfpathlineto{\pgfqpoint{4.500000in}{0.440000in}}%
\pgfusepath{stroke}%
\end{pgfscope}%
\begin{pgfscope}%
\pgfsetrectcap%
\pgfsetmiterjoin%
\pgfsetlinewidth{0.803000pt}%
\definecolor{currentstroke}{rgb}{0.000000,0.000000,0.000000}%
\pgfsetstrokecolor{currentstroke}%
\pgfsetdash{}{0pt}%
\pgfpathmoveto{\pgfqpoint{0.625000in}{3.520000in}}%
\pgfpathlineto{\pgfqpoint{4.500000in}{3.520000in}}%
\pgfusepath{stroke}%
\end{pgfscope}%
\begin{pgfscope}%
\pgfsetbuttcap%
\pgfsetmiterjoin%
\definecolor{currentfill}{rgb}{1.000000,1.000000,1.000000}%
\pgfsetfillcolor{currentfill}%
\pgfsetfillopacity{0.800000}%
\pgfsetlinewidth{1.003750pt}%
\definecolor{currentstroke}{rgb}{0.800000,0.800000,0.800000}%
\pgfsetstrokecolor{currentstroke}%
\pgfsetstrokeopacity{0.800000}%
\pgfsetdash{}{0pt}%
\pgfpathmoveto{\pgfqpoint{2.423855in}{3.199199in}}%
\pgfpathlineto{\pgfqpoint{4.402778in}{3.199199in}}%
\pgfpathquadraticcurveto{\pgfqpoint{4.430556in}{3.199199in}}{\pgfqpoint{4.430556in}{3.226977in}}%
\pgfpathlineto{\pgfqpoint{4.430556in}{3.422778in}}%
\pgfpathquadraticcurveto{\pgfqpoint{4.430556in}{3.450556in}}{\pgfqpoint{4.402778in}{3.450556in}}%
\pgfpathlineto{\pgfqpoint{2.423855in}{3.450556in}}%
\pgfpathquadraticcurveto{\pgfqpoint{2.396077in}{3.450556in}}{\pgfqpoint{2.396077in}{3.422778in}}%
\pgfpathlineto{\pgfqpoint{2.396077in}{3.226977in}}%
\pgfpathquadraticcurveto{\pgfqpoint{2.396077in}{3.199199in}}{\pgfqpoint{2.423855in}{3.199199in}}%
\pgfpathclose%
\pgfusepath{stroke,fill}%
\end{pgfscope}%
\begin{pgfscope}%
\pgfsetbuttcap%
\pgfsetroundjoin%
\pgfsetlinewidth{1.505625pt}%
\definecolor{currentstroke}{rgb}{0.000000,0.000000,0.000000}%
\pgfsetstrokecolor{currentstroke}%
\pgfsetdash{{5.550000pt}{2.400000pt}}{0.000000pt}%
\pgfpathmoveto{\pgfqpoint{2.451633in}{3.338088in}}%
\pgfpathlineto{\pgfqpoint{2.729411in}{3.338088in}}%
\pgfusepath{stroke}%
\end{pgfscope}%
\begin{pgfscope}%
\definecolor{textcolor}{rgb}{0.000000,0.000000,0.000000}%
\pgfsetstrokecolor{textcolor}%
\pgfsetfillcolor{textcolor}%
\pgftext[x=2.840522in,y=3.289477in,left,base]{\color{textcolor}\sffamily\fontsize{10.000000}{12.000000}\selectfont \(\displaystyle \alpha = 1.3375 - 0.1569\mathrm{ln}(k)\)}%
\end{pgfscope}%
\end{pgfpicture}%
\makeatother%
\endgroup%

    \end{subfigure}
\caption{Plots of the dependence of $C$ and $\alpha$ on $k$ in \cref{eq:qmcerrorform} for $Q(u) = \int_D u$. Observe the $x$-axis is on a $\log_{10}$ scale, but $\loge$ is the natural logarithm. \label{fig:integralCalpha}}
\end{figure}
\optodo{Figure out how to rotate plot labels}
\begin{figure}[h]
    \centering
    \begin{subfigure}{\textwidth}
            \centering
%% Creator: Matplotlib, PGF backend
%%
%% To include the figure in your LaTeX document, write
%%   \input{<filename>.pgf}
%%
%% Make sure the required packages are loaded in your preamble
%%   \usepackage{pgf}
%%
%% Figures using additional raster images can only be included by \input if
%% they are in the same directory as the main LaTeX file. For loading figures
%% from other directories you can use the `import` package
%%   \usepackage{import}
%% and then include the figures with
%%   \import{<path to file>}{<filename>.pgf}
%%
%% Matplotlib used the following preamble
%%   \usepackage{fontspec}
%%   \setmainfont{DejaVuSerif.ttf}[Path=/home/owen/progs/firedrake-complex/firedrake/lib/python3.5/site-packages/matplotlib/mpl-data/fonts/ttf/]
%%   \setsansfont{DejaVuSans.ttf}[Path=/home/owen/progs/firedrake-complex/firedrake/lib/python3.5/site-packages/matplotlib/mpl-data/fonts/ttf/]
%%   \setmonofont{DejaVuSansMono.ttf}[Path=/home/owen/progs/firedrake-complex/firedrake/lib/python3.5/site-packages/matplotlib/mpl-data/fonts/ttf/]
%%
\begingroup%
\makeatletter%
\begin{pgfpicture}%
\pgfpathrectangle{\pgfpointorigin}{\pgfqpoint{5.000000in}{4.000000in}}%
\pgfusepath{use as bounding box, clip}%
\begin{pgfscope}%
\pgfsetbuttcap%
\pgfsetmiterjoin%
\definecolor{currentfill}{rgb}{1.000000,1.000000,1.000000}%
\pgfsetfillcolor{currentfill}%
\pgfsetlinewidth{0.000000pt}%
\definecolor{currentstroke}{rgb}{1.000000,1.000000,1.000000}%
\pgfsetstrokecolor{currentstroke}%
\pgfsetdash{}{0pt}%
\pgfpathmoveto{\pgfqpoint{0.000000in}{0.000000in}}%
\pgfpathlineto{\pgfqpoint{5.000000in}{0.000000in}}%
\pgfpathlineto{\pgfqpoint{5.000000in}{4.000000in}}%
\pgfpathlineto{\pgfqpoint{0.000000in}{4.000000in}}%
\pgfpathclose%
\pgfusepath{fill}%
\end{pgfscope}%
\begin{pgfscope}%
\pgfsetbuttcap%
\pgfsetmiterjoin%
\definecolor{currentfill}{rgb}{1.000000,1.000000,1.000000}%
\pgfsetfillcolor{currentfill}%
\pgfsetlinewidth{0.000000pt}%
\definecolor{currentstroke}{rgb}{0.000000,0.000000,0.000000}%
\pgfsetstrokecolor{currentstroke}%
\pgfsetstrokeopacity{0.000000}%
\pgfsetdash{}{0pt}%
\pgfpathmoveto{\pgfqpoint{0.625000in}{0.440000in}}%
\pgfpathlineto{\pgfqpoint{4.500000in}{0.440000in}}%
\pgfpathlineto{\pgfqpoint{4.500000in}{3.520000in}}%
\pgfpathlineto{\pgfqpoint{0.625000in}{3.520000in}}%
\pgfpathclose%
\pgfusepath{fill}%
\end{pgfscope}%
\begin{pgfscope}%
\pgfsetbuttcap%
\pgfsetroundjoin%
\definecolor{currentfill}{rgb}{0.000000,0.000000,0.000000}%
\pgfsetfillcolor{currentfill}%
\pgfsetlinewidth{0.803000pt}%
\definecolor{currentstroke}{rgb}{0.000000,0.000000,0.000000}%
\pgfsetstrokecolor{currentstroke}%
\pgfsetdash{}{0pt}%
\pgfsys@defobject{currentmarker}{\pgfqpoint{0.000000in}{-0.048611in}}{\pgfqpoint{0.000000in}{0.000000in}}{%
\pgfpathmoveto{\pgfqpoint{0.000000in}{0.000000in}}%
\pgfpathlineto{\pgfqpoint{0.000000in}{-0.048611in}}%
\pgfusepath{stroke,fill}%
}%
\begin{pgfscope}%
\pgfsys@transformshift{0.801136in}{0.440000in}%
\pgfsys@useobject{currentmarker}{}%
\end{pgfscope}%
\end{pgfscope}%
\begin{pgfscope}%
\definecolor{textcolor}{rgb}{0.000000,0.000000,0.000000}%
\pgfsetstrokecolor{textcolor}%
\pgfsetfillcolor{textcolor}%
\pgftext[x=0.801136in,y=0.342778in,,top]{\color{textcolor}\sffamily\fontsize{10.000000}{12.000000}\selectfont \(\displaystyle {10^{1}}\)}%
\end{pgfscope}%
\begin{pgfscope}%
\pgfsetbuttcap%
\pgfsetroundjoin%
\definecolor{currentfill}{rgb}{0.000000,0.000000,0.000000}%
\pgfsetfillcolor{currentfill}%
\pgfsetlinewidth{0.602250pt}%
\definecolor{currentstroke}{rgb}{0.000000,0.000000,0.000000}%
\pgfsetstrokecolor{currentstroke}%
\pgfsetdash{}{0pt}%
\pgfsys@defobject{currentmarker}{\pgfqpoint{0.000000in}{-0.027778in}}{\pgfqpoint{0.000000in}{0.000000in}}{%
\pgfpathmoveto{\pgfqpoint{0.000000in}{0.000000in}}%
\pgfpathlineto{\pgfqpoint{0.000000in}{-0.027778in}}%
\pgfusepath{stroke,fill}%
}%
\begin{pgfscope}%
\pgfsys@transformshift{2.163913in}{0.440000in}%
\pgfsys@useobject{currentmarker}{}%
\end{pgfscope}%
\end{pgfscope}%
\begin{pgfscope}%
\definecolor{textcolor}{rgb}{0.000000,0.000000,0.000000}%
\pgfsetstrokecolor{textcolor}%
\pgfsetfillcolor{textcolor}%
\pgftext[x=2.163913in,y=0.365000in,,top]{\color{textcolor}\sffamily\fontsize{10.000000}{12.000000}\selectfont \(\displaystyle {2\times10^{1}}\)}%
\end{pgfscope}%
\begin{pgfscope}%
\pgfsetbuttcap%
\pgfsetroundjoin%
\definecolor{currentfill}{rgb}{0.000000,0.000000,0.000000}%
\pgfsetfillcolor{currentfill}%
\pgfsetlinewidth{0.602250pt}%
\definecolor{currentstroke}{rgb}{0.000000,0.000000,0.000000}%
\pgfsetstrokecolor{currentstroke}%
\pgfsetdash{}{0pt}%
\pgfsys@defobject{currentmarker}{\pgfqpoint{0.000000in}{-0.027778in}}{\pgfqpoint{0.000000in}{0.000000in}}{%
\pgfpathmoveto{\pgfqpoint{0.000000in}{0.000000in}}%
\pgfpathlineto{\pgfqpoint{0.000000in}{-0.027778in}}%
\pgfusepath{stroke,fill}%
}%
\begin{pgfscope}%
\pgfsys@transformshift{2.961087in}{0.440000in}%
\pgfsys@useobject{currentmarker}{}%
\end{pgfscope}%
\end{pgfscope}%
\begin{pgfscope}%
\definecolor{textcolor}{rgb}{0.000000,0.000000,0.000000}%
\pgfsetstrokecolor{textcolor}%
\pgfsetfillcolor{textcolor}%
\pgftext[x=2.961087in,y=0.365000in,,top]{\color{textcolor}\sffamily\fontsize{10.000000}{12.000000}\selectfont \(\displaystyle {3\times10^{1}}\)}%
\end{pgfscope}%
\begin{pgfscope}%
\pgfsetbuttcap%
\pgfsetroundjoin%
\definecolor{currentfill}{rgb}{0.000000,0.000000,0.000000}%
\pgfsetfillcolor{currentfill}%
\pgfsetlinewidth{0.602250pt}%
\definecolor{currentstroke}{rgb}{0.000000,0.000000,0.000000}%
\pgfsetstrokecolor{currentstroke}%
\pgfsetdash{}{0pt}%
\pgfsys@defobject{currentmarker}{\pgfqpoint{0.000000in}{-0.027778in}}{\pgfqpoint{0.000000in}{0.000000in}}{%
\pgfpathmoveto{\pgfqpoint{0.000000in}{0.000000in}}%
\pgfpathlineto{\pgfqpoint{0.000000in}{-0.027778in}}%
\pgfusepath{stroke,fill}%
}%
\begin{pgfscope}%
\pgfsys@transformshift{3.526690in}{0.440000in}%
\pgfsys@useobject{currentmarker}{}%
\end{pgfscope}%
\end{pgfscope}%
\begin{pgfscope}%
\definecolor{textcolor}{rgb}{0.000000,0.000000,0.000000}%
\pgfsetstrokecolor{textcolor}%
\pgfsetfillcolor{textcolor}%
\pgftext[x=3.526690in,y=0.365000in,,top]{\color{textcolor}\sffamily\fontsize{10.000000}{12.000000}\selectfont \(\displaystyle {4\times10^{1}}\)}%
\end{pgfscope}%
\begin{pgfscope}%
\pgfsetbuttcap%
\pgfsetroundjoin%
\definecolor{currentfill}{rgb}{0.000000,0.000000,0.000000}%
\pgfsetfillcolor{currentfill}%
\pgfsetlinewidth{0.602250pt}%
\definecolor{currentstroke}{rgb}{0.000000,0.000000,0.000000}%
\pgfsetstrokecolor{currentstroke}%
\pgfsetdash{}{0pt}%
\pgfsys@defobject{currentmarker}{\pgfqpoint{0.000000in}{-0.027778in}}{\pgfqpoint{0.000000in}{0.000000in}}{%
\pgfpathmoveto{\pgfqpoint{0.000000in}{0.000000in}}%
\pgfpathlineto{\pgfqpoint{0.000000in}{-0.027778in}}%
\pgfusepath{stroke,fill}%
}%
\begin{pgfscope}%
\pgfsys@transformshift{3.965406in}{0.440000in}%
\pgfsys@useobject{currentmarker}{}%
\end{pgfscope}%
\end{pgfscope}%
\begin{pgfscope}%
\pgfsetbuttcap%
\pgfsetroundjoin%
\definecolor{currentfill}{rgb}{0.000000,0.000000,0.000000}%
\pgfsetfillcolor{currentfill}%
\pgfsetlinewidth{0.602250pt}%
\definecolor{currentstroke}{rgb}{0.000000,0.000000,0.000000}%
\pgfsetstrokecolor{currentstroke}%
\pgfsetdash{}{0pt}%
\pgfsys@defobject{currentmarker}{\pgfqpoint{0.000000in}{-0.027778in}}{\pgfqpoint{0.000000in}{0.000000in}}{%
\pgfpathmoveto{\pgfqpoint{0.000000in}{0.000000in}}%
\pgfpathlineto{\pgfqpoint{0.000000in}{-0.027778in}}%
\pgfusepath{stroke,fill}%
}%
\begin{pgfscope}%
\pgfsys@transformshift{4.323864in}{0.440000in}%
\pgfsys@useobject{currentmarker}{}%
\end{pgfscope}%
\end{pgfscope}%
\begin{pgfscope}%
\definecolor{textcolor}{rgb}{0.000000,0.000000,0.000000}%
\pgfsetstrokecolor{textcolor}%
\pgfsetfillcolor{textcolor}%
\pgftext[x=4.323864in,y=0.365000in,,top]{\color{textcolor}\sffamily\fontsize{10.000000}{12.000000}\selectfont \(\displaystyle {6\times10^{1}}\)}%
\end{pgfscope}%
\begin{pgfscope}%
\definecolor{textcolor}{rgb}{0.000000,0.000000,0.000000}%
\pgfsetstrokecolor{textcolor}%
\pgfsetfillcolor{textcolor}%
\pgftext[x=2.562500in,y=0.152809in,,top]{\color{textcolor}\sffamily\fontsize{10.000000}{12.000000}\selectfont k}%
\end{pgfscope}%
\begin{pgfscope}%
\pgfsetbuttcap%
\pgfsetroundjoin%
\definecolor{currentfill}{rgb}{0.000000,0.000000,0.000000}%
\pgfsetfillcolor{currentfill}%
\pgfsetlinewidth{0.803000pt}%
\definecolor{currentstroke}{rgb}{0.000000,0.000000,0.000000}%
\pgfsetstrokecolor{currentstroke}%
\pgfsetdash{}{0pt}%
\pgfsys@defobject{currentmarker}{\pgfqpoint{-0.048611in}{0.000000in}}{\pgfqpoint{0.000000in}{0.000000in}}{%
\pgfpathmoveto{\pgfqpoint{0.000000in}{0.000000in}}%
\pgfpathlineto{\pgfqpoint{-0.048611in}{0.000000in}}%
\pgfusepath{stroke,fill}%
}%
\begin{pgfscope}%
\pgfsys@transformshift{0.625000in}{3.491714in}%
\pgfsys@useobject{currentmarker}{}%
\end{pgfscope}%
\end{pgfscope}%
\begin{pgfscope}%
\definecolor{textcolor}{rgb}{0.000000,0.000000,0.000000}%
\pgfsetstrokecolor{textcolor}%
\pgfsetfillcolor{textcolor}%
\pgftext[x=0.239775in,y=3.438953in,left,base]{\color{textcolor}\sffamily\fontsize{10.000000}{12.000000}\selectfont \(\displaystyle {10^{-2}}\)}%
\end{pgfscope}%
\begin{pgfscope}%
\pgfsetbuttcap%
\pgfsetroundjoin%
\definecolor{currentfill}{rgb}{0.000000,0.000000,0.000000}%
\pgfsetfillcolor{currentfill}%
\pgfsetlinewidth{0.602250pt}%
\definecolor{currentstroke}{rgb}{0.000000,0.000000,0.000000}%
\pgfsetstrokecolor{currentstroke}%
\pgfsetdash{}{0pt}%
\pgfsys@defobject{currentmarker}{\pgfqpoint{-0.027778in}{0.000000in}}{\pgfqpoint{0.000000in}{0.000000in}}{%
\pgfpathmoveto{\pgfqpoint{0.000000in}{0.000000in}}%
\pgfpathlineto{\pgfqpoint{-0.027778in}{0.000000in}}%
\pgfusepath{stroke,fill}%
}%
\begin{pgfscope}%
\pgfsys@transformshift{0.625000in}{0.893670in}%
\pgfsys@useobject{currentmarker}{}%
\end{pgfscope}%
\end{pgfscope}%
\begin{pgfscope}%
\definecolor{textcolor}{rgb}{0.000000,0.000000,0.000000}%
\pgfsetstrokecolor{textcolor}%
\pgfsetfillcolor{textcolor}%
\pgftext[x=0.022801in,y=0.840909in,left,base]{\color{textcolor}\sffamily\fontsize{10.000000}{12.000000}\selectfont \(\displaystyle {8\times10^{-3}}\)}%
\end{pgfscope}%
\begin{pgfscope}%
\pgfsetbuttcap%
\pgfsetroundjoin%
\definecolor{currentfill}{rgb}{0.000000,0.000000,0.000000}%
\pgfsetfillcolor{currentfill}%
\pgfsetlinewidth{0.602250pt}%
\definecolor{currentstroke}{rgb}{0.000000,0.000000,0.000000}%
\pgfsetstrokecolor{currentstroke}%
\pgfsetdash{}{0pt}%
\pgfsys@defobject{currentmarker}{\pgfqpoint{-0.027778in}{0.000000in}}{\pgfqpoint{0.000000in}{0.000000in}}{%
\pgfpathmoveto{\pgfqpoint{0.000000in}{0.000000in}}%
\pgfpathlineto{\pgfqpoint{-0.027778in}{0.000000in}}%
\pgfusepath{stroke,fill}%
}%
\begin{pgfscope}%
\pgfsys@transformshift{0.625000in}{2.265010in}%
\pgfsys@useobject{currentmarker}{}%
\end{pgfscope}%
\end{pgfscope}%
\begin{pgfscope}%
\definecolor{textcolor}{rgb}{0.000000,0.000000,0.000000}%
\pgfsetstrokecolor{textcolor}%
\pgfsetfillcolor{textcolor}%
\pgftext[x=0.022801in,y=2.212248in,left,base]{\color{textcolor}\sffamily\fontsize{10.000000}{12.000000}\selectfont \(\displaystyle {9\times10^{-3}}\)}%
\end{pgfscope}%
\begin{pgfscope}%
\definecolor{textcolor}{rgb}{0.000000,0.000000,0.000000}%
\pgfsetstrokecolor{textcolor}%
\pgfsetfillcolor{textcolor}%
\pgftext[x=-0.032755in,y=1.980000in,,bottom,rotate=90.000000]{\color{textcolor}\sffamily\fontsize{10.000000}{12.000000}\selectfont C}%
\end{pgfscope}%
\begin{pgfscope}%
\pgfpathrectangle{\pgfqpoint{0.625000in}{0.440000in}}{\pgfqpoint{3.875000in}{3.080000in}}%
\pgfusepath{clip}%
\pgfsetbuttcap%
\pgfsetroundjoin%
\definecolor{currentfill}{rgb}{0.000000,0.000000,0.000000}%
\pgfsetfillcolor{currentfill}%
\pgfsetlinewidth{1.003750pt}%
\definecolor{currentstroke}{rgb}{0.000000,0.000000,0.000000}%
\pgfsetstrokecolor{currentstroke}%
\pgfsetdash{}{0pt}%
\pgfsys@defobject{currentmarker}{\pgfqpoint{-0.041667in}{-0.041667in}}{\pgfqpoint{0.041667in}{0.041667in}}{%
\pgfpathmoveto{\pgfqpoint{0.000000in}{-0.041667in}}%
\pgfpathcurveto{\pgfqpoint{0.011050in}{-0.041667in}}{\pgfqpoint{0.021649in}{-0.037276in}}{\pgfqpoint{0.029463in}{-0.029463in}}%
\pgfpathcurveto{\pgfqpoint{0.037276in}{-0.021649in}}{\pgfqpoint{0.041667in}{-0.011050in}}{\pgfqpoint{0.041667in}{0.000000in}}%
\pgfpathcurveto{\pgfqpoint{0.041667in}{0.011050in}}{\pgfqpoint{0.037276in}{0.021649in}}{\pgfqpoint{0.029463in}{0.029463in}}%
\pgfpathcurveto{\pgfqpoint{0.021649in}{0.037276in}}{\pgfqpoint{0.011050in}{0.041667in}}{\pgfqpoint{0.000000in}{0.041667in}}%
\pgfpathcurveto{\pgfqpoint{-0.011050in}{0.041667in}}{\pgfqpoint{-0.021649in}{0.037276in}}{\pgfqpoint{-0.029463in}{0.029463in}}%
\pgfpathcurveto{\pgfqpoint{-0.037276in}{0.021649in}}{\pgfqpoint{-0.041667in}{0.011050in}}{\pgfqpoint{-0.041667in}{0.000000in}}%
\pgfpathcurveto{\pgfqpoint{-0.041667in}{-0.011050in}}{\pgfqpoint{-0.037276in}{-0.021649in}}{\pgfqpoint{-0.029463in}{-0.029463in}}%
\pgfpathcurveto{\pgfqpoint{-0.021649in}{-0.037276in}}{\pgfqpoint{-0.011050in}{-0.041667in}}{\pgfqpoint{0.000000in}{-0.041667in}}%
\pgfpathclose%
\pgfusepath{stroke,fill}%
}%
\begin{pgfscope}%
\pgfsys@transformshift{0.801136in}{0.639416in}%
\pgfsys@useobject{currentmarker}{}%
\end{pgfscope}%
\begin{pgfscope}%
\pgfsys@transformshift{2.163913in}{1.106347in}%
\pgfsys@useobject{currentmarker}{}%
\end{pgfscope}%
\begin{pgfscope}%
\pgfsys@transformshift{2.961087in}{0.580000in}%
\pgfsys@useobject{currentmarker}{}%
\end{pgfscope}%
\begin{pgfscope}%
\pgfsys@transformshift{3.526690in}{3.380000in}%
\pgfsys@useobject{currentmarker}{}%
\end{pgfscope}%
\begin{pgfscope}%
\pgfsys@transformshift{3.965406in}{1.950632in}%
\pgfsys@useobject{currentmarker}{}%
\end{pgfscope}%
\begin{pgfscope}%
\pgfsys@transformshift{4.323864in}{1.870142in}%
\pgfsys@useobject{currentmarker}{}%
\end{pgfscope}%
\end{pgfscope}%
\begin{pgfscope}%
\pgfsetrectcap%
\pgfsetmiterjoin%
\pgfsetlinewidth{0.803000pt}%
\definecolor{currentstroke}{rgb}{0.000000,0.000000,0.000000}%
\pgfsetstrokecolor{currentstroke}%
\pgfsetdash{}{0pt}%
\pgfpathmoveto{\pgfqpoint{0.625000in}{0.440000in}}%
\pgfpathlineto{\pgfqpoint{0.625000in}{3.520000in}}%
\pgfusepath{stroke}%
\end{pgfscope}%
\begin{pgfscope}%
\pgfsetrectcap%
\pgfsetmiterjoin%
\pgfsetlinewidth{0.803000pt}%
\definecolor{currentstroke}{rgb}{0.000000,0.000000,0.000000}%
\pgfsetstrokecolor{currentstroke}%
\pgfsetdash{}{0pt}%
\pgfpathmoveto{\pgfqpoint{4.500000in}{0.440000in}}%
\pgfpathlineto{\pgfqpoint{4.500000in}{3.520000in}}%
\pgfusepath{stroke}%
\end{pgfscope}%
\begin{pgfscope}%
\pgfsetrectcap%
\pgfsetmiterjoin%
\pgfsetlinewidth{0.803000pt}%
\definecolor{currentstroke}{rgb}{0.000000,0.000000,0.000000}%
\pgfsetstrokecolor{currentstroke}%
\pgfsetdash{}{0pt}%
\pgfpathmoveto{\pgfqpoint{0.625000in}{0.440000in}}%
\pgfpathlineto{\pgfqpoint{4.500000in}{0.440000in}}%
\pgfusepath{stroke}%
\end{pgfscope}%
\begin{pgfscope}%
\pgfsetrectcap%
\pgfsetmiterjoin%
\pgfsetlinewidth{0.803000pt}%
\definecolor{currentstroke}{rgb}{0.000000,0.000000,0.000000}%
\pgfsetstrokecolor{currentstroke}%
\pgfsetdash{}{0pt}%
\pgfpathmoveto{\pgfqpoint{0.625000in}{3.520000in}}%
\pgfpathlineto{\pgfqpoint{4.500000in}{3.520000in}}%
\pgfusepath{stroke}%
\end{pgfscope}%
\end{pgfpicture}%
\makeatother%
\endgroup%

  \end{subfigure}
    \begin{subfigure}{\textwidth}
                \centering
%% Creator: Matplotlib, PGF backend
%%
%% To include the figure in your LaTeX document, write
%%   \input{<filename>.pgf}
%%
%% Make sure the required packages are loaded in your preamble
%%   \usepackage{pgf}
%%
%% Figures using additional raster images can only be included by \input if
%% they are in the same directory as the main LaTeX file. For loading figures
%% from other directories you can use the `import` package
%%   \usepackage{import}
%% and then include the figures with
%%   \import{<path to file>}{<filename>.pgf}
%%
%% Matplotlib used the following preamble
%%   \usepackage{fontspec}
%%   \setmainfont{DejaVuSerif.ttf}[Path=/home/owen/progs/firedrake-complex/firedrake/lib/python3.5/site-packages/matplotlib/mpl-data/fonts/ttf/]
%%   \setsansfont{DejaVuSans.ttf}[Path=/home/owen/progs/firedrake-complex/firedrake/lib/python3.5/site-packages/matplotlib/mpl-data/fonts/ttf/]
%%   \setmonofont{DejaVuSansMono.ttf}[Path=/home/owen/progs/firedrake-complex/firedrake/lib/python3.5/site-packages/matplotlib/mpl-data/fonts/ttf/]
%%
\begingroup%
\makeatletter%
\begin{pgfpicture}%
\pgfpathrectangle{\pgfpointorigin}{\pgfqpoint{5.000000in}{4.000000in}}%
\pgfusepath{use as bounding box, clip}%
\begin{pgfscope}%
\pgfsetbuttcap%
\pgfsetmiterjoin%
\definecolor{currentfill}{rgb}{1.000000,1.000000,1.000000}%
\pgfsetfillcolor{currentfill}%
\pgfsetlinewidth{0.000000pt}%
\definecolor{currentstroke}{rgb}{1.000000,1.000000,1.000000}%
\pgfsetstrokecolor{currentstroke}%
\pgfsetdash{}{0pt}%
\pgfpathmoveto{\pgfqpoint{0.000000in}{0.000000in}}%
\pgfpathlineto{\pgfqpoint{5.000000in}{0.000000in}}%
\pgfpathlineto{\pgfqpoint{5.000000in}{4.000000in}}%
\pgfpathlineto{\pgfqpoint{0.000000in}{4.000000in}}%
\pgfpathclose%
\pgfusepath{fill}%
\end{pgfscope}%
\begin{pgfscope}%
\pgfsetbuttcap%
\pgfsetmiterjoin%
\definecolor{currentfill}{rgb}{1.000000,1.000000,1.000000}%
\pgfsetfillcolor{currentfill}%
\pgfsetlinewidth{0.000000pt}%
\definecolor{currentstroke}{rgb}{0.000000,0.000000,0.000000}%
\pgfsetstrokecolor{currentstroke}%
\pgfsetstrokeopacity{0.000000}%
\pgfsetdash{}{0pt}%
\pgfpathmoveto{\pgfqpoint{0.625000in}{0.440000in}}%
\pgfpathlineto{\pgfqpoint{4.500000in}{0.440000in}}%
\pgfpathlineto{\pgfqpoint{4.500000in}{3.520000in}}%
\pgfpathlineto{\pgfqpoint{0.625000in}{3.520000in}}%
\pgfpathclose%
\pgfusepath{fill}%
\end{pgfscope}%
\begin{pgfscope}%
\pgfsetbuttcap%
\pgfsetroundjoin%
\definecolor{currentfill}{rgb}{0.000000,0.000000,0.000000}%
\pgfsetfillcolor{currentfill}%
\pgfsetlinewidth{0.803000pt}%
\definecolor{currentstroke}{rgb}{0.000000,0.000000,0.000000}%
\pgfsetstrokecolor{currentstroke}%
\pgfsetdash{}{0pt}%
\pgfsys@defobject{currentmarker}{\pgfqpoint{0.000000in}{-0.048611in}}{\pgfqpoint{0.000000in}{0.000000in}}{%
\pgfpathmoveto{\pgfqpoint{0.000000in}{0.000000in}}%
\pgfpathlineto{\pgfqpoint{0.000000in}{-0.048611in}}%
\pgfusepath{stroke,fill}%
}%
\begin{pgfscope}%
\pgfsys@transformshift{0.801136in}{0.440000in}%
\pgfsys@useobject{currentmarker}{}%
\end{pgfscope}%
\end{pgfscope}%
\begin{pgfscope}%
\definecolor{textcolor}{rgb}{0.000000,0.000000,0.000000}%
\pgfsetstrokecolor{textcolor}%
\pgfsetfillcolor{textcolor}%
\pgftext[x=0.801136in,y=0.342778in,,top]{\color{textcolor}\sffamily\fontsize{10.000000}{12.000000}\selectfont \(\displaystyle {10^{1}}\)}%
\end{pgfscope}%
\begin{pgfscope}%
\pgfsetbuttcap%
\pgfsetroundjoin%
\definecolor{currentfill}{rgb}{0.000000,0.000000,0.000000}%
\pgfsetfillcolor{currentfill}%
\pgfsetlinewidth{0.602250pt}%
\definecolor{currentstroke}{rgb}{0.000000,0.000000,0.000000}%
\pgfsetstrokecolor{currentstroke}%
\pgfsetdash{}{0pt}%
\pgfsys@defobject{currentmarker}{\pgfqpoint{0.000000in}{-0.027778in}}{\pgfqpoint{0.000000in}{0.000000in}}{%
\pgfpathmoveto{\pgfqpoint{0.000000in}{0.000000in}}%
\pgfpathlineto{\pgfqpoint{0.000000in}{-0.027778in}}%
\pgfusepath{stroke,fill}%
}%
\begin{pgfscope}%
\pgfsys@transformshift{2.163913in}{0.440000in}%
\pgfsys@useobject{currentmarker}{}%
\end{pgfscope}%
\end{pgfscope}%
\begin{pgfscope}%
\definecolor{textcolor}{rgb}{0.000000,0.000000,0.000000}%
\pgfsetstrokecolor{textcolor}%
\pgfsetfillcolor{textcolor}%
\pgftext[x=2.163913in,y=0.365000in,,top]{\color{textcolor}\sffamily\fontsize{10.000000}{12.000000}\selectfont \(\displaystyle {2\times10^{1}}\)}%
\end{pgfscope}%
\begin{pgfscope}%
\pgfsetbuttcap%
\pgfsetroundjoin%
\definecolor{currentfill}{rgb}{0.000000,0.000000,0.000000}%
\pgfsetfillcolor{currentfill}%
\pgfsetlinewidth{0.602250pt}%
\definecolor{currentstroke}{rgb}{0.000000,0.000000,0.000000}%
\pgfsetstrokecolor{currentstroke}%
\pgfsetdash{}{0pt}%
\pgfsys@defobject{currentmarker}{\pgfqpoint{0.000000in}{-0.027778in}}{\pgfqpoint{0.000000in}{0.000000in}}{%
\pgfpathmoveto{\pgfqpoint{0.000000in}{0.000000in}}%
\pgfpathlineto{\pgfqpoint{0.000000in}{-0.027778in}}%
\pgfusepath{stroke,fill}%
}%
\begin{pgfscope}%
\pgfsys@transformshift{2.961087in}{0.440000in}%
\pgfsys@useobject{currentmarker}{}%
\end{pgfscope}%
\end{pgfscope}%
\begin{pgfscope}%
\definecolor{textcolor}{rgb}{0.000000,0.000000,0.000000}%
\pgfsetstrokecolor{textcolor}%
\pgfsetfillcolor{textcolor}%
\pgftext[x=2.961087in,y=0.365000in,,top]{\color{textcolor}\sffamily\fontsize{10.000000}{12.000000}\selectfont \(\displaystyle {3\times10^{1}}\)}%
\end{pgfscope}%
\begin{pgfscope}%
\pgfsetbuttcap%
\pgfsetroundjoin%
\definecolor{currentfill}{rgb}{0.000000,0.000000,0.000000}%
\pgfsetfillcolor{currentfill}%
\pgfsetlinewidth{0.602250pt}%
\definecolor{currentstroke}{rgb}{0.000000,0.000000,0.000000}%
\pgfsetstrokecolor{currentstroke}%
\pgfsetdash{}{0pt}%
\pgfsys@defobject{currentmarker}{\pgfqpoint{0.000000in}{-0.027778in}}{\pgfqpoint{0.000000in}{0.000000in}}{%
\pgfpathmoveto{\pgfqpoint{0.000000in}{0.000000in}}%
\pgfpathlineto{\pgfqpoint{0.000000in}{-0.027778in}}%
\pgfusepath{stroke,fill}%
}%
\begin{pgfscope}%
\pgfsys@transformshift{3.526690in}{0.440000in}%
\pgfsys@useobject{currentmarker}{}%
\end{pgfscope}%
\end{pgfscope}%
\begin{pgfscope}%
\definecolor{textcolor}{rgb}{0.000000,0.000000,0.000000}%
\pgfsetstrokecolor{textcolor}%
\pgfsetfillcolor{textcolor}%
\pgftext[x=3.526690in,y=0.365000in,,top]{\color{textcolor}\sffamily\fontsize{10.000000}{12.000000}\selectfont \(\displaystyle {4\times10^{1}}\)}%
\end{pgfscope}%
\begin{pgfscope}%
\pgfsetbuttcap%
\pgfsetroundjoin%
\definecolor{currentfill}{rgb}{0.000000,0.000000,0.000000}%
\pgfsetfillcolor{currentfill}%
\pgfsetlinewidth{0.602250pt}%
\definecolor{currentstroke}{rgb}{0.000000,0.000000,0.000000}%
\pgfsetstrokecolor{currentstroke}%
\pgfsetdash{}{0pt}%
\pgfsys@defobject{currentmarker}{\pgfqpoint{0.000000in}{-0.027778in}}{\pgfqpoint{0.000000in}{0.000000in}}{%
\pgfpathmoveto{\pgfqpoint{0.000000in}{0.000000in}}%
\pgfpathlineto{\pgfqpoint{0.000000in}{-0.027778in}}%
\pgfusepath{stroke,fill}%
}%
\begin{pgfscope}%
\pgfsys@transformshift{3.965406in}{0.440000in}%
\pgfsys@useobject{currentmarker}{}%
\end{pgfscope}%
\end{pgfscope}%
\begin{pgfscope}%
\pgfsetbuttcap%
\pgfsetroundjoin%
\definecolor{currentfill}{rgb}{0.000000,0.000000,0.000000}%
\pgfsetfillcolor{currentfill}%
\pgfsetlinewidth{0.602250pt}%
\definecolor{currentstroke}{rgb}{0.000000,0.000000,0.000000}%
\pgfsetstrokecolor{currentstroke}%
\pgfsetdash{}{0pt}%
\pgfsys@defobject{currentmarker}{\pgfqpoint{0.000000in}{-0.027778in}}{\pgfqpoint{0.000000in}{0.000000in}}{%
\pgfpathmoveto{\pgfqpoint{0.000000in}{0.000000in}}%
\pgfpathlineto{\pgfqpoint{0.000000in}{-0.027778in}}%
\pgfusepath{stroke,fill}%
}%
\begin{pgfscope}%
\pgfsys@transformshift{4.323864in}{0.440000in}%
\pgfsys@useobject{currentmarker}{}%
\end{pgfscope}%
\end{pgfscope}%
\begin{pgfscope}%
\definecolor{textcolor}{rgb}{0.000000,0.000000,0.000000}%
\pgfsetstrokecolor{textcolor}%
\pgfsetfillcolor{textcolor}%
\pgftext[x=4.323864in,y=0.365000in,,top]{\color{textcolor}\sffamily\fontsize{10.000000}{12.000000}\selectfont \(\displaystyle {6\times10^{1}}\)}%
\end{pgfscope}%
\begin{pgfscope}%
\definecolor{textcolor}{rgb}{0.000000,0.000000,0.000000}%
\pgfsetstrokecolor{textcolor}%
\pgfsetfillcolor{textcolor}%
\pgftext[x=2.562500in,y=0.152809in,,top]{\color{textcolor}\sffamily\fontsize{10.000000}{12.000000}\selectfont \(\displaystyle k\)}%
\end{pgfscope}%
\begin{pgfscope}%
\pgfsetbuttcap%
\pgfsetroundjoin%
\definecolor{currentfill}{rgb}{0.000000,0.000000,0.000000}%
\pgfsetfillcolor{currentfill}%
\pgfsetlinewidth{0.803000pt}%
\definecolor{currentstroke}{rgb}{0.000000,0.000000,0.000000}%
\pgfsetstrokecolor{currentstroke}%
\pgfsetdash{}{0pt}%
\pgfsys@defobject{currentmarker}{\pgfqpoint{-0.048611in}{0.000000in}}{\pgfqpoint{0.000000in}{0.000000in}}{%
\pgfpathmoveto{\pgfqpoint{0.000000in}{0.000000in}}%
\pgfpathlineto{\pgfqpoint{-0.048611in}{0.000000in}}%
\pgfusepath{stroke,fill}%
}%
\begin{pgfscope}%
\pgfsys@transformshift{0.625000in}{0.560614in}%
\pgfsys@useobject{currentmarker}{}%
\end{pgfscope}%
\end{pgfscope}%
\begin{pgfscope}%
\definecolor{textcolor}{rgb}{0.000000,0.000000,0.000000}%
\pgfsetstrokecolor{textcolor}%
\pgfsetfillcolor{textcolor}%
\pgftext[x=0.218533in,y=0.507853in,left,base]{\color{textcolor}\sffamily\fontsize{10.000000}{12.000000}\selectfont 0.60}%
\end{pgfscope}%
\begin{pgfscope}%
\pgfsetbuttcap%
\pgfsetroundjoin%
\definecolor{currentfill}{rgb}{0.000000,0.000000,0.000000}%
\pgfsetfillcolor{currentfill}%
\pgfsetlinewidth{0.803000pt}%
\definecolor{currentstroke}{rgb}{0.000000,0.000000,0.000000}%
\pgfsetstrokecolor{currentstroke}%
\pgfsetdash{}{0pt}%
\pgfsys@defobject{currentmarker}{\pgfqpoint{-0.048611in}{0.000000in}}{\pgfqpoint{0.000000in}{0.000000in}}{%
\pgfpathmoveto{\pgfqpoint{0.000000in}{0.000000in}}%
\pgfpathlineto{\pgfqpoint{-0.048611in}{0.000000in}}%
\pgfusepath{stroke,fill}%
}%
\begin{pgfscope}%
\pgfsys@transformshift{0.625000in}{0.956056in}%
\pgfsys@useobject{currentmarker}{}%
\end{pgfscope}%
\end{pgfscope}%
\begin{pgfscope}%
\definecolor{textcolor}{rgb}{0.000000,0.000000,0.000000}%
\pgfsetstrokecolor{textcolor}%
\pgfsetfillcolor{textcolor}%
\pgftext[x=0.218533in,y=0.903294in,left,base]{\color{textcolor}\sffamily\fontsize{10.000000}{12.000000}\selectfont 0.65}%
\end{pgfscope}%
\begin{pgfscope}%
\pgfsetbuttcap%
\pgfsetroundjoin%
\definecolor{currentfill}{rgb}{0.000000,0.000000,0.000000}%
\pgfsetfillcolor{currentfill}%
\pgfsetlinewidth{0.803000pt}%
\definecolor{currentstroke}{rgb}{0.000000,0.000000,0.000000}%
\pgfsetstrokecolor{currentstroke}%
\pgfsetdash{}{0pt}%
\pgfsys@defobject{currentmarker}{\pgfqpoint{-0.048611in}{0.000000in}}{\pgfqpoint{0.000000in}{0.000000in}}{%
\pgfpathmoveto{\pgfqpoint{0.000000in}{0.000000in}}%
\pgfpathlineto{\pgfqpoint{-0.048611in}{0.000000in}}%
\pgfusepath{stroke,fill}%
}%
\begin{pgfscope}%
\pgfsys@transformshift{0.625000in}{1.351498in}%
\pgfsys@useobject{currentmarker}{}%
\end{pgfscope}%
\end{pgfscope}%
\begin{pgfscope}%
\definecolor{textcolor}{rgb}{0.000000,0.000000,0.000000}%
\pgfsetstrokecolor{textcolor}%
\pgfsetfillcolor{textcolor}%
\pgftext[x=0.218533in,y=1.298736in,left,base]{\color{textcolor}\sffamily\fontsize{10.000000}{12.000000}\selectfont 0.70}%
\end{pgfscope}%
\begin{pgfscope}%
\pgfsetbuttcap%
\pgfsetroundjoin%
\definecolor{currentfill}{rgb}{0.000000,0.000000,0.000000}%
\pgfsetfillcolor{currentfill}%
\pgfsetlinewidth{0.803000pt}%
\definecolor{currentstroke}{rgb}{0.000000,0.000000,0.000000}%
\pgfsetstrokecolor{currentstroke}%
\pgfsetdash{}{0pt}%
\pgfsys@defobject{currentmarker}{\pgfqpoint{-0.048611in}{0.000000in}}{\pgfqpoint{0.000000in}{0.000000in}}{%
\pgfpathmoveto{\pgfqpoint{0.000000in}{0.000000in}}%
\pgfpathlineto{\pgfqpoint{-0.048611in}{0.000000in}}%
\pgfusepath{stroke,fill}%
}%
\begin{pgfscope}%
\pgfsys@transformshift{0.625000in}{1.746939in}%
\pgfsys@useobject{currentmarker}{}%
\end{pgfscope}%
\end{pgfscope}%
\begin{pgfscope}%
\definecolor{textcolor}{rgb}{0.000000,0.000000,0.000000}%
\pgfsetstrokecolor{textcolor}%
\pgfsetfillcolor{textcolor}%
\pgftext[x=0.218533in,y=1.694178in,left,base]{\color{textcolor}\sffamily\fontsize{10.000000}{12.000000}\selectfont 0.75}%
\end{pgfscope}%
\begin{pgfscope}%
\pgfsetbuttcap%
\pgfsetroundjoin%
\definecolor{currentfill}{rgb}{0.000000,0.000000,0.000000}%
\pgfsetfillcolor{currentfill}%
\pgfsetlinewidth{0.803000pt}%
\definecolor{currentstroke}{rgb}{0.000000,0.000000,0.000000}%
\pgfsetstrokecolor{currentstroke}%
\pgfsetdash{}{0pt}%
\pgfsys@defobject{currentmarker}{\pgfqpoint{-0.048611in}{0.000000in}}{\pgfqpoint{0.000000in}{0.000000in}}{%
\pgfpathmoveto{\pgfqpoint{0.000000in}{0.000000in}}%
\pgfpathlineto{\pgfqpoint{-0.048611in}{0.000000in}}%
\pgfusepath{stroke,fill}%
}%
\begin{pgfscope}%
\pgfsys@transformshift{0.625000in}{2.142381in}%
\pgfsys@useobject{currentmarker}{}%
\end{pgfscope}%
\end{pgfscope}%
\begin{pgfscope}%
\definecolor{textcolor}{rgb}{0.000000,0.000000,0.000000}%
\pgfsetstrokecolor{textcolor}%
\pgfsetfillcolor{textcolor}%
\pgftext[x=0.218533in,y=2.089619in,left,base]{\color{textcolor}\sffamily\fontsize{10.000000}{12.000000}\selectfont 0.80}%
\end{pgfscope}%
\begin{pgfscope}%
\pgfsetbuttcap%
\pgfsetroundjoin%
\definecolor{currentfill}{rgb}{0.000000,0.000000,0.000000}%
\pgfsetfillcolor{currentfill}%
\pgfsetlinewidth{0.803000pt}%
\definecolor{currentstroke}{rgb}{0.000000,0.000000,0.000000}%
\pgfsetstrokecolor{currentstroke}%
\pgfsetdash{}{0pt}%
\pgfsys@defobject{currentmarker}{\pgfqpoint{-0.048611in}{0.000000in}}{\pgfqpoint{0.000000in}{0.000000in}}{%
\pgfpathmoveto{\pgfqpoint{0.000000in}{0.000000in}}%
\pgfpathlineto{\pgfqpoint{-0.048611in}{0.000000in}}%
\pgfusepath{stroke,fill}%
}%
\begin{pgfscope}%
\pgfsys@transformshift{0.625000in}{2.537823in}%
\pgfsys@useobject{currentmarker}{}%
\end{pgfscope}%
\end{pgfscope}%
\begin{pgfscope}%
\definecolor{textcolor}{rgb}{0.000000,0.000000,0.000000}%
\pgfsetstrokecolor{textcolor}%
\pgfsetfillcolor{textcolor}%
\pgftext[x=0.218533in,y=2.485061in,left,base]{\color{textcolor}\sffamily\fontsize{10.000000}{12.000000}\selectfont 0.85}%
\end{pgfscope}%
\begin{pgfscope}%
\pgfsetbuttcap%
\pgfsetroundjoin%
\definecolor{currentfill}{rgb}{0.000000,0.000000,0.000000}%
\pgfsetfillcolor{currentfill}%
\pgfsetlinewidth{0.803000pt}%
\definecolor{currentstroke}{rgb}{0.000000,0.000000,0.000000}%
\pgfsetstrokecolor{currentstroke}%
\pgfsetdash{}{0pt}%
\pgfsys@defobject{currentmarker}{\pgfqpoint{-0.048611in}{0.000000in}}{\pgfqpoint{0.000000in}{0.000000in}}{%
\pgfpathmoveto{\pgfqpoint{0.000000in}{0.000000in}}%
\pgfpathlineto{\pgfqpoint{-0.048611in}{0.000000in}}%
\pgfusepath{stroke,fill}%
}%
\begin{pgfscope}%
\pgfsys@transformshift{0.625000in}{2.933264in}%
\pgfsys@useobject{currentmarker}{}%
\end{pgfscope}%
\end{pgfscope}%
\begin{pgfscope}%
\definecolor{textcolor}{rgb}{0.000000,0.000000,0.000000}%
\pgfsetstrokecolor{textcolor}%
\pgfsetfillcolor{textcolor}%
\pgftext[x=0.218533in,y=2.880503in,left,base]{\color{textcolor}\sffamily\fontsize{10.000000}{12.000000}\selectfont 0.90}%
\end{pgfscope}%
\begin{pgfscope}%
\pgfsetbuttcap%
\pgfsetroundjoin%
\definecolor{currentfill}{rgb}{0.000000,0.000000,0.000000}%
\pgfsetfillcolor{currentfill}%
\pgfsetlinewidth{0.803000pt}%
\definecolor{currentstroke}{rgb}{0.000000,0.000000,0.000000}%
\pgfsetstrokecolor{currentstroke}%
\pgfsetdash{}{0pt}%
\pgfsys@defobject{currentmarker}{\pgfqpoint{-0.048611in}{0.000000in}}{\pgfqpoint{0.000000in}{0.000000in}}{%
\pgfpathmoveto{\pgfqpoint{0.000000in}{0.000000in}}%
\pgfpathlineto{\pgfqpoint{-0.048611in}{0.000000in}}%
\pgfusepath{stroke,fill}%
}%
\begin{pgfscope}%
\pgfsys@transformshift{0.625000in}{3.328706in}%
\pgfsys@useobject{currentmarker}{}%
\end{pgfscope}%
\end{pgfscope}%
\begin{pgfscope}%
\definecolor{textcolor}{rgb}{0.000000,0.000000,0.000000}%
\pgfsetstrokecolor{textcolor}%
\pgfsetfillcolor{textcolor}%
\pgftext[x=0.218533in,y=3.275945in,left,base]{\color{textcolor}\sffamily\fontsize{10.000000}{12.000000}\selectfont 0.95}%
\end{pgfscope}%
\begin{pgfscope}%
\definecolor{textcolor}{rgb}{0.000000,0.000000,0.000000}%
\pgfsetstrokecolor{textcolor}%
\pgfsetfillcolor{textcolor}%
\pgftext[x=0.162977in,y=1.980000in,,bottom,rotate=90.000000]{\color{textcolor}\sffamily\fontsize{10.000000}{12.000000}\selectfont \(\displaystyle \alpha\)}%
\end{pgfscope}%
\begin{pgfscope}%
\pgfpathrectangle{\pgfqpoint{0.625000in}{0.440000in}}{\pgfqpoint{3.875000in}{3.080000in}}%
\pgfusepath{clip}%
\pgfsetbuttcap%
\pgfsetroundjoin%
\pgfsetlinewidth{1.505625pt}%
\definecolor{currentstroke}{rgb}{0.000000,0.000000,0.000000}%
\pgfsetstrokecolor{currentstroke}%
\pgfsetdash{{5.550000pt}{2.400000pt}}{0.000000pt}%
\pgfpathmoveto{\pgfqpoint{0.801136in}{3.269246in}}%
\pgfpathlineto{\pgfqpoint{2.163913in}{2.228904in}}%
\pgfpathlineto{\pgfqpoint{2.961087in}{1.620343in}}%
\pgfpathlineto{\pgfqpoint{3.526690in}{1.188561in}}%
\pgfpathlineto{\pgfqpoint{3.965406in}{0.853646in}}%
\pgfpathlineto{\pgfqpoint{4.323864in}{0.580000in}}%
\pgfusepath{stroke}%
\end{pgfscope}%
\begin{pgfscope}%
\pgfpathrectangle{\pgfqpoint{0.625000in}{0.440000in}}{\pgfqpoint{3.875000in}{3.080000in}}%
\pgfusepath{clip}%
\pgfsetbuttcap%
\pgfsetroundjoin%
\definecolor{currentfill}{rgb}{0.000000,0.000000,0.000000}%
\pgfsetfillcolor{currentfill}%
\pgfsetlinewidth{1.003750pt}%
\definecolor{currentstroke}{rgb}{0.000000,0.000000,0.000000}%
\pgfsetstrokecolor{currentstroke}%
\pgfsetdash{}{0pt}%
\pgfsys@defobject{currentmarker}{\pgfqpoint{-0.041667in}{-0.041667in}}{\pgfqpoint{0.041667in}{0.041667in}}{%
\pgfpathmoveto{\pgfqpoint{0.000000in}{-0.041667in}}%
\pgfpathcurveto{\pgfqpoint{0.011050in}{-0.041667in}}{\pgfqpoint{0.021649in}{-0.037276in}}{\pgfqpoint{0.029463in}{-0.029463in}}%
\pgfpathcurveto{\pgfqpoint{0.037276in}{-0.021649in}}{\pgfqpoint{0.041667in}{-0.011050in}}{\pgfqpoint{0.041667in}{0.000000in}}%
\pgfpathcurveto{\pgfqpoint{0.041667in}{0.011050in}}{\pgfqpoint{0.037276in}{0.021649in}}{\pgfqpoint{0.029463in}{0.029463in}}%
\pgfpathcurveto{\pgfqpoint{0.021649in}{0.037276in}}{\pgfqpoint{0.011050in}{0.041667in}}{\pgfqpoint{0.000000in}{0.041667in}}%
\pgfpathcurveto{\pgfqpoint{-0.011050in}{0.041667in}}{\pgfqpoint{-0.021649in}{0.037276in}}{\pgfqpoint{-0.029463in}{0.029463in}}%
\pgfpathcurveto{\pgfqpoint{-0.037276in}{0.021649in}}{\pgfqpoint{-0.041667in}{0.011050in}}{\pgfqpoint{-0.041667in}{0.000000in}}%
\pgfpathcurveto{\pgfqpoint{-0.041667in}{-0.011050in}}{\pgfqpoint{-0.037276in}{-0.021649in}}{\pgfqpoint{-0.029463in}{-0.029463in}}%
\pgfpathcurveto{\pgfqpoint{-0.021649in}{-0.037276in}}{\pgfqpoint{-0.011050in}{-0.041667in}}{\pgfqpoint{0.000000in}{-0.041667in}}%
\pgfpathclose%
\pgfusepath{stroke,fill}%
}%
\begin{pgfscope}%
\pgfsys@transformshift{0.801136in}{3.380000in}%
\pgfsys@useobject{currentmarker}{}%
\end{pgfscope}%
\begin{pgfscope}%
\pgfsys@transformshift{2.163913in}{2.359432in}%
\pgfsys@useobject{currentmarker}{}%
\end{pgfscope}%
\begin{pgfscope}%
\pgfsys@transformshift{2.961087in}{1.117792in}%
\pgfsys@useobject{currentmarker}{}%
\end{pgfscope}%
\begin{pgfscope}%
\pgfsys@transformshift{3.526690in}{1.211877in}%
\pgfsys@useobject{currentmarker}{}%
\end{pgfscope}%
\begin{pgfscope}%
\pgfsys@transformshift{3.965406in}{0.837435in}%
\pgfsys@useobject{currentmarker}{}%
\end{pgfscope}%
\begin{pgfscope}%
\pgfsys@transformshift{4.323864in}{0.834164in}%
\pgfsys@useobject{currentmarker}{}%
\end{pgfscope}%
\end{pgfscope}%
\begin{pgfscope}%
\pgfsetrectcap%
\pgfsetmiterjoin%
\pgfsetlinewidth{0.803000pt}%
\definecolor{currentstroke}{rgb}{0.000000,0.000000,0.000000}%
\pgfsetstrokecolor{currentstroke}%
\pgfsetdash{}{0pt}%
\pgfpathmoveto{\pgfqpoint{0.625000in}{0.440000in}}%
\pgfpathlineto{\pgfqpoint{0.625000in}{3.520000in}}%
\pgfusepath{stroke}%
\end{pgfscope}%
\begin{pgfscope}%
\pgfsetrectcap%
\pgfsetmiterjoin%
\pgfsetlinewidth{0.803000pt}%
\definecolor{currentstroke}{rgb}{0.000000,0.000000,0.000000}%
\pgfsetstrokecolor{currentstroke}%
\pgfsetdash{}{0pt}%
\pgfpathmoveto{\pgfqpoint{4.500000in}{0.440000in}}%
\pgfpathlineto{\pgfqpoint{4.500000in}{3.520000in}}%
\pgfusepath{stroke}%
\end{pgfscope}%
\begin{pgfscope}%
\pgfsetrectcap%
\pgfsetmiterjoin%
\pgfsetlinewidth{0.803000pt}%
\definecolor{currentstroke}{rgb}{0.000000,0.000000,0.000000}%
\pgfsetstrokecolor{currentstroke}%
\pgfsetdash{}{0pt}%
\pgfpathmoveto{\pgfqpoint{0.625000in}{0.440000in}}%
\pgfpathlineto{\pgfqpoint{4.500000in}{0.440000in}}%
\pgfusepath{stroke}%
\end{pgfscope}%
\begin{pgfscope}%
\pgfsetrectcap%
\pgfsetmiterjoin%
\pgfsetlinewidth{0.803000pt}%
\definecolor{currentstroke}{rgb}{0.000000,0.000000,0.000000}%
\pgfsetstrokecolor{currentstroke}%
\pgfsetdash{}{0pt}%
\pgfpathmoveto{\pgfqpoint{0.625000in}{3.520000in}}%
\pgfpathlineto{\pgfqpoint{4.500000in}{3.520000in}}%
\pgfusepath{stroke}%
\end{pgfscope}%
\begin{pgfscope}%
\pgfsetbuttcap%
\pgfsetmiterjoin%
\definecolor{currentfill}{rgb}{1.000000,1.000000,1.000000}%
\pgfsetfillcolor{currentfill}%
\pgfsetfillopacity{0.800000}%
\pgfsetlinewidth{1.003750pt}%
\definecolor{currentstroke}{rgb}{0.800000,0.800000,0.800000}%
\pgfsetstrokecolor{currentstroke}%
\pgfsetstrokeopacity{0.800000}%
\pgfsetdash{}{0pt}%
\pgfpathmoveto{\pgfqpoint{2.423855in}{3.199199in}}%
\pgfpathlineto{\pgfqpoint{4.402778in}{3.199199in}}%
\pgfpathquadraticcurveto{\pgfqpoint{4.430556in}{3.199199in}}{\pgfqpoint{4.430556in}{3.226977in}}%
\pgfpathlineto{\pgfqpoint{4.430556in}{3.422778in}}%
\pgfpathquadraticcurveto{\pgfqpoint{4.430556in}{3.450556in}}{\pgfqpoint{4.402778in}{3.450556in}}%
\pgfpathlineto{\pgfqpoint{2.423855in}{3.450556in}}%
\pgfpathquadraticcurveto{\pgfqpoint{2.396077in}{3.450556in}}{\pgfqpoint{2.396077in}{3.422778in}}%
\pgfpathlineto{\pgfqpoint{2.396077in}{3.226977in}}%
\pgfpathquadraticcurveto{\pgfqpoint{2.396077in}{3.199199in}}{\pgfqpoint{2.423855in}{3.199199in}}%
\pgfpathclose%
\pgfusepath{stroke,fill}%
\end{pgfscope}%
\begin{pgfscope}%
\pgfsetbuttcap%
\pgfsetroundjoin%
\pgfsetlinewidth{1.505625pt}%
\definecolor{currentstroke}{rgb}{0.000000,0.000000,0.000000}%
\pgfsetstrokecolor{currentstroke}%
\pgfsetdash{{5.550000pt}{2.400000pt}}{0.000000pt}%
\pgfpathmoveto{\pgfqpoint{2.451633in}{3.338088in}}%
\pgfpathlineto{\pgfqpoint{2.729411in}{3.338088in}}%
\pgfusepath{stroke}%
\end{pgfscope}%
\begin{pgfscope}%
\definecolor{textcolor}{rgb}{0.000000,0.000000,0.000000}%
\pgfsetstrokecolor{textcolor}%
\pgfsetfillcolor{textcolor}%
\pgftext[x=2.840522in,y=3.289477in,left,base]{\color{textcolor}\sffamily\fontsize{10.000000}{12.000000}\selectfont \(\displaystyle \alpha = 1.3794 - 0.1897\mathrm{ln}(k)\)}%
\end{pgfscope}%
\end{pgfpicture}%
\makeatother%
\endgroup%

    \end{subfigure}

\caption{Plots of the dependence of $C$ and $\alpha$ on $k$ in \cref{eq:qmcerrorform} for $Q(u) =  u(\bzero)$. Observe the $x$-axis is on a $\log_{10}$ scale, but $\loge$ is the natural logarithm. \label{fig:originCalpha}}
\end{figure}

\begin{figure}[h]
    \centering
    \begin{subfigure}{\textwidth}
            \centering
%% Creator: Matplotlib, PGF backend
%%
%% To include the figure in your LaTeX document, write
%%   \input{<filename>.pgf}
%%
%% Make sure the required packages are loaded in your preamble
%%   \usepackage{pgf}
%%
%% Figures using additional raster images can only be included by \input if
%% they are in the same directory as the main LaTeX file. For loading figures
%% from other directories you can use the `import` package
%%   \usepackage{import}
%% and then include the figures with
%%   \import{<path to file>}{<filename>.pgf}
%%
%% Matplotlib used the following preamble
%%   \usepackage{fontspec}
%%   \setmainfont{DejaVuSerif.ttf}[Path=/home/owen/progs/firedrake-complex/firedrake/lib/python3.5/site-packages/matplotlib/mpl-data/fonts/ttf/]
%%   \setsansfont{DejaVuSans.ttf}[Path=/home/owen/progs/firedrake-complex/firedrake/lib/python3.5/site-packages/matplotlib/mpl-data/fonts/ttf/]
%%   \setmonofont{DejaVuSansMono.ttf}[Path=/home/owen/progs/firedrake-complex/firedrake/lib/python3.5/site-packages/matplotlib/mpl-data/fonts/ttf/]
%%
\begingroup%
\makeatletter%
\begin{pgfpicture}%
\pgfpathrectangle{\pgfpointorigin}{\pgfqpoint{5.000000in}{4.000000in}}%
\pgfusepath{use as bounding box, clip}%
\begin{pgfscope}%
\pgfsetbuttcap%
\pgfsetmiterjoin%
\definecolor{currentfill}{rgb}{1.000000,1.000000,1.000000}%
\pgfsetfillcolor{currentfill}%
\pgfsetlinewidth{0.000000pt}%
\definecolor{currentstroke}{rgb}{1.000000,1.000000,1.000000}%
\pgfsetstrokecolor{currentstroke}%
\pgfsetdash{}{0pt}%
\pgfpathmoveto{\pgfqpoint{0.000000in}{0.000000in}}%
\pgfpathlineto{\pgfqpoint{5.000000in}{0.000000in}}%
\pgfpathlineto{\pgfqpoint{5.000000in}{4.000000in}}%
\pgfpathlineto{\pgfqpoint{0.000000in}{4.000000in}}%
\pgfpathclose%
\pgfusepath{fill}%
\end{pgfscope}%
\begin{pgfscope}%
\pgfsetbuttcap%
\pgfsetmiterjoin%
\definecolor{currentfill}{rgb}{1.000000,1.000000,1.000000}%
\pgfsetfillcolor{currentfill}%
\pgfsetlinewidth{0.000000pt}%
\definecolor{currentstroke}{rgb}{0.000000,0.000000,0.000000}%
\pgfsetstrokecolor{currentstroke}%
\pgfsetstrokeopacity{0.000000}%
\pgfsetdash{}{0pt}%
\pgfpathmoveto{\pgfqpoint{0.688770in}{0.582778in}}%
\pgfpathlineto{\pgfqpoint{4.810222in}{0.582778in}}%
\pgfpathlineto{\pgfqpoint{4.810222in}{3.815000in}}%
\pgfpathlineto{\pgfqpoint{0.688770in}{3.815000in}}%
\pgfpathclose%
\pgfusepath{fill}%
\end{pgfscope}%
\begin{pgfscope}%
\pgfsetbuttcap%
\pgfsetroundjoin%
\definecolor{currentfill}{rgb}{0.000000,0.000000,0.000000}%
\pgfsetfillcolor{currentfill}%
\pgfsetlinewidth{0.803000pt}%
\definecolor{currentstroke}{rgb}{0.000000,0.000000,0.000000}%
\pgfsetstrokecolor{currentstroke}%
\pgfsetdash{}{0pt}%
\pgfsys@defobject{currentmarker}{\pgfqpoint{0.000000in}{-0.048611in}}{\pgfqpoint{0.000000in}{0.000000in}}{%
\pgfpathmoveto{\pgfqpoint{0.000000in}{0.000000in}}%
\pgfpathlineto{\pgfqpoint{0.000000in}{-0.048611in}}%
\pgfusepath{stroke,fill}%
}%
\begin{pgfscope}%
\pgfsys@transformshift{0.876108in}{0.582778in}%
\pgfsys@useobject{currentmarker}{}%
\end{pgfscope}%
\end{pgfscope}%
\begin{pgfscope}%
\definecolor{textcolor}{rgb}{0.000000,0.000000,0.000000}%
\pgfsetstrokecolor{textcolor}%
\pgfsetfillcolor{textcolor}%
\pgftext[x=0.876108in,y=0.485556in,,top]{\color{textcolor}\sffamily\fontsize{10.000000}{12.000000}\selectfont \(\displaystyle 10^{1}\)}%
\end{pgfscope}%
\begin{pgfscope}%
\pgfsetbuttcap%
\pgfsetroundjoin%
\definecolor{currentfill}{rgb}{0.000000,0.000000,0.000000}%
\pgfsetfillcolor{currentfill}%
\pgfsetlinewidth{0.602250pt}%
\definecolor{currentstroke}{rgb}{0.000000,0.000000,0.000000}%
\pgfsetstrokecolor{currentstroke}%
\pgfsetdash{}{0pt}%
\pgfsys@defobject{currentmarker}{\pgfqpoint{0.000000in}{-0.027778in}}{\pgfqpoint{0.000000in}{0.000000in}}{%
\pgfpathmoveto{\pgfqpoint{0.000000in}{0.000000in}}%
\pgfpathlineto{\pgfqpoint{0.000000in}{-0.027778in}}%
\pgfusepath{stroke,fill}%
}%
\begin{pgfscope}%
\pgfsys@transformshift{2.325559in}{0.582778in}%
\pgfsys@useobject{currentmarker}{}%
\end{pgfscope}%
\end{pgfscope}%
\begin{pgfscope}%
\definecolor{textcolor}{rgb}{0.000000,0.000000,0.000000}%
\pgfsetstrokecolor{textcolor}%
\pgfsetfillcolor{textcolor}%
\pgftext[x=2.325559in,y=0.507778in,,top]{\color{textcolor}\sffamily\fontsize{10.000000}{12.000000}\selectfont \(\displaystyle 2\times10^{1}\)}%
\end{pgfscope}%
\begin{pgfscope}%
\pgfsetbuttcap%
\pgfsetroundjoin%
\definecolor{currentfill}{rgb}{0.000000,0.000000,0.000000}%
\pgfsetfillcolor{currentfill}%
\pgfsetlinewidth{0.602250pt}%
\definecolor{currentstroke}{rgb}{0.000000,0.000000,0.000000}%
\pgfsetstrokecolor{currentstroke}%
\pgfsetdash{}{0pt}%
\pgfsys@defobject{currentmarker}{\pgfqpoint{0.000000in}{-0.027778in}}{\pgfqpoint{0.000000in}{0.000000in}}{%
\pgfpathmoveto{\pgfqpoint{0.000000in}{0.000000in}}%
\pgfpathlineto{\pgfqpoint{0.000000in}{-0.027778in}}%
\pgfusepath{stroke,fill}%
}%
\begin{pgfscope}%
\pgfsys@transformshift{3.173433in}{0.582778in}%
\pgfsys@useobject{currentmarker}{}%
\end{pgfscope}%
\end{pgfscope}%
\begin{pgfscope}%
\definecolor{textcolor}{rgb}{0.000000,0.000000,0.000000}%
\pgfsetstrokecolor{textcolor}%
\pgfsetfillcolor{textcolor}%
\pgftext[x=3.173433in,y=0.507778in,,top]{\color{textcolor}\sffamily\fontsize{10.000000}{12.000000}\selectfont \(\displaystyle 3\times10^{1}\)}%
\end{pgfscope}%
\begin{pgfscope}%
\pgfsetbuttcap%
\pgfsetroundjoin%
\definecolor{currentfill}{rgb}{0.000000,0.000000,0.000000}%
\pgfsetfillcolor{currentfill}%
\pgfsetlinewidth{0.602250pt}%
\definecolor{currentstroke}{rgb}{0.000000,0.000000,0.000000}%
\pgfsetstrokecolor{currentstroke}%
\pgfsetdash{}{0pt}%
\pgfsys@defobject{currentmarker}{\pgfqpoint{0.000000in}{-0.027778in}}{\pgfqpoint{0.000000in}{0.000000in}}{%
\pgfpathmoveto{\pgfqpoint{0.000000in}{0.000000in}}%
\pgfpathlineto{\pgfqpoint{0.000000in}{-0.027778in}}%
\pgfusepath{stroke,fill}%
}%
\begin{pgfscope}%
\pgfsys@transformshift{3.775009in}{0.582778in}%
\pgfsys@useobject{currentmarker}{}%
\end{pgfscope}%
\end{pgfscope}%
\begin{pgfscope}%
\definecolor{textcolor}{rgb}{0.000000,0.000000,0.000000}%
\pgfsetstrokecolor{textcolor}%
\pgfsetfillcolor{textcolor}%
\pgftext[x=3.775009in,y=0.507778in,,top]{\color{textcolor}\sffamily\fontsize{10.000000}{12.000000}\selectfont \(\displaystyle 4\times10^{1}\)}%
\end{pgfscope}%
\begin{pgfscope}%
\pgfsetbuttcap%
\pgfsetroundjoin%
\definecolor{currentfill}{rgb}{0.000000,0.000000,0.000000}%
\pgfsetfillcolor{currentfill}%
\pgfsetlinewidth{0.602250pt}%
\definecolor{currentstroke}{rgb}{0.000000,0.000000,0.000000}%
\pgfsetstrokecolor{currentstroke}%
\pgfsetdash{}{0pt}%
\pgfsys@defobject{currentmarker}{\pgfqpoint{0.000000in}{-0.027778in}}{\pgfqpoint{0.000000in}{0.000000in}}{%
\pgfpathmoveto{\pgfqpoint{0.000000in}{0.000000in}}%
\pgfpathlineto{\pgfqpoint{0.000000in}{-0.027778in}}%
\pgfusepath{stroke,fill}%
}%
\begin{pgfscope}%
\pgfsys@transformshift{4.241628in}{0.582778in}%
\pgfsys@useobject{currentmarker}{}%
\end{pgfscope}%
\end{pgfscope}%
\begin{pgfscope}%
\pgfsetbuttcap%
\pgfsetroundjoin%
\definecolor{currentfill}{rgb}{0.000000,0.000000,0.000000}%
\pgfsetfillcolor{currentfill}%
\pgfsetlinewidth{0.602250pt}%
\definecolor{currentstroke}{rgb}{0.000000,0.000000,0.000000}%
\pgfsetstrokecolor{currentstroke}%
\pgfsetdash{}{0pt}%
\pgfsys@defobject{currentmarker}{\pgfqpoint{0.000000in}{-0.027778in}}{\pgfqpoint{0.000000in}{0.000000in}}{%
\pgfpathmoveto{\pgfqpoint{0.000000in}{0.000000in}}%
\pgfpathlineto{\pgfqpoint{0.000000in}{-0.027778in}}%
\pgfusepath{stroke,fill}%
}%
\begin{pgfscope}%
\pgfsys@transformshift{4.622883in}{0.582778in}%
\pgfsys@useobject{currentmarker}{}%
\end{pgfscope}%
\end{pgfscope}%
\begin{pgfscope}%
\definecolor{textcolor}{rgb}{0.000000,0.000000,0.000000}%
\pgfsetstrokecolor{textcolor}%
\pgfsetfillcolor{textcolor}%
\pgftext[x=4.622883in,y=0.507778in,,top]{\color{textcolor}\sffamily\fontsize{10.000000}{12.000000}\selectfont \(\displaystyle 6\times10^{1}\)}%
\end{pgfscope}%
\begin{pgfscope}%
\definecolor{textcolor}{rgb}{0.000000,0.000000,0.000000}%
\pgfsetstrokecolor{textcolor}%
\pgfsetfillcolor{textcolor}%
\pgftext[x=2.749496in,y=0.295587in,,top]{\color{textcolor}\sffamily\fontsize{10.000000}{12.000000}\selectfont \(\displaystyle k\)}%
\end{pgfscope}%
\begin{pgfscope}%
\pgfsetbuttcap%
\pgfsetroundjoin%
\definecolor{currentfill}{rgb}{0.000000,0.000000,0.000000}%
\pgfsetfillcolor{currentfill}%
\pgfsetlinewidth{0.803000pt}%
\definecolor{currentstroke}{rgb}{0.000000,0.000000,0.000000}%
\pgfsetstrokecolor{currentstroke}%
\pgfsetdash{}{0pt}%
\pgfsys@defobject{currentmarker}{\pgfqpoint{-0.048611in}{0.000000in}}{\pgfqpoint{0.000000in}{0.000000in}}{%
\pgfpathmoveto{\pgfqpoint{0.000000in}{0.000000in}}%
\pgfpathlineto{\pgfqpoint{-0.048611in}{0.000000in}}%
\pgfusepath{stroke,fill}%
}%
\begin{pgfscope}%
\pgfsys@transformshift{0.688770in}{0.934647in}%
\pgfsys@useobject{currentmarker}{}%
\end{pgfscope}%
\end{pgfscope}%
\begin{pgfscope}%
\definecolor{textcolor}{rgb}{0.000000,0.000000,0.000000}%
\pgfsetstrokecolor{textcolor}%
\pgfsetfillcolor{textcolor}%
\pgftext[x=0.344633in,y=0.881886in,left,base]{\color{textcolor}\sffamily\fontsize{10.000000}{12.000000}\selectfont \(\displaystyle 0.10\)}%
\end{pgfscope}%
\begin{pgfscope}%
\pgfsetbuttcap%
\pgfsetroundjoin%
\definecolor{currentfill}{rgb}{0.000000,0.000000,0.000000}%
\pgfsetfillcolor{currentfill}%
\pgfsetlinewidth{0.803000pt}%
\definecolor{currentstroke}{rgb}{0.000000,0.000000,0.000000}%
\pgfsetstrokecolor{currentstroke}%
\pgfsetdash{}{0pt}%
\pgfsys@defobject{currentmarker}{\pgfqpoint{-0.048611in}{0.000000in}}{\pgfqpoint{0.000000in}{0.000000in}}{%
\pgfpathmoveto{\pgfqpoint{0.000000in}{0.000000in}}%
\pgfpathlineto{\pgfqpoint{-0.048611in}{0.000000in}}%
\pgfusepath{stroke,fill}%
}%
\begin{pgfscope}%
\pgfsys@transformshift{0.688770in}{1.553460in}%
\pgfsys@useobject{currentmarker}{}%
\end{pgfscope}%
\end{pgfscope}%
\begin{pgfscope}%
\definecolor{textcolor}{rgb}{0.000000,0.000000,0.000000}%
\pgfsetstrokecolor{textcolor}%
\pgfsetfillcolor{textcolor}%
\pgftext[x=0.344633in,y=1.500698in,left,base]{\color{textcolor}\sffamily\fontsize{10.000000}{12.000000}\selectfont \(\displaystyle 0.12\)}%
\end{pgfscope}%
\begin{pgfscope}%
\pgfsetbuttcap%
\pgfsetroundjoin%
\definecolor{currentfill}{rgb}{0.000000,0.000000,0.000000}%
\pgfsetfillcolor{currentfill}%
\pgfsetlinewidth{0.803000pt}%
\definecolor{currentstroke}{rgb}{0.000000,0.000000,0.000000}%
\pgfsetstrokecolor{currentstroke}%
\pgfsetdash{}{0pt}%
\pgfsys@defobject{currentmarker}{\pgfqpoint{-0.048611in}{0.000000in}}{\pgfqpoint{0.000000in}{0.000000in}}{%
\pgfpathmoveto{\pgfqpoint{0.000000in}{0.000000in}}%
\pgfpathlineto{\pgfqpoint{-0.048611in}{0.000000in}}%
\pgfusepath{stroke,fill}%
}%
\begin{pgfscope}%
\pgfsys@transformshift{0.688770in}{2.172272in}%
\pgfsys@useobject{currentmarker}{}%
\end{pgfscope}%
\end{pgfscope}%
\begin{pgfscope}%
\definecolor{textcolor}{rgb}{0.000000,0.000000,0.000000}%
\pgfsetstrokecolor{textcolor}%
\pgfsetfillcolor{textcolor}%
\pgftext[x=0.344633in,y=2.119510in,left,base]{\color{textcolor}\sffamily\fontsize{10.000000}{12.000000}\selectfont \(\displaystyle 0.14\)}%
\end{pgfscope}%
\begin{pgfscope}%
\pgfsetbuttcap%
\pgfsetroundjoin%
\definecolor{currentfill}{rgb}{0.000000,0.000000,0.000000}%
\pgfsetfillcolor{currentfill}%
\pgfsetlinewidth{0.803000pt}%
\definecolor{currentstroke}{rgb}{0.000000,0.000000,0.000000}%
\pgfsetstrokecolor{currentstroke}%
\pgfsetdash{}{0pt}%
\pgfsys@defobject{currentmarker}{\pgfqpoint{-0.048611in}{0.000000in}}{\pgfqpoint{0.000000in}{0.000000in}}{%
\pgfpathmoveto{\pgfqpoint{0.000000in}{0.000000in}}%
\pgfpathlineto{\pgfqpoint{-0.048611in}{0.000000in}}%
\pgfusepath{stroke,fill}%
}%
\begin{pgfscope}%
\pgfsys@transformshift{0.688770in}{2.791084in}%
\pgfsys@useobject{currentmarker}{}%
\end{pgfscope}%
\end{pgfscope}%
\begin{pgfscope}%
\definecolor{textcolor}{rgb}{0.000000,0.000000,0.000000}%
\pgfsetstrokecolor{textcolor}%
\pgfsetfillcolor{textcolor}%
\pgftext[x=0.344633in,y=2.738322in,left,base]{\color{textcolor}\sffamily\fontsize{10.000000}{12.000000}\selectfont \(\displaystyle 0.16\)}%
\end{pgfscope}%
\begin{pgfscope}%
\pgfsetbuttcap%
\pgfsetroundjoin%
\definecolor{currentfill}{rgb}{0.000000,0.000000,0.000000}%
\pgfsetfillcolor{currentfill}%
\pgfsetlinewidth{0.803000pt}%
\definecolor{currentstroke}{rgb}{0.000000,0.000000,0.000000}%
\pgfsetstrokecolor{currentstroke}%
\pgfsetdash{}{0pt}%
\pgfsys@defobject{currentmarker}{\pgfqpoint{-0.048611in}{0.000000in}}{\pgfqpoint{0.000000in}{0.000000in}}{%
\pgfpathmoveto{\pgfqpoint{0.000000in}{0.000000in}}%
\pgfpathlineto{\pgfqpoint{-0.048611in}{0.000000in}}%
\pgfusepath{stroke,fill}%
}%
\begin{pgfscope}%
\pgfsys@transformshift{0.688770in}{3.409896in}%
\pgfsys@useobject{currentmarker}{}%
\end{pgfscope}%
\end{pgfscope}%
\begin{pgfscope}%
\definecolor{textcolor}{rgb}{0.000000,0.000000,0.000000}%
\pgfsetstrokecolor{textcolor}%
\pgfsetfillcolor{textcolor}%
\pgftext[x=0.344633in,y=3.357135in,left,base]{\color{textcolor}\sffamily\fontsize{10.000000}{12.000000}\selectfont \(\displaystyle 0.18\)}%
\end{pgfscope}%
\begin{pgfscope}%
\definecolor{textcolor}{rgb}{0.000000,0.000000,0.000000}%
\pgfsetstrokecolor{textcolor}%
\pgfsetfillcolor{textcolor}%
\pgftext[x=0.289078in,y=2.198889in,,bottom,rotate=90.000000]{\color{textcolor}\sffamily\fontsize{10.000000}{12.000000}\selectfont \(\displaystyle C\)}%
\end{pgfscope}%
\begin{pgfscope}%
\pgfpathrectangle{\pgfqpoint{0.688770in}{0.582778in}}{\pgfqpoint{4.121452in}{3.232222in}}%
\pgfusepath{clip}%
\pgfsetbuttcap%
\pgfsetroundjoin%
\definecolor{currentfill}{rgb}{0.000000,0.000000,0.000000}%
\pgfsetfillcolor{currentfill}%
\pgfsetlinewidth{1.003750pt}%
\definecolor{currentstroke}{rgb}{0.000000,0.000000,0.000000}%
\pgfsetstrokecolor{currentstroke}%
\pgfsetdash{}{0pt}%
\pgfsys@defobject{currentmarker}{\pgfqpoint{-0.041667in}{-0.041667in}}{\pgfqpoint{0.041667in}{0.041667in}}{%
\pgfpathmoveto{\pgfqpoint{0.000000in}{-0.041667in}}%
\pgfpathcurveto{\pgfqpoint{0.011050in}{-0.041667in}}{\pgfqpoint{0.021649in}{-0.037276in}}{\pgfqpoint{0.029463in}{-0.029463in}}%
\pgfpathcurveto{\pgfqpoint{0.037276in}{-0.021649in}}{\pgfqpoint{0.041667in}{-0.011050in}}{\pgfqpoint{0.041667in}{0.000000in}}%
\pgfpathcurveto{\pgfqpoint{0.041667in}{0.011050in}}{\pgfqpoint{0.037276in}{0.021649in}}{\pgfqpoint{0.029463in}{0.029463in}}%
\pgfpathcurveto{\pgfqpoint{0.021649in}{0.037276in}}{\pgfqpoint{0.011050in}{0.041667in}}{\pgfqpoint{0.000000in}{0.041667in}}%
\pgfpathcurveto{\pgfqpoint{-0.011050in}{0.041667in}}{\pgfqpoint{-0.021649in}{0.037276in}}{\pgfqpoint{-0.029463in}{0.029463in}}%
\pgfpathcurveto{\pgfqpoint{-0.037276in}{0.021649in}}{\pgfqpoint{-0.041667in}{0.011050in}}{\pgfqpoint{-0.041667in}{0.000000in}}%
\pgfpathcurveto{\pgfqpoint{-0.041667in}{-0.011050in}}{\pgfqpoint{-0.037276in}{-0.021649in}}{\pgfqpoint{-0.029463in}{-0.029463in}}%
\pgfpathcurveto{\pgfqpoint{-0.021649in}{-0.037276in}}{\pgfqpoint{-0.011050in}{-0.041667in}}{\pgfqpoint{0.000000in}{-0.041667in}}%
\pgfpathclose%
\pgfusepath{stroke,fill}%
}%
\begin{pgfscope}%
\pgfsys@transformshift{0.876108in}{0.729697in}%
\pgfsys@useobject{currentmarker}{}%
\end{pgfscope}%
\begin{pgfscope}%
\pgfsys@transformshift{2.325559in}{1.966349in}%
\pgfsys@useobject{currentmarker}{}%
\end{pgfscope}%
\begin{pgfscope}%
\pgfsys@transformshift{3.173433in}{3.441230in}%
\pgfsys@useobject{currentmarker}{}%
\end{pgfscope}%
\begin{pgfscope}%
\pgfsys@transformshift{3.775009in}{2.830242in}%
\pgfsys@useobject{currentmarker}{}%
\end{pgfscope}%
\begin{pgfscope}%
\pgfsys@transformshift{4.241628in}{3.668081in}%
\pgfsys@useobject{currentmarker}{}%
\end{pgfscope}%
\begin{pgfscope}%
\pgfsys@transformshift{4.622883in}{2.286315in}%
\pgfsys@useobject{currentmarker}{}%
\end{pgfscope}%
\end{pgfscope}%
\begin{pgfscope}%
\pgfsetrectcap%
\pgfsetmiterjoin%
\pgfsetlinewidth{0.803000pt}%
\definecolor{currentstroke}{rgb}{0.000000,0.000000,0.000000}%
\pgfsetstrokecolor{currentstroke}%
\pgfsetdash{}{0pt}%
\pgfpathmoveto{\pgfqpoint{0.688770in}{0.582778in}}%
\pgfpathlineto{\pgfqpoint{0.688770in}{3.815000in}}%
\pgfusepath{stroke}%
\end{pgfscope}%
\begin{pgfscope}%
\pgfsetrectcap%
\pgfsetmiterjoin%
\pgfsetlinewidth{0.803000pt}%
\definecolor{currentstroke}{rgb}{0.000000,0.000000,0.000000}%
\pgfsetstrokecolor{currentstroke}%
\pgfsetdash{}{0pt}%
\pgfpathmoveto{\pgfqpoint{4.810222in}{0.582778in}}%
\pgfpathlineto{\pgfqpoint{4.810222in}{3.815000in}}%
\pgfusepath{stroke}%
\end{pgfscope}%
\begin{pgfscope}%
\pgfsetrectcap%
\pgfsetmiterjoin%
\pgfsetlinewidth{0.803000pt}%
\definecolor{currentstroke}{rgb}{0.000000,0.000000,0.000000}%
\pgfsetstrokecolor{currentstroke}%
\pgfsetdash{}{0pt}%
\pgfpathmoveto{\pgfqpoint{0.688770in}{0.582778in}}%
\pgfpathlineto{\pgfqpoint{4.810222in}{0.582778in}}%
\pgfusepath{stroke}%
\end{pgfscope}%
\begin{pgfscope}%
\pgfsetrectcap%
\pgfsetmiterjoin%
\pgfsetlinewidth{0.803000pt}%
\definecolor{currentstroke}{rgb}{0.000000,0.000000,0.000000}%
\pgfsetstrokecolor{currentstroke}%
\pgfsetdash{}{0pt}%
\pgfpathmoveto{\pgfqpoint{0.688770in}{3.815000in}}%
\pgfpathlineto{\pgfqpoint{4.810222in}{3.815000in}}%
\pgfusepath{stroke}%
\end{pgfscope}%
\end{pgfpicture}%
\makeatother%
\endgroup%

  \end{subfigure}
    \begin{subfigure}{\textwidth}
            \centering
%% Creator: Matplotlib, PGF backend
%%
%% To include the figure in your LaTeX document, write
%%   \input{<filename>.pgf}
%%
%% Make sure the required packages are loaded in your preamble
%%   \usepackage{pgf}
%%
%% Figures using additional raster images can only be included by \input if
%% they are in the same directory as the main LaTeX file. For loading figures
%% from other directories you can use the `import` package
%%   \usepackage{import}
%% and then include the figures with
%%   \import{<path to file>}{<filename>.pgf}
%%
%% Matplotlib used the following preamble
%%   \usepackage{fontspec}
%%   \setmainfont{DejaVuSerif.ttf}[Path=/home/owen/progs/firedrake-complex/firedrake/lib/python3.5/site-packages/matplotlib/mpl-data/fonts/ttf/]
%%   \setsansfont{DejaVuSans.ttf}[Path=/home/owen/progs/firedrake-complex/firedrake/lib/python3.5/site-packages/matplotlib/mpl-data/fonts/ttf/]
%%   \setmonofont{DejaVuSansMono.ttf}[Path=/home/owen/progs/firedrake-complex/firedrake/lib/python3.5/site-packages/matplotlib/mpl-data/fonts/ttf/]
%%
\begingroup%
\makeatletter%
\begin{pgfpicture}%
\pgfpathrectangle{\pgfpointorigin}{\pgfqpoint{5.000000in}{4.000000in}}%
\pgfusepath{use as bounding box, clip}%
\begin{pgfscope}%
\pgfsetbuttcap%
\pgfsetmiterjoin%
\definecolor{currentfill}{rgb}{1.000000,1.000000,1.000000}%
\pgfsetfillcolor{currentfill}%
\pgfsetlinewidth{0.000000pt}%
\definecolor{currentstroke}{rgb}{1.000000,1.000000,1.000000}%
\pgfsetstrokecolor{currentstroke}%
\pgfsetdash{}{0pt}%
\pgfpathmoveto{\pgfqpoint{0.000000in}{0.000000in}}%
\pgfpathlineto{\pgfqpoint{5.000000in}{0.000000in}}%
\pgfpathlineto{\pgfqpoint{5.000000in}{4.000000in}}%
\pgfpathlineto{\pgfqpoint{0.000000in}{4.000000in}}%
\pgfpathclose%
\pgfusepath{fill}%
\end{pgfscope}%
\begin{pgfscope}%
\pgfsetbuttcap%
\pgfsetmiterjoin%
\definecolor{currentfill}{rgb}{1.000000,1.000000,1.000000}%
\pgfsetfillcolor{currentfill}%
\pgfsetlinewidth{0.000000pt}%
\definecolor{currentstroke}{rgb}{0.000000,0.000000,0.000000}%
\pgfsetstrokecolor{currentstroke}%
\pgfsetstrokeopacity{0.000000}%
\pgfsetdash{}{0pt}%
\pgfpathmoveto{\pgfqpoint{0.625000in}{0.440000in}}%
\pgfpathlineto{\pgfqpoint{4.500000in}{0.440000in}}%
\pgfpathlineto{\pgfqpoint{4.500000in}{3.520000in}}%
\pgfpathlineto{\pgfqpoint{0.625000in}{3.520000in}}%
\pgfpathclose%
\pgfusepath{fill}%
\end{pgfscope}%
\begin{pgfscope}%
\pgfsetbuttcap%
\pgfsetroundjoin%
\definecolor{currentfill}{rgb}{0.000000,0.000000,0.000000}%
\pgfsetfillcolor{currentfill}%
\pgfsetlinewidth{0.803000pt}%
\definecolor{currentstroke}{rgb}{0.000000,0.000000,0.000000}%
\pgfsetstrokecolor{currentstroke}%
\pgfsetdash{}{0pt}%
\pgfsys@defobject{currentmarker}{\pgfqpoint{0.000000in}{-0.048611in}}{\pgfqpoint{0.000000in}{0.000000in}}{%
\pgfpathmoveto{\pgfqpoint{0.000000in}{0.000000in}}%
\pgfpathlineto{\pgfqpoint{0.000000in}{-0.048611in}}%
\pgfusepath{stroke,fill}%
}%
\begin{pgfscope}%
\pgfsys@transformshift{0.801136in}{0.440000in}%
\pgfsys@useobject{currentmarker}{}%
\end{pgfscope}%
\end{pgfscope}%
\begin{pgfscope}%
\definecolor{textcolor}{rgb}{0.000000,0.000000,0.000000}%
\pgfsetstrokecolor{textcolor}%
\pgfsetfillcolor{textcolor}%
\pgftext[x=0.801136in,y=0.342778in,,top]{\color{textcolor}\sffamily\fontsize{10.000000}{12.000000}\selectfont \(\displaystyle 10^{1}\)}%
\end{pgfscope}%
\begin{pgfscope}%
\pgfsetbuttcap%
\pgfsetroundjoin%
\definecolor{currentfill}{rgb}{0.000000,0.000000,0.000000}%
\pgfsetfillcolor{currentfill}%
\pgfsetlinewidth{0.602250pt}%
\definecolor{currentstroke}{rgb}{0.000000,0.000000,0.000000}%
\pgfsetstrokecolor{currentstroke}%
\pgfsetdash{}{0pt}%
\pgfsys@defobject{currentmarker}{\pgfqpoint{0.000000in}{-0.027778in}}{\pgfqpoint{0.000000in}{0.000000in}}{%
\pgfpathmoveto{\pgfqpoint{0.000000in}{0.000000in}}%
\pgfpathlineto{\pgfqpoint{0.000000in}{-0.027778in}}%
\pgfusepath{stroke,fill}%
}%
\begin{pgfscope}%
\pgfsys@transformshift{2.163913in}{0.440000in}%
\pgfsys@useobject{currentmarker}{}%
\end{pgfscope}%
\end{pgfscope}%
\begin{pgfscope}%
\definecolor{textcolor}{rgb}{0.000000,0.000000,0.000000}%
\pgfsetstrokecolor{textcolor}%
\pgfsetfillcolor{textcolor}%
\pgftext[x=2.163913in,y=0.365000in,,top]{\color{textcolor}\sffamily\fontsize{10.000000}{12.000000}\selectfont \(\displaystyle 2\times10^{1}\)}%
\end{pgfscope}%
\begin{pgfscope}%
\pgfsetbuttcap%
\pgfsetroundjoin%
\definecolor{currentfill}{rgb}{0.000000,0.000000,0.000000}%
\pgfsetfillcolor{currentfill}%
\pgfsetlinewidth{0.602250pt}%
\definecolor{currentstroke}{rgb}{0.000000,0.000000,0.000000}%
\pgfsetstrokecolor{currentstroke}%
\pgfsetdash{}{0pt}%
\pgfsys@defobject{currentmarker}{\pgfqpoint{0.000000in}{-0.027778in}}{\pgfqpoint{0.000000in}{0.000000in}}{%
\pgfpathmoveto{\pgfqpoint{0.000000in}{0.000000in}}%
\pgfpathlineto{\pgfqpoint{0.000000in}{-0.027778in}}%
\pgfusepath{stroke,fill}%
}%
\begin{pgfscope}%
\pgfsys@transformshift{2.961087in}{0.440000in}%
\pgfsys@useobject{currentmarker}{}%
\end{pgfscope}%
\end{pgfscope}%
\begin{pgfscope}%
\definecolor{textcolor}{rgb}{0.000000,0.000000,0.000000}%
\pgfsetstrokecolor{textcolor}%
\pgfsetfillcolor{textcolor}%
\pgftext[x=2.961087in,y=0.365000in,,top]{\color{textcolor}\sffamily\fontsize{10.000000}{12.000000}\selectfont \(\displaystyle 3\times10^{1}\)}%
\end{pgfscope}%
\begin{pgfscope}%
\pgfsetbuttcap%
\pgfsetroundjoin%
\definecolor{currentfill}{rgb}{0.000000,0.000000,0.000000}%
\pgfsetfillcolor{currentfill}%
\pgfsetlinewidth{0.602250pt}%
\definecolor{currentstroke}{rgb}{0.000000,0.000000,0.000000}%
\pgfsetstrokecolor{currentstroke}%
\pgfsetdash{}{0pt}%
\pgfsys@defobject{currentmarker}{\pgfqpoint{0.000000in}{-0.027778in}}{\pgfqpoint{0.000000in}{0.000000in}}{%
\pgfpathmoveto{\pgfqpoint{0.000000in}{0.000000in}}%
\pgfpathlineto{\pgfqpoint{0.000000in}{-0.027778in}}%
\pgfusepath{stroke,fill}%
}%
\begin{pgfscope}%
\pgfsys@transformshift{3.526690in}{0.440000in}%
\pgfsys@useobject{currentmarker}{}%
\end{pgfscope}%
\end{pgfscope}%
\begin{pgfscope}%
\definecolor{textcolor}{rgb}{0.000000,0.000000,0.000000}%
\pgfsetstrokecolor{textcolor}%
\pgfsetfillcolor{textcolor}%
\pgftext[x=3.526690in,y=0.365000in,,top]{\color{textcolor}\sffamily\fontsize{10.000000}{12.000000}\selectfont \(\displaystyle 4\times10^{1}\)}%
\end{pgfscope}%
\begin{pgfscope}%
\pgfsetbuttcap%
\pgfsetroundjoin%
\definecolor{currentfill}{rgb}{0.000000,0.000000,0.000000}%
\pgfsetfillcolor{currentfill}%
\pgfsetlinewidth{0.602250pt}%
\definecolor{currentstroke}{rgb}{0.000000,0.000000,0.000000}%
\pgfsetstrokecolor{currentstroke}%
\pgfsetdash{}{0pt}%
\pgfsys@defobject{currentmarker}{\pgfqpoint{0.000000in}{-0.027778in}}{\pgfqpoint{0.000000in}{0.000000in}}{%
\pgfpathmoveto{\pgfqpoint{0.000000in}{0.000000in}}%
\pgfpathlineto{\pgfqpoint{0.000000in}{-0.027778in}}%
\pgfusepath{stroke,fill}%
}%
\begin{pgfscope}%
\pgfsys@transformshift{3.965406in}{0.440000in}%
\pgfsys@useobject{currentmarker}{}%
\end{pgfscope}%
\end{pgfscope}%
\begin{pgfscope}%
\pgfsetbuttcap%
\pgfsetroundjoin%
\definecolor{currentfill}{rgb}{0.000000,0.000000,0.000000}%
\pgfsetfillcolor{currentfill}%
\pgfsetlinewidth{0.602250pt}%
\definecolor{currentstroke}{rgb}{0.000000,0.000000,0.000000}%
\pgfsetstrokecolor{currentstroke}%
\pgfsetdash{}{0pt}%
\pgfsys@defobject{currentmarker}{\pgfqpoint{0.000000in}{-0.027778in}}{\pgfqpoint{0.000000in}{0.000000in}}{%
\pgfpathmoveto{\pgfqpoint{0.000000in}{0.000000in}}%
\pgfpathlineto{\pgfqpoint{0.000000in}{-0.027778in}}%
\pgfusepath{stroke,fill}%
}%
\begin{pgfscope}%
\pgfsys@transformshift{4.323864in}{0.440000in}%
\pgfsys@useobject{currentmarker}{}%
\end{pgfscope}%
\end{pgfscope}%
\begin{pgfscope}%
\definecolor{textcolor}{rgb}{0.000000,0.000000,0.000000}%
\pgfsetstrokecolor{textcolor}%
\pgfsetfillcolor{textcolor}%
\pgftext[x=4.323864in,y=0.365000in,,top]{\color{textcolor}\sffamily\fontsize{10.000000}{12.000000}\selectfont \(\displaystyle 6\times10^{1}\)}%
\end{pgfscope}%
\begin{pgfscope}%
\definecolor{textcolor}{rgb}{0.000000,0.000000,0.000000}%
\pgfsetstrokecolor{textcolor}%
\pgfsetfillcolor{textcolor}%
\pgftext[x=2.562500in,y=0.152809in,,top]{\color{textcolor}\sffamily\fontsize{10.000000}{12.000000}\selectfont \(\displaystyle k\)}%
\end{pgfscope}%
\begin{pgfscope}%
\pgfsetbuttcap%
\pgfsetroundjoin%
\definecolor{currentfill}{rgb}{0.000000,0.000000,0.000000}%
\pgfsetfillcolor{currentfill}%
\pgfsetlinewidth{0.803000pt}%
\definecolor{currentstroke}{rgb}{0.000000,0.000000,0.000000}%
\pgfsetstrokecolor{currentstroke}%
\pgfsetdash{}{0pt}%
\pgfsys@defobject{currentmarker}{\pgfqpoint{-0.048611in}{0.000000in}}{\pgfqpoint{0.000000in}{0.000000in}}{%
\pgfpathmoveto{\pgfqpoint{0.000000in}{0.000000in}}%
\pgfpathlineto{\pgfqpoint{-0.048611in}{0.000000in}}%
\pgfusepath{stroke,fill}%
}%
\begin{pgfscope}%
\pgfsys@transformshift{0.625000in}{0.750342in}%
\pgfsys@useobject{currentmarker}{}%
\end{pgfscope}%
\end{pgfscope}%
\begin{pgfscope}%
\definecolor{textcolor}{rgb}{0.000000,0.000000,0.000000}%
\pgfsetstrokecolor{textcolor}%
\pgfsetfillcolor{textcolor}%
\pgftext[x=0.350308in,y=0.697581in,left,base]{\color{textcolor}\sffamily\fontsize{10.000000}{12.000000}\selectfont \(\displaystyle 0.6\)}%
\end{pgfscope}%
\begin{pgfscope}%
\pgfsetbuttcap%
\pgfsetroundjoin%
\definecolor{currentfill}{rgb}{0.000000,0.000000,0.000000}%
\pgfsetfillcolor{currentfill}%
\pgfsetlinewidth{0.803000pt}%
\definecolor{currentstroke}{rgb}{0.000000,0.000000,0.000000}%
\pgfsetstrokecolor{currentstroke}%
\pgfsetdash{}{0pt}%
\pgfsys@defobject{currentmarker}{\pgfqpoint{-0.048611in}{0.000000in}}{\pgfqpoint{0.000000in}{0.000000in}}{%
\pgfpathmoveto{\pgfqpoint{0.000000in}{0.000000in}}%
\pgfpathlineto{\pgfqpoint{-0.048611in}{0.000000in}}%
\pgfusepath{stroke,fill}%
}%
\begin{pgfscope}%
\pgfsys@transformshift{0.625000in}{1.353780in}%
\pgfsys@useobject{currentmarker}{}%
\end{pgfscope}%
\end{pgfscope}%
\begin{pgfscope}%
\definecolor{textcolor}{rgb}{0.000000,0.000000,0.000000}%
\pgfsetstrokecolor{textcolor}%
\pgfsetfillcolor{textcolor}%
\pgftext[x=0.350308in,y=1.301019in,left,base]{\color{textcolor}\sffamily\fontsize{10.000000}{12.000000}\selectfont \(\displaystyle 0.7\)}%
\end{pgfscope}%
\begin{pgfscope}%
\pgfsetbuttcap%
\pgfsetroundjoin%
\definecolor{currentfill}{rgb}{0.000000,0.000000,0.000000}%
\pgfsetfillcolor{currentfill}%
\pgfsetlinewidth{0.803000pt}%
\definecolor{currentstroke}{rgb}{0.000000,0.000000,0.000000}%
\pgfsetstrokecolor{currentstroke}%
\pgfsetdash{}{0pt}%
\pgfsys@defobject{currentmarker}{\pgfqpoint{-0.048611in}{0.000000in}}{\pgfqpoint{0.000000in}{0.000000in}}{%
\pgfpathmoveto{\pgfqpoint{0.000000in}{0.000000in}}%
\pgfpathlineto{\pgfqpoint{-0.048611in}{0.000000in}}%
\pgfusepath{stroke,fill}%
}%
\begin{pgfscope}%
\pgfsys@transformshift{0.625000in}{1.957218in}%
\pgfsys@useobject{currentmarker}{}%
\end{pgfscope}%
\end{pgfscope}%
\begin{pgfscope}%
\definecolor{textcolor}{rgb}{0.000000,0.000000,0.000000}%
\pgfsetstrokecolor{textcolor}%
\pgfsetfillcolor{textcolor}%
\pgftext[x=0.350308in,y=1.904456in,left,base]{\color{textcolor}\sffamily\fontsize{10.000000}{12.000000}\selectfont \(\displaystyle 0.8\)}%
\end{pgfscope}%
\begin{pgfscope}%
\pgfsetbuttcap%
\pgfsetroundjoin%
\definecolor{currentfill}{rgb}{0.000000,0.000000,0.000000}%
\pgfsetfillcolor{currentfill}%
\pgfsetlinewidth{0.803000pt}%
\definecolor{currentstroke}{rgb}{0.000000,0.000000,0.000000}%
\pgfsetstrokecolor{currentstroke}%
\pgfsetdash{}{0pt}%
\pgfsys@defobject{currentmarker}{\pgfqpoint{-0.048611in}{0.000000in}}{\pgfqpoint{0.000000in}{0.000000in}}{%
\pgfpathmoveto{\pgfqpoint{0.000000in}{0.000000in}}%
\pgfpathlineto{\pgfqpoint{-0.048611in}{0.000000in}}%
\pgfusepath{stroke,fill}%
}%
\begin{pgfscope}%
\pgfsys@transformshift{0.625000in}{2.560656in}%
\pgfsys@useobject{currentmarker}{}%
\end{pgfscope}%
\end{pgfscope}%
\begin{pgfscope}%
\definecolor{textcolor}{rgb}{0.000000,0.000000,0.000000}%
\pgfsetstrokecolor{textcolor}%
\pgfsetfillcolor{textcolor}%
\pgftext[x=0.350308in,y=2.507894in,left,base]{\color{textcolor}\sffamily\fontsize{10.000000}{12.000000}\selectfont \(\displaystyle 0.9\)}%
\end{pgfscope}%
\begin{pgfscope}%
\pgfsetbuttcap%
\pgfsetroundjoin%
\definecolor{currentfill}{rgb}{0.000000,0.000000,0.000000}%
\pgfsetfillcolor{currentfill}%
\pgfsetlinewidth{0.803000pt}%
\definecolor{currentstroke}{rgb}{0.000000,0.000000,0.000000}%
\pgfsetstrokecolor{currentstroke}%
\pgfsetdash{}{0pt}%
\pgfsys@defobject{currentmarker}{\pgfqpoint{-0.048611in}{0.000000in}}{\pgfqpoint{0.000000in}{0.000000in}}{%
\pgfpathmoveto{\pgfqpoint{0.000000in}{0.000000in}}%
\pgfpathlineto{\pgfqpoint{-0.048611in}{0.000000in}}%
\pgfusepath{stroke,fill}%
}%
\begin{pgfscope}%
\pgfsys@transformshift{0.625000in}{3.164093in}%
\pgfsys@useobject{currentmarker}{}%
\end{pgfscope}%
\end{pgfscope}%
\begin{pgfscope}%
\definecolor{textcolor}{rgb}{0.000000,0.000000,0.000000}%
\pgfsetstrokecolor{textcolor}%
\pgfsetfillcolor{textcolor}%
\pgftext[x=0.350308in,y=3.111332in,left,base]{\color{textcolor}\sffamily\fontsize{10.000000}{12.000000}\selectfont \(\displaystyle 1.0\)}%
\end{pgfscope}%
\begin{pgfscope}%
\definecolor{textcolor}{rgb}{0.000000,0.000000,0.000000}%
\pgfsetstrokecolor{textcolor}%
\pgfsetfillcolor{textcolor}%
\pgftext[x=0.294753in,y=1.980000in,,bottom,rotate=90.000000]{\color{textcolor}\sffamily\fontsize{10.000000}{12.000000}\selectfont \(\displaystyle \alpha\)}%
\end{pgfscope}%
\begin{pgfscope}%
\pgfpathrectangle{\pgfqpoint{0.625000in}{0.440000in}}{\pgfqpoint{3.875000in}{3.080000in}}%
\pgfusepath{clip}%
\pgfsetbuttcap%
\pgfsetroundjoin%
\pgfsetlinewidth{1.505625pt}%
\definecolor{currentstroke}{rgb}{0.000000,0.000000,0.000000}%
\pgfsetstrokecolor{currentstroke}%
\pgfsetdash{{5.550000pt}{2.400000pt}}{0.000000pt}%
\pgfpathmoveto{\pgfqpoint{0.801136in}{3.380000in}}%
\pgfpathlineto{\pgfqpoint{2.163913in}{2.516130in}}%
\pgfpathlineto{\pgfqpoint{2.961087in}{2.010799in}}%
\pgfpathlineto{\pgfqpoint{3.526690in}{1.652260in}}%
\pgfpathlineto{\pgfqpoint{3.965406in}{1.374156in}}%
\pgfpathlineto{\pgfqpoint{4.323864in}{1.146929in}}%
\pgfusepath{stroke}%
\end{pgfscope}%
\begin{pgfscope}%
\pgfpathrectangle{\pgfqpoint{0.625000in}{0.440000in}}{\pgfqpoint{3.875000in}{3.080000in}}%
\pgfusepath{clip}%
\pgfsetbuttcap%
\pgfsetroundjoin%
\definecolor{currentfill}{rgb}{0.000000,0.000000,0.000000}%
\pgfsetfillcolor{currentfill}%
\pgfsetlinewidth{1.003750pt}%
\definecolor{currentstroke}{rgb}{0.000000,0.000000,0.000000}%
\pgfsetstrokecolor{currentstroke}%
\pgfsetdash{}{0pt}%
\pgfsys@defobject{currentmarker}{\pgfqpoint{-0.041667in}{-0.041667in}}{\pgfqpoint{0.041667in}{0.041667in}}{%
\pgfpathmoveto{\pgfqpoint{0.000000in}{-0.041667in}}%
\pgfpathcurveto{\pgfqpoint{0.011050in}{-0.041667in}}{\pgfqpoint{0.021649in}{-0.037276in}}{\pgfqpoint{0.029463in}{-0.029463in}}%
\pgfpathcurveto{\pgfqpoint{0.037276in}{-0.021649in}}{\pgfqpoint{0.041667in}{-0.011050in}}{\pgfqpoint{0.041667in}{0.000000in}}%
\pgfpathcurveto{\pgfqpoint{0.041667in}{0.011050in}}{\pgfqpoint{0.037276in}{0.021649in}}{\pgfqpoint{0.029463in}{0.029463in}}%
\pgfpathcurveto{\pgfqpoint{0.021649in}{0.037276in}}{\pgfqpoint{0.011050in}{0.041667in}}{\pgfqpoint{0.000000in}{0.041667in}}%
\pgfpathcurveto{\pgfqpoint{-0.011050in}{0.041667in}}{\pgfqpoint{-0.021649in}{0.037276in}}{\pgfqpoint{-0.029463in}{0.029463in}}%
\pgfpathcurveto{\pgfqpoint{-0.037276in}{0.021649in}}{\pgfqpoint{-0.041667in}{0.011050in}}{\pgfqpoint{-0.041667in}{0.000000in}}%
\pgfpathcurveto{\pgfqpoint{-0.041667in}{-0.011050in}}{\pgfqpoint{-0.037276in}{-0.021649in}}{\pgfqpoint{-0.029463in}{-0.029463in}}%
\pgfpathcurveto{\pgfqpoint{-0.021649in}{-0.037276in}}{\pgfqpoint{-0.011050in}{-0.041667in}}{\pgfqpoint{0.000000in}{-0.041667in}}%
\pgfpathclose%
\pgfusepath{stroke,fill}%
}%
\begin{pgfscope}%
\pgfsys@transformshift{0.801136in}{3.062735in}%
\pgfsys@useobject{currentmarker}{}%
\end{pgfscope}%
\begin{pgfscope}%
\pgfsys@transformshift{2.163913in}{2.696609in}%
\pgfsys@useobject{currentmarker}{}%
\end{pgfscope}%
\begin{pgfscope}%
\pgfsys@transformshift{2.961087in}{2.475483in}%
\pgfsys@useobject{currentmarker}{}%
\end{pgfscope}%
\begin{pgfscope}%
\pgfsys@transformshift{3.526690in}{1.672474in}%
\pgfsys@useobject{currentmarker}{}%
\end{pgfscope}%
\begin{pgfscope}%
\pgfsys@transformshift{3.965406in}{1.592973in}%
\pgfsys@useobject{currentmarker}{}%
\end{pgfscope}%
\begin{pgfscope}%
\pgfsys@transformshift{4.323864in}{0.580000in}%
\pgfsys@useobject{currentmarker}{}%
\end{pgfscope}%
\end{pgfscope}%
\begin{pgfscope}%
\pgfsetrectcap%
\pgfsetmiterjoin%
\pgfsetlinewidth{0.803000pt}%
\definecolor{currentstroke}{rgb}{0.000000,0.000000,0.000000}%
\pgfsetstrokecolor{currentstroke}%
\pgfsetdash{}{0pt}%
\pgfpathmoveto{\pgfqpoint{0.625000in}{0.440000in}}%
\pgfpathlineto{\pgfqpoint{0.625000in}{3.520000in}}%
\pgfusepath{stroke}%
\end{pgfscope}%
\begin{pgfscope}%
\pgfsetrectcap%
\pgfsetmiterjoin%
\pgfsetlinewidth{0.000000pt}%
\definecolor{currentstroke}{rgb}{0.000000,0.000000,0.000000}%
\pgfsetstrokecolor{currentstroke}%
\pgfsetstrokeopacity{0.000000}%
\pgfsetdash{}{0pt}%
\pgfpathmoveto{\pgfqpoint{4.500000in}{0.440000in}}%
\pgfpathlineto{\pgfqpoint{4.500000in}{3.520000in}}%
\pgfusepath{}%
\end{pgfscope}%
\begin{pgfscope}%
\pgfsetrectcap%
\pgfsetmiterjoin%
\pgfsetlinewidth{0.803000pt}%
\definecolor{currentstroke}{rgb}{0.000000,0.000000,0.000000}%
\pgfsetstrokecolor{currentstroke}%
\pgfsetdash{}{0pt}%
\pgfpathmoveto{\pgfqpoint{0.625000in}{0.440000in}}%
\pgfpathlineto{\pgfqpoint{4.500000in}{0.440000in}}%
\pgfusepath{stroke}%
\end{pgfscope}%
\begin{pgfscope}%
\pgfsetrectcap%
\pgfsetmiterjoin%
\pgfsetlinewidth{0.000000pt}%
\definecolor{currentstroke}{rgb}{0.000000,0.000000,0.000000}%
\pgfsetstrokecolor{currentstroke}%
\pgfsetstrokeopacity{0.000000}%
\pgfsetdash{}{0pt}%
\pgfpathmoveto{\pgfqpoint{0.625000in}{3.520000in}}%
\pgfpathlineto{\pgfqpoint{4.500000in}{3.520000in}}%
\pgfusepath{}%
\end{pgfscope}%
\begin{pgfscope}%
\pgfsetbuttcap%
\pgfsetmiterjoin%
\definecolor{currentfill}{rgb}{1.000000,1.000000,1.000000}%
\pgfsetfillcolor{currentfill}%
\pgfsetfillopacity{0.800000}%
\pgfsetlinewidth{1.003750pt}%
\definecolor{currentstroke}{rgb}{0.800000,0.800000,0.800000}%
\pgfsetstrokecolor{currentstroke}%
\pgfsetstrokeopacity{0.800000}%
\pgfsetdash{}{0pt}%
\pgfpathmoveto{\pgfqpoint{2.423855in}{3.199199in}}%
\pgfpathlineto{\pgfqpoint{4.402778in}{3.199199in}}%
\pgfpathquadraticcurveto{\pgfqpoint{4.430556in}{3.199199in}}{\pgfqpoint{4.430556in}{3.226977in}}%
\pgfpathlineto{\pgfqpoint{4.430556in}{3.422778in}}%
\pgfpathquadraticcurveto{\pgfqpoint{4.430556in}{3.450556in}}{\pgfqpoint{4.402778in}{3.450556in}}%
\pgfpathlineto{\pgfqpoint{2.423855in}{3.450556in}}%
\pgfpathquadraticcurveto{\pgfqpoint{2.396077in}{3.450556in}}{\pgfqpoint{2.396077in}{3.422778in}}%
\pgfpathlineto{\pgfqpoint{2.396077in}{3.226977in}}%
\pgfpathquadraticcurveto{\pgfqpoint{2.396077in}{3.199199in}}{\pgfqpoint{2.423855in}{3.199199in}}%
\pgfpathclose%
\pgfusepath{stroke,fill}%
\end{pgfscope}%
\begin{pgfscope}%
\pgfsetbuttcap%
\pgfsetroundjoin%
\pgfsetlinewidth{1.505625pt}%
\definecolor{currentstroke}{rgb}{0.000000,0.000000,0.000000}%
\pgfsetstrokecolor{currentstroke}%
\pgfsetdash{{5.550000pt}{2.400000pt}}{0.000000pt}%
\pgfpathmoveto{\pgfqpoint{2.451633in}{3.338088in}}%
\pgfpathlineto{\pgfqpoint{2.729411in}{3.338088in}}%
\pgfusepath{stroke}%
\end{pgfscope}%
\begin{pgfscope}%
\definecolor{textcolor}{rgb}{0.000000,0.000000,0.000000}%
\pgfsetstrokecolor{textcolor}%
\pgfsetfillcolor{textcolor}%
\pgftext[x=2.840522in,y=3.289477in,left,base]{\color{textcolor}\sffamily\fontsize{10.000000}{12.000000}\selectfont \(\displaystyle \alpha = 1.5113 - 0.2065\mathrm{ln}(k)\)}%
\end{pgfscope}%
\end{pgfpicture}%
\makeatother%
\endgroup%

    \end{subfigure}
\caption{Plots of the dependence of $C$ and $\alpha$ on $k$ in \cref{eq:qmcerrorform} for $Q(u) = u((1,1))$. Observe the $x$-axis is on a $\log_{10}$ scale, but $\loge$ is the natural logarithm.  \label{fig:toprightCalpha}}
\end{figure}

\begin{figure}[h]
    \centering
    \begin{subfigure}{\textwidth}
            \centering
%% Creator: Matplotlib, PGF backend
%%
%% To include the figure in your LaTeX document, write
%%   \input{<filename>.pgf}
%%
%% Make sure the required packages are loaded in your preamble
%%   \usepackage{pgf}
%%
%% Figures using additional raster images can only be included by \input if
%% they are in the same directory as the main LaTeX file. For loading figures
%% from other directories you can use the `import` package
%%   \usepackage{import}
%% and then include the figures with
%%   \import{<path to file>}{<filename>.pgf}
%%
%% Matplotlib used the following preamble
%%   \usepackage{fontspec}
%%   \setmainfont{DejaVuSerif.ttf}[Path=/home/owen/progs/firedrake-complex/firedrake/lib/python3.5/site-packages/matplotlib/mpl-data/fonts/ttf/]
%%   \setsansfont{DejaVuSans.ttf}[Path=/home/owen/progs/firedrake-complex/firedrake/lib/python3.5/site-packages/matplotlib/mpl-data/fonts/ttf/]
%%   \setmonofont{DejaVuSansMono.ttf}[Path=/home/owen/progs/firedrake-complex/firedrake/lib/python3.5/site-packages/matplotlib/mpl-data/fonts/ttf/]
%%
\begingroup%
\makeatletter%
\begin{pgfpicture}%
\pgfpathrectangle{\pgfpointorigin}{\pgfqpoint{5.000000in}{4.000000in}}%
\pgfusepath{use as bounding box, clip}%
\begin{pgfscope}%
\pgfsetbuttcap%
\pgfsetmiterjoin%
\definecolor{currentfill}{rgb}{1.000000,1.000000,1.000000}%
\pgfsetfillcolor{currentfill}%
\pgfsetlinewidth{0.000000pt}%
\definecolor{currentstroke}{rgb}{1.000000,1.000000,1.000000}%
\pgfsetstrokecolor{currentstroke}%
\pgfsetdash{}{0pt}%
\pgfpathmoveto{\pgfqpoint{0.000000in}{0.000000in}}%
\pgfpathlineto{\pgfqpoint{5.000000in}{0.000000in}}%
\pgfpathlineto{\pgfqpoint{5.000000in}{4.000000in}}%
\pgfpathlineto{\pgfqpoint{0.000000in}{4.000000in}}%
\pgfpathclose%
\pgfusepath{fill}%
\end{pgfscope}%
\begin{pgfscope}%
\pgfsetbuttcap%
\pgfsetmiterjoin%
\definecolor{currentfill}{rgb}{1.000000,1.000000,1.000000}%
\pgfsetfillcolor{currentfill}%
\pgfsetlinewidth{0.000000pt}%
\definecolor{currentstroke}{rgb}{0.000000,0.000000,0.000000}%
\pgfsetstrokecolor{currentstroke}%
\pgfsetstrokeopacity{0.000000}%
\pgfsetdash{}{0pt}%
\pgfpathmoveto{\pgfqpoint{0.625000in}{0.440000in}}%
\pgfpathlineto{\pgfqpoint{4.500000in}{0.440000in}}%
\pgfpathlineto{\pgfqpoint{4.500000in}{3.520000in}}%
\pgfpathlineto{\pgfqpoint{0.625000in}{3.520000in}}%
\pgfpathclose%
\pgfusepath{fill}%
\end{pgfscope}%
\begin{pgfscope}%
\pgfsetbuttcap%
\pgfsetroundjoin%
\definecolor{currentfill}{rgb}{0.000000,0.000000,0.000000}%
\pgfsetfillcolor{currentfill}%
\pgfsetlinewidth{0.803000pt}%
\definecolor{currentstroke}{rgb}{0.000000,0.000000,0.000000}%
\pgfsetstrokecolor{currentstroke}%
\pgfsetdash{}{0pt}%
\pgfsys@defobject{currentmarker}{\pgfqpoint{0.000000in}{-0.048611in}}{\pgfqpoint{0.000000in}{0.000000in}}{%
\pgfpathmoveto{\pgfqpoint{0.000000in}{0.000000in}}%
\pgfpathlineto{\pgfqpoint{0.000000in}{-0.048611in}}%
\pgfusepath{stroke,fill}%
}%
\begin{pgfscope}%
\pgfsys@transformshift{0.801136in}{0.440000in}%
\pgfsys@useobject{currentmarker}{}%
\end{pgfscope}%
\end{pgfscope}%
\begin{pgfscope}%
\definecolor{textcolor}{rgb}{0.000000,0.000000,0.000000}%
\pgfsetstrokecolor{textcolor}%
\pgfsetfillcolor{textcolor}%
\pgftext[x=0.801136in,y=0.342778in,,top]{\color{textcolor}\sffamily\fontsize{10.000000}{12.000000}\selectfont \(\displaystyle {10^{1}}\)}%
\end{pgfscope}%
\begin{pgfscope}%
\pgfsetbuttcap%
\pgfsetroundjoin%
\definecolor{currentfill}{rgb}{0.000000,0.000000,0.000000}%
\pgfsetfillcolor{currentfill}%
\pgfsetlinewidth{0.602250pt}%
\definecolor{currentstroke}{rgb}{0.000000,0.000000,0.000000}%
\pgfsetstrokecolor{currentstroke}%
\pgfsetdash{}{0pt}%
\pgfsys@defobject{currentmarker}{\pgfqpoint{0.000000in}{-0.027778in}}{\pgfqpoint{0.000000in}{0.000000in}}{%
\pgfpathmoveto{\pgfqpoint{0.000000in}{0.000000in}}%
\pgfpathlineto{\pgfqpoint{0.000000in}{-0.027778in}}%
\pgfusepath{stroke,fill}%
}%
\begin{pgfscope}%
\pgfsys@transformshift{2.163913in}{0.440000in}%
\pgfsys@useobject{currentmarker}{}%
\end{pgfscope}%
\end{pgfscope}%
\begin{pgfscope}%
\definecolor{textcolor}{rgb}{0.000000,0.000000,0.000000}%
\pgfsetstrokecolor{textcolor}%
\pgfsetfillcolor{textcolor}%
\pgftext[x=2.163913in,y=0.365000in,,top]{\color{textcolor}\sffamily\fontsize{10.000000}{12.000000}\selectfont \(\displaystyle {2\times10^{1}}\)}%
\end{pgfscope}%
\begin{pgfscope}%
\pgfsetbuttcap%
\pgfsetroundjoin%
\definecolor{currentfill}{rgb}{0.000000,0.000000,0.000000}%
\pgfsetfillcolor{currentfill}%
\pgfsetlinewidth{0.602250pt}%
\definecolor{currentstroke}{rgb}{0.000000,0.000000,0.000000}%
\pgfsetstrokecolor{currentstroke}%
\pgfsetdash{}{0pt}%
\pgfsys@defobject{currentmarker}{\pgfqpoint{0.000000in}{-0.027778in}}{\pgfqpoint{0.000000in}{0.000000in}}{%
\pgfpathmoveto{\pgfqpoint{0.000000in}{0.000000in}}%
\pgfpathlineto{\pgfqpoint{0.000000in}{-0.027778in}}%
\pgfusepath{stroke,fill}%
}%
\begin{pgfscope}%
\pgfsys@transformshift{2.961087in}{0.440000in}%
\pgfsys@useobject{currentmarker}{}%
\end{pgfscope}%
\end{pgfscope}%
\begin{pgfscope}%
\definecolor{textcolor}{rgb}{0.000000,0.000000,0.000000}%
\pgfsetstrokecolor{textcolor}%
\pgfsetfillcolor{textcolor}%
\pgftext[x=2.961087in,y=0.365000in,,top]{\color{textcolor}\sffamily\fontsize{10.000000}{12.000000}\selectfont \(\displaystyle {3\times10^{1}}\)}%
\end{pgfscope}%
\begin{pgfscope}%
\pgfsetbuttcap%
\pgfsetroundjoin%
\definecolor{currentfill}{rgb}{0.000000,0.000000,0.000000}%
\pgfsetfillcolor{currentfill}%
\pgfsetlinewidth{0.602250pt}%
\definecolor{currentstroke}{rgb}{0.000000,0.000000,0.000000}%
\pgfsetstrokecolor{currentstroke}%
\pgfsetdash{}{0pt}%
\pgfsys@defobject{currentmarker}{\pgfqpoint{0.000000in}{-0.027778in}}{\pgfqpoint{0.000000in}{0.000000in}}{%
\pgfpathmoveto{\pgfqpoint{0.000000in}{0.000000in}}%
\pgfpathlineto{\pgfqpoint{0.000000in}{-0.027778in}}%
\pgfusepath{stroke,fill}%
}%
\begin{pgfscope}%
\pgfsys@transformshift{3.526690in}{0.440000in}%
\pgfsys@useobject{currentmarker}{}%
\end{pgfscope}%
\end{pgfscope}%
\begin{pgfscope}%
\definecolor{textcolor}{rgb}{0.000000,0.000000,0.000000}%
\pgfsetstrokecolor{textcolor}%
\pgfsetfillcolor{textcolor}%
\pgftext[x=3.526690in,y=0.365000in,,top]{\color{textcolor}\sffamily\fontsize{10.000000}{12.000000}\selectfont \(\displaystyle {4\times10^{1}}\)}%
\end{pgfscope}%
\begin{pgfscope}%
\pgfsetbuttcap%
\pgfsetroundjoin%
\definecolor{currentfill}{rgb}{0.000000,0.000000,0.000000}%
\pgfsetfillcolor{currentfill}%
\pgfsetlinewidth{0.602250pt}%
\definecolor{currentstroke}{rgb}{0.000000,0.000000,0.000000}%
\pgfsetstrokecolor{currentstroke}%
\pgfsetdash{}{0pt}%
\pgfsys@defobject{currentmarker}{\pgfqpoint{0.000000in}{-0.027778in}}{\pgfqpoint{0.000000in}{0.000000in}}{%
\pgfpathmoveto{\pgfqpoint{0.000000in}{0.000000in}}%
\pgfpathlineto{\pgfqpoint{0.000000in}{-0.027778in}}%
\pgfusepath{stroke,fill}%
}%
\begin{pgfscope}%
\pgfsys@transformshift{3.965406in}{0.440000in}%
\pgfsys@useobject{currentmarker}{}%
\end{pgfscope}%
\end{pgfscope}%
\begin{pgfscope}%
\pgfsetbuttcap%
\pgfsetroundjoin%
\definecolor{currentfill}{rgb}{0.000000,0.000000,0.000000}%
\pgfsetfillcolor{currentfill}%
\pgfsetlinewidth{0.602250pt}%
\definecolor{currentstroke}{rgb}{0.000000,0.000000,0.000000}%
\pgfsetstrokecolor{currentstroke}%
\pgfsetdash{}{0pt}%
\pgfsys@defobject{currentmarker}{\pgfqpoint{0.000000in}{-0.027778in}}{\pgfqpoint{0.000000in}{0.000000in}}{%
\pgfpathmoveto{\pgfqpoint{0.000000in}{0.000000in}}%
\pgfpathlineto{\pgfqpoint{0.000000in}{-0.027778in}}%
\pgfusepath{stroke,fill}%
}%
\begin{pgfscope}%
\pgfsys@transformshift{4.323864in}{0.440000in}%
\pgfsys@useobject{currentmarker}{}%
\end{pgfscope}%
\end{pgfscope}%
\begin{pgfscope}%
\definecolor{textcolor}{rgb}{0.000000,0.000000,0.000000}%
\pgfsetstrokecolor{textcolor}%
\pgfsetfillcolor{textcolor}%
\pgftext[x=4.323864in,y=0.365000in,,top]{\color{textcolor}\sffamily\fontsize{10.000000}{12.000000}\selectfont \(\displaystyle {6\times10^{1}}\)}%
\end{pgfscope}%
\begin{pgfscope}%
\definecolor{textcolor}{rgb}{0.000000,0.000000,0.000000}%
\pgfsetstrokecolor{textcolor}%
\pgfsetfillcolor{textcolor}%
\pgftext[x=2.562500in,y=0.152809in,,top]{\color{textcolor}\sffamily\fontsize{10.000000}{12.000000}\selectfont k}%
\end{pgfscope}%
\begin{pgfscope}%
\pgfsetbuttcap%
\pgfsetroundjoin%
\definecolor{currentfill}{rgb}{0.000000,0.000000,0.000000}%
\pgfsetfillcolor{currentfill}%
\pgfsetlinewidth{0.803000pt}%
\definecolor{currentstroke}{rgb}{0.000000,0.000000,0.000000}%
\pgfsetstrokecolor{currentstroke}%
\pgfsetdash{}{0pt}%
\pgfsys@defobject{currentmarker}{\pgfqpoint{-0.048611in}{0.000000in}}{\pgfqpoint{0.000000in}{0.000000in}}{%
\pgfpathmoveto{\pgfqpoint{0.000000in}{0.000000in}}%
\pgfpathlineto{\pgfqpoint{-0.048611in}{0.000000in}}%
\pgfusepath{stroke,fill}%
}%
\begin{pgfscope}%
\pgfsys@transformshift{0.625000in}{0.897834in}%
\pgfsys@useobject{currentmarker}{}%
\end{pgfscope}%
\end{pgfscope}%
\begin{pgfscope}%
\definecolor{textcolor}{rgb}{0.000000,0.000000,0.000000}%
\pgfsetstrokecolor{textcolor}%
\pgfsetfillcolor{textcolor}%
\pgftext[x=0.326581in,y=0.845072in,left,base]{\color{textcolor}\sffamily\fontsize{10.000000}{12.000000}\selectfont \(\displaystyle {10^{0}}\)}%
\end{pgfscope}%
\begin{pgfscope}%
\pgfsetbuttcap%
\pgfsetroundjoin%
\definecolor{currentfill}{rgb}{0.000000,0.000000,0.000000}%
\pgfsetfillcolor{currentfill}%
\pgfsetlinewidth{0.602250pt}%
\definecolor{currentstroke}{rgb}{0.000000,0.000000,0.000000}%
\pgfsetstrokecolor{currentstroke}%
\pgfsetdash{}{0pt}%
\pgfsys@defobject{currentmarker}{\pgfqpoint{-0.027778in}{0.000000in}}{\pgfqpoint{0.000000in}{0.000000in}}{%
\pgfpathmoveto{\pgfqpoint{0.000000in}{0.000000in}}%
\pgfpathlineto{\pgfqpoint{-0.027778in}{0.000000in}}%
\pgfusepath{stroke,fill}%
}%
\begin{pgfscope}%
\pgfsys@transformshift{0.625000in}{0.550354in}%
\pgfsys@useobject{currentmarker}{}%
\end{pgfscope}%
\end{pgfscope}%
\begin{pgfscope}%
\pgfsetbuttcap%
\pgfsetroundjoin%
\definecolor{currentfill}{rgb}{0.000000,0.000000,0.000000}%
\pgfsetfillcolor{currentfill}%
\pgfsetlinewidth{0.602250pt}%
\definecolor{currentstroke}{rgb}{0.000000,0.000000,0.000000}%
\pgfsetstrokecolor{currentstroke}%
\pgfsetdash{}{0pt}%
\pgfsys@defobject{currentmarker}{\pgfqpoint{-0.027778in}{0.000000in}}{\pgfqpoint{0.000000in}{0.000000in}}{%
\pgfpathmoveto{\pgfqpoint{0.000000in}{0.000000in}}%
\pgfpathlineto{\pgfqpoint{-0.027778in}{0.000000in}}%
\pgfusepath{stroke,fill}%
}%
\begin{pgfscope}%
\pgfsys@transformshift{0.625000in}{0.733766in}%
\pgfsys@useobject{currentmarker}{}%
\end{pgfscope}%
\end{pgfscope}%
\begin{pgfscope}%
\pgfsetbuttcap%
\pgfsetroundjoin%
\definecolor{currentfill}{rgb}{0.000000,0.000000,0.000000}%
\pgfsetfillcolor{currentfill}%
\pgfsetlinewidth{0.602250pt}%
\definecolor{currentstroke}{rgb}{0.000000,0.000000,0.000000}%
\pgfsetstrokecolor{currentstroke}%
\pgfsetdash{}{0pt}%
\pgfsys@defobject{currentmarker}{\pgfqpoint{-0.027778in}{0.000000in}}{\pgfqpoint{0.000000in}{0.000000in}}{%
\pgfpathmoveto{\pgfqpoint{0.000000in}{0.000000in}}%
\pgfpathlineto{\pgfqpoint{-0.027778in}{0.000000in}}%
\pgfusepath{stroke,fill}%
}%
\begin{pgfscope}%
\pgfsys@transformshift{0.625000in}{1.977204in}%
\pgfsys@useobject{currentmarker}{}%
\end{pgfscope}%
\end{pgfscope}%
\begin{pgfscope}%
\definecolor{textcolor}{rgb}{0.000000,0.000000,0.000000}%
\pgfsetstrokecolor{textcolor}%
\pgfsetfillcolor{textcolor}%
\pgftext[x=0.109607in,y=1.924443in,left,base]{\color{textcolor}\sffamily\fontsize{10.000000}{12.000000}\selectfont \(\displaystyle {2\times10^{0}}\)}%
\end{pgfscope}%
\begin{pgfscope}%
\pgfsetbuttcap%
\pgfsetroundjoin%
\definecolor{currentfill}{rgb}{0.000000,0.000000,0.000000}%
\pgfsetfillcolor{currentfill}%
\pgfsetlinewidth{0.602250pt}%
\definecolor{currentstroke}{rgb}{0.000000,0.000000,0.000000}%
\pgfsetstrokecolor{currentstroke}%
\pgfsetdash{}{0pt}%
\pgfsys@defobject{currentmarker}{\pgfqpoint{-0.027778in}{0.000000in}}{\pgfqpoint{0.000000in}{0.000000in}}{%
\pgfpathmoveto{\pgfqpoint{0.000000in}{0.000000in}}%
\pgfpathlineto{\pgfqpoint{-0.027778in}{0.000000in}}%
\pgfusepath{stroke,fill}%
}%
\begin{pgfscope}%
\pgfsys@transformshift{0.625000in}{2.608596in}%
\pgfsys@useobject{currentmarker}{}%
\end{pgfscope}%
\end{pgfscope}%
\begin{pgfscope}%
\definecolor{textcolor}{rgb}{0.000000,0.000000,0.000000}%
\pgfsetstrokecolor{textcolor}%
\pgfsetfillcolor{textcolor}%
\pgftext[x=0.109607in,y=2.555834in,left,base]{\color{textcolor}\sffamily\fontsize{10.000000}{12.000000}\selectfont \(\displaystyle {3\times10^{0}}\)}%
\end{pgfscope}%
\begin{pgfscope}%
\pgfsetbuttcap%
\pgfsetroundjoin%
\definecolor{currentfill}{rgb}{0.000000,0.000000,0.000000}%
\pgfsetfillcolor{currentfill}%
\pgfsetlinewidth{0.602250pt}%
\definecolor{currentstroke}{rgb}{0.000000,0.000000,0.000000}%
\pgfsetstrokecolor{currentstroke}%
\pgfsetdash{}{0pt}%
\pgfsys@defobject{currentmarker}{\pgfqpoint{-0.027778in}{0.000000in}}{\pgfqpoint{0.000000in}{0.000000in}}{%
\pgfpathmoveto{\pgfqpoint{0.000000in}{0.000000in}}%
\pgfpathlineto{\pgfqpoint{-0.027778in}{0.000000in}}%
\pgfusepath{stroke,fill}%
}%
\begin{pgfscope}%
\pgfsys@transformshift{0.625000in}{3.056575in}%
\pgfsys@useobject{currentmarker}{}%
\end{pgfscope}%
\end{pgfscope}%
\begin{pgfscope}%
\definecolor{textcolor}{rgb}{0.000000,0.000000,0.000000}%
\pgfsetstrokecolor{textcolor}%
\pgfsetfillcolor{textcolor}%
\pgftext[x=0.109607in,y=3.003814in,left,base]{\color{textcolor}\sffamily\fontsize{10.000000}{12.000000}\selectfont \(\displaystyle {4\times10^{0}}\)}%
\end{pgfscope}%
\begin{pgfscope}%
\pgfsetbuttcap%
\pgfsetroundjoin%
\definecolor{currentfill}{rgb}{0.000000,0.000000,0.000000}%
\pgfsetfillcolor{currentfill}%
\pgfsetlinewidth{0.602250pt}%
\definecolor{currentstroke}{rgb}{0.000000,0.000000,0.000000}%
\pgfsetstrokecolor{currentstroke}%
\pgfsetdash{}{0pt}%
\pgfsys@defobject{currentmarker}{\pgfqpoint{-0.027778in}{0.000000in}}{\pgfqpoint{0.000000in}{0.000000in}}{%
\pgfpathmoveto{\pgfqpoint{0.000000in}{0.000000in}}%
\pgfpathlineto{\pgfqpoint{-0.027778in}{0.000000in}}%
\pgfusepath{stroke,fill}%
}%
\begin{pgfscope}%
\pgfsys@transformshift{0.625000in}{3.404055in}%
\pgfsys@useobject{currentmarker}{}%
\end{pgfscope}%
\end{pgfscope}%
\begin{pgfscope}%
\definecolor{textcolor}{rgb}{0.000000,0.000000,0.000000}%
\pgfsetstrokecolor{textcolor}%
\pgfsetfillcolor{textcolor}%
\pgftext[x=0.054051in,y=1.980000in,,bottom,rotate=90.000000]{\color{textcolor}\sffamily\fontsize{10.000000}{12.000000}\selectfont C}%
\end{pgfscope}%
\begin{pgfscope}%
\pgfpathrectangle{\pgfqpoint{0.625000in}{0.440000in}}{\pgfqpoint{3.875000in}{3.080000in}}%
\pgfusepath{clip}%
\pgfsetbuttcap%
\pgfsetroundjoin%
\definecolor{currentfill}{rgb}{0.000000,0.000000,0.000000}%
\pgfsetfillcolor{currentfill}%
\pgfsetlinewidth{1.003750pt}%
\definecolor{currentstroke}{rgb}{0.000000,0.000000,0.000000}%
\pgfsetstrokecolor{currentstroke}%
\pgfsetdash{}{0pt}%
\pgfsys@defobject{currentmarker}{\pgfqpoint{-0.041667in}{-0.041667in}}{\pgfqpoint{0.041667in}{0.041667in}}{%
\pgfpathmoveto{\pgfqpoint{0.000000in}{-0.041667in}}%
\pgfpathcurveto{\pgfqpoint{0.011050in}{-0.041667in}}{\pgfqpoint{0.021649in}{-0.037276in}}{\pgfqpoint{0.029463in}{-0.029463in}}%
\pgfpathcurveto{\pgfqpoint{0.037276in}{-0.021649in}}{\pgfqpoint{0.041667in}{-0.011050in}}{\pgfqpoint{0.041667in}{0.000000in}}%
\pgfpathcurveto{\pgfqpoint{0.041667in}{0.011050in}}{\pgfqpoint{0.037276in}{0.021649in}}{\pgfqpoint{0.029463in}{0.029463in}}%
\pgfpathcurveto{\pgfqpoint{0.021649in}{0.037276in}}{\pgfqpoint{0.011050in}{0.041667in}}{\pgfqpoint{0.000000in}{0.041667in}}%
\pgfpathcurveto{\pgfqpoint{-0.011050in}{0.041667in}}{\pgfqpoint{-0.021649in}{0.037276in}}{\pgfqpoint{-0.029463in}{0.029463in}}%
\pgfpathcurveto{\pgfqpoint{-0.037276in}{0.021649in}}{\pgfqpoint{-0.041667in}{0.011050in}}{\pgfqpoint{-0.041667in}{0.000000in}}%
\pgfpathcurveto{\pgfqpoint{-0.041667in}{-0.011050in}}{\pgfqpoint{-0.037276in}{-0.021649in}}{\pgfqpoint{-0.029463in}{-0.029463in}}%
\pgfpathcurveto{\pgfqpoint{-0.021649in}{-0.037276in}}{\pgfqpoint{-0.011050in}{-0.041667in}}{\pgfqpoint{0.000000in}{-0.041667in}}%
\pgfpathclose%
\pgfusepath{stroke,fill}%
}%
\begin{pgfscope}%
\pgfsys@transformshift{0.801136in}{0.580000in}%
\pgfsys@useobject{currentmarker}{}%
\end{pgfscope}%
\begin{pgfscope}%
\pgfsys@transformshift{2.163913in}{1.211949in}%
\pgfsys@useobject{currentmarker}{}%
\end{pgfscope}%
\begin{pgfscope}%
\pgfsys@transformshift{2.961087in}{2.605341in}%
\pgfsys@useobject{currentmarker}{}%
\end{pgfscope}%
\begin{pgfscope}%
\pgfsys@transformshift{3.526690in}{2.532871in}%
\pgfsys@useobject{currentmarker}{}%
\end{pgfscope}%
\begin{pgfscope}%
\pgfsys@transformshift{3.965406in}{3.380000in}%
\pgfsys@useobject{currentmarker}{}%
\end{pgfscope}%
\begin{pgfscope}%
\pgfsys@transformshift{4.323864in}{3.162925in}%
\pgfsys@useobject{currentmarker}{}%
\end{pgfscope}%
\end{pgfscope}%
\begin{pgfscope}%
\pgfsetrectcap%
\pgfsetmiterjoin%
\pgfsetlinewidth{0.803000pt}%
\definecolor{currentstroke}{rgb}{0.000000,0.000000,0.000000}%
\pgfsetstrokecolor{currentstroke}%
\pgfsetdash{}{0pt}%
\pgfpathmoveto{\pgfqpoint{0.625000in}{0.440000in}}%
\pgfpathlineto{\pgfqpoint{0.625000in}{3.520000in}}%
\pgfusepath{stroke}%
\end{pgfscope}%
\begin{pgfscope}%
\pgfsetrectcap%
\pgfsetmiterjoin%
\pgfsetlinewidth{0.803000pt}%
\definecolor{currentstroke}{rgb}{0.000000,0.000000,0.000000}%
\pgfsetstrokecolor{currentstroke}%
\pgfsetdash{}{0pt}%
\pgfpathmoveto{\pgfqpoint{4.500000in}{0.440000in}}%
\pgfpathlineto{\pgfqpoint{4.500000in}{3.520000in}}%
\pgfusepath{stroke}%
\end{pgfscope}%
\begin{pgfscope}%
\pgfsetrectcap%
\pgfsetmiterjoin%
\pgfsetlinewidth{0.803000pt}%
\definecolor{currentstroke}{rgb}{0.000000,0.000000,0.000000}%
\pgfsetstrokecolor{currentstroke}%
\pgfsetdash{}{0pt}%
\pgfpathmoveto{\pgfqpoint{0.625000in}{0.440000in}}%
\pgfpathlineto{\pgfqpoint{4.500000in}{0.440000in}}%
\pgfusepath{stroke}%
\end{pgfscope}%
\begin{pgfscope}%
\pgfsetrectcap%
\pgfsetmiterjoin%
\pgfsetlinewidth{0.803000pt}%
\definecolor{currentstroke}{rgb}{0.000000,0.000000,0.000000}%
\pgfsetstrokecolor{currentstroke}%
\pgfsetdash{}{0pt}%
\pgfpathmoveto{\pgfqpoint{0.625000in}{3.520000in}}%
\pgfpathlineto{\pgfqpoint{4.500000in}{3.520000in}}%
\pgfusepath{stroke}%
\end{pgfscope}%
\end{pgfpicture}%
\makeatother%
\endgroup%

  \end{subfigure}
    \begin{subfigure}{\textwidth}
            \centering 
%% Creator: Matplotlib, PGF backend
%%
%% To include the figure in your LaTeX document, write
%%   \input{<filename>.pgf}
%%
%% Make sure the required packages are loaded in your preamble
%%   \usepackage{pgf}
%%
%% Figures using additional raster images can only be included by \input if
%% they are in the same directory as the main LaTeX file. For loading figures
%% from other directories you can use the `import` package
%%   \usepackage{import}
%% and then include the figures with
%%   \import{<path to file>}{<filename>.pgf}
%%
%% Matplotlib used the following preamble
%%   \usepackage{fontspec}
%%   \setmainfont{DejaVuSerif.ttf}[Path=/home/owen/progs/firedrake-complex/firedrake/lib/python3.5/site-packages/matplotlib/mpl-data/fonts/ttf/]
%%   \setsansfont{DejaVuSans.ttf}[Path=/home/owen/progs/firedrake-complex/firedrake/lib/python3.5/site-packages/matplotlib/mpl-data/fonts/ttf/]
%%   \setmonofont{DejaVuSansMono.ttf}[Path=/home/owen/progs/firedrake-complex/firedrake/lib/python3.5/site-packages/matplotlib/mpl-data/fonts/ttf/]
%%
\begingroup%
\makeatletter%
\begin{pgfpicture}%
\pgfpathrectangle{\pgfpointorigin}{\pgfqpoint{5.000000in}{4.000000in}}%
\pgfusepath{use as bounding box, clip}%
\begin{pgfscope}%
\pgfsetbuttcap%
\pgfsetmiterjoin%
\definecolor{currentfill}{rgb}{1.000000,1.000000,1.000000}%
\pgfsetfillcolor{currentfill}%
\pgfsetlinewidth{0.000000pt}%
\definecolor{currentstroke}{rgb}{1.000000,1.000000,1.000000}%
\pgfsetstrokecolor{currentstroke}%
\pgfsetdash{}{0pt}%
\pgfpathmoveto{\pgfqpoint{0.000000in}{0.000000in}}%
\pgfpathlineto{\pgfqpoint{5.000000in}{0.000000in}}%
\pgfpathlineto{\pgfqpoint{5.000000in}{4.000000in}}%
\pgfpathlineto{\pgfqpoint{0.000000in}{4.000000in}}%
\pgfpathclose%
\pgfusepath{fill}%
\end{pgfscope}%
\begin{pgfscope}%
\pgfsetbuttcap%
\pgfsetmiterjoin%
\definecolor{currentfill}{rgb}{1.000000,1.000000,1.000000}%
\pgfsetfillcolor{currentfill}%
\pgfsetlinewidth{0.000000pt}%
\definecolor{currentstroke}{rgb}{0.000000,0.000000,0.000000}%
\pgfsetstrokecolor{currentstroke}%
\pgfsetstrokeopacity{0.000000}%
\pgfsetdash{}{0pt}%
\pgfpathmoveto{\pgfqpoint{0.625000in}{0.440000in}}%
\pgfpathlineto{\pgfqpoint{4.500000in}{0.440000in}}%
\pgfpathlineto{\pgfqpoint{4.500000in}{3.520000in}}%
\pgfpathlineto{\pgfqpoint{0.625000in}{3.520000in}}%
\pgfpathclose%
\pgfusepath{fill}%
\end{pgfscope}%
\begin{pgfscope}%
\pgfsetbuttcap%
\pgfsetroundjoin%
\definecolor{currentfill}{rgb}{0.000000,0.000000,0.000000}%
\pgfsetfillcolor{currentfill}%
\pgfsetlinewidth{0.803000pt}%
\definecolor{currentstroke}{rgb}{0.000000,0.000000,0.000000}%
\pgfsetstrokecolor{currentstroke}%
\pgfsetdash{}{0pt}%
\pgfsys@defobject{currentmarker}{\pgfqpoint{0.000000in}{-0.048611in}}{\pgfqpoint{0.000000in}{0.000000in}}{%
\pgfpathmoveto{\pgfqpoint{0.000000in}{0.000000in}}%
\pgfpathlineto{\pgfqpoint{0.000000in}{-0.048611in}}%
\pgfusepath{stroke,fill}%
}%
\begin{pgfscope}%
\pgfsys@transformshift{2.172304in}{0.440000in}%
\pgfsys@useobject{currentmarker}{}%
\end{pgfscope}%
\end{pgfscope}%
\begin{pgfscope}%
\definecolor{textcolor}{rgb}{0.000000,0.000000,0.000000}%
\pgfsetstrokecolor{textcolor}%
\pgfsetfillcolor{textcolor}%
\pgftext[x=2.172304in,y=0.342778in,,top]{\color{textcolor}\sffamily\fontsize{10.000000}{12.000000}\selectfont \(\displaystyle {2.718281828459045^{3}}\)}%
\end{pgfscope}%
\begin{pgfscope}%
\pgfsetbuttcap%
\pgfsetroundjoin%
\definecolor{currentfill}{rgb}{0.000000,0.000000,0.000000}%
\pgfsetfillcolor{currentfill}%
\pgfsetlinewidth{0.803000pt}%
\definecolor{currentstroke}{rgb}{0.000000,0.000000,0.000000}%
\pgfsetstrokecolor{currentstroke}%
\pgfsetdash{}{0pt}%
\pgfsys@defobject{currentmarker}{\pgfqpoint{0.000000in}{-0.048611in}}{\pgfqpoint{0.000000in}{0.000000in}}{%
\pgfpathmoveto{\pgfqpoint{0.000000in}{0.000000in}}%
\pgfpathlineto{\pgfqpoint{0.000000in}{-0.048611in}}%
\pgfusepath{stroke,fill}%
}%
\begin{pgfscope}%
\pgfsys@transformshift{4.138375in}{0.440000in}%
\pgfsys@useobject{currentmarker}{}%
\end{pgfscope}%
\end{pgfscope}%
\begin{pgfscope}%
\definecolor{textcolor}{rgb}{0.000000,0.000000,0.000000}%
\pgfsetstrokecolor{textcolor}%
\pgfsetfillcolor{textcolor}%
\pgftext[x=4.138375in,y=0.342778in,,top]{\color{textcolor}\sffamily\fontsize{10.000000}{12.000000}\selectfont \(\displaystyle {2.718281828459045^{4}}\)}%
\end{pgfscope}%
\begin{pgfscope}%
\definecolor{textcolor}{rgb}{0.000000,0.000000,0.000000}%
\pgfsetstrokecolor{textcolor}%
\pgfsetfillcolor{textcolor}%
\pgftext[x=2.562500in,y=0.152809in,,top]{\color{textcolor}\sffamily\fontsize{10.000000}{12.000000}\selectfont \(\displaystyle k\)}%
\end{pgfscope}%
\begin{pgfscope}%
\pgfsetbuttcap%
\pgfsetroundjoin%
\definecolor{currentfill}{rgb}{0.000000,0.000000,0.000000}%
\pgfsetfillcolor{currentfill}%
\pgfsetlinewidth{0.803000pt}%
\definecolor{currentstroke}{rgb}{0.000000,0.000000,0.000000}%
\pgfsetstrokecolor{currentstroke}%
\pgfsetdash{}{0pt}%
\pgfsys@defobject{currentmarker}{\pgfqpoint{-0.048611in}{0.000000in}}{\pgfqpoint{0.000000in}{0.000000in}}{%
\pgfpathmoveto{\pgfqpoint{0.000000in}{0.000000in}}%
\pgfpathlineto{\pgfqpoint{-0.048611in}{0.000000in}}%
\pgfusepath{stroke,fill}%
}%
\begin{pgfscope}%
\pgfsys@transformshift{0.625000in}{0.750163in}%
\pgfsys@useobject{currentmarker}{}%
\end{pgfscope}%
\end{pgfscope}%
\begin{pgfscope}%
\definecolor{textcolor}{rgb}{0.000000,0.000000,0.000000}%
\pgfsetstrokecolor{textcolor}%
\pgfsetfillcolor{textcolor}%
\pgftext[x=0.306898in,y=0.697401in,left,base]{\color{textcolor}\sffamily\fontsize{10.000000}{12.000000}\selectfont 0.6}%
\end{pgfscope}%
\begin{pgfscope}%
\pgfsetbuttcap%
\pgfsetroundjoin%
\definecolor{currentfill}{rgb}{0.000000,0.000000,0.000000}%
\pgfsetfillcolor{currentfill}%
\pgfsetlinewidth{0.803000pt}%
\definecolor{currentstroke}{rgb}{0.000000,0.000000,0.000000}%
\pgfsetstrokecolor{currentstroke}%
\pgfsetdash{}{0pt}%
\pgfsys@defobject{currentmarker}{\pgfqpoint{-0.048611in}{0.000000in}}{\pgfqpoint{0.000000in}{0.000000in}}{%
\pgfpathmoveto{\pgfqpoint{0.000000in}{0.000000in}}%
\pgfpathlineto{\pgfqpoint{-0.048611in}{0.000000in}}%
\pgfusepath{stroke,fill}%
}%
\begin{pgfscope}%
\pgfsys@transformshift{0.625000in}{1.353507in}%
\pgfsys@useobject{currentmarker}{}%
\end{pgfscope}%
\end{pgfscope}%
\begin{pgfscope}%
\definecolor{textcolor}{rgb}{0.000000,0.000000,0.000000}%
\pgfsetstrokecolor{textcolor}%
\pgfsetfillcolor{textcolor}%
\pgftext[x=0.306898in,y=1.300746in,left,base]{\color{textcolor}\sffamily\fontsize{10.000000}{12.000000}\selectfont 0.7}%
\end{pgfscope}%
\begin{pgfscope}%
\pgfsetbuttcap%
\pgfsetroundjoin%
\definecolor{currentfill}{rgb}{0.000000,0.000000,0.000000}%
\pgfsetfillcolor{currentfill}%
\pgfsetlinewidth{0.803000pt}%
\definecolor{currentstroke}{rgb}{0.000000,0.000000,0.000000}%
\pgfsetstrokecolor{currentstroke}%
\pgfsetdash{}{0pt}%
\pgfsys@defobject{currentmarker}{\pgfqpoint{-0.048611in}{0.000000in}}{\pgfqpoint{0.000000in}{0.000000in}}{%
\pgfpathmoveto{\pgfqpoint{0.000000in}{0.000000in}}%
\pgfpathlineto{\pgfqpoint{-0.048611in}{0.000000in}}%
\pgfusepath{stroke,fill}%
}%
\begin{pgfscope}%
\pgfsys@transformshift{0.625000in}{1.956852in}%
\pgfsys@useobject{currentmarker}{}%
\end{pgfscope}%
\end{pgfscope}%
\begin{pgfscope}%
\definecolor{textcolor}{rgb}{0.000000,0.000000,0.000000}%
\pgfsetstrokecolor{textcolor}%
\pgfsetfillcolor{textcolor}%
\pgftext[x=0.306898in,y=1.904091in,left,base]{\color{textcolor}\sffamily\fontsize{10.000000}{12.000000}\selectfont 0.8}%
\end{pgfscope}%
\begin{pgfscope}%
\pgfsetbuttcap%
\pgfsetroundjoin%
\definecolor{currentfill}{rgb}{0.000000,0.000000,0.000000}%
\pgfsetfillcolor{currentfill}%
\pgfsetlinewidth{0.803000pt}%
\definecolor{currentstroke}{rgb}{0.000000,0.000000,0.000000}%
\pgfsetstrokecolor{currentstroke}%
\pgfsetdash{}{0pt}%
\pgfsys@defobject{currentmarker}{\pgfqpoint{-0.048611in}{0.000000in}}{\pgfqpoint{0.000000in}{0.000000in}}{%
\pgfpathmoveto{\pgfqpoint{0.000000in}{0.000000in}}%
\pgfpathlineto{\pgfqpoint{-0.048611in}{0.000000in}}%
\pgfusepath{stroke,fill}%
}%
\begin{pgfscope}%
\pgfsys@transformshift{0.625000in}{2.560197in}%
\pgfsys@useobject{currentmarker}{}%
\end{pgfscope}%
\end{pgfscope}%
\begin{pgfscope}%
\definecolor{textcolor}{rgb}{0.000000,0.000000,0.000000}%
\pgfsetstrokecolor{textcolor}%
\pgfsetfillcolor{textcolor}%
\pgftext[x=0.306898in,y=2.507435in,left,base]{\color{textcolor}\sffamily\fontsize{10.000000}{12.000000}\selectfont 0.9}%
\end{pgfscope}%
\begin{pgfscope}%
\pgfsetbuttcap%
\pgfsetroundjoin%
\definecolor{currentfill}{rgb}{0.000000,0.000000,0.000000}%
\pgfsetfillcolor{currentfill}%
\pgfsetlinewidth{0.803000pt}%
\definecolor{currentstroke}{rgb}{0.000000,0.000000,0.000000}%
\pgfsetstrokecolor{currentstroke}%
\pgfsetdash{}{0pt}%
\pgfsys@defobject{currentmarker}{\pgfqpoint{-0.048611in}{0.000000in}}{\pgfqpoint{0.000000in}{0.000000in}}{%
\pgfpathmoveto{\pgfqpoint{0.000000in}{0.000000in}}%
\pgfpathlineto{\pgfqpoint{-0.048611in}{0.000000in}}%
\pgfusepath{stroke,fill}%
}%
\begin{pgfscope}%
\pgfsys@transformshift{0.625000in}{3.163542in}%
\pgfsys@useobject{currentmarker}{}%
\end{pgfscope}%
\end{pgfscope}%
\begin{pgfscope}%
\definecolor{textcolor}{rgb}{0.000000,0.000000,0.000000}%
\pgfsetstrokecolor{textcolor}%
\pgfsetfillcolor{textcolor}%
\pgftext[x=0.306898in,y=3.110780in,left,base]{\color{textcolor}\sffamily\fontsize{10.000000}{12.000000}\selectfont 1.0}%
\end{pgfscope}%
\begin{pgfscope}%
\definecolor{textcolor}{rgb}{0.000000,0.000000,0.000000}%
\pgfsetstrokecolor{textcolor}%
\pgfsetfillcolor{textcolor}%
\pgftext[x=0.251343in,y=1.980000in,,bottom,rotate=90.000000]{\color{textcolor}\sffamily\fontsize{10.000000}{12.000000}\selectfont \(\displaystyle \alpha\)}%
\end{pgfscope}%
\begin{pgfscope}%
\pgfpathrectangle{\pgfqpoint{0.625000in}{0.440000in}}{\pgfqpoint{3.875000in}{3.080000in}}%
\pgfusepath{clip}%
\pgfsetbuttcap%
\pgfsetroundjoin%
\pgfsetlinewidth{1.505625pt}%
\definecolor{currentstroke}{rgb}{0.000000,0.000000,0.000000}%
\pgfsetstrokecolor{currentstroke}%
\pgfsetdash{{5.550000pt}{2.400000pt}}{0.000000pt}%
\pgfpathmoveto{\pgfqpoint{0.801136in}{3.380000in}}%
\pgfpathlineto{\pgfqpoint{2.163913in}{2.516354in}}%
\pgfpathlineto{\pgfqpoint{2.961087in}{2.011153in}}%
\pgfpathlineto{\pgfqpoint{3.526690in}{1.652708in}}%
\pgfpathlineto{\pgfqpoint{3.965406in}{1.374676in}}%
\pgfpathlineto{\pgfqpoint{4.323864in}{1.147507in}}%
\pgfusepath{stroke}%
\end{pgfscope}%
\begin{pgfscope}%
\pgfpathrectangle{\pgfqpoint{0.625000in}{0.440000in}}{\pgfqpoint{3.875000in}{3.080000in}}%
\pgfusepath{clip}%
\pgfsetbuttcap%
\pgfsetroundjoin%
\definecolor{currentfill}{rgb}{0.000000,0.000000,0.000000}%
\pgfsetfillcolor{currentfill}%
\pgfsetlinewidth{1.003750pt}%
\definecolor{currentstroke}{rgb}{0.000000,0.000000,0.000000}%
\pgfsetstrokecolor{currentstroke}%
\pgfsetdash{}{0pt}%
\pgfsys@defobject{currentmarker}{\pgfqpoint{-0.041667in}{-0.041667in}}{\pgfqpoint{0.041667in}{0.041667in}}{%
\pgfpathmoveto{\pgfqpoint{0.000000in}{-0.041667in}}%
\pgfpathcurveto{\pgfqpoint{0.011050in}{-0.041667in}}{\pgfqpoint{0.021649in}{-0.037276in}}{\pgfqpoint{0.029463in}{-0.029463in}}%
\pgfpathcurveto{\pgfqpoint{0.037276in}{-0.021649in}}{\pgfqpoint{0.041667in}{-0.011050in}}{\pgfqpoint{0.041667in}{0.000000in}}%
\pgfpathcurveto{\pgfqpoint{0.041667in}{0.011050in}}{\pgfqpoint{0.037276in}{0.021649in}}{\pgfqpoint{0.029463in}{0.029463in}}%
\pgfpathcurveto{\pgfqpoint{0.021649in}{0.037276in}}{\pgfqpoint{0.011050in}{0.041667in}}{\pgfqpoint{0.000000in}{0.041667in}}%
\pgfpathcurveto{\pgfqpoint{-0.011050in}{0.041667in}}{\pgfqpoint{-0.021649in}{0.037276in}}{\pgfqpoint{-0.029463in}{0.029463in}}%
\pgfpathcurveto{\pgfqpoint{-0.037276in}{0.021649in}}{\pgfqpoint{-0.041667in}{0.011050in}}{\pgfqpoint{-0.041667in}{0.000000in}}%
\pgfpathcurveto{\pgfqpoint{-0.041667in}{-0.011050in}}{\pgfqpoint{-0.037276in}{-0.021649in}}{\pgfqpoint{-0.029463in}{-0.029463in}}%
\pgfpathcurveto{\pgfqpoint{-0.021649in}{-0.037276in}}{\pgfqpoint{-0.011050in}{-0.041667in}}{\pgfqpoint{0.000000in}{-0.041667in}}%
\pgfpathclose%
\pgfusepath{stroke,fill}%
}%
\begin{pgfscope}%
\pgfsys@transformshift{0.801136in}{3.062742in}%
\pgfsys@useobject{currentmarker}{}%
\end{pgfscope}%
\begin{pgfscope}%
\pgfsys@transformshift{2.163913in}{2.696563in}%
\pgfsys@useobject{currentmarker}{}%
\end{pgfscope}%
\begin{pgfscope}%
\pgfsys@transformshift{2.961087in}{2.476225in}%
\pgfsys@useobject{currentmarker}{}%
\end{pgfscope}%
\begin{pgfscope}%
\pgfsys@transformshift{3.526690in}{1.672621in}%
\pgfsys@useobject{currentmarker}{}%
\end{pgfscope}%
\begin{pgfscope}%
\pgfsys@transformshift{3.965406in}{1.594247in}%
\pgfsys@useobject{currentmarker}{}%
\end{pgfscope}%
\begin{pgfscope}%
\pgfsys@transformshift{4.323864in}{0.580000in}%
\pgfsys@useobject{currentmarker}{}%
\end{pgfscope}%
\end{pgfscope}%
\begin{pgfscope}%
\pgfsetrectcap%
\pgfsetmiterjoin%
\pgfsetlinewidth{0.803000pt}%
\definecolor{currentstroke}{rgb}{0.000000,0.000000,0.000000}%
\pgfsetstrokecolor{currentstroke}%
\pgfsetdash{}{0pt}%
\pgfpathmoveto{\pgfqpoint{0.625000in}{0.440000in}}%
\pgfpathlineto{\pgfqpoint{0.625000in}{3.520000in}}%
\pgfusepath{stroke}%
\end{pgfscope}%
\begin{pgfscope}%
\pgfsetrectcap%
\pgfsetmiterjoin%
\pgfsetlinewidth{0.803000pt}%
\definecolor{currentstroke}{rgb}{0.000000,0.000000,0.000000}%
\pgfsetstrokecolor{currentstroke}%
\pgfsetdash{}{0pt}%
\pgfpathmoveto{\pgfqpoint{4.500000in}{0.440000in}}%
\pgfpathlineto{\pgfqpoint{4.500000in}{3.520000in}}%
\pgfusepath{stroke}%
\end{pgfscope}%
\begin{pgfscope}%
\pgfsetrectcap%
\pgfsetmiterjoin%
\pgfsetlinewidth{0.803000pt}%
\definecolor{currentstroke}{rgb}{0.000000,0.000000,0.000000}%
\pgfsetstrokecolor{currentstroke}%
\pgfsetdash{}{0pt}%
\pgfpathmoveto{\pgfqpoint{0.625000in}{0.440000in}}%
\pgfpathlineto{\pgfqpoint{4.500000in}{0.440000in}}%
\pgfusepath{stroke}%
\end{pgfscope}%
\begin{pgfscope}%
\pgfsetrectcap%
\pgfsetmiterjoin%
\pgfsetlinewidth{0.803000pt}%
\definecolor{currentstroke}{rgb}{0.000000,0.000000,0.000000}%
\pgfsetstrokecolor{currentstroke}%
\pgfsetdash{}{0pt}%
\pgfpathmoveto{\pgfqpoint{0.625000in}{3.520000in}}%
\pgfpathlineto{\pgfqpoint{4.500000in}{3.520000in}}%
\pgfusepath{stroke}%
\end{pgfscope}%
\begin{pgfscope}%
\pgfsetbuttcap%
\pgfsetmiterjoin%
\definecolor{currentfill}{rgb}{1.000000,1.000000,1.000000}%
\pgfsetfillcolor{currentfill}%
\pgfsetfillopacity{0.800000}%
\pgfsetlinewidth{1.003750pt}%
\definecolor{currentstroke}{rgb}{0.800000,0.800000,0.800000}%
\pgfsetstrokecolor{currentstroke}%
\pgfsetstrokeopacity{0.800000}%
\pgfsetdash{}{0pt}%
\pgfpathmoveto{\pgfqpoint{2.317632in}{3.199199in}}%
\pgfpathlineto{\pgfqpoint{4.402778in}{3.199199in}}%
\pgfpathquadraticcurveto{\pgfqpoint{4.430556in}{3.199199in}}{\pgfqpoint{4.430556in}{3.226977in}}%
\pgfpathlineto{\pgfqpoint{4.430556in}{3.422778in}}%
\pgfpathquadraticcurveto{\pgfqpoint{4.430556in}{3.450556in}}{\pgfqpoint{4.402778in}{3.450556in}}%
\pgfpathlineto{\pgfqpoint{2.317632in}{3.450556in}}%
\pgfpathquadraticcurveto{\pgfqpoint{2.289854in}{3.450556in}}{\pgfqpoint{2.289854in}{3.422778in}}%
\pgfpathlineto{\pgfqpoint{2.289854in}{3.226977in}}%
\pgfpathquadraticcurveto{\pgfqpoint{2.289854in}{3.199199in}}{\pgfqpoint{2.317632in}{3.199199in}}%
\pgfpathclose%
\pgfusepath{stroke,fill}%
\end{pgfscope}%
\begin{pgfscope}%
\pgfsetbuttcap%
\pgfsetroundjoin%
\pgfsetlinewidth{1.505625pt}%
\definecolor{currentstroke}{rgb}{0.000000,0.000000,0.000000}%
\pgfsetstrokecolor{currentstroke}%
\pgfsetdash{{5.550000pt}{2.400000pt}}{0.000000pt}%
\pgfpathmoveto{\pgfqpoint{2.345410in}{3.338088in}}%
\pgfpathlineto{\pgfqpoint{2.623187in}{3.338088in}}%
\pgfusepath{stroke}%
\end{pgfscope}%
\begin{pgfscope}%
\definecolor{textcolor}{rgb}{0.000000,0.000000,0.000000}%
\pgfsetstrokecolor{textcolor}%
\pgfsetfillcolor{textcolor}%
\pgftext[x=2.734298in,y=3.289477in,left,base]{\color{textcolor}\sffamily\fontsize{10.000000}{12.000000}\selectfont \(\displaystyle \alpha = 1.5113 - 0.2065\log(k)\)}%
\end{pgfscope}%
\end{pgfpicture}%
\makeatother%
\endgroup%

    \end{subfigure}
\caption{Plots of the dependence of $C$ and $\alpha$ on $k$ in \cref{eq:qmcerrorform} for $Q(u) = \gradu((1,1))$. Observe the $x$-axis is on a $\log_{10}$ scale, but $\loge$ is the natural logarithm.  \label{fig:gradienttoprightCalpha}}
\end{figure}


\begin{table}[h!]
  \centering
  \begin{tabular}{Sc Sc Sc Sc Sc}
\toprule
{} & \$Q = \textbackslash int\_D u\$ & \$Q = u(\textbackslash bzero)\$ & \$Q = u((1,1))\$ & \$Q = \textbackslash gradu((1,1))\$ \\
\midrule
\$\textbackslash alphaz\$        &           1.28 &            1.36 &           1.43 &                1.43 \\
\$\textbackslash alpha\$         &           0.30 &            0.38 &           0.41 &                0.41 \\
\$\textbackslash alphaz/alphao\$ &           0.24 &            0.28 &           0.29 &                0.29 \\
\bottomrule
\end{tabular}

  \caption{The quantities $\alphaz$ and $\alphao$ for differents QoIs, where the QMC error $Err \approx C \NQMC^{\alphaz - \alphao\loge(k)}$.}\label{tab:qmcalpha}
  \end{table}

Based on QMC theory results for the stationary diffusion equation, e.g., \cite[Equation 4.2]{GrKuNuScSl:11}, we make the assumption that the QMC error (with one shift) satisfies
\beq\label{eq:qmcerrorform}
\QMCerror{Q}{\Nshifts} = C \NQMC^{-\alpha},
\eeq
for some $C, \alpha > 0.$ Based on this assumption, \cref{fig:integralCalpha,fig:originCalpha,fig:toprightCalpha,fig:gradienttoprightCalpha} plot $C$ and $\alpha$ for increasing $k$. (In \cref{app:hhqmcconv}, we plot the QMC error for increasing $\NQMC$ for each $k \in \set{10,20,30,40,50,60}$ and for each QoI---these plots allow us to determine the values of $C$ and $\alpha$ for each value of $k.$) For the QoIs that are point evaluations (\cref{fig:originCalpha,fig:toprightCalpha}), $C$ appears to be approximately constant; we assume $C$ is constant in all of the following calculations.

Based on the evidence in \cref{fig:integralCalpha,fig:originCalpha,fig:toprightCalpha,fig:gradienttoprightCalpha}, we conjecture
\beq\label{eq:alphaform}
\alpha(k) = \alphaz - \alphao\loge(k),
\eeq
for some constants $\alphaz,\alphao > 0.$ (Throughout this \lcnamecref{sec:nbpcqmcnumerics}, $\loge$ denotes the natural logarithm.) We fitted $\alphaz$ and $\alphao$ numerically, and have plotted the resulting line of best fit on \cref{fig:integralCalpha,fig:originCalpha,fig:toprightCalpha,fig:gradienttoprightCalpha}. (Observe that the conjectured form \cref{eq:alphaform} cannot hold for $k$ very large, as then $\alpha(k)$ would be negative. Nevertheless, for the range of $k$ we consider in these numerical experiments, the form \cref{eq:alphaform} seems to give a good fit with the data.) The values of $C$ and $\alpha$ for the different QoIs are given in \cref{fig:integralCalpha,fig:originCalpha,fig:toprightCalpha,fig:gradienttoprightCalpha}.

Having understood how the QMC error increases with $k$ for fixed $\NQMC$, we now use this knowledge to determine how one should increase $\NQMC$ with $k$ in order to keep the QMC error bounded. Recalling that we assume $C$ in \cref{eq:qmcerrorform} is constant, if we take
\beq\label{eq:Nform}
\NQMC(k) = \exp\mleft(\Ctilde \alpha(k)^{-1}\mright),
\eeq
for some constant $\Ctilde > 0$, then substituting \cref{eq:Nform} into \cref{eq:qmcerrorform}, we see that the QMC error should remain bounded, with
\beqs
\QMCerror{Q}{\Nshifts} = C \exp\mleft(-\Ctilde\mright).
\eeqs

In our numerical experiments with increasing $\NQMC(k)$ below, we choose $\Ctilde$ so that $\NQMC(10) = 2048,$ because in our numerical experiments to determine the behaviour of the QMC error, we used $\NQMC = 2048$ (with 20 shifts). Also in our numerical experiments below we take the number of QMC points to be a power of 2, because the lattice rule we use to generate the points is a complete lattice rule if $\NQMC$ is a power of 2 (see \cite{NuREADME}). We choose $\NQMC$ to be a power of 2 by setting $\NQMC(k) = 2^{M(k)},$ where
\beqs
M(k) = \round{\logtwo\mleft(\exp\mleft(\Ctilde \alpha(k)^{-1}\mright)\mright)}.
\eeqs

Based on the results for the QoIs in \cref{tab:qmcalpha} (excluding the results for the QoI being the integral of the solution, as this seems to display slightly different convergence characteristics), in our numerical experiments below we take $\alpha(k) = 1.38 - 0.19  \loge(k).$ The resulting values of $\NQMC$ are summarised in \cref{tab:nqmc}.

\begin{table}[h]
  \centering
  \begin{tabular}{Sc Sc Sc }

\toprule

$k$ & $\exp\mleft(\Ctilde\alpha(k)^{-1}\mright)$ & $\NQMC$\\
\midrule

10 &                                   $2^{11}$ &  $2^{11}$ \\

20 &                                $2^{12.78}$ &  $2^{13}$ \\

30 &                                $2^{14.12}$ &  $2^{14}$ \\

40 &                                $2^{15.26}$ &  $2^{15}$ \\

50 &                                $2^{16.28}$ &  $2^{16}$ \\

60 &                                $2^{17.21}$ &  $2^{17}$ \\

\bottomrule

\end{tabular}


  \caption{The ideal and actual number of QMC points $\NQMC$, chosen so that the QMC error is empirically bounded for all $k$.}\label{tab:qmcpoints}
  \end{table}

\subsubsection{Numerical results for nearby preconditioning applied to QMC}

Now that we understand how the number of QMC points should scale with $k$ in order to keep the QMC error bounded, we apply nearby preconditioning to QMC (with the number of points scaling as above) and observe how the computational work of this nearby-preconditioning-QMC (NP-QMC) algorithm scales with $k.$\ednote{Should I do the QMC calculations with multiple shifts, but this point scaling, again, and check that the estimate of the QMC error is roughly constant? It could take a while, and be rather expensive....}

As outlined above, we combine our sequential- and parallel-NPQMC algorithms:
\bit
\item We first use the sequential algorithm for low $k$ (fixing the maximum number of GMRES iterations) and observe how the number of preconditioners (as a proportion of the number of QMC points) changes with $k$.
  \item We then use the parallel algorithm (with the above proportion of preconditioners) for higher values of $k.$
    \eit
    We remark that, in principle, one could use the sequential algorithm for all values of $k$, however, this would take an incredibly long time--- we see in \cref{tab:qmcpoints} that for $k=60$ we must perform $2^{17}$ Helmholtz solves; if we performed these solves sequentially, and each solve took 10 seconds, this computation would take over 2 weeks to complete.

    The results for the sequential algorithm are summarised in \cref{tab:nbpcqmcseq}, for $k = 10,\,20,\,30$. These results show that nearby preconditioning is effective, with the number of preconditioners growing (approximately) linearly in $k$, but at a very low percentage of the total number of solves. This result indicates the decreasing radius (on the order $1/k$) over which nearby preconditioning is effective is mostly offset by the growing number of QMC points. Observe that if the number of QMC points remained constant in $k$, we would expect the number of preconditioners to (potentially) increase like $k^J$, because the number of balls of radius $\sim 1/k$ in $\cube{J}$ is $\sim k^J.$

    \optodo{Rewrite the next two paragraphs later}
    Based on these sequential results, we then used the parallel algorithm with a target proportion of preconditioners of 0.2\%. (Although recall from our discussion above that the actual proportion of preconditioners used can vary due to rounding in the algorithm.) The results of these computations are summarised in \cref{tab:par}. We observe that the fraction of preconditioners is approximate 0.2\%, but the maximum (and average) number of GMRES iterations appears to grow slowly with $k.$ This may be because the placement of the preconditioning points is not optimal with respect to the $\dQMC$ metric; we conjecture that oversampling the number of preconditioners needed (for example, taking a proportion of 0.5\%) may result in a bounded number of GMRES iterations\ednote{Both---Should I run these computations?} Nevertheless, we see that nearby preconditioning gives considerable speedup, drastically reducing the number of preconditioners that must be calculated.

    \begin{table}
  \centering
  \begin{tabular}{Sc Sc Sc Sc Sc Sc}
\toprule

$k$ & \# LU factorisations & \makecell{Total \#\\linear systems} & \makecell{\# LU factorisations$/$\\\# linear systems}(\%) & \makecell{Average \#\\GMRES iterations} & \makecell{Max. \#\\GMRES iterations}\\
\midrule

10 &                    5 &                                2048 &                                               0.24 &                                    7.13 &                                   10 \\

20 &                   44 &                               16384 &                                               0.27 &                                    7.24 &                                   10 \\

30 &                  216 &                              131072 &                                               0.16 &                                    7.23 &                                   10 \\

\bottomrule

\end{tabular}


  \caption{Results applying our sequential nearby-preconditioning-Quasi-Monte-Carlo algorithm, with the maximum number of GMRES iterations $=10$.}\label{tab:nbpcqmcseq}
\end{table}

\begin{table}
  \centering
  \begin{tabular}{Sc Sc Sc Sc Sc Sc}
\toprule

$k$ & \# LU factorisations & \makecell{Total \#\\linear systems} & \makecell{\# LU factorisations$/$\\\# linear systems}(\%) & \makecell{Average \#\\GMRES iterations} & \makecell{Max. \#\\GMRES iterations}\\
\midrule

10 &                    4 &                                2048 &                                               0.20 &                                    6.46 &                                   10 \\

20 &                   33 &                                8192 &                                               0.40 &                                    6.42 &                                   11 \\

30 &                  127 &                               16384 &                                               0.78 &                                    6.66 &                                   13 \\

40 &                  207 &                               32768 &                                               0.63 &                                    7.16 &                                   15 \\

50 &                 1027 &                               65536 &                                               1.57 &                                    7.07 &                                   14 \\

60 &                 1444 &                              131072 &                                               1.10 &                                    7.41 &                                   16 \\

\bottomrule

\end{tabular}


  \caption{Results applying our parallel nearby-preconditioning-Quasi-Monte-Carlo algorithm with the target proportion of preconditioners as $0.2$\%.}\label{tab:nbpcqmcspar}
  \end{table}


    In conclusion, we see that nearby preconditioning gives a very significant speedup when applied to a QMC model problem. We therefore expect that this technique will give significant speed up when applied to other, more realistic problems.
    

    
\section{Extension of the results to the truncated exterior Dirichlet problem}\label{sec:TEDP}

We now briefly outline how the results in \cref{sec:main} above can be extended to \cref{prob:vtedp}, the Truncated Exterior Dirichlet Problem.

%\subsection{Definition of the TEDP and analogues of the results in \cref{sec:3}}

%% \paragraph{The impedance boundary $\Gamma_I$.} By comparing \cref{eq:src,eq:ibc}, we see that, in the case $g_I=0$, the TEDP approximates the DtN operator $T_R$ by $\ri k$. Indeed, by using Green's first identity and the definition of the normal derivative (see, e.g., \cite[Lemma 4.3]{Mc:00}), show that the boundary condition on $\Gamma_I$ imposed in the variational problem \cref{prob:vtedp} is 
%% %In this BVP, the DtN operator $T_R$ Sommerfeld radiation condition 
%% \beq\label{eq:imp}
%% \dudnu - \ri k\gamma u = g_I \ton \Gamma_I.
%% \eeq
%% where $\nu$ is the unit outward-pointing normal vector to $\Omega$ on $\Gamma_I$.

\paragraph{Existence and uniqueness of a solution to the TEDP.} The sesquilinear form $\aT(\cdot,\cdot)$ defined in \cref{eq:aT} satisfies the G\aa rding inequality \cref{eq:gardingbrief}, and existence and uniqueness of a solution to the TEDP follow under the same condition on $A$ (piecewise-Lipschitz) as for the EDP, as discussed in \cref{thm:tedp}.%sec:vpGm}; in the case of Lipschitz scalar $A$, these unique-continuation arguments are summarised in \cite[\S2]{GrSa:18}.

\paragraph{Finite-element/Galerkin solution.}
The Galerkin matrix $\Amat$ is defined exactly as in \cref{eq:matrixAdef}, except that 
\beq\label{eq:NTEDP}
\big(\Nmat\big)_{ij}\de \ri k\int_{\Gamma_I}  (\gamma\phi_i) \,\gamma \phi_j.
\eeq

\paragraph{The adjoint sesquilinear form.} For the TEDP, the adjoint sesquilinear form is given by 
\beq\label{eq:TEDPadjoint}
a^*(u,v) \de \int_{\DR} 
\Big((A \grad u)\cdot\grad \vb
 - k^2 n u\vb\Big) +\ri k\int_{\Gamma_I} \gamma u\, \overline{\gamma v};
\eeq
then \cref{eq:A*} holds (with $\Nmat$ now given by \cref{eq:NTEDP}), and the analogue of \cref{lem:adjoint} follows in a straightforward way.


\paragraph{The analogues of \cref{cond:1nbpc,cond:2}.}
The statement of the TEDP analogues of \cref{cond:1nbpc,cond:2} are the same as for the EDP, apart from the following.
\ben
\item
$\supp \,f$ need not be a subset of $\widetilde{\Omega}$ (i.e.~the support of $f$ can go up to the impedance boundary $\Gamma_I$), and
\item the assumption $g_I= 0$ needs to be added to \cref{cond:1nbpc} and Part (i) of \cref{cond:2}.
\een
 Note that, since $\aT(\cdot,\cdot)$ for the TEDP satisfies the same G\aa rding inequality \cref{eq:gardingbrief} as $a(\cdot,\cdot)$ for the EDP, \cref{lem:H1} holds for the TEDP under the TEDP-analogue of \cref{cond:1nbpc}.

\paragraph{The main results \cref{thm:1,cor:1}.}
Since \cref{cond:1nbpc,cond:2} are essentially unchanged from the EDP case, \cref{lem:keylemma1,lem:keylemma2} hold for the TEDP, and thus so do \cref{thm:1,cor:1,cor:1a}.

\paragraph{The PDE results \cref{thm:2} and \cref{lem:sharp}.}

The PDE bound \cref{thm:2} relies only on \cref{lem:H1}, which, as stated above, also holds for the TEDP. Therefore \cref{thm:2} holds for the TEDP under the TEDP-analogue of \cref{cond:1nbpc} described above. The construction in \cref{lem:sharp} to show sharpness of the bound in \cref{thm:1} (at least when $\Aso= \Ast= I$) also holds for the TEDP; this is because one can choose the supports of $\chi$ and $\widetilde{\chi}$ to be contained inside $\widetilde{\Omega}$, and then $u^{(1)}$ and $u^{(2)}$ defined in \cref{lem:sharp} satisfy the impedance boundary condition \cref{eq:imp} on $\Gamma_I$.

%% \paragraph{When the TEDP-analogue of \cref{cond:1nbpc} holds.}

%% In \cref{sec:cond1hold} we discussed 4 situations (Cases 1-4) where \cref{cond:1nbpc} is proved to hold for the EDP. We now discuss the TEDP-analogues of these.
%% %Cases 1, 3, and 4 (there is no proof yet for the TEDP-analogue of Case 2).

%% \emph{Cases 1 and 2: $\Aso$, $\nso$, and $\Gamma_I$  are $C^\infty$.} 
%% With the rays defined as in the EDP case (by the Melrose--Sj{\"o}strand generalized bicharacteristic flow 
%% \cite[\S24.3]{Ho:85}), the TEDP-analogue of nontrapping for the EDP is the assumption that 
%% every ray eventually hits the boundary at a \emph{non-diffractive point} (defined in \cite[Page 1037]{BaLeRa:92}). Note that, in the case $\Dm=\emptyset$ $\Aso= I$, and $\nso=1$, every ray eventually hits the boundary at a non-diffractive point by \cite[Lemma 5.3]{BaSpWu:16}.
%% Under the additional assumption that $\nso= 1$, \cref{cond:1nbpc} follows from the results of \cite{BaLeRa:92} by combining \cite[Theorem 1.8]{BaSpWu:16} and \cite[Remark 5.6]{BaSpWu:16}, but $C^{(1)}_{\rm bound}$ is not given explicitly.

%% \emph{Case 3: $\Dm$ is starshaped with respect to the origin, $\Aso$ and $\nso$ are Lipschitz and satisfy radial monotonicity-like conditions.}
%% When $\Gamma_I$ is also starshaped with respect to the origin and $A$ and $n$ satisfy \cref{eq:A1nbpc} and \cref{eq:n1nbpc} respectively (with $\Dp$ replaced by $\Omega$), 
%% \cite[Theorem A.6(i)]{GrPeSp:19} proves that
%% \cref{cond:1nbpc} holds, with an explicit expression for $C^{(1)}_{\rm bound}$. Analogous results when (a) $2\Aso - (\bx\cdot\nabla)\Aso \geq \mu_1$ and $\nso= 1$,
%% and  (b) $\Aso= I$ and  $2\nso + \bx \cdot \nabla \nso \geq \mu_2$, 
%% are contained in \cite[Theorem A.6(ii)]{GrPeSp:19} and \cite[Theorem A.6(iii)]{GrPeSp:19} respectively.
%% When $A$ is scalar, these results were also proved in \cite[Theorem 1]{BrGaPe:17} and, when $\Aso= I$ and $\Dm=\emptyset$, also in \cite[Theorem 3.2]{GrSa:18}.

%% \emph{Case 4: %\item[Case 4:]
%%  $\Aso$ and $\nso$ are allowed to be discontinuous.}
%% %\een
%% \cref{cond:1nbpc} is proved in \cite{CaVo:10} (without an explicit expression for $C^{(1)}_{\rm bound}$) when $\Dm$ is $C^\infty$ and nontrapping, $\Gamma_I$ is $C^\infty$, $\Aso= I $, and $\nso$ is a piecewise-constant, monotonically non-decreasing function, jumping on interfaces that are $C^\infty$ with strictly positive curvature.
%% Recall from \cref{cond:1nbpc} that \cite[Theorem 2.7]{GrPeSp:19} proves that \cref{cond:1nbpc} holds for the EDP (with an explicit expression for $C^{(1)}_{\rm bound}$) when $\Dm$ is starshaped with respect to the origin, $A$ and $n$ are $L^\infty$, with $A$ monotonically \emph{non-increasing} in the radial direction, and $n$ monotonically \emph{non-decreasing}. This proof can be extended to the TEDP, with the additional assumption that $\Gamma_I$ is star-shaped with respect to the origin; see the discussion in \cite[Section A.2]{GrPeSp:19}.

%\cref{cond:1nbpc} is proved, with an explicit expression for $C^{(1)}_{\rm bound}$, when 

%\newpage
%
%\section*{Questions for Th\'eo}
%
%\ben
%\item At the place marked A on the scanned pages, you seem to use the inequality 
%\beq\label{eq:Theo1}
%\vert\vert\vert \xi - \cP_h \xi\vert\vert\vert \lesssim h^\alpha \N{u_\phi- \cP_h u_\phi}_{0,\Omega}.
%\eeq
%\een
%
%\newpag

%% \section*{Owen to do list}
%% \ben
%% \item Varying  $\|\Aso-\Ast\|_{L^\infty}$ and $\|\nso-\nst\|_{L^\infty}$ in standard GMRES.
%% \item Computations where $\|\Aso-\Ast\|_{L^\infty}$ and $\|\nso-\nst\|_{L^\infty}$ are sometimes large; is having the standard deviations of these $\sim 1/k$ good enough for $k$-independent GMRES iterations?
%% \item ***on backburner*** Checking under what conditions (if any) Part (ii) \cref{cond:2} holds by running the following experiment:
%% %\item Exciting experiments for random $n$ that you told us about last week.
%% %\item In the weighted norm, the condition on $A$ is ``$k \|\Aso-\Ast\|_{L^\infty}$ sufficiently small" but in the Euclidean norm the best we have so far is ``$h^{-1} \|\Aso-\Ast\|_{L^\infty}$ sufficiently small". You indicated before that experiments seemed to indicate that ``$k \|\Aso-\Ast\|_{L^\infty}$ sufficiently small" seemed correct for the Euclidean norm too. The next time we meet, can you show me these results please?
%% %\item Please run the following numerical experiment.
%% \bit
%% \item TEDP with $\Omega$ a square/rectangle.
%% \item $\Aso$ being at least Lipschitz (but smooth is fine). To keep things simple, just take scalar- (as opposed to matrix-) valued $\Aso$ and don't worry about making it nontrapping.
%% \item Smoothness of $\nso$ doesn't really matter, just take smooth in the first instance for simplicity (and also don't worry about nontrapping).
%% \item $\Vhp$ piecewise linear.
%% \item Linear system $\Amato \bu = \Smat_{A} \balpha$ for some arbitrary complex-valued vector $\balpha$ and some arbitrary $A\in L^\infty$. (I claim this corresponds to the problem described in Part (ii) of \cref{cond:2},  but please check this!)
%% \item For each $\Aso, \nso, \balpha$, solve linear system for increasing values of $k$, first with $h\sim k^{-2}$, and then with $h\sim k^{-3/2}$.
%% \item Goal: see if the bound \cref{eq:bound4} holds, using 
%% \beqs
%% \N{\sum_j \alpha_j (A\nabla \phi_j)}_{\LtDR} \quad \text{ as a proxy for } \quad \N{\LE}_{(\HokDR)'}.
%% \eeqs
%% \eit
%% %\item Varying  $\|\Aso-\Ast\|_{L^\infty}$ and $\|\nso-\nst\|_{L^\infty}$ in \emph{weighted} GMRES.
%% \een

