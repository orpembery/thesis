\section{Introduction and Motivation from UQ}\label{sec:intronbpc}

\subsection{Motivation from uncertainty quantification of the Helmholtz equation} 
Consider the stochastic Helmholtz equation 
\beq\label{eq:nbpchh}
\nabla\cdot\big(A(\omega,\bx) \nabla u(\omega,\bx) \big) + k^2 n(\omega,\bx) u(\omega,\bx) =-f(\bx), \quad \bx\in\Dp,
\eeq
as defined in \cref{chap:stochastic}. If $Q(u)$ is some quantity of interest of the solution, then the simplest way to approximate $\EXP{Q(u)}$ is via a sampling-based method, i.e. using the approximation
\beq\label{eq:samplingexp}
\EXP{Q(u)} \approx \frac1N \sum_{l=1}^N Q(u(\omegasl)),
\eeq
where the $\omegasl$ are elements of the sample space $\Omega.$ To calculate the right-hand side of \cref{eq:samplingexp},  one must solve many deterministic Helmholtz problems, corresponding to different samples $\omega^l$, i.e. corresponding to different realisations of the coefficients $A(\omega,\cdot)$ and $n(\omega,\cdot)$.
Solving all these deterministic problems is a very computationally-intensive task because linear systems arising from discretisations of the Helmholtz equation are notoriously difficult to solve; see the discussion in \cref{sec:numsolve} above. In particular, direct solvers involving the LU decomposition of the linear system have a computational cost of the order $\cO\mleft(N^{3/2}\mright)$ in 2-d and $\cO\mleft(N^2\mright)$ in 3-d.

However, if one already has access to the LU decomposition, then the cost of applying a direct solver using the LU decomposition is much cheaper; $\cO\mleft(N\log N\mright)$ in 2-d and $\cO\mleft(N^{4/3}\mright)$ in 3-d. In the context of Uncertainty Quantification for the Helmholtz equation this reduction in cost when one has access to an LU decomposition suggests the folowing question: When can the LU decomposition corresponding to a particular realisation of \cref{eq:nbpchh} be used as a preconditioner for other realisations of \cref{eq:nbpchh}?

This question of reusing preconditioners is more widely applicable than just for LU decompositions. For \emph{any} preconditioner for the Helmholtz equation, one could ask when the preconditioner corresponding to one realisation of \cref{eq:nbpchh} can be re-used for other realisations. In this \lcnamecref{chap:nbpc}, for simplicity, we restrict our attention to the case where the preconditioner is an exact LU decomposition.

One expects this reuse of the preconditioner to work well if the two realisations are `nearby' in some sense. This idea of reusing preconditioners is the `nearby preconditioning' strategy proposed in this \lcnamecref{chap:nbpc}. To analyse this `nearby preconditioning' strategy rigorously, we first consider the following problem and question.

%% Therefore, in this \lcnamecref{chap:nbpc} we consider whether one can reuse the preconditioner from one solve for subsequent solves, and thereby reduce the overall computational effort.  

%%  We note that the above suggestion of re-using preconditioners should, at first glance, reduce the computational effort needed because Helmholtz preconditioners typically require more effort to \emph{construct} than they do to \emph{apply}.

%%  For example, if we calculate the action of $\AmatI$ exactly (say via an $LU$ factorisation) then (in 3-d) the cost of calculating the preconditioner is $\cO\mleft(N^2\mright)$ and the cost of applying the preconditioner is $\cO\mleft(N^{4/3}\mright)$ (where $N$ is the number of unknowns), see, e.g., \cite[Section 4]{GaZh:19}. This relationship between the cost of construction and application is also seen in recent Helmholtz preconditioners such as a sweeping preconditioners and a domain-decompostion preconditioners (see the recent review article \cite{GaZh:19} for an overview of many types of Helmholtz preconditioners). For these preconditioners one still performs $LU$-factorisations (or direct solves) on subdomains of $\Dp$ (see, e.g., \cite[Section 2]{GaZh:19}). Therefore resuing preconditioners, and thereby calculating fewer $LU$ factorisations may still lead to considerable savings. .

Let $\Aj, \nj$, $j=1,2$ satisfy the properties of $A$ and $n$ in \cref{prob:vedp} or \cref{prob:vtedp} (we will prove results for both problems), with $\uj$ the corresponding solution and $\Dm,$ $f$, etc. as in \cref{prob:vedp} or \cref{prob:vtedp}. Let $\Amatj$, $j=1,2,$ be the Galerkin matrices for the corresponding $h$-finite-element discretisations (see \cref{eq:matrixAjdef} below for a precise definition of $\Amatj$).

% Inspired by the usage of enumitem here: https://tex.stackexchange.com/a/58714
\ben[label=Q1., ref=Q1]
\item\label[itemblank]{it:nbpcq1} How small must $\N{\Aso - \Ast}$ and 
$\N{\nso - \nst}$ be (in some norm to be defined, in terms of $k$-dependence) for GMRES 
applied to $(\Amat^{(1)})^{-1}\Amat^{(2)}$ to converge in a $k$-independent number of iterations
%$ to be a good preconditioner for $\Amat^{(2)}$
 for arbitrarily large $k$? 
 \een

 \subsection{Outline of the chapter}% Don't know why cleveref didn't work here
In \cref{sec:main} we state the main results of this \lcnamecref{chap:nbpc} on the effectiveness of nearby preconditioning, and give their analogues on the PDE level. In \cref{sec:num} we describe numerical experiments investigating the sharpness of the nearby-preconditioning results in \cref{sec:main}. In \cref{sec:3} we prove the results in \cref{sec:main}. In \cref{sec:weaknorm} we extend the results in \cref{sec:main} to hold in weaker spatial norms, and we describe numerical experiments investigating the sharpness of these new results. In \cref{sec:nbpcqmc} we then apply the idea of nearby preconditioning to a Quasi-Monte-Carlo (QMC) method for the stochastic Helmholtz equation; in \cref{sec:nbpcqmcnum} we describe two algorithms for applying nearby preconditioning to QMC methods and in \cref{sec:nbpcqmcnumerics} we describe numerical experiments on the effectiveness of nearby preconditioning applied to QMC methods. In \cref{sec:nbpclitreview} we briefly review the related literature. Finally, in \cref{sec:nbpcstochastic}, we show how one can prove probabalistic results about the behaviour of nearby preconditioning, and we describe numerical experiments that investigate these probabalistic results.

\subsection{Statement of the main results}\label{sec:main}
Our main results about \cref{it:nbpcq1} are \cref{cor:1,cor:1a} below. \Cref{cor:1a} gives results in the \emph{Euclidean} norm on matrices, denoted by $\|\cdot\|_2$ (induced by the Euclidean norm on vectors), whereas \Cref{cor:1} gives results in the \emph{weighted} norms $\NDmatk{\cdot}$ and $\NDmatkI{\cdot}$. These weighted norms are induced by the corresponding vector norms
\beq\label{eq:Dk}
\NDmatk{\vvec}^2\de \big( \Dmatk \vvec, \vvec\big)_2 = %\big( (\Smat_I + k^2 \Mmat_1)\vvec,\vvec\big)_2 
\N{v_h}^2_{\HokD}
\quad \tand
\quad \NDmatkI{\vvec}^2\de \big( \Dmatk^{-1} \vvec, \vvec\big)_2 %= %\big( (\Smat_I + k^2 \Mmat_1)\vvec,\vvec\big)_2 
%\N{v_h}^2_{\HokD}
\eeq
where $\Dmatk$ is given in terms of familiar finite-element stiffness- and mass-matrices by \cref{eq:Dk2} below and $\vh =\sum_i \vi \phii$, where the $\phii$ are the finite-element basis functions.

As described in \cref{sec:wpdisc}, the PDE analysis of the Helmholtz equation naturally takes place in the norm $\NHokD{\cdot}$, and \cref{eq:Dk} shows that the norm $\NDmatk{\cdot}$ is simply the norm on the finite-element space induced by  $\NHokD{\cdot}$. %See \cref{eq:Dk2} and \cref{eq:Dk3} below for this norm expressed in terms of the Euc
The norms $\NDmatk{\cdot}$ and $\NDmatkI{\cdot}$ recently appeared in results about the convergence of domain-decomposition methods %in this norm are proved 
for the Helmholtz equation \cite{GrSpVa:17,GrSpZo:18}, and a related norm appeared in similar results for the time-harmonic Maxwell equations \cite{BoDoGrSpTo:19}. 

\Cref{cor:1,cor:1a} are proved under the \Cref{cond:1nbpc,cond:2} below. These \lcnamecrefs{cond:1nbpc} can be informally stated as 
\bit
\item the obstacle $\Dm$ and the coefficients $\Aso$ and $\nso$ are such that $u^{(1)}$ exists,is unique, and the problem is \emph{nontrapping} (in the sense described in  \cref{sec:wpdisc} above), and
\item the meshsize $h$ and polynomial degree $p$ in the finite-element method are chosen to depend on $k$ to ensure that the 
finite-element approximation to the solution of the problem with coefficients $\Aso$ and $\nso$ exists, is unique, and has bounded error in the $H^1_k$-norm
%Galerkin method (with coefficients $\Aso$ and $\no$) 
$k\tendi$. 
\eit 





\bth[Answer to \cref{it:nbpcq1}: $k$-independent weighted GMRES iterations]\label{cor:1}

\

\noindent Assume that $\Dm$, $\Aso$, and $\nso$ satisfy \cref{cond:1nbpc} below, $h$ and $p$ satisfy \cref{cond:2} below, and $\Ast$ and $\nst$ are as in \cref{prob:vgen} below. Given $k_0>0$, there exist constants $\Co$ and $\Ct$  independent of $h$ and $k$ (but dependent on $\Dm, \Aso, \nso, p$, and $\kz$) such that if 
% there exists $C_2>0$, independent of $h$ and $k$ (but dependent on $\Dm, \Aso, \nso$, $p$, and $k_0$) and given explicitly in \cref{eq:C2} below,
% such that if 
\beq\label{eq:cond}
C_1 \,k \,\NLiDRRdtd{\Aso-\Ast} +C_2 \, k\, \NLiDRR{\nso-\nst}
\leq \frac{1}{2}
\eeq
for all $k\geq k_0$, then \emph{both} weighted GMRES working in $\|\cdot\|_{\Dmat_k}$ (and the associated inner product) applied to 
\beq\label{eq:pcsystem1}
(\Amat^{(1)})^{-1}\Amat^{(2)}\uvec = \fvec
\eeq
\emph{and} weighted GMRES working in $\|\cdot\|_{(\Dmat_k)^{-1}}$ (and the associated inner product) applied to 
\beq\label{eq:pcsystem2}
\Amat^{(2)}(\Amat^{(1)})^{-1}\vvec = \fvec
\eeq
 converge in a $k$-independent number of iterations for all $k\geq k_0$.
 \enth

The constants $\Co$ and $\Ct$ are given explicitly  in \cref{eq:C1nbpc,eq:C2} below. The proof of \cref{cor:1} is on \cpageref{page:cor1proof} below.

\bth[Answer to \cref{it:nbpcq1}: $k$-independent (unweighted) GMRES iterations]\label{cor:1a}

\

\noindent Assume that $\Dm$, $\Aso$, and $\nso$ satisfy \cref{cond:1nbpc} below, $h$ and $p$ satisfy \cref{cond:2} below, and $\Ast$ and $\nst$ are as in \cref{prob:vgen} below. Given $k_0>0$,
let $C_1$ and $C_2$ be as in \cref{cor:1}, and let $s_{+}>0$ and $m_{\pm}>0$ be as in \cref{lem:normequiv} below (note that all these constants are independent of $k$, $h$, and $p$). Then if 
% there exists $C_2>0$, independent of $h$ and $k$ (but dependent on $\Dm, \Aso, \nso$, $p$, and $k_0$) and given explicitly in \cref{eq:C2} below,
% such that if 
\beq\label{eq:conda}
 C_1 \,\left(\frac{s_+}{m_-}\right) \,\frac{1}{h} \,
\NLiDRRdtd{\Aso-\Ast} + C_2 \, \left(\frac{m_+}{m_-} \right)k \, \NLiDRR{\nso-\nst}
\leq \frac{1}{2}
\eeq
for all $k\geq k_0$, then standard GMRES (working in the Euclidean norm and inner product) applied to either of the equations \cref{eq:pcsystem1} or \cref{eq:pcsystem2}
%\beqs
%(\Amat^{(1)})^{-1}\Amat^{(2)}\uvec = \fvec\quad\text{ or } \quad\Amat^{(2)}(\Amat^{(1)})^{-1}\vvec = \fvec
%\eeqs
 converges in a $k$-independent number of iterations for all $k\geq k_0$.
 \enth

 The proof of \cref{cor:1a} is on \cpageref{page:cor1aproof} below.

Three notes regarding \cref{cor:1,cor:1a}: (i) The $L^\infty$ norms of $\Ao-\At$ and $\no-\nt$ in \cref{cor:1,cor:1a} can be replaced by $L^p$ norms with $p < \infty$, at the price of making the conditions \cref{eq:cond,eq:conda} more restrictive; see \cref{sec:weaknorm} for more details. (ii) The $L^\infty(D)$ norm on a matrix-valued functions appearing on the right-hand sides of \cref{eq:main1} and \cref{eq:main1a} is defined by
\beqs
\NLiDRRdtd{A}\de \esssup_{\bx\in D}\N{A(\bx)}_2.
\eeqs
(iii) The factor $1/2$ on the right-hand sides of \cref{eq:cond} and \cref{eq:conda} can be replaced by any number $<1$ and the result still holds, although the number of GMRES iterations may then be different---but is still independent of $k$.
%In \cref{sec:proofFEM}, the constant $C_2$ is expressed explicitly in terms of $C_1$.

\bre
When $h\sim  k^{-1}$, the bounds \cref{eq:cond,eq:conda} are identical in their $k$-dependence; however, when $h\ll k^{-1}$ (as one needs to take to overcome the pollution effect, as discussed in \cref{sec:helmfedisc}) the bound \cref{eq:conda} for standard GMRES is more pessimistic than the bound \cref{eq:cond} for weighted GMRES.
\ere


\paragraph{How sharp are \cref{cor:1,cor:1a} in their $k$-dependence?}
Numerical experiments in \cref{sec:num} indicate that the condition \cref{eq:cond} is sharp, i.e., that the $k$ in \cref{eq:cond} cannot be replaced by $k^\alpha$ for $\alpha<1$. This indicated sharpness of \cref{eq:cond} is also supported by the PDE-result \cref{thm:2} below. Indeed, \cref{thm:2} % and \cref{lem:1} 
 shows that the condition
\beq\label{eq:sufficientlysmall}
k\,
\NLiDRRdtd{\Aso-\Ast} \quad\text{ and } \quad k\,\NLiDRR{\nso-\nst}
%\Big) 
\quad\text{ both sufficiently small}
\eeq
is not only an answer to \cref{it:nbpcq1} (about finite-element discretisations), but is also the natural answer to the analogue of \cref{it:nbpcq1} at the level of PDEs, namely 
\ben[label=Q2., ref=Q2]
\item\label[itemblank]{it:nbpcq2}
How small must $\NLiDRRdtd{\Aso - \Ast}$ and 
$\NLiDRR{\nso - \nst}$ be (in terms of their $k$-dependence) for the relative error in approximating 
%$u^{(1)}$ to be a good approximation to 
$u^{(2)}$ by $u^{(1)}$ to be bounded independently of $k$ for arbitrarily-large $k$? 
\een
\Cref{lem:sharp} shows that the condition ``$k\NLiDRR{\nso - \nst}$ sufficiently small" is the \emph{provably-sharp} answer to \cref{it:nbpcq2} when $\Aso= \Ast= I$.

%Before stating these PDE results, we define the weighted $H^1$ norm
%\beq\label{eq:1knorm}
%\N{v}^2_{\HokD} \de \N{\grad v}^2_{L^2(\D)} + k^2 \N{v}^2_{L^2(D)} \quad \tfor v \in H^1_{0,D}(D),
%\eeq
%where the space $H^1_{0,D}(D)$, defined by \cref{eq:spaceEDP} below, is the natural space containing the solution of the exterior Dirichlet problem. 
To state these PDE results, we use the notation for $a,b>0$ that $a\lesssim b$ when $a\leq C b$ for some $C>0$, independent of $k$, and $a\sim b$ if $a\lesssim b$ and $b\lesssim a$.


%The sharpness of \cref{eq:cond} and \cref{eq:main1} is also supported by the answer to the analogue of \cref{it:nbpcq1} at the level of PDEs. Indeed, the following \cref{thm:2} is the analogue of \cref{thm:1} 

\begin{theorem}[Answer to \cref{it:nbpcq2} (the PDE analogue of \cref{it:nbpcq1})]\label{thm:2}
%Given $f\in L^2(D)$ such that $\supp \, f \subset \BR$, 
Let $\Dm$, $\Aso$, and $\nso$ satisfy \cref{cond:1nbpc} below applied to \cref{prob:vedp} (so that the solution of \cref{prob:vedp} $\uso$ exists, is unique, and satisfies a $k$-independent a priori bound). Let $\Dm$, $\Ast$, and $\nst$ be such that $u^{(2)}$ exists
for any $f\in L^2(D)$ such that $\supp \, f \subset \BR$. 
Then, given $k_0>0$, there exists $C_3>0$, independent of $k$ and given explicitly in terms of $\Dm$, $\Aso$, and $\nso$ in \cref{eq:C3} below, such that
\beq\label{eq:PDEbound}
\frac{\big\|u^{(1)}-u^{(2)}\big\|_{\HokD}
}{
\N{u^{(2)}}_{\HokD}
}\leq C_3 \,k\, \max\set{\NLiDRRdtd{\Aso-\Ast}\,,\, \NLiDRR{\nso-\nst}}
\eeq
for all $k\geq k_0$. 
\end{theorem}

The proof of \cref{thm:2} is on \cpageref{page:thm2proof} below.

\ble[Sharpness of the bound \cref{eq:PDEbound} when $\Aso = \Ast= I$]\label{lem:sharp}
There exist particular choices of  $f, \,\nso$, and $\nst$ (with $\nso\neq \nst$ both continuous) such that 
the corresponding solutions $u^{(1)}$ and $u^{(2)}$ of \cref{prob:edp} with $\Aso = \Ast= I$ exist, are unique, and satisfy
\beq\label{eq:sharp1}
\frac{\N{u^{(1)}-u^{(2)}}_{\HokD}
}{
\N{u^{(2)}}_{\HokD}
}
\sim 
\frac{\N{u^{(1)}-u^{(2)}}_{L^2(D)}
}{
\N{u^{(2)}}_{L^2(D)}
}\sim k \NLiDRR{\nso-\nst}.
\eeq
%\noi (ii) There exist $f, \Aso, \Ast$, (with $\Aso\not\equiv \Ast$), such that 
%the corresponding solutions $u^{(1)}$ and $u^{(2)}$ of the exterior Dirichlet problem with $\nso \equiv \nst\equiv 1$ exist, are unique, and satisfy
%%There exist $f\in L^2(D), \Aj \in C^{0,1}(D)$, $j=1,2$ (with $\Aso\not\equiv \Ast$), such that the corresponding solutions $u^{(1)}$ and $u^{(2)}$ of the exterior Dirichlet problem with $\nso \equiv \nst\equiv 1$ satisfy
%\beq\label{eq:sharp2}
%\frac{\N{u^{(1)}-u^{(2)}}_{\HokD}
%}{
%\N{u^{(2)}}_{\HokD}
%}
%\sim 
%\frac{\N{u^{(1)}-u^{(2)}}_{L^2(D)}
%}{
%\N{u^{(2)}}_{L^2(D)}
%}\sim k \big\|\Aso-\Ast\big\|_{L^\infty(D)}.
%\eeq
\ele

The proof of \cref{lem:sharp} is on \cpageref{page:lemsharpproof} below.

\bre[Physical interpretation for $k$-dependence]\label{rem:physical1k}
It is perhaps unsurprising that the condition \cref{eq:sufficientlysmall} is a sufficient condition to answer both \cref{it:nbpcq1} and \cref{it:nbpcq2}. Recall that $1/k$ is proportional to the wavelength $2\pi/k$ of the wave $u$ (at least when $A=I$ and $n=1$). As the wavelength is the natural length scale associated with the wave $u$, one expects perturbations of magnitude up to $\cO(1/k)$ to be `unseen' by the PDE or numerical method. This is exactly what we see; perturbations of size (up to) $1/k$ give bounded relative difference (in \cref{it:nbpcq2}) and bounded GMRES iterations for the nearby-preconditioned linear system (in \cref{it:nbpcq1}). Also, on a PDE level, perturbations of order $1/k$ being `unseen' by the PDE can also be seen in bounds proved for $u$ where $n = \no + \eta,$ with $\no$ nontrapping and $\NLiDRR{\eta} \lesssim 1/k,$ see \cref{rem:kdep} above.
\ere


\section[Numerical experiments investigating sharpness]{Numerical experiments investigating the sharpness of \cref{thm:1,cor:1}}\label{sec:num}

The numerical experiments in this section seek to verify \cref{thm:1,cor:1} for \cref{prob:vtedp}, and investigate their sharpness. More specifically, the experiments seek to verify whether the condition \cref{eq:cond} is:
\ben
\item sufficient, and
\item necessary
  \een
  for \emph{standard} GMRES applied to \cref{eq:pcsystem1} to converge in a number of iterations that is independent of $k.$

Based on the PDE results \cref{thm:2,lem:sharp} above, we expect that the condition \cref{eq:sufficientlysmall} is a necessary and sufficient condition for standard GMRES applied to \cref{eq:pcsystem1} to converge in a $k$-independent number of iterations, even though we can only prove this is a sufficient condition for \emph{weighted} GMRES. We expect this because \cref{eq:sufficientlysmall} is a sufficient condition for \cref{it:nbpcq2}, the PDE analogue of \cref{it:nbpcq1}. Indeed, this is exactly the behaviour we observe in numerical experiments; we see that if \cref{eq:sufficientlysmall} holds, then standard GMRES applied to \cref{eq:pcsystem1} converges in a $k$-independent number of iterations, and moreover, \cref{eq:sufficientlysmall} may be sharp. We now describe our numerical experiments in more detail.

To verify this expected behaviour, we perform numerical experiments with the setup described in \cref{app:compsetup} with $\Aso = I$ and $\nso = 1$. We define $f$ and $\gI$ to correspond to a plane wave incident from the bottom left passing through a homogeneous medium given by coefficients $\Aso$ and $\nso$. We perform experiments for $A$ and $n$ separately, i.e., first we perform experiments with $\Ast=I$ and $\nst$ varying, and then we perform experiments with $\Ast$ varying and $\nst=1.$

We define $\Ast$ and $\nst$ to be piecewise constant (matrix-valued and real-valued respectively) on a $10\times10$ square grid, with their values on each square chosen independently at random from a $\Unif\mleft(1-\alpha,1+\alpha\mright)$ distribution, with $\alpha \in (0,1)$ chosen as described below. For $\Ast,$ we impose the restriction that on each square $\Ast$ is positive-definite almost surely. We solve the linear systems \cref{eq:pcsystem1} for $k = 20,40,60,80,100$ using standard GMRES and record the number of GMRES iterations taken to achieve a (relative) tolerance of $10^{-5}$ (relative to $\Nt{\bfb}$).

We perform experiments taking $\alpha = 0.5 \times k^{-\beta}$ for $\beta \in 0,0.1,\ldots,0.9,1.$ We expect that when $\beta \neq 1$  the number of GMRES iterations required for convergence will increase as $k$ increases, whereas we expect that when $\beta = 1$ the number of GMRES iterations required for convergence will remain bounded as $k$ increases, even though this behaviour for $\beta=1$ has only been proved for $\NLiDRRdtd{\Aso-\Ast}$ for weighted GMRES (compare the restrictions on $\NLiDRRdtd{\Aso-\Ast}$ in \cref{cor:1,cor:1a}).


In \cref{fig:linfinityA0,fig:linfinityA1,fig:linfinityA2,fig:linfinityn0,fig:linfinityn1,fig:linfinityn2}, when $\beta \in \set{0,\ldots,0.3}$ (for $\NLiDRRdtd{\Aso-\Ast}$) and $\beta \in \set{0,\ldots,0.5}$ (for $\NLiDRR{\nso-\nst}$) we see growth in the maximum number of GMRES iterations needed (over all realisations) to achieve convergence, otherwise we see that the number of GMRES iterations is bounded as $k$ increases. This behaviour is better than expected; as the number of GMRES iterations is apparently bounded for a range of $\beta < 1.$ However, we note that this behaviour could be (i) because we are in a pre-asymptotic regime, and the number of GMRES iterations would grow if we increased $k$ further, or (ii) the particular structure of $\nst$ (being piecewise constant, with the pieces independently, randomly chosen) could result in some kind of `averaging' behaviour, meaning the preconditioner is better than would otherwise be expected. However, we do not investigate these issues further in this thesis.

    \begin{figure}
      \centering
%% Creator: Matplotlib, PGF backend
%%
%% To include the figure in your LaTeX document, write
%%   \input{<filename>.pgf}
%%
%% Make sure the required packages are loaded in your preamble
%%   \usepackage{pgf}
%%
%% Figures using additional raster images can only be included by \input if
%% they are in the same directory as the main LaTeX file. For loading figures
%% from other directories you can use the `import` package
%%   \usepackage{import}
%% and then include the figures with
%%   \import{<path to file>}{<filename>.pgf}
%%
%% Matplotlib used the following preamble
%%   \usepackage{fontspec}
%%   \setmainfont{DejaVuSerif.ttf}[Path=/home/owen/progs/firedrake-complex/firedrake/lib/python3.5/site-packages/matplotlib/mpl-data/fonts/ttf/]
%%   \setsansfont{DejaVuSans.ttf}[Path=/home/owen/progs/firedrake-complex/firedrake/lib/python3.5/site-packages/matplotlib/mpl-data/fonts/ttf/]
%%   \setmonofont{DejaVuSansMono.ttf}[Path=/home/owen/progs/firedrake-complex/firedrake/lib/python3.5/site-packages/matplotlib/mpl-data/fonts/ttf/]
%%
\begingroup%
\makeatletter%
\begin{pgfpicture}%
\pgfpathrectangle{\pgfpointorigin}{\pgfqpoint{5.000000in}{2.500000in}}%
\pgfusepath{use as bounding box, clip}%
\begin{pgfscope}%
\pgfsetbuttcap%
\pgfsetmiterjoin%
\definecolor{currentfill}{rgb}{1.000000,1.000000,1.000000}%
\pgfsetfillcolor{currentfill}%
\pgfsetlinewidth{0.000000pt}%
\definecolor{currentstroke}{rgb}{1.000000,1.000000,1.000000}%
\pgfsetstrokecolor{currentstroke}%
\pgfsetdash{}{0pt}%
\pgfpathmoveto{\pgfqpoint{0.000000in}{0.000000in}}%
\pgfpathlineto{\pgfqpoint{5.000000in}{0.000000in}}%
\pgfpathlineto{\pgfqpoint{5.000000in}{2.500000in}}%
\pgfpathlineto{\pgfqpoint{0.000000in}{2.500000in}}%
\pgfpathclose%
\pgfusepath{fill}%
\end{pgfscope}%
\begin{pgfscope}%
\pgfsetbuttcap%
\pgfsetmiterjoin%
\definecolor{currentfill}{rgb}{1.000000,1.000000,1.000000}%
\pgfsetfillcolor{currentfill}%
\pgfsetlinewidth{0.000000pt}%
\definecolor{currentstroke}{rgb}{0.000000,0.000000,0.000000}%
\pgfsetstrokecolor{currentstroke}%
\pgfsetstrokeopacity{0.000000}%
\pgfsetdash{}{0pt}%
\pgfpathmoveto{\pgfqpoint{0.625000in}{0.275000in}}%
\pgfpathlineto{\pgfqpoint{4.500000in}{0.275000in}}%
\pgfpathlineto{\pgfqpoint{4.500000in}{2.200000in}}%
\pgfpathlineto{\pgfqpoint{0.625000in}{2.200000in}}%
\pgfpathclose%
\pgfusepath{fill}%
\end{pgfscope}%
\begin{pgfscope}%
\pgfsetbuttcap%
\pgfsetroundjoin%
\definecolor{currentfill}{rgb}{0.000000,0.000000,0.000000}%
\pgfsetfillcolor{currentfill}%
\pgfsetlinewidth{0.803000pt}%
\definecolor{currentstroke}{rgb}{0.000000,0.000000,0.000000}%
\pgfsetstrokecolor{currentstroke}%
\pgfsetdash{}{0pt}%
\pgfsys@defobject{currentmarker}{\pgfqpoint{0.000000in}{-0.048611in}}{\pgfqpoint{0.000000in}{0.000000in}}{%
\pgfpathmoveto{\pgfqpoint{0.000000in}{0.000000in}}%
\pgfpathlineto{\pgfqpoint{0.000000in}{-0.048611in}}%
\pgfusepath{stroke,fill}%
}%
\begin{pgfscope}%
\pgfsys@transformshift{0.801136in}{0.275000in}%
\pgfsys@useobject{currentmarker}{}%
\end{pgfscope}%
\end{pgfscope}%
\begin{pgfscope}%
\definecolor{textcolor}{rgb}{0.000000,0.000000,0.000000}%
\pgfsetstrokecolor{textcolor}%
\pgfsetfillcolor{textcolor}%
\pgftext[x=0.801136in,y=0.177778in,,top]{\color{textcolor}\sffamily\fontsize{10.000000}{12.000000}\selectfont 20}%
\end{pgfscope}%
\begin{pgfscope}%
\pgfsetbuttcap%
\pgfsetroundjoin%
\definecolor{currentfill}{rgb}{0.000000,0.000000,0.000000}%
\pgfsetfillcolor{currentfill}%
\pgfsetlinewidth{0.803000pt}%
\definecolor{currentstroke}{rgb}{0.000000,0.000000,0.000000}%
\pgfsetstrokecolor{currentstroke}%
\pgfsetdash{}{0pt}%
\pgfsys@defobject{currentmarker}{\pgfqpoint{0.000000in}{-0.048611in}}{\pgfqpoint{0.000000in}{0.000000in}}{%
\pgfpathmoveto{\pgfqpoint{0.000000in}{0.000000in}}%
\pgfpathlineto{\pgfqpoint{0.000000in}{-0.048611in}}%
\pgfusepath{stroke,fill}%
}%
\begin{pgfscope}%
\pgfsys@transformshift{1.975379in}{0.275000in}%
\pgfsys@useobject{currentmarker}{}%
\end{pgfscope}%
\end{pgfscope}%
\begin{pgfscope}%
\definecolor{textcolor}{rgb}{0.000000,0.000000,0.000000}%
\pgfsetstrokecolor{textcolor}%
\pgfsetfillcolor{textcolor}%
\pgftext[x=1.975379in,y=0.177778in,,top]{\color{textcolor}\sffamily\fontsize{10.000000}{12.000000}\selectfont 40}%
\end{pgfscope}%
\begin{pgfscope}%
\pgfsetbuttcap%
\pgfsetroundjoin%
\definecolor{currentfill}{rgb}{0.000000,0.000000,0.000000}%
\pgfsetfillcolor{currentfill}%
\pgfsetlinewidth{0.803000pt}%
\definecolor{currentstroke}{rgb}{0.000000,0.000000,0.000000}%
\pgfsetstrokecolor{currentstroke}%
\pgfsetdash{}{0pt}%
\pgfsys@defobject{currentmarker}{\pgfqpoint{0.000000in}{-0.048611in}}{\pgfqpoint{0.000000in}{0.000000in}}{%
\pgfpathmoveto{\pgfqpoint{0.000000in}{0.000000in}}%
\pgfpathlineto{\pgfqpoint{0.000000in}{-0.048611in}}%
\pgfusepath{stroke,fill}%
}%
\begin{pgfscope}%
\pgfsys@transformshift{3.149621in}{0.275000in}%
\pgfsys@useobject{currentmarker}{}%
\end{pgfscope}%
\end{pgfscope}%
\begin{pgfscope}%
\definecolor{textcolor}{rgb}{0.000000,0.000000,0.000000}%
\pgfsetstrokecolor{textcolor}%
\pgfsetfillcolor{textcolor}%
\pgftext[x=3.149621in,y=0.177778in,,top]{\color{textcolor}\sffamily\fontsize{10.000000}{12.000000}\selectfont 60}%
\end{pgfscope}%
\begin{pgfscope}%
\pgfsetbuttcap%
\pgfsetroundjoin%
\definecolor{currentfill}{rgb}{0.000000,0.000000,0.000000}%
\pgfsetfillcolor{currentfill}%
\pgfsetlinewidth{0.803000pt}%
\definecolor{currentstroke}{rgb}{0.000000,0.000000,0.000000}%
\pgfsetstrokecolor{currentstroke}%
\pgfsetdash{}{0pt}%
\pgfsys@defobject{currentmarker}{\pgfqpoint{0.000000in}{-0.048611in}}{\pgfqpoint{0.000000in}{0.000000in}}{%
\pgfpathmoveto{\pgfqpoint{0.000000in}{0.000000in}}%
\pgfpathlineto{\pgfqpoint{0.000000in}{-0.048611in}}%
\pgfusepath{stroke,fill}%
}%
\begin{pgfscope}%
\pgfsys@transformshift{4.323864in}{0.275000in}%
\pgfsys@useobject{currentmarker}{}%
\end{pgfscope}%
\end{pgfscope}%
\begin{pgfscope}%
\definecolor{textcolor}{rgb}{0.000000,0.000000,0.000000}%
\pgfsetstrokecolor{textcolor}%
\pgfsetfillcolor{textcolor}%
\pgftext[x=4.323864in,y=0.177778in,,top]{\color{textcolor}\sffamily\fontsize{10.000000}{12.000000}\selectfont 80}%
\end{pgfscope}%
\begin{pgfscope}%
\definecolor{textcolor}{rgb}{0.000000,0.000000,0.000000}%
\pgfsetstrokecolor{textcolor}%
\pgfsetfillcolor{textcolor}%
\pgftext[x=2.562500in,y=-0.012191in,,top]{\color{textcolor}\sffamily\fontsize{10.000000}{12.000000}\selectfont \(\displaystyle k\)}%
\end{pgfscope}%
\begin{pgfscope}%
\pgfsetbuttcap%
\pgfsetroundjoin%
\definecolor{currentfill}{rgb}{0.000000,0.000000,0.000000}%
\pgfsetfillcolor{currentfill}%
\pgfsetlinewidth{0.803000pt}%
\definecolor{currentstroke}{rgb}{0.000000,0.000000,0.000000}%
\pgfsetstrokecolor{currentstroke}%
\pgfsetdash{}{0pt}%
\pgfsys@defobject{currentmarker}{\pgfqpoint{-0.048611in}{0.000000in}}{\pgfqpoint{0.000000in}{0.000000in}}{%
\pgfpathmoveto{\pgfqpoint{0.000000in}{0.000000in}}%
\pgfpathlineto{\pgfqpoint{-0.048611in}{0.000000in}}%
\pgfusepath{stroke,fill}%
}%
\begin{pgfscope}%
\pgfsys@transformshift{0.625000in}{0.312642in}%
\pgfsys@useobject{currentmarker}{}%
\end{pgfscope}%
\end{pgfscope}%
\begin{pgfscope}%
\definecolor{textcolor}{rgb}{0.000000,0.000000,0.000000}%
\pgfsetstrokecolor{textcolor}%
\pgfsetfillcolor{textcolor}%
\pgftext[x=0.439412in,y=0.259881in,left,base]{\color{textcolor}\sffamily\fontsize{10.000000}{12.000000}\selectfont 0}%
\end{pgfscope}%
\begin{pgfscope}%
\pgfsetbuttcap%
\pgfsetroundjoin%
\definecolor{currentfill}{rgb}{0.000000,0.000000,0.000000}%
\pgfsetfillcolor{currentfill}%
\pgfsetlinewidth{0.803000pt}%
\definecolor{currentstroke}{rgb}{0.000000,0.000000,0.000000}%
\pgfsetstrokecolor{currentstroke}%
\pgfsetdash{}{0pt}%
\pgfsys@defobject{currentmarker}{\pgfqpoint{-0.048611in}{0.000000in}}{\pgfqpoint{0.000000in}{0.000000in}}{%
\pgfpathmoveto{\pgfqpoint{0.000000in}{0.000000in}}%
\pgfpathlineto{\pgfqpoint{-0.048611in}{0.000000in}}%
\pgfusepath{stroke,fill}%
}%
\begin{pgfscope}%
\pgfsys@transformshift{0.625000in}{0.512073in}%
\pgfsys@useobject{currentmarker}{}%
\end{pgfscope}%
\end{pgfscope}%
\begin{pgfscope}%
\definecolor{textcolor}{rgb}{0.000000,0.000000,0.000000}%
\pgfsetstrokecolor{textcolor}%
\pgfsetfillcolor{textcolor}%
\pgftext[x=0.351047in,y=0.459311in,left,base]{\color{textcolor}\sffamily\fontsize{10.000000}{12.000000}\selectfont 40}%
\end{pgfscope}%
\begin{pgfscope}%
\pgfsetbuttcap%
\pgfsetroundjoin%
\definecolor{currentfill}{rgb}{0.000000,0.000000,0.000000}%
\pgfsetfillcolor{currentfill}%
\pgfsetlinewidth{0.803000pt}%
\definecolor{currentstroke}{rgb}{0.000000,0.000000,0.000000}%
\pgfsetstrokecolor{currentstroke}%
\pgfsetdash{}{0pt}%
\pgfsys@defobject{currentmarker}{\pgfqpoint{-0.048611in}{0.000000in}}{\pgfqpoint{0.000000in}{0.000000in}}{%
\pgfpathmoveto{\pgfqpoint{0.000000in}{0.000000in}}%
\pgfpathlineto{\pgfqpoint{-0.048611in}{0.000000in}}%
\pgfusepath{stroke,fill}%
}%
\begin{pgfscope}%
\pgfsys@transformshift{0.625000in}{0.711503in}%
\pgfsys@useobject{currentmarker}{}%
\end{pgfscope}%
\end{pgfscope}%
\begin{pgfscope}%
\definecolor{textcolor}{rgb}{0.000000,0.000000,0.000000}%
\pgfsetstrokecolor{textcolor}%
\pgfsetfillcolor{textcolor}%
\pgftext[x=0.351047in,y=0.658741in,left,base]{\color{textcolor}\sffamily\fontsize{10.000000}{12.000000}\selectfont 80}%
\end{pgfscope}%
\begin{pgfscope}%
\pgfsetbuttcap%
\pgfsetroundjoin%
\definecolor{currentfill}{rgb}{0.000000,0.000000,0.000000}%
\pgfsetfillcolor{currentfill}%
\pgfsetlinewidth{0.803000pt}%
\definecolor{currentstroke}{rgb}{0.000000,0.000000,0.000000}%
\pgfsetstrokecolor{currentstroke}%
\pgfsetdash{}{0pt}%
\pgfsys@defobject{currentmarker}{\pgfqpoint{-0.048611in}{0.000000in}}{\pgfqpoint{0.000000in}{0.000000in}}{%
\pgfpathmoveto{\pgfqpoint{0.000000in}{0.000000in}}%
\pgfpathlineto{\pgfqpoint{-0.048611in}{0.000000in}}%
\pgfusepath{stroke,fill}%
}%
\begin{pgfscope}%
\pgfsys@transformshift{0.625000in}{0.910933in}%
\pgfsys@useobject{currentmarker}{}%
\end{pgfscope}%
\end{pgfscope}%
\begin{pgfscope}%
\definecolor{textcolor}{rgb}{0.000000,0.000000,0.000000}%
\pgfsetstrokecolor{textcolor}%
\pgfsetfillcolor{textcolor}%
\pgftext[x=0.262682in,y=0.858172in,left,base]{\color{textcolor}\sffamily\fontsize{10.000000}{12.000000}\selectfont 120}%
\end{pgfscope}%
\begin{pgfscope}%
\pgfsetbuttcap%
\pgfsetroundjoin%
\definecolor{currentfill}{rgb}{0.000000,0.000000,0.000000}%
\pgfsetfillcolor{currentfill}%
\pgfsetlinewidth{0.803000pt}%
\definecolor{currentstroke}{rgb}{0.000000,0.000000,0.000000}%
\pgfsetstrokecolor{currentstroke}%
\pgfsetdash{}{0pt}%
\pgfsys@defobject{currentmarker}{\pgfqpoint{-0.048611in}{0.000000in}}{\pgfqpoint{0.000000in}{0.000000in}}{%
\pgfpathmoveto{\pgfqpoint{0.000000in}{0.000000in}}%
\pgfpathlineto{\pgfqpoint{-0.048611in}{0.000000in}}%
\pgfusepath{stroke,fill}%
}%
\begin{pgfscope}%
\pgfsys@transformshift{0.625000in}{1.110363in}%
\pgfsys@useobject{currentmarker}{}%
\end{pgfscope}%
\end{pgfscope}%
\begin{pgfscope}%
\definecolor{textcolor}{rgb}{0.000000,0.000000,0.000000}%
\pgfsetstrokecolor{textcolor}%
\pgfsetfillcolor{textcolor}%
\pgftext[x=0.262682in,y=1.057602in,left,base]{\color{textcolor}\sffamily\fontsize{10.000000}{12.000000}\selectfont 160}%
\end{pgfscope}%
\begin{pgfscope}%
\pgfsetbuttcap%
\pgfsetroundjoin%
\definecolor{currentfill}{rgb}{0.000000,0.000000,0.000000}%
\pgfsetfillcolor{currentfill}%
\pgfsetlinewidth{0.803000pt}%
\definecolor{currentstroke}{rgb}{0.000000,0.000000,0.000000}%
\pgfsetstrokecolor{currentstroke}%
\pgfsetdash{}{0pt}%
\pgfsys@defobject{currentmarker}{\pgfqpoint{-0.048611in}{0.000000in}}{\pgfqpoint{0.000000in}{0.000000in}}{%
\pgfpathmoveto{\pgfqpoint{0.000000in}{0.000000in}}%
\pgfpathlineto{\pgfqpoint{-0.048611in}{0.000000in}}%
\pgfusepath{stroke,fill}%
}%
\begin{pgfscope}%
\pgfsys@transformshift{0.625000in}{1.309793in}%
\pgfsys@useobject{currentmarker}{}%
\end{pgfscope}%
\end{pgfscope}%
\begin{pgfscope}%
\definecolor{textcolor}{rgb}{0.000000,0.000000,0.000000}%
\pgfsetstrokecolor{textcolor}%
\pgfsetfillcolor{textcolor}%
\pgftext[x=0.262682in,y=1.257032in,left,base]{\color{textcolor}\sffamily\fontsize{10.000000}{12.000000}\selectfont 200}%
\end{pgfscope}%
\begin{pgfscope}%
\pgfsetbuttcap%
\pgfsetroundjoin%
\definecolor{currentfill}{rgb}{0.000000,0.000000,0.000000}%
\pgfsetfillcolor{currentfill}%
\pgfsetlinewidth{0.803000pt}%
\definecolor{currentstroke}{rgb}{0.000000,0.000000,0.000000}%
\pgfsetstrokecolor{currentstroke}%
\pgfsetdash{}{0pt}%
\pgfsys@defobject{currentmarker}{\pgfqpoint{-0.048611in}{0.000000in}}{\pgfqpoint{0.000000in}{0.000000in}}{%
\pgfpathmoveto{\pgfqpoint{0.000000in}{0.000000in}}%
\pgfpathlineto{\pgfqpoint{-0.048611in}{0.000000in}}%
\pgfusepath{stroke,fill}%
}%
\begin{pgfscope}%
\pgfsys@transformshift{0.625000in}{1.509224in}%
\pgfsys@useobject{currentmarker}{}%
\end{pgfscope}%
\end{pgfscope}%
\begin{pgfscope}%
\definecolor{textcolor}{rgb}{0.000000,0.000000,0.000000}%
\pgfsetstrokecolor{textcolor}%
\pgfsetfillcolor{textcolor}%
\pgftext[x=0.262682in,y=1.456462in,left,base]{\color{textcolor}\sffamily\fontsize{10.000000}{12.000000}\selectfont 240}%
\end{pgfscope}%
\begin{pgfscope}%
\pgfsetbuttcap%
\pgfsetroundjoin%
\definecolor{currentfill}{rgb}{0.000000,0.000000,0.000000}%
\pgfsetfillcolor{currentfill}%
\pgfsetlinewidth{0.803000pt}%
\definecolor{currentstroke}{rgb}{0.000000,0.000000,0.000000}%
\pgfsetstrokecolor{currentstroke}%
\pgfsetdash{}{0pt}%
\pgfsys@defobject{currentmarker}{\pgfqpoint{-0.048611in}{0.000000in}}{\pgfqpoint{0.000000in}{0.000000in}}{%
\pgfpathmoveto{\pgfqpoint{0.000000in}{0.000000in}}%
\pgfpathlineto{\pgfqpoint{-0.048611in}{0.000000in}}%
\pgfusepath{stroke,fill}%
}%
\begin{pgfscope}%
\pgfsys@transformshift{0.625000in}{1.708654in}%
\pgfsys@useobject{currentmarker}{}%
\end{pgfscope}%
\end{pgfscope}%
\begin{pgfscope}%
\definecolor{textcolor}{rgb}{0.000000,0.000000,0.000000}%
\pgfsetstrokecolor{textcolor}%
\pgfsetfillcolor{textcolor}%
\pgftext[x=0.262682in,y=1.655892in,left,base]{\color{textcolor}\sffamily\fontsize{10.000000}{12.000000}\selectfont 280}%
\end{pgfscope}%
\begin{pgfscope}%
\pgfsetbuttcap%
\pgfsetroundjoin%
\definecolor{currentfill}{rgb}{0.000000,0.000000,0.000000}%
\pgfsetfillcolor{currentfill}%
\pgfsetlinewidth{0.803000pt}%
\definecolor{currentstroke}{rgb}{0.000000,0.000000,0.000000}%
\pgfsetstrokecolor{currentstroke}%
\pgfsetdash{}{0pt}%
\pgfsys@defobject{currentmarker}{\pgfqpoint{-0.048611in}{0.000000in}}{\pgfqpoint{0.000000in}{0.000000in}}{%
\pgfpathmoveto{\pgfqpoint{0.000000in}{0.000000in}}%
\pgfpathlineto{\pgfqpoint{-0.048611in}{0.000000in}}%
\pgfusepath{stroke,fill}%
}%
\begin{pgfscope}%
\pgfsys@transformshift{0.625000in}{1.908084in}%
\pgfsys@useobject{currentmarker}{}%
\end{pgfscope}%
\end{pgfscope}%
\begin{pgfscope}%
\definecolor{textcolor}{rgb}{0.000000,0.000000,0.000000}%
\pgfsetstrokecolor{textcolor}%
\pgfsetfillcolor{textcolor}%
\pgftext[x=0.262682in,y=1.855323in,left,base]{\color{textcolor}\sffamily\fontsize{10.000000}{12.000000}\selectfont 320}%
\end{pgfscope}%
\begin{pgfscope}%
\pgfsetbuttcap%
\pgfsetroundjoin%
\definecolor{currentfill}{rgb}{0.000000,0.000000,0.000000}%
\pgfsetfillcolor{currentfill}%
\pgfsetlinewidth{0.803000pt}%
\definecolor{currentstroke}{rgb}{0.000000,0.000000,0.000000}%
\pgfsetstrokecolor{currentstroke}%
\pgfsetdash{}{0pt}%
\pgfsys@defobject{currentmarker}{\pgfqpoint{-0.048611in}{0.000000in}}{\pgfqpoint{0.000000in}{0.000000in}}{%
\pgfpathmoveto{\pgfqpoint{0.000000in}{0.000000in}}%
\pgfpathlineto{\pgfqpoint{-0.048611in}{0.000000in}}%
\pgfusepath{stroke,fill}%
}%
\begin{pgfscope}%
\pgfsys@transformshift{0.625000in}{2.107514in}%
\pgfsys@useobject{currentmarker}{}%
\end{pgfscope}%
\end{pgfscope}%
\begin{pgfscope}%
\definecolor{textcolor}{rgb}{0.000000,0.000000,0.000000}%
\pgfsetstrokecolor{textcolor}%
\pgfsetfillcolor{textcolor}%
\pgftext[x=0.262682in,y=2.054753in,left,base]{\color{textcolor}\sffamily\fontsize{10.000000}{12.000000}\selectfont 360}%
\end{pgfscope}%
\begin{pgfscope}%
\definecolor{textcolor}{rgb}{0.000000,0.000000,0.000000}%
\pgfsetstrokecolor{textcolor}%
\pgfsetfillcolor{textcolor}%
\pgftext[x=0.207126in,y=1.237500in,,bottom,rotate=90.000000]{\color{textcolor}\sffamily\fontsize{10.000000}{12.000000}\selectfont Number of GMRES Iterations}%
\end{pgfscope}%
\begin{pgfscope}%
\pgfpathrectangle{\pgfqpoint{0.625000in}{0.275000in}}{\pgfqpoint{3.875000in}{1.925000in}}%
\pgfusepath{clip}%
\pgfsetbuttcap%
\pgfsetroundjoin%
\definecolor{currentfill}{rgb}{0.000000,0.000000,0.000000}%
\pgfsetfillcolor{currentfill}%
\pgfsetlinewidth{1.003750pt}%
\definecolor{currentstroke}{rgb}{0.000000,0.000000,0.000000}%
\pgfsetstrokecolor{currentstroke}%
\pgfsetdash{}{0pt}%
\pgfsys@defobject{currentmarker}{\pgfqpoint{-0.041667in}{-0.041667in}}{\pgfqpoint{0.041667in}{0.041667in}}{%
\pgfpathmoveto{\pgfqpoint{0.000000in}{-0.041667in}}%
\pgfpathcurveto{\pgfqpoint{0.011050in}{-0.041667in}}{\pgfqpoint{0.021649in}{-0.037276in}}{\pgfqpoint{0.029463in}{-0.029463in}}%
\pgfpathcurveto{\pgfqpoint{0.037276in}{-0.021649in}}{\pgfqpoint{0.041667in}{-0.011050in}}{\pgfqpoint{0.041667in}{0.000000in}}%
\pgfpathcurveto{\pgfqpoint{0.041667in}{0.011050in}}{\pgfqpoint{0.037276in}{0.021649in}}{\pgfqpoint{0.029463in}{0.029463in}}%
\pgfpathcurveto{\pgfqpoint{0.021649in}{0.037276in}}{\pgfqpoint{0.011050in}{0.041667in}}{\pgfqpoint{0.000000in}{0.041667in}}%
\pgfpathcurveto{\pgfqpoint{-0.011050in}{0.041667in}}{\pgfqpoint{-0.021649in}{0.037276in}}{\pgfqpoint{-0.029463in}{0.029463in}}%
\pgfpathcurveto{\pgfqpoint{-0.037276in}{0.021649in}}{\pgfqpoint{-0.041667in}{0.011050in}}{\pgfqpoint{-0.041667in}{0.000000in}}%
\pgfpathcurveto{\pgfqpoint{-0.041667in}{-0.011050in}}{\pgfqpoint{-0.037276in}{-0.021649in}}{\pgfqpoint{-0.029463in}{-0.029463in}}%
\pgfpathcurveto{\pgfqpoint{-0.021649in}{-0.037276in}}{\pgfqpoint{-0.011050in}{-0.041667in}}{\pgfqpoint{0.000000in}{-0.041667in}}%
\pgfpathclose%
\pgfusepath{stroke,fill}%
}%
\begin{pgfscope}%
\pgfsys@transformshift{0.801136in}{0.402386in}%
\pgfsys@useobject{currentmarker}{}%
\end{pgfscope}%
\begin{pgfscope}%
\pgfsys@transformshift{0.801136in}{0.407372in}%
\pgfsys@useobject{currentmarker}{}%
\end{pgfscope}%
\begin{pgfscope}%
\pgfsys@transformshift{0.801136in}{0.412358in}%
\pgfsys@useobject{currentmarker}{}%
\end{pgfscope}%
\begin{pgfscope}%
\pgfsys@transformshift{0.801136in}{0.417343in}%
\pgfsys@useobject{currentmarker}{}%
\end{pgfscope}%
\begin{pgfscope}%
\pgfsys@transformshift{0.801136in}{0.422329in}%
\pgfsys@useobject{currentmarker}{}%
\end{pgfscope}%
\begin{pgfscope}%
\pgfsys@transformshift{0.801136in}{0.427315in}%
\pgfsys@useobject{currentmarker}{}%
\end{pgfscope}%
\begin{pgfscope}%
\pgfsys@transformshift{0.801136in}{0.432301in}%
\pgfsys@useobject{currentmarker}{}%
\end{pgfscope}%
\begin{pgfscope}%
\pgfsys@transformshift{0.801136in}{0.437286in}%
\pgfsys@useobject{currentmarker}{}%
\end{pgfscope}%
\end{pgfscope}%
\begin{pgfscope}%
\pgfpathrectangle{\pgfqpoint{0.625000in}{0.275000in}}{\pgfqpoint{3.875000in}{1.925000in}}%
\pgfusepath{clip}%
\pgfsetbuttcap%
\pgfsetroundjoin%
\definecolor{currentfill}{rgb}{0.000000,0.000000,0.000000}%
\pgfsetfillcolor{currentfill}%
\pgfsetlinewidth{1.003750pt}%
\definecolor{currentstroke}{rgb}{0.000000,0.000000,0.000000}%
\pgfsetstrokecolor{currentstroke}%
\pgfsetdash{}{0pt}%
\pgfsys@defobject{currentmarker}{\pgfqpoint{-0.041667in}{-0.041667in}}{\pgfqpoint{0.041667in}{0.041667in}}{%
\pgfpathmoveto{\pgfqpoint{0.000000in}{-0.041667in}}%
\pgfpathcurveto{\pgfqpoint{0.011050in}{-0.041667in}}{\pgfqpoint{0.021649in}{-0.037276in}}{\pgfqpoint{0.029463in}{-0.029463in}}%
\pgfpathcurveto{\pgfqpoint{0.037276in}{-0.021649in}}{\pgfqpoint{0.041667in}{-0.011050in}}{\pgfqpoint{0.041667in}{0.000000in}}%
\pgfpathcurveto{\pgfqpoint{0.041667in}{0.011050in}}{\pgfqpoint{0.037276in}{0.021649in}}{\pgfqpoint{0.029463in}{0.029463in}}%
\pgfpathcurveto{\pgfqpoint{0.021649in}{0.037276in}}{\pgfqpoint{0.011050in}{0.041667in}}{\pgfqpoint{0.000000in}{0.041667in}}%
\pgfpathcurveto{\pgfqpoint{-0.011050in}{0.041667in}}{\pgfqpoint{-0.021649in}{0.037276in}}{\pgfqpoint{-0.029463in}{0.029463in}}%
\pgfpathcurveto{\pgfqpoint{-0.037276in}{0.021649in}}{\pgfqpoint{-0.041667in}{0.011050in}}{\pgfqpoint{-0.041667in}{0.000000in}}%
\pgfpathcurveto{\pgfqpoint{-0.041667in}{-0.011050in}}{\pgfqpoint{-0.037276in}{-0.021649in}}{\pgfqpoint{-0.029463in}{-0.029463in}}%
\pgfpathcurveto{\pgfqpoint{-0.021649in}{-0.037276in}}{\pgfqpoint{-0.011050in}{-0.041667in}}{\pgfqpoint{0.000000in}{-0.041667in}}%
\pgfpathclose%
\pgfusepath{stroke,fill}%
}%
\begin{pgfscope}%
\pgfsys@transformshift{1.975379in}{0.517058in}%
\pgfsys@useobject{currentmarker}{}%
\end{pgfscope}%
\begin{pgfscope}%
\pgfsys@transformshift{1.975379in}{0.537001in}%
\pgfsys@useobject{currentmarker}{}%
\end{pgfscope}%
\begin{pgfscope}%
\pgfsys@transformshift{1.975379in}{0.546973in}%
\pgfsys@useobject{currentmarker}{}%
\end{pgfscope}%
\begin{pgfscope}%
\pgfsys@transformshift{1.975379in}{0.551959in}%
\pgfsys@useobject{currentmarker}{}%
\end{pgfscope}%
\begin{pgfscope}%
\pgfsys@transformshift{1.975379in}{0.556944in}%
\pgfsys@useobject{currentmarker}{}%
\end{pgfscope}%
\begin{pgfscope}%
\pgfsys@transformshift{1.975379in}{0.561930in}%
\pgfsys@useobject{currentmarker}{}%
\end{pgfscope}%
\begin{pgfscope}%
\pgfsys@transformshift{1.975379in}{0.566916in}%
\pgfsys@useobject{currentmarker}{}%
\end{pgfscope}%
\begin{pgfscope}%
\pgfsys@transformshift{1.975379in}{0.571902in}%
\pgfsys@useobject{currentmarker}{}%
\end{pgfscope}%
\begin{pgfscope}%
\pgfsys@transformshift{1.975379in}{0.576887in}%
\pgfsys@useobject{currentmarker}{}%
\end{pgfscope}%
\begin{pgfscope}%
\pgfsys@transformshift{1.975379in}{0.581873in}%
\pgfsys@useobject{currentmarker}{}%
\end{pgfscope}%
\begin{pgfscope}%
\pgfsys@transformshift{1.975379in}{0.586859in}%
\pgfsys@useobject{currentmarker}{}%
\end{pgfscope}%
\begin{pgfscope}%
\pgfsys@transformshift{1.975379in}{0.591845in}%
\pgfsys@useobject{currentmarker}{}%
\end{pgfscope}%
\begin{pgfscope}%
\pgfsys@transformshift{1.975379in}{0.596830in}%
\pgfsys@useobject{currentmarker}{}%
\end{pgfscope}%
\begin{pgfscope}%
\pgfsys@transformshift{1.975379in}{0.601816in}%
\pgfsys@useobject{currentmarker}{}%
\end{pgfscope}%
\begin{pgfscope}%
\pgfsys@transformshift{1.975379in}{0.606802in}%
\pgfsys@useobject{currentmarker}{}%
\end{pgfscope}%
\begin{pgfscope}%
\pgfsys@transformshift{1.975379in}{0.611788in}%
\pgfsys@useobject{currentmarker}{}%
\end{pgfscope}%
\begin{pgfscope}%
\pgfsys@transformshift{1.975379in}{0.616774in}%
\pgfsys@useobject{currentmarker}{}%
\end{pgfscope}%
\begin{pgfscope}%
\pgfsys@transformshift{1.975379in}{0.621759in}%
\pgfsys@useobject{currentmarker}{}%
\end{pgfscope}%
\begin{pgfscope}%
\pgfsys@transformshift{1.975379in}{0.626745in}%
\pgfsys@useobject{currentmarker}{}%
\end{pgfscope}%
\begin{pgfscope}%
\pgfsys@transformshift{1.975379in}{0.631731in}%
\pgfsys@useobject{currentmarker}{}%
\end{pgfscope}%
\begin{pgfscope}%
\pgfsys@transformshift{1.975379in}{0.636717in}%
\pgfsys@useobject{currentmarker}{}%
\end{pgfscope}%
\begin{pgfscope}%
\pgfsys@transformshift{1.975379in}{0.641702in}%
\pgfsys@useobject{currentmarker}{}%
\end{pgfscope}%
\begin{pgfscope}%
\pgfsys@transformshift{1.975379in}{0.646688in}%
\pgfsys@useobject{currentmarker}{}%
\end{pgfscope}%
\begin{pgfscope}%
\pgfsys@transformshift{1.975379in}{0.651674in}%
\pgfsys@useobject{currentmarker}{}%
\end{pgfscope}%
\begin{pgfscope}%
\pgfsys@transformshift{1.975379in}{0.656660in}%
\pgfsys@useobject{currentmarker}{}%
\end{pgfscope}%
\begin{pgfscope}%
\pgfsys@transformshift{1.975379in}{0.661645in}%
\pgfsys@useobject{currentmarker}{}%
\end{pgfscope}%
\begin{pgfscope}%
\pgfsys@transformshift{1.975379in}{0.686574in}%
\pgfsys@useobject{currentmarker}{}%
\end{pgfscope}%
\begin{pgfscope}%
\pgfsys@transformshift{1.975379in}{0.716489in}%
\pgfsys@useobject{currentmarker}{}%
\end{pgfscope}%
\end{pgfscope}%
\begin{pgfscope}%
\pgfpathrectangle{\pgfqpoint{0.625000in}{0.275000in}}{\pgfqpoint{3.875000in}{1.925000in}}%
\pgfusepath{clip}%
\pgfsetbuttcap%
\pgfsetroundjoin%
\definecolor{currentfill}{rgb}{0.000000,0.000000,0.000000}%
\pgfsetfillcolor{currentfill}%
\pgfsetlinewidth{1.003750pt}%
\definecolor{currentstroke}{rgb}{0.000000,0.000000,0.000000}%
\pgfsetstrokecolor{currentstroke}%
\pgfsetdash{}{0pt}%
\pgfsys@defobject{currentmarker}{\pgfqpoint{-0.041667in}{-0.041667in}}{\pgfqpoint{0.041667in}{0.041667in}}{%
\pgfpathmoveto{\pgfqpoint{0.000000in}{-0.041667in}}%
\pgfpathcurveto{\pgfqpoint{0.011050in}{-0.041667in}}{\pgfqpoint{0.021649in}{-0.037276in}}{\pgfqpoint{0.029463in}{-0.029463in}}%
\pgfpathcurveto{\pgfqpoint{0.037276in}{-0.021649in}}{\pgfqpoint{0.041667in}{-0.011050in}}{\pgfqpoint{0.041667in}{0.000000in}}%
\pgfpathcurveto{\pgfqpoint{0.041667in}{0.011050in}}{\pgfqpoint{0.037276in}{0.021649in}}{\pgfqpoint{0.029463in}{0.029463in}}%
\pgfpathcurveto{\pgfqpoint{0.021649in}{0.037276in}}{\pgfqpoint{0.011050in}{0.041667in}}{\pgfqpoint{0.000000in}{0.041667in}}%
\pgfpathcurveto{\pgfqpoint{-0.011050in}{0.041667in}}{\pgfqpoint{-0.021649in}{0.037276in}}{\pgfqpoint{-0.029463in}{0.029463in}}%
\pgfpathcurveto{\pgfqpoint{-0.037276in}{0.021649in}}{\pgfqpoint{-0.041667in}{0.011050in}}{\pgfqpoint{-0.041667in}{0.000000in}}%
\pgfpathcurveto{\pgfqpoint{-0.041667in}{-0.011050in}}{\pgfqpoint{-0.037276in}{-0.021649in}}{\pgfqpoint{-0.029463in}{-0.029463in}}%
\pgfpathcurveto{\pgfqpoint{-0.021649in}{-0.037276in}}{\pgfqpoint{-0.011050in}{-0.041667in}}{\pgfqpoint{0.000000in}{-0.041667in}}%
\pgfpathclose%
\pgfusepath{stroke,fill}%
}%
\begin{pgfscope}%
\pgfsys@transformshift{3.149621in}{0.846118in}%
\pgfsys@useobject{currentmarker}{}%
\end{pgfscope}%
\begin{pgfscope}%
\pgfsys@transformshift{3.149621in}{0.851104in}%
\pgfsys@useobject{currentmarker}{}%
\end{pgfscope}%
\begin{pgfscope}%
\pgfsys@transformshift{3.149621in}{0.856090in}%
\pgfsys@useobject{currentmarker}{}%
\end{pgfscope}%
\begin{pgfscope}%
\pgfsys@transformshift{3.149621in}{0.861075in}%
\pgfsys@useobject{currentmarker}{}%
\end{pgfscope}%
\begin{pgfscope}%
\pgfsys@transformshift{3.149621in}{0.866061in}%
\pgfsys@useobject{currentmarker}{}%
\end{pgfscope}%
\begin{pgfscope}%
\pgfsys@transformshift{3.149621in}{0.871047in}%
\pgfsys@useobject{currentmarker}{}%
\end{pgfscope}%
\begin{pgfscope}%
\pgfsys@transformshift{3.149621in}{0.876033in}%
\pgfsys@useobject{currentmarker}{}%
\end{pgfscope}%
\begin{pgfscope}%
\pgfsys@transformshift{3.149621in}{0.881019in}%
\pgfsys@useobject{currentmarker}{}%
\end{pgfscope}%
\begin{pgfscope}%
\pgfsys@transformshift{3.149621in}{0.886004in}%
\pgfsys@useobject{currentmarker}{}%
\end{pgfscope}%
\begin{pgfscope}%
\pgfsys@transformshift{3.149621in}{0.890990in}%
\pgfsys@useobject{currentmarker}{}%
\end{pgfscope}%
\begin{pgfscope}%
\pgfsys@transformshift{3.149621in}{0.895976in}%
\pgfsys@useobject{currentmarker}{}%
\end{pgfscope}%
\begin{pgfscope}%
\pgfsys@transformshift{3.149621in}{0.900962in}%
\pgfsys@useobject{currentmarker}{}%
\end{pgfscope}%
\begin{pgfscope}%
\pgfsys@transformshift{3.149621in}{0.905947in}%
\pgfsys@useobject{currentmarker}{}%
\end{pgfscope}%
\begin{pgfscope}%
\pgfsys@transformshift{3.149621in}{0.910933in}%
\pgfsys@useobject{currentmarker}{}%
\end{pgfscope}%
\begin{pgfscope}%
\pgfsys@transformshift{3.149621in}{0.920905in}%
\pgfsys@useobject{currentmarker}{}%
\end{pgfscope}%
\begin{pgfscope}%
\pgfsys@transformshift{3.149621in}{0.935862in}%
\pgfsys@useobject{currentmarker}{}%
\end{pgfscope}%
\begin{pgfscope}%
\pgfsys@transformshift{3.149621in}{0.940848in}%
\pgfsys@useobject{currentmarker}{}%
\end{pgfscope}%
\begin{pgfscope}%
\pgfsys@transformshift{3.149621in}{0.945833in}%
\pgfsys@useobject{currentmarker}{}%
\end{pgfscope}%
\begin{pgfscope}%
\pgfsys@transformshift{3.149621in}{0.955805in}%
\pgfsys@useobject{currentmarker}{}%
\end{pgfscope}%
\begin{pgfscope}%
\pgfsys@transformshift{3.149621in}{0.960791in}%
\pgfsys@useobject{currentmarker}{}%
\end{pgfscope}%
\begin{pgfscope}%
\pgfsys@transformshift{3.149621in}{0.965776in}%
\pgfsys@useobject{currentmarker}{}%
\end{pgfscope}%
\begin{pgfscope}%
\pgfsys@transformshift{3.149621in}{0.970762in}%
\pgfsys@useobject{currentmarker}{}%
\end{pgfscope}%
\begin{pgfscope}%
\pgfsys@transformshift{3.149621in}{0.975748in}%
\pgfsys@useobject{currentmarker}{}%
\end{pgfscope}%
\begin{pgfscope}%
\pgfsys@transformshift{3.149621in}{0.980734in}%
\pgfsys@useobject{currentmarker}{}%
\end{pgfscope}%
\begin{pgfscope}%
\pgfsys@transformshift{3.149621in}{0.985719in}%
\pgfsys@useobject{currentmarker}{}%
\end{pgfscope}%
\begin{pgfscope}%
\pgfsys@transformshift{3.149621in}{0.990705in}%
\pgfsys@useobject{currentmarker}{}%
\end{pgfscope}%
\begin{pgfscope}%
\pgfsys@transformshift{3.149621in}{1.000677in}%
\pgfsys@useobject{currentmarker}{}%
\end{pgfscope}%
\begin{pgfscope}%
\pgfsys@transformshift{3.149621in}{1.010648in}%
\pgfsys@useobject{currentmarker}{}%
\end{pgfscope}%
\begin{pgfscope}%
\pgfsys@transformshift{3.149621in}{1.015634in}%
\pgfsys@useobject{currentmarker}{}%
\end{pgfscope}%
\begin{pgfscope}%
\pgfsys@transformshift{3.149621in}{1.020620in}%
\pgfsys@useobject{currentmarker}{}%
\end{pgfscope}%
\begin{pgfscope}%
\pgfsys@transformshift{3.149621in}{1.025605in}%
\pgfsys@useobject{currentmarker}{}%
\end{pgfscope}%
\begin{pgfscope}%
\pgfsys@transformshift{3.149621in}{1.030591in}%
\pgfsys@useobject{currentmarker}{}%
\end{pgfscope}%
\begin{pgfscope}%
\pgfsys@transformshift{3.149621in}{1.035577in}%
\pgfsys@useobject{currentmarker}{}%
\end{pgfscope}%
\begin{pgfscope}%
\pgfsys@transformshift{3.149621in}{1.040563in}%
\pgfsys@useobject{currentmarker}{}%
\end{pgfscope}%
\begin{pgfscope}%
\pgfsys@transformshift{3.149621in}{1.045548in}%
\pgfsys@useobject{currentmarker}{}%
\end{pgfscope}%
\begin{pgfscope}%
\pgfsys@transformshift{3.149621in}{1.050534in}%
\pgfsys@useobject{currentmarker}{}%
\end{pgfscope}%
\begin{pgfscope}%
\pgfsys@transformshift{3.149621in}{1.055520in}%
\pgfsys@useobject{currentmarker}{}%
\end{pgfscope}%
\begin{pgfscope}%
\pgfsys@transformshift{3.149621in}{1.070477in}%
\pgfsys@useobject{currentmarker}{}%
\end{pgfscope}%
\begin{pgfscope}%
\pgfsys@transformshift{3.149621in}{1.085434in}%
\pgfsys@useobject{currentmarker}{}%
\end{pgfscope}%
\begin{pgfscope}%
\pgfsys@transformshift{3.149621in}{1.120335in}%
\pgfsys@useobject{currentmarker}{}%
\end{pgfscope}%
\begin{pgfscope}%
\pgfsys@transformshift{3.149621in}{1.185150in}%
\pgfsys@useobject{currentmarker}{}%
\end{pgfscope}%
\end{pgfscope}%
\begin{pgfscope}%
\pgfpathrectangle{\pgfqpoint{0.625000in}{0.275000in}}{\pgfqpoint{3.875000in}{1.925000in}}%
\pgfusepath{clip}%
\pgfsetbuttcap%
\pgfsetroundjoin%
\definecolor{currentfill}{rgb}{0.000000,0.000000,0.000000}%
\pgfsetfillcolor{currentfill}%
\pgfsetlinewidth{1.003750pt}%
\definecolor{currentstroke}{rgb}{0.000000,0.000000,0.000000}%
\pgfsetstrokecolor{currentstroke}%
\pgfsetdash{}{0pt}%
\pgfsys@defobject{currentmarker}{\pgfqpoint{-0.041667in}{-0.041667in}}{\pgfqpoint{0.041667in}{0.041667in}}{%
\pgfpathmoveto{\pgfqpoint{0.000000in}{-0.041667in}}%
\pgfpathcurveto{\pgfqpoint{0.011050in}{-0.041667in}}{\pgfqpoint{0.021649in}{-0.037276in}}{\pgfqpoint{0.029463in}{-0.029463in}}%
\pgfpathcurveto{\pgfqpoint{0.037276in}{-0.021649in}}{\pgfqpoint{0.041667in}{-0.011050in}}{\pgfqpoint{0.041667in}{0.000000in}}%
\pgfpathcurveto{\pgfqpoint{0.041667in}{0.011050in}}{\pgfqpoint{0.037276in}{0.021649in}}{\pgfqpoint{0.029463in}{0.029463in}}%
\pgfpathcurveto{\pgfqpoint{0.021649in}{0.037276in}}{\pgfqpoint{0.011050in}{0.041667in}}{\pgfqpoint{0.000000in}{0.041667in}}%
\pgfpathcurveto{\pgfqpoint{-0.011050in}{0.041667in}}{\pgfqpoint{-0.021649in}{0.037276in}}{\pgfqpoint{-0.029463in}{0.029463in}}%
\pgfpathcurveto{\pgfqpoint{-0.037276in}{0.021649in}}{\pgfqpoint{-0.041667in}{0.011050in}}{\pgfqpoint{-0.041667in}{0.000000in}}%
\pgfpathcurveto{\pgfqpoint{-0.041667in}{-0.011050in}}{\pgfqpoint{-0.037276in}{-0.021649in}}{\pgfqpoint{-0.029463in}{-0.029463in}}%
\pgfpathcurveto{\pgfqpoint{-0.021649in}{-0.037276in}}{\pgfqpoint{-0.011050in}{-0.041667in}}{\pgfqpoint{0.000000in}{-0.041667in}}%
\pgfpathclose%
\pgfusepath{stroke,fill}%
}%
\begin{pgfscope}%
\pgfsys@transformshift{4.323864in}{1.559081in}%
\pgfsys@useobject{currentmarker}{}%
\end{pgfscope}%
\begin{pgfscope}%
\pgfsys@transformshift{4.323864in}{1.593981in}%
\pgfsys@useobject{currentmarker}{}%
\end{pgfscope}%
\begin{pgfscope}%
\pgfsys@transformshift{4.323864in}{1.598967in}%
\pgfsys@useobject{currentmarker}{}%
\end{pgfscope}%
\begin{pgfscope}%
\pgfsys@transformshift{4.323864in}{1.603953in}%
\pgfsys@useobject{currentmarker}{}%
\end{pgfscope}%
\begin{pgfscope}%
\pgfsys@transformshift{4.323864in}{1.613925in}%
\pgfsys@useobject{currentmarker}{}%
\end{pgfscope}%
\begin{pgfscope}%
\pgfsys@transformshift{4.323864in}{1.618910in}%
\pgfsys@useobject{currentmarker}{}%
\end{pgfscope}%
\begin{pgfscope}%
\pgfsys@transformshift{4.323864in}{1.628882in}%
\pgfsys@useobject{currentmarker}{}%
\end{pgfscope}%
\begin{pgfscope}%
\pgfsys@transformshift{4.323864in}{1.633868in}%
\pgfsys@useobject{currentmarker}{}%
\end{pgfscope}%
\begin{pgfscope}%
\pgfsys@transformshift{4.323864in}{1.638853in}%
\pgfsys@useobject{currentmarker}{}%
\end{pgfscope}%
\begin{pgfscope}%
\pgfsys@transformshift{4.323864in}{1.643839in}%
\pgfsys@useobject{currentmarker}{}%
\end{pgfscope}%
\begin{pgfscope}%
\pgfsys@transformshift{4.323864in}{1.653811in}%
\pgfsys@useobject{currentmarker}{}%
\end{pgfscope}%
\begin{pgfscope}%
\pgfsys@transformshift{4.323864in}{1.658796in}%
\pgfsys@useobject{currentmarker}{}%
\end{pgfscope}%
\begin{pgfscope}%
\pgfsys@transformshift{4.323864in}{1.663782in}%
\pgfsys@useobject{currentmarker}{}%
\end{pgfscope}%
\begin{pgfscope}%
\pgfsys@transformshift{4.323864in}{1.668768in}%
\pgfsys@useobject{currentmarker}{}%
\end{pgfscope}%
\begin{pgfscope}%
\pgfsys@transformshift{4.323864in}{1.673754in}%
\pgfsys@useobject{currentmarker}{}%
\end{pgfscope}%
\begin{pgfscope}%
\pgfsys@transformshift{4.323864in}{1.678739in}%
\pgfsys@useobject{currentmarker}{}%
\end{pgfscope}%
\begin{pgfscope}%
\pgfsys@transformshift{4.323864in}{1.683725in}%
\pgfsys@useobject{currentmarker}{}%
\end{pgfscope}%
\begin{pgfscope}%
\pgfsys@transformshift{4.323864in}{1.688711in}%
\pgfsys@useobject{currentmarker}{}%
\end{pgfscope}%
\begin{pgfscope}%
\pgfsys@transformshift{4.323864in}{1.693697in}%
\pgfsys@useobject{currentmarker}{}%
\end{pgfscope}%
\begin{pgfscope}%
\pgfsys@transformshift{4.323864in}{1.698682in}%
\pgfsys@useobject{currentmarker}{}%
\end{pgfscope}%
\begin{pgfscope}%
\pgfsys@transformshift{4.323864in}{1.703668in}%
\pgfsys@useobject{currentmarker}{}%
\end{pgfscope}%
\begin{pgfscope}%
\pgfsys@transformshift{4.323864in}{1.708654in}%
\pgfsys@useobject{currentmarker}{}%
\end{pgfscope}%
\begin{pgfscope}%
\pgfsys@transformshift{4.323864in}{1.713640in}%
\pgfsys@useobject{currentmarker}{}%
\end{pgfscope}%
\begin{pgfscope}%
\pgfsys@transformshift{4.323864in}{1.718625in}%
\pgfsys@useobject{currentmarker}{}%
\end{pgfscope}%
\begin{pgfscope}%
\pgfsys@transformshift{4.323864in}{1.723611in}%
\pgfsys@useobject{currentmarker}{}%
\end{pgfscope}%
\begin{pgfscope}%
\pgfsys@transformshift{4.323864in}{1.733583in}%
\pgfsys@useobject{currentmarker}{}%
\end{pgfscope}%
\begin{pgfscope}%
\pgfsys@transformshift{4.323864in}{1.738568in}%
\pgfsys@useobject{currentmarker}{}%
\end{pgfscope}%
\begin{pgfscope}%
\pgfsys@transformshift{4.323864in}{1.743554in}%
\pgfsys@useobject{currentmarker}{}%
\end{pgfscope}%
\begin{pgfscope}%
\pgfsys@transformshift{4.323864in}{1.748540in}%
\pgfsys@useobject{currentmarker}{}%
\end{pgfscope}%
\begin{pgfscope}%
\pgfsys@transformshift{4.323864in}{1.753526in}%
\pgfsys@useobject{currentmarker}{}%
\end{pgfscope}%
\begin{pgfscope}%
\pgfsys@transformshift{4.323864in}{1.758511in}%
\pgfsys@useobject{currentmarker}{}%
\end{pgfscope}%
\begin{pgfscope}%
\pgfsys@transformshift{4.323864in}{1.768483in}%
\pgfsys@useobject{currentmarker}{}%
\end{pgfscope}%
\begin{pgfscope}%
\pgfsys@transformshift{4.323864in}{1.773469in}%
\pgfsys@useobject{currentmarker}{}%
\end{pgfscope}%
\begin{pgfscope}%
\pgfsys@transformshift{4.323864in}{1.778454in}%
\pgfsys@useobject{currentmarker}{}%
\end{pgfscope}%
\begin{pgfscope}%
\pgfsys@transformshift{4.323864in}{1.783440in}%
\pgfsys@useobject{currentmarker}{}%
\end{pgfscope}%
\begin{pgfscope}%
\pgfsys@transformshift{4.323864in}{1.788426in}%
\pgfsys@useobject{currentmarker}{}%
\end{pgfscope}%
\begin{pgfscope}%
\pgfsys@transformshift{4.323864in}{1.793412in}%
\pgfsys@useobject{currentmarker}{}%
\end{pgfscope}%
\begin{pgfscope}%
\pgfsys@transformshift{4.323864in}{1.803383in}%
\pgfsys@useobject{currentmarker}{}%
\end{pgfscope}%
\begin{pgfscope}%
\pgfsys@transformshift{4.323864in}{1.808369in}%
\pgfsys@useobject{currentmarker}{}%
\end{pgfscope}%
\begin{pgfscope}%
\pgfsys@transformshift{4.323864in}{1.813355in}%
\pgfsys@useobject{currentmarker}{}%
\end{pgfscope}%
\begin{pgfscope}%
\pgfsys@transformshift{4.323864in}{1.838283in}%
\pgfsys@useobject{currentmarker}{}%
\end{pgfscope}%
\begin{pgfscope}%
\pgfsys@transformshift{4.323864in}{1.843269in}%
\pgfsys@useobject{currentmarker}{}%
\end{pgfscope}%
\begin{pgfscope}%
\pgfsys@transformshift{4.323864in}{1.848255in}%
\pgfsys@useobject{currentmarker}{}%
\end{pgfscope}%
\begin{pgfscope}%
\pgfsys@transformshift{4.323864in}{1.858226in}%
\pgfsys@useobject{currentmarker}{}%
\end{pgfscope}%
\begin{pgfscope}%
\pgfsys@transformshift{4.323864in}{1.873184in}%
\pgfsys@useobject{currentmarker}{}%
\end{pgfscope}%
\begin{pgfscope}%
\pgfsys@transformshift{4.323864in}{1.878170in}%
\pgfsys@useobject{currentmarker}{}%
\end{pgfscope}%
\begin{pgfscope}%
\pgfsys@transformshift{4.323864in}{1.883155in}%
\pgfsys@useobject{currentmarker}{}%
\end{pgfscope}%
\begin{pgfscope}%
\pgfsys@transformshift{4.323864in}{1.893127in}%
\pgfsys@useobject{currentmarker}{}%
\end{pgfscope}%
\begin{pgfscope}%
\pgfsys@transformshift{4.323864in}{1.898113in}%
\pgfsys@useobject{currentmarker}{}%
\end{pgfscope}%
\begin{pgfscope}%
\pgfsys@transformshift{4.323864in}{1.913070in}%
\pgfsys@useobject{currentmarker}{}%
\end{pgfscope}%
\begin{pgfscope}%
\pgfsys@transformshift{4.323864in}{1.918056in}%
\pgfsys@useobject{currentmarker}{}%
\end{pgfscope}%
\begin{pgfscope}%
\pgfsys@transformshift{4.323864in}{1.937999in}%
\pgfsys@useobject{currentmarker}{}%
\end{pgfscope}%
\begin{pgfscope}%
\pgfsys@transformshift{4.323864in}{1.972899in}%
\pgfsys@useobject{currentmarker}{}%
\end{pgfscope}%
\begin{pgfscope}%
\pgfsys@transformshift{4.323864in}{1.997828in}%
\pgfsys@useobject{currentmarker}{}%
\end{pgfscope}%
\begin{pgfscope}%
\pgfsys@transformshift{4.323864in}{2.007799in}%
\pgfsys@useobject{currentmarker}{}%
\end{pgfscope}%
\begin{pgfscope}%
\pgfsys@transformshift{4.323864in}{2.012785in}%
\pgfsys@useobject{currentmarker}{}%
\end{pgfscope}%
\begin{pgfscope}%
\pgfsys@transformshift{4.323864in}{2.037714in}%
\pgfsys@useobject{currentmarker}{}%
\end{pgfscope}%
\begin{pgfscope}%
\pgfsys@transformshift{4.323864in}{2.042699in}%
\pgfsys@useobject{currentmarker}{}%
\end{pgfscope}%
\begin{pgfscope}%
\pgfsys@transformshift{4.323864in}{2.072614in}%
\pgfsys@useobject{currentmarker}{}%
\end{pgfscope}%
\begin{pgfscope}%
\pgfsys@transformshift{4.323864in}{2.092557in}%
\pgfsys@useobject{currentmarker}{}%
\end{pgfscope}%
\begin{pgfscope}%
\pgfsys@transformshift{4.323864in}{2.112500in}%
\pgfsys@useobject{currentmarker}{}%
\end{pgfscope}%
\end{pgfscope}%
\begin{pgfscope}%
\pgfpathrectangle{\pgfqpoint{0.625000in}{0.275000in}}{\pgfqpoint{3.875000in}{1.925000in}}%
\pgfusepath{clip}%
\pgfsetbuttcap%
\pgfsetmiterjoin%
\definecolor{currentfill}{rgb}{0.000000,0.000000,0.000000}%
\pgfsetfillcolor{currentfill}%
\pgfsetlinewidth{1.003750pt}%
\definecolor{currentstroke}{rgb}{0.000000,0.000000,0.000000}%
\pgfsetstrokecolor{currentstroke}%
\pgfsetdash{}{0pt}%
\pgfsys@defobject{currentmarker}{\pgfqpoint{-0.041667in}{-0.041667in}}{\pgfqpoint{0.041667in}{0.041667in}}{%
\pgfpathmoveto{\pgfqpoint{-0.000000in}{-0.041667in}}%
\pgfpathlineto{\pgfqpoint{0.041667in}{0.041667in}}%
\pgfpathlineto{\pgfqpoint{-0.041667in}{0.041667in}}%
\pgfpathclose%
\pgfusepath{stroke,fill}%
}%
\begin{pgfscope}%
\pgfsys@transformshift{0.801136in}{0.387429in}%
\pgfsys@useobject{currentmarker}{}%
\end{pgfscope}%
\begin{pgfscope}%
\pgfsys@transformshift{0.801136in}{0.392415in}%
\pgfsys@useobject{currentmarker}{}%
\end{pgfscope}%
\begin{pgfscope}%
\pgfsys@transformshift{0.801136in}{0.397400in}%
\pgfsys@useobject{currentmarker}{}%
\end{pgfscope}%
\begin{pgfscope}%
\pgfsys@transformshift{0.801136in}{0.402386in}%
\pgfsys@useobject{currentmarker}{}%
\end{pgfscope}%
\begin{pgfscope}%
\pgfsys@transformshift{0.801136in}{0.407372in}%
\pgfsys@useobject{currentmarker}{}%
\end{pgfscope}%
\end{pgfscope}%
\begin{pgfscope}%
\pgfpathrectangle{\pgfqpoint{0.625000in}{0.275000in}}{\pgfqpoint{3.875000in}{1.925000in}}%
\pgfusepath{clip}%
\pgfsetbuttcap%
\pgfsetmiterjoin%
\definecolor{currentfill}{rgb}{0.000000,0.000000,0.000000}%
\pgfsetfillcolor{currentfill}%
\pgfsetlinewidth{1.003750pt}%
\definecolor{currentstroke}{rgb}{0.000000,0.000000,0.000000}%
\pgfsetstrokecolor{currentstroke}%
\pgfsetdash{}{0pt}%
\pgfsys@defobject{currentmarker}{\pgfqpoint{-0.041667in}{-0.041667in}}{\pgfqpoint{0.041667in}{0.041667in}}{%
\pgfpathmoveto{\pgfqpoint{-0.000000in}{-0.041667in}}%
\pgfpathlineto{\pgfqpoint{0.041667in}{0.041667in}}%
\pgfpathlineto{\pgfqpoint{-0.041667in}{0.041667in}}%
\pgfpathclose%
\pgfusepath{stroke,fill}%
}%
\begin{pgfscope}%
\pgfsys@transformshift{1.975379in}{0.432301in}%
\pgfsys@useobject{currentmarker}{}%
\end{pgfscope}%
\begin{pgfscope}%
\pgfsys@transformshift{1.975379in}{0.437286in}%
\pgfsys@useobject{currentmarker}{}%
\end{pgfscope}%
\begin{pgfscope}%
\pgfsys@transformshift{1.975379in}{0.442272in}%
\pgfsys@useobject{currentmarker}{}%
\end{pgfscope}%
\begin{pgfscope}%
\pgfsys@transformshift{1.975379in}{0.447258in}%
\pgfsys@useobject{currentmarker}{}%
\end{pgfscope}%
\begin{pgfscope}%
\pgfsys@transformshift{1.975379in}{0.452244in}%
\pgfsys@useobject{currentmarker}{}%
\end{pgfscope}%
\begin{pgfscope}%
\pgfsys@transformshift{1.975379in}{0.457229in}%
\pgfsys@useobject{currentmarker}{}%
\end{pgfscope}%
\begin{pgfscope}%
\pgfsys@transformshift{1.975379in}{0.462215in}%
\pgfsys@useobject{currentmarker}{}%
\end{pgfscope}%
\begin{pgfscope}%
\pgfsys@transformshift{1.975379in}{0.467201in}%
\pgfsys@useobject{currentmarker}{}%
\end{pgfscope}%
\begin{pgfscope}%
\pgfsys@transformshift{1.975379in}{0.472187in}%
\pgfsys@useobject{currentmarker}{}%
\end{pgfscope}%
\begin{pgfscope}%
\pgfsys@transformshift{1.975379in}{0.477172in}%
\pgfsys@useobject{currentmarker}{}%
\end{pgfscope}%
\begin{pgfscope}%
\pgfsys@transformshift{1.975379in}{0.482158in}%
\pgfsys@useobject{currentmarker}{}%
\end{pgfscope}%
\begin{pgfscope}%
\pgfsys@transformshift{1.975379in}{0.487144in}%
\pgfsys@useobject{currentmarker}{}%
\end{pgfscope}%
\begin{pgfscope}%
\pgfsys@transformshift{1.975379in}{0.497115in}%
\pgfsys@useobject{currentmarker}{}%
\end{pgfscope}%
\end{pgfscope}%
\begin{pgfscope}%
\pgfpathrectangle{\pgfqpoint{0.625000in}{0.275000in}}{\pgfqpoint{3.875000in}{1.925000in}}%
\pgfusepath{clip}%
\pgfsetbuttcap%
\pgfsetmiterjoin%
\definecolor{currentfill}{rgb}{0.000000,0.000000,0.000000}%
\pgfsetfillcolor{currentfill}%
\pgfsetlinewidth{1.003750pt}%
\definecolor{currentstroke}{rgb}{0.000000,0.000000,0.000000}%
\pgfsetstrokecolor{currentstroke}%
\pgfsetdash{}{0pt}%
\pgfsys@defobject{currentmarker}{\pgfqpoint{-0.041667in}{-0.041667in}}{\pgfqpoint{0.041667in}{0.041667in}}{%
\pgfpathmoveto{\pgfqpoint{-0.000000in}{-0.041667in}}%
\pgfpathlineto{\pgfqpoint{0.041667in}{0.041667in}}%
\pgfpathlineto{\pgfqpoint{-0.041667in}{0.041667in}}%
\pgfpathclose%
\pgfusepath{stroke,fill}%
}%
\begin{pgfscope}%
\pgfsys@transformshift{3.149621in}{0.522044in}%
\pgfsys@useobject{currentmarker}{}%
\end{pgfscope}%
\begin{pgfscope}%
\pgfsys@transformshift{3.149621in}{0.527030in}%
\pgfsys@useobject{currentmarker}{}%
\end{pgfscope}%
\begin{pgfscope}%
\pgfsys@transformshift{3.149621in}{0.532016in}%
\pgfsys@useobject{currentmarker}{}%
\end{pgfscope}%
\begin{pgfscope}%
\pgfsys@transformshift{3.149621in}{0.537001in}%
\pgfsys@useobject{currentmarker}{}%
\end{pgfscope}%
\begin{pgfscope}%
\pgfsys@transformshift{3.149621in}{0.541987in}%
\pgfsys@useobject{currentmarker}{}%
\end{pgfscope}%
\begin{pgfscope}%
\pgfsys@transformshift{3.149621in}{0.546973in}%
\pgfsys@useobject{currentmarker}{}%
\end{pgfscope}%
\begin{pgfscope}%
\pgfsys@transformshift{3.149621in}{0.551959in}%
\pgfsys@useobject{currentmarker}{}%
\end{pgfscope}%
\begin{pgfscope}%
\pgfsys@transformshift{3.149621in}{0.556944in}%
\pgfsys@useobject{currentmarker}{}%
\end{pgfscope}%
\begin{pgfscope}%
\pgfsys@transformshift{3.149621in}{0.561930in}%
\pgfsys@useobject{currentmarker}{}%
\end{pgfscope}%
\begin{pgfscope}%
\pgfsys@transformshift{3.149621in}{0.566916in}%
\pgfsys@useobject{currentmarker}{}%
\end{pgfscope}%
\begin{pgfscope}%
\pgfsys@transformshift{3.149621in}{0.586859in}%
\pgfsys@useobject{currentmarker}{}%
\end{pgfscope}%
\end{pgfscope}%
\begin{pgfscope}%
\pgfpathrectangle{\pgfqpoint{0.625000in}{0.275000in}}{\pgfqpoint{3.875000in}{1.925000in}}%
\pgfusepath{clip}%
\pgfsetbuttcap%
\pgfsetmiterjoin%
\definecolor{currentfill}{rgb}{0.000000,0.000000,0.000000}%
\pgfsetfillcolor{currentfill}%
\pgfsetlinewidth{1.003750pt}%
\definecolor{currentstroke}{rgb}{0.000000,0.000000,0.000000}%
\pgfsetstrokecolor{currentstroke}%
\pgfsetdash{}{0pt}%
\pgfsys@defobject{currentmarker}{\pgfqpoint{-0.041667in}{-0.041667in}}{\pgfqpoint{0.041667in}{0.041667in}}{%
\pgfpathmoveto{\pgfqpoint{-0.000000in}{-0.041667in}}%
\pgfpathlineto{\pgfqpoint{0.041667in}{0.041667in}}%
\pgfpathlineto{\pgfqpoint{-0.041667in}{0.041667in}}%
\pgfpathclose%
\pgfusepath{stroke,fill}%
}%
\begin{pgfscope}%
\pgfsys@transformshift{4.323864in}{0.611788in}%
\pgfsys@useobject{currentmarker}{}%
\end{pgfscope}%
\begin{pgfscope}%
\pgfsys@transformshift{4.323864in}{0.621759in}%
\pgfsys@useobject{currentmarker}{}%
\end{pgfscope}%
\begin{pgfscope}%
\pgfsys@transformshift{4.323864in}{0.626745in}%
\pgfsys@useobject{currentmarker}{}%
\end{pgfscope}%
\begin{pgfscope}%
\pgfsys@transformshift{4.323864in}{0.631731in}%
\pgfsys@useobject{currentmarker}{}%
\end{pgfscope}%
\begin{pgfscope}%
\pgfsys@transformshift{4.323864in}{0.636717in}%
\pgfsys@useobject{currentmarker}{}%
\end{pgfscope}%
\begin{pgfscope}%
\pgfsys@transformshift{4.323864in}{0.641702in}%
\pgfsys@useobject{currentmarker}{}%
\end{pgfscope}%
\begin{pgfscope}%
\pgfsys@transformshift{4.323864in}{0.646688in}%
\pgfsys@useobject{currentmarker}{}%
\end{pgfscope}%
\begin{pgfscope}%
\pgfsys@transformshift{4.323864in}{0.651674in}%
\pgfsys@useobject{currentmarker}{}%
\end{pgfscope}%
\begin{pgfscope}%
\pgfsys@transformshift{4.323864in}{0.656660in}%
\pgfsys@useobject{currentmarker}{}%
\end{pgfscope}%
\begin{pgfscope}%
\pgfsys@transformshift{4.323864in}{0.661645in}%
\pgfsys@useobject{currentmarker}{}%
\end{pgfscope}%
\begin{pgfscope}%
\pgfsys@transformshift{4.323864in}{0.666631in}%
\pgfsys@useobject{currentmarker}{}%
\end{pgfscope}%
\begin{pgfscope}%
\pgfsys@transformshift{4.323864in}{0.671617in}%
\pgfsys@useobject{currentmarker}{}%
\end{pgfscope}%
\begin{pgfscope}%
\pgfsys@transformshift{4.323864in}{0.681588in}%
\pgfsys@useobject{currentmarker}{}%
\end{pgfscope}%
\begin{pgfscope}%
\pgfsys@transformshift{4.323864in}{0.686574in}%
\pgfsys@useobject{currentmarker}{}%
\end{pgfscope}%
\begin{pgfscope}%
\pgfsys@transformshift{4.323864in}{0.691560in}%
\pgfsys@useobject{currentmarker}{}%
\end{pgfscope}%
\begin{pgfscope}%
\pgfsys@transformshift{4.323864in}{0.701531in}%
\pgfsys@useobject{currentmarker}{}%
\end{pgfscope}%
\end{pgfscope}%
\begin{pgfscope}%
\pgfpathrectangle{\pgfqpoint{0.625000in}{0.275000in}}{\pgfqpoint{3.875000in}{1.925000in}}%
\pgfusepath{clip}%
\pgfsetbuttcap%
\pgfsetmiterjoin%
\definecolor{currentfill}{rgb}{0.000000,0.000000,0.000000}%
\pgfsetfillcolor{currentfill}%
\pgfsetlinewidth{1.003750pt}%
\definecolor{currentstroke}{rgb}{0.000000,0.000000,0.000000}%
\pgfsetstrokecolor{currentstroke}%
\pgfsetdash{}{0pt}%
\pgfsys@defobject{currentmarker}{\pgfqpoint{-0.041667in}{-0.041667in}}{\pgfqpoint{0.041667in}{0.041667in}}{%
\pgfpathmoveto{\pgfqpoint{-0.041667in}{-0.041667in}}%
\pgfpathlineto{\pgfqpoint{0.041667in}{-0.041667in}}%
\pgfpathlineto{\pgfqpoint{0.041667in}{0.041667in}}%
\pgfpathlineto{\pgfqpoint{-0.041667in}{0.041667in}}%
\pgfpathclose%
\pgfusepath{stroke,fill}%
}%
\begin{pgfscope}%
\pgfsys@transformshift{0.801136in}{0.372472in}%
\pgfsys@useobject{currentmarker}{}%
\end{pgfscope}%
\begin{pgfscope}%
\pgfsys@transformshift{0.801136in}{0.377457in}%
\pgfsys@useobject{currentmarker}{}%
\end{pgfscope}%
\begin{pgfscope}%
\pgfsys@transformshift{0.801136in}{0.382443in}%
\pgfsys@useobject{currentmarker}{}%
\end{pgfscope}%
\begin{pgfscope}%
\pgfsys@transformshift{0.801136in}{0.387429in}%
\pgfsys@useobject{currentmarker}{}%
\end{pgfscope}%
\end{pgfscope}%
\begin{pgfscope}%
\pgfpathrectangle{\pgfqpoint{0.625000in}{0.275000in}}{\pgfqpoint{3.875000in}{1.925000in}}%
\pgfusepath{clip}%
\pgfsetbuttcap%
\pgfsetmiterjoin%
\definecolor{currentfill}{rgb}{0.000000,0.000000,0.000000}%
\pgfsetfillcolor{currentfill}%
\pgfsetlinewidth{1.003750pt}%
\definecolor{currentstroke}{rgb}{0.000000,0.000000,0.000000}%
\pgfsetstrokecolor{currentstroke}%
\pgfsetdash{}{0pt}%
\pgfsys@defobject{currentmarker}{\pgfqpoint{-0.041667in}{-0.041667in}}{\pgfqpoint{0.041667in}{0.041667in}}{%
\pgfpathmoveto{\pgfqpoint{-0.041667in}{-0.041667in}}%
\pgfpathlineto{\pgfqpoint{0.041667in}{-0.041667in}}%
\pgfpathlineto{\pgfqpoint{0.041667in}{0.041667in}}%
\pgfpathlineto{\pgfqpoint{-0.041667in}{0.041667in}}%
\pgfpathclose%
\pgfusepath{stroke,fill}%
}%
\begin{pgfscope}%
\pgfsys@transformshift{1.975379in}{0.392415in}%
\pgfsys@useobject{currentmarker}{}%
\end{pgfscope}%
\begin{pgfscope}%
\pgfsys@transformshift{1.975379in}{0.397400in}%
\pgfsys@useobject{currentmarker}{}%
\end{pgfscope}%
\begin{pgfscope}%
\pgfsys@transformshift{1.975379in}{0.402386in}%
\pgfsys@useobject{currentmarker}{}%
\end{pgfscope}%
\begin{pgfscope}%
\pgfsys@transformshift{1.975379in}{0.407372in}%
\pgfsys@useobject{currentmarker}{}%
\end{pgfscope}%
\begin{pgfscope}%
\pgfsys@transformshift{1.975379in}{0.412358in}%
\pgfsys@useobject{currentmarker}{}%
\end{pgfscope}%
\begin{pgfscope}%
\pgfsys@transformshift{1.975379in}{0.417343in}%
\pgfsys@useobject{currentmarker}{}%
\end{pgfscope}%
\begin{pgfscope}%
\pgfsys@transformshift{1.975379in}{0.422329in}%
\pgfsys@useobject{currentmarker}{}%
\end{pgfscope}%
\end{pgfscope}%
\begin{pgfscope}%
\pgfpathrectangle{\pgfqpoint{0.625000in}{0.275000in}}{\pgfqpoint{3.875000in}{1.925000in}}%
\pgfusepath{clip}%
\pgfsetbuttcap%
\pgfsetmiterjoin%
\definecolor{currentfill}{rgb}{0.000000,0.000000,0.000000}%
\pgfsetfillcolor{currentfill}%
\pgfsetlinewidth{1.003750pt}%
\definecolor{currentstroke}{rgb}{0.000000,0.000000,0.000000}%
\pgfsetstrokecolor{currentstroke}%
\pgfsetdash{}{0pt}%
\pgfsys@defobject{currentmarker}{\pgfqpoint{-0.041667in}{-0.041667in}}{\pgfqpoint{0.041667in}{0.041667in}}{%
\pgfpathmoveto{\pgfqpoint{-0.041667in}{-0.041667in}}%
\pgfpathlineto{\pgfqpoint{0.041667in}{-0.041667in}}%
\pgfpathlineto{\pgfqpoint{0.041667in}{0.041667in}}%
\pgfpathlineto{\pgfqpoint{-0.041667in}{0.041667in}}%
\pgfpathclose%
\pgfusepath{stroke,fill}%
}%
\begin{pgfscope}%
\pgfsys@transformshift{3.149621in}{0.427315in}%
\pgfsys@useobject{currentmarker}{}%
\end{pgfscope}%
\begin{pgfscope}%
\pgfsys@transformshift{3.149621in}{0.432301in}%
\pgfsys@useobject{currentmarker}{}%
\end{pgfscope}%
\begin{pgfscope}%
\pgfsys@transformshift{3.149621in}{0.437286in}%
\pgfsys@useobject{currentmarker}{}%
\end{pgfscope}%
\begin{pgfscope}%
\pgfsys@transformshift{3.149621in}{0.442272in}%
\pgfsys@useobject{currentmarker}{}%
\end{pgfscope}%
\begin{pgfscope}%
\pgfsys@transformshift{3.149621in}{0.447258in}%
\pgfsys@useobject{currentmarker}{}%
\end{pgfscope}%
\end{pgfscope}%
\begin{pgfscope}%
\pgfpathrectangle{\pgfqpoint{0.625000in}{0.275000in}}{\pgfqpoint{3.875000in}{1.925000in}}%
\pgfusepath{clip}%
\pgfsetbuttcap%
\pgfsetmiterjoin%
\definecolor{currentfill}{rgb}{0.000000,0.000000,0.000000}%
\pgfsetfillcolor{currentfill}%
\pgfsetlinewidth{1.003750pt}%
\definecolor{currentstroke}{rgb}{0.000000,0.000000,0.000000}%
\pgfsetstrokecolor{currentstroke}%
\pgfsetdash{}{0pt}%
\pgfsys@defobject{currentmarker}{\pgfqpoint{-0.041667in}{-0.041667in}}{\pgfqpoint{0.041667in}{0.041667in}}{%
\pgfpathmoveto{\pgfqpoint{-0.041667in}{-0.041667in}}%
\pgfpathlineto{\pgfqpoint{0.041667in}{-0.041667in}}%
\pgfpathlineto{\pgfqpoint{0.041667in}{0.041667in}}%
\pgfpathlineto{\pgfqpoint{-0.041667in}{0.041667in}}%
\pgfpathclose%
\pgfusepath{stroke,fill}%
}%
\begin{pgfscope}%
\pgfsys@transformshift{4.323864in}{0.452244in}%
\pgfsys@useobject{currentmarker}{}%
\end{pgfscope}%
\begin{pgfscope}%
\pgfsys@transformshift{4.323864in}{0.457229in}%
\pgfsys@useobject{currentmarker}{}%
\end{pgfscope}%
\begin{pgfscope}%
\pgfsys@transformshift{4.323864in}{0.462215in}%
\pgfsys@useobject{currentmarker}{}%
\end{pgfscope}%
\begin{pgfscope}%
\pgfsys@transformshift{4.323864in}{0.467201in}%
\pgfsys@useobject{currentmarker}{}%
\end{pgfscope}%
\begin{pgfscope}%
\pgfsys@transformshift{4.323864in}{0.472187in}%
\pgfsys@useobject{currentmarker}{}%
\end{pgfscope}%
\begin{pgfscope}%
\pgfsys@transformshift{4.323864in}{0.477172in}%
\pgfsys@useobject{currentmarker}{}%
\end{pgfscope}%
\end{pgfscope}%
\begin{pgfscope}%
\pgfpathrectangle{\pgfqpoint{0.625000in}{0.275000in}}{\pgfqpoint{3.875000in}{1.925000in}}%
\pgfusepath{clip}%
\pgfsetbuttcap%
\pgfsetmiterjoin%
\definecolor{currentfill}{rgb}{0.000000,0.000000,0.000000}%
\pgfsetfillcolor{currentfill}%
\pgfsetlinewidth{1.003750pt}%
\definecolor{currentstroke}{rgb}{0.000000,0.000000,0.000000}%
\pgfsetstrokecolor{currentstroke}%
\pgfsetdash{}{0pt}%
\pgfsys@defobject{currentmarker}{\pgfqpoint{-0.035355in}{-0.058926in}}{\pgfqpoint{0.035355in}{0.058926in}}{%
\pgfpathmoveto{\pgfqpoint{-0.000000in}{-0.058926in}}%
\pgfpathlineto{\pgfqpoint{0.035355in}{0.000000in}}%
\pgfpathlineto{\pgfqpoint{0.000000in}{0.058926in}}%
\pgfpathlineto{\pgfqpoint{-0.035355in}{0.000000in}}%
\pgfpathclose%
\pgfusepath{stroke,fill}%
}%
\begin{pgfscope}%
\pgfsys@transformshift{0.801136in}{0.362500in}%
\pgfsys@useobject{currentmarker}{}%
\end{pgfscope}%
\begin{pgfscope}%
\pgfsys@transformshift{0.801136in}{0.367486in}%
\pgfsys@useobject{currentmarker}{}%
\end{pgfscope}%
\begin{pgfscope}%
\pgfsys@transformshift{0.801136in}{0.372472in}%
\pgfsys@useobject{currentmarker}{}%
\end{pgfscope}%
\end{pgfscope}%
\begin{pgfscope}%
\pgfpathrectangle{\pgfqpoint{0.625000in}{0.275000in}}{\pgfqpoint{3.875000in}{1.925000in}}%
\pgfusepath{clip}%
\pgfsetbuttcap%
\pgfsetmiterjoin%
\definecolor{currentfill}{rgb}{0.000000,0.000000,0.000000}%
\pgfsetfillcolor{currentfill}%
\pgfsetlinewidth{1.003750pt}%
\definecolor{currentstroke}{rgb}{0.000000,0.000000,0.000000}%
\pgfsetstrokecolor{currentstroke}%
\pgfsetdash{}{0pt}%
\pgfsys@defobject{currentmarker}{\pgfqpoint{-0.035355in}{-0.058926in}}{\pgfqpoint{0.035355in}{0.058926in}}{%
\pgfpathmoveto{\pgfqpoint{-0.000000in}{-0.058926in}}%
\pgfpathlineto{\pgfqpoint{0.035355in}{0.000000in}}%
\pgfpathlineto{\pgfqpoint{0.000000in}{0.058926in}}%
\pgfpathlineto{\pgfqpoint{-0.035355in}{0.000000in}}%
\pgfpathclose%
\pgfusepath{stroke,fill}%
}%
\begin{pgfscope}%
\pgfsys@transformshift{1.975379in}{0.372472in}%
\pgfsys@useobject{currentmarker}{}%
\end{pgfscope}%
\begin{pgfscope}%
\pgfsys@transformshift{1.975379in}{0.377457in}%
\pgfsys@useobject{currentmarker}{}%
\end{pgfscope}%
\begin{pgfscope}%
\pgfsys@transformshift{1.975379in}{0.382443in}%
\pgfsys@useobject{currentmarker}{}%
\end{pgfscope}%
\begin{pgfscope}%
\pgfsys@transformshift{1.975379in}{0.387429in}%
\pgfsys@useobject{currentmarker}{}%
\end{pgfscope}%
\begin{pgfscope}%
\pgfsys@transformshift{1.975379in}{0.392415in}%
\pgfsys@useobject{currentmarker}{}%
\end{pgfscope}%
\end{pgfscope}%
\begin{pgfscope}%
\pgfpathrectangle{\pgfqpoint{0.625000in}{0.275000in}}{\pgfqpoint{3.875000in}{1.925000in}}%
\pgfusepath{clip}%
\pgfsetbuttcap%
\pgfsetmiterjoin%
\definecolor{currentfill}{rgb}{0.000000,0.000000,0.000000}%
\pgfsetfillcolor{currentfill}%
\pgfsetlinewidth{1.003750pt}%
\definecolor{currentstroke}{rgb}{0.000000,0.000000,0.000000}%
\pgfsetstrokecolor{currentstroke}%
\pgfsetdash{}{0pt}%
\pgfsys@defobject{currentmarker}{\pgfqpoint{-0.035355in}{-0.058926in}}{\pgfqpoint{0.035355in}{0.058926in}}{%
\pgfpathmoveto{\pgfqpoint{-0.000000in}{-0.058926in}}%
\pgfpathlineto{\pgfqpoint{0.035355in}{0.000000in}}%
\pgfpathlineto{\pgfqpoint{0.000000in}{0.058926in}}%
\pgfpathlineto{\pgfqpoint{-0.035355in}{0.000000in}}%
\pgfpathclose%
\pgfusepath{stroke,fill}%
}%
\begin{pgfscope}%
\pgfsys@transformshift{3.149621in}{0.387429in}%
\pgfsys@useobject{currentmarker}{}%
\end{pgfscope}%
\begin{pgfscope}%
\pgfsys@transformshift{3.149621in}{0.392415in}%
\pgfsys@useobject{currentmarker}{}%
\end{pgfscope}%
\begin{pgfscope}%
\pgfsys@transformshift{3.149621in}{0.397400in}%
\pgfsys@useobject{currentmarker}{}%
\end{pgfscope}%
\end{pgfscope}%
\begin{pgfscope}%
\pgfpathrectangle{\pgfqpoint{0.625000in}{0.275000in}}{\pgfqpoint{3.875000in}{1.925000in}}%
\pgfusepath{clip}%
\pgfsetbuttcap%
\pgfsetmiterjoin%
\definecolor{currentfill}{rgb}{0.000000,0.000000,0.000000}%
\pgfsetfillcolor{currentfill}%
\pgfsetlinewidth{1.003750pt}%
\definecolor{currentstroke}{rgb}{0.000000,0.000000,0.000000}%
\pgfsetstrokecolor{currentstroke}%
\pgfsetdash{}{0pt}%
\pgfsys@defobject{currentmarker}{\pgfqpoint{-0.035355in}{-0.058926in}}{\pgfqpoint{0.035355in}{0.058926in}}{%
\pgfpathmoveto{\pgfqpoint{-0.000000in}{-0.058926in}}%
\pgfpathlineto{\pgfqpoint{0.035355in}{0.000000in}}%
\pgfpathlineto{\pgfqpoint{0.000000in}{0.058926in}}%
\pgfpathlineto{\pgfqpoint{-0.035355in}{0.000000in}}%
\pgfpathclose%
\pgfusepath{stroke,fill}%
}%
\begin{pgfscope}%
\pgfsys@transformshift{4.323864in}{0.397400in}%
\pgfsys@useobject{currentmarker}{}%
\end{pgfscope}%
\begin{pgfscope}%
\pgfsys@transformshift{4.323864in}{0.402386in}%
\pgfsys@useobject{currentmarker}{}%
\end{pgfscope}%
\begin{pgfscope}%
\pgfsys@transformshift{4.323864in}{0.407372in}%
\pgfsys@useobject{currentmarker}{}%
\end{pgfscope}%
\end{pgfscope}%
\begin{pgfscope}%
\pgfsetrectcap%
\pgfsetmiterjoin%
\pgfsetlinewidth{0.803000pt}%
\definecolor{currentstroke}{rgb}{0.000000,0.000000,0.000000}%
\pgfsetstrokecolor{currentstroke}%
\pgfsetdash{}{0pt}%
\pgfpathmoveto{\pgfqpoint{0.625000in}{0.275000in}}%
\pgfpathlineto{\pgfqpoint{0.625000in}{2.200000in}}%
\pgfusepath{stroke}%
\end{pgfscope}%
\begin{pgfscope}%
\pgfsetrectcap%
\pgfsetmiterjoin%
\pgfsetlinewidth{0.803000pt}%
\definecolor{currentstroke}{rgb}{0.000000,0.000000,0.000000}%
\pgfsetstrokecolor{currentstroke}%
\pgfsetdash{}{0pt}%
\pgfpathmoveto{\pgfqpoint{4.500000in}{0.275000in}}%
\pgfpathlineto{\pgfqpoint{4.500000in}{2.200000in}}%
\pgfusepath{stroke}%
\end{pgfscope}%
\begin{pgfscope}%
\pgfsetrectcap%
\pgfsetmiterjoin%
\pgfsetlinewidth{0.803000pt}%
\definecolor{currentstroke}{rgb}{0.000000,0.000000,0.000000}%
\pgfsetstrokecolor{currentstroke}%
\pgfsetdash{}{0pt}%
\pgfpathmoveto{\pgfqpoint{0.625000in}{0.275000in}}%
\pgfpathlineto{\pgfqpoint{4.500000in}{0.275000in}}%
\pgfusepath{stroke}%
\end{pgfscope}%
\begin{pgfscope}%
\pgfsetrectcap%
\pgfsetmiterjoin%
\pgfsetlinewidth{0.803000pt}%
\definecolor{currentstroke}{rgb}{0.000000,0.000000,0.000000}%
\pgfsetstrokecolor{currentstroke}%
\pgfsetdash{}{0pt}%
\pgfpathmoveto{\pgfqpoint{0.625000in}{2.200000in}}%
\pgfpathlineto{\pgfqpoint{4.500000in}{2.200000in}}%
\pgfusepath{stroke}%
\end{pgfscope}%
\end{pgfpicture}%
\makeatother%
\endgroup%

  \caption[Maximum GMRES iteration counts when $\NLiDRRdtd{\Aso-\Ast} = 0.5\times  k^{\beta}$ for $\beta = 0,0.1,0.2,0.3.$]{Maximum GMRES iteration counts for solving systems with matrix $\AmatoI\Amatt$, where $\nso=\nst=1$ and $\NLiDRRdtd{\Aso-\Ast} = 0.5\times  k^{\beta}$ for $\beta = 0,0.1,0.2,0.3.$}\label{fig:linfinityA0}
    \end{figure}
    
    \begin{figure}
      \centering
%% Creator: Matplotlib, PGF backend
%%
%% To include the figure in your LaTeX document, write
%%   \input{<filename>.pgf}
%%
%% Make sure the required packages are loaded in your preamble
%%   \usepackage{pgf}
%%
%% Figures using additional raster images can only be included by \input if
%% they are in the same directory as the main LaTeX file. For loading figures
%% from other directories you can use the `import` package
%%   \usepackage{import}
%% and then include the figures with
%%   \import{<path to file>}{<filename>.pgf}
%%
%% Matplotlib used the following preamble
%%   \usepackage{fontspec}
%%   \setmainfont{DejaVuSerif.ttf}[Path=/home/owen/progs/firedrake-complex/firedrake/lib/python3.5/site-packages/matplotlib/mpl-data/fonts/ttf/]
%%   \setsansfont{DejaVuSans.ttf}[Path=/home/owen/progs/firedrake-complex/firedrake/lib/python3.5/site-packages/matplotlib/mpl-data/fonts/ttf/]
%%   \setmonofont{DejaVuSansMono.ttf}[Path=/home/owen/progs/firedrake-complex/firedrake/lib/python3.5/site-packages/matplotlib/mpl-data/fonts/ttf/]
%%
\begingroup%
\makeatletter%
\begin{pgfpicture}%
\pgfpathrectangle{\pgfpointorigin}{\pgfqpoint{6.400000in}{4.800000in}}%
\pgfusepath{use as bounding box, clip}%
\begin{pgfscope}%
\pgfsetbuttcap%
\pgfsetmiterjoin%
\definecolor{currentfill}{rgb}{1.000000,1.000000,1.000000}%
\pgfsetfillcolor{currentfill}%
\pgfsetlinewidth{0.000000pt}%
\definecolor{currentstroke}{rgb}{1.000000,1.000000,1.000000}%
\pgfsetstrokecolor{currentstroke}%
\pgfsetdash{}{0pt}%
\pgfpathmoveto{\pgfqpoint{0.000000in}{0.000000in}}%
\pgfpathlineto{\pgfqpoint{6.400000in}{0.000000in}}%
\pgfpathlineto{\pgfqpoint{6.400000in}{4.800000in}}%
\pgfpathlineto{\pgfqpoint{0.000000in}{4.800000in}}%
\pgfpathclose%
\pgfusepath{fill}%
\end{pgfscope}%
\begin{pgfscope}%
\pgfsetbuttcap%
\pgfsetmiterjoin%
\definecolor{currentfill}{rgb}{1.000000,1.000000,1.000000}%
\pgfsetfillcolor{currentfill}%
\pgfsetlinewidth{0.000000pt}%
\definecolor{currentstroke}{rgb}{0.000000,0.000000,0.000000}%
\pgfsetstrokecolor{currentstroke}%
\pgfsetstrokeopacity{0.000000}%
\pgfsetdash{}{0pt}%
\pgfpathmoveto{\pgfqpoint{0.800000in}{0.528000in}}%
\pgfpathlineto{\pgfqpoint{5.760000in}{0.528000in}}%
\pgfpathlineto{\pgfqpoint{5.760000in}{4.224000in}}%
\pgfpathlineto{\pgfqpoint{0.800000in}{4.224000in}}%
\pgfpathclose%
\pgfusepath{fill}%
\end{pgfscope}%
\begin{pgfscope}%
\pgfsetbuttcap%
\pgfsetroundjoin%
\definecolor{currentfill}{rgb}{0.000000,0.000000,0.000000}%
\pgfsetfillcolor{currentfill}%
\pgfsetlinewidth{0.803000pt}%
\definecolor{currentstroke}{rgb}{0.000000,0.000000,0.000000}%
\pgfsetstrokecolor{currentstroke}%
\pgfsetdash{}{0pt}%
\pgfsys@defobject{currentmarker}{\pgfqpoint{0.000000in}{-0.048611in}}{\pgfqpoint{0.000000in}{0.000000in}}{%
\pgfpathmoveto{\pgfqpoint{0.000000in}{0.000000in}}%
\pgfpathlineto{\pgfqpoint{0.000000in}{-0.048611in}}%
\pgfusepath{stroke,fill}%
}%
\begin{pgfscope}%
\pgfsys@transformshift{1.025455in}{0.528000in}%
\pgfsys@useobject{currentmarker}{}%
\end{pgfscope}%
\end{pgfscope}%
\begin{pgfscope}%
\definecolor{textcolor}{rgb}{0.000000,0.000000,0.000000}%
\pgfsetstrokecolor{textcolor}%
\pgfsetfillcolor{textcolor}%
\pgftext[x=1.025455in,y=0.430778in,,top]{\color{textcolor}\sffamily\fontsize{10.000000}{12.000000}\selectfont \(\displaystyle 20\)}%
\end{pgfscope}%
\begin{pgfscope}%
\pgfsetbuttcap%
\pgfsetroundjoin%
\definecolor{currentfill}{rgb}{0.000000,0.000000,0.000000}%
\pgfsetfillcolor{currentfill}%
\pgfsetlinewidth{0.803000pt}%
\definecolor{currentstroke}{rgb}{0.000000,0.000000,0.000000}%
\pgfsetstrokecolor{currentstroke}%
\pgfsetdash{}{0pt}%
\pgfsys@defobject{currentmarker}{\pgfqpoint{0.000000in}{-0.048611in}}{\pgfqpoint{0.000000in}{0.000000in}}{%
\pgfpathmoveto{\pgfqpoint{0.000000in}{0.000000in}}%
\pgfpathlineto{\pgfqpoint{0.000000in}{-0.048611in}}%
\pgfusepath{stroke,fill}%
}%
\begin{pgfscope}%
\pgfsys@transformshift{2.528485in}{0.528000in}%
\pgfsys@useobject{currentmarker}{}%
\end{pgfscope}%
\end{pgfscope}%
\begin{pgfscope}%
\definecolor{textcolor}{rgb}{0.000000,0.000000,0.000000}%
\pgfsetstrokecolor{textcolor}%
\pgfsetfillcolor{textcolor}%
\pgftext[x=2.528485in,y=0.430778in,,top]{\color{textcolor}\sffamily\fontsize{10.000000}{12.000000}\selectfont \(\displaystyle 40\)}%
\end{pgfscope}%
\begin{pgfscope}%
\pgfsetbuttcap%
\pgfsetroundjoin%
\definecolor{currentfill}{rgb}{0.000000,0.000000,0.000000}%
\pgfsetfillcolor{currentfill}%
\pgfsetlinewidth{0.803000pt}%
\definecolor{currentstroke}{rgb}{0.000000,0.000000,0.000000}%
\pgfsetstrokecolor{currentstroke}%
\pgfsetdash{}{0pt}%
\pgfsys@defobject{currentmarker}{\pgfqpoint{0.000000in}{-0.048611in}}{\pgfqpoint{0.000000in}{0.000000in}}{%
\pgfpathmoveto{\pgfqpoint{0.000000in}{0.000000in}}%
\pgfpathlineto{\pgfqpoint{0.000000in}{-0.048611in}}%
\pgfusepath{stroke,fill}%
}%
\begin{pgfscope}%
\pgfsys@transformshift{4.031515in}{0.528000in}%
\pgfsys@useobject{currentmarker}{}%
\end{pgfscope}%
\end{pgfscope}%
\begin{pgfscope}%
\definecolor{textcolor}{rgb}{0.000000,0.000000,0.000000}%
\pgfsetstrokecolor{textcolor}%
\pgfsetfillcolor{textcolor}%
\pgftext[x=4.031515in,y=0.430778in,,top]{\color{textcolor}\sffamily\fontsize{10.000000}{12.000000}\selectfont \(\displaystyle 60\)}%
\end{pgfscope}%
\begin{pgfscope}%
\pgfsetbuttcap%
\pgfsetroundjoin%
\definecolor{currentfill}{rgb}{0.000000,0.000000,0.000000}%
\pgfsetfillcolor{currentfill}%
\pgfsetlinewidth{0.803000pt}%
\definecolor{currentstroke}{rgb}{0.000000,0.000000,0.000000}%
\pgfsetstrokecolor{currentstroke}%
\pgfsetdash{}{0pt}%
\pgfsys@defobject{currentmarker}{\pgfqpoint{0.000000in}{-0.048611in}}{\pgfqpoint{0.000000in}{0.000000in}}{%
\pgfpathmoveto{\pgfqpoint{0.000000in}{0.000000in}}%
\pgfpathlineto{\pgfqpoint{0.000000in}{-0.048611in}}%
\pgfusepath{stroke,fill}%
}%
\begin{pgfscope}%
\pgfsys@transformshift{5.534545in}{0.528000in}%
\pgfsys@useobject{currentmarker}{}%
\end{pgfscope}%
\end{pgfscope}%
\begin{pgfscope}%
\definecolor{textcolor}{rgb}{0.000000,0.000000,0.000000}%
\pgfsetstrokecolor{textcolor}%
\pgfsetfillcolor{textcolor}%
\pgftext[x=5.534545in,y=0.430778in,,top]{\color{textcolor}\sffamily\fontsize{10.000000}{12.000000}\selectfont \(\displaystyle 80\)}%
\end{pgfscope}%
\begin{pgfscope}%
\definecolor{textcolor}{rgb}{0.000000,0.000000,0.000000}%
\pgfsetstrokecolor{textcolor}%
\pgfsetfillcolor{textcolor}%
\pgftext[x=3.280000in,y=0.240809in,,top]{\color{textcolor}\sffamily\fontsize{10.000000}{12.000000}\selectfont \(\displaystyle k\)}%
\end{pgfscope}%
\begin{pgfscope}%
\pgfsetbuttcap%
\pgfsetroundjoin%
\definecolor{currentfill}{rgb}{0.000000,0.000000,0.000000}%
\pgfsetfillcolor{currentfill}%
\pgfsetlinewidth{0.803000pt}%
\definecolor{currentstroke}{rgb}{0.000000,0.000000,0.000000}%
\pgfsetstrokecolor{currentstroke}%
\pgfsetdash{}{0pt}%
\pgfsys@defobject{currentmarker}{\pgfqpoint{-0.048611in}{0.000000in}}{\pgfqpoint{0.000000in}{0.000000in}}{%
\pgfpathmoveto{\pgfqpoint{0.000000in}{0.000000in}}%
\pgfpathlineto{\pgfqpoint{-0.048611in}{0.000000in}}%
\pgfusepath{stroke,fill}%
}%
\begin{pgfscope}%
\pgfsys@transformshift{0.800000in}{0.696000in}%
\pgfsys@useobject{currentmarker}{}%
\end{pgfscope}%
\end{pgfscope}%
\begin{pgfscope}%
\definecolor{textcolor}{rgb}{0.000000,0.000000,0.000000}%
\pgfsetstrokecolor{textcolor}%
\pgfsetfillcolor{textcolor}%
\pgftext[x=0.633333in,y=0.643238in,left,base]{\color{textcolor}\sffamily\fontsize{10.000000}{12.000000}\selectfont \(\displaystyle 7\)}%
\end{pgfscope}%
\begin{pgfscope}%
\pgfsetbuttcap%
\pgfsetroundjoin%
\definecolor{currentfill}{rgb}{0.000000,0.000000,0.000000}%
\pgfsetfillcolor{currentfill}%
\pgfsetlinewidth{0.803000pt}%
\definecolor{currentstroke}{rgb}{0.000000,0.000000,0.000000}%
\pgfsetstrokecolor{currentstroke}%
\pgfsetdash{}{0pt}%
\pgfsys@defobject{currentmarker}{\pgfqpoint{-0.048611in}{0.000000in}}{\pgfqpoint{0.000000in}{0.000000in}}{%
\pgfpathmoveto{\pgfqpoint{0.000000in}{0.000000in}}%
\pgfpathlineto{\pgfqpoint{-0.048611in}{0.000000in}}%
\pgfusepath{stroke,fill}%
}%
\begin{pgfscope}%
\pgfsys@transformshift{0.800000in}{1.256000in}%
\pgfsys@useobject{currentmarker}{}%
\end{pgfscope}%
\end{pgfscope}%
\begin{pgfscope}%
\definecolor{textcolor}{rgb}{0.000000,0.000000,0.000000}%
\pgfsetstrokecolor{textcolor}%
\pgfsetfillcolor{textcolor}%
\pgftext[x=0.633333in,y=1.203238in,left,base]{\color{textcolor}\sffamily\fontsize{10.000000}{12.000000}\selectfont \(\displaystyle 8\)}%
\end{pgfscope}%
\begin{pgfscope}%
\pgfsetbuttcap%
\pgfsetroundjoin%
\definecolor{currentfill}{rgb}{0.000000,0.000000,0.000000}%
\pgfsetfillcolor{currentfill}%
\pgfsetlinewidth{0.803000pt}%
\definecolor{currentstroke}{rgb}{0.000000,0.000000,0.000000}%
\pgfsetstrokecolor{currentstroke}%
\pgfsetdash{}{0pt}%
\pgfsys@defobject{currentmarker}{\pgfqpoint{-0.048611in}{0.000000in}}{\pgfqpoint{0.000000in}{0.000000in}}{%
\pgfpathmoveto{\pgfqpoint{0.000000in}{0.000000in}}%
\pgfpathlineto{\pgfqpoint{-0.048611in}{0.000000in}}%
\pgfusepath{stroke,fill}%
}%
\begin{pgfscope}%
\pgfsys@transformshift{0.800000in}{1.816000in}%
\pgfsys@useobject{currentmarker}{}%
\end{pgfscope}%
\end{pgfscope}%
\begin{pgfscope}%
\definecolor{textcolor}{rgb}{0.000000,0.000000,0.000000}%
\pgfsetstrokecolor{textcolor}%
\pgfsetfillcolor{textcolor}%
\pgftext[x=0.633333in,y=1.763238in,left,base]{\color{textcolor}\sffamily\fontsize{10.000000}{12.000000}\selectfont \(\displaystyle 9\)}%
\end{pgfscope}%
\begin{pgfscope}%
\pgfsetbuttcap%
\pgfsetroundjoin%
\definecolor{currentfill}{rgb}{0.000000,0.000000,0.000000}%
\pgfsetfillcolor{currentfill}%
\pgfsetlinewidth{0.803000pt}%
\definecolor{currentstroke}{rgb}{0.000000,0.000000,0.000000}%
\pgfsetstrokecolor{currentstroke}%
\pgfsetdash{}{0pt}%
\pgfsys@defobject{currentmarker}{\pgfqpoint{-0.048611in}{0.000000in}}{\pgfqpoint{0.000000in}{0.000000in}}{%
\pgfpathmoveto{\pgfqpoint{0.000000in}{0.000000in}}%
\pgfpathlineto{\pgfqpoint{-0.048611in}{0.000000in}}%
\pgfusepath{stroke,fill}%
}%
\begin{pgfscope}%
\pgfsys@transformshift{0.800000in}{2.376000in}%
\pgfsys@useobject{currentmarker}{}%
\end{pgfscope}%
\end{pgfscope}%
\begin{pgfscope}%
\definecolor{textcolor}{rgb}{0.000000,0.000000,0.000000}%
\pgfsetstrokecolor{textcolor}%
\pgfsetfillcolor{textcolor}%
\pgftext[x=0.563888in,y=2.323238in,left,base]{\color{textcolor}\sffamily\fontsize{10.000000}{12.000000}\selectfont \(\displaystyle 10\)}%
\end{pgfscope}%
\begin{pgfscope}%
\pgfsetbuttcap%
\pgfsetroundjoin%
\definecolor{currentfill}{rgb}{0.000000,0.000000,0.000000}%
\pgfsetfillcolor{currentfill}%
\pgfsetlinewidth{0.803000pt}%
\definecolor{currentstroke}{rgb}{0.000000,0.000000,0.000000}%
\pgfsetstrokecolor{currentstroke}%
\pgfsetdash{}{0pt}%
\pgfsys@defobject{currentmarker}{\pgfqpoint{-0.048611in}{0.000000in}}{\pgfqpoint{0.000000in}{0.000000in}}{%
\pgfpathmoveto{\pgfqpoint{0.000000in}{0.000000in}}%
\pgfpathlineto{\pgfqpoint{-0.048611in}{0.000000in}}%
\pgfusepath{stroke,fill}%
}%
\begin{pgfscope}%
\pgfsys@transformshift{0.800000in}{2.936000in}%
\pgfsys@useobject{currentmarker}{}%
\end{pgfscope}%
\end{pgfscope}%
\begin{pgfscope}%
\definecolor{textcolor}{rgb}{0.000000,0.000000,0.000000}%
\pgfsetstrokecolor{textcolor}%
\pgfsetfillcolor{textcolor}%
\pgftext[x=0.563888in,y=2.883238in,left,base]{\color{textcolor}\sffamily\fontsize{10.000000}{12.000000}\selectfont \(\displaystyle 11\)}%
\end{pgfscope}%
\begin{pgfscope}%
\pgfsetbuttcap%
\pgfsetroundjoin%
\definecolor{currentfill}{rgb}{0.000000,0.000000,0.000000}%
\pgfsetfillcolor{currentfill}%
\pgfsetlinewidth{0.803000pt}%
\definecolor{currentstroke}{rgb}{0.000000,0.000000,0.000000}%
\pgfsetstrokecolor{currentstroke}%
\pgfsetdash{}{0pt}%
\pgfsys@defobject{currentmarker}{\pgfqpoint{-0.048611in}{0.000000in}}{\pgfqpoint{0.000000in}{0.000000in}}{%
\pgfpathmoveto{\pgfqpoint{0.000000in}{0.000000in}}%
\pgfpathlineto{\pgfqpoint{-0.048611in}{0.000000in}}%
\pgfusepath{stroke,fill}%
}%
\begin{pgfscope}%
\pgfsys@transformshift{0.800000in}{3.496000in}%
\pgfsys@useobject{currentmarker}{}%
\end{pgfscope}%
\end{pgfscope}%
\begin{pgfscope}%
\definecolor{textcolor}{rgb}{0.000000,0.000000,0.000000}%
\pgfsetstrokecolor{textcolor}%
\pgfsetfillcolor{textcolor}%
\pgftext[x=0.563888in,y=3.443238in,left,base]{\color{textcolor}\sffamily\fontsize{10.000000}{12.000000}\selectfont \(\displaystyle 12\)}%
\end{pgfscope}%
\begin{pgfscope}%
\pgfsetbuttcap%
\pgfsetroundjoin%
\definecolor{currentfill}{rgb}{0.000000,0.000000,0.000000}%
\pgfsetfillcolor{currentfill}%
\pgfsetlinewidth{0.803000pt}%
\definecolor{currentstroke}{rgb}{0.000000,0.000000,0.000000}%
\pgfsetstrokecolor{currentstroke}%
\pgfsetdash{}{0pt}%
\pgfsys@defobject{currentmarker}{\pgfqpoint{-0.048611in}{0.000000in}}{\pgfqpoint{0.000000in}{0.000000in}}{%
\pgfpathmoveto{\pgfqpoint{0.000000in}{0.000000in}}%
\pgfpathlineto{\pgfqpoint{-0.048611in}{0.000000in}}%
\pgfusepath{stroke,fill}%
}%
\begin{pgfscope}%
\pgfsys@transformshift{0.800000in}{4.056000in}%
\pgfsys@useobject{currentmarker}{}%
\end{pgfscope}%
\end{pgfscope}%
\begin{pgfscope}%
\definecolor{textcolor}{rgb}{0.000000,0.000000,0.000000}%
\pgfsetstrokecolor{textcolor}%
\pgfsetfillcolor{textcolor}%
\pgftext[x=0.563888in,y=4.003238in,left,base]{\color{textcolor}\sffamily\fontsize{10.000000}{12.000000}\selectfont \(\displaystyle 13\)}%
\end{pgfscope}%
\begin{pgfscope}%
\definecolor{textcolor}{rgb}{0.000000,0.000000,0.000000}%
\pgfsetstrokecolor{textcolor}%
\pgfsetfillcolor{textcolor}%
\pgftext[x=0.508333in,y=2.376000in,,bottom,rotate=90.000000]{\color{textcolor}\sffamily\fontsize{10.000000}{12.000000}\selectfont Maximum Number of GMRES Iterations}%
\end{pgfscope}%
\begin{pgfscope}%
\pgfpathrectangle{\pgfqpoint{0.800000in}{0.528000in}}{\pgfqpoint{4.960000in}{3.696000in}}%
\pgfusepath{clip}%
\pgfsetbuttcap%
\pgfsetroundjoin%
\pgfsetlinewidth{1.505625pt}%
\definecolor{currentstroke}{rgb}{0.843137,0.000000,0.000000}%
\pgfsetstrokecolor{currentstroke}%
\pgfsetdash{{5.550000pt}{2.400000pt}}{0.000000pt}%
\pgfpathmoveto{\pgfqpoint{1.025455in}{2.376000in}}%
\pgfpathlineto{\pgfqpoint{2.528485in}{3.496000in}}%
\pgfpathlineto{\pgfqpoint{4.031515in}{4.056000in}}%
\pgfpathlineto{\pgfqpoint{5.534545in}{4.056000in}}%
\pgfusepath{stroke}%
\end{pgfscope}%
\begin{pgfscope}%
\pgfpathrectangle{\pgfqpoint{0.800000in}{0.528000in}}{\pgfqpoint{4.960000in}{3.696000in}}%
\pgfusepath{clip}%
\pgfsetbuttcap%
\pgfsetroundjoin%
\definecolor{currentfill}{rgb}{0.843137,0.000000,0.000000}%
\pgfsetfillcolor{currentfill}%
\pgfsetlinewidth{1.003750pt}%
\definecolor{currentstroke}{rgb}{0.843137,0.000000,0.000000}%
\pgfsetstrokecolor{currentstroke}%
\pgfsetdash{}{0pt}%
\pgfsys@defobject{currentmarker}{\pgfqpoint{-0.041667in}{-0.041667in}}{\pgfqpoint{0.041667in}{0.041667in}}{%
\pgfpathmoveto{\pgfqpoint{0.000000in}{-0.041667in}}%
\pgfpathcurveto{\pgfqpoint{0.011050in}{-0.041667in}}{\pgfqpoint{0.021649in}{-0.037276in}}{\pgfqpoint{0.029463in}{-0.029463in}}%
\pgfpathcurveto{\pgfqpoint{0.037276in}{-0.021649in}}{\pgfqpoint{0.041667in}{-0.011050in}}{\pgfqpoint{0.041667in}{0.000000in}}%
\pgfpathcurveto{\pgfqpoint{0.041667in}{0.011050in}}{\pgfqpoint{0.037276in}{0.021649in}}{\pgfqpoint{0.029463in}{0.029463in}}%
\pgfpathcurveto{\pgfqpoint{0.021649in}{0.037276in}}{\pgfqpoint{0.011050in}{0.041667in}}{\pgfqpoint{0.000000in}{0.041667in}}%
\pgfpathcurveto{\pgfqpoint{-0.011050in}{0.041667in}}{\pgfqpoint{-0.021649in}{0.037276in}}{\pgfqpoint{-0.029463in}{0.029463in}}%
\pgfpathcurveto{\pgfqpoint{-0.037276in}{0.021649in}}{\pgfqpoint{-0.041667in}{0.011050in}}{\pgfqpoint{-0.041667in}{0.000000in}}%
\pgfpathcurveto{\pgfqpoint{-0.041667in}{-0.011050in}}{\pgfqpoint{-0.037276in}{-0.021649in}}{\pgfqpoint{-0.029463in}{-0.029463in}}%
\pgfpathcurveto{\pgfqpoint{-0.021649in}{-0.037276in}}{\pgfqpoint{-0.011050in}{-0.041667in}}{\pgfqpoint{0.000000in}{-0.041667in}}%
\pgfpathclose%
\pgfusepath{stroke,fill}%
}%
\begin{pgfscope}%
\pgfsys@transformshift{1.025455in}{2.376000in}%
\pgfsys@useobject{currentmarker}{}%
\end{pgfscope}%
\begin{pgfscope}%
\pgfsys@transformshift{2.528485in}{3.496000in}%
\pgfsys@useobject{currentmarker}{}%
\end{pgfscope}%
\begin{pgfscope}%
\pgfsys@transformshift{4.031515in}{4.056000in}%
\pgfsys@useobject{currentmarker}{}%
\end{pgfscope}%
\begin{pgfscope}%
\pgfsys@transformshift{5.534545in}{4.056000in}%
\pgfsys@useobject{currentmarker}{}%
\end{pgfscope}%
\end{pgfscope}%
\begin{pgfscope}%
\pgfpathrectangle{\pgfqpoint{0.800000in}{0.528000in}}{\pgfqpoint{4.960000in}{3.696000in}}%
\pgfusepath{clip}%
\pgfsetbuttcap%
\pgfsetroundjoin%
\pgfsetlinewidth{1.505625pt}%
\definecolor{currentstroke}{rgb}{0.549020,0.235294,1.000000}%
\pgfsetstrokecolor{currentstroke}%
\pgfsetdash{{5.550000pt}{2.400000pt}}{0.000000pt}%
\pgfpathmoveto{\pgfqpoint{1.025455in}{1.816000in}}%
\pgfpathlineto{\pgfqpoint{2.528485in}{2.376000in}}%
\pgfpathlineto{\pgfqpoint{4.031515in}{2.376000in}}%
\pgfpathlineto{\pgfqpoint{5.534545in}{2.376000in}}%
\pgfusepath{stroke}%
\end{pgfscope}%
\begin{pgfscope}%
\pgfpathrectangle{\pgfqpoint{0.800000in}{0.528000in}}{\pgfqpoint{4.960000in}{3.696000in}}%
\pgfusepath{clip}%
\pgfsetbuttcap%
\pgfsetmiterjoin%
\definecolor{currentfill}{rgb}{0.549020,0.235294,1.000000}%
\pgfsetfillcolor{currentfill}%
\pgfsetlinewidth{1.003750pt}%
\definecolor{currentstroke}{rgb}{0.549020,0.235294,1.000000}%
\pgfsetstrokecolor{currentstroke}%
\pgfsetdash{}{0pt}%
\pgfsys@defobject{currentmarker}{\pgfqpoint{-0.041667in}{-0.041667in}}{\pgfqpoint{0.041667in}{0.041667in}}{%
\pgfpathmoveto{\pgfqpoint{0.000000in}{0.041667in}}%
\pgfpathlineto{\pgfqpoint{-0.041667in}{-0.041667in}}%
\pgfpathlineto{\pgfqpoint{0.041667in}{-0.041667in}}%
\pgfpathclose%
\pgfusepath{stroke,fill}%
}%
\begin{pgfscope}%
\pgfsys@transformshift{1.025455in}{1.816000in}%
\pgfsys@useobject{currentmarker}{}%
\end{pgfscope}%
\begin{pgfscope}%
\pgfsys@transformshift{2.528485in}{2.376000in}%
\pgfsys@useobject{currentmarker}{}%
\end{pgfscope}%
\begin{pgfscope}%
\pgfsys@transformshift{4.031515in}{2.376000in}%
\pgfsys@useobject{currentmarker}{}%
\end{pgfscope}%
\begin{pgfscope}%
\pgfsys@transformshift{5.534545in}{2.376000in}%
\pgfsys@useobject{currentmarker}{}%
\end{pgfscope}%
\end{pgfscope}%
\begin{pgfscope}%
\pgfpathrectangle{\pgfqpoint{0.800000in}{0.528000in}}{\pgfqpoint{4.960000in}{3.696000in}}%
\pgfusepath{clip}%
\pgfsetbuttcap%
\pgfsetroundjoin%
\pgfsetlinewidth{1.505625pt}%
\definecolor{currentstroke}{rgb}{0.007843,0.533333,0.000000}%
\pgfsetstrokecolor{currentstroke}%
\pgfsetdash{{5.550000pt}{2.400000pt}}{0.000000pt}%
\pgfpathmoveto{\pgfqpoint{1.025455in}{1.256000in}}%
\pgfpathlineto{\pgfqpoint{2.528485in}{1.256000in}}%
\pgfpathlineto{\pgfqpoint{4.031515in}{1.256000in}}%
\pgfpathlineto{\pgfqpoint{5.534545in}{1.256000in}}%
\pgfusepath{stroke}%
\end{pgfscope}%
\begin{pgfscope}%
\pgfpathrectangle{\pgfqpoint{0.800000in}{0.528000in}}{\pgfqpoint{4.960000in}{3.696000in}}%
\pgfusepath{clip}%
\pgfsetbuttcap%
\pgfsetmiterjoin%
\definecolor{currentfill}{rgb}{0.007843,0.533333,0.000000}%
\pgfsetfillcolor{currentfill}%
\pgfsetlinewidth{1.003750pt}%
\definecolor{currentstroke}{rgb}{0.007843,0.533333,0.000000}%
\pgfsetstrokecolor{currentstroke}%
\pgfsetdash{}{0pt}%
\pgfsys@defobject{currentmarker}{\pgfqpoint{-0.041667in}{-0.041667in}}{\pgfqpoint{0.041667in}{0.041667in}}{%
\pgfpathmoveto{\pgfqpoint{-0.000000in}{-0.041667in}}%
\pgfpathlineto{\pgfqpoint{0.041667in}{0.041667in}}%
\pgfpathlineto{\pgfqpoint{-0.041667in}{0.041667in}}%
\pgfpathclose%
\pgfusepath{stroke,fill}%
}%
\begin{pgfscope}%
\pgfsys@transformshift{1.025455in}{1.256000in}%
\pgfsys@useobject{currentmarker}{}%
\end{pgfscope}%
\begin{pgfscope}%
\pgfsys@transformshift{2.528485in}{1.256000in}%
\pgfsys@useobject{currentmarker}{}%
\end{pgfscope}%
\begin{pgfscope}%
\pgfsys@transformshift{4.031515in}{1.256000in}%
\pgfsys@useobject{currentmarker}{}%
\end{pgfscope}%
\begin{pgfscope}%
\pgfsys@transformshift{5.534545in}{1.256000in}%
\pgfsys@useobject{currentmarker}{}%
\end{pgfscope}%
\end{pgfscope}%
\begin{pgfscope}%
\pgfpathrectangle{\pgfqpoint{0.800000in}{0.528000in}}{\pgfqpoint{4.960000in}{3.696000in}}%
\pgfusepath{clip}%
\pgfsetbuttcap%
\pgfsetroundjoin%
\pgfsetlinewidth{1.505625pt}%
\definecolor{currentstroke}{rgb}{0.000000,0.674510,0.780392}%
\pgfsetstrokecolor{currentstroke}%
\pgfsetdash{{5.550000pt}{2.400000pt}}{0.000000pt}%
\pgfpathmoveto{\pgfqpoint{1.025455in}{0.696000in}}%
\pgfpathlineto{\pgfqpoint{2.528485in}{0.696000in}}%
\pgfpathlineto{\pgfqpoint{4.031515in}{0.696000in}}%
\pgfpathlineto{\pgfqpoint{5.534545in}{0.696000in}}%
\pgfusepath{stroke}%
\end{pgfscope}%
\begin{pgfscope}%
\pgfpathrectangle{\pgfqpoint{0.800000in}{0.528000in}}{\pgfqpoint{4.960000in}{3.696000in}}%
\pgfusepath{clip}%
\pgfsetbuttcap%
\pgfsetmiterjoin%
\definecolor{currentfill}{rgb}{0.000000,0.674510,0.780392}%
\pgfsetfillcolor{currentfill}%
\pgfsetlinewidth{1.003750pt}%
\definecolor{currentstroke}{rgb}{0.000000,0.674510,0.780392}%
\pgfsetstrokecolor{currentstroke}%
\pgfsetdash{}{0pt}%
\pgfsys@defobject{currentmarker}{\pgfqpoint{-0.041667in}{-0.041667in}}{\pgfqpoint{0.041667in}{0.041667in}}{%
\pgfpathmoveto{\pgfqpoint{0.041667in}{-0.000000in}}%
\pgfpathlineto{\pgfqpoint{-0.041667in}{0.041667in}}%
\pgfpathlineto{\pgfqpoint{-0.041667in}{-0.041667in}}%
\pgfpathclose%
\pgfusepath{stroke,fill}%
}%
\begin{pgfscope}%
\pgfsys@transformshift{1.025455in}{0.696000in}%
\pgfsys@useobject{currentmarker}{}%
\end{pgfscope}%
\begin{pgfscope}%
\pgfsys@transformshift{2.528485in}{0.696000in}%
\pgfsys@useobject{currentmarker}{}%
\end{pgfscope}%
\begin{pgfscope}%
\pgfsys@transformshift{4.031515in}{0.696000in}%
\pgfsys@useobject{currentmarker}{}%
\end{pgfscope}%
\begin{pgfscope}%
\pgfsys@transformshift{5.534545in}{0.696000in}%
\pgfsys@useobject{currentmarker}{}%
\end{pgfscope}%
\end{pgfscope}%
\begin{pgfscope}%
\pgfsetrectcap%
\pgfsetmiterjoin%
\pgfsetlinewidth{0.803000pt}%
\definecolor{currentstroke}{rgb}{0.000000,0.000000,0.000000}%
\pgfsetstrokecolor{currentstroke}%
\pgfsetdash{}{0pt}%
\pgfpathmoveto{\pgfqpoint{0.800000in}{0.528000in}}%
\pgfpathlineto{\pgfqpoint{0.800000in}{4.224000in}}%
\pgfusepath{stroke}%
\end{pgfscope}%
\begin{pgfscope}%
\pgfsetrectcap%
\pgfsetmiterjoin%
\pgfsetlinewidth{0.803000pt}%
\definecolor{currentstroke}{rgb}{0.000000,0.000000,0.000000}%
\pgfsetstrokecolor{currentstroke}%
\pgfsetdash{}{0pt}%
\pgfpathmoveto{\pgfqpoint{5.760000in}{0.528000in}}%
\pgfpathlineto{\pgfqpoint{5.760000in}{4.224000in}}%
\pgfusepath{stroke}%
\end{pgfscope}%
\begin{pgfscope}%
\pgfsetrectcap%
\pgfsetmiterjoin%
\pgfsetlinewidth{0.803000pt}%
\definecolor{currentstroke}{rgb}{0.000000,0.000000,0.000000}%
\pgfsetstrokecolor{currentstroke}%
\pgfsetdash{}{0pt}%
\pgfpathmoveto{\pgfqpoint{0.800000in}{0.528000in}}%
\pgfpathlineto{\pgfqpoint{5.760000in}{0.528000in}}%
\pgfusepath{stroke}%
\end{pgfscope}%
\begin{pgfscope}%
\pgfsetrectcap%
\pgfsetmiterjoin%
\pgfsetlinewidth{0.803000pt}%
\definecolor{currentstroke}{rgb}{0.000000,0.000000,0.000000}%
\pgfsetstrokecolor{currentstroke}%
\pgfsetdash{}{0pt}%
\pgfpathmoveto{\pgfqpoint{0.800000in}{4.224000in}}%
\pgfpathlineto{\pgfqpoint{5.760000in}{4.224000in}}%
\pgfusepath{stroke}%
\end{pgfscope}%
\begin{pgfscope}%
\pgfsetbuttcap%
\pgfsetmiterjoin%
\definecolor{currentfill}{rgb}{1.000000,1.000000,1.000000}%
\pgfsetfillcolor{currentfill}%
\pgfsetfillopacity{0.800000}%
\pgfsetlinewidth{1.003750pt}%
\definecolor{currentstroke}{rgb}{0.800000,0.800000,0.800000}%
\pgfsetstrokecolor{currentstroke}%
\pgfsetstrokeopacity{0.800000}%
\pgfsetdash{}{0pt}%
\pgfpathmoveto{\pgfqpoint{0.897222in}{3.297460in}}%
\pgfpathlineto{\pgfqpoint{1.795040in}{3.297460in}}%
\pgfpathquadraticcurveto{\pgfqpoint{1.822817in}{3.297460in}}{\pgfqpoint{1.822817in}{3.325238in}}%
\pgfpathlineto{\pgfqpoint{1.822817in}{4.126778in}}%
\pgfpathquadraticcurveto{\pgfqpoint{1.822817in}{4.154556in}}{\pgfqpoint{1.795040in}{4.154556in}}%
\pgfpathlineto{\pgfqpoint{0.897222in}{4.154556in}}%
\pgfpathquadraticcurveto{\pgfqpoint{0.869444in}{4.154556in}}{\pgfqpoint{0.869444in}{4.126778in}}%
\pgfpathlineto{\pgfqpoint{0.869444in}{3.325238in}}%
\pgfpathquadraticcurveto{\pgfqpoint{0.869444in}{3.297460in}}{\pgfqpoint{0.897222in}{3.297460in}}%
\pgfpathclose%
\pgfusepath{stroke,fill}%
\end{pgfscope}%
\begin{pgfscope}%
\pgfsetbuttcap%
\pgfsetroundjoin%
\pgfsetlinewidth{1.505625pt}%
\definecolor{currentstroke}{rgb}{0.843137,0.000000,0.000000}%
\pgfsetstrokecolor{currentstroke}%
\pgfsetdash{{5.550000pt}{2.400000pt}}{0.000000pt}%
\pgfpathmoveto{\pgfqpoint{0.925000in}{4.042088in}}%
\pgfpathlineto{\pgfqpoint{1.202778in}{4.042088in}}%
\pgfusepath{stroke}%
\end{pgfscope}%
\begin{pgfscope}%
\pgfsetbuttcap%
\pgfsetroundjoin%
\definecolor{currentfill}{rgb}{0.843137,0.000000,0.000000}%
\pgfsetfillcolor{currentfill}%
\pgfsetlinewidth{1.003750pt}%
\definecolor{currentstroke}{rgb}{0.843137,0.000000,0.000000}%
\pgfsetstrokecolor{currentstroke}%
\pgfsetdash{}{0pt}%
\pgfsys@defobject{currentmarker}{\pgfqpoint{-0.041667in}{-0.041667in}}{\pgfqpoint{0.041667in}{0.041667in}}{%
\pgfpathmoveto{\pgfqpoint{0.000000in}{-0.041667in}}%
\pgfpathcurveto{\pgfqpoint{0.011050in}{-0.041667in}}{\pgfqpoint{0.021649in}{-0.037276in}}{\pgfqpoint{0.029463in}{-0.029463in}}%
\pgfpathcurveto{\pgfqpoint{0.037276in}{-0.021649in}}{\pgfqpoint{0.041667in}{-0.011050in}}{\pgfqpoint{0.041667in}{0.000000in}}%
\pgfpathcurveto{\pgfqpoint{0.041667in}{0.011050in}}{\pgfqpoint{0.037276in}{0.021649in}}{\pgfqpoint{0.029463in}{0.029463in}}%
\pgfpathcurveto{\pgfqpoint{0.021649in}{0.037276in}}{\pgfqpoint{0.011050in}{0.041667in}}{\pgfqpoint{0.000000in}{0.041667in}}%
\pgfpathcurveto{\pgfqpoint{-0.011050in}{0.041667in}}{\pgfqpoint{-0.021649in}{0.037276in}}{\pgfqpoint{-0.029463in}{0.029463in}}%
\pgfpathcurveto{\pgfqpoint{-0.037276in}{0.021649in}}{\pgfqpoint{-0.041667in}{0.011050in}}{\pgfqpoint{-0.041667in}{0.000000in}}%
\pgfpathcurveto{\pgfqpoint{-0.041667in}{-0.011050in}}{\pgfqpoint{-0.037276in}{-0.021649in}}{\pgfqpoint{-0.029463in}{-0.029463in}}%
\pgfpathcurveto{\pgfqpoint{-0.021649in}{-0.037276in}}{\pgfqpoint{-0.011050in}{-0.041667in}}{\pgfqpoint{0.000000in}{-0.041667in}}%
\pgfpathclose%
\pgfusepath{stroke,fill}%
}%
\begin{pgfscope}%
\pgfsys@transformshift{1.063889in}{4.042088in}%
\pgfsys@useobject{currentmarker}{}%
\end{pgfscope}%
\end{pgfscope}%
\begin{pgfscope}%
\definecolor{textcolor}{rgb}{0.000000,0.000000,0.000000}%
\pgfsetstrokecolor{textcolor}%
\pgfsetfillcolor{textcolor}%
\pgftext[x=1.313889in,y=3.993477in,left,base]{\color{textcolor}\sffamily\fontsize{10.000000}{12.000000}\selectfont \(\displaystyle \beta = \)0.4}%
\end{pgfscope}%
\begin{pgfscope}%
\pgfsetbuttcap%
\pgfsetroundjoin%
\pgfsetlinewidth{1.505625pt}%
\definecolor{currentstroke}{rgb}{0.549020,0.235294,1.000000}%
\pgfsetstrokecolor{currentstroke}%
\pgfsetdash{{5.550000pt}{2.400000pt}}{0.000000pt}%
\pgfpathmoveto{\pgfqpoint{0.925000in}{3.838231in}}%
\pgfpathlineto{\pgfqpoint{1.202778in}{3.838231in}}%
\pgfusepath{stroke}%
\end{pgfscope}%
\begin{pgfscope}%
\pgfsetbuttcap%
\pgfsetmiterjoin%
\definecolor{currentfill}{rgb}{0.549020,0.235294,1.000000}%
\pgfsetfillcolor{currentfill}%
\pgfsetlinewidth{1.003750pt}%
\definecolor{currentstroke}{rgb}{0.549020,0.235294,1.000000}%
\pgfsetstrokecolor{currentstroke}%
\pgfsetdash{}{0pt}%
\pgfsys@defobject{currentmarker}{\pgfqpoint{-0.041667in}{-0.041667in}}{\pgfqpoint{0.041667in}{0.041667in}}{%
\pgfpathmoveto{\pgfqpoint{0.000000in}{0.041667in}}%
\pgfpathlineto{\pgfqpoint{-0.041667in}{-0.041667in}}%
\pgfpathlineto{\pgfqpoint{0.041667in}{-0.041667in}}%
\pgfpathclose%
\pgfusepath{stroke,fill}%
}%
\begin{pgfscope}%
\pgfsys@transformshift{1.063889in}{3.838231in}%
\pgfsys@useobject{currentmarker}{}%
\end{pgfscope}%
\end{pgfscope}%
\begin{pgfscope}%
\definecolor{textcolor}{rgb}{0.000000,0.000000,0.000000}%
\pgfsetstrokecolor{textcolor}%
\pgfsetfillcolor{textcolor}%
\pgftext[x=1.313889in,y=3.789620in,left,base]{\color{textcolor}\sffamily\fontsize{10.000000}{12.000000}\selectfont \(\displaystyle \beta = \)0.5}%
\end{pgfscope}%
\begin{pgfscope}%
\pgfsetbuttcap%
\pgfsetroundjoin%
\pgfsetlinewidth{1.505625pt}%
\definecolor{currentstroke}{rgb}{0.007843,0.533333,0.000000}%
\pgfsetstrokecolor{currentstroke}%
\pgfsetdash{{5.550000pt}{2.400000pt}}{0.000000pt}%
\pgfpathmoveto{\pgfqpoint{0.925000in}{3.634374in}}%
\pgfpathlineto{\pgfqpoint{1.202778in}{3.634374in}}%
\pgfusepath{stroke}%
\end{pgfscope}%
\begin{pgfscope}%
\pgfsetbuttcap%
\pgfsetmiterjoin%
\definecolor{currentfill}{rgb}{0.007843,0.533333,0.000000}%
\pgfsetfillcolor{currentfill}%
\pgfsetlinewidth{1.003750pt}%
\definecolor{currentstroke}{rgb}{0.007843,0.533333,0.000000}%
\pgfsetstrokecolor{currentstroke}%
\pgfsetdash{}{0pt}%
\pgfsys@defobject{currentmarker}{\pgfqpoint{-0.041667in}{-0.041667in}}{\pgfqpoint{0.041667in}{0.041667in}}{%
\pgfpathmoveto{\pgfqpoint{-0.000000in}{-0.041667in}}%
\pgfpathlineto{\pgfqpoint{0.041667in}{0.041667in}}%
\pgfpathlineto{\pgfqpoint{-0.041667in}{0.041667in}}%
\pgfpathclose%
\pgfusepath{stroke,fill}%
}%
\begin{pgfscope}%
\pgfsys@transformshift{1.063889in}{3.634374in}%
\pgfsys@useobject{currentmarker}{}%
\end{pgfscope}%
\end{pgfscope}%
\begin{pgfscope}%
\definecolor{textcolor}{rgb}{0.000000,0.000000,0.000000}%
\pgfsetstrokecolor{textcolor}%
\pgfsetfillcolor{textcolor}%
\pgftext[x=1.313889in,y=3.585762in,left,base]{\color{textcolor}\sffamily\fontsize{10.000000}{12.000000}\selectfont \(\displaystyle \beta = \)0.6}%
\end{pgfscope}%
\begin{pgfscope}%
\pgfsetbuttcap%
\pgfsetroundjoin%
\pgfsetlinewidth{1.505625pt}%
\definecolor{currentstroke}{rgb}{0.000000,0.674510,0.780392}%
\pgfsetstrokecolor{currentstroke}%
\pgfsetdash{{5.550000pt}{2.400000pt}}{0.000000pt}%
\pgfpathmoveto{\pgfqpoint{0.925000in}{3.430516in}}%
\pgfpathlineto{\pgfqpoint{1.202778in}{3.430516in}}%
\pgfusepath{stroke}%
\end{pgfscope}%
\begin{pgfscope}%
\pgfsetbuttcap%
\pgfsetmiterjoin%
\definecolor{currentfill}{rgb}{0.000000,0.674510,0.780392}%
\pgfsetfillcolor{currentfill}%
\pgfsetlinewidth{1.003750pt}%
\definecolor{currentstroke}{rgb}{0.000000,0.674510,0.780392}%
\pgfsetstrokecolor{currentstroke}%
\pgfsetdash{}{0pt}%
\pgfsys@defobject{currentmarker}{\pgfqpoint{-0.041667in}{-0.041667in}}{\pgfqpoint{0.041667in}{0.041667in}}{%
\pgfpathmoveto{\pgfqpoint{0.041667in}{-0.000000in}}%
\pgfpathlineto{\pgfqpoint{-0.041667in}{0.041667in}}%
\pgfpathlineto{\pgfqpoint{-0.041667in}{-0.041667in}}%
\pgfpathclose%
\pgfusepath{stroke,fill}%
}%
\begin{pgfscope}%
\pgfsys@transformshift{1.063889in}{3.430516in}%
\pgfsys@useobject{currentmarker}{}%
\end{pgfscope}%
\end{pgfscope}%
\begin{pgfscope}%
\definecolor{textcolor}{rgb}{0.000000,0.000000,0.000000}%
\pgfsetstrokecolor{textcolor}%
\pgfsetfillcolor{textcolor}%
\pgftext[x=1.313889in,y=3.381905in,left,base]{\color{textcolor}\sffamily\fontsize{10.000000}{12.000000}\selectfont \(\displaystyle \beta = \)0.7}%
\end{pgfscope}%
\end{pgfpicture}%
\makeatother%
\endgroup%

   \caption[Maximum GMRES iteration counts when $\NLiDRRdtd{\Aso-\Ast} = 0.5\times  k^{\beta}$ for $\beta = 0.4,0.5,0.6,0.7.$]{Maximum GMRES iteration counts for solving systems with matrix $\AmatoI\Amatt$, where $\nso=\nst=1$ and $\NLiDRRdtd{\Aso-\Ast} = 0.5\times  k^{\beta}$ for $\beta = 0.4,0.5 ,0.6,0.7.$}\label{fig:linfinityA1}
\end{figure}

    \begin{figure}
      \centering
%% Creator: Matplotlib, PGF backend
%%
%% To include the figure in your LaTeX document, write
%%   \input{<filename>.pgf}
%%
%% Make sure the required packages are loaded in your preamble
%%   \usepackage{pgf}
%%
%% Figures using additional raster images can only be included by \input if
%% they are in the same directory as the main LaTeX file. For loading figures
%% from other directories you can use the `import` package
%%   \usepackage{import}
%% and then include the figures with
%%   \import{<path to file>}{<filename>.pgf}
%%
%% Matplotlib used the following preamble
%%   \usepackage{fontspec}
%%   \setmainfont{DejaVuSerif.ttf}[Path=/home/owen/progs/firedrake-complex/firedrake/lib/python3.5/site-packages/matplotlib/mpl-data/fonts/ttf/]
%%   \setsansfont{DejaVuSans.ttf}[Path=/home/owen/progs/firedrake-complex/firedrake/lib/python3.5/site-packages/matplotlib/mpl-data/fonts/ttf/]
%%   \setmonofont{DejaVuSansMono.ttf}[Path=/home/owen/progs/firedrake-complex/firedrake/lib/python3.5/site-packages/matplotlib/mpl-data/fonts/ttf/]
%%
\begingroup%
\makeatletter%
\begin{pgfpicture}%
\pgfpathrectangle{\pgfpointorigin}{\pgfqpoint{6.400000in}{4.800000in}}%
\pgfusepath{use as bounding box, clip}%
\begin{pgfscope}%
\pgfsetbuttcap%
\pgfsetmiterjoin%
\definecolor{currentfill}{rgb}{1.000000,1.000000,1.000000}%
\pgfsetfillcolor{currentfill}%
\pgfsetlinewidth{0.000000pt}%
\definecolor{currentstroke}{rgb}{1.000000,1.000000,1.000000}%
\pgfsetstrokecolor{currentstroke}%
\pgfsetdash{}{0pt}%
\pgfpathmoveto{\pgfqpoint{0.000000in}{0.000000in}}%
\pgfpathlineto{\pgfqpoint{6.400000in}{0.000000in}}%
\pgfpathlineto{\pgfqpoint{6.400000in}{4.800000in}}%
\pgfpathlineto{\pgfqpoint{0.000000in}{4.800000in}}%
\pgfpathclose%
\pgfusepath{fill}%
\end{pgfscope}%
\begin{pgfscope}%
\pgfsetbuttcap%
\pgfsetmiterjoin%
\definecolor{currentfill}{rgb}{1.000000,1.000000,1.000000}%
\pgfsetfillcolor{currentfill}%
\pgfsetlinewidth{0.000000pt}%
\definecolor{currentstroke}{rgb}{0.000000,0.000000,0.000000}%
\pgfsetstrokecolor{currentstroke}%
\pgfsetstrokeopacity{0.000000}%
\pgfsetdash{}{0pt}%
\pgfpathmoveto{\pgfqpoint{0.800000in}{0.528000in}}%
\pgfpathlineto{\pgfqpoint{5.760000in}{0.528000in}}%
\pgfpathlineto{\pgfqpoint{5.760000in}{4.224000in}}%
\pgfpathlineto{\pgfqpoint{0.800000in}{4.224000in}}%
\pgfpathclose%
\pgfusepath{fill}%
\end{pgfscope}%
\begin{pgfscope}%
\pgfsetbuttcap%
\pgfsetroundjoin%
\definecolor{currentfill}{rgb}{0.000000,0.000000,0.000000}%
\pgfsetfillcolor{currentfill}%
\pgfsetlinewidth{0.803000pt}%
\definecolor{currentstroke}{rgb}{0.000000,0.000000,0.000000}%
\pgfsetstrokecolor{currentstroke}%
\pgfsetdash{}{0pt}%
\pgfsys@defobject{currentmarker}{\pgfqpoint{0.000000in}{-0.048611in}}{\pgfqpoint{0.000000in}{0.000000in}}{%
\pgfpathmoveto{\pgfqpoint{0.000000in}{0.000000in}}%
\pgfpathlineto{\pgfqpoint{0.000000in}{-0.048611in}}%
\pgfusepath{stroke,fill}%
}%
\begin{pgfscope}%
\pgfsys@transformshift{1.025455in}{0.528000in}%
\pgfsys@useobject{currentmarker}{}%
\end{pgfscope}%
\end{pgfscope}%
\begin{pgfscope}%
\definecolor{textcolor}{rgb}{0.000000,0.000000,0.000000}%
\pgfsetstrokecolor{textcolor}%
\pgfsetfillcolor{textcolor}%
\pgftext[x=1.025455in,y=0.430778in,,top]{\color{textcolor}\sffamily\fontsize{10.000000}{12.000000}\selectfont 20}%
\end{pgfscope}%
\begin{pgfscope}%
\pgfsetbuttcap%
\pgfsetroundjoin%
\definecolor{currentfill}{rgb}{0.000000,0.000000,0.000000}%
\pgfsetfillcolor{currentfill}%
\pgfsetlinewidth{0.803000pt}%
\definecolor{currentstroke}{rgb}{0.000000,0.000000,0.000000}%
\pgfsetstrokecolor{currentstroke}%
\pgfsetdash{}{0pt}%
\pgfsys@defobject{currentmarker}{\pgfqpoint{0.000000in}{-0.048611in}}{\pgfqpoint{0.000000in}{0.000000in}}{%
\pgfpathmoveto{\pgfqpoint{0.000000in}{0.000000in}}%
\pgfpathlineto{\pgfqpoint{0.000000in}{-0.048611in}}%
\pgfusepath{stroke,fill}%
}%
\begin{pgfscope}%
\pgfsys@transformshift{2.528485in}{0.528000in}%
\pgfsys@useobject{currentmarker}{}%
\end{pgfscope}%
\end{pgfscope}%
\begin{pgfscope}%
\definecolor{textcolor}{rgb}{0.000000,0.000000,0.000000}%
\pgfsetstrokecolor{textcolor}%
\pgfsetfillcolor{textcolor}%
\pgftext[x=2.528485in,y=0.430778in,,top]{\color{textcolor}\sffamily\fontsize{10.000000}{12.000000}\selectfont 40}%
\end{pgfscope}%
\begin{pgfscope}%
\pgfsetbuttcap%
\pgfsetroundjoin%
\definecolor{currentfill}{rgb}{0.000000,0.000000,0.000000}%
\pgfsetfillcolor{currentfill}%
\pgfsetlinewidth{0.803000pt}%
\definecolor{currentstroke}{rgb}{0.000000,0.000000,0.000000}%
\pgfsetstrokecolor{currentstroke}%
\pgfsetdash{}{0pt}%
\pgfsys@defobject{currentmarker}{\pgfqpoint{0.000000in}{-0.048611in}}{\pgfqpoint{0.000000in}{0.000000in}}{%
\pgfpathmoveto{\pgfqpoint{0.000000in}{0.000000in}}%
\pgfpathlineto{\pgfqpoint{0.000000in}{-0.048611in}}%
\pgfusepath{stroke,fill}%
}%
\begin{pgfscope}%
\pgfsys@transformshift{4.031515in}{0.528000in}%
\pgfsys@useobject{currentmarker}{}%
\end{pgfscope}%
\end{pgfscope}%
\begin{pgfscope}%
\definecolor{textcolor}{rgb}{0.000000,0.000000,0.000000}%
\pgfsetstrokecolor{textcolor}%
\pgfsetfillcolor{textcolor}%
\pgftext[x=4.031515in,y=0.430778in,,top]{\color{textcolor}\sffamily\fontsize{10.000000}{12.000000}\selectfont 60}%
\end{pgfscope}%
\begin{pgfscope}%
\pgfsetbuttcap%
\pgfsetroundjoin%
\definecolor{currentfill}{rgb}{0.000000,0.000000,0.000000}%
\pgfsetfillcolor{currentfill}%
\pgfsetlinewidth{0.803000pt}%
\definecolor{currentstroke}{rgb}{0.000000,0.000000,0.000000}%
\pgfsetstrokecolor{currentstroke}%
\pgfsetdash{}{0pt}%
\pgfsys@defobject{currentmarker}{\pgfqpoint{0.000000in}{-0.048611in}}{\pgfqpoint{0.000000in}{0.000000in}}{%
\pgfpathmoveto{\pgfqpoint{0.000000in}{0.000000in}}%
\pgfpathlineto{\pgfqpoint{0.000000in}{-0.048611in}}%
\pgfusepath{stroke,fill}%
}%
\begin{pgfscope}%
\pgfsys@transformshift{5.534545in}{0.528000in}%
\pgfsys@useobject{currentmarker}{}%
\end{pgfscope}%
\end{pgfscope}%
\begin{pgfscope}%
\definecolor{textcolor}{rgb}{0.000000,0.000000,0.000000}%
\pgfsetstrokecolor{textcolor}%
\pgfsetfillcolor{textcolor}%
\pgftext[x=5.534545in,y=0.430778in,,top]{\color{textcolor}\sffamily\fontsize{10.000000}{12.000000}\selectfont 80}%
\end{pgfscope}%
\begin{pgfscope}%
\definecolor{textcolor}{rgb}{0.000000,0.000000,0.000000}%
\pgfsetstrokecolor{textcolor}%
\pgfsetfillcolor{textcolor}%
\pgftext[x=3.280000in,y=0.240809in,,top]{\color{textcolor}\sffamily\fontsize{10.000000}{12.000000}\selectfont \(\displaystyle k\)}%
\end{pgfscope}%
\begin{pgfscope}%
\pgfsetbuttcap%
\pgfsetroundjoin%
\definecolor{currentfill}{rgb}{0.000000,0.000000,0.000000}%
\pgfsetfillcolor{currentfill}%
\pgfsetlinewidth{0.803000pt}%
\definecolor{currentstroke}{rgb}{0.000000,0.000000,0.000000}%
\pgfsetstrokecolor{currentstroke}%
\pgfsetdash{}{0pt}%
\pgfsys@defobject{currentmarker}{\pgfqpoint{-0.048611in}{0.000000in}}{\pgfqpoint{0.000000in}{0.000000in}}{%
\pgfpathmoveto{\pgfqpoint{0.000000in}{0.000000in}}%
\pgfpathlineto{\pgfqpoint{-0.048611in}{0.000000in}}%
\pgfusepath{stroke,fill}%
}%
\begin{pgfscope}%
\pgfsys@transformshift{0.800000in}{0.696000in}%
\pgfsys@useobject{currentmarker}{}%
\end{pgfscope}%
\end{pgfscope}%
\begin{pgfscope}%
\definecolor{textcolor}{rgb}{0.000000,0.000000,0.000000}%
\pgfsetstrokecolor{textcolor}%
\pgfsetfillcolor{textcolor}%
\pgftext[x=0.614413in,y=0.643238in,left,base]{\color{textcolor}\sffamily\fontsize{10.000000}{12.000000}\selectfont 5}%
\end{pgfscope}%
\begin{pgfscope}%
\pgfsetbuttcap%
\pgfsetroundjoin%
\definecolor{currentfill}{rgb}{0.000000,0.000000,0.000000}%
\pgfsetfillcolor{currentfill}%
\pgfsetlinewidth{0.803000pt}%
\definecolor{currentstroke}{rgb}{0.000000,0.000000,0.000000}%
\pgfsetstrokecolor{currentstroke}%
\pgfsetdash{}{0pt}%
\pgfsys@defobject{currentmarker}{\pgfqpoint{-0.048611in}{0.000000in}}{\pgfqpoint{0.000000in}{0.000000in}}{%
\pgfpathmoveto{\pgfqpoint{0.000000in}{0.000000in}}%
\pgfpathlineto{\pgfqpoint{-0.048611in}{0.000000in}}%
\pgfusepath{stroke,fill}%
}%
\begin{pgfscope}%
\pgfsys@transformshift{0.800000in}{4.056000in}%
\pgfsys@useobject{currentmarker}{}%
\end{pgfscope}%
\end{pgfscope}%
\begin{pgfscope}%
\definecolor{textcolor}{rgb}{0.000000,0.000000,0.000000}%
\pgfsetstrokecolor{textcolor}%
\pgfsetfillcolor{textcolor}%
\pgftext[x=0.614413in,y=4.003238in,left,base]{\color{textcolor}\sffamily\fontsize{10.000000}{12.000000}\selectfont 6}%
\end{pgfscope}%
\begin{pgfscope}%
\definecolor{textcolor}{rgb}{0.000000,0.000000,0.000000}%
\pgfsetstrokecolor{textcolor}%
\pgfsetfillcolor{textcolor}%
\pgftext[x=0.558857in,y=2.376000in,,bottom,rotate=90.000000]{\color{textcolor}\sffamily\fontsize{10.000000}{12.000000}\selectfont Number of GMRES Iterations}%
\end{pgfscope}%
\begin{pgfscope}%
\pgfpathrectangle{\pgfqpoint{0.800000in}{0.528000in}}{\pgfqpoint{4.960000in}{3.696000in}}%
\pgfusepath{clip}%
\pgfsetbuttcap%
\pgfsetroundjoin%
\pgfsetlinewidth{1.505625pt}%
\definecolor{currentstroke}{rgb}{0.000000,0.000000,0.000000}%
\pgfsetstrokecolor{currentstroke}%
\pgfsetdash{{5.550000pt}{2.400000pt}}{0.000000pt}%
\pgfpathmoveto{\pgfqpoint{1.025455in}{4.056000in}}%
\pgfpathlineto{\pgfqpoint{2.528485in}{4.056000in}}%
\pgfpathlineto{\pgfqpoint{4.031515in}{4.056000in}}%
\pgfpathlineto{\pgfqpoint{5.534545in}{4.056000in}}%
\pgfusepath{stroke}%
\end{pgfscope}%
\begin{pgfscope}%
\pgfpathrectangle{\pgfqpoint{0.800000in}{0.528000in}}{\pgfqpoint{4.960000in}{3.696000in}}%
\pgfusepath{clip}%
\pgfsetbuttcap%
\pgfsetroundjoin%
\definecolor{currentfill}{rgb}{0.000000,0.000000,0.000000}%
\pgfsetfillcolor{currentfill}%
\pgfsetlinewidth{1.003750pt}%
\definecolor{currentstroke}{rgb}{0.000000,0.000000,0.000000}%
\pgfsetstrokecolor{currentstroke}%
\pgfsetdash{}{0pt}%
\pgfsys@defobject{currentmarker}{\pgfqpoint{-0.041667in}{-0.041667in}}{\pgfqpoint{0.041667in}{0.041667in}}{%
\pgfpathmoveto{\pgfqpoint{0.000000in}{-0.041667in}}%
\pgfpathcurveto{\pgfqpoint{0.011050in}{-0.041667in}}{\pgfqpoint{0.021649in}{-0.037276in}}{\pgfqpoint{0.029463in}{-0.029463in}}%
\pgfpathcurveto{\pgfqpoint{0.037276in}{-0.021649in}}{\pgfqpoint{0.041667in}{-0.011050in}}{\pgfqpoint{0.041667in}{0.000000in}}%
\pgfpathcurveto{\pgfqpoint{0.041667in}{0.011050in}}{\pgfqpoint{0.037276in}{0.021649in}}{\pgfqpoint{0.029463in}{0.029463in}}%
\pgfpathcurveto{\pgfqpoint{0.021649in}{0.037276in}}{\pgfqpoint{0.011050in}{0.041667in}}{\pgfqpoint{0.000000in}{0.041667in}}%
\pgfpathcurveto{\pgfqpoint{-0.011050in}{0.041667in}}{\pgfqpoint{-0.021649in}{0.037276in}}{\pgfqpoint{-0.029463in}{0.029463in}}%
\pgfpathcurveto{\pgfqpoint{-0.037276in}{0.021649in}}{\pgfqpoint{-0.041667in}{0.011050in}}{\pgfqpoint{-0.041667in}{0.000000in}}%
\pgfpathcurveto{\pgfqpoint{-0.041667in}{-0.011050in}}{\pgfqpoint{-0.037276in}{-0.021649in}}{\pgfqpoint{-0.029463in}{-0.029463in}}%
\pgfpathcurveto{\pgfqpoint{-0.021649in}{-0.037276in}}{\pgfqpoint{-0.011050in}{-0.041667in}}{\pgfqpoint{0.000000in}{-0.041667in}}%
\pgfpathclose%
\pgfusepath{stroke,fill}%
}%
\begin{pgfscope}%
\pgfsys@transformshift{1.025455in}{4.056000in}%
\pgfsys@useobject{currentmarker}{}%
\end{pgfscope}%
\begin{pgfscope}%
\pgfsys@transformshift{2.528485in}{4.056000in}%
\pgfsys@useobject{currentmarker}{}%
\end{pgfscope}%
\begin{pgfscope}%
\pgfsys@transformshift{4.031515in}{4.056000in}%
\pgfsys@useobject{currentmarker}{}%
\end{pgfscope}%
\begin{pgfscope}%
\pgfsys@transformshift{5.534545in}{4.056000in}%
\pgfsys@useobject{currentmarker}{}%
\end{pgfscope}%
\end{pgfscope}%
\begin{pgfscope}%
\pgfpathrectangle{\pgfqpoint{0.800000in}{0.528000in}}{\pgfqpoint{4.960000in}{3.696000in}}%
\pgfusepath{clip}%
\pgfsetbuttcap%
\pgfsetroundjoin%
\pgfsetlinewidth{1.505625pt}%
\definecolor{currentstroke}{rgb}{0.000000,0.000000,0.000000}%
\pgfsetstrokecolor{currentstroke}%
\pgfsetdash{{5.550000pt}{2.400000pt}}{0.000000pt}%
\pgfpathmoveto{\pgfqpoint{1.025455in}{4.056000in}}%
\pgfpathlineto{\pgfqpoint{2.528485in}{4.056000in}}%
\pgfpathlineto{\pgfqpoint{4.031515in}{0.696000in}}%
\pgfpathlineto{\pgfqpoint{5.534545in}{0.696000in}}%
\pgfusepath{stroke}%
\end{pgfscope}%
\begin{pgfscope}%
\pgfpathrectangle{\pgfqpoint{0.800000in}{0.528000in}}{\pgfqpoint{4.960000in}{3.696000in}}%
\pgfusepath{clip}%
\pgfsetbuttcap%
\pgfsetmiterjoin%
\definecolor{currentfill}{rgb}{0.000000,0.000000,0.000000}%
\pgfsetfillcolor{currentfill}%
\pgfsetlinewidth{1.003750pt}%
\definecolor{currentstroke}{rgb}{0.000000,0.000000,0.000000}%
\pgfsetstrokecolor{currentstroke}%
\pgfsetdash{}{0pt}%
\pgfsys@defobject{currentmarker}{\pgfqpoint{-0.041667in}{-0.041667in}}{\pgfqpoint{0.041667in}{0.041667in}}{%
\pgfpathmoveto{\pgfqpoint{-0.000000in}{-0.041667in}}%
\pgfpathlineto{\pgfqpoint{0.041667in}{0.041667in}}%
\pgfpathlineto{\pgfqpoint{-0.041667in}{0.041667in}}%
\pgfpathclose%
\pgfusepath{stroke,fill}%
}%
\begin{pgfscope}%
\pgfsys@transformshift{1.025455in}{4.056000in}%
\pgfsys@useobject{currentmarker}{}%
\end{pgfscope}%
\begin{pgfscope}%
\pgfsys@transformshift{2.528485in}{4.056000in}%
\pgfsys@useobject{currentmarker}{}%
\end{pgfscope}%
\begin{pgfscope}%
\pgfsys@transformshift{4.031515in}{0.696000in}%
\pgfsys@useobject{currentmarker}{}%
\end{pgfscope}%
\begin{pgfscope}%
\pgfsys@transformshift{5.534545in}{0.696000in}%
\pgfsys@useobject{currentmarker}{}%
\end{pgfscope}%
\end{pgfscope}%
\begin{pgfscope}%
\pgfpathrectangle{\pgfqpoint{0.800000in}{0.528000in}}{\pgfqpoint{4.960000in}{3.696000in}}%
\pgfusepath{clip}%
\pgfsetbuttcap%
\pgfsetroundjoin%
\pgfsetlinewidth{1.505625pt}%
\definecolor{currentstroke}{rgb}{0.000000,0.000000,0.000000}%
\pgfsetstrokecolor{currentstroke}%
\pgfsetdash{{5.550000pt}{2.400000pt}}{0.000000pt}%
\pgfpathmoveto{\pgfqpoint{1.025455in}{0.696000in}}%
\pgfpathlineto{\pgfqpoint{2.528485in}{0.696000in}}%
\pgfpathlineto{\pgfqpoint{4.031515in}{0.696000in}}%
\pgfpathlineto{\pgfqpoint{5.534545in}{0.696000in}}%
\pgfusepath{stroke}%
\end{pgfscope}%
\begin{pgfscope}%
\pgfpathrectangle{\pgfqpoint{0.800000in}{0.528000in}}{\pgfqpoint{4.960000in}{3.696000in}}%
\pgfusepath{clip}%
\pgfsetbuttcap%
\pgfsetmiterjoin%
\definecolor{currentfill}{rgb}{0.000000,0.000000,0.000000}%
\pgfsetfillcolor{currentfill}%
\pgfsetlinewidth{1.003750pt}%
\definecolor{currentstroke}{rgb}{0.000000,0.000000,0.000000}%
\pgfsetstrokecolor{currentstroke}%
\pgfsetdash{}{0pt}%
\pgfsys@defobject{currentmarker}{\pgfqpoint{-0.041667in}{-0.041667in}}{\pgfqpoint{0.041667in}{0.041667in}}{%
\pgfpathmoveto{\pgfqpoint{-0.020833in}{-0.041667in}}%
\pgfpathlineto{\pgfqpoint{0.000000in}{-0.020833in}}%
\pgfpathlineto{\pgfqpoint{0.020833in}{-0.041667in}}%
\pgfpathlineto{\pgfqpoint{0.041667in}{-0.020833in}}%
\pgfpathlineto{\pgfqpoint{0.020833in}{0.000000in}}%
\pgfpathlineto{\pgfqpoint{0.041667in}{0.020833in}}%
\pgfpathlineto{\pgfqpoint{0.020833in}{0.041667in}}%
\pgfpathlineto{\pgfqpoint{0.000000in}{0.020833in}}%
\pgfpathlineto{\pgfqpoint{-0.020833in}{0.041667in}}%
\pgfpathlineto{\pgfqpoint{-0.041667in}{0.020833in}}%
\pgfpathlineto{\pgfqpoint{-0.020833in}{0.000000in}}%
\pgfpathlineto{\pgfqpoint{-0.041667in}{-0.020833in}}%
\pgfpathclose%
\pgfusepath{stroke,fill}%
}%
\begin{pgfscope}%
\pgfsys@transformshift{1.025455in}{0.696000in}%
\pgfsys@useobject{currentmarker}{}%
\end{pgfscope}%
\begin{pgfscope}%
\pgfsys@transformshift{2.528485in}{0.696000in}%
\pgfsys@useobject{currentmarker}{}%
\end{pgfscope}%
\begin{pgfscope}%
\pgfsys@transformshift{4.031515in}{0.696000in}%
\pgfsys@useobject{currentmarker}{}%
\end{pgfscope}%
\begin{pgfscope}%
\pgfsys@transformshift{5.534545in}{0.696000in}%
\pgfsys@useobject{currentmarker}{}%
\end{pgfscope}%
\end{pgfscope}%
\begin{pgfscope}%
\pgfsetrectcap%
\pgfsetmiterjoin%
\pgfsetlinewidth{0.803000pt}%
\definecolor{currentstroke}{rgb}{0.000000,0.000000,0.000000}%
\pgfsetstrokecolor{currentstroke}%
\pgfsetdash{}{0pt}%
\pgfpathmoveto{\pgfqpoint{0.800000in}{0.528000in}}%
\pgfpathlineto{\pgfqpoint{0.800000in}{4.224000in}}%
\pgfusepath{stroke}%
\end{pgfscope}%
\begin{pgfscope}%
\pgfsetrectcap%
\pgfsetmiterjoin%
\pgfsetlinewidth{0.803000pt}%
\definecolor{currentstroke}{rgb}{0.000000,0.000000,0.000000}%
\pgfsetstrokecolor{currentstroke}%
\pgfsetdash{}{0pt}%
\pgfpathmoveto{\pgfqpoint{5.760000in}{0.528000in}}%
\pgfpathlineto{\pgfqpoint{5.760000in}{4.224000in}}%
\pgfusepath{stroke}%
\end{pgfscope}%
\begin{pgfscope}%
\pgfsetrectcap%
\pgfsetmiterjoin%
\pgfsetlinewidth{0.803000pt}%
\definecolor{currentstroke}{rgb}{0.000000,0.000000,0.000000}%
\pgfsetstrokecolor{currentstroke}%
\pgfsetdash{}{0pt}%
\pgfpathmoveto{\pgfqpoint{0.800000in}{0.528000in}}%
\pgfpathlineto{\pgfqpoint{5.760000in}{0.528000in}}%
\pgfusepath{stroke}%
\end{pgfscope}%
\begin{pgfscope}%
\pgfsetrectcap%
\pgfsetmiterjoin%
\pgfsetlinewidth{0.803000pt}%
\definecolor{currentstroke}{rgb}{0.000000,0.000000,0.000000}%
\pgfsetstrokecolor{currentstroke}%
\pgfsetdash{}{0pt}%
\pgfpathmoveto{\pgfqpoint{0.800000in}{4.224000in}}%
\pgfpathlineto{\pgfqpoint{5.760000in}{4.224000in}}%
\pgfusepath{stroke}%
\end{pgfscope}%
\begin{pgfscope}%
\pgfsetbuttcap%
\pgfsetmiterjoin%
\definecolor{currentfill}{rgb}{1.000000,1.000000,1.000000}%
\pgfsetfillcolor{currentfill}%
\pgfsetfillopacity{0.800000}%
\pgfsetlinewidth{1.003750pt}%
\definecolor{currentstroke}{rgb}{0.800000,0.800000,0.800000}%
\pgfsetstrokecolor{currentstroke}%
\pgfsetstrokeopacity{0.800000}%
\pgfsetdash{}{0pt}%
\pgfpathmoveto{\pgfqpoint{4.764960in}{2.049381in}}%
\pgfpathlineto{\pgfqpoint{5.662778in}{2.049381in}}%
\pgfpathquadraticcurveto{\pgfqpoint{5.690556in}{2.049381in}}{\pgfqpoint{5.690556in}{2.077159in}}%
\pgfpathlineto{\pgfqpoint{5.690556in}{2.674841in}}%
\pgfpathquadraticcurveto{\pgfqpoint{5.690556in}{2.702619in}}{\pgfqpoint{5.662778in}{2.702619in}}%
\pgfpathlineto{\pgfqpoint{4.764960in}{2.702619in}}%
\pgfpathquadraticcurveto{\pgfqpoint{4.737183in}{2.702619in}}{\pgfqpoint{4.737183in}{2.674841in}}%
\pgfpathlineto{\pgfqpoint{4.737183in}{2.077159in}}%
\pgfpathquadraticcurveto{\pgfqpoint{4.737183in}{2.049381in}}{\pgfqpoint{4.764960in}{2.049381in}}%
\pgfpathclose%
\pgfusepath{stroke,fill}%
\end{pgfscope}%
\begin{pgfscope}%
\pgfsetbuttcap%
\pgfsetroundjoin%
\pgfsetlinewidth{1.505625pt}%
\definecolor{currentstroke}{rgb}{0.000000,0.000000,0.000000}%
\pgfsetstrokecolor{currentstroke}%
\pgfsetdash{{5.550000pt}{2.400000pt}}{0.000000pt}%
\pgfpathmoveto{\pgfqpoint{4.792738in}{2.590152in}}%
\pgfpathlineto{\pgfqpoint{5.070516in}{2.590152in}}%
\pgfusepath{stroke}%
\end{pgfscope}%
\begin{pgfscope}%
\pgfsetbuttcap%
\pgfsetroundjoin%
\definecolor{currentfill}{rgb}{0.000000,0.000000,0.000000}%
\pgfsetfillcolor{currentfill}%
\pgfsetlinewidth{1.003750pt}%
\definecolor{currentstroke}{rgb}{0.000000,0.000000,0.000000}%
\pgfsetstrokecolor{currentstroke}%
\pgfsetdash{}{0pt}%
\pgfsys@defobject{currentmarker}{\pgfqpoint{-0.041667in}{-0.041667in}}{\pgfqpoint{0.041667in}{0.041667in}}{%
\pgfpathmoveto{\pgfqpoint{0.000000in}{-0.041667in}}%
\pgfpathcurveto{\pgfqpoint{0.011050in}{-0.041667in}}{\pgfqpoint{0.021649in}{-0.037276in}}{\pgfqpoint{0.029463in}{-0.029463in}}%
\pgfpathcurveto{\pgfqpoint{0.037276in}{-0.021649in}}{\pgfqpoint{0.041667in}{-0.011050in}}{\pgfqpoint{0.041667in}{0.000000in}}%
\pgfpathcurveto{\pgfqpoint{0.041667in}{0.011050in}}{\pgfqpoint{0.037276in}{0.021649in}}{\pgfqpoint{0.029463in}{0.029463in}}%
\pgfpathcurveto{\pgfqpoint{0.021649in}{0.037276in}}{\pgfqpoint{0.011050in}{0.041667in}}{\pgfqpoint{0.000000in}{0.041667in}}%
\pgfpathcurveto{\pgfqpoint{-0.011050in}{0.041667in}}{\pgfqpoint{-0.021649in}{0.037276in}}{\pgfqpoint{-0.029463in}{0.029463in}}%
\pgfpathcurveto{\pgfqpoint{-0.037276in}{0.021649in}}{\pgfqpoint{-0.041667in}{0.011050in}}{\pgfqpoint{-0.041667in}{0.000000in}}%
\pgfpathcurveto{\pgfqpoint{-0.041667in}{-0.011050in}}{\pgfqpoint{-0.037276in}{-0.021649in}}{\pgfqpoint{-0.029463in}{-0.029463in}}%
\pgfpathcurveto{\pgfqpoint{-0.021649in}{-0.037276in}}{\pgfqpoint{-0.011050in}{-0.041667in}}{\pgfqpoint{0.000000in}{-0.041667in}}%
\pgfpathclose%
\pgfusepath{stroke,fill}%
}%
\begin{pgfscope}%
\pgfsys@transformshift{4.931627in}{2.590152in}%
\pgfsys@useobject{currentmarker}{}%
\end{pgfscope}%
\end{pgfscope}%
\begin{pgfscope}%
\definecolor{textcolor}{rgb}{0.000000,0.000000,0.000000}%
\pgfsetstrokecolor{textcolor}%
\pgfsetfillcolor{textcolor}%
\pgftext[x=5.181627in,y=2.541541in,left,base]{\color{textcolor}\sffamily\fontsize{10.000000}{12.000000}\selectfont \(\displaystyle \beta = \)0.8}%
\end{pgfscope}%
\begin{pgfscope}%
\pgfsetbuttcap%
\pgfsetroundjoin%
\pgfsetlinewidth{1.505625pt}%
\definecolor{currentstroke}{rgb}{0.000000,0.000000,0.000000}%
\pgfsetstrokecolor{currentstroke}%
\pgfsetdash{{5.550000pt}{2.400000pt}}{0.000000pt}%
\pgfpathmoveto{\pgfqpoint{4.792738in}{2.386294in}}%
\pgfpathlineto{\pgfqpoint{5.070516in}{2.386294in}}%
\pgfusepath{stroke}%
\end{pgfscope}%
\begin{pgfscope}%
\pgfsetbuttcap%
\pgfsetmiterjoin%
\definecolor{currentfill}{rgb}{0.000000,0.000000,0.000000}%
\pgfsetfillcolor{currentfill}%
\pgfsetlinewidth{1.003750pt}%
\definecolor{currentstroke}{rgb}{0.000000,0.000000,0.000000}%
\pgfsetstrokecolor{currentstroke}%
\pgfsetdash{}{0pt}%
\pgfsys@defobject{currentmarker}{\pgfqpoint{-0.041667in}{-0.041667in}}{\pgfqpoint{0.041667in}{0.041667in}}{%
\pgfpathmoveto{\pgfqpoint{-0.000000in}{-0.041667in}}%
\pgfpathlineto{\pgfqpoint{0.041667in}{0.041667in}}%
\pgfpathlineto{\pgfqpoint{-0.041667in}{0.041667in}}%
\pgfpathclose%
\pgfusepath{stroke,fill}%
}%
\begin{pgfscope}%
\pgfsys@transformshift{4.931627in}{2.386294in}%
\pgfsys@useobject{currentmarker}{}%
\end{pgfscope}%
\end{pgfscope}%
\begin{pgfscope}%
\definecolor{textcolor}{rgb}{0.000000,0.000000,0.000000}%
\pgfsetstrokecolor{textcolor}%
\pgfsetfillcolor{textcolor}%
\pgftext[x=5.181627in,y=2.337683in,left,base]{\color{textcolor}\sffamily\fontsize{10.000000}{12.000000}\selectfont \(\displaystyle \beta = \)0.9}%
\end{pgfscope}%
\begin{pgfscope}%
\pgfsetbuttcap%
\pgfsetroundjoin%
\pgfsetlinewidth{1.505625pt}%
\definecolor{currentstroke}{rgb}{0.000000,0.000000,0.000000}%
\pgfsetstrokecolor{currentstroke}%
\pgfsetdash{{5.550000pt}{2.400000pt}}{0.000000pt}%
\pgfpathmoveto{\pgfqpoint{4.792738in}{2.182437in}}%
\pgfpathlineto{\pgfqpoint{5.070516in}{2.182437in}}%
\pgfusepath{stroke}%
\end{pgfscope}%
\begin{pgfscope}%
\pgfsetbuttcap%
\pgfsetmiterjoin%
\definecolor{currentfill}{rgb}{0.000000,0.000000,0.000000}%
\pgfsetfillcolor{currentfill}%
\pgfsetlinewidth{1.003750pt}%
\definecolor{currentstroke}{rgb}{0.000000,0.000000,0.000000}%
\pgfsetstrokecolor{currentstroke}%
\pgfsetdash{}{0pt}%
\pgfsys@defobject{currentmarker}{\pgfqpoint{-0.041667in}{-0.041667in}}{\pgfqpoint{0.041667in}{0.041667in}}{%
\pgfpathmoveto{\pgfqpoint{-0.020833in}{-0.041667in}}%
\pgfpathlineto{\pgfqpoint{0.000000in}{-0.020833in}}%
\pgfpathlineto{\pgfqpoint{0.020833in}{-0.041667in}}%
\pgfpathlineto{\pgfqpoint{0.041667in}{-0.020833in}}%
\pgfpathlineto{\pgfqpoint{0.020833in}{0.000000in}}%
\pgfpathlineto{\pgfqpoint{0.041667in}{0.020833in}}%
\pgfpathlineto{\pgfqpoint{0.020833in}{0.041667in}}%
\pgfpathlineto{\pgfqpoint{0.000000in}{0.020833in}}%
\pgfpathlineto{\pgfqpoint{-0.020833in}{0.041667in}}%
\pgfpathlineto{\pgfqpoint{-0.041667in}{0.020833in}}%
\pgfpathlineto{\pgfqpoint{-0.020833in}{0.000000in}}%
\pgfpathlineto{\pgfqpoint{-0.041667in}{-0.020833in}}%
\pgfpathclose%
\pgfusepath{stroke,fill}%
}%
\begin{pgfscope}%
\pgfsys@transformshift{4.931627in}{2.182437in}%
\pgfsys@useobject{currentmarker}{}%
\end{pgfscope}%
\end{pgfscope}%
\begin{pgfscope}%
\definecolor{textcolor}{rgb}{0.000000,0.000000,0.000000}%
\pgfsetstrokecolor{textcolor}%
\pgfsetfillcolor{textcolor}%
\pgftext[x=5.181627in,y=2.133826in,left,base]{\color{textcolor}\sffamily\fontsize{10.000000}{12.000000}\selectfont \(\displaystyle \beta = \)1}%
\end{pgfscope}%
\end{pgfpicture}%
\makeatother%
\endgroup%

    \caption[Maximum GMRES iteration counts when $\NLiDRRdtd{\Aso-\Ast} = 0.5\times  k^{\beta}$ for $\beta = 0.8,0.9,1.$]{Maximum GMRES iteration counts for solving systems with matrix $\AmatoI\Amatt$, where $\nso=\nst=1$ and $\NLiDRRdtd{\Aso-\Ast} = 0.5\times  k^{\beta}$ for $\beta = 0.8,0.9,1.$}\label{fig:linfinityA2}
\end{figure}

    \begin{figure}
      \centering
%% Creator: Matplotlib, PGF backend
%%
%% To include the figure in your LaTeX document, write
%%   \input{<filename>.pgf}
%%
%% Make sure the required packages are loaded in your preamble
%%   \usepackage{pgf}
%%
%% Figures using additional raster images can only be included by \input if
%% they are in the same directory as the main LaTeX file. For loading figures
%% from other directories you can use the `import` package
%%   \usepackage{import}
%% and then include the figures with
%%   \import{<path to file>}{<filename>.pgf}
%%
%% Matplotlib used the following preamble
%%   \usepackage{fontspec}
%%   \setmainfont{DejaVuSerif.ttf}[Path=/home/owen/progs/firedrake-complex/firedrake/lib/python3.5/site-packages/matplotlib/mpl-data/fonts/ttf/]
%%   \setsansfont{DejaVuSans.ttf}[Path=/home/owen/progs/firedrake-complex/firedrake/lib/python3.5/site-packages/matplotlib/mpl-data/fonts/ttf/]
%%   \setmonofont{DejaVuSansMono.ttf}[Path=/home/owen/progs/firedrake-complex/firedrake/lib/python3.5/site-packages/matplotlib/mpl-data/fonts/ttf/]
%%
\begingroup%
\makeatletter%
\begin{pgfpicture}%
\pgfpathrectangle{\pgfpointorigin}{\pgfqpoint{5.500000in}{5.500000in}}%
\pgfusepath{use as bounding box, clip}%
\begin{pgfscope}%
\pgfsetbuttcap%
\pgfsetmiterjoin%
\definecolor{currentfill}{rgb}{1.000000,1.000000,1.000000}%
\pgfsetfillcolor{currentfill}%
\pgfsetlinewidth{0.000000pt}%
\definecolor{currentstroke}{rgb}{1.000000,1.000000,1.000000}%
\pgfsetstrokecolor{currentstroke}%
\pgfsetdash{}{0pt}%
\pgfpathmoveto{\pgfqpoint{0.000000in}{0.000000in}}%
\pgfpathlineto{\pgfqpoint{5.500000in}{0.000000in}}%
\pgfpathlineto{\pgfqpoint{5.500000in}{5.500000in}}%
\pgfpathlineto{\pgfqpoint{0.000000in}{5.500000in}}%
\pgfpathclose%
\pgfusepath{fill}%
\end{pgfscope}%
\begin{pgfscope}%
\pgfsetbuttcap%
\pgfsetmiterjoin%
\definecolor{currentfill}{rgb}{1.000000,1.000000,1.000000}%
\pgfsetfillcolor{currentfill}%
\pgfsetlinewidth{0.000000pt}%
\definecolor{currentstroke}{rgb}{0.000000,0.000000,0.000000}%
\pgfsetstrokecolor{currentstroke}%
\pgfsetstrokeopacity{0.000000}%
\pgfsetdash{}{0pt}%
\pgfpathmoveto{\pgfqpoint{0.687500in}{0.605000in}}%
\pgfpathlineto{\pgfqpoint{4.950000in}{0.605000in}}%
\pgfpathlineto{\pgfqpoint{4.950000in}{4.840000in}}%
\pgfpathlineto{\pgfqpoint{0.687500in}{4.840000in}}%
\pgfpathclose%
\pgfusepath{fill}%
\end{pgfscope}%
\begin{pgfscope}%
\pgfsetbuttcap%
\pgfsetroundjoin%
\definecolor{currentfill}{rgb}{0.000000,0.000000,0.000000}%
\pgfsetfillcolor{currentfill}%
\pgfsetlinewidth{0.803000pt}%
\definecolor{currentstroke}{rgb}{0.000000,0.000000,0.000000}%
\pgfsetstrokecolor{currentstroke}%
\pgfsetdash{}{0pt}%
\pgfsys@defobject{currentmarker}{\pgfqpoint{0.000000in}{-0.048611in}}{\pgfqpoint{0.000000in}{0.000000in}}{%
\pgfpathmoveto{\pgfqpoint{0.000000in}{0.000000in}}%
\pgfpathlineto{\pgfqpoint{0.000000in}{-0.048611in}}%
\pgfusepath{stroke,fill}%
}%
\begin{pgfscope}%
\pgfsys@transformshift{0.881250in}{0.605000in}%
\pgfsys@useobject{currentmarker}{}%
\end{pgfscope}%
\end{pgfscope}%
\begin{pgfscope}%
\definecolor{textcolor}{rgb}{0.000000,0.000000,0.000000}%
\pgfsetstrokecolor{textcolor}%
\pgfsetfillcolor{textcolor}%
\pgftext[x=0.881250in,y=0.507778in,,top]{\color{textcolor}\sffamily\fontsize{11.000000}{13.200000}\selectfont \(\displaystyle 20\)}%
\end{pgfscope}%
\begin{pgfscope}%
\pgfsetbuttcap%
\pgfsetroundjoin%
\definecolor{currentfill}{rgb}{0.000000,0.000000,0.000000}%
\pgfsetfillcolor{currentfill}%
\pgfsetlinewidth{0.803000pt}%
\definecolor{currentstroke}{rgb}{0.000000,0.000000,0.000000}%
\pgfsetstrokecolor{currentstroke}%
\pgfsetdash{}{0pt}%
\pgfsys@defobject{currentmarker}{\pgfqpoint{0.000000in}{-0.048611in}}{\pgfqpoint{0.000000in}{0.000000in}}{%
\pgfpathmoveto{\pgfqpoint{0.000000in}{0.000000in}}%
\pgfpathlineto{\pgfqpoint{0.000000in}{-0.048611in}}%
\pgfusepath{stroke,fill}%
}%
\begin{pgfscope}%
\pgfsys@transformshift{2.172917in}{0.605000in}%
\pgfsys@useobject{currentmarker}{}%
\end{pgfscope}%
\end{pgfscope}%
\begin{pgfscope}%
\definecolor{textcolor}{rgb}{0.000000,0.000000,0.000000}%
\pgfsetstrokecolor{textcolor}%
\pgfsetfillcolor{textcolor}%
\pgftext[x=2.172917in,y=0.507778in,,top]{\color{textcolor}\sffamily\fontsize{11.000000}{13.200000}\selectfont \(\displaystyle 40\)}%
\end{pgfscope}%
\begin{pgfscope}%
\pgfsetbuttcap%
\pgfsetroundjoin%
\definecolor{currentfill}{rgb}{0.000000,0.000000,0.000000}%
\pgfsetfillcolor{currentfill}%
\pgfsetlinewidth{0.803000pt}%
\definecolor{currentstroke}{rgb}{0.000000,0.000000,0.000000}%
\pgfsetstrokecolor{currentstroke}%
\pgfsetdash{}{0pt}%
\pgfsys@defobject{currentmarker}{\pgfqpoint{0.000000in}{-0.048611in}}{\pgfqpoint{0.000000in}{0.000000in}}{%
\pgfpathmoveto{\pgfqpoint{0.000000in}{0.000000in}}%
\pgfpathlineto{\pgfqpoint{0.000000in}{-0.048611in}}%
\pgfusepath{stroke,fill}%
}%
\begin{pgfscope}%
\pgfsys@transformshift{3.464583in}{0.605000in}%
\pgfsys@useobject{currentmarker}{}%
\end{pgfscope}%
\end{pgfscope}%
\begin{pgfscope}%
\definecolor{textcolor}{rgb}{0.000000,0.000000,0.000000}%
\pgfsetstrokecolor{textcolor}%
\pgfsetfillcolor{textcolor}%
\pgftext[x=3.464583in,y=0.507778in,,top]{\color{textcolor}\sffamily\fontsize{11.000000}{13.200000}\selectfont \(\displaystyle 60\)}%
\end{pgfscope}%
\begin{pgfscope}%
\pgfsetbuttcap%
\pgfsetroundjoin%
\definecolor{currentfill}{rgb}{0.000000,0.000000,0.000000}%
\pgfsetfillcolor{currentfill}%
\pgfsetlinewidth{0.803000pt}%
\definecolor{currentstroke}{rgb}{0.000000,0.000000,0.000000}%
\pgfsetstrokecolor{currentstroke}%
\pgfsetdash{}{0pt}%
\pgfsys@defobject{currentmarker}{\pgfqpoint{0.000000in}{-0.048611in}}{\pgfqpoint{0.000000in}{0.000000in}}{%
\pgfpathmoveto{\pgfqpoint{0.000000in}{0.000000in}}%
\pgfpathlineto{\pgfqpoint{0.000000in}{-0.048611in}}%
\pgfusepath{stroke,fill}%
}%
\begin{pgfscope}%
\pgfsys@transformshift{4.756250in}{0.605000in}%
\pgfsys@useobject{currentmarker}{}%
\end{pgfscope}%
\end{pgfscope}%
\begin{pgfscope}%
\definecolor{textcolor}{rgb}{0.000000,0.000000,0.000000}%
\pgfsetstrokecolor{textcolor}%
\pgfsetfillcolor{textcolor}%
\pgftext[x=4.756250in,y=0.507778in,,top]{\color{textcolor}\sffamily\fontsize{11.000000}{13.200000}\selectfont \(\displaystyle 80\)}%
\end{pgfscope}%
\begin{pgfscope}%
\definecolor{textcolor}{rgb}{0.000000,0.000000,0.000000}%
\pgfsetstrokecolor{textcolor}%
\pgfsetfillcolor{textcolor}%
\pgftext[x=2.818750in,y=0.304368in,,top]{\color{textcolor}\sffamily\fontsize{11.000000}{13.200000}\selectfont \(\displaystyle k\)}%
\end{pgfscope}%
\begin{pgfscope}%
\pgfsetbuttcap%
\pgfsetroundjoin%
\definecolor{currentfill}{rgb}{0.000000,0.000000,0.000000}%
\pgfsetfillcolor{currentfill}%
\pgfsetlinewidth{0.803000pt}%
\definecolor{currentstroke}{rgb}{0.000000,0.000000,0.000000}%
\pgfsetstrokecolor{currentstroke}%
\pgfsetdash{}{0pt}%
\pgfsys@defobject{currentmarker}{\pgfqpoint{-0.048611in}{0.000000in}}{\pgfqpoint{0.000000in}{0.000000in}}{%
\pgfpathmoveto{\pgfqpoint{0.000000in}{0.000000in}}%
\pgfpathlineto{\pgfqpoint{-0.048611in}{0.000000in}}%
\pgfusepath{stroke,fill}%
}%
\begin{pgfscope}%
\pgfsys@transformshift{0.687500in}{0.979595in}%
\pgfsys@useobject{currentmarker}{}%
\end{pgfscope}%
\end{pgfscope}%
\begin{pgfscope}%
\definecolor{textcolor}{rgb}{0.000000,0.000000,0.000000}%
\pgfsetstrokecolor{textcolor}%
\pgfsetfillcolor{textcolor}%
\pgftext[x=0.438194in,y=0.921557in,left,base]{\color{textcolor}\sffamily\fontsize{11.000000}{13.200000}\selectfont \(\displaystyle 20\)}%
\end{pgfscope}%
\begin{pgfscope}%
\pgfsetbuttcap%
\pgfsetroundjoin%
\definecolor{currentfill}{rgb}{0.000000,0.000000,0.000000}%
\pgfsetfillcolor{currentfill}%
\pgfsetlinewidth{0.803000pt}%
\definecolor{currentstroke}{rgb}{0.000000,0.000000,0.000000}%
\pgfsetstrokecolor{currentstroke}%
\pgfsetdash{}{0pt}%
\pgfsys@defobject{currentmarker}{\pgfqpoint{-0.048611in}{0.000000in}}{\pgfqpoint{0.000000in}{0.000000in}}{%
\pgfpathmoveto{\pgfqpoint{0.000000in}{0.000000in}}%
\pgfpathlineto{\pgfqpoint{-0.048611in}{0.000000in}}%
\pgfusepath{stroke,fill}%
}%
\begin{pgfscope}%
\pgfsys@transformshift{0.687500in}{1.499865in}%
\pgfsys@useobject{currentmarker}{}%
\end{pgfscope}%
\end{pgfscope}%
\begin{pgfscope}%
\definecolor{textcolor}{rgb}{0.000000,0.000000,0.000000}%
\pgfsetstrokecolor{textcolor}%
\pgfsetfillcolor{textcolor}%
\pgftext[x=0.438194in,y=1.441827in,left,base]{\color{textcolor}\sffamily\fontsize{11.000000}{13.200000}\selectfont \(\displaystyle 40\)}%
\end{pgfscope}%
\begin{pgfscope}%
\pgfsetbuttcap%
\pgfsetroundjoin%
\definecolor{currentfill}{rgb}{0.000000,0.000000,0.000000}%
\pgfsetfillcolor{currentfill}%
\pgfsetlinewidth{0.803000pt}%
\definecolor{currentstroke}{rgb}{0.000000,0.000000,0.000000}%
\pgfsetstrokecolor{currentstroke}%
\pgfsetdash{}{0pt}%
\pgfsys@defobject{currentmarker}{\pgfqpoint{-0.048611in}{0.000000in}}{\pgfqpoint{0.000000in}{0.000000in}}{%
\pgfpathmoveto{\pgfqpoint{0.000000in}{0.000000in}}%
\pgfpathlineto{\pgfqpoint{-0.048611in}{0.000000in}}%
\pgfusepath{stroke,fill}%
}%
\begin{pgfscope}%
\pgfsys@transformshift{0.687500in}{2.020135in}%
\pgfsys@useobject{currentmarker}{}%
\end{pgfscope}%
\end{pgfscope}%
\begin{pgfscope}%
\definecolor{textcolor}{rgb}{0.000000,0.000000,0.000000}%
\pgfsetstrokecolor{textcolor}%
\pgfsetfillcolor{textcolor}%
\pgftext[x=0.438194in,y=1.962097in,left,base]{\color{textcolor}\sffamily\fontsize{11.000000}{13.200000}\selectfont \(\displaystyle 60\)}%
\end{pgfscope}%
\begin{pgfscope}%
\pgfsetbuttcap%
\pgfsetroundjoin%
\definecolor{currentfill}{rgb}{0.000000,0.000000,0.000000}%
\pgfsetfillcolor{currentfill}%
\pgfsetlinewidth{0.803000pt}%
\definecolor{currentstroke}{rgb}{0.000000,0.000000,0.000000}%
\pgfsetstrokecolor{currentstroke}%
\pgfsetdash{}{0pt}%
\pgfsys@defobject{currentmarker}{\pgfqpoint{-0.048611in}{0.000000in}}{\pgfqpoint{0.000000in}{0.000000in}}{%
\pgfpathmoveto{\pgfqpoint{0.000000in}{0.000000in}}%
\pgfpathlineto{\pgfqpoint{-0.048611in}{0.000000in}}%
\pgfusepath{stroke,fill}%
}%
\begin{pgfscope}%
\pgfsys@transformshift{0.687500in}{2.540405in}%
\pgfsys@useobject{currentmarker}{}%
\end{pgfscope}%
\end{pgfscope}%
\begin{pgfscope}%
\definecolor{textcolor}{rgb}{0.000000,0.000000,0.000000}%
\pgfsetstrokecolor{textcolor}%
\pgfsetfillcolor{textcolor}%
\pgftext[x=0.438194in,y=2.482368in,left,base]{\color{textcolor}\sffamily\fontsize{11.000000}{13.200000}\selectfont \(\displaystyle 80\)}%
\end{pgfscope}%
\begin{pgfscope}%
\pgfsetbuttcap%
\pgfsetroundjoin%
\definecolor{currentfill}{rgb}{0.000000,0.000000,0.000000}%
\pgfsetfillcolor{currentfill}%
\pgfsetlinewidth{0.803000pt}%
\definecolor{currentstroke}{rgb}{0.000000,0.000000,0.000000}%
\pgfsetstrokecolor{currentstroke}%
\pgfsetdash{}{0pt}%
\pgfsys@defobject{currentmarker}{\pgfqpoint{-0.048611in}{0.000000in}}{\pgfqpoint{0.000000in}{0.000000in}}{%
\pgfpathmoveto{\pgfqpoint{0.000000in}{0.000000in}}%
\pgfpathlineto{\pgfqpoint{-0.048611in}{0.000000in}}%
\pgfusepath{stroke,fill}%
}%
\begin{pgfscope}%
\pgfsys@transformshift{0.687500in}{3.060676in}%
\pgfsys@useobject{currentmarker}{}%
\end{pgfscope}%
\end{pgfscope}%
\begin{pgfscope}%
\definecolor{textcolor}{rgb}{0.000000,0.000000,0.000000}%
\pgfsetstrokecolor{textcolor}%
\pgfsetfillcolor{textcolor}%
\pgftext[x=0.362152in,y=3.002638in,left,base]{\color{textcolor}\sffamily\fontsize{11.000000}{13.200000}\selectfont \(\displaystyle 100\)}%
\end{pgfscope}%
\begin{pgfscope}%
\pgfsetbuttcap%
\pgfsetroundjoin%
\definecolor{currentfill}{rgb}{0.000000,0.000000,0.000000}%
\pgfsetfillcolor{currentfill}%
\pgfsetlinewidth{0.803000pt}%
\definecolor{currentstroke}{rgb}{0.000000,0.000000,0.000000}%
\pgfsetstrokecolor{currentstroke}%
\pgfsetdash{}{0pt}%
\pgfsys@defobject{currentmarker}{\pgfqpoint{-0.048611in}{0.000000in}}{\pgfqpoint{0.000000in}{0.000000in}}{%
\pgfpathmoveto{\pgfqpoint{0.000000in}{0.000000in}}%
\pgfpathlineto{\pgfqpoint{-0.048611in}{0.000000in}}%
\pgfusepath{stroke,fill}%
}%
\begin{pgfscope}%
\pgfsys@transformshift{0.687500in}{3.580946in}%
\pgfsys@useobject{currentmarker}{}%
\end{pgfscope}%
\end{pgfscope}%
\begin{pgfscope}%
\definecolor{textcolor}{rgb}{0.000000,0.000000,0.000000}%
\pgfsetstrokecolor{textcolor}%
\pgfsetfillcolor{textcolor}%
\pgftext[x=0.362152in,y=3.522908in,left,base]{\color{textcolor}\sffamily\fontsize{11.000000}{13.200000}\selectfont \(\displaystyle 120\)}%
\end{pgfscope}%
\begin{pgfscope}%
\pgfsetbuttcap%
\pgfsetroundjoin%
\definecolor{currentfill}{rgb}{0.000000,0.000000,0.000000}%
\pgfsetfillcolor{currentfill}%
\pgfsetlinewidth{0.803000pt}%
\definecolor{currentstroke}{rgb}{0.000000,0.000000,0.000000}%
\pgfsetstrokecolor{currentstroke}%
\pgfsetdash{}{0pt}%
\pgfsys@defobject{currentmarker}{\pgfqpoint{-0.048611in}{0.000000in}}{\pgfqpoint{0.000000in}{0.000000in}}{%
\pgfpathmoveto{\pgfqpoint{0.000000in}{0.000000in}}%
\pgfpathlineto{\pgfqpoint{-0.048611in}{0.000000in}}%
\pgfusepath{stroke,fill}%
}%
\begin{pgfscope}%
\pgfsys@transformshift{0.687500in}{4.101216in}%
\pgfsys@useobject{currentmarker}{}%
\end{pgfscope}%
\end{pgfscope}%
\begin{pgfscope}%
\definecolor{textcolor}{rgb}{0.000000,0.000000,0.000000}%
\pgfsetstrokecolor{textcolor}%
\pgfsetfillcolor{textcolor}%
\pgftext[x=0.362152in,y=4.043179in,left,base]{\color{textcolor}\sffamily\fontsize{11.000000}{13.200000}\selectfont \(\displaystyle 140\)}%
\end{pgfscope}%
\begin{pgfscope}%
\pgfsetbuttcap%
\pgfsetroundjoin%
\definecolor{currentfill}{rgb}{0.000000,0.000000,0.000000}%
\pgfsetfillcolor{currentfill}%
\pgfsetlinewidth{0.803000pt}%
\definecolor{currentstroke}{rgb}{0.000000,0.000000,0.000000}%
\pgfsetstrokecolor{currentstroke}%
\pgfsetdash{}{0pt}%
\pgfsys@defobject{currentmarker}{\pgfqpoint{-0.048611in}{0.000000in}}{\pgfqpoint{0.000000in}{0.000000in}}{%
\pgfpathmoveto{\pgfqpoint{0.000000in}{0.000000in}}%
\pgfpathlineto{\pgfqpoint{-0.048611in}{0.000000in}}%
\pgfusepath{stroke,fill}%
}%
\begin{pgfscope}%
\pgfsys@transformshift{0.687500in}{4.621486in}%
\pgfsys@useobject{currentmarker}{}%
\end{pgfscope}%
\end{pgfscope}%
\begin{pgfscope}%
\definecolor{textcolor}{rgb}{0.000000,0.000000,0.000000}%
\pgfsetstrokecolor{textcolor}%
\pgfsetfillcolor{textcolor}%
\pgftext[x=0.362152in,y=4.563449in,left,base]{\color{textcolor}\sffamily\fontsize{11.000000}{13.200000}\selectfont \(\displaystyle 160\)}%
\end{pgfscope}%
\begin{pgfscope}%
\definecolor{textcolor}{rgb}{0.000000,0.000000,0.000000}%
\pgfsetstrokecolor{textcolor}%
\pgfsetfillcolor{textcolor}%
\pgftext[x=0.306597in,y=2.722500in,,bottom,rotate=90.000000]{\color{textcolor}\sffamily\fontsize{11.000000}{13.200000}\selectfont \textrm{Maximum Number of GMRES Iterations}}%
\end{pgfscope}%
\begin{pgfscope}%
\pgfpathrectangle{\pgfqpoint{0.687500in}{0.605000in}}{\pgfqpoint{4.262500in}{4.235000in}}%
\pgfusepath{clip}%
\pgfsetbuttcap%
\pgfsetroundjoin%
\pgfsetlinewidth{1.505625pt}%
\definecolor{currentstroke}{rgb}{0.843137,0.000000,0.000000}%
\pgfsetstrokecolor{currentstroke}%
\pgfsetdash{{5.550000pt}{2.400000pt}}{0.000000pt}%
\pgfpathmoveto{\pgfqpoint{0.881250in}{1.135676in}}%
\pgfpathlineto{\pgfqpoint{2.172917in}{1.786014in}}%
\pgfpathlineto{\pgfqpoint{3.464583in}{3.242770in}}%
\pgfpathlineto{\pgfqpoint{4.756250in}{4.647500in}}%
\pgfusepath{stroke}%
\end{pgfscope}%
\begin{pgfscope}%
\pgfpathrectangle{\pgfqpoint{0.687500in}{0.605000in}}{\pgfqpoint{4.262500in}{4.235000in}}%
\pgfusepath{clip}%
\pgfsetbuttcap%
\pgfsetroundjoin%
\definecolor{currentfill}{rgb}{0.843137,0.000000,0.000000}%
\pgfsetfillcolor{currentfill}%
\pgfsetlinewidth{1.003750pt}%
\definecolor{currentstroke}{rgb}{0.843137,0.000000,0.000000}%
\pgfsetstrokecolor{currentstroke}%
\pgfsetdash{}{0pt}%
\pgfsys@defobject{currentmarker}{\pgfqpoint{-0.041667in}{-0.041667in}}{\pgfqpoint{0.041667in}{0.041667in}}{%
\pgfpathmoveto{\pgfqpoint{0.000000in}{-0.041667in}}%
\pgfpathcurveto{\pgfqpoint{0.011050in}{-0.041667in}}{\pgfqpoint{0.021649in}{-0.037276in}}{\pgfqpoint{0.029463in}{-0.029463in}}%
\pgfpathcurveto{\pgfqpoint{0.037276in}{-0.021649in}}{\pgfqpoint{0.041667in}{-0.011050in}}{\pgfqpoint{0.041667in}{0.000000in}}%
\pgfpathcurveto{\pgfqpoint{0.041667in}{0.011050in}}{\pgfqpoint{0.037276in}{0.021649in}}{\pgfqpoint{0.029463in}{0.029463in}}%
\pgfpathcurveto{\pgfqpoint{0.021649in}{0.037276in}}{\pgfqpoint{0.011050in}{0.041667in}}{\pgfqpoint{0.000000in}{0.041667in}}%
\pgfpathcurveto{\pgfqpoint{-0.011050in}{0.041667in}}{\pgfqpoint{-0.021649in}{0.037276in}}{\pgfqpoint{-0.029463in}{0.029463in}}%
\pgfpathcurveto{\pgfqpoint{-0.037276in}{0.021649in}}{\pgfqpoint{-0.041667in}{0.011050in}}{\pgfqpoint{-0.041667in}{0.000000in}}%
\pgfpathcurveto{\pgfqpoint{-0.041667in}{-0.011050in}}{\pgfqpoint{-0.037276in}{-0.021649in}}{\pgfqpoint{-0.029463in}{-0.029463in}}%
\pgfpathcurveto{\pgfqpoint{-0.021649in}{-0.037276in}}{\pgfqpoint{-0.011050in}{-0.041667in}}{\pgfqpoint{0.000000in}{-0.041667in}}%
\pgfpathclose%
\pgfusepath{stroke,fill}%
}%
\begin{pgfscope}%
\pgfsys@transformshift{0.881250in}{1.135676in}%
\pgfsys@useobject{currentmarker}{}%
\end{pgfscope}%
\begin{pgfscope}%
\pgfsys@transformshift{2.172917in}{1.786014in}%
\pgfsys@useobject{currentmarker}{}%
\end{pgfscope}%
\begin{pgfscope}%
\pgfsys@transformshift{3.464583in}{3.242770in}%
\pgfsys@useobject{currentmarker}{}%
\end{pgfscope}%
\begin{pgfscope}%
\pgfsys@transformshift{4.756250in}{4.647500in}%
\pgfsys@useobject{currentmarker}{}%
\end{pgfscope}%
\end{pgfscope}%
\begin{pgfscope}%
\pgfpathrectangle{\pgfqpoint{0.687500in}{0.605000in}}{\pgfqpoint{4.262500in}{4.235000in}}%
\pgfusepath{clip}%
\pgfsetbuttcap%
\pgfsetroundjoin%
\pgfsetlinewidth{1.505625pt}%
\definecolor{currentstroke}{rgb}{0.549020,0.235294,1.000000}%
\pgfsetstrokecolor{currentstroke}%
\pgfsetdash{{5.550000pt}{2.400000pt}}{0.000000pt}%
\pgfpathmoveto{\pgfqpoint{0.881250in}{0.979595in}}%
\pgfpathlineto{\pgfqpoint{2.172917in}{1.265743in}}%
\pgfpathlineto{\pgfqpoint{3.464583in}{1.786014in}}%
\pgfpathlineto{\pgfqpoint{4.756250in}{2.436351in}}%
\pgfusepath{stroke}%
\end{pgfscope}%
\begin{pgfscope}%
\pgfpathrectangle{\pgfqpoint{0.687500in}{0.605000in}}{\pgfqpoint{4.262500in}{4.235000in}}%
\pgfusepath{clip}%
\pgfsetbuttcap%
\pgfsetmiterjoin%
\definecolor{currentfill}{rgb}{0.549020,0.235294,1.000000}%
\pgfsetfillcolor{currentfill}%
\pgfsetlinewidth{1.003750pt}%
\definecolor{currentstroke}{rgb}{0.549020,0.235294,1.000000}%
\pgfsetstrokecolor{currentstroke}%
\pgfsetdash{}{0pt}%
\pgfsys@defobject{currentmarker}{\pgfqpoint{-0.041667in}{-0.041667in}}{\pgfqpoint{0.041667in}{0.041667in}}{%
\pgfpathmoveto{\pgfqpoint{0.000000in}{0.041667in}}%
\pgfpathlineto{\pgfqpoint{-0.041667in}{-0.041667in}}%
\pgfpathlineto{\pgfqpoint{0.041667in}{-0.041667in}}%
\pgfpathclose%
\pgfusepath{stroke,fill}%
}%
\begin{pgfscope}%
\pgfsys@transformshift{0.881250in}{0.979595in}%
\pgfsys@useobject{currentmarker}{}%
\end{pgfscope}%
\begin{pgfscope}%
\pgfsys@transformshift{2.172917in}{1.265743in}%
\pgfsys@useobject{currentmarker}{}%
\end{pgfscope}%
\begin{pgfscope}%
\pgfsys@transformshift{3.464583in}{1.786014in}%
\pgfsys@useobject{currentmarker}{}%
\end{pgfscope}%
\begin{pgfscope}%
\pgfsys@transformshift{4.756250in}{2.436351in}%
\pgfsys@useobject{currentmarker}{}%
\end{pgfscope}%
\end{pgfscope}%
\begin{pgfscope}%
\pgfpathrectangle{\pgfqpoint{0.687500in}{0.605000in}}{\pgfqpoint{4.262500in}{4.235000in}}%
\pgfusepath{clip}%
\pgfsetbuttcap%
\pgfsetroundjoin%
\pgfsetlinewidth{1.505625pt}%
\definecolor{currentstroke}{rgb}{0.007843,0.533333,0.000000}%
\pgfsetstrokecolor{currentstroke}%
\pgfsetdash{{5.550000pt}{2.400000pt}}{0.000000pt}%
\pgfpathmoveto{\pgfqpoint{0.881250in}{0.875541in}}%
\pgfpathlineto{\pgfqpoint{2.172917in}{1.005608in}}%
\pgfpathlineto{\pgfqpoint{3.464583in}{1.213716in}}%
\pgfpathlineto{\pgfqpoint{4.756250in}{1.447838in}}%
\pgfusepath{stroke}%
\end{pgfscope}%
\begin{pgfscope}%
\pgfpathrectangle{\pgfqpoint{0.687500in}{0.605000in}}{\pgfqpoint{4.262500in}{4.235000in}}%
\pgfusepath{clip}%
\pgfsetbuttcap%
\pgfsetmiterjoin%
\definecolor{currentfill}{rgb}{0.007843,0.533333,0.000000}%
\pgfsetfillcolor{currentfill}%
\pgfsetlinewidth{1.003750pt}%
\definecolor{currentstroke}{rgb}{0.007843,0.533333,0.000000}%
\pgfsetstrokecolor{currentstroke}%
\pgfsetdash{}{0pt}%
\pgfsys@defobject{currentmarker}{\pgfqpoint{-0.041667in}{-0.041667in}}{\pgfqpoint{0.041667in}{0.041667in}}{%
\pgfpathmoveto{\pgfqpoint{-0.000000in}{-0.041667in}}%
\pgfpathlineto{\pgfqpoint{0.041667in}{0.041667in}}%
\pgfpathlineto{\pgfqpoint{-0.041667in}{0.041667in}}%
\pgfpathclose%
\pgfusepath{stroke,fill}%
}%
\begin{pgfscope}%
\pgfsys@transformshift{0.881250in}{0.875541in}%
\pgfsys@useobject{currentmarker}{}%
\end{pgfscope}%
\begin{pgfscope}%
\pgfsys@transformshift{2.172917in}{1.005608in}%
\pgfsys@useobject{currentmarker}{}%
\end{pgfscope}%
\begin{pgfscope}%
\pgfsys@transformshift{3.464583in}{1.213716in}%
\pgfsys@useobject{currentmarker}{}%
\end{pgfscope}%
\begin{pgfscope}%
\pgfsys@transformshift{4.756250in}{1.447838in}%
\pgfsys@useobject{currentmarker}{}%
\end{pgfscope}%
\end{pgfscope}%
\begin{pgfscope}%
\pgfpathrectangle{\pgfqpoint{0.687500in}{0.605000in}}{\pgfqpoint{4.262500in}{4.235000in}}%
\pgfusepath{clip}%
\pgfsetbuttcap%
\pgfsetroundjoin%
\pgfsetlinewidth{1.505625pt}%
\definecolor{currentstroke}{rgb}{0.000000,0.674510,0.780392}%
\pgfsetstrokecolor{currentstroke}%
\pgfsetdash{{5.550000pt}{2.400000pt}}{0.000000pt}%
\pgfpathmoveto{\pgfqpoint{0.881250in}{0.797500in}}%
\pgfpathlineto{\pgfqpoint{2.172917in}{0.875541in}}%
\pgfpathlineto{\pgfqpoint{3.464583in}{0.979595in}}%
\pgfpathlineto{\pgfqpoint{4.756250in}{1.057635in}}%
\pgfusepath{stroke}%
\end{pgfscope}%
\begin{pgfscope}%
\pgfpathrectangle{\pgfqpoint{0.687500in}{0.605000in}}{\pgfqpoint{4.262500in}{4.235000in}}%
\pgfusepath{clip}%
\pgfsetbuttcap%
\pgfsetmiterjoin%
\definecolor{currentfill}{rgb}{0.000000,0.674510,0.780392}%
\pgfsetfillcolor{currentfill}%
\pgfsetlinewidth{1.003750pt}%
\definecolor{currentstroke}{rgb}{0.000000,0.674510,0.780392}%
\pgfsetstrokecolor{currentstroke}%
\pgfsetdash{}{0pt}%
\pgfsys@defobject{currentmarker}{\pgfqpoint{-0.041667in}{-0.041667in}}{\pgfqpoint{0.041667in}{0.041667in}}{%
\pgfpathmoveto{\pgfqpoint{0.041667in}{-0.000000in}}%
\pgfpathlineto{\pgfqpoint{-0.041667in}{0.041667in}}%
\pgfpathlineto{\pgfqpoint{-0.041667in}{-0.041667in}}%
\pgfpathclose%
\pgfusepath{stroke,fill}%
}%
\begin{pgfscope}%
\pgfsys@transformshift{0.881250in}{0.797500in}%
\pgfsys@useobject{currentmarker}{}%
\end{pgfscope}%
\begin{pgfscope}%
\pgfsys@transformshift{2.172917in}{0.875541in}%
\pgfsys@useobject{currentmarker}{}%
\end{pgfscope}%
\begin{pgfscope}%
\pgfsys@transformshift{3.464583in}{0.979595in}%
\pgfsys@useobject{currentmarker}{}%
\end{pgfscope}%
\begin{pgfscope}%
\pgfsys@transformshift{4.756250in}{1.057635in}%
\pgfsys@useobject{currentmarker}{}%
\end{pgfscope}%
\end{pgfscope}%
\begin{pgfscope}%
\pgfsetrectcap%
\pgfsetmiterjoin%
\pgfsetlinewidth{0.803000pt}%
\definecolor{currentstroke}{rgb}{0.000000,0.000000,0.000000}%
\pgfsetstrokecolor{currentstroke}%
\pgfsetdash{}{0pt}%
\pgfpathmoveto{\pgfqpoint{0.687500in}{0.605000in}}%
\pgfpathlineto{\pgfqpoint{0.687500in}{4.840000in}}%
\pgfusepath{stroke}%
\end{pgfscope}%
\begin{pgfscope}%
\pgfsetrectcap%
\pgfsetmiterjoin%
\pgfsetlinewidth{0.000000pt}%
\definecolor{currentstroke}{rgb}{0.000000,0.000000,0.000000}%
\pgfsetstrokecolor{currentstroke}%
\pgfsetstrokeopacity{0.000000}%
\pgfsetdash{}{0pt}%
\pgfpathmoveto{\pgfqpoint{4.950000in}{0.605000in}}%
\pgfpathlineto{\pgfqpoint{4.950000in}{4.840000in}}%
\pgfusepath{}%
\end{pgfscope}%
\begin{pgfscope}%
\pgfsetrectcap%
\pgfsetmiterjoin%
\pgfsetlinewidth{0.803000pt}%
\definecolor{currentstroke}{rgb}{0.000000,0.000000,0.000000}%
\pgfsetstrokecolor{currentstroke}%
\pgfsetdash{}{0pt}%
\pgfpathmoveto{\pgfqpoint{0.687500in}{0.605000in}}%
\pgfpathlineto{\pgfqpoint{4.950000in}{0.605000in}}%
\pgfusepath{stroke}%
\end{pgfscope}%
\begin{pgfscope}%
\pgfsetrectcap%
\pgfsetmiterjoin%
\pgfsetlinewidth{0.000000pt}%
\definecolor{currentstroke}{rgb}{0.000000,0.000000,0.000000}%
\pgfsetstrokecolor{currentstroke}%
\pgfsetstrokeopacity{0.000000}%
\pgfsetdash{}{0pt}%
\pgfpathmoveto{\pgfqpoint{0.687500in}{4.840000in}}%
\pgfpathlineto{\pgfqpoint{4.950000in}{4.840000in}}%
\pgfusepath{}%
\end{pgfscope}%
\begin{pgfscope}%
\pgfsetbuttcap%
\pgfsetmiterjoin%
\definecolor{currentfill}{rgb}{1.000000,1.000000,1.000000}%
\pgfsetfillcolor{currentfill}%
\pgfsetfillopacity{0.800000}%
\pgfsetlinewidth{1.003750pt}%
\definecolor{currentstroke}{rgb}{0.800000,0.800000,0.800000}%
\pgfsetstrokecolor{currentstroke}%
\pgfsetstrokeopacity{0.800000}%
\pgfsetdash{}{0pt}%
\pgfpathmoveto{\pgfqpoint{0.794444in}{3.820806in}}%
\pgfpathlineto{\pgfqpoint{1.780882in}{3.820806in}}%
\pgfpathquadraticcurveto{\pgfqpoint{1.811437in}{3.820806in}}{\pgfqpoint{1.811437in}{3.851362in}}%
\pgfpathlineto{\pgfqpoint{1.811437in}{4.733056in}}%
\pgfpathquadraticcurveto{\pgfqpoint{1.811437in}{4.763611in}}{\pgfqpoint{1.780882in}{4.763611in}}%
\pgfpathlineto{\pgfqpoint{0.794444in}{4.763611in}}%
\pgfpathquadraticcurveto{\pgfqpoint{0.763889in}{4.763611in}}{\pgfqpoint{0.763889in}{4.733056in}}%
\pgfpathlineto{\pgfqpoint{0.763889in}{3.851362in}}%
\pgfpathquadraticcurveto{\pgfqpoint{0.763889in}{3.820806in}}{\pgfqpoint{0.794444in}{3.820806in}}%
\pgfpathclose%
\pgfusepath{stroke,fill}%
\end{pgfscope}%
\begin{pgfscope}%
\pgfsetbuttcap%
\pgfsetroundjoin%
\pgfsetlinewidth{1.505625pt}%
\definecolor{currentstroke}{rgb}{0.843137,0.000000,0.000000}%
\pgfsetstrokecolor{currentstroke}%
\pgfsetdash{{5.550000pt}{2.400000pt}}{0.000000pt}%
\pgfpathmoveto{\pgfqpoint{0.825000in}{4.639897in}}%
\pgfpathlineto{\pgfqpoint{1.130556in}{4.639897in}}%
\pgfusepath{stroke}%
\end{pgfscope}%
\begin{pgfscope}%
\pgfsetbuttcap%
\pgfsetroundjoin%
\definecolor{currentfill}{rgb}{0.843137,0.000000,0.000000}%
\pgfsetfillcolor{currentfill}%
\pgfsetlinewidth{1.003750pt}%
\definecolor{currentstroke}{rgb}{0.843137,0.000000,0.000000}%
\pgfsetstrokecolor{currentstroke}%
\pgfsetdash{}{0pt}%
\pgfsys@defobject{currentmarker}{\pgfqpoint{-0.041667in}{-0.041667in}}{\pgfqpoint{0.041667in}{0.041667in}}{%
\pgfpathmoveto{\pgfqpoint{0.000000in}{-0.041667in}}%
\pgfpathcurveto{\pgfqpoint{0.011050in}{-0.041667in}}{\pgfqpoint{0.021649in}{-0.037276in}}{\pgfqpoint{0.029463in}{-0.029463in}}%
\pgfpathcurveto{\pgfqpoint{0.037276in}{-0.021649in}}{\pgfqpoint{0.041667in}{-0.011050in}}{\pgfqpoint{0.041667in}{0.000000in}}%
\pgfpathcurveto{\pgfqpoint{0.041667in}{0.011050in}}{\pgfqpoint{0.037276in}{0.021649in}}{\pgfqpoint{0.029463in}{0.029463in}}%
\pgfpathcurveto{\pgfqpoint{0.021649in}{0.037276in}}{\pgfqpoint{0.011050in}{0.041667in}}{\pgfqpoint{0.000000in}{0.041667in}}%
\pgfpathcurveto{\pgfqpoint{-0.011050in}{0.041667in}}{\pgfqpoint{-0.021649in}{0.037276in}}{\pgfqpoint{-0.029463in}{0.029463in}}%
\pgfpathcurveto{\pgfqpoint{-0.037276in}{0.021649in}}{\pgfqpoint{-0.041667in}{0.011050in}}{\pgfqpoint{-0.041667in}{0.000000in}}%
\pgfpathcurveto{\pgfqpoint{-0.041667in}{-0.011050in}}{\pgfqpoint{-0.037276in}{-0.021649in}}{\pgfqpoint{-0.029463in}{-0.029463in}}%
\pgfpathcurveto{\pgfqpoint{-0.021649in}{-0.037276in}}{\pgfqpoint{-0.011050in}{-0.041667in}}{\pgfqpoint{0.000000in}{-0.041667in}}%
\pgfpathclose%
\pgfusepath{stroke,fill}%
}%
\begin{pgfscope}%
\pgfsys@transformshift{0.977778in}{4.639897in}%
\pgfsys@useobject{currentmarker}{}%
\end{pgfscope}%
\end{pgfscope}%
\begin{pgfscope}%
\definecolor{textcolor}{rgb}{0.000000,0.000000,0.000000}%
\pgfsetstrokecolor{textcolor}%
\pgfsetfillcolor{textcolor}%
\pgftext[x=1.252778in,y=4.586425in,left,base]{\color{textcolor}\sffamily\fontsize{11.000000}{13.200000}\selectfont \(\displaystyle \beta = \)0}%
\end{pgfscope}%
\begin{pgfscope}%
\pgfsetbuttcap%
\pgfsetroundjoin%
\pgfsetlinewidth{1.505625pt}%
\definecolor{currentstroke}{rgb}{0.549020,0.235294,1.000000}%
\pgfsetstrokecolor{currentstroke}%
\pgfsetdash{{5.550000pt}{2.400000pt}}{0.000000pt}%
\pgfpathmoveto{\pgfqpoint{0.825000in}{4.415654in}}%
\pgfpathlineto{\pgfqpoint{1.130556in}{4.415654in}}%
\pgfusepath{stroke}%
\end{pgfscope}%
\begin{pgfscope}%
\pgfsetbuttcap%
\pgfsetmiterjoin%
\definecolor{currentfill}{rgb}{0.549020,0.235294,1.000000}%
\pgfsetfillcolor{currentfill}%
\pgfsetlinewidth{1.003750pt}%
\definecolor{currentstroke}{rgb}{0.549020,0.235294,1.000000}%
\pgfsetstrokecolor{currentstroke}%
\pgfsetdash{}{0pt}%
\pgfsys@defobject{currentmarker}{\pgfqpoint{-0.041667in}{-0.041667in}}{\pgfqpoint{0.041667in}{0.041667in}}{%
\pgfpathmoveto{\pgfqpoint{0.000000in}{0.041667in}}%
\pgfpathlineto{\pgfqpoint{-0.041667in}{-0.041667in}}%
\pgfpathlineto{\pgfqpoint{0.041667in}{-0.041667in}}%
\pgfpathclose%
\pgfusepath{stroke,fill}%
}%
\begin{pgfscope}%
\pgfsys@transformshift{0.977778in}{4.415654in}%
\pgfsys@useobject{currentmarker}{}%
\end{pgfscope}%
\end{pgfscope}%
\begin{pgfscope}%
\definecolor{textcolor}{rgb}{0.000000,0.000000,0.000000}%
\pgfsetstrokecolor{textcolor}%
\pgfsetfillcolor{textcolor}%
\pgftext[x=1.252778in,y=4.362182in,left,base]{\color{textcolor}\sffamily\fontsize{11.000000}{13.200000}\selectfont \(\displaystyle \beta = \)0.1}%
\end{pgfscope}%
\begin{pgfscope}%
\pgfsetbuttcap%
\pgfsetroundjoin%
\pgfsetlinewidth{1.505625pt}%
\definecolor{currentstroke}{rgb}{0.007843,0.533333,0.000000}%
\pgfsetstrokecolor{currentstroke}%
\pgfsetdash{{5.550000pt}{2.400000pt}}{0.000000pt}%
\pgfpathmoveto{\pgfqpoint{0.825000in}{4.191411in}}%
\pgfpathlineto{\pgfqpoint{1.130556in}{4.191411in}}%
\pgfusepath{stroke}%
\end{pgfscope}%
\begin{pgfscope}%
\pgfsetbuttcap%
\pgfsetmiterjoin%
\definecolor{currentfill}{rgb}{0.007843,0.533333,0.000000}%
\pgfsetfillcolor{currentfill}%
\pgfsetlinewidth{1.003750pt}%
\definecolor{currentstroke}{rgb}{0.007843,0.533333,0.000000}%
\pgfsetstrokecolor{currentstroke}%
\pgfsetdash{}{0pt}%
\pgfsys@defobject{currentmarker}{\pgfqpoint{-0.041667in}{-0.041667in}}{\pgfqpoint{0.041667in}{0.041667in}}{%
\pgfpathmoveto{\pgfqpoint{-0.000000in}{-0.041667in}}%
\pgfpathlineto{\pgfqpoint{0.041667in}{0.041667in}}%
\pgfpathlineto{\pgfqpoint{-0.041667in}{0.041667in}}%
\pgfpathclose%
\pgfusepath{stroke,fill}%
}%
\begin{pgfscope}%
\pgfsys@transformshift{0.977778in}{4.191411in}%
\pgfsys@useobject{currentmarker}{}%
\end{pgfscope}%
\end{pgfscope}%
\begin{pgfscope}%
\definecolor{textcolor}{rgb}{0.000000,0.000000,0.000000}%
\pgfsetstrokecolor{textcolor}%
\pgfsetfillcolor{textcolor}%
\pgftext[x=1.252778in,y=4.137939in,left,base]{\color{textcolor}\sffamily\fontsize{11.000000}{13.200000}\selectfont \(\displaystyle \beta = \)0.2}%
\end{pgfscope}%
\begin{pgfscope}%
\pgfsetbuttcap%
\pgfsetroundjoin%
\pgfsetlinewidth{1.505625pt}%
\definecolor{currentstroke}{rgb}{0.000000,0.674510,0.780392}%
\pgfsetstrokecolor{currentstroke}%
\pgfsetdash{{5.550000pt}{2.400000pt}}{0.000000pt}%
\pgfpathmoveto{\pgfqpoint{0.825000in}{3.967168in}}%
\pgfpathlineto{\pgfqpoint{1.130556in}{3.967168in}}%
\pgfusepath{stroke}%
\end{pgfscope}%
\begin{pgfscope}%
\pgfsetbuttcap%
\pgfsetmiterjoin%
\definecolor{currentfill}{rgb}{0.000000,0.674510,0.780392}%
\pgfsetfillcolor{currentfill}%
\pgfsetlinewidth{1.003750pt}%
\definecolor{currentstroke}{rgb}{0.000000,0.674510,0.780392}%
\pgfsetstrokecolor{currentstroke}%
\pgfsetdash{}{0pt}%
\pgfsys@defobject{currentmarker}{\pgfqpoint{-0.041667in}{-0.041667in}}{\pgfqpoint{0.041667in}{0.041667in}}{%
\pgfpathmoveto{\pgfqpoint{0.041667in}{-0.000000in}}%
\pgfpathlineto{\pgfqpoint{-0.041667in}{0.041667in}}%
\pgfpathlineto{\pgfqpoint{-0.041667in}{-0.041667in}}%
\pgfpathclose%
\pgfusepath{stroke,fill}%
}%
\begin{pgfscope}%
\pgfsys@transformshift{0.977778in}{3.967168in}%
\pgfsys@useobject{currentmarker}{}%
\end{pgfscope}%
\end{pgfscope}%
\begin{pgfscope}%
\definecolor{textcolor}{rgb}{0.000000,0.000000,0.000000}%
\pgfsetstrokecolor{textcolor}%
\pgfsetfillcolor{textcolor}%
\pgftext[x=1.252778in,y=3.913696in,left,base]{\color{textcolor}\sffamily\fontsize{11.000000}{13.200000}\selectfont \(\displaystyle \beta = \)0.3}%
\end{pgfscope}%
\end{pgfpicture}%
\makeatother%
\endgroup%

  \caption[Maximum GMRES iteration counts when $\NLiDRR{\nso-\nst} = 0.5\times  k^{\beta}$ for $\beta = 0,0.1,0.2,0.3.$]{Maximum GMRES iteration counts for solving systems with matrix $\AmatoI\Amatt$, where $\Aso=\Ast=1$ and $\NLiDRR{\nso-\nst} = 0.5\times  k^{\beta}$ for $\beta = 0,0.1,0.2,0.3.$}\label{fig:linfinityn0}
    \end{figure}
    
    \begin{figure}
      \centering
%% Creator: Matplotlib, PGF backend
%%
%% To include the figure in your LaTeX document, write
%%   \input{<filename>.pgf}
%%
%% Make sure the required packages are loaded in your preamble
%%   \usepackage{pgf}
%%
%% Figures using additional raster images can only be included by \input if
%% they are in the same directory as the main LaTeX file. For loading figures
%% from other directories you can use the `import` package
%%   \usepackage{import}
%% and then include the figures with
%%   \import{<path to file>}{<filename>.pgf}
%%
%% Matplotlib used the following preamble
%%   \usepackage{fontspec}
%%   \setmainfont{DejaVuSerif.ttf}[Path=/home/owen/progs/firedrake-complex/firedrake/lib/python3.5/site-packages/matplotlib/mpl-data/fonts/ttf/]
%%   \setsansfont{DejaVuSans.ttf}[Path=/home/owen/progs/firedrake-complex/firedrake/lib/python3.5/site-packages/matplotlib/mpl-data/fonts/ttf/]
%%   \setmonofont{DejaVuSansMono.ttf}[Path=/home/owen/progs/firedrake-complex/firedrake/lib/python3.5/site-packages/matplotlib/mpl-data/fonts/ttf/]
%%
\begingroup%
\makeatletter%
\begin{pgfpicture}%
\pgfpathrectangle{\pgfpointorigin}{\pgfqpoint{3.000000in}{3.000000in}}%
\pgfusepath{use as bounding box, clip}%
\begin{pgfscope}%
\pgfsetbuttcap%
\pgfsetmiterjoin%
\definecolor{currentfill}{rgb}{1.000000,1.000000,1.000000}%
\pgfsetfillcolor{currentfill}%
\pgfsetlinewidth{0.000000pt}%
\definecolor{currentstroke}{rgb}{1.000000,1.000000,1.000000}%
\pgfsetstrokecolor{currentstroke}%
\pgfsetdash{}{0pt}%
\pgfpathmoveto{\pgfqpoint{0.000000in}{0.000000in}}%
\pgfpathlineto{\pgfqpoint{3.000000in}{0.000000in}}%
\pgfpathlineto{\pgfqpoint{3.000000in}{3.000000in}}%
\pgfpathlineto{\pgfqpoint{0.000000in}{3.000000in}}%
\pgfpathclose%
\pgfusepath{fill}%
\end{pgfscope}%
\begin{pgfscope}%
\pgfsetbuttcap%
\pgfsetmiterjoin%
\definecolor{currentfill}{rgb}{1.000000,1.000000,1.000000}%
\pgfsetfillcolor{currentfill}%
\pgfsetlinewidth{0.000000pt}%
\definecolor{currentstroke}{rgb}{0.000000,0.000000,0.000000}%
\pgfsetstrokecolor{currentstroke}%
\pgfsetstrokeopacity{0.000000}%
\pgfsetdash{}{0pt}%
\pgfpathmoveto{\pgfqpoint{0.375000in}{0.330000in}}%
\pgfpathlineto{\pgfqpoint{2.700000in}{0.330000in}}%
\pgfpathlineto{\pgfqpoint{2.700000in}{2.640000in}}%
\pgfpathlineto{\pgfqpoint{0.375000in}{2.640000in}}%
\pgfpathclose%
\pgfusepath{fill}%
\end{pgfscope}%
\begin{pgfscope}%
\pgfpathrectangle{\pgfqpoint{0.375000in}{0.330000in}}{\pgfqpoint{2.325000in}{2.310000in}}%
\pgfusepath{clip}%
\pgfsetbuttcap%
\pgfsetroundjoin%
\definecolor{currentfill}{rgb}{0.000000,0.000000,0.000000}%
\pgfsetfillcolor{currentfill}%
\pgfsetlinewidth{1.003750pt}%
\definecolor{currentstroke}{rgb}{0.000000,0.000000,0.000000}%
\pgfsetstrokecolor{currentstroke}%
\pgfsetdash{}{0pt}%
\pgfpathmoveto{\pgfqpoint{0.480841in}{0.402751in}}%
\pgfpathcurveto{\pgfqpoint{0.491891in}{0.402751in}}{\pgfqpoint{0.502490in}{0.407141in}}{\pgfqpoint{0.510303in}{0.414955in}}%
\pgfpathcurveto{\pgfqpoint{0.518117in}{0.422768in}}{\pgfqpoint{0.522507in}{0.433367in}}{\pgfqpoint{0.522507in}{0.444417in}}%
\pgfpathcurveto{\pgfqpoint{0.522507in}{0.455468in}}{\pgfqpoint{0.518117in}{0.466067in}}{\pgfqpoint{0.510303in}{0.473880in}}%
\pgfpathcurveto{\pgfqpoint{0.502490in}{0.481694in}}{\pgfqpoint{0.491891in}{0.486084in}}{\pgfqpoint{0.480841in}{0.486084in}}%
\pgfpathcurveto{\pgfqpoint{0.469790in}{0.486084in}}{\pgfqpoint{0.459191in}{0.481694in}}{\pgfqpoint{0.451378in}{0.473880in}}%
\pgfpathcurveto{\pgfqpoint{0.443564in}{0.466067in}}{\pgfqpoint{0.439174in}{0.455468in}}{\pgfqpoint{0.439174in}{0.444417in}}%
\pgfpathcurveto{\pgfqpoint{0.439174in}{0.433367in}}{\pgfqpoint{0.443564in}{0.422768in}}{\pgfqpoint{0.451378in}{0.414955in}}%
\pgfpathcurveto{\pgfqpoint{0.459191in}{0.407141in}}{\pgfqpoint{0.469790in}{0.402751in}}{\pgfqpoint{0.480841in}{0.402751in}}%
\pgfpathclose%
\pgfusepath{stroke,fill}%
\end{pgfscope}%
\begin{pgfscope}%
\pgfpathrectangle{\pgfqpoint{0.375000in}{0.330000in}}{\pgfqpoint{2.325000in}{2.310000in}}%
\pgfusepath{clip}%
\pgfsetbuttcap%
\pgfsetroundjoin%
\definecolor{currentfill}{rgb}{0.000000,0.000000,0.000000}%
\pgfsetfillcolor{currentfill}%
\pgfsetlinewidth{1.003750pt}%
\definecolor{currentstroke}{rgb}{0.000000,0.000000,0.000000}%
\pgfsetstrokecolor{currentstroke}%
\pgfsetdash{}{0pt}%
\pgfpathmoveto{\pgfqpoint{0.480841in}{0.402751in}}%
\pgfpathcurveto{\pgfqpoint{0.491891in}{0.402751in}}{\pgfqpoint{0.502490in}{0.407141in}}{\pgfqpoint{0.510303in}{0.414955in}}%
\pgfpathcurveto{\pgfqpoint{0.518117in}{0.422768in}}{\pgfqpoint{0.522507in}{0.433367in}}{\pgfqpoint{0.522507in}{0.444417in}}%
\pgfpathcurveto{\pgfqpoint{0.522507in}{0.455468in}}{\pgfqpoint{0.518117in}{0.466067in}}{\pgfqpoint{0.510303in}{0.473880in}}%
\pgfpathcurveto{\pgfqpoint{0.502490in}{0.481694in}}{\pgfqpoint{0.491891in}{0.486084in}}{\pgfqpoint{0.480841in}{0.486084in}}%
\pgfpathcurveto{\pgfqpoint{0.469790in}{0.486084in}}{\pgfqpoint{0.459191in}{0.481694in}}{\pgfqpoint{0.451378in}{0.473880in}}%
\pgfpathcurveto{\pgfqpoint{0.443564in}{0.466067in}}{\pgfqpoint{0.439174in}{0.455468in}}{\pgfqpoint{0.439174in}{0.444417in}}%
\pgfpathcurveto{\pgfqpoint{0.439174in}{0.433367in}}{\pgfqpoint{0.443564in}{0.422768in}}{\pgfqpoint{0.451378in}{0.414955in}}%
\pgfpathcurveto{\pgfqpoint{0.459191in}{0.407141in}}{\pgfqpoint{0.469790in}{0.402751in}}{\pgfqpoint{0.480841in}{0.402751in}}%
\pgfpathclose%
\pgfusepath{stroke,fill}%
\end{pgfscope}%
\begin{pgfscope}%
\pgfpathrectangle{\pgfqpoint{0.375000in}{0.330000in}}{\pgfqpoint{2.325000in}{2.310000in}}%
\pgfusepath{clip}%
\pgfsetbuttcap%
\pgfsetroundjoin%
\definecolor{currentfill}{rgb}{0.000000,0.000000,0.000000}%
\pgfsetfillcolor{currentfill}%
\pgfsetlinewidth{1.003750pt}%
\definecolor{currentstroke}{rgb}{0.000000,0.000000,0.000000}%
\pgfsetstrokecolor{currentstroke}%
\pgfsetdash{}{0pt}%
\pgfpathmoveto{\pgfqpoint{0.480841in}{0.402751in}}%
\pgfpathcurveto{\pgfqpoint{0.491891in}{0.402751in}}{\pgfqpoint{0.502490in}{0.407141in}}{\pgfqpoint{0.510303in}{0.414955in}}%
\pgfpathcurveto{\pgfqpoint{0.518117in}{0.422768in}}{\pgfqpoint{0.522507in}{0.433367in}}{\pgfqpoint{0.522507in}{0.444417in}}%
\pgfpathcurveto{\pgfqpoint{0.522507in}{0.455468in}}{\pgfqpoint{0.518117in}{0.466067in}}{\pgfqpoint{0.510303in}{0.473880in}}%
\pgfpathcurveto{\pgfqpoint{0.502490in}{0.481694in}}{\pgfqpoint{0.491891in}{0.486084in}}{\pgfqpoint{0.480841in}{0.486084in}}%
\pgfpathcurveto{\pgfqpoint{0.469790in}{0.486084in}}{\pgfqpoint{0.459191in}{0.481694in}}{\pgfqpoint{0.451378in}{0.473880in}}%
\pgfpathcurveto{\pgfqpoint{0.443564in}{0.466067in}}{\pgfqpoint{0.439174in}{0.455468in}}{\pgfqpoint{0.439174in}{0.444417in}}%
\pgfpathcurveto{\pgfqpoint{0.439174in}{0.433367in}}{\pgfqpoint{0.443564in}{0.422768in}}{\pgfqpoint{0.451378in}{0.414955in}}%
\pgfpathcurveto{\pgfqpoint{0.459191in}{0.407141in}}{\pgfqpoint{0.469790in}{0.402751in}}{\pgfqpoint{0.480841in}{0.402751in}}%
\pgfpathclose%
\pgfusepath{stroke,fill}%
\end{pgfscope}%
\begin{pgfscope}%
\pgfpathrectangle{\pgfqpoint{0.375000in}{0.330000in}}{\pgfqpoint{2.325000in}{2.310000in}}%
\pgfusepath{clip}%
\pgfsetbuttcap%
\pgfsetroundjoin%
\definecolor{currentfill}{rgb}{0.000000,0.000000,0.000000}%
\pgfsetfillcolor{currentfill}%
\pgfsetlinewidth{1.003750pt}%
\definecolor{currentstroke}{rgb}{0.000000,0.000000,0.000000}%
\pgfsetstrokecolor{currentstroke}%
\pgfsetdash{}{0pt}%
\pgfpathmoveto{\pgfqpoint{0.480841in}{0.402751in}}%
\pgfpathcurveto{\pgfqpoint{0.491891in}{0.402751in}}{\pgfqpoint{0.502490in}{0.407141in}}{\pgfqpoint{0.510303in}{0.414955in}}%
\pgfpathcurveto{\pgfqpoint{0.518117in}{0.422768in}}{\pgfqpoint{0.522507in}{0.433367in}}{\pgfqpoint{0.522507in}{0.444417in}}%
\pgfpathcurveto{\pgfqpoint{0.522507in}{0.455468in}}{\pgfqpoint{0.518117in}{0.466067in}}{\pgfqpoint{0.510303in}{0.473880in}}%
\pgfpathcurveto{\pgfqpoint{0.502490in}{0.481694in}}{\pgfqpoint{0.491891in}{0.486084in}}{\pgfqpoint{0.480841in}{0.486084in}}%
\pgfpathcurveto{\pgfqpoint{0.469790in}{0.486084in}}{\pgfqpoint{0.459191in}{0.481694in}}{\pgfqpoint{0.451378in}{0.473880in}}%
\pgfpathcurveto{\pgfqpoint{0.443564in}{0.466067in}}{\pgfqpoint{0.439174in}{0.455468in}}{\pgfqpoint{0.439174in}{0.444417in}}%
\pgfpathcurveto{\pgfqpoint{0.439174in}{0.433367in}}{\pgfqpoint{0.443564in}{0.422768in}}{\pgfqpoint{0.451378in}{0.414955in}}%
\pgfpathcurveto{\pgfqpoint{0.459191in}{0.407141in}}{\pgfqpoint{0.469790in}{0.402751in}}{\pgfqpoint{0.480841in}{0.402751in}}%
\pgfpathclose%
\pgfusepath{stroke,fill}%
\end{pgfscope}%
\begin{pgfscope}%
\pgfpathrectangle{\pgfqpoint{0.375000in}{0.330000in}}{\pgfqpoint{2.325000in}{2.310000in}}%
\pgfusepath{clip}%
\pgfsetbuttcap%
\pgfsetroundjoin%
\definecolor{currentfill}{rgb}{0.000000,0.000000,0.000000}%
\pgfsetfillcolor{currentfill}%
\pgfsetlinewidth{1.003750pt}%
\definecolor{currentstroke}{rgb}{0.000000,0.000000,0.000000}%
\pgfsetstrokecolor{currentstroke}%
\pgfsetdash{}{0pt}%
\pgfpathmoveto{\pgfqpoint{0.480841in}{0.402751in}}%
\pgfpathcurveto{\pgfqpoint{0.491891in}{0.402751in}}{\pgfqpoint{0.502490in}{0.407141in}}{\pgfqpoint{0.510303in}{0.414955in}}%
\pgfpathcurveto{\pgfqpoint{0.518117in}{0.422768in}}{\pgfqpoint{0.522507in}{0.433367in}}{\pgfqpoint{0.522507in}{0.444417in}}%
\pgfpathcurveto{\pgfqpoint{0.522507in}{0.455468in}}{\pgfqpoint{0.518117in}{0.466067in}}{\pgfqpoint{0.510303in}{0.473880in}}%
\pgfpathcurveto{\pgfqpoint{0.502490in}{0.481694in}}{\pgfqpoint{0.491891in}{0.486084in}}{\pgfqpoint{0.480841in}{0.486084in}}%
\pgfpathcurveto{\pgfqpoint{0.469790in}{0.486084in}}{\pgfqpoint{0.459191in}{0.481694in}}{\pgfqpoint{0.451378in}{0.473880in}}%
\pgfpathcurveto{\pgfqpoint{0.443564in}{0.466067in}}{\pgfqpoint{0.439174in}{0.455468in}}{\pgfqpoint{0.439174in}{0.444417in}}%
\pgfpathcurveto{\pgfqpoint{0.439174in}{0.433367in}}{\pgfqpoint{0.443564in}{0.422768in}}{\pgfqpoint{0.451378in}{0.414955in}}%
\pgfpathcurveto{\pgfqpoint{0.459191in}{0.407141in}}{\pgfqpoint{0.469790in}{0.402751in}}{\pgfqpoint{0.480841in}{0.402751in}}%
\pgfpathclose%
\pgfusepath{stroke,fill}%
\end{pgfscope}%
\begin{pgfscope}%
\pgfpathrectangle{\pgfqpoint{0.375000in}{0.330000in}}{\pgfqpoint{2.325000in}{2.310000in}}%
\pgfusepath{clip}%
\pgfsetbuttcap%
\pgfsetroundjoin%
\definecolor{currentfill}{rgb}{0.000000,0.000000,0.000000}%
\pgfsetfillcolor{currentfill}%
\pgfsetlinewidth{1.003750pt}%
\definecolor{currentstroke}{rgb}{0.000000,0.000000,0.000000}%
\pgfsetstrokecolor{currentstroke}%
\pgfsetdash{}{0pt}%
\pgfpathmoveto{\pgfqpoint{0.480841in}{0.402751in}}%
\pgfpathcurveto{\pgfqpoint{0.491891in}{0.402751in}}{\pgfqpoint{0.502490in}{0.407141in}}{\pgfqpoint{0.510303in}{0.414955in}}%
\pgfpathcurveto{\pgfqpoint{0.518117in}{0.422768in}}{\pgfqpoint{0.522507in}{0.433367in}}{\pgfqpoint{0.522507in}{0.444417in}}%
\pgfpathcurveto{\pgfqpoint{0.522507in}{0.455468in}}{\pgfqpoint{0.518117in}{0.466067in}}{\pgfqpoint{0.510303in}{0.473880in}}%
\pgfpathcurveto{\pgfqpoint{0.502490in}{0.481694in}}{\pgfqpoint{0.491891in}{0.486084in}}{\pgfqpoint{0.480841in}{0.486084in}}%
\pgfpathcurveto{\pgfqpoint{0.469790in}{0.486084in}}{\pgfqpoint{0.459191in}{0.481694in}}{\pgfqpoint{0.451378in}{0.473880in}}%
\pgfpathcurveto{\pgfqpoint{0.443564in}{0.466067in}}{\pgfqpoint{0.439174in}{0.455468in}}{\pgfqpoint{0.439174in}{0.444417in}}%
\pgfpathcurveto{\pgfqpoint{0.439174in}{0.433367in}}{\pgfqpoint{0.443564in}{0.422768in}}{\pgfqpoint{0.451378in}{0.414955in}}%
\pgfpathcurveto{\pgfqpoint{0.459191in}{0.407141in}}{\pgfqpoint{0.469790in}{0.402751in}}{\pgfqpoint{0.480841in}{0.402751in}}%
\pgfpathclose%
\pgfusepath{stroke,fill}%
\end{pgfscope}%
\begin{pgfscope}%
\pgfpathrectangle{\pgfqpoint{0.375000in}{0.330000in}}{\pgfqpoint{2.325000in}{2.310000in}}%
\pgfusepath{clip}%
\pgfsetbuttcap%
\pgfsetroundjoin%
\definecolor{currentfill}{rgb}{0.000000,0.000000,0.000000}%
\pgfsetfillcolor{currentfill}%
\pgfsetlinewidth{1.003750pt}%
\definecolor{currentstroke}{rgb}{0.000000,0.000000,0.000000}%
\pgfsetstrokecolor{currentstroke}%
\pgfsetdash{}{0pt}%
\pgfpathmoveto{\pgfqpoint{0.480841in}{0.402751in}}%
\pgfpathcurveto{\pgfqpoint{0.491891in}{0.402751in}}{\pgfqpoint{0.502490in}{0.407141in}}{\pgfqpoint{0.510303in}{0.414955in}}%
\pgfpathcurveto{\pgfqpoint{0.518117in}{0.422768in}}{\pgfqpoint{0.522507in}{0.433367in}}{\pgfqpoint{0.522507in}{0.444417in}}%
\pgfpathcurveto{\pgfqpoint{0.522507in}{0.455468in}}{\pgfqpoint{0.518117in}{0.466067in}}{\pgfqpoint{0.510303in}{0.473880in}}%
\pgfpathcurveto{\pgfqpoint{0.502490in}{0.481694in}}{\pgfqpoint{0.491891in}{0.486084in}}{\pgfqpoint{0.480841in}{0.486084in}}%
\pgfpathcurveto{\pgfqpoint{0.469790in}{0.486084in}}{\pgfqpoint{0.459191in}{0.481694in}}{\pgfqpoint{0.451378in}{0.473880in}}%
\pgfpathcurveto{\pgfqpoint{0.443564in}{0.466067in}}{\pgfqpoint{0.439174in}{0.455468in}}{\pgfqpoint{0.439174in}{0.444417in}}%
\pgfpathcurveto{\pgfqpoint{0.439174in}{0.433367in}}{\pgfqpoint{0.443564in}{0.422768in}}{\pgfqpoint{0.451378in}{0.414955in}}%
\pgfpathcurveto{\pgfqpoint{0.459191in}{0.407141in}}{\pgfqpoint{0.469790in}{0.402751in}}{\pgfqpoint{0.480841in}{0.402751in}}%
\pgfpathclose%
\pgfusepath{stroke,fill}%
\end{pgfscope}%
\begin{pgfscope}%
\pgfpathrectangle{\pgfqpoint{0.375000in}{0.330000in}}{\pgfqpoint{2.325000in}{2.310000in}}%
\pgfusepath{clip}%
\pgfsetbuttcap%
\pgfsetroundjoin%
\definecolor{currentfill}{rgb}{0.000000,0.000000,0.000000}%
\pgfsetfillcolor{currentfill}%
\pgfsetlinewidth{1.003750pt}%
\definecolor{currentstroke}{rgb}{0.000000,0.000000,0.000000}%
\pgfsetstrokecolor{currentstroke}%
\pgfsetdash{}{0pt}%
\pgfpathmoveto{\pgfqpoint{0.480841in}{0.402751in}}%
\pgfpathcurveto{\pgfqpoint{0.491891in}{0.402751in}}{\pgfqpoint{0.502490in}{0.407141in}}{\pgfqpoint{0.510303in}{0.414955in}}%
\pgfpathcurveto{\pgfqpoint{0.518117in}{0.422768in}}{\pgfqpoint{0.522507in}{0.433367in}}{\pgfqpoint{0.522507in}{0.444417in}}%
\pgfpathcurveto{\pgfqpoint{0.522507in}{0.455468in}}{\pgfqpoint{0.518117in}{0.466067in}}{\pgfqpoint{0.510303in}{0.473880in}}%
\pgfpathcurveto{\pgfqpoint{0.502490in}{0.481694in}}{\pgfqpoint{0.491891in}{0.486084in}}{\pgfqpoint{0.480841in}{0.486084in}}%
\pgfpathcurveto{\pgfqpoint{0.469790in}{0.486084in}}{\pgfqpoint{0.459191in}{0.481694in}}{\pgfqpoint{0.451378in}{0.473880in}}%
\pgfpathcurveto{\pgfqpoint{0.443564in}{0.466067in}}{\pgfqpoint{0.439174in}{0.455468in}}{\pgfqpoint{0.439174in}{0.444417in}}%
\pgfpathcurveto{\pgfqpoint{0.439174in}{0.433367in}}{\pgfqpoint{0.443564in}{0.422768in}}{\pgfqpoint{0.451378in}{0.414955in}}%
\pgfpathcurveto{\pgfqpoint{0.459191in}{0.407141in}}{\pgfqpoint{0.469790in}{0.402751in}}{\pgfqpoint{0.480841in}{0.402751in}}%
\pgfpathclose%
\pgfusepath{stroke,fill}%
\end{pgfscope}%
\begin{pgfscope}%
\pgfpathrectangle{\pgfqpoint{0.375000in}{0.330000in}}{\pgfqpoint{2.325000in}{2.310000in}}%
\pgfusepath{clip}%
\pgfsetbuttcap%
\pgfsetroundjoin%
\definecolor{currentfill}{rgb}{0.000000,0.000000,0.000000}%
\pgfsetfillcolor{currentfill}%
\pgfsetlinewidth{1.003750pt}%
\definecolor{currentstroke}{rgb}{0.000000,0.000000,0.000000}%
\pgfsetstrokecolor{currentstroke}%
\pgfsetdash{}{0pt}%
\pgfpathmoveto{\pgfqpoint{0.480841in}{0.402751in}}%
\pgfpathcurveto{\pgfqpoint{0.491891in}{0.402751in}}{\pgfqpoint{0.502490in}{0.407141in}}{\pgfqpoint{0.510303in}{0.414955in}}%
\pgfpathcurveto{\pgfqpoint{0.518117in}{0.422768in}}{\pgfqpoint{0.522507in}{0.433367in}}{\pgfqpoint{0.522507in}{0.444417in}}%
\pgfpathcurveto{\pgfqpoint{0.522507in}{0.455468in}}{\pgfqpoint{0.518117in}{0.466067in}}{\pgfqpoint{0.510303in}{0.473880in}}%
\pgfpathcurveto{\pgfqpoint{0.502490in}{0.481694in}}{\pgfqpoint{0.491891in}{0.486084in}}{\pgfqpoint{0.480841in}{0.486084in}}%
\pgfpathcurveto{\pgfqpoint{0.469790in}{0.486084in}}{\pgfqpoint{0.459191in}{0.481694in}}{\pgfqpoint{0.451378in}{0.473880in}}%
\pgfpathcurveto{\pgfqpoint{0.443564in}{0.466067in}}{\pgfqpoint{0.439174in}{0.455468in}}{\pgfqpoint{0.439174in}{0.444417in}}%
\pgfpathcurveto{\pgfqpoint{0.439174in}{0.433367in}}{\pgfqpoint{0.443564in}{0.422768in}}{\pgfqpoint{0.451378in}{0.414955in}}%
\pgfpathcurveto{\pgfqpoint{0.459191in}{0.407141in}}{\pgfqpoint{0.469790in}{0.402751in}}{\pgfqpoint{0.480841in}{0.402751in}}%
\pgfpathclose%
\pgfusepath{stroke,fill}%
\end{pgfscope}%
\begin{pgfscope}%
\pgfpathrectangle{\pgfqpoint{0.375000in}{0.330000in}}{\pgfqpoint{2.325000in}{2.310000in}}%
\pgfusepath{clip}%
\pgfsetbuttcap%
\pgfsetroundjoin%
\definecolor{currentfill}{rgb}{0.000000,0.000000,0.000000}%
\pgfsetfillcolor{currentfill}%
\pgfsetlinewidth{1.003750pt}%
\definecolor{currentstroke}{rgb}{0.000000,0.000000,0.000000}%
\pgfsetstrokecolor{currentstroke}%
\pgfsetdash{}{0pt}%
\pgfpathmoveto{\pgfqpoint{0.480841in}{0.402751in}}%
\pgfpathcurveto{\pgfqpoint{0.491891in}{0.402751in}}{\pgfqpoint{0.502490in}{0.407141in}}{\pgfqpoint{0.510303in}{0.414955in}}%
\pgfpathcurveto{\pgfqpoint{0.518117in}{0.422768in}}{\pgfqpoint{0.522507in}{0.433367in}}{\pgfqpoint{0.522507in}{0.444417in}}%
\pgfpathcurveto{\pgfqpoint{0.522507in}{0.455468in}}{\pgfqpoint{0.518117in}{0.466067in}}{\pgfqpoint{0.510303in}{0.473880in}}%
\pgfpathcurveto{\pgfqpoint{0.502490in}{0.481694in}}{\pgfqpoint{0.491891in}{0.486084in}}{\pgfqpoint{0.480841in}{0.486084in}}%
\pgfpathcurveto{\pgfqpoint{0.469790in}{0.486084in}}{\pgfqpoint{0.459191in}{0.481694in}}{\pgfqpoint{0.451378in}{0.473880in}}%
\pgfpathcurveto{\pgfqpoint{0.443564in}{0.466067in}}{\pgfqpoint{0.439174in}{0.455468in}}{\pgfqpoint{0.439174in}{0.444417in}}%
\pgfpathcurveto{\pgfqpoint{0.439174in}{0.433367in}}{\pgfqpoint{0.443564in}{0.422768in}}{\pgfqpoint{0.451378in}{0.414955in}}%
\pgfpathcurveto{\pgfqpoint{0.459191in}{0.407141in}}{\pgfqpoint{0.469790in}{0.402751in}}{\pgfqpoint{0.480841in}{0.402751in}}%
\pgfpathclose%
\pgfusepath{stroke,fill}%
\end{pgfscope}%
\begin{pgfscope}%
\pgfpathrectangle{\pgfqpoint{0.375000in}{0.330000in}}{\pgfqpoint{2.325000in}{2.310000in}}%
\pgfusepath{clip}%
\pgfsetbuttcap%
\pgfsetroundjoin%
\definecolor{currentfill}{rgb}{0.000000,0.000000,0.000000}%
\pgfsetfillcolor{currentfill}%
\pgfsetlinewidth{1.003750pt}%
\definecolor{currentstroke}{rgb}{0.000000,0.000000,0.000000}%
\pgfsetstrokecolor{currentstroke}%
\pgfsetdash{}{0pt}%
\pgfpathmoveto{\pgfqpoint{0.480841in}{1.438667in}}%
\pgfpathcurveto{\pgfqpoint{0.491891in}{1.438667in}}{\pgfqpoint{0.502490in}{1.443057in}}{\pgfqpoint{0.510303in}{1.450871in}}%
\pgfpathcurveto{\pgfqpoint{0.518117in}{1.458685in}}{\pgfqpoint{0.522507in}{1.469284in}}{\pgfqpoint{0.522507in}{1.480334in}}%
\pgfpathcurveto{\pgfqpoint{0.522507in}{1.491384in}}{\pgfqpoint{0.518117in}{1.501983in}}{\pgfqpoint{0.510303in}{1.509796in}}%
\pgfpathcurveto{\pgfqpoint{0.502490in}{1.517610in}}{\pgfqpoint{0.491891in}{1.522000in}}{\pgfqpoint{0.480841in}{1.522000in}}%
\pgfpathcurveto{\pgfqpoint{0.469790in}{1.522000in}}{\pgfqpoint{0.459191in}{1.517610in}}{\pgfqpoint{0.451378in}{1.509796in}}%
\pgfpathcurveto{\pgfqpoint{0.443564in}{1.501983in}}{\pgfqpoint{0.439174in}{1.491384in}}{\pgfqpoint{0.439174in}{1.480334in}}%
\pgfpathcurveto{\pgfqpoint{0.439174in}{1.469284in}}{\pgfqpoint{0.443564in}{1.458685in}}{\pgfqpoint{0.451378in}{1.450871in}}%
\pgfpathcurveto{\pgfqpoint{0.459191in}{1.443057in}}{\pgfqpoint{0.469790in}{1.438667in}}{\pgfqpoint{0.480841in}{1.438667in}}%
\pgfpathclose%
\pgfusepath{stroke,fill}%
\end{pgfscope}%
\begin{pgfscope}%
\pgfpathrectangle{\pgfqpoint{0.375000in}{0.330000in}}{\pgfqpoint{2.325000in}{2.310000in}}%
\pgfusepath{clip}%
\pgfsetbuttcap%
\pgfsetroundjoin%
\definecolor{currentfill}{rgb}{0.000000,0.000000,0.000000}%
\pgfsetfillcolor{currentfill}%
\pgfsetlinewidth{1.003750pt}%
\definecolor{currentstroke}{rgb}{0.000000,0.000000,0.000000}%
\pgfsetstrokecolor{currentstroke}%
\pgfsetdash{}{0pt}%
\pgfpathmoveto{\pgfqpoint{0.480841in}{0.402751in}}%
\pgfpathcurveto{\pgfqpoint{0.491891in}{0.402751in}}{\pgfqpoint{0.502490in}{0.407141in}}{\pgfqpoint{0.510303in}{0.414955in}}%
\pgfpathcurveto{\pgfqpoint{0.518117in}{0.422768in}}{\pgfqpoint{0.522507in}{0.433367in}}{\pgfqpoint{0.522507in}{0.444417in}}%
\pgfpathcurveto{\pgfqpoint{0.522507in}{0.455468in}}{\pgfqpoint{0.518117in}{0.466067in}}{\pgfqpoint{0.510303in}{0.473880in}}%
\pgfpathcurveto{\pgfqpoint{0.502490in}{0.481694in}}{\pgfqpoint{0.491891in}{0.486084in}}{\pgfqpoint{0.480841in}{0.486084in}}%
\pgfpathcurveto{\pgfqpoint{0.469790in}{0.486084in}}{\pgfqpoint{0.459191in}{0.481694in}}{\pgfqpoint{0.451378in}{0.473880in}}%
\pgfpathcurveto{\pgfqpoint{0.443564in}{0.466067in}}{\pgfqpoint{0.439174in}{0.455468in}}{\pgfqpoint{0.439174in}{0.444417in}}%
\pgfpathcurveto{\pgfqpoint{0.439174in}{0.433367in}}{\pgfqpoint{0.443564in}{0.422768in}}{\pgfqpoint{0.451378in}{0.414955in}}%
\pgfpathcurveto{\pgfqpoint{0.459191in}{0.407141in}}{\pgfqpoint{0.469790in}{0.402751in}}{\pgfqpoint{0.480841in}{0.402751in}}%
\pgfpathclose%
\pgfusepath{stroke,fill}%
\end{pgfscope}%
\begin{pgfscope}%
\pgfpathrectangle{\pgfqpoint{0.375000in}{0.330000in}}{\pgfqpoint{2.325000in}{2.310000in}}%
\pgfusepath{clip}%
\pgfsetbuttcap%
\pgfsetroundjoin%
\definecolor{currentfill}{rgb}{0.000000,0.000000,0.000000}%
\pgfsetfillcolor{currentfill}%
\pgfsetlinewidth{1.003750pt}%
\definecolor{currentstroke}{rgb}{0.000000,0.000000,0.000000}%
\pgfsetstrokecolor{currentstroke}%
\pgfsetdash{}{0pt}%
\pgfpathmoveto{\pgfqpoint{0.480841in}{0.402751in}}%
\pgfpathcurveto{\pgfqpoint{0.491891in}{0.402751in}}{\pgfqpoint{0.502490in}{0.407141in}}{\pgfqpoint{0.510303in}{0.414955in}}%
\pgfpathcurveto{\pgfqpoint{0.518117in}{0.422768in}}{\pgfqpoint{0.522507in}{0.433367in}}{\pgfqpoint{0.522507in}{0.444417in}}%
\pgfpathcurveto{\pgfqpoint{0.522507in}{0.455468in}}{\pgfqpoint{0.518117in}{0.466067in}}{\pgfqpoint{0.510303in}{0.473880in}}%
\pgfpathcurveto{\pgfqpoint{0.502490in}{0.481694in}}{\pgfqpoint{0.491891in}{0.486084in}}{\pgfqpoint{0.480841in}{0.486084in}}%
\pgfpathcurveto{\pgfqpoint{0.469790in}{0.486084in}}{\pgfqpoint{0.459191in}{0.481694in}}{\pgfqpoint{0.451378in}{0.473880in}}%
\pgfpathcurveto{\pgfqpoint{0.443564in}{0.466067in}}{\pgfqpoint{0.439174in}{0.455468in}}{\pgfqpoint{0.439174in}{0.444417in}}%
\pgfpathcurveto{\pgfqpoint{0.439174in}{0.433367in}}{\pgfqpoint{0.443564in}{0.422768in}}{\pgfqpoint{0.451378in}{0.414955in}}%
\pgfpathcurveto{\pgfqpoint{0.459191in}{0.407141in}}{\pgfqpoint{0.469790in}{0.402751in}}{\pgfqpoint{0.480841in}{0.402751in}}%
\pgfpathclose%
\pgfusepath{stroke,fill}%
\end{pgfscope}%
\begin{pgfscope}%
\pgfpathrectangle{\pgfqpoint{0.375000in}{0.330000in}}{\pgfqpoint{2.325000in}{2.310000in}}%
\pgfusepath{clip}%
\pgfsetbuttcap%
\pgfsetroundjoin%
\definecolor{currentfill}{rgb}{0.000000,0.000000,0.000000}%
\pgfsetfillcolor{currentfill}%
\pgfsetlinewidth{1.003750pt}%
\definecolor{currentstroke}{rgb}{0.000000,0.000000,0.000000}%
\pgfsetstrokecolor{currentstroke}%
\pgfsetdash{}{0pt}%
\pgfpathmoveto{\pgfqpoint{0.480841in}{0.402751in}}%
\pgfpathcurveto{\pgfqpoint{0.491891in}{0.402751in}}{\pgfqpoint{0.502490in}{0.407141in}}{\pgfqpoint{0.510303in}{0.414955in}}%
\pgfpathcurveto{\pgfqpoint{0.518117in}{0.422768in}}{\pgfqpoint{0.522507in}{0.433367in}}{\pgfqpoint{0.522507in}{0.444417in}}%
\pgfpathcurveto{\pgfqpoint{0.522507in}{0.455468in}}{\pgfqpoint{0.518117in}{0.466067in}}{\pgfqpoint{0.510303in}{0.473880in}}%
\pgfpathcurveto{\pgfqpoint{0.502490in}{0.481694in}}{\pgfqpoint{0.491891in}{0.486084in}}{\pgfqpoint{0.480841in}{0.486084in}}%
\pgfpathcurveto{\pgfqpoint{0.469790in}{0.486084in}}{\pgfqpoint{0.459191in}{0.481694in}}{\pgfqpoint{0.451378in}{0.473880in}}%
\pgfpathcurveto{\pgfqpoint{0.443564in}{0.466067in}}{\pgfqpoint{0.439174in}{0.455468in}}{\pgfqpoint{0.439174in}{0.444417in}}%
\pgfpathcurveto{\pgfqpoint{0.439174in}{0.433367in}}{\pgfqpoint{0.443564in}{0.422768in}}{\pgfqpoint{0.451378in}{0.414955in}}%
\pgfpathcurveto{\pgfqpoint{0.459191in}{0.407141in}}{\pgfqpoint{0.469790in}{0.402751in}}{\pgfqpoint{0.480841in}{0.402751in}}%
\pgfpathclose%
\pgfusepath{stroke,fill}%
\end{pgfscope}%
\begin{pgfscope}%
\pgfpathrectangle{\pgfqpoint{0.375000in}{0.330000in}}{\pgfqpoint{2.325000in}{2.310000in}}%
\pgfusepath{clip}%
\pgfsetbuttcap%
\pgfsetroundjoin%
\definecolor{currentfill}{rgb}{0.000000,0.000000,0.000000}%
\pgfsetfillcolor{currentfill}%
\pgfsetlinewidth{1.003750pt}%
\definecolor{currentstroke}{rgb}{0.000000,0.000000,0.000000}%
\pgfsetstrokecolor{currentstroke}%
\pgfsetdash{}{0pt}%
\pgfpathmoveto{\pgfqpoint{0.480841in}{0.402751in}}%
\pgfpathcurveto{\pgfqpoint{0.491891in}{0.402751in}}{\pgfqpoint{0.502490in}{0.407141in}}{\pgfqpoint{0.510303in}{0.414955in}}%
\pgfpathcurveto{\pgfqpoint{0.518117in}{0.422768in}}{\pgfqpoint{0.522507in}{0.433367in}}{\pgfqpoint{0.522507in}{0.444417in}}%
\pgfpathcurveto{\pgfqpoint{0.522507in}{0.455468in}}{\pgfqpoint{0.518117in}{0.466067in}}{\pgfqpoint{0.510303in}{0.473880in}}%
\pgfpathcurveto{\pgfqpoint{0.502490in}{0.481694in}}{\pgfqpoint{0.491891in}{0.486084in}}{\pgfqpoint{0.480841in}{0.486084in}}%
\pgfpathcurveto{\pgfqpoint{0.469790in}{0.486084in}}{\pgfqpoint{0.459191in}{0.481694in}}{\pgfqpoint{0.451378in}{0.473880in}}%
\pgfpathcurveto{\pgfqpoint{0.443564in}{0.466067in}}{\pgfqpoint{0.439174in}{0.455468in}}{\pgfqpoint{0.439174in}{0.444417in}}%
\pgfpathcurveto{\pgfqpoint{0.439174in}{0.433367in}}{\pgfqpoint{0.443564in}{0.422768in}}{\pgfqpoint{0.451378in}{0.414955in}}%
\pgfpathcurveto{\pgfqpoint{0.459191in}{0.407141in}}{\pgfqpoint{0.469790in}{0.402751in}}{\pgfqpoint{0.480841in}{0.402751in}}%
\pgfpathclose%
\pgfusepath{stroke,fill}%
\end{pgfscope}%
\begin{pgfscope}%
\pgfpathrectangle{\pgfqpoint{0.375000in}{0.330000in}}{\pgfqpoint{2.325000in}{2.310000in}}%
\pgfusepath{clip}%
\pgfsetbuttcap%
\pgfsetroundjoin%
\definecolor{currentfill}{rgb}{0.000000,0.000000,0.000000}%
\pgfsetfillcolor{currentfill}%
\pgfsetlinewidth{1.003750pt}%
\definecolor{currentstroke}{rgb}{0.000000,0.000000,0.000000}%
\pgfsetstrokecolor{currentstroke}%
\pgfsetdash{}{0pt}%
\pgfpathmoveto{\pgfqpoint{0.480841in}{1.438667in}}%
\pgfpathcurveto{\pgfqpoint{0.491891in}{1.438667in}}{\pgfqpoint{0.502490in}{1.443057in}}{\pgfqpoint{0.510303in}{1.450871in}}%
\pgfpathcurveto{\pgfqpoint{0.518117in}{1.458685in}}{\pgfqpoint{0.522507in}{1.469284in}}{\pgfqpoint{0.522507in}{1.480334in}}%
\pgfpathcurveto{\pgfqpoint{0.522507in}{1.491384in}}{\pgfqpoint{0.518117in}{1.501983in}}{\pgfqpoint{0.510303in}{1.509796in}}%
\pgfpathcurveto{\pgfqpoint{0.502490in}{1.517610in}}{\pgfqpoint{0.491891in}{1.522000in}}{\pgfqpoint{0.480841in}{1.522000in}}%
\pgfpathcurveto{\pgfqpoint{0.469790in}{1.522000in}}{\pgfqpoint{0.459191in}{1.517610in}}{\pgfqpoint{0.451378in}{1.509796in}}%
\pgfpathcurveto{\pgfqpoint{0.443564in}{1.501983in}}{\pgfqpoint{0.439174in}{1.491384in}}{\pgfqpoint{0.439174in}{1.480334in}}%
\pgfpathcurveto{\pgfqpoint{0.439174in}{1.469284in}}{\pgfqpoint{0.443564in}{1.458685in}}{\pgfqpoint{0.451378in}{1.450871in}}%
\pgfpathcurveto{\pgfqpoint{0.459191in}{1.443057in}}{\pgfqpoint{0.469790in}{1.438667in}}{\pgfqpoint{0.480841in}{1.438667in}}%
\pgfpathclose%
\pgfusepath{stroke,fill}%
\end{pgfscope}%
\begin{pgfscope}%
\pgfpathrectangle{\pgfqpoint{0.375000in}{0.330000in}}{\pgfqpoint{2.325000in}{2.310000in}}%
\pgfusepath{clip}%
\pgfsetbuttcap%
\pgfsetroundjoin%
\definecolor{currentfill}{rgb}{0.000000,0.000000,0.000000}%
\pgfsetfillcolor{currentfill}%
\pgfsetlinewidth{1.003750pt}%
\definecolor{currentstroke}{rgb}{0.000000,0.000000,0.000000}%
\pgfsetstrokecolor{currentstroke}%
\pgfsetdash{}{0pt}%
\pgfpathmoveto{\pgfqpoint{0.480841in}{0.402751in}}%
\pgfpathcurveto{\pgfqpoint{0.491891in}{0.402751in}}{\pgfqpoint{0.502490in}{0.407141in}}{\pgfqpoint{0.510303in}{0.414955in}}%
\pgfpathcurveto{\pgfqpoint{0.518117in}{0.422768in}}{\pgfqpoint{0.522507in}{0.433367in}}{\pgfqpoint{0.522507in}{0.444417in}}%
\pgfpathcurveto{\pgfqpoint{0.522507in}{0.455468in}}{\pgfqpoint{0.518117in}{0.466067in}}{\pgfqpoint{0.510303in}{0.473880in}}%
\pgfpathcurveto{\pgfqpoint{0.502490in}{0.481694in}}{\pgfqpoint{0.491891in}{0.486084in}}{\pgfqpoint{0.480841in}{0.486084in}}%
\pgfpathcurveto{\pgfqpoint{0.469790in}{0.486084in}}{\pgfqpoint{0.459191in}{0.481694in}}{\pgfqpoint{0.451378in}{0.473880in}}%
\pgfpathcurveto{\pgfqpoint{0.443564in}{0.466067in}}{\pgfqpoint{0.439174in}{0.455468in}}{\pgfqpoint{0.439174in}{0.444417in}}%
\pgfpathcurveto{\pgfqpoint{0.439174in}{0.433367in}}{\pgfqpoint{0.443564in}{0.422768in}}{\pgfqpoint{0.451378in}{0.414955in}}%
\pgfpathcurveto{\pgfqpoint{0.459191in}{0.407141in}}{\pgfqpoint{0.469790in}{0.402751in}}{\pgfqpoint{0.480841in}{0.402751in}}%
\pgfpathclose%
\pgfusepath{stroke,fill}%
\end{pgfscope}%
\begin{pgfscope}%
\pgfpathrectangle{\pgfqpoint{0.375000in}{0.330000in}}{\pgfqpoint{2.325000in}{2.310000in}}%
\pgfusepath{clip}%
\pgfsetbuttcap%
\pgfsetroundjoin%
\definecolor{currentfill}{rgb}{0.000000,0.000000,0.000000}%
\pgfsetfillcolor{currentfill}%
\pgfsetlinewidth{1.003750pt}%
\definecolor{currentstroke}{rgb}{0.000000,0.000000,0.000000}%
\pgfsetstrokecolor{currentstroke}%
\pgfsetdash{}{0pt}%
\pgfpathmoveto{\pgfqpoint{0.480841in}{0.402751in}}%
\pgfpathcurveto{\pgfqpoint{0.491891in}{0.402751in}}{\pgfqpoint{0.502490in}{0.407141in}}{\pgfqpoint{0.510303in}{0.414955in}}%
\pgfpathcurveto{\pgfqpoint{0.518117in}{0.422768in}}{\pgfqpoint{0.522507in}{0.433367in}}{\pgfqpoint{0.522507in}{0.444417in}}%
\pgfpathcurveto{\pgfqpoint{0.522507in}{0.455468in}}{\pgfqpoint{0.518117in}{0.466067in}}{\pgfqpoint{0.510303in}{0.473880in}}%
\pgfpathcurveto{\pgfqpoint{0.502490in}{0.481694in}}{\pgfqpoint{0.491891in}{0.486084in}}{\pgfqpoint{0.480841in}{0.486084in}}%
\pgfpathcurveto{\pgfqpoint{0.469790in}{0.486084in}}{\pgfqpoint{0.459191in}{0.481694in}}{\pgfqpoint{0.451378in}{0.473880in}}%
\pgfpathcurveto{\pgfqpoint{0.443564in}{0.466067in}}{\pgfqpoint{0.439174in}{0.455468in}}{\pgfqpoint{0.439174in}{0.444417in}}%
\pgfpathcurveto{\pgfqpoint{0.439174in}{0.433367in}}{\pgfqpoint{0.443564in}{0.422768in}}{\pgfqpoint{0.451378in}{0.414955in}}%
\pgfpathcurveto{\pgfqpoint{0.459191in}{0.407141in}}{\pgfqpoint{0.469790in}{0.402751in}}{\pgfqpoint{0.480841in}{0.402751in}}%
\pgfpathclose%
\pgfusepath{stroke,fill}%
\end{pgfscope}%
\begin{pgfscope}%
\pgfpathrectangle{\pgfqpoint{0.375000in}{0.330000in}}{\pgfqpoint{2.325000in}{2.310000in}}%
\pgfusepath{clip}%
\pgfsetbuttcap%
\pgfsetroundjoin%
\definecolor{currentfill}{rgb}{0.000000,0.000000,0.000000}%
\pgfsetfillcolor{currentfill}%
\pgfsetlinewidth{1.003750pt}%
\definecolor{currentstroke}{rgb}{0.000000,0.000000,0.000000}%
\pgfsetstrokecolor{currentstroke}%
\pgfsetdash{}{0pt}%
\pgfpathmoveto{\pgfqpoint{0.480841in}{0.402751in}}%
\pgfpathcurveto{\pgfqpoint{0.491891in}{0.402751in}}{\pgfqpoint{0.502490in}{0.407141in}}{\pgfqpoint{0.510303in}{0.414955in}}%
\pgfpathcurveto{\pgfqpoint{0.518117in}{0.422768in}}{\pgfqpoint{0.522507in}{0.433367in}}{\pgfqpoint{0.522507in}{0.444417in}}%
\pgfpathcurveto{\pgfqpoint{0.522507in}{0.455468in}}{\pgfqpoint{0.518117in}{0.466067in}}{\pgfqpoint{0.510303in}{0.473880in}}%
\pgfpathcurveto{\pgfqpoint{0.502490in}{0.481694in}}{\pgfqpoint{0.491891in}{0.486084in}}{\pgfqpoint{0.480841in}{0.486084in}}%
\pgfpathcurveto{\pgfqpoint{0.469790in}{0.486084in}}{\pgfqpoint{0.459191in}{0.481694in}}{\pgfqpoint{0.451378in}{0.473880in}}%
\pgfpathcurveto{\pgfqpoint{0.443564in}{0.466067in}}{\pgfqpoint{0.439174in}{0.455468in}}{\pgfqpoint{0.439174in}{0.444417in}}%
\pgfpathcurveto{\pgfqpoint{0.439174in}{0.433367in}}{\pgfqpoint{0.443564in}{0.422768in}}{\pgfqpoint{0.451378in}{0.414955in}}%
\pgfpathcurveto{\pgfqpoint{0.459191in}{0.407141in}}{\pgfqpoint{0.469790in}{0.402751in}}{\pgfqpoint{0.480841in}{0.402751in}}%
\pgfpathclose%
\pgfusepath{stroke,fill}%
\end{pgfscope}%
\begin{pgfscope}%
\pgfpathrectangle{\pgfqpoint{0.375000in}{0.330000in}}{\pgfqpoint{2.325000in}{2.310000in}}%
\pgfusepath{clip}%
\pgfsetbuttcap%
\pgfsetroundjoin%
\definecolor{currentfill}{rgb}{0.000000,0.000000,0.000000}%
\pgfsetfillcolor{currentfill}%
\pgfsetlinewidth{1.003750pt}%
\definecolor{currentstroke}{rgb}{0.000000,0.000000,0.000000}%
\pgfsetstrokecolor{currentstroke}%
\pgfsetdash{}{0pt}%
\pgfpathmoveto{\pgfqpoint{0.480841in}{0.402751in}}%
\pgfpathcurveto{\pgfqpoint{0.491891in}{0.402751in}}{\pgfqpoint{0.502490in}{0.407141in}}{\pgfqpoint{0.510303in}{0.414955in}}%
\pgfpathcurveto{\pgfqpoint{0.518117in}{0.422768in}}{\pgfqpoint{0.522507in}{0.433367in}}{\pgfqpoint{0.522507in}{0.444417in}}%
\pgfpathcurveto{\pgfqpoint{0.522507in}{0.455468in}}{\pgfqpoint{0.518117in}{0.466067in}}{\pgfqpoint{0.510303in}{0.473880in}}%
\pgfpathcurveto{\pgfqpoint{0.502490in}{0.481694in}}{\pgfqpoint{0.491891in}{0.486084in}}{\pgfqpoint{0.480841in}{0.486084in}}%
\pgfpathcurveto{\pgfqpoint{0.469790in}{0.486084in}}{\pgfqpoint{0.459191in}{0.481694in}}{\pgfqpoint{0.451378in}{0.473880in}}%
\pgfpathcurveto{\pgfqpoint{0.443564in}{0.466067in}}{\pgfqpoint{0.439174in}{0.455468in}}{\pgfqpoint{0.439174in}{0.444417in}}%
\pgfpathcurveto{\pgfqpoint{0.439174in}{0.433367in}}{\pgfqpoint{0.443564in}{0.422768in}}{\pgfqpoint{0.451378in}{0.414955in}}%
\pgfpathcurveto{\pgfqpoint{0.459191in}{0.407141in}}{\pgfqpoint{0.469790in}{0.402751in}}{\pgfqpoint{0.480841in}{0.402751in}}%
\pgfpathclose%
\pgfusepath{stroke,fill}%
\end{pgfscope}%
\begin{pgfscope}%
\pgfpathrectangle{\pgfqpoint{0.375000in}{0.330000in}}{\pgfqpoint{2.325000in}{2.310000in}}%
\pgfusepath{clip}%
\pgfsetbuttcap%
\pgfsetroundjoin%
\definecolor{currentfill}{rgb}{0.000000,0.000000,0.000000}%
\pgfsetfillcolor{currentfill}%
\pgfsetlinewidth{1.003750pt}%
\definecolor{currentstroke}{rgb}{0.000000,0.000000,0.000000}%
\pgfsetstrokecolor{currentstroke}%
\pgfsetdash{}{0pt}%
\pgfpathmoveto{\pgfqpoint{0.480841in}{0.402751in}}%
\pgfpathcurveto{\pgfqpoint{0.491891in}{0.402751in}}{\pgfqpoint{0.502490in}{0.407141in}}{\pgfqpoint{0.510303in}{0.414955in}}%
\pgfpathcurveto{\pgfqpoint{0.518117in}{0.422768in}}{\pgfqpoint{0.522507in}{0.433367in}}{\pgfqpoint{0.522507in}{0.444417in}}%
\pgfpathcurveto{\pgfqpoint{0.522507in}{0.455468in}}{\pgfqpoint{0.518117in}{0.466067in}}{\pgfqpoint{0.510303in}{0.473880in}}%
\pgfpathcurveto{\pgfqpoint{0.502490in}{0.481694in}}{\pgfqpoint{0.491891in}{0.486084in}}{\pgfqpoint{0.480841in}{0.486084in}}%
\pgfpathcurveto{\pgfqpoint{0.469790in}{0.486084in}}{\pgfqpoint{0.459191in}{0.481694in}}{\pgfqpoint{0.451378in}{0.473880in}}%
\pgfpathcurveto{\pgfqpoint{0.443564in}{0.466067in}}{\pgfqpoint{0.439174in}{0.455468in}}{\pgfqpoint{0.439174in}{0.444417in}}%
\pgfpathcurveto{\pgfqpoint{0.439174in}{0.433367in}}{\pgfqpoint{0.443564in}{0.422768in}}{\pgfqpoint{0.451378in}{0.414955in}}%
\pgfpathcurveto{\pgfqpoint{0.459191in}{0.407141in}}{\pgfqpoint{0.469790in}{0.402751in}}{\pgfqpoint{0.480841in}{0.402751in}}%
\pgfpathclose%
\pgfusepath{stroke,fill}%
\end{pgfscope}%
\begin{pgfscope}%
\pgfpathrectangle{\pgfqpoint{0.375000in}{0.330000in}}{\pgfqpoint{2.325000in}{2.310000in}}%
\pgfusepath{clip}%
\pgfsetbuttcap%
\pgfsetroundjoin%
\definecolor{currentfill}{rgb}{0.000000,0.000000,0.000000}%
\pgfsetfillcolor{currentfill}%
\pgfsetlinewidth{1.003750pt}%
\definecolor{currentstroke}{rgb}{0.000000,0.000000,0.000000}%
\pgfsetstrokecolor{currentstroke}%
\pgfsetdash{}{0pt}%
\pgfpathmoveto{\pgfqpoint{0.480841in}{0.402751in}}%
\pgfpathcurveto{\pgfqpoint{0.491891in}{0.402751in}}{\pgfqpoint{0.502490in}{0.407141in}}{\pgfqpoint{0.510303in}{0.414955in}}%
\pgfpathcurveto{\pgfqpoint{0.518117in}{0.422768in}}{\pgfqpoint{0.522507in}{0.433367in}}{\pgfqpoint{0.522507in}{0.444417in}}%
\pgfpathcurveto{\pgfqpoint{0.522507in}{0.455468in}}{\pgfqpoint{0.518117in}{0.466067in}}{\pgfqpoint{0.510303in}{0.473880in}}%
\pgfpathcurveto{\pgfqpoint{0.502490in}{0.481694in}}{\pgfqpoint{0.491891in}{0.486084in}}{\pgfqpoint{0.480841in}{0.486084in}}%
\pgfpathcurveto{\pgfqpoint{0.469790in}{0.486084in}}{\pgfqpoint{0.459191in}{0.481694in}}{\pgfqpoint{0.451378in}{0.473880in}}%
\pgfpathcurveto{\pgfqpoint{0.443564in}{0.466067in}}{\pgfqpoint{0.439174in}{0.455468in}}{\pgfqpoint{0.439174in}{0.444417in}}%
\pgfpathcurveto{\pgfqpoint{0.439174in}{0.433367in}}{\pgfqpoint{0.443564in}{0.422768in}}{\pgfqpoint{0.451378in}{0.414955in}}%
\pgfpathcurveto{\pgfqpoint{0.459191in}{0.407141in}}{\pgfqpoint{0.469790in}{0.402751in}}{\pgfqpoint{0.480841in}{0.402751in}}%
\pgfpathclose%
\pgfusepath{stroke,fill}%
\end{pgfscope}%
\begin{pgfscope}%
\pgfpathrectangle{\pgfqpoint{0.375000in}{0.330000in}}{\pgfqpoint{2.325000in}{2.310000in}}%
\pgfusepath{clip}%
\pgfsetbuttcap%
\pgfsetroundjoin%
\definecolor{currentfill}{rgb}{0.000000,0.000000,0.000000}%
\pgfsetfillcolor{currentfill}%
\pgfsetlinewidth{1.003750pt}%
\definecolor{currentstroke}{rgb}{0.000000,0.000000,0.000000}%
\pgfsetstrokecolor{currentstroke}%
\pgfsetdash{}{0pt}%
\pgfpathmoveto{\pgfqpoint{0.480841in}{0.402751in}}%
\pgfpathcurveto{\pgfqpoint{0.491891in}{0.402751in}}{\pgfqpoint{0.502490in}{0.407141in}}{\pgfqpoint{0.510303in}{0.414955in}}%
\pgfpathcurveto{\pgfqpoint{0.518117in}{0.422768in}}{\pgfqpoint{0.522507in}{0.433367in}}{\pgfqpoint{0.522507in}{0.444417in}}%
\pgfpathcurveto{\pgfqpoint{0.522507in}{0.455468in}}{\pgfqpoint{0.518117in}{0.466067in}}{\pgfqpoint{0.510303in}{0.473880in}}%
\pgfpathcurveto{\pgfqpoint{0.502490in}{0.481694in}}{\pgfqpoint{0.491891in}{0.486084in}}{\pgfqpoint{0.480841in}{0.486084in}}%
\pgfpathcurveto{\pgfqpoint{0.469790in}{0.486084in}}{\pgfqpoint{0.459191in}{0.481694in}}{\pgfqpoint{0.451378in}{0.473880in}}%
\pgfpathcurveto{\pgfqpoint{0.443564in}{0.466067in}}{\pgfqpoint{0.439174in}{0.455468in}}{\pgfqpoint{0.439174in}{0.444417in}}%
\pgfpathcurveto{\pgfqpoint{0.439174in}{0.433367in}}{\pgfqpoint{0.443564in}{0.422768in}}{\pgfqpoint{0.451378in}{0.414955in}}%
\pgfpathcurveto{\pgfqpoint{0.459191in}{0.407141in}}{\pgfqpoint{0.469790in}{0.402751in}}{\pgfqpoint{0.480841in}{0.402751in}}%
\pgfpathclose%
\pgfusepath{stroke,fill}%
\end{pgfscope}%
\begin{pgfscope}%
\pgfpathrectangle{\pgfqpoint{0.375000in}{0.330000in}}{\pgfqpoint{2.325000in}{2.310000in}}%
\pgfusepath{clip}%
\pgfsetbuttcap%
\pgfsetroundjoin%
\definecolor{currentfill}{rgb}{0.000000,0.000000,0.000000}%
\pgfsetfillcolor{currentfill}%
\pgfsetlinewidth{1.003750pt}%
\definecolor{currentstroke}{rgb}{0.000000,0.000000,0.000000}%
\pgfsetstrokecolor{currentstroke}%
\pgfsetdash{}{0pt}%
\pgfpathmoveto{\pgfqpoint{0.480841in}{0.402751in}}%
\pgfpathcurveto{\pgfqpoint{0.491891in}{0.402751in}}{\pgfqpoint{0.502490in}{0.407141in}}{\pgfqpoint{0.510303in}{0.414955in}}%
\pgfpathcurveto{\pgfqpoint{0.518117in}{0.422768in}}{\pgfqpoint{0.522507in}{0.433367in}}{\pgfqpoint{0.522507in}{0.444417in}}%
\pgfpathcurveto{\pgfqpoint{0.522507in}{0.455468in}}{\pgfqpoint{0.518117in}{0.466067in}}{\pgfqpoint{0.510303in}{0.473880in}}%
\pgfpathcurveto{\pgfqpoint{0.502490in}{0.481694in}}{\pgfqpoint{0.491891in}{0.486084in}}{\pgfqpoint{0.480841in}{0.486084in}}%
\pgfpathcurveto{\pgfqpoint{0.469790in}{0.486084in}}{\pgfqpoint{0.459191in}{0.481694in}}{\pgfqpoint{0.451378in}{0.473880in}}%
\pgfpathcurveto{\pgfqpoint{0.443564in}{0.466067in}}{\pgfqpoint{0.439174in}{0.455468in}}{\pgfqpoint{0.439174in}{0.444417in}}%
\pgfpathcurveto{\pgfqpoint{0.439174in}{0.433367in}}{\pgfqpoint{0.443564in}{0.422768in}}{\pgfqpoint{0.451378in}{0.414955in}}%
\pgfpathcurveto{\pgfqpoint{0.459191in}{0.407141in}}{\pgfqpoint{0.469790in}{0.402751in}}{\pgfqpoint{0.480841in}{0.402751in}}%
\pgfpathclose%
\pgfusepath{stroke,fill}%
\end{pgfscope}%
\begin{pgfscope}%
\pgfpathrectangle{\pgfqpoint{0.375000in}{0.330000in}}{\pgfqpoint{2.325000in}{2.310000in}}%
\pgfusepath{clip}%
\pgfsetbuttcap%
\pgfsetroundjoin%
\definecolor{currentfill}{rgb}{0.000000,0.000000,0.000000}%
\pgfsetfillcolor{currentfill}%
\pgfsetlinewidth{1.003750pt}%
\definecolor{currentstroke}{rgb}{0.000000,0.000000,0.000000}%
\pgfsetstrokecolor{currentstroke}%
\pgfsetdash{}{0pt}%
\pgfpathmoveto{\pgfqpoint{0.480841in}{0.402751in}}%
\pgfpathcurveto{\pgfqpoint{0.491891in}{0.402751in}}{\pgfqpoint{0.502490in}{0.407141in}}{\pgfqpoint{0.510303in}{0.414955in}}%
\pgfpathcurveto{\pgfqpoint{0.518117in}{0.422768in}}{\pgfqpoint{0.522507in}{0.433367in}}{\pgfqpoint{0.522507in}{0.444417in}}%
\pgfpathcurveto{\pgfqpoint{0.522507in}{0.455468in}}{\pgfqpoint{0.518117in}{0.466067in}}{\pgfqpoint{0.510303in}{0.473880in}}%
\pgfpathcurveto{\pgfqpoint{0.502490in}{0.481694in}}{\pgfqpoint{0.491891in}{0.486084in}}{\pgfqpoint{0.480841in}{0.486084in}}%
\pgfpathcurveto{\pgfqpoint{0.469790in}{0.486084in}}{\pgfqpoint{0.459191in}{0.481694in}}{\pgfqpoint{0.451378in}{0.473880in}}%
\pgfpathcurveto{\pgfqpoint{0.443564in}{0.466067in}}{\pgfqpoint{0.439174in}{0.455468in}}{\pgfqpoint{0.439174in}{0.444417in}}%
\pgfpathcurveto{\pgfqpoint{0.439174in}{0.433367in}}{\pgfqpoint{0.443564in}{0.422768in}}{\pgfqpoint{0.451378in}{0.414955in}}%
\pgfpathcurveto{\pgfqpoint{0.459191in}{0.407141in}}{\pgfqpoint{0.469790in}{0.402751in}}{\pgfqpoint{0.480841in}{0.402751in}}%
\pgfpathclose%
\pgfusepath{stroke,fill}%
\end{pgfscope}%
\begin{pgfscope}%
\pgfpathrectangle{\pgfqpoint{0.375000in}{0.330000in}}{\pgfqpoint{2.325000in}{2.310000in}}%
\pgfusepath{clip}%
\pgfsetbuttcap%
\pgfsetroundjoin%
\definecolor{currentfill}{rgb}{0.000000,0.000000,0.000000}%
\pgfsetfillcolor{currentfill}%
\pgfsetlinewidth{1.003750pt}%
\definecolor{currentstroke}{rgb}{0.000000,0.000000,0.000000}%
\pgfsetstrokecolor{currentstroke}%
\pgfsetdash{}{0pt}%
\pgfpathmoveto{\pgfqpoint{0.480841in}{0.402751in}}%
\pgfpathcurveto{\pgfqpoint{0.491891in}{0.402751in}}{\pgfqpoint{0.502490in}{0.407141in}}{\pgfqpoint{0.510303in}{0.414955in}}%
\pgfpathcurveto{\pgfqpoint{0.518117in}{0.422768in}}{\pgfqpoint{0.522507in}{0.433367in}}{\pgfqpoint{0.522507in}{0.444417in}}%
\pgfpathcurveto{\pgfqpoint{0.522507in}{0.455468in}}{\pgfqpoint{0.518117in}{0.466067in}}{\pgfqpoint{0.510303in}{0.473880in}}%
\pgfpathcurveto{\pgfqpoint{0.502490in}{0.481694in}}{\pgfqpoint{0.491891in}{0.486084in}}{\pgfqpoint{0.480841in}{0.486084in}}%
\pgfpathcurveto{\pgfqpoint{0.469790in}{0.486084in}}{\pgfqpoint{0.459191in}{0.481694in}}{\pgfqpoint{0.451378in}{0.473880in}}%
\pgfpathcurveto{\pgfqpoint{0.443564in}{0.466067in}}{\pgfqpoint{0.439174in}{0.455468in}}{\pgfqpoint{0.439174in}{0.444417in}}%
\pgfpathcurveto{\pgfqpoint{0.439174in}{0.433367in}}{\pgfqpoint{0.443564in}{0.422768in}}{\pgfqpoint{0.451378in}{0.414955in}}%
\pgfpathcurveto{\pgfqpoint{0.459191in}{0.407141in}}{\pgfqpoint{0.469790in}{0.402751in}}{\pgfqpoint{0.480841in}{0.402751in}}%
\pgfpathclose%
\pgfusepath{stroke,fill}%
\end{pgfscope}%
\begin{pgfscope}%
\pgfpathrectangle{\pgfqpoint{0.375000in}{0.330000in}}{\pgfqpoint{2.325000in}{2.310000in}}%
\pgfusepath{clip}%
\pgfsetbuttcap%
\pgfsetroundjoin%
\definecolor{currentfill}{rgb}{0.000000,0.000000,0.000000}%
\pgfsetfillcolor{currentfill}%
\pgfsetlinewidth{1.003750pt}%
\definecolor{currentstroke}{rgb}{0.000000,0.000000,0.000000}%
\pgfsetstrokecolor{currentstroke}%
\pgfsetdash{}{0pt}%
\pgfpathmoveto{\pgfqpoint{0.480841in}{1.438667in}}%
\pgfpathcurveto{\pgfqpoint{0.491891in}{1.438667in}}{\pgfqpoint{0.502490in}{1.443057in}}{\pgfqpoint{0.510303in}{1.450871in}}%
\pgfpathcurveto{\pgfqpoint{0.518117in}{1.458685in}}{\pgfqpoint{0.522507in}{1.469284in}}{\pgfqpoint{0.522507in}{1.480334in}}%
\pgfpathcurveto{\pgfqpoint{0.522507in}{1.491384in}}{\pgfqpoint{0.518117in}{1.501983in}}{\pgfqpoint{0.510303in}{1.509796in}}%
\pgfpathcurveto{\pgfqpoint{0.502490in}{1.517610in}}{\pgfqpoint{0.491891in}{1.522000in}}{\pgfqpoint{0.480841in}{1.522000in}}%
\pgfpathcurveto{\pgfqpoint{0.469790in}{1.522000in}}{\pgfqpoint{0.459191in}{1.517610in}}{\pgfqpoint{0.451378in}{1.509796in}}%
\pgfpathcurveto{\pgfqpoint{0.443564in}{1.501983in}}{\pgfqpoint{0.439174in}{1.491384in}}{\pgfqpoint{0.439174in}{1.480334in}}%
\pgfpathcurveto{\pgfqpoint{0.439174in}{1.469284in}}{\pgfqpoint{0.443564in}{1.458685in}}{\pgfqpoint{0.451378in}{1.450871in}}%
\pgfpathcurveto{\pgfqpoint{0.459191in}{1.443057in}}{\pgfqpoint{0.469790in}{1.438667in}}{\pgfqpoint{0.480841in}{1.438667in}}%
\pgfpathclose%
\pgfusepath{stroke,fill}%
\end{pgfscope}%
\begin{pgfscope}%
\pgfpathrectangle{\pgfqpoint{0.375000in}{0.330000in}}{\pgfqpoint{2.325000in}{2.310000in}}%
\pgfusepath{clip}%
\pgfsetbuttcap%
\pgfsetroundjoin%
\definecolor{currentfill}{rgb}{0.000000,0.000000,0.000000}%
\pgfsetfillcolor{currentfill}%
\pgfsetlinewidth{1.003750pt}%
\definecolor{currentstroke}{rgb}{0.000000,0.000000,0.000000}%
\pgfsetstrokecolor{currentstroke}%
\pgfsetdash{}{0pt}%
\pgfpathmoveto{\pgfqpoint{0.480841in}{0.402751in}}%
\pgfpathcurveto{\pgfqpoint{0.491891in}{0.402751in}}{\pgfqpoint{0.502490in}{0.407141in}}{\pgfqpoint{0.510303in}{0.414955in}}%
\pgfpathcurveto{\pgfqpoint{0.518117in}{0.422768in}}{\pgfqpoint{0.522507in}{0.433367in}}{\pgfqpoint{0.522507in}{0.444417in}}%
\pgfpathcurveto{\pgfqpoint{0.522507in}{0.455468in}}{\pgfqpoint{0.518117in}{0.466067in}}{\pgfqpoint{0.510303in}{0.473880in}}%
\pgfpathcurveto{\pgfqpoint{0.502490in}{0.481694in}}{\pgfqpoint{0.491891in}{0.486084in}}{\pgfqpoint{0.480841in}{0.486084in}}%
\pgfpathcurveto{\pgfqpoint{0.469790in}{0.486084in}}{\pgfqpoint{0.459191in}{0.481694in}}{\pgfqpoint{0.451378in}{0.473880in}}%
\pgfpathcurveto{\pgfqpoint{0.443564in}{0.466067in}}{\pgfqpoint{0.439174in}{0.455468in}}{\pgfqpoint{0.439174in}{0.444417in}}%
\pgfpathcurveto{\pgfqpoint{0.439174in}{0.433367in}}{\pgfqpoint{0.443564in}{0.422768in}}{\pgfqpoint{0.451378in}{0.414955in}}%
\pgfpathcurveto{\pgfqpoint{0.459191in}{0.407141in}}{\pgfqpoint{0.469790in}{0.402751in}}{\pgfqpoint{0.480841in}{0.402751in}}%
\pgfpathclose%
\pgfusepath{stroke,fill}%
\end{pgfscope}%
\begin{pgfscope}%
\pgfpathrectangle{\pgfqpoint{0.375000in}{0.330000in}}{\pgfqpoint{2.325000in}{2.310000in}}%
\pgfusepath{clip}%
\pgfsetbuttcap%
\pgfsetroundjoin%
\definecolor{currentfill}{rgb}{0.000000,0.000000,0.000000}%
\pgfsetfillcolor{currentfill}%
\pgfsetlinewidth{1.003750pt}%
\definecolor{currentstroke}{rgb}{0.000000,0.000000,0.000000}%
\pgfsetstrokecolor{currentstroke}%
\pgfsetdash{}{0pt}%
\pgfpathmoveto{\pgfqpoint{0.480841in}{0.402751in}}%
\pgfpathcurveto{\pgfqpoint{0.491891in}{0.402751in}}{\pgfqpoint{0.502490in}{0.407141in}}{\pgfqpoint{0.510303in}{0.414955in}}%
\pgfpathcurveto{\pgfqpoint{0.518117in}{0.422768in}}{\pgfqpoint{0.522507in}{0.433367in}}{\pgfqpoint{0.522507in}{0.444417in}}%
\pgfpathcurveto{\pgfqpoint{0.522507in}{0.455468in}}{\pgfqpoint{0.518117in}{0.466067in}}{\pgfqpoint{0.510303in}{0.473880in}}%
\pgfpathcurveto{\pgfqpoint{0.502490in}{0.481694in}}{\pgfqpoint{0.491891in}{0.486084in}}{\pgfqpoint{0.480841in}{0.486084in}}%
\pgfpathcurveto{\pgfqpoint{0.469790in}{0.486084in}}{\pgfqpoint{0.459191in}{0.481694in}}{\pgfqpoint{0.451378in}{0.473880in}}%
\pgfpathcurveto{\pgfqpoint{0.443564in}{0.466067in}}{\pgfqpoint{0.439174in}{0.455468in}}{\pgfqpoint{0.439174in}{0.444417in}}%
\pgfpathcurveto{\pgfqpoint{0.439174in}{0.433367in}}{\pgfqpoint{0.443564in}{0.422768in}}{\pgfqpoint{0.451378in}{0.414955in}}%
\pgfpathcurveto{\pgfqpoint{0.459191in}{0.407141in}}{\pgfqpoint{0.469790in}{0.402751in}}{\pgfqpoint{0.480841in}{0.402751in}}%
\pgfpathclose%
\pgfusepath{stroke,fill}%
\end{pgfscope}%
\begin{pgfscope}%
\pgfpathrectangle{\pgfqpoint{0.375000in}{0.330000in}}{\pgfqpoint{2.325000in}{2.310000in}}%
\pgfusepath{clip}%
\pgfsetbuttcap%
\pgfsetroundjoin%
\definecolor{currentfill}{rgb}{0.000000,0.000000,0.000000}%
\pgfsetfillcolor{currentfill}%
\pgfsetlinewidth{1.003750pt}%
\definecolor{currentstroke}{rgb}{0.000000,0.000000,0.000000}%
\pgfsetstrokecolor{currentstroke}%
\pgfsetdash{}{0pt}%
\pgfpathmoveto{\pgfqpoint{0.480841in}{0.402751in}}%
\pgfpathcurveto{\pgfqpoint{0.491891in}{0.402751in}}{\pgfqpoint{0.502490in}{0.407141in}}{\pgfqpoint{0.510303in}{0.414955in}}%
\pgfpathcurveto{\pgfqpoint{0.518117in}{0.422768in}}{\pgfqpoint{0.522507in}{0.433367in}}{\pgfqpoint{0.522507in}{0.444417in}}%
\pgfpathcurveto{\pgfqpoint{0.522507in}{0.455468in}}{\pgfqpoint{0.518117in}{0.466067in}}{\pgfqpoint{0.510303in}{0.473880in}}%
\pgfpathcurveto{\pgfqpoint{0.502490in}{0.481694in}}{\pgfqpoint{0.491891in}{0.486084in}}{\pgfqpoint{0.480841in}{0.486084in}}%
\pgfpathcurveto{\pgfqpoint{0.469790in}{0.486084in}}{\pgfqpoint{0.459191in}{0.481694in}}{\pgfqpoint{0.451378in}{0.473880in}}%
\pgfpathcurveto{\pgfqpoint{0.443564in}{0.466067in}}{\pgfqpoint{0.439174in}{0.455468in}}{\pgfqpoint{0.439174in}{0.444417in}}%
\pgfpathcurveto{\pgfqpoint{0.439174in}{0.433367in}}{\pgfqpoint{0.443564in}{0.422768in}}{\pgfqpoint{0.451378in}{0.414955in}}%
\pgfpathcurveto{\pgfqpoint{0.459191in}{0.407141in}}{\pgfqpoint{0.469790in}{0.402751in}}{\pgfqpoint{0.480841in}{0.402751in}}%
\pgfpathclose%
\pgfusepath{stroke,fill}%
\end{pgfscope}%
\begin{pgfscope}%
\pgfpathrectangle{\pgfqpoint{0.375000in}{0.330000in}}{\pgfqpoint{2.325000in}{2.310000in}}%
\pgfusepath{clip}%
\pgfsetbuttcap%
\pgfsetroundjoin%
\definecolor{currentfill}{rgb}{0.000000,0.000000,0.000000}%
\pgfsetfillcolor{currentfill}%
\pgfsetlinewidth{1.003750pt}%
\definecolor{currentstroke}{rgb}{0.000000,0.000000,0.000000}%
\pgfsetstrokecolor{currentstroke}%
\pgfsetdash{}{0pt}%
\pgfpathmoveto{\pgfqpoint{0.480841in}{0.402751in}}%
\pgfpathcurveto{\pgfqpoint{0.491891in}{0.402751in}}{\pgfqpoint{0.502490in}{0.407141in}}{\pgfqpoint{0.510303in}{0.414955in}}%
\pgfpathcurveto{\pgfqpoint{0.518117in}{0.422768in}}{\pgfqpoint{0.522507in}{0.433367in}}{\pgfqpoint{0.522507in}{0.444417in}}%
\pgfpathcurveto{\pgfqpoint{0.522507in}{0.455468in}}{\pgfqpoint{0.518117in}{0.466067in}}{\pgfqpoint{0.510303in}{0.473880in}}%
\pgfpathcurveto{\pgfqpoint{0.502490in}{0.481694in}}{\pgfqpoint{0.491891in}{0.486084in}}{\pgfqpoint{0.480841in}{0.486084in}}%
\pgfpathcurveto{\pgfqpoint{0.469790in}{0.486084in}}{\pgfqpoint{0.459191in}{0.481694in}}{\pgfqpoint{0.451378in}{0.473880in}}%
\pgfpathcurveto{\pgfqpoint{0.443564in}{0.466067in}}{\pgfqpoint{0.439174in}{0.455468in}}{\pgfqpoint{0.439174in}{0.444417in}}%
\pgfpathcurveto{\pgfqpoint{0.439174in}{0.433367in}}{\pgfqpoint{0.443564in}{0.422768in}}{\pgfqpoint{0.451378in}{0.414955in}}%
\pgfpathcurveto{\pgfqpoint{0.459191in}{0.407141in}}{\pgfqpoint{0.469790in}{0.402751in}}{\pgfqpoint{0.480841in}{0.402751in}}%
\pgfpathclose%
\pgfusepath{stroke,fill}%
\end{pgfscope}%
\begin{pgfscope}%
\pgfpathrectangle{\pgfqpoint{0.375000in}{0.330000in}}{\pgfqpoint{2.325000in}{2.310000in}}%
\pgfusepath{clip}%
\pgfsetbuttcap%
\pgfsetroundjoin%
\definecolor{currentfill}{rgb}{0.000000,0.000000,0.000000}%
\pgfsetfillcolor{currentfill}%
\pgfsetlinewidth{1.003750pt}%
\definecolor{currentstroke}{rgb}{0.000000,0.000000,0.000000}%
\pgfsetstrokecolor{currentstroke}%
\pgfsetdash{}{0pt}%
\pgfpathmoveto{\pgfqpoint{0.480841in}{0.402751in}}%
\pgfpathcurveto{\pgfqpoint{0.491891in}{0.402751in}}{\pgfqpoint{0.502490in}{0.407141in}}{\pgfqpoint{0.510303in}{0.414955in}}%
\pgfpathcurveto{\pgfqpoint{0.518117in}{0.422768in}}{\pgfqpoint{0.522507in}{0.433367in}}{\pgfqpoint{0.522507in}{0.444417in}}%
\pgfpathcurveto{\pgfqpoint{0.522507in}{0.455468in}}{\pgfqpoint{0.518117in}{0.466067in}}{\pgfqpoint{0.510303in}{0.473880in}}%
\pgfpathcurveto{\pgfqpoint{0.502490in}{0.481694in}}{\pgfqpoint{0.491891in}{0.486084in}}{\pgfqpoint{0.480841in}{0.486084in}}%
\pgfpathcurveto{\pgfqpoint{0.469790in}{0.486084in}}{\pgfqpoint{0.459191in}{0.481694in}}{\pgfqpoint{0.451378in}{0.473880in}}%
\pgfpathcurveto{\pgfqpoint{0.443564in}{0.466067in}}{\pgfqpoint{0.439174in}{0.455468in}}{\pgfqpoint{0.439174in}{0.444417in}}%
\pgfpathcurveto{\pgfqpoint{0.439174in}{0.433367in}}{\pgfqpoint{0.443564in}{0.422768in}}{\pgfqpoint{0.451378in}{0.414955in}}%
\pgfpathcurveto{\pgfqpoint{0.459191in}{0.407141in}}{\pgfqpoint{0.469790in}{0.402751in}}{\pgfqpoint{0.480841in}{0.402751in}}%
\pgfpathclose%
\pgfusepath{stroke,fill}%
\end{pgfscope}%
\begin{pgfscope}%
\pgfpathrectangle{\pgfqpoint{0.375000in}{0.330000in}}{\pgfqpoint{2.325000in}{2.310000in}}%
\pgfusepath{clip}%
\pgfsetbuttcap%
\pgfsetroundjoin%
\definecolor{currentfill}{rgb}{0.000000,0.000000,0.000000}%
\pgfsetfillcolor{currentfill}%
\pgfsetlinewidth{1.003750pt}%
\definecolor{currentstroke}{rgb}{0.000000,0.000000,0.000000}%
\pgfsetstrokecolor{currentstroke}%
\pgfsetdash{}{0pt}%
\pgfpathmoveto{\pgfqpoint{0.480841in}{0.402751in}}%
\pgfpathcurveto{\pgfqpoint{0.491891in}{0.402751in}}{\pgfqpoint{0.502490in}{0.407141in}}{\pgfqpoint{0.510303in}{0.414955in}}%
\pgfpathcurveto{\pgfqpoint{0.518117in}{0.422768in}}{\pgfqpoint{0.522507in}{0.433367in}}{\pgfqpoint{0.522507in}{0.444417in}}%
\pgfpathcurveto{\pgfqpoint{0.522507in}{0.455468in}}{\pgfqpoint{0.518117in}{0.466067in}}{\pgfqpoint{0.510303in}{0.473880in}}%
\pgfpathcurveto{\pgfqpoint{0.502490in}{0.481694in}}{\pgfqpoint{0.491891in}{0.486084in}}{\pgfqpoint{0.480841in}{0.486084in}}%
\pgfpathcurveto{\pgfqpoint{0.469790in}{0.486084in}}{\pgfqpoint{0.459191in}{0.481694in}}{\pgfqpoint{0.451378in}{0.473880in}}%
\pgfpathcurveto{\pgfqpoint{0.443564in}{0.466067in}}{\pgfqpoint{0.439174in}{0.455468in}}{\pgfqpoint{0.439174in}{0.444417in}}%
\pgfpathcurveto{\pgfqpoint{0.439174in}{0.433367in}}{\pgfqpoint{0.443564in}{0.422768in}}{\pgfqpoint{0.451378in}{0.414955in}}%
\pgfpathcurveto{\pgfqpoint{0.459191in}{0.407141in}}{\pgfqpoint{0.469790in}{0.402751in}}{\pgfqpoint{0.480841in}{0.402751in}}%
\pgfpathclose%
\pgfusepath{stroke,fill}%
\end{pgfscope}%
\begin{pgfscope}%
\pgfpathrectangle{\pgfqpoint{0.375000in}{0.330000in}}{\pgfqpoint{2.325000in}{2.310000in}}%
\pgfusepath{clip}%
\pgfsetbuttcap%
\pgfsetroundjoin%
\definecolor{currentfill}{rgb}{0.000000,0.000000,0.000000}%
\pgfsetfillcolor{currentfill}%
\pgfsetlinewidth{1.003750pt}%
\definecolor{currentstroke}{rgb}{0.000000,0.000000,0.000000}%
\pgfsetstrokecolor{currentstroke}%
\pgfsetdash{}{0pt}%
\pgfpathmoveto{\pgfqpoint{0.480841in}{0.402751in}}%
\pgfpathcurveto{\pgfqpoint{0.491891in}{0.402751in}}{\pgfqpoint{0.502490in}{0.407141in}}{\pgfqpoint{0.510303in}{0.414955in}}%
\pgfpathcurveto{\pgfqpoint{0.518117in}{0.422768in}}{\pgfqpoint{0.522507in}{0.433367in}}{\pgfqpoint{0.522507in}{0.444417in}}%
\pgfpathcurveto{\pgfqpoint{0.522507in}{0.455468in}}{\pgfqpoint{0.518117in}{0.466067in}}{\pgfqpoint{0.510303in}{0.473880in}}%
\pgfpathcurveto{\pgfqpoint{0.502490in}{0.481694in}}{\pgfqpoint{0.491891in}{0.486084in}}{\pgfqpoint{0.480841in}{0.486084in}}%
\pgfpathcurveto{\pgfqpoint{0.469790in}{0.486084in}}{\pgfqpoint{0.459191in}{0.481694in}}{\pgfqpoint{0.451378in}{0.473880in}}%
\pgfpathcurveto{\pgfqpoint{0.443564in}{0.466067in}}{\pgfqpoint{0.439174in}{0.455468in}}{\pgfqpoint{0.439174in}{0.444417in}}%
\pgfpathcurveto{\pgfqpoint{0.439174in}{0.433367in}}{\pgfqpoint{0.443564in}{0.422768in}}{\pgfqpoint{0.451378in}{0.414955in}}%
\pgfpathcurveto{\pgfqpoint{0.459191in}{0.407141in}}{\pgfqpoint{0.469790in}{0.402751in}}{\pgfqpoint{0.480841in}{0.402751in}}%
\pgfpathclose%
\pgfusepath{stroke,fill}%
\end{pgfscope}%
\begin{pgfscope}%
\pgfpathrectangle{\pgfqpoint{0.375000in}{0.330000in}}{\pgfqpoint{2.325000in}{2.310000in}}%
\pgfusepath{clip}%
\pgfsetbuttcap%
\pgfsetroundjoin%
\definecolor{currentfill}{rgb}{0.000000,0.000000,0.000000}%
\pgfsetfillcolor{currentfill}%
\pgfsetlinewidth{1.003750pt}%
\definecolor{currentstroke}{rgb}{0.000000,0.000000,0.000000}%
\pgfsetstrokecolor{currentstroke}%
\pgfsetdash{}{0pt}%
\pgfpathmoveto{\pgfqpoint{0.480841in}{0.402751in}}%
\pgfpathcurveto{\pgfqpoint{0.491891in}{0.402751in}}{\pgfqpoint{0.502490in}{0.407141in}}{\pgfqpoint{0.510303in}{0.414955in}}%
\pgfpathcurveto{\pgfqpoint{0.518117in}{0.422768in}}{\pgfqpoint{0.522507in}{0.433367in}}{\pgfqpoint{0.522507in}{0.444417in}}%
\pgfpathcurveto{\pgfqpoint{0.522507in}{0.455468in}}{\pgfqpoint{0.518117in}{0.466067in}}{\pgfqpoint{0.510303in}{0.473880in}}%
\pgfpathcurveto{\pgfqpoint{0.502490in}{0.481694in}}{\pgfqpoint{0.491891in}{0.486084in}}{\pgfqpoint{0.480841in}{0.486084in}}%
\pgfpathcurveto{\pgfqpoint{0.469790in}{0.486084in}}{\pgfqpoint{0.459191in}{0.481694in}}{\pgfqpoint{0.451378in}{0.473880in}}%
\pgfpathcurveto{\pgfqpoint{0.443564in}{0.466067in}}{\pgfqpoint{0.439174in}{0.455468in}}{\pgfqpoint{0.439174in}{0.444417in}}%
\pgfpathcurveto{\pgfqpoint{0.439174in}{0.433367in}}{\pgfqpoint{0.443564in}{0.422768in}}{\pgfqpoint{0.451378in}{0.414955in}}%
\pgfpathcurveto{\pgfqpoint{0.459191in}{0.407141in}}{\pgfqpoint{0.469790in}{0.402751in}}{\pgfqpoint{0.480841in}{0.402751in}}%
\pgfpathclose%
\pgfusepath{stroke,fill}%
\end{pgfscope}%
\begin{pgfscope}%
\pgfpathrectangle{\pgfqpoint{0.375000in}{0.330000in}}{\pgfqpoint{2.325000in}{2.310000in}}%
\pgfusepath{clip}%
\pgfsetbuttcap%
\pgfsetroundjoin%
\definecolor{currentfill}{rgb}{0.000000,0.000000,0.000000}%
\pgfsetfillcolor{currentfill}%
\pgfsetlinewidth{1.003750pt}%
\definecolor{currentstroke}{rgb}{0.000000,0.000000,0.000000}%
\pgfsetstrokecolor{currentstroke}%
\pgfsetdash{}{0pt}%
\pgfpathmoveto{\pgfqpoint{0.480841in}{0.402751in}}%
\pgfpathcurveto{\pgfqpoint{0.491891in}{0.402751in}}{\pgfqpoint{0.502490in}{0.407141in}}{\pgfqpoint{0.510303in}{0.414955in}}%
\pgfpathcurveto{\pgfqpoint{0.518117in}{0.422768in}}{\pgfqpoint{0.522507in}{0.433367in}}{\pgfqpoint{0.522507in}{0.444417in}}%
\pgfpathcurveto{\pgfqpoint{0.522507in}{0.455468in}}{\pgfqpoint{0.518117in}{0.466067in}}{\pgfqpoint{0.510303in}{0.473880in}}%
\pgfpathcurveto{\pgfqpoint{0.502490in}{0.481694in}}{\pgfqpoint{0.491891in}{0.486084in}}{\pgfqpoint{0.480841in}{0.486084in}}%
\pgfpathcurveto{\pgfqpoint{0.469790in}{0.486084in}}{\pgfqpoint{0.459191in}{0.481694in}}{\pgfqpoint{0.451378in}{0.473880in}}%
\pgfpathcurveto{\pgfqpoint{0.443564in}{0.466067in}}{\pgfqpoint{0.439174in}{0.455468in}}{\pgfqpoint{0.439174in}{0.444417in}}%
\pgfpathcurveto{\pgfqpoint{0.439174in}{0.433367in}}{\pgfqpoint{0.443564in}{0.422768in}}{\pgfqpoint{0.451378in}{0.414955in}}%
\pgfpathcurveto{\pgfqpoint{0.459191in}{0.407141in}}{\pgfqpoint{0.469790in}{0.402751in}}{\pgfqpoint{0.480841in}{0.402751in}}%
\pgfpathclose%
\pgfusepath{stroke,fill}%
\end{pgfscope}%
\begin{pgfscope}%
\pgfpathrectangle{\pgfqpoint{0.375000in}{0.330000in}}{\pgfqpoint{2.325000in}{2.310000in}}%
\pgfusepath{clip}%
\pgfsetbuttcap%
\pgfsetroundjoin%
\definecolor{currentfill}{rgb}{0.000000,0.000000,0.000000}%
\pgfsetfillcolor{currentfill}%
\pgfsetlinewidth{1.003750pt}%
\definecolor{currentstroke}{rgb}{0.000000,0.000000,0.000000}%
\pgfsetstrokecolor{currentstroke}%
\pgfsetdash{}{0pt}%
\pgfpathmoveto{\pgfqpoint{0.480841in}{0.402751in}}%
\pgfpathcurveto{\pgfqpoint{0.491891in}{0.402751in}}{\pgfqpoint{0.502490in}{0.407141in}}{\pgfqpoint{0.510303in}{0.414955in}}%
\pgfpathcurveto{\pgfqpoint{0.518117in}{0.422768in}}{\pgfqpoint{0.522507in}{0.433367in}}{\pgfqpoint{0.522507in}{0.444417in}}%
\pgfpathcurveto{\pgfqpoint{0.522507in}{0.455468in}}{\pgfqpoint{0.518117in}{0.466067in}}{\pgfqpoint{0.510303in}{0.473880in}}%
\pgfpathcurveto{\pgfqpoint{0.502490in}{0.481694in}}{\pgfqpoint{0.491891in}{0.486084in}}{\pgfqpoint{0.480841in}{0.486084in}}%
\pgfpathcurveto{\pgfqpoint{0.469790in}{0.486084in}}{\pgfqpoint{0.459191in}{0.481694in}}{\pgfqpoint{0.451378in}{0.473880in}}%
\pgfpathcurveto{\pgfqpoint{0.443564in}{0.466067in}}{\pgfqpoint{0.439174in}{0.455468in}}{\pgfqpoint{0.439174in}{0.444417in}}%
\pgfpathcurveto{\pgfqpoint{0.439174in}{0.433367in}}{\pgfqpoint{0.443564in}{0.422768in}}{\pgfqpoint{0.451378in}{0.414955in}}%
\pgfpathcurveto{\pgfqpoint{0.459191in}{0.407141in}}{\pgfqpoint{0.469790in}{0.402751in}}{\pgfqpoint{0.480841in}{0.402751in}}%
\pgfpathclose%
\pgfusepath{stroke,fill}%
\end{pgfscope}%
\begin{pgfscope}%
\pgfpathrectangle{\pgfqpoint{0.375000in}{0.330000in}}{\pgfqpoint{2.325000in}{2.310000in}}%
\pgfusepath{clip}%
\pgfsetbuttcap%
\pgfsetroundjoin%
\definecolor{currentfill}{rgb}{0.000000,0.000000,0.000000}%
\pgfsetfillcolor{currentfill}%
\pgfsetlinewidth{1.003750pt}%
\definecolor{currentstroke}{rgb}{0.000000,0.000000,0.000000}%
\pgfsetstrokecolor{currentstroke}%
\pgfsetdash{}{0pt}%
\pgfpathmoveto{\pgfqpoint{0.480841in}{0.402751in}}%
\pgfpathcurveto{\pgfqpoint{0.491891in}{0.402751in}}{\pgfqpoint{0.502490in}{0.407141in}}{\pgfqpoint{0.510303in}{0.414955in}}%
\pgfpathcurveto{\pgfqpoint{0.518117in}{0.422768in}}{\pgfqpoint{0.522507in}{0.433367in}}{\pgfqpoint{0.522507in}{0.444417in}}%
\pgfpathcurveto{\pgfqpoint{0.522507in}{0.455468in}}{\pgfqpoint{0.518117in}{0.466067in}}{\pgfqpoint{0.510303in}{0.473880in}}%
\pgfpathcurveto{\pgfqpoint{0.502490in}{0.481694in}}{\pgfqpoint{0.491891in}{0.486084in}}{\pgfqpoint{0.480841in}{0.486084in}}%
\pgfpathcurveto{\pgfqpoint{0.469790in}{0.486084in}}{\pgfqpoint{0.459191in}{0.481694in}}{\pgfqpoint{0.451378in}{0.473880in}}%
\pgfpathcurveto{\pgfqpoint{0.443564in}{0.466067in}}{\pgfqpoint{0.439174in}{0.455468in}}{\pgfqpoint{0.439174in}{0.444417in}}%
\pgfpathcurveto{\pgfqpoint{0.439174in}{0.433367in}}{\pgfqpoint{0.443564in}{0.422768in}}{\pgfqpoint{0.451378in}{0.414955in}}%
\pgfpathcurveto{\pgfqpoint{0.459191in}{0.407141in}}{\pgfqpoint{0.469790in}{0.402751in}}{\pgfqpoint{0.480841in}{0.402751in}}%
\pgfpathclose%
\pgfusepath{stroke,fill}%
\end{pgfscope}%
\begin{pgfscope}%
\pgfpathrectangle{\pgfqpoint{0.375000in}{0.330000in}}{\pgfqpoint{2.325000in}{2.310000in}}%
\pgfusepath{clip}%
\pgfsetbuttcap%
\pgfsetroundjoin%
\definecolor{currentfill}{rgb}{0.000000,0.000000,0.000000}%
\pgfsetfillcolor{currentfill}%
\pgfsetlinewidth{1.003750pt}%
\definecolor{currentstroke}{rgb}{0.000000,0.000000,0.000000}%
\pgfsetstrokecolor{currentstroke}%
\pgfsetdash{}{0pt}%
\pgfpathmoveto{\pgfqpoint{0.480841in}{0.402751in}}%
\pgfpathcurveto{\pgfqpoint{0.491891in}{0.402751in}}{\pgfqpoint{0.502490in}{0.407141in}}{\pgfqpoint{0.510303in}{0.414955in}}%
\pgfpathcurveto{\pgfqpoint{0.518117in}{0.422768in}}{\pgfqpoint{0.522507in}{0.433367in}}{\pgfqpoint{0.522507in}{0.444417in}}%
\pgfpathcurveto{\pgfqpoint{0.522507in}{0.455468in}}{\pgfqpoint{0.518117in}{0.466067in}}{\pgfqpoint{0.510303in}{0.473880in}}%
\pgfpathcurveto{\pgfqpoint{0.502490in}{0.481694in}}{\pgfqpoint{0.491891in}{0.486084in}}{\pgfqpoint{0.480841in}{0.486084in}}%
\pgfpathcurveto{\pgfqpoint{0.469790in}{0.486084in}}{\pgfqpoint{0.459191in}{0.481694in}}{\pgfqpoint{0.451378in}{0.473880in}}%
\pgfpathcurveto{\pgfqpoint{0.443564in}{0.466067in}}{\pgfqpoint{0.439174in}{0.455468in}}{\pgfqpoint{0.439174in}{0.444417in}}%
\pgfpathcurveto{\pgfqpoint{0.439174in}{0.433367in}}{\pgfqpoint{0.443564in}{0.422768in}}{\pgfqpoint{0.451378in}{0.414955in}}%
\pgfpathcurveto{\pgfqpoint{0.459191in}{0.407141in}}{\pgfqpoint{0.469790in}{0.402751in}}{\pgfqpoint{0.480841in}{0.402751in}}%
\pgfpathclose%
\pgfusepath{stroke,fill}%
\end{pgfscope}%
\begin{pgfscope}%
\pgfpathrectangle{\pgfqpoint{0.375000in}{0.330000in}}{\pgfqpoint{2.325000in}{2.310000in}}%
\pgfusepath{clip}%
\pgfsetbuttcap%
\pgfsetroundjoin%
\definecolor{currentfill}{rgb}{0.000000,0.000000,0.000000}%
\pgfsetfillcolor{currentfill}%
\pgfsetlinewidth{1.003750pt}%
\definecolor{currentstroke}{rgb}{0.000000,0.000000,0.000000}%
\pgfsetstrokecolor{currentstroke}%
\pgfsetdash{}{0pt}%
\pgfpathmoveto{\pgfqpoint{0.480841in}{0.402751in}}%
\pgfpathcurveto{\pgfqpoint{0.491891in}{0.402751in}}{\pgfqpoint{0.502490in}{0.407141in}}{\pgfqpoint{0.510303in}{0.414955in}}%
\pgfpathcurveto{\pgfqpoint{0.518117in}{0.422768in}}{\pgfqpoint{0.522507in}{0.433367in}}{\pgfqpoint{0.522507in}{0.444417in}}%
\pgfpathcurveto{\pgfqpoint{0.522507in}{0.455468in}}{\pgfqpoint{0.518117in}{0.466067in}}{\pgfqpoint{0.510303in}{0.473880in}}%
\pgfpathcurveto{\pgfqpoint{0.502490in}{0.481694in}}{\pgfqpoint{0.491891in}{0.486084in}}{\pgfqpoint{0.480841in}{0.486084in}}%
\pgfpathcurveto{\pgfqpoint{0.469790in}{0.486084in}}{\pgfqpoint{0.459191in}{0.481694in}}{\pgfqpoint{0.451378in}{0.473880in}}%
\pgfpathcurveto{\pgfqpoint{0.443564in}{0.466067in}}{\pgfqpoint{0.439174in}{0.455468in}}{\pgfqpoint{0.439174in}{0.444417in}}%
\pgfpathcurveto{\pgfqpoint{0.439174in}{0.433367in}}{\pgfqpoint{0.443564in}{0.422768in}}{\pgfqpoint{0.451378in}{0.414955in}}%
\pgfpathcurveto{\pgfqpoint{0.459191in}{0.407141in}}{\pgfqpoint{0.469790in}{0.402751in}}{\pgfqpoint{0.480841in}{0.402751in}}%
\pgfpathclose%
\pgfusepath{stroke,fill}%
\end{pgfscope}%
\begin{pgfscope}%
\pgfpathrectangle{\pgfqpoint{0.375000in}{0.330000in}}{\pgfqpoint{2.325000in}{2.310000in}}%
\pgfusepath{clip}%
\pgfsetbuttcap%
\pgfsetroundjoin%
\definecolor{currentfill}{rgb}{0.000000,0.000000,0.000000}%
\pgfsetfillcolor{currentfill}%
\pgfsetlinewidth{1.003750pt}%
\definecolor{currentstroke}{rgb}{0.000000,0.000000,0.000000}%
\pgfsetstrokecolor{currentstroke}%
\pgfsetdash{}{0pt}%
\pgfpathmoveto{\pgfqpoint{0.480841in}{0.402751in}}%
\pgfpathcurveto{\pgfqpoint{0.491891in}{0.402751in}}{\pgfqpoint{0.502490in}{0.407141in}}{\pgfqpoint{0.510303in}{0.414955in}}%
\pgfpathcurveto{\pgfqpoint{0.518117in}{0.422768in}}{\pgfqpoint{0.522507in}{0.433367in}}{\pgfqpoint{0.522507in}{0.444417in}}%
\pgfpathcurveto{\pgfqpoint{0.522507in}{0.455468in}}{\pgfqpoint{0.518117in}{0.466067in}}{\pgfqpoint{0.510303in}{0.473880in}}%
\pgfpathcurveto{\pgfqpoint{0.502490in}{0.481694in}}{\pgfqpoint{0.491891in}{0.486084in}}{\pgfqpoint{0.480841in}{0.486084in}}%
\pgfpathcurveto{\pgfqpoint{0.469790in}{0.486084in}}{\pgfqpoint{0.459191in}{0.481694in}}{\pgfqpoint{0.451378in}{0.473880in}}%
\pgfpathcurveto{\pgfqpoint{0.443564in}{0.466067in}}{\pgfqpoint{0.439174in}{0.455468in}}{\pgfqpoint{0.439174in}{0.444417in}}%
\pgfpathcurveto{\pgfqpoint{0.439174in}{0.433367in}}{\pgfqpoint{0.443564in}{0.422768in}}{\pgfqpoint{0.451378in}{0.414955in}}%
\pgfpathcurveto{\pgfqpoint{0.459191in}{0.407141in}}{\pgfqpoint{0.469790in}{0.402751in}}{\pgfqpoint{0.480841in}{0.402751in}}%
\pgfpathclose%
\pgfusepath{stroke,fill}%
\end{pgfscope}%
\begin{pgfscope}%
\pgfpathrectangle{\pgfqpoint{0.375000in}{0.330000in}}{\pgfqpoint{2.325000in}{2.310000in}}%
\pgfusepath{clip}%
\pgfsetbuttcap%
\pgfsetroundjoin%
\definecolor{currentfill}{rgb}{0.000000,0.000000,0.000000}%
\pgfsetfillcolor{currentfill}%
\pgfsetlinewidth{1.003750pt}%
\definecolor{currentstroke}{rgb}{0.000000,0.000000,0.000000}%
\pgfsetstrokecolor{currentstroke}%
\pgfsetdash{}{0pt}%
\pgfpathmoveto{\pgfqpoint{0.480841in}{0.402751in}}%
\pgfpathcurveto{\pgfqpoint{0.491891in}{0.402751in}}{\pgfqpoint{0.502490in}{0.407141in}}{\pgfqpoint{0.510303in}{0.414955in}}%
\pgfpathcurveto{\pgfqpoint{0.518117in}{0.422768in}}{\pgfqpoint{0.522507in}{0.433367in}}{\pgfqpoint{0.522507in}{0.444417in}}%
\pgfpathcurveto{\pgfqpoint{0.522507in}{0.455468in}}{\pgfqpoint{0.518117in}{0.466067in}}{\pgfqpoint{0.510303in}{0.473880in}}%
\pgfpathcurveto{\pgfqpoint{0.502490in}{0.481694in}}{\pgfqpoint{0.491891in}{0.486084in}}{\pgfqpoint{0.480841in}{0.486084in}}%
\pgfpathcurveto{\pgfqpoint{0.469790in}{0.486084in}}{\pgfqpoint{0.459191in}{0.481694in}}{\pgfqpoint{0.451378in}{0.473880in}}%
\pgfpathcurveto{\pgfqpoint{0.443564in}{0.466067in}}{\pgfqpoint{0.439174in}{0.455468in}}{\pgfqpoint{0.439174in}{0.444417in}}%
\pgfpathcurveto{\pgfqpoint{0.439174in}{0.433367in}}{\pgfqpoint{0.443564in}{0.422768in}}{\pgfqpoint{0.451378in}{0.414955in}}%
\pgfpathcurveto{\pgfqpoint{0.459191in}{0.407141in}}{\pgfqpoint{0.469790in}{0.402751in}}{\pgfqpoint{0.480841in}{0.402751in}}%
\pgfpathclose%
\pgfusepath{stroke,fill}%
\end{pgfscope}%
\begin{pgfscope}%
\pgfpathrectangle{\pgfqpoint{0.375000in}{0.330000in}}{\pgfqpoint{2.325000in}{2.310000in}}%
\pgfusepath{clip}%
\pgfsetbuttcap%
\pgfsetroundjoin%
\definecolor{currentfill}{rgb}{0.000000,0.000000,0.000000}%
\pgfsetfillcolor{currentfill}%
\pgfsetlinewidth{1.003750pt}%
\definecolor{currentstroke}{rgb}{0.000000,0.000000,0.000000}%
\pgfsetstrokecolor{currentstroke}%
\pgfsetdash{}{0pt}%
\pgfpathmoveto{\pgfqpoint{0.480841in}{0.402751in}}%
\pgfpathcurveto{\pgfqpoint{0.491891in}{0.402751in}}{\pgfqpoint{0.502490in}{0.407141in}}{\pgfqpoint{0.510303in}{0.414955in}}%
\pgfpathcurveto{\pgfqpoint{0.518117in}{0.422768in}}{\pgfqpoint{0.522507in}{0.433367in}}{\pgfqpoint{0.522507in}{0.444417in}}%
\pgfpathcurveto{\pgfqpoint{0.522507in}{0.455468in}}{\pgfqpoint{0.518117in}{0.466067in}}{\pgfqpoint{0.510303in}{0.473880in}}%
\pgfpathcurveto{\pgfqpoint{0.502490in}{0.481694in}}{\pgfqpoint{0.491891in}{0.486084in}}{\pgfqpoint{0.480841in}{0.486084in}}%
\pgfpathcurveto{\pgfqpoint{0.469790in}{0.486084in}}{\pgfqpoint{0.459191in}{0.481694in}}{\pgfqpoint{0.451378in}{0.473880in}}%
\pgfpathcurveto{\pgfqpoint{0.443564in}{0.466067in}}{\pgfqpoint{0.439174in}{0.455468in}}{\pgfqpoint{0.439174in}{0.444417in}}%
\pgfpathcurveto{\pgfqpoint{0.439174in}{0.433367in}}{\pgfqpoint{0.443564in}{0.422768in}}{\pgfqpoint{0.451378in}{0.414955in}}%
\pgfpathcurveto{\pgfqpoint{0.459191in}{0.407141in}}{\pgfqpoint{0.469790in}{0.402751in}}{\pgfqpoint{0.480841in}{0.402751in}}%
\pgfpathclose%
\pgfusepath{stroke,fill}%
\end{pgfscope}%
\begin{pgfscope}%
\pgfpathrectangle{\pgfqpoint{0.375000in}{0.330000in}}{\pgfqpoint{2.325000in}{2.310000in}}%
\pgfusepath{clip}%
\pgfsetbuttcap%
\pgfsetroundjoin%
\definecolor{currentfill}{rgb}{0.000000,0.000000,0.000000}%
\pgfsetfillcolor{currentfill}%
\pgfsetlinewidth{1.003750pt}%
\definecolor{currentstroke}{rgb}{0.000000,0.000000,0.000000}%
\pgfsetstrokecolor{currentstroke}%
\pgfsetdash{}{0pt}%
\pgfpathmoveto{\pgfqpoint{0.480841in}{0.402751in}}%
\pgfpathcurveto{\pgfqpoint{0.491891in}{0.402751in}}{\pgfqpoint{0.502490in}{0.407141in}}{\pgfqpoint{0.510303in}{0.414955in}}%
\pgfpathcurveto{\pgfqpoint{0.518117in}{0.422768in}}{\pgfqpoint{0.522507in}{0.433367in}}{\pgfqpoint{0.522507in}{0.444417in}}%
\pgfpathcurveto{\pgfqpoint{0.522507in}{0.455468in}}{\pgfqpoint{0.518117in}{0.466067in}}{\pgfqpoint{0.510303in}{0.473880in}}%
\pgfpathcurveto{\pgfqpoint{0.502490in}{0.481694in}}{\pgfqpoint{0.491891in}{0.486084in}}{\pgfqpoint{0.480841in}{0.486084in}}%
\pgfpathcurveto{\pgfqpoint{0.469790in}{0.486084in}}{\pgfqpoint{0.459191in}{0.481694in}}{\pgfqpoint{0.451378in}{0.473880in}}%
\pgfpathcurveto{\pgfqpoint{0.443564in}{0.466067in}}{\pgfqpoint{0.439174in}{0.455468in}}{\pgfqpoint{0.439174in}{0.444417in}}%
\pgfpathcurveto{\pgfqpoint{0.439174in}{0.433367in}}{\pgfqpoint{0.443564in}{0.422768in}}{\pgfqpoint{0.451378in}{0.414955in}}%
\pgfpathcurveto{\pgfqpoint{0.459191in}{0.407141in}}{\pgfqpoint{0.469790in}{0.402751in}}{\pgfqpoint{0.480841in}{0.402751in}}%
\pgfpathclose%
\pgfusepath{stroke,fill}%
\end{pgfscope}%
\begin{pgfscope}%
\pgfpathrectangle{\pgfqpoint{0.375000in}{0.330000in}}{\pgfqpoint{2.325000in}{2.310000in}}%
\pgfusepath{clip}%
\pgfsetbuttcap%
\pgfsetroundjoin%
\definecolor{currentfill}{rgb}{0.000000,0.000000,0.000000}%
\pgfsetfillcolor{currentfill}%
\pgfsetlinewidth{1.003750pt}%
\definecolor{currentstroke}{rgb}{0.000000,0.000000,0.000000}%
\pgfsetstrokecolor{currentstroke}%
\pgfsetdash{}{0pt}%
\pgfpathmoveto{\pgfqpoint{0.480841in}{0.402751in}}%
\pgfpathcurveto{\pgfqpoint{0.491891in}{0.402751in}}{\pgfqpoint{0.502490in}{0.407141in}}{\pgfqpoint{0.510303in}{0.414955in}}%
\pgfpathcurveto{\pgfqpoint{0.518117in}{0.422768in}}{\pgfqpoint{0.522507in}{0.433367in}}{\pgfqpoint{0.522507in}{0.444417in}}%
\pgfpathcurveto{\pgfqpoint{0.522507in}{0.455468in}}{\pgfqpoint{0.518117in}{0.466067in}}{\pgfqpoint{0.510303in}{0.473880in}}%
\pgfpathcurveto{\pgfqpoint{0.502490in}{0.481694in}}{\pgfqpoint{0.491891in}{0.486084in}}{\pgfqpoint{0.480841in}{0.486084in}}%
\pgfpathcurveto{\pgfqpoint{0.469790in}{0.486084in}}{\pgfqpoint{0.459191in}{0.481694in}}{\pgfqpoint{0.451378in}{0.473880in}}%
\pgfpathcurveto{\pgfqpoint{0.443564in}{0.466067in}}{\pgfqpoint{0.439174in}{0.455468in}}{\pgfqpoint{0.439174in}{0.444417in}}%
\pgfpathcurveto{\pgfqpoint{0.439174in}{0.433367in}}{\pgfqpoint{0.443564in}{0.422768in}}{\pgfqpoint{0.451378in}{0.414955in}}%
\pgfpathcurveto{\pgfqpoint{0.459191in}{0.407141in}}{\pgfqpoint{0.469790in}{0.402751in}}{\pgfqpoint{0.480841in}{0.402751in}}%
\pgfpathclose%
\pgfusepath{stroke,fill}%
\end{pgfscope}%
\begin{pgfscope}%
\pgfpathrectangle{\pgfqpoint{0.375000in}{0.330000in}}{\pgfqpoint{2.325000in}{2.310000in}}%
\pgfusepath{clip}%
\pgfsetbuttcap%
\pgfsetroundjoin%
\definecolor{currentfill}{rgb}{0.000000,0.000000,0.000000}%
\pgfsetfillcolor{currentfill}%
\pgfsetlinewidth{1.003750pt}%
\definecolor{currentstroke}{rgb}{0.000000,0.000000,0.000000}%
\pgfsetstrokecolor{currentstroke}%
\pgfsetdash{}{0pt}%
\pgfpathmoveto{\pgfqpoint{0.480841in}{0.402751in}}%
\pgfpathcurveto{\pgfqpoint{0.491891in}{0.402751in}}{\pgfqpoint{0.502490in}{0.407141in}}{\pgfqpoint{0.510303in}{0.414955in}}%
\pgfpathcurveto{\pgfqpoint{0.518117in}{0.422768in}}{\pgfqpoint{0.522507in}{0.433367in}}{\pgfqpoint{0.522507in}{0.444417in}}%
\pgfpathcurveto{\pgfqpoint{0.522507in}{0.455468in}}{\pgfqpoint{0.518117in}{0.466067in}}{\pgfqpoint{0.510303in}{0.473880in}}%
\pgfpathcurveto{\pgfqpoint{0.502490in}{0.481694in}}{\pgfqpoint{0.491891in}{0.486084in}}{\pgfqpoint{0.480841in}{0.486084in}}%
\pgfpathcurveto{\pgfqpoint{0.469790in}{0.486084in}}{\pgfqpoint{0.459191in}{0.481694in}}{\pgfqpoint{0.451378in}{0.473880in}}%
\pgfpathcurveto{\pgfqpoint{0.443564in}{0.466067in}}{\pgfqpoint{0.439174in}{0.455468in}}{\pgfqpoint{0.439174in}{0.444417in}}%
\pgfpathcurveto{\pgfqpoint{0.439174in}{0.433367in}}{\pgfqpoint{0.443564in}{0.422768in}}{\pgfqpoint{0.451378in}{0.414955in}}%
\pgfpathcurveto{\pgfqpoint{0.459191in}{0.407141in}}{\pgfqpoint{0.469790in}{0.402751in}}{\pgfqpoint{0.480841in}{0.402751in}}%
\pgfpathclose%
\pgfusepath{stroke,fill}%
\end{pgfscope}%
\begin{pgfscope}%
\pgfpathrectangle{\pgfqpoint{0.375000in}{0.330000in}}{\pgfqpoint{2.325000in}{2.310000in}}%
\pgfusepath{clip}%
\pgfsetbuttcap%
\pgfsetroundjoin%
\definecolor{currentfill}{rgb}{0.000000,0.000000,0.000000}%
\pgfsetfillcolor{currentfill}%
\pgfsetlinewidth{1.003750pt}%
\definecolor{currentstroke}{rgb}{0.000000,0.000000,0.000000}%
\pgfsetstrokecolor{currentstroke}%
\pgfsetdash{}{0pt}%
\pgfpathmoveto{\pgfqpoint{0.480841in}{0.402751in}}%
\pgfpathcurveto{\pgfqpoint{0.491891in}{0.402751in}}{\pgfqpoint{0.502490in}{0.407141in}}{\pgfqpoint{0.510303in}{0.414955in}}%
\pgfpathcurveto{\pgfqpoint{0.518117in}{0.422768in}}{\pgfqpoint{0.522507in}{0.433367in}}{\pgfqpoint{0.522507in}{0.444417in}}%
\pgfpathcurveto{\pgfqpoint{0.522507in}{0.455468in}}{\pgfqpoint{0.518117in}{0.466067in}}{\pgfqpoint{0.510303in}{0.473880in}}%
\pgfpathcurveto{\pgfqpoint{0.502490in}{0.481694in}}{\pgfqpoint{0.491891in}{0.486084in}}{\pgfqpoint{0.480841in}{0.486084in}}%
\pgfpathcurveto{\pgfqpoint{0.469790in}{0.486084in}}{\pgfqpoint{0.459191in}{0.481694in}}{\pgfqpoint{0.451378in}{0.473880in}}%
\pgfpathcurveto{\pgfqpoint{0.443564in}{0.466067in}}{\pgfqpoint{0.439174in}{0.455468in}}{\pgfqpoint{0.439174in}{0.444417in}}%
\pgfpathcurveto{\pgfqpoint{0.439174in}{0.433367in}}{\pgfqpoint{0.443564in}{0.422768in}}{\pgfqpoint{0.451378in}{0.414955in}}%
\pgfpathcurveto{\pgfqpoint{0.459191in}{0.407141in}}{\pgfqpoint{0.469790in}{0.402751in}}{\pgfqpoint{0.480841in}{0.402751in}}%
\pgfpathclose%
\pgfusepath{stroke,fill}%
\end{pgfscope}%
\begin{pgfscope}%
\pgfpathrectangle{\pgfqpoint{0.375000in}{0.330000in}}{\pgfqpoint{2.325000in}{2.310000in}}%
\pgfusepath{clip}%
\pgfsetbuttcap%
\pgfsetroundjoin%
\definecolor{currentfill}{rgb}{0.000000,0.000000,0.000000}%
\pgfsetfillcolor{currentfill}%
\pgfsetlinewidth{1.003750pt}%
\definecolor{currentstroke}{rgb}{0.000000,0.000000,0.000000}%
\pgfsetstrokecolor{currentstroke}%
\pgfsetdash{}{0pt}%
\pgfpathmoveto{\pgfqpoint{0.480841in}{0.402751in}}%
\pgfpathcurveto{\pgfqpoint{0.491891in}{0.402751in}}{\pgfqpoint{0.502490in}{0.407141in}}{\pgfqpoint{0.510303in}{0.414955in}}%
\pgfpathcurveto{\pgfqpoint{0.518117in}{0.422768in}}{\pgfqpoint{0.522507in}{0.433367in}}{\pgfqpoint{0.522507in}{0.444417in}}%
\pgfpathcurveto{\pgfqpoint{0.522507in}{0.455468in}}{\pgfqpoint{0.518117in}{0.466067in}}{\pgfqpoint{0.510303in}{0.473880in}}%
\pgfpathcurveto{\pgfqpoint{0.502490in}{0.481694in}}{\pgfqpoint{0.491891in}{0.486084in}}{\pgfqpoint{0.480841in}{0.486084in}}%
\pgfpathcurveto{\pgfqpoint{0.469790in}{0.486084in}}{\pgfqpoint{0.459191in}{0.481694in}}{\pgfqpoint{0.451378in}{0.473880in}}%
\pgfpathcurveto{\pgfqpoint{0.443564in}{0.466067in}}{\pgfqpoint{0.439174in}{0.455468in}}{\pgfqpoint{0.439174in}{0.444417in}}%
\pgfpathcurveto{\pgfqpoint{0.439174in}{0.433367in}}{\pgfqpoint{0.443564in}{0.422768in}}{\pgfqpoint{0.451378in}{0.414955in}}%
\pgfpathcurveto{\pgfqpoint{0.459191in}{0.407141in}}{\pgfqpoint{0.469790in}{0.402751in}}{\pgfqpoint{0.480841in}{0.402751in}}%
\pgfpathclose%
\pgfusepath{stroke,fill}%
\end{pgfscope}%
\begin{pgfscope}%
\pgfpathrectangle{\pgfqpoint{0.375000in}{0.330000in}}{\pgfqpoint{2.325000in}{2.310000in}}%
\pgfusepath{clip}%
\pgfsetbuttcap%
\pgfsetroundjoin%
\definecolor{currentfill}{rgb}{0.000000,0.000000,0.000000}%
\pgfsetfillcolor{currentfill}%
\pgfsetlinewidth{1.003750pt}%
\definecolor{currentstroke}{rgb}{0.000000,0.000000,0.000000}%
\pgfsetstrokecolor{currentstroke}%
\pgfsetdash{}{0pt}%
\pgfpathmoveto{\pgfqpoint{0.480841in}{1.438667in}}%
\pgfpathcurveto{\pgfqpoint{0.491891in}{1.438667in}}{\pgfqpoint{0.502490in}{1.443057in}}{\pgfqpoint{0.510303in}{1.450871in}}%
\pgfpathcurveto{\pgfqpoint{0.518117in}{1.458685in}}{\pgfqpoint{0.522507in}{1.469284in}}{\pgfqpoint{0.522507in}{1.480334in}}%
\pgfpathcurveto{\pgfqpoint{0.522507in}{1.491384in}}{\pgfqpoint{0.518117in}{1.501983in}}{\pgfqpoint{0.510303in}{1.509796in}}%
\pgfpathcurveto{\pgfqpoint{0.502490in}{1.517610in}}{\pgfqpoint{0.491891in}{1.522000in}}{\pgfqpoint{0.480841in}{1.522000in}}%
\pgfpathcurveto{\pgfqpoint{0.469790in}{1.522000in}}{\pgfqpoint{0.459191in}{1.517610in}}{\pgfqpoint{0.451378in}{1.509796in}}%
\pgfpathcurveto{\pgfqpoint{0.443564in}{1.501983in}}{\pgfqpoint{0.439174in}{1.491384in}}{\pgfqpoint{0.439174in}{1.480334in}}%
\pgfpathcurveto{\pgfqpoint{0.439174in}{1.469284in}}{\pgfqpoint{0.443564in}{1.458685in}}{\pgfqpoint{0.451378in}{1.450871in}}%
\pgfpathcurveto{\pgfqpoint{0.459191in}{1.443057in}}{\pgfqpoint{0.469790in}{1.438667in}}{\pgfqpoint{0.480841in}{1.438667in}}%
\pgfpathclose%
\pgfusepath{stroke,fill}%
\end{pgfscope}%
\begin{pgfscope}%
\pgfpathrectangle{\pgfqpoint{0.375000in}{0.330000in}}{\pgfqpoint{2.325000in}{2.310000in}}%
\pgfusepath{clip}%
\pgfsetbuttcap%
\pgfsetroundjoin%
\definecolor{currentfill}{rgb}{0.000000,0.000000,0.000000}%
\pgfsetfillcolor{currentfill}%
\pgfsetlinewidth{1.003750pt}%
\definecolor{currentstroke}{rgb}{0.000000,0.000000,0.000000}%
\pgfsetstrokecolor{currentstroke}%
\pgfsetdash{}{0pt}%
\pgfpathmoveto{\pgfqpoint{0.480841in}{0.402751in}}%
\pgfpathcurveto{\pgfqpoint{0.491891in}{0.402751in}}{\pgfqpoint{0.502490in}{0.407141in}}{\pgfqpoint{0.510303in}{0.414955in}}%
\pgfpathcurveto{\pgfqpoint{0.518117in}{0.422768in}}{\pgfqpoint{0.522507in}{0.433367in}}{\pgfqpoint{0.522507in}{0.444417in}}%
\pgfpathcurveto{\pgfqpoint{0.522507in}{0.455468in}}{\pgfqpoint{0.518117in}{0.466067in}}{\pgfqpoint{0.510303in}{0.473880in}}%
\pgfpathcurveto{\pgfqpoint{0.502490in}{0.481694in}}{\pgfqpoint{0.491891in}{0.486084in}}{\pgfqpoint{0.480841in}{0.486084in}}%
\pgfpathcurveto{\pgfqpoint{0.469790in}{0.486084in}}{\pgfqpoint{0.459191in}{0.481694in}}{\pgfqpoint{0.451378in}{0.473880in}}%
\pgfpathcurveto{\pgfqpoint{0.443564in}{0.466067in}}{\pgfqpoint{0.439174in}{0.455468in}}{\pgfqpoint{0.439174in}{0.444417in}}%
\pgfpathcurveto{\pgfqpoint{0.439174in}{0.433367in}}{\pgfqpoint{0.443564in}{0.422768in}}{\pgfqpoint{0.451378in}{0.414955in}}%
\pgfpathcurveto{\pgfqpoint{0.459191in}{0.407141in}}{\pgfqpoint{0.469790in}{0.402751in}}{\pgfqpoint{0.480841in}{0.402751in}}%
\pgfpathclose%
\pgfusepath{stroke,fill}%
\end{pgfscope}%
\begin{pgfscope}%
\pgfpathrectangle{\pgfqpoint{0.375000in}{0.330000in}}{\pgfqpoint{2.325000in}{2.310000in}}%
\pgfusepath{clip}%
\pgfsetbuttcap%
\pgfsetroundjoin%
\definecolor{currentfill}{rgb}{0.000000,0.000000,0.000000}%
\pgfsetfillcolor{currentfill}%
\pgfsetlinewidth{1.003750pt}%
\definecolor{currentstroke}{rgb}{0.000000,0.000000,0.000000}%
\pgfsetstrokecolor{currentstroke}%
\pgfsetdash{}{0pt}%
\pgfpathmoveto{\pgfqpoint{0.480841in}{1.438667in}}%
\pgfpathcurveto{\pgfqpoint{0.491891in}{1.438667in}}{\pgfqpoint{0.502490in}{1.443057in}}{\pgfqpoint{0.510303in}{1.450871in}}%
\pgfpathcurveto{\pgfqpoint{0.518117in}{1.458685in}}{\pgfqpoint{0.522507in}{1.469284in}}{\pgfqpoint{0.522507in}{1.480334in}}%
\pgfpathcurveto{\pgfqpoint{0.522507in}{1.491384in}}{\pgfqpoint{0.518117in}{1.501983in}}{\pgfqpoint{0.510303in}{1.509796in}}%
\pgfpathcurveto{\pgfqpoint{0.502490in}{1.517610in}}{\pgfqpoint{0.491891in}{1.522000in}}{\pgfqpoint{0.480841in}{1.522000in}}%
\pgfpathcurveto{\pgfqpoint{0.469790in}{1.522000in}}{\pgfqpoint{0.459191in}{1.517610in}}{\pgfqpoint{0.451378in}{1.509796in}}%
\pgfpathcurveto{\pgfqpoint{0.443564in}{1.501983in}}{\pgfqpoint{0.439174in}{1.491384in}}{\pgfqpoint{0.439174in}{1.480334in}}%
\pgfpathcurveto{\pgfqpoint{0.439174in}{1.469284in}}{\pgfqpoint{0.443564in}{1.458685in}}{\pgfqpoint{0.451378in}{1.450871in}}%
\pgfpathcurveto{\pgfqpoint{0.459191in}{1.443057in}}{\pgfqpoint{0.469790in}{1.438667in}}{\pgfqpoint{0.480841in}{1.438667in}}%
\pgfpathclose%
\pgfusepath{stroke,fill}%
\end{pgfscope}%
\begin{pgfscope}%
\pgfpathrectangle{\pgfqpoint{0.375000in}{0.330000in}}{\pgfqpoint{2.325000in}{2.310000in}}%
\pgfusepath{clip}%
\pgfsetbuttcap%
\pgfsetroundjoin%
\definecolor{currentfill}{rgb}{0.000000,0.000000,0.000000}%
\pgfsetfillcolor{currentfill}%
\pgfsetlinewidth{1.003750pt}%
\definecolor{currentstroke}{rgb}{0.000000,0.000000,0.000000}%
\pgfsetstrokecolor{currentstroke}%
\pgfsetdash{}{0pt}%
\pgfpathmoveto{\pgfqpoint{0.480841in}{0.402751in}}%
\pgfpathcurveto{\pgfqpoint{0.491891in}{0.402751in}}{\pgfqpoint{0.502490in}{0.407141in}}{\pgfqpoint{0.510303in}{0.414955in}}%
\pgfpathcurveto{\pgfqpoint{0.518117in}{0.422768in}}{\pgfqpoint{0.522507in}{0.433367in}}{\pgfqpoint{0.522507in}{0.444417in}}%
\pgfpathcurveto{\pgfqpoint{0.522507in}{0.455468in}}{\pgfqpoint{0.518117in}{0.466067in}}{\pgfqpoint{0.510303in}{0.473880in}}%
\pgfpathcurveto{\pgfqpoint{0.502490in}{0.481694in}}{\pgfqpoint{0.491891in}{0.486084in}}{\pgfqpoint{0.480841in}{0.486084in}}%
\pgfpathcurveto{\pgfqpoint{0.469790in}{0.486084in}}{\pgfqpoint{0.459191in}{0.481694in}}{\pgfqpoint{0.451378in}{0.473880in}}%
\pgfpathcurveto{\pgfqpoint{0.443564in}{0.466067in}}{\pgfqpoint{0.439174in}{0.455468in}}{\pgfqpoint{0.439174in}{0.444417in}}%
\pgfpathcurveto{\pgfqpoint{0.439174in}{0.433367in}}{\pgfqpoint{0.443564in}{0.422768in}}{\pgfqpoint{0.451378in}{0.414955in}}%
\pgfpathcurveto{\pgfqpoint{0.459191in}{0.407141in}}{\pgfqpoint{0.469790in}{0.402751in}}{\pgfqpoint{0.480841in}{0.402751in}}%
\pgfpathclose%
\pgfusepath{stroke,fill}%
\end{pgfscope}%
\begin{pgfscope}%
\pgfpathrectangle{\pgfqpoint{0.375000in}{0.330000in}}{\pgfqpoint{2.325000in}{2.310000in}}%
\pgfusepath{clip}%
\pgfsetbuttcap%
\pgfsetroundjoin%
\definecolor{currentfill}{rgb}{0.000000,0.000000,0.000000}%
\pgfsetfillcolor{currentfill}%
\pgfsetlinewidth{1.003750pt}%
\definecolor{currentstroke}{rgb}{0.000000,0.000000,0.000000}%
\pgfsetstrokecolor{currentstroke}%
\pgfsetdash{}{0pt}%
\pgfpathmoveto{\pgfqpoint{0.480841in}{0.402751in}}%
\pgfpathcurveto{\pgfqpoint{0.491891in}{0.402751in}}{\pgfqpoint{0.502490in}{0.407141in}}{\pgfqpoint{0.510303in}{0.414955in}}%
\pgfpathcurveto{\pgfqpoint{0.518117in}{0.422768in}}{\pgfqpoint{0.522507in}{0.433367in}}{\pgfqpoint{0.522507in}{0.444417in}}%
\pgfpathcurveto{\pgfqpoint{0.522507in}{0.455468in}}{\pgfqpoint{0.518117in}{0.466067in}}{\pgfqpoint{0.510303in}{0.473880in}}%
\pgfpathcurveto{\pgfqpoint{0.502490in}{0.481694in}}{\pgfqpoint{0.491891in}{0.486084in}}{\pgfqpoint{0.480841in}{0.486084in}}%
\pgfpathcurveto{\pgfqpoint{0.469790in}{0.486084in}}{\pgfqpoint{0.459191in}{0.481694in}}{\pgfqpoint{0.451378in}{0.473880in}}%
\pgfpathcurveto{\pgfqpoint{0.443564in}{0.466067in}}{\pgfqpoint{0.439174in}{0.455468in}}{\pgfqpoint{0.439174in}{0.444417in}}%
\pgfpathcurveto{\pgfqpoint{0.439174in}{0.433367in}}{\pgfqpoint{0.443564in}{0.422768in}}{\pgfqpoint{0.451378in}{0.414955in}}%
\pgfpathcurveto{\pgfqpoint{0.459191in}{0.407141in}}{\pgfqpoint{0.469790in}{0.402751in}}{\pgfqpoint{0.480841in}{0.402751in}}%
\pgfpathclose%
\pgfusepath{stroke,fill}%
\end{pgfscope}%
\begin{pgfscope}%
\pgfpathrectangle{\pgfqpoint{0.375000in}{0.330000in}}{\pgfqpoint{2.325000in}{2.310000in}}%
\pgfusepath{clip}%
\pgfsetbuttcap%
\pgfsetroundjoin%
\definecolor{currentfill}{rgb}{0.000000,0.000000,0.000000}%
\pgfsetfillcolor{currentfill}%
\pgfsetlinewidth{1.003750pt}%
\definecolor{currentstroke}{rgb}{0.000000,0.000000,0.000000}%
\pgfsetstrokecolor{currentstroke}%
\pgfsetdash{}{0pt}%
\pgfpathmoveto{\pgfqpoint{0.480841in}{0.402751in}}%
\pgfpathcurveto{\pgfqpoint{0.491891in}{0.402751in}}{\pgfqpoint{0.502490in}{0.407141in}}{\pgfqpoint{0.510303in}{0.414955in}}%
\pgfpathcurveto{\pgfqpoint{0.518117in}{0.422768in}}{\pgfqpoint{0.522507in}{0.433367in}}{\pgfqpoint{0.522507in}{0.444417in}}%
\pgfpathcurveto{\pgfqpoint{0.522507in}{0.455468in}}{\pgfqpoint{0.518117in}{0.466067in}}{\pgfqpoint{0.510303in}{0.473880in}}%
\pgfpathcurveto{\pgfqpoint{0.502490in}{0.481694in}}{\pgfqpoint{0.491891in}{0.486084in}}{\pgfqpoint{0.480841in}{0.486084in}}%
\pgfpathcurveto{\pgfqpoint{0.469790in}{0.486084in}}{\pgfqpoint{0.459191in}{0.481694in}}{\pgfqpoint{0.451378in}{0.473880in}}%
\pgfpathcurveto{\pgfqpoint{0.443564in}{0.466067in}}{\pgfqpoint{0.439174in}{0.455468in}}{\pgfqpoint{0.439174in}{0.444417in}}%
\pgfpathcurveto{\pgfqpoint{0.439174in}{0.433367in}}{\pgfqpoint{0.443564in}{0.422768in}}{\pgfqpoint{0.451378in}{0.414955in}}%
\pgfpathcurveto{\pgfqpoint{0.459191in}{0.407141in}}{\pgfqpoint{0.469790in}{0.402751in}}{\pgfqpoint{0.480841in}{0.402751in}}%
\pgfpathclose%
\pgfusepath{stroke,fill}%
\end{pgfscope}%
\begin{pgfscope}%
\pgfpathrectangle{\pgfqpoint{0.375000in}{0.330000in}}{\pgfqpoint{2.325000in}{2.310000in}}%
\pgfusepath{clip}%
\pgfsetbuttcap%
\pgfsetroundjoin%
\definecolor{currentfill}{rgb}{0.000000,0.000000,0.000000}%
\pgfsetfillcolor{currentfill}%
\pgfsetlinewidth{1.003750pt}%
\definecolor{currentstroke}{rgb}{0.000000,0.000000,0.000000}%
\pgfsetstrokecolor{currentstroke}%
\pgfsetdash{}{0pt}%
\pgfpathmoveto{\pgfqpoint{0.480841in}{1.438667in}}%
\pgfpathcurveto{\pgfqpoint{0.491891in}{1.438667in}}{\pgfqpoint{0.502490in}{1.443057in}}{\pgfqpoint{0.510303in}{1.450871in}}%
\pgfpathcurveto{\pgfqpoint{0.518117in}{1.458685in}}{\pgfqpoint{0.522507in}{1.469284in}}{\pgfqpoint{0.522507in}{1.480334in}}%
\pgfpathcurveto{\pgfqpoint{0.522507in}{1.491384in}}{\pgfqpoint{0.518117in}{1.501983in}}{\pgfqpoint{0.510303in}{1.509796in}}%
\pgfpathcurveto{\pgfqpoint{0.502490in}{1.517610in}}{\pgfqpoint{0.491891in}{1.522000in}}{\pgfqpoint{0.480841in}{1.522000in}}%
\pgfpathcurveto{\pgfqpoint{0.469790in}{1.522000in}}{\pgfqpoint{0.459191in}{1.517610in}}{\pgfqpoint{0.451378in}{1.509796in}}%
\pgfpathcurveto{\pgfqpoint{0.443564in}{1.501983in}}{\pgfqpoint{0.439174in}{1.491384in}}{\pgfqpoint{0.439174in}{1.480334in}}%
\pgfpathcurveto{\pgfqpoint{0.439174in}{1.469284in}}{\pgfqpoint{0.443564in}{1.458685in}}{\pgfqpoint{0.451378in}{1.450871in}}%
\pgfpathcurveto{\pgfqpoint{0.459191in}{1.443057in}}{\pgfqpoint{0.469790in}{1.438667in}}{\pgfqpoint{0.480841in}{1.438667in}}%
\pgfpathclose%
\pgfusepath{stroke,fill}%
\end{pgfscope}%
\begin{pgfscope}%
\pgfpathrectangle{\pgfqpoint{0.375000in}{0.330000in}}{\pgfqpoint{2.325000in}{2.310000in}}%
\pgfusepath{clip}%
\pgfsetbuttcap%
\pgfsetroundjoin%
\definecolor{currentfill}{rgb}{0.000000,0.000000,0.000000}%
\pgfsetfillcolor{currentfill}%
\pgfsetlinewidth{1.003750pt}%
\definecolor{currentstroke}{rgb}{0.000000,0.000000,0.000000}%
\pgfsetstrokecolor{currentstroke}%
\pgfsetdash{}{0pt}%
\pgfpathmoveto{\pgfqpoint{0.480841in}{0.402751in}}%
\pgfpathcurveto{\pgfqpoint{0.491891in}{0.402751in}}{\pgfqpoint{0.502490in}{0.407141in}}{\pgfqpoint{0.510303in}{0.414955in}}%
\pgfpathcurveto{\pgfqpoint{0.518117in}{0.422768in}}{\pgfqpoint{0.522507in}{0.433367in}}{\pgfqpoint{0.522507in}{0.444417in}}%
\pgfpathcurveto{\pgfqpoint{0.522507in}{0.455468in}}{\pgfqpoint{0.518117in}{0.466067in}}{\pgfqpoint{0.510303in}{0.473880in}}%
\pgfpathcurveto{\pgfqpoint{0.502490in}{0.481694in}}{\pgfqpoint{0.491891in}{0.486084in}}{\pgfqpoint{0.480841in}{0.486084in}}%
\pgfpathcurveto{\pgfqpoint{0.469790in}{0.486084in}}{\pgfqpoint{0.459191in}{0.481694in}}{\pgfqpoint{0.451378in}{0.473880in}}%
\pgfpathcurveto{\pgfqpoint{0.443564in}{0.466067in}}{\pgfqpoint{0.439174in}{0.455468in}}{\pgfqpoint{0.439174in}{0.444417in}}%
\pgfpathcurveto{\pgfqpoint{0.439174in}{0.433367in}}{\pgfqpoint{0.443564in}{0.422768in}}{\pgfqpoint{0.451378in}{0.414955in}}%
\pgfpathcurveto{\pgfqpoint{0.459191in}{0.407141in}}{\pgfqpoint{0.469790in}{0.402751in}}{\pgfqpoint{0.480841in}{0.402751in}}%
\pgfpathclose%
\pgfusepath{stroke,fill}%
\end{pgfscope}%
\begin{pgfscope}%
\pgfpathrectangle{\pgfqpoint{0.375000in}{0.330000in}}{\pgfqpoint{2.325000in}{2.310000in}}%
\pgfusepath{clip}%
\pgfsetbuttcap%
\pgfsetroundjoin%
\definecolor{currentfill}{rgb}{0.000000,0.000000,0.000000}%
\pgfsetfillcolor{currentfill}%
\pgfsetlinewidth{1.003750pt}%
\definecolor{currentstroke}{rgb}{0.000000,0.000000,0.000000}%
\pgfsetstrokecolor{currentstroke}%
\pgfsetdash{}{0pt}%
\pgfpathmoveto{\pgfqpoint{0.480841in}{0.402751in}}%
\pgfpathcurveto{\pgfqpoint{0.491891in}{0.402751in}}{\pgfqpoint{0.502490in}{0.407141in}}{\pgfqpoint{0.510303in}{0.414955in}}%
\pgfpathcurveto{\pgfqpoint{0.518117in}{0.422768in}}{\pgfqpoint{0.522507in}{0.433367in}}{\pgfqpoint{0.522507in}{0.444417in}}%
\pgfpathcurveto{\pgfqpoint{0.522507in}{0.455468in}}{\pgfqpoint{0.518117in}{0.466067in}}{\pgfqpoint{0.510303in}{0.473880in}}%
\pgfpathcurveto{\pgfqpoint{0.502490in}{0.481694in}}{\pgfqpoint{0.491891in}{0.486084in}}{\pgfqpoint{0.480841in}{0.486084in}}%
\pgfpathcurveto{\pgfqpoint{0.469790in}{0.486084in}}{\pgfqpoint{0.459191in}{0.481694in}}{\pgfqpoint{0.451378in}{0.473880in}}%
\pgfpathcurveto{\pgfqpoint{0.443564in}{0.466067in}}{\pgfqpoint{0.439174in}{0.455468in}}{\pgfqpoint{0.439174in}{0.444417in}}%
\pgfpathcurveto{\pgfqpoint{0.439174in}{0.433367in}}{\pgfqpoint{0.443564in}{0.422768in}}{\pgfqpoint{0.451378in}{0.414955in}}%
\pgfpathcurveto{\pgfqpoint{0.459191in}{0.407141in}}{\pgfqpoint{0.469790in}{0.402751in}}{\pgfqpoint{0.480841in}{0.402751in}}%
\pgfpathclose%
\pgfusepath{stroke,fill}%
\end{pgfscope}%
\begin{pgfscope}%
\pgfpathrectangle{\pgfqpoint{0.375000in}{0.330000in}}{\pgfqpoint{2.325000in}{2.310000in}}%
\pgfusepath{clip}%
\pgfsetbuttcap%
\pgfsetroundjoin%
\definecolor{currentfill}{rgb}{0.000000,0.000000,0.000000}%
\pgfsetfillcolor{currentfill}%
\pgfsetlinewidth{1.003750pt}%
\definecolor{currentstroke}{rgb}{0.000000,0.000000,0.000000}%
\pgfsetstrokecolor{currentstroke}%
\pgfsetdash{}{0pt}%
\pgfpathmoveto{\pgfqpoint{0.480841in}{0.402751in}}%
\pgfpathcurveto{\pgfqpoint{0.491891in}{0.402751in}}{\pgfqpoint{0.502490in}{0.407141in}}{\pgfqpoint{0.510303in}{0.414955in}}%
\pgfpathcurveto{\pgfqpoint{0.518117in}{0.422768in}}{\pgfqpoint{0.522507in}{0.433367in}}{\pgfqpoint{0.522507in}{0.444417in}}%
\pgfpathcurveto{\pgfqpoint{0.522507in}{0.455468in}}{\pgfqpoint{0.518117in}{0.466067in}}{\pgfqpoint{0.510303in}{0.473880in}}%
\pgfpathcurveto{\pgfqpoint{0.502490in}{0.481694in}}{\pgfqpoint{0.491891in}{0.486084in}}{\pgfqpoint{0.480841in}{0.486084in}}%
\pgfpathcurveto{\pgfqpoint{0.469790in}{0.486084in}}{\pgfqpoint{0.459191in}{0.481694in}}{\pgfqpoint{0.451378in}{0.473880in}}%
\pgfpathcurveto{\pgfqpoint{0.443564in}{0.466067in}}{\pgfqpoint{0.439174in}{0.455468in}}{\pgfqpoint{0.439174in}{0.444417in}}%
\pgfpathcurveto{\pgfqpoint{0.439174in}{0.433367in}}{\pgfqpoint{0.443564in}{0.422768in}}{\pgfqpoint{0.451378in}{0.414955in}}%
\pgfpathcurveto{\pgfqpoint{0.459191in}{0.407141in}}{\pgfqpoint{0.469790in}{0.402751in}}{\pgfqpoint{0.480841in}{0.402751in}}%
\pgfpathclose%
\pgfusepath{stroke,fill}%
\end{pgfscope}%
\begin{pgfscope}%
\pgfpathrectangle{\pgfqpoint{0.375000in}{0.330000in}}{\pgfqpoint{2.325000in}{2.310000in}}%
\pgfusepath{clip}%
\pgfsetbuttcap%
\pgfsetroundjoin%
\definecolor{currentfill}{rgb}{0.000000,0.000000,0.000000}%
\pgfsetfillcolor{currentfill}%
\pgfsetlinewidth{1.003750pt}%
\definecolor{currentstroke}{rgb}{0.000000,0.000000,0.000000}%
\pgfsetstrokecolor{currentstroke}%
\pgfsetdash{}{0pt}%
\pgfpathmoveto{\pgfqpoint{0.480841in}{0.402751in}}%
\pgfpathcurveto{\pgfqpoint{0.491891in}{0.402751in}}{\pgfqpoint{0.502490in}{0.407141in}}{\pgfqpoint{0.510303in}{0.414955in}}%
\pgfpathcurveto{\pgfqpoint{0.518117in}{0.422768in}}{\pgfqpoint{0.522507in}{0.433367in}}{\pgfqpoint{0.522507in}{0.444417in}}%
\pgfpathcurveto{\pgfqpoint{0.522507in}{0.455468in}}{\pgfqpoint{0.518117in}{0.466067in}}{\pgfqpoint{0.510303in}{0.473880in}}%
\pgfpathcurveto{\pgfqpoint{0.502490in}{0.481694in}}{\pgfqpoint{0.491891in}{0.486084in}}{\pgfqpoint{0.480841in}{0.486084in}}%
\pgfpathcurveto{\pgfqpoint{0.469790in}{0.486084in}}{\pgfqpoint{0.459191in}{0.481694in}}{\pgfqpoint{0.451378in}{0.473880in}}%
\pgfpathcurveto{\pgfqpoint{0.443564in}{0.466067in}}{\pgfqpoint{0.439174in}{0.455468in}}{\pgfqpoint{0.439174in}{0.444417in}}%
\pgfpathcurveto{\pgfqpoint{0.439174in}{0.433367in}}{\pgfqpoint{0.443564in}{0.422768in}}{\pgfqpoint{0.451378in}{0.414955in}}%
\pgfpathcurveto{\pgfqpoint{0.459191in}{0.407141in}}{\pgfqpoint{0.469790in}{0.402751in}}{\pgfqpoint{0.480841in}{0.402751in}}%
\pgfpathclose%
\pgfusepath{stroke,fill}%
\end{pgfscope}%
\begin{pgfscope}%
\pgfpathrectangle{\pgfqpoint{0.375000in}{0.330000in}}{\pgfqpoint{2.325000in}{2.310000in}}%
\pgfusepath{clip}%
\pgfsetbuttcap%
\pgfsetroundjoin%
\definecolor{currentfill}{rgb}{0.000000,0.000000,0.000000}%
\pgfsetfillcolor{currentfill}%
\pgfsetlinewidth{1.003750pt}%
\definecolor{currentstroke}{rgb}{0.000000,0.000000,0.000000}%
\pgfsetstrokecolor{currentstroke}%
\pgfsetdash{}{0pt}%
\pgfpathmoveto{\pgfqpoint{0.480841in}{0.402751in}}%
\pgfpathcurveto{\pgfqpoint{0.491891in}{0.402751in}}{\pgfqpoint{0.502490in}{0.407141in}}{\pgfqpoint{0.510303in}{0.414955in}}%
\pgfpathcurveto{\pgfqpoint{0.518117in}{0.422768in}}{\pgfqpoint{0.522507in}{0.433367in}}{\pgfqpoint{0.522507in}{0.444417in}}%
\pgfpathcurveto{\pgfqpoint{0.522507in}{0.455468in}}{\pgfqpoint{0.518117in}{0.466067in}}{\pgfqpoint{0.510303in}{0.473880in}}%
\pgfpathcurveto{\pgfqpoint{0.502490in}{0.481694in}}{\pgfqpoint{0.491891in}{0.486084in}}{\pgfqpoint{0.480841in}{0.486084in}}%
\pgfpathcurveto{\pgfqpoint{0.469790in}{0.486084in}}{\pgfqpoint{0.459191in}{0.481694in}}{\pgfqpoint{0.451378in}{0.473880in}}%
\pgfpathcurveto{\pgfqpoint{0.443564in}{0.466067in}}{\pgfqpoint{0.439174in}{0.455468in}}{\pgfqpoint{0.439174in}{0.444417in}}%
\pgfpathcurveto{\pgfqpoint{0.439174in}{0.433367in}}{\pgfqpoint{0.443564in}{0.422768in}}{\pgfqpoint{0.451378in}{0.414955in}}%
\pgfpathcurveto{\pgfqpoint{0.459191in}{0.407141in}}{\pgfqpoint{0.469790in}{0.402751in}}{\pgfqpoint{0.480841in}{0.402751in}}%
\pgfpathclose%
\pgfusepath{stroke,fill}%
\end{pgfscope}%
\begin{pgfscope}%
\pgfpathrectangle{\pgfqpoint{0.375000in}{0.330000in}}{\pgfqpoint{2.325000in}{2.310000in}}%
\pgfusepath{clip}%
\pgfsetbuttcap%
\pgfsetroundjoin%
\definecolor{currentfill}{rgb}{0.000000,0.000000,0.000000}%
\pgfsetfillcolor{currentfill}%
\pgfsetlinewidth{1.003750pt}%
\definecolor{currentstroke}{rgb}{0.000000,0.000000,0.000000}%
\pgfsetstrokecolor{currentstroke}%
\pgfsetdash{}{0pt}%
\pgfpathmoveto{\pgfqpoint{0.480841in}{0.402751in}}%
\pgfpathcurveto{\pgfqpoint{0.491891in}{0.402751in}}{\pgfqpoint{0.502490in}{0.407141in}}{\pgfqpoint{0.510303in}{0.414955in}}%
\pgfpathcurveto{\pgfqpoint{0.518117in}{0.422768in}}{\pgfqpoint{0.522507in}{0.433367in}}{\pgfqpoint{0.522507in}{0.444417in}}%
\pgfpathcurveto{\pgfqpoint{0.522507in}{0.455468in}}{\pgfqpoint{0.518117in}{0.466067in}}{\pgfqpoint{0.510303in}{0.473880in}}%
\pgfpathcurveto{\pgfqpoint{0.502490in}{0.481694in}}{\pgfqpoint{0.491891in}{0.486084in}}{\pgfqpoint{0.480841in}{0.486084in}}%
\pgfpathcurveto{\pgfqpoint{0.469790in}{0.486084in}}{\pgfqpoint{0.459191in}{0.481694in}}{\pgfqpoint{0.451378in}{0.473880in}}%
\pgfpathcurveto{\pgfqpoint{0.443564in}{0.466067in}}{\pgfqpoint{0.439174in}{0.455468in}}{\pgfqpoint{0.439174in}{0.444417in}}%
\pgfpathcurveto{\pgfqpoint{0.439174in}{0.433367in}}{\pgfqpoint{0.443564in}{0.422768in}}{\pgfqpoint{0.451378in}{0.414955in}}%
\pgfpathcurveto{\pgfqpoint{0.459191in}{0.407141in}}{\pgfqpoint{0.469790in}{0.402751in}}{\pgfqpoint{0.480841in}{0.402751in}}%
\pgfpathclose%
\pgfusepath{stroke,fill}%
\end{pgfscope}%
\begin{pgfscope}%
\pgfpathrectangle{\pgfqpoint{0.375000in}{0.330000in}}{\pgfqpoint{2.325000in}{2.310000in}}%
\pgfusepath{clip}%
\pgfsetbuttcap%
\pgfsetroundjoin%
\definecolor{currentfill}{rgb}{0.000000,0.000000,0.000000}%
\pgfsetfillcolor{currentfill}%
\pgfsetlinewidth{1.003750pt}%
\definecolor{currentstroke}{rgb}{0.000000,0.000000,0.000000}%
\pgfsetstrokecolor{currentstroke}%
\pgfsetdash{}{0pt}%
\pgfpathmoveto{\pgfqpoint{0.480841in}{0.402751in}}%
\pgfpathcurveto{\pgfqpoint{0.491891in}{0.402751in}}{\pgfqpoint{0.502490in}{0.407141in}}{\pgfqpoint{0.510303in}{0.414955in}}%
\pgfpathcurveto{\pgfqpoint{0.518117in}{0.422768in}}{\pgfqpoint{0.522507in}{0.433367in}}{\pgfqpoint{0.522507in}{0.444417in}}%
\pgfpathcurveto{\pgfqpoint{0.522507in}{0.455468in}}{\pgfqpoint{0.518117in}{0.466067in}}{\pgfqpoint{0.510303in}{0.473880in}}%
\pgfpathcurveto{\pgfqpoint{0.502490in}{0.481694in}}{\pgfqpoint{0.491891in}{0.486084in}}{\pgfqpoint{0.480841in}{0.486084in}}%
\pgfpathcurveto{\pgfqpoint{0.469790in}{0.486084in}}{\pgfqpoint{0.459191in}{0.481694in}}{\pgfqpoint{0.451378in}{0.473880in}}%
\pgfpathcurveto{\pgfqpoint{0.443564in}{0.466067in}}{\pgfqpoint{0.439174in}{0.455468in}}{\pgfqpoint{0.439174in}{0.444417in}}%
\pgfpathcurveto{\pgfqpoint{0.439174in}{0.433367in}}{\pgfqpoint{0.443564in}{0.422768in}}{\pgfqpoint{0.451378in}{0.414955in}}%
\pgfpathcurveto{\pgfqpoint{0.459191in}{0.407141in}}{\pgfqpoint{0.469790in}{0.402751in}}{\pgfqpoint{0.480841in}{0.402751in}}%
\pgfpathclose%
\pgfusepath{stroke,fill}%
\end{pgfscope}%
\begin{pgfscope}%
\pgfpathrectangle{\pgfqpoint{0.375000in}{0.330000in}}{\pgfqpoint{2.325000in}{2.310000in}}%
\pgfusepath{clip}%
\pgfsetbuttcap%
\pgfsetroundjoin%
\definecolor{currentfill}{rgb}{0.000000,0.000000,0.000000}%
\pgfsetfillcolor{currentfill}%
\pgfsetlinewidth{1.003750pt}%
\definecolor{currentstroke}{rgb}{0.000000,0.000000,0.000000}%
\pgfsetstrokecolor{currentstroke}%
\pgfsetdash{}{0pt}%
\pgfpathmoveto{\pgfqpoint{0.480841in}{0.402751in}}%
\pgfpathcurveto{\pgfqpoint{0.491891in}{0.402751in}}{\pgfqpoint{0.502490in}{0.407141in}}{\pgfqpoint{0.510303in}{0.414955in}}%
\pgfpathcurveto{\pgfqpoint{0.518117in}{0.422768in}}{\pgfqpoint{0.522507in}{0.433367in}}{\pgfqpoint{0.522507in}{0.444417in}}%
\pgfpathcurveto{\pgfqpoint{0.522507in}{0.455468in}}{\pgfqpoint{0.518117in}{0.466067in}}{\pgfqpoint{0.510303in}{0.473880in}}%
\pgfpathcurveto{\pgfqpoint{0.502490in}{0.481694in}}{\pgfqpoint{0.491891in}{0.486084in}}{\pgfqpoint{0.480841in}{0.486084in}}%
\pgfpathcurveto{\pgfqpoint{0.469790in}{0.486084in}}{\pgfqpoint{0.459191in}{0.481694in}}{\pgfqpoint{0.451378in}{0.473880in}}%
\pgfpathcurveto{\pgfqpoint{0.443564in}{0.466067in}}{\pgfqpoint{0.439174in}{0.455468in}}{\pgfqpoint{0.439174in}{0.444417in}}%
\pgfpathcurveto{\pgfqpoint{0.439174in}{0.433367in}}{\pgfqpoint{0.443564in}{0.422768in}}{\pgfqpoint{0.451378in}{0.414955in}}%
\pgfpathcurveto{\pgfqpoint{0.459191in}{0.407141in}}{\pgfqpoint{0.469790in}{0.402751in}}{\pgfqpoint{0.480841in}{0.402751in}}%
\pgfpathclose%
\pgfusepath{stroke,fill}%
\end{pgfscope}%
\begin{pgfscope}%
\pgfpathrectangle{\pgfqpoint{0.375000in}{0.330000in}}{\pgfqpoint{2.325000in}{2.310000in}}%
\pgfusepath{clip}%
\pgfsetbuttcap%
\pgfsetroundjoin%
\definecolor{currentfill}{rgb}{0.000000,0.000000,0.000000}%
\pgfsetfillcolor{currentfill}%
\pgfsetlinewidth{1.003750pt}%
\definecolor{currentstroke}{rgb}{0.000000,0.000000,0.000000}%
\pgfsetstrokecolor{currentstroke}%
\pgfsetdash{}{0pt}%
\pgfpathmoveto{\pgfqpoint{0.480841in}{0.402751in}}%
\pgfpathcurveto{\pgfqpoint{0.491891in}{0.402751in}}{\pgfqpoint{0.502490in}{0.407141in}}{\pgfqpoint{0.510303in}{0.414955in}}%
\pgfpathcurveto{\pgfqpoint{0.518117in}{0.422768in}}{\pgfqpoint{0.522507in}{0.433367in}}{\pgfqpoint{0.522507in}{0.444417in}}%
\pgfpathcurveto{\pgfqpoint{0.522507in}{0.455468in}}{\pgfqpoint{0.518117in}{0.466067in}}{\pgfqpoint{0.510303in}{0.473880in}}%
\pgfpathcurveto{\pgfqpoint{0.502490in}{0.481694in}}{\pgfqpoint{0.491891in}{0.486084in}}{\pgfqpoint{0.480841in}{0.486084in}}%
\pgfpathcurveto{\pgfqpoint{0.469790in}{0.486084in}}{\pgfqpoint{0.459191in}{0.481694in}}{\pgfqpoint{0.451378in}{0.473880in}}%
\pgfpathcurveto{\pgfqpoint{0.443564in}{0.466067in}}{\pgfqpoint{0.439174in}{0.455468in}}{\pgfqpoint{0.439174in}{0.444417in}}%
\pgfpathcurveto{\pgfqpoint{0.439174in}{0.433367in}}{\pgfqpoint{0.443564in}{0.422768in}}{\pgfqpoint{0.451378in}{0.414955in}}%
\pgfpathcurveto{\pgfqpoint{0.459191in}{0.407141in}}{\pgfqpoint{0.469790in}{0.402751in}}{\pgfqpoint{0.480841in}{0.402751in}}%
\pgfpathclose%
\pgfusepath{stroke,fill}%
\end{pgfscope}%
\begin{pgfscope}%
\pgfpathrectangle{\pgfqpoint{0.375000in}{0.330000in}}{\pgfqpoint{2.325000in}{2.310000in}}%
\pgfusepath{clip}%
\pgfsetbuttcap%
\pgfsetroundjoin%
\definecolor{currentfill}{rgb}{0.000000,0.000000,0.000000}%
\pgfsetfillcolor{currentfill}%
\pgfsetlinewidth{1.003750pt}%
\definecolor{currentstroke}{rgb}{0.000000,0.000000,0.000000}%
\pgfsetstrokecolor{currentstroke}%
\pgfsetdash{}{0pt}%
\pgfpathmoveto{\pgfqpoint{0.480841in}{0.402751in}}%
\pgfpathcurveto{\pgfqpoint{0.491891in}{0.402751in}}{\pgfqpoint{0.502490in}{0.407141in}}{\pgfqpoint{0.510303in}{0.414955in}}%
\pgfpathcurveto{\pgfqpoint{0.518117in}{0.422768in}}{\pgfqpoint{0.522507in}{0.433367in}}{\pgfqpoint{0.522507in}{0.444417in}}%
\pgfpathcurveto{\pgfqpoint{0.522507in}{0.455468in}}{\pgfqpoint{0.518117in}{0.466067in}}{\pgfqpoint{0.510303in}{0.473880in}}%
\pgfpathcurveto{\pgfqpoint{0.502490in}{0.481694in}}{\pgfqpoint{0.491891in}{0.486084in}}{\pgfqpoint{0.480841in}{0.486084in}}%
\pgfpathcurveto{\pgfqpoint{0.469790in}{0.486084in}}{\pgfqpoint{0.459191in}{0.481694in}}{\pgfqpoint{0.451378in}{0.473880in}}%
\pgfpathcurveto{\pgfqpoint{0.443564in}{0.466067in}}{\pgfqpoint{0.439174in}{0.455468in}}{\pgfqpoint{0.439174in}{0.444417in}}%
\pgfpathcurveto{\pgfqpoint{0.439174in}{0.433367in}}{\pgfqpoint{0.443564in}{0.422768in}}{\pgfqpoint{0.451378in}{0.414955in}}%
\pgfpathcurveto{\pgfqpoint{0.459191in}{0.407141in}}{\pgfqpoint{0.469790in}{0.402751in}}{\pgfqpoint{0.480841in}{0.402751in}}%
\pgfpathclose%
\pgfusepath{stroke,fill}%
\end{pgfscope}%
\begin{pgfscope}%
\pgfpathrectangle{\pgfqpoint{0.375000in}{0.330000in}}{\pgfqpoint{2.325000in}{2.310000in}}%
\pgfusepath{clip}%
\pgfsetbuttcap%
\pgfsetroundjoin%
\definecolor{currentfill}{rgb}{0.000000,0.000000,0.000000}%
\pgfsetfillcolor{currentfill}%
\pgfsetlinewidth{1.003750pt}%
\definecolor{currentstroke}{rgb}{0.000000,0.000000,0.000000}%
\pgfsetstrokecolor{currentstroke}%
\pgfsetdash{}{0pt}%
\pgfpathmoveto{\pgfqpoint{0.480841in}{0.402751in}}%
\pgfpathcurveto{\pgfqpoint{0.491891in}{0.402751in}}{\pgfqpoint{0.502490in}{0.407141in}}{\pgfqpoint{0.510303in}{0.414955in}}%
\pgfpathcurveto{\pgfqpoint{0.518117in}{0.422768in}}{\pgfqpoint{0.522507in}{0.433367in}}{\pgfqpoint{0.522507in}{0.444417in}}%
\pgfpathcurveto{\pgfqpoint{0.522507in}{0.455468in}}{\pgfqpoint{0.518117in}{0.466067in}}{\pgfqpoint{0.510303in}{0.473880in}}%
\pgfpathcurveto{\pgfqpoint{0.502490in}{0.481694in}}{\pgfqpoint{0.491891in}{0.486084in}}{\pgfqpoint{0.480841in}{0.486084in}}%
\pgfpathcurveto{\pgfqpoint{0.469790in}{0.486084in}}{\pgfqpoint{0.459191in}{0.481694in}}{\pgfqpoint{0.451378in}{0.473880in}}%
\pgfpathcurveto{\pgfqpoint{0.443564in}{0.466067in}}{\pgfqpoint{0.439174in}{0.455468in}}{\pgfqpoint{0.439174in}{0.444417in}}%
\pgfpathcurveto{\pgfqpoint{0.439174in}{0.433367in}}{\pgfqpoint{0.443564in}{0.422768in}}{\pgfqpoint{0.451378in}{0.414955in}}%
\pgfpathcurveto{\pgfqpoint{0.459191in}{0.407141in}}{\pgfqpoint{0.469790in}{0.402751in}}{\pgfqpoint{0.480841in}{0.402751in}}%
\pgfpathclose%
\pgfusepath{stroke,fill}%
\end{pgfscope}%
\begin{pgfscope}%
\pgfpathrectangle{\pgfqpoint{0.375000in}{0.330000in}}{\pgfqpoint{2.325000in}{2.310000in}}%
\pgfusepath{clip}%
\pgfsetbuttcap%
\pgfsetroundjoin%
\definecolor{currentfill}{rgb}{0.000000,0.000000,0.000000}%
\pgfsetfillcolor{currentfill}%
\pgfsetlinewidth{1.003750pt}%
\definecolor{currentstroke}{rgb}{0.000000,0.000000,0.000000}%
\pgfsetstrokecolor{currentstroke}%
\pgfsetdash{}{0pt}%
\pgfpathmoveto{\pgfqpoint{0.480841in}{0.402751in}}%
\pgfpathcurveto{\pgfqpoint{0.491891in}{0.402751in}}{\pgfqpoint{0.502490in}{0.407141in}}{\pgfqpoint{0.510303in}{0.414955in}}%
\pgfpathcurveto{\pgfqpoint{0.518117in}{0.422768in}}{\pgfqpoint{0.522507in}{0.433367in}}{\pgfqpoint{0.522507in}{0.444417in}}%
\pgfpathcurveto{\pgfqpoint{0.522507in}{0.455468in}}{\pgfqpoint{0.518117in}{0.466067in}}{\pgfqpoint{0.510303in}{0.473880in}}%
\pgfpathcurveto{\pgfqpoint{0.502490in}{0.481694in}}{\pgfqpoint{0.491891in}{0.486084in}}{\pgfqpoint{0.480841in}{0.486084in}}%
\pgfpathcurveto{\pgfqpoint{0.469790in}{0.486084in}}{\pgfqpoint{0.459191in}{0.481694in}}{\pgfqpoint{0.451378in}{0.473880in}}%
\pgfpathcurveto{\pgfqpoint{0.443564in}{0.466067in}}{\pgfqpoint{0.439174in}{0.455468in}}{\pgfqpoint{0.439174in}{0.444417in}}%
\pgfpathcurveto{\pgfqpoint{0.439174in}{0.433367in}}{\pgfqpoint{0.443564in}{0.422768in}}{\pgfqpoint{0.451378in}{0.414955in}}%
\pgfpathcurveto{\pgfqpoint{0.459191in}{0.407141in}}{\pgfqpoint{0.469790in}{0.402751in}}{\pgfqpoint{0.480841in}{0.402751in}}%
\pgfpathclose%
\pgfusepath{stroke,fill}%
\end{pgfscope}%
\begin{pgfscope}%
\pgfpathrectangle{\pgfqpoint{0.375000in}{0.330000in}}{\pgfqpoint{2.325000in}{2.310000in}}%
\pgfusepath{clip}%
\pgfsetbuttcap%
\pgfsetroundjoin%
\definecolor{currentfill}{rgb}{0.000000,0.000000,0.000000}%
\pgfsetfillcolor{currentfill}%
\pgfsetlinewidth{1.003750pt}%
\definecolor{currentstroke}{rgb}{0.000000,0.000000,0.000000}%
\pgfsetstrokecolor{currentstroke}%
\pgfsetdash{}{0pt}%
\pgfpathmoveto{\pgfqpoint{0.480841in}{0.402751in}}%
\pgfpathcurveto{\pgfqpoint{0.491891in}{0.402751in}}{\pgfqpoint{0.502490in}{0.407141in}}{\pgfqpoint{0.510303in}{0.414955in}}%
\pgfpathcurveto{\pgfqpoint{0.518117in}{0.422768in}}{\pgfqpoint{0.522507in}{0.433367in}}{\pgfqpoint{0.522507in}{0.444417in}}%
\pgfpathcurveto{\pgfqpoint{0.522507in}{0.455468in}}{\pgfqpoint{0.518117in}{0.466067in}}{\pgfqpoint{0.510303in}{0.473880in}}%
\pgfpathcurveto{\pgfqpoint{0.502490in}{0.481694in}}{\pgfqpoint{0.491891in}{0.486084in}}{\pgfqpoint{0.480841in}{0.486084in}}%
\pgfpathcurveto{\pgfqpoint{0.469790in}{0.486084in}}{\pgfqpoint{0.459191in}{0.481694in}}{\pgfqpoint{0.451378in}{0.473880in}}%
\pgfpathcurveto{\pgfqpoint{0.443564in}{0.466067in}}{\pgfqpoint{0.439174in}{0.455468in}}{\pgfqpoint{0.439174in}{0.444417in}}%
\pgfpathcurveto{\pgfqpoint{0.439174in}{0.433367in}}{\pgfqpoint{0.443564in}{0.422768in}}{\pgfqpoint{0.451378in}{0.414955in}}%
\pgfpathcurveto{\pgfqpoint{0.459191in}{0.407141in}}{\pgfqpoint{0.469790in}{0.402751in}}{\pgfqpoint{0.480841in}{0.402751in}}%
\pgfpathclose%
\pgfusepath{stroke,fill}%
\end{pgfscope}%
\begin{pgfscope}%
\pgfpathrectangle{\pgfqpoint{0.375000in}{0.330000in}}{\pgfqpoint{2.325000in}{2.310000in}}%
\pgfusepath{clip}%
\pgfsetbuttcap%
\pgfsetroundjoin%
\definecolor{currentfill}{rgb}{0.000000,0.000000,0.000000}%
\pgfsetfillcolor{currentfill}%
\pgfsetlinewidth{1.003750pt}%
\definecolor{currentstroke}{rgb}{0.000000,0.000000,0.000000}%
\pgfsetstrokecolor{currentstroke}%
\pgfsetdash{}{0pt}%
\pgfpathmoveto{\pgfqpoint{0.480841in}{0.402751in}}%
\pgfpathcurveto{\pgfqpoint{0.491891in}{0.402751in}}{\pgfqpoint{0.502490in}{0.407141in}}{\pgfqpoint{0.510303in}{0.414955in}}%
\pgfpathcurveto{\pgfqpoint{0.518117in}{0.422768in}}{\pgfqpoint{0.522507in}{0.433367in}}{\pgfqpoint{0.522507in}{0.444417in}}%
\pgfpathcurveto{\pgfqpoint{0.522507in}{0.455468in}}{\pgfqpoint{0.518117in}{0.466067in}}{\pgfqpoint{0.510303in}{0.473880in}}%
\pgfpathcurveto{\pgfqpoint{0.502490in}{0.481694in}}{\pgfqpoint{0.491891in}{0.486084in}}{\pgfqpoint{0.480841in}{0.486084in}}%
\pgfpathcurveto{\pgfqpoint{0.469790in}{0.486084in}}{\pgfqpoint{0.459191in}{0.481694in}}{\pgfqpoint{0.451378in}{0.473880in}}%
\pgfpathcurveto{\pgfqpoint{0.443564in}{0.466067in}}{\pgfqpoint{0.439174in}{0.455468in}}{\pgfqpoint{0.439174in}{0.444417in}}%
\pgfpathcurveto{\pgfqpoint{0.439174in}{0.433367in}}{\pgfqpoint{0.443564in}{0.422768in}}{\pgfqpoint{0.451378in}{0.414955in}}%
\pgfpathcurveto{\pgfqpoint{0.459191in}{0.407141in}}{\pgfqpoint{0.469790in}{0.402751in}}{\pgfqpoint{0.480841in}{0.402751in}}%
\pgfpathclose%
\pgfusepath{stroke,fill}%
\end{pgfscope}%
\begin{pgfscope}%
\pgfpathrectangle{\pgfqpoint{0.375000in}{0.330000in}}{\pgfqpoint{2.325000in}{2.310000in}}%
\pgfusepath{clip}%
\pgfsetbuttcap%
\pgfsetroundjoin%
\definecolor{currentfill}{rgb}{0.000000,0.000000,0.000000}%
\pgfsetfillcolor{currentfill}%
\pgfsetlinewidth{1.003750pt}%
\definecolor{currentstroke}{rgb}{0.000000,0.000000,0.000000}%
\pgfsetstrokecolor{currentstroke}%
\pgfsetdash{}{0pt}%
\pgfpathmoveto{\pgfqpoint{0.480841in}{0.402751in}}%
\pgfpathcurveto{\pgfqpoint{0.491891in}{0.402751in}}{\pgfqpoint{0.502490in}{0.407141in}}{\pgfqpoint{0.510303in}{0.414955in}}%
\pgfpathcurveto{\pgfqpoint{0.518117in}{0.422768in}}{\pgfqpoint{0.522507in}{0.433367in}}{\pgfqpoint{0.522507in}{0.444417in}}%
\pgfpathcurveto{\pgfqpoint{0.522507in}{0.455468in}}{\pgfqpoint{0.518117in}{0.466067in}}{\pgfqpoint{0.510303in}{0.473880in}}%
\pgfpathcurveto{\pgfqpoint{0.502490in}{0.481694in}}{\pgfqpoint{0.491891in}{0.486084in}}{\pgfqpoint{0.480841in}{0.486084in}}%
\pgfpathcurveto{\pgfqpoint{0.469790in}{0.486084in}}{\pgfqpoint{0.459191in}{0.481694in}}{\pgfqpoint{0.451378in}{0.473880in}}%
\pgfpathcurveto{\pgfqpoint{0.443564in}{0.466067in}}{\pgfqpoint{0.439174in}{0.455468in}}{\pgfqpoint{0.439174in}{0.444417in}}%
\pgfpathcurveto{\pgfqpoint{0.439174in}{0.433367in}}{\pgfqpoint{0.443564in}{0.422768in}}{\pgfqpoint{0.451378in}{0.414955in}}%
\pgfpathcurveto{\pgfqpoint{0.459191in}{0.407141in}}{\pgfqpoint{0.469790in}{0.402751in}}{\pgfqpoint{0.480841in}{0.402751in}}%
\pgfpathclose%
\pgfusepath{stroke,fill}%
\end{pgfscope}%
\begin{pgfscope}%
\pgfpathrectangle{\pgfqpoint{0.375000in}{0.330000in}}{\pgfqpoint{2.325000in}{2.310000in}}%
\pgfusepath{clip}%
\pgfsetbuttcap%
\pgfsetroundjoin%
\definecolor{currentfill}{rgb}{0.000000,0.000000,0.000000}%
\pgfsetfillcolor{currentfill}%
\pgfsetlinewidth{1.003750pt}%
\definecolor{currentstroke}{rgb}{0.000000,0.000000,0.000000}%
\pgfsetstrokecolor{currentstroke}%
\pgfsetdash{}{0pt}%
\pgfpathmoveto{\pgfqpoint{0.480841in}{0.402751in}}%
\pgfpathcurveto{\pgfqpoint{0.491891in}{0.402751in}}{\pgfqpoint{0.502490in}{0.407141in}}{\pgfqpoint{0.510303in}{0.414955in}}%
\pgfpathcurveto{\pgfqpoint{0.518117in}{0.422768in}}{\pgfqpoint{0.522507in}{0.433367in}}{\pgfqpoint{0.522507in}{0.444417in}}%
\pgfpathcurveto{\pgfqpoint{0.522507in}{0.455468in}}{\pgfqpoint{0.518117in}{0.466067in}}{\pgfqpoint{0.510303in}{0.473880in}}%
\pgfpathcurveto{\pgfqpoint{0.502490in}{0.481694in}}{\pgfqpoint{0.491891in}{0.486084in}}{\pgfqpoint{0.480841in}{0.486084in}}%
\pgfpathcurveto{\pgfqpoint{0.469790in}{0.486084in}}{\pgfqpoint{0.459191in}{0.481694in}}{\pgfqpoint{0.451378in}{0.473880in}}%
\pgfpathcurveto{\pgfqpoint{0.443564in}{0.466067in}}{\pgfqpoint{0.439174in}{0.455468in}}{\pgfqpoint{0.439174in}{0.444417in}}%
\pgfpathcurveto{\pgfqpoint{0.439174in}{0.433367in}}{\pgfqpoint{0.443564in}{0.422768in}}{\pgfqpoint{0.451378in}{0.414955in}}%
\pgfpathcurveto{\pgfqpoint{0.459191in}{0.407141in}}{\pgfqpoint{0.469790in}{0.402751in}}{\pgfqpoint{0.480841in}{0.402751in}}%
\pgfpathclose%
\pgfusepath{stroke,fill}%
\end{pgfscope}%
\begin{pgfscope}%
\pgfpathrectangle{\pgfqpoint{0.375000in}{0.330000in}}{\pgfqpoint{2.325000in}{2.310000in}}%
\pgfusepath{clip}%
\pgfsetbuttcap%
\pgfsetroundjoin%
\definecolor{currentfill}{rgb}{0.000000,0.000000,0.000000}%
\pgfsetfillcolor{currentfill}%
\pgfsetlinewidth{1.003750pt}%
\definecolor{currentstroke}{rgb}{0.000000,0.000000,0.000000}%
\pgfsetstrokecolor{currentstroke}%
\pgfsetdash{}{0pt}%
\pgfpathmoveto{\pgfqpoint{0.480841in}{0.402751in}}%
\pgfpathcurveto{\pgfqpoint{0.491891in}{0.402751in}}{\pgfqpoint{0.502490in}{0.407141in}}{\pgfqpoint{0.510303in}{0.414955in}}%
\pgfpathcurveto{\pgfqpoint{0.518117in}{0.422768in}}{\pgfqpoint{0.522507in}{0.433367in}}{\pgfqpoint{0.522507in}{0.444417in}}%
\pgfpathcurveto{\pgfqpoint{0.522507in}{0.455468in}}{\pgfqpoint{0.518117in}{0.466067in}}{\pgfqpoint{0.510303in}{0.473880in}}%
\pgfpathcurveto{\pgfqpoint{0.502490in}{0.481694in}}{\pgfqpoint{0.491891in}{0.486084in}}{\pgfqpoint{0.480841in}{0.486084in}}%
\pgfpathcurveto{\pgfqpoint{0.469790in}{0.486084in}}{\pgfqpoint{0.459191in}{0.481694in}}{\pgfqpoint{0.451378in}{0.473880in}}%
\pgfpathcurveto{\pgfqpoint{0.443564in}{0.466067in}}{\pgfqpoint{0.439174in}{0.455468in}}{\pgfqpoint{0.439174in}{0.444417in}}%
\pgfpathcurveto{\pgfqpoint{0.439174in}{0.433367in}}{\pgfqpoint{0.443564in}{0.422768in}}{\pgfqpoint{0.451378in}{0.414955in}}%
\pgfpathcurveto{\pgfqpoint{0.459191in}{0.407141in}}{\pgfqpoint{0.469790in}{0.402751in}}{\pgfqpoint{0.480841in}{0.402751in}}%
\pgfpathclose%
\pgfusepath{stroke,fill}%
\end{pgfscope}%
\begin{pgfscope}%
\pgfpathrectangle{\pgfqpoint{0.375000in}{0.330000in}}{\pgfqpoint{2.325000in}{2.310000in}}%
\pgfusepath{clip}%
\pgfsetbuttcap%
\pgfsetroundjoin%
\definecolor{currentfill}{rgb}{0.000000,0.000000,0.000000}%
\pgfsetfillcolor{currentfill}%
\pgfsetlinewidth{1.003750pt}%
\definecolor{currentstroke}{rgb}{0.000000,0.000000,0.000000}%
\pgfsetstrokecolor{currentstroke}%
\pgfsetdash{}{0pt}%
\pgfpathmoveto{\pgfqpoint{0.480841in}{0.402751in}}%
\pgfpathcurveto{\pgfqpoint{0.491891in}{0.402751in}}{\pgfqpoint{0.502490in}{0.407141in}}{\pgfqpoint{0.510303in}{0.414955in}}%
\pgfpathcurveto{\pgfqpoint{0.518117in}{0.422768in}}{\pgfqpoint{0.522507in}{0.433367in}}{\pgfqpoint{0.522507in}{0.444417in}}%
\pgfpathcurveto{\pgfqpoint{0.522507in}{0.455468in}}{\pgfqpoint{0.518117in}{0.466067in}}{\pgfqpoint{0.510303in}{0.473880in}}%
\pgfpathcurveto{\pgfqpoint{0.502490in}{0.481694in}}{\pgfqpoint{0.491891in}{0.486084in}}{\pgfqpoint{0.480841in}{0.486084in}}%
\pgfpathcurveto{\pgfqpoint{0.469790in}{0.486084in}}{\pgfqpoint{0.459191in}{0.481694in}}{\pgfqpoint{0.451378in}{0.473880in}}%
\pgfpathcurveto{\pgfqpoint{0.443564in}{0.466067in}}{\pgfqpoint{0.439174in}{0.455468in}}{\pgfqpoint{0.439174in}{0.444417in}}%
\pgfpathcurveto{\pgfqpoint{0.439174in}{0.433367in}}{\pgfqpoint{0.443564in}{0.422768in}}{\pgfqpoint{0.451378in}{0.414955in}}%
\pgfpathcurveto{\pgfqpoint{0.459191in}{0.407141in}}{\pgfqpoint{0.469790in}{0.402751in}}{\pgfqpoint{0.480841in}{0.402751in}}%
\pgfpathclose%
\pgfusepath{stroke,fill}%
\end{pgfscope}%
\begin{pgfscope}%
\pgfpathrectangle{\pgfqpoint{0.375000in}{0.330000in}}{\pgfqpoint{2.325000in}{2.310000in}}%
\pgfusepath{clip}%
\pgfsetbuttcap%
\pgfsetroundjoin%
\definecolor{currentfill}{rgb}{0.000000,0.000000,0.000000}%
\pgfsetfillcolor{currentfill}%
\pgfsetlinewidth{1.003750pt}%
\definecolor{currentstroke}{rgb}{0.000000,0.000000,0.000000}%
\pgfsetstrokecolor{currentstroke}%
\pgfsetdash{}{0pt}%
\pgfpathmoveto{\pgfqpoint{0.480841in}{0.402751in}}%
\pgfpathcurveto{\pgfqpoint{0.491891in}{0.402751in}}{\pgfqpoint{0.502490in}{0.407141in}}{\pgfqpoint{0.510303in}{0.414955in}}%
\pgfpathcurveto{\pgfqpoint{0.518117in}{0.422768in}}{\pgfqpoint{0.522507in}{0.433367in}}{\pgfqpoint{0.522507in}{0.444417in}}%
\pgfpathcurveto{\pgfqpoint{0.522507in}{0.455468in}}{\pgfqpoint{0.518117in}{0.466067in}}{\pgfqpoint{0.510303in}{0.473880in}}%
\pgfpathcurveto{\pgfqpoint{0.502490in}{0.481694in}}{\pgfqpoint{0.491891in}{0.486084in}}{\pgfqpoint{0.480841in}{0.486084in}}%
\pgfpathcurveto{\pgfqpoint{0.469790in}{0.486084in}}{\pgfqpoint{0.459191in}{0.481694in}}{\pgfqpoint{0.451378in}{0.473880in}}%
\pgfpathcurveto{\pgfqpoint{0.443564in}{0.466067in}}{\pgfqpoint{0.439174in}{0.455468in}}{\pgfqpoint{0.439174in}{0.444417in}}%
\pgfpathcurveto{\pgfqpoint{0.439174in}{0.433367in}}{\pgfqpoint{0.443564in}{0.422768in}}{\pgfqpoint{0.451378in}{0.414955in}}%
\pgfpathcurveto{\pgfqpoint{0.459191in}{0.407141in}}{\pgfqpoint{0.469790in}{0.402751in}}{\pgfqpoint{0.480841in}{0.402751in}}%
\pgfpathclose%
\pgfusepath{stroke,fill}%
\end{pgfscope}%
\begin{pgfscope}%
\pgfpathrectangle{\pgfqpoint{0.375000in}{0.330000in}}{\pgfqpoint{2.325000in}{2.310000in}}%
\pgfusepath{clip}%
\pgfsetbuttcap%
\pgfsetroundjoin%
\definecolor{currentfill}{rgb}{0.000000,0.000000,0.000000}%
\pgfsetfillcolor{currentfill}%
\pgfsetlinewidth{1.003750pt}%
\definecolor{currentstroke}{rgb}{0.000000,0.000000,0.000000}%
\pgfsetstrokecolor{currentstroke}%
\pgfsetdash{}{0pt}%
\pgfpathmoveto{\pgfqpoint{0.480841in}{0.402751in}}%
\pgfpathcurveto{\pgfqpoint{0.491891in}{0.402751in}}{\pgfqpoint{0.502490in}{0.407141in}}{\pgfqpoint{0.510303in}{0.414955in}}%
\pgfpathcurveto{\pgfqpoint{0.518117in}{0.422768in}}{\pgfqpoint{0.522507in}{0.433367in}}{\pgfqpoint{0.522507in}{0.444417in}}%
\pgfpathcurveto{\pgfqpoint{0.522507in}{0.455468in}}{\pgfqpoint{0.518117in}{0.466067in}}{\pgfqpoint{0.510303in}{0.473880in}}%
\pgfpathcurveto{\pgfqpoint{0.502490in}{0.481694in}}{\pgfqpoint{0.491891in}{0.486084in}}{\pgfqpoint{0.480841in}{0.486084in}}%
\pgfpathcurveto{\pgfqpoint{0.469790in}{0.486084in}}{\pgfqpoint{0.459191in}{0.481694in}}{\pgfqpoint{0.451378in}{0.473880in}}%
\pgfpathcurveto{\pgfqpoint{0.443564in}{0.466067in}}{\pgfqpoint{0.439174in}{0.455468in}}{\pgfqpoint{0.439174in}{0.444417in}}%
\pgfpathcurveto{\pgfqpoint{0.439174in}{0.433367in}}{\pgfqpoint{0.443564in}{0.422768in}}{\pgfqpoint{0.451378in}{0.414955in}}%
\pgfpathcurveto{\pgfqpoint{0.459191in}{0.407141in}}{\pgfqpoint{0.469790in}{0.402751in}}{\pgfqpoint{0.480841in}{0.402751in}}%
\pgfpathclose%
\pgfusepath{stroke,fill}%
\end{pgfscope}%
\begin{pgfscope}%
\pgfpathrectangle{\pgfqpoint{0.375000in}{0.330000in}}{\pgfqpoint{2.325000in}{2.310000in}}%
\pgfusepath{clip}%
\pgfsetbuttcap%
\pgfsetroundjoin%
\definecolor{currentfill}{rgb}{0.000000,0.000000,0.000000}%
\pgfsetfillcolor{currentfill}%
\pgfsetlinewidth{1.003750pt}%
\definecolor{currentstroke}{rgb}{0.000000,0.000000,0.000000}%
\pgfsetstrokecolor{currentstroke}%
\pgfsetdash{}{0pt}%
\pgfpathmoveto{\pgfqpoint{0.480841in}{1.438667in}}%
\pgfpathcurveto{\pgfqpoint{0.491891in}{1.438667in}}{\pgfqpoint{0.502490in}{1.443057in}}{\pgfqpoint{0.510303in}{1.450871in}}%
\pgfpathcurveto{\pgfqpoint{0.518117in}{1.458685in}}{\pgfqpoint{0.522507in}{1.469284in}}{\pgfqpoint{0.522507in}{1.480334in}}%
\pgfpathcurveto{\pgfqpoint{0.522507in}{1.491384in}}{\pgfqpoint{0.518117in}{1.501983in}}{\pgfqpoint{0.510303in}{1.509796in}}%
\pgfpathcurveto{\pgfqpoint{0.502490in}{1.517610in}}{\pgfqpoint{0.491891in}{1.522000in}}{\pgfqpoint{0.480841in}{1.522000in}}%
\pgfpathcurveto{\pgfqpoint{0.469790in}{1.522000in}}{\pgfqpoint{0.459191in}{1.517610in}}{\pgfqpoint{0.451378in}{1.509796in}}%
\pgfpathcurveto{\pgfqpoint{0.443564in}{1.501983in}}{\pgfqpoint{0.439174in}{1.491384in}}{\pgfqpoint{0.439174in}{1.480334in}}%
\pgfpathcurveto{\pgfqpoint{0.439174in}{1.469284in}}{\pgfqpoint{0.443564in}{1.458685in}}{\pgfqpoint{0.451378in}{1.450871in}}%
\pgfpathcurveto{\pgfqpoint{0.459191in}{1.443057in}}{\pgfqpoint{0.469790in}{1.438667in}}{\pgfqpoint{0.480841in}{1.438667in}}%
\pgfpathclose%
\pgfusepath{stroke,fill}%
\end{pgfscope}%
\begin{pgfscope}%
\pgfpathrectangle{\pgfqpoint{0.375000in}{0.330000in}}{\pgfqpoint{2.325000in}{2.310000in}}%
\pgfusepath{clip}%
\pgfsetbuttcap%
\pgfsetroundjoin%
\definecolor{currentfill}{rgb}{0.000000,0.000000,0.000000}%
\pgfsetfillcolor{currentfill}%
\pgfsetlinewidth{1.003750pt}%
\definecolor{currentstroke}{rgb}{0.000000,0.000000,0.000000}%
\pgfsetstrokecolor{currentstroke}%
\pgfsetdash{}{0pt}%
\pgfpathmoveto{\pgfqpoint{0.480841in}{0.402751in}}%
\pgfpathcurveto{\pgfqpoint{0.491891in}{0.402751in}}{\pgfqpoint{0.502490in}{0.407141in}}{\pgfqpoint{0.510303in}{0.414955in}}%
\pgfpathcurveto{\pgfqpoint{0.518117in}{0.422768in}}{\pgfqpoint{0.522507in}{0.433367in}}{\pgfqpoint{0.522507in}{0.444417in}}%
\pgfpathcurveto{\pgfqpoint{0.522507in}{0.455468in}}{\pgfqpoint{0.518117in}{0.466067in}}{\pgfqpoint{0.510303in}{0.473880in}}%
\pgfpathcurveto{\pgfqpoint{0.502490in}{0.481694in}}{\pgfqpoint{0.491891in}{0.486084in}}{\pgfqpoint{0.480841in}{0.486084in}}%
\pgfpathcurveto{\pgfqpoint{0.469790in}{0.486084in}}{\pgfqpoint{0.459191in}{0.481694in}}{\pgfqpoint{0.451378in}{0.473880in}}%
\pgfpathcurveto{\pgfqpoint{0.443564in}{0.466067in}}{\pgfqpoint{0.439174in}{0.455468in}}{\pgfqpoint{0.439174in}{0.444417in}}%
\pgfpathcurveto{\pgfqpoint{0.439174in}{0.433367in}}{\pgfqpoint{0.443564in}{0.422768in}}{\pgfqpoint{0.451378in}{0.414955in}}%
\pgfpathcurveto{\pgfqpoint{0.459191in}{0.407141in}}{\pgfqpoint{0.469790in}{0.402751in}}{\pgfqpoint{0.480841in}{0.402751in}}%
\pgfpathclose%
\pgfusepath{stroke,fill}%
\end{pgfscope}%
\begin{pgfscope}%
\pgfpathrectangle{\pgfqpoint{0.375000in}{0.330000in}}{\pgfqpoint{2.325000in}{2.310000in}}%
\pgfusepath{clip}%
\pgfsetbuttcap%
\pgfsetroundjoin%
\definecolor{currentfill}{rgb}{0.000000,0.000000,0.000000}%
\pgfsetfillcolor{currentfill}%
\pgfsetlinewidth{1.003750pt}%
\definecolor{currentstroke}{rgb}{0.000000,0.000000,0.000000}%
\pgfsetstrokecolor{currentstroke}%
\pgfsetdash{}{0pt}%
\pgfpathmoveto{\pgfqpoint{0.480841in}{1.438667in}}%
\pgfpathcurveto{\pgfqpoint{0.491891in}{1.438667in}}{\pgfqpoint{0.502490in}{1.443057in}}{\pgfqpoint{0.510303in}{1.450871in}}%
\pgfpathcurveto{\pgfqpoint{0.518117in}{1.458685in}}{\pgfqpoint{0.522507in}{1.469284in}}{\pgfqpoint{0.522507in}{1.480334in}}%
\pgfpathcurveto{\pgfqpoint{0.522507in}{1.491384in}}{\pgfqpoint{0.518117in}{1.501983in}}{\pgfqpoint{0.510303in}{1.509796in}}%
\pgfpathcurveto{\pgfqpoint{0.502490in}{1.517610in}}{\pgfqpoint{0.491891in}{1.522000in}}{\pgfqpoint{0.480841in}{1.522000in}}%
\pgfpathcurveto{\pgfqpoint{0.469790in}{1.522000in}}{\pgfqpoint{0.459191in}{1.517610in}}{\pgfqpoint{0.451378in}{1.509796in}}%
\pgfpathcurveto{\pgfqpoint{0.443564in}{1.501983in}}{\pgfqpoint{0.439174in}{1.491384in}}{\pgfqpoint{0.439174in}{1.480334in}}%
\pgfpathcurveto{\pgfqpoint{0.439174in}{1.469284in}}{\pgfqpoint{0.443564in}{1.458685in}}{\pgfqpoint{0.451378in}{1.450871in}}%
\pgfpathcurveto{\pgfqpoint{0.459191in}{1.443057in}}{\pgfqpoint{0.469790in}{1.438667in}}{\pgfqpoint{0.480841in}{1.438667in}}%
\pgfpathclose%
\pgfusepath{stroke,fill}%
\end{pgfscope}%
\begin{pgfscope}%
\pgfpathrectangle{\pgfqpoint{0.375000in}{0.330000in}}{\pgfqpoint{2.325000in}{2.310000in}}%
\pgfusepath{clip}%
\pgfsetbuttcap%
\pgfsetroundjoin%
\definecolor{currentfill}{rgb}{0.000000,0.000000,0.000000}%
\pgfsetfillcolor{currentfill}%
\pgfsetlinewidth{1.003750pt}%
\definecolor{currentstroke}{rgb}{0.000000,0.000000,0.000000}%
\pgfsetstrokecolor{currentstroke}%
\pgfsetdash{}{0pt}%
\pgfpathmoveto{\pgfqpoint{0.480841in}{0.402751in}}%
\pgfpathcurveto{\pgfqpoint{0.491891in}{0.402751in}}{\pgfqpoint{0.502490in}{0.407141in}}{\pgfqpoint{0.510303in}{0.414955in}}%
\pgfpathcurveto{\pgfqpoint{0.518117in}{0.422768in}}{\pgfqpoint{0.522507in}{0.433367in}}{\pgfqpoint{0.522507in}{0.444417in}}%
\pgfpathcurveto{\pgfqpoint{0.522507in}{0.455468in}}{\pgfqpoint{0.518117in}{0.466067in}}{\pgfqpoint{0.510303in}{0.473880in}}%
\pgfpathcurveto{\pgfqpoint{0.502490in}{0.481694in}}{\pgfqpoint{0.491891in}{0.486084in}}{\pgfqpoint{0.480841in}{0.486084in}}%
\pgfpathcurveto{\pgfqpoint{0.469790in}{0.486084in}}{\pgfqpoint{0.459191in}{0.481694in}}{\pgfqpoint{0.451378in}{0.473880in}}%
\pgfpathcurveto{\pgfqpoint{0.443564in}{0.466067in}}{\pgfqpoint{0.439174in}{0.455468in}}{\pgfqpoint{0.439174in}{0.444417in}}%
\pgfpathcurveto{\pgfqpoint{0.439174in}{0.433367in}}{\pgfqpoint{0.443564in}{0.422768in}}{\pgfqpoint{0.451378in}{0.414955in}}%
\pgfpathcurveto{\pgfqpoint{0.459191in}{0.407141in}}{\pgfqpoint{0.469790in}{0.402751in}}{\pgfqpoint{0.480841in}{0.402751in}}%
\pgfpathclose%
\pgfusepath{stroke,fill}%
\end{pgfscope}%
\begin{pgfscope}%
\pgfpathrectangle{\pgfqpoint{0.375000in}{0.330000in}}{\pgfqpoint{2.325000in}{2.310000in}}%
\pgfusepath{clip}%
\pgfsetbuttcap%
\pgfsetroundjoin%
\definecolor{currentfill}{rgb}{0.000000,0.000000,0.000000}%
\pgfsetfillcolor{currentfill}%
\pgfsetlinewidth{1.003750pt}%
\definecolor{currentstroke}{rgb}{0.000000,0.000000,0.000000}%
\pgfsetstrokecolor{currentstroke}%
\pgfsetdash{}{0pt}%
\pgfpathmoveto{\pgfqpoint{0.480841in}{0.402751in}}%
\pgfpathcurveto{\pgfqpoint{0.491891in}{0.402751in}}{\pgfqpoint{0.502490in}{0.407141in}}{\pgfqpoint{0.510303in}{0.414955in}}%
\pgfpathcurveto{\pgfqpoint{0.518117in}{0.422768in}}{\pgfqpoint{0.522507in}{0.433367in}}{\pgfqpoint{0.522507in}{0.444417in}}%
\pgfpathcurveto{\pgfqpoint{0.522507in}{0.455468in}}{\pgfqpoint{0.518117in}{0.466067in}}{\pgfqpoint{0.510303in}{0.473880in}}%
\pgfpathcurveto{\pgfqpoint{0.502490in}{0.481694in}}{\pgfqpoint{0.491891in}{0.486084in}}{\pgfqpoint{0.480841in}{0.486084in}}%
\pgfpathcurveto{\pgfqpoint{0.469790in}{0.486084in}}{\pgfqpoint{0.459191in}{0.481694in}}{\pgfqpoint{0.451378in}{0.473880in}}%
\pgfpathcurveto{\pgfqpoint{0.443564in}{0.466067in}}{\pgfqpoint{0.439174in}{0.455468in}}{\pgfqpoint{0.439174in}{0.444417in}}%
\pgfpathcurveto{\pgfqpoint{0.439174in}{0.433367in}}{\pgfqpoint{0.443564in}{0.422768in}}{\pgfqpoint{0.451378in}{0.414955in}}%
\pgfpathcurveto{\pgfqpoint{0.459191in}{0.407141in}}{\pgfqpoint{0.469790in}{0.402751in}}{\pgfqpoint{0.480841in}{0.402751in}}%
\pgfpathclose%
\pgfusepath{stroke,fill}%
\end{pgfscope}%
\begin{pgfscope}%
\pgfpathrectangle{\pgfqpoint{0.375000in}{0.330000in}}{\pgfqpoint{2.325000in}{2.310000in}}%
\pgfusepath{clip}%
\pgfsetbuttcap%
\pgfsetroundjoin%
\definecolor{currentfill}{rgb}{0.000000,0.000000,0.000000}%
\pgfsetfillcolor{currentfill}%
\pgfsetlinewidth{1.003750pt}%
\definecolor{currentstroke}{rgb}{0.000000,0.000000,0.000000}%
\pgfsetstrokecolor{currentstroke}%
\pgfsetdash{}{0pt}%
\pgfpathmoveto{\pgfqpoint{0.480841in}{0.402751in}}%
\pgfpathcurveto{\pgfqpoint{0.491891in}{0.402751in}}{\pgfqpoint{0.502490in}{0.407141in}}{\pgfqpoint{0.510303in}{0.414955in}}%
\pgfpathcurveto{\pgfqpoint{0.518117in}{0.422768in}}{\pgfqpoint{0.522507in}{0.433367in}}{\pgfqpoint{0.522507in}{0.444417in}}%
\pgfpathcurveto{\pgfqpoint{0.522507in}{0.455468in}}{\pgfqpoint{0.518117in}{0.466067in}}{\pgfqpoint{0.510303in}{0.473880in}}%
\pgfpathcurveto{\pgfqpoint{0.502490in}{0.481694in}}{\pgfqpoint{0.491891in}{0.486084in}}{\pgfqpoint{0.480841in}{0.486084in}}%
\pgfpathcurveto{\pgfqpoint{0.469790in}{0.486084in}}{\pgfqpoint{0.459191in}{0.481694in}}{\pgfqpoint{0.451378in}{0.473880in}}%
\pgfpathcurveto{\pgfqpoint{0.443564in}{0.466067in}}{\pgfqpoint{0.439174in}{0.455468in}}{\pgfqpoint{0.439174in}{0.444417in}}%
\pgfpathcurveto{\pgfqpoint{0.439174in}{0.433367in}}{\pgfqpoint{0.443564in}{0.422768in}}{\pgfqpoint{0.451378in}{0.414955in}}%
\pgfpathcurveto{\pgfqpoint{0.459191in}{0.407141in}}{\pgfqpoint{0.469790in}{0.402751in}}{\pgfqpoint{0.480841in}{0.402751in}}%
\pgfpathclose%
\pgfusepath{stroke,fill}%
\end{pgfscope}%
\begin{pgfscope}%
\pgfpathrectangle{\pgfqpoint{0.375000in}{0.330000in}}{\pgfqpoint{2.325000in}{2.310000in}}%
\pgfusepath{clip}%
\pgfsetbuttcap%
\pgfsetroundjoin%
\definecolor{currentfill}{rgb}{0.000000,0.000000,0.000000}%
\pgfsetfillcolor{currentfill}%
\pgfsetlinewidth{1.003750pt}%
\definecolor{currentstroke}{rgb}{0.000000,0.000000,0.000000}%
\pgfsetstrokecolor{currentstroke}%
\pgfsetdash{}{0pt}%
\pgfpathmoveto{\pgfqpoint{0.480841in}{0.402751in}}%
\pgfpathcurveto{\pgfqpoint{0.491891in}{0.402751in}}{\pgfqpoint{0.502490in}{0.407141in}}{\pgfqpoint{0.510303in}{0.414955in}}%
\pgfpathcurveto{\pgfqpoint{0.518117in}{0.422768in}}{\pgfqpoint{0.522507in}{0.433367in}}{\pgfqpoint{0.522507in}{0.444417in}}%
\pgfpathcurveto{\pgfqpoint{0.522507in}{0.455468in}}{\pgfqpoint{0.518117in}{0.466067in}}{\pgfqpoint{0.510303in}{0.473880in}}%
\pgfpathcurveto{\pgfqpoint{0.502490in}{0.481694in}}{\pgfqpoint{0.491891in}{0.486084in}}{\pgfqpoint{0.480841in}{0.486084in}}%
\pgfpathcurveto{\pgfqpoint{0.469790in}{0.486084in}}{\pgfqpoint{0.459191in}{0.481694in}}{\pgfqpoint{0.451378in}{0.473880in}}%
\pgfpathcurveto{\pgfqpoint{0.443564in}{0.466067in}}{\pgfqpoint{0.439174in}{0.455468in}}{\pgfqpoint{0.439174in}{0.444417in}}%
\pgfpathcurveto{\pgfqpoint{0.439174in}{0.433367in}}{\pgfqpoint{0.443564in}{0.422768in}}{\pgfqpoint{0.451378in}{0.414955in}}%
\pgfpathcurveto{\pgfqpoint{0.459191in}{0.407141in}}{\pgfqpoint{0.469790in}{0.402751in}}{\pgfqpoint{0.480841in}{0.402751in}}%
\pgfpathclose%
\pgfusepath{stroke,fill}%
\end{pgfscope}%
\begin{pgfscope}%
\pgfpathrectangle{\pgfqpoint{0.375000in}{0.330000in}}{\pgfqpoint{2.325000in}{2.310000in}}%
\pgfusepath{clip}%
\pgfsetbuttcap%
\pgfsetroundjoin%
\definecolor{currentfill}{rgb}{0.000000,0.000000,0.000000}%
\pgfsetfillcolor{currentfill}%
\pgfsetlinewidth{1.003750pt}%
\definecolor{currentstroke}{rgb}{0.000000,0.000000,0.000000}%
\pgfsetstrokecolor{currentstroke}%
\pgfsetdash{}{0pt}%
\pgfpathmoveto{\pgfqpoint{0.480841in}{0.402751in}}%
\pgfpathcurveto{\pgfqpoint{0.491891in}{0.402751in}}{\pgfqpoint{0.502490in}{0.407141in}}{\pgfqpoint{0.510303in}{0.414955in}}%
\pgfpathcurveto{\pgfqpoint{0.518117in}{0.422768in}}{\pgfqpoint{0.522507in}{0.433367in}}{\pgfqpoint{0.522507in}{0.444417in}}%
\pgfpathcurveto{\pgfqpoint{0.522507in}{0.455468in}}{\pgfqpoint{0.518117in}{0.466067in}}{\pgfqpoint{0.510303in}{0.473880in}}%
\pgfpathcurveto{\pgfqpoint{0.502490in}{0.481694in}}{\pgfqpoint{0.491891in}{0.486084in}}{\pgfqpoint{0.480841in}{0.486084in}}%
\pgfpathcurveto{\pgfqpoint{0.469790in}{0.486084in}}{\pgfqpoint{0.459191in}{0.481694in}}{\pgfqpoint{0.451378in}{0.473880in}}%
\pgfpathcurveto{\pgfqpoint{0.443564in}{0.466067in}}{\pgfqpoint{0.439174in}{0.455468in}}{\pgfqpoint{0.439174in}{0.444417in}}%
\pgfpathcurveto{\pgfqpoint{0.439174in}{0.433367in}}{\pgfqpoint{0.443564in}{0.422768in}}{\pgfqpoint{0.451378in}{0.414955in}}%
\pgfpathcurveto{\pgfqpoint{0.459191in}{0.407141in}}{\pgfqpoint{0.469790in}{0.402751in}}{\pgfqpoint{0.480841in}{0.402751in}}%
\pgfpathclose%
\pgfusepath{stroke,fill}%
\end{pgfscope}%
\begin{pgfscope}%
\pgfpathrectangle{\pgfqpoint{0.375000in}{0.330000in}}{\pgfqpoint{2.325000in}{2.310000in}}%
\pgfusepath{clip}%
\pgfsetbuttcap%
\pgfsetroundjoin%
\definecolor{currentfill}{rgb}{0.000000,0.000000,0.000000}%
\pgfsetfillcolor{currentfill}%
\pgfsetlinewidth{1.003750pt}%
\definecolor{currentstroke}{rgb}{0.000000,0.000000,0.000000}%
\pgfsetstrokecolor{currentstroke}%
\pgfsetdash{}{0pt}%
\pgfpathmoveto{\pgfqpoint{0.480841in}{0.402751in}}%
\pgfpathcurveto{\pgfqpoint{0.491891in}{0.402751in}}{\pgfqpoint{0.502490in}{0.407141in}}{\pgfqpoint{0.510303in}{0.414955in}}%
\pgfpathcurveto{\pgfqpoint{0.518117in}{0.422768in}}{\pgfqpoint{0.522507in}{0.433367in}}{\pgfqpoint{0.522507in}{0.444417in}}%
\pgfpathcurveto{\pgfqpoint{0.522507in}{0.455468in}}{\pgfqpoint{0.518117in}{0.466067in}}{\pgfqpoint{0.510303in}{0.473880in}}%
\pgfpathcurveto{\pgfqpoint{0.502490in}{0.481694in}}{\pgfqpoint{0.491891in}{0.486084in}}{\pgfqpoint{0.480841in}{0.486084in}}%
\pgfpathcurveto{\pgfqpoint{0.469790in}{0.486084in}}{\pgfqpoint{0.459191in}{0.481694in}}{\pgfqpoint{0.451378in}{0.473880in}}%
\pgfpathcurveto{\pgfqpoint{0.443564in}{0.466067in}}{\pgfqpoint{0.439174in}{0.455468in}}{\pgfqpoint{0.439174in}{0.444417in}}%
\pgfpathcurveto{\pgfqpoint{0.439174in}{0.433367in}}{\pgfqpoint{0.443564in}{0.422768in}}{\pgfqpoint{0.451378in}{0.414955in}}%
\pgfpathcurveto{\pgfqpoint{0.459191in}{0.407141in}}{\pgfqpoint{0.469790in}{0.402751in}}{\pgfqpoint{0.480841in}{0.402751in}}%
\pgfpathclose%
\pgfusepath{stroke,fill}%
\end{pgfscope}%
\begin{pgfscope}%
\pgfpathrectangle{\pgfqpoint{0.375000in}{0.330000in}}{\pgfqpoint{2.325000in}{2.310000in}}%
\pgfusepath{clip}%
\pgfsetbuttcap%
\pgfsetroundjoin%
\definecolor{currentfill}{rgb}{0.000000,0.000000,0.000000}%
\pgfsetfillcolor{currentfill}%
\pgfsetlinewidth{1.003750pt}%
\definecolor{currentstroke}{rgb}{0.000000,0.000000,0.000000}%
\pgfsetstrokecolor{currentstroke}%
\pgfsetdash{}{0pt}%
\pgfpathmoveto{\pgfqpoint{0.480841in}{0.402751in}}%
\pgfpathcurveto{\pgfqpoint{0.491891in}{0.402751in}}{\pgfqpoint{0.502490in}{0.407141in}}{\pgfqpoint{0.510303in}{0.414955in}}%
\pgfpathcurveto{\pgfqpoint{0.518117in}{0.422768in}}{\pgfqpoint{0.522507in}{0.433367in}}{\pgfqpoint{0.522507in}{0.444417in}}%
\pgfpathcurveto{\pgfqpoint{0.522507in}{0.455468in}}{\pgfqpoint{0.518117in}{0.466067in}}{\pgfqpoint{0.510303in}{0.473880in}}%
\pgfpathcurveto{\pgfqpoint{0.502490in}{0.481694in}}{\pgfqpoint{0.491891in}{0.486084in}}{\pgfqpoint{0.480841in}{0.486084in}}%
\pgfpathcurveto{\pgfqpoint{0.469790in}{0.486084in}}{\pgfqpoint{0.459191in}{0.481694in}}{\pgfqpoint{0.451378in}{0.473880in}}%
\pgfpathcurveto{\pgfqpoint{0.443564in}{0.466067in}}{\pgfqpoint{0.439174in}{0.455468in}}{\pgfqpoint{0.439174in}{0.444417in}}%
\pgfpathcurveto{\pgfqpoint{0.439174in}{0.433367in}}{\pgfqpoint{0.443564in}{0.422768in}}{\pgfqpoint{0.451378in}{0.414955in}}%
\pgfpathcurveto{\pgfqpoint{0.459191in}{0.407141in}}{\pgfqpoint{0.469790in}{0.402751in}}{\pgfqpoint{0.480841in}{0.402751in}}%
\pgfpathclose%
\pgfusepath{stroke,fill}%
\end{pgfscope}%
\begin{pgfscope}%
\pgfpathrectangle{\pgfqpoint{0.375000in}{0.330000in}}{\pgfqpoint{2.325000in}{2.310000in}}%
\pgfusepath{clip}%
\pgfsetbuttcap%
\pgfsetroundjoin%
\definecolor{currentfill}{rgb}{0.000000,0.000000,0.000000}%
\pgfsetfillcolor{currentfill}%
\pgfsetlinewidth{1.003750pt}%
\definecolor{currentstroke}{rgb}{0.000000,0.000000,0.000000}%
\pgfsetstrokecolor{currentstroke}%
\pgfsetdash{}{0pt}%
\pgfpathmoveto{\pgfqpoint{0.480841in}{0.402751in}}%
\pgfpathcurveto{\pgfqpoint{0.491891in}{0.402751in}}{\pgfqpoint{0.502490in}{0.407141in}}{\pgfqpoint{0.510303in}{0.414955in}}%
\pgfpathcurveto{\pgfqpoint{0.518117in}{0.422768in}}{\pgfqpoint{0.522507in}{0.433367in}}{\pgfqpoint{0.522507in}{0.444417in}}%
\pgfpathcurveto{\pgfqpoint{0.522507in}{0.455468in}}{\pgfqpoint{0.518117in}{0.466067in}}{\pgfqpoint{0.510303in}{0.473880in}}%
\pgfpathcurveto{\pgfqpoint{0.502490in}{0.481694in}}{\pgfqpoint{0.491891in}{0.486084in}}{\pgfqpoint{0.480841in}{0.486084in}}%
\pgfpathcurveto{\pgfqpoint{0.469790in}{0.486084in}}{\pgfqpoint{0.459191in}{0.481694in}}{\pgfqpoint{0.451378in}{0.473880in}}%
\pgfpathcurveto{\pgfqpoint{0.443564in}{0.466067in}}{\pgfqpoint{0.439174in}{0.455468in}}{\pgfqpoint{0.439174in}{0.444417in}}%
\pgfpathcurveto{\pgfqpoint{0.439174in}{0.433367in}}{\pgfqpoint{0.443564in}{0.422768in}}{\pgfqpoint{0.451378in}{0.414955in}}%
\pgfpathcurveto{\pgfqpoint{0.459191in}{0.407141in}}{\pgfqpoint{0.469790in}{0.402751in}}{\pgfqpoint{0.480841in}{0.402751in}}%
\pgfpathclose%
\pgfusepath{stroke,fill}%
\end{pgfscope}%
\begin{pgfscope}%
\pgfpathrectangle{\pgfqpoint{0.375000in}{0.330000in}}{\pgfqpoint{2.325000in}{2.310000in}}%
\pgfusepath{clip}%
\pgfsetbuttcap%
\pgfsetroundjoin%
\definecolor{currentfill}{rgb}{0.000000,0.000000,0.000000}%
\pgfsetfillcolor{currentfill}%
\pgfsetlinewidth{1.003750pt}%
\definecolor{currentstroke}{rgb}{0.000000,0.000000,0.000000}%
\pgfsetstrokecolor{currentstroke}%
\pgfsetdash{}{0pt}%
\pgfpathmoveto{\pgfqpoint{0.480841in}{0.402751in}}%
\pgfpathcurveto{\pgfqpoint{0.491891in}{0.402751in}}{\pgfqpoint{0.502490in}{0.407141in}}{\pgfqpoint{0.510303in}{0.414955in}}%
\pgfpathcurveto{\pgfqpoint{0.518117in}{0.422768in}}{\pgfqpoint{0.522507in}{0.433367in}}{\pgfqpoint{0.522507in}{0.444417in}}%
\pgfpathcurveto{\pgfqpoint{0.522507in}{0.455468in}}{\pgfqpoint{0.518117in}{0.466067in}}{\pgfqpoint{0.510303in}{0.473880in}}%
\pgfpathcurveto{\pgfqpoint{0.502490in}{0.481694in}}{\pgfqpoint{0.491891in}{0.486084in}}{\pgfqpoint{0.480841in}{0.486084in}}%
\pgfpathcurveto{\pgfqpoint{0.469790in}{0.486084in}}{\pgfqpoint{0.459191in}{0.481694in}}{\pgfqpoint{0.451378in}{0.473880in}}%
\pgfpathcurveto{\pgfqpoint{0.443564in}{0.466067in}}{\pgfqpoint{0.439174in}{0.455468in}}{\pgfqpoint{0.439174in}{0.444417in}}%
\pgfpathcurveto{\pgfqpoint{0.439174in}{0.433367in}}{\pgfqpoint{0.443564in}{0.422768in}}{\pgfqpoint{0.451378in}{0.414955in}}%
\pgfpathcurveto{\pgfqpoint{0.459191in}{0.407141in}}{\pgfqpoint{0.469790in}{0.402751in}}{\pgfqpoint{0.480841in}{0.402751in}}%
\pgfpathclose%
\pgfusepath{stroke,fill}%
\end{pgfscope}%
\begin{pgfscope}%
\pgfpathrectangle{\pgfqpoint{0.375000in}{0.330000in}}{\pgfqpoint{2.325000in}{2.310000in}}%
\pgfusepath{clip}%
\pgfsetbuttcap%
\pgfsetroundjoin%
\definecolor{currentfill}{rgb}{0.000000,0.000000,0.000000}%
\pgfsetfillcolor{currentfill}%
\pgfsetlinewidth{1.003750pt}%
\definecolor{currentstroke}{rgb}{0.000000,0.000000,0.000000}%
\pgfsetstrokecolor{currentstroke}%
\pgfsetdash{}{0pt}%
\pgfpathmoveto{\pgfqpoint{0.480841in}{0.402751in}}%
\pgfpathcurveto{\pgfqpoint{0.491891in}{0.402751in}}{\pgfqpoint{0.502490in}{0.407141in}}{\pgfqpoint{0.510303in}{0.414955in}}%
\pgfpathcurveto{\pgfqpoint{0.518117in}{0.422768in}}{\pgfqpoint{0.522507in}{0.433367in}}{\pgfqpoint{0.522507in}{0.444417in}}%
\pgfpathcurveto{\pgfqpoint{0.522507in}{0.455468in}}{\pgfqpoint{0.518117in}{0.466067in}}{\pgfqpoint{0.510303in}{0.473880in}}%
\pgfpathcurveto{\pgfqpoint{0.502490in}{0.481694in}}{\pgfqpoint{0.491891in}{0.486084in}}{\pgfqpoint{0.480841in}{0.486084in}}%
\pgfpathcurveto{\pgfqpoint{0.469790in}{0.486084in}}{\pgfqpoint{0.459191in}{0.481694in}}{\pgfqpoint{0.451378in}{0.473880in}}%
\pgfpathcurveto{\pgfqpoint{0.443564in}{0.466067in}}{\pgfqpoint{0.439174in}{0.455468in}}{\pgfqpoint{0.439174in}{0.444417in}}%
\pgfpathcurveto{\pgfqpoint{0.439174in}{0.433367in}}{\pgfqpoint{0.443564in}{0.422768in}}{\pgfqpoint{0.451378in}{0.414955in}}%
\pgfpathcurveto{\pgfqpoint{0.459191in}{0.407141in}}{\pgfqpoint{0.469790in}{0.402751in}}{\pgfqpoint{0.480841in}{0.402751in}}%
\pgfpathclose%
\pgfusepath{stroke,fill}%
\end{pgfscope}%
\begin{pgfscope}%
\pgfpathrectangle{\pgfqpoint{0.375000in}{0.330000in}}{\pgfqpoint{2.325000in}{2.310000in}}%
\pgfusepath{clip}%
\pgfsetbuttcap%
\pgfsetroundjoin%
\definecolor{currentfill}{rgb}{0.000000,0.000000,0.000000}%
\pgfsetfillcolor{currentfill}%
\pgfsetlinewidth{1.003750pt}%
\definecolor{currentstroke}{rgb}{0.000000,0.000000,0.000000}%
\pgfsetstrokecolor{currentstroke}%
\pgfsetdash{}{0pt}%
\pgfpathmoveto{\pgfqpoint{0.480841in}{0.402751in}}%
\pgfpathcurveto{\pgfqpoint{0.491891in}{0.402751in}}{\pgfqpoint{0.502490in}{0.407141in}}{\pgfqpoint{0.510303in}{0.414955in}}%
\pgfpathcurveto{\pgfqpoint{0.518117in}{0.422768in}}{\pgfqpoint{0.522507in}{0.433367in}}{\pgfqpoint{0.522507in}{0.444417in}}%
\pgfpathcurveto{\pgfqpoint{0.522507in}{0.455468in}}{\pgfqpoint{0.518117in}{0.466067in}}{\pgfqpoint{0.510303in}{0.473880in}}%
\pgfpathcurveto{\pgfqpoint{0.502490in}{0.481694in}}{\pgfqpoint{0.491891in}{0.486084in}}{\pgfqpoint{0.480841in}{0.486084in}}%
\pgfpathcurveto{\pgfqpoint{0.469790in}{0.486084in}}{\pgfqpoint{0.459191in}{0.481694in}}{\pgfqpoint{0.451378in}{0.473880in}}%
\pgfpathcurveto{\pgfqpoint{0.443564in}{0.466067in}}{\pgfqpoint{0.439174in}{0.455468in}}{\pgfqpoint{0.439174in}{0.444417in}}%
\pgfpathcurveto{\pgfqpoint{0.439174in}{0.433367in}}{\pgfqpoint{0.443564in}{0.422768in}}{\pgfqpoint{0.451378in}{0.414955in}}%
\pgfpathcurveto{\pgfqpoint{0.459191in}{0.407141in}}{\pgfqpoint{0.469790in}{0.402751in}}{\pgfqpoint{0.480841in}{0.402751in}}%
\pgfpathclose%
\pgfusepath{stroke,fill}%
\end{pgfscope}%
\begin{pgfscope}%
\pgfpathrectangle{\pgfqpoint{0.375000in}{0.330000in}}{\pgfqpoint{2.325000in}{2.310000in}}%
\pgfusepath{clip}%
\pgfsetbuttcap%
\pgfsetroundjoin%
\definecolor{currentfill}{rgb}{0.000000,0.000000,0.000000}%
\pgfsetfillcolor{currentfill}%
\pgfsetlinewidth{1.003750pt}%
\definecolor{currentstroke}{rgb}{0.000000,0.000000,0.000000}%
\pgfsetstrokecolor{currentstroke}%
\pgfsetdash{}{0pt}%
\pgfpathmoveto{\pgfqpoint{0.480841in}{0.402751in}}%
\pgfpathcurveto{\pgfqpoint{0.491891in}{0.402751in}}{\pgfqpoint{0.502490in}{0.407141in}}{\pgfqpoint{0.510303in}{0.414955in}}%
\pgfpathcurveto{\pgfqpoint{0.518117in}{0.422768in}}{\pgfqpoint{0.522507in}{0.433367in}}{\pgfqpoint{0.522507in}{0.444417in}}%
\pgfpathcurveto{\pgfqpoint{0.522507in}{0.455468in}}{\pgfqpoint{0.518117in}{0.466067in}}{\pgfqpoint{0.510303in}{0.473880in}}%
\pgfpathcurveto{\pgfqpoint{0.502490in}{0.481694in}}{\pgfqpoint{0.491891in}{0.486084in}}{\pgfqpoint{0.480841in}{0.486084in}}%
\pgfpathcurveto{\pgfqpoint{0.469790in}{0.486084in}}{\pgfqpoint{0.459191in}{0.481694in}}{\pgfqpoint{0.451378in}{0.473880in}}%
\pgfpathcurveto{\pgfqpoint{0.443564in}{0.466067in}}{\pgfqpoint{0.439174in}{0.455468in}}{\pgfqpoint{0.439174in}{0.444417in}}%
\pgfpathcurveto{\pgfqpoint{0.439174in}{0.433367in}}{\pgfqpoint{0.443564in}{0.422768in}}{\pgfqpoint{0.451378in}{0.414955in}}%
\pgfpathcurveto{\pgfqpoint{0.459191in}{0.407141in}}{\pgfqpoint{0.469790in}{0.402751in}}{\pgfqpoint{0.480841in}{0.402751in}}%
\pgfpathclose%
\pgfusepath{stroke,fill}%
\end{pgfscope}%
\begin{pgfscope}%
\pgfpathrectangle{\pgfqpoint{0.375000in}{0.330000in}}{\pgfqpoint{2.325000in}{2.310000in}}%
\pgfusepath{clip}%
\pgfsetbuttcap%
\pgfsetroundjoin%
\definecolor{currentfill}{rgb}{0.000000,0.000000,0.000000}%
\pgfsetfillcolor{currentfill}%
\pgfsetlinewidth{1.003750pt}%
\definecolor{currentstroke}{rgb}{0.000000,0.000000,0.000000}%
\pgfsetstrokecolor{currentstroke}%
\pgfsetdash{}{0pt}%
\pgfpathmoveto{\pgfqpoint{0.480841in}{0.402751in}}%
\pgfpathcurveto{\pgfqpoint{0.491891in}{0.402751in}}{\pgfqpoint{0.502490in}{0.407141in}}{\pgfqpoint{0.510303in}{0.414955in}}%
\pgfpathcurveto{\pgfqpoint{0.518117in}{0.422768in}}{\pgfqpoint{0.522507in}{0.433367in}}{\pgfqpoint{0.522507in}{0.444417in}}%
\pgfpathcurveto{\pgfqpoint{0.522507in}{0.455468in}}{\pgfqpoint{0.518117in}{0.466067in}}{\pgfqpoint{0.510303in}{0.473880in}}%
\pgfpathcurveto{\pgfqpoint{0.502490in}{0.481694in}}{\pgfqpoint{0.491891in}{0.486084in}}{\pgfqpoint{0.480841in}{0.486084in}}%
\pgfpathcurveto{\pgfqpoint{0.469790in}{0.486084in}}{\pgfqpoint{0.459191in}{0.481694in}}{\pgfqpoint{0.451378in}{0.473880in}}%
\pgfpathcurveto{\pgfqpoint{0.443564in}{0.466067in}}{\pgfqpoint{0.439174in}{0.455468in}}{\pgfqpoint{0.439174in}{0.444417in}}%
\pgfpathcurveto{\pgfqpoint{0.439174in}{0.433367in}}{\pgfqpoint{0.443564in}{0.422768in}}{\pgfqpoint{0.451378in}{0.414955in}}%
\pgfpathcurveto{\pgfqpoint{0.459191in}{0.407141in}}{\pgfqpoint{0.469790in}{0.402751in}}{\pgfqpoint{0.480841in}{0.402751in}}%
\pgfpathclose%
\pgfusepath{stroke,fill}%
\end{pgfscope}%
\begin{pgfscope}%
\pgfpathrectangle{\pgfqpoint{0.375000in}{0.330000in}}{\pgfqpoint{2.325000in}{2.310000in}}%
\pgfusepath{clip}%
\pgfsetbuttcap%
\pgfsetroundjoin%
\definecolor{currentfill}{rgb}{0.000000,0.000000,0.000000}%
\pgfsetfillcolor{currentfill}%
\pgfsetlinewidth{1.003750pt}%
\definecolor{currentstroke}{rgb}{0.000000,0.000000,0.000000}%
\pgfsetstrokecolor{currentstroke}%
\pgfsetdash{}{0pt}%
\pgfpathmoveto{\pgfqpoint{0.480841in}{0.402751in}}%
\pgfpathcurveto{\pgfqpoint{0.491891in}{0.402751in}}{\pgfqpoint{0.502490in}{0.407141in}}{\pgfqpoint{0.510303in}{0.414955in}}%
\pgfpathcurveto{\pgfqpoint{0.518117in}{0.422768in}}{\pgfqpoint{0.522507in}{0.433367in}}{\pgfqpoint{0.522507in}{0.444417in}}%
\pgfpathcurveto{\pgfqpoint{0.522507in}{0.455468in}}{\pgfqpoint{0.518117in}{0.466067in}}{\pgfqpoint{0.510303in}{0.473880in}}%
\pgfpathcurveto{\pgfqpoint{0.502490in}{0.481694in}}{\pgfqpoint{0.491891in}{0.486084in}}{\pgfqpoint{0.480841in}{0.486084in}}%
\pgfpathcurveto{\pgfqpoint{0.469790in}{0.486084in}}{\pgfqpoint{0.459191in}{0.481694in}}{\pgfqpoint{0.451378in}{0.473880in}}%
\pgfpathcurveto{\pgfqpoint{0.443564in}{0.466067in}}{\pgfqpoint{0.439174in}{0.455468in}}{\pgfqpoint{0.439174in}{0.444417in}}%
\pgfpathcurveto{\pgfqpoint{0.439174in}{0.433367in}}{\pgfqpoint{0.443564in}{0.422768in}}{\pgfqpoint{0.451378in}{0.414955in}}%
\pgfpathcurveto{\pgfqpoint{0.459191in}{0.407141in}}{\pgfqpoint{0.469790in}{0.402751in}}{\pgfqpoint{0.480841in}{0.402751in}}%
\pgfpathclose%
\pgfusepath{stroke,fill}%
\end{pgfscope}%
\begin{pgfscope}%
\pgfpathrectangle{\pgfqpoint{0.375000in}{0.330000in}}{\pgfqpoint{2.325000in}{2.310000in}}%
\pgfusepath{clip}%
\pgfsetbuttcap%
\pgfsetroundjoin%
\definecolor{currentfill}{rgb}{0.000000,0.000000,0.000000}%
\pgfsetfillcolor{currentfill}%
\pgfsetlinewidth{1.003750pt}%
\definecolor{currentstroke}{rgb}{0.000000,0.000000,0.000000}%
\pgfsetstrokecolor{currentstroke}%
\pgfsetdash{}{0pt}%
\pgfpathmoveto{\pgfqpoint{0.480841in}{0.402751in}}%
\pgfpathcurveto{\pgfqpoint{0.491891in}{0.402751in}}{\pgfqpoint{0.502490in}{0.407141in}}{\pgfqpoint{0.510303in}{0.414955in}}%
\pgfpathcurveto{\pgfqpoint{0.518117in}{0.422768in}}{\pgfqpoint{0.522507in}{0.433367in}}{\pgfqpoint{0.522507in}{0.444417in}}%
\pgfpathcurveto{\pgfqpoint{0.522507in}{0.455468in}}{\pgfqpoint{0.518117in}{0.466067in}}{\pgfqpoint{0.510303in}{0.473880in}}%
\pgfpathcurveto{\pgfqpoint{0.502490in}{0.481694in}}{\pgfqpoint{0.491891in}{0.486084in}}{\pgfqpoint{0.480841in}{0.486084in}}%
\pgfpathcurveto{\pgfqpoint{0.469790in}{0.486084in}}{\pgfqpoint{0.459191in}{0.481694in}}{\pgfqpoint{0.451378in}{0.473880in}}%
\pgfpathcurveto{\pgfqpoint{0.443564in}{0.466067in}}{\pgfqpoint{0.439174in}{0.455468in}}{\pgfqpoint{0.439174in}{0.444417in}}%
\pgfpathcurveto{\pgfqpoint{0.439174in}{0.433367in}}{\pgfqpoint{0.443564in}{0.422768in}}{\pgfqpoint{0.451378in}{0.414955in}}%
\pgfpathcurveto{\pgfqpoint{0.459191in}{0.407141in}}{\pgfqpoint{0.469790in}{0.402751in}}{\pgfqpoint{0.480841in}{0.402751in}}%
\pgfpathclose%
\pgfusepath{stroke,fill}%
\end{pgfscope}%
\begin{pgfscope}%
\pgfpathrectangle{\pgfqpoint{0.375000in}{0.330000in}}{\pgfqpoint{2.325000in}{2.310000in}}%
\pgfusepath{clip}%
\pgfsetbuttcap%
\pgfsetroundjoin%
\definecolor{currentfill}{rgb}{0.000000,0.000000,0.000000}%
\pgfsetfillcolor{currentfill}%
\pgfsetlinewidth{1.003750pt}%
\definecolor{currentstroke}{rgb}{0.000000,0.000000,0.000000}%
\pgfsetstrokecolor{currentstroke}%
\pgfsetdash{}{0pt}%
\pgfpathmoveto{\pgfqpoint{0.480841in}{0.402751in}}%
\pgfpathcurveto{\pgfqpoint{0.491891in}{0.402751in}}{\pgfqpoint{0.502490in}{0.407141in}}{\pgfqpoint{0.510303in}{0.414955in}}%
\pgfpathcurveto{\pgfqpoint{0.518117in}{0.422768in}}{\pgfqpoint{0.522507in}{0.433367in}}{\pgfqpoint{0.522507in}{0.444417in}}%
\pgfpathcurveto{\pgfqpoint{0.522507in}{0.455468in}}{\pgfqpoint{0.518117in}{0.466067in}}{\pgfqpoint{0.510303in}{0.473880in}}%
\pgfpathcurveto{\pgfqpoint{0.502490in}{0.481694in}}{\pgfqpoint{0.491891in}{0.486084in}}{\pgfqpoint{0.480841in}{0.486084in}}%
\pgfpathcurveto{\pgfqpoint{0.469790in}{0.486084in}}{\pgfqpoint{0.459191in}{0.481694in}}{\pgfqpoint{0.451378in}{0.473880in}}%
\pgfpathcurveto{\pgfqpoint{0.443564in}{0.466067in}}{\pgfqpoint{0.439174in}{0.455468in}}{\pgfqpoint{0.439174in}{0.444417in}}%
\pgfpathcurveto{\pgfqpoint{0.439174in}{0.433367in}}{\pgfqpoint{0.443564in}{0.422768in}}{\pgfqpoint{0.451378in}{0.414955in}}%
\pgfpathcurveto{\pgfqpoint{0.459191in}{0.407141in}}{\pgfqpoint{0.469790in}{0.402751in}}{\pgfqpoint{0.480841in}{0.402751in}}%
\pgfpathclose%
\pgfusepath{stroke,fill}%
\end{pgfscope}%
\begin{pgfscope}%
\pgfpathrectangle{\pgfqpoint{0.375000in}{0.330000in}}{\pgfqpoint{2.325000in}{2.310000in}}%
\pgfusepath{clip}%
\pgfsetbuttcap%
\pgfsetroundjoin%
\definecolor{currentfill}{rgb}{0.000000,0.000000,0.000000}%
\pgfsetfillcolor{currentfill}%
\pgfsetlinewidth{1.003750pt}%
\definecolor{currentstroke}{rgb}{0.000000,0.000000,0.000000}%
\pgfsetstrokecolor{currentstroke}%
\pgfsetdash{}{0pt}%
\pgfpathmoveto{\pgfqpoint{0.480841in}{0.402751in}}%
\pgfpathcurveto{\pgfqpoint{0.491891in}{0.402751in}}{\pgfqpoint{0.502490in}{0.407141in}}{\pgfqpoint{0.510303in}{0.414955in}}%
\pgfpathcurveto{\pgfqpoint{0.518117in}{0.422768in}}{\pgfqpoint{0.522507in}{0.433367in}}{\pgfqpoint{0.522507in}{0.444417in}}%
\pgfpathcurveto{\pgfqpoint{0.522507in}{0.455468in}}{\pgfqpoint{0.518117in}{0.466067in}}{\pgfqpoint{0.510303in}{0.473880in}}%
\pgfpathcurveto{\pgfqpoint{0.502490in}{0.481694in}}{\pgfqpoint{0.491891in}{0.486084in}}{\pgfqpoint{0.480841in}{0.486084in}}%
\pgfpathcurveto{\pgfqpoint{0.469790in}{0.486084in}}{\pgfqpoint{0.459191in}{0.481694in}}{\pgfqpoint{0.451378in}{0.473880in}}%
\pgfpathcurveto{\pgfqpoint{0.443564in}{0.466067in}}{\pgfqpoint{0.439174in}{0.455468in}}{\pgfqpoint{0.439174in}{0.444417in}}%
\pgfpathcurveto{\pgfqpoint{0.439174in}{0.433367in}}{\pgfqpoint{0.443564in}{0.422768in}}{\pgfqpoint{0.451378in}{0.414955in}}%
\pgfpathcurveto{\pgfqpoint{0.459191in}{0.407141in}}{\pgfqpoint{0.469790in}{0.402751in}}{\pgfqpoint{0.480841in}{0.402751in}}%
\pgfpathclose%
\pgfusepath{stroke,fill}%
\end{pgfscope}%
\begin{pgfscope}%
\pgfpathrectangle{\pgfqpoint{0.375000in}{0.330000in}}{\pgfqpoint{2.325000in}{2.310000in}}%
\pgfusepath{clip}%
\pgfsetbuttcap%
\pgfsetroundjoin%
\definecolor{currentfill}{rgb}{0.000000,0.000000,0.000000}%
\pgfsetfillcolor{currentfill}%
\pgfsetlinewidth{1.003750pt}%
\definecolor{currentstroke}{rgb}{0.000000,0.000000,0.000000}%
\pgfsetstrokecolor{currentstroke}%
\pgfsetdash{}{0pt}%
\pgfpathmoveto{\pgfqpoint{0.480841in}{0.402751in}}%
\pgfpathcurveto{\pgfqpoint{0.491891in}{0.402751in}}{\pgfqpoint{0.502490in}{0.407141in}}{\pgfqpoint{0.510303in}{0.414955in}}%
\pgfpathcurveto{\pgfqpoint{0.518117in}{0.422768in}}{\pgfqpoint{0.522507in}{0.433367in}}{\pgfqpoint{0.522507in}{0.444417in}}%
\pgfpathcurveto{\pgfqpoint{0.522507in}{0.455468in}}{\pgfqpoint{0.518117in}{0.466067in}}{\pgfqpoint{0.510303in}{0.473880in}}%
\pgfpathcurveto{\pgfqpoint{0.502490in}{0.481694in}}{\pgfqpoint{0.491891in}{0.486084in}}{\pgfqpoint{0.480841in}{0.486084in}}%
\pgfpathcurveto{\pgfqpoint{0.469790in}{0.486084in}}{\pgfqpoint{0.459191in}{0.481694in}}{\pgfqpoint{0.451378in}{0.473880in}}%
\pgfpathcurveto{\pgfqpoint{0.443564in}{0.466067in}}{\pgfqpoint{0.439174in}{0.455468in}}{\pgfqpoint{0.439174in}{0.444417in}}%
\pgfpathcurveto{\pgfqpoint{0.439174in}{0.433367in}}{\pgfqpoint{0.443564in}{0.422768in}}{\pgfqpoint{0.451378in}{0.414955in}}%
\pgfpathcurveto{\pgfqpoint{0.459191in}{0.407141in}}{\pgfqpoint{0.469790in}{0.402751in}}{\pgfqpoint{0.480841in}{0.402751in}}%
\pgfpathclose%
\pgfusepath{stroke,fill}%
\end{pgfscope}%
\begin{pgfscope}%
\pgfpathrectangle{\pgfqpoint{0.375000in}{0.330000in}}{\pgfqpoint{2.325000in}{2.310000in}}%
\pgfusepath{clip}%
\pgfsetbuttcap%
\pgfsetroundjoin%
\definecolor{currentfill}{rgb}{0.000000,0.000000,0.000000}%
\pgfsetfillcolor{currentfill}%
\pgfsetlinewidth{1.003750pt}%
\definecolor{currentstroke}{rgb}{0.000000,0.000000,0.000000}%
\pgfsetstrokecolor{currentstroke}%
\pgfsetdash{}{0pt}%
\pgfpathmoveto{\pgfqpoint{0.480841in}{0.402751in}}%
\pgfpathcurveto{\pgfqpoint{0.491891in}{0.402751in}}{\pgfqpoint{0.502490in}{0.407141in}}{\pgfqpoint{0.510303in}{0.414955in}}%
\pgfpathcurveto{\pgfqpoint{0.518117in}{0.422768in}}{\pgfqpoint{0.522507in}{0.433367in}}{\pgfqpoint{0.522507in}{0.444417in}}%
\pgfpathcurveto{\pgfqpoint{0.522507in}{0.455468in}}{\pgfqpoint{0.518117in}{0.466067in}}{\pgfqpoint{0.510303in}{0.473880in}}%
\pgfpathcurveto{\pgfqpoint{0.502490in}{0.481694in}}{\pgfqpoint{0.491891in}{0.486084in}}{\pgfqpoint{0.480841in}{0.486084in}}%
\pgfpathcurveto{\pgfqpoint{0.469790in}{0.486084in}}{\pgfqpoint{0.459191in}{0.481694in}}{\pgfqpoint{0.451378in}{0.473880in}}%
\pgfpathcurveto{\pgfqpoint{0.443564in}{0.466067in}}{\pgfqpoint{0.439174in}{0.455468in}}{\pgfqpoint{0.439174in}{0.444417in}}%
\pgfpathcurveto{\pgfqpoint{0.439174in}{0.433367in}}{\pgfqpoint{0.443564in}{0.422768in}}{\pgfqpoint{0.451378in}{0.414955in}}%
\pgfpathcurveto{\pgfqpoint{0.459191in}{0.407141in}}{\pgfqpoint{0.469790in}{0.402751in}}{\pgfqpoint{0.480841in}{0.402751in}}%
\pgfpathclose%
\pgfusepath{stroke,fill}%
\end{pgfscope}%
\begin{pgfscope}%
\pgfpathrectangle{\pgfqpoint{0.375000in}{0.330000in}}{\pgfqpoint{2.325000in}{2.310000in}}%
\pgfusepath{clip}%
\pgfsetbuttcap%
\pgfsetroundjoin%
\definecolor{currentfill}{rgb}{0.000000,0.000000,0.000000}%
\pgfsetfillcolor{currentfill}%
\pgfsetlinewidth{1.003750pt}%
\definecolor{currentstroke}{rgb}{0.000000,0.000000,0.000000}%
\pgfsetstrokecolor{currentstroke}%
\pgfsetdash{}{0pt}%
\pgfpathmoveto{\pgfqpoint{0.480841in}{0.402751in}}%
\pgfpathcurveto{\pgfqpoint{0.491891in}{0.402751in}}{\pgfqpoint{0.502490in}{0.407141in}}{\pgfqpoint{0.510303in}{0.414955in}}%
\pgfpathcurveto{\pgfqpoint{0.518117in}{0.422768in}}{\pgfqpoint{0.522507in}{0.433367in}}{\pgfqpoint{0.522507in}{0.444417in}}%
\pgfpathcurveto{\pgfqpoint{0.522507in}{0.455468in}}{\pgfqpoint{0.518117in}{0.466067in}}{\pgfqpoint{0.510303in}{0.473880in}}%
\pgfpathcurveto{\pgfqpoint{0.502490in}{0.481694in}}{\pgfqpoint{0.491891in}{0.486084in}}{\pgfqpoint{0.480841in}{0.486084in}}%
\pgfpathcurveto{\pgfqpoint{0.469790in}{0.486084in}}{\pgfqpoint{0.459191in}{0.481694in}}{\pgfqpoint{0.451378in}{0.473880in}}%
\pgfpathcurveto{\pgfqpoint{0.443564in}{0.466067in}}{\pgfqpoint{0.439174in}{0.455468in}}{\pgfqpoint{0.439174in}{0.444417in}}%
\pgfpathcurveto{\pgfqpoint{0.439174in}{0.433367in}}{\pgfqpoint{0.443564in}{0.422768in}}{\pgfqpoint{0.451378in}{0.414955in}}%
\pgfpathcurveto{\pgfqpoint{0.459191in}{0.407141in}}{\pgfqpoint{0.469790in}{0.402751in}}{\pgfqpoint{0.480841in}{0.402751in}}%
\pgfpathclose%
\pgfusepath{stroke,fill}%
\end{pgfscope}%
\begin{pgfscope}%
\pgfpathrectangle{\pgfqpoint{0.375000in}{0.330000in}}{\pgfqpoint{2.325000in}{2.310000in}}%
\pgfusepath{clip}%
\pgfsetbuttcap%
\pgfsetroundjoin%
\definecolor{currentfill}{rgb}{0.000000,0.000000,0.000000}%
\pgfsetfillcolor{currentfill}%
\pgfsetlinewidth{1.003750pt}%
\definecolor{currentstroke}{rgb}{0.000000,0.000000,0.000000}%
\pgfsetstrokecolor{currentstroke}%
\pgfsetdash{}{0pt}%
\pgfpathmoveto{\pgfqpoint{0.480841in}{0.402751in}}%
\pgfpathcurveto{\pgfqpoint{0.491891in}{0.402751in}}{\pgfqpoint{0.502490in}{0.407141in}}{\pgfqpoint{0.510303in}{0.414955in}}%
\pgfpathcurveto{\pgfqpoint{0.518117in}{0.422768in}}{\pgfqpoint{0.522507in}{0.433367in}}{\pgfqpoint{0.522507in}{0.444417in}}%
\pgfpathcurveto{\pgfqpoint{0.522507in}{0.455468in}}{\pgfqpoint{0.518117in}{0.466067in}}{\pgfqpoint{0.510303in}{0.473880in}}%
\pgfpathcurveto{\pgfqpoint{0.502490in}{0.481694in}}{\pgfqpoint{0.491891in}{0.486084in}}{\pgfqpoint{0.480841in}{0.486084in}}%
\pgfpathcurveto{\pgfqpoint{0.469790in}{0.486084in}}{\pgfqpoint{0.459191in}{0.481694in}}{\pgfqpoint{0.451378in}{0.473880in}}%
\pgfpathcurveto{\pgfqpoint{0.443564in}{0.466067in}}{\pgfqpoint{0.439174in}{0.455468in}}{\pgfqpoint{0.439174in}{0.444417in}}%
\pgfpathcurveto{\pgfqpoint{0.439174in}{0.433367in}}{\pgfqpoint{0.443564in}{0.422768in}}{\pgfqpoint{0.451378in}{0.414955in}}%
\pgfpathcurveto{\pgfqpoint{0.459191in}{0.407141in}}{\pgfqpoint{0.469790in}{0.402751in}}{\pgfqpoint{0.480841in}{0.402751in}}%
\pgfpathclose%
\pgfusepath{stroke,fill}%
\end{pgfscope}%
\begin{pgfscope}%
\pgfpathrectangle{\pgfqpoint{0.375000in}{0.330000in}}{\pgfqpoint{2.325000in}{2.310000in}}%
\pgfusepath{clip}%
\pgfsetbuttcap%
\pgfsetroundjoin%
\definecolor{currentfill}{rgb}{0.000000,0.000000,0.000000}%
\pgfsetfillcolor{currentfill}%
\pgfsetlinewidth{1.003750pt}%
\definecolor{currentstroke}{rgb}{0.000000,0.000000,0.000000}%
\pgfsetstrokecolor{currentstroke}%
\pgfsetdash{}{0pt}%
\pgfpathmoveto{\pgfqpoint{0.480841in}{0.402751in}}%
\pgfpathcurveto{\pgfqpoint{0.491891in}{0.402751in}}{\pgfqpoint{0.502490in}{0.407141in}}{\pgfqpoint{0.510303in}{0.414955in}}%
\pgfpathcurveto{\pgfqpoint{0.518117in}{0.422768in}}{\pgfqpoint{0.522507in}{0.433367in}}{\pgfqpoint{0.522507in}{0.444417in}}%
\pgfpathcurveto{\pgfqpoint{0.522507in}{0.455468in}}{\pgfqpoint{0.518117in}{0.466067in}}{\pgfqpoint{0.510303in}{0.473880in}}%
\pgfpathcurveto{\pgfqpoint{0.502490in}{0.481694in}}{\pgfqpoint{0.491891in}{0.486084in}}{\pgfqpoint{0.480841in}{0.486084in}}%
\pgfpathcurveto{\pgfqpoint{0.469790in}{0.486084in}}{\pgfqpoint{0.459191in}{0.481694in}}{\pgfqpoint{0.451378in}{0.473880in}}%
\pgfpathcurveto{\pgfqpoint{0.443564in}{0.466067in}}{\pgfqpoint{0.439174in}{0.455468in}}{\pgfqpoint{0.439174in}{0.444417in}}%
\pgfpathcurveto{\pgfqpoint{0.439174in}{0.433367in}}{\pgfqpoint{0.443564in}{0.422768in}}{\pgfqpoint{0.451378in}{0.414955in}}%
\pgfpathcurveto{\pgfqpoint{0.459191in}{0.407141in}}{\pgfqpoint{0.469790in}{0.402751in}}{\pgfqpoint{0.480841in}{0.402751in}}%
\pgfpathclose%
\pgfusepath{stroke,fill}%
\end{pgfscope}%
\begin{pgfscope}%
\pgfpathrectangle{\pgfqpoint{0.375000in}{0.330000in}}{\pgfqpoint{2.325000in}{2.310000in}}%
\pgfusepath{clip}%
\pgfsetbuttcap%
\pgfsetroundjoin%
\definecolor{currentfill}{rgb}{0.000000,0.000000,0.000000}%
\pgfsetfillcolor{currentfill}%
\pgfsetlinewidth{1.003750pt}%
\definecolor{currentstroke}{rgb}{0.000000,0.000000,0.000000}%
\pgfsetstrokecolor{currentstroke}%
\pgfsetdash{}{0pt}%
\pgfpathmoveto{\pgfqpoint{1.005694in}{1.438667in}}%
\pgfpathcurveto{\pgfqpoint{1.016744in}{1.438667in}}{\pgfqpoint{1.027343in}{1.443057in}}{\pgfqpoint{1.035157in}{1.450871in}}%
\pgfpathcurveto{\pgfqpoint{1.042971in}{1.458685in}}{\pgfqpoint{1.047361in}{1.469284in}}{\pgfqpoint{1.047361in}{1.480334in}}%
\pgfpathcurveto{\pgfqpoint{1.047361in}{1.491384in}}{\pgfqpoint{1.042971in}{1.501983in}}{\pgfqpoint{1.035157in}{1.509796in}}%
\pgfpathcurveto{\pgfqpoint{1.027343in}{1.517610in}}{\pgfqpoint{1.016744in}{1.522000in}}{\pgfqpoint{1.005694in}{1.522000in}}%
\pgfpathcurveto{\pgfqpoint{0.994644in}{1.522000in}}{\pgfqpoint{0.984045in}{1.517610in}}{\pgfqpoint{0.976232in}{1.509796in}}%
\pgfpathcurveto{\pgfqpoint{0.968418in}{1.501983in}}{\pgfqpoint{0.964028in}{1.491384in}}{\pgfqpoint{0.964028in}{1.480334in}}%
\pgfpathcurveto{\pgfqpoint{0.964028in}{1.469284in}}{\pgfqpoint{0.968418in}{1.458685in}}{\pgfqpoint{0.976232in}{1.450871in}}%
\pgfpathcurveto{\pgfqpoint{0.984045in}{1.443057in}}{\pgfqpoint{0.994644in}{1.438667in}}{\pgfqpoint{1.005694in}{1.438667in}}%
\pgfpathclose%
\pgfusepath{stroke,fill}%
\end{pgfscope}%
\begin{pgfscope}%
\pgfpathrectangle{\pgfqpoint{0.375000in}{0.330000in}}{\pgfqpoint{2.325000in}{2.310000in}}%
\pgfusepath{clip}%
\pgfsetbuttcap%
\pgfsetroundjoin%
\definecolor{currentfill}{rgb}{0.000000,0.000000,0.000000}%
\pgfsetfillcolor{currentfill}%
\pgfsetlinewidth{1.003750pt}%
\definecolor{currentstroke}{rgb}{0.000000,0.000000,0.000000}%
\pgfsetstrokecolor{currentstroke}%
\pgfsetdash{}{0pt}%
\pgfpathmoveto{\pgfqpoint{1.005694in}{1.438667in}}%
\pgfpathcurveto{\pgfqpoint{1.016744in}{1.438667in}}{\pgfqpoint{1.027343in}{1.443057in}}{\pgfqpoint{1.035157in}{1.450871in}}%
\pgfpathcurveto{\pgfqpoint{1.042971in}{1.458685in}}{\pgfqpoint{1.047361in}{1.469284in}}{\pgfqpoint{1.047361in}{1.480334in}}%
\pgfpathcurveto{\pgfqpoint{1.047361in}{1.491384in}}{\pgfqpoint{1.042971in}{1.501983in}}{\pgfqpoint{1.035157in}{1.509796in}}%
\pgfpathcurveto{\pgfqpoint{1.027343in}{1.517610in}}{\pgfqpoint{1.016744in}{1.522000in}}{\pgfqpoint{1.005694in}{1.522000in}}%
\pgfpathcurveto{\pgfqpoint{0.994644in}{1.522000in}}{\pgfqpoint{0.984045in}{1.517610in}}{\pgfqpoint{0.976232in}{1.509796in}}%
\pgfpathcurveto{\pgfqpoint{0.968418in}{1.501983in}}{\pgfqpoint{0.964028in}{1.491384in}}{\pgfqpoint{0.964028in}{1.480334in}}%
\pgfpathcurveto{\pgfqpoint{0.964028in}{1.469284in}}{\pgfqpoint{0.968418in}{1.458685in}}{\pgfqpoint{0.976232in}{1.450871in}}%
\pgfpathcurveto{\pgfqpoint{0.984045in}{1.443057in}}{\pgfqpoint{0.994644in}{1.438667in}}{\pgfqpoint{1.005694in}{1.438667in}}%
\pgfpathclose%
\pgfusepath{stroke,fill}%
\end{pgfscope}%
\begin{pgfscope}%
\pgfpathrectangle{\pgfqpoint{0.375000in}{0.330000in}}{\pgfqpoint{2.325000in}{2.310000in}}%
\pgfusepath{clip}%
\pgfsetbuttcap%
\pgfsetroundjoin%
\definecolor{currentfill}{rgb}{0.000000,0.000000,0.000000}%
\pgfsetfillcolor{currentfill}%
\pgfsetlinewidth{1.003750pt}%
\definecolor{currentstroke}{rgb}{0.000000,0.000000,0.000000}%
\pgfsetstrokecolor{currentstroke}%
\pgfsetdash{}{0pt}%
\pgfpathmoveto{\pgfqpoint{1.005694in}{1.438667in}}%
\pgfpathcurveto{\pgfqpoint{1.016744in}{1.438667in}}{\pgfqpoint{1.027343in}{1.443057in}}{\pgfqpoint{1.035157in}{1.450871in}}%
\pgfpathcurveto{\pgfqpoint{1.042971in}{1.458685in}}{\pgfqpoint{1.047361in}{1.469284in}}{\pgfqpoint{1.047361in}{1.480334in}}%
\pgfpathcurveto{\pgfqpoint{1.047361in}{1.491384in}}{\pgfqpoint{1.042971in}{1.501983in}}{\pgfqpoint{1.035157in}{1.509796in}}%
\pgfpathcurveto{\pgfqpoint{1.027343in}{1.517610in}}{\pgfqpoint{1.016744in}{1.522000in}}{\pgfqpoint{1.005694in}{1.522000in}}%
\pgfpathcurveto{\pgfqpoint{0.994644in}{1.522000in}}{\pgfqpoint{0.984045in}{1.517610in}}{\pgfqpoint{0.976232in}{1.509796in}}%
\pgfpathcurveto{\pgfqpoint{0.968418in}{1.501983in}}{\pgfqpoint{0.964028in}{1.491384in}}{\pgfqpoint{0.964028in}{1.480334in}}%
\pgfpathcurveto{\pgfqpoint{0.964028in}{1.469284in}}{\pgfqpoint{0.968418in}{1.458685in}}{\pgfqpoint{0.976232in}{1.450871in}}%
\pgfpathcurveto{\pgfqpoint{0.984045in}{1.443057in}}{\pgfqpoint{0.994644in}{1.438667in}}{\pgfqpoint{1.005694in}{1.438667in}}%
\pgfpathclose%
\pgfusepath{stroke,fill}%
\end{pgfscope}%
\begin{pgfscope}%
\pgfpathrectangle{\pgfqpoint{0.375000in}{0.330000in}}{\pgfqpoint{2.325000in}{2.310000in}}%
\pgfusepath{clip}%
\pgfsetbuttcap%
\pgfsetroundjoin%
\definecolor{currentfill}{rgb}{0.000000,0.000000,0.000000}%
\pgfsetfillcolor{currentfill}%
\pgfsetlinewidth{1.003750pt}%
\definecolor{currentstroke}{rgb}{0.000000,0.000000,0.000000}%
\pgfsetstrokecolor{currentstroke}%
\pgfsetdash{}{0pt}%
\pgfpathmoveto{\pgfqpoint{1.005694in}{1.438667in}}%
\pgfpathcurveto{\pgfqpoint{1.016744in}{1.438667in}}{\pgfqpoint{1.027343in}{1.443057in}}{\pgfqpoint{1.035157in}{1.450871in}}%
\pgfpathcurveto{\pgfqpoint{1.042971in}{1.458685in}}{\pgfqpoint{1.047361in}{1.469284in}}{\pgfqpoint{1.047361in}{1.480334in}}%
\pgfpathcurveto{\pgfqpoint{1.047361in}{1.491384in}}{\pgfqpoint{1.042971in}{1.501983in}}{\pgfqpoint{1.035157in}{1.509796in}}%
\pgfpathcurveto{\pgfqpoint{1.027343in}{1.517610in}}{\pgfqpoint{1.016744in}{1.522000in}}{\pgfqpoint{1.005694in}{1.522000in}}%
\pgfpathcurveto{\pgfqpoint{0.994644in}{1.522000in}}{\pgfqpoint{0.984045in}{1.517610in}}{\pgfqpoint{0.976232in}{1.509796in}}%
\pgfpathcurveto{\pgfqpoint{0.968418in}{1.501983in}}{\pgfqpoint{0.964028in}{1.491384in}}{\pgfqpoint{0.964028in}{1.480334in}}%
\pgfpathcurveto{\pgfqpoint{0.964028in}{1.469284in}}{\pgfqpoint{0.968418in}{1.458685in}}{\pgfqpoint{0.976232in}{1.450871in}}%
\pgfpathcurveto{\pgfqpoint{0.984045in}{1.443057in}}{\pgfqpoint{0.994644in}{1.438667in}}{\pgfqpoint{1.005694in}{1.438667in}}%
\pgfpathclose%
\pgfusepath{stroke,fill}%
\end{pgfscope}%
\begin{pgfscope}%
\pgfpathrectangle{\pgfqpoint{0.375000in}{0.330000in}}{\pgfqpoint{2.325000in}{2.310000in}}%
\pgfusepath{clip}%
\pgfsetbuttcap%
\pgfsetroundjoin%
\definecolor{currentfill}{rgb}{0.000000,0.000000,0.000000}%
\pgfsetfillcolor{currentfill}%
\pgfsetlinewidth{1.003750pt}%
\definecolor{currentstroke}{rgb}{0.000000,0.000000,0.000000}%
\pgfsetstrokecolor{currentstroke}%
\pgfsetdash{}{0pt}%
\pgfpathmoveto{\pgfqpoint{1.005694in}{1.438667in}}%
\pgfpathcurveto{\pgfqpoint{1.016744in}{1.438667in}}{\pgfqpoint{1.027343in}{1.443057in}}{\pgfqpoint{1.035157in}{1.450871in}}%
\pgfpathcurveto{\pgfqpoint{1.042971in}{1.458685in}}{\pgfqpoint{1.047361in}{1.469284in}}{\pgfqpoint{1.047361in}{1.480334in}}%
\pgfpathcurveto{\pgfqpoint{1.047361in}{1.491384in}}{\pgfqpoint{1.042971in}{1.501983in}}{\pgfqpoint{1.035157in}{1.509796in}}%
\pgfpathcurveto{\pgfqpoint{1.027343in}{1.517610in}}{\pgfqpoint{1.016744in}{1.522000in}}{\pgfqpoint{1.005694in}{1.522000in}}%
\pgfpathcurveto{\pgfqpoint{0.994644in}{1.522000in}}{\pgfqpoint{0.984045in}{1.517610in}}{\pgfqpoint{0.976232in}{1.509796in}}%
\pgfpathcurveto{\pgfqpoint{0.968418in}{1.501983in}}{\pgfqpoint{0.964028in}{1.491384in}}{\pgfqpoint{0.964028in}{1.480334in}}%
\pgfpathcurveto{\pgfqpoint{0.964028in}{1.469284in}}{\pgfqpoint{0.968418in}{1.458685in}}{\pgfqpoint{0.976232in}{1.450871in}}%
\pgfpathcurveto{\pgfqpoint{0.984045in}{1.443057in}}{\pgfqpoint{0.994644in}{1.438667in}}{\pgfqpoint{1.005694in}{1.438667in}}%
\pgfpathclose%
\pgfusepath{stroke,fill}%
\end{pgfscope}%
\begin{pgfscope}%
\pgfpathrectangle{\pgfqpoint{0.375000in}{0.330000in}}{\pgfqpoint{2.325000in}{2.310000in}}%
\pgfusepath{clip}%
\pgfsetbuttcap%
\pgfsetroundjoin%
\definecolor{currentfill}{rgb}{0.000000,0.000000,0.000000}%
\pgfsetfillcolor{currentfill}%
\pgfsetlinewidth{1.003750pt}%
\definecolor{currentstroke}{rgb}{0.000000,0.000000,0.000000}%
\pgfsetstrokecolor{currentstroke}%
\pgfsetdash{}{0pt}%
\pgfpathmoveto{\pgfqpoint{1.005694in}{1.438667in}}%
\pgfpathcurveto{\pgfqpoint{1.016744in}{1.438667in}}{\pgfqpoint{1.027343in}{1.443057in}}{\pgfqpoint{1.035157in}{1.450871in}}%
\pgfpathcurveto{\pgfqpoint{1.042971in}{1.458685in}}{\pgfqpoint{1.047361in}{1.469284in}}{\pgfqpoint{1.047361in}{1.480334in}}%
\pgfpathcurveto{\pgfqpoint{1.047361in}{1.491384in}}{\pgfqpoint{1.042971in}{1.501983in}}{\pgfqpoint{1.035157in}{1.509796in}}%
\pgfpathcurveto{\pgfqpoint{1.027343in}{1.517610in}}{\pgfqpoint{1.016744in}{1.522000in}}{\pgfqpoint{1.005694in}{1.522000in}}%
\pgfpathcurveto{\pgfqpoint{0.994644in}{1.522000in}}{\pgfqpoint{0.984045in}{1.517610in}}{\pgfqpoint{0.976232in}{1.509796in}}%
\pgfpathcurveto{\pgfqpoint{0.968418in}{1.501983in}}{\pgfqpoint{0.964028in}{1.491384in}}{\pgfqpoint{0.964028in}{1.480334in}}%
\pgfpathcurveto{\pgfqpoint{0.964028in}{1.469284in}}{\pgfqpoint{0.968418in}{1.458685in}}{\pgfqpoint{0.976232in}{1.450871in}}%
\pgfpathcurveto{\pgfqpoint{0.984045in}{1.443057in}}{\pgfqpoint{0.994644in}{1.438667in}}{\pgfqpoint{1.005694in}{1.438667in}}%
\pgfpathclose%
\pgfusepath{stroke,fill}%
\end{pgfscope}%
\begin{pgfscope}%
\pgfpathrectangle{\pgfqpoint{0.375000in}{0.330000in}}{\pgfqpoint{2.325000in}{2.310000in}}%
\pgfusepath{clip}%
\pgfsetbuttcap%
\pgfsetroundjoin%
\definecolor{currentfill}{rgb}{0.000000,0.000000,0.000000}%
\pgfsetfillcolor{currentfill}%
\pgfsetlinewidth{1.003750pt}%
\definecolor{currentstroke}{rgb}{0.000000,0.000000,0.000000}%
\pgfsetstrokecolor{currentstroke}%
\pgfsetdash{}{0pt}%
\pgfpathmoveto{\pgfqpoint{1.005694in}{1.438667in}}%
\pgfpathcurveto{\pgfqpoint{1.016744in}{1.438667in}}{\pgfqpoint{1.027343in}{1.443057in}}{\pgfqpoint{1.035157in}{1.450871in}}%
\pgfpathcurveto{\pgfqpoint{1.042971in}{1.458685in}}{\pgfqpoint{1.047361in}{1.469284in}}{\pgfqpoint{1.047361in}{1.480334in}}%
\pgfpathcurveto{\pgfqpoint{1.047361in}{1.491384in}}{\pgfqpoint{1.042971in}{1.501983in}}{\pgfqpoint{1.035157in}{1.509796in}}%
\pgfpathcurveto{\pgfqpoint{1.027343in}{1.517610in}}{\pgfqpoint{1.016744in}{1.522000in}}{\pgfqpoint{1.005694in}{1.522000in}}%
\pgfpathcurveto{\pgfqpoint{0.994644in}{1.522000in}}{\pgfqpoint{0.984045in}{1.517610in}}{\pgfqpoint{0.976232in}{1.509796in}}%
\pgfpathcurveto{\pgfqpoint{0.968418in}{1.501983in}}{\pgfqpoint{0.964028in}{1.491384in}}{\pgfqpoint{0.964028in}{1.480334in}}%
\pgfpathcurveto{\pgfqpoint{0.964028in}{1.469284in}}{\pgfqpoint{0.968418in}{1.458685in}}{\pgfqpoint{0.976232in}{1.450871in}}%
\pgfpathcurveto{\pgfqpoint{0.984045in}{1.443057in}}{\pgfqpoint{0.994644in}{1.438667in}}{\pgfqpoint{1.005694in}{1.438667in}}%
\pgfpathclose%
\pgfusepath{stroke,fill}%
\end{pgfscope}%
\begin{pgfscope}%
\pgfpathrectangle{\pgfqpoint{0.375000in}{0.330000in}}{\pgfqpoint{2.325000in}{2.310000in}}%
\pgfusepath{clip}%
\pgfsetbuttcap%
\pgfsetroundjoin%
\definecolor{currentfill}{rgb}{0.000000,0.000000,0.000000}%
\pgfsetfillcolor{currentfill}%
\pgfsetlinewidth{1.003750pt}%
\definecolor{currentstroke}{rgb}{0.000000,0.000000,0.000000}%
\pgfsetstrokecolor{currentstroke}%
\pgfsetdash{}{0pt}%
\pgfpathmoveto{\pgfqpoint{1.005694in}{1.438667in}}%
\pgfpathcurveto{\pgfqpoint{1.016744in}{1.438667in}}{\pgfqpoint{1.027343in}{1.443057in}}{\pgfqpoint{1.035157in}{1.450871in}}%
\pgfpathcurveto{\pgfqpoint{1.042971in}{1.458685in}}{\pgfqpoint{1.047361in}{1.469284in}}{\pgfqpoint{1.047361in}{1.480334in}}%
\pgfpathcurveto{\pgfqpoint{1.047361in}{1.491384in}}{\pgfqpoint{1.042971in}{1.501983in}}{\pgfqpoint{1.035157in}{1.509796in}}%
\pgfpathcurveto{\pgfqpoint{1.027343in}{1.517610in}}{\pgfqpoint{1.016744in}{1.522000in}}{\pgfqpoint{1.005694in}{1.522000in}}%
\pgfpathcurveto{\pgfqpoint{0.994644in}{1.522000in}}{\pgfqpoint{0.984045in}{1.517610in}}{\pgfqpoint{0.976232in}{1.509796in}}%
\pgfpathcurveto{\pgfqpoint{0.968418in}{1.501983in}}{\pgfqpoint{0.964028in}{1.491384in}}{\pgfqpoint{0.964028in}{1.480334in}}%
\pgfpathcurveto{\pgfqpoint{0.964028in}{1.469284in}}{\pgfqpoint{0.968418in}{1.458685in}}{\pgfqpoint{0.976232in}{1.450871in}}%
\pgfpathcurveto{\pgfqpoint{0.984045in}{1.443057in}}{\pgfqpoint{0.994644in}{1.438667in}}{\pgfqpoint{1.005694in}{1.438667in}}%
\pgfpathclose%
\pgfusepath{stroke,fill}%
\end{pgfscope}%
\begin{pgfscope}%
\pgfpathrectangle{\pgfqpoint{0.375000in}{0.330000in}}{\pgfqpoint{2.325000in}{2.310000in}}%
\pgfusepath{clip}%
\pgfsetbuttcap%
\pgfsetroundjoin%
\definecolor{currentfill}{rgb}{0.000000,0.000000,0.000000}%
\pgfsetfillcolor{currentfill}%
\pgfsetlinewidth{1.003750pt}%
\definecolor{currentstroke}{rgb}{0.000000,0.000000,0.000000}%
\pgfsetstrokecolor{currentstroke}%
\pgfsetdash{}{0pt}%
\pgfpathmoveto{\pgfqpoint{1.005694in}{1.438667in}}%
\pgfpathcurveto{\pgfqpoint{1.016744in}{1.438667in}}{\pgfqpoint{1.027343in}{1.443057in}}{\pgfqpoint{1.035157in}{1.450871in}}%
\pgfpathcurveto{\pgfqpoint{1.042971in}{1.458685in}}{\pgfqpoint{1.047361in}{1.469284in}}{\pgfqpoint{1.047361in}{1.480334in}}%
\pgfpathcurveto{\pgfqpoint{1.047361in}{1.491384in}}{\pgfqpoint{1.042971in}{1.501983in}}{\pgfqpoint{1.035157in}{1.509796in}}%
\pgfpathcurveto{\pgfqpoint{1.027343in}{1.517610in}}{\pgfqpoint{1.016744in}{1.522000in}}{\pgfqpoint{1.005694in}{1.522000in}}%
\pgfpathcurveto{\pgfqpoint{0.994644in}{1.522000in}}{\pgfqpoint{0.984045in}{1.517610in}}{\pgfqpoint{0.976232in}{1.509796in}}%
\pgfpathcurveto{\pgfqpoint{0.968418in}{1.501983in}}{\pgfqpoint{0.964028in}{1.491384in}}{\pgfqpoint{0.964028in}{1.480334in}}%
\pgfpathcurveto{\pgfqpoint{0.964028in}{1.469284in}}{\pgfqpoint{0.968418in}{1.458685in}}{\pgfqpoint{0.976232in}{1.450871in}}%
\pgfpathcurveto{\pgfqpoint{0.984045in}{1.443057in}}{\pgfqpoint{0.994644in}{1.438667in}}{\pgfqpoint{1.005694in}{1.438667in}}%
\pgfpathclose%
\pgfusepath{stroke,fill}%
\end{pgfscope}%
\begin{pgfscope}%
\pgfpathrectangle{\pgfqpoint{0.375000in}{0.330000in}}{\pgfqpoint{2.325000in}{2.310000in}}%
\pgfusepath{clip}%
\pgfsetbuttcap%
\pgfsetroundjoin%
\definecolor{currentfill}{rgb}{0.000000,0.000000,0.000000}%
\pgfsetfillcolor{currentfill}%
\pgfsetlinewidth{1.003750pt}%
\definecolor{currentstroke}{rgb}{0.000000,0.000000,0.000000}%
\pgfsetstrokecolor{currentstroke}%
\pgfsetdash{}{0pt}%
\pgfpathmoveto{\pgfqpoint{1.005694in}{1.438667in}}%
\pgfpathcurveto{\pgfqpoint{1.016744in}{1.438667in}}{\pgfqpoint{1.027343in}{1.443057in}}{\pgfqpoint{1.035157in}{1.450871in}}%
\pgfpathcurveto{\pgfqpoint{1.042971in}{1.458685in}}{\pgfqpoint{1.047361in}{1.469284in}}{\pgfqpoint{1.047361in}{1.480334in}}%
\pgfpathcurveto{\pgfqpoint{1.047361in}{1.491384in}}{\pgfqpoint{1.042971in}{1.501983in}}{\pgfqpoint{1.035157in}{1.509796in}}%
\pgfpathcurveto{\pgfqpoint{1.027343in}{1.517610in}}{\pgfqpoint{1.016744in}{1.522000in}}{\pgfqpoint{1.005694in}{1.522000in}}%
\pgfpathcurveto{\pgfqpoint{0.994644in}{1.522000in}}{\pgfqpoint{0.984045in}{1.517610in}}{\pgfqpoint{0.976232in}{1.509796in}}%
\pgfpathcurveto{\pgfqpoint{0.968418in}{1.501983in}}{\pgfqpoint{0.964028in}{1.491384in}}{\pgfqpoint{0.964028in}{1.480334in}}%
\pgfpathcurveto{\pgfqpoint{0.964028in}{1.469284in}}{\pgfqpoint{0.968418in}{1.458685in}}{\pgfqpoint{0.976232in}{1.450871in}}%
\pgfpathcurveto{\pgfqpoint{0.984045in}{1.443057in}}{\pgfqpoint{0.994644in}{1.438667in}}{\pgfqpoint{1.005694in}{1.438667in}}%
\pgfpathclose%
\pgfusepath{stroke,fill}%
\end{pgfscope}%
\begin{pgfscope}%
\pgfpathrectangle{\pgfqpoint{0.375000in}{0.330000in}}{\pgfqpoint{2.325000in}{2.310000in}}%
\pgfusepath{clip}%
\pgfsetbuttcap%
\pgfsetroundjoin%
\definecolor{currentfill}{rgb}{0.000000,0.000000,0.000000}%
\pgfsetfillcolor{currentfill}%
\pgfsetlinewidth{1.003750pt}%
\definecolor{currentstroke}{rgb}{0.000000,0.000000,0.000000}%
\pgfsetstrokecolor{currentstroke}%
\pgfsetdash{}{0pt}%
\pgfpathmoveto{\pgfqpoint{1.005694in}{1.438667in}}%
\pgfpathcurveto{\pgfqpoint{1.016744in}{1.438667in}}{\pgfqpoint{1.027343in}{1.443057in}}{\pgfqpoint{1.035157in}{1.450871in}}%
\pgfpathcurveto{\pgfqpoint{1.042971in}{1.458685in}}{\pgfqpoint{1.047361in}{1.469284in}}{\pgfqpoint{1.047361in}{1.480334in}}%
\pgfpathcurveto{\pgfqpoint{1.047361in}{1.491384in}}{\pgfqpoint{1.042971in}{1.501983in}}{\pgfqpoint{1.035157in}{1.509796in}}%
\pgfpathcurveto{\pgfqpoint{1.027343in}{1.517610in}}{\pgfqpoint{1.016744in}{1.522000in}}{\pgfqpoint{1.005694in}{1.522000in}}%
\pgfpathcurveto{\pgfqpoint{0.994644in}{1.522000in}}{\pgfqpoint{0.984045in}{1.517610in}}{\pgfqpoint{0.976232in}{1.509796in}}%
\pgfpathcurveto{\pgfqpoint{0.968418in}{1.501983in}}{\pgfqpoint{0.964028in}{1.491384in}}{\pgfqpoint{0.964028in}{1.480334in}}%
\pgfpathcurveto{\pgfqpoint{0.964028in}{1.469284in}}{\pgfqpoint{0.968418in}{1.458685in}}{\pgfqpoint{0.976232in}{1.450871in}}%
\pgfpathcurveto{\pgfqpoint{0.984045in}{1.443057in}}{\pgfqpoint{0.994644in}{1.438667in}}{\pgfqpoint{1.005694in}{1.438667in}}%
\pgfpathclose%
\pgfusepath{stroke,fill}%
\end{pgfscope}%
\begin{pgfscope}%
\pgfpathrectangle{\pgfqpoint{0.375000in}{0.330000in}}{\pgfqpoint{2.325000in}{2.310000in}}%
\pgfusepath{clip}%
\pgfsetbuttcap%
\pgfsetroundjoin%
\definecolor{currentfill}{rgb}{0.000000,0.000000,0.000000}%
\pgfsetfillcolor{currentfill}%
\pgfsetlinewidth{1.003750pt}%
\definecolor{currentstroke}{rgb}{0.000000,0.000000,0.000000}%
\pgfsetstrokecolor{currentstroke}%
\pgfsetdash{}{0pt}%
\pgfpathmoveto{\pgfqpoint{1.005694in}{1.438667in}}%
\pgfpathcurveto{\pgfqpoint{1.016744in}{1.438667in}}{\pgfqpoint{1.027343in}{1.443057in}}{\pgfqpoint{1.035157in}{1.450871in}}%
\pgfpathcurveto{\pgfqpoint{1.042971in}{1.458685in}}{\pgfqpoint{1.047361in}{1.469284in}}{\pgfqpoint{1.047361in}{1.480334in}}%
\pgfpathcurveto{\pgfqpoint{1.047361in}{1.491384in}}{\pgfqpoint{1.042971in}{1.501983in}}{\pgfqpoint{1.035157in}{1.509796in}}%
\pgfpathcurveto{\pgfqpoint{1.027343in}{1.517610in}}{\pgfqpoint{1.016744in}{1.522000in}}{\pgfqpoint{1.005694in}{1.522000in}}%
\pgfpathcurveto{\pgfqpoint{0.994644in}{1.522000in}}{\pgfqpoint{0.984045in}{1.517610in}}{\pgfqpoint{0.976232in}{1.509796in}}%
\pgfpathcurveto{\pgfqpoint{0.968418in}{1.501983in}}{\pgfqpoint{0.964028in}{1.491384in}}{\pgfqpoint{0.964028in}{1.480334in}}%
\pgfpathcurveto{\pgfqpoint{0.964028in}{1.469284in}}{\pgfqpoint{0.968418in}{1.458685in}}{\pgfqpoint{0.976232in}{1.450871in}}%
\pgfpathcurveto{\pgfqpoint{0.984045in}{1.443057in}}{\pgfqpoint{0.994644in}{1.438667in}}{\pgfqpoint{1.005694in}{1.438667in}}%
\pgfpathclose%
\pgfusepath{stroke,fill}%
\end{pgfscope}%
\begin{pgfscope}%
\pgfpathrectangle{\pgfqpoint{0.375000in}{0.330000in}}{\pgfqpoint{2.325000in}{2.310000in}}%
\pgfusepath{clip}%
\pgfsetbuttcap%
\pgfsetroundjoin%
\definecolor{currentfill}{rgb}{0.000000,0.000000,0.000000}%
\pgfsetfillcolor{currentfill}%
\pgfsetlinewidth{1.003750pt}%
\definecolor{currentstroke}{rgb}{0.000000,0.000000,0.000000}%
\pgfsetstrokecolor{currentstroke}%
\pgfsetdash{}{0pt}%
\pgfpathmoveto{\pgfqpoint{1.005694in}{1.438667in}}%
\pgfpathcurveto{\pgfqpoint{1.016744in}{1.438667in}}{\pgfqpoint{1.027343in}{1.443057in}}{\pgfqpoint{1.035157in}{1.450871in}}%
\pgfpathcurveto{\pgfqpoint{1.042971in}{1.458685in}}{\pgfqpoint{1.047361in}{1.469284in}}{\pgfqpoint{1.047361in}{1.480334in}}%
\pgfpathcurveto{\pgfqpoint{1.047361in}{1.491384in}}{\pgfqpoint{1.042971in}{1.501983in}}{\pgfqpoint{1.035157in}{1.509796in}}%
\pgfpathcurveto{\pgfqpoint{1.027343in}{1.517610in}}{\pgfqpoint{1.016744in}{1.522000in}}{\pgfqpoint{1.005694in}{1.522000in}}%
\pgfpathcurveto{\pgfqpoint{0.994644in}{1.522000in}}{\pgfqpoint{0.984045in}{1.517610in}}{\pgfqpoint{0.976232in}{1.509796in}}%
\pgfpathcurveto{\pgfqpoint{0.968418in}{1.501983in}}{\pgfqpoint{0.964028in}{1.491384in}}{\pgfqpoint{0.964028in}{1.480334in}}%
\pgfpathcurveto{\pgfqpoint{0.964028in}{1.469284in}}{\pgfqpoint{0.968418in}{1.458685in}}{\pgfqpoint{0.976232in}{1.450871in}}%
\pgfpathcurveto{\pgfqpoint{0.984045in}{1.443057in}}{\pgfqpoint{0.994644in}{1.438667in}}{\pgfqpoint{1.005694in}{1.438667in}}%
\pgfpathclose%
\pgfusepath{stroke,fill}%
\end{pgfscope}%
\begin{pgfscope}%
\pgfpathrectangle{\pgfqpoint{0.375000in}{0.330000in}}{\pgfqpoint{2.325000in}{2.310000in}}%
\pgfusepath{clip}%
\pgfsetbuttcap%
\pgfsetroundjoin%
\definecolor{currentfill}{rgb}{0.000000,0.000000,0.000000}%
\pgfsetfillcolor{currentfill}%
\pgfsetlinewidth{1.003750pt}%
\definecolor{currentstroke}{rgb}{0.000000,0.000000,0.000000}%
\pgfsetstrokecolor{currentstroke}%
\pgfsetdash{}{0pt}%
\pgfpathmoveto{\pgfqpoint{1.005694in}{1.438667in}}%
\pgfpathcurveto{\pgfqpoint{1.016744in}{1.438667in}}{\pgfqpoint{1.027343in}{1.443057in}}{\pgfqpoint{1.035157in}{1.450871in}}%
\pgfpathcurveto{\pgfqpoint{1.042971in}{1.458685in}}{\pgfqpoint{1.047361in}{1.469284in}}{\pgfqpoint{1.047361in}{1.480334in}}%
\pgfpathcurveto{\pgfqpoint{1.047361in}{1.491384in}}{\pgfqpoint{1.042971in}{1.501983in}}{\pgfqpoint{1.035157in}{1.509796in}}%
\pgfpathcurveto{\pgfqpoint{1.027343in}{1.517610in}}{\pgfqpoint{1.016744in}{1.522000in}}{\pgfqpoint{1.005694in}{1.522000in}}%
\pgfpathcurveto{\pgfqpoint{0.994644in}{1.522000in}}{\pgfqpoint{0.984045in}{1.517610in}}{\pgfqpoint{0.976232in}{1.509796in}}%
\pgfpathcurveto{\pgfqpoint{0.968418in}{1.501983in}}{\pgfqpoint{0.964028in}{1.491384in}}{\pgfqpoint{0.964028in}{1.480334in}}%
\pgfpathcurveto{\pgfqpoint{0.964028in}{1.469284in}}{\pgfqpoint{0.968418in}{1.458685in}}{\pgfqpoint{0.976232in}{1.450871in}}%
\pgfpathcurveto{\pgfqpoint{0.984045in}{1.443057in}}{\pgfqpoint{0.994644in}{1.438667in}}{\pgfqpoint{1.005694in}{1.438667in}}%
\pgfpathclose%
\pgfusepath{stroke,fill}%
\end{pgfscope}%
\begin{pgfscope}%
\pgfpathrectangle{\pgfqpoint{0.375000in}{0.330000in}}{\pgfqpoint{2.325000in}{2.310000in}}%
\pgfusepath{clip}%
\pgfsetbuttcap%
\pgfsetroundjoin%
\definecolor{currentfill}{rgb}{0.000000,0.000000,0.000000}%
\pgfsetfillcolor{currentfill}%
\pgfsetlinewidth{1.003750pt}%
\definecolor{currentstroke}{rgb}{0.000000,0.000000,0.000000}%
\pgfsetstrokecolor{currentstroke}%
\pgfsetdash{}{0pt}%
\pgfpathmoveto{\pgfqpoint{1.005694in}{1.438667in}}%
\pgfpathcurveto{\pgfqpoint{1.016744in}{1.438667in}}{\pgfqpoint{1.027343in}{1.443057in}}{\pgfqpoint{1.035157in}{1.450871in}}%
\pgfpathcurveto{\pgfqpoint{1.042971in}{1.458685in}}{\pgfqpoint{1.047361in}{1.469284in}}{\pgfqpoint{1.047361in}{1.480334in}}%
\pgfpathcurveto{\pgfqpoint{1.047361in}{1.491384in}}{\pgfqpoint{1.042971in}{1.501983in}}{\pgfqpoint{1.035157in}{1.509796in}}%
\pgfpathcurveto{\pgfqpoint{1.027343in}{1.517610in}}{\pgfqpoint{1.016744in}{1.522000in}}{\pgfqpoint{1.005694in}{1.522000in}}%
\pgfpathcurveto{\pgfqpoint{0.994644in}{1.522000in}}{\pgfqpoint{0.984045in}{1.517610in}}{\pgfqpoint{0.976232in}{1.509796in}}%
\pgfpathcurveto{\pgfqpoint{0.968418in}{1.501983in}}{\pgfqpoint{0.964028in}{1.491384in}}{\pgfqpoint{0.964028in}{1.480334in}}%
\pgfpathcurveto{\pgfqpoint{0.964028in}{1.469284in}}{\pgfqpoint{0.968418in}{1.458685in}}{\pgfqpoint{0.976232in}{1.450871in}}%
\pgfpathcurveto{\pgfqpoint{0.984045in}{1.443057in}}{\pgfqpoint{0.994644in}{1.438667in}}{\pgfqpoint{1.005694in}{1.438667in}}%
\pgfpathclose%
\pgfusepath{stroke,fill}%
\end{pgfscope}%
\begin{pgfscope}%
\pgfpathrectangle{\pgfqpoint{0.375000in}{0.330000in}}{\pgfqpoint{2.325000in}{2.310000in}}%
\pgfusepath{clip}%
\pgfsetbuttcap%
\pgfsetroundjoin%
\definecolor{currentfill}{rgb}{0.000000,0.000000,0.000000}%
\pgfsetfillcolor{currentfill}%
\pgfsetlinewidth{1.003750pt}%
\definecolor{currentstroke}{rgb}{0.000000,0.000000,0.000000}%
\pgfsetstrokecolor{currentstroke}%
\pgfsetdash{}{0pt}%
\pgfpathmoveto{\pgfqpoint{1.005694in}{1.438667in}}%
\pgfpathcurveto{\pgfqpoint{1.016744in}{1.438667in}}{\pgfqpoint{1.027343in}{1.443057in}}{\pgfqpoint{1.035157in}{1.450871in}}%
\pgfpathcurveto{\pgfqpoint{1.042971in}{1.458685in}}{\pgfqpoint{1.047361in}{1.469284in}}{\pgfqpoint{1.047361in}{1.480334in}}%
\pgfpathcurveto{\pgfqpoint{1.047361in}{1.491384in}}{\pgfqpoint{1.042971in}{1.501983in}}{\pgfqpoint{1.035157in}{1.509796in}}%
\pgfpathcurveto{\pgfqpoint{1.027343in}{1.517610in}}{\pgfqpoint{1.016744in}{1.522000in}}{\pgfqpoint{1.005694in}{1.522000in}}%
\pgfpathcurveto{\pgfqpoint{0.994644in}{1.522000in}}{\pgfqpoint{0.984045in}{1.517610in}}{\pgfqpoint{0.976232in}{1.509796in}}%
\pgfpathcurveto{\pgfqpoint{0.968418in}{1.501983in}}{\pgfqpoint{0.964028in}{1.491384in}}{\pgfqpoint{0.964028in}{1.480334in}}%
\pgfpathcurveto{\pgfqpoint{0.964028in}{1.469284in}}{\pgfqpoint{0.968418in}{1.458685in}}{\pgfqpoint{0.976232in}{1.450871in}}%
\pgfpathcurveto{\pgfqpoint{0.984045in}{1.443057in}}{\pgfqpoint{0.994644in}{1.438667in}}{\pgfqpoint{1.005694in}{1.438667in}}%
\pgfpathclose%
\pgfusepath{stroke,fill}%
\end{pgfscope}%
\begin{pgfscope}%
\pgfpathrectangle{\pgfqpoint{0.375000in}{0.330000in}}{\pgfqpoint{2.325000in}{2.310000in}}%
\pgfusepath{clip}%
\pgfsetbuttcap%
\pgfsetroundjoin%
\definecolor{currentfill}{rgb}{0.000000,0.000000,0.000000}%
\pgfsetfillcolor{currentfill}%
\pgfsetlinewidth{1.003750pt}%
\definecolor{currentstroke}{rgb}{0.000000,0.000000,0.000000}%
\pgfsetstrokecolor{currentstroke}%
\pgfsetdash{}{0pt}%
\pgfpathmoveto{\pgfqpoint{1.005694in}{1.438667in}}%
\pgfpathcurveto{\pgfqpoint{1.016744in}{1.438667in}}{\pgfqpoint{1.027343in}{1.443057in}}{\pgfqpoint{1.035157in}{1.450871in}}%
\pgfpathcurveto{\pgfqpoint{1.042971in}{1.458685in}}{\pgfqpoint{1.047361in}{1.469284in}}{\pgfqpoint{1.047361in}{1.480334in}}%
\pgfpathcurveto{\pgfqpoint{1.047361in}{1.491384in}}{\pgfqpoint{1.042971in}{1.501983in}}{\pgfqpoint{1.035157in}{1.509796in}}%
\pgfpathcurveto{\pgfqpoint{1.027343in}{1.517610in}}{\pgfqpoint{1.016744in}{1.522000in}}{\pgfqpoint{1.005694in}{1.522000in}}%
\pgfpathcurveto{\pgfqpoint{0.994644in}{1.522000in}}{\pgfqpoint{0.984045in}{1.517610in}}{\pgfqpoint{0.976232in}{1.509796in}}%
\pgfpathcurveto{\pgfqpoint{0.968418in}{1.501983in}}{\pgfqpoint{0.964028in}{1.491384in}}{\pgfqpoint{0.964028in}{1.480334in}}%
\pgfpathcurveto{\pgfqpoint{0.964028in}{1.469284in}}{\pgfqpoint{0.968418in}{1.458685in}}{\pgfqpoint{0.976232in}{1.450871in}}%
\pgfpathcurveto{\pgfqpoint{0.984045in}{1.443057in}}{\pgfqpoint{0.994644in}{1.438667in}}{\pgfqpoint{1.005694in}{1.438667in}}%
\pgfpathclose%
\pgfusepath{stroke,fill}%
\end{pgfscope}%
\begin{pgfscope}%
\pgfpathrectangle{\pgfqpoint{0.375000in}{0.330000in}}{\pgfqpoint{2.325000in}{2.310000in}}%
\pgfusepath{clip}%
\pgfsetbuttcap%
\pgfsetroundjoin%
\definecolor{currentfill}{rgb}{0.000000,0.000000,0.000000}%
\pgfsetfillcolor{currentfill}%
\pgfsetlinewidth{1.003750pt}%
\definecolor{currentstroke}{rgb}{0.000000,0.000000,0.000000}%
\pgfsetstrokecolor{currentstroke}%
\pgfsetdash{}{0pt}%
\pgfpathmoveto{\pgfqpoint{1.005694in}{1.438667in}}%
\pgfpathcurveto{\pgfqpoint{1.016744in}{1.438667in}}{\pgfqpoint{1.027343in}{1.443057in}}{\pgfqpoint{1.035157in}{1.450871in}}%
\pgfpathcurveto{\pgfqpoint{1.042971in}{1.458685in}}{\pgfqpoint{1.047361in}{1.469284in}}{\pgfqpoint{1.047361in}{1.480334in}}%
\pgfpathcurveto{\pgfqpoint{1.047361in}{1.491384in}}{\pgfqpoint{1.042971in}{1.501983in}}{\pgfqpoint{1.035157in}{1.509796in}}%
\pgfpathcurveto{\pgfqpoint{1.027343in}{1.517610in}}{\pgfqpoint{1.016744in}{1.522000in}}{\pgfqpoint{1.005694in}{1.522000in}}%
\pgfpathcurveto{\pgfqpoint{0.994644in}{1.522000in}}{\pgfqpoint{0.984045in}{1.517610in}}{\pgfqpoint{0.976232in}{1.509796in}}%
\pgfpathcurveto{\pgfqpoint{0.968418in}{1.501983in}}{\pgfqpoint{0.964028in}{1.491384in}}{\pgfqpoint{0.964028in}{1.480334in}}%
\pgfpathcurveto{\pgfqpoint{0.964028in}{1.469284in}}{\pgfqpoint{0.968418in}{1.458685in}}{\pgfqpoint{0.976232in}{1.450871in}}%
\pgfpathcurveto{\pgfqpoint{0.984045in}{1.443057in}}{\pgfqpoint{0.994644in}{1.438667in}}{\pgfqpoint{1.005694in}{1.438667in}}%
\pgfpathclose%
\pgfusepath{stroke,fill}%
\end{pgfscope}%
\begin{pgfscope}%
\pgfpathrectangle{\pgfqpoint{0.375000in}{0.330000in}}{\pgfqpoint{2.325000in}{2.310000in}}%
\pgfusepath{clip}%
\pgfsetbuttcap%
\pgfsetroundjoin%
\definecolor{currentfill}{rgb}{0.000000,0.000000,0.000000}%
\pgfsetfillcolor{currentfill}%
\pgfsetlinewidth{1.003750pt}%
\definecolor{currentstroke}{rgb}{0.000000,0.000000,0.000000}%
\pgfsetstrokecolor{currentstroke}%
\pgfsetdash{}{0pt}%
\pgfpathmoveto{\pgfqpoint{1.005694in}{1.438667in}}%
\pgfpathcurveto{\pgfqpoint{1.016744in}{1.438667in}}{\pgfqpoint{1.027343in}{1.443057in}}{\pgfqpoint{1.035157in}{1.450871in}}%
\pgfpathcurveto{\pgfqpoint{1.042971in}{1.458685in}}{\pgfqpoint{1.047361in}{1.469284in}}{\pgfqpoint{1.047361in}{1.480334in}}%
\pgfpathcurveto{\pgfqpoint{1.047361in}{1.491384in}}{\pgfqpoint{1.042971in}{1.501983in}}{\pgfqpoint{1.035157in}{1.509796in}}%
\pgfpathcurveto{\pgfqpoint{1.027343in}{1.517610in}}{\pgfqpoint{1.016744in}{1.522000in}}{\pgfqpoint{1.005694in}{1.522000in}}%
\pgfpathcurveto{\pgfqpoint{0.994644in}{1.522000in}}{\pgfqpoint{0.984045in}{1.517610in}}{\pgfqpoint{0.976232in}{1.509796in}}%
\pgfpathcurveto{\pgfqpoint{0.968418in}{1.501983in}}{\pgfqpoint{0.964028in}{1.491384in}}{\pgfqpoint{0.964028in}{1.480334in}}%
\pgfpathcurveto{\pgfqpoint{0.964028in}{1.469284in}}{\pgfqpoint{0.968418in}{1.458685in}}{\pgfqpoint{0.976232in}{1.450871in}}%
\pgfpathcurveto{\pgfqpoint{0.984045in}{1.443057in}}{\pgfqpoint{0.994644in}{1.438667in}}{\pgfqpoint{1.005694in}{1.438667in}}%
\pgfpathclose%
\pgfusepath{stroke,fill}%
\end{pgfscope}%
\begin{pgfscope}%
\pgfpathrectangle{\pgfqpoint{0.375000in}{0.330000in}}{\pgfqpoint{2.325000in}{2.310000in}}%
\pgfusepath{clip}%
\pgfsetbuttcap%
\pgfsetroundjoin%
\definecolor{currentfill}{rgb}{0.000000,0.000000,0.000000}%
\pgfsetfillcolor{currentfill}%
\pgfsetlinewidth{1.003750pt}%
\definecolor{currentstroke}{rgb}{0.000000,0.000000,0.000000}%
\pgfsetstrokecolor{currentstroke}%
\pgfsetdash{}{0pt}%
\pgfpathmoveto{\pgfqpoint{1.005694in}{1.438667in}}%
\pgfpathcurveto{\pgfqpoint{1.016744in}{1.438667in}}{\pgfqpoint{1.027343in}{1.443057in}}{\pgfqpoint{1.035157in}{1.450871in}}%
\pgfpathcurveto{\pgfqpoint{1.042971in}{1.458685in}}{\pgfqpoint{1.047361in}{1.469284in}}{\pgfqpoint{1.047361in}{1.480334in}}%
\pgfpathcurveto{\pgfqpoint{1.047361in}{1.491384in}}{\pgfqpoint{1.042971in}{1.501983in}}{\pgfqpoint{1.035157in}{1.509796in}}%
\pgfpathcurveto{\pgfqpoint{1.027343in}{1.517610in}}{\pgfqpoint{1.016744in}{1.522000in}}{\pgfqpoint{1.005694in}{1.522000in}}%
\pgfpathcurveto{\pgfqpoint{0.994644in}{1.522000in}}{\pgfqpoint{0.984045in}{1.517610in}}{\pgfqpoint{0.976232in}{1.509796in}}%
\pgfpathcurveto{\pgfqpoint{0.968418in}{1.501983in}}{\pgfqpoint{0.964028in}{1.491384in}}{\pgfqpoint{0.964028in}{1.480334in}}%
\pgfpathcurveto{\pgfqpoint{0.964028in}{1.469284in}}{\pgfqpoint{0.968418in}{1.458685in}}{\pgfqpoint{0.976232in}{1.450871in}}%
\pgfpathcurveto{\pgfqpoint{0.984045in}{1.443057in}}{\pgfqpoint{0.994644in}{1.438667in}}{\pgfqpoint{1.005694in}{1.438667in}}%
\pgfpathclose%
\pgfusepath{stroke,fill}%
\end{pgfscope}%
\begin{pgfscope}%
\pgfpathrectangle{\pgfqpoint{0.375000in}{0.330000in}}{\pgfqpoint{2.325000in}{2.310000in}}%
\pgfusepath{clip}%
\pgfsetbuttcap%
\pgfsetroundjoin%
\definecolor{currentfill}{rgb}{0.000000,0.000000,0.000000}%
\pgfsetfillcolor{currentfill}%
\pgfsetlinewidth{1.003750pt}%
\definecolor{currentstroke}{rgb}{0.000000,0.000000,0.000000}%
\pgfsetstrokecolor{currentstroke}%
\pgfsetdash{}{0pt}%
\pgfpathmoveto{\pgfqpoint{1.005694in}{1.438667in}}%
\pgfpathcurveto{\pgfqpoint{1.016744in}{1.438667in}}{\pgfqpoint{1.027343in}{1.443057in}}{\pgfqpoint{1.035157in}{1.450871in}}%
\pgfpathcurveto{\pgfqpoint{1.042971in}{1.458685in}}{\pgfqpoint{1.047361in}{1.469284in}}{\pgfqpoint{1.047361in}{1.480334in}}%
\pgfpathcurveto{\pgfqpoint{1.047361in}{1.491384in}}{\pgfqpoint{1.042971in}{1.501983in}}{\pgfqpoint{1.035157in}{1.509796in}}%
\pgfpathcurveto{\pgfqpoint{1.027343in}{1.517610in}}{\pgfqpoint{1.016744in}{1.522000in}}{\pgfqpoint{1.005694in}{1.522000in}}%
\pgfpathcurveto{\pgfqpoint{0.994644in}{1.522000in}}{\pgfqpoint{0.984045in}{1.517610in}}{\pgfqpoint{0.976232in}{1.509796in}}%
\pgfpathcurveto{\pgfqpoint{0.968418in}{1.501983in}}{\pgfqpoint{0.964028in}{1.491384in}}{\pgfqpoint{0.964028in}{1.480334in}}%
\pgfpathcurveto{\pgfqpoint{0.964028in}{1.469284in}}{\pgfqpoint{0.968418in}{1.458685in}}{\pgfqpoint{0.976232in}{1.450871in}}%
\pgfpathcurveto{\pgfqpoint{0.984045in}{1.443057in}}{\pgfqpoint{0.994644in}{1.438667in}}{\pgfqpoint{1.005694in}{1.438667in}}%
\pgfpathclose%
\pgfusepath{stroke,fill}%
\end{pgfscope}%
\begin{pgfscope}%
\pgfpathrectangle{\pgfqpoint{0.375000in}{0.330000in}}{\pgfqpoint{2.325000in}{2.310000in}}%
\pgfusepath{clip}%
\pgfsetbuttcap%
\pgfsetroundjoin%
\definecolor{currentfill}{rgb}{0.000000,0.000000,0.000000}%
\pgfsetfillcolor{currentfill}%
\pgfsetlinewidth{1.003750pt}%
\definecolor{currentstroke}{rgb}{0.000000,0.000000,0.000000}%
\pgfsetstrokecolor{currentstroke}%
\pgfsetdash{}{0pt}%
\pgfpathmoveto{\pgfqpoint{1.005694in}{1.438667in}}%
\pgfpathcurveto{\pgfqpoint{1.016744in}{1.438667in}}{\pgfqpoint{1.027343in}{1.443057in}}{\pgfqpoint{1.035157in}{1.450871in}}%
\pgfpathcurveto{\pgfqpoint{1.042971in}{1.458685in}}{\pgfqpoint{1.047361in}{1.469284in}}{\pgfqpoint{1.047361in}{1.480334in}}%
\pgfpathcurveto{\pgfqpoint{1.047361in}{1.491384in}}{\pgfqpoint{1.042971in}{1.501983in}}{\pgfqpoint{1.035157in}{1.509796in}}%
\pgfpathcurveto{\pgfqpoint{1.027343in}{1.517610in}}{\pgfqpoint{1.016744in}{1.522000in}}{\pgfqpoint{1.005694in}{1.522000in}}%
\pgfpathcurveto{\pgfqpoint{0.994644in}{1.522000in}}{\pgfqpoint{0.984045in}{1.517610in}}{\pgfqpoint{0.976232in}{1.509796in}}%
\pgfpathcurveto{\pgfqpoint{0.968418in}{1.501983in}}{\pgfqpoint{0.964028in}{1.491384in}}{\pgfqpoint{0.964028in}{1.480334in}}%
\pgfpathcurveto{\pgfqpoint{0.964028in}{1.469284in}}{\pgfqpoint{0.968418in}{1.458685in}}{\pgfqpoint{0.976232in}{1.450871in}}%
\pgfpathcurveto{\pgfqpoint{0.984045in}{1.443057in}}{\pgfqpoint{0.994644in}{1.438667in}}{\pgfqpoint{1.005694in}{1.438667in}}%
\pgfpathclose%
\pgfusepath{stroke,fill}%
\end{pgfscope}%
\begin{pgfscope}%
\pgfpathrectangle{\pgfqpoint{0.375000in}{0.330000in}}{\pgfqpoint{2.325000in}{2.310000in}}%
\pgfusepath{clip}%
\pgfsetbuttcap%
\pgfsetroundjoin%
\definecolor{currentfill}{rgb}{0.000000,0.000000,0.000000}%
\pgfsetfillcolor{currentfill}%
\pgfsetlinewidth{1.003750pt}%
\definecolor{currentstroke}{rgb}{0.000000,0.000000,0.000000}%
\pgfsetstrokecolor{currentstroke}%
\pgfsetdash{}{0pt}%
\pgfpathmoveto{\pgfqpoint{1.005694in}{1.438667in}}%
\pgfpathcurveto{\pgfqpoint{1.016744in}{1.438667in}}{\pgfqpoint{1.027343in}{1.443057in}}{\pgfqpoint{1.035157in}{1.450871in}}%
\pgfpathcurveto{\pgfqpoint{1.042971in}{1.458685in}}{\pgfqpoint{1.047361in}{1.469284in}}{\pgfqpoint{1.047361in}{1.480334in}}%
\pgfpathcurveto{\pgfqpoint{1.047361in}{1.491384in}}{\pgfqpoint{1.042971in}{1.501983in}}{\pgfqpoint{1.035157in}{1.509796in}}%
\pgfpathcurveto{\pgfqpoint{1.027343in}{1.517610in}}{\pgfqpoint{1.016744in}{1.522000in}}{\pgfqpoint{1.005694in}{1.522000in}}%
\pgfpathcurveto{\pgfqpoint{0.994644in}{1.522000in}}{\pgfqpoint{0.984045in}{1.517610in}}{\pgfqpoint{0.976232in}{1.509796in}}%
\pgfpathcurveto{\pgfqpoint{0.968418in}{1.501983in}}{\pgfqpoint{0.964028in}{1.491384in}}{\pgfqpoint{0.964028in}{1.480334in}}%
\pgfpathcurveto{\pgfqpoint{0.964028in}{1.469284in}}{\pgfqpoint{0.968418in}{1.458685in}}{\pgfqpoint{0.976232in}{1.450871in}}%
\pgfpathcurveto{\pgfqpoint{0.984045in}{1.443057in}}{\pgfqpoint{0.994644in}{1.438667in}}{\pgfqpoint{1.005694in}{1.438667in}}%
\pgfpathclose%
\pgfusepath{stroke,fill}%
\end{pgfscope}%
\begin{pgfscope}%
\pgfpathrectangle{\pgfqpoint{0.375000in}{0.330000in}}{\pgfqpoint{2.325000in}{2.310000in}}%
\pgfusepath{clip}%
\pgfsetbuttcap%
\pgfsetroundjoin%
\definecolor{currentfill}{rgb}{0.000000,0.000000,0.000000}%
\pgfsetfillcolor{currentfill}%
\pgfsetlinewidth{1.003750pt}%
\definecolor{currentstroke}{rgb}{0.000000,0.000000,0.000000}%
\pgfsetstrokecolor{currentstroke}%
\pgfsetdash{}{0pt}%
\pgfpathmoveto{\pgfqpoint{1.005694in}{1.438667in}}%
\pgfpathcurveto{\pgfqpoint{1.016744in}{1.438667in}}{\pgfqpoint{1.027343in}{1.443057in}}{\pgfqpoint{1.035157in}{1.450871in}}%
\pgfpathcurveto{\pgfqpoint{1.042971in}{1.458685in}}{\pgfqpoint{1.047361in}{1.469284in}}{\pgfqpoint{1.047361in}{1.480334in}}%
\pgfpathcurveto{\pgfqpoint{1.047361in}{1.491384in}}{\pgfqpoint{1.042971in}{1.501983in}}{\pgfqpoint{1.035157in}{1.509796in}}%
\pgfpathcurveto{\pgfqpoint{1.027343in}{1.517610in}}{\pgfqpoint{1.016744in}{1.522000in}}{\pgfqpoint{1.005694in}{1.522000in}}%
\pgfpathcurveto{\pgfqpoint{0.994644in}{1.522000in}}{\pgfqpoint{0.984045in}{1.517610in}}{\pgfqpoint{0.976232in}{1.509796in}}%
\pgfpathcurveto{\pgfqpoint{0.968418in}{1.501983in}}{\pgfqpoint{0.964028in}{1.491384in}}{\pgfqpoint{0.964028in}{1.480334in}}%
\pgfpathcurveto{\pgfqpoint{0.964028in}{1.469284in}}{\pgfqpoint{0.968418in}{1.458685in}}{\pgfqpoint{0.976232in}{1.450871in}}%
\pgfpathcurveto{\pgfqpoint{0.984045in}{1.443057in}}{\pgfqpoint{0.994644in}{1.438667in}}{\pgfqpoint{1.005694in}{1.438667in}}%
\pgfpathclose%
\pgfusepath{stroke,fill}%
\end{pgfscope}%
\begin{pgfscope}%
\pgfpathrectangle{\pgfqpoint{0.375000in}{0.330000in}}{\pgfqpoint{2.325000in}{2.310000in}}%
\pgfusepath{clip}%
\pgfsetbuttcap%
\pgfsetroundjoin%
\definecolor{currentfill}{rgb}{0.000000,0.000000,0.000000}%
\pgfsetfillcolor{currentfill}%
\pgfsetlinewidth{1.003750pt}%
\definecolor{currentstroke}{rgb}{0.000000,0.000000,0.000000}%
\pgfsetstrokecolor{currentstroke}%
\pgfsetdash{}{0pt}%
\pgfpathmoveto{\pgfqpoint{1.005694in}{1.438667in}}%
\pgfpathcurveto{\pgfqpoint{1.016744in}{1.438667in}}{\pgfqpoint{1.027343in}{1.443057in}}{\pgfqpoint{1.035157in}{1.450871in}}%
\pgfpathcurveto{\pgfqpoint{1.042971in}{1.458685in}}{\pgfqpoint{1.047361in}{1.469284in}}{\pgfqpoint{1.047361in}{1.480334in}}%
\pgfpathcurveto{\pgfqpoint{1.047361in}{1.491384in}}{\pgfqpoint{1.042971in}{1.501983in}}{\pgfqpoint{1.035157in}{1.509796in}}%
\pgfpathcurveto{\pgfqpoint{1.027343in}{1.517610in}}{\pgfqpoint{1.016744in}{1.522000in}}{\pgfqpoint{1.005694in}{1.522000in}}%
\pgfpathcurveto{\pgfqpoint{0.994644in}{1.522000in}}{\pgfqpoint{0.984045in}{1.517610in}}{\pgfqpoint{0.976232in}{1.509796in}}%
\pgfpathcurveto{\pgfqpoint{0.968418in}{1.501983in}}{\pgfqpoint{0.964028in}{1.491384in}}{\pgfqpoint{0.964028in}{1.480334in}}%
\pgfpathcurveto{\pgfqpoint{0.964028in}{1.469284in}}{\pgfqpoint{0.968418in}{1.458685in}}{\pgfqpoint{0.976232in}{1.450871in}}%
\pgfpathcurveto{\pgfqpoint{0.984045in}{1.443057in}}{\pgfqpoint{0.994644in}{1.438667in}}{\pgfqpoint{1.005694in}{1.438667in}}%
\pgfpathclose%
\pgfusepath{stroke,fill}%
\end{pgfscope}%
\begin{pgfscope}%
\pgfpathrectangle{\pgfqpoint{0.375000in}{0.330000in}}{\pgfqpoint{2.325000in}{2.310000in}}%
\pgfusepath{clip}%
\pgfsetbuttcap%
\pgfsetroundjoin%
\definecolor{currentfill}{rgb}{0.000000,0.000000,0.000000}%
\pgfsetfillcolor{currentfill}%
\pgfsetlinewidth{1.003750pt}%
\definecolor{currentstroke}{rgb}{0.000000,0.000000,0.000000}%
\pgfsetstrokecolor{currentstroke}%
\pgfsetdash{}{0pt}%
\pgfpathmoveto{\pgfqpoint{1.005694in}{1.438667in}}%
\pgfpathcurveto{\pgfqpoint{1.016744in}{1.438667in}}{\pgfqpoint{1.027343in}{1.443057in}}{\pgfqpoint{1.035157in}{1.450871in}}%
\pgfpathcurveto{\pgfqpoint{1.042971in}{1.458685in}}{\pgfqpoint{1.047361in}{1.469284in}}{\pgfqpoint{1.047361in}{1.480334in}}%
\pgfpathcurveto{\pgfqpoint{1.047361in}{1.491384in}}{\pgfqpoint{1.042971in}{1.501983in}}{\pgfqpoint{1.035157in}{1.509796in}}%
\pgfpathcurveto{\pgfqpoint{1.027343in}{1.517610in}}{\pgfqpoint{1.016744in}{1.522000in}}{\pgfqpoint{1.005694in}{1.522000in}}%
\pgfpathcurveto{\pgfqpoint{0.994644in}{1.522000in}}{\pgfqpoint{0.984045in}{1.517610in}}{\pgfqpoint{0.976232in}{1.509796in}}%
\pgfpathcurveto{\pgfqpoint{0.968418in}{1.501983in}}{\pgfqpoint{0.964028in}{1.491384in}}{\pgfqpoint{0.964028in}{1.480334in}}%
\pgfpathcurveto{\pgfqpoint{0.964028in}{1.469284in}}{\pgfqpoint{0.968418in}{1.458685in}}{\pgfqpoint{0.976232in}{1.450871in}}%
\pgfpathcurveto{\pgfqpoint{0.984045in}{1.443057in}}{\pgfqpoint{0.994644in}{1.438667in}}{\pgfqpoint{1.005694in}{1.438667in}}%
\pgfpathclose%
\pgfusepath{stroke,fill}%
\end{pgfscope}%
\begin{pgfscope}%
\pgfpathrectangle{\pgfqpoint{0.375000in}{0.330000in}}{\pgfqpoint{2.325000in}{2.310000in}}%
\pgfusepath{clip}%
\pgfsetbuttcap%
\pgfsetroundjoin%
\definecolor{currentfill}{rgb}{0.000000,0.000000,0.000000}%
\pgfsetfillcolor{currentfill}%
\pgfsetlinewidth{1.003750pt}%
\definecolor{currentstroke}{rgb}{0.000000,0.000000,0.000000}%
\pgfsetstrokecolor{currentstroke}%
\pgfsetdash{}{0pt}%
\pgfpathmoveto{\pgfqpoint{1.005694in}{1.438667in}}%
\pgfpathcurveto{\pgfqpoint{1.016744in}{1.438667in}}{\pgfqpoint{1.027343in}{1.443057in}}{\pgfqpoint{1.035157in}{1.450871in}}%
\pgfpathcurveto{\pgfqpoint{1.042971in}{1.458685in}}{\pgfqpoint{1.047361in}{1.469284in}}{\pgfqpoint{1.047361in}{1.480334in}}%
\pgfpathcurveto{\pgfqpoint{1.047361in}{1.491384in}}{\pgfqpoint{1.042971in}{1.501983in}}{\pgfqpoint{1.035157in}{1.509796in}}%
\pgfpathcurveto{\pgfqpoint{1.027343in}{1.517610in}}{\pgfqpoint{1.016744in}{1.522000in}}{\pgfqpoint{1.005694in}{1.522000in}}%
\pgfpathcurveto{\pgfqpoint{0.994644in}{1.522000in}}{\pgfqpoint{0.984045in}{1.517610in}}{\pgfqpoint{0.976232in}{1.509796in}}%
\pgfpathcurveto{\pgfqpoint{0.968418in}{1.501983in}}{\pgfqpoint{0.964028in}{1.491384in}}{\pgfqpoint{0.964028in}{1.480334in}}%
\pgfpathcurveto{\pgfqpoint{0.964028in}{1.469284in}}{\pgfqpoint{0.968418in}{1.458685in}}{\pgfqpoint{0.976232in}{1.450871in}}%
\pgfpathcurveto{\pgfqpoint{0.984045in}{1.443057in}}{\pgfqpoint{0.994644in}{1.438667in}}{\pgfqpoint{1.005694in}{1.438667in}}%
\pgfpathclose%
\pgfusepath{stroke,fill}%
\end{pgfscope}%
\begin{pgfscope}%
\pgfpathrectangle{\pgfqpoint{0.375000in}{0.330000in}}{\pgfqpoint{2.325000in}{2.310000in}}%
\pgfusepath{clip}%
\pgfsetbuttcap%
\pgfsetroundjoin%
\definecolor{currentfill}{rgb}{0.000000,0.000000,0.000000}%
\pgfsetfillcolor{currentfill}%
\pgfsetlinewidth{1.003750pt}%
\definecolor{currentstroke}{rgb}{0.000000,0.000000,0.000000}%
\pgfsetstrokecolor{currentstroke}%
\pgfsetdash{}{0pt}%
\pgfpathmoveto{\pgfqpoint{1.005694in}{1.438667in}}%
\pgfpathcurveto{\pgfqpoint{1.016744in}{1.438667in}}{\pgfqpoint{1.027343in}{1.443057in}}{\pgfqpoint{1.035157in}{1.450871in}}%
\pgfpathcurveto{\pgfqpoint{1.042971in}{1.458685in}}{\pgfqpoint{1.047361in}{1.469284in}}{\pgfqpoint{1.047361in}{1.480334in}}%
\pgfpathcurveto{\pgfqpoint{1.047361in}{1.491384in}}{\pgfqpoint{1.042971in}{1.501983in}}{\pgfqpoint{1.035157in}{1.509796in}}%
\pgfpathcurveto{\pgfqpoint{1.027343in}{1.517610in}}{\pgfqpoint{1.016744in}{1.522000in}}{\pgfqpoint{1.005694in}{1.522000in}}%
\pgfpathcurveto{\pgfqpoint{0.994644in}{1.522000in}}{\pgfqpoint{0.984045in}{1.517610in}}{\pgfqpoint{0.976232in}{1.509796in}}%
\pgfpathcurveto{\pgfqpoint{0.968418in}{1.501983in}}{\pgfqpoint{0.964028in}{1.491384in}}{\pgfqpoint{0.964028in}{1.480334in}}%
\pgfpathcurveto{\pgfqpoint{0.964028in}{1.469284in}}{\pgfqpoint{0.968418in}{1.458685in}}{\pgfqpoint{0.976232in}{1.450871in}}%
\pgfpathcurveto{\pgfqpoint{0.984045in}{1.443057in}}{\pgfqpoint{0.994644in}{1.438667in}}{\pgfqpoint{1.005694in}{1.438667in}}%
\pgfpathclose%
\pgfusepath{stroke,fill}%
\end{pgfscope}%
\begin{pgfscope}%
\pgfpathrectangle{\pgfqpoint{0.375000in}{0.330000in}}{\pgfqpoint{2.325000in}{2.310000in}}%
\pgfusepath{clip}%
\pgfsetbuttcap%
\pgfsetroundjoin%
\definecolor{currentfill}{rgb}{0.000000,0.000000,0.000000}%
\pgfsetfillcolor{currentfill}%
\pgfsetlinewidth{1.003750pt}%
\definecolor{currentstroke}{rgb}{0.000000,0.000000,0.000000}%
\pgfsetstrokecolor{currentstroke}%
\pgfsetdash{}{0pt}%
\pgfpathmoveto{\pgfqpoint{1.005694in}{1.438667in}}%
\pgfpathcurveto{\pgfqpoint{1.016744in}{1.438667in}}{\pgfqpoint{1.027343in}{1.443057in}}{\pgfqpoint{1.035157in}{1.450871in}}%
\pgfpathcurveto{\pgfqpoint{1.042971in}{1.458685in}}{\pgfqpoint{1.047361in}{1.469284in}}{\pgfqpoint{1.047361in}{1.480334in}}%
\pgfpathcurveto{\pgfqpoint{1.047361in}{1.491384in}}{\pgfqpoint{1.042971in}{1.501983in}}{\pgfqpoint{1.035157in}{1.509796in}}%
\pgfpathcurveto{\pgfqpoint{1.027343in}{1.517610in}}{\pgfqpoint{1.016744in}{1.522000in}}{\pgfqpoint{1.005694in}{1.522000in}}%
\pgfpathcurveto{\pgfqpoint{0.994644in}{1.522000in}}{\pgfqpoint{0.984045in}{1.517610in}}{\pgfqpoint{0.976232in}{1.509796in}}%
\pgfpathcurveto{\pgfqpoint{0.968418in}{1.501983in}}{\pgfqpoint{0.964028in}{1.491384in}}{\pgfqpoint{0.964028in}{1.480334in}}%
\pgfpathcurveto{\pgfqpoint{0.964028in}{1.469284in}}{\pgfqpoint{0.968418in}{1.458685in}}{\pgfqpoint{0.976232in}{1.450871in}}%
\pgfpathcurveto{\pgfqpoint{0.984045in}{1.443057in}}{\pgfqpoint{0.994644in}{1.438667in}}{\pgfqpoint{1.005694in}{1.438667in}}%
\pgfpathclose%
\pgfusepath{stroke,fill}%
\end{pgfscope}%
\begin{pgfscope}%
\pgfpathrectangle{\pgfqpoint{0.375000in}{0.330000in}}{\pgfqpoint{2.325000in}{2.310000in}}%
\pgfusepath{clip}%
\pgfsetbuttcap%
\pgfsetroundjoin%
\definecolor{currentfill}{rgb}{0.000000,0.000000,0.000000}%
\pgfsetfillcolor{currentfill}%
\pgfsetlinewidth{1.003750pt}%
\definecolor{currentstroke}{rgb}{0.000000,0.000000,0.000000}%
\pgfsetstrokecolor{currentstroke}%
\pgfsetdash{}{0pt}%
\pgfpathmoveto{\pgfqpoint{1.005694in}{1.438667in}}%
\pgfpathcurveto{\pgfqpoint{1.016744in}{1.438667in}}{\pgfqpoint{1.027343in}{1.443057in}}{\pgfqpoint{1.035157in}{1.450871in}}%
\pgfpathcurveto{\pgfqpoint{1.042971in}{1.458685in}}{\pgfqpoint{1.047361in}{1.469284in}}{\pgfqpoint{1.047361in}{1.480334in}}%
\pgfpathcurveto{\pgfqpoint{1.047361in}{1.491384in}}{\pgfqpoint{1.042971in}{1.501983in}}{\pgfqpoint{1.035157in}{1.509796in}}%
\pgfpathcurveto{\pgfqpoint{1.027343in}{1.517610in}}{\pgfqpoint{1.016744in}{1.522000in}}{\pgfqpoint{1.005694in}{1.522000in}}%
\pgfpathcurveto{\pgfqpoint{0.994644in}{1.522000in}}{\pgfqpoint{0.984045in}{1.517610in}}{\pgfqpoint{0.976232in}{1.509796in}}%
\pgfpathcurveto{\pgfqpoint{0.968418in}{1.501983in}}{\pgfqpoint{0.964028in}{1.491384in}}{\pgfqpoint{0.964028in}{1.480334in}}%
\pgfpathcurveto{\pgfqpoint{0.964028in}{1.469284in}}{\pgfqpoint{0.968418in}{1.458685in}}{\pgfqpoint{0.976232in}{1.450871in}}%
\pgfpathcurveto{\pgfqpoint{0.984045in}{1.443057in}}{\pgfqpoint{0.994644in}{1.438667in}}{\pgfqpoint{1.005694in}{1.438667in}}%
\pgfpathclose%
\pgfusepath{stroke,fill}%
\end{pgfscope}%
\begin{pgfscope}%
\pgfpathrectangle{\pgfqpoint{0.375000in}{0.330000in}}{\pgfqpoint{2.325000in}{2.310000in}}%
\pgfusepath{clip}%
\pgfsetbuttcap%
\pgfsetroundjoin%
\definecolor{currentfill}{rgb}{0.000000,0.000000,0.000000}%
\pgfsetfillcolor{currentfill}%
\pgfsetlinewidth{1.003750pt}%
\definecolor{currentstroke}{rgb}{0.000000,0.000000,0.000000}%
\pgfsetstrokecolor{currentstroke}%
\pgfsetdash{}{0pt}%
\pgfpathmoveto{\pgfqpoint{1.005694in}{1.438667in}}%
\pgfpathcurveto{\pgfqpoint{1.016744in}{1.438667in}}{\pgfqpoint{1.027343in}{1.443057in}}{\pgfqpoint{1.035157in}{1.450871in}}%
\pgfpathcurveto{\pgfqpoint{1.042971in}{1.458685in}}{\pgfqpoint{1.047361in}{1.469284in}}{\pgfqpoint{1.047361in}{1.480334in}}%
\pgfpathcurveto{\pgfqpoint{1.047361in}{1.491384in}}{\pgfqpoint{1.042971in}{1.501983in}}{\pgfqpoint{1.035157in}{1.509796in}}%
\pgfpathcurveto{\pgfqpoint{1.027343in}{1.517610in}}{\pgfqpoint{1.016744in}{1.522000in}}{\pgfqpoint{1.005694in}{1.522000in}}%
\pgfpathcurveto{\pgfqpoint{0.994644in}{1.522000in}}{\pgfqpoint{0.984045in}{1.517610in}}{\pgfqpoint{0.976232in}{1.509796in}}%
\pgfpathcurveto{\pgfqpoint{0.968418in}{1.501983in}}{\pgfqpoint{0.964028in}{1.491384in}}{\pgfqpoint{0.964028in}{1.480334in}}%
\pgfpathcurveto{\pgfqpoint{0.964028in}{1.469284in}}{\pgfqpoint{0.968418in}{1.458685in}}{\pgfqpoint{0.976232in}{1.450871in}}%
\pgfpathcurveto{\pgfqpoint{0.984045in}{1.443057in}}{\pgfqpoint{0.994644in}{1.438667in}}{\pgfqpoint{1.005694in}{1.438667in}}%
\pgfpathclose%
\pgfusepath{stroke,fill}%
\end{pgfscope}%
\begin{pgfscope}%
\pgfpathrectangle{\pgfqpoint{0.375000in}{0.330000in}}{\pgfqpoint{2.325000in}{2.310000in}}%
\pgfusepath{clip}%
\pgfsetbuttcap%
\pgfsetroundjoin%
\definecolor{currentfill}{rgb}{0.000000,0.000000,0.000000}%
\pgfsetfillcolor{currentfill}%
\pgfsetlinewidth{1.003750pt}%
\definecolor{currentstroke}{rgb}{0.000000,0.000000,0.000000}%
\pgfsetstrokecolor{currentstroke}%
\pgfsetdash{}{0pt}%
\pgfpathmoveto{\pgfqpoint{1.005694in}{1.438667in}}%
\pgfpathcurveto{\pgfqpoint{1.016744in}{1.438667in}}{\pgfqpoint{1.027343in}{1.443057in}}{\pgfqpoint{1.035157in}{1.450871in}}%
\pgfpathcurveto{\pgfqpoint{1.042971in}{1.458685in}}{\pgfqpoint{1.047361in}{1.469284in}}{\pgfqpoint{1.047361in}{1.480334in}}%
\pgfpathcurveto{\pgfqpoint{1.047361in}{1.491384in}}{\pgfqpoint{1.042971in}{1.501983in}}{\pgfqpoint{1.035157in}{1.509796in}}%
\pgfpathcurveto{\pgfqpoint{1.027343in}{1.517610in}}{\pgfqpoint{1.016744in}{1.522000in}}{\pgfqpoint{1.005694in}{1.522000in}}%
\pgfpathcurveto{\pgfqpoint{0.994644in}{1.522000in}}{\pgfqpoint{0.984045in}{1.517610in}}{\pgfqpoint{0.976232in}{1.509796in}}%
\pgfpathcurveto{\pgfqpoint{0.968418in}{1.501983in}}{\pgfqpoint{0.964028in}{1.491384in}}{\pgfqpoint{0.964028in}{1.480334in}}%
\pgfpathcurveto{\pgfqpoint{0.964028in}{1.469284in}}{\pgfqpoint{0.968418in}{1.458685in}}{\pgfqpoint{0.976232in}{1.450871in}}%
\pgfpathcurveto{\pgfqpoint{0.984045in}{1.443057in}}{\pgfqpoint{0.994644in}{1.438667in}}{\pgfqpoint{1.005694in}{1.438667in}}%
\pgfpathclose%
\pgfusepath{stroke,fill}%
\end{pgfscope}%
\begin{pgfscope}%
\pgfpathrectangle{\pgfqpoint{0.375000in}{0.330000in}}{\pgfqpoint{2.325000in}{2.310000in}}%
\pgfusepath{clip}%
\pgfsetbuttcap%
\pgfsetroundjoin%
\definecolor{currentfill}{rgb}{0.000000,0.000000,0.000000}%
\pgfsetfillcolor{currentfill}%
\pgfsetlinewidth{1.003750pt}%
\definecolor{currentstroke}{rgb}{0.000000,0.000000,0.000000}%
\pgfsetstrokecolor{currentstroke}%
\pgfsetdash{}{0pt}%
\pgfpathmoveto{\pgfqpoint{1.005694in}{1.438667in}}%
\pgfpathcurveto{\pgfqpoint{1.016744in}{1.438667in}}{\pgfqpoint{1.027343in}{1.443057in}}{\pgfqpoint{1.035157in}{1.450871in}}%
\pgfpathcurveto{\pgfqpoint{1.042971in}{1.458685in}}{\pgfqpoint{1.047361in}{1.469284in}}{\pgfqpoint{1.047361in}{1.480334in}}%
\pgfpathcurveto{\pgfqpoint{1.047361in}{1.491384in}}{\pgfqpoint{1.042971in}{1.501983in}}{\pgfqpoint{1.035157in}{1.509796in}}%
\pgfpathcurveto{\pgfqpoint{1.027343in}{1.517610in}}{\pgfqpoint{1.016744in}{1.522000in}}{\pgfqpoint{1.005694in}{1.522000in}}%
\pgfpathcurveto{\pgfqpoint{0.994644in}{1.522000in}}{\pgfqpoint{0.984045in}{1.517610in}}{\pgfqpoint{0.976232in}{1.509796in}}%
\pgfpathcurveto{\pgfqpoint{0.968418in}{1.501983in}}{\pgfqpoint{0.964028in}{1.491384in}}{\pgfqpoint{0.964028in}{1.480334in}}%
\pgfpathcurveto{\pgfqpoint{0.964028in}{1.469284in}}{\pgfqpoint{0.968418in}{1.458685in}}{\pgfqpoint{0.976232in}{1.450871in}}%
\pgfpathcurveto{\pgfqpoint{0.984045in}{1.443057in}}{\pgfqpoint{0.994644in}{1.438667in}}{\pgfqpoint{1.005694in}{1.438667in}}%
\pgfpathclose%
\pgfusepath{stroke,fill}%
\end{pgfscope}%
\begin{pgfscope}%
\pgfpathrectangle{\pgfqpoint{0.375000in}{0.330000in}}{\pgfqpoint{2.325000in}{2.310000in}}%
\pgfusepath{clip}%
\pgfsetbuttcap%
\pgfsetroundjoin%
\definecolor{currentfill}{rgb}{0.000000,0.000000,0.000000}%
\pgfsetfillcolor{currentfill}%
\pgfsetlinewidth{1.003750pt}%
\definecolor{currentstroke}{rgb}{0.000000,0.000000,0.000000}%
\pgfsetstrokecolor{currentstroke}%
\pgfsetdash{}{0pt}%
\pgfpathmoveto{\pgfqpoint{1.005694in}{1.438667in}}%
\pgfpathcurveto{\pgfqpoint{1.016744in}{1.438667in}}{\pgfqpoint{1.027343in}{1.443057in}}{\pgfqpoint{1.035157in}{1.450871in}}%
\pgfpathcurveto{\pgfqpoint{1.042971in}{1.458685in}}{\pgfqpoint{1.047361in}{1.469284in}}{\pgfqpoint{1.047361in}{1.480334in}}%
\pgfpathcurveto{\pgfqpoint{1.047361in}{1.491384in}}{\pgfqpoint{1.042971in}{1.501983in}}{\pgfqpoint{1.035157in}{1.509796in}}%
\pgfpathcurveto{\pgfqpoint{1.027343in}{1.517610in}}{\pgfqpoint{1.016744in}{1.522000in}}{\pgfqpoint{1.005694in}{1.522000in}}%
\pgfpathcurveto{\pgfqpoint{0.994644in}{1.522000in}}{\pgfqpoint{0.984045in}{1.517610in}}{\pgfqpoint{0.976232in}{1.509796in}}%
\pgfpathcurveto{\pgfqpoint{0.968418in}{1.501983in}}{\pgfqpoint{0.964028in}{1.491384in}}{\pgfqpoint{0.964028in}{1.480334in}}%
\pgfpathcurveto{\pgfqpoint{0.964028in}{1.469284in}}{\pgfqpoint{0.968418in}{1.458685in}}{\pgfqpoint{0.976232in}{1.450871in}}%
\pgfpathcurveto{\pgfqpoint{0.984045in}{1.443057in}}{\pgfqpoint{0.994644in}{1.438667in}}{\pgfqpoint{1.005694in}{1.438667in}}%
\pgfpathclose%
\pgfusepath{stroke,fill}%
\end{pgfscope}%
\begin{pgfscope}%
\pgfpathrectangle{\pgfqpoint{0.375000in}{0.330000in}}{\pgfqpoint{2.325000in}{2.310000in}}%
\pgfusepath{clip}%
\pgfsetbuttcap%
\pgfsetroundjoin%
\definecolor{currentfill}{rgb}{0.000000,0.000000,0.000000}%
\pgfsetfillcolor{currentfill}%
\pgfsetlinewidth{1.003750pt}%
\definecolor{currentstroke}{rgb}{0.000000,0.000000,0.000000}%
\pgfsetstrokecolor{currentstroke}%
\pgfsetdash{}{0pt}%
\pgfpathmoveto{\pgfqpoint{1.005694in}{1.438667in}}%
\pgfpathcurveto{\pgfqpoint{1.016744in}{1.438667in}}{\pgfqpoint{1.027343in}{1.443057in}}{\pgfqpoint{1.035157in}{1.450871in}}%
\pgfpathcurveto{\pgfqpoint{1.042971in}{1.458685in}}{\pgfqpoint{1.047361in}{1.469284in}}{\pgfqpoint{1.047361in}{1.480334in}}%
\pgfpathcurveto{\pgfqpoint{1.047361in}{1.491384in}}{\pgfqpoint{1.042971in}{1.501983in}}{\pgfqpoint{1.035157in}{1.509796in}}%
\pgfpathcurveto{\pgfqpoint{1.027343in}{1.517610in}}{\pgfqpoint{1.016744in}{1.522000in}}{\pgfqpoint{1.005694in}{1.522000in}}%
\pgfpathcurveto{\pgfqpoint{0.994644in}{1.522000in}}{\pgfqpoint{0.984045in}{1.517610in}}{\pgfqpoint{0.976232in}{1.509796in}}%
\pgfpathcurveto{\pgfqpoint{0.968418in}{1.501983in}}{\pgfqpoint{0.964028in}{1.491384in}}{\pgfqpoint{0.964028in}{1.480334in}}%
\pgfpathcurveto{\pgfqpoint{0.964028in}{1.469284in}}{\pgfqpoint{0.968418in}{1.458685in}}{\pgfqpoint{0.976232in}{1.450871in}}%
\pgfpathcurveto{\pgfqpoint{0.984045in}{1.443057in}}{\pgfqpoint{0.994644in}{1.438667in}}{\pgfqpoint{1.005694in}{1.438667in}}%
\pgfpathclose%
\pgfusepath{stroke,fill}%
\end{pgfscope}%
\begin{pgfscope}%
\pgfpathrectangle{\pgfqpoint{0.375000in}{0.330000in}}{\pgfqpoint{2.325000in}{2.310000in}}%
\pgfusepath{clip}%
\pgfsetbuttcap%
\pgfsetroundjoin%
\definecolor{currentfill}{rgb}{0.000000,0.000000,0.000000}%
\pgfsetfillcolor{currentfill}%
\pgfsetlinewidth{1.003750pt}%
\definecolor{currentstroke}{rgb}{0.000000,0.000000,0.000000}%
\pgfsetstrokecolor{currentstroke}%
\pgfsetdash{}{0pt}%
\pgfpathmoveto{\pgfqpoint{1.005694in}{1.438667in}}%
\pgfpathcurveto{\pgfqpoint{1.016744in}{1.438667in}}{\pgfqpoint{1.027343in}{1.443057in}}{\pgfqpoint{1.035157in}{1.450871in}}%
\pgfpathcurveto{\pgfqpoint{1.042971in}{1.458685in}}{\pgfqpoint{1.047361in}{1.469284in}}{\pgfqpoint{1.047361in}{1.480334in}}%
\pgfpathcurveto{\pgfqpoint{1.047361in}{1.491384in}}{\pgfqpoint{1.042971in}{1.501983in}}{\pgfqpoint{1.035157in}{1.509796in}}%
\pgfpathcurveto{\pgfqpoint{1.027343in}{1.517610in}}{\pgfqpoint{1.016744in}{1.522000in}}{\pgfqpoint{1.005694in}{1.522000in}}%
\pgfpathcurveto{\pgfqpoint{0.994644in}{1.522000in}}{\pgfqpoint{0.984045in}{1.517610in}}{\pgfqpoint{0.976232in}{1.509796in}}%
\pgfpathcurveto{\pgfqpoint{0.968418in}{1.501983in}}{\pgfqpoint{0.964028in}{1.491384in}}{\pgfqpoint{0.964028in}{1.480334in}}%
\pgfpathcurveto{\pgfqpoint{0.964028in}{1.469284in}}{\pgfqpoint{0.968418in}{1.458685in}}{\pgfqpoint{0.976232in}{1.450871in}}%
\pgfpathcurveto{\pgfqpoint{0.984045in}{1.443057in}}{\pgfqpoint{0.994644in}{1.438667in}}{\pgfqpoint{1.005694in}{1.438667in}}%
\pgfpathclose%
\pgfusepath{stroke,fill}%
\end{pgfscope}%
\begin{pgfscope}%
\pgfpathrectangle{\pgfqpoint{0.375000in}{0.330000in}}{\pgfqpoint{2.325000in}{2.310000in}}%
\pgfusepath{clip}%
\pgfsetbuttcap%
\pgfsetroundjoin%
\definecolor{currentfill}{rgb}{0.000000,0.000000,0.000000}%
\pgfsetfillcolor{currentfill}%
\pgfsetlinewidth{1.003750pt}%
\definecolor{currentstroke}{rgb}{0.000000,0.000000,0.000000}%
\pgfsetstrokecolor{currentstroke}%
\pgfsetdash{}{0pt}%
\pgfpathmoveto{\pgfqpoint{1.005694in}{1.438667in}}%
\pgfpathcurveto{\pgfqpoint{1.016744in}{1.438667in}}{\pgfqpoint{1.027343in}{1.443057in}}{\pgfqpoint{1.035157in}{1.450871in}}%
\pgfpathcurveto{\pgfqpoint{1.042971in}{1.458685in}}{\pgfqpoint{1.047361in}{1.469284in}}{\pgfqpoint{1.047361in}{1.480334in}}%
\pgfpathcurveto{\pgfqpoint{1.047361in}{1.491384in}}{\pgfqpoint{1.042971in}{1.501983in}}{\pgfqpoint{1.035157in}{1.509796in}}%
\pgfpathcurveto{\pgfqpoint{1.027343in}{1.517610in}}{\pgfqpoint{1.016744in}{1.522000in}}{\pgfqpoint{1.005694in}{1.522000in}}%
\pgfpathcurveto{\pgfqpoint{0.994644in}{1.522000in}}{\pgfqpoint{0.984045in}{1.517610in}}{\pgfqpoint{0.976232in}{1.509796in}}%
\pgfpathcurveto{\pgfqpoint{0.968418in}{1.501983in}}{\pgfqpoint{0.964028in}{1.491384in}}{\pgfqpoint{0.964028in}{1.480334in}}%
\pgfpathcurveto{\pgfqpoint{0.964028in}{1.469284in}}{\pgfqpoint{0.968418in}{1.458685in}}{\pgfqpoint{0.976232in}{1.450871in}}%
\pgfpathcurveto{\pgfqpoint{0.984045in}{1.443057in}}{\pgfqpoint{0.994644in}{1.438667in}}{\pgfqpoint{1.005694in}{1.438667in}}%
\pgfpathclose%
\pgfusepath{stroke,fill}%
\end{pgfscope}%
\begin{pgfscope}%
\pgfpathrectangle{\pgfqpoint{0.375000in}{0.330000in}}{\pgfqpoint{2.325000in}{2.310000in}}%
\pgfusepath{clip}%
\pgfsetbuttcap%
\pgfsetroundjoin%
\definecolor{currentfill}{rgb}{0.000000,0.000000,0.000000}%
\pgfsetfillcolor{currentfill}%
\pgfsetlinewidth{1.003750pt}%
\definecolor{currentstroke}{rgb}{0.000000,0.000000,0.000000}%
\pgfsetstrokecolor{currentstroke}%
\pgfsetdash{}{0pt}%
\pgfpathmoveto{\pgfqpoint{1.005694in}{1.438667in}}%
\pgfpathcurveto{\pgfqpoint{1.016744in}{1.438667in}}{\pgfqpoint{1.027343in}{1.443057in}}{\pgfqpoint{1.035157in}{1.450871in}}%
\pgfpathcurveto{\pgfqpoint{1.042971in}{1.458685in}}{\pgfqpoint{1.047361in}{1.469284in}}{\pgfqpoint{1.047361in}{1.480334in}}%
\pgfpathcurveto{\pgfqpoint{1.047361in}{1.491384in}}{\pgfqpoint{1.042971in}{1.501983in}}{\pgfqpoint{1.035157in}{1.509796in}}%
\pgfpathcurveto{\pgfqpoint{1.027343in}{1.517610in}}{\pgfqpoint{1.016744in}{1.522000in}}{\pgfqpoint{1.005694in}{1.522000in}}%
\pgfpathcurveto{\pgfqpoint{0.994644in}{1.522000in}}{\pgfqpoint{0.984045in}{1.517610in}}{\pgfqpoint{0.976232in}{1.509796in}}%
\pgfpathcurveto{\pgfqpoint{0.968418in}{1.501983in}}{\pgfqpoint{0.964028in}{1.491384in}}{\pgfqpoint{0.964028in}{1.480334in}}%
\pgfpathcurveto{\pgfqpoint{0.964028in}{1.469284in}}{\pgfqpoint{0.968418in}{1.458685in}}{\pgfqpoint{0.976232in}{1.450871in}}%
\pgfpathcurveto{\pgfqpoint{0.984045in}{1.443057in}}{\pgfqpoint{0.994644in}{1.438667in}}{\pgfqpoint{1.005694in}{1.438667in}}%
\pgfpathclose%
\pgfusepath{stroke,fill}%
\end{pgfscope}%
\begin{pgfscope}%
\pgfpathrectangle{\pgfqpoint{0.375000in}{0.330000in}}{\pgfqpoint{2.325000in}{2.310000in}}%
\pgfusepath{clip}%
\pgfsetbuttcap%
\pgfsetroundjoin%
\definecolor{currentfill}{rgb}{0.000000,0.000000,0.000000}%
\pgfsetfillcolor{currentfill}%
\pgfsetlinewidth{1.003750pt}%
\definecolor{currentstroke}{rgb}{0.000000,0.000000,0.000000}%
\pgfsetstrokecolor{currentstroke}%
\pgfsetdash{}{0pt}%
\pgfpathmoveto{\pgfqpoint{1.005694in}{1.438667in}}%
\pgfpathcurveto{\pgfqpoint{1.016744in}{1.438667in}}{\pgfqpoint{1.027343in}{1.443057in}}{\pgfqpoint{1.035157in}{1.450871in}}%
\pgfpathcurveto{\pgfqpoint{1.042971in}{1.458685in}}{\pgfqpoint{1.047361in}{1.469284in}}{\pgfqpoint{1.047361in}{1.480334in}}%
\pgfpathcurveto{\pgfqpoint{1.047361in}{1.491384in}}{\pgfqpoint{1.042971in}{1.501983in}}{\pgfqpoint{1.035157in}{1.509796in}}%
\pgfpathcurveto{\pgfqpoint{1.027343in}{1.517610in}}{\pgfqpoint{1.016744in}{1.522000in}}{\pgfqpoint{1.005694in}{1.522000in}}%
\pgfpathcurveto{\pgfqpoint{0.994644in}{1.522000in}}{\pgfqpoint{0.984045in}{1.517610in}}{\pgfqpoint{0.976232in}{1.509796in}}%
\pgfpathcurveto{\pgfqpoint{0.968418in}{1.501983in}}{\pgfqpoint{0.964028in}{1.491384in}}{\pgfqpoint{0.964028in}{1.480334in}}%
\pgfpathcurveto{\pgfqpoint{0.964028in}{1.469284in}}{\pgfqpoint{0.968418in}{1.458685in}}{\pgfqpoint{0.976232in}{1.450871in}}%
\pgfpathcurveto{\pgfqpoint{0.984045in}{1.443057in}}{\pgfqpoint{0.994644in}{1.438667in}}{\pgfqpoint{1.005694in}{1.438667in}}%
\pgfpathclose%
\pgfusepath{stroke,fill}%
\end{pgfscope}%
\begin{pgfscope}%
\pgfpathrectangle{\pgfqpoint{0.375000in}{0.330000in}}{\pgfqpoint{2.325000in}{2.310000in}}%
\pgfusepath{clip}%
\pgfsetbuttcap%
\pgfsetroundjoin%
\definecolor{currentfill}{rgb}{0.000000,0.000000,0.000000}%
\pgfsetfillcolor{currentfill}%
\pgfsetlinewidth{1.003750pt}%
\definecolor{currentstroke}{rgb}{0.000000,0.000000,0.000000}%
\pgfsetstrokecolor{currentstroke}%
\pgfsetdash{}{0pt}%
\pgfpathmoveto{\pgfqpoint{1.005694in}{1.438667in}}%
\pgfpathcurveto{\pgfqpoint{1.016744in}{1.438667in}}{\pgfqpoint{1.027343in}{1.443057in}}{\pgfqpoint{1.035157in}{1.450871in}}%
\pgfpathcurveto{\pgfqpoint{1.042971in}{1.458685in}}{\pgfqpoint{1.047361in}{1.469284in}}{\pgfqpoint{1.047361in}{1.480334in}}%
\pgfpathcurveto{\pgfqpoint{1.047361in}{1.491384in}}{\pgfqpoint{1.042971in}{1.501983in}}{\pgfqpoint{1.035157in}{1.509796in}}%
\pgfpathcurveto{\pgfqpoint{1.027343in}{1.517610in}}{\pgfqpoint{1.016744in}{1.522000in}}{\pgfqpoint{1.005694in}{1.522000in}}%
\pgfpathcurveto{\pgfqpoint{0.994644in}{1.522000in}}{\pgfqpoint{0.984045in}{1.517610in}}{\pgfqpoint{0.976232in}{1.509796in}}%
\pgfpathcurveto{\pgfqpoint{0.968418in}{1.501983in}}{\pgfqpoint{0.964028in}{1.491384in}}{\pgfqpoint{0.964028in}{1.480334in}}%
\pgfpathcurveto{\pgfqpoint{0.964028in}{1.469284in}}{\pgfqpoint{0.968418in}{1.458685in}}{\pgfqpoint{0.976232in}{1.450871in}}%
\pgfpathcurveto{\pgfqpoint{0.984045in}{1.443057in}}{\pgfqpoint{0.994644in}{1.438667in}}{\pgfqpoint{1.005694in}{1.438667in}}%
\pgfpathclose%
\pgfusepath{stroke,fill}%
\end{pgfscope}%
\begin{pgfscope}%
\pgfpathrectangle{\pgfqpoint{0.375000in}{0.330000in}}{\pgfqpoint{2.325000in}{2.310000in}}%
\pgfusepath{clip}%
\pgfsetbuttcap%
\pgfsetroundjoin%
\definecolor{currentfill}{rgb}{0.000000,0.000000,0.000000}%
\pgfsetfillcolor{currentfill}%
\pgfsetlinewidth{1.003750pt}%
\definecolor{currentstroke}{rgb}{0.000000,0.000000,0.000000}%
\pgfsetstrokecolor{currentstroke}%
\pgfsetdash{}{0pt}%
\pgfpathmoveto{\pgfqpoint{1.005694in}{1.438667in}}%
\pgfpathcurveto{\pgfqpoint{1.016744in}{1.438667in}}{\pgfqpoint{1.027343in}{1.443057in}}{\pgfqpoint{1.035157in}{1.450871in}}%
\pgfpathcurveto{\pgfqpoint{1.042971in}{1.458685in}}{\pgfqpoint{1.047361in}{1.469284in}}{\pgfqpoint{1.047361in}{1.480334in}}%
\pgfpathcurveto{\pgfqpoint{1.047361in}{1.491384in}}{\pgfqpoint{1.042971in}{1.501983in}}{\pgfqpoint{1.035157in}{1.509796in}}%
\pgfpathcurveto{\pgfqpoint{1.027343in}{1.517610in}}{\pgfqpoint{1.016744in}{1.522000in}}{\pgfqpoint{1.005694in}{1.522000in}}%
\pgfpathcurveto{\pgfqpoint{0.994644in}{1.522000in}}{\pgfqpoint{0.984045in}{1.517610in}}{\pgfqpoint{0.976232in}{1.509796in}}%
\pgfpathcurveto{\pgfqpoint{0.968418in}{1.501983in}}{\pgfqpoint{0.964028in}{1.491384in}}{\pgfqpoint{0.964028in}{1.480334in}}%
\pgfpathcurveto{\pgfqpoint{0.964028in}{1.469284in}}{\pgfqpoint{0.968418in}{1.458685in}}{\pgfqpoint{0.976232in}{1.450871in}}%
\pgfpathcurveto{\pgfqpoint{0.984045in}{1.443057in}}{\pgfqpoint{0.994644in}{1.438667in}}{\pgfqpoint{1.005694in}{1.438667in}}%
\pgfpathclose%
\pgfusepath{stroke,fill}%
\end{pgfscope}%
\begin{pgfscope}%
\pgfpathrectangle{\pgfqpoint{0.375000in}{0.330000in}}{\pgfqpoint{2.325000in}{2.310000in}}%
\pgfusepath{clip}%
\pgfsetbuttcap%
\pgfsetroundjoin%
\definecolor{currentfill}{rgb}{0.000000,0.000000,0.000000}%
\pgfsetfillcolor{currentfill}%
\pgfsetlinewidth{1.003750pt}%
\definecolor{currentstroke}{rgb}{0.000000,0.000000,0.000000}%
\pgfsetstrokecolor{currentstroke}%
\pgfsetdash{}{0pt}%
\pgfpathmoveto{\pgfqpoint{1.005694in}{1.438667in}}%
\pgfpathcurveto{\pgfqpoint{1.016744in}{1.438667in}}{\pgfqpoint{1.027343in}{1.443057in}}{\pgfqpoint{1.035157in}{1.450871in}}%
\pgfpathcurveto{\pgfqpoint{1.042971in}{1.458685in}}{\pgfqpoint{1.047361in}{1.469284in}}{\pgfqpoint{1.047361in}{1.480334in}}%
\pgfpathcurveto{\pgfqpoint{1.047361in}{1.491384in}}{\pgfqpoint{1.042971in}{1.501983in}}{\pgfqpoint{1.035157in}{1.509796in}}%
\pgfpathcurveto{\pgfqpoint{1.027343in}{1.517610in}}{\pgfqpoint{1.016744in}{1.522000in}}{\pgfqpoint{1.005694in}{1.522000in}}%
\pgfpathcurveto{\pgfqpoint{0.994644in}{1.522000in}}{\pgfqpoint{0.984045in}{1.517610in}}{\pgfqpoint{0.976232in}{1.509796in}}%
\pgfpathcurveto{\pgfqpoint{0.968418in}{1.501983in}}{\pgfqpoint{0.964028in}{1.491384in}}{\pgfqpoint{0.964028in}{1.480334in}}%
\pgfpathcurveto{\pgfqpoint{0.964028in}{1.469284in}}{\pgfqpoint{0.968418in}{1.458685in}}{\pgfqpoint{0.976232in}{1.450871in}}%
\pgfpathcurveto{\pgfqpoint{0.984045in}{1.443057in}}{\pgfqpoint{0.994644in}{1.438667in}}{\pgfqpoint{1.005694in}{1.438667in}}%
\pgfpathclose%
\pgfusepath{stroke,fill}%
\end{pgfscope}%
\begin{pgfscope}%
\pgfpathrectangle{\pgfqpoint{0.375000in}{0.330000in}}{\pgfqpoint{2.325000in}{2.310000in}}%
\pgfusepath{clip}%
\pgfsetbuttcap%
\pgfsetroundjoin%
\definecolor{currentfill}{rgb}{0.000000,0.000000,0.000000}%
\pgfsetfillcolor{currentfill}%
\pgfsetlinewidth{1.003750pt}%
\definecolor{currentstroke}{rgb}{0.000000,0.000000,0.000000}%
\pgfsetstrokecolor{currentstroke}%
\pgfsetdash{}{0pt}%
\pgfpathmoveto{\pgfqpoint{1.005694in}{1.438667in}}%
\pgfpathcurveto{\pgfqpoint{1.016744in}{1.438667in}}{\pgfqpoint{1.027343in}{1.443057in}}{\pgfqpoint{1.035157in}{1.450871in}}%
\pgfpathcurveto{\pgfqpoint{1.042971in}{1.458685in}}{\pgfqpoint{1.047361in}{1.469284in}}{\pgfqpoint{1.047361in}{1.480334in}}%
\pgfpathcurveto{\pgfqpoint{1.047361in}{1.491384in}}{\pgfqpoint{1.042971in}{1.501983in}}{\pgfqpoint{1.035157in}{1.509796in}}%
\pgfpathcurveto{\pgfqpoint{1.027343in}{1.517610in}}{\pgfqpoint{1.016744in}{1.522000in}}{\pgfqpoint{1.005694in}{1.522000in}}%
\pgfpathcurveto{\pgfqpoint{0.994644in}{1.522000in}}{\pgfqpoint{0.984045in}{1.517610in}}{\pgfqpoint{0.976232in}{1.509796in}}%
\pgfpathcurveto{\pgfqpoint{0.968418in}{1.501983in}}{\pgfqpoint{0.964028in}{1.491384in}}{\pgfqpoint{0.964028in}{1.480334in}}%
\pgfpathcurveto{\pgfqpoint{0.964028in}{1.469284in}}{\pgfqpoint{0.968418in}{1.458685in}}{\pgfqpoint{0.976232in}{1.450871in}}%
\pgfpathcurveto{\pgfqpoint{0.984045in}{1.443057in}}{\pgfqpoint{0.994644in}{1.438667in}}{\pgfqpoint{1.005694in}{1.438667in}}%
\pgfpathclose%
\pgfusepath{stroke,fill}%
\end{pgfscope}%
\begin{pgfscope}%
\pgfpathrectangle{\pgfqpoint{0.375000in}{0.330000in}}{\pgfqpoint{2.325000in}{2.310000in}}%
\pgfusepath{clip}%
\pgfsetbuttcap%
\pgfsetroundjoin%
\definecolor{currentfill}{rgb}{0.000000,0.000000,0.000000}%
\pgfsetfillcolor{currentfill}%
\pgfsetlinewidth{1.003750pt}%
\definecolor{currentstroke}{rgb}{0.000000,0.000000,0.000000}%
\pgfsetstrokecolor{currentstroke}%
\pgfsetdash{}{0pt}%
\pgfpathmoveto{\pgfqpoint{1.005694in}{1.438667in}}%
\pgfpathcurveto{\pgfqpoint{1.016744in}{1.438667in}}{\pgfqpoint{1.027343in}{1.443057in}}{\pgfqpoint{1.035157in}{1.450871in}}%
\pgfpathcurveto{\pgfqpoint{1.042971in}{1.458685in}}{\pgfqpoint{1.047361in}{1.469284in}}{\pgfqpoint{1.047361in}{1.480334in}}%
\pgfpathcurveto{\pgfqpoint{1.047361in}{1.491384in}}{\pgfqpoint{1.042971in}{1.501983in}}{\pgfqpoint{1.035157in}{1.509796in}}%
\pgfpathcurveto{\pgfqpoint{1.027343in}{1.517610in}}{\pgfqpoint{1.016744in}{1.522000in}}{\pgfqpoint{1.005694in}{1.522000in}}%
\pgfpathcurveto{\pgfqpoint{0.994644in}{1.522000in}}{\pgfqpoint{0.984045in}{1.517610in}}{\pgfqpoint{0.976232in}{1.509796in}}%
\pgfpathcurveto{\pgfqpoint{0.968418in}{1.501983in}}{\pgfqpoint{0.964028in}{1.491384in}}{\pgfqpoint{0.964028in}{1.480334in}}%
\pgfpathcurveto{\pgfqpoint{0.964028in}{1.469284in}}{\pgfqpoint{0.968418in}{1.458685in}}{\pgfqpoint{0.976232in}{1.450871in}}%
\pgfpathcurveto{\pgfqpoint{0.984045in}{1.443057in}}{\pgfqpoint{0.994644in}{1.438667in}}{\pgfqpoint{1.005694in}{1.438667in}}%
\pgfpathclose%
\pgfusepath{stroke,fill}%
\end{pgfscope}%
\begin{pgfscope}%
\pgfpathrectangle{\pgfqpoint{0.375000in}{0.330000in}}{\pgfqpoint{2.325000in}{2.310000in}}%
\pgfusepath{clip}%
\pgfsetbuttcap%
\pgfsetroundjoin%
\definecolor{currentfill}{rgb}{0.000000,0.000000,0.000000}%
\pgfsetfillcolor{currentfill}%
\pgfsetlinewidth{1.003750pt}%
\definecolor{currentstroke}{rgb}{0.000000,0.000000,0.000000}%
\pgfsetstrokecolor{currentstroke}%
\pgfsetdash{}{0pt}%
\pgfpathmoveto{\pgfqpoint{1.005694in}{1.438667in}}%
\pgfpathcurveto{\pgfqpoint{1.016744in}{1.438667in}}{\pgfqpoint{1.027343in}{1.443057in}}{\pgfqpoint{1.035157in}{1.450871in}}%
\pgfpathcurveto{\pgfqpoint{1.042971in}{1.458685in}}{\pgfqpoint{1.047361in}{1.469284in}}{\pgfqpoint{1.047361in}{1.480334in}}%
\pgfpathcurveto{\pgfqpoint{1.047361in}{1.491384in}}{\pgfqpoint{1.042971in}{1.501983in}}{\pgfqpoint{1.035157in}{1.509796in}}%
\pgfpathcurveto{\pgfqpoint{1.027343in}{1.517610in}}{\pgfqpoint{1.016744in}{1.522000in}}{\pgfqpoint{1.005694in}{1.522000in}}%
\pgfpathcurveto{\pgfqpoint{0.994644in}{1.522000in}}{\pgfqpoint{0.984045in}{1.517610in}}{\pgfqpoint{0.976232in}{1.509796in}}%
\pgfpathcurveto{\pgfqpoint{0.968418in}{1.501983in}}{\pgfqpoint{0.964028in}{1.491384in}}{\pgfqpoint{0.964028in}{1.480334in}}%
\pgfpathcurveto{\pgfqpoint{0.964028in}{1.469284in}}{\pgfqpoint{0.968418in}{1.458685in}}{\pgfqpoint{0.976232in}{1.450871in}}%
\pgfpathcurveto{\pgfqpoint{0.984045in}{1.443057in}}{\pgfqpoint{0.994644in}{1.438667in}}{\pgfqpoint{1.005694in}{1.438667in}}%
\pgfpathclose%
\pgfusepath{stroke,fill}%
\end{pgfscope}%
\begin{pgfscope}%
\pgfpathrectangle{\pgfqpoint{0.375000in}{0.330000in}}{\pgfqpoint{2.325000in}{2.310000in}}%
\pgfusepath{clip}%
\pgfsetbuttcap%
\pgfsetroundjoin%
\definecolor{currentfill}{rgb}{0.000000,0.000000,0.000000}%
\pgfsetfillcolor{currentfill}%
\pgfsetlinewidth{1.003750pt}%
\definecolor{currentstroke}{rgb}{0.000000,0.000000,0.000000}%
\pgfsetstrokecolor{currentstroke}%
\pgfsetdash{}{0pt}%
\pgfpathmoveto{\pgfqpoint{1.005694in}{1.438667in}}%
\pgfpathcurveto{\pgfqpoint{1.016744in}{1.438667in}}{\pgfqpoint{1.027343in}{1.443057in}}{\pgfqpoint{1.035157in}{1.450871in}}%
\pgfpathcurveto{\pgfqpoint{1.042971in}{1.458685in}}{\pgfqpoint{1.047361in}{1.469284in}}{\pgfqpoint{1.047361in}{1.480334in}}%
\pgfpathcurveto{\pgfqpoint{1.047361in}{1.491384in}}{\pgfqpoint{1.042971in}{1.501983in}}{\pgfqpoint{1.035157in}{1.509796in}}%
\pgfpathcurveto{\pgfqpoint{1.027343in}{1.517610in}}{\pgfqpoint{1.016744in}{1.522000in}}{\pgfqpoint{1.005694in}{1.522000in}}%
\pgfpathcurveto{\pgfqpoint{0.994644in}{1.522000in}}{\pgfqpoint{0.984045in}{1.517610in}}{\pgfqpoint{0.976232in}{1.509796in}}%
\pgfpathcurveto{\pgfqpoint{0.968418in}{1.501983in}}{\pgfqpoint{0.964028in}{1.491384in}}{\pgfqpoint{0.964028in}{1.480334in}}%
\pgfpathcurveto{\pgfqpoint{0.964028in}{1.469284in}}{\pgfqpoint{0.968418in}{1.458685in}}{\pgfqpoint{0.976232in}{1.450871in}}%
\pgfpathcurveto{\pgfqpoint{0.984045in}{1.443057in}}{\pgfqpoint{0.994644in}{1.438667in}}{\pgfqpoint{1.005694in}{1.438667in}}%
\pgfpathclose%
\pgfusepath{stroke,fill}%
\end{pgfscope}%
\begin{pgfscope}%
\pgfpathrectangle{\pgfqpoint{0.375000in}{0.330000in}}{\pgfqpoint{2.325000in}{2.310000in}}%
\pgfusepath{clip}%
\pgfsetbuttcap%
\pgfsetroundjoin%
\definecolor{currentfill}{rgb}{0.000000,0.000000,0.000000}%
\pgfsetfillcolor{currentfill}%
\pgfsetlinewidth{1.003750pt}%
\definecolor{currentstroke}{rgb}{0.000000,0.000000,0.000000}%
\pgfsetstrokecolor{currentstroke}%
\pgfsetdash{}{0pt}%
\pgfpathmoveto{\pgfqpoint{1.005694in}{1.438667in}}%
\pgfpathcurveto{\pgfqpoint{1.016744in}{1.438667in}}{\pgfqpoint{1.027343in}{1.443057in}}{\pgfqpoint{1.035157in}{1.450871in}}%
\pgfpathcurveto{\pgfqpoint{1.042971in}{1.458685in}}{\pgfqpoint{1.047361in}{1.469284in}}{\pgfqpoint{1.047361in}{1.480334in}}%
\pgfpathcurveto{\pgfqpoint{1.047361in}{1.491384in}}{\pgfqpoint{1.042971in}{1.501983in}}{\pgfqpoint{1.035157in}{1.509796in}}%
\pgfpathcurveto{\pgfqpoint{1.027343in}{1.517610in}}{\pgfqpoint{1.016744in}{1.522000in}}{\pgfqpoint{1.005694in}{1.522000in}}%
\pgfpathcurveto{\pgfqpoint{0.994644in}{1.522000in}}{\pgfqpoint{0.984045in}{1.517610in}}{\pgfqpoint{0.976232in}{1.509796in}}%
\pgfpathcurveto{\pgfqpoint{0.968418in}{1.501983in}}{\pgfqpoint{0.964028in}{1.491384in}}{\pgfqpoint{0.964028in}{1.480334in}}%
\pgfpathcurveto{\pgfqpoint{0.964028in}{1.469284in}}{\pgfqpoint{0.968418in}{1.458685in}}{\pgfqpoint{0.976232in}{1.450871in}}%
\pgfpathcurveto{\pgfqpoint{0.984045in}{1.443057in}}{\pgfqpoint{0.994644in}{1.438667in}}{\pgfqpoint{1.005694in}{1.438667in}}%
\pgfpathclose%
\pgfusepath{stroke,fill}%
\end{pgfscope}%
\begin{pgfscope}%
\pgfpathrectangle{\pgfqpoint{0.375000in}{0.330000in}}{\pgfqpoint{2.325000in}{2.310000in}}%
\pgfusepath{clip}%
\pgfsetbuttcap%
\pgfsetroundjoin%
\definecolor{currentfill}{rgb}{0.000000,0.000000,0.000000}%
\pgfsetfillcolor{currentfill}%
\pgfsetlinewidth{1.003750pt}%
\definecolor{currentstroke}{rgb}{0.000000,0.000000,0.000000}%
\pgfsetstrokecolor{currentstroke}%
\pgfsetdash{}{0pt}%
\pgfpathmoveto{\pgfqpoint{1.005694in}{1.438667in}}%
\pgfpathcurveto{\pgfqpoint{1.016744in}{1.438667in}}{\pgfqpoint{1.027343in}{1.443057in}}{\pgfqpoint{1.035157in}{1.450871in}}%
\pgfpathcurveto{\pgfqpoint{1.042971in}{1.458685in}}{\pgfqpoint{1.047361in}{1.469284in}}{\pgfqpoint{1.047361in}{1.480334in}}%
\pgfpathcurveto{\pgfqpoint{1.047361in}{1.491384in}}{\pgfqpoint{1.042971in}{1.501983in}}{\pgfqpoint{1.035157in}{1.509796in}}%
\pgfpathcurveto{\pgfqpoint{1.027343in}{1.517610in}}{\pgfqpoint{1.016744in}{1.522000in}}{\pgfqpoint{1.005694in}{1.522000in}}%
\pgfpathcurveto{\pgfqpoint{0.994644in}{1.522000in}}{\pgfqpoint{0.984045in}{1.517610in}}{\pgfqpoint{0.976232in}{1.509796in}}%
\pgfpathcurveto{\pgfqpoint{0.968418in}{1.501983in}}{\pgfqpoint{0.964028in}{1.491384in}}{\pgfqpoint{0.964028in}{1.480334in}}%
\pgfpathcurveto{\pgfqpoint{0.964028in}{1.469284in}}{\pgfqpoint{0.968418in}{1.458685in}}{\pgfqpoint{0.976232in}{1.450871in}}%
\pgfpathcurveto{\pgfqpoint{0.984045in}{1.443057in}}{\pgfqpoint{0.994644in}{1.438667in}}{\pgfqpoint{1.005694in}{1.438667in}}%
\pgfpathclose%
\pgfusepath{stroke,fill}%
\end{pgfscope}%
\begin{pgfscope}%
\pgfpathrectangle{\pgfqpoint{0.375000in}{0.330000in}}{\pgfqpoint{2.325000in}{2.310000in}}%
\pgfusepath{clip}%
\pgfsetbuttcap%
\pgfsetroundjoin%
\definecolor{currentfill}{rgb}{0.000000,0.000000,0.000000}%
\pgfsetfillcolor{currentfill}%
\pgfsetlinewidth{1.003750pt}%
\definecolor{currentstroke}{rgb}{0.000000,0.000000,0.000000}%
\pgfsetstrokecolor{currentstroke}%
\pgfsetdash{}{0pt}%
\pgfpathmoveto{\pgfqpoint{1.005694in}{1.438667in}}%
\pgfpathcurveto{\pgfqpoint{1.016744in}{1.438667in}}{\pgfqpoint{1.027343in}{1.443057in}}{\pgfqpoint{1.035157in}{1.450871in}}%
\pgfpathcurveto{\pgfqpoint{1.042971in}{1.458685in}}{\pgfqpoint{1.047361in}{1.469284in}}{\pgfqpoint{1.047361in}{1.480334in}}%
\pgfpathcurveto{\pgfqpoint{1.047361in}{1.491384in}}{\pgfqpoint{1.042971in}{1.501983in}}{\pgfqpoint{1.035157in}{1.509796in}}%
\pgfpathcurveto{\pgfqpoint{1.027343in}{1.517610in}}{\pgfqpoint{1.016744in}{1.522000in}}{\pgfqpoint{1.005694in}{1.522000in}}%
\pgfpathcurveto{\pgfqpoint{0.994644in}{1.522000in}}{\pgfqpoint{0.984045in}{1.517610in}}{\pgfqpoint{0.976232in}{1.509796in}}%
\pgfpathcurveto{\pgfqpoint{0.968418in}{1.501983in}}{\pgfqpoint{0.964028in}{1.491384in}}{\pgfqpoint{0.964028in}{1.480334in}}%
\pgfpathcurveto{\pgfqpoint{0.964028in}{1.469284in}}{\pgfqpoint{0.968418in}{1.458685in}}{\pgfqpoint{0.976232in}{1.450871in}}%
\pgfpathcurveto{\pgfqpoint{0.984045in}{1.443057in}}{\pgfqpoint{0.994644in}{1.438667in}}{\pgfqpoint{1.005694in}{1.438667in}}%
\pgfpathclose%
\pgfusepath{stroke,fill}%
\end{pgfscope}%
\begin{pgfscope}%
\pgfpathrectangle{\pgfqpoint{0.375000in}{0.330000in}}{\pgfqpoint{2.325000in}{2.310000in}}%
\pgfusepath{clip}%
\pgfsetbuttcap%
\pgfsetroundjoin%
\definecolor{currentfill}{rgb}{0.000000,0.000000,0.000000}%
\pgfsetfillcolor{currentfill}%
\pgfsetlinewidth{1.003750pt}%
\definecolor{currentstroke}{rgb}{0.000000,0.000000,0.000000}%
\pgfsetstrokecolor{currentstroke}%
\pgfsetdash{}{0pt}%
\pgfpathmoveto{\pgfqpoint{1.005694in}{1.438667in}}%
\pgfpathcurveto{\pgfqpoint{1.016744in}{1.438667in}}{\pgfqpoint{1.027343in}{1.443057in}}{\pgfqpoint{1.035157in}{1.450871in}}%
\pgfpathcurveto{\pgfqpoint{1.042971in}{1.458685in}}{\pgfqpoint{1.047361in}{1.469284in}}{\pgfqpoint{1.047361in}{1.480334in}}%
\pgfpathcurveto{\pgfqpoint{1.047361in}{1.491384in}}{\pgfqpoint{1.042971in}{1.501983in}}{\pgfqpoint{1.035157in}{1.509796in}}%
\pgfpathcurveto{\pgfqpoint{1.027343in}{1.517610in}}{\pgfqpoint{1.016744in}{1.522000in}}{\pgfqpoint{1.005694in}{1.522000in}}%
\pgfpathcurveto{\pgfqpoint{0.994644in}{1.522000in}}{\pgfqpoint{0.984045in}{1.517610in}}{\pgfqpoint{0.976232in}{1.509796in}}%
\pgfpathcurveto{\pgfqpoint{0.968418in}{1.501983in}}{\pgfqpoint{0.964028in}{1.491384in}}{\pgfqpoint{0.964028in}{1.480334in}}%
\pgfpathcurveto{\pgfqpoint{0.964028in}{1.469284in}}{\pgfqpoint{0.968418in}{1.458685in}}{\pgfqpoint{0.976232in}{1.450871in}}%
\pgfpathcurveto{\pgfqpoint{0.984045in}{1.443057in}}{\pgfqpoint{0.994644in}{1.438667in}}{\pgfqpoint{1.005694in}{1.438667in}}%
\pgfpathclose%
\pgfusepath{stroke,fill}%
\end{pgfscope}%
\begin{pgfscope}%
\pgfpathrectangle{\pgfqpoint{0.375000in}{0.330000in}}{\pgfqpoint{2.325000in}{2.310000in}}%
\pgfusepath{clip}%
\pgfsetbuttcap%
\pgfsetroundjoin%
\definecolor{currentfill}{rgb}{0.000000,0.000000,0.000000}%
\pgfsetfillcolor{currentfill}%
\pgfsetlinewidth{1.003750pt}%
\definecolor{currentstroke}{rgb}{0.000000,0.000000,0.000000}%
\pgfsetstrokecolor{currentstroke}%
\pgfsetdash{}{0pt}%
\pgfpathmoveto{\pgfqpoint{1.005694in}{1.438667in}}%
\pgfpathcurveto{\pgfqpoint{1.016744in}{1.438667in}}{\pgfqpoint{1.027343in}{1.443057in}}{\pgfqpoint{1.035157in}{1.450871in}}%
\pgfpathcurveto{\pgfqpoint{1.042971in}{1.458685in}}{\pgfqpoint{1.047361in}{1.469284in}}{\pgfqpoint{1.047361in}{1.480334in}}%
\pgfpathcurveto{\pgfqpoint{1.047361in}{1.491384in}}{\pgfqpoint{1.042971in}{1.501983in}}{\pgfqpoint{1.035157in}{1.509796in}}%
\pgfpathcurveto{\pgfqpoint{1.027343in}{1.517610in}}{\pgfqpoint{1.016744in}{1.522000in}}{\pgfqpoint{1.005694in}{1.522000in}}%
\pgfpathcurveto{\pgfqpoint{0.994644in}{1.522000in}}{\pgfqpoint{0.984045in}{1.517610in}}{\pgfqpoint{0.976232in}{1.509796in}}%
\pgfpathcurveto{\pgfqpoint{0.968418in}{1.501983in}}{\pgfqpoint{0.964028in}{1.491384in}}{\pgfqpoint{0.964028in}{1.480334in}}%
\pgfpathcurveto{\pgfqpoint{0.964028in}{1.469284in}}{\pgfqpoint{0.968418in}{1.458685in}}{\pgfqpoint{0.976232in}{1.450871in}}%
\pgfpathcurveto{\pgfqpoint{0.984045in}{1.443057in}}{\pgfqpoint{0.994644in}{1.438667in}}{\pgfqpoint{1.005694in}{1.438667in}}%
\pgfpathclose%
\pgfusepath{stroke,fill}%
\end{pgfscope}%
\begin{pgfscope}%
\pgfpathrectangle{\pgfqpoint{0.375000in}{0.330000in}}{\pgfqpoint{2.325000in}{2.310000in}}%
\pgfusepath{clip}%
\pgfsetbuttcap%
\pgfsetroundjoin%
\definecolor{currentfill}{rgb}{0.000000,0.000000,0.000000}%
\pgfsetfillcolor{currentfill}%
\pgfsetlinewidth{1.003750pt}%
\definecolor{currentstroke}{rgb}{0.000000,0.000000,0.000000}%
\pgfsetstrokecolor{currentstroke}%
\pgfsetdash{}{0pt}%
\pgfpathmoveto{\pgfqpoint{1.005694in}{1.438667in}}%
\pgfpathcurveto{\pgfqpoint{1.016744in}{1.438667in}}{\pgfqpoint{1.027343in}{1.443057in}}{\pgfqpoint{1.035157in}{1.450871in}}%
\pgfpathcurveto{\pgfqpoint{1.042971in}{1.458685in}}{\pgfqpoint{1.047361in}{1.469284in}}{\pgfqpoint{1.047361in}{1.480334in}}%
\pgfpathcurveto{\pgfqpoint{1.047361in}{1.491384in}}{\pgfqpoint{1.042971in}{1.501983in}}{\pgfqpoint{1.035157in}{1.509796in}}%
\pgfpathcurveto{\pgfqpoint{1.027343in}{1.517610in}}{\pgfqpoint{1.016744in}{1.522000in}}{\pgfqpoint{1.005694in}{1.522000in}}%
\pgfpathcurveto{\pgfqpoint{0.994644in}{1.522000in}}{\pgfqpoint{0.984045in}{1.517610in}}{\pgfqpoint{0.976232in}{1.509796in}}%
\pgfpathcurveto{\pgfqpoint{0.968418in}{1.501983in}}{\pgfqpoint{0.964028in}{1.491384in}}{\pgfqpoint{0.964028in}{1.480334in}}%
\pgfpathcurveto{\pgfqpoint{0.964028in}{1.469284in}}{\pgfqpoint{0.968418in}{1.458685in}}{\pgfqpoint{0.976232in}{1.450871in}}%
\pgfpathcurveto{\pgfqpoint{0.984045in}{1.443057in}}{\pgfqpoint{0.994644in}{1.438667in}}{\pgfqpoint{1.005694in}{1.438667in}}%
\pgfpathclose%
\pgfusepath{stroke,fill}%
\end{pgfscope}%
\begin{pgfscope}%
\pgfpathrectangle{\pgfqpoint{0.375000in}{0.330000in}}{\pgfqpoint{2.325000in}{2.310000in}}%
\pgfusepath{clip}%
\pgfsetbuttcap%
\pgfsetroundjoin%
\definecolor{currentfill}{rgb}{0.000000,0.000000,0.000000}%
\pgfsetfillcolor{currentfill}%
\pgfsetlinewidth{1.003750pt}%
\definecolor{currentstroke}{rgb}{0.000000,0.000000,0.000000}%
\pgfsetstrokecolor{currentstroke}%
\pgfsetdash{}{0pt}%
\pgfpathmoveto{\pgfqpoint{1.005694in}{1.438667in}}%
\pgfpathcurveto{\pgfqpoint{1.016744in}{1.438667in}}{\pgfqpoint{1.027343in}{1.443057in}}{\pgfqpoint{1.035157in}{1.450871in}}%
\pgfpathcurveto{\pgfqpoint{1.042971in}{1.458685in}}{\pgfqpoint{1.047361in}{1.469284in}}{\pgfqpoint{1.047361in}{1.480334in}}%
\pgfpathcurveto{\pgfqpoint{1.047361in}{1.491384in}}{\pgfqpoint{1.042971in}{1.501983in}}{\pgfqpoint{1.035157in}{1.509796in}}%
\pgfpathcurveto{\pgfqpoint{1.027343in}{1.517610in}}{\pgfqpoint{1.016744in}{1.522000in}}{\pgfqpoint{1.005694in}{1.522000in}}%
\pgfpathcurveto{\pgfqpoint{0.994644in}{1.522000in}}{\pgfqpoint{0.984045in}{1.517610in}}{\pgfqpoint{0.976232in}{1.509796in}}%
\pgfpathcurveto{\pgfqpoint{0.968418in}{1.501983in}}{\pgfqpoint{0.964028in}{1.491384in}}{\pgfqpoint{0.964028in}{1.480334in}}%
\pgfpathcurveto{\pgfqpoint{0.964028in}{1.469284in}}{\pgfqpoint{0.968418in}{1.458685in}}{\pgfqpoint{0.976232in}{1.450871in}}%
\pgfpathcurveto{\pgfqpoint{0.984045in}{1.443057in}}{\pgfqpoint{0.994644in}{1.438667in}}{\pgfqpoint{1.005694in}{1.438667in}}%
\pgfpathclose%
\pgfusepath{stroke,fill}%
\end{pgfscope}%
\begin{pgfscope}%
\pgfpathrectangle{\pgfqpoint{0.375000in}{0.330000in}}{\pgfqpoint{2.325000in}{2.310000in}}%
\pgfusepath{clip}%
\pgfsetbuttcap%
\pgfsetroundjoin%
\definecolor{currentfill}{rgb}{0.000000,0.000000,0.000000}%
\pgfsetfillcolor{currentfill}%
\pgfsetlinewidth{1.003750pt}%
\definecolor{currentstroke}{rgb}{0.000000,0.000000,0.000000}%
\pgfsetstrokecolor{currentstroke}%
\pgfsetdash{}{0pt}%
\pgfpathmoveto{\pgfqpoint{1.005694in}{1.438667in}}%
\pgfpathcurveto{\pgfqpoint{1.016744in}{1.438667in}}{\pgfqpoint{1.027343in}{1.443057in}}{\pgfqpoint{1.035157in}{1.450871in}}%
\pgfpathcurveto{\pgfqpoint{1.042971in}{1.458685in}}{\pgfqpoint{1.047361in}{1.469284in}}{\pgfqpoint{1.047361in}{1.480334in}}%
\pgfpathcurveto{\pgfqpoint{1.047361in}{1.491384in}}{\pgfqpoint{1.042971in}{1.501983in}}{\pgfqpoint{1.035157in}{1.509796in}}%
\pgfpathcurveto{\pgfqpoint{1.027343in}{1.517610in}}{\pgfqpoint{1.016744in}{1.522000in}}{\pgfqpoint{1.005694in}{1.522000in}}%
\pgfpathcurveto{\pgfqpoint{0.994644in}{1.522000in}}{\pgfqpoint{0.984045in}{1.517610in}}{\pgfqpoint{0.976232in}{1.509796in}}%
\pgfpathcurveto{\pgfqpoint{0.968418in}{1.501983in}}{\pgfqpoint{0.964028in}{1.491384in}}{\pgfqpoint{0.964028in}{1.480334in}}%
\pgfpathcurveto{\pgfqpoint{0.964028in}{1.469284in}}{\pgfqpoint{0.968418in}{1.458685in}}{\pgfqpoint{0.976232in}{1.450871in}}%
\pgfpathcurveto{\pgfqpoint{0.984045in}{1.443057in}}{\pgfqpoint{0.994644in}{1.438667in}}{\pgfqpoint{1.005694in}{1.438667in}}%
\pgfpathclose%
\pgfusepath{stroke,fill}%
\end{pgfscope}%
\begin{pgfscope}%
\pgfpathrectangle{\pgfqpoint{0.375000in}{0.330000in}}{\pgfqpoint{2.325000in}{2.310000in}}%
\pgfusepath{clip}%
\pgfsetbuttcap%
\pgfsetroundjoin%
\definecolor{currentfill}{rgb}{0.000000,0.000000,0.000000}%
\pgfsetfillcolor{currentfill}%
\pgfsetlinewidth{1.003750pt}%
\definecolor{currentstroke}{rgb}{0.000000,0.000000,0.000000}%
\pgfsetstrokecolor{currentstroke}%
\pgfsetdash{}{0pt}%
\pgfpathmoveto{\pgfqpoint{1.005694in}{1.438667in}}%
\pgfpathcurveto{\pgfqpoint{1.016744in}{1.438667in}}{\pgfqpoint{1.027343in}{1.443057in}}{\pgfqpoint{1.035157in}{1.450871in}}%
\pgfpathcurveto{\pgfqpoint{1.042971in}{1.458685in}}{\pgfqpoint{1.047361in}{1.469284in}}{\pgfqpoint{1.047361in}{1.480334in}}%
\pgfpathcurveto{\pgfqpoint{1.047361in}{1.491384in}}{\pgfqpoint{1.042971in}{1.501983in}}{\pgfqpoint{1.035157in}{1.509796in}}%
\pgfpathcurveto{\pgfqpoint{1.027343in}{1.517610in}}{\pgfqpoint{1.016744in}{1.522000in}}{\pgfqpoint{1.005694in}{1.522000in}}%
\pgfpathcurveto{\pgfqpoint{0.994644in}{1.522000in}}{\pgfqpoint{0.984045in}{1.517610in}}{\pgfqpoint{0.976232in}{1.509796in}}%
\pgfpathcurveto{\pgfqpoint{0.968418in}{1.501983in}}{\pgfqpoint{0.964028in}{1.491384in}}{\pgfqpoint{0.964028in}{1.480334in}}%
\pgfpathcurveto{\pgfqpoint{0.964028in}{1.469284in}}{\pgfqpoint{0.968418in}{1.458685in}}{\pgfqpoint{0.976232in}{1.450871in}}%
\pgfpathcurveto{\pgfqpoint{0.984045in}{1.443057in}}{\pgfqpoint{0.994644in}{1.438667in}}{\pgfqpoint{1.005694in}{1.438667in}}%
\pgfpathclose%
\pgfusepath{stroke,fill}%
\end{pgfscope}%
\begin{pgfscope}%
\pgfpathrectangle{\pgfqpoint{0.375000in}{0.330000in}}{\pgfqpoint{2.325000in}{2.310000in}}%
\pgfusepath{clip}%
\pgfsetbuttcap%
\pgfsetroundjoin%
\definecolor{currentfill}{rgb}{0.000000,0.000000,0.000000}%
\pgfsetfillcolor{currentfill}%
\pgfsetlinewidth{1.003750pt}%
\definecolor{currentstroke}{rgb}{0.000000,0.000000,0.000000}%
\pgfsetstrokecolor{currentstroke}%
\pgfsetdash{}{0pt}%
\pgfpathmoveto{\pgfqpoint{1.005694in}{1.438667in}}%
\pgfpathcurveto{\pgfqpoint{1.016744in}{1.438667in}}{\pgfqpoint{1.027343in}{1.443057in}}{\pgfqpoint{1.035157in}{1.450871in}}%
\pgfpathcurveto{\pgfqpoint{1.042971in}{1.458685in}}{\pgfqpoint{1.047361in}{1.469284in}}{\pgfqpoint{1.047361in}{1.480334in}}%
\pgfpathcurveto{\pgfqpoint{1.047361in}{1.491384in}}{\pgfqpoint{1.042971in}{1.501983in}}{\pgfqpoint{1.035157in}{1.509796in}}%
\pgfpathcurveto{\pgfqpoint{1.027343in}{1.517610in}}{\pgfqpoint{1.016744in}{1.522000in}}{\pgfqpoint{1.005694in}{1.522000in}}%
\pgfpathcurveto{\pgfqpoint{0.994644in}{1.522000in}}{\pgfqpoint{0.984045in}{1.517610in}}{\pgfqpoint{0.976232in}{1.509796in}}%
\pgfpathcurveto{\pgfqpoint{0.968418in}{1.501983in}}{\pgfqpoint{0.964028in}{1.491384in}}{\pgfqpoint{0.964028in}{1.480334in}}%
\pgfpathcurveto{\pgfqpoint{0.964028in}{1.469284in}}{\pgfqpoint{0.968418in}{1.458685in}}{\pgfqpoint{0.976232in}{1.450871in}}%
\pgfpathcurveto{\pgfqpoint{0.984045in}{1.443057in}}{\pgfqpoint{0.994644in}{1.438667in}}{\pgfqpoint{1.005694in}{1.438667in}}%
\pgfpathclose%
\pgfusepath{stroke,fill}%
\end{pgfscope}%
\begin{pgfscope}%
\pgfpathrectangle{\pgfqpoint{0.375000in}{0.330000in}}{\pgfqpoint{2.325000in}{2.310000in}}%
\pgfusepath{clip}%
\pgfsetbuttcap%
\pgfsetroundjoin%
\definecolor{currentfill}{rgb}{0.000000,0.000000,0.000000}%
\pgfsetfillcolor{currentfill}%
\pgfsetlinewidth{1.003750pt}%
\definecolor{currentstroke}{rgb}{0.000000,0.000000,0.000000}%
\pgfsetstrokecolor{currentstroke}%
\pgfsetdash{}{0pt}%
\pgfpathmoveto{\pgfqpoint{1.005694in}{1.438667in}}%
\pgfpathcurveto{\pgfqpoint{1.016744in}{1.438667in}}{\pgfqpoint{1.027343in}{1.443057in}}{\pgfqpoint{1.035157in}{1.450871in}}%
\pgfpathcurveto{\pgfqpoint{1.042971in}{1.458685in}}{\pgfqpoint{1.047361in}{1.469284in}}{\pgfqpoint{1.047361in}{1.480334in}}%
\pgfpathcurveto{\pgfqpoint{1.047361in}{1.491384in}}{\pgfqpoint{1.042971in}{1.501983in}}{\pgfqpoint{1.035157in}{1.509796in}}%
\pgfpathcurveto{\pgfqpoint{1.027343in}{1.517610in}}{\pgfqpoint{1.016744in}{1.522000in}}{\pgfqpoint{1.005694in}{1.522000in}}%
\pgfpathcurveto{\pgfqpoint{0.994644in}{1.522000in}}{\pgfqpoint{0.984045in}{1.517610in}}{\pgfqpoint{0.976232in}{1.509796in}}%
\pgfpathcurveto{\pgfqpoint{0.968418in}{1.501983in}}{\pgfqpoint{0.964028in}{1.491384in}}{\pgfqpoint{0.964028in}{1.480334in}}%
\pgfpathcurveto{\pgfqpoint{0.964028in}{1.469284in}}{\pgfqpoint{0.968418in}{1.458685in}}{\pgfqpoint{0.976232in}{1.450871in}}%
\pgfpathcurveto{\pgfqpoint{0.984045in}{1.443057in}}{\pgfqpoint{0.994644in}{1.438667in}}{\pgfqpoint{1.005694in}{1.438667in}}%
\pgfpathclose%
\pgfusepath{stroke,fill}%
\end{pgfscope}%
\begin{pgfscope}%
\pgfpathrectangle{\pgfqpoint{0.375000in}{0.330000in}}{\pgfqpoint{2.325000in}{2.310000in}}%
\pgfusepath{clip}%
\pgfsetbuttcap%
\pgfsetroundjoin%
\definecolor{currentfill}{rgb}{0.000000,0.000000,0.000000}%
\pgfsetfillcolor{currentfill}%
\pgfsetlinewidth{1.003750pt}%
\definecolor{currentstroke}{rgb}{0.000000,0.000000,0.000000}%
\pgfsetstrokecolor{currentstroke}%
\pgfsetdash{}{0pt}%
\pgfpathmoveto{\pgfqpoint{1.005694in}{1.438667in}}%
\pgfpathcurveto{\pgfqpoint{1.016744in}{1.438667in}}{\pgfqpoint{1.027343in}{1.443057in}}{\pgfqpoint{1.035157in}{1.450871in}}%
\pgfpathcurveto{\pgfqpoint{1.042971in}{1.458685in}}{\pgfqpoint{1.047361in}{1.469284in}}{\pgfqpoint{1.047361in}{1.480334in}}%
\pgfpathcurveto{\pgfqpoint{1.047361in}{1.491384in}}{\pgfqpoint{1.042971in}{1.501983in}}{\pgfqpoint{1.035157in}{1.509796in}}%
\pgfpathcurveto{\pgfqpoint{1.027343in}{1.517610in}}{\pgfqpoint{1.016744in}{1.522000in}}{\pgfqpoint{1.005694in}{1.522000in}}%
\pgfpathcurveto{\pgfqpoint{0.994644in}{1.522000in}}{\pgfqpoint{0.984045in}{1.517610in}}{\pgfqpoint{0.976232in}{1.509796in}}%
\pgfpathcurveto{\pgfqpoint{0.968418in}{1.501983in}}{\pgfqpoint{0.964028in}{1.491384in}}{\pgfqpoint{0.964028in}{1.480334in}}%
\pgfpathcurveto{\pgfqpoint{0.964028in}{1.469284in}}{\pgfqpoint{0.968418in}{1.458685in}}{\pgfqpoint{0.976232in}{1.450871in}}%
\pgfpathcurveto{\pgfqpoint{0.984045in}{1.443057in}}{\pgfqpoint{0.994644in}{1.438667in}}{\pgfqpoint{1.005694in}{1.438667in}}%
\pgfpathclose%
\pgfusepath{stroke,fill}%
\end{pgfscope}%
\begin{pgfscope}%
\pgfpathrectangle{\pgfqpoint{0.375000in}{0.330000in}}{\pgfqpoint{2.325000in}{2.310000in}}%
\pgfusepath{clip}%
\pgfsetbuttcap%
\pgfsetroundjoin%
\definecolor{currentfill}{rgb}{0.000000,0.000000,0.000000}%
\pgfsetfillcolor{currentfill}%
\pgfsetlinewidth{1.003750pt}%
\definecolor{currentstroke}{rgb}{0.000000,0.000000,0.000000}%
\pgfsetstrokecolor{currentstroke}%
\pgfsetdash{}{0pt}%
\pgfpathmoveto{\pgfqpoint{1.005694in}{1.438667in}}%
\pgfpathcurveto{\pgfqpoint{1.016744in}{1.438667in}}{\pgfqpoint{1.027343in}{1.443057in}}{\pgfqpoint{1.035157in}{1.450871in}}%
\pgfpathcurveto{\pgfqpoint{1.042971in}{1.458685in}}{\pgfqpoint{1.047361in}{1.469284in}}{\pgfqpoint{1.047361in}{1.480334in}}%
\pgfpathcurveto{\pgfqpoint{1.047361in}{1.491384in}}{\pgfqpoint{1.042971in}{1.501983in}}{\pgfqpoint{1.035157in}{1.509796in}}%
\pgfpathcurveto{\pgfqpoint{1.027343in}{1.517610in}}{\pgfqpoint{1.016744in}{1.522000in}}{\pgfqpoint{1.005694in}{1.522000in}}%
\pgfpathcurveto{\pgfqpoint{0.994644in}{1.522000in}}{\pgfqpoint{0.984045in}{1.517610in}}{\pgfqpoint{0.976232in}{1.509796in}}%
\pgfpathcurveto{\pgfqpoint{0.968418in}{1.501983in}}{\pgfqpoint{0.964028in}{1.491384in}}{\pgfqpoint{0.964028in}{1.480334in}}%
\pgfpathcurveto{\pgfqpoint{0.964028in}{1.469284in}}{\pgfqpoint{0.968418in}{1.458685in}}{\pgfqpoint{0.976232in}{1.450871in}}%
\pgfpathcurveto{\pgfqpoint{0.984045in}{1.443057in}}{\pgfqpoint{0.994644in}{1.438667in}}{\pgfqpoint{1.005694in}{1.438667in}}%
\pgfpathclose%
\pgfusepath{stroke,fill}%
\end{pgfscope}%
\begin{pgfscope}%
\pgfpathrectangle{\pgfqpoint{0.375000in}{0.330000in}}{\pgfqpoint{2.325000in}{2.310000in}}%
\pgfusepath{clip}%
\pgfsetbuttcap%
\pgfsetroundjoin%
\definecolor{currentfill}{rgb}{0.000000,0.000000,0.000000}%
\pgfsetfillcolor{currentfill}%
\pgfsetlinewidth{1.003750pt}%
\definecolor{currentstroke}{rgb}{0.000000,0.000000,0.000000}%
\pgfsetstrokecolor{currentstroke}%
\pgfsetdash{}{0pt}%
\pgfpathmoveto{\pgfqpoint{1.005694in}{1.438667in}}%
\pgfpathcurveto{\pgfqpoint{1.016744in}{1.438667in}}{\pgfqpoint{1.027343in}{1.443057in}}{\pgfqpoint{1.035157in}{1.450871in}}%
\pgfpathcurveto{\pgfqpoint{1.042971in}{1.458685in}}{\pgfqpoint{1.047361in}{1.469284in}}{\pgfqpoint{1.047361in}{1.480334in}}%
\pgfpathcurveto{\pgfqpoint{1.047361in}{1.491384in}}{\pgfqpoint{1.042971in}{1.501983in}}{\pgfqpoint{1.035157in}{1.509796in}}%
\pgfpathcurveto{\pgfqpoint{1.027343in}{1.517610in}}{\pgfqpoint{1.016744in}{1.522000in}}{\pgfqpoint{1.005694in}{1.522000in}}%
\pgfpathcurveto{\pgfqpoint{0.994644in}{1.522000in}}{\pgfqpoint{0.984045in}{1.517610in}}{\pgfqpoint{0.976232in}{1.509796in}}%
\pgfpathcurveto{\pgfqpoint{0.968418in}{1.501983in}}{\pgfqpoint{0.964028in}{1.491384in}}{\pgfqpoint{0.964028in}{1.480334in}}%
\pgfpathcurveto{\pgfqpoint{0.964028in}{1.469284in}}{\pgfqpoint{0.968418in}{1.458685in}}{\pgfqpoint{0.976232in}{1.450871in}}%
\pgfpathcurveto{\pgfqpoint{0.984045in}{1.443057in}}{\pgfqpoint{0.994644in}{1.438667in}}{\pgfqpoint{1.005694in}{1.438667in}}%
\pgfpathclose%
\pgfusepath{stroke,fill}%
\end{pgfscope}%
\begin{pgfscope}%
\pgfpathrectangle{\pgfqpoint{0.375000in}{0.330000in}}{\pgfqpoint{2.325000in}{2.310000in}}%
\pgfusepath{clip}%
\pgfsetbuttcap%
\pgfsetroundjoin%
\definecolor{currentfill}{rgb}{0.000000,0.000000,0.000000}%
\pgfsetfillcolor{currentfill}%
\pgfsetlinewidth{1.003750pt}%
\definecolor{currentstroke}{rgb}{0.000000,0.000000,0.000000}%
\pgfsetstrokecolor{currentstroke}%
\pgfsetdash{}{0pt}%
\pgfpathmoveto{\pgfqpoint{1.005694in}{1.438667in}}%
\pgfpathcurveto{\pgfqpoint{1.016744in}{1.438667in}}{\pgfqpoint{1.027343in}{1.443057in}}{\pgfqpoint{1.035157in}{1.450871in}}%
\pgfpathcurveto{\pgfqpoint{1.042971in}{1.458685in}}{\pgfqpoint{1.047361in}{1.469284in}}{\pgfqpoint{1.047361in}{1.480334in}}%
\pgfpathcurveto{\pgfqpoint{1.047361in}{1.491384in}}{\pgfqpoint{1.042971in}{1.501983in}}{\pgfqpoint{1.035157in}{1.509796in}}%
\pgfpathcurveto{\pgfqpoint{1.027343in}{1.517610in}}{\pgfqpoint{1.016744in}{1.522000in}}{\pgfqpoint{1.005694in}{1.522000in}}%
\pgfpathcurveto{\pgfqpoint{0.994644in}{1.522000in}}{\pgfqpoint{0.984045in}{1.517610in}}{\pgfqpoint{0.976232in}{1.509796in}}%
\pgfpathcurveto{\pgfqpoint{0.968418in}{1.501983in}}{\pgfqpoint{0.964028in}{1.491384in}}{\pgfqpoint{0.964028in}{1.480334in}}%
\pgfpathcurveto{\pgfqpoint{0.964028in}{1.469284in}}{\pgfqpoint{0.968418in}{1.458685in}}{\pgfqpoint{0.976232in}{1.450871in}}%
\pgfpathcurveto{\pgfqpoint{0.984045in}{1.443057in}}{\pgfqpoint{0.994644in}{1.438667in}}{\pgfqpoint{1.005694in}{1.438667in}}%
\pgfpathclose%
\pgfusepath{stroke,fill}%
\end{pgfscope}%
\begin{pgfscope}%
\pgfpathrectangle{\pgfqpoint{0.375000in}{0.330000in}}{\pgfqpoint{2.325000in}{2.310000in}}%
\pgfusepath{clip}%
\pgfsetbuttcap%
\pgfsetroundjoin%
\definecolor{currentfill}{rgb}{0.000000,0.000000,0.000000}%
\pgfsetfillcolor{currentfill}%
\pgfsetlinewidth{1.003750pt}%
\definecolor{currentstroke}{rgb}{0.000000,0.000000,0.000000}%
\pgfsetstrokecolor{currentstroke}%
\pgfsetdash{}{0pt}%
\pgfpathmoveto{\pgfqpoint{1.005694in}{1.438667in}}%
\pgfpathcurveto{\pgfqpoint{1.016744in}{1.438667in}}{\pgfqpoint{1.027343in}{1.443057in}}{\pgfqpoint{1.035157in}{1.450871in}}%
\pgfpathcurveto{\pgfqpoint{1.042971in}{1.458685in}}{\pgfqpoint{1.047361in}{1.469284in}}{\pgfqpoint{1.047361in}{1.480334in}}%
\pgfpathcurveto{\pgfqpoint{1.047361in}{1.491384in}}{\pgfqpoint{1.042971in}{1.501983in}}{\pgfqpoint{1.035157in}{1.509796in}}%
\pgfpathcurveto{\pgfqpoint{1.027343in}{1.517610in}}{\pgfqpoint{1.016744in}{1.522000in}}{\pgfqpoint{1.005694in}{1.522000in}}%
\pgfpathcurveto{\pgfqpoint{0.994644in}{1.522000in}}{\pgfqpoint{0.984045in}{1.517610in}}{\pgfqpoint{0.976232in}{1.509796in}}%
\pgfpathcurveto{\pgfqpoint{0.968418in}{1.501983in}}{\pgfqpoint{0.964028in}{1.491384in}}{\pgfqpoint{0.964028in}{1.480334in}}%
\pgfpathcurveto{\pgfqpoint{0.964028in}{1.469284in}}{\pgfqpoint{0.968418in}{1.458685in}}{\pgfqpoint{0.976232in}{1.450871in}}%
\pgfpathcurveto{\pgfqpoint{0.984045in}{1.443057in}}{\pgfqpoint{0.994644in}{1.438667in}}{\pgfqpoint{1.005694in}{1.438667in}}%
\pgfpathclose%
\pgfusepath{stroke,fill}%
\end{pgfscope}%
\begin{pgfscope}%
\pgfpathrectangle{\pgfqpoint{0.375000in}{0.330000in}}{\pgfqpoint{2.325000in}{2.310000in}}%
\pgfusepath{clip}%
\pgfsetbuttcap%
\pgfsetroundjoin%
\definecolor{currentfill}{rgb}{0.000000,0.000000,0.000000}%
\pgfsetfillcolor{currentfill}%
\pgfsetlinewidth{1.003750pt}%
\definecolor{currentstroke}{rgb}{0.000000,0.000000,0.000000}%
\pgfsetstrokecolor{currentstroke}%
\pgfsetdash{}{0pt}%
\pgfpathmoveto{\pgfqpoint{1.005694in}{1.438667in}}%
\pgfpathcurveto{\pgfqpoint{1.016744in}{1.438667in}}{\pgfqpoint{1.027343in}{1.443057in}}{\pgfqpoint{1.035157in}{1.450871in}}%
\pgfpathcurveto{\pgfqpoint{1.042971in}{1.458685in}}{\pgfqpoint{1.047361in}{1.469284in}}{\pgfqpoint{1.047361in}{1.480334in}}%
\pgfpathcurveto{\pgfqpoint{1.047361in}{1.491384in}}{\pgfqpoint{1.042971in}{1.501983in}}{\pgfqpoint{1.035157in}{1.509796in}}%
\pgfpathcurveto{\pgfqpoint{1.027343in}{1.517610in}}{\pgfqpoint{1.016744in}{1.522000in}}{\pgfqpoint{1.005694in}{1.522000in}}%
\pgfpathcurveto{\pgfqpoint{0.994644in}{1.522000in}}{\pgfqpoint{0.984045in}{1.517610in}}{\pgfqpoint{0.976232in}{1.509796in}}%
\pgfpathcurveto{\pgfqpoint{0.968418in}{1.501983in}}{\pgfqpoint{0.964028in}{1.491384in}}{\pgfqpoint{0.964028in}{1.480334in}}%
\pgfpathcurveto{\pgfqpoint{0.964028in}{1.469284in}}{\pgfqpoint{0.968418in}{1.458685in}}{\pgfqpoint{0.976232in}{1.450871in}}%
\pgfpathcurveto{\pgfqpoint{0.984045in}{1.443057in}}{\pgfqpoint{0.994644in}{1.438667in}}{\pgfqpoint{1.005694in}{1.438667in}}%
\pgfpathclose%
\pgfusepath{stroke,fill}%
\end{pgfscope}%
\begin{pgfscope}%
\pgfpathrectangle{\pgfqpoint{0.375000in}{0.330000in}}{\pgfqpoint{2.325000in}{2.310000in}}%
\pgfusepath{clip}%
\pgfsetbuttcap%
\pgfsetroundjoin%
\definecolor{currentfill}{rgb}{0.000000,0.000000,0.000000}%
\pgfsetfillcolor{currentfill}%
\pgfsetlinewidth{1.003750pt}%
\definecolor{currentstroke}{rgb}{0.000000,0.000000,0.000000}%
\pgfsetstrokecolor{currentstroke}%
\pgfsetdash{}{0pt}%
\pgfpathmoveto{\pgfqpoint{1.005694in}{1.438667in}}%
\pgfpathcurveto{\pgfqpoint{1.016744in}{1.438667in}}{\pgfqpoint{1.027343in}{1.443057in}}{\pgfqpoint{1.035157in}{1.450871in}}%
\pgfpathcurveto{\pgfqpoint{1.042971in}{1.458685in}}{\pgfqpoint{1.047361in}{1.469284in}}{\pgfqpoint{1.047361in}{1.480334in}}%
\pgfpathcurveto{\pgfqpoint{1.047361in}{1.491384in}}{\pgfqpoint{1.042971in}{1.501983in}}{\pgfqpoint{1.035157in}{1.509796in}}%
\pgfpathcurveto{\pgfqpoint{1.027343in}{1.517610in}}{\pgfqpoint{1.016744in}{1.522000in}}{\pgfqpoint{1.005694in}{1.522000in}}%
\pgfpathcurveto{\pgfqpoint{0.994644in}{1.522000in}}{\pgfqpoint{0.984045in}{1.517610in}}{\pgfqpoint{0.976232in}{1.509796in}}%
\pgfpathcurveto{\pgfqpoint{0.968418in}{1.501983in}}{\pgfqpoint{0.964028in}{1.491384in}}{\pgfqpoint{0.964028in}{1.480334in}}%
\pgfpathcurveto{\pgfqpoint{0.964028in}{1.469284in}}{\pgfqpoint{0.968418in}{1.458685in}}{\pgfqpoint{0.976232in}{1.450871in}}%
\pgfpathcurveto{\pgfqpoint{0.984045in}{1.443057in}}{\pgfqpoint{0.994644in}{1.438667in}}{\pgfqpoint{1.005694in}{1.438667in}}%
\pgfpathclose%
\pgfusepath{stroke,fill}%
\end{pgfscope}%
\begin{pgfscope}%
\pgfpathrectangle{\pgfqpoint{0.375000in}{0.330000in}}{\pgfqpoint{2.325000in}{2.310000in}}%
\pgfusepath{clip}%
\pgfsetbuttcap%
\pgfsetroundjoin%
\definecolor{currentfill}{rgb}{0.000000,0.000000,0.000000}%
\pgfsetfillcolor{currentfill}%
\pgfsetlinewidth{1.003750pt}%
\definecolor{currentstroke}{rgb}{0.000000,0.000000,0.000000}%
\pgfsetstrokecolor{currentstroke}%
\pgfsetdash{}{0pt}%
\pgfpathmoveto{\pgfqpoint{1.005694in}{1.438667in}}%
\pgfpathcurveto{\pgfqpoint{1.016744in}{1.438667in}}{\pgfqpoint{1.027343in}{1.443057in}}{\pgfqpoint{1.035157in}{1.450871in}}%
\pgfpathcurveto{\pgfqpoint{1.042971in}{1.458685in}}{\pgfqpoint{1.047361in}{1.469284in}}{\pgfqpoint{1.047361in}{1.480334in}}%
\pgfpathcurveto{\pgfqpoint{1.047361in}{1.491384in}}{\pgfqpoint{1.042971in}{1.501983in}}{\pgfqpoint{1.035157in}{1.509796in}}%
\pgfpathcurveto{\pgfqpoint{1.027343in}{1.517610in}}{\pgfqpoint{1.016744in}{1.522000in}}{\pgfqpoint{1.005694in}{1.522000in}}%
\pgfpathcurveto{\pgfqpoint{0.994644in}{1.522000in}}{\pgfqpoint{0.984045in}{1.517610in}}{\pgfqpoint{0.976232in}{1.509796in}}%
\pgfpathcurveto{\pgfqpoint{0.968418in}{1.501983in}}{\pgfqpoint{0.964028in}{1.491384in}}{\pgfqpoint{0.964028in}{1.480334in}}%
\pgfpathcurveto{\pgfqpoint{0.964028in}{1.469284in}}{\pgfqpoint{0.968418in}{1.458685in}}{\pgfqpoint{0.976232in}{1.450871in}}%
\pgfpathcurveto{\pgfqpoint{0.984045in}{1.443057in}}{\pgfqpoint{0.994644in}{1.438667in}}{\pgfqpoint{1.005694in}{1.438667in}}%
\pgfpathclose%
\pgfusepath{stroke,fill}%
\end{pgfscope}%
\begin{pgfscope}%
\pgfpathrectangle{\pgfqpoint{0.375000in}{0.330000in}}{\pgfqpoint{2.325000in}{2.310000in}}%
\pgfusepath{clip}%
\pgfsetbuttcap%
\pgfsetroundjoin%
\definecolor{currentfill}{rgb}{0.000000,0.000000,0.000000}%
\pgfsetfillcolor{currentfill}%
\pgfsetlinewidth{1.003750pt}%
\definecolor{currentstroke}{rgb}{0.000000,0.000000,0.000000}%
\pgfsetstrokecolor{currentstroke}%
\pgfsetdash{}{0pt}%
\pgfpathmoveto{\pgfqpoint{1.005694in}{1.438667in}}%
\pgfpathcurveto{\pgfqpoint{1.016744in}{1.438667in}}{\pgfqpoint{1.027343in}{1.443057in}}{\pgfqpoint{1.035157in}{1.450871in}}%
\pgfpathcurveto{\pgfqpoint{1.042971in}{1.458685in}}{\pgfqpoint{1.047361in}{1.469284in}}{\pgfqpoint{1.047361in}{1.480334in}}%
\pgfpathcurveto{\pgfqpoint{1.047361in}{1.491384in}}{\pgfqpoint{1.042971in}{1.501983in}}{\pgfqpoint{1.035157in}{1.509796in}}%
\pgfpathcurveto{\pgfqpoint{1.027343in}{1.517610in}}{\pgfqpoint{1.016744in}{1.522000in}}{\pgfqpoint{1.005694in}{1.522000in}}%
\pgfpathcurveto{\pgfqpoint{0.994644in}{1.522000in}}{\pgfqpoint{0.984045in}{1.517610in}}{\pgfqpoint{0.976232in}{1.509796in}}%
\pgfpathcurveto{\pgfqpoint{0.968418in}{1.501983in}}{\pgfqpoint{0.964028in}{1.491384in}}{\pgfqpoint{0.964028in}{1.480334in}}%
\pgfpathcurveto{\pgfqpoint{0.964028in}{1.469284in}}{\pgfqpoint{0.968418in}{1.458685in}}{\pgfqpoint{0.976232in}{1.450871in}}%
\pgfpathcurveto{\pgfqpoint{0.984045in}{1.443057in}}{\pgfqpoint{0.994644in}{1.438667in}}{\pgfqpoint{1.005694in}{1.438667in}}%
\pgfpathclose%
\pgfusepath{stroke,fill}%
\end{pgfscope}%
\begin{pgfscope}%
\pgfpathrectangle{\pgfqpoint{0.375000in}{0.330000in}}{\pgfqpoint{2.325000in}{2.310000in}}%
\pgfusepath{clip}%
\pgfsetbuttcap%
\pgfsetroundjoin%
\definecolor{currentfill}{rgb}{0.000000,0.000000,0.000000}%
\pgfsetfillcolor{currentfill}%
\pgfsetlinewidth{1.003750pt}%
\definecolor{currentstroke}{rgb}{0.000000,0.000000,0.000000}%
\pgfsetstrokecolor{currentstroke}%
\pgfsetdash{}{0pt}%
\pgfpathmoveto{\pgfqpoint{1.005694in}{1.438667in}}%
\pgfpathcurveto{\pgfqpoint{1.016744in}{1.438667in}}{\pgfqpoint{1.027343in}{1.443057in}}{\pgfqpoint{1.035157in}{1.450871in}}%
\pgfpathcurveto{\pgfqpoint{1.042971in}{1.458685in}}{\pgfqpoint{1.047361in}{1.469284in}}{\pgfqpoint{1.047361in}{1.480334in}}%
\pgfpathcurveto{\pgfqpoint{1.047361in}{1.491384in}}{\pgfqpoint{1.042971in}{1.501983in}}{\pgfqpoint{1.035157in}{1.509796in}}%
\pgfpathcurveto{\pgfqpoint{1.027343in}{1.517610in}}{\pgfqpoint{1.016744in}{1.522000in}}{\pgfqpoint{1.005694in}{1.522000in}}%
\pgfpathcurveto{\pgfqpoint{0.994644in}{1.522000in}}{\pgfqpoint{0.984045in}{1.517610in}}{\pgfqpoint{0.976232in}{1.509796in}}%
\pgfpathcurveto{\pgfqpoint{0.968418in}{1.501983in}}{\pgfqpoint{0.964028in}{1.491384in}}{\pgfqpoint{0.964028in}{1.480334in}}%
\pgfpathcurveto{\pgfqpoint{0.964028in}{1.469284in}}{\pgfqpoint{0.968418in}{1.458685in}}{\pgfqpoint{0.976232in}{1.450871in}}%
\pgfpathcurveto{\pgfqpoint{0.984045in}{1.443057in}}{\pgfqpoint{0.994644in}{1.438667in}}{\pgfqpoint{1.005694in}{1.438667in}}%
\pgfpathclose%
\pgfusepath{stroke,fill}%
\end{pgfscope}%
\begin{pgfscope}%
\pgfpathrectangle{\pgfqpoint{0.375000in}{0.330000in}}{\pgfqpoint{2.325000in}{2.310000in}}%
\pgfusepath{clip}%
\pgfsetbuttcap%
\pgfsetroundjoin%
\definecolor{currentfill}{rgb}{0.000000,0.000000,0.000000}%
\pgfsetfillcolor{currentfill}%
\pgfsetlinewidth{1.003750pt}%
\definecolor{currentstroke}{rgb}{0.000000,0.000000,0.000000}%
\pgfsetstrokecolor{currentstroke}%
\pgfsetdash{}{0pt}%
\pgfpathmoveto{\pgfqpoint{1.005694in}{1.438667in}}%
\pgfpathcurveto{\pgfqpoint{1.016744in}{1.438667in}}{\pgfqpoint{1.027343in}{1.443057in}}{\pgfqpoint{1.035157in}{1.450871in}}%
\pgfpathcurveto{\pgfqpoint{1.042971in}{1.458685in}}{\pgfqpoint{1.047361in}{1.469284in}}{\pgfqpoint{1.047361in}{1.480334in}}%
\pgfpathcurveto{\pgfqpoint{1.047361in}{1.491384in}}{\pgfqpoint{1.042971in}{1.501983in}}{\pgfqpoint{1.035157in}{1.509796in}}%
\pgfpathcurveto{\pgfqpoint{1.027343in}{1.517610in}}{\pgfqpoint{1.016744in}{1.522000in}}{\pgfqpoint{1.005694in}{1.522000in}}%
\pgfpathcurveto{\pgfqpoint{0.994644in}{1.522000in}}{\pgfqpoint{0.984045in}{1.517610in}}{\pgfqpoint{0.976232in}{1.509796in}}%
\pgfpathcurveto{\pgfqpoint{0.968418in}{1.501983in}}{\pgfqpoint{0.964028in}{1.491384in}}{\pgfqpoint{0.964028in}{1.480334in}}%
\pgfpathcurveto{\pgfqpoint{0.964028in}{1.469284in}}{\pgfqpoint{0.968418in}{1.458685in}}{\pgfqpoint{0.976232in}{1.450871in}}%
\pgfpathcurveto{\pgfqpoint{0.984045in}{1.443057in}}{\pgfqpoint{0.994644in}{1.438667in}}{\pgfqpoint{1.005694in}{1.438667in}}%
\pgfpathclose%
\pgfusepath{stroke,fill}%
\end{pgfscope}%
\begin{pgfscope}%
\pgfpathrectangle{\pgfqpoint{0.375000in}{0.330000in}}{\pgfqpoint{2.325000in}{2.310000in}}%
\pgfusepath{clip}%
\pgfsetbuttcap%
\pgfsetroundjoin%
\definecolor{currentfill}{rgb}{0.000000,0.000000,0.000000}%
\pgfsetfillcolor{currentfill}%
\pgfsetlinewidth{1.003750pt}%
\definecolor{currentstroke}{rgb}{0.000000,0.000000,0.000000}%
\pgfsetstrokecolor{currentstroke}%
\pgfsetdash{}{0pt}%
\pgfpathmoveto{\pgfqpoint{1.005694in}{1.438667in}}%
\pgfpathcurveto{\pgfqpoint{1.016744in}{1.438667in}}{\pgfqpoint{1.027343in}{1.443057in}}{\pgfqpoint{1.035157in}{1.450871in}}%
\pgfpathcurveto{\pgfqpoint{1.042971in}{1.458685in}}{\pgfqpoint{1.047361in}{1.469284in}}{\pgfqpoint{1.047361in}{1.480334in}}%
\pgfpathcurveto{\pgfqpoint{1.047361in}{1.491384in}}{\pgfqpoint{1.042971in}{1.501983in}}{\pgfqpoint{1.035157in}{1.509796in}}%
\pgfpathcurveto{\pgfqpoint{1.027343in}{1.517610in}}{\pgfqpoint{1.016744in}{1.522000in}}{\pgfqpoint{1.005694in}{1.522000in}}%
\pgfpathcurveto{\pgfqpoint{0.994644in}{1.522000in}}{\pgfqpoint{0.984045in}{1.517610in}}{\pgfqpoint{0.976232in}{1.509796in}}%
\pgfpathcurveto{\pgfqpoint{0.968418in}{1.501983in}}{\pgfqpoint{0.964028in}{1.491384in}}{\pgfqpoint{0.964028in}{1.480334in}}%
\pgfpathcurveto{\pgfqpoint{0.964028in}{1.469284in}}{\pgfqpoint{0.968418in}{1.458685in}}{\pgfqpoint{0.976232in}{1.450871in}}%
\pgfpathcurveto{\pgfqpoint{0.984045in}{1.443057in}}{\pgfqpoint{0.994644in}{1.438667in}}{\pgfqpoint{1.005694in}{1.438667in}}%
\pgfpathclose%
\pgfusepath{stroke,fill}%
\end{pgfscope}%
\begin{pgfscope}%
\pgfpathrectangle{\pgfqpoint{0.375000in}{0.330000in}}{\pgfqpoint{2.325000in}{2.310000in}}%
\pgfusepath{clip}%
\pgfsetbuttcap%
\pgfsetroundjoin%
\definecolor{currentfill}{rgb}{0.000000,0.000000,0.000000}%
\pgfsetfillcolor{currentfill}%
\pgfsetlinewidth{1.003750pt}%
\definecolor{currentstroke}{rgb}{0.000000,0.000000,0.000000}%
\pgfsetstrokecolor{currentstroke}%
\pgfsetdash{}{0pt}%
\pgfpathmoveto{\pgfqpoint{1.005694in}{1.438667in}}%
\pgfpathcurveto{\pgfqpoint{1.016744in}{1.438667in}}{\pgfqpoint{1.027343in}{1.443057in}}{\pgfqpoint{1.035157in}{1.450871in}}%
\pgfpathcurveto{\pgfqpoint{1.042971in}{1.458685in}}{\pgfqpoint{1.047361in}{1.469284in}}{\pgfqpoint{1.047361in}{1.480334in}}%
\pgfpathcurveto{\pgfqpoint{1.047361in}{1.491384in}}{\pgfqpoint{1.042971in}{1.501983in}}{\pgfqpoint{1.035157in}{1.509796in}}%
\pgfpathcurveto{\pgfqpoint{1.027343in}{1.517610in}}{\pgfqpoint{1.016744in}{1.522000in}}{\pgfqpoint{1.005694in}{1.522000in}}%
\pgfpathcurveto{\pgfqpoint{0.994644in}{1.522000in}}{\pgfqpoint{0.984045in}{1.517610in}}{\pgfqpoint{0.976232in}{1.509796in}}%
\pgfpathcurveto{\pgfqpoint{0.968418in}{1.501983in}}{\pgfqpoint{0.964028in}{1.491384in}}{\pgfqpoint{0.964028in}{1.480334in}}%
\pgfpathcurveto{\pgfqpoint{0.964028in}{1.469284in}}{\pgfqpoint{0.968418in}{1.458685in}}{\pgfqpoint{0.976232in}{1.450871in}}%
\pgfpathcurveto{\pgfqpoint{0.984045in}{1.443057in}}{\pgfqpoint{0.994644in}{1.438667in}}{\pgfqpoint{1.005694in}{1.438667in}}%
\pgfpathclose%
\pgfusepath{stroke,fill}%
\end{pgfscope}%
\begin{pgfscope}%
\pgfpathrectangle{\pgfqpoint{0.375000in}{0.330000in}}{\pgfqpoint{2.325000in}{2.310000in}}%
\pgfusepath{clip}%
\pgfsetbuttcap%
\pgfsetroundjoin%
\definecolor{currentfill}{rgb}{0.000000,0.000000,0.000000}%
\pgfsetfillcolor{currentfill}%
\pgfsetlinewidth{1.003750pt}%
\definecolor{currentstroke}{rgb}{0.000000,0.000000,0.000000}%
\pgfsetstrokecolor{currentstroke}%
\pgfsetdash{}{0pt}%
\pgfpathmoveto{\pgfqpoint{1.005694in}{1.438667in}}%
\pgfpathcurveto{\pgfqpoint{1.016744in}{1.438667in}}{\pgfqpoint{1.027343in}{1.443057in}}{\pgfqpoint{1.035157in}{1.450871in}}%
\pgfpathcurveto{\pgfqpoint{1.042971in}{1.458685in}}{\pgfqpoint{1.047361in}{1.469284in}}{\pgfqpoint{1.047361in}{1.480334in}}%
\pgfpathcurveto{\pgfqpoint{1.047361in}{1.491384in}}{\pgfqpoint{1.042971in}{1.501983in}}{\pgfqpoint{1.035157in}{1.509796in}}%
\pgfpathcurveto{\pgfqpoint{1.027343in}{1.517610in}}{\pgfqpoint{1.016744in}{1.522000in}}{\pgfqpoint{1.005694in}{1.522000in}}%
\pgfpathcurveto{\pgfqpoint{0.994644in}{1.522000in}}{\pgfqpoint{0.984045in}{1.517610in}}{\pgfqpoint{0.976232in}{1.509796in}}%
\pgfpathcurveto{\pgfqpoint{0.968418in}{1.501983in}}{\pgfqpoint{0.964028in}{1.491384in}}{\pgfqpoint{0.964028in}{1.480334in}}%
\pgfpathcurveto{\pgfqpoint{0.964028in}{1.469284in}}{\pgfqpoint{0.968418in}{1.458685in}}{\pgfqpoint{0.976232in}{1.450871in}}%
\pgfpathcurveto{\pgfqpoint{0.984045in}{1.443057in}}{\pgfqpoint{0.994644in}{1.438667in}}{\pgfqpoint{1.005694in}{1.438667in}}%
\pgfpathclose%
\pgfusepath{stroke,fill}%
\end{pgfscope}%
\begin{pgfscope}%
\pgfpathrectangle{\pgfqpoint{0.375000in}{0.330000in}}{\pgfqpoint{2.325000in}{2.310000in}}%
\pgfusepath{clip}%
\pgfsetbuttcap%
\pgfsetroundjoin%
\definecolor{currentfill}{rgb}{0.000000,0.000000,0.000000}%
\pgfsetfillcolor{currentfill}%
\pgfsetlinewidth{1.003750pt}%
\definecolor{currentstroke}{rgb}{0.000000,0.000000,0.000000}%
\pgfsetstrokecolor{currentstroke}%
\pgfsetdash{}{0pt}%
\pgfpathmoveto{\pgfqpoint{1.005694in}{1.438667in}}%
\pgfpathcurveto{\pgfqpoint{1.016744in}{1.438667in}}{\pgfqpoint{1.027343in}{1.443057in}}{\pgfqpoint{1.035157in}{1.450871in}}%
\pgfpathcurveto{\pgfqpoint{1.042971in}{1.458685in}}{\pgfqpoint{1.047361in}{1.469284in}}{\pgfqpoint{1.047361in}{1.480334in}}%
\pgfpathcurveto{\pgfqpoint{1.047361in}{1.491384in}}{\pgfqpoint{1.042971in}{1.501983in}}{\pgfqpoint{1.035157in}{1.509796in}}%
\pgfpathcurveto{\pgfqpoint{1.027343in}{1.517610in}}{\pgfqpoint{1.016744in}{1.522000in}}{\pgfqpoint{1.005694in}{1.522000in}}%
\pgfpathcurveto{\pgfqpoint{0.994644in}{1.522000in}}{\pgfqpoint{0.984045in}{1.517610in}}{\pgfqpoint{0.976232in}{1.509796in}}%
\pgfpathcurveto{\pgfqpoint{0.968418in}{1.501983in}}{\pgfqpoint{0.964028in}{1.491384in}}{\pgfqpoint{0.964028in}{1.480334in}}%
\pgfpathcurveto{\pgfqpoint{0.964028in}{1.469284in}}{\pgfqpoint{0.968418in}{1.458685in}}{\pgfqpoint{0.976232in}{1.450871in}}%
\pgfpathcurveto{\pgfqpoint{0.984045in}{1.443057in}}{\pgfqpoint{0.994644in}{1.438667in}}{\pgfqpoint{1.005694in}{1.438667in}}%
\pgfpathclose%
\pgfusepath{stroke,fill}%
\end{pgfscope}%
\begin{pgfscope}%
\pgfpathrectangle{\pgfqpoint{0.375000in}{0.330000in}}{\pgfqpoint{2.325000in}{2.310000in}}%
\pgfusepath{clip}%
\pgfsetbuttcap%
\pgfsetroundjoin%
\definecolor{currentfill}{rgb}{0.000000,0.000000,0.000000}%
\pgfsetfillcolor{currentfill}%
\pgfsetlinewidth{1.003750pt}%
\definecolor{currentstroke}{rgb}{0.000000,0.000000,0.000000}%
\pgfsetstrokecolor{currentstroke}%
\pgfsetdash{}{0pt}%
\pgfpathmoveto{\pgfqpoint{1.005694in}{1.438667in}}%
\pgfpathcurveto{\pgfqpoint{1.016744in}{1.438667in}}{\pgfqpoint{1.027343in}{1.443057in}}{\pgfqpoint{1.035157in}{1.450871in}}%
\pgfpathcurveto{\pgfqpoint{1.042971in}{1.458685in}}{\pgfqpoint{1.047361in}{1.469284in}}{\pgfqpoint{1.047361in}{1.480334in}}%
\pgfpathcurveto{\pgfqpoint{1.047361in}{1.491384in}}{\pgfqpoint{1.042971in}{1.501983in}}{\pgfqpoint{1.035157in}{1.509796in}}%
\pgfpathcurveto{\pgfqpoint{1.027343in}{1.517610in}}{\pgfqpoint{1.016744in}{1.522000in}}{\pgfqpoint{1.005694in}{1.522000in}}%
\pgfpathcurveto{\pgfqpoint{0.994644in}{1.522000in}}{\pgfqpoint{0.984045in}{1.517610in}}{\pgfqpoint{0.976232in}{1.509796in}}%
\pgfpathcurveto{\pgfqpoint{0.968418in}{1.501983in}}{\pgfqpoint{0.964028in}{1.491384in}}{\pgfqpoint{0.964028in}{1.480334in}}%
\pgfpathcurveto{\pgfqpoint{0.964028in}{1.469284in}}{\pgfqpoint{0.968418in}{1.458685in}}{\pgfqpoint{0.976232in}{1.450871in}}%
\pgfpathcurveto{\pgfqpoint{0.984045in}{1.443057in}}{\pgfqpoint{0.994644in}{1.438667in}}{\pgfqpoint{1.005694in}{1.438667in}}%
\pgfpathclose%
\pgfusepath{stroke,fill}%
\end{pgfscope}%
\begin{pgfscope}%
\pgfpathrectangle{\pgfqpoint{0.375000in}{0.330000in}}{\pgfqpoint{2.325000in}{2.310000in}}%
\pgfusepath{clip}%
\pgfsetbuttcap%
\pgfsetroundjoin%
\definecolor{currentfill}{rgb}{0.000000,0.000000,0.000000}%
\pgfsetfillcolor{currentfill}%
\pgfsetlinewidth{1.003750pt}%
\definecolor{currentstroke}{rgb}{0.000000,0.000000,0.000000}%
\pgfsetstrokecolor{currentstroke}%
\pgfsetdash{}{0pt}%
\pgfpathmoveto{\pgfqpoint{1.005694in}{1.438667in}}%
\pgfpathcurveto{\pgfqpoint{1.016744in}{1.438667in}}{\pgfqpoint{1.027343in}{1.443057in}}{\pgfqpoint{1.035157in}{1.450871in}}%
\pgfpathcurveto{\pgfqpoint{1.042971in}{1.458685in}}{\pgfqpoint{1.047361in}{1.469284in}}{\pgfqpoint{1.047361in}{1.480334in}}%
\pgfpathcurveto{\pgfqpoint{1.047361in}{1.491384in}}{\pgfqpoint{1.042971in}{1.501983in}}{\pgfqpoint{1.035157in}{1.509796in}}%
\pgfpathcurveto{\pgfqpoint{1.027343in}{1.517610in}}{\pgfqpoint{1.016744in}{1.522000in}}{\pgfqpoint{1.005694in}{1.522000in}}%
\pgfpathcurveto{\pgfqpoint{0.994644in}{1.522000in}}{\pgfqpoint{0.984045in}{1.517610in}}{\pgfqpoint{0.976232in}{1.509796in}}%
\pgfpathcurveto{\pgfqpoint{0.968418in}{1.501983in}}{\pgfqpoint{0.964028in}{1.491384in}}{\pgfqpoint{0.964028in}{1.480334in}}%
\pgfpathcurveto{\pgfqpoint{0.964028in}{1.469284in}}{\pgfqpoint{0.968418in}{1.458685in}}{\pgfqpoint{0.976232in}{1.450871in}}%
\pgfpathcurveto{\pgfqpoint{0.984045in}{1.443057in}}{\pgfqpoint{0.994644in}{1.438667in}}{\pgfqpoint{1.005694in}{1.438667in}}%
\pgfpathclose%
\pgfusepath{stroke,fill}%
\end{pgfscope}%
\begin{pgfscope}%
\pgfpathrectangle{\pgfqpoint{0.375000in}{0.330000in}}{\pgfqpoint{2.325000in}{2.310000in}}%
\pgfusepath{clip}%
\pgfsetbuttcap%
\pgfsetroundjoin%
\definecolor{currentfill}{rgb}{0.000000,0.000000,0.000000}%
\pgfsetfillcolor{currentfill}%
\pgfsetlinewidth{1.003750pt}%
\definecolor{currentstroke}{rgb}{0.000000,0.000000,0.000000}%
\pgfsetstrokecolor{currentstroke}%
\pgfsetdash{}{0pt}%
\pgfpathmoveto{\pgfqpoint{1.005694in}{1.438667in}}%
\pgfpathcurveto{\pgfqpoint{1.016744in}{1.438667in}}{\pgfqpoint{1.027343in}{1.443057in}}{\pgfqpoint{1.035157in}{1.450871in}}%
\pgfpathcurveto{\pgfqpoint{1.042971in}{1.458685in}}{\pgfqpoint{1.047361in}{1.469284in}}{\pgfqpoint{1.047361in}{1.480334in}}%
\pgfpathcurveto{\pgfqpoint{1.047361in}{1.491384in}}{\pgfqpoint{1.042971in}{1.501983in}}{\pgfqpoint{1.035157in}{1.509796in}}%
\pgfpathcurveto{\pgfqpoint{1.027343in}{1.517610in}}{\pgfqpoint{1.016744in}{1.522000in}}{\pgfqpoint{1.005694in}{1.522000in}}%
\pgfpathcurveto{\pgfqpoint{0.994644in}{1.522000in}}{\pgfqpoint{0.984045in}{1.517610in}}{\pgfqpoint{0.976232in}{1.509796in}}%
\pgfpathcurveto{\pgfqpoint{0.968418in}{1.501983in}}{\pgfqpoint{0.964028in}{1.491384in}}{\pgfqpoint{0.964028in}{1.480334in}}%
\pgfpathcurveto{\pgfqpoint{0.964028in}{1.469284in}}{\pgfqpoint{0.968418in}{1.458685in}}{\pgfqpoint{0.976232in}{1.450871in}}%
\pgfpathcurveto{\pgfqpoint{0.984045in}{1.443057in}}{\pgfqpoint{0.994644in}{1.438667in}}{\pgfqpoint{1.005694in}{1.438667in}}%
\pgfpathclose%
\pgfusepath{stroke,fill}%
\end{pgfscope}%
\begin{pgfscope}%
\pgfpathrectangle{\pgfqpoint{0.375000in}{0.330000in}}{\pgfqpoint{2.325000in}{2.310000in}}%
\pgfusepath{clip}%
\pgfsetbuttcap%
\pgfsetroundjoin%
\definecolor{currentfill}{rgb}{0.000000,0.000000,0.000000}%
\pgfsetfillcolor{currentfill}%
\pgfsetlinewidth{1.003750pt}%
\definecolor{currentstroke}{rgb}{0.000000,0.000000,0.000000}%
\pgfsetstrokecolor{currentstroke}%
\pgfsetdash{}{0pt}%
\pgfpathmoveto{\pgfqpoint{1.005694in}{1.438667in}}%
\pgfpathcurveto{\pgfqpoint{1.016744in}{1.438667in}}{\pgfqpoint{1.027343in}{1.443057in}}{\pgfqpoint{1.035157in}{1.450871in}}%
\pgfpathcurveto{\pgfqpoint{1.042971in}{1.458685in}}{\pgfqpoint{1.047361in}{1.469284in}}{\pgfqpoint{1.047361in}{1.480334in}}%
\pgfpathcurveto{\pgfqpoint{1.047361in}{1.491384in}}{\pgfqpoint{1.042971in}{1.501983in}}{\pgfqpoint{1.035157in}{1.509796in}}%
\pgfpathcurveto{\pgfqpoint{1.027343in}{1.517610in}}{\pgfqpoint{1.016744in}{1.522000in}}{\pgfqpoint{1.005694in}{1.522000in}}%
\pgfpathcurveto{\pgfqpoint{0.994644in}{1.522000in}}{\pgfqpoint{0.984045in}{1.517610in}}{\pgfqpoint{0.976232in}{1.509796in}}%
\pgfpathcurveto{\pgfqpoint{0.968418in}{1.501983in}}{\pgfqpoint{0.964028in}{1.491384in}}{\pgfqpoint{0.964028in}{1.480334in}}%
\pgfpathcurveto{\pgfqpoint{0.964028in}{1.469284in}}{\pgfqpoint{0.968418in}{1.458685in}}{\pgfqpoint{0.976232in}{1.450871in}}%
\pgfpathcurveto{\pgfqpoint{0.984045in}{1.443057in}}{\pgfqpoint{0.994644in}{1.438667in}}{\pgfqpoint{1.005694in}{1.438667in}}%
\pgfpathclose%
\pgfusepath{stroke,fill}%
\end{pgfscope}%
\begin{pgfscope}%
\pgfpathrectangle{\pgfqpoint{0.375000in}{0.330000in}}{\pgfqpoint{2.325000in}{2.310000in}}%
\pgfusepath{clip}%
\pgfsetbuttcap%
\pgfsetroundjoin%
\definecolor{currentfill}{rgb}{0.000000,0.000000,0.000000}%
\pgfsetfillcolor{currentfill}%
\pgfsetlinewidth{1.003750pt}%
\definecolor{currentstroke}{rgb}{0.000000,0.000000,0.000000}%
\pgfsetstrokecolor{currentstroke}%
\pgfsetdash{}{0pt}%
\pgfpathmoveto{\pgfqpoint{1.005694in}{1.438667in}}%
\pgfpathcurveto{\pgfqpoint{1.016744in}{1.438667in}}{\pgfqpoint{1.027343in}{1.443057in}}{\pgfqpoint{1.035157in}{1.450871in}}%
\pgfpathcurveto{\pgfqpoint{1.042971in}{1.458685in}}{\pgfqpoint{1.047361in}{1.469284in}}{\pgfqpoint{1.047361in}{1.480334in}}%
\pgfpathcurveto{\pgfqpoint{1.047361in}{1.491384in}}{\pgfqpoint{1.042971in}{1.501983in}}{\pgfqpoint{1.035157in}{1.509796in}}%
\pgfpathcurveto{\pgfqpoint{1.027343in}{1.517610in}}{\pgfqpoint{1.016744in}{1.522000in}}{\pgfqpoint{1.005694in}{1.522000in}}%
\pgfpathcurveto{\pgfqpoint{0.994644in}{1.522000in}}{\pgfqpoint{0.984045in}{1.517610in}}{\pgfqpoint{0.976232in}{1.509796in}}%
\pgfpathcurveto{\pgfqpoint{0.968418in}{1.501983in}}{\pgfqpoint{0.964028in}{1.491384in}}{\pgfqpoint{0.964028in}{1.480334in}}%
\pgfpathcurveto{\pgfqpoint{0.964028in}{1.469284in}}{\pgfqpoint{0.968418in}{1.458685in}}{\pgfqpoint{0.976232in}{1.450871in}}%
\pgfpathcurveto{\pgfqpoint{0.984045in}{1.443057in}}{\pgfqpoint{0.994644in}{1.438667in}}{\pgfqpoint{1.005694in}{1.438667in}}%
\pgfpathclose%
\pgfusepath{stroke,fill}%
\end{pgfscope}%
\begin{pgfscope}%
\pgfpathrectangle{\pgfqpoint{0.375000in}{0.330000in}}{\pgfqpoint{2.325000in}{2.310000in}}%
\pgfusepath{clip}%
\pgfsetbuttcap%
\pgfsetroundjoin%
\definecolor{currentfill}{rgb}{0.000000,0.000000,0.000000}%
\pgfsetfillcolor{currentfill}%
\pgfsetlinewidth{1.003750pt}%
\definecolor{currentstroke}{rgb}{0.000000,0.000000,0.000000}%
\pgfsetstrokecolor{currentstroke}%
\pgfsetdash{}{0pt}%
\pgfpathmoveto{\pgfqpoint{1.005694in}{1.438667in}}%
\pgfpathcurveto{\pgfqpoint{1.016744in}{1.438667in}}{\pgfqpoint{1.027343in}{1.443057in}}{\pgfqpoint{1.035157in}{1.450871in}}%
\pgfpathcurveto{\pgfqpoint{1.042971in}{1.458685in}}{\pgfqpoint{1.047361in}{1.469284in}}{\pgfqpoint{1.047361in}{1.480334in}}%
\pgfpathcurveto{\pgfqpoint{1.047361in}{1.491384in}}{\pgfqpoint{1.042971in}{1.501983in}}{\pgfqpoint{1.035157in}{1.509796in}}%
\pgfpathcurveto{\pgfqpoint{1.027343in}{1.517610in}}{\pgfqpoint{1.016744in}{1.522000in}}{\pgfqpoint{1.005694in}{1.522000in}}%
\pgfpathcurveto{\pgfqpoint{0.994644in}{1.522000in}}{\pgfqpoint{0.984045in}{1.517610in}}{\pgfqpoint{0.976232in}{1.509796in}}%
\pgfpathcurveto{\pgfqpoint{0.968418in}{1.501983in}}{\pgfqpoint{0.964028in}{1.491384in}}{\pgfqpoint{0.964028in}{1.480334in}}%
\pgfpathcurveto{\pgfqpoint{0.964028in}{1.469284in}}{\pgfqpoint{0.968418in}{1.458685in}}{\pgfqpoint{0.976232in}{1.450871in}}%
\pgfpathcurveto{\pgfqpoint{0.984045in}{1.443057in}}{\pgfqpoint{0.994644in}{1.438667in}}{\pgfqpoint{1.005694in}{1.438667in}}%
\pgfpathclose%
\pgfusepath{stroke,fill}%
\end{pgfscope}%
\begin{pgfscope}%
\pgfpathrectangle{\pgfqpoint{0.375000in}{0.330000in}}{\pgfqpoint{2.325000in}{2.310000in}}%
\pgfusepath{clip}%
\pgfsetbuttcap%
\pgfsetroundjoin%
\definecolor{currentfill}{rgb}{0.000000,0.000000,0.000000}%
\pgfsetfillcolor{currentfill}%
\pgfsetlinewidth{1.003750pt}%
\definecolor{currentstroke}{rgb}{0.000000,0.000000,0.000000}%
\pgfsetstrokecolor{currentstroke}%
\pgfsetdash{}{0pt}%
\pgfpathmoveto{\pgfqpoint{1.005694in}{1.438667in}}%
\pgfpathcurveto{\pgfqpoint{1.016744in}{1.438667in}}{\pgfqpoint{1.027343in}{1.443057in}}{\pgfqpoint{1.035157in}{1.450871in}}%
\pgfpathcurveto{\pgfqpoint{1.042971in}{1.458685in}}{\pgfqpoint{1.047361in}{1.469284in}}{\pgfqpoint{1.047361in}{1.480334in}}%
\pgfpathcurveto{\pgfqpoint{1.047361in}{1.491384in}}{\pgfqpoint{1.042971in}{1.501983in}}{\pgfqpoint{1.035157in}{1.509796in}}%
\pgfpathcurveto{\pgfqpoint{1.027343in}{1.517610in}}{\pgfqpoint{1.016744in}{1.522000in}}{\pgfqpoint{1.005694in}{1.522000in}}%
\pgfpathcurveto{\pgfqpoint{0.994644in}{1.522000in}}{\pgfqpoint{0.984045in}{1.517610in}}{\pgfqpoint{0.976232in}{1.509796in}}%
\pgfpathcurveto{\pgfqpoint{0.968418in}{1.501983in}}{\pgfqpoint{0.964028in}{1.491384in}}{\pgfqpoint{0.964028in}{1.480334in}}%
\pgfpathcurveto{\pgfqpoint{0.964028in}{1.469284in}}{\pgfqpoint{0.968418in}{1.458685in}}{\pgfqpoint{0.976232in}{1.450871in}}%
\pgfpathcurveto{\pgfqpoint{0.984045in}{1.443057in}}{\pgfqpoint{0.994644in}{1.438667in}}{\pgfqpoint{1.005694in}{1.438667in}}%
\pgfpathclose%
\pgfusepath{stroke,fill}%
\end{pgfscope}%
\begin{pgfscope}%
\pgfpathrectangle{\pgfqpoint{0.375000in}{0.330000in}}{\pgfqpoint{2.325000in}{2.310000in}}%
\pgfusepath{clip}%
\pgfsetbuttcap%
\pgfsetroundjoin%
\definecolor{currentfill}{rgb}{0.000000,0.000000,0.000000}%
\pgfsetfillcolor{currentfill}%
\pgfsetlinewidth{1.003750pt}%
\definecolor{currentstroke}{rgb}{0.000000,0.000000,0.000000}%
\pgfsetstrokecolor{currentstroke}%
\pgfsetdash{}{0pt}%
\pgfpathmoveto{\pgfqpoint{1.005694in}{1.438667in}}%
\pgfpathcurveto{\pgfqpoint{1.016744in}{1.438667in}}{\pgfqpoint{1.027343in}{1.443057in}}{\pgfqpoint{1.035157in}{1.450871in}}%
\pgfpathcurveto{\pgfqpoint{1.042971in}{1.458685in}}{\pgfqpoint{1.047361in}{1.469284in}}{\pgfqpoint{1.047361in}{1.480334in}}%
\pgfpathcurveto{\pgfqpoint{1.047361in}{1.491384in}}{\pgfqpoint{1.042971in}{1.501983in}}{\pgfqpoint{1.035157in}{1.509796in}}%
\pgfpathcurveto{\pgfqpoint{1.027343in}{1.517610in}}{\pgfqpoint{1.016744in}{1.522000in}}{\pgfqpoint{1.005694in}{1.522000in}}%
\pgfpathcurveto{\pgfqpoint{0.994644in}{1.522000in}}{\pgfqpoint{0.984045in}{1.517610in}}{\pgfqpoint{0.976232in}{1.509796in}}%
\pgfpathcurveto{\pgfqpoint{0.968418in}{1.501983in}}{\pgfqpoint{0.964028in}{1.491384in}}{\pgfqpoint{0.964028in}{1.480334in}}%
\pgfpathcurveto{\pgfqpoint{0.964028in}{1.469284in}}{\pgfqpoint{0.968418in}{1.458685in}}{\pgfqpoint{0.976232in}{1.450871in}}%
\pgfpathcurveto{\pgfqpoint{0.984045in}{1.443057in}}{\pgfqpoint{0.994644in}{1.438667in}}{\pgfqpoint{1.005694in}{1.438667in}}%
\pgfpathclose%
\pgfusepath{stroke,fill}%
\end{pgfscope}%
\begin{pgfscope}%
\pgfpathrectangle{\pgfqpoint{0.375000in}{0.330000in}}{\pgfqpoint{2.325000in}{2.310000in}}%
\pgfusepath{clip}%
\pgfsetbuttcap%
\pgfsetroundjoin%
\definecolor{currentfill}{rgb}{0.000000,0.000000,0.000000}%
\pgfsetfillcolor{currentfill}%
\pgfsetlinewidth{1.003750pt}%
\definecolor{currentstroke}{rgb}{0.000000,0.000000,0.000000}%
\pgfsetstrokecolor{currentstroke}%
\pgfsetdash{}{0pt}%
\pgfpathmoveto{\pgfqpoint{1.005694in}{1.438667in}}%
\pgfpathcurveto{\pgfqpoint{1.016744in}{1.438667in}}{\pgfqpoint{1.027343in}{1.443057in}}{\pgfqpoint{1.035157in}{1.450871in}}%
\pgfpathcurveto{\pgfqpoint{1.042971in}{1.458685in}}{\pgfqpoint{1.047361in}{1.469284in}}{\pgfqpoint{1.047361in}{1.480334in}}%
\pgfpathcurveto{\pgfqpoint{1.047361in}{1.491384in}}{\pgfqpoint{1.042971in}{1.501983in}}{\pgfqpoint{1.035157in}{1.509796in}}%
\pgfpathcurveto{\pgfqpoint{1.027343in}{1.517610in}}{\pgfqpoint{1.016744in}{1.522000in}}{\pgfqpoint{1.005694in}{1.522000in}}%
\pgfpathcurveto{\pgfqpoint{0.994644in}{1.522000in}}{\pgfqpoint{0.984045in}{1.517610in}}{\pgfqpoint{0.976232in}{1.509796in}}%
\pgfpathcurveto{\pgfqpoint{0.968418in}{1.501983in}}{\pgfqpoint{0.964028in}{1.491384in}}{\pgfqpoint{0.964028in}{1.480334in}}%
\pgfpathcurveto{\pgfqpoint{0.964028in}{1.469284in}}{\pgfqpoint{0.968418in}{1.458685in}}{\pgfqpoint{0.976232in}{1.450871in}}%
\pgfpathcurveto{\pgfqpoint{0.984045in}{1.443057in}}{\pgfqpoint{0.994644in}{1.438667in}}{\pgfqpoint{1.005694in}{1.438667in}}%
\pgfpathclose%
\pgfusepath{stroke,fill}%
\end{pgfscope}%
\begin{pgfscope}%
\pgfpathrectangle{\pgfqpoint{0.375000in}{0.330000in}}{\pgfqpoint{2.325000in}{2.310000in}}%
\pgfusepath{clip}%
\pgfsetbuttcap%
\pgfsetroundjoin%
\definecolor{currentfill}{rgb}{0.000000,0.000000,0.000000}%
\pgfsetfillcolor{currentfill}%
\pgfsetlinewidth{1.003750pt}%
\definecolor{currentstroke}{rgb}{0.000000,0.000000,0.000000}%
\pgfsetstrokecolor{currentstroke}%
\pgfsetdash{}{0pt}%
\pgfpathmoveto{\pgfqpoint{1.005694in}{1.438667in}}%
\pgfpathcurveto{\pgfqpoint{1.016744in}{1.438667in}}{\pgfqpoint{1.027343in}{1.443057in}}{\pgfqpoint{1.035157in}{1.450871in}}%
\pgfpathcurveto{\pgfqpoint{1.042971in}{1.458685in}}{\pgfqpoint{1.047361in}{1.469284in}}{\pgfqpoint{1.047361in}{1.480334in}}%
\pgfpathcurveto{\pgfqpoint{1.047361in}{1.491384in}}{\pgfqpoint{1.042971in}{1.501983in}}{\pgfqpoint{1.035157in}{1.509796in}}%
\pgfpathcurveto{\pgfqpoint{1.027343in}{1.517610in}}{\pgfqpoint{1.016744in}{1.522000in}}{\pgfqpoint{1.005694in}{1.522000in}}%
\pgfpathcurveto{\pgfqpoint{0.994644in}{1.522000in}}{\pgfqpoint{0.984045in}{1.517610in}}{\pgfqpoint{0.976232in}{1.509796in}}%
\pgfpathcurveto{\pgfqpoint{0.968418in}{1.501983in}}{\pgfqpoint{0.964028in}{1.491384in}}{\pgfqpoint{0.964028in}{1.480334in}}%
\pgfpathcurveto{\pgfqpoint{0.964028in}{1.469284in}}{\pgfqpoint{0.968418in}{1.458685in}}{\pgfqpoint{0.976232in}{1.450871in}}%
\pgfpathcurveto{\pgfqpoint{0.984045in}{1.443057in}}{\pgfqpoint{0.994644in}{1.438667in}}{\pgfqpoint{1.005694in}{1.438667in}}%
\pgfpathclose%
\pgfusepath{stroke,fill}%
\end{pgfscope}%
\begin{pgfscope}%
\pgfpathrectangle{\pgfqpoint{0.375000in}{0.330000in}}{\pgfqpoint{2.325000in}{2.310000in}}%
\pgfusepath{clip}%
\pgfsetbuttcap%
\pgfsetroundjoin%
\definecolor{currentfill}{rgb}{0.000000,0.000000,0.000000}%
\pgfsetfillcolor{currentfill}%
\pgfsetlinewidth{1.003750pt}%
\definecolor{currentstroke}{rgb}{0.000000,0.000000,0.000000}%
\pgfsetstrokecolor{currentstroke}%
\pgfsetdash{}{0pt}%
\pgfpathmoveto{\pgfqpoint{1.005694in}{1.438667in}}%
\pgfpathcurveto{\pgfqpoint{1.016744in}{1.438667in}}{\pgfqpoint{1.027343in}{1.443057in}}{\pgfqpoint{1.035157in}{1.450871in}}%
\pgfpathcurveto{\pgfqpoint{1.042971in}{1.458685in}}{\pgfqpoint{1.047361in}{1.469284in}}{\pgfqpoint{1.047361in}{1.480334in}}%
\pgfpathcurveto{\pgfqpoint{1.047361in}{1.491384in}}{\pgfqpoint{1.042971in}{1.501983in}}{\pgfqpoint{1.035157in}{1.509796in}}%
\pgfpathcurveto{\pgfqpoint{1.027343in}{1.517610in}}{\pgfqpoint{1.016744in}{1.522000in}}{\pgfqpoint{1.005694in}{1.522000in}}%
\pgfpathcurveto{\pgfqpoint{0.994644in}{1.522000in}}{\pgfqpoint{0.984045in}{1.517610in}}{\pgfqpoint{0.976232in}{1.509796in}}%
\pgfpathcurveto{\pgfqpoint{0.968418in}{1.501983in}}{\pgfqpoint{0.964028in}{1.491384in}}{\pgfqpoint{0.964028in}{1.480334in}}%
\pgfpathcurveto{\pgfqpoint{0.964028in}{1.469284in}}{\pgfqpoint{0.968418in}{1.458685in}}{\pgfqpoint{0.976232in}{1.450871in}}%
\pgfpathcurveto{\pgfqpoint{0.984045in}{1.443057in}}{\pgfqpoint{0.994644in}{1.438667in}}{\pgfqpoint{1.005694in}{1.438667in}}%
\pgfpathclose%
\pgfusepath{stroke,fill}%
\end{pgfscope}%
\begin{pgfscope}%
\pgfpathrectangle{\pgfqpoint{0.375000in}{0.330000in}}{\pgfqpoint{2.325000in}{2.310000in}}%
\pgfusepath{clip}%
\pgfsetbuttcap%
\pgfsetroundjoin%
\definecolor{currentfill}{rgb}{0.000000,0.000000,0.000000}%
\pgfsetfillcolor{currentfill}%
\pgfsetlinewidth{1.003750pt}%
\definecolor{currentstroke}{rgb}{0.000000,0.000000,0.000000}%
\pgfsetstrokecolor{currentstroke}%
\pgfsetdash{}{0pt}%
\pgfpathmoveto{\pgfqpoint{1.005694in}{1.438667in}}%
\pgfpathcurveto{\pgfqpoint{1.016744in}{1.438667in}}{\pgfqpoint{1.027343in}{1.443057in}}{\pgfqpoint{1.035157in}{1.450871in}}%
\pgfpathcurveto{\pgfqpoint{1.042971in}{1.458685in}}{\pgfqpoint{1.047361in}{1.469284in}}{\pgfqpoint{1.047361in}{1.480334in}}%
\pgfpathcurveto{\pgfqpoint{1.047361in}{1.491384in}}{\pgfqpoint{1.042971in}{1.501983in}}{\pgfqpoint{1.035157in}{1.509796in}}%
\pgfpathcurveto{\pgfqpoint{1.027343in}{1.517610in}}{\pgfqpoint{1.016744in}{1.522000in}}{\pgfqpoint{1.005694in}{1.522000in}}%
\pgfpathcurveto{\pgfqpoint{0.994644in}{1.522000in}}{\pgfqpoint{0.984045in}{1.517610in}}{\pgfqpoint{0.976232in}{1.509796in}}%
\pgfpathcurveto{\pgfqpoint{0.968418in}{1.501983in}}{\pgfqpoint{0.964028in}{1.491384in}}{\pgfqpoint{0.964028in}{1.480334in}}%
\pgfpathcurveto{\pgfqpoint{0.964028in}{1.469284in}}{\pgfqpoint{0.968418in}{1.458685in}}{\pgfqpoint{0.976232in}{1.450871in}}%
\pgfpathcurveto{\pgfqpoint{0.984045in}{1.443057in}}{\pgfqpoint{0.994644in}{1.438667in}}{\pgfqpoint{1.005694in}{1.438667in}}%
\pgfpathclose%
\pgfusepath{stroke,fill}%
\end{pgfscope}%
\begin{pgfscope}%
\pgfpathrectangle{\pgfqpoint{0.375000in}{0.330000in}}{\pgfqpoint{2.325000in}{2.310000in}}%
\pgfusepath{clip}%
\pgfsetbuttcap%
\pgfsetroundjoin%
\definecolor{currentfill}{rgb}{0.000000,0.000000,0.000000}%
\pgfsetfillcolor{currentfill}%
\pgfsetlinewidth{1.003750pt}%
\definecolor{currentstroke}{rgb}{0.000000,0.000000,0.000000}%
\pgfsetstrokecolor{currentstroke}%
\pgfsetdash{}{0pt}%
\pgfpathmoveto{\pgfqpoint{1.005694in}{1.438667in}}%
\pgfpathcurveto{\pgfqpoint{1.016744in}{1.438667in}}{\pgfqpoint{1.027343in}{1.443057in}}{\pgfqpoint{1.035157in}{1.450871in}}%
\pgfpathcurveto{\pgfqpoint{1.042971in}{1.458685in}}{\pgfqpoint{1.047361in}{1.469284in}}{\pgfqpoint{1.047361in}{1.480334in}}%
\pgfpathcurveto{\pgfqpoint{1.047361in}{1.491384in}}{\pgfqpoint{1.042971in}{1.501983in}}{\pgfqpoint{1.035157in}{1.509796in}}%
\pgfpathcurveto{\pgfqpoint{1.027343in}{1.517610in}}{\pgfqpoint{1.016744in}{1.522000in}}{\pgfqpoint{1.005694in}{1.522000in}}%
\pgfpathcurveto{\pgfqpoint{0.994644in}{1.522000in}}{\pgfqpoint{0.984045in}{1.517610in}}{\pgfqpoint{0.976232in}{1.509796in}}%
\pgfpathcurveto{\pgfqpoint{0.968418in}{1.501983in}}{\pgfqpoint{0.964028in}{1.491384in}}{\pgfqpoint{0.964028in}{1.480334in}}%
\pgfpathcurveto{\pgfqpoint{0.964028in}{1.469284in}}{\pgfqpoint{0.968418in}{1.458685in}}{\pgfqpoint{0.976232in}{1.450871in}}%
\pgfpathcurveto{\pgfqpoint{0.984045in}{1.443057in}}{\pgfqpoint{0.994644in}{1.438667in}}{\pgfqpoint{1.005694in}{1.438667in}}%
\pgfpathclose%
\pgfusepath{stroke,fill}%
\end{pgfscope}%
\begin{pgfscope}%
\pgfpathrectangle{\pgfqpoint{0.375000in}{0.330000in}}{\pgfqpoint{2.325000in}{2.310000in}}%
\pgfusepath{clip}%
\pgfsetbuttcap%
\pgfsetroundjoin%
\definecolor{currentfill}{rgb}{0.000000,0.000000,0.000000}%
\pgfsetfillcolor{currentfill}%
\pgfsetlinewidth{1.003750pt}%
\definecolor{currentstroke}{rgb}{0.000000,0.000000,0.000000}%
\pgfsetstrokecolor{currentstroke}%
\pgfsetdash{}{0pt}%
\pgfpathmoveto{\pgfqpoint{1.005694in}{1.438667in}}%
\pgfpathcurveto{\pgfqpoint{1.016744in}{1.438667in}}{\pgfqpoint{1.027343in}{1.443057in}}{\pgfqpoint{1.035157in}{1.450871in}}%
\pgfpathcurveto{\pgfqpoint{1.042971in}{1.458685in}}{\pgfqpoint{1.047361in}{1.469284in}}{\pgfqpoint{1.047361in}{1.480334in}}%
\pgfpathcurveto{\pgfqpoint{1.047361in}{1.491384in}}{\pgfqpoint{1.042971in}{1.501983in}}{\pgfqpoint{1.035157in}{1.509796in}}%
\pgfpathcurveto{\pgfqpoint{1.027343in}{1.517610in}}{\pgfqpoint{1.016744in}{1.522000in}}{\pgfqpoint{1.005694in}{1.522000in}}%
\pgfpathcurveto{\pgfqpoint{0.994644in}{1.522000in}}{\pgfqpoint{0.984045in}{1.517610in}}{\pgfqpoint{0.976232in}{1.509796in}}%
\pgfpathcurveto{\pgfqpoint{0.968418in}{1.501983in}}{\pgfqpoint{0.964028in}{1.491384in}}{\pgfqpoint{0.964028in}{1.480334in}}%
\pgfpathcurveto{\pgfqpoint{0.964028in}{1.469284in}}{\pgfqpoint{0.968418in}{1.458685in}}{\pgfqpoint{0.976232in}{1.450871in}}%
\pgfpathcurveto{\pgfqpoint{0.984045in}{1.443057in}}{\pgfqpoint{0.994644in}{1.438667in}}{\pgfqpoint{1.005694in}{1.438667in}}%
\pgfpathclose%
\pgfusepath{stroke,fill}%
\end{pgfscope}%
\begin{pgfscope}%
\pgfpathrectangle{\pgfqpoint{0.375000in}{0.330000in}}{\pgfqpoint{2.325000in}{2.310000in}}%
\pgfusepath{clip}%
\pgfsetbuttcap%
\pgfsetroundjoin%
\definecolor{currentfill}{rgb}{0.000000,0.000000,0.000000}%
\pgfsetfillcolor{currentfill}%
\pgfsetlinewidth{1.003750pt}%
\definecolor{currentstroke}{rgb}{0.000000,0.000000,0.000000}%
\pgfsetstrokecolor{currentstroke}%
\pgfsetdash{}{0pt}%
\pgfpathmoveto{\pgfqpoint{1.005694in}{1.438667in}}%
\pgfpathcurveto{\pgfqpoint{1.016744in}{1.438667in}}{\pgfqpoint{1.027343in}{1.443057in}}{\pgfqpoint{1.035157in}{1.450871in}}%
\pgfpathcurveto{\pgfqpoint{1.042971in}{1.458685in}}{\pgfqpoint{1.047361in}{1.469284in}}{\pgfqpoint{1.047361in}{1.480334in}}%
\pgfpathcurveto{\pgfqpoint{1.047361in}{1.491384in}}{\pgfqpoint{1.042971in}{1.501983in}}{\pgfqpoint{1.035157in}{1.509796in}}%
\pgfpathcurveto{\pgfqpoint{1.027343in}{1.517610in}}{\pgfqpoint{1.016744in}{1.522000in}}{\pgfqpoint{1.005694in}{1.522000in}}%
\pgfpathcurveto{\pgfqpoint{0.994644in}{1.522000in}}{\pgfqpoint{0.984045in}{1.517610in}}{\pgfqpoint{0.976232in}{1.509796in}}%
\pgfpathcurveto{\pgfqpoint{0.968418in}{1.501983in}}{\pgfqpoint{0.964028in}{1.491384in}}{\pgfqpoint{0.964028in}{1.480334in}}%
\pgfpathcurveto{\pgfqpoint{0.964028in}{1.469284in}}{\pgfqpoint{0.968418in}{1.458685in}}{\pgfqpoint{0.976232in}{1.450871in}}%
\pgfpathcurveto{\pgfqpoint{0.984045in}{1.443057in}}{\pgfqpoint{0.994644in}{1.438667in}}{\pgfqpoint{1.005694in}{1.438667in}}%
\pgfpathclose%
\pgfusepath{stroke,fill}%
\end{pgfscope}%
\begin{pgfscope}%
\pgfpathrectangle{\pgfqpoint{0.375000in}{0.330000in}}{\pgfqpoint{2.325000in}{2.310000in}}%
\pgfusepath{clip}%
\pgfsetbuttcap%
\pgfsetroundjoin%
\definecolor{currentfill}{rgb}{0.000000,0.000000,0.000000}%
\pgfsetfillcolor{currentfill}%
\pgfsetlinewidth{1.003750pt}%
\definecolor{currentstroke}{rgb}{0.000000,0.000000,0.000000}%
\pgfsetstrokecolor{currentstroke}%
\pgfsetdash{}{0pt}%
\pgfpathmoveto{\pgfqpoint{1.005694in}{1.438667in}}%
\pgfpathcurveto{\pgfqpoint{1.016744in}{1.438667in}}{\pgfqpoint{1.027343in}{1.443057in}}{\pgfqpoint{1.035157in}{1.450871in}}%
\pgfpathcurveto{\pgfqpoint{1.042971in}{1.458685in}}{\pgfqpoint{1.047361in}{1.469284in}}{\pgfqpoint{1.047361in}{1.480334in}}%
\pgfpathcurveto{\pgfqpoint{1.047361in}{1.491384in}}{\pgfqpoint{1.042971in}{1.501983in}}{\pgfqpoint{1.035157in}{1.509796in}}%
\pgfpathcurveto{\pgfqpoint{1.027343in}{1.517610in}}{\pgfqpoint{1.016744in}{1.522000in}}{\pgfqpoint{1.005694in}{1.522000in}}%
\pgfpathcurveto{\pgfqpoint{0.994644in}{1.522000in}}{\pgfqpoint{0.984045in}{1.517610in}}{\pgfqpoint{0.976232in}{1.509796in}}%
\pgfpathcurveto{\pgfqpoint{0.968418in}{1.501983in}}{\pgfqpoint{0.964028in}{1.491384in}}{\pgfqpoint{0.964028in}{1.480334in}}%
\pgfpathcurveto{\pgfqpoint{0.964028in}{1.469284in}}{\pgfqpoint{0.968418in}{1.458685in}}{\pgfqpoint{0.976232in}{1.450871in}}%
\pgfpathcurveto{\pgfqpoint{0.984045in}{1.443057in}}{\pgfqpoint{0.994644in}{1.438667in}}{\pgfqpoint{1.005694in}{1.438667in}}%
\pgfpathclose%
\pgfusepath{stroke,fill}%
\end{pgfscope}%
\begin{pgfscope}%
\pgfpathrectangle{\pgfqpoint{0.375000in}{0.330000in}}{\pgfqpoint{2.325000in}{2.310000in}}%
\pgfusepath{clip}%
\pgfsetbuttcap%
\pgfsetroundjoin%
\definecolor{currentfill}{rgb}{0.000000,0.000000,0.000000}%
\pgfsetfillcolor{currentfill}%
\pgfsetlinewidth{1.003750pt}%
\definecolor{currentstroke}{rgb}{0.000000,0.000000,0.000000}%
\pgfsetstrokecolor{currentstroke}%
\pgfsetdash{}{0pt}%
\pgfpathmoveto{\pgfqpoint{1.005694in}{1.438667in}}%
\pgfpathcurveto{\pgfqpoint{1.016744in}{1.438667in}}{\pgfqpoint{1.027343in}{1.443057in}}{\pgfqpoint{1.035157in}{1.450871in}}%
\pgfpathcurveto{\pgfqpoint{1.042971in}{1.458685in}}{\pgfqpoint{1.047361in}{1.469284in}}{\pgfqpoint{1.047361in}{1.480334in}}%
\pgfpathcurveto{\pgfqpoint{1.047361in}{1.491384in}}{\pgfqpoint{1.042971in}{1.501983in}}{\pgfqpoint{1.035157in}{1.509796in}}%
\pgfpathcurveto{\pgfqpoint{1.027343in}{1.517610in}}{\pgfqpoint{1.016744in}{1.522000in}}{\pgfqpoint{1.005694in}{1.522000in}}%
\pgfpathcurveto{\pgfqpoint{0.994644in}{1.522000in}}{\pgfqpoint{0.984045in}{1.517610in}}{\pgfqpoint{0.976232in}{1.509796in}}%
\pgfpathcurveto{\pgfqpoint{0.968418in}{1.501983in}}{\pgfqpoint{0.964028in}{1.491384in}}{\pgfqpoint{0.964028in}{1.480334in}}%
\pgfpathcurveto{\pgfqpoint{0.964028in}{1.469284in}}{\pgfqpoint{0.968418in}{1.458685in}}{\pgfqpoint{0.976232in}{1.450871in}}%
\pgfpathcurveto{\pgfqpoint{0.984045in}{1.443057in}}{\pgfqpoint{0.994644in}{1.438667in}}{\pgfqpoint{1.005694in}{1.438667in}}%
\pgfpathclose%
\pgfusepath{stroke,fill}%
\end{pgfscope}%
\begin{pgfscope}%
\pgfpathrectangle{\pgfqpoint{0.375000in}{0.330000in}}{\pgfqpoint{2.325000in}{2.310000in}}%
\pgfusepath{clip}%
\pgfsetbuttcap%
\pgfsetroundjoin%
\definecolor{currentfill}{rgb}{0.000000,0.000000,0.000000}%
\pgfsetfillcolor{currentfill}%
\pgfsetlinewidth{1.003750pt}%
\definecolor{currentstroke}{rgb}{0.000000,0.000000,0.000000}%
\pgfsetstrokecolor{currentstroke}%
\pgfsetdash{}{0pt}%
\pgfpathmoveto{\pgfqpoint{1.005694in}{1.438667in}}%
\pgfpathcurveto{\pgfqpoint{1.016744in}{1.438667in}}{\pgfqpoint{1.027343in}{1.443057in}}{\pgfqpoint{1.035157in}{1.450871in}}%
\pgfpathcurveto{\pgfqpoint{1.042971in}{1.458685in}}{\pgfqpoint{1.047361in}{1.469284in}}{\pgfqpoint{1.047361in}{1.480334in}}%
\pgfpathcurveto{\pgfqpoint{1.047361in}{1.491384in}}{\pgfqpoint{1.042971in}{1.501983in}}{\pgfqpoint{1.035157in}{1.509796in}}%
\pgfpathcurveto{\pgfqpoint{1.027343in}{1.517610in}}{\pgfqpoint{1.016744in}{1.522000in}}{\pgfqpoint{1.005694in}{1.522000in}}%
\pgfpathcurveto{\pgfqpoint{0.994644in}{1.522000in}}{\pgfqpoint{0.984045in}{1.517610in}}{\pgfqpoint{0.976232in}{1.509796in}}%
\pgfpathcurveto{\pgfqpoint{0.968418in}{1.501983in}}{\pgfqpoint{0.964028in}{1.491384in}}{\pgfqpoint{0.964028in}{1.480334in}}%
\pgfpathcurveto{\pgfqpoint{0.964028in}{1.469284in}}{\pgfqpoint{0.968418in}{1.458685in}}{\pgfqpoint{0.976232in}{1.450871in}}%
\pgfpathcurveto{\pgfqpoint{0.984045in}{1.443057in}}{\pgfqpoint{0.994644in}{1.438667in}}{\pgfqpoint{1.005694in}{1.438667in}}%
\pgfpathclose%
\pgfusepath{stroke,fill}%
\end{pgfscope}%
\begin{pgfscope}%
\pgfpathrectangle{\pgfqpoint{0.375000in}{0.330000in}}{\pgfqpoint{2.325000in}{2.310000in}}%
\pgfusepath{clip}%
\pgfsetbuttcap%
\pgfsetroundjoin%
\definecolor{currentfill}{rgb}{0.000000,0.000000,0.000000}%
\pgfsetfillcolor{currentfill}%
\pgfsetlinewidth{1.003750pt}%
\definecolor{currentstroke}{rgb}{0.000000,0.000000,0.000000}%
\pgfsetstrokecolor{currentstroke}%
\pgfsetdash{}{0pt}%
\pgfpathmoveto{\pgfqpoint{1.005694in}{1.438667in}}%
\pgfpathcurveto{\pgfqpoint{1.016744in}{1.438667in}}{\pgfqpoint{1.027343in}{1.443057in}}{\pgfqpoint{1.035157in}{1.450871in}}%
\pgfpathcurveto{\pgfqpoint{1.042971in}{1.458685in}}{\pgfqpoint{1.047361in}{1.469284in}}{\pgfqpoint{1.047361in}{1.480334in}}%
\pgfpathcurveto{\pgfqpoint{1.047361in}{1.491384in}}{\pgfqpoint{1.042971in}{1.501983in}}{\pgfqpoint{1.035157in}{1.509796in}}%
\pgfpathcurveto{\pgfqpoint{1.027343in}{1.517610in}}{\pgfqpoint{1.016744in}{1.522000in}}{\pgfqpoint{1.005694in}{1.522000in}}%
\pgfpathcurveto{\pgfqpoint{0.994644in}{1.522000in}}{\pgfqpoint{0.984045in}{1.517610in}}{\pgfqpoint{0.976232in}{1.509796in}}%
\pgfpathcurveto{\pgfqpoint{0.968418in}{1.501983in}}{\pgfqpoint{0.964028in}{1.491384in}}{\pgfqpoint{0.964028in}{1.480334in}}%
\pgfpathcurveto{\pgfqpoint{0.964028in}{1.469284in}}{\pgfqpoint{0.968418in}{1.458685in}}{\pgfqpoint{0.976232in}{1.450871in}}%
\pgfpathcurveto{\pgfqpoint{0.984045in}{1.443057in}}{\pgfqpoint{0.994644in}{1.438667in}}{\pgfqpoint{1.005694in}{1.438667in}}%
\pgfpathclose%
\pgfusepath{stroke,fill}%
\end{pgfscope}%
\begin{pgfscope}%
\pgfpathrectangle{\pgfqpoint{0.375000in}{0.330000in}}{\pgfqpoint{2.325000in}{2.310000in}}%
\pgfusepath{clip}%
\pgfsetbuttcap%
\pgfsetroundjoin%
\definecolor{currentfill}{rgb}{0.000000,0.000000,0.000000}%
\pgfsetfillcolor{currentfill}%
\pgfsetlinewidth{1.003750pt}%
\definecolor{currentstroke}{rgb}{0.000000,0.000000,0.000000}%
\pgfsetstrokecolor{currentstroke}%
\pgfsetdash{}{0pt}%
\pgfpathmoveto{\pgfqpoint{1.005694in}{1.438667in}}%
\pgfpathcurveto{\pgfqpoint{1.016744in}{1.438667in}}{\pgfqpoint{1.027343in}{1.443057in}}{\pgfqpoint{1.035157in}{1.450871in}}%
\pgfpathcurveto{\pgfqpoint{1.042971in}{1.458685in}}{\pgfqpoint{1.047361in}{1.469284in}}{\pgfqpoint{1.047361in}{1.480334in}}%
\pgfpathcurveto{\pgfqpoint{1.047361in}{1.491384in}}{\pgfqpoint{1.042971in}{1.501983in}}{\pgfqpoint{1.035157in}{1.509796in}}%
\pgfpathcurveto{\pgfqpoint{1.027343in}{1.517610in}}{\pgfqpoint{1.016744in}{1.522000in}}{\pgfqpoint{1.005694in}{1.522000in}}%
\pgfpathcurveto{\pgfqpoint{0.994644in}{1.522000in}}{\pgfqpoint{0.984045in}{1.517610in}}{\pgfqpoint{0.976232in}{1.509796in}}%
\pgfpathcurveto{\pgfqpoint{0.968418in}{1.501983in}}{\pgfqpoint{0.964028in}{1.491384in}}{\pgfqpoint{0.964028in}{1.480334in}}%
\pgfpathcurveto{\pgfqpoint{0.964028in}{1.469284in}}{\pgfqpoint{0.968418in}{1.458685in}}{\pgfqpoint{0.976232in}{1.450871in}}%
\pgfpathcurveto{\pgfqpoint{0.984045in}{1.443057in}}{\pgfqpoint{0.994644in}{1.438667in}}{\pgfqpoint{1.005694in}{1.438667in}}%
\pgfpathclose%
\pgfusepath{stroke,fill}%
\end{pgfscope}%
\begin{pgfscope}%
\pgfpathrectangle{\pgfqpoint{0.375000in}{0.330000in}}{\pgfqpoint{2.325000in}{2.310000in}}%
\pgfusepath{clip}%
\pgfsetbuttcap%
\pgfsetroundjoin%
\definecolor{currentfill}{rgb}{0.000000,0.000000,0.000000}%
\pgfsetfillcolor{currentfill}%
\pgfsetlinewidth{1.003750pt}%
\definecolor{currentstroke}{rgb}{0.000000,0.000000,0.000000}%
\pgfsetstrokecolor{currentstroke}%
\pgfsetdash{}{0pt}%
\pgfpathmoveto{\pgfqpoint{1.005694in}{1.438667in}}%
\pgfpathcurveto{\pgfqpoint{1.016744in}{1.438667in}}{\pgfqpoint{1.027343in}{1.443057in}}{\pgfqpoint{1.035157in}{1.450871in}}%
\pgfpathcurveto{\pgfqpoint{1.042971in}{1.458685in}}{\pgfqpoint{1.047361in}{1.469284in}}{\pgfqpoint{1.047361in}{1.480334in}}%
\pgfpathcurveto{\pgfqpoint{1.047361in}{1.491384in}}{\pgfqpoint{1.042971in}{1.501983in}}{\pgfqpoint{1.035157in}{1.509796in}}%
\pgfpathcurveto{\pgfqpoint{1.027343in}{1.517610in}}{\pgfqpoint{1.016744in}{1.522000in}}{\pgfqpoint{1.005694in}{1.522000in}}%
\pgfpathcurveto{\pgfqpoint{0.994644in}{1.522000in}}{\pgfqpoint{0.984045in}{1.517610in}}{\pgfqpoint{0.976232in}{1.509796in}}%
\pgfpathcurveto{\pgfqpoint{0.968418in}{1.501983in}}{\pgfqpoint{0.964028in}{1.491384in}}{\pgfqpoint{0.964028in}{1.480334in}}%
\pgfpathcurveto{\pgfqpoint{0.964028in}{1.469284in}}{\pgfqpoint{0.968418in}{1.458685in}}{\pgfqpoint{0.976232in}{1.450871in}}%
\pgfpathcurveto{\pgfqpoint{0.984045in}{1.443057in}}{\pgfqpoint{0.994644in}{1.438667in}}{\pgfqpoint{1.005694in}{1.438667in}}%
\pgfpathclose%
\pgfusepath{stroke,fill}%
\end{pgfscope}%
\begin{pgfscope}%
\pgfpathrectangle{\pgfqpoint{0.375000in}{0.330000in}}{\pgfqpoint{2.325000in}{2.310000in}}%
\pgfusepath{clip}%
\pgfsetbuttcap%
\pgfsetroundjoin%
\definecolor{currentfill}{rgb}{0.000000,0.000000,0.000000}%
\pgfsetfillcolor{currentfill}%
\pgfsetlinewidth{1.003750pt}%
\definecolor{currentstroke}{rgb}{0.000000,0.000000,0.000000}%
\pgfsetstrokecolor{currentstroke}%
\pgfsetdash{}{0pt}%
\pgfpathmoveto{\pgfqpoint{1.005694in}{1.438667in}}%
\pgfpathcurveto{\pgfqpoint{1.016744in}{1.438667in}}{\pgfqpoint{1.027343in}{1.443057in}}{\pgfqpoint{1.035157in}{1.450871in}}%
\pgfpathcurveto{\pgfqpoint{1.042971in}{1.458685in}}{\pgfqpoint{1.047361in}{1.469284in}}{\pgfqpoint{1.047361in}{1.480334in}}%
\pgfpathcurveto{\pgfqpoint{1.047361in}{1.491384in}}{\pgfqpoint{1.042971in}{1.501983in}}{\pgfqpoint{1.035157in}{1.509796in}}%
\pgfpathcurveto{\pgfqpoint{1.027343in}{1.517610in}}{\pgfqpoint{1.016744in}{1.522000in}}{\pgfqpoint{1.005694in}{1.522000in}}%
\pgfpathcurveto{\pgfqpoint{0.994644in}{1.522000in}}{\pgfqpoint{0.984045in}{1.517610in}}{\pgfqpoint{0.976232in}{1.509796in}}%
\pgfpathcurveto{\pgfqpoint{0.968418in}{1.501983in}}{\pgfqpoint{0.964028in}{1.491384in}}{\pgfqpoint{0.964028in}{1.480334in}}%
\pgfpathcurveto{\pgfqpoint{0.964028in}{1.469284in}}{\pgfqpoint{0.968418in}{1.458685in}}{\pgfqpoint{0.976232in}{1.450871in}}%
\pgfpathcurveto{\pgfqpoint{0.984045in}{1.443057in}}{\pgfqpoint{0.994644in}{1.438667in}}{\pgfqpoint{1.005694in}{1.438667in}}%
\pgfpathclose%
\pgfusepath{stroke,fill}%
\end{pgfscope}%
\begin{pgfscope}%
\pgfpathrectangle{\pgfqpoint{0.375000in}{0.330000in}}{\pgfqpoint{2.325000in}{2.310000in}}%
\pgfusepath{clip}%
\pgfsetbuttcap%
\pgfsetroundjoin%
\definecolor{currentfill}{rgb}{0.000000,0.000000,0.000000}%
\pgfsetfillcolor{currentfill}%
\pgfsetlinewidth{1.003750pt}%
\definecolor{currentstroke}{rgb}{0.000000,0.000000,0.000000}%
\pgfsetstrokecolor{currentstroke}%
\pgfsetdash{}{0pt}%
\pgfpathmoveto{\pgfqpoint{1.005694in}{1.438667in}}%
\pgfpathcurveto{\pgfqpoint{1.016744in}{1.438667in}}{\pgfqpoint{1.027343in}{1.443057in}}{\pgfqpoint{1.035157in}{1.450871in}}%
\pgfpathcurveto{\pgfqpoint{1.042971in}{1.458685in}}{\pgfqpoint{1.047361in}{1.469284in}}{\pgfqpoint{1.047361in}{1.480334in}}%
\pgfpathcurveto{\pgfqpoint{1.047361in}{1.491384in}}{\pgfqpoint{1.042971in}{1.501983in}}{\pgfqpoint{1.035157in}{1.509796in}}%
\pgfpathcurveto{\pgfqpoint{1.027343in}{1.517610in}}{\pgfqpoint{1.016744in}{1.522000in}}{\pgfqpoint{1.005694in}{1.522000in}}%
\pgfpathcurveto{\pgfqpoint{0.994644in}{1.522000in}}{\pgfqpoint{0.984045in}{1.517610in}}{\pgfqpoint{0.976232in}{1.509796in}}%
\pgfpathcurveto{\pgfqpoint{0.968418in}{1.501983in}}{\pgfqpoint{0.964028in}{1.491384in}}{\pgfqpoint{0.964028in}{1.480334in}}%
\pgfpathcurveto{\pgfqpoint{0.964028in}{1.469284in}}{\pgfqpoint{0.968418in}{1.458685in}}{\pgfqpoint{0.976232in}{1.450871in}}%
\pgfpathcurveto{\pgfqpoint{0.984045in}{1.443057in}}{\pgfqpoint{0.994644in}{1.438667in}}{\pgfqpoint{1.005694in}{1.438667in}}%
\pgfpathclose%
\pgfusepath{stroke,fill}%
\end{pgfscope}%
\begin{pgfscope}%
\pgfpathrectangle{\pgfqpoint{0.375000in}{0.330000in}}{\pgfqpoint{2.325000in}{2.310000in}}%
\pgfusepath{clip}%
\pgfsetbuttcap%
\pgfsetroundjoin%
\definecolor{currentfill}{rgb}{0.000000,0.000000,0.000000}%
\pgfsetfillcolor{currentfill}%
\pgfsetlinewidth{1.003750pt}%
\definecolor{currentstroke}{rgb}{0.000000,0.000000,0.000000}%
\pgfsetstrokecolor{currentstroke}%
\pgfsetdash{}{0pt}%
\pgfpathmoveto{\pgfqpoint{1.005694in}{1.438667in}}%
\pgfpathcurveto{\pgfqpoint{1.016744in}{1.438667in}}{\pgfqpoint{1.027343in}{1.443057in}}{\pgfqpoint{1.035157in}{1.450871in}}%
\pgfpathcurveto{\pgfqpoint{1.042971in}{1.458685in}}{\pgfqpoint{1.047361in}{1.469284in}}{\pgfqpoint{1.047361in}{1.480334in}}%
\pgfpathcurveto{\pgfqpoint{1.047361in}{1.491384in}}{\pgfqpoint{1.042971in}{1.501983in}}{\pgfqpoint{1.035157in}{1.509796in}}%
\pgfpathcurveto{\pgfqpoint{1.027343in}{1.517610in}}{\pgfqpoint{1.016744in}{1.522000in}}{\pgfqpoint{1.005694in}{1.522000in}}%
\pgfpathcurveto{\pgfqpoint{0.994644in}{1.522000in}}{\pgfqpoint{0.984045in}{1.517610in}}{\pgfqpoint{0.976232in}{1.509796in}}%
\pgfpathcurveto{\pgfqpoint{0.968418in}{1.501983in}}{\pgfqpoint{0.964028in}{1.491384in}}{\pgfqpoint{0.964028in}{1.480334in}}%
\pgfpathcurveto{\pgfqpoint{0.964028in}{1.469284in}}{\pgfqpoint{0.968418in}{1.458685in}}{\pgfqpoint{0.976232in}{1.450871in}}%
\pgfpathcurveto{\pgfqpoint{0.984045in}{1.443057in}}{\pgfqpoint{0.994644in}{1.438667in}}{\pgfqpoint{1.005694in}{1.438667in}}%
\pgfpathclose%
\pgfusepath{stroke,fill}%
\end{pgfscope}%
\begin{pgfscope}%
\pgfpathrectangle{\pgfqpoint{0.375000in}{0.330000in}}{\pgfqpoint{2.325000in}{2.310000in}}%
\pgfusepath{clip}%
\pgfsetbuttcap%
\pgfsetroundjoin%
\definecolor{currentfill}{rgb}{0.000000,0.000000,0.000000}%
\pgfsetfillcolor{currentfill}%
\pgfsetlinewidth{1.003750pt}%
\definecolor{currentstroke}{rgb}{0.000000,0.000000,0.000000}%
\pgfsetstrokecolor{currentstroke}%
\pgfsetdash{}{0pt}%
\pgfpathmoveto{\pgfqpoint{1.005694in}{1.438667in}}%
\pgfpathcurveto{\pgfqpoint{1.016744in}{1.438667in}}{\pgfqpoint{1.027343in}{1.443057in}}{\pgfqpoint{1.035157in}{1.450871in}}%
\pgfpathcurveto{\pgfqpoint{1.042971in}{1.458685in}}{\pgfqpoint{1.047361in}{1.469284in}}{\pgfqpoint{1.047361in}{1.480334in}}%
\pgfpathcurveto{\pgfqpoint{1.047361in}{1.491384in}}{\pgfqpoint{1.042971in}{1.501983in}}{\pgfqpoint{1.035157in}{1.509796in}}%
\pgfpathcurveto{\pgfqpoint{1.027343in}{1.517610in}}{\pgfqpoint{1.016744in}{1.522000in}}{\pgfqpoint{1.005694in}{1.522000in}}%
\pgfpathcurveto{\pgfqpoint{0.994644in}{1.522000in}}{\pgfqpoint{0.984045in}{1.517610in}}{\pgfqpoint{0.976232in}{1.509796in}}%
\pgfpathcurveto{\pgfqpoint{0.968418in}{1.501983in}}{\pgfqpoint{0.964028in}{1.491384in}}{\pgfqpoint{0.964028in}{1.480334in}}%
\pgfpathcurveto{\pgfqpoint{0.964028in}{1.469284in}}{\pgfqpoint{0.968418in}{1.458685in}}{\pgfqpoint{0.976232in}{1.450871in}}%
\pgfpathcurveto{\pgfqpoint{0.984045in}{1.443057in}}{\pgfqpoint{0.994644in}{1.438667in}}{\pgfqpoint{1.005694in}{1.438667in}}%
\pgfpathclose%
\pgfusepath{stroke,fill}%
\end{pgfscope}%
\begin{pgfscope}%
\pgfpathrectangle{\pgfqpoint{0.375000in}{0.330000in}}{\pgfqpoint{2.325000in}{2.310000in}}%
\pgfusepath{clip}%
\pgfsetbuttcap%
\pgfsetroundjoin%
\definecolor{currentfill}{rgb}{0.000000,0.000000,0.000000}%
\pgfsetfillcolor{currentfill}%
\pgfsetlinewidth{1.003750pt}%
\definecolor{currentstroke}{rgb}{0.000000,0.000000,0.000000}%
\pgfsetstrokecolor{currentstroke}%
\pgfsetdash{}{0pt}%
\pgfpathmoveto{\pgfqpoint{1.005694in}{1.438667in}}%
\pgfpathcurveto{\pgfqpoint{1.016744in}{1.438667in}}{\pgfqpoint{1.027343in}{1.443057in}}{\pgfqpoint{1.035157in}{1.450871in}}%
\pgfpathcurveto{\pgfqpoint{1.042971in}{1.458685in}}{\pgfqpoint{1.047361in}{1.469284in}}{\pgfqpoint{1.047361in}{1.480334in}}%
\pgfpathcurveto{\pgfqpoint{1.047361in}{1.491384in}}{\pgfqpoint{1.042971in}{1.501983in}}{\pgfqpoint{1.035157in}{1.509796in}}%
\pgfpathcurveto{\pgfqpoint{1.027343in}{1.517610in}}{\pgfqpoint{1.016744in}{1.522000in}}{\pgfqpoint{1.005694in}{1.522000in}}%
\pgfpathcurveto{\pgfqpoint{0.994644in}{1.522000in}}{\pgfqpoint{0.984045in}{1.517610in}}{\pgfqpoint{0.976232in}{1.509796in}}%
\pgfpathcurveto{\pgfqpoint{0.968418in}{1.501983in}}{\pgfqpoint{0.964028in}{1.491384in}}{\pgfqpoint{0.964028in}{1.480334in}}%
\pgfpathcurveto{\pgfqpoint{0.964028in}{1.469284in}}{\pgfqpoint{0.968418in}{1.458685in}}{\pgfqpoint{0.976232in}{1.450871in}}%
\pgfpathcurveto{\pgfqpoint{0.984045in}{1.443057in}}{\pgfqpoint{0.994644in}{1.438667in}}{\pgfqpoint{1.005694in}{1.438667in}}%
\pgfpathclose%
\pgfusepath{stroke,fill}%
\end{pgfscope}%
\begin{pgfscope}%
\pgfpathrectangle{\pgfqpoint{0.375000in}{0.330000in}}{\pgfqpoint{2.325000in}{2.310000in}}%
\pgfusepath{clip}%
\pgfsetbuttcap%
\pgfsetroundjoin%
\definecolor{currentfill}{rgb}{0.000000,0.000000,0.000000}%
\pgfsetfillcolor{currentfill}%
\pgfsetlinewidth{1.003750pt}%
\definecolor{currentstroke}{rgb}{0.000000,0.000000,0.000000}%
\pgfsetstrokecolor{currentstroke}%
\pgfsetdash{}{0pt}%
\pgfpathmoveto{\pgfqpoint{1.005694in}{1.438667in}}%
\pgfpathcurveto{\pgfqpoint{1.016744in}{1.438667in}}{\pgfqpoint{1.027343in}{1.443057in}}{\pgfqpoint{1.035157in}{1.450871in}}%
\pgfpathcurveto{\pgfqpoint{1.042971in}{1.458685in}}{\pgfqpoint{1.047361in}{1.469284in}}{\pgfqpoint{1.047361in}{1.480334in}}%
\pgfpathcurveto{\pgfqpoint{1.047361in}{1.491384in}}{\pgfqpoint{1.042971in}{1.501983in}}{\pgfqpoint{1.035157in}{1.509796in}}%
\pgfpathcurveto{\pgfqpoint{1.027343in}{1.517610in}}{\pgfqpoint{1.016744in}{1.522000in}}{\pgfqpoint{1.005694in}{1.522000in}}%
\pgfpathcurveto{\pgfqpoint{0.994644in}{1.522000in}}{\pgfqpoint{0.984045in}{1.517610in}}{\pgfqpoint{0.976232in}{1.509796in}}%
\pgfpathcurveto{\pgfqpoint{0.968418in}{1.501983in}}{\pgfqpoint{0.964028in}{1.491384in}}{\pgfqpoint{0.964028in}{1.480334in}}%
\pgfpathcurveto{\pgfqpoint{0.964028in}{1.469284in}}{\pgfqpoint{0.968418in}{1.458685in}}{\pgfqpoint{0.976232in}{1.450871in}}%
\pgfpathcurveto{\pgfqpoint{0.984045in}{1.443057in}}{\pgfqpoint{0.994644in}{1.438667in}}{\pgfqpoint{1.005694in}{1.438667in}}%
\pgfpathclose%
\pgfusepath{stroke,fill}%
\end{pgfscope}%
\begin{pgfscope}%
\pgfpathrectangle{\pgfqpoint{0.375000in}{0.330000in}}{\pgfqpoint{2.325000in}{2.310000in}}%
\pgfusepath{clip}%
\pgfsetbuttcap%
\pgfsetroundjoin%
\definecolor{currentfill}{rgb}{0.000000,0.000000,0.000000}%
\pgfsetfillcolor{currentfill}%
\pgfsetlinewidth{1.003750pt}%
\definecolor{currentstroke}{rgb}{0.000000,0.000000,0.000000}%
\pgfsetstrokecolor{currentstroke}%
\pgfsetdash{}{0pt}%
\pgfpathmoveto{\pgfqpoint{1.005694in}{1.438667in}}%
\pgfpathcurveto{\pgfqpoint{1.016744in}{1.438667in}}{\pgfqpoint{1.027343in}{1.443057in}}{\pgfqpoint{1.035157in}{1.450871in}}%
\pgfpathcurveto{\pgfqpoint{1.042971in}{1.458685in}}{\pgfqpoint{1.047361in}{1.469284in}}{\pgfqpoint{1.047361in}{1.480334in}}%
\pgfpathcurveto{\pgfqpoint{1.047361in}{1.491384in}}{\pgfqpoint{1.042971in}{1.501983in}}{\pgfqpoint{1.035157in}{1.509796in}}%
\pgfpathcurveto{\pgfqpoint{1.027343in}{1.517610in}}{\pgfqpoint{1.016744in}{1.522000in}}{\pgfqpoint{1.005694in}{1.522000in}}%
\pgfpathcurveto{\pgfqpoint{0.994644in}{1.522000in}}{\pgfqpoint{0.984045in}{1.517610in}}{\pgfqpoint{0.976232in}{1.509796in}}%
\pgfpathcurveto{\pgfqpoint{0.968418in}{1.501983in}}{\pgfqpoint{0.964028in}{1.491384in}}{\pgfqpoint{0.964028in}{1.480334in}}%
\pgfpathcurveto{\pgfqpoint{0.964028in}{1.469284in}}{\pgfqpoint{0.968418in}{1.458685in}}{\pgfqpoint{0.976232in}{1.450871in}}%
\pgfpathcurveto{\pgfqpoint{0.984045in}{1.443057in}}{\pgfqpoint{0.994644in}{1.438667in}}{\pgfqpoint{1.005694in}{1.438667in}}%
\pgfpathclose%
\pgfusepath{stroke,fill}%
\end{pgfscope}%
\begin{pgfscope}%
\pgfpathrectangle{\pgfqpoint{0.375000in}{0.330000in}}{\pgfqpoint{2.325000in}{2.310000in}}%
\pgfusepath{clip}%
\pgfsetbuttcap%
\pgfsetroundjoin%
\definecolor{currentfill}{rgb}{0.000000,0.000000,0.000000}%
\pgfsetfillcolor{currentfill}%
\pgfsetlinewidth{1.003750pt}%
\definecolor{currentstroke}{rgb}{0.000000,0.000000,0.000000}%
\pgfsetstrokecolor{currentstroke}%
\pgfsetdash{}{0pt}%
\pgfpathmoveto{\pgfqpoint{1.530548in}{1.438667in}}%
\pgfpathcurveto{\pgfqpoint{1.541598in}{1.438667in}}{\pgfqpoint{1.552197in}{1.443057in}}{\pgfqpoint{1.560011in}{1.450871in}}%
\pgfpathcurveto{\pgfqpoint{1.567825in}{1.458685in}}{\pgfqpoint{1.572215in}{1.469284in}}{\pgfqpoint{1.572215in}{1.480334in}}%
\pgfpathcurveto{\pgfqpoint{1.572215in}{1.491384in}}{\pgfqpoint{1.567825in}{1.501983in}}{\pgfqpoint{1.560011in}{1.509796in}}%
\pgfpathcurveto{\pgfqpoint{1.552197in}{1.517610in}}{\pgfqpoint{1.541598in}{1.522000in}}{\pgfqpoint{1.530548in}{1.522000in}}%
\pgfpathcurveto{\pgfqpoint{1.519498in}{1.522000in}}{\pgfqpoint{1.508899in}{1.517610in}}{\pgfqpoint{1.501085in}{1.509796in}}%
\pgfpathcurveto{\pgfqpoint{1.493272in}{1.501983in}}{\pgfqpoint{1.488881in}{1.491384in}}{\pgfqpoint{1.488881in}{1.480334in}}%
\pgfpathcurveto{\pgfqpoint{1.488881in}{1.469284in}}{\pgfqpoint{1.493272in}{1.458685in}}{\pgfqpoint{1.501085in}{1.450871in}}%
\pgfpathcurveto{\pgfqpoint{1.508899in}{1.443057in}}{\pgfqpoint{1.519498in}{1.438667in}}{\pgfqpoint{1.530548in}{1.438667in}}%
\pgfpathclose%
\pgfusepath{stroke,fill}%
\end{pgfscope}%
\begin{pgfscope}%
\pgfpathrectangle{\pgfqpoint{0.375000in}{0.330000in}}{\pgfqpoint{2.325000in}{2.310000in}}%
\pgfusepath{clip}%
\pgfsetbuttcap%
\pgfsetroundjoin%
\definecolor{currentfill}{rgb}{0.000000,0.000000,0.000000}%
\pgfsetfillcolor{currentfill}%
\pgfsetlinewidth{1.003750pt}%
\definecolor{currentstroke}{rgb}{0.000000,0.000000,0.000000}%
\pgfsetstrokecolor{currentstroke}%
\pgfsetdash{}{0pt}%
\pgfpathmoveto{\pgfqpoint{1.530548in}{1.438667in}}%
\pgfpathcurveto{\pgfqpoint{1.541598in}{1.438667in}}{\pgfqpoint{1.552197in}{1.443057in}}{\pgfqpoint{1.560011in}{1.450871in}}%
\pgfpathcurveto{\pgfqpoint{1.567825in}{1.458685in}}{\pgfqpoint{1.572215in}{1.469284in}}{\pgfqpoint{1.572215in}{1.480334in}}%
\pgfpathcurveto{\pgfqpoint{1.572215in}{1.491384in}}{\pgfqpoint{1.567825in}{1.501983in}}{\pgfqpoint{1.560011in}{1.509796in}}%
\pgfpathcurveto{\pgfqpoint{1.552197in}{1.517610in}}{\pgfqpoint{1.541598in}{1.522000in}}{\pgfqpoint{1.530548in}{1.522000in}}%
\pgfpathcurveto{\pgfqpoint{1.519498in}{1.522000in}}{\pgfqpoint{1.508899in}{1.517610in}}{\pgfqpoint{1.501085in}{1.509796in}}%
\pgfpathcurveto{\pgfqpoint{1.493272in}{1.501983in}}{\pgfqpoint{1.488881in}{1.491384in}}{\pgfqpoint{1.488881in}{1.480334in}}%
\pgfpathcurveto{\pgfqpoint{1.488881in}{1.469284in}}{\pgfqpoint{1.493272in}{1.458685in}}{\pgfqpoint{1.501085in}{1.450871in}}%
\pgfpathcurveto{\pgfqpoint{1.508899in}{1.443057in}}{\pgfqpoint{1.519498in}{1.438667in}}{\pgfqpoint{1.530548in}{1.438667in}}%
\pgfpathclose%
\pgfusepath{stroke,fill}%
\end{pgfscope}%
\begin{pgfscope}%
\pgfpathrectangle{\pgfqpoint{0.375000in}{0.330000in}}{\pgfqpoint{2.325000in}{2.310000in}}%
\pgfusepath{clip}%
\pgfsetbuttcap%
\pgfsetroundjoin%
\definecolor{currentfill}{rgb}{0.000000,0.000000,0.000000}%
\pgfsetfillcolor{currentfill}%
\pgfsetlinewidth{1.003750pt}%
\definecolor{currentstroke}{rgb}{0.000000,0.000000,0.000000}%
\pgfsetstrokecolor{currentstroke}%
\pgfsetdash{}{0pt}%
\pgfpathmoveto{\pgfqpoint{1.530548in}{1.438667in}}%
\pgfpathcurveto{\pgfqpoint{1.541598in}{1.438667in}}{\pgfqpoint{1.552197in}{1.443057in}}{\pgfqpoint{1.560011in}{1.450871in}}%
\pgfpathcurveto{\pgfqpoint{1.567825in}{1.458685in}}{\pgfqpoint{1.572215in}{1.469284in}}{\pgfqpoint{1.572215in}{1.480334in}}%
\pgfpathcurveto{\pgfqpoint{1.572215in}{1.491384in}}{\pgfqpoint{1.567825in}{1.501983in}}{\pgfqpoint{1.560011in}{1.509796in}}%
\pgfpathcurveto{\pgfqpoint{1.552197in}{1.517610in}}{\pgfqpoint{1.541598in}{1.522000in}}{\pgfqpoint{1.530548in}{1.522000in}}%
\pgfpathcurveto{\pgfqpoint{1.519498in}{1.522000in}}{\pgfqpoint{1.508899in}{1.517610in}}{\pgfqpoint{1.501085in}{1.509796in}}%
\pgfpathcurveto{\pgfqpoint{1.493272in}{1.501983in}}{\pgfqpoint{1.488881in}{1.491384in}}{\pgfqpoint{1.488881in}{1.480334in}}%
\pgfpathcurveto{\pgfqpoint{1.488881in}{1.469284in}}{\pgfqpoint{1.493272in}{1.458685in}}{\pgfqpoint{1.501085in}{1.450871in}}%
\pgfpathcurveto{\pgfqpoint{1.508899in}{1.443057in}}{\pgfqpoint{1.519498in}{1.438667in}}{\pgfqpoint{1.530548in}{1.438667in}}%
\pgfpathclose%
\pgfusepath{stroke,fill}%
\end{pgfscope}%
\begin{pgfscope}%
\pgfpathrectangle{\pgfqpoint{0.375000in}{0.330000in}}{\pgfqpoint{2.325000in}{2.310000in}}%
\pgfusepath{clip}%
\pgfsetbuttcap%
\pgfsetroundjoin%
\definecolor{currentfill}{rgb}{0.000000,0.000000,0.000000}%
\pgfsetfillcolor{currentfill}%
\pgfsetlinewidth{1.003750pt}%
\definecolor{currentstroke}{rgb}{0.000000,0.000000,0.000000}%
\pgfsetstrokecolor{currentstroke}%
\pgfsetdash{}{0pt}%
\pgfpathmoveto{\pgfqpoint{1.530548in}{1.438667in}}%
\pgfpathcurveto{\pgfqpoint{1.541598in}{1.438667in}}{\pgfqpoint{1.552197in}{1.443057in}}{\pgfqpoint{1.560011in}{1.450871in}}%
\pgfpathcurveto{\pgfqpoint{1.567825in}{1.458685in}}{\pgfqpoint{1.572215in}{1.469284in}}{\pgfqpoint{1.572215in}{1.480334in}}%
\pgfpathcurveto{\pgfqpoint{1.572215in}{1.491384in}}{\pgfqpoint{1.567825in}{1.501983in}}{\pgfqpoint{1.560011in}{1.509796in}}%
\pgfpathcurveto{\pgfqpoint{1.552197in}{1.517610in}}{\pgfqpoint{1.541598in}{1.522000in}}{\pgfqpoint{1.530548in}{1.522000in}}%
\pgfpathcurveto{\pgfqpoint{1.519498in}{1.522000in}}{\pgfqpoint{1.508899in}{1.517610in}}{\pgfqpoint{1.501085in}{1.509796in}}%
\pgfpathcurveto{\pgfqpoint{1.493272in}{1.501983in}}{\pgfqpoint{1.488881in}{1.491384in}}{\pgfqpoint{1.488881in}{1.480334in}}%
\pgfpathcurveto{\pgfqpoint{1.488881in}{1.469284in}}{\pgfqpoint{1.493272in}{1.458685in}}{\pgfqpoint{1.501085in}{1.450871in}}%
\pgfpathcurveto{\pgfqpoint{1.508899in}{1.443057in}}{\pgfqpoint{1.519498in}{1.438667in}}{\pgfqpoint{1.530548in}{1.438667in}}%
\pgfpathclose%
\pgfusepath{stroke,fill}%
\end{pgfscope}%
\begin{pgfscope}%
\pgfpathrectangle{\pgfqpoint{0.375000in}{0.330000in}}{\pgfqpoint{2.325000in}{2.310000in}}%
\pgfusepath{clip}%
\pgfsetbuttcap%
\pgfsetroundjoin%
\definecolor{currentfill}{rgb}{0.000000,0.000000,0.000000}%
\pgfsetfillcolor{currentfill}%
\pgfsetlinewidth{1.003750pt}%
\definecolor{currentstroke}{rgb}{0.000000,0.000000,0.000000}%
\pgfsetstrokecolor{currentstroke}%
\pgfsetdash{}{0pt}%
\pgfpathmoveto{\pgfqpoint{1.530548in}{1.438667in}}%
\pgfpathcurveto{\pgfqpoint{1.541598in}{1.438667in}}{\pgfqpoint{1.552197in}{1.443057in}}{\pgfqpoint{1.560011in}{1.450871in}}%
\pgfpathcurveto{\pgfqpoint{1.567825in}{1.458685in}}{\pgfqpoint{1.572215in}{1.469284in}}{\pgfqpoint{1.572215in}{1.480334in}}%
\pgfpathcurveto{\pgfqpoint{1.572215in}{1.491384in}}{\pgfqpoint{1.567825in}{1.501983in}}{\pgfqpoint{1.560011in}{1.509796in}}%
\pgfpathcurveto{\pgfqpoint{1.552197in}{1.517610in}}{\pgfqpoint{1.541598in}{1.522000in}}{\pgfqpoint{1.530548in}{1.522000in}}%
\pgfpathcurveto{\pgfqpoint{1.519498in}{1.522000in}}{\pgfqpoint{1.508899in}{1.517610in}}{\pgfqpoint{1.501085in}{1.509796in}}%
\pgfpathcurveto{\pgfqpoint{1.493272in}{1.501983in}}{\pgfqpoint{1.488881in}{1.491384in}}{\pgfqpoint{1.488881in}{1.480334in}}%
\pgfpathcurveto{\pgfqpoint{1.488881in}{1.469284in}}{\pgfqpoint{1.493272in}{1.458685in}}{\pgfqpoint{1.501085in}{1.450871in}}%
\pgfpathcurveto{\pgfqpoint{1.508899in}{1.443057in}}{\pgfqpoint{1.519498in}{1.438667in}}{\pgfqpoint{1.530548in}{1.438667in}}%
\pgfpathclose%
\pgfusepath{stroke,fill}%
\end{pgfscope}%
\begin{pgfscope}%
\pgfpathrectangle{\pgfqpoint{0.375000in}{0.330000in}}{\pgfqpoint{2.325000in}{2.310000in}}%
\pgfusepath{clip}%
\pgfsetbuttcap%
\pgfsetroundjoin%
\definecolor{currentfill}{rgb}{0.000000,0.000000,0.000000}%
\pgfsetfillcolor{currentfill}%
\pgfsetlinewidth{1.003750pt}%
\definecolor{currentstroke}{rgb}{0.000000,0.000000,0.000000}%
\pgfsetstrokecolor{currentstroke}%
\pgfsetdash{}{0pt}%
\pgfpathmoveto{\pgfqpoint{1.530548in}{1.438667in}}%
\pgfpathcurveto{\pgfqpoint{1.541598in}{1.438667in}}{\pgfqpoint{1.552197in}{1.443057in}}{\pgfqpoint{1.560011in}{1.450871in}}%
\pgfpathcurveto{\pgfqpoint{1.567825in}{1.458685in}}{\pgfqpoint{1.572215in}{1.469284in}}{\pgfqpoint{1.572215in}{1.480334in}}%
\pgfpathcurveto{\pgfqpoint{1.572215in}{1.491384in}}{\pgfqpoint{1.567825in}{1.501983in}}{\pgfqpoint{1.560011in}{1.509796in}}%
\pgfpathcurveto{\pgfqpoint{1.552197in}{1.517610in}}{\pgfqpoint{1.541598in}{1.522000in}}{\pgfqpoint{1.530548in}{1.522000in}}%
\pgfpathcurveto{\pgfqpoint{1.519498in}{1.522000in}}{\pgfqpoint{1.508899in}{1.517610in}}{\pgfqpoint{1.501085in}{1.509796in}}%
\pgfpathcurveto{\pgfqpoint{1.493272in}{1.501983in}}{\pgfqpoint{1.488881in}{1.491384in}}{\pgfqpoint{1.488881in}{1.480334in}}%
\pgfpathcurveto{\pgfqpoint{1.488881in}{1.469284in}}{\pgfqpoint{1.493272in}{1.458685in}}{\pgfqpoint{1.501085in}{1.450871in}}%
\pgfpathcurveto{\pgfqpoint{1.508899in}{1.443057in}}{\pgfqpoint{1.519498in}{1.438667in}}{\pgfqpoint{1.530548in}{1.438667in}}%
\pgfpathclose%
\pgfusepath{stroke,fill}%
\end{pgfscope}%
\begin{pgfscope}%
\pgfpathrectangle{\pgfqpoint{0.375000in}{0.330000in}}{\pgfqpoint{2.325000in}{2.310000in}}%
\pgfusepath{clip}%
\pgfsetbuttcap%
\pgfsetroundjoin%
\definecolor{currentfill}{rgb}{0.000000,0.000000,0.000000}%
\pgfsetfillcolor{currentfill}%
\pgfsetlinewidth{1.003750pt}%
\definecolor{currentstroke}{rgb}{0.000000,0.000000,0.000000}%
\pgfsetstrokecolor{currentstroke}%
\pgfsetdash{}{0pt}%
\pgfpathmoveto{\pgfqpoint{1.530548in}{1.438667in}}%
\pgfpathcurveto{\pgfqpoint{1.541598in}{1.438667in}}{\pgfqpoint{1.552197in}{1.443057in}}{\pgfqpoint{1.560011in}{1.450871in}}%
\pgfpathcurveto{\pgfqpoint{1.567825in}{1.458685in}}{\pgfqpoint{1.572215in}{1.469284in}}{\pgfqpoint{1.572215in}{1.480334in}}%
\pgfpathcurveto{\pgfqpoint{1.572215in}{1.491384in}}{\pgfqpoint{1.567825in}{1.501983in}}{\pgfqpoint{1.560011in}{1.509796in}}%
\pgfpathcurveto{\pgfqpoint{1.552197in}{1.517610in}}{\pgfqpoint{1.541598in}{1.522000in}}{\pgfqpoint{1.530548in}{1.522000in}}%
\pgfpathcurveto{\pgfqpoint{1.519498in}{1.522000in}}{\pgfqpoint{1.508899in}{1.517610in}}{\pgfqpoint{1.501085in}{1.509796in}}%
\pgfpathcurveto{\pgfqpoint{1.493272in}{1.501983in}}{\pgfqpoint{1.488881in}{1.491384in}}{\pgfqpoint{1.488881in}{1.480334in}}%
\pgfpathcurveto{\pgfqpoint{1.488881in}{1.469284in}}{\pgfqpoint{1.493272in}{1.458685in}}{\pgfqpoint{1.501085in}{1.450871in}}%
\pgfpathcurveto{\pgfqpoint{1.508899in}{1.443057in}}{\pgfqpoint{1.519498in}{1.438667in}}{\pgfqpoint{1.530548in}{1.438667in}}%
\pgfpathclose%
\pgfusepath{stroke,fill}%
\end{pgfscope}%
\begin{pgfscope}%
\pgfpathrectangle{\pgfqpoint{0.375000in}{0.330000in}}{\pgfqpoint{2.325000in}{2.310000in}}%
\pgfusepath{clip}%
\pgfsetbuttcap%
\pgfsetroundjoin%
\definecolor{currentfill}{rgb}{0.000000,0.000000,0.000000}%
\pgfsetfillcolor{currentfill}%
\pgfsetlinewidth{1.003750pt}%
\definecolor{currentstroke}{rgb}{0.000000,0.000000,0.000000}%
\pgfsetstrokecolor{currentstroke}%
\pgfsetdash{}{0pt}%
\pgfpathmoveto{\pgfqpoint{1.530548in}{1.438667in}}%
\pgfpathcurveto{\pgfqpoint{1.541598in}{1.438667in}}{\pgfqpoint{1.552197in}{1.443057in}}{\pgfqpoint{1.560011in}{1.450871in}}%
\pgfpathcurveto{\pgfqpoint{1.567825in}{1.458685in}}{\pgfqpoint{1.572215in}{1.469284in}}{\pgfqpoint{1.572215in}{1.480334in}}%
\pgfpathcurveto{\pgfqpoint{1.572215in}{1.491384in}}{\pgfqpoint{1.567825in}{1.501983in}}{\pgfqpoint{1.560011in}{1.509796in}}%
\pgfpathcurveto{\pgfqpoint{1.552197in}{1.517610in}}{\pgfqpoint{1.541598in}{1.522000in}}{\pgfqpoint{1.530548in}{1.522000in}}%
\pgfpathcurveto{\pgfqpoint{1.519498in}{1.522000in}}{\pgfqpoint{1.508899in}{1.517610in}}{\pgfqpoint{1.501085in}{1.509796in}}%
\pgfpathcurveto{\pgfqpoint{1.493272in}{1.501983in}}{\pgfqpoint{1.488881in}{1.491384in}}{\pgfqpoint{1.488881in}{1.480334in}}%
\pgfpathcurveto{\pgfqpoint{1.488881in}{1.469284in}}{\pgfqpoint{1.493272in}{1.458685in}}{\pgfqpoint{1.501085in}{1.450871in}}%
\pgfpathcurveto{\pgfqpoint{1.508899in}{1.443057in}}{\pgfqpoint{1.519498in}{1.438667in}}{\pgfqpoint{1.530548in}{1.438667in}}%
\pgfpathclose%
\pgfusepath{stroke,fill}%
\end{pgfscope}%
\begin{pgfscope}%
\pgfpathrectangle{\pgfqpoint{0.375000in}{0.330000in}}{\pgfqpoint{2.325000in}{2.310000in}}%
\pgfusepath{clip}%
\pgfsetbuttcap%
\pgfsetroundjoin%
\definecolor{currentfill}{rgb}{0.000000,0.000000,0.000000}%
\pgfsetfillcolor{currentfill}%
\pgfsetlinewidth{1.003750pt}%
\definecolor{currentstroke}{rgb}{0.000000,0.000000,0.000000}%
\pgfsetstrokecolor{currentstroke}%
\pgfsetdash{}{0pt}%
\pgfpathmoveto{\pgfqpoint{1.530548in}{1.438667in}}%
\pgfpathcurveto{\pgfqpoint{1.541598in}{1.438667in}}{\pgfqpoint{1.552197in}{1.443057in}}{\pgfqpoint{1.560011in}{1.450871in}}%
\pgfpathcurveto{\pgfqpoint{1.567825in}{1.458685in}}{\pgfqpoint{1.572215in}{1.469284in}}{\pgfqpoint{1.572215in}{1.480334in}}%
\pgfpathcurveto{\pgfqpoint{1.572215in}{1.491384in}}{\pgfqpoint{1.567825in}{1.501983in}}{\pgfqpoint{1.560011in}{1.509796in}}%
\pgfpathcurveto{\pgfqpoint{1.552197in}{1.517610in}}{\pgfqpoint{1.541598in}{1.522000in}}{\pgfqpoint{1.530548in}{1.522000in}}%
\pgfpathcurveto{\pgfqpoint{1.519498in}{1.522000in}}{\pgfqpoint{1.508899in}{1.517610in}}{\pgfqpoint{1.501085in}{1.509796in}}%
\pgfpathcurveto{\pgfqpoint{1.493272in}{1.501983in}}{\pgfqpoint{1.488881in}{1.491384in}}{\pgfqpoint{1.488881in}{1.480334in}}%
\pgfpathcurveto{\pgfqpoint{1.488881in}{1.469284in}}{\pgfqpoint{1.493272in}{1.458685in}}{\pgfqpoint{1.501085in}{1.450871in}}%
\pgfpathcurveto{\pgfqpoint{1.508899in}{1.443057in}}{\pgfqpoint{1.519498in}{1.438667in}}{\pgfqpoint{1.530548in}{1.438667in}}%
\pgfpathclose%
\pgfusepath{stroke,fill}%
\end{pgfscope}%
\begin{pgfscope}%
\pgfpathrectangle{\pgfqpoint{0.375000in}{0.330000in}}{\pgfqpoint{2.325000in}{2.310000in}}%
\pgfusepath{clip}%
\pgfsetbuttcap%
\pgfsetroundjoin%
\definecolor{currentfill}{rgb}{0.000000,0.000000,0.000000}%
\pgfsetfillcolor{currentfill}%
\pgfsetlinewidth{1.003750pt}%
\definecolor{currentstroke}{rgb}{0.000000,0.000000,0.000000}%
\pgfsetstrokecolor{currentstroke}%
\pgfsetdash{}{0pt}%
\pgfpathmoveto{\pgfqpoint{1.530548in}{1.438667in}}%
\pgfpathcurveto{\pgfqpoint{1.541598in}{1.438667in}}{\pgfqpoint{1.552197in}{1.443057in}}{\pgfqpoint{1.560011in}{1.450871in}}%
\pgfpathcurveto{\pgfqpoint{1.567825in}{1.458685in}}{\pgfqpoint{1.572215in}{1.469284in}}{\pgfqpoint{1.572215in}{1.480334in}}%
\pgfpathcurveto{\pgfqpoint{1.572215in}{1.491384in}}{\pgfqpoint{1.567825in}{1.501983in}}{\pgfqpoint{1.560011in}{1.509796in}}%
\pgfpathcurveto{\pgfqpoint{1.552197in}{1.517610in}}{\pgfqpoint{1.541598in}{1.522000in}}{\pgfqpoint{1.530548in}{1.522000in}}%
\pgfpathcurveto{\pgfqpoint{1.519498in}{1.522000in}}{\pgfqpoint{1.508899in}{1.517610in}}{\pgfqpoint{1.501085in}{1.509796in}}%
\pgfpathcurveto{\pgfqpoint{1.493272in}{1.501983in}}{\pgfqpoint{1.488881in}{1.491384in}}{\pgfqpoint{1.488881in}{1.480334in}}%
\pgfpathcurveto{\pgfqpoint{1.488881in}{1.469284in}}{\pgfqpoint{1.493272in}{1.458685in}}{\pgfqpoint{1.501085in}{1.450871in}}%
\pgfpathcurveto{\pgfqpoint{1.508899in}{1.443057in}}{\pgfqpoint{1.519498in}{1.438667in}}{\pgfqpoint{1.530548in}{1.438667in}}%
\pgfpathclose%
\pgfusepath{stroke,fill}%
\end{pgfscope}%
\begin{pgfscope}%
\pgfpathrectangle{\pgfqpoint{0.375000in}{0.330000in}}{\pgfqpoint{2.325000in}{2.310000in}}%
\pgfusepath{clip}%
\pgfsetbuttcap%
\pgfsetroundjoin%
\definecolor{currentfill}{rgb}{0.000000,0.000000,0.000000}%
\pgfsetfillcolor{currentfill}%
\pgfsetlinewidth{1.003750pt}%
\definecolor{currentstroke}{rgb}{0.000000,0.000000,0.000000}%
\pgfsetstrokecolor{currentstroke}%
\pgfsetdash{}{0pt}%
\pgfpathmoveto{\pgfqpoint{1.530548in}{1.438667in}}%
\pgfpathcurveto{\pgfqpoint{1.541598in}{1.438667in}}{\pgfqpoint{1.552197in}{1.443057in}}{\pgfqpoint{1.560011in}{1.450871in}}%
\pgfpathcurveto{\pgfqpoint{1.567825in}{1.458685in}}{\pgfqpoint{1.572215in}{1.469284in}}{\pgfqpoint{1.572215in}{1.480334in}}%
\pgfpathcurveto{\pgfqpoint{1.572215in}{1.491384in}}{\pgfqpoint{1.567825in}{1.501983in}}{\pgfqpoint{1.560011in}{1.509796in}}%
\pgfpathcurveto{\pgfqpoint{1.552197in}{1.517610in}}{\pgfqpoint{1.541598in}{1.522000in}}{\pgfqpoint{1.530548in}{1.522000in}}%
\pgfpathcurveto{\pgfqpoint{1.519498in}{1.522000in}}{\pgfqpoint{1.508899in}{1.517610in}}{\pgfqpoint{1.501085in}{1.509796in}}%
\pgfpathcurveto{\pgfqpoint{1.493272in}{1.501983in}}{\pgfqpoint{1.488881in}{1.491384in}}{\pgfqpoint{1.488881in}{1.480334in}}%
\pgfpathcurveto{\pgfqpoint{1.488881in}{1.469284in}}{\pgfqpoint{1.493272in}{1.458685in}}{\pgfqpoint{1.501085in}{1.450871in}}%
\pgfpathcurveto{\pgfqpoint{1.508899in}{1.443057in}}{\pgfqpoint{1.519498in}{1.438667in}}{\pgfqpoint{1.530548in}{1.438667in}}%
\pgfpathclose%
\pgfusepath{stroke,fill}%
\end{pgfscope}%
\begin{pgfscope}%
\pgfpathrectangle{\pgfqpoint{0.375000in}{0.330000in}}{\pgfqpoint{2.325000in}{2.310000in}}%
\pgfusepath{clip}%
\pgfsetbuttcap%
\pgfsetroundjoin%
\definecolor{currentfill}{rgb}{0.000000,0.000000,0.000000}%
\pgfsetfillcolor{currentfill}%
\pgfsetlinewidth{1.003750pt}%
\definecolor{currentstroke}{rgb}{0.000000,0.000000,0.000000}%
\pgfsetstrokecolor{currentstroke}%
\pgfsetdash{}{0pt}%
\pgfpathmoveto{\pgfqpoint{1.530548in}{1.438667in}}%
\pgfpathcurveto{\pgfqpoint{1.541598in}{1.438667in}}{\pgfqpoint{1.552197in}{1.443057in}}{\pgfqpoint{1.560011in}{1.450871in}}%
\pgfpathcurveto{\pgfqpoint{1.567825in}{1.458685in}}{\pgfqpoint{1.572215in}{1.469284in}}{\pgfqpoint{1.572215in}{1.480334in}}%
\pgfpathcurveto{\pgfqpoint{1.572215in}{1.491384in}}{\pgfqpoint{1.567825in}{1.501983in}}{\pgfqpoint{1.560011in}{1.509796in}}%
\pgfpathcurveto{\pgfqpoint{1.552197in}{1.517610in}}{\pgfqpoint{1.541598in}{1.522000in}}{\pgfqpoint{1.530548in}{1.522000in}}%
\pgfpathcurveto{\pgfqpoint{1.519498in}{1.522000in}}{\pgfqpoint{1.508899in}{1.517610in}}{\pgfqpoint{1.501085in}{1.509796in}}%
\pgfpathcurveto{\pgfqpoint{1.493272in}{1.501983in}}{\pgfqpoint{1.488881in}{1.491384in}}{\pgfqpoint{1.488881in}{1.480334in}}%
\pgfpathcurveto{\pgfqpoint{1.488881in}{1.469284in}}{\pgfqpoint{1.493272in}{1.458685in}}{\pgfqpoint{1.501085in}{1.450871in}}%
\pgfpathcurveto{\pgfqpoint{1.508899in}{1.443057in}}{\pgfqpoint{1.519498in}{1.438667in}}{\pgfqpoint{1.530548in}{1.438667in}}%
\pgfpathclose%
\pgfusepath{stroke,fill}%
\end{pgfscope}%
\begin{pgfscope}%
\pgfpathrectangle{\pgfqpoint{0.375000in}{0.330000in}}{\pgfqpoint{2.325000in}{2.310000in}}%
\pgfusepath{clip}%
\pgfsetbuttcap%
\pgfsetroundjoin%
\definecolor{currentfill}{rgb}{0.000000,0.000000,0.000000}%
\pgfsetfillcolor{currentfill}%
\pgfsetlinewidth{1.003750pt}%
\definecolor{currentstroke}{rgb}{0.000000,0.000000,0.000000}%
\pgfsetstrokecolor{currentstroke}%
\pgfsetdash{}{0pt}%
\pgfpathmoveto{\pgfqpoint{1.530548in}{1.438667in}}%
\pgfpathcurveto{\pgfqpoint{1.541598in}{1.438667in}}{\pgfqpoint{1.552197in}{1.443057in}}{\pgfqpoint{1.560011in}{1.450871in}}%
\pgfpathcurveto{\pgfqpoint{1.567825in}{1.458685in}}{\pgfqpoint{1.572215in}{1.469284in}}{\pgfqpoint{1.572215in}{1.480334in}}%
\pgfpathcurveto{\pgfqpoint{1.572215in}{1.491384in}}{\pgfqpoint{1.567825in}{1.501983in}}{\pgfqpoint{1.560011in}{1.509796in}}%
\pgfpathcurveto{\pgfqpoint{1.552197in}{1.517610in}}{\pgfqpoint{1.541598in}{1.522000in}}{\pgfqpoint{1.530548in}{1.522000in}}%
\pgfpathcurveto{\pgfqpoint{1.519498in}{1.522000in}}{\pgfqpoint{1.508899in}{1.517610in}}{\pgfqpoint{1.501085in}{1.509796in}}%
\pgfpathcurveto{\pgfqpoint{1.493272in}{1.501983in}}{\pgfqpoint{1.488881in}{1.491384in}}{\pgfqpoint{1.488881in}{1.480334in}}%
\pgfpathcurveto{\pgfqpoint{1.488881in}{1.469284in}}{\pgfqpoint{1.493272in}{1.458685in}}{\pgfqpoint{1.501085in}{1.450871in}}%
\pgfpathcurveto{\pgfqpoint{1.508899in}{1.443057in}}{\pgfqpoint{1.519498in}{1.438667in}}{\pgfqpoint{1.530548in}{1.438667in}}%
\pgfpathclose%
\pgfusepath{stroke,fill}%
\end{pgfscope}%
\begin{pgfscope}%
\pgfpathrectangle{\pgfqpoint{0.375000in}{0.330000in}}{\pgfqpoint{2.325000in}{2.310000in}}%
\pgfusepath{clip}%
\pgfsetbuttcap%
\pgfsetroundjoin%
\definecolor{currentfill}{rgb}{0.000000,0.000000,0.000000}%
\pgfsetfillcolor{currentfill}%
\pgfsetlinewidth{1.003750pt}%
\definecolor{currentstroke}{rgb}{0.000000,0.000000,0.000000}%
\pgfsetstrokecolor{currentstroke}%
\pgfsetdash{}{0pt}%
\pgfpathmoveto{\pgfqpoint{1.530548in}{1.438667in}}%
\pgfpathcurveto{\pgfqpoint{1.541598in}{1.438667in}}{\pgfqpoint{1.552197in}{1.443057in}}{\pgfqpoint{1.560011in}{1.450871in}}%
\pgfpathcurveto{\pgfqpoint{1.567825in}{1.458685in}}{\pgfqpoint{1.572215in}{1.469284in}}{\pgfqpoint{1.572215in}{1.480334in}}%
\pgfpathcurveto{\pgfqpoint{1.572215in}{1.491384in}}{\pgfqpoint{1.567825in}{1.501983in}}{\pgfqpoint{1.560011in}{1.509796in}}%
\pgfpathcurveto{\pgfqpoint{1.552197in}{1.517610in}}{\pgfqpoint{1.541598in}{1.522000in}}{\pgfqpoint{1.530548in}{1.522000in}}%
\pgfpathcurveto{\pgfqpoint{1.519498in}{1.522000in}}{\pgfqpoint{1.508899in}{1.517610in}}{\pgfqpoint{1.501085in}{1.509796in}}%
\pgfpathcurveto{\pgfqpoint{1.493272in}{1.501983in}}{\pgfqpoint{1.488881in}{1.491384in}}{\pgfqpoint{1.488881in}{1.480334in}}%
\pgfpathcurveto{\pgfqpoint{1.488881in}{1.469284in}}{\pgfqpoint{1.493272in}{1.458685in}}{\pgfqpoint{1.501085in}{1.450871in}}%
\pgfpathcurveto{\pgfqpoint{1.508899in}{1.443057in}}{\pgfqpoint{1.519498in}{1.438667in}}{\pgfqpoint{1.530548in}{1.438667in}}%
\pgfpathclose%
\pgfusepath{stroke,fill}%
\end{pgfscope}%
\begin{pgfscope}%
\pgfpathrectangle{\pgfqpoint{0.375000in}{0.330000in}}{\pgfqpoint{2.325000in}{2.310000in}}%
\pgfusepath{clip}%
\pgfsetbuttcap%
\pgfsetroundjoin%
\definecolor{currentfill}{rgb}{0.000000,0.000000,0.000000}%
\pgfsetfillcolor{currentfill}%
\pgfsetlinewidth{1.003750pt}%
\definecolor{currentstroke}{rgb}{0.000000,0.000000,0.000000}%
\pgfsetstrokecolor{currentstroke}%
\pgfsetdash{}{0pt}%
\pgfpathmoveto{\pgfqpoint{1.530548in}{1.438667in}}%
\pgfpathcurveto{\pgfqpoint{1.541598in}{1.438667in}}{\pgfqpoint{1.552197in}{1.443057in}}{\pgfqpoint{1.560011in}{1.450871in}}%
\pgfpathcurveto{\pgfqpoint{1.567825in}{1.458685in}}{\pgfqpoint{1.572215in}{1.469284in}}{\pgfqpoint{1.572215in}{1.480334in}}%
\pgfpathcurveto{\pgfqpoint{1.572215in}{1.491384in}}{\pgfqpoint{1.567825in}{1.501983in}}{\pgfqpoint{1.560011in}{1.509796in}}%
\pgfpathcurveto{\pgfqpoint{1.552197in}{1.517610in}}{\pgfqpoint{1.541598in}{1.522000in}}{\pgfqpoint{1.530548in}{1.522000in}}%
\pgfpathcurveto{\pgfqpoint{1.519498in}{1.522000in}}{\pgfqpoint{1.508899in}{1.517610in}}{\pgfqpoint{1.501085in}{1.509796in}}%
\pgfpathcurveto{\pgfqpoint{1.493272in}{1.501983in}}{\pgfqpoint{1.488881in}{1.491384in}}{\pgfqpoint{1.488881in}{1.480334in}}%
\pgfpathcurveto{\pgfqpoint{1.488881in}{1.469284in}}{\pgfqpoint{1.493272in}{1.458685in}}{\pgfqpoint{1.501085in}{1.450871in}}%
\pgfpathcurveto{\pgfqpoint{1.508899in}{1.443057in}}{\pgfqpoint{1.519498in}{1.438667in}}{\pgfqpoint{1.530548in}{1.438667in}}%
\pgfpathclose%
\pgfusepath{stroke,fill}%
\end{pgfscope}%
\begin{pgfscope}%
\pgfpathrectangle{\pgfqpoint{0.375000in}{0.330000in}}{\pgfqpoint{2.325000in}{2.310000in}}%
\pgfusepath{clip}%
\pgfsetbuttcap%
\pgfsetroundjoin%
\definecolor{currentfill}{rgb}{0.000000,0.000000,0.000000}%
\pgfsetfillcolor{currentfill}%
\pgfsetlinewidth{1.003750pt}%
\definecolor{currentstroke}{rgb}{0.000000,0.000000,0.000000}%
\pgfsetstrokecolor{currentstroke}%
\pgfsetdash{}{0pt}%
\pgfpathmoveto{\pgfqpoint{1.530548in}{1.438667in}}%
\pgfpathcurveto{\pgfqpoint{1.541598in}{1.438667in}}{\pgfqpoint{1.552197in}{1.443057in}}{\pgfqpoint{1.560011in}{1.450871in}}%
\pgfpathcurveto{\pgfqpoint{1.567825in}{1.458685in}}{\pgfqpoint{1.572215in}{1.469284in}}{\pgfqpoint{1.572215in}{1.480334in}}%
\pgfpathcurveto{\pgfqpoint{1.572215in}{1.491384in}}{\pgfqpoint{1.567825in}{1.501983in}}{\pgfqpoint{1.560011in}{1.509796in}}%
\pgfpathcurveto{\pgfqpoint{1.552197in}{1.517610in}}{\pgfqpoint{1.541598in}{1.522000in}}{\pgfqpoint{1.530548in}{1.522000in}}%
\pgfpathcurveto{\pgfqpoint{1.519498in}{1.522000in}}{\pgfqpoint{1.508899in}{1.517610in}}{\pgfqpoint{1.501085in}{1.509796in}}%
\pgfpathcurveto{\pgfqpoint{1.493272in}{1.501983in}}{\pgfqpoint{1.488881in}{1.491384in}}{\pgfqpoint{1.488881in}{1.480334in}}%
\pgfpathcurveto{\pgfqpoint{1.488881in}{1.469284in}}{\pgfqpoint{1.493272in}{1.458685in}}{\pgfqpoint{1.501085in}{1.450871in}}%
\pgfpathcurveto{\pgfqpoint{1.508899in}{1.443057in}}{\pgfqpoint{1.519498in}{1.438667in}}{\pgfqpoint{1.530548in}{1.438667in}}%
\pgfpathclose%
\pgfusepath{stroke,fill}%
\end{pgfscope}%
\begin{pgfscope}%
\pgfpathrectangle{\pgfqpoint{0.375000in}{0.330000in}}{\pgfqpoint{2.325000in}{2.310000in}}%
\pgfusepath{clip}%
\pgfsetbuttcap%
\pgfsetroundjoin%
\definecolor{currentfill}{rgb}{0.000000,0.000000,0.000000}%
\pgfsetfillcolor{currentfill}%
\pgfsetlinewidth{1.003750pt}%
\definecolor{currentstroke}{rgb}{0.000000,0.000000,0.000000}%
\pgfsetstrokecolor{currentstroke}%
\pgfsetdash{}{0pt}%
\pgfpathmoveto{\pgfqpoint{1.530548in}{1.438667in}}%
\pgfpathcurveto{\pgfqpoint{1.541598in}{1.438667in}}{\pgfqpoint{1.552197in}{1.443057in}}{\pgfqpoint{1.560011in}{1.450871in}}%
\pgfpathcurveto{\pgfqpoint{1.567825in}{1.458685in}}{\pgfqpoint{1.572215in}{1.469284in}}{\pgfqpoint{1.572215in}{1.480334in}}%
\pgfpathcurveto{\pgfqpoint{1.572215in}{1.491384in}}{\pgfqpoint{1.567825in}{1.501983in}}{\pgfqpoint{1.560011in}{1.509796in}}%
\pgfpathcurveto{\pgfqpoint{1.552197in}{1.517610in}}{\pgfqpoint{1.541598in}{1.522000in}}{\pgfqpoint{1.530548in}{1.522000in}}%
\pgfpathcurveto{\pgfqpoint{1.519498in}{1.522000in}}{\pgfqpoint{1.508899in}{1.517610in}}{\pgfqpoint{1.501085in}{1.509796in}}%
\pgfpathcurveto{\pgfqpoint{1.493272in}{1.501983in}}{\pgfqpoint{1.488881in}{1.491384in}}{\pgfqpoint{1.488881in}{1.480334in}}%
\pgfpathcurveto{\pgfqpoint{1.488881in}{1.469284in}}{\pgfqpoint{1.493272in}{1.458685in}}{\pgfqpoint{1.501085in}{1.450871in}}%
\pgfpathcurveto{\pgfqpoint{1.508899in}{1.443057in}}{\pgfqpoint{1.519498in}{1.438667in}}{\pgfqpoint{1.530548in}{1.438667in}}%
\pgfpathclose%
\pgfusepath{stroke,fill}%
\end{pgfscope}%
\begin{pgfscope}%
\pgfpathrectangle{\pgfqpoint{0.375000in}{0.330000in}}{\pgfqpoint{2.325000in}{2.310000in}}%
\pgfusepath{clip}%
\pgfsetbuttcap%
\pgfsetroundjoin%
\definecolor{currentfill}{rgb}{0.000000,0.000000,0.000000}%
\pgfsetfillcolor{currentfill}%
\pgfsetlinewidth{1.003750pt}%
\definecolor{currentstroke}{rgb}{0.000000,0.000000,0.000000}%
\pgfsetstrokecolor{currentstroke}%
\pgfsetdash{}{0pt}%
\pgfpathmoveto{\pgfqpoint{1.530548in}{1.438667in}}%
\pgfpathcurveto{\pgfqpoint{1.541598in}{1.438667in}}{\pgfqpoint{1.552197in}{1.443057in}}{\pgfqpoint{1.560011in}{1.450871in}}%
\pgfpathcurveto{\pgfqpoint{1.567825in}{1.458685in}}{\pgfqpoint{1.572215in}{1.469284in}}{\pgfqpoint{1.572215in}{1.480334in}}%
\pgfpathcurveto{\pgfqpoint{1.572215in}{1.491384in}}{\pgfqpoint{1.567825in}{1.501983in}}{\pgfqpoint{1.560011in}{1.509796in}}%
\pgfpathcurveto{\pgfqpoint{1.552197in}{1.517610in}}{\pgfqpoint{1.541598in}{1.522000in}}{\pgfqpoint{1.530548in}{1.522000in}}%
\pgfpathcurveto{\pgfqpoint{1.519498in}{1.522000in}}{\pgfqpoint{1.508899in}{1.517610in}}{\pgfqpoint{1.501085in}{1.509796in}}%
\pgfpathcurveto{\pgfqpoint{1.493272in}{1.501983in}}{\pgfqpoint{1.488881in}{1.491384in}}{\pgfqpoint{1.488881in}{1.480334in}}%
\pgfpathcurveto{\pgfqpoint{1.488881in}{1.469284in}}{\pgfqpoint{1.493272in}{1.458685in}}{\pgfqpoint{1.501085in}{1.450871in}}%
\pgfpathcurveto{\pgfqpoint{1.508899in}{1.443057in}}{\pgfqpoint{1.519498in}{1.438667in}}{\pgfqpoint{1.530548in}{1.438667in}}%
\pgfpathclose%
\pgfusepath{stroke,fill}%
\end{pgfscope}%
\begin{pgfscope}%
\pgfpathrectangle{\pgfqpoint{0.375000in}{0.330000in}}{\pgfqpoint{2.325000in}{2.310000in}}%
\pgfusepath{clip}%
\pgfsetbuttcap%
\pgfsetroundjoin%
\definecolor{currentfill}{rgb}{0.000000,0.000000,0.000000}%
\pgfsetfillcolor{currentfill}%
\pgfsetlinewidth{1.003750pt}%
\definecolor{currentstroke}{rgb}{0.000000,0.000000,0.000000}%
\pgfsetstrokecolor{currentstroke}%
\pgfsetdash{}{0pt}%
\pgfpathmoveto{\pgfqpoint{1.530548in}{1.438667in}}%
\pgfpathcurveto{\pgfqpoint{1.541598in}{1.438667in}}{\pgfqpoint{1.552197in}{1.443057in}}{\pgfqpoint{1.560011in}{1.450871in}}%
\pgfpathcurveto{\pgfqpoint{1.567825in}{1.458685in}}{\pgfqpoint{1.572215in}{1.469284in}}{\pgfqpoint{1.572215in}{1.480334in}}%
\pgfpathcurveto{\pgfqpoint{1.572215in}{1.491384in}}{\pgfqpoint{1.567825in}{1.501983in}}{\pgfqpoint{1.560011in}{1.509796in}}%
\pgfpathcurveto{\pgfqpoint{1.552197in}{1.517610in}}{\pgfqpoint{1.541598in}{1.522000in}}{\pgfqpoint{1.530548in}{1.522000in}}%
\pgfpathcurveto{\pgfqpoint{1.519498in}{1.522000in}}{\pgfqpoint{1.508899in}{1.517610in}}{\pgfqpoint{1.501085in}{1.509796in}}%
\pgfpathcurveto{\pgfqpoint{1.493272in}{1.501983in}}{\pgfqpoint{1.488881in}{1.491384in}}{\pgfqpoint{1.488881in}{1.480334in}}%
\pgfpathcurveto{\pgfqpoint{1.488881in}{1.469284in}}{\pgfqpoint{1.493272in}{1.458685in}}{\pgfqpoint{1.501085in}{1.450871in}}%
\pgfpathcurveto{\pgfqpoint{1.508899in}{1.443057in}}{\pgfqpoint{1.519498in}{1.438667in}}{\pgfqpoint{1.530548in}{1.438667in}}%
\pgfpathclose%
\pgfusepath{stroke,fill}%
\end{pgfscope}%
\begin{pgfscope}%
\pgfpathrectangle{\pgfqpoint{0.375000in}{0.330000in}}{\pgfqpoint{2.325000in}{2.310000in}}%
\pgfusepath{clip}%
\pgfsetbuttcap%
\pgfsetroundjoin%
\definecolor{currentfill}{rgb}{0.000000,0.000000,0.000000}%
\pgfsetfillcolor{currentfill}%
\pgfsetlinewidth{1.003750pt}%
\definecolor{currentstroke}{rgb}{0.000000,0.000000,0.000000}%
\pgfsetstrokecolor{currentstroke}%
\pgfsetdash{}{0pt}%
\pgfpathmoveto{\pgfqpoint{1.530548in}{1.438667in}}%
\pgfpathcurveto{\pgfqpoint{1.541598in}{1.438667in}}{\pgfqpoint{1.552197in}{1.443057in}}{\pgfqpoint{1.560011in}{1.450871in}}%
\pgfpathcurveto{\pgfqpoint{1.567825in}{1.458685in}}{\pgfqpoint{1.572215in}{1.469284in}}{\pgfqpoint{1.572215in}{1.480334in}}%
\pgfpathcurveto{\pgfqpoint{1.572215in}{1.491384in}}{\pgfqpoint{1.567825in}{1.501983in}}{\pgfqpoint{1.560011in}{1.509796in}}%
\pgfpathcurveto{\pgfqpoint{1.552197in}{1.517610in}}{\pgfqpoint{1.541598in}{1.522000in}}{\pgfqpoint{1.530548in}{1.522000in}}%
\pgfpathcurveto{\pgfqpoint{1.519498in}{1.522000in}}{\pgfqpoint{1.508899in}{1.517610in}}{\pgfqpoint{1.501085in}{1.509796in}}%
\pgfpathcurveto{\pgfqpoint{1.493272in}{1.501983in}}{\pgfqpoint{1.488881in}{1.491384in}}{\pgfqpoint{1.488881in}{1.480334in}}%
\pgfpathcurveto{\pgfqpoint{1.488881in}{1.469284in}}{\pgfqpoint{1.493272in}{1.458685in}}{\pgfqpoint{1.501085in}{1.450871in}}%
\pgfpathcurveto{\pgfqpoint{1.508899in}{1.443057in}}{\pgfqpoint{1.519498in}{1.438667in}}{\pgfqpoint{1.530548in}{1.438667in}}%
\pgfpathclose%
\pgfusepath{stroke,fill}%
\end{pgfscope}%
\begin{pgfscope}%
\pgfpathrectangle{\pgfqpoint{0.375000in}{0.330000in}}{\pgfqpoint{2.325000in}{2.310000in}}%
\pgfusepath{clip}%
\pgfsetbuttcap%
\pgfsetroundjoin%
\definecolor{currentfill}{rgb}{0.000000,0.000000,0.000000}%
\pgfsetfillcolor{currentfill}%
\pgfsetlinewidth{1.003750pt}%
\definecolor{currentstroke}{rgb}{0.000000,0.000000,0.000000}%
\pgfsetstrokecolor{currentstroke}%
\pgfsetdash{}{0pt}%
\pgfpathmoveto{\pgfqpoint{1.530548in}{1.438667in}}%
\pgfpathcurveto{\pgfqpoint{1.541598in}{1.438667in}}{\pgfqpoint{1.552197in}{1.443057in}}{\pgfqpoint{1.560011in}{1.450871in}}%
\pgfpathcurveto{\pgfqpoint{1.567825in}{1.458685in}}{\pgfqpoint{1.572215in}{1.469284in}}{\pgfqpoint{1.572215in}{1.480334in}}%
\pgfpathcurveto{\pgfqpoint{1.572215in}{1.491384in}}{\pgfqpoint{1.567825in}{1.501983in}}{\pgfqpoint{1.560011in}{1.509796in}}%
\pgfpathcurveto{\pgfqpoint{1.552197in}{1.517610in}}{\pgfqpoint{1.541598in}{1.522000in}}{\pgfqpoint{1.530548in}{1.522000in}}%
\pgfpathcurveto{\pgfqpoint{1.519498in}{1.522000in}}{\pgfqpoint{1.508899in}{1.517610in}}{\pgfqpoint{1.501085in}{1.509796in}}%
\pgfpathcurveto{\pgfqpoint{1.493272in}{1.501983in}}{\pgfqpoint{1.488881in}{1.491384in}}{\pgfqpoint{1.488881in}{1.480334in}}%
\pgfpathcurveto{\pgfqpoint{1.488881in}{1.469284in}}{\pgfqpoint{1.493272in}{1.458685in}}{\pgfqpoint{1.501085in}{1.450871in}}%
\pgfpathcurveto{\pgfqpoint{1.508899in}{1.443057in}}{\pgfqpoint{1.519498in}{1.438667in}}{\pgfqpoint{1.530548in}{1.438667in}}%
\pgfpathclose%
\pgfusepath{stroke,fill}%
\end{pgfscope}%
\begin{pgfscope}%
\pgfpathrectangle{\pgfqpoint{0.375000in}{0.330000in}}{\pgfqpoint{2.325000in}{2.310000in}}%
\pgfusepath{clip}%
\pgfsetbuttcap%
\pgfsetroundjoin%
\definecolor{currentfill}{rgb}{0.000000,0.000000,0.000000}%
\pgfsetfillcolor{currentfill}%
\pgfsetlinewidth{1.003750pt}%
\definecolor{currentstroke}{rgb}{0.000000,0.000000,0.000000}%
\pgfsetstrokecolor{currentstroke}%
\pgfsetdash{}{0pt}%
\pgfpathmoveto{\pgfqpoint{1.530548in}{1.438667in}}%
\pgfpathcurveto{\pgfqpoint{1.541598in}{1.438667in}}{\pgfqpoint{1.552197in}{1.443057in}}{\pgfqpoint{1.560011in}{1.450871in}}%
\pgfpathcurveto{\pgfqpoint{1.567825in}{1.458685in}}{\pgfqpoint{1.572215in}{1.469284in}}{\pgfqpoint{1.572215in}{1.480334in}}%
\pgfpathcurveto{\pgfqpoint{1.572215in}{1.491384in}}{\pgfqpoint{1.567825in}{1.501983in}}{\pgfqpoint{1.560011in}{1.509796in}}%
\pgfpathcurveto{\pgfqpoint{1.552197in}{1.517610in}}{\pgfqpoint{1.541598in}{1.522000in}}{\pgfqpoint{1.530548in}{1.522000in}}%
\pgfpathcurveto{\pgfqpoint{1.519498in}{1.522000in}}{\pgfqpoint{1.508899in}{1.517610in}}{\pgfqpoint{1.501085in}{1.509796in}}%
\pgfpathcurveto{\pgfqpoint{1.493272in}{1.501983in}}{\pgfqpoint{1.488881in}{1.491384in}}{\pgfqpoint{1.488881in}{1.480334in}}%
\pgfpathcurveto{\pgfqpoint{1.488881in}{1.469284in}}{\pgfqpoint{1.493272in}{1.458685in}}{\pgfqpoint{1.501085in}{1.450871in}}%
\pgfpathcurveto{\pgfqpoint{1.508899in}{1.443057in}}{\pgfqpoint{1.519498in}{1.438667in}}{\pgfqpoint{1.530548in}{1.438667in}}%
\pgfpathclose%
\pgfusepath{stroke,fill}%
\end{pgfscope}%
\begin{pgfscope}%
\pgfpathrectangle{\pgfqpoint{0.375000in}{0.330000in}}{\pgfqpoint{2.325000in}{2.310000in}}%
\pgfusepath{clip}%
\pgfsetbuttcap%
\pgfsetroundjoin%
\definecolor{currentfill}{rgb}{0.000000,0.000000,0.000000}%
\pgfsetfillcolor{currentfill}%
\pgfsetlinewidth{1.003750pt}%
\definecolor{currentstroke}{rgb}{0.000000,0.000000,0.000000}%
\pgfsetstrokecolor{currentstroke}%
\pgfsetdash{}{0pt}%
\pgfpathmoveto{\pgfqpoint{1.530548in}{1.438667in}}%
\pgfpathcurveto{\pgfqpoint{1.541598in}{1.438667in}}{\pgfqpoint{1.552197in}{1.443057in}}{\pgfqpoint{1.560011in}{1.450871in}}%
\pgfpathcurveto{\pgfqpoint{1.567825in}{1.458685in}}{\pgfqpoint{1.572215in}{1.469284in}}{\pgfqpoint{1.572215in}{1.480334in}}%
\pgfpathcurveto{\pgfqpoint{1.572215in}{1.491384in}}{\pgfqpoint{1.567825in}{1.501983in}}{\pgfqpoint{1.560011in}{1.509796in}}%
\pgfpathcurveto{\pgfqpoint{1.552197in}{1.517610in}}{\pgfqpoint{1.541598in}{1.522000in}}{\pgfqpoint{1.530548in}{1.522000in}}%
\pgfpathcurveto{\pgfqpoint{1.519498in}{1.522000in}}{\pgfqpoint{1.508899in}{1.517610in}}{\pgfqpoint{1.501085in}{1.509796in}}%
\pgfpathcurveto{\pgfqpoint{1.493272in}{1.501983in}}{\pgfqpoint{1.488881in}{1.491384in}}{\pgfqpoint{1.488881in}{1.480334in}}%
\pgfpathcurveto{\pgfqpoint{1.488881in}{1.469284in}}{\pgfqpoint{1.493272in}{1.458685in}}{\pgfqpoint{1.501085in}{1.450871in}}%
\pgfpathcurveto{\pgfqpoint{1.508899in}{1.443057in}}{\pgfqpoint{1.519498in}{1.438667in}}{\pgfqpoint{1.530548in}{1.438667in}}%
\pgfpathclose%
\pgfusepath{stroke,fill}%
\end{pgfscope}%
\begin{pgfscope}%
\pgfpathrectangle{\pgfqpoint{0.375000in}{0.330000in}}{\pgfqpoint{2.325000in}{2.310000in}}%
\pgfusepath{clip}%
\pgfsetbuttcap%
\pgfsetroundjoin%
\definecolor{currentfill}{rgb}{0.000000,0.000000,0.000000}%
\pgfsetfillcolor{currentfill}%
\pgfsetlinewidth{1.003750pt}%
\definecolor{currentstroke}{rgb}{0.000000,0.000000,0.000000}%
\pgfsetstrokecolor{currentstroke}%
\pgfsetdash{}{0pt}%
\pgfpathmoveto{\pgfqpoint{1.530548in}{1.438667in}}%
\pgfpathcurveto{\pgfqpoint{1.541598in}{1.438667in}}{\pgfqpoint{1.552197in}{1.443057in}}{\pgfqpoint{1.560011in}{1.450871in}}%
\pgfpathcurveto{\pgfqpoint{1.567825in}{1.458685in}}{\pgfqpoint{1.572215in}{1.469284in}}{\pgfqpoint{1.572215in}{1.480334in}}%
\pgfpathcurveto{\pgfqpoint{1.572215in}{1.491384in}}{\pgfqpoint{1.567825in}{1.501983in}}{\pgfqpoint{1.560011in}{1.509796in}}%
\pgfpathcurveto{\pgfqpoint{1.552197in}{1.517610in}}{\pgfqpoint{1.541598in}{1.522000in}}{\pgfqpoint{1.530548in}{1.522000in}}%
\pgfpathcurveto{\pgfqpoint{1.519498in}{1.522000in}}{\pgfqpoint{1.508899in}{1.517610in}}{\pgfqpoint{1.501085in}{1.509796in}}%
\pgfpathcurveto{\pgfqpoint{1.493272in}{1.501983in}}{\pgfqpoint{1.488881in}{1.491384in}}{\pgfqpoint{1.488881in}{1.480334in}}%
\pgfpathcurveto{\pgfqpoint{1.488881in}{1.469284in}}{\pgfqpoint{1.493272in}{1.458685in}}{\pgfqpoint{1.501085in}{1.450871in}}%
\pgfpathcurveto{\pgfqpoint{1.508899in}{1.443057in}}{\pgfqpoint{1.519498in}{1.438667in}}{\pgfqpoint{1.530548in}{1.438667in}}%
\pgfpathclose%
\pgfusepath{stroke,fill}%
\end{pgfscope}%
\begin{pgfscope}%
\pgfpathrectangle{\pgfqpoint{0.375000in}{0.330000in}}{\pgfqpoint{2.325000in}{2.310000in}}%
\pgfusepath{clip}%
\pgfsetbuttcap%
\pgfsetroundjoin%
\definecolor{currentfill}{rgb}{0.000000,0.000000,0.000000}%
\pgfsetfillcolor{currentfill}%
\pgfsetlinewidth{1.003750pt}%
\definecolor{currentstroke}{rgb}{0.000000,0.000000,0.000000}%
\pgfsetstrokecolor{currentstroke}%
\pgfsetdash{}{0pt}%
\pgfpathmoveto{\pgfqpoint{1.530548in}{1.438667in}}%
\pgfpathcurveto{\pgfqpoint{1.541598in}{1.438667in}}{\pgfqpoint{1.552197in}{1.443057in}}{\pgfqpoint{1.560011in}{1.450871in}}%
\pgfpathcurveto{\pgfqpoint{1.567825in}{1.458685in}}{\pgfqpoint{1.572215in}{1.469284in}}{\pgfqpoint{1.572215in}{1.480334in}}%
\pgfpathcurveto{\pgfqpoint{1.572215in}{1.491384in}}{\pgfqpoint{1.567825in}{1.501983in}}{\pgfqpoint{1.560011in}{1.509796in}}%
\pgfpathcurveto{\pgfqpoint{1.552197in}{1.517610in}}{\pgfqpoint{1.541598in}{1.522000in}}{\pgfqpoint{1.530548in}{1.522000in}}%
\pgfpathcurveto{\pgfqpoint{1.519498in}{1.522000in}}{\pgfqpoint{1.508899in}{1.517610in}}{\pgfqpoint{1.501085in}{1.509796in}}%
\pgfpathcurveto{\pgfqpoint{1.493272in}{1.501983in}}{\pgfqpoint{1.488881in}{1.491384in}}{\pgfqpoint{1.488881in}{1.480334in}}%
\pgfpathcurveto{\pgfqpoint{1.488881in}{1.469284in}}{\pgfqpoint{1.493272in}{1.458685in}}{\pgfqpoint{1.501085in}{1.450871in}}%
\pgfpathcurveto{\pgfqpoint{1.508899in}{1.443057in}}{\pgfqpoint{1.519498in}{1.438667in}}{\pgfqpoint{1.530548in}{1.438667in}}%
\pgfpathclose%
\pgfusepath{stroke,fill}%
\end{pgfscope}%
\begin{pgfscope}%
\pgfpathrectangle{\pgfqpoint{0.375000in}{0.330000in}}{\pgfqpoint{2.325000in}{2.310000in}}%
\pgfusepath{clip}%
\pgfsetbuttcap%
\pgfsetroundjoin%
\definecolor{currentfill}{rgb}{0.000000,0.000000,0.000000}%
\pgfsetfillcolor{currentfill}%
\pgfsetlinewidth{1.003750pt}%
\definecolor{currentstroke}{rgb}{0.000000,0.000000,0.000000}%
\pgfsetstrokecolor{currentstroke}%
\pgfsetdash{}{0pt}%
\pgfpathmoveto{\pgfqpoint{1.530548in}{1.438667in}}%
\pgfpathcurveto{\pgfqpoint{1.541598in}{1.438667in}}{\pgfqpoint{1.552197in}{1.443057in}}{\pgfqpoint{1.560011in}{1.450871in}}%
\pgfpathcurveto{\pgfqpoint{1.567825in}{1.458685in}}{\pgfqpoint{1.572215in}{1.469284in}}{\pgfqpoint{1.572215in}{1.480334in}}%
\pgfpathcurveto{\pgfqpoint{1.572215in}{1.491384in}}{\pgfqpoint{1.567825in}{1.501983in}}{\pgfqpoint{1.560011in}{1.509796in}}%
\pgfpathcurveto{\pgfqpoint{1.552197in}{1.517610in}}{\pgfqpoint{1.541598in}{1.522000in}}{\pgfqpoint{1.530548in}{1.522000in}}%
\pgfpathcurveto{\pgfqpoint{1.519498in}{1.522000in}}{\pgfqpoint{1.508899in}{1.517610in}}{\pgfqpoint{1.501085in}{1.509796in}}%
\pgfpathcurveto{\pgfqpoint{1.493272in}{1.501983in}}{\pgfqpoint{1.488881in}{1.491384in}}{\pgfqpoint{1.488881in}{1.480334in}}%
\pgfpathcurveto{\pgfqpoint{1.488881in}{1.469284in}}{\pgfqpoint{1.493272in}{1.458685in}}{\pgfqpoint{1.501085in}{1.450871in}}%
\pgfpathcurveto{\pgfqpoint{1.508899in}{1.443057in}}{\pgfqpoint{1.519498in}{1.438667in}}{\pgfqpoint{1.530548in}{1.438667in}}%
\pgfpathclose%
\pgfusepath{stroke,fill}%
\end{pgfscope}%
\begin{pgfscope}%
\pgfpathrectangle{\pgfqpoint{0.375000in}{0.330000in}}{\pgfqpoint{2.325000in}{2.310000in}}%
\pgfusepath{clip}%
\pgfsetbuttcap%
\pgfsetroundjoin%
\definecolor{currentfill}{rgb}{0.000000,0.000000,0.000000}%
\pgfsetfillcolor{currentfill}%
\pgfsetlinewidth{1.003750pt}%
\definecolor{currentstroke}{rgb}{0.000000,0.000000,0.000000}%
\pgfsetstrokecolor{currentstroke}%
\pgfsetdash{}{0pt}%
\pgfpathmoveto{\pgfqpoint{1.530548in}{1.438667in}}%
\pgfpathcurveto{\pgfqpoint{1.541598in}{1.438667in}}{\pgfqpoint{1.552197in}{1.443057in}}{\pgfqpoint{1.560011in}{1.450871in}}%
\pgfpathcurveto{\pgfqpoint{1.567825in}{1.458685in}}{\pgfqpoint{1.572215in}{1.469284in}}{\pgfqpoint{1.572215in}{1.480334in}}%
\pgfpathcurveto{\pgfqpoint{1.572215in}{1.491384in}}{\pgfqpoint{1.567825in}{1.501983in}}{\pgfqpoint{1.560011in}{1.509796in}}%
\pgfpathcurveto{\pgfqpoint{1.552197in}{1.517610in}}{\pgfqpoint{1.541598in}{1.522000in}}{\pgfqpoint{1.530548in}{1.522000in}}%
\pgfpathcurveto{\pgfqpoint{1.519498in}{1.522000in}}{\pgfqpoint{1.508899in}{1.517610in}}{\pgfqpoint{1.501085in}{1.509796in}}%
\pgfpathcurveto{\pgfqpoint{1.493272in}{1.501983in}}{\pgfqpoint{1.488881in}{1.491384in}}{\pgfqpoint{1.488881in}{1.480334in}}%
\pgfpathcurveto{\pgfqpoint{1.488881in}{1.469284in}}{\pgfqpoint{1.493272in}{1.458685in}}{\pgfqpoint{1.501085in}{1.450871in}}%
\pgfpathcurveto{\pgfqpoint{1.508899in}{1.443057in}}{\pgfqpoint{1.519498in}{1.438667in}}{\pgfqpoint{1.530548in}{1.438667in}}%
\pgfpathclose%
\pgfusepath{stroke,fill}%
\end{pgfscope}%
\begin{pgfscope}%
\pgfpathrectangle{\pgfqpoint{0.375000in}{0.330000in}}{\pgfqpoint{2.325000in}{2.310000in}}%
\pgfusepath{clip}%
\pgfsetbuttcap%
\pgfsetroundjoin%
\definecolor{currentfill}{rgb}{0.000000,0.000000,0.000000}%
\pgfsetfillcolor{currentfill}%
\pgfsetlinewidth{1.003750pt}%
\definecolor{currentstroke}{rgb}{0.000000,0.000000,0.000000}%
\pgfsetstrokecolor{currentstroke}%
\pgfsetdash{}{0pt}%
\pgfpathmoveto{\pgfqpoint{1.530548in}{1.438667in}}%
\pgfpathcurveto{\pgfqpoint{1.541598in}{1.438667in}}{\pgfqpoint{1.552197in}{1.443057in}}{\pgfqpoint{1.560011in}{1.450871in}}%
\pgfpathcurveto{\pgfqpoint{1.567825in}{1.458685in}}{\pgfqpoint{1.572215in}{1.469284in}}{\pgfqpoint{1.572215in}{1.480334in}}%
\pgfpathcurveto{\pgfqpoint{1.572215in}{1.491384in}}{\pgfqpoint{1.567825in}{1.501983in}}{\pgfqpoint{1.560011in}{1.509796in}}%
\pgfpathcurveto{\pgfqpoint{1.552197in}{1.517610in}}{\pgfqpoint{1.541598in}{1.522000in}}{\pgfqpoint{1.530548in}{1.522000in}}%
\pgfpathcurveto{\pgfqpoint{1.519498in}{1.522000in}}{\pgfqpoint{1.508899in}{1.517610in}}{\pgfqpoint{1.501085in}{1.509796in}}%
\pgfpathcurveto{\pgfqpoint{1.493272in}{1.501983in}}{\pgfqpoint{1.488881in}{1.491384in}}{\pgfqpoint{1.488881in}{1.480334in}}%
\pgfpathcurveto{\pgfqpoint{1.488881in}{1.469284in}}{\pgfqpoint{1.493272in}{1.458685in}}{\pgfqpoint{1.501085in}{1.450871in}}%
\pgfpathcurveto{\pgfqpoint{1.508899in}{1.443057in}}{\pgfqpoint{1.519498in}{1.438667in}}{\pgfqpoint{1.530548in}{1.438667in}}%
\pgfpathclose%
\pgfusepath{stroke,fill}%
\end{pgfscope}%
\begin{pgfscope}%
\pgfpathrectangle{\pgfqpoint{0.375000in}{0.330000in}}{\pgfqpoint{2.325000in}{2.310000in}}%
\pgfusepath{clip}%
\pgfsetbuttcap%
\pgfsetroundjoin%
\definecolor{currentfill}{rgb}{0.000000,0.000000,0.000000}%
\pgfsetfillcolor{currentfill}%
\pgfsetlinewidth{1.003750pt}%
\definecolor{currentstroke}{rgb}{0.000000,0.000000,0.000000}%
\pgfsetstrokecolor{currentstroke}%
\pgfsetdash{}{0pt}%
\pgfpathmoveto{\pgfqpoint{1.530548in}{1.438667in}}%
\pgfpathcurveto{\pgfqpoint{1.541598in}{1.438667in}}{\pgfqpoint{1.552197in}{1.443057in}}{\pgfqpoint{1.560011in}{1.450871in}}%
\pgfpathcurveto{\pgfqpoint{1.567825in}{1.458685in}}{\pgfqpoint{1.572215in}{1.469284in}}{\pgfqpoint{1.572215in}{1.480334in}}%
\pgfpathcurveto{\pgfqpoint{1.572215in}{1.491384in}}{\pgfqpoint{1.567825in}{1.501983in}}{\pgfqpoint{1.560011in}{1.509796in}}%
\pgfpathcurveto{\pgfqpoint{1.552197in}{1.517610in}}{\pgfqpoint{1.541598in}{1.522000in}}{\pgfqpoint{1.530548in}{1.522000in}}%
\pgfpathcurveto{\pgfqpoint{1.519498in}{1.522000in}}{\pgfqpoint{1.508899in}{1.517610in}}{\pgfqpoint{1.501085in}{1.509796in}}%
\pgfpathcurveto{\pgfqpoint{1.493272in}{1.501983in}}{\pgfqpoint{1.488881in}{1.491384in}}{\pgfqpoint{1.488881in}{1.480334in}}%
\pgfpathcurveto{\pgfqpoint{1.488881in}{1.469284in}}{\pgfqpoint{1.493272in}{1.458685in}}{\pgfqpoint{1.501085in}{1.450871in}}%
\pgfpathcurveto{\pgfqpoint{1.508899in}{1.443057in}}{\pgfqpoint{1.519498in}{1.438667in}}{\pgfqpoint{1.530548in}{1.438667in}}%
\pgfpathclose%
\pgfusepath{stroke,fill}%
\end{pgfscope}%
\begin{pgfscope}%
\pgfpathrectangle{\pgfqpoint{0.375000in}{0.330000in}}{\pgfqpoint{2.325000in}{2.310000in}}%
\pgfusepath{clip}%
\pgfsetbuttcap%
\pgfsetroundjoin%
\definecolor{currentfill}{rgb}{0.000000,0.000000,0.000000}%
\pgfsetfillcolor{currentfill}%
\pgfsetlinewidth{1.003750pt}%
\definecolor{currentstroke}{rgb}{0.000000,0.000000,0.000000}%
\pgfsetstrokecolor{currentstroke}%
\pgfsetdash{}{0pt}%
\pgfpathmoveto{\pgfqpoint{1.530548in}{1.438667in}}%
\pgfpathcurveto{\pgfqpoint{1.541598in}{1.438667in}}{\pgfqpoint{1.552197in}{1.443057in}}{\pgfqpoint{1.560011in}{1.450871in}}%
\pgfpathcurveto{\pgfqpoint{1.567825in}{1.458685in}}{\pgfqpoint{1.572215in}{1.469284in}}{\pgfqpoint{1.572215in}{1.480334in}}%
\pgfpathcurveto{\pgfqpoint{1.572215in}{1.491384in}}{\pgfqpoint{1.567825in}{1.501983in}}{\pgfqpoint{1.560011in}{1.509796in}}%
\pgfpathcurveto{\pgfqpoint{1.552197in}{1.517610in}}{\pgfqpoint{1.541598in}{1.522000in}}{\pgfqpoint{1.530548in}{1.522000in}}%
\pgfpathcurveto{\pgfqpoint{1.519498in}{1.522000in}}{\pgfqpoint{1.508899in}{1.517610in}}{\pgfqpoint{1.501085in}{1.509796in}}%
\pgfpathcurveto{\pgfqpoint{1.493272in}{1.501983in}}{\pgfqpoint{1.488881in}{1.491384in}}{\pgfqpoint{1.488881in}{1.480334in}}%
\pgfpathcurveto{\pgfqpoint{1.488881in}{1.469284in}}{\pgfqpoint{1.493272in}{1.458685in}}{\pgfqpoint{1.501085in}{1.450871in}}%
\pgfpathcurveto{\pgfqpoint{1.508899in}{1.443057in}}{\pgfqpoint{1.519498in}{1.438667in}}{\pgfqpoint{1.530548in}{1.438667in}}%
\pgfpathclose%
\pgfusepath{stroke,fill}%
\end{pgfscope}%
\begin{pgfscope}%
\pgfpathrectangle{\pgfqpoint{0.375000in}{0.330000in}}{\pgfqpoint{2.325000in}{2.310000in}}%
\pgfusepath{clip}%
\pgfsetbuttcap%
\pgfsetroundjoin%
\definecolor{currentfill}{rgb}{0.000000,0.000000,0.000000}%
\pgfsetfillcolor{currentfill}%
\pgfsetlinewidth{1.003750pt}%
\definecolor{currentstroke}{rgb}{0.000000,0.000000,0.000000}%
\pgfsetstrokecolor{currentstroke}%
\pgfsetdash{}{0pt}%
\pgfpathmoveto{\pgfqpoint{1.530548in}{1.438667in}}%
\pgfpathcurveto{\pgfqpoint{1.541598in}{1.438667in}}{\pgfqpoint{1.552197in}{1.443057in}}{\pgfqpoint{1.560011in}{1.450871in}}%
\pgfpathcurveto{\pgfqpoint{1.567825in}{1.458685in}}{\pgfqpoint{1.572215in}{1.469284in}}{\pgfqpoint{1.572215in}{1.480334in}}%
\pgfpathcurveto{\pgfqpoint{1.572215in}{1.491384in}}{\pgfqpoint{1.567825in}{1.501983in}}{\pgfqpoint{1.560011in}{1.509796in}}%
\pgfpathcurveto{\pgfqpoint{1.552197in}{1.517610in}}{\pgfqpoint{1.541598in}{1.522000in}}{\pgfqpoint{1.530548in}{1.522000in}}%
\pgfpathcurveto{\pgfqpoint{1.519498in}{1.522000in}}{\pgfqpoint{1.508899in}{1.517610in}}{\pgfqpoint{1.501085in}{1.509796in}}%
\pgfpathcurveto{\pgfqpoint{1.493272in}{1.501983in}}{\pgfqpoint{1.488881in}{1.491384in}}{\pgfqpoint{1.488881in}{1.480334in}}%
\pgfpathcurveto{\pgfqpoint{1.488881in}{1.469284in}}{\pgfqpoint{1.493272in}{1.458685in}}{\pgfqpoint{1.501085in}{1.450871in}}%
\pgfpathcurveto{\pgfqpoint{1.508899in}{1.443057in}}{\pgfqpoint{1.519498in}{1.438667in}}{\pgfqpoint{1.530548in}{1.438667in}}%
\pgfpathclose%
\pgfusepath{stroke,fill}%
\end{pgfscope}%
\begin{pgfscope}%
\pgfpathrectangle{\pgfqpoint{0.375000in}{0.330000in}}{\pgfqpoint{2.325000in}{2.310000in}}%
\pgfusepath{clip}%
\pgfsetbuttcap%
\pgfsetroundjoin%
\definecolor{currentfill}{rgb}{0.000000,0.000000,0.000000}%
\pgfsetfillcolor{currentfill}%
\pgfsetlinewidth{1.003750pt}%
\definecolor{currentstroke}{rgb}{0.000000,0.000000,0.000000}%
\pgfsetstrokecolor{currentstroke}%
\pgfsetdash{}{0pt}%
\pgfpathmoveto{\pgfqpoint{1.530548in}{1.438667in}}%
\pgfpathcurveto{\pgfqpoint{1.541598in}{1.438667in}}{\pgfqpoint{1.552197in}{1.443057in}}{\pgfqpoint{1.560011in}{1.450871in}}%
\pgfpathcurveto{\pgfqpoint{1.567825in}{1.458685in}}{\pgfqpoint{1.572215in}{1.469284in}}{\pgfqpoint{1.572215in}{1.480334in}}%
\pgfpathcurveto{\pgfqpoint{1.572215in}{1.491384in}}{\pgfqpoint{1.567825in}{1.501983in}}{\pgfqpoint{1.560011in}{1.509796in}}%
\pgfpathcurveto{\pgfqpoint{1.552197in}{1.517610in}}{\pgfqpoint{1.541598in}{1.522000in}}{\pgfqpoint{1.530548in}{1.522000in}}%
\pgfpathcurveto{\pgfqpoint{1.519498in}{1.522000in}}{\pgfqpoint{1.508899in}{1.517610in}}{\pgfqpoint{1.501085in}{1.509796in}}%
\pgfpathcurveto{\pgfqpoint{1.493272in}{1.501983in}}{\pgfqpoint{1.488881in}{1.491384in}}{\pgfqpoint{1.488881in}{1.480334in}}%
\pgfpathcurveto{\pgfqpoint{1.488881in}{1.469284in}}{\pgfqpoint{1.493272in}{1.458685in}}{\pgfqpoint{1.501085in}{1.450871in}}%
\pgfpathcurveto{\pgfqpoint{1.508899in}{1.443057in}}{\pgfqpoint{1.519498in}{1.438667in}}{\pgfqpoint{1.530548in}{1.438667in}}%
\pgfpathclose%
\pgfusepath{stroke,fill}%
\end{pgfscope}%
\begin{pgfscope}%
\pgfpathrectangle{\pgfqpoint{0.375000in}{0.330000in}}{\pgfqpoint{2.325000in}{2.310000in}}%
\pgfusepath{clip}%
\pgfsetbuttcap%
\pgfsetroundjoin%
\definecolor{currentfill}{rgb}{0.000000,0.000000,0.000000}%
\pgfsetfillcolor{currentfill}%
\pgfsetlinewidth{1.003750pt}%
\definecolor{currentstroke}{rgb}{0.000000,0.000000,0.000000}%
\pgfsetstrokecolor{currentstroke}%
\pgfsetdash{}{0pt}%
\pgfpathmoveto{\pgfqpoint{1.530548in}{1.438667in}}%
\pgfpathcurveto{\pgfqpoint{1.541598in}{1.438667in}}{\pgfqpoint{1.552197in}{1.443057in}}{\pgfqpoint{1.560011in}{1.450871in}}%
\pgfpathcurveto{\pgfqpoint{1.567825in}{1.458685in}}{\pgfqpoint{1.572215in}{1.469284in}}{\pgfqpoint{1.572215in}{1.480334in}}%
\pgfpathcurveto{\pgfqpoint{1.572215in}{1.491384in}}{\pgfqpoint{1.567825in}{1.501983in}}{\pgfqpoint{1.560011in}{1.509796in}}%
\pgfpathcurveto{\pgfqpoint{1.552197in}{1.517610in}}{\pgfqpoint{1.541598in}{1.522000in}}{\pgfqpoint{1.530548in}{1.522000in}}%
\pgfpathcurveto{\pgfqpoint{1.519498in}{1.522000in}}{\pgfqpoint{1.508899in}{1.517610in}}{\pgfqpoint{1.501085in}{1.509796in}}%
\pgfpathcurveto{\pgfqpoint{1.493272in}{1.501983in}}{\pgfqpoint{1.488881in}{1.491384in}}{\pgfqpoint{1.488881in}{1.480334in}}%
\pgfpathcurveto{\pgfqpoint{1.488881in}{1.469284in}}{\pgfqpoint{1.493272in}{1.458685in}}{\pgfqpoint{1.501085in}{1.450871in}}%
\pgfpathcurveto{\pgfqpoint{1.508899in}{1.443057in}}{\pgfqpoint{1.519498in}{1.438667in}}{\pgfqpoint{1.530548in}{1.438667in}}%
\pgfpathclose%
\pgfusepath{stroke,fill}%
\end{pgfscope}%
\begin{pgfscope}%
\pgfpathrectangle{\pgfqpoint{0.375000in}{0.330000in}}{\pgfqpoint{2.325000in}{2.310000in}}%
\pgfusepath{clip}%
\pgfsetbuttcap%
\pgfsetroundjoin%
\definecolor{currentfill}{rgb}{0.000000,0.000000,0.000000}%
\pgfsetfillcolor{currentfill}%
\pgfsetlinewidth{1.003750pt}%
\definecolor{currentstroke}{rgb}{0.000000,0.000000,0.000000}%
\pgfsetstrokecolor{currentstroke}%
\pgfsetdash{}{0pt}%
\pgfpathmoveto{\pgfqpoint{1.530548in}{1.438667in}}%
\pgfpathcurveto{\pgfqpoint{1.541598in}{1.438667in}}{\pgfqpoint{1.552197in}{1.443057in}}{\pgfqpoint{1.560011in}{1.450871in}}%
\pgfpathcurveto{\pgfqpoint{1.567825in}{1.458685in}}{\pgfqpoint{1.572215in}{1.469284in}}{\pgfqpoint{1.572215in}{1.480334in}}%
\pgfpathcurveto{\pgfqpoint{1.572215in}{1.491384in}}{\pgfqpoint{1.567825in}{1.501983in}}{\pgfqpoint{1.560011in}{1.509796in}}%
\pgfpathcurveto{\pgfqpoint{1.552197in}{1.517610in}}{\pgfqpoint{1.541598in}{1.522000in}}{\pgfqpoint{1.530548in}{1.522000in}}%
\pgfpathcurveto{\pgfqpoint{1.519498in}{1.522000in}}{\pgfqpoint{1.508899in}{1.517610in}}{\pgfqpoint{1.501085in}{1.509796in}}%
\pgfpathcurveto{\pgfqpoint{1.493272in}{1.501983in}}{\pgfqpoint{1.488881in}{1.491384in}}{\pgfqpoint{1.488881in}{1.480334in}}%
\pgfpathcurveto{\pgfqpoint{1.488881in}{1.469284in}}{\pgfqpoint{1.493272in}{1.458685in}}{\pgfqpoint{1.501085in}{1.450871in}}%
\pgfpathcurveto{\pgfqpoint{1.508899in}{1.443057in}}{\pgfqpoint{1.519498in}{1.438667in}}{\pgfqpoint{1.530548in}{1.438667in}}%
\pgfpathclose%
\pgfusepath{stroke,fill}%
\end{pgfscope}%
\begin{pgfscope}%
\pgfpathrectangle{\pgfqpoint{0.375000in}{0.330000in}}{\pgfqpoint{2.325000in}{2.310000in}}%
\pgfusepath{clip}%
\pgfsetbuttcap%
\pgfsetroundjoin%
\definecolor{currentfill}{rgb}{0.000000,0.000000,0.000000}%
\pgfsetfillcolor{currentfill}%
\pgfsetlinewidth{1.003750pt}%
\definecolor{currentstroke}{rgb}{0.000000,0.000000,0.000000}%
\pgfsetstrokecolor{currentstroke}%
\pgfsetdash{}{0pt}%
\pgfpathmoveto{\pgfqpoint{1.530548in}{1.438667in}}%
\pgfpathcurveto{\pgfqpoint{1.541598in}{1.438667in}}{\pgfqpoint{1.552197in}{1.443057in}}{\pgfqpoint{1.560011in}{1.450871in}}%
\pgfpathcurveto{\pgfqpoint{1.567825in}{1.458685in}}{\pgfqpoint{1.572215in}{1.469284in}}{\pgfqpoint{1.572215in}{1.480334in}}%
\pgfpathcurveto{\pgfqpoint{1.572215in}{1.491384in}}{\pgfqpoint{1.567825in}{1.501983in}}{\pgfqpoint{1.560011in}{1.509796in}}%
\pgfpathcurveto{\pgfqpoint{1.552197in}{1.517610in}}{\pgfqpoint{1.541598in}{1.522000in}}{\pgfqpoint{1.530548in}{1.522000in}}%
\pgfpathcurveto{\pgfqpoint{1.519498in}{1.522000in}}{\pgfqpoint{1.508899in}{1.517610in}}{\pgfqpoint{1.501085in}{1.509796in}}%
\pgfpathcurveto{\pgfqpoint{1.493272in}{1.501983in}}{\pgfqpoint{1.488881in}{1.491384in}}{\pgfqpoint{1.488881in}{1.480334in}}%
\pgfpathcurveto{\pgfqpoint{1.488881in}{1.469284in}}{\pgfqpoint{1.493272in}{1.458685in}}{\pgfqpoint{1.501085in}{1.450871in}}%
\pgfpathcurveto{\pgfqpoint{1.508899in}{1.443057in}}{\pgfqpoint{1.519498in}{1.438667in}}{\pgfqpoint{1.530548in}{1.438667in}}%
\pgfpathclose%
\pgfusepath{stroke,fill}%
\end{pgfscope}%
\begin{pgfscope}%
\pgfpathrectangle{\pgfqpoint{0.375000in}{0.330000in}}{\pgfqpoint{2.325000in}{2.310000in}}%
\pgfusepath{clip}%
\pgfsetbuttcap%
\pgfsetroundjoin%
\definecolor{currentfill}{rgb}{0.000000,0.000000,0.000000}%
\pgfsetfillcolor{currentfill}%
\pgfsetlinewidth{1.003750pt}%
\definecolor{currentstroke}{rgb}{0.000000,0.000000,0.000000}%
\pgfsetstrokecolor{currentstroke}%
\pgfsetdash{}{0pt}%
\pgfpathmoveto{\pgfqpoint{1.530548in}{1.438667in}}%
\pgfpathcurveto{\pgfqpoint{1.541598in}{1.438667in}}{\pgfqpoint{1.552197in}{1.443057in}}{\pgfqpoint{1.560011in}{1.450871in}}%
\pgfpathcurveto{\pgfqpoint{1.567825in}{1.458685in}}{\pgfqpoint{1.572215in}{1.469284in}}{\pgfqpoint{1.572215in}{1.480334in}}%
\pgfpathcurveto{\pgfqpoint{1.572215in}{1.491384in}}{\pgfqpoint{1.567825in}{1.501983in}}{\pgfqpoint{1.560011in}{1.509796in}}%
\pgfpathcurveto{\pgfqpoint{1.552197in}{1.517610in}}{\pgfqpoint{1.541598in}{1.522000in}}{\pgfqpoint{1.530548in}{1.522000in}}%
\pgfpathcurveto{\pgfqpoint{1.519498in}{1.522000in}}{\pgfqpoint{1.508899in}{1.517610in}}{\pgfqpoint{1.501085in}{1.509796in}}%
\pgfpathcurveto{\pgfqpoint{1.493272in}{1.501983in}}{\pgfqpoint{1.488881in}{1.491384in}}{\pgfqpoint{1.488881in}{1.480334in}}%
\pgfpathcurveto{\pgfqpoint{1.488881in}{1.469284in}}{\pgfqpoint{1.493272in}{1.458685in}}{\pgfqpoint{1.501085in}{1.450871in}}%
\pgfpathcurveto{\pgfqpoint{1.508899in}{1.443057in}}{\pgfqpoint{1.519498in}{1.438667in}}{\pgfqpoint{1.530548in}{1.438667in}}%
\pgfpathclose%
\pgfusepath{stroke,fill}%
\end{pgfscope}%
\begin{pgfscope}%
\pgfpathrectangle{\pgfqpoint{0.375000in}{0.330000in}}{\pgfqpoint{2.325000in}{2.310000in}}%
\pgfusepath{clip}%
\pgfsetbuttcap%
\pgfsetroundjoin%
\definecolor{currentfill}{rgb}{0.000000,0.000000,0.000000}%
\pgfsetfillcolor{currentfill}%
\pgfsetlinewidth{1.003750pt}%
\definecolor{currentstroke}{rgb}{0.000000,0.000000,0.000000}%
\pgfsetstrokecolor{currentstroke}%
\pgfsetdash{}{0pt}%
\pgfpathmoveto{\pgfqpoint{1.530548in}{1.438667in}}%
\pgfpathcurveto{\pgfqpoint{1.541598in}{1.438667in}}{\pgfqpoint{1.552197in}{1.443057in}}{\pgfqpoint{1.560011in}{1.450871in}}%
\pgfpathcurveto{\pgfqpoint{1.567825in}{1.458685in}}{\pgfqpoint{1.572215in}{1.469284in}}{\pgfqpoint{1.572215in}{1.480334in}}%
\pgfpathcurveto{\pgfqpoint{1.572215in}{1.491384in}}{\pgfqpoint{1.567825in}{1.501983in}}{\pgfqpoint{1.560011in}{1.509796in}}%
\pgfpathcurveto{\pgfqpoint{1.552197in}{1.517610in}}{\pgfqpoint{1.541598in}{1.522000in}}{\pgfqpoint{1.530548in}{1.522000in}}%
\pgfpathcurveto{\pgfqpoint{1.519498in}{1.522000in}}{\pgfqpoint{1.508899in}{1.517610in}}{\pgfqpoint{1.501085in}{1.509796in}}%
\pgfpathcurveto{\pgfqpoint{1.493272in}{1.501983in}}{\pgfqpoint{1.488881in}{1.491384in}}{\pgfqpoint{1.488881in}{1.480334in}}%
\pgfpathcurveto{\pgfqpoint{1.488881in}{1.469284in}}{\pgfqpoint{1.493272in}{1.458685in}}{\pgfqpoint{1.501085in}{1.450871in}}%
\pgfpathcurveto{\pgfqpoint{1.508899in}{1.443057in}}{\pgfqpoint{1.519498in}{1.438667in}}{\pgfqpoint{1.530548in}{1.438667in}}%
\pgfpathclose%
\pgfusepath{stroke,fill}%
\end{pgfscope}%
\begin{pgfscope}%
\pgfpathrectangle{\pgfqpoint{0.375000in}{0.330000in}}{\pgfqpoint{2.325000in}{2.310000in}}%
\pgfusepath{clip}%
\pgfsetbuttcap%
\pgfsetroundjoin%
\definecolor{currentfill}{rgb}{0.000000,0.000000,0.000000}%
\pgfsetfillcolor{currentfill}%
\pgfsetlinewidth{1.003750pt}%
\definecolor{currentstroke}{rgb}{0.000000,0.000000,0.000000}%
\pgfsetstrokecolor{currentstroke}%
\pgfsetdash{}{0pt}%
\pgfpathmoveto{\pgfqpoint{1.530548in}{1.438667in}}%
\pgfpathcurveto{\pgfqpoint{1.541598in}{1.438667in}}{\pgfqpoint{1.552197in}{1.443057in}}{\pgfqpoint{1.560011in}{1.450871in}}%
\pgfpathcurveto{\pgfqpoint{1.567825in}{1.458685in}}{\pgfqpoint{1.572215in}{1.469284in}}{\pgfqpoint{1.572215in}{1.480334in}}%
\pgfpathcurveto{\pgfqpoint{1.572215in}{1.491384in}}{\pgfqpoint{1.567825in}{1.501983in}}{\pgfqpoint{1.560011in}{1.509796in}}%
\pgfpathcurveto{\pgfqpoint{1.552197in}{1.517610in}}{\pgfqpoint{1.541598in}{1.522000in}}{\pgfqpoint{1.530548in}{1.522000in}}%
\pgfpathcurveto{\pgfqpoint{1.519498in}{1.522000in}}{\pgfqpoint{1.508899in}{1.517610in}}{\pgfqpoint{1.501085in}{1.509796in}}%
\pgfpathcurveto{\pgfqpoint{1.493272in}{1.501983in}}{\pgfqpoint{1.488881in}{1.491384in}}{\pgfqpoint{1.488881in}{1.480334in}}%
\pgfpathcurveto{\pgfqpoint{1.488881in}{1.469284in}}{\pgfqpoint{1.493272in}{1.458685in}}{\pgfqpoint{1.501085in}{1.450871in}}%
\pgfpathcurveto{\pgfqpoint{1.508899in}{1.443057in}}{\pgfqpoint{1.519498in}{1.438667in}}{\pgfqpoint{1.530548in}{1.438667in}}%
\pgfpathclose%
\pgfusepath{stroke,fill}%
\end{pgfscope}%
\begin{pgfscope}%
\pgfpathrectangle{\pgfqpoint{0.375000in}{0.330000in}}{\pgfqpoint{2.325000in}{2.310000in}}%
\pgfusepath{clip}%
\pgfsetbuttcap%
\pgfsetroundjoin%
\definecolor{currentfill}{rgb}{0.000000,0.000000,0.000000}%
\pgfsetfillcolor{currentfill}%
\pgfsetlinewidth{1.003750pt}%
\definecolor{currentstroke}{rgb}{0.000000,0.000000,0.000000}%
\pgfsetstrokecolor{currentstroke}%
\pgfsetdash{}{0pt}%
\pgfpathmoveto{\pgfqpoint{1.530548in}{1.438667in}}%
\pgfpathcurveto{\pgfqpoint{1.541598in}{1.438667in}}{\pgfqpoint{1.552197in}{1.443057in}}{\pgfqpoint{1.560011in}{1.450871in}}%
\pgfpathcurveto{\pgfqpoint{1.567825in}{1.458685in}}{\pgfqpoint{1.572215in}{1.469284in}}{\pgfqpoint{1.572215in}{1.480334in}}%
\pgfpathcurveto{\pgfqpoint{1.572215in}{1.491384in}}{\pgfqpoint{1.567825in}{1.501983in}}{\pgfqpoint{1.560011in}{1.509796in}}%
\pgfpathcurveto{\pgfqpoint{1.552197in}{1.517610in}}{\pgfqpoint{1.541598in}{1.522000in}}{\pgfqpoint{1.530548in}{1.522000in}}%
\pgfpathcurveto{\pgfqpoint{1.519498in}{1.522000in}}{\pgfqpoint{1.508899in}{1.517610in}}{\pgfqpoint{1.501085in}{1.509796in}}%
\pgfpathcurveto{\pgfqpoint{1.493272in}{1.501983in}}{\pgfqpoint{1.488881in}{1.491384in}}{\pgfqpoint{1.488881in}{1.480334in}}%
\pgfpathcurveto{\pgfqpoint{1.488881in}{1.469284in}}{\pgfqpoint{1.493272in}{1.458685in}}{\pgfqpoint{1.501085in}{1.450871in}}%
\pgfpathcurveto{\pgfqpoint{1.508899in}{1.443057in}}{\pgfqpoint{1.519498in}{1.438667in}}{\pgfqpoint{1.530548in}{1.438667in}}%
\pgfpathclose%
\pgfusepath{stroke,fill}%
\end{pgfscope}%
\begin{pgfscope}%
\pgfpathrectangle{\pgfqpoint{0.375000in}{0.330000in}}{\pgfqpoint{2.325000in}{2.310000in}}%
\pgfusepath{clip}%
\pgfsetbuttcap%
\pgfsetroundjoin%
\definecolor{currentfill}{rgb}{0.000000,0.000000,0.000000}%
\pgfsetfillcolor{currentfill}%
\pgfsetlinewidth{1.003750pt}%
\definecolor{currentstroke}{rgb}{0.000000,0.000000,0.000000}%
\pgfsetstrokecolor{currentstroke}%
\pgfsetdash{}{0pt}%
\pgfpathmoveto{\pgfqpoint{1.530548in}{1.438667in}}%
\pgfpathcurveto{\pgfqpoint{1.541598in}{1.438667in}}{\pgfqpoint{1.552197in}{1.443057in}}{\pgfqpoint{1.560011in}{1.450871in}}%
\pgfpathcurveto{\pgfqpoint{1.567825in}{1.458685in}}{\pgfqpoint{1.572215in}{1.469284in}}{\pgfqpoint{1.572215in}{1.480334in}}%
\pgfpathcurveto{\pgfqpoint{1.572215in}{1.491384in}}{\pgfqpoint{1.567825in}{1.501983in}}{\pgfqpoint{1.560011in}{1.509796in}}%
\pgfpathcurveto{\pgfqpoint{1.552197in}{1.517610in}}{\pgfqpoint{1.541598in}{1.522000in}}{\pgfqpoint{1.530548in}{1.522000in}}%
\pgfpathcurveto{\pgfqpoint{1.519498in}{1.522000in}}{\pgfqpoint{1.508899in}{1.517610in}}{\pgfqpoint{1.501085in}{1.509796in}}%
\pgfpathcurveto{\pgfqpoint{1.493272in}{1.501983in}}{\pgfqpoint{1.488881in}{1.491384in}}{\pgfqpoint{1.488881in}{1.480334in}}%
\pgfpathcurveto{\pgfqpoint{1.488881in}{1.469284in}}{\pgfqpoint{1.493272in}{1.458685in}}{\pgfqpoint{1.501085in}{1.450871in}}%
\pgfpathcurveto{\pgfqpoint{1.508899in}{1.443057in}}{\pgfqpoint{1.519498in}{1.438667in}}{\pgfqpoint{1.530548in}{1.438667in}}%
\pgfpathclose%
\pgfusepath{stroke,fill}%
\end{pgfscope}%
\begin{pgfscope}%
\pgfpathrectangle{\pgfqpoint{0.375000in}{0.330000in}}{\pgfqpoint{2.325000in}{2.310000in}}%
\pgfusepath{clip}%
\pgfsetbuttcap%
\pgfsetroundjoin%
\definecolor{currentfill}{rgb}{0.000000,0.000000,0.000000}%
\pgfsetfillcolor{currentfill}%
\pgfsetlinewidth{1.003750pt}%
\definecolor{currentstroke}{rgb}{0.000000,0.000000,0.000000}%
\pgfsetstrokecolor{currentstroke}%
\pgfsetdash{}{0pt}%
\pgfpathmoveto{\pgfqpoint{1.530548in}{1.438667in}}%
\pgfpathcurveto{\pgfqpoint{1.541598in}{1.438667in}}{\pgfqpoint{1.552197in}{1.443057in}}{\pgfqpoint{1.560011in}{1.450871in}}%
\pgfpathcurveto{\pgfqpoint{1.567825in}{1.458685in}}{\pgfqpoint{1.572215in}{1.469284in}}{\pgfqpoint{1.572215in}{1.480334in}}%
\pgfpathcurveto{\pgfqpoint{1.572215in}{1.491384in}}{\pgfqpoint{1.567825in}{1.501983in}}{\pgfqpoint{1.560011in}{1.509796in}}%
\pgfpathcurveto{\pgfqpoint{1.552197in}{1.517610in}}{\pgfqpoint{1.541598in}{1.522000in}}{\pgfqpoint{1.530548in}{1.522000in}}%
\pgfpathcurveto{\pgfqpoint{1.519498in}{1.522000in}}{\pgfqpoint{1.508899in}{1.517610in}}{\pgfqpoint{1.501085in}{1.509796in}}%
\pgfpathcurveto{\pgfqpoint{1.493272in}{1.501983in}}{\pgfqpoint{1.488881in}{1.491384in}}{\pgfqpoint{1.488881in}{1.480334in}}%
\pgfpathcurveto{\pgfqpoint{1.488881in}{1.469284in}}{\pgfqpoint{1.493272in}{1.458685in}}{\pgfqpoint{1.501085in}{1.450871in}}%
\pgfpathcurveto{\pgfqpoint{1.508899in}{1.443057in}}{\pgfqpoint{1.519498in}{1.438667in}}{\pgfqpoint{1.530548in}{1.438667in}}%
\pgfpathclose%
\pgfusepath{stroke,fill}%
\end{pgfscope}%
\begin{pgfscope}%
\pgfpathrectangle{\pgfqpoint{0.375000in}{0.330000in}}{\pgfqpoint{2.325000in}{2.310000in}}%
\pgfusepath{clip}%
\pgfsetbuttcap%
\pgfsetroundjoin%
\definecolor{currentfill}{rgb}{0.000000,0.000000,0.000000}%
\pgfsetfillcolor{currentfill}%
\pgfsetlinewidth{1.003750pt}%
\definecolor{currentstroke}{rgb}{0.000000,0.000000,0.000000}%
\pgfsetstrokecolor{currentstroke}%
\pgfsetdash{}{0pt}%
\pgfpathmoveto{\pgfqpoint{1.530548in}{1.438667in}}%
\pgfpathcurveto{\pgfqpoint{1.541598in}{1.438667in}}{\pgfqpoint{1.552197in}{1.443057in}}{\pgfqpoint{1.560011in}{1.450871in}}%
\pgfpathcurveto{\pgfqpoint{1.567825in}{1.458685in}}{\pgfqpoint{1.572215in}{1.469284in}}{\pgfqpoint{1.572215in}{1.480334in}}%
\pgfpathcurveto{\pgfqpoint{1.572215in}{1.491384in}}{\pgfqpoint{1.567825in}{1.501983in}}{\pgfqpoint{1.560011in}{1.509796in}}%
\pgfpathcurveto{\pgfqpoint{1.552197in}{1.517610in}}{\pgfqpoint{1.541598in}{1.522000in}}{\pgfqpoint{1.530548in}{1.522000in}}%
\pgfpathcurveto{\pgfqpoint{1.519498in}{1.522000in}}{\pgfqpoint{1.508899in}{1.517610in}}{\pgfqpoint{1.501085in}{1.509796in}}%
\pgfpathcurveto{\pgfqpoint{1.493272in}{1.501983in}}{\pgfqpoint{1.488881in}{1.491384in}}{\pgfqpoint{1.488881in}{1.480334in}}%
\pgfpathcurveto{\pgfqpoint{1.488881in}{1.469284in}}{\pgfqpoint{1.493272in}{1.458685in}}{\pgfqpoint{1.501085in}{1.450871in}}%
\pgfpathcurveto{\pgfqpoint{1.508899in}{1.443057in}}{\pgfqpoint{1.519498in}{1.438667in}}{\pgfqpoint{1.530548in}{1.438667in}}%
\pgfpathclose%
\pgfusepath{stroke,fill}%
\end{pgfscope}%
\begin{pgfscope}%
\pgfpathrectangle{\pgfqpoint{0.375000in}{0.330000in}}{\pgfqpoint{2.325000in}{2.310000in}}%
\pgfusepath{clip}%
\pgfsetbuttcap%
\pgfsetroundjoin%
\definecolor{currentfill}{rgb}{0.000000,0.000000,0.000000}%
\pgfsetfillcolor{currentfill}%
\pgfsetlinewidth{1.003750pt}%
\definecolor{currentstroke}{rgb}{0.000000,0.000000,0.000000}%
\pgfsetstrokecolor{currentstroke}%
\pgfsetdash{}{0pt}%
\pgfpathmoveto{\pgfqpoint{1.530548in}{1.438667in}}%
\pgfpathcurveto{\pgfqpoint{1.541598in}{1.438667in}}{\pgfqpoint{1.552197in}{1.443057in}}{\pgfqpoint{1.560011in}{1.450871in}}%
\pgfpathcurveto{\pgfqpoint{1.567825in}{1.458685in}}{\pgfqpoint{1.572215in}{1.469284in}}{\pgfqpoint{1.572215in}{1.480334in}}%
\pgfpathcurveto{\pgfqpoint{1.572215in}{1.491384in}}{\pgfqpoint{1.567825in}{1.501983in}}{\pgfqpoint{1.560011in}{1.509796in}}%
\pgfpathcurveto{\pgfqpoint{1.552197in}{1.517610in}}{\pgfqpoint{1.541598in}{1.522000in}}{\pgfqpoint{1.530548in}{1.522000in}}%
\pgfpathcurveto{\pgfqpoint{1.519498in}{1.522000in}}{\pgfqpoint{1.508899in}{1.517610in}}{\pgfqpoint{1.501085in}{1.509796in}}%
\pgfpathcurveto{\pgfqpoint{1.493272in}{1.501983in}}{\pgfqpoint{1.488881in}{1.491384in}}{\pgfqpoint{1.488881in}{1.480334in}}%
\pgfpathcurveto{\pgfqpoint{1.488881in}{1.469284in}}{\pgfqpoint{1.493272in}{1.458685in}}{\pgfqpoint{1.501085in}{1.450871in}}%
\pgfpathcurveto{\pgfqpoint{1.508899in}{1.443057in}}{\pgfqpoint{1.519498in}{1.438667in}}{\pgfqpoint{1.530548in}{1.438667in}}%
\pgfpathclose%
\pgfusepath{stroke,fill}%
\end{pgfscope}%
\begin{pgfscope}%
\pgfpathrectangle{\pgfqpoint{0.375000in}{0.330000in}}{\pgfqpoint{2.325000in}{2.310000in}}%
\pgfusepath{clip}%
\pgfsetbuttcap%
\pgfsetroundjoin%
\definecolor{currentfill}{rgb}{0.000000,0.000000,0.000000}%
\pgfsetfillcolor{currentfill}%
\pgfsetlinewidth{1.003750pt}%
\definecolor{currentstroke}{rgb}{0.000000,0.000000,0.000000}%
\pgfsetstrokecolor{currentstroke}%
\pgfsetdash{}{0pt}%
\pgfpathmoveto{\pgfqpoint{1.530548in}{1.438667in}}%
\pgfpathcurveto{\pgfqpoint{1.541598in}{1.438667in}}{\pgfqpoint{1.552197in}{1.443057in}}{\pgfqpoint{1.560011in}{1.450871in}}%
\pgfpathcurveto{\pgfqpoint{1.567825in}{1.458685in}}{\pgfqpoint{1.572215in}{1.469284in}}{\pgfqpoint{1.572215in}{1.480334in}}%
\pgfpathcurveto{\pgfqpoint{1.572215in}{1.491384in}}{\pgfqpoint{1.567825in}{1.501983in}}{\pgfqpoint{1.560011in}{1.509796in}}%
\pgfpathcurveto{\pgfqpoint{1.552197in}{1.517610in}}{\pgfqpoint{1.541598in}{1.522000in}}{\pgfqpoint{1.530548in}{1.522000in}}%
\pgfpathcurveto{\pgfqpoint{1.519498in}{1.522000in}}{\pgfqpoint{1.508899in}{1.517610in}}{\pgfqpoint{1.501085in}{1.509796in}}%
\pgfpathcurveto{\pgfqpoint{1.493272in}{1.501983in}}{\pgfqpoint{1.488881in}{1.491384in}}{\pgfqpoint{1.488881in}{1.480334in}}%
\pgfpathcurveto{\pgfqpoint{1.488881in}{1.469284in}}{\pgfqpoint{1.493272in}{1.458685in}}{\pgfqpoint{1.501085in}{1.450871in}}%
\pgfpathcurveto{\pgfqpoint{1.508899in}{1.443057in}}{\pgfqpoint{1.519498in}{1.438667in}}{\pgfqpoint{1.530548in}{1.438667in}}%
\pgfpathclose%
\pgfusepath{stroke,fill}%
\end{pgfscope}%
\begin{pgfscope}%
\pgfpathrectangle{\pgfqpoint{0.375000in}{0.330000in}}{\pgfqpoint{2.325000in}{2.310000in}}%
\pgfusepath{clip}%
\pgfsetbuttcap%
\pgfsetroundjoin%
\definecolor{currentfill}{rgb}{0.000000,0.000000,0.000000}%
\pgfsetfillcolor{currentfill}%
\pgfsetlinewidth{1.003750pt}%
\definecolor{currentstroke}{rgb}{0.000000,0.000000,0.000000}%
\pgfsetstrokecolor{currentstroke}%
\pgfsetdash{}{0pt}%
\pgfpathmoveto{\pgfqpoint{1.530548in}{1.438667in}}%
\pgfpathcurveto{\pgfqpoint{1.541598in}{1.438667in}}{\pgfqpoint{1.552197in}{1.443057in}}{\pgfqpoint{1.560011in}{1.450871in}}%
\pgfpathcurveto{\pgfqpoint{1.567825in}{1.458685in}}{\pgfqpoint{1.572215in}{1.469284in}}{\pgfqpoint{1.572215in}{1.480334in}}%
\pgfpathcurveto{\pgfqpoint{1.572215in}{1.491384in}}{\pgfqpoint{1.567825in}{1.501983in}}{\pgfqpoint{1.560011in}{1.509796in}}%
\pgfpathcurveto{\pgfqpoint{1.552197in}{1.517610in}}{\pgfqpoint{1.541598in}{1.522000in}}{\pgfqpoint{1.530548in}{1.522000in}}%
\pgfpathcurveto{\pgfqpoint{1.519498in}{1.522000in}}{\pgfqpoint{1.508899in}{1.517610in}}{\pgfqpoint{1.501085in}{1.509796in}}%
\pgfpathcurveto{\pgfqpoint{1.493272in}{1.501983in}}{\pgfqpoint{1.488881in}{1.491384in}}{\pgfqpoint{1.488881in}{1.480334in}}%
\pgfpathcurveto{\pgfqpoint{1.488881in}{1.469284in}}{\pgfqpoint{1.493272in}{1.458685in}}{\pgfqpoint{1.501085in}{1.450871in}}%
\pgfpathcurveto{\pgfqpoint{1.508899in}{1.443057in}}{\pgfqpoint{1.519498in}{1.438667in}}{\pgfqpoint{1.530548in}{1.438667in}}%
\pgfpathclose%
\pgfusepath{stroke,fill}%
\end{pgfscope}%
\begin{pgfscope}%
\pgfpathrectangle{\pgfqpoint{0.375000in}{0.330000in}}{\pgfqpoint{2.325000in}{2.310000in}}%
\pgfusepath{clip}%
\pgfsetbuttcap%
\pgfsetroundjoin%
\definecolor{currentfill}{rgb}{0.000000,0.000000,0.000000}%
\pgfsetfillcolor{currentfill}%
\pgfsetlinewidth{1.003750pt}%
\definecolor{currentstroke}{rgb}{0.000000,0.000000,0.000000}%
\pgfsetstrokecolor{currentstroke}%
\pgfsetdash{}{0pt}%
\pgfpathmoveto{\pgfqpoint{1.530548in}{1.438667in}}%
\pgfpathcurveto{\pgfqpoint{1.541598in}{1.438667in}}{\pgfqpoint{1.552197in}{1.443057in}}{\pgfqpoint{1.560011in}{1.450871in}}%
\pgfpathcurveto{\pgfqpoint{1.567825in}{1.458685in}}{\pgfqpoint{1.572215in}{1.469284in}}{\pgfqpoint{1.572215in}{1.480334in}}%
\pgfpathcurveto{\pgfqpoint{1.572215in}{1.491384in}}{\pgfqpoint{1.567825in}{1.501983in}}{\pgfqpoint{1.560011in}{1.509796in}}%
\pgfpathcurveto{\pgfqpoint{1.552197in}{1.517610in}}{\pgfqpoint{1.541598in}{1.522000in}}{\pgfqpoint{1.530548in}{1.522000in}}%
\pgfpathcurveto{\pgfqpoint{1.519498in}{1.522000in}}{\pgfqpoint{1.508899in}{1.517610in}}{\pgfqpoint{1.501085in}{1.509796in}}%
\pgfpathcurveto{\pgfqpoint{1.493272in}{1.501983in}}{\pgfqpoint{1.488881in}{1.491384in}}{\pgfqpoint{1.488881in}{1.480334in}}%
\pgfpathcurveto{\pgfqpoint{1.488881in}{1.469284in}}{\pgfqpoint{1.493272in}{1.458685in}}{\pgfqpoint{1.501085in}{1.450871in}}%
\pgfpathcurveto{\pgfqpoint{1.508899in}{1.443057in}}{\pgfqpoint{1.519498in}{1.438667in}}{\pgfqpoint{1.530548in}{1.438667in}}%
\pgfpathclose%
\pgfusepath{stroke,fill}%
\end{pgfscope}%
\begin{pgfscope}%
\pgfpathrectangle{\pgfqpoint{0.375000in}{0.330000in}}{\pgfqpoint{2.325000in}{2.310000in}}%
\pgfusepath{clip}%
\pgfsetbuttcap%
\pgfsetroundjoin%
\definecolor{currentfill}{rgb}{0.000000,0.000000,0.000000}%
\pgfsetfillcolor{currentfill}%
\pgfsetlinewidth{1.003750pt}%
\definecolor{currentstroke}{rgb}{0.000000,0.000000,0.000000}%
\pgfsetstrokecolor{currentstroke}%
\pgfsetdash{}{0pt}%
\pgfpathmoveto{\pgfqpoint{1.530548in}{1.438667in}}%
\pgfpathcurveto{\pgfqpoint{1.541598in}{1.438667in}}{\pgfqpoint{1.552197in}{1.443057in}}{\pgfqpoint{1.560011in}{1.450871in}}%
\pgfpathcurveto{\pgfqpoint{1.567825in}{1.458685in}}{\pgfqpoint{1.572215in}{1.469284in}}{\pgfqpoint{1.572215in}{1.480334in}}%
\pgfpathcurveto{\pgfqpoint{1.572215in}{1.491384in}}{\pgfqpoint{1.567825in}{1.501983in}}{\pgfqpoint{1.560011in}{1.509796in}}%
\pgfpathcurveto{\pgfqpoint{1.552197in}{1.517610in}}{\pgfqpoint{1.541598in}{1.522000in}}{\pgfqpoint{1.530548in}{1.522000in}}%
\pgfpathcurveto{\pgfqpoint{1.519498in}{1.522000in}}{\pgfqpoint{1.508899in}{1.517610in}}{\pgfqpoint{1.501085in}{1.509796in}}%
\pgfpathcurveto{\pgfqpoint{1.493272in}{1.501983in}}{\pgfqpoint{1.488881in}{1.491384in}}{\pgfqpoint{1.488881in}{1.480334in}}%
\pgfpathcurveto{\pgfqpoint{1.488881in}{1.469284in}}{\pgfqpoint{1.493272in}{1.458685in}}{\pgfqpoint{1.501085in}{1.450871in}}%
\pgfpathcurveto{\pgfqpoint{1.508899in}{1.443057in}}{\pgfqpoint{1.519498in}{1.438667in}}{\pgfqpoint{1.530548in}{1.438667in}}%
\pgfpathclose%
\pgfusepath{stroke,fill}%
\end{pgfscope}%
\begin{pgfscope}%
\pgfpathrectangle{\pgfqpoint{0.375000in}{0.330000in}}{\pgfqpoint{2.325000in}{2.310000in}}%
\pgfusepath{clip}%
\pgfsetbuttcap%
\pgfsetroundjoin%
\definecolor{currentfill}{rgb}{0.000000,0.000000,0.000000}%
\pgfsetfillcolor{currentfill}%
\pgfsetlinewidth{1.003750pt}%
\definecolor{currentstroke}{rgb}{0.000000,0.000000,0.000000}%
\pgfsetstrokecolor{currentstroke}%
\pgfsetdash{}{0pt}%
\pgfpathmoveto{\pgfqpoint{1.530548in}{1.438667in}}%
\pgfpathcurveto{\pgfqpoint{1.541598in}{1.438667in}}{\pgfqpoint{1.552197in}{1.443057in}}{\pgfqpoint{1.560011in}{1.450871in}}%
\pgfpathcurveto{\pgfqpoint{1.567825in}{1.458685in}}{\pgfqpoint{1.572215in}{1.469284in}}{\pgfqpoint{1.572215in}{1.480334in}}%
\pgfpathcurveto{\pgfqpoint{1.572215in}{1.491384in}}{\pgfqpoint{1.567825in}{1.501983in}}{\pgfqpoint{1.560011in}{1.509796in}}%
\pgfpathcurveto{\pgfqpoint{1.552197in}{1.517610in}}{\pgfqpoint{1.541598in}{1.522000in}}{\pgfqpoint{1.530548in}{1.522000in}}%
\pgfpathcurveto{\pgfqpoint{1.519498in}{1.522000in}}{\pgfqpoint{1.508899in}{1.517610in}}{\pgfqpoint{1.501085in}{1.509796in}}%
\pgfpathcurveto{\pgfqpoint{1.493272in}{1.501983in}}{\pgfqpoint{1.488881in}{1.491384in}}{\pgfqpoint{1.488881in}{1.480334in}}%
\pgfpathcurveto{\pgfqpoint{1.488881in}{1.469284in}}{\pgfqpoint{1.493272in}{1.458685in}}{\pgfqpoint{1.501085in}{1.450871in}}%
\pgfpathcurveto{\pgfqpoint{1.508899in}{1.443057in}}{\pgfqpoint{1.519498in}{1.438667in}}{\pgfqpoint{1.530548in}{1.438667in}}%
\pgfpathclose%
\pgfusepath{stroke,fill}%
\end{pgfscope}%
\begin{pgfscope}%
\pgfpathrectangle{\pgfqpoint{0.375000in}{0.330000in}}{\pgfqpoint{2.325000in}{2.310000in}}%
\pgfusepath{clip}%
\pgfsetbuttcap%
\pgfsetroundjoin%
\definecolor{currentfill}{rgb}{0.000000,0.000000,0.000000}%
\pgfsetfillcolor{currentfill}%
\pgfsetlinewidth{1.003750pt}%
\definecolor{currentstroke}{rgb}{0.000000,0.000000,0.000000}%
\pgfsetstrokecolor{currentstroke}%
\pgfsetdash{}{0pt}%
\pgfpathmoveto{\pgfqpoint{1.530548in}{1.438667in}}%
\pgfpathcurveto{\pgfqpoint{1.541598in}{1.438667in}}{\pgfqpoint{1.552197in}{1.443057in}}{\pgfqpoint{1.560011in}{1.450871in}}%
\pgfpathcurveto{\pgfqpoint{1.567825in}{1.458685in}}{\pgfqpoint{1.572215in}{1.469284in}}{\pgfqpoint{1.572215in}{1.480334in}}%
\pgfpathcurveto{\pgfqpoint{1.572215in}{1.491384in}}{\pgfqpoint{1.567825in}{1.501983in}}{\pgfqpoint{1.560011in}{1.509796in}}%
\pgfpathcurveto{\pgfqpoint{1.552197in}{1.517610in}}{\pgfqpoint{1.541598in}{1.522000in}}{\pgfqpoint{1.530548in}{1.522000in}}%
\pgfpathcurveto{\pgfqpoint{1.519498in}{1.522000in}}{\pgfqpoint{1.508899in}{1.517610in}}{\pgfqpoint{1.501085in}{1.509796in}}%
\pgfpathcurveto{\pgfqpoint{1.493272in}{1.501983in}}{\pgfqpoint{1.488881in}{1.491384in}}{\pgfqpoint{1.488881in}{1.480334in}}%
\pgfpathcurveto{\pgfqpoint{1.488881in}{1.469284in}}{\pgfqpoint{1.493272in}{1.458685in}}{\pgfqpoint{1.501085in}{1.450871in}}%
\pgfpathcurveto{\pgfqpoint{1.508899in}{1.443057in}}{\pgfqpoint{1.519498in}{1.438667in}}{\pgfqpoint{1.530548in}{1.438667in}}%
\pgfpathclose%
\pgfusepath{stroke,fill}%
\end{pgfscope}%
\begin{pgfscope}%
\pgfpathrectangle{\pgfqpoint{0.375000in}{0.330000in}}{\pgfqpoint{2.325000in}{2.310000in}}%
\pgfusepath{clip}%
\pgfsetbuttcap%
\pgfsetroundjoin%
\definecolor{currentfill}{rgb}{0.000000,0.000000,0.000000}%
\pgfsetfillcolor{currentfill}%
\pgfsetlinewidth{1.003750pt}%
\definecolor{currentstroke}{rgb}{0.000000,0.000000,0.000000}%
\pgfsetstrokecolor{currentstroke}%
\pgfsetdash{}{0pt}%
\pgfpathmoveto{\pgfqpoint{1.530548in}{1.438667in}}%
\pgfpathcurveto{\pgfqpoint{1.541598in}{1.438667in}}{\pgfqpoint{1.552197in}{1.443057in}}{\pgfqpoint{1.560011in}{1.450871in}}%
\pgfpathcurveto{\pgfqpoint{1.567825in}{1.458685in}}{\pgfqpoint{1.572215in}{1.469284in}}{\pgfqpoint{1.572215in}{1.480334in}}%
\pgfpathcurveto{\pgfqpoint{1.572215in}{1.491384in}}{\pgfqpoint{1.567825in}{1.501983in}}{\pgfqpoint{1.560011in}{1.509796in}}%
\pgfpathcurveto{\pgfqpoint{1.552197in}{1.517610in}}{\pgfqpoint{1.541598in}{1.522000in}}{\pgfqpoint{1.530548in}{1.522000in}}%
\pgfpathcurveto{\pgfqpoint{1.519498in}{1.522000in}}{\pgfqpoint{1.508899in}{1.517610in}}{\pgfqpoint{1.501085in}{1.509796in}}%
\pgfpathcurveto{\pgfqpoint{1.493272in}{1.501983in}}{\pgfqpoint{1.488881in}{1.491384in}}{\pgfqpoint{1.488881in}{1.480334in}}%
\pgfpathcurveto{\pgfqpoint{1.488881in}{1.469284in}}{\pgfqpoint{1.493272in}{1.458685in}}{\pgfqpoint{1.501085in}{1.450871in}}%
\pgfpathcurveto{\pgfqpoint{1.508899in}{1.443057in}}{\pgfqpoint{1.519498in}{1.438667in}}{\pgfqpoint{1.530548in}{1.438667in}}%
\pgfpathclose%
\pgfusepath{stroke,fill}%
\end{pgfscope}%
\begin{pgfscope}%
\pgfpathrectangle{\pgfqpoint{0.375000in}{0.330000in}}{\pgfqpoint{2.325000in}{2.310000in}}%
\pgfusepath{clip}%
\pgfsetbuttcap%
\pgfsetroundjoin%
\definecolor{currentfill}{rgb}{0.000000,0.000000,0.000000}%
\pgfsetfillcolor{currentfill}%
\pgfsetlinewidth{1.003750pt}%
\definecolor{currentstroke}{rgb}{0.000000,0.000000,0.000000}%
\pgfsetstrokecolor{currentstroke}%
\pgfsetdash{}{0pt}%
\pgfpathmoveto{\pgfqpoint{1.530548in}{1.438667in}}%
\pgfpathcurveto{\pgfqpoint{1.541598in}{1.438667in}}{\pgfqpoint{1.552197in}{1.443057in}}{\pgfqpoint{1.560011in}{1.450871in}}%
\pgfpathcurveto{\pgfqpoint{1.567825in}{1.458685in}}{\pgfqpoint{1.572215in}{1.469284in}}{\pgfqpoint{1.572215in}{1.480334in}}%
\pgfpathcurveto{\pgfqpoint{1.572215in}{1.491384in}}{\pgfqpoint{1.567825in}{1.501983in}}{\pgfqpoint{1.560011in}{1.509796in}}%
\pgfpathcurveto{\pgfqpoint{1.552197in}{1.517610in}}{\pgfqpoint{1.541598in}{1.522000in}}{\pgfqpoint{1.530548in}{1.522000in}}%
\pgfpathcurveto{\pgfqpoint{1.519498in}{1.522000in}}{\pgfqpoint{1.508899in}{1.517610in}}{\pgfqpoint{1.501085in}{1.509796in}}%
\pgfpathcurveto{\pgfqpoint{1.493272in}{1.501983in}}{\pgfqpoint{1.488881in}{1.491384in}}{\pgfqpoint{1.488881in}{1.480334in}}%
\pgfpathcurveto{\pgfqpoint{1.488881in}{1.469284in}}{\pgfqpoint{1.493272in}{1.458685in}}{\pgfqpoint{1.501085in}{1.450871in}}%
\pgfpathcurveto{\pgfqpoint{1.508899in}{1.443057in}}{\pgfqpoint{1.519498in}{1.438667in}}{\pgfqpoint{1.530548in}{1.438667in}}%
\pgfpathclose%
\pgfusepath{stroke,fill}%
\end{pgfscope}%
\begin{pgfscope}%
\pgfpathrectangle{\pgfqpoint{0.375000in}{0.330000in}}{\pgfqpoint{2.325000in}{2.310000in}}%
\pgfusepath{clip}%
\pgfsetbuttcap%
\pgfsetroundjoin%
\definecolor{currentfill}{rgb}{0.000000,0.000000,0.000000}%
\pgfsetfillcolor{currentfill}%
\pgfsetlinewidth{1.003750pt}%
\definecolor{currentstroke}{rgb}{0.000000,0.000000,0.000000}%
\pgfsetstrokecolor{currentstroke}%
\pgfsetdash{}{0pt}%
\pgfpathmoveto{\pgfqpoint{1.530548in}{1.438667in}}%
\pgfpathcurveto{\pgfqpoint{1.541598in}{1.438667in}}{\pgfqpoint{1.552197in}{1.443057in}}{\pgfqpoint{1.560011in}{1.450871in}}%
\pgfpathcurveto{\pgfqpoint{1.567825in}{1.458685in}}{\pgfqpoint{1.572215in}{1.469284in}}{\pgfqpoint{1.572215in}{1.480334in}}%
\pgfpathcurveto{\pgfqpoint{1.572215in}{1.491384in}}{\pgfqpoint{1.567825in}{1.501983in}}{\pgfqpoint{1.560011in}{1.509796in}}%
\pgfpathcurveto{\pgfqpoint{1.552197in}{1.517610in}}{\pgfqpoint{1.541598in}{1.522000in}}{\pgfqpoint{1.530548in}{1.522000in}}%
\pgfpathcurveto{\pgfqpoint{1.519498in}{1.522000in}}{\pgfqpoint{1.508899in}{1.517610in}}{\pgfqpoint{1.501085in}{1.509796in}}%
\pgfpathcurveto{\pgfqpoint{1.493272in}{1.501983in}}{\pgfqpoint{1.488881in}{1.491384in}}{\pgfqpoint{1.488881in}{1.480334in}}%
\pgfpathcurveto{\pgfqpoint{1.488881in}{1.469284in}}{\pgfqpoint{1.493272in}{1.458685in}}{\pgfqpoint{1.501085in}{1.450871in}}%
\pgfpathcurveto{\pgfqpoint{1.508899in}{1.443057in}}{\pgfqpoint{1.519498in}{1.438667in}}{\pgfqpoint{1.530548in}{1.438667in}}%
\pgfpathclose%
\pgfusepath{stroke,fill}%
\end{pgfscope}%
\begin{pgfscope}%
\pgfpathrectangle{\pgfqpoint{0.375000in}{0.330000in}}{\pgfqpoint{2.325000in}{2.310000in}}%
\pgfusepath{clip}%
\pgfsetbuttcap%
\pgfsetroundjoin%
\definecolor{currentfill}{rgb}{0.000000,0.000000,0.000000}%
\pgfsetfillcolor{currentfill}%
\pgfsetlinewidth{1.003750pt}%
\definecolor{currentstroke}{rgb}{0.000000,0.000000,0.000000}%
\pgfsetstrokecolor{currentstroke}%
\pgfsetdash{}{0pt}%
\pgfpathmoveto{\pgfqpoint{1.530548in}{1.438667in}}%
\pgfpathcurveto{\pgfqpoint{1.541598in}{1.438667in}}{\pgfqpoint{1.552197in}{1.443057in}}{\pgfqpoint{1.560011in}{1.450871in}}%
\pgfpathcurveto{\pgfqpoint{1.567825in}{1.458685in}}{\pgfqpoint{1.572215in}{1.469284in}}{\pgfqpoint{1.572215in}{1.480334in}}%
\pgfpathcurveto{\pgfqpoint{1.572215in}{1.491384in}}{\pgfqpoint{1.567825in}{1.501983in}}{\pgfqpoint{1.560011in}{1.509796in}}%
\pgfpathcurveto{\pgfqpoint{1.552197in}{1.517610in}}{\pgfqpoint{1.541598in}{1.522000in}}{\pgfqpoint{1.530548in}{1.522000in}}%
\pgfpathcurveto{\pgfqpoint{1.519498in}{1.522000in}}{\pgfqpoint{1.508899in}{1.517610in}}{\pgfqpoint{1.501085in}{1.509796in}}%
\pgfpathcurveto{\pgfqpoint{1.493272in}{1.501983in}}{\pgfqpoint{1.488881in}{1.491384in}}{\pgfqpoint{1.488881in}{1.480334in}}%
\pgfpathcurveto{\pgfqpoint{1.488881in}{1.469284in}}{\pgfqpoint{1.493272in}{1.458685in}}{\pgfqpoint{1.501085in}{1.450871in}}%
\pgfpathcurveto{\pgfqpoint{1.508899in}{1.443057in}}{\pgfqpoint{1.519498in}{1.438667in}}{\pgfqpoint{1.530548in}{1.438667in}}%
\pgfpathclose%
\pgfusepath{stroke,fill}%
\end{pgfscope}%
\begin{pgfscope}%
\pgfpathrectangle{\pgfqpoint{0.375000in}{0.330000in}}{\pgfqpoint{2.325000in}{2.310000in}}%
\pgfusepath{clip}%
\pgfsetbuttcap%
\pgfsetroundjoin%
\definecolor{currentfill}{rgb}{0.000000,0.000000,0.000000}%
\pgfsetfillcolor{currentfill}%
\pgfsetlinewidth{1.003750pt}%
\definecolor{currentstroke}{rgb}{0.000000,0.000000,0.000000}%
\pgfsetstrokecolor{currentstroke}%
\pgfsetdash{}{0pt}%
\pgfpathmoveto{\pgfqpoint{1.530548in}{1.438667in}}%
\pgfpathcurveto{\pgfqpoint{1.541598in}{1.438667in}}{\pgfqpoint{1.552197in}{1.443057in}}{\pgfqpoint{1.560011in}{1.450871in}}%
\pgfpathcurveto{\pgfqpoint{1.567825in}{1.458685in}}{\pgfqpoint{1.572215in}{1.469284in}}{\pgfqpoint{1.572215in}{1.480334in}}%
\pgfpathcurveto{\pgfqpoint{1.572215in}{1.491384in}}{\pgfqpoint{1.567825in}{1.501983in}}{\pgfqpoint{1.560011in}{1.509796in}}%
\pgfpathcurveto{\pgfqpoint{1.552197in}{1.517610in}}{\pgfqpoint{1.541598in}{1.522000in}}{\pgfqpoint{1.530548in}{1.522000in}}%
\pgfpathcurveto{\pgfqpoint{1.519498in}{1.522000in}}{\pgfqpoint{1.508899in}{1.517610in}}{\pgfqpoint{1.501085in}{1.509796in}}%
\pgfpathcurveto{\pgfqpoint{1.493272in}{1.501983in}}{\pgfqpoint{1.488881in}{1.491384in}}{\pgfqpoint{1.488881in}{1.480334in}}%
\pgfpathcurveto{\pgfqpoint{1.488881in}{1.469284in}}{\pgfqpoint{1.493272in}{1.458685in}}{\pgfqpoint{1.501085in}{1.450871in}}%
\pgfpathcurveto{\pgfqpoint{1.508899in}{1.443057in}}{\pgfqpoint{1.519498in}{1.438667in}}{\pgfqpoint{1.530548in}{1.438667in}}%
\pgfpathclose%
\pgfusepath{stroke,fill}%
\end{pgfscope}%
\begin{pgfscope}%
\pgfpathrectangle{\pgfqpoint{0.375000in}{0.330000in}}{\pgfqpoint{2.325000in}{2.310000in}}%
\pgfusepath{clip}%
\pgfsetbuttcap%
\pgfsetroundjoin%
\definecolor{currentfill}{rgb}{0.000000,0.000000,0.000000}%
\pgfsetfillcolor{currentfill}%
\pgfsetlinewidth{1.003750pt}%
\definecolor{currentstroke}{rgb}{0.000000,0.000000,0.000000}%
\pgfsetstrokecolor{currentstroke}%
\pgfsetdash{}{0pt}%
\pgfpathmoveto{\pgfqpoint{1.530548in}{1.438667in}}%
\pgfpathcurveto{\pgfqpoint{1.541598in}{1.438667in}}{\pgfqpoint{1.552197in}{1.443057in}}{\pgfqpoint{1.560011in}{1.450871in}}%
\pgfpathcurveto{\pgfqpoint{1.567825in}{1.458685in}}{\pgfqpoint{1.572215in}{1.469284in}}{\pgfqpoint{1.572215in}{1.480334in}}%
\pgfpathcurveto{\pgfqpoint{1.572215in}{1.491384in}}{\pgfqpoint{1.567825in}{1.501983in}}{\pgfqpoint{1.560011in}{1.509796in}}%
\pgfpathcurveto{\pgfqpoint{1.552197in}{1.517610in}}{\pgfqpoint{1.541598in}{1.522000in}}{\pgfqpoint{1.530548in}{1.522000in}}%
\pgfpathcurveto{\pgfqpoint{1.519498in}{1.522000in}}{\pgfqpoint{1.508899in}{1.517610in}}{\pgfqpoint{1.501085in}{1.509796in}}%
\pgfpathcurveto{\pgfqpoint{1.493272in}{1.501983in}}{\pgfqpoint{1.488881in}{1.491384in}}{\pgfqpoint{1.488881in}{1.480334in}}%
\pgfpathcurveto{\pgfqpoint{1.488881in}{1.469284in}}{\pgfqpoint{1.493272in}{1.458685in}}{\pgfqpoint{1.501085in}{1.450871in}}%
\pgfpathcurveto{\pgfqpoint{1.508899in}{1.443057in}}{\pgfqpoint{1.519498in}{1.438667in}}{\pgfqpoint{1.530548in}{1.438667in}}%
\pgfpathclose%
\pgfusepath{stroke,fill}%
\end{pgfscope}%
\begin{pgfscope}%
\pgfpathrectangle{\pgfqpoint{0.375000in}{0.330000in}}{\pgfqpoint{2.325000in}{2.310000in}}%
\pgfusepath{clip}%
\pgfsetbuttcap%
\pgfsetroundjoin%
\definecolor{currentfill}{rgb}{0.000000,0.000000,0.000000}%
\pgfsetfillcolor{currentfill}%
\pgfsetlinewidth{1.003750pt}%
\definecolor{currentstroke}{rgb}{0.000000,0.000000,0.000000}%
\pgfsetstrokecolor{currentstroke}%
\pgfsetdash{}{0pt}%
\pgfpathmoveto{\pgfqpoint{1.530548in}{1.438667in}}%
\pgfpathcurveto{\pgfqpoint{1.541598in}{1.438667in}}{\pgfqpoint{1.552197in}{1.443057in}}{\pgfqpoint{1.560011in}{1.450871in}}%
\pgfpathcurveto{\pgfqpoint{1.567825in}{1.458685in}}{\pgfqpoint{1.572215in}{1.469284in}}{\pgfqpoint{1.572215in}{1.480334in}}%
\pgfpathcurveto{\pgfqpoint{1.572215in}{1.491384in}}{\pgfqpoint{1.567825in}{1.501983in}}{\pgfqpoint{1.560011in}{1.509796in}}%
\pgfpathcurveto{\pgfqpoint{1.552197in}{1.517610in}}{\pgfqpoint{1.541598in}{1.522000in}}{\pgfqpoint{1.530548in}{1.522000in}}%
\pgfpathcurveto{\pgfqpoint{1.519498in}{1.522000in}}{\pgfqpoint{1.508899in}{1.517610in}}{\pgfqpoint{1.501085in}{1.509796in}}%
\pgfpathcurveto{\pgfqpoint{1.493272in}{1.501983in}}{\pgfqpoint{1.488881in}{1.491384in}}{\pgfqpoint{1.488881in}{1.480334in}}%
\pgfpathcurveto{\pgfqpoint{1.488881in}{1.469284in}}{\pgfqpoint{1.493272in}{1.458685in}}{\pgfqpoint{1.501085in}{1.450871in}}%
\pgfpathcurveto{\pgfqpoint{1.508899in}{1.443057in}}{\pgfqpoint{1.519498in}{1.438667in}}{\pgfqpoint{1.530548in}{1.438667in}}%
\pgfpathclose%
\pgfusepath{stroke,fill}%
\end{pgfscope}%
\begin{pgfscope}%
\pgfpathrectangle{\pgfqpoint{0.375000in}{0.330000in}}{\pgfqpoint{2.325000in}{2.310000in}}%
\pgfusepath{clip}%
\pgfsetbuttcap%
\pgfsetroundjoin%
\definecolor{currentfill}{rgb}{0.000000,0.000000,0.000000}%
\pgfsetfillcolor{currentfill}%
\pgfsetlinewidth{1.003750pt}%
\definecolor{currentstroke}{rgb}{0.000000,0.000000,0.000000}%
\pgfsetstrokecolor{currentstroke}%
\pgfsetdash{}{0pt}%
\pgfpathmoveto{\pgfqpoint{1.530548in}{1.438667in}}%
\pgfpathcurveto{\pgfqpoint{1.541598in}{1.438667in}}{\pgfqpoint{1.552197in}{1.443057in}}{\pgfqpoint{1.560011in}{1.450871in}}%
\pgfpathcurveto{\pgfqpoint{1.567825in}{1.458685in}}{\pgfqpoint{1.572215in}{1.469284in}}{\pgfqpoint{1.572215in}{1.480334in}}%
\pgfpathcurveto{\pgfqpoint{1.572215in}{1.491384in}}{\pgfqpoint{1.567825in}{1.501983in}}{\pgfqpoint{1.560011in}{1.509796in}}%
\pgfpathcurveto{\pgfqpoint{1.552197in}{1.517610in}}{\pgfqpoint{1.541598in}{1.522000in}}{\pgfqpoint{1.530548in}{1.522000in}}%
\pgfpathcurveto{\pgfqpoint{1.519498in}{1.522000in}}{\pgfqpoint{1.508899in}{1.517610in}}{\pgfqpoint{1.501085in}{1.509796in}}%
\pgfpathcurveto{\pgfqpoint{1.493272in}{1.501983in}}{\pgfqpoint{1.488881in}{1.491384in}}{\pgfqpoint{1.488881in}{1.480334in}}%
\pgfpathcurveto{\pgfqpoint{1.488881in}{1.469284in}}{\pgfqpoint{1.493272in}{1.458685in}}{\pgfqpoint{1.501085in}{1.450871in}}%
\pgfpathcurveto{\pgfqpoint{1.508899in}{1.443057in}}{\pgfqpoint{1.519498in}{1.438667in}}{\pgfqpoint{1.530548in}{1.438667in}}%
\pgfpathclose%
\pgfusepath{stroke,fill}%
\end{pgfscope}%
\begin{pgfscope}%
\pgfpathrectangle{\pgfqpoint{0.375000in}{0.330000in}}{\pgfqpoint{2.325000in}{2.310000in}}%
\pgfusepath{clip}%
\pgfsetbuttcap%
\pgfsetroundjoin%
\definecolor{currentfill}{rgb}{0.000000,0.000000,0.000000}%
\pgfsetfillcolor{currentfill}%
\pgfsetlinewidth{1.003750pt}%
\definecolor{currentstroke}{rgb}{0.000000,0.000000,0.000000}%
\pgfsetstrokecolor{currentstroke}%
\pgfsetdash{}{0pt}%
\pgfpathmoveto{\pgfqpoint{1.530548in}{1.438667in}}%
\pgfpathcurveto{\pgfqpoint{1.541598in}{1.438667in}}{\pgfqpoint{1.552197in}{1.443057in}}{\pgfqpoint{1.560011in}{1.450871in}}%
\pgfpathcurveto{\pgfqpoint{1.567825in}{1.458685in}}{\pgfqpoint{1.572215in}{1.469284in}}{\pgfqpoint{1.572215in}{1.480334in}}%
\pgfpathcurveto{\pgfqpoint{1.572215in}{1.491384in}}{\pgfqpoint{1.567825in}{1.501983in}}{\pgfqpoint{1.560011in}{1.509796in}}%
\pgfpathcurveto{\pgfqpoint{1.552197in}{1.517610in}}{\pgfqpoint{1.541598in}{1.522000in}}{\pgfqpoint{1.530548in}{1.522000in}}%
\pgfpathcurveto{\pgfqpoint{1.519498in}{1.522000in}}{\pgfqpoint{1.508899in}{1.517610in}}{\pgfqpoint{1.501085in}{1.509796in}}%
\pgfpathcurveto{\pgfqpoint{1.493272in}{1.501983in}}{\pgfqpoint{1.488881in}{1.491384in}}{\pgfqpoint{1.488881in}{1.480334in}}%
\pgfpathcurveto{\pgfqpoint{1.488881in}{1.469284in}}{\pgfqpoint{1.493272in}{1.458685in}}{\pgfqpoint{1.501085in}{1.450871in}}%
\pgfpathcurveto{\pgfqpoint{1.508899in}{1.443057in}}{\pgfqpoint{1.519498in}{1.438667in}}{\pgfqpoint{1.530548in}{1.438667in}}%
\pgfpathclose%
\pgfusepath{stroke,fill}%
\end{pgfscope}%
\begin{pgfscope}%
\pgfpathrectangle{\pgfqpoint{0.375000in}{0.330000in}}{\pgfqpoint{2.325000in}{2.310000in}}%
\pgfusepath{clip}%
\pgfsetbuttcap%
\pgfsetroundjoin%
\definecolor{currentfill}{rgb}{0.000000,0.000000,0.000000}%
\pgfsetfillcolor{currentfill}%
\pgfsetlinewidth{1.003750pt}%
\definecolor{currentstroke}{rgb}{0.000000,0.000000,0.000000}%
\pgfsetstrokecolor{currentstroke}%
\pgfsetdash{}{0pt}%
\pgfpathmoveto{\pgfqpoint{1.530548in}{1.438667in}}%
\pgfpathcurveto{\pgfqpoint{1.541598in}{1.438667in}}{\pgfqpoint{1.552197in}{1.443057in}}{\pgfqpoint{1.560011in}{1.450871in}}%
\pgfpathcurveto{\pgfqpoint{1.567825in}{1.458685in}}{\pgfqpoint{1.572215in}{1.469284in}}{\pgfqpoint{1.572215in}{1.480334in}}%
\pgfpathcurveto{\pgfqpoint{1.572215in}{1.491384in}}{\pgfqpoint{1.567825in}{1.501983in}}{\pgfqpoint{1.560011in}{1.509796in}}%
\pgfpathcurveto{\pgfqpoint{1.552197in}{1.517610in}}{\pgfqpoint{1.541598in}{1.522000in}}{\pgfqpoint{1.530548in}{1.522000in}}%
\pgfpathcurveto{\pgfqpoint{1.519498in}{1.522000in}}{\pgfqpoint{1.508899in}{1.517610in}}{\pgfqpoint{1.501085in}{1.509796in}}%
\pgfpathcurveto{\pgfqpoint{1.493272in}{1.501983in}}{\pgfqpoint{1.488881in}{1.491384in}}{\pgfqpoint{1.488881in}{1.480334in}}%
\pgfpathcurveto{\pgfqpoint{1.488881in}{1.469284in}}{\pgfqpoint{1.493272in}{1.458685in}}{\pgfqpoint{1.501085in}{1.450871in}}%
\pgfpathcurveto{\pgfqpoint{1.508899in}{1.443057in}}{\pgfqpoint{1.519498in}{1.438667in}}{\pgfqpoint{1.530548in}{1.438667in}}%
\pgfpathclose%
\pgfusepath{stroke,fill}%
\end{pgfscope}%
\begin{pgfscope}%
\pgfpathrectangle{\pgfqpoint{0.375000in}{0.330000in}}{\pgfqpoint{2.325000in}{2.310000in}}%
\pgfusepath{clip}%
\pgfsetbuttcap%
\pgfsetroundjoin%
\definecolor{currentfill}{rgb}{0.000000,0.000000,0.000000}%
\pgfsetfillcolor{currentfill}%
\pgfsetlinewidth{1.003750pt}%
\definecolor{currentstroke}{rgb}{0.000000,0.000000,0.000000}%
\pgfsetstrokecolor{currentstroke}%
\pgfsetdash{}{0pt}%
\pgfpathmoveto{\pgfqpoint{1.530548in}{1.438667in}}%
\pgfpathcurveto{\pgfqpoint{1.541598in}{1.438667in}}{\pgfqpoint{1.552197in}{1.443057in}}{\pgfqpoint{1.560011in}{1.450871in}}%
\pgfpathcurveto{\pgfqpoint{1.567825in}{1.458685in}}{\pgfqpoint{1.572215in}{1.469284in}}{\pgfqpoint{1.572215in}{1.480334in}}%
\pgfpathcurveto{\pgfqpoint{1.572215in}{1.491384in}}{\pgfqpoint{1.567825in}{1.501983in}}{\pgfqpoint{1.560011in}{1.509796in}}%
\pgfpathcurveto{\pgfqpoint{1.552197in}{1.517610in}}{\pgfqpoint{1.541598in}{1.522000in}}{\pgfqpoint{1.530548in}{1.522000in}}%
\pgfpathcurveto{\pgfqpoint{1.519498in}{1.522000in}}{\pgfqpoint{1.508899in}{1.517610in}}{\pgfqpoint{1.501085in}{1.509796in}}%
\pgfpathcurveto{\pgfqpoint{1.493272in}{1.501983in}}{\pgfqpoint{1.488881in}{1.491384in}}{\pgfqpoint{1.488881in}{1.480334in}}%
\pgfpathcurveto{\pgfqpoint{1.488881in}{1.469284in}}{\pgfqpoint{1.493272in}{1.458685in}}{\pgfqpoint{1.501085in}{1.450871in}}%
\pgfpathcurveto{\pgfqpoint{1.508899in}{1.443057in}}{\pgfqpoint{1.519498in}{1.438667in}}{\pgfqpoint{1.530548in}{1.438667in}}%
\pgfpathclose%
\pgfusepath{stroke,fill}%
\end{pgfscope}%
\begin{pgfscope}%
\pgfpathrectangle{\pgfqpoint{0.375000in}{0.330000in}}{\pgfqpoint{2.325000in}{2.310000in}}%
\pgfusepath{clip}%
\pgfsetbuttcap%
\pgfsetroundjoin%
\definecolor{currentfill}{rgb}{0.000000,0.000000,0.000000}%
\pgfsetfillcolor{currentfill}%
\pgfsetlinewidth{1.003750pt}%
\definecolor{currentstroke}{rgb}{0.000000,0.000000,0.000000}%
\pgfsetstrokecolor{currentstroke}%
\pgfsetdash{}{0pt}%
\pgfpathmoveto{\pgfqpoint{1.530548in}{1.438667in}}%
\pgfpathcurveto{\pgfqpoint{1.541598in}{1.438667in}}{\pgfqpoint{1.552197in}{1.443057in}}{\pgfqpoint{1.560011in}{1.450871in}}%
\pgfpathcurveto{\pgfqpoint{1.567825in}{1.458685in}}{\pgfqpoint{1.572215in}{1.469284in}}{\pgfqpoint{1.572215in}{1.480334in}}%
\pgfpathcurveto{\pgfqpoint{1.572215in}{1.491384in}}{\pgfqpoint{1.567825in}{1.501983in}}{\pgfqpoint{1.560011in}{1.509796in}}%
\pgfpathcurveto{\pgfqpoint{1.552197in}{1.517610in}}{\pgfqpoint{1.541598in}{1.522000in}}{\pgfqpoint{1.530548in}{1.522000in}}%
\pgfpathcurveto{\pgfqpoint{1.519498in}{1.522000in}}{\pgfqpoint{1.508899in}{1.517610in}}{\pgfqpoint{1.501085in}{1.509796in}}%
\pgfpathcurveto{\pgfqpoint{1.493272in}{1.501983in}}{\pgfqpoint{1.488881in}{1.491384in}}{\pgfqpoint{1.488881in}{1.480334in}}%
\pgfpathcurveto{\pgfqpoint{1.488881in}{1.469284in}}{\pgfqpoint{1.493272in}{1.458685in}}{\pgfqpoint{1.501085in}{1.450871in}}%
\pgfpathcurveto{\pgfqpoint{1.508899in}{1.443057in}}{\pgfqpoint{1.519498in}{1.438667in}}{\pgfqpoint{1.530548in}{1.438667in}}%
\pgfpathclose%
\pgfusepath{stroke,fill}%
\end{pgfscope}%
\begin{pgfscope}%
\pgfpathrectangle{\pgfqpoint{0.375000in}{0.330000in}}{\pgfqpoint{2.325000in}{2.310000in}}%
\pgfusepath{clip}%
\pgfsetbuttcap%
\pgfsetroundjoin%
\definecolor{currentfill}{rgb}{0.000000,0.000000,0.000000}%
\pgfsetfillcolor{currentfill}%
\pgfsetlinewidth{1.003750pt}%
\definecolor{currentstroke}{rgb}{0.000000,0.000000,0.000000}%
\pgfsetstrokecolor{currentstroke}%
\pgfsetdash{}{0pt}%
\pgfpathmoveto{\pgfqpoint{1.530548in}{1.438667in}}%
\pgfpathcurveto{\pgfqpoint{1.541598in}{1.438667in}}{\pgfqpoint{1.552197in}{1.443057in}}{\pgfqpoint{1.560011in}{1.450871in}}%
\pgfpathcurveto{\pgfqpoint{1.567825in}{1.458685in}}{\pgfqpoint{1.572215in}{1.469284in}}{\pgfqpoint{1.572215in}{1.480334in}}%
\pgfpathcurveto{\pgfqpoint{1.572215in}{1.491384in}}{\pgfqpoint{1.567825in}{1.501983in}}{\pgfqpoint{1.560011in}{1.509796in}}%
\pgfpathcurveto{\pgfqpoint{1.552197in}{1.517610in}}{\pgfqpoint{1.541598in}{1.522000in}}{\pgfqpoint{1.530548in}{1.522000in}}%
\pgfpathcurveto{\pgfqpoint{1.519498in}{1.522000in}}{\pgfqpoint{1.508899in}{1.517610in}}{\pgfqpoint{1.501085in}{1.509796in}}%
\pgfpathcurveto{\pgfqpoint{1.493272in}{1.501983in}}{\pgfqpoint{1.488881in}{1.491384in}}{\pgfqpoint{1.488881in}{1.480334in}}%
\pgfpathcurveto{\pgfqpoint{1.488881in}{1.469284in}}{\pgfqpoint{1.493272in}{1.458685in}}{\pgfqpoint{1.501085in}{1.450871in}}%
\pgfpathcurveto{\pgfqpoint{1.508899in}{1.443057in}}{\pgfqpoint{1.519498in}{1.438667in}}{\pgfqpoint{1.530548in}{1.438667in}}%
\pgfpathclose%
\pgfusepath{stroke,fill}%
\end{pgfscope}%
\begin{pgfscope}%
\pgfpathrectangle{\pgfqpoint{0.375000in}{0.330000in}}{\pgfqpoint{2.325000in}{2.310000in}}%
\pgfusepath{clip}%
\pgfsetbuttcap%
\pgfsetroundjoin%
\definecolor{currentfill}{rgb}{0.000000,0.000000,0.000000}%
\pgfsetfillcolor{currentfill}%
\pgfsetlinewidth{1.003750pt}%
\definecolor{currentstroke}{rgb}{0.000000,0.000000,0.000000}%
\pgfsetstrokecolor{currentstroke}%
\pgfsetdash{}{0pt}%
\pgfpathmoveto{\pgfqpoint{1.530548in}{1.438667in}}%
\pgfpathcurveto{\pgfqpoint{1.541598in}{1.438667in}}{\pgfqpoint{1.552197in}{1.443057in}}{\pgfqpoint{1.560011in}{1.450871in}}%
\pgfpathcurveto{\pgfqpoint{1.567825in}{1.458685in}}{\pgfqpoint{1.572215in}{1.469284in}}{\pgfqpoint{1.572215in}{1.480334in}}%
\pgfpathcurveto{\pgfqpoint{1.572215in}{1.491384in}}{\pgfqpoint{1.567825in}{1.501983in}}{\pgfqpoint{1.560011in}{1.509796in}}%
\pgfpathcurveto{\pgfqpoint{1.552197in}{1.517610in}}{\pgfqpoint{1.541598in}{1.522000in}}{\pgfqpoint{1.530548in}{1.522000in}}%
\pgfpathcurveto{\pgfqpoint{1.519498in}{1.522000in}}{\pgfqpoint{1.508899in}{1.517610in}}{\pgfqpoint{1.501085in}{1.509796in}}%
\pgfpathcurveto{\pgfqpoint{1.493272in}{1.501983in}}{\pgfqpoint{1.488881in}{1.491384in}}{\pgfqpoint{1.488881in}{1.480334in}}%
\pgfpathcurveto{\pgfqpoint{1.488881in}{1.469284in}}{\pgfqpoint{1.493272in}{1.458685in}}{\pgfqpoint{1.501085in}{1.450871in}}%
\pgfpathcurveto{\pgfqpoint{1.508899in}{1.443057in}}{\pgfqpoint{1.519498in}{1.438667in}}{\pgfqpoint{1.530548in}{1.438667in}}%
\pgfpathclose%
\pgfusepath{stroke,fill}%
\end{pgfscope}%
\begin{pgfscope}%
\pgfpathrectangle{\pgfqpoint{0.375000in}{0.330000in}}{\pgfqpoint{2.325000in}{2.310000in}}%
\pgfusepath{clip}%
\pgfsetbuttcap%
\pgfsetroundjoin%
\definecolor{currentfill}{rgb}{0.000000,0.000000,0.000000}%
\pgfsetfillcolor{currentfill}%
\pgfsetlinewidth{1.003750pt}%
\definecolor{currentstroke}{rgb}{0.000000,0.000000,0.000000}%
\pgfsetstrokecolor{currentstroke}%
\pgfsetdash{}{0pt}%
\pgfpathmoveto{\pgfqpoint{1.530548in}{1.438667in}}%
\pgfpathcurveto{\pgfqpoint{1.541598in}{1.438667in}}{\pgfqpoint{1.552197in}{1.443057in}}{\pgfqpoint{1.560011in}{1.450871in}}%
\pgfpathcurveto{\pgfqpoint{1.567825in}{1.458685in}}{\pgfqpoint{1.572215in}{1.469284in}}{\pgfqpoint{1.572215in}{1.480334in}}%
\pgfpathcurveto{\pgfqpoint{1.572215in}{1.491384in}}{\pgfqpoint{1.567825in}{1.501983in}}{\pgfqpoint{1.560011in}{1.509796in}}%
\pgfpathcurveto{\pgfqpoint{1.552197in}{1.517610in}}{\pgfqpoint{1.541598in}{1.522000in}}{\pgfqpoint{1.530548in}{1.522000in}}%
\pgfpathcurveto{\pgfqpoint{1.519498in}{1.522000in}}{\pgfqpoint{1.508899in}{1.517610in}}{\pgfqpoint{1.501085in}{1.509796in}}%
\pgfpathcurveto{\pgfqpoint{1.493272in}{1.501983in}}{\pgfqpoint{1.488881in}{1.491384in}}{\pgfqpoint{1.488881in}{1.480334in}}%
\pgfpathcurveto{\pgfqpoint{1.488881in}{1.469284in}}{\pgfqpoint{1.493272in}{1.458685in}}{\pgfqpoint{1.501085in}{1.450871in}}%
\pgfpathcurveto{\pgfqpoint{1.508899in}{1.443057in}}{\pgfqpoint{1.519498in}{1.438667in}}{\pgfqpoint{1.530548in}{1.438667in}}%
\pgfpathclose%
\pgfusepath{stroke,fill}%
\end{pgfscope}%
\begin{pgfscope}%
\pgfpathrectangle{\pgfqpoint{0.375000in}{0.330000in}}{\pgfqpoint{2.325000in}{2.310000in}}%
\pgfusepath{clip}%
\pgfsetbuttcap%
\pgfsetroundjoin%
\definecolor{currentfill}{rgb}{0.000000,0.000000,0.000000}%
\pgfsetfillcolor{currentfill}%
\pgfsetlinewidth{1.003750pt}%
\definecolor{currentstroke}{rgb}{0.000000,0.000000,0.000000}%
\pgfsetstrokecolor{currentstroke}%
\pgfsetdash{}{0pt}%
\pgfpathmoveto{\pgfqpoint{1.530548in}{1.438667in}}%
\pgfpathcurveto{\pgfqpoint{1.541598in}{1.438667in}}{\pgfqpoint{1.552197in}{1.443057in}}{\pgfqpoint{1.560011in}{1.450871in}}%
\pgfpathcurveto{\pgfqpoint{1.567825in}{1.458685in}}{\pgfqpoint{1.572215in}{1.469284in}}{\pgfqpoint{1.572215in}{1.480334in}}%
\pgfpathcurveto{\pgfqpoint{1.572215in}{1.491384in}}{\pgfqpoint{1.567825in}{1.501983in}}{\pgfqpoint{1.560011in}{1.509796in}}%
\pgfpathcurveto{\pgfqpoint{1.552197in}{1.517610in}}{\pgfqpoint{1.541598in}{1.522000in}}{\pgfqpoint{1.530548in}{1.522000in}}%
\pgfpathcurveto{\pgfqpoint{1.519498in}{1.522000in}}{\pgfqpoint{1.508899in}{1.517610in}}{\pgfqpoint{1.501085in}{1.509796in}}%
\pgfpathcurveto{\pgfqpoint{1.493272in}{1.501983in}}{\pgfqpoint{1.488881in}{1.491384in}}{\pgfqpoint{1.488881in}{1.480334in}}%
\pgfpathcurveto{\pgfqpoint{1.488881in}{1.469284in}}{\pgfqpoint{1.493272in}{1.458685in}}{\pgfqpoint{1.501085in}{1.450871in}}%
\pgfpathcurveto{\pgfqpoint{1.508899in}{1.443057in}}{\pgfqpoint{1.519498in}{1.438667in}}{\pgfqpoint{1.530548in}{1.438667in}}%
\pgfpathclose%
\pgfusepath{stroke,fill}%
\end{pgfscope}%
\begin{pgfscope}%
\pgfpathrectangle{\pgfqpoint{0.375000in}{0.330000in}}{\pgfqpoint{2.325000in}{2.310000in}}%
\pgfusepath{clip}%
\pgfsetbuttcap%
\pgfsetroundjoin%
\definecolor{currentfill}{rgb}{0.000000,0.000000,0.000000}%
\pgfsetfillcolor{currentfill}%
\pgfsetlinewidth{1.003750pt}%
\definecolor{currentstroke}{rgb}{0.000000,0.000000,0.000000}%
\pgfsetstrokecolor{currentstroke}%
\pgfsetdash{}{0pt}%
\pgfpathmoveto{\pgfqpoint{1.530548in}{1.438667in}}%
\pgfpathcurveto{\pgfqpoint{1.541598in}{1.438667in}}{\pgfqpoint{1.552197in}{1.443057in}}{\pgfqpoint{1.560011in}{1.450871in}}%
\pgfpathcurveto{\pgfqpoint{1.567825in}{1.458685in}}{\pgfqpoint{1.572215in}{1.469284in}}{\pgfqpoint{1.572215in}{1.480334in}}%
\pgfpathcurveto{\pgfqpoint{1.572215in}{1.491384in}}{\pgfqpoint{1.567825in}{1.501983in}}{\pgfqpoint{1.560011in}{1.509796in}}%
\pgfpathcurveto{\pgfqpoint{1.552197in}{1.517610in}}{\pgfqpoint{1.541598in}{1.522000in}}{\pgfqpoint{1.530548in}{1.522000in}}%
\pgfpathcurveto{\pgfqpoint{1.519498in}{1.522000in}}{\pgfqpoint{1.508899in}{1.517610in}}{\pgfqpoint{1.501085in}{1.509796in}}%
\pgfpathcurveto{\pgfqpoint{1.493272in}{1.501983in}}{\pgfqpoint{1.488881in}{1.491384in}}{\pgfqpoint{1.488881in}{1.480334in}}%
\pgfpathcurveto{\pgfqpoint{1.488881in}{1.469284in}}{\pgfqpoint{1.493272in}{1.458685in}}{\pgfqpoint{1.501085in}{1.450871in}}%
\pgfpathcurveto{\pgfqpoint{1.508899in}{1.443057in}}{\pgfqpoint{1.519498in}{1.438667in}}{\pgfqpoint{1.530548in}{1.438667in}}%
\pgfpathclose%
\pgfusepath{stroke,fill}%
\end{pgfscope}%
\begin{pgfscope}%
\pgfpathrectangle{\pgfqpoint{0.375000in}{0.330000in}}{\pgfqpoint{2.325000in}{2.310000in}}%
\pgfusepath{clip}%
\pgfsetbuttcap%
\pgfsetroundjoin%
\definecolor{currentfill}{rgb}{0.000000,0.000000,0.000000}%
\pgfsetfillcolor{currentfill}%
\pgfsetlinewidth{1.003750pt}%
\definecolor{currentstroke}{rgb}{0.000000,0.000000,0.000000}%
\pgfsetstrokecolor{currentstroke}%
\pgfsetdash{}{0pt}%
\pgfpathmoveto{\pgfqpoint{1.530548in}{1.438667in}}%
\pgfpathcurveto{\pgfqpoint{1.541598in}{1.438667in}}{\pgfqpoint{1.552197in}{1.443057in}}{\pgfqpoint{1.560011in}{1.450871in}}%
\pgfpathcurveto{\pgfqpoint{1.567825in}{1.458685in}}{\pgfqpoint{1.572215in}{1.469284in}}{\pgfqpoint{1.572215in}{1.480334in}}%
\pgfpathcurveto{\pgfqpoint{1.572215in}{1.491384in}}{\pgfqpoint{1.567825in}{1.501983in}}{\pgfqpoint{1.560011in}{1.509796in}}%
\pgfpathcurveto{\pgfqpoint{1.552197in}{1.517610in}}{\pgfqpoint{1.541598in}{1.522000in}}{\pgfqpoint{1.530548in}{1.522000in}}%
\pgfpathcurveto{\pgfqpoint{1.519498in}{1.522000in}}{\pgfqpoint{1.508899in}{1.517610in}}{\pgfqpoint{1.501085in}{1.509796in}}%
\pgfpathcurveto{\pgfqpoint{1.493272in}{1.501983in}}{\pgfqpoint{1.488881in}{1.491384in}}{\pgfqpoint{1.488881in}{1.480334in}}%
\pgfpathcurveto{\pgfqpoint{1.488881in}{1.469284in}}{\pgfqpoint{1.493272in}{1.458685in}}{\pgfqpoint{1.501085in}{1.450871in}}%
\pgfpathcurveto{\pgfqpoint{1.508899in}{1.443057in}}{\pgfqpoint{1.519498in}{1.438667in}}{\pgfqpoint{1.530548in}{1.438667in}}%
\pgfpathclose%
\pgfusepath{stroke,fill}%
\end{pgfscope}%
\begin{pgfscope}%
\pgfpathrectangle{\pgfqpoint{0.375000in}{0.330000in}}{\pgfqpoint{2.325000in}{2.310000in}}%
\pgfusepath{clip}%
\pgfsetbuttcap%
\pgfsetroundjoin%
\definecolor{currentfill}{rgb}{0.000000,0.000000,0.000000}%
\pgfsetfillcolor{currentfill}%
\pgfsetlinewidth{1.003750pt}%
\definecolor{currentstroke}{rgb}{0.000000,0.000000,0.000000}%
\pgfsetstrokecolor{currentstroke}%
\pgfsetdash{}{0pt}%
\pgfpathmoveto{\pgfqpoint{1.530548in}{1.438667in}}%
\pgfpathcurveto{\pgfqpoint{1.541598in}{1.438667in}}{\pgfqpoint{1.552197in}{1.443057in}}{\pgfqpoint{1.560011in}{1.450871in}}%
\pgfpathcurveto{\pgfqpoint{1.567825in}{1.458685in}}{\pgfqpoint{1.572215in}{1.469284in}}{\pgfqpoint{1.572215in}{1.480334in}}%
\pgfpathcurveto{\pgfqpoint{1.572215in}{1.491384in}}{\pgfqpoint{1.567825in}{1.501983in}}{\pgfqpoint{1.560011in}{1.509796in}}%
\pgfpathcurveto{\pgfqpoint{1.552197in}{1.517610in}}{\pgfqpoint{1.541598in}{1.522000in}}{\pgfqpoint{1.530548in}{1.522000in}}%
\pgfpathcurveto{\pgfqpoint{1.519498in}{1.522000in}}{\pgfqpoint{1.508899in}{1.517610in}}{\pgfqpoint{1.501085in}{1.509796in}}%
\pgfpathcurveto{\pgfqpoint{1.493272in}{1.501983in}}{\pgfqpoint{1.488881in}{1.491384in}}{\pgfqpoint{1.488881in}{1.480334in}}%
\pgfpathcurveto{\pgfqpoint{1.488881in}{1.469284in}}{\pgfqpoint{1.493272in}{1.458685in}}{\pgfqpoint{1.501085in}{1.450871in}}%
\pgfpathcurveto{\pgfqpoint{1.508899in}{1.443057in}}{\pgfqpoint{1.519498in}{1.438667in}}{\pgfqpoint{1.530548in}{1.438667in}}%
\pgfpathclose%
\pgfusepath{stroke,fill}%
\end{pgfscope}%
\begin{pgfscope}%
\pgfpathrectangle{\pgfqpoint{0.375000in}{0.330000in}}{\pgfqpoint{2.325000in}{2.310000in}}%
\pgfusepath{clip}%
\pgfsetbuttcap%
\pgfsetroundjoin%
\definecolor{currentfill}{rgb}{0.000000,0.000000,0.000000}%
\pgfsetfillcolor{currentfill}%
\pgfsetlinewidth{1.003750pt}%
\definecolor{currentstroke}{rgb}{0.000000,0.000000,0.000000}%
\pgfsetstrokecolor{currentstroke}%
\pgfsetdash{}{0pt}%
\pgfpathmoveto{\pgfqpoint{1.530548in}{1.438667in}}%
\pgfpathcurveto{\pgfqpoint{1.541598in}{1.438667in}}{\pgfqpoint{1.552197in}{1.443057in}}{\pgfqpoint{1.560011in}{1.450871in}}%
\pgfpathcurveto{\pgfqpoint{1.567825in}{1.458685in}}{\pgfqpoint{1.572215in}{1.469284in}}{\pgfqpoint{1.572215in}{1.480334in}}%
\pgfpathcurveto{\pgfqpoint{1.572215in}{1.491384in}}{\pgfqpoint{1.567825in}{1.501983in}}{\pgfqpoint{1.560011in}{1.509796in}}%
\pgfpathcurveto{\pgfqpoint{1.552197in}{1.517610in}}{\pgfqpoint{1.541598in}{1.522000in}}{\pgfqpoint{1.530548in}{1.522000in}}%
\pgfpathcurveto{\pgfqpoint{1.519498in}{1.522000in}}{\pgfqpoint{1.508899in}{1.517610in}}{\pgfqpoint{1.501085in}{1.509796in}}%
\pgfpathcurveto{\pgfqpoint{1.493272in}{1.501983in}}{\pgfqpoint{1.488881in}{1.491384in}}{\pgfqpoint{1.488881in}{1.480334in}}%
\pgfpathcurveto{\pgfqpoint{1.488881in}{1.469284in}}{\pgfqpoint{1.493272in}{1.458685in}}{\pgfqpoint{1.501085in}{1.450871in}}%
\pgfpathcurveto{\pgfqpoint{1.508899in}{1.443057in}}{\pgfqpoint{1.519498in}{1.438667in}}{\pgfqpoint{1.530548in}{1.438667in}}%
\pgfpathclose%
\pgfusepath{stroke,fill}%
\end{pgfscope}%
\begin{pgfscope}%
\pgfpathrectangle{\pgfqpoint{0.375000in}{0.330000in}}{\pgfqpoint{2.325000in}{2.310000in}}%
\pgfusepath{clip}%
\pgfsetbuttcap%
\pgfsetroundjoin%
\definecolor{currentfill}{rgb}{0.000000,0.000000,0.000000}%
\pgfsetfillcolor{currentfill}%
\pgfsetlinewidth{1.003750pt}%
\definecolor{currentstroke}{rgb}{0.000000,0.000000,0.000000}%
\pgfsetstrokecolor{currentstroke}%
\pgfsetdash{}{0pt}%
\pgfpathmoveto{\pgfqpoint{1.530548in}{1.438667in}}%
\pgfpathcurveto{\pgfqpoint{1.541598in}{1.438667in}}{\pgfqpoint{1.552197in}{1.443057in}}{\pgfqpoint{1.560011in}{1.450871in}}%
\pgfpathcurveto{\pgfqpoint{1.567825in}{1.458685in}}{\pgfqpoint{1.572215in}{1.469284in}}{\pgfqpoint{1.572215in}{1.480334in}}%
\pgfpathcurveto{\pgfqpoint{1.572215in}{1.491384in}}{\pgfqpoint{1.567825in}{1.501983in}}{\pgfqpoint{1.560011in}{1.509796in}}%
\pgfpathcurveto{\pgfqpoint{1.552197in}{1.517610in}}{\pgfqpoint{1.541598in}{1.522000in}}{\pgfqpoint{1.530548in}{1.522000in}}%
\pgfpathcurveto{\pgfqpoint{1.519498in}{1.522000in}}{\pgfqpoint{1.508899in}{1.517610in}}{\pgfqpoint{1.501085in}{1.509796in}}%
\pgfpathcurveto{\pgfqpoint{1.493272in}{1.501983in}}{\pgfqpoint{1.488881in}{1.491384in}}{\pgfqpoint{1.488881in}{1.480334in}}%
\pgfpathcurveto{\pgfqpoint{1.488881in}{1.469284in}}{\pgfqpoint{1.493272in}{1.458685in}}{\pgfqpoint{1.501085in}{1.450871in}}%
\pgfpathcurveto{\pgfqpoint{1.508899in}{1.443057in}}{\pgfqpoint{1.519498in}{1.438667in}}{\pgfqpoint{1.530548in}{1.438667in}}%
\pgfpathclose%
\pgfusepath{stroke,fill}%
\end{pgfscope}%
\begin{pgfscope}%
\pgfpathrectangle{\pgfqpoint{0.375000in}{0.330000in}}{\pgfqpoint{2.325000in}{2.310000in}}%
\pgfusepath{clip}%
\pgfsetbuttcap%
\pgfsetroundjoin%
\definecolor{currentfill}{rgb}{0.000000,0.000000,0.000000}%
\pgfsetfillcolor{currentfill}%
\pgfsetlinewidth{1.003750pt}%
\definecolor{currentstroke}{rgb}{0.000000,0.000000,0.000000}%
\pgfsetstrokecolor{currentstroke}%
\pgfsetdash{}{0pt}%
\pgfpathmoveto{\pgfqpoint{1.530548in}{1.438667in}}%
\pgfpathcurveto{\pgfqpoint{1.541598in}{1.438667in}}{\pgfqpoint{1.552197in}{1.443057in}}{\pgfqpoint{1.560011in}{1.450871in}}%
\pgfpathcurveto{\pgfqpoint{1.567825in}{1.458685in}}{\pgfqpoint{1.572215in}{1.469284in}}{\pgfqpoint{1.572215in}{1.480334in}}%
\pgfpathcurveto{\pgfqpoint{1.572215in}{1.491384in}}{\pgfqpoint{1.567825in}{1.501983in}}{\pgfqpoint{1.560011in}{1.509796in}}%
\pgfpathcurveto{\pgfqpoint{1.552197in}{1.517610in}}{\pgfqpoint{1.541598in}{1.522000in}}{\pgfqpoint{1.530548in}{1.522000in}}%
\pgfpathcurveto{\pgfqpoint{1.519498in}{1.522000in}}{\pgfqpoint{1.508899in}{1.517610in}}{\pgfqpoint{1.501085in}{1.509796in}}%
\pgfpathcurveto{\pgfqpoint{1.493272in}{1.501983in}}{\pgfqpoint{1.488881in}{1.491384in}}{\pgfqpoint{1.488881in}{1.480334in}}%
\pgfpathcurveto{\pgfqpoint{1.488881in}{1.469284in}}{\pgfqpoint{1.493272in}{1.458685in}}{\pgfqpoint{1.501085in}{1.450871in}}%
\pgfpathcurveto{\pgfqpoint{1.508899in}{1.443057in}}{\pgfqpoint{1.519498in}{1.438667in}}{\pgfqpoint{1.530548in}{1.438667in}}%
\pgfpathclose%
\pgfusepath{stroke,fill}%
\end{pgfscope}%
\begin{pgfscope}%
\pgfpathrectangle{\pgfqpoint{0.375000in}{0.330000in}}{\pgfqpoint{2.325000in}{2.310000in}}%
\pgfusepath{clip}%
\pgfsetbuttcap%
\pgfsetroundjoin%
\definecolor{currentfill}{rgb}{0.000000,0.000000,0.000000}%
\pgfsetfillcolor{currentfill}%
\pgfsetlinewidth{1.003750pt}%
\definecolor{currentstroke}{rgb}{0.000000,0.000000,0.000000}%
\pgfsetstrokecolor{currentstroke}%
\pgfsetdash{}{0pt}%
\pgfpathmoveto{\pgfqpoint{1.530548in}{1.438667in}}%
\pgfpathcurveto{\pgfqpoint{1.541598in}{1.438667in}}{\pgfqpoint{1.552197in}{1.443057in}}{\pgfqpoint{1.560011in}{1.450871in}}%
\pgfpathcurveto{\pgfqpoint{1.567825in}{1.458685in}}{\pgfqpoint{1.572215in}{1.469284in}}{\pgfqpoint{1.572215in}{1.480334in}}%
\pgfpathcurveto{\pgfqpoint{1.572215in}{1.491384in}}{\pgfqpoint{1.567825in}{1.501983in}}{\pgfqpoint{1.560011in}{1.509796in}}%
\pgfpathcurveto{\pgfqpoint{1.552197in}{1.517610in}}{\pgfqpoint{1.541598in}{1.522000in}}{\pgfqpoint{1.530548in}{1.522000in}}%
\pgfpathcurveto{\pgfqpoint{1.519498in}{1.522000in}}{\pgfqpoint{1.508899in}{1.517610in}}{\pgfqpoint{1.501085in}{1.509796in}}%
\pgfpathcurveto{\pgfqpoint{1.493272in}{1.501983in}}{\pgfqpoint{1.488881in}{1.491384in}}{\pgfqpoint{1.488881in}{1.480334in}}%
\pgfpathcurveto{\pgfqpoint{1.488881in}{1.469284in}}{\pgfqpoint{1.493272in}{1.458685in}}{\pgfqpoint{1.501085in}{1.450871in}}%
\pgfpathcurveto{\pgfqpoint{1.508899in}{1.443057in}}{\pgfqpoint{1.519498in}{1.438667in}}{\pgfqpoint{1.530548in}{1.438667in}}%
\pgfpathclose%
\pgfusepath{stroke,fill}%
\end{pgfscope}%
\begin{pgfscope}%
\pgfpathrectangle{\pgfqpoint{0.375000in}{0.330000in}}{\pgfqpoint{2.325000in}{2.310000in}}%
\pgfusepath{clip}%
\pgfsetbuttcap%
\pgfsetroundjoin%
\definecolor{currentfill}{rgb}{0.000000,0.000000,0.000000}%
\pgfsetfillcolor{currentfill}%
\pgfsetlinewidth{1.003750pt}%
\definecolor{currentstroke}{rgb}{0.000000,0.000000,0.000000}%
\pgfsetstrokecolor{currentstroke}%
\pgfsetdash{}{0pt}%
\pgfpathmoveto{\pgfqpoint{1.530548in}{1.438667in}}%
\pgfpathcurveto{\pgfqpoint{1.541598in}{1.438667in}}{\pgfqpoint{1.552197in}{1.443057in}}{\pgfqpoint{1.560011in}{1.450871in}}%
\pgfpathcurveto{\pgfqpoint{1.567825in}{1.458685in}}{\pgfqpoint{1.572215in}{1.469284in}}{\pgfqpoint{1.572215in}{1.480334in}}%
\pgfpathcurveto{\pgfqpoint{1.572215in}{1.491384in}}{\pgfqpoint{1.567825in}{1.501983in}}{\pgfqpoint{1.560011in}{1.509796in}}%
\pgfpathcurveto{\pgfqpoint{1.552197in}{1.517610in}}{\pgfqpoint{1.541598in}{1.522000in}}{\pgfqpoint{1.530548in}{1.522000in}}%
\pgfpathcurveto{\pgfqpoint{1.519498in}{1.522000in}}{\pgfqpoint{1.508899in}{1.517610in}}{\pgfqpoint{1.501085in}{1.509796in}}%
\pgfpathcurveto{\pgfqpoint{1.493272in}{1.501983in}}{\pgfqpoint{1.488881in}{1.491384in}}{\pgfqpoint{1.488881in}{1.480334in}}%
\pgfpathcurveto{\pgfqpoint{1.488881in}{1.469284in}}{\pgfqpoint{1.493272in}{1.458685in}}{\pgfqpoint{1.501085in}{1.450871in}}%
\pgfpathcurveto{\pgfqpoint{1.508899in}{1.443057in}}{\pgfqpoint{1.519498in}{1.438667in}}{\pgfqpoint{1.530548in}{1.438667in}}%
\pgfpathclose%
\pgfusepath{stroke,fill}%
\end{pgfscope}%
\begin{pgfscope}%
\pgfpathrectangle{\pgfqpoint{0.375000in}{0.330000in}}{\pgfqpoint{2.325000in}{2.310000in}}%
\pgfusepath{clip}%
\pgfsetbuttcap%
\pgfsetroundjoin%
\definecolor{currentfill}{rgb}{0.000000,0.000000,0.000000}%
\pgfsetfillcolor{currentfill}%
\pgfsetlinewidth{1.003750pt}%
\definecolor{currentstroke}{rgb}{0.000000,0.000000,0.000000}%
\pgfsetstrokecolor{currentstroke}%
\pgfsetdash{}{0pt}%
\pgfpathmoveto{\pgfqpoint{1.530548in}{1.438667in}}%
\pgfpathcurveto{\pgfqpoint{1.541598in}{1.438667in}}{\pgfqpoint{1.552197in}{1.443057in}}{\pgfqpoint{1.560011in}{1.450871in}}%
\pgfpathcurveto{\pgfqpoint{1.567825in}{1.458685in}}{\pgfqpoint{1.572215in}{1.469284in}}{\pgfqpoint{1.572215in}{1.480334in}}%
\pgfpathcurveto{\pgfqpoint{1.572215in}{1.491384in}}{\pgfqpoint{1.567825in}{1.501983in}}{\pgfqpoint{1.560011in}{1.509796in}}%
\pgfpathcurveto{\pgfqpoint{1.552197in}{1.517610in}}{\pgfqpoint{1.541598in}{1.522000in}}{\pgfqpoint{1.530548in}{1.522000in}}%
\pgfpathcurveto{\pgfqpoint{1.519498in}{1.522000in}}{\pgfqpoint{1.508899in}{1.517610in}}{\pgfqpoint{1.501085in}{1.509796in}}%
\pgfpathcurveto{\pgfqpoint{1.493272in}{1.501983in}}{\pgfqpoint{1.488881in}{1.491384in}}{\pgfqpoint{1.488881in}{1.480334in}}%
\pgfpathcurveto{\pgfqpoint{1.488881in}{1.469284in}}{\pgfqpoint{1.493272in}{1.458685in}}{\pgfqpoint{1.501085in}{1.450871in}}%
\pgfpathcurveto{\pgfqpoint{1.508899in}{1.443057in}}{\pgfqpoint{1.519498in}{1.438667in}}{\pgfqpoint{1.530548in}{1.438667in}}%
\pgfpathclose%
\pgfusepath{stroke,fill}%
\end{pgfscope}%
\begin{pgfscope}%
\pgfpathrectangle{\pgfqpoint{0.375000in}{0.330000in}}{\pgfqpoint{2.325000in}{2.310000in}}%
\pgfusepath{clip}%
\pgfsetbuttcap%
\pgfsetroundjoin%
\definecolor{currentfill}{rgb}{0.000000,0.000000,0.000000}%
\pgfsetfillcolor{currentfill}%
\pgfsetlinewidth{1.003750pt}%
\definecolor{currentstroke}{rgb}{0.000000,0.000000,0.000000}%
\pgfsetstrokecolor{currentstroke}%
\pgfsetdash{}{0pt}%
\pgfpathmoveto{\pgfqpoint{1.530548in}{1.438667in}}%
\pgfpathcurveto{\pgfqpoint{1.541598in}{1.438667in}}{\pgfqpoint{1.552197in}{1.443057in}}{\pgfqpoint{1.560011in}{1.450871in}}%
\pgfpathcurveto{\pgfqpoint{1.567825in}{1.458685in}}{\pgfqpoint{1.572215in}{1.469284in}}{\pgfqpoint{1.572215in}{1.480334in}}%
\pgfpathcurveto{\pgfqpoint{1.572215in}{1.491384in}}{\pgfqpoint{1.567825in}{1.501983in}}{\pgfqpoint{1.560011in}{1.509796in}}%
\pgfpathcurveto{\pgfqpoint{1.552197in}{1.517610in}}{\pgfqpoint{1.541598in}{1.522000in}}{\pgfqpoint{1.530548in}{1.522000in}}%
\pgfpathcurveto{\pgfqpoint{1.519498in}{1.522000in}}{\pgfqpoint{1.508899in}{1.517610in}}{\pgfqpoint{1.501085in}{1.509796in}}%
\pgfpathcurveto{\pgfqpoint{1.493272in}{1.501983in}}{\pgfqpoint{1.488881in}{1.491384in}}{\pgfqpoint{1.488881in}{1.480334in}}%
\pgfpathcurveto{\pgfqpoint{1.488881in}{1.469284in}}{\pgfqpoint{1.493272in}{1.458685in}}{\pgfqpoint{1.501085in}{1.450871in}}%
\pgfpathcurveto{\pgfqpoint{1.508899in}{1.443057in}}{\pgfqpoint{1.519498in}{1.438667in}}{\pgfqpoint{1.530548in}{1.438667in}}%
\pgfpathclose%
\pgfusepath{stroke,fill}%
\end{pgfscope}%
\begin{pgfscope}%
\pgfpathrectangle{\pgfqpoint{0.375000in}{0.330000in}}{\pgfqpoint{2.325000in}{2.310000in}}%
\pgfusepath{clip}%
\pgfsetbuttcap%
\pgfsetroundjoin%
\definecolor{currentfill}{rgb}{0.000000,0.000000,0.000000}%
\pgfsetfillcolor{currentfill}%
\pgfsetlinewidth{1.003750pt}%
\definecolor{currentstroke}{rgb}{0.000000,0.000000,0.000000}%
\pgfsetstrokecolor{currentstroke}%
\pgfsetdash{}{0pt}%
\pgfpathmoveto{\pgfqpoint{1.530548in}{1.438667in}}%
\pgfpathcurveto{\pgfqpoint{1.541598in}{1.438667in}}{\pgfqpoint{1.552197in}{1.443057in}}{\pgfqpoint{1.560011in}{1.450871in}}%
\pgfpathcurveto{\pgfqpoint{1.567825in}{1.458685in}}{\pgfqpoint{1.572215in}{1.469284in}}{\pgfqpoint{1.572215in}{1.480334in}}%
\pgfpathcurveto{\pgfqpoint{1.572215in}{1.491384in}}{\pgfqpoint{1.567825in}{1.501983in}}{\pgfqpoint{1.560011in}{1.509796in}}%
\pgfpathcurveto{\pgfqpoint{1.552197in}{1.517610in}}{\pgfqpoint{1.541598in}{1.522000in}}{\pgfqpoint{1.530548in}{1.522000in}}%
\pgfpathcurveto{\pgfqpoint{1.519498in}{1.522000in}}{\pgfqpoint{1.508899in}{1.517610in}}{\pgfqpoint{1.501085in}{1.509796in}}%
\pgfpathcurveto{\pgfqpoint{1.493272in}{1.501983in}}{\pgfqpoint{1.488881in}{1.491384in}}{\pgfqpoint{1.488881in}{1.480334in}}%
\pgfpathcurveto{\pgfqpoint{1.488881in}{1.469284in}}{\pgfqpoint{1.493272in}{1.458685in}}{\pgfqpoint{1.501085in}{1.450871in}}%
\pgfpathcurveto{\pgfqpoint{1.508899in}{1.443057in}}{\pgfqpoint{1.519498in}{1.438667in}}{\pgfqpoint{1.530548in}{1.438667in}}%
\pgfpathclose%
\pgfusepath{stroke,fill}%
\end{pgfscope}%
\begin{pgfscope}%
\pgfpathrectangle{\pgfqpoint{0.375000in}{0.330000in}}{\pgfqpoint{2.325000in}{2.310000in}}%
\pgfusepath{clip}%
\pgfsetbuttcap%
\pgfsetroundjoin%
\definecolor{currentfill}{rgb}{0.000000,0.000000,0.000000}%
\pgfsetfillcolor{currentfill}%
\pgfsetlinewidth{1.003750pt}%
\definecolor{currentstroke}{rgb}{0.000000,0.000000,0.000000}%
\pgfsetstrokecolor{currentstroke}%
\pgfsetdash{}{0pt}%
\pgfpathmoveto{\pgfqpoint{1.530548in}{1.438667in}}%
\pgfpathcurveto{\pgfqpoint{1.541598in}{1.438667in}}{\pgfqpoint{1.552197in}{1.443057in}}{\pgfqpoint{1.560011in}{1.450871in}}%
\pgfpathcurveto{\pgfqpoint{1.567825in}{1.458685in}}{\pgfqpoint{1.572215in}{1.469284in}}{\pgfqpoint{1.572215in}{1.480334in}}%
\pgfpathcurveto{\pgfqpoint{1.572215in}{1.491384in}}{\pgfqpoint{1.567825in}{1.501983in}}{\pgfqpoint{1.560011in}{1.509796in}}%
\pgfpathcurveto{\pgfqpoint{1.552197in}{1.517610in}}{\pgfqpoint{1.541598in}{1.522000in}}{\pgfqpoint{1.530548in}{1.522000in}}%
\pgfpathcurveto{\pgfqpoint{1.519498in}{1.522000in}}{\pgfqpoint{1.508899in}{1.517610in}}{\pgfqpoint{1.501085in}{1.509796in}}%
\pgfpathcurveto{\pgfqpoint{1.493272in}{1.501983in}}{\pgfqpoint{1.488881in}{1.491384in}}{\pgfqpoint{1.488881in}{1.480334in}}%
\pgfpathcurveto{\pgfqpoint{1.488881in}{1.469284in}}{\pgfqpoint{1.493272in}{1.458685in}}{\pgfqpoint{1.501085in}{1.450871in}}%
\pgfpathcurveto{\pgfqpoint{1.508899in}{1.443057in}}{\pgfqpoint{1.519498in}{1.438667in}}{\pgfqpoint{1.530548in}{1.438667in}}%
\pgfpathclose%
\pgfusepath{stroke,fill}%
\end{pgfscope}%
\begin{pgfscope}%
\pgfpathrectangle{\pgfqpoint{0.375000in}{0.330000in}}{\pgfqpoint{2.325000in}{2.310000in}}%
\pgfusepath{clip}%
\pgfsetbuttcap%
\pgfsetroundjoin%
\definecolor{currentfill}{rgb}{0.000000,0.000000,0.000000}%
\pgfsetfillcolor{currentfill}%
\pgfsetlinewidth{1.003750pt}%
\definecolor{currentstroke}{rgb}{0.000000,0.000000,0.000000}%
\pgfsetstrokecolor{currentstroke}%
\pgfsetdash{}{0pt}%
\pgfpathmoveto{\pgfqpoint{1.530548in}{1.438667in}}%
\pgfpathcurveto{\pgfqpoint{1.541598in}{1.438667in}}{\pgfqpoint{1.552197in}{1.443057in}}{\pgfqpoint{1.560011in}{1.450871in}}%
\pgfpathcurveto{\pgfqpoint{1.567825in}{1.458685in}}{\pgfqpoint{1.572215in}{1.469284in}}{\pgfqpoint{1.572215in}{1.480334in}}%
\pgfpathcurveto{\pgfqpoint{1.572215in}{1.491384in}}{\pgfqpoint{1.567825in}{1.501983in}}{\pgfqpoint{1.560011in}{1.509796in}}%
\pgfpathcurveto{\pgfqpoint{1.552197in}{1.517610in}}{\pgfqpoint{1.541598in}{1.522000in}}{\pgfqpoint{1.530548in}{1.522000in}}%
\pgfpathcurveto{\pgfqpoint{1.519498in}{1.522000in}}{\pgfqpoint{1.508899in}{1.517610in}}{\pgfqpoint{1.501085in}{1.509796in}}%
\pgfpathcurveto{\pgfqpoint{1.493272in}{1.501983in}}{\pgfqpoint{1.488881in}{1.491384in}}{\pgfqpoint{1.488881in}{1.480334in}}%
\pgfpathcurveto{\pgfqpoint{1.488881in}{1.469284in}}{\pgfqpoint{1.493272in}{1.458685in}}{\pgfqpoint{1.501085in}{1.450871in}}%
\pgfpathcurveto{\pgfqpoint{1.508899in}{1.443057in}}{\pgfqpoint{1.519498in}{1.438667in}}{\pgfqpoint{1.530548in}{1.438667in}}%
\pgfpathclose%
\pgfusepath{stroke,fill}%
\end{pgfscope}%
\begin{pgfscope}%
\pgfpathrectangle{\pgfqpoint{0.375000in}{0.330000in}}{\pgfqpoint{2.325000in}{2.310000in}}%
\pgfusepath{clip}%
\pgfsetbuttcap%
\pgfsetroundjoin%
\definecolor{currentfill}{rgb}{0.000000,0.000000,0.000000}%
\pgfsetfillcolor{currentfill}%
\pgfsetlinewidth{1.003750pt}%
\definecolor{currentstroke}{rgb}{0.000000,0.000000,0.000000}%
\pgfsetstrokecolor{currentstroke}%
\pgfsetdash{}{0pt}%
\pgfpathmoveto{\pgfqpoint{1.530548in}{1.438667in}}%
\pgfpathcurveto{\pgfqpoint{1.541598in}{1.438667in}}{\pgfqpoint{1.552197in}{1.443057in}}{\pgfqpoint{1.560011in}{1.450871in}}%
\pgfpathcurveto{\pgfqpoint{1.567825in}{1.458685in}}{\pgfqpoint{1.572215in}{1.469284in}}{\pgfqpoint{1.572215in}{1.480334in}}%
\pgfpathcurveto{\pgfqpoint{1.572215in}{1.491384in}}{\pgfqpoint{1.567825in}{1.501983in}}{\pgfqpoint{1.560011in}{1.509796in}}%
\pgfpathcurveto{\pgfqpoint{1.552197in}{1.517610in}}{\pgfqpoint{1.541598in}{1.522000in}}{\pgfqpoint{1.530548in}{1.522000in}}%
\pgfpathcurveto{\pgfqpoint{1.519498in}{1.522000in}}{\pgfqpoint{1.508899in}{1.517610in}}{\pgfqpoint{1.501085in}{1.509796in}}%
\pgfpathcurveto{\pgfqpoint{1.493272in}{1.501983in}}{\pgfqpoint{1.488881in}{1.491384in}}{\pgfqpoint{1.488881in}{1.480334in}}%
\pgfpathcurveto{\pgfqpoint{1.488881in}{1.469284in}}{\pgfqpoint{1.493272in}{1.458685in}}{\pgfqpoint{1.501085in}{1.450871in}}%
\pgfpathcurveto{\pgfqpoint{1.508899in}{1.443057in}}{\pgfqpoint{1.519498in}{1.438667in}}{\pgfqpoint{1.530548in}{1.438667in}}%
\pgfpathclose%
\pgfusepath{stroke,fill}%
\end{pgfscope}%
\begin{pgfscope}%
\pgfpathrectangle{\pgfqpoint{0.375000in}{0.330000in}}{\pgfqpoint{2.325000in}{2.310000in}}%
\pgfusepath{clip}%
\pgfsetbuttcap%
\pgfsetroundjoin%
\definecolor{currentfill}{rgb}{0.000000,0.000000,0.000000}%
\pgfsetfillcolor{currentfill}%
\pgfsetlinewidth{1.003750pt}%
\definecolor{currentstroke}{rgb}{0.000000,0.000000,0.000000}%
\pgfsetstrokecolor{currentstroke}%
\pgfsetdash{}{0pt}%
\pgfpathmoveto{\pgfqpoint{1.530548in}{1.438667in}}%
\pgfpathcurveto{\pgfqpoint{1.541598in}{1.438667in}}{\pgfqpoint{1.552197in}{1.443057in}}{\pgfqpoint{1.560011in}{1.450871in}}%
\pgfpathcurveto{\pgfqpoint{1.567825in}{1.458685in}}{\pgfqpoint{1.572215in}{1.469284in}}{\pgfqpoint{1.572215in}{1.480334in}}%
\pgfpathcurveto{\pgfqpoint{1.572215in}{1.491384in}}{\pgfqpoint{1.567825in}{1.501983in}}{\pgfqpoint{1.560011in}{1.509796in}}%
\pgfpathcurveto{\pgfqpoint{1.552197in}{1.517610in}}{\pgfqpoint{1.541598in}{1.522000in}}{\pgfqpoint{1.530548in}{1.522000in}}%
\pgfpathcurveto{\pgfqpoint{1.519498in}{1.522000in}}{\pgfqpoint{1.508899in}{1.517610in}}{\pgfqpoint{1.501085in}{1.509796in}}%
\pgfpathcurveto{\pgfqpoint{1.493272in}{1.501983in}}{\pgfqpoint{1.488881in}{1.491384in}}{\pgfqpoint{1.488881in}{1.480334in}}%
\pgfpathcurveto{\pgfqpoint{1.488881in}{1.469284in}}{\pgfqpoint{1.493272in}{1.458685in}}{\pgfqpoint{1.501085in}{1.450871in}}%
\pgfpathcurveto{\pgfqpoint{1.508899in}{1.443057in}}{\pgfqpoint{1.519498in}{1.438667in}}{\pgfqpoint{1.530548in}{1.438667in}}%
\pgfpathclose%
\pgfusepath{stroke,fill}%
\end{pgfscope}%
\begin{pgfscope}%
\pgfpathrectangle{\pgfqpoint{0.375000in}{0.330000in}}{\pgfqpoint{2.325000in}{2.310000in}}%
\pgfusepath{clip}%
\pgfsetbuttcap%
\pgfsetroundjoin%
\definecolor{currentfill}{rgb}{0.000000,0.000000,0.000000}%
\pgfsetfillcolor{currentfill}%
\pgfsetlinewidth{1.003750pt}%
\definecolor{currentstroke}{rgb}{0.000000,0.000000,0.000000}%
\pgfsetstrokecolor{currentstroke}%
\pgfsetdash{}{0pt}%
\pgfpathmoveto{\pgfqpoint{1.530548in}{1.438667in}}%
\pgfpathcurveto{\pgfqpoint{1.541598in}{1.438667in}}{\pgfqpoint{1.552197in}{1.443057in}}{\pgfqpoint{1.560011in}{1.450871in}}%
\pgfpathcurveto{\pgfqpoint{1.567825in}{1.458685in}}{\pgfqpoint{1.572215in}{1.469284in}}{\pgfqpoint{1.572215in}{1.480334in}}%
\pgfpathcurveto{\pgfqpoint{1.572215in}{1.491384in}}{\pgfqpoint{1.567825in}{1.501983in}}{\pgfqpoint{1.560011in}{1.509796in}}%
\pgfpathcurveto{\pgfqpoint{1.552197in}{1.517610in}}{\pgfqpoint{1.541598in}{1.522000in}}{\pgfqpoint{1.530548in}{1.522000in}}%
\pgfpathcurveto{\pgfqpoint{1.519498in}{1.522000in}}{\pgfqpoint{1.508899in}{1.517610in}}{\pgfqpoint{1.501085in}{1.509796in}}%
\pgfpathcurveto{\pgfqpoint{1.493272in}{1.501983in}}{\pgfqpoint{1.488881in}{1.491384in}}{\pgfqpoint{1.488881in}{1.480334in}}%
\pgfpathcurveto{\pgfqpoint{1.488881in}{1.469284in}}{\pgfqpoint{1.493272in}{1.458685in}}{\pgfqpoint{1.501085in}{1.450871in}}%
\pgfpathcurveto{\pgfqpoint{1.508899in}{1.443057in}}{\pgfqpoint{1.519498in}{1.438667in}}{\pgfqpoint{1.530548in}{1.438667in}}%
\pgfpathclose%
\pgfusepath{stroke,fill}%
\end{pgfscope}%
\begin{pgfscope}%
\pgfpathrectangle{\pgfqpoint{0.375000in}{0.330000in}}{\pgfqpoint{2.325000in}{2.310000in}}%
\pgfusepath{clip}%
\pgfsetbuttcap%
\pgfsetroundjoin%
\definecolor{currentfill}{rgb}{0.000000,0.000000,0.000000}%
\pgfsetfillcolor{currentfill}%
\pgfsetlinewidth{1.003750pt}%
\definecolor{currentstroke}{rgb}{0.000000,0.000000,0.000000}%
\pgfsetstrokecolor{currentstroke}%
\pgfsetdash{}{0pt}%
\pgfpathmoveto{\pgfqpoint{1.530548in}{1.438667in}}%
\pgfpathcurveto{\pgfqpoint{1.541598in}{1.438667in}}{\pgfqpoint{1.552197in}{1.443057in}}{\pgfqpoint{1.560011in}{1.450871in}}%
\pgfpathcurveto{\pgfqpoint{1.567825in}{1.458685in}}{\pgfqpoint{1.572215in}{1.469284in}}{\pgfqpoint{1.572215in}{1.480334in}}%
\pgfpathcurveto{\pgfqpoint{1.572215in}{1.491384in}}{\pgfqpoint{1.567825in}{1.501983in}}{\pgfqpoint{1.560011in}{1.509796in}}%
\pgfpathcurveto{\pgfqpoint{1.552197in}{1.517610in}}{\pgfqpoint{1.541598in}{1.522000in}}{\pgfqpoint{1.530548in}{1.522000in}}%
\pgfpathcurveto{\pgfqpoint{1.519498in}{1.522000in}}{\pgfqpoint{1.508899in}{1.517610in}}{\pgfqpoint{1.501085in}{1.509796in}}%
\pgfpathcurveto{\pgfqpoint{1.493272in}{1.501983in}}{\pgfqpoint{1.488881in}{1.491384in}}{\pgfqpoint{1.488881in}{1.480334in}}%
\pgfpathcurveto{\pgfqpoint{1.488881in}{1.469284in}}{\pgfqpoint{1.493272in}{1.458685in}}{\pgfqpoint{1.501085in}{1.450871in}}%
\pgfpathcurveto{\pgfqpoint{1.508899in}{1.443057in}}{\pgfqpoint{1.519498in}{1.438667in}}{\pgfqpoint{1.530548in}{1.438667in}}%
\pgfpathclose%
\pgfusepath{stroke,fill}%
\end{pgfscope}%
\begin{pgfscope}%
\pgfpathrectangle{\pgfqpoint{0.375000in}{0.330000in}}{\pgfqpoint{2.325000in}{2.310000in}}%
\pgfusepath{clip}%
\pgfsetbuttcap%
\pgfsetroundjoin%
\definecolor{currentfill}{rgb}{0.000000,0.000000,0.000000}%
\pgfsetfillcolor{currentfill}%
\pgfsetlinewidth{1.003750pt}%
\definecolor{currentstroke}{rgb}{0.000000,0.000000,0.000000}%
\pgfsetstrokecolor{currentstroke}%
\pgfsetdash{}{0pt}%
\pgfpathmoveto{\pgfqpoint{1.530548in}{1.438667in}}%
\pgfpathcurveto{\pgfqpoint{1.541598in}{1.438667in}}{\pgfqpoint{1.552197in}{1.443057in}}{\pgfqpoint{1.560011in}{1.450871in}}%
\pgfpathcurveto{\pgfqpoint{1.567825in}{1.458685in}}{\pgfqpoint{1.572215in}{1.469284in}}{\pgfqpoint{1.572215in}{1.480334in}}%
\pgfpathcurveto{\pgfqpoint{1.572215in}{1.491384in}}{\pgfqpoint{1.567825in}{1.501983in}}{\pgfqpoint{1.560011in}{1.509796in}}%
\pgfpathcurveto{\pgfqpoint{1.552197in}{1.517610in}}{\pgfqpoint{1.541598in}{1.522000in}}{\pgfqpoint{1.530548in}{1.522000in}}%
\pgfpathcurveto{\pgfqpoint{1.519498in}{1.522000in}}{\pgfqpoint{1.508899in}{1.517610in}}{\pgfqpoint{1.501085in}{1.509796in}}%
\pgfpathcurveto{\pgfqpoint{1.493272in}{1.501983in}}{\pgfqpoint{1.488881in}{1.491384in}}{\pgfqpoint{1.488881in}{1.480334in}}%
\pgfpathcurveto{\pgfqpoint{1.488881in}{1.469284in}}{\pgfqpoint{1.493272in}{1.458685in}}{\pgfqpoint{1.501085in}{1.450871in}}%
\pgfpathcurveto{\pgfqpoint{1.508899in}{1.443057in}}{\pgfqpoint{1.519498in}{1.438667in}}{\pgfqpoint{1.530548in}{1.438667in}}%
\pgfpathclose%
\pgfusepath{stroke,fill}%
\end{pgfscope}%
\begin{pgfscope}%
\pgfpathrectangle{\pgfqpoint{0.375000in}{0.330000in}}{\pgfqpoint{2.325000in}{2.310000in}}%
\pgfusepath{clip}%
\pgfsetbuttcap%
\pgfsetroundjoin%
\definecolor{currentfill}{rgb}{0.000000,0.000000,0.000000}%
\pgfsetfillcolor{currentfill}%
\pgfsetlinewidth{1.003750pt}%
\definecolor{currentstroke}{rgb}{0.000000,0.000000,0.000000}%
\pgfsetstrokecolor{currentstroke}%
\pgfsetdash{}{0pt}%
\pgfpathmoveto{\pgfqpoint{1.530548in}{1.438667in}}%
\pgfpathcurveto{\pgfqpoint{1.541598in}{1.438667in}}{\pgfqpoint{1.552197in}{1.443057in}}{\pgfqpoint{1.560011in}{1.450871in}}%
\pgfpathcurveto{\pgfqpoint{1.567825in}{1.458685in}}{\pgfqpoint{1.572215in}{1.469284in}}{\pgfqpoint{1.572215in}{1.480334in}}%
\pgfpathcurveto{\pgfqpoint{1.572215in}{1.491384in}}{\pgfqpoint{1.567825in}{1.501983in}}{\pgfqpoint{1.560011in}{1.509796in}}%
\pgfpathcurveto{\pgfqpoint{1.552197in}{1.517610in}}{\pgfqpoint{1.541598in}{1.522000in}}{\pgfqpoint{1.530548in}{1.522000in}}%
\pgfpathcurveto{\pgfqpoint{1.519498in}{1.522000in}}{\pgfqpoint{1.508899in}{1.517610in}}{\pgfqpoint{1.501085in}{1.509796in}}%
\pgfpathcurveto{\pgfqpoint{1.493272in}{1.501983in}}{\pgfqpoint{1.488881in}{1.491384in}}{\pgfqpoint{1.488881in}{1.480334in}}%
\pgfpathcurveto{\pgfqpoint{1.488881in}{1.469284in}}{\pgfqpoint{1.493272in}{1.458685in}}{\pgfqpoint{1.501085in}{1.450871in}}%
\pgfpathcurveto{\pgfqpoint{1.508899in}{1.443057in}}{\pgfqpoint{1.519498in}{1.438667in}}{\pgfqpoint{1.530548in}{1.438667in}}%
\pgfpathclose%
\pgfusepath{stroke,fill}%
\end{pgfscope}%
\begin{pgfscope}%
\pgfpathrectangle{\pgfqpoint{0.375000in}{0.330000in}}{\pgfqpoint{2.325000in}{2.310000in}}%
\pgfusepath{clip}%
\pgfsetbuttcap%
\pgfsetroundjoin%
\definecolor{currentfill}{rgb}{0.000000,0.000000,0.000000}%
\pgfsetfillcolor{currentfill}%
\pgfsetlinewidth{1.003750pt}%
\definecolor{currentstroke}{rgb}{0.000000,0.000000,0.000000}%
\pgfsetstrokecolor{currentstroke}%
\pgfsetdash{}{0pt}%
\pgfpathmoveto{\pgfqpoint{1.530548in}{1.438667in}}%
\pgfpathcurveto{\pgfqpoint{1.541598in}{1.438667in}}{\pgfqpoint{1.552197in}{1.443057in}}{\pgfqpoint{1.560011in}{1.450871in}}%
\pgfpathcurveto{\pgfqpoint{1.567825in}{1.458685in}}{\pgfqpoint{1.572215in}{1.469284in}}{\pgfqpoint{1.572215in}{1.480334in}}%
\pgfpathcurveto{\pgfqpoint{1.572215in}{1.491384in}}{\pgfqpoint{1.567825in}{1.501983in}}{\pgfqpoint{1.560011in}{1.509796in}}%
\pgfpathcurveto{\pgfqpoint{1.552197in}{1.517610in}}{\pgfqpoint{1.541598in}{1.522000in}}{\pgfqpoint{1.530548in}{1.522000in}}%
\pgfpathcurveto{\pgfqpoint{1.519498in}{1.522000in}}{\pgfqpoint{1.508899in}{1.517610in}}{\pgfqpoint{1.501085in}{1.509796in}}%
\pgfpathcurveto{\pgfqpoint{1.493272in}{1.501983in}}{\pgfqpoint{1.488881in}{1.491384in}}{\pgfqpoint{1.488881in}{1.480334in}}%
\pgfpathcurveto{\pgfqpoint{1.488881in}{1.469284in}}{\pgfqpoint{1.493272in}{1.458685in}}{\pgfqpoint{1.501085in}{1.450871in}}%
\pgfpathcurveto{\pgfqpoint{1.508899in}{1.443057in}}{\pgfqpoint{1.519498in}{1.438667in}}{\pgfqpoint{1.530548in}{1.438667in}}%
\pgfpathclose%
\pgfusepath{stroke,fill}%
\end{pgfscope}%
\begin{pgfscope}%
\pgfpathrectangle{\pgfqpoint{0.375000in}{0.330000in}}{\pgfqpoint{2.325000in}{2.310000in}}%
\pgfusepath{clip}%
\pgfsetbuttcap%
\pgfsetroundjoin%
\definecolor{currentfill}{rgb}{0.000000,0.000000,0.000000}%
\pgfsetfillcolor{currentfill}%
\pgfsetlinewidth{1.003750pt}%
\definecolor{currentstroke}{rgb}{0.000000,0.000000,0.000000}%
\pgfsetstrokecolor{currentstroke}%
\pgfsetdash{}{0pt}%
\pgfpathmoveto{\pgfqpoint{1.530548in}{1.438667in}}%
\pgfpathcurveto{\pgfqpoint{1.541598in}{1.438667in}}{\pgfqpoint{1.552197in}{1.443057in}}{\pgfqpoint{1.560011in}{1.450871in}}%
\pgfpathcurveto{\pgfqpoint{1.567825in}{1.458685in}}{\pgfqpoint{1.572215in}{1.469284in}}{\pgfqpoint{1.572215in}{1.480334in}}%
\pgfpathcurveto{\pgfqpoint{1.572215in}{1.491384in}}{\pgfqpoint{1.567825in}{1.501983in}}{\pgfqpoint{1.560011in}{1.509796in}}%
\pgfpathcurveto{\pgfqpoint{1.552197in}{1.517610in}}{\pgfqpoint{1.541598in}{1.522000in}}{\pgfqpoint{1.530548in}{1.522000in}}%
\pgfpathcurveto{\pgfqpoint{1.519498in}{1.522000in}}{\pgfqpoint{1.508899in}{1.517610in}}{\pgfqpoint{1.501085in}{1.509796in}}%
\pgfpathcurveto{\pgfqpoint{1.493272in}{1.501983in}}{\pgfqpoint{1.488881in}{1.491384in}}{\pgfqpoint{1.488881in}{1.480334in}}%
\pgfpathcurveto{\pgfqpoint{1.488881in}{1.469284in}}{\pgfqpoint{1.493272in}{1.458685in}}{\pgfqpoint{1.501085in}{1.450871in}}%
\pgfpathcurveto{\pgfqpoint{1.508899in}{1.443057in}}{\pgfqpoint{1.519498in}{1.438667in}}{\pgfqpoint{1.530548in}{1.438667in}}%
\pgfpathclose%
\pgfusepath{stroke,fill}%
\end{pgfscope}%
\begin{pgfscope}%
\pgfpathrectangle{\pgfqpoint{0.375000in}{0.330000in}}{\pgfqpoint{2.325000in}{2.310000in}}%
\pgfusepath{clip}%
\pgfsetbuttcap%
\pgfsetroundjoin%
\definecolor{currentfill}{rgb}{0.000000,0.000000,0.000000}%
\pgfsetfillcolor{currentfill}%
\pgfsetlinewidth{1.003750pt}%
\definecolor{currentstroke}{rgb}{0.000000,0.000000,0.000000}%
\pgfsetstrokecolor{currentstroke}%
\pgfsetdash{}{0pt}%
\pgfpathmoveto{\pgfqpoint{1.530548in}{1.438667in}}%
\pgfpathcurveto{\pgfqpoint{1.541598in}{1.438667in}}{\pgfqpoint{1.552197in}{1.443057in}}{\pgfqpoint{1.560011in}{1.450871in}}%
\pgfpathcurveto{\pgfqpoint{1.567825in}{1.458685in}}{\pgfqpoint{1.572215in}{1.469284in}}{\pgfqpoint{1.572215in}{1.480334in}}%
\pgfpathcurveto{\pgfqpoint{1.572215in}{1.491384in}}{\pgfqpoint{1.567825in}{1.501983in}}{\pgfqpoint{1.560011in}{1.509796in}}%
\pgfpathcurveto{\pgfqpoint{1.552197in}{1.517610in}}{\pgfqpoint{1.541598in}{1.522000in}}{\pgfqpoint{1.530548in}{1.522000in}}%
\pgfpathcurveto{\pgfqpoint{1.519498in}{1.522000in}}{\pgfqpoint{1.508899in}{1.517610in}}{\pgfqpoint{1.501085in}{1.509796in}}%
\pgfpathcurveto{\pgfqpoint{1.493272in}{1.501983in}}{\pgfqpoint{1.488881in}{1.491384in}}{\pgfqpoint{1.488881in}{1.480334in}}%
\pgfpathcurveto{\pgfqpoint{1.488881in}{1.469284in}}{\pgfqpoint{1.493272in}{1.458685in}}{\pgfqpoint{1.501085in}{1.450871in}}%
\pgfpathcurveto{\pgfqpoint{1.508899in}{1.443057in}}{\pgfqpoint{1.519498in}{1.438667in}}{\pgfqpoint{1.530548in}{1.438667in}}%
\pgfpathclose%
\pgfusepath{stroke,fill}%
\end{pgfscope}%
\begin{pgfscope}%
\pgfpathrectangle{\pgfqpoint{0.375000in}{0.330000in}}{\pgfqpoint{2.325000in}{2.310000in}}%
\pgfusepath{clip}%
\pgfsetbuttcap%
\pgfsetroundjoin%
\definecolor{currentfill}{rgb}{0.000000,0.000000,0.000000}%
\pgfsetfillcolor{currentfill}%
\pgfsetlinewidth{1.003750pt}%
\definecolor{currentstroke}{rgb}{0.000000,0.000000,0.000000}%
\pgfsetstrokecolor{currentstroke}%
\pgfsetdash{}{0pt}%
\pgfpathmoveto{\pgfqpoint{1.530548in}{1.438667in}}%
\pgfpathcurveto{\pgfqpoint{1.541598in}{1.438667in}}{\pgfqpoint{1.552197in}{1.443057in}}{\pgfqpoint{1.560011in}{1.450871in}}%
\pgfpathcurveto{\pgfqpoint{1.567825in}{1.458685in}}{\pgfqpoint{1.572215in}{1.469284in}}{\pgfqpoint{1.572215in}{1.480334in}}%
\pgfpathcurveto{\pgfqpoint{1.572215in}{1.491384in}}{\pgfqpoint{1.567825in}{1.501983in}}{\pgfqpoint{1.560011in}{1.509796in}}%
\pgfpathcurveto{\pgfqpoint{1.552197in}{1.517610in}}{\pgfqpoint{1.541598in}{1.522000in}}{\pgfqpoint{1.530548in}{1.522000in}}%
\pgfpathcurveto{\pgfqpoint{1.519498in}{1.522000in}}{\pgfqpoint{1.508899in}{1.517610in}}{\pgfqpoint{1.501085in}{1.509796in}}%
\pgfpathcurveto{\pgfqpoint{1.493272in}{1.501983in}}{\pgfqpoint{1.488881in}{1.491384in}}{\pgfqpoint{1.488881in}{1.480334in}}%
\pgfpathcurveto{\pgfqpoint{1.488881in}{1.469284in}}{\pgfqpoint{1.493272in}{1.458685in}}{\pgfqpoint{1.501085in}{1.450871in}}%
\pgfpathcurveto{\pgfqpoint{1.508899in}{1.443057in}}{\pgfqpoint{1.519498in}{1.438667in}}{\pgfqpoint{1.530548in}{1.438667in}}%
\pgfpathclose%
\pgfusepath{stroke,fill}%
\end{pgfscope}%
\begin{pgfscope}%
\pgfpathrectangle{\pgfqpoint{0.375000in}{0.330000in}}{\pgfqpoint{2.325000in}{2.310000in}}%
\pgfusepath{clip}%
\pgfsetbuttcap%
\pgfsetroundjoin%
\definecolor{currentfill}{rgb}{0.000000,0.000000,0.000000}%
\pgfsetfillcolor{currentfill}%
\pgfsetlinewidth{1.003750pt}%
\definecolor{currentstroke}{rgb}{0.000000,0.000000,0.000000}%
\pgfsetstrokecolor{currentstroke}%
\pgfsetdash{}{0pt}%
\pgfpathmoveto{\pgfqpoint{1.530548in}{1.438667in}}%
\pgfpathcurveto{\pgfqpoint{1.541598in}{1.438667in}}{\pgfqpoint{1.552197in}{1.443057in}}{\pgfqpoint{1.560011in}{1.450871in}}%
\pgfpathcurveto{\pgfqpoint{1.567825in}{1.458685in}}{\pgfqpoint{1.572215in}{1.469284in}}{\pgfqpoint{1.572215in}{1.480334in}}%
\pgfpathcurveto{\pgfqpoint{1.572215in}{1.491384in}}{\pgfqpoint{1.567825in}{1.501983in}}{\pgfqpoint{1.560011in}{1.509796in}}%
\pgfpathcurveto{\pgfqpoint{1.552197in}{1.517610in}}{\pgfqpoint{1.541598in}{1.522000in}}{\pgfqpoint{1.530548in}{1.522000in}}%
\pgfpathcurveto{\pgfqpoint{1.519498in}{1.522000in}}{\pgfqpoint{1.508899in}{1.517610in}}{\pgfqpoint{1.501085in}{1.509796in}}%
\pgfpathcurveto{\pgfqpoint{1.493272in}{1.501983in}}{\pgfqpoint{1.488881in}{1.491384in}}{\pgfqpoint{1.488881in}{1.480334in}}%
\pgfpathcurveto{\pgfqpoint{1.488881in}{1.469284in}}{\pgfqpoint{1.493272in}{1.458685in}}{\pgfqpoint{1.501085in}{1.450871in}}%
\pgfpathcurveto{\pgfqpoint{1.508899in}{1.443057in}}{\pgfqpoint{1.519498in}{1.438667in}}{\pgfqpoint{1.530548in}{1.438667in}}%
\pgfpathclose%
\pgfusepath{stroke,fill}%
\end{pgfscope}%
\begin{pgfscope}%
\pgfpathrectangle{\pgfqpoint{0.375000in}{0.330000in}}{\pgfqpoint{2.325000in}{2.310000in}}%
\pgfusepath{clip}%
\pgfsetbuttcap%
\pgfsetroundjoin%
\definecolor{currentfill}{rgb}{0.000000,0.000000,0.000000}%
\pgfsetfillcolor{currentfill}%
\pgfsetlinewidth{1.003750pt}%
\definecolor{currentstroke}{rgb}{0.000000,0.000000,0.000000}%
\pgfsetstrokecolor{currentstroke}%
\pgfsetdash{}{0pt}%
\pgfpathmoveto{\pgfqpoint{1.530548in}{1.438667in}}%
\pgfpathcurveto{\pgfqpoint{1.541598in}{1.438667in}}{\pgfqpoint{1.552197in}{1.443057in}}{\pgfqpoint{1.560011in}{1.450871in}}%
\pgfpathcurveto{\pgfqpoint{1.567825in}{1.458685in}}{\pgfqpoint{1.572215in}{1.469284in}}{\pgfqpoint{1.572215in}{1.480334in}}%
\pgfpathcurveto{\pgfqpoint{1.572215in}{1.491384in}}{\pgfqpoint{1.567825in}{1.501983in}}{\pgfqpoint{1.560011in}{1.509796in}}%
\pgfpathcurveto{\pgfqpoint{1.552197in}{1.517610in}}{\pgfqpoint{1.541598in}{1.522000in}}{\pgfqpoint{1.530548in}{1.522000in}}%
\pgfpathcurveto{\pgfqpoint{1.519498in}{1.522000in}}{\pgfqpoint{1.508899in}{1.517610in}}{\pgfqpoint{1.501085in}{1.509796in}}%
\pgfpathcurveto{\pgfqpoint{1.493272in}{1.501983in}}{\pgfqpoint{1.488881in}{1.491384in}}{\pgfqpoint{1.488881in}{1.480334in}}%
\pgfpathcurveto{\pgfqpoint{1.488881in}{1.469284in}}{\pgfqpoint{1.493272in}{1.458685in}}{\pgfqpoint{1.501085in}{1.450871in}}%
\pgfpathcurveto{\pgfqpoint{1.508899in}{1.443057in}}{\pgfqpoint{1.519498in}{1.438667in}}{\pgfqpoint{1.530548in}{1.438667in}}%
\pgfpathclose%
\pgfusepath{stroke,fill}%
\end{pgfscope}%
\begin{pgfscope}%
\pgfpathrectangle{\pgfqpoint{0.375000in}{0.330000in}}{\pgfqpoint{2.325000in}{2.310000in}}%
\pgfusepath{clip}%
\pgfsetbuttcap%
\pgfsetroundjoin%
\definecolor{currentfill}{rgb}{0.000000,0.000000,0.000000}%
\pgfsetfillcolor{currentfill}%
\pgfsetlinewidth{1.003750pt}%
\definecolor{currentstroke}{rgb}{0.000000,0.000000,0.000000}%
\pgfsetstrokecolor{currentstroke}%
\pgfsetdash{}{0pt}%
\pgfpathmoveto{\pgfqpoint{1.530548in}{1.438667in}}%
\pgfpathcurveto{\pgfqpoint{1.541598in}{1.438667in}}{\pgfqpoint{1.552197in}{1.443057in}}{\pgfqpoint{1.560011in}{1.450871in}}%
\pgfpathcurveto{\pgfqpoint{1.567825in}{1.458685in}}{\pgfqpoint{1.572215in}{1.469284in}}{\pgfqpoint{1.572215in}{1.480334in}}%
\pgfpathcurveto{\pgfqpoint{1.572215in}{1.491384in}}{\pgfqpoint{1.567825in}{1.501983in}}{\pgfqpoint{1.560011in}{1.509796in}}%
\pgfpathcurveto{\pgfqpoint{1.552197in}{1.517610in}}{\pgfqpoint{1.541598in}{1.522000in}}{\pgfqpoint{1.530548in}{1.522000in}}%
\pgfpathcurveto{\pgfqpoint{1.519498in}{1.522000in}}{\pgfqpoint{1.508899in}{1.517610in}}{\pgfqpoint{1.501085in}{1.509796in}}%
\pgfpathcurveto{\pgfqpoint{1.493272in}{1.501983in}}{\pgfqpoint{1.488881in}{1.491384in}}{\pgfqpoint{1.488881in}{1.480334in}}%
\pgfpathcurveto{\pgfqpoint{1.488881in}{1.469284in}}{\pgfqpoint{1.493272in}{1.458685in}}{\pgfqpoint{1.501085in}{1.450871in}}%
\pgfpathcurveto{\pgfqpoint{1.508899in}{1.443057in}}{\pgfqpoint{1.519498in}{1.438667in}}{\pgfqpoint{1.530548in}{1.438667in}}%
\pgfpathclose%
\pgfusepath{stroke,fill}%
\end{pgfscope}%
\begin{pgfscope}%
\pgfpathrectangle{\pgfqpoint{0.375000in}{0.330000in}}{\pgfqpoint{2.325000in}{2.310000in}}%
\pgfusepath{clip}%
\pgfsetbuttcap%
\pgfsetroundjoin%
\definecolor{currentfill}{rgb}{0.000000,0.000000,0.000000}%
\pgfsetfillcolor{currentfill}%
\pgfsetlinewidth{1.003750pt}%
\definecolor{currentstroke}{rgb}{0.000000,0.000000,0.000000}%
\pgfsetstrokecolor{currentstroke}%
\pgfsetdash{}{0pt}%
\pgfpathmoveto{\pgfqpoint{1.530548in}{1.438667in}}%
\pgfpathcurveto{\pgfqpoint{1.541598in}{1.438667in}}{\pgfqpoint{1.552197in}{1.443057in}}{\pgfqpoint{1.560011in}{1.450871in}}%
\pgfpathcurveto{\pgfqpoint{1.567825in}{1.458685in}}{\pgfqpoint{1.572215in}{1.469284in}}{\pgfqpoint{1.572215in}{1.480334in}}%
\pgfpathcurveto{\pgfqpoint{1.572215in}{1.491384in}}{\pgfqpoint{1.567825in}{1.501983in}}{\pgfqpoint{1.560011in}{1.509796in}}%
\pgfpathcurveto{\pgfqpoint{1.552197in}{1.517610in}}{\pgfqpoint{1.541598in}{1.522000in}}{\pgfqpoint{1.530548in}{1.522000in}}%
\pgfpathcurveto{\pgfqpoint{1.519498in}{1.522000in}}{\pgfqpoint{1.508899in}{1.517610in}}{\pgfqpoint{1.501085in}{1.509796in}}%
\pgfpathcurveto{\pgfqpoint{1.493272in}{1.501983in}}{\pgfqpoint{1.488881in}{1.491384in}}{\pgfqpoint{1.488881in}{1.480334in}}%
\pgfpathcurveto{\pgfqpoint{1.488881in}{1.469284in}}{\pgfqpoint{1.493272in}{1.458685in}}{\pgfqpoint{1.501085in}{1.450871in}}%
\pgfpathcurveto{\pgfqpoint{1.508899in}{1.443057in}}{\pgfqpoint{1.519498in}{1.438667in}}{\pgfqpoint{1.530548in}{1.438667in}}%
\pgfpathclose%
\pgfusepath{stroke,fill}%
\end{pgfscope}%
\begin{pgfscope}%
\pgfpathrectangle{\pgfqpoint{0.375000in}{0.330000in}}{\pgfqpoint{2.325000in}{2.310000in}}%
\pgfusepath{clip}%
\pgfsetbuttcap%
\pgfsetroundjoin%
\definecolor{currentfill}{rgb}{0.000000,0.000000,0.000000}%
\pgfsetfillcolor{currentfill}%
\pgfsetlinewidth{1.003750pt}%
\definecolor{currentstroke}{rgb}{0.000000,0.000000,0.000000}%
\pgfsetstrokecolor{currentstroke}%
\pgfsetdash{}{0pt}%
\pgfpathmoveto{\pgfqpoint{1.530548in}{1.438667in}}%
\pgfpathcurveto{\pgfqpoint{1.541598in}{1.438667in}}{\pgfqpoint{1.552197in}{1.443057in}}{\pgfqpoint{1.560011in}{1.450871in}}%
\pgfpathcurveto{\pgfqpoint{1.567825in}{1.458685in}}{\pgfqpoint{1.572215in}{1.469284in}}{\pgfqpoint{1.572215in}{1.480334in}}%
\pgfpathcurveto{\pgfqpoint{1.572215in}{1.491384in}}{\pgfqpoint{1.567825in}{1.501983in}}{\pgfqpoint{1.560011in}{1.509796in}}%
\pgfpathcurveto{\pgfqpoint{1.552197in}{1.517610in}}{\pgfqpoint{1.541598in}{1.522000in}}{\pgfqpoint{1.530548in}{1.522000in}}%
\pgfpathcurveto{\pgfqpoint{1.519498in}{1.522000in}}{\pgfqpoint{1.508899in}{1.517610in}}{\pgfqpoint{1.501085in}{1.509796in}}%
\pgfpathcurveto{\pgfqpoint{1.493272in}{1.501983in}}{\pgfqpoint{1.488881in}{1.491384in}}{\pgfqpoint{1.488881in}{1.480334in}}%
\pgfpathcurveto{\pgfqpoint{1.488881in}{1.469284in}}{\pgfqpoint{1.493272in}{1.458685in}}{\pgfqpoint{1.501085in}{1.450871in}}%
\pgfpathcurveto{\pgfqpoint{1.508899in}{1.443057in}}{\pgfqpoint{1.519498in}{1.438667in}}{\pgfqpoint{1.530548in}{1.438667in}}%
\pgfpathclose%
\pgfusepath{stroke,fill}%
\end{pgfscope}%
\begin{pgfscope}%
\pgfpathrectangle{\pgfqpoint{0.375000in}{0.330000in}}{\pgfqpoint{2.325000in}{2.310000in}}%
\pgfusepath{clip}%
\pgfsetbuttcap%
\pgfsetroundjoin%
\definecolor{currentfill}{rgb}{0.000000,0.000000,0.000000}%
\pgfsetfillcolor{currentfill}%
\pgfsetlinewidth{1.003750pt}%
\definecolor{currentstroke}{rgb}{0.000000,0.000000,0.000000}%
\pgfsetstrokecolor{currentstroke}%
\pgfsetdash{}{0pt}%
\pgfpathmoveto{\pgfqpoint{1.530548in}{1.438667in}}%
\pgfpathcurveto{\pgfqpoint{1.541598in}{1.438667in}}{\pgfqpoint{1.552197in}{1.443057in}}{\pgfqpoint{1.560011in}{1.450871in}}%
\pgfpathcurveto{\pgfqpoint{1.567825in}{1.458685in}}{\pgfqpoint{1.572215in}{1.469284in}}{\pgfqpoint{1.572215in}{1.480334in}}%
\pgfpathcurveto{\pgfqpoint{1.572215in}{1.491384in}}{\pgfqpoint{1.567825in}{1.501983in}}{\pgfqpoint{1.560011in}{1.509796in}}%
\pgfpathcurveto{\pgfqpoint{1.552197in}{1.517610in}}{\pgfqpoint{1.541598in}{1.522000in}}{\pgfqpoint{1.530548in}{1.522000in}}%
\pgfpathcurveto{\pgfqpoint{1.519498in}{1.522000in}}{\pgfqpoint{1.508899in}{1.517610in}}{\pgfqpoint{1.501085in}{1.509796in}}%
\pgfpathcurveto{\pgfqpoint{1.493272in}{1.501983in}}{\pgfqpoint{1.488881in}{1.491384in}}{\pgfqpoint{1.488881in}{1.480334in}}%
\pgfpathcurveto{\pgfqpoint{1.488881in}{1.469284in}}{\pgfqpoint{1.493272in}{1.458685in}}{\pgfqpoint{1.501085in}{1.450871in}}%
\pgfpathcurveto{\pgfqpoint{1.508899in}{1.443057in}}{\pgfqpoint{1.519498in}{1.438667in}}{\pgfqpoint{1.530548in}{1.438667in}}%
\pgfpathclose%
\pgfusepath{stroke,fill}%
\end{pgfscope}%
\begin{pgfscope}%
\pgfpathrectangle{\pgfqpoint{0.375000in}{0.330000in}}{\pgfqpoint{2.325000in}{2.310000in}}%
\pgfusepath{clip}%
\pgfsetbuttcap%
\pgfsetroundjoin%
\definecolor{currentfill}{rgb}{0.000000,0.000000,0.000000}%
\pgfsetfillcolor{currentfill}%
\pgfsetlinewidth{1.003750pt}%
\definecolor{currentstroke}{rgb}{0.000000,0.000000,0.000000}%
\pgfsetstrokecolor{currentstroke}%
\pgfsetdash{}{0pt}%
\pgfpathmoveto{\pgfqpoint{1.530548in}{1.438667in}}%
\pgfpathcurveto{\pgfqpoint{1.541598in}{1.438667in}}{\pgfqpoint{1.552197in}{1.443057in}}{\pgfqpoint{1.560011in}{1.450871in}}%
\pgfpathcurveto{\pgfqpoint{1.567825in}{1.458685in}}{\pgfqpoint{1.572215in}{1.469284in}}{\pgfqpoint{1.572215in}{1.480334in}}%
\pgfpathcurveto{\pgfqpoint{1.572215in}{1.491384in}}{\pgfqpoint{1.567825in}{1.501983in}}{\pgfqpoint{1.560011in}{1.509796in}}%
\pgfpathcurveto{\pgfqpoint{1.552197in}{1.517610in}}{\pgfqpoint{1.541598in}{1.522000in}}{\pgfqpoint{1.530548in}{1.522000in}}%
\pgfpathcurveto{\pgfqpoint{1.519498in}{1.522000in}}{\pgfqpoint{1.508899in}{1.517610in}}{\pgfqpoint{1.501085in}{1.509796in}}%
\pgfpathcurveto{\pgfqpoint{1.493272in}{1.501983in}}{\pgfqpoint{1.488881in}{1.491384in}}{\pgfqpoint{1.488881in}{1.480334in}}%
\pgfpathcurveto{\pgfqpoint{1.488881in}{1.469284in}}{\pgfqpoint{1.493272in}{1.458685in}}{\pgfqpoint{1.501085in}{1.450871in}}%
\pgfpathcurveto{\pgfqpoint{1.508899in}{1.443057in}}{\pgfqpoint{1.519498in}{1.438667in}}{\pgfqpoint{1.530548in}{1.438667in}}%
\pgfpathclose%
\pgfusepath{stroke,fill}%
\end{pgfscope}%
\begin{pgfscope}%
\pgfpathrectangle{\pgfqpoint{0.375000in}{0.330000in}}{\pgfqpoint{2.325000in}{2.310000in}}%
\pgfusepath{clip}%
\pgfsetbuttcap%
\pgfsetroundjoin%
\definecolor{currentfill}{rgb}{0.000000,0.000000,0.000000}%
\pgfsetfillcolor{currentfill}%
\pgfsetlinewidth{1.003750pt}%
\definecolor{currentstroke}{rgb}{0.000000,0.000000,0.000000}%
\pgfsetstrokecolor{currentstroke}%
\pgfsetdash{}{0pt}%
\pgfpathmoveto{\pgfqpoint{1.530548in}{1.438667in}}%
\pgfpathcurveto{\pgfqpoint{1.541598in}{1.438667in}}{\pgfqpoint{1.552197in}{1.443057in}}{\pgfqpoint{1.560011in}{1.450871in}}%
\pgfpathcurveto{\pgfqpoint{1.567825in}{1.458685in}}{\pgfqpoint{1.572215in}{1.469284in}}{\pgfqpoint{1.572215in}{1.480334in}}%
\pgfpathcurveto{\pgfqpoint{1.572215in}{1.491384in}}{\pgfqpoint{1.567825in}{1.501983in}}{\pgfqpoint{1.560011in}{1.509796in}}%
\pgfpathcurveto{\pgfqpoint{1.552197in}{1.517610in}}{\pgfqpoint{1.541598in}{1.522000in}}{\pgfqpoint{1.530548in}{1.522000in}}%
\pgfpathcurveto{\pgfqpoint{1.519498in}{1.522000in}}{\pgfqpoint{1.508899in}{1.517610in}}{\pgfqpoint{1.501085in}{1.509796in}}%
\pgfpathcurveto{\pgfqpoint{1.493272in}{1.501983in}}{\pgfqpoint{1.488881in}{1.491384in}}{\pgfqpoint{1.488881in}{1.480334in}}%
\pgfpathcurveto{\pgfqpoint{1.488881in}{1.469284in}}{\pgfqpoint{1.493272in}{1.458685in}}{\pgfqpoint{1.501085in}{1.450871in}}%
\pgfpathcurveto{\pgfqpoint{1.508899in}{1.443057in}}{\pgfqpoint{1.519498in}{1.438667in}}{\pgfqpoint{1.530548in}{1.438667in}}%
\pgfpathclose%
\pgfusepath{stroke,fill}%
\end{pgfscope}%
\begin{pgfscope}%
\pgfpathrectangle{\pgfqpoint{0.375000in}{0.330000in}}{\pgfqpoint{2.325000in}{2.310000in}}%
\pgfusepath{clip}%
\pgfsetbuttcap%
\pgfsetroundjoin%
\definecolor{currentfill}{rgb}{0.000000,0.000000,0.000000}%
\pgfsetfillcolor{currentfill}%
\pgfsetlinewidth{1.003750pt}%
\definecolor{currentstroke}{rgb}{0.000000,0.000000,0.000000}%
\pgfsetstrokecolor{currentstroke}%
\pgfsetdash{}{0pt}%
\pgfpathmoveto{\pgfqpoint{1.530548in}{1.438667in}}%
\pgfpathcurveto{\pgfqpoint{1.541598in}{1.438667in}}{\pgfqpoint{1.552197in}{1.443057in}}{\pgfqpoint{1.560011in}{1.450871in}}%
\pgfpathcurveto{\pgfqpoint{1.567825in}{1.458685in}}{\pgfqpoint{1.572215in}{1.469284in}}{\pgfqpoint{1.572215in}{1.480334in}}%
\pgfpathcurveto{\pgfqpoint{1.572215in}{1.491384in}}{\pgfqpoint{1.567825in}{1.501983in}}{\pgfqpoint{1.560011in}{1.509796in}}%
\pgfpathcurveto{\pgfqpoint{1.552197in}{1.517610in}}{\pgfqpoint{1.541598in}{1.522000in}}{\pgfqpoint{1.530548in}{1.522000in}}%
\pgfpathcurveto{\pgfqpoint{1.519498in}{1.522000in}}{\pgfqpoint{1.508899in}{1.517610in}}{\pgfqpoint{1.501085in}{1.509796in}}%
\pgfpathcurveto{\pgfqpoint{1.493272in}{1.501983in}}{\pgfqpoint{1.488881in}{1.491384in}}{\pgfqpoint{1.488881in}{1.480334in}}%
\pgfpathcurveto{\pgfqpoint{1.488881in}{1.469284in}}{\pgfqpoint{1.493272in}{1.458685in}}{\pgfqpoint{1.501085in}{1.450871in}}%
\pgfpathcurveto{\pgfqpoint{1.508899in}{1.443057in}}{\pgfqpoint{1.519498in}{1.438667in}}{\pgfqpoint{1.530548in}{1.438667in}}%
\pgfpathclose%
\pgfusepath{stroke,fill}%
\end{pgfscope}%
\begin{pgfscope}%
\pgfpathrectangle{\pgfqpoint{0.375000in}{0.330000in}}{\pgfqpoint{2.325000in}{2.310000in}}%
\pgfusepath{clip}%
\pgfsetbuttcap%
\pgfsetroundjoin%
\definecolor{currentfill}{rgb}{0.000000,0.000000,0.000000}%
\pgfsetfillcolor{currentfill}%
\pgfsetlinewidth{1.003750pt}%
\definecolor{currentstroke}{rgb}{0.000000,0.000000,0.000000}%
\pgfsetstrokecolor{currentstroke}%
\pgfsetdash{}{0pt}%
\pgfpathmoveto{\pgfqpoint{1.530548in}{1.438667in}}%
\pgfpathcurveto{\pgfqpoint{1.541598in}{1.438667in}}{\pgfqpoint{1.552197in}{1.443057in}}{\pgfqpoint{1.560011in}{1.450871in}}%
\pgfpathcurveto{\pgfqpoint{1.567825in}{1.458685in}}{\pgfqpoint{1.572215in}{1.469284in}}{\pgfqpoint{1.572215in}{1.480334in}}%
\pgfpathcurveto{\pgfqpoint{1.572215in}{1.491384in}}{\pgfqpoint{1.567825in}{1.501983in}}{\pgfqpoint{1.560011in}{1.509796in}}%
\pgfpathcurveto{\pgfqpoint{1.552197in}{1.517610in}}{\pgfqpoint{1.541598in}{1.522000in}}{\pgfqpoint{1.530548in}{1.522000in}}%
\pgfpathcurveto{\pgfqpoint{1.519498in}{1.522000in}}{\pgfqpoint{1.508899in}{1.517610in}}{\pgfqpoint{1.501085in}{1.509796in}}%
\pgfpathcurveto{\pgfqpoint{1.493272in}{1.501983in}}{\pgfqpoint{1.488881in}{1.491384in}}{\pgfqpoint{1.488881in}{1.480334in}}%
\pgfpathcurveto{\pgfqpoint{1.488881in}{1.469284in}}{\pgfqpoint{1.493272in}{1.458685in}}{\pgfqpoint{1.501085in}{1.450871in}}%
\pgfpathcurveto{\pgfqpoint{1.508899in}{1.443057in}}{\pgfqpoint{1.519498in}{1.438667in}}{\pgfqpoint{1.530548in}{1.438667in}}%
\pgfpathclose%
\pgfusepath{stroke,fill}%
\end{pgfscope}%
\begin{pgfscope}%
\pgfpathrectangle{\pgfqpoint{0.375000in}{0.330000in}}{\pgfqpoint{2.325000in}{2.310000in}}%
\pgfusepath{clip}%
\pgfsetbuttcap%
\pgfsetroundjoin%
\definecolor{currentfill}{rgb}{0.000000,0.000000,0.000000}%
\pgfsetfillcolor{currentfill}%
\pgfsetlinewidth{1.003750pt}%
\definecolor{currentstroke}{rgb}{0.000000,0.000000,0.000000}%
\pgfsetstrokecolor{currentstroke}%
\pgfsetdash{}{0pt}%
\pgfpathmoveto{\pgfqpoint{1.530548in}{1.438667in}}%
\pgfpathcurveto{\pgfqpoint{1.541598in}{1.438667in}}{\pgfqpoint{1.552197in}{1.443057in}}{\pgfqpoint{1.560011in}{1.450871in}}%
\pgfpathcurveto{\pgfqpoint{1.567825in}{1.458685in}}{\pgfqpoint{1.572215in}{1.469284in}}{\pgfqpoint{1.572215in}{1.480334in}}%
\pgfpathcurveto{\pgfqpoint{1.572215in}{1.491384in}}{\pgfqpoint{1.567825in}{1.501983in}}{\pgfqpoint{1.560011in}{1.509796in}}%
\pgfpathcurveto{\pgfqpoint{1.552197in}{1.517610in}}{\pgfqpoint{1.541598in}{1.522000in}}{\pgfqpoint{1.530548in}{1.522000in}}%
\pgfpathcurveto{\pgfqpoint{1.519498in}{1.522000in}}{\pgfqpoint{1.508899in}{1.517610in}}{\pgfqpoint{1.501085in}{1.509796in}}%
\pgfpathcurveto{\pgfqpoint{1.493272in}{1.501983in}}{\pgfqpoint{1.488881in}{1.491384in}}{\pgfqpoint{1.488881in}{1.480334in}}%
\pgfpathcurveto{\pgfqpoint{1.488881in}{1.469284in}}{\pgfqpoint{1.493272in}{1.458685in}}{\pgfqpoint{1.501085in}{1.450871in}}%
\pgfpathcurveto{\pgfqpoint{1.508899in}{1.443057in}}{\pgfqpoint{1.519498in}{1.438667in}}{\pgfqpoint{1.530548in}{1.438667in}}%
\pgfpathclose%
\pgfusepath{stroke,fill}%
\end{pgfscope}%
\begin{pgfscope}%
\pgfpathrectangle{\pgfqpoint{0.375000in}{0.330000in}}{\pgfqpoint{2.325000in}{2.310000in}}%
\pgfusepath{clip}%
\pgfsetbuttcap%
\pgfsetroundjoin%
\definecolor{currentfill}{rgb}{0.000000,0.000000,0.000000}%
\pgfsetfillcolor{currentfill}%
\pgfsetlinewidth{1.003750pt}%
\definecolor{currentstroke}{rgb}{0.000000,0.000000,0.000000}%
\pgfsetstrokecolor{currentstroke}%
\pgfsetdash{}{0pt}%
\pgfpathmoveto{\pgfqpoint{1.530548in}{1.438667in}}%
\pgfpathcurveto{\pgfqpoint{1.541598in}{1.438667in}}{\pgfqpoint{1.552197in}{1.443057in}}{\pgfqpoint{1.560011in}{1.450871in}}%
\pgfpathcurveto{\pgfqpoint{1.567825in}{1.458685in}}{\pgfqpoint{1.572215in}{1.469284in}}{\pgfqpoint{1.572215in}{1.480334in}}%
\pgfpathcurveto{\pgfqpoint{1.572215in}{1.491384in}}{\pgfqpoint{1.567825in}{1.501983in}}{\pgfqpoint{1.560011in}{1.509796in}}%
\pgfpathcurveto{\pgfqpoint{1.552197in}{1.517610in}}{\pgfqpoint{1.541598in}{1.522000in}}{\pgfqpoint{1.530548in}{1.522000in}}%
\pgfpathcurveto{\pgfqpoint{1.519498in}{1.522000in}}{\pgfqpoint{1.508899in}{1.517610in}}{\pgfqpoint{1.501085in}{1.509796in}}%
\pgfpathcurveto{\pgfqpoint{1.493272in}{1.501983in}}{\pgfqpoint{1.488881in}{1.491384in}}{\pgfqpoint{1.488881in}{1.480334in}}%
\pgfpathcurveto{\pgfqpoint{1.488881in}{1.469284in}}{\pgfqpoint{1.493272in}{1.458685in}}{\pgfqpoint{1.501085in}{1.450871in}}%
\pgfpathcurveto{\pgfqpoint{1.508899in}{1.443057in}}{\pgfqpoint{1.519498in}{1.438667in}}{\pgfqpoint{1.530548in}{1.438667in}}%
\pgfpathclose%
\pgfusepath{stroke,fill}%
\end{pgfscope}%
\begin{pgfscope}%
\pgfpathrectangle{\pgfqpoint{0.375000in}{0.330000in}}{\pgfqpoint{2.325000in}{2.310000in}}%
\pgfusepath{clip}%
\pgfsetbuttcap%
\pgfsetroundjoin%
\definecolor{currentfill}{rgb}{0.000000,0.000000,0.000000}%
\pgfsetfillcolor{currentfill}%
\pgfsetlinewidth{1.003750pt}%
\definecolor{currentstroke}{rgb}{0.000000,0.000000,0.000000}%
\pgfsetstrokecolor{currentstroke}%
\pgfsetdash{}{0pt}%
\pgfpathmoveto{\pgfqpoint{2.055402in}{1.438667in}}%
\pgfpathcurveto{\pgfqpoint{2.066452in}{1.438667in}}{\pgfqpoint{2.077051in}{1.443057in}}{\pgfqpoint{2.084865in}{1.450871in}}%
\pgfpathcurveto{\pgfqpoint{2.092678in}{1.458685in}}{\pgfqpoint{2.097069in}{1.469284in}}{\pgfqpoint{2.097069in}{1.480334in}}%
\pgfpathcurveto{\pgfqpoint{2.097069in}{1.491384in}}{\pgfqpoint{2.092678in}{1.501983in}}{\pgfqpoint{2.084865in}{1.509796in}}%
\pgfpathcurveto{\pgfqpoint{2.077051in}{1.517610in}}{\pgfqpoint{2.066452in}{1.522000in}}{\pgfqpoint{2.055402in}{1.522000in}}%
\pgfpathcurveto{\pgfqpoint{2.044352in}{1.522000in}}{\pgfqpoint{2.033753in}{1.517610in}}{\pgfqpoint{2.025939in}{1.509796in}}%
\pgfpathcurveto{\pgfqpoint{2.018125in}{1.501983in}}{\pgfqpoint{2.013735in}{1.491384in}}{\pgfqpoint{2.013735in}{1.480334in}}%
\pgfpathcurveto{\pgfqpoint{2.013735in}{1.469284in}}{\pgfqpoint{2.018125in}{1.458685in}}{\pgfqpoint{2.025939in}{1.450871in}}%
\pgfpathcurveto{\pgfqpoint{2.033753in}{1.443057in}}{\pgfqpoint{2.044352in}{1.438667in}}{\pgfqpoint{2.055402in}{1.438667in}}%
\pgfpathclose%
\pgfusepath{stroke,fill}%
\end{pgfscope}%
\begin{pgfscope}%
\pgfpathrectangle{\pgfqpoint{0.375000in}{0.330000in}}{\pgfqpoint{2.325000in}{2.310000in}}%
\pgfusepath{clip}%
\pgfsetbuttcap%
\pgfsetroundjoin%
\definecolor{currentfill}{rgb}{0.000000,0.000000,0.000000}%
\pgfsetfillcolor{currentfill}%
\pgfsetlinewidth{1.003750pt}%
\definecolor{currentstroke}{rgb}{0.000000,0.000000,0.000000}%
\pgfsetstrokecolor{currentstroke}%
\pgfsetdash{}{0pt}%
\pgfpathmoveto{\pgfqpoint{2.055402in}{1.438667in}}%
\pgfpathcurveto{\pgfqpoint{2.066452in}{1.438667in}}{\pgfqpoint{2.077051in}{1.443057in}}{\pgfqpoint{2.084865in}{1.450871in}}%
\pgfpathcurveto{\pgfqpoint{2.092678in}{1.458685in}}{\pgfqpoint{2.097069in}{1.469284in}}{\pgfqpoint{2.097069in}{1.480334in}}%
\pgfpathcurveto{\pgfqpoint{2.097069in}{1.491384in}}{\pgfqpoint{2.092678in}{1.501983in}}{\pgfqpoint{2.084865in}{1.509796in}}%
\pgfpathcurveto{\pgfqpoint{2.077051in}{1.517610in}}{\pgfqpoint{2.066452in}{1.522000in}}{\pgfqpoint{2.055402in}{1.522000in}}%
\pgfpathcurveto{\pgfqpoint{2.044352in}{1.522000in}}{\pgfqpoint{2.033753in}{1.517610in}}{\pgfqpoint{2.025939in}{1.509796in}}%
\pgfpathcurveto{\pgfqpoint{2.018125in}{1.501983in}}{\pgfqpoint{2.013735in}{1.491384in}}{\pgfqpoint{2.013735in}{1.480334in}}%
\pgfpathcurveto{\pgfqpoint{2.013735in}{1.469284in}}{\pgfqpoint{2.018125in}{1.458685in}}{\pgfqpoint{2.025939in}{1.450871in}}%
\pgfpathcurveto{\pgfqpoint{2.033753in}{1.443057in}}{\pgfqpoint{2.044352in}{1.438667in}}{\pgfqpoint{2.055402in}{1.438667in}}%
\pgfpathclose%
\pgfusepath{stroke,fill}%
\end{pgfscope}%
\begin{pgfscope}%
\pgfpathrectangle{\pgfqpoint{0.375000in}{0.330000in}}{\pgfqpoint{2.325000in}{2.310000in}}%
\pgfusepath{clip}%
\pgfsetbuttcap%
\pgfsetroundjoin%
\definecolor{currentfill}{rgb}{0.000000,0.000000,0.000000}%
\pgfsetfillcolor{currentfill}%
\pgfsetlinewidth{1.003750pt}%
\definecolor{currentstroke}{rgb}{0.000000,0.000000,0.000000}%
\pgfsetstrokecolor{currentstroke}%
\pgfsetdash{}{0pt}%
\pgfpathmoveto{\pgfqpoint{2.055402in}{1.438667in}}%
\pgfpathcurveto{\pgfqpoint{2.066452in}{1.438667in}}{\pgfqpoint{2.077051in}{1.443057in}}{\pgfqpoint{2.084865in}{1.450871in}}%
\pgfpathcurveto{\pgfqpoint{2.092678in}{1.458685in}}{\pgfqpoint{2.097069in}{1.469284in}}{\pgfqpoint{2.097069in}{1.480334in}}%
\pgfpathcurveto{\pgfqpoint{2.097069in}{1.491384in}}{\pgfqpoint{2.092678in}{1.501983in}}{\pgfqpoint{2.084865in}{1.509796in}}%
\pgfpathcurveto{\pgfqpoint{2.077051in}{1.517610in}}{\pgfqpoint{2.066452in}{1.522000in}}{\pgfqpoint{2.055402in}{1.522000in}}%
\pgfpathcurveto{\pgfqpoint{2.044352in}{1.522000in}}{\pgfqpoint{2.033753in}{1.517610in}}{\pgfqpoint{2.025939in}{1.509796in}}%
\pgfpathcurveto{\pgfqpoint{2.018125in}{1.501983in}}{\pgfqpoint{2.013735in}{1.491384in}}{\pgfqpoint{2.013735in}{1.480334in}}%
\pgfpathcurveto{\pgfqpoint{2.013735in}{1.469284in}}{\pgfqpoint{2.018125in}{1.458685in}}{\pgfqpoint{2.025939in}{1.450871in}}%
\pgfpathcurveto{\pgfqpoint{2.033753in}{1.443057in}}{\pgfqpoint{2.044352in}{1.438667in}}{\pgfqpoint{2.055402in}{1.438667in}}%
\pgfpathclose%
\pgfusepath{stroke,fill}%
\end{pgfscope}%
\begin{pgfscope}%
\pgfpathrectangle{\pgfqpoint{0.375000in}{0.330000in}}{\pgfqpoint{2.325000in}{2.310000in}}%
\pgfusepath{clip}%
\pgfsetbuttcap%
\pgfsetroundjoin%
\definecolor{currentfill}{rgb}{0.000000,0.000000,0.000000}%
\pgfsetfillcolor{currentfill}%
\pgfsetlinewidth{1.003750pt}%
\definecolor{currentstroke}{rgb}{0.000000,0.000000,0.000000}%
\pgfsetstrokecolor{currentstroke}%
\pgfsetdash{}{0pt}%
\pgfpathmoveto{\pgfqpoint{2.055402in}{1.438667in}}%
\pgfpathcurveto{\pgfqpoint{2.066452in}{1.438667in}}{\pgfqpoint{2.077051in}{1.443057in}}{\pgfqpoint{2.084865in}{1.450871in}}%
\pgfpathcurveto{\pgfqpoint{2.092678in}{1.458685in}}{\pgfqpoint{2.097069in}{1.469284in}}{\pgfqpoint{2.097069in}{1.480334in}}%
\pgfpathcurveto{\pgfqpoint{2.097069in}{1.491384in}}{\pgfqpoint{2.092678in}{1.501983in}}{\pgfqpoint{2.084865in}{1.509796in}}%
\pgfpathcurveto{\pgfqpoint{2.077051in}{1.517610in}}{\pgfqpoint{2.066452in}{1.522000in}}{\pgfqpoint{2.055402in}{1.522000in}}%
\pgfpathcurveto{\pgfqpoint{2.044352in}{1.522000in}}{\pgfqpoint{2.033753in}{1.517610in}}{\pgfqpoint{2.025939in}{1.509796in}}%
\pgfpathcurveto{\pgfqpoint{2.018125in}{1.501983in}}{\pgfqpoint{2.013735in}{1.491384in}}{\pgfqpoint{2.013735in}{1.480334in}}%
\pgfpathcurveto{\pgfqpoint{2.013735in}{1.469284in}}{\pgfqpoint{2.018125in}{1.458685in}}{\pgfqpoint{2.025939in}{1.450871in}}%
\pgfpathcurveto{\pgfqpoint{2.033753in}{1.443057in}}{\pgfqpoint{2.044352in}{1.438667in}}{\pgfqpoint{2.055402in}{1.438667in}}%
\pgfpathclose%
\pgfusepath{stroke,fill}%
\end{pgfscope}%
\begin{pgfscope}%
\pgfpathrectangle{\pgfqpoint{0.375000in}{0.330000in}}{\pgfqpoint{2.325000in}{2.310000in}}%
\pgfusepath{clip}%
\pgfsetbuttcap%
\pgfsetroundjoin%
\definecolor{currentfill}{rgb}{0.000000,0.000000,0.000000}%
\pgfsetfillcolor{currentfill}%
\pgfsetlinewidth{1.003750pt}%
\definecolor{currentstroke}{rgb}{0.000000,0.000000,0.000000}%
\pgfsetstrokecolor{currentstroke}%
\pgfsetdash{}{0pt}%
\pgfpathmoveto{\pgfqpoint{2.055402in}{1.438667in}}%
\pgfpathcurveto{\pgfqpoint{2.066452in}{1.438667in}}{\pgfqpoint{2.077051in}{1.443057in}}{\pgfqpoint{2.084865in}{1.450871in}}%
\pgfpathcurveto{\pgfqpoint{2.092678in}{1.458685in}}{\pgfqpoint{2.097069in}{1.469284in}}{\pgfqpoint{2.097069in}{1.480334in}}%
\pgfpathcurveto{\pgfqpoint{2.097069in}{1.491384in}}{\pgfqpoint{2.092678in}{1.501983in}}{\pgfqpoint{2.084865in}{1.509796in}}%
\pgfpathcurveto{\pgfqpoint{2.077051in}{1.517610in}}{\pgfqpoint{2.066452in}{1.522000in}}{\pgfqpoint{2.055402in}{1.522000in}}%
\pgfpathcurveto{\pgfqpoint{2.044352in}{1.522000in}}{\pgfqpoint{2.033753in}{1.517610in}}{\pgfqpoint{2.025939in}{1.509796in}}%
\pgfpathcurveto{\pgfqpoint{2.018125in}{1.501983in}}{\pgfqpoint{2.013735in}{1.491384in}}{\pgfqpoint{2.013735in}{1.480334in}}%
\pgfpathcurveto{\pgfqpoint{2.013735in}{1.469284in}}{\pgfqpoint{2.018125in}{1.458685in}}{\pgfqpoint{2.025939in}{1.450871in}}%
\pgfpathcurveto{\pgfqpoint{2.033753in}{1.443057in}}{\pgfqpoint{2.044352in}{1.438667in}}{\pgfqpoint{2.055402in}{1.438667in}}%
\pgfpathclose%
\pgfusepath{stroke,fill}%
\end{pgfscope}%
\begin{pgfscope}%
\pgfpathrectangle{\pgfqpoint{0.375000in}{0.330000in}}{\pgfqpoint{2.325000in}{2.310000in}}%
\pgfusepath{clip}%
\pgfsetbuttcap%
\pgfsetroundjoin%
\definecolor{currentfill}{rgb}{0.000000,0.000000,0.000000}%
\pgfsetfillcolor{currentfill}%
\pgfsetlinewidth{1.003750pt}%
\definecolor{currentstroke}{rgb}{0.000000,0.000000,0.000000}%
\pgfsetstrokecolor{currentstroke}%
\pgfsetdash{}{0pt}%
\pgfpathmoveto{\pgfqpoint{2.055402in}{1.438667in}}%
\pgfpathcurveto{\pgfqpoint{2.066452in}{1.438667in}}{\pgfqpoint{2.077051in}{1.443057in}}{\pgfqpoint{2.084865in}{1.450871in}}%
\pgfpathcurveto{\pgfqpoint{2.092678in}{1.458685in}}{\pgfqpoint{2.097069in}{1.469284in}}{\pgfqpoint{2.097069in}{1.480334in}}%
\pgfpathcurveto{\pgfqpoint{2.097069in}{1.491384in}}{\pgfqpoint{2.092678in}{1.501983in}}{\pgfqpoint{2.084865in}{1.509796in}}%
\pgfpathcurveto{\pgfqpoint{2.077051in}{1.517610in}}{\pgfqpoint{2.066452in}{1.522000in}}{\pgfqpoint{2.055402in}{1.522000in}}%
\pgfpathcurveto{\pgfqpoint{2.044352in}{1.522000in}}{\pgfqpoint{2.033753in}{1.517610in}}{\pgfqpoint{2.025939in}{1.509796in}}%
\pgfpathcurveto{\pgfqpoint{2.018125in}{1.501983in}}{\pgfqpoint{2.013735in}{1.491384in}}{\pgfqpoint{2.013735in}{1.480334in}}%
\pgfpathcurveto{\pgfqpoint{2.013735in}{1.469284in}}{\pgfqpoint{2.018125in}{1.458685in}}{\pgfqpoint{2.025939in}{1.450871in}}%
\pgfpathcurveto{\pgfqpoint{2.033753in}{1.443057in}}{\pgfqpoint{2.044352in}{1.438667in}}{\pgfqpoint{2.055402in}{1.438667in}}%
\pgfpathclose%
\pgfusepath{stroke,fill}%
\end{pgfscope}%
\begin{pgfscope}%
\pgfpathrectangle{\pgfqpoint{0.375000in}{0.330000in}}{\pgfqpoint{2.325000in}{2.310000in}}%
\pgfusepath{clip}%
\pgfsetbuttcap%
\pgfsetroundjoin%
\definecolor{currentfill}{rgb}{0.000000,0.000000,0.000000}%
\pgfsetfillcolor{currentfill}%
\pgfsetlinewidth{1.003750pt}%
\definecolor{currentstroke}{rgb}{0.000000,0.000000,0.000000}%
\pgfsetstrokecolor{currentstroke}%
\pgfsetdash{}{0pt}%
\pgfpathmoveto{\pgfqpoint{2.055402in}{1.438667in}}%
\pgfpathcurveto{\pgfqpoint{2.066452in}{1.438667in}}{\pgfqpoint{2.077051in}{1.443057in}}{\pgfqpoint{2.084865in}{1.450871in}}%
\pgfpathcurveto{\pgfqpoint{2.092678in}{1.458685in}}{\pgfqpoint{2.097069in}{1.469284in}}{\pgfqpoint{2.097069in}{1.480334in}}%
\pgfpathcurveto{\pgfqpoint{2.097069in}{1.491384in}}{\pgfqpoint{2.092678in}{1.501983in}}{\pgfqpoint{2.084865in}{1.509796in}}%
\pgfpathcurveto{\pgfqpoint{2.077051in}{1.517610in}}{\pgfqpoint{2.066452in}{1.522000in}}{\pgfqpoint{2.055402in}{1.522000in}}%
\pgfpathcurveto{\pgfqpoint{2.044352in}{1.522000in}}{\pgfqpoint{2.033753in}{1.517610in}}{\pgfqpoint{2.025939in}{1.509796in}}%
\pgfpathcurveto{\pgfqpoint{2.018125in}{1.501983in}}{\pgfqpoint{2.013735in}{1.491384in}}{\pgfqpoint{2.013735in}{1.480334in}}%
\pgfpathcurveto{\pgfqpoint{2.013735in}{1.469284in}}{\pgfqpoint{2.018125in}{1.458685in}}{\pgfqpoint{2.025939in}{1.450871in}}%
\pgfpathcurveto{\pgfqpoint{2.033753in}{1.443057in}}{\pgfqpoint{2.044352in}{1.438667in}}{\pgfqpoint{2.055402in}{1.438667in}}%
\pgfpathclose%
\pgfusepath{stroke,fill}%
\end{pgfscope}%
\begin{pgfscope}%
\pgfpathrectangle{\pgfqpoint{0.375000in}{0.330000in}}{\pgfqpoint{2.325000in}{2.310000in}}%
\pgfusepath{clip}%
\pgfsetbuttcap%
\pgfsetroundjoin%
\definecolor{currentfill}{rgb}{0.000000,0.000000,0.000000}%
\pgfsetfillcolor{currentfill}%
\pgfsetlinewidth{1.003750pt}%
\definecolor{currentstroke}{rgb}{0.000000,0.000000,0.000000}%
\pgfsetstrokecolor{currentstroke}%
\pgfsetdash{}{0pt}%
\pgfpathmoveto{\pgfqpoint{2.055402in}{1.438667in}}%
\pgfpathcurveto{\pgfqpoint{2.066452in}{1.438667in}}{\pgfqpoint{2.077051in}{1.443057in}}{\pgfqpoint{2.084865in}{1.450871in}}%
\pgfpathcurveto{\pgfqpoint{2.092678in}{1.458685in}}{\pgfqpoint{2.097069in}{1.469284in}}{\pgfqpoint{2.097069in}{1.480334in}}%
\pgfpathcurveto{\pgfqpoint{2.097069in}{1.491384in}}{\pgfqpoint{2.092678in}{1.501983in}}{\pgfqpoint{2.084865in}{1.509796in}}%
\pgfpathcurveto{\pgfqpoint{2.077051in}{1.517610in}}{\pgfqpoint{2.066452in}{1.522000in}}{\pgfqpoint{2.055402in}{1.522000in}}%
\pgfpathcurveto{\pgfqpoint{2.044352in}{1.522000in}}{\pgfqpoint{2.033753in}{1.517610in}}{\pgfqpoint{2.025939in}{1.509796in}}%
\pgfpathcurveto{\pgfqpoint{2.018125in}{1.501983in}}{\pgfqpoint{2.013735in}{1.491384in}}{\pgfqpoint{2.013735in}{1.480334in}}%
\pgfpathcurveto{\pgfqpoint{2.013735in}{1.469284in}}{\pgfqpoint{2.018125in}{1.458685in}}{\pgfqpoint{2.025939in}{1.450871in}}%
\pgfpathcurveto{\pgfqpoint{2.033753in}{1.443057in}}{\pgfqpoint{2.044352in}{1.438667in}}{\pgfqpoint{2.055402in}{1.438667in}}%
\pgfpathclose%
\pgfusepath{stroke,fill}%
\end{pgfscope}%
\begin{pgfscope}%
\pgfpathrectangle{\pgfqpoint{0.375000in}{0.330000in}}{\pgfqpoint{2.325000in}{2.310000in}}%
\pgfusepath{clip}%
\pgfsetbuttcap%
\pgfsetroundjoin%
\definecolor{currentfill}{rgb}{0.000000,0.000000,0.000000}%
\pgfsetfillcolor{currentfill}%
\pgfsetlinewidth{1.003750pt}%
\definecolor{currentstroke}{rgb}{0.000000,0.000000,0.000000}%
\pgfsetstrokecolor{currentstroke}%
\pgfsetdash{}{0pt}%
\pgfpathmoveto{\pgfqpoint{2.055402in}{1.438667in}}%
\pgfpathcurveto{\pgfqpoint{2.066452in}{1.438667in}}{\pgfqpoint{2.077051in}{1.443057in}}{\pgfqpoint{2.084865in}{1.450871in}}%
\pgfpathcurveto{\pgfqpoint{2.092678in}{1.458685in}}{\pgfqpoint{2.097069in}{1.469284in}}{\pgfqpoint{2.097069in}{1.480334in}}%
\pgfpathcurveto{\pgfqpoint{2.097069in}{1.491384in}}{\pgfqpoint{2.092678in}{1.501983in}}{\pgfqpoint{2.084865in}{1.509796in}}%
\pgfpathcurveto{\pgfqpoint{2.077051in}{1.517610in}}{\pgfqpoint{2.066452in}{1.522000in}}{\pgfqpoint{2.055402in}{1.522000in}}%
\pgfpathcurveto{\pgfqpoint{2.044352in}{1.522000in}}{\pgfqpoint{2.033753in}{1.517610in}}{\pgfqpoint{2.025939in}{1.509796in}}%
\pgfpathcurveto{\pgfqpoint{2.018125in}{1.501983in}}{\pgfqpoint{2.013735in}{1.491384in}}{\pgfqpoint{2.013735in}{1.480334in}}%
\pgfpathcurveto{\pgfqpoint{2.013735in}{1.469284in}}{\pgfqpoint{2.018125in}{1.458685in}}{\pgfqpoint{2.025939in}{1.450871in}}%
\pgfpathcurveto{\pgfqpoint{2.033753in}{1.443057in}}{\pgfqpoint{2.044352in}{1.438667in}}{\pgfqpoint{2.055402in}{1.438667in}}%
\pgfpathclose%
\pgfusepath{stroke,fill}%
\end{pgfscope}%
\begin{pgfscope}%
\pgfpathrectangle{\pgfqpoint{0.375000in}{0.330000in}}{\pgfqpoint{2.325000in}{2.310000in}}%
\pgfusepath{clip}%
\pgfsetbuttcap%
\pgfsetroundjoin%
\definecolor{currentfill}{rgb}{0.000000,0.000000,0.000000}%
\pgfsetfillcolor{currentfill}%
\pgfsetlinewidth{1.003750pt}%
\definecolor{currentstroke}{rgb}{0.000000,0.000000,0.000000}%
\pgfsetstrokecolor{currentstroke}%
\pgfsetdash{}{0pt}%
\pgfpathmoveto{\pgfqpoint{2.055402in}{1.438667in}}%
\pgfpathcurveto{\pgfqpoint{2.066452in}{1.438667in}}{\pgfqpoint{2.077051in}{1.443057in}}{\pgfqpoint{2.084865in}{1.450871in}}%
\pgfpathcurveto{\pgfqpoint{2.092678in}{1.458685in}}{\pgfqpoint{2.097069in}{1.469284in}}{\pgfqpoint{2.097069in}{1.480334in}}%
\pgfpathcurveto{\pgfqpoint{2.097069in}{1.491384in}}{\pgfqpoint{2.092678in}{1.501983in}}{\pgfqpoint{2.084865in}{1.509796in}}%
\pgfpathcurveto{\pgfqpoint{2.077051in}{1.517610in}}{\pgfqpoint{2.066452in}{1.522000in}}{\pgfqpoint{2.055402in}{1.522000in}}%
\pgfpathcurveto{\pgfqpoint{2.044352in}{1.522000in}}{\pgfqpoint{2.033753in}{1.517610in}}{\pgfqpoint{2.025939in}{1.509796in}}%
\pgfpathcurveto{\pgfqpoint{2.018125in}{1.501983in}}{\pgfqpoint{2.013735in}{1.491384in}}{\pgfqpoint{2.013735in}{1.480334in}}%
\pgfpathcurveto{\pgfqpoint{2.013735in}{1.469284in}}{\pgfqpoint{2.018125in}{1.458685in}}{\pgfqpoint{2.025939in}{1.450871in}}%
\pgfpathcurveto{\pgfqpoint{2.033753in}{1.443057in}}{\pgfqpoint{2.044352in}{1.438667in}}{\pgfqpoint{2.055402in}{1.438667in}}%
\pgfpathclose%
\pgfusepath{stroke,fill}%
\end{pgfscope}%
\begin{pgfscope}%
\pgfpathrectangle{\pgfqpoint{0.375000in}{0.330000in}}{\pgfqpoint{2.325000in}{2.310000in}}%
\pgfusepath{clip}%
\pgfsetbuttcap%
\pgfsetroundjoin%
\definecolor{currentfill}{rgb}{0.000000,0.000000,0.000000}%
\pgfsetfillcolor{currentfill}%
\pgfsetlinewidth{1.003750pt}%
\definecolor{currentstroke}{rgb}{0.000000,0.000000,0.000000}%
\pgfsetstrokecolor{currentstroke}%
\pgfsetdash{}{0pt}%
\pgfpathmoveto{\pgfqpoint{2.055402in}{1.438667in}}%
\pgfpathcurveto{\pgfqpoint{2.066452in}{1.438667in}}{\pgfqpoint{2.077051in}{1.443057in}}{\pgfqpoint{2.084865in}{1.450871in}}%
\pgfpathcurveto{\pgfqpoint{2.092678in}{1.458685in}}{\pgfqpoint{2.097069in}{1.469284in}}{\pgfqpoint{2.097069in}{1.480334in}}%
\pgfpathcurveto{\pgfqpoint{2.097069in}{1.491384in}}{\pgfqpoint{2.092678in}{1.501983in}}{\pgfqpoint{2.084865in}{1.509796in}}%
\pgfpathcurveto{\pgfqpoint{2.077051in}{1.517610in}}{\pgfqpoint{2.066452in}{1.522000in}}{\pgfqpoint{2.055402in}{1.522000in}}%
\pgfpathcurveto{\pgfqpoint{2.044352in}{1.522000in}}{\pgfqpoint{2.033753in}{1.517610in}}{\pgfqpoint{2.025939in}{1.509796in}}%
\pgfpathcurveto{\pgfqpoint{2.018125in}{1.501983in}}{\pgfqpoint{2.013735in}{1.491384in}}{\pgfqpoint{2.013735in}{1.480334in}}%
\pgfpathcurveto{\pgfqpoint{2.013735in}{1.469284in}}{\pgfqpoint{2.018125in}{1.458685in}}{\pgfqpoint{2.025939in}{1.450871in}}%
\pgfpathcurveto{\pgfqpoint{2.033753in}{1.443057in}}{\pgfqpoint{2.044352in}{1.438667in}}{\pgfqpoint{2.055402in}{1.438667in}}%
\pgfpathclose%
\pgfusepath{stroke,fill}%
\end{pgfscope}%
\begin{pgfscope}%
\pgfpathrectangle{\pgfqpoint{0.375000in}{0.330000in}}{\pgfqpoint{2.325000in}{2.310000in}}%
\pgfusepath{clip}%
\pgfsetbuttcap%
\pgfsetroundjoin%
\definecolor{currentfill}{rgb}{0.000000,0.000000,0.000000}%
\pgfsetfillcolor{currentfill}%
\pgfsetlinewidth{1.003750pt}%
\definecolor{currentstroke}{rgb}{0.000000,0.000000,0.000000}%
\pgfsetstrokecolor{currentstroke}%
\pgfsetdash{}{0pt}%
\pgfpathmoveto{\pgfqpoint{2.055402in}{1.438667in}}%
\pgfpathcurveto{\pgfqpoint{2.066452in}{1.438667in}}{\pgfqpoint{2.077051in}{1.443057in}}{\pgfqpoint{2.084865in}{1.450871in}}%
\pgfpathcurveto{\pgfqpoint{2.092678in}{1.458685in}}{\pgfqpoint{2.097069in}{1.469284in}}{\pgfqpoint{2.097069in}{1.480334in}}%
\pgfpathcurveto{\pgfqpoint{2.097069in}{1.491384in}}{\pgfqpoint{2.092678in}{1.501983in}}{\pgfqpoint{2.084865in}{1.509796in}}%
\pgfpathcurveto{\pgfqpoint{2.077051in}{1.517610in}}{\pgfqpoint{2.066452in}{1.522000in}}{\pgfqpoint{2.055402in}{1.522000in}}%
\pgfpathcurveto{\pgfqpoint{2.044352in}{1.522000in}}{\pgfqpoint{2.033753in}{1.517610in}}{\pgfqpoint{2.025939in}{1.509796in}}%
\pgfpathcurveto{\pgfqpoint{2.018125in}{1.501983in}}{\pgfqpoint{2.013735in}{1.491384in}}{\pgfqpoint{2.013735in}{1.480334in}}%
\pgfpathcurveto{\pgfqpoint{2.013735in}{1.469284in}}{\pgfqpoint{2.018125in}{1.458685in}}{\pgfqpoint{2.025939in}{1.450871in}}%
\pgfpathcurveto{\pgfqpoint{2.033753in}{1.443057in}}{\pgfqpoint{2.044352in}{1.438667in}}{\pgfqpoint{2.055402in}{1.438667in}}%
\pgfpathclose%
\pgfusepath{stroke,fill}%
\end{pgfscope}%
\begin{pgfscope}%
\pgfpathrectangle{\pgfqpoint{0.375000in}{0.330000in}}{\pgfqpoint{2.325000in}{2.310000in}}%
\pgfusepath{clip}%
\pgfsetbuttcap%
\pgfsetroundjoin%
\definecolor{currentfill}{rgb}{0.000000,0.000000,0.000000}%
\pgfsetfillcolor{currentfill}%
\pgfsetlinewidth{1.003750pt}%
\definecolor{currentstroke}{rgb}{0.000000,0.000000,0.000000}%
\pgfsetstrokecolor{currentstroke}%
\pgfsetdash{}{0pt}%
\pgfpathmoveto{\pgfqpoint{2.055402in}{1.438667in}}%
\pgfpathcurveto{\pgfqpoint{2.066452in}{1.438667in}}{\pgfqpoint{2.077051in}{1.443057in}}{\pgfqpoint{2.084865in}{1.450871in}}%
\pgfpathcurveto{\pgfqpoint{2.092678in}{1.458685in}}{\pgfqpoint{2.097069in}{1.469284in}}{\pgfqpoint{2.097069in}{1.480334in}}%
\pgfpathcurveto{\pgfqpoint{2.097069in}{1.491384in}}{\pgfqpoint{2.092678in}{1.501983in}}{\pgfqpoint{2.084865in}{1.509796in}}%
\pgfpathcurveto{\pgfqpoint{2.077051in}{1.517610in}}{\pgfqpoint{2.066452in}{1.522000in}}{\pgfqpoint{2.055402in}{1.522000in}}%
\pgfpathcurveto{\pgfqpoint{2.044352in}{1.522000in}}{\pgfqpoint{2.033753in}{1.517610in}}{\pgfqpoint{2.025939in}{1.509796in}}%
\pgfpathcurveto{\pgfqpoint{2.018125in}{1.501983in}}{\pgfqpoint{2.013735in}{1.491384in}}{\pgfqpoint{2.013735in}{1.480334in}}%
\pgfpathcurveto{\pgfqpoint{2.013735in}{1.469284in}}{\pgfqpoint{2.018125in}{1.458685in}}{\pgfqpoint{2.025939in}{1.450871in}}%
\pgfpathcurveto{\pgfqpoint{2.033753in}{1.443057in}}{\pgfqpoint{2.044352in}{1.438667in}}{\pgfqpoint{2.055402in}{1.438667in}}%
\pgfpathclose%
\pgfusepath{stroke,fill}%
\end{pgfscope}%
\begin{pgfscope}%
\pgfpathrectangle{\pgfqpoint{0.375000in}{0.330000in}}{\pgfqpoint{2.325000in}{2.310000in}}%
\pgfusepath{clip}%
\pgfsetbuttcap%
\pgfsetroundjoin%
\definecolor{currentfill}{rgb}{0.000000,0.000000,0.000000}%
\pgfsetfillcolor{currentfill}%
\pgfsetlinewidth{1.003750pt}%
\definecolor{currentstroke}{rgb}{0.000000,0.000000,0.000000}%
\pgfsetstrokecolor{currentstroke}%
\pgfsetdash{}{0pt}%
\pgfpathmoveto{\pgfqpoint{2.055402in}{1.438667in}}%
\pgfpathcurveto{\pgfqpoint{2.066452in}{1.438667in}}{\pgfqpoint{2.077051in}{1.443057in}}{\pgfqpoint{2.084865in}{1.450871in}}%
\pgfpathcurveto{\pgfqpoint{2.092678in}{1.458685in}}{\pgfqpoint{2.097069in}{1.469284in}}{\pgfqpoint{2.097069in}{1.480334in}}%
\pgfpathcurveto{\pgfqpoint{2.097069in}{1.491384in}}{\pgfqpoint{2.092678in}{1.501983in}}{\pgfqpoint{2.084865in}{1.509796in}}%
\pgfpathcurveto{\pgfqpoint{2.077051in}{1.517610in}}{\pgfqpoint{2.066452in}{1.522000in}}{\pgfqpoint{2.055402in}{1.522000in}}%
\pgfpathcurveto{\pgfqpoint{2.044352in}{1.522000in}}{\pgfqpoint{2.033753in}{1.517610in}}{\pgfqpoint{2.025939in}{1.509796in}}%
\pgfpathcurveto{\pgfqpoint{2.018125in}{1.501983in}}{\pgfqpoint{2.013735in}{1.491384in}}{\pgfqpoint{2.013735in}{1.480334in}}%
\pgfpathcurveto{\pgfqpoint{2.013735in}{1.469284in}}{\pgfqpoint{2.018125in}{1.458685in}}{\pgfqpoint{2.025939in}{1.450871in}}%
\pgfpathcurveto{\pgfqpoint{2.033753in}{1.443057in}}{\pgfqpoint{2.044352in}{1.438667in}}{\pgfqpoint{2.055402in}{1.438667in}}%
\pgfpathclose%
\pgfusepath{stroke,fill}%
\end{pgfscope}%
\begin{pgfscope}%
\pgfpathrectangle{\pgfqpoint{0.375000in}{0.330000in}}{\pgfqpoint{2.325000in}{2.310000in}}%
\pgfusepath{clip}%
\pgfsetbuttcap%
\pgfsetroundjoin%
\definecolor{currentfill}{rgb}{0.000000,0.000000,0.000000}%
\pgfsetfillcolor{currentfill}%
\pgfsetlinewidth{1.003750pt}%
\definecolor{currentstroke}{rgb}{0.000000,0.000000,0.000000}%
\pgfsetstrokecolor{currentstroke}%
\pgfsetdash{}{0pt}%
\pgfpathmoveto{\pgfqpoint{2.055402in}{1.438667in}}%
\pgfpathcurveto{\pgfqpoint{2.066452in}{1.438667in}}{\pgfqpoint{2.077051in}{1.443057in}}{\pgfqpoint{2.084865in}{1.450871in}}%
\pgfpathcurveto{\pgfqpoint{2.092678in}{1.458685in}}{\pgfqpoint{2.097069in}{1.469284in}}{\pgfqpoint{2.097069in}{1.480334in}}%
\pgfpathcurveto{\pgfqpoint{2.097069in}{1.491384in}}{\pgfqpoint{2.092678in}{1.501983in}}{\pgfqpoint{2.084865in}{1.509796in}}%
\pgfpathcurveto{\pgfqpoint{2.077051in}{1.517610in}}{\pgfqpoint{2.066452in}{1.522000in}}{\pgfqpoint{2.055402in}{1.522000in}}%
\pgfpathcurveto{\pgfqpoint{2.044352in}{1.522000in}}{\pgfqpoint{2.033753in}{1.517610in}}{\pgfqpoint{2.025939in}{1.509796in}}%
\pgfpathcurveto{\pgfqpoint{2.018125in}{1.501983in}}{\pgfqpoint{2.013735in}{1.491384in}}{\pgfqpoint{2.013735in}{1.480334in}}%
\pgfpathcurveto{\pgfqpoint{2.013735in}{1.469284in}}{\pgfqpoint{2.018125in}{1.458685in}}{\pgfqpoint{2.025939in}{1.450871in}}%
\pgfpathcurveto{\pgfqpoint{2.033753in}{1.443057in}}{\pgfqpoint{2.044352in}{1.438667in}}{\pgfqpoint{2.055402in}{1.438667in}}%
\pgfpathclose%
\pgfusepath{stroke,fill}%
\end{pgfscope}%
\begin{pgfscope}%
\pgfpathrectangle{\pgfqpoint{0.375000in}{0.330000in}}{\pgfqpoint{2.325000in}{2.310000in}}%
\pgfusepath{clip}%
\pgfsetbuttcap%
\pgfsetroundjoin%
\definecolor{currentfill}{rgb}{0.000000,0.000000,0.000000}%
\pgfsetfillcolor{currentfill}%
\pgfsetlinewidth{1.003750pt}%
\definecolor{currentstroke}{rgb}{0.000000,0.000000,0.000000}%
\pgfsetstrokecolor{currentstroke}%
\pgfsetdash{}{0pt}%
\pgfpathmoveto{\pgfqpoint{2.055402in}{1.438667in}}%
\pgfpathcurveto{\pgfqpoint{2.066452in}{1.438667in}}{\pgfqpoint{2.077051in}{1.443057in}}{\pgfqpoint{2.084865in}{1.450871in}}%
\pgfpathcurveto{\pgfqpoint{2.092678in}{1.458685in}}{\pgfqpoint{2.097069in}{1.469284in}}{\pgfqpoint{2.097069in}{1.480334in}}%
\pgfpathcurveto{\pgfqpoint{2.097069in}{1.491384in}}{\pgfqpoint{2.092678in}{1.501983in}}{\pgfqpoint{2.084865in}{1.509796in}}%
\pgfpathcurveto{\pgfqpoint{2.077051in}{1.517610in}}{\pgfqpoint{2.066452in}{1.522000in}}{\pgfqpoint{2.055402in}{1.522000in}}%
\pgfpathcurveto{\pgfqpoint{2.044352in}{1.522000in}}{\pgfqpoint{2.033753in}{1.517610in}}{\pgfqpoint{2.025939in}{1.509796in}}%
\pgfpathcurveto{\pgfqpoint{2.018125in}{1.501983in}}{\pgfqpoint{2.013735in}{1.491384in}}{\pgfqpoint{2.013735in}{1.480334in}}%
\pgfpathcurveto{\pgfqpoint{2.013735in}{1.469284in}}{\pgfqpoint{2.018125in}{1.458685in}}{\pgfqpoint{2.025939in}{1.450871in}}%
\pgfpathcurveto{\pgfqpoint{2.033753in}{1.443057in}}{\pgfqpoint{2.044352in}{1.438667in}}{\pgfqpoint{2.055402in}{1.438667in}}%
\pgfpathclose%
\pgfusepath{stroke,fill}%
\end{pgfscope}%
\begin{pgfscope}%
\pgfpathrectangle{\pgfqpoint{0.375000in}{0.330000in}}{\pgfqpoint{2.325000in}{2.310000in}}%
\pgfusepath{clip}%
\pgfsetbuttcap%
\pgfsetroundjoin%
\definecolor{currentfill}{rgb}{0.000000,0.000000,0.000000}%
\pgfsetfillcolor{currentfill}%
\pgfsetlinewidth{1.003750pt}%
\definecolor{currentstroke}{rgb}{0.000000,0.000000,0.000000}%
\pgfsetstrokecolor{currentstroke}%
\pgfsetdash{}{0pt}%
\pgfpathmoveto{\pgfqpoint{2.055402in}{1.438667in}}%
\pgfpathcurveto{\pgfqpoint{2.066452in}{1.438667in}}{\pgfqpoint{2.077051in}{1.443057in}}{\pgfqpoint{2.084865in}{1.450871in}}%
\pgfpathcurveto{\pgfqpoint{2.092678in}{1.458685in}}{\pgfqpoint{2.097069in}{1.469284in}}{\pgfqpoint{2.097069in}{1.480334in}}%
\pgfpathcurveto{\pgfqpoint{2.097069in}{1.491384in}}{\pgfqpoint{2.092678in}{1.501983in}}{\pgfqpoint{2.084865in}{1.509796in}}%
\pgfpathcurveto{\pgfqpoint{2.077051in}{1.517610in}}{\pgfqpoint{2.066452in}{1.522000in}}{\pgfqpoint{2.055402in}{1.522000in}}%
\pgfpathcurveto{\pgfqpoint{2.044352in}{1.522000in}}{\pgfqpoint{2.033753in}{1.517610in}}{\pgfqpoint{2.025939in}{1.509796in}}%
\pgfpathcurveto{\pgfqpoint{2.018125in}{1.501983in}}{\pgfqpoint{2.013735in}{1.491384in}}{\pgfqpoint{2.013735in}{1.480334in}}%
\pgfpathcurveto{\pgfqpoint{2.013735in}{1.469284in}}{\pgfqpoint{2.018125in}{1.458685in}}{\pgfqpoint{2.025939in}{1.450871in}}%
\pgfpathcurveto{\pgfqpoint{2.033753in}{1.443057in}}{\pgfqpoint{2.044352in}{1.438667in}}{\pgfqpoint{2.055402in}{1.438667in}}%
\pgfpathclose%
\pgfusepath{stroke,fill}%
\end{pgfscope}%
\begin{pgfscope}%
\pgfpathrectangle{\pgfqpoint{0.375000in}{0.330000in}}{\pgfqpoint{2.325000in}{2.310000in}}%
\pgfusepath{clip}%
\pgfsetbuttcap%
\pgfsetroundjoin%
\definecolor{currentfill}{rgb}{0.000000,0.000000,0.000000}%
\pgfsetfillcolor{currentfill}%
\pgfsetlinewidth{1.003750pt}%
\definecolor{currentstroke}{rgb}{0.000000,0.000000,0.000000}%
\pgfsetstrokecolor{currentstroke}%
\pgfsetdash{}{0pt}%
\pgfpathmoveto{\pgfqpoint{2.055402in}{1.438667in}}%
\pgfpathcurveto{\pgfqpoint{2.066452in}{1.438667in}}{\pgfqpoint{2.077051in}{1.443057in}}{\pgfqpoint{2.084865in}{1.450871in}}%
\pgfpathcurveto{\pgfqpoint{2.092678in}{1.458685in}}{\pgfqpoint{2.097069in}{1.469284in}}{\pgfqpoint{2.097069in}{1.480334in}}%
\pgfpathcurveto{\pgfqpoint{2.097069in}{1.491384in}}{\pgfqpoint{2.092678in}{1.501983in}}{\pgfqpoint{2.084865in}{1.509796in}}%
\pgfpathcurveto{\pgfqpoint{2.077051in}{1.517610in}}{\pgfqpoint{2.066452in}{1.522000in}}{\pgfqpoint{2.055402in}{1.522000in}}%
\pgfpathcurveto{\pgfqpoint{2.044352in}{1.522000in}}{\pgfqpoint{2.033753in}{1.517610in}}{\pgfqpoint{2.025939in}{1.509796in}}%
\pgfpathcurveto{\pgfqpoint{2.018125in}{1.501983in}}{\pgfqpoint{2.013735in}{1.491384in}}{\pgfqpoint{2.013735in}{1.480334in}}%
\pgfpathcurveto{\pgfqpoint{2.013735in}{1.469284in}}{\pgfqpoint{2.018125in}{1.458685in}}{\pgfqpoint{2.025939in}{1.450871in}}%
\pgfpathcurveto{\pgfqpoint{2.033753in}{1.443057in}}{\pgfqpoint{2.044352in}{1.438667in}}{\pgfqpoint{2.055402in}{1.438667in}}%
\pgfpathclose%
\pgfusepath{stroke,fill}%
\end{pgfscope}%
\begin{pgfscope}%
\pgfpathrectangle{\pgfqpoint{0.375000in}{0.330000in}}{\pgfqpoint{2.325000in}{2.310000in}}%
\pgfusepath{clip}%
\pgfsetbuttcap%
\pgfsetroundjoin%
\definecolor{currentfill}{rgb}{0.000000,0.000000,0.000000}%
\pgfsetfillcolor{currentfill}%
\pgfsetlinewidth{1.003750pt}%
\definecolor{currentstroke}{rgb}{0.000000,0.000000,0.000000}%
\pgfsetstrokecolor{currentstroke}%
\pgfsetdash{}{0pt}%
\pgfpathmoveto{\pgfqpoint{2.055402in}{1.438667in}}%
\pgfpathcurveto{\pgfqpoint{2.066452in}{1.438667in}}{\pgfqpoint{2.077051in}{1.443057in}}{\pgfqpoint{2.084865in}{1.450871in}}%
\pgfpathcurveto{\pgfqpoint{2.092678in}{1.458685in}}{\pgfqpoint{2.097069in}{1.469284in}}{\pgfqpoint{2.097069in}{1.480334in}}%
\pgfpathcurveto{\pgfqpoint{2.097069in}{1.491384in}}{\pgfqpoint{2.092678in}{1.501983in}}{\pgfqpoint{2.084865in}{1.509796in}}%
\pgfpathcurveto{\pgfqpoint{2.077051in}{1.517610in}}{\pgfqpoint{2.066452in}{1.522000in}}{\pgfqpoint{2.055402in}{1.522000in}}%
\pgfpathcurveto{\pgfqpoint{2.044352in}{1.522000in}}{\pgfqpoint{2.033753in}{1.517610in}}{\pgfqpoint{2.025939in}{1.509796in}}%
\pgfpathcurveto{\pgfqpoint{2.018125in}{1.501983in}}{\pgfqpoint{2.013735in}{1.491384in}}{\pgfqpoint{2.013735in}{1.480334in}}%
\pgfpathcurveto{\pgfqpoint{2.013735in}{1.469284in}}{\pgfqpoint{2.018125in}{1.458685in}}{\pgfqpoint{2.025939in}{1.450871in}}%
\pgfpathcurveto{\pgfqpoint{2.033753in}{1.443057in}}{\pgfqpoint{2.044352in}{1.438667in}}{\pgfqpoint{2.055402in}{1.438667in}}%
\pgfpathclose%
\pgfusepath{stroke,fill}%
\end{pgfscope}%
\begin{pgfscope}%
\pgfpathrectangle{\pgfqpoint{0.375000in}{0.330000in}}{\pgfqpoint{2.325000in}{2.310000in}}%
\pgfusepath{clip}%
\pgfsetbuttcap%
\pgfsetroundjoin%
\definecolor{currentfill}{rgb}{0.000000,0.000000,0.000000}%
\pgfsetfillcolor{currentfill}%
\pgfsetlinewidth{1.003750pt}%
\definecolor{currentstroke}{rgb}{0.000000,0.000000,0.000000}%
\pgfsetstrokecolor{currentstroke}%
\pgfsetdash{}{0pt}%
\pgfpathmoveto{\pgfqpoint{2.055402in}{1.438667in}}%
\pgfpathcurveto{\pgfqpoint{2.066452in}{1.438667in}}{\pgfqpoint{2.077051in}{1.443057in}}{\pgfqpoint{2.084865in}{1.450871in}}%
\pgfpathcurveto{\pgfqpoint{2.092678in}{1.458685in}}{\pgfqpoint{2.097069in}{1.469284in}}{\pgfqpoint{2.097069in}{1.480334in}}%
\pgfpathcurveto{\pgfqpoint{2.097069in}{1.491384in}}{\pgfqpoint{2.092678in}{1.501983in}}{\pgfqpoint{2.084865in}{1.509796in}}%
\pgfpathcurveto{\pgfqpoint{2.077051in}{1.517610in}}{\pgfqpoint{2.066452in}{1.522000in}}{\pgfqpoint{2.055402in}{1.522000in}}%
\pgfpathcurveto{\pgfqpoint{2.044352in}{1.522000in}}{\pgfqpoint{2.033753in}{1.517610in}}{\pgfqpoint{2.025939in}{1.509796in}}%
\pgfpathcurveto{\pgfqpoint{2.018125in}{1.501983in}}{\pgfqpoint{2.013735in}{1.491384in}}{\pgfqpoint{2.013735in}{1.480334in}}%
\pgfpathcurveto{\pgfqpoint{2.013735in}{1.469284in}}{\pgfqpoint{2.018125in}{1.458685in}}{\pgfqpoint{2.025939in}{1.450871in}}%
\pgfpathcurveto{\pgfqpoint{2.033753in}{1.443057in}}{\pgfqpoint{2.044352in}{1.438667in}}{\pgfqpoint{2.055402in}{1.438667in}}%
\pgfpathclose%
\pgfusepath{stroke,fill}%
\end{pgfscope}%
\begin{pgfscope}%
\pgfpathrectangle{\pgfqpoint{0.375000in}{0.330000in}}{\pgfqpoint{2.325000in}{2.310000in}}%
\pgfusepath{clip}%
\pgfsetbuttcap%
\pgfsetroundjoin%
\definecolor{currentfill}{rgb}{0.000000,0.000000,0.000000}%
\pgfsetfillcolor{currentfill}%
\pgfsetlinewidth{1.003750pt}%
\definecolor{currentstroke}{rgb}{0.000000,0.000000,0.000000}%
\pgfsetstrokecolor{currentstroke}%
\pgfsetdash{}{0pt}%
\pgfpathmoveto{\pgfqpoint{2.055402in}{1.438667in}}%
\pgfpathcurveto{\pgfqpoint{2.066452in}{1.438667in}}{\pgfqpoint{2.077051in}{1.443057in}}{\pgfqpoint{2.084865in}{1.450871in}}%
\pgfpathcurveto{\pgfqpoint{2.092678in}{1.458685in}}{\pgfqpoint{2.097069in}{1.469284in}}{\pgfqpoint{2.097069in}{1.480334in}}%
\pgfpathcurveto{\pgfqpoint{2.097069in}{1.491384in}}{\pgfqpoint{2.092678in}{1.501983in}}{\pgfqpoint{2.084865in}{1.509796in}}%
\pgfpathcurveto{\pgfqpoint{2.077051in}{1.517610in}}{\pgfqpoint{2.066452in}{1.522000in}}{\pgfqpoint{2.055402in}{1.522000in}}%
\pgfpathcurveto{\pgfqpoint{2.044352in}{1.522000in}}{\pgfqpoint{2.033753in}{1.517610in}}{\pgfqpoint{2.025939in}{1.509796in}}%
\pgfpathcurveto{\pgfqpoint{2.018125in}{1.501983in}}{\pgfqpoint{2.013735in}{1.491384in}}{\pgfqpoint{2.013735in}{1.480334in}}%
\pgfpathcurveto{\pgfqpoint{2.013735in}{1.469284in}}{\pgfqpoint{2.018125in}{1.458685in}}{\pgfqpoint{2.025939in}{1.450871in}}%
\pgfpathcurveto{\pgfqpoint{2.033753in}{1.443057in}}{\pgfqpoint{2.044352in}{1.438667in}}{\pgfqpoint{2.055402in}{1.438667in}}%
\pgfpathclose%
\pgfusepath{stroke,fill}%
\end{pgfscope}%
\begin{pgfscope}%
\pgfpathrectangle{\pgfqpoint{0.375000in}{0.330000in}}{\pgfqpoint{2.325000in}{2.310000in}}%
\pgfusepath{clip}%
\pgfsetbuttcap%
\pgfsetroundjoin%
\definecolor{currentfill}{rgb}{0.000000,0.000000,0.000000}%
\pgfsetfillcolor{currentfill}%
\pgfsetlinewidth{1.003750pt}%
\definecolor{currentstroke}{rgb}{0.000000,0.000000,0.000000}%
\pgfsetstrokecolor{currentstroke}%
\pgfsetdash{}{0pt}%
\pgfpathmoveto{\pgfqpoint{2.055402in}{1.438667in}}%
\pgfpathcurveto{\pgfqpoint{2.066452in}{1.438667in}}{\pgfqpoint{2.077051in}{1.443057in}}{\pgfqpoint{2.084865in}{1.450871in}}%
\pgfpathcurveto{\pgfqpoint{2.092678in}{1.458685in}}{\pgfqpoint{2.097069in}{1.469284in}}{\pgfqpoint{2.097069in}{1.480334in}}%
\pgfpathcurveto{\pgfqpoint{2.097069in}{1.491384in}}{\pgfqpoint{2.092678in}{1.501983in}}{\pgfqpoint{2.084865in}{1.509796in}}%
\pgfpathcurveto{\pgfqpoint{2.077051in}{1.517610in}}{\pgfqpoint{2.066452in}{1.522000in}}{\pgfqpoint{2.055402in}{1.522000in}}%
\pgfpathcurveto{\pgfqpoint{2.044352in}{1.522000in}}{\pgfqpoint{2.033753in}{1.517610in}}{\pgfqpoint{2.025939in}{1.509796in}}%
\pgfpathcurveto{\pgfqpoint{2.018125in}{1.501983in}}{\pgfqpoint{2.013735in}{1.491384in}}{\pgfqpoint{2.013735in}{1.480334in}}%
\pgfpathcurveto{\pgfqpoint{2.013735in}{1.469284in}}{\pgfqpoint{2.018125in}{1.458685in}}{\pgfqpoint{2.025939in}{1.450871in}}%
\pgfpathcurveto{\pgfqpoint{2.033753in}{1.443057in}}{\pgfqpoint{2.044352in}{1.438667in}}{\pgfqpoint{2.055402in}{1.438667in}}%
\pgfpathclose%
\pgfusepath{stroke,fill}%
\end{pgfscope}%
\begin{pgfscope}%
\pgfpathrectangle{\pgfqpoint{0.375000in}{0.330000in}}{\pgfqpoint{2.325000in}{2.310000in}}%
\pgfusepath{clip}%
\pgfsetbuttcap%
\pgfsetroundjoin%
\definecolor{currentfill}{rgb}{0.000000,0.000000,0.000000}%
\pgfsetfillcolor{currentfill}%
\pgfsetlinewidth{1.003750pt}%
\definecolor{currentstroke}{rgb}{0.000000,0.000000,0.000000}%
\pgfsetstrokecolor{currentstroke}%
\pgfsetdash{}{0pt}%
\pgfpathmoveto{\pgfqpoint{2.055402in}{1.438667in}}%
\pgfpathcurveto{\pgfqpoint{2.066452in}{1.438667in}}{\pgfqpoint{2.077051in}{1.443057in}}{\pgfqpoint{2.084865in}{1.450871in}}%
\pgfpathcurveto{\pgfqpoint{2.092678in}{1.458685in}}{\pgfqpoint{2.097069in}{1.469284in}}{\pgfqpoint{2.097069in}{1.480334in}}%
\pgfpathcurveto{\pgfqpoint{2.097069in}{1.491384in}}{\pgfqpoint{2.092678in}{1.501983in}}{\pgfqpoint{2.084865in}{1.509796in}}%
\pgfpathcurveto{\pgfqpoint{2.077051in}{1.517610in}}{\pgfqpoint{2.066452in}{1.522000in}}{\pgfqpoint{2.055402in}{1.522000in}}%
\pgfpathcurveto{\pgfqpoint{2.044352in}{1.522000in}}{\pgfqpoint{2.033753in}{1.517610in}}{\pgfqpoint{2.025939in}{1.509796in}}%
\pgfpathcurveto{\pgfqpoint{2.018125in}{1.501983in}}{\pgfqpoint{2.013735in}{1.491384in}}{\pgfqpoint{2.013735in}{1.480334in}}%
\pgfpathcurveto{\pgfqpoint{2.013735in}{1.469284in}}{\pgfqpoint{2.018125in}{1.458685in}}{\pgfqpoint{2.025939in}{1.450871in}}%
\pgfpathcurveto{\pgfqpoint{2.033753in}{1.443057in}}{\pgfqpoint{2.044352in}{1.438667in}}{\pgfqpoint{2.055402in}{1.438667in}}%
\pgfpathclose%
\pgfusepath{stroke,fill}%
\end{pgfscope}%
\begin{pgfscope}%
\pgfpathrectangle{\pgfqpoint{0.375000in}{0.330000in}}{\pgfqpoint{2.325000in}{2.310000in}}%
\pgfusepath{clip}%
\pgfsetbuttcap%
\pgfsetroundjoin%
\definecolor{currentfill}{rgb}{0.000000,0.000000,0.000000}%
\pgfsetfillcolor{currentfill}%
\pgfsetlinewidth{1.003750pt}%
\definecolor{currentstroke}{rgb}{0.000000,0.000000,0.000000}%
\pgfsetstrokecolor{currentstroke}%
\pgfsetdash{}{0pt}%
\pgfpathmoveto{\pgfqpoint{2.055402in}{1.438667in}}%
\pgfpathcurveto{\pgfqpoint{2.066452in}{1.438667in}}{\pgfqpoint{2.077051in}{1.443057in}}{\pgfqpoint{2.084865in}{1.450871in}}%
\pgfpathcurveto{\pgfqpoint{2.092678in}{1.458685in}}{\pgfqpoint{2.097069in}{1.469284in}}{\pgfqpoint{2.097069in}{1.480334in}}%
\pgfpathcurveto{\pgfqpoint{2.097069in}{1.491384in}}{\pgfqpoint{2.092678in}{1.501983in}}{\pgfqpoint{2.084865in}{1.509796in}}%
\pgfpathcurveto{\pgfqpoint{2.077051in}{1.517610in}}{\pgfqpoint{2.066452in}{1.522000in}}{\pgfqpoint{2.055402in}{1.522000in}}%
\pgfpathcurveto{\pgfqpoint{2.044352in}{1.522000in}}{\pgfqpoint{2.033753in}{1.517610in}}{\pgfqpoint{2.025939in}{1.509796in}}%
\pgfpathcurveto{\pgfqpoint{2.018125in}{1.501983in}}{\pgfqpoint{2.013735in}{1.491384in}}{\pgfqpoint{2.013735in}{1.480334in}}%
\pgfpathcurveto{\pgfqpoint{2.013735in}{1.469284in}}{\pgfqpoint{2.018125in}{1.458685in}}{\pgfqpoint{2.025939in}{1.450871in}}%
\pgfpathcurveto{\pgfqpoint{2.033753in}{1.443057in}}{\pgfqpoint{2.044352in}{1.438667in}}{\pgfqpoint{2.055402in}{1.438667in}}%
\pgfpathclose%
\pgfusepath{stroke,fill}%
\end{pgfscope}%
\begin{pgfscope}%
\pgfpathrectangle{\pgfqpoint{0.375000in}{0.330000in}}{\pgfqpoint{2.325000in}{2.310000in}}%
\pgfusepath{clip}%
\pgfsetbuttcap%
\pgfsetroundjoin%
\definecolor{currentfill}{rgb}{0.000000,0.000000,0.000000}%
\pgfsetfillcolor{currentfill}%
\pgfsetlinewidth{1.003750pt}%
\definecolor{currentstroke}{rgb}{0.000000,0.000000,0.000000}%
\pgfsetstrokecolor{currentstroke}%
\pgfsetdash{}{0pt}%
\pgfpathmoveto{\pgfqpoint{2.055402in}{1.438667in}}%
\pgfpathcurveto{\pgfqpoint{2.066452in}{1.438667in}}{\pgfqpoint{2.077051in}{1.443057in}}{\pgfqpoint{2.084865in}{1.450871in}}%
\pgfpathcurveto{\pgfqpoint{2.092678in}{1.458685in}}{\pgfqpoint{2.097069in}{1.469284in}}{\pgfqpoint{2.097069in}{1.480334in}}%
\pgfpathcurveto{\pgfqpoint{2.097069in}{1.491384in}}{\pgfqpoint{2.092678in}{1.501983in}}{\pgfqpoint{2.084865in}{1.509796in}}%
\pgfpathcurveto{\pgfqpoint{2.077051in}{1.517610in}}{\pgfqpoint{2.066452in}{1.522000in}}{\pgfqpoint{2.055402in}{1.522000in}}%
\pgfpathcurveto{\pgfqpoint{2.044352in}{1.522000in}}{\pgfqpoint{2.033753in}{1.517610in}}{\pgfqpoint{2.025939in}{1.509796in}}%
\pgfpathcurveto{\pgfqpoint{2.018125in}{1.501983in}}{\pgfqpoint{2.013735in}{1.491384in}}{\pgfqpoint{2.013735in}{1.480334in}}%
\pgfpathcurveto{\pgfqpoint{2.013735in}{1.469284in}}{\pgfqpoint{2.018125in}{1.458685in}}{\pgfqpoint{2.025939in}{1.450871in}}%
\pgfpathcurveto{\pgfqpoint{2.033753in}{1.443057in}}{\pgfqpoint{2.044352in}{1.438667in}}{\pgfqpoint{2.055402in}{1.438667in}}%
\pgfpathclose%
\pgfusepath{stroke,fill}%
\end{pgfscope}%
\begin{pgfscope}%
\pgfpathrectangle{\pgfqpoint{0.375000in}{0.330000in}}{\pgfqpoint{2.325000in}{2.310000in}}%
\pgfusepath{clip}%
\pgfsetbuttcap%
\pgfsetroundjoin%
\definecolor{currentfill}{rgb}{0.000000,0.000000,0.000000}%
\pgfsetfillcolor{currentfill}%
\pgfsetlinewidth{1.003750pt}%
\definecolor{currentstroke}{rgb}{0.000000,0.000000,0.000000}%
\pgfsetstrokecolor{currentstroke}%
\pgfsetdash{}{0pt}%
\pgfpathmoveto{\pgfqpoint{2.055402in}{1.438667in}}%
\pgfpathcurveto{\pgfqpoint{2.066452in}{1.438667in}}{\pgfqpoint{2.077051in}{1.443057in}}{\pgfqpoint{2.084865in}{1.450871in}}%
\pgfpathcurveto{\pgfqpoint{2.092678in}{1.458685in}}{\pgfqpoint{2.097069in}{1.469284in}}{\pgfqpoint{2.097069in}{1.480334in}}%
\pgfpathcurveto{\pgfqpoint{2.097069in}{1.491384in}}{\pgfqpoint{2.092678in}{1.501983in}}{\pgfqpoint{2.084865in}{1.509796in}}%
\pgfpathcurveto{\pgfqpoint{2.077051in}{1.517610in}}{\pgfqpoint{2.066452in}{1.522000in}}{\pgfqpoint{2.055402in}{1.522000in}}%
\pgfpathcurveto{\pgfqpoint{2.044352in}{1.522000in}}{\pgfqpoint{2.033753in}{1.517610in}}{\pgfqpoint{2.025939in}{1.509796in}}%
\pgfpathcurveto{\pgfqpoint{2.018125in}{1.501983in}}{\pgfqpoint{2.013735in}{1.491384in}}{\pgfqpoint{2.013735in}{1.480334in}}%
\pgfpathcurveto{\pgfqpoint{2.013735in}{1.469284in}}{\pgfqpoint{2.018125in}{1.458685in}}{\pgfqpoint{2.025939in}{1.450871in}}%
\pgfpathcurveto{\pgfqpoint{2.033753in}{1.443057in}}{\pgfqpoint{2.044352in}{1.438667in}}{\pgfqpoint{2.055402in}{1.438667in}}%
\pgfpathclose%
\pgfusepath{stroke,fill}%
\end{pgfscope}%
\begin{pgfscope}%
\pgfpathrectangle{\pgfqpoint{0.375000in}{0.330000in}}{\pgfqpoint{2.325000in}{2.310000in}}%
\pgfusepath{clip}%
\pgfsetbuttcap%
\pgfsetroundjoin%
\definecolor{currentfill}{rgb}{0.000000,0.000000,0.000000}%
\pgfsetfillcolor{currentfill}%
\pgfsetlinewidth{1.003750pt}%
\definecolor{currentstroke}{rgb}{0.000000,0.000000,0.000000}%
\pgfsetstrokecolor{currentstroke}%
\pgfsetdash{}{0pt}%
\pgfpathmoveto{\pgfqpoint{2.055402in}{1.438667in}}%
\pgfpathcurveto{\pgfqpoint{2.066452in}{1.438667in}}{\pgfqpoint{2.077051in}{1.443057in}}{\pgfqpoint{2.084865in}{1.450871in}}%
\pgfpathcurveto{\pgfqpoint{2.092678in}{1.458685in}}{\pgfqpoint{2.097069in}{1.469284in}}{\pgfqpoint{2.097069in}{1.480334in}}%
\pgfpathcurveto{\pgfqpoint{2.097069in}{1.491384in}}{\pgfqpoint{2.092678in}{1.501983in}}{\pgfqpoint{2.084865in}{1.509796in}}%
\pgfpathcurveto{\pgfqpoint{2.077051in}{1.517610in}}{\pgfqpoint{2.066452in}{1.522000in}}{\pgfqpoint{2.055402in}{1.522000in}}%
\pgfpathcurveto{\pgfqpoint{2.044352in}{1.522000in}}{\pgfqpoint{2.033753in}{1.517610in}}{\pgfqpoint{2.025939in}{1.509796in}}%
\pgfpathcurveto{\pgfqpoint{2.018125in}{1.501983in}}{\pgfqpoint{2.013735in}{1.491384in}}{\pgfqpoint{2.013735in}{1.480334in}}%
\pgfpathcurveto{\pgfqpoint{2.013735in}{1.469284in}}{\pgfqpoint{2.018125in}{1.458685in}}{\pgfqpoint{2.025939in}{1.450871in}}%
\pgfpathcurveto{\pgfqpoint{2.033753in}{1.443057in}}{\pgfqpoint{2.044352in}{1.438667in}}{\pgfqpoint{2.055402in}{1.438667in}}%
\pgfpathclose%
\pgfusepath{stroke,fill}%
\end{pgfscope}%
\begin{pgfscope}%
\pgfpathrectangle{\pgfqpoint{0.375000in}{0.330000in}}{\pgfqpoint{2.325000in}{2.310000in}}%
\pgfusepath{clip}%
\pgfsetbuttcap%
\pgfsetroundjoin%
\definecolor{currentfill}{rgb}{0.000000,0.000000,0.000000}%
\pgfsetfillcolor{currentfill}%
\pgfsetlinewidth{1.003750pt}%
\definecolor{currentstroke}{rgb}{0.000000,0.000000,0.000000}%
\pgfsetstrokecolor{currentstroke}%
\pgfsetdash{}{0pt}%
\pgfpathmoveto{\pgfqpoint{2.055402in}{1.438667in}}%
\pgfpathcurveto{\pgfqpoint{2.066452in}{1.438667in}}{\pgfqpoint{2.077051in}{1.443057in}}{\pgfqpoint{2.084865in}{1.450871in}}%
\pgfpathcurveto{\pgfqpoint{2.092678in}{1.458685in}}{\pgfqpoint{2.097069in}{1.469284in}}{\pgfqpoint{2.097069in}{1.480334in}}%
\pgfpathcurveto{\pgfqpoint{2.097069in}{1.491384in}}{\pgfqpoint{2.092678in}{1.501983in}}{\pgfqpoint{2.084865in}{1.509796in}}%
\pgfpathcurveto{\pgfqpoint{2.077051in}{1.517610in}}{\pgfqpoint{2.066452in}{1.522000in}}{\pgfqpoint{2.055402in}{1.522000in}}%
\pgfpathcurveto{\pgfqpoint{2.044352in}{1.522000in}}{\pgfqpoint{2.033753in}{1.517610in}}{\pgfqpoint{2.025939in}{1.509796in}}%
\pgfpathcurveto{\pgfqpoint{2.018125in}{1.501983in}}{\pgfqpoint{2.013735in}{1.491384in}}{\pgfqpoint{2.013735in}{1.480334in}}%
\pgfpathcurveto{\pgfqpoint{2.013735in}{1.469284in}}{\pgfqpoint{2.018125in}{1.458685in}}{\pgfqpoint{2.025939in}{1.450871in}}%
\pgfpathcurveto{\pgfqpoint{2.033753in}{1.443057in}}{\pgfqpoint{2.044352in}{1.438667in}}{\pgfqpoint{2.055402in}{1.438667in}}%
\pgfpathclose%
\pgfusepath{stroke,fill}%
\end{pgfscope}%
\begin{pgfscope}%
\pgfpathrectangle{\pgfqpoint{0.375000in}{0.330000in}}{\pgfqpoint{2.325000in}{2.310000in}}%
\pgfusepath{clip}%
\pgfsetbuttcap%
\pgfsetroundjoin%
\definecolor{currentfill}{rgb}{0.000000,0.000000,0.000000}%
\pgfsetfillcolor{currentfill}%
\pgfsetlinewidth{1.003750pt}%
\definecolor{currentstroke}{rgb}{0.000000,0.000000,0.000000}%
\pgfsetstrokecolor{currentstroke}%
\pgfsetdash{}{0pt}%
\pgfpathmoveto{\pgfqpoint{2.055402in}{1.438667in}}%
\pgfpathcurveto{\pgfqpoint{2.066452in}{1.438667in}}{\pgfqpoint{2.077051in}{1.443057in}}{\pgfqpoint{2.084865in}{1.450871in}}%
\pgfpathcurveto{\pgfqpoint{2.092678in}{1.458685in}}{\pgfqpoint{2.097069in}{1.469284in}}{\pgfqpoint{2.097069in}{1.480334in}}%
\pgfpathcurveto{\pgfqpoint{2.097069in}{1.491384in}}{\pgfqpoint{2.092678in}{1.501983in}}{\pgfqpoint{2.084865in}{1.509796in}}%
\pgfpathcurveto{\pgfqpoint{2.077051in}{1.517610in}}{\pgfqpoint{2.066452in}{1.522000in}}{\pgfqpoint{2.055402in}{1.522000in}}%
\pgfpathcurveto{\pgfqpoint{2.044352in}{1.522000in}}{\pgfqpoint{2.033753in}{1.517610in}}{\pgfqpoint{2.025939in}{1.509796in}}%
\pgfpathcurveto{\pgfqpoint{2.018125in}{1.501983in}}{\pgfqpoint{2.013735in}{1.491384in}}{\pgfqpoint{2.013735in}{1.480334in}}%
\pgfpathcurveto{\pgfqpoint{2.013735in}{1.469284in}}{\pgfqpoint{2.018125in}{1.458685in}}{\pgfqpoint{2.025939in}{1.450871in}}%
\pgfpathcurveto{\pgfqpoint{2.033753in}{1.443057in}}{\pgfqpoint{2.044352in}{1.438667in}}{\pgfqpoint{2.055402in}{1.438667in}}%
\pgfpathclose%
\pgfusepath{stroke,fill}%
\end{pgfscope}%
\begin{pgfscope}%
\pgfpathrectangle{\pgfqpoint{0.375000in}{0.330000in}}{\pgfqpoint{2.325000in}{2.310000in}}%
\pgfusepath{clip}%
\pgfsetbuttcap%
\pgfsetroundjoin%
\definecolor{currentfill}{rgb}{0.000000,0.000000,0.000000}%
\pgfsetfillcolor{currentfill}%
\pgfsetlinewidth{1.003750pt}%
\definecolor{currentstroke}{rgb}{0.000000,0.000000,0.000000}%
\pgfsetstrokecolor{currentstroke}%
\pgfsetdash{}{0pt}%
\pgfpathmoveto{\pgfqpoint{2.055402in}{1.438667in}}%
\pgfpathcurveto{\pgfqpoint{2.066452in}{1.438667in}}{\pgfqpoint{2.077051in}{1.443057in}}{\pgfqpoint{2.084865in}{1.450871in}}%
\pgfpathcurveto{\pgfqpoint{2.092678in}{1.458685in}}{\pgfqpoint{2.097069in}{1.469284in}}{\pgfqpoint{2.097069in}{1.480334in}}%
\pgfpathcurveto{\pgfqpoint{2.097069in}{1.491384in}}{\pgfqpoint{2.092678in}{1.501983in}}{\pgfqpoint{2.084865in}{1.509796in}}%
\pgfpathcurveto{\pgfqpoint{2.077051in}{1.517610in}}{\pgfqpoint{2.066452in}{1.522000in}}{\pgfqpoint{2.055402in}{1.522000in}}%
\pgfpathcurveto{\pgfqpoint{2.044352in}{1.522000in}}{\pgfqpoint{2.033753in}{1.517610in}}{\pgfqpoint{2.025939in}{1.509796in}}%
\pgfpathcurveto{\pgfqpoint{2.018125in}{1.501983in}}{\pgfqpoint{2.013735in}{1.491384in}}{\pgfqpoint{2.013735in}{1.480334in}}%
\pgfpathcurveto{\pgfqpoint{2.013735in}{1.469284in}}{\pgfqpoint{2.018125in}{1.458685in}}{\pgfqpoint{2.025939in}{1.450871in}}%
\pgfpathcurveto{\pgfqpoint{2.033753in}{1.443057in}}{\pgfqpoint{2.044352in}{1.438667in}}{\pgfqpoint{2.055402in}{1.438667in}}%
\pgfpathclose%
\pgfusepath{stroke,fill}%
\end{pgfscope}%
\begin{pgfscope}%
\pgfpathrectangle{\pgfqpoint{0.375000in}{0.330000in}}{\pgfqpoint{2.325000in}{2.310000in}}%
\pgfusepath{clip}%
\pgfsetbuttcap%
\pgfsetroundjoin%
\definecolor{currentfill}{rgb}{0.000000,0.000000,0.000000}%
\pgfsetfillcolor{currentfill}%
\pgfsetlinewidth{1.003750pt}%
\definecolor{currentstroke}{rgb}{0.000000,0.000000,0.000000}%
\pgfsetstrokecolor{currentstroke}%
\pgfsetdash{}{0pt}%
\pgfpathmoveto{\pgfqpoint{2.055402in}{1.438667in}}%
\pgfpathcurveto{\pgfqpoint{2.066452in}{1.438667in}}{\pgfqpoint{2.077051in}{1.443057in}}{\pgfqpoint{2.084865in}{1.450871in}}%
\pgfpathcurveto{\pgfqpoint{2.092678in}{1.458685in}}{\pgfqpoint{2.097069in}{1.469284in}}{\pgfqpoint{2.097069in}{1.480334in}}%
\pgfpathcurveto{\pgfqpoint{2.097069in}{1.491384in}}{\pgfqpoint{2.092678in}{1.501983in}}{\pgfqpoint{2.084865in}{1.509796in}}%
\pgfpathcurveto{\pgfqpoint{2.077051in}{1.517610in}}{\pgfqpoint{2.066452in}{1.522000in}}{\pgfqpoint{2.055402in}{1.522000in}}%
\pgfpathcurveto{\pgfqpoint{2.044352in}{1.522000in}}{\pgfqpoint{2.033753in}{1.517610in}}{\pgfqpoint{2.025939in}{1.509796in}}%
\pgfpathcurveto{\pgfqpoint{2.018125in}{1.501983in}}{\pgfqpoint{2.013735in}{1.491384in}}{\pgfqpoint{2.013735in}{1.480334in}}%
\pgfpathcurveto{\pgfqpoint{2.013735in}{1.469284in}}{\pgfqpoint{2.018125in}{1.458685in}}{\pgfqpoint{2.025939in}{1.450871in}}%
\pgfpathcurveto{\pgfqpoint{2.033753in}{1.443057in}}{\pgfqpoint{2.044352in}{1.438667in}}{\pgfqpoint{2.055402in}{1.438667in}}%
\pgfpathclose%
\pgfusepath{stroke,fill}%
\end{pgfscope}%
\begin{pgfscope}%
\pgfpathrectangle{\pgfqpoint{0.375000in}{0.330000in}}{\pgfqpoint{2.325000in}{2.310000in}}%
\pgfusepath{clip}%
\pgfsetbuttcap%
\pgfsetroundjoin%
\definecolor{currentfill}{rgb}{0.000000,0.000000,0.000000}%
\pgfsetfillcolor{currentfill}%
\pgfsetlinewidth{1.003750pt}%
\definecolor{currentstroke}{rgb}{0.000000,0.000000,0.000000}%
\pgfsetstrokecolor{currentstroke}%
\pgfsetdash{}{0pt}%
\pgfpathmoveto{\pgfqpoint{2.055402in}{1.438667in}}%
\pgfpathcurveto{\pgfqpoint{2.066452in}{1.438667in}}{\pgfqpoint{2.077051in}{1.443057in}}{\pgfqpoint{2.084865in}{1.450871in}}%
\pgfpathcurveto{\pgfqpoint{2.092678in}{1.458685in}}{\pgfqpoint{2.097069in}{1.469284in}}{\pgfqpoint{2.097069in}{1.480334in}}%
\pgfpathcurveto{\pgfqpoint{2.097069in}{1.491384in}}{\pgfqpoint{2.092678in}{1.501983in}}{\pgfqpoint{2.084865in}{1.509796in}}%
\pgfpathcurveto{\pgfqpoint{2.077051in}{1.517610in}}{\pgfqpoint{2.066452in}{1.522000in}}{\pgfqpoint{2.055402in}{1.522000in}}%
\pgfpathcurveto{\pgfqpoint{2.044352in}{1.522000in}}{\pgfqpoint{2.033753in}{1.517610in}}{\pgfqpoint{2.025939in}{1.509796in}}%
\pgfpathcurveto{\pgfqpoint{2.018125in}{1.501983in}}{\pgfqpoint{2.013735in}{1.491384in}}{\pgfqpoint{2.013735in}{1.480334in}}%
\pgfpathcurveto{\pgfqpoint{2.013735in}{1.469284in}}{\pgfqpoint{2.018125in}{1.458685in}}{\pgfqpoint{2.025939in}{1.450871in}}%
\pgfpathcurveto{\pgfqpoint{2.033753in}{1.443057in}}{\pgfqpoint{2.044352in}{1.438667in}}{\pgfqpoint{2.055402in}{1.438667in}}%
\pgfpathclose%
\pgfusepath{stroke,fill}%
\end{pgfscope}%
\begin{pgfscope}%
\pgfpathrectangle{\pgfqpoint{0.375000in}{0.330000in}}{\pgfqpoint{2.325000in}{2.310000in}}%
\pgfusepath{clip}%
\pgfsetbuttcap%
\pgfsetroundjoin%
\definecolor{currentfill}{rgb}{0.000000,0.000000,0.000000}%
\pgfsetfillcolor{currentfill}%
\pgfsetlinewidth{1.003750pt}%
\definecolor{currentstroke}{rgb}{0.000000,0.000000,0.000000}%
\pgfsetstrokecolor{currentstroke}%
\pgfsetdash{}{0pt}%
\pgfpathmoveto{\pgfqpoint{2.055402in}{1.438667in}}%
\pgfpathcurveto{\pgfqpoint{2.066452in}{1.438667in}}{\pgfqpoint{2.077051in}{1.443057in}}{\pgfqpoint{2.084865in}{1.450871in}}%
\pgfpathcurveto{\pgfqpoint{2.092678in}{1.458685in}}{\pgfqpoint{2.097069in}{1.469284in}}{\pgfqpoint{2.097069in}{1.480334in}}%
\pgfpathcurveto{\pgfqpoint{2.097069in}{1.491384in}}{\pgfqpoint{2.092678in}{1.501983in}}{\pgfqpoint{2.084865in}{1.509796in}}%
\pgfpathcurveto{\pgfqpoint{2.077051in}{1.517610in}}{\pgfqpoint{2.066452in}{1.522000in}}{\pgfqpoint{2.055402in}{1.522000in}}%
\pgfpathcurveto{\pgfqpoint{2.044352in}{1.522000in}}{\pgfqpoint{2.033753in}{1.517610in}}{\pgfqpoint{2.025939in}{1.509796in}}%
\pgfpathcurveto{\pgfqpoint{2.018125in}{1.501983in}}{\pgfqpoint{2.013735in}{1.491384in}}{\pgfqpoint{2.013735in}{1.480334in}}%
\pgfpathcurveto{\pgfqpoint{2.013735in}{1.469284in}}{\pgfqpoint{2.018125in}{1.458685in}}{\pgfqpoint{2.025939in}{1.450871in}}%
\pgfpathcurveto{\pgfqpoint{2.033753in}{1.443057in}}{\pgfqpoint{2.044352in}{1.438667in}}{\pgfqpoint{2.055402in}{1.438667in}}%
\pgfpathclose%
\pgfusepath{stroke,fill}%
\end{pgfscope}%
\begin{pgfscope}%
\pgfpathrectangle{\pgfqpoint{0.375000in}{0.330000in}}{\pgfqpoint{2.325000in}{2.310000in}}%
\pgfusepath{clip}%
\pgfsetbuttcap%
\pgfsetroundjoin%
\definecolor{currentfill}{rgb}{0.000000,0.000000,0.000000}%
\pgfsetfillcolor{currentfill}%
\pgfsetlinewidth{1.003750pt}%
\definecolor{currentstroke}{rgb}{0.000000,0.000000,0.000000}%
\pgfsetstrokecolor{currentstroke}%
\pgfsetdash{}{0pt}%
\pgfpathmoveto{\pgfqpoint{2.055402in}{1.438667in}}%
\pgfpathcurveto{\pgfqpoint{2.066452in}{1.438667in}}{\pgfqpoint{2.077051in}{1.443057in}}{\pgfqpoint{2.084865in}{1.450871in}}%
\pgfpathcurveto{\pgfqpoint{2.092678in}{1.458685in}}{\pgfqpoint{2.097069in}{1.469284in}}{\pgfqpoint{2.097069in}{1.480334in}}%
\pgfpathcurveto{\pgfqpoint{2.097069in}{1.491384in}}{\pgfqpoint{2.092678in}{1.501983in}}{\pgfqpoint{2.084865in}{1.509796in}}%
\pgfpathcurveto{\pgfqpoint{2.077051in}{1.517610in}}{\pgfqpoint{2.066452in}{1.522000in}}{\pgfqpoint{2.055402in}{1.522000in}}%
\pgfpathcurveto{\pgfqpoint{2.044352in}{1.522000in}}{\pgfqpoint{2.033753in}{1.517610in}}{\pgfqpoint{2.025939in}{1.509796in}}%
\pgfpathcurveto{\pgfqpoint{2.018125in}{1.501983in}}{\pgfqpoint{2.013735in}{1.491384in}}{\pgfqpoint{2.013735in}{1.480334in}}%
\pgfpathcurveto{\pgfqpoint{2.013735in}{1.469284in}}{\pgfqpoint{2.018125in}{1.458685in}}{\pgfqpoint{2.025939in}{1.450871in}}%
\pgfpathcurveto{\pgfqpoint{2.033753in}{1.443057in}}{\pgfqpoint{2.044352in}{1.438667in}}{\pgfqpoint{2.055402in}{1.438667in}}%
\pgfpathclose%
\pgfusepath{stroke,fill}%
\end{pgfscope}%
\begin{pgfscope}%
\pgfpathrectangle{\pgfqpoint{0.375000in}{0.330000in}}{\pgfqpoint{2.325000in}{2.310000in}}%
\pgfusepath{clip}%
\pgfsetbuttcap%
\pgfsetroundjoin%
\definecolor{currentfill}{rgb}{0.000000,0.000000,0.000000}%
\pgfsetfillcolor{currentfill}%
\pgfsetlinewidth{1.003750pt}%
\definecolor{currentstroke}{rgb}{0.000000,0.000000,0.000000}%
\pgfsetstrokecolor{currentstroke}%
\pgfsetdash{}{0pt}%
\pgfpathmoveto{\pgfqpoint{2.055402in}{1.438667in}}%
\pgfpathcurveto{\pgfqpoint{2.066452in}{1.438667in}}{\pgfqpoint{2.077051in}{1.443057in}}{\pgfqpoint{2.084865in}{1.450871in}}%
\pgfpathcurveto{\pgfqpoint{2.092678in}{1.458685in}}{\pgfqpoint{2.097069in}{1.469284in}}{\pgfqpoint{2.097069in}{1.480334in}}%
\pgfpathcurveto{\pgfqpoint{2.097069in}{1.491384in}}{\pgfqpoint{2.092678in}{1.501983in}}{\pgfqpoint{2.084865in}{1.509796in}}%
\pgfpathcurveto{\pgfqpoint{2.077051in}{1.517610in}}{\pgfqpoint{2.066452in}{1.522000in}}{\pgfqpoint{2.055402in}{1.522000in}}%
\pgfpathcurveto{\pgfqpoint{2.044352in}{1.522000in}}{\pgfqpoint{2.033753in}{1.517610in}}{\pgfqpoint{2.025939in}{1.509796in}}%
\pgfpathcurveto{\pgfqpoint{2.018125in}{1.501983in}}{\pgfqpoint{2.013735in}{1.491384in}}{\pgfqpoint{2.013735in}{1.480334in}}%
\pgfpathcurveto{\pgfqpoint{2.013735in}{1.469284in}}{\pgfqpoint{2.018125in}{1.458685in}}{\pgfqpoint{2.025939in}{1.450871in}}%
\pgfpathcurveto{\pgfqpoint{2.033753in}{1.443057in}}{\pgfqpoint{2.044352in}{1.438667in}}{\pgfqpoint{2.055402in}{1.438667in}}%
\pgfpathclose%
\pgfusepath{stroke,fill}%
\end{pgfscope}%
\begin{pgfscope}%
\pgfpathrectangle{\pgfqpoint{0.375000in}{0.330000in}}{\pgfqpoint{2.325000in}{2.310000in}}%
\pgfusepath{clip}%
\pgfsetbuttcap%
\pgfsetroundjoin%
\definecolor{currentfill}{rgb}{0.000000,0.000000,0.000000}%
\pgfsetfillcolor{currentfill}%
\pgfsetlinewidth{1.003750pt}%
\definecolor{currentstroke}{rgb}{0.000000,0.000000,0.000000}%
\pgfsetstrokecolor{currentstroke}%
\pgfsetdash{}{0pt}%
\pgfpathmoveto{\pgfqpoint{2.055402in}{1.438667in}}%
\pgfpathcurveto{\pgfqpoint{2.066452in}{1.438667in}}{\pgfqpoint{2.077051in}{1.443057in}}{\pgfqpoint{2.084865in}{1.450871in}}%
\pgfpathcurveto{\pgfqpoint{2.092678in}{1.458685in}}{\pgfqpoint{2.097069in}{1.469284in}}{\pgfqpoint{2.097069in}{1.480334in}}%
\pgfpathcurveto{\pgfqpoint{2.097069in}{1.491384in}}{\pgfqpoint{2.092678in}{1.501983in}}{\pgfqpoint{2.084865in}{1.509796in}}%
\pgfpathcurveto{\pgfqpoint{2.077051in}{1.517610in}}{\pgfqpoint{2.066452in}{1.522000in}}{\pgfqpoint{2.055402in}{1.522000in}}%
\pgfpathcurveto{\pgfqpoint{2.044352in}{1.522000in}}{\pgfqpoint{2.033753in}{1.517610in}}{\pgfqpoint{2.025939in}{1.509796in}}%
\pgfpathcurveto{\pgfqpoint{2.018125in}{1.501983in}}{\pgfqpoint{2.013735in}{1.491384in}}{\pgfqpoint{2.013735in}{1.480334in}}%
\pgfpathcurveto{\pgfqpoint{2.013735in}{1.469284in}}{\pgfqpoint{2.018125in}{1.458685in}}{\pgfqpoint{2.025939in}{1.450871in}}%
\pgfpathcurveto{\pgfqpoint{2.033753in}{1.443057in}}{\pgfqpoint{2.044352in}{1.438667in}}{\pgfqpoint{2.055402in}{1.438667in}}%
\pgfpathclose%
\pgfusepath{stroke,fill}%
\end{pgfscope}%
\begin{pgfscope}%
\pgfpathrectangle{\pgfqpoint{0.375000in}{0.330000in}}{\pgfqpoint{2.325000in}{2.310000in}}%
\pgfusepath{clip}%
\pgfsetbuttcap%
\pgfsetroundjoin%
\definecolor{currentfill}{rgb}{0.000000,0.000000,0.000000}%
\pgfsetfillcolor{currentfill}%
\pgfsetlinewidth{1.003750pt}%
\definecolor{currentstroke}{rgb}{0.000000,0.000000,0.000000}%
\pgfsetstrokecolor{currentstroke}%
\pgfsetdash{}{0pt}%
\pgfpathmoveto{\pgfqpoint{2.055402in}{1.438667in}}%
\pgfpathcurveto{\pgfqpoint{2.066452in}{1.438667in}}{\pgfqpoint{2.077051in}{1.443057in}}{\pgfqpoint{2.084865in}{1.450871in}}%
\pgfpathcurveto{\pgfqpoint{2.092678in}{1.458685in}}{\pgfqpoint{2.097069in}{1.469284in}}{\pgfqpoint{2.097069in}{1.480334in}}%
\pgfpathcurveto{\pgfqpoint{2.097069in}{1.491384in}}{\pgfqpoint{2.092678in}{1.501983in}}{\pgfqpoint{2.084865in}{1.509796in}}%
\pgfpathcurveto{\pgfqpoint{2.077051in}{1.517610in}}{\pgfqpoint{2.066452in}{1.522000in}}{\pgfqpoint{2.055402in}{1.522000in}}%
\pgfpathcurveto{\pgfqpoint{2.044352in}{1.522000in}}{\pgfqpoint{2.033753in}{1.517610in}}{\pgfqpoint{2.025939in}{1.509796in}}%
\pgfpathcurveto{\pgfqpoint{2.018125in}{1.501983in}}{\pgfqpoint{2.013735in}{1.491384in}}{\pgfqpoint{2.013735in}{1.480334in}}%
\pgfpathcurveto{\pgfqpoint{2.013735in}{1.469284in}}{\pgfqpoint{2.018125in}{1.458685in}}{\pgfqpoint{2.025939in}{1.450871in}}%
\pgfpathcurveto{\pgfqpoint{2.033753in}{1.443057in}}{\pgfqpoint{2.044352in}{1.438667in}}{\pgfqpoint{2.055402in}{1.438667in}}%
\pgfpathclose%
\pgfusepath{stroke,fill}%
\end{pgfscope}%
\begin{pgfscope}%
\pgfpathrectangle{\pgfqpoint{0.375000in}{0.330000in}}{\pgfqpoint{2.325000in}{2.310000in}}%
\pgfusepath{clip}%
\pgfsetbuttcap%
\pgfsetroundjoin%
\definecolor{currentfill}{rgb}{0.000000,0.000000,0.000000}%
\pgfsetfillcolor{currentfill}%
\pgfsetlinewidth{1.003750pt}%
\definecolor{currentstroke}{rgb}{0.000000,0.000000,0.000000}%
\pgfsetstrokecolor{currentstroke}%
\pgfsetdash{}{0pt}%
\pgfpathmoveto{\pgfqpoint{2.055402in}{1.438667in}}%
\pgfpathcurveto{\pgfqpoint{2.066452in}{1.438667in}}{\pgfqpoint{2.077051in}{1.443057in}}{\pgfqpoint{2.084865in}{1.450871in}}%
\pgfpathcurveto{\pgfqpoint{2.092678in}{1.458685in}}{\pgfqpoint{2.097069in}{1.469284in}}{\pgfqpoint{2.097069in}{1.480334in}}%
\pgfpathcurveto{\pgfqpoint{2.097069in}{1.491384in}}{\pgfqpoint{2.092678in}{1.501983in}}{\pgfqpoint{2.084865in}{1.509796in}}%
\pgfpathcurveto{\pgfqpoint{2.077051in}{1.517610in}}{\pgfqpoint{2.066452in}{1.522000in}}{\pgfqpoint{2.055402in}{1.522000in}}%
\pgfpathcurveto{\pgfqpoint{2.044352in}{1.522000in}}{\pgfqpoint{2.033753in}{1.517610in}}{\pgfqpoint{2.025939in}{1.509796in}}%
\pgfpathcurveto{\pgfqpoint{2.018125in}{1.501983in}}{\pgfqpoint{2.013735in}{1.491384in}}{\pgfqpoint{2.013735in}{1.480334in}}%
\pgfpathcurveto{\pgfqpoint{2.013735in}{1.469284in}}{\pgfqpoint{2.018125in}{1.458685in}}{\pgfqpoint{2.025939in}{1.450871in}}%
\pgfpathcurveto{\pgfqpoint{2.033753in}{1.443057in}}{\pgfqpoint{2.044352in}{1.438667in}}{\pgfqpoint{2.055402in}{1.438667in}}%
\pgfpathclose%
\pgfusepath{stroke,fill}%
\end{pgfscope}%
\begin{pgfscope}%
\pgfpathrectangle{\pgfqpoint{0.375000in}{0.330000in}}{\pgfqpoint{2.325000in}{2.310000in}}%
\pgfusepath{clip}%
\pgfsetbuttcap%
\pgfsetroundjoin%
\definecolor{currentfill}{rgb}{0.000000,0.000000,0.000000}%
\pgfsetfillcolor{currentfill}%
\pgfsetlinewidth{1.003750pt}%
\definecolor{currentstroke}{rgb}{0.000000,0.000000,0.000000}%
\pgfsetstrokecolor{currentstroke}%
\pgfsetdash{}{0pt}%
\pgfpathmoveto{\pgfqpoint{2.055402in}{1.438667in}}%
\pgfpathcurveto{\pgfqpoint{2.066452in}{1.438667in}}{\pgfqpoint{2.077051in}{1.443057in}}{\pgfqpoint{2.084865in}{1.450871in}}%
\pgfpathcurveto{\pgfqpoint{2.092678in}{1.458685in}}{\pgfqpoint{2.097069in}{1.469284in}}{\pgfqpoint{2.097069in}{1.480334in}}%
\pgfpathcurveto{\pgfqpoint{2.097069in}{1.491384in}}{\pgfqpoint{2.092678in}{1.501983in}}{\pgfqpoint{2.084865in}{1.509796in}}%
\pgfpathcurveto{\pgfqpoint{2.077051in}{1.517610in}}{\pgfqpoint{2.066452in}{1.522000in}}{\pgfqpoint{2.055402in}{1.522000in}}%
\pgfpathcurveto{\pgfqpoint{2.044352in}{1.522000in}}{\pgfqpoint{2.033753in}{1.517610in}}{\pgfqpoint{2.025939in}{1.509796in}}%
\pgfpathcurveto{\pgfqpoint{2.018125in}{1.501983in}}{\pgfqpoint{2.013735in}{1.491384in}}{\pgfqpoint{2.013735in}{1.480334in}}%
\pgfpathcurveto{\pgfqpoint{2.013735in}{1.469284in}}{\pgfqpoint{2.018125in}{1.458685in}}{\pgfqpoint{2.025939in}{1.450871in}}%
\pgfpathcurveto{\pgfqpoint{2.033753in}{1.443057in}}{\pgfqpoint{2.044352in}{1.438667in}}{\pgfqpoint{2.055402in}{1.438667in}}%
\pgfpathclose%
\pgfusepath{stroke,fill}%
\end{pgfscope}%
\begin{pgfscope}%
\pgfpathrectangle{\pgfqpoint{0.375000in}{0.330000in}}{\pgfqpoint{2.325000in}{2.310000in}}%
\pgfusepath{clip}%
\pgfsetbuttcap%
\pgfsetroundjoin%
\definecolor{currentfill}{rgb}{0.000000,0.000000,0.000000}%
\pgfsetfillcolor{currentfill}%
\pgfsetlinewidth{1.003750pt}%
\definecolor{currentstroke}{rgb}{0.000000,0.000000,0.000000}%
\pgfsetstrokecolor{currentstroke}%
\pgfsetdash{}{0pt}%
\pgfpathmoveto{\pgfqpoint{2.055402in}{1.438667in}}%
\pgfpathcurveto{\pgfqpoint{2.066452in}{1.438667in}}{\pgfqpoint{2.077051in}{1.443057in}}{\pgfqpoint{2.084865in}{1.450871in}}%
\pgfpathcurveto{\pgfqpoint{2.092678in}{1.458685in}}{\pgfqpoint{2.097069in}{1.469284in}}{\pgfqpoint{2.097069in}{1.480334in}}%
\pgfpathcurveto{\pgfqpoint{2.097069in}{1.491384in}}{\pgfqpoint{2.092678in}{1.501983in}}{\pgfqpoint{2.084865in}{1.509796in}}%
\pgfpathcurveto{\pgfqpoint{2.077051in}{1.517610in}}{\pgfqpoint{2.066452in}{1.522000in}}{\pgfqpoint{2.055402in}{1.522000in}}%
\pgfpathcurveto{\pgfqpoint{2.044352in}{1.522000in}}{\pgfqpoint{2.033753in}{1.517610in}}{\pgfqpoint{2.025939in}{1.509796in}}%
\pgfpathcurveto{\pgfqpoint{2.018125in}{1.501983in}}{\pgfqpoint{2.013735in}{1.491384in}}{\pgfqpoint{2.013735in}{1.480334in}}%
\pgfpathcurveto{\pgfqpoint{2.013735in}{1.469284in}}{\pgfqpoint{2.018125in}{1.458685in}}{\pgfqpoint{2.025939in}{1.450871in}}%
\pgfpathcurveto{\pgfqpoint{2.033753in}{1.443057in}}{\pgfqpoint{2.044352in}{1.438667in}}{\pgfqpoint{2.055402in}{1.438667in}}%
\pgfpathclose%
\pgfusepath{stroke,fill}%
\end{pgfscope}%
\begin{pgfscope}%
\pgfpathrectangle{\pgfqpoint{0.375000in}{0.330000in}}{\pgfqpoint{2.325000in}{2.310000in}}%
\pgfusepath{clip}%
\pgfsetbuttcap%
\pgfsetroundjoin%
\definecolor{currentfill}{rgb}{0.000000,0.000000,0.000000}%
\pgfsetfillcolor{currentfill}%
\pgfsetlinewidth{1.003750pt}%
\definecolor{currentstroke}{rgb}{0.000000,0.000000,0.000000}%
\pgfsetstrokecolor{currentstroke}%
\pgfsetdash{}{0pt}%
\pgfpathmoveto{\pgfqpoint{2.055402in}{1.438667in}}%
\pgfpathcurveto{\pgfqpoint{2.066452in}{1.438667in}}{\pgfqpoint{2.077051in}{1.443057in}}{\pgfqpoint{2.084865in}{1.450871in}}%
\pgfpathcurveto{\pgfqpoint{2.092678in}{1.458685in}}{\pgfqpoint{2.097069in}{1.469284in}}{\pgfqpoint{2.097069in}{1.480334in}}%
\pgfpathcurveto{\pgfqpoint{2.097069in}{1.491384in}}{\pgfqpoint{2.092678in}{1.501983in}}{\pgfqpoint{2.084865in}{1.509796in}}%
\pgfpathcurveto{\pgfqpoint{2.077051in}{1.517610in}}{\pgfqpoint{2.066452in}{1.522000in}}{\pgfqpoint{2.055402in}{1.522000in}}%
\pgfpathcurveto{\pgfqpoint{2.044352in}{1.522000in}}{\pgfqpoint{2.033753in}{1.517610in}}{\pgfqpoint{2.025939in}{1.509796in}}%
\pgfpathcurveto{\pgfqpoint{2.018125in}{1.501983in}}{\pgfqpoint{2.013735in}{1.491384in}}{\pgfqpoint{2.013735in}{1.480334in}}%
\pgfpathcurveto{\pgfqpoint{2.013735in}{1.469284in}}{\pgfqpoint{2.018125in}{1.458685in}}{\pgfqpoint{2.025939in}{1.450871in}}%
\pgfpathcurveto{\pgfqpoint{2.033753in}{1.443057in}}{\pgfqpoint{2.044352in}{1.438667in}}{\pgfqpoint{2.055402in}{1.438667in}}%
\pgfpathclose%
\pgfusepath{stroke,fill}%
\end{pgfscope}%
\begin{pgfscope}%
\pgfpathrectangle{\pgfqpoint{0.375000in}{0.330000in}}{\pgfqpoint{2.325000in}{2.310000in}}%
\pgfusepath{clip}%
\pgfsetbuttcap%
\pgfsetroundjoin%
\definecolor{currentfill}{rgb}{0.000000,0.000000,0.000000}%
\pgfsetfillcolor{currentfill}%
\pgfsetlinewidth{1.003750pt}%
\definecolor{currentstroke}{rgb}{0.000000,0.000000,0.000000}%
\pgfsetstrokecolor{currentstroke}%
\pgfsetdash{}{0pt}%
\pgfpathmoveto{\pgfqpoint{2.055402in}{1.438667in}}%
\pgfpathcurveto{\pgfqpoint{2.066452in}{1.438667in}}{\pgfqpoint{2.077051in}{1.443057in}}{\pgfqpoint{2.084865in}{1.450871in}}%
\pgfpathcurveto{\pgfqpoint{2.092678in}{1.458685in}}{\pgfqpoint{2.097069in}{1.469284in}}{\pgfqpoint{2.097069in}{1.480334in}}%
\pgfpathcurveto{\pgfqpoint{2.097069in}{1.491384in}}{\pgfqpoint{2.092678in}{1.501983in}}{\pgfqpoint{2.084865in}{1.509796in}}%
\pgfpathcurveto{\pgfqpoint{2.077051in}{1.517610in}}{\pgfqpoint{2.066452in}{1.522000in}}{\pgfqpoint{2.055402in}{1.522000in}}%
\pgfpathcurveto{\pgfqpoint{2.044352in}{1.522000in}}{\pgfqpoint{2.033753in}{1.517610in}}{\pgfqpoint{2.025939in}{1.509796in}}%
\pgfpathcurveto{\pgfqpoint{2.018125in}{1.501983in}}{\pgfqpoint{2.013735in}{1.491384in}}{\pgfqpoint{2.013735in}{1.480334in}}%
\pgfpathcurveto{\pgfqpoint{2.013735in}{1.469284in}}{\pgfqpoint{2.018125in}{1.458685in}}{\pgfqpoint{2.025939in}{1.450871in}}%
\pgfpathcurveto{\pgfqpoint{2.033753in}{1.443057in}}{\pgfqpoint{2.044352in}{1.438667in}}{\pgfqpoint{2.055402in}{1.438667in}}%
\pgfpathclose%
\pgfusepath{stroke,fill}%
\end{pgfscope}%
\begin{pgfscope}%
\pgfpathrectangle{\pgfqpoint{0.375000in}{0.330000in}}{\pgfqpoint{2.325000in}{2.310000in}}%
\pgfusepath{clip}%
\pgfsetbuttcap%
\pgfsetroundjoin%
\definecolor{currentfill}{rgb}{0.000000,0.000000,0.000000}%
\pgfsetfillcolor{currentfill}%
\pgfsetlinewidth{1.003750pt}%
\definecolor{currentstroke}{rgb}{0.000000,0.000000,0.000000}%
\pgfsetstrokecolor{currentstroke}%
\pgfsetdash{}{0pt}%
\pgfpathmoveto{\pgfqpoint{2.055402in}{1.438667in}}%
\pgfpathcurveto{\pgfqpoint{2.066452in}{1.438667in}}{\pgfqpoint{2.077051in}{1.443057in}}{\pgfqpoint{2.084865in}{1.450871in}}%
\pgfpathcurveto{\pgfqpoint{2.092678in}{1.458685in}}{\pgfqpoint{2.097069in}{1.469284in}}{\pgfqpoint{2.097069in}{1.480334in}}%
\pgfpathcurveto{\pgfqpoint{2.097069in}{1.491384in}}{\pgfqpoint{2.092678in}{1.501983in}}{\pgfqpoint{2.084865in}{1.509796in}}%
\pgfpathcurveto{\pgfqpoint{2.077051in}{1.517610in}}{\pgfqpoint{2.066452in}{1.522000in}}{\pgfqpoint{2.055402in}{1.522000in}}%
\pgfpathcurveto{\pgfqpoint{2.044352in}{1.522000in}}{\pgfqpoint{2.033753in}{1.517610in}}{\pgfqpoint{2.025939in}{1.509796in}}%
\pgfpathcurveto{\pgfqpoint{2.018125in}{1.501983in}}{\pgfqpoint{2.013735in}{1.491384in}}{\pgfqpoint{2.013735in}{1.480334in}}%
\pgfpathcurveto{\pgfqpoint{2.013735in}{1.469284in}}{\pgfqpoint{2.018125in}{1.458685in}}{\pgfqpoint{2.025939in}{1.450871in}}%
\pgfpathcurveto{\pgfqpoint{2.033753in}{1.443057in}}{\pgfqpoint{2.044352in}{1.438667in}}{\pgfqpoint{2.055402in}{1.438667in}}%
\pgfpathclose%
\pgfusepath{stroke,fill}%
\end{pgfscope}%
\begin{pgfscope}%
\pgfpathrectangle{\pgfqpoint{0.375000in}{0.330000in}}{\pgfqpoint{2.325000in}{2.310000in}}%
\pgfusepath{clip}%
\pgfsetbuttcap%
\pgfsetroundjoin%
\definecolor{currentfill}{rgb}{0.000000,0.000000,0.000000}%
\pgfsetfillcolor{currentfill}%
\pgfsetlinewidth{1.003750pt}%
\definecolor{currentstroke}{rgb}{0.000000,0.000000,0.000000}%
\pgfsetstrokecolor{currentstroke}%
\pgfsetdash{}{0pt}%
\pgfpathmoveto{\pgfqpoint{2.055402in}{1.438667in}}%
\pgfpathcurveto{\pgfqpoint{2.066452in}{1.438667in}}{\pgfqpoint{2.077051in}{1.443057in}}{\pgfqpoint{2.084865in}{1.450871in}}%
\pgfpathcurveto{\pgfqpoint{2.092678in}{1.458685in}}{\pgfqpoint{2.097069in}{1.469284in}}{\pgfqpoint{2.097069in}{1.480334in}}%
\pgfpathcurveto{\pgfqpoint{2.097069in}{1.491384in}}{\pgfqpoint{2.092678in}{1.501983in}}{\pgfqpoint{2.084865in}{1.509796in}}%
\pgfpathcurveto{\pgfqpoint{2.077051in}{1.517610in}}{\pgfqpoint{2.066452in}{1.522000in}}{\pgfqpoint{2.055402in}{1.522000in}}%
\pgfpathcurveto{\pgfqpoint{2.044352in}{1.522000in}}{\pgfqpoint{2.033753in}{1.517610in}}{\pgfqpoint{2.025939in}{1.509796in}}%
\pgfpathcurveto{\pgfqpoint{2.018125in}{1.501983in}}{\pgfqpoint{2.013735in}{1.491384in}}{\pgfqpoint{2.013735in}{1.480334in}}%
\pgfpathcurveto{\pgfqpoint{2.013735in}{1.469284in}}{\pgfqpoint{2.018125in}{1.458685in}}{\pgfqpoint{2.025939in}{1.450871in}}%
\pgfpathcurveto{\pgfqpoint{2.033753in}{1.443057in}}{\pgfqpoint{2.044352in}{1.438667in}}{\pgfqpoint{2.055402in}{1.438667in}}%
\pgfpathclose%
\pgfusepath{stroke,fill}%
\end{pgfscope}%
\begin{pgfscope}%
\pgfpathrectangle{\pgfqpoint{0.375000in}{0.330000in}}{\pgfqpoint{2.325000in}{2.310000in}}%
\pgfusepath{clip}%
\pgfsetbuttcap%
\pgfsetroundjoin%
\definecolor{currentfill}{rgb}{0.000000,0.000000,0.000000}%
\pgfsetfillcolor{currentfill}%
\pgfsetlinewidth{1.003750pt}%
\definecolor{currentstroke}{rgb}{0.000000,0.000000,0.000000}%
\pgfsetstrokecolor{currentstroke}%
\pgfsetdash{}{0pt}%
\pgfpathmoveto{\pgfqpoint{2.055402in}{1.438667in}}%
\pgfpathcurveto{\pgfqpoint{2.066452in}{1.438667in}}{\pgfqpoint{2.077051in}{1.443057in}}{\pgfqpoint{2.084865in}{1.450871in}}%
\pgfpathcurveto{\pgfqpoint{2.092678in}{1.458685in}}{\pgfqpoint{2.097069in}{1.469284in}}{\pgfqpoint{2.097069in}{1.480334in}}%
\pgfpathcurveto{\pgfqpoint{2.097069in}{1.491384in}}{\pgfqpoint{2.092678in}{1.501983in}}{\pgfqpoint{2.084865in}{1.509796in}}%
\pgfpathcurveto{\pgfqpoint{2.077051in}{1.517610in}}{\pgfqpoint{2.066452in}{1.522000in}}{\pgfqpoint{2.055402in}{1.522000in}}%
\pgfpathcurveto{\pgfqpoint{2.044352in}{1.522000in}}{\pgfqpoint{2.033753in}{1.517610in}}{\pgfqpoint{2.025939in}{1.509796in}}%
\pgfpathcurveto{\pgfqpoint{2.018125in}{1.501983in}}{\pgfqpoint{2.013735in}{1.491384in}}{\pgfqpoint{2.013735in}{1.480334in}}%
\pgfpathcurveto{\pgfqpoint{2.013735in}{1.469284in}}{\pgfqpoint{2.018125in}{1.458685in}}{\pgfqpoint{2.025939in}{1.450871in}}%
\pgfpathcurveto{\pgfqpoint{2.033753in}{1.443057in}}{\pgfqpoint{2.044352in}{1.438667in}}{\pgfqpoint{2.055402in}{1.438667in}}%
\pgfpathclose%
\pgfusepath{stroke,fill}%
\end{pgfscope}%
\begin{pgfscope}%
\pgfpathrectangle{\pgfqpoint{0.375000in}{0.330000in}}{\pgfqpoint{2.325000in}{2.310000in}}%
\pgfusepath{clip}%
\pgfsetbuttcap%
\pgfsetroundjoin%
\definecolor{currentfill}{rgb}{0.000000,0.000000,0.000000}%
\pgfsetfillcolor{currentfill}%
\pgfsetlinewidth{1.003750pt}%
\definecolor{currentstroke}{rgb}{0.000000,0.000000,0.000000}%
\pgfsetstrokecolor{currentstroke}%
\pgfsetdash{}{0pt}%
\pgfpathmoveto{\pgfqpoint{2.055402in}{1.438667in}}%
\pgfpathcurveto{\pgfqpoint{2.066452in}{1.438667in}}{\pgfqpoint{2.077051in}{1.443057in}}{\pgfqpoint{2.084865in}{1.450871in}}%
\pgfpathcurveto{\pgfqpoint{2.092678in}{1.458685in}}{\pgfqpoint{2.097069in}{1.469284in}}{\pgfqpoint{2.097069in}{1.480334in}}%
\pgfpathcurveto{\pgfqpoint{2.097069in}{1.491384in}}{\pgfqpoint{2.092678in}{1.501983in}}{\pgfqpoint{2.084865in}{1.509796in}}%
\pgfpathcurveto{\pgfqpoint{2.077051in}{1.517610in}}{\pgfqpoint{2.066452in}{1.522000in}}{\pgfqpoint{2.055402in}{1.522000in}}%
\pgfpathcurveto{\pgfqpoint{2.044352in}{1.522000in}}{\pgfqpoint{2.033753in}{1.517610in}}{\pgfqpoint{2.025939in}{1.509796in}}%
\pgfpathcurveto{\pgfqpoint{2.018125in}{1.501983in}}{\pgfqpoint{2.013735in}{1.491384in}}{\pgfqpoint{2.013735in}{1.480334in}}%
\pgfpathcurveto{\pgfqpoint{2.013735in}{1.469284in}}{\pgfqpoint{2.018125in}{1.458685in}}{\pgfqpoint{2.025939in}{1.450871in}}%
\pgfpathcurveto{\pgfqpoint{2.033753in}{1.443057in}}{\pgfqpoint{2.044352in}{1.438667in}}{\pgfqpoint{2.055402in}{1.438667in}}%
\pgfpathclose%
\pgfusepath{stroke,fill}%
\end{pgfscope}%
\begin{pgfscope}%
\pgfpathrectangle{\pgfqpoint{0.375000in}{0.330000in}}{\pgfqpoint{2.325000in}{2.310000in}}%
\pgfusepath{clip}%
\pgfsetbuttcap%
\pgfsetroundjoin%
\definecolor{currentfill}{rgb}{0.000000,0.000000,0.000000}%
\pgfsetfillcolor{currentfill}%
\pgfsetlinewidth{1.003750pt}%
\definecolor{currentstroke}{rgb}{0.000000,0.000000,0.000000}%
\pgfsetstrokecolor{currentstroke}%
\pgfsetdash{}{0pt}%
\pgfpathmoveto{\pgfqpoint{2.055402in}{1.438667in}}%
\pgfpathcurveto{\pgfqpoint{2.066452in}{1.438667in}}{\pgfqpoint{2.077051in}{1.443057in}}{\pgfqpoint{2.084865in}{1.450871in}}%
\pgfpathcurveto{\pgfqpoint{2.092678in}{1.458685in}}{\pgfqpoint{2.097069in}{1.469284in}}{\pgfqpoint{2.097069in}{1.480334in}}%
\pgfpathcurveto{\pgfqpoint{2.097069in}{1.491384in}}{\pgfqpoint{2.092678in}{1.501983in}}{\pgfqpoint{2.084865in}{1.509796in}}%
\pgfpathcurveto{\pgfqpoint{2.077051in}{1.517610in}}{\pgfqpoint{2.066452in}{1.522000in}}{\pgfqpoint{2.055402in}{1.522000in}}%
\pgfpathcurveto{\pgfqpoint{2.044352in}{1.522000in}}{\pgfqpoint{2.033753in}{1.517610in}}{\pgfqpoint{2.025939in}{1.509796in}}%
\pgfpathcurveto{\pgfqpoint{2.018125in}{1.501983in}}{\pgfqpoint{2.013735in}{1.491384in}}{\pgfqpoint{2.013735in}{1.480334in}}%
\pgfpathcurveto{\pgfqpoint{2.013735in}{1.469284in}}{\pgfqpoint{2.018125in}{1.458685in}}{\pgfqpoint{2.025939in}{1.450871in}}%
\pgfpathcurveto{\pgfqpoint{2.033753in}{1.443057in}}{\pgfqpoint{2.044352in}{1.438667in}}{\pgfqpoint{2.055402in}{1.438667in}}%
\pgfpathclose%
\pgfusepath{stroke,fill}%
\end{pgfscope}%
\begin{pgfscope}%
\pgfpathrectangle{\pgfqpoint{0.375000in}{0.330000in}}{\pgfqpoint{2.325000in}{2.310000in}}%
\pgfusepath{clip}%
\pgfsetbuttcap%
\pgfsetroundjoin%
\definecolor{currentfill}{rgb}{0.000000,0.000000,0.000000}%
\pgfsetfillcolor{currentfill}%
\pgfsetlinewidth{1.003750pt}%
\definecolor{currentstroke}{rgb}{0.000000,0.000000,0.000000}%
\pgfsetstrokecolor{currentstroke}%
\pgfsetdash{}{0pt}%
\pgfpathmoveto{\pgfqpoint{2.055402in}{1.438667in}}%
\pgfpathcurveto{\pgfqpoint{2.066452in}{1.438667in}}{\pgfqpoint{2.077051in}{1.443057in}}{\pgfqpoint{2.084865in}{1.450871in}}%
\pgfpathcurveto{\pgfqpoint{2.092678in}{1.458685in}}{\pgfqpoint{2.097069in}{1.469284in}}{\pgfqpoint{2.097069in}{1.480334in}}%
\pgfpathcurveto{\pgfqpoint{2.097069in}{1.491384in}}{\pgfqpoint{2.092678in}{1.501983in}}{\pgfqpoint{2.084865in}{1.509796in}}%
\pgfpathcurveto{\pgfqpoint{2.077051in}{1.517610in}}{\pgfqpoint{2.066452in}{1.522000in}}{\pgfqpoint{2.055402in}{1.522000in}}%
\pgfpathcurveto{\pgfqpoint{2.044352in}{1.522000in}}{\pgfqpoint{2.033753in}{1.517610in}}{\pgfqpoint{2.025939in}{1.509796in}}%
\pgfpathcurveto{\pgfqpoint{2.018125in}{1.501983in}}{\pgfqpoint{2.013735in}{1.491384in}}{\pgfqpoint{2.013735in}{1.480334in}}%
\pgfpathcurveto{\pgfqpoint{2.013735in}{1.469284in}}{\pgfqpoint{2.018125in}{1.458685in}}{\pgfqpoint{2.025939in}{1.450871in}}%
\pgfpathcurveto{\pgfqpoint{2.033753in}{1.443057in}}{\pgfqpoint{2.044352in}{1.438667in}}{\pgfqpoint{2.055402in}{1.438667in}}%
\pgfpathclose%
\pgfusepath{stroke,fill}%
\end{pgfscope}%
\begin{pgfscope}%
\pgfpathrectangle{\pgfqpoint{0.375000in}{0.330000in}}{\pgfqpoint{2.325000in}{2.310000in}}%
\pgfusepath{clip}%
\pgfsetbuttcap%
\pgfsetroundjoin%
\definecolor{currentfill}{rgb}{0.000000,0.000000,0.000000}%
\pgfsetfillcolor{currentfill}%
\pgfsetlinewidth{1.003750pt}%
\definecolor{currentstroke}{rgb}{0.000000,0.000000,0.000000}%
\pgfsetstrokecolor{currentstroke}%
\pgfsetdash{}{0pt}%
\pgfpathmoveto{\pgfqpoint{2.055402in}{1.438667in}}%
\pgfpathcurveto{\pgfqpoint{2.066452in}{1.438667in}}{\pgfqpoint{2.077051in}{1.443057in}}{\pgfqpoint{2.084865in}{1.450871in}}%
\pgfpathcurveto{\pgfqpoint{2.092678in}{1.458685in}}{\pgfqpoint{2.097069in}{1.469284in}}{\pgfqpoint{2.097069in}{1.480334in}}%
\pgfpathcurveto{\pgfqpoint{2.097069in}{1.491384in}}{\pgfqpoint{2.092678in}{1.501983in}}{\pgfqpoint{2.084865in}{1.509796in}}%
\pgfpathcurveto{\pgfqpoint{2.077051in}{1.517610in}}{\pgfqpoint{2.066452in}{1.522000in}}{\pgfqpoint{2.055402in}{1.522000in}}%
\pgfpathcurveto{\pgfqpoint{2.044352in}{1.522000in}}{\pgfqpoint{2.033753in}{1.517610in}}{\pgfqpoint{2.025939in}{1.509796in}}%
\pgfpathcurveto{\pgfqpoint{2.018125in}{1.501983in}}{\pgfqpoint{2.013735in}{1.491384in}}{\pgfqpoint{2.013735in}{1.480334in}}%
\pgfpathcurveto{\pgfqpoint{2.013735in}{1.469284in}}{\pgfqpoint{2.018125in}{1.458685in}}{\pgfqpoint{2.025939in}{1.450871in}}%
\pgfpathcurveto{\pgfqpoint{2.033753in}{1.443057in}}{\pgfqpoint{2.044352in}{1.438667in}}{\pgfqpoint{2.055402in}{1.438667in}}%
\pgfpathclose%
\pgfusepath{stroke,fill}%
\end{pgfscope}%
\begin{pgfscope}%
\pgfpathrectangle{\pgfqpoint{0.375000in}{0.330000in}}{\pgfqpoint{2.325000in}{2.310000in}}%
\pgfusepath{clip}%
\pgfsetbuttcap%
\pgfsetroundjoin%
\definecolor{currentfill}{rgb}{0.000000,0.000000,0.000000}%
\pgfsetfillcolor{currentfill}%
\pgfsetlinewidth{1.003750pt}%
\definecolor{currentstroke}{rgb}{0.000000,0.000000,0.000000}%
\pgfsetstrokecolor{currentstroke}%
\pgfsetdash{}{0pt}%
\pgfpathmoveto{\pgfqpoint{2.055402in}{1.438667in}}%
\pgfpathcurveto{\pgfqpoint{2.066452in}{1.438667in}}{\pgfqpoint{2.077051in}{1.443057in}}{\pgfqpoint{2.084865in}{1.450871in}}%
\pgfpathcurveto{\pgfqpoint{2.092678in}{1.458685in}}{\pgfqpoint{2.097069in}{1.469284in}}{\pgfqpoint{2.097069in}{1.480334in}}%
\pgfpathcurveto{\pgfqpoint{2.097069in}{1.491384in}}{\pgfqpoint{2.092678in}{1.501983in}}{\pgfqpoint{2.084865in}{1.509796in}}%
\pgfpathcurveto{\pgfqpoint{2.077051in}{1.517610in}}{\pgfqpoint{2.066452in}{1.522000in}}{\pgfqpoint{2.055402in}{1.522000in}}%
\pgfpathcurveto{\pgfqpoint{2.044352in}{1.522000in}}{\pgfqpoint{2.033753in}{1.517610in}}{\pgfqpoint{2.025939in}{1.509796in}}%
\pgfpathcurveto{\pgfqpoint{2.018125in}{1.501983in}}{\pgfqpoint{2.013735in}{1.491384in}}{\pgfqpoint{2.013735in}{1.480334in}}%
\pgfpathcurveto{\pgfqpoint{2.013735in}{1.469284in}}{\pgfqpoint{2.018125in}{1.458685in}}{\pgfqpoint{2.025939in}{1.450871in}}%
\pgfpathcurveto{\pgfqpoint{2.033753in}{1.443057in}}{\pgfqpoint{2.044352in}{1.438667in}}{\pgfqpoint{2.055402in}{1.438667in}}%
\pgfpathclose%
\pgfusepath{stroke,fill}%
\end{pgfscope}%
\begin{pgfscope}%
\pgfpathrectangle{\pgfqpoint{0.375000in}{0.330000in}}{\pgfqpoint{2.325000in}{2.310000in}}%
\pgfusepath{clip}%
\pgfsetbuttcap%
\pgfsetroundjoin%
\definecolor{currentfill}{rgb}{0.000000,0.000000,0.000000}%
\pgfsetfillcolor{currentfill}%
\pgfsetlinewidth{1.003750pt}%
\definecolor{currentstroke}{rgb}{0.000000,0.000000,0.000000}%
\pgfsetstrokecolor{currentstroke}%
\pgfsetdash{}{0pt}%
\pgfpathmoveto{\pgfqpoint{2.055402in}{1.438667in}}%
\pgfpathcurveto{\pgfqpoint{2.066452in}{1.438667in}}{\pgfqpoint{2.077051in}{1.443057in}}{\pgfqpoint{2.084865in}{1.450871in}}%
\pgfpathcurveto{\pgfqpoint{2.092678in}{1.458685in}}{\pgfqpoint{2.097069in}{1.469284in}}{\pgfqpoint{2.097069in}{1.480334in}}%
\pgfpathcurveto{\pgfqpoint{2.097069in}{1.491384in}}{\pgfqpoint{2.092678in}{1.501983in}}{\pgfqpoint{2.084865in}{1.509796in}}%
\pgfpathcurveto{\pgfqpoint{2.077051in}{1.517610in}}{\pgfqpoint{2.066452in}{1.522000in}}{\pgfqpoint{2.055402in}{1.522000in}}%
\pgfpathcurveto{\pgfqpoint{2.044352in}{1.522000in}}{\pgfqpoint{2.033753in}{1.517610in}}{\pgfqpoint{2.025939in}{1.509796in}}%
\pgfpathcurveto{\pgfqpoint{2.018125in}{1.501983in}}{\pgfqpoint{2.013735in}{1.491384in}}{\pgfqpoint{2.013735in}{1.480334in}}%
\pgfpathcurveto{\pgfqpoint{2.013735in}{1.469284in}}{\pgfqpoint{2.018125in}{1.458685in}}{\pgfqpoint{2.025939in}{1.450871in}}%
\pgfpathcurveto{\pgfqpoint{2.033753in}{1.443057in}}{\pgfqpoint{2.044352in}{1.438667in}}{\pgfqpoint{2.055402in}{1.438667in}}%
\pgfpathclose%
\pgfusepath{stroke,fill}%
\end{pgfscope}%
\begin{pgfscope}%
\pgfpathrectangle{\pgfqpoint{0.375000in}{0.330000in}}{\pgfqpoint{2.325000in}{2.310000in}}%
\pgfusepath{clip}%
\pgfsetbuttcap%
\pgfsetroundjoin%
\definecolor{currentfill}{rgb}{0.000000,0.000000,0.000000}%
\pgfsetfillcolor{currentfill}%
\pgfsetlinewidth{1.003750pt}%
\definecolor{currentstroke}{rgb}{0.000000,0.000000,0.000000}%
\pgfsetstrokecolor{currentstroke}%
\pgfsetdash{}{0pt}%
\pgfpathmoveto{\pgfqpoint{2.055402in}{1.438667in}}%
\pgfpathcurveto{\pgfqpoint{2.066452in}{1.438667in}}{\pgfqpoint{2.077051in}{1.443057in}}{\pgfqpoint{2.084865in}{1.450871in}}%
\pgfpathcurveto{\pgfqpoint{2.092678in}{1.458685in}}{\pgfqpoint{2.097069in}{1.469284in}}{\pgfqpoint{2.097069in}{1.480334in}}%
\pgfpathcurveto{\pgfqpoint{2.097069in}{1.491384in}}{\pgfqpoint{2.092678in}{1.501983in}}{\pgfqpoint{2.084865in}{1.509796in}}%
\pgfpathcurveto{\pgfqpoint{2.077051in}{1.517610in}}{\pgfqpoint{2.066452in}{1.522000in}}{\pgfqpoint{2.055402in}{1.522000in}}%
\pgfpathcurveto{\pgfqpoint{2.044352in}{1.522000in}}{\pgfqpoint{2.033753in}{1.517610in}}{\pgfqpoint{2.025939in}{1.509796in}}%
\pgfpathcurveto{\pgfqpoint{2.018125in}{1.501983in}}{\pgfqpoint{2.013735in}{1.491384in}}{\pgfqpoint{2.013735in}{1.480334in}}%
\pgfpathcurveto{\pgfqpoint{2.013735in}{1.469284in}}{\pgfqpoint{2.018125in}{1.458685in}}{\pgfqpoint{2.025939in}{1.450871in}}%
\pgfpathcurveto{\pgfqpoint{2.033753in}{1.443057in}}{\pgfqpoint{2.044352in}{1.438667in}}{\pgfqpoint{2.055402in}{1.438667in}}%
\pgfpathclose%
\pgfusepath{stroke,fill}%
\end{pgfscope}%
\begin{pgfscope}%
\pgfpathrectangle{\pgfqpoint{0.375000in}{0.330000in}}{\pgfqpoint{2.325000in}{2.310000in}}%
\pgfusepath{clip}%
\pgfsetbuttcap%
\pgfsetroundjoin%
\definecolor{currentfill}{rgb}{0.000000,0.000000,0.000000}%
\pgfsetfillcolor{currentfill}%
\pgfsetlinewidth{1.003750pt}%
\definecolor{currentstroke}{rgb}{0.000000,0.000000,0.000000}%
\pgfsetstrokecolor{currentstroke}%
\pgfsetdash{}{0pt}%
\pgfpathmoveto{\pgfqpoint{2.055402in}{1.438667in}}%
\pgfpathcurveto{\pgfqpoint{2.066452in}{1.438667in}}{\pgfqpoint{2.077051in}{1.443057in}}{\pgfqpoint{2.084865in}{1.450871in}}%
\pgfpathcurveto{\pgfqpoint{2.092678in}{1.458685in}}{\pgfqpoint{2.097069in}{1.469284in}}{\pgfqpoint{2.097069in}{1.480334in}}%
\pgfpathcurveto{\pgfqpoint{2.097069in}{1.491384in}}{\pgfqpoint{2.092678in}{1.501983in}}{\pgfqpoint{2.084865in}{1.509796in}}%
\pgfpathcurveto{\pgfqpoint{2.077051in}{1.517610in}}{\pgfqpoint{2.066452in}{1.522000in}}{\pgfqpoint{2.055402in}{1.522000in}}%
\pgfpathcurveto{\pgfqpoint{2.044352in}{1.522000in}}{\pgfqpoint{2.033753in}{1.517610in}}{\pgfqpoint{2.025939in}{1.509796in}}%
\pgfpathcurveto{\pgfqpoint{2.018125in}{1.501983in}}{\pgfqpoint{2.013735in}{1.491384in}}{\pgfqpoint{2.013735in}{1.480334in}}%
\pgfpathcurveto{\pgfqpoint{2.013735in}{1.469284in}}{\pgfqpoint{2.018125in}{1.458685in}}{\pgfqpoint{2.025939in}{1.450871in}}%
\pgfpathcurveto{\pgfqpoint{2.033753in}{1.443057in}}{\pgfqpoint{2.044352in}{1.438667in}}{\pgfqpoint{2.055402in}{1.438667in}}%
\pgfpathclose%
\pgfusepath{stroke,fill}%
\end{pgfscope}%
\begin{pgfscope}%
\pgfpathrectangle{\pgfqpoint{0.375000in}{0.330000in}}{\pgfqpoint{2.325000in}{2.310000in}}%
\pgfusepath{clip}%
\pgfsetbuttcap%
\pgfsetroundjoin%
\definecolor{currentfill}{rgb}{0.000000,0.000000,0.000000}%
\pgfsetfillcolor{currentfill}%
\pgfsetlinewidth{1.003750pt}%
\definecolor{currentstroke}{rgb}{0.000000,0.000000,0.000000}%
\pgfsetstrokecolor{currentstroke}%
\pgfsetdash{}{0pt}%
\pgfpathmoveto{\pgfqpoint{2.055402in}{1.438667in}}%
\pgfpathcurveto{\pgfqpoint{2.066452in}{1.438667in}}{\pgfqpoint{2.077051in}{1.443057in}}{\pgfqpoint{2.084865in}{1.450871in}}%
\pgfpathcurveto{\pgfqpoint{2.092678in}{1.458685in}}{\pgfqpoint{2.097069in}{1.469284in}}{\pgfqpoint{2.097069in}{1.480334in}}%
\pgfpathcurveto{\pgfqpoint{2.097069in}{1.491384in}}{\pgfqpoint{2.092678in}{1.501983in}}{\pgfqpoint{2.084865in}{1.509796in}}%
\pgfpathcurveto{\pgfqpoint{2.077051in}{1.517610in}}{\pgfqpoint{2.066452in}{1.522000in}}{\pgfqpoint{2.055402in}{1.522000in}}%
\pgfpathcurveto{\pgfqpoint{2.044352in}{1.522000in}}{\pgfqpoint{2.033753in}{1.517610in}}{\pgfqpoint{2.025939in}{1.509796in}}%
\pgfpathcurveto{\pgfqpoint{2.018125in}{1.501983in}}{\pgfqpoint{2.013735in}{1.491384in}}{\pgfqpoint{2.013735in}{1.480334in}}%
\pgfpathcurveto{\pgfqpoint{2.013735in}{1.469284in}}{\pgfqpoint{2.018125in}{1.458685in}}{\pgfqpoint{2.025939in}{1.450871in}}%
\pgfpathcurveto{\pgfqpoint{2.033753in}{1.443057in}}{\pgfqpoint{2.044352in}{1.438667in}}{\pgfqpoint{2.055402in}{1.438667in}}%
\pgfpathclose%
\pgfusepath{stroke,fill}%
\end{pgfscope}%
\begin{pgfscope}%
\pgfpathrectangle{\pgfqpoint{0.375000in}{0.330000in}}{\pgfqpoint{2.325000in}{2.310000in}}%
\pgfusepath{clip}%
\pgfsetbuttcap%
\pgfsetroundjoin%
\definecolor{currentfill}{rgb}{0.000000,0.000000,0.000000}%
\pgfsetfillcolor{currentfill}%
\pgfsetlinewidth{1.003750pt}%
\definecolor{currentstroke}{rgb}{0.000000,0.000000,0.000000}%
\pgfsetstrokecolor{currentstroke}%
\pgfsetdash{}{0pt}%
\pgfpathmoveto{\pgfqpoint{2.055402in}{1.438667in}}%
\pgfpathcurveto{\pgfqpoint{2.066452in}{1.438667in}}{\pgfqpoint{2.077051in}{1.443057in}}{\pgfqpoint{2.084865in}{1.450871in}}%
\pgfpathcurveto{\pgfqpoint{2.092678in}{1.458685in}}{\pgfqpoint{2.097069in}{1.469284in}}{\pgfqpoint{2.097069in}{1.480334in}}%
\pgfpathcurveto{\pgfqpoint{2.097069in}{1.491384in}}{\pgfqpoint{2.092678in}{1.501983in}}{\pgfqpoint{2.084865in}{1.509796in}}%
\pgfpathcurveto{\pgfqpoint{2.077051in}{1.517610in}}{\pgfqpoint{2.066452in}{1.522000in}}{\pgfqpoint{2.055402in}{1.522000in}}%
\pgfpathcurveto{\pgfqpoint{2.044352in}{1.522000in}}{\pgfqpoint{2.033753in}{1.517610in}}{\pgfqpoint{2.025939in}{1.509796in}}%
\pgfpathcurveto{\pgfqpoint{2.018125in}{1.501983in}}{\pgfqpoint{2.013735in}{1.491384in}}{\pgfqpoint{2.013735in}{1.480334in}}%
\pgfpathcurveto{\pgfqpoint{2.013735in}{1.469284in}}{\pgfqpoint{2.018125in}{1.458685in}}{\pgfqpoint{2.025939in}{1.450871in}}%
\pgfpathcurveto{\pgfqpoint{2.033753in}{1.443057in}}{\pgfqpoint{2.044352in}{1.438667in}}{\pgfqpoint{2.055402in}{1.438667in}}%
\pgfpathclose%
\pgfusepath{stroke,fill}%
\end{pgfscope}%
\begin{pgfscope}%
\pgfpathrectangle{\pgfqpoint{0.375000in}{0.330000in}}{\pgfqpoint{2.325000in}{2.310000in}}%
\pgfusepath{clip}%
\pgfsetbuttcap%
\pgfsetroundjoin%
\definecolor{currentfill}{rgb}{0.000000,0.000000,0.000000}%
\pgfsetfillcolor{currentfill}%
\pgfsetlinewidth{1.003750pt}%
\definecolor{currentstroke}{rgb}{0.000000,0.000000,0.000000}%
\pgfsetstrokecolor{currentstroke}%
\pgfsetdash{}{0pt}%
\pgfpathmoveto{\pgfqpoint{2.055402in}{1.438667in}}%
\pgfpathcurveto{\pgfqpoint{2.066452in}{1.438667in}}{\pgfqpoint{2.077051in}{1.443057in}}{\pgfqpoint{2.084865in}{1.450871in}}%
\pgfpathcurveto{\pgfqpoint{2.092678in}{1.458685in}}{\pgfqpoint{2.097069in}{1.469284in}}{\pgfqpoint{2.097069in}{1.480334in}}%
\pgfpathcurveto{\pgfqpoint{2.097069in}{1.491384in}}{\pgfqpoint{2.092678in}{1.501983in}}{\pgfqpoint{2.084865in}{1.509796in}}%
\pgfpathcurveto{\pgfqpoint{2.077051in}{1.517610in}}{\pgfqpoint{2.066452in}{1.522000in}}{\pgfqpoint{2.055402in}{1.522000in}}%
\pgfpathcurveto{\pgfqpoint{2.044352in}{1.522000in}}{\pgfqpoint{2.033753in}{1.517610in}}{\pgfqpoint{2.025939in}{1.509796in}}%
\pgfpathcurveto{\pgfqpoint{2.018125in}{1.501983in}}{\pgfqpoint{2.013735in}{1.491384in}}{\pgfqpoint{2.013735in}{1.480334in}}%
\pgfpathcurveto{\pgfqpoint{2.013735in}{1.469284in}}{\pgfqpoint{2.018125in}{1.458685in}}{\pgfqpoint{2.025939in}{1.450871in}}%
\pgfpathcurveto{\pgfqpoint{2.033753in}{1.443057in}}{\pgfqpoint{2.044352in}{1.438667in}}{\pgfqpoint{2.055402in}{1.438667in}}%
\pgfpathclose%
\pgfusepath{stroke,fill}%
\end{pgfscope}%
\begin{pgfscope}%
\pgfpathrectangle{\pgfqpoint{0.375000in}{0.330000in}}{\pgfqpoint{2.325000in}{2.310000in}}%
\pgfusepath{clip}%
\pgfsetbuttcap%
\pgfsetroundjoin%
\definecolor{currentfill}{rgb}{0.000000,0.000000,0.000000}%
\pgfsetfillcolor{currentfill}%
\pgfsetlinewidth{1.003750pt}%
\definecolor{currentstroke}{rgb}{0.000000,0.000000,0.000000}%
\pgfsetstrokecolor{currentstroke}%
\pgfsetdash{}{0pt}%
\pgfpathmoveto{\pgfqpoint{2.055402in}{1.438667in}}%
\pgfpathcurveto{\pgfqpoint{2.066452in}{1.438667in}}{\pgfqpoint{2.077051in}{1.443057in}}{\pgfqpoint{2.084865in}{1.450871in}}%
\pgfpathcurveto{\pgfqpoint{2.092678in}{1.458685in}}{\pgfqpoint{2.097069in}{1.469284in}}{\pgfqpoint{2.097069in}{1.480334in}}%
\pgfpathcurveto{\pgfqpoint{2.097069in}{1.491384in}}{\pgfqpoint{2.092678in}{1.501983in}}{\pgfqpoint{2.084865in}{1.509796in}}%
\pgfpathcurveto{\pgfqpoint{2.077051in}{1.517610in}}{\pgfqpoint{2.066452in}{1.522000in}}{\pgfqpoint{2.055402in}{1.522000in}}%
\pgfpathcurveto{\pgfqpoint{2.044352in}{1.522000in}}{\pgfqpoint{2.033753in}{1.517610in}}{\pgfqpoint{2.025939in}{1.509796in}}%
\pgfpathcurveto{\pgfqpoint{2.018125in}{1.501983in}}{\pgfqpoint{2.013735in}{1.491384in}}{\pgfqpoint{2.013735in}{1.480334in}}%
\pgfpathcurveto{\pgfqpoint{2.013735in}{1.469284in}}{\pgfqpoint{2.018125in}{1.458685in}}{\pgfqpoint{2.025939in}{1.450871in}}%
\pgfpathcurveto{\pgfqpoint{2.033753in}{1.443057in}}{\pgfqpoint{2.044352in}{1.438667in}}{\pgfqpoint{2.055402in}{1.438667in}}%
\pgfpathclose%
\pgfusepath{stroke,fill}%
\end{pgfscope}%
\begin{pgfscope}%
\pgfpathrectangle{\pgfqpoint{0.375000in}{0.330000in}}{\pgfqpoint{2.325000in}{2.310000in}}%
\pgfusepath{clip}%
\pgfsetbuttcap%
\pgfsetroundjoin%
\definecolor{currentfill}{rgb}{0.000000,0.000000,0.000000}%
\pgfsetfillcolor{currentfill}%
\pgfsetlinewidth{1.003750pt}%
\definecolor{currentstroke}{rgb}{0.000000,0.000000,0.000000}%
\pgfsetstrokecolor{currentstroke}%
\pgfsetdash{}{0pt}%
\pgfpathmoveto{\pgfqpoint{2.055402in}{1.438667in}}%
\pgfpathcurveto{\pgfqpoint{2.066452in}{1.438667in}}{\pgfqpoint{2.077051in}{1.443057in}}{\pgfqpoint{2.084865in}{1.450871in}}%
\pgfpathcurveto{\pgfqpoint{2.092678in}{1.458685in}}{\pgfqpoint{2.097069in}{1.469284in}}{\pgfqpoint{2.097069in}{1.480334in}}%
\pgfpathcurveto{\pgfqpoint{2.097069in}{1.491384in}}{\pgfqpoint{2.092678in}{1.501983in}}{\pgfqpoint{2.084865in}{1.509796in}}%
\pgfpathcurveto{\pgfqpoint{2.077051in}{1.517610in}}{\pgfqpoint{2.066452in}{1.522000in}}{\pgfqpoint{2.055402in}{1.522000in}}%
\pgfpathcurveto{\pgfqpoint{2.044352in}{1.522000in}}{\pgfqpoint{2.033753in}{1.517610in}}{\pgfqpoint{2.025939in}{1.509796in}}%
\pgfpathcurveto{\pgfqpoint{2.018125in}{1.501983in}}{\pgfqpoint{2.013735in}{1.491384in}}{\pgfqpoint{2.013735in}{1.480334in}}%
\pgfpathcurveto{\pgfqpoint{2.013735in}{1.469284in}}{\pgfqpoint{2.018125in}{1.458685in}}{\pgfqpoint{2.025939in}{1.450871in}}%
\pgfpathcurveto{\pgfqpoint{2.033753in}{1.443057in}}{\pgfqpoint{2.044352in}{1.438667in}}{\pgfqpoint{2.055402in}{1.438667in}}%
\pgfpathclose%
\pgfusepath{stroke,fill}%
\end{pgfscope}%
\begin{pgfscope}%
\pgfpathrectangle{\pgfqpoint{0.375000in}{0.330000in}}{\pgfqpoint{2.325000in}{2.310000in}}%
\pgfusepath{clip}%
\pgfsetbuttcap%
\pgfsetroundjoin%
\definecolor{currentfill}{rgb}{0.000000,0.000000,0.000000}%
\pgfsetfillcolor{currentfill}%
\pgfsetlinewidth{1.003750pt}%
\definecolor{currentstroke}{rgb}{0.000000,0.000000,0.000000}%
\pgfsetstrokecolor{currentstroke}%
\pgfsetdash{}{0pt}%
\pgfpathmoveto{\pgfqpoint{2.055402in}{1.438667in}}%
\pgfpathcurveto{\pgfqpoint{2.066452in}{1.438667in}}{\pgfqpoint{2.077051in}{1.443057in}}{\pgfqpoint{2.084865in}{1.450871in}}%
\pgfpathcurveto{\pgfqpoint{2.092678in}{1.458685in}}{\pgfqpoint{2.097069in}{1.469284in}}{\pgfqpoint{2.097069in}{1.480334in}}%
\pgfpathcurveto{\pgfqpoint{2.097069in}{1.491384in}}{\pgfqpoint{2.092678in}{1.501983in}}{\pgfqpoint{2.084865in}{1.509796in}}%
\pgfpathcurveto{\pgfqpoint{2.077051in}{1.517610in}}{\pgfqpoint{2.066452in}{1.522000in}}{\pgfqpoint{2.055402in}{1.522000in}}%
\pgfpathcurveto{\pgfqpoint{2.044352in}{1.522000in}}{\pgfqpoint{2.033753in}{1.517610in}}{\pgfqpoint{2.025939in}{1.509796in}}%
\pgfpathcurveto{\pgfqpoint{2.018125in}{1.501983in}}{\pgfqpoint{2.013735in}{1.491384in}}{\pgfqpoint{2.013735in}{1.480334in}}%
\pgfpathcurveto{\pgfqpoint{2.013735in}{1.469284in}}{\pgfqpoint{2.018125in}{1.458685in}}{\pgfqpoint{2.025939in}{1.450871in}}%
\pgfpathcurveto{\pgfqpoint{2.033753in}{1.443057in}}{\pgfqpoint{2.044352in}{1.438667in}}{\pgfqpoint{2.055402in}{1.438667in}}%
\pgfpathclose%
\pgfusepath{stroke,fill}%
\end{pgfscope}%
\begin{pgfscope}%
\pgfpathrectangle{\pgfqpoint{0.375000in}{0.330000in}}{\pgfqpoint{2.325000in}{2.310000in}}%
\pgfusepath{clip}%
\pgfsetbuttcap%
\pgfsetroundjoin%
\definecolor{currentfill}{rgb}{0.000000,0.000000,0.000000}%
\pgfsetfillcolor{currentfill}%
\pgfsetlinewidth{1.003750pt}%
\definecolor{currentstroke}{rgb}{0.000000,0.000000,0.000000}%
\pgfsetstrokecolor{currentstroke}%
\pgfsetdash{}{0pt}%
\pgfpathmoveto{\pgfqpoint{2.055402in}{1.438667in}}%
\pgfpathcurveto{\pgfqpoint{2.066452in}{1.438667in}}{\pgfqpoint{2.077051in}{1.443057in}}{\pgfqpoint{2.084865in}{1.450871in}}%
\pgfpathcurveto{\pgfqpoint{2.092678in}{1.458685in}}{\pgfqpoint{2.097069in}{1.469284in}}{\pgfqpoint{2.097069in}{1.480334in}}%
\pgfpathcurveto{\pgfqpoint{2.097069in}{1.491384in}}{\pgfqpoint{2.092678in}{1.501983in}}{\pgfqpoint{2.084865in}{1.509796in}}%
\pgfpathcurveto{\pgfqpoint{2.077051in}{1.517610in}}{\pgfqpoint{2.066452in}{1.522000in}}{\pgfqpoint{2.055402in}{1.522000in}}%
\pgfpathcurveto{\pgfqpoint{2.044352in}{1.522000in}}{\pgfqpoint{2.033753in}{1.517610in}}{\pgfqpoint{2.025939in}{1.509796in}}%
\pgfpathcurveto{\pgfqpoint{2.018125in}{1.501983in}}{\pgfqpoint{2.013735in}{1.491384in}}{\pgfqpoint{2.013735in}{1.480334in}}%
\pgfpathcurveto{\pgfqpoint{2.013735in}{1.469284in}}{\pgfqpoint{2.018125in}{1.458685in}}{\pgfqpoint{2.025939in}{1.450871in}}%
\pgfpathcurveto{\pgfqpoint{2.033753in}{1.443057in}}{\pgfqpoint{2.044352in}{1.438667in}}{\pgfqpoint{2.055402in}{1.438667in}}%
\pgfpathclose%
\pgfusepath{stroke,fill}%
\end{pgfscope}%
\begin{pgfscope}%
\pgfpathrectangle{\pgfqpoint{0.375000in}{0.330000in}}{\pgfqpoint{2.325000in}{2.310000in}}%
\pgfusepath{clip}%
\pgfsetbuttcap%
\pgfsetroundjoin%
\definecolor{currentfill}{rgb}{0.000000,0.000000,0.000000}%
\pgfsetfillcolor{currentfill}%
\pgfsetlinewidth{1.003750pt}%
\definecolor{currentstroke}{rgb}{0.000000,0.000000,0.000000}%
\pgfsetstrokecolor{currentstroke}%
\pgfsetdash{}{0pt}%
\pgfpathmoveto{\pgfqpoint{2.055402in}{1.438667in}}%
\pgfpathcurveto{\pgfqpoint{2.066452in}{1.438667in}}{\pgfqpoint{2.077051in}{1.443057in}}{\pgfqpoint{2.084865in}{1.450871in}}%
\pgfpathcurveto{\pgfqpoint{2.092678in}{1.458685in}}{\pgfqpoint{2.097069in}{1.469284in}}{\pgfqpoint{2.097069in}{1.480334in}}%
\pgfpathcurveto{\pgfqpoint{2.097069in}{1.491384in}}{\pgfqpoint{2.092678in}{1.501983in}}{\pgfqpoint{2.084865in}{1.509796in}}%
\pgfpathcurveto{\pgfqpoint{2.077051in}{1.517610in}}{\pgfqpoint{2.066452in}{1.522000in}}{\pgfqpoint{2.055402in}{1.522000in}}%
\pgfpathcurveto{\pgfqpoint{2.044352in}{1.522000in}}{\pgfqpoint{2.033753in}{1.517610in}}{\pgfqpoint{2.025939in}{1.509796in}}%
\pgfpathcurveto{\pgfqpoint{2.018125in}{1.501983in}}{\pgfqpoint{2.013735in}{1.491384in}}{\pgfqpoint{2.013735in}{1.480334in}}%
\pgfpathcurveto{\pgfqpoint{2.013735in}{1.469284in}}{\pgfqpoint{2.018125in}{1.458685in}}{\pgfqpoint{2.025939in}{1.450871in}}%
\pgfpathcurveto{\pgfqpoint{2.033753in}{1.443057in}}{\pgfqpoint{2.044352in}{1.438667in}}{\pgfqpoint{2.055402in}{1.438667in}}%
\pgfpathclose%
\pgfusepath{stroke,fill}%
\end{pgfscope}%
\begin{pgfscope}%
\pgfpathrectangle{\pgfqpoint{0.375000in}{0.330000in}}{\pgfqpoint{2.325000in}{2.310000in}}%
\pgfusepath{clip}%
\pgfsetbuttcap%
\pgfsetroundjoin%
\definecolor{currentfill}{rgb}{0.000000,0.000000,0.000000}%
\pgfsetfillcolor{currentfill}%
\pgfsetlinewidth{1.003750pt}%
\definecolor{currentstroke}{rgb}{0.000000,0.000000,0.000000}%
\pgfsetstrokecolor{currentstroke}%
\pgfsetdash{}{0pt}%
\pgfpathmoveto{\pgfqpoint{2.055402in}{1.438667in}}%
\pgfpathcurveto{\pgfqpoint{2.066452in}{1.438667in}}{\pgfqpoint{2.077051in}{1.443057in}}{\pgfqpoint{2.084865in}{1.450871in}}%
\pgfpathcurveto{\pgfqpoint{2.092678in}{1.458685in}}{\pgfqpoint{2.097069in}{1.469284in}}{\pgfqpoint{2.097069in}{1.480334in}}%
\pgfpathcurveto{\pgfqpoint{2.097069in}{1.491384in}}{\pgfqpoint{2.092678in}{1.501983in}}{\pgfqpoint{2.084865in}{1.509796in}}%
\pgfpathcurveto{\pgfqpoint{2.077051in}{1.517610in}}{\pgfqpoint{2.066452in}{1.522000in}}{\pgfqpoint{2.055402in}{1.522000in}}%
\pgfpathcurveto{\pgfqpoint{2.044352in}{1.522000in}}{\pgfqpoint{2.033753in}{1.517610in}}{\pgfqpoint{2.025939in}{1.509796in}}%
\pgfpathcurveto{\pgfqpoint{2.018125in}{1.501983in}}{\pgfqpoint{2.013735in}{1.491384in}}{\pgfqpoint{2.013735in}{1.480334in}}%
\pgfpathcurveto{\pgfqpoint{2.013735in}{1.469284in}}{\pgfqpoint{2.018125in}{1.458685in}}{\pgfqpoint{2.025939in}{1.450871in}}%
\pgfpathcurveto{\pgfqpoint{2.033753in}{1.443057in}}{\pgfqpoint{2.044352in}{1.438667in}}{\pgfqpoint{2.055402in}{1.438667in}}%
\pgfpathclose%
\pgfusepath{stroke,fill}%
\end{pgfscope}%
\begin{pgfscope}%
\pgfpathrectangle{\pgfqpoint{0.375000in}{0.330000in}}{\pgfqpoint{2.325000in}{2.310000in}}%
\pgfusepath{clip}%
\pgfsetbuttcap%
\pgfsetroundjoin%
\definecolor{currentfill}{rgb}{0.000000,0.000000,0.000000}%
\pgfsetfillcolor{currentfill}%
\pgfsetlinewidth{1.003750pt}%
\definecolor{currentstroke}{rgb}{0.000000,0.000000,0.000000}%
\pgfsetstrokecolor{currentstroke}%
\pgfsetdash{}{0pt}%
\pgfpathmoveto{\pgfqpoint{2.055402in}{1.438667in}}%
\pgfpathcurveto{\pgfqpoint{2.066452in}{1.438667in}}{\pgfqpoint{2.077051in}{1.443057in}}{\pgfqpoint{2.084865in}{1.450871in}}%
\pgfpathcurveto{\pgfqpoint{2.092678in}{1.458685in}}{\pgfqpoint{2.097069in}{1.469284in}}{\pgfqpoint{2.097069in}{1.480334in}}%
\pgfpathcurveto{\pgfqpoint{2.097069in}{1.491384in}}{\pgfqpoint{2.092678in}{1.501983in}}{\pgfqpoint{2.084865in}{1.509796in}}%
\pgfpathcurveto{\pgfqpoint{2.077051in}{1.517610in}}{\pgfqpoint{2.066452in}{1.522000in}}{\pgfqpoint{2.055402in}{1.522000in}}%
\pgfpathcurveto{\pgfqpoint{2.044352in}{1.522000in}}{\pgfqpoint{2.033753in}{1.517610in}}{\pgfqpoint{2.025939in}{1.509796in}}%
\pgfpathcurveto{\pgfqpoint{2.018125in}{1.501983in}}{\pgfqpoint{2.013735in}{1.491384in}}{\pgfqpoint{2.013735in}{1.480334in}}%
\pgfpathcurveto{\pgfqpoint{2.013735in}{1.469284in}}{\pgfqpoint{2.018125in}{1.458685in}}{\pgfqpoint{2.025939in}{1.450871in}}%
\pgfpathcurveto{\pgfqpoint{2.033753in}{1.443057in}}{\pgfqpoint{2.044352in}{1.438667in}}{\pgfqpoint{2.055402in}{1.438667in}}%
\pgfpathclose%
\pgfusepath{stroke,fill}%
\end{pgfscope}%
\begin{pgfscope}%
\pgfpathrectangle{\pgfqpoint{0.375000in}{0.330000in}}{\pgfqpoint{2.325000in}{2.310000in}}%
\pgfusepath{clip}%
\pgfsetbuttcap%
\pgfsetroundjoin%
\definecolor{currentfill}{rgb}{0.000000,0.000000,0.000000}%
\pgfsetfillcolor{currentfill}%
\pgfsetlinewidth{1.003750pt}%
\definecolor{currentstroke}{rgb}{0.000000,0.000000,0.000000}%
\pgfsetstrokecolor{currentstroke}%
\pgfsetdash{}{0pt}%
\pgfpathmoveto{\pgfqpoint{2.055402in}{1.438667in}}%
\pgfpathcurveto{\pgfqpoint{2.066452in}{1.438667in}}{\pgfqpoint{2.077051in}{1.443057in}}{\pgfqpoint{2.084865in}{1.450871in}}%
\pgfpathcurveto{\pgfqpoint{2.092678in}{1.458685in}}{\pgfqpoint{2.097069in}{1.469284in}}{\pgfqpoint{2.097069in}{1.480334in}}%
\pgfpathcurveto{\pgfqpoint{2.097069in}{1.491384in}}{\pgfqpoint{2.092678in}{1.501983in}}{\pgfqpoint{2.084865in}{1.509796in}}%
\pgfpathcurveto{\pgfqpoint{2.077051in}{1.517610in}}{\pgfqpoint{2.066452in}{1.522000in}}{\pgfqpoint{2.055402in}{1.522000in}}%
\pgfpathcurveto{\pgfqpoint{2.044352in}{1.522000in}}{\pgfqpoint{2.033753in}{1.517610in}}{\pgfqpoint{2.025939in}{1.509796in}}%
\pgfpathcurveto{\pgfqpoint{2.018125in}{1.501983in}}{\pgfqpoint{2.013735in}{1.491384in}}{\pgfqpoint{2.013735in}{1.480334in}}%
\pgfpathcurveto{\pgfqpoint{2.013735in}{1.469284in}}{\pgfqpoint{2.018125in}{1.458685in}}{\pgfqpoint{2.025939in}{1.450871in}}%
\pgfpathcurveto{\pgfqpoint{2.033753in}{1.443057in}}{\pgfqpoint{2.044352in}{1.438667in}}{\pgfqpoint{2.055402in}{1.438667in}}%
\pgfpathclose%
\pgfusepath{stroke,fill}%
\end{pgfscope}%
\begin{pgfscope}%
\pgfpathrectangle{\pgfqpoint{0.375000in}{0.330000in}}{\pgfqpoint{2.325000in}{2.310000in}}%
\pgfusepath{clip}%
\pgfsetbuttcap%
\pgfsetroundjoin%
\definecolor{currentfill}{rgb}{0.000000,0.000000,0.000000}%
\pgfsetfillcolor{currentfill}%
\pgfsetlinewidth{1.003750pt}%
\definecolor{currentstroke}{rgb}{0.000000,0.000000,0.000000}%
\pgfsetstrokecolor{currentstroke}%
\pgfsetdash{}{0pt}%
\pgfpathmoveto{\pgfqpoint{2.055402in}{1.438667in}}%
\pgfpathcurveto{\pgfqpoint{2.066452in}{1.438667in}}{\pgfqpoint{2.077051in}{1.443057in}}{\pgfqpoint{2.084865in}{1.450871in}}%
\pgfpathcurveto{\pgfqpoint{2.092678in}{1.458685in}}{\pgfqpoint{2.097069in}{1.469284in}}{\pgfqpoint{2.097069in}{1.480334in}}%
\pgfpathcurveto{\pgfqpoint{2.097069in}{1.491384in}}{\pgfqpoint{2.092678in}{1.501983in}}{\pgfqpoint{2.084865in}{1.509796in}}%
\pgfpathcurveto{\pgfqpoint{2.077051in}{1.517610in}}{\pgfqpoint{2.066452in}{1.522000in}}{\pgfqpoint{2.055402in}{1.522000in}}%
\pgfpathcurveto{\pgfqpoint{2.044352in}{1.522000in}}{\pgfqpoint{2.033753in}{1.517610in}}{\pgfqpoint{2.025939in}{1.509796in}}%
\pgfpathcurveto{\pgfqpoint{2.018125in}{1.501983in}}{\pgfqpoint{2.013735in}{1.491384in}}{\pgfqpoint{2.013735in}{1.480334in}}%
\pgfpathcurveto{\pgfqpoint{2.013735in}{1.469284in}}{\pgfqpoint{2.018125in}{1.458685in}}{\pgfqpoint{2.025939in}{1.450871in}}%
\pgfpathcurveto{\pgfqpoint{2.033753in}{1.443057in}}{\pgfqpoint{2.044352in}{1.438667in}}{\pgfqpoint{2.055402in}{1.438667in}}%
\pgfpathclose%
\pgfusepath{stroke,fill}%
\end{pgfscope}%
\begin{pgfscope}%
\pgfpathrectangle{\pgfqpoint{0.375000in}{0.330000in}}{\pgfqpoint{2.325000in}{2.310000in}}%
\pgfusepath{clip}%
\pgfsetbuttcap%
\pgfsetroundjoin%
\definecolor{currentfill}{rgb}{0.000000,0.000000,0.000000}%
\pgfsetfillcolor{currentfill}%
\pgfsetlinewidth{1.003750pt}%
\definecolor{currentstroke}{rgb}{0.000000,0.000000,0.000000}%
\pgfsetstrokecolor{currentstroke}%
\pgfsetdash{}{0pt}%
\pgfpathmoveto{\pgfqpoint{2.055402in}{1.438667in}}%
\pgfpathcurveto{\pgfqpoint{2.066452in}{1.438667in}}{\pgfqpoint{2.077051in}{1.443057in}}{\pgfqpoint{2.084865in}{1.450871in}}%
\pgfpathcurveto{\pgfqpoint{2.092678in}{1.458685in}}{\pgfqpoint{2.097069in}{1.469284in}}{\pgfqpoint{2.097069in}{1.480334in}}%
\pgfpathcurveto{\pgfqpoint{2.097069in}{1.491384in}}{\pgfqpoint{2.092678in}{1.501983in}}{\pgfqpoint{2.084865in}{1.509796in}}%
\pgfpathcurveto{\pgfqpoint{2.077051in}{1.517610in}}{\pgfqpoint{2.066452in}{1.522000in}}{\pgfqpoint{2.055402in}{1.522000in}}%
\pgfpathcurveto{\pgfqpoint{2.044352in}{1.522000in}}{\pgfqpoint{2.033753in}{1.517610in}}{\pgfqpoint{2.025939in}{1.509796in}}%
\pgfpathcurveto{\pgfqpoint{2.018125in}{1.501983in}}{\pgfqpoint{2.013735in}{1.491384in}}{\pgfqpoint{2.013735in}{1.480334in}}%
\pgfpathcurveto{\pgfqpoint{2.013735in}{1.469284in}}{\pgfqpoint{2.018125in}{1.458685in}}{\pgfqpoint{2.025939in}{1.450871in}}%
\pgfpathcurveto{\pgfqpoint{2.033753in}{1.443057in}}{\pgfqpoint{2.044352in}{1.438667in}}{\pgfqpoint{2.055402in}{1.438667in}}%
\pgfpathclose%
\pgfusepath{stroke,fill}%
\end{pgfscope}%
\begin{pgfscope}%
\pgfpathrectangle{\pgfqpoint{0.375000in}{0.330000in}}{\pgfqpoint{2.325000in}{2.310000in}}%
\pgfusepath{clip}%
\pgfsetbuttcap%
\pgfsetroundjoin%
\definecolor{currentfill}{rgb}{0.000000,0.000000,0.000000}%
\pgfsetfillcolor{currentfill}%
\pgfsetlinewidth{1.003750pt}%
\definecolor{currentstroke}{rgb}{0.000000,0.000000,0.000000}%
\pgfsetstrokecolor{currentstroke}%
\pgfsetdash{}{0pt}%
\pgfpathmoveto{\pgfqpoint{2.055402in}{1.438667in}}%
\pgfpathcurveto{\pgfqpoint{2.066452in}{1.438667in}}{\pgfqpoint{2.077051in}{1.443057in}}{\pgfqpoint{2.084865in}{1.450871in}}%
\pgfpathcurveto{\pgfqpoint{2.092678in}{1.458685in}}{\pgfqpoint{2.097069in}{1.469284in}}{\pgfqpoint{2.097069in}{1.480334in}}%
\pgfpathcurveto{\pgfqpoint{2.097069in}{1.491384in}}{\pgfqpoint{2.092678in}{1.501983in}}{\pgfqpoint{2.084865in}{1.509796in}}%
\pgfpathcurveto{\pgfqpoint{2.077051in}{1.517610in}}{\pgfqpoint{2.066452in}{1.522000in}}{\pgfqpoint{2.055402in}{1.522000in}}%
\pgfpathcurveto{\pgfqpoint{2.044352in}{1.522000in}}{\pgfqpoint{2.033753in}{1.517610in}}{\pgfqpoint{2.025939in}{1.509796in}}%
\pgfpathcurveto{\pgfqpoint{2.018125in}{1.501983in}}{\pgfqpoint{2.013735in}{1.491384in}}{\pgfqpoint{2.013735in}{1.480334in}}%
\pgfpathcurveto{\pgfqpoint{2.013735in}{1.469284in}}{\pgfqpoint{2.018125in}{1.458685in}}{\pgfqpoint{2.025939in}{1.450871in}}%
\pgfpathcurveto{\pgfqpoint{2.033753in}{1.443057in}}{\pgfqpoint{2.044352in}{1.438667in}}{\pgfqpoint{2.055402in}{1.438667in}}%
\pgfpathclose%
\pgfusepath{stroke,fill}%
\end{pgfscope}%
\begin{pgfscope}%
\pgfpathrectangle{\pgfqpoint{0.375000in}{0.330000in}}{\pgfqpoint{2.325000in}{2.310000in}}%
\pgfusepath{clip}%
\pgfsetbuttcap%
\pgfsetroundjoin%
\definecolor{currentfill}{rgb}{0.000000,0.000000,0.000000}%
\pgfsetfillcolor{currentfill}%
\pgfsetlinewidth{1.003750pt}%
\definecolor{currentstroke}{rgb}{0.000000,0.000000,0.000000}%
\pgfsetstrokecolor{currentstroke}%
\pgfsetdash{}{0pt}%
\pgfpathmoveto{\pgfqpoint{2.055402in}{1.438667in}}%
\pgfpathcurveto{\pgfqpoint{2.066452in}{1.438667in}}{\pgfqpoint{2.077051in}{1.443057in}}{\pgfqpoint{2.084865in}{1.450871in}}%
\pgfpathcurveto{\pgfqpoint{2.092678in}{1.458685in}}{\pgfqpoint{2.097069in}{1.469284in}}{\pgfqpoint{2.097069in}{1.480334in}}%
\pgfpathcurveto{\pgfqpoint{2.097069in}{1.491384in}}{\pgfqpoint{2.092678in}{1.501983in}}{\pgfqpoint{2.084865in}{1.509796in}}%
\pgfpathcurveto{\pgfqpoint{2.077051in}{1.517610in}}{\pgfqpoint{2.066452in}{1.522000in}}{\pgfqpoint{2.055402in}{1.522000in}}%
\pgfpathcurveto{\pgfqpoint{2.044352in}{1.522000in}}{\pgfqpoint{2.033753in}{1.517610in}}{\pgfqpoint{2.025939in}{1.509796in}}%
\pgfpathcurveto{\pgfqpoint{2.018125in}{1.501983in}}{\pgfqpoint{2.013735in}{1.491384in}}{\pgfqpoint{2.013735in}{1.480334in}}%
\pgfpathcurveto{\pgfqpoint{2.013735in}{1.469284in}}{\pgfqpoint{2.018125in}{1.458685in}}{\pgfqpoint{2.025939in}{1.450871in}}%
\pgfpathcurveto{\pgfqpoint{2.033753in}{1.443057in}}{\pgfqpoint{2.044352in}{1.438667in}}{\pgfqpoint{2.055402in}{1.438667in}}%
\pgfpathclose%
\pgfusepath{stroke,fill}%
\end{pgfscope}%
\begin{pgfscope}%
\pgfpathrectangle{\pgfqpoint{0.375000in}{0.330000in}}{\pgfqpoint{2.325000in}{2.310000in}}%
\pgfusepath{clip}%
\pgfsetbuttcap%
\pgfsetroundjoin%
\definecolor{currentfill}{rgb}{0.000000,0.000000,0.000000}%
\pgfsetfillcolor{currentfill}%
\pgfsetlinewidth{1.003750pt}%
\definecolor{currentstroke}{rgb}{0.000000,0.000000,0.000000}%
\pgfsetstrokecolor{currentstroke}%
\pgfsetdash{}{0pt}%
\pgfpathmoveto{\pgfqpoint{2.055402in}{1.438667in}}%
\pgfpathcurveto{\pgfqpoint{2.066452in}{1.438667in}}{\pgfqpoint{2.077051in}{1.443057in}}{\pgfqpoint{2.084865in}{1.450871in}}%
\pgfpathcurveto{\pgfqpoint{2.092678in}{1.458685in}}{\pgfqpoint{2.097069in}{1.469284in}}{\pgfqpoint{2.097069in}{1.480334in}}%
\pgfpathcurveto{\pgfqpoint{2.097069in}{1.491384in}}{\pgfqpoint{2.092678in}{1.501983in}}{\pgfqpoint{2.084865in}{1.509796in}}%
\pgfpathcurveto{\pgfqpoint{2.077051in}{1.517610in}}{\pgfqpoint{2.066452in}{1.522000in}}{\pgfqpoint{2.055402in}{1.522000in}}%
\pgfpathcurveto{\pgfqpoint{2.044352in}{1.522000in}}{\pgfqpoint{2.033753in}{1.517610in}}{\pgfqpoint{2.025939in}{1.509796in}}%
\pgfpathcurveto{\pgfqpoint{2.018125in}{1.501983in}}{\pgfqpoint{2.013735in}{1.491384in}}{\pgfqpoint{2.013735in}{1.480334in}}%
\pgfpathcurveto{\pgfqpoint{2.013735in}{1.469284in}}{\pgfqpoint{2.018125in}{1.458685in}}{\pgfqpoint{2.025939in}{1.450871in}}%
\pgfpathcurveto{\pgfqpoint{2.033753in}{1.443057in}}{\pgfqpoint{2.044352in}{1.438667in}}{\pgfqpoint{2.055402in}{1.438667in}}%
\pgfpathclose%
\pgfusepath{stroke,fill}%
\end{pgfscope}%
\begin{pgfscope}%
\pgfpathrectangle{\pgfqpoint{0.375000in}{0.330000in}}{\pgfqpoint{2.325000in}{2.310000in}}%
\pgfusepath{clip}%
\pgfsetbuttcap%
\pgfsetroundjoin%
\definecolor{currentfill}{rgb}{0.000000,0.000000,0.000000}%
\pgfsetfillcolor{currentfill}%
\pgfsetlinewidth{1.003750pt}%
\definecolor{currentstroke}{rgb}{0.000000,0.000000,0.000000}%
\pgfsetstrokecolor{currentstroke}%
\pgfsetdash{}{0pt}%
\pgfpathmoveto{\pgfqpoint{2.055402in}{1.438667in}}%
\pgfpathcurveto{\pgfqpoint{2.066452in}{1.438667in}}{\pgfqpoint{2.077051in}{1.443057in}}{\pgfqpoint{2.084865in}{1.450871in}}%
\pgfpathcurveto{\pgfqpoint{2.092678in}{1.458685in}}{\pgfqpoint{2.097069in}{1.469284in}}{\pgfqpoint{2.097069in}{1.480334in}}%
\pgfpathcurveto{\pgfqpoint{2.097069in}{1.491384in}}{\pgfqpoint{2.092678in}{1.501983in}}{\pgfqpoint{2.084865in}{1.509796in}}%
\pgfpathcurveto{\pgfqpoint{2.077051in}{1.517610in}}{\pgfqpoint{2.066452in}{1.522000in}}{\pgfqpoint{2.055402in}{1.522000in}}%
\pgfpathcurveto{\pgfqpoint{2.044352in}{1.522000in}}{\pgfqpoint{2.033753in}{1.517610in}}{\pgfqpoint{2.025939in}{1.509796in}}%
\pgfpathcurveto{\pgfqpoint{2.018125in}{1.501983in}}{\pgfqpoint{2.013735in}{1.491384in}}{\pgfqpoint{2.013735in}{1.480334in}}%
\pgfpathcurveto{\pgfqpoint{2.013735in}{1.469284in}}{\pgfqpoint{2.018125in}{1.458685in}}{\pgfqpoint{2.025939in}{1.450871in}}%
\pgfpathcurveto{\pgfqpoint{2.033753in}{1.443057in}}{\pgfqpoint{2.044352in}{1.438667in}}{\pgfqpoint{2.055402in}{1.438667in}}%
\pgfpathclose%
\pgfusepath{stroke,fill}%
\end{pgfscope}%
\begin{pgfscope}%
\pgfpathrectangle{\pgfqpoint{0.375000in}{0.330000in}}{\pgfqpoint{2.325000in}{2.310000in}}%
\pgfusepath{clip}%
\pgfsetbuttcap%
\pgfsetroundjoin%
\definecolor{currentfill}{rgb}{0.000000,0.000000,0.000000}%
\pgfsetfillcolor{currentfill}%
\pgfsetlinewidth{1.003750pt}%
\definecolor{currentstroke}{rgb}{0.000000,0.000000,0.000000}%
\pgfsetstrokecolor{currentstroke}%
\pgfsetdash{}{0pt}%
\pgfpathmoveto{\pgfqpoint{2.055402in}{1.438667in}}%
\pgfpathcurveto{\pgfqpoint{2.066452in}{1.438667in}}{\pgfqpoint{2.077051in}{1.443057in}}{\pgfqpoint{2.084865in}{1.450871in}}%
\pgfpathcurveto{\pgfqpoint{2.092678in}{1.458685in}}{\pgfqpoint{2.097069in}{1.469284in}}{\pgfqpoint{2.097069in}{1.480334in}}%
\pgfpathcurveto{\pgfqpoint{2.097069in}{1.491384in}}{\pgfqpoint{2.092678in}{1.501983in}}{\pgfqpoint{2.084865in}{1.509796in}}%
\pgfpathcurveto{\pgfqpoint{2.077051in}{1.517610in}}{\pgfqpoint{2.066452in}{1.522000in}}{\pgfqpoint{2.055402in}{1.522000in}}%
\pgfpathcurveto{\pgfqpoint{2.044352in}{1.522000in}}{\pgfqpoint{2.033753in}{1.517610in}}{\pgfqpoint{2.025939in}{1.509796in}}%
\pgfpathcurveto{\pgfqpoint{2.018125in}{1.501983in}}{\pgfqpoint{2.013735in}{1.491384in}}{\pgfqpoint{2.013735in}{1.480334in}}%
\pgfpathcurveto{\pgfqpoint{2.013735in}{1.469284in}}{\pgfqpoint{2.018125in}{1.458685in}}{\pgfqpoint{2.025939in}{1.450871in}}%
\pgfpathcurveto{\pgfqpoint{2.033753in}{1.443057in}}{\pgfqpoint{2.044352in}{1.438667in}}{\pgfqpoint{2.055402in}{1.438667in}}%
\pgfpathclose%
\pgfusepath{stroke,fill}%
\end{pgfscope}%
\begin{pgfscope}%
\pgfpathrectangle{\pgfqpoint{0.375000in}{0.330000in}}{\pgfqpoint{2.325000in}{2.310000in}}%
\pgfusepath{clip}%
\pgfsetbuttcap%
\pgfsetroundjoin%
\definecolor{currentfill}{rgb}{0.000000,0.000000,0.000000}%
\pgfsetfillcolor{currentfill}%
\pgfsetlinewidth{1.003750pt}%
\definecolor{currentstroke}{rgb}{0.000000,0.000000,0.000000}%
\pgfsetstrokecolor{currentstroke}%
\pgfsetdash{}{0pt}%
\pgfpathmoveto{\pgfqpoint{2.055402in}{1.438667in}}%
\pgfpathcurveto{\pgfqpoint{2.066452in}{1.438667in}}{\pgfqpoint{2.077051in}{1.443057in}}{\pgfqpoint{2.084865in}{1.450871in}}%
\pgfpathcurveto{\pgfqpoint{2.092678in}{1.458685in}}{\pgfqpoint{2.097069in}{1.469284in}}{\pgfqpoint{2.097069in}{1.480334in}}%
\pgfpathcurveto{\pgfqpoint{2.097069in}{1.491384in}}{\pgfqpoint{2.092678in}{1.501983in}}{\pgfqpoint{2.084865in}{1.509796in}}%
\pgfpathcurveto{\pgfqpoint{2.077051in}{1.517610in}}{\pgfqpoint{2.066452in}{1.522000in}}{\pgfqpoint{2.055402in}{1.522000in}}%
\pgfpathcurveto{\pgfqpoint{2.044352in}{1.522000in}}{\pgfqpoint{2.033753in}{1.517610in}}{\pgfqpoint{2.025939in}{1.509796in}}%
\pgfpathcurveto{\pgfqpoint{2.018125in}{1.501983in}}{\pgfqpoint{2.013735in}{1.491384in}}{\pgfqpoint{2.013735in}{1.480334in}}%
\pgfpathcurveto{\pgfqpoint{2.013735in}{1.469284in}}{\pgfqpoint{2.018125in}{1.458685in}}{\pgfqpoint{2.025939in}{1.450871in}}%
\pgfpathcurveto{\pgfqpoint{2.033753in}{1.443057in}}{\pgfqpoint{2.044352in}{1.438667in}}{\pgfqpoint{2.055402in}{1.438667in}}%
\pgfpathclose%
\pgfusepath{stroke,fill}%
\end{pgfscope}%
\begin{pgfscope}%
\pgfpathrectangle{\pgfqpoint{0.375000in}{0.330000in}}{\pgfqpoint{2.325000in}{2.310000in}}%
\pgfusepath{clip}%
\pgfsetbuttcap%
\pgfsetroundjoin%
\definecolor{currentfill}{rgb}{0.000000,0.000000,0.000000}%
\pgfsetfillcolor{currentfill}%
\pgfsetlinewidth{1.003750pt}%
\definecolor{currentstroke}{rgb}{0.000000,0.000000,0.000000}%
\pgfsetstrokecolor{currentstroke}%
\pgfsetdash{}{0pt}%
\pgfpathmoveto{\pgfqpoint{2.055402in}{1.438667in}}%
\pgfpathcurveto{\pgfqpoint{2.066452in}{1.438667in}}{\pgfqpoint{2.077051in}{1.443057in}}{\pgfqpoint{2.084865in}{1.450871in}}%
\pgfpathcurveto{\pgfqpoint{2.092678in}{1.458685in}}{\pgfqpoint{2.097069in}{1.469284in}}{\pgfqpoint{2.097069in}{1.480334in}}%
\pgfpathcurveto{\pgfqpoint{2.097069in}{1.491384in}}{\pgfqpoint{2.092678in}{1.501983in}}{\pgfqpoint{2.084865in}{1.509796in}}%
\pgfpathcurveto{\pgfqpoint{2.077051in}{1.517610in}}{\pgfqpoint{2.066452in}{1.522000in}}{\pgfqpoint{2.055402in}{1.522000in}}%
\pgfpathcurveto{\pgfqpoint{2.044352in}{1.522000in}}{\pgfqpoint{2.033753in}{1.517610in}}{\pgfqpoint{2.025939in}{1.509796in}}%
\pgfpathcurveto{\pgfqpoint{2.018125in}{1.501983in}}{\pgfqpoint{2.013735in}{1.491384in}}{\pgfqpoint{2.013735in}{1.480334in}}%
\pgfpathcurveto{\pgfqpoint{2.013735in}{1.469284in}}{\pgfqpoint{2.018125in}{1.458685in}}{\pgfqpoint{2.025939in}{1.450871in}}%
\pgfpathcurveto{\pgfqpoint{2.033753in}{1.443057in}}{\pgfqpoint{2.044352in}{1.438667in}}{\pgfqpoint{2.055402in}{1.438667in}}%
\pgfpathclose%
\pgfusepath{stroke,fill}%
\end{pgfscope}%
\begin{pgfscope}%
\pgfpathrectangle{\pgfqpoint{0.375000in}{0.330000in}}{\pgfqpoint{2.325000in}{2.310000in}}%
\pgfusepath{clip}%
\pgfsetbuttcap%
\pgfsetroundjoin%
\definecolor{currentfill}{rgb}{0.000000,0.000000,0.000000}%
\pgfsetfillcolor{currentfill}%
\pgfsetlinewidth{1.003750pt}%
\definecolor{currentstroke}{rgb}{0.000000,0.000000,0.000000}%
\pgfsetstrokecolor{currentstroke}%
\pgfsetdash{}{0pt}%
\pgfpathmoveto{\pgfqpoint{2.055402in}{1.438667in}}%
\pgfpathcurveto{\pgfqpoint{2.066452in}{1.438667in}}{\pgfqpoint{2.077051in}{1.443057in}}{\pgfqpoint{2.084865in}{1.450871in}}%
\pgfpathcurveto{\pgfqpoint{2.092678in}{1.458685in}}{\pgfqpoint{2.097069in}{1.469284in}}{\pgfqpoint{2.097069in}{1.480334in}}%
\pgfpathcurveto{\pgfqpoint{2.097069in}{1.491384in}}{\pgfqpoint{2.092678in}{1.501983in}}{\pgfqpoint{2.084865in}{1.509796in}}%
\pgfpathcurveto{\pgfqpoint{2.077051in}{1.517610in}}{\pgfqpoint{2.066452in}{1.522000in}}{\pgfqpoint{2.055402in}{1.522000in}}%
\pgfpathcurveto{\pgfqpoint{2.044352in}{1.522000in}}{\pgfqpoint{2.033753in}{1.517610in}}{\pgfqpoint{2.025939in}{1.509796in}}%
\pgfpathcurveto{\pgfqpoint{2.018125in}{1.501983in}}{\pgfqpoint{2.013735in}{1.491384in}}{\pgfqpoint{2.013735in}{1.480334in}}%
\pgfpathcurveto{\pgfqpoint{2.013735in}{1.469284in}}{\pgfqpoint{2.018125in}{1.458685in}}{\pgfqpoint{2.025939in}{1.450871in}}%
\pgfpathcurveto{\pgfqpoint{2.033753in}{1.443057in}}{\pgfqpoint{2.044352in}{1.438667in}}{\pgfqpoint{2.055402in}{1.438667in}}%
\pgfpathclose%
\pgfusepath{stroke,fill}%
\end{pgfscope}%
\begin{pgfscope}%
\pgfpathrectangle{\pgfqpoint{0.375000in}{0.330000in}}{\pgfqpoint{2.325000in}{2.310000in}}%
\pgfusepath{clip}%
\pgfsetbuttcap%
\pgfsetroundjoin%
\definecolor{currentfill}{rgb}{0.000000,0.000000,0.000000}%
\pgfsetfillcolor{currentfill}%
\pgfsetlinewidth{1.003750pt}%
\definecolor{currentstroke}{rgb}{0.000000,0.000000,0.000000}%
\pgfsetstrokecolor{currentstroke}%
\pgfsetdash{}{0pt}%
\pgfpathmoveto{\pgfqpoint{2.055402in}{1.438667in}}%
\pgfpathcurveto{\pgfqpoint{2.066452in}{1.438667in}}{\pgfqpoint{2.077051in}{1.443057in}}{\pgfqpoint{2.084865in}{1.450871in}}%
\pgfpathcurveto{\pgfqpoint{2.092678in}{1.458685in}}{\pgfqpoint{2.097069in}{1.469284in}}{\pgfqpoint{2.097069in}{1.480334in}}%
\pgfpathcurveto{\pgfqpoint{2.097069in}{1.491384in}}{\pgfqpoint{2.092678in}{1.501983in}}{\pgfqpoint{2.084865in}{1.509796in}}%
\pgfpathcurveto{\pgfqpoint{2.077051in}{1.517610in}}{\pgfqpoint{2.066452in}{1.522000in}}{\pgfqpoint{2.055402in}{1.522000in}}%
\pgfpathcurveto{\pgfqpoint{2.044352in}{1.522000in}}{\pgfqpoint{2.033753in}{1.517610in}}{\pgfqpoint{2.025939in}{1.509796in}}%
\pgfpathcurveto{\pgfqpoint{2.018125in}{1.501983in}}{\pgfqpoint{2.013735in}{1.491384in}}{\pgfqpoint{2.013735in}{1.480334in}}%
\pgfpathcurveto{\pgfqpoint{2.013735in}{1.469284in}}{\pgfqpoint{2.018125in}{1.458685in}}{\pgfqpoint{2.025939in}{1.450871in}}%
\pgfpathcurveto{\pgfqpoint{2.033753in}{1.443057in}}{\pgfqpoint{2.044352in}{1.438667in}}{\pgfqpoint{2.055402in}{1.438667in}}%
\pgfpathclose%
\pgfusepath{stroke,fill}%
\end{pgfscope}%
\begin{pgfscope}%
\pgfpathrectangle{\pgfqpoint{0.375000in}{0.330000in}}{\pgfqpoint{2.325000in}{2.310000in}}%
\pgfusepath{clip}%
\pgfsetbuttcap%
\pgfsetroundjoin%
\definecolor{currentfill}{rgb}{0.000000,0.000000,0.000000}%
\pgfsetfillcolor{currentfill}%
\pgfsetlinewidth{1.003750pt}%
\definecolor{currentstroke}{rgb}{0.000000,0.000000,0.000000}%
\pgfsetstrokecolor{currentstroke}%
\pgfsetdash{}{0pt}%
\pgfpathmoveto{\pgfqpoint{2.055402in}{1.438667in}}%
\pgfpathcurveto{\pgfqpoint{2.066452in}{1.438667in}}{\pgfqpoint{2.077051in}{1.443057in}}{\pgfqpoint{2.084865in}{1.450871in}}%
\pgfpathcurveto{\pgfqpoint{2.092678in}{1.458685in}}{\pgfqpoint{2.097069in}{1.469284in}}{\pgfqpoint{2.097069in}{1.480334in}}%
\pgfpathcurveto{\pgfqpoint{2.097069in}{1.491384in}}{\pgfqpoint{2.092678in}{1.501983in}}{\pgfqpoint{2.084865in}{1.509796in}}%
\pgfpathcurveto{\pgfqpoint{2.077051in}{1.517610in}}{\pgfqpoint{2.066452in}{1.522000in}}{\pgfqpoint{2.055402in}{1.522000in}}%
\pgfpathcurveto{\pgfqpoint{2.044352in}{1.522000in}}{\pgfqpoint{2.033753in}{1.517610in}}{\pgfqpoint{2.025939in}{1.509796in}}%
\pgfpathcurveto{\pgfqpoint{2.018125in}{1.501983in}}{\pgfqpoint{2.013735in}{1.491384in}}{\pgfqpoint{2.013735in}{1.480334in}}%
\pgfpathcurveto{\pgfqpoint{2.013735in}{1.469284in}}{\pgfqpoint{2.018125in}{1.458685in}}{\pgfqpoint{2.025939in}{1.450871in}}%
\pgfpathcurveto{\pgfqpoint{2.033753in}{1.443057in}}{\pgfqpoint{2.044352in}{1.438667in}}{\pgfqpoint{2.055402in}{1.438667in}}%
\pgfpathclose%
\pgfusepath{stroke,fill}%
\end{pgfscope}%
\begin{pgfscope}%
\pgfpathrectangle{\pgfqpoint{0.375000in}{0.330000in}}{\pgfqpoint{2.325000in}{2.310000in}}%
\pgfusepath{clip}%
\pgfsetbuttcap%
\pgfsetroundjoin%
\definecolor{currentfill}{rgb}{0.000000,0.000000,0.000000}%
\pgfsetfillcolor{currentfill}%
\pgfsetlinewidth{1.003750pt}%
\definecolor{currentstroke}{rgb}{0.000000,0.000000,0.000000}%
\pgfsetstrokecolor{currentstroke}%
\pgfsetdash{}{0pt}%
\pgfpathmoveto{\pgfqpoint{2.055402in}{1.438667in}}%
\pgfpathcurveto{\pgfqpoint{2.066452in}{1.438667in}}{\pgfqpoint{2.077051in}{1.443057in}}{\pgfqpoint{2.084865in}{1.450871in}}%
\pgfpathcurveto{\pgfqpoint{2.092678in}{1.458685in}}{\pgfqpoint{2.097069in}{1.469284in}}{\pgfqpoint{2.097069in}{1.480334in}}%
\pgfpathcurveto{\pgfqpoint{2.097069in}{1.491384in}}{\pgfqpoint{2.092678in}{1.501983in}}{\pgfqpoint{2.084865in}{1.509796in}}%
\pgfpathcurveto{\pgfqpoint{2.077051in}{1.517610in}}{\pgfqpoint{2.066452in}{1.522000in}}{\pgfqpoint{2.055402in}{1.522000in}}%
\pgfpathcurveto{\pgfqpoint{2.044352in}{1.522000in}}{\pgfqpoint{2.033753in}{1.517610in}}{\pgfqpoint{2.025939in}{1.509796in}}%
\pgfpathcurveto{\pgfqpoint{2.018125in}{1.501983in}}{\pgfqpoint{2.013735in}{1.491384in}}{\pgfqpoint{2.013735in}{1.480334in}}%
\pgfpathcurveto{\pgfqpoint{2.013735in}{1.469284in}}{\pgfqpoint{2.018125in}{1.458685in}}{\pgfqpoint{2.025939in}{1.450871in}}%
\pgfpathcurveto{\pgfqpoint{2.033753in}{1.443057in}}{\pgfqpoint{2.044352in}{1.438667in}}{\pgfqpoint{2.055402in}{1.438667in}}%
\pgfpathclose%
\pgfusepath{stroke,fill}%
\end{pgfscope}%
\begin{pgfscope}%
\pgfpathrectangle{\pgfqpoint{0.375000in}{0.330000in}}{\pgfqpoint{2.325000in}{2.310000in}}%
\pgfusepath{clip}%
\pgfsetbuttcap%
\pgfsetroundjoin%
\definecolor{currentfill}{rgb}{0.000000,0.000000,0.000000}%
\pgfsetfillcolor{currentfill}%
\pgfsetlinewidth{1.003750pt}%
\definecolor{currentstroke}{rgb}{0.000000,0.000000,0.000000}%
\pgfsetstrokecolor{currentstroke}%
\pgfsetdash{}{0pt}%
\pgfpathmoveto{\pgfqpoint{2.055402in}{1.438667in}}%
\pgfpathcurveto{\pgfqpoint{2.066452in}{1.438667in}}{\pgfqpoint{2.077051in}{1.443057in}}{\pgfqpoint{2.084865in}{1.450871in}}%
\pgfpathcurveto{\pgfqpoint{2.092678in}{1.458685in}}{\pgfqpoint{2.097069in}{1.469284in}}{\pgfqpoint{2.097069in}{1.480334in}}%
\pgfpathcurveto{\pgfqpoint{2.097069in}{1.491384in}}{\pgfqpoint{2.092678in}{1.501983in}}{\pgfqpoint{2.084865in}{1.509796in}}%
\pgfpathcurveto{\pgfqpoint{2.077051in}{1.517610in}}{\pgfqpoint{2.066452in}{1.522000in}}{\pgfqpoint{2.055402in}{1.522000in}}%
\pgfpathcurveto{\pgfqpoint{2.044352in}{1.522000in}}{\pgfqpoint{2.033753in}{1.517610in}}{\pgfqpoint{2.025939in}{1.509796in}}%
\pgfpathcurveto{\pgfqpoint{2.018125in}{1.501983in}}{\pgfqpoint{2.013735in}{1.491384in}}{\pgfqpoint{2.013735in}{1.480334in}}%
\pgfpathcurveto{\pgfqpoint{2.013735in}{1.469284in}}{\pgfqpoint{2.018125in}{1.458685in}}{\pgfqpoint{2.025939in}{1.450871in}}%
\pgfpathcurveto{\pgfqpoint{2.033753in}{1.443057in}}{\pgfqpoint{2.044352in}{1.438667in}}{\pgfqpoint{2.055402in}{1.438667in}}%
\pgfpathclose%
\pgfusepath{stroke,fill}%
\end{pgfscope}%
\begin{pgfscope}%
\pgfpathrectangle{\pgfqpoint{0.375000in}{0.330000in}}{\pgfqpoint{2.325000in}{2.310000in}}%
\pgfusepath{clip}%
\pgfsetbuttcap%
\pgfsetroundjoin%
\definecolor{currentfill}{rgb}{0.000000,0.000000,0.000000}%
\pgfsetfillcolor{currentfill}%
\pgfsetlinewidth{1.003750pt}%
\definecolor{currentstroke}{rgb}{0.000000,0.000000,0.000000}%
\pgfsetstrokecolor{currentstroke}%
\pgfsetdash{}{0pt}%
\pgfpathmoveto{\pgfqpoint{2.055402in}{1.438667in}}%
\pgfpathcurveto{\pgfqpoint{2.066452in}{1.438667in}}{\pgfqpoint{2.077051in}{1.443057in}}{\pgfqpoint{2.084865in}{1.450871in}}%
\pgfpathcurveto{\pgfqpoint{2.092678in}{1.458685in}}{\pgfqpoint{2.097069in}{1.469284in}}{\pgfqpoint{2.097069in}{1.480334in}}%
\pgfpathcurveto{\pgfqpoint{2.097069in}{1.491384in}}{\pgfqpoint{2.092678in}{1.501983in}}{\pgfqpoint{2.084865in}{1.509796in}}%
\pgfpathcurveto{\pgfqpoint{2.077051in}{1.517610in}}{\pgfqpoint{2.066452in}{1.522000in}}{\pgfqpoint{2.055402in}{1.522000in}}%
\pgfpathcurveto{\pgfqpoint{2.044352in}{1.522000in}}{\pgfqpoint{2.033753in}{1.517610in}}{\pgfqpoint{2.025939in}{1.509796in}}%
\pgfpathcurveto{\pgfqpoint{2.018125in}{1.501983in}}{\pgfqpoint{2.013735in}{1.491384in}}{\pgfqpoint{2.013735in}{1.480334in}}%
\pgfpathcurveto{\pgfqpoint{2.013735in}{1.469284in}}{\pgfqpoint{2.018125in}{1.458685in}}{\pgfqpoint{2.025939in}{1.450871in}}%
\pgfpathcurveto{\pgfqpoint{2.033753in}{1.443057in}}{\pgfqpoint{2.044352in}{1.438667in}}{\pgfqpoint{2.055402in}{1.438667in}}%
\pgfpathclose%
\pgfusepath{stroke,fill}%
\end{pgfscope}%
\begin{pgfscope}%
\pgfpathrectangle{\pgfqpoint{0.375000in}{0.330000in}}{\pgfqpoint{2.325000in}{2.310000in}}%
\pgfusepath{clip}%
\pgfsetbuttcap%
\pgfsetroundjoin%
\definecolor{currentfill}{rgb}{0.000000,0.000000,0.000000}%
\pgfsetfillcolor{currentfill}%
\pgfsetlinewidth{1.003750pt}%
\definecolor{currentstroke}{rgb}{0.000000,0.000000,0.000000}%
\pgfsetstrokecolor{currentstroke}%
\pgfsetdash{}{0pt}%
\pgfpathmoveto{\pgfqpoint{2.055402in}{1.438667in}}%
\pgfpathcurveto{\pgfqpoint{2.066452in}{1.438667in}}{\pgfqpoint{2.077051in}{1.443057in}}{\pgfqpoint{2.084865in}{1.450871in}}%
\pgfpathcurveto{\pgfqpoint{2.092678in}{1.458685in}}{\pgfqpoint{2.097069in}{1.469284in}}{\pgfqpoint{2.097069in}{1.480334in}}%
\pgfpathcurveto{\pgfqpoint{2.097069in}{1.491384in}}{\pgfqpoint{2.092678in}{1.501983in}}{\pgfqpoint{2.084865in}{1.509796in}}%
\pgfpathcurveto{\pgfqpoint{2.077051in}{1.517610in}}{\pgfqpoint{2.066452in}{1.522000in}}{\pgfqpoint{2.055402in}{1.522000in}}%
\pgfpathcurveto{\pgfqpoint{2.044352in}{1.522000in}}{\pgfqpoint{2.033753in}{1.517610in}}{\pgfqpoint{2.025939in}{1.509796in}}%
\pgfpathcurveto{\pgfqpoint{2.018125in}{1.501983in}}{\pgfqpoint{2.013735in}{1.491384in}}{\pgfqpoint{2.013735in}{1.480334in}}%
\pgfpathcurveto{\pgfqpoint{2.013735in}{1.469284in}}{\pgfqpoint{2.018125in}{1.458685in}}{\pgfqpoint{2.025939in}{1.450871in}}%
\pgfpathcurveto{\pgfqpoint{2.033753in}{1.443057in}}{\pgfqpoint{2.044352in}{1.438667in}}{\pgfqpoint{2.055402in}{1.438667in}}%
\pgfpathclose%
\pgfusepath{stroke,fill}%
\end{pgfscope}%
\begin{pgfscope}%
\pgfpathrectangle{\pgfqpoint{0.375000in}{0.330000in}}{\pgfqpoint{2.325000in}{2.310000in}}%
\pgfusepath{clip}%
\pgfsetbuttcap%
\pgfsetroundjoin%
\definecolor{currentfill}{rgb}{0.000000,0.000000,0.000000}%
\pgfsetfillcolor{currentfill}%
\pgfsetlinewidth{1.003750pt}%
\definecolor{currentstroke}{rgb}{0.000000,0.000000,0.000000}%
\pgfsetstrokecolor{currentstroke}%
\pgfsetdash{}{0pt}%
\pgfpathmoveto{\pgfqpoint{2.055402in}{1.438667in}}%
\pgfpathcurveto{\pgfqpoint{2.066452in}{1.438667in}}{\pgfqpoint{2.077051in}{1.443057in}}{\pgfqpoint{2.084865in}{1.450871in}}%
\pgfpathcurveto{\pgfqpoint{2.092678in}{1.458685in}}{\pgfqpoint{2.097069in}{1.469284in}}{\pgfqpoint{2.097069in}{1.480334in}}%
\pgfpathcurveto{\pgfqpoint{2.097069in}{1.491384in}}{\pgfqpoint{2.092678in}{1.501983in}}{\pgfqpoint{2.084865in}{1.509796in}}%
\pgfpathcurveto{\pgfqpoint{2.077051in}{1.517610in}}{\pgfqpoint{2.066452in}{1.522000in}}{\pgfqpoint{2.055402in}{1.522000in}}%
\pgfpathcurveto{\pgfqpoint{2.044352in}{1.522000in}}{\pgfqpoint{2.033753in}{1.517610in}}{\pgfqpoint{2.025939in}{1.509796in}}%
\pgfpathcurveto{\pgfqpoint{2.018125in}{1.501983in}}{\pgfqpoint{2.013735in}{1.491384in}}{\pgfqpoint{2.013735in}{1.480334in}}%
\pgfpathcurveto{\pgfqpoint{2.013735in}{1.469284in}}{\pgfqpoint{2.018125in}{1.458685in}}{\pgfqpoint{2.025939in}{1.450871in}}%
\pgfpathcurveto{\pgfqpoint{2.033753in}{1.443057in}}{\pgfqpoint{2.044352in}{1.438667in}}{\pgfqpoint{2.055402in}{1.438667in}}%
\pgfpathclose%
\pgfusepath{stroke,fill}%
\end{pgfscope}%
\begin{pgfscope}%
\pgfpathrectangle{\pgfqpoint{0.375000in}{0.330000in}}{\pgfqpoint{2.325000in}{2.310000in}}%
\pgfusepath{clip}%
\pgfsetbuttcap%
\pgfsetroundjoin%
\definecolor{currentfill}{rgb}{0.000000,0.000000,0.000000}%
\pgfsetfillcolor{currentfill}%
\pgfsetlinewidth{1.003750pt}%
\definecolor{currentstroke}{rgb}{0.000000,0.000000,0.000000}%
\pgfsetstrokecolor{currentstroke}%
\pgfsetdash{}{0pt}%
\pgfpathmoveto{\pgfqpoint{2.055402in}{1.438667in}}%
\pgfpathcurveto{\pgfqpoint{2.066452in}{1.438667in}}{\pgfqpoint{2.077051in}{1.443057in}}{\pgfqpoint{2.084865in}{1.450871in}}%
\pgfpathcurveto{\pgfqpoint{2.092678in}{1.458685in}}{\pgfqpoint{2.097069in}{1.469284in}}{\pgfqpoint{2.097069in}{1.480334in}}%
\pgfpathcurveto{\pgfqpoint{2.097069in}{1.491384in}}{\pgfqpoint{2.092678in}{1.501983in}}{\pgfqpoint{2.084865in}{1.509796in}}%
\pgfpathcurveto{\pgfqpoint{2.077051in}{1.517610in}}{\pgfqpoint{2.066452in}{1.522000in}}{\pgfqpoint{2.055402in}{1.522000in}}%
\pgfpathcurveto{\pgfqpoint{2.044352in}{1.522000in}}{\pgfqpoint{2.033753in}{1.517610in}}{\pgfqpoint{2.025939in}{1.509796in}}%
\pgfpathcurveto{\pgfqpoint{2.018125in}{1.501983in}}{\pgfqpoint{2.013735in}{1.491384in}}{\pgfqpoint{2.013735in}{1.480334in}}%
\pgfpathcurveto{\pgfqpoint{2.013735in}{1.469284in}}{\pgfqpoint{2.018125in}{1.458685in}}{\pgfqpoint{2.025939in}{1.450871in}}%
\pgfpathcurveto{\pgfqpoint{2.033753in}{1.443057in}}{\pgfqpoint{2.044352in}{1.438667in}}{\pgfqpoint{2.055402in}{1.438667in}}%
\pgfpathclose%
\pgfusepath{stroke,fill}%
\end{pgfscope}%
\begin{pgfscope}%
\pgfpathrectangle{\pgfqpoint{0.375000in}{0.330000in}}{\pgfqpoint{2.325000in}{2.310000in}}%
\pgfusepath{clip}%
\pgfsetbuttcap%
\pgfsetroundjoin%
\definecolor{currentfill}{rgb}{0.000000,0.000000,0.000000}%
\pgfsetfillcolor{currentfill}%
\pgfsetlinewidth{1.003750pt}%
\definecolor{currentstroke}{rgb}{0.000000,0.000000,0.000000}%
\pgfsetstrokecolor{currentstroke}%
\pgfsetdash{}{0pt}%
\pgfpathmoveto{\pgfqpoint{2.055402in}{1.438667in}}%
\pgfpathcurveto{\pgfqpoint{2.066452in}{1.438667in}}{\pgfqpoint{2.077051in}{1.443057in}}{\pgfqpoint{2.084865in}{1.450871in}}%
\pgfpathcurveto{\pgfqpoint{2.092678in}{1.458685in}}{\pgfqpoint{2.097069in}{1.469284in}}{\pgfqpoint{2.097069in}{1.480334in}}%
\pgfpathcurveto{\pgfqpoint{2.097069in}{1.491384in}}{\pgfqpoint{2.092678in}{1.501983in}}{\pgfqpoint{2.084865in}{1.509796in}}%
\pgfpathcurveto{\pgfqpoint{2.077051in}{1.517610in}}{\pgfqpoint{2.066452in}{1.522000in}}{\pgfqpoint{2.055402in}{1.522000in}}%
\pgfpathcurveto{\pgfqpoint{2.044352in}{1.522000in}}{\pgfqpoint{2.033753in}{1.517610in}}{\pgfqpoint{2.025939in}{1.509796in}}%
\pgfpathcurveto{\pgfqpoint{2.018125in}{1.501983in}}{\pgfqpoint{2.013735in}{1.491384in}}{\pgfqpoint{2.013735in}{1.480334in}}%
\pgfpathcurveto{\pgfqpoint{2.013735in}{1.469284in}}{\pgfqpoint{2.018125in}{1.458685in}}{\pgfqpoint{2.025939in}{1.450871in}}%
\pgfpathcurveto{\pgfqpoint{2.033753in}{1.443057in}}{\pgfqpoint{2.044352in}{1.438667in}}{\pgfqpoint{2.055402in}{1.438667in}}%
\pgfpathclose%
\pgfusepath{stroke,fill}%
\end{pgfscope}%
\begin{pgfscope}%
\pgfpathrectangle{\pgfqpoint{0.375000in}{0.330000in}}{\pgfqpoint{2.325000in}{2.310000in}}%
\pgfusepath{clip}%
\pgfsetbuttcap%
\pgfsetroundjoin%
\definecolor{currentfill}{rgb}{0.000000,0.000000,0.000000}%
\pgfsetfillcolor{currentfill}%
\pgfsetlinewidth{1.003750pt}%
\definecolor{currentstroke}{rgb}{0.000000,0.000000,0.000000}%
\pgfsetstrokecolor{currentstroke}%
\pgfsetdash{}{0pt}%
\pgfpathmoveto{\pgfqpoint{2.055402in}{1.438667in}}%
\pgfpathcurveto{\pgfqpoint{2.066452in}{1.438667in}}{\pgfqpoint{2.077051in}{1.443057in}}{\pgfqpoint{2.084865in}{1.450871in}}%
\pgfpathcurveto{\pgfqpoint{2.092678in}{1.458685in}}{\pgfqpoint{2.097069in}{1.469284in}}{\pgfqpoint{2.097069in}{1.480334in}}%
\pgfpathcurveto{\pgfqpoint{2.097069in}{1.491384in}}{\pgfqpoint{2.092678in}{1.501983in}}{\pgfqpoint{2.084865in}{1.509796in}}%
\pgfpathcurveto{\pgfqpoint{2.077051in}{1.517610in}}{\pgfqpoint{2.066452in}{1.522000in}}{\pgfqpoint{2.055402in}{1.522000in}}%
\pgfpathcurveto{\pgfqpoint{2.044352in}{1.522000in}}{\pgfqpoint{2.033753in}{1.517610in}}{\pgfqpoint{2.025939in}{1.509796in}}%
\pgfpathcurveto{\pgfqpoint{2.018125in}{1.501983in}}{\pgfqpoint{2.013735in}{1.491384in}}{\pgfqpoint{2.013735in}{1.480334in}}%
\pgfpathcurveto{\pgfqpoint{2.013735in}{1.469284in}}{\pgfqpoint{2.018125in}{1.458685in}}{\pgfqpoint{2.025939in}{1.450871in}}%
\pgfpathcurveto{\pgfqpoint{2.033753in}{1.443057in}}{\pgfqpoint{2.044352in}{1.438667in}}{\pgfqpoint{2.055402in}{1.438667in}}%
\pgfpathclose%
\pgfusepath{stroke,fill}%
\end{pgfscope}%
\begin{pgfscope}%
\pgfpathrectangle{\pgfqpoint{0.375000in}{0.330000in}}{\pgfqpoint{2.325000in}{2.310000in}}%
\pgfusepath{clip}%
\pgfsetbuttcap%
\pgfsetroundjoin%
\definecolor{currentfill}{rgb}{0.000000,0.000000,0.000000}%
\pgfsetfillcolor{currentfill}%
\pgfsetlinewidth{1.003750pt}%
\definecolor{currentstroke}{rgb}{0.000000,0.000000,0.000000}%
\pgfsetstrokecolor{currentstroke}%
\pgfsetdash{}{0pt}%
\pgfpathmoveto{\pgfqpoint{2.055402in}{1.438667in}}%
\pgfpathcurveto{\pgfqpoint{2.066452in}{1.438667in}}{\pgfqpoint{2.077051in}{1.443057in}}{\pgfqpoint{2.084865in}{1.450871in}}%
\pgfpathcurveto{\pgfqpoint{2.092678in}{1.458685in}}{\pgfqpoint{2.097069in}{1.469284in}}{\pgfqpoint{2.097069in}{1.480334in}}%
\pgfpathcurveto{\pgfqpoint{2.097069in}{1.491384in}}{\pgfqpoint{2.092678in}{1.501983in}}{\pgfqpoint{2.084865in}{1.509796in}}%
\pgfpathcurveto{\pgfqpoint{2.077051in}{1.517610in}}{\pgfqpoint{2.066452in}{1.522000in}}{\pgfqpoint{2.055402in}{1.522000in}}%
\pgfpathcurveto{\pgfqpoint{2.044352in}{1.522000in}}{\pgfqpoint{2.033753in}{1.517610in}}{\pgfqpoint{2.025939in}{1.509796in}}%
\pgfpathcurveto{\pgfqpoint{2.018125in}{1.501983in}}{\pgfqpoint{2.013735in}{1.491384in}}{\pgfqpoint{2.013735in}{1.480334in}}%
\pgfpathcurveto{\pgfqpoint{2.013735in}{1.469284in}}{\pgfqpoint{2.018125in}{1.458685in}}{\pgfqpoint{2.025939in}{1.450871in}}%
\pgfpathcurveto{\pgfqpoint{2.033753in}{1.443057in}}{\pgfqpoint{2.044352in}{1.438667in}}{\pgfqpoint{2.055402in}{1.438667in}}%
\pgfpathclose%
\pgfusepath{stroke,fill}%
\end{pgfscope}%
\begin{pgfscope}%
\pgfpathrectangle{\pgfqpoint{0.375000in}{0.330000in}}{\pgfqpoint{2.325000in}{2.310000in}}%
\pgfusepath{clip}%
\pgfsetbuttcap%
\pgfsetroundjoin%
\definecolor{currentfill}{rgb}{0.000000,0.000000,0.000000}%
\pgfsetfillcolor{currentfill}%
\pgfsetlinewidth{1.003750pt}%
\definecolor{currentstroke}{rgb}{0.000000,0.000000,0.000000}%
\pgfsetstrokecolor{currentstroke}%
\pgfsetdash{}{0pt}%
\pgfpathmoveto{\pgfqpoint{2.055402in}{1.438667in}}%
\pgfpathcurveto{\pgfqpoint{2.066452in}{1.438667in}}{\pgfqpoint{2.077051in}{1.443057in}}{\pgfqpoint{2.084865in}{1.450871in}}%
\pgfpathcurveto{\pgfqpoint{2.092678in}{1.458685in}}{\pgfqpoint{2.097069in}{1.469284in}}{\pgfqpoint{2.097069in}{1.480334in}}%
\pgfpathcurveto{\pgfqpoint{2.097069in}{1.491384in}}{\pgfqpoint{2.092678in}{1.501983in}}{\pgfqpoint{2.084865in}{1.509796in}}%
\pgfpathcurveto{\pgfqpoint{2.077051in}{1.517610in}}{\pgfqpoint{2.066452in}{1.522000in}}{\pgfqpoint{2.055402in}{1.522000in}}%
\pgfpathcurveto{\pgfqpoint{2.044352in}{1.522000in}}{\pgfqpoint{2.033753in}{1.517610in}}{\pgfqpoint{2.025939in}{1.509796in}}%
\pgfpathcurveto{\pgfqpoint{2.018125in}{1.501983in}}{\pgfqpoint{2.013735in}{1.491384in}}{\pgfqpoint{2.013735in}{1.480334in}}%
\pgfpathcurveto{\pgfqpoint{2.013735in}{1.469284in}}{\pgfqpoint{2.018125in}{1.458685in}}{\pgfqpoint{2.025939in}{1.450871in}}%
\pgfpathcurveto{\pgfqpoint{2.033753in}{1.443057in}}{\pgfqpoint{2.044352in}{1.438667in}}{\pgfqpoint{2.055402in}{1.438667in}}%
\pgfpathclose%
\pgfusepath{stroke,fill}%
\end{pgfscope}%
\begin{pgfscope}%
\pgfpathrectangle{\pgfqpoint{0.375000in}{0.330000in}}{\pgfqpoint{2.325000in}{2.310000in}}%
\pgfusepath{clip}%
\pgfsetbuttcap%
\pgfsetroundjoin%
\definecolor{currentfill}{rgb}{0.000000,0.000000,0.000000}%
\pgfsetfillcolor{currentfill}%
\pgfsetlinewidth{1.003750pt}%
\definecolor{currentstroke}{rgb}{0.000000,0.000000,0.000000}%
\pgfsetstrokecolor{currentstroke}%
\pgfsetdash{}{0pt}%
\pgfpathmoveto{\pgfqpoint{2.055402in}{1.438667in}}%
\pgfpathcurveto{\pgfqpoint{2.066452in}{1.438667in}}{\pgfqpoint{2.077051in}{1.443057in}}{\pgfqpoint{2.084865in}{1.450871in}}%
\pgfpathcurveto{\pgfqpoint{2.092678in}{1.458685in}}{\pgfqpoint{2.097069in}{1.469284in}}{\pgfqpoint{2.097069in}{1.480334in}}%
\pgfpathcurveto{\pgfqpoint{2.097069in}{1.491384in}}{\pgfqpoint{2.092678in}{1.501983in}}{\pgfqpoint{2.084865in}{1.509796in}}%
\pgfpathcurveto{\pgfqpoint{2.077051in}{1.517610in}}{\pgfqpoint{2.066452in}{1.522000in}}{\pgfqpoint{2.055402in}{1.522000in}}%
\pgfpathcurveto{\pgfqpoint{2.044352in}{1.522000in}}{\pgfqpoint{2.033753in}{1.517610in}}{\pgfqpoint{2.025939in}{1.509796in}}%
\pgfpathcurveto{\pgfqpoint{2.018125in}{1.501983in}}{\pgfqpoint{2.013735in}{1.491384in}}{\pgfqpoint{2.013735in}{1.480334in}}%
\pgfpathcurveto{\pgfqpoint{2.013735in}{1.469284in}}{\pgfqpoint{2.018125in}{1.458685in}}{\pgfqpoint{2.025939in}{1.450871in}}%
\pgfpathcurveto{\pgfqpoint{2.033753in}{1.443057in}}{\pgfqpoint{2.044352in}{1.438667in}}{\pgfqpoint{2.055402in}{1.438667in}}%
\pgfpathclose%
\pgfusepath{stroke,fill}%
\end{pgfscope}%
\begin{pgfscope}%
\pgfpathrectangle{\pgfqpoint{0.375000in}{0.330000in}}{\pgfqpoint{2.325000in}{2.310000in}}%
\pgfusepath{clip}%
\pgfsetbuttcap%
\pgfsetroundjoin%
\definecolor{currentfill}{rgb}{0.000000,0.000000,0.000000}%
\pgfsetfillcolor{currentfill}%
\pgfsetlinewidth{1.003750pt}%
\definecolor{currentstroke}{rgb}{0.000000,0.000000,0.000000}%
\pgfsetstrokecolor{currentstroke}%
\pgfsetdash{}{0pt}%
\pgfpathmoveto{\pgfqpoint{2.055402in}{1.438667in}}%
\pgfpathcurveto{\pgfqpoint{2.066452in}{1.438667in}}{\pgfqpoint{2.077051in}{1.443057in}}{\pgfqpoint{2.084865in}{1.450871in}}%
\pgfpathcurveto{\pgfqpoint{2.092678in}{1.458685in}}{\pgfqpoint{2.097069in}{1.469284in}}{\pgfqpoint{2.097069in}{1.480334in}}%
\pgfpathcurveto{\pgfqpoint{2.097069in}{1.491384in}}{\pgfqpoint{2.092678in}{1.501983in}}{\pgfqpoint{2.084865in}{1.509796in}}%
\pgfpathcurveto{\pgfqpoint{2.077051in}{1.517610in}}{\pgfqpoint{2.066452in}{1.522000in}}{\pgfqpoint{2.055402in}{1.522000in}}%
\pgfpathcurveto{\pgfqpoint{2.044352in}{1.522000in}}{\pgfqpoint{2.033753in}{1.517610in}}{\pgfqpoint{2.025939in}{1.509796in}}%
\pgfpathcurveto{\pgfqpoint{2.018125in}{1.501983in}}{\pgfqpoint{2.013735in}{1.491384in}}{\pgfqpoint{2.013735in}{1.480334in}}%
\pgfpathcurveto{\pgfqpoint{2.013735in}{1.469284in}}{\pgfqpoint{2.018125in}{1.458685in}}{\pgfqpoint{2.025939in}{1.450871in}}%
\pgfpathcurveto{\pgfqpoint{2.033753in}{1.443057in}}{\pgfqpoint{2.044352in}{1.438667in}}{\pgfqpoint{2.055402in}{1.438667in}}%
\pgfpathclose%
\pgfusepath{stroke,fill}%
\end{pgfscope}%
\begin{pgfscope}%
\pgfpathrectangle{\pgfqpoint{0.375000in}{0.330000in}}{\pgfqpoint{2.325000in}{2.310000in}}%
\pgfusepath{clip}%
\pgfsetbuttcap%
\pgfsetroundjoin%
\definecolor{currentfill}{rgb}{0.000000,0.000000,0.000000}%
\pgfsetfillcolor{currentfill}%
\pgfsetlinewidth{1.003750pt}%
\definecolor{currentstroke}{rgb}{0.000000,0.000000,0.000000}%
\pgfsetstrokecolor{currentstroke}%
\pgfsetdash{}{0pt}%
\pgfpathmoveto{\pgfqpoint{2.055402in}{1.438667in}}%
\pgfpathcurveto{\pgfqpoint{2.066452in}{1.438667in}}{\pgfqpoint{2.077051in}{1.443057in}}{\pgfqpoint{2.084865in}{1.450871in}}%
\pgfpathcurveto{\pgfqpoint{2.092678in}{1.458685in}}{\pgfqpoint{2.097069in}{1.469284in}}{\pgfqpoint{2.097069in}{1.480334in}}%
\pgfpathcurveto{\pgfqpoint{2.097069in}{1.491384in}}{\pgfqpoint{2.092678in}{1.501983in}}{\pgfqpoint{2.084865in}{1.509796in}}%
\pgfpathcurveto{\pgfqpoint{2.077051in}{1.517610in}}{\pgfqpoint{2.066452in}{1.522000in}}{\pgfqpoint{2.055402in}{1.522000in}}%
\pgfpathcurveto{\pgfqpoint{2.044352in}{1.522000in}}{\pgfqpoint{2.033753in}{1.517610in}}{\pgfqpoint{2.025939in}{1.509796in}}%
\pgfpathcurveto{\pgfqpoint{2.018125in}{1.501983in}}{\pgfqpoint{2.013735in}{1.491384in}}{\pgfqpoint{2.013735in}{1.480334in}}%
\pgfpathcurveto{\pgfqpoint{2.013735in}{1.469284in}}{\pgfqpoint{2.018125in}{1.458685in}}{\pgfqpoint{2.025939in}{1.450871in}}%
\pgfpathcurveto{\pgfqpoint{2.033753in}{1.443057in}}{\pgfqpoint{2.044352in}{1.438667in}}{\pgfqpoint{2.055402in}{1.438667in}}%
\pgfpathclose%
\pgfusepath{stroke,fill}%
\end{pgfscope}%
\begin{pgfscope}%
\pgfpathrectangle{\pgfqpoint{0.375000in}{0.330000in}}{\pgfqpoint{2.325000in}{2.310000in}}%
\pgfusepath{clip}%
\pgfsetbuttcap%
\pgfsetroundjoin%
\definecolor{currentfill}{rgb}{0.000000,0.000000,0.000000}%
\pgfsetfillcolor{currentfill}%
\pgfsetlinewidth{1.003750pt}%
\definecolor{currentstroke}{rgb}{0.000000,0.000000,0.000000}%
\pgfsetstrokecolor{currentstroke}%
\pgfsetdash{}{0pt}%
\pgfpathmoveto{\pgfqpoint{2.055402in}{1.438667in}}%
\pgfpathcurveto{\pgfqpoint{2.066452in}{1.438667in}}{\pgfqpoint{2.077051in}{1.443057in}}{\pgfqpoint{2.084865in}{1.450871in}}%
\pgfpathcurveto{\pgfqpoint{2.092678in}{1.458685in}}{\pgfqpoint{2.097069in}{1.469284in}}{\pgfqpoint{2.097069in}{1.480334in}}%
\pgfpathcurveto{\pgfqpoint{2.097069in}{1.491384in}}{\pgfqpoint{2.092678in}{1.501983in}}{\pgfqpoint{2.084865in}{1.509796in}}%
\pgfpathcurveto{\pgfqpoint{2.077051in}{1.517610in}}{\pgfqpoint{2.066452in}{1.522000in}}{\pgfqpoint{2.055402in}{1.522000in}}%
\pgfpathcurveto{\pgfqpoint{2.044352in}{1.522000in}}{\pgfqpoint{2.033753in}{1.517610in}}{\pgfqpoint{2.025939in}{1.509796in}}%
\pgfpathcurveto{\pgfqpoint{2.018125in}{1.501983in}}{\pgfqpoint{2.013735in}{1.491384in}}{\pgfqpoint{2.013735in}{1.480334in}}%
\pgfpathcurveto{\pgfqpoint{2.013735in}{1.469284in}}{\pgfqpoint{2.018125in}{1.458685in}}{\pgfqpoint{2.025939in}{1.450871in}}%
\pgfpathcurveto{\pgfqpoint{2.033753in}{1.443057in}}{\pgfqpoint{2.044352in}{1.438667in}}{\pgfqpoint{2.055402in}{1.438667in}}%
\pgfpathclose%
\pgfusepath{stroke,fill}%
\end{pgfscope}%
\begin{pgfscope}%
\pgfpathrectangle{\pgfqpoint{0.375000in}{0.330000in}}{\pgfqpoint{2.325000in}{2.310000in}}%
\pgfusepath{clip}%
\pgfsetbuttcap%
\pgfsetroundjoin%
\definecolor{currentfill}{rgb}{0.000000,0.000000,0.000000}%
\pgfsetfillcolor{currentfill}%
\pgfsetlinewidth{1.003750pt}%
\definecolor{currentstroke}{rgb}{0.000000,0.000000,0.000000}%
\pgfsetstrokecolor{currentstroke}%
\pgfsetdash{}{0pt}%
\pgfpathmoveto{\pgfqpoint{2.055402in}{1.438667in}}%
\pgfpathcurveto{\pgfqpoint{2.066452in}{1.438667in}}{\pgfqpoint{2.077051in}{1.443057in}}{\pgfqpoint{2.084865in}{1.450871in}}%
\pgfpathcurveto{\pgfqpoint{2.092678in}{1.458685in}}{\pgfqpoint{2.097069in}{1.469284in}}{\pgfqpoint{2.097069in}{1.480334in}}%
\pgfpathcurveto{\pgfqpoint{2.097069in}{1.491384in}}{\pgfqpoint{2.092678in}{1.501983in}}{\pgfqpoint{2.084865in}{1.509796in}}%
\pgfpathcurveto{\pgfqpoint{2.077051in}{1.517610in}}{\pgfqpoint{2.066452in}{1.522000in}}{\pgfqpoint{2.055402in}{1.522000in}}%
\pgfpathcurveto{\pgfqpoint{2.044352in}{1.522000in}}{\pgfqpoint{2.033753in}{1.517610in}}{\pgfqpoint{2.025939in}{1.509796in}}%
\pgfpathcurveto{\pgfqpoint{2.018125in}{1.501983in}}{\pgfqpoint{2.013735in}{1.491384in}}{\pgfqpoint{2.013735in}{1.480334in}}%
\pgfpathcurveto{\pgfqpoint{2.013735in}{1.469284in}}{\pgfqpoint{2.018125in}{1.458685in}}{\pgfqpoint{2.025939in}{1.450871in}}%
\pgfpathcurveto{\pgfqpoint{2.033753in}{1.443057in}}{\pgfqpoint{2.044352in}{1.438667in}}{\pgfqpoint{2.055402in}{1.438667in}}%
\pgfpathclose%
\pgfusepath{stroke,fill}%
\end{pgfscope}%
\begin{pgfscope}%
\pgfpathrectangle{\pgfqpoint{0.375000in}{0.330000in}}{\pgfqpoint{2.325000in}{2.310000in}}%
\pgfusepath{clip}%
\pgfsetbuttcap%
\pgfsetroundjoin%
\definecolor{currentfill}{rgb}{0.000000,0.000000,0.000000}%
\pgfsetfillcolor{currentfill}%
\pgfsetlinewidth{1.003750pt}%
\definecolor{currentstroke}{rgb}{0.000000,0.000000,0.000000}%
\pgfsetstrokecolor{currentstroke}%
\pgfsetdash{}{0pt}%
\pgfpathmoveto{\pgfqpoint{2.055402in}{1.438667in}}%
\pgfpathcurveto{\pgfqpoint{2.066452in}{1.438667in}}{\pgfqpoint{2.077051in}{1.443057in}}{\pgfqpoint{2.084865in}{1.450871in}}%
\pgfpathcurveto{\pgfqpoint{2.092678in}{1.458685in}}{\pgfqpoint{2.097069in}{1.469284in}}{\pgfqpoint{2.097069in}{1.480334in}}%
\pgfpathcurveto{\pgfqpoint{2.097069in}{1.491384in}}{\pgfqpoint{2.092678in}{1.501983in}}{\pgfqpoint{2.084865in}{1.509796in}}%
\pgfpathcurveto{\pgfqpoint{2.077051in}{1.517610in}}{\pgfqpoint{2.066452in}{1.522000in}}{\pgfqpoint{2.055402in}{1.522000in}}%
\pgfpathcurveto{\pgfqpoint{2.044352in}{1.522000in}}{\pgfqpoint{2.033753in}{1.517610in}}{\pgfqpoint{2.025939in}{1.509796in}}%
\pgfpathcurveto{\pgfqpoint{2.018125in}{1.501983in}}{\pgfqpoint{2.013735in}{1.491384in}}{\pgfqpoint{2.013735in}{1.480334in}}%
\pgfpathcurveto{\pgfqpoint{2.013735in}{1.469284in}}{\pgfqpoint{2.018125in}{1.458685in}}{\pgfqpoint{2.025939in}{1.450871in}}%
\pgfpathcurveto{\pgfqpoint{2.033753in}{1.443057in}}{\pgfqpoint{2.044352in}{1.438667in}}{\pgfqpoint{2.055402in}{1.438667in}}%
\pgfpathclose%
\pgfusepath{stroke,fill}%
\end{pgfscope}%
\begin{pgfscope}%
\pgfpathrectangle{\pgfqpoint{0.375000in}{0.330000in}}{\pgfqpoint{2.325000in}{2.310000in}}%
\pgfusepath{clip}%
\pgfsetbuttcap%
\pgfsetroundjoin%
\definecolor{currentfill}{rgb}{0.000000,0.000000,0.000000}%
\pgfsetfillcolor{currentfill}%
\pgfsetlinewidth{1.003750pt}%
\definecolor{currentstroke}{rgb}{0.000000,0.000000,0.000000}%
\pgfsetstrokecolor{currentstroke}%
\pgfsetdash{}{0pt}%
\pgfpathmoveto{\pgfqpoint{2.055402in}{1.438667in}}%
\pgfpathcurveto{\pgfqpoint{2.066452in}{1.438667in}}{\pgfqpoint{2.077051in}{1.443057in}}{\pgfqpoint{2.084865in}{1.450871in}}%
\pgfpathcurveto{\pgfqpoint{2.092678in}{1.458685in}}{\pgfqpoint{2.097069in}{1.469284in}}{\pgfqpoint{2.097069in}{1.480334in}}%
\pgfpathcurveto{\pgfqpoint{2.097069in}{1.491384in}}{\pgfqpoint{2.092678in}{1.501983in}}{\pgfqpoint{2.084865in}{1.509796in}}%
\pgfpathcurveto{\pgfqpoint{2.077051in}{1.517610in}}{\pgfqpoint{2.066452in}{1.522000in}}{\pgfqpoint{2.055402in}{1.522000in}}%
\pgfpathcurveto{\pgfqpoint{2.044352in}{1.522000in}}{\pgfqpoint{2.033753in}{1.517610in}}{\pgfqpoint{2.025939in}{1.509796in}}%
\pgfpathcurveto{\pgfqpoint{2.018125in}{1.501983in}}{\pgfqpoint{2.013735in}{1.491384in}}{\pgfqpoint{2.013735in}{1.480334in}}%
\pgfpathcurveto{\pgfqpoint{2.013735in}{1.469284in}}{\pgfqpoint{2.018125in}{1.458685in}}{\pgfqpoint{2.025939in}{1.450871in}}%
\pgfpathcurveto{\pgfqpoint{2.033753in}{1.443057in}}{\pgfqpoint{2.044352in}{1.438667in}}{\pgfqpoint{2.055402in}{1.438667in}}%
\pgfpathclose%
\pgfusepath{stroke,fill}%
\end{pgfscope}%
\begin{pgfscope}%
\pgfpathrectangle{\pgfqpoint{0.375000in}{0.330000in}}{\pgfqpoint{2.325000in}{2.310000in}}%
\pgfusepath{clip}%
\pgfsetbuttcap%
\pgfsetroundjoin%
\definecolor{currentfill}{rgb}{0.000000,0.000000,0.000000}%
\pgfsetfillcolor{currentfill}%
\pgfsetlinewidth{1.003750pt}%
\definecolor{currentstroke}{rgb}{0.000000,0.000000,0.000000}%
\pgfsetstrokecolor{currentstroke}%
\pgfsetdash{}{0pt}%
\pgfpathmoveto{\pgfqpoint{2.055402in}{1.438667in}}%
\pgfpathcurveto{\pgfqpoint{2.066452in}{1.438667in}}{\pgfqpoint{2.077051in}{1.443057in}}{\pgfqpoint{2.084865in}{1.450871in}}%
\pgfpathcurveto{\pgfqpoint{2.092678in}{1.458685in}}{\pgfqpoint{2.097069in}{1.469284in}}{\pgfqpoint{2.097069in}{1.480334in}}%
\pgfpathcurveto{\pgfqpoint{2.097069in}{1.491384in}}{\pgfqpoint{2.092678in}{1.501983in}}{\pgfqpoint{2.084865in}{1.509796in}}%
\pgfpathcurveto{\pgfqpoint{2.077051in}{1.517610in}}{\pgfqpoint{2.066452in}{1.522000in}}{\pgfqpoint{2.055402in}{1.522000in}}%
\pgfpathcurveto{\pgfqpoint{2.044352in}{1.522000in}}{\pgfqpoint{2.033753in}{1.517610in}}{\pgfqpoint{2.025939in}{1.509796in}}%
\pgfpathcurveto{\pgfqpoint{2.018125in}{1.501983in}}{\pgfqpoint{2.013735in}{1.491384in}}{\pgfqpoint{2.013735in}{1.480334in}}%
\pgfpathcurveto{\pgfqpoint{2.013735in}{1.469284in}}{\pgfqpoint{2.018125in}{1.458685in}}{\pgfqpoint{2.025939in}{1.450871in}}%
\pgfpathcurveto{\pgfqpoint{2.033753in}{1.443057in}}{\pgfqpoint{2.044352in}{1.438667in}}{\pgfqpoint{2.055402in}{1.438667in}}%
\pgfpathclose%
\pgfusepath{stroke,fill}%
\end{pgfscope}%
\begin{pgfscope}%
\pgfpathrectangle{\pgfqpoint{0.375000in}{0.330000in}}{\pgfqpoint{2.325000in}{2.310000in}}%
\pgfusepath{clip}%
\pgfsetbuttcap%
\pgfsetroundjoin%
\definecolor{currentfill}{rgb}{0.000000,0.000000,0.000000}%
\pgfsetfillcolor{currentfill}%
\pgfsetlinewidth{1.003750pt}%
\definecolor{currentstroke}{rgb}{0.000000,0.000000,0.000000}%
\pgfsetstrokecolor{currentstroke}%
\pgfsetdash{}{0pt}%
\pgfpathmoveto{\pgfqpoint{2.055402in}{1.438667in}}%
\pgfpathcurveto{\pgfqpoint{2.066452in}{1.438667in}}{\pgfqpoint{2.077051in}{1.443057in}}{\pgfqpoint{2.084865in}{1.450871in}}%
\pgfpathcurveto{\pgfqpoint{2.092678in}{1.458685in}}{\pgfqpoint{2.097069in}{1.469284in}}{\pgfqpoint{2.097069in}{1.480334in}}%
\pgfpathcurveto{\pgfqpoint{2.097069in}{1.491384in}}{\pgfqpoint{2.092678in}{1.501983in}}{\pgfqpoint{2.084865in}{1.509796in}}%
\pgfpathcurveto{\pgfqpoint{2.077051in}{1.517610in}}{\pgfqpoint{2.066452in}{1.522000in}}{\pgfqpoint{2.055402in}{1.522000in}}%
\pgfpathcurveto{\pgfqpoint{2.044352in}{1.522000in}}{\pgfqpoint{2.033753in}{1.517610in}}{\pgfqpoint{2.025939in}{1.509796in}}%
\pgfpathcurveto{\pgfqpoint{2.018125in}{1.501983in}}{\pgfqpoint{2.013735in}{1.491384in}}{\pgfqpoint{2.013735in}{1.480334in}}%
\pgfpathcurveto{\pgfqpoint{2.013735in}{1.469284in}}{\pgfqpoint{2.018125in}{1.458685in}}{\pgfqpoint{2.025939in}{1.450871in}}%
\pgfpathcurveto{\pgfqpoint{2.033753in}{1.443057in}}{\pgfqpoint{2.044352in}{1.438667in}}{\pgfqpoint{2.055402in}{1.438667in}}%
\pgfpathclose%
\pgfusepath{stroke,fill}%
\end{pgfscope}%
\begin{pgfscope}%
\pgfpathrectangle{\pgfqpoint{0.375000in}{0.330000in}}{\pgfqpoint{2.325000in}{2.310000in}}%
\pgfusepath{clip}%
\pgfsetbuttcap%
\pgfsetroundjoin%
\definecolor{currentfill}{rgb}{0.000000,0.000000,0.000000}%
\pgfsetfillcolor{currentfill}%
\pgfsetlinewidth{1.003750pt}%
\definecolor{currentstroke}{rgb}{0.000000,0.000000,0.000000}%
\pgfsetstrokecolor{currentstroke}%
\pgfsetdash{}{0pt}%
\pgfpathmoveto{\pgfqpoint{2.055402in}{1.438667in}}%
\pgfpathcurveto{\pgfqpoint{2.066452in}{1.438667in}}{\pgfqpoint{2.077051in}{1.443057in}}{\pgfqpoint{2.084865in}{1.450871in}}%
\pgfpathcurveto{\pgfqpoint{2.092678in}{1.458685in}}{\pgfqpoint{2.097069in}{1.469284in}}{\pgfqpoint{2.097069in}{1.480334in}}%
\pgfpathcurveto{\pgfqpoint{2.097069in}{1.491384in}}{\pgfqpoint{2.092678in}{1.501983in}}{\pgfqpoint{2.084865in}{1.509796in}}%
\pgfpathcurveto{\pgfqpoint{2.077051in}{1.517610in}}{\pgfqpoint{2.066452in}{1.522000in}}{\pgfqpoint{2.055402in}{1.522000in}}%
\pgfpathcurveto{\pgfqpoint{2.044352in}{1.522000in}}{\pgfqpoint{2.033753in}{1.517610in}}{\pgfqpoint{2.025939in}{1.509796in}}%
\pgfpathcurveto{\pgfqpoint{2.018125in}{1.501983in}}{\pgfqpoint{2.013735in}{1.491384in}}{\pgfqpoint{2.013735in}{1.480334in}}%
\pgfpathcurveto{\pgfqpoint{2.013735in}{1.469284in}}{\pgfqpoint{2.018125in}{1.458685in}}{\pgfqpoint{2.025939in}{1.450871in}}%
\pgfpathcurveto{\pgfqpoint{2.033753in}{1.443057in}}{\pgfqpoint{2.044352in}{1.438667in}}{\pgfqpoint{2.055402in}{1.438667in}}%
\pgfpathclose%
\pgfusepath{stroke,fill}%
\end{pgfscope}%
\begin{pgfscope}%
\pgfpathrectangle{\pgfqpoint{0.375000in}{0.330000in}}{\pgfqpoint{2.325000in}{2.310000in}}%
\pgfusepath{clip}%
\pgfsetbuttcap%
\pgfsetroundjoin%
\definecolor{currentfill}{rgb}{0.000000,0.000000,0.000000}%
\pgfsetfillcolor{currentfill}%
\pgfsetlinewidth{1.003750pt}%
\definecolor{currentstroke}{rgb}{0.000000,0.000000,0.000000}%
\pgfsetstrokecolor{currentstroke}%
\pgfsetdash{}{0pt}%
\pgfpathmoveto{\pgfqpoint{2.055402in}{1.438667in}}%
\pgfpathcurveto{\pgfqpoint{2.066452in}{1.438667in}}{\pgfqpoint{2.077051in}{1.443057in}}{\pgfqpoint{2.084865in}{1.450871in}}%
\pgfpathcurveto{\pgfqpoint{2.092678in}{1.458685in}}{\pgfqpoint{2.097069in}{1.469284in}}{\pgfqpoint{2.097069in}{1.480334in}}%
\pgfpathcurveto{\pgfqpoint{2.097069in}{1.491384in}}{\pgfqpoint{2.092678in}{1.501983in}}{\pgfqpoint{2.084865in}{1.509796in}}%
\pgfpathcurveto{\pgfqpoint{2.077051in}{1.517610in}}{\pgfqpoint{2.066452in}{1.522000in}}{\pgfqpoint{2.055402in}{1.522000in}}%
\pgfpathcurveto{\pgfqpoint{2.044352in}{1.522000in}}{\pgfqpoint{2.033753in}{1.517610in}}{\pgfqpoint{2.025939in}{1.509796in}}%
\pgfpathcurveto{\pgfqpoint{2.018125in}{1.501983in}}{\pgfqpoint{2.013735in}{1.491384in}}{\pgfqpoint{2.013735in}{1.480334in}}%
\pgfpathcurveto{\pgfqpoint{2.013735in}{1.469284in}}{\pgfqpoint{2.018125in}{1.458685in}}{\pgfqpoint{2.025939in}{1.450871in}}%
\pgfpathcurveto{\pgfqpoint{2.033753in}{1.443057in}}{\pgfqpoint{2.044352in}{1.438667in}}{\pgfqpoint{2.055402in}{1.438667in}}%
\pgfpathclose%
\pgfusepath{stroke,fill}%
\end{pgfscope}%
\begin{pgfscope}%
\pgfpathrectangle{\pgfqpoint{0.375000in}{0.330000in}}{\pgfqpoint{2.325000in}{2.310000in}}%
\pgfusepath{clip}%
\pgfsetbuttcap%
\pgfsetroundjoin%
\definecolor{currentfill}{rgb}{0.000000,0.000000,0.000000}%
\pgfsetfillcolor{currentfill}%
\pgfsetlinewidth{1.003750pt}%
\definecolor{currentstroke}{rgb}{0.000000,0.000000,0.000000}%
\pgfsetstrokecolor{currentstroke}%
\pgfsetdash{}{0pt}%
\pgfpathmoveto{\pgfqpoint{2.055402in}{1.438667in}}%
\pgfpathcurveto{\pgfqpoint{2.066452in}{1.438667in}}{\pgfqpoint{2.077051in}{1.443057in}}{\pgfqpoint{2.084865in}{1.450871in}}%
\pgfpathcurveto{\pgfqpoint{2.092678in}{1.458685in}}{\pgfqpoint{2.097069in}{1.469284in}}{\pgfqpoint{2.097069in}{1.480334in}}%
\pgfpathcurveto{\pgfqpoint{2.097069in}{1.491384in}}{\pgfqpoint{2.092678in}{1.501983in}}{\pgfqpoint{2.084865in}{1.509796in}}%
\pgfpathcurveto{\pgfqpoint{2.077051in}{1.517610in}}{\pgfqpoint{2.066452in}{1.522000in}}{\pgfqpoint{2.055402in}{1.522000in}}%
\pgfpathcurveto{\pgfqpoint{2.044352in}{1.522000in}}{\pgfqpoint{2.033753in}{1.517610in}}{\pgfqpoint{2.025939in}{1.509796in}}%
\pgfpathcurveto{\pgfqpoint{2.018125in}{1.501983in}}{\pgfqpoint{2.013735in}{1.491384in}}{\pgfqpoint{2.013735in}{1.480334in}}%
\pgfpathcurveto{\pgfqpoint{2.013735in}{1.469284in}}{\pgfqpoint{2.018125in}{1.458685in}}{\pgfqpoint{2.025939in}{1.450871in}}%
\pgfpathcurveto{\pgfqpoint{2.033753in}{1.443057in}}{\pgfqpoint{2.044352in}{1.438667in}}{\pgfqpoint{2.055402in}{1.438667in}}%
\pgfpathclose%
\pgfusepath{stroke,fill}%
\end{pgfscope}%
\begin{pgfscope}%
\pgfpathrectangle{\pgfqpoint{0.375000in}{0.330000in}}{\pgfqpoint{2.325000in}{2.310000in}}%
\pgfusepath{clip}%
\pgfsetbuttcap%
\pgfsetroundjoin%
\definecolor{currentfill}{rgb}{0.000000,0.000000,0.000000}%
\pgfsetfillcolor{currentfill}%
\pgfsetlinewidth{1.003750pt}%
\definecolor{currentstroke}{rgb}{0.000000,0.000000,0.000000}%
\pgfsetstrokecolor{currentstroke}%
\pgfsetdash{}{0pt}%
\pgfpathmoveto{\pgfqpoint{2.055402in}{1.438667in}}%
\pgfpathcurveto{\pgfqpoint{2.066452in}{1.438667in}}{\pgfqpoint{2.077051in}{1.443057in}}{\pgfqpoint{2.084865in}{1.450871in}}%
\pgfpathcurveto{\pgfqpoint{2.092678in}{1.458685in}}{\pgfqpoint{2.097069in}{1.469284in}}{\pgfqpoint{2.097069in}{1.480334in}}%
\pgfpathcurveto{\pgfqpoint{2.097069in}{1.491384in}}{\pgfqpoint{2.092678in}{1.501983in}}{\pgfqpoint{2.084865in}{1.509796in}}%
\pgfpathcurveto{\pgfqpoint{2.077051in}{1.517610in}}{\pgfqpoint{2.066452in}{1.522000in}}{\pgfqpoint{2.055402in}{1.522000in}}%
\pgfpathcurveto{\pgfqpoint{2.044352in}{1.522000in}}{\pgfqpoint{2.033753in}{1.517610in}}{\pgfqpoint{2.025939in}{1.509796in}}%
\pgfpathcurveto{\pgfqpoint{2.018125in}{1.501983in}}{\pgfqpoint{2.013735in}{1.491384in}}{\pgfqpoint{2.013735in}{1.480334in}}%
\pgfpathcurveto{\pgfqpoint{2.013735in}{1.469284in}}{\pgfqpoint{2.018125in}{1.458685in}}{\pgfqpoint{2.025939in}{1.450871in}}%
\pgfpathcurveto{\pgfqpoint{2.033753in}{1.443057in}}{\pgfqpoint{2.044352in}{1.438667in}}{\pgfqpoint{2.055402in}{1.438667in}}%
\pgfpathclose%
\pgfusepath{stroke,fill}%
\end{pgfscope}%
\begin{pgfscope}%
\pgfpathrectangle{\pgfqpoint{0.375000in}{0.330000in}}{\pgfqpoint{2.325000in}{2.310000in}}%
\pgfusepath{clip}%
\pgfsetbuttcap%
\pgfsetroundjoin%
\definecolor{currentfill}{rgb}{0.000000,0.000000,0.000000}%
\pgfsetfillcolor{currentfill}%
\pgfsetlinewidth{1.003750pt}%
\definecolor{currentstroke}{rgb}{0.000000,0.000000,0.000000}%
\pgfsetstrokecolor{currentstroke}%
\pgfsetdash{}{0pt}%
\pgfpathmoveto{\pgfqpoint{2.055402in}{1.438667in}}%
\pgfpathcurveto{\pgfqpoint{2.066452in}{1.438667in}}{\pgfqpoint{2.077051in}{1.443057in}}{\pgfqpoint{2.084865in}{1.450871in}}%
\pgfpathcurveto{\pgfqpoint{2.092678in}{1.458685in}}{\pgfqpoint{2.097069in}{1.469284in}}{\pgfqpoint{2.097069in}{1.480334in}}%
\pgfpathcurveto{\pgfqpoint{2.097069in}{1.491384in}}{\pgfqpoint{2.092678in}{1.501983in}}{\pgfqpoint{2.084865in}{1.509796in}}%
\pgfpathcurveto{\pgfqpoint{2.077051in}{1.517610in}}{\pgfqpoint{2.066452in}{1.522000in}}{\pgfqpoint{2.055402in}{1.522000in}}%
\pgfpathcurveto{\pgfqpoint{2.044352in}{1.522000in}}{\pgfqpoint{2.033753in}{1.517610in}}{\pgfqpoint{2.025939in}{1.509796in}}%
\pgfpathcurveto{\pgfqpoint{2.018125in}{1.501983in}}{\pgfqpoint{2.013735in}{1.491384in}}{\pgfqpoint{2.013735in}{1.480334in}}%
\pgfpathcurveto{\pgfqpoint{2.013735in}{1.469284in}}{\pgfqpoint{2.018125in}{1.458685in}}{\pgfqpoint{2.025939in}{1.450871in}}%
\pgfpathcurveto{\pgfqpoint{2.033753in}{1.443057in}}{\pgfqpoint{2.044352in}{1.438667in}}{\pgfqpoint{2.055402in}{1.438667in}}%
\pgfpathclose%
\pgfusepath{stroke,fill}%
\end{pgfscope}%
\begin{pgfscope}%
\pgfpathrectangle{\pgfqpoint{0.375000in}{0.330000in}}{\pgfqpoint{2.325000in}{2.310000in}}%
\pgfusepath{clip}%
\pgfsetbuttcap%
\pgfsetroundjoin%
\definecolor{currentfill}{rgb}{0.000000,0.000000,0.000000}%
\pgfsetfillcolor{currentfill}%
\pgfsetlinewidth{1.003750pt}%
\definecolor{currentstroke}{rgb}{0.000000,0.000000,0.000000}%
\pgfsetstrokecolor{currentstroke}%
\pgfsetdash{}{0pt}%
\pgfpathmoveto{\pgfqpoint{2.580256in}{2.474583in}}%
\pgfpathcurveto{\pgfqpoint{2.591306in}{2.474583in}}{\pgfqpoint{2.601905in}{2.478974in}}{\pgfqpoint{2.609718in}{2.486787in}}%
\pgfpathcurveto{\pgfqpoint{2.617532in}{2.494601in}}{\pgfqpoint{2.621922in}{2.505200in}}{\pgfqpoint{2.621922in}{2.516250in}}%
\pgfpathcurveto{\pgfqpoint{2.621922in}{2.527300in}}{\pgfqpoint{2.617532in}{2.537899in}}{\pgfqpoint{2.609718in}{2.545713in}}%
\pgfpathcurveto{\pgfqpoint{2.601905in}{2.553526in}}{\pgfqpoint{2.591306in}{2.557917in}}{\pgfqpoint{2.580256in}{2.557917in}}%
\pgfpathcurveto{\pgfqpoint{2.569206in}{2.557917in}}{\pgfqpoint{2.558607in}{2.553526in}}{\pgfqpoint{2.550793in}{2.545713in}}%
\pgfpathcurveto{\pgfqpoint{2.542979in}{2.537899in}}{\pgfqpoint{2.538589in}{2.527300in}}{\pgfqpoint{2.538589in}{2.516250in}}%
\pgfpathcurveto{\pgfqpoint{2.538589in}{2.505200in}}{\pgfqpoint{2.542979in}{2.494601in}}{\pgfqpoint{2.550793in}{2.486787in}}%
\pgfpathcurveto{\pgfqpoint{2.558607in}{2.478974in}}{\pgfqpoint{2.569206in}{2.474583in}}{\pgfqpoint{2.580256in}{2.474583in}}%
\pgfpathclose%
\pgfusepath{stroke,fill}%
\end{pgfscope}%
\begin{pgfscope}%
\pgfpathrectangle{\pgfqpoint{0.375000in}{0.330000in}}{\pgfqpoint{2.325000in}{2.310000in}}%
\pgfusepath{clip}%
\pgfsetbuttcap%
\pgfsetroundjoin%
\definecolor{currentfill}{rgb}{0.000000,0.000000,0.000000}%
\pgfsetfillcolor{currentfill}%
\pgfsetlinewidth{1.003750pt}%
\definecolor{currentstroke}{rgb}{0.000000,0.000000,0.000000}%
\pgfsetstrokecolor{currentstroke}%
\pgfsetdash{}{0pt}%
\pgfpathmoveto{\pgfqpoint{2.580256in}{2.474583in}}%
\pgfpathcurveto{\pgfqpoint{2.591306in}{2.474583in}}{\pgfqpoint{2.601905in}{2.478974in}}{\pgfqpoint{2.609718in}{2.486787in}}%
\pgfpathcurveto{\pgfqpoint{2.617532in}{2.494601in}}{\pgfqpoint{2.621922in}{2.505200in}}{\pgfqpoint{2.621922in}{2.516250in}}%
\pgfpathcurveto{\pgfqpoint{2.621922in}{2.527300in}}{\pgfqpoint{2.617532in}{2.537899in}}{\pgfqpoint{2.609718in}{2.545713in}}%
\pgfpathcurveto{\pgfqpoint{2.601905in}{2.553526in}}{\pgfqpoint{2.591306in}{2.557917in}}{\pgfqpoint{2.580256in}{2.557917in}}%
\pgfpathcurveto{\pgfqpoint{2.569206in}{2.557917in}}{\pgfqpoint{2.558607in}{2.553526in}}{\pgfqpoint{2.550793in}{2.545713in}}%
\pgfpathcurveto{\pgfqpoint{2.542979in}{2.537899in}}{\pgfqpoint{2.538589in}{2.527300in}}{\pgfqpoint{2.538589in}{2.516250in}}%
\pgfpathcurveto{\pgfqpoint{2.538589in}{2.505200in}}{\pgfqpoint{2.542979in}{2.494601in}}{\pgfqpoint{2.550793in}{2.486787in}}%
\pgfpathcurveto{\pgfqpoint{2.558607in}{2.478974in}}{\pgfqpoint{2.569206in}{2.474583in}}{\pgfqpoint{2.580256in}{2.474583in}}%
\pgfpathclose%
\pgfusepath{stroke,fill}%
\end{pgfscope}%
\begin{pgfscope}%
\pgfpathrectangle{\pgfqpoint{0.375000in}{0.330000in}}{\pgfqpoint{2.325000in}{2.310000in}}%
\pgfusepath{clip}%
\pgfsetbuttcap%
\pgfsetroundjoin%
\definecolor{currentfill}{rgb}{0.000000,0.000000,0.000000}%
\pgfsetfillcolor{currentfill}%
\pgfsetlinewidth{1.003750pt}%
\definecolor{currentstroke}{rgb}{0.000000,0.000000,0.000000}%
\pgfsetstrokecolor{currentstroke}%
\pgfsetdash{}{0pt}%
\pgfpathmoveto{\pgfqpoint{2.580256in}{2.474583in}}%
\pgfpathcurveto{\pgfqpoint{2.591306in}{2.474583in}}{\pgfqpoint{2.601905in}{2.478974in}}{\pgfqpoint{2.609718in}{2.486787in}}%
\pgfpathcurveto{\pgfqpoint{2.617532in}{2.494601in}}{\pgfqpoint{2.621922in}{2.505200in}}{\pgfqpoint{2.621922in}{2.516250in}}%
\pgfpathcurveto{\pgfqpoint{2.621922in}{2.527300in}}{\pgfqpoint{2.617532in}{2.537899in}}{\pgfqpoint{2.609718in}{2.545713in}}%
\pgfpathcurveto{\pgfqpoint{2.601905in}{2.553526in}}{\pgfqpoint{2.591306in}{2.557917in}}{\pgfqpoint{2.580256in}{2.557917in}}%
\pgfpathcurveto{\pgfqpoint{2.569206in}{2.557917in}}{\pgfqpoint{2.558607in}{2.553526in}}{\pgfqpoint{2.550793in}{2.545713in}}%
\pgfpathcurveto{\pgfqpoint{2.542979in}{2.537899in}}{\pgfqpoint{2.538589in}{2.527300in}}{\pgfqpoint{2.538589in}{2.516250in}}%
\pgfpathcurveto{\pgfqpoint{2.538589in}{2.505200in}}{\pgfqpoint{2.542979in}{2.494601in}}{\pgfqpoint{2.550793in}{2.486787in}}%
\pgfpathcurveto{\pgfqpoint{2.558607in}{2.478974in}}{\pgfqpoint{2.569206in}{2.474583in}}{\pgfqpoint{2.580256in}{2.474583in}}%
\pgfpathclose%
\pgfusepath{stroke,fill}%
\end{pgfscope}%
\begin{pgfscope}%
\pgfpathrectangle{\pgfqpoint{0.375000in}{0.330000in}}{\pgfqpoint{2.325000in}{2.310000in}}%
\pgfusepath{clip}%
\pgfsetbuttcap%
\pgfsetroundjoin%
\definecolor{currentfill}{rgb}{0.000000,0.000000,0.000000}%
\pgfsetfillcolor{currentfill}%
\pgfsetlinewidth{1.003750pt}%
\definecolor{currentstroke}{rgb}{0.000000,0.000000,0.000000}%
\pgfsetstrokecolor{currentstroke}%
\pgfsetdash{}{0pt}%
\pgfpathmoveto{\pgfqpoint{2.580256in}{2.474583in}}%
\pgfpathcurveto{\pgfqpoint{2.591306in}{2.474583in}}{\pgfqpoint{2.601905in}{2.478974in}}{\pgfqpoint{2.609718in}{2.486787in}}%
\pgfpathcurveto{\pgfqpoint{2.617532in}{2.494601in}}{\pgfqpoint{2.621922in}{2.505200in}}{\pgfqpoint{2.621922in}{2.516250in}}%
\pgfpathcurveto{\pgfqpoint{2.621922in}{2.527300in}}{\pgfqpoint{2.617532in}{2.537899in}}{\pgfqpoint{2.609718in}{2.545713in}}%
\pgfpathcurveto{\pgfqpoint{2.601905in}{2.553526in}}{\pgfqpoint{2.591306in}{2.557917in}}{\pgfqpoint{2.580256in}{2.557917in}}%
\pgfpathcurveto{\pgfqpoint{2.569206in}{2.557917in}}{\pgfqpoint{2.558607in}{2.553526in}}{\pgfqpoint{2.550793in}{2.545713in}}%
\pgfpathcurveto{\pgfqpoint{2.542979in}{2.537899in}}{\pgfqpoint{2.538589in}{2.527300in}}{\pgfqpoint{2.538589in}{2.516250in}}%
\pgfpathcurveto{\pgfqpoint{2.538589in}{2.505200in}}{\pgfqpoint{2.542979in}{2.494601in}}{\pgfqpoint{2.550793in}{2.486787in}}%
\pgfpathcurveto{\pgfqpoint{2.558607in}{2.478974in}}{\pgfqpoint{2.569206in}{2.474583in}}{\pgfqpoint{2.580256in}{2.474583in}}%
\pgfpathclose%
\pgfusepath{stroke,fill}%
\end{pgfscope}%
\begin{pgfscope}%
\pgfpathrectangle{\pgfqpoint{0.375000in}{0.330000in}}{\pgfqpoint{2.325000in}{2.310000in}}%
\pgfusepath{clip}%
\pgfsetbuttcap%
\pgfsetroundjoin%
\definecolor{currentfill}{rgb}{0.000000,0.000000,0.000000}%
\pgfsetfillcolor{currentfill}%
\pgfsetlinewidth{1.003750pt}%
\definecolor{currentstroke}{rgb}{0.000000,0.000000,0.000000}%
\pgfsetstrokecolor{currentstroke}%
\pgfsetdash{}{0pt}%
\pgfpathmoveto{\pgfqpoint{2.580256in}{2.474583in}}%
\pgfpathcurveto{\pgfqpoint{2.591306in}{2.474583in}}{\pgfqpoint{2.601905in}{2.478974in}}{\pgfqpoint{2.609718in}{2.486787in}}%
\pgfpathcurveto{\pgfqpoint{2.617532in}{2.494601in}}{\pgfqpoint{2.621922in}{2.505200in}}{\pgfqpoint{2.621922in}{2.516250in}}%
\pgfpathcurveto{\pgfqpoint{2.621922in}{2.527300in}}{\pgfqpoint{2.617532in}{2.537899in}}{\pgfqpoint{2.609718in}{2.545713in}}%
\pgfpathcurveto{\pgfqpoint{2.601905in}{2.553526in}}{\pgfqpoint{2.591306in}{2.557917in}}{\pgfqpoint{2.580256in}{2.557917in}}%
\pgfpathcurveto{\pgfqpoint{2.569206in}{2.557917in}}{\pgfqpoint{2.558607in}{2.553526in}}{\pgfqpoint{2.550793in}{2.545713in}}%
\pgfpathcurveto{\pgfqpoint{2.542979in}{2.537899in}}{\pgfqpoint{2.538589in}{2.527300in}}{\pgfqpoint{2.538589in}{2.516250in}}%
\pgfpathcurveto{\pgfqpoint{2.538589in}{2.505200in}}{\pgfqpoint{2.542979in}{2.494601in}}{\pgfqpoint{2.550793in}{2.486787in}}%
\pgfpathcurveto{\pgfqpoint{2.558607in}{2.478974in}}{\pgfqpoint{2.569206in}{2.474583in}}{\pgfqpoint{2.580256in}{2.474583in}}%
\pgfpathclose%
\pgfusepath{stroke,fill}%
\end{pgfscope}%
\begin{pgfscope}%
\pgfpathrectangle{\pgfqpoint{0.375000in}{0.330000in}}{\pgfqpoint{2.325000in}{2.310000in}}%
\pgfusepath{clip}%
\pgfsetbuttcap%
\pgfsetroundjoin%
\definecolor{currentfill}{rgb}{0.000000,0.000000,0.000000}%
\pgfsetfillcolor{currentfill}%
\pgfsetlinewidth{1.003750pt}%
\definecolor{currentstroke}{rgb}{0.000000,0.000000,0.000000}%
\pgfsetstrokecolor{currentstroke}%
\pgfsetdash{}{0pt}%
\pgfpathmoveto{\pgfqpoint{2.580256in}{2.474583in}}%
\pgfpathcurveto{\pgfqpoint{2.591306in}{2.474583in}}{\pgfqpoint{2.601905in}{2.478974in}}{\pgfqpoint{2.609718in}{2.486787in}}%
\pgfpathcurveto{\pgfqpoint{2.617532in}{2.494601in}}{\pgfqpoint{2.621922in}{2.505200in}}{\pgfqpoint{2.621922in}{2.516250in}}%
\pgfpathcurveto{\pgfqpoint{2.621922in}{2.527300in}}{\pgfqpoint{2.617532in}{2.537899in}}{\pgfqpoint{2.609718in}{2.545713in}}%
\pgfpathcurveto{\pgfqpoint{2.601905in}{2.553526in}}{\pgfqpoint{2.591306in}{2.557917in}}{\pgfqpoint{2.580256in}{2.557917in}}%
\pgfpathcurveto{\pgfqpoint{2.569206in}{2.557917in}}{\pgfqpoint{2.558607in}{2.553526in}}{\pgfqpoint{2.550793in}{2.545713in}}%
\pgfpathcurveto{\pgfqpoint{2.542979in}{2.537899in}}{\pgfqpoint{2.538589in}{2.527300in}}{\pgfqpoint{2.538589in}{2.516250in}}%
\pgfpathcurveto{\pgfqpoint{2.538589in}{2.505200in}}{\pgfqpoint{2.542979in}{2.494601in}}{\pgfqpoint{2.550793in}{2.486787in}}%
\pgfpathcurveto{\pgfqpoint{2.558607in}{2.478974in}}{\pgfqpoint{2.569206in}{2.474583in}}{\pgfqpoint{2.580256in}{2.474583in}}%
\pgfpathclose%
\pgfusepath{stroke,fill}%
\end{pgfscope}%
\begin{pgfscope}%
\pgfpathrectangle{\pgfqpoint{0.375000in}{0.330000in}}{\pgfqpoint{2.325000in}{2.310000in}}%
\pgfusepath{clip}%
\pgfsetbuttcap%
\pgfsetroundjoin%
\definecolor{currentfill}{rgb}{0.000000,0.000000,0.000000}%
\pgfsetfillcolor{currentfill}%
\pgfsetlinewidth{1.003750pt}%
\definecolor{currentstroke}{rgb}{0.000000,0.000000,0.000000}%
\pgfsetstrokecolor{currentstroke}%
\pgfsetdash{}{0pt}%
\pgfpathmoveto{\pgfqpoint{2.580256in}{2.474583in}}%
\pgfpathcurveto{\pgfqpoint{2.591306in}{2.474583in}}{\pgfqpoint{2.601905in}{2.478974in}}{\pgfqpoint{2.609718in}{2.486787in}}%
\pgfpathcurveto{\pgfqpoint{2.617532in}{2.494601in}}{\pgfqpoint{2.621922in}{2.505200in}}{\pgfqpoint{2.621922in}{2.516250in}}%
\pgfpathcurveto{\pgfqpoint{2.621922in}{2.527300in}}{\pgfqpoint{2.617532in}{2.537899in}}{\pgfqpoint{2.609718in}{2.545713in}}%
\pgfpathcurveto{\pgfqpoint{2.601905in}{2.553526in}}{\pgfqpoint{2.591306in}{2.557917in}}{\pgfqpoint{2.580256in}{2.557917in}}%
\pgfpathcurveto{\pgfqpoint{2.569206in}{2.557917in}}{\pgfqpoint{2.558607in}{2.553526in}}{\pgfqpoint{2.550793in}{2.545713in}}%
\pgfpathcurveto{\pgfqpoint{2.542979in}{2.537899in}}{\pgfqpoint{2.538589in}{2.527300in}}{\pgfqpoint{2.538589in}{2.516250in}}%
\pgfpathcurveto{\pgfqpoint{2.538589in}{2.505200in}}{\pgfqpoint{2.542979in}{2.494601in}}{\pgfqpoint{2.550793in}{2.486787in}}%
\pgfpathcurveto{\pgfqpoint{2.558607in}{2.478974in}}{\pgfqpoint{2.569206in}{2.474583in}}{\pgfqpoint{2.580256in}{2.474583in}}%
\pgfpathclose%
\pgfusepath{stroke,fill}%
\end{pgfscope}%
\begin{pgfscope}%
\pgfpathrectangle{\pgfqpoint{0.375000in}{0.330000in}}{\pgfqpoint{2.325000in}{2.310000in}}%
\pgfusepath{clip}%
\pgfsetbuttcap%
\pgfsetroundjoin%
\definecolor{currentfill}{rgb}{0.000000,0.000000,0.000000}%
\pgfsetfillcolor{currentfill}%
\pgfsetlinewidth{1.003750pt}%
\definecolor{currentstroke}{rgb}{0.000000,0.000000,0.000000}%
\pgfsetstrokecolor{currentstroke}%
\pgfsetdash{}{0pt}%
\pgfpathmoveto{\pgfqpoint{2.580256in}{1.438667in}}%
\pgfpathcurveto{\pgfqpoint{2.591306in}{1.438667in}}{\pgfqpoint{2.601905in}{1.443057in}}{\pgfqpoint{2.609718in}{1.450871in}}%
\pgfpathcurveto{\pgfqpoint{2.617532in}{1.458685in}}{\pgfqpoint{2.621922in}{1.469284in}}{\pgfqpoint{2.621922in}{1.480334in}}%
\pgfpathcurveto{\pgfqpoint{2.621922in}{1.491384in}}{\pgfqpoint{2.617532in}{1.501983in}}{\pgfqpoint{2.609718in}{1.509796in}}%
\pgfpathcurveto{\pgfqpoint{2.601905in}{1.517610in}}{\pgfqpoint{2.591306in}{1.522000in}}{\pgfqpoint{2.580256in}{1.522000in}}%
\pgfpathcurveto{\pgfqpoint{2.569206in}{1.522000in}}{\pgfqpoint{2.558607in}{1.517610in}}{\pgfqpoint{2.550793in}{1.509796in}}%
\pgfpathcurveto{\pgfqpoint{2.542979in}{1.501983in}}{\pgfqpoint{2.538589in}{1.491384in}}{\pgfqpoint{2.538589in}{1.480334in}}%
\pgfpathcurveto{\pgfqpoint{2.538589in}{1.469284in}}{\pgfqpoint{2.542979in}{1.458685in}}{\pgfqpoint{2.550793in}{1.450871in}}%
\pgfpathcurveto{\pgfqpoint{2.558607in}{1.443057in}}{\pgfqpoint{2.569206in}{1.438667in}}{\pgfqpoint{2.580256in}{1.438667in}}%
\pgfpathclose%
\pgfusepath{stroke,fill}%
\end{pgfscope}%
\begin{pgfscope}%
\pgfpathrectangle{\pgfqpoint{0.375000in}{0.330000in}}{\pgfqpoint{2.325000in}{2.310000in}}%
\pgfusepath{clip}%
\pgfsetbuttcap%
\pgfsetroundjoin%
\definecolor{currentfill}{rgb}{0.000000,0.000000,0.000000}%
\pgfsetfillcolor{currentfill}%
\pgfsetlinewidth{1.003750pt}%
\definecolor{currentstroke}{rgb}{0.000000,0.000000,0.000000}%
\pgfsetstrokecolor{currentstroke}%
\pgfsetdash{}{0pt}%
\pgfpathmoveto{\pgfqpoint{2.580256in}{1.438667in}}%
\pgfpathcurveto{\pgfqpoint{2.591306in}{1.438667in}}{\pgfqpoint{2.601905in}{1.443057in}}{\pgfqpoint{2.609718in}{1.450871in}}%
\pgfpathcurveto{\pgfqpoint{2.617532in}{1.458685in}}{\pgfqpoint{2.621922in}{1.469284in}}{\pgfqpoint{2.621922in}{1.480334in}}%
\pgfpathcurveto{\pgfqpoint{2.621922in}{1.491384in}}{\pgfqpoint{2.617532in}{1.501983in}}{\pgfqpoint{2.609718in}{1.509796in}}%
\pgfpathcurveto{\pgfqpoint{2.601905in}{1.517610in}}{\pgfqpoint{2.591306in}{1.522000in}}{\pgfqpoint{2.580256in}{1.522000in}}%
\pgfpathcurveto{\pgfqpoint{2.569206in}{1.522000in}}{\pgfqpoint{2.558607in}{1.517610in}}{\pgfqpoint{2.550793in}{1.509796in}}%
\pgfpathcurveto{\pgfqpoint{2.542979in}{1.501983in}}{\pgfqpoint{2.538589in}{1.491384in}}{\pgfqpoint{2.538589in}{1.480334in}}%
\pgfpathcurveto{\pgfqpoint{2.538589in}{1.469284in}}{\pgfqpoint{2.542979in}{1.458685in}}{\pgfqpoint{2.550793in}{1.450871in}}%
\pgfpathcurveto{\pgfqpoint{2.558607in}{1.443057in}}{\pgfqpoint{2.569206in}{1.438667in}}{\pgfqpoint{2.580256in}{1.438667in}}%
\pgfpathclose%
\pgfusepath{stroke,fill}%
\end{pgfscope}%
\begin{pgfscope}%
\pgfpathrectangle{\pgfqpoint{0.375000in}{0.330000in}}{\pgfqpoint{2.325000in}{2.310000in}}%
\pgfusepath{clip}%
\pgfsetbuttcap%
\pgfsetroundjoin%
\definecolor{currentfill}{rgb}{0.000000,0.000000,0.000000}%
\pgfsetfillcolor{currentfill}%
\pgfsetlinewidth{1.003750pt}%
\definecolor{currentstroke}{rgb}{0.000000,0.000000,0.000000}%
\pgfsetstrokecolor{currentstroke}%
\pgfsetdash{}{0pt}%
\pgfpathmoveto{\pgfqpoint{2.580256in}{2.474583in}}%
\pgfpathcurveto{\pgfqpoint{2.591306in}{2.474583in}}{\pgfqpoint{2.601905in}{2.478974in}}{\pgfqpoint{2.609718in}{2.486787in}}%
\pgfpathcurveto{\pgfqpoint{2.617532in}{2.494601in}}{\pgfqpoint{2.621922in}{2.505200in}}{\pgfqpoint{2.621922in}{2.516250in}}%
\pgfpathcurveto{\pgfqpoint{2.621922in}{2.527300in}}{\pgfqpoint{2.617532in}{2.537899in}}{\pgfqpoint{2.609718in}{2.545713in}}%
\pgfpathcurveto{\pgfqpoint{2.601905in}{2.553526in}}{\pgfqpoint{2.591306in}{2.557917in}}{\pgfqpoint{2.580256in}{2.557917in}}%
\pgfpathcurveto{\pgfqpoint{2.569206in}{2.557917in}}{\pgfqpoint{2.558607in}{2.553526in}}{\pgfqpoint{2.550793in}{2.545713in}}%
\pgfpathcurveto{\pgfqpoint{2.542979in}{2.537899in}}{\pgfqpoint{2.538589in}{2.527300in}}{\pgfqpoint{2.538589in}{2.516250in}}%
\pgfpathcurveto{\pgfqpoint{2.538589in}{2.505200in}}{\pgfqpoint{2.542979in}{2.494601in}}{\pgfqpoint{2.550793in}{2.486787in}}%
\pgfpathcurveto{\pgfqpoint{2.558607in}{2.478974in}}{\pgfqpoint{2.569206in}{2.474583in}}{\pgfqpoint{2.580256in}{2.474583in}}%
\pgfpathclose%
\pgfusepath{stroke,fill}%
\end{pgfscope}%
\begin{pgfscope}%
\pgfpathrectangle{\pgfqpoint{0.375000in}{0.330000in}}{\pgfqpoint{2.325000in}{2.310000in}}%
\pgfusepath{clip}%
\pgfsetbuttcap%
\pgfsetroundjoin%
\definecolor{currentfill}{rgb}{0.000000,0.000000,0.000000}%
\pgfsetfillcolor{currentfill}%
\pgfsetlinewidth{1.003750pt}%
\definecolor{currentstroke}{rgb}{0.000000,0.000000,0.000000}%
\pgfsetstrokecolor{currentstroke}%
\pgfsetdash{}{0pt}%
\pgfpathmoveto{\pgfqpoint{2.580256in}{2.474583in}}%
\pgfpathcurveto{\pgfqpoint{2.591306in}{2.474583in}}{\pgfqpoint{2.601905in}{2.478974in}}{\pgfqpoint{2.609718in}{2.486787in}}%
\pgfpathcurveto{\pgfqpoint{2.617532in}{2.494601in}}{\pgfqpoint{2.621922in}{2.505200in}}{\pgfqpoint{2.621922in}{2.516250in}}%
\pgfpathcurveto{\pgfqpoint{2.621922in}{2.527300in}}{\pgfqpoint{2.617532in}{2.537899in}}{\pgfqpoint{2.609718in}{2.545713in}}%
\pgfpathcurveto{\pgfqpoint{2.601905in}{2.553526in}}{\pgfqpoint{2.591306in}{2.557917in}}{\pgfqpoint{2.580256in}{2.557917in}}%
\pgfpathcurveto{\pgfqpoint{2.569206in}{2.557917in}}{\pgfqpoint{2.558607in}{2.553526in}}{\pgfqpoint{2.550793in}{2.545713in}}%
\pgfpathcurveto{\pgfqpoint{2.542979in}{2.537899in}}{\pgfqpoint{2.538589in}{2.527300in}}{\pgfqpoint{2.538589in}{2.516250in}}%
\pgfpathcurveto{\pgfqpoint{2.538589in}{2.505200in}}{\pgfqpoint{2.542979in}{2.494601in}}{\pgfqpoint{2.550793in}{2.486787in}}%
\pgfpathcurveto{\pgfqpoint{2.558607in}{2.478974in}}{\pgfqpoint{2.569206in}{2.474583in}}{\pgfqpoint{2.580256in}{2.474583in}}%
\pgfpathclose%
\pgfusepath{stroke,fill}%
\end{pgfscope}%
\begin{pgfscope}%
\pgfpathrectangle{\pgfqpoint{0.375000in}{0.330000in}}{\pgfqpoint{2.325000in}{2.310000in}}%
\pgfusepath{clip}%
\pgfsetbuttcap%
\pgfsetroundjoin%
\definecolor{currentfill}{rgb}{0.000000,0.000000,0.000000}%
\pgfsetfillcolor{currentfill}%
\pgfsetlinewidth{1.003750pt}%
\definecolor{currentstroke}{rgb}{0.000000,0.000000,0.000000}%
\pgfsetstrokecolor{currentstroke}%
\pgfsetdash{}{0pt}%
\pgfpathmoveto{\pgfqpoint{2.580256in}{2.474583in}}%
\pgfpathcurveto{\pgfqpoint{2.591306in}{2.474583in}}{\pgfqpoint{2.601905in}{2.478974in}}{\pgfqpoint{2.609718in}{2.486787in}}%
\pgfpathcurveto{\pgfqpoint{2.617532in}{2.494601in}}{\pgfqpoint{2.621922in}{2.505200in}}{\pgfqpoint{2.621922in}{2.516250in}}%
\pgfpathcurveto{\pgfqpoint{2.621922in}{2.527300in}}{\pgfqpoint{2.617532in}{2.537899in}}{\pgfqpoint{2.609718in}{2.545713in}}%
\pgfpathcurveto{\pgfqpoint{2.601905in}{2.553526in}}{\pgfqpoint{2.591306in}{2.557917in}}{\pgfqpoint{2.580256in}{2.557917in}}%
\pgfpathcurveto{\pgfqpoint{2.569206in}{2.557917in}}{\pgfqpoint{2.558607in}{2.553526in}}{\pgfqpoint{2.550793in}{2.545713in}}%
\pgfpathcurveto{\pgfqpoint{2.542979in}{2.537899in}}{\pgfqpoint{2.538589in}{2.527300in}}{\pgfqpoint{2.538589in}{2.516250in}}%
\pgfpathcurveto{\pgfqpoint{2.538589in}{2.505200in}}{\pgfqpoint{2.542979in}{2.494601in}}{\pgfqpoint{2.550793in}{2.486787in}}%
\pgfpathcurveto{\pgfqpoint{2.558607in}{2.478974in}}{\pgfqpoint{2.569206in}{2.474583in}}{\pgfqpoint{2.580256in}{2.474583in}}%
\pgfpathclose%
\pgfusepath{stroke,fill}%
\end{pgfscope}%
\begin{pgfscope}%
\pgfpathrectangle{\pgfqpoint{0.375000in}{0.330000in}}{\pgfqpoint{2.325000in}{2.310000in}}%
\pgfusepath{clip}%
\pgfsetbuttcap%
\pgfsetroundjoin%
\definecolor{currentfill}{rgb}{0.000000,0.000000,0.000000}%
\pgfsetfillcolor{currentfill}%
\pgfsetlinewidth{1.003750pt}%
\definecolor{currentstroke}{rgb}{0.000000,0.000000,0.000000}%
\pgfsetstrokecolor{currentstroke}%
\pgfsetdash{}{0pt}%
\pgfpathmoveto{\pgfqpoint{2.580256in}{2.474583in}}%
\pgfpathcurveto{\pgfqpoint{2.591306in}{2.474583in}}{\pgfqpoint{2.601905in}{2.478974in}}{\pgfqpoint{2.609718in}{2.486787in}}%
\pgfpathcurveto{\pgfqpoint{2.617532in}{2.494601in}}{\pgfqpoint{2.621922in}{2.505200in}}{\pgfqpoint{2.621922in}{2.516250in}}%
\pgfpathcurveto{\pgfqpoint{2.621922in}{2.527300in}}{\pgfqpoint{2.617532in}{2.537899in}}{\pgfqpoint{2.609718in}{2.545713in}}%
\pgfpathcurveto{\pgfqpoint{2.601905in}{2.553526in}}{\pgfqpoint{2.591306in}{2.557917in}}{\pgfqpoint{2.580256in}{2.557917in}}%
\pgfpathcurveto{\pgfqpoint{2.569206in}{2.557917in}}{\pgfqpoint{2.558607in}{2.553526in}}{\pgfqpoint{2.550793in}{2.545713in}}%
\pgfpathcurveto{\pgfqpoint{2.542979in}{2.537899in}}{\pgfqpoint{2.538589in}{2.527300in}}{\pgfqpoint{2.538589in}{2.516250in}}%
\pgfpathcurveto{\pgfqpoint{2.538589in}{2.505200in}}{\pgfqpoint{2.542979in}{2.494601in}}{\pgfqpoint{2.550793in}{2.486787in}}%
\pgfpathcurveto{\pgfqpoint{2.558607in}{2.478974in}}{\pgfqpoint{2.569206in}{2.474583in}}{\pgfqpoint{2.580256in}{2.474583in}}%
\pgfpathclose%
\pgfusepath{stroke,fill}%
\end{pgfscope}%
\begin{pgfscope}%
\pgfpathrectangle{\pgfqpoint{0.375000in}{0.330000in}}{\pgfqpoint{2.325000in}{2.310000in}}%
\pgfusepath{clip}%
\pgfsetbuttcap%
\pgfsetroundjoin%
\definecolor{currentfill}{rgb}{0.000000,0.000000,0.000000}%
\pgfsetfillcolor{currentfill}%
\pgfsetlinewidth{1.003750pt}%
\definecolor{currentstroke}{rgb}{0.000000,0.000000,0.000000}%
\pgfsetstrokecolor{currentstroke}%
\pgfsetdash{}{0pt}%
\pgfpathmoveto{\pgfqpoint{2.580256in}{2.474583in}}%
\pgfpathcurveto{\pgfqpoint{2.591306in}{2.474583in}}{\pgfqpoint{2.601905in}{2.478974in}}{\pgfqpoint{2.609718in}{2.486787in}}%
\pgfpathcurveto{\pgfqpoint{2.617532in}{2.494601in}}{\pgfqpoint{2.621922in}{2.505200in}}{\pgfqpoint{2.621922in}{2.516250in}}%
\pgfpathcurveto{\pgfqpoint{2.621922in}{2.527300in}}{\pgfqpoint{2.617532in}{2.537899in}}{\pgfqpoint{2.609718in}{2.545713in}}%
\pgfpathcurveto{\pgfqpoint{2.601905in}{2.553526in}}{\pgfqpoint{2.591306in}{2.557917in}}{\pgfqpoint{2.580256in}{2.557917in}}%
\pgfpathcurveto{\pgfqpoint{2.569206in}{2.557917in}}{\pgfqpoint{2.558607in}{2.553526in}}{\pgfqpoint{2.550793in}{2.545713in}}%
\pgfpathcurveto{\pgfqpoint{2.542979in}{2.537899in}}{\pgfqpoint{2.538589in}{2.527300in}}{\pgfqpoint{2.538589in}{2.516250in}}%
\pgfpathcurveto{\pgfqpoint{2.538589in}{2.505200in}}{\pgfqpoint{2.542979in}{2.494601in}}{\pgfqpoint{2.550793in}{2.486787in}}%
\pgfpathcurveto{\pgfqpoint{2.558607in}{2.478974in}}{\pgfqpoint{2.569206in}{2.474583in}}{\pgfqpoint{2.580256in}{2.474583in}}%
\pgfpathclose%
\pgfusepath{stroke,fill}%
\end{pgfscope}%
\begin{pgfscope}%
\pgfpathrectangle{\pgfqpoint{0.375000in}{0.330000in}}{\pgfqpoint{2.325000in}{2.310000in}}%
\pgfusepath{clip}%
\pgfsetbuttcap%
\pgfsetroundjoin%
\definecolor{currentfill}{rgb}{0.000000,0.000000,0.000000}%
\pgfsetfillcolor{currentfill}%
\pgfsetlinewidth{1.003750pt}%
\definecolor{currentstroke}{rgb}{0.000000,0.000000,0.000000}%
\pgfsetstrokecolor{currentstroke}%
\pgfsetdash{}{0pt}%
\pgfpathmoveto{\pgfqpoint{2.580256in}{2.474583in}}%
\pgfpathcurveto{\pgfqpoint{2.591306in}{2.474583in}}{\pgfqpoint{2.601905in}{2.478974in}}{\pgfqpoint{2.609718in}{2.486787in}}%
\pgfpathcurveto{\pgfqpoint{2.617532in}{2.494601in}}{\pgfqpoint{2.621922in}{2.505200in}}{\pgfqpoint{2.621922in}{2.516250in}}%
\pgfpathcurveto{\pgfqpoint{2.621922in}{2.527300in}}{\pgfqpoint{2.617532in}{2.537899in}}{\pgfqpoint{2.609718in}{2.545713in}}%
\pgfpathcurveto{\pgfqpoint{2.601905in}{2.553526in}}{\pgfqpoint{2.591306in}{2.557917in}}{\pgfqpoint{2.580256in}{2.557917in}}%
\pgfpathcurveto{\pgfqpoint{2.569206in}{2.557917in}}{\pgfqpoint{2.558607in}{2.553526in}}{\pgfqpoint{2.550793in}{2.545713in}}%
\pgfpathcurveto{\pgfqpoint{2.542979in}{2.537899in}}{\pgfqpoint{2.538589in}{2.527300in}}{\pgfqpoint{2.538589in}{2.516250in}}%
\pgfpathcurveto{\pgfqpoint{2.538589in}{2.505200in}}{\pgfqpoint{2.542979in}{2.494601in}}{\pgfqpoint{2.550793in}{2.486787in}}%
\pgfpathcurveto{\pgfqpoint{2.558607in}{2.478974in}}{\pgfqpoint{2.569206in}{2.474583in}}{\pgfqpoint{2.580256in}{2.474583in}}%
\pgfpathclose%
\pgfusepath{stroke,fill}%
\end{pgfscope}%
\begin{pgfscope}%
\pgfpathrectangle{\pgfqpoint{0.375000in}{0.330000in}}{\pgfqpoint{2.325000in}{2.310000in}}%
\pgfusepath{clip}%
\pgfsetbuttcap%
\pgfsetroundjoin%
\definecolor{currentfill}{rgb}{0.000000,0.000000,0.000000}%
\pgfsetfillcolor{currentfill}%
\pgfsetlinewidth{1.003750pt}%
\definecolor{currentstroke}{rgb}{0.000000,0.000000,0.000000}%
\pgfsetstrokecolor{currentstroke}%
\pgfsetdash{}{0pt}%
\pgfpathmoveto{\pgfqpoint{2.580256in}{2.474583in}}%
\pgfpathcurveto{\pgfqpoint{2.591306in}{2.474583in}}{\pgfqpoint{2.601905in}{2.478974in}}{\pgfqpoint{2.609718in}{2.486787in}}%
\pgfpathcurveto{\pgfqpoint{2.617532in}{2.494601in}}{\pgfqpoint{2.621922in}{2.505200in}}{\pgfqpoint{2.621922in}{2.516250in}}%
\pgfpathcurveto{\pgfqpoint{2.621922in}{2.527300in}}{\pgfqpoint{2.617532in}{2.537899in}}{\pgfqpoint{2.609718in}{2.545713in}}%
\pgfpathcurveto{\pgfqpoint{2.601905in}{2.553526in}}{\pgfqpoint{2.591306in}{2.557917in}}{\pgfqpoint{2.580256in}{2.557917in}}%
\pgfpathcurveto{\pgfqpoint{2.569206in}{2.557917in}}{\pgfqpoint{2.558607in}{2.553526in}}{\pgfqpoint{2.550793in}{2.545713in}}%
\pgfpathcurveto{\pgfqpoint{2.542979in}{2.537899in}}{\pgfqpoint{2.538589in}{2.527300in}}{\pgfqpoint{2.538589in}{2.516250in}}%
\pgfpathcurveto{\pgfqpoint{2.538589in}{2.505200in}}{\pgfqpoint{2.542979in}{2.494601in}}{\pgfqpoint{2.550793in}{2.486787in}}%
\pgfpathcurveto{\pgfqpoint{2.558607in}{2.478974in}}{\pgfqpoint{2.569206in}{2.474583in}}{\pgfqpoint{2.580256in}{2.474583in}}%
\pgfpathclose%
\pgfusepath{stroke,fill}%
\end{pgfscope}%
\begin{pgfscope}%
\pgfpathrectangle{\pgfqpoint{0.375000in}{0.330000in}}{\pgfqpoint{2.325000in}{2.310000in}}%
\pgfusepath{clip}%
\pgfsetbuttcap%
\pgfsetroundjoin%
\definecolor{currentfill}{rgb}{0.000000,0.000000,0.000000}%
\pgfsetfillcolor{currentfill}%
\pgfsetlinewidth{1.003750pt}%
\definecolor{currentstroke}{rgb}{0.000000,0.000000,0.000000}%
\pgfsetstrokecolor{currentstroke}%
\pgfsetdash{}{0pt}%
\pgfpathmoveto{\pgfqpoint{2.580256in}{2.474583in}}%
\pgfpathcurveto{\pgfqpoint{2.591306in}{2.474583in}}{\pgfqpoint{2.601905in}{2.478974in}}{\pgfqpoint{2.609718in}{2.486787in}}%
\pgfpathcurveto{\pgfqpoint{2.617532in}{2.494601in}}{\pgfqpoint{2.621922in}{2.505200in}}{\pgfqpoint{2.621922in}{2.516250in}}%
\pgfpathcurveto{\pgfqpoint{2.621922in}{2.527300in}}{\pgfqpoint{2.617532in}{2.537899in}}{\pgfqpoint{2.609718in}{2.545713in}}%
\pgfpathcurveto{\pgfqpoint{2.601905in}{2.553526in}}{\pgfqpoint{2.591306in}{2.557917in}}{\pgfqpoint{2.580256in}{2.557917in}}%
\pgfpathcurveto{\pgfqpoint{2.569206in}{2.557917in}}{\pgfqpoint{2.558607in}{2.553526in}}{\pgfqpoint{2.550793in}{2.545713in}}%
\pgfpathcurveto{\pgfqpoint{2.542979in}{2.537899in}}{\pgfqpoint{2.538589in}{2.527300in}}{\pgfqpoint{2.538589in}{2.516250in}}%
\pgfpathcurveto{\pgfqpoint{2.538589in}{2.505200in}}{\pgfqpoint{2.542979in}{2.494601in}}{\pgfqpoint{2.550793in}{2.486787in}}%
\pgfpathcurveto{\pgfqpoint{2.558607in}{2.478974in}}{\pgfqpoint{2.569206in}{2.474583in}}{\pgfqpoint{2.580256in}{2.474583in}}%
\pgfpathclose%
\pgfusepath{stroke,fill}%
\end{pgfscope}%
\begin{pgfscope}%
\pgfpathrectangle{\pgfqpoint{0.375000in}{0.330000in}}{\pgfqpoint{2.325000in}{2.310000in}}%
\pgfusepath{clip}%
\pgfsetbuttcap%
\pgfsetroundjoin%
\definecolor{currentfill}{rgb}{0.000000,0.000000,0.000000}%
\pgfsetfillcolor{currentfill}%
\pgfsetlinewidth{1.003750pt}%
\definecolor{currentstroke}{rgb}{0.000000,0.000000,0.000000}%
\pgfsetstrokecolor{currentstroke}%
\pgfsetdash{}{0pt}%
\pgfpathmoveto{\pgfqpoint{2.580256in}{2.474583in}}%
\pgfpathcurveto{\pgfqpoint{2.591306in}{2.474583in}}{\pgfqpoint{2.601905in}{2.478974in}}{\pgfqpoint{2.609718in}{2.486787in}}%
\pgfpathcurveto{\pgfqpoint{2.617532in}{2.494601in}}{\pgfqpoint{2.621922in}{2.505200in}}{\pgfqpoint{2.621922in}{2.516250in}}%
\pgfpathcurveto{\pgfqpoint{2.621922in}{2.527300in}}{\pgfqpoint{2.617532in}{2.537899in}}{\pgfqpoint{2.609718in}{2.545713in}}%
\pgfpathcurveto{\pgfqpoint{2.601905in}{2.553526in}}{\pgfqpoint{2.591306in}{2.557917in}}{\pgfqpoint{2.580256in}{2.557917in}}%
\pgfpathcurveto{\pgfqpoint{2.569206in}{2.557917in}}{\pgfqpoint{2.558607in}{2.553526in}}{\pgfqpoint{2.550793in}{2.545713in}}%
\pgfpathcurveto{\pgfqpoint{2.542979in}{2.537899in}}{\pgfqpoint{2.538589in}{2.527300in}}{\pgfqpoint{2.538589in}{2.516250in}}%
\pgfpathcurveto{\pgfqpoint{2.538589in}{2.505200in}}{\pgfqpoint{2.542979in}{2.494601in}}{\pgfqpoint{2.550793in}{2.486787in}}%
\pgfpathcurveto{\pgfqpoint{2.558607in}{2.478974in}}{\pgfqpoint{2.569206in}{2.474583in}}{\pgfqpoint{2.580256in}{2.474583in}}%
\pgfpathclose%
\pgfusepath{stroke,fill}%
\end{pgfscope}%
\begin{pgfscope}%
\pgfpathrectangle{\pgfqpoint{0.375000in}{0.330000in}}{\pgfqpoint{2.325000in}{2.310000in}}%
\pgfusepath{clip}%
\pgfsetbuttcap%
\pgfsetroundjoin%
\definecolor{currentfill}{rgb}{0.000000,0.000000,0.000000}%
\pgfsetfillcolor{currentfill}%
\pgfsetlinewidth{1.003750pt}%
\definecolor{currentstroke}{rgb}{0.000000,0.000000,0.000000}%
\pgfsetstrokecolor{currentstroke}%
\pgfsetdash{}{0pt}%
\pgfpathmoveto{\pgfqpoint{2.580256in}{2.474583in}}%
\pgfpathcurveto{\pgfqpoint{2.591306in}{2.474583in}}{\pgfqpoint{2.601905in}{2.478974in}}{\pgfqpoint{2.609718in}{2.486787in}}%
\pgfpathcurveto{\pgfqpoint{2.617532in}{2.494601in}}{\pgfqpoint{2.621922in}{2.505200in}}{\pgfqpoint{2.621922in}{2.516250in}}%
\pgfpathcurveto{\pgfqpoint{2.621922in}{2.527300in}}{\pgfqpoint{2.617532in}{2.537899in}}{\pgfqpoint{2.609718in}{2.545713in}}%
\pgfpathcurveto{\pgfqpoint{2.601905in}{2.553526in}}{\pgfqpoint{2.591306in}{2.557917in}}{\pgfqpoint{2.580256in}{2.557917in}}%
\pgfpathcurveto{\pgfqpoint{2.569206in}{2.557917in}}{\pgfqpoint{2.558607in}{2.553526in}}{\pgfqpoint{2.550793in}{2.545713in}}%
\pgfpathcurveto{\pgfqpoint{2.542979in}{2.537899in}}{\pgfqpoint{2.538589in}{2.527300in}}{\pgfqpoint{2.538589in}{2.516250in}}%
\pgfpathcurveto{\pgfqpoint{2.538589in}{2.505200in}}{\pgfqpoint{2.542979in}{2.494601in}}{\pgfqpoint{2.550793in}{2.486787in}}%
\pgfpathcurveto{\pgfqpoint{2.558607in}{2.478974in}}{\pgfqpoint{2.569206in}{2.474583in}}{\pgfqpoint{2.580256in}{2.474583in}}%
\pgfpathclose%
\pgfusepath{stroke,fill}%
\end{pgfscope}%
\begin{pgfscope}%
\pgfpathrectangle{\pgfqpoint{0.375000in}{0.330000in}}{\pgfqpoint{2.325000in}{2.310000in}}%
\pgfusepath{clip}%
\pgfsetbuttcap%
\pgfsetroundjoin%
\definecolor{currentfill}{rgb}{0.000000,0.000000,0.000000}%
\pgfsetfillcolor{currentfill}%
\pgfsetlinewidth{1.003750pt}%
\definecolor{currentstroke}{rgb}{0.000000,0.000000,0.000000}%
\pgfsetstrokecolor{currentstroke}%
\pgfsetdash{}{0pt}%
\pgfpathmoveto{\pgfqpoint{2.580256in}{2.474583in}}%
\pgfpathcurveto{\pgfqpoint{2.591306in}{2.474583in}}{\pgfqpoint{2.601905in}{2.478974in}}{\pgfqpoint{2.609718in}{2.486787in}}%
\pgfpathcurveto{\pgfqpoint{2.617532in}{2.494601in}}{\pgfqpoint{2.621922in}{2.505200in}}{\pgfqpoint{2.621922in}{2.516250in}}%
\pgfpathcurveto{\pgfqpoint{2.621922in}{2.527300in}}{\pgfqpoint{2.617532in}{2.537899in}}{\pgfqpoint{2.609718in}{2.545713in}}%
\pgfpathcurveto{\pgfqpoint{2.601905in}{2.553526in}}{\pgfqpoint{2.591306in}{2.557917in}}{\pgfqpoint{2.580256in}{2.557917in}}%
\pgfpathcurveto{\pgfqpoint{2.569206in}{2.557917in}}{\pgfqpoint{2.558607in}{2.553526in}}{\pgfqpoint{2.550793in}{2.545713in}}%
\pgfpathcurveto{\pgfqpoint{2.542979in}{2.537899in}}{\pgfqpoint{2.538589in}{2.527300in}}{\pgfqpoint{2.538589in}{2.516250in}}%
\pgfpathcurveto{\pgfqpoint{2.538589in}{2.505200in}}{\pgfqpoint{2.542979in}{2.494601in}}{\pgfqpoint{2.550793in}{2.486787in}}%
\pgfpathcurveto{\pgfqpoint{2.558607in}{2.478974in}}{\pgfqpoint{2.569206in}{2.474583in}}{\pgfqpoint{2.580256in}{2.474583in}}%
\pgfpathclose%
\pgfusepath{stroke,fill}%
\end{pgfscope}%
\begin{pgfscope}%
\pgfpathrectangle{\pgfqpoint{0.375000in}{0.330000in}}{\pgfqpoint{2.325000in}{2.310000in}}%
\pgfusepath{clip}%
\pgfsetbuttcap%
\pgfsetroundjoin%
\definecolor{currentfill}{rgb}{0.000000,0.000000,0.000000}%
\pgfsetfillcolor{currentfill}%
\pgfsetlinewidth{1.003750pt}%
\definecolor{currentstroke}{rgb}{0.000000,0.000000,0.000000}%
\pgfsetstrokecolor{currentstroke}%
\pgfsetdash{}{0pt}%
\pgfpathmoveto{\pgfqpoint{2.580256in}{2.474583in}}%
\pgfpathcurveto{\pgfqpoint{2.591306in}{2.474583in}}{\pgfqpoint{2.601905in}{2.478974in}}{\pgfqpoint{2.609718in}{2.486787in}}%
\pgfpathcurveto{\pgfqpoint{2.617532in}{2.494601in}}{\pgfqpoint{2.621922in}{2.505200in}}{\pgfqpoint{2.621922in}{2.516250in}}%
\pgfpathcurveto{\pgfqpoint{2.621922in}{2.527300in}}{\pgfqpoint{2.617532in}{2.537899in}}{\pgfqpoint{2.609718in}{2.545713in}}%
\pgfpathcurveto{\pgfqpoint{2.601905in}{2.553526in}}{\pgfqpoint{2.591306in}{2.557917in}}{\pgfqpoint{2.580256in}{2.557917in}}%
\pgfpathcurveto{\pgfqpoint{2.569206in}{2.557917in}}{\pgfqpoint{2.558607in}{2.553526in}}{\pgfqpoint{2.550793in}{2.545713in}}%
\pgfpathcurveto{\pgfqpoint{2.542979in}{2.537899in}}{\pgfqpoint{2.538589in}{2.527300in}}{\pgfqpoint{2.538589in}{2.516250in}}%
\pgfpathcurveto{\pgfqpoint{2.538589in}{2.505200in}}{\pgfqpoint{2.542979in}{2.494601in}}{\pgfqpoint{2.550793in}{2.486787in}}%
\pgfpathcurveto{\pgfqpoint{2.558607in}{2.478974in}}{\pgfqpoint{2.569206in}{2.474583in}}{\pgfqpoint{2.580256in}{2.474583in}}%
\pgfpathclose%
\pgfusepath{stroke,fill}%
\end{pgfscope}%
\begin{pgfscope}%
\pgfpathrectangle{\pgfqpoint{0.375000in}{0.330000in}}{\pgfqpoint{2.325000in}{2.310000in}}%
\pgfusepath{clip}%
\pgfsetbuttcap%
\pgfsetroundjoin%
\definecolor{currentfill}{rgb}{0.000000,0.000000,0.000000}%
\pgfsetfillcolor{currentfill}%
\pgfsetlinewidth{1.003750pt}%
\definecolor{currentstroke}{rgb}{0.000000,0.000000,0.000000}%
\pgfsetstrokecolor{currentstroke}%
\pgfsetdash{}{0pt}%
\pgfpathmoveto{\pgfqpoint{2.580256in}{2.474583in}}%
\pgfpathcurveto{\pgfqpoint{2.591306in}{2.474583in}}{\pgfqpoint{2.601905in}{2.478974in}}{\pgfqpoint{2.609718in}{2.486787in}}%
\pgfpathcurveto{\pgfqpoint{2.617532in}{2.494601in}}{\pgfqpoint{2.621922in}{2.505200in}}{\pgfqpoint{2.621922in}{2.516250in}}%
\pgfpathcurveto{\pgfqpoint{2.621922in}{2.527300in}}{\pgfqpoint{2.617532in}{2.537899in}}{\pgfqpoint{2.609718in}{2.545713in}}%
\pgfpathcurveto{\pgfqpoint{2.601905in}{2.553526in}}{\pgfqpoint{2.591306in}{2.557917in}}{\pgfqpoint{2.580256in}{2.557917in}}%
\pgfpathcurveto{\pgfqpoint{2.569206in}{2.557917in}}{\pgfqpoint{2.558607in}{2.553526in}}{\pgfqpoint{2.550793in}{2.545713in}}%
\pgfpathcurveto{\pgfqpoint{2.542979in}{2.537899in}}{\pgfqpoint{2.538589in}{2.527300in}}{\pgfqpoint{2.538589in}{2.516250in}}%
\pgfpathcurveto{\pgfqpoint{2.538589in}{2.505200in}}{\pgfqpoint{2.542979in}{2.494601in}}{\pgfqpoint{2.550793in}{2.486787in}}%
\pgfpathcurveto{\pgfqpoint{2.558607in}{2.478974in}}{\pgfqpoint{2.569206in}{2.474583in}}{\pgfqpoint{2.580256in}{2.474583in}}%
\pgfpathclose%
\pgfusepath{stroke,fill}%
\end{pgfscope}%
\begin{pgfscope}%
\pgfpathrectangle{\pgfqpoint{0.375000in}{0.330000in}}{\pgfqpoint{2.325000in}{2.310000in}}%
\pgfusepath{clip}%
\pgfsetbuttcap%
\pgfsetroundjoin%
\definecolor{currentfill}{rgb}{0.000000,0.000000,0.000000}%
\pgfsetfillcolor{currentfill}%
\pgfsetlinewidth{1.003750pt}%
\definecolor{currentstroke}{rgb}{0.000000,0.000000,0.000000}%
\pgfsetstrokecolor{currentstroke}%
\pgfsetdash{}{0pt}%
\pgfpathmoveto{\pgfqpoint{2.580256in}{2.474583in}}%
\pgfpathcurveto{\pgfqpoint{2.591306in}{2.474583in}}{\pgfqpoint{2.601905in}{2.478974in}}{\pgfqpoint{2.609718in}{2.486787in}}%
\pgfpathcurveto{\pgfqpoint{2.617532in}{2.494601in}}{\pgfqpoint{2.621922in}{2.505200in}}{\pgfqpoint{2.621922in}{2.516250in}}%
\pgfpathcurveto{\pgfqpoint{2.621922in}{2.527300in}}{\pgfqpoint{2.617532in}{2.537899in}}{\pgfqpoint{2.609718in}{2.545713in}}%
\pgfpathcurveto{\pgfqpoint{2.601905in}{2.553526in}}{\pgfqpoint{2.591306in}{2.557917in}}{\pgfqpoint{2.580256in}{2.557917in}}%
\pgfpathcurveto{\pgfqpoint{2.569206in}{2.557917in}}{\pgfqpoint{2.558607in}{2.553526in}}{\pgfqpoint{2.550793in}{2.545713in}}%
\pgfpathcurveto{\pgfqpoint{2.542979in}{2.537899in}}{\pgfqpoint{2.538589in}{2.527300in}}{\pgfqpoint{2.538589in}{2.516250in}}%
\pgfpathcurveto{\pgfqpoint{2.538589in}{2.505200in}}{\pgfqpoint{2.542979in}{2.494601in}}{\pgfqpoint{2.550793in}{2.486787in}}%
\pgfpathcurveto{\pgfqpoint{2.558607in}{2.478974in}}{\pgfqpoint{2.569206in}{2.474583in}}{\pgfqpoint{2.580256in}{2.474583in}}%
\pgfpathclose%
\pgfusepath{stroke,fill}%
\end{pgfscope}%
\begin{pgfscope}%
\pgfpathrectangle{\pgfqpoint{0.375000in}{0.330000in}}{\pgfqpoint{2.325000in}{2.310000in}}%
\pgfusepath{clip}%
\pgfsetbuttcap%
\pgfsetroundjoin%
\definecolor{currentfill}{rgb}{0.000000,0.000000,0.000000}%
\pgfsetfillcolor{currentfill}%
\pgfsetlinewidth{1.003750pt}%
\definecolor{currentstroke}{rgb}{0.000000,0.000000,0.000000}%
\pgfsetstrokecolor{currentstroke}%
\pgfsetdash{}{0pt}%
\pgfpathmoveto{\pgfqpoint{2.580256in}{2.474583in}}%
\pgfpathcurveto{\pgfqpoint{2.591306in}{2.474583in}}{\pgfqpoint{2.601905in}{2.478974in}}{\pgfqpoint{2.609718in}{2.486787in}}%
\pgfpathcurveto{\pgfqpoint{2.617532in}{2.494601in}}{\pgfqpoint{2.621922in}{2.505200in}}{\pgfqpoint{2.621922in}{2.516250in}}%
\pgfpathcurveto{\pgfqpoint{2.621922in}{2.527300in}}{\pgfqpoint{2.617532in}{2.537899in}}{\pgfqpoint{2.609718in}{2.545713in}}%
\pgfpathcurveto{\pgfqpoint{2.601905in}{2.553526in}}{\pgfqpoint{2.591306in}{2.557917in}}{\pgfqpoint{2.580256in}{2.557917in}}%
\pgfpathcurveto{\pgfqpoint{2.569206in}{2.557917in}}{\pgfqpoint{2.558607in}{2.553526in}}{\pgfqpoint{2.550793in}{2.545713in}}%
\pgfpathcurveto{\pgfqpoint{2.542979in}{2.537899in}}{\pgfqpoint{2.538589in}{2.527300in}}{\pgfqpoint{2.538589in}{2.516250in}}%
\pgfpathcurveto{\pgfqpoint{2.538589in}{2.505200in}}{\pgfqpoint{2.542979in}{2.494601in}}{\pgfqpoint{2.550793in}{2.486787in}}%
\pgfpathcurveto{\pgfqpoint{2.558607in}{2.478974in}}{\pgfqpoint{2.569206in}{2.474583in}}{\pgfqpoint{2.580256in}{2.474583in}}%
\pgfpathclose%
\pgfusepath{stroke,fill}%
\end{pgfscope}%
\begin{pgfscope}%
\pgfpathrectangle{\pgfqpoint{0.375000in}{0.330000in}}{\pgfqpoint{2.325000in}{2.310000in}}%
\pgfusepath{clip}%
\pgfsetbuttcap%
\pgfsetroundjoin%
\definecolor{currentfill}{rgb}{0.000000,0.000000,0.000000}%
\pgfsetfillcolor{currentfill}%
\pgfsetlinewidth{1.003750pt}%
\definecolor{currentstroke}{rgb}{0.000000,0.000000,0.000000}%
\pgfsetstrokecolor{currentstroke}%
\pgfsetdash{}{0pt}%
\pgfpathmoveto{\pgfqpoint{2.580256in}{2.474583in}}%
\pgfpathcurveto{\pgfqpoint{2.591306in}{2.474583in}}{\pgfqpoint{2.601905in}{2.478974in}}{\pgfqpoint{2.609718in}{2.486787in}}%
\pgfpathcurveto{\pgfqpoint{2.617532in}{2.494601in}}{\pgfqpoint{2.621922in}{2.505200in}}{\pgfqpoint{2.621922in}{2.516250in}}%
\pgfpathcurveto{\pgfqpoint{2.621922in}{2.527300in}}{\pgfqpoint{2.617532in}{2.537899in}}{\pgfqpoint{2.609718in}{2.545713in}}%
\pgfpathcurveto{\pgfqpoint{2.601905in}{2.553526in}}{\pgfqpoint{2.591306in}{2.557917in}}{\pgfqpoint{2.580256in}{2.557917in}}%
\pgfpathcurveto{\pgfqpoint{2.569206in}{2.557917in}}{\pgfqpoint{2.558607in}{2.553526in}}{\pgfqpoint{2.550793in}{2.545713in}}%
\pgfpathcurveto{\pgfqpoint{2.542979in}{2.537899in}}{\pgfqpoint{2.538589in}{2.527300in}}{\pgfqpoint{2.538589in}{2.516250in}}%
\pgfpathcurveto{\pgfqpoint{2.538589in}{2.505200in}}{\pgfqpoint{2.542979in}{2.494601in}}{\pgfqpoint{2.550793in}{2.486787in}}%
\pgfpathcurveto{\pgfqpoint{2.558607in}{2.478974in}}{\pgfqpoint{2.569206in}{2.474583in}}{\pgfqpoint{2.580256in}{2.474583in}}%
\pgfpathclose%
\pgfusepath{stroke,fill}%
\end{pgfscope}%
\begin{pgfscope}%
\pgfpathrectangle{\pgfqpoint{0.375000in}{0.330000in}}{\pgfqpoint{2.325000in}{2.310000in}}%
\pgfusepath{clip}%
\pgfsetbuttcap%
\pgfsetroundjoin%
\definecolor{currentfill}{rgb}{0.000000,0.000000,0.000000}%
\pgfsetfillcolor{currentfill}%
\pgfsetlinewidth{1.003750pt}%
\definecolor{currentstroke}{rgb}{0.000000,0.000000,0.000000}%
\pgfsetstrokecolor{currentstroke}%
\pgfsetdash{}{0pt}%
\pgfpathmoveto{\pgfqpoint{2.580256in}{2.474583in}}%
\pgfpathcurveto{\pgfqpoint{2.591306in}{2.474583in}}{\pgfqpoint{2.601905in}{2.478974in}}{\pgfqpoint{2.609718in}{2.486787in}}%
\pgfpathcurveto{\pgfqpoint{2.617532in}{2.494601in}}{\pgfqpoint{2.621922in}{2.505200in}}{\pgfqpoint{2.621922in}{2.516250in}}%
\pgfpathcurveto{\pgfqpoint{2.621922in}{2.527300in}}{\pgfqpoint{2.617532in}{2.537899in}}{\pgfqpoint{2.609718in}{2.545713in}}%
\pgfpathcurveto{\pgfqpoint{2.601905in}{2.553526in}}{\pgfqpoint{2.591306in}{2.557917in}}{\pgfqpoint{2.580256in}{2.557917in}}%
\pgfpathcurveto{\pgfqpoint{2.569206in}{2.557917in}}{\pgfqpoint{2.558607in}{2.553526in}}{\pgfqpoint{2.550793in}{2.545713in}}%
\pgfpathcurveto{\pgfqpoint{2.542979in}{2.537899in}}{\pgfqpoint{2.538589in}{2.527300in}}{\pgfqpoint{2.538589in}{2.516250in}}%
\pgfpathcurveto{\pgfqpoint{2.538589in}{2.505200in}}{\pgfqpoint{2.542979in}{2.494601in}}{\pgfqpoint{2.550793in}{2.486787in}}%
\pgfpathcurveto{\pgfqpoint{2.558607in}{2.478974in}}{\pgfqpoint{2.569206in}{2.474583in}}{\pgfqpoint{2.580256in}{2.474583in}}%
\pgfpathclose%
\pgfusepath{stroke,fill}%
\end{pgfscope}%
\begin{pgfscope}%
\pgfpathrectangle{\pgfqpoint{0.375000in}{0.330000in}}{\pgfqpoint{2.325000in}{2.310000in}}%
\pgfusepath{clip}%
\pgfsetbuttcap%
\pgfsetroundjoin%
\definecolor{currentfill}{rgb}{0.000000,0.000000,0.000000}%
\pgfsetfillcolor{currentfill}%
\pgfsetlinewidth{1.003750pt}%
\definecolor{currentstroke}{rgb}{0.000000,0.000000,0.000000}%
\pgfsetstrokecolor{currentstroke}%
\pgfsetdash{}{0pt}%
\pgfpathmoveto{\pgfqpoint{2.580256in}{2.474583in}}%
\pgfpathcurveto{\pgfqpoint{2.591306in}{2.474583in}}{\pgfqpoint{2.601905in}{2.478974in}}{\pgfqpoint{2.609718in}{2.486787in}}%
\pgfpathcurveto{\pgfqpoint{2.617532in}{2.494601in}}{\pgfqpoint{2.621922in}{2.505200in}}{\pgfqpoint{2.621922in}{2.516250in}}%
\pgfpathcurveto{\pgfqpoint{2.621922in}{2.527300in}}{\pgfqpoint{2.617532in}{2.537899in}}{\pgfqpoint{2.609718in}{2.545713in}}%
\pgfpathcurveto{\pgfqpoint{2.601905in}{2.553526in}}{\pgfqpoint{2.591306in}{2.557917in}}{\pgfqpoint{2.580256in}{2.557917in}}%
\pgfpathcurveto{\pgfqpoint{2.569206in}{2.557917in}}{\pgfqpoint{2.558607in}{2.553526in}}{\pgfqpoint{2.550793in}{2.545713in}}%
\pgfpathcurveto{\pgfqpoint{2.542979in}{2.537899in}}{\pgfqpoint{2.538589in}{2.527300in}}{\pgfqpoint{2.538589in}{2.516250in}}%
\pgfpathcurveto{\pgfqpoint{2.538589in}{2.505200in}}{\pgfqpoint{2.542979in}{2.494601in}}{\pgfqpoint{2.550793in}{2.486787in}}%
\pgfpathcurveto{\pgfqpoint{2.558607in}{2.478974in}}{\pgfqpoint{2.569206in}{2.474583in}}{\pgfqpoint{2.580256in}{2.474583in}}%
\pgfpathclose%
\pgfusepath{stroke,fill}%
\end{pgfscope}%
\begin{pgfscope}%
\pgfpathrectangle{\pgfqpoint{0.375000in}{0.330000in}}{\pgfqpoint{2.325000in}{2.310000in}}%
\pgfusepath{clip}%
\pgfsetbuttcap%
\pgfsetroundjoin%
\definecolor{currentfill}{rgb}{0.000000,0.000000,0.000000}%
\pgfsetfillcolor{currentfill}%
\pgfsetlinewidth{1.003750pt}%
\definecolor{currentstroke}{rgb}{0.000000,0.000000,0.000000}%
\pgfsetstrokecolor{currentstroke}%
\pgfsetdash{}{0pt}%
\pgfpathmoveto{\pgfqpoint{2.580256in}{1.438667in}}%
\pgfpathcurveto{\pgfqpoint{2.591306in}{1.438667in}}{\pgfqpoint{2.601905in}{1.443057in}}{\pgfqpoint{2.609718in}{1.450871in}}%
\pgfpathcurveto{\pgfqpoint{2.617532in}{1.458685in}}{\pgfqpoint{2.621922in}{1.469284in}}{\pgfqpoint{2.621922in}{1.480334in}}%
\pgfpathcurveto{\pgfqpoint{2.621922in}{1.491384in}}{\pgfqpoint{2.617532in}{1.501983in}}{\pgfqpoint{2.609718in}{1.509796in}}%
\pgfpathcurveto{\pgfqpoint{2.601905in}{1.517610in}}{\pgfqpoint{2.591306in}{1.522000in}}{\pgfqpoint{2.580256in}{1.522000in}}%
\pgfpathcurveto{\pgfqpoint{2.569206in}{1.522000in}}{\pgfqpoint{2.558607in}{1.517610in}}{\pgfqpoint{2.550793in}{1.509796in}}%
\pgfpathcurveto{\pgfqpoint{2.542979in}{1.501983in}}{\pgfqpoint{2.538589in}{1.491384in}}{\pgfqpoint{2.538589in}{1.480334in}}%
\pgfpathcurveto{\pgfqpoint{2.538589in}{1.469284in}}{\pgfqpoint{2.542979in}{1.458685in}}{\pgfqpoint{2.550793in}{1.450871in}}%
\pgfpathcurveto{\pgfqpoint{2.558607in}{1.443057in}}{\pgfqpoint{2.569206in}{1.438667in}}{\pgfqpoint{2.580256in}{1.438667in}}%
\pgfpathclose%
\pgfusepath{stroke,fill}%
\end{pgfscope}%
\begin{pgfscope}%
\pgfpathrectangle{\pgfqpoint{0.375000in}{0.330000in}}{\pgfqpoint{2.325000in}{2.310000in}}%
\pgfusepath{clip}%
\pgfsetbuttcap%
\pgfsetroundjoin%
\definecolor{currentfill}{rgb}{0.000000,0.000000,0.000000}%
\pgfsetfillcolor{currentfill}%
\pgfsetlinewidth{1.003750pt}%
\definecolor{currentstroke}{rgb}{0.000000,0.000000,0.000000}%
\pgfsetstrokecolor{currentstroke}%
\pgfsetdash{}{0pt}%
\pgfpathmoveto{\pgfqpoint{2.580256in}{2.474583in}}%
\pgfpathcurveto{\pgfqpoint{2.591306in}{2.474583in}}{\pgfqpoint{2.601905in}{2.478974in}}{\pgfqpoint{2.609718in}{2.486787in}}%
\pgfpathcurveto{\pgfqpoint{2.617532in}{2.494601in}}{\pgfqpoint{2.621922in}{2.505200in}}{\pgfqpoint{2.621922in}{2.516250in}}%
\pgfpathcurveto{\pgfqpoint{2.621922in}{2.527300in}}{\pgfqpoint{2.617532in}{2.537899in}}{\pgfqpoint{2.609718in}{2.545713in}}%
\pgfpathcurveto{\pgfqpoint{2.601905in}{2.553526in}}{\pgfqpoint{2.591306in}{2.557917in}}{\pgfqpoint{2.580256in}{2.557917in}}%
\pgfpathcurveto{\pgfqpoint{2.569206in}{2.557917in}}{\pgfqpoint{2.558607in}{2.553526in}}{\pgfqpoint{2.550793in}{2.545713in}}%
\pgfpathcurveto{\pgfqpoint{2.542979in}{2.537899in}}{\pgfqpoint{2.538589in}{2.527300in}}{\pgfqpoint{2.538589in}{2.516250in}}%
\pgfpathcurveto{\pgfqpoint{2.538589in}{2.505200in}}{\pgfqpoint{2.542979in}{2.494601in}}{\pgfqpoint{2.550793in}{2.486787in}}%
\pgfpathcurveto{\pgfqpoint{2.558607in}{2.478974in}}{\pgfqpoint{2.569206in}{2.474583in}}{\pgfqpoint{2.580256in}{2.474583in}}%
\pgfpathclose%
\pgfusepath{stroke,fill}%
\end{pgfscope}%
\begin{pgfscope}%
\pgfpathrectangle{\pgfqpoint{0.375000in}{0.330000in}}{\pgfqpoint{2.325000in}{2.310000in}}%
\pgfusepath{clip}%
\pgfsetbuttcap%
\pgfsetroundjoin%
\definecolor{currentfill}{rgb}{0.000000,0.000000,0.000000}%
\pgfsetfillcolor{currentfill}%
\pgfsetlinewidth{1.003750pt}%
\definecolor{currentstroke}{rgb}{0.000000,0.000000,0.000000}%
\pgfsetstrokecolor{currentstroke}%
\pgfsetdash{}{0pt}%
\pgfpathmoveto{\pgfqpoint{2.580256in}{2.474583in}}%
\pgfpathcurveto{\pgfqpoint{2.591306in}{2.474583in}}{\pgfqpoint{2.601905in}{2.478974in}}{\pgfqpoint{2.609718in}{2.486787in}}%
\pgfpathcurveto{\pgfqpoint{2.617532in}{2.494601in}}{\pgfqpoint{2.621922in}{2.505200in}}{\pgfqpoint{2.621922in}{2.516250in}}%
\pgfpathcurveto{\pgfqpoint{2.621922in}{2.527300in}}{\pgfqpoint{2.617532in}{2.537899in}}{\pgfqpoint{2.609718in}{2.545713in}}%
\pgfpathcurveto{\pgfqpoint{2.601905in}{2.553526in}}{\pgfqpoint{2.591306in}{2.557917in}}{\pgfqpoint{2.580256in}{2.557917in}}%
\pgfpathcurveto{\pgfqpoint{2.569206in}{2.557917in}}{\pgfqpoint{2.558607in}{2.553526in}}{\pgfqpoint{2.550793in}{2.545713in}}%
\pgfpathcurveto{\pgfqpoint{2.542979in}{2.537899in}}{\pgfqpoint{2.538589in}{2.527300in}}{\pgfqpoint{2.538589in}{2.516250in}}%
\pgfpathcurveto{\pgfqpoint{2.538589in}{2.505200in}}{\pgfqpoint{2.542979in}{2.494601in}}{\pgfqpoint{2.550793in}{2.486787in}}%
\pgfpathcurveto{\pgfqpoint{2.558607in}{2.478974in}}{\pgfqpoint{2.569206in}{2.474583in}}{\pgfqpoint{2.580256in}{2.474583in}}%
\pgfpathclose%
\pgfusepath{stroke,fill}%
\end{pgfscope}%
\begin{pgfscope}%
\pgfpathrectangle{\pgfqpoint{0.375000in}{0.330000in}}{\pgfqpoint{2.325000in}{2.310000in}}%
\pgfusepath{clip}%
\pgfsetbuttcap%
\pgfsetroundjoin%
\definecolor{currentfill}{rgb}{0.000000,0.000000,0.000000}%
\pgfsetfillcolor{currentfill}%
\pgfsetlinewidth{1.003750pt}%
\definecolor{currentstroke}{rgb}{0.000000,0.000000,0.000000}%
\pgfsetstrokecolor{currentstroke}%
\pgfsetdash{}{0pt}%
\pgfpathmoveto{\pgfqpoint{2.580256in}{2.474583in}}%
\pgfpathcurveto{\pgfqpoint{2.591306in}{2.474583in}}{\pgfqpoint{2.601905in}{2.478974in}}{\pgfqpoint{2.609718in}{2.486787in}}%
\pgfpathcurveto{\pgfqpoint{2.617532in}{2.494601in}}{\pgfqpoint{2.621922in}{2.505200in}}{\pgfqpoint{2.621922in}{2.516250in}}%
\pgfpathcurveto{\pgfqpoint{2.621922in}{2.527300in}}{\pgfqpoint{2.617532in}{2.537899in}}{\pgfqpoint{2.609718in}{2.545713in}}%
\pgfpathcurveto{\pgfqpoint{2.601905in}{2.553526in}}{\pgfqpoint{2.591306in}{2.557917in}}{\pgfqpoint{2.580256in}{2.557917in}}%
\pgfpathcurveto{\pgfqpoint{2.569206in}{2.557917in}}{\pgfqpoint{2.558607in}{2.553526in}}{\pgfqpoint{2.550793in}{2.545713in}}%
\pgfpathcurveto{\pgfqpoint{2.542979in}{2.537899in}}{\pgfqpoint{2.538589in}{2.527300in}}{\pgfqpoint{2.538589in}{2.516250in}}%
\pgfpathcurveto{\pgfqpoint{2.538589in}{2.505200in}}{\pgfqpoint{2.542979in}{2.494601in}}{\pgfqpoint{2.550793in}{2.486787in}}%
\pgfpathcurveto{\pgfqpoint{2.558607in}{2.478974in}}{\pgfqpoint{2.569206in}{2.474583in}}{\pgfqpoint{2.580256in}{2.474583in}}%
\pgfpathclose%
\pgfusepath{stroke,fill}%
\end{pgfscope}%
\begin{pgfscope}%
\pgfpathrectangle{\pgfqpoint{0.375000in}{0.330000in}}{\pgfqpoint{2.325000in}{2.310000in}}%
\pgfusepath{clip}%
\pgfsetbuttcap%
\pgfsetroundjoin%
\definecolor{currentfill}{rgb}{0.000000,0.000000,0.000000}%
\pgfsetfillcolor{currentfill}%
\pgfsetlinewidth{1.003750pt}%
\definecolor{currentstroke}{rgb}{0.000000,0.000000,0.000000}%
\pgfsetstrokecolor{currentstroke}%
\pgfsetdash{}{0pt}%
\pgfpathmoveto{\pgfqpoint{2.580256in}{2.474583in}}%
\pgfpathcurveto{\pgfqpoint{2.591306in}{2.474583in}}{\pgfqpoint{2.601905in}{2.478974in}}{\pgfqpoint{2.609718in}{2.486787in}}%
\pgfpathcurveto{\pgfqpoint{2.617532in}{2.494601in}}{\pgfqpoint{2.621922in}{2.505200in}}{\pgfqpoint{2.621922in}{2.516250in}}%
\pgfpathcurveto{\pgfqpoint{2.621922in}{2.527300in}}{\pgfqpoint{2.617532in}{2.537899in}}{\pgfqpoint{2.609718in}{2.545713in}}%
\pgfpathcurveto{\pgfqpoint{2.601905in}{2.553526in}}{\pgfqpoint{2.591306in}{2.557917in}}{\pgfqpoint{2.580256in}{2.557917in}}%
\pgfpathcurveto{\pgfqpoint{2.569206in}{2.557917in}}{\pgfqpoint{2.558607in}{2.553526in}}{\pgfqpoint{2.550793in}{2.545713in}}%
\pgfpathcurveto{\pgfqpoint{2.542979in}{2.537899in}}{\pgfqpoint{2.538589in}{2.527300in}}{\pgfqpoint{2.538589in}{2.516250in}}%
\pgfpathcurveto{\pgfqpoint{2.538589in}{2.505200in}}{\pgfqpoint{2.542979in}{2.494601in}}{\pgfqpoint{2.550793in}{2.486787in}}%
\pgfpathcurveto{\pgfqpoint{2.558607in}{2.478974in}}{\pgfqpoint{2.569206in}{2.474583in}}{\pgfqpoint{2.580256in}{2.474583in}}%
\pgfpathclose%
\pgfusepath{stroke,fill}%
\end{pgfscope}%
\begin{pgfscope}%
\pgfpathrectangle{\pgfqpoint{0.375000in}{0.330000in}}{\pgfqpoint{2.325000in}{2.310000in}}%
\pgfusepath{clip}%
\pgfsetbuttcap%
\pgfsetroundjoin%
\definecolor{currentfill}{rgb}{0.000000,0.000000,0.000000}%
\pgfsetfillcolor{currentfill}%
\pgfsetlinewidth{1.003750pt}%
\definecolor{currentstroke}{rgb}{0.000000,0.000000,0.000000}%
\pgfsetstrokecolor{currentstroke}%
\pgfsetdash{}{0pt}%
\pgfpathmoveto{\pgfqpoint{2.580256in}{2.474583in}}%
\pgfpathcurveto{\pgfqpoint{2.591306in}{2.474583in}}{\pgfqpoint{2.601905in}{2.478974in}}{\pgfqpoint{2.609718in}{2.486787in}}%
\pgfpathcurveto{\pgfqpoint{2.617532in}{2.494601in}}{\pgfqpoint{2.621922in}{2.505200in}}{\pgfqpoint{2.621922in}{2.516250in}}%
\pgfpathcurveto{\pgfqpoint{2.621922in}{2.527300in}}{\pgfqpoint{2.617532in}{2.537899in}}{\pgfqpoint{2.609718in}{2.545713in}}%
\pgfpathcurveto{\pgfqpoint{2.601905in}{2.553526in}}{\pgfqpoint{2.591306in}{2.557917in}}{\pgfqpoint{2.580256in}{2.557917in}}%
\pgfpathcurveto{\pgfqpoint{2.569206in}{2.557917in}}{\pgfqpoint{2.558607in}{2.553526in}}{\pgfqpoint{2.550793in}{2.545713in}}%
\pgfpathcurveto{\pgfqpoint{2.542979in}{2.537899in}}{\pgfqpoint{2.538589in}{2.527300in}}{\pgfqpoint{2.538589in}{2.516250in}}%
\pgfpathcurveto{\pgfqpoint{2.538589in}{2.505200in}}{\pgfqpoint{2.542979in}{2.494601in}}{\pgfqpoint{2.550793in}{2.486787in}}%
\pgfpathcurveto{\pgfqpoint{2.558607in}{2.478974in}}{\pgfqpoint{2.569206in}{2.474583in}}{\pgfqpoint{2.580256in}{2.474583in}}%
\pgfpathclose%
\pgfusepath{stroke,fill}%
\end{pgfscope}%
\begin{pgfscope}%
\pgfpathrectangle{\pgfqpoint{0.375000in}{0.330000in}}{\pgfqpoint{2.325000in}{2.310000in}}%
\pgfusepath{clip}%
\pgfsetbuttcap%
\pgfsetroundjoin%
\definecolor{currentfill}{rgb}{0.000000,0.000000,0.000000}%
\pgfsetfillcolor{currentfill}%
\pgfsetlinewidth{1.003750pt}%
\definecolor{currentstroke}{rgb}{0.000000,0.000000,0.000000}%
\pgfsetstrokecolor{currentstroke}%
\pgfsetdash{}{0pt}%
\pgfpathmoveto{\pgfqpoint{2.580256in}{1.438667in}}%
\pgfpathcurveto{\pgfqpoint{2.591306in}{1.438667in}}{\pgfqpoint{2.601905in}{1.443057in}}{\pgfqpoint{2.609718in}{1.450871in}}%
\pgfpathcurveto{\pgfqpoint{2.617532in}{1.458685in}}{\pgfqpoint{2.621922in}{1.469284in}}{\pgfqpoint{2.621922in}{1.480334in}}%
\pgfpathcurveto{\pgfqpoint{2.621922in}{1.491384in}}{\pgfqpoint{2.617532in}{1.501983in}}{\pgfqpoint{2.609718in}{1.509796in}}%
\pgfpathcurveto{\pgfqpoint{2.601905in}{1.517610in}}{\pgfqpoint{2.591306in}{1.522000in}}{\pgfqpoint{2.580256in}{1.522000in}}%
\pgfpathcurveto{\pgfqpoint{2.569206in}{1.522000in}}{\pgfqpoint{2.558607in}{1.517610in}}{\pgfqpoint{2.550793in}{1.509796in}}%
\pgfpathcurveto{\pgfqpoint{2.542979in}{1.501983in}}{\pgfqpoint{2.538589in}{1.491384in}}{\pgfqpoint{2.538589in}{1.480334in}}%
\pgfpathcurveto{\pgfqpoint{2.538589in}{1.469284in}}{\pgfqpoint{2.542979in}{1.458685in}}{\pgfqpoint{2.550793in}{1.450871in}}%
\pgfpathcurveto{\pgfqpoint{2.558607in}{1.443057in}}{\pgfqpoint{2.569206in}{1.438667in}}{\pgfqpoint{2.580256in}{1.438667in}}%
\pgfpathclose%
\pgfusepath{stroke,fill}%
\end{pgfscope}%
\begin{pgfscope}%
\pgfpathrectangle{\pgfqpoint{0.375000in}{0.330000in}}{\pgfqpoint{2.325000in}{2.310000in}}%
\pgfusepath{clip}%
\pgfsetbuttcap%
\pgfsetroundjoin%
\definecolor{currentfill}{rgb}{0.000000,0.000000,0.000000}%
\pgfsetfillcolor{currentfill}%
\pgfsetlinewidth{1.003750pt}%
\definecolor{currentstroke}{rgb}{0.000000,0.000000,0.000000}%
\pgfsetstrokecolor{currentstroke}%
\pgfsetdash{}{0pt}%
\pgfpathmoveto{\pgfqpoint{2.580256in}{2.474583in}}%
\pgfpathcurveto{\pgfqpoint{2.591306in}{2.474583in}}{\pgfqpoint{2.601905in}{2.478974in}}{\pgfqpoint{2.609718in}{2.486787in}}%
\pgfpathcurveto{\pgfqpoint{2.617532in}{2.494601in}}{\pgfqpoint{2.621922in}{2.505200in}}{\pgfqpoint{2.621922in}{2.516250in}}%
\pgfpathcurveto{\pgfqpoint{2.621922in}{2.527300in}}{\pgfqpoint{2.617532in}{2.537899in}}{\pgfqpoint{2.609718in}{2.545713in}}%
\pgfpathcurveto{\pgfqpoint{2.601905in}{2.553526in}}{\pgfqpoint{2.591306in}{2.557917in}}{\pgfqpoint{2.580256in}{2.557917in}}%
\pgfpathcurveto{\pgfqpoint{2.569206in}{2.557917in}}{\pgfqpoint{2.558607in}{2.553526in}}{\pgfqpoint{2.550793in}{2.545713in}}%
\pgfpathcurveto{\pgfqpoint{2.542979in}{2.537899in}}{\pgfqpoint{2.538589in}{2.527300in}}{\pgfqpoint{2.538589in}{2.516250in}}%
\pgfpathcurveto{\pgfqpoint{2.538589in}{2.505200in}}{\pgfqpoint{2.542979in}{2.494601in}}{\pgfqpoint{2.550793in}{2.486787in}}%
\pgfpathcurveto{\pgfqpoint{2.558607in}{2.478974in}}{\pgfqpoint{2.569206in}{2.474583in}}{\pgfqpoint{2.580256in}{2.474583in}}%
\pgfpathclose%
\pgfusepath{stroke,fill}%
\end{pgfscope}%
\begin{pgfscope}%
\pgfpathrectangle{\pgfqpoint{0.375000in}{0.330000in}}{\pgfqpoint{2.325000in}{2.310000in}}%
\pgfusepath{clip}%
\pgfsetbuttcap%
\pgfsetroundjoin%
\definecolor{currentfill}{rgb}{0.000000,0.000000,0.000000}%
\pgfsetfillcolor{currentfill}%
\pgfsetlinewidth{1.003750pt}%
\definecolor{currentstroke}{rgb}{0.000000,0.000000,0.000000}%
\pgfsetstrokecolor{currentstroke}%
\pgfsetdash{}{0pt}%
\pgfpathmoveto{\pgfqpoint{2.580256in}{2.474583in}}%
\pgfpathcurveto{\pgfqpoint{2.591306in}{2.474583in}}{\pgfqpoint{2.601905in}{2.478974in}}{\pgfqpoint{2.609718in}{2.486787in}}%
\pgfpathcurveto{\pgfqpoint{2.617532in}{2.494601in}}{\pgfqpoint{2.621922in}{2.505200in}}{\pgfqpoint{2.621922in}{2.516250in}}%
\pgfpathcurveto{\pgfqpoint{2.621922in}{2.527300in}}{\pgfqpoint{2.617532in}{2.537899in}}{\pgfqpoint{2.609718in}{2.545713in}}%
\pgfpathcurveto{\pgfqpoint{2.601905in}{2.553526in}}{\pgfqpoint{2.591306in}{2.557917in}}{\pgfqpoint{2.580256in}{2.557917in}}%
\pgfpathcurveto{\pgfqpoint{2.569206in}{2.557917in}}{\pgfqpoint{2.558607in}{2.553526in}}{\pgfqpoint{2.550793in}{2.545713in}}%
\pgfpathcurveto{\pgfqpoint{2.542979in}{2.537899in}}{\pgfqpoint{2.538589in}{2.527300in}}{\pgfqpoint{2.538589in}{2.516250in}}%
\pgfpathcurveto{\pgfqpoint{2.538589in}{2.505200in}}{\pgfqpoint{2.542979in}{2.494601in}}{\pgfqpoint{2.550793in}{2.486787in}}%
\pgfpathcurveto{\pgfqpoint{2.558607in}{2.478974in}}{\pgfqpoint{2.569206in}{2.474583in}}{\pgfqpoint{2.580256in}{2.474583in}}%
\pgfpathclose%
\pgfusepath{stroke,fill}%
\end{pgfscope}%
\begin{pgfscope}%
\pgfpathrectangle{\pgfqpoint{0.375000in}{0.330000in}}{\pgfqpoint{2.325000in}{2.310000in}}%
\pgfusepath{clip}%
\pgfsetbuttcap%
\pgfsetroundjoin%
\definecolor{currentfill}{rgb}{0.000000,0.000000,0.000000}%
\pgfsetfillcolor{currentfill}%
\pgfsetlinewidth{1.003750pt}%
\definecolor{currentstroke}{rgb}{0.000000,0.000000,0.000000}%
\pgfsetstrokecolor{currentstroke}%
\pgfsetdash{}{0pt}%
\pgfpathmoveto{\pgfqpoint{2.580256in}{2.474583in}}%
\pgfpathcurveto{\pgfqpoint{2.591306in}{2.474583in}}{\pgfqpoint{2.601905in}{2.478974in}}{\pgfqpoint{2.609718in}{2.486787in}}%
\pgfpathcurveto{\pgfqpoint{2.617532in}{2.494601in}}{\pgfqpoint{2.621922in}{2.505200in}}{\pgfqpoint{2.621922in}{2.516250in}}%
\pgfpathcurveto{\pgfqpoint{2.621922in}{2.527300in}}{\pgfqpoint{2.617532in}{2.537899in}}{\pgfqpoint{2.609718in}{2.545713in}}%
\pgfpathcurveto{\pgfqpoint{2.601905in}{2.553526in}}{\pgfqpoint{2.591306in}{2.557917in}}{\pgfqpoint{2.580256in}{2.557917in}}%
\pgfpathcurveto{\pgfqpoint{2.569206in}{2.557917in}}{\pgfqpoint{2.558607in}{2.553526in}}{\pgfqpoint{2.550793in}{2.545713in}}%
\pgfpathcurveto{\pgfqpoint{2.542979in}{2.537899in}}{\pgfqpoint{2.538589in}{2.527300in}}{\pgfqpoint{2.538589in}{2.516250in}}%
\pgfpathcurveto{\pgfqpoint{2.538589in}{2.505200in}}{\pgfqpoint{2.542979in}{2.494601in}}{\pgfqpoint{2.550793in}{2.486787in}}%
\pgfpathcurveto{\pgfqpoint{2.558607in}{2.478974in}}{\pgfqpoint{2.569206in}{2.474583in}}{\pgfqpoint{2.580256in}{2.474583in}}%
\pgfpathclose%
\pgfusepath{stroke,fill}%
\end{pgfscope}%
\begin{pgfscope}%
\pgfpathrectangle{\pgfqpoint{0.375000in}{0.330000in}}{\pgfqpoint{2.325000in}{2.310000in}}%
\pgfusepath{clip}%
\pgfsetbuttcap%
\pgfsetroundjoin%
\definecolor{currentfill}{rgb}{0.000000,0.000000,0.000000}%
\pgfsetfillcolor{currentfill}%
\pgfsetlinewidth{1.003750pt}%
\definecolor{currentstroke}{rgb}{0.000000,0.000000,0.000000}%
\pgfsetstrokecolor{currentstroke}%
\pgfsetdash{}{0pt}%
\pgfpathmoveto{\pgfqpoint{2.580256in}{2.474583in}}%
\pgfpathcurveto{\pgfqpoint{2.591306in}{2.474583in}}{\pgfqpoint{2.601905in}{2.478974in}}{\pgfqpoint{2.609718in}{2.486787in}}%
\pgfpathcurveto{\pgfqpoint{2.617532in}{2.494601in}}{\pgfqpoint{2.621922in}{2.505200in}}{\pgfqpoint{2.621922in}{2.516250in}}%
\pgfpathcurveto{\pgfqpoint{2.621922in}{2.527300in}}{\pgfqpoint{2.617532in}{2.537899in}}{\pgfqpoint{2.609718in}{2.545713in}}%
\pgfpathcurveto{\pgfqpoint{2.601905in}{2.553526in}}{\pgfqpoint{2.591306in}{2.557917in}}{\pgfqpoint{2.580256in}{2.557917in}}%
\pgfpathcurveto{\pgfqpoint{2.569206in}{2.557917in}}{\pgfqpoint{2.558607in}{2.553526in}}{\pgfqpoint{2.550793in}{2.545713in}}%
\pgfpathcurveto{\pgfqpoint{2.542979in}{2.537899in}}{\pgfqpoint{2.538589in}{2.527300in}}{\pgfqpoint{2.538589in}{2.516250in}}%
\pgfpathcurveto{\pgfqpoint{2.538589in}{2.505200in}}{\pgfqpoint{2.542979in}{2.494601in}}{\pgfqpoint{2.550793in}{2.486787in}}%
\pgfpathcurveto{\pgfqpoint{2.558607in}{2.478974in}}{\pgfqpoint{2.569206in}{2.474583in}}{\pgfqpoint{2.580256in}{2.474583in}}%
\pgfpathclose%
\pgfusepath{stroke,fill}%
\end{pgfscope}%
\begin{pgfscope}%
\pgfpathrectangle{\pgfqpoint{0.375000in}{0.330000in}}{\pgfqpoint{2.325000in}{2.310000in}}%
\pgfusepath{clip}%
\pgfsetbuttcap%
\pgfsetroundjoin%
\definecolor{currentfill}{rgb}{0.000000,0.000000,0.000000}%
\pgfsetfillcolor{currentfill}%
\pgfsetlinewidth{1.003750pt}%
\definecolor{currentstroke}{rgb}{0.000000,0.000000,0.000000}%
\pgfsetstrokecolor{currentstroke}%
\pgfsetdash{}{0pt}%
\pgfpathmoveto{\pgfqpoint{2.580256in}{1.438667in}}%
\pgfpathcurveto{\pgfqpoint{2.591306in}{1.438667in}}{\pgfqpoint{2.601905in}{1.443057in}}{\pgfqpoint{2.609718in}{1.450871in}}%
\pgfpathcurveto{\pgfqpoint{2.617532in}{1.458685in}}{\pgfqpoint{2.621922in}{1.469284in}}{\pgfqpoint{2.621922in}{1.480334in}}%
\pgfpathcurveto{\pgfqpoint{2.621922in}{1.491384in}}{\pgfqpoint{2.617532in}{1.501983in}}{\pgfqpoint{2.609718in}{1.509796in}}%
\pgfpathcurveto{\pgfqpoint{2.601905in}{1.517610in}}{\pgfqpoint{2.591306in}{1.522000in}}{\pgfqpoint{2.580256in}{1.522000in}}%
\pgfpathcurveto{\pgfqpoint{2.569206in}{1.522000in}}{\pgfqpoint{2.558607in}{1.517610in}}{\pgfqpoint{2.550793in}{1.509796in}}%
\pgfpathcurveto{\pgfqpoint{2.542979in}{1.501983in}}{\pgfqpoint{2.538589in}{1.491384in}}{\pgfqpoint{2.538589in}{1.480334in}}%
\pgfpathcurveto{\pgfqpoint{2.538589in}{1.469284in}}{\pgfqpoint{2.542979in}{1.458685in}}{\pgfqpoint{2.550793in}{1.450871in}}%
\pgfpathcurveto{\pgfqpoint{2.558607in}{1.443057in}}{\pgfqpoint{2.569206in}{1.438667in}}{\pgfqpoint{2.580256in}{1.438667in}}%
\pgfpathclose%
\pgfusepath{stroke,fill}%
\end{pgfscope}%
\begin{pgfscope}%
\pgfpathrectangle{\pgfqpoint{0.375000in}{0.330000in}}{\pgfqpoint{2.325000in}{2.310000in}}%
\pgfusepath{clip}%
\pgfsetbuttcap%
\pgfsetroundjoin%
\definecolor{currentfill}{rgb}{0.000000,0.000000,0.000000}%
\pgfsetfillcolor{currentfill}%
\pgfsetlinewidth{1.003750pt}%
\definecolor{currentstroke}{rgb}{0.000000,0.000000,0.000000}%
\pgfsetstrokecolor{currentstroke}%
\pgfsetdash{}{0pt}%
\pgfpathmoveto{\pgfqpoint{2.580256in}{2.474583in}}%
\pgfpathcurveto{\pgfqpoint{2.591306in}{2.474583in}}{\pgfqpoint{2.601905in}{2.478974in}}{\pgfqpoint{2.609718in}{2.486787in}}%
\pgfpathcurveto{\pgfqpoint{2.617532in}{2.494601in}}{\pgfqpoint{2.621922in}{2.505200in}}{\pgfqpoint{2.621922in}{2.516250in}}%
\pgfpathcurveto{\pgfqpoint{2.621922in}{2.527300in}}{\pgfqpoint{2.617532in}{2.537899in}}{\pgfqpoint{2.609718in}{2.545713in}}%
\pgfpathcurveto{\pgfqpoint{2.601905in}{2.553526in}}{\pgfqpoint{2.591306in}{2.557917in}}{\pgfqpoint{2.580256in}{2.557917in}}%
\pgfpathcurveto{\pgfqpoint{2.569206in}{2.557917in}}{\pgfqpoint{2.558607in}{2.553526in}}{\pgfqpoint{2.550793in}{2.545713in}}%
\pgfpathcurveto{\pgfqpoint{2.542979in}{2.537899in}}{\pgfqpoint{2.538589in}{2.527300in}}{\pgfqpoint{2.538589in}{2.516250in}}%
\pgfpathcurveto{\pgfqpoint{2.538589in}{2.505200in}}{\pgfqpoint{2.542979in}{2.494601in}}{\pgfqpoint{2.550793in}{2.486787in}}%
\pgfpathcurveto{\pgfqpoint{2.558607in}{2.478974in}}{\pgfqpoint{2.569206in}{2.474583in}}{\pgfqpoint{2.580256in}{2.474583in}}%
\pgfpathclose%
\pgfusepath{stroke,fill}%
\end{pgfscope}%
\begin{pgfscope}%
\pgfpathrectangle{\pgfqpoint{0.375000in}{0.330000in}}{\pgfqpoint{2.325000in}{2.310000in}}%
\pgfusepath{clip}%
\pgfsetbuttcap%
\pgfsetroundjoin%
\definecolor{currentfill}{rgb}{0.000000,0.000000,0.000000}%
\pgfsetfillcolor{currentfill}%
\pgfsetlinewidth{1.003750pt}%
\definecolor{currentstroke}{rgb}{0.000000,0.000000,0.000000}%
\pgfsetstrokecolor{currentstroke}%
\pgfsetdash{}{0pt}%
\pgfpathmoveto{\pgfqpoint{2.580256in}{2.474583in}}%
\pgfpathcurveto{\pgfqpoint{2.591306in}{2.474583in}}{\pgfqpoint{2.601905in}{2.478974in}}{\pgfqpoint{2.609718in}{2.486787in}}%
\pgfpathcurveto{\pgfqpoint{2.617532in}{2.494601in}}{\pgfqpoint{2.621922in}{2.505200in}}{\pgfqpoint{2.621922in}{2.516250in}}%
\pgfpathcurveto{\pgfqpoint{2.621922in}{2.527300in}}{\pgfqpoint{2.617532in}{2.537899in}}{\pgfqpoint{2.609718in}{2.545713in}}%
\pgfpathcurveto{\pgfqpoint{2.601905in}{2.553526in}}{\pgfqpoint{2.591306in}{2.557917in}}{\pgfqpoint{2.580256in}{2.557917in}}%
\pgfpathcurveto{\pgfqpoint{2.569206in}{2.557917in}}{\pgfqpoint{2.558607in}{2.553526in}}{\pgfqpoint{2.550793in}{2.545713in}}%
\pgfpathcurveto{\pgfqpoint{2.542979in}{2.537899in}}{\pgfqpoint{2.538589in}{2.527300in}}{\pgfqpoint{2.538589in}{2.516250in}}%
\pgfpathcurveto{\pgfqpoint{2.538589in}{2.505200in}}{\pgfqpoint{2.542979in}{2.494601in}}{\pgfqpoint{2.550793in}{2.486787in}}%
\pgfpathcurveto{\pgfqpoint{2.558607in}{2.478974in}}{\pgfqpoint{2.569206in}{2.474583in}}{\pgfqpoint{2.580256in}{2.474583in}}%
\pgfpathclose%
\pgfusepath{stroke,fill}%
\end{pgfscope}%
\begin{pgfscope}%
\pgfpathrectangle{\pgfqpoint{0.375000in}{0.330000in}}{\pgfqpoint{2.325000in}{2.310000in}}%
\pgfusepath{clip}%
\pgfsetbuttcap%
\pgfsetroundjoin%
\definecolor{currentfill}{rgb}{0.000000,0.000000,0.000000}%
\pgfsetfillcolor{currentfill}%
\pgfsetlinewidth{1.003750pt}%
\definecolor{currentstroke}{rgb}{0.000000,0.000000,0.000000}%
\pgfsetstrokecolor{currentstroke}%
\pgfsetdash{}{0pt}%
\pgfpathmoveto{\pgfqpoint{2.580256in}{1.438667in}}%
\pgfpathcurveto{\pgfqpoint{2.591306in}{1.438667in}}{\pgfqpoint{2.601905in}{1.443057in}}{\pgfqpoint{2.609718in}{1.450871in}}%
\pgfpathcurveto{\pgfqpoint{2.617532in}{1.458685in}}{\pgfqpoint{2.621922in}{1.469284in}}{\pgfqpoint{2.621922in}{1.480334in}}%
\pgfpathcurveto{\pgfqpoint{2.621922in}{1.491384in}}{\pgfqpoint{2.617532in}{1.501983in}}{\pgfqpoint{2.609718in}{1.509796in}}%
\pgfpathcurveto{\pgfqpoint{2.601905in}{1.517610in}}{\pgfqpoint{2.591306in}{1.522000in}}{\pgfqpoint{2.580256in}{1.522000in}}%
\pgfpathcurveto{\pgfqpoint{2.569206in}{1.522000in}}{\pgfqpoint{2.558607in}{1.517610in}}{\pgfqpoint{2.550793in}{1.509796in}}%
\pgfpathcurveto{\pgfqpoint{2.542979in}{1.501983in}}{\pgfqpoint{2.538589in}{1.491384in}}{\pgfqpoint{2.538589in}{1.480334in}}%
\pgfpathcurveto{\pgfqpoint{2.538589in}{1.469284in}}{\pgfqpoint{2.542979in}{1.458685in}}{\pgfqpoint{2.550793in}{1.450871in}}%
\pgfpathcurveto{\pgfqpoint{2.558607in}{1.443057in}}{\pgfqpoint{2.569206in}{1.438667in}}{\pgfqpoint{2.580256in}{1.438667in}}%
\pgfpathclose%
\pgfusepath{stroke,fill}%
\end{pgfscope}%
\begin{pgfscope}%
\pgfpathrectangle{\pgfqpoint{0.375000in}{0.330000in}}{\pgfqpoint{2.325000in}{2.310000in}}%
\pgfusepath{clip}%
\pgfsetbuttcap%
\pgfsetroundjoin%
\definecolor{currentfill}{rgb}{0.000000,0.000000,0.000000}%
\pgfsetfillcolor{currentfill}%
\pgfsetlinewidth{1.003750pt}%
\definecolor{currentstroke}{rgb}{0.000000,0.000000,0.000000}%
\pgfsetstrokecolor{currentstroke}%
\pgfsetdash{}{0pt}%
\pgfpathmoveto{\pgfqpoint{2.580256in}{2.474583in}}%
\pgfpathcurveto{\pgfqpoint{2.591306in}{2.474583in}}{\pgfqpoint{2.601905in}{2.478974in}}{\pgfqpoint{2.609718in}{2.486787in}}%
\pgfpathcurveto{\pgfqpoint{2.617532in}{2.494601in}}{\pgfqpoint{2.621922in}{2.505200in}}{\pgfqpoint{2.621922in}{2.516250in}}%
\pgfpathcurveto{\pgfqpoint{2.621922in}{2.527300in}}{\pgfqpoint{2.617532in}{2.537899in}}{\pgfqpoint{2.609718in}{2.545713in}}%
\pgfpathcurveto{\pgfqpoint{2.601905in}{2.553526in}}{\pgfqpoint{2.591306in}{2.557917in}}{\pgfqpoint{2.580256in}{2.557917in}}%
\pgfpathcurveto{\pgfqpoint{2.569206in}{2.557917in}}{\pgfqpoint{2.558607in}{2.553526in}}{\pgfqpoint{2.550793in}{2.545713in}}%
\pgfpathcurveto{\pgfqpoint{2.542979in}{2.537899in}}{\pgfqpoint{2.538589in}{2.527300in}}{\pgfqpoint{2.538589in}{2.516250in}}%
\pgfpathcurveto{\pgfqpoint{2.538589in}{2.505200in}}{\pgfqpoint{2.542979in}{2.494601in}}{\pgfqpoint{2.550793in}{2.486787in}}%
\pgfpathcurveto{\pgfqpoint{2.558607in}{2.478974in}}{\pgfqpoint{2.569206in}{2.474583in}}{\pgfqpoint{2.580256in}{2.474583in}}%
\pgfpathclose%
\pgfusepath{stroke,fill}%
\end{pgfscope}%
\begin{pgfscope}%
\pgfpathrectangle{\pgfqpoint{0.375000in}{0.330000in}}{\pgfqpoint{2.325000in}{2.310000in}}%
\pgfusepath{clip}%
\pgfsetbuttcap%
\pgfsetroundjoin%
\definecolor{currentfill}{rgb}{0.000000,0.000000,0.000000}%
\pgfsetfillcolor{currentfill}%
\pgfsetlinewidth{1.003750pt}%
\definecolor{currentstroke}{rgb}{0.000000,0.000000,0.000000}%
\pgfsetstrokecolor{currentstroke}%
\pgfsetdash{}{0pt}%
\pgfpathmoveto{\pgfqpoint{2.580256in}{2.474583in}}%
\pgfpathcurveto{\pgfqpoint{2.591306in}{2.474583in}}{\pgfqpoint{2.601905in}{2.478974in}}{\pgfqpoint{2.609718in}{2.486787in}}%
\pgfpathcurveto{\pgfqpoint{2.617532in}{2.494601in}}{\pgfqpoint{2.621922in}{2.505200in}}{\pgfqpoint{2.621922in}{2.516250in}}%
\pgfpathcurveto{\pgfqpoint{2.621922in}{2.527300in}}{\pgfqpoint{2.617532in}{2.537899in}}{\pgfqpoint{2.609718in}{2.545713in}}%
\pgfpathcurveto{\pgfqpoint{2.601905in}{2.553526in}}{\pgfqpoint{2.591306in}{2.557917in}}{\pgfqpoint{2.580256in}{2.557917in}}%
\pgfpathcurveto{\pgfqpoint{2.569206in}{2.557917in}}{\pgfqpoint{2.558607in}{2.553526in}}{\pgfqpoint{2.550793in}{2.545713in}}%
\pgfpathcurveto{\pgfqpoint{2.542979in}{2.537899in}}{\pgfqpoint{2.538589in}{2.527300in}}{\pgfqpoint{2.538589in}{2.516250in}}%
\pgfpathcurveto{\pgfqpoint{2.538589in}{2.505200in}}{\pgfqpoint{2.542979in}{2.494601in}}{\pgfqpoint{2.550793in}{2.486787in}}%
\pgfpathcurveto{\pgfqpoint{2.558607in}{2.478974in}}{\pgfqpoint{2.569206in}{2.474583in}}{\pgfqpoint{2.580256in}{2.474583in}}%
\pgfpathclose%
\pgfusepath{stroke,fill}%
\end{pgfscope}%
\begin{pgfscope}%
\pgfpathrectangle{\pgfqpoint{0.375000in}{0.330000in}}{\pgfqpoint{2.325000in}{2.310000in}}%
\pgfusepath{clip}%
\pgfsetbuttcap%
\pgfsetroundjoin%
\definecolor{currentfill}{rgb}{0.000000,0.000000,0.000000}%
\pgfsetfillcolor{currentfill}%
\pgfsetlinewidth{1.003750pt}%
\definecolor{currentstroke}{rgb}{0.000000,0.000000,0.000000}%
\pgfsetstrokecolor{currentstroke}%
\pgfsetdash{}{0pt}%
\pgfpathmoveto{\pgfqpoint{2.580256in}{2.474583in}}%
\pgfpathcurveto{\pgfqpoint{2.591306in}{2.474583in}}{\pgfqpoint{2.601905in}{2.478974in}}{\pgfqpoint{2.609718in}{2.486787in}}%
\pgfpathcurveto{\pgfqpoint{2.617532in}{2.494601in}}{\pgfqpoint{2.621922in}{2.505200in}}{\pgfqpoint{2.621922in}{2.516250in}}%
\pgfpathcurveto{\pgfqpoint{2.621922in}{2.527300in}}{\pgfqpoint{2.617532in}{2.537899in}}{\pgfqpoint{2.609718in}{2.545713in}}%
\pgfpathcurveto{\pgfqpoint{2.601905in}{2.553526in}}{\pgfqpoint{2.591306in}{2.557917in}}{\pgfqpoint{2.580256in}{2.557917in}}%
\pgfpathcurveto{\pgfqpoint{2.569206in}{2.557917in}}{\pgfqpoint{2.558607in}{2.553526in}}{\pgfqpoint{2.550793in}{2.545713in}}%
\pgfpathcurveto{\pgfqpoint{2.542979in}{2.537899in}}{\pgfqpoint{2.538589in}{2.527300in}}{\pgfqpoint{2.538589in}{2.516250in}}%
\pgfpathcurveto{\pgfqpoint{2.538589in}{2.505200in}}{\pgfqpoint{2.542979in}{2.494601in}}{\pgfqpoint{2.550793in}{2.486787in}}%
\pgfpathcurveto{\pgfqpoint{2.558607in}{2.478974in}}{\pgfqpoint{2.569206in}{2.474583in}}{\pgfqpoint{2.580256in}{2.474583in}}%
\pgfpathclose%
\pgfusepath{stroke,fill}%
\end{pgfscope}%
\begin{pgfscope}%
\pgfpathrectangle{\pgfqpoint{0.375000in}{0.330000in}}{\pgfqpoint{2.325000in}{2.310000in}}%
\pgfusepath{clip}%
\pgfsetbuttcap%
\pgfsetroundjoin%
\definecolor{currentfill}{rgb}{0.000000,0.000000,0.000000}%
\pgfsetfillcolor{currentfill}%
\pgfsetlinewidth{1.003750pt}%
\definecolor{currentstroke}{rgb}{0.000000,0.000000,0.000000}%
\pgfsetstrokecolor{currentstroke}%
\pgfsetdash{}{0pt}%
\pgfpathmoveto{\pgfqpoint{2.580256in}{2.474583in}}%
\pgfpathcurveto{\pgfqpoint{2.591306in}{2.474583in}}{\pgfqpoint{2.601905in}{2.478974in}}{\pgfqpoint{2.609718in}{2.486787in}}%
\pgfpathcurveto{\pgfqpoint{2.617532in}{2.494601in}}{\pgfqpoint{2.621922in}{2.505200in}}{\pgfqpoint{2.621922in}{2.516250in}}%
\pgfpathcurveto{\pgfqpoint{2.621922in}{2.527300in}}{\pgfqpoint{2.617532in}{2.537899in}}{\pgfqpoint{2.609718in}{2.545713in}}%
\pgfpathcurveto{\pgfqpoint{2.601905in}{2.553526in}}{\pgfqpoint{2.591306in}{2.557917in}}{\pgfqpoint{2.580256in}{2.557917in}}%
\pgfpathcurveto{\pgfqpoint{2.569206in}{2.557917in}}{\pgfqpoint{2.558607in}{2.553526in}}{\pgfqpoint{2.550793in}{2.545713in}}%
\pgfpathcurveto{\pgfqpoint{2.542979in}{2.537899in}}{\pgfqpoint{2.538589in}{2.527300in}}{\pgfqpoint{2.538589in}{2.516250in}}%
\pgfpathcurveto{\pgfqpoint{2.538589in}{2.505200in}}{\pgfqpoint{2.542979in}{2.494601in}}{\pgfqpoint{2.550793in}{2.486787in}}%
\pgfpathcurveto{\pgfqpoint{2.558607in}{2.478974in}}{\pgfqpoint{2.569206in}{2.474583in}}{\pgfqpoint{2.580256in}{2.474583in}}%
\pgfpathclose%
\pgfusepath{stroke,fill}%
\end{pgfscope}%
\begin{pgfscope}%
\pgfpathrectangle{\pgfqpoint{0.375000in}{0.330000in}}{\pgfqpoint{2.325000in}{2.310000in}}%
\pgfusepath{clip}%
\pgfsetbuttcap%
\pgfsetroundjoin%
\definecolor{currentfill}{rgb}{0.000000,0.000000,0.000000}%
\pgfsetfillcolor{currentfill}%
\pgfsetlinewidth{1.003750pt}%
\definecolor{currentstroke}{rgb}{0.000000,0.000000,0.000000}%
\pgfsetstrokecolor{currentstroke}%
\pgfsetdash{}{0pt}%
\pgfpathmoveto{\pgfqpoint{2.580256in}{2.474583in}}%
\pgfpathcurveto{\pgfqpoint{2.591306in}{2.474583in}}{\pgfqpoint{2.601905in}{2.478974in}}{\pgfqpoint{2.609718in}{2.486787in}}%
\pgfpathcurveto{\pgfqpoint{2.617532in}{2.494601in}}{\pgfqpoint{2.621922in}{2.505200in}}{\pgfqpoint{2.621922in}{2.516250in}}%
\pgfpathcurveto{\pgfqpoint{2.621922in}{2.527300in}}{\pgfqpoint{2.617532in}{2.537899in}}{\pgfqpoint{2.609718in}{2.545713in}}%
\pgfpathcurveto{\pgfqpoint{2.601905in}{2.553526in}}{\pgfqpoint{2.591306in}{2.557917in}}{\pgfqpoint{2.580256in}{2.557917in}}%
\pgfpathcurveto{\pgfqpoint{2.569206in}{2.557917in}}{\pgfqpoint{2.558607in}{2.553526in}}{\pgfqpoint{2.550793in}{2.545713in}}%
\pgfpathcurveto{\pgfqpoint{2.542979in}{2.537899in}}{\pgfqpoint{2.538589in}{2.527300in}}{\pgfqpoint{2.538589in}{2.516250in}}%
\pgfpathcurveto{\pgfqpoint{2.538589in}{2.505200in}}{\pgfqpoint{2.542979in}{2.494601in}}{\pgfqpoint{2.550793in}{2.486787in}}%
\pgfpathcurveto{\pgfqpoint{2.558607in}{2.478974in}}{\pgfqpoint{2.569206in}{2.474583in}}{\pgfqpoint{2.580256in}{2.474583in}}%
\pgfpathclose%
\pgfusepath{stroke,fill}%
\end{pgfscope}%
\begin{pgfscope}%
\pgfpathrectangle{\pgfqpoint{0.375000in}{0.330000in}}{\pgfqpoint{2.325000in}{2.310000in}}%
\pgfusepath{clip}%
\pgfsetbuttcap%
\pgfsetroundjoin%
\definecolor{currentfill}{rgb}{0.000000,0.000000,0.000000}%
\pgfsetfillcolor{currentfill}%
\pgfsetlinewidth{1.003750pt}%
\definecolor{currentstroke}{rgb}{0.000000,0.000000,0.000000}%
\pgfsetstrokecolor{currentstroke}%
\pgfsetdash{}{0pt}%
\pgfpathmoveto{\pgfqpoint{2.580256in}{2.474583in}}%
\pgfpathcurveto{\pgfqpoint{2.591306in}{2.474583in}}{\pgfqpoint{2.601905in}{2.478974in}}{\pgfqpoint{2.609718in}{2.486787in}}%
\pgfpathcurveto{\pgfqpoint{2.617532in}{2.494601in}}{\pgfqpoint{2.621922in}{2.505200in}}{\pgfqpoint{2.621922in}{2.516250in}}%
\pgfpathcurveto{\pgfqpoint{2.621922in}{2.527300in}}{\pgfqpoint{2.617532in}{2.537899in}}{\pgfqpoint{2.609718in}{2.545713in}}%
\pgfpathcurveto{\pgfqpoint{2.601905in}{2.553526in}}{\pgfqpoint{2.591306in}{2.557917in}}{\pgfqpoint{2.580256in}{2.557917in}}%
\pgfpathcurveto{\pgfqpoint{2.569206in}{2.557917in}}{\pgfqpoint{2.558607in}{2.553526in}}{\pgfqpoint{2.550793in}{2.545713in}}%
\pgfpathcurveto{\pgfqpoint{2.542979in}{2.537899in}}{\pgfqpoint{2.538589in}{2.527300in}}{\pgfqpoint{2.538589in}{2.516250in}}%
\pgfpathcurveto{\pgfqpoint{2.538589in}{2.505200in}}{\pgfqpoint{2.542979in}{2.494601in}}{\pgfqpoint{2.550793in}{2.486787in}}%
\pgfpathcurveto{\pgfqpoint{2.558607in}{2.478974in}}{\pgfqpoint{2.569206in}{2.474583in}}{\pgfqpoint{2.580256in}{2.474583in}}%
\pgfpathclose%
\pgfusepath{stroke,fill}%
\end{pgfscope}%
\begin{pgfscope}%
\pgfpathrectangle{\pgfqpoint{0.375000in}{0.330000in}}{\pgfqpoint{2.325000in}{2.310000in}}%
\pgfusepath{clip}%
\pgfsetbuttcap%
\pgfsetroundjoin%
\definecolor{currentfill}{rgb}{0.000000,0.000000,0.000000}%
\pgfsetfillcolor{currentfill}%
\pgfsetlinewidth{1.003750pt}%
\definecolor{currentstroke}{rgb}{0.000000,0.000000,0.000000}%
\pgfsetstrokecolor{currentstroke}%
\pgfsetdash{}{0pt}%
\pgfpathmoveto{\pgfqpoint{2.580256in}{2.474583in}}%
\pgfpathcurveto{\pgfqpoint{2.591306in}{2.474583in}}{\pgfqpoint{2.601905in}{2.478974in}}{\pgfqpoint{2.609718in}{2.486787in}}%
\pgfpathcurveto{\pgfqpoint{2.617532in}{2.494601in}}{\pgfqpoint{2.621922in}{2.505200in}}{\pgfqpoint{2.621922in}{2.516250in}}%
\pgfpathcurveto{\pgfqpoint{2.621922in}{2.527300in}}{\pgfqpoint{2.617532in}{2.537899in}}{\pgfqpoint{2.609718in}{2.545713in}}%
\pgfpathcurveto{\pgfqpoint{2.601905in}{2.553526in}}{\pgfqpoint{2.591306in}{2.557917in}}{\pgfqpoint{2.580256in}{2.557917in}}%
\pgfpathcurveto{\pgfqpoint{2.569206in}{2.557917in}}{\pgfqpoint{2.558607in}{2.553526in}}{\pgfqpoint{2.550793in}{2.545713in}}%
\pgfpathcurveto{\pgfqpoint{2.542979in}{2.537899in}}{\pgfqpoint{2.538589in}{2.527300in}}{\pgfqpoint{2.538589in}{2.516250in}}%
\pgfpathcurveto{\pgfqpoint{2.538589in}{2.505200in}}{\pgfqpoint{2.542979in}{2.494601in}}{\pgfqpoint{2.550793in}{2.486787in}}%
\pgfpathcurveto{\pgfqpoint{2.558607in}{2.478974in}}{\pgfqpoint{2.569206in}{2.474583in}}{\pgfqpoint{2.580256in}{2.474583in}}%
\pgfpathclose%
\pgfusepath{stroke,fill}%
\end{pgfscope}%
\begin{pgfscope}%
\pgfpathrectangle{\pgfqpoint{0.375000in}{0.330000in}}{\pgfqpoint{2.325000in}{2.310000in}}%
\pgfusepath{clip}%
\pgfsetbuttcap%
\pgfsetroundjoin%
\definecolor{currentfill}{rgb}{0.000000,0.000000,0.000000}%
\pgfsetfillcolor{currentfill}%
\pgfsetlinewidth{1.003750pt}%
\definecolor{currentstroke}{rgb}{0.000000,0.000000,0.000000}%
\pgfsetstrokecolor{currentstroke}%
\pgfsetdash{}{0pt}%
\pgfpathmoveto{\pgfqpoint{2.580256in}{2.474583in}}%
\pgfpathcurveto{\pgfqpoint{2.591306in}{2.474583in}}{\pgfqpoint{2.601905in}{2.478974in}}{\pgfqpoint{2.609718in}{2.486787in}}%
\pgfpathcurveto{\pgfqpoint{2.617532in}{2.494601in}}{\pgfqpoint{2.621922in}{2.505200in}}{\pgfqpoint{2.621922in}{2.516250in}}%
\pgfpathcurveto{\pgfqpoint{2.621922in}{2.527300in}}{\pgfqpoint{2.617532in}{2.537899in}}{\pgfqpoint{2.609718in}{2.545713in}}%
\pgfpathcurveto{\pgfqpoint{2.601905in}{2.553526in}}{\pgfqpoint{2.591306in}{2.557917in}}{\pgfqpoint{2.580256in}{2.557917in}}%
\pgfpathcurveto{\pgfqpoint{2.569206in}{2.557917in}}{\pgfqpoint{2.558607in}{2.553526in}}{\pgfqpoint{2.550793in}{2.545713in}}%
\pgfpathcurveto{\pgfqpoint{2.542979in}{2.537899in}}{\pgfqpoint{2.538589in}{2.527300in}}{\pgfqpoint{2.538589in}{2.516250in}}%
\pgfpathcurveto{\pgfqpoint{2.538589in}{2.505200in}}{\pgfqpoint{2.542979in}{2.494601in}}{\pgfqpoint{2.550793in}{2.486787in}}%
\pgfpathcurveto{\pgfqpoint{2.558607in}{2.478974in}}{\pgfqpoint{2.569206in}{2.474583in}}{\pgfqpoint{2.580256in}{2.474583in}}%
\pgfpathclose%
\pgfusepath{stroke,fill}%
\end{pgfscope}%
\begin{pgfscope}%
\pgfpathrectangle{\pgfqpoint{0.375000in}{0.330000in}}{\pgfqpoint{2.325000in}{2.310000in}}%
\pgfusepath{clip}%
\pgfsetbuttcap%
\pgfsetroundjoin%
\definecolor{currentfill}{rgb}{0.000000,0.000000,0.000000}%
\pgfsetfillcolor{currentfill}%
\pgfsetlinewidth{1.003750pt}%
\definecolor{currentstroke}{rgb}{0.000000,0.000000,0.000000}%
\pgfsetstrokecolor{currentstroke}%
\pgfsetdash{}{0pt}%
\pgfpathmoveto{\pgfqpoint{2.580256in}{2.474583in}}%
\pgfpathcurveto{\pgfqpoint{2.591306in}{2.474583in}}{\pgfqpoint{2.601905in}{2.478974in}}{\pgfqpoint{2.609718in}{2.486787in}}%
\pgfpathcurveto{\pgfqpoint{2.617532in}{2.494601in}}{\pgfqpoint{2.621922in}{2.505200in}}{\pgfqpoint{2.621922in}{2.516250in}}%
\pgfpathcurveto{\pgfqpoint{2.621922in}{2.527300in}}{\pgfqpoint{2.617532in}{2.537899in}}{\pgfqpoint{2.609718in}{2.545713in}}%
\pgfpathcurveto{\pgfqpoint{2.601905in}{2.553526in}}{\pgfqpoint{2.591306in}{2.557917in}}{\pgfqpoint{2.580256in}{2.557917in}}%
\pgfpathcurveto{\pgfqpoint{2.569206in}{2.557917in}}{\pgfqpoint{2.558607in}{2.553526in}}{\pgfqpoint{2.550793in}{2.545713in}}%
\pgfpathcurveto{\pgfqpoint{2.542979in}{2.537899in}}{\pgfqpoint{2.538589in}{2.527300in}}{\pgfqpoint{2.538589in}{2.516250in}}%
\pgfpathcurveto{\pgfqpoint{2.538589in}{2.505200in}}{\pgfqpoint{2.542979in}{2.494601in}}{\pgfqpoint{2.550793in}{2.486787in}}%
\pgfpathcurveto{\pgfqpoint{2.558607in}{2.478974in}}{\pgfqpoint{2.569206in}{2.474583in}}{\pgfqpoint{2.580256in}{2.474583in}}%
\pgfpathclose%
\pgfusepath{stroke,fill}%
\end{pgfscope}%
\begin{pgfscope}%
\pgfpathrectangle{\pgfqpoint{0.375000in}{0.330000in}}{\pgfqpoint{2.325000in}{2.310000in}}%
\pgfusepath{clip}%
\pgfsetbuttcap%
\pgfsetroundjoin%
\definecolor{currentfill}{rgb}{0.000000,0.000000,0.000000}%
\pgfsetfillcolor{currentfill}%
\pgfsetlinewidth{1.003750pt}%
\definecolor{currentstroke}{rgb}{0.000000,0.000000,0.000000}%
\pgfsetstrokecolor{currentstroke}%
\pgfsetdash{}{0pt}%
\pgfpathmoveto{\pgfqpoint{2.580256in}{2.474583in}}%
\pgfpathcurveto{\pgfqpoint{2.591306in}{2.474583in}}{\pgfqpoint{2.601905in}{2.478974in}}{\pgfqpoint{2.609718in}{2.486787in}}%
\pgfpathcurveto{\pgfqpoint{2.617532in}{2.494601in}}{\pgfqpoint{2.621922in}{2.505200in}}{\pgfqpoint{2.621922in}{2.516250in}}%
\pgfpathcurveto{\pgfqpoint{2.621922in}{2.527300in}}{\pgfqpoint{2.617532in}{2.537899in}}{\pgfqpoint{2.609718in}{2.545713in}}%
\pgfpathcurveto{\pgfqpoint{2.601905in}{2.553526in}}{\pgfqpoint{2.591306in}{2.557917in}}{\pgfqpoint{2.580256in}{2.557917in}}%
\pgfpathcurveto{\pgfqpoint{2.569206in}{2.557917in}}{\pgfqpoint{2.558607in}{2.553526in}}{\pgfqpoint{2.550793in}{2.545713in}}%
\pgfpathcurveto{\pgfqpoint{2.542979in}{2.537899in}}{\pgfqpoint{2.538589in}{2.527300in}}{\pgfqpoint{2.538589in}{2.516250in}}%
\pgfpathcurveto{\pgfqpoint{2.538589in}{2.505200in}}{\pgfqpoint{2.542979in}{2.494601in}}{\pgfqpoint{2.550793in}{2.486787in}}%
\pgfpathcurveto{\pgfqpoint{2.558607in}{2.478974in}}{\pgfqpoint{2.569206in}{2.474583in}}{\pgfqpoint{2.580256in}{2.474583in}}%
\pgfpathclose%
\pgfusepath{stroke,fill}%
\end{pgfscope}%
\begin{pgfscope}%
\pgfpathrectangle{\pgfqpoint{0.375000in}{0.330000in}}{\pgfqpoint{2.325000in}{2.310000in}}%
\pgfusepath{clip}%
\pgfsetbuttcap%
\pgfsetroundjoin%
\definecolor{currentfill}{rgb}{0.000000,0.000000,0.000000}%
\pgfsetfillcolor{currentfill}%
\pgfsetlinewidth{1.003750pt}%
\definecolor{currentstroke}{rgb}{0.000000,0.000000,0.000000}%
\pgfsetstrokecolor{currentstroke}%
\pgfsetdash{}{0pt}%
\pgfpathmoveto{\pgfqpoint{2.580256in}{2.474583in}}%
\pgfpathcurveto{\pgfqpoint{2.591306in}{2.474583in}}{\pgfqpoint{2.601905in}{2.478974in}}{\pgfqpoint{2.609718in}{2.486787in}}%
\pgfpathcurveto{\pgfqpoint{2.617532in}{2.494601in}}{\pgfqpoint{2.621922in}{2.505200in}}{\pgfqpoint{2.621922in}{2.516250in}}%
\pgfpathcurveto{\pgfqpoint{2.621922in}{2.527300in}}{\pgfqpoint{2.617532in}{2.537899in}}{\pgfqpoint{2.609718in}{2.545713in}}%
\pgfpathcurveto{\pgfqpoint{2.601905in}{2.553526in}}{\pgfqpoint{2.591306in}{2.557917in}}{\pgfqpoint{2.580256in}{2.557917in}}%
\pgfpathcurveto{\pgfqpoint{2.569206in}{2.557917in}}{\pgfqpoint{2.558607in}{2.553526in}}{\pgfqpoint{2.550793in}{2.545713in}}%
\pgfpathcurveto{\pgfqpoint{2.542979in}{2.537899in}}{\pgfqpoint{2.538589in}{2.527300in}}{\pgfqpoint{2.538589in}{2.516250in}}%
\pgfpathcurveto{\pgfqpoint{2.538589in}{2.505200in}}{\pgfqpoint{2.542979in}{2.494601in}}{\pgfqpoint{2.550793in}{2.486787in}}%
\pgfpathcurveto{\pgfqpoint{2.558607in}{2.478974in}}{\pgfqpoint{2.569206in}{2.474583in}}{\pgfqpoint{2.580256in}{2.474583in}}%
\pgfpathclose%
\pgfusepath{stroke,fill}%
\end{pgfscope}%
\begin{pgfscope}%
\pgfpathrectangle{\pgfqpoint{0.375000in}{0.330000in}}{\pgfqpoint{2.325000in}{2.310000in}}%
\pgfusepath{clip}%
\pgfsetbuttcap%
\pgfsetroundjoin%
\definecolor{currentfill}{rgb}{0.000000,0.000000,0.000000}%
\pgfsetfillcolor{currentfill}%
\pgfsetlinewidth{1.003750pt}%
\definecolor{currentstroke}{rgb}{0.000000,0.000000,0.000000}%
\pgfsetstrokecolor{currentstroke}%
\pgfsetdash{}{0pt}%
\pgfpathmoveto{\pgfqpoint{2.580256in}{2.474583in}}%
\pgfpathcurveto{\pgfqpoint{2.591306in}{2.474583in}}{\pgfqpoint{2.601905in}{2.478974in}}{\pgfqpoint{2.609718in}{2.486787in}}%
\pgfpathcurveto{\pgfqpoint{2.617532in}{2.494601in}}{\pgfqpoint{2.621922in}{2.505200in}}{\pgfqpoint{2.621922in}{2.516250in}}%
\pgfpathcurveto{\pgfqpoint{2.621922in}{2.527300in}}{\pgfqpoint{2.617532in}{2.537899in}}{\pgfqpoint{2.609718in}{2.545713in}}%
\pgfpathcurveto{\pgfqpoint{2.601905in}{2.553526in}}{\pgfqpoint{2.591306in}{2.557917in}}{\pgfqpoint{2.580256in}{2.557917in}}%
\pgfpathcurveto{\pgfqpoint{2.569206in}{2.557917in}}{\pgfqpoint{2.558607in}{2.553526in}}{\pgfqpoint{2.550793in}{2.545713in}}%
\pgfpathcurveto{\pgfqpoint{2.542979in}{2.537899in}}{\pgfqpoint{2.538589in}{2.527300in}}{\pgfqpoint{2.538589in}{2.516250in}}%
\pgfpathcurveto{\pgfqpoint{2.538589in}{2.505200in}}{\pgfqpoint{2.542979in}{2.494601in}}{\pgfqpoint{2.550793in}{2.486787in}}%
\pgfpathcurveto{\pgfqpoint{2.558607in}{2.478974in}}{\pgfqpoint{2.569206in}{2.474583in}}{\pgfqpoint{2.580256in}{2.474583in}}%
\pgfpathclose%
\pgfusepath{stroke,fill}%
\end{pgfscope}%
\begin{pgfscope}%
\pgfpathrectangle{\pgfqpoint{0.375000in}{0.330000in}}{\pgfqpoint{2.325000in}{2.310000in}}%
\pgfusepath{clip}%
\pgfsetbuttcap%
\pgfsetroundjoin%
\definecolor{currentfill}{rgb}{0.000000,0.000000,0.000000}%
\pgfsetfillcolor{currentfill}%
\pgfsetlinewidth{1.003750pt}%
\definecolor{currentstroke}{rgb}{0.000000,0.000000,0.000000}%
\pgfsetstrokecolor{currentstroke}%
\pgfsetdash{}{0pt}%
\pgfpathmoveto{\pgfqpoint{2.580256in}{2.474583in}}%
\pgfpathcurveto{\pgfqpoint{2.591306in}{2.474583in}}{\pgfqpoint{2.601905in}{2.478974in}}{\pgfqpoint{2.609718in}{2.486787in}}%
\pgfpathcurveto{\pgfqpoint{2.617532in}{2.494601in}}{\pgfqpoint{2.621922in}{2.505200in}}{\pgfqpoint{2.621922in}{2.516250in}}%
\pgfpathcurveto{\pgfqpoint{2.621922in}{2.527300in}}{\pgfqpoint{2.617532in}{2.537899in}}{\pgfqpoint{2.609718in}{2.545713in}}%
\pgfpathcurveto{\pgfqpoint{2.601905in}{2.553526in}}{\pgfqpoint{2.591306in}{2.557917in}}{\pgfqpoint{2.580256in}{2.557917in}}%
\pgfpathcurveto{\pgfqpoint{2.569206in}{2.557917in}}{\pgfqpoint{2.558607in}{2.553526in}}{\pgfqpoint{2.550793in}{2.545713in}}%
\pgfpathcurveto{\pgfqpoint{2.542979in}{2.537899in}}{\pgfqpoint{2.538589in}{2.527300in}}{\pgfqpoint{2.538589in}{2.516250in}}%
\pgfpathcurveto{\pgfqpoint{2.538589in}{2.505200in}}{\pgfqpoint{2.542979in}{2.494601in}}{\pgfqpoint{2.550793in}{2.486787in}}%
\pgfpathcurveto{\pgfqpoint{2.558607in}{2.478974in}}{\pgfqpoint{2.569206in}{2.474583in}}{\pgfqpoint{2.580256in}{2.474583in}}%
\pgfpathclose%
\pgfusepath{stroke,fill}%
\end{pgfscope}%
\begin{pgfscope}%
\pgfpathrectangle{\pgfqpoint{0.375000in}{0.330000in}}{\pgfqpoint{2.325000in}{2.310000in}}%
\pgfusepath{clip}%
\pgfsetbuttcap%
\pgfsetroundjoin%
\definecolor{currentfill}{rgb}{0.000000,0.000000,0.000000}%
\pgfsetfillcolor{currentfill}%
\pgfsetlinewidth{1.003750pt}%
\definecolor{currentstroke}{rgb}{0.000000,0.000000,0.000000}%
\pgfsetstrokecolor{currentstroke}%
\pgfsetdash{}{0pt}%
\pgfpathmoveto{\pgfqpoint{2.580256in}{2.474583in}}%
\pgfpathcurveto{\pgfqpoint{2.591306in}{2.474583in}}{\pgfqpoint{2.601905in}{2.478974in}}{\pgfqpoint{2.609718in}{2.486787in}}%
\pgfpathcurveto{\pgfqpoint{2.617532in}{2.494601in}}{\pgfqpoint{2.621922in}{2.505200in}}{\pgfqpoint{2.621922in}{2.516250in}}%
\pgfpathcurveto{\pgfqpoint{2.621922in}{2.527300in}}{\pgfqpoint{2.617532in}{2.537899in}}{\pgfqpoint{2.609718in}{2.545713in}}%
\pgfpathcurveto{\pgfqpoint{2.601905in}{2.553526in}}{\pgfqpoint{2.591306in}{2.557917in}}{\pgfqpoint{2.580256in}{2.557917in}}%
\pgfpathcurveto{\pgfqpoint{2.569206in}{2.557917in}}{\pgfqpoint{2.558607in}{2.553526in}}{\pgfqpoint{2.550793in}{2.545713in}}%
\pgfpathcurveto{\pgfqpoint{2.542979in}{2.537899in}}{\pgfqpoint{2.538589in}{2.527300in}}{\pgfqpoint{2.538589in}{2.516250in}}%
\pgfpathcurveto{\pgfqpoint{2.538589in}{2.505200in}}{\pgfqpoint{2.542979in}{2.494601in}}{\pgfqpoint{2.550793in}{2.486787in}}%
\pgfpathcurveto{\pgfqpoint{2.558607in}{2.478974in}}{\pgfqpoint{2.569206in}{2.474583in}}{\pgfqpoint{2.580256in}{2.474583in}}%
\pgfpathclose%
\pgfusepath{stroke,fill}%
\end{pgfscope}%
\begin{pgfscope}%
\pgfpathrectangle{\pgfqpoint{0.375000in}{0.330000in}}{\pgfqpoint{2.325000in}{2.310000in}}%
\pgfusepath{clip}%
\pgfsetbuttcap%
\pgfsetroundjoin%
\definecolor{currentfill}{rgb}{0.000000,0.000000,0.000000}%
\pgfsetfillcolor{currentfill}%
\pgfsetlinewidth{1.003750pt}%
\definecolor{currentstroke}{rgb}{0.000000,0.000000,0.000000}%
\pgfsetstrokecolor{currentstroke}%
\pgfsetdash{}{0pt}%
\pgfpathmoveto{\pgfqpoint{2.580256in}{2.474583in}}%
\pgfpathcurveto{\pgfqpoint{2.591306in}{2.474583in}}{\pgfqpoint{2.601905in}{2.478974in}}{\pgfqpoint{2.609718in}{2.486787in}}%
\pgfpathcurveto{\pgfqpoint{2.617532in}{2.494601in}}{\pgfqpoint{2.621922in}{2.505200in}}{\pgfqpoint{2.621922in}{2.516250in}}%
\pgfpathcurveto{\pgfqpoint{2.621922in}{2.527300in}}{\pgfqpoint{2.617532in}{2.537899in}}{\pgfqpoint{2.609718in}{2.545713in}}%
\pgfpathcurveto{\pgfqpoint{2.601905in}{2.553526in}}{\pgfqpoint{2.591306in}{2.557917in}}{\pgfqpoint{2.580256in}{2.557917in}}%
\pgfpathcurveto{\pgfqpoint{2.569206in}{2.557917in}}{\pgfqpoint{2.558607in}{2.553526in}}{\pgfqpoint{2.550793in}{2.545713in}}%
\pgfpathcurveto{\pgfqpoint{2.542979in}{2.537899in}}{\pgfqpoint{2.538589in}{2.527300in}}{\pgfqpoint{2.538589in}{2.516250in}}%
\pgfpathcurveto{\pgfqpoint{2.538589in}{2.505200in}}{\pgfqpoint{2.542979in}{2.494601in}}{\pgfqpoint{2.550793in}{2.486787in}}%
\pgfpathcurveto{\pgfqpoint{2.558607in}{2.478974in}}{\pgfqpoint{2.569206in}{2.474583in}}{\pgfqpoint{2.580256in}{2.474583in}}%
\pgfpathclose%
\pgfusepath{stroke,fill}%
\end{pgfscope}%
\begin{pgfscope}%
\pgfpathrectangle{\pgfqpoint{0.375000in}{0.330000in}}{\pgfqpoint{2.325000in}{2.310000in}}%
\pgfusepath{clip}%
\pgfsetbuttcap%
\pgfsetroundjoin%
\definecolor{currentfill}{rgb}{0.000000,0.000000,0.000000}%
\pgfsetfillcolor{currentfill}%
\pgfsetlinewidth{1.003750pt}%
\definecolor{currentstroke}{rgb}{0.000000,0.000000,0.000000}%
\pgfsetstrokecolor{currentstroke}%
\pgfsetdash{}{0pt}%
\pgfpathmoveto{\pgfqpoint{2.580256in}{2.474583in}}%
\pgfpathcurveto{\pgfqpoint{2.591306in}{2.474583in}}{\pgfqpoint{2.601905in}{2.478974in}}{\pgfqpoint{2.609718in}{2.486787in}}%
\pgfpathcurveto{\pgfqpoint{2.617532in}{2.494601in}}{\pgfqpoint{2.621922in}{2.505200in}}{\pgfqpoint{2.621922in}{2.516250in}}%
\pgfpathcurveto{\pgfqpoint{2.621922in}{2.527300in}}{\pgfqpoint{2.617532in}{2.537899in}}{\pgfqpoint{2.609718in}{2.545713in}}%
\pgfpathcurveto{\pgfqpoint{2.601905in}{2.553526in}}{\pgfqpoint{2.591306in}{2.557917in}}{\pgfqpoint{2.580256in}{2.557917in}}%
\pgfpathcurveto{\pgfqpoint{2.569206in}{2.557917in}}{\pgfqpoint{2.558607in}{2.553526in}}{\pgfqpoint{2.550793in}{2.545713in}}%
\pgfpathcurveto{\pgfqpoint{2.542979in}{2.537899in}}{\pgfqpoint{2.538589in}{2.527300in}}{\pgfqpoint{2.538589in}{2.516250in}}%
\pgfpathcurveto{\pgfqpoint{2.538589in}{2.505200in}}{\pgfqpoint{2.542979in}{2.494601in}}{\pgfqpoint{2.550793in}{2.486787in}}%
\pgfpathcurveto{\pgfqpoint{2.558607in}{2.478974in}}{\pgfqpoint{2.569206in}{2.474583in}}{\pgfqpoint{2.580256in}{2.474583in}}%
\pgfpathclose%
\pgfusepath{stroke,fill}%
\end{pgfscope}%
\begin{pgfscope}%
\pgfpathrectangle{\pgfqpoint{0.375000in}{0.330000in}}{\pgfqpoint{2.325000in}{2.310000in}}%
\pgfusepath{clip}%
\pgfsetbuttcap%
\pgfsetroundjoin%
\definecolor{currentfill}{rgb}{0.000000,0.000000,0.000000}%
\pgfsetfillcolor{currentfill}%
\pgfsetlinewidth{1.003750pt}%
\definecolor{currentstroke}{rgb}{0.000000,0.000000,0.000000}%
\pgfsetstrokecolor{currentstroke}%
\pgfsetdash{}{0pt}%
\pgfpathmoveto{\pgfqpoint{2.580256in}{2.474583in}}%
\pgfpathcurveto{\pgfqpoint{2.591306in}{2.474583in}}{\pgfqpoint{2.601905in}{2.478974in}}{\pgfqpoint{2.609718in}{2.486787in}}%
\pgfpathcurveto{\pgfqpoint{2.617532in}{2.494601in}}{\pgfqpoint{2.621922in}{2.505200in}}{\pgfqpoint{2.621922in}{2.516250in}}%
\pgfpathcurveto{\pgfqpoint{2.621922in}{2.527300in}}{\pgfqpoint{2.617532in}{2.537899in}}{\pgfqpoint{2.609718in}{2.545713in}}%
\pgfpathcurveto{\pgfqpoint{2.601905in}{2.553526in}}{\pgfqpoint{2.591306in}{2.557917in}}{\pgfqpoint{2.580256in}{2.557917in}}%
\pgfpathcurveto{\pgfqpoint{2.569206in}{2.557917in}}{\pgfqpoint{2.558607in}{2.553526in}}{\pgfqpoint{2.550793in}{2.545713in}}%
\pgfpathcurveto{\pgfqpoint{2.542979in}{2.537899in}}{\pgfqpoint{2.538589in}{2.527300in}}{\pgfqpoint{2.538589in}{2.516250in}}%
\pgfpathcurveto{\pgfqpoint{2.538589in}{2.505200in}}{\pgfqpoint{2.542979in}{2.494601in}}{\pgfqpoint{2.550793in}{2.486787in}}%
\pgfpathcurveto{\pgfqpoint{2.558607in}{2.478974in}}{\pgfqpoint{2.569206in}{2.474583in}}{\pgfqpoint{2.580256in}{2.474583in}}%
\pgfpathclose%
\pgfusepath{stroke,fill}%
\end{pgfscope}%
\begin{pgfscope}%
\pgfpathrectangle{\pgfqpoint{0.375000in}{0.330000in}}{\pgfqpoint{2.325000in}{2.310000in}}%
\pgfusepath{clip}%
\pgfsetbuttcap%
\pgfsetroundjoin%
\definecolor{currentfill}{rgb}{0.000000,0.000000,0.000000}%
\pgfsetfillcolor{currentfill}%
\pgfsetlinewidth{1.003750pt}%
\definecolor{currentstroke}{rgb}{0.000000,0.000000,0.000000}%
\pgfsetstrokecolor{currentstroke}%
\pgfsetdash{}{0pt}%
\pgfpathmoveto{\pgfqpoint{2.580256in}{2.474583in}}%
\pgfpathcurveto{\pgfqpoint{2.591306in}{2.474583in}}{\pgfqpoint{2.601905in}{2.478974in}}{\pgfqpoint{2.609718in}{2.486787in}}%
\pgfpathcurveto{\pgfqpoint{2.617532in}{2.494601in}}{\pgfqpoint{2.621922in}{2.505200in}}{\pgfqpoint{2.621922in}{2.516250in}}%
\pgfpathcurveto{\pgfqpoint{2.621922in}{2.527300in}}{\pgfqpoint{2.617532in}{2.537899in}}{\pgfqpoint{2.609718in}{2.545713in}}%
\pgfpathcurveto{\pgfqpoint{2.601905in}{2.553526in}}{\pgfqpoint{2.591306in}{2.557917in}}{\pgfqpoint{2.580256in}{2.557917in}}%
\pgfpathcurveto{\pgfqpoint{2.569206in}{2.557917in}}{\pgfqpoint{2.558607in}{2.553526in}}{\pgfqpoint{2.550793in}{2.545713in}}%
\pgfpathcurveto{\pgfqpoint{2.542979in}{2.537899in}}{\pgfqpoint{2.538589in}{2.527300in}}{\pgfqpoint{2.538589in}{2.516250in}}%
\pgfpathcurveto{\pgfqpoint{2.538589in}{2.505200in}}{\pgfqpoint{2.542979in}{2.494601in}}{\pgfqpoint{2.550793in}{2.486787in}}%
\pgfpathcurveto{\pgfqpoint{2.558607in}{2.478974in}}{\pgfqpoint{2.569206in}{2.474583in}}{\pgfqpoint{2.580256in}{2.474583in}}%
\pgfpathclose%
\pgfusepath{stroke,fill}%
\end{pgfscope}%
\begin{pgfscope}%
\pgfpathrectangle{\pgfqpoint{0.375000in}{0.330000in}}{\pgfqpoint{2.325000in}{2.310000in}}%
\pgfusepath{clip}%
\pgfsetbuttcap%
\pgfsetroundjoin%
\definecolor{currentfill}{rgb}{0.000000,0.000000,0.000000}%
\pgfsetfillcolor{currentfill}%
\pgfsetlinewidth{1.003750pt}%
\definecolor{currentstroke}{rgb}{0.000000,0.000000,0.000000}%
\pgfsetstrokecolor{currentstroke}%
\pgfsetdash{}{0pt}%
\pgfpathmoveto{\pgfqpoint{2.580256in}{2.474583in}}%
\pgfpathcurveto{\pgfqpoint{2.591306in}{2.474583in}}{\pgfqpoint{2.601905in}{2.478974in}}{\pgfqpoint{2.609718in}{2.486787in}}%
\pgfpathcurveto{\pgfqpoint{2.617532in}{2.494601in}}{\pgfqpoint{2.621922in}{2.505200in}}{\pgfqpoint{2.621922in}{2.516250in}}%
\pgfpathcurveto{\pgfqpoint{2.621922in}{2.527300in}}{\pgfqpoint{2.617532in}{2.537899in}}{\pgfqpoint{2.609718in}{2.545713in}}%
\pgfpathcurveto{\pgfqpoint{2.601905in}{2.553526in}}{\pgfqpoint{2.591306in}{2.557917in}}{\pgfqpoint{2.580256in}{2.557917in}}%
\pgfpathcurveto{\pgfqpoint{2.569206in}{2.557917in}}{\pgfqpoint{2.558607in}{2.553526in}}{\pgfqpoint{2.550793in}{2.545713in}}%
\pgfpathcurveto{\pgfqpoint{2.542979in}{2.537899in}}{\pgfqpoint{2.538589in}{2.527300in}}{\pgfqpoint{2.538589in}{2.516250in}}%
\pgfpathcurveto{\pgfqpoint{2.538589in}{2.505200in}}{\pgfqpoint{2.542979in}{2.494601in}}{\pgfqpoint{2.550793in}{2.486787in}}%
\pgfpathcurveto{\pgfqpoint{2.558607in}{2.478974in}}{\pgfqpoint{2.569206in}{2.474583in}}{\pgfqpoint{2.580256in}{2.474583in}}%
\pgfpathclose%
\pgfusepath{stroke,fill}%
\end{pgfscope}%
\begin{pgfscope}%
\pgfpathrectangle{\pgfqpoint{0.375000in}{0.330000in}}{\pgfqpoint{2.325000in}{2.310000in}}%
\pgfusepath{clip}%
\pgfsetbuttcap%
\pgfsetroundjoin%
\definecolor{currentfill}{rgb}{0.000000,0.000000,0.000000}%
\pgfsetfillcolor{currentfill}%
\pgfsetlinewidth{1.003750pt}%
\definecolor{currentstroke}{rgb}{0.000000,0.000000,0.000000}%
\pgfsetstrokecolor{currentstroke}%
\pgfsetdash{}{0pt}%
\pgfpathmoveto{\pgfqpoint{2.580256in}{2.474583in}}%
\pgfpathcurveto{\pgfqpoint{2.591306in}{2.474583in}}{\pgfqpoint{2.601905in}{2.478974in}}{\pgfqpoint{2.609718in}{2.486787in}}%
\pgfpathcurveto{\pgfqpoint{2.617532in}{2.494601in}}{\pgfqpoint{2.621922in}{2.505200in}}{\pgfqpoint{2.621922in}{2.516250in}}%
\pgfpathcurveto{\pgfqpoint{2.621922in}{2.527300in}}{\pgfqpoint{2.617532in}{2.537899in}}{\pgfqpoint{2.609718in}{2.545713in}}%
\pgfpathcurveto{\pgfqpoint{2.601905in}{2.553526in}}{\pgfqpoint{2.591306in}{2.557917in}}{\pgfqpoint{2.580256in}{2.557917in}}%
\pgfpathcurveto{\pgfqpoint{2.569206in}{2.557917in}}{\pgfqpoint{2.558607in}{2.553526in}}{\pgfqpoint{2.550793in}{2.545713in}}%
\pgfpathcurveto{\pgfqpoint{2.542979in}{2.537899in}}{\pgfqpoint{2.538589in}{2.527300in}}{\pgfqpoint{2.538589in}{2.516250in}}%
\pgfpathcurveto{\pgfqpoint{2.538589in}{2.505200in}}{\pgfqpoint{2.542979in}{2.494601in}}{\pgfqpoint{2.550793in}{2.486787in}}%
\pgfpathcurveto{\pgfqpoint{2.558607in}{2.478974in}}{\pgfqpoint{2.569206in}{2.474583in}}{\pgfqpoint{2.580256in}{2.474583in}}%
\pgfpathclose%
\pgfusepath{stroke,fill}%
\end{pgfscope}%
\begin{pgfscope}%
\pgfpathrectangle{\pgfqpoint{0.375000in}{0.330000in}}{\pgfqpoint{2.325000in}{2.310000in}}%
\pgfusepath{clip}%
\pgfsetbuttcap%
\pgfsetroundjoin%
\definecolor{currentfill}{rgb}{0.000000,0.000000,0.000000}%
\pgfsetfillcolor{currentfill}%
\pgfsetlinewidth{1.003750pt}%
\definecolor{currentstroke}{rgb}{0.000000,0.000000,0.000000}%
\pgfsetstrokecolor{currentstroke}%
\pgfsetdash{}{0pt}%
\pgfpathmoveto{\pgfqpoint{2.580256in}{2.474583in}}%
\pgfpathcurveto{\pgfqpoint{2.591306in}{2.474583in}}{\pgfqpoint{2.601905in}{2.478974in}}{\pgfqpoint{2.609718in}{2.486787in}}%
\pgfpathcurveto{\pgfqpoint{2.617532in}{2.494601in}}{\pgfqpoint{2.621922in}{2.505200in}}{\pgfqpoint{2.621922in}{2.516250in}}%
\pgfpathcurveto{\pgfqpoint{2.621922in}{2.527300in}}{\pgfqpoint{2.617532in}{2.537899in}}{\pgfqpoint{2.609718in}{2.545713in}}%
\pgfpathcurveto{\pgfqpoint{2.601905in}{2.553526in}}{\pgfqpoint{2.591306in}{2.557917in}}{\pgfqpoint{2.580256in}{2.557917in}}%
\pgfpathcurveto{\pgfqpoint{2.569206in}{2.557917in}}{\pgfqpoint{2.558607in}{2.553526in}}{\pgfqpoint{2.550793in}{2.545713in}}%
\pgfpathcurveto{\pgfqpoint{2.542979in}{2.537899in}}{\pgfqpoint{2.538589in}{2.527300in}}{\pgfqpoint{2.538589in}{2.516250in}}%
\pgfpathcurveto{\pgfqpoint{2.538589in}{2.505200in}}{\pgfqpoint{2.542979in}{2.494601in}}{\pgfqpoint{2.550793in}{2.486787in}}%
\pgfpathcurveto{\pgfqpoint{2.558607in}{2.478974in}}{\pgfqpoint{2.569206in}{2.474583in}}{\pgfqpoint{2.580256in}{2.474583in}}%
\pgfpathclose%
\pgfusepath{stroke,fill}%
\end{pgfscope}%
\begin{pgfscope}%
\pgfpathrectangle{\pgfqpoint{0.375000in}{0.330000in}}{\pgfqpoint{2.325000in}{2.310000in}}%
\pgfusepath{clip}%
\pgfsetbuttcap%
\pgfsetroundjoin%
\definecolor{currentfill}{rgb}{0.000000,0.000000,0.000000}%
\pgfsetfillcolor{currentfill}%
\pgfsetlinewidth{1.003750pt}%
\definecolor{currentstroke}{rgb}{0.000000,0.000000,0.000000}%
\pgfsetstrokecolor{currentstroke}%
\pgfsetdash{}{0pt}%
\pgfpathmoveto{\pgfqpoint{2.580256in}{2.474583in}}%
\pgfpathcurveto{\pgfqpoint{2.591306in}{2.474583in}}{\pgfqpoint{2.601905in}{2.478974in}}{\pgfqpoint{2.609718in}{2.486787in}}%
\pgfpathcurveto{\pgfqpoint{2.617532in}{2.494601in}}{\pgfqpoint{2.621922in}{2.505200in}}{\pgfqpoint{2.621922in}{2.516250in}}%
\pgfpathcurveto{\pgfqpoint{2.621922in}{2.527300in}}{\pgfqpoint{2.617532in}{2.537899in}}{\pgfqpoint{2.609718in}{2.545713in}}%
\pgfpathcurveto{\pgfqpoint{2.601905in}{2.553526in}}{\pgfqpoint{2.591306in}{2.557917in}}{\pgfqpoint{2.580256in}{2.557917in}}%
\pgfpathcurveto{\pgfqpoint{2.569206in}{2.557917in}}{\pgfqpoint{2.558607in}{2.553526in}}{\pgfqpoint{2.550793in}{2.545713in}}%
\pgfpathcurveto{\pgfqpoint{2.542979in}{2.537899in}}{\pgfqpoint{2.538589in}{2.527300in}}{\pgfqpoint{2.538589in}{2.516250in}}%
\pgfpathcurveto{\pgfqpoint{2.538589in}{2.505200in}}{\pgfqpoint{2.542979in}{2.494601in}}{\pgfqpoint{2.550793in}{2.486787in}}%
\pgfpathcurveto{\pgfqpoint{2.558607in}{2.478974in}}{\pgfqpoint{2.569206in}{2.474583in}}{\pgfqpoint{2.580256in}{2.474583in}}%
\pgfpathclose%
\pgfusepath{stroke,fill}%
\end{pgfscope}%
\begin{pgfscope}%
\pgfpathrectangle{\pgfqpoint{0.375000in}{0.330000in}}{\pgfqpoint{2.325000in}{2.310000in}}%
\pgfusepath{clip}%
\pgfsetbuttcap%
\pgfsetroundjoin%
\definecolor{currentfill}{rgb}{0.000000,0.000000,0.000000}%
\pgfsetfillcolor{currentfill}%
\pgfsetlinewidth{1.003750pt}%
\definecolor{currentstroke}{rgb}{0.000000,0.000000,0.000000}%
\pgfsetstrokecolor{currentstroke}%
\pgfsetdash{}{0pt}%
\pgfpathmoveto{\pgfqpoint{2.580256in}{2.474583in}}%
\pgfpathcurveto{\pgfqpoint{2.591306in}{2.474583in}}{\pgfqpoint{2.601905in}{2.478974in}}{\pgfqpoint{2.609718in}{2.486787in}}%
\pgfpathcurveto{\pgfqpoint{2.617532in}{2.494601in}}{\pgfqpoint{2.621922in}{2.505200in}}{\pgfqpoint{2.621922in}{2.516250in}}%
\pgfpathcurveto{\pgfqpoint{2.621922in}{2.527300in}}{\pgfqpoint{2.617532in}{2.537899in}}{\pgfqpoint{2.609718in}{2.545713in}}%
\pgfpathcurveto{\pgfqpoint{2.601905in}{2.553526in}}{\pgfqpoint{2.591306in}{2.557917in}}{\pgfqpoint{2.580256in}{2.557917in}}%
\pgfpathcurveto{\pgfqpoint{2.569206in}{2.557917in}}{\pgfqpoint{2.558607in}{2.553526in}}{\pgfqpoint{2.550793in}{2.545713in}}%
\pgfpathcurveto{\pgfqpoint{2.542979in}{2.537899in}}{\pgfqpoint{2.538589in}{2.527300in}}{\pgfqpoint{2.538589in}{2.516250in}}%
\pgfpathcurveto{\pgfqpoint{2.538589in}{2.505200in}}{\pgfqpoint{2.542979in}{2.494601in}}{\pgfqpoint{2.550793in}{2.486787in}}%
\pgfpathcurveto{\pgfqpoint{2.558607in}{2.478974in}}{\pgfqpoint{2.569206in}{2.474583in}}{\pgfqpoint{2.580256in}{2.474583in}}%
\pgfpathclose%
\pgfusepath{stroke,fill}%
\end{pgfscope}%
\begin{pgfscope}%
\pgfpathrectangle{\pgfqpoint{0.375000in}{0.330000in}}{\pgfqpoint{2.325000in}{2.310000in}}%
\pgfusepath{clip}%
\pgfsetbuttcap%
\pgfsetroundjoin%
\definecolor{currentfill}{rgb}{0.000000,0.000000,0.000000}%
\pgfsetfillcolor{currentfill}%
\pgfsetlinewidth{1.003750pt}%
\definecolor{currentstroke}{rgb}{0.000000,0.000000,0.000000}%
\pgfsetstrokecolor{currentstroke}%
\pgfsetdash{}{0pt}%
\pgfpathmoveto{\pgfqpoint{2.580256in}{2.474583in}}%
\pgfpathcurveto{\pgfqpoint{2.591306in}{2.474583in}}{\pgfqpoint{2.601905in}{2.478974in}}{\pgfqpoint{2.609718in}{2.486787in}}%
\pgfpathcurveto{\pgfqpoint{2.617532in}{2.494601in}}{\pgfqpoint{2.621922in}{2.505200in}}{\pgfqpoint{2.621922in}{2.516250in}}%
\pgfpathcurveto{\pgfqpoint{2.621922in}{2.527300in}}{\pgfqpoint{2.617532in}{2.537899in}}{\pgfqpoint{2.609718in}{2.545713in}}%
\pgfpathcurveto{\pgfqpoint{2.601905in}{2.553526in}}{\pgfqpoint{2.591306in}{2.557917in}}{\pgfqpoint{2.580256in}{2.557917in}}%
\pgfpathcurveto{\pgfqpoint{2.569206in}{2.557917in}}{\pgfqpoint{2.558607in}{2.553526in}}{\pgfqpoint{2.550793in}{2.545713in}}%
\pgfpathcurveto{\pgfqpoint{2.542979in}{2.537899in}}{\pgfqpoint{2.538589in}{2.527300in}}{\pgfqpoint{2.538589in}{2.516250in}}%
\pgfpathcurveto{\pgfqpoint{2.538589in}{2.505200in}}{\pgfqpoint{2.542979in}{2.494601in}}{\pgfqpoint{2.550793in}{2.486787in}}%
\pgfpathcurveto{\pgfqpoint{2.558607in}{2.478974in}}{\pgfqpoint{2.569206in}{2.474583in}}{\pgfqpoint{2.580256in}{2.474583in}}%
\pgfpathclose%
\pgfusepath{stroke,fill}%
\end{pgfscope}%
\begin{pgfscope}%
\pgfpathrectangle{\pgfqpoint{0.375000in}{0.330000in}}{\pgfqpoint{2.325000in}{2.310000in}}%
\pgfusepath{clip}%
\pgfsetbuttcap%
\pgfsetroundjoin%
\definecolor{currentfill}{rgb}{0.000000,0.000000,0.000000}%
\pgfsetfillcolor{currentfill}%
\pgfsetlinewidth{1.003750pt}%
\definecolor{currentstroke}{rgb}{0.000000,0.000000,0.000000}%
\pgfsetstrokecolor{currentstroke}%
\pgfsetdash{}{0pt}%
\pgfpathmoveto{\pgfqpoint{2.580256in}{2.474583in}}%
\pgfpathcurveto{\pgfqpoint{2.591306in}{2.474583in}}{\pgfqpoint{2.601905in}{2.478974in}}{\pgfqpoint{2.609718in}{2.486787in}}%
\pgfpathcurveto{\pgfqpoint{2.617532in}{2.494601in}}{\pgfqpoint{2.621922in}{2.505200in}}{\pgfqpoint{2.621922in}{2.516250in}}%
\pgfpathcurveto{\pgfqpoint{2.621922in}{2.527300in}}{\pgfqpoint{2.617532in}{2.537899in}}{\pgfqpoint{2.609718in}{2.545713in}}%
\pgfpathcurveto{\pgfqpoint{2.601905in}{2.553526in}}{\pgfqpoint{2.591306in}{2.557917in}}{\pgfqpoint{2.580256in}{2.557917in}}%
\pgfpathcurveto{\pgfqpoint{2.569206in}{2.557917in}}{\pgfqpoint{2.558607in}{2.553526in}}{\pgfqpoint{2.550793in}{2.545713in}}%
\pgfpathcurveto{\pgfqpoint{2.542979in}{2.537899in}}{\pgfqpoint{2.538589in}{2.527300in}}{\pgfqpoint{2.538589in}{2.516250in}}%
\pgfpathcurveto{\pgfqpoint{2.538589in}{2.505200in}}{\pgfqpoint{2.542979in}{2.494601in}}{\pgfqpoint{2.550793in}{2.486787in}}%
\pgfpathcurveto{\pgfqpoint{2.558607in}{2.478974in}}{\pgfqpoint{2.569206in}{2.474583in}}{\pgfqpoint{2.580256in}{2.474583in}}%
\pgfpathclose%
\pgfusepath{stroke,fill}%
\end{pgfscope}%
\begin{pgfscope}%
\pgfpathrectangle{\pgfqpoint{0.375000in}{0.330000in}}{\pgfqpoint{2.325000in}{2.310000in}}%
\pgfusepath{clip}%
\pgfsetbuttcap%
\pgfsetroundjoin%
\definecolor{currentfill}{rgb}{0.000000,0.000000,0.000000}%
\pgfsetfillcolor{currentfill}%
\pgfsetlinewidth{1.003750pt}%
\definecolor{currentstroke}{rgb}{0.000000,0.000000,0.000000}%
\pgfsetstrokecolor{currentstroke}%
\pgfsetdash{}{0pt}%
\pgfpathmoveto{\pgfqpoint{2.580256in}{2.474583in}}%
\pgfpathcurveto{\pgfqpoint{2.591306in}{2.474583in}}{\pgfqpoint{2.601905in}{2.478974in}}{\pgfqpoint{2.609718in}{2.486787in}}%
\pgfpathcurveto{\pgfqpoint{2.617532in}{2.494601in}}{\pgfqpoint{2.621922in}{2.505200in}}{\pgfqpoint{2.621922in}{2.516250in}}%
\pgfpathcurveto{\pgfqpoint{2.621922in}{2.527300in}}{\pgfqpoint{2.617532in}{2.537899in}}{\pgfqpoint{2.609718in}{2.545713in}}%
\pgfpathcurveto{\pgfqpoint{2.601905in}{2.553526in}}{\pgfqpoint{2.591306in}{2.557917in}}{\pgfqpoint{2.580256in}{2.557917in}}%
\pgfpathcurveto{\pgfqpoint{2.569206in}{2.557917in}}{\pgfqpoint{2.558607in}{2.553526in}}{\pgfqpoint{2.550793in}{2.545713in}}%
\pgfpathcurveto{\pgfqpoint{2.542979in}{2.537899in}}{\pgfqpoint{2.538589in}{2.527300in}}{\pgfqpoint{2.538589in}{2.516250in}}%
\pgfpathcurveto{\pgfqpoint{2.538589in}{2.505200in}}{\pgfqpoint{2.542979in}{2.494601in}}{\pgfqpoint{2.550793in}{2.486787in}}%
\pgfpathcurveto{\pgfqpoint{2.558607in}{2.478974in}}{\pgfqpoint{2.569206in}{2.474583in}}{\pgfqpoint{2.580256in}{2.474583in}}%
\pgfpathclose%
\pgfusepath{stroke,fill}%
\end{pgfscope}%
\begin{pgfscope}%
\pgfpathrectangle{\pgfqpoint{0.375000in}{0.330000in}}{\pgfqpoint{2.325000in}{2.310000in}}%
\pgfusepath{clip}%
\pgfsetbuttcap%
\pgfsetroundjoin%
\definecolor{currentfill}{rgb}{0.000000,0.000000,0.000000}%
\pgfsetfillcolor{currentfill}%
\pgfsetlinewidth{1.003750pt}%
\definecolor{currentstroke}{rgb}{0.000000,0.000000,0.000000}%
\pgfsetstrokecolor{currentstroke}%
\pgfsetdash{}{0pt}%
\pgfpathmoveto{\pgfqpoint{2.580256in}{2.474583in}}%
\pgfpathcurveto{\pgfqpoint{2.591306in}{2.474583in}}{\pgfqpoint{2.601905in}{2.478974in}}{\pgfqpoint{2.609718in}{2.486787in}}%
\pgfpathcurveto{\pgfqpoint{2.617532in}{2.494601in}}{\pgfqpoint{2.621922in}{2.505200in}}{\pgfqpoint{2.621922in}{2.516250in}}%
\pgfpathcurveto{\pgfqpoint{2.621922in}{2.527300in}}{\pgfqpoint{2.617532in}{2.537899in}}{\pgfqpoint{2.609718in}{2.545713in}}%
\pgfpathcurveto{\pgfqpoint{2.601905in}{2.553526in}}{\pgfqpoint{2.591306in}{2.557917in}}{\pgfqpoint{2.580256in}{2.557917in}}%
\pgfpathcurveto{\pgfqpoint{2.569206in}{2.557917in}}{\pgfqpoint{2.558607in}{2.553526in}}{\pgfqpoint{2.550793in}{2.545713in}}%
\pgfpathcurveto{\pgfqpoint{2.542979in}{2.537899in}}{\pgfqpoint{2.538589in}{2.527300in}}{\pgfqpoint{2.538589in}{2.516250in}}%
\pgfpathcurveto{\pgfqpoint{2.538589in}{2.505200in}}{\pgfqpoint{2.542979in}{2.494601in}}{\pgfqpoint{2.550793in}{2.486787in}}%
\pgfpathcurveto{\pgfqpoint{2.558607in}{2.478974in}}{\pgfqpoint{2.569206in}{2.474583in}}{\pgfqpoint{2.580256in}{2.474583in}}%
\pgfpathclose%
\pgfusepath{stroke,fill}%
\end{pgfscope}%
\begin{pgfscope}%
\pgfpathrectangle{\pgfqpoint{0.375000in}{0.330000in}}{\pgfqpoint{2.325000in}{2.310000in}}%
\pgfusepath{clip}%
\pgfsetbuttcap%
\pgfsetroundjoin%
\definecolor{currentfill}{rgb}{0.000000,0.000000,0.000000}%
\pgfsetfillcolor{currentfill}%
\pgfsetlinewidth{1.003750pt}%
\definecolor{currentstroke}{rgb}{0.000000,0.000000,0.000000}%
\pgfsetstrokecolor{currentstroke}%
\pgfsetdash{}{0pt}%
\pgfpathmoveto{\pgfqpoint{2.580256in}{2.474583in}}%
\pgfpathcurveto{\pgfqpoint{2.591306in}{2.474583in}}{\pgfqpoint{2.601905in}{2.478974in}}{\pgfqpoint{2.609718in}{2.486787in}}%
\pgfpathcurveto{\pgfqpoint{2.617532in}{2.494601in}}{\pgfqpoint{2.621922in}{2.505200in}}{\pgfqpoint{2.621922in}{2.516250in}}%
\pgfpathcurveto{\pgfqpoint{2.621922in}{2.527300in}}{\pgfqpoint{2.617532in}{2.537899in}}{\pgfqpoint{2.609718in}{2.545713in}}%
\pgfpathcurveto{\pgfqpoint{2.601905in}{2.553526in}}{\pgfqpoint{2.591306in}{2.557917in}}{\pgfqpoint{2.580256in}{2.557917in}}%
\pgfpathcurveto{\pgfqpoint{2.569206in}{2.557917in}}{\pgfqpoint{2.558607in}{2.553526in}}{\pgfqpoint{2.550793in}{2.545713in}}%
\pgfpathcurveto{\pgfqpoint{2.542979in}{2.537899in}}{\pgfqpoint{2.538589in}{2.527300in}}{\pgfqpoint{2.538589in}{2.516250in}}%
\pgfpathcurveto{\pgfqpoint{2.538589in}{2.505200in}}{\pgfqpoint{2.542979in}{2.494601in}}{\pgfqpoint{2.550793in}{2.486787in}}%
\pgfpathcurveto{\pgfqpoint{2.558607in}{2.478974in}}{\pgfqpoint{2.569206in}{2.474583in}}{\pgfqpoint{2.580256in}{2.474583in}}%
\pgfpathclose%
\pgfusepath{stroke,fill}%
\end{pgfscope}%
\begin{pgfscope}%
\pgfpathrectangle{\pgfqpoint{0.375000in}{0.330000in}}{\pgfqpoint{2.325000in}{2.310000in}}%
\pgfusepath{clip}%
\pgfsetbuttcap%
\pgfsetroundjoin%
\definecolor{currentfill}{rgb}{0.000000,0.000000,0.000000}%
\pgfsetfillcolor{currentfill}%
\pgfsetlinewidth{1.003750pt}%
\definecolor{currentstroke}{rgb}{0.000000,0.000000,0.000000}%
\pgfsetstrokecolor{currentstroke}%
\pgfsetdash{}{0pt}%
\pgfpathmoveto{\pgfqpoint{2.580256in}{2.474583in}}%
\pgfpathcurveto{\pgfqpoint{2.591306in}{2.474583in}}{\pgfqpoint{2.601905in}{2.478974in}}{\pgfqpoint{2.609718in}{2.486787in}}%
\pgfpathcurveto{\pgfqpoint{2.617532in}{2.494601in}}{\pgfqpoint{2.621922in}{2.505200in}}{\pgfqpoint{2.621922in}{2.516250in}}%
\pgfpathcurveto{\pgfqpoint{2.621922in}{2.527300in}}{\pgfqpoint{2.617532in}{2.537899in}}{\pgfqpoint{2.609718in}{2.545713in}}%
\pgfpathcurveto{\pgfqpoint{2.601905in}{2.553526in}}{\pgfqpoint{2.591306in}{2.557917in}}{\pgfqpoint{2.580256in}{2.557917in}}%
\pgfpathcurveto{\pgfqpoint{2.569206in}{2.557917in}}{\pgfqpoint{2.558607in}{2.553526in}}{\pgfqpoint{2.550793in}{2.545713in}}%
\pgfpathcurveto{\pgfqpoint{2.542979in}{2.537899in}}{\pgfqpoint{2.538589in}{2.527300in}}{\pgfqpoint{2.538589in}{2.516250in}}%
\pgfpathcurveto{\pgfqpoint{2.538589in}{2.505200in}}{\pgfqpoint{2.542979in}{2.494601in}}{\pgfqpoint{2.550793in}{2.486787in}}%
\pgfpathcurveto{\pgfqpoint{2.558607in}{2.478974in}}{\pgfqpoint{2.569206in}{2.474583in}}{\pgfqpoint{2.580256in}{2.474583in}}%
\pgfpathclose%
\pgfusepath{stroke,fill}%
\end{pgfscope}%
\begin{pgfscope}%
\pgfpathrectangle{\pgfqpoint{0.375000in}{0.330000in}}{\pgfqpoint{2.325000in}{2.310000in}}%
\pgfusepath{clip}%
\pgfsetbuttcap%
\pgfsetroundjoin%
\definecolor{currentfill}{rgb}{0.000000,0.000000,0.000000}%
\pgfsetfillcolor{currentfill}%
\pgfsetlinewidth{1.003750pt}%
\definecolor{currentstroke}{rgb}{0.000000,0.000000,0.000000}%
\pgfsetstrokecolor{currentstroke}%
\pgfsetdash{}{0pt}%
\pgfpathmoveto{\pgfqpoint{2.580256in}{2.474583in}}%
\pgfpathcurveto{\pgfqpoint{2.591306in}{2.474583in}}{\pgfqpoint{2.601905in}{2.478974in}}{\pgfqpoint{2.609718in}{2.486787in}}%
\pgfpathcurveto{\pgfqpoint{2.617532in}{2.494601in}}{\pgfqpoint{2.621922in}{2.505200in}}{\pgfqpoint{2.621922in}{2.516250in}}%
\pgfpathcurveto{\pgfqpoint{2.621922in}{2.527300in}}{\pgfqpoint{2.617532in}{2.537899in}}{\pgfqpoint{2.609718in}{2.545713in}}%
\pgfpathcurveto{\pgfqpoint{2.601905in}{2.553526in}}{\pgfqpoint{2.591306in}{2.557917in}}{\pgfqpoint{2.580256in}{2.557917in}}%
\pgfpathcurveto{\pgfqpoint{2.569206in}{2.557917in}}{\pgfqpoint{2.558607in}{2.553526in}}{\pgfqpoint{2.550793in}{2.545713in}}%
\pgfpathcurveto{\pgfqpoint{2.542979in}{2.537899in}}{\pgfqpoint{2.538589in}{2.527300in}}{\pgfqpoint{2.538589in}{2.516250in}}%
\pgfpathcurveto{\pgfqpoint{2.538589in}{2.505200in}}{\pgfqpoint{2.542979in}{2.494601in}}{\pgfqpoint{2.550793in}{2.486787in}}%
\pgfpathcurveto{\pgfqpoint{2.558607in}{2.478974in}}{\pgfqpoint{2.569206in}{2.474583in}}{\pgfqpoint{2.580256in}{2.474583in}}%
\pgfpathclose%
\pgfusepath{stroke,fill}%
\end{pgfscope}%
\begin{pgfscope}%
\pgfpathrectangle{\pgfqpoint{0.375000in}{0.330000in}}{\pgfqpoint{2.325000in}{2.310000in}}%
\pgfusepath{clip}%
\pgfsetbuttcap%
\pgfsetroundjoin%
\definecolor{currentfill}{rgb}{0.000000,0.000000,0.000000}%
\pgfsetfillcolor{currentfill}%
\pgfsetlinewidth{1.003750pt}%
\definecolor{currentstroke}{rgb}{0.000000,0.000000,0.000000}%
\pgfsetstrokecolor{currentstroke}%
\pgfsetdash{}{0pt}%
\pgfpathmoveto{\pgfqpoint{2.580256in}{2.474583in}}%
\pgfpathcurveto{\pgfqpoint{2.591306in}{2.474583in}}{\pgfqpoint{2.601905in}{2.478974in}}{\pgfqpoint{2.609718in}{2.486787in}}%
\pgfpathcurveto{\pgfqpoint{2.617532in}{2.494601in}}{\pgfqpoint{2.621922in}{2.505200in}}{\pgfqpoint{2.621922in}{2.516250in}}%
\pgfpathcurveto{\pgfqpoint{2.621922in}{2.527300in}}{\pgfqpoint{2.617532in}{2.537899in}}{\pgfqpoint{2.609718in}{2.545713in}}%
\pgfpathcurveto{\pgfqpoint{2.601905in}{2.553526in}}{\pgfqpoint{2.591306in}{2.557917in}}{\pgfqpoint{2.580256in}{2.557917in}}%
\pgfpathcurveto{\pgfqpoint{2.569206in}{2.557917in}}{\pgfqpoint{2.558607in}{2.553526in}}{\pgfqpoint{2.550793in}{2.545713in}}%
\pgfpathcurveto{\pgfqpoint{2.542979in}{2.537899in}}{\pgfqpoint{2.538589in}{2.527300in}}{\pgfqpoint{2.538589in}{2.516250in}}%
\pgfpathcurveto{\pgfqpoint{2.538589in}{2.505200in}}{\pgfqpoint{2.542979in}{2.494601in}}{\pgfqpoint{2.550793in}{2.486787in}}%
\pgfpathcurveto{\pgfqpoint{2.558607in}{2.478974in}}{\pgfqpoint{2.569206in}{2.474583in}}{\pgfqpoint{2.580256in}{2.474583in}}%
\pgfpathclose%
\pgfusepath{stroke,fill}%
\end{pgfscope}%
\begin{pgfscope}%
\pgfpathrectangle{\pgfqpoint{0.375000in}{0.330000in}}{\pgfqpoint{2.325000in}{2.310000in}}%
\pgfusepath{clip}%
\pgfsetbuttcap%
\pgfsetroundjoin%
\definecolor{currentfill}{rgb}{0.000000,0.000000,0.000000}%
\pgfsetfillcolor{currentfill}%
\pgfsetlinewidth{1.003750pt}%
\definecolor{currentstroke}{rgb}{0.000000,0.000000,0.000000}%
\pgfsetstrokecolor{currentstroke}%
\pgfsetdash{}{0pt}%
\pgfpathmoveto{\pgfqpoint{2.580256in}{1.438667in}}%
\pgfpathcurveto{\pgfqpoint{2.591306in}{1.438667in}}{\pgfqpoint{2.601905in}{1.443057in}}{\pgfqpoint{2.609718in}{1.450871in}}%
\pgfpathcurveto{\pgfqpoint{2.617532in}{1.458685in}}{\pgfqpoint{2.621922in}{1.469284in}}{\pgfqpoint{2.621922in}{1.480334in}}%
\pgfpathcurveto{\pgfqpoint{2.621922in}{1.491384in}}{\pgfqpoint{2.617532in}{1.501983in}}{\pgfqpoint{2.609718in}{1.509796in}}%
\pgfpathcurveto{\pgfqpoint{2.601905in}{1.517610in}}{\pgfqpoint{2.591306in}{1.522000in}}{\pgfqpoint{2.580256in}{1.522000in}}%
\pgfpathcurveto{\pgfqpoint{2.569206in}{1.522000in}}{\pgfqpoint{2.558607in}{1.517610in}}{\pgfqpoint{2.550793in}{1.509796in}}%
\pgfpathcurveto{\pgfqpoint{2.542979in}{1.501983in}}{\pgfqpoint{2.538589in}{1.491384in}}{\pgfqpoint{2.538589in}{1.480334in}}%
\pgfpathcurveto{\pgfqpoint{2.538589in}{1.469284in}}{\pgfqpoint{2.542979in}{1.458685in}}{\pgfqpoint{2.550793in}{1.450871in}}%
\pgfpathcurveto{\pgfqpoint{2.558607in}{1.443057in}}{\pgfqpoint{2.569206in}{1.438667in}}{\pgfqpoint{2.580256in}{1.438667in}}%
\pgfpathclose%
\pgfusepath{stroke,fill}%
\end{pgfscope}%
\begin{pgfscope}%
\pgfpathrectangle{\pgfqpoint{0.375000in}{0.330000in}}{\pgfqpoint{2.325000in}{2.310000in}}%
\pgfusepath{clip}%
\pgfsetbuttcap%
\pgfsetroundjoin%
\definecolor{currentfill}{rgb}{0.000000,0.000000,0.000000}%
\pgfsetfillcolor{currentfill}%
\pgfsetlinewidth{1.003750pt}%
\definecolor{currentstroke}{rgb}{0.000000,0.000000,0.000000}%
\pgfsetstrokecolor{currentstroke}%
\pgfsetdash{}{0pt}%
\pgfpathmoveto{\pgfqpoint{2.580256in}{2.474583in}}%
\pgfpathcurveto{\pgfqpoint{2.591306in}{2.474583in}}{\pgfqpoint{2.601905in}{2.478974in}}{\pgfqpoint{2.609718in}{2.486787in}}%
\pgfpathcurveto{\pgfqpoint{2.617532in}{2.494601in}}{\pgfqpoint{2.621922in}{2.505200in}}{\pgfqpoint{2.621922in}{2.516250in}}%
\pgfpathcurveto{\pgfqpoint{2.621922in}{2.527300in}}{\pgfqpoint{2.617532in}{2.537899in}}{\pgfqpoint{2.609718in}{2.545713in}}%
\pgfpathcurveto{\pgfqpoint{2.601905in}{2.553526in}}{\pgfqpoint{2.591306in}{2.557917in}}{\pgfqpoint{2.580256in}{2.557917in}}%
\pgfpathcurveto{\pgfqpoint{2.569206in}{2.557917in}}{\pgfqpoint{2.558607in}{2.553526in}}{\pgfqpoint{2.550793in}{2.545713in}}%
\pgfpathcurveto{\pgfqpoint{2.542979in}{2.537899in}}{\pgfqpoint{2.538589in}{2.527300in}}{\pgfqpoint{2.538589in}{2.516250in}}%
\pgfpathcurveto{\pgfqpoint{2.538589in}{2.505200in}}{\pgfqpoint{2.542979in}{2.494601in}}{\pgfqpoint{2.550793in}{2.486787in}}%
\pgfpathcurveto{\pgfqpoint{2.558607in}{2.478974in}}{\pgfqpoint{2.569206in}{2.474583in}}{\pgfqpoint{2.580256in}{2.474583in}}%
\pgfpathclose%
\pgfusepath{stroke,fill}%
\end{pgfscope}%
\begin{pgfscope}%
\pgfpathrectangle{\pgfqpoint{0.375000in}{0.330000in}}{\pgfqpoint{2.325000in}{2.310000in}}%
\pgfusepath{clip}%
\pgfsetbuttcap%
\pgfsetroundjoin%
\definecolor{currentfill}{rgb}{0.000000,0.000000,0.000000}%
\pgfsetfillcolor{currentfill}%
\pgfsetlinewidth{1.003750pt}%
\definecolor{currentstroke}{rgb}{0.000000,0.000000,0.000000}%
\pgfsetstrokecolor{currentstroke}%
\pgfsetdash{}{0pt}%
\pgfpathmoveto{\pgfqpoint{2.580256in}{1.438667in}}%
\pgfpathcurveto{\pgfqpoint{2.591306in}{1.438667in}}{\pgfqpoint{2.601905in}{1.443057in}}{\pgfqpoint{2.609718in}{1.450871in}}%
\pgfpathcurveto{\pgfqpoint{2.617532in}{1.458685in}}{\pgfqpoint{2.621922in}{1.469284in}}{\pgfqpoint{2.621922in}{1.480334in}}%
\pgfpathcurveto{\pgfqpoint{2.621922in}{1.491384in}}{\pgfqpoint{2.617532in}{1.501983in}}{\pgfqpoint{2.609718in}{1.509796in}}%
\pgfpathcurveto{\pgfqpoint{2.601905in}{1.517610in}}{\pgfqpoint{2.591306in}{1.522000in}}{\pgfqpoint{2.580256in}{1.522000in}}%
\pgfpathcurveto{\pgfqpoint{2.569206in}{1.522000in}}{\pgfqpoint{2.558607in}{1.517610in}}{\pgfqpoint{2.550793in}{1.509796in}}%
\pgfpathcurveto{\pgfqpoint{2.542979in}{1.501983in}}{\pgfqpoint{2.538589in}{1.491384in}}{\pgfqpoint{2.538589in}{1.480334in}}%
\pgfpathcurveto{\pgfqpoint{2.538589in}{1.469284in}}{\pgfqpoint{2.542979in}{1.458685in}}{\pgfqpoint{2.550793in}{1.450871in}}%
\pgfpathcurveto{\pgfqpoint{2.558607in}{1.443057in}}{\pgfqpoint{2.569206in}{1.438667in}}{\pgfqpoint{2.580256in}{1.438667in}}%
\pgfpathclose%
\pgfusepath{stroke,fill}%
\end{pgfscope}%
\begin{pgfscope}%
\pgfpathrectangle{\pgfqpoint{0.375000in}{0.330000in}}{\pgfqpoint{2.325000in}{2.310000in}}%
\pgfusepath{clip}%
\pgfsetbuttcap%
\pgfsetroundjoin%
\definecolor{currentfill}{rgb}{0.000000,0.000000,0.000000}%
\pgfsetfillcolor{currentfill}%
\pgfsetlinewidth{1.003750pt}%
\definecolor{currentstroke}{rgb}{0.000000,0.000000,0.000000}%
\pgfsetstrokecolor{currentstroke}%
\pgfsetdash{}{0pt}%
\pgfpathmoveto{\pgfqpoint{2.580256in}{2.474583in}}%
\pgfpathcurveto{\pgfqpoint{2.591306in}{2.474583in}}{\pgfqpoint{2.601905in}{2.478974in}}{\pgfqpoint{2.609718in}{2.486787in}}%
\pgfpathcurveto{\pgfqpoint{2.617532in}{2.494601in}}{\pgfqpoint{2.621922in}{2.505200in}}{\pgfqpoint{2.621922in}{2.516250in}}%
\pgfpathcurveto{\pgfqpoint{2.621922in}{2.527300in}}{\pgfqpoint{2.617532in}{2.537899in}}{\pgfqpoint{2.609718in}{2.545713in}}%
\pgfpathcurveto{\pgfqpoint{2.601905in}{2.553526in}}{\pgfqpoint{2.591306in}{2.557917in}}{\pgfqpoint{2.580256in}{2.557917in}}%
\pgfpathcurveto{\pgfqpoint{2.569206in}{2.557917in}}{\pgfqpoint{2.558607in}{2.553526in}}{\pgfqpoint{2.550793in}{2.545713in}}%
\pgfpathcurveto{\pgfqpoint{2.542979in}{2.537899in}}{\pgfqpoint{2.538589in}{2.527300in}}{\pgfqpoint{2.538589in}{2.516250in}}%
\pgfpathcurveto{\pgfqpoint{2.538589in}{2.505200in}}{\pgfqpoint{2.542979in}{2.494601in}}{\pgfqpoint{2.550793in}{2.486787in}}%
\pgfpathcurveto{\pgfqpoint{2.558607in}{2.478974in}}{\pgfqpoint{2.569206in}{2.474583in}}{\pgfqpoint{2.580256in}{2.474583in}}%
\pgfpathclose%
\pgfusepath{stroke,fill}%
\end{pgfscope}%
\begin{pgfscope}%
\pgfpathrectangle{\pgfqpoint{0.375000in}{0.330000in}}{\pgfqpoint{2.325000in}{2.310000in}}%
\pgfusepath{clip}%
\pgfsetbuttcap%
\pgfsetroundjoin%
\definecolor{currentfill}{rgb}{0.000000,0.000000,0.000000}%
\pgfsetfillcolor{currentfill}%
\pgfsetlinewidth{1.003750pt}%
\definecolor{currentstroke}{rgb}{0.000000,0.000000,0.000000}%
\pgfsetstrokecolor{currentstroke}%
\pgfsetdash{}{0pt}%
\pgfpathmoveto{\pgfqpoint{2.580256in}{2.474583in}}%
\pgfpathcurveto{\pgfqpoint{2.591306in}{2.474583in}}{\pgfqpoint{2.601905in}{2.478974in}}{\pgfqpoint{2.609718in}{2.486787in}}%
\pgfpathcurveto{\pgfqpoint{2.617532in}{2.494601in}}{\pgfqpoint{2.621922in}{2.505200in}}{\pgfqpoint{2.621922in}{2.516250in}}%
\pgfpathcurveto{\pgfqpoint{2.621922in}{2.527300in}}{\pgfqpoint{2.617532in}{2.537899in}}{\pgfqpoint{2.609718in}{2.545713in}}%
\pgfpathcurveto{\pgfqpoint{2.601905in}{2.553526in}}{\pgfqpoint{2.591306in}{2.557917in}}{\pgfqpoint{2.580256in}{2.557917in}}%
\pgfpathcurveto{\pgfqpoint{2.569206in}{2.557917in}}{\pgfqpoint{2.558607in}{2.553526in}}{\pgfqpoint{2.550793in}{2.545713in}}%
\pgfpathcurveto{\pgfqpoint{2.542979in}{2.537899in}}{\pgfqpoint{2.538589in}{2.527300in}}{\pgfqpoint{2.538589in}{2.516250in}}%
\pgfpathcurveto{\pgfqpoint{2.538589in}{2.505200in}}{\pgfqpoint{2.542979in}{2.494601in}}{\pgfqpoint{2.550793in}{2.486787in}}%
\pgfpathcurveto{\pgfqpoint{2.558607in}{2.478974in}}{\pgfqpoint{2.569206in}{2.474583in}}{\pgfqpoint{2.580256in}{2.474583in}}%
\pgfpathclose%
\pgfusepath{stroke,fill}%
\end{pgfscope}%
\begin{pgfscope}%
\pgfpathrectangle{\pgfqpoint{0.375000in}{0.330000in}}{\pgfqpoint{2.325000in}{2.310000in}}%
\pgfusepath{clip}%
\pgfsetbuttcap%
\pgfsetroundjoin%
\definecolor{currentfill}{rgb}{0.000000,0.000000,0.000000}%
\pgfsetfillcolor{currentfill}%
\pgfsetlinewidth{1.003750pt}%
\definecolor{currentstroke}{rgb}{0.000000,0.000000,0.000000}%
\pgfsetstrokecolor{currentstroke}%
\pgfsetdash{}{0pt}%
\pgfpathmoveto{\pgfqpoint{2.580256in}{2.474583in}}%
\pgfpathcurveto{\pgfqpoint{2.591306in}{2.474583in}}{\pgfqpoint{2.601905in}{2.478974in}}{\pgfqpoint{2.609718in}{2.486787in}}%
\pgfpathcurveto{\pgfqpoint{2.617532in}{2.494601in}}{\pgfqpoint{2.621922in}{2.505200in}}{\pgfqpoint{2.621922in}{2.516250in}}%
\pgfpathcurveto{\pgfqpoint{2.621922in}{2.527300in}}{\pgfqpoint{2.617532in}{2.537899in}}{\pgfqpoint{2.609718in}{2.545713in}}%
\pgfpathcurveto{\pgfqpoint{2.601905in}{2.553526in}}{\pgfqpoint{2.591306in}{2.557917in}}{\pgfqpoint{2.580256in}{2.557917in}}%
\pgfpathcurveto{\pgfqpoint{2.569206in}{2.557917in}}{\pgfqpoint{2.558607in}{2.553526in}}{\pgfqpoint{2.550793in}{2.545713in}}%
\pgfpathcurveto{\pgfqpoint{2.542979in}{2.537899in}}{\pgfqpoint{2.538589in}{2.527300in}}{\pgfqpoint{2.538589in}{2.516250in}}%
\pgfpathcurveto{\pgfqpoint{2.538589in}{2.505200in}}{\pgfqpoint{2.542979in}{2.494601in}}{\pgfqpoint{2.550793in}{2.486787in}}%
\pgfpathcurveto{\pgfqpoint{2.558607in}{2.478974in}}{\pgfqpoint{2.569206in}{2.474583in}}{\pgfqpoint{2.580256in}{2.474583in}}%
\pgfpathclose%
\pgfusepath{stroke,fill}%
\end{pgfscope}%
\begin{pgfscope}%
\pgfpathrectangle{\pgfqpoint{0.375000in}{0.330000in}}{\pgfqpoint{2.325000in}{2.310000in}}%
\pgfusepath{clip}%
\pgfsetbuttcap%
\pgfsetroundjoin%
\definecolor{currentfill}{rgb}{0.000000,0.000000,0.000000}%
\pgfsetfillcolor{currentfill}%
\pgfsetlinewidth{1.003750pt}%
\definecolor{currentstroke}{rgb}{0.000000,0.000000,0.000000}%
\pgfsetstrokecolor{currentstroke}%
\pgfsetdash{}{0pt}%
\pgfpathmoveto{\pgfqpoint{2.580256in}{2.474583in}}%
\pgfpathcurveto{\pgfqpoint{2.591306in}{2.474583in}}{\pgfqpoint{2.601905in}{2.478974in}}{\pgfqpoint{2.609718in}{2.486787in}}%
\pgfpathcurveto{\pgfqpoint{2.617532in}{2.494601in}}{\pgfqpoint{2.621922in}{2.505200in}}{\pgfqpoint{2.621922in}{2.516250in}}%
\pgfpathcurveto{\pgfqpoint{2.621922in}{2.527300in}}{\pgfqpoint{2.617532in}{2.537899in}}{\pgfqpoint{2.609718in}{2.545713in}}%
\pgfpathcurveto{\pgfqpoint{2.601905in}{2.553526in}}{\pgfqpoint{2.591306in}{2.557917in}}{\pgfqpoint{2.580256in}{2.557917in}}%
\pgfpathcurveto{\pgfqpoint{2.569206in}{2.557917in}}{\pgfqpoint{2.558607in}{2.553526in}}{\pgfqpoint{2.550793in}{2.545713in}}%
\pgfpathcurveto{\pgfqpoint{2.542979in}{2.537899in}}{\pgfqpoint{2.538589in}{2.527300in}}{\pgfqpoint{2.538589in}{2.516250in}}%
\pgfpathcurveto{\pgfqpoint{2.538589in}{2.505200in}}{\pgfqpoint{2.542979in}{2.494601in}}{\pgfqpoint{2.550793in}{2.486787in}}%
\pgfpathcurveto{\pgfqpoint{2.558607in}{2.478974in}}{\pgfqpoint{2.569206in}{2.474583in}}{\pgfqpoint{2.580256in}{2.474583in}}%
\pgfpathclose%
\pgfusepath{stroke,fill}%
\end{pgfscope}%
\begin{pgfscope}%
\pgfpathrectangle{\pgfqpoint{0.375000in}{0.330000in}}{\pgfqpoint{2.325000in}{2.310000in}}%
\pgfusepath{clip}%
\pgfsetbuttcap%
\pgfsetroundjoin%
\definecolor{currentfill}{rgb}{0.000000,0.000000,0.000000}%
\pgfsetfillcolor{currentfill}%
\pgfsetlinewidth{1.003750pt}%
\definecolor{currentstroke}{rgb}{0.000000,0.000000,0.000000}%
\pgfsetstrokecolor{currentstroke}%
\pgfsetdash{}{0pt}%
\pgfpathmoveto{\pgfqpoint{2.580256in}{2.474583in}}%
\pgfpathcurveto{\pgfqpoint{2.591306in}{2.474583in}}{\pgfqpoint{2.601905in}{2.478974in}}{\pgfqpoint{2.609718in}{2.486787in}}%
\pgfpathcurveto{\pgfqpoint{2.617532in}{2.494601in}}{\pgfqpoint{2.621922in}{2.505200in}}{\pgfqpoint{2.621922in}{2.516250in}}%
\pgfpathcurveto{\pgfqpoint{2.621922in}{2.527300in}}{\pgfqpoint{2.617532in}{2.537899in}}{\pgfqpoint{2.609718in}{2.545713in}}%
\pgfpathcurveto{\pgfqpoint{2.601905in}{2.553526in}}{\pgfqpoint{2.591306in}{2.557917in}}{\pgfqpoint{2.580256in}{2.557917in}}%
\pgfpathcurveto{\pgfqpoint{2.569206in}{2.557917in}}{\pgfqpoint{2.558607in}{2.553526in}}{\pgfqpoint{2.550793in}{2.545713in}}%
\pgfpathcurveto{\pgfqpoint{2.542979in}{2.537899in}}{\pgfqpoint{2.538589in}{2.527300in}}{\pgfqpoint{2.538589in}{2.516250in}}%
\pgfpathcurveto{\pgfqpoint{2.538589in}{2.505200in}}{\pgfqpoint{2.542979in}{2.494601in}}{\pgfqpoint{2.550793in}{2.486787in}}%
\pgfpathcurveto{\pgfqpoint{2.558607in}{2.478974in}}{\pgfqpoint{2.569206in}{2.474583in}}{\pgfqpoint{2.580256in}{2.474583in}}%
\pgfpathclose%
\pgfusepath{stroke,fill}%
\end{pgfscope}%
\begin{pgfscope}%
\pgfpathrectangle{\pgfqpoint{0.375000in}{0.330000in}}{\pgfqpoint{2.325000in}{2.310000in}}%
\pgfusepath{clip}%
\pgfsetbuttcap%
\pgfsetroundjoin%
\definecolor{currentfill}{rgb}{0.000000,0.000000,0.000000}%
\pgfsetfillcolor{currentfill}%
\pgfsetlinewidth{1.003750pt}%
\definecolor{currentstroke}{rgb}{0.000000,0.000000,0.000000}%
\pgfsetstrokecolor{currentstroke}%
\pgfsetdash{}{0pt}%
\pgfpathmoveto{\pgfqpoint{2.580256in}{2.474583in}}%
\pgfpathcurveto{\pgfqpoint{2.591306in}{2.474583in}}{\pgfqpoint{2.601905in}{2.478974in}}{\pgfqpoint{2.609718in}{2.486787in}}%
\pgfpathcurveto{\pgfqpoint{2.617532in}{2.494601in}}{\pgfqpoint{2.621922in}{2.505200in}}{\pgfqpoint{2.621922in}{2.516250in}}%
\pgfpathcurveto{\pgfqpoint{2.621922in}{2.527300in}}{\pgfqpoint{2.617532in}{2.537899in}}{\pgfqpoint{2.609718in}{2.545713in}}%
\pgfpathcurveto{\pgfqpoint{2.601905in}{2.553526in}}{\pgfqpoint{2.591306in}{2.557917in}}{\pgfqpoint{2.580256in}{2.557917in}}%
\pgfpathcurveto{\pgfqpoint{2.569206in}{2.557917in}}{\pgfqpoint{2.558607in}{2.553526in}}{\pgfqpoint{2.550793in}{2.545713in}}%
\pgfpathcurveto{\pgfqpoint{2.542979in}{2.537899in}}{\pgfqpoint{2.538589in}{2.527300in}}{\pgfqpoint{2.538589in}{2.516250in}}%
\pgfpathcurveto{\pgfqpoint{2.538589in}{2.505200in}}{\pgfqpoint{2.542979in}{2.494601in}}{\pgfqpoint{2.550793in}{2.486787in}}%
\pgfpathcurveto{\pgfqpoint{2.558607in}{2.478974in}}{\pgfqpoint{2.569206in}{2.474583in}}{\pgfqpoint{2.580256in}{2.474583in}}%
\pgfpathclose%
\pgfusepath{stroke,fill}%
\end{pgfscope}%
\begin{pgfscope}%
\pgfpathrectangle{\pgfqpoint{0.375000in}{0.330000in}}{\pgfqpoint{2.325000in}{2.310000in}}%
\pgfusepath{clip}%
\pgfsetbuttcap%
\pgfsetroundjoin%
\definecolor{currentfill}{rgb}{0.000000,0.000000,0.000000}%
\pgfsetfillcolor{currentfill}%
\pgfsetlinewidth{1.003750pt}%
\definecolor{currentstroke}{rgb}{0.000000,0.000000,0.000000}%
\pgfsetstrokecolor{currentstroke}%
\pgfsetdash{}{0pt}%
\pgfpathmoveto{\pgfqpoint{2.580256in}{2.474583in}}%
\pgfpathcurveto{\pgfqpoint{2.591306in}{2.474583in}}{\pgfqpoint{2.601905in}{2.478974in}}{\pgfqpoint{2.609718in}{2.486787in}}%
\pgfpathcurveto{\pgfqpoint{2.617532in}{2.494601in}}{\pgfqpoint{2.621922in}{2.505200in}}{\pgfqpoint{2.621922in}{2.516250in}}%
\pgfpathcurveto{\pgfqpoint{2.621922in}{2.527300in}}{\pgfqpoint{2.617532in}{2.537899in}}{\pgfqpoint{2.609718in}{2.545713in}}%
\pgfpathcurveto{\pgfqpoint{2.601905in}{2.553526in}}{\pgfqpoint{2.591306in}{2.557917in}}{\pgfqpoint{2.580256in}{2.557917in}}%
\pgfpathcurveto{\pgfqpoint{2.569206in}{2.557917in}}{\pgfqpoint{2.558607in}{2.553526in}}{\pgfqpoint{2.550793in}{2.545713in}}%
\pgfpathcurveto{\pgfqpoint{2.542979in}{2.537899in}}{\pgfqpoint{2.538589in}{2.527300in}}{\pgfqpoint{2.538589in}{2.516250in}}%
\pgfpathcurveto{\pgfqpoint{2.538589in}{2.505200in}}{\pgfqpoint{2.542979in}{2.494601in}}{\pgfqpoint{2.550793in}{2.486787in}}%
\pgfpathcurveto{\pgfqpoint{2.558607in}{2.478974in}}{\pgfqpoint{2.569206in}{2.474583in}}{\pgfqpoint{2.580256in}{2.474583in}}%
\pgfpathclose%
\pgfusepath{stroke,fill}%
\end{pgfscope}%
\begin{pgfscope}%
\pgfpathrectangle{\pgfqpoint{0.375000in}{0.330000in}}{\pgfqpoint{2.325000in}{2.310000in}}%
\pgfusepath{clip}%
\pgfsetbuttcap%
\pgfsetroundjoin%
\definecolor{currentfill}{rgb}{0.000000,0.000000,0.000000}%
\pgfsetfillcolor{currentfill}%
\pgfsetlinewidth{1.003750pt}%
\definecolor{currentstroke}{rgb}{0.000000,0.000000,0.000000}%
\pgfsetstrokecolor{currentstroke}%
\pgfsetdash{}{0pt}%
\pgfpathmoveto{\pgfqpoint{2.580256in}{1.438667in}}%
\pgfpathcurveto{\pgfqpoint{2.591306in}{1.438667in}}{\pgfqpoint{2.601905in}{1.443057in}}{\pgfqpoint{2.609718in}{1.450871in}}%
\pgfpathcurveto{\pgfqpoint{2.617532in}{1.458685in}}{\pgfqpoint{2.621922in}{1.469284in}}{\pgfqpoint{2.621922in}{1.480334in}}%
\pgfpathcurveto{\pgfqpoint{2.621922in}{1.491384in}}{\pgfqpoint{2.617532in}{1.501983in}}{\pgfqpoint{2.609718in}{1.509796in}}%
\pgfpathcurveto{\pgfqpoint{2.601905in}{1.517610in}}{\pgfqpoint{2.591306in}{1.522000in}}{\pgfqpoint{2.580256in}{1.522000in}}%
\pgfpathcurveto{\pgfqpoint{2.569206in}{1.522000in}}{\pgfqpoint{2.558607in}{1.517610in}}{\pgfqpoint{2.550793in}{1.509796in}}%
\pgfpathcurveto{\pgfqpoint{2.542979in}{1.501983in}}{\pgfqpoint{2.538589in}{1.491384in}}{\pgfqpoint{2.538589in}{1.480334in}}%
\pgfpathcurveto{\pgfqpoint{2.538589in}{1.469284in}}{\pgfqpoint{2.542979in}{1.458685in}}{\pgfqpoint{2.550793in}{1.450871in}}%
\pgfpathcurveto{\pgfqpoint{2.558607in}{1.443057in}}{\pgfqpoint{2.569206in}{1.438667in}}{\pgfqpoint{2.580256in}{1.438667in}}%
\pgfpathclose%
\pgfusepath{stroke,fill}%
\end{pgfscope}%
\begin{pgfscope}%
\pgfpathrectangle{\pgfqpoint{0.375000in}{0.330000in}}{\pgfqpoint{2.325000in}{2.310000in}}%
\pgfusepath{clip}%
\pgfsetbuttcap%
\pgfsetroundjoin%
\definecolor{currentfill}{rgb}{0.000000,0.000000,0.000000}%
\pgfsetfillcolor{currentfill}%
\pgfsetlinewidth{1.003750pt}%
\definecolor{currentstroke}{rgb}{0.000000,0.000000,0.000000}%
\pgfsetstrokecolor{currentstroke}%
\pgfsetdash{}{0pt}%
\pgfpathmoveto{\pgfqpoint{2.580256in}{2.474583in}}%
\pgfpathcurveto{\pgfqpoint{2.591306in}{2.474583in}}{\pgfqpoint{2.601905in}{2.478974in}}{\pgfqpoint{2.609718in}{2.486787in}}%
\pgfpathcurveto{\pgfqpoint{2.617532in}{2.494601in}}{\pgfqpoint{2.621922in}{2.505200in}}{\pgfqpoint{2.621922in}{2.516250in}}%
\pgfpathcurveto{\pgfqpoint{2.621922in}{2.527300in}}{\pgfqpoint{2.617532in}{2.537899in}}{\pgfqpoint{2.609718in}{2.545713in}}%
\pgfpathcurveto{\pgfqpoint{2.601905in}{2.553526in}}{\pgfqpoint{2.591306in}{2.557917in}}{\pgfqpoint{2.580256in}{2.557917in}}%
\pgfpathcurveto{\pgfqpoint{2.569206in}{2.557917in}}{\pgfqpoint{2.558607in}{2.553526in}}{\pgfqpoint{2.550793in}{2.545713in}}%
\pgfpathcurveto{\pgfqpoint{2.542979in}{2.537899in}}{\pgfqpoint{2.538589in}{2.527300in}}{\pgfqpoint{2.538589in}{2.516250in}}%
\pgfpathcurveto{\pgfqpoint{2.538589in}{2.505200in}}{\pgfqpoint{2.542979in}{2.494601in}}{\pgfqpoint{2.550793in}{2.486787in}}%
\pgfpathcurveto{\pgfqpoint{2.558607in}{2.478974in}}{\pgfqpoint{2.569206in}{2.474583in}}{\pgfqpoint{2.580256in}{2.474583in}}%
\pgfpathclose%
\pgfusepath{stroke,fill}%
\end{pgfscope}%
\begin{pgfscope}%
\pgfpathrectangle{\pgfqpoint{0.375000in}{0.330000in}}{\pgfqpoint{2.325000in}{2.310000in}}%
\pgfusepath{clip}%
\pgfsetbuttcap%
\pgfsetroundjoin%
\definecolor{currentfill}{rgb}{0.000000,0.000000,0.000000}%
\pgfsetfillcolor{currentfill}%
\pgfsetlinewidth{1.003750pt}%
\definecolor{currentstroke}{rgb}{0.000000,0.000000,0.000000}%
\pgfsetstrokecolor{currentstroke}%
\pgfsetdash{}{0pt}%
\pgfpathmoveto{\pgfqpoint{2.580256in}{2.474583in}}%
\pgfpathcurveto{\pgfqpoint{2.591306in}{2.474583in}}{\pgfqpoint{2.601905in}{2.478974in}}{\pgfqpoint{2.609718in}{2.486787in}}%
\pgfpathcurveto{\pgfqpoint{2.617532in}{2.494601in}}{\pgfqpoint{2.621922in}{2.505200in}}{\pgfqpoint{2.621922in}{2.516250in}}%
\pgfpathcurveto{\pgfqpoint{2.621922in}{2.527300in}}{\pgfqpoint{2.617532in}{2.537899in}}{\pgfqpoint{2.609718in}{2.545713in}}%
\pgfpathcurveto{\pgfqpoint{2.601905in}{2.553526in}}{\pgfqpoint{2.591306in}{2.557917in}}{\pgfqpoint{2.580256in}{2.557917in}}%
\pgfpathcurveto{\pgfqpoint{2.569206in}{2.557917in}}{\pgfqpoint{2.558607in}{2.553526in}}{\pgfqpoint{2.550793in}{2.545713in}}%
\pgfpathcurveto{\pgfqpoint{2.542979in}{2.537899in}}{\pgfqpoint{2.538589in}{2.527300in}}{\pgfqpoint{2.538589in}{2.516250in}}%
\pgfpathcurveto{\pgfqpoint{2.538589in}{2.505200in}}{\pgfqpoint{2.542979in}{2.494601in}}{\pgfqpoint{2.550793in}{2.486787in}}%
\pgfpathcurveto{\pgfqpoint{2.558607in}{2.478974in}}{\pgfqpoint{2.569206in}{2.474583in}}{\pgfqpoint{2.580256in}{2.474583in}}%
\pgfpathclose%
\pgfusepath{stroke,fill}%
\end{pgfscope}%
\begin{pgfscope}%
\pgfpathrectangle{\pgfqpoint{0.375000in}{0.330000in}}{\pgfqpoint{2.325000in}{2.310000in}}%
\pgfusepath{clip}%
\pgfsetbuttcap%
\pgfsetroundjoin%
\definecolor{currentfill}{rgb}{0.000000,0.000000,0.000000}%
\pgfsetfillcolor{currentfill}%
\pgfsetlinewidth{1.003750pt}%
\definecolor{currentstroke}{rgb}{0.000000,0.000000,0.000000}%
\pgfsetstrokecolor{currentstroke}%
\pgfsetdash{}{0pt}%
\pgfpathmoveto{\pgfqpoint{2.580256in}{2.474583in}}%
\pgfpathcurveto{\pgfqpoint{2.591306in}{2.474583in}}{\pgfqpoint{2.601905in}{2.478974in}}{\pgfqpoint{2.609718in}{2.486787in}}%
\pgfpathcurveto{\pgfqpoint{2.617532in}{2.494601in}}{\pgfqpoint{2.621922in}{2.505200in}}{\pgfqpoint{2.621922in}{2.516250in}}%
\pgfpathcurveto{\pgfqpoint{2.621922in}{2.527300in}}{\pgfqpoint{2.617532in}{2.537899in}}{\pgfqpoint{2.609718in}{2.545713in}}%
\pgfpathcurveto{\pgfqpoint{2.601905in}{2.553526in}}{\pgfqpoint{2.591306in}{2.557917in}}{\pgfqpoint{2.580256in}{2.557917in}}%
\pgfpathcurveto{\pgfqpoint{2.569206in}{2.557917in}}{\pgfqpoint{2.558607in}{2.553526in}}{\pgfqpoint{2.550793in}{2.545713in}}%
\pgfpathcurveto{\pgfqpoint{2.542979in}{2.537899in}}{\pgfqpoint{2.538589in}{2.527300in}}{\pgfqpoint{2.538589in}{2.516250in}}%
\pgfpathcurveto{\pgfqpoint{2.538589in}{2.505200in}}{\pgfqpoint{2.542979in}{2.494601in}}{\pgfqpoint{2.550793in}{2.486787in}}%
\pgfpathcurveto{\pgfqpoint{2.558607in}{2.478974in}}{\pgfqpoint{2.569206in}{2.474583in}}{\pgfqpoint{2.580256in}{2.474583in}}%
\pgfpathclose%
\pgfusepath{stroke,fill}%
\end{pgfscope}%
\begin{pgfscope}%
\pgfpathrectangle{\pgfqpoint{0.375000in}{0.330000in}}{\pgfqpoint{2.325000in}{2.310000in}}%
\pgfusepath{clip}%
\pgfsetbuttcap%
\pgfsetroundjoin%
\definecolor{currentfill}{rgb}{0.000000,0.000000,0.000000}%
\pgfsetfillcolor{currentfill}%
\pgfsetlinewidth{1.003750pt}%
\definecolor{currentstroke}{rgb}{0.000000,0.000000,0.000000}%
\pgfsetstrokecolor{currentstroke}%
\pgfsetdash{}{0pt}%
\pgfpathmoveto{\pgfqpoint{2.580256in}{2.474583in}}%
\pgfpathcurveto{\pgfqpoint{2.591306in}{2.474583in}}{\pgfqpoint{2.601905in}{2.478974in}}{\pgfqpoint{2.609718in}{2.486787in}}%
\pgfpathcurveto{\pgfqpoint{2.617532in}{2.494601in}}{\pgfqpoint{2.621922in}{2.505200in}}{\pgfqpoint{2.621922in}{2.516250in}}%
\pgfpathcurveto{\pgfqpoint{2.621922in}{2.527300in}}{\pgfqpoint{2.617532in}{2.537899in}}{\pgfqpoint{2.609718in}{2.545713in}}%
\pgfpathcurveto{\pgfqpoint{2.601905in}{2.553526in}}{\pgfqpoint{2.591306in}{2.557917in}}{\pgfqpoint{2.580256in}{2.557917in}}%
\pgfpathcurveto{\pgfqpoint{2.569206in}{2.557917in}}{\pgfqpoint{2.558607in}{2.553526in}}{\pgfqpoint{2.550793in}{2.545713in}}%
\pgfpathcurveto{\pgfqpoint{2.542979in}{2.537899in}}{\pgfqpoint{2.538589in}{2.527300in}}{\pgfqpoint{2.538589in}{2.516250in}}%
\pgfpathcurveto{\pgfqpoint{2.538589in}{2.505200in}}{\pgfqpoint{2.542979in}{2.494601in}}{\pgfqpoint{2.550793in}{2.486787in}}%
\pgfpathcurveto{\pgfqpoint{2.558607in}{2.478974in}}{\pgfqpoint{2.569206in}{2.474583in}}{\pgfqpoint{2.580256in}{2.474583in}}%
\pgfpathclose%
\pgfusepath{stroke,fill}%
\end{pgfscope}%
\begin{pgfscope}%
\pgfpathrectangle{\pgfqpoint{0.375000in}{0.330000in}}{\pgfqpoint{2.325000in}{2.310000in}}%
\pgfusepath{clip}%
\pgfsetbuttcap%
\pgfsetroundjoin%
\definecolor{currentfill}{rgb}{0.000000,0.000000,0.000000}%
\pgfsetfillcolor{currentfill}%
\pgfsetlinewidth{1.003750pt}%
\definecolor{currentstroke}{rgb}{0.000000,0.000000,0.000000}%
\pgfsetstrokecolor{currentstroke}%
\pgfsetdash{}{0pt}%
\pgfpathmoveto{\pgfqpoint{2.580256in}{2.474583in}}%
\pgfpathcurveto{\pgfqpoint{2.591306in}{2.474583in}}{\pgfqpoint{2.601905in}{2.478974in}}{\pgfqpoint{2.609718in}{2.486787in}}%
\pgfpathcurveto{\pgfqpoint{2.617532in}{2.494601in}}{\pgfqpoint{2.621922in}{2.505200in}}{\pgfqpoint{2.621922in}{2.516250in}}%
\pgfpathcurveto{\pgfqpoint{2.621922in}{2.527300in}}{\pgfqpoint{2.617532in}{2.537899in}}{\pgfqpoint{2.609718in}{2.545713in}}%
\pgfpathcurveto{\pgfqpoint{2.601905in}{2.553526in}}{\pgfqpoint{2.591306in}{2.557917in}}{\pgfqpoint{2.580256in}{2.557917in}}%
\pgfpathcurveto{\pgfqpoint{2.569206in}{2.557917in}}{\pgfqpoint{2.558607in}{2.553526in}}{\pgfqpoint{2.550793in}{2.545713in}}%
\pgfpathcurveto{\pgfqpoint{2.542979in}{2.537899in}}{\pgfqpoint{2.538589in}{2.527300in}}{\pgfqpoint{2.538589in}{2.516250in}}%
\pgfpathcurveto{\pgfqpoint{2.538589in}{2.505200in}}{\pgfqpoint{2.542979in}{2.494601in}}{\pgfqpoint{2.550793in}{2.486787in}}%
\pgfpathcurveto{\pgfqpoint{2.558607in}{2.478974in}}{\pgfqpoint{2.569206in}{2.474583in}}{\pgfqpoint{2.580256in}{2.474583in}}%
\pgfpathclose%
\pgfusepath{stroke,fill}%
\end{pgfscope}%
\begin{pgfscope}%
\pgfpathrectangle{\pgfqpoint{0.375000in}{0.330000in}}{\pgfqpoint{2.325000in}{2.310000in}}%
\pgfusepath{clip}%
\pgfsetbuttcap%
\pgfsetroundjoin%
\definecolor{currentfill}{rgb}{0.000000,0.000000,0.000000}%
\pgfsetfillcolor{currentfill}%
\pgfsetlinewidth{1.003750pt}%
\definecolor{currentstroke}{rgb}{0.000000,0.000000,0.000000}%
\pgfsetstrokecolor{currentstroke}%
\pgfsetdash{}{0pt}%
\pgfpathmoveto{\pgfqpoint{2.580256in}{2.474583in}}%
\pgfpathcurveto{\pgfqpoint{2.591306in}{2.474583in}}{\pgfqpoint{2.601905in}{2.478974in}}{\pgfqpoint{2.609718in}{2.486787in}}%
\pgfpathcurveto{\pgfqpoint{2.617532in}{2.494601in}}{\pgfqpoint{2.621922in}{2.505200in}}{\pgfqpoint{2.621922in}{2.516250in}}%
\pgfpathcurveto{\pgfqpoint{2.621922in}{2.527300in}}{\pgfqpoint{2.617532in}{2.537899in}}{\pgfqpoint{2.609718in}{2.545713in}}%
\pgfpathcurveto{\pgfqpoint{2.601905in}{2.553526in}}{\pgfqpoint{2.591306in}{2.557917in}}{\pgfqpoint{2.580256in}{2.557917in}}%
\pgfpathcurveto{\pgfqpoint{2.569206in}{2.557917in}}{\pgfqpoint{2.558607in}{2.553526in}}{\pgfqpoint{2.550793in}{2.545713in}}%
\pgfpathcurveto{\pgfqpoint{2.542979in}{2.537899in}}{\pgfqpoint{2.538589in}{2.527300in}}{\pgfqpoint{2.538589in}{2.516250in}}%
\pgfpathcurveto{\pgfqpoint{2.538589in}{2.505200in}}{\pgfqpoint{2.542979in}{2.494601in}}{\pgfqpoint{2.550793in}{2.486787in}}%
\pgfpathcurveto{\pgfqpoint{2.558607in}{2.478974in}}{\pgfqpoint{2.569206in}{2.474583in}}{\pgfqpoint{2.580256in}{2.474583in}}%
\pgfpathclose%
\pgfusepath{stroke,fill}%
\end{pgfscope}%
\begin{pgfscope}%
\pgfpathrectangle{\pgfqpoint{0.375000in}{0.330000in}}{\pgfqpoint{2.325000in}{2.310000in}}%
\pgfusepath{clip}%
\pgfsetbuttcap%
\pgfsetroundjoin%
\definecolor{currentfill}{rgb}{0.000000,0.000000,0.000000}%
\pgfsetfillcolor{currentfill}%
\pgfsetlinewidth{1.003750pt}%
\definecolor{currentstroke}{rgb}{0.000000,0.000000,0.000000}%
\pgfsetstrokecolor{currentstroke}%
\pgfsetdash{}{0pt}%
\pgfpathmoveto{\pgfqpoint{2.580256in}{2.474583in}}%
\pgfpathcurveto{\pgfqpoint{2.591306in}{2.474583in}}{\pgfqpoint{2.601905in}{2.478974in}}{\pgfqpoint{2.609718in}{2.486787in}}%
\pgfpathcurveto{\pgfqpoint{2.617532in}{2.494601in}}{\pgfqpoint{2.621922in}{2.505200in}}{\pgfqpoint{2.621922in}{2.516250in}}%
\pgfpathcurveto{\pgfqpoint{2.621922in}{2.527300in}}{\pgfqpoint{2.617532in}{2.537899in}}{\pgfqpoint{2.609718in}{2.545713in}}%
\pgfpathcurveto{\pgfqpoint{2.601905in}{2.553526in}}{\pgfqpoint{2.591306in}{2.557917in}}{\pgfqpoint{2.580256in}{2.557917in}}%
\pgfpathcurveto{\pgfqpoint{2.569206in}{2.557917in}}{\pgfqpoint{2.558607in}{2.553526in}}{\pgfqpoint{2.550793in}{2.545713in}}%
\pgfpathcurveto{\pgfqpoint{2.542979in}{2.537899in}}{\pgfqpoint{2.538589in}{2.527300in}}{\pgfqpoint{2.538589in}{2.516250in}}%
\pgfpathcurveto{\pgfqpoint{2.538589in}{2.505200in}}{\pgfqpoint{2.542979in}{2.494601in}}{\pgfqpoint{2.550793in}{2.486787in}}%
\pgfpathcurveto{\pgfqpoint{2.558607in}{2.478974in}}{\pgfqpoint{2.569206in}{2.474583in}}{\pgfqpoint{2.580256in}{2.474583in}}%
\pgfpathclose%
\pgfusepath{stroke,fill}%
\end{pgfscope}%
\begin{pgfscope}%
\pgfpathrectangle{\pgfqpoint{0.375000in}{0.330000in}}{\pgfqpoint{2.325000in}{2.310000in}}%
\pgfusepath{clip}%
\pgfsetbuttcap%
\pgfsetroundjoin%
\definecolor{currentfill}{rgb}{0.000000,0.000000,0.000000}%
\pgfsetfillcolor{currentfill}%
\pgfsetlinewidth{1.003750pt}%
\definecolor{currentstroke}{rgb}{0.000000,0.000000,0.000000}%
\pgfsetstrokecolor{currentstroke}%
\pgfsetdash{}{0pt}%
\pgfpathmoveto{\pgfqpoint{2.580256in}{2.474583in}}%
\pgfpathcurveto{\pgfqpoint{2.591306in}{2.474583in}}{\pgfqpoint{2.601905in}{2.478974in}}{\pgfqpoint{2.609718in}{2.486787in}}%
\pgfpathcurveto{\pgfqpoint{2.617532in}{2.494601in}}{\pgfqpoint{2.621922in}{2.505200in}}{\pgfqpoint{2.621922in}{2.516250in}}%
\pgfpathcurveto{\pgfqpoint{2.621922in}{2.527300in}}{\pgfqpoint{2.617532in}{2.537899in}}{\pgfqpoint{2.609718in}{2.545713in}}%
\pgfpathcurveto{\pgfqpoint{2.601905in}{2.553526in}}{\pgfqpoint{2.591306in}{2.557917in}}{\pgfqpoint{2.580256in}{2.557917in}}%
\pgfpathcurveto{\pgfqpoint{2.569206in}{2.557917in}}{\pgfqpoint{2.558607in}{2.553526in}}{\pgfqpoint{2.550793in}{2.545713in}}%
\pgfpathcurveto{\pgfqpoint{2.542979in}{2.537899in}}{\pgfqpoint{2.538589in}{2.527300in}}{\pgfqpoint{2.538589in}{2.516250in}}%
\pgfpathcurveto{\pgfqpoint{2.538589in}{2.505200in}}{\pgfqpoint{2.542979in}{2.494601in}}{\pgfqpoint{2.550793in}{2.486787in}}%
\pgfpathcurveto{\pgfqpoint{2.558607in}{2.478974in}}{\pgfqpoint{2.569206in}{2.474583in}}{\pgfqpoint{2.580256in}{2.474583in}}%
\pgfpathclose%
\pgfusepath{stroke,fill}%
\end{pgfscope}%
\begin{pgfscope}%
\pgfpathrectangle{\pgfqpoint{0.375000in}{0.330000in}}{\pgfqpoint{2.325000in}{2.310000in}}%
\pgfusepath{clip}%
\pgfsetbuttcap%
\pgfsetroundjoin%
\definecolor{currentfill}{rgb}{0.000000,0.000000,0.000000}%
\pgfsetfillcolor{currentfill}%
\pgfsetlinewidth{1.003750pt}%
\definecolor{currentstroke}{rgb}{0.000000,0.000000,0.000000}%
\pgfsetstrokecolor{currentstroke}%
\pgfsetdash{}{0pt}%
\pgfpathmoveto{\pgfqpoint{2.580256in}{2.474583in}}%
\pgfpathcurveto{\pgfqpoint{2.591306in}{2.474583in}}{\pgfqpoint{2.601905in}{2.478974in}}{\pgfqpoint{2.609718in}{2.486787in}}%
\pgfpathcurveto{\pgfqpoint{2.617532in}{2.494601in}}{\pgfqpoint{2.621922in}{2.505200in}}{\pgfqpoint{2.621922in}{2.516250in}}%
\pgfpathcurveto{\pgfqpoint{2.621922in}{2.527300in}}{\pgfqpoint{2.617532in}{2.537899in}}{\pgfqpoint{2.609718in}{2.545713in}}%
\pgfpathcurveto{\pgfqpoint{2.601905in}{2.553526in}}{\pgfqpoint{2.591306in}{2.557917in}}{\pgfqpoint{2.580256in}{2.557917in}}%
\pgfpathcurveto{\pgfqpoint{2.569206in}{2.557917in}}{\pgfqpoint{2.558607in}{2.553526in}}{\pgfqpoint{2.550793in}{2.545713in}}%
\pgfpathcurveto{\pgfqpoint{2.542979in}{2.537899in}}{\pgfqpoint{2.538589in}{2.527300in}}{\pgfqpoint{2.538589in}{2.516250in}}%
\pgfpathcurveto{\pgfqpoint{2.538589in}{2.505200in}}{\pgfqpoint{2.542979in}{2.494601in}}{\pgfqpoint{2.550793in}{2.486787in}}%
\pgfpathcurveto{\pgfqpoint{2.558607in}{2.478974in}}{\pgfqpoint{2.569206in}{2.474583in}}{\pgfqpoint{2.580256in}{2.474583in}}%
\pgfpathclose%
\pgfusepath{stroke,fill}%
\end{pgfscope}%
\begin{pgfscope}%
\pgfpathrectangle{\pgfqpoint{0.375000in}{0.330000in}}{\pgfqpoint{2.325000in}{2.310000in}}%
\pgfusepath{clip}%
\pgfsetbuttcap%
\pgfsetroundjoin%
\definecolor{currentfill}{rgb}{0.000000,0.000000,0.000000}%
\pgfsetfillcolor{currentfill}%
\pgfsetlinewidth{1.003750pt}%
\definecolor{currentstroke}{rgb}{0.000000,0.000000,0.000000}%
\pgfsetstrokecolor{currentstroke}%
\pgfsetdash{}{0pt}%
\pgfpathmoveto{\pgfqpoint{2.580256in}{2.474583in}}%
\pgfpathcurveto{\pgfqpoint{2.591306in}{2.474583in}}{\pgfqpoint{2.601905in}{2.478974in}}{\pgfqpoint{2.609718in}{2.486787in}}%
\pgfpathcurveto{\pgfqpoint{2.617532in}{2.494601in}}{\pgfqpoint{2.621922in}{2.505200in}}{\pgfqpoint{2.621922in}{2.516250in}}%
\pgfpathcurveto{\pgfqpoint{2.621922in}{2.527300in}}{\pgfqpoint{2.617532in}{2.537899in}}{\pgfqpoint{2.609718in}{2.545713in}}%
\pgfpathcurveto{\pgfqpoint{2.601905in}{2.553526in}}{\pgfqpoint{2.591306in}{2.557917in}}{\pgfqpoint{2.580256in}{2.557917in}}%
\pgfpathcurveto{\pgfqpoint{2.569206in}{2.557917in}}{\pgfqpoint{2.558607in}{2.553526in}}{\pgfqpoint{2.550793in}{2.545713in}}%
\pgfpathcurveto{\pgfqpoint{2.542979in}{2.537899in}}{\pgfqpoint{2.538589in}{2.527300in}}{\pgfqpoint{2.538589in}{2.516250in}}%
\pgfpathcurveto{\pgfqpoint{2.538589in}{2.505200in}}{\pgfqpoint{2.542979in}{2.494601in}}{\pgfqpoint{2.550793in}{2.486787in}}%
\pgfpathcurveto{\pgfqpoint{2.558607in}{2.478974in}}{\pgfqpoint{2.569206in}{2.474583in}}{\pgfqpoint{2.580256in}{2.474583in}}%
\pgfpathclose%
\pgfusepath{stroke,fill}%
\end{pgfscope}%
\begin{pgfscope}%
\pgfpathrectangle{\pgfqpoint{0.375000in}{0.330000in}}{\pgfqpoint{2.325000in}{2.310000in}}%
\pgfusepath{clip}%
\pgfsetbuttcap%
\pgfsetroundjoin%
\definecolor{currentfill}{rgb}{0.000000,0.000000,0.000000}%
\pgfsetfillcolor{currentfill}%
\pgfsetlinewidth{1.003750pt}%
\definecolor{currentstroke}{rgb}{0.000000,0.000000,0.000000}%
\pgfsetstrokecolor{currentstroke}%
\pgfsetdash{}{0pt}%
\pgfpathmoveto{\pgfqpoint{2.580256in}{2.474583in}}%
\pgfpathcurveto{\pgfqpoint{2.591306in}{2.474583in}}{\pgfqpoint{2.601905in}{2.478974in}}{\pgfqpoint{2.609718in}{2.486787in}}%
\pgfpathcurveto{\pgfqpoint{2.617532in}{2.494601in}}{\pgfqpoint{2.621922in}{2.505200in}}{\pgfqpoint{2.621922in}{2.516250in}}%
\pgfpathcurveto{\pgfqpoint{2.621922in}{2.527300in}}{\pgfqpoint{2.617532in}{2.537899in}}{\pgfqpoint{2.609718in}{2.545713in}}%
\pgfpathcurveto{\pgfqpoint{2.601905in}{2.553526in}}{\pgfqpoint{2.591306in}{2.557917in}}{\pgfqpoint{2.580256in}{2.557917in}}%
\pgfpathcurveto{\pgfqpoint{2.569206in}{2.557917in}}{\pgfqpoint{2.558607in}{2.553526in}}{\pgfqpoint{2.550793in}{2.545713in}}%
\pgfpathcurveto{\pgfqpoint{2.542979in}{2.537899in}}{\pgfqpoint{2.538589in}{2.527300in}}{\pgfqpoint{2.538589in}{2.516250in}}%
\pgfpathcurveto{\pgfqpoint{2.538589in}{2.505200in}}{\pgfqpoint{2.542979in}{2.494601in}}{\pgfqpoint{2.550793in}{2.486787in}}%
\pgfpathcurveto{\pgfqpoint{2.558607in}{2.478974in}}{\pgfqpoint{2.569206in}{2.474583in}}{\pgfqpoint{2.580256in}{2.474583in}}%
\pgfpathclose%
\pgfusepath{stroke,fill}%
\end{pgfscope}%
\begin{pgfscope}%
\pgfpathrectangle{\pgfqpoint{0.375000in}{0.330000in}}{\pgfqpoint{2.325000in}{2.310000in}}%
\pgfusepath{clip}%
\pgfsetbuttcap%
\pgfsetroundjoin%
\definecolor{currentfill}{rgb}{0.000000,0.000000,0.000000}%
\pgfsetfillcolor{currentfill}%
\pgfsetlinewidth{1.003750pt}%
\definecolor{currentstroke}{rgb}{0.000000,0.000000,0.000000}%
\pgfsetstrokecolor{currentstroke}%
\pgfsetdash{}{0pt}%
\pgfpathmoveto{\pgfqpoint{2.580256in}{2.474583in}}%
\pgfpathcurveto{\pgfqpoint{2.591306in}{2.474583in}}{\pgfqpoint{2.601905in}{2.478974in}}{\pgfqpoint{2.609718in}{2.486787in}}%
\pgfpathcurveto{\pgfqpoint{2.617532in}{2.494601in}}{\pgfqpoint{2.621922in}{2.505200in}}{\pgfqpoint{2.621922in}{2.516250in}}%
\pgfpathcurveto{\pgfqpoint{2.621922in}{2.527300in}}{\pgfqpoint{2.617532in}{2.537899in}}{\pgfqpoint{2.609718in}{2.545713in}}%
\pgfpathcurveto{\pgfqpoint{2.601905in}{2.553526in}}{\pgfqpoint{2.591306in}{2.557917in}}{\pgfqpoint{2.580256in}{2.557917in}}%
\pgfpathcurveto{\pgfqpoint{2.569206in}{2.557917in}}{\pgfqpoint{2.558607in}{2.553526in}}{\pgfqpoint{2.550793in}{2.545713in}}%
\pgfpathcurveto{\pgfqpoint{2.542979in}{2.537899in}}{\pgfqpoint{2.538589in}{2.527300in}}{\pgfqpoint{2.538589in}{2.516250in}}%
\pgfpathcurveto{\pgfqpoint{2.538589in}{2.505200in}}{\pgfqpoint{2.542979in}{2.494601in}}{\pgfqpoint{2.550793in}{2.486787in}}%
\pgfpathcurveto{\pgfqpoint{2.558607in}{2.478974in}}{\pgfqpoint{2.569206in}{2.474583in}}{\pgfqpoint{2.580256in}{2.474583in}}%
\pgfpathclose%
\pgfusepath{stroke,fill}%
\end{pgfscope}%
\begin{pgfscope}%
\pgfpathrectangle{\pgfqpoint{0.375000in}{0.330000in}}{\pgfqpoint{2.325000in}{2.310000in}}%
\pgfusepath{clip}%
\pgfsetbuttcap%
\pgfsetroundjoin%
\definecolor{currentfill}{rgb}{0.000000,0.000000,0.000000}%
\pgfsetfillcolor{currentfill}%
\pgfsetlinewidth{1.003750pt}%
\definecolor{currentstroke}{rgb}{0.000000,0.000000,0.000000}%
\pgfsetstrokecolor{currentstroke}%
\pgfsetdash{}{0pt}%
\pgfpathmoveto{\pgfqpoint{2.580256in}{2.474583in}}%
\pgfpathcurveto{\pgfqpoint{2.591306in}{2.474583in}}{\pgfqpoint{2.601905in}{2.478974in}}{\pgfqpoint{2.609718in}{2.486787in}}%
\pgfpathcurveto{\pgfqpoint{2.617532in}{2.494601in}}{\pgfqpoint{2.621922in}{2.505200in}}{\pgfqpoint{2.621922in}{2.516250in}}%
\pgfpathcurveto{\pgfqpoint{2.621922in}{2.527300in}}{\pgfqpoint{2.617532in}{2.537899in}}{\pgfqpoint{2.609718in}{2.545713in}}%
\pgfpathcurveto{\pgfqpoint{2.601905in}{2.553526in}}{\pgfqpoint{2.591306in}{2.557917in}}{\pgfqpoint{2.580256in}{2.557917in}}%
\pgfpathcurveto{\pgfqpoint{2.569206in}{2.557917in}}{\pgfqpoint{2.558607in}{2.553526in}}{\pgfqpoint{2.550793in}{2.545713in}}%
\pgfpathcurveto{\pgfqpoint{2.542979in}{2.537899in}}{\pgfqpoint{2.538589in}{2.527300in}}{\pgfqpoint{2.538589in}{2.516250in}}%
\pgfpathcurveto{\pgfqpoint{2.538589in}{2.505200in}}{\pgfqpoint{2.542979in}{2.494601in}}{\pgfqpoint{2.550793in}{2.486787in}}%
\pgfpathcurveto{\pgfqpoint{2.558607in}{2.478974in}}{\pgfqpoint{2.569206in}{2.474583in}}{\pgfqpoint{2.580256in}{2.474583in}}%
\pgfpathclose%
\pgfusepath{stroke,fill}%
\end{pgfscope}%
\begin{pgfscope}%
\pgfpathrectangle{\pgfqpoint{0.375000in}{0.330000in}}{\pgfqpoint{2.325000in}{2.310000in}}%
\pgfusepath{clip}%
\pgfsetbuttcap%
\pgfsetroundjoin%
\definecolor{currentfill}{rgb}{0.000000,0.000000,0.000000}%
\pgfsetfillcolor{currentfill}%
\pgfsetlinewidth{1.003750pt}%
\definecolor{currentstroke}{rgb}{0.000000,0.000000,0.000000}%
\pgfsetstrokecolor{currentstroke}%
\pgfsetdash{}{0pt}%
\pgfpathmoveto{\pgfqpoint{2.580256in}{1.438667in}}%
\pgfpathcurveto{\pgfqpoint{2.591306in}{1.438667in}}{\pgfqpoint{2.601905in}{1.443057in}}{\pgfqpoint{2.609718in}{1.450871in}}%
\pgfpathcurveto{\pgfqpoint{2.617532in}{1.458685in}}{\pgfqpoint{2.621922in}{1.469284in}}{\pgfqpoint{2.621922in}{1.480334in}}%
\pgfpathcurveto{\pgfqpoint{2.621922in}{1.491384in}}{\pgfqpoint{2.617532in}{1.501983in}}{\pgfqpoint{2.609718in}{1.509796in}}%
\pgfpathcurveto{\pgfqpoint{2.601905in}{1.517610in}}{\pgfqpoint{2.591306in}{1.522000in}}{\pgfqpoint{2.580256in}{1.522000in}}%
\pgfpathcurveto{\pgfqpoint{2.569206in}{1.522000in}}{\pgfqpoint{2.558607in}{1.517610in}}{\pgfqpoint{2.550793in}{1.509796in}}%
\pgfpathcurveto{\pgfqpoint{2.542979in}{1.501983in}}{\pgfqpoint{2.538589in}{1.491384in}}{\pgfqpoint{2.538589in}{1.480334in}}%
\pgfpathcurveto{\pgfqpoint{2.538589in}{1.469284in}}{\pgfqpoint{2.542979in}{1.458685in}}{\pgfqpoint{2.550793in}{1.450871in}}%
\pgfpathcurveto{\pgfqpoint{2.558607in}{1.443057in}}{\pgfqpoint{2.569206in}{1.438667in}}{\pgfqpoint{2.580256in}{1.438667in}}%
\pgfpathclose%
\pgfusepath{stroke,fill}%
\end{pgfscope}%
\begin{pgfscope}%
\pgfpathrectangle{\pgfqpoint{0.375000in}{0.330000in}}{\pgfqpoint{2.325000in}{2.310000in}}%
\pgfusepath{clip}%
\pgfsetbuttcap%
\pgfsetroundjoin%
\definecolor{currentfill}{rgb}{0.000000,0.000000,0.000000}%
\pgfsetfillcolor{currentfill}%
\pgfsetlinewidth{1.003750pt}%
\definecolor{currentstroke}{rgb}{0.000000,0.000000,0.000000}%
\pgfsetstrokecolor{currentstroke}%
\pgfsetdash{}{0pt}%
\pgfpathmoveto{\pgfqpoint{2.580256in}{2.474583in}}%
\pgfpathcurveto{\pgfqpoint{2.591306in}{2.474583in}}{\pgfqpoint{2.601905in}{2.478974in}}{\pgfqpoint{2.609718in}{2.486787in}}%
\pgfpathcurveto{\pgfqpoint{2.617532in}{2.494601in}}{\pgfqpoint{2.621922in}{2.505200in}}{\pgfqpoint{2.621922in}{2.516250in}}%
\pgfpathcurveto{\pgfqpoint{2.621922in}{2.527300in}}{\pgfqpoint{2.617532in}{2.537899in}}{\pgfqpoint{2.609718in}{2.545713in}}%
\pgfpathcurveto{\pgfqpoint{2.601905in}{2.553526in}}{\pgfqpoint{2.591306in}{2.557917in}}{\pgfqpoint{2.580256in}{2.557917in}}%
\pgfpathcurveto{\pgfqpoint{2.569206in}{2.557917in}}{\pgfqpoint{2.558607in}{2.553526in}}{\pgfqpoint{2.550793in}{2.545713in}}%
\pgfpathcurveto{\pgfqpoint{2.542979in}{2.537899in}}{\pgfqpoint{2.538589in}{2.527300in}}{\pgfqpoint{2.538589in}{2.516250in}}%
\pgfpathcurveto{\pgfqpoint{2.538589in}{2.505200in}}{\pgfqpoint{2.542979in}{2.494601in}}{\pgfqpoint{2.550793in}{2.486787in}}%
\pgfpathcurveto{\pgfqpoint{2.558607in}{2.478974in}}{\pgfqpoint{2.569206in}{2.474583in}}{\pgfqpoint{2.580256in}{2.474583in}}%
\pgfpathclose%
\pgfusepath{stroke,fill}%
\end{pgfscope}%
\begin{pgfscope}%
\pgfpathrectangle{\pgfqpoint{0.375000in}{0.330000in}}{\pgfqpoint{2.325000in}{2.310000in}}%
\pgfusepath{clip}%
\pgfsetbuttcap%
\pgfsetroundjoin%
\definecolor{currentfill}{rgb}{0.000000,0.000000,0.000000}%
\pgfsetfillcolor{currentfill}%
\pgfsetlinewidth{1.003750pt}%
\definecolor{currentstroke}{rgb}{0.000000,0.000000,0.000000}%
\pgfsetstrokecolor{currentstroke}%
\pgfsetdash{}{0pt}%
\pgfpathmoveto{\pgfqpoint{2.580256in}{1.438667in}}%
\pgfpathcurveto{\pgfqpoint{2.591306in}{1.438667in}}{\pgfqpoint{2.601905in}{1.443057in}}{\pgfqpoint{2.609718in}{1.450871in}}%
\pgfpathcurveto{\pgfqpoint{2.617532in}{1.458685in}}{\pgfqpoint{2.621922in}{1.469284in}}{\pgfqpoint{2.621922in}{1.480334in}}%
\pgfpathcurveto{\pgfqpoint{2.621922in}{1.491384in}}{\pgfqpoint{2.617532in}{1.501983in}}{\pgfqpoint{2.609718in}{1.509796in}}%
\pgfpathcurveto{\pgfqpoint{2.601905in}{1.517610in}}{\pgfqpoint{2.591306in}{1.522000in}}{\pgfqpoint{2.580256in}{1.522000in}}%
\pgfpathcurveto{\pgfqpoint{2.569206in}{1.522000in}}{\pgfqpoint{2.558607in}{1.517610in}}{\pgfqpoint{2.550793in}{1.509796in}}%
\pgfpathcurveto{\pgfqpoint{2.542979in}{1.501983in}}{\pgfqpoint{2.538589in}{1.491384in}}{\pgfqpoint{2.538589in}{1.480334in}}%
\pgfpathcurveto{\pgfqpoint{2.538589in}{1.469284in}}{\pgfqpoint{2.542979in}{1.458685in}}{\pgfqpoint{2.550793in}{1.450871in}}%
\pgfpathcurveto{\pgfqpoint{2.558607in}{1.443057in}}{\pgfqpoint{2.569206in}{1.438667in}}{\pgfqpoint{2.580256in}{1.438667in}}%
\pgfpathclose%
\pgfusepath{stroke,fill}%
\end{pgfscope}%
\begin{pgfscope}%
\pgfsetbuttcap%
\pgfsetroundjoin%
\definecolor{currentfill}{rgb}{0.000000,0.000000,0.000000}%
\pgfsetfillcolor{currentfill}%
\pgfsetlinewidth{0.803000pt}%
\definecolor{currentstroke}{rgb}{0.000000,0.000000,0.000000}%
\pgfsetstrokecolor{currentstroke}%
\pgfsetdash{}{0pt}%
\pgfsys@defobject{currentmarker}{\pgfqpoint{0.000000in}{-0.048611in}}{\pgfqpoint{0.000000in}{0.000000in}}{%
\pgfpathmoveto{\pgfqpoint{0.000000in}{0.000000in}}%
\pgfpathlineto{\pgfqpoint{0.000000in}{-0.048611in}}%
\pgfusepath{stroke,fill}%
}%
\begin{pgfscope}%
\pgfsys@transformshift{0.480841in}{0.330000in}%
\pgfsys@useobject{currentmarker}{}%
\end{pgfscope}%
\end{pgfscope}%
\begin{pgfscope}%
\definecolor{textcolor}{rgb}{0.000000,0.000000,0.000000}%
\pgfsetstrokecolor{textcolor}%
\pgfsetfillcolor{textcolor}%
\pgftext[x=0.480841in,y=0.232778in,,top]{\color{textcolor}\sffamily\fontsize{10.000000}{12.000000}\selectfont 20}%
\end{pgfscope}%
\begin{pgfscope}%
\pgfsetbuttcap%
\pgfsetroundjoin%
\definecolor{currentfill}{rgb}{0.000000,0.000000,0.000000}%
\pgfsetfillcolor{currentfill}%
\pgfsetlinewidth{0.803000pt}%
\definecolor{currentstroke}{rgb}{0.000000,0.000000,0.000000}%
\pgfsetstrokecolor{currentstroke}%
\pgfsetdash{}{0pt}%
\pgfsys@defobject{currentmarker}{\pgfqpoint{0.000000in}{-0.048611in}}{\pgfqpoint{0.000000in}{0.000000in}}{%
\pgfpathmoveto{\pgfqpoint{0.000000in}{0.000000in}}%
\pgfpathlineto{\pgfqpoint{0.000000in}{-0.048611in}}%
\pgfusepath{stroke,fill}%
}%
\begin{pgfscope}%
\pgfsys@transformshift{1.005694in}{0.330000in}%
\pgfsys@useobject{currentmarker}{}%
\end{pgfscope}%
\end{pgfscope}%
\begin{pgfscope}%
\definecolor{textcolor}{rgb}{0.000000,0.000000,0.000000}%
\pgfsetstrokecolor{textcolor}%
\pgfsetfillcolor{textcolor}%
\pgftext[x=1.005694in,y=0.232778in,,top]{\color{textcolor}\sffamily\fontsize{10.000000}{12.000000}\selectfont 40}%
\end{pgfscope}%
\begin{pgfscope}%
\pgfsetbuttcap%
\pgfsetroundjoin%
\definecolor{currentfill}{rgb}{0.000000,0.000000,0.000000}%
\pgfsetfillcolor{currentfill}%
\pgfsetlinewidth{0.803000pt}%
\definecolor{currentstroke}{rgb}{0.000000,0.000000,0.000000}%
\pgfsetstrokecolor{currentstroke}%
\pgfsetdash{}{0pt}%
\pgfsys@defobject{currentmarker}{\pgfqpoint{0.000000in}{-0.048611in}}{\pgfqpoint{0.000000in}{0.000000in}}{%
\pgfpathmoveto{\pgfqpoint{0.000000in}{0.000000in}}%
\pgfpathlineto{\pgfqpoint{0.000000in}{-0.048611in}}%
\pgfusepath{stroke,fill}%
}%
\begin{pgfscope}%
\pgfsys@transformshift{1.530548in}{0.330000in}%
\pgfsys@useobject{currentmarker}{}%
\end{pgfscope}%
\end{pgfscope}%
\begin{pgfscope}%
\definecolor{textcolor}{rgb}{0.000000,0.000000,0.000000}%
\pgfsetstrokecolor{textcolor}%
\pgfsetfillcolor{textcolor}%
\pgftext[x=1.530548in,y=0.232778in,,top]{\color{textcolor}\sffamily\fontsize{10.000000}{12.000000}\selectfont 60}%
\end{pgfscope}%
\begin{pgfscope}%
\pgfsetbuttcap%
\pgfsetroundjoin%
\definecolor{currentfill}{rgb}{0.000000,0.000000,0.000000}%
\pgfsetfillcolor{currentfill}%
\pgfsetlinewidth{0.803000pt}%
\definecolor{currentstroke}{rgb}{0.000000,0.000000,0.000000}%
\pgfsetstrokecolor{currentstroke}%
\pgfsetdash{}{0pt}%
\pgfsys@defobject{currentmarker}{\pgfqpoint{0.000000in}{-0.048611in}}{\pgfqpoint{0.000000in}{0.000000in}}{%
\pgfpathmoveto{\pgfqpoint{0.000000in}{0.000000in}}%
\pgfpathlineto{\pgfqpoint{0.000000in}{-0.048611in}}%
\pgfusepath{stroke,fill}%
}%
\begin{pgfscope}%
\pgfsys@transformshift{2.055402in}{0.330000in}%
\pgfsys@useobject{currentmarker}{}%
\end{pgfscope}%
\end{pgfscope}%
\begin{pgfscope}%
\definecolor{textcolor}{rgb}{0.000000,0.000000,0.000000}%
\pgfsetstrokecolor{textcolor}%
\pgfsetfillcolor{textcolor}%
\pgftext[x=2.055402in,y=0.232778in,,top]{\color{textcolor}\sffamily\fontsize{10.000000}{12.000000}\selectfont 80}%
\end{pgfscope}%
\begin{pgfscope}%
\pgfsetbuttcap%
\pgfsetroundjoin%
\definecolor{currentfill}{rgb}{0.000000,0.000000,0.000000}%
\pgfsetfillcolor{currentfill}%
\pgfsetlinewidth{0.803000pt}%
\definecolor{currentstroke}{rgb}{0.000000,0.000000,0.000000}%
\pgfsetstrokecolor{currentstroke}%
\pgfsetdash{}{0pt}%
\pgfsys@defobject{currentmarker}{\pgfqpoint{0.000000in}{-0.048611in}}{\pgfqpoint{0.000000in}{0.000000in}}{%
\pgfpathmoveto{\pgfqpoint{0.000000in}{0.000000in}}%
\pgfpathlineto{\pgfqpoint{0.000000in}{-0.048611in}}%
\pgfusepath{stroke,fill}%
}%
\begin{pgfscope}%
\pgfsys@transformshift{2.580256in}{0.330000in}%
\pgfsys@useobject{currentmarker}{}%
\end{pgfscope}%
\end{pgfscope}%
\begin{pgfscope}%
\definecolor{textcolor}{rgb}{0.000000,0.000000,0.000000}%
\pgfsetstrokecolor{textcolor}%
\pgfsetfillcolor{textcolor}%
\pgftext[x=2.580256in,y=0.232778in,,top]{\color{textcolor}\sffamily\fontsize{10.000000}{12.000000}\selectfont 100}%
\end{pgfscope}%
\begin{pgfscope}%
\definecolor{textcolor}{rgb}{0.000000,0.000000,0.000000}%
\pgfsetstrokecolor{textcolor}%
\pgfsetfillcolor{textcolor}%
\pgftext[x=1.537500in,y=0.042809in,,top]{\color{textcolor}\sffamily\fontsize{10.000000}{12.000000}\selectfont \(\displaystyle k\)}%
\end{pgfscope}%
\begin{pgfscope}%
\pgfsetbuttcap%
\pgfsetroundjoin%
\definecolor{currentfill}{rgb}{0.000000,0.000000,0.000000}%
\pgfsetfillcolor{currentfill}%
\pgfsetlinewidth{0.803000pt}%
\definecolor{currentstroke}{rgb}{0.000000,0.000000,0.000000}%
\pgfsetstrokecolor{currentstroke}%
\pgfsetdash{}{0pt}%
\pgfsys@defobject{currentmarker}{\pgfqpoint{-0.048611in}{0.000000in}}{\pgfqpoint{0.000000in}{0.000000in}}{%
\pgfpathmoveto{\pgfqpoint{0.000000in}{0.000000in}}%
\pgfpathlineto{\pgfqpoint{-0.048611in}{0.000000in}}%
\pgfusepath{stroke,fill}%
}%
\begin{pgfscope}%
\pgfsys@transformshift{0.375000in}{0.444417in}%
\pgfsys@useobject{currentmarker}{}%
\end{pgfscope}%
\end{pgfscope}%
\begin{pgfscope}%
\definecolor{textcolor}{rgb}{0.000000,0.000000,0.000000}%
\pgfsetstrokecolor{textcolor}%
\pgfsetfillcolor{textcolor}%
\pgftext[x=0.189413in,y=0.391656in,left,base]{\color{textcolor}\sffamily\fontsize{10.000000}{12.000000}\selectfont 6}%
\end{pgfscope}%
\begin{pgfscope}%
\pgfsetbuttcap%
\pgfsetroundjoin%
\definecolor{currentfill}{rgb}{0.000000,0.000000,0.000000}%
\pgfsetfillcolor{currentfill}%
\pgfsetlinewidth{0.803000pt}%
\definecolor{currentstroke}{rgb}{0.000000,0.000000,0.000000}%
\pgfsetstrokecolor{currentstroke}%
\pgfsetdash{}{0pt}%
\pgfsys@defobject{currentmarker}{\pgfqpoint{-0.048611in}{0.000000in}}{\pgfqpoint{0.000000in}{0.000000in}}{%
\pgfpathmoveto{\pgfqpoint{0.000000in}{0.000000in}}%
\pgfpathlineto{\pgfqpoint{-0.048611in}{0.000000in}}%
\pgfusepath{stroke,fill}%
}%
\begin{pgfscope}%
\pgfsys@transformshift{0.375000in}{1.480334in}%
\pgfsys@useobject{currentmarker}{}%
\end{pgfscope}%
\end{pgfscope}%
\begin{pgfscope}%
\definecolor{textcolor}{rgb}{0.000000,0.000000,0.000000}%
\pgfsetstrokecolor{textcolor}%
\pgfsetfillcolor{textcolor}%
\pgftext[x=0.189413in,y=1.427572in,left,base]{\color{textcolor}\sffamily\fontsize{10.000000}{12.000000}\selectfont 7}%
\end{pgfscope}%
\begin{pgfscope}%
\pgfsetbuttcap%
\pgfsetroundjoin%
\definecolor{currentfill}{rgb}{0.000000,0.000000,0.000000}%
\pgfsetfillcolor{currentfill}%
\pgfsetlinewidth{0.803000pt}%
\definecolor{currentstroke}{rgb}{0.000000,0.000000,0.000000}%
\pgfsetstrokecolor{currentstroke}%
\pgfsetdash{}{0pt}%
\pgfsys@defobject{currentmarker}{\pgfqpoint{-0.048611in}{0.000000in}}{\pgfqpoint{0.000000in}{0.000000in}}{%
\pgfpathmoveto{\pgfqpoint{0.000000in}{0.000000in}}%
\pgfpathlineto{\pgfqpoint{-0.048611in}{0.000000in}}%
\pgfusepath{stroke,fill}%
}%
\begin{pgfscope}%
\pgfsys@transformshift{0.375000in}{2.516250in}%
\pgfsys@useobject{currentmarker}{}%
\end{pgfscope}%
\end{pgfscope}%
\begin{pgfscope}%
\definecolor{textcolor}{rgb}{0.000000,0.000000,0.000000}%
\pgfsetstrokecolor{textcolor}%
\pgfsetfillcolor{textcolor}%
\pgftext[x=0.189413in,y=2.463488in,left,base]{\color{textcolor}\sffamily\fontsize{10.000000}{12.000000}\selectfont 8}%
\end{pgfscope}%
\begin{pgfscope}%
\definecolor{textcolor}{rgb}{0.000000,0.000000,0.000000}%
\pgfsetstrokecolor{textcolor}%
\pgfsetfillcolor{textcolor}%
\pgftext[x=0.133857in,y=1.485000in,,bottom,rotate=90.000000]{\color{textcolor}\sffamily\fontsize{10.000000}{12.000000}\selectfont Number of GMRES Iterations}%
\end{pgfscope}%
\begin{pgfscope}%
\pgfsetrectcap%
\pgfsetmiterjoin%
\pgfsetlinewidth{0.803000pt}%
\definecolor{currentstroke}{rgb}{0.000000,0.000000,0.000000}%
\pgfsetstrokecolor{currentstroke}%
\pgfsetdash{}{0pt}%
\pgfpathmoveto{\pgfqpoint{0.375000in}{0.330000in}}%
\pgfpathlineto{\pgfqpoint{0.375000in}{2.640000in}}%
\pgfusepath{stroke}%
\end{pgfscope}%
\begin{pgfscope}%
\pgfsetrectcap%
\pgfsetmiterjoin%
\pgfsetlinewidth{0.803000pt}%
\definecolor{currentstroke}{rgb}{0.000000,0.000000,0.000000}%
\pgfsetstrokecolor{currentstroke}%
\pgfsetdash{}{0pt}%
\pgfpathmoveto{\pgfqpoint{2.700000in}{0.330000in}}%
\pgfpathlineto{\pgfqpoint{2.700000in}{2.640000in}}%
\pgfusepath{stroke}%
\end{pgfscope}%
\begin{pgfscope}%
\pgfsetrectcap%
\pgfsetmiterjoin%
\pgfsetlinewidth{0.803000pt}%
\definecolor{currentstroke}{rgb}{0.000000,0.000000,0.000000}%
\pgfsetstrokecolor{currentstroke}%
\pgfsetdash{}{0pt}%
\pgfpathmoveto{\pgfqpoint{0.375000in}{0.330000in}}%
\pgfpathlineto{\pgfqpoint{2.700000in}{0.330000in}}%
\pgfusepath{stroke}%
\end{pgfscope}%
\begin{pgfscope}%
\pgfsetrectcap%
\pgfsetmiterjoin%
\pgfsetlinewidth{0.803000pt}%
\definecolor{currentstroke}{rgb}{0.000000,0.000000,0.000000}%
\pgfsetstrokecolor{currentstroke}%
\pgfsetdash{}{0pt}%
\pgfpathmoveto{\pgfqpoint{0.375000in}{2.640000in}}%
\pgfpathlineto{\pgfqpoint{2.700000in}{2.640000in}}%
\pgfusepath{stroke}%
\end{pgfscope}%
\end{pgfpicture}%
\makeatother%
\endgroup%

   \caption[Maximum GMRES iteration counts when $\NLiDRR{\nso-\nst} = 0.5\times  k^{\beta}$ for $\beta = 0.4,0.5,0.6,0.7.$]{Maximum GMRES iteration counts for solving systems with matrix $\AmatoI\Amatt$, where $\Aso=\Ast=1$ and $\NLiDRR{\nso-\nst} = 0.5\times  k^{\beta}$ for $\beta = 0.4,0.5,0.6,0.7.$}\label{fig:linfinityn1}
\end{figure}

    \begin{figure}
      \centering
%% Creator: Matplotlib, PGF backend
%%
%% To include the figure in your LaTeX document, write
%%   \input{<filename>.pgf}
%%
%% Make sure the required packages are loaded in your preamble
%%   \usepackage{pgf}
%%
%% Figures using additional raster images can only be included by \input if
%% they are in the same directory as the main LaTeX file. For loading figures
%% from other directories you can use the `import` package
%%   \usepackage{import}
%% and then include the figures with
%%   \import{<path to file>}{<filename>.pgf}
%%
%% Matplotlib used the following preamble
%%   \usepackage{fontspec}
%%   \setmainfont{DejaVuSerif.ttf}[Path=/home/owen/progs/firedrake-complex/firedrake/lib/python3.5/site-packages/matplotlib/mpl-data/fonts/ttf/]
%%   \setsansfont{DejaVuSans.ttf}[Path=/home/owen/progs/firedrake-complex/firedrake/lib/python3.5/site-packages/matplotlib/mpl-data/fonts/ttf/]
%%   \setmonofont{DejaVuSansMono.ttf}[Path=/home/owen/progs/firedrake-complex/firedrake/lib/python3.5/site-packages/matplotlib/mpl-data/fonts/ttf/]
%%
\begingroup%
\makeatletter%
\begin{pgfpicture}%
\pgfpathrectangle{\pgfpointorigin}{\pgfqpoint{6.400000in}{4.800000in}}%
\pgfusepath{use as bounding box, clip}%
\begin{pgfscope}%
\pgfsetbuttcap%
\pgfsetmiterjoin%
\definecolor{currentfill}{rgb}{1.000000,1.000000,1.000000}%
\pgfsetfillcolor{currentfill}%
\pgfsetlinewidth{0.000000pt}%
\definecolor{currentstroke}{rgb}{1.000000,1.000000,1.000000}%
\pgfsetstrokecolor{currentstroke}%
\pgfsetdash{}{0pt}%
\pgfpathmoveto{\pgfqpoint{0.000000in}{0.000000in}}%
\pgfpathlineto{\pgfqpoint{6.400000in}{0.000000in}}%
\pgfpathlineto{\pgfqpoint{6.400000in}{4.800000in}}%
\pgfpathlineto{\pgfqpoint{0.000000in}{4.800000in}}%
\pgfpathclose%
\pgfusepath{fill}%
\end{pgfscope}%
\begin{pgfscope}%
\pgfsetbuttcap%
\pgfsetmiterjoin%
\definecolor{currentfill}{rgb}{1.000000,1.000000,1.000000}%
\pgfsetfillcolor{currentfill}%
\pgfsetlinewidth{0.000000pt}%
\definecolor{currentstroke}{rgb}{0.000000,0.000000,0.000000}%
\pgfsetstrokecolor{currentstroke}%
\pgfsetstrokeopacity{0.000000}%
\pgfsetdash{}{0pt}%
\pgfpathmoveto{\pgfqpoint{0.800000in}{0.528000in}}%
\pgfpathlineto{\pgfqpoint{5.760000in}{0.528000in}}%
\pgfpathlineto{\pgfqpoint{5.760000in}{4.224000in}}%
\pgfpathlineto{\pgfqpoint{0.800000in}{4.224000in}}%
\pgfpathclose%
\pgfusepath{fill}%
\end{pgfscope}%
\begin{pgfscope}%
\pgfsetbuttcap%
\pgfsetroundjoin%
\definecolor{currentfill}{rgb}{0.000000,0.000000,0.000000}%
\pgfsetfillcolor{currentfill}%
\pgfsetlinewidth{0.803000pt}%
\definecolor{currentstroke}{rgb}{0.000000,0.000000,0.000000}%
\pgfsetstrokecolor{currentstroke}%
\pgfsetdash{}{0pt}%
\pgfsys@defobject{currentmarker}{\pgfqpoint{0.000000in}{-0.048611in}}{\pgfqpoint{0.000000in}{0.000000in}}{%
\pgfpathmoveto{\pgfqpoint{0.000000in}{0.000000in}}%
\pgfpathlineto{\pgfqpoint{0.000000in}{-0.048611in}}%
\pgfusepath{stroke,fill}%
}%
\begin{pgfscope}%
\pgfsys@transformshift{1.025455in}{0.528000in}%
\pgfsys@useobject{currentmarker}{}%
\end{pgfscope}%
\end{pgfscope}%
\begin{pgfscope}%
\definecolor{textcolor}{rgb}{0.000000,0.000000,0.000000}%
\pgfsetstrokecolor{textcolor}%
\pgfsetfillcolor{textcolor}%
\pgftext[x=1.025455in,y=0.430778in,,top]{\color{textcolor}\sffamily\fontsize{10.000000}{12.000000}\selectfont 20}%
\end{pgfscope}%
\begin{pgfscope}%
\pgfsetbuttcap%
\pgfsetroundjoin%
\definecolor{currentfill}{rgb}{0.000000,0.000000,0.000000}%
\pgfsetfillcolor{currentfill}%
\pgfsetlinewidth{0.803000pt}%
\definecolor{currentstroke}{rgb}{0.000000,0.000000,0.000000}%
\pgfsetstrokecolor{currentstroke}%
\pgfsetdash{}{0pt}%
\pgfsys@defobject{currentmarker}{\pgfqpoint{0.000000in}{-0.048611in}}{\pgfqpoint{0.000000in}{0.000000in}}{%
\pgfpathmoveto{\pgfqpoint{0.000000in}{0.000000in}}%
\pgfpathlineto{\pgfqpoint{0.000000in}{-0.048611in}}%
\pgfusepath{stroke,fill}%
}%
\begin{pgfscope}%
\pgfsys@transformshift{2.528485in}{0.528000in}%
\pgfsys@useobject{currentmarker}{}%
\end{pgfscope}%
\end{pgfscope}%
\begin{pgfscope}%
\definecolor{textcolor}{rgb}{0.000000,0.000000,0.000000}%
\pgfsetstrokecolor{textcolor}%
\pgfsetfillcolor{textcolor}%
\pgftext[x=2.528485in,y=0.430778in,,top]{\color{textcolor}\sffamily\fontsize{10.000000}{12.000000}\selectfont 40}%
\end{pgfscope}%
\begin{pgfscope}%
\pgfsetbuttcap%
\pgfsetroundjoin%
\definecolor{currentfill}{rgb}{0.000000,0.000000,0.000000}%
\pgfsetfillcolor{currentfill}%
\pgfsetlinewidth{0.803000pt}%
\definecolor{currentstroke}{rgb}{0.000000,0.000000,0.000000}%
\pgfsetstrokecolor{currentstroke}%
\pgfsetdash{}{0pt}%
\pgfsys@defobject{currentmarker}{\pgfqpoint{0.000000in}{-0.048611in}}{\pgfqpoint{0.000000in}{0.000000in}}{%
\pgfpathmoveto{\pgfqpoint{0.000000in}{0.000000in}}%
\pgfpathlineto{\pgfqpoint{0.000000in}{-0.048611in}}%
\pgfusepath{stroke,fill}%
}%
\begin{pgfscope}%
\pgfsys@transformshift{4.031515in}{0.528000in}%
\pgfsys@useobject{currentmarker}{}%
\end{pgfscope}%
\end{pgfscope}%
\begin{pgfscope}%
\definecolor{textcolor}{rgb}{0.000000,0.000000,0.000000}%
\pgfsetstrokecolor{textcolor}%
\pgfsetfillcolor{textcolor}%
\pgftext[x=4.031515in,y=0.430778in,,top]{\color{textcolor}\sffamily\fontsize{10.000000}{12.000000}\selectfont 60}%
\end{pgfscope}%
\begin{pgfscope}%
\pgfsetbuttcap%
\pgfsetroundjoin%
\definecolor{currentfill}{rgb}{0.000000,0.000000,0.000000}%
\pgfsetfillcolor{currentfill}%
\pgfsetlinewidth{0.803000pt}%
\definecolor{currentstroke}{rgb}{0.000000,0.000000,0.000000}%
\pgfsetstrokecolor{currentstroke}%
\pgfsetdash{}{0pt}%
\pgfsys@defobject{currentmarker}{\pgfqpoint{0.000000in}{-0.048611in}}{\pgfqpoint{0.000000in}{0.000000in}}{%
\pgfpathmoveto{\pgfqpoint{0.000000in}{0.000000in}}%
\pgfpathlineto{\pgfqpoint{0.000000in}{-0.048611in}}%
\pgfusepath{stroke,fill}%
}%
\begin{pgfscope}%
\pgfsys@transformshift{5.534545in}{0.528000in}%
\pgfsys@useobject{currentmarker}{}%
\end{pgfscope}%
\end{pgfscope}%
\begin{pgfscope}%
\definecolor{textcolor}{rgb}{0.000000,0.000000,0.000000}%
\pgfsetstrokecolor{textcolor}%
\pgfsetfillcolor{textcolor}%
\pgftext[x=5.534545in,y=0.430778in,,top]{\color{textcolor}\sffamily\fontsize{10.000000}{12.000000}\selectfont 80}%
\end{pgfscope}%
\begin{pgfscope}%
\definecolor{textcolor}{rgb}{0.000000,0.000000,0.000000}%
\pgfsetstrokecolor{textcolor}%
\pgfsetfillcolor{textcolor}%
\pgftext[x=3.280000in,y=0.240809in,,top]{\color{textcolor}\sffamily\fontsize{10.000000}{12.000000}\selectfont \(\displaystyle k\)}%
\end{pgfscope}%
\begin{pgfscope}%
\pgfsetbuttcap%
\pgfsetroundjoin%
\definecolor{currentfill}{rgb}{0.000000,0.000000,0.000000}%
\pgfsetfillcolor{currentfill}%
\pgfsetlinewidth{0.803000pt}%
\definecolor{currentstroke}{rgb}{0.000000,0.000000,0.000000}%
\pgfsetstrokecolor{currentstroke}%
\pgfsetdash{}{0pt}%
\pgfsys@defobject{currentmarker}{\pgfqpoint{-0.048611in}{0.000000in}}{\pgfqpoint{0.000000in}{0.000000in}}{%
\pgfpathmoveto{\pgfqpoint{0.000000in}{0.000000in}}%
\pgfpathlineto{\pgfqpoint{-0.048611in}{0.000000in}}%
\pgfusepath{stroke,fill}%
}%
\begin{pgfscope}%
\pgfsys@transformshift{0.800000in}{0.696000in}%
\pgfsys@useobject{currentmarker}{}%
\end{pgfscope}%
\end{pgfscope}%
\begin{pgfscope}%
\definecolor{textcolor}{rgb}{0.000000,0.000000,0.000000}%
\pgfsetstrokecolor{textcolor}%
\pgfsetfillcolor{textcolor}%
\pgftext[x=0.614413in,y=0.643238in,left,base]{\color{textcolor}\sffamily\fontsize{10.000000}{12.000000}\selectfont 5}%
\end{pgfscope}%
\begin{pgfscope}%
\pgfsetbuttcap%
\pgfsetroundjoin%
\definecolor{currentfill}{rgb}{0.000000,0.000000,0.000000}%
\pgfsetfillcolor{currentfill}%
\pgfsetlinewidth{0.803000pt}%
\definecolor{currentstroke}{rgb}{0.000000,0.000000,0.000000}%
\pgfsetstrokecolor{currentstroke}%
\pgfsetdash{}{0pt}%
\pgfsys@defobject{currentmarker}{\pgfqpoint{-0.048611in}{0.000000in}}{\pgfqpoint{0.000000in}{0.000000in}}{%
\pgfpathmoveto{\pgfqpoint{0.000000in}{0.000000in}}%
\pgfpathlineto{\pgfqpoint{-0.048611in}{0.000000in}}%
\pgfusepath{stroke,fill}%
}%
\begin{pgfscope}%
\pgfsys@transformshift{0.800000in}{2.376000in}%
\pgfsys@useobject{currentmarker}{}%
\end{pgfscope}%
\end{pgfscope}%
\begin{pgfscope}%
\definecolor{textcolor}{rgb}{0.000000,0.000000,0.000000}%
\pgfsetstrokecolor{textcolor}%
\pgfsetfillcolor{textcolor}%
\pgftext[x=0.614413in,y=2.323238in,left,base]{\color{textcolor}\sffamily\fontsize{10.000000}{12.000000}\selectfont 6}%
\end{pgfscope}%
\begin{pgfscope}%
\pgfsetbuttcap%
\pgfsetroundjoin%
\definecolor{currentfill}{rgb}{0.000000,0.000000,0.000000}%
\pgfsetfillcolor{currentfill}%
\pgfsetlinewidth{0.803000pt}%
\definecolor{currentstroke}{rgb}{0.000000,0.000000,0.000000}%
\pgfsetstrokecolor{currentstroke}%
\pgfsetdash{}{0pt}%
\pgfsys@defobject{currentmarker}{\pgfqpoint{-0.048611in}{0.000000in}}{\pgfqpoint{0.000000in}{0.000000in}}{%
\pgfpathmoveto{\pgfqpoint{0.000000in}{0.000000in}}%
\pgfpathlineto{\pgfqpoint{-0.048611in}{0.000000in}}%
\pgfusepath{stroke,fill}%
}%
\begin{pgfscope}%
\pgfsys@transformshift{0.800000in}{4.056000in}%
\pgfsys@useobject{currentmarker}{}%
\end{pgfscope}%
\end{pgfscope}%
\begin{pgfscope}%
\definecolor{textcolor}{rgb}{0.000000,0.000000,0.000000}%
\pgfsetstrokecolor{textcolor}%
\pgfsetfillcolor{textcolor}%
\pgftext[x=0.614413in,y=4.003238in,left,base]{\color{textcolor}\sffamily\fontsize{10.000000}{12.000000}\selectfont 7}%
\end{pgfscope}%
\begin{pgfscope}%
\definecolor{textcolor}{rgb}{0.000000,0.000000,0.000000}%
\pgfsetstrokecolor{textcolor}%
\pgfsetfillcolor{textcolor}%
\pgftext[x=0.558857in,y=2.376000in,,bottom,rotate=90.000000]{\color{textcolor}\sffamily\fontsize{10.000000}{12.000000}\selectfont Number of GMRES Iterations}%
\end{pgfscope}%
\begin{pgfscope}%
\pgfpathrectangle{\pgfqpoint{0.800000in}{0.528000in}}{\pgfqpoint{4.960000in}{3.696000in}}%
\pgfusepath{clip}%
\pgfsetbuttcap%
\pgfsetroundjoin%
\pgfsetlinewidth{1.505625pt}%
\definecolor{currentstroke}{rgb}{0.000000,0.000000,0.000000}%
\pgfsetstrokecolor{currentstroke}%
\pgfsetdash{{5.550000pt}{2.400000pt}}{0.000000pt}%
\pgfpathmoveto{\pgfqpoint{1.025455in}{4.056000in}}%
\pgfpathlineto{\pgfqpoint{2.528485in}{4.056000in}}%
\pgfpathlineto{\pgfqpoint{4.031515in}{4.056000in}}%
\pgfpathlineto{\pgfqpoint{5.534545in}{2.376000in}}%
\pgfusepath{stroke}%
\end{pgfscope}%
\begin{pgfscope}%
\pgfpathrectangle{\pgfqpoint{0.800000in}{0.528000in}}{\pgfqpoint{4.960000in}{3.696000in}}%
\pgfusepath{clip}%
\pgfsetbuttcap%
\pgfsetroundjoin%
\definecolor{currentfill}{rgb}{0.000000,0.000000,0.000000}%
\pgfsetfillcolor{currentfill}%
\pgfsetlinewidth{1.003750pt}%
\definecolor{currentstroke}{rgb}{0.000000,0.000000,0.000000}%
\pgfsetstrokecolor{currentstroke}%
\pgfsetdash{}{0pt}%
\pgfsys@defobject{currentmarker}{\pgfqpoint{-0.041667in}{-0.041667in}}{\pgfqpoint{0.041667in}{0.041667in}}{%
\pgfpathmoveto{\pgfqpoint{0.000000in}{-0.041667in}}%
\pgfpathcurveto{\pgfqpoint{0.011050in}{-0.041667in}}{\pgfqpoint{0.021649in}{-0.037276in}}{\pgfqpoint{0.029463in}{-0.029463in}}%
\pgfpathcurveto{\pgfqpoint{0.037276in}{-0.021649in}}{\pgfqpoint{0.041667in}{-0.011050in}}{\pgfqpoint{0.041667in}{0.000000in}}%
\pgfpathcurveto{\pgfqpoint{0.041667in}{0.011050in}}{\pgfqpoint{0.037276in}{0.021649in}}{\pgfqpoint{0.029463in}{0.029463in}}%
\pgfpathcurveto{\pgfqpoint{0.021649in}{0.037276in}}{\pgfqpoint{0.011050in}{0.041667in}}{\pgfqpoint{0.000000in}{0.041667in}}%
\pgfpathcurveto{\pgfqpoint{-0.011050in}{0.041667in}}{\pgfqpoint{-0.021649in}{0.037276in}}{\pgfqpoint{-0.029463in}{0.029463in}}%
\pgfpathcurveto{\pgfqpoint{-0.037276in}{0.021649in}}{\pgfqpoint{-0.041667in}{0.011050in}}{\pgfqpoint{-0.041667in}{0.000000in}}%
\pgfpathcurveto{\pgfqpoint{-0.041667in}{-0.011050in}}{\pgfqpoint{-0.037276in}{-0.021649in}}{\pgfqpoint{-0.029463in}{-0.029463in}}%
\pgfpathcurveto{\pgfqpoint{-0.021649in}{-0.037276in}}{\pgfqpoint{-0.011050in}{-0.041667in}}{\pgfqpoint{0.000000in}{-0.041667in}}%
\pgfpathclose%
\pgfusepath{stroke,fill}%
}%
\begin{pgfscope}%
\pgfsys@transformshift{1.025455in}{4.056000in}%
\pgfsys@useobject{currentmarker}{}%
\end{pgfscope}%
\begin{pgfscope}%
\pgfsys@transformshift{2.528485in}{4.056000in}%
\pgfsys@useobject{currentmarker}{}%
\end{pgfscope}%
\begin{pgfscope}%
\pgfsys@transformshift{4.031515in}{4.056000in}%
\pgfsys@useobject{currentmarker}{}%
\end{pgfscope}%
\begin{pgfscope}%
\pgfsys@transformshift{5.534545in}{2.376000in}%
\pgfsys@useobject{currentmarker}{}%
\end{pgfscope}%
\end{pgfscope}%
\begin{pgfscope}%
\pgfpathrectangle{\pgfqpoint{0.800000in}{0.528000in}}{\pgfqpoint{4.960000in}{3.696000in}}%
\pgfusepath{clip}%
\pgfsetbuttcap%
\pgfsetroundjoin%
\pgfsetlinewidth{1.505625pt}%
\definecolor{currentstroke}{rgb}{0.000000,0.000000,0.000000}%
\pgfsetstrokecolor{currentstroke}%
\pgfsetdash{{5.550000pt}{2.400000pt}}{0.000000pt}%
\pgfpathmoveto{\pgfqpoint{1.025455in}{2.376000in}}%
\pgfpathlineto{\pgfqpoint{2.528485in}{2.376000in}}%
\pgfpathlineto{\pgfqpoint{4.031515in}{2.376000in}}%
\pgfpathlineto{\pgfqpoint{5.534545in}{2.376000in}}%
\pgfusepath{stroke}%
\end{pgfscope}%
\begin{pgfscope}%
\pgfpathrectangle{\pgfqpoint{0.800000in}{0.528000in}}{\pgfqpoint{4.960000in}{3.696000in}}%
\pgfusepath{clip}%
\pgfsetbuttcap%
\pgfsetmiterjoin%
\definecolor{currentfill}{rgb}{0.000000,0.000000,0.000000}%
\pgfsetfillcolor{currentfill}%
\pgfsetlinewidth{1.003750pt}%
\definecolor{currentstroke}{rgb}{0.000000,0.000000,0.000000}%
\pgfsetstrokecolor{currentstroke}%
\pgfsetdash{}{0pt}%
\pgfsys@defobject{currentmarker}{\pgfqpoint{-0.041667in}{-0.041667in}}{\pgfqpoint{0.041667in}{0.041667in}}{%
\pgfpathmoveto{\pgfqpoint{-0.000000in}{-0.041667in}}%
\pgfpathlineto{\pgfqpoint{0.041667in}{0.041667in}}%
\pgfpathlineto{\pgfqpoint{-0.041667in}{0.041667in}}%
\pgfpathclose%
\pgfusepath{stroke,fill}%
}%
\begin{pgfscope}%
\pgfsys@transformshift{1.025455in}{2.376000in}%
\pgfsys@useobject{currentmarker}{}%
\end{pgfscope}%
\begin{pgfscope}%
\pgfsys@transformshift{2.528485in}{2.376000in}%
\pgfsys@useobject{currentmarker}{}%
\end{pgfscope}%
\begin{pgfscope}%
\pgfsys@transformshift{4.031515in}{2.376000in}%
\pgfsys@useobject{currentmarker}{}%
\end{pgfscope}%
\begin{pgfscope}%
\pgfsys@transformshift{5.534545in}{2.376000in}%
\pgfsys@useobject{currentmarker}{}%
\end{pgfscope}%
\end{pgfscope}%
\begin{pgfscope}%
\pgfpathrectangle{\pgfqpoint{0.800000in}{0.528000in}}{\pgfqpoint{4.960000in}{3.696000in}}%
\pgfusepath{clip}%
\pgfsetbuttcap%
\pgfsetroundjoin%
\pgfsetlinewidth{1.505625pt}%
\definecolor{currentstroke}{rgb}{0.000000,0.000000,0.000000}%
\pgfsetstrokecolor{currentstroke}%
\pgfsetdash{{5.550000pt}{2.400000pt}}{0.000000pt}%
\pgfpathmoveto{\pgfqpoint{1.025455in}{2.376000in}}%
\pgfpathlineto{\pgfqpoint{2.528485in}{0.696000in}}%
\pgfpathlineto{\pgfqpoint{4.031515in}{0.696000in}}%
\pgfpathlineto{\pgfqpoint{5.534545in}{0.696000in}}%
\pgfusepath{stroke}%
\end{pgfscope}%
\begin{pgfscope}%
\pgfpathrectangle{\pgfqpoint{0.800000in}{0.528000in}}{\pgfqpoint{4.960000in}{3.696000in}}%
\pgfusepath{clip}%
\pgfsetbuttcap%
\pgfsetmiterjoin%
\definecolor{currentfill}{rgb}{0.000000,0.000000,0.000000}%
\pgfsetfillcolor{currentfill}%
\pgfsetlinewidth{1.003750pt}%
\definecolor{currentstroke}{rgb}{0.000000,0.000000,0.000000}%
\pgfsetstrokecolor{currentstroke}%
\pgfsetdash{}{0pt}%
\pgfsys@defobject{currentmarker}{\pgfqpoint{-0.041667in}{-0.041667in}}{\pgfqpoint{0.041667in}{0.041667in}}{%
\pgfpathmoveto{\pgfqpoint{-0.020833in}{-0.041667in}}%
\pgfpathlineto{\pgfqpoint{0.000000in}{-0.020833in}}%
\pgfpathlineto{\pgfqpoint{0.020833in}{-0.041667in}}%
\pgfpathlineto{\pgfqpoint{0.041667in}{-0.020833in}}%
\pgfpathlineto{\pgfqpoint{0.020833in}{0.000000in}}%
\pgfpathlineto{\pgfqpoint{0.041667in}{0.020833in}}%
\pgfpathlineto{\pgfqpoint{0.020833in}{0.041667in}}%
\pgfpathlineto{\pgfqpoint{0.000000in}{0.020833in}}%
\pgfpathlineto{\pgfqpoint{-0.020833in}{0.041667in}}%
\pgfpathlineto{\pgfqpoint{-0.041667in}{0.020833in}}%
\pgfpathlineto{\pgfqpoint{-0.020833in}{0.000000in}}%
\pgfpathlineto{\pgfqpoint{-0.041667in}{-0.020833in}}%
\pgfpathclose%
\pgfusepath{stroke,fill}%
}%
\begin{pgfscope}%
\pgfsys@transformshift{1.025455in}{2.376000in}%
\pgfsys@useobject{currentmarker}{}%
\end{pgfscope}%
\begin{pgfscope}%
\pgfsys@transformshift{2.528485in}{0.696000in}%
\pgfsys@useobject{currentmarker}{}%
\end{pgfscope}%
\begin{pgfscope}%
\pgfsys@transformshift{4.031515in}{0.696000in}%
\pgfsys@useobject{currentmarker}{}%
\end{pgfscope}%
\begin{pgfscope}%
\pgfsys@transformshift{5.534545in}{0.696000in}%
\pgfsys@useobject{currentmarker}{}%
\end{pgfscope}%
\end{pgfscope}%
\begin{pgfscope}%
\pgfsetrectcap%
\pgfsetmiterjoin%
\pgfsetlinewidth{0.803000pt}%
\definecolor{currentstroke}{rgb}{0.000000,0.000000,0.000000}%
\pgfsetstrokecolor{currentstroke}%
\pgfsetdash{}{0pt}%
\pgfpathmoveto{\pgfqpoint{0.800000in}{0.528000in}}%
\pgfpathlineto{\pgfqpoint{0.800000in}{4.224000in}}%
\pgfusepath{stroke}%
\end{pgfscope}%
\begin{pgfscope}%
\pgfsetrectcap%
\pgfsetmiterjoin%
\pgfsetlinewidth{0.803000pt}%
\definecolor{currentstroke}{rgb}{0.000000,0.000000,0.000000}%
\pgfsetstrokecolor{currentstroke}%
\pgfsetdash{}{0pt}%
\pgfpathmoveto{\pgfqpoint{5.760000in}{0.528000in}}%
\pgfpathlineto{\pgfqpoint{5.760000in}{4.224000in}}%
\pgfusepath{stroke}%
\end{pgfscope}%
\begin{pgfscope}%
\pgfsetrectcap%
\pgfsetmiterjoin%
\pgfsetlinewidth{0.803000pt}%
\definecolor{currentstroke}{rgb}{0.000000,0.000000,0.000000}%
\pgfsetstrokecolor{currentstroke}%
\pgfsetdash{}{0pt}%
\pgfpathmoveto{\pgfqpoint{0.800000in}{0.528000in}}%
\pgfpathlineto{\pgfqpoint{5.760000in}{0.528000in}}%
\pgfusepath{stroke}%
\end{pgfscope}%
\begin{pgfscope}%
\pgfsetrectcap%
\pgfsetmiterjoin%
\pgfsetlinewidth{0.803000pt}%
\definecolor{currentstroke}{rgb}{0.000000,0.000000,0.000000}%
\pgfsetstrokecolor{currentstroke}%
\pgfsetdash{}{0pt}%
\pgfpathmoveto{\pgfqpoint{0.800000in}{4.224000in}}%
\pgfpathlineto{\pgfqpoint{5.760000in}{4.224000in}}%
\pgfusepath{stroke}%
\end{pgfscope}%
\begin{pgfscope}%
\pgfsetbuttcap%
\pgfsetmiterjoin%
\definecolor{currentfill}{rgb}{1.000000,1.000000,1.000000}%
\pgfsetfillcolor{currentfill}%
\pgfsetfillopacity{0.800000}%
\pgfsetlinewidth{1.003750pt}%
\definecolor{currentstroke}{rgb}{0.800000,0.800000,0.800000}%
\pgfsetstrokecolor{currentstroke}%
\pgfsetstrokeopacity{0.800000}%
\pgfsetdash{}{0pt}%
\pgfpathmoveto{\pgfqpoint{4.764960in}{3.501317in}}%
\pgfpathlineto{\pgfqpoint{5.662778in}{3.501317in}}%
\pgfpathquadraticcurveto{\pgfqpoint{5.690556in}{3.501317in}}{\pgfqpoint{5.690556in}{3.529095in}}%
\pgfpathlineto{\pgfqpoint{5.690556in}{4.126778in}}%
\pgfpathquadraticcurveto{\pgfqpoint{5.690556in}{4.154556in}}{\pgfqpoint{5.662778in}{4.154556in}}%
\pgfpathlineto{\pgfqpoint{4.764960in}{4.154556in}}%
\pgfpathquadraticcurveto{\pgfqpoint{4.737183in}{4.154556in}}{\pgfqpoint{4.737183in}{4.126778in}}%
\pgfpathlineto{\pgfqpoint{4.737183in}{3.529095in}}%
\pgfpathquadraticcurveto{\pgfqpoint{4.737183in}{3.501317in}}{\pgfqpoint{4.764960in}{3.501317in}}%
\pgfpathclose%
\pgfusepath{stroke,fill}%
\end{pgfscope}%
\begin{pgfscope}%
\pgfsetbuttcap%
\pgfsetroundjoin%
\pgfsetlinewidth{1.505625pt}%
\definecolor{currentstroke}{rgb}{0.000000,0.000000,0.000000}%
\pgfsetstrokecolor{currentstroke}%
\pgfsetdash{{5.550000pt}{2.400000pt}}{0.000000pt}%
\pgfpathmoveto{\pgfqpoint{4.792738in}{4.042088in}}%
\pgfpathlineto{\pgfqpoint{5.070516in}{4.042088in}}%
\pgfusepath{stroke}%
\end{pgfscope}%
\begin{pgfscope}%
\pgfsetbuttcap%
\pgfsetroundjoin%
\definecolor{currentfill}{rgb}{0.000000,0.000000,0.000000}%
\pgfsetfillcolor{currentfill}%
\pgfsetlinewidth{1.003750pt}%
\definecolor{currentstroke}{rgb}{0.000000,0.000000,0.000000}%
\pgfsetstrokecolor{currentstroke}%
\pgfsetdash{}{0pt}%
\pgfsys@defobject{currentmarker}{\pgfqpoint{-0.041667in}{-0.041667in}}{\pgfqpoint{0.041667in}{0.041667in}}{%
\pgfpathmoveto{\pgfqpoint{0.000000in}{-0.041667in}}%
\pgfpathcurveto{\pgfqpoint{0.011050in}{-0.041667in}}{\pgfqpoint{0.021649in}{-0.037276in}}{\pgfqpoint{0.029463in}{-0.029463in}}%
\pgfpathcurveto{\pgfqpoint{0.037276in}{-0.021649in}}{\pgfqpoint{0.041667in}{-0.011050in}}{\pgfqpoint{0.041667in}{0.000000in}}%
\pgfpathcurveto{\pgfqpoint{0.041667in}{0.011050in}}{\pgfqpoint{0.037276in}{0.021649in}}{\pgfqpoint{0.029463in}{0.029463in}}%
\pgfpathcurveto{\pgfqpoint{0.021649in}{0.037276in}}{\pgfqpoint{0.011050in}{0.041667in}}{\pgfqpoint{0.000000in}{0.041667in}}%
\pgfpathcurveto{\pgfqpoint{-0.011050in}{0.041667in}}{\pgfqpoint{-0.021649in}{0.037276in}}{\pgfqpoint{-0.029463in}{0.029463in}}%
\pgfpathcurveto{\pgfqpoint{-0.037276in}{0.021649in}}{\pgfqpoint{-0.041667in}{0.011050in}}{\pgfqpoint{-0.041667in}{0.000000in}}%
\pgfpathcurveto{\pgfqpoint{-0.041667in}{-0.011050in}}{\pgfqpoint{-0.037276in}{-0.021649in}}{\pgfqpoint{-0.029463in}{-0.029463in}}%
\pgfpathcurveto{\pgfqpoint{-0.021649in}{-0.037276in}}{\pgfqpoint{-0.011050in}{-0.041667in}}{\pgfqpoint{0.000000in}{-0.041667in}}%
\pgfpathclose%
\pgfusepath{stroke,fill}%
}%
\begin{pgfscope}%
\pgfsys@transformshift{4.931627in}{4.042088in}%
\pgfsys@useobject{currentmarker}{}%
\end{pgfscope}%
\end{pgfscope}%
\begin{pgfscope}%
\definecolor{textcolor}{rgb}{0.000000,0.000000,0.000000}%
\pgfsetstrokecolor{textcolor}%
\pgfsetfillcolor{textcolor}%
\pgftext[x=5.181627in,y=3.993477in,left,base]{\color{textcolor}\sffamily\fontsize{10.000000}{12.000000}\selectfont \(\displaystyle \beta = \)0.8}%
\end{pgfscope}%
\begin{pgfscope}%
\pgfsetbuttcap%
\pgfsetroundjoin%
\pgfsetlinewidth{1.505625pt}%
\definecolor{currentstroke}{rgb}{0.000000,0.000000,0.000000}%
\pgfsetstrokecolor{currentstroke}%
\pgfsetdash{{5.550000pt}{2.400000pt}}{0.000000pt}%
\pgfpathmoveto{\pgfqpoint{4.792738in}{3.838231in}}%
\pgfpathlineto{\pgfqpoint{5.070516in}{3.838231in}}%
\pgfusepath{stroke}%
\end{pgfscope}%
\begin{pgfscope}%
\pgfsetbuttcap%
\pgfsetmiterjoin%
\definecolor{currentfill}{rgb}{0.000000,0.000000,0.000000}%
\pgfsetfillcolor{currentfill}%
\pgfsetlinewidth{1.003750pt}%
\definecolor{currentstroke}{rgb}{0.000000,0.000000,0.000000}%
\pgfsetstrokecolor{currentstroke}%
\pgfsetdash{}{0pt}%
\pgfsys@defobject{currentmarker}{\pgfqpoint{-0.041667in}{-0.041667in}}{\pgfqpoint{0.041667in}{0.041667in}}{%
\pgfpathmoveto{\pgfqpoint{-0.000000in}{-0.041667in}}%
\pgfpathlineto{\pgfqpoint{0.041667in}{0.041667in}}%
\pgfpathlineto{\pgfqpoint{-0.041667in}{0.041667in}}%
\pgfpathclose%
\pgfusepath{stroke,fill}%
}%
\begin{pgfscope}%
\pgfsys@transformshift{4.931627in}{3.838231in}%
\pgfsys@useobject{currentmarker}{}%
\end{pgfscope}%
\end{pgfscope}%
\begin{pgfscope}%
\definecolor{textcolor}{rgb}{0.000000,0.000000,0.000000}%
\pgfsetstrokecolor{textcolor}%
\pgfsetfillcolor{textcolor}%
\pgftext[x=5.181627in,y=3.789620in,left,base]{\color{textcolor}\sffamily\fontsize{10.000000}{12.000000}\selectfont \(\displaystyle \beta = \)0.9}%
\end{pgfscope}%
\begin{pgfscope}%
\pgfsetbuttcap%
\pgfsetroundjoin%
\pgfsetlinewidth{1.505625pt}%
\definecolor{currentstroke}{rgb}{0.000000,0.000000,0.000000}%
\pgfsetstrokecolor{currentstroke}%
\pgfsetdash{{5.550000pt}{2.400000pt}}{0.000000pt}%
\pgfpathmoveto{\pgfqpoint{4.792738in}{3.634374in}}%
\pgfpathlineto{\pgfqpoint{5.070516in}{3.634374in}}%
\pgfusepath{stroke}%
\end{pgfscope}%
\begin{pgfscope}%
\pgfsetbuttcap%
\pgfsetmiterjoin%
\definecolor{currentfill}{rgb}{0.000000,0.000000,0.000000}%
\pgfsetfillcolor{currentfill}%
\pgfsetlinewidth{1.003750pt}%
\definecolor{currentstroke}{rgb}{0.000000,0.000000,0.000000}%
\pgfsetstrokecolor{currentstroke}%
\pgfsetdash{}{0pt}%
\pgfsys@defobject{currentmarker}{\pgfqpoint{-0.041667in}{-0.041667in}}{\pgfqpoint{0.041667in}{0.041667in}}{%
\pgfpathmoveto{\pgfqpoint{-0.020833in}{-0.041667in}}%
\pgfpathlineto{\pgfqpoint{0.000000in}{-0.020833in}}%
\pgfpathlineto{\pgfqpoint{0.020833in}{-0.041667in}}%
\pgfpathlineto{\pgfqpoint{0.041667in}{-0.020833in}}%
\pgfpathlineto{\pgfqpoint{0.020833in}{0.000000in}}%
\pgfpathlineto{\pgfqpoint{0.041667in}{0.020833in}}%
\pgfpathlineto{\pgfqpoint{0.020833in}{0.041667in}}%
\pgfpathlineto{\pgfqpoint{0.000000in}{0.020833in}}%
\pgfpathlineto{\pgfqpoint{-0.020833in}{0.041667in}}%
\pgfpathlineto{\pgfqpoint{-0.041667in}{0.020833in}}%
\pgfpathlineto{\pgfqpoint{-0.020833in}{0.000000in}}%
\pgfpathlineto{\pgfqpoint{-0.041667in}{-0.020833in}}%
\pgfpathclose%
\pgfusepath{stroke,fill}%
}%
\begin{pgfscope}%
\pgfsys@transformshift{4.931627in}{3.634374in}%
\pgfsys@useobject{currentmarker}{}%
\end{pgfscope}%
\end{pgfscope}%
\begin{pgfscope}%
\definecolor{textcolor}{rgb}{0.000000,0.000000,0.000000}%
\pgfsetstrokecolor{textcolor}%
\pgfsetfillcolor{textcolor}%
\pgftext[x=5.181627in,y=3.585763in,left,base]{\color{textcolor}\sffamily\fontsize{10.000000}{12.000000}\selectfont \(\displaystyle \beta = \)1}%
\end{pgfscope}%
\end{pgfpicture}%
\makeatother%
\endgroup%

   \caption[Maximum GMRES iteration counts when $\NLiDRR{\nso-\nst} = 0.5\times  k^{\beta}$ for $\beta = 0.8,0.9,1.$]{Maximum GMRES iteration counts for solving systems with matrix $\AmatoI\Amatt$, where $\Aso=\Ast=1$ and $\NLiDRR{\nso-\nst} = 0.5\times  k^{\beta}$ for $\beta = 0.8,0.9,1.$}\label{fig:linfinityn2}
\end{figure}
  
%  \ednote{Euan---I had to redo these computations, as there was a bug in my code, so we now don't see \emph{really} rapid growth in the number of GMRES iterations as $k$ inreases. I think the previous code may have may the variations in $n$ too large, meaning we got really rapid growth.}


%Say that these are for the TEDP defined in \cref{def:TEDP}.

\section[Proofs of Theorems \MakeLowercase{\ref{cor:1}, \ref{cor:1a}, and \ref{thm:2}} and Lemma \MakeLowercase{\ref{lem:sharp}}]{Proofs of \cref{cor:1,cor:1a,thm:2,lem:sharp}}\label{sec:3}

We now state the necessary technical definitions to prove \cref{cor:1,cor:1a} above.

\subsection{Definition of variational problems and conditions used to prove main results}\label{sec:vpGm}
As this \lcnamecref{chap:nbpc} concerns finite-element discretisations of the Helmholtz equation, we will work with the variational formulation of the Helmholtz equation. However, because the arguments we use do not directly rely on the boundary condition used to truncate the computational domain, we will state our Helmholtz problems in sufficient generality to include both the EDP (\cref{prob:vedp}) and TEDP (\cref{prob:vtedp}) above.

\bprob[General variational Helmholtz problem]\label{prob:vgen}
Let $D, A,$ and $n$ be as in \cref{prob:tedp}. We say $u \in \HozDD$ satisfies the \defn{variational formulation of the truncated exterior Dirichlet problem} with $\gD = 0$ if
\beq\label{eq:vgen}
\aG(u,v) = \LG(v) \tfa v \in \HozDD,
\eeq
where
\beq\label{eq:agen}
\aG(w,v) \de \int_{D} \mleft(\mleft(A \grad w\mright)\cdot\grad \vbar - k^2 n\minispace w \vbar\mright) - \DPGI{\T \trGI w}{\trGI v},
\eeq
$\T:\HhGI\rightarrow \HmhGI$ is a bounded linear map,  $\DPGI{\cdot}{\cdot}$ is the duality pairing on $\GI,$ and $\LG  \in \HozDDs.$
\eprob

\bre[\Cref{prob:vgen} is a generalisation of \cref{prob:vedp,prob:vtedp}]
With the exception of some overlap in notation, it is straightforward to see that appropriate choices of $D,$ $\GI$, $\T$ and $\LG$ allow \cref{prob:vgen} to be either \cref{prob:vedp} or \cref{prob:vtedp}. Taking $D = D$, $\GI = \GR$, $\T = \DtN$ and $\LG(v) = \int_{D} f\minispace\vbar$ (for $f$ as in \cref{prob:vedp}) in \cref{prob:vgen}, we see \cref{prob:vgen} becomes \cref{prob:vedp}. However taking $D$ and $\GI$ in \cref{prob:vgen} to be the same as the $D$ and $\GI$ in \cref{prob:vtedp}, taking $\T=ik,$ and $\LG(v) = \int_{D} f\minispace\vbar + \int_{\GI} \gI \minispace\trGI \vbar$ (for $f$ and $\gI$ as in \cref{prob:vtedp}), \cref{prob:vgen} becomes \cref{prob:vtedp}.
\ere

\bre[\Cref{prob:vgen} allows for other boundary conditions]
The strength of the general formulation \cref{prob:vgen} is that it allows us to treat a wide variety of Helmholtz problems at once. Indeed, \emph{any} Helmholtz problem that can be written in the form \cref{eq:vgen,eq:agen} and satisfies \cref{cond:1nbpc,cond:2} below can be treated using the analysis in this \lcnamecref{chap:nbpc}.
\ere 

For the remainder of this \lcnamecref{chap:nbpc}, we let $(\Vhp)_{h>0}$ be a family of finite-dimensional subspaces of $\HozDD$, whose union is dense in $\HozDD$. More specifically, we let $\Vhp$ consist of piecewise-polynomials on a quasi-uniform simplicial mesh $\cTh$ with mesh-size $h$
%\ednote{Euan says: have problem that want to allow $C^{1,1}$ $\Dm$, so that statements later about $H^2$ regularity are covered, but easiest to define triangulation and hence subspaces on Lipschitz domains -- Euan to discuss with Ivan}
and fixed polynomial degree $p$. Note that the dimension $N$ of $\Vhp$ then satisfies $N\sim h^{-d}$, with hidden constant dependent on $p$. (The assumption of quasi-uniformity can, in principle, be relaxed, see \cref{rem:ggsqu} below.) As in \cref{rem:crimes} above, we ignore any variational crimes resulting from this discretisation. We now define the finite-element approximation of \cref{prob:vgen}.
\bprob[Finite-element approximation of \cref{prob:vgen}]\label{prob:fevgen}
    Find $\uh \in \Vhp$ such that
\beq\label{eq:galerkin}
\aG(\uh,\vh) = \LG(\vh) \tforall \vh \in \Vhp.
\eeq
We say that $\uh \in \Vhp$ is the \defn{finite-element approximation of $u$} (the solution to \cref{prob:vgen}).
\eprob
%Observe that implicit in our use of $\aE$ in \cref{eq:galerkin} is the fact that we are realising the Dirichlet-to-Neumann map $\TR$ exactly on $\GR.$
%Definition of Galerkin method

We now define the matrices associated with our finite-element discretisation. Let $\{\phi_i, i= 1, \ldots, N\}$ be a basis for $\Vhp$ with each $\phi_i$ \emph{real-valued}.
Let 
\beq\label{eq:matrixSjdef}
\big(\Smat_{A}\big)_{ij}\de \int_D \big(A \nabla \phi_j)\cdot\nabla \phi_i, \quad
\big(\Mmat_{n}\big)_{ij}\de \int_D n\,\phi_i\, \phi_j,
\quad\tand\quad
\big(\Nmat\big)_{ij}\de \int_{\GR} \T (\gamma\phi_j) \,\gamma \phi_i
\eeq
be the stiffness, domain-mass, and boundary-mass matrices, respectively. Note that both $\Smat_A$ and $\Mmat_n$ are \emph{real-valued}, but in general $\Nmat$ is \emph{complex-valued} (because both the  DtN operator $\DtN$ and the impedance operator $ik$ are complex-valued).
Let
\beq\label{eq:matrixAdef}
\Amat \de \Smat_{A} - k^2 \Mmat_{n} - \Nmat,
\eeq
and let $u_h\de \sum_j u_j \phi_j$. Then \cref{eq:galerkin} implies that
\beqs
\Amat \uvec = \fvec,
\eeqs
where $(\fvec)_i \de \FE(\phi_i)$.
Similarly to above we let 
\beq\label{eq:matrixAjdef}
\Amatj \de \Smat_{A^{(j)}} - k^2 \Mmat_{n^{(j)}} - \Nmat.
\eeq
%then the weighted norm $\|\cdot\|_{\Dmat_k}$ is given by 
%\beq\label{eq:Dk3}
%\N{\vvec}_{\Dmat_k}^2\de   \N{v_h}^2_{\HokD}=\big( \Dmat_k \vvec,\vvec\big)_2,
%\eeq
%for
%$v_h =\sum_i v_i \phi_i$.

\ble[Norm equivalences of FE functions]\label{lem:normequiv}
There exist $m_\pm >0$ and $s_+>0$, independent of $h$ (but dependent on $p$), such that
\beq\label{eq:normequiv1}
m_- h^{d/2} \N{\vvec}_2 \leq \N{v_h}_{\LtD} \leq m_+ h^{d/2} \N{\vvec}_2,
\eeq
and
\beq\label{eq:normequiv2}
\N{\nabla v_h}_{\LtD} \leq s_+ h^{d/2-1} \N{\vvec}_2,
\eeq
for all finite-element functions $v_h =\sum_i v_i \phi_i \in \Vhp$.
\ele

\bpf[Proof of \cref{lem:normequiv}]
We first show \cref{eq:normequiv1} by direct computation, before concluding \cref{eq:normequiv2} from \cref{eq:normequiv1} and a standard inverse inequality. Throughout this proof, when we use $\sim$ the hidden constants are independent of $\tau \in \Th,$ the mesh size $h$, and $\vh \in \Vhp,$ but may depend on $p.$

For any $\vh \in \Vhp$ we have (letting $\bnj$ denote a node of $\Th$)
\begin{align}
  \NLtD{\vh}^2 &= \sum_{\tau \in \Th} \int_\tau \abs{\vh}^2\label{eq:twiddlenumber1}\\
  &\sim \sum_{\tau \in \Th} \abs{\tau} \sum_{\bnj \in \tau} \abs{\vh(\bnj)}^2 \label{eq:twiddlenumber2}\\
  &\sim h^d \sum_{\tau \in \Th} \sum_{\bnj \in \tau} \abs{\vh(\bnj)}^2, \text{ by quasi-uniformity,}\nonumber\\
  &\sim h^2 \Nt{\bu},
\end{align}
i.e., \cref{eq:normequiv1}. The expression \cref{eq:twiddlenumber2} follows from \cref{eq:twiddlenumber1} because the terms defined on $\tau$ are equivalent norms on $\tau$ of functions in $\Vhp.$

To show \cref{eq:normequiv2} we recall the standard inverse inequality (see, e.g., \cite[Theorem 4.5.11 and Remark 4.5.20]{BrSc:08})
\beq\label{eq:twiddleinverse}
\NHoD{\vh} \lesssim h^{-1} \NLtD{\vh}.
\eeq
By combining \cref{eq:twiddleinverse} and the right-hand side of \cref{eq:normequiv1}, we obtain \cref{eq:normequiv2}.
\epf

\bre[Relaxing the assumption of quasi-uniformity]\label{rem:ggsqu}
We assume that $\set{\Th}_{h>0}$ is a quasi-uniform family of meshes so that the proof of \cref{lem:normequiv} is straightforward. However, this assumption can almost certainly be relaxed. In \cite{GaGrSp:15} (on which the bulk of the arguments in this \lcnamecref{chap:nbpc} are based) Gander, Graham, and Spence prove results both for quasi-uniform meshes and also for shape-regular meshes (see \cite[Section 3.4 and 4.1.2]{GaGrSp:15}). Given the results in \cite{GaGrSp:15} for shape-regular meshes are analagous to those they obtain for quasi-uniform meshes, we expect the results in this \lcnamecref{chap:nbpc} can also be extended to shape-regular meshes. However, we note that \cite{GaGrSp:15} only contains bounds on preconditioned mass matrices (analagous to \cref{lem:keylemma1} below) but not preconditioned stiffness matrices (analogous to \cref{lem:keylemma2} below. Therefore it remains open to prove that our results in this \lcnamecref{chap:nbpc} can be extended in their entirety to shape-regular meshes.
\ere

Written in terms of the matrices $\Mmat_1$ and $\Smat_I$ defined in \cref{eq:matrixSjdef}, the bounds \cref{eq:normequiv1} and \cref{eq:normequiv2} are, respectively, the bounds
\beqs
(\Mmat_1 \vvec,\vvec)_2 \sim h^d \N{\vvec}^2_2 \quad\tand\quad (\Smat_I \vvec,\vvec)_2 \lesssim h^{d-2} \N{\vvec}^2_2.
\eeqs

%For a proof of \cref{lem:normequiv}, see\footnote{In \cite[Chapter V, Lemma 2.5]{Br:07} the assumption is made that the meshes underlying $\Vhp$ are \emph{uniform}. However, the definition of uniformity in \cite[Chapter 2, Definition 5.1(4)]{Br:07} is the same as the more standard definition of quasi-uniformity in \cref{def:quasiuniform}.} \cite[Chapter V, Lemma 2.5]{Br:07}.

%% \bpf[Sketch proof of \cref{lem:normequiv}]\opntodo{can omit this if can find a good reference. One possibility . Need to check basis scaling business.}
%% The inequalities in \cref{eq:normequiv1} follow from writing $\|v_h\|_{\LtD}$ as a sum of integrals over elements of the mesh, and then mapping to the reference element \ednote{Euan to discuss with Ivan}.
%% %\beqs
%% %\N{v_h}^2_{L^2(\O
%% %\eeqs
%% Then, \cref{eq:normequiv2} follows from \cref{eq:normequiv1} and the inequalities
%% \beqs
%% \N{v_h}_{L^2(D)}\lesssim \N{\nabla v_h}_{L^2(D)}\lesssim \frac{1}{h} \N{v_h}_{L^2(D)},
%% \eeqs
%% the first of which follows from the Poincar\'e inequality, since $v_h \in \HozDD$
%% (see, e.g., \cite[Proposition 5.3.4]{BrSc:00}), the second of which follows from a standard inverse estimate (see, e.g., \cite[Theorem 4.5.11]{BrSc:00}).
%% \epf


Finally, we need the concept of the \emph{adjoint} sesquilinear form to $\aG(\cdot,\cdot)$.
\begin{definition}[The adjoint sesquilinear form $\aGs(\cdot,\cdot)$]\label{def:adjoint}
Let $D$, $A$, and $n$, be as in \cref{prob:vgen}. The adjoint sesquilinear form, $\aGs(\cdot,\cdot)$, to $\aG(\cdot,\cdot)$ defined in \cref{eq:agen} is given by
\beq\label{eq:EDPadjoint}
\aGs(w,v) \de \int_{D} 
\Big((A \grad w)\cdot\grad \vb
 - k^2 n w\vb\Big) -  \DPGI{\trI w}{\T(\trI v)}.
\eeq
\end{definition}

\noi It is then straightforward to check that
\beq\label{eq:A*}
\Amat^\dagger \de \Smat_A -k^2 \Mmat_n - \Nmat^\dagger
\eeq
(where $^\dagger$ denotes conjugate transpose) is the Galerkin matrix for the sesquilinear form $\aGs(\cdot,\cdot)$; i.e.~$(\Amat^\dagger)_{ij} = \aGs(\phi_j, \phi_i)$.

\ble[Link between variational problems involving $\aG(\cdot,\cdot)$ and $\aGs(\cdot,\cdot)$]\label{lem:adjoint}

\

\noindent If $\LG$ is as in \cref{prob:vgen}, $u$ is the solution to the variational problem \cref{prob:vgen}, the boundary operator $\T$ satisfies
\beq\label{eq:DPconj}
\DPGI{\T \psi}{\phibar} = \DPGI{\T \phi}{\psibar} \tfa \phi,\psi \in \HhGI,
\eeq
and
\beq\label{eq:adjoint1}
\aGs(u,v)= \LG(v) \quad\tfa v\in \HozDD,
\eeq
then $\overline{u}$ satisfies
\beq\label{eq:adjoint2}
\aG(\overline{u},v)= \overline{\LG(\overline{v})} \quad\tfa v\in \HozDD.
\eeq
\ele

\bpf[Proof of \cref{lem:adjoint}]
From \cref{eq:adjoint1} we have that 
\beqs
\overline{\aGs(u,\overline{v})}= \overline{\LG(\overline{v})} \quad\tfa v\in \HozDD.
\eeqs
Using the definition of $\aGs(\cdot,\cdot)$ and the property \cref{eq:DPconj} in the left-hand side of this last equation, we find \cref{eq:adjoint2}.
\epf

\bco[\Cref{eq:adjoint2} holds for \cref{prob:vedp,prob:vtedp}]\label{cor:adjoint}
If \cref{prob:vgen} is chosen to represent either \cref{prob:vedp} or \cref{prob:vtedp}, then \cref{eq:adjoint2} holds.
\eco

\bpf[Proof of \cref{cor:adjoint}]
The only thing we need to check is that \cref{eq:DPconj} holds for both \cref{prob:vedp,prob:vtedp}. For \cref{prob:vtedp}, when $\T = ik$, the proof is straightforward. For \cref{prob:vedp} when $\T = \DtN$ we need the following property of the DtN map $\DtN$:
\beq\label{eq:DtN}
\DPGR{\DtN\psi}{\phibar} = \DPGR{\DtN \phi}{\psibar} \quad\tfa \phi,\psi \in H^{1/2}(\GR).
\eeq
This property follows from the fact that, if $\uo$ and $\ut$ are solutions of the homogeneous Helmholtz equation $\Delta u +k^2 u=0$ in $\RRd\setminus \overline{\BR}$, both satisfying the Sommerfeld radiation condition \cref{eq:src}, then
\beqs
\int_{\GR} (\gamma \uo)\, \dn \ut = \int_{\GR} (\gamma \ut)\,\dn \uo;
\eeqs
which follows from Green's theorem and, e.g., \cite[Lemma 4.10]{Sp:15}.
\epf

\Cref{cor:1,cor:1a} are proved under the following two \lcnamecrefs{cond:1nbpc}, which are the minimal conditions needed for the proof of \cref{cor:1,cor:1a}. Therefore, in particular, \cref{cond:2} is a very weak condition on the finite-element space $\Vhp.$

\begin{condition}[Nontrapping bound on $u^{(1)}$]\label{cond:1nbpc}
$\Aso, \nso,$ and $D$ are such that, given $f\in L^2(D)$
  the solution of \cref{prob:vgen} with
  \beq\label{eq:LGf}
  \LG(v) = \int_D f\vbar,
  \eeq
$u^{(1)}$, exists, is unique, and, given $k_0>0$, $u^{(1)}$ satisfies the bound 
\beq\label{eq:bound1}
\big\|u^{(1)}\big\|_{\HokD} \leq C^{(1)}_{\rm bound} \N{f}_{L^2(D)} \quad \tfa k\geq k_0,
\eeq
where $C^{(1)}_{\rm bound}$ is independent of $k$, but dependent on $\Aso, \nso, D$, and $k_0$.
\end{condition}

\begin{condition}[$k$-independent accuracy of the FE solution for $a^{(1)}(\cdot,\cdot)$]
\label{cond:2}

\

\noindent\ben

\item\label[itempart]{it:femasspt1} Given $\kz>0$, $h$ and $p$ are such that, if $f= n\sum_j \alpha_j\phi_j$ for some $\alpha_j \in \CC$ and  $n\in \LiDR$  (i.e.~$f$ is an arbitrary element of $\Vhp$ multiplied by $n$), then for all $k\geq k_0$,
  \bit
\item The solution $u_h$ of \cref{prob:fevgen} (with $\aG = \aGo$, and $\LG(v)$ given by \cref{eq:LGf}) exists and is unique, and
\item The error bound
  \beq\label{eq:bound3}
\N{u-u_h}_{\HokD} \leq C^{(1)}_{\rm FEM1} \N{f}_{\LtD} \quad\tfa k\geq k_0, 
\eeq
holds, where $C^{(1)}_{\rm FEM1}$  is independent of $k$ and $h$, but dependent on $\Aso, \nso, D, k_0$, and $p$.
  \eit

\item Given $k_0>0$, $h$ and $p$ are such that, if $\LG(v)= (A\nabla \widetilde{f},\nabla v)_{\LtD}$, where $A\in \LiDRRdtd$ and $\widetilde{f} \de \sum_j \alpha_j \phi_j$ with $\alpha_j\in \CC$  (i.e.~$\widetilde{f}$ is an arbitrary element of $\Vhp$), then for all $k \geq \kz$,
  \bit
\item The solution $u_h$ of \cref{prob:fevgen} with $\aG = \aGo$ exists and is unique, and
\item The error bound
\beq\label{eq:bound4}
\N{u-u_h}_{\HokD} \leq C^{(1)}_{\rm FEM2}\,k\, \N{\LG}_{\HokDs} \quad\tfa k\geq k_0, 
\eeq
holds, where $C^{(1)}_{\rm FEM2}$  is independent of $k$ and $h$, but dependent on $\Aso, \nso, D, k_0$, and $p$.  
  \eit
\een
\end{condition}

For details of when \cref{cond:1nbpc,cond:2} are satisfied (for \cref{prob:vedp,prob:vtedp}), see \cref{sec:pdetheory} (for \cref{cond:1nbpc}) and \cref{sec:helmfedisc} (for \cref{cond:2} \cref{it:femasspt1}).

\bre[\Cref{eq:bound4} has the same $k$-dependence as \cref{eq:bound3}]\label{rem:yesitis}
Observe that the bound \cref{eq:bound4} has the same $k$-dependence as \cref{eq:bound3} despite the fact that a factor $k$ appears on the right-hand side. If $\LG(v) = \int_{D} f\vbar,$ then
\begin{align*}
  \NHokDs{\LG} = \sup_{v \in \HozDD} \frac{\abs{\LG(v)}}{\NHokD{v}} &\leq\sup_{v \in \HozDD} \frac{\NLtD{f}\NLtD{v}}{\NHokD{v}}\\
  &\lesssim \frac1k \sup_{v \in \HozDD} \frac{\NLtD{f}\NLtD{v}}{\NLtD{v}}\\
  &= \frac{\NLtD{f}}k.
\end{align*}
The factor $k$ appears in \cref{eq:bound4} since we use the weighted norm $\NHokD{\cdot}$ in the definition of $\NHokDs{\cdot},$ rather than the unweighted norm $\NHoD{\cdot}.$
\ere


\subsection{Proof of the main ingredient of the proofs of \cref{cor:1,cor:1a}}\label{sec:proofs}

The main part of the proofs of \cref{cor:1,cor:1a} is the following \lcnamecref{thm:1}, where we work in the weighted norm
\beq\label{eq:Dk2}
\Dmat_k\de \Smat_I + k^2 \Mmat_1;
\eeq
which defines the norm on $\Vhp$ induced by $\NHokD{\cdot},$ see \cref{eq:Dk}.

\begin{theorem}[Main ingredient of the answer to \cref{it:nbpcq1}]\label{thm:1}
Assume $\Dm$, $\Aso$, and $\nso$ satisfy \cref{cond:1nbpc}, and assume that $h$ and $p$ satisfy \cref{cond:2}. 
Let the $k$- and $h$-independent constants $\mpm$ and $\splus$ be given as in \cref{lem:normequiv}.
Then, given $\kz>0$, there exist $\Co, \Ct>0$, independent of $h$ and $k$ (but dependent on $\Dm, \Aso, \nso$, $p$, and $\kz$) such that
\begin{align}\nonumber
&\max\Big\{
\NDmatk{\Imat - (\Amat^{(1)})^{-1}\Amat^{(2)}}, 
\N{\Imat -\Amat^{(2)} (\Amat^{(1)})^{-1}}_{(\Dmat_k)^{-1}}
\Big\}\\
&\hspace{3cm} 
\leq C_1 \,k \,
\NLiDRRdtd{\Aso-\Ast} + C_2 \, k \, \NLiDRR{\nso-\nst}
\label{eq:main1}
\end{align}
and 
\begin{align}\nonumber
&\max\Big\{
\N{\Imat - (\Amat^{(1)})^{-1}\Amat^{(2)}}_2, 
\N{\Imat -\Amat^{(2)} (\Amat^{(1)})^{-1}}_2
\Big\}\\
&\hspace{0cm} 
\leq C_1 \,\left(\frac{s_+}{m_-}\right) \,\frac{1}{h} \,
\NLiDRRdtd{\Aso-\Ast} + C_2 \, \left(\frac{m_+}{m_-} \right)k \, \NLiDRR{\nso-\nst}
\label{eq:main1a}
\end{align}
for all $k\geq k_0$. 
\end{theorem}

The proof of \cref{thm:1} is given after the following two lemmas, that are the heart of the proof of \cref{thm:1}.

\ble[Bounds on $(\Amato)^{-1} \Mmat_{n}$]\label{lem:keylemma1}
Assume that \cref{cond:1nbpc} holds, and assume that Part (i) of \cref{cond:2} holds. Then, for $n\in \LiDRR$,
\beq\label{eq:keybound1}
\max\Big\{\big\| (\Amato)^{-1} \Mmat_{n} \big\|_{\Dmat_k}, \,\,
\big\|  \Mmat_{n}(\Amato)^{-1} \big\|_{(\Dmat_k)^{-1}}
\Big\}\leq 
C_2
%\frac{m_+}{m_-} \left[ C_{\rm FEM1}^{(1)} + C_{\rm bound}^{(1)}\right] 
\frac{\NLiDRR{n}}{k}
\eeq
and 
\beq\label{eq:keybound1a}
\max\Big\{\big\| (\Amato)^{-1} \Mmat_{n} \big\|_2, \,\,
\big\|  \Mmat_{n}(\Amato)^{-1} \big\|_2 
\Big\}\leq 
C_2 
%\frac{m_+}{m_-} \left[ C_{\rm FEM1}^{(1)} + C_{\rm bound}^{(1)}\right] 
\left(\frac{m_+}{m_-}\right) \frac{\NLiDRR{n}}{k}
\eeq
for all $k\geq k_0$,
where
\beq\label{eq:C2}
C_2\de%\frac{m_+}{m_-} 
%\left[ 
C_{\rm FEM1}^{(1)} + C_{\rm bound}^{(1)}.%\right].
\eeq
\ele

The proof of \cref{lem:keylemma1} is on \cpageref{page:lemkeylemma1proof} below.

\ble[Bounds on $(\Amato)^{-1} \Smat_A$]\label{lem:keylemma2}
Assume that \cref{cond:1nbpc} holds, and assume that Part (ii) of \cref{cond:2} holds. Then, for $A\in \LiDRRdtd$,
\beq\label{eq:keybound2}
\max\Big\{\big\| (\Amato)^{-1} \Smat_A \big\|_{(\Dmat_k)^{-1}}, \,\,
\big\| \Smat_A (\Amato)^{-1} \big\|_{\Dmat_k}\Big\} \leq C_1\, k\NLiDRRdtd{A}
\eeq
and
\beq\label{eq:keybound2a}
\max\Big\{\big\| (\Amato)^{-1} \Smat_A \big\|_2, \,\,
\big\| \Smat_A (\Amato)^{-1} \big\|_2\Big\} \leq C_1\,\left(\frac{s_+}{m_-}\right) \frac{1}{h}\NLiDRRdtd{A}
\eeq
%\begin{align}\nonumber
%&\max\Big\{\big\| (\Amato)^{-1} \Smat_A \big\|_2, \,\,
%\big\| \Smat_A (\Amato)^{-1} \big\|_2\Big\}\nonumber \\
%&\hspace{2cm}
% \leq \frac{s_+}{s_-} \left[ C_{\rm FEM2}^{(1)} + 
% \frac{1}{\min\big\{\Asomin,\nsomin\big\}}\left( \frac{1}{k_0} + 2 C^{(1)}_{\rm bound}\nsomax  \right) \right]k\N{A}_{L^\infty(D)}\label{eq:keybound2}
%% + C_{\rm bound}^{(1)}\right) \frac{\N{n}_{L^\infty(D)}}{k}.
%\end{align}
for all $k\geq k_0$, where
\beq\label{eq:C1nbpc}
C_1\de%\frac{s_+}{s_-} 
\left[ C_{\rm FEM2}^{(1)} + 
 \frac{1}{\min\big\{\Asomin,\nsomin\big\}}\left( \frac{1}{k_0} + 2 C^{(1)}_{\rm bound}\nsomax  \right) \right]
\eeq
\ele

The proof of \cref{lem:keylemma2} is on \cpageref{page:lemkeylemma2proof} below.

\bpf[Proof of \cref{thm:1} using \cref{lem:keylemma1,lem:keylemma2}]
Using the definition of the matrices $\Amatj, \SmatA$, and $\Mmatn$ in \cref{eq:matrixAjdef} and \cref{eq:matrixSjdef}, we have
\begin{align}\nonumber
\Imat - (\Amato)^{-1}\Amatt = (\Amato)^{-1}\big(\Amato-\Amatt\big) &=  (\Amato)^{-1}\left( \Smat_{A^{(1)}} - \Smat_{A^{(2)}} - k^2 \big(\Mmat_{n^{(1)}}-\Mmat_{n^{(2)}}\big)\right)\\
&= (\Amato)^{-1}\left( \Smat_{A^{(1)}-A^{(2)}} - k^2 \Mmat_{n^{(1)}-n^{(2)}}\right),\label{eq:idea1}
\end{align}
and similarly 
\beq\label{eq:idea2}
\Imat -\Amatt  (\Amato)^{-1}= \left( \Smat_{A^{(1)}-A^{(2)}} - k^2 \Mmat_{n^{(1)}-n^{(2)}}\right)(\Amato)^{-1}.
\eeq
The bounds in  \cref{eq:main1} on $\NDk{\Imat - (\Amato)^{-1}\Amatt}$ and  $\NDkI{\Imat - \Amatt(\Amato)^{-1}}$ then follow from using the bounds \cref{eq:keybound1,eq:keybound2} in \cref{eq:idea1,eq:idea2}. The bounds in \Cref{eq:main1a} on $\Nt{\Imat - (\Amato)^{-1}\Amatt}$ and  $\Nt{\Imat - \Amatt(\Amato)^{-1}}$ follow completely analagously, except we use the bounds \cref{eq:keybound1a,eq:keybound2a} instead of the bounds \cref{eq:keybound1,eq:keybound2}.
%
%, and $C_1$, $C_2$ in \cref{eq:main1} are given explicitly by
%\beq\label{eq:C1nbpc}
%C_1\de%\frac{s_+}{s_-} 
%\left[ C_{\rm FEM2}^{(1)} + 
% \frac{1}{\min\big\{\Asomin,\nsomin\big\}}\left( \frac{1}{k_0} + 2 C^{(1)}_{\rm bound}\nsomax  \right) \right] \,\,\tand\,\,
%\quad C_2\de  %+ \frac{m_+}{m_-} 
% \left[ C_{\rm FEM1}^{(1)} + C_{\rm bound}^{(1)}\right].
%\eeq
\epf

\

\bpf[Proof of \cref{lem:keylemma1}]
\label{page:lemkeylemma1proof}
We first concentrate on proving \cref{eq:keybound1}.
Given $\fvec \in \CC^N$ and $n\in \LiDRR$, we create a variational problem whose Galerkin discretisation leads to the equation $\Amato \tbu = \Mmat_n\,\fvec$.
Indeed, let $\widetilde{f} \de \sum_j f_j \phi_j\in \HozDD$. Define $\widetilde{u}$ to be the solution of the variational problem 
\beq\label{eq:411}
a^{(1)}(\widetilde{u},v)= \IPLtD{n\widetilde{f}}{v} \quad\text{ for all } v\in \HozDD,
\eeq
and let $\tu_h$ be the solution of the finite-element approximation of \cref{eq:411}, i.e.,
\beq\label{eq:41}
a^{(1)}(\tu_h,v_h)= \IPLtD{(n\widetilde{f}}{v_h} \quad\text{ for all } v_h\in \Vhp,
\eeq
and let $\tbu$ be the vector of nodal values of $\tu_h$. The definition of $\widetilde{f}$ then implies that \cref{eq:41} is equivalent to the linear system $\Amato \tbu = \Mmat_{n}\,\fvec$, and so to obtain a bound on $\|(\Amato)^{-1}\Mmat_n\|_{\Dmat_k}$ we need to bound $\|\tbu\|_{\Dmat_k}$ in terms of $\|\fvec\|_{\Dmat_k}$. (Recall $\fvec \in \CCN$ was arbitrary.) Because of the definition 
of $\|\cdot\|_{\Dmat_k}$ in \cref{eq:Dk}, this is bound equivalent to bounding $\|\tu_h\|_{\HokD}$ in terms of $\|\widetilde{f}\|_{\HokD}$.

%First observe that the bound \cref{eq:bound3} from Part (i) of \cref{cond:2} holds for the solution of the variational problem
%\beqs%\label{eq:411}
%a^{(1)}(u,v)= (n\phi_j,v)_{L^2(\Omega)} \quad\text{ for all } v\in H^1(\Omega),
%\eeqs
%and hence, by linearity, it also holds for the solution $\widetilde{u}$ of the variational problem \cref{eq:411}.

Using %the bounds in \cref{eq:normequiv1}, 
the triangle inequality and the bounds \cref{eq:bound3} and \cref{eq:bound1} from \cref{cond:2,cond:1nbpc} respectively, we find
%Note that the hypotheses imply that the bound on the solution operator 
%\cref{eq:bound_unif} holds (by \cref{cor:uniform}), and also that if $h k\sqrt{|k^2-\eps|} \leq C_1$ then quasi-optimality \cref{eq:qoeps_lemma} holds (by \cref{lem:qo}).
%Starting with \cref{eq:equiv} we then have 
\begin{align}
%m_- h^{d/2}k \N{\tbu}_2 \leq k\N{\tu_h}_{\LtD}\leq  
\N{\tu_h}_{\HokD} \leq
\N{\tu-\tu_h}_{\HokD} + \N{\tu}_{\HokD} \label{eq:mainevent1}
& \leq C^{(1)}_{\rm FEM1}\NLtD{n\ftilde} + C^{(1)}_{\rm bound}\NLtD{n\ftilde} \\ 
& \leq \mleft(C^{(1)}_{\rm FEM1} + C^{(1)}_{\rm bound}\mright)\NLiDRR{n}\NLtD{\ftilde} \label{eq:mainevent1a} \\
& \leq\big(C^{(1)}_{\rm FEM1}+  C^{(1)}_{\rm bound}\big)\NLiDRR{n}\frac{\big\|\widetilde{f}\big\|_{\HokD}}{k};\nonumber
%& \leq\big(C^{(1)}_{\rm FEM1}+  C^{(1)}_{\rm bound}\big)\N{n}_{L^\infty(D)} m_+ h^{d/2} \N{\fvec}_2,
\end{align}
the bound on $\|(\Amato)^{-1}\Mmat_n\|_{\Dmat_k}$ in \cref{eq:keybound1} then follows from the definition of $\|\cdot\|_{\Dmat_k}$ in \cref{eq:Dk} and the definition of $C_2$ \cref{eq:C2}.

To prove the bound on $\|\Mmat_n(\Amato)^{-1}\|_{(\Dmat_k)^{-1}}$ in \cref{eq:keybound1}, first observe that the definitions of $\|\cdot\|_{\Dmat_k}$ and $\|\cdot\|_{(\Dmat_k)^{-1}}$ in \cref{eq:Dk} imply that, for any matrix $\Cmat \in \CCNtN$ and for any $\vvec\in \CC^N$,
\beq\label{eq:A380-0}
\frac{
\big\|\matrixC \vvec \big\|_{(\Dmat_k)^{-1}}
}{
\big\|\vvec\|_{(\Dmat_k)^{-1}}
} = 
\frac{
\big\|\matrixC^\dagger \wvec \big\|_{\Dmat_k}
}{
\big\|\wvec\|_{\Dmat_k}
}
\eeq
where $\wvec \de (\Dmat_k)^{1/2}\vvec$, and where $\matrixC^\dagger$ is the conjugate transpose of $\matrixC$ (i.e.~the adjoint with respect to $(\cdot,\cdot)_2$).
Therefore, since $\Mmat_n$ is a real, symmetric matrix,
\beqs
\frac{
\big\|\Mmat_n (\Amato)^{-1}\vvec\big\|_{(\Dmat_k)^{-1}}
}{
\N{\vvec}_{(\Dmat_k)^{-1}}
}
=
\frac{\NDk{\mleft(\AmatoI\Mmatn\mright)^\dagger \wvec}}{\NDk{\wvec}}
= 
\frac{
\big\|((\Amato)^\dagger)^{-1}\Mmat_n\wvec\big\|_{\Dmat_k}
}{
\N{\wvec}_{\Dmat_k}
},
 \eeqs
 so that 
\beq\label{eq:A380} 
 \big\|\Mmat_n (\Amato)^{-1}\big\|_{(\Dmat_k)^{-1}}=\big\|((\Amato)^\dagger)^{-1}\Mmat_n\big\|_{\Dmat_k}.
 \eeq 
Recall from the text below \cref{eq:A*} that $(\Amato)^\dagger$ is the Galerkin matrix corresponding to the variational problem \cref{eq:adjoint1} -- the adjoint problem. \cref{lem:adjoint} implies that if the EDP %with coefficients $A^{(1)}$ and $n^{(1)}$ 
satisfies \cref{cond:1nbpc,cond:2}, then so does the adjoint problem. Therefore, the argument above leading to the bound on $\|(\Amato)^{-1}\Mmat_n\|_{\Dmat_k}$ under \cref{cond:1nbpc} and Part (i) of \cref{cond:2} proves the same bound on $\|((\Amato)^\dagger)^{-1}\Mmat_n\|_{\Dmat_k}$, and then, using \cref{eq:A380}, also on $\big\|\Mmat_n(\Amato)^{-1}\big\|_{(\Dmat_k)^{-1}}$.

To prove the bound on  $\|(\Amato)^{-1}\Mmat_n\|_{2}$ in \cref{eq:keybound1a}, we use the bounds 
\beqs
m_- h^{d/2} k \N{\tbu}_2 \leq k \N{\widetilde{u}_h}_{\LtD} \leq \N{\widetilde{u}_h}_{\HokD}
\,\tand\,
\big\|\widetilde{f}\big\|_{\LtD} \leq m_+ h^{d/2}\N{\fvec}_2,
\eeqs
on either side of the inequality \cref{eq:mainevent1}, with these bounds coming from \cref{eq:normequiv1}. The proof of the bound on 
$\|\Mmat_n((\Amato)^\dagger)^{-1}\|_{2}$ in \cref{eq:keybound1a} follows in a similar way to above, using the fact that 
$\|\Mmat_n (\Amato)^{-1}\|_2=\|((\Amato)^\dagger)^{-1}\Mmat_n\|_2$ (compare to \cref{eq:A380}).
%, namely the variational problem \cref{eq:EDPvar} with the operator $T_R$ in $a^{(1)}(\cdot,\cdot)$ replaced by $\overline{T_R}$ (corresponding to the $-\ri k$ in the radiation condition \cref{eq:src} being changed to $+i k$).
%
%Now, if $u$ is the solution of the adjoint problem with data $\LE(v)$, then $\overline{u}$ is the solution of the original problem with data $\overline{\LE(\overline{v})}$; 
%
%in particular if $\LE(v)$ is as in \cref{eq:EDPa}, then the $L^2$ data of the adjoint problem is just $\overline{f}$. Therefore, if the EDP satisfies \cref{cond:1nbpc,cond:2}, then so does its adjoint, and
% the bound in \cref{eq:keybound1} on $\|(\Amato)^{-1}\Mmat_n\|_{2}$ also holds for $\|((\Amato)^\dagger)^{-1}\Mmat_n\|_{2}$.
\epf

The proof of \cref{lem:keylemma2} uses the following \lcnamecref{lem:H1}, which one can prove using the G\aa rding inequality \cref{eq:gardingbrief}; see \cite[Lemma 5.1]{GrPeSp:19}.

\ble[Bound for data in $\HozDDs$]\label{lem:H1}
%With the sesquilinear form $a(\cdot,\cdot)$ defined by \cref{eq:EDPa} with $A=\Aso$ and $n=\nso$, 
Given $\LGtilde\in \HozDDs$, let $\widetilde{u}$ be the solution of the variational problem
\beqs
\text{ find } \,\,\widetilde{u} \in H^1_{0,D}(D) \,\,\tst \,\,
a^{(1)}(\widetilde{u},v)=\LGtilde(v) \,\, \tfa v\in H^1_{0,D}(D).
\eeqs
If \cref{cond:1nbpc} holds, then $\widetilde{u}$ exists, is unique, and satisfies the bound
\beq\label{eq:bound2}
\N{\widetilde{u}}_{\HokD} \leq \frac{1}{\min\{\Asomin,\nsomin\}}\left( 1 + 2 C^{(1)}_{\rm bound}\nsomax  k\right) \big\|\LGtilde\big\|_{(\HokD)^*}
\eeq
for all $k\geq k_0$.
\ele
%Observe that, similar to \cref{rem:yesitis}, \cref{eq:bound2} is a $k$-independent bound, due to the norm $\NHokDs{\LE}$ on the right-hand side.


\bpf[Proof of \cref{lem:keylemma2}]
\label{page:lemkeylemma2proof}
In a similar way to the proof of \cref{lem:keylemma1}, given $\fvec \in \CC^N$ and $A\in \LiDRRdtd$, let $\widetilde{f} \de \sum_j f_j \phi_j$ and observe that $\widetilde{f} \in \HozDD$. Define $\widetilde{u}$ to be the solution of the variational problem 
\beq\label{eq:411a}
a^{(1)}(\widetilde{u},v)= \LGtilde(v) \quad\text{ for all } v\in \HozDD,
\quad\text{ where } \quad
 \LGtilde(v) \de \IPLtD{(A\nabla\widetilde{f}}{\nabla v}.
\eeq
Observe that the definition of the norms $\|\cdot\|_{(\HokD)^*}$ and $\|\cdot\|_{\HokD}$ \cref{eq:weightednorm} and the Cauchy-Schwarz inequality imply that
\begin{align}
\big\| \LGtilde\big\|_{(\HokD)^*}&\leq \big\|A\nabla \widetilde{f}\big\|_{\LtD}\nonumber\\
&\leq \NLiDRRdtd{A} \big\|\nabla \widetilde{f}\big\|_{\LtD}\label{eq:Fbounda}\\
&\leq \NLiDRRdtd{A} \big\| \widetilde{f}\big\|_{\HokD}.\label{eq:Fbound}
\end{align}
Let $\tu_h$ be the solution of the finite element approximation of \cref{eq:411a}, i.e.,
\beq\label{eq:41a}
a^{(1)}(\tu_h,v_h)= \LGtilde(v_h) \quad\text{ for all } v_h\in \Vhp,
\eeq
and let $\tbu$ be the vector of nodal values of $\tu_h$. The definition of $\widetilde{f}$ then implies that \cref{eq:41a} is equivalent to $\Amato \tbu = \Smat_A\,\fvec$. 

Similar to the proof of \cref{lem:keylemma1},
using the triangle inequality, the bound \cref{eq:bound4} from \cref{cond:2}, the bound \cref{eq:bound2} from \cref{lem:H1}, the bound \cref{eq:Fbound}, and the definition of $C_1$ \cref{eq:C1nbpc},
we find
%Note that the hypotheses imply that the bound on the solution operator 
%\cref{eq:bound_unif} holds (by \cref{cor:uniform}), and also that if $h k\sqrt{|k^2-\eps|} \leq C_1$ then quasi-optimality \cref{eq:qoeps_lemma} holds (by \cref{lem:qo}).
%Starting with \cref{eq:equiv} we then have 
\begin{align}\nonumber 
%s_- h^{(d-2)/2} \N{\tbu}_2 &\leq \N{\nabla \tu_h}_{\LtD}\leq  
\N{\tu_h}_{\HokD} &\leq
\N{\tu-\tu_h}_{\HokD} + \N{\tu}_{\HokD},\nonumber \\ \nonumber
& \leq \left[ C^{(1)}_{\rm FEM2} k + 
\frac{1}{\min\{\Asomin,\nsomin\}}\left( 1 + 2 C^{(1)}_{\rm bound}\nsomax k  \right) 
\right]\big\|\LGtilde\big\|_{(\HokD)^*},\\
&\leq C_1 \, k\, 
%\left[C^{(1)}_{\rm FEM2} k + \frac{1}{\min\{\Asomin,\nsomin\}}\left( 1 + 2 C^{(1)}_{\rm bound}\nsomaxk  \right) \right]
\NLiDRRdtd{A} \big\|\nabla\widetilde{f}\big\|_{\LtD},\label{eq:mainevent2}\\
&\leq C_1 \, k\, 
%\left[C^{(1)}_{\rm FEM2} k + \frac{1}{\min\{\Asomin,\nsomin\}}\left( 1 + 2 C^{(1)}_{\rm bound}\nsomaxk  \right) \right]
\NLiDRRdtd{A} \big\|\widetilde{f}\big\|_{\HokD},\nonumber
%&\leq \left[ C^{(1)}_{\rm FEM2} k + 
%\frac{1}{\min\{\Asomin,\nsomin\}}\left( 1 + 2 C^{(1)}_{\rm bound}\nsomaxk  \right) 
%\right]\big\|A\big\|_{L^\infty(D)}s_+ h^{(d-2)/2} \N{\fvec}_2,
\end{align}
and the bound on $\|(\Amato)^{-1}\Smat_A\|_{\Dmat_k}$ in \cref{eq:keybound2} follows.

The bound on $\|\Smat_A(\Amato)^{-1}\|_{(\Dmat_k)^{-1}}$ follows in a similar way to how we obtained the 
bound on  $\|\Mmat_n(\Amato)^{-1}\|_{(\Dmat_k)^{-1}}$ from the bound on $\|(\Amato)^{-1}\Mmat_n\|_{\Dmat_k}$ in Part (i). Indeed, 
\cref{eq:A380-0} and the fact that $\Smat_A$ is a real, symmetric matrix imply that 
\beq\label{eq:A380-2} 
 \big\|\Smat_A (\Amato)^{-1}\big\|_{(\Dmat_k)^{-1}}=\big\|\big((\Amato)^\dagger\big)^{-1}\Smat_A\big\|_{\Dmat_k}
 \eeq 
%since 
%\beqs
%\big\|\Smat_A(\Amato)^{-1}\big\|_{2}=\big\|(\Smat_A(\Amato)^{-1})^\dagger\big\|_{2}=\big\|((\Amato)^\dagger)^{-1}\Smat_A\big\|_{2},
%\eeqs
(c.f. \cref{eq:A380}),
and then the arguments in the proof of part (i) imply that 
the bound in \cref{eq:keybound2} on $\|(\Amato)^{-1}\Smat_A\|_{\Dmat_k}$ also holds for $\|((\Amato)^\dagger)^{-1}\Smat_A\|_{\Dmat_k}$.

To prove the bound on  $\|(\Amato)^{-1}\Smat_A\|_{2}$ in \cref{eq:keybound2a}, we use the bounds 
\beqs
m_- h^{d/2} k \N{\tbu}_2 \leq k \N{\widetilde{u}_h}_{\LtD} \leq \N{\widetilde{u}_h}_{\HokD}
\,\tand\,
\big\|\nabla \widetilde{f}\big\|_{\LtD} \leq s_+ h^{d/2-1}\N{\fvec}_2,
\eeqs
on either side of the inequality \cref{eq:mainevent2}, with these bounds coming from \cref{eq:normequiv1} and \cref{eq:normequiv2} respectively. The proof of the bound on 
$\|\Smat_A((\Amato)^\dagger)^{-1}\|_{2}$ in \cref{eq:keybound2a} follows in a similar way to above, using \cref{eq:Fbound}.
\epf

\bre[Link to the results of \cite{GaGrSp:15}]
A result analogous to the Euclidean-norm bounds in \cref{thm:1} was proved in \cite{GaGrSp:15} for the case that $\Aso= \Ast= I$, $\nst= 1$, and $\nso = 1 + i\eps/k^2$, with the `absorption parameter' or `shift' $\eps$ satisfying $0<\eps\lesssim k^2$. (Recall that the proof strategy used in this \lcnamecref{chap:nbpc} is based on the strategy in \cite{GaGrSp:15}.) The motivation for proving this result was that the so-called `shifted Laplacian preconditioning' of the Helmholtz equation is based on preconditioning (with these choices of parameters) $\Amatt$ with an approximation of $\Amato$. Similar to \cref{cor:1}, bounds on $\|\Imat -  (\Amato)^{-1}\Amatt \|_2$ and 
$\|\Imat - \Amatt  (\Amato)^{-1}\|_2$
 then give upper bounds on how large the `shift' $\eps$ can be for GMRES for $\AmatoI\Amatt$ to converge in a $k$-independent number of iterations in the case when the action of $(\Amato)^{-1}$ is computed exactly.

%\cite[Lemma 4.1]{GaGrSp:15}
The main differences between \cite{GaGrSp:15} and this work are that (i)  \cite{GaGrSp:15} focused on the TEDP, not the EDP,
(ii) \cite{GaGrSp:15} focused on the particular case that $\Dm$ is star-shaped with respect to a ball, finding a $k$- and $\eps$-explicit expression for $C^{(1)}_{\rm bound}$ in this case using Morawetz identities, whereas we assume the existence of $\Cboundo,$
(iii) \cite{GaGrSp:15} required a bound on 
$(\Amato)^{-1}\Mmat_{n}$, analogous to the bounds in \cref{lem:keylemma1} along with one on $(\Amato)^{-1}\Nmat$ (in the case that $T_R$ is approximated by $i k$), but \emph{not} on 
$(\Amato)^{-1}\Smat_{A}$, and (iv) \cite{GaGrSp:15} only proved bounds in the $\|\cdot\|_2$ norm.
%The result of \cref{thm:1}
\ere

%\bre[Analogue of \cref{thm:1} in a weighted norm]\label{rem:weight1}
%The PDE analysis of the Helmholtz equation naturally takes place in the weighted $H^1$ norm $\|\cdot\|_{\HokD}$ defined by \cref{eq:1knorm}. The discrete analogue of this norm is the norm $\|\cdot\|_{\Dmat_k}$ defined by 
%\beq\label{eq:Dk}
%\N{\vvec}_{\Dmat_k}^2\de \big( (\Smat_I + k^2 \Mmat_1)\vvec,\vvec\big)_2 = \N{v_h}^2_{\HokD}
%\eeq
%for
%$v_h =\sum_i v_i \phi_i$. 
%This norm is used, e.g., in recent results about the convergence of domain-decomposition methods %in this norm are proved 
%for the Helmholtz equation \cite{GrSpVa:17}, \cite{GrSpZo:18}, and for the time-harmonic Maxwell equations \cite{BoDoGrSpTo:19}. 
%
%Inspecting the proof of \cref{lem:keylemma}, we see that the bounds \cref{eq:keybound1} and \cref{eq:keybound2} hold with the $\|\cdot\|_2$ norm replaced by the $\|\cdot\|_{\Dmat_k}$ norm and without the terms involving $m_\pm$ and $s_\pm$ on the right-hand side. \cref{thm:1} 
%%(and also \cref{cor:1}) 
%therefore also holds with the $\|\cdot\|_2$ norm replaced by the $\|\cdot\|_{\Dmat_k}$ norm and the constant $C_1$ modified appropriately.
%\ere

\subsection{Proofs of the finite-element results \cref{cor:1,cor:1a}}\label{sec:mainproofs}

We first give the set-up for weighted GMRES, first introduced in by Essai in \cite{Es:98}, largely following \cite[Section 5]{GrSpVa:17}. Consider the abstract  linear system 
% \begin{equation*}
$\matrixC \xvec = \dvec$
%\end{equation*}
in $\mathbb{C}^N$, where $\matrixC \in \CC^{N\times N}$ is invertible. Let $\xvecz$ be the initial guess, and define the initial residual $\rvec^0 \de \dvec- C \xvec^0$ and the standard Krylov spaces:  
\beqs  
\cK^m(\Cmat, \rvec^0) \de \mathrm{span}\big\{\matrixC^j \rvec^0 : j = 0, \ldots, m-1\big\}.
\eeqs
Analagously to the definition of $\NDk{\cdot}$ above, let $(\cdot , \cdot )_{\Dmat}$ denote the inner product on $\CC^n$ 
induced by some Hermitian positive-definite matrix $\Dmat$, i.e.~
%\begin{equation*}
$(\vvec,\wvec)_{\Dmat} \de (\Dmat \vvec, \wvec)_2,$
%\end{equation*} 
and let $\Vert \cdot \Vert_\Dmat$ be the induced norm. For $m \geq 1$, define the $m$th GMRES iterate $\xvec^m$  to be  the unique element of $\cK^m$ satisfying  the  
 minimal residual  property: 
$$ \ \Vert \rvecm \Vert_\Dmat \de \Vert \dvec - \matrixC \xvec^m \Vert_\Dmat \ = \ \min_{\xvec \in \cK^m(C, \rvec^0)} \Vert {\dvec} - {\matrixC} {\xvec} \Vert_\Dmat. $$
Observe that when $\Dmat = \Imat$ this is the standard GMRES algorithm. We also note that in general, weighted GMRES requires the use of weighted Arnoldi process, also introduced by Essai in \cite{Es:98}, see also the alternative implementations of the weighted Arnoldi process in \cite{GuPe:14}.


Let 
\beq\label{eq:fov}
W_\Dmat(\matrixC)\de \Big\{ (\matrixC \xvec, \xvec)_{\Dmat} : \xvec \in \CCN, \|\xvec\|_\Dmat=1\Big\};
\eeq
$W_\Dmat(\matrixC)$ is called the \emph{numerical range} or \emph{field of values} of $\matrixC$ (in the $(\cdot,\cdot)_\Dmat$ inner product).

%Recall the so-called ``Elman estimate" for GMRES

\begin{theorem}[Elman estimate for weighted GMRES]\label{thm:GMRES1_intro} 
Let $\matrixC$ be a matrix with $\zerovec\notin W(\matrixC)$. Let $\beta\in[0,\pi/2)$ be defined such that
\beq\label{eq:cosbeta}
\cos \beta \de \frac{\mathrm{dist}\big(\zerovec, W(\matrixC)\big)}{\N{\matrixC}_{2}}.
\eeq
If the matrix equation $\matrixC \xvec = \by$ is solved using weighted GMRES then, 
for $m\in \mathbb{N}$, the GMRES residual $\rvecm$ %\de \matrixC \xvec_m - \by$ 
satisfies
\beq\label{eq:Elman}
\frac{\N{\rvecm}_{\Dmat}}{\N{\rvecz}_{\Dmat}} \leq \sin^m \beta. %, \quad \text{ where}\quad 
\eeq
\end{theorem}
The bound \cref{eq:Elman} with $\Dmat=\Imat$ was first proved in \cite[Theorem 6.3]{El:82} (see also \cite[Theorem 3.3]{EiElSc:83}) and was written in the above form in \cite[Equation 1.2]{BeGoTy:06}. The bound \cref{eq:Elman} (for arbitrary Hermitian positive-definite $\Dmat$) was stated implicitly (without proof) in \cite[p. 247]{CaWi:92} and proved in \cite[Theorem 5.1]{GrSpVa:17}. % (see also \cite[Remark 5.2]{GrSpVa:17}). 



\cref{thm:GMRES1_intro} has the following corollary, and the proofs of \cref{cor:1,cor:1a} follow from combining this with \cref{thm:1}.

\begin{corollary}
\label{cor:GMRES_intro} 
If $\|\Imat - \matrixC \|_\Dmat \leq \alpha < 1$, then, with $\beta$ defined as in \cref{eq:cosbeta},
\beqs
\cos \beta \geq \frac{1-\alpha}{1+\alpha}\eeqs
and
\beq\label{eq:gmressin}
\sin \beta \leq \frac{2 \sqrt{\alpha}}{(1+\alpha)^2}.
\eeq
\end{corollary}

\bpf[Proof of \cref{cor:1}]
\label{page:cor1proof}
This follows from \cref{thm:1} by applying \cref{thm:GMRES1_intro,cor:GMRES_intro} first with $\matrixC= (\Amato)^{-1} \Amatt$, $\Dmat=\Dmat_k$, and $\alpha=1/2$, and then with $\matrixC= \Amatt(\Amato)^{-1} $, $\Dmat=(\Dmat_k)^{-1}$, and $\alpha=1/2$.
\epf

\

\bpf[Proof of \cref{cor:1a}]
\label{page:cor1aproof}
This follows from \cref{thm:1} by applying \cref{cor:GMRES_intro} first with $\matrixC= (\Amato)^{-1} \Amatt$, $\Dmat=\Imat$, and $\alpha=1/2$, and then with $\matrixC= \Amatt(\Amato)^{-1} $, $\Dmat=\Imat$, and $\alpha=1/2$.
\epf


\bre[The improvement of the Elman estimate \cref{eq:Elman} in \cite{BeGoTy:06}]
A stronger result than \cref{eq:Elman} is given for standard (unweighted) GMRES in \cite[Theorem 2.1]{BeGoTy:06}, and then converted to a result about weighted GMRES in \cite[Theorem 5.3]{BoDoGrSpTo:19}; indeed, the convergence factor $\sin \beta$ is replaced by a function of $\beta$ strictly less than $\sin\beta$ for $\beta\in (0,\pi/2)$. Using this stronger result, however, does not improve the $k$-dependence of \cref{cor:1}.
\ere


%\section{Proof of }\label{sec:proofPDE}

\subsection{Proofs of the PDE results \cref{thm:2,lem:sharp}}\label{sec:pdeproofs}

\bpf[Proof of \cref{thm:2}]
\label{page:thm2proof}
%We first prove the upper bound \cref{eq:PDEbound}.
Because we assumed \cref{cond:1nbpc} holds for the EDP (\cref{prob:vedp}), $u^{(1)}$ and $u^{(2)}$ exist, are unique, satisfy $a^{(1)}(u^{(1)}, v) = \LG(v)$  and $a^{(2)}(u^{(2)}, v) = \LG(v)$ for all $v \in \HozDD$, respectively, where $\LG$ is given by \cref{eq:Ledp}. Subtracting these equations, we find that $u^{(1)}- u^{(2)}$ satisfies the variational problem
\beq\label{eq:vp1}
a^{(1)}(u^{(1)}-u^{(2)},v) = \LGtilde(v) \quad\tfa v\in H^1_{0,D}(D)
\eeq
where
\beqs
 \LGtilde(v)\de \int_{D} \left((\Ast-\Aso) \nabla u^{(2)}\right) \cdot\overline{\nabla v} + k^2 (\nso-\nst) u^{(2)}\overline{v}.
\eeqs
Now, by the Cauchy-Schwarz inequality and the definition of the norm $\|\cdot\|_{\HokD}$ (see \cref{eq:weightednorm}), we have
\begin{align*}
| \LGtilde(v)| &\leq \NLiDRRdtd{\Aso-\Ast} \big\|\nabla u^{(2)}\big\|_{L^2(D)}
\N{\nabla v}_{L^2(D)} 
\\& \hspace{5cm}+ k^2 
\NLiDRR{\nso-\nst} \big\| u^{(2)}\big\|_{L^2(D)}
\N{v}_{L^2(D)}\\
&\leq\max\Big\{\NLiDRRdtd{\Aso-\Ast}\,,\, \NLiDRR{\nso-\nst}\Big\}
\big\| u^{(2)}\big\|_{\HokD} \N{v}_{\HokD}
\end{align*}
(by Cauchy--Schwarz in $\RR^2$). Therefore, by the definition of the norm $\|\cdot\|_{(\HokD)^*}$
\beqs
\big\|\LGtilde\big\|_{(\HokD)^*}\leq \max\set{\NLiDop{\Aso-\Ast},\NLiDRR{\nso-\nst}}.
\big\| u^{(2)}\big\|_{\HokD}.
\eeqs
Since \cref{cond:1nbpc} holds, we can then apply \cref{lem:H1}, i.e.~the bound \cref{eq:bound2}, to the solution of the variational problem \cref{eq:vp1}  to find that 
\begin{align*}
\frac{\big\| u^{(1)} - u^{(1)}\big\|_{\HokD}}
{\big\| u^{(2)}\big\|_{\HokD}, 
}
 \leq 
\,&\frac{1}{\min\big\{\Asomin,\nsomin\big\}}\left( 1 + 2 C^{(1)}_{\rm bound}\nsomax  k\right)
\\
&\quad \mleft(\max\set{\NLiDop{\Aso-\Ast},\NLiDRR{\nso-\nst}}\mright),
\end{align*}
and then the result \cref{eq:PDEbound} follows with 
\beq\label{eq:C3}
C_3\de \frac{1}{\min\big\{\Asomin,\nsomin\big\}}\left( \frac{1}{k_0} + 2 C^{(1)}_{\rm bound}\nsomax  \right).
\eeq
\epf

\bpf[Proof of \cref{lem:sharp}]
\label{page:lemsharpproof}
We actually prove the stronger result that given any function $c(k)$ such that $c(k)>0$ for all $k>0$, there exist 
$f, \nso,$ and $ \nst$ (with $\nso\not= \nst$) with
\beq\label{eq:nck}
\NLiDRR{\nso-\nst} \sim c(k)
\eeq
such that the corresponding solutions $u^{(1)}$ and $u^{(2)}$ of \cref{prob:vedp} with $\Aso = \Ast= I$ exist, are unique, and satisfy \cref{eq:sharp1}. 

The heart of the proof is the equation
\beq\label{eq:obs1}
(\Delta + k^2) \big(e^{i k r}\chi(r)\big) =  e^{i k r}\mleft(\Delta \chi(r) + 2ik\frac{\partial \chi}{\partial r}(r) + i k \frac{d-1}{r} \chi(r)\mright)=: -\widetilde{f}(r),
\eeq
where $\chi(r)$ is chosen to have $\supp \chi \subset D$. Observe that \cref{eq:obs1} is the Helmholtz operator applied to a circular wave $e^{ikr}$, with the added factor $\chi$ which can be chosen to have compact support. The equation \cref{eq:obs1} can be proved using the formula for the Laplacian in $d$-dimensional spherical coordinates
\beq\label{eq:sphericallaplacian}
\Delta \chi = \frac{1}{r^{d-1}} \frac{\partial}{\partial r}\mleft(r^{d-1} \frac{\partial \chi}{\partial r} \mright) + \frac1{r^2} \LapBel \chi,
\eeq
where $\LapBel$ is Laplace--Beltrami operator on the $d-1$-dimensional sphere (see, e.g., \cite[Equations (17.23) and (17.25)]{RiHoBe:97} for \cref{eq:sphericallaplacian} in $d=2$ and $3.$). Observe that $e^{ikr} \chi(r)$ has $\LapBel e^{ikr} \chi(r) = 0.$

We expect that \cref{eq:obs1} will be key in the proof of the sharpness of \cref{eq:PDEbound}, for the following reasons. Observe that \cref{eq:obs1} proves the sharpness of the nontrapping resolvent estimate \cref{eq:bound1}, since $\NLtD{\ftilde}\sim k$ and $\NHokD{e^{ikr}\chi(r)}\sim k$  and hence $\NHokD{e^{ikr}\chi(r)} \sim \NLtD{\ftilde}$ (see, e.g., \cite[Lemma 3.10]{ChMo:08},  \cite[Lemma 4.12]{Sp:14}).

Also, recall that  the nontrapping resolvent estimate \cref{eq:bound1} was used in the proof of the PDE bound \cref{eq:PDEbound} applied to $\uso-\ust.$ Therefore we expect that if we set things up so that
\beq\label{eq:sharpdiff}
\uso-\ust = e^{ikr}\chi(r),
\eeq
then  combining \cref{eq:obs1} and \cref{eq:sharpdiff} will show the sharpness of the PDE bound \cref{eq:PDEbound}. Moreover, the function $e^{ikr} \chi(r)$ was used to prove the sharpness of resolvent estimates in \cite[Discussion on p. 1445 and Lemma 3.10]{ChMo:08} and \cite[Lemma 4.12]{Sp:14}, and so we can expect it will also be effective for proving sharpness in our setting.

We now set things up so that \cref{eq:sharpdiff} holds. We define $\nso = 1$ and
\beq\label{eq:fiddlyntdone}
\nst = \nso + c(k) \chitilde(r),
\eeq
for some function $\chitilde(r)$ such that $\widetilde{\chi}\in C^{\infty}(D)$, $\widetilde{\chi}\not = 1$ (so that $\nst\not = \nso$), $\supp \, \widetilde{\chi} \subset D$, and $\NLiDRR{\chitilde} = 1$ (so that $\NLiDRR{\nso-\nst} = c(k)$).   As above, let $\chi=\chi(r)$ with $\chi \in C^{\infty}(D)$ and $\supp \,\chi$ a compact subset of $D$. We will specify $\chitilde$ and $\chi$ in more detail later.

Let $\ftilde(r)$ be as defined in \cref{eq:obs1}, and define
\beq\label{eq:obs3}
u^{(2)}(\bx)\de -\frac{1}{k^2 c(k)}\frac{\widetilde{f}(r)}{\widetilde{\chi}(r)}
\eeq
and
\beq\label{eq:fiddlyf}
f(\bx)\de -\big(\Delta +k^2 \nst(\bx)\big) u^{(2)}(\bx).
\eeq
I.e., $\ust$ solves \cref{prob:vedp} with coefficients $\Ast = I$ and $\nst$ given by \cref{eq:fiddlyntdone}, and right-hand side $f.$ We will define $\chi$ and $\chitilde$ below in such a way that $\ust \in \HoD$ and $f \in \LtD.$ In particular, we choose $\chi$ and $\chitilde$ so that if $\chitilde=0,$ then $\chi = 0$. Since $\ftilde$ depends on $\chi,$ this relation means we understand the right-hand side of \cref{eq:obs3} to be zero if $\chitilde$ is zero. In addition, since $\widetilde{\chi}(r)$ has compact support and $\ftilde$ depends on $\chi,$ we need to tie both the support of $\widetilde{\chi}$ and how fast $\widetilde{\chi}$ vanishes in a neighbourhood of its support to the definition of $\chi$ for both $u^{(2)}$ and $f$ to be well defined. As the final part of the setup, let $\uso$ solve
\beqs
\mleft(\Delta + k^2\mright) \uso = -f.
\eeqs
I.e., $\uso$ solves \cref{prob:vedp} with coefficients $\Aso = I$ and $\nso =1$ and right-hand side $f$.

Now observe that by construction (since $\nst$ is given by \cref{eq:fiddlyntdone})

\begin{align*}
  \mleft(\Delta + k^2\mright) \mleft(\uso-\ust\mright) &= \mleft(\Delta + k^2\mright)\uso - \mleft(\Delta - k^2 \nst - k^2\mleft(\nst - 1\mright)\mright)\ust\\
  &= -f + f + k^2\mleft(\nst-1\mright)\ust\\
   &= k^2 \mleft(\nst-1\mright)\ust\\
  &= k^2 c(k) \chitilde \frac{-1}{k^2 c(k)} \frac{\ftilde}{\chitilde}\\
  &= -\ftilde\\
  &= \mleft(\Delta + k^2 \mright)\mleft(e^{ikr}\chi(r)\mright).
\end{align*}
Therefore, by uniqueness of the solution of \cref{prob:vedp} \beq\label{eq:obs4}
u^{(1)}(\bx)- u^{(2)}(\bx) = e^{i k r}\chi(r).
\eeq
Therefore, by \cref{eq:obs4} and the properties of $e^{ikr} \chi(r)$ discussed above,  we have
\beqs
\big\|u^{(1)}-u^{(2)}\big\|_{L^2(D)} \sim 1
\quad \tand \quad
\big\|u^{(1)}-u^{(2)}\big\|_{\HokD} \sim k.
\eeqs
Furthermore, the definitions of $u^{(2)}$ and $\widetilde{f}$ imply that
\beqs
\big\| u^{(2)}\big\|_{L^2(D)} \sim \frac{1}{k\, c(k)} \quad\tand \quad 
\big\| u^{(2)}\big\|_{\HokD} \sim \frac{1}{c(k)},
\eeqs 
and, since $\|\nso- \nst\|_{L^\infty(D)} = c(k)$, \cref{eq:nck} holds, as required.

Therefore, to complete the proof, we only need to show that there exists a choice of $\chi$ and $\widetilde{\chi}$ for which $u^{(2)}$ and $f$ defined by \cref{eq:obs3,eq:fiddlyf} are 
in $H^{1}(D)$ and $\LtD$ respectively (in fact, we prove that they are in $W^{1,\infty}(D)$ and $L^\infty(D)$ respectively).
%well-defined. 
Because $\chi$ and $\widetilde{\chi}$ are in $C^\infty(D)$ and we choose $\chi$ and $\chitilde$ so that if $\chitilde=0,$ then $\chi=0$, the only issue is what happens at the boundary of the support of $\widetilde{\chi}$, where $u^{(2)}$ has the potential to be singular.
Since $\clos{\Dm} \subset \BR$, there exist $0<R_1<R_2<R$ such that $\overline{\Dm} \subset B_{R_2}\setminus B_{R_1} \subset \BR$. Let $\supp \chi = B_{R_2}\setminus B_{R_1}$ and let $\chi$ vanish to order $m$ at $r= R_1$ and $r=R_2$; i.e.~$\chi(r) \sim (r-R_1)^m$ as $r \downarrow R_1$ and 
$\chi(r) \sim (R_2-r)^m$ as $r \uparrow R_2$. The definition of $\widetilde{f}$ \cref{eq:obs3} then implies that $\widetilde{f}$ vanishes to order $m-2$. Let $\widetilde{\chi}(r)$ vanish to order $\mtilde$ at $r= R_1$ and $r=R_2$. 
We now claim that if $m >\mtilde+4$, then $u^{(2)}\in W^{1,\infty}(D)$ and $f$ $\in L^\infty(D)$. Indeed,  
straightforward calculation from \cref{eq:obs3} shows that  $u^{(2)}(r) \sim (r-R_1)^{m-\mtilde-2}$, $\nabla u^{(2)}(r) \sim (r-R_1)^{m-\mtilde-3}$, and $\Delta u^{(2)}(r) \sim (r-R_1)^{m-\mtilde-4}$ as $r \downarrow R_1$, with analogous behaviour at $r=R_2$.
The assumption 
$m >\mtilde+4$ therefore implies that $u^{(2)}$, $\nabla u ^{(2)}$, and $\Delta u^{(2)}$ vanish (and hence are finite) at $r=R_1$ and $r=R_2$.
\epf

\bre[Why doesn't \cref{lem:sharp} cover the case $\Aso\neq  \Ast$?]
When $\nj\de1$, $j=1,2,$, $\Aso\de I$, and $\Ast\de I + c(k)\widetilde{\chi}$, the variational problem \cref{eq:vp1} implies that 
\beq\label{eq:obs2}
\Delta \big( u^{(1)} - u^{(2)}\big) + k^2 \big( u^{(1)} - u^{(2)}\big) = c(k)\nabla\cdot \big(\widetilde{\chi}\nabla u^{(2)}\big).
\eeq
It is now much harder than in \cref{eq:obs2} to set things up so that $ u^{(1)}(\bx) - u^{(2)}(\bx)=e^{i kr}\chi(r)$ (so that one can then use \cref{eq:obs1}).
\ere

%\section{Proof of \cref{lem:2}}

\section[Extension to weaker norms]{Extension of nearby preconditioning results to weaker norms}\label{sec:weaknorm}
Recall from \cref{sec:num,sec:main} that GMRES applied to $\AmatoI \Amatt$ converges in a $k$-indepen\-dent number of iterations if $k\NLiDRR{\nso-\nst}$ is sufficiently small (with an analagous result for $\Aso-\Ast$). This result (and the related numerics) shows that $1/k$ may be a sharp threshold when we consider the maximum norm of the difference between $\nso$ and $\nst$. However, this result does not say anything if $\nso-\nst$ is merely small in some integral norm. For example if $\nso$ and $\nst$ (defined on the unit square) are given by
\beq\label{eq:noweak}
\nso(\bx) =
\begin{dcases}
  1 &\tif \bx_1 \leq \half\\
  2  &\tif \bx_1 > \half
  \end{dcases}
\eeq
and
\beq\label{eq:ntweak}
\nst(\bx) =
\begin{dcases}
  1 &\tif \bx_1 \leq \half+\alpha\\
  2  &\tif \bx_1 > \half+\alpha
  \end{dcases}
\eeq
for some $0 < \alpha < 1/2,$ then $\NLiDRR{\nso-\nst} = 1$ for all $\alpha$, but one would expect that for small $\alpha$ the corresponding solutions of \cref{prob:edp} would satisfy $\uso \approx \ust.$ In addition, one might expect that GMRES applied to $\AmatoI\Amatt$ would converge in a $k$-independent number of iterations. Therefore, in this \lcnamecref{sec:weaknorm} we seek to obtain analogues of \cref{thm:1,cor:1,cor:1a} with the difference in $\nso-\nst$ and $\Aso-\Ast$ measured in weaker norms than the $L^\infty$ norm.

The (realistic) best-case result we could obtain would be that GMRES applied to $\AmatoI\Amatt$ converges in a $k$-independent number of iterations if $\NLoDRR{\nso-\nst} \lesssim 1/k$. This result is `best' in the sense that it depends optimally on $k$; recall the discussion in \cref{rem:physical1k} that $1/k$ is the length scale governing the behaviour of Helmholtz problems. It is also `best' with regards to the norm used to measure $\nso-\nst$. When we measure $\nso-\nst$ in the $L^\infty$ norm as above, we are able to control the magnitude of $\nso-\nst$, but not the spatial variability; if $\nso-\nst \neq 0$ only on a set of small (but nonzero) measure, and $\nso-\nst=1$ on this small set, then $\NLiDRR{\nso-\nst} = 1$, regardless of the measure of the set. In contrast, the $L^1$ norm allows us to control both the magnitude of $\nso-\nst$ and the measure of the sets on which it is nonzero.

We will give numerical results indicating that this theoretical best-case result can be achieved (our numerical results actually indicate that we can obtain $k$-independent convergence when $\NLqDRR{\nso-\nst}\sim 1/k$ for any $1 \leq q < \infty$). We will also provide theory results that are, to our knowledge, the best one can prove, although they are sub-optimal in both $q$ and the dependence on $k.$


\subsection{Theory in weaker norms}\label{sec:weakertheory}
Before we prove results analogous to \cref{cor:1,cor:1a} in weaker norms (using a result analogous to \cref{thm:1} in weaker norms), we first recap why the terms  $\NLiDRRdtd{\Aso-\Ast}$ and $\NLiDRR{\nso-\nst}$ appear in \cref{thm:1}. These terms appear in \cref{thm:1} because the terms $\NLiDRR{n}$ and $\NLiDRRdtd{A}$ appear in \cref{lem:keylemma1,lem:keylemma2}, respectively. These terms appear in these \lcnamecrefs{lem:keylemma1} because in \cref{eq:mainevent1a,eq:Fbounda} we use the bounds
\beq\label{eq:keynbound}
\NLtD{n\ftilde} \leq \NLiDRR{n}\NLtD{\ftilde}
\eeq
and
\beq\label{eq:keyAbound}
\NLtD{A \grad \ftilde} \leq \NLiDRRdtd{A}\NLtD{\grad \ftilde}
\eeq
respectively, for an arbitrary function $\ftilde \in \Vhp,$ and these bounds are carried through the rest of the proof.

However, we observe that we have the following generalisation of H\"older's inequality: If $q,s > 2$ such that $1/2 = 1/q+1/s,$ then
\beq\label{eq:genholder}
\NLtD{\vo\vt} \leq \NLqD{\vo}\NLsD{\vt}.
\eeq

If we instead use \cref{eq:genholder} to bound \cref{eq:keynbound,eq:keyAbound} we obtain
\beq\label{eq:keynbound2}
\NLtD{n\ftilde} \leq \NLqDRR{n}\NLsD{\ftilde}
\eeq
and
\beq\label{eq:keyAbound2}
\NLtD{A\grad\ftilde} \leq \NLqDRRdtd{A}\NLsD{\ftilde}.
\eeq

As $\ftilde \in \Vhp$, we can apply an inverse inequality to bound $\NLsD{\ftilde}$ by $\NLtD{\ftilde}$. The required inverse inequality is (see \cite[Theorem 4.5.11 and Remark 4.5.20]{BrSc:08}
\beq\label{eq:inverses}
\NLsD{\ftilde} \leq \Cinvs h^{d\mleft(\frac1{s} - \half\mright)} \NLtD{\ftilde}.
\eeq
If we then apply \cref{eq:inverses} to \cref{eq:keynbound2,eq:keyAbound2} we obtain
\beq\label{eq:keynboundfinal}
\NLtD{n\ftilde} \leq \Cinvs \NLqDRR{n} h^{d\mleft(\frac1{s} - \half\mright)} \NLtD{\ftilde}
\eeq
and
\beq\label{eq:keyAboundfinal}
\NLtD{A\grad\ftilde} \leq \Cinvs \NLqDRRdtd{A} h^{d\mleft(\frac1{s} - \half\mright)} \NLtD{\grad\ftilde}.
\eeq

Replacing \cref{eq:mainevent1a,eq:Fbounda} with \cref{eq:keynboundfinal,eq:keyAboundfinal}, as in the proofs of \cref{lem:keylemma1,lem:keylemma2}, and proceeding as in those proofs, we can obtain the following \lcnamecrefs{cor:1alt}, the analogues of \cref{cor:1,cor:1a}.

\bth[Alternative answer to \cref{it:nbpcq1}: $k$-independent weighted GMRES iterations]\label{cor:1alt}

\

\noindent Let the assumptions of \cref{cor:1a} hold.  Given $k_0>0$ and $q >2$, let $\Cotilde$ and $\Cttilde$ be as in \cref{thm:1alt}. Then if 
% there exists $C_2>0$, independent of $h$ and $k$ (but dependent on $\Dm, \Aso, \nso$, $p$, and $k_0$) and given explicitly in \cref{eq:C2} below,
% such that if 
\beq\label{eq:condalt}
\Cotilde kh^{-\frac{d}{q}} \NLqDRRdtd{\Aso-\Ast} +\Cttilde  kh^{-\frac{d}{q}} \NLqDRR{\nso-\nst}
\leq \frac{1}{2}
\eeq
for all $k\geq k_0$, then \emph{both} weighted GMRES working in $\|\cdot\|_{\Dmat_k}$ (and the associated inner product) applied to \cref{eq:pcsystem1} \emph{and} weighted GMRES working in $\|\cdot\|_{(\Dmat_k)^{-1}}$ (and the associated inner product) applied to \cref{eq:pcsystem2}  converge in a $k$-independent number of iterations for all $k\geq k_0$.
\enth

\bth[Alternative answer to \cref{it:nbpcq1}: $k$-independent (unweighted) GMRES iterations]\label{cor:1aalt}

\

\noindent Let the assumptions of \cref{cor:1a} hold.  Given $k_0>0$, and $q >2$, let $\Cotilde$ and $\Cttilde$ be as in \cref{thm:1alt}. Then if 
% there exists $C_2>0$, independent of $h$ and $k$ (but dependent on $\Dm, \Aso, \nso$, $p$, and $k_0$) and given explicitly in \cref{eq:C2} below,
% such that if 
\beq\label{eq:condaalt}
\Cotilde \mleft(\frac{\splus}{\mminus}\mright) h^{-\frac{d}{q}-1} \NLqDRRdtd{\Aso-\Ast} + \Cttilde \mleft(\frac{\mplus}{\mminus}\mright) kh^{-\frac{d}{q}} \NLqDRR{\nso-\nst} \leq \half
\eeq
for all $k\geq k_0$, then standard GMRES (working in the Euclidean norm and inner product) applied to either of the equations \cref{eq:pcsystem1} or \cref{eq:pcsystem2}
%\beqs
%(\Amat^{(1)})^{-1}\Amat^{(2)}\uvec = \fvec\quad\text{ or } \quad\Amat^{(2)}(\Amat^{(1)})^{-1}\vvec = \fvec
%\eeqs
 converges in a $k$-independent number of iterations for all $k\geq k_0$.
 \enth

 A sketch proof of \cref{cor:1alt,cor:1aalt} is on \cpageref{page:cor1altcor1aaltproof} below.

\bre[Trade off between the type of norm and powers of $h$ and $k$]
Observe that in \cref{cor:1alt,cor:1aalt} there is a trade-off between the norm that one uses to measure $\no-\nt$ and the restriction on the magnitude of this norm. E.g., the condition on $\no-\nt$ in both \cref{cor:1alt,cor:1aalt} can be summarised as
\beq\label{eq:altsufficientlysmall}
\NLqDRR{\no-\nt} k h^{-\frac{d}{q}} \text{ is sufficiently small}.
\eeq
Observe that as $q \downarrow 2,$ we measure $\no-\nt$ in a weaker norm, but the condition \cref{eq:altsufficientlysmall} becomes more restrictive; the power of $h$ increases. Conversely, as $q \uparrow \infty,$ we measure $\no-\nt$ in a stronger norm, but the condition \cref{eq:altsufficientlysmall} becomes less restrictive; the power of $h$ decreases. (Also observe that in the $q\uparrow\infty$ limit we recover the condition \cref{eq:sufficientlysmall} we previously proved for $\NLiDRR{\no-\nt}.$
\ere

\bre[\Cref{cor:1,cor:1a} are a special case of \cref{cor:1alt,cor:1aalt}]
Observe that in the case $q=\infty$ \cref{cor:1alt,cor:1aalt} become our previous results in the $L^\infty$ norm, \cref{cor:1,cor:1a}.
\ere

The numerical experiments in \cref{sec:weakernumerics} below suggest that, at least in certain cases, a sufficient condition for nearby preconditioning to be effective is
\beq\label{eq:experimentalsufficientlysmall}
\NLqDRR{\no-\nt} k \quad\text{is sufficiently small},
\eeq
for \emph{any} $q \geq 1$, and moreover \cref{eq:experimentalsufficientlysmall} appears sharp in its $k$-dependence. (This requirement would fit with our previous observation about $1/k$ being the length scale below which perturbations cannot be seen---see \cref{rem:physical1k} above.) However, we do not say that \cref{eq:experimentalsufficientlysmall} is sufficient for all cases; recall that for transmission problems, very small perturbations in $n$ can lead to very different behaviour in the solution $u$ if $k$ is a quasi-resonance for $\no$ or $\nt$; see the discussion at the end of \cref{sec:wpdisc} above.


\subsubsection{Proof of \cref{cor:1alt,cor:1aalt}}

We first state analogues of \cref{lem:keylemma1,lem:keylemma2} in weaker norms; these \lcnamecrefs{lem:keylemma1} are the key to the proofs of \cref{cor:1alt,cor:1aalt} above. The essence of the proofs of \cref{lem:keylemma1a,lem:keylemma2a} are the discussion at the start of \cref{sec:weakertheory}.

\ble[Alternative bounds on $(\Amato)^{-1} \Mmat_{n}$]\label{lem:keylemma1a}
Under the assumptions of \cref{lem:keylemma1}, for $n\in \LiDRR$ and for any $q > 2$,
\beq\label{eq:keybound12}
\max\set{\NDk{\AmatoI \Mmatn},\NDkI{\Mmatn\AmatoI}} \leq \Cttilde h^{-\frac{d}{q}} \frac{\NLqDRR{n}}k
\eeq
and 
\beq\label{eq:keybound1a2}
\max\set{\Nt{\AmatoI \Mmatn},\Nt{\Mmatn\AmatoI}} \leq \Cttilde\mleft(\frac{\mplus}{\mminus}\mright) h^{-\frac{d}q} \frac{\NLqDRR{n}}k
\eeq
for all $k\geq \kz$,
where
\beq\label{eq:C2tilde}
\Cttilde\de%\frac{m_+}{m_-} 
%\left[ 
\Cinvs\Ct,
\eeq
where $\Ct$ is defined by \cref{eq:C2} and $1/s = 1/2 - 1/q.$
\ele

\ble[Alternative bounds on $(\Amato)^{-1} \Smat_A$]\label{lem:keylemma2a}
Under the assumptions of \cref{lem:keylemma2}, for $A\in L^\infty(D,\RR^{d\times d})$ and for any $q > 2$
\beq\label{eq:keybound22}
\max\set{\NDk{\AmatoI \SmatA},\NDkI{\SmatA\AmatoI}} \leq \Cotilde h^{-\frac{d}q}k \NLqDRRdtd{A}
\eeq
and
\beq\label{eq:keybound2a2}
\max\set{\Nt{\AmatoI \SmatA},\Nt{\SmatA\AmatoI}} \leq \Cotilde\mleft(\frac{\splus}{\mminus}\mright) h^{-\frac{d}q-1} \NLqDRR{A}
\eeq
%\begin{align}\nonumber
%&\max\Big\{\big\| (\Amato)^{-1} \Smat_A \big\|_2, \,\,
%\big\| \Smat_A (\Amato)^{-1} \big\|_2\Big\}\nonumber \\
%&\hspace{2cm}
% \leq \frac{s_+}{s_-} \left[ C_{\rm FEM2}^{(1)} + 
% \frac{1}{\min\big\{\Asomin,\nsomin\big\}}\left( \frac{1}{k_0} + 2 C^{(1)}_{\rm bound}\nsomax  \right) \right]k\N{A}_{L^\infty(D)}\label{eq:keybound2}
%% + C_{\rm bound}^{(1)}\right) \frac{\N{n}_{L^\infty(D)}}{k}.
%\end{align}
for all $k\geq k_0$, where
\beq\label{eq:C1tildenbpc}
\Cotilde \de \Cinvs\Co,
\eeq
where $\Co$ is given by \cref{eq:C1nbpc} and $1/s = 1/2 - 1/q.$
\ele

The proofs of \cref{lem:keylemma1a,lem:keylemma2a} are virtually identical to the proofs of \cref{lem:keylemma1,lem:keylemma2}, with the modifications for $L^q$ norms detailed at the beginning of \cref{sec:weakertheory}.

\bre[Reduction to \cref{lem:keylemma1,lem:keylemma2}]
Observe that in the case $s=2$ and $q=\infty$ \cref{lem:keylemma1a,lem:keylemma2a} reduce to the our previous results \cref{lem:keylemma1,lem:keylemma2}.
\ere

We can use \cref{lem:keylemma1a,lem:keylemma2a} in place of \cref{lem:keylemma1,lem:keylemma2} to obtain the following analogue of \cref{thm:1} in weaker norms.

\begin{theorem}[Alternative main ingredient to answer to \cref{it:nbpcq1}]\label{thm:1alt}
If all the assumptions of \namecref{thm:1} \ref{thm:1} hold, then, given $\kz>0$ and $q >2$, there exist $\Cotilde, \Cttilde>0$, independent of $h$ and $k$ (but dependent on $d, \Dm, \Aso, \nso$, $p$, $q$, and $\kz$) such that
\begin{align}\nonumber
&\max\set{\NDk{\Imat - \AmatoI\Amatt},\NDkI{\Imat -\Amatt\AmatoI}}\\
&\hspace{3cm} 
\leq \Cotilde kh^{-\frac{d}q} \NLqDRRdtd{\Aso-\Ast} + \Cttilde  kh^{-\frac{d}q}  \NLqDRR{\nso-\nst}
\label{eq:main1alt}
\end{align}
and 
\begin{align}\nonumber
&\max\set{\Nt{\Imat - \AmatoI\Amatt}, \Nt{\Imat -\Amatt\AmatoI}}\\
&\hspace{0cm}
\leq \Cotilde \mleft(\frac{\splus}{\mminus}\mright) h^{-\frac{d}q-1}\NLqDRRdtd{\Aso-\Ast} + \Cttilde \mleft(\frac{\mplus}{\mminus}\mright) kh^{-\frac{d}q}\NLqDRR{\nso-\nst}
\label{eq:main1aalt}
\end{align}
for all $k\geq k_0$. 
\end{theorem}

The proof of \cref{thm:1alt} is identical to the proof of \cref{thm:1}, with \cref{lem:keylemma1,lem:keylemma2} replaced by \cref{lem:keylemma1a,lem:keylemma2a}.

%We can now use \cref{thm:1alt} to obtain the following analogues to \cref{cor:1,cor:1a} in weaker norms.

\bpf[Sketch proof of \cref{cor:1alt,cor:1aalt}]
\label{page:cor1altcor1aaltproof}
The proofs of \cref{cor:1alt,cor:1aalt} are completely analagous to the proofs of \cref{cor:1,cor:1a}, with the exception that we use \cref{thm:1alt} in place of \cref{thm:1}.
\epf


\subsection{Numerics in weaker norms}\label{sec:weakernumerics}
For our computations, we use the computational setup as in \cref{app:compsetup}, with $f$ and $\gI$ corresponding to a plane wave passing through homogeneous media.  We let $\Aso=\Ast=I,$ and we define $\nso$ and $\nst$ by \cref{eq:noweak,eq:ntweak}. For $\alpha = 0.2k^{-\beta},$ $\beta = 0,0.1,\ldots,0.9,1$ and for $k=10,20,\ldots,100$ we used GMRES to solve $\AmatoI\Amatt = \AmatoI \fvec$ (for $\fvec$ given by the Helmholtz problem), and we record the number of GMRES iterations taken to achieve convergence.

Our results in \cref{fig:l1low,fig:l1med,fig:l1high} (also displayed in \cref{tab:l1}) indicate the following conclusions for $\NLqDRR{\nso-\nst} \sim 0.1/k^{\beta}$, for all $1 \leq q < \infty$:
\bit
\item For $\beta \in (0,0.6)$ there is clear growth of the number of GMRES iterations with $k$,
\item For $\beta = 1$ there is clear boundedness of the number of GMRES iterations with $k$, and
  \item for $\beta \in (0.7,0.9)$ it is unclear if the number of GMRES iterations grows with $k.$
    \eit
We note that the results in \cref{fig:l1low,fig:l1med,fig:l1high} are the analogues of those in \cref{fig:linfinityn0,fig:linfinityn1,fig:linfinityn2}.

If we compare our numerical results with the theory results in \cref{cor:1aalt}, we see that the theory (if $h \sim k^{-3/2}$ and $d=2$, as in our computational experiments) predicts that the number of iterations will remain bounded if $\NLqDRR{\nso-\nst} k^{1+3/q}$ is sufficiently small, for any $q > 2.$ Our computed results indicate that this result is not sharp. The computed results indicate that if $\NLqDRR{\nso-\nst} \sim k^{-1}$ for any $q \geq 1,$ then the number of GMRES iterations is bounded as $k$ increases. Observe again that the `best case' $1/k$ condition is only predicted by the theory in the $q\rightarrow \infty$ limit.

\begin{figure}
%% Creator: Matplotlib, PGF backend
%%
%% To include the figure in your LaTeX document, write
%%   \input{<filename>.pgf}
%%
%% Make sure the required packages are loaded in your preamble
%%   \usepackage{pgf}
%%
%% Figures using additional raster images can only be included by \input if
%% they are in the same directory as the main LaTeX file. For loading figures
%% from other directories you can use the `import` package
%%   \usepackage{import}
%% and then include the figures with
%%   \import{<path to file>}{<filename>.pgf}
%%
%% Matplotlib used the following preamble
%%   \usepackage{fontspec}
%%   \setmainfont{DejaVuSerif.ttf}[Path=/home/owen/progs/firedrake-complex/firedrake/lib/python3.5/site-packages/matplotlib/mpl-data/fonts/ttf/]
%%   \setsansfont{DejaVuSans.ttf}[Path=/home/owen/progs/firedrake-complex/firedrake/lib/python3.5/site-packages/matplotlib/mpl-data/fonts/ttf/]
%%   \setmonofont{DejaVuSansMono.ttf}[Path=/home/owen/progs/firedrake-complex/firedrake/lib/python3.5/site-packages/matplotlib/mpl-data/fonts/ttf/]
%%
\begingroup%
\makeatletter%
\begin{pgfpicture}%
\pgfpathrectangle{\pgfpointorigin}{\pgfqpoint{6.400000in}{4.800000in}}%
\pgfusepath{use as bounding box, clip}%
\begin{pgfscope}%
\pgfsetbuttcap%
\pgfsetmiterjoin%
\definecolor{currentfill}{rgb}{1.000000,1.000000,1.000000}%
\pgfsetfillcolor{currentfill}%
\pgfsetlinewidth{0.000000pt}%
\definecolor{currentstroke}{rgb}{1.000000,1.000000,1.000000}%
\pgfsetstrokecolor{currentstroke}%
\pgfsetdash{}{0pt}%
\pgfpathmoveto{\pgfqpoint{0.000000in}{0.000000in}}%
\pgfpathlineto{\pgfqpoint{6.400000in}{0.000000in}}%
\pgfpathlineto{\pgfqpoint{6.400000in}{4.800000in}}%
\pgfpathlineto{\pgfqpoint{0.000000in}{4.800000in}}%
\pgfpathclose%
\pgfusepath{fill}%
\end{pgfscope}%
\begin{pgfscope}%
\pgfsetbuttcap%
\pgfsetmiterjoin%
\definecolor{currentfill}{rgb}{1.000000,1.000000,1.000000}%
\pgfsetfillcolor{currentfill}%
\pgfsetlinewidth{0.000000pt}%
\definecolor{currentstroke}{rgb}{0.000000,0.000000,0.000000}%
\pgfsetstrokecolor{currentstroke}%
\pgfsetstrokeopacity{0.000000}%
\pgfsetdash{}{0pt}%
\pgfpathmoveto{\pgfqpoint{0.800000in}{0.528000in}}%
\pgfpathlineto{\pgfqpoint{5.760000in}{0.528000in}}%
\pgfpathlineto{\pgfqpoint{5.760000in}{4.224000in}}%
\pgfpathlineto{\pgfqpoint{0.800000in}{4.224000in}}%
\pgfpathclose%
\pgfusepath{fill}%
\end{pgfscope}%
\begin{pgfscope}%
\pgfsetbuttcap%
\pgfsetroundjoin%
\definecolor{currentfill}{rgb}{0.000000,0.000000,0.000000}%
\pgfsetfillcolor{currentfill}%
\pgfsetlinewidth{0.803000pt}%
\definecolor{currentstroke}{rgb}{0.000000,0.000000,0.000000}%
\pgfsetstrokecolor{currentstroke}%
\pgfsetdash{}{0pt}%
\pgfsys@defobject{currentmarker}{\pgfqpoint{0.000000in}{-0.048611in}}{\pgfqpoint{0.000000in}{0.000000in}}{%
\pgfpathmoveto{\pgfqpoint{0.000000in}{0.000000in}}%
\pgfpathlineto{\pgfqpoint{0.000000in}{-0.048611in}}%
\pgfusepath{stroke,fill}%
}%
\begin{pgfscope}%
\pgfsys@transformshift{1.250909in}{0.528000in}%
\pgfsys@useobject{currentmarker}{}%
\end{pgfscope}%
\end{pgfscope}%
\begin{pgfscope}%
\definecolor{textcolor}{rgb}{0.000000,0.000000,0.000000}%
\pgfsetstrokecolor{textcolor}%
\pgfsetfillcolor{textcolor}%
\pgftext[x=1.250909in,y=0.430778in,,top]{\color{textcolor}\sffamily\fontsize{10.000000}{12.000000}\selectfont \(\displaystyle 10\)}%
\end{pgfscope}%
\begin{pgfscope}%
\pgfsetbuttcap%
\pgfsetroundjoin%
\definecolor{currentfill}{rgb}{0.000000,0.000000,0.000000}%
\pgfsetfillcolor{currentfill}%
\pgfsetlinewidth{0.803000pt}%
\definecolor{currentstroke}{rgb}{0.000000,0.000000,0.000000}%
\pgfsetstrokecolor{currentstroke}%
\pgfsetdash{}{0pt}%
\pgfsys@defobject{currentmarker}{\pgfqpoint{0.000000in}{-0.048611in}}{\pgfqpoint{0.000000in}{0.000000in}}{%
\pgfpathmoveto{\pgfqpoint{0.000000in}{0.000000in}}%
\pgfpathlineto{\pgfqpoint{0.000000in}{-0.048611in}}%
\pgfusepath{stroke,fill}%
}%
\begin{pgfscope}%
\pgfsys@transformshift{1.701818in}{0.528000in}%
\pgfsys@useobject{currentmarker}{}%
\end{pgfscope}%
\end{pgfscope}%
\begin{pgfscope}%
\definecolor{textcolor}{rgb}{0.000000,0.000000,0.000000}%
\pgfsetstrokecolor{textcolor}%
\pgfsetfillcolor{textcolor}%
\pgftext[x=1.701818in,y=0.430778in,,top]{\color{textcolor}\sffamily\fontsize{10.000000}{12.000000}\selectfont \(\displaystyle 20\)}%
\end{pgfscope}%
\begin{pgfscope}%
\pgfsetbuttcap%
\pgfsetroundjoin%
\definecolor{currentfill}{rgb}{0.000000,0.000000,0.000000}%
\pgfsetfillcolor{currentfill}%
\pgfsetlinewidth{0.803000pt}%
\definecolor{currentstroke}{rgb}{0.000000,0.000000,0.000000}%
\pgfsetstrokecolor{currentstroke}%
\pgfsetdash{}{0pt}%
\pgfsys@defobject{currentmarker}{\pgfqpoint{0.000000in}{-0.048611in}}{\pgfqpoint{0.000000in}{0.000000in}}{%
\pgfpathmoveto{\pgfqpoint{0.000000in}{0.000000in}}%
\pgfpathlineto{\pgfqpoint{0.000000in}{-0.048611in}}%
\pgfusepath{stroke,fill}%
}%
\begin{pgfscope}%
\pgfsys@transformshift{2.152727in}{0.528000in}%
\pgfsys@useobject{currentmarker}{}%
\end{pgfscope}%
\end{pgfscope}%
\begin{pgfscope}%
\definecolor{textcolor}{rgb}{0.000000,0.000000,0.000000}%
\pgfsetstrokecolor{textcolor}%
\pgfsetfillcolor{textcolor}%
\pgftext[x=2.152727in,y=0.430778in,,top]{\color{textcolor}\sffamily\fontsize{10.000000}{12.000000}\selectfont \(\displaystyle 30\)}%
\end{pgfscope}%
\begin{pgfscope}%
\pgfsetbuttcap%
\pgfsetroundjoin%
\definecolor{currentfill}{rgb}{0.000000,0.000000,0.000000}%
\pgfsetfillcolor{currentfill}%
\pgfsetlinewidth{0.803000pt}%
\definecolor{currentstroke}{rgb}{0.000000,0.000000,0.000000}%
\pgfsetstrokecolor{currentstroke}%
\pgfsetdash{}{0pt}%
\pgfsys@defobject{currentmarker}{\pgfqpoint{0.000000in}{-0.048611in}}{\pgfqpoint{0.000000in}{0.000000in}}{%
\pgfpathmoveto{\pgfqpoint{0.000000in}{0.000000in}}%
\pgfpathlineto{\pgfqpoint{0.000000in}{-0.048611in}}%
\pgfusepath{stroke,fill}%
}%
\begin{pgfscope}%
\pgfsys@transformshift{2.603636in}{0.528000in}%
\pgfsys@useobject{currentmarker}{}%
\end{pgfscope}%
\end{pgfscope}%
\begin{pgfscope}%
\definecolor{textcolor}{rgb}{0.000000,0.000000,0.000000}%
\pgfsetstrokecolor{textcolor}%
\pgfsetfillcolor{textcolor}%
\pgftext[x=2.603636in,y=0.430778in,,top]{\color{textcolor}\sffamily\fontsize{10.000000}{12.000000}\selectfont \(\displaystyle 40\)}%
\end{pgfscope}%
\begin{pgfscope}%
\pgfsetbuttcap%
\pgfsetroundjoin%
\definecolor{currentfill}{rgb}{0.000000,0.000000,0.000000}%
\pgfsetfillcolor{currentfill}%
\pgfsetlinewidth{0.803000pt}%
\definecolor{currentstroke}{rgb}{0.000000,0.000000,0.000000}%
\pgfsetstrokecolor{currentstroke}%
\pgfsetdash{}{0pt}%
\pgfsys@defobject{currentmarker}{\pgfqpoint{0.000000in}{-0.048611in}}{\pgfqpoint{0.000000in}{0.000000in}}{%
\pgfpathmoveto{\pgfqpoint{0.000000in}{0.000000in}}%
\pgfpathlineto{\pgfqpoint{0.000000in}{-0.048611in}}%
\pgfusepath{stroke,fill}%
}%
\begin{pgfscope}%
\pgfsys@transformshift{3.054545in}{0.528000in}%
\pgfsys@useobject{currentmarker}{}%
\end{pgfscope}%
\end{pgfscope}%
\begin{pgfscope}%
\definecolor{textcolor}{rgb}{0.000000,0.000000,0.000000}%
\pgfsetstrokecolor{textcolor}%
\pgfsetfillcolor{textcolor}%
\pgftext[x=3.054545in,y=0.430778in,,top]{\color{textcolor}\sffamily\fontsize{10.000000}{12.000000}\selectfont \(\displaystyle 50\)}%
\end{pgfscope}%
\begin{pgfscope}%
\pgfsetbuttcap%
\pgfsetroundjoin%
\definecolor{currentfill}{rgb}{0.000000,0.000000,0.000000}%
\pgfsetfillcolor{currentfill}%
\pgfsetlinewidth{0.803000pt}%
\definecolor{currentstroke}{rgb}{0.000000,0.000000,0.000000}%
\pgfsetstrokecolor{currentstroke}%
\pgfsetdash{}{0pt}%
\pgfsys@defobject{currentmarker}{\pgfqpoint{0.000000in}{-0.048611in}}{\pgfqpoint{0.000000in}{0.000000in}}{%
\pgfpathmoveto{\pgfqpoint{0.000000in}{0.000000in}}%
\pgfpathlineto{\pgfqpoint{0.000000in}{-0.048611in}}%
\pgfusepath{stroke,fill}%
}%
\begin{pgfscope}%
\pgfsys@transformshift{3.505455in}{0.528000in}%
\pgfsys@useobject{currentmarker}{}%
\end{pgfscope}%
\end{pgfscope}%
\begin{pgfscope}%
\definecolor{textcolor}{rgb}{0.000000,0.000000,0.000000}%
\pgfsetstrokecolor{textcolor}%
\pgfsetfillcolor{textcolor}%
\pgftext[x=3.505455in,y=0.430778in,,top]{\color{textcolor}\sffamily\fontsize{10.000000}{12.000000}\selectfont \(\displaystyle 60\)}%
\end{pgfscope}%
\begin{pgfscope}%
\pgfsetbuttcap%
\pgfsetroundjoin%
\definecolor{currentfill}{rgb}{0.000000,0.000000,0.000000}%
\pgfsetfillcolor{currentfill}%
\pgfsetlinewidth{0.803000pt}%
\definecolor{currentstroke}{rgb}{0.000000,0.000000,0.000000}%
\pgfsetstrokecolor{currentstroke}%
\pgfsetdash{}{0pt}%
\pgfsys@defobject{currentmarker}{\pgfqpoint{0.000000in}{-0.048611in}}{\pgfqpoint{0.000000in}{0.000000in}}{%
\pgfpathmoveto{\pgfqpoint{0.000000in}{0.000000in}}%
\pgfpathlineto{\pgfqpoint{0.000000in}{-0.048611in}}%
\pgfusepath{stroke,fill}%
}%
\begin{pgfscope}%
\pgfsys@transformshift{3.956364in}{0.528000in}%
\pgfsys@useobject{currentmarker}{}%
\end{pgfscope}%
\end{pgfscope}%
\begin{pgfscope}%
\definecolor{textcolor}{rgb}{0.000000,0.000000,0.000000}%
\pgfsetstrokecolor{textcolor}%
\pgfsetfillcolor{textcolor}%
\pgftext[x=3.956364in,y=0.430778in,,top]{\color{textcolor}\sffamily\fontsize{10.000000}{12.000000}\selectfont \(\displaystyle 70\)}%
\end{pgfscope}%
\begin{pgfscope}%
\pgfsetbuttcap%
\pgfsetroundjoin%
\definecolor{currentfill}{rgb}{0.000000,0.000000,0.000000}%
\pgfsetfillcolor{currentfill}%
\pgfsetlinewidth{0.803000pt}%
\definecolor{currentstroke}{rgb}{0.000000,0.000000,0.000000}%
\pgfsetstrokecolor{currentstroke}%
\pgfsetdash{}{0pt}%
\pgfsys@defobject{currentmarker}{\pgfqpoint{0.000000in}{-0.048611in}}{\pgfqpoint{0.000000in}{0.000000in}}{%
\pgfpathmoveto{\pgfqpoint{0.000000in}{0.000000in}}%
\pgfpathlineto{\pgfqpoint{0.000000in}{-0.048611in}}%
\pgfusepath{stroke,fill}%
}%
\begin{pgfscope}%
\pgfsys@transformshift{4.407273in}{0.528000in}%
\pgfsys@useobject{currentmarker}{}%
\end{pgfscope}%
\end{pgfscope}%
\begin{pgfscope}%
\definecolor{textcolor}{rgb}{0.000000,0.000000,0.000000}%
\pgfsetstrokecolor{textcolor}%
\pgfsetfillcolor{textcolor}%
\pgftext[x=4.407273in,y=0.430778in,,top]{\color{textcolor}\sffamily\fontsize{10.000000}{12.000000}\selectfont \(\displaystyle 80\)}%
\end{pgfscope}%
\begin{pgfscope}%
\pgfsetbuttcap%
\pgfsetroundjoin%
\definecolor{currentfill}{rgb}{0.000000,0.000000,0.000000}%
\pgfsetfillcolor{currentfill}%
\pgfsetlinewidth{0.803000pt}%
\definecolor{currentstroke}{rgb}{0.000000,0.000000,0.000000}%
\pgfsetstrokecolor{currentstroke}%
\pgfsetdash{}{0pt}%
\pgfsys@defobject{currentmarker}{\pgfqpoint{0.000000in}{-0.048611in}}{\pgfqpoint{0.000000in}{0.000000in}}{%
\pgfpathmoveto{\pgfqpoint{0.000000in}{0.000000in}}%
\pgfpathlineto{\pgfqpoint{0.000000in}{-0.048611in}}%
\pgfusepath{stroke,fill}%
}%
\begin{pgfscope}%
\pgfsys@transformshift{4.858182in}{0.528000in}%
\pgfsys@useobject{currentmarker}{}%
\end{pgfscope}%
\end{pgfscope}%
\begin{pgfscope}%
\definecolor{textcolor}{rgb}{0.000000,0.000000,0.000000}%
\pgfsetstrokecolor{textcolor}%
\pgfsetfillcolor{textcolor}%
\pgftext[x=4.858182in,y=0.430778in,,top]{\color{textcolor}\sffamily\fontsize{10.000000}{12.000000}\selectfont \(\displaystyle 90\)}%
\end{pgfscope}%
\begin{pgfscope}%
\pgfsetbuttcap%
\pgfsetroundjoin%
\definecolor{currentfill}{rgb}{0.000000,0.000000,0.000000}%
\pgfsetfillcolor{currentfill}%
\pgfsetlinewidth{0.803000pt}%
\definecolor{currentstroke}{rgb}{0.000000,0.000000,0.000000}%
\pgfsetstrokecolor{currentstroke}%
\pgfsetdash{}{0pt}%
\pgfsys@defobject{currentmarker}{\pgfqpoint{0.000000in}{-0.048611in}}{\pgfqpoint{0.000000in}{0.000000in}}{%
\pgfpathmoveto{\pgfqpoint{0.000000in}{0.000000in}}%
\pgfpathlineto{\pgfqpoint{0.000000in}{-0.048611in}}%
\pgfusepath{stroke,fill}%
}%
\begin{pgfscope}%
\pgfsys@transformshift{5.309091in}{0.528000in}%
\pgfsys@useobject{currentmarker}{}%
\end{pgfscope}%
\end{pgfscope}%
\begin{pgfscope}%
\definecolor{textcolor}{rgb}{0.000000,0.000000,0.000000}%
\pgfsetstrokecolor{textcolor}%
\pgfsetfillcolor{textcolor}%
\pgftext[x=5.309091in,y=0.430778in,,top]{\color{textcolor}\sffamily\fontsize{10.000000}{12.000000}\selectfont \(\displaystyle 100\)}%
\end{pgfscope}%
\begin{pgfscope}%
\definecolor{textcolor}{rgb}{0.000000,0.000000,0.000000}%
\pgfsetstrokecolor{textcolor}%
\pgfsetfillcolor{textcolor}%
\pgftext[x=3.280000in,y=0.240809in,,top]{\color{textcolor}\sffamily\fontsize{10.000000}{12.000000}\selectfont \(\displaystyle k\)}%
\end{pgfscope}%
\begin{pgfscope}%
\pgfsetbuttcap%
\pgfsetroundjoin%
\definecolor{currentfill}{rgb}{0.000000,0.000000,0.000000}%
\pgfsetfillcolor{currentfill}%
\pgfsetlinewidth{0.803000pt}%
\definecolor{currentstroke}{rgb}{0.000000,0.000000,0.000000}%
\pgfsetstrokecolor{currentstroke}%
\pgfsetdash{}{0pt}%
\pgfsys@defobject{currentmarker}{\pgfqpoint{-0.048611in}{0.000000in}}{\pgfqpoint{0.000000in}{0.000000in}}{%
\pgfpathmoveto{\pgfqpoint{0.000000in}{0.000000in}}%
\pgfpathlineto{\pgfqpoint{-0.048611in}{0.000000in}}%
\pgfusepath{stroke,fill}%
}%
\begin{pgfscope}%
\pgfsys@transformshift{0.800000in}{0.678442in}%
\pgfsys@useobject{currentmarker}{}%
\end{pgfscope}%
\end{pgfscope}%
\begin{pgfscope}%
\definecolor{textcolor}{rgb}{0.000000,0.000000,0.000000}%
\pgfsetstrokecolor{textcolor}%
\pgfsetfillcolor{textcolor}%
\pgftext[x=0.633333in,y=0.625680in,left,base]{\color{textcolor}\sffamily\fontsize{10.000000}{12.000000}\selectfont \(\displaystyle 0\)}%
\end{pgfscope}%
\begin{pgfscope}%
\pgfsetbuttcap%
\pgfsetroundjoin%
\definecolor{currentfill}{rgb}{0.000000,0.000000,0.000000}%
\pgfsetfillcolor{currentfill}%
\pgfsetlinewidth{0.803000pt}%
\definecolor{currentstroke}{rgb}{0.000000,0.000000,0.000000}%
\pgfsetstrokecolor{currentstroke}%
\pgfsetdash{}{0pt}%
\pgfsys@defobject{currentmarker}{\pgfqpoint{-0.048611in}{0.000000in}}{\pgfqpoint{0.000000in}{0.000000in}}{%
\pgfpathmoveto{\pgfqpoint{0.000000in}{0.000000in}}%
\pgfpathlineto{\pgfqpoint{-0.048611in}{0.000000in}}%
\pgfusepath{stroke,fill}%
}%
\begin{pgfscope}%
\pgfsys@transformshift{0.800000in}{1.077492in}%
\pgfsys@useobject{currentmarker}{}%
\end{pgfscope}%
\end{pgfscope}%
\begin{pgfscope}%
\definecolor{textcolor}{rgb}{0.000000,0.000000,0.000000}%
\pgfsetstrokecolor{textcolor}%
\pgfsetfillcolor{textcolor}%
\pgftext[x=0.494444in,y=1.024730in,left,base]{\color{textcolor}\sffamily\fontsize{10.000000}{12.000000}\selectfont \(\displaystyle 250\)}%
\end{pgfscope}%
\begin{pgfscope}%
\pgfsetbuttcap%
\pgfsetroundjoin%
\definecolor{currentfill}{rgb}{0.000000,0.000000,0.000000}%
\pgfsetfillcolor{currentfill}%
\pgfsetlinewidth{0.803000pt}%
\definecolor{currentstroke}{rgb}{0.000000,0.000000,0.000000}%
\pgfsetstrokecolor{currentstroke}%
\pgfsetdash{}{0pt}%
\pgfsys@defobject{currentmarker}{\pgfqpoint{-0.048611in}{0.000000in}}{\pgfqpoint{0.000000in}{0.000000in}}{%
\pgfpathmoveto{\pgfqpoint{0.000000in}{0.000000in}}%
\pgfpathlineto{\pgfqpoint{-0.048611in}{0.000000in}}%
\pgfusepath{stroke,fill}%
}%
\begin{pgfscope}%
\pgfsys@transformshift{0.800000in}{1.476542in}%
\pgfsys@useobject{currentmarker}{}%
\end{pgfscope}%
\end{pgfscope}%
\begin{pgfscope}%
\definecolor{textcolor}{rgb}{0.000000,0.000000,0.000000}%
\pgfsetstrokecolor{textcolor}%
\pgfsetfillcolor{textcolor}%
\pgftext[x=0.494444in,y=1.423780in,left,base]{\color{textcolor}\sffamily\fontsize{10.000000}{12.000000}\selectfont \(\displaystyle 500\)}%
\end{pgfscope}%
\begin{pgfscope}%
\pgfsetbuttcap%
\pgfsetroundjoin%
\definecolor{currentfill}{rgb}{0.000000,0.000000,0.000000}%
\pgfsetfillcolor{currentfill}%
\pgfsetlinewidth{0.803000pt}%
\definecolor{currentstroke}{rgb}{0.000000,0.000000,0.000000}%
\pgfsetstrokecolor{currentstroke}%
\pgfsetdash{}{0pt}%
\pgfsys@defobject{currentmarker}{\pgfqpoint{-0.048611in}{0.000000in}}{\pgfqpoint{0.000000in}{0.000000in}}{%
\pgfpathmoveto{\pgfqpoint{0.000000in}{0.000000in}}%
\pgfpathlineto{\pgfqpoint{-0.048611in}{0.000000in}}%
\pgfusepath{stroke,fill}%
}%
\begin{pgfscope}%
\pgfsys@transformshift{0.800000in}{1.875591in}%
\pgfsys@useobject{currentmarker}{}%
\end{pgfscope}%
\end{pgfscope}%
\begin{pgfscope}%
\definecolor{textcolor}{rgb}{0.000000,0.000000,0.000000}%
\pgfsetstrokecolor{textcolor}%
\pgfsetfillcolor{textcolor}%
\pgftext[x=0.494444in,y=1.822830in,left,base]{\color{textcolor}\sffamily\fontsize{10.000000}{12.000000}\selectfont \(\displaystyle 750\)}%
\end{pgfscope}%
\begin{pgfscope}%
\pgfsetbuttcap%
\pgfsetroundjoin%
\definecolor{currentfill}{rgb}{0.000000,0.000000,0.000000}%
\pgfsetfillcolor{currentfill}%
\pgfsetlinewidth{0.803000pt}%
\definecolor{currentstroke}{rgb}{0.000000,0.000000,0.000000}%
\pgfsetstrokecolor{currentstroke}%
\pgfsetdash{}{0pt}%
\pgfsys@defobject{currentmarker}{\pgfqpoint{-0.048611in}{0.000000in}}{\pgfqpoint{0.000000in}{0.000000in}}{%
\pgfpathmoveto{\pgfqpoint{0.000000in}{0.000000in}}%
\pgfpathlineto{\pgfqpoint{-0.048611in}{0.000000in}}%
\pgfusepath{stroke,fill}%
}%
\begin{pgfscope}%
\pgfsys@transformshift{0.800000in}{2.274641in}%
\pgfsys@useobject{currentmarker}{}%
\end{pgfscope}%
\end{pgfscope}%
\begin{pgfscope}%
\definecolor{textcolor}{rgb}{0.000000,0.000000,0.000000}%
\pgfsetstrokecolor{textcolor}%
\pgfsetfillcolor{textcolor}%
\pgftext[x=0.424999in,y=2.221880in,left,base]{\color{textcolor}\sffamily\fontsize{10.000000}{12.000000}\selectfont \(\displaystyle 1000\)}%
\end{pgfscope}%
\begin{pgfscope}%
\pgfsetbuttcap%
\pgfsetroundjoin%
\definecolor{currentfill}{rgb}{0.000000,0.000000,0.000000}%
\pgfsetfillcolor{currentfill}%
\pgfsetlinewidth{0.803000pt}%
\definecolor{currentstroke}{rgb}{0.000000,0.000000,0.000000}%
\pgfsetstrokecolor{currentstroke}%
\pgfsetdash{}{0pt}%
\pgfsys@defobject{currentmarker}{\pgfqpoint{-0.048611in}{0.000000in}}{\pgfqpoint{0.000000in}{0.000000in}}{%
\pgfpathmoveto{\pgfqpoint{0.000000in}{0.000000in}}%
\pgfpathlineto{\pgfqpoint{-0.048611in}{0.000000in}}%
\pgfusepath{stroke,fill}%
}%
\begin{pgfscope}%
\pgfsys@transformshift{0.800000in}{2.673691in}%
\pgfsys@useobject{currentmarker}{}%
\end{pgfscope}%
\end{pgfscope}%
\begin{pgfscope}%
\definecolor{textcolor}{rgb}{0.000000,0.000000,0.000000}%
\pgfsetstrokecolor{textcolor}%
\pgfsetfillcolor{textcolor}%
\pgftext[x=0.424999in,y=2.620930in,left,base]{\color{textcolor}\sffamily\fontsize{10.000000}{12.000000}\selectfont \(\displaystyle 1250\)}%
\end{pgfscope}%
\begin{pgfscope}%
\pgfsetbuttcap%
\pgfsetroundjoin%
\definecolor{currentfill}{rgb}{0.000000,0.000000,0.000000}%
\pgfsetfillcolor{currentfill}%
\pgfsetlinewidth{0.803000pt}%
\definecolor{currentstroke}{rgb}{0.000000,0.000000,0.000000}%
\pgfsetstrokecolor{currentstroke}%
\pgfsetdash{}{0pt}%
\pgfsys@defobject{currentmarker}{\pgfqpoint{-0.048611in}{0.000000in}}{\pgfqpoint{0.000000in}{0.000000in}}{%
\pgfpathmoveto{\pgfqpoint{0.000000in}{0.000000in}}%
\pgfpathlineto{\pgfqpoint{-0.048611in}{0.000000in}}%
\pgfusepath{stroke,fill}%
}%
\begin{pgfscope}%
\pgfsys@transformshift{0.800000in}{3.072741in}%
\pgfsys@useobject{currentmarker}{}%
\end{pgfscope}%
\end{pgfscope}%
\begin{pgfscope}%
\definecolor{textcolor}{rgb}{0.000000,0.000000,0.000000}%
\pgfsetstrokecolor{textcolor}%
\pgfsetfillcolor{textcolor}%
\pgftext[x=0.424999in,y=3.019980in,left,base]{\color{textcolor}\sffamily\fontsize{10.000000}{12.000000}\selectfont \(\displaystyle 1500\)}%
\end{pgfscope}%
\begin{pgfscope}%
\pgfsetbuttcap%
\pgfsetroundjoin%
\definecolor{currentfill}{rgb}{0.000000,0.000000,0.000000}%
\pgfsetfillcolor{currentfill}%
\pgfsetlinewidth{0.803000pt}%
\definecolor{currentstroke}{rgb}{0.000000,0.000000,0.000000}%
\pgfsetstrokecolor{currentstroke}%
\pgfsetdash{}{0pt}%
\pgfsys@defobject{currentmarker}{\pgfqpoint{-0.048611in}{0.000000in}}{\pgfqpoint{0.000000in}{0.000000in}}{%
\pgfpathmoveto{\pgfqpoint{0.000000in}{0.000000in}}%
\pgfpathlineto{\pgfqpoint{-0.048611in}{0.000000in}}%
\pgfusepath{stroke,fill}%
}%
\begin{pgfscope}%
\pgfsys@transformshift{0.800000in}{3.471791in}%
\pgfsys@useobject{currentmarker}{}%
\end{pgfscope}%
\end{pgfscope}%
\begin{pgfscope}%
\definecolor{textcolor}{rgb}{0.000000,0.000000,0.000000}%
\pgfsetstrokecolor{textcolor}%
\pgfsetfillcolor{textcolor}%
\pgftext[x=0.424999in,y=3.419029in,left,base]{\color{textcolor}\sffamily\fontsize{10.000000}{12.000000}\selectfont \(\displaystyle 1750\)}%
\end{pgfscope}%
\begin{pgfscope}%
\pgfsetbuttcap%
\pgfsetroundjoin%
\definecolor{currentfill}{rgb}{0.000000,0.000000,0.000000}%
\pgfsetfillcolor{currentfill}%
\pgfsetlinewidth{0.803000pt}%
\definecolor{currentstroke}{rgb}{0.000000,0.000000,0.000000}%
\pgfsetstrokecolor{currentstroke}%
\pgfsetdash{}{0pt}%
\pgfsys@defobject{currentmarker}{\pgfqpoint{-0.048611in}{0.000000in}}{\pgfqpoint{0.000000in}{0.000000in}}{%
\pgfpathmoveto{\pgfqpoint{0.000000in}{0.000000in}}%
\pgfpathlineto{\pgfqpoint{-0.048611in}{0.000000in}}%
\pgfusepath{stroke,fill}%
}%
\begin{pgfscope}%
\pgfsys@transformshift{0.800000in}{3.870841in}%
\pgfsys@useobject{currentmarker}{}%
\end{pgfscope}%
\end{pgfscope}%
\begin{pgfscope}%
\definecolor{textcolor}{rgb}{0.000000,0.000000,0.000000}%
\pgfsetstrokecolor{textcolor}%
\pgfsetfillcolor{textcolor}%
\pgftext[x=0.424999in,y=3.818079in,left,base]{\color{textcolor}\sffamily\fontsize{10.000000}{12.000000}\selectfont \(\displaystyle 2000\)}%
\end{pgfscope}%
\begin{pgfscope}%
\definecolor{textcolor}{rgb}{0.000000,0.000000,0.000000}%
\pgfsetstrokecolor{textcolor}%
\pgfsetfillcolor{textcolor}%
\pgftext[x=0.369444in,y=2.376000in,,bottom,rotate=90.000000]{\color{textcolor}\sffamily\fontsize{10.000000}{12.000000}\selectfont Number of GMRES iterations}%
\end{pgfscope}%
\begin{pgfscope}%
\pgfpathrectangle{\pgfqpoint{0.800000in}{0.528000in}}{\pgfqpoint{4.960000in}{3.696000in}}%
\pgfusepath{clip}%
\pgfsetbuttcap%
\pgfsetroundjoin%
\pgfsetlinewidth{1.505625pt}%
\definecolor{currentstroke}{rgb}{0.843137,0.000000,0.000000}%
\pgfsetstrokecolor{currentstroke}%
\pgfsetdash{{5.550000pt}{2.400000pt}}{0.000000pt}%
\pgfpathmoveto{\pgfqpoint{1.250909in}{0.700789in}}%
\pgfpathlineto{\pgfqpoint{1.701818in}{0.742290in}}%
\pgfpathlineto{\pgfqpoint{2.152727in}{0.868390in}}%
\pgfpathlineto{\pgfqpoint{2.603636in}{1.090261in}}%
\pgfpathlineto{\pgfqpoint{3.054545in}{1.360019in}}%
\pgfpathlineto{\pgfqpoint{3.505455in}{1.679259in}}%
\pgfpathlineto{\pgfqpoint{3.956364in}{2.178869in}}%
\pgfpathlineto{\pgfqpoint{4.407273in}{2.712000in}}%
\pgfpathlineto{\pgfqpoint{4.858182in}{3.384000in}}%
\pgfpathlineto{\pgfqpoint{5.309091in}{4.056000in}}%
\pgfusepath{stroke}%
\end{pgfscope}%
\begin{pgfscope}%
\pgfpathrectangle{\pgfqpoint{0.800000in}{0.528000in}}{\pgfqpoint{4.960000in}{3.696000in}}%
\pgfusepath{clip}%
\pgfsetbuttcap%
\pgfsetroundjoin%
\definecolor{currentfill}{rgb}{0.843137,0.000000,0.000000}%
\pgfsetfillcolor{currentfill}%
\pgfsetlinewidth{1.003750pt}%
\definecolor{currentstroke}{rgb}{0.843137,0.000000,0.000000}%
\pgfsetstrokecolor{currentstroke}%
\pgfsetdash{}{0pt}%
\pgfsys@defobject{currentmarker}{\pgfqpoint{-0.041667in}{-0.041667in}}{\pgfqpoint{0.041667in}{0.041667in}}{%
\pgfpathmoveto{\pgfqpoint{0.000000in}{-0.041667in}}%
\pgfpathcurveto{\pgfqpoint{0.011050in}{-0.041667in}}{\pgfqpoint{0.021649in}{-0.037276in}}{\pgfqpoint{0.029463in}{-0.029463in}}%
\pgfpathcurveto{\pgfqpoint{0.037276in}{-0.021649in}}{\pgfqpoint{0.041667in}{-0.011050in}}{\pgfqpoint{0.041667in}{0.000000in}}%
\pgfpathcurveto{\pgfqpoint{0.041667in}{0.011050in}}{\pgfqpoint{0.037276in}{0.021649in}}{\pgfqpoint{0.029463in}{0.029463in}}%
\pgfpathcurveto{\pgfqpoint{0.021649in}{0.037276in}}{\pgfqpoint{0.011050in}{0.041667in}}{\pgfqpoint{0.000000in}{0.041667in}}%
\pgfpathcurveto{\pgfqpoint{-0.011050in}{0.041667in}}{\pgfqpoint{-0.021649in}{0.037276in}}{\pgfqpoint{-0.029463in}{0.029463in}}%
\pgfpathcurveto{\pgfqpoint{-0.037276in}{0.021649in}}{\pgfqpoint{-0.041667in}{0.011050in}}{\pgfqpoint{-0.041667in}{0.000000in}}%
\pgfpathcurveto{\pgfqpoint{-0.041667in}{-0.011050in}}{\pgfqpoint{-0.037276in}{-0.021649in}}{\pgfqpoint{-0.029463in}{-0.029463in}}%
\pgfpathcurveto{\pgfqpoint{-0.021649in}{-0.037276in}}{\pgfqpoint{-0.011050in}{-0.041667in}}{\pgfqpoint{0.000000in}{-0.041667in}}%
\pgfpathclose%
\pgfusepath{stroke,fill}%
}%
\begin{pgfscope}%
\pgfsys@transformshift{1.250909in}{0.700789in}%
\pgfsys@useobject{currentmarker}{}%
\end{pgfscope}%
\begin{pgfscope}%
\pgfsys@transformshift{1.701818in}{0.742290in}%
\pgfsys@useobject{currentmarker}{}%
\end{pgfscope}%
\begin{pgfscope}%
\pgfsys@transformshift{2.152727in}{0.868390in}%
\pgfsys@useobject{currentmarker}{}%
\end{pgfscope}%
\begin{pgfscope}%
\pgfsys@transformshift{2.603636in}{1.090261in}%
\pgfsys@useobject{currentmarker}{}%
\end{pgfscope}%
\begin{pgfscope}%
\pgfsys@transformshift{3.054545in}{1.360019in}%
\pgfsys@useobject{currentmarker}{}%
\end{pgfscope}%
\begin{pgfscope}%
\pgfsys@transformshift{3.505455in}{1.679259in}%
\pgfsys@useobject{currentmarker}{}%
\end{pgfscope}%
\begin{pgfscope}%
\pgfsys@transformshift{3.956364in}{2.178869in}%
\pgfsys@useobject{currentmarker}{}%
\end{pgfscope}%
\begin{pgfscope}%
\pgfsys@transformshift{4.407273in}{2.712000in}%
\pgfsys@useobject{currentmarker}{}%
\end{pgfscope}%
\begin{pgfscope}%
\pgfsys@transformshift{4.858182in}{3.384000in}%
\pgfsys@useobject{currentmarker}{}%
\end{pgfscope}%
\begin{pgfscope}%
\pgfsys@transformshift{5.309091in}{4.056000in}%
\pgfsys@useobject{currentmarker}{}%
\end{pgfscope}%
\end{pgfscope}%
\begin{pgfscope}%
\pgfpathrectangle{\pgfqpoint{0.800000in}{0.528000in}}{\pgfqpoint{4.960000in}{3.696000in}}%
\pgfusepath{clip}%
\pgfsetbuttcap%
\pgfsetroundjoin%
\pgfsetlinewidth{1.505625pt}%
\definecolor{currentstroke}{rgb}{0.549020,0.235294,1.000000}%
\pgfsetstrokecolor{currentstroke}%
\pgfsetdash{{5.550000pt}{2.400000pt}}{0.000000pt}%
\pgfpathmoveto{\pgfqpoint{1.250909in}{0.699192in}}%
\pgfpathlineto{\pgfqpoint{1.701818in}{0.721539in}}%
\pgfpathlineto{\pgfqpoint{2.152727in}{0.790176in}}%
\pgfpathlineto{\pgfqpoint{2.603636in}{0.913083in}}%
\pgfpathlineto{\pgfqpoint{3.054545in}{1.096646in}}%
\pgfpathlineto{\pgfqpoint{3.505455in}{1.307344in}}%
\pgfpathlineto{\pgfqpoint{3.956364in}{1.620200in}}%
\pgfpathlineto{\pgfqpoint{4.407273in}{1.995306in}}%
\pgfpathlineto{\pgfqpoint{4.858182in}{2.478955in}}%
\pgfpathlineto{\pgfqpoint{5.309091in}{2.901948in}}%
\pgfusepath{stroke}%
\end{pgfscope}%
\begin{pgfscope}%
\pgfpathrectangle{\pgfqpoint{0.800000in}{0.528000in}}{\pgfqpoint{4.960000in}{3.696000in}}%
\pgfusepath{clip}%
\pgfsetbuttcap%
\pgfsetmiterjoin%
\definecolor{currentfill}{rgb}{0.549020,0.235294,1.000000}%
\pgfsetfillcolor{currentfill}%
\pgfsetlinewidth{1.003750pt}%
\definecolor{currentstroke}{rgb}{0.549020,0.235294,1.000000}%
\pgfsetstrokecolor{currentstroke}%
\pgfsetdash{}{0pt}%
\pgfsys@defobject{currentmarker}{\pgfqpoint{-0.041667in}{-0.041667in}}{\pgfqpoint{0.041667in}{0.041667in}}{%
\pgfpathmoveto{\pgfqpoint{0.000000in}{0.041667in}}%
\pgfpathlineto{\pgfqpoint{-0.041667in}{-0.041667in}}%
\pgfpathlineto{\pgfqpoint{0.041667in}{-0.041667in}}%
\pgfpathclose%
\pgfusepath{stroke,fill}%
}%
\begin{pgfscope}%
\pgfsys@transformshift{1.250909in}{0.699192in}%
\pgfsys@useobject{currentmarker}{}%
\end{pgfscope}%
\begin{pgfscope}%
\pgfsys@transformshift{1.701818in}{0.721539in}%
\pgfsys@useobject{currentmarker}{}%
\end{pgfscope}%
\begin{pgfscope}%
\pgfsys@transformshift{2.152727in}{0.790176in}%
\pgfsys@useobject{currentmarker}{}%
\end{pgfscope}%
\begin{pgfscope}%
\pgfsys@transformshift{2.603636in}{0.913083in}%
\pgfsys@useobject{currentmarker}{}%
\end{pgfscope}%
\begin{pgfscope}%
\pgfsys@transformshift{3.054545in}{1.096646in}%
\pgfsys@useobject{currentmarker}{}%
\end{pgfscope}%
\begin{pgfscope}%
\pgfsys@transformshift{3.505455in}{1.307344in}%
\pgfsys@useobject{currentmarker}{}%
\end{pgfscope}%
\begin{pgfscope}%
\pgfsys@transformshift{3.956364in}{1.620200in}%
\pgfsys@useobject{currentmarker}{}%
\end{pgfscope}%
\begin{pgfscope}%
\pgfsys@transformshift{4.407273in}{1.995306in}%
\pgfsys@useobject{currentmarker}{}%
\end{pgfscope}%
\begin{pgfscope}%
\pgfsys@transformshift{4.858182in}{2.478955in}%
\pgfsys@useobject{currentmarker}{}%
\end{pgfscope}%
\begin{pgfscope}%
\pgfsys@transformshift{5.309091in}{2.901948in}%
\pgfsys@useobject{currentmarker}{}%
\end{pgfscope}%
\end{pgfscope}%
\begin{pgfscope}%
\pgfpathrectangle{\pgfqpoint{0.800000in}{0.528000in}}{\pgfqpoint{4.960000in}{3.696000in}}%
\pgfusepath{clip}%
\pgfsetbuttcap%
\pgfsetroundjoin%
\pgfsetlinewidth{1.505625pt}%
\definecolor{currentstroke}{rgb}{0.007843,0.533333,0.000000}%
\pgfsetstrokecolor{currentstroke}%
\pgfsetdash{{5.550000pt}{2.400000pt}}{0.000000pt}%
\pgfpathmoveto{\pgfqpoint{1.250909in}{0.697596in}}%
\pgfpathlineto{\pgfqpoint{1.701818in}{0.713558in}}%
\pgfpathlineto{\pgfqpoint{2.152727in}{0.742290in}}%
\pgfpathlineto{\pgfqpoint{2.603636in}{0.801349in}}%
\pgfpathlineto{\pgfqpoint{3.054545in}{0.892333in}}%
\pgfpathlineto{\pgfqpoint{3.505455in}{0.996086in}}%
\pgfpathlineto{\pgfqpoint{3.956364in}{1.144532in}}%
\pgfpathlineto{\pgfqpoint{4.407273in}{1.347249in}}%
\pgfpathlineto{\pgfqpoint{4.858182in}{1.557948in}}%
\pgfpathlineto{\pgfqpoint{5.309091in}{1.837283in}}%
\pgfusepath{stroke}%
\end{pgfscope}%
\begin{pgfscope}%
\pgfpathrectangle{\pgfqpoint{0.800000in}{0.528000in}}{\pgfqpoint{4.960000in}{3.696000in}}%
\pgfusepath{clip}%
\pgfsetbuttcap%
\pgfsetmiterjoin%
\definecolor{currentfill}{rgb}{0.007843,0.533333,0.000000}%
\pgfsetfillcolor{currentfill}%
\pgfsetlinewidth{1.003750pt}%
\definecolor{currentstroke}{rgb}{0.007843,0.533333,0.000000}%
\pgfsetstrokecolor{currentstroke}%
\pgfsetdash{}{0pt}%
\pgfsys@defobject{currentmarker}{\pgfqpoint{-0.041667in}{-0.041667in}}{\pgfqpoint{0.041667in}{0.041667in}}{%
\pgfpathmoveto{\pgfqpoint{-0.000000in}{-0.041667in}}%
\pgfpathlineto{\pgfqpoint{0.041667in}{0.041667in}}%
\pgfpathlineto{\pgfqpoint{-0.041667in}{0.041667in}}%
\pgfpathclose%
\pgfusepath{stroke,fill}%
}%
\begin{pgfscope}%
\pgfsys@transformshift{1.250909in}{0.697596in}%
\pgfsys@useobject{currentmarker}{}%
\end{pgfscope}%
\begin{pgfscope}%
\pgfsys@transformshift{1.701818in}{0.713558in}%
\pgfsys@useobject{currentmarker}{}%
\end{pgfscope}%
\begin{pgfscope}%
\pgfsys@transformshift{2.152727in}{0.742290in}%
\pgfsys@useobject{currentmarker}{}%
\end{pgfscope}%
\begin{pgfscope}%
\pgfsys@transformshift{2.603636in}{0.801349in}%
\pgfsys@useobject{currentmarker}{}%
\end{pgfscope}%
\begin{pgfscope}%
\pgfsys@transformshift{3.054545in}{0.892333in}%
\pgfsys@useobject{currentmarker}{}%
\end{pgfscope}%
\begin{pgfscope}%
\pgfsys@transformshift{3.505455in}{0.996086in}%
\pgfsys@useobject{currentmarker}{}%
\end{pgfscope}%
\begin{pgfscope}%
\pgfsys@transformshift{3.956364in}{1.144532in}%
\pgfsys@useobject{currentmarker}{}%
\end{pgfscope}%
\begin{pgfscope}%
\pgfsys@transformshift{4.407273in}{1.347249in}%
\pgfsys@useobject{currentmarker}{}%
\end{pgfscope}%
\begin{pgfscope}%
\pgfsys@transformshift{4.858182in}{1.557948in}%
\pgfsys@useobject{currentmarker}{}%
\end{pgfscope}%
\begin{pgfscope}%
\pgfsys@transformshift{5.309091in}{1.837283in}%
\pgfsys@useobject{currentmarker}{}%
\end{pgfscope}%
\end{pgfscope}%
\begin{pgfscope}%
\pgfpathrectangle{\pgfqpoint{0.800000in}{0.528000in}}{\pgfqpoint{4.960000in}{3.696000in}}%
\pgfusepath{clip}%
\pgfsetbuttcap%
\pgfsetroundjoin%
\pgfsetlinewidth{1.505625pt}%
\definecolor{currentstroke}{rgb}{0.000000,0.674510,0.780392}%
\pgfsetstrokecolor{currentstroke}%
\pgfsetdash{{5.550000pt}{2.400000pt}}{0.000000pt}%
\pgfpathmoveto{\pgfqpoint{1.250909in}{0.696000in}}%
\pgfpathlineto{\pgfqpoint{1.701818in}{0.707173in}}%
\pgfpathlineto{\pgfqpoint{2.152727in}{0.718347in}}%
\pgfpathlineto{\pgfqpoint{2.603636in}{0.742290in}}%
\pgfpathlineto{\pgfqpoint{3.054545in}{0.771021in}}%
\pgfpathlineto{\pgfqpoint{3.505455in}{0.815715in}}%
\pgfpathlineto{\pgfqpoint{3.956364in}{0.868390in}}%
\pgfpathlineto{\pgfqpoint{4.407273in}{0.938622in}}%
\pgfpathlineto{\pgfqpoint{4.858182in}{1.012048in}}%
\pgfpathlineto{\pgfqpoint{5.309091in}{1.109416in}}%
\pgfusepath{stroke}%
\end{pgfscope}%
\begin{pgfscope}%
\pgfpathrectangle{\pgfqpoint{0.800000in}{0.528000in}}{\pgfqpoint{4.960000in}{3.696000in}}%
\pgfusepath{clip}%
\pgfsetbuttcap%
\pgfsetmiterjoin%
\definecolor{currentfill}{rgb}{0.000000,0.674510,0.780392}%
\pgfsetfillcolor{currentfill}%
\pgfsetlinewidth{1.003750pt}%
\definecolor{currentstroke}{rgb}{0.000000,0.674510,0.780392}%
\pgfsetstrokecolor{currentstroke}%
\pgfsetdash{}{0pt}%
\pgfsys@defobject{currentmarker}{\pgfqpoint{-0.041667in}{-0.041667in}}{\pgfqpoint{0.041667in}{0.041667in}}{%
\pgfpathmoveto{\pgfqpoint{0.041667in}{-0.000000in}}%
\pgfpathlineto{\pgfqpoint{-0.041667in}{0.041667in}}%
\pgfpathlineto{\pgfqpoint{-0.041667in}{-0.041667in}}%
\pgfpathclose%
\pgfusepath{stroke,fill}%
}%
\begin{pgfscope}%
\pgfsys@transformshift{1.250909in}{0.696000in}%
\pgfsys@useobject{currentmarker}{}%
\end{pgfscope}%
\begin{pgfscope}%
\pgfsys@transformshift{1.701818in}{0.707173in}%
\pgfsys@useobject{currentmarker}{}%
\end{pgfscope}%
\begin{pgfscope}%
\pgfsys@transformshift{2.152727in}{0.718347in}%
\pgfsys@useobject{currentmarker}{}%
\end{pgfscope}%
\begin{pgfscope}%
\pgfsys@transformshift{2.603636in}{0.742290in}%
\pgfsys@useobject{currentmarker}{}%
\end{pgfscope}%
\begin{pgfscope}%
\pgfsys@transformshift{3.054545in}{0.771021in}%
\pgfsys@useobject{currentmarker}{}%
\end{pgfscope}%
\begin{pgfscope}%
\pgfsys@transformshift{3.505455in}{0.815715in}%
\pgfsys@useobject{currentmarker}{}%
\end{pgfscope}%
\begin{pgfscope}%
\pgfsys@transformshift{3.956364in}{0.868390in}%
\pgfsys@useobject{currentmarker}{}%
\end{pgfscope}%
\begin{pgfscope}%
\pgfsys@transformshift{4.407273in}{0.938622in}%
\pgfsys@useobject{currentmarker}{}%
\end{pgfscope}%
\begin{pgfscope}%
\pgfsys@transformshift{4.858182in}{1.012048in}%
\pgfsys@useobject{currentmarker}{}%
\end{pgfscope}%
\begin{pgfscope}%
\pgfsys@transformshift{5.309091in}{1.109416in}%
\pgfsys@useobject{currentmarker}{}%
\end{pgfscope}%
\end{pgfscope}%
\begin{pgfscope}%
\pgfsetrectcap%
\pgfsetmiterjoin%
\pgfsetlinewidth{0.803000pt}%
\definecolor{currentstroke}{rgb}{0.000000,0.000000,0.000000}%
\pgfsetstrokecolor{currentstroke}%
\pgfsetdash{}{0pt}%
\pgfpathmoveto{\pgfqpoint{0.800000in}{0.528000in}}%
\pgfpathlineto{\pgfqpoint{0.800000in}{4.224000in}}%
\pgfusepath{stroke}%
\end{pgfscope}%
\begin{pgfscope}%
\pgfsetrectcap%
\pgfsetmiterjoin%
\pgfsetlinewidth{0.000000pt}%
\definecolor{currentstroke}{rgb}{0.000000,0.000000,0.000000}%
\pgfsetstrokecolor{currentstroke}%
\pgfsetstrokeopacity{0.000000}%
\pgfsetdash{}{0pt}%
\pgfpathmoveto{\pgfqpoint{5.760000in}{0.528000in}}%
\pgfpathlineto{\pgfqpoint{5.760000in}{4.224000in}}%
\pgfusepath{}%
\end{pgfscope}%
\begin{pgfscope}%
\pgfsetrectcap%
\pgfsetmiterjoin%
\pgfsetlinewidth{0.803000pt}%
\definecolor{currentstroke}{rgb}{0.000000,0.000000,0.000000}%
\pgfsetstrokecolor{currentstroke}%
\pgfsetdash{}{0pt}%
\pgfpathmoveto{\pgfqpoint{0.800000in}{0.528000in}}%
\pgfpathlineto{\pgfqpoint{5.760000in}{0.528000in}}%
\pgfusepath{stroke}%
\end{pgfscope}%
\begin{pgfscope}%
\pgfsetrectcap%
\pgfsetmiterjoin%
\pgfsetlinewidth{0.000000pt}%
\definecolor{currentstroke}{rgb}{0.000000,0.000000,0.000000}%
\pgfsetstrokecolor{currentstroke}%
\pgfsetstrokeopacity{0.000000}%
\pgfsetdash{}{0pt}%
\pgfpathmoveto{\pgfqpoint{0.800000in}{4.224000in}}%
\pgfpathlineto{\pgfqpoint{5.760000in}{4.224000in}}%
\pgfusepath{}%
\end{pgfscope}%
\begin{pgfscope}%
\pgfsetbuttcap%
\pgfsetmiterjoin%
\definecolor{currentfill}{rgb}{1.000000,1.000000,1.000000}%
\pgfsetfillcolor{currentfill}%
\pgfsetfillopacity{0.800000}%
\pgfsetlinewidth{1.003750pt}%
\definecolor{currentstroke}{rgb}{0.800000,0.800000,0.800000}%
\pgfsetstrokecolor{currentstroke}%
\pgfsetstrokeopacity{0.800000}%
\pgfsetdash{}{0pt}%
\pgfpathmoveto{\pgfqpoint{0.897222in}{3.297460in}}%
\pgfpathlineto{\pgfqpoint{1.790209in}{3.297460in}}%
\pgfpathquadraticcurveto{\pgfqpoint{1.817987in}{3.297460in}}{\pgfqpoint{1.817987in}{3.325238in}}%
\pgfpathlineto{\pgfqpoint{1.817987in}{4.126778in}}%
\pgfpathquadraticcurveto{\pgfqpoint{1.817987in}{4.154556in}}{\pgfqpoint{1.790209in}{4.154556in}}%
\pgfpathlineto{\pgfqpoint{0.897222in}{4.154556in}}%
\pgfpathquadraticcurveto{\pgfqpoint{0.869444in}{4.154556in}}{\pgfqpoint{0.869444in}{4.126778in}}%
\pgfpathlineto{\pgfqpoint{0.869444in}{3.325238in}}%
\pgfpathquadraticcurveto{\pgfqpoint{0.869444in}{3.297460in}}{\pgfqpoint{0.897222in}{3.297460in}}%
\pgfpathclose%
\pgfusepath{stroke,fill}%
\end{pgfscope}%
\begin{pgfscope}%
\pgfsetbuttcap%
\pgfsetroundjoin%
\pgfsetlinewidth{1.505625pt}%
\definecolor{currentstroke}{rgb}{0.843137,0.000000,0.000000}%
\pgfsetstrokecolor{currentstroke}%
\pgfsetdash{{5.550000pt}{2.400000pt}}{0.000000pt}%
\pgfpathmoveto{\pgfqpoint{0.925000in}{4.042088in}}%
\pgfpathlineto{\pgfqpoint{1.202778in}{4.042088in}}%
\pgfusepath{stroke}%
\end{pgfscope}%
\begin{pgfscope}%
\pgfsetbuttcap%
\pgfsetroundjoin%
\definecolor{currentfill}{rgb}{0.843137,0.000000,0.000000}%
\pgfsetfillcolor{currentfill}%
\pgfsetlinewidth{1.003750pt}%
\definecolor{currentstroke}{rgb}{0.843137,0.000000,0.000000}%
\pgfsetstrokecolor{currentstroke}%
\pgfsetdash{}{0pt}%
\pgfsys@defobject{currentmarker}{\pgfqpoint{-0.041667in}{-0.041667in}}{\pgfqpoint{0.041667in}{0.041667in}}{%
\pgfpathmoveto{\pgfqpoint{0.000000in}{-0.041667in}}%
\pgfpathcurveto{\pgfqpoint{0.011050in}{-0.041667in}}{\pgfqpoint{0.021649in}{-0.037276in}}{\pgfqpoint{0.029463in}{-0.029463in}}%
\pgfpathcurveto{\pgfqpoint{0.037276in}{-0.021649in}}{\pgfqpoint{0.041667in}{-0.011050in}}{\pgfqpoint{0.041667in}{0.000000in}}%
\pgfpathcurveto{\pgfqpoint{0.041667in}{0.011050in}}{\pgfqpoint{0.037276in}{0.021649in}}{\pgfqpoint{0.029463in}{0.029463in}}%
\pgfpathcurveto{\pgfqpoint{0.021649in}{0.037276in}}{\pgfqpoint{0.011050in}{0.041667in}}{\pgfqpoint{0.000000in}{0.041667in}}%
\pgfpathcurveto{\pgfqpoint{-0.011050in}{0.041667in}}{\pgfqpoint{-0.021649in}{0.037276in}}{\pgfqpoint{-0.029463in}{0.029463in}}%
\pgfpathcurveto{\pgfqpoint{-0.037276in}{0.021649in}}{\pgfqpoint{-0.041667in}{0.011050in}}{\pgfqpoint{-0.041667in}{0.000000in}}%
\pgfpathcurveto{\pgfqpoint{-0.041667in}{-0.011050in}}{\pgfqpoint{-0.037276in}{-0.021649in}}{\pgfqpoint{-0.029463in}{-0.029463in}}%
\pgfpathcurveto{\pgfqpoint{-0.021649in}{-0.037276in}}{\pgfqpoint{-0.011050in}{-0.041667in}}{\pgfqpoint{0.000000in}{-0.041667in}}%
\pgfpathclose%
\pgfusepath{stroke,fill}%
}%
\begin{pgfscope}%
\pgfsys@transformshift{1.063889in}{4.042088in}%
\pgfsys@useobject{currentmarker}{}%
\end{pgfscope}%
\end{pgfscope}%
\begin{pgfscope}%
\definecolor{textcolor}{rgb}{0.000000,0.000000,0.000000}%
\pgfsetstrokecolor{textcolor}%
\pgfsetfillcolor{textcolor}%
\pgftext[x=1.313889in,y=3.993477in,left,base]{\color{textcolor}\sffamily\fontsize{10.000000}{12.000000}\selectfont \(\displaystyle \beta = 0\)}%
\end{pgfscope}%
\begin{pgfscope}%
\pgfsetbuttcap%
\pgfsetroundjoin%
\pgfsetlinewidth{1.505625pt}%
\definecolor{currentstroke}{rgb}{0.549020,0.235294,1.000000}%
\pgfsetstrokecolor{currentstroke}%
\pgfsetdash{{5.550000pt}{2.400000pt}}{0.000000pt}%
\pgfpathmoveto{\pgfqpoint{0.925000in}{3.838231in}}%
\pgfpathlineto{\pgfqpoint{1.202778in}{3.838231in}}%
\pgfusepath{stroke}%
\end{pgfscope}%
\begin{pgfscope}%
\pgfsetbuttcap%
\pgfsetmiterjoin%
\definecolor{currentfill}{rgb}{0.549020,0.235294,1.000000}%
\pgfsetfillcolor{currentfill}%
\pgfsetlinewidth{1.003750pt}%
\definecolor{currentstroke}{rgb}{0.549020,0.235294,1.000000}%
\pgfsetstrokecolor{currentstroke}%
\pgfsetdash{}{0pt}%
\pgfsys@defobject{currentmarker}{\pgfqpoint{-0.041667in}{-0.041667in}}{\pgfqpoint{0.041667in}{0.041667in}}{%
\pgfpathmoveto{\pgfqpoint{0.000000in}{0.041667in}}%
\pgfpathlineto{\pgfqpoint{-0.041667in}{-0.041667in}}%
\pgfpathlineto{\pgfqpoint{0.041667in}{-0.041667in}}%
\pgfpathclose%
\pgfusepath{stroke,fill}%
}%
\begin{pgfscope}%
\pgfsys@transformshift{1.063889in}{3.838231in}%
\pgfsys@useobject{currentmarker}{}%
\end{pgfscope}%
\end{pgfscope}%
\begin{pgfscope}%
\definecolor{textcolor}{rgb}{0.000000,0.000000,0.000000}%
\pgfsetstrokecolor{textcolor}%
\pgfsetfillcolor{textcolor}%
\pgftext[x=1.313889in,y=3.789620in,left,base]{\color{textcolor}\sffamily\fontsize{10.000000}{12.000000}\selectfont \(\displaystyle \beta = 0.1\)}%
\end{pgfscope}%
\begin{pgfscope}%
\pgfsetbuttcap%
\pgfsetroundjoin%
\pgfsetlinewidth{1.505625pt}%
\definecolor{currentstroke}{rgb}{0.007843,0.533333,0.000000}%
\pgfsetstrokecolor{currentstroke}%
\pgfsetdash{{5.550000pt}{2.400000pt}}{0.000000pt}%
\pgfpathmoveto{\pgfqpoint{0.925000in}{3.634374in}}%
\pgfpathlineto{\pgfqpoint{1.202778in}{3.634374in}}%
\pgfusepath{stroke}%
\end{pgfscope}%
\begin{pgfscope}%
\pgfsetbuttcap%
\pgfsetmiterjoin%
\definecolor{currentfill}{rgb}{0.007843,0.533333,0.000000}%
\pgfsetfillcolor{currentfill}%
\pgfsetlinewidth{1.003750pt}%
\definecolor{currentstroke}{rgb}{0.007843,0.533333,0.000000}%
\pgfsetstrokecolor{currentstroke}%
\pgfsetdash{}{0pt}%
\pgfsys@defobject{currentmarker}{\pgfqpoint{-0.041667in}{-0.041667in}}{\pgfqpoint{0.041667in}{0.041667in}}{%
\pgfpathmoveto{\pgfqpoint{-0.000000in}{-0.041667in}}%
\pgfpathlineto{\pgfqpoint{0.041667in}{0.041667in}}%
\pgfpathlineto{\pgfqpoint{-0.041667in}{0.041667in}}%
\pgfpathclose%
\pgfusepath{stroke,fill}%
}%
\begin{pgfscope}%
\pgfsys@transformshift{1.063889in}{3.634374in}%
\pgfsys@useobject{currentmarker}{}%
\end{pgfscope}%
\end{pgfscope}%
\begin{pgfscope}%
\definecolor{textcolor}{rgb}{0.000000,0.000000,0.000000}%
\pgfsetstrokecolor{textcolor}%
\pgfsetfillcolor{textcolor}%
\pgftext[x=1.313889in,y=3.585762in,left,base]{\color{textcolor}\sffamily\fontsize{10.000000}{12.000000}\selectfont \(\displaystyle \beta = 0.2\)}%
\end{pgfscope}%
\begin{pgfscope}%
\pgfsetbuttcap%
\pgfsetroundjoin%
\pgfsetlinewidth{1.505625pt}%
\definecolor{currentstroke}{rgb}{0.000000,0.674510,0.780392}%
\pgfsetstrokecolor{currentstroke}%
\pgfsetdash{{5.550000pt}{2.400000pt}}{0.000000pt}%
\pgfpathmoveto{\pgfqpoint{0.925000in}{3.430516in}}%
\pgfpathlineto{\pgfqpoint{1.202778in}{3.430516in}}%
\pgfusepath{stroke}%
\end{pgfscope}%
\begin{pgfscope}%
\pgfsetbuttcap%
\pgfsetmiterjoin%
\definecolor{currentfill}{rgb}{0.000000,0.674510,0.780392}%
\pgfsetfillcolor{currentfill}%
\pgfsetlinewidth{1.003750pt}%
\definecolor{currentstroke}{rgb}{0.000000,0.674510,0.780392}%
\pgfsetstrokecolor{currentstroke}%
\pgfsetdash{}{0pt}%
\pgfsys@defobject{currentmarker}{\pgfqpoint{-0.041667in}{-0.041667in}}{\pgfqpoint{0.041667in}{0.041667in}}{%
\pgfpathmoveto{\pgfqpoint{0.041667in}{-0.000000in}}%
\pgfpathlineto{\pgfqpoint{-0.041667in}{0.041667in}}%
\pgfpathlineto{\pgfqpoint{-0.041667in}{-0.041667in}}%
\pgfpathclose%
\pgfusepath{stroke,fill}%
}%
\begin{pgfscope}%
\pgfsys@transformshift{1.063889in}{3.430516in}%
\pgfsys@useobject{currentmarker}{}%
\end{pgfscope}%
\end{pgfscope}%
\begin{pgfscope}%
\definecolor{textcolor}{rgb}{0.000000,0.000000,0.000000}%
\pgfsetstrokecolor{textcolor}%
\pgfsetfillcolor{textcolor}%
\pgftext[x=1.313889in,y=3.381905in,left,base]{\color{textcolor}\sffamily\fontsize{10.000000}{12.000000}\selectfont \(\displaystyle \beta = 0.3\)}%
\end{pgfscope}%
\end{pgfpicture}%
\makeatother%
\endgroup%

  \caption[GMRES iteration counts when $\NLqDRR{\nso-\nst} = 0.2\times k^{-\beta},$ for any $1 \leq q < \infty$ and $\beta = 0,0.1,0.2,0.3$.]{GMRES iteration counts for $\AmatoI\Amatt$ given by \cref{eq:noweak,eq:ntweak}, where $\alpha = 0.2\times k^{-\beta},$ for $\beta = 0,0.1,0.2,0.3.$}\label{fig:l1low}
\end{figure}

\begin{figure}
  %% Creator: Matplotlib, PGF backend
%%
%% To include the figure in your LaTeX document, write
%%   \input{<filename>.pgf}
%%
%% Make sure the required packages are loaded in your preamble
%%   \usepackage{pgf}
%%
%% Figures using additional raster images can only be included by \input if
%% they are in the same directory as the main LaTeX file. For loading figures
%% from other directories you can use the `import` package
%%   \usepackage{import}
%% and then include the figures with
%%   \import{<path to file>}{<filename>.pgf}
%%
%% Matplotlib used the following preamble
%%   \usepackage{fontspec}
%%   \setmainfont{DejaVuSerif.ttf}[Path=/home/owen/progs/firedrake-complex/firedrake/lib/python3.5/site-packages/matplotlib/mpl-data/fonts/ttf/]
%%   \setsansfont{DejaVuSans.ttf}[Path=/home/owen/progs/firedrake-complex/firedrake/lib/python3.5/site-packages/matplotlib/mpl-data/fonts/ttf/]
%%   \setmonofont{DejaVuSansMono.ttf}[Path=/home/owen/progs/firedrake-complex/firedrake/lib/python3.5/site-packages/matplotlib/mpl-data/fonts/ttf/]
%%
\begingroup%
\makeatletter%
\begin{pgfpicture}%
\pgfpathrectangle{\pgfpointorigin}{\pgfqpoint{6.400000in}{4.800000in}}%
\pgfusepath{use as bounding box, clip}%
\begin{pgfscope}%
\pgfsetbuttcap%
\pgfsetmiterjoin%
\definecolor{currentfill}{rgb}{1.000000,1.000000,1.000000}%
\pgfsetfillcolor{currentfill}%
\pgfsetlinewidth{0.000000pt}%
\definecolor{currentstroke}{rgb}{1.000000,1.000000,1.000000}%
\pgfsetstrokecolor{currentstroke}%
\pgfsetdash{}{0pt}%
\pgfpathmoveto{\pgfqpoint{0.000000in}{0.000000in}}%
\pgfpathlineto{\pgfqpoint{6.400000in}{0.000000in}}%
\pgfpathlineto{\pgfqpoint{6.400000in}{4.800000in}}%
\pgfpathlineto{\pgfqpoint{0.000000in}{4.800000in}}%
\pgfpathclose%
\pgfusepath{fill}%
\end{pgfscope}%
\begin{pgfscope}%
\pgfsetbuttcap%
\pgfsetmiterjoin%
\definecolor{currentfill}{rgb}{1.000000,1.000000,1.000000}%
\pgfsetfillcolor{currentfill}%
\pgfsetlinewidth{0.000000pt}%
\definecolor{currentstroke}{rgb}{0.000000,0.000000,0.000000}%
\pgfsetstrokecolor{currentstroke}%
\pgfsetstrokeopacity{0.000000}%
\pgfsetdash{}{0pt}%
\pgfpathmoveto{\pgfqpoint{0.800000in}{0.528000in}}%
\pgfpathlineto{\pgfqpoint{5.760000in}{0.528000in}}%
\pgfpathlineto{\pgfqpoint{5.760000in}{4.224000in}}%
\pgfpathlineto{\pgfqpoint{0.800000in}{4.224000in}}%
\pgfpathclose%
\pgfusepath{fill}%
\end{pgfscope}%
\begin{pgfscope}%
\pgfsetbuttcap%
\pgfsetroundjoin%
\definecolor{currentfill}{rgb}{0.000000,0.000000,0.000000}%
\pgfsetfillcolor{currentfill}%
\pgfsetlinewidth{0.803000pt}%
\definecolor{currentstroke}{rgb}{0.000000,0.000000,0.000000}%
\pgfsetstrokecolor{currentstroke}%
\pgfsetdash{}{0pt}%
\pgfsys@defobject{currentmarker}{\pgfqpoint{0.000000in}{-0.048611in}}{\pgfqpoint{0.000000in}{0.000000in}}{%
\pgfpathmoveto{\pgfqpoint{0.000000in}{0.000000in}}%
\pgfpathlineto{\pgfqpoint{0.000000in}{-0.048611in}}%
\pgfusepath{stroke,fill}%
}%
\begin{pgfscope}%
\pgfsys@transformshift{1.250909in}{0.528000in}%
\pgfsys@useobject{currentmarker}{}%
\end{pgfscope}%
\end{pgfscope}%
\begin{pgfscope}%
\definecolor{textcolor}{rgb}{0.000000,0.000000,0.000000}%
\pgfsetstrokecolor{textcolor}%
\pgfsetfillcolor{textcolor}%
\pgftext[x=1.250909in,y=0.430778in,,top]{\color{textcolor}\sffamily\fontsize{10.000000}{12.000000}\selectfont 10}%
\end{pgfscope}%
\begin{pgfscope}%
\pgfsetbuttcap%
\pgfsetroundjoin%
\definecolor{currentfill}{rgb}{0.000000,0.000000,0.000000}%
\pgfsetfillcolor{currentfill}%
\pgfsetlinewidth{0.803000pt}%
\definecolor{currentstroke}{rgb}{0.000000,0.000000,0.000000}%
\pgfsetstrokecolor{currentstroke}%
\pgfsetdash{}{0pt}%
\pgfsys@defobject{currentmarker}{\pgfqpoint{0.000000in}{-0.048611in}}{\pgfqpoint{0.000000in}{0.000000in}}{%
\pgfpathmoveto{\pgfqpoint{0.000000in}{0.000000in}}%
\pgfpathlineto{\pgfqpoint{0.000000in}{-0.048611in}}%
\pgfusepath{stroke,fill}%
}%
\begin{pgfscope}%
\pgfsys@transformshift{1.701818in}{0.528000in}%
\pgfsys@useobject{currentmarker}{}%
\end{pgfscope}%
\end{pgfscope}%
\begin{pgfscope}%
\definecolor{textcolor}{rgb}{0.000000,0.000000,0.000000}%
\pgfsetstrokecolor{textcolor}%
\pgfsetfillcolor{textcolor}%
\pgftext[x=1.701818in,y=0.430778in,,top]{\color{textcolor}\sffamily\fontsize{10.000000}{12.000000}\selectfont 20}%
\end{pgfscope}%
\begin{pgfscope}%
\pgfsetbuttcap%
\pgfsetroundjoin%
\definecolor{currentfill}{rgb}{0.000000,0.000000,0.000000}%
\pgfsetfillcolor{currentfill}%
\pgfsetlinewidth{0.803000pt}%
\definecolor{currentstroke}{rgb}{0.000000,0.000000,0.000000}%
\pgfsetstrokecolor{currentstroke}%
\pgfsetdash{}{0pt}%
\pgfsys@defobject{currentmarker}{\pgfqpoint{0.000000in}{-0.048611in}}{\pgfqpoint{0.000000in}{0.000000in}}{%
\pgfpathmoveto{\pgfqpoint{0.000000in}{0.000000in}}%
\pgfpathlineto{\pgfqpoint{0.000000in}{-0.048611in}}%
\pgfusepath{stroke,fill}%
}%
\begin{pgfscope}%
\pgfsys@transformshift{2.152727in}{0.528000in}%
\pgfsys@useobject{currentmarker}{}%
\end{pgfscope}%
\end{pgfscope}%
\begin{pgfscope}%
\definecolor{textcolor}{rgb}{0.000000,0.000000,0.000000}%
\pgfsetstrokecolor{textcolor}%
\pgfsetfillcolor{textcolor}%
\pgftext[x=2.152727in,y=0.430778in,,top]{\color{textcolor}\sffamily\fontsize{10.000000}{12.000000}\selectfont 30}%
\end{pgfscope}%
\begin{pgfscope}%
\pgfsetbuttcap%
\pgfsetroundjoin%
\definecolor{currentfill}{rgb}{0.000000,0.000000,0.000000}%
\pgfsetfillcolor{currentfill}%
\pgfsetlinewidth{0.803000pt}%
\definecolor{currentstroke}{rgb}{0.000000,0.000000,0.000000}%
\pgfsetstrokecolor{currentstroke}%
\pgfsetdash{}{0pt}%
\pgfsys@defobject{currentmarker}{\pgfqpoint{0.000000in}{-0.048611in}}{\pgfqpoint{0.000000in}{0.000000in}}{%
\pgfpathmoveto{\pgfqpoint{0.000000in}{0.000000in}}%
\pgfpathlineto{\pgfqpoint{0.000000in}{-0.048611in}}%
\pgfusepath{stroke,fill}%
}%
\begin{pgfscope}%
\pgfsys@transformshift{2.603636in}{0.528000in}%
\pgfsys@useobject{currentmarker}{}%
\end{pgfscope}%
\end{pgfscope}%
\begin{pgfscope}%
\definecolor{textcolor}{rgb}{0.000000,0.000000,0.000000}%
\pgfsetstrokecolor{textcolor}%
\pgfsetfillcolor{textcolor}%
\pgftext[x=2.603636in,y=0.430778in,,top]{\color{textcolor}\sffamily\fontsize{10.000000}{12.000000}\selectfont 40}%
\end{pgfscope}%
\begin{pgfscope}%
\pgfsetbuttcap%
\pgfsetroundjoin%
\definecolor{currentfill}{rgb}{0.000000,0.000000,0.000000}%
\pgfsetfillcolor{currentfill}%
\pgfsetlinewidth{0.803000pt}%
\definecolor{currentstroke}{rgb}{0.000000,0.000000,0.000000}%
\pgfsetstrokecolor{currentstroke}%
\pgfsetdash{}{0pt}%
\pgfsys@defobject{currentmarker}{\pgfqpoint{0.000000in}{-0.048611in}}{\pgfqpoint{0.000000in}{0.000000in}}{%
\pgfpathmoveto{\pgfqpoint{0.000000in}{0.000000in}}%
\pgfpathlineto{\pgfqpoint{0.000000in}{-0.048611in}}%
\pgfusepath{stroke,fill}%
}%
\begin{pgfscope}%
\pgfsys@transformshift{3.054545in}{0.528000in}%
\pgfsys@useobject{currentmarker}{}%
\end{pgfscope}%
\end{pgfscope}%
\begin{pgfscope}%
\definecolor{textcolor}{rgb}{0.000000,0.000000,0.000000}%
\pgfsetstrokecolor{textcolor}%
\pgfsetfillcolor{textcolor}%
\pgftext[x=3.054545in,y=0.430778in,,top]{\color{textcolor}\sffamily\fontsize{10.000000}{12.000000}\selectfont 50}%
\end{pgfscope}%
\begin{pgfscope}%
\pgfsetbuttcap%
\pgfsetroundjoin%
\definecolor{currentfill}{rgb}{0.000000,0.000000,0.000000}%
\pgfsetfillcolor{currentfill}%
\pgfsetlinewidth{0.803000pt}%
\definecolor{currentstroke}{rgb}{0.000000,0.000000,0.000000}%
\pgfsetstrokecolor{currentstroke}%
\pgfsetdash{}{0pt}%
\pgfsys@defobject{currentmarker}{\pgfqpoint{0.000000in}{-0.048611in}}{\pgfqpoint{0.000000in}{0.000000in}}{%
\pgfpathmoveto{\pgfqpoint{0.000000in}{0.000000in}}%
\pgfpathlineto{\pgfqpoint{0.000000in}{-0.048611in}}%
\pgfusepath{stroke,fill}%
}%
\begin{pgfscope}%
\pgfsys@transformshift{3.505455in}{0.528000in}%
\pgfsys@useobject{currentmarker}{}%
\end{pgfscope}%
\end{pgfscope}%
\begin{pgfscope}%
\definecolor{textcolor}{rgb}{0.000000,0.000000,0.000000}%
\pgfsetstrokecolor{textcolor}%
\pgfsetfillcolor{textcolor}%
\pgftext[x=3.505455in,y=0.430778in,,top]{\color{textcolor}\sffamily\fontsize{10.000000}{12.000000}\selectfont 60}%
\end{pgfscope}%
\begin{pgfscope}%
\pgfsetbuttcap%
\pgfsetroundjoin%
\definecolor{currentfill}{rgb}{0.000000,0.000000,0.000000}%
\pgfsetfillcolor{currentfill}%
\pgfsetlinewidth{0.803000pt}%
\definecolor{currentstroke}{rgb}{0.000000,0.000000,0.000000}%
\pgfsetstrokecolor{currentstroke}%
\pgfsetdash{}{0pt}%
\pgfsys@defobject{currentmarker}{\pgfqpoint{0.000000in}{-0.048611in}}{\pgfqpoint{0.000000in}{0.000000in}}{%
\pgfpathmoveto{\pgfqpoint{0.000000in}{0.000000in}}%
\pgfpathlineto{\pgfqpoint{0.000000in}{-0.048611in}}%
\pgfusepath{stroke,fill}%
}%
\begin{pgfscope}%
\pgfsys@transformshift{3.956364in}{0.528000in}%
\pgfsys@useobject{currentmarker}{}%
\end{pgfscope}%
\end{pgfscope}%
\begin{pgfscope}%
\definecolor{textcolor}{rgb}{0.000000,0.000000,0.000000}%
\pgfsetstrokecolor{textcolor}%
\pgfsetfillcolor{textcolor}%
\pgftext[x=3.956364in,y=0.430778in,,top]{\color{textcolor}\sffamily\fontsize{10.000000}{12.000000}\selectfont 70}%
\end{pgfscope}%
\begin{pgfscope}%
\pgfsetbuttcap%
\pgfsetroundjoin%
\definecolor{currentfill}{rgb}{0.000000,0.000000,0.000000}%
\pgfsetfillcolor{currentfill}%
\pgfsetlinewidth{0.803000pt}%
\definecolor{currentstroke}{rgb}{0.000000,0.000000,0.000000}%
\pgfsetstrokecolor{currentstroke}%
\pgfsetdash{}{0pt}%
\pgfsys@defobject{currentmarker}{\pgfqpoint{0.000000in}{-0.048611in}}{\pgfqpoint{0.000000in}{0.000000in}}{%
\pgfpathmoveto{\pgfqpoint{0.000000in}{0.000000in}}%
\pgfpathlineto{\pgfqpoint{0.000000in}{-0.048611in}}%
\pgfusepath{stroke,fill}%
}%
\begin{pgfscope}%
\pgfsys@transformshift{4.407273in}{0.528000in}%
\pgfsys@useobject{currentmarker}{}%
\end{pgfscope}%
\end{pgfscope}%
\begin{pgfscope}%
\definecolor{textcolor}{rgb}{0.000000,0.000000,0.000000}%
\pgfsetstrokecolor{textcolor}%
\pgfsetfillcolor{textcolor}%
\pgftext[x=4.407273in,y=0.430778in,,top]{\color{textcolor}\sffamily\fontsize{10.000000}{12.000000}\selectfont 80}%
\end{pgfscope}%
\begin{pgfscope}%
\pgfsetbuttcap%
\pgfsetroundjoin%
\definecolor{currentfill}{rgb}{0.000000,0.000000,0.000000}%
\pgfsetfillcolor{currentfill}%
\pgfsetlinewidth{0.803000pt}%
\definecolor{currentstroke}{rgb}{0.000000,0.000000,0.000000}%
\pgfsetstrokecolor{currentstroke}%
\pgfsetdash{}{0pt}%
\pgfsys@defobject{currentmarker}{\pgfqpoint{0.000000in}{-0.048611in}}{\pgfqpoint{0.000000in}{0.000000in}}{%
\pgfpathmoveto{\pgfqpoint{0.000000in}{0.000000in}}%
\pgfpathlineto{\pgfqpoint{0.000000in}{-0.048611in}}%
\pgfusepath{stroke,fill}%
}%
\begin{pgfscope}%
\pgfsys@transformshift{4.858182in}{0.528000in}%
\pgfsys@useobject{currentmarker}{}%
\end{pgfscope}%
\end{pgfscope}%
\begin{pgfscope}%
\definecolor{textcolor}{rgb}{0.000000,0.000000,0.000000}%
\pgfsetstrokecolor{textcolor}%
\pgfsetfillcolor{textcolor}%
\pgftext[x=4.858182in,y=0.430778in,,top]{\color{textcolor}\sffamily\fontsize{10.000000}{12.000000}\selectfont 90}%
\end{pgfscope}%
\begin{pgfscope}%
\pgfsetbuttcap%
\pgfsetroundjoin%
\definecolor{currentfill}{rgb}{0.000000,0.000000,0.000000}%
\pgfsetfillcolor{currentfill}%
\pgfsetlinewidth{0.803000pt}%
\definecolor{currentstroke}{rgb}{0.000000,0.000000,0.000000}%
\pgfsetstrokecolor{currentstroke}%
\pgfsetdash{}{0pt}%
\pgfsys@defobject{currentmarker}{\pgfqpoint{0.000000in}{-0.048611in}}{\pgfqpoint{0.000000in}{0.000000in}}{%
\pgfpathmoveto{\pgfqpoint{0.000000in}{0.000000in}}%
\pgfpathlineto{\pgfqpoint{0.000000in}{-0.048611in}}%
\pgfusepath{stroke,fill}%
}%
\begin{pgfscope}%
\pgfsys@transformshift{5.309091in}{0.528000in}%
\pgfsys@useobject{currentmarker}{}%
\end{pgfscope}%
\end{pgfscope}%
\begin{pgfscope}%
\definecolor{textcolor}{rgb}{0.000000,0.000000,0.000000}%
\pgfsetstrokecolor{textcolor}%
\pgfsetfillcolor{textcolor}%
\pgftext[x=5.309091in,y=0.430778in,,top]{\color{textcolor}\sffamily\fontsize{10.000000}{12.000000}\selectfont 100}%
\end{pgfscope}%
\begin{pgfscope}%
\definecolor{textcolor}{rgb}{0.000000,0.000000,0.000000}%
\pgfsetstrokecolor{textcolor}%
\pgfsetfillcolor{textcolor}%
\pgftext[x=3.280000in,y=0.240809in,,top]{\color{textcolor}\sffamily\fontsize{10.000000}{12.000000}\selectfont \(\displaystyle k\)}%
\end{pgfscope}%
\begin{pgfscope}%
\pgfsetbuttcap%
\pgfsetroundjoin%
\definecolor{currentfill}{rgb}{0.000000,0.000000,0.000000}%
\pgfsetfillcolor{currentfill}%
\pgfsetlinewidth{0.803000pt}%
\definecolor{currentstroke}{rgb}{0.000000,0.000000,0.000000}%
\pgfsetstrokecolor{currentstroke}%
\pgfsetdash{}{0pt}%
\pgfsys@defobject{currentmarker}{\pgfqpoint{-0.048611in}{0.000000in}}{\pgfqpoint{0.000000in}{0.000000in}}{%
\pgfpathmoveto{\pgfqpoint{0.000000in}{0.000000in}}%
\pgfpathlineto{\pgfqpoint{-0.048611in}{0.000000in}}%
\pgfusepath{stroke,fill}%
}%
\begin{pgfscope}%
\pgfsys@transformshift{0.800000in}{0.770667in}%
\pgfsys@useobject{currentmarker}{}%
\end{pgfscope}%
\end{pgfscope}%
\begin{pgfscope}%
\definecolor{textcolor}{rgb}{0.000000,0.000000,0.000000}%
\pgfsetstrokecolor{textcolor}%
\pgfsetfillcolor{textcolor}%
\pgftext[x=0.526047in,y=0.717905in,left,base]{\color{textcolor}\sffamily\fontsize{10.000000}{12.000000}\selectfont 10}%
\end{pgfscope}%
\begin{pgfscope}%
\pgfsetbuttcap%
\pgfsetroundjoin%
\definecolor{currentfill}{rgb}{0.000000,0.000000,0.000000}%
\pgfsetfillcolor{currentfill}%
\pgfsetlinewidth{0.803000pt}%
\definecolor{currentstroke}{rgb}{0.000000,0.000000,0.000000}%
\pgfsetstrokecolor{currentstroke}%
\pgfsetdash{}{0pt}%
\pgfsys@defobject{currentmarker}{\pgfqpoint{-0.048611in}{0.000000in}}{\pgfqpoint{0.000000in}{0.000000in}}{%
\pgfpathmoveto{\pgfqpoint{0.000000in}{0.000000in}}%
\pgfpathlineto{\pgfqpoint{-0.048611in}{0.000000in}}%
\pgfusepath{stroke,fill}%
}%
\begin{pgfscope}%
\pgfsys@transformshift{0.800000in}{1.144000in}%
\pgfsys@useobject{currentmarker}{}%
\end{pgfscope}%
\end{pgfscope}%
\begin{pgfscope}%
\definecolor{textcolor}{rgb}{0.000000,0.000000,0.000000}%
\pgfsetstrokecolor{textcolor}%
\pgfsetfillcolor{textcolor}%
\pgftext[x=0.526047in,y=1.091238in,left,base]{\color{textcolor}\sffamily\fontsize{10.000000}{12.000000}\selectfont 20}%
\end{pgfscope}%
\begin{pgfscope}%
\pgfsetbuttcap%
\pgfsetroundjoin%
\definecolor{currentfill}{rgb}{0.000000,0.000000,0.000000}%
\pgfsetfillcolor{currentfill}%
\pgfsetlinewidth{0.803000pt}%
\definecolor{currentstroke}{rgb}{0.000000,0.000000,0.000000}%
\pgfsetstrokecolor{currentstroke}%
\pgfsetdash{}{0pt}%
\pgfsys@defobject{currentmarker}{\pgfqpoint{-0.048611in}{0.000000in}}{\pgfqpoint{0.000000in}{0.000000in}}{%
\pgfpathmoveto{\pgfqpoint{0.000000in}{0.000000in}}%
\pgfpathlineto{\pgfqpoint{-0.048611in}{0.000000in}}%
\pgfusepath{stroke,fill}%
}%
\begin{pgfscope}%
\pgfsys@transformshift{0.800000in}{1.517333in}%
\pgfsys@useobject{currentmarker}{}%
\end{pgfscope}%
\end{pgfscope}%
\begin{pgfscope}%
\definecolor{textcolor}{rgb}{0.000000,0.000000,0.000000}%
\pgfsetstrokecolor{textcolor}%
\pgfsetfillcolor{textcolor}%
\pgftext[x=0.526047in,y=1.464572in,left,base]{\color{textcolor}\sffamily\fontsize{10.000000}{12.000000}\selectfont 30}%
\end{pgfscope}%
\begin{pgfscope}%
\pgfsetbuttcap%
\pgfsetroundjoin%
\definecolor{currentfill}{rgb}{0.000000,0.000000,0.000000}%
\pgfsetfillcolor{currentfill}%
\pgfsetlinewidth{0.803000pt}%
\definecolor{currentstroke}{rgb}{0.000000,0.000000,0.000000}%
\pgfsetstrokecolor{currentstroke}%
\pgfsetdash{}{0pt}%
\pgfsys@defobject{currentmarker}{\pgfqpoint{-0.048611in}{0.000000in}}{\pgfqpoint{0.000000in}{0.000000in}}{%
\pgfpathmoveto{\pgfqpoint{0.000000in}{0.000000in}}%
\pgfpathlineto{\pgfqpoint{-0.048611in}{0.000000in}}%
\pgfusepath{stroke,fill}%
}%
\begin{pgfscope}%
\pgfsys@transformshift{0.800000in}{1.890667in}%
\pgfsys@useobject{currentmarker}{}%
\end{pgfscope}%
\end{pgfscope}%
\begin{pgfscope}%
\definecolor{textcolor}{rgb}{0.000000,0.000000,0.000000}%
\pgfsetstrokecolor{textcolor}%
\pgfsetfillcolor{textcolor}%
\pgftext[x=0.526047in,y=1.837905in,left,base]{\color{textcolor}\sffamily\fontsize{10.000000}{12.000000}\selectfont 40}%
\end{pgfscope}%
\begin{pgfscope}%
\pgfsetbuttcap%
\pgfsetroundjoin%
\definecolor{currentfill}{rgb}{0.000000,0.000000,0.000000}%
\pgfsetfillcolor{currentfill}%
\pgfsetlinewidth{0.803000pt}%
\definecolor{currentstroke}{rgb}{0.000000,0.000000,0.000000}%
\pgfsetstrokecolor{currentstroke}%
\pgfsetdash{}{0pt}%
\pgfsys@defobject{currentmarker}{\pgfqpoint{-0.048611in}{0.000000in}}{\pgfqpoint{0.000000in}{0.000000in}}{%
\pgfpathmoveto{\pgfqpoint{0.000000in}{0.000000in}}%
\pgfpathlineto{\pgfqpoint{-0.048611in}{0.000000in}}%
\pgfusepath{stroke,fill}%
}%
\begin{pgfscope}%
\pgfsys@transformshift{0.800000in}{2.264000in}%
\pgfsys@useobject{currentmarker}{}%
\end{pgfscope}%
\end{pgfscope}%
\begin{pgfscope}%
\definecolor{textcolor}{rgb}{0.000000,0.000000,0.000000}%
\pgfsetstrokecolor{textcolor}%
\pgfsetfillcolor{textcolor}%
\pgftext[x=0.526047in,y=2.211238in,left,base]{\color{textcolor}\sffamily\fontsize{10.000000}{12.000000}\selectfont 50}%
\end{pgfscope}%
\begin{pgfscope}%
\pgfsetbuttcap%
\pgfsetroundjoin%
\definecolor{currentfill}{rgb}{0.000000,0.000000,0.000000}%
\pgfsetfillcolor{currentfill}%
\pgfsetlinewidth{0.803000pt}%
\definecolor{currentstroke}{rgb}{0.000000,0.000000,0.000000}%
\pgfsetstrokecolor{currentstroke}%
\pgfsetdash{}{0pt}%
\pgfsys@defobject{currentmarker}{\pgfqpoint{-0.048611in}{0.000000in}}{\pgfqpoint{0.000000in}{0.000000in}}{%
\pgfpathmoveto{\pgfqpoint{0.000000in}{0.000000in}}%
\pgfpathlineto{\pgfqpoint{-0.048611in}{0.000000in}}%
\pgfusepath{stroke,fill}%
}%
\begin{pgfscope}%
\pgfsys@transformshift{0.800000in}{2.637333in}%
\pgfsys@useobject{currentmarker}{}%
\end{pgfscope}%
\end{pgfscope}%
\begin{pgfscope}%
\definecolor{textcolor}{rgb}{0.000000,0.000000,0.000000}%
\pgfsetstrokecolor{textcolor}%
\pgfsetfillcolor{textcolor}%
\pgftext[x=0.526047in,y=2.584572in,left,base]{\color{textcolor}\sffamily\fontsize{10.000000}{12.000000}\selectfont 60}%
\end{pgfscope}%
\begin{pgfscope}%
\pgfsetbuttcap%
\pgfsetroundjoin%
\definecolor{currentfill}{rgb}{0.000000,0.000000,0.000000}%
\pgfsetfillcolor{currentfill}%
\pgfsetlinewidth{0.803000pt}%
\definecolor{currentstroke}{rgb}{0.000000,0.000000,0.000000}%
\pgfsetstrokecolor{currentstroke}%
\pgfsetdash{}{0pt}%
\pgfsys@defobject{currentmarker}{\pgfqpoint{-0.048611in}{0.000000in}}{\pgfqpoint{0.000000in}{0.000000in}}{%
\pgfpathmoveto{\pgfqpoint{0.000000in}{0.000000in}}%
\pgfpathlineto{\pgfqpoint{-0.048611in}{0.000000in}}%
\pgfusepath{stroke,fill}%
}%
\begin{pgfscope}%
\pgfsys@transformshift{0.800000in}{3.010667in}%
\pgfsys@useobject{currentmarker}{}%
\end{pgfscope}%
\end{pgfscope}%
\begin{pgfscope}%
\definecolor{textcolor}{rgb}{0.000000,0.000000,0.000000}%
\pgfsetstrokecolor{textcolor}%
\pgfsetfillcolor{textcolor}%
\pgftext[x=0.526047in,y=2.957905in,left,base]{\color{textcolor}\sffamily\fontsize{10.000000}{12.000000}\selectfont 70}%
\end{pgfscope}%
\begin{pgfscope}%
\pgfsetbuttcap%
\pgfsetroundjoin%
\definecolor{currentfill}{rgb}{0.000000,0.000000,0.000000}%
\pgfsetfillcolor{currentfill}%
\pgfsetlinewidth{0.803000pt}%
\definecolor{currentstroke}{rgb}{0.000000,0.000000,0.000000}%
\pgfsetstrokecolor{currentstroke}%
\pgfsetdash{}{0pt}%
\pgfsys@defobject{currentmarker}{\pgfqpoint{-0.048611in}{0.000000in}}{\pgfqpoint{0.000000in}{0.000000in}}{%
\pgfpathmoveto{\pgfqpoint{0.000000in}{0.000000in}}%
\pgfpathlineto{\pgfqpoint{-0.048611in}{0.000000in}}%
\pgfusepath{stroke,fill}%
}%
\begin{pgfscope}%
\pgfsys@transformshift{0.800000in}{3.384000in}%
\pgfsys@useobject{currentmarker}{}%
\end{pgfscope}%
\end{pgfscope}%
\begin{pgfscope}%
\definecolor{textcolor}{rgb}{0.000000,0.000000,0.000000}%
\pgfsetstrokecolor{textcolor}%
\pgfsetfillcolor{textcolor}%
\pgftext[x=0.526047in,y=3.331238in,left,base]{\color{textcolor}\sffamily\fontsize{10.000000}{12.000000}\selectfont 80}%
\end{pgfscope}%
\begin{pgfscope}%
\pgfsetbuttcap%
\pgfsetroundjoin%
\definecolor{currentfill}{rgb}{0.000000,0.000000,0.000000}%
\pgfsetfillcolor{currentfill}%
\pgfsetlinewidth{0.803000pt}%
\definecolor{currentstroke}{rgb}{0.000000,0.000000,0.000000}%
\pgfsetstrokecolor{currentstroke}%
\pgfsetdash{}{0pt}%
\pgfsys@defobject{currentmarker}{\pgfqpoint{-0.048611in}{0.000000in}}{\pgfqpoint{0.000000in}{0.000000in}}{%
\pgfpathmoveto{\pgfqpoint{0.000000in}{0.000000in}}%
\pgfpathlineto{\pgfqpoint{-0.048611in}{0.000000in}}%
\pgfusepath{stroke,fill}%
}%
\begin{pgfscope}%
\pgfsys@transformshift{0.800000in}{3.757333in}%
\pgfsys@useobject{currentmarker}{}%
\end{pgfscope}%
\end{pgfscope}%
\begin{pgfscope}%
\definecolor{textcolor}{rgb}{0.000000,0.000000,0.000000}%
\pgfsetstrokecolor{textcolor}%
\pgfsetfillcolor{textcolor}%
\pgftext[x=0.526047in,y=3.704572in,left,base]{\color{textcolor}\sffamily\fontsize{10.000000}{12.000000}\selectfont 90}%
\end{pgfscope}%
\begin{pgfscope}%
\pgfsetbuttcap%
\pgfsetroundjoin%
\definecolor{currentfill}{rgb}{0.000000,0.000000,0.000000}%
\pgfsetfillcolor{currentfill}%
\pgfsetlinewidth{0.803000pt}%
\definecolor{currentstroke}{rgb}{0.000000,0.000000,0.000000}%
\pgfsetstrokecolor{currentstroke}%
\pgfsetdash{}{0pt}%
\pgfsys@defobject{currentmarker}{\pgfqpoint{-0.048611in}{0.000000in}}{\pgfqpoint{0.000000in}{0.000000in}}{%
\pgfpathmoveto{\pgfqpoint{0.000000in}{0.000000in}}%
\pgfpathlineto{\pgfqpoint{-0.048611in}{0.000000in}}%
\pgfusepath{stroke,fill}%
}%
\begin{pgfscope}%
\pgfsys@transformshift{0.800000in}{4.130667in}%
\pgfsys@useobject{currentmarker}{}%
\end{pgfscope}%
\end{pgfscope}%
\begin{pgfscope}%
\definecolor{textcolor}{rgb}{0.000000,0.000000,0.000000}%
\pgfsetstrokecolor{textcolor}%
\pgfsetfillcolor{textcolor}%
\pgftext[x=0.437682in,y=4.077905in,left,base]{\color{textcolor}\sffamily\fontsize{10.000000}{12.000000}\selectfont 100}%
\end{pgfscope}%
\begin{pgfscope}%
\definecolor{textcolor}{rgb}{0.000000,0.000000,0.000000}%
\pgfsetstrokecolor{textcolor}%
\pgfsetfillcolor{textcolor}%
\pgftext[x=0.382126in,y=2.376000in,,bottom,rotate=90.000000]{\color{textcolor}\sffamily\fontsize{10.000000}{12.000000}\selectfont Number of GMRES iterations}%
\end{pgfscope}%
\begin{pgfscope}%
\pgfpathrectangle{\pgfqpoint{0.800000in}{0.528000in}}{\pgfqpoint{4.960000in}{3.696000in}}%
\pgfusepath{clip}%
\pgfsetbuttcap%
\pgfsetroundjoin%
\pgfsetlinewidth{1.505625pt}%
\definecolor{currentstroke}{rgb}{0.000000,0.000000,0.000000}%
\pgfsetstrokecolor{currentstroke}%
\pgfsetdash{{5.550000pt}{2.400000pt}}{0.000000pt}%
\pgfpathmoveto{\pgfqpoint{1.250909in}{0.770667in}}%
\pgfpathlineto{\pgfqpoint{1.701818in}{0.957333in}}%
\pgfpathlineto{\pgfqpoint{2.152727in}{1.144000in}}%
\pgfpathlineto{\pgfqpoint{2.603636in}{1.330667in}}%
\pgfpathlineto{\pgfqpoint{3.054545in}{1.517333in}}%
\pgfpathlineto{\pgfqpoint{3.505455in}{1.965333in}}%
\pgfpathlineto{\pgfqpoint{3.956364in}{2.376000in}}%
\pgfpathlineto{\pgfqpoint{4.407273in}{2.786667in}}%
\pgfpathlineto{\pgfqpoint{4.858182in}{3.421333in}}%
\pgfpathlineto{\pgfqpoint{5.309091in}{4.056000in}}%
\pgfusepath{stroke}%
\end{pgfscope}%
\begin{pgfscope}%
\pgfpathrectangle{\pgfqpoint{0.800000in}{0.528000in}}{\pgfqpoint{4.960000in}{3.696000in}}%
\pgfusepath{clip}%
\pgfsetbuttcap%
\pgfsetroundjoin%
\definecolor{currentfill}{rgb}{0.000000,0.000000,0.000000}%
\pgfsetfillcolor{currentfill}%
\pgfsetlinewidth{1.003750pt}%
\definecolor{currentstroke}{rgb}{0.000000,0.000000,0.000000}%
\pgfsetstrokecolor{currentstroke}%
\pgfsetdash{}{0pt}%
\pgfsys@defobject{currentmarker}{\pgfqpoint{-0.041667in}{-0.041667in}}{\pgfqpoint{0.041667in}{0.041667in}}{%
\pgfpathmoveto{\pgfqpoint{0.000000in}{-0.041667in}}%
\pgfpathcurveto{\pgfqpoint{0.011050in}{-0.041667in}}{\pgfqpoint{0.021649in}{-0.037276in}}{\pgfqpoint{0.029463in}{-0.029463in}}%
\pgfpathcurveto{\pgfqpoint{0.037276in}{-0.021649in}}{\pgfqpoint{0.041667in}{-0.011050in}}{\pgfqpoint{0.041667in}{0.000000in}}%
\pgfpathcurveto{\pgfqpoint{0.041667in}{0.011050in}}{\pgfqpoint{0.037276in}{0.021649in}}{\pgfqpoint{0.029463in}{0.029463in}}%
\pgfpathcurveto{\pgfqpoint{0.021649in}{0.037276in}}{\pgfqpoint{0.011050in}{0.041667in}}{\pgfqpoint{0.000000in}{0.041667in}}%
\pgfpathcurveto{\pgfqpoint{-0.011050in}{0.041667in}}{\pgfqpoint{-0.021649in}{0.037276in}}{\pgfqpoint{-0.029463in}{0.029463in}}%
\pgfpathcurveto{\pgfqpoint{-0.037276in}{0.021649in}}{\pgfqpoint{-0.041667in}{0.011050in}}{\pgfqpoint{-0.041667in}{0.000000in}}%
\pgfpathcurveto{\pgfqpoint{-0.041667in}{-0.011050in}}{\pgfqpoint{-0.037276in}{-0.021649in}}{\pgfqpoint{-0.029463in}{-0.029463in}}%
\pgfpathcurveto{\pgfqpoint{-0.021649in}{-0.037276in}}{\pgfqpoint{-0.011050in}{-0.041667in}}{\pgfqpoint{0.000000in}{-0.041667in}}%
\pgfpathclose%
\pgfusepath{stroke,fill}%
}%
\begin{pgfscope}%
\pgfsys@transformshift{1.250909in}{0.770667in}%
\pgfsys@useobject{currentmarker}{}%
\end{pgfscope}%
\begin{pgfscope}%
\pgfsys@transformshift{1.701818in}{0.957333in}%
\pgfsys@useobject{currentmarker}{}%
\end{pgfscope}%
\begin{pgfscope}%
\pgfsys@transformshift{2.152727in}{1.144000in}%
\pgfsys@useobject{currentmarker}{}%
\end{pgfscope}%
\begin{pgfscope}%
\pgfsys@transformshift{2.603636in}{1.330667in}%
\pgfsys@useobject{currentmarker}{}%
\end{pgfscope}%
\begin{pgfscope}%
\pgfsys@transformshift{3.054545in}{1.517333in}%
\pgfsys@useobject{currentmarker}{}%
\end{pgfscope}%
\begin{pgfscope}%
\pgfsys@transformshift{3.505455in}{1.965333in}%
\pgfsys@useobject{currentmarker}{}%
\end{pgfscope}%
\begin{pgfscope}%
\pgfsys@transformshift{3.956364in}{2.376000in}%
\pgfsys@useobject{currentmarker}{}%
\end{pgfscope}%
\begin{pgfscope}%
\pgfsys@transformshift{4.407273in}{2.786667in}%
\pgfsys@useobject{currentmarker}{}%
\end{pgfscope}%
\begin{pgfscope}%
\pgfsys@transformshift{4.858182in}{3.421333in}%
\pgfsys@useobject{currentmarker}{}%
\end{pgfscope}%
\begin{pgfscope}%
\pgfsys@transformshift{5.309091in}{4.056000in}%
\pgfsys@useobject{currentmarker}{}%
\end{pgfscope}%
\end{pgfscope}%
\begin{pgfscope}%
\pgfpathrectangle{\pgfqpoint{0.800000in}{0.528000in}}{\pgfqpoint{4.960000in}{3.696000in}}%
\pgfusepath{clip}%
\pgfsetbuttcap%
\pgfsetroundjoin%
\pgfsetlinewidth{1.505625pt}%
\definecolor{currentstroke}{rgb}{0.000000,0.000000,0.000000}%
\pgfsetstrokecolor{currentstroke}%
\pgfsetdash{{5.550000pt}{2.400000pt}}{0.000000pt}%
\pgfpathmoveto{\pgfqpoint{1.250909in}{0.770667in}}%
\pgfpathlineto{\pgfqpoint{1.701818in}{0.882667in}}%
\pgfpathlineto{\pgfqpoint{2.152727in}{0.994667in}}%
\pgfpathlineto{\pgfqpoint{2.603636in}{1.106667in}}%
\pgfpathlineto{\pgfqpoint{3.054545in}{1.218667in}}%
\pgfpathlineto{\pgfqpoint{3.505455in}{1.330667in}}%
\pgfpathlineto{\pgfqpoint{3.956364in}{1.442667in}}%
\pgfpathlineto{\pgfqpoint{4.407273in}{1.554667in}}%
\pgfpathlineto{\pgfqpoint{4.858182in}{1.778667in}}%
\pgfpathlineto{\pgfqpoint{5.309091in}{1.928000in}}%
\pgfusepath{stroke}%
\end{pgfscope}%
\begin{pgfscope}%
\pgfpathrectangle{\pgfqpoint{0.800000in}{0.528000in}}{\pgfqpoint{4.960000in}{3.696000in}}%
\pgfusepath{clip}%
\pgfsetbuttcap%
\pgfsetmiterjoin%
\definecolor{currentfill}{rgb}{0.000000,0.000000,0.000000}%
\pgfsetfillcolor{currentfill}%
\pgfsetlinewidth{1.003750pt}%
\definecolor{currentstroke}{rgb}{0.000000,0.000000,0.000000}%
\pgfsetstrokecolor{currentstroke}%
\pgfsetdash{}{0pt}%
\pgfsys@defobject{currentmarker}{\pgfqpoint{-0.041667in}{-0.041667in}}{\pgfqpoint{0.041667in}{0.041667in}}{%
\pgfpathmoveto{\pgfqpoint{-0.000000in}{-0.041667in}}%
\pgfpathlineto{\pgfqpoint{0.041667in}{0.041667in}}%
\pgfpathlineto{\pgfqpoint{-0.041667in}{0.041667in}}%
\pgfpathclose%
\pgfusepath{stroke,fill}%
}%
\begin{pgfscope}%
\pgfsys@transformshift{1.250909in}{0.770667in}%
\pgfsys@useobject{currentmarker}{}%
\end{pgfscope}%
\begin{pgfscope}%
\pgfsys@transformshift{1.701818in}{0.882667in}%
\pgfsys@useobject{currentmarker}{}%
\end{pgfscope}%
\begin{pgfscope}%
\pgfsys@transformshift{2.152727in}{0.994667in}%
\pgfsys@useobject{currentmarker}{}%
\end{pgfscope}%
\begin{pgfscope}%
\pgfsys@transformshift{2.603636in}{1.106667in}%
\pgfsys@useobject{currentmarker}{}%
\end{pgfscope}%
\begin{pgfscope}%
\pgfsys@transformshift{3.054545in}{1.218667in}%
\pgfsys@useobject{currentmarker}{}%
\end{pgfscope}%
\begin{pgfscope}%
\pgfsys@transformshift{3.505455in}{1.330667in}%
\pgfsys@useobject{currentmarker}{}%
\end{pgfscope}%
\begin{pgfscope}%
\pgfsys@transformshift{3.956364in}{1.442667in}%
\pgfsys@useobject{currentmarker}{}%
\end{pgfscope}%
\begin{pgfscope}%
\pgfsys@transformshift{4.407273in}{1.554667in}%
\pgfsys@useobject{currentmarker}{}%
\end{pgfscope}%
\begin{pgfscope}%
\pgfsys@transformshift{4.858182in}{1.778667in}%
\pgfsys@useobject{currentmarker}{}%
\end{pgfscope}%
\begin{pgfscope}%
\pgfsys@transformshift{5.309091in}{1.928000in}%
\pgfsys@useobject{currentmarker}{}%
\end{pgfscope}%
\end{pgfscope}%
\begin{pgfscope}%
\pgfpathrectangle{\pgfqpoint{0.800000in}{0.528000in}}{\pgfqpoint{4.960000in}{3.696000in}}%
\pgfusepath{clip}%
\pgfsetbuttcap%
\pgfsetroundjoin%
\pgfsetlinewidth{1.505625pt}%
\definecolor{currentstroke}{rgb}{0.000000,0.000000,0.000000}%
\pgfsetstrokecolor{currentstroke}%
\pgfsetdash{{5.550000pt}{2.400000pt}}{0.000000pt}%
\pgfpathmoveto{\pgfqpoint{1.250909in}{0.733333in}}%
\pgfpathlineto{\pgfqpoint{1.701818in}{0.808000in}}%
\pgfpathlineto{\pgfqpoint{2.152727in}{0.882667in}}%
\pgfpathlineto{\pgfqpoint{2.603636in}{0.920000in}}%
\pgfpathlineto{\pgfqpoint{3.054545in}{0.994667in}}%
\pgfpathlineto{\pgfqpoint{3.505455in}{1.032000in}}%
\pgfpathlineto{\pgfqpoint{3.956364in}{1.106667in}}%
\pgfpathlineto{\pgfqpoint{4.407273in}{1.106667in}}%
\pgfpathlineto{\pgfqpoint{4.858182in}{1.181333in}}%
\pgfpathlineto{\pgfqpoint{5.309091in}{1.218667in}}%
\pgfusepath{stroke}%
\end{pgfscope}%
\begin{pgfscope}%
\pgfpathrectangle{\pgfqpoint{0.800000in}{0.528000in}}{\pgfqpoint{4.960000in}{3.696000in}}%
\pgfusepath{clip}%
\pgfsetbuttcap%
\pgfsetmiterjoin%
\definecolor{currentfill}{rgb}{0.000000,0.000000,0.000000}%
\pgfsetfillcolor{currentfill}%
\pgfsetlinewidth{1.003750pt}%
\definecolor{currentstroke}{rgb}{0.000000,0.000000,0.000000}%
\pgfsetstrokecolor{currentstroke}%
\pgfsetdash{}{0pt}%
\pgfsys@defobject{currentmarker}{\pgfqpoint{-0.041667in}{-0.041667in}}{\pgfqpoint{0.041667in}{0.041667in}}{%
\pgfpathmoveto{\pgfqpoint{-0.020833in}{-0.041667in}}%
\pgfpathlineto{\pgfqpoint{0.000000in}{-0.020833in}}%
\pgfpathlineto{\pgfqpoint{0.020833in}{-0.041667in}}%
\pgfpathlineto{\pgfqpoint{0.041667in}{-0.020833in}}%
\pgfpathlineto{\pgfqpoint{0.020833in}{0.000000in}}%
\pgfpathlineto{\pgfqpoint{0.041667in}{0.020833in}}%
\pgfpathlineto{\pgfqpoint{0.020833in}{0.041667in}}%
\pgfpathlineto{\pgfqpoint{0.000000in}{0.020833in}}%
\pgfpathlineto{\pgfqpoint{-0.020833in}{0.041667in}}%
\pgfpathlineto{\pgfqpoint{-0.041667in}{0.020833in}}%
\pgfpathlineto{\pgfqpoint{-0.020833in}{0.000000in}}%
\pgfpathlineto{\pgfqpoint{-0.041667in}{-0.020833in}}%
\pgfpathclose%
\pgfusepath{stroke,fill}%
}%
\begin{pgfscope}%
\pgfsys@transformshift{1.250909in}{0.733333in}%
\pgfsys@useobject{currentmarker}{}%
\end{pgfscope}%
\begin{pgfscope}%
\pgfsys@transformshift{1.701818in}{0.808000in}%
\pgfsys@useobject{currentmarker}{}%
\end{pgfscope}%
\begin{pgfscope}%
\pgfsys@transformshift{2.152727in}{0.882667in}%
\pgfsys@useobject{currentmarker}{}%
\end{pgfscope}%
\begin{pgfscope}%
\pgfsys@transformshift{2.603636in}{0.920000in}%
\pgfsys@useobject{currentmarker}{}%
\end{pgfscope}%
\begin{pgfscope}%
\pgfsys@transformshift{3.054545in}{0.994667in}%
\pgfsys@useobject{currentmarker}{}%
\end{pgfscope}%
\begin{pgfscope}%
\pgfsys@transformshift{3.505455in}{1.032000in}%
\pgfsys@useobject{currentmarker}{}%
\end{pgfscope}%
\begin{pgfscope}%
\pgfsys@transformshift{3.956364in}{1.106667in}%
\pgfsys@useobject{currentmarker}{}%
\end{pgfscope}%
\begin{pgfscope}%
\pgfsys@transformshift{4.407273in}{1.106667in}%
\pgfsys@useobject{currentmarker}{}%
\end{pgfscope}%
\begin{pgfscope}%
\pgfsys@transformshift{4.858182in}{1.181333in}%
\pgfsys@useobject{currentmarker}{}%
\end{pgfscope}%
\begin{pgfscope}%
\pgfsys@transformshift{5.309091in}{1.218667in}%
\pgfsys@useobject{currentmarker}{}%
\end{pgfscope}%
\end{pgfscope}%
\begin{pgfscope}%
\pgfpathrectangle{\pgfqpoint{0.800000in}{0.528000in}}{\pgfqpoint{4.960000in}{3.696000in}}%
\pgfusepath{clip}%
\pgfsetbuttcap%
\pgfsetroundjoin%
\pgfsetlinewidth{1.505625pt}%
\definecolor{currentstroke}{rgb}{0.000000,0.000000,0.000000}%
\pgfsetstrokecolor{currentstroke}%
\pgfsetdash{{5.550000pt}{2.400000pt}}{0.000000pt}%
\pgfpathmoveto{\pgfqpoint{1.250909in}{0.696000in}}%
\pgfpathlineto{\pgfqpoint{1.701818in}{0.733333in}}%
\pgfpathlineto{\pgfqpoint{2.152727in}{0.770667in}}%
\pgfpathlineto{\pgfqpoint{2.603636in}{0.808000in}}%
\pgfpathlineto{\pgfqpoint{3.054545in}{0.845333in}}%
\pgfpathlineto{\pgfqpoint{3.505455in}{0.882667in}}%
\pgfpathlineto{\pgfqpoint{3.956364in}{0.882667in}}%
\pgfpathlineto{\pgfqpoint{4.407273in}{0.920000in}}%
\pgfpathlineto{\pgfqpoint{4.858182in}{0.920000in}}%
\pgfpathlineto{\pgfqpoint{5.309091in}{0.920000in}}%
\pgfusepath{stroke}%
\end{pgfscope}%
\begin{pgfscope}%
\pgfpathrectangle{\pgfqpoint{0.800000in}{0.528000in}}{\pgfqpoint{4.960000in}{3.696000in}}%
\pgfusepath{clip}%
\pgfsetbuttcap%
\pgfsetmiterjoin%
\definecolor{currentfill}{rgb}{0.000000,0.000000,0.000000}%
\pgfsetfillcolor{currentfill}%
\pgfsetlinewidth{1.003750pt}%
\definecolor{currentstroke}{rgb}{0.000000,0.000000,0.000000}%
\pgfsetstrokecolor{currentstroke}%
\pgfsetdash{}{0pt}%
\pgfsys@defobject{currentmarker}{\pgfqpoint{-0.035355in}{-0.058926in}}{\pgfqpoint{0.035355in}{0.058926in}}{%
\pgfpathmoveto{\pgfqpoint{-0.000000in}{-0.058926in}}%
\pgfpathlineto{\pgfqpoint{0.035355in}{0.000000in}}%
\pgfpathlineto{\pgfqpoint{0.000000in}{0.058926in}}%
\pgfpathlineto{\pgfqpoint{-0.035355in}{0.000000in}}%
\pgfpathclose%
\pgfusepath{stroke,fill}%
}%
\begin{pgfscope}%
\pgfsys@transformshift{1.250909in}{0.696000in}%
\pgfsys@useobject{currentmarker}{}%
\end{pgfscope}%
\begin{pgfscope}%
\pgfsys@transformshift{1.701818in}{0.733333in}%
\pgfsys@useobject{currentmarker}{}%
\end{pgfscope}%
\begin{pgfscope}%
\pgfsys@transformshift{2.152727in}{0.770667in}%
\pgfsys@useobject{currentmarker}{}%
\end{pgfscope}%
\begin{pgfscope}%
\pgfsys@transformshift{2.603636in}{0.808000in}%
\pgfsys@useobject{currentmarker}{}%
\end{pgfscope}%
\begin{pgfscope}%
\pgfsys@transformshift{3.054545in}{0.845333in}%
\pgfsys@useobject{currentmarker}{}%
\end{pgfscope}%
\begin{pgfscope}%
\pgfsys@transformshift{3.505455in}{0.882667in}%
\pgfsys@useobject{currentmarker}{}%
\end{pgfscope}%
\begin{pgfscope}%
\pgfsys@transformshift{3.956364in}{0.882667in}%
\pgfsys@useobject{currentmarker}{}%
\end{pgfscope}%
\begin{pgfscope}%
\pgfsys@transformshift{4.407273in}{0.920000in}%
\pgfsys@useobject{currentmarker}{}%
\end{pgfscope}%
\begin{pgfscope}%
\pgfsys@transformshift{4.858182in}{0.920000in}%
\pgfsys@useobject{currentmarker}{}%
\end{pgfscope}%
\begin{pgfscope}%
\pgfsys@transformshift{5.309091in}{0.920000in}%
\pgfsys@useobject{currentmarker}{}%
\end{pgfscope}%
\end{pgfscope}%
\begin{pgfscope}%
\pgfsetrectcap%
\pgfsetmiterjoin%
\pgfsetlinewidth{0.803000pt}%
\definecolor{currentstroke}{rgb}{0.000000,0.000000,0.000000}%
\pgfsetstrokecolor{currentstroke}%
\pgfsetdash{}{0pt}%
\pgfpathmoveto{\pgfqpoint{0.800000in}{0.528000in}}%
\pgfpathlineto{\pgfqpoint{0.800000in}{4.224000in}}%
\pgfusepath{stroke}%
\end{pgfscope}%
\begin{pgfscope}%
\pgfsetrectcap%
\pgfsetmiterjoin%
\pgfsetlinewidth{0.803000pt}%
\definecolor{currentstroke}{rgb}{0.000000,0.000000,0.000000}%
\pgfsetstrokecolor{currentstroke}%
\pgfsetdash{}{0pt}%
\pgfpathmoveto{\pgfqpoint{5.760000in}{0.528000in}}%
\pgfpathlineto{\pgfqpoint{5.760000in}{4.224000in}}%
\pgfusepath{stroke}%
\end{pgfscope}%
\begin{pgfscope}%
\pgfsetrectcap%
\pgfsetmiterjoin%
\pgfsetlinewidth{0.803000pt}%
\definecolor{currentstroke}{rgb}{0.000000,0.000000,0.000000}%
\pgfsetstrokecolor{currentstroke}%
\pgfsetdash{}{0pt}%
\pgfpathmoveto{\pgfqpoint{0.800000in}{0.528000in}}%
\pgfpathlineto{\pgfqpoint{5.760000in}{0.528000in}}%
\pgfusepath{stroke}%
\end{pgfscope}%
\begin{pgfscope}%
\pgfsetrectcap%
\pgfsetmiterjoin%
\pgfsetlinewidth{0.803000pt}%
\definecolor{currentstroke}{rgb}{0.000000,0.000000,0.000000}%
\pgfsetstrokecolor{currentstroke}%
\pgfsetdash{}{0pt}%
\pgfpathmoveto{\pgfqpoint{0.800000in}{4.224000in}}%
\pgfpathlineto{\pgfqpoint{5.760000in}{4.224000in}}%
\pgfusepath{stroke}%
\end{pgfscope}%
\begin{pgfscope}%
\pgfsetbuttcap%
\pgfsetmiterjoin%
\definecolor{currentfill}{rgb}{1.000000,1.000000,1.000000}%
\pgfsetfillcolor{currentfill}%
\pgfsetfillopacity{0.800000}%
\pgfsetlinewidth{1.003750pt}%
\definecolor{currentstroke}{rgb}{0.800000,0.800000,0.800000}%
\pgfsetstrokecolor{currentstroke}%
\pgfsetstrokeopacity{0.800000}%
\pgfsetdash{}{0pt}%
\pgfpathmoveto{\pgfqpoint{0.897222in}{3.297460in}}%
\pgfpathlineto{\pgfqpoint{1.790209in}{3.297460in}}%
\pgfpathquadraticcurveto{\pgfqpoint{1.817987in}{3.297460in}}{\pgfqpoint{1.817987in}{3.325238in}}%
\pgfpathlineto{\pgfqpoint{1.817987in}{4.126778in}}%
\pgfpathquadraticcurveto{\pgfqpoint{1.817987in}{4.154556in}}{\pgfqpoint{1.790209in}{4.154556in}}%
\pgfpathlineto{\pgfqpoint{0.897222in}{4.154556in}}%
\pgfpathquadraticcurveto{\pgfqpoint{0.869444in}{4.154556in}}{\pgfqpoint{0.869444in}{4.126778in}}%
\pgfpathlineto{\pgfqpoint{0.869444in}{3.325238in}}%
\pgfpathquadraticcurveto{\pgfqpoint{0.869444in}{3.297460in}}{\pgfqpoint{0.897222in}{3.297460in}}%
\pgfpathclose%
\pgfusepath{stroke,fill}%
\end{pgfscope}%
\begin{pgfscope}%
\pgfsetbuttcap%
\pgfsetroundjoin%
\pgfsetlinewidth{1.505625pt}%
\definecolor{currentstroke}{rgb}{0.000000,0.000000,0.000000}%
\pgfsetstrokecolor{currentstroke}%
\pgfsetdash{{5.550000pt}{2.400000pt}}{0.000000pt}%
\pgfpathmoveto{\pgfqpoint{0.925000in}{4.042088in}}%
\pgfpathlineto{\pgfqpoint{1.202778in}{4.042088in}}%
\pgfusepath{stroke}%
\end{pgfscope}%
\begin{pgfscope}%
\pgfsetbuttcap%
\pgfsetroundjoin%
\definecolor{currentfill}{rgb}{0.000000,0.000000,0.000000}%
\pgfsetfillcolor{currentfill}%
\pgfsetlinewidth{1.003750pt}%
\definecolor{currentstroke}{rgb}{0.000000,0.000000,0.000000}%
\pgfsetstrokecolor{currentstroke}%
\pgfsetdash{}{0pt}%
\pgfsys@defobject{currentmarker}{\pgfqpoint{-0.041667in}{-0.041667in}}{\pgfqpoint{0.041667in}{0.041667in}}{%
\pgfpathmoveto{\pgfqpoint{0.000000in}{-0.041667in}}%
\pgfpathcurveto{\pgfqpoint{0.011050in}{-0.041667in}}{\pgfqpoint{0.021649in}{-0.037276in}}{\pgfqpoint{0.029463in}{-0.029463in}}%
\pgfpathcurveto{\pgfqpoint{0.037276in}{-0.021649in}}{\pgfqpoint{0.041667in}{-0.011050in}}{\pgfqpoint{0.041667in}{0.000000in}}%
\pgfpathcurveto{\pgfqpoint{0.041667in}{0.011050in}}{\pgfqpoint{0.037276in}{0.021649in}}{\pgfqpoint{0.029463in}{0.029463in}}%
\pgfpathcurveto{\pgfqpoint{0.021649in}{0.037276in}}{\pgfqpoint{0.011050in}{0.041667in}}{\pgfqpoint{0.000000in}{0.041667in}}%
\pgfpathcurveto{\pgfqpoint{-0.011050in}{0.041667in}}{\pgfqpoint{-0.021649in}{0.037276in}}{\pgfqpoint{-0.029463in}{0.029463in}}%
\pgfpathcurveto{\pgfqpoint{-0.037276in}{0.021649in}}{\pgfqpoint{-0.041667in}{0.011050in}}{\pgfqpoint{-0.041667in}{0.000000in}}%
\pgfpathcurveto{\pgfqpoint{-0.041667in}{-0.011050in}}{\pgfqpoint{-0.037276in}{-0.021649in}}{\pgfqpoint{-0.029463in}{-0.029463in}}%
\pgfpathcurveto{\pgfqpoint{-0.021649in}{-0.037276in}}{\pgfqpoint{-0.011050in}{-0.041667in}}{\pgfqpoint{0.000000in}{-0.041667in}}%
\pgfpathclose%
\pgfusepath{stroke,fill}%
}%
\begin{pgfscope}%
\pgfsys@transformshift{1.063889in}{4.042088in}%
\pgfsys@useobject{currentmarker}{}%
\end{pgfscope}%
\end{pgfscope}%
\begin{pgfscope}%
\definecolor{textcolor}{rgb}{0.000000,0.000000,0.000000}%
\pgfsetstrokecolor{textcolor}%
\pgfsetfillcolor{textcolor}%
\pgftext[x=1.313889in,y=3.993477in,left,base]{\color{textcolor}\sffamily\fontsize{10.000000}{12.000000}\selectfont \(\displaystyle \beta = 0.4\)}%
\end{pgfscope}%
\begin{pgfscope}%
\pgfsetbuttcap%
\pgfsetroundjoin%
\pgfsetlinewidth{1.505625pt}%
\definecolor{currentstroke}{rgb}{0.000000,0.000000,0.000000}%
\pgfsetstrokecolor{currentstroke}%
\pgfsetdash{{5.550000pt}{2.400000pt}}{0.000000pt}%
\pgfpathmoveto{\pgfqpoint{0.925000in}{3.838231in}}%
\pgfpathlineto{\pgfqpoint{1.202778in}{3.838231in}}%
\pgfusepath{stroke}%
\end{pgfscope}%
\begin{pgfscope}%
\pgfsetbuttcap%
\pgfsetmiterjoin%
\definecolor{currentfill}{rgb}{0.000000,0.000000,0.000000}%
\pgfsetfillcolor{currentfill}%
\pgfsetlinewidth{1.003750pt}%
\definecolor{currentstroke}{rgb}{0.000000,0.000000,0.000000}%
\pgfsetstrokecolor{currentstroke}%
\pgfsetdash{}{0pt}%
\pgfsys@defobject{currentmarker}{\pgfqpoint{-0.041667in}{-0.041667in}}{\pgfqpoint{0.041667in}{0.041667in}}{%
\pgfpathmoveto{\pgfqpoint{-0.000000in}{-0.041667in}}%
\pgfpathlineto{\pgfqpoint{0.041667in}{0.041667in}}%
\pgfpathlineto{\pgfqpoint{-0.041667in}{0.041667in}}%
\pgfpathclose%
\pgfusepath{stroke,fill}%
}%
\begin{pgfscope}%
\pgfsys@transformshift{1.063889in}{3.838231in}%
\pgfsys@useobject{currentmarker}{}%
\end{pgfscope}%
\end{pgfscope}%
\begin{pgfscope}%
\definecolor{textcolor}{rgb}{0.000000,0.000000,0.000000}%
\pgfsetstrokecolor{textcolor}%
\pgfsetfillcolor{textcolor}%
\pgftext[x=1.313889in,y=3.789620in,left,base]{\color{textcolor}\sffamily\fontsize{10.000000}{12.000000}\selectfont \(\displaystyle \beta = 0.5\)}%
\end{pgfscope}%
\begin{pgfscope}%
\pgfsetbuttcap%
\pgfsetroundjoin%
\pgfsetlinewidth{1.505625pt}%
\definecolor{currentstroke}{rgb}{0.000000,0.000000,0.000000}%
\pgfsetstrokecolor{currentstroke}%
\pgfsetdash{{5.550000pt}{2.400000pt}}{0.000000pt}%
\pgfpathmoveto{\pgfqpoint{0.925000in}{3.634374in}}%
\pgfpathlineto{\pgfqpoint{1.202778in}{3.634374in}}%
\pgfusepath{stroke}%
\end{pgfscope}%
\begin{pgfscope}%
\pgfsetbuttcap%
\pgfsetmiterjoin%
\definecolor{currentfill}{rgb}{0.000000,0.000000,0.000000}%
\pgfsetfillcolor{currentfill}%
\pgfsetlinewidth{1.003750pt}%
\definecolor{currentstroke}{rgb}{0.000000,0.000000,0.000000}%
\pgfsetstrokecolor{currentstroke}%
\pgfsetdash{}{0pt}%
\pgfsys@defobject{currentmarker}{\pgfqpoint{-0.041667in}{-0.041667in}}{\pgfqpoint{0.041667in}{0.041667in}}{%
\pgfpathmoveto{\pgfqpoint{-0.020833in}{-0.041667in}}%
\pgfpathlineto{\pgfqpoint{0.000000in}{-0.020833in}}%
\pgfpathlineto{\pgfqpoint{0.020833in}{-0.041667in}}%
\pgfpathlineto{\pgfqpoint{0.041667in}{-0.020833in}}%
\pgfpathlineto{\pgfqpoint{0.020833in}{0.000000in}}%
\pgfpathlineto{\pgfqpoint{0.041667in}{0.020833in}}%
\pgfpathlineto{\pgfqpoint{0.020833in}{0.041667in}}%
\pgfpathlineto{\pgfqpoint{0.000000in}{0.020833in}}%
\pgfpathlineto{\pgfqpoint{-0.020833in}{0.041667in}}%
\pgfpathlineto{\pgfqpoint{-0.041667in}{0.020833in}}%
\pgfpathlineto{\pgfqpoint{-0.020833in}{0.000000in}}%
\pgfpathlineto{\pgfqpoint{-0.041667in}{-0.020833in}}%
\pgfpathclose%
\pgfusepath{stroke,fill}%
}%
\begin{pgfscope}%
\pgfsys@transformshift{1.063889in}{3.634374in}%
\pgfsys@useobject{currentmarker}{}%
\end{pgfscope}%
\end{pgfscope}%
\begin{pgfscope}%
\definecolor{textcolor}{rgb}{0.000000,0.000000,0.000000}%
\pgfsetstrokecolor{textcolor}%
\pgfsetfillcolor{textcolor}%
\pgftext[x=1.313889in,y=3.585762in,left,base]{\color{textcolor}\sffamily\fontsize{10.000000}{12.000000}\selectfont \(\displaystyle \beta = 0.6\)}%
\end{pgfscope}%
\begin{pgfscope}%
\pgfsetbuttcap%
\pgfsetroundjoin%
\pgfsetlinewidth{1.505625pt}%
\definecolor{currentstroke}{rgb}{0.000000,0.000000,0.000000}%
\pgfsetstrokecolor{currentstroke}%
\pgfsetdash{{5.550000pt}{2.400000pt}}{0.000000pt}%
\pgfpathmoveto{\pgfqpoint{0.925000in}{3.430516in}}%
\pgfpathlineto{\pgfqpoint{1.202778in}{3.430516in}}%
\pgfusepath{stroke}%
\end{pgfscope}%
\begin{pgfscope}%
\pgfsetbuttcap%
\pgfsetmiterjoin%
\definecolor{currentfill}{rgb}{0.000000,0.000000,0.000000}%
\pgfsetfillcolor{currentfill}%
\pgfsetlinewidth{1.003750pt}%
\definecolor{currentstroke}{rgb}{0.000000,0.000000,0.000000}%
\pgfsetstrokecolor{currentstroke}%
\pgfsetdash{}{0pt}%
\pgfsys@defobject{currentmarker}{\pgfqpoint{-0.035355in}{-0.058926in}}{\pgfqpoint{0.035355in}{0.058926in}}{%
\pgfpathmoveto{\pgfqpoint{-0.000000in}{-0.058926in}}%
\pgfpathlineto{\pgfqpoint{0.035355in}{0.000000in}}%
\pgfpathlineto{\pgfqpoint{0.000000in}{0.058926in}}%
\pgfpathlineto{\pgfqpoint{-0.035355in}{0.000000in}}%
\pgfpathclose%
\pgfusepath{stroke,fill}%
}%
\begin{pgfscope}%
\pgfsys@transformshift{1.063889in}{3.430516in}%
\pgfsys@useobject{currentmarker}{}%
\end{pgfscope}%
\end{pgfscope}%
\begin{pgfscope}%
\definecolor{textcolor}{rgb}{0.000000,0.000000,0.000000}%
\pgfsetstrokecolor{textcolor}%
\pgfsetfillcolor{textcolor}%
\pgftext[x=1.313889in,y=3.381905in,left,base]{\color{textcolor}\sffamily\fontsize{10.000000}{12.000000}\selectfont \(\displaystyle \beta = 0.7\)}%
\end{pgfscope}%
\end{pgfpicture}%
\makeatother%
\endgroup%

    \caption[GMRES iteration counts when $\NLqDRR{\nso-\nst} = 0.2\times k^{-\beta},$ for any $1 \leq q < \infty$ and $\beta = 0.4,0.5,0.6,0.7$.]{GMRES iteration counts for $\AmatoI\Amatt$ given by \cref{eq:noweak,eq:ntweak}, where $\alpha = 0.2\times k^{-\beta},$ for $\beta = 0.4,0.5,0.6,0.7.$}\label{fig:l1med}
\end{figure}
    
    \begin{figure}
    %% Creator: Matplotlib, PGF backend
%%
%% To include the figure in your LaTeX document, write
%%   \input{<filename>.pgf}
%%
%% Make sure the required packages are loaded in your preamble
%%   \usepackage{pgf}
%%
%% Figures using additional raster images can only be included by \input if
%% they are in the same directory as the main LaTeX file. For loading figures
%% from other directories you can use the `import` package
%%   \usepackage{import}
%% and then include the figures with
%%   \import{<path to file>}{<filename>.pgf}
%%
%% Matplotlib used the following preamble
%%   \usepackage{fontspec}
%%   \setmainfont{DejaVuSerif.ttf}[Path=/home/owen/progs/firedrake-complex/firedrake/lib/python3.5/site-packages/matplotlib/mpl-data/fonts/ttf/]
%%   \setsansfont{DejaVuSans.ttf}[Path=/home/owen/progs/firedrake-complex/firedrake/lib/python3.5/site-packages/matplotlib/mpl-data/fonts/ttf/]
%%   \setmonofont{DejaVuSansMono.ttf}[Path=/home/owen/progs/firedrake-complex/firedrake/lib/python3.5/site-packages/matplotlib/mpl-data/fonts/ttf/]
%%
\begingroup%
\makeatletter%
\begin{pgfpicture}%
\pgfpathrectangle{\pgfpointorigin}{\pgfqpoint{6.400000in}{4.800000in}}%
\pgfusepath{use as bounding box, clip}%
\begin{pgfscope}%
\pgfsetbuttcap%
\pgfsetmiterjoin%
\definecolor{currentfill}{rgb}{1.000000,1.000000,1.000000}%
\pgfsetfillcolor{currentfill}%
\pgfsetlinewidth{0.000000pt}%
\definecolor{currentstroke}{rgb}{1.000000,1.000000,1.000000}%
\pgfsetstrokecolor{currentstroke}%
\pgfsetdash{}{0pt}%
\pgfpathmoveto{\pgfqpoint{0.000000in}{0.000000in}}%
\pgfpathlineto{\pgfqpoint{6.400000in}{0.000000in}}%
\pgfpathlineto{\pgfqpoint{6.400000in}{4.800000in}}%
\pgfpathlineto{\pgfqpoint{0.000000in}{4.800000in}}%
\pgfpathclose%
\pgfusepath{fill}%
\end{pgfscope}%
\begin{pgfscope}%
\pgfsetbuttcap%
\pgfsetmiterjoin%
\definecolor{currentfill}{rgb}{1.000000,1.000000,1.000000}%
\pgfsetfillcolor{currentfill}%
\pgfsetlinewidth{0.000000pt}%
\definecolor{currentstroke}{rgb}{0.000000,0.000000,0.000000}%
\pgfsetstrokecolor{currentstroke}%
\pgfsetstrokeopacity{0.000000}%
\pgfsetdash{}{0pt}%
\pgfpathmoveto{\pgfqpoint{0.800000in}{0.528000in}}%
\pgfpathlineto{\pgfqpoint{5.760000in}{0.528000in}}%
\pgfpathlineto{\pgfqpoint{5.760000in}{4.224000in}}%
\pgfpathlineto{\pgfqpoint{0.800000in}{4.224000in}}%
\pgfpathclose%
\pgfusepath{fill}%
\end{pgfscope}%
\begin{pgfscope}%
\pgfsetbuttcap%
\pgfsetroundjoin%
\definecolor{currentfill}{rgb}{0.000000,0.000000,0.000000}%
\pgfsetfillcolor{currentfill}%
\pgfsetlinewidth{0.803000pt}%
\definecolor{currentstroke}{rgb}{0.000000,0.000000,0.000000}%
\pgfsetstrokecolor{currentstroke}%
\pgfsetdash{}{0pt}%
\pgfsys@defobject{currentmarker}{\pgfqpoint{0.000000in}{-0.048611in}}{\pgfqpoint{0.000000in}{0.000000in}}{%
\pgfpathmoveto{\pgfqpoint{0.000000in}{0.000000in}}%
\pgfpathlineto{\pgfqpoint{0.000000in}{-0.048611in}}%
\pgfusepath{stroke,fill}%
}%
\begin{pgfscope}%
\pgfsys@transformshift{1.250909in}{0.528000in}%
\pgfsys@useobject{currentmarker}{}%
\end{pgfscope}%
\end{pgfscope}%
\begin{pgfscope}%
\definecolor{textcolor}{rgb}{0.000000,0.000000,0.000000}%
\pgfsetstrokecolor{textcolor}%
\pgfsetfillcolor{textcolor}%
\pgftext[x=1.250909in,y=0.430778in,,top]{\color{textcolor}\sffamily\fontsize{10.000000}{12.000000}\selectfont 10}%
\end{pgfscope}%
\begin{pgfscope}%
\pgfsetbuttcap%
\pgfsetroundjoin%
\definecolor{currentfill}{rgb}{0.000000,0.000000,0.000000}%
\pgfsetfillcolor{currentfill}%
\pgfsetlinewidth{0.803000pt}%
\definecolor{currentstroke}{rgb}{0.000000,0.000000,0.000000}%
\pgfsetstrokecolor{currentstroke}%
\pgfsetdash{}{0pt}%
\pgfsys@defobject{currentmarker}{\pgfqpoint{0.000000in}{-0.048611in}}{\pgfqpoint{0.000000in}{0.000000in}}{%
\pgfpathmoveto{\pgfqpoint{0.000000in}{0.000000in}}%
\pgfpathlineto{\pgfqpoint{0.000000in}{-0.048611in}}%
\pgfusepath{stroke,fill}%
}%
\begin{pgfscope}%
\pgfsys@transformshift{1.701818in}{0.528000in}%
\pgfsys@useobject{currentmarker}{}%
\end{pgfscope}%
\end{pgfscope}%
\begin{pgfscope}%
\definecolor{textcolor}{rgb}{0.000000,0.000000,0.000000}%
\pgfsetstrokecolor{textcolor}%
\pgfsetfillcolor{textcolor}%
\pgftext[x=1.701818in,y=0.430778in,,top]{\color{textcolor}\sffamily\fontsize{10.000000}{12.000000}\selectfont 20}%
\end{pgfscope}%
\begin{pgfscope}%
\pgfsetbuttcap%
\pgfsetroundjoin%
\definecolor{currentfill}{rgb}{0.000000,0.000000,0.000000}%
\pgfsetfillcolor{currentfill}%
\pgfsetlinewidth{0.803000pt}%
\definecolor{currentstroke}{rgb}{0.000000,0.000000,0.000000}%
\pgfsetstrokecolor{currentstroke}%
\pgfsetdash{}{0pt}%
\pgfsys@defobject{currentmarker}{\pgfqpoint{0.000000in}{-0.048611in}}{\pgfqpoint{0.000000in}{0.000000in}}{%
\pgfpathmoveto{\pgfqpoint{0.000000in}{0.000000in}}%
\pgfpathlineto{\pgfqpoint{0.000000in}{-0.048611in}}%
\pgfusepath{stroke,fill}%
}%
\begin{pgfscope}%
\pgfsys@transformshift{2.152727in}{0.528000in}%
\pgfsys@useobject{currentmarker}{}%
\end{pgfscope}%
\end{pgfscope}%
\begin{pgfscope}%
\definecolor{textcolor}{rgb}{0.000000,0.000000,0.000000}%
\pgfsetstrokecolor{textcolor}%
\pgfsetfillcolor{textcolor}%
\pgftext[x=2.152727in,y=0.430778in,,top]{\color{textcolor}\sffamily\fontsize{10.000000}{12.000000}\selectfont 30}%
\end{pgfscope}%
\begin{pgfscope}%
\pgfsetbuttcap%
\pgfsetroundjoin%
\definecolor{currentfill}{rgb}{0.000000,0.000000,0.000000}%
\pgfsetfillcolor{currentfill}%
\pgfsetlinewidth{0.803000pt}%
\definecolor{currentstroke}{rgb}{0.000000,0.000000,0.000000}%
\pgfsetstrokecolor{currentstroke}%
\pgfsetdash{}{0pt}%
\pgfsys@defobject{currentmarker}{\pgfqpoint{0.000000in}{-0.048611in}}{\pgfqpoint{0.000000in}{0.000000in}}{%
\pgfpathmoveto{\pgfqpoint{0.000000in}{0.000000in}}%
\pgfpathlineto{\pgfqpoint{0.000000in}{-0.048611in}}%
\pgfusepath{stroke,fill}%
}%
\begin{pgfscope}%
\pgfsys@transformshift{2.603636in}{0.528000in}%
\pgfsys@useobject{currentmarker}{}%
\end{pgfscope}%
\end{pgfscope}%
\begin{pgfscope}%
\definecolor{textcolor}{rgb}{0.000000,0.000000,0.000000}%
\pgfsetstrokecolor{textcolor}%
\pgfsetfillcolor{textcolor}%
\pgftext[x=2.603636in,y=0.430778in,,top]{\color{textcolor}\sffamily\fontsize{10.000000}{12.000000}\selectfont 40}%
\end{pgfscope}%
\begin{pgfscope}%
\pgfsetbuttcap%
\pgfsetroundjoin%
\definecolor{currentfill}{rgb}{0.000000,0.000000,0.000000}%
\pgfsetfillcolor{currentfill}%
\pgfsetlinewidth{0.803000pt}%
\definecolor{currentstroke}{rgb}{0.000000,0.000000,0.000000}%
\pgfsetstrokecolor{currentstroke}%
\pgfsetdash{}{0pt}%
\pgfsys@defobject{currentmarker}{\pgfqpoint{0.000000in}{-0.048611in}}{\pgfqpoint{0.000000in}{0.000000in}}{%
\pgfpathmoveto{\pgfqpoint{0.000000in}{0.000000in}}%
\pgfpathlineto{\pgfqpoint{0.000000in}{-0.048611in}}%
\pgfusepath{stroke,fill}%
}%
\begin{pgfscope}%
\pgfsys@transformshift{3.054545in}{0.528000in}%
\pgfsys@useobject{currentmarker}{}%
\end{pgfscope}%
\end{pgfscope}%
\begin{pgfscope}%
\definecolor{textcolor}{rgb}{0.000000,0.000000,0.000000}%
\pgfsetstrokecolor{textcolor}%
\pgfsetfillcolor{textcolor}%
\pgftext[x=3.054545in,y=0.430778in,,top]{\color{textcolor}\sffamily\fontsize{10.000000}{12.000000}\selectfont 50}%
\end{pgfscope}%
\begin{pgfscope}%
\pgfsetbuttcap%
\pgfsetroundjoin%
\definecolor{currentfill}{rgb}{0.000000,0.000000,0.000000}%
\pgfsetfillcolor{currentfill}%
\pgfsetlinewidth{0.803000pt}%
\definecolor{currentstroke}{rgb}{0.000000,0.000000,0.000000}%
\pgfsetstrokecolor{currentstroke}%
\pgfsetdash{}{0pt}%
\pgfsys@defobject{currentmarker}{\pgfqpoint{0.000000in}{-0.048611in}}{\pgfqpoint{0.000000in}{0.000000in}}{%
\pgfpathmoveto{\pgfqpoint{0.000000in}{0.000000in}}%
\pgfpathlineto{\pgfqpoint{0.000000in}{-0.048611in}}%
\pgfusepath{stroke,fill}%
}%
\begin{pgfscope}%
\pgfsys@transformshift{3.505455in}{0.528000in}%
\pgfsys@useobject{currentmarker}{}%
\end{pgfscope}%
\end{pgfscope}%
\begin{pgfscope}%
\definecolor{textcolor}{rgb}{0.000000,0.000000,0.000000}%
\pgfsetstrokecolor{textcolor}%
\pgfsetfillcolor{textcolor}%
\pgftext[x=3.505455in,y=0.430778in,,top]{\color{textcolor}\sffamily\fontsize{10.000000}{12.000000}\selectfont 60}%
\end{pgfscope}%
\begin{pgfscope}%
\pgfsetbuttcap%
\pgfsetroundjoin%
\definecolor{currentfill}{rgb}{0.000000,0.000000,0.000000}%
\pgfsetfillcolor{currentfill}%
\pgfsetlinewidth{0.803000pt}%
\definecolor{currentstroke}{rgb}{0.000000,0.000000,0.000000}%
\pgfsetstrokecolor{currentstroke}%
\pgfsetdash{}{0pt}%
\pgfsys@defobject{currentmarker}{\pgfqpoint{0.000000in}{-0.048611in}}{\pgfqpoint{0.000000in}{0.000000in}}{%
\pgfpathmoveto{\pgfqpoint{0.000000in}{0.000000in}}%
\pgfpathlineto{\pgfqpoint{0.000000in}{-0.048611in}}%
\pgfusepath{stroke,fill}%
}%
\begin{pgfscope}%
\pgfsys@transformshift{3.956364in}{0.528000in}%
\pgfsys@useobject{currentmarker}{}%
\end{pgfscope}%
\end{pgfscope}%
\begin{pgfscope}%
\definecolor{textcolor}{rgb}{0.000000,0.000000,0.000000}%
\pgfsetstrokecolor{textcolor}%
\pgfsetfillcolor{textcolor}%
\pgftext[x=3.956364in,y=0.430778in,,top]{\color{textcolor}\sffamily\fontsize{10.000000}{12.000000}\selectfont 70}%
\end{pgfscope}%
\begin{pgfscope}%
\pgfsetbuttcap%
\pgfsetroundjoin%
\definecolor{currentfill}{rgb}{0.000000,0.000000,0.000000}%
\pgfsetfillcolor{currentfill}%
\pgfsetlinewidth{0.803000pt}%
\definecolor{currentstroke}{rgb}{0.000000,0.000000,0.000000}%
\pgfsetstrokecolor{currentstroke}%
\pgfsetdash{}{0pt}%
\pgfsys@defobject{currentmarker}{\pgfqpoint{0.000000in}{-0.048611in}}{\pgfqpoint{0.000000in}{0.000000in}}{%
\pgfpathmoveto{\pgfqpoint{0.000000in}{0.000000in}}%
\pgfpathlineto{\pgfqpoint{0.000000in}{-0.048611in}}%
\pgfusepath{stroke,fill}%
}%
\begin{pgfscope}%
\pgfsys@transformshift{4.407273in}{0.528000in}%
\pgfsys@useobject{currentmarker}{}%
\end{pgfscope}%
\end{pgfscope}%
\begin{pgfscope}%
\definecolor{textcolor}{rgb}{0.000000,0.000000,0.000000}%
\pgfsetstrokecolor{textcolor}%
\pgfsetfillcolor{textcolor}%
\pgftext[x=4.407273in,y=0.430778in,,top]{\color{textcolor}\sffamily\fontsize{10.000000}{12.000000}\selectfont 80}%
\end{pgfscope}%
\begin{pgfscope}%
\pgfsetbuttcap%
\pgfsetroundjoin%
\definecolor{currentfill}{rgb}{0.000000,0.000000,0.000000}%
\pgfsetfillcolor{currentfill}%
\pgfsetlinewidth{0.803000pt}%
\definecolor{currentstroke}{rgb}{0.000000,0.000000,0.000000}%
\pgfsetstrokecolor{currentstroke}%
\pgfsetdash{}{0pt}%
\pgfsys@defobject{currentmarker}{\pgfqpoint{0.000000in}{-0.048611in}}{\pgfqpoint{0.000000in}{0.000000in}}{%
\pgfpathmoveto{\pgfqpoint{0.000000in}{0.000000in}}%
\pgfpathlineto{\pgfqpoint{0.000000in}{-0.048611in}}%
\pgfusepath{stroke,fill}%
}%
\begin{pgfscope}%
\pgfsys@transformshift{4.858182in}{0.528000in}%
\pgfsys@useobject{currentmarker}{}%
\end{pgfscope}%
\end{pgfscope}%
\begin{pgfscope}%
\definecolor{textcolor}{rgb}{0.000000,0.000000,0.000000}%
\pgfsetstrokecolor{textcolor}%
\pgfsetfillcolor{textcolor}%
\pgftext[x=4.858182in,y=0.430778in,,top]{\color{textcolor}\sffamily\fontsize{10.000000}{12.000000}\selectfont 90}%
\end{pgfscope}%
\begin{pgfscope}%
\pgfsetbuttcap%
\pgfsetroundjoin%
\definecolor{currentfill}{rgb}{0.000000,0.000000,0.000000}%
\pgfsetfillcolor{currentfill}%
\pgfsetlinewidth{0.803000pt}%
\definecolor{currentstroke}{rgb}{0.000000,0.000000,0.000000}%
\pgfsetstrokecolor{currentstroke}%
\pgfsetdash{}{0pt}%
\pgfsys@defobject{currentmarker}{\pgfqpoint{0.000000in}{-0.048611in}}{\pgfqpoint{0.000000in}{0.000000in}}{%
\pgfpathmoveto{\pgfqpoint{0.000000in}{0.000000in}}%
\pgfpathlineto{\pgfqpoint{0.000000in}{-0.048611in}}%
\pgfusepath{stroke,fill}%
}%
\begin{pgfscope}%
\pgfsys@transformshift{5.309091in}{0.528000in}%
\pgfsys@useobject{currentmarker}{}%
\end{pgfscope}%
\end{pgfscope}%
\begin{pgfscope}%
\definecolor{textcolor}{rgb}{0.000000,0.000000,0.000000}%
\pgfsetstrokecolor{textcolor}%
\pgfsetfillcolor{textcolor}%
\pgftext[x=5.309091in,y=0.430778in,,top]{\color{textcolor}\sffamily\fontsize{10.000000}{12.000000}\selectfont 100}%
\end{pgfscope}%
\begin{pgfscope}%
\definecolor{textcolor}{rgb}{0.000000,0.000000,0.000000}%
\pgfsetstrokecolor{textcolor}%
\pgfsetfillcolor{textcolor}%
\pgftext[x=3.280000in,y=0.240809in,,top]{\color{textcolor}\sffamily\fontsize{10.000000}{12.000000}\selectfont \(\displaystyle k\)}%
\end{pgfscope}%
\begin{pgfscope}%
\pgfsetbuttcap%
\pgfsetroundjoin%
\definecolor{currentfill}{rgb}{0.000000,0.000000,0.000000}%
\pgfsetfillcolor{currentfill}%
\pgfsetlinewidth{0.803000pt}%
\definecolor{currentstroke}{rgb}{0.000000,0.000000,0.000000}%
\pgfsetstrokecolor{currentstroke}%
\pgfsetdash{}{0pt}%
\pgfsys@defobject{currentmarker}{\pgfqpoint{-0.048611in}{0.000000in}}{\pgfqpoint{0.000000in}{0.000000in}}{%
\pgfpathmoveto{\pgfqpoint{0.000000in}{0.000000in}}%
\pgfpathlineto{\pgfqpoint{-0.048611in}{0.000000in}}%
\pgfusepath{stroke,fill}%
}%
\begin{pgfscope}%
\pgfsys@transformshift{0.800000in}{0.696000in}%
\pgfsys@useobject{currentmarker}{}%
\end{pgfscope}%
\end{pgfscope}%
\begin{pgfscope}%
\definecolor{textcolor}{rgb}{0.000000,0.000000,0.000000}%
\pgfsetstrokecolor{textcolor}%
\pgfsetfillcolor{textcolor}%
\pgftext[x=0.614413in,y=0.643238in,left,base]{\color{textcolor}\sffamily\fontsize{10.000000}{12.000000}\selectfont 6}%
\end{pgfscope}%
\begin{pgfscope}%
\pgfsetbuttcap%
\pgfsetroundjoin%
\definecolor{currentfill}{rgb}{0.000000,0.000000,0.000000}%
\pgfsetfillcolor{currentfill}%
\pgfsetlinewidth{0.803000pt}%
\definecolor{currentstroke}{rgb}{0.000000,0.000000,0.000000}%
\pgfsetstrokecolor{currentstroke}%
\pgfsetdash{}{0pt}%
\pgfsys@defobject{currentmarker}{\pgfqpoint{-0.048611in}{0.000000in}}{\pgfqpoint{0.000000in}{0.000000in}}{%
\pgfpathmoveto{\pgfqpoint{0.000000in}{0.000000in}}%
\pgfpathlineto{\pgfqpoint{-0.048611in}{0.000000in}}%
\pgfusepath{stroke,fill}%
}%
\begin{pgfscope}%
\pgfsys@transformshift{0.800000in}{1.536000in}%
\pgfsys@useobject{currentmarker}{}%
\end{pgfscope}%
\end{pgfscope}%
\begin{pgfscope}%
\definecolor{textcolor}{rgb}{0.000000,0.000000,0.000000}%
\pgfsetstrokecolor{textcolor}%
\pgfsetfillcolor{textcolor}%
\pgftext[x=0.614413in,y=1.483238in,left,base]{\color{textcolor}\sffamily\fontsize{10.000000}{12.000000}\selectfont 7}%
\end{pgfscope}%
\begin{pgfscope}%
\pgfsetbuttcap%
\pgfsetroundjoin%
\definecolor{currentfill}{rgb}{0.000000,0.000000,0.000000}%
\pgfsetfillcolor{currentfill}%
\pgfsetlinewidth{0.803000pt}%
\definecolor{currentstroke}{rgb}{0.000000,0.000000,0.000000}%
\pgfsetstrokecolor{currentstroke}%
\pgfsetdash{}{0pt}%
\pgfsys@defobject{currentmarker}{\pgfqpoint{-0.048611in}{0.000000in}}{\pgfqpoint{0.000000in}{0.000000in}}{%
\pgfpathmoveto{\pgfqpoint{0.000000in}{0.000000in}}%
\pgfpathlineto{\pgfqpoint{-0.048611in}{0.000000in}}%
\pgfusepath{stroke,fill}%
}%
\begin{pgfscope}%
\pgfsys@transformshift{0.800000in}{2.376000in}%
\pgfsys@useobject{currentmarker}{}%
\end{pgfscope}%
\end{pgfscope}%
\begin{pgfscope}%
\definecolor{textcolor}{rgb}{0.000000,0.000000,0.000000}%
\pgfsetstrokecolor{textcolor}%
\pgfsetfillcolor{textcolor}%
\pgftext[x=0.614413in,y=2.323238in,left,base]{\color{textcolor}\sffamily\fontsize{10.000000}{12.000000}\selectfont 8}%
\end{pgfscope}%
\begin{pgfscope}%
\pgfsetbuttcap%
\pgfsetroundjoin%
\definecolor{currentfill}{rgb}{0.000000,0.000000,0.000000}%
\pgfsetfillcolor{currentfill}%
\pgfsetlinewidth{0.803000pt}%
\definecolor{currentstroke}{rgb}{0.000000,0.000000,0.000000}%
\pgfsetstrokecolor{currentstroke}%
\pgfsetdash{}{0pt}%
\pgfsys@defobject{currentmarker}{\pgfqpoint{-0.048611in}{0.000000in}}{\pgfqpoint{0.000000in}{0.000000in}}{%
\pgfpathmoveto{\pgfqpoint{0.000000in}{0.000000in}}%
\pgfpathlineto{\pgfqpoint{-0.048611in}{0.000000in}}%
\pgfusepath{stroke,fill}%
}%
\begin{pgfscope}%
\pgfsys@transformshift{0.800000in}{3.216000in}%
\pgfsys@useobject{currentmarker}{}%
\end{pgfscope}%
\end{pgfscope}%
\begin{pgfscope}%
\definecolor{textcolor}{rgb}{0.000000,0.000000,0.000000}%
\pgfsetstrokecolor{textcolor}%
\pgfsetfillcolor{textcolor}%
\pgftext[x=0.614413in,y=3.163238in,left,base]{\color{textcolor}\sffamily\fontsize{10.000000}{12.000000}\selectfont 9}%
\end{pgfscope}%
\begin{pgfscope}%
\pgfsetbuttcap%
\pgfsetroundjoin%
\definecolor{currentfill}{rgb}{0.000000,0.000000,0.000000}%
\pgfsetfillcolor{currentfill}%
\pgfsetlinewidth{0.803000pt}%
\definecolor{currentstroke}{rgb}{0.000000,0.000000,0.000000}%
\pgfsetstrokecolor{currentstroke}%
\pgfsetdash{}{0pt}%
\pgfsys@defobject{currentmarker}{\pgfqpoint{-0.048611in}{0.000000in}}{\pgfqpoint{0.000000in}{0.000000in}}{%
\pgfpathmoveto{\pgfqpoint{0.000000in}{0.000000in}}%
\pgfpathlineto{\pgfqpoint{-0.048611in}{0.000000in}}%
\pgfusepath{stroke,fill}%
}%
\begin{pgfscope}%
\pgfsys@transformshift{0.800000in}{4.056000in}%
\pgfsys@useobject{currentmarker}{}%
\end{pgfscope}%
\end{pgfscope}%
\begin{pgfscope}%
\definecolor{textcolor}{rgb}{0.000000,0.000000,0.000000}%
\pgfsetstrokecolor{textcolor}%
\pgfsetfillcolor{textcolor}%
\pgftext[x=0.526047in,y=4.003238in,left,base]{\color{textcolor}\sffamily\fontsize{10.000000}{12.000000}\selectfont 10}%
\end{pgfscope}%
\begin{pgfscope}%
\definecolor{textcolor}{rgb}{0.000000,0.000000,0.000000}%
\pgfsetstrokecolor{textcolor}%
\pgfsetfillcolor{textcolor}%
\pgftext[x=0.470492in,y=2.376000in,,bottom,rotate=90.000000]{\color{textcolor}\sffamily\fontsize{10.000000}{12.000000}\selectfont Number of GMRES iterations}%
\end{pgfscope}%
\begin{pgfscope}%
\pgfpathrectangle{\pgfqpoint{0.800000in}{0.528000in}}{\pgfqpoint{4.960000in}{3.696000in}}%
\pgfusepath{clip}%
\pgfsetbuttcap%
\pgfsetroundjoin%
\pgfsetlinewidth{1.505625pt}%
\definecolor{currentstroke}{rgb}{0.843137,0.000000,0.000000}%
\pgfsetstrokecolor{currentstroke}%
\pgfsetdash{{5.550000pt}{2.400000pt}}{0.000000pt}%
\pgfpathmoveto{\pgfqpoint{1.250909in}{2.376000in}}%
\pgfpathlineto{\pgfqpoint{1.701818in}{2.376000in}}%
\pgfpathlineto{\pgfqpoint{2.152727in}{3.216000in}}%
\pgfpathlineto{\pgfqpoint{2.603636in}{3.216000in}}%
\pgfpathlineto{\pgfqpoint{3.054545in}{4.056000in}}%
\pgfpathlineto{\pgfqpoint{3.505455in}{4.056000in}}%
\pgfpathlineto{\pgfqpoint{3.956364in}{4.056000in}}%
\pgfpathlineto{\pgfqpoint{4.407273in}{4.056000in}}%
\pgfpathlineto{\pgfqpoint{4.858182in}{4.056000in}}%
\pgfpathlineto{\pgfqpoint{5.309091in}{4.056000in}}%
\pgfusepath{stroke}%
\end{pgfscope}%
\begin{pgfscope}%
\pgfpathrectangle{\pgfqpoint{0.800000in}{0.528000in}}{\pgfqpoint{4.960000in}{3.696000in}}%
\pgfusepath{clip}%
\pgfsetbuttcap%
\pgfsetroundjoin%
\definecolor{currentfill}{rgb}{0.843137,0.000000,0.000000}%
\pgfsetfillcolor{currentfill}%
\pgfsetlinewidth{1.003750pt}%
\definecolor{currentstroke}{rgb}{0.843137,0.000000,0.000000}%
\pgfsetstrokecolor{currentstroke}%
\pgfsetdash{}{0pt}%
\pgfsys@defobject{currentmarker}{\pgfqpoint{-0.041667in}{-0.041667in}}{\pgfqpoint{0.041667in}{0.041667in}}{%
\pgfpathmoveto{\pgfqpoint{0.000000in}{-0.041667in}}%
\pgfpathcurveto{\pgfqpoint{0.011050in}{-0.041667in}}{\pgfqpoint{0.021649in}{-0.037276in}}{\pgfqpoint{0.029463in}{-0.029463in}}%
\pgfpathcurveto{\pgfqpoint{0.037276in}{-0.021649in}}{\pgfqpoint{0.041667in}{-0.011050in}}{\pgfqpoint{0.041667in}{0.000000in}}%
\pgfpathcurveto{\pgfqpoint{0.041667in}{0.011050in}}{\pgfqpoint{0.037276in}{0.021649in}}{\pgfqpoint{0.029463in}{0.029463in}}%
\pgfpathcurveto{\pgfqpoint{0.021649in}{0.037276in}}{\pgfqpoint{0.011050in}{0.041667in}}{\pgfqpoint{0.000000in}{0.041667in}}%
\pgfpathcurveto{\pgfqpoint{-0.011050in}{0.041667in}}{\pgfqpoint{-0.021649in}{0.037276in}}{\pgfqpoint{-0.029463in}{0.029463in}}%
\pgfpathcurveto{\pgfqpoint{-0.037276in}{0.021649in}}{\pgfqpoint{-0.041667in}{0.011050in}}{\pgfqpoint{-0.041667in}{0.000000in}}%
\pgfpathcurveto{\pgfqpoint{-0.041667in}{-0.011050in}}{\pgfqpoint{-0.037276in}{-0.021649in}}{\pgfqpoint{-0.029463in}{-0.029463in}}%
\pgfpathcurveto{\pgfqpoint{-0.021649in}{-0.037276in}}{\pgfqpoint{-0.011050in}{-0.041667in}}{\pgfqpoint{0.000000in}{-0.041667in}}%
\pgfpathclose%
\pgfusepath{stroke,fill}%
}%
\begin{pgfscope}%
\pgfsys@transformshift{1.250909in}{2.376000in}%
\pgfsys@useobject{currentmarker}{}%
\end{pgfscope}%
\begin{pgfscope}%
\pgfsys@transformshift{1.701818in}{2.376000in}%
\pgfsys@useobject{currentmarker}{}%
\end{pgfscope}%
\begin{pgfscope}%
\pgfsys@transformshift{2.152727in}{3.216000in}%
\pgfsys@useobject{currentmarker}{}%
\end{pgfscope}%
\begin{pgfscope}%
\pgfsys@transformshift{2.603636in}{3.216000in}%
\pgfsys@useobject{currentmarker}{}%
\end{pgfscope}%
\begin{pgfscope}%
\pgfsys@transformshift{3.054545in}{4.056000in}%
\pgfsys@useobject{currentmarker}{}%
\end{pgfscope}%
\begin{pgfscope}%
\pgfsys@transformshift{3.505455in}{4.056000in}%
\pgfsys@useobject{currentmarker}{}%
\end{pgfscope}%
\begin{pgfscope}%
\pgfsys@transformshift{3.956364in}{4.056000in}%
\pgfsys@useobject{currentmarker}{}%
\end{pgfscope}%
\begin{pgfscope}%
\pgfsys@transformshift{4.407273in}{4.056000in}%
\pgfsys@useobject{currentmarker}{}%
\end{pgfscope}%
\begin{pgfscope}%
\pgfsys@transformshift{4.858182in}{4.056000in}%
\pgfsys@useobject{currentmarker}{}%
\end{pgfscope}%
\begin{pgfscope}%
\pgfsys@transformshift{5.309091in}{4.056000in}%
\pgfsys@useobject{currentmarker}{}%
\end{pgfscope}%
\end{pgfscope}%
\begin{pgfscope}%
\pgfpathrectangle{\pgfqpoint{0.800000in}{0.528000in}}{\pgfqpoint{4.960000in}{3.696000in}}%
\pgfusepath{clip}%
\pgfsetbuttcap%
\pgfsetroundjoin%
\pgfsetlinewidth{1.505625pt}%
\definecolor{currentstroke}{rgb}{0.549020,0.235294,1.000000}%
\pgfsetstrokecolor{currentstroke}%
\pgfsetdash{{5.550000pt}{2.400000pt}}{0.000000pt}%
\pgfpathmoveto{\pgfqpoint{1.250909in}{1.536000in}}%
\pgfpathlineto{\pgfqpoint{1.701818in}{1.536000in}}%
\pgfpathlineto{\pgfqpoint{2.152727in}{2.376000in}}%
\pgfpathlineto{\pgfqpoint{2.603636in}{2.376000in}}%
\pgfpathlineto{\pgfqpoint{3.054545in}{2.376000in}}%
\pgfpathlineto{\pgfqpoint{3.505455in}{2.376000in}}%
\pgfpathlineto{\pgfqpoint{3.956364in}{2.376000in}}%
\pgfpathlineto{\pgfqpoint{4.407273in}{2.376000in}}%
\pgfpathlineto{\pgfqpoint{4.858182in}{2.376000in}}%
\pgfpathlineto{\pgfqpoint{5.309091in}{2.376000in}}%
\pgfusepath{stroke}%
\end{pgfscope}%
\begin{pgfscope}%
\pgfpathrectangle{\pgfqpoint{0.800000in}{0.528000in}}{\pgfqpoint{4.960000in}{3.696000in}}%
\pgfusepath{clip}%
\pgfsetbuttcap%
\pgfsetmiterjoin%
\definecolor{currentfill}{rgb}{0.549020,0.235294,1.000000}%
\pgfsetfillcolor{currentfill}%
\pgfsetlinewidth{1.003750pt}%
\definecolor{currentstroke}{rgb}{0.549020,0.235294,1.000000}%
\pgfsetstrokecolor{currentstroke}%
\pgfsetdash{}{0pt}%
\pgfsys@defobject{currentmarker}{\pgfqpoint{-0.041667in}{-0.041667in}}{\pgfqpoint{0.041667in}{0.041667in}}{%
\pgfpathmoveto{\pgfqpoint{0.000000in}{0.041667in}}%
\pgfpathlineto{\pgfqpoint{-0.041667in}{-0.041667in}}%
\pgfpathlineto{\pgfqpoint{0.041667in}{-0.041667in}}%
\pgfpathclose%
\pgfusepath{stroke,fill}%
}%
\begin{pgfscope}%
\pgfsys@transformshift{1.250909in}{1.536000in}%
\pgfsys@useobject{currentmarker}{}%
\end{pgfscope}%
\begin{pgfscope}%
\pgfsys@transformshift{1.701818in}{1.536000in}%
\pgfsys@useobject{currentmarker}{}%
\end{pgfscope}%
\begin{pgfscope}%
\pgfsys@transformshift{2.152727in}{2.376000in}%
\pgfsys@useobject{currentmarker}{}%
\end{pgfscope}%
\begin{pgfscope}%
\pgfsys@transformshift{2.603636in}{2.376000in}%
\pgfsys@useobject{currentmarker}{}%
\end{pgfscope}%
\begin{pgfscope}%
\pgfsys@transformshift{3.054545in}{2.376000in}%
\pgfsys@useobject{currentmarker}{}%
\end{pgfscope}%
\begin{pgfscope}%
\pgfsys@transformshift{3.505455in}{2.376000in}%
\pgfsys@useobject{currentmarker}{}%
\end{pgfscope}%
\begin{pgfscope}%
\pgfsys@transformshift{3.956364in}{2.376000in}%
\pgfsys@useobject{currentmarker}{}%
\end{pgfscope}%
\begin{pgfscope}%
\pgfsys@transformshift{4.407273in}{2.376000in}%
\pgfsys@useobject{currentmarker}{}%
\end{pgfscope}%
\begin{pgfscope}%
\pgfsys@transformshift{4.858182in}{2.376000in}%
\pgfsys@useobject{currentmarker}{}%
\end{pgfscope}%
\begin{pgfscope}%
\pgfsys@transformshift{5.309091in}{2.376000in}%
\pgfsys@useobject{currentmarker}{}%
\end{pgfscope}%
\end{pgfscope}%
\begin{pgfscope}%
\pgfpathrectangle{\pgfqpoint{0.800000in}{0.528000in}}{\pgfqpoint{4.960000in}{3.696000in}}%
\pgfusepath{clip}%
\pgfsetbuttcap%
\pgfsetroundjoin%
\pgfsetlinewidth{1.505625pt}%
\definecolor{currentstroke}{rgb}{0.007843,0.533333,0.000000}%
\pgfsetstrokecolor{currentstroke}%
\pgfsetdash{{5.550000pt}{2.400000pt}}{0.000000pt}%
\pgfpathmoveto{\pgfqpoint{1.250909in}{1.536000in}}%
\pgfpathlineto{\pgfqpoint{1.701818in}{0.696000in}}%
\pgfpathlineto{\pgfqpoint{2.152727in}{1.536000in}}%
\pgfpathlineto{\pgfqpoint{2.603636in}{1.536000in}}%
\pgfpathlineto{\pgfqpoint{3.054545in}{1.536000in}}%
\pgfpathlineto{\pgfqpoint{3.505455in}{1.536000in}}%
\pgfpathlineto{\pgfqpoint{3.956364in}{1.536000in}}%
\pgfpathlineto{\pgfqpoint{4.407273in}{1.536000in}}%
\pgfpathlineto{\pgfqpoint{4.858182in}{1.536000in}}%
\pgfpathlineto{\pgfqpoint{5.309091in}{1.536000in}}%
\pgfusepath{stroke}%
\end{pgfscope}%
\begin{pgfscope}%
\pgfpathrectangle{\pgfqpoint{0.800000in}{0.528000in}}{\pgfqpoint{4.960000in}{3.696000in}}%
\pgfusepath{clip}%
\pgfsetbuttcap%
\pgfsetmiterjoin%
\definecolor{currentfill}{rgb}{0.007843,0.533333,0.000000}%
\pgfsetfillcolor{currentfill}%
\pgfsetlinewidth{1.003750pt}%
\definecolor{currentstroke}{rgb}{0.007843,0.533333,0.000000}%
\pgfsetstrokecolor{currentstroke}%
\pgfsetdash{}{0pt}%
\pgfsys@defobject{currentmarker}{\pgfqpoint{-0.041667in}{-0.041667in}}{\pgfqpoint{0.041667in}{0.041667in}}{%
\pgfpathmoveto{\pgfqpoint{-0.000000in}{-0.041667in}}%
\pgfpathlineto{\pgfqpoint{0.041667in}{0.041667in}}%
\pgfpathlineto{\pgfqpoint{-0.041667in}{0.041667in}}%
\pgfpathclose%
\pgfusepath{stroke,fill}%
}%
\begin{pgfscope}%
\pgfsys@transformshift{1.250909in}{1.536000in}%
\pgfsys@useobject{currentmarker}{}%
\end{pgfscope}%
\begin{pgfscope}%
\pgfsys@transformshift{1.701818in}{0.696000in}%
\pgfsys@useobject{currentmarker}{}%
\end{pgfscope}%
\begin{pgfscope}%
\pgfsys@transformshift{2.152727in}{1.536000in}%
\pgfsys@useobject{currentmarker}{}%
\end{pgfscope}%
\begin{pgfscope}%
\pgfsys@transformshift{2.603636in}{1.536000in}%
\pgfsys@useobject{currentmarker}{}%
\end{pgfscope}%
\begin{pgfscope}%
\pgfsys@transformshift{3.054545in}{1.536000in}%
\pgfsys@useobject{currentmarker}{}%
\end{pgfscope}%
\begin{pgfscope}%
\pgfsys@transformshift{3.505455in}{1.536000in}%
\pgfsys@useobject{currentmarker}{}%
\end{pgfscope}%
\begin{pgfscope}%
\pgfsys@transformshift{3.956364in}{1.536000in}%
\pgfsys@useobject{currentmarker}{}%
\end{pgfscope}%
\begin{pgfscope}%
\pgfsys@transformshift{4.407273in}{1.536000in}%
\pgfsys@useobject{currentmarker}{}%
\end{pgfscope}%
\begin{pgfscope}%
\pgfsys@transformshift{4.858182in}{1.536000in}%
\pgfsys@useobject{currentmarker}{}%
\end{pgfscope}%
\begin{pgfscope}%
\pgfsys@transformshift{5.309091in}{1.536000in}%
\pgfsys@useobject{currentmarker}{}%
\end{pgfscope}%
\end{pgfscope}%
\begin{pgfscope}%
\pgfsetrectcap%
\pgfsetmiterjoin%
\pgfsetlinewidth{0.803000pt}%
\definecolor{currentstroke}{rgb}{0.000000,0.000000,0.000000}%
\pgfsetstrokecolor{currentstroke}%
\pgfsetdash{}{0pt}%
\pgfpathmoveto{\pgfqpoint{0.800000in}{0.528000in}}%
\pgfpathlineto{\pgfqpoint{0.800000in}{4.224000in}}%
\pgfusepath{stroke}%
\end{pgfscope}%
\begin{pgfscope}%
\pgfsetrectcap%
\pgfsetmiterjoin%
\pgfsetlinewidth{0.803000pt}%
\definecolor{currentstroke}{rgb}{0.000000,0.000000,0.000000}%
\pgfsetstrokecolor{currentstroke}%
\pgfsetdash{}{0pt}%
\pgfpathmoveto{\pgfqpoint{5.760000in}{0.528000in}}%
\pgfpathlineto{\pgfqpoint{5.760000in}{4.224000in}}%
\pgfusepath{stroke}%
\end{pgfscope}%
\begin{pgfscope}%
\pgfsetrectcap%
\pgfsetmiterjoin%
\pgfsetlinewidth{0.803000pt}%
\definecolor{currentstroke}{rgb}{0.000000,0.000000,0.000000}%
\pgfsetstrokecolor{currentstroke}%
\pgfsetdash{}{0pt}%
\pgfpathmoveto{\pgfqpoint{0.800000in}{0.528000in}}%
\pgfpathlineto{\pgfqpoint{5.760000in}{0.528000in}}%
\pgfusepath{stroke}%
\end{pgfscope}%
\begin{pgfscope}%
\pgfsetrectcap%
\pgfsetmiterjoin%
\pgfsetlinewidth{0.803000pt}%
\definecolor{currentstroke}{rgb}{0.000000,0.000000,0.000000}%
\pgfsetstrokecolor{currentstroke}%
\pgfsetdash{}{0pt}%
\pgfpathmoveto{\pgfqpoint{0.800000in}{4.224000in}}%
\pgfpathlineto{\pgfqpoint{5.760000in}{4.224000in}}%
\pgfusepath{stroke}%
\end{pgfscope}%
\begin{pgfscope}%
\pgfsetbuttcap%
\pgfsetmiterjoin%
\definecolor{currentfill}{rgb}{1.000000,1.000000,1.000000}%
\pgfsetfillcolor{currentfill}%
\pgfsetfillopacity{0.800000}%
\pgfsetlinewidth{1.003750pt}%
\definecolor{currentstroke}{rgb}{0.800000,0.800000,0.800000}%
\pgfsetstrokecolor{currentstroke}%
\pgfsetstrokeopacity{0.800000}%
\pgfsetdash{}{0pt}%
\pgfpathmoveto{\pgfqpoint{0.897222in}{3.501317in}}%
\pgfpathlineto{\pgfqpoint{1.790209in}{3.501317in}}%
\pgfpathquadraticcurveto{\pgfqpoint{1.817987in}{3.501317in}}{\pgfqpoint{1.817987in}{3.529095in}}%
\pgfpathlineto{\pgfqpoint{1.817987in}{4.126778in}}%
\pgfpathquadraticcurveto{\pgfqpoint{1.817987in}{4.154556in}}{\pgfqpoint{1.790209in}{4.154556in}}%
\pgfpathlineto{\pgfqpoint{0.897222in}{4.154556in}}%
\pgfpathquadraticcurveto{\pgfqpoint{0.869444in}{4.154556in}}{\pgfqpoint{0.869444in}{4.126778in}}%
\pgfpathlineto{\pgfqpoint{0.869444in}{3.529095in}}%
\pgfpathquadraticcurveto{\pgfqpoint{0.869444in}{3.501317in}}{\pgfqpoint{0.897222in}{3.501317in}}%
\pgfpathclose%
\pgfusepath{stroke,fill}%
\end{pgfscope}%
\begin{pgfscope}%
\pgfsetbuttcap%
\pgfsetroundjoin%
\pgfsetlinewidth{1.505625pt}%
\definecolor{currentstroke}{rgb}{0.843137,0.000000,0.000000}%
\pgfsetstrokecolor{currentstroke}%
\pgfsetdash{{5.550000pt}{2.400000pt}}{0.000000pt}%
\pgfpathmoveto{\pgfqpoint{0.925000in}{4.042088in}}%
\pgfpathlineto{\pgfqpoint{1.202778in}{4.042088in}}%
\pgfusepath{stroke}%
\end{pgfscope}%
\begin{pgfscope}%
\pgfsetbuttcap%
\pgfsetroundjoin%
\definecolor{currentfill}{rgb}{0.843137,0.000000,0.000000}%
\pgfsetfillcolor{currentfill}%
\pgfsetlinewidth{1.003750pt}%
\definecolor{currentstroke}{rgb}{0.843137,0.000000,0.000000}%
\pgfsetstrokecolor{currentstroke}%
\pgfsetdash{}{0pt}%
\pgfsys@defobject{currentmarker}{\pgfqpoint{-0.041667in}{-0.041667in}}{\pgfqpoint{0.041667in}{0.041667in}}{%
\pgfpathmoveto{\pgfqpoint{0.000000in}{-0.041667in}}%
\pgfpathcurveto{\pgfqpoint{0.011050in}{-0.041667in}}{\pgfqpoint{0.021649in}{-0.037276in}}{\pgfqpoint{0.029463in}{-0.029463in}}%
\pgfpathcurveto{\pgfqpoint{0.037276in}{-0.021649in}}{\pgfqpoint{0.041667in}{-0.011050in}}{\pgfqpoint{0.041667in}{0.000000in}}%
\pgfpathcurveto{\pgfqpoint{0.041667in}{0.011050in}}{\pgfqpoint{0.037276in}{0.021649in}}{\pgfqpoint{0.029463in}{0.029463in}}%
\pgfpathcurveto{\pgfqpoint{0.021649in}{0.037276in}}{\pgfqpoint{0.011050in}{0.041667in}}{\pgfqpoint{0.000000in}{0.041667in}}%
\pgfpathcurveto{\pgfqpoint{-0.011050in}{0.041667in}}{\pgfqpoint{-0.021649in}{0.037276in}}{\pgfqpoint{-0.029463in}{0.029463in}}%
\pgfpathcurveto{\pgfqpoint{-0.037276in}{0.021649in}}{\pgfqpoint{-0.041667in}{0.011050in}}{\pgfqpoint{-0.041667in}{0.000000in}}%
\pgfpathcurveto{\pgfqpoint{-0.041667in}{-0.011050in}}{\pgfqpoint{-0.037276in}{-0.021649in}}{\pgfqpoint{-0.029463in}{-0.029463in}}%
\pgfpathcurveto{\pgfqpoint{-0.021649in}{-0.037276in}}{\pgfqpoint{-0.011050in}{-0.041667in}}{\pgfqpoint{0.000000in}{-0.041667in}}%
\pgfpathclose%
\pgfusepath{stroke,fill}%
}%
\begin{pgfscope}%
\pgfsys@transformshift{1.063889in}{4.042088in}%
\pgfsys@useobject{currentmarker}{}%
\end{pgfscope}%
\end{pgfscope}%
\begin{pgfscope}%
\definecolor{textcolor}{rgb}{0.000000,0.000000,0.000000}%
\pgfsetstrokecolor{textcolor}%
\pgfsetfillcolor{textcolor}%
\pgftext[x=1.313889in,y=3.993477in,left,base]{\color{textcolor}\sffamily\fontsize{10.000000}{12.000000}\selectfont \(\displaystyle \beta = 0.8\)}%
\end{pgfscope}%
\begin{pgfscope}%
\pgfsetbuttcap%
\pgfsetroundjoin%
\pgfsetlinewidth{1.505625pt}%
\definecolor{currentstroke}{rgb}{0.549020,0.235294,1.000000}%
\pgfsetstrokecolor{currentstroke}%
\pgfsetdash{{5.550000pt}{2.400000pt}}{0.000000pt}%
\pgfpathmoveto{\pgfqpoint{0.925000in}{3.838231in}}%
\pgfpathlineto{\pgfqpoint{1.202778in}{3.838231in}}%
\pgfusepath{stroke}%
\end{pgfscope}%
\begin{pgfscope}%
\pgfsetbuttcap%
\pgfsetmiterjoin%
\definecolor{currentfill}{rgb}{0.549020,0.235294,1.000000}%
\pgfsetfillcolor{currentfill}%
\pgfsetlinewidth{1.003750pt}%
\definecolor{currentstroke}{rgb}{0.549020,0.235294,1.000000}%
\pgfsetstrokecolor{currentstroke}%
\pgfsetdash{}{0pt}%
\pgfsys@defobject{currentmarker}{\pgfqpoint{-0.041667in}{-0.041667in}}{\pgfqpoint{0.041667in}{0.041667in}}{%
\pgfpathmoveto{\pgfqpoint{0.000000in}{0.041667in}}%
\pgfpathlineto{\pgfqpoint{-0.041667in}{-0.041667in}}%
\pgfpathlineto{\pgfqpoint{0.041667in}{-0.041667in}}%
\pgfpathclose%
\pgfusepath{stroke,fill}%
}%
\begin{pgfscope}%
\pgfsys@transformshift{1.063889in}{3.838231in}%
\pgfsys@useobject{currentmarker}{}%
\end{pgfscope}%
\end{pgfscope}%
\begin{pgfscope}%
\definecolor{textcolor}{rgb}{0.000000,0.000000,0.000000}%
\pgfsetstrokecolor{textcolor}%
\pgfsetfillcolor{textcolor}%
\pgftext[x=1.313889in,y=3.789620in,left,base]{\color{textcolor}\sffamily\fontsize{10.000000}{12.000000}\selectfont \(\displaystyle \beta = 0.9\)}%
\end{pgfscope}%
\begin{pgfscope}%
\pgfsetbuttcap%
\pgfsetroundjoin%
\pgfsetlinewidth{1.505625pt}%
\definecolor{currentstroke}{rgb}{0.007843,0.533333,0.000000}%
\pgfsetstrokecolor{currentstroke}%
\pgfsetdash{{5.550000pt}{2.400000pt}}{0.000000pt}%
\pgfpathmoveto{\pgfqpoint{0.925000in}{3.634374in}}%
\pgfpathlineto{\pgfqpoint{1.202778in}{3.634374in}}%
\pgfusepath{stroke}%
\end{pgfscope}%
\begin{pgfscope}%
\pgfsetbuttcap%
\pgfsetmiterjoin%
\definecolor{currentfill}{rgb}{0.007843,0.533333,0.000000}%
\pgfsetfillcolor{currentfill}%
\pgfsetlinewidth{1.003750pt}%
\definecolor{currentstroke}{rgb}{0.007843,0.533333,0.000000}%
\pgfsetstrokecolor{currentstroke}%
\pgfsetdash{}{0pt}%
\pgfsys@defobject{currentmarker}{\pgfqpoint{-0.041667in}{-0.041667in}}{\pgfqpoint{0.041667in}{0.041667in}}{%
\pgfpathmoveto{\pgfqpoint{-0.000000in}{-0.041667in}}%
\pgfpathlineto{\pgfqpoint{0.041667in}{0.041667in}}%
\pgfpathlineto{\pgfqpoint{-0.041667in}{0.041667in}}%
\pgfpathclose%
\pgfusepath{stroke,fill}%
}%
\begin{pgfscope}%
\pgfsys@transformshift{1.063889in}{3.634374in}%
\pgfsys@useobject{currentmarker}{}%
\end{pgfscope}%
\end{pgfscope}%
\begin{pgfscope}%
\definecolor{textcolor}{rgb}{0.000000,0.000000,0.000000}%
\pgfsetstrokecolor{textcolor}%
\pgfsetfillcolor{textcolor}%
\pgftext[x=1.313889in,y=3.585763in,left,base]{\color{textcolor}\sffamily\fontsize{10.000000}{12.000000}\selectfont \(\displaystyle \beta = 1\)}%
\end{pgfscope}%
\end{pgfpicture}%
\makeatother%
\endgroup%

      \caption[GMRES iteration counts when $\NLqDRR{\nso-\nst} = 0.2\times k^{-\beta},$ for any $1 \leq q < \infty$ and $\beta = 0.8,0.9,1$.]{GMRES iteration counts for $\AmatoI\Amatt$ given by \cref{eq:noweak,eq:ntweak}, where $\alpha = 0.2\times k^{-\beta},$ for $\beta = 0.8,0.9,1.$}\label{fig:l1high}
\end{figure}

\begin{table}
  \centering
  \begin{tabular}{Sc Sc Sc Sc Sc Sc Sc Sc Sc ScSc }
\toprule

$\eps$\textbackslash$k$ &  10.0  &  20.0  &  30.0  &  40.0  &  50.0  &  60.0  &  70.0  &  80.0  &  90.0  &  100.0 \\

\midrule

0.0 &     14 &     40 &    119 &    258 &    427 &    627 &    940 &   1274 &   1695 &   2116 \\

0.1 &     13 &     27 &     70 &    147 &    262 &    394 &    590 &    825 &   1128 &   1393 \\

0.2 &     12 &     22 &     40 &     77 &    134 &    199 &    292 &    419 &    551 &    726 \\

0.3 &     11 &     18 &     25 &     40 &     58 &     86 &    119 &    163 &    209 &    270 \\

0.4 &     10 &     15 &     20 &     25 &     30 &     42 &     53 &     64 &     81 &     98 \\

0.5 &     10 &     13 &     16 &     19 &     22 &     25 &     28 &     31 &     37 &     41 \\

0.6 &      9 &     11 &     13 &     14 &     16 &     17 &     19 &     19 &     21 &     22 \\

0.7 &      8 &      9 &     10 &     11 &     12 &     13 &     13 &     14 &     14 &     14 \\

0.8 &      8 &      8 &      9 &      9 &     10 &     10 &     10 &     10 &     10 &     10 \\

0.9 &      7 &      7 &      8 &      8 &      8 &      8 &      8 &      8 &      8 &      8 \\

1.0 &      7 &      6 &      7 &      7 &      7 &      7 &      7 &      7 &      7 &      7 \\

\bottomrule

\end{tabular}


  \caption[GMRES iteration counts when $\NLqDRR{\nso-\nst} = 0.2\times k^{-\beta},$ for any $1 \leq q < \infty$ and $\beta = 0,0.1,\ldots,1$.]{GMRES iteration counts for $\AmatoI\Amatt$ given by \cref{eq:noweak,eq:ntweak}, where $\alpha = 0.2\times k^{-\beta}.$}\label{tab:l1}
  \end{table}



\section[Applying nearby preconditioning to QMC]{Applying nearby preconditioning to Quasi-Monte-Carlo methods for the Helmholtz equation}\label{sec:nbpcqmc}

We now apply nearby preconditioning in the implementation of a Quasi-Monte-Carlo (QMC) method for the Helmholtz equation. We begin with a brief description of QMC methods, before detailing two ways in which we apply nearby preconditioning to these methods. Finally, we give computational results illustrating this application.

\subsection{Brief description of QMC}

QMC methods (or rules) are high-dimensional quadrature rules, designed to give rates of convergence (with respect to the number of integration points) which are superior to those of Monte-Carlo methods, under certain conditions. Suppose one wants to approximate $\EXP{Q},$ where $Q$ is some random variable (later in this \lcnamecref{sec:nbpcqmc}, $Q$ will be a function of the solution $u(\omega)$ of a stochastic Helmholtz equation). By definition, the expectation is
\beq\label{eq:qmcexpdef}
\EXP{Q} = \int_\Omega Q(\omega)\ \ddPPomega.
\eeq

If we now suppose $Q$ depends on the sample space $\Omega$ via a finite set of random variables $\Uo,\ldots,\UJ$, then we can rewrite \cref{eq:qmcexpdef} as
\beq\label{eq:qmcexp2}
\EXP{Q} = \int_\Omega Q\mleft((\Uo(\omega),\ldots,\UJ(\omega)\mright)\, \ddPPomega.
\eeq
If, for example, the $\Uj$ are all uniform random variables on $\mleft[-1/2,1/2\mright]$, then \cref{eq:qmcexp2} can be rewritten as
\beq\label{eq:qmcexp3}
\EXP{Q} = \int_{\cube{J}} \hspace{-3em}Q\mleft(\by\mright)\, \dd\lambda(\by),
\eeq
where $\lambda$ denotes Lebesgue measure.

Any quadrature rule, or method for approximating $\EXP{Q}$, can then be seen as a method for approximating the $J$-dimensional integral on the right-hand side of \cref{eq:qmcexp3} and vice-versa. Equal-weight quadrature rules choose points $\byo,\ldots,\byNpoints \in \cube{J}$ and use the approximation
\beqs
\EXP{Q} \approx \frac1{\Npoints}\sum_{l=1}^{\Npoints} Q\mleft(\byl\mright).
\eeqs
Monte-Carlo and Quasi-Monte-Carlo rules correspond to different choices of the points $\byl$. In a Monte-Carlo rule the points are chosen at random in accordance with the associated probability distribution. For example, in the case that the $\Uj$ are $\Unif(-1/2,1/2)$ random variables, the points $\byl$ are chosen according to the Uniform distribution on $\cube{J}$. Observe that Monte-Carlo rules do not need the dependence of $Q$ on $\omega$ to take the form prescribed in \cref{eq:qmcexp2}, indeed, they apply to any random variable.

Quasi-Monte-Carlo rules, in contrast to Monte-Carlo rules, do require the dependence on $\omega$ to be via finitely- or countable-many random variables. This is because QMC rules are high-dimensional quadrature rules (in the simplest case performing quadrature on the high-dim\-en\-sion\-al cube $\cube{J}).$ In pure QMC rules the points $\byl$ are chosen deterministically, unlike Monte-Carlo rules.

The main advantage of QMC rules is that they can exhibit higher rates of convergence compared to Monte-Carlo rules; Monte Carlo rules typically converge with rate $\Npoints^{-1/2}$ (see, e.g., \cite[Section 1.1]{Gi:15}), whereas QMC rules can converge with rates up to $\Npoints^{-1}$  or with even higher rates for higher-order QMC rules, see, e.g., \cite[Penultimate paragraph of Section 1.2]{KuNu:16}.

In applying QMC rules to stochastic PDEs, we assume that the random coefficient ($n$ in our case) is defined via finitely many (or countably many) random variables, as in \cref{eq:qmcexp2} above, and we then use QMC rules to estimate expectations of quantities of interest of the solution $u$, i.e., $Q = Q(u).$ We note that applying QMC rules to stochastic PDEs is a vibrant and active research area. For recent overviews of this field, see \cite{KuNu:16,KuNu:18b} (and the associated tutorial \cite{KuNu:18a}). We note that there is currently no rigorous study of how QMC methods behave for the Helmholtz equation, although we understand some such work is currently underway by Ganesh, Kuo, and Sloan \cite{GaKuSl}.

\subsection{Methods for applying nearby preconditioning to QMC}\label{sec:nbpcqmcnum}
In all of our previous uses of nearby preconditioning, we have fixed $\nso$, the value for which we calculate the preconditioner, and have then used $\Amato$ to precondition $\Amatt$ for different values of $\nst.$ However, the key idea for applying nearby preconditioning to QMC methods for the Helmholtz equation is to choose \emph{a number} of different realisations of $\nso$ and use each realisation of $\nso$ as a preconditioner only for those  realisations of $\nst$ for which $\Amato$ is a \emph{good} preconditioner for $\Amatt.$ We adopt this approach because it is highly unlikely that a single realisation of $\nso$ will be a good preconditioner for every realisation of $\nst.$

Therefore, the algorithms presented in this \lcnamecref{sec:nbpcqmcnum} seek to answer the two questions:
\ben
\item For which realisations of $n$ should a preconditioner be calculated?
  \item To which realisations of $n$ should each preconditioner be applied?
\een

We now detail two methods for using nearby preconditioning to speed up QMC methods for the Helmholtz equation. To apply these methods, we use the following model problem: We consider the Interior Impedance Problem in 2-d with $f=1$ and $\gI=0$, $A = I$, and $n$ given by
\beq\label{eq:artificialkl}
n(\omega,\bx) = 1 + \sum_{j=1}^{10} \Uj(\omega) \sqrt{\lambdaj} \psij(\bx),
\eeq
where
\beq\label{eq:artificialkllambdas}
\sqrt{\lambdaj} = j^{-2}
\eeq
and
\beq\label{eq:artificialklfuns}
\psij(\bx) = \cos\mleft(\frac{j\pi}4 x\mright)\cos\mleft(\frac{\mleft(j+1\mright)\pi}4 y\mright).
\eeq
Observe that $\NLiDR{\psij}=1$ for all $j,$ and $\sqrt{\lambdaj} \rightarrow 0$ as $j \rightarrow \infty.$ This expansion is based on the random-field expansion used in \cite[Section 5.1]{GiGrKuScSl:19}, although the main change we make from \cite{GiGrKuScSl:19} is to introduce the factors $1/4$ in \cref{eq:artificialklfuns}. We introduce this factor to ensure that the oscillations in the medium $n$ are `low frequency' compared to the frequency $k$ of the waves passing through the medium\footnote{The highest `frequency' associated with the oscillations in the medium is $(10+1)\pi/4 \approx 26$, whereas we consider waves with frequencies $k=10,\ldots,60$. Therefore (for $k > 26$) the waves are of a `higher frequency' than the medium. Moreover, we would see if there is any change in the behaviour of our algorithm as the frequency of the waves increases past the `frequency' of the medium. However, we do not see any such change.}. Expansions similar to \cref{eq:artificialkl} are often decribed as `artificial Karhunen--Lo\`eve expansions' due to their similarity with the Karhunen--Lo\`eve expansion of a random field. In a Karhunen--Lo\`eve expansion the $\Uj$ are independent random variables whose distribution is determined by the distribution of the random field, and the $\lambdaj$ and $\psij$ are the eigenvalues and eigenvectors of the covariance operator, see, e.g., \cite[Section 7.4]{LoPoSh:14}. Since we will be using QMC methods to approximate $\EXP{Q(u)}$ (for some quantities of interest $Q$), we will sometimes instead write $n(\by)$ for $\by \in \cube{10}$, by which we mean
\beqs
n(\by) = 1 + \sum_{j=1}^{10} \by_{j} \sqrt{\lambdaj} \psij.
\eeqs
There is no a priori reason that one must have such an affine dependence of the random field on the randomness in order to apply nearby preconditioning to QMC methods. One could, for example, take $n$ to be a lognormal random field, in which case $n$ would take the form $n(\by) = \exp\mleft(\nz + \sum_j \Nj \sqrt{\lambdaj} \psij\mright)$ where the $\Nj$ are Normal$(0,1)$ random variables.. However, in the case of affine dependence there is a `parallelisable' nearby-preconditioning-QMC algorithm which we present below.

We stress that the results in this \lcnamecref{sec:nbpcqmcnum} are strictly numerical; there is no current theory to support these calculations. In particular, we observe in \cref{sec:nbpcqmcnumerics} below that in these experiments, for the QMC error for Helmholtz problems to remain bounded as $k$ increases, one must increase the number of QMC points with $k.$ We again remark that there is currently no theoretical justification for this behaviour.


\paragraph{Terminology} Before we describe the nearby-preconditioning-QMC algorithms in detail, we establish two pieces of terminology that will be of use in describing them. Firstly, we will use the word `point' to refer to a point in the parameter space $\cube{J}$, and use phrases such as `calculate a preconditioner at the point $\by$' as shorthand for `calculate the LU decomposition of the system matrix $\Amat$ corresponding to the finite-element discretisation of the Helmholtz IIP (as described above) with coefficient $n(\by)$'.

We also use the words `nearby' and `nearest' (when referring to QMC points) to mean: nearest in an approximation of the metric
\beq\label{eq:dqmc}
\dQMC(\byo,\byt) = \NLiDRR{n(\byo)-n(\byt)},
\eeq
i.e., the metric on $\cube{J}$ induced by the spatial $L^\infty$ norm. (See \cref{eq:dapprox} below for the approximation of $\dQMC$ that we use.) We use this metric to describe the geometry of our QMC points as the best rigorous results (in terms of $k$-dependence) are proved in \cref{sec:3} for the $L^\infty$-norm\footnote{Although, in line with the results in \cref{sec:weaknorm}, we could instead use a spatial $L^q$ norm, for some $q \geq 1$ in \cref{eq:dqmc}.}. Therefore, when considering which QMC points will give preconditioners suitable for use with other QMC points, this metric is a natural metric to use.


\subsubsection{A sequential algorithm}
We first describe a straightforward algorithm that uses nearby preconditioning to speed up QMC calculation. We call this a `sequential' algorithm because, unlike the `parallel' algorithm that we describe below, it is intrinsically sequential and cannot be parallelised, i.e., finite-element solves for different realisations of the random field $n$ cannot be treated in parallel. Although, when performing the individual finite-element solves one is not restricted to a single core, i.e., one can use parallelisation for each finite-element solve  if the linear systems $\Amat$ are large enough to warrant this.

An overview of the algorithm is:
\ben
\item Choose a QMC point $\by$ for which to calculate a preconditioner
\item\label[itemstep]{it:nearest} Find the nearest QMC point $\byp$ to $\by$ and attempt a GMRES solve of the problem at $\byp$ using the LU decomposition of the system at $\by$ as a preconditioner.
    \item If GMRES converges quickly (i.e., in fewer than a preset number of iterations), return to \cref{it:nearest}.
\item If GMRES takes too long to converge, recalculate the preconditioner at the current point, and return to \cref{it:nearest}.
  \een
  The algorithm is written in more formal pseudocode in \cref{alg:seq}.
\begin{algorithm}[h]
\DontPrintSemicolon
\SetKwInOut{Input}{Input}\SetKwInOut{Output}{Output}
\SetKwFunction{Nearest}{nearest}

\Input{$\maxGMRES$,$\SQMC$}
\BlankLine
Choose starting point $\bystart$\;
$\bypre \defined \bystart$\;
$\Sremaining \defined \SQMC\setminus\set{\bypre}$\;
Calculate and store preconditioner $\Lmat\Umat = \AmatpreI$\;
$\bycurrent \defined$ \Nearest{$\bypre,\Sremaining$}\;
\While{$\Sremaining \neq \emptyset$}{
\eIf{GMRES applied to $\UmatI\LmatI\Amat\bycurrent = \UmatI\LmatI \bff$ converges in fewer than $\maxGMRES$ iterations}{
$\Sremaining \defined \Sremaining\setminus\set{\bycurrent}$\;
$\bycurrent \defined$ \Nearest{$\bypre,\Sremaining$}\;
}{
$\bypre \defined \bycurrent$\;
Calculate and store preconditioner $\Lmat\Umat = \AmatpreI$\;
}
}
\caption[The sequential nearby-preconditioning-Quasi-Monte-Carlo algorithm]{The sequential nearby-preconditioning-Quasi-Monte-Carlo algorithm\label{alg:seq}. $\maxGMRES$ is the maximum allowed number of GMRES iterations and $\SQMC$ is the set of all QMC points. $\nearest(\bypre,\Sremaining)$ denotes the point in $\SQMC$ nearest to $\bypre$ in the $\dQMC$ metric.}
\end{algorithm}
\subsubsection{A parallel algorithm}

The main disadvantage of the `sequential' algorithm described above is that the points at which preconditioners are calculated are identified as the algorithm progresses. The algorithm cannot be parallelised by sending different collections of QMC points to different processors (as one does not know  a priori which preconditioner to use for each QMC point). Therefore, we now suggest an alternative algorithm that allows one to specify the number of preconditioning points \emph{before} the algorithm begins. The algorithm then calculates which points to use as preconditioning points, before performing the linear solves. Because the preconditioners are known in advance, the solves can be computed in parallel if required. The most complicated part of the algorithm is deciding at which points to calculate the preconditioners, and so we describe this part of the algorithm in more detail here. A more formal pseudocode description of the algorithm is given in \cref{alg:par}.

Suppose we are given a set $\SQMC = \set{\byo,\ldots,\byNQMC}$ of QMC points and a number $\Npretarget$; the target number of preconditioners to compute. The aim of this algorithm is to select (approximately) $\Npretarget$ QMC points that are (approximately) equally spaced with respect to the $\dQMC$ metric defined above. If such a goal is acheived, then one expects that the preconditioning points are best located to minimise the total number of GMRES iterations across the solves for all of the QMC points.

The algorithm contains two key ideas:
\ben
  \item Use a surrogate metric in place of $\dQMC$, and
\item Locate the preconditioning points according to a tensor-product rule.
  \een
  We now describe each of these two ideas in turn, before describing our final algorithm.

\paragraph{Surrogate metric} Whilst the metric $\dQMC$ is the metric in which  nearby preconditioning is analysed (as described in \cref{sec:intronbpc} above), in practice $\dQMC$ is difficult to work with, since the  geometry it induces on $\cube{J}$ is nontrivial as this geometry is dependent on the interaction between the functions $\psij$ in the expansion \cref{eq:artificialkl}. Therefore, we work in an alternative, although related metric
  \beq\label{eq:dapprox}
\dapprox(\byo,\byt) \de \sum_{j=1}^{J} \sqrt{\lambdaj} \abs{{\byo}_{j} - {\byt}_{j}}, \tfor \byo,\by \in \cube{J}.
\eeq
Observe that the structure of $\dapprox$ is similar to that of $\dQMC$ and for all $\byo,\byt \in \cube{J},$ the inequality $\dQMC\mleft(\byo,\byt\mright) \leq \dapprox\mleft(\byo,\byt\mright)$ holds. Also observe that $\dapprox$ is a weighted $L^1$ metric on $\cube{J}$, with the weights corresponding to the terms in \cref{eq:artificialkl}. Recall that $\sqrt{\lambdaj} \rightarrow 0$ as $j \rightarrow \infty$; therefore the higher dimensions contribute less to the value of $\dapprox$ (or, informally, points are `closer' in higher dimensions, or higher dimensions are `smaller' than lower dimensions). However, it is easy to compute with $\dapprox,$ and it is obvious that it enables one to think of $\cube{J}$ as the high-dimensional rectangle $\mleft[0,\sqrt{\lambdao}\mright]\times\cdots\times\mleft[0,\sqrt{\lambdaJ}\mright]$ equipped with the standard $L^1$ metric.

\paragraph{Tensor-product algoritm for locating preconditioning points} We first describe the intuition behind our use of a tensor-product rule to locate the preconditioning points (even though we do not use this intuition in the final algorithm). Once we have described this intution, we will then show how it can be adapted to provide the final algorithm. To understand why we use locate the preconditioning points using a tensor-product rule, we first decribe the heuristic we use. Let us assume we want to cover $\cube{J}$ with balls of radius $r$ (where these balls are measured in the $\dapprox$ metric, and therefore have a similar geometry to balls on $\cube{J}$ in the $L^1$ metric). Therefore, given the centres $\bcone$ and $\bct$ of two adjacent balls, we will have
\beq\label{eq:centres2r}
\dapprox(\bcone,\bct) = 2r.
\eeq
The question now arises of how we choose $\bcone$ and $\bct$ so that \cref{eq:centres2r} holds. We observe that, by the definition of $\dapprox$, if we choose $\bcone$ and $\bct$ such that
\beqs
\sqrt{\lambdaj}\abs{{\bcone}_{j}-{\bct}_j} = \frac{2r}{J} \tforall j = 1,\ldots,J,
\eeqs
then we will have \cref{eq:centres2r} by construction, because
\beqs
\dapprox(\bcone,\bct) = \sum_{j=1}^J \frac{2r}J = 2r.
\eeqs
Therefore, in dimension $j$ we choose the centres of the balls to be spaced
\beqs
\min\set{\frac{2r}{J\sqrt{\lambdaj}},1}
\eeqs
apart (where we include the minimum so that, for high dimensions, we include at least one centre). That is, in dimension $j$, we take
\beq\label{eq:Nj}
\Nj \de \max\set{1,\frac{J\sqrt{\lambdaj}}{2r}}
\eeq
equally spaced points in the sets $\centresj = \set{c_{j,1},\ldots,c_{j,\Nj}},$ and then we form the centres $\bcone,\ldots,\bcNpre$ by taking tensor products of the points in $\centreso,\ldots,\centresJ,$ giving a total of
\beq\label{eq:Npre}
\Npre = \No \times \cdots \times \NJ
\eeq
preconditioning points.

However, we face three immediate difficulties with  the above approach:
\ben
\item The above procedure assumes we know the radius $r$, and then returns the total number of preconditioning points, and their locations. However, we only know in advance the ideal total number of preconditioning points.
\item There is no guarantee that the numbers of points $\Nj$ calculated above are integers.
  \item There is no guarantee the preconditioning points given by the above procedure are QMC points.
    \een
    These questions are all completely valid, and so we slightly modify the above procedure to deal with them.

    \paragraph{Definition of the parallel algorithm} Recall that we assume that we are given a target number of preconditioners $\Npretarget$. The above procedure (amongst other things) defines a map $\Npreideal:\RRp \rightarrow \RRp$ given by $r \mapsto \Npre,$ where $\Npre$ is defined by \cref{eq:Npre} and the number of preconditioners in each dimension is given by \cref{eq:Nj}.  Therefore we can numerically invert this map (or more precisely, calculate numerically the value $\rideal$ such that $\Npreideal(\rideal) = \Npretarget$). (In our computations, we do this calculation via interval bisection.)

Given we expect that the size of the balls over which nearby preconditioning is effective decreases with $\cO\mleft(1/k\mright)$ (in line with \cref{cor:1}), and the number of QMC points needed to keep the error bounded increases with $k$ (see \cref{sec:nbpcqmcnumerics} below), it is not obvious that we should know $\Npreideal$ in advance. See \cpageref{page:seqandpar} for how we use the sequential algorithm to determine how $\Npreideal$ scales with $k$ for the parallel algorithm.

    Once we know the value of $\rideal,$ we can then calculate the numbers of centres in each dimension $\No(\rideal),\ldots,\NJ(\rideal)$ as above (recalling that the $\Nj(\rideal)$ are not necessarily integers). We then obtain integers $\Npreactualj = \round{\Nj(\rideal)}$, where $\round{\cdot}$ denotes rounding to the nearest integer. (Recall $\Nj(\rideal) \geq 1$ for all $j$ by construction, so $\Npreactualj$ will be a positive integer for all $j.$)

We then take $\Npreactualj$ centres in each dimension, as described above. We then obtain a total of $\Npreactual = \Npreactualo \times \cdots \times \NpreactualJ$ preconditioning points.

These points may not be QMC points. We could simply calculate the preconditioners at these non-QMC points. However we instead replace each calculated centre with its nearest QMC point and calculate the preconditioners at these QMC points. Finally, we calculate the map $\Prenearest:\SQMC\rightarrow\Spre$, i.e., for each QMC point we find its nearest preconditioning point, and use the corresponding preconditioner for the linear solve.

    This algorithm is summarised more formally in \cref{alg:par}.
    
    \bre[Is calculating $\Prenearest$ computationally expensive?]
    We note that calculating the map $\Prenearest:\SQMC\rightarrow \Spre$ is an $\cO\mleft(\NQMC\Npre\mright)$ operation, because for each QMC point we must find the nearest preconditioning point. Given that $\Spre \subseteq \SQMC,$ it is possible that calculating $\Prenearest$ could actually be an $\cO\mleft(\NQMC^2\mright)$ operation.

    However, we expect that $\Npre$ will be small relative to $\NQMC$ (and this is borne out in the numerical experiments summarised in \cref{tab:nbpcqmcpar} below) and therefore we expect $\cO\mleft(\NQMC\Npre\mright) \approx \cO\mleft(\NQMC\mright).$ Hence calculating $\Prenearest$ should not be an expensive computational task.

    A similar line of reasoning shows that calculating the nearest QMC point to each of calculated tensor-product points (as outlined above) should also be an $\cO\mleft(\NQMC\mright)$ task.
    \ere

%% Define
%% \beqs
%% \Npreidealj(r) = \max\set{\frac{J \sqrt{\lambdaj}}{2r},1}.
%% \eeqs
%% The `ideal' total number of QMC points is
%% \beqs
%% \Npreideal(r)=\prod_{j=1}^J  \Npreidealj(r)
%% \eeqs

%% Want to calculate the number of preconditioners $\Npre$, the set
%% \beqs
%% \Spre=\set{\ypreo,\ldots,\ypreNpre}
%% \eeqs
%% of QMC points at which to calculate the preconditioner and the map
%% \beqs
%% \nearestpre:\SQMC\rightarrow\Spre
%% \eeqs
%% taking each QMC point to its nearest (in the induced spatial $L^\infty$ norm) preconditioner, where $\SQMC$ is the set of QMC points.

\begin{algorithm}[h]
\DontPrintSemicolon
\SetKwInOut{Input}{Input}\SetKwInOut{Output}{Output}
\SetKwFunction{Round}{round}

\Input{$\Npretarget \in \NN$}
\Output{The set $\Spre$, the map $\nearestpre:\SQMC\rightarrow\Spre$}
\BlankLine
Solve (numerically) $\Npreideal(\rideal) = \Npretarget$ for $\rideal$\;
\For{j $= 1$ \KwTo $J$}{
Calculate $\Npreactualj =$ \Round{$\Npreidealj(\rideal)$}\;
Define $\Sprej$ to be set of $\Npreactualj$ equally spaced points in $\mleft[-1/2,1/2\mright]$\;
}
Define $\displaystyle\Npre = \prod_{j=1}^J \Npreactualj$\;
Define $\Spre$ by taking all possible tensor products of points in $\Sprej$, and then finding the nearest QMC point to each one\;
\For{l $=1$ \KwTo $\NQMC$}{
Calculate $\nearestpre\mleft(\by^{(l)}\mright)$\;
}
\caption[The main part of the parallel nearby-preconditioning-Quasi-Monte-Carlo algorithm.]{The main part of the parallel nearby-preconditioning-Quasi-Monte-Carlo algorithm. This part of the algorithm determines $\Spre$ and $\nearestpre$. $\Spre$ is the set of preconditioning points, and $\nearestpre:\SQMC\rightarrow\Spre$ maps each QMC points to its nearest preconditioning point.\label{alg:par}}
\end{algorithm}

\subsubsection{Comparing and Constrasting the two algorithms}

We now briefly list the main differences in the two algorithms given above.

\paragraph{Complexity} The sequential algorithm is simple and intuitive to describe, given that it mainly revolves around `finding the nearest point'. However, the parallel algorithm is much more complicated, both in the underlying ideas, but also in its technical definition.

\paragraph{Heuristics} The sequential algorithm has very minimal heuristics; one only needs to specify the maximum number of GMRES iterations and this could be determined, for example, by the memory constraints of the machine one is using. In contrast, for the parallel algorithm one needs a heuristic for how many preconditioning points to choose, as this is not given by the algorithm. (In our numerical experiments below, we obtain this heuristic by using the sequential algorithm for low $k$, and then extrapolating the proportion of preconditioning points used for low values of $k$ to larger values of $k.$

\paragraph{Parallelisability} Unsurprisingly (given the name) the sequential algorithm is inherently serial; one must see whether a given solve converges in the required number of GMRES iterations before knowing whether we must recalculate the preconditioner for subsequent solves. (In principle one could parallelise the algorithm by splitting the QMC points up onto different groups of processors, and then use the sequential algorithm on each group of processors. However, there is no guarantee one would split the QMC points up in a way that grouped nearby points, therefore this approach could lead to a substantial increase in computational work.) In contrast, the parallel algorithm is fully parallelisable; once the preconditioning points and the map $\nearestpre:\SQMC\rightarrow\Spre$ have been calculated, one can send different linear solves to different groups of processors as one chooses. (Although note that, unless one sends all of the QMC points corresponding to a single preconditioner to the \emph{same} group of processors, one may need to calculate the same preconditioner several times, on different groups of processes\footnote{In our code, we split up the points with respect to the order they are generated by the QMC code. This was purely to make the code simpler.} However, the decrease in computational time gained from parallelisation should more than offset this increase in computational effort.)

\paragraph{Choice of preconditioning points} Neither algorithm will necessarily pick the optimal set of preconditioning points (optimal in the sense of the minimal number of preconditioning points needed). In the sequential algorithm, there is no guarantee that this method for exploring the sample space and choosing the preconditioning points will give an optimal collection of preconditioning points. Also, whilst for the parallel algorithm the preconditioning points should fill the parameter space `well' (given the points are chosen a priori to be well spaced according to the $\dapprox$ metric), the number of preconditioning points generated is not exactly $\Npretarget$ due to rounding the `ideal' number of centres in each dimension to the nearest integer. Therefore, even in the parallel case, one may not end up with an optimal set of preconditioning points.

\subsection{Numerical Experiments}\label{sec:nbpcqmcnumerics}
We now describe numerical experiments that demostrate the effectiveness of the above algorithms.         Our main result is that, for a particular QMC model problem nearby preconditioning gives a substantial speedup, with around 98\% of solves being computed using a previously-calculated LU decomposition.

For the computational setup, including the algorithm we use to generate our QMC points, see \cref{app:compsetup}.

Before we perform our numerical experiments, we need to determine:
\bit
\item How the number of QMC points should scale with $k$, and
  \item How many preconditioners we should choose.
    \eit
    Throughout this \lcnamecref{sec:nbpcqmcnumerics} we use the model problem detailed in \cref{eq:artificialkl,eq:artificialkllambdas,eq:artificialklfuns} above.

\subsubsection{QMC error estimators}
    
    To determine how the number of QMC points should scale with $k$, we first estimate how the QMC error grows as $k$ increases. The QMC rule we use is a randomly shifted QMC rule, we use such a rule because there exists an error estimator for this rule, see \cref{eq:errest} below. Our exposition below follows that in \cite[Section 4.2]{GrKuNuScSl:11}.

    Suppose our QMC points are $\byo,\ldots,\byNQMC$, and the resulting QMC rule is
    \beqs
\QMC{Q} = \frac1{\NQMC}\sum_{l=1}^{\NQMC} Q\mleft(u\mleft(\byl\mright)\mright).
\eeqs
For a `shift' $\shift \in \cube{J}$ we define the shifted QMC rule
\beqs
\QMCshift{Q}{\shift} = \frac1{\NQMC}\sum_{l=1}^{\NQMC} Q\mleft(u\mleft(\byl\oplus\shift\mright)\mright),
\eeqs
where $\by \oplus \shift$ denotes $\by + \shift$ `wrapped around' onto the hypercube $\cube{J}$. (Formally $\by \oplus \shift = \fracoperator{\mleft(\by + \bhalf\mright)+\shift} - \bhalf,$ where $\fracoperator{\cdot}$ denotes the fractional part and $\bhalf$ denotes the $J$-dimensional vector with every entry $1/2.$)

We can then define the randomly-shifted QMC rule (with multiple randomly-chosen shifts $\shifto,\ldots,\shiftNshifts$)
\beqs
\QMCrandshift{Q}{\Nshifts} = \frac1{\Nshifts}\sum_{s=1}^{\Nshifts} \QMCshift{Q}{\shifts} = \frac1{\NQMC\Nshifts}\sum_{s=1}^{\Nshifts}\sum_{l=1}^{\NQMC} Q\mleft(u\mleft(\byl\oplus \shifts\mright)\mright).
\eeqs

Having defined the randomly shifted QMC rule, one can use the standard statistical estimator of the standard deviation of the statistical error in $\QMCrandshift{Q}{\Nshifts}$ \cite[Equation (4.6)]{GrKuNuScSl:11}
\beq\label{eq:errest}
\QMCerror{\NQMC}{\Nshifts} = \mleft(\frac1{\Nshifts\mleft(\Nshifts-1\mright)}\sum_{s=1}^{\Nshifts} \mleft(\QMCshift{Q}{\shifts} - \QMCrandshift{Q}{\Nshifts}\mright)^2\mright)^{\half}.
\eeq
(See \cref{app:complexerror} for proof that $\QMCerror{\NQMC}{\Nshifts}^2$ is an unbiased estimator of the variance of $\QMCrandshift{Q}{\Nshifts}$; recall that it does \emph{not} then follow that $\QMCerror{\NQMC}{\Nshifts}$ is an \emph{unbiased} estimator of the standard deviation of $\QMCrandshift{Q}{\Nshifts}$.)

\subsubsection{$k$-dependence of the number of QMC points}

We first sought to determine how $\QMCerror{\NQMC}{\Nshifts}$ depends on $k.$ We estimated the error $\QMCerror{\NQMC}{\Nshifts}$ for the setup described in \cref{app:compsetup} with $\NQMC = 2048$ and $\Nshifts=20$ (i.e., 40,960 PDE solves in total) for $k = 10,20,30,40,50,60$. We set $h = 0.002$ for all of the computations (as $0.002 \approx 60^{-3/2}$), as then by \cref{thm:fembound} the finite-element error is of the order $h^2k^3 \sim (k/60)^3 \lesssim 1$ for all the values of $k$ we consider. The quantities of interest (QoIs) we considered were:
\bit
\item The integral of $u$ over the whole domain $\mleft[0,1\mright]^2$,
\item The value of $u$ at the origin,
\item The value of $u$ at the top-right corner of the domain, and
\item The $x$-component of $\grad u$ at the top-right corner of the domain.
  \eit
  Observe that these QoIs require a certain amount of regularity of the solution. (The integral is defined for functions in $\LoD$, point evaluation for functions in $\Hfn{}{3/2 + \eps}{D}$ and point evaluation of the gradient for functions in $\Hfn{}{5/2+\eps}{D}$ (in 3-d - the corresponding function spaces are $\Hfn{}{1+\eps}{D}$ and $\Hfn{}{2+\eps}{D}$ in 2-d) for any $\eps > 0.$) Therefore computing for this range of QoIs will give a good insight into the behaviour of QMC applied to the Helmholtz equation\footnote{We can evaluate point values of $\uh$ because $\uh$ is continuous, and we use the constant value of $\grad \uh$ on the upper-rightmost mesh element as a proxy for $\grad \uh((1,1))$; such a use is possible due to the structure of our mesh, see \cref{fig:grid}, and the fact that we use first-order finite elements.}.
%%     That is, we randomly choose $\shifto,\ldots,\shiftNshifts$ points in $\cube{J}$ (the `shifts')rause the standard error estimator
%%     \beqs
%%     \mleft(\frac1{\nu\mleft(\nu-1\mright)} \sum_{s=1}^{\Nshifts} 
%%     \eeqs$h = 0.002$ (relation to $k=60$ - maximum?)

%In \cref{fig:integralCalpha,fig:originCalpha,fig:toprightCalpha,fig:gradienttoprightCalpha} we plot how $C$ and $\alpha$ depend on $k$, for the plots of the QMC error with increasing $\NQMC$ for each value of $k,$ see \cref{app:hhqmcconv}.

\begin{figure}[h]
    \centering
    \begin{subfigure}{\textwidth}
      \centering
%% Creator: Matplotlib, PGF backend
%%
%% To include the figure in your LaTeX document, write
%%   \input{<filename>.pgf}
%%
%% Make sure the required packages are loaded in your preamble
%%   \usepackage{pgf}
%%
%% Figures using additional raster images can only be included by \input if
%% they are in the same directory as the main LaTeX file. For loading figures
%% from other directories you can use the `import` package
%%   \usepackage{import}
%% and then include the figures with
%%   \import{<path to file>}{<filename>.pgf}
%%
%% Matplotlib used the following preamble
%%   \usepackage{fontspec}
%%   \setmainfont{DejaVuSerif.ttf}[Path=/home/owen/progs/firedrake-complex/firedrake/lib/python3.5/site-packages/matplotlib/mpl-data/fonts/ttf/]
%%   \setsansfont{DejaVuSans.ttf}[Path=/home/owen/progs/firedrake-complex/firedrake/lib/python3.5/site-packages/matplotlib/mpl-data/fonts/ttf/]
%%   \setmonofont{DejaVuSansMono.ttf}[Path=/home/owen/progs/firedrake-complex/firedrake/lib/python3.5/site-packages/matplotlib/mpl-data/fonts/ttf/]
%%
\begingroup%
\makeatletter%
\begin{pgfpicture}%
\pgfpathrectangle{\pgfpointorigin}{\pgfqpoint{5.000000in}{4.000000in}}%
\pgfusepath{use as bounding box, clip}%
\begin{pgfscope}%
\pgfsetbuttcap%
\pgfsetmiterjoin%
\definecolor{currentfill}{rgb}{1.000000,1.000000,1.000000}%
\pgfsetfillcolor{currentfill}%
\pgfsetlinewidth{0.000000pt}%
\definecolor{currentstroke}{rgb}{1.000000,1.000000,1.000000}%
\pgfsetstrokecolor{currentstroke}%
\pgfsetdash{}{0pt}%
\pgfpathmoveto{\pgfqpoint{0.000000in}{0.000000in}}%
\pgfpathlineto{\pgfqpoint{5.000000in}{0.000000in}}%
\pgfpathlineto{\pgfqpoint{5.000000in}{4.000000in}}%
\pgfpathlineto{\pgfqpoint{0.000000in}{4.000000in}}%
\pgfpathclose%
\pgfusepath{fill}%
\end{pgfscope}%
\begin{pgfscope}%
\pgfsetbuttcap%
\pgfsetmiterjoin%
\definecolor{currentfill}{rgb}{1.000000,1.000000,1.000000}%
\pgfsetfillcolor{currentfill}%
\pgfsetlinewidth{0.000000pt}%
\definecolor{currentstroke}{rgb}{0.000000,0.000000,0.000000}%
\pgfsetstrokecolor{currentstroke}%
\pgfsetstrokeopacity{0.000000}%
\pgfsetdash{}{0pt}%
\pgfpathmoveto{\pgfqpoint{0.625000in}{0.440000in}}%
\pgfpathlineto{\pgfqpoint{4.500000in}{0.440000in}}%
\pgfpathlineto{\pgfqpoint{4.500000in}{3.520000in}}%
\pgfpathlineto{\pgfqpoint{0.625000in}{3.520000in}}%
\pgfpathclose%
\pgfusepath{fill}%
\end{pgfscope}%
\begin{pgfscope}%
\pgfsetbuttcap%
\pgfsetroundjoin%
\definecolor{currentfill}{rgb}{0.000000,0.000000,0.000000}%
\pgfsetfillcolor{currentfill}%
\pgfsetlinewidth{0.803000pt}%
\definecolor{currentstroke}{rgb}{0.000000,0.000000,0.000000}%
\pgfsetstrokecolor{currentstroke}%
\pgfsetdash{}{0pt}%
\pgfsys@defobject{currentmarker}{\pgfqpoint{0.000000in}{-0.048611in}}{\pgfqpoint{0.000000in}{0.000000in}}{%
\pgfpathmoveto{\pgfqpoint{0.000000in}{0.000000in}}%
\pgfpathlineto{\pgfqpoint{0.000000in}{-0.048611in}}%
\pgfusepath{stroke,fill}%
}%
\begin{pgfscope}%
\pgfsys@transformshift{0.801136in}{0.440000in}%
\pgfsys@useobject{currentmarker}{}%
\end{pgfscope}%
\end{pgfscope}%
\begin{pgfscope}%
\definecolor{textcolor}{rgb}{0.000000,0.000000,0.000000}%
\pgfsetstrokecolor{textcolor}%
\pgfsetfillcolor{textcolor}%
\pgftext[x=0.801136in,y=0.342778in,,top]{\color{textcolor}\sffamily\fontsize{10.000000}{12.000000}\selectfont \(\displaystyle {10^{1}}\)}%
\end{pgfscope}%
\begin{pgfscope}%
\pgfsetbuttcap%
\pgfsetroundjoin%
\definecolor{currentfill}{rgb}{0.000000,0.000000,0.000000}%
\pgfsetfillcolor{currentfill}%
\pgfsetlinewidth{0.602250pt}%
\definecolor{currentstroke}{rgb}{0.000000,0.000000,0.000000}%
\pgfsetstrokecolor{currentstroke}%
\pgfsetdash{}{0pt}%
\pgfsys@defobject{currentmarker}{\pgfqpoint{0.000000in}{-0.027778in}}{\pgfqpoint{0.000000in}{0.000000in}}{%
\pgfpathmoveto{\pgfqpoint{0.000000in}{0.000000in}}%
\pgfpathlineto{\pgfqpoint{0.000000in}{-0.027778in}}%
\pgfusepath{stroke,fill}%
}%
\begin{pgfscope}%
\pgfsys@transformshift{2.163913in}{0.440000in}%
\pgfsys@useobject{currentmarker}{}%
\end{pgfscope}%
\end{pgfscope}%
\begin{pgfscope}%
\definecolor{textcolor}{rgb}{0.000000,0.000000,0.000000}%
\pgfsetstrokecolor{textcolor}%
\pgfsetfillcolor{textcolor}%
\pgftext[x=2.163913in,y=0.365000in,,top]{\color{textcolor}\sffamily\fontsize{10.000000}{12.000000}\selectfont \(\displaystyle {2\times10^{1}}\)}%
\end{pgfscope}%
\begin{pgfscope}%
\pgfsetbuttcap%
\pgfsetroundjoin%
\definecolor{currentfill}{rgb}{0.000000,0.000000,0.000000}%
\pgfsetfillcolor{currentfill}%
\pgfsetlinewidth{0.602250pt}%
\definecolor{currentstroke}{rgb}{0.000000,0.000000,0.000000}%
\pgfsetstrokecolor{currentstroke}%
\pgfsetdash{}{0pt}%
\pgfsys@defobject{currentmarker}{\pgfqpoint{0.000000in}{-0.027778in}}{\pgfqpoint{0.000000in}{0.000000in}}{%
\pgfpathmoveto{\pgfqpoint{0.000000in}{0.000000in}}%
\pgfpathlineto{\pgfqpoint{0.000000in}{-0.027778in}}%
\pgfusepath{stroke,fill}%
}%
\begin{pgfscope}%
\pgfsys@transformshift{2.961087in}{0.440000in}%
\pgfsys@useobject{currentmarker}{}%
\end{pgfscope}%
\end{pgfscope}%
\begin{pgfscope}%
\definecolor{textcolor}{rgb}{0.000000,0.000000,0.000000}%
\pgfsetstrokecolor{textcolor}%
\pgfsetfillcolor{textcolor}%
\pgftext[x=2.961087in,y=0.365000in,,top]{\color{textcolor}\sffamily\fontsize{10.000000}{12.000000}\selectfont \(\displaystyle {3\times10^{1}}\)}%
\end{pgfscope}%
\begin{pgfscope}%
\pgfsetbuttcap%
\pgfsetroundjoin%
\definecolor{currentfill}{rgb}{0.000000,0.000000,0.000000}%
\pgfsetfillcolor{currentfill}%
\pgfsetlinewidth{0.602250pt}%
\definecolor{currentstroke}{rgb}{0.000000,0.000000,0.000000}%
\pgfsetstrokecolor{currentstroke}%
\pgfsetdash{}{0pt}%
\pgfsys@defobject{currentmarker}{\pgfqpoint{0.000000in}{-0.027778in}}{\pgfqpoint{0.000000in}{0.000000in}}{%
\pgfpathmoveto{\pgfqpoint{0.000000in}{0.000000in}}%
\pgfpathlineto{\pgfqpoint{0.000000in}{-0.027778in}}%
\pgfusepath{stroke,fill}%
}%
\begin{pgfscope}%
\pgfsys@transformshift{3.526690in}{0.440000in}%
\pgfsys@useobject{currentmarker}{}%
\end{pgfscope}%
\end{pgfscope}%
\begin{pgfscope}%
\definecolor{textcolor}{rgb}{0.000000,0.000000,0.000000}%
\pgfsetstrokecolor{textcolor}%
\pgfsetfillcolor{textcolor}%
\pgftext[x=3.526690in,y=0.365000in,,top]{\color{textcolor}\sffamily\fontsize{10.000000}{12.000000}\selectfont \(\displaystyle {4\times10^{1}}\)}%
\end{pgfscope}%
\begin{pgfscope}%
\pgfsetbuttcap%
\pgfsetroundjoin%
\definecolor{currentfill}{rgb}{0.000000,0.000000,0.000000}%
\pgfsetfillcolor{currentfill}%
\pgfsetlinewidth{0.602250pt}%
\definecolor{currentstroke}{rgb}{0.000000,0.000000,0.000000}%
\pgfsetstrokecolor{currentstroke}%
\pgfsetdash{}{0pt}%
\pgfsys@defobject{currentmarker}{\pgfqpoint{0.000000in}{-0.027778in}}{\pgfqpoint{0.000000in}{0.000000in}}{%
\pgfpathmoveto{\pgfqpoint{0.000000in}{0.000000in}}%
\pgfpathlineto{\pgfqpoint{0.000000in}{-0.027778in}}%
\pgfusepath{stroke,fill}%
}%
\begin{pgfscope}%
\pgfsys@transformshift{3.965406in}{0.440000in}%
\pgfsys@useobject{currentmarker}{}%
\end{pgfscope}%
\end{pgfscope}%
\begin{pgfscope}%
\pgfsetbuttcap%
\pgfsetroundjoin%
\definecolor{currentfill}{rgb}{0.000000,0.000000,0.000000}%
\pgfsetfillcolor{currentfill}%
\pgfsetlinewidth{0.602250pt}%
\definecolor{currentstroke}{rgb}{0.000000,0.000000,0.000000}%
\pgfsetstrokecolor{currentstroke}%
\pgfsetdash{}{0pt}%
\pgfsys@defobject{currentmarker}{\pgfqpoint{0.000000in}{-0.027778in}}{\pgfqpoint{0.000000in}{0.000000in}}{%
\pgfpathmoveto{\pgfqpoint{0.000000in}{0.000000in}}%
\pgfpathlineto{\pgfqpoint{0.000000in}{-0.027778in}}%
\pgfusepath{stroke,fill}%
}%
\begin{pgfscope}%
\pgfsys@transformshift{4.323864in}{0.440000in}%
\pgfsys@useobject{currentmarker}{}%
\end{pgfscope}%
\end{pgfscope}%
\begin{pgfscope}%
\definecolor{textcolor}{rgb}{0.000000,0.000000,0.000000}%
\pgfsetstrokecolor{textcolor}%
\pgfsetfillcolor{textcolor}%
\pgftext[x=4.323864in,y=0.365000in,,top]{\color{textcolor}\sffamily\fontsize{10.000000}{12.000000}\selectfont \(\displaystyle {6\times10^{1}}\)}%
\end{pgfscope}%
\begin{pgfscope}%
\definecolor{textcolor}{rgb}{0.000000,0.000000,0.000000}%
\pgfsetstrokecolor{textcolor}%
\pgfsetfillcolor{textcolor}%
\pgftext[x=2.562500in,y=0.152809in,,top]{\color{textcolor}\sffamily\fontsize{10.000000}{12.000000}\selectfont \(\displaystyle k\)}%
\end{pgfscope}%
\begin{pgfscope}%
\pgfsetbuttcap%
\pgfsetroundjoin%
\definecolor{currentfill}{rgb}{0.000000,0.000000,0.000000}%
\pgfsetfillcolor{currentfill}%
\pgfsetlinewidth{0.803000pt}%
\definecolor{currentstroke}{rgb}{0.000000,0.000000,0.000000}%
\pgfsetstrokecolor{currentstroke}%
\pgfsetdash{}{0pt}%
\pgfsys@defobject{currentmarker}{\pgfqpoint{-0.048611in}{0.000000in}}{\pgfqpoint{0.000000in}{0.000000in}}{%
\pgfpathmoveto{\pgfqpoint{0.000000in}{0.000000in}}%
\pgfpathlineto{\pgfqpoint{-0.048611in}{0.000000in}}%
\pgfusepath{stroke,fill}%
}%
\begin{pgfscope}%
\pgfsys@transformshift{0.625000in}{1.908201in}%
\pgfsys@useobject{currentmarker}{}%
\end{pgfscope}%
\end{pgfscope}%
\begin{pgfscope}%
\definecolor{textcolor}{rgb}{0.000000,0.000000,0.000000}%
\pgfsetstrokecolor{textcolor}%
\pgfsetfillcolor{textcolor}%
\pgftext[x=0.239775in,y=1.855439in,left,base]{\color{textcolor}\sffamily\fontsize{10.000000}{12.000000}\selectfont \(\displaystyle {10^{-3}}\)}%
\end{pgfscope}%
\begin{pgfscope}%
\pgfsetbuttcap%
\pgfsetroundjoin%
\definecolor{currentfill}{rgb}{0.000000,0.000000,0.000000}%
\pgfsetfillcolor{currentfill}%
\pgfsetlinewidth{0.602250pt}%
\definecolor{currentstroke}{rgb}{0.000000,0.000000,0.000000}%
\pgfsetstrokecolor{currentstroke}%
\pgfsetdash{}{0pt}%
\pgfsys@defobject{currentmarker}{\pgfqpoint{-0.027778in}{0.000000in}}{\pgfqpoint{0.000000in}{0.000000in}}{%
\pgfpathmoveto{\pgfqpoint{0.000000in}{0.000000in}}%
\pgfpathlineto{\pgfqpoint{-0.027778in}{0.000000in}}%
\pgfusepath{stroke,fill}%
}%
\begin{pgfscope}%
\pgfsys@transformshift{0.625000in}{0.494966in}%
\pgfsys@useobject{currentmarker}{}%
\end{pgfscope}%
\end{pgfscope}%
\begin{pgfscope}%
\pgfsetbuttcap%
\pgfsetroundjoin%
\definecolor{currentfill}{rgb}{0.000000,0.000000,0.000000}%
\pgfsetfillcolor{currentfill}%
\pgfsetlinewidth{0.602250pt}%
\definecolor{currentstroke}{rgb}{0.000000,0.000000,0.000000}%
\pgfsetstrokecolor{currentstroke}%
\pgfsetdash{}{0pt}%
\pgfsys@defobject{currentmarker}{\pgfqpoint{-0.027778in}{0.000000in}}{\pgfqpoint{0.000000in}{0.000000in}}{%
\pgfpathmoveto{\pgfqpoint{0.000000in}{0.000000in}}%
\pgfpathlineto{\pgfqpoint{-0.027778in}{0.000000in}}%
\pgfusepath{stroke,fill}%
}%
\begin{pgfscope}%
\pgfsys@transformshift{0.625000in}{0.851002in}%
\pgfsys@useobject{currentmarker}{}%
\end{pgfscope}%
\end{pgfscope}%
\begin{pgfscope}%
\pgfsetbuttcap%
\pgfsetroundjoin%
\definecolor{currentfill}{rgb}{0.000000,0.000000,0.000000}%
\pgfsetfillcolor{currentfill}%
\pgfsetlinewidth{0.602250pt}%
\definecolor{currentstroke}{rgb}{0.000000,0.000000,0.000000}%
\pgfsetstrokecolor{currentstroke}%
\pgfsetdash{}{0pt}%
\pgfsys@defobject{currentmarker}{\pgfqpoint{-0.027778in}{0.000000in}}{\pgfqpoint{0.000000in}{0.000000in}}{%
\pgfpathmoveto{\pgfqpoint{0.000000in}{0.000000in}}%
\pgfpathlineto{\pgfqpoint{-0.027778in}{0.000000in}}%
\pgfusepath{stroke,fill}%
}%
\begin{pgfscope}%
\pgfsys@transformshift{0.625000in}{1.103613in}%
\pgfsys@useobject{currentmarker}{}%
\end{pgfscope}%
\end{pgfscope}%
\begin{pgfscope}%
\pgfsetbuttcap%
\pgfsetroundjoin%
\definecolor{currentfill}{rgb}{0.000000,0.000000,0.000000}%
\pgfsetfillcolor{currentfill}%
\pgfsetlinewidth{0.602250pt}%
\definecolor{currentstroke}{rgb}{0.000000,0.000000,0.000000}%
\pgfsetstrokecolor{currentstroke}%
\pgfsetdash{}{0pt}%
\pgfsys@defobject{currentmarker}{\pgfqpoint{-0.027778in}{0.000000in}}{\pgfqpoint{0.000000in}{0.000000in}}{%
\pgfpathmoveto{\pgfqpoint{0.000000in}{0.000000in}}%
\pgfpathlineto{\pgfqpoint{-0.027778in}{0.000000in}}%
\pgfusepath{stroke,fill}%
}%
\begin{pgfscope}%
\pgfsys@transformshift{0.625000in}{1.299554in}%
\pgfsys@useobject{currentmarker}{}%
\end{pgfscope}%
\end{pgfscope}%
\begin{pgfscope}%
\pgfsetbuttcap%
\pgfsetroundjoin%
\definecolor{currentfill}{rgb}{0.000000,0.000000,0.000000}%
\pgfsetfillcolor{currentfill}%
\pgfsetlinewidth{0.602250pt}%
\definecolor{currentstroke}{rgb}{0.000000,0.000000,0.000000}%
\pgfsetstrokecolor{currentstroke}%
\pgfsetdash{}{0pt}%
\pgfsys@defobject{currentmarker}{\pgfqpoint{-0.027778in}{0.000000in}}{\pgfqpoint{0.000000in}{0.000000in}}{%
\pgfpathmoveto{\pgfqpoint{0.000000in}{0.000000in}}%
\pgfpathlineto{\pgfqpoint{-0.027778in}{0.000000in}}%
\pgfusepath{stroke,fill}%
}%
\begin{pgfscope}%
\pgfsys@transformshift{0.625000in}{1.459649in}%
\pgfsys@useobject{currentmarker}{}%
\end{pgfscope}%
\end{pgfscope}%
\begin{pgfscope}%
\pgfsetbuttcap%
\pgfsetroundjoin%
\definecolor{currentfill}{rgb}{0.000000,0.000000,0.000000}%
\pgfsetfillcolor{currentfill}%
\pgfsetlinewidth{0.602250pt}%
\definecolor{currentstroke}{rgb}{0.000000,0.000000,0.000000}%
\pgfsetstrokecolor{currentstroke}%
\pgfsetdash{}{0pt}%
\pgfsys@defobject{currentmarker}{\pgfqpoint{-0.027778in}{0.000000in}}{\pgfqpoint{0.000000in}{0.000000in}}{%
\pgfpathmoveto{\pgfqpoint{0.000000in}{0.000000in}}%
\pgfpathlineto{\pgfqpoint{-0.027778in}{0.000000in}}%
\pgfusepath{stroke,fill}%
}%
\begin{pgfscope}%
\pgfsys@transformshift{0.625000in}{1.595007in}%
\pgfsys@useobject{currentmarker}{}%
\end{pgfscope}%
\end{pgfscope}%
\begin{pgfscope}%
\pgfsetbuttcap%
\pgfsetroundjoin%
\definecolor{currentfill}{rgb}{0.000000,0.000000,0.000000}%
\pgfsetfillcolor{currentfill}%
\pgfsetlinewidth{0.602250pt}%
\definecolor{currentstroke}{rgb}{0.000000,0.000000,0.000000}%
\pgfsetstrokecolor{currentstroke}%
\pgfsetdash{}{0pt}%
\pgfsys@defobject{currentmarker}{\pgfqpoint{-0.027778in}{0.000000in}}{\pgfqpoint{0.000000in}{0.000000in}}{%
\pgfpathmoveto{\pgfqpoint{0.000000in}{0.000000in}}%
\pgfpathlineto{\pgfqpoint{-0.027778in}{0.000000in}}%
\pgfusepath{stroke,fill}%
}%
\begin{pgfscope}%
\pgfsys@transformshift{0.625000in}{1.712260in}%
\pgfsys@useobject{currentmarker}{}%
\end{pgfscope}%
\end{pgfscope}%
\begin{pgfscope}%
\pgfsetbuttcap%
\pgfsetroundjoin%
\definecolor{currentfill}{rgb}{0.000000,0.000000,0.000000}%
\pgfsetfillcolor{currentfill}%
\pgfsetlinewidth{0.602250pt}%
\definecolor{currentstroke}{rgb}{0.000000,0.000000,0.000000}%
\pgfsetstrokecolor{currentstroke}%
\pgfsetdash{}{0pt}%
\pgfsys@defobject{currentmarker}{\pgfqpoint{-0.027778in}{0.000000in}}{\pgfqpoint{0.000000in}{0.000000in}}{%
\pgfpathmoveto{\pgfqpoint{0.000000in}{0.000000in}}%
\pgfpathlineto{\pgfqpoint{-0.027778in}{0.000000in}}%
\pgfusepath{stroke,fill}%
}%
\begin{pgfscope}%
\pgfsys@transformshift{0.625000in}{1.815684in}%
\pgfsys@useobject{currentmarker}{}%
\end{pgfscope}%
\end{pgfscope}%
\begin{pgfscope}%
\pgfsetbuttcap%
\pgfsetroundjoin%
\definecolor{currentfill}{rgb}{0.000000,0.000000,0.000000}%
\pgfsetfillcolor{currentfill}%
\pgfsetlinewidth{0.602250pt}%
\definecolor{currentstroke}{rgb}{0.000000,0.000000,0.000000}%
\pgfsetstrokecolor{currentstroke}%
\pgfsetdash{}{0pt}%
\pgfsys@defobject{currentmarker}{\pgfqpoint{-0.027778in}{0.000000in}}{\pgfqpoint{0.000000in}{0.000000in}}{%
\pgfpathmoveto{\pgfqpoint{0.000000in}{0.000000in}}%
\pgfpathlineto{\pgfqpoint{-0.027778in}{0.000000in}}%
\pgfusepath{stroke,fill}%
}%
\begin{pgfscope}%
\pgfsys@transformshift{0.625000in}{2.516848in}%
\pgfsys@useobject{currentmarker}{}%
\end{pgfscope}%
\end{pgfscope}%
\begin{pgfscope}%
\pgfsetbuttcap%
\pgfsetroundjoin%
\definecolor{currentfill}{rgb}{0.000000,0.000000,0.000000}%
\pgfsetfillcolor{currentfill}%
\pgfsetlinewidth{0.602250pt}%
\definecolor{currentstroke}{rgb}{0.000000,0.000000,0.000000}%
\pgfsetstrokecolor{currentstroke}%
\pgfsetdash{}{0pt}%
\pgfsys@defobject{currentmarker}{\pgfqpoint{-0.027778in}{0.000000in}}{\pgfqpoint{0.000000in}{0.000000in}}{%
\pgfpathmoveto{\pgfqpoint{0.000000in}{0.000000in}}%
\pgfpathlineto{\pgfqpoint{-0.027778in}{0.000000in}}%
\pgfusepath{stroke,fill}%
}%
\begin{pgfscope}%
\pgfsys@transformshift{0.625000in}{2.872884in}%
\pgfsys@useobject{currentmarker}{}%
\end{pgfscope}%
\end{pgfscope}%
\begin{pgfscope}%
\pgfsetbuttcap%
\pgfsetroundjoin%
\definecolor{currentfill}{rgb}{0.000000,0.000000,0.000000}%
\pgfsetfillcolor{currentfill}%
\pgfsetlinewidth{0.602250pt}%
\definecolor{currentstroke}{rgb}{0.000000,0.000000,0.000000}%
\pgfsetstrokecolor{currentstroke}%
\pgfsetdash{}{0pt}%
\pgfsys@defobject{currentmarker}{\pgfqpoint{-0.027778in}{0.000000in}}{\pgfqpoint{0.000000in}{0.000000in}}{%
\pgfpathmoveto{\pgfqpoint{0.000000in}{0.000000in}}%
\pgfpathlineto{\pgfqpoint{-0.027778in}{0.000000in}}%
\pgfusepath{stroke,fill}%
}%
\begin{pgfscope}%
\pgfsys@transformshift{0.625000in}{3.125495in}%
\pgfsys@useobject{currentmarker}{}%
\end{pgfscope}%
\end{pgfscope}%
\begin{pgfscope}%
\pgfsetbuttcap%
\pgfsetroundjoin%
\definecolor{currentfill}{rgb}{0.000000,0.000000,0.000000}%
\pgfsetfillcolor{currentfill}%
\pgfsetlinewidth{0.602250pt}%
\definecolor{currentstroke}{rgb}{0.000000,0.000000,0.000000}%
\pgfsetstrokecolor{currentstroke}%
\pgfsetdash{}{0pt}%
\pgfsys@defobject{currentmarker}{\pgfqpoint{-0.027778in}{0.000000in}}{\pgfqpoint{0.000000in}{0.000000in}}{%
\pgfpathmoveto{\pgfqpoint{0.000000in}{0.000000in}}%
\pgfpathlineto{\pgfqpoint{-0.027778in}{0.000000in}}%
\pgfusepath{stroke,fill}%
}%
\begin{pgfscope}%
\pgfsys@transformshift{0.625000in}{3.321436in}%
\pgfsys@useobject{currentmarker}{}%
\end{pgfscope}%
\end{pgfscope}%
\begin{pgfscope}%
\pgfsetbuttcap%
\pgfsetroundjoin%
\definecolor{currentfill}{rgb}{0.000000,0.000000,0.000000}%
\pgfsetfillcolor{currentfill}%
\pgfsetlinewidth{0.602250pt}%
\definecolor{currentstroke}{rgb}{0.000000,0.000000,0.000000}%
\pgfsetstrokecolor{currentstroke}%
\pgfsetdash{}{0pt}%
\pgfsys@defobject{currentmarker}{\pgfqpoint{-0.027778in}{0.000000in}}{\pgfqpoint{0.000000in}{0.000000in}}{%
\pgfpathmoveto{\pgfqpoint{0.000000in}{0.000000in}}%
\pgfpathlineto{\pgfqpoint{-0.027778in}{0.000000in}}%
\pgfusepath{stroke,fill}%
}%
\begin{pgfscope}%
\pgfsys@transformshift{0.625000in}{3.481531in}%
\pgfsys@useobject{currentmarker}{}%
\end{pgfscope}%
\end{pgfscope}%
\begin{pgfscope}%
\definecolor{textcolor}{rgb}{0.000000,0.000000,0.000000}%
\pgfsetstrokecolor{textcolor}%
\pgfsetfillcolor{textcolor}%
\pgftext[x=0.184220in,y=1.980000in,,bottom,rotate=90.000000]{\color{textcolor}\sffamily\fontsize{10.000000}{12.000000}\selectfont \(\displaystyle C\)}%
\end{pgfscope}%
\begin{pgfscope}%
\pgfpathrectangle{\pgfqpoint{0.625000in}{0.440000in}}{\pgfqpoint{3.875000in}{3.080000in}}%
\pgfusepath{clip}%
\pgfsetbuttcap%
\pgfsetroundjoin%
\definecolor{currentfill}{rgb}{0.000000,0.000000,0.000000}%
\pgfsetfillcolor{currentfill}%
\pgfsetlinewidth{1.003750pt}%
\definecolor{currentstroke}{rgb}{0.000000,0.000000,0.000000}%
\pgfsetstrokecolor{currentstroke}%
\pgfsetdash{}{0pt}%
\pgfsys@defobject{currentmarker}{\pgfqpoint{-0.041667in}{-0.041667in}}{\pgfqpoint{0.041667in}{0.041667in}}{%
\pgfpathmoveto{\pgfqpoint{0.000000in}{-0.041667in}}%
\pgfpathcurveto{\pgfqpoint{0.011050in}{-0.041667in}}{\pgfqpoint{0.021649in}{-0.037276in}}{\pgfqpoint{0.029463in}{-0.029463in}}%
\pgfpathcurveto{\pgfqpoint{0.037276in}{-0.021649in}}{\pgfqpoint{0.041667in}{-0.011050in}}{\pgfqpoint{0.041667in}{0.000000in}}%
\pgfpathcurveto{\pgfqpoint{0.041667in}{0.011050in}}{\pgfqpoint{0.037276in}{0.021649in}}{\pgfqpoint{0.029463in}{0.029463in}}%
\pgfpathcurveto{\pgfqpoint{0.021649in}{0.037276in}}{\pgfqpoint{0.011050in}{0.041667in}}{\pgfqpoint{0.000000in}{0.041667in}}%
\pgfpathcurveto{\pgfqpoint{-0.011050in}{0.041667in}}{\pgfqpoint{-0.021649in}{0.037276in}}{\pgfqpoint{-0.029463in}{0.029463in}}%
\pgfpathcurveto{\pgfqpoint{-0.037276in}{0.021649in}}{\pgfqpoint{-0.041667in}{0.011050in}}{\pgfqpoint{-0.041667in}{0.000000in}}%
\pgfpathcurveto{\pgfqpoint{-0.041667in}{-0.011050in}}{\pgfqpoint{-0.037276in}{-0.021649in}}{\pgfqpoint{-0.029463in}{-0.029463in}}%
\pgfpathcurveto{\pgfqpoint{-0.021649in}{-0.037276in}}{\pgfqpoint{-0.011050in}{-0.041667in}}{\pgfqpoint{0.000000in}{-0.041667in}}%
\pgfpathclose%
\pgfusepath{stroke,fill}%
}%
\begin{pgfscope}%
\pgfsys@transformshift{0.801136in}{3.380000in}%
\pgfsys@useobject{currentmarker}{}%
\end{pgfscope}%
\begin{pgfscope}%
\pgfsys@transformshift{2.163913in}{2.625858in}%
\pgfsys@useobject{currentmarker}{}%
\end{pgfscope}%
\begin{pgfscope}%
\pgfsys@transformshift{2.961087in}{2.076879in}%
\pgfsys@useobject{currentmarker}{}%
\end{pgfscope}%
\begin{pgfscope}%
\pgfsys@transformshift{3.526690in}{1.198547in}%
\pgfsys@useobject{currentmarker}{}%
\end{pgfscope}%
\begin{pgfscope}%
\pgfsys@transformshift{3.965406in}{0.972270in}%
\pgfsys@useobject{currentmarker}{}%
\end{pgfscope}%
\begin{pgfscope}%
\pgfsys@transformshift{4.323864in}{0.580000in}%
\pgfsys@useobject{currentmarker}{}%
\end{pgfscope}%
\end{pgfscope}%
\begin{pgfscope}%
\pgfsetrectcap%
\pgfsetmiterjoin%
\pgfsetlinewidth{0.803000pt}%
\definecolor{currentstroke}{rgb}{0.000000,0.000000,0.000000}%
\pgfsetstrokecolor{currentstroke}%
\pgfsetdash{}{0pt}%
\pgfpathmoveto{\pgfqpoint{0.625000in}{0.440000in}}%
\pgfpathlineto{\pgfqpoint{0.625000in}{3.520000in}}%
\pgfusepath{stroke}%
\end{pgfscope}%
\begin{pgfscope}%
\pgfsetrectcap%
\pgfsetmiterjoin%
\pgfsetlinewidth{0.803000pt}%
\definecolor{currentstroke}{rgb}{0.000000,0.000000,0.000000}%
\pgfsetstrokecolor{currentstroke}%
\pgfsetdash{}{0pt}%
\pgfpathmoveto{\pgfqpoint{4.500000in}{0.440000in}}%
\pgfpathlineto{\pgfqpoint{4.500000in}{3.520000in}}%
\pgfusepath{stroke}%
\end{pgfscope}%
\begin{pgfscope}%
\pgfsetrectcap%
\pgfsetmiterjoin%
\pgfsetlinewidth{0.803000pt}%
\definecolor{currentstroke}{rgb}{0.000000,0.000000,0.000000}%
\pgfsetstrokecolor{currentstroke}%
\pgfsetdash{}{0pt}%
\pgfpathmoveto{\pgfqpoint{0.625000in}{0.440000in}}%
\pgfpathlineto{\pgfqpoint{4.500000in}{0.440000in}}%
\pgfusepath{stroke}%
\end{pgfscope}%
\begin{pgfscope}%
\pgfsetrectcap%
\pgfsetmiterjoin%
\pgfsetlinewidth{0.803000pt}%
\definecolor{currentstroke}{rgb}{0.000000,0.000000,0.000000}%
\pgfsetstrokecolor{currentstroke}%
\pgfsetdash{}{0pt}%
\pgfpathmoveto{\pgfqpoint{0.625000in}{3.520000in}}%
\pgfpathlineto{\pgfqpoint{4.500000in}{3.520000in}}%
\pgfusepath{stroke}%
\end{pgfscope}%
\end{pgfpicture}%
\makeatother%
\endgroup%

    \end{subfigure}
    \begin{subfigure}{\textwidth}
                \centering
      %% Creator: Matplotlib, PGF backend
%%
%% To include the figure in your LaTeX document, write
%%   \input{<filename>.pgf}
%%
%% Make sure the required packages are loaded in your preamble
%%   \usepackage{pgf}
%%
%% Figures using additional raster images can only be included by \input if
%% they are in the same directory as the main LaTeX file. For loading figures
%% from other directories you can use the `import` package
%%   \usepackage{import}
%% and then include the figures with
%%   \import{<path to file>}{<filename>.pgf}
%%
%% Matplotlib used the following preamble
%%   \usepackage{fontspec}
%%   \setmainfont{DejaVuSerif.ttf}[Path=/home/owen/progs/firedrake-complex/firedrake/lib/python3.5/site-packages/matplotlib/mpl-data/fonts/ttf/]
%%   \setsansfont{DejaVuSans.ttf}[Path=/home/owen/progs/firedrake-complex/firedrake/lib/python3.5/site-packages/matplotlib/mpl-data/fonts/ttf/]
%%   \setmonofont{DejaVuSansMono.ttf}[Path=/home/owen/progs/firedrake-complex/firedrake/lib/python3.5/site-packages/matplotlib/mpl-data/fonts/ttf/]
%%
\begingroup%
\makeatletter%
\begin{pgfpicture}%
\pgfpathrectangle{\pgfpointorigin}{\pgfqpoint{5.000000in}{4.000000in}}%
\pgfusepath{use as bounding box, clip}%
\begin{pgfscope}%
\pgfsetbuttcap%
\pgfsetmiterjoin%
\definecolor{currentfill}{rgb}{1.000000,1.000000,1.000000}%
\pgfsetfillcolor{currentfill}%
\pgfsetlinewidth{0.000000pt}%
\definecolor{currentstroke}{rgb}{1.000000,1.000000,1.000000}%
\pgfsetstrokecolor{currentstroke}%
\pgfsetdash{}{0pt}%
\pgfpathmoveto{\pgfqpoint{0.000000in}{0.000000in}}%
\pgfpathlineto{\pgfqpoint{5.000000in}{0.000000in}}%
\pgfpathlineto{\pgfqpoint{5.000000in}{4.000000in}}%
\pgfpathlineto{\pgfqpoint{0.000000in}{4.000000in}}%
\pgfpathclose%
\pgfusepath{fill}%
\end{pgfscope}%
\begin{pgfscope}%
\pgfsetbuttcap%
\pgfsetmiterjoin%
\definecolor{currentfill}{rgb}{1.000000,1.000000,1.000000}%
\pgfsetfillcolor{currentfill}%
\pgfsetlinewidth{0.000000pt}%
\definecolor{currentstroke}{rgb}{0.000000,0.000000,0.000000}%
\pgfsetstrokecolor{currentstroke}%
\pgfsetstrokeopacity{0.000000}%
\pgfsetdash{}{0pt}%
\pgfpathmoveto{\pgfqpoint{0.625000in}{0.440000in}}%
\pgfpathlineto{\pgfqpoint{4.500000in}{0.440000in}}%
\pgfpathlineto{\pgfqpoint{4.500000in}{3.520000in}}%
\pgfpathlineto{\pgfqpoint{0.625000in}{3.520000in}}%
\pgfpathclose%
\pgfusepath{fill}%
\end{pgfscope}%
\begin{pgfscope}%
\pgfsetbuttcap%
\pgfsetroundjoin%
\definecolor{currentfill}{rgb}{0.000000,0.000000,0.000000}%
\pgfsetfillcolor{currentfill}%
\pgfsetlinewidth{0.803000pt}%
\definecolor{currentstroke}{rgb}{0.000000,0.000000,0.000000}%
\pgfsetstrokecolor{currentstroke}%
\pgfsetdash{}{0pt}%
\pgfsys@defobject{currentmarker}{\pgfqpoint{0.000000in}{-0.048611in}}{\pgfqpoint{0.000000in}{0.000000in}}{%
\pgfpathmoveto{\pgfqpoint{0.000000in}{0.000000in}}%
\pgfpathlineto{\pgfqpoint{0.000000in}{-0.048611in}}%
\pgfusepath{stroke,fill}%
}%
\begin{pgfscope}%
\pgfsys@transformshift{0.801136in}{0.440000in}%
\pgfsys@useobject{currentmarker}{}%
\end{pgfscope}%
\end{pgfscope}%
\begin{pgfscope}%
\definecolor{textcolor}{rgb}{0.000000,0.000000,0.000000}%
\pgfsetstrokecolor{textcolor}%
\pgfsetfillcolor{textcolor}%
\pgftext[x=0.801136in,y=0.342778in,,top]{\color{textcolor}\sffamily\fontsize{10.000000}{12.000000}\selectfont \(\displaystyle 10^{1}\)}%
\end{pgfscope}%
\begin{pgfscope}%
\pgfsetbuttcap%
\pgfsetroundjoin%
\definecolor{currentfill}{rgb}{0.000000,0.000000,0.000000}%
\pgfsetfillcolor{currentfill}%
\pgfsetlinewidth{0.602250pt}%
\definecolor{currentstroke}{rgb}{0.000000,0.000000,0.000000}%
\pgfsetstrokecolor{currentstroke}%
\pgfsetdash{}{0pt}%
\pgfsys@defobject{currentmarker}{\pgfqpoint{0.000000in}{-0.027778in}}{\pgfqpoint{0.000000in}{0.000000in}}{%
\pgfpathmoveto{\pgfqpoint{0.000000in}{0.000000in}}%
\pgfpathlineto{\pgfqpoint{0.000000in}{-0.027778in}}%
\pgfusepath{stroke,fill}%
}%
\begin{pgfscope}%
\pgfsys@transformshift{2.163913in}{0.440000in}%
\pgfsys@useobject{currentmarker}{}%
\end{pgfscope}%
\end{pgfscope}%
\begin{pgfscope}%
\definecolor{textcolor}{rgb}{0.000000,0.000000,0.000000}%
\pgfsetstrokecolor{textcolor}%
\pgfsetfillcolor{textcolor}%
\pgftext[x=2.163913in,y=0.365000in,,top]{\color{textcolor}\sffamily\fontsize{10.000000}{12.000000}\selectfont \(\displaystyle 2\times10^{1}\)}%
\end{pgfscope}%
\begin{pgfscope}%
\pgfsetbuttcap%
\pgfsetroundjoin%
\definecolor{currentfill}{rgb}{0.000000,0.000000,0.000000}%
\pgfsetfillcolor{currentfill}%
\pgfsetlinewidth{0.602250pt}%
\definecolor{currentstroke}{rgb}{0.000000,0.000000,0.000000}%
\pgfsetstrokecolor{currentstroke}%
\pgfsetdash{}{0pt}%
\pgfsys@defobject{currentmarker}{\pgfqpoint{0.000000in}{-0.027778in}}{\pgfqpoint{0.000000in}{0.000000in}}{%
\pgfpathmoveto{\pgfqpoint{0.000000in}{0.000000in}}%
\pgfpathlineto{\pgfqpoint{0.000000in}{-0.027778in}}%
\pgfusepath{stroke,fill}%
}%
\begin{pgfscope}%
\pgfsys@transformshift{2.961087in}{0.440000in}%
\pgfsys@useobject{currentmarker}{}%
\end{pgfscope}%
\end{pgfscope}%
\begin{pgfscope}%
\definecolor{textcolor}{rgb}{0.000000,0.000000,0.000000}%
\pgfsetstrokecolor{textcolor}%
\pgfsetfillcolor{textcolor}%
\pgftext[x=2.961087in,y=0.365000in,,top]{\color{textcolor}\sffamily\fontsize{10.000000}{12.000000}\selectfont \(\displaystyle 3\times10^{1}\)}%
\end{pgfscope}%
\begin{pgfscope}%
\pgfsetbuttcap%
\pgfsetroundjoin%
\definecolor{currentfill}{rgb}{0.000000,0.000000,0.000000}%
\pgfsetfillcolor{currentfill}%
\pgfsetlinewidth{0.602250pt}%
\definecolor{currentstroke}{rgb}{0.000000,0.000000,0.000000}%
\pgfsetstrokecolor{currentstroke}%
\pgfsetdash{}{0pt}%
\pgfsys@defobject{currentmarker}{\pgfqpoint{0.000000in}{-0.027778in}}{\pgfqpoint{0.000000in}{0.000000in}}{%
\pgfpathmoveto{\pgfqpoint{0.000000in}{0.000000in}}%
\pgfpathlineto{\pgfqpoint{0.000000in}{-0.027778in}}%
\pgfusepath{stroke,fill}%
}%
\begin{pgfscope}%
\pgfsys@transformshift{3.526690in}{0.440000in}%
\pgfsys@useobject{currentmarker}{}%
\end{pgfscope}%
\end{pgfscope}%
\begin{pgfscope}%
\definecolor{textcolor}{rgb}{0.000000,0.000000,0.000000}%
\pgfsetstrokecolor{textcolor}%
\pgfsetfillcolor{textcolor}%
\pgftext[x=3.526690in,y=0.365000in,,top]{\color{textcolor}\sffamily\fontsize{10.000000}{12.000000}\selectfont \(\displaystyle 4\times10^{1}\)}%
\end{pgfscope}%
\begin{pgfscope}%
\pgfsetbuttcap%
\pgfsetroundjoin%
\definecolor{currentfill}{rgb}{0.000000,0.000000,0.000000}%
\pgfsetfillcolor{currentfill}%
\pgfsetlinewidth{0.602250pt}%
\definecolor{currentstroke}{rgb}{0.000000,0.000000,0.000000}%
\pgfsetstrokecolor{currentstroke}%
\pgfsetdash{}{0pt}%
\pgfsys@defobject{currentmarker}{\pgfqpoint{0.000000in}{-0.027778in}}{\pgfqpoint{0.000000in}{0.000000in}}{%
\pgfpathmoveto{\pgfqpoint{0.000000in}{0.000000in}}%
\pgfpathlineto{\pgfqpoint{0.000000in}{-0.027778in}}%
\pgfusepath{stroke,fill}%
}%
\begin{pgfscope}%
\pgfsys@transformshift{3.965406in}{0.440000in}%
\pgfsys@useobject{currentmarker}{}%
\end{pgfscope}%
\end{pgfscope}%
\begin{pgfscope}%
\pgfsetbuttcap%
\pgfsetroundjoin%
\definecolor{currentfill}{rgb}{0.000000,0.000000,0.000000}%
\pgfsetfillcolor{currentfill}%
\pgfsetlinewidth{0.602250pt}%
\definecolor{currentstroke}{rgb}{0.000000,0.000000,0.000000}%
\pgfsetstrokecolor{currentstroke}%
\pgfsetdash{}{0pt}%
\pgfsys@defobject{currentmarker}{\pgfqpoint{0.000000in}{-0.027778in}}{\pgfqpoint{0.000000in}{0.000000in}}{%
\pgfpathmoveto{\pgfqpoint{0.000000in}{0.000000in}}%
\pgfpathlineto{\pgfqpoint{0.000000in}{-0.027778in}}%
\pgfusepath{stroke,fill}%
}%
\begin{pgfscope}%
\pgfsys@transformshift{4.323864in}{0.440000in}%
\pgfsys@useobject{currentmarker}{}%
\end{pgfscope}%
\end{pgfscope}%
\begin{pgfscope}%
\definecolor{textcolor}{rgb}{0.000000,0.000000,0.000000}%
\pgfsetstrokecolor{textcolor}%
\pgfsetfillcolor{textcolor}%
\pgftext[x=4.323864in,y=0.365000in,,top]{\color{textcolor}\sffamily\fontsize{10.000000}{12.000000}\selectfont \(\displaystyle 6\times10^{1}\)}%
\end{pgfscope}%
\begin{pgfscope}%
\definecolor{textcolor}{rgb}{0.000000,0.000000,0.000000}%
\pgfsetstrokecolor{textcolor}%
\pgfsetfillcolor{textcolor}%
\pgftext[x=2.562500in,y=0.152809in,,top]{\color{textcolor}\sffamily\fontsize{10.000000}{12.000000}\selectfont \(\displaystyle k\)}%
\end{pgfscope}%
\begin{pgfscope}%
\pgfsetbuttcap%
\pgfsetroundjoin%
\definecolor{currentfill}{rgb}{0.000000,0.000000,0.000000}%
\pgfsetfillcolor{currentfill}%
\pgfsetlinewidth{0.803000pt}%
\definecolor{currentstroke}{rgb}{0.000000,0.000000,0.000000}%
\pgfsetstrokecolor{currentstroke}%
\pgfsetdash{}{0pt}%
\pgfsys@defobject{currentmarker}{\pgfqpoint{-0.048611in}{0.000000in}}{\pgfqpoint{0.000000in}{0.000000in}}{%
\pgfpathmoveto{\pgfqpoint{0.000000in}{0.000000in}}%
\pgfpathlineto{\pgfqpoint{-0.048611in}{0.000000in}}%
\pgfusepath{stroke,fill}%
}%
\begin{pgfscope}%
\pgfsys@transformshift{0.625000in}{0.623639in}%
\pgfsys@useobject{currentmarker}{}%
\end{pgfscope}%
\end{pgfscope}%
\begin{pgfscope}%
\definecolor{textcolor}{rgb}{0.000000,0.000000,0.000000}%
\pgfsetstrokecolor{textcolor}%
\pgfsetfillcolor{textcolor}%
\pgftext[x=0.280863in,y=0.570877in,left,base]{\color{textcolor}\sffamily\fontsize{10.000000}{12.000000}\selectfont \(\displaystyle 0.60\)}%
\end{pgfscope}%
\begin{pgfscope}%
\pgfsetbuttcap%
\pgfsetroundjoin%
\definecolor{currentfill}{rgb}{0.000000,0.000000,0.000000}%
\pgfsetfillcolor{currentfill}%
\pgfsetlinewidth{0.803000pt}%
\definecolor{currentstroke}{rgb}{0.000000,0.000000,0.000000}%
\pgfsetstrokecolor{currentstroke}%
\pgfsetdash{}{0pt}%
\pgfsys@defobject{currentmarker}{\pgfqpoint{-0.048611in}{0.000000in}}{\pgfqpoint{0.000000in}{0.000000in}}{%
\pgfpathmoveto{\pgfqpoint{0.000000in}{0.000000in}}%
\pgfpathlineto{\pgfqpoint{-0.048611in}{0.000000in}}%
\pgfusepath{stroke,fill}%
}%
\begin{pgfscope}%
\pgfsys@transformshift{0.625000in}{0.990129in}%
\pgfsys@useobject{currentmarker}{}%
\end{pgfscope}%
\end{pgfscope}%
\begin{pgfscope}%
\definecolor{textcolor}{rgb}{0.000000,0.000000,0.000000}%
\pgfsetstrokecolor{textcolor}%
\pgfsetfillcolor{textcolor}%
\pgftext[x=0.280863in,y=0.937367in,left,base]{\color{textcolor}\sffamily\fontsize{10.000000}{12.000000}\selectfont \(\displaystyle 0.65\)}%
\end{pgfscope}%
\begin{pgfscope}%
\pgfsetbuttcap%
\pgfsetroundjoin%
\definecolor{currentfill}{rgb}{0.000000,0.000000,0.000000}%
\pgfsetfillcolor{currentfill}%
\pgfsetlinewidth{0.803000pt}%
\definecolor{currentstroke}{rgb}{0.000000,0.000000,0.000000}%
\pgfsetstrokecolor{currentstroke}%
\pgfsetdash{}{0pt}%
\pgfsys@defobject{currentmarker}{\pgfqpoint{-0.048611in}{0.000000in}}{\pgfqpoint{0.000000in}{0.000000in}}{%
\pgfpathmoveto{\pgfqpoint{0.000000in}{0.000000in}}%
\pgfpathlineto{\pgfqpoint{-0.048611in}{0.000000in}}%
\pgfusepath{stroke,fill}%
}%
\begin{pgfscope}%
\pgfsys@transformshift{0.625000in}{1.356618in}%
\pgfsys@useobject{currentmarker}{}%
\end{pgfscope}%
\end{pgfscope}%
\begin{pgfscope}%
\definecolor{textcolor}{rgb}{0.000000,0.000000,0.000000}%
\pgfsetstrokecolor{textcolor}%
\pgfsetfillcolor{textcolor}%
\pgftext[x=0.280863in,y=1.303857in,left,base]{\color{textcolor}\sffamily\fontsize{10.000000}{12.000000}\selectfont \(\displaystyle 0.70\)}%
\end{pgfscope}%
\begin{pgfscope}%
\pgfsetbuttcap%
\pgfsetroundjoin%
\definecolor{currentfill}{rgb}{0.000000,0.000000,0.000000}%
\pgfsetfillcolor{currentfill}%
\pgfsetlinewidth{0.803000pt}%
\definecolor{currentstroke}{rgb}{0.000000,0.000000,0.000000}%
\pgfsetstrokecolor{currentstroke}%
\pgfsetdash{}{0pt}%
\pgfsys@defobject{currentmarker}{\pgfqpoint{-0.048611in}{0.000000in}}{\pgfqpoint{0.000000in}{0.000000in}}{%
\pgfpathmoveto{\pgfqpoint{0.000000in}{0.000000in}}%
\pgfpathlineto{\pgfqpoint{-0.048611in}{0.000000in}}%
\pgfusepath{stroke,fill}%
}%
\begin{pgfscope}%
\pgfsys@transformshift{0.625000in}{1.723108in}%
\pgfsys@useobject{currentmarker}{}%
\end{pgfscope}%
\end{pgfscope}%
\begin{pgfscope}%
\definecolor{textcolor}{rgb}{0.000000,0.000000,0.000000}%
\pgfsetstrokecolor{textcolor}%
\pgfsetfillcolor{textcolor}%
\pgftext[x=0.280863in,y=1.670347in,left,base]{\color{textcolor}\sffamily\fontsize{10.000000}{12.000000}\selectfont \(\displaystyle 0.75\)}%
\end{pgfscope}%
\begin{pgfscope}%
\pgfsetbuttcap%
\pgfsetroundjoin%
\definecolor{currentfill}{rgb}{0.000000,0.000000,0.000000}%
\pgfsetfillcolor{currentfill}%
\pgfsetlinewidth{0.803000pt}%
\definecolor{currentstroke}{rgb}{0.000000,0.000000,0.000000}%
\pgfsetstrokecolor{currentstroke}%
\pgfsetdash{}{0pt}%
\pgfsys@defobject{currentmarker}{\pgfqpoint{-0.048611in}{0.000000in}}{\pgfqpoint{0.000000in}{0.000000in}}{%
\pgfpathmoveto{\pgfqpoint{0.000000in}{0.000000in}}%
\pgfpathlineto{\pgfqpoint{-0.048611in}{0.000000in}}%
\pgfusepath{stroke,fill}%
}%
\begin{pgfscope}%
\pgfsys@transformshift{0.625000in}{2.089598in}%
\pgfsys@useobject{currentmarker}{}%
\end{pgfscope}%
\end{pgfscope}%
\begin{pgfscope}%
\definecolor{textcolor}{rgb}{0.000000,0.000000,0.000000}%
\pgfsetstrokecolor{textcolor}%
\pgfsetfillcolor{textcolor}%
\pgftext[x=0.280863in,y=2.036837in,left,base]{\color{textcolor}\sffamily\fontsize{10.000000}{12.000000}\selectfont \(\displaystyle 0.80\)}%
\end{pgfscope}%
\begin{pgfscope}%
\pgfsetbuttcap%
\pgfsetroundjoin%
\definecolor{currentfill}{rgb}{0.000000,0.000000,0.000000}%
\pgfsetfillcolor{currentfill}%
\pgfsetlinewidth{0.803000pt}%
\definecolor{currentstroke}{rgb}{0.000000,0.000000,0.000000}%
\pgfsetstrokecolor{currentstroke}%
\pgfsetdash{}{0pt}%
\pgfsys@defobject{currentmarker}{\pgfqpoint{-0.048611in}{0.000000in}}{\pgfqpoint{0.000000in}{0.000000in}}{%
\pgfpathmoveto{\pgfqpoint{0.000000in}{0.000000in}}%
\pgfpathlineto{\pgfqpoint{-0.048611in}{0.000000in}}%
\pgfusepath{stroke,fill}%
}%
\begin{pgfscope}%
\pgfsys@transformshift{0.625000in}{2.456088in}%
\pgfsys@useobject{currentmarker}{}%
\end{pgfscope}%
\end{pgfscope}%
\begin{pgfscope}%
\definecolor{textcolor}{rgb}{0.000000,0.000000,0.000000}%
\pgfsetstrokecolor{textcolor}%
\pgfsetfillcolor{textcolor}%
\pgftext[x=0.280863in,y=2.403327in,left,base]{\color{textcolor}\sffamily\fontsize{10.000000}{12.000000}\selectfont \(\displaystyle 0.85\)}%
\end{pgfscope}%
\begin{pgfscope}%
\pgfsetbuttcap%
\pgfsetroundjoin%
\definecolor{currentfill}{rgb}{0.000000,0.000000,0.000000}%
\pgfsetfillcolor{currentfill}%
\pgfsetlinewidth{0.803000pt}%
\definecolor{currentstroke}{rgb}{0.000000,0.000000,0.000000}%
\pgfsetstrokecolor{currentstroke}%
\pgfsetdash{}{0pt}%
\pgfsys@defobject{currentmarker}{\pgfqpoint{-0.048611in}{0.000000in}}{\pgfqpoint{0.000000in}{0.000000in}}{%
\pgfpathmoveto{\pgfqpoint{0.000000in}{0.000000in}}%
\pgfpathlineto{\pgfqpoint{-0.048611in}{0.000000in}}%
\pgfusepath{stroke,fill}%
}%
\begin{pgfscope}%
\pgfsys@transformshift{0.625000in}{2.822578in}%
\pgfsys@useobject{currentmarker}{}%
\end{pgfscope}%
\end{pgfscope}%
\begin{pgfscope}%
\definecolor{textcolor}{rgb}{0.000000,0.000000,0.000000}%
\pgfsetstrokecolor{textcolor}%
\pgfsetfillcolor{textcolor}%
\pgftext[x=0.280863in,y=2.769817in,left,base]{\color{textcolor}\sffamily\fontsize{10.000000}{12.000000}\selectfont \(\displaystyle 0.90\)}%
\end{pgfscope}%
\begin{pgfscope}%
\pgfsetbuttcap%
\pgfsetroundjoin%
\definecolor{currentfill}{rgb}{0.000000,0.000000,0.000000}%
\pgfsetfillcolor{currentfill}%
\pgfsetlinewidth{0.803000pt}%
\definecolor{currentstroke}{rgb}{0.000000,0.000000,0.000000}%
\pgfsetstrokecolor{currentstroke}%
\pgfsetdash{}{0pt}%
\pgfsys@defobject{currentmarker}{\pgfqpoint{-0.048611in}{0.000000in}}{\pgfqpoint{0.000000in}{0.000000in}}{%
\pgfpathmoveto{\pgfqpoint{0.000000in}{0.000000in}}%
\pgfpathlineto{\pgfqpoint{-0.048611in}{0.000000in}}%
\pgfusepath{stroke,fill}%
}%
\begin{pgfscope}%
\pgfsys@transformshift{0.625000in}{3.189068in}%
\pgfsys@useobject{currentmarker}{}%
\end{pgfscope}%
\end{pgfscope}%
\begin{pgfscope}%
\definecolor{textcolor}{rgb}{0.000000,0.000000,0.000000}%
\pgfsetstrokecolor{textcolor}%
\pgfsetfillcolor{textcolor}%
\pgftext[x=0.280863in,y=3.136307in,left,base]{\color{textcolor}\sffamily\fontsize{10.000000}{12.000000}\selectfont \(\displaystyle 0.95\)}%
\end{pgfscope}%
\begin{pgfscope}%
\definecolor{textcolor}{rgb}{0.000000,0.000000,0.000000}%
\pgfsetstrokecolor{textcolor}%
\pgfsetfillcolor{textcolor}%
\pgftext[x=0.225308in,y=1.980000in,,bottom,rotate=90.000000]{\color{textcolor}\sffamily\fontsize{10.000000}{12.000000}\selectfont \(\displaystyle \alpha\)}%
\end{pgfscope}%
\begin{pgfscope}%
\pgfpathrectangle{\pgfqpoint{0.625000in}{0.440000in}}{\pgfqpoint{3.875000in}{3.080000in}}%
\pgfusepath{clip}%
\pgfsetbuttcap%
\pgfsetroundjoin%
\pgfsetlinewidth{1.505625pt}%
\definecolor{currentstroke}{rgb}{0.000000,0.000000,0.000000}%
\pgfsetstrokecolor{currentstroke}%
\pgfsetdash{{5.550000pt}{2.400000pt}}{0.000000pt}%
\pgfpathmoveto{\pgfqpoint{0.801136in}{3.380000in}}%
\pgfpathlineto{\pgfqpoint{2.163913in}{2.582426in}}%
\pgfpathlineto{\pgfqpoint{2.961087in}{2.115875in}}%
\pgfpathlineto{\pgfqpoint{3.526690in}{1.784851in}}%
\pgfpathlineto{\pgfqpoint{3.965406in}{1.528090in}}%
\pgfpathlineto{\pgfqpoint{4.323864in}{1.318300in}}%
\pgfusepath{stroke}%
\end{pgfscope}%
\begin{pgfscope}%
\pgfpathrectangle{\pgfqpoint{0.625000in}{0.440000in}}{\pgfqpoint{3.875000in}{3.080000in}}%
\pgfusepath{clip}%
\pgfsetbuttcap%
\pgfsetroundjoin%
\definecolor{currentfill}{rgb}{0.000000,0.000000,0.000000}%
\pgfsetfillcolor{currentfill}%
\pgfsetlinewidth{1.003750pt}%
\definecolor{currentstroke}{rgb}{0.000000,0.000000,0.000000}%
\pgfsetstrokecolor{currentstroke}%
\pgfsetdash{}{0pt}%
\pgfsys@defobject{currentmarker}{\pgfqpoint{-0.041667in}{-0.041667in}}{\pgfqpoint{0.041667in}{0.041667in}}{%
\pgfpathmoveto{\pgfqpoint{0.000000in}{-0.041667in}}%
\pgfpathcurveto{\pgfqpoint{0.011050in}{-0.041667in}}{\pgfqpoint{0.021649in}{-0.037276in}}{\pgfqpoint{0.029463in}{-0.029463in}}%
\pgfpathcurveto{\pgfqpoint{0.037276in}{-0.021649in}}{\pgfqpoint{0.041667in}{-0.011050in}}{\pgfqpoint{0.041667in}{0.000000in}}%
\pgfpathcurveto{\pgfqpoint{0.041667in}{0.011050in}}{\pgfqpoint{0.037276in}{0.021649in}}{\pgfqpoint{0.029463in}{0.029463in}}%
\pgfpathcurveto{\pgfqpoint{0.021649in}{0.037276in}}{\pgfqpoint{0.011050in}{0.041667in}}{\pgfqpoint{0.000000in}{0.041667in}}%
\pgfpathcurveto{\pgfqpoint{-0.011050in}{0.041667in}}{\pgfqpoint{-0.021649in}{0.037276in}}{\pgfqpoint{-0.029463in}{0.029463in}}%
\pgfpathcurveto{\pgfqpoint{-0.037276in}{0.021649in}}{\pgfqpoint{-0.041667in}{0.011050in}}{\pgfqpoint{-0.041667in}{0.000000in}}%
\pgfpathcurveto{\pgfqpoint{-0.041667in}{-0.011050in}}{\pgfqpoint{-0.037276in}{-0.021649in}}{\pgfqpoint{-0.029463in}{-0.029463in}}%
\pgfpathcurveto{\pgfqpoint{-0.021649in}{-0.037276in}}{\pgfqpoint{-0.011050in}{-0.041667in}}{\pgfqpoint{0.000000in}{-0.041667in}}%
\pgfpathclose%
\pgfusepath{stroke,fill}%
}%
\begin{pgfscope}%
\pgfsys@transformshift{0.801136in}{2.792835in}%
\pgfsys@useobject{currentmarker}{}%
\end{pgfscope}%
\begin{pgfscope}%
\pgfsys@transformshift{2.163913in}{2.980508in}%
\pgfsys@useobject{currentmarker}{}%
\end{pgfscope}%
\begin{pgfscope}%
\pgfsys@transformshift{2.961087in}{2.884867in}%
\pgfsys@useobject{currentmarker}{}%
\end{pgfscope}%
\begin{pgfscope}%
\pgfsys@transformshift{3.526690in}{2.021546in}%
\pgfsys@useobject{currentmarker}{}%
\end{pgfscope}%
\begin{pgfscope}%
\pgfsys@transformshift{3.965406in}{1.449785in}%
\pgfsys@useobject{currentmarker}{}%
\end{pgfscope}%
\begin{pgfscope}%
\pgfsys@transformshift{4.323864in}{0.580000in}%
\pgfsys@useobject{currentmarker}{}%
\end{pgfscope}%
\end{pgfscope}%
\begin{pgfscope}%
\pgfsetrectcap%
\pgfsetmiterjoin%
\pgfsetlinewidth{0.803000pt}%
\definecolor{currentstroke}{rgb}{0.000000,0.000000,0.000000}%
\pgfsetstrokecolor{currentstroke}%
\pgfsetdash{}{0pt}%
\pgfpathmoveto{\pgfqpoint{0.625000in}{0.440000in}}%
\pgfpathlineto{\pgfqpoint{0.625000in}{3.520000in}}%
\pgfusepath{stroke}%
\end{pgfscope}%
\begin{pgfscope}%
\pgfsetrectcap%
\pgfsetmiterjoin%
\pgfsetlinewidth{0.803000pt}%
\definecolor{currentstroke}{rgb}{0.000000,0.000000,0.000000}%
\pgfsetstrokecolor{currentstroke}%
\pgfsetdash{}{0pt}%
\pgfpathmoveto{\pgfqpoint{4.500000in}{0.440000in}}%
\pgfpathlineto{\pgfqpoint{4.500000in}{3.520000in}}%
\pgfusepath{stroke}%
\end{pgfscope}%
\begin{pgfscope}%
\pgfsetrectcap%
\pgfsetmiterjoin%
\pgfsetlinewidth{0.803000pt}%
\definecolor{currentstroke}{rgb}{0.000000,0.000000,0.000000}%
\pgfsetstrokecolor{currentstroke}%
\pgfsetdash{}{0pt}%
\pgfpathmoveto{\pgfqpoint{0.625000in}{0.440000in}}%
\pgfpathlineto{\pgfqpoint{4.500000in}{0.440000in}}%
\pgfusepath{stroke}%
\end{pgfscope}%
\begin{pgfscope}%
\pgfsetrectcap%
\pgfsetmiterjoin%
\pgfsetlinewidth{0.803000pt}%
\definecolor{currentstroke}{rgb}{0.000000,0.000000,0.000000}%
\pgfsetstrokecolor{currentstroke}%
\pgfsetdash{}{0pt}%
\pgfpathmoveto{\pgfqpoint{0.625000in}{3.520000in}}%
\pgfpathlineto{\pgfqpoint{4.500000in}{3.520000in}}%
\pgfusepath{stroke}%
\end{pgfscope}%
\begin{pgfscope}%
\pgfsetbuttcap%
\pgfsetmiterjoin%
\definecolor{currentfill}{rgb}{1.000000,1.000000,1.000000}%
\pgfsetfillcolor{currentfill}%
\pgfsetfillopacity{0.800000}%
\pgfsetlinewidth{1.003750pt}%
\definecolor{currentstroke}{rgb}{0.800000,0.800000,0.800000}%
\pgfsetstrokecolor{currentstroke}%
\pgfsetstrokeopacity{0.800000}%
\pgfsetdash{}{0pt}%
\pgfpathmoveto{\pgfqpoint{2.423855in}{3.199199in}}%
\pgfpathlineto{\pgfqpoint{4.402778in}{3.199199in}}%
\pgfpathquadraticcurveto{\pgfqpoint{4.430556in}{3.199199in}}{\pgfqpoint{4.430556in}{3.226977in}}%
\pgfpathlineto{\pgfqpoint{4.430556in}{3.422778in}}%
\pgfpathquadraticcurveto{\pgfqpoint{4.430556in}{3.450556in}}{\pgfqpoint{4.402778in}{3.450556in}}%
\pgfpathlineto{\pgfqpoint{2.423855in}{3.450556in}}%
\pgfpathquadraticcurveto{\pgfqpoint{2.396077in}{3.450556in}}{\pgfqpoint{2.396077in}{3.422778in}}%
\pgfpathlineto{\pgfqpoint{2.396077in}{3.226977in}}%
\pgfpathquadraticcurveto{\pgfqpoint{2.396077in}{3.199199in}}{\pgfqpoint{2.423855in}{3.199199in}}%
\pgfpathclose%
\pgfusepath{stroke,fill}%
\end{pgfscope}%
\begin{pgfscope}%
\pgfsetbuttcap%
\pgfsetroundjoin%
\pgfsetlinewidth{1.505625pt}%
\definecolor{currentstroke}{rgb}{0.000000,0.000000,0.000000}%
\pgfsetstrokecolor{currentstroke}%
\pgfsetdash{{5.550000pt}{2.400000pt}}{0.000000pt}%
\pgfpathmoveto{\pgfqpoint{2.451633in}{3.338088in}}%
\pgfpathlineto{\pgfqpoint{2.729411in}{3.338088in}}%
\pgfusepath{stroke}%
\end{pgfscope}%
\begin{pgfscope}%
\definecolor{textcolor}{rgb}{0.000000,0.000000,0.000000}%
\pgfsetstrokecolor{textcolor}%
\pgfsetfillcolor{textcolor}%
\pgftext[x=2.840522in,y=3.289477in,left,base]{\color{textcolor}\sffamily\fontsize{10.000000}{12.000000}\selectfont \(\displaystyle \alpha = 1.3375 - 0.1569\mathrm{ln}(k)\)}%
\end{pgfscope}%
\end{pgfpicture}%
\makeatother%
\endgroup%

    \end{subfigure}
\caption[The computed Quasi-Monte-Carlo convergence rate for $Q(u) =  \int_D u$.]{Plots of the computed values of $C$ (top) and $\alpha$ (bottom) against $k$ in \cref{eq:qmcerrorform} for $Q(u) = \int_D u$. Observe the $x$-axes are on a $\log_{10}$ scale, but $\loge$ is the natural logarithm. \label{fig:integralCalpha}}
\end{figure}

\begin{figure}[h]
    \centering
    \begin{subfigure}{\textwidth}
            \centering
%% Creator: Matplotlib, PGF backend
%%
%% To include the figure in your LaTeX document, write
%%   \input{<filename>.pgf}
%%
%% Make sure the required packages are loaded in your preamble
%%   \usepackage{pgf}
%%
%% Figures using additional raster images can only be included by \input if
%% they are in the same directory as the main LaTeX file. For loading figures
%% from other directories you can use the `import` package
%%   \usepackage{import}
%% and then include the figures with
%%   \import{<path to file>}{<filename>.pgf}
%%
%% Matplotlib used the following preamble
%%   \usepackage{fontspec}
%%   \setmainfont{DejaVuSerif.ttf}[Path=/home/owen/progs/firedrake-complex/firedrake/lib/python3.5/site-packages/matplotlib/mpl-data/fonts/ttf/]
%%   \setsansfont{DejaVuSans.ttf}[Path=/home/owen/progs/firedrake-complex/firedrake/lib/python3.5/site-packages/matplotlib/mpl-data/fonts/ttf/]
%%   \setmonofont{DejaVuSansMono.ttf}[Path=/home/owen/progs/firedrake-complex/firedrake/lib/python3.5/site-packages/matplotlib/mpl-data/fonts/ttf/]
%%
\begingroup%
\makeatletter%
\begin{pgfpicture}%
\pgfpathrectangle{\pgfpointorigin}{\pgfqpoint{5.000000in}{4.000000in}}%
\pgfusepath{use as bounding box, clip}%
\begin{pgfscope}%
\pgfsetbuttcap%
\pgfsetmiterjoin%
\definecolor{currentfill}{rgb}{1.000000,1.000000,1.000000}%
\pgfsetfillcolor{currentfill}%
\pgfsetlinewidth{0.000000pt}%
\definecolor{currentstroke}{rgb}{1.000000,1.000000,1.000000}%
\pgfsetstrokecolor{currentstroke}%
\pgfsetdash{}{0pt}%
\pgfpathmoveto{\pgfqpoint{0.000000in}{0.000000in}}%
\pgfpathlineto{\pgfqpoint{5.000000in}{0.000000in}}%
\pgfpathlineto{\pgfqpoint{5.000000in}{4.000000in}}%
\pgfpathlineto{\pgfqpoint{0.000000in}{4.000000in}}%
\pgfpathclose%
\pgfusepath{fill}%
\end{pgfscope}%
\begin{pgfscope}%
\pgfsetbuttcap%
\pgfsetmiterjoin%
\definecolor{currentfill}{rgb}{1.000000,1.000000,1.000000}%
\pgfsetfillcolor{currentfill}%
\pgfsetlinewidth{0.000000pt}%
\definecolor{currentstroke}{rgb}{0.000000,0.000000,0.000000}%
\pgfsetstrokecolor{currentstroke}%
\pgfsetstrokeopacity{0.000000}%
\pgfsetdash{}{0pt}%
\pgfpathmoveto{\pgfqpoint{0.625000in}{0.440000in}}%
\pgfpathlineto{\pgfqpoint{4.500000in}{0.440000in}}%
\pgfpathlineto{\pgfqpoint{4.500000in}{3.520000in}}%
\pgfpathlineto{\pgfqpoint{0.625000in}{3.520000in}}%
\pgfpathclose%
\pgfusepath{fill}%
\end{pgfscope}%
\begin{pgfscope}%
\pgfsetbuttcap%
\pgfsetroundjoin%
\definecolor{currentfill}{rgb}{0.000000,0.000000,0.000000}%
\pgfsetfillcolor{currentfill}%
\pgfsetlinewidth{0.803000pt}%
\definecolor{currentstroke}{rgb}{0.000000,0.000000,0.000000}%
\pgfsetstrokecolor{currentstroke}%
\pgfsetdash{}{0pt}%
\pgfsys@defobject{currentmarker}{\pgfqpoint{0.000000in}{-0.048611in}}{\pgfqpoint{0.000000in}{0.000000in}}{%
\pgfpathmoveto{\pgfqpoint{0.000000in}{0.000000in}}%
\pgfpathlineto{\pgfqpoint{0.000000in}{-0.048611in}}%
\pgfusepath{stroke,fill}%
}%
\begin{pgfscope}%
\pgfsys@transformshift{0.801136in}{0.440000in}%
\pgfsys@useobject{currentmarker}{}%
\end{pgfscope}%
\end{pgfscope}%
\begin{pgfscope}%
\definecolor{textcolor}{rgb}{0.000000,0.000000,0.000000}%
\pgfsetstrokecolor{textcolor}%
\pgfsetfillcolor{textcolor}%
\pgftext[x=0.801136in,y=0.342778in,,top]{\color{textcolor}\sffamily\fontsize{10.000000}{12.000000}\selectfont \(\displaystyle {10^{1}}\)}%
\end{pgfscope}%
\begin{pgfscope}%
\pgfsetbuttcap%
\pgfsetroundjoin%
\definecolor{currentfill}{rgb}{0.000000,0.000000,0.000000}%
\pgfsetfillcolor{currentfill}%
\pgfsetlinewidth{0.602250pt}%
\definecolor{currentstroke}{rgb}{0.000000,0.000000,0.000000}%
\pgfsetstrokecolor{currentstroke}%
\pgfsetdash{}{0pt}%
\pgfsys@defobject{currentmarker}{\pgfqpoint{0.000000in}{-0.027778in}}{\pgfqpoint{0.000000in}{0.000000in}}{%
\pgfpathmoveto{\pgfqpoint{0.000000in}{0.000000in}}%
\pgfpathlineto{\pgfqpoint{0.000000in}{-0.027778in}}%
\pgfusepath{stroke,fill}%
}%
\begin{pgfscope}%
\pgfsys@transformshift{2.163913in}{0.440000in}%
\pgfsys@useobject{currentmarker}{}%
\end{pgfscope}%
\end{pgfscope}%
\begin{pgfscope}%
\definecolor{textcolor}{rgb}{0.000000,0.000000,0.000000}%
\pgfsetstrokecolor{textcolor}%
\pgfsetfillcolor{textcolor}%
\pgftext[x=2.163913in,y=0.365000in,,top]{\color{textcolor}\sffamily\fontsize{10.000000}{12.000000}\selectfont \(\displaystyle {2\times10^{1}}\)}%
\end{pgfscope}%
\begin{pgfscope}%
\pgfsetbuttcap%
\pgfsetroundjoin%
\definecolor{currentfill}{rgb}{0.000000,0.000000,0.000000}%
\pgfsetfillcolor{currentfill}%
\pgfsetlinewidth{0.602250pt}%
\definecolor{currentstroke}{rgb}{0.000000,0.000000,0.000000}%
\pgfsetstrokecolor{currentstroke}%
\pgfsetdash{}{0pt}%
\pgfsys@defobject{currentmarker}{\pgfqpoint{0.000000in}{-0.027778in}}{\pgfqpoint{0.000000in}{0.000000in}}{%
\pgfpathmoveto{\pgfqpoint{0.000000in}{0.000000in}}%
\pgfpathlineto{\pgfqpoint{0.000000in}{-0.027778in}}%
\pgfusepath{stroke,fill}%
}%
\begin{pgfscope}%
\pgfsys@transformshift{2.961087in}{0.440000in}%
\pgfsys@useobject{currentmarker}{}%
\end{pgfscope}%
\end{pgfscope}%
\begin{pgfscope}%
\definecolor{textcolor}{rgb}{0.000000,0.000000,0.000000}%
\pgfsetstrokecolor{textcolor}%
\pgfsetfillcolor{textcolor}%
\pgftext[x=2.961087in,y=0.365000in,,top]{\color{textcolor}\sffamily\fontsize{10.000000}{12.000000}\selectfont \(\displaystyle {3\times10^{1}}\)}%
\end{pgfscope}%
\begin{pgfscope}%
\pgfsetbuttcap%
\pgfsetroundjoin%
\definecolor{currentfill}{rgb}{0.000000,0.000000,0.000000}%
\pgfsetfillcolor{currentfill}%
\pgfsetlinewidth{0.602250pt}%
\definecolor{currentstroke}{rgb}{0.000000,0.000000,0.000000}%
\pgfsetstrokecolor{currentstroke}%
\pgfsetdash{}{0pt}%
\pgfsys@defobject{currentmarker}{\pgfqpoint{0.000000in}{-0.027778in}}{\pgfqpoint{0.000000in}{0.000000in}}{%
\pgfpathmoveto{\pgfqpoint{0.000000in}{0.000000in}}%
\pgfpathlineto{\pgfqpoint{0.000000in}{-0.027778in}}%
\pgfusepath{stroke,fill}%
}%
\begin{pgfscope}%
\pgfsys@transformshift{3.526690in}{0.440000in}%
\pgfsys@useobject{currentmarker}{}%
\end{pgfscope}%
\end{pgfscope}%
\begin{pgfscope}%
\definecolor{textcolor}{rgb}{0.000000,0.000000,0.000000}%
\pgfsetstrokecolor{textcolor}%
\pgfsetfillcolor{textcolor}%
\pgftext[x=3.526690in,y=0.365000in,,top]{\color{textcolor}\sffamily\fontsize{10.000000}{12.000000}\selectfont \(\displaystyle {4\times10^{1}}\)}%
\end{pgfscope}%
\begin{pgfscope}%
\pgfsetbuttcap%
\pgfsetroundjoin%
\definecolor{currentfill}{rgb}{0.000000,0.000000,0.000000}%
\pgfsetfillcolor{currentfill}%
\pgfsetlinewidth{0.602250pt}%
\definecolor{currentstroke}{rgb}{0.000000,0.000000,0.000000}%
\pgfsetstrokecolor{currentstroke}%
\pgfsetdash{}{0pt}%
\pgfsys@defobject{currentmarker}{\pgfqpoint{0.000000in}{-0.027778in}}{\pgfqpoint{0.000000in}{0.000000in}}{%
\pgfpathmoveto{\pgfqpoint{0.000000in}{0.000000in}}%
\pgfpathlineto{\pgfqpoint{0.000000in}{-0.027778in}}%
\pgfusepath{stroke,fill}%
}%
\begin{pgfscope}%
\pgfsys@transformshift{3.965406in}{0.440000in}%
\pgfsys@useobject{currentmarker}{}%
\end{pgfscope}%
\end{pgfscope}%
\begin{pgfscope}%
\pgfsetbuttcap%
\pgfsetroundjoin%
\definecolor{currentfill}{rgb}{0.000000,0.000000,0.000000}%
\pgfsetfillcolor{currentfill}%
\pgfsetlinewidth{0.602250pt}%
\definecolor{currentstroke}{rgb}{0.000000,0.000000,0.000000}%
\pgfsetstrokecolor{currentstroke}%
\pgfsetdash{}{0pt}%
\pgfsys@defobject{currentmarker}{\pgfqpoint{0.000000in}{-0.027778in}}{\pgfqpoint{0.000000in}{0.000000in}}{%
\pgfpathmoveto{\pgfqpoint{0.000000in}{0.000000in}}%
\pgfpathlineto{\pgfqpoint{0.000000in}{-0.027778in}}%
\pgfusepath{stroke,fill}%
}%
\begin{pgfscope}%
\pgfsys@transformshift{4.323864in}{0.440000in}%
\pgfsys@useobject{currentmarker}{}%
\end{pgfscope}%
\end{pgfscope}%
\begin{pgfscope}%
\definecolor{textcolor}{rgb}{0.000000,0.000000,0.000000}%
\pgfsetstrokecolor{textcolor}%
\pgfsetfillcolor{textcolor}%
\pgftext[x=4.323864in,y=0.365000in,,top]{\color{textcolor}\sffamily\fontsize{10.000000}{12.000000}\selectfont \(\displaystyle {6\times10^{1}}\)}%
\end{pgfscope}%
\begin{pgfscope}%
\definecolor{textcolor}{rgb}{0.000000,0.000000,0.000000}%
\pgfsetstrokecolor{textcolor}%
\pgfsetfillcolor{textcolor}%
\pgftext[x=2.562500in,y=0.152809in,,top]{\color{textcolor}\sffamily\fontsize{10.000000}{12.000000}\selectfont k}%
\end{pgfscope}%
\begin{pgfscope}%
\pgfsetbuttcap%
\pgfsetroundjoin%
\definecolor{currentfill}{rgb}{0.000000,0.000000,0.000000}%
\pgfsetfillcolor{currentfill}%
\pgfsetlinewidth{0.803000pt}%
\definecolor{currentstroke}{rgb}{0.000000,0.000000,0.000000}%
\pgfsetstrokecolor{currentstroke}%
\pgfsetdash{}{0pt}%
\pgfsys@defobject{currentmarker}{\pgfqpoint{-0.048611in}{0.000000in}}{\pgfqpoint{0.000000in}{0.000000in}}{%
\pgfpathmoveto{\pgfqpoint{0.000000in}{0.000000in}}%
\pgfpathlineto{\pgfqpoint{-0.048611in}{0.000000in}}%
\pgfusepath{stroke,fill}%
}%
\begin{pgfscope}%
\pgfsys@transformshift{0.625000in}{3.491714in}%
\pgfsys@useobject{currentmarker}{}%
\end{pgfscope}%
\end{pgfscope}%
\begin{pgfscope}%
\definecolor{textcolor}{rgb}{0.000000,0.000000,0.000000}%
\pgfsetstrokecolor{textcolor}%
\pgfsetfillcolor{textcolor}%
\pgftext[x=0.239775in,y=3.438953in,left,base]{\color{textcolor}\sffamily\fontsize{10.000000}{12.000000}\selectfont \(\displaystyle {10^{-2}}\)}%
\end{pgfscope}%
\begin{pgfscope}%
\pgfsetbuttcap%
\pgfsetroundjoin%
\definecolor{currentfill}{rgb}{0.000000,0.000000,0.000000}%
\pgfsetfillcolor{currentfill}%
\pgfsetlinewidth{0.602250pt}%
\definecolor{currentstroke}{rgb}{0.000000,0.000000,0.000000}%
\pgfsetstrokecolor{currentstroke}%
\pgfsetdash{}{0pt}%
\pgfsys@defobject{currentmarker}{\pgfqpoint{-0.027778in}{0.000000in}}{\pgfqpoint{0.000000in}{0.000000in}}{%
\pgfpathmoveto{\pgfqpoint{0.000000in}{0.000000in}}%
\pgfpathlineto{\pgfqpoint{-0.027778in}{0.000000in}}%
\pgfusepath{stroke,fill}%
}%
\begin{pgfscope}%
\pgfsys@transformshift{0.625000in}{0.893670in}%
\pgfsys@useobject{currentmarker}{}%
\end{pgfscope}%
\end{pgfscope}%
\begin{pgfscope}%
\definecolor{textcolor}{rgb}{0.000000,0.000000,0.000000}%
\pgfsetstrokecolor{textcolor}%
\pgfsetfillcolor{textcolor}%
\pgftext[x=0.022801in,y=0.840909in,left,base]{\color{textcolor}\sffamily\fontsize{10.000000}{12.000000}\selectfont \(\displaystyle {8\times10^{-3}}\)}%
\end{pgfscope}%
\begin{pgfscope}%
\pgfsetbuttcap%
\pgfsetroundjoin%
\definecolor{currentfill}{rgb}{0.000000,0.000000,0.000000}%
\pgfsetfillcolor{currentfill}%
\pgfsetlinewidth{0.602250pt}%
\definecolor{currentstroke}{rgb}{0.000000,0.000000,0.000000}%
\pgfsetstrokecolor{currentstroke}%
\pgfsetdash{}{0pt}%
\pgfsys@defobject{currentmarker}{\pgfqpoint{-0.027778in}{0.000000in}}{\pgfqpoint{0.000000in}{0.000000in}}{%
\pgfpathmoveto{\pgfqpoint{0.000000in}{0.000000in}}%
\pgfpathlineto{\pgfqpoint{-0.027778in}{0.000000in}}%
\pgfusepath{stroke,fill}%
}%
\begin{pgfscope}%
\pgfsys@transformshift{0.625000in}{2.265010in}%
\pgfsys@useobject{currentmarker}{}%
\end{pgfscope}%
\end{pgfscope}%
\begin{pgfscope}%
\definecolor{textcolor}{rgb}{0.000000,0.000000,0.000000}%
\pgfsetstrokecolor{textcolor}%
\pgfsetfillcolor{textcolor}%
\pgftext[x=0.022801in,y=2.212248in,left,base]{\color{textcolor}\sffamily\fontsize{10.000000}{12.000000}\selectfont \(\displaystyle {9\times10^{-3}}\)}%
\end{pgfscope}%
\begin{pgfscope}%
\definecolor{textcolor}{rgb}{0.000000,0.000000,0.000000}%
\pgfsetstrokecolor{textcolor}%
\pgfsetfillcolor{textcolor}%
\pgftext[x=-0.032755in,y=1.980000in,,bottom,rotate=90.000000]{\color{textcolor}\sffamily\fontsize{10.000000}{12.000000}\selectfont C}%
\end{pgfscope}%
\begin{pgfscope}%
\pgfpathrectangle{\pgfqpoint{0.625000in}{0.440000in}}{\pgfqpoint{3.875000in}{3.080000in}}%
\pgfusepath{clip}%
\pgfsetbuttcap%
\pgfsetroundjoin%
\definecolor{currentfill}{rgb}{0.000000,0.000000,0.000000}%
\pgfsetfillcolor{currentfill}%
\pgfsetlinewidth{1.003750pt}%
\definecolor{currentstroke}{rgb}{0.000000,0.000000,0.000000}%
\pgfsetstrokecolor{currentstroke}%
\pgfsetdash{}{0pt}%
\pgfsys@defobject{currentmarker}{\pgfqpoint{-0.041667in}{-0.041667in}}{\pgfqpoint{0.041667in}{0.041667in}}{%
\pgfpathmoveto{\pgfqpoint{0.000000in}{-0.041667in}}%
\pgfpathcurveto{\pgfqpoint{0.011050in}{-0.041667in}}{\pgfqpoint{0.021649in}{-0.037276in}}{\pgfqpoint{0.029463in}{-0.029463in}}%
\pgfpathcurveto{\pgfqpoint{0.037276in}{-0.021649in}}{\pgfqpoint{0.041667in}{-0.011050in}}{\pgfqpoint{0.041667in}{0.000000in}}%
\pgfpathcurveto{\pgfqpoint{0.041667in}{0.011050in}}{\pgfqpoint{0.037276in}{0.021649in}}{\pgfqpoint{0.029463in}{0.029463in}}%
\pgfpathcurveto{\pgfqpoint{0.021649in}{0.037276in}}{\pgfqpoint{0.011050in}{0.041667in}}{\pgfqpoint{0.000000in}{0.041667in}}%
\pgfpathcurveto{\pgfqpoint{-0.011050in}{0.041667in}}{\pgfqpoint{-0.021649in}{0.037276in}}{\pgfqpoint{-0.029463in}{0.029463in}}%
\pgfpathcurveto{\pgfqpoint{-0.037276in}{0.021649in}}{\pgfqpoint{-0.041667in}{0.011050in}}{\pgfqpoint{-0.041667in}{0.000000in}}%
\pgfpathcurveto{\pgfqpoint{-0.041667in}{-0.011050in}}{\pgfqpoint{-0.037276in}{-0.021649in}}{\pgfqpoint{-0.029463in}{-0.029463in}}%
\pgfpathcurveto{\pgfqpoint{-0.021649in}{-0.037276in}}{\pgfqpoint{-0.011050in}{-0.041667in}}{\pgfqpoint{0.000000in}{-0.041667in}}%
\pgfpathclose%
\pgfusepath{stroke,fill}%
}%
\begin{pgfscope}%
\pgfsys@transformshift{0.801136in}{0.639416in}%
\pgfsys@useobject{currentmarker}{}%
\end{pgfscope}%
\begin{pgfscope}%
\pgfsys@transformshift{2.163913in}{1.106347in}%
\pgfsys@useobject{currentmarker}{}%
\end{pgfscope}%
\begin{pgfscope}%
\pgfsys@transformshift{2.961087in}{0.580000in}%
\pgfsys@useobject{currentmarker}{}%
\end{pgfscope}%
\begin{pgfscope}%
\pgfsys@transformshift{3.526690in}{3.380000in}%
\pgfsys@useobject{currentmarker}{}%
\end{pgfscope}%
\begin{pgfscope}%
\pgfsys@transformshift{3.965406in}{1.950632in}%
\pgfsys@useobject{currentmarker}{}%
\end{pgfscope}%
\begin{pgfscope}%
\pgfsys@transformshift{4.323864in}{1.870142in}%
\pgfsys@useobject{currentmarker}{}%
\end{pgfscope}%
\end{pgfscope}%
\begin{pgfscope}%
\pgfsetrectcap%
\pgfsetmiterjoin%
\pgfsetlinewidth{0.803000pt}%
\definecolor{currentstroke}{rgb}{0.000000,0.000000,0.000000}%
\pgfsetstrokecolor{currentstroke}%
\pgfsetdash{}{0pt}%
\pgfpathmoveto{\pgfqpoint{0.625000in}{0.440000in}}%
\pgfpathlineto{\pgfqpoint{0.625000in}{3.520000in}}%
\pgfusepath{stroke}%
\end{pgfscope}%
\begin{pgfscope}%
\pgfsetrectcap%
\pgfsetmiterjoin%
\pgfsetlinewidth{0.803000pt}%
\definecolor{currentstroke}{rgb}{0.000000,0.000000,0.000000}%
\pgfsetstrokecolor{currentstroke}%
\pgfsetdash{}{0pt}%
\pgfpathmoveto{\pgfqpoint{4.500000in}{0.440000in}}%
\pgfpathlineto{\pgfqpoint{4.500000in}{3.520000in}}%
\pgfusepath{stroke}%
\end{pgfscope}%
\begin{pgfscope}%
\pgfsetrectcap%
\pgfsetmiterjoin%
\pgfsetlinewidth{0.803000pt}%
\definecolor{currentstroke}{rgb}{0.000000,0.000000,0.000000}%
\pgfsetstrokecolor{currentstroke}%
\pgfsetdash{}{0pt}%
\pgfpathmoveto{\pgfqpoint{0.625000in}{0.440000in}}%
\pgfpathlineto{\pgfqpoint{4.500000in}{0.440000in}}%
\pgfusepath{stroke}%
\end{pgfscope}%
\begin{pgfscope}%
\pgfsetrectcap%
\pgfsetmiterjoin%
\pgfsetlinewidth{0.803000pt}%
\definecolor{currentstroke}{rgb}{0.000000,0.000000,0.000000}%
\pgfsetstrokecolor{currentstroke}%
\pgfsetdash{}{0pt}%
\pgfpathmoveto{\pgfqpoint{0.625000in}{3.520000in}}%
\pgfpathlineto{\pgfqpoint{4.500000in}{3.520000in}}%
\pgfusepath{stroke}%
\end{pgfscope}%
\end{pgfpicture}%
\makeatother%
\endgroup%

  \end{subfigure}
    \begin{subfigure}{\textwidth}
                \centering
%% Creator: Matplotlib, PGF backend
%%
%% To include the figure in your LaTeX document, write
%%   \input{<filename>.pgf}
%%
%% Make sure the required packages are loaded in your preamble
%%   \usepackage{pgf}
%%
%% Figures using additional raster images can only be included by \input if
%% they are in the same directory as the main LaTeX file. For loading figures
%% from other directories you can use the `import` package
%%   \usepackage{import}
%% and then include the figures with
%%   \import{<path to file>}{<filename>.pgf}
%%
%% Matplotlib used the following preamble
%%   \usepackage{fontspec}
%%   \setmainfont{DejaVuSerif.ttf}[Path=/home/owen/progs/firedrake-complex/firedrake/lib/python3.5/site-packages/matplotlib/mpl-data/fonts/ttf/]
%%   \setsansfont{DejaVuSans.ttf}[Path=/home/owen/progs/firedrake-complex/firedrake/lib/python3.5/site-packages/matplotlib/mpl-data/fonts/ttf/]
%%   \setmonofont{DejaVuSansMono.ttf}[Path=/home/owen/progs/firedrake-complex/firedrake/lib/python3.5/site-packages/matplotlib/mpl-data/fonts/ttf/]
%%
\begingroup%
\makeatletter%
\begin{pgfpicture}%
\pgfpathrectangle{\pgfpointorigin}{\pgfqpoint{5.000000in}{4.000000in}}%
\pgfusepath{use as bounding box, clip}%
\begin{pgfscope}%
\pgfsetbuttcap%
\pgfsetmiterjoin%
\definecolor{currentfill}{rgb}{1.000000,1.000000,1.000000}%
\pgfsetfillcolor{currentfill}%
\pgfsetlinewidth{0.000000pt}%
\definecolor{currentstroke}{rgb}{1.000000,1.000000,1.000000}%
\pgfsetstrokecolor{currentstroke}%
\pgfsetdash{}{0pt}%
\pgfpathmoveto{\pgfqpoint{0.000000in}{0.000000in}}%
\pgfpathlineto{\pgfqpoint{5.000000in}{0.000000in}}%
\pgfpathlineto{\pgfqpoint{5.000000in}{4.000000in}}%
\pgfpathlineto{\pgfqpoint{0.000000in}{4.000000in}}%
\pgfpathclose%
\pgfusepath{fill}%
\end{pgfscope}%
\begin{pgfscope}%
\pgfsetbuttcap%
\pgfsetmiterjoin%
\definecolor{currentfill}{rgb}{1.000000,1.000000,1.000000}%
\pgfsetfillcolor{currentfill}%
\pgfsetlinewidth{0.000000pt}%
\definecolor{currentstroke}{rgb}{0.000000,0.000000,0.000000}%
\pgfsetstrokecolor{currentstroke}%
\pgfsetstrokeopacity{0.000000}%
\pgfsetdash{}{0pt}%
\pgfpathmoveto{\pgfqpoint{0.625000in}{0.440000in}}%
\pgfpathlineto{\pgfqpoint{4.500000in}{0.440000in}}%
\pgfpathlineto{\pgfqpoint{4.500000in}{3.520000in}}%
\pgfpathlineto{\pgfqpoint{0.625000in}{3.520000in}}%
\pgfpathclose%
\pgfusepath{fill}%
\end{pgfscope}%
\begin{pgfscope}%
\pgfsetbuttcap%
\pgfsetroundjoin%
\definecolor{currentfill}{rgb}{0.000000,0.000000,0.000000}%
\pgfsetfillcolor{currentfill}%
\pgfsetlinewidth{0.803000pt}%
\definecolor{currentstroke}{rgb}{0.000000,0.000000,0.000000}%
\pgfsetstrokecolor{currentstroke}%
\pgfsetdash{}{0pt}%
\pgfsys@defobject{currentmarker}{\pgfqpoint{0.000000in}{-0.048611in}}{\pgfqpoint{0.000000in}{0.000000in}}{%
\pgfpathmoveto{\pgfqpoint{0.000000in}{0.000000in}}%
\pgfpathlineto{\pgfqpoint{0.000000in}{-0.048611in}}%
\pgfusepath{stroke,fill}%
}%
\begin{pgfscope}%
\pgfsys@transformshift{0.801136in}{0.440000in}%
\pgfsys@useobject{currentmarker}{}%
\end{pgfscope}%
\end{pgfscope}%
\begin{pgfscope}%
\definecolor{textcolor}{rgb}{0.000000,0.000000,0.000000}%
\pgfsetstrokecolor{textcolor}%
\pgfsetfillcolor{textcolor}%
\pgftext[x=0.801136in,y=0.342778in,,top]{\color{textcolor}\sffamily\fontsize{10.000000}{12.000000}\selectfont \(\displaystyle {10^{1}}\)}%
\end{pgfscope}%
\begin{pgfscope}%
\pgfsetbuttcap%
\pgfsetroundjoin%
\definecolor{currentfill}{rgb}{0.000000,0.000000,0.000000}%
\pgfsetfillcolor{currentfill}%
\pgfsetlinewidth{0.602250pt}%
\definecolor{currentstroke}{rgb}{0.000000,0.000000,0.000000}%
\pgfsetstrokecolor{currentstroke}%
\pgfsetdash{}{0pt}%
\pgfsys@defobject{currentmarker}{\pgfqpoint{0.000000in}{-0.027778in}}{\pgfqpoint{0.000000in}{0.000000in}}{%
\pgfpathmoveto{\pgfqpoint{0.000000in}{0.000000in}}%
\pgfpathlineto{\pgfqpoint{0.000000in}{-0.027778in}}%
\pgfusepath{stroke,fill}%
}%
\begin{pgfscope}%
\pgfsys@transformshift{2.163913in}{0.440000in}%
\pgfsys@useobject{currentmarker}{}%
\end{pgfscope}%
\end{pgfscope}%
\begin{pgfscope}%
\definecolor{textcolor}{rgb}{0.000000,0.000000,0.000000}%
\pgfsetstrokecolor{textcolor}%
\pgfsetfillcolor{textcolor}%
\pgftext[x=2.163913in,y=0.365000in,,top]{\color{textcolor}\sffamily\fontsize{10.000000}{12.000000}\selectfont \(\displaystyle {2\times10^{1}}\)}%
\end{pgfscope}%
\begin{pgfscope}%
\pgfsetbuttcap%
\pgfsetroundjoin%
\definecolor{currentfill}{rgb}{0.000000,0.000000,0.000000}%
\pgfsetfillcolor{currentfill}%
\pgfsetlinewidth{0.602250pt}%
\definecolor{currentstroke}{rgb}{0.000000,0.000000,0.000000}%
\pgfsetstrokecolor{currentstroke}%
\pgfsetdash{}{0pt}%
\pgfsys@defobject{currentmarker}{\pgfqpoint{0.000000in}{-0.027778in}}{\pgfqpoint{0.000000in}{0.000000in}}{%
\pgfpathmoveto{\pgfqpoint{0.000000in}{0.000000in}}%
\pgfpathlineto{\pgfqpoint{0.000000in}{-0.027778in}}%
\pgfusepath{stroke,fill}%
}%
\begin{pgfscope}%
\pgfsys@transformshift{2.961087in}{0.440000in}%
\pgfsys@useobject{currentmarker}{}%
\end{pgfscope}%
\end{pgfscope}%
\begin{pgfscope}%
\definecolor{textcolor}{rgb}{0.000000,0.000000,0.000000}%
\pgfsetstrokecolor{textcolor}%
\pgfsetfillcolor{textcolor}%
\pgftext[x=2.961087in,y=0.365000in,,top]{\color{textcolor}\sffamily\fontsize{10.000000}{12.000000}\selectfont \(\displaystyle {3\times10^{1}}\)}%
\end{pgfscope}%
\begin{pgfscope}%
\pgfsetbuttcap%
\pgfsetroundjoin%
\definecolor{currentfill}{rgb}{0.000000,0.000000,0.000000}%
\pgfsetfillcolor{currentfill}%
\pgfsetlinewidth{0.602250pt}%
\definecolor{currentstroke}{rgb}{0.000000,0.000000,0.000000}%
\pgfsetstrokecolor{currentstroke}%
\pgfsetdash{}{0pt}%
\pgfsys@defobject{currentmarker}{\pgfqpoint{0.000000in}{-0.027778in}}{\pgfqpoint{0.000000in}{0.000000in}}{%
\pgfpathmoveto{\pgfqpoint{0.000000in}{0.000000in}}%
\pgfpathlineto{\pgfqpoint{0.000000in}{-0.027778in}}%
\pgfusepath{stroke,fill}%
}%
\begin{pgfscope}%
\pgfsys@transformshift{3.526690in}{0.440000in}%
\pgfsys@useobject{currentmarker}{}%
\end{pgfscope}%
\end{pgfscope}%
\begin{pgfscope}%
\definecolor{textcolor}{rgb}{0.000000,0.000000,0.000000}%
\pgfsetstrokecolor{textcolor}%
\pgfsetfillcolor{textcolor}%
\pgftext[x=3.526690in,y=0.365000in,,top]{\color{textcolor}\sffamily\fontsize{10.000000}{12.000000}\selectfont \(\displaystyle {4\times10^{1}}\)}%
\end{pgfscope}%
\begin{pgfscope}%
\pgfsetbuttcap%
\pgfsetroundjoin%
\definecolor{currentfill}{rgb}{0.000000,0.000000,0.000000}%
\pgfsetfillcolor{currentfill}%
\pgfsetlinewidth{0.602250pt}%
\definecolor{currentstroke}{rgb}{0.000000,0.000000,0.000000}%
\pgfsetstrokecolor{currentstroke}%
\pgfsetdash{}{0pt}%
\pgfsys@defobject{currentmarker}{\pgfqpoint{0.000000in}{-0.027778in}}{\pgfqpoint{0.000000in}{0.000000in}}{%
\pgfpathmoveto{\pgfqpoint{0.000000in}{0.000000in}}%
\pgfpathlineto{\pgfqpoint{0.000000in}{-0.027778in}}%
\pgfusepath{stroke,fill}%
}%
\begin{pgfscope}%
\pgfsys@transformshift{3.965406in}{0.440000in}%
\pgfsys@useobject{currentmarker}{}%
\end{pgfscope}%
\end{pgfscope}%
\begin{pgfscope}%
\pgfsetbuttcap%
\pgfsetroundjoin%
\definecolor{currentfill}{rgb}{0.000000,0.000000,0.000000}%
\pgfsetfillcolor{currentfill}%
\pgfsetlinewidth{0.602250pt}%
\definecolor{currentstroke}{rgb}{0.000000,0.000000,0.000000}%
\pgfsetstrokecolor{currentstroke}%
\pgfsetdash{}{0pt}%
\pgfsys@defobject{currentmarker}{\pgfqpoint{0.000000in}{-0.027778in}}{\pgfqpoint{0.000000in}{0.000000in}}{%
\pgfpathmoveto{\pgfqpoint{0.000000in}{0.000000in}}%
\pgfpathlineto{\pgfqpoint{0.000000in}{-0.027778in}}%
\pgfusepath{stroke,fill}%
}%
\begin{pgfscope}%
\pgfsys@transformshift{4.323864in}{0.440000in}%
\pgfsys@useobject{currentmarker}{}%
\end{pgfscope}%
\end{pgfscope}%
\begin{pgfscope}%
\definecolor{textcolor}{rgb}{0.000000,0.000000,0.000000}%
\pgfsetstrokecolor{textcolor}%
\pgfsetfillcolor{textcolor}%
\pgftext[x=4.323864in,y=0.365000in,,top]{\color{textcolor}\sffamily\fontsize{10.000000}{12.000000}\selectfont \(\displaystyle {6\times10^{1}}\)}%
\end{pgfscope}%
\begin{pgfscope}%
\definecolor{textcolor}{rgb}{0.000000,0.000000,0.000000}%
\pgfsetstrokecolor{textcolor}%
\pgfsetfillcolor{textcolor}%
\pgftext[x=2.562500in,y=0.152809in,,top]{\color{textcolor}\sffamily\fontsize{10.000000}{12.000000}\selectfont \(\displaystyle k\)}%
\end{pgfscope}%
\begin{pgfscope}%
\pgfsetbuttcap%
\pgfsetroundjoin%
\definecolor{currentfill}{rgb}{0.000000,0.000000,0.000000}%
\pgfsetfillcolor{currentfill}%
\pgfsetlinewidth{0.803000pt}%
\definecolor{currentstroke}{rgb}{0.000000,0.000000,0.000000}%
\pgfsetstrokecolor{currentstroke}%
\pgfsetdash{}{0pt}%
\pgfsys@defobject{currentmarker}{\pgfqpoint{-0.048611in}{0.000000in}}{\pgfqpoint{0.000000in}{0.000000in}}{%
\pgfpathmoveto{\pgfqpoint{0.000000in}{0.000000in}}%
\pgfpathlineto{\pgfqpoint{-0.048611in}{0.000000in}}%
\pgfusepath{stroke,fill}%
}%
\begin{pgfscope}%
\pgfsys@transformshift{0.625000in}{0.560614in}%
\pgfsys@useobject{currentmarker}{}%
\end{pgfscope}%
\end{pgfscope}%
\begin{pgfscope}%
\definecolor{textcolor}{rgb}{0.000000,0.000000,0.000000}%
\pgfsetstrokecolor{textcolor}%
\pgfsetfillcolor{textcolor}%
\pgftext[x=0.218533in,y=0.507853in,left,base]{\color{textcolor}\sffamily\fontsize{10.000000}{12.000000}\selectfont 0.60}%
\end{pgfscope}%
\begin{pgfscope}%
\pgfsetbuttcap%
\pgfsetroundjoin%
\definecolor{currentfill}{rgb}{0.000000,0.000000,0.000000}%
\pgfsetfillcolor{currentfill}%
\pgfsetlinewidth{0.803000pt}%
\definecolor{currentstroke}{rgb}{0.000000,0.000000,0.000000}%
\pgfsetstrokecolor{currentstroke}%
\pgfsetdash{}{0pt}%
\pgfsys@defobject{currentmarker}{\pgfqpoint{-0.048611in}{0.000000in}}{\pgfqpoint{0.000000in}{0.000000in}}{%
\pgfpathmoveto{\pgfqpoint{0.000000in}{0.000000in}}%
\pgfpathlineto{\pgfqpoint{-0.048611in}{0.000000in}}%
\pgfusepath{stroke,fill}%
}%
\begin{pgfscope}%
\pgfsys@transformshift{0.625000in}{0.956056in}%
\pgfsys@useobject{currentmarker}{}%
\end{pgfscope}%
\end{pgfscope}%
\begin{pgfscope}%
\definecolor{textcolor}{rgb}{0.000000,0.000000,0.000000}%
\pgfsetstrokecolor{textcolor}%
\pgfsetfillcolor{textcolor}%
\pgftext[x=0.218533in,y=0.903294in,left,base]{\color{textcolor}\sffamily\fontsize{10.000000}{12.000000}\selectfont 0.65}%
\end{pgfscope}%
\begin{pgfscope}%
\pgfsetbuttcap%
\pgfsetroundjoin%
\definecolor{currentfill}{rgb}{0.000000,0.000000,0.000000}%
\pgfsetfillcolor{currentfill}%
\pgfsetlinewidth{0.803000pt}%
\definecolor{currentstroke}{rgb}{0.000000,0.000000,0.000000}%
\pgfsetstrokecolor{currentstroke}%
\pgfsetdash{}{0pt}%
\pgfsys@defobject{currentmarker}{\pgfqpoint{-0.048611in}{0.000000in}}{\pgfqpoint{0.000000in}{0.000000in}}{%
\pgfpathmoveto{\pgfqpoint{0.000000in}{0.000000in}}%
\pgfpathlineto{\pgfqpoint{-0.048611in}{0.000000in}}%
\pgfusepath{stroke,fill}%
}%
\begin{pgfscope}%
\pgfsys@transformshift{0.625000in}{1.351498in}%
\pgfsys@useobject{currentmarker}{}%
\end{pgfscope}%
\end{pgfscope}%
\begin{pgfscope}%
\definecolor{textcolor}{rgb}{0.000000,0.000000,0.000000}%
\pgfsetstrokecolor{textcolor}%
\pgfsetfillcolor{textcolor}%
\pgftext[x=0.218533in,y=1.298736in,left,base]{\color{textcolor}\sffamily\fontsize{10.000000}{12.000000}\selectfont 0.70}%
\end{pgfscope}%
\begin{pgfscope}%
\pgfsetbuttcap%
\pgfsetroundjoin%
\definecolor{currentfill}{rgb}{0.000000,0.000000,0.000000}%
\pgfsetfillcolor{currentfill}%
\pgfsetlinewidth{0.803000pt}%
\definecolor{currentstroke}{rgb}{0.000000,0.000000,0.000000}%
\pgfsetstrokecolor{currentstroke}%
\pgfsetdash{}{0pt}%
\pgfsys@defobject{currentmarker}{\pgfqpoint{-0.048611in}{0.000000in}}{\pgfqpoint{0.000000in}{0.000000in}}{%
\pgfpathmoveto{\pgfqpoint{0.000000in}{0.000000in}}%
\pgfpathlineto{\pgfqpoint{-0.048611in}{0.000000in}}%
\pgfusepath{stroke,fill}%
}%
\begin{pgfscope}%
\pgfsys@transformshift{0.625000in}{1.746939in}%
\pgfsys@useobject{currentmarker}{}%
\end{pgfscope}%
\end{pgfscope}%
\begin{pgfscope}%
\definecolor{textcolor}{rgb}{0.000000,0.000000,0.000000}%
\pgfsetstrokecolor{textcolor}%
\pgfsetfillcolor{textcolor}%
\pgftext[x=0.218533in,y=1.694178in,left,base]{\color{textcolor}\sffamily\fontsize{10.000000}{12.000000}\selectfont 0.75}%
\end{pgfscope}%
\begin{pgfscope}%
\pgfsetbuttcap%
\pgfsetroundjoin%
\definecolor{currentfill}{rgb}{0.000000,0.000000,0.000000}%
\pgfsetfillcolor{currentfill}%
\pgfsetlinewidth{0.803000pt}%
\definecolor{currentstroke}{rgb}{0.000000,0.000000,0.000000}%
\pgfsetstrokecolor{currentstroke}%
\pgfsetdash{}{0pt}%
\pgfsys@defobject{currentmarker}{\pgfqpoint{-0.048611in}{0.000000in}}{\pgfqpoint{0.000000in}{0.000000in}}{%
\pgfpathmoveto{\pgfqpoint{0.000000in}{0.000000in}}%
\pgfpathlineto{\pgfqpoint{-0.048611in}{0.000000in}}%
\pgfusepath{stroke,fill}%
}%
\begin{pgfscope}%
\pgfsys@transformshift{0.625000in}{2.142381in}%
\pgfsys@useobject{currentmarker}{}%
\end{pgfscope}%
\end{pgfscope}%
\begin{pgfscope}%
\definecolor{textcolor}{rgb}{0.000000,0.000000,0.000000}%
\pgfsetstrokecolor{textcolor}%
\pgfsetfillcolor{textcolor}%
\pgftext[x=0.218533in,y=2.089619in,left,base]{\color{textcolor}\sffamily\fontsize{10.000000}{12.000000}\selectfont 0.80}%
\end{pgfscope}%
\begin{pgfscope}%
\pgfsetbuttcap%
\pgfsetroundjoin%
\definecolor{currentfill}{rgb}{0.000000,0.000000,0.000000}%
\pgfsetfillcolor{currentfill}%
\pgfsetlinewidth{0.803000pt}%
\definecolor{currentstroke}{rgb}{0.000000,0.000000,0.000000}%
\pgfsetstrokecolor{currentstroke}%
\pgfsetdash{}{0pt}%
\pgfsys@defobject{currentmarker}{\pgfqpoint{-0.048611in}{0.000000in}}{\pgfqpoint{0.000000in}{0.000000in}}{%
\pgfpathmoveto{\pgfqpoint{0.000000in}{0.000000in}}%
\pgfpathlineto{\pgfqpoint{-0.048611in}{0.000000in}}%
\pgfusepath{stroke,fill}%
}%
\begin{pgfscope}%
\pgfsys@transformshift{0.625000in}{2.537823in}%
\pgfsys@useobject{currentmarker}{}%
\end{pgfscope}%
\end{pgfscope}%
\begin{pgfscope}%
\definecolor{textcolor}{rgb}{0.000000,0.000000,0.000000}%
\pgfsetstrokecolor{textcolor}%
\pgfsetfillcolor{textcolor}%
\pgftext[x=0.218533in,y=2.485061in,left,base]{\color{textcolor}\sffamily\fontsize{10.000000}{12.000000}\selectfont 0.85}%
\end{pgfscope}%
\begin{pgfscope}%
\pgfsetbuttcap%
\pgfsetroundjoin%
\definecolor{currentfill}{rgb}{0.000000,0.000000,0.000000}%
\pgfsetfillcolor{currentfill}%
\pgfsetlinewidth{0.803000pt}%
\definecolor{currentstroke}{rgb}{0.000000,0.000000,0.000000}%
\pgfsetstrokecolor{currentstroke}%
\pgfsetdash{}{0pt}%
\pgfsys@defobject{currentmarker}{\pgfqpoint{-0.048611in}{0.000000in}}{\pgfqpoint{0.000000in}{0.000000in}}{%
\pgfpathmoveto{\pgfqpoint{0.000000in}{0.000000in}}%
\pgfpathlineto{\pgfqpoint{-0.048611in}{0.000000in}}%
\pgfusepath{stroke,fill}%
}%
\begin{pgfscope}%
\pgfsys@transformshift{0.625000in}{2.933264in}%
\pgfsys@useobject{currentmarker}{}%
\end{pgfscope}%
\end{pgfscope}%
\begin{pgfscope}%
\definecolor{textcolor}{rgb}{0.000000,0.000000,0.000000}%
\pgfsetstrokecolor{textcolor}%
\pgfsetfillcolor{textcolor}%
\pgftext[x=0.218533in,y=2.880503in,left,base]{\color{textcolor}\sffamily\fontsize{10.000000}{12.000000}\selectfont 0.90}%
\end{pgfscope}%
\begin{pgfscope}%
\pgfsetbuttcap%
\pgfsetroundjoin%
\definecolor{currentfill}{rgb}{0.000000,0.000000,0.000000}%
\pgfsetfillcolor{currentfill}%
\pgfsetlinewidth{0.803000pt}%
\definecolor{currentstroke}{rgb}{0.000000,0.000000,0.000000}%
\pgfsetstrokecolor{currentstroke}%
\pgfsetdash{}{0pt}%
\pgfsys@defobject{currentmarker}{\pgfqpoint{-0.048611in}{0.000000in}}{\pgfqpoint{0.000000in}{0.000000in}}{%
\pgfpathmoveto{\pgfqpoint{0.000000in}{0.000000in}}%
\pgfpathlineto{\pgfqpoint{-0.048611in}{0.000000in}}%
\pgfusepath{stroke,fill}%
}%
\begin{pgfscope}%
\pgfsys@transformshift{0.625000in}{3.328706in}%
\pgfsys@useobject{currentmarker}{}%
\end{pgfscope}%
\end{pgfscope}%
\begin{pgfscope}%
\definecolor{textcolor}{rgb}{0.000000,0.000000,0.000000}%
\pgfsetstrokecolor{textcolor}%
\pgfsetfillcolor{textcolor}%
\pgftext[x=0.218533in,y=3.275945in,left,base]{\color{textcolor}\sffamily\fontsize{10.000000}{12.000000}\selectfont 0.95}%
\end{pgfscope}%
\begin{pgfscope}%
\definecolor{textcolor}{rgb}{0.000000,0.000000,0.000000}%
\pgfsetstrokecolor{textcolor}%
\pgfsetfillcolor{textcolor}%
\pgftext[x=0.162977in,y=1.980000in,,bottom,rotate=90.000000]{\color{textcolor}\sffamily\fontsize{10.000000}{12.000000}\selectfont \(\displaystyle \alpha\)}%
\end{pgfscope}%
\begin{pgfscope}%
\pgfpathrectangle{\pgfqpoint{0.625000in}{0.440000in}}{\pgfqpoint{3.875000in}{3.080000in}}%
\pgfusepath{clip}%
\pgfsetbuttcap%
\pgfsetroundjoin%
\pgfsetlinewidth{1.505625pt}%
\definecolor{currentstroke}{rgb}{0.000000,0.000000,0.000000}%
\pgfsetstrokecolor{currentstroke}%
\pgfsetdash{{5.550000pt}{2.400000pt}}{0.000000pt}%
\pgfpathmoveto{\pgfqpoint{0.801136in}{3.269246in}}%
\pgfpathlineto{\pgfqpoint{2.163913in}{2.228904in}}%
\pgfpathlineto{\pgfqpoint{2.961087in}{1.620343in}}%
\pgfpathlineto{\pgfqpoint{3.526690in}{1.188561in}}%
\pgfpathlineto{\pgfqpoint{3.965406in}{0.853646in}}%
\pgfpathlineto{\pgfqpoint{4.323864in}{0.580000in}}%
\pgfusepath{stroke}%
\end{pgfscope}%
\begin{pgfscope}%
\pgfpathrectangle{\pgfqpoint{0.625000in}{0.440000in}}{\pgfqpoint{3.875000in}{3.080000in}}%
\pgfusepath{clip}%
\pgfsetbuttcap%
\pgfsetroundjoin%
\definecolor{currentfill}{rgb}{0.000000,0.000000,0.000000}%
\pgfsetfillcolor{currentfill}%
\pgfsetlinewidth{1.003750pt}%
\definecolor{currentstroke}{rgb}{0.000000,0.000000,0.000000}%
\pgfsetstrokecolor{currentstroke}%
\pgfsetdash{}{0pt}%
\pgfsys@defobject{currentmarker}{\pgfqpoint{-0.041667in}{-0.041667in}}{\pgfqpoint{0.041667in}{0.041667in}}{%
\pgfpathmoveto{\pgfqpoint{0.000000in}{-0.041667in}}%
\pgfpathcurveto{\pgfqpoint{0.011050in}{-0.041667in}}{\pgfqpoint{0.021649in}{-0.037276in}}{\pgfqpoint{0.029463in}{-0.029463in}}%
\pgfpathcurveto{\pgfqpoint{0.037276in}{-0.021649in}}{\pgfqpoint{0.041667in}{-0.011050in}}{\pgfqpoint{0.041667in}{0.000000in}}%
\pgfpathcurveto{\pgfqpoint{0.041667in}{0.011050in}}{\pgfqpoint{0.037276in}{0.021649in}}{\pgfqpoint{0.029463in}{0.029463in}}%
\pgfpathcurveto{\pgfqpoint{0.021649in}{0.037276in}}{\pgfqpoint{0.011050in}{0.041667in}}{\pgfqpoint{0.000000in}{0.041667in}}%
\pgfpathcurveto{\pgfqpoint{-0.011050in}{0.041667in}}{\pgfqpoint{-0.021649in}{0.037276in}}{\pgfqpoint{-0.029463in}{0.029463in}}%
\pgfpathcurveto{\pgfqpoint{-0.037276in}{0.021649in}}{\pgfqpoint{-0.041667in}{0.011050in}}{\pgfqpoint{-0.041667in}{0.000000in}}%
\pgfpathcurveto{\pgfqpoint{-0.041667in}{-0.011050in}}{\pgfqpoint{-0.037276in}{-0.021649in}}{\pgfqpoint{-0.029463in}{-0.029463in}}%
\pgfpathcurveto{\pgfqpoint{-0.021649in}{-0.037276in}}{\pgfqpoint{-0.011050in}{-0.041667in}}{\pgfqpoint{0.000000in}{-0.041667in}}%
\pgfpathclose%
\pgfusepath{stroke,fill}%
}%
\begin{pgfscope}%
\pgfsys@transformshift{0.801136in}{3.380000in}%
\pgfsys@useobject{currentmarker}{}%
\end{pgfscope}%
\begin{pgfscope}%
\pgfsys@transformshift{2.163913in}{2.359432in}%
\pgfsys@useobject{currentmarker}{}%
\end{pgfscope}%
\begin{pgfscope}%
\pgfsys@transformshift{2.961087in}{1.117792in}%
\pgfsys@useobject{currentmarker}{}%
\end{pgfscope}%
\begin{pgfscope}%
\pgfsys@transformshift{3.526690in}{1.211877in}%
\pgfsys@useobject{currentmarker}{}%
\end{pgfscope}%
\begin{pgfscope}%
\pgfsys@transformshift{3.965406in}{0.837435in}%
\pgfsys@useobject{currentmarker}{}%
\end{pgfscope}%
\begin{pgfscope}%
\pgfsys@transformshift{4.323864in}{0.834164in}%
\pgfsys@useobject{currentmarker}{}%
\end{pgfscope}%
\end{pgfscope}%
\begin{pgfscope}%
\pgfsetrectcap%
\pgfsetmiterjoin%
\pgfsetlinewidth{0.803000pt}%
\definecolor{currentstroke}{rgb}{0.000000,0.000000,0.000000}%
\pgfsetstrokecolor{currentstroke}%
\pgfsetdash{}{0pt}%
\pgfpathmoveto{\pgfqpoint{0.625000in}{0.440000in}}%
\pgfpathlineto{\pgfqpoint{0.625000in}{3.520000in}}%
\pgfusepath{stroke}%
\end{pgfscope}%
\begin{pgfscope}%
\pgfsetrectcap%
\pgfsetmiterjoin%
\pgfsetlinewidth{0.803000pt}%
\definecolor{currentstroke}{rgb}{0.000000,0.000000,0.000000}%
\pgfsetstrokecolor{currentstroke}%
\pgfsetdash{}{0pt}%
\pgfpathmoveto{\pgfqpoint{4.500000in}{0.440000in}}%
\pgfpathlineto{\pgfqpoint{4.500000in}{3.520000in}}%
\pgfusepath{stroke}%
\end{pgfscope}%
\begin{pgfscope}%
\pgfsetrectcap%
\pgfsetmiterjoin%
\pgfsetlinewidth{0.803000pt}%
\definecolor{currentstroke}{rgb}{0.000000,0.000000,0.000000}%
\pgfsetstrokecolor{currentstroke}%
\pgfsetdash{}{0pt}%
\pgfpathmoveto{\pgfqpoint{0.625000in}{0.440000in}}%
\pgfpathlineto{\pgfqpoint{4.500000in}{0.440000in}}%
\pgfusepath{stroke}%
\end{pgfscope}%
\begin{pgfscope}%
\pgfsetrectcap%
\pgfsetmiterjoin%
\pgfsetlinewidth{0.803000pt}%
\definecolor{currentstroke}{rgb}{0.000000,0.000000,0.000000}%
\pgfsetstrokecolor{currentstroke}%
\pgfsetdash{}{0pt}%
\pgfpathmoveto{\pgfqpoint{0.625000in}{3.520000in}}%
\pgfpathlineto{\pgfqpoint{4.500000in}{3.520000in}}%
\pgfusepath{stroke}%
\end{pgfscope}%
\begin{pgfscope}%
\pgfsetbuttcap%
\pgfsetmiterjoin%
\definecolor{currentfill}{rgb}{1.000000,1.000000,1.000000}%
\pgfsetfillcolor{currentfill}%
\pgfsetfillopacity{0.800000}%
\pgfsetlinewidth{1.003750pt}%
\definecolor{currentstroke}{rgb}{0.800000,0.800000,0.800000}%
\pgfsetstrokecolor{currentstroke}%
\pgfsetstrokeopacity{0.800000}%
\pgfsetdash{}{0pt}%
\pgfpathmoveto{\pgfqpoint{2.423855in}{3.199199in}}%
\pgfpathlineto{\pgfqpoint{4.402778in}{3.199199in}}%
\pgfpathquadraticcurveto{\pgfqpoint{4.430556in}{3.199199in}}{\pgfqpoint{4.430556in}{3.226977in}}%
\pgfpathlineto{\pgfqpoint{4.430556in}{3.422778in}}%
\pgfpathquadraticcurveto{\pgfqpoint{4.430556in}{3.450556in}}{\pgfqpoint{4.402778in}{3.450556in}}%
\pgfpathlineto{\pgfqpoint{2.423855in}{3.450556in}}%
\pgfpathquadraticcurveto{\pgfqpoint{2.396077in}{3.450556in}}{\pgfqpoint{2.396077in}{3.422778in}}%
\pgfpathlineto{\pgfqpoint{2.396077in}{3.226977in}}%
\pgfpathquadraticcurveto{\pgfqpoint{2.396077in}{3.199199in}}{\pgfqpoint{2.423855in}{3.199199in}}%
\pgfpathclose%
\pgfusepath{stroke,fill}%
\end{pgfscope}%
\begin{pgfscope}%
\pgfsetbuttcap%
\pgfsetroundjoin%
\pgfsetlinewidth{1.505625pt}%
\definecolor{currentstroke}{rgb}{0.000000,0.000000,0.000000}%
\pgfsetstrokecolor{currentstroke}%
\pgfsetdash{{5.550000pt}{2.400000pt}}{0.000000pt}%
\pgfpathmoveto{\pgfqpoint{2.451633in}{3.338088in}}%
\pgfpathlineto{\pgfqpoint{2.729411in}{3.338088in}}%
\pgfusepath{stroke}%
\end{pgfscope}%
\begin{pgfscope}%
\definecolor{textcolor}{rgb}{0.000000,0.000000,0.000000}%
\pgfsetstrokecolor{textcolor}%
\pgfsetfillcolor{textcolor}%
\pgftext[x=2.840522in,y=3.289477in,left,base]{\color{textcolor}\sffamily\fontsize{10.000000}{12.000000}\selectfont \(\displaystyle \alpha = 1.3794 - 0.1897\mathrm{ln}(k)\)}%
\end{pgfscope}%
\end{pgfpicture}%
\makeatother%
\endgroup%

    \end{subfigure}

\caption[The computed Quasi-Monte-Carlo convergence rate for $Q(u) =  u(\bzero)$.]{The computed values of $C$ (top) and $\alpha$ (bottom) against $k$ in \cref{eq:qmcerrorform} for $Q(u) =  u(\bzero)$. Observe the $x$-axes are on a $\log_{10}$ scale, but $\loge$ is the natural logarithm. \label{fig:originCalpha}}
\end{figure}

\begin{figure}[h]
    \centering
    \begin{subfigure}{\textwidth}
            \centering
%% Creator: Matplotlib, PGF backend
%%
%% To include the figure in your LaTeX document, write
%%   \input{<filename>.pgf}
%%
%% Make sure the required packages are loaded in your preamble
%%   \usepackage{pgf}
%%
%% Figures using additional raster images can only be included by \input if
%% they are in the same directory as the main LaTeX file. For loading figures
%% from other directories you can use the `import` package
%%   \usepackage{import}
%% and then include the figures with
%%   \import{<path to file>}{<filename>.pgf}
%%
%% Matplotlib used the following preamble
%%   \usepackage{fontspec}
%%   \setmainfont{DejaVuSerif.ttf}[Path=/home/owen/progs/firedrake-complex/firedrake/lib/python3.5/site-packages/matplotlib/mpl-data/fonts/ttf/]
%%   \setsansfont{DejaVuSans.ttf}[Path=/home/owen/progs/firedrake-complex/firedrake/lib/python3.5/site-packages/matplotlib/mpl-data/fonts/ttf/]
%%   \setmonofont{DejaVuSansMono.ttf}[Path=/home/owen/progs/firedrake-complex/firedrake/lib/python3.5/site-packages/matplotlib/mpl-data/fonts/ttf/]
%%
\begingroup%
\makeatletter%
\begin{pgfpicture}%
\pgfpathrectangle{\pgfpointorigin}{\pgfqpoint{5.000000in}{4.000000in}}%
\pgfusepath{use as bounding box, clip}%
\begin{pgfscope}%
\pgfsetbuttcap%
\pgfsetmiterjoin%
\definecolor{currentfill}{rgb}{1.000000,1.000000,1.000000}%
\pgfsetfillcolor{currentfill}%
\pgfsetlinewidth{0.000000pt}%
\definecolor{currentstroke}{rgb}{1.000000,1.000000,1.000000}%
\pgfsetstrokecolor{currentstroke}%
\pgfsetdash{}{0pt}%
\pgfpathmoveto{\pgfqpoint{0.000000in}{0.000000in}}%
\pgfpathlineto{\pgfqpoint{5.000000in}{0.000000in}}%
\pgfpathlineto{\pgfqpoint{5.000000in}{4.000000in}}%
\pgfpathlineto{\pgfqpoint{0.000000in}{4.000000in}}%
\pgfpathclose%
\pgfusepath{fill}%
\end{pgfscope}%
\begin{pgfscope}%
\pgfsetbuttcap%
\pgfsetmiterjoin%
\definecolor{currentfill}{rgb}{1.000000,1.000000,1.000000}%
\pgfsetfillcolor{currentfill}%
\pgfsetlinewidth{0.000000pt}%
\definecolor{currentstroke}{rgb}{0.000000,0.000000,0.000000}%
\pgfsetstrokecolor{currentstroke}%
\pgfsetstrokeopacity{0.000000}%
\pgfsetdash{}{0pt}%
\pgfpathmoveto{\pgfqpoint{0.688770in}{0.582778in}}%
\pgfpathlineto{\pgfqpoint{4.810222in}{0.582778in}}%
\pgfpathlineto{\pgfqpoint{4.810222in}{3.815000in}}%
\pgfpathlineto{\pgfqpoint{0.688770in}{3.815000in}}%
\pgfpathclose%
\pgfusepath{fill}%
\end{pgfscope}%
\begin{pgfscope}%
\pgfsetbuttcap%
\pgfsetroundjoin%
\definecolor{currentfill}{rgb}{0.000000,0.000000,0.000000}%
\pgfsetfillcolor{currentfill}%
\pgfsetlinewidth{0.803000pt}%
\definecolor{currentstroke}{rgb}{0.000000,0.000000,0.000000}%
\pgfsetstrokecolor{currentstroke}%
\pgfsetdash{}{0pt}%
\pgfsys@defobject{currentmarker}{\pgfqpoint{0.000000in}{-0.048611in}}{\pgfqpoint{0.000000in}{0.000000in}}{%
\pgfpathmoveto{\pgfqpoint{0.000000in}{0.000000in}}%
\pgfpathlineto{\pgfqpoint{0.000000in}{-0.048611in}}%
\pgfusepath{stroke,fill}%
}%
\begin{pgfscope}%
\pgfsys@transformshift{0.876108in}{0.582778in}%
\pgfsys@useobject{currentmarker}{}%
\end{pgfscope}%
\end{pgfscope}%
\begin{pgfscope}%
\definecolor{textcolor}{rgb}{0.000000,0.000000,0.000000}%
\pgfsetstrokecolor{textcolor}%
\pgfsetfillcolor{textcolor}%
\pgftext[x=0.876108in,y=0.485556in,,top]{\color{textcolor}\sffamily\fontsize{10.000000}{12.000000}\selectfont \(\displaystyle 10^{1}\)}%
\end{pgfscope}%
\begin{pgfscope}%
\pgfsetbuttcap%
\pgfsetroundjoin%
\definecolor{currentfill}{rgb}{0.000000,0.000000,0.000000}%
\pgfsetfillcolor{currentfill}%
\pgfsetlinewidth{0.602250pt}%
\definecolor{currentstroke}{rgb}{0.000000,0.000000,0.000000}%
\pgfsetstrokecolor{currentstroke}%
\pgfsetdash{}{0pt}%
\pgfsys@defobject{currentmarker}{\pgfqpoint{0.000000in}{-0.027778in}}{\pgfqpoint{0.000000in}{0.000000in}}{%
\pgfpathmoveto{\pgfqpoint{0.000000in}{0.000000in}}%
\pgfpathlineto{\pgfqpoint{0.000000in}{-0.027778in}}%
\pgfusepath{stroke,fill}%
}%
\begin{pgfscope}%
\pgfsys@transformshift{2.325559in}{0.582778in}%
\pgfsys@useobject{currentmarker}{}%
\end{pgfscope}%
\end{pgfscope}%
\begin{pgfscope}%
\definecolor{textcolor}{rgb}{0.000000,0.000000,0.000000}%
\pgfsetstrokecolor{textcolor}%
\pgfsetfillcolor{textcolor}%
\pgftext[x=2.325559in,y=0.507778in,,top]{\color{textcolor}\sffamily\fontsize{10.000000}{12.000000}\selectfont \(\displaystyle 2\times10^{1}\)}%
\end{pgfscope}%
\begin{pgfscope}%
\pgfsetbuttcap%
\pgfsetroundjoin%
\definecolor{currentfill}{rgb}{0.000000,0.000000,0.000000}%
\pgfsetfillcolor{currentfill}%
\pgfsetlinewidth{0.602250pt}%
\definecolor{currentstroke}{rgb}{0.000000,0.000000,0.000000}%
\pgfsetstrokecolor{currentstroke}%
\pgfsetdash{}{0pt}%
\pgfsys@defobject{currentmarker}{\pgfqpoint{0.000000in}{-0.027778in}}{\pgfqpoint{0.000000in}{0.000000in}}{%
\pgfpathmoveto{\pgfqpoint{0.000000in}{0.000000in}}%
\pgfpathlineto{\pgfqpoint{0.000000in}{-0.027778in}}%
\pgfusepath{stroke,fill}%
}%
\begin{pgfscope}%
\pgfsys@transformshift{3.173433in}{0.582778in}%
\pgfsys@useobject{currentmarker}{}%
\end{pgfscope}%
\end{pgfscope}%
\begin{pgfscope}%
\definecolor{textcolor}{rgb}{0.000000,0.000000,0.000000}%
\pgfsetstrokecolor{textcolor}%
\pgfsetfillcolor{textcolor}%
\pgftext[x=3.173433in,y=0.507778in,,top]{\color{textcolor}\sffamily\fontsize{10.000000}{12.000000}\selectfont \(\displaystyle 3\times10^{1}\)}%
\end{pgfscope}%
\begin{pgfscope}%
\pgfsetbuttcap%
\pgfsetroundjoin%
\definecolor{currentfill}{rgb}{0.000000,0.000000,0.000000}%
\pgfsetfillcolor{currentfill}%
\pgfsetlinewidth{0.602250pt}%
\definecolor{currentstroke}{rgb}{0.000000,0.000000,0.000000}%
\pgfsetstrokecolor{currentstroke}%
\pgfsetdash{}{0pt}%
\pgfsys@defobject{currentmarker}{\pgfqpoint{0.000000in}{-0.027778in}}{\pgfqpoint{0.000000in}{0.000000in}}{%
\pgfpathmoveto{\pgfqpoint{0.000000in}{0.000000in}}%
\pgfpathlineto{\pgfqpoint{0.000000in}{-0.027778in}}%
\pgfusepath{stroke,fill}%
}%
\begin{pgfscope}%
\pgfsys@transformshift{3.775009in}{0.582778in}%
\pgfsys@useobject{currentmarker}{}%
\end{pgfscope}%
\end{pgfscope}%
\begin{pgfscope}%
\definecolor{textcolor}{rgb}{0.000000,0.000000,0.000000}%
\pgfsetstrokecolor{textcolor}%
\pgfsetfillcolor{textcolor}%
\pgftext[x=3.775009in,y=0.507778in,,top]{\color{textcolor}\sffamily\fontsize{10.000000}{12.000000}\selectfont \(\displaystyle 4\times10^{1}\)}%
\end{pgfscope}%
\begin{pgfscope}%
\pgfsetbuttcap%
\pgfsetroundjoin%
\definecolor{currentfill}{rgb}{0.000000,0.000000,0.000000}%
\pgfsetfillcolor{currentfill}%
\pgfsetlinewidth{0.602250pt}%
\definecolor{currentstroke}{rgb}{0.000000,0.000000,0.000000}%
\pgfsetstrokecolor{currentstroke}%
\pgfsetdash{}{0pt}%
\pgfsys@defobject{currentmarker}{\pgfqpoint{0.000000in}{-0.027778in}}{\pgfqpoint{0.000000in}{0.000000in}}{%
\pgfpathmoveto{\pgfqpoint{0.000000in}{0.000000in}}%
\pgfpathlineto{\pgfqpoint{0.000000in}{-0.027778in}}%
\pgfusepath{stroke,fill}%
}%
\begin{pgfscope}%
\pgfsys@transformshift{4.241628in}{0.582778in}%
\pgfsys@useobject{currentmarker}{}%
\end{pgfscope}%
\end{pgfscope}%
\begin{pgfscope}%
\pgfsetbuttcap%
\pgfsetroundjoin%
\definecolor{currentfill}{rgb}{0.000000,0.000000,0.000000}%
\pgfsetfillcolor{currentfill}%
\pgfsetlinewidth{0.602250pt}%
\definecolor{currentstroke}{rgb}{0.000000,0.000000,0.000000}%
\pgfsetstrokecolor{currentstroke}%
\pgfsetdash{}{0pt}%
\pgfsys@defobject{currentmarker}{\pgfqpoint{0.000000in}{-0.027778in}}{\pgfqpoint{0.000000in}{0.000000in}}{%
\pgfpathmoveto{\pgfqpoint{0.000000in}{0.000000in}}%
\pgfpathlineto{\pgfqpoint{0.000000in}{-0.027778in}}%
\pgfusepath{stroke,fill}%
}%
\begin{pgfscope}%
\pgfsys@transformshift{4.622883in}{0.582778in}%
\pgfsys@useobject{currentmarker}{}%
\end{pgfscope}%
\end{pgfscope}%
\begin{pgfscope}%
\definecolor{textcolor}{rgb}{0.000000,0.000000,0.000000}%
\pgfsetstrokecolor{textcolor}%
\pgfsetfillcolor{textcolor}%
\pgftext[x=4.622883in,y=0.507778in,,top]{\color{textcolor}\sffamily\fontsize{10.000000}{12.000000}\selectfont \(\displaystyle 6\times10^{1}\)}%
\end{pgfscope}%
\begin{pgfscope}%
\definecolor{textcolor}{rgb}{0.000000,0.000000,0.000000}%
\pgfsetstrokecolor{textcolor}%
\pgfsetfillcolor{textcolor}%
\pgftext[x=2.749496in,y=0.295587in,,top]{\color{textcolor}\sffamily\fontsize{10.000000}{12.000000}\selectfont \(\displaystyle k\)}%
\end{pgfscope}%
\begin{pgfscope}%
\pgfsetbuttcap%
\pgfsetroundjoin%
\definecolor{currentfill}{rgb}{0.000000,0.000000,0.000000}%
\pgfsetfillcolor{currentfill}%
\pgfsetlinewidth{0.803000pt}%
\definecolor{currentstroke}{rgb}{0.000000,0.000000,0.000000}%
\pgfsetstrokecolor{currentstroke}%
\pgfsetdash{}{0pt}%
\pgfsys@defobject{currentmarker}{\pgfqpoint{-0.048611in}{0.000000in}}{\pgfqpoint{0.000000in}{0.000000in}}{%
\pgfpathmoveto{\pgfqpoint{0.000000in}{0.000000in}}%
\pgfpathlineto{\pgfqpoint{-0.048611in}{0.000000in}}%
\pgfusepath{stroke,fill}%
}%
\begin{pgfscope}%
\pgfsys@transformshift{0.688770in}{0.934647in}%
\pgfsys@useobject{currentmarker}{}%
\end{pgfscope}%
\end{pgfscope}%
\begin{pgfscope}%
\definecolor{textcolor}{rgb}{0.000000,0.000000,0.000000}%
\pgfsetstrokecolor{textcolor}%
\pgfsetfillcolor{textcolor}%
\pgftext[x=0.344633in,y=0.881886in,left,base]{\color{textcolor}\sffamily\fontsize{10.000000}{12.000000}\selectfont \(\displaystyle 0.10\)}%
\end{pgfscope}%
\begin{pgfscope}%
\pgfsetbuttcap%
\pgfsetroundjoin%
\definecolor{currentfill}{rgb}{0.000000,0.000000,0.000000}%
\pgfsetfillcolor{currentfill}%
\pgfsetlinewidth{0.803000pt}%
\definecolor{currentstroke}{rgb}{0.000000,0.000000,0.000000}%
\pgfsetstrokecolor{currentstroke}%
\pgfsetdash{}{0pt}%
\pgfsys@defobject{currentmarker}{\pgfqpoint{-0.048611in}{0.000000in}}{\pgfqpoint{0.000000in}{0.000000in}}{%
\pgfpathmoveto{\pgfqpoint{0.000000in}{0.000000in}}%
\pgfpathlineto{\pgfqpoint{-0.048611in}{0.000000in}}%
\pgfusepath{stroke,fill}%
}%
\begin{pgfscope}%
\pgfsys@transformshift{0.688770in}{1.553460in}%
\pgfsys@useobject{currentmarker}{}%
\end{pgfscope}%
\end{pgfscope}%
\begin{pgfscope}%
\definecolor{textcolor}{rgb}{0.000000,0.000000,0.000000}%
\pgfsetstrokecolor{textcolor}%
\pgfsetfillcolor{textcolor}%
\pgftext[x=0.344633in,y=1.500698in,left,base]{\color{textcolor}\sffamily\fontsize{10.000000}{12.000000}\selectfont \(\displaystyle 0.12\)}%
\end{pgfscope}%
\begin{pgfscope}%
\pgfsetbuttcap%
\pgfsetroundjoin%
\definecolor{currentfill}{rgb}{0.000000,0.000000,0.000000}%
\pgfsetfillcolor{currentfill}%
\pgfsetlinewidth{0.803000pt}%
\definecolor{currentstroke}{rgb}{0.000000,0.000000,0.000000}%
\pgfsetstrokecolor{currentstroke}%
\pgfsetdash{}{0pt}%
\pgfsys@defobject{currentmarker}{\pgfqpoint{-0.048611in}{0.000000in}}{\pgfqpoint{0.000000in}{0.000000in}}{%
\pgfpathmoveto{\pgfqpoint{0.000000in}{0.000000in}}%
\pgfpathlineto{\pgfqpoint{-0.048611in}{0.000000in}}%
\pgfusepath{stroke,fill}%
}%
\begin{pgfscope}%
\pgfsys@transformshift{0.688770in}{2.172272in}%
\pgfsys@useobject{currentmarker}{}%
\end{pgfscope}%
\end{pgfscope}%
\begin{pgfscope}%
\definecolor{textcolor}{rgb}{0.000000,0.000000,0.000000}%
\pgfsetstrokecolor{textcolor}%
\pgfsetfillcolor{textcolor}%
\pgftext[x=0.344633in,y=2.119510in,left,base]{\color{textcolor}\sffamily\fontsize{10.000000}{12.000000}\selectfont \(\displaystyle 0.14\)}%
\end{pgfscope}%
\begin{pgfscope}%
\pgfsetbuttcap%
\pgfsetroundjoin%
\definecolor{currentfill}{rgb}{0.000000,0.000000,0.000000}%
\pgfsetfillcolor{currentfill}%
\pgfsetlinewidth{0.803000pt}%
\definecolor{currentstroke}{rgb}{0.000000,0.000000,0.000000}%
\pgfsetstrokecolor{currentstroke}%
\pgfsetdash{}{0pt}%
\pgfsys@defobject{currentmarker}{\pgfqpoint{-0.048611in}{0.000000in}}{\pgfqpoint{0.000000in}{0.000000in}}{%
\pgfpathmoveto{\pgfqpoint{0.000000in}{0.000000in}}%
\pgfpathlineto{\pgfqpoint{-0.048611in}{0.000000in}}%
\pgfusepath{stroke,fill}%
}%
\begin{pgfscope}%
\pgfsys@transformshift{0.688770in}{2.791084in}%
\pgfsys@useobject{currentmarker}{}%
\end{pgfscope}%
\end{pgfscope}%
\begin{pgfscope}%
\definecolor{textcolor}{rgb}{0.000000,0.000000,0.000000}%
\pgfsetstrokecolor{textcolor}%
\pgfsetfillcolor{textcolor}%
\pgftext[x=0.344633in,y=2.738322in,left,base]{\color{textcolor}\sffamily\fontsize{10.000000}{12.000000}\selectfont \(\displaystyle 0.16\)}%
\end{pgfscope}%
\begin{pgfscope}%
\pgfsetbuttcap%
\pgfsetroundjoin%
\definecolor{currentfill}{rgb}{0.000000,0.000000,0.000000}%
\pgfsetfillcolor{currentfill}%
\pgfsetlinewidth{0.803000pt}%
\definecolor{currentstroke}{rgb}{0.000000,0.000000,0.000000}%
\pgfsetstrokecolor{currentstroke}%
\pgfsetdash{}{0pt}%
\pgfsys@defobject{currentmarker}{\pgfqpoint{-0.048611in}{0.000000in}}{\pgfqpoint{0.000000in}{0.000000in}}{%
\pgfpathmoveto{\pgfqpoint{0.000000in}{0.000000in}}%
\pgfpathlineto{\pgfqpoint{-0.048611in}{0.000000in}}%
\pgfusepath{stroke,fill}%
}%
\begin{pgfscope}%
\pgfsys@transformshift{0.688770in}{3.409896in}%
\pgfsys@useobject{currentmarker}{}%
\end{pgfscope}%
\end{pgfscope}%
\begin{pgfscope}%
\definecolor{textcolor}{rgb}{0.000000,0.000000,0.000000}%
\pgfsetstrokecolor{textcolor}%
\pgfsetfillcolor{textcolor}%
\pgftext[x=0.344633in,y=3.357135in,left,base]{\color{textcolor}\sffamily\fontsize{10.000000}{12.000000}\selectfont \(\displaystyle 0.18\)}%
\end{pgfscope}%
\begin{pgfscope}%
\definecolor{textcolor}{rgb}{0.000000,0.000000,0.000000}%
\pgfsetstrokecolor{textcolor}%
\pgfsetfillcolor{textcolor}%
\pgftext[x=0.289078in,y=2.198889in,,bottom,rotate=90.000000]{\color{textcolor}\sffamily\fontsize{10.000000}{12.000000}\selectfont \(\displaystyle C\)}%
\end{pgfscope}%
\begin{pgfscope}%
\pgfpathrectangle{\pgfqpoint{0.688770in}{0.582778in}}{\pgfqpoint{4.121452in}{3.232222in}}%
\pgfusepath{clip}%
\pgfsetbuttcap%
\pgfsetroundjoin%
\definecolor{currentfill}{rgb}{0.000000,0.000000,0.000000}%
\pgfsetfillcolor{currentfill}%
\pgfsetlinewidth{1.003750pt}%
\definecolor{currentstroke}{rgb}{0.000000,0.000000,0.000000}%
\pgfsetstrokecolor{currentstroke}%
\pgfsetdash{}{0pt}%
\pgfsys@defobject{currentmarker}{\pgfqpoint{-0.041667in}{-0.041667in}}{\pgfqpoint{0.041667in}{0.041667in}}{%
\pgfpathmoveto{\pgfqpoint{0.000000in}{-0.041667in}}%
\pgfpathcurveto{\pgfqpoint{0.011050in}{-0.041667in}}{\pgfqpoint{0.021649in}{-0.037276in}}{\pgfqpoint{0.029463in}{-0.029463in}}%
\pgfpathcurveto{\pgfqpoint{0.037276in}{-0.021649in}}{\pgfqpoint{0.041667in}{-0.011050in}}{\pgfqpoint{0.041667in}{0.000000in}}%
\pgfpathcurveto{\pgfqpoint{0.041667in}{0.011050in}}{\pgfqpoint{0.037276in}{0.021649in}}{\pgfqpoint{0.029463in}{0.029463in}}%
\pgfpathcurveto{\pgfqpoint{0.021649in}{0.037276in}}{\pgfqpoint{0.011050in}{0.041667in}}{\pgfqpoint{0.000000in}{0.041667in}}%
\pgfpathcurveto{\pgfqpoint{-0.011050in}{0.041667in}}{\pgfqpoint{-0.021649in}{0.037276in}}{\pgfqpoint{-0.029463in}{0.029463in}}%
\pgfpathcurveto{\pgfqpoint{-0.037276in}{0.021649in}}{\pgfqpoint{-0.041667in}{0.011050in}}{\pgfqpoint{-0.041667in}{0.000000in}}%
\pgfpathcurveto{\pgfqpoint{-0.041667in}{-0.011050in}}{\pgfqpoint{-0.037276in}{-0.021649in}}{\pgfqpoint{-0.029463in}{-0.029463in}}%
\pgfpathcurveto{\pgfqpoint{-0.021649in}{-0.037276in}}{\pgfqpoint{-0.011050in}{-0.041667in}}{\pgfqpoint{0.000000in}{-0.041667in}}%
\pgfpathclose%
\pgfusepath{stroke,fill}%
}%
\begin{pgfscope}%
\pgfsys@transformshift{0.876108in}{0.729697in}%
\pgfsys@useobject{currentmarker}{}%
\end{pgfscope}%
\begin{pgfscope}%
\pgfsys@transformshift{2.325559in}{1.966349in}%
\pgfsys@useobject{currentmarker}{}%
\end{pgfscope}%
\begin{pgfscope}%
\pgfsys@transformshift{3.173433in}{3.441230in}%
\pgfsys@useobject{currentmarker}{}%
\end{pgfscope}%
\begin{pgfscope}%
\pgfsys@transformshift{3.775009in}{2.830242in}%
\pgfsys@useobject{currentmarker}{}%
\end{pgfscope}%
\begin{pgfscope}%
\pgfsys@transformshift{4.241628in}{3.668081in}%
\pgfsys@useobject{currentmarker}{}%
\end{pgfscope}%
\begin{pgfscope}%
\pgfsys@transformshift{4.622883in}{2.286315in}%
\pgfsys@useobject{currentmarker}{}%
\end{pgfscope}%
\end{pgfscope}%
\begin{pgfscope}%
\pgfsetrectcap%
\pgfsetmiterjoin%
\pgfsetlinewidth{0.803000pt}%
\definecolor{currentstroke}{rgb}{0.000000,0.000000,0.000000}%
\pgfsetstrokecolor{currentstroke}%
\pgfsetdash{}{0pt}%
\pgfpathmoveto{\pgfqpoint{0.688770in}{0.582778in}}%
\pgfpathlineto{\pgfqpoint{0.688770in}{3.815000in}}%
\pgfusepath{stroke}%
\end{pgfscope}%
\begin{pgfscope}%
\pgfsetrectcap%
\pgfsetmiterjoin%
\pgfsetlinewidth{0.803000pt}%
\definecolor{currentstroke}{rgb}{0.000000,0.000000,0.000000}%
\pgfsetstrokecolor{currentstroke}%
\pgfsetdash{}{0pt}%
\pgfpathmoveto{\pgfqpoint{4.810222in}{0.582778in}}%
\pgfpathlineto{\pgfqpoint{4.810222in}{3.815000in}}%
\pgfusepath{stroke}%
\end{pgfscope}%
\begin{pgfscope}%
\pgfsetrectcap%
\pgfsetmiterjoin%
\pgfsetlinewidth{0.803000pt}%
\definecolor{currentstroke}{rgb}{0.000000,0.000000,0.000000}%
\pgfsetstrokecolor{currentstroke}%
\pgfsetdash{}{0pt}%
\pgfpathmoveto{\pgfqpoint{0.688770in}{0.582778in}}%
\pgfpathlineto{\pgfqpoint{4.810222in}{0.582778in}}%
\pgfusepath{stroke}%
\end{pgfscope}%
\begin{pgfscope}%
\pgfsetrectcap%
\pgfsetmiterjoin%
\pgfsetlinewidth{0.803000pt}%
\definecolor{currentstroke}{rgb}{0.000000,0.000000,0.000000}%
\pgfsetstrokecolor{currentstroke}%
\pgfsetdash{}{0pt}%
\pgfpathmoveto{\pgfqpoint{0.688770in}{3.815000in}}%
\pgfpathlineto{\pgfqpoint{4.810222in}{3.815000in}}%
\pgfusepath{stroke}%
\end{pgfscope}%
\end{pgfpicture}%
\makeatother%
\endgroup%

  \end{subfigure}
    \begin{subfigure}{\textwidth}
            \centering
%% Creator: Matplotlib, PGF backend
%%
%% To include the figure in your LaTeX document, write
%%   \input{<filename>.pgf}
%%
%% Make sure the required packages are loaded in your preamble
%%   \usepackage{pgf}
%%
%% Figures using additional raster images can only be included by \input if
%% they are in the same directory as the main LaTeX file. For loading figures
%% from other directories you can use the `import` package
%%   \usepackage{import}
%% and then include the figures with
%%   \import{<path to file>}{<filename>.pgf}
%%
%% Matplotlib used the following preamble
%%   \usepackage{fontspec}
%%   \setmainfont{DejaVuSerif.ttf}[Path=/home/owen/progs/firedrake-complex/firedrake/lib/python3.5/site-packages/matplotlib/mpl-data/fonts/ttf/]
%%   \setsansfont{DejaVuSans.ttf}[Path=/home/owen/progs/firedrake-complex/firedrake/lib/python3.5/site-packages/matplotlib/mpl-data/fonts/ttf/]
%%   \setmonofont{DejaVuSansMono.ttf}[Path=/home/owen/progs/firedrake-complex/firedrake/lib/python3.5/site-packages/matplotlib/mpl-data/fonts/ttf/]
%%
\begingroup%
\makeatletter%
\begin{pgfpicture}%
\pgfpathrectangle{\pgfpointorigin}{\pgfqpoint{5.000000in}{4.000000in}}%
\pgfusepath{use as bounding box, clip}%
\begin{pgfscope}%
\pgfsetbuttcap%
\pgfsetmiterjoin%
\definecolor{currentfill}{rgb}{1.000000,1.000000,1.000000}%
\pgfsetfillcolor{currentfill}%
\pgfsetlinewidth{0.000000pt}%
\definecolor{currentstroke}{rgb}{1.000000,1.000000,1.000000}%
\pgfsetstrokecolor{currentstroke}%
\pgfsetdash{}{0pt}%
\pgfpathmoveto{\pgfqpoint{0.000000in}{0.000000in}}%
\pgfpathlineto{\pgfqpoint{5.000000in}{0.000000in}}%
\pgfpathlineto{\pgfqpoint{5.000000in}{4.000000in}}%
\pgfpathlineto{\pgfqpoint{0.000000in}{4.000000in}}%
\pgfpathclose%
\pgfusepath{fill}%
\end{pgfscope}%
\begin{pgfscope}%
\pgfsetbuttcap%
\pgfsetmiterjoin%
\definecolor{currentfill}{rgb}{1.000000,1.000000,1.000000}%
\pgfsetfillcolor{currentfill}%
\pgfsetlinewidth{0.000000pt}%
\definecolor{currentstroke}{rgb}{0.000000,0.000000,0.000000}%
\pgfsetstrokecolor{currentstroke}%
\pgfsetstrokeopacity{0.000000}%
\pgfsetdash{}{0pt}%
\pgfpathmoveto{\pgfqpoint{0.625000in}{0.440000in}}%
\pgfpathlineto{\pgfqpoint{4.500000in}{0.440000in}}%
\pgfpathlineto{\pgfqpoint{4.500000in}{3.520000in}}%
\pgfpathlineto{\pgfqpoint{0.625000in}{3.520000in}}%
\pgfpathclose%
\pgfusepath{fill}%
\end{pgfscope}%
\begin{pgfscope}%
\pgfsetbuttcap%
\pgfsetroundjoin%
\definecolor{currentfill}{rgb}{0.000000,0.000000,0.000000}%
\pgfsetfillcolor{currentfill}%
\pgfsetlinewidth{0.803000pt}%
\definecolor{currentstroke}{rgb}{0.000000,0.000000,0.000000}%
\pgfsetstrokecolor{currentstroke}%
\pgfsetdash{}{0pt}%
\pgfsys@defobject{currentmarker}{\pgfqpoint{0.000000in}{-0.048611in}}{\pgfqpoint{0.000000in}{0.000000in}}{%
\pgfpathmoveto{\pgfqpoint{0.000000in}{0.000000in}}%
\pgfpathlineto{\pgfqpoint{0.000000in}{-0.048611in}}%
\pgfusepath{stroke,fill}%
}%
\begin{pgfscope}%
\pgfsys@transformshift{0.801136in}{0.440000in}%
\pgfsys@useobject{currentmarker}{}%
\end{pgfscope}%
\end{pgfscope}%
\begin{pgfscope}%
\definecolor{textcolor}{rgb}{0.000000,0.000000,0.000000}%
\pgfsetstrokecolor{textcolor}%
\pgfsetfillcolor{textcolor}%
\pgftext[x=0.801136in,y=0.342778in,,top]{\color{textcolor}\sffamily\fontsize{10.000000}{12.000000}\selectfont \(\displaystyle 10^{1}\)}%
\end{pgfscope}%
\begin{pgfscope}%
\pgfsetbuttcap%
\pgfsetroundjoin%
\definecolor{currentfill}{rgb}{0.000000,0.000000,0.000000}%
\pgfsetfillcolor{currentfill}%
\pgfsetlinewidth{0.602250pt}%
\definecolor{currentstroke}{rgb}{0.000000,0.000000,0.000000}%
\pgfsetstrokecolor{currentstroke}%
\pgfsetdash{}{0pt}%
\pgfsys@defobject{currentmarker}{\pgfqpoint{0.000000in}{-0.027778in}}{\pgfqpoint{0.000000in}{0.000000in}}{%
\pgfpathmoveto{\pgfqpoint{0.000000in}{0.000000in}}%
\pgfpathlineto{\pgfqpoint{0.000000in}{-0.027778in}}%
\pgfusepath{stroke,fill}%
}%
\begin{pgfscope}%
\pgfsys@transformshift{2.163913in}{0.440000in}%
\pgfsys@useobject{currentmarker}{}%
\end{pgfscope}%
\end{pgfscope}%
\begin{pgfscope}%
\definecolor{textcolor}{rgb}{0.000000,0.000000,0.000000}%
\pgfsetstrokecolor{textcolor}%
\pgfsetfillcolor{textcolor}%
\pgftext[x=2.163913in,y=0.365000in,,top]{\color{textcolor}\sffamily\fontsize{10.000000}{12.000000}\selectfont \(\displaystyle 2\times10^{1}\)}%
\end{pgfscope}%
\begin{pgfscope}%
\pgfsetbuttcap%
\pgfsetroundjoin%
\definecolor{currentfill}{rgb}{0.000000,0.000000,0.000000}%
\pgfsetfillcolor{currentfill}%
\pgfsetlinewidth{0.602250pt}%
\definecolor{currentstroke}{rgb}{0.000000,0.000000,0.000000}%
\pgfsetstrokecolor{currentstroke}%
\pgfsetdash{}{0pt}%
\pgfsys@defobject{currentmarker}{\pgfqpoint{0.000000in}{-0.027778in}}{\pgfqpoint{0.000000in}{0.000000in}}{%
\pgfpathmoveto{\pgfqpoint{0.000000in}{0.000000in}}%
\pgfpathlineto{\pgfqpoint{0.000000in}{-0.027778in}}%
\pgfusepath{stroke,fill}%
}%
\begin{pgfscope}%
\pgfsys@transformshift{2.961087in}{0.440000in}%
\pgfsys@useobject{currentmarker}{}%
\end{pgfscope}%
\end{pgfscope}%
\begin{pgfscope}%
\definecolor{textcolor}{rgb}{0.000000,0.000000,0.000000}%
\pgfsetstrokecolor{textcolor}%
\pgfsetfillcolor{textcolor}%
\pgftext[x=2.961087in,y=0.365000in,,top]{\color{textcolor}\sffamily\fontsize{10.000000}{12.000000}\selectfont \(\displaystyle 3\times10^{1}\)}%
\end{pgfscope}%
\begin{pgfscope}%
\pgfsetbuttcap%
\pgfsetroundjoin%
\definecolor{currentfill}{rgb}{0.000000,0.000000,0.000000}%
\pgfsetfillcolor{currentfill}%
\pgfsetlinewidth{0.602250pt}%
\definecolor{currentstroke}{rgb}{0.000000,0.000000,0.000000}%
\pgfsetstrokecolor{currentstroke}%
\pgfsetdash{}{0pt}%
\pgfsys@defobject{currentmarker}{\pgfqpoint{0.000000in}{-0.027778in}}{\pgfqpoint{0.000000in}{0.000000in}}{%
\pgfpathmoveto{\pgfqpoint{0.000000in}{0.000000in}}%
\pgfpathlineto{\pgfqpoint{0.000000in}{-0.027778in}}%
\pgfusepath{stroke,fill}%
}%
\begin{pgfscope}%
\pgfsys@transformshift{3.526690in}{0.440000in}%
\pgfsys@useobject{currentmarker}{}%
\end{pgfscope}%
\end{pgfscope}%
\begin{pgfscope}%
\definecolor{textcolor}{rgb}{0.000000,0.000000,0.000000}%
\pgfsetstrokecolor{textcolor}%
\pgfsetfillcolor{textcolor}%
\pgftext[x=3.526690in,y=0.365000in,,top]{\color{textcolor}\sffamily\fontsize{10.000000}{12.000000}\selectfont \(\displaystyle 4\times10^{1}\)}%
\end{pgfscope}%
\begin{pgfscope}%
\pgfsetbuttcap%
\pgfsetroundjoin%
\definecolor{currentfill}{rgb}{0.000000,0.000000,0.000000}%
\pgfsetfillcolor{currentfill}%
\pgfsetlinewidth{0.602250pt}%
\definecolor{currentstroke}{rgb}{0.000000,0.000000,0.000000}%
\pgfsetstrokecolor{currentstroke}%
\pgfsetdash{}{0pt}%
\pgfsys@defobject{currentmarker}{\pgfqpoint{0.000000in}{-0.027778in}}{\pgfqpoint{0.000000in}{0.000000in}}{%
\pgfpathmoveto{\pgfqpoint{0.000000in}{0.000000in}}%
\pgfpathlineto{\pgfqpoint{0.000000in}{-0.027778in}}%
\pgfusepath{stroke,fill}%
}%
\begin{pgfscope}%
\pgfsys@transformshift{3.965406in}{0.440000in}%
\pgfsys@useobject{currentmarker}{}%
\end{pgfscope}%
\end{pgfscope}%
\begin{pgfscope}%
\pgfsetbuttcap%
\pgfsetroundjoin%
\definecolor{currentfill}{rgb}{0.000000,0.000000,0.000000}%
\pgfsetfillcolor{currentfill}%
\pgfsetlinewidth{0.602250pt}%
\definecolor{currentstroke}{rgb}{0.000000,0.000000,0.000000}%
\pgfsetstrokecolor{currentstroke}%
\pgfsetdash{}{0pt}%
\pgfsys@defobject{currentmarker}{\pgfqpoint{0.000000in}{-0.027778in}}{\pgfqpoint{0.000000in}{0.000000in}}{%
\pgfpathmoveto{\pgfqpoint{0.000000in}{0.000000in}}%
\pgfpathlineto{\pgfqpoint{0.000000in}{-0.027778in}}%
\pgfusepath{stroke,fill}%
}%
\begin{pgfscope}%
\pgfsys@transformshift{4.323864in}{0.440000in}%
\pgfsys@useobject{currentmarker}{}%
\end{pgfscope}%
\end{pgfscope}%
\begin{pgfscope}%
\definecolor{textcolor}{rgb}{0.000000,0.000000,0.000000}%
\pgfsetstrokecolor{textcolor}%
\pgfsetfillcolor{textcolor}%
\pgftext[x=4.323864in,y=0.365000in,,top]{\color{textcolor}\sffamily\fontsize{10.000000}{12.000000}\selectfont \(\displaystyle 6\times10^{1}\)}%
\end{pgfscope}%
\begin{pgfscope}%
\definecolor{textcolor}{rgb}{0.000000,0.000000,0.000000}%
\pgfsetstrokecolor{textcolor}%
\pgfsetfillcolor{textcolor}%
\pgftext[x=2.562500in,y=0.152809in,,top]{\color{textcolor}\sffamily\fontsize{10.000000}{12.000000}\selectfont \(\displaystyle k\)}%
\end{pgfscope}%
\begin{pgfscope}%
\pgfsetbuttcap%
\pgfsetroundjoin%
\definecolor{currentfill}{rgb}{0.000000,0.000000,0.000000}%
\pgfsetfillcolor{currentfill}%
\pgfsetlinewidth{0.803000pt}%
\definecolor{currentstroke}{rgb}{0.000000,0.000000,0.000000}%
\pgfsetstrokecolor{currentstroke}%
\pgfsetdash{}{0pt}%
\pgfsys@defobject{currentmarker}{\pgfqpoint{-0.048611in}{0.000000in}}{\pgfqpoint{0.000000in}{0.000000in}}{%
\pgfpathmoveto{\pgfqpoint{0.000000in}{0.000000in}}%
\pgfpathlineto{\pgfqpoint{-0.048611in}{0.000000in}}%
\pgfusepath{stroke,fill}%
}%
\begin{pgfscope}%
\pgfsys@transformshift{0.625000in}{0.750342in}%
\pgfsys@useobject{currentmarker}{}%
\end{pgfscope}%
\end{pgfscope}%
\begin{pgfscope}%
\definecolor{textcolor}{rgb}{0.000000,0.000000,0.000000}%
\pgfsetstrokecolor{textcolor}%
\pgfsetfillcolor{textcolor}%
\pgftext[x=0.350308in,y=0.697581in,left,base]{\color{textcolor}\sffamily\fontsize{10.000000}{12.000000}\selectfont \(\displaystyle 0.6\)}%
\end{pgfscope}%
\begin{pgfscope}%
\pgfsetbuttcap%
\pgfsetroundjoin%
\definecolor{currentfill}{rgb}{0.000000,0.000000,0.000000}%
\pgfsetfillcolor{currentfill}%
\pgfsetlinewidth{0.803000pt}%
\definecolor{currentstroke}{rgb}{0.000000,0.000000,0.000000}%
\pgfsetstrokecolor{currentstroke}%
\pgfsetdash{}{0pt}%
\pgfsys@defobject{currentmarker}{\pgfqpoint{-0.048611in}{0.000000in}}{\pgfqpoint{0.000000in}{0.000000in}}{%
\pgfpathmoveto{\pgfqpoint{0.000000in}{0.000000in}}%
\pgfpathlineto{\pgfqpoint{-0.048611in}{0.000000in}}%
\pgfusepath{stroke,fill}%
}%
\begin{pgfscope}%
\pgfsys@transformshift{0.625000in}{1.353780in}%
\pgfsys@useobject{currentmarker}{}%
\end{pgfscope}%
\end{pgfscope}%
\begin{pgfscope}%
\definecolor{textcolor}{rgb}{0.000000,0.000000,0.000000}%
\pgfsetstrokecolor{textcolor}%
\pgfsetfillcolor{textcolor}%
\pgftext[x=0.350308in,y=1.301019in,left,base]{\color{textcolor}\sffamily\fontsize{10.000000}{12.000000}\selectfont \(\displaystyle 0.7\)}%
\end{pgfscope}%
\begin{pgfscope}%
\pgfsetbuttcap%
\pgfsetroundjoin%
\definecolor{currentfill}{rgb}{0.000000,0.000000,0.000000}%
\pgfsetfillcolor{currentfill}%
\pgfsetlinewidth{0.803000pt}%
\definecolor{currentstroke}{rgb}{0.000000,0.000000,0.000000}%
\pgfsetstrokecolor{currentstroke}%
\pgfsetdash{}{0pt}%
\pgfsys@defobject{currentmarker}{\pgfqpoint{-0.048611in}{0.000000in}}{\pgfqpoint{0.000000in}{0.000000in}}{%
\pgfpathmoveto{\pgfqpoint{0.000000in}{0.000000in}}%
\pgfpathlineto{\pgfqpoint{-0.048611in}{0.000000in}}%
\pgfusepath{stroke,fill}%
}%
\begin{pgfscope}%
\pgfsys@transformshift{0.625000in}{1.957218in}%
\pgfsys@useobject{currentmarker}{}%
\end{pgfscope}%
\end{pgfscope}%
\begin{pgfscope}%
\definecolor{textcolor}{rgb}{0.000000,0.000000,0.000000}%
\pgfsetstrokecolor{textcolor}%
\pgfsetfillcolor{textcolor}%
\pgftext[x=0.350308in,y=1.904456in,left,base]{\color{textcolor}\sffamily\fontsize{10.000000}{12.000000}\selectfont \(\displaystyle 0.8\)}%
\end{pgfscope}%
\begin{pgfscope}%
\pgfsetbuttcap%
\pgfsetroundjoin%
\definecolor{currentfill}{rgb}{0.000000,0.000000,0.000000}%
\pgfsetfillcolor{currentfill}%
\pgfsetlinewidth{0.803000pt}%
\definecolor{currentstroke}{rgb}{0.000000,0.000000,0.000000}%
\pgfsetstrokecolor{currentstroke}%
\pgfsetdash{}{0pt}%
\pgfsys@defobject{currentmarker}{\pgfqpoint{-0.048611in}{0.000000in}}{\pgfqpoint{0.000000in}{0.000000in}}{%
\pgfpathmoveto{\pgfqpoint{0.000000in}{0.000000in}}%
\pgfpathlineto{\pgfqpoint{-0.048611in}{0.000000in}}%
\pgfusepath{stroke,fill}%
}%
\begin{pgfscope}%
\pgfsys@transformshift{0.625000in}{2.560656in}%
\pgfsys@useobject{currentmarker}{}%
\end{pgfscope}%
\end{pgfscope}%
\begin{pgfscope}%
\definecolor{textcolor}{rgb}{0.000000,0.000000,0.000000}%
\pgfsetstrokecolor{textcolor}%
\pgfsetfillcolor{textcolor}%
\pgftext[x=0.350308in,y=2.507894in,left,base]{\color{textcolor}\sffamily\fontsize{10.000000}{12.000000}\selectfont \(\displaystyle 0.9\)}%
\end{pgfscope}%
\begin{pgfscope}%
\pgfsetbuttcap%
\pgfsetroundjoin%
\definecolor{currentfill}{rgb}{0.000000,0.000000,0.000000}%
\pgfsetfillcolor{currentfill}%
\pgfsetlinewidth{0.803000pt}%
\definecolor{currentstroke}{rgb}{0.000000,0.000000,0.000000}%
\pgfsetstrokecolor{currentstroke}%
\pgfsetdash{}{0pt}%
\pgfsys@defobject{currentmarker}{\pgfqpoint{-0.048611in}{0.000000in}}{\pgfqpoint{0.000000in}{0.000000in}}{%
\pgfpathmoveto{\pgfqpoint{0.000000in}{0.000000in}}%
\pgfpathlineto{\pgfqpoint{-0.048611in}{0.000000in}}%
\pgfusepath{stroke,fill}%
}%
\begin{pgfscope}%
\pgfsys@transformshift{0.625000in}{3.164093in}%
\pgfsys@useobject{currentmarker}{}%
\end{pgfscope}%
\end{pgfscope}%
\begin{pgfscope}%
\definecolor{textcolor}{rgb}{0.000000,0.000000,0.000000}%
\pgfsetstrokecolor{textcolor}%
\pgfsetfillcolor{textcolor}%
\pgftext[x=0.350308in,y=3.111332in,left,base]{\color{textcolor}\sffamily\fontsize{10.000000}{12.000000}\selectfont \(\displaystyle 1.0\)}%
\end{pgfscope}%
\begin{pgfscope}%
\definecolor{textcolor}{rgb}{0.000000,0.000000,0.000000}%
\pgfsetstrokecolor{textcolor}%
\pgfsetfillcolor{textcolor}%
\pgftext[x=0.294753in,y=1.980000in,,bottom,rotate=90.000000]{\color{textcolor}\sffamily\fontsize{10.000000}{12.000000}\selectfont \(\displaystyle \alpha\)}%
\end{pgfscope}%
\begin{pgfscope}%
\pgfpathrectangle{\pgfqpoint{0.625000in}{0.440000in}}{\pgfqpoint{3.875000in}{3.080000in}}%
\pgfusepath{clip}%
\pgfsetbuttcap%
\pgfsetroundjoin%
\pgfsetlinewidth{1.505625pt}%
\definecolor{currentstroke}{rgb}{0.000000,0.000000,0.000000}%
\pgfsetstrokecolor{currentstroke}%
\pgfsetdash{{5.550000pt}{2.400000pt}}{0.000000pt}%
\pgfpathmoveto{\pgfqpoint{0.801136in}{3.380000in}}%
\pgfpathlineto{\pgfqpoint{2.163913in}{2.516130in}}%
\pgfpathlineto{\pgfqpoint{2.961087in}{2.010799in}}%
\pgfpathlineto{\pgfqpoint{3.526690in}{1.652260in}}%
\pgfpathlineto{\pgfqpoint{3.965406in}{1.374156in}}%
\pgfpathlineto{\pgfqpoint{4.323864in}{1.146929in}}%
\pgfusepath{stroke}%
\end{pgfscope}%
\begin{pgfscope}%
\pgfpathrectangle{\pgfqpoint{0.625000in}{0.440000in}}{\pgfqpoint{3.875000in}{3.080000in}}%
\pgfusepath{clip}%
\pgfsetbuttcap%
\pgfsetroundjoin%
\definecolor{currentfill}{rgb}{0.000000,0.000000,0.000000}%
\pgfsetfillcolor{currentfill}%
\pgfsetlinewidth{1.003750pt}%
\definecolor{currentstroke}{rgb}{0.000000,0.000000,0.000000}%
\pgfsetstrokecolor{currentstroke}%
\pgfsetdash{}{0pt}%
\pgfsys@defobject{currentmarker}{\pgfqpoint{-0.041667in}{-0.041667in}}{\pgfqpoint{0.041667in}{0.041667in}}{%
\pgfpathmoveto{\pgfqpoint{0.000000in}{-0.041667in}}%
\pgfpathcurveto{\pgfqpoint{0.011050in}{-0.041667in}}{\pgfqpoint{0.021649in}{-0.037276in}}{\pgfqpoint{0.029463in}{-0.029463in}}%
\pgfpathcurveto{\pgfqpoint{0.037276in}{-0.021649in}}{\pgfqpoint{0.041667in}{-0.011050in}}{\pgfqpoint{0.041667in}{0.000000in}}%
\pgfpathcurveto{\pgfqpoint{0.041667in}{0.011050in}}{\pgfqpoint{0.037276in}{0.021649in}}{\pgfqpoint{0.029463in}{0.029463in}}%
\pgfpathcurveto{\pgfqpoint{0.021649in}{0.037276in}}{\pgfqpoint{0.011050in}{0.041667in}}{\pgfqpoint{0.000000in}{0.041667in}}%
\pgfpathcurveto{\pgfqpoint{-0.011050in}{0.041667in}}{\pgfqpoint{-0.021649in}{0.037276in}}{\pgfqpoint{-0.029463in}{0.029463in}}%
\pgfpathcurveto{\pgfqpoint{-0.037276in}{0.021649in}}{\pgfqpoint{-0.041667in}{0.011050in}}{\pgfqpoint{-0.041667in}{0.000000in}}%
\pgfpathcurveto{\pgfqpoint{-0.041667in}{-0.011050in}}{\pgfqpoint{-0.037276in}{-0.021649in}}{\pgfqpoint{-0.029463in}{-0.029463in}}%
\pgfpathcurveto{\pgfqpoint{-0.021649in}{-0.037276in}}{\pgfqpoint{-0.011050in}{-0.041667in}}{\pgfqpoint{0.000000in}{-0.041667in}}%
\pgfpathclose%
\pgfusepath{stroke,fill}%
}%
\begin{pgfscope}%
\pgfsys@transformshift{0.801136in}{3.062735in}%
\pgfsys@useobject{currentmarker}{}%
\end{pgfscope}%
\begin{pgfscope}%
\pgfsys@transformshift{2.163913in}{2.696609in}%
\pgfsys@useobject{currentmarker}{}%
\end{pgfscope}%
\begin{pgfscope}%
\pgfsys@transformshift{2.961087in}{2.475483in}%
\pgfsys@useobject{currentmarker}{}%
\end{pgfscope}%
\begin{pgfscope}%
\pgfsys@transformshift{3.526690in}{1.672474in}%
\pgfsys@useobject{currentmarker}{}%
\end{pgfscope}%
\begin{pgfscope}%
\pgfsys@transformshift{3.965406in}{1.592973in}%
\pgfsys@useobject{currentmarker}{}%
\end{pgfscope}%
\begin{pgfscope}%
\pgfsys@transformshift{4.323864in}{0.580000in}%
\pgfsys@useobject{currentmarker}{}%
\end{pgfscope}%
\end{pgfscope}%
\begin{pgfscope}%
\pgfsetrectcap%
\pgfsetmiterjoin%
\pgfsetlinewidth{0.803000pt}%
\definecolor{currentstroke}{rgb}{0.000000,0.000000,0.000000}%
\pgfsetstrokecolor{currentstroke}%
\pgfsetdash{}{0pt}%
\pgfpathmoveto{\pgfqpoint{0.625000in}{0.440000in}}%
\pgfpathlineto{\pgfqpoint{0.625000in}{3.520000in}}%
\pgfusepath{stroke}%
\end{pgfscope}%
\begin{pgfscope}%
\pgfsetrectcap%
\pgfsetmiterjoin%
\pgfsetlinewidth{0.000000pt}%
\definecolor{currentstroke}{rgb}{0.000000,0.000000,0.000000}%
\pgfsetstrokecolor{currentstroke}%
\pgfsetstrokeopacity{0.000000}%
\pgfsetdash{}{0pt}%
\pgfpathmoveto{\pgfqpoint{4.500000in}{0.440000in}}%
\pgfpathlineto{\pgfqpoint{4.500000in}{3.520000in}}%
\pgfusepath{}%
\end{pgfscope}%
\begin{pgfscope}%
\pgfsetrectcap%
\pgfsetmiterjoin%
\pgfsetlinewidth{0.803000pt}%
\definecolor{currentstroke}{rgb}{0.000000,0.000000,0.000000}%
\pgfsetstrokecolor{currentstroke}%
\pgfsetdash{}{0pt}%
\pgfpathmoveto{\pgfqpoint{0.625000in}{0.440000in}}%
\pgfpathlineto{\pgfqpoint{4.500000in}{0.440000in}}%
\pgfusepath{stroke}%
\end{pgfscope}%
\begin{pgfscope}%
\pgfsetrectcap%
\pgfsetmiterjoin%
\pgfsetlinewidth{0.000000pt}%
\definecolor{currentstroke}{rgb}{0.000000,0.000000,0.000000}%
\pgfsetstrokecolor{currentstroke}%
\pgfsetstrokeopacity{0.000000}%
\pgfsetdash{}{0pt}%
\pgfpathmoveto{\pgfqpoint{0.625000in}{3.520000in}}%
\pgfpathlineto{\pgfqpoint{4.500000in}{3.520000in}}%
\pgfusepath{}%
\end{pgfscope}%
\begin{pgfscope}%
\pgfsetbuttcap%
\pgfsetmiterjoin%
\definecolor{currentfill}{rgb}{1.000000,1.000000,1.000000}%
\pgfsetfillcolor{currentfill}%
\pgfsetfillopacity{0.800000}%
\pgfsetlinewidth{1.003750pt}%
\definecolor{currentstroke}{rgb}{0.800000,0.800000,0.800000}%
\pgfsetstrokecolor{currentstroke}%
\pgfsetstrokeopacity{0.800000}%
\pgfsetdash{}{0pt}%
\pgfpathmoveto{\pgfqpoint{2.423855in}{3.199199in}}%
\pgfpathlineto{\pgfqpoint{4.402778in}{3.199199in}}%
\pgfpathquadraticcurveto{\pgfqpoint{4.430556in}{3.199199in}}{\pgfqpoint{4.430556in}{3.226977in}}%
\pgfpathlineto{\pgfqpoint{4.430556in}{3.422778in}}%
\pgfpathquadraticcurveto{\pgfqpoint{4.430556in}{3.450556in}}{\pgfqpoint{4.402778in}{3.450556in}}%
\pgfpathlineto{\pgfqpoint{2.423855in}{3.450556in}}%
\pgfpathquadraticcurveto{\pgfqpoint{2.396077in}{3.450556in}}{\pgfqpoint{2.396077in}{3.422778in}}%
\pgfpathlineto{\pgfqpoint{2.396077in}{3.226977in}}%
\pgfpathquadraticcurveto{\pgfqpoint{2.396077in}{3.199199in}}{\pgfqpoint{2.423855in}{3.199199in}}%
\pgfpathclose%
\pgfusepath{stroke,fill}%
\end{pgfscope}%
\begin{pgfscope}%
\pgfsetbuttcap%
\pgfsetroundjoin%
\pgfsetlinewidth{1.505625pt}%
\definecolor{currentstroke}{rgb}{0.000000,0.000000,0.000000}%
\pgfsetstrokecolor{currentstroke}%
\pgfsetdash{{5.550000pt}{2.400000pt}}{0.000000pt}%
\pgfpathmoveto{\pgfqpoint{2.451633in}{3.338088in}}%
\pgfpathlineto{\pgfqpoint{2.729411in}{3.338088in}}%
\pgfusepath{stroke}%
\end{pgfscope}%
\begin{pgfscope}%
\definecolor{textcolor}{rgb}{0.000000,0.000000,0.000000}%
\pgfsetstrokecolor{textcolor}%
\pgfsetfillcolor{textcolor}%
\pgftext[x=2.840522in,y=3.289477in,left,base]{\color{textcolor}\sffamily\fontsize{10.000000}{12.000000}\selectfont \(\displaystyle \alpha = 1.5113 - 0.2065\mathrm{ln}(k)\)}%
\end{pgfscope}%
\end{pgfpicture}%
\makeatother%
\endgroup%

    \end{subfigure}
\caption[The computed Quasi-Monte-Carlo convergence rate for $Q(u) =  u(1,1)$.]{The computed values of $C$ (top) and $\alpha$ (bottom) against $k$ in \cref{eq:qmcerrorform} for $Q(u) = u((1,1))$. Observe the $x$-axes are on a $\log_{10}$ scale, but $\loge$ is the natural logarithm.  \label{fig:toprightCalpha}}
\end{figure}

\begin{figure}[h]
    \centering
    \begin{subfigure}{\textwidth}
            \centering
%% Creator: Matplotlib, PGF backend
%%
%% To include the figure in your LaTeX document, write
%%   \input{<filename>.pgf}
%%
%% Make sure the required packages are loaded in your preamble
%%   \usepackage{pgf}
%%
%% Figures using additional raster images can only be included by \input if
%% they are in the same directory as the main LaTeX file. For loading figures
%% from other directories you can use the `import` package
%%   \usepackage{import}
%% and then include the figures with
%%   \import{<path to file>}{<filename>.pgf}
%%
%% Matplotlib used the following preamble
%%   \usepackage{fontspec}
%%   \setmainfont{DejaVuSerif.ttf}[Path=/home/owen/progs/firedrake-complex/firedrake/lib/python3.5/site-packages/matplotlib/mpl-data/fonts/ttf/]
%%   \setsansfont{DejaVuSans.ttf}[Path=/home/owen/progs/firedrake-complex/firedrake/lib/python3.5/site-packages/matplotlib/mpl-data/fonts/ttf/]
%%   \setmonofont{DejaVuSansMono.ttf}[Path=/home/owen/progs/firedrake-complex/firedrake/lib/python3.5/site-packages/matplotlib/mpl-data/fonts/ttf/]
%%
\begingroup%
\makeatletter%
\begin{pgfpicture}%
\pgfpathrectangle{\pgfpointorigin}{\pgfqpoint{5.000000in}{4.000000in}}%
\pgfusepath{use as bounding box, clip}%
\begin{pgfscope}%
\pgfsetbuttcap%
\pgfsetmiterjoin%
\definecolor{currentfill}{rgb}{1.000000,1.000000,1.000000}%
\pgfsetfillcolor{currentfill}%
\pgfsetlinewidth{0.000000pt}%
\definecolor{currentstroke}{rgb}{1.000000,1.000000,1.000000}%
\pgfsetstrokecolor{currentstroke}%
\pgfsetdash{}{0pt}%
\pgfpathmoveto{\pgfqpoint{0.000000in}{0.000000in}}%
\pgfpathlineto{\pgfqpoint{5.000000in}{0.000000in}}%
\pgfpathlineto{\pgfqpoint{5.000000in}{4.000000in}}%
\pgfpathlineto{\pgfqpoint{0.000000in}{4.000000in}}%
\pgfpathclose%
\pgfusepath{fill}%
\end{pgfscope}%
\begin{pgfscope}%
\pgfsetbuttcap%
\pgfsetmiterjoin%
\definecolor{currentfill}{rgb}{1.000000,1.000000,1.000000}%
\pgfsetfillcolor{currentfill}%
\pgfsetlinewidth{0.000000pt}%
\definecolor{currentstroke}{rgb}{0.000000,0.000000,0.000000}%
\pgfsetstrokecolor{currentstroke}%
\pgfsetstrokeopacity{0.000000}%
\pgfsetdash{}{0pt}%
\pgfpathmoveto{\pgfqpoint{0.625000in}{0.440000in}}%
\pgfpathlineto{\pgfqpoint{4.500000in}{0.440000in}}%
\pgfpathlineto{\pgfqpoint{4.500000in}{3.520000in}}%
\pgfpathlineto{\pgfqpoint{0.625000in}{3.520000in}}%
\pgfpathclose%
\pgfusepath{fill}%
\end{pgfscope}%
\begin{pgfscope}%
\pgfsetbuttcap%
\pgfsetroundjoin%
\definecolor{currentfill}{rgb}{0.000000,0.000000,0.000000}%
\pgfsetfillcolor{currentfill}%
\pgfsetlinewidth{0.803000pt}%
\definecolor{currentstroke}{rgb}{0.000000,0.000000,0.000000}%
\pgfsetstrokecolor{currentstroke}%
\pgfsetdash{}{0pt}%
\pgfsys@defobject{currentmarker}{\pgfqpoint{0.000000in}{-0.048611in}}{\pgfqpoint{0.000000in}{0.000000in}}{%
\pgfpathmoveto{\pgfqpoint{0.000000in}{0.000000in}}%
\pgfpathlineto{\pgfqpoint{0.000000in}{-0.048611in}}%
\pgfusepath{stroke,fill}%
}%
\begin{pgfscope}%
\pgfsys@transformshift{0.801136in}{0.440000in}%
\pgfsys@useobject{currentmarker}{}%
\end{pgfscope}%
\end{pgfscope}%
\begin{pgfscope}%
\definecolor{textcolor}{rgb}{0.000000,0.000000,0.000000}%
\pgfsetstrokecolor{textcolor}%
\pgfsetfillcolor{textcolor}%
\pgftext[x=0.801136in,y=0.342778in,,top]{\color{textcolor}\sffamily\fontsize{10.000000}{12.000000}\selectfont \(\displaystyle {10^{1}}\)}%
\end{pgfscope}%
\begin{pgfscope}%
\pgfsetbuttcap%
\pgfsetroundjoin%
\definecolor{currentfill}{rgb}{0.000000,0.000000,0.000000}%
\pgfsetfillcolor{currentfill}%
\pgfsetlinewidth{0.602250pt}%
\definecolor{currentstroke}{rgb}{0.000000,0.000000,0.000000}%
\pgfsetstrokecolor{currentstroke}%
\pgfsetdash{}{0pt}%
\pgfsys@defobject{currentmarker}{\pgfqpoint{0.000000in}{-0.027778in}}{\pgfqpoint{0.000000in}{0.000000in}}{%
\pgfpathmoveto{\pgfqpoint{0.000000in}{0.000000in}}%
\pgfpathlineto{\pgfqpoint{0.000000in}{-0.027778in}}%
\pgfusepath{stroke,fill}%
}%
\begin{pgfscope}%
\pgfsys@transformshift{2.163913in}{0.440000in}%
\pgfsys@useobject{currentmarker}{}%
\end{pgfscope}%
\end{pgfscope}%
\begin{pgfscope}%
\definecolor{textcolor}{rgb}{0.000000,0.000000,0.000000}%
\pgfsetstrokecolor{textcolor}%
\pgfsetfillcolor{textcolor}%
\pgftext[x=2.163913in,y=0.365000in,,top]{\color{textcolor}\sffamily\fontsize{10.000000}{12.000000}\selectfont \(\displaystyle {2\times10^{1}}\)}%
\end{pgfscope}%
\begin{pgfscope}%
\pgfsetbuttcap%
\pgfsetroundjoin%
\definecolor{currentfill}{rgb}{0.000000,0.000000,0.000000}%
\pgfsetfillcolor{currentfill}%
\pgfsetlinewidth{0.602250pt}%
\definecolor{currentstroke}{rgb}{0.000000,0.000000,0.000000}%
\pgfsetstrokecolor{currentstroke}%
\pgfsetdash{}{0pt}%
\pgfsys@defobject{currentmarker}{\pgfqpoint{0.000000in}{-0.027778in}}{\pgfqpoint{0.000000in}{0.000000in}}{%
\pgfpathmoveto{\pgfqpoint{0.000000in}{0.000000in}}%
\pgfpathlineto{\pgfqpoint{0.000000in}{-0.027778in}}%
\pgfusepath{stroke,fill}%
}%
\begin{pgfscope}%
\pgfsys@transformshift{2.961087in}{0.440000in}%
\pgfsys@useobject{currentmarker}{}%
\end{pgfscope}%
\end{pgfscope}%
\begin{pgfscope}%
\definecolor{textcolor}{rgb}{0.000000,0.000000,0.000000}%
\pgfsetstrokecolor{textcolor}%
\pgfsetfillcolor{textcolor}%
\pgftext[x=2.961087in,y=0.365000in,,top]{\color{textcolor}\sffamily\fontsize{10.000000}{12.000000}\selectfont \(\displaystyle {3\times10^{1}}\)}%
\end{pgfscope}%
\begin{pgfscope}%
\pgfsetbuttcap%
\pgfsetroundjoin%
\definecolor{currentfill}{rgb}{0.000000,0.000000,0.000000}%
\pgfsetfillcolor{currentfill}%
\pgfsetlinewidth{0.602250pt}%
\definecolor{currentstroke}{rgb}{0.000000,0.000000,0.000000}%
\pgfsetstrokecolor{currentstroke}%
\pgfsetdash{}{0pt}%
\pgfsys@defobject{currentmarker}{\pgfqpoint{0.000000in}{-0.027778in}}{\pgfqpoint{0.000000in}{0.000000in}}{%
\pgfpathmoveto{\pgfqpoint{0.000000in}{0.000000in}}%
\pgfpathlineto{\pgfqpoint{0.000000in}{-0.027778in}}%
\pgfusepath{stroke,fill}%
}%
\begin{pgfscope}%
\pgfsys@transformshift{3.526690in}{0.440000in}%
\pgfsys@useobject{currentmarker}{}%
\end{pgfscope}%
\end{pgfscope}%
\begin{pgfscope}%
\definecolor{textcolor}{rgb}{0.000000,0.000000,0.000000}%
\pgfsetstrokecolor{textcolor}%
\pgfsetfillcolor{textcolor}%
\pgftext[x=3.526690in,y=0.365000in,,top]{\color{textcolor}\sffamily\fontsize{10.000000}{12.000000}\selectfont \(\displaystyle {4\times10^{1}}\)}%
\end{pgfscope}%
\begin{pgfscope}%
\pgfsetbuttcap%
\pgfsetroundjoin%
\definecolor{currentfill}{rgb}{0.000000,0.000000,0.000000}%
\pgfsetfillcolor{currentfill}%
\pgfsetlinewidth{0.602250pt}%
\definecolor{currentstroke}{rgb}{0.000000,0.000000,0.000000}%
\pgfsetstrokecolor{currentstroke}%
\pgfsetdash{}{0pt}%
\pgfsys@defobject{currentmarker}{\pgfqpoint{0.000000in}{-0.027778in}}{\pgfqpoint{0.000000in}{0.000000in}}{%
\pgfpathmoveto{\pgfqpoint{0.000000in}{0.000000in}}%
\pgfpathlineto{\pgfqpoint{0.000000in}{-0.027778in}}%
\pgfusepath{stroke,fill}%
}%
\begin{pgfscope}%
\pgfsys@transformshift{3.965406in}{0.440000in}%
\pgfsys@useobject{currentmarker}{}%
\end{pgfscope}%
\end{pgfscope}%
\begin{pgfscope}%
\pgfsetbuttcap%
\pgfsetroundjoin%
\definecolor{currentfill}{rgb}{0.000000,0.000000,0.000000}%
\pgfsetfillcolor{currentfill}%
\pgfsetlinewidth{0.602250pt}%
\definecolor{currentstroke}{rgb}{0.000000,0.000000,0.000000}%
\pgfsetstrokecolor{currentstroke}%
\pgfsetdash{}{0pt}%
\pgfsys@defobject{currentmarker}{\pgfqpoint{0.000000in}{-0.027778in}}{\pgfqpoint{0.000000in}{0.000000in}}{%
\pgfpathmoveto{\pgfqpoint{0.000000in}{0.000000in}}%
\pgfpathlineto{\pgfqpoint{0.000000in}{-0.027778in}}%
\pgfusepath{stroke,fill}%
}%
\begin{pgfscope}%
\pgfsys@transformshift{4.323864in}{0.440000in}%
\pgfsys@useobject{currentmarker}{}%
\end{pgfscope}%
\end{pgfscope}%
\begin{pgfscope}%
\definecolor{textcolor}{rgb}{0.000000,0.000000,0.000000}%
\pgfsetstrokecolor{textcolor}%
\pgfsetfillcolor{textcolor}%
\pgftext[x=4.323864in,y=0.365000in,,top]{\color{textcolor}\sffamily\fontsize{10.000000}{12.000000}\selectfont \(\displaystyle {6\times10^{1}}\)}%
\end{pgfscope}%
\begin{pgfscope}%
\definecolor{textcolor}{rgb}{0.000000,0.000000,0.000000}%
\pgfsetstrokecolor{textcolor}%
\pgfsetfillcolor{textcolor}%
\pgftext[x=2.562500in,y=0.152809in,,top]{\color{textcolor}\sffamily\fontsize{10.000000}{12.000000}\selectfont k}%
\end{pgfscope}%
\begin{pgfscope}%
\pgfsetbuttcap%
\pgfsetroundjoin%
\definecolor{currentfill}{rgb}{0.000000,0.000000,0.000000}%
\pgfsetfillcolor{currentfill}%
\pgfsetlinewidth{0.803000pt}%
\definecolor{currentstroke}{rgb}{0.000000,0.000000,0.000000}%
\pgfsetstrokecolor{currentstroke}%
\pgfsetdash{}{0pt}%
\pgfsys@defobject{currentmarker}{\pgfqpoint{-0.048611in}{0.000000in}}{\pgfqpoint{0.000000in}{0.000000in}}{%
\pgfpathmoveto{\pgfqpoint{0.000000in}{0.000000in}}%
\pgfpathlineto{\pgfqpoint{-0.048611in}{0.000000in}}%
\pgfusepath{stroke,fill}%
}%
\begin{pgfscope}%
\pgfsys@transformshift{0.625000in}{0.897834in}%
\pgfsys@useobject{currentmarker}{}%
\end{pgfscope}%
\end{pgfscope}%
\begin{pgfscope}%
\definecolor{textcolor}{rgb}{0.000000,0.000000,0.000000}%
\pgfsetstrokecolor{textcolor}%
\pgfsetfillcolor{textcolor}%
\pgftext[x=0.326581in,y=0.845072in,left,base]{\color{textcolor}\sffamily\fontsize{10.000000}{12.000000}\selectfont \(\displaystyle {10^{0}}\)}%
\end{pgfscope}%
\begin{pgfscope}%
\pgfsetbuttcap%
\pgfsetroundjoin%
\definecolor{currentfill}{rgb}{0.000000,0.000000,0.000000}%
\pgfsetfillcolor{currentfill}%
\pgfsetlinewidth{0.602250pt}%
\definecolor{currentstroke}{rgb}{0.000000,0.000000,0.000000}%
\pgfsetstrokecolor{currentstroke}%
\pgfsetdash{}{0pt}%
\pgfsys@defobject{currentmarker}{\pgfqpoint{-0.027778in}{0.000000in}}{\pgfqpoint{0.000000in}{0.000000in}}{%
\pgfpathmoveto{\pgfqpoint{0.000000in}{0.000000in}}%
\pgfpathlineto{\pgfqpoint{-0.027778in}{0.000000in}}%
\pgfusepath{stroke,fill}%
}%
\begin{pgfscope}%
\pgfsys@transformshift{0.625000in}{0.550354in}%
\pgfsys@useobject{currentmarker}{}%
\end{pgfscope}%
\end{pgfscope}%
\begin{pgfscope}%
\pgfsetbuttcap%
\pgfsetroundjoin%
\definecolor{currentfill}{rgb}{0.000000,0.000000,0.000000}%
\pgfsetfillcolor{currentfill}%
\pgfsetlinewidth{0.602250pt}%
\definecolor{currentstroke}{rgb}{0.000000,0.000000,0.000000}%
\pgfsetstrokecolor{currentstroke}%
\pgfsetdash{}{0pt}%
\pgfsys@defobject{currentmarker}{\pgfqpoint{-0.027778in}{0.000000in}}{\pgfqpoint{0.000000in}{0.000000in}}{%
\pgfpathmoveto{\pgfqpoint{0.000000in}{0.000000in}}%
\pgfpathlineto{\pgfqpoint{-0.027778in}{0.000000in}}%
\pgfusepath{stroke,fill}%
}%
\begin{pgfscope}%
\pgfsys@transformshift{0.625000in}{0.733766in}%
\pgfsys@useobject{currentmarker}{}%
\end{pgfscope}%
\end{pgfscope}%
\begin{pgfscope}%
\pgfsetbuttcap%
\pgfsetroundjoin%
\definecolor{currentfill}{rgb}{0.000000,0.000000,0.000000}%
\pgfsetfillcolor{currentfill}%
\pgfsetlinewidth{0.602250pt}%
\definecolor{currentstroke}{rgb}{0.000000,0.000000,0.000000}%
\pgfsetstrokecolor{currentstroke}%
\pgfsetdash{}{0pt}%
\pgfsys@defobject{currentmarker}{\pgfqpoint{-0.027778in}{0.000000in}}{\pgfqpoint{0.000000in}{0.000000in}}{%
\pgfpathmoveto{\pgfqpoint{0.000000in}{0.000000in}}%
\pgfpathlineto{\pgfqpoint{-0.027778in}{0.000000in}}%
\pgfusepath{stroke,fill}%
}%
\begin{pgfscope}%
\pgfsys@transformshift{0.625000in}{1.977204in}%
\pgfsys@useobject{currentmarker}{}%
\end{pgfscope}%
\end{pgfscope}%
\begin{pgfscope}%
\definecolor{textcolor}{rgb}{0.000000,0.000000,0.000000}%
\pgfsetstrokecolor{textcolor}%
\pgfsetfillcolor{textcolor}%
\pgftext[x=0.109607in,y=1.924443in,left,base]{\color{textcolor}\sffamily\fontsize{10.000000}{12.000000}\selectfont \(\displaystyle {2\times10^{0}}\)}%
\end{pgfscope}%
\begin{pgfscope}%
\pgfsetbuttcap%
\pgfsetroundjoin%
\definecolor{currentfill}{rgb}{0.000000,0.000000,0.000000}%
\pgfsetfillcolor{currentfill}%
\pgfsetlinewidth{0.602250pt}%
\definecolor{currentstroke}{rgb}{0.000000,0.000000,0.000000}%
\pgfsetstrokecolor{currentstroke}%
\pgfsetdash{}{0pt}%
\pgfsys@defobject{currentmarker}{\pgfqpoint{-0.027778in}{0.000000in}}{\pgfqpoint{0.000000in}{0.000000in}}{%
\pgfpathmoveto{\pgfqpoint{0.000000in}{0.000000in}}%
\pgfpathlineto{\pgfqpoint{-0.027778in}{0.000000in}}%
\pgfusepath{stroke,fill}%
}%
\begin{pgfscope}%
\pgfsys@transformshift{0.625000in}{2.608596in}%
\pgfsys@useobject{currentmarker}{}%
\end{pgfscope}%
\end{pgfscope}%
\begin{pgfscope}%
\definecolor{textcolor}{rgb}{0.000000,0.000000,0.000000}%
\pgfsetstrokecolor{textcolor}%
\pgfsetfillcolor{textcolor}%
\pgftext[x=0.109607in,y=2.555834in,left,base]{\color{textcolor}\sffamily\fontsize{10.000000}{12.000000}\selectfont \(\displaystyle {3\times10^{0}}\)}%
\end{pgfscope}%
\begin{pgfscope}%
\pgfsetbuttcap%
\pgfsetroundjoin%
\definecolor{currentfill}{rgb}{0.000000,0.000000,0.000000}%
\pgfsetfillcolor{currentfill}%
\pgfsetlinewidth{0.602250pt}%
\definecolor{currentstroke}{rgb}{0.000000,0.000000,0.000000}%
\pgfsetstrokecolor{currentstroke}%
\pgfsetdash{}{0pt}%
\pgfsys@defobject{currentmarker}{\pgfqpoint{-0.027778in}{0.000000in}}{\pgfqpoint{0.000000in}{0.000000in}}{%
\pgfpathmoveto{\pgfqpoint{0.000000in}{0.000000in}}%
\pgfpathlineto{\pgfqpoint{-0.027778in}{0.000000in}}%
\pgfusepath{stroke,fill}%
}%
\begin{pgfscope}%
\pgfsys@transformshift{0.625000in}{3.056575in}%
\pgfsys@useobject{currentmarker}{}%
\end{pgfscope}%
\end{pgfscope}%
\begin{pgfscope}%
\definecolor{textcolor}{rgb}{0.000000,0.000000,0.000000}%
\pgfsetstrokecolor{textcolor}%
\pgfsetfillcolor{textcolor}%
\pgftext[x=0.109607in,y=3.003814in,left,base]{\color{textcolor}\sffamily\fontsize{10.000000}{12.000000}\selectfont \(\displaystyle {4\times10^{0}}\)}%
\end{pgfscope}%
\begin{pgfscope}%
\pgfsetbuttcap%
\pgfsetroundjoin%
\definecolor{currentfill}{rgb}{0.000000,0.000000,0.000000}%
\pgfsetfillcolor{currentfill}%
\pgfsetlinewidth{0.602250pt}%
\definecolor{currentstroke}{rgb}{0.000000,0.000000,0.000000}%
\pgfsetstrokecolor{currentstroke}%
\pgfsetdash{}{0pt}%
\pgfsys@defobject{currentmarker}{\pgfqpoint{-0.027778in}{0.000000in}}{\pgfqpoint{0.000000in}{0.000000in}}{%
\pgfpathmoveto{\pgfqpoint{0.000000in}{0.000000in}}%
\pgfpathlineto{\pgfqpoint{-0.027778in}{0.000000in}}%
\pgfusepath{stroke,fill}%
}%
\begin{pgfscope}%
\pgfsys@transformshift{0.625000in}{3.404055in}%
\pgfsys@useobject{currentmarker}{}%
\end{pgfscope}%
\end{pgfscope}%
\begin{pgfscope}%
\definecolor{textcolor}{rgb}{0.000000,0.000000,0.000000}%
\pgfsetstrokecolor{textcolor}%
\pgfsetfillcolor{textcolor}%
\pgftext[x=0.054051in,y=1.980000in,,bottom,rotate=90.000000]{\color{textcolor}\sffamily\fontsize{10.000000}{12.000000}\selectfont C}%
\end{pgfscope}%
\begin{pgfscope}%
\pgfpathrectangle{\pgfqpoint{0.625000in}{0.440000in}}{\pgfqpoint{3.875000in}{3.080000in}}%
\pgfusepath{clip}%
\pgfsetbuttcap%
\pgfsetroundjoin%
\definecolor{currentfill}{rgb}{0.000000,0.000000,0.000000}%
\pgfsetfillcolor{currentfill}%
\pgfsetlinewidth{1.003750pt}%
\definecolor{currentstroke}{rgb}{0.000000,0.000000,0.000000}%
\pgfsetstrokecolor{currentstroke}%
\pgfsetdash{}{0pt}%
\pgfsys@defobject{currentmarker}{\pgfqpoint{-0.041667in}{-0.041667in}}{\pgfqpoint{0.041667in}{0.041667in}}{%
\pgfpathmoveto{\pgfqpoint{0.000000in}{-0.041667in}}%
\pgfpathcurveto{\pgfqpoint{0.011050in}{-0.041667in}}{\pgfqpoint{0.021649in}{-0.037276in}}{\pgfqpoint{0.029463in}{-0.029463in}}%
\pgfpathcurveto{\pgfqpoint{0.037276in}{-0.021649in}}{\pgfqpoint{0.041667in}{-0.011050in}}{\pgfqpoint{0.041667in}{0.000000in}}%
\pgfpathcurveto{\pgfqpoint{0.041667in}{0.011050in}}{\pgfqpoint{0.037276in}{0.021649in}}{\pgfqpoint{0.029463in}{0.029463in}}%
\pgfpathcurveto{\pgfqpoint{0.021649in}{0.037276in}}{\pgfqpoint{0.011050in}{0.041667in}}{\pgfqpoint{0.000000in}{0.041667in}}%
\pgfpathcurveto{\pgfqpoint{-0.011050in}{0.041667in}}{\pgfqpoint{-0.021649in}{0.037276in}}{\pgfqpoint{-0.029463in}{0.029463in}}%
\pgfpathcurveto{\pgfqpoint{-0.037276in}{0.021649in}}{\pgfqpoint{-0.041667in}{0.011050in}}{\pgfqpoint{-0.041667in}{0.000000in}}%
\pgfpathcurveto{\pgfqpoint{-0.041667in}{-0.011050in}}{\pgfqpoint{-0.037276in}{-0.021649in}}{\pgfqpoint{-0.029463in}{-0.029463in}}%
\pgfpathcurveto{\pgfqpoint{-0.021649in}{-0.037276in}}{\pgfqpoint{-0.011050in}{-0.041667in}}{\pgfqpoint{0.000000in}{-0.041667in}}%
\pgfpathclose%
\pgfusepath{stroke,fill}%
}%
\begin{pgfscope}%
\pgfsys@transformshift{0.801136in}{0.580000in}%
\pgfsys@useobject{currentmarker}{}%
\end{pgfscope}%
\begin{pgfscope}%
\pgfsys@transformshift{2.163913in}{1.211949in}%
\pgfsys@useobject{currentmarker}{}%
\end{pgfscope}%
\begin{pgfscope}%
\pgfsys@transformshift{2.961087in}{2.605341in}%
\pgfsys@useobject{currentmarker}{}%
\end{pgfscope}%
\begin{pgfscope}%
\pgfsys@transformshift{3.526690in}{2.532871in}%
\pgfsys@useobject{currentmarker}{}%
\end{pgfscope}%
\begin{pgfscope}%
\pgfsys@transformshift{3.965406in}{3.380000in}%
\pgfsys@useobject{currentmarker}{}%
\end{pgfscope}%
\begin{pgfscope}%
\pgfsys@transformshift{4.323864in}{3.162925in}%
\pgfsys@useobject{currentmarker}{}%
\end{pgfscope}%
\end{pgfscope}%
\begin{pgfscope}%
\pgfsetrectcap%
\pgfsetmiterjoin%
\pgfsetlinewidth{0.803000pt}%
\definecolor{currentstroke}{rgb}{0.000000,0.000000,0.000000}%
\pgfsetstrokecolor{currentstroke}%
\pgfsetdash{}{0pt}%
\pgfpathmoveto{\pgfqpoint{0.625000in}{0.440000in}}%
\pgfpathlineto{\pgfqpoint{0.625000in}{3.520000in}}%
\pgfusepath{stroke}%
\end{pgfscope}%
\begin{pgfscope}%
\pgfsetrectcap%
\pgfsetmiterjoin%
\pgfsetlinewidth{0.803000pt}%
\definecolor{currentstroke}{rgb}{0.000000,0.000000,0.000000}%
\pgfsetstrokecolor{currentstroke}%
\pgfsetdash{}{0pt}%
\pgfpathmoveto{\pgfqpoint{4.500000in}{0.440000in}}%
\pgfpathlineto{\pgfqpoint{4.500000in}{3.520000in}}%
\pgfusepath{stroke}%
\end{pgfscope}%
\begin{pgfscope}%
\pgfsetrectcap%
\pgfsetmiterjoin%
\pgfsetlinewidth{0.803000pt}%
\definecolor{currentstroke}{rgb}{0.000000,0.000000,0.000000}%
\pgfsetstrokecolor{currentstroke}%
\pgfsetdash{}{0pt}%
\pgfpathmoveto{\pgfqpoint{0.625000in}{0.440000in}}%
\pgfpathlineto{\pgfqpoint{4.500000in}{0.440000in}}%
\pgfusepath{stroke}%
\end{pgfscope}%
\begin{pgfscope}%
\pgfsetrectcap%
\pgfsetmiterjoin%
\pgfsetlinewidth{0.803000pt}%
\definecolor{currentstroke}{rgb}{0.000000,0.000000,0.000000}%
\pgfsetstrokecolor{currentstroke}%
\pgfsetdash{}{0pt}%
\pgfpathmoveto{\pgfqpoint{0.625000in}{3.520000in}}%
\pgfpathlineto{\pgfqpoint{4.500000in}{3.520000in}}%
\pgfusepath{stroke}%
\end{pgfscope}%
\end{pgfpicture}%
\makeatother%
\endgroup%

  \end{subfigure}
    \begin{subfigure}{\textwidth}
            \centering 
%% Creator: Matplotlib, PGF backend
%%
%% To include the figure in your LaTeX document, write
%%   \input{<filename>.pgf}
%%
%% Make sure the required packages are loaded in your preamble
%%   \usepackage{pgf}
%%
%% Figures using additional raster images can only be included by \input if
%% they are in the same directory as the main LaTeX file. For loading figures
%% from other directories you can use the `import` package
%%   \usepackage{import}
%% and then include the figures with
%%   \import{<path to file>}{<filename>.pgf}
%%
%% Matplotlib used the following preamble
%%   \usepackage{fontspec}
%%   \setmainfont{DejaVuSerif.ttf}[Path=/home/owen/progs/firedrake-complex/firedrake/lib/python3.5/site-packages/matplotlib/mpl-data/fonts/ttf/]
%%   \setsansfont{DejaVuSans.ttf}[Path=/home/owen/progs/firedrake-complex/firedrake/lib/python3.5/site-packages/matplotlib/mpl-data/fonts/ttf/]
%%   \setmonofont{DejaVuSansMono.ttf}[Path=/home/owen/progs/firedrake-complex/firedrake/lib/python3.5/site-packages/matplotlib/mpl-data/fonts/ttf/]
%%
\begingroup%
\makeatletter%
\begin{pgfpicture}%
\pgfpathrectangle{\pgfpointorigin}{\pgfqpoint{5.000000in}{4.000000in}}%
\pgfusepath{use as bounding box, clip}%
\begin{pgfscope}%
\pgfsetbuttcap%
\pgfsetmiterjoin%
\definecolor{currentfill}{rgb}{1.000000,1.000000,1.000000}%
\pgfsetfillcolor{currentfill}%
\pgfsetlinewidth{0.000000pt}%
\definecolor{currentstroke}{rgb}{1.000000,1.000000,1.000000}%
\pgfsetstrokecolor{currentstroke}%
\pgfsetdash{}{0pt}%
\pgfpathmoveto{\pgfqpoint{0.000000in}{0.000000in}}%
\pgfpathlineto{\pgfqpoint{5.000000in}{0.000000in}}%
\pgfpathlineto{\pgfqpoint{5.000000in}{4.000000in}}%
\pgfpathlineto{\pgfqpoint{0.000000in}{4.000000in}}%
\pgfpathclose%
\pgfusepath{fill}%
\end{pgfscope}%
\begin{pgfscope}%
\pgfsetbuttcap%
\pgfsetmiterjoin%
\definecolor{currentfill}{rgb}{1.000000,1.000000,1.000000}%
\pgfsetfillcolor{currentfill}%
\pgfsetlinewidth{0.000000pt}%
\definecolor{currentstroke}{rgb}{0.000000,0.000000,0.000000}%
\pgfsetstrokecolor{currentstroke}%
\pgfsetstrokeopacity{0.000000}%
\pgfsetdash{}{0pt}%
\pgfpathmoveto{\pgfqpoint{0.625000in}{0.440000in}}%
\pgfpathlineto{\pgfqpoint{4.500000in}{0.440000in}}%
\pgfpathlineto{\pgfqpoint{4.500000in}{3.520000in}}%
\pgfpathlineto{\pgfqpoint{0.625000in}{3.520000in}}%
\pgfpathclose%
\pgfusepath{fill}%
\end{pgfscope}%
\begin{pgfscope}%
\pgfsetbuttcap%
\pgfsetroundjoin%
\definecolor{currentfill}{rgb}{0.000000,0.000000,0.000000}%
\pgfsetfillcolor{currentfill}%
\pgfsetlinewidth{0.803000pt}%
\definecolor{currentstroke}{rgb}{0.000000,0.000000,0.000000}%
\pgfsetstrokecolor{currentstroke}%
\pgfsetdash{}{0pt}%
\pgfsys@defobject{currentmarker}{\pgfqpoint{0.000000in}{-0.048611in}}{\pgfqpoint{0.000000in}{0.000000in}}{%
\pgfpathmoveto{\pgfqpoint{0.000000in}{0.000000in}}%
\pgfpathlineto{\pgfqpoint{0.000000in}{-0.048611in}}%
\pgfusepath{stroke,fill}%
}%
\begin{pgfscope}%
\pgfsys@transformshift{2.172304in}{0.440000in}%
\pgfsys@useobject{currentmarker}{}%
\end{pgfscope}%
\end{pgfscope}%
\begin{pgfscope}%
\definecolor{textcolor}{rgb}{0.000000,0.000000,0.000000}%
\pgfsetstrokecolor{textcolor}%
\pgfsetfillcolor{textcolor}%
\pgftext[x=2.172304in,y=0.342778in,,top]{\color{textcolor}\sffamily\fontsize{10.000000}{12.000000}\selectfont \(\displaystyle {2.718281828459045^{3}}\)}%
\end{pgfscope}%
\begin{pgfscope}%
\pgfsetbuttcap%
\pgfsetroundjoin%
\definecolor{currentfill}{rgb}{0.000000,0.000000,0.000000}%
\pgfsetfillcolor{currentfill}%
\pgfsetlinewidth{0.803000pt}%
\definecolor{currentstroke}{rgb}{0.000000,0.000000,0.000000}%
\pgfsetstrokecolor{currentstroke}%
\pgfsetdash{}{0pt}%
\pgfsys@defobject{currentmarker}{\pgfqpoint{0.000000in}{-0.048611in}}{\pgfqpoint{0.000000in}{0.000000in}}{%
\pgfpathmoveto{\pgfqpoint{0.000000in}{0.000000in}}%
\pgfpathlineto{\pgfqpoint{0.000000in}{-0.048611in}}%
\pgfusepath{stroke,fill}%
}%
\begin{pgfscope}%
\pgfsys@transformshift{4.138375in}{0.440000in}%
\pgfsys@useobject{currentmarker}{}%
\end{pgfscope}%
\end{pgfscope}%
\begin{pgfscope}%
\definecolor{textcolor}{rgb}{0.000000,0.000000,0.000000}%
\pgfsetstrokecolor{textcolor}%
\pgfsetfillcolor{textcolor}%
\pgftext[x=4.138375in,y=0.342778in,,top]{\color{textcolor}\sffamily\fontsize{10.000000}{12.000000}\selectfont \(\displaystyle {2.718281828459045^{4}}\)}%
\end{pgfscope}%
\begin{pgfscope}%
\definecolor{textcolor}{rgb}{0.000000,0.000000,0.000000}%
\pgfsetstrokecolor{textcolor}%
\pgfsetfillcolor{textcolor}%
\pgftext[x=2.562500in,y=0.152809in,,top]{\color{textcolor}\sffamily\fontsize{10.000000}{12.000000}\selectfont \(\displaystyle k\)}%
\end{pgfscope}%
\begin{pgfscope}%
\pgfsetbuttcap%
\pgfsetroundjoin%
\definecolor{currentfill}{rgb}{0.000000,0.000000,0.000000}%
\pgfsetfillcolor{currentfill}%
\pgfsetlinewidth{0.803000pt}%
\definecolor{currentstroke}{rgb}{0.000000,0.000000,0.000000}%
\pgfsetstrokecolor{currentstroke}%
\pgfsetdash{}{0pt}%
\pgfsys@defobject{currentmarker}{\pgfqpoint{-0.048611in}{0.000000in}}{\pgfqpoint{0.000000in}{0.000000in}}{%
\pgfpathmoveto{\pgfqpoint{0.000000in}{0.000000in}}%
\pgfpathlineto{\pgfqpoint{-0.048611in}{0.000000in}}%
\pgfusepath{stroke,fill}%
}%
\begin{pgfscope}%
\pgfsys@transformshift{0.625000in}{0.750163in}%
\pgfsys@useobject{currentmarker}{}%
\end{pgfscope}%
\end{pgfscope}%
\begin{pgfscope}%
\definecolor{textcolor}{rgb}{0.000000,0.000000,0.000000}%
\pgfsetstrokecolor{textcolor}%
\pgfsetfillcolor{textcolor}%
\pgftext[x=0.306898in,y=0.697401in,left,base]{\color{textcolor}\sffamily\fontsize{10.000000}{12.000000}\selectfont 0.6}%
\end{pgfscope}%
\begin{pgfscope}%
\pgfsetbuttcap%
\pgfsetroundjoin%
\definecolor{currentfill}{rgb}{0.000000,0.000000,0.000000}%
\pgfsetfillcolor{currentfill}%
\pgfsetlinewidth{0.803000pt}%
\definecolor{currentstroke}{rgb}{0.000000,0.000000,0.000000}%
\pgfsetstrokecolor{currentstroke}%
\pgfsetdash{}{0pt}%
\pgfsys@defobject{currentmarker}{\pgfqpoint{-0.048611in}{0.000000in}}{\pgfqpoint{0.000000in}{0.000000in}}{%
\pgfpathmoveto{\pgfqpoint{0.000000in}{0.000000in}}%
\pgfpathlineto{\pgfqpoint{-0.048611in}{0.000000in}}%
\pgfusepath{stroke,fill}%
}%
\begin{pgfscope}%
\pgfsys@transformshift{0.625000in}{1.353507in}%
\pgfsys@useobject{currentmarker}{}%
\end{pgfscope}%
\end{pgfscope}%
\begin{pgfscope}%
\definecolor{textcolor}{rgb}{0.000000,0.000000,0.000000}%
\pgfsetstrokecolor{textcolor}%
\pgfsetfillcolor{textcolor}%
\pgftext[x=0.306898in,y=1.300746in,left,base]{\color{textcolor}\sffamily\fontsize{10.000000}{12.000000}\selectfont 0.7}%
\end{pgfscope}%
\begin{pgfscope}%
\pgfsetbuttcap%
\pgfsetroundjoin%
\definecolor{currentfill}{rgb}{0.000000,0.000000,0.000000}%
\pgfsetfillcolor{currentfill}%
\pgfsetlinewidth{0.803000pt}%
\definecolor{currentstroke}{rgb}{0.000000,0.000000,0.000000}%
\pgfsetstrokecolor{currentstroke}%
\pgfsetdash{}{0pt}%
\pgfsys@defobject{currentmarker}{\pgfqpoint{-0.048611in}{0.000000in}}{\pgfqpoint{0.000000in}{0.000000in}}{%
\pgfpathmoveto{\pgfqpoint{0.000000in}{0.000000in}}%
\pgfpathlineto{\pgfqpoint{-0.048611in}{0.000000in}}%
\pgfusepath{stroke,fill}%
}%
\begin{pgfscope}%
\pgfsys@transformshift{0.625000in}{1.956852in}%
\pgfsys@useobject{currentmarker}{}%
\end{pgfscope}%
\end{pgfscope}%
\begin{pgfscope}%
\definecolor{textcolor}{rgb}{0.000000,0.000000,0.000000}%
\pgfsetstrokecolor{textcolor}%
\pgfsetfillcolor{textcolor}%
\pgftext[x=0.306898in,y=1.904091in,left,base]{\color{textcolor}\sffamily\fontsize{10.000000}{12.000000}\selectfont 0.8}%
\end{pgfscope}%
\begin{pgfscope}%
\pgfsetbuttcap%
\pgfsetroundjoin%
\definecolor{currentfill}{rgb}{0.000000,0.000000,0.000000}%
\pgfsetfillcolor{currentfill}%
\pgfsetlinewidth{0.803000pt}%
\definecolor{currentstroke}{rgb}{0.000000,0.000000,0.000000}%
\pgfsetstrokecolor{currentstroke}%
\pgfsetdash{}{0pt}%
\pgfsys@defobject{currentmarker}{\pgfqpoint{-0.048611in}{0.000000in}}{\pgfqpoint{0.000000in}{0.000000in}}{%
\pgfpathmoveto{\pgfqpoint{0.000000in}{0.000000in}}%
\pgfpathlineto{\pgfqpoint{-0.048611in}{0.000000in}}%
\pgfusepath{stroke,fill}%
}%
\begin{pgfscope}%
\pgfsys@transformshift{0.625000in}{2.560197in}%
\pgfsys@useobject{currentmarker}{}%
\end{pgfscope}%
\end{pgfscope}%
\begin{pgfscope}%
\definecolor{textcolor}{rgb}{0.000000,0.000000,0.000000}%
\pgfsetstrokecolor{textcolor}%
\pgfsetfillcolor{textcolor}%
\pgftext[x=0.306898in,y=2.507435in,left,base]{\color{textcolor}\sffamily\fontsize{10.000000}{12.000000}\selectfont 0.9}%
\end{pgfscope}%
\begin{pgfscope}%
\pgfsetbuttcap%
\pgfsetroundjoin%
\definecolor{currentfill}{rgb}{0.000000,0.000000,0.000000}%
\pgfsetfillcolor{currentfill}%
\pgfsetlinewidth{0.803000pt}%
\definecolor{currentstroke}{rgb}{0.000000,0.000000,0.000000}%
\pgfsetstrokecolor{currentstroke}%
\pgfsetdash{}{0pt}%
\pgfsys@defobject{currentmarker}{\pgfqpoint{-0.048611in}{0.000000in}}{\pgfqpoint{0.000000in}{0.000000in}}{%
\pgfpathmoveto{\pgfqpoint{0.000000in}{0.000000in}}%
\pgfpathlineto{\pgfqpoint{-0.048611in}{0.000000in}}%
\pgfusepath{stroke,fill}%
}%
\begin{pgfscope}%
\pgfsys@transformshift{0.625000in}{3.163542in}%
\pgfsys@useobject{currentmarker}{}%
\end{pgfscope}%
\end{pgfscope}%
\begin{pgfscope}%
\definecolor{textcolor}{rgb}{0.000000,0.000000,0.000000}%
\pgfsetstrokecolor{textcolor}%
\pgfsetfillcolor{textcolor}%
\pgftext[x=0.306898in,y=3.110780in,left,base]{\color{textcolor}\sffamily\fontsize{10.000000}{12.000000}\selectfont 1.0}%
\end{pgfscope}%
\begin{pgfscope}%
\definecolor{textcolor}{rgb}{0.000000,0.000000,0.000000}%
\pgfsetstrokecolor{textcolor}%
\pgfsetfillcolor{textcolor}%
\pgftext[x=0.251343in,y=1.980000in,,bottom,rotate=90.000000]{\color{textcolor}\sffamily\fontsize{10.000000}{12.000000}\selectfont \(\displaystyle \alpha\)}%
\end{pgfscope}%
\begin{pgfscope}%
\pgfpathrectangle{\pgfqpoint{0.625000in}{0.440000in}}{\pgfqpoint{3.875000in}{3.080000in}}%
\pgfusepath{clip}%
\pgfsetbuttcap%
\pgfsetroundjoin%
\pgfsetlinewidth{1.505625pt}%
\definecolor{currentstroke}{rgb}{0.000000,0.000000,0.000000}%
\pgfsetstrokecolor{currentstroke}%
\pgfsetdash{{5.550000pt}{2.400000pt}}{0.000000pt}%
\pgfpathmoveto{\pgfqpoint{0.801136in}{3.380000in}}%
\pgfpathlineto{\pgfqpoint{2.163913in}{2.516354in}}%
\pgfpathlineto{\pgfqpoint{2.961087in}{2.011153in}}%
\pgfpathlineto{\pgfqpoint{3.526690in}{1.652708in}}%
\pgfpathlineto{\pgfqpoint{3.965406in}{1.374676in}}%
\pgfpathlineto{\pgfqpoint{4.323864in}{1.147507in}}%
\pgfusepath{stroke}%
\end{pgfscope}%
\begin{pgfscope}%
\pgfpathrectangle{\pgfqpoint{0.625000in}{0.440000in}}{\pgfqpoint{3.875000in}{3.080000in}}%
\pgfusepath{clip}%
\pgfsetbuttcap%
\pgfsetroundjoin%
\definecolor{currentfill}{rgb}{0.000000,0.000000,0.000000}%
\pgfsetfillcolor{currentfill}%
\pgfsetlinewidth{1.003750pt}%
\definecolor{currentstroke}{rgb}{0.000000,0.000000,0.000000}%
\pgfsetstrokecolor{currentstroke}%
\pgfsetdash{}{0pt}%
\pgfsys@defobject{currentmarker}{\pgfqpoint{-0.041667in}{-0.041667in}}{\pgfqpoint{0.041667in}{0.041667in}}{%
\pgfpathmoveto{\pgfqpoint{0.000000in}{-0.041667in}}%
\pgfpathcurveto{\pgfqpoint{0.011050in}{-0.041667in}}{\pgfqpoint{0.021649in}{-0.037276in}}{\pgfqpoint{0.029463in}{-0.029463in}}%
\pgfpathcurveto{\pgfqpoint{0.037276in}{-0.021649in}}{\pgfqpoint{0.041667in}{-0.011050in}}{\pgfqpoint{0.041667in}{0.000000in}}%
\pgfpathcurveto{\pgfqpoint{0.041667in}{0.011050in}}{\pgfqpoint{0.037276in}{0.021649in}}{\pgfqpoint{0.029463in}{0.029463in}}%
\pgfpathcurveto{\pgfqpoint{0.021649in}{0.037276in}}{\pgfqpoint{0.011050in}{0.041667in}}{\pgfqpoint{0.000000in}{0.041667in}}%
\pgfpathcurveto{\pgfqpoint{-0.011050in}{0.041667in}}{\pgfqpoint{-0.021649in}{0.037276in}}{\pgfqpoint{-0.029463in}{0.029463in}}%
\pgfpathcurveto{\pgfqpoint{-0.037276in}{0.021649in}}{\pgfqpoint{-0.041667in}{0.011050in}}{\pgfqpoint{-0.041667in}{0.000000in}}%
\pgfpathcurveto{\pgfqpoint{-0.041667in}{-0.011050in}}{\pgfqpoint{-0.037276in}{-0.021649in}}{\pgfqpoint{-0.029463in}{-0.029463in}}%
\pgfpathcurveto{\pgfqpoint{-0.021649in}{-0.037276in}}{\pgfqpoint{-0.011050in}{-0.041667in}}{\pgfqpoint{0.000000in}{-0.041667in}}%
\pgfpathclose%
\pgfusepath{stroke,fill}%
}%
\begin{pgfscope}%
\pgfsys@transformshift{0.801136in}{3.062742in}%
\pgfsys@useobject{currentmarker}{}%
\end{pgfscope}%
\begin{pgfscope}%
\pgfsys@transformshift{2.163913in}{2.696563in}%
\pgfsys@useobject{currentmarker}{}%
\end{pgfscope}%
\begin{pgfscope}%
\pgfsys@transformshift{2.961087in}{2.476225in}%
\pgfsys@useobject{currentmarker}{}%
\end{pgfscope}%
\begin{pgfscope}%
\pgfsys@transformshift{3.526690in}{1.672621in}%
\pgfsys@useobject{currentmarker}{}%
\end{pgfscope}%
\begin{pgfscope}%
\pgfsys@transformshift{3.965406in}{1.594247in}%
\pgfsys@useobject{currentmarker}{}%
\end{pgfscope}%
\begin{pgfscope}%
\pgfsys@transformshift{4.323864in}{0.580000in}%
\pgfsys@useobject{currentmarker}{}%
\end{pgfscope}%
\end{pgfscope}%
\begin{pgfscope}%
\pgfsetrectcap%
\pgfsetmiterjoin%
\pgfsetlinewidth{0.803000pt}%
\definecolor{currentstroke}{rgb}{0.000000,0.000000,0.000000}%
\pgfsetstrokecolor{currentstroke}%
\pgfsetdash{}{0pt}%
\pgfpathmoveto{\pgfqpoint{0.625000in}{0.440000in}}%
\pgfpathlineto{\pgfqpoint{0.625000in}{3.520000in}}%
\pgfusepath{stroke}%
\end{pgfscope}%
\begin{pgfscope}%
\pgfsetrectcap%
\pgfsetmiterjoin%
\pgfsetlinewidth{0.803000pt}%
\definecolor{currentstroke}{rgb}{0.000000,0.000000,0.000000}%
\pgfsetstrokecolor{currentstroke}%
\pgfsetdash{}{0pt}%
\pgfpathmoveto{\pgfqpoint{4.500000in}{0.440000in}}%
\pgfpathlineto{\pgfqpoint{4.500000in}{3.520000in}}%
\pgfusepath{stroke}%
\end{pgfscope}%
\begin{pgfscope}%
\pgfsetrectcap%
\pgfsetmiterjoin%
\pgfsetlinewidth{0.803000pt}%
\definecolor{currentstroke}{rgb}{0.000000,0.000000,0.000000}%
\pgfsetstrokecolor{currentstroke}%
\pgfsetdash{}{0pt}%
\pgfpathmoveto{\pgfqpoint{0.625000in}{0.440000in}}%
\pgfpathlineto{\pgfqpoint{4.500000in}{0.440000in}}%
\pgfusepath{stroke}%
\end{pgfscope}%
\begin{pgfscope}%
\pgfsetrectcap%
\pgfsetmiterjoin%
\pgfsetlinewidth{0.803000pt}%
\definecolor{currentstroke}{rgb}{0.000000,0.000000,0.000000}%
\pgfsetstrokecolor{currentstroke}%
\pgfsetdash{}{0pt}%
\pgfpathmoveto{\pgfqpoint{0.625000in}{3.520000in}}%
\pgfpathlineto{\pgfqpoint{4.500000in}{3.520000in}}%
\pgfusepath{stroke}%
\end{pgfscope}%
\begin{pgfscope}%
\pgfsetbuttcap%
\pgfsetmiterjoin%
\definecolor{currentfill}{rgb}{1.000000,1.000000,1.000000}%
\pgfsetfillcolor{currentfill}%
\pgfsetfillopacity{0.800000}%
\pgfsetlinewidth{1.003750pt}%
\definecolor{currentstroke}{rgb}{0.800000,0.800000,0.800000}%
\pgfsetstrokecolor{currentstroke}%
\pgfsetstrokeopacity{0.800000}%
\pgfsetdash{}{0pt}%
\pgfpathmoveto{\pgfqpoint{2.317632in}{3.199199in}}%
\pgfpathlineto{\pgfqpoint{4.402778in}{3.199199in}}%
\pgfpathquadraticcurveto{\pgfqpoint{4.430556in}{3.199199in}}{\pgfqpoint{4.430556in}{3.226977in}}%
\pgfpathlineto{\pgfqpoint{4.430556in}{3.422778in}}%
\pgfpathquadraticcurveto{\pgfqpoint{4.430556in}{3.450556in}}{\pgfqpoint{4.402778in}{3.450556in}}%
\pgfpathlineto{\pgfqpoint{2.317632in}{3.450556in}}%
\pgfpathquadraticcurveto{\pgfqpoint{2.289854in}{3.450556in}}{\pgfqpoint{2.289854in}{3.422778in}}%
\pgfpathlineto{\pgfqpoint{2.289854in}{3.226977in}}%
\pgfpathquadraticcurveto{\pgfqpoint{2.289854in}{3.199199in}}{\pgfqpoint{2.317632in}{3.199199in}}%
\pgfpathclose%
\pgfusepath{stroke,fill}%
\end{pgfscope}%
\begin{pgfscope}%
\pgfsetbuttcap%
\pgfsetroundjoin%
\pgfsetlinewidth{1.505625pt}%
\definecolor{currentstroke}{rgb}{0.000000,0.000000,0.000000}%
\pgfsetstrokecolor{currentstroke}%
\pgfsetdash{{5.550000pt}{2.400000pt}}{0.000000pt}%
\pgfpathmoveto{\pgfqpoint{2.345410in}{3.338088in}}%
\pgfpathlineto{\pgfqpoint{2.623187in}{3.338088in}}%
\pgfusepath{stroke}%
\end{pgfscope}%
\begin{pgfscope}%
\definecolor{textcolor}{rgb}{0.000000,0.000000,0.000000}%
\pgfsetstrokecolor{textcolor}%
\pgfsetfillcolor{textcolor}%
\pgftext[x=2.734298in,y=3.289477in,left,base]{\color{textcolor}\sffamily\fontsize{10.000000}{12.000000}\selectfont \(\displaystyle \alpha = 1.5113 - 0.2065\log(k)\)}%
\end{pgfscope}%
\end{pgfpicture}%
\makeatother%
\endgroup%

    \end{subfigure}
\caption[The computed Quasi-Monte-Carlo convergence rate for $Q(u) =  \gradu(1,1)$.]{The computed values of $C$ (top) and $\alpha$ (bottom) against $k$ in \cref{eq:qmcerrorform} for $Q(u) = \gradu((1,1))$. Observe the $x$-axes are on a $\log_{10}$ scale, but $\loge$ is the natural logarithm.  \label{fig:gradienttoprightCalpha}}
\end{figure}


\begin{table}[h!]
  \centering
  \begin{tabular}{Sc Sc Sc Sc Sc}
\toprule
{} & \$Q = \textbackslash int\_D u\$ & \$Q = u(\textbackslash bzero)\$ & \$Q = u((1,1))\$ & \$Q = \textbackslash gradu((1,1))\$ \\
\midrule
\$\textbackslash alphaz\$        &           1.28 &            1.36 &           1.43 &                1.43 \\
\$\textbackslash alpha\$         &           0.30 &            0.38 &           0.41 &                0.41 \\
\$\textbackslash alphaz/alphao\$ &           0.24 &            0.28 &           0.29 &                0.29 \\
\bottomrule
\end{tabular}

  \caption{The quantities $\alphaz$ and $\alphao$ for different QoIs, where the QMC error $\approx C \NQMC^{\alphaz - \alphao\loge(k)}$.}\label{tab:qmcalpha}
  \end{table}

Motivated by QMC theory for other applications, e.g., \cite[Equation 4.2]{GrKuNuScSl:11}, we test experimentally the assumption that the QMC error satisfies
\beq\label{eq:qmcerrorform}
\QMCerror{Q}{\Nshifts} = C \NQMC^{-\alpha},
\eeq
for some $C, \alpha > 0.$ Using data for the values of $k$ listed above, \cref{fig:integralCalpha,fig:originCalpha,fig:toprightCalpha,fig:gradienttoprightCalpha} plot the computed values of $C$ and $\alpha$ against $k$. (In \cref{app:hhqmcconv}, we plot the QMC error for increasing $\NQMC$ for each $k \in \set{10,20,30,40,50,60}$ and for each QoI---these plots allow us to determine the values of $C$ and $\alpha$ for each value of $k.$) For the QoIs that are point evaluations (\cref{fig:originCalpha,fig:toprightCalpha}), $C$ appears not to vary very much; thus we assume $C$ is constant in all of the following calculations.

The \cref{fig:integralCalpha,fig:originCalpha,fig:toprightCalpha,fig:gradienttoprightCalpha} (bottom panes) show $\alpha$ decreasing at a rate proportional to $\log k$. Therefore we conjecture
\beq\label{eq:alphaform}
\alpha(k) = \alphaz - \alphao\loge(k),
\eeq
for some constants $\alphaz,\alphao > 0.$ (Throughout this \lcnamecref{sec:nbpcqmcnumerics}, $\loge$ denotes the natural logarithm.) We fitted $\alphaz$ and $\alphao$ numerically, and have plotted the resulting line of best fit on \cref{fig:integralCalpha,fig:originCalpha,fig:toprightCalpha,fig:gradienttoprightCalpha}. (Observe that the conjectured form \cref{eq:alphaform} cannot hold for $k$ very large, as then $\alpha(k)$ would be negative, and there would be no convergence as the number of QMC points is increased. Nevertheless, for the range of $k$ we consider in these numerical experiments, the form \cref{eq:alphaform} seems to give a good fit with the data.) The values of $C$ and $\alpha$ for the different QoIs are given in \cref{fig:integralCalpha,fig:originCalpha,fig:toprightCalpha,fig:gradienttoprightCalpha}.

Having understood how the QMC error increases with $k$ for fixed $\NQMC$, we now use this knowledge to determine how one should increase $\NQMC$ with $k$ in order to keep the QMC error bounded. Recalling that we assume $C$ in \cref{eq:qmcerrorform} is constant, if we take
\beq\label{eq:Nform}
\NQMC(k) = \exp\mleft(\Ctilde \alpha(k)^{-1}\mright),
\eeq
for some constant $\Ctilde > 0$, then substituting \cref{eq:Nform} into \cref{eq:qmcerrorform}, we see that the QMC error should remain bounded, with
\beqs
\QMCerror{Q}{\Nshifts} = C \exp\mleft(-\Ctilde\mright).
\eeqs
Observe that, since $\alpha(k)$ decreases as $k$ increases, \cref{eq:Nform} will increase as $k$ increases.

In our numerical experiments with increasing $\NQMC(k)$ below, we set $\Ctilde$ so that $\NQMC(10) = 2048,$ because in our numerical experiments to determine the behaviour of the QMC error, we used $\NQMC = 2048$ (with 20 shifts). Also in our numerical experiments below we take the number of QMC points to be a power of 2, because the lattice rule we use to generate the points is a complete lattice rule if $\NQMC$ is a power of 2 (see \cite{NuREADME}). We choose $\NQMC$ to be a power of 2 by setting $\NQMC(k) = 2^{M(k)},$ where
\beqs
M(k) = \round{\logtwo\mleft(\exp\mleft(\Ctilde \alpha(k)^{-1}\mright)\mright)}.
\eeqs

Based on the results for the QoIs in \cref{tab:qmcalpha} (excluding the results for the QoI being the integral of $u$ and $\grad u((1,1))$, as these seem to display slightly different convergence characteristics), in our numerical experiments below we take $\alpha(k) = 1.38 - 0.19  \loge(k).$ The resulting values of $\NQMC$ are summarised in \cref{tab:qmcpoints}.

\begin{table}[h]
  \centering
  \begin{tabular}{Sc Sc Sc }

\toprule

$k$ & $\exp\mleft(\Ctilde\alpha(k)^{-1}\mright)$ & $\NQMC$\\
\midrule

10 &                                   $2^{11}$ &  $2^{11}$ \\

20 &                                $2^{12.78}$ &  $2^{13}$ \\

30 &                                $2^{14.12}$ &  $2^{14}$ \\

40 &                                $2^{15.26}$ &  $2^{15}$ \\

50 &                                $2^{16.28}$ &  $2^{16}$ \\

60 &                                $2^{17.21}$ &  $2^{17}$ \\

\bottomrule

\end{tabular}


  \caption{The ideal and actual number of QMC points $\NQMC$, chosen so that the QMC error is empirically bounded for all $k$.\label{tab:qmcpoints}}
  \end{table}

\subsubsection{Numerical results for nearby preconditioning applied to QMC}

Now that we have an estimate of how the number of QMC points should scale with $k$ in order to keep the QMC error bounded, we apply nearby preconditioning to QMC (with the number of points chosen as in \cref{tab:qmcpoints}) and observe how the computational work of this nearby-preconditioning-QMC (NP-QMC) algorithm scales with $k.$

As outlined above, we combine our sequential- and parallel-NPQMC algorithms:\label{page:seqandpar}
\bit
\item We first use the sequential algorithm for low $k$ (fixing the maximum number of GMRES iterations) and observe how the number of preconditioners (as a proportion of the number of QMC points) changes with $k$. We thus obtain an empirical relationship between $k$ and the proportion of QMC points used to construct preconditioners.
  \item We then use the parallel algorithm (with the above proportion of preconditioners) for higher values of $k.$
    \eit
    We remark that, in principle, one could use the sequential algorithm for all values of $k$, however, this would take an incredibly long time--- we see in \cref{tab:qmcpoints} that for $k=60$ we must perform $2^{17}$ Helmholtz solves; if we performed these solves sequentially, and each solve took 10 seconds, this computation would take over 2 weeks to complete.

    The results for the sequential algorithm are summarised in \cref{tab:nbpcqmcseq}, for $k = 10,\,20,\,30$. The results show that nearby preconditioning is effective, with the number of preconditioners growing (approximately) linearly in $k$, but at a very low percentage of the total number of solves. Performing a linear fit for the percentage of LU-factorisations, we obtain that the percentage of LU-factorisations grows like $-0.04 + 0.02k$ (see \cref{fig:lu}). This result indicates that although the radius of the balls in which nearby preconditioning is effective decreases with $\cO\mleft(1/k\mright)$, the fact that the number of QMC points increases with $k$ means that a large proportion of the solves are computed using a previously-calculated LU decomposition. Observe that if the number of QMC points remained constant in $k$, we would expect the number of preconditioners to (potentially) increase like $k^J$, because the number of balls of radius $\sim 1/k$ in $\cube{J}$ is $\sim k^J.$

    Based on these sequential results, we then used the parallel algorithm with a target proportion of preconditioners of ($-0.04 + 0.02k$)\%. (Although recall from our discussion above that the actual proportion of preconditioners used can vary due to rounding in the algorithm.) The results of these computations are summarised in \cref{tab:nbpcqmcpar}. We observe that the fraction of preconditioners is approximately $-0.04 + 0.02k$, but the maximum (and average) number of GMRES iterations appears to grow slowly with $k.$ This growth (which did not occur with the sequential algorithm) may be because the placement of the preconditioning points is not optimal with respect to the $\dQMC$ metric; we conjecture that oversampling the number of preconditioners needed (for example, taking a proportion of ($0.05k$)\%) may result in a bounded number of GMRES iterations Nevertheless, we see that nearby preconditioning gives considerable speedup, drastically reducing the number of preconditioners that must be calculated.

    \begin{figure}
      %% Creator: Matplotlib, PGF backend
%%
%% To include the figure in your LaTeX document, write
%%   \input{<filename>.pgf}
%%
%% Make sure the required packages are loaded in your preamble
%%   \usepackage{pgf}
%%
%% Figures using additional raster images can only be included by \input if
%% they are in the same directory as the main LaTeX file. For loading figures
%% from other directories you can use the `import` package
%%   \usepackage{import}
%% and then include the figures with
%%   \import{<path to file>}{<filename>.pgf}
%%
%% Matplotlib used the following preamble
%%   \usepackage{fontspec}
%%   \setmainfont{DejaVuSerif.ttf}[Path=/home/owen/progs/firedrake-complex/firedrake/lib/python3.5/site-packages/matplotlib/mpl-data/fonts/ttf/]
%%   \setsansfont{DejaVuSans.ttf}[Path=/home/owen/progs/firedrake-complex/firedrake/lib/python3.5/site-packages/matplotlib/mpl-data/fonts/ttf/]
%%   \setmonofont{DejaVuSansMono.ttf}[Path=/home/owen/progs/firedrake-complex/firedrake/lib/python3.5/site-packages/matplotlib/mpl-data/fonts/ttf/]
%%
\begingroup%
\makeatletter%
\begin{pgfpicture}%
\pgfpathrectangle{\pgfpointorigin}{\pgfqpoint{5.500000in}{5.500000in}}%
\pgfusepath{use as bounding box, clip}%
\begin{pgfscope}%
\pgfsetbuttcap%
\pgfsetmiterjoin%
\definecolor{currentfill}{rgb}{1.000000,1.000000,1.000000}%
\pgfsetfillcolor{currentfill}%
\pgfsetlinewidth{0.000000pt}%
\definecolor{currentstroke}{rgb}{1.000000,1.000000,1.000000}%
\pgfsetstrokecolor{currentstroke}%
\pgfsetdash{}{0pt}%
\pgfpathmoveto{\pgfqpoint{0.000000in}{0.000000in}}%
\pgfpathlineto{\pgfqpoint{5.500000in}{0.000000in}}%
\pgfpathlineto{\pgfqpoint{5.500000in}{5.500000in}}%
\pgfpathlineto{\pgfqpoint{0.000000in}{5.500000in}}%
\pgfpathclose%
\pgfusepath{fill}%
\end{pgfscope}%
\begin{pgfscope}%
\pgfsetbuttcap%
\pgfsetmiterjoin%
\definecolor{currentfill}{rgb}{1.000000,1.000000,1.000000}%
\pgfsetfillcolor{currentfill}%
\pgfsetlinewidth{0.000000pt}%
\definecolor{currentstroke}{rgb}{0.000000,0.000000,0.000000}%
\pgfsetstrokecolor{currentstroke}%
\pgfsetstrokeopacity{0.000000}%
\pgfsetdash{}{0pt}%
\pgfpathmoveto{\pgfqpoint{0.687500in}{0.605000in}}%
\pgfpathlineto{\pgfqpoint{4.950000in}{0.605000in}}%
\pgfpathlineto{\pgfqpoint{4.950000in}{4.840000in}}%
\pgfpathlineto{\pgfqpoint{0.687500in}{4.840000in}}%
\pgfpathclose%
\pgfusepath{fill}%
\end{pgfscope}%
\begin{pgfscope}%
\pgfsetbuttcap%
\pgfsetroundjoin%
\definecolor{currentfill}{rgb}{0.000000,0.000000,0.000000}%
\pgfsetfillcolor{currentfill}%
\pgfsetlinewidth{0.803000pt}%
\definecolor{currentstroke}{rgb}{0.000000,0.000000,0.000000}%
\pgfsetstrokecolor{currentstroke}%
\pgfsetdash{}{0pt}%
\pgfsys@defobject{currentmarker}{\pgfqpoint{0.000000in}{-0.048611in}}{\pgfqpoint{0.000000in}{0.000000in}}{%
\pgfpathmoveto{\pgfqpoint{0.000000in}{0.000000in}}%
\pgfpathlineto{\pgfqpoint{0.000000in}{-0.048611in}}%
\pgfusepath{stroke,fill}%
}%
\begin{pgfscope}%
\pgfsys@transformshift{0.881250in}{0.605000in}%
\pgfsys@useobject{currentmarker}{}%
\end{pgfscope}%
\end{pgfscope}%
\begin{pgfscope}%
\definecolor{textcolor}{rgb}{0.000000,0.000000,0.000000}%
\pgfsetstrokecolor{textcolor}%
\pgfsetfillcolor{textcolor}%
\pgftext[x=0.881250in,y=0.507778in,,top]{\color{textcolor}\sffamily\fontsize{11.000000}{13.200000}\selectfont \(\displaystyle 10\)}%
\end{pgfscope}%
\begin{pgfscope}%
\pgfsetbuttcap%
\pgfsetroundjoin%
\definecolor{currentfill}{rgb}{0.000000,0.000000,0.000000}%
\pgfsetfillcolor{currentfill}%
\pgfsetlinewidth{0.803000pt}%
\definecolor{currentstroke}{rgb}{0.000000,0.000000,0.000000}%
\pgfsetstrokecolor{currentstroke}%
\pgfsetdash{}{0pt}%
\pgfsys@defobject{currentmarker}{\pgfqpoint{0.000000in}{-0.048611in}}{\pgfqpoint{0.000000in}{0.000000in}}{%
\pgfpathmoveto{\pgfqpoint{0.000000in}{0.000000in}}%
\pgfpathlineto{\pgfqpoint{0.000000in}{-0.048611in}}%
\pgfusepath{stroke,fill}%
}%
\begin{pgfscope}%
\pgfsys@transformshift{2.818750in}{0.605000in}%
\pgfsys@useobject{currentmarker}{}%
\end{pgfscope}%
\end{pgfscope}%
\begin{pgfscope}%
\definecolor{textcolor}{rgb}{0.000000,0.000000,0.000000}%
\pgfsetstrokecolor{textcolor}%
\pgfsetfillcolor{textcolor}%
\pgftext[x=2.818750in,y=0.507778in,,top]{\color{textcolor}\sffamily\fontsize{11.000000}{13.200000}\selectfont \(\displaystyle 20\)}%
\end{pgfscope}%
\begin{pgfscope}%
\pgfsetbuttcap%
\pgfsetroundjoin%
\definecolor{currentfill}{rgb}{0.000000,0.000000,0.000000}%
\pgfsetfillcolor{currentfill}%
\pgfsetlinewidth{0.803000pt}%
\definecolor{currentstroke}{rgb}{0.000000,0.000000,0.000000}%
\pgfsetstrokecolor{currentstroke}%
\pgfsetdash{}{0pt}%
\pgfsys@defobject{currentmarker}{\pgfqpoint{0.000000in}{-0.048611in}}{\pgfqpoint{0.000000in}{0.000000in}}{%
\pgfpathmoveto{\pgfqpoint{0.000000in}{0.000000in}}%
\pgfpathlineto{\pgfqpoint{0.000000in}{-0.048611in}}%
\pgfusepath{stroke,fill}%
}%
\begin{pgfscope}%
\pgfsys@transformshift{4.756250in}{0.605000in}%
\pgfsys@useobject{currentmarker}{}%
\end{pgfscope}%
\end{pgfscope}%
\begin{pgfscope}%
\definecolor{textcolor}{rgb}{0.000000,0.000000,0.000000}%
\pgfsetstrokecolor{textcolor}%
\pgfsetfillcolor{textcolor}%
\pgftext[x=4.756250in,y=0.507778in,,top]{\color{textcolor}\sffamily\fontsize{11.000000}{13.200000}\selectfont \(\displaystyle 30\)}%
\end{pgfscope}%
\begin{pgfscope}%
\definecolor{textcolor}{rgb}{0.000000,0.000000,0.000000}%
\pgfsetstrokecolor{textcolor}%
\pgfsetfillcolor{textcolor}%
\pgftext[x=2.818750in,y=0.304368in,,top]{\color{textcolor}\sffamily\fontsize{11.000000}{13.200000}\selectfont \(\displaystyle k\)}%
\end{pgfscope}%
\begin{pgfscope}%
\pgfsetbuttcap%
\pgfsetroundjoin%
\definecolor{currentfill}{rgb}{0.000000,0.000000,0.000000}%
\pgfsetfillcolor{currentfill}%
\pgfsetlinewidth{0.803000pt}%
\definecolor{currentstroke}{rgb}{0.000000,0.000000,0.000000}%
\pgfsetstrokecolor{currentstroke}%
\pgfsetdash{}{0pt}%
\pgfsys@defobject{currentmarker}{\pgfqpoint{-0.048611in}{0.000000in}}{\pgfqpoint{0.000000in}{0.000000in}}{%
\pgfpathmoveto{\pgfqpoint{0.000000in}{0.000000in}}%
\pgfpathlineto{\pgfqpoint{-0.048611in}{0.000000in}}%
\pgfusepath{stroke,fill}%
}%
\begin{pgfscope}%
\pgfsys@transformshift{0.687500in}{1.291863in}%
\pgfsys@useobject{currentmarker}{}%
\end{pgfscope}%
\end{pgfscope}%
\begin{pgfscope}%
\definecolor{textcolor}{rgb}{0.000000,0.000000,0.000000}%
\pgfsetstrokecolor{textcolor}%
\pgfsetfillcolor{textcolor}%
\pgftext[x=0.395949in,y=1.233825in,left,base]{\color{textcolor}\sffamily\fontsize{11.000000}{13.200000}\selectfont \(\displaystyle 0.3\)}%
\end{pgfscope}%
\begin{pgfscope}%
\pgfsetbuttcap%
\pgfsetroundjoin%
\definecolor{currentfill}{rgb}{0.000000,0.000000,0.000000}%
\pgfsetfillcolor{currentfill}%
\pgfsetlinewidth{0.803000pt}%
\definecolor{currentstroke}{rgb}{0.000000,0.000000,0.000000}%
\pgfsetstrokecolor{currentstroke}%
\pgfsetdash{}{0pt}%
\pgfsys@defobject{currentmarker}{\pgfqpoint{-0.048611in}{0.000000in}}{\pgfqpoint{0.000000in}{0.000000in}}{%
\pgfpathmoveto{\pgfqpoint{0.000000in}{0.000000in}}%
\pgfpathlineto{\pgfqpoint{-0.048611in}{0.000000in}}%
\pgfusepath{stroke,fill}%
}%
\begin{pgfscope}%
\pgfsys@transformshift{0.687500in}{1.998095in}%
\pgfsys@useobject{currentmarker}{}%
\end{pgfscope}%
\end{pgfscope}%
\begin{pgfscope}%
\definecolor{textcolor}{rgb}{0.000000,0.000000,0.000000}%
\pgfsetstrokecolor{textcolor}%
\pgfsetfillcolor{textcolor}%
\pgftext[x=0.395949in,y=1.940057in,left,base]{\color{textcolor}\sffamily\fontsize{11.000000}{13.200000}\selectfont \(\displaystyle 0.4\)}%
\end{pgfscope}%
\begin{pgfscope}%
\pgfsetbuttcap%
\pgfsetroundjoin%
\definecolor{currentfill}{rgb}{0.000000,0.000000,0.000000}%
\pgfsetfillcolor{currentfill}%
\pgfsetlinewidth{0.803000pt}%
\definecolor{currentstroke}{rgb}{0.000000,0.000000,0.000000}%
\pgfsetstrokecolor{currentstroke}%
\pgfsetdash{}{0pt}%
\pgfsys@defobject{currentmarker}{\pgfqpoint{-0.048611in}{0.000000in}}{\pgfqpoint{0.000000in}{0.000000in}}{%
\pgfpathmoveto{\pgfqpoint{0.000000in}{0.000000in}}%
\pgfpathlineto{\pgfqpoint{-0.048611in}{0.000000in}}%
\pgfusepath{stroke,fill}%
}%
\begin{pgfscope}%
\pgfsys@transformshift{0.687500in}{2.704327in}%
\pgfsys@useobject{currentmarker}{}%
\end{pgfscope}%
\end{pgfscope}%
\begin{pgfscope}%
\definecolor{textcolor}{rgb}{0.000000,0.000000,0.000000}%
\pgfsetstrokecolor{textcolor}%
\pgfsetfillcolor{textcolor}%
\pgftext[x=0.395949in,y=2.646289in,left,base]{\color{textcolor}\sffamily\fontsize{11.000000}{13.200000}\selectfont \(\displaystyle 0.5\)}%
\end{pgfscope}%
\begin{pgfscope}%
\pgfsetbuttcap%
\pgfsetroundjoin%
\definecolor{currentfill}{rgb}{0.000000,0.000000,0.000000}%
\pgfsetfillcolor{currentfill}%
\pgfsetlinewidth{0.803000pt}%
\definecolor{currentstroke}{rgb}{0.000000,0.000000,0.000000}%
\pgfsetstrokecolor{currentstroke}%
\pgfsetdash{}{0pt}%
\pgfsys@defobject{currentmarker}{\pgfqpoint{-0.048611in}{0.000000in}}{\pgfqpoint{0.000000in}{0.000000in}}{%
\pgfpathmoveto{\pgfqpoint{0.000000in}{0.000000in}}%
\pgfpathlineto{\pgfqpoint{-0.048611in}{0.000000in}}%
\pgfusepath{stroke,fill}%
}%
\begin{pgfscope}%
\pgfsys@transformshift{0.687500in}{3.410559in}%
\pgfsys@useobject{currentmarker}{}%
\end{pgfscope}%
\end{pgfscope}%
\begin{pgfscope}%
\definecolor{textcolor}{rgb}{0.000000,0.000000,0.000000}%
\pgfsetstrokecolor{textcolor}%
\pgfsetfillcolor{textcolor}%
\pgftext[x=0.395949in,y=3.352521in,left,base]{\color{textcolor}\sffamily\fontsize{11.000000}{13.200000}\selectfont \(\displaystyle 0.6\)}%
\end{pgfscope}%
\begin{pgfscope}%
\pgfsetbuttcap%
\pgfsetroundjoin%
\definecolor{currentfill}{rgb}{0.000000,0.000000,0.000000}%
\pgfsetfillcolor{currentfill}%
\pgfsetlinewidth{0.803000pt}%
\definecolor{currentstroke}{rgb}{0.000000,0.000000,0.000000}%
\pgfsetstrokecolor{currentstroke}%
\pgfsetdash{}{0pt}%
\pgfsys@defobject{currentmarker}{\pgfqpoint{-0.048611in}{0.000000in}}{\pgfqpoint{0.000000in}{0.000000in}}{%
\pgfpathmoveto{\pgfqpoint{0.000000in}{0.000000in}}%
\pgfpathlineto{\pgfqpoint{-0.048611in}{0.000000in}}%
\pgfusepath{stroke,fill}%
}%
\begin{pgfscope}%
\pgfsys@transformshift{0.687500in}{4.116791in}%
\pgfsys@useobject{currentmarker}{}%
\end{pgfscope}%
\end{pgfscope}%
\begin{pgfscope}%
\definecolor{textcolor}{rgb}{0.000000,0.000000,0.000000}%
\pgfsetstrokecolor{textcolor}%
\pgfsetfillcolor{textcolor}%
\pgftext[x=0.395949in,y=4.058754in,left,base]{\color{textcolor}\sffamily\fontsize{11.000000}{13.200000}\selectfont \(\displaystyle 0.7\)}%
\end{pgfscope}%
\begin{pgfscope}%
\pgfsetbuttcap%
\pgfsetroundjoin%
\definecolor{currentfill}{rgb}{0.000000,0.000000,0.000000}%
\pgfsetfillcolor{currentfill}%
\pgfsetlinewidth{0.803000pt}%
\definecolor{currentstroke}{rgb}{0.000000,0.000000,0.000000}%
\pgfsetstrokecolor{currentstroke}%
\pgfsetdash{}{0pt}%
\pgfsys@defobject{currentmarker}{\pgfqpoint{-0.048611in}{0.000000in}}{\pgfqpoint{0.000000in}{0.000000in}}{%
\pgfpathmoveto{\pgfqpoint{0.000000in}{0.000000in}}%
\pgfpathlineto{\pgfqpoint{-0.048611in}{0.000000in}}%
\pgfusepath{stroke,fill}%
}%
\begin{pgfscope}%
\pgfsys@transformshift{0.687500in}{4.823024in}%
\pgfsys@useobject{currentmarker}{}%
\end{pgfscope}%
\end{pgfscope}%
\begin{pgfscope}%
\definecolor{textcolor}{rgb}{0.000000,0.000000,0.000000}%
\pgfsetstrokecolor{textcolor}%
\pgfsetfillcolor{textcolor}%
\pgftext[x=0.395949in,y=4.764986in,left,base]{\color{textcolor}\sffamily\fontsize{11.000000}{13.200000}\selectfont \(\displaystyle 0.8\)}%
\end{pgfscope}%
\begin{pgfscope}%
\definecolor{textcolor}{rgb}{0.000000,0.000000,0.000000}%
\pgfsetstrokecolor{textcolor}%
\pgfsetfillcolor{textcolor}%
\pgftext[x=0.340393in,y=2.722500in,,bottom,rotate=90.000000]{\color{textcolor}\sffamily\fontsize{11.000000}{13.200000}\selectfont \textrm{Number of LU factorisations/Number of QMC points (\%)}}%
\end{pgfscope}%
\begin{pgfscope}%
\pgfpathrectangle{\pgfqpoint{0.687500in}{0.605000in}}{\pgfqpoint{4.262500in}{4.235000in}}%
\pgfusepath{clip}%
\pgfsetbuttcap%
\pgfsetroundjoin%
\definecolor{currentfill}{rgb}{0.000000,0.000000,0.000000}%
\pgfsetfillcolor{currentfill}%
\pgfsetlinewidth{1.003750pt}%
\definecolor{currentstroke}{rgb}{0.000000,0.000000,0.000000}%
\pgfsetstrokecolor{currentstroke}%
\pgfsetdash{}{0pt}%
\pgfsys@defobject{currentmarker}{\pgfqpoint{-0.041667in}{-0.041667in}}{\pgfqpoint{0.041667in}{0.041667in}}{%
\pgfpathmoveto{\pgfqpoint{0.000000in}{-0.041667in}}%
\pgfpathcurveto{\pgfqpoint{0.011050in}{-0.041667in}}{\pgfqpoint{0.021649in}{-0.037276in}}{\pgfqpoint{0.029463in}{-0.029463in}}%
\pgfpathcurveto{\pgfqpoint{0.037276in}{-0.021649in}}{\pgfqpoint{0.041667in}{-0.011050in}}{\pgfqpoint{0.041667in}{0.000000in}}%
\pgfpathcurveto{\pgfqpoint{0.041667in}{0.011050in}}{\pgfqpoint{0.037276in}{0.021649in}}{\pgfqpoint{0.029463in}{0.029463in}}%
\pgfpathcurveto{\pgfqpoint{0.021649in}{0.037276in}}{\pgfqpoint{0.011050in}{0.041667in}}{\pgfqpoint{0.000000in}{0.041667in}}%
\pgfpathcurveto{\pgfqpoint{-0.011050in}{0.041667in}}{\pgfqpoint{-0.021649in}{0.037276in}}{\pgfqpoint{-0.029463in}{0.029463in}}%
\pgfpathcurveto{\pgfqpoint{-0.037276in}{0.021649in}}{\pgfqpoint{-0.041667in}{0.011050in}}{\pgfqpoint{-0.041667in}{0.000000in}}%
\pgfpathcurveto{\pgfqpoint{-0.041667in}{-0.011050in}}{\pgfqpoint{-0.037276in}{-0.021649in}}{\pgfqpoint{-0.029463in}{-0.029463in}}%
\pgfpathcurveto{\pgfqpoint{-0.021649in}{-0.037276in}}{\pgfqpoint{-0.011050in}{-0.041667in}}{\pgfqpoint{0.000000in}{-0.041667in}}%
\pgfpathclose%
\pgfusepath{stroke,fill}%
}%
\begin{pgfscope}%
\pgfsys@transformshift{0.881250in}{0.897366in}%
\pgfsys@useobject{currentmarker}{}%
\end{pgfscope}%
\begin{pgfscope}%
\pgfsys@transformshift{2.818750in}{2.535355in}%
\pgfsys@useobject{currentmarker}{}%
\end{pgfscope}%
\begin{pgfscope}%
\pgfsys@transformshift{4.756250in}{4.647500in}%
\pgfsys@useobject{currentmarker}{}%
\end{pgfscope}%
\end{pgfscope}%
\begin{pgfscope}%
\pgfpathrectangle{\pgfqpoint{0.687500in}{0.605000in}}{\pgfqpoint{4.262500in}{4.235000in}}%
\pgfusepath{clip}%
\pgfsetbuttcap%
\pgfsetroundjoin%
\pgfsetlinewidth{1.505625pt}%
\definecolor{currentstroke}{rgb}{0.000000,0.000000,0.000000}%
\pgfsetstrokecolor{currentstroke}%
\pgfsetdash{{5.550000pt}{2.400000pt}}{0.000000pt}%
\pgfpathmoveto{\pgfqpoint{0.881250in}{0.797500in}}%
\pgfpathlineto{\pgfqpoint{2.818750in}{2.704327in}}%
\pgfpathlineto{\pgfqpoint{4.756250in}{4.611154in}}%
\pgfusepath{stroke}%
\end{pgfscope}%
\begin{pgfscope}%
\pgfsetrectcap%
\pgfsetmiterjoin%
\pgfsetlinewidth{0.803000pt}%
\definecolor{currentstroke}{rgb}{0.000000,0.000000,0.000000}%
\pgfsetstrokecolor{currentstroke}%
\pgfsetdash{}{0pt}%
\pgfpathmoveto{\pgfqpoint{0.687500in}{0.605000in}}%
\pgfpathlineto{\pgfqpoint{0.687500in}{4.840000in}}%
\pgfusepath{stroke}%
\end{pgfscope}%
\begin{pgfscope}%
\pgfsetrectcap%
\pgfsetmiterjoin%
\pgfsetlinewidth{0.000000pt}%
\definecolor{currentstroke}{rgb}{0.000000,0.000000,0.000000}%
\pgfsetstrokecolor{currentstroke}%
\pgfsetstrokeopacity{0.000000}%
\pgfsetdash{}{0pt}%
\pgfpathmoveto{\pgfqpoint{4.950000in}{0.605000in}}%
\pgfpathlineto{\pgfqpoint{4.950000in}{4.840000in}}%
\pgfusepath{}%
\end{pgfscope}%
\begin{pgfscope}%
\pgfsetrectcap%
\pgfsetmiterjoin%
\pgfsetlinewidth{0.803000pt}%
\definecolor{currentstroke}{rgb}{0.000000,0.000000,0.000000}%
\pgfsetstrokecolor{currentstroke}%
\pgfsetdash{}{0pt}%
\pgfpathmoveto{\pgfqpoint{0.687500in}{0.605000in}}%
\pgfpathlineto{\pgfqpoint{4.950000in}{0.605000in}}%
\pgfusepath{stroke}%
\end{pgfscope}%
\begin{pgfscope}%
\pgfsetrectcap%
\pgfsetmiterjoin%
\pgfsetlinewidth{0.000000pt}%
\definecolor{currentstroke}{rgb}{0.000000,0.000000,0.000000}%
\pgfsetstrokecolor{currentstroke}%
\pgfsetstrokeopacity{0.000000}%
\pgfsetdash{}{0pt}%
\pgfpathmoveto{\pgfqpoint{0.687500in}{4.840000in}}%
\pgfpathlineto{\pgfqpoint{4.950000in}{4.840000in}}%
\pgfusepath{}%
\end{pgfscope}%
\begin{pgfscope}%
\pgfsetbuttcap%
\pgfsetmiterjoin%
\definecolor{currentfill}{rgb}{1.000000,1.000000,1.000000}%
\pgfsetfillcolor{currentfill}%
\pgfsetfillopacity{0.800000}%
\pgfsetlinewidth{1.003750pt}%
\definecolor{currentstroke}{rgb}{0.800000,0.800000,0.800000}%
\pgfsetstrokecolor{currentstroke}%
\pgfsetstrokeopacity{0.800000}%
\pgfsetdash{}{0pt}%
\pgfpathmoveto{\pgfqpoint{0.794444in}{4.493535in}}%
\pgfpathlineto{\pgfqpoint{2.212204in}{4.493535in}}%
\pgfpathquadraticcurveto{\pgfqpoint{2.242759in}{4.493535in}}{\pgfqpoint{2.242759in}{4.524090in}}%
\pgfpathlineto{\pgfqpoint{2.242759in}{4.733056in}}%
\pgfpathquadraticcurveto{\pgfqpoint{2.242759in}{4.763611in}}{\pgfqpoint{2.212204in}{4.763611in}}%
\pgfpathlineto{\pgfqpoint{0.794444in}{4.763611in}}%
\pgfpathquadraticcurveto{\pgfqpoint{0.763889in}{4.763611in}}{\pgfqpoint{0.763889in}{4.733056in}}%
\pgfpathlineto{\pgfqpoint{0.763889in}{4.524090in}}%
\pgfpathquadraticcurveto{\pgfqpoint{0.763889in}{4.493535in}}{\pgfqpoint{0.794444in}{4.493535in}}%
\pgfpathclose%
\pgfusepath{stroke,fill}%
\end{pgfscope}%
\begin{pgfscope}%
\pgfsetbuttcap%
\pgfsetroundjoin%
\pgfsetlinewidth{1.505625pt}%
\definecolor{currentstroke}{rgb}{0.000000,0.000000,0.000000}%
\pgfsetstrokecolor{currentstroke}%
\pgfsetdash{{5.550000pt}{2.400000pt}}{0.000000pt}%
\pgfpathmoveto{\pgfqpoint{0.825000in}{4.639897in}}%
\pgfpathlineto{\pgfqpoint{1.130556in}{4.639897in}}%
\pgfusepath{stroke}%
\end{pgfscope}%
\begin{pgfscope}%
\definecolor{textcolor}{rgb}{0.000000,0.000000,0.000000}%
\pgfsetstrokecolor{textcolor}%
\pgfsetfillcolor{textcolor}%
\pgftext[x=1.252778in,y=4.586425in,left,base]{\color{textcolor}\sffamily\fontsize{11.000000}{13.200000}\selectfont \(\displaystyle -0.04 + 0.02k\)}%
\end{pgfscope}%
\end{pgfpicture}%
\makeatother%
\endgroup%

      \caption[The number of LU factorisations in the sequential nearby-preconditioning-QMC algorithm as a percentage of the total number of solves.]{The number of LU factorisations in the sequential algorithm as a percentage of the total number of solves.\label{fig:lu}}
      \end{figure}
    \afterpage{% Heard about from https://tex.stackexchange.com/questions/11471/how-to-wrap-text-around-landscape-page
\begin{landscape} % Heard about from https://tex.stackexchange.com/questions/19017/how-to-place-a-table-on-a-new-page-with-landscape-orientation-without-clearing-t/19021#19021 and https://tex.stackexchange.com/questions/25369/how-to-rotate-a-table 
    \begin{table}
  \centering
  \begin{tabular}{Sc Sc Sc Sc Sc Sc}
\toprule

$k$ & \# LU factorisations & \makecell{Total \#\\linear systems} & \makecell{\# LU factorisations$/$\\\# linear systems}(\%) & \makecell{Average \#\\GMRES iterations} & \makecell{Max. \#\\GMRES iterations}\\
\midrule

10 &                    5 &                                2048 &                                               0.24 &                                    7.13 &                                   10 \\

20 &                   44 &                               16384 &                                               0.27 &                                    7.24 &                                   10 \\

30 &                  216 &                              131072 &                                               0.16 &                                    7.23 &                                   10 \\

\bottomrule

\end{tabular}


  \caption{Results applying our sequential nearby-preconditioning-Quasi-Monte-Carlo algorithm, with the maximum number of GMRES iterations $=10$.}\label{tab:nbpcqmcseq}
\end{table}

\begin{table}
  \centering
  \begin{tabular}{Sc Sc Sc Sc Sc Sc}
\toprule

$k$ & \# LU factorisations & \makecell{Total \#\\linear systems} & \makecell{\# LU factorisations$/$\\\# linear systems}(\%) & \makecell{Average \#\\GMRES iterations} & \makecell{Max. \#\\GMRES iterations}\\
\midrule

10 &                    4 &                                2048 &                                               0.20 &                                    6.46 &                                   10 \\

20 &                   33 &                                8192 &                                               0.40 &                                    6.42 &                                   11 \\

30 &                  127 &                               16384 &                                               0.78 &                                    6.66 &                                   13 \\

40 &                  207 &                               32768 &                                               0.63 &                                    7.16 &                                   15 \\

50 &                 1027 &                               65536 &                                               1.57 &                                    7.07 &                                   14 \\

60 &                 1444 &                              131072 &                                               1.10 &                                    7.41 &                                   16 \\

\bottomrule

\end{tabular}


  \caption{Results applying our parallel nearby-preconditioning-Quasi-Monte-Carlo algorithm with the target proportion of preconditioners as $(-0.04+0.02k)$\%.}\label{tab:nbpcqmcpar}
\end{table}
\end{landscape}
}

    In conclusion, we see that nearby preconditioning gives a significant speedup when applied to a QMC model problem.%, with around 98\% of solves being computed using a previously-calculated LU decomposition.% We therefore expect that this technique will give significant speed up when applied to other, more realistic problems.
    

    
%% \section{Extension of the results to the truncated exterior Dirichlet problem}\label{sec:TEDP}

%% We now briefly outline how the results in \cref{sec:main} above can be extended to \cref{prob:vtedp}, the Truncated Exterior Dirichlet Problem.

%% %\subsection{Definition of the TEDP and analogues of the results in \cref{sec:3}}

%% %% \paragraph{The impedance boundary $\Gamma_I$.} By comparing \cref{eq:src,eq:ibc}, we see that, in the case $g_I=0$, the TEDP approximates the DtN operator $T_R$ by $\ri k$. Indeed, by using Green's first identity and the definition of the normal derivative (see, e.g., \cite[Lemma 4.3]{Mc:00}), show that the boundary condition on $\Gamma_I$ imposed in the variational problem \cref{prob:vtedp} is 
%% %% %In this BVP, the DtN operator $T_R$ Sommerfeld radiation condition 
%% %% \beq\label{eq:imp}
%% %% \dudnu - \ri k\gamma u = g_I \ton \Gamma_I.
%% %% \eeq
%% %% where $\nu$ is the unit outward-pointing normal vector to $\Omega$ on $\Gamma_I$.

%% \paragraph{Existence and uniqueness of a solution to the TEDP.} The sesquilinear form $\aT(\cdot,\cdot)$ defined in \cref{eq:aT} satisfies the G\aa rding inequality \cref{eq:gardingbrief}, and existence and uniqueness of a solution to the TEDP follow under the same condition on $A$ (piecewise-Lipschitz) as for the EDP, as discussed in \cref{thm:tedp}.%sec:vpGm}; in the case of Lipschitz scalar $A$, these unique-continuation arguments are summarised in \cite[\S2]{GrSa:18}.

%% \paragraph{Finite-element/Galerkin solution.}
%% The Galerkin matrix $\Amat$ is defined exactly as in \cref{eq:matrixAdef}, except that 
%% \beq\label{eq:NTEDP}
%% \big(\Nmat\big)_{ij}\de i k\int_{\Gamma_I}  (\gamma\phi_i) \,\gamma \phi_j.
%% \eeq

%% \paragraph{The adjoint sesquilinear form.} For the TEDP, the adjoint sesquilinear form is given by 
%% \beq\label{eq:TEDPadjoint}
%% a^\dagger(u,v) \de \int_{D} 
%% \Big((A \grad u)\cdot\grad \vb
%%  - k^2 n u\vb\Big) +i k\int_{\Gamma_I} \gamma u\, \overline{\gamma v};
%% \eeq
%% then \cref{eq:A*} holds (with $\Nmat$ now given by \cref{eq:NTEDP}), and the analogue of \cref{lem:adjoint} follows in a straightforward way.


%% \paragraph{The analogues of \cref{cond:1nbpc,cond:2}.}
%% The statement of the TEDP analogues of \cref{cond:1nbpc,cond:2} are the same as for the EDP, apart from the following.
%% \ben
%% \item
%% $\supp \,f$ need not be a subset of $\widetilde{\Omega}$ (i.e.~the support of $f$ can go up to the impedance boundary $\Gamma_I$), and
%% \item the assumption $g_I= 0$ needs to be added to \cref{cond:1nbpc} and Part (i) of \cref{cond:2}.
%% \een
%%  Note that, since $\aT(\cdot,\cdot)$ for the TEDP satisfies the same G\aa rding inequality \cref{eq:gardingbrief} as $a(\cdot,\cdot)$ for the EDP, \cref{lem:H1} holds for the TEDP under the TEDP-analogue of \cref{cond:1nbpc}.

%% \paragraph{The main results \cref{thm:1,cor:1}.}
%% Since \cref{cond:1nbpc,cond:2} are essentially unchanged from the EDP case, \cref{lem:keylemma1,lem:keylemma2} hold for the TEDP, and thus so do \cref{thm:1,cor:1,cor:1a}.

%% \paragraph{The PDE results \cref{thm:2} and \cref{lem:sharp}.}

%% The PDE bound \cref{thm:2} relies only on \cref{lem:H1}, which, as stated above, also holds for the TEDP. Therefore \cref{thm:2} holds for the TEDP under the TEDP-analogue of \cref{cond:1nbpc} described above. The construction in \cref{lem:sharp} to show sharpness of the bound in \cref{thm:1} (at least when $\Aso= \Ast= I$) also holds for the TEDP; this is because one can choose the supports of $\chi$ and $\widetilde{\chi}$ to be contained inside $\widetilde{\Omega}$, and then $u^{(1)}$ and $u^{(2)}$ defined in \cref{lem:sharp} satisfy the impedance boundary condition \cref{eq:imp} on $\Gamma_I$.

%% %% \paragraph{When the TEDP-analogue of \cref{cond:1nbpc} holds.}

%% %% In \cref{sec:cond1hold} we discussed 4 situations (Cases 1-4) where \cref{cond:1nbpc} is proved to hold for the EDP. We now discuss the TEDP-analogues of these.
%% %% %Cases 1, 3, and 4 (there is no proof yet for the TEDP-analogue of Case 2).

%% %% \emph{Cases 1 and 2: $\Aso$, $\nso$, and $\Gamma_I$  are $C^\infty$.} 
%% %% With the rays defined as in the EDP case (by the Melrose--Sj{\"o}strand generalized bicharacteristic flow 
%% %% \cite[\S24.3]{Ho:85}), the TEDP-analogue of nontrapping for the EDP is the assumption that 
%% %% every ray eventually hits the boundary at a \emph{non-diffractive point} (defined in \cite[Page 1037]{BaLeRa:92}). Note that, in the case $\Dm=\emptyset$ $\Aso= I$, and $\nso=1$, every ray eventually hits the boundary at a non-diffractive point by \cite[Lemma 5.3]{BaSpWu:16}.
%% %% Under the additional assumption that $\nso= 1$, \cref{cond:1nbpc} follows from the results of \cite{BaLeRa:92} by combining \cite[Theorem 1.8]{BaSpWu:16} and \cite[Remark 5.6]{BaSpWu:16}, but $C^{(1)}_{\rm bound}$ is not given explicitly.

%% %% \emph{Case 3: $\Dm$ is starshaped with respect to the origin, $\Aso$ and $\nso$ are Lipschitz and satisfy radial monotonicity-like conditions.}
%% %% When $\Gamma_I$ is also starshaped with respect to the origin and $A$ and $n$ satisfy \cref{eq:A1nbpc} and \cref{eq:n1nbpc} respectively (with $\Dp$ replaced by $\Omega$), 
%% %% \cite[Theorem A.6(i)]{GrPeSp:19} proves that
%% %% \cref{cond:1nbpc} holds, with an explicit expression for $C^{(1)}_{\rm bound}$. Analogous results when (a) $2\Aso - (\bx\cdot\nabla)\Aso \geq \mu_1$ and $\nso= 1$,
%% %% and  (b) $\Aso= I$ and  $2\nso + \bx \cdot \nabla \nso \geq \mu_2$, 
%% %% are contained in \cite[Theorem A.6(ii)]{GrPeSp:19} and \cite[Theorem A.6(iii)]{GrPeSp:19} respectively.
%% %% When $A$ is scalar, these results were also proved in \cite[Theorem 1]{BrGaPe:17} and, when $\Aso= I$ and $\Dm=\emptyset$, also in \cite[Theorem 3.2]{GrSa:18}.

%% %% \emph{Case 4: %\item[Case 4:]
%% %%  $\Aso$ and $\nso$ are allowed to be discontinuous.}
%% %% %\een
%% %% \cref{cond:1nbpc} is proved in \cite{CaVo:10} (without an explicit expression for $C^{(1)}_{\rm bound}$) when $\Dm$ is $C^\infty$ and nontrapping, $\Gamma_I$ is $C^\infty$, $\Aso= I $, and $\nso$ is a piecewise-constant, monotonically non-decreasing function, jumping on interfaces that are $C^\infty$ with strictly positive curvature.
%% %% Recall from \cref{cond:1nbpc} that \cite[Theorem 2.7]{GrPeSp:19} proves that \cref{cond:1nbpc} holds for the EDP (with an explicit expression for $C^{(1)}_{\rm bound}$) when $\Dm$ is starshaped with respect to the origin, $A$ and $n$ are $L^\infty$, with $A$ monotonically \emph{non-increasing} in the radial direction, and $n$ monotonically \emph{non-decreasing}. This proof can be extended to the TEDP, with the additional assumption that $\Gamma_I$ is star-shaped with respect to the origin; see the discussion in \cite[Section A.2]{GrPeSp:19}.

%% %\cref{cond:1nbpc} is proved, with an explicit expression for $C^{(1)}_{\rm bound}$, when 

%% %\newpage
%% %
%% %\section*{Questions for Th\'eo}
%% %
%% %\ben
%% %\item At the place marked A on the scanned pages, you seem to use the inequality 
%% %\beq\label{eq:Theo1}
%% %\vert\vert\vert \xi - \cP_h \xi\vert\vert\vert \lesssim h^\alpha \N{u_\phi- \cP_h u_\phi}_{0,\Omega}.
%% %\eeq
%% %\een
%% %
%% %\newpag

%% %% \section*{Owen to do list}
%% %% \ben
%% %% \item Varying  $\|\Aso-\Ast\|_{L^\infty}$ and $\|\nso-\nst\|_{L^\infty}$ in standard GMRES.
%% %% \item Computations where $\|\Aso-\Ast\|_{L^\infty}$ and $\|\nso-\nst\|_{L^\infty}$ are sometimes large; is having the standard deviations of these $\sim 1/k$ good enough for $k$-independent GMRES iterations?
%% %% \item ***on backburner*** Checking under what conditions (if any) Part (ii) \cref{cond:2} holds by running the following experiment:
%% %% %\item Exciting experiments for random $n$ that you told us about last week.
%% %% %\item In the weighted norm, the condition on $A$ is ``$k \|\Aso-\Ast\|_{L^\infty}$ sufficiently small" but in the Euclidean norm the best we have so far is ``$h^{-1} \|\Aso-\Ast\|_{L^\infty}$ sufficiently small". You indicated before that experiments seemed to indicate that ``$k \|\Aso-\Ast\|_{L^\infty}$ sufficiently small" seemed correct for the Euclidean norm too. The next time we meet, can you show me these results please?
%% %% %\item Please run the following numerical experiment.
%% %% \bit
%% %% \item TEDP with $\Omega$ a square/rectangle.
%% %% \item $\Aso$ being at least Lipschitz (but smooth is fine). To keep things simple, just take scalar- (as opposed to matrix-) valued $\Aso$ and don't worry about making it nontrapping.
%% %% \item Smoothness of $\nso$ doesn't really matter, just take smooth in the first instance for simplicity (and also don't worry about nontrapping).
%% %% \item $\Vhp$ piecewise linear.
%% %% \item Linear system $\Amato \uvec = \Smat_{A} \balpha$ for some arbitrary complex-valued vector $\balpha$ and some arbitrary $A\in L^\infty$. (I claim this corresponds to the problem described in Part (ii) of \cref{cond:2},  but please check this!)
%% %% \item For each $\Aso, \nso, \balpha$, solve linear system for increasing values of $k$, first with $h\sim k^{-2}$, and then with $h\sim k^{-3/2}$.
%% %% \item Goal: see if the bound \cref{eq:bound4} holds, using 
%% %% \beqs
%% %% \N{\sum_j \alpha_j (A\nabla \phi_j)}_{\LtD} \quad \text{ as a proxy for } \quad \N{\LE}_{(\HokD)'}.
%% %% \eeqs
%% %% \eit
%% %% %\item Varying  $\|\Aso-\Ast\|_{L^\infty}$ and $\|\nso-\nst\|_{L^\infty}$ in \emph{weighted} GMRES.
%% %% \een

\section{Review of related techniques in the literature}\label{sec:nbpclitreview}
   
Having proved rigorous results on the effectiveness of nearby preconditioning, and also applied it to a UQ algorithm, we now review similar computational techniques (applied to other problems) which can be found in the literature. Whilst the idea of \emph{nearby} preconditioning introduced here is, as far as we are aware, novel, there has been a body of work on the closely-related idea of \emph{mean-based} preconditioning. In mean-based preconditioning a \emph{single} preconditioner is calculated corresponding to the mean of the random coefficient. This is in contrast to nearby preconditioning, where \emph{multiple} preconditioners are calculated, corresponding to each realisation in a particular subset of all the realisations. Mean-based preconditioning has been most extensively studied for the stationary diffusion equation
    \beqs
\grad \cdot \mleft(a\grad u\mright)  = -f,
\eeqs
with a small number of works analysing other PDEs, including two works on the Helmholtz equation. We will first explain the idea of mean-based preconditioning before we review the literature applying it to the stationary diffusion equation and other PDEs, and finally turning our attention to mean-based preconditioning for the Helmholtz equation. In general, the computational and mathematical results in the literature show that mean-based preconditioning  is effective if the variance of the random parameters are small enough, i.e.,  if most of the samples are sufficiently close to the mean.

Mean-based preconditioning was first developed for the stationary diffusion equation in the context of so-called Stochastic Spectral Finite-Element Methods (SSFEM). In these methods, the random field $a$ is given by a series expansion, such as a Karhunen--Lo\`eve expansion, and the dependence of $u$ on the random parameters is computed using a Polynomial Chaos expansion (see, e.g., \cite[Section 2.4.2]{GhSp:12}. The resulting problem is then discretised in the whole space $D \times \Omega$, where $D$ is the spatial domain and $\Omega$ the probability space. The resulting discrete problems involve very large matrices of the form
\beq\label{eq:sgmatrix}
\Amat \otimes \Gmat,
\eeq
where $\Amat$ is a standard finite-element matrix, $\Gmat$ is a matrix corresponding to the discretisation in $\Omega,$ and $\otimes$ is the Kronecker product. For SSFEMs (and the closely-related stochastic-Galerkin FEMs, see, e.g. \cite{BaTeZo:04}, which also have discretisations of the form \cref{eq:sgmatrix}) a mean-based preconditioner is a matrix of the form
\beq\label{eq:mbsg}
\Amatmean \otimes \ImatOmega,
\eeq
where $\Amatmean$ is the standard finite-element matrix corresponding to the mean and $\ImatOmega$ is the identity matrix associated with the discretisation on $\Omega.$ Using a mean-based preconditioner of the form \cref{eq:mbsg} gives considerable computational savings, as only one preconditioner of a standard finite-element matrix need be calculated.

When stochastic Galerkin methods are used with so-called `doubly-orthogonal bases' (see, e.g., \cite[Section 3.2]{ErPoSiUl:09}), then the linear system \cref{eq:sgmatrix} decouples into many distinct standard finite-element matrices; mean-based preconditioning has also been investigated in this context (and in the context of stochastic collocation methods (see, e.g., \cite{BaNoTe:07}), where one similarly obtains many different standard finite-element matrices) as will be discussed below.

The main insight gleaned from studies of mean-based preconditioning is that, as stated above, if the variance of $a$ (or any other stochastic coefficients) is sufficiently small, then mean-based preconditioning is effective.

The initial computational work on mean-based preconditioning for the stationary diffusion equation was carried out by Ghanem and Kruger \cite{GhKr:96}, Pellissetti and Ghanem \cite{PeGh:00}, and Keese \cite{Ke:04}, with theory (proving bounds on the eigenvalues of the preconditioned matrices) following from Powell and Elman \cite{PoEl:09} and Ernst, Powell, Silvester, and Ullmann \cite{ErPoSiUl:09}. These eigenvalue bounds are analagous to results in \cref{sec:main} above, as they allow one to infer convergence properties of the iterative method used. All of the above results were for $a$ given by a (real or artificial) Karhunen--Lo\`eve expansion; that is, in the case where $a$ depends linearly on the random parameter. In the case where $a$ is a lognormal random field (and so the dependence is no longer linear), Powell and Ullmann \cite{PoUl:10} declared mean-based preconditioners to be ineffective, and so developed more advanced preconditioners; in contrast, Ullmann, Elman and Ernst \cite{UlElEr:12} transformed a stationary diffusion problem with lognormal coefficient into a stationary convection-diffusion problem with a random coefficient depending linearly on the noise, before proving eigenvalue bounds as before. With a more computational slant, Tipireddy, Phipps and Ghanem \cite{TiPhGh:10} and Rosseel and Vandewalle \cite{RoVa:10} compared the computational properties of several mean-based preconditioners and Elman, Miller, Phipps and Tuminaro \cite{ElMiPhTu:11} compared the computational cost of mean-based preconditioners for stochastic Galerkin and stochastic collocation methods.

Seeking to apply mean-based preconditioning to more challenging problems, Powell and Silvester \cite{PoSi:12} performed computational investigations for mean-based preconditioners applied to stochastic Galerkin discretisations of the steady-state Navier--Stokes equations, and Soused\'ik and Elman \cite{SoEl:16} introduced a Gauss--Seidel-type preconditioner, using mean-based ideas, for the steady-state Navier--Stokes equations. Finally, Khan, Powell, and Silvester \cite{KhPoSi:19} applied mean-based preconditioning to stochastic Galerkin discretisations of the equations for nearly-incompressible elasticity.

The works applying mean-based preconditioning to many individual systems (for the stationary diffusion equation) are those of Eiermann, Ernst and Ullmann \cite{EiErUl:07}; Ernst, Powell, Silvester, and Ullmann \cite{ErPoSiUl:09}; and Gordon and Powell \cite{GoPo:12}. The first work contained computational results in a (decoupled) stochastic Galerkin setting; the second proved eigenvalue bounds in the same setting, and the last proved rigorous eigenvalue bounds in a stochastic collocation setting. All these works assume linear dependence on the noise, and show that mean-based preconditioning works well when the variance is sufficiently small.

We now turn our attention to mean-based preconditioning for the Helmholtz equation. The first work we discuss is the recent work of Wang and Liao \cite{WaLi:19}. They discretise a stochastic Helmholtz problem with $k=10$ and $n$ given by a truncated Karhunen--Lo\`eve expansion (with either 4 terms or 1 term) and use a generalised polynomial chaos (gPC) expansion (see, e.g., \cite{XiKa:02}) for the solution $u$. Whilst they use mean-based preconditioning (in the `Kronecker product' sense) they are more interested in investigating the effect of the number of terms in the gPC expansion on the accuracy of the discrete solution. Nonetheless, they see convergence using the mean-based preconditioner, although more iterations are needed when the random field is `close to' exciting a resonant frequency (see \cite[Example 4.2]{WaLi:19}).

The work most similar to ours is the work of Jin and Cai \cite{JiCa:09}, who use a stochastic Galerkin discretisation with a doubly-orthogonal basis for a stochastic Helmholtz equation, resulting in around 5000 linear systems. They take $k = 225$ and a Karhunen--Lo\`eve expansion with 4 terms for both (scalar-valued) $A$ and $n$. The random variables in the Karhunen--Lo\`eve expansions are $\Unif(-\sqrt{3},\sqrt{3})$ and $\Unif(-45\sqrt{3},45\sqrt{3})$ for $A$ and $n$ respectively. Their mean-based preconditioner is a 1-level additive Schwarz preconditioner, and they compare resuing the preconditioner with reusing the Krylov subspaces (an idea first introduced by Parks, De Sturler, Mackey, Johnson, and Maiti in \cite{PadeMaJoMa:06}), as well as combining both techniques. Intriguingly, they see no additional benefit from reusing the preconditioner, but considerable benefit from recycling the Krylov subspaces. Based on our results in this \lcnamecref{chap:nbpc}, we conjecture that they see no benefit from a single mean-based preconditioner because $k$ is reasonably large, and therefore for most of the realisations, $k\NLiDRR{\EXP{n}-\nsj}$ and $k\NLiDRRdtd{\EXP{A}-\Asj}$ are not sufficiently small. and so there is little-to-no effect on the number of GMRES iterations from mean-based preconditioning. We conjecture that if they had used multiple preconditioners distributed around the stochastic parameter space, they would have seen computational improvements, as described in this \lcnamecref{chap:nbpc}.
    
\section{Probabilistic nearby preconditioning results}\label{sec:nbpcstochastic}

We now briefly overview how one can prove probabalistic results on the effectiveness of nearby preconditioning. All of the results in \cref{sec:intronbpc,sec:num,sec:3,sec:weaknorm} above have been for deterministic (as opposed to stochastic) coefficients $A$ and $n$ (and we then applied these deterministic results to QMC methods for the Helmholtz equation in \cref{sec:nbpcqmc}). Therefore we now turn our attention to obtaining probabilistic results on the effectiveness of nearby preconditioning for stochastic Helmholtz problems, i.e., \cref{prob:msedp,prob:somsedp,prob:svsedp} from \cref{chap:stochastic}. Firstly, in \cref{cor:stonbpcas} below, we prove an `essentially deterministic' result on the effectiveness of nearby preconditioning, before proving probabilistic results on the effectiveness of nearby preconditioning applied to stochastic problems. However, we will see that our efforts to prove probabilistic results are restricted by the applicability of the Elman estimate (\cref{thm:GMRES1_intro} above).

Throughout this \cref{sec:nbpcstochastic} we consider \cref{prob:msedp} from \cref{chap:stochastic} but with $A=I$, i.e., for simplicity we only consider the case of random $n$, although everything we say could be easily extended to include random $A$. To maintain consistent notation with the rest of this \lcnamecref{chap:nbpc} we will use a superscript ${}^{(2)}$ to refer to the stochastic problem (e.g., the random coefficient will be $\nst(\omega)$, the solution will be $\ust(\omega)$, the matrices arising from the finite-element discretiation will be $\Amatt(\omega),$ etc.). We let $\nso \in \LiDRR$ define a \emph{deterministic} Helmholtz problem. We will use the discretisation of this deterministic Helmholtz problem to precondition the discretisations of the realisations of the stochastic Helmholtz problem. I.e., we will consider the performance of GMRES applied to
\beq\label{eq:stopc}
\AmatoI\Amatt(\omega)\uvec = \AmatoI \fvec.
\eeq
For simplicity, in all that follows we will measure $\no-\nt$ in the $L^{\infty}$ norm, although one could use any of the weaker norms discussed in \cref{sec:weaknorm} above, and obtain analogous results.

\subsection{Probabilistic theory for nearby preconditioning}
\bde[Number of GMRES iterations required for convergence]

\

\noindent Let $\GMRES{\eps}{\nso}{\nst}$ denote the number of iterations required for GMRES in the unweighted norm $\Nt{\cdot}$ with $\Nt{\rvecz} = 1,$ applied to
\beqs
\AmatoI\Amatt  \uvec = \AmatoI \fvec
\eeqs
to converge to within a tolerance $\eps,$ i.e., to achieve
\beqs
\frac{\Nt{\rvecm}}{\Nt{\fvec}} < \eps.
\eeqs
\ede

Note that $\GMRES{\eps}{\nso}{\nst}$ is a random variable, see \cref{lem:randomvariable} below.

If we apply \cref{cor:1a} to the problem \cref{eq:stopc} we can straightforwardly conclude the following \lcnamecref{cor:stonbpcas}

\bco[Almost-sure nearby preconditioning]\label{cor:stonbpcas}
Let $0 < \eps < 1,$ $\nst:\Omega \rightarrow \LiDRR$ satisfy the assumptions at the start of \cref{sec:hh-results}, $\no, \,\Dm,$ and $f$ be as in \cref{prob:vgen}, and let the assumptions of \cref{cor:1a} hold. Then $\GMRES{\eps}{\nso}{\nst}$ is bounded independently of $k$ almost surely if
\beq\label{eq:nbpcas}
\NLiDRR{\nso-\nst(\omega)} \leq \frac1{2\Ct k}
\eeq
almost surely.
\eco

%Observe that implicit in \cref{cor:stonbpcas} is the fact that $\GMRES{\eps}{\nso}{\nst}$ is a random variable; we sketch a proof of this fact now.

\ble[$\GMRES{\eps}{\no}{\nt}$ is a random variable]\label{lem:randomvariable}
Under the assumptions of \cref{cor:stonbpcas}, $\GMRES{\eps}{\nso}{\nst}$ is a random variable, i.e., $\GMRES{\eps}{\nso}{\nst}:\Omega\rightarrow \RR$ is measurable.
\ele

\bpf[Sketch Proof of \cref{lem:randomvariable}]
All of the operations used in constructing the vectors $\xvecm$ in the GMRES algorithm are measurable functions of $\xvecmmo$ and $\AmatoI\Amatt$ (see, e.g., \cite[Algorithms 11.4.2 and 5.1.3]{GoVa:13}), therefore $\mleft(\rvecm\mright)_{m=1}^N$ is a sequence of random variables, i.e., a stochastic process (see, e.g., \cite[Definition 2.1.4]{Ok:13}). The stopping criterion $\Nt{\rvecm}/\Nt{\fvec} < \eps$ is an exit time for the stochastic process $\xvecm$ from the set $\CCN \setminus \ball{\CCN}{\xvecs}{\eps\Nt{\fvec}},$ where $\xvecs$ is the true solution. Therefore, because we assume $\OFP$ is a complete probability space, it follows from, e.g.,  \cite[Example 7.2.2]{Ok:13} that $\GMRES{\eps}{\no}{\nt}$ is a stopping time (see \cite[Definition 7.2.1]{Ok:13}). Because $\GMRES{\eps}{\no}{\nt}$ is a stopping time, it is measurable with respect to the associated filtration (see, e.g., \cite[Definition 3.2.2]{Ok:13}), and so is measurable with respect to $\cF$; i.e., $\GMRES{\eps}{\no}{\nt}$ is a random variable.
\epf

The numerical results in \cref{sec:num} above can be seen (in part) as confirming \cref{cor:stonbpcas}. Recall that in \cref{sec:num} we let $\nso-\nst$ be a piecewise-constant random field, and we fixed $\alpha = \NLiDRR{\nso-\nst}$ or $\NLiDRRdtd{\Aso-\Ast}$ almost surely. When we fixed $\alpha = 0.5/k$ almost surely (see \cref{fig:linfinityA2,fig:linfinityn2}) we saw that the number of GMRES iterations was bounded independently of $k.$ This behaviour is precisely that given in \cref{cor:stonbpcas}.

\bre[Drawbacks of \cref{cor:stonbpcas}]\label{rem:notideal}
There are two drawbacks of \cref{cor:stonbpcas}:
\ben
\item\label[itemdrawback]{it:notideal1} The condition \cref{eq:nbpcas} must hold almost surely, and
  \item\label[itemdrawback]{it:notideal2} \Cref{cor:stonbpcas} does not give any explicit information on how the distribution of the number of GMRES iterations depends on the distribution of $\NLiDRR{\nso-\nst}.$
    \een
    \Cref{it:notideal1} is not ideal because in many physically realistic problems $\NLiDRR{\no-\nt(\omega)}$ may be unbounded (e.g., if $\nt$ is a lognormal random field) or even if bounded may not satisfy the condition \cref{eq:nbpcas} almost surely.% \Cref{it:notideal2} is not ideal because it means one cannot infer information about the distribution of the number of GMRES iterations from the distribution of$\NLiDRR{\nso-\nst(\omega)}.$
    \ere

    To correct the deficiencies described in \cref{rem:notideal} one would aim to  prove a bound on the number of GMRES iterations depending explicitly on $\NLiDRR{\nso-\nst(\omega)}$, and then use this bound to prove a probabalistic estimate for the number of GMRES iterations. Such a bound is given in \cref{lem:probgmres1} in \cref{app:probnbpc}. However, such a bound will be highly pessimistic, and will impart little useful information. The reason for this lack of information is that the Elman estimate (\cref{cor:GMRES_intro} above) when applied to the nearby-preconditioned system $\AmatoI\Amatt$) only applies when $k\NLiDRRdtd{\Aso-\Ast}$ and $k\NLiDRR{\nso-\nst}$ are sufficiently small (as we saw in \cref{cor:1} above). Therefore one can only obtain detailed information on how the number of GMRES iterations depends on $\NLiDRRdtd{\Aso-\Ast}$ and $\NLiDRR{\nso-\nst}$ when these quantities are small (informally, when they are $\lesssim 1/k$). In all other cases (again, informally, when these quantities are $\gtrsim 1/k$) the only statement one can make about the convergence of GMRES is that there will be at most $N$ iterations, where $N$ is the number of degrees of freedom (this result is recalled in \cref{cor:gmresguaranteed} below). In summary, current results on GMRES convergence will only allow us to prove what are likely to be very pessimistic bounds on how the number of GMRES iterations for $\AmatoI\Amatt$ depends on $\NLiDRRdtd{\Aso-\Ast}$ and $\NLiDRR{\nso-\nst}$. For completeness, we record these results in \cref{app:probnbpc}.






%%  TO HERE BRO

%%     We first define notation for the number of GMRES iterations required for convergence. Because \cref{lem:probgmres1} below is a \emph{deterministic} result, i.e., it does not require $\nst$ to be a random field. Therefore for this \lcnamecref{lem:probgmres1} only, we assume $\nst$ is as given at the beginning of this \lcnamecref{chap:nbpc}.


    


%% \bre[\Cref{thm:probgmres} is pessimistic]\label{rem:pessimistic}
%% Observe that we expect the bound in \cref{thm:probgmres} to be pessimistic, i.e., we expect that \cref{eq:GMRESprob} is not sharp in its dependence on $\alpha$. In particular, we expect \cref{eq:GMRESprob} is not sharp for large values of $R.$ We now show that in most cases, for large values of $R$ the left-hand side of \cref{eq:GMRESprob} is independent of $R$.

%% One can show via elementary calculus that for $\alpha < 1$ $\Gfnname$ achieves its maximum when $\alpha = 1/3$ (assuming that for $\alpha < 1$ the expression involving $\alpha$ in \cref{eq:gdef} is always  at most $N$). Also observe that $\Gfnname$ over the range $\alpha  \in (0,1)$ only depends on $k$ through the dependence of $\alpha$ on $k.$ Therefore, the maximum of $\Gfnname$ over $\alpha \in (0,1)$ is independent of $k.$ Let $\Gfnmaxlo$ denote the value of this maximum. Then, for any $R \in \mleft(\Gfnmaxlo,N\mright)$, the estimate $\PP\mleft(\Gfn{\nso-\nst} \leq R\mright)$ is equal to $\PP\mleft(\alpha<1\mright) = \PP\mleft(\NLiDRR{\nso-\nst} < 1/\mleft(\Ct k\mright)\mright),$ i.e., the lower-bound in \cref{eq:GMRESprob} is \emph{independent} of $R$, for $R \in \mleft(\Gfnmaxlo,N\mright)$  This is almost certainly not sharp - we would expect $\PP\mleft(\GMRES{\eps}{\nso}{\nst} \leq R\mright)$ to increase with $R$. However, because the only rigorous result we have available if $\alpha \geq 1$ is \cref{cor:gmresguaranteed}, we cannot prove a better bound.
%% \ere

\subsection{Numerical probabalistic results for nearby preconditioning}\label{sec:qualgmres}

Notwithstanding the fact that we are limited in the probabilistic results that we can \emph{prove} about nearby preconditioning, we will now see that we \emph{observe} reasonable probabailistic behaviour when we perform numerical experiments. We again recall that (informally) \cref{cor:stonbpcas} states that we obtain almost-surely bounded GMRES iterations if $\NLiDRR{\nso-\nst} \lesssim 1/k.$ A plausible probabalistic analogue of this result would be that we have bounded \emph{average} number of GMRES iterations if the standard deviation of $\NLiDRR{\nso-\nst}$ is of the order $1/k$. We expect this result because the standard deviation of a random variable is a (probabilistic) measure of its variation. In \cref{cor:stonbpcas} we show that the number of GMRES iterations is bounded almost surely if the variation in $\NLiDRR{\nso-\nst}$ is bounded (of the order $1/k$) almost surely. Therefore, it reasonable to assume that the probabilistic analogue of the number of GMRES iterations (the average) is bounded if the probabilistic analogue of the variation in $\NLiDRR{\nso-\nst}$ (the standard deviation) is bounded (of the order $1/k$). Will we see exactly this behaviour in our numerical experiments.

In our numerical experiments we use the computational setup described in \cref{app:compsetup}, with $f=1$ and $\gI=0,$ $\Aso=\Ast=I$, and $\NLiDRR{\nso-\nst}$ given by an exponential random variables with standard deviation $\sigma.$  We consider three cases:
\ben
\item\label[itemcase]{it:sigma1} $\displaystyle \sigma  = 1,$
\item\label[itemcase]{it:sigma2} $\displaystyle \sigma  = \frac{1}k,$ and
  \item\label[itemcase]{it:sigma3} $\displaystyle \sigma  = \frac{1}{k^2}$.
    \een

    For each of these cases we calculate
    \beq\label{eq:gmresprob}
    \PP\mleft(\GMRES{\eps}{\no}{\nt} \leq 12\mright).
    \eeq

    Based on the reasoning above we expect that in \cref{it:sigma2} the probability \cref{eq:gmresprob} \emph{is constant} as $k$ increases, and using similar reasoning, we expect that in \cref{it:sigma1} the probability \cref{eq:gmresprob} \emph{decreases} as $k$ increases and in \cref{it:sigma3} the probability \cref{eq:gmresprob} \emph{increases} as $k$ increases.   This is approximately the behaviour we observe in \cref{fig:prob-plot-0.0,fig:prob-plot-1.0,fig:prob-plot-2.0}. This behaviour demonstrates that whilst the theory developed in the rest of this \lcnamecref{chap:nbpc} does not allow us to easily prove useful results about the probabalistic behaviour of nearby preconditioning, the theory does give us \emph{intuition} as to what the probabilistic behavour will be.

%% \begin{figure}[p]
%%   \centering
%%   \begin{subfigure}{\textwidth}
%%     \centering
%% %% Creator: Matplotlib, PGF backend
%%
%% To include the figure in your LaTeX document, write
%%   \input{<filename>.pgf}
%%
%% Make sure the required packages are loaded in your preamble
%%   \usepackage{pgf}
%%
%% Figures using additional raster images can only be included by \input if
%% they are in the same directory as the main LaTeX file. For loading figures
%% from other directories you can use the `import` package
%%   \usepackage{import}
%% and then include the figures with
%%   \import{<path to file>}{<filename>.pgf}
%%
%% Matplotlib used the following preamble
%%   \usepackage{fontspec}
%%   \setmainfont{DejaVuSerif.ttf}[Path=/home/owen/progs/firedrake-complex/firedrake/lib/python3.5/site-packages/matplotlib/mpl-data/fonts/ttf/]
%%   \setsansfont{DejaVuSans.ttf}[Path=/home/owen/progs/firedrake-complex/firedrake/lib/python3.5/site-packages/matplotlib/mpl-data/fonts/ttf/]
%%   \setmonofont{DejaVuSansMono.ttf}[Path=/home/owen/progs/firedrake-complex/firedrake/lib/python3.5/site-packages/matplotlib/mpl-data/fonts/ttf/]
%%
\begingroup%
\makeatletter%
\begin{pgfpicture}%
\pgfpathrectangle{\pgfpointorigin}{\pgfqpoint{6.400000in}{4.800000in}}%
\pgfusepath{use as bounding box, clip}%
\begin{pgfscope}%
\pgfsetbuttcap%
\pgfsetmiterjoin%
\definecolor{currentfill}{rgb}{1.000000,1.000000,1.000000}%
\pgfsetfillcolor{currentfill}%
\pgfsetlinewidth{0.000000pt}%
\definecolor{currentstroke}{rgb}{1.000000,1.000000,1.000000}%
\pgfsetstrokecolor{currentstroke}%
\pgfsetdash{}{0pt}%
\pgfpathmoveto{\pgfqpoint{0.000000in}{0.000000in}}%
\pgfpathlineto{\pgfqpoint{6.400000in}{0.000000in}}%
\pgfpathlineto{\pgfqpoint{6.400000in}{4.800000in}}%
\pgfpathlineto{\pgfqpoint{0.000000in}{4.800000in}}%
\pgfpathclose%
\pgfusepath{fill}%
\end{pgfscope}%
\begin{pgfscope}%
\pgfsetbuttcap%
\pgfsetmiterjoin%
\definecolor{currentfill}{rgb}{1.000000,1.000000,1.000000}%
\pgfsetfillcolor{currentfill}%
\pgfsetlinewidth{0.000000pt}%
\definecolor{currentstroke}{rgb}{0.000000,0.000000,0.000000}%
\pgfsetstrokecolor{currentstroke}%
\pgfsetstrokeopacity{0.000000}%
\pgfsetdash{}{0pt}%
\pgfpathmoveto{\pgfqpoint{0.800000in}{0.528000in}}%
\pgfpathlineto{\pgfqpoint{5.760000in}{0.528000in}}%
\pgfpathlineto{\pgfqpoint{5.760000in}{4.224000in}}%
\pgfpathlineto{\pgfqpoint{0.800000in}{4.224000in}}%
\pgfpathclose%
\pgfusepath{fill}%
\end{pgfscope}%
\begin{pgfscope}%
\pgfsetbuttcap%
\pgfsetroundjoin%
\definecolor{currentfill}{rgb}{0.000000,0.000000,0.000000}%
\pgfsetfillcolor{currentfill}%
\pgfsetlinewidth{0.803000pt}%
\definecolor{currentstroke}{rgb}{0.000000,0.000000,0.000000}%
\pgfsetstrokecolor{currentstroke}%
\pgfsetdash{}{0pt}%
\pgfsys@defobject{currentmarker}{\pgfqpoint{0.000000in}{-0.048611in}}{\pgfqpoint{0.000000in}{0.000000in}}{%
\pgfpathmoveto{\pgfqpoint{0.000000in}{0.000000in}}%
\pgfpathlineto{\pgfqpoint{0.000000in}{-0.048611in}}%
\pgfusepath{stroke,fill}%
}%
\begin{pgfscope}%
\pgfsys@transformshift{1.025455in}{0.528000in}%
\pgfsys@useobject{currentmarker}{}%
\end{pgfscope}%
\end{pgfscope}%
\begin{pgfscope}%
\definecolor{textcolor}{rgb}{0.000000,0.000000,0.000000}%
\pgfsetstrokecolor{textcolor}%
\pgfsetfillcolor{textcolor}%
\pgftext[x=1.025455in,y=0.430778in,,top]{\color{textcolor}\sffamily\fontsize{10.000000}{12.000000}\selectfont 10}%
\end{pgfscope}%
\begin{pgfscope}%
\pgfsetbuttcap%
\pgfsetroundjoin%
\definecolor{currentfill}{rgb}{0.000000,0.000000,0.000000}%
\pgfsetfillcolor{currentfill}%
\pgfsetlinewidth{0.803000pt}%
\definecolor{currentstroke}{rgb}{0.000000,0.000000,0.000000}%
\pgfsetstrokecolor{currentstroke}%
\pgfsetdash{}{0pt}%
\pgfsys@defobject{currentmarker}{\pgfqpoint{0.000000in}{-0.048611in}}{\pgfqpoint{0.000000in}{0.000000in}}{%
\pgfpathmoveto{\pgfqpoint{0.000000in}{0.000000in}}%
\pgfpathlineto{\pgfqpoint{0.000000in}{-0.048611in}}%
\pgfusepath{stroke,fill}%
}%
\begin{pgfscope}%
\pgfsys@transformshift{1.776970in}{0.528000in}%
\pgfsys@useobject{currentmarker}{}%
\end{pgfscope}%
\end{pgfscope}%
\begin{pgfscope}%
\definecolor{textcolor}{rgb}{0.000000,0.000000,0.000000}%
\pgfsetstrokecolor{textcolor}%
\pgfsetfillcolor{textcolor}%
\pgftext[x=1.776970in,y=0.430778in,,top]{\color{textcolor}\sffamily\fontsize{10.000000}{12.000000}\selectfont 15}%
\end{pgfscope}%
\begin{pgfscope}%
\pgfsetbuttcap%
\pgfsetroundjoin%
\definecolor{currentfill}{rgb}{0.000000,0.000000,0.000000}%
\pgfsetfillcolor{currentfill}%
\pgfsetlinewidth{0.803000pt}%
\definecolor{currentstroke}{rgb}{0.000000,0.000000,0.000000}%
\pgfsetstrokecolor{currentstroke}%
\pgfsetdash{}{0pt}%
\pgfsys@defobject{currentmarker}{\pgfqpoint{0.000000in}{-0.048611in}}{\pgfqpoint{0.000000in}{0.000000in}}{%
\pgfpathmoveto{\pgfqpoint{0.000000in}{0.000000in}}%
\pgfpathlineto{\pgfqpoint{0.000000in}{-0.048611in}}%
\pgfusepath{stroke,fill}%
}%
\begin{pgfscope}%
\pgfsys@transformshift{2.528485in}{0.528000in}%
\pgfsys@useobject{currentmarker}{}%
\end{pgfscope}%
\end{pgfscope}%
\begin{pgfscope}%
\definecolor{textcolor}{rgb}{0.000000,0.000000,0.000000}%
\pgfsetstrokecolor{textcolor}%
\pgfsetfillcolor{textcolor}%
\pgftext[x=2.528485in,y=0.430778in,,top]{\color{textcolor}\sffamily\fontsize{10.000000}{12.000000}\selectfont 20}%
\end{pgfscope}%
\begin{pgfscope}%
\pgfsetbuttcap%
\pgfsetroundjoin%
\definecolor{currentfill}{rgb}{0.000000,0.000000,0.000000}%
\pgfsetfillcolor{currentfill}%
\pgfsetlinewidth{0.803000pt}%
\definecolor{currentstroke}{rgb}{0.000000,0.000000,0.000000}%
\pgfsetstrokecolor{currentstroke}%
\pgfsetdash{}{0pt}%
\pgfsys@defobject{currentmarker}{\pgfqpoint{0.000000in}{-0.048611in}}{\pgfqpoint{0.000000in}{0.000000in}}{%
\pgfpathmoveto{\pgfqpoint{0.000000in}{0.000000in}}%
\pgfpathlineto{\pgfqpoint{0.000000in}{-0.048611in}}%
\pgfusepath{stroke,fill}%
}%
\begin{pgfscope}%
\pgfsys@transformshift{3.280000in}{0.528000in}%
\pgfsys@useobject{currentmarker}{}%
\end{pgfscope}%
\end{pgfscope}%
\begin{pgfscope}%
\definecolor{textcolor}{rgb}{0.000000,0.000000,0.000000}%
\pgfsetstrokecolor{textcolor}%
\pgfsetfillcolor{textcolor}%
\pgftext[x=3.280000in,y=0.430778in,,top]{\color{textcolor}\sffamily\fontsize{10.000000}{12.000000}\selectfont 25}%
\end{pgfscope}%
\begin{pgfscope}%
\pgfsetbuttcap%
\pgfsetroundjoin%
\definecolor{currentfill}{rgb}{0.000000,0.000000,0.000000}%
\pgfsetfillcolor{currentfill}%
\pgfsetlinewidth{0.803000pt}%
\definecolor{currentstroke}{rgb}{0.000000,0.000000,0.000000}%
\pgfsetstrokecolor{currentstroke}%
\pgfsetdash{}{0pt}%
\pgfsys@defobject{currentmarker}{\pgfqpoint{0.000000in}{-0.048611in}}{\pgfqpoint{0.000000in}{0.000000in}}{%
\pgfpathmoveto{\pgfqpoint{0.000000in}{0.000000in}}%
\pgfpathlineto{\pgfqpoint{0.000000in}{-0.048611in}}%
\pgfusepath{stroke,fill}%
}%
\begin{pgfscope}%
\pgfsys@transformshift{4.031515in}{0.528000in}%
\pgfsys@useobject{currentmarker}{}%
\end{pgfscope}%
\end{pgfscope}%
\begin{pgfscope}%
\definecolor{textcolor}{rgb}{0.000000,0.000000,0.000000}%
\pgfsetstrokecolor{textcolor}%
\pgfsetfillcolor{textcolor}%
\pgftext[x=4.031515in,y=0.430778in,,top]{\color{textcolor}\sffamily\fontsize{10.000000}{12.000000}\selectfont 30}%
\end{pgfscope}%
\begin{pgfscope}%
\pgfsetbuttcap%
\pgfsetroundjoin%
\definecolor{currentfill}{rgb}{0.000000,0.000000,0.000000}%
\pgfsetfillcolor{currentfill}%
\pgfsetlinewidth{0.803000pt}%
\definecolor{currentstroke}{rgb}{0.000000,0.000000,0.000000}%
\pgfsetstrokecolor{currentstroke}%
\pgfsetdash{}{0pt}%
\pgfsys@defobject{currentmarker}{\pgfqpoint{0.000000in}{-0.048611in}}{\pgfqpoint{0.000000in}{0.000000in}}{%
\pgfpathmoveto{\pgfqpoint{0.000000in}{0.000000in}}%
\pgfpathlineto{\pgfqpoint{0.000000in}{-0.048611in}}%
\pgfusepath{stroke,fill}%
}%
\begin{pgfscope}%
\pgfsys@transformshift{4.783030in}{0.528000in}%
\pgfsys@useobject{currentmarker}{}%
\end{pgfscope}%
\end{pgfscope}%
\begin{pgfscope}%
\definecolor{textcolor}{rgb}{0.000000,0.000000,0.000000}%
\pgfsetstrokecolor{textcolor}%
\pgfsetfillcolor{textcolor}%
\pgftext[x=4.783030in,y=0.430778in,,top]{\color{textcolor}\sffamily\fontsize{10.000000}{12.000000}\selectfont 35}%
\end{pgfscope}%
\begin{pgfscope}%
\pgfsetbuttcap%
\pgfsetroundjoin%
\definecolor{currentfill}{rgb}{0.000000,0.000000,0.000000}%
\pgfsetfillcolor{currentfill}%
\pgfsetlinewidth{0.803000pt}%
\definecolor{currentstroke}{rgb}{0.000000,0.000000,0.000000}%
\pgfsetstrokecolor{currentstroke}%
\pgfsetdash{}{0pt}%
\pgfsys@defobject{currentmarker}{\pgfqpoint{0.000000in}{-0.048611in}}{\pgfqpoint{0.000000in}{0.000000in}}{%
\pgfpathmoveto{\pgfqpoint{0.000000in}{0.000000in}}%
\pgfpathlineto{\pgfqpoint{0.000000in}{-0.048611in}}%
\pgfusepath{stroke,fill}%
}%
\begin{pgfscope}%
\pgfsys@transformshift{5.534545in}{0.528000in}%
\pgfsys@useobject{currentmarker}{}%
\end{pgfscope}%
\end{pgfscope}%
\begin{pgfscope}%
\definecolor{textcolor}{rgb}{0.000000,0.000000,0.000000}%
\pgfsetstrokecolor{textcolor}%
\pgfsetfillcolor{textcolor}%
\pgftext[x=5.534545in,y=0.430778in,,top]{\color{textcolor}\sffamily\fontsize{10.000000}{12.000000}\selectfont 40}%
\end{pgfscope}%
\begin{pgfscope}%
\definecolor{textcolor}{rgb}{0.000000,0.000000,0.000000}%
\pgfsetstrokecolor{textcolor}%
\pgfsetfillcolor{textcolor}%
\pgftext[x=3.280000in,y=0.240809in,,top]{\color{textcolor}\sffamily\fontsize{10.000000}{12.000000}\selectfont \(\displaystyle k\)}%
\end{pgfscope}%
\begin{pgfscope}%
\pgfsetbuttcap%
\pgfsetroundjoin%
\definecolor{currentfill}{rgb}{0.000000,0.000000,0.000000}%
\pgfsetfillcolor{currentfill}%
\pgfsetlinewidth{0.803000pt}%
\definecolor{currentstroke}{rgb}{0.000000,0.000000,0.000000}%
\pgfsetstrokecolor{currentstroke}%
\pgfsetdash{}{0pt}%
\pgfsys@defobject{currentmarker}{\pgfqpoint{-0.048611in}{0.000000in}}{\pgfqpoint{0.000000in}{0.000000in}}{%
\pgfpathmoveto{\pgfqpoint{0.000000in}{0.000000in}}%
\pgfpathlineto{\pgfqpoint{-0.048611in}{0.000000in}}%
\pgfusepath{stroke,fill}%
}%
\begin{pgfscope}%
\pgfsys@transformshift{0.800000in}{0.848886in}%
\pgfsys@useobject{currentmarker}{}%
\end{pgfscope}%
\end{pgfscope}%
\begin{pgfscope}%
\definecolor{textcolor}{rgb}{0.000000,0.000000,0.000000}%
\pgfsetstrokecolor{textcolor}%
\pgfsetfillcolor{textcolor}%
\pgftext[x=0.393533in,y=0.796124in,left,base]{\color{textcolor}\sffamily\fontsize{10.000000}{12.000000}\selectfont 0.01}%
\end{pgfscope}%
\begin{pgfscope}%
\pgfsetbuttcap%
\pgfsetroundjoin%
\definecolor{currentfill}{rgb}{0.000000,0.000000,0.000000}%
\pgfsetfillcolor{currentfill}%
\pgfsetlinewidth{0.803000pt}%
\definecolor{currentstroke}{rgb}{0.000000,0.000000,0.000000}%
\pgfsetstrokecolor{currentstroke}%
\pgfsetdash{}{0pt}%
\pgfsys@defobject{currentmarker}{\pgfqpoint{-0.048611in}{0.000000in}}{\pgfqpoint{0.000000in}{0.000000in}}{%
\pgfpathmoveto{\pgfqpoint{0.000000in}{0.000000in}}%
\pgfpathlineto{\pgfqpoint{-0.048611in}{0.000000in}}%
\pgfusepath{stroke,fill}%
}%
\begin{pgfscope}%
\pgfsys@transformshift{0.800000in}{2.141757in}%
\pgfsys@useobject{currentmarker}{}%
\end{pgfscope}%
\end{pgfscope}%
\begin{pgfscope}%
\definecolor{textcolor}{rgb}{0.000000,0.000000,0.000000}%
\pgfsetstrokecolor{textcolor}%
\pgfsetfillcolor{textcolor}%
\pgftext[x=0.393533in,y=2.088995in,left,base]{\color{textcolor}\sffamily\fontsize{10.000000}{12.000000}\selectfont 0.02}%
\end{pgfscope}%
\begin{pgfscope}%
\pgfsetbuttcap%
\pgfsetroundjoin%
\definecolor{currentfill}{rgb}{0.000000,0.000000,0.000000}%
\pgfsetfillcolor{currentfill}%
\pgfsetlinewidth{0.803000pt}%
\definecolor{currentstroke}{rgb}{0.000000,0.000000,0.000000}%
\pgfsetstrokecolor{currentstroke}%
\pgfsetdash{}{0pt}%
\pgfsys@defobject{currentmarker}{\pgfqpoint{-0.048611in}{0.000000in}}{\pgfqpoint{0.000000in}{0.000000in}}{%
\pgfpathmoveto{\pgfqpoint{0.000000in}{0.000000in}}%
\pgfpathlineto{\pgfqpoint{-0.048611in}{0.000000in}}%
\pgfusepath{stroke,fill}%
}%
\begin{pgfscope}%
\pgfsys@transformshift{0.800000in}{3.434628in}%
\pgfsys@useobject{currentmarker}{}%
\end{pgfscope}%
\end{pgfscope}%
\begin{pgfscope}%
\definecolor{textcolor}{rgb}{0.000000,0.000000,0.000000}%
\pgfsetstrokecolor{textcolor}%
\pgfsetfillcolor{textcolor}%
\pgftext[x=0.393533in,y=3.381866in,left,base]{\color{textcolor}\sffamily\fontsize{10.000000}{12.000000}\selectfont 0.03}%
\end{pgfscope}%
\begin{pgfscope}%
\definecolor{textcolor}{rgb}{0.000000,0.000000,0.000000}%
\pgfsetstrokecolor{textcolor}%
\pgfsetfillcolor{textcolor}%
\pgftext[x=0.337977in,y=2.376000in,,bottom,rotate=90.000000]{\color{textcolor}\sffamily\fontsize{10.000000}{12.000000}\selectfont Probability Number of GMRES iterations is at most 12}%
\end{pgfscope}%
\begin{pgfscope}%
\pgfpathrectangle{\pgfqpoint{0.800000in}{0.528000in}}{\pgfqpoint{4.960000in}{3.696000in}}%
\pgfusepath{clip}%
\pgfsetbuttcap%
\pgfsetroundjoin%
\definecolor{currentfill}{rgb}{0.121569,0.466667,0.705882}%
\pgfsetfillcolor{currentfill}%
\pgfsetlinewidth{1.003750pt}%
\definecolor{currentstroke}{rgb}{0.121569,0.466667,0.705882}%
\pgfsetstrokecolor{currentstroke}%
\pgfsetdash{}{0pt}%
\pgfsys@defobject{currentmarker}{\pgfqpoint{-0.020833in}{-0.020833in}}{\pgfqpoint{0.020833in}{0.020833in}}{%
\pgfpathmoveto{\pgfqpoint{0.000000in}{-0.020833in}}%
\pgfpathcurveto{\pgfqpoint{0.005525in}{-0.020833in}}{\pgfqpoint{0.010825in}{-0.018638in}}{\pgfqpoint{0.014731in}{-0.014731in}}%
\pgfpathcurveto{\pgfqpoint{0.018638in}{-0.010825in}}{\pgfqpoint{0.020833in}{-0.005525in}}{\pgfqpoint{0.020833in}{0.000000in}}%
\pgfpathcurveto{\pgfqpoint{0.020833in}{0.005525in}}{\pgfqpoint{0.018638in}{0.010825in}}{\pgfqpoint{0.014731in}{0.014731in}}%
\pgfpathcurveto{\pgfqpoint{0.010825in}{0.018638in}}{\pgfqpoint{0.005525in}{0.020833in}}{\pgfqpoint{0.000000in}{0.020833in}}%
\pgfpathcurveto{\pgfqpoint{-0.005525in}{0.020833in}}{\pgfqpoint{-0.010825in}{0.018638in}}{\pgfqpoint{-0.014731in}{0.014731in}}%
\pgfpathcurveto{\pgfqpoint{-0.018638in}{0.010825in}}{\pgfqpoint{-0.020833in}{0.005525in}}{\pgfqpoint{-0.020833in}{0.000000in}}%
\pgfpathcurveto{\pgfqpoint{-0.020833in}{-0.005525in}}{\pgfqpoint{-0.018638in}{-0.010825in}}{\pgfqpoint{-0.014731in}{-0.014731in}}%
\pgfpathcurveto{\pgfqpoint{-0.010825in}{-0.018638in}}{\pgfqpoint{-0.005525in}{-0.020833in}}{\pgfqpoint{0.000000in}{-0.020833in}}%
\pgfpathclose%
\pgfusepath{stroke,fill}%
}%
\begin{pgfscope}%
\pgfsys@transformshift{1.025455in}{4.056000in}%
\pgfsys@useobject{currentmarker}{}%
\end{pgfscope}%
\begin{pgfscope}%
\pgfsys@transformshift{1.070545in}{3.927174in}%
\pgfsys@useobject{currentmarker}{}%
\end{pgfscope}%
\begin{pgfscope}%
\pgfsys@transformshift{1.115636in}{3.805518in}%
\pgfsys@useobject{currentmarker}{}%
\end{pgfscope}%
\begin{pgfscope}%
\pgfsys@transformshift{1.160727in}{3.690450in}%
\pgfsys@useobject{currentmarker}{}%
\end{pgfscope}%
\begin{pgfscope}%
\pgfsys@transformshift{1.205818in}{3.581448in}%
\pgfsys@useobject{currentmarker}{}%
\end{pgfscope}%
\begin{pgfscope}%
\pgfsys@transformshift{1.250909in}{3.478046in}%
\pgfsys@useobject{currentmarker}{}%
\end{pgfscope}%
\begin{pgfscope}%
\pgfsys@transformshift{1.296000in}{3.379823in}%
\pgfsys@useobject{currentmarker}{}%
\end{pgfscope}%
\begin{pgfscope}%
\pgfsys@transformshift{1.341091in}{3.286399in}%
\pgfsys@useobject{currentmarker}{}%
\end{pgfscope}%
\begin{pgfscope}%
\pgfsys@transformshift{1.386182in}{3.197431in}%
\pgfsys@useobject{currentmarker}{}%
\end{pgfscope}%
\begin{pgfscope}%
\pgfsys@transformshift{1.431273in}{3.112608in}%
\pgfsys@useobject{currentmarker}{}%
\end{pgfscope}%
\begin{pgfscope}%
\pgfsys@transformshift{1.476364in}{3.031646in}%
\pgfsys@useobject{currentmarker}{}%
\end{pgfscope}%
\begin{pgfscope}%
\pgfsys@transformshift{1.521455in}{2.954288in}%
\pgfsys@useobject{currentmarker}{}%
\end{pgfscope}%
\begin{pgfscope}%
\pgfsys@transformshift{1.566545in}{2.880298in}%
\pgfsys@useobject{currentmarker}{}%
\end{pgfscope}%
\begin{pgfscope}%
\pgfsys@transformshift{1.611636in}{2.809462in}%
\pgfsys@useobject{currentmarker}{}%
\end{pgfscope}%
\begin{pgfscope}%
\pgfsys@transformshift{1.656727in}{2.741581in}%
\pgfsys@useobject{currentmarker}{}%
\end{pgfscope}%
\begin{pgfscope}%
\pgfsys@transformshift{1.701818in}{2.676474in}%
\pgfsys@useobject{currentmarker}{}%
\end{pgfscope}%
\begin{pgfscope}%
\pgfsys@transformshift{1.746909in}{2.613976in}%
\pgfsys@useobject{currentmarker}{}%
\end{pgfscope}%
\begin{pgfscope}%
\pgfsys@transformshift{1.792000in}{2.553932in}%
\pgfsys@useobject{currentmarker}{}%
\end{pgfscope}%
\begin{pgfscope}%
\pgfsys@transformshift{1.837091in}{2.496200in}%
\pgfsys@useobject{currentmarker}{}%
\end{pgfscope}%
\begin{pgfscope}%
\pgfsys@transformshift{1.882182in}{2.440649in}%
\pgfsys@useobject{currentmarker}{}%
\end{pgfscope}%
\begin{pgfscope}%
\pgfsys@transformshift{1.927273in}{2.387159in}%
\pgfsys@useobject{currentmarker}{}%
\end{pgfscope}%
\begin{pgfscope}%
\pgfsys@transformshift{1.972364in}{2.335616in}%
\pgfsys@useobject{currentmarker}{}%
\end{pgfscope}%
\begin{pgfscope}%
\pgfsys@transformshift{2.017455in}{2.285917in}%
\pgfsys@useobject{currentmarker}{}%
\end{pgfscope}%
\begin{pgfscope}%
\pgfsys@transformshift{2.062545in}{2.237963in}%
\pgfsys@useobject{currentmarker}{}%
\end{pgfscope}%
\begin{pgfscope}%
\pgfsys@transformshift{2.107636in}{2.191665in}%
\pgfsys@useobject{currentmarker}{}%
\end{pgfscope}%
\begin{pgfscope}%
\pgfsys@transformshift{2.152727in}{2.146938in}%
\pgfsys@useobject{currentmarker}{}%
\end{pgfscope}%
\begin{pgfscope}%
\pgfsys@transformshift{2.197818in}{2.103704in}%
\pgfsys@useobject{currentmarker}{}%
\end{pgfscope}%
\begin{pgfscope}%
\pgfsys@transformshift{2.242909in}{2.061889in}%
\pgfsys@useobject{currentmarker}{}%
\end{pgfscope}%
\begin{pgfscope}%
\pgfsys@transformshift{2.288000in}{2.021425in}%
\pgfsys@useobject{currentmarker}{}%
\end{pgfscope}%
\begin{pgfscope}%
\pgfsys@transformshift{2.333091in}{1.982246in}%
\pgfsys@useobject{currentmarker}{}%
\end{pgfscope}%
\begin{pgfscope}%
\pgfsys@transformshift{2.378182in}{1.944293in}%
\pgfsys@useobject{currentmarker}{}%
\end{pgfscope}%
\begin{pgfscope}%
\pgfsys@transformshift{2.423273in}{1.907509in}%
\pgfsys@useobject{currentmarker}{}%
\end{pgfscope}%
\begin{pgfscope}%
\pgfsys@transformshift{2.468364in}{1.871841in}%
\pgfsys@useobject{currentmarker}{}%
\end{pgfscope}%
\begin{pgfscope}%
\pgfsys@transformshift{2.513455in}{1.837239in}%
\pgfsys@useobject{currentmarker}{}%
\end{pgfscope}%
\begin{pgfscope}%
\pgfsys@transformshift{2.558545in}{1.803656in}%
\pgfsys@useobject{currentmarker}{}%
\end{pgfscope}%
\begin{pgfscope}%
\pgfsys@transformshift{2.603636in}{1.771047in}%
\pgfsys@useobject{currentmarker}{}%
\end{pgfscope}%
\begin{pgfscope}%
\pgfsys@transformshift{2.648727in}{1.739371in}%
\pgfsys@useobject{currentmarker}{}%
\end{pgfscope}%
\begin{pgfscope}%
\pgfsys@transformshift{2.693818in}{1.708588in}%
\pgfsys@useobject{currentmarker}{}%
\end{pgfscope}%
\begin{pgfscope}%
\pgfsys@transformshift{2.738909in}{1.678660in}%
\pgfsys@useobject{currentmarker}{}%
\end{pgfscope}%
\begin{pgfscope}%
\pgfsys@transformshift{2.784000in}{1.649554in}%
\pgfsys@useobject{currentmarker}{}%
\end{pgfscope}%
\begin{pgfscope}%
\pgfsys@transformshift{2.829091in}{1.621235in}%
\pgfsys@useobject{currentmarker}{}%
\end{pgfscope}%
\begin{pgfscope}%
\pgfsys@transformshift{2.874182in}{1.593672in}%
\pgfsys@useobject{currentmarker}{}%
\end{pgfscope}%
\begin{pgfscope}%
\pgfsys@transformshift{2.919273in}{1.566835in}%
\pgfsys@useobject{currentmarker}{}%
\end{pgfscope}%
\begin{pgfscope}%
\pgfsys@transformshift{2.964364in}{1.540695in}%
\pgfsys@useobject{currentmarker}{}%
\end{pgfscope}%
\begin{pgfscope}%
\pgfsys@transformshift{3.009455in}{1.515227in}%
\pgfsys@useobject{currentmarker}{}%
\end{pgfscope}%
\begin{pgfscope}%
\pgfsys@transformshift{3.054545in}{1.490404in}%
\pgfsys@useobject{currentmarker}{}%
\end{pgfscope}%
\begin{pgfscope}%
\pgfsys@transformshift{3.099636in}{1.466202in}%
\pgfsys@useobject{currentmarker}{}%
\end{pgfscope}%
\begin{pgfscope}%
\pgfsys@transformshift{3.144727in}{1.442598in}%
\pgfsys@useobject{currentmarker}{}%
\end{pgfscope}%
\begin{pgfscope}%
\pgfsys@transformshift{3.189818in}{1.419570in}%
\pgfsys@useobject{currentmarker}{}%
\end{pgfscope}%
\begin{pgfscope}%
\pgfsys@transformshift{3.234909in}{1.397098in}%
\pgfsys@useobject{currentmarker}{}%
\end{pgfscope}%
\begin{pgfscope}%
\pgfsys@transformshift{3.280000in}{1.375161in}%
\pgfsys@useobject{currentmarker}{}%
\end{pgfscope}%
\begin{pgfscope}%
\pgfsys@transformshift{3.325091in}{1.353741in}%
\pgfsys@useobject{currentmarker}{}%
\end{pgfscope}%
\begin{pgfscope}%
\pgfsys@transformshift{3.370182in}{1.332819in}%
\pgfsys@useobject{currentmarker}{}%
\end{pgfscope}%
\begin{pgfscope}%
\pgfsys@transformshift{3.415273in}{1.312379in}%
\pgfsys@useobject{currentmarker}{}%
\end{pgfscope}%
\begin{pgfscope}%
\pgfsys@transformshift{3.460364in}{1.292403in}%
\pgfsys@useobject{currentmarker}{}%
\end{pgfscope}%
\begin{pgfscope}%
\pgfsys@transformshift{3.505455in}{1.272877in}%
\pgfsys@useobject{currentmarker}{}%
\end{pgfscope}%
\begin{pgfscope}%
\pgfsys@transformshift{3.550545in}{1.253785in}%
\pgfsys@useobject{currentmarker}{}%
\end{pgfscope}%
\begin{pgfscope}%
\pgfsys@transformshift{3.595636in}{1.235113in}%
\pgfsys@useobject{currentmarker}{}%
\end{pgfscope}%
\begin{pgfscope}%
\pgfsys@transformshift{3.640727in}{1.216848in}%
\pgfsys@useobject{currentmarker}{}%
\end{pgfscope}%
\begin{pgfscope}%
\pgfsys@transformshift{3.685818in}{1.198975in}%
\pgfsys@useobject{currentmarker}{}%
\end{pgfscope}%
\begin{pgfscope}%
\pgfsys@transformshift{3.730909in}{1.181483in}%
\pgfsys@useobject{currentmarker}{}%
\end{pgfscope}%
\begin{pgfscope}%
\pgfsys@transformshift{3.776000in}{1.164360in}%
\pgfsys@useobject{currentmarker}{}%
\end{pgfscope}%
\begin{pgfscope}%
\pgfsys@transformshift{3.821091in}{1.147593in}%
\pgfsys@useobject{currentmarker}{}%
\end{pgfscope}%
\begin{pgfscope}%
\pgfsys@transformshift{3.866182in}{1.131173in}%
\pgfsys@useobject{currentmarker}{}%
\end{pgfscope}%
\begin{pgfscope}%
\pgfsys@transformshift{3.911273in}{1.115087in}%
\pgfsys@useobject{currentmarker}{}%
\end{pgfscope}%
\begin{pgfscope}%
\pgfsys@transformshift{3.956364in}{1.099327in}%
\pgfsys@useobject{currentmarker}{}%
\end{pgfscope}%
\begin{pgfscope}%
\pgfsys@transformshift{4.001455in}{1.083883in}%
\pgfsys@useobject{currentmarker}{}%
\end{pgfscope}%
\begin{pgfscope}%
\pgfsys@transformshift{4.046545in}{1.068744in}%
\pgfsys@useobject{currentmarker}{}%
\end{pgfscope}%
\begin{pgfscope}%
\pgfsys@transformshift{4.091636in}{1.053903in}%
\pgfsys@useobject{currentmarker}{}%
\end{pgfscope}%
\begin{pgfscope}%
\pgfsys@transformshift{4.136727in}{1.039350in}%
\pgfsys@useobject{currentmarker}{}%
\end{pgfscope}%
\begin{pgfscope}%
\pgfsys@transformshift{4.181818in}{1.025077in}%
\pgfsys@useobject{currentmarker}{}%
\end{pgfscope}%
\begin{pgfscope}%
\pgfsys@transformshift{4.226909in}{1.011076in}%
\pgfsys@useobject{currentmarker}{}%
\end{pgfscope}%
\begin{pgfscope}%
\pgfsys@transformshift{4.272000in}{0.997339in}%
\pgfsys@useobject{currentmarker}{}%
\end{pgfscope}%
\begin{pgfscope}%
\pgfsys@transformshift{4.317091in}{0.983860in}%
\pgfsys@useobject{currentmarker}{}%
\end{pgfscope}%
\begin{pgfscope}%
\pgfsys@transformshift{4.362182in}{0.970630in}%
\pgfsys@useobject{currentmarker}{}%
\end{pgfscope}%
\begin{pgfscope}%
\pgfsys@transformshift{4.407273in}{0.957643in}%
\pgfsys@useobject{currentmarker}{}%
\end{pgfscope}%
\begin{pgfscope}%
\pgfsys@transformshift{4.452364in}{0.944892in}%
\pgfsys@useobject{currentmarker}{}%
\end{pgfscope}%
\begin{pgfscope}%
\pgfsys@transformshift{4.497455in}{0.932371in}%
\pgfsys@useobject{currentmarker}{}%
\end{pgfscope}%
\begin{pgfscope}%
\pgfsys@transformshift{4.542545in}{0.920074in}%
\pgfsys@useobject{currentmarker}{}%
\end{pgfscope}%
\begin{pgfscope}%
\pgfsys@transformshift{4.587636in}{0.907995in}%
\pgfsys@useobject{currentmarker}{}%
\end{pgfscope}%
\begin{pgfscope}%
\pgfsys@transformshift{4.632727in}{0.896128in}%
\pgfsys@useobject{currentmarker}{}%
\end{pgfscope}%
\begin{pgfscope}%
\pgfsys@transformshift{4.677818in}{0.884467in}%
\pgfsys@useobject{currentmarker}{}%
\end{pgfscope}%
\begin{pgfscope}%
\pgfsys@transformshift{4.722909in}{0.873008in}%
\pgfsys@useobject{currentmarker}{}%
\end{pgfscope}%
\begin{pgfscope}%
\pgfsys@transformshift{4.768000in}{0.861744in}%
\pgfsys@useobject{currentmarker}{}%
\end{pgfscope}%
\begin{pgfscope}%
\pgfsys@transformshift{4.813091in}{0.850672in}%
\pgfsys@useobject{currentmarker}{}%
\end{pgfscope}%
\begin{pgfscope}%
\pgfsys@transformshift{4.858182in}{0.839786in}%
\pgfsys@useobject{currentmarker}{}%
\end{pgfscope}%
\begin{pgfscope}%
\pgfsys@transformshift{4.903273in}{0.829081in}%
\pgfsys@useobject{currentmarker}{}%
\end{pgfscope}%
\begin{pgfscope}%
\pgfsys@transformshift{4.948364in}{0.818553in}%
\pgfsys@useobject{currentmarker}{}%
\end{pgfscope}%
\begin{pgfscope}%
\pgfsys@transformshift{4.993455in}{0.808198in}%
\pgfsys@useobject{currentmarker}{}%
\end{pgfscope}%
\begin{pgfscope}%
\pgfsys@transformshift{5.038545in}{0.798012in}%
\pgfsys@useobject{currentmarker}{}%
\end{pgfscope}%
\begin{pgfscope}%
\pgfsys@transformshift{5.083636in}{0.787990in}%
\pgfsys@useobject{currentmarker}{}%
\end{pgfscope}%
\begin{pgfscope}%
\pgfsys@transformshift{5.128727in}{0.778128in}%
\pgfsys@useobject{currentmarker}{}%
\end{pgfscope}%
\begin{pgfscope}%
\pgfsys@transformshift{5.173818in}{0.768423in}%
\pgfsys@useobject{currentmarker}{}%
\end{pgfscope}%
\begin{pgfscope}%
\pgfsys@transformshift{5.218909in}{0.758871in}%
\pgfsys@useobject{currentmarker}{}%
\end{pgfscope}%
\begin{pgfscope}%
\pgfsys@transformshift{5.264000in}{0.749468in}%
\pgfsys@useobject{currentmarker}{}%
\end{pgfscope}%
\begin{pgfscope}%
\pgfsys@transformshift{5.309091in}{0.740211in}%
\pgfsys@useobject{currentmarker}{}%
\end{pgfscope}%
\begin{pgfscope}%
\pgfsys@transformshift{5.354182in}{0.731097in}%
\pgfsys@useobject{currentmarker}{}%
\end{pgfscope}%
\begin{pgfscope}%
\pgfsys@transformshift{5.399273in}{0.722121in}%
\pgfsys@useobject{currentmarker}{}%
\end{pgfscope}%
\begin{pgfscope}%
\pgfsys@transformshift{5.444364in}{0.713282in}%
\pgfsys@useobject{currentmarker}{}%
\end{pgfscope}%
\begin{pgfscope}%
\pgfsys@transformshift{5.489455in}{0.704576in}%
\pgfsys@useobject{currentmarker}{}%
\end{pgfscope}%
\begin{pgfscope}%
\pgfsys@transformshift{5.534545in}{0.696000in}%
\pgfsys@useobject{currentmarker}{}%
\end{pgfscope}%
\end{pgfscope}%
\begin{pgfscope}%
\pgfsetrectcap%
\pgfsetmiterjoin%
\pgfsetlinewidth{0.803000pt}%
\definecolor{currentstroke}{rgb}{0.000000,0.000000,0.000000}%
\pgfsetstrokecolor{currentstroke}%
\pgfsetdash{}{0pt}%
\pgfpathmoveto{\pgfqpoint{0.800000in}{0.528000in}}%
\pgfpathlineto{\pgfqpoint{0.800000in}{4.224000in}}%
\pgfusepath{stroke}%
\end{pgfscope}%
\begin{pgfscope}%
\pgfsetrectcap%
\pgfsetmiterjoin%
\pgfsetlinewidth{0.803000pt}%
\definecolor{currentstroke}{rgb}{0.000000,0.000000,0.000000}%
\pgfsetstrokecolor{currentstroke}%
\pgfsetdash{}{0pt}%
\pgfpathmoveto{\pgfqpoint{5.760000in}{0.528000in}}%
\pgfpathlineto{\pgfqpoint{5.760000in}{4.224000in}}%
\pgfusepath{stroke}%
\end{pgfscope}%
\begin{pgfscope}%
\pgfsetrectcap%
\pgfsetmiterjoin%
\pgfsetlinewidth{0.803000pt}%
\definecolor{currentstroke}{rgb}{0.000000,0.000000,0.000000}%
\pgfsetstrokecolor{currentstroke}%
\pgfsetdash{}{0pt}%
\pgfpathmoveto{\pgfqpoint{0.800000in}{0.528000in}}%
\pgfpathlineto{\pgfqpoint{5.760000in}{0.528000in}}%
\pgfusepath{stroke}%
\end{pgfscope}%
\begin{pgfscope}%
\pgfsetrectcap%
\pgfsetmiterjoin%
\pgfsetlinewidth{0.803000pt}%
\definecolor{currentstroke}{rgb}{0.000000,0.000000,0.000000}%
\pgfsetstrokecolor{currentstroke}%
\pgfsetdash{}{0pt}%
\pgfpathmoveto{\pgfqpoint{0.800000in}{4.224000in}}%
\pgfpathlineto{\pgfqpoint{5.760000in}{4.224000in}}%
\pgfusepath{stroke}%
\end{pgfscope}%
\end{pgfpicture}%
\makeatother%
\endgroup%

%% \caption{The lower bound in \cref{eq:GMRESprob} with $\NLiDR{\no-\nt} \sim \Exp{\sigma}$ with $\sigma = 1.$\label{fig:prob-theory-plot-0.0}}
%% \end{subfigure}

%% \begin{subfigure}{\textwidth}
%%     \centering
%% %% Creator: Matplotlib, PGF backend
%%
%% To include the figure in your LaTeX document, write
%%   \input{<filename>.pgf}
%%
%% Make sure the required packages are loaded in your preamble
%%   \usepackage{pgf}
%%
%% Figures using additional raster images can only be included by \input if
%% they are in the same directory as the main LaTeX file. For loading figures
%% from other directories you can use the `import` package
%%   \usepackage{import}
%% and then include the figures with
%%   \import{<path to file>}{<filename>.pgf}
%%
%% Matplotlib used the following preamble
%%   \usepackage{fontspec}
%%   \setmainfont{DejaVuSerif.ttf}[Path=/home/owen/progs/firedrake-complex/firedrake/lib/python3.5/site-packages/matplotlib/mpl-data/fonts/ttf/]
%%   \setsansfont{DejaVuSans.ttf}[Path=/home/owen/progs/firedrake-complex/firedrake/lib/python3.5/site-packages/matplotlib/mpl-data/fonts/ttf/]
%%   \setmonofont{DejaVuSansMono.ttf}[Path=/home/owen/progs/firedrake-complex/firedrake/lib/python3.5/site-packages/matplotlib/mpl-data/fonts/ttf/]
%%
\begingroup%
\makeatletter%
\begin{pgfpicture}%
\pgfpathrectangle{\pgfpointorigin}{\pgfqpoint{6.000000in}{2.500000in}}%
\pgfusepath{use as bounding box, clip}%
\begin{pgfscope}%
\pgfsetbuttcap%
\pgfsetmiterjoin%
\definecolor{currentfill}{rgb}{1.000000,1.000000,1.000000}%
\pgfsetfillcolor{currentfill}%
\pgfsetlinewidth{0.000000pt}%
\definecolor{currentstroke}{rgb}{1.000000,1.000000,1.000000}%
\pgfsetstrokecolor{currentstroke}%
\pgfsetdash{}{0pt}%
\pgfpathmoveto{\pgfqpoint{0.000000in}{0.000000in}}%
\pgfpathlineto{\pgfqpoint{6.000000in}{0.000000in}}%
\pgfpathlineto{\pgfqpoint{6.000000in}{2.500000in}}%
\pgfpathlineto{\pgfqpoint{0.000000in}{2.500000in}}%
\pgfpathclose%
\pgfusepath{fill}%
\end{pgfscope}%
\begin{pgfscope}%
\pgfsetbuttcap%
\pgfsetmiterjoin%
\definecolor{currentfill}{rgb}{1.000000,1.000000,1.000000}%
\pgfsetfillcolor{currentfill}%
\pgfsetlinewidth{0.000000pt}%
\definecolor{currentstroke}{rgb}{0.000000,0.000000,0.000000}%
\pgfsetstrokecolor{currentstroke}%
\pgfsetstrokeopacity{0.000000}%
\pgfsetdash{}{0pt}%
\pgfpathmoveto{\pgfqpoint{0.750000in}{0.275000in}}%
\pgfpathlineto{\pgfqpoint{5.400000in}{0.275000in}}%
\pgfpathlineto{\pgfqpoint{5.400000in}{2.200000in}}%
\pgfpathlineto{\pgfqpoint{0.750000in}{2.200000in}}%
\pgfpathclose%
\pgfusepath{fill}%
\end{pgfscope}%
\begin{pgfscope}%
\pgfsetbuttcap%
\pgfsetroundjoin%
\definecolor{currentfill}{rgb}{0.000000,0.000000,0.000000}%
\pgfsetfillcolor{currentfill}%
\pgfsetlinewidth{0.803000pt}%
\definecolor{currentstroke}{rgb}{0.000000,0.000000,0.000000}%
\pgfsetstrokecolor{currentstroke}%
\pgfsetdash{}{0pt}%
\pgfsys@defobject{currentmarker}{\pgfqpoint{0.000000in}{-0.048611in}}{\pgfqpoint{0.000000in}{0.000000in}}{%
\pgfpathmoveto{\pgfqpoint{0.000000in}{0.000000in}}%
\pgfpathlineto{\pgfqpoint{0.000000in}{-0.048611in}}%
\pgfusepath{stroke,fill}%
}%
\begin{pgfscope}%
\pgfsys@transformshift{0.961364in}{0.275000in}%
\pgfsys@useobject{currentmarker}{}%
\end{pgfscope}%
\end{pgfscope}%
\begin{pgfscope}%
\definecolor{textcolor}{rgb}{0.000000,0.000000,0.000000}%
\pgfsetstrokecolor{textcolor}%
\pgfsetfillcolor{textcolor}%
\pgftext[x=0.961364in,y=0.177778in,,top]{\color{textcolor}\sffamily\fontsize{10.000000}{12.000000}\selectfont 10}%
\end{pgfscope}%
\begin{pgfscope}%
\pgfsetbuttcap%
\pgfsetroundjoin%
\definecolor{currentfill}{rgb}{0.000000,0.000000,0.000000}%
\pgfsetfillcolor{currentfill}%
\pgfsetlinewidth{0.803000pt}%
\definecolor{currentstroke}{rgb}{0.000000,0.000000,0.000000}%
\pgfsetstrokecolor{currentstroke}%
\pgfsetdash{}{0pt}%
\pgfsys@defobject{currentmarker}{\pgfqpoint{0.000000in}{-0.048611in}}{\pgfqpoint{0.000000in}{0.000000in}}{%
\pgfpathmoveto{\pgfqpoint{0.000000in}{0.000000in}}%
\pgfpathlineto{\pgfqpoint{0.000000in}{-0.048611in}}%
\pgfusepath{stroke,fill}%
}%
\begin{pgfscope}%
\pgfsys@transformshift{1.665909in}{0.275000in}%
\pgfsys@useobject{currentmarker}{}%
\end{pgfscope}%
\end{pgfscope}%
\begin{pgfscope}%
\definecolor{textcolor}{rgb}{0.000000,0.000000,0.000000}%
\pgfsetstrokecolor{textcolor}%
\pgfsetfillcolor{textcolor}%
\pgftext[x=1.665909in,y=0.177778in,,top]{\color{textcolor}\sffamily\fontsize{10.000000}{12.000000}\selectfont 15}%
\end{pgfscope}%
\begin{pgfscope}%
\pgfsetbuttcap%
\pgfsetroundjoin%
\definecolor{currentfill}{rgb}{0.000000,0.000000,0.000000}%
\pgfsetfillcolor{currentfill}%
\pgfsetlinewidth{0.803000pt}%
\definecolor{currentstroke}{rgb}{0.000000,0.000000,0.000000}%
\pgfsetstrokecolor{currentstroke}%
\pgfsetdash{}{0pt}%
\pgfsys@defobject{currentmarker}{\pgfqpoint{0.000000in}{-0.048611in}}{\pgfqpoint{0.000000in}{0.000000in}}{%
\pgfpathmoveto{\pgfqpoint{0.000000in}{0.000000in}}%
\pgfpathlineto{\pgfqpoint{0.000000in}{-0.048611in}}%
\pgfusepath{stroke,fill}%
}%
\begin{pgfscope}%
\pgfsys@transformshift{2.370455in}{0.275000in}%
\pgfsys@useobject{currentmarker}{}%
\end{pgfscope}%
\end{pgfscope}%
\begin{pgfscope}%
\definecolor{textcolor}{rgb}{0.000000,0.000000,0.000000}%
\pgfsetstrokecolor{textcolor}%
\pgfsetfillcolor{textcolor}%
\pgftext[x=2.370455in,y=0.177778in,,top]{\color{textcolor}\sffamily\fontsize{10.000000}{12.000000}\selectfont 20}%
\end{pgfscope}%
\begin{pgfscope}%
\pgfsetbuttcap%
\pgfsetroundjoin%
\definecolor{currentfill}{rgb}{0.000000,0.000000,0.000000}%
\pgfsetfillcolor{currentfill}%
\pgfsetlinewidth{0.803000pt}%
\definecolor{currentstroke}{rgb}{0.000000,0.000000,0.000000}%
\pgfsetstrokecolor{currentstroke}%
\pgfsetdash{}{0pt}%
\pgfsys@defobject{currentmarker}{\pgfqpoint{0.000000in}{-0.048611in}}{\pgfqpoint{0.000000in}{0.000000in}}{%
\pgfpathmoveto{\pgfqpoint{0.000000in}{0.000000in}}%
\pgfpathlineto{\pgfqpoint{0.000000in}{-0.048611in}}%
\pgfusepath{stroke,fill}%
}%
\begin{pgfscope}%
\pgfsys@transformshift{3.075000in}{0.275000in}%
\pgfsys@useobject{currentmarker}{}%
\end{pgfscope}%
\end{pgfscope}%
\begin{pgfscope}%
\definecolor{textcolor}{rgb}{0.000000,0.000000,0.000000}%
\pgfsetstrokecolor{textcolor}%
\pgfsetfillcolor{textcolor}%
\pgftext[x=3.075000in,y=0.177778in,,top]{\color{textcolor}\sffamily\fontsize{10.000000}{12.000000}\selectfont 25}%
\end{pgfscope}%
\begin{pgfscope}%
\pgfsetbuttcap%
\pgfsetroundjoin%
\definecolor{currentfill}{rgb}{0.000000,0.000000,0.000000}%
\pgfsetfillcolor{currentfill}%
\pgfsetlinewidth{0.803000pt}%
\definecolor{currentstroke}{rgb}{0.000000,0.000000,0.000000}%
\pgfsetstrokecolor{currentstroke}%
\pgfsetdash{}{0pt}%
\pgfsys@defobject{currentmarker}{\pgfqpoint{0.000000in}{-0.048611in}}{\pgfqpoint{0.000000in}{0.000000in}}{%
\pgfpathmoveto{\pgfqpoint{0.000000in}{0.000000in}}%
\pgfpathlineto{\pgfqpoint{0.000000in}{-0.048611in}}%
\pgfusepath{stroke,fill}%
}%
\begin{pgfscope}%
\pgfsys@transformshift{3.779545in}{0.275000in}%
\pgfsys@useobject{currentmarker}{}%
\end{pgfscope}%
\end{pgfscope}%
\begin{pgfscope}%
\definecolor{textcolor}{rgb}{0.000000,0.000000,0.000000}%
\pgfsetstrokecolor{textcolor}%
\pgfsetfillcolor{textcolor}%
\pgftext[x=3.779545in,y=0.177778in,,top]{\color{textcolor}\sffamily\fontsize{10.000000}{12.000000}\selectfont 30}%
\end{pgfscope}%
\begin{pgfscope}%
\pgfsetbuttcap%
\pgfsetroundjoin%
\definecolor{currentfill}{rgb}{0.000000,0.000000,0.000000}%
\pgfsetfillcolor{currentfill}%
\pgfsetlinewidth{0.803000pt}%
\definecolor{currentstroke}{rgb}{0.000000,0.000000,0.000000}%
\pgfsetstrokecolor{currentstroke}%
\pgfsetdash{}{0pt}%
\pgfsys@defobject{currentmarker}{\pgfqpoint{0.000000in}{-0.048611in}}{\pgfqpoint{0.000000in}{0.000000in}}{%
\pgfpathmoveto{\pgfqpoint{0.000000in}{0.000000in}}%
\pgfpathlineto{\pgfqpoint{0.000000in}{-0.048611in}}%
\pgfusepath{stroke,fill}%
}%
\begin{pgfscope}%
\pgfsys@transformshift{4.484091in}{0.275000in}%
\pgfsys@useobject{currentmarker}{}%
\end{pgfscope}%
\end{pgfscope}%
\begin{pgfscope}%
\definecolor{textcolor}{rgb}{0.000000,0.000000,0.000000}%
\pgfsetstrokecolor{textcolor}%
\pgfsetfillcolor{textcolor}%
\pgftext[x=4.484091in,y=0.177778in,,top]{\color{textcolor}\sffamily\fontsize{10.000000}{12.000000}\selectfont 35}%
\end{pgfscope}%
\begin{pgfscope}%
\pgfsetbuttcap%
\pgfsetroundjoin%
\definecolor{currentfill}{rgb}{0.000000,0.000000,0.000000}%
\pgfsetfillcolor{currentfill}%
\pgfsetlinewidth{0.803000pt}%
\definecolor{currentstroke}{rgb}{0.000000,0.000000,0.000000}%
\pgfsetstrokecolor{currentstroke}%
\pgfsetdash{}{0pt}%
\pgfsys@defobject{currentmarker}{\pgfqpoint{0.000000in}{-0.048611in}}{\pgfqpoint{0.000000in}{0.000000in}}{%
\pgfpathmoveto{\pgfqpoint{0.000000in}{0.000000in}}%
\pgfpathlineto{\pgfqpoint{0.000000in}{-0.048611in}}%
\pgfusepath{stroke,fill}%
}%
\begin{pgfscope}%
\pgfsys@transformshift{5.188636in}{0.275000in}%
\pgfsys@useobject{currentmarker}{}%
\end{pgfscope}%
\end{pgfscope}%
\begin{pgfscope}%
\definecolor{textcolor}{rgb}{0.000000,0.000000,0.000000}%
\pgfsetstrokecolor{textcolor}%
\pgfsetfillcolor{textcolor}%
\pgftext[x=5.188636in,y=0.177778in,,top]{\color{textcolor}\sffamily\fontsize{10.000000}{12.000000}\selectfont 40}%
\end{pgfscope}%
\begin{pgfscope}%
\definecolor{textcolor}{rgb}{0.000000,0.000000,0.000000}%
\pgfsetstrokecolor{textcolor}%
\pgfsetfillcolor{textcolor}%
\pgftext[x=3.075000in,y=-0.012191in,,top]{\color{textcolor}\sffamily\fontsize{10.000000}{12.000000}\selectfont \(\displaystyle k\)}%
\end{pgfscope}%
\begin{pgfscope}%
\pgfsetbuttcap%
\pgfsetroundjoin%
\definecolor{currentfill}{rgb}{0.000000,0.000000,0.000000}%
\pgfsetfillcolor{currentfill}%
\pgfsetlinewidth{0.803000pt}%
\definecolor{currentstroke}{rgb}{0.000000,0.000000,0.000000}%
\pgfsetstrokecolor{currentstroke}%
\pgfsetdash{}{0pt}%
\pgfsys@defobject{currentmarker}{\pgfqpoint{-0.048611in}{0.000000in}}{\pgfqpoint{0.000000in}{0.000000in}}{%
\pgfpathmoveto{\pgfqpoint{0.000000in}{0.000000in}}%
\pgfpathlineto{\pgfqpoint{-0.048611in}{0.000000in}}%
\pgfusepath{stroke,fill}%
}%
\begin{pgfscope}%
\pgfsys@transformshift{0.750000in}{0.552046in}%
\pgfsys@useobject{currentmarker}{}%
\end{pgfscope}%
\end{pgfscope}%
\begin{pgfscope}%
\definecolor{textcolor}{rgb}{0.000000,0.000000,0.000000}%
\pgfsetstrokecolor{textcolor}%
\pgfsetfillcolor{textcolor}%
\pgftext[x=0.431898in,y=0.499284in,left,base]{\color{textcolor}\sffamily\fontsize{10.000000}{12.000000}\selectfont 6.5}%
\end{pgfscope}%
\begin{pgfscope}%
\pgfsetbuttcap%
\pgfsetroundjoin%
\definecolor{currentfill}{rgb}{0.000000,0.000000,0.000000}%
\pgfsetfillcolor{currentfill}%
\pgfsetlinewidth{0.803000pt}%
\definecolor{currentstroke}{rgb}{0.000000,0.000000,0.000000}%
\pgfsetstrokecolor{currentstroke}%
\pgfsetdash{}{0pt}%
\pgfsys@defobject{currentmarker}{\pgfqpoint{-0.048611in}{0.000000in}}{\pgfqpoint{0.000000in}{0.000000in}}{%
\pgfpathmoveto{\pgfqpoint{0.000000in}{0.000000in}}%
\pgfpathlineto{\pgfqpoint{-0.048611in}{0.000000in}}%
\pgfusepath{stroke,fill}%
}%
\begin{pgfscope}%
\pgfsys@transformshift{0.750000in}{1.066191in}%
\pgfsys@useobject{currentmarker}{}%
\end{pgfscope}%
\end{pgfscope}%
\begin{pgfscope}%
\definecolor{textcolor}{rgb}{0.000000,0.000000,0.000000}%
\pgfsetstrokecolor{textcolor}%
\pgfsetfillcolor{textcolor}%
\pgftext[x=0.431898in,y=1.013430in,left,base]{\color{textcolor}\sffamily\fontsize{10.000000}{12.000000}\selectfont 7.0}%
\end{pgfscope}%
\begin{pgfscope}%
\pgfsetbuttcap%
\pgfsetroundjoin%
\definecolor{currentfill}{rgb}{0.000000,0.000000,0.000000}%
\pgfsetfillcolor{currentfill}%
\pgfsetlinewidth{0.803000pt}%
\definecolor{currentstroke}{rgb}{0.000000,0.000000,0.000000}%
\pgfsetstrokecolor{currentstroke}%
\pgfsetdash{}{0pt}%
\pgfsys@defobject{currentmarker}{\pgfqpoint{-0.048611in}{0.000000in}}{\pgfqpoint{0.000000in}{0.000000in}}{%
\pgfpathmoveto{\pgfqpoint{0.000000in}{0.000000in}}%
\pgfpathlineto{\pgfqpoint{-0.048611in}{0.000000in}}%
\pgfusepath{stroke,fill}%
}%
\begin{pgfscope}%
\pgfsys@transformshift{0.750000in}{1.580342in}%
\pgfsys@useobject{currentmarker}{}%
\end{pgfscope}%
\end{pgfscope}%
\begin{pgfscope}%
\definecolor{textcolor}{rgb}{0.000000,0.000000,0.000000}%
\pgfsetstrokecolor{textcolor}%
\pgfsetfillcolor{textcolor}%
\pgftext[x=0.431898in,y=1.527580in,left,base]{\color{textcolor}\sffamily\fontsize{10.000000}{12.000000}\selectfont 7.5}%
\end{pgfscope}%
\begin{pgfscope}%
\pgfsetbuttcap%
\pgfsetroundjoin%
\definecolor{currentfill}{rgb}{0.000000,0.000000,0.000000}%
\pgfsetfillcolor{currentfill}%
\pgfsetlinewidth{0.803000pt}%
\definecolor{currentstroke}{rgb}{0.000000,0.000000,0.000000}%
\pgfsetstrokecolor{currentstroke}%
\pgfsetdash{}{0pt}%
\pgfsys@defobject{currentmarker}{\pgfqpoint{-0.048611in}{0.000000in}}{\pgfqpoint{0.000000in}{0.000000in}}{%
\pgfpathmoveto{\pgfqpoint{0.000000in}{0.000000in}}%
\pgfpathlineto{\pgfqpoint{-0.048611in}{0.000000in}}%
\pgfusepath{stroke,fill}%
}%
\begin{pgfscope}%
\pgfsys@transformshift{0.750000in}{2.094492in}%
\pgfsys@useobject{currentmarker}{}%
\end{pgfscope}%
\end{pgfscope}%
\begin{pgfscope}%
\definecolor{textcolor}{rgb}{0.000000,0.000000,0.000000}%
\pgfsetstrokecolor{textcolor}%
\pgfsetfillcolor{textcolor}%
\pgftext[x=0.431898in,y=2.041731in,left,base]{\color{textcolor}\sffamily\fontsize{10.000000}{12.000000}\selectfont 8.0}%
\end{pgfscope}%
\begin{pgfscope}%
\definecolor{textcolor}{rgb}{0.000000,0.000000,0.000000}%
\pgfsetstrokecolor{textcolor}%
\pgfsetfillcolor{textcolor}%
\pgftext[x=0.191936in,y=0.325194in,left,base,rotate=90.000000]{\color{textcolor}\sffamily\fontsize{10.000000}{12.000000}\selectfont Probability that number of}%
\end{pgfscope}%
\begin{pgfscope}%
\definecolor{textcolor}{rgb}{0.000000,0.000000,0.000000}%
\pgfsetstrokecolor{textcolor}%
\pgfsetfillcolor{textcolor}%
\pgftext[x=0.347453in,y=0.160467in,left,base,rotate=90.000000]{\color{textcolor}\sffamily\fontsize{10.000000}{12.000000}\selectfont GMRES iterations is at most 12}%
\end{pgfscope}%
\begin{pgfscope}%
\definecolor{textcolor}{rgb}{0.000000,0.000000,0.000000}%
\pgfsetstrokecolor{textcolor}%
\pgfsetfillcolor{textcolor}%
\pgftext[x=0.750000in,y=2.241667in,left,base]{\color{textcolor}\sffamily\fontsize{10.000000}{12.000000}\selectfont 1e−11+2.983096487e−1}%
\end{pgfscope}%
\begin{pgfscope}%
\pgfpathrectangle{\pgfqpoint{0.750000in}{0.275000in}}{\pgfqpoint{4.650000in}{1.925000in}}%
\pgfusepath{clip}%
\pgfsetbuttcap%
\pgfsetroundjoin%
\definecolor{currentfill}{rgb}{0.000000,0.000000,0.000000}%
\pgfsetfillcolor{currentfill}%
\pgfsetlinewidth{1.003750pt}%
\definecolor{currentstroke}{rgb}{0.000000,0.000000,0.000000}%
\pgfsetstrokecolor{currentstroke}%
\pgfsetdash{}{0pt}%
\pgfsys@defobject{currentmarker}{\pgfqpoint{-0.020833in}{-0.020833in}}{\pgfqpoint{0.020833in}{0.020833in}}{%
\pgfpathmoveto{\pgfqpoint{0.000000in}{-0.020833in}}%
\pgfpathcurveto{\pgfqpoint{0.005525in}{-0.020833in}}{\pgfqpoint{0.010825in}{-0.018638in}}{\pgfqpoint{0.014731in}{-0.014731in}}%
\pgfpathcurveto{\pgfqpoint{0.018638in}{-0.010825in}}{\pgfqpoint{0.020833in}{-0.005525in}}{\pgfqpoint{0.020833in}{0.000000in}}%
\pgfpathcurveto{\pgfqpoint{0.020833in}{0.005525in}}{\pgfqpoint{0.018638in}{0.010825in}}{\pgfqpoint{0.014731in}{0.014731in}}%
\pgfpathcurveto{\pgfqpoint{0.010825in}{0.018638in}}{\pgfqpoint{0.005525in}{0.020833in}}{\pgfqpoint{0.000000in}{0.020833in}}%
\pgfpathcurveto{\pgfqpoint{-0.005525in}{0.020833in}}{\pgfqpoint{-0.010825in}{0.018638in}}{\pgfqpoint{-0.014731in}{0.014731in}}%
\pgfpathcurveto{\pgfqpoint{-0.018638in}{0.010825in}}{\pgfqpoint{-0.020833in}{0.005525in}}{\pgfqpoint{-0.020833in}{0.000000in}}%
\pgfpathcurveto{\pgfqpoint{-0.020833in}{-0.005525in}}{\pgfqpoint{-0.018638in}{-0.010825in}}{\pgfqpoint{-0.014731in}{-0.014731in}}%
\pgfpathcurveto{\pgfqpoint{-0.010825in}{-0.018638in}}{\pgfqpoint{-0.005525in}{-0.020833in}}{\pgfqpoint{0.000000in}{-0.020833in}}%
\pgfpathclose%
\pgfusepath{stroke,fill}%
}%
\begin{pgfscope}%
\pgfsys@transformshift{0.961364in}{1.237490in}%
\pgfsys@useobject{currentmarker}{}%
\end{pgfscope}%
\begin{pgfscope}%
\pgfsys@transformshift{1.003636in}{1.237485in}%
\pgfsys@useobject{currentmarker}{}%
\end{pgfscope}%
\begin{pgfscope}%
\pgfsys@transformshift{1.045909in}{1.237490in}%
\pgfsys@useobject{currentmarker}{}%
\end{pgfscope}%
\begin{pgfscope}%
\pgfsys@transformshift{1.088182in}{1.237495in}%
\pgfsys@useobject{currentmarker}{}%
\end{pgfscope}%
\begin{pgfscope}%
\pgfsys@transformshift{1.130455in}{1.237485in}%
\pgfsys@useobject{currentmarker}{}%
\end{pgfscope}%
\begin{pgfscope}%
\pgfsys@transformshift{1.172727in}{1.237480in}%
\pgfsys@useobject{currentmarker}{}%
\end{pgfscope}%
\begin{pgfscope}%
\pgfsys@transformshift{1.215000in}{1.237490in}%
\pgfsys@useobject{currentmarker}{}%
\end{pgfscope}%
\begin{pgfscope}%
\pgfsys@transformshift{1.257273in}{1.237485in}%
\pgfsys@useobject{currentmarker}{}%
\end{pgfscope}%
\begin{pgfscope}%
\pgfsys@transformshift{1.299545in}{2.112480in}%
\pgfsys@useobject{currentmarker}{}%
\end{pgfscope}%
\begin{pgfscope}%
\pgfsys@transformshift{1.341818in}{2.112480in}%
\pgfsys@useobject{currentmarker}{}%
\end{pgfscope}%
\begin{pgfscope}%
\pgfsys@transformshift{1.384091in}{2.112495in}%
\pgfsys@useobject{currentmarker}{}%
\end{pgfscope}%
\begin{pgfscope}%
\pgfsys@transformshift{1.426364in}{2.112490in}%
\pgfsys@useobject{currentmarker}{}%
\end{pgfscope}%
\begin{pgfscope}%
\pgfsys@transformshift{1.468636in}{2.112490in}%
\pgfsys@useobject{currentmarker}{}%
\end{pgfscope}%
\begin{pgfscope}%
\pgfsys@transformshift{1.510909in}{2.112485in}%
\pgfsys@useobject{currentmarker}{}%
\end{pgfscope}%
\begin{pgfscope}%
\pgfsys@transformshift{1.553182in}{2.112485in}%
\pgfsys@useobject{currentmarker}{}%
\end{pgfscope}%
\begin{pgfscope}%
\pgfsys@transformshift{1.595455in}{2.112495in}%
\pgfsys@useobject{currentmarker}{}%
\end{pgfscope}%
\begin{pgfscope}%
\pgfsys@transformshift{1.637727in}{2.112495in}%
\pgfsys@useobject{currentmarker}{}%
\end{pgfscope}%
\begin{pgfscope}%
\pgfsys@transformshift{1.680000in}{2.112480in}%
\pgfsys@useobject{currentmarker}{}%
\end{pgfscope}%
\begin{pgfscope}%
\pgfsys@transformshift{1.722273in}{2.112490in}%
\pgfsys@useobject{currentmarker}{}%
\end{pgfscope}%
\begin{pgfscope}%
\pgfsys@transformshift{1.764545in}{2.112485in}%
\pgfsys@useobject{currentmarker}{}%
\end{pgfscope}%
\begin{pgfscope}%
\pgfsys@transformshift{1.806818in}{2.112490in}%
\pgfsys@useobject{currentmarker}{}%
\end{pgfscope}%
\begin{pgfscope}%
\pgfsys@transformshift{1.849091in}{2.112480in}%
\pgfsys@useobject{currentmarker}{}%
\end{pgfscope}%
\begin{pgfscope}%
\pgfsys@transformshift{1.891364in}{2.112485in}%
\pgfsys@useobject{currentmarker}{}%
\end{pgfscope}%
\begin{pgfscope}%
\pgfsys@transformshift{1.933636in}{2.112485in}%
\pgfsys@useobject{currentmarker}{}%
\end{pgfscope}%
\begin{pgfscope}%
\pgfsys@transformshift{1.975909in}{2.112490in}%
\pgfsys@useobject{currentmarker}{}%
\end{pgfscope}%
\begin{pgfscope}%
\pgfsys@transformshift{2.018182in}{2.112485in}%
\pgfsys@useobject{currentmarker}{}%
\end{pgfscope}%
\begin{pgfscope}%
\pgfsys@transformshift{2.060455in}{2.112485in}%
\pgfsys@useobject{currentmarker}{}%
\end{pgfscope}%
\begin{pgfscope}%
\pgfsys@transformshift{2.102727in}{2.112480in}%
\pgfsys@useobject{currentmarker}{}%
\end{pgfscope}%
\begin{pgfscope}%
\pgfsys@transformshift{2.145000in}{2.112490in}%
\pgfsys@useobject{currentmarker}{}%
\end{pgfscope}%
\begin{pgfscope}%
\pgfsys@transformshift{2.187273in}{2.112495in}%
\pgfsys@useobject{currentmarker}{}%
\end{pgfscope}%
\begin{pgfscope}%
\pgfsys@transformshift{2.229545in}{2.112480in}%
\pgfsys@useobject{currentmarker}{}%
\end{pgfscope}%
\begin{pgfscope}%
\pgfsys@transformshift{2.271818in}{2.112485in}%
\pgfsys@useobject{currentmarker}{}%
\end{pgfscope}%
\begin{pgfscope}%
\pgfsys@transformshift{2.314091in}{2.112471in}%
\pgfsys@useobject{currentmarker}{}%
\end{pgfscope}%
\begin{pgfscope}%
\pgfsys@transformshift{2.356364in}{2.112485in}%
\pgfsys@useobject{currentmarker}{}%
\end{pgfscope}%
\begin{pgfscope}%
\pgfsys@transformshift{2.398636in}{2.112495in}%
\pgfsys@useobject{currentmarker}{}%
\end{pgfscope}%
\begin{pgfscope}%
\pgfsys@transformshift{2.440909in}{2.112490in}%
\pgfsys@useobject{currentmarker}{}%
\end{pgfscope}%
\begin{pgfscope}%
\pgfsys@transformshift{2.483182in}{2.112485in}%
\pgfsys@useobject{currentmarker}{}%
\end{pgfscope}%
\begin{pgfscope}%
\pgfsys@transformshift{2.525455in}{2.112485in}%
\pgfsys@useobject{currentmarker}{}%
\end{pgfscope}%
\begin{pgfscope}%
\pgfsys@transformshift{2.567727in}{2.112471in}%
\pgfsys@useobject{currentmarker}{}%
\end{pgfscope}%
\begin{pgfscope}%
\pgfsys@transformshift{2.610000in}{2.112490in}%
\pgfsys@useobject{currentmarker}{}%
\end{pgfscope}%
\begin{pgfscope}%
\pgfsys@transformshift{2.652273in}{2.112480in}%
\pgfsys@useobject{currentmarker}{}%
\end{pgfscope}%
\begin{pgfscope}%
\pgfsys@transformshift{2.694545in}{2.112485in}%
\pgfsys@useobject{currentmarker}{}%
\end{pgfscope}%
\begin{pgfscope}%
\pgfsys@transformshift{2.736818in}{2.112485in}%
\pgfsys@useobject{currentmarker}{}%
\end{pgfscope}%
\begin{pgfscope}%
\pgfsys@transformshift{2.779091in}{2.112495in}%
\pgfsys@useobject{currentmarker}{}%
\end{pgfscope}%
\begin{pgfscope}%
\pgfsys@transformshift{2.821364in}{2.112490in}%
\pgfsys@useobject{currentmarker}{}%
\end{pgfscope}%
\begin{pgfscope}%
\pgfsys@transformshift{2.863636in}{2.112490in}%
\pgfsys@useobject{currentmarker}{}%
\end{pgfscope}%
\begin{pgfscope}%
\pgfsys@transformshift{2.905909in}{2.112490in}%
\pgfsys@useobject{currentmarker}{}%
\end{pgfscope}%
\begin{pgfscope}%
\pgfsys@transformshift{2.948182in}{2.112490in}%
\pgfsys@useobject{currentmarker}{}%
\end{pgfscope}%
\begin{pgfscope}%
\pgfsys@transformshift{2.990455in}{0.362505in}%
\pgfsys@useobject{currentmarker}{}%
\end{pgfscope}%
\begin{pgfscope}%
\pgfsys@transformshift{3.032727in}{0.362510in}%
\pgfsys@useobject{currentmarker}{}%
\end{pgfscope}%
\begin{pgfscope}%
\pgfsys@transformshift{3.075000in}{0.362505in}%
\pgfsys@useobject{currentmarker}{}%
\end{pgfscope}%
\begin{pgfscope}%
\pgfsys@transformshift{3.117273in}{0.362510in}%
\pgfsys@useobject{currentmarker}{}%
\end{pgfscope}%
\begin{pgfscope}%
\pgfsys@transformshift{3.159545in}{0.362510in}%
\pgfsys@useobject{currentmarker}{}%
\end{pgfscope}%
\begin{pgfscope}%
\pgfsys@transformshift{3.201818in}{0.362510in}%
\pgfsys@useobject{currentmarker}{}%
\end{pgfscope}%
\begin{pgfscope}%
\pgfsys@transformshift{3.244091in}{0.362505in}%
\pgfsys@useobject{currentmarker}{}%
\end{pgfscope}%
\begin{pgfscope}%
\pgfsys@transformshift{3.286364in}{0.362505in}%
\pgfsys@useobject{currentmarker}{}%
\end{pgfscope}%
\begin{pgfscope}%
\pgfsys@transformshift{3.328636in}{0.362510in}%
\pgfsys@useobject{currentmarker}{}%
\end{pgfscope}%
\begin{pgfscope}%
\pgfsys@transformshift{3.370909in}{0.362510in}%
\pgfsys@useobject{currentmarker}{}%
\end{pgfscope}%
\begin{pgfscope}%
\pgfsys@transformshift{3.413182in}{0.362510in}%
\pgfsys@useobject{currentmarker}{}%
\end{pgfscope}%
\begin{pgfscope}%
\pgfsys@transformshift{3.455455in}{0.362510in}%
\pgfsys@useobject{currentmarker}{}%
\end{pgfscope}%
\begin{pgfscope}%
\pgfsys@transformshift{3.497727in}{0.362505in}%
\pgfsys@useobject{currentmarker}{}%
\end{pgfscope}%
\begin{pgfscope}%
\pgfsys@transformshift{3.540000in}{0.362505in}%
\pgfsys@useobject{currentmarker}{}%
\end{pgfscope}%
\begin{pgfscope}%
\pgfsys@transformshift{3.582273in}{0.362505in}%
\pgfsys@useobject{currentmarker}{}%
\end{pgfscope}%
\begin{pgfscope}%
\pgfsys@transformshift{3.624545in}{0.362510in}%
\pgfsys@useobject{currentmarker}{}%
\end{pgfscope}%
\begin{pgfscope}%
\pgfsys@transformshift{3.666818in}{0.362510in}%
\pgfsys@useobject{currentmarker}{}%
\end{pgfscope}%
\begin{pgfscope}%
\pgfsys@transformshift{3.709091in}{0.362505in}%
\pgfsys@useobject{currentmarker}{}%
\end{pgfscope}%
\begin{pgfscope}%
\pgfsys@transformshift{3.751364in}{0.362510in}%
\pgfsys@useobject{currentmarker}{}%
\end{pgfscope}%
\begin{pgfscope}%
\pgfsys@transformshift{3.793636in}{0.362510in}%
\pgfsys@useobject{currentmarker}{}%
\end{pgfscope}%
\begin{pgfscope}%
\pgfsys@transformshift{3.835909in}{0.362510in}%
\pgfsys@useobject{currentmarker}{}%
\end{pgfscope}%
\begin{pgfscope}%
\pgfsys@transformshift{3.878182in}{0.362510in}%
\pgfsys@useobject{currentmarker}{}%
\end{pgfscope}%
\begin{pgfscope}%
\pgfsys@transformshift{3.920455in}{0.362510in}%
\pgfsys@useobject{currentmarker}{}%
\end{pgfscope}%
\begin{pgfscope}%
\pgfsys@transformshift{3.962727in}{0.362510in}%
\pgfsys@useobject{currentmarker}{}%
\end{pgfscope}%
\begin{pgfscope}%
\pgfsys@transformshift{4.005000in}{0.362505in}%
\pgfsys@useobject{currentmarker}{}%
\end{pgfscope}%
\begin{pgfscope}%
\pgfsys@transformshift{4.047273in}{0.362510in}%
\pgfsys@useobject{currentmarker}{}%
\end{pgfscope}%
\begin{pgfscope}%
\pgfsys@transformshift{4.089545in}{0.362505in}%
\pgfsys@useobject{currentmarker}{}%
\end{pgfscope}%
\begin{pgfscope}%
\pgfsys@transformshift{4.131818in}{0.362510in}%
\pgfsys@useobject{currentmarker}{}%
\end{pgfscope}%
\begin{pgfscope}%
\pgfsys@transformshift{4.174091in}{0.362510in}%
\pgfsys@useobject{currentmarker}{}%
\end{pgfscope}%
\begin{pgfscope}%
\pgfsys@transformshift{4.216364in}{0.362510in}%
\pgfsys@useobject{currentmarker}{}%
\end{pgfscope}%
\begin{pgfscope}%
\pgfsys@transformshift{4.258636in}{0.362510in}%
\pgfsys@useobject{currentmarker}{}%
\end{pgfscope}%
\begin{pgfscope}%
\pgfsys@transformshift{4.300909in}{0.362505in}%
\pgfsys@useobject{currentmarker}{}%
\end{pgfscope}%
\begin{pgfscope}%
\pgfsys@transformshift{4.343182in}{0.362510in}%
\pgfsys@useobject{currentmarker}{}%
\end{pgfscope}%
\begin{pgfscope}%
\pgfsys@transformshift{4.385455in}{0.362505in}%
\pgfsys@useobject{currentmarker}{}%
\end{pgfscope}%
\begin{pgfscope}%
\pgfsys@transformshift{4.427727in}{0.362505in}%
\pgfsys@useobject{currentmarker}{}%
\end{pgfscope}%
\begin{pgfscope}%
\pgfsys@transformshift{4.470000in}{0.362505in}%
\pgfsys@useobject{currentmarker}{}%
\end{pgfscope}%
\begin{pgfscope}%
\pgfsys@transformshift{4.512273in}{0.362505in}%
\pgfsys@useobject{currentmarker}{}%
\end{pgfscope}%
\begin{pgfscope}%
\pgfsys@transformshift{4.554545in}{0.362505in}%
\pgfsys@useobject{currentmarker}{}%
\end{pgfscope}%
\begin{pgfscope}%
\pgfsys@transformshift{4.596818in}{0.362505in}%
\pgfsys@useobject{currentmarker}{}%
\end{pgfscope}%
\begin{pgfscope}%
\pgfsys@transformshift{4.639091in}{0.362505in}%
\pgfsys@useobject{currentmarker}{}%
\end{pgfscope}%
\begin{pgfscope}%
\pgfsys@transformshift{4.681364in}{0.362505in}%
\pgfsys@useobject{currentmarker}{}%
\end{pgfscope}%
\begin{pgfscope}%
\pgfsys@transformshift{4.723636in}{0.362510in}%
\pgfsys@useobject{currentmarker}{}%
\end{pgfscope}%
\begin{pgfscope}%
\pgfsys@transformshift{4.765909in}{0.362510in}%
\pgfsys@useobject{currentmarker}{}%
\end{pgfscope}%
\begin{pgfscope}%
\pgfsys@transformshift{4.808182in}{0.362510in}%
\pgfsys@useobject{currentmarker}{}%
\end{pgfscope}%
\begin{pgfscope}%
\pgfsys@transformshift{4.850455in}{0.362510in}%
\pgfsys@useobject{currentmarker}{}%
\end{pgfscope}%
\begin{pgfscope}%
\pgfsys@transformshift{4.892727in}{0.362505in}%
\pgfsys@useobject{currentmarker}{}%
\end{pgfscope}%
\begin{pgfscope}%
\pgfsys@transformshift{4.935000in}{0.362510in}%
\pgfsys@useobject{currentmarker}{}%
\end{pgfscope}%
\begin{pgfscope}%
\pgfsys@transformshift{4.977273in}{0.362500in}%
\pgfsys@useobject{currentmarker}{}%
\end{pgfscope}%
\begin{pgfscope}%
\pgfsys@transformshift{5.019545in}{0.362505in}%
\pgfsys@useobject{currentmarker}{}%
\end{pgfscope}%
\begin{pgfscope}%
\pgfsys@transformshift{5.061818in}{0.362510in}%
\pgfsys@useobject{currentmarker}{}%
\end{pgfscope}%
\begin{pgfscope}%
\pgfsys@transformshift{5.104091in}{0.362505in}%
\pgfsys@useobject{currentmarker}{}%
\end{pgfscope}%
\begin{pgfscope}%
\pgfsys@transformshift{5.146364in}{0.362505in}%
\pgfsys@useobject{currentmarker}{}%
\end{pgfscope}%
\begin{pgfscope}%
\pgfsys@transformshift{5.188636in}{0.362510in}%
\pgfsys@useobject{currentmarker}{}%
\end{pgfscope}%
\end{pgfscope}%
\begin{pgfscope}%
\pgfsetrectcap%
\pgfsetmiterjoin%
\pgfsetlinewidth{0.803000pt}%
\definecolor{currentstroke}{rgb}{0.000000,0.000000,0.000000}%
\pgfsetstrokecolor{currentstroke}%
\pgfsetdash{}{0pt}%
\pgfpathmoveto{\pgfqpoint{0.750000in}{0.274995in}}%
\pgfpathlineto{\pgfqpoint{0.750000in}{2.199995in}}%
\pgfusepath{stroke}%
\end{pgfscope}%
\begin{pgfscope}%
\pgfsetrectcap%
\pgfsetmiterjoin%
\pgfsetlinewidth{0.803000pt}%
\definecolor{currentstroke}{rgb}{0.000000,0.000000,0.000000}%
\pgfsetstrokecolor{currentstroke}%
\pgfsetdash{}{0pt}%
\pgfpathmoveto{\pgfqpoint{5.400000in}{0.274995in}}%
\pgfpathlineto{\pgfqpoint{5.400000in}{2.199995in}}%
\pgfusepath{stroke}%
\end{pgfscope}%
\begin{pgfscope}%
\pgfsetrectcap%
\pgfsetmiterjoin%
\pgfsetlinewidth{0.803000pt}%
\definecolor{currentstroke}{rgb}{0.000000,0.000000,0.000000}%
\pgfsetstrokecolor{currentstroke}%
\pgfsetdash{}{0pt}%
\pgfpathmoveto{\pgfqpoint{0.750000in}{0.275000in}}%
\pgfpathlineto{\pgfqpoint{5.400000in}{0.275000in}}%
\pgfusepath{stroke}%
\end{pgfscope}%
\begin{pgfscope}%
\pgfsetrectcap%
\pgfsetmiterjoin%
\pgfsetlinewidth{0.803000pt}%
\definecolor{currentstroke}{rgb}{0.000000,0.000000,0.000000}%
\pgfsetstrokecolor{currentstroke}%
\pgfsetdash{}{0pt}%
\pgfpathmoveto{\pgfqpoint{0.750000in}{2.200000in}}%
\pgfpathlineto{\pgfqpoint{5.400000in}{2.200000in}}%
\pgfusepath{stroke}%
\end{pgfscope}%
\end{pgfpicture}%
\makeatother%
\endgroup%

%% \caption{The lower bound in \cref{eq:GMRESprob} with $\NLiDR{\no-\nt} \sim \Exp{\sigma}$ with $\sigma = 1/k.$\label{fig:prob-theory-plot-1.0}}
%% \end{subfigure}

%% \begin{subfigure}{\textwidth}
%%     \centering
%%     %% Creator: Matplotlib, PGF backend
%%
%% To include the figure in your LaTeX document, write
%%   \input{<filename>.pgf}
%%
%% Make sure the required packages are loaded in your preamble
%%   \usepackage{pgf}
%%
%% Figures using additional raster images can only be included by \input if
%% they are in the same directory as the main LaTeX file. For loading figures
%% from other directories you can use the `import` package
%%   \usepackage{import}
%% and then include the figures with
%%   \import{<path to file>}{<filename>.pgf}
%%
%% Matplotlib used the following preamble
%%   \usepackage{fontspec}
%%   \setmainfont{DejaVuSerif.ttf}[Path=/home/owen/progs/firedrake-complex/firedrake/lib/python3.5/site-packages/matplotlib/mpl-data/fonts/ttf/]
%%   \setsansfont{DejaVuSans.ttf}[Path=/home/owen/progs/firedrake-complex/firedrake/lib/python3.5/site-packages/matplotlib/mpl-data/fonts/ttf/]
%%   \setmonofont{DejaVuSansMono.ttf}[Path=/home/owen/progs/firedrake-complex/firedrake/lib/python3.5/site-packages/matplotlib/mpl-data/fonts/ttf/]
%%
\begingroup%
\makeatletter%
\begin{pgfpicture}%
\pgfpathrectangle{\pgfpointorigin}{\pgfqpoint{6.000000in}{2.500000in}}%
\pgfusepath{use as bounding box, clip}%
\begin{pgfscope}%
\pgfsetbuttcap%
\pgfsetmiterjoin%
\definecolor{currentfill}{rgb}{1.000000,1.000000,1.000000}%
\pgfsetfillcolor{currentfill}%
\pgfsetlinewidth{0.000000pt}%
\definecolor{currentstroke}{rgb}{1.000000,1.000000,1.000000}%
\pgfsetstrokecolor{currentstroke}%
\pgfsetdash{}{0pt}%
\pgfpathmoveto{\pgfqpoint{0.000000in}{0.000000in}}%
\pgfpathlineto{\pgfqpoint{6.000000in}{0.000000in}}%
\pgfpathlineto{\pgfqpoint{6.000000in}{2.500000in}}%
\pgfpathlineto{\pgfqpoint{0.000000in}{2.500000in}}%
\pgfpathclose%
\pgfusepath{fill}%
\end{pgfscope}%
\begin{pgfscope}%
\pgfsetbuttcap%
\pgfsetmiterjoin%
\definecolor{currentfill}{rgb}{1.000000,1.000000,1.000000}%
\pgfsetfillcolor{currentfill}%
\pgfsetlinewidth{0.000000pt}%
\definecolor{currentstroke}{rgb}{0.000000,0.000000,0.000000}%
\pgfsetstrokecolor{currentstroke}%
\pgfsetstrokeopacity{0.000000}%
\pgfsetdash{}{0pt}%
\pgfpathmoveto{\pgfqpoint{0.750000in}{0.275000in}}%
\pgfpathlineto{\pgfqpoint{5.400000in}{0.275000in}}%
\pgfpathlineto{\pgfqpoint{5.400000in}{2.200000in}}%
\pgfpathlineto{\pgfqpoint{0.750000in}{2.200000in}}%
\pgfpathclose%
\pgfusepath{fill}%
\end{pgfscope}%
\begin{pgfscope}%
\pgfsetbuttcap%
\pgfsetroundjoin%
\definecolor{currentfill}{rgb}{0.000000,0.000000,0.000000}%
\pgfsetfillcolor{currentfill}%
\pgfsetlinewidth{0.803000pt}%
\definecolor{currentstroke}{rgb}{0.000000,0.000000,0.000000}%
\pgfsetstrokecolor{currentstroke}%
\pgfsetdash{}{0pt}%
\pgfsys@defobject{currentmarker}{\pgfqpoint{0.000000in}{-0.048611in}}{\pgfqpoint{0.000000in}{0.000000in}}{%
\pgfpathmoveto{\pgfqpoint{0.000000in}{0.000000in}}%
\pgfpathlineto{\pgfqpoint{0.000000in}{-0.048611in}}%
\pgfusepath{stroke,fill}%
}%
\begin{pgfscope}%
\pgfsys@transformshift{0.961364in}{0.275000in}%
\pgfsys@useobject{currentmarker}{}%
\end{pgfscope}%
\end{pgfscope}%
\begin{pgfscope}%
\definecolor{textcolor}{rgb}{0.000000,0.000000,0.000000}%
\pgfsetstrokecolor{textcolor}%
\pgfsetfillcolor{textcolor}%
\pgftext[x=0.961364in,y=0.177778in,,top]{\color{textcolor}\sffamily\fontsize{10.000000}{12.000000}\selectfont 10}%
\end{pgfscope}%
\begin{pgfscope}%
\pgfsetbuttcap%
\pgfsetroundjoin%
\definecolor{currentfill}{rgb}{0.000000,0.000000,0.000000}%
\pgfsetfillcolor{currentfill}%
\pgfsetlinewidth{0.803000pt}%
\definecolor{currentstroke}{rgb}{0.000000,0.000000,0.000000}%
\pgfsetstrokecolor{currentstroke}%
\pgfsetdash{}{0pt}%
\pgfsys@defobject{currentmarker}{\pgfqpoint{0.000000in}{-0.048611in}}{\pgfqpoint{0.000000in}{0.000000in}}{%
\pgfpathmoveto{\pgfqpoint{0.000000in}{0.000000in}}%
\pgfpathlineto{\pgfqpoint{0.000000in}{-0.048611in}}%
\pgfusepath{stroke,fill}%
}%
\begin{pgfscope}%
\pgfsys@transformshift{1.665909in}{0.275000in}%
\pgfsys@useobject{currentmarker}{}%
\end{pgfscope}%
\end{pgfscope}%
\begin{pgfscope}%
\definecolor{textcolor}{rgb}{0.000000,0.000000,0.000000}%
\pgfsetstrokecolor{textcolor}%
\pgfsetfillcolor{textcolor}%
\pgftext[x=1.665909in,y=0.177778in,,top]{\color{textcolor}\sffamily\fontsize{10.000000}{12.000000}\selectfont 15}%
\end{pgfscope}%
\begin{pgfscope}%
\pgfsetbuttcap%
\pgfsetroundjoin%
\definecolor{currentfill}{rgb}{0.000000,0.000000,0.000000}%
\pgfsetfillcolor{currentfill}%
\pgfsetlinewidth{0.803000pt}%
\definecolor{currentstroke}{rgb}{0.000000,0.000000,0.000000}%
\pgfsetstrokecolor{currentstroke}%
\pgfsetdash{}{0pt}%
\pgfsys@defobject{currentmarker}{\pgfqpoint{0.000000in}{-0.048611in}}{\pgfqpoint{0.000000in}{0.000000in}}{%
\pgfpathmoveto{\pgfqpoint{0.000000in}{0.000000in}}%
\pgfpathlineto{\pgfqpoint{0.000000in}{-0.048611in}}%
\pgfusepath{stroke,fill}%
}%
\begin{pgfscope}%
\pgfsys@transformshift{2.370455in}{0.275000in}%
\pgfsys@useobject{currentmarker}{}%
\end{pgfscope}%
\end{pgfscope}%
\begin{pgfscope}%
\definecolor{textcolor}{rgb}{0.000000,0.000000,0.000000}%
\pgfsetstrokecolor{textcolor}%
\pgfsetfillcolor{textcolor}%
\pgftext[x=2.370455in,y=0.177778in,,top]{\color{textcolor}\sffamily\fontsize{10.000000}{12.000000}\selectfont 20}%
\end{pgfscope}%
\begin{pgfscope}%
\pgfsetbuttcap%
\pgfsetroundjoin%
\definecolor{currentfill}{rgb}{0.000000,0.000000,0.000000}%
\pgfsetfillcolor{currentfill}%
\pgfsetlinewidth{0.803000pt}%
\definecolor{currentstroke}{rgb}{0.000000,0.000000,0.000000}%
\pgfsetstrokecolor{currentstroke}%
\pgfsetdash{}{0pt}%
\pgfsys@defobject{currentmarker}{\pgfqpoint{0.000000in}{-0.048611in}}{\pgfqpoint{0.000000in}{0.000000in}}{%
\pgfpathmoveto{\pgfqpoint{0.000000in}{0.000000in}}%
\pgfpathlineto{\pgfqpoint{0.000000in}{-0.048611in}}%
\pgfusepath{stroke,fill}%
}%
\begin{pgfscope}%
\pgfsys@transformshift{3.075000in}{0.275000in}%
\pgfsys@useobject{currentmarker}{}%
\end{pgfscope}%
\end{pgfscope}%
\begin{pgfscope}%
\definecolor{textcolor}{rgb}{0.000000,0.000000,0.000000}%
\pgfsetstrokecolor{textcolor}%
\pgfsetfillcolor{textcolor}%
\pgftext[x=3.075000in,y=0.177778in,,top]{\color{textcolor}\sffamily\fontsize{10.000000}{12.000000}\selectfont 25}%
\end{pgfscope}%
\begin{pgfscope}%
\pgfsetbuttcap%
\pgfsetroundjoin%
\definecolor{currentfill}{rgb}{0.000000,0.000000,0.000000}%
\pgfsetfillcolor{currentfill}%
\pgfsetlinewidth{0.803000pt}%
\definecolor{currentstroke}{rgb}{0.000000,0.000000,0.000000}%
\pgfsetstrokecolor{currentstroke}%
\pgfsetdash{}{0pt}%
\pgfsys@defobject{currentmarker}{\pgfqpoint{0.000000in}{-0.048611in}}{\pgfqpoint{0.000000in}{0.000000in}}{%
\pgfpathmoveto{\pgfqpoint{0.000000in}{0.000000in}}%
\pgfpathlineto{\pgfqpoint{0.000000in}{-0.048611in}}%
\pgfusepath{stroke,fill}%
}%
\begin{pgfscope}%
\pgfsys@transformshift{3.779545in}{0.275000in}%
\pgfsys@useobject{currentmarker}{}%
\end{pgfscope}%
\end{pgfscope}%
\begin{pgfscope}%
\definecolor{textcolor}{rgb}{0.000000,0.000000,0.000000}%
\pgfsetstrokecolor{textcolor}%
\pgfsetfillcolor{textcolor}%
\pgftext[x=3.779545in,y=0.177778in,,top]{\color{textcolor}\sffamily\fontsize{10.000000}{12.000000}\selectfont 30}%
\end{pgfscope}%
\begin{pgfscope}%
\pgfsetbuttcap%
\pgfsetroundjoin%
\definecolor{currentfill}{rgb}{0.000000,0.000000,0.000000}%
\pgfsetfillcolor{currentfill}%
\pgfsetlinewidth{0.803000pt}%
\definecolor{currentstroke}{rgb}{0.000000,0.000000,0.000000}%
\pgfsetstrokecolor{currentstroke}%
\pgfsetdash{}{0pt}%
\pgfsys@defobject{currentmarker}{\pgfqpoint{0.000000in}{-0.048611in}}{\pgfqpoint{0.000000in}{0.000000in}}{%
\pgfpathmoveto{\pgfqpoint{0.000000in}{0.000000in}}%
\pgfpathlineto{\pgfqpoint{0.000000in}{-0.048611in}}%
\pgfusepath{stroke,fill}%
}%
\begin{pgfscope}%
\pgfsys@transformshift{4.484091in}{0.275000in}%
\pgfsys@useobject{currentmarker}{}%
\end{pgfscope}%
\end{pgfscope}%
\begin{pgfscope}%
\definecolor{textcolor}{rgb}{0.000000,0.000000,0.000000}%
\pgfsetstrokecolor{textcolor}%
\pgfsetfillcolor{textcolor}%
\pgftext[x=4.484091in,y=0.177778in,,top]{\color{textcolor}\sffamily\fontsize{10.000000}{12.000000}\selectfont 35}%
\end{pgfscope}%
\begin{pgfscope}%
\pgfsetbuttcap%
\pgfsetroundjoin%
\definecolor{currentfill}{rgb}{0.000000,0.000000,0.000000}%
\pgfsetfillcolor{currentfill}%
\pgfsetlinewidth{0.803000pt}%
\definecolor{currentstroke}{rgb}{0.000000,0.000000,0.000000}%
\pgfsetstrokecolor{currentstroke}%
\pgfsetdash{}{0pt}%
\pgfsys@defobject{currentmarker}{\pgfqpoint{0.000000in}{-0.048611in}}{\pgfqpoint{0.000000in}{0.000000in}}{%
\pgfpathmoveto{\pgfqpoint{0.000000in}{0.000000in}}%
\pgfpathlineto{\pgfqpoint{0.000000in}{-0.048611in}}%
\pgfusepath{stroke,fill}%
}%
\begin{pgfscope}%
\pgfsys@transformshift{5.188636in}{0.275000in}%
\pgfsys@useobject{currentmarker}{}%
\end{pgfscope}%
\end{pgfscope}%
\begin{pgfscope}%
\definecolor{textcolor}{rgb}{0.000000,0.000000,0.000000}%
\pgfsetstrokecolor{textcolor}%
\pgfsetfillcolor{textcolor}%
\pgftext[x=5.188636in,y=0.177778in,,top]{\color{textcolor}\sffamily\fontsize{10.000000}{12.000000}\selectfont 40}%
\end{pgfscope}%
\begin{pgfscope}%
\definecolor{textcolor}{rgb}{0.000000,0.000000,0.000000}%
\pgfsetstrokecolor{textcolor}%
\pgfsetfillcolor{textcolor}%
\pgftext[x=3.075000in,y=-0.012191in,,top]{\color{textcolor}\sffamily\fontsize{10.000000}{12.000000}\selectfont \(\displaystyle k\)}%
\end{pgfscope}%
\begin{pgfscope}%
\pgfsetbuttcap%
\pgfsetroundjoin%
\definecolor{currentfill}{rgb}{0.000000,0.000000,0.000000}%
\pgfsetfillcolor{currentfill}%
\pgfsetlinewidth{0.803000pt}%
\definecolor{currentstroke}{rgb}{0.000000,0.000000,0.000000}%
\pgfsetstrokecolor{currentstroke}%
\pgfsetdash{}{0pt}%
\pgfsys@defobject{currentmarker}{\pgfqpoint{-0.048611in}{0.000000in}}{\pgfqpoint{0.000000in}{0.000000in}}{%
\pgfpathmoveto{\pgfqpoint{0.000000in}{0.000000in}}%
\pgfpathlineto{\pgfqpoint{-0.048611in}{0.000000in}}%
\pgfusepath{stroke,fill}%
}%
\begin{pgfscope}%
\pgfsys@transformshift{0.750000in}{0.298219in}%
\pgfsys@useobject{currentmarker}{}%
\end{pgfscope}%
\end{pgfscope}%
\begin{pgfscope}%
\definecolor{textcolor}{rgb}{0.000000,0.000000,0.000000}%
\pgfsetstrokecolor{textcolor}%
\pgfsetfillcolor{textcolor}%
\pgftext[x=0.343533in,y=0.245457in,left,base]{\color{textcolor}\sffamily\fontsize{10.000000}{12.000000}\selectfont 0.97}%
\end{pgfscope}%
\begin{pgfscope}%
\pgfsetbuttcap%
\pgfsetroundjoin%
\definecolor{currentfill}{rgb}{0.000000,0.000000,0.000000}%
\pgfsetfillcolor{currentfill}%
\pgfsetlinewidth{0.803000pt}%
\definecolor{currentstroke}{rgb}{0.000000,0.000000,0.000000}%
\pgfsetstrokecolor{currentstroke}%
\pgfsetdash{}{0pt}%
\pgfsys@defobject{currentmarker}{\pgfqpoint{-0.048611in}{0.000000in}}{\pgfqpoint{0.000000in}{0.000000in}}{%
\pgfpathmoveto{\pgfqpoint{0.000000in}{0.000000in}}%
\pgfpathlineto{\pgfqpoint{-0.048611in}{0.000000in}}%
\pgfusepath{stroke,fill}%
}%
\begin{pgfscope}%
\pgfsys@transformshift{0.750000in}{0.902993in}%
\pgfsys@useobject{currentmarker}{}%
\end{pgfscope}%
\end{pgfscope}%
\begin{pgfscope}%
\definecolor{textcolor}{rgb}{0.000000,0.000000,0.000000}%
\pgfsetstrokecolor{textcolor}%
\pgfsetfillcolor{textcolor}%
\pgftext[x=0.343533in,y=0.850232in,left,base]{\color{textcolor}\sffamily\fontsize{10.000000}{12.000000}\selectfont 0.98}%
\end{pgfscope}%
\begin{pgfscope}%
\pgfsetbuttcap%
\pgfsetroundjoin%
\definecolor{currentfill}{rgb}{0.000000,0.000000,0.000000}%
\pgfsetfillcolor{currentfill}%
\pgfsetlinewidth{0.803000pt}%
\definecolor{currentstroke}{rgb}{0.000000,0.000000,0.000000}%
\pgfsetstrokecolor{currentstroke}%
\pgfsetdash{}{0pt}%
\pgfsys@defobject{currentmarker}{\pgfqpoint{-0.048611in}{0.000000in}}{\pgfqpoint{0.000000in}{0.000000in}}{%
\pgfpathmoveto{\pgfqpoint{0.000000in}{0.000000in}}%
\pgfpathlineto{\pgfqpoint{-0.048611in}{0.000000in}}%
\pgfusepath{stroke,fill}%
}%
\begin{pgfscope}%
\pgfsys@transformshift{0.750000in}{1.507768in}%
\pgfsys@useobject{currentmarker}{}%
\end{pgfscope}%
\end{pgfscope}%
\begin{pgfscope}%
\definecolor{textcolor}{rgb}{0.000000,0.000000,0.000000}%
\pgfsetstrokecolor{textcolor}%
\pgfsetfillcolor{textcolor}%
\pgftext[x=0.343533in,y=1.455006in,left,base]{\color{textcolor}\sffamily\fontsize{10.000000}{12.000000}\selectfont 0.99}%
\end{pgfscope}%
\begin{pgfscope}%
\pgfsetbuttcap%
\pgfsetroundjoin%
\definecolor{currentfill}{rgb}{0.000000,0.000000,0.000000}%
\pgfsetfillcolor{currentfill}%
\pgfsetlinewidth{0.803000pt}%
\definecolor{currentstroke}{rgb}{0.000000,0.000000,0.000000}%
\pgfsetstrokecolor{currentstroke}%
\pgfsetdash{}{0pt}%
\pgfsys@defobject{currentmarker}{\pgfqpoint{-0.048611in}{0.000000in}}{\pgfqpoint{0.000000in}{0.000000in}}{%
\pgfpathmoveto{\pgfqpoint{0.000000in}{0.000000in}}%
\pgfpathlineto{\pgfqpoint{-0.048611in}{0.000000in}}%
\pgfusepath{stroke,fill}%
}%
\begin{pgfscope}%
\pgfsys@transformshift{0.750000in}{2.112542in}%
\pgfsys@useobject{currentmarker}{}%
\end{pgfscope}%
\end{pgfscope}%
\begin{pgfscope}%
\definecolor{textcolor}{rgb}{0.000000,0.000000,0.000000}%
\pgfsetstrokecolor{textcolor}%
\pgfsetfillcolor{textcolor}%
\pgftext[x=0.343533in,y=2.059781in,left,base]{\color{textcolor}\sffamily\fontsize{10.000000}{12.000000}\selectfont 1.00}%
\end{pgfscope}%
\begin{pgfscope}%
\definecolor{textcolor}{rgb}{0.000000,0.000000,0.000000}%
\pgfsetstrokecolor{textcolor}%
\pgfsetfillcolor{textcolor}%
\pgftext[x=0.103570in,y=0.325194in,left,base,rotate=90.000000]{\color{textcolor}\sffamily\fontsize{10.000000}{12.000000}\selectfont Probability that number of}%
\end{pgfscope}%
\begin{pgfscope}%
\definecolor{textcolor}{rgb}{0.000000,0.000000,0.000000}%
\pgfsetstrokecolor{textcolor}%
\pgfsetfillcolor{textcolor}%
\pgftext[x=0.259088in,y=0.160467in,left,base,rotate=90.000000]{\color{textcolor}\sffamily\fontsize{10.000000}{12.000000}\selectfont GMRES iterations is at most 12}%
\end{pgfscope}%
\begin{pgfscope}%
\pgfpathrectangle{\pgfqpoint{0.750000in}{0.275000in}}{\pgfqpoint{4.650000in}{1.925000in}}%
\pgfusepath{clip}%
\pgfsetbuttcap%
\pgfsetroundjoin%
\definecolor{currentfill}{rgb}{0.000000,0.000000,0.000000}%
\pgfsetfillcolor{currentfill}%
\pgfsetlinewidth{1.003750pt}%
\definecolor{currentstroke}{rgb}{0.000000,0.000000,0.000000}%
\pgfsetstrokecolor{currentstroke}%
\pgfsetdash{}{0pt}%
\pgfsys@defobject{currentmarker}{\pgfqpoint{-0.020833in}{-0.020833in}}{\pgfqpoint{0.020833in}{0.020833in}}{%
\pgfpathmoveto{\pgfqpoint{0.000000in}{-0.020833in}}%
\pgfpathcurveto{\pgfqpoint{0.005525in}{-0.020833in}}{\pgfqpoint{0.010825in}{-0.018638in}}{\pgfqpoint{0.014731in}{-0.014731in}}%
\pgfpathcurveto{\pgfqpoint{0.018638in}{-0.010825in}}{\pgfqpoint{0.020833in}{-0.005525in}}{\pgfqpoint{0.020833in}{0.000000in}}%
\pgfpathcurveto{\pgfqpoint{0.020833in}{0.005525in}}{\pgfqpoint{0.018638in}{0.010825in}}{\pgfqpoint{0.014731in}{0.014731in}}%
\pgfpathcurveto{\pgfqpoint{0.010825in}{0.018638in}}{\pgfqpoint{0.005525in}{0.020833in}}{\pgfqpoint{0.000000in}{0.020833in}}%
\pgfpathcurveto{\pgfqpoint{-0.005525in}{0.020833in}}{\pgfqpoint{-0.010825in}{0.018638in}}{\pgfqpoint{-0.014731in}{0.014731in}}%
\pgfpathcurveto{\pgfqpoint{-0.018638in}{0.010825in}}{\pgfqpoint{-0.020833in}{0.005525in}}{\pgfqpoint{-0.020833in}{0.000000in}}%
\pgfpathcurveto{\pgfqpoint{-0.020833in}{-0.005525in}}{\pgfqpoint{-0.018638in}{-0.010825in}}{\pgfqpoint{-0.014731in}{-0.014731in}}%
\pgfpathcurveto{\pgfqpoint{-0.010825in}{-0.018638in}}{\pgfqpoint{-0.005525in}{-0.020833in}}{\pgfqpoint{0.000000in}{-0.020833in}}%
\pgfpathclose%
\pgfusepath{stroke,fill}%
}%
\begin{pgfscope}%
\pgfsys@transformshift{0.961364in}{0.362500in}%
\pgfsys@useobject{currentmarker}{}%
\end{pgfscope}%
\begin{pgfscope}%
\pgfsys@transformshift{1.003636in}{0.538950in}%
\pgfsys@useobject{currentmarker}{}%
\end{pgfscope}%
\begin{pgfscope}%
\pgfsys@transformshift{1.045909in}{0.697609in}%
\pgfsys@useobject{currentmarker}{}%
\end{pgfscope}%
\begin{pgfscope}%
\pgfsys@transformshift{1.088182in}{0.840272in}%
\pgfsys@useobject{currentmarker}{}%
\end{pgfscope}%
\begin{pgfscope}%
\pgfsys@transformshift{1.130455in}{0.968550in}%
\pgfsys@useobject{currentmarker}{}%
\end{pgfscope}%
\begin{pgfscope}%
\pgfsys@transformshift{1.172727in}{1.083894in}%
\pgfsys@useobject{currentmarker}{}%
\end{pgfscope}%
\begin{pgfscope}%
\pgfsys@transformshift{1.215000in}{1.187609in}%
\pgfsys@useobject{currentmarker}{}%
\end{pgfscope}%
\begin{pgfscope}%
\pgfsys@transformshift{1.257273in}{1.280866in}%
\pgfsys@useobject{currentmarker}{}%
\end{pgfscope}%
\begin{pgfscope}%
\pgfsys@transformshift{1.299545in}{1.364721in}%
\pgfsys@useobject{currentmarker}{}%
\end{pgfscope}%
\begin{pgfscope}%
\pgfsys@transformshift{1.341818in}{1.440121in}%
\pgfsys@useobject{currentmarker}{}%
\end{pgfscope}%
\begin{pgfscope}%
\pgfsys@transformshift{1.384091in}{1.507919in}%
\pgfsys@useobject{currentmarker}{}%
\end{pgfscope}%
\begin{pgfscope}%
\pgfsys@transformshift{1.426364in}{1.568881in}%
\pgfsys@useobject{currentmarker}{}%
\end{pgfscope}%
\begin{pgfscope}%
\pgfsys@transformshift{1.468636in}{1.623696in}%
\pgfsys@useobject{currentmarker}{}%
\end{pgfscope}%
\begin{pgfscope}%
\pgfsys@transformshift{1.510909in}{1.672985in}%
\pgfsys@useobject{currentmarker}{}%
\end{pgfscope}%
\begin{pgfscope}%
\pgfsys@transformshift{1.553182in}{1.717303in}%
\pgfsys@useobject{currentmarker}{}%
\end{pgfscope}%
\begin{pgfscope}%
\pgfsys@transformshift{1.595455in}{1.757154in}%
\pgfsys@useobject{currentmarker}{}%
\end{pgfscope}%
\begin{pgfscope}%
\pgfsys@transformshift{1.637727in}{1.792986in}%
\pgfsys@useobject{currentmarker}{}%
\end{pgfscope}%
\begin{pgfscope}%
\pgfsys@transformshift{1.680000in}{1.825206in}%
\pgfsys@useobject{currentmarker}{}%
\end{pgfscope}%
\begin{pgfscope}%
\pgfsys@transformshift{1.722273in}{1.854177in}%
\pgfsys@useobject{currentmarker}{}%
\end{pgfscope}%
\begin{pgfscope}%
\pgfsys@transformshift{1.764545in}{1.880227in}%
\pgfsys@useobject{currentmarker}{}%
\end{pgfscope}%
\begin{pgfscope}%
\pgfsys@transformshift{1.806818in}{1.903651in}%
\pgfsys@useobject{currentmarker}{}%
\end{pgfscope}%
\begin{pgfscope}%
\pgfsys@transformshift{1.849091in}{1.924712in}%
\pgfsys@useobject{currentmarker}{}%
\end{pgfscope}%
\begin{pgfscope}%
\pgfsys@transformshift{1.891364in}{1.943650in}%
\pgfsys@useobject{currentmarker}{}%
\end{pgfscope}%
\begin{pgfscope}%
\pgfsys@transformshift{1.933636in}{1.960679in}%
\pgfsys@useobject{currentmarker}{}%
\end{pgfscope}%
\begin{pgfscope}%
\pgfsys@transformshift{1.975909in}{1.975991in}%
\pgfsys@useobject{currentmarker}{}%
\end{pgfscope}%
\begin{pgfscope}%
\pgfsys@transformshift{2.018182in}{1.989759in}%
\pgfsys@useobject{currentmarker}{}%
\end{pgfscope}%
\begin{pgfscope}%
\pgfsys@transformshift{2.060455in}{2.002139in}%
\pgfsys@useobject{currentmarker}{}%
\end{pgfscope}%
\begin{pgfscope}%
\pgfsys@transformshift{2.102727in}{2.013270in}%
\pgfsys@useobject{currentmarker}{}%
\end{pgfscope}%
\begin{pgfscope}%
\pgfsys@transformshift{2.145000in}{2.023280in}%
\pgfsys@useobject{currentmarker}{}%
\end{pgfscope}%
\begin{pgfscope}%
\pgfsys@transformshift{2.187273in}{2.032280in}%
\pgfsys@useobject{currentmarker}{}%
\end{pgfscope}%
\begin{pgfscope}%
\pgfsys@transformshift{2.229545in}{2.040372in}%
\pgfsys@useobject{currentmarker}{}%
\end{pgfscope}%
\begin{pgfscope}%
\pgfsys@transformshift{2.271818in}{2.047649in}%
\pgfsys@useobject{currentmarker}{}%
\end{pgfscope}%
\begin{pgfscope}%
\pgfsys@transformshift{2.314091in}{2.054192in}%
\pgfsys@useobject{currentmarker}{}%
\end{pgfscope}%
\begin{pgfscope}%
\pgfsys@transformshift{2.356364in}{2.060075in}%
\pgfsys@useobject{currentmarker}{}%
\end{pgfscope}%
\begin{pgfscope}%
\pgfsys@transformshift{2.398636in}{2.065365in}%
\pgfsys@useobject{currentmarker}{}%
\end{pgfscope}%
\begin{pgfscope}%
\pgfsys@transformshift{2.440909in}{2.070122in}%
\pgfsys@useobject{currentmarker}{}%
\end{pgfscope}%
\begin{pgfscope}%
\pgfsys@transformshift{2.483182in}{2.074399in}%
\pgfsys@useobject{currentmarker}{}%
\end{pgfscope}%
\begin{pgfscope}%
\pgfsys@transformshift{2.525455in}{2.078245in}%
\pgfsys@useobject{currentmarker}{}%
\end{pgfscope}%
\begin{pgfscope}%
\pgfsys@transformshift{2.567727in}{2.081703in}%
\pgfsys@useobject{currentmarker}{}%
\end{pgfscope}%
\begin{pgfscope}%
\pgfsys@transformshift{2.610000in}{2.084812in}%
\pgfsys@useobject{currentmarker}{}%
\end{pgfscope}%
\begin{pgfscope}%
\pgfsys@transformshift{2.652273in}{2.087608in}%
\pgfsys@useobject{currentmarker}{}%
\end{pgfscope}%
\begin{pgfscope}%
\pgfsys@transformshift{2.694545in}{2.090122in}%
\pgfsys@useobject{currentmarker}{}%
\end{pgfscope}%
\begin{pgfscope}%
\pgfsys@transformshift{2.736818in}{2.092383in}%
\pgfsys@useobject{currentmarker}{}%
\end{pgfscope}%
\begin{pgfscope}%
\pgfsys@transformshift{2.779091in}{2.094415in}%
\pgfsys@useobject{currentmarker}{}%
\end{pgfscope}%
\begin{pgfscope}%
\pgfsys@transformshift{2.821364in}{2.096243in}%
\pgfsys@useobject{currentmarker}{}%
\end{pgfscope}%
\begin{pgfscope}%
\pgfsys@transformshift{2.863636in}{2.097886in}%
\pgfsys@useobject{currentmarker}{}%
\end{pgfscope}%
\begin{pgfscope}%
\pgfsys@transformshift{2.905909in}{2.099364in}%
\pgfsys@useobject{currentmarker}{}%
\end{pgfscope}%
\begin{pgfscope}%
\pgfsys@transformshift{2.948182in}{2.100693in}%
\pgfsys@useobject{currentmarker}{}%
\end{pgfscope}%
\begin{pgfscope}%
\pgfsys@transformshift{2.990455in}{2.101888in}%
\pgfsys@useobject{currentmarker}{}%
\end{pgfscope}%
\begin{pgfscope}%
\pgfsys@transformshift{3.032727in}{2.102962in}%
\pgfsys@useobject{currentmarker}{}%
\end{pgfscope}%
\begin{pgfscope}%
\pgfsys@transformshift{3.075000in}{2.103928in}%
\pgfsys@useobject{currentmarker}{}%
\end{pgfscope}%
\begin{pgfscope}%
\pgfsys@transformshift{3.117273in}{2.104796in}%
\pgfsys@useobject{currentmarker}{}%
\end{pgfscope}%
\begin{pgfscope}%
\pgfsys@transformshift{3.159545in}{2.105577in}%
\pgfsys@useobject{currentmarker}{}%
\end{pgfscope}%
\begin{pgfscope}%
\pgfsys@transformshift{3.201818in}{2.106280in}%
\pgfsys@useobject{currentmarker}{}%
\end{pgfscope}%
\begin{pgfscope}%
\pgfsys@transformshift{3.244091in}{2.106911in}%
\pgfsys@useobject{currentmarker}{}%
\end{pgfscope}%
\begin{pgfscope}%
\pgfsys@transformshift{3.286364in}{2.107479in}%
\pgfsys@useobject{currentmarker}{}%
\end{pgfscope}%
\begin{pgfscope}%
\pgfsys@transformshift{3.328636in}{2.107989in}%
\pgfsys@useobject{currentmarker}{}%
\end{pgfscope}%
\begin{pgfscope}%
\pgfsys@transformshift{3.370909in}{2.108449in}%
\pgfsys@useobject{currentmarker}{}%
\end{pgfscope}%
\begin{pgfscope}%
\pgfsys@transformshift{3.413182in}{2.108861in}%
\pgfsys@useobject{currentmarker}{}%
\end{pgfscope}%
\begin{pgfscope}%
\pgfsys@transformshift{3.455455in}{2.109232in}%
\pgfsys@useobject{currentmarker}{}%
\end{pgfscope}%
\begin{pgfscope}%
\pgfsys@transformshift{3.497727in}{2.109566in}%
\pgfsys@useobject{currentmarker}{}%
\end{pgfscope}%
\begin{pgfscope}%
\pgfsys@transformshift{3.540000in}{2.109866in}%
\pgfsys@useobject{currentmarker}{}%
\end{pgfscope}%
\begin{pgfscope}%
\pgfsys@transformshift{3.582273in}{2.110136in}%
\pgfsys@useobject{currentmarker}{}%
\end{pgfscope}%
\begin{pgfscope}%
\pgfsys@transformshift{3.624545in}{2.110379in}%
\pgfsys@useobject{currentmarker}{}%
\end{pgfscope}%
\begin{pgfscope}%
\pgfsys@transformshift{3.666818in}{2.110597in}%
\pgfsys@useobject{currentmarker}{}%
\end{pgfscope}%
\begin{pgfscope}%
\pgfsys@transformshift{3.709091in}{2.110793in}%
\pgfsys@useobject{currentmarker}{}%
\end{pgfscope}%
\begin{pgfscope}%
\pgfsys@transformshift{3.751364in}{2.110969in}%
\pgfsys@useobject{currentmarker}{}%
\end{pgfscope}%
\begin{pgfscope}%
\pgfsys@transformshift{3.793636in}{2.111128in}%
\pgfsys@useobject{currentmarker}{}%
\end{pgfscope}%
\begin{pgfscope}%
\pgfsys@transformshift{3.835909in}{2.111271in}%
\pgfsys@useobject{currentmarker}{}%
\end{pgfscope}%
\begin{pgfscope}%
\pgfsys@transformshift{3.878182in}{2.111399in}%
\pgfsys@useobject{currentmarker}{}%
\end{pgfscope}%
\begin{pgfscope}%
\pgfsys@transformshift{3.920455in}{2.111514in}%
\pgfsys@useobject{currentmarker}{}%
\end{pgfscope}%
\begin{pgfscope}%
\pgfsys@transformshift{3.962727in}{2.111618in}%
\pgfsys@useobject{currentmarker}{}%
\end{pgfscope}%
\begin{pgfscope}%
\pgfsys@transformshift{4.005000in}{2.111711in}%
\pgfsys@useobject{currentmarker}{}%
\end{pgfscope}%
\begin{pgfscope}%
\pgfsys@transformshift{4.047273in}{2.111795in}%
\pgfsys@useobject{currentmarker}{}%
\end{pgfscope}%
\begin{pgfscope}%
\pgfsys@transformshift{4.089545in}{2.111870in}%
\pgfsys@useobject{currentmarker}{}%
\end{pgfscope}%
\begin{pgfscope}%
\pgfsys@transformshift{4.131818in}{2.111938in}%
\pgfsys@useobject{currentmarker}{}%
\end{pgfscope}%
\begin{pgfscope}%
\pgfsys@transformshift{4.174091in}{2.111999in}%
\pgfsys@useobject{currentmarker}{}%
\end{pgfscope}%
\begin{pgfscope}%
\pgfsys@transformshift{4.216364in}{2.112054in}%
\pgfsys@useobject{currentmarker}{}%
\end{pgfscope}%
\begin{pgfscope}%
\pgfsys@transformshift{4.258636in}{2.112103in}%
\pgfsys@useobject{currentmarker}{}%
\end{pgfscope}%
\begin{pgfscope}%
\pgfsys@transformshift{4.300909in}{2.112147in}%
\pgfsys@useobject{currentmarker}{}%
\end{pgfscope}%
\begin{pgfscope}%
\pgfsys@transformshift{4.343182in}{2.112187in}%
\pgfsys@useobject{currentmarker}{}%
\end{pgfscope}%
\begin{pgfscope}%
\pgfsys@transformshift{4.385455in}{2.112223in}%
\pgfsys@useobject{currentmarker}{}%
\end{pgfscope}%
\begin{pgfscope}%
\pgfsys@transformshift{4.427727in}{2.112255in}%
\pgfsys@useobject{currentmarker}{}%
\end{pgfscope}%
\begin{pgfscope}%
\pgfsys@transformshift{4.470000in}{2.112284in}%
\pgfsys@useobject{currentmarker}{}%
\end{pgfscope}%
\begin{pgfscope}%
\pgfsys@transformshift{4.512273in}{2.112310in}%
\pgfsys@useobject{currentmarker}{}%
\end{pgfscope}%
\begin{pgfscope}%
\pgfsys@transformshift{4.554545in}{2.112334in}%
\pgfsys@useobject{currentmarker}{}%
\end{pgfscope}%
\begin{pgfscope}%
\pgfsys@transformshift{4.596818in}{2.112355in}%
\pgfsys@useobject{currentmarker}{}%
\end{pgfscope}%
\begin{pgfscope}%
\pgfsys@transformshift{4.639091in}{2.112374in}%
\pgfsys@useobject{currentmarker}{}%
\end{pgfscope}%
\begin{pgfscope}%
\pgfsys@transformshift{4.681364in}{2.112391in}%
\pgfsys@useobject{currentmarker}{}%
\end{pgfscope}%
\begin{pgfscope}%
\pgfsys@transformshift{4.723636in}{2.112406in}%
\pgfsys@useobject{currentmarker}{}%
\end{pgfscope}%
\begin{pgfscope}%
\pgfsys@transformshift{4.765909in}{2.112420in}%
\pgfsys@useobject{currentmarker}{}%
\end{pgfscope}%
\begin{pgfscope}%
\pgfsys@transformshift{4.808182in}{2.112432in}%
\pgfsys@useobject{currentmarker}{}%
\end{pgfscope}%
\begin{pgfscope}%
\pgfsys@transformshift{4.850455in}{2.112443in}%
\pgfsys@useobject{currentmarker}{}%
\end{pgfscope}%
\begin{pgfscope}%
\pgfsys@transformshift{4.892727in}{2.112453in}%
\pgfsys@useobject{currentmarker}{}%
\end{pgfscope}%
\begin{pgfscope}%
\pgfsys@transformshift{4.935000in}{2.112462in}%
\pgfsys@useobject{currentmarker}{}%
\end{pgfscope}%
\begin{pgfscope}%
\pgfsys@transformshift{4.977273in}{2.112470in}%
\pgfsys@useobject{currentmarker}{}%
\end{pgfscope}%
\begin{pgfscope}%
\pgfsys@transformshift{5.019545in}{2.112478in}%
\pgfsys@useobject{currentmarker}{}%
\end{pgfscope}%
\begin{pgfscope}%
\pgfsys@transformshift{5.061818in}{2.112484in}%
\pgfsys@useobject{currentmarker}{}%
\end{pgfscope}%
\begin{pgfscope}%
\pgfsys@transformshift{5.104091in}{2.112490in}%
\pgfsys@useobject{currentmarker}{}%
\end{pgfscope}%
\begin{pgfscope}%
\pgfsys@transformshift{5.146364in}{2.112495in}%
\pgfsys@useobject{currentmarker}{}%
\end{pgfscope}%
\begin{pgfscope}%
\pgfsys@transformshift{5.188636in}{2.112500in}%
\pgfsys@useobject{currentmarker}{}%
\end{pgfscope}%
\end{pgfscope}%
\begin{pgfscope}%
\pgfsetrectcap%
\pgfsetmiterjoin%
\pgfsetlinewidth{0.803000pt}%
\definecolor{currentstroke}{rgb}{0.000000,0.000000,0.000000}%
\pgfsetstrokecolor{currentstroke}%
\pgfsetdash{}{0pt}%
\pgfpathmoveto{\pgfqpoint{0.750000in}{0.275000in}}%
\pgfpathlineto{\pgfqpoint{0.750000in}{2.200000in}}%
\pgfusepath{stroke}%
\end{pgfscope}%
\begin{pgfscope}%
\pgfsetrectcap%
\pgfsetmiterjoin%
\pgfsetlinewidth{0.803000pt}%
\definecolor{currentstroke}{rgb}{0.000000,0.000000,0.000000}%
\pgfsetstrokecolor{currentstroke}%
\pgfsetdash{}{0pt}%
\pgfpathmoveto{\pgfqpoint{5.400000in}{0.275000in}}%
\pgfpathlineto{\pgfqpoint{5.400000in}{2.200000in}}%
\pgfusepath{stroke}%
\end{pgfscope}%
\begin{pgfscope}%
\pgfsetrectcap%
\pgfsetmiterjoin%
\pgfsetlinewidth{0.803000pt}%
\definecolor{currentstroke}{rgb}{0.000000,0.000000,0.000000}%
\pgfsetstrokecolor{currentstroke}%
\pgfsetdash{}{0pt}%
\pgfpathmoveto{\pgfqpoint{0.750000in}{0.275000in}}%
\pgfpathlineto{\pgfqpoint{5.400000in}{0.275000in}}%
\pgfusepath{stroke}%
\end{pgfscope}%
\begin{pgfscope}%
\pgfsetrectcap%
\pgfsetmiterjoin%
\pgfsetlinewidth{0.803000pt}%
\definecolor{currentstroke}{rgb}{0.000000,0.000000,0.000000}%
\pgfsetstrokecolor{currentstroke}%
\pgfsetdash{}{0pt}%
\pgfpathmoveto{\pgfqpoint{0.750000in}{2.200000in}}%
\pgfpathlineto{\pgfqpoint{5.400000in}{2.200000in}}%
\pgfusepath{stroke}%
\end{pgfscope}%
\end{pgfpicture}%
\makeatother%
\endgroup%

%%     \caption{The lower bound in \cref{eq:GMRESprob} with $\NLiDR{\no-\nt} \sim \Exp{\sigma}$ with $\sigma = 1/k^2.$\label{fig:prob-theory-plot-2.0}}
%% \end{subfigure}
%% \caption{The lower bound in \cref{eq:GMRESprob} for $R=12$, $\eps = 10^{-5}$, $N = \ceil{k^{3}}$, and $\Ct=0.1,$ for different functional forms of $\NLiDR{\no-\nt}$.}
%% \end{figure}
%% \ednote{Euan---the y axis on \cref{fig:prob-theory-plot-1.0} isn't correct. But basically, the values are all around 0.29, to within $10^{-11}$.}

\begin{figure}[p]
  \centering
  \begin{subfigure}{\textwidth}
    \centering
%% Creator: Matplotlib, PGF backend
%%
%% To include the figure in your LaTeX document, write
%%   \input{<filename>.pgf}
%%
%% Make sure the required packages are loaded in your preamble
%%   \usepackage{pgf}
%%
%% Figures using additional raster images can only be included by \input if
%% they are in the same directory as the main LaTeX file. For loading figures
%% from other directories you can use the `import` package
%%   \usepackage{import}
%% and then include the figures with
%%   \import{<path to file>}{<filename>.pgf}
%%
%% Matplotlib used the following preamble
%%   \usepackage{fontspec}
%%   \setmainfont{DejaVuSerif.ttf}[Path=/home/owen/progs/firedrake-complex/firedrake/lib/python3.5/site-packages/matplotlib/mpl-data/fonts/ttf/]
%%   \setsansfont{DejaVuSans.ttf}[Path=/home/owen/progs/firedrake-complex/firedrake/lib/python3.5/site-packages/matplotlib/mpl-data/fonts/ttf/]
%%   \setmonofont{DejaVuSansMono.ttf}[Path=/home/owen/progs/firedrake-complex/firedrake/lib/python3.5/site-packages/matplotlib/mpl-data/fonts/ttf/]
%%
\begingroup%
\makeatletter%
\begin{pgfpicture}%
\pgfpathrectangle{\pgfpointorigin}{\pgfqpoint{6.400000in}{4.800000in}}%
\pgfusepath{use as bounding box, clip}%
\begin{pgfscope}%
\pgfsetbuttcap%
\pgfsetmiterjoin%
\definecolor{currentfill}{rgb}{1.000000,1.000000,1.000000}%
\pgfsetfillcolor{currentfill}%
\pgfsetlinewidth{0.000000pt}%
\definecolor{currentstroke}{rgb}{1.000000,1.000000,1.000000}%
\pgfsetstrokecolor{currentstroke}%
\pgfsetdash{}{0pt}%
\pgfpathmoveto{\pgfqpoint{0.000000in}{0.000000in}}%
\pgfpathlineto{\pgfqpoint{6.400000in}{0.000000in}}%
\pgfpathlineto{\pgfqpoint{6.400000in}{4.800000in}}%
\pgfpathlineto{\pgfqpoint{0.000000in}{4.800000in}}%
\pgfpathclose%
\pgfusepath{fill}%
\end{pgfscope}%
\begin{pgfscope}%
\pgfsetbuttcap%
\pgfsetmiterjoin%
\definecolor{currentfill}{rgb}{1.000000,1.000000,1.000000}%
\pgfsetfillcolor{currentfill}%
\pgfsetlinewidth{0.000000pt}%
\definecolor{currentstroke}{rgb}{0.000000,0.000000,0.000000}%
\pgfsetstrokecolor{currentstroke}%
\pgfsetstrokeopacity{0.000000}%
\pgfsetdash{}{0pt}%
\pgfpathmoveto{\pgfqpoint{0.800000in}{0.528000in}}%
\pgfpathlineto{\pgfqpoint{5.760000in}{0.528000in}}%
\pgfpathlineto{\pgfqpoint{5.760000in}{4.224000in}}%
\pgfpathlineto{\pgfqpoint{0.800000in}{4.224000in}}%
\pgfpathclose%
\pgfusepath{fill}%
\end{pgfscope}%
\begin{pgfscope}%
\pgfsetbuttcap%
\pgfsetroundjoin%
\definecolor{currentfill}{rgb}{0.000000,0.000000,0.000000}%
\pgfsetfillcolor{currentfill}%
\pgfsetlinewidth{0.803000pt}%
\definecolor{currentstroke}{rgb}{0.000000,0.000000,0.000000}%
\pgfsetstrokecolor{currentstroke}%
\pgfsetdash{}{0pt}%
\pgfsys@defobject{currentmarker}{\pgfqpoint{0.000000in}{-0.048611in}}{\pgfqpoint{0.000000in}{0.000000in}}{%
\pgfpathmoveto{\pgfqpoint{0.000000in}{0.000000in}}%
\pgfpathlineto{\pgfqpoint{0.000000in}{-0.048611in}}%
\pgfusepath{stroke,fill}%
}%
\begin{pgfscope}%
\pgfsys@transformshift{1.025455in}{0.528000in}%
\pgfsys@useobject{currentmarker}{}%
\end{pgfscope}%
\end{pgfscope}%
\begin{pgfscope}%
\definecolor{textcolor}{rgb}{0.000000,0.000000,0.000000}%
\pgfsetstrokecolor{textcolor}%
\pgfsetfillcolor{textcolor}%
\pgftext[x=1.025455in,y=0.430778in,,top]{\color{textcolor}\sffamily\fontsize{10.000000}{12.000000}\selectfont 10}%
\end{pgfscope}%
\begin{pgfscope}%
\pgfsetbuttcap%
\pgfsetroundjoin%
\definecolor{currentfill}{rgb}{0.000000,0.000000,0.000000}%
\pgfsetfillcolor{currentfill}%
\pgfsetlinewidth{0.803000pt}%
\definecolor{currentstroke}{rgb}{0.000000,0.000000,0.000000}%
\pgfsetstrokecolor{currentstroke}%
\pgfsetdash{}{0pt}%
\pgfsys@defobject{currentmarker}{\pgfqpoint{0.000000in}{-0.048611in}}{\pgfqpoint{0.000000in}{0.000000in}}{%
\pgfpathmoveto{\pgfqpoint{0.000000in}{0.000000in}}%
\pgfpathlineto{\pgfqpoint{0.000000in}{-0.048611in}}%
\pgfusepath{stroke,fill}%
}%
\begin{pgfscope}%
\pgfsys@transformshift{2.528485in}{0.528000in}%
\pgfsys@useobject{currentmarker}{}%
\end{pgfscope}%
\end{pgfscope}%
\begin{pgfscope}%
\definecolor{textcolor}{rgb}{0.000000,0.000000,0.000000}%
\pgfsetstrokecolor{textcolor}%
\pgfsetfillcolor{textcolor}%
\pgftext[x=2.528485in,y=0.430778in,,top]{\color{textcolor}\sffamily\fontsize{10.000000}{12.000000}\selectfont 20}%
\end{pgfscope}%
\begin{pgfscope}%
\pgfsetbuttcap%
\pgfsetroundjoin%
\definecolor{currentfill}{rgb}{0.000000,0.000000,0.000000}%
\pgfsetfillcolor{currentfill}%
\pgfsetlinewidth{0.803000pt}%
\definecolor{currentstroke}{rgb}{0.000000,0.000000,0.000000}%
\pgfsetstrokecolor{currentstroke}%
\pgfsetdash{}{0pt}%
\pgfsys@defobject{currentmarker}{\pgfqpoint{0.000000in}{-0.048611in}}{\pgfqpoint{0.000000in}{0.000000in}}{%
\pgfpathmoveto{\pgfqpoint{0.000000in}{0.000000in}}%
\pgfpathlineto{\pgfqpoint{0.000000in}{-0.048611in}}%
\pgfusepath{stroke,fill}%
}%
\begin{pgfscope}%
\pgfsys@transformshift{4.031515in}{0.528000in}%
\pgfsys@useobject{currentmarker}{}%
\end{pgfscope}%
\end{pgfscope}%
\begin{pgfscope}%
\definecolor{textcolor}{rgb}{0.000000,0.000000,0.000000}%
\pgfsetstrokecolor{textcolor}%
\pgfsetfillcolor{textcolor}%
\pgftext[x=4.031515in,y=0.430778in,,top]{\color{textcolor}\sffamily\fontsize{10.000000}{12.000000}\selectfont 30}%
\end{pgfscope}%
\begin{pgfscope}%
\pgfsetbuttcap%
\pgfsetroundjoin%
\definecolor{currentfill}{rgb}{0.000000,0.000000,0.000000}%
\pgfsetfillcolor{currentfill}%
\pgfsetlinewidth{0.803000pt}%
\definecolor{currentstroke}{rgb}{0.000000,0.000000,0.000000}%
\pgfsetstrokecolor{currentstroke}%
\pgfsetdash{}{0pt}%
\pgfsys@defobject{currentmarker}{\pgfqpoint{0.000000in}{-0.048611in}}{\pgfqpoint{0.000000in}{0.000000in}}{%
\pgfpathmoveto{\pgfqpoint{0.000000in}{0.000000in}}%
\pgfpathlineto{\pgfqpoint{0.000000in}{-0.048611in}}%
\pgfusepath{stroke,fill}%
}%
\begin{pgfscope}%
\pgfsys@transformshift{5.534545in}{0.528000in}%
\pgfsys@useobject{currentmarker}{}%
\end{pgfscope}%
\end{pgfscope}%
\begin{pgfscope}%
\definecolor{textcolor}{rgb}{0.000000,0.000000,0.000000}%
\pgfsetstrokecolor{textcolor}%
\pgfsetfillcolor{textcolor}%
\pgftext[x=5.534545in,y=0.430778in,,top]{\color{textcolor}\sffamily\fontsize{10.000000}{12.000000}\selectfont 40}%
\end{pgfscope}%
\begin{pgfscope}%
\definecolor{textcolor}{rgb}{0.000000,0.000000,0.000000}%
\pgfsetstrokecolor{textcolor}%
\pgfsetfillcolor{textcolor}%
\pgftext[x=3.280000in,y=0.240809in,,top]{\color{textcolor}\sffamily\fontsize{10.000000}{12.000000}\selectfont \(\displaystyle k\)}%
\end{pgfscope}%
\begin{pgfscope}%
\pgfsetbuttcap%
\pgfsetroundjoin%
\definecolor{currentfill}{rgb}{0.000000,0.000000,0.000000}%
\pgfsetfillcolor{currentfill}%
\pgfsetlinewidth{0.803000pt}%
\definecolor{currentstroke}{rgb}{0.000000,0.000000,0.000000}%
\pgfsetstrokecolor{currentstroke}%
\pgfsetdash{}{0pt}%
\pgfsys@defobject{currentmarker}{\pgfqpoint{-0.048611in}{0.000000in}}{\pgfqpoint{0.000000in}{0.000000in}}{%
\pgfpathmoveto{\pgfqpoint{0.000000in}{0.000000in}}%
\pgfpathlineto{\pgfqpoint{-0.048611in}{0.000000in}}%
\pgfusepath{stroke,fill}%
}%
\begin{pgfscope}%
\pgfsys@transformshift{0.800000in}{0.696000in}%
\pgfsys@useobject{currentmarker}{}%
\end{pgfscope}%
\end{pgfscope}%
\begin{pgfscope}%
\definecolor{textcolor}{rgb}{0.000000,0.000000,0.000000}%
\pgfsetstrokecolor{textcolor}%
\pgfsetfillcolor{textcolor}%
\pgftext[x=0.481898in,y=0.643238in,left,base]{\color{textcolor}\sffamily\fontsize{10.000000}{12.000000}\selectfont 0.0}%
\end{pgfscope}%
\begin{pgfscope}%
\pgfsetbuttcap%
\pgfsetroundjoin%
\definecolor{currentfill}{rgb}{0.000000,0.000000,0.000000}%
\pgfsetfillcolor{currentfill}%
\pgfsetlinewidth{0.803000pt}%
\definecolor{currentstroke}{rgb}{0.000000,0.000000,0.000000}%
\pgfsetstrokecolor{currentstroke}%
\pgfsetdash{}{0pt}%
\pgfsys@defobject{currentmarker}{\pgfqpoint{-0.048611in}{0.000000in}}{\pgfqpoint{0.000000in}{0.000000in}}{%
\pgfpathmoveto{\pgfqpoint{0.000000in}{0.000000in}}%
\pgfpathlineto{\pgfqpoint{-0.048611in}{0.000000in}}%
\pgfusepath{stroke,fill}%
}%
\begin{pgfscope}%
\pgfsys@transformshift{0.800000in}{1.515512in}%
\pgfsys@useobject{currentmarker}{}%
\end{pgfscope}%
\end{pgfscope}%
\begin{pgfscope}%
\definecolor{textcolor}{rgb}{0.000000,0.000000,0.000000}%
\pgfsetstrokecolor{textcolor}%
\pgfsetfillcolor{textcolor}%
\pgftext[x=0.481898in,y=1.462751in,left,base]{\color{textcolor}\sffamily\fontsize{10.000000}{12.000000}\selectfont 0.1}%
\end{pgfscope}%
\begin{pgfscope}%
\pgfsetbuttcap%
\pgfsetroundjoin%
\definecolor{currentfill}{rgb}{0.000000,0.000000,0.000000}%
\pgfsetfillcolor{currentfill}%
\pgfsetlinewidth{0.803000pt}%
\definecolor{currentstroke}{rgb}{0.000000,0.000000,0.000000}%
\pgfsetstrokecolor{currentstroke}%
\pgfsetdash{}{0pt}%
\pgfsys@defobject{currentmarker}{\pgfqpoint{-0.048611in}{0.000000in}}{\pgfqpoint{0.000000in}{0.000000in}}{%
\pgfpathmoveto{\pgfqpoint{0.000000in}{0.000000in}}%
\pgfpathlineto{\pgfqpoint{-0.048611in}{0.000000in}}%
\pgfusepath{stroke,fill}%
}%
\begin{pgfscope}%
\pgfsys@transformshift{0.800000in}{2.335024in}%
\pgfsys@useobject{currentmarker}{}%
\end{pgfscope}%
\end{pgfscope}%
\begin{pgfscope}%
\definecolor{textcolor}{rgb}{0.000000,0.000000,0.000000}%
\pgfsetstrokecolor{textcolor}%
\pgfsetfillcolor{textcolor}%
\pgftext[x=0.481898in,y=2.282263in,left,base]{\color{textcolor}\sffamily\fontsize{10.000000}{12.000000}\selectfont 0.2}%
\end{pgfscope}%
\begin{pgfscope}%
\pgfsetbuttcap%
\pgfsetroundjoin%
\definecolor{currentfill}{rgb}{0.000000,0.000000,0.000000}%
\pgfsetfillcolor{currentfill}%
\pgfsetlinewidth{0.803000pt}%
\definecolor{currentstroke}{rgb}{0.000000,0.000000,0.000000}%
\pgfsetstrokecolor{currentstroke}%
\pgfsetdash{}{0pt}%
\pgfsys@defobject{currentmarker}{\pgfqpoint{-0.048611in}{0.000000in}}{\pgfqpoint{0.000000in}{0.000000in}}{%
\pgfpathmoveto{\pgfqpoint{0.000000in}{0.000000in}}%
\pgfpathlineto{\pgfqpoint{-0.048611in}{0.000000in}}%
\pgfusepath{stroke,fill}%
}%
\begin{pgfscope}%
\pgfsys@transformshift{0.800000in}{3.154537in}%
\pgfsys@useobject{currentmarker}{}%
\end{pgfscope}%
\end{pgfscope}%
\begin{pgfscope}%
\definecolor{textcolor}{rgb}{0.000000,0.000000,0.000000}%
\pgfsetstrokecolor{textcolor}%
\pgfsetfillcolor{textcolor}%
\pgftext[x=0.481898in,y=3.101775in,left,base]{\color{textcolor}\sffamily\fontsize{10.000000}{12.000000}\selectfont 0.3}%
\end{pgfscope}%
\begin{pgfscope}%
\pgfsetbuttcap%
\pgfsetroundjoin%
\definecolor{currentfill}{rgb}{0.000000,0.000000,0.000000}%
\pgfsetfillcolor{currentfill}%
\pgfsetlinewidth{0.803000pt}%
\definecolor{currentstroke}{rgb}{0.000000,0.000000,0.000000}%
\pgfsetstrokecolor{currentstroke}%
\pgfsetdash{}{0pt}%
\pgfsys@defobject{currentmarker}{\pgfqpoint{-0.048611in}{0.000000in}}{\pgfqpoint{0.000000in}{0.000000in}}{%
\pgfpathmoveto{\pgfqpoint{0.000000in}{0.000000in}}%
\pgfpathlineto{\pgfqpoint{-0.048611in}{0.000000in}}%
\pgfusepath{stroke,fill}%
}%
\begin{pgfscope}%
\pgfsys@transformshift{0.800000in}{3.974049in}%
\pgfsys@useobject{currentmarker}{}%
\end{pgfscope}%
\end{pgfscope}%
\begin{pgfscope}%
\definecolor{textcolor}{rgb}{0.000000,0.000000,0.000000}%
\pgfsetstrokecolor{textcolor}%
\pgfsetfillcolor{textcolor}%
\pgftext[x=0.481898in,y=3.921287in,left,base]{\color{textcolor}\sffamily\fontsize{10.000000}{12.000000}\selectfont 0.4}%
\end{pgfscope}%
\begin{pgfscope}%
\definecolor{textcolor}{rgb}{0.000000,0.000000,0.000000}%
\pgfsetstrokecolor{textcolor}%
\pgfsetfillcolor{textcolor}%
\pgftext[x=0.426343in,y=2.376000in,,bottom,rotate=90.000000]{\color{textcolor}\sffamily\fontsize{10.000000}{12.000000}\selectfont Number of GMRES iterations}%
\end{pgfscope}%
\begin{pgfscope}%
\pgfpathrectangle{\pgfqpoint{0.800000in}{0.528000in}}{\pgfqpoint{4.960000in}{3.696000in}}%
\pgfusepath{clip}%
\pgfsetbuttcap%
\pgfsetroundjoin%
\definecolor{currentfill}{rgb}{0.000000,0.000000,0.000000}%
\pgfsetfillcolor{currentfill}%
\pgfsetlinewidth{1.003750pt}%
\definecolor{currentstroke}{rgb}{0.000000,0.000000,0.000000}%
\pgfsetstrokecolor{currentstroke}%
\pgfsetdash{}{0pt}%
\pgfsys@defobject{currentmarker}{\pgfqpoint{-0.041667in}{-0.041667in}}{\pgfqpoint{0.041667in}{0.041667in}}{%
\pgfpathmoveto{\pgfqpoint{0.000000in}{-0.041667in}}%
\pgfpathcurveto{\pgfqpoint{0.011050in}{-0.041667in}}{\pgfqpoint{0.021649in}{-0.037276in}}{\pgfqpoint{0.029463in}{-0.029463in}}%
\pgfpathcurveto{\pgfqpoint{0.037276in}{-0.021649in}}{\pgfqpoint{0.041667in}{-0.011050in}}{\pgfqpoint{0.041667in}{0.000000in}}%
\pgfpathcurveto{\pgfqpoint{0.041667in}{0.011050in}}{\pgfqpoint{0.037276in}{0.021649in}}{\pgfqpoint{0.029463in}{0.029463in}}%
\pgfpathcurveto{\pgfqpoint{0.021649in}{0.037276in}}{\pgfqpoint{0.011050in}{0.041667in}}{\pgfqpoint{0.000000in}{0.041667in}}%
\pgfpathcurveto{\pgfqpoint{-0.011050in}{0.041667in}}{\pgfqpoint{-0.021649in}{0.037276in}}{\pgfqpoint{-0.029463in}{0.029463in}}%
\pgfpathcurveto{\pgfqpoint{-0.037276in}{0.021649in}}{\pgfqpoint{-0.041667in}{0.011050in}}{\pgfqpoint{-0.041667in}{0.000000in}}%
\pgfpathcurveto{\pgfqpoint{-0.041667in}{-0.011050in}}{\pgfqpoint{-0.037276in}{-0.021649in}}{\pgfqpoint{-0.029463in}{-0.029463in}}%
\pgfpathcurveto{\pgfqpoint{-0.021649in}{-0.037276in}}{\pgfqpoint{-0.011050in}{-0.041667in}}{\pgfqpoint{0.000000in}{-0.041667in}}%
\pgfpathclose%
\pgfusepath{stroke,fill}%
}%
\begin{pgfscope}%
\pgfsys@transformshift{1.025455in}{4.056000in}%
\pgfsys@useobject{currentmarker}{}%
\end{pgfscope}%
\end{pgfscope}%
\begin{pgfscope}%
\pgfpathrectangle{\pgfqpoint{0.800000in}{0.528000in}}{\pgfqpoint{4.960000in}{3.696000in}}%
\pgfusepath{clip}%
\pgfsetbuttcap%
\pgfsetroundjoin%
\definecolor{currentfill}{rgb}{0.000000,0.000000,0.000000}%
\pgfsetfillcolor{currentfill}%
\pgfsetlinewidth{1.003750pt}%
\definecolor{currentstroke}{rgb}{0.000000,0.000000,0.000000}%
\pgfsetstrokecolor{currentstroke}%
\pgfsetdash{}{0pt}%
\pgfsys@defobject{currentmarker}{\pgfqpoint{-0.041667in}{-0.041667in}}{\pgfqpoint{0.041667in}{0.041667in}}{%
\pgfpathmoveto{\pgfqpoint{0.000000in}{-0.041667in}}%
\pgfpathcurveto{\pgfqpoint{0.011050in}{-0.041667in}}{\pgfqpoint{0.021649in}{-0.037276in}}{\pgfqpoint{0.029463in}{-0.029463in}}%
\pgfpathcurveto{\pgfqpoint{0.037276in}{-0.021649in}}{\pgfqpoint{0.041667in}{-0.011050in}}{\pgfqpoint{0.041667in}{0.000000in}}%
\pgfpathcurveto{\pgfqpoint{0.041667in}{0.011050in}}{\pgfqpoint{0.037276in}{0.021649in}}{\pgfqpoint{0.029463in}{0.029463in}}%
\pgfpathcurveto{\pgfqpoint{0.021649in}{0.037276in}}{\pgfqpoint{0.011050in}{0.041667in}}{\pgfqpoint{0.000000in}{0.041667in}}%
\pgfpathcurveto{\pgfqpoint{-0.011050in}{0.041667in}}{\pgfqpoint{-0.021649in}{0.037276in}}{\pgfqpoint{-0.029463in}{0.029463in}}%
\pgfpathcurveto{\pgfqpoint{-0.037276in}{0.021649in}}{\pgfqpoint{-0.041667in}{0.011050in}}{\pgfqpoint{-0.041667in}{0.000000in}}%
\pgfpathcurveto{\pgfqpoint{-0.041667in}{-0.011050in}}{\pgfqpoint{-0.037276in}{-0.021649in}}{\pgfqpoint{-0.029463in}{-0.029463in}}%
\pgfpathcurveto{\pgfqpoint{-0.021649in}{-0.037276in}}{\pgfqpoint{-0.011050in}{-0.041667in}}{\pgfqpoint{0.000000in}{-0.041667in}}%
\pgfpathclose%
\pgfusepath{stroke,fill}%
}%
\begin{pgfscope}%
\pgfsys@transformshift{2.528485in}{0.933659in}%
\pgfsys@useobject{currentmarker}{}%
\end{pgfscope}%
\end{pgfscope}%
\begin{pgfscope}%
\pgfpathrectangle{\pgfqpoint{0.800000in}{0.528000in}}{\pgfqpoint{4.960000in}{3.696000in}}%
\pgfusepath{clip}%
\pgfsetbuttcap%
\pgfsetroundjoin%
\definecolor{currentfill}{rgb}{0.000000,0.000000,0.000000}%
\pgfsetfillcolor{currentfill}%
\pgfsetlinewidth{1.003750pt}%
\definecolor{currentstroke}{rgb}{0.000000,0.000000,0.000000}%
\pgfsetstrokecolor{currentstroke}%
\pgfsetdash{}{0pt}%
\pgfsys@defobject{currentmarker}{\pgfqpoint{-0.041667in}{-0.041667in}}{\pgfqpoint{0.041667in}{0.041667in}}{%
\pgfpathmoveto{\pgfqpoint{0.000000in}{-0.041667in}}%
\pgfpathcurveto{\pgfqpoint{0.011050in}{-0.041667in}}{\pgfqpoint{0.021649in}{-0.037276in}}{\pgfqpoint{0.029463in}{-0.029463in}}%
\pgfpathcurveto{\pgfqpoint{0.037276in}{-0.021649in}}{\pgfqpoint{0.041667in}{-0.011050in}}{\pgfqpoint{0.041667in}{0.000000in}}%
\pgfpathcurveto{\pgfqpoint{0.041667in}{0.011050in}}{\pgfqpoint{0.037276in}{0.021649in}}{\pgfqpoint{0.029463in}{0.029463in}}%
\pgfpathcurveto{\pgfqpoint{0.021649in}{0.037276in}}{\pgfqpoint{0.011050in}{0.041667in}}{\pgfqpoint{0.000000in}{0.041667in}}%
\pgfpathcurveto{\pgfqpoint{-0.011050in}{0.041667in}}{\pgfqpoint{-0.021649in}{0.037276in}}{\pgfqpoint{-0.029463in}{0.029463in}}%
\pgfpathcurveto{\pgfqpoint{-0.037276in}{0.021649in}}{\pgfqpoint{-0.041667in}{0.011050in}}{\pgfqpoint{-0.041667in}{0.000000in}}%
\pgfpathcurveto{\pgfqpoint{-0.041667in}{-0.011050in}}{\pgfqpoint{-0.037276in}{-0.021649in}}{\pgfqpoint{-0.029463in}{-0.029463in}}%
\pgfpathcurveto{\pgfqpoint{-0.021649in}{-0.037276in}}{\pgfqpoint{-0.011050in}{-0.041667in}}{\pgfqpoint{0.000000in}{-0.041667in}}%
\pgfpathclose%
\pgfusepath{stroke,fill}%
}%
\begin{pgfscope}%
\pgfsys@transformshift{4.031515in}{0.696000in}%
\pgfsys@useobject{currentmarker}{}%
\end{pgfscope}%
\end{pgfscope}%
\begin{pgfscope}%
\pgfpathrectangle{\pgfqpoint{0.800000in}{0.528000in}}{\pgfqpoint{4.960000in}{3.696000in}}%
\pgfusepath{clip}%
\pgfsetbuttcap%
\pgfsetroundjoin%
\definecolor{currentfill}{rgb}{0.000000,0.000000,0.000000}%
\pgfsetfillcolor{currentfill}%
\pgfsetlinewidth{1.003750pt}%
\definecolor{currentstroke}{rgb}{0.000000,0.000000,0.000000}%
\pgfsetstrokecolor{currentstroke}%
\pgfsetdash{}{0pt}%
\pgfsys@defobject{currentmarker}{\pgfqpoint{-0.041667in}{-0.041667in}}{\pgfqpoint{0.041667in}{0.041667in}}{%
\pgfpathmoveto{\pgfqpoint{0.000000in}{-0.041667in}}%
\pgfpathcurveto{\pgfqpoint{0.011050in}{-0.041667in}}{\pgfqpoint{0.021649in}{-0.037276in}}{\pgfqpoint{0.029463in}{-0.029463in}}%
\pgfpathcurveto{\pgfqpoint{0.037276in}{-0.021649in}}{\pgfqpoint{0.041667in}{-0.011050in}}{\pgfqpoint{0.041667in}{0.000000in}}%
\pgfpathcurveto{\pgfqpoint{0.041667in}{0.011050in}}{\pgfqpoint{0.037276in}{0.021649in}}{\pgfqpoint{0.029463in}{0.029463in}}%
\pgfpathcurveto{\pgfqpoint{0.021649in}{0.037276in}}{\pgfqpoint{0.011050in}{0.041667in}}{\pgfqpoint{0.000000in}{0.041667in}}%
\pgfpathcurveto{\pgfqpoint{-0.011050in}{0.041667in}}{\pgfqpoint{-0.021649in}{0.037276in}}{\pgfqpoint{-0.029463in}{0.029463in}}%
\pgfpathcurveto{\pgfqpoint{-0.037276in}{0.021649in}}{\pgfqpoint{-0.041667in}{0.011050in}}{\pgfqpoint{-0.041667in}{0.000000in}}%
\pgfpathcurveto{\pgfqpoint{-0.041667in}{-0.011050in}}{\pgfqpoint{-0.037276in}{-0.021649in}}{\pgfqpoint{-0.029463in}{-0.029463in}}%
\pgfpathcurveto{\pgfqpoint{-0.021649in}{-0.037276in}}{\pgfqpoint{-0.011050in}{-0.041667in}}{\pgfqpoint{0.000000in}{-0.041667in}}%
\pgfpathclose%
\pgfusepath{stroke,fill}%
}%
\begin{pgfscope}%
\pgfsys@transformshift{5.534545in}{0.696000in}%
\pgfsys@useobject{currentmarker}{}%
\end{pgfscope}%
\end{pgfscope}%
\begin{pgfscope}%
\pgfsetrectcap%
\pgfsetmiterjoin%
\pgfsetlinewidth{0.803000pt}%
\definecolor{currentstroke}{rgb}{0.000000,0.000000,0.000000}%
\pgfsetstrokecolor{currentstroke}%
\pgfsetdash{}{0pt}%
\pgfpathmoveto{\pgfqpoint{0.800000in}{0.528000in}}%
\pgfpathlineto{\pgfqpoint{0.800000in}{4.224000in}}%
\pgfusepath{stroke}%
\end{pgfscope}%
\begin{pgfscope}%
\pgfsetrectcap%
\pgfsetmiterjoin%
\pgfsetlinewidth{0.803000pt}%
\definecolor{currentstroke}{rgb}{0.000000,0.000000,0.000000}%
\pgfsetstrokecolor{currentstroke}%
\pgfsetdash{}{0pt}%
\pgfpathmoveto{\pgfqpoint{5.760000in}{0.528000in}}%
\pgfpathlineto{\pgfqpoint{5.760000in}{4.224000in}}%
\pgfusepath{stroke}%
\end{pgfscope}%
\begin{pgfscope}%
\pgfsetrectcap%
\pgfsetmiterjoin%
\pgfsetlinewidth{0.803000pt}%
\definecolor{currentstroke}{rgb}{0.000000,0.000000,0.000000}%
\pgfsetstrokecolor{currentstroke}%
\pgfsetdash{}{0pt}%
\pgfpathmoveto{\pgfqpoint{0.800000in}{0.528000in}}%
\pgfpathlineto{\pgfqpoint{5.760000in}{0.528000in}}%
\pgfusepath{stroke}%
\end{pgfscope}%
\begin{pgfscope}%
\pgfsetrectcap%
\pgfsetmiterjoin%
\pgfsetlinewidth{0.803000pt}%
\definecolor{currentstroke}{rgb}{0.000000,0.000000,0.000000}%
\pgfsetstrokecolor{currentstroke}%
\pgfsetdash{}{0pt}%
\pgfpathmoveto{\pgfqpoint{0.800000in}{4.224000in}}%
\pgfpathlineto{\pgfqpoint{5.760000in}{4.224000in}}%
\pgfusepath{stroke}%
\end{pgfscope}%
\end{pgfpicture}%
\makeatother%
\endgroup%

\caption{The empirical probability that $\GMRES{\eps}{\no}{\nt}\leq 12$ for $\sigma = 1.$\label{fig:prob-plot-0.0}}
\end{subfigure}

\begin{subfigure}{\textwidth}
    \centering
%% Creator: Matplotlib, PGF backend
%%
%% To include the figure in your LaTeX document, write
%%   \input{<filename>.pgf}
%%
%% Make sure the required packages are loaded in your preamble
%%   \usepackage{pgf}
%%
%% Figures using additional raster images can only be included by \input if
%% they are in the same directory as the main LaTeX file. For loading figures
%% from other directories you can use the `import` package
%%   \usepackage{import}
%% and then include the figures with
%%   \import{<path to file>}{<filename>.pgf}
%%
%% Matplotlib used the following preamble
%%   \usepackage{fontspec}
%%   \setmainfont{DejaVuSerif.ttf}[Path=/home/owen/progs/firedrake-complex/firedrake/lib/python3.5/site-packages/matplotlib/mpl-data/fonts/ttf/]
%%   \setsansfont{DejaVuSans.ttf}[Path=/home/owen/progs/firedrake-complex/firedrake/lib/python3.5/site-packages/matplotlib/mpl-data/fonts/ttf/]
%%   \setmonofont{DejaVuSansMono.ttf}[Path=/home/owen/progs/firedrake-complex/firedrake/lib/python3.5/site-packages/matplotlib/mpl-data/fonts/ttf/]
%%
\begingroup%
\makeatletter%
\begin{pgfpicture}%
\pgfpathrectangle{\pgfpointorigin}{\pgfqpoint{6.400000in}{4.800000in}}%
\pgfusepath{use as bounding box, clip}%
\begin{pgfscope}%
\pgfsetbuttcap%
\pgfsetmiterjoin%
\definecolor{currentfill}{rgb}{1.000000,1.000000,1.000000}%
\pgfsetfillcolor{currentfill}%
\pgfsetlinewidth{0.000000pt}%
\definecolor{currentstroke}{rgb}{1.000000,1.000000,1.000000}%
\pgfsetstrokecolor{currentstroke}%
\pgfsetdash{}{0pt}%
\pgfpathmoveto{\pgfqpoint{0.000000in}{0.000000in}}%
\pgfpathlineto{\pgfqpoint{6.400000in}{0.000000in}}%
\pgfpathlineto{\pgfqpoint{6.400000in}{4.800000in}}%
\pgfpathlineto{\pgfqpoint{0.000000in}{4.800000in}}%
\pgfpathclose%
\pgfusepath{fill}%
\end{pgfscope}%
\begin{pgfscope}%
\pgfsetbuttcap%
\pgfsetmiterjoin%
\definecolor{currentfill}{rgb}{1.000000,1.000000,1.000000}%
\pgfsetfillcolor{currentfill}%
\pgfsetlinewidth{0.000000pt}%
\definecolor{currentstroke}{rgb}{0.000000,0.000000,0.000000}%
\pgfsetstrokecolor{currentstroke}%
\pgfsetstrokeopacity{0.000000}%
\pgfsetdash{}{0pt}%
\pgfpathmoveto{\pgfqpoint{0.800000in}{0.528000in}}%
\pgfpathlineto{\pgfqpoint{5.760000in}{0.528000in}}%
\pgfpathlineto{\pgfqpoint{5.760000in}{4.224000in}}%
\pgfpathlineto{\pgfqpoint{0.800000in}{4.224000in}}%
\pgfpathclose%
\pgfusepath{fill}%
\end{pgfscope}%
\begin{pgfscope}%
\pgfsetbuttcap%
\pgfsetroundjoin%
\definecolor{currentfill}{rgb}{0.000000,0.000000,0.000000}%
\pgfsetfillcolor{currentfill}%
\pgfsetlinewidth{0.803000pt}%
\definecolor{currentstroke}{rgb}{0.000000,0.000000,0.000000}%
\pgfsetstrokecolor{currentstroke}%
\pgfsetdash{}{0pt}%
\pgfsys@defobject{currentmarker}{\pgfqpoint{0.000000in}{-0.048611in}}{\pgfqpoint{0.000000in}{0.000000in}}{%
\pgfpathmoveto{\pgfqpoint{0.000000in}{0.000000in}}%
\pgfpathlineto{\pgfqpoint{0.000000in}{-0.048611in}}%
\pgfusepath{stroke,fill}%
}%
\begin{pgfscope}%
\pgfsys@transformshift{1.025455in}{0.528000in}%
\pgfsys@useobject{currentmarker}{}%
\end{pgfscope}%
\end{pgfscope}%
\begin{pgfscope}%
\definecolor{textcolor}{rgb}{0.000000,0.000000,0.000000}%
\pgfsetstrokecolor{textcolor}%
\pgfsetfillcolor{textcolor}%
\pgftext[x=1.025455in,y=0.430778in,,top]{\color{textcolor}\sffamily\fontsize{10.000000}{12.000000}\selectfont 10}%
\end{pgfscope}%
\begin{pgfscope}%
\pgfsetbuttcap%
\pgfsetroundjoin%
\definecolor{currentfill}{rgb}{0.000000,0.000000,0.000000}%
\pgfsetfillcolor{currentfill}%
\pgfsetlinewidth{0.803000pt}%
\definecolor{currentstroke}{rgb}{0.000000,0.000000,0.000000}%
\pgfsetstrokecolor{currentstroke}%
\pgfsetdash{}{0pt}%
\pgfsys@defobject{currentmarker}{\pgfqpoint{0.000000in}{-0.048611in}}{\pgfqpoint{0.000000in}{0.000000in}}{%
\pgfpathmoveto{\pgfqpoint{0.000000in}{0.000000in}}%
\pgfpathlineto{\pgfqpoint{0.000000in}{-0.048611in}}%
\pgfusepath{stroke,fill}%
}%
\begin{pgfscope}%
\pgfsys@transformshift{2.528485in}{0.528000in}%
\pgfsys@useobject{currentmarker}{}%
\end{pgfscope}%
\end{pgfscope}%
\begin{pgfscope}%
\definecolor{textcolor}{rgb}{0.000000,0.000000,0.000000}%
\pgfsetstrokecolor{textcolor}%
\pgfsetfillcolor{textcolor}%
\pgftext[x=2.528485in,y=0.430778in,,top]{\color{textcolor}\sffamily\fontsize{10.000000}{12.000000}\selectfont 20}%
\end{pgfscope}%
\begin{pgfscope}%
\pgfsetbuttcap%
\pgfsetroundjoin%
\definecolor{currentfill}{rgb}{0.000000,0.000000,0.000000}%
\pgfsetfillcolor{currentfill}%
\pgfsetlinewidth{0.803000pt}%
\definecolor{currentstroke}{rgb}{0.000000,0.000000,0.000000}%
\pgfsetstrokecolor{currentstroke}%
\pgfsetdash{}{0pt}%
\pgfsys@defobject{currentmarker}{\pgfqpoint{0.000000in}{-0.048611in}}{\pgfqpoint{0.000000in}{0.000000in}}{%
\pgfpathmoveto{\pgfqpoint{0.000000in}{0.000000in}}%
\pgfpathlineto{\pgfqpoint{0.000000in}{-0.048611in}}%
\pgfusepath{stroke,fill}%
}%
\begin{pgfscope}%
\pgfsys@transformshift{4.031515in}{0.528000in}%
\pgfsys@useobject{currentmarker}{}%
\end{pgfscope}%
\end{pgfscope}%
\begin{pgfscope}%
\definecolor{textcolor}{rgb}{0.000000,0.000000,0.000000}%
\pgfsetstrokecolor{textcolor}%
\pgfsetfillcolor{textcolor}%
\pgftext[x=4.031515in,y=0.430778in,,top]{\color{textcolor}\sffamily\fontsize{10.000000}{12.000000}\selectfont 30}%
\end{pgfscope}%
\begin{pgfscope}%
\pgfsetbuttcap%
\pgfsetroundjoin%
\definecolor{currentfill}{rgb}{0.000000,0.000000,0.000000}%
\pgfsetfillcolor{currentfill}%
\pgfsetlinewidth{0.803000pt}%
\definecolor{currentstroke}{rgb}{0.000000,0.000000,0.000000}%
\pgfsetstrokecolor{currentstroke}%
\pgfsetdash{}{0pt}%
\pgfsys@defobject{currentmarker}{\pgfqpoint{0.000000in}{-0.048611in}}{\pgfqpoint{0.000000in}{0.000000in}}{%
\pgfpathmoveto{\pgfqpoint{0.000000in}{0.000000in}}%
\pgfpathlineto{\pgfqpoint{0.000000in}{-0.048611in}}%
\pgfusepath{stroke,fill}%
}%
\begin{pgfscope}%
\pgfsys@transformshift{5.534545in}{0.528000in}%
\pgfsys@useobject{currentmarker}{}%
\end{pgfscope}%
\end{pgfscope}%
\begin{pgfscope}%
\definecolor{textcolor}{rgb}{0.000000,0.000000,0.000000}%
\pgfsetstrokecolor{textcolor}%
\pgfsetfillcolor{textcolor}%
\pgftext[x=5.534545in,y=0.430778in,,top]{\color{textcolor}\sffamily\fontsize{10.000000}{12.000000}\selectfont 40}%
\end{pgfscope}%
\begin{pgfscope}%
\definecolor{textcolor}{rgb}{0.000000,0.000000,0.000000}%
\pgfsetstrokecolor{textcolor}%
\pgfsetfillcolor{textcolor}%
\pgftext[x=3.280000in,y=0.240809in,,top]{\color{textcolor}\sffamily\fontsize{10.000000}{12.000000}\selectfont \(\displaystyle k\)}%
\end{pgfscope}%
\begin{pgfscope}%
\pgfsetbuttcap%
\pgfsetroundjoin%
\definecolor{currentfill}{rgb}{0.000000,0.000000,0.000000}%
\pgfsetfillcolor{currentfill}%
\pgfsetlinewidth{0.803000pt}%
\definecolor{currentstroke}{rgb}{0.000000,0.000000,0.000000}%
\pgfsetstrokecolor{currentstroke}%
\pgfsetdash{}{0pt}%
\pgfsys@defobject{currentmarker}{\pgfqpoint{-0.048611in}{0.000000in}}{\pgfqpoint{0.000000in}{0.000000in}}{%
\pgfpathmoveto{\pgfqpoint{0.000000in}{0.000000in}}%
\pgfpathlineto{\pgfqpoint{-0.048611in}{0.000000in}}%
\pgfusepath{stroke,fill}%
}%
\begin{pgfscope}%
\pgfsys@transformshift{0.800000in}{0.682922in}%
\pgfsys@useobject{currentmarker}{}%
\end{pgfscope}%
\end{pgfscope}%
\begin{pgfscope}%
\definecolor{textcolor}{rgb}{0.000000,0.000000,0.000000}%
\pgfsetstrokecolor{textcolor}%
\pgfsetfillcolor{textcolor}%
\pgftext[x=0.305168in,y=0.630161in,left,base]{\color{textcolor}\sffamily\fontsize{10.000000}{12.000000}\selectfont 0.992}%
\end{pgfscope}%
\begin{pgfscope}%
\pgfsetbuttcap%
\pgfsetroundjoin%
\definecolor{currentfill}{rgb}{0.000000,0.000000,0.000000}%
\pgfsetfillcolor{currentfill}%
\pgfsetlinewidth{0.803000pt}%
\definecolor{currentstroke}{rgb}{0.000000,0.000000,0.000000}%
\pgfsetstrokecolor{currentstroke}%
\pgfsetdash{}{0pt}%
\pgfsys@defobject{currentmarker}{\pgfqpoint{-0.048611in}{0.000000in}}{\pgfqpoint{0.000000in}{0.000000in}}{%
\pgfpathmoveto{\pgfqpoint{0.000000in}{0.000000in}}%
\pgfpathlineto{\pgfqpoint{-0.048611in}{0.000000in}}%
\pgfusepath{stroke,fill}%
}%
\begin{pgfscope}%
\pgfsys@transformshift{0.800000in}{1.568192in}%
\pgfsys@useobject{currentmarker}{}%
\end{pgfscope}%
\end{pgfscope}%
\begin{pgfscope}%
\definecolor{textcolor}{rgb}{0.000000,0.000000,0.000000}%
\pgfsetstrokecolor{textcolor}%
\pgfsetfillcolor{textcolor}%
\pgftext[x=0.305168in,y=1.515430in,left,base]{\color{textcolor}\sffamily\fontsize{10.000000}{12.000000}\selectfont 0.994}%
\end{pgfscope}%
\begin{pgfscope}%
\pgfsetbuttcap%
\pgfsetroundjoin%
\definecolor{currentfill}{rgb}{0.000000,0.000000,0.000000}%
\pgfsetfillcolor{currentfill}%
\pgfsetlinewidth{0.803000pt}%
\definecolor{currentstroke}{rgb}{0.000000,0.000000,0.000000}%
\pgfsetstrokecolor{currentstroke}%
\pgfsetdash{}{0pt}%
\pgfsys@defobject{currentmarker}{\pgfqpoint{-0.048611in}{0.000000in}}{\pgfqpoint{0.000000in}{0.000000in}}{%
\pgfpathmoveto{\pgfqpoint{0.000000in}{0.000000in}}%
\pgfpathlineto{\pgfqpoint{-0.048611in}{0.000000in}}%
\pgfusepath{stroke,fill}%
}%
\begin{pgfscope}%
\pgfsys@transformshift{0.800000in}{2.453461in}%
\pgfsys@useobject{currentmarker}{}%
\end{pgfscope}%
\end{pgfscope}%
\begin{pgfscope}%
\definecolor{textcolor}{rgb}{0.000000,0.000000,0.000000}%
\pgfsetstrokecolor{textcolor}%
\pgfsetfillcolor{textcolor}%
\pgftext[x=0.305168in,y=2.400700in,left,base]{\color{textcolor}\sffamily\fontsize{10.000000}{12.000000}\selectfont 0.996}%
\end{pgfscope}%
\begin{pgfscope}%
\pgfsetbuttcap%
\pgfsetroundjoin%
\definecolor{currentfill}{rgb}{0.000000,0.000000,0.000000}%
\pgfsetfillcolor{currentfill}%
\pgfsetlinewidth{0.803000pt}%
\definecolor{currentstroke}{rgb}{0.000000,0.000000,0.000000}%
\pgfsetstrokecolor{currentstroke}%
\pgfsetdash{}{0pt}%
\pgfsys@defobject{currentmarker}{\pgfqpoint{-0.048611in}{0.000000in}}{\pgfqpoint{0.000000in}{0.000000in}}{%
\pgfpathmoveto{\pgfqpoint{0.000000in}{0.000000in}}%
\pgfpathlineto{\pgfqpoint{-0.048611in}{0.000000in}}%
\pgfusepath{stroke,fill}%
}%
\begin{pgfscope}%
\pgfsys@transformshift{0.800000in}{3.338731in}%
\pgfsys@useobject{currentmarker}{}%
\end{pgfscope}%
\end{pgfscope}%
\begin{pgfscope}%
\definecolor{textcolor}{rgb}{0.000000,0.000000,0.000000}%
\pgfsetstrokecolor{textcolor}%
\pgfsetfillcolor{textcolor}%
\pgftext[x=0.305168in,y=3.285969in,left,base]{\color{textcolor}\sffamily\fontsize{10.000000}{12.000000}\selectfont 0.998}%
\end{pgfscope}%
\begin{pgfscope}%
\pgfsetbuttcap%
\pgfsetroundjoin%
\definecolor{currentfill}{rgb}{0.000000,0.000000,0.000000}%
\pgfsetfillcolor{currentfill}%
\pgfsetlinewidth{0.803000pt}%
\definecolor{currentstroke}{rgb}{0.000000,0.000000,0.000000}%
\pgfsetstrokecolor{currentstroke}%
\pgfsetdash{}{0pt}%
\pgfsys@defobject{currentmarker}{\pgfqpoint{-0.048611in}{0.000000in}}{\pgfqpoint{0.000000in}{0.000000in}}{%
\pgfpathmoveto{\pgfqpoint{0.000000in}{0.000000in}}%
\pgfpathlineto{\pgfqpoint{-0.048611in}{0.000000in}}%
\pgfusepath{stroke,fill}%
}%
\begin{pgfscope}%
\pgfsys@transformshift{0.800000in}{4.224000in}%
\pgfsys@useobject{currentmarker}{}%
\end{pgfscope}%
\end{pgfscope}%
\begin{pgfscope}%
\definecolor{textcolor}{rgb}{0.000000,0.000000,0.000000}%
\pgfsetstrokecolor{textcolor}%
\pgfsetfillcolor{textcolor}%
\pgftext[x=0.305168in,y=4.171238in,left,base]{\color{textcolor}\sffamily\fontsize{10.000000}{12.000000}\selectfont 1.000}%
\end{pgfscope}%
\begin{pgfscope}%
\definecolor{textcolor}{rgb}{0.000000,0.000000,0.000000}%
\pgfsetstrokecolor{textcolor}%
\pgfsetfillcolor{textcolor}%
\pgftext[x=0.249612in,y=2.376000in,,bottom,rotate=90.000000]{\color{textcolor}\sffamily\fontsize{10.000000}{12.000000}\selectfont Number of GMRES iterations}%
\end{pgfscope}%
\begin{pgfscope}%
\pgfpathrectangle{\pgfqpoint{0.800000in}{0.528000in}}{\pgfqpoint{4.960000in}{3.696000in}}%
\pgfusepath{clip}%
\pgfsetbuttcap%
\pgfsetroundjoin%
\definecolor{currentfill}{rgb}{0.000000,0.000000,0.000000}%
\pgfsetfillcolor{currentfill}%
\pgfsetlinewidth{1.003750pt}%
\definecolor{currentstroke}{rgb}{0.000000,0.000000,0.000000}%
\pgfsetstrokecolor{currentstroke}%
\pgfsetdash{}{0pt}%
\pgfsys@defobject{currentmarker}{\pgfqpoint{-0.041667in}{-0.041667in}}{\pgfqpoint{0.041667in}{0.041667in}}{%
\pgfpathmoveto{\pgfqpoint{0.000000in}{-0.041667in}}%
\pgfpathcurveto{\pgfqpoint{0.011050in}{-0.041667in}}{\pgfqpoint{0.021649in}{-0.037276in}}{\pgfqpoint{0.029463in}{-0.029463in}}%
\pgfpathcurveto{\pgfqpoint{0.037276in}{-0.021649in}}{\pgfqpoint{0.041667in}{-0.011050in}}{\pgfqpoint{0.041667in}{0.000000in}}%
\pgfpathcurveto{\pgfqpoint{0.041667in}{0.011050in}}{\pgfqpoint{0.037276in}{0.021649in}}{\pgfqpoint{0.029463in}{0.029463in}}%
\pgfpathcurveto{\pgfqpoint{0.021649in}{0.037276in}}{\pgfqpoint{0.011050in}{0.041667in}}{\pgfqpoint{0.000000in}{0.041667in}}%
\pgfpathcurveto{\pgfqpoint{-0.011050in}{0.041667in}}{\pgfqpoint{-0.021649in}{0.037276in}}{\pgfqpoint{-0.029463in}{0.029463in}}%
\pgfpathcurveto{\pgfqpoint{-0.037276in}{0.021649in}}{\pgfqpoint{-0.041667in}{0.011050in}}{\pgfqpoint{-0.041667in}{0.000000in}}%
\pgfpathcurveto{\pgfqpoint{-0.041667in}{-0.011050in}}{\pgfqpoint{-0.037276in}{-0.021649in}}{\pgfqpoint{-0.029463in}{-0.029463in}}%
\pgfpathcurveto{\pgfqpoint{-0.021649in}{-0.037276in}}{\pgfqpoint{-0.011050in}{-0.041667in}}{\pgfqpoint{0.000000in}{-0.041667in}}%
\pgfpathclose%
\pgfusepath{stroke,fill}%
}%
\begin{pgfscope}%
\pgfsys@transformshift{1.025455in}{3.781365in}%
\pgfsys@useobject{currentmarker}{}%
\end{pgfscope}%
\end{pgfscope}%
\begin{pgfscope}%
\pgfpathrectangle{\pgfqpoint{0.800000in}{0.528000in}}{\pgfqpoint{4.960000in}{3.696000in}}%
\pgfusepath{clip}%
\pgfsetbuttcap%
\pgfsetroundjoin%
\definecolor{currentfill}{rgb}{0.000000,0.000000,0.000000}%
\pgfsetfillcolor{currentfill}%
\pgfsetlinewidth{1.003750pt}%
\definecolor{currentstroke}{rgb}{0.000000,0.000000,0.000000}%
\pgfsetstrokecolor{currentstroke}%
\pgfsetdash{}{0pt}%
\pgfsys@defobject{currentmarker}{\pgfqpoint{-0.041667in}{-0.041667in}}{\pgfqpoint{0.041667in}{0.041667in}}{%
\pgfpathmoveto{\pgfqpoint{0.000000in}{-0.041667in}}%
\pgfpathcurveto{\pgfqpoint{0.011050in}{-0.041667in}}{\pgfqpoint{0.021649in}{-0.037276in}}{\pgfqpoint{0.029463in}{-0.029463in}}%
\pgfpathcurveto{\pgfqpoint{0.037276in}{-0.021649in}}{\pgfqpoint{0.041667in}{-0.011050in}}{\pgfqpoint{0.041667in}{0.000000in}}%
\pgfpathcurveto{\pgfqpoint{0.041667in}{0.011050in}}{\pgfqpoint{0.037276in}{0.021649in}}{\pgfqpoint{0.029463in}{0.029463in}}%
\pgfpathcurveto{\pgfqpoint{0.021649in}{0.037276in}}{\pgfqpoint{0.011050in}{0.041667in}}{\pgfqpoint{0.000000in}{0.041667in}}%
\pgfpathcurveto{\pgfqpoint{-0.011050in}{0.041667in}}{\pgfqpoint{-0.021649in}{0.037276in}}{\pgfqpoint{-0.029463in}{0.029463in}}%
\pgfpathcurveto{\pgfqpoint{-0.037276in}{0.021649in}}{\pgfqpoint{-0.041667in}{0.011050in}}{\pgfqpoint{-0.041667in}{0.000000in}}%
\pgfpathcurveto{\pgfqpoint{-0.041667in}{-0.011050in}}{\pgfqpoint{-0.037276in}{-0.021649in}}{\pgfqpoint{-0.029463in}{-0.029463in}}%
\pgfpathcurveto{\pgfqpoint{-0.021649in}{-0.037276in}}{\pgfqpoint{-0.011050in}{-0.041667in}}{\pgfqpoint{0.000000in}{-0.041667in}}%
\pgfpathclose%
\pgfusepath{stroke,fill}%
}%
\begin{pgfscope}%
\pgfsys@transformshift{2.528485in}{0.682922in}%
\pgfsys@useobject{currentmarker}{}%
\end{pgfscope}%
\end{pgfscope}%
\begin{pgfscope}%
\pgfpathrectangle{\pgfqpoint{0.800000in}{0.528000in}}{\pgfqpoint{4.960000in}{3.696000in}}%
\pgfusepath{clip}%
\pgfsetbuttcap%
\pgfsetroundjoin%
\definecolor{currentfill}{rgb}{0.000000,0.000000,0.000000}%
\pgfsetfillcolor{currentfill}%
\pgfsetlinewidth{1.003750pt}%
\definecolor{currentstroke}{rgb}{0.000000,0.000000,0.000000}%
\pgfsetstrokecolor{currentstroke}%
\pgfsetdash{}{0pt}%
\pgfsys@defobject{currentmarker}{\pgfqpoint{-0.041667in}{-0.041667in}}{\pgfqpoint{0.041667in}{0.041667in}}{%
\pgfpathmoveto{\pgfqpoint{0.000000in}{-0.041667in}}%
\pgfpathcurveto{\pgfqpoint{0.011050in}{-0.041667in}}{\pgfqpoint{0.021649in}{-0.037276in}}{\pgfqpoint{0.029463in}{-0.029463in}}%
\pgfpathcurveto{\pgfqpoint{0.037276in}{-0.021649in}}{\pgfqpoint{0.041667in}{-0.011050in}}{\pgfqpoint{0.041667in}{0.000000in}}%
\pgfpathcurveto{\pgfqpoint{0.041667in}{0.011050in}}{\pgfqpoint{0.037276in}{0.021649in}}{\pgfqpoint{0.029463in}{0.029463in}}%
\pgfpathcurveto{\pgfqpoint{0.021649in}{0.037276in}}{\pgfqpoint{0.011050in}{0.041667in}}{\pgfqpoint{0.000000in}{0.041667in}}%
\pgfpathcurveto{\pgfqpoint{-0.011050in}{0.041667in}}{\pgfqpoint{-0.021649in}{0.037276in}}{\pgfqpoint{-0.029463in}{0.029463in}}%
\pgfpathcurveto{\pgfqpoint{-0.037276in}{0.021649in}}{\pgfqpoint{-0.041667in}{0.011050in}}{\pgfqpoint{-0.041667in}{0.000000in}}%
\pgfpathcurveto{\pgfqpoint{-0.041667in}{-0.011050in}}{\pgfqpoint{-0.037276in}{-0.021649in}}{\pgfqpoint{-0.029463in}{-0.029463in}}%
\pgfpathcurveto{\pgfqpoint{-0.021649in}{-0.037276in}}{\pgfqpoint{-0.011050in}{-0.041667in}}{\pgfqpoint{0.000000in}{-0.041667in}}%
\pgfpathclose%
\pgfusepath{stroke,fill}%
}%
\begin{pgfscope}%
\pgfsys@transformshift{4.031515in}{0.682922in}%
\pgfsys@useobject{currentmarker}{}%
\end{pgfscope}%
\end{pgfscope}%
\begin{pgfscope}%
\pgfpathrectangle{\pgfqpoint{0.800000in}{0.528000in}}{\pgfqpoint{4.960000in}{3.696000in}}%
\pgfusepath{clip}%
\pgfsetbuttcap%
\pgfsetroundjoin%
\definecolor{currentfill}{rgb}{0.000000,0.000000,0.000000}%
\pgfsetfillcolor{currentfill}%
\pgfsetlinewidth{1.003750pt}%
\definecolor{currentstroke}{rgb}{0.000000,0.000000,0.000000}%
\pgfsetstrokecolor{currentstroke}%
\pgfsetdash{}{0pt}%
\pgfsys@defobject{currentmarker}{\pgfqpoint{-0.041667in}{-0.041667in}}{\pgfqpoint{0.041667in}{0.041667in}}{%
\pgfpathmoveto{\pgfqpoint{0.000000in}{-0.041667in}}%
\pgfpathcurveto{\pgfqpoint{0.011050in}{-0.041667in}}{\pgfqpoint{0.021649in}{-0.037276in}}{\pgfqpoint{0.029463in}{-0.029463in}}%
\pgfpathcurveto{\pgfqpoint{0.037276in}{-0.021649in}}{\pgfqpoint{0.041667in}{-0.011050in}}{\pgfqpoint{0.041667in}{0.000000in}}%
\pgfpathcurveto{\pgfqpoint{0.041667in}{0.011050in}}{\pgfqpoint{0.037276in}{0.021649in}}{\pgfqpoint{0.029463in}{0.029463in}}%
\pgfpathcurveto{\pgfqpoint{0.021649in}{0.037276in}}{\pgfqpoint{0.011050in}{0.041667in}}{\pgfqpoint{0.000000in}{0.041667in}}%
\pgfpathcurveto{\pgfqpoint{-0.011050in}{0.041667in}}{\pgfqpoint{-0.021649in}{0.037276in}}{\pgfqpoint{-0.029463in}{0.029463in}}%
\pgfpathcurveto{\pgfqpoint{-0.037276in}{0.021649in}}{\pgfqpoint{-0.041667in}{0.011050in}}{\pgfqpoint{-0.041667in}{0.000000in}}%
\pgfpathcurveto{\pgfqpoint{-0.041667in}{-0.011050in}}{\pgfqpoint{-0.037276in}{-0.021649in}}{\pgfqpoint{-0.029463in}{-0.029463in}}%
\pgfpathcurveto{\pgfqpoint{-0.021649in}{-0.037276in}}{\pgfqpoint{-0.011050in}{-0.041667in}}{\pgfqpoint{0.000000in}{-0.041667in}}%
\pgfpathclose%
\pgfusepath{stroke,fill}%
}%
\begin{pgfscope}%
\pgfsys@transformshift{5.534545in}{1.568192in}%
\pgfsys@useobject{currentmarker}{}%
\end{pgfscope}%
\end{pgfscope}%
\begin{pgfscope}%
\pgfsetrectcap%
\pgfsetmiterjoin%
\pgfsetlinewidth{0.803000pt}%
\definecolor{currentstroke}{rgb}{0.000000,0.000000,0.000000}%
\pgfsetstrokecolor{currentstroke}%
\pgfsetdash{}{0pt}%
\pgfpathmoveto{\pgfqpoint{0.800000in}{0.528000in}}%
\pgfpathlineto{\pgfqpoint{0.800000in}{4.224000in}}%
\pgfusepath{stroke}%
\end{pgfscope}%
\begin{pgfscope}%
\pgfsetrectcap%
\pgfsetmiterjoin%
\pgfsetlinewidth{0.803000pt}%
\definecolor{currentstroke}{rgb}{0.000000,0.000000,0.000000}%
\pgfsetstrokecolor{currentstroke}%
\pgfsetdash{}{0pt}%
\pgfpathmoveto{\pgfqpoint{5.760000in}{0.528000in}}%
\pgfpathlineto{\pgfqpoint{5.760000in}{4.224000in}}%
\pgfusepath{stroke}%
\end{pgfscope}%
\begin{pgfscope}%
\pgfsetrectcap%
\pgfsetmiterjoin%
\pgfsetlinewidth{0.803000pt}%
\definecolor{currentstroke}{rgb}{0.000000,0.000000,0.000000}%
\pgfsetstrokecolor{currentstroke}%
\pgfsetdash{}{0pt}%
\pgfpathmoveto{\pgfqpoint{0.800000in}{0.528000in}}%
\pgfpathlineto{\pgfqpoint{5.760000in}{0.528000in}}%
\pgfusepath{stroke}%
\end{pgfscope}%
\begin{pgfscope}%
\pgfsetrectcap%
\pgfsetmiterjoin%
\pgfsetlinewidth{0.803000pt}%
\definecolor{currentstroke}{rgb}{0.000000,0.000000,0.000000}%
\pgfsetstrokecolor{currentstroke}%
\pgfsetdash{}{0pt}%
\pgfpathmoveto{\pgfqpoint{0.800000in}{4.224000in}}%
\pgfpathlineto{\pgfqpoint{5.760000in}{4.224000in}}%
\pgfusepath{stroke}%
\end{pgfscope}%
\end{pgfpicture}%
\makeatother%
\endgroup%

\caption{The empirical probability that $\GMRES{\eps}{\no}{\nt}\leq 12$ for $\sigma = 1/k$\label{fig:prob-plot-1.0}}
\end{subfigure}

\begin{subfigure}{\textwidth}
    \centering
%% Creator: Matplotlib, PGF backend
%%
%% To include the figure in your LaTeX document, write
%%   \input{<filename>.pgf}
%%
%% Make sure the required packages are loaded in your preamble
%%   \usepackage{pgf}
%%
%% Figures using additional raster images can only be included by \input if
%% they are in the same directory as the main LaTeX file. For loading figures
%% from other directories you can use the `import` package
%%   \usepackage{import}
%% and then include the figures with
%%   \import{<path to file>}{<filename>.pgf}
%%
%% Matplotlib used the following preamble
%%   \usepackage{fontspec}
%%   \setmainfont{DejaVuSerif.ttf}[Path=/home/owen/progs/firedrake-complex/firedrake/lib/python3.5/site-packages/matplotlib/mpl-data/fonts/ttf/]
%%   \setsansfont{DejaVuSans.ttf}[Path=/home/owen/progs/firedrake-complex/firedrake/lib/python3.5/site-packages/matplotlib/mpl-data/fonts/ttf/]
%%   \setmonofont{DejaVuSansMono.ttf}[Path=/home/owen/progs/firedrake-complex/firedrake/lib/python3.5/site-packages/matplotlib/mpl-data/fonts/ttf/]
%%
\begingroup%
\makeatletter%
\begin{pgfpicture}%
\pgfpathrectangle{\pgfpointorigin}{\pgfqpoint{6.400000in}{4.800000in}}%
\pgfusepath{use as bounding box, clip}%
\begin{pgfscope}%
\pgfsetbuttcap%
\pgfsetmiterjoin%
\definecolor{currentfill}{rgb}{1.000000,1.000000,1.000000}%
\pgfsetfillcolor{currentfill}%
\pgfsetlinewidth{0.000000pt}%
\definecolor{currentstroke}{rgb}{1.000000,1.000000,1.000000}%
\pgfsetstrokecolor{currentstroke}%
\pgfsetdash{}{0pt}%
\pgfpathmoveto{\pgfqpoint{0.000000in}{0.000000in}}%
\pgfpathlineto{\pgfqpoint{6.400000in}{0.000000in}}%
\pgfpathlineto{\pgfqpoint{6.400000in}{4.800000in}}%
\pgfpathlineto{\pgfqpoint{0.000000in}{4.800000in}}%
\pgfpathclose%
\pgfusepath{fill}%
\end{pgfscope}%
\begin{pgfscope}%
\pgfsetbuttcap%
\pgfsetmiterjoin%
\definecolor{currentfill}{rgb}{1.000000,1.000000,1.000000}%
\pgfsetfillcolor{currentfill}%
\pgfsetlinewidth{0.000000pt}%
\definecolor{currentstroke}{rgb}{0.000000,0.000000,0.000000}%
\pgfsetstrokecolor{currentstroke}%
\pgfsetstrokeopacity{0.000000}%
\pgfsetdash{}{0pt}%
\pgfpathmoveto{\pgfqpoint{0.800000in}{0.528000in}}%
\pgfpathlineto{\pgfqpoint{5.760000in}{0.528000in}}%
\pgfpathlineto{\pgfqpoint{5.760000in}{4.224000in}}%
\pgfpathlineto{\pgfqpoint{0.800000in}{4.224000in}}%
\pgfpathclose%
\pgfusepath{fill}%
\end{pgfscope}%
\begin{pgfscope}%
\pgfsetbuttcap%
\pgfsetroundjoin%
\definecolor{currentfill}{rgb}{0.000000,0.000000,0.000000}%
\pgfsetfillcolor{currentfill}%
\pgfsetlinewidth{0.803000pt}%
\definecolor{currentstroke}{rgb}{0.000000,0.000000,0.000000}%
\pgfsetstrokecolor{currentstroke}%
\pgfsetdash{}{0pt}%
\pgfsys@defobject{currentmarker}{\pgfqpoint{0.000000in}{-0.048611in}}{\pgfqpoint{0.000000in}{0.000000in}}{%
\pgfpathmoveto{\pgfqpoint{0.000000in}{0.000000in}}%
\pgfpathlineto{\pgfqpoint{0.000000in}{-0.048611in}}%
\pgfusepath{stroke,fill}%
}%
\begin{pgfscope}%
\pgfsys@transformshift{1.025455in}{0.528000in}%
\pgfsys@useobject{currentmarker}{}%
\end{pgfscope}%
\end{pgfscope}%
\begin{pgfscope}%
\definecolor{textcolor}{rgb}{0.000000,0.000000,0.000000}%
\pgfsetstrokecolor{textcolor}%
\pgfsetfillcolor{textcolor}%
\pgftext[x=1.025455in,y=0.430778in,,top]{\color{textcolor}\sffamily\fontsize{10.000000}{12.000000}\selectfont 10}%
\end{pgfscope}%
\begin{pgfscope}%
\pgfsetbuttcap%
\pgfsetroundjoin%
\definecolor{currentfill}{rgb}{0.000000,0.000000,0.000000}%
\pgfsetfillcolor{currentfill}%
\pgfsetlinewidth{0.803000pt}%
\definecolor{currentstroke}{rgb}{0.000000,0.000000,0.000000}%
\pgfsetstrokecolor{currentstroke}%
\pgfsetdash{}{0pt}%
\pgfsys@defobject{currentmarker}{\pgfqpoint{0.000000in}{-0.048611in}}{\pgfqpoint{0.000000in}{0.000000in}}{%
\pgfpathmoveto{\pgfqpoint{0.000000in}{0.000000in}}%
\pgfpathlineto{\pgfqpoint{0.000000in}{-0.048611in}}%
\pgfusepath{stroke,fill}%
}%
\begin{pgfscope}%
\pgfsys@transformshift{2.528485in}{0.528000in}%
\pgfsys@useobject{currentmarker}{}%
\end{pgfscope}%
\end{pgfscope}%
\begin{pgfscope}%
\definecolor{textcolor}{rgb}{0.000000,0.000000,0.000000}%
\pgfsetstrokecolor{textcolor}%
\pgfsetfillcolor{textcolor}%
\pgftext[x=2.528485in,y=0.430778in,,top]{\color{textcolor}\sffamily\fontsize{10.000000}{12.000000}\selectfont 20}%
\end{pgfscope}%
\begin{pgfscope}%
\pgfsetbuttcap%
\pgfsetroundjoin%
\definecolor{currentfill}{rgb}{0.000000,0.000000,0.000000}%
\pgfsetfillcolor{currentfill}%
\pgfsetlinewidth{0.803000pt}%
\definecolor{currentstroke}{rgb}{0.000000,0.000000,0.000000}%
\pgfsetstrokecolor{currentstroke}%
\pgfsetdash{}{0pt}%
\pgfsys@defobject{currentmarker}{\pgfqpoint{0.000000in}{-0.048611in}}{\pgfqpoint{0.000000in}{0.000000in}}{%
\pgfpathmoveto{\pgfqpoint{0.000000in}{0.000000in}}%
\pgfpathlineto{\pgfqpoint{0.000000in}{-0.048611in}}%
\pgfusepath{stroke,fill}%
}%
\begin{pgfscope}%
\pgfsys@transformshift{4.031515in}{0.528000in}%
\pgfsys@useobject{currentmarker}{}%
\end{pgfscope}%
\end{pgfscope}%
\begin{pgfscope}%
\definecolor{textcolor}{rgb}{0.000000,0.000000,0.000000}%
\pgfsetstrokecolor{textcolor}%
\pgfsetfillcolor{textcolor}%
\pgftext[x=4.031515in,y=0.430778in,,top]{\color{textcolor}\sffamily\fontsize{10.000000}{12.000000}\selectfont 30}%
\end{pgfscope}%
\begin{pgfscope}%
\pgfsetbuttcap%
\pgfsetroundjoin%
\definecolor{currentfill}{rgb}{0.000000,0.000000,0.000000}%
\pgfsetfillcolor{currentfill}%
\pgfsetlinewidth{0.803000pt}%
\definecolor{currentstroke}{rgb}{0.000000,0.000000,0.000000}%
\pgfsetstrokecolor{currentstroke}%
\pgfsetdash{}{0pt}%
\pgfsys@defobject{currentmarker}{\pgfqpoint{0.000000in}{-0.048611in}}{\pgfqpoint{0.000000in}{0.000000in}}{%
\pgfpathmoveto{\pgfqpoint{0.000000in}{0.000000in}}%
\pgfpathlineto{\pgfqpoint{0.000000in}{-0.048611in}}%
\pgfusepath{stroke,fill}%
}%
\begin{pgfscope}%
\pgfsys@transformshift{5.534545in}{0.528000in}%
\pgfsys@useobject{currentmarker}{}%
\end{pgfscope}%
\end{pgfscope}%
\begin{pgfscope}%
\definecolor{textcolor}{rgb}{0.000000,0.000000,0.000000}%
\pgfsetstrokecolor{textcolor}%
\pgfsetfillcolor{textcolor}%
\pgftext[x=5.534545in,y=0.430778in,,top]{\color{textcolor}\sffamily\fontsize{10.000000}{12.000000}\selectfont 40}%
\end{pgfscope}%
\begin{pgfscope}%
\definecolor{textcolor}{rgb}{0.000000,0.000000,0.000000}%
\pgfsetstrokecolor{textcolor}%
\pgfsetfillcolor{textcolor}%
\pgftext[x=3.280000in,y=0.240809in,,top]{\color{textcolor}\sffamily\fontsize{10.000000}{12.000000}\selectfont \(\displaystyle k\)}%
\end{pgfscope}%
\begin{pgfscope}%
\pgfsetbuttcap%
\pgfsetroundjoin%
\definecolor{currentfill}{rgb}{0.000000,0.000000,0.000000}%
\pgfsetfillcolor{currentfill}%
\pgfsetlinewidth{0.803000pt}%
\definecolor{currentstroke}{rgb}{0.000000,0.000000,0.000000}%
\pgfsetstrokecolor{currentstroke}%
\pgfsetdash{}{0pt}%
\pgfsys@defobject{currentmarker}{\pgfqpoint{-0.048611in}{0.000000in}}{\pgfqpoint{0.000000in}{0.000000in}}{%
\pgfpathmoveto{\pgfqpoint{0.000000in}{0.000000in}}%
\pgfpathlineto{\pgfqpoint{-0.048611in}{0.000000in}}%
\pgfusepath{stroke,fill}%
}%
\begin{pgfscope}%
\pgfsys@transformshift{0.800000in}{2.376000in}%
\pgfsys@useobject{currentmarker}{}%
\end{pgfscope}%
\end{pgfscope}%
\begin{pgfscope}%
\definecolor{textcolor}{rgb}{0.000000,0.000000,0.000000}%
\pgfsetstrokecolor{textcolor}%
\pgfsetfillcolor{textcolor}%
\pgftext[x=0.614413in,y=2.323238in,left,base]{\color{textcolor}\sffamily\fontsize{10.000000}{12.000000}\selectfont 1}%
\end{pgfscope}%
\begin{pgfscope}%
\definecolor{textcolor}{rgb}{0.000000,0.000000,0.000000}%
\pgfsetstrokecolor{textcolor}%
\pgfsetfillcolor{textcolor}%
\pgftext[x=0.558857in,y=2.376000in,,bottom,rotate=90.000000]{\color{textcolor}\sffamily\fontsize{10.000000}{12.000000}\selectfont Number of GMRES iterations}%
\end{pgfscope}%
\begin{pgfscope}%
\pgfpathrectangle{\pgfqpoint{0.800000in}{0.528000in}}{\pgfqpoint{4.960000in}{3.696000in}}%
\pgfusepath{clip}%
\pgfsetbuttcap%
\pgfsetroundjoin%
\definecolor{currentfill}{rgb}{0.000000,0.000000,0.000000}%
\pgfsetfillcolor{currentfill}%
\pgfsetlinewidth{1.003750pt}%
\definecolor{currentstroke}{rgb}{0.000000,0.000000,0.000000}%
\pgfsetstrokecolor{currentstroke}%
\pgfsetdash{}{0pt}%
\pgfsys@defobject{currentmarker}{\pgfqpoint{-0.041667in}{-0.041667in}}{\pgfqpoint{0.041667in}{0.041667in}}{%
\pgfpathmoveto{\pgfqpoint{0.000000in}{-0.041667in}}%
\pgfpathcurveto{\pgfqpoint{0.011050in}{-0.041667in}}{\pgfqpoint{0.021649in}{-0.037276in}}{\pgfqpoint{0.029463in}{-0.029463in}}%
\pgfpathcurveto{\pgfqpoint{0.037276in}{-0.021649in}}{\pgfqpoint{0.041667in}{-0.011050in}}{\pgfqpoint{0.041667in}{0.000000in}}%
\pgfpathcurveto{\pgfqpoint{0.041667in}{0.011050in}}{\pgfqpoint{0.037276in}{0.021649in}}{\pgfqpoint{0.029463in}{0.029463in}}%
\pgfpathcurveto{\pgfqpoint{0.021649in}{0.037276in}}{\pgfqpoint{0.011050in}{0.041667in}}{\pgfqpoint{0.000000in}{0.041667in}}%
\pgfpathcurveto{\pgfqpoint{-0.011050in}{0.041667in}}{\pgfqpoint{-0.021649in}{0.037276in}}{\pgfqpoint{-0.029463in}{0.029463in}}%
\pgfpathcurveto{\pgfqpoint{-0.037276in}{0.021649in}}{\pgfqpoint{-0.041667in}{0.011050in}}{\pgfqpoint{-0.041667in}{0.000000in}}%
\pgfpathcurveto{\pgfqpoint{-0.041667in}{-0.011050in}}{\pgfqpoint{-0.037276in}{-0.021649in}}{\pgfqpoint{-0.029463in}{-0.029463in}}%
\pgfpathcurveto{\pgfqpoint{-0.021649in}{-0.037276in}}{\pgfqpoint{-0.011050in}{-0.041667in}}{\pgfqpoint{0.000000in}{-0.041667in}}%
\pgfpathclose%
\pgfusepath{stroke,fill}%
}%
\begin{pgfscope}%
\pgfsys@transformshift{1.025455in}{2.376000in}%
\pgfsys@useobject{currentmarker}{}%
\end{pgfscope}%
\end{pgfscope}%
\begin{pgfscope}%
\pgfpathrectangle{\pgfqpoint{0.800000in}{0.528000in}}{\pgfqpoint{4.960000in}{3.696000in}}%
\pgfusepath{clip}%
\pgfsetbuttcap%
\pgfsetroundjoin%
\definecolor{currentfill}{rgb}{0.000000,0.000000,0.000000}%
\pgfsetfillcolor{currentfill}%
\pgfsetlinewidth{1.003750pt}%
\definecolor{currentstroke}{rgb}{0.000000,0.000000,0.000000}%
\pgfsetstrokecolor{currentstroke}%
\pgfsetdash{}{0pt}%
\pgfsys@defobject{currentmarker}{\pgfqpoint{-0.041667in}{-0.041667in}}{\pgfqpoint{0.041667in}{0.041667in}}{%
\pgfpathmoveto{\pgfqpoint{0.000000in}{-0.041667in}}%
\pgfpathcurveto{\pgfqpoint{0.011050in}{-0.041667in}}{\pgfqpoint{0.021649in}{-0.037276in}}{\pgfqpoint{0.029463in}{-0.029463in}}%
\pgfpathcurveto{\pgfqpoint{0.037276in}{-0.021649in}}{\pgfqpoint{0.041667in}{-0.011050in}}{\pgfqpoint{0.041667in}{0.000000in}}%
\pgfpathcurveto{\pgfqpoint{0.041667in}{0.011050in}}{\pgfqpoint{0.037276in}{0.021649in}}{\pgfqpoint{0.029463in}{0.029463in}}%
\pgfpathcurveto{\pgfqpoint{0.021649in}{0.037276in}}{\pgfqpoint{0.011050in}{0.041667in}}{\pgfqpoint{0.000000in}{0.041667in}}%
\pgfpathcurveto{\pgfqpoint{-0.011050in}{0.041667in}}{\pgfqpoint{-0.021649in}{0.037276in}}{\pgfqpoint{-0.029463in}{0.029463in}}%
\pgfpathcurveto{\pgfqpoint{-0.037276in}{0.021649in}}{\pgfqpoint{-0.041667in}{0.011050in}}{\pgfqpoint{-0.041667in}{0.000000in}}%
\pgfpathcurveto{\pgfqpoint{-0.041667in}{-0.011050in}}{\pgfqpoint{-0.037276in}{-0.021649in}}{\pgfqpoint{-0.029463in}{-0.029463in}}%
\pgfpathcurveto{\pgfqpoint{-0.021649in}{-0.037276in}}{\pgfqpoint{-0.011050in}{-0.041667in}}{\pgfqpoint{0.000000in}{-0.041667in}}%
\pgfpathclose%
\pgfusepath{stroke,fill}%
}%
\begin{pgfscope}%
\pgfsys@transformshift{2.528485in}{2.376000in}%
\pgfsys@useobject{currentmarker}{}%
\end{pgfscope}%
\end{pgfscope}%
\begin{pgfscope}%
\pgfpathrectangle{\pgfqpoint{0.800000in}{0.528000in}}{\pgfqpoint{4.960000in}{3.696000in}}%
\pgfusepath{clip}%
\pgfsetbuttcap%
\pgfsetroundjoin%
\definecolor{currentfill}{rgb}{0.000000,0.000000,0.000000}%
\pgfsetfillcolor{currentfill}%
\pgfsetlinewidth{1.003750pt}%
\definecolor{currentstroke}{rgb}{0.000000,0.000000,0.000000}%
\pgfsetstrokecolor{currentstroke}%
\pgfsetdash{}{0pt}%
\pgfsys@defobject{currentmarker}{\pgfqpoint{-0.041667in}{-0.041667in}}{\pgfqpoint{0.041667in}{0.041667in}}{%
\pgfpathmoveto{\pgfqpoint{0.000000in}{-0.041667in}}%
\pgfpathcurveto{\pgfqpoint{0.011050in}{-0.041667in}}{\pgfqpoint{0.021649in}{-0.037276in}}{\pgfqpoint{0.029463in}{-0.029463in}}%
\pgfpathcurveto{\pgfqpoint{0.037276in}{-0.021649in}}{\pgfqpoint{0.041667in}{-0.011050in}}{\pgfqpoint{0.041667in}{0.000000in}}%
\pgfpathcurveto{\pgfqpoint{0.041667in}{0.011050in}}{\pgfqpoint{0.037276in}{0.021649in}}{\pgfqpoint{0.029463in}{0.029463in}}%
\pgfpathcurveto{\pgfqpoint{0.021649in}{0.037276in}}{\pgfqpoint{0.011050in}{0.041667in}}{\pgfqpoint{0.000000in}{0.041667in}}%
\pgfpathcurveto{\pgfqpoint{-0.011050in}{0.041667in}}{\pgfqpoint{-0.021649in}{0.037276in}}{\pgfqpoint{-0.029463in}{0.029463in}}%
\pgfpathcurveto{\pgfqpoint{-0.037276in}{0.021649in}}{\pgfqpoint{-0.041667in}{0.011050in}}{\pgfqpoint{-0.041667in}{0.000000in}}%
\pgfpathcurveto{\pgfqpoint{-0.041667in}{-0.011050in}}{\pgfqpoint{-0.037276in}{-0.021649in}}{\pgfqpoint{-0.029463in}{-0.029463in}}%
\pgfpathcurveto{\pgfqpoint{-0.021649in}{-0.037276in}}{\pgfqpoint{-0.011050in}{-0.041667in}}{\pgfqpoint{0.000000in}{-0.041667in}}%
\pgfpathclose%
\pgfusepath{stroke,fill}%
}%
\begin{pgfscope}%
\pgfsys@transformshift{4.031515in}{2.376000in}%
\pgfsys@useobject{currentmarker}{}%
\end{pgfscope}%
\end{pgfscope}%
\begin{pgfscope}%
\pgfpathrectangle{\pgfqpoint{0.800000in}{0.528000in}}{\pgfqpoint{4.960000in}{3.696000in}}%
\pgfusepath{clip}%
\pgfsetbuttcap%
\pgfsetroundjoin%
\definecolor{currentfill}{rgb}{0.000000,0.000000,0.000000}%
\pgfsetfillcolor{currentfill}%
\pgfsetlinewidth{1.003750pt}%
\definecolor{currentstroke}{rgb}{0.000000,0.000000,0.000000}%
\pgfsetstrokecolor{currentstroke}%
\pgfsetdash{}{0pt}%
\pgfsys@defobject{currentmarker}{\pgfqpoint{-0.041667in}{-0.041667in}}{\pgfqpoint{0.041667in}{0.041667in}}{%
\pgfpathmoveto{\pgfqpoint{0.000000in}{-0.041667in}}%
\pgfpathcurveto{\pgfqpoint{0.011050in}{-0.041667in}}{\pgfqpoint{0.021649in}{-0.037276in}}{\pgfqpoint{0.029463in}{-0.029463in}}%
\pgfpathcurveto{\pgfqpoint{0.037276in}{-0.021649in}}{\pgfqpoint{0.041667in}{-0.011050in}}{\pgfqpoint{0.041667in}{0.000000in}}%
\pgfpathcurveto{\pgfqpoint{0.041667in}{0.011050in}}{\pgfqpoint{0.037276in}{0.021649in}}{\pgfqpoint{0.029463in}{0.029463in}}%
\pgfpathcurveto{\pgfqpoint{0.021649in}{0.037276in}}{\pgfqpoint{0.011050in}{0.041667in}}{\pgfqpoint{0.000000in}{0.041667in}}%
\pgfpathcurveto{\pgfqpoint{-0.011050in}{0.041667in}}{\pgfqpoint{-0.021649in}{0.037276in}}{\pgfqpoint{-0.029463in}{0.029463in}}%
\pgfpathcurveto{\pgfqpoint{-0.037276in}{0.021649in}}{\pgfqpoint{-0.041667in}{0.011050in}}{\pgfqpoint{-0.041667in}{0.000000in}}%
\pgfpathcurveto{\pgfqpoint{-0.041667in}{-0.011050in}}{\pgfqpoint{-0.037276in}{-0.021649in}}{\pgfqpoint{-0.029463in}{-0.029463in}}%
\pgfpathcurveto{\pgfqpoint{-0.021649in}{-0.037276in}}{\pgfqpoint{-0.011050in}{-0.041667in}}{\pgfqpoint{0.000000in}{-0.041667in}}%
\pgfpathclose%
\pgfusepath{stroke,fill}%
}%
\begin{pgfscope}%
\pgfsys@transformshift{5.534545in}{2.376000in}%
\pgfsys@useobject{currentmarker}{}%
\end{pgfscope}%
\end{pgfscope}%
\begin{pgfscope}%
\pgfsetrectcap%
\pgfsetmiterjoin%
\pgfsetlinewidth{0.803000pt}%
\definecolor{currentstroke}{rgb}{0.000000,0.000000,0.000000}%
\pgfsetstrokecolor{currentstroke}%
\pgfsetdash{}{0pt}%
\pgfpathmoveto{\pgfqpoint{0.800000in}{0.528000in}}%
\pgfpathlineto{\pgfqpoint{0.800000in}{4.224000in}}%
\pgfusepath{stroke}%
\end{pgfscope}%
\begin{pgfscope}%
\pgfsetrectcap%
\pgfsetmiterjoin%
\pgfsetlinewidth{0.803000pt}%
\definecolor{currentstroke}{rgb}{0.000000,0.000000,0.000000}%
\pgfsetstrokecolor{currentstroke}%
\pgfsetdash{}{0pt}%
\pgfpathmoveto{\pgfqpoint{5.760000in}{0.528000in}}%
\pgfpathlineto{\pgfqpoint{5.760000in}{4.224000in}}%
\pgfusepath{stroke}%
\end{pgfscope}%
\begin{pgfscope}%
\pgfsetrectcap%
\pgfsetmiterjoin%
\pgfsetlinewidth{0.803000pt}%
\definecolor{currentstroke}{rgb}{0.000000,0.000000,0.000000}%
\pgfsetstrokecolor{currentstroke}%
\pgfsetdash{}{0pt}%
\pgfpathmoveto{\pgfqpoint{0.800000in}{0.528000in}}%
\pgfpathlineto{\pgfqpoint{5.760000in}{0.528000in}}%
\pgfusepath{stroke}%
\end{pgfscope}%
\begin{pgfscope}%
\pgfsetrectcap%
\pgfsetmiterjoin%
\pgfsetlinewidth{0.803000pt}%
\definecolor{currentstroke}{rgb}{0.000000,0.000000,0.000000}%
\pgfsetstrokecolor{currentstroke}%
\pgfsetdash{}{0pt}%
\pgfpathmoveto{\pgfqpoint{0.800000in}{4.224000in}}%
\pgfpathlineto{\pgfqpoint{5.760000in}{4.224000in}}%
\pgfusepath{stroke}%
\end{pgfscope}%
\end{pgfpicture}%
\makeatother%
\endgroup%

\caption{The empirical probability that $\GMRES{\eps}{\no}{\nt}\leq 12$ for $\sigma = 1/k^2.$\label{fig:prob-plot-2.0}}
\end{subfigure}
\caption[The empirical probability that GMRES applied to a nearby-preconditioned linear system converges in at most 12 iterations.]{The empirical probability (calculated from 1000 realisations) that $\GMRES{\eps}{\no}{\nt}\leq 12$ for $k = 10, 20, 30, 40,$ where $R=12$, $\eps = 10^{-5}$, $N = \ceil{k^{3}}$, and $\Ct=0.1,$ with $\NLiDR{\no-\nt} \sim \Exp{\sigma}$ for different functional forms of $\sigma.$}
\end{figure}
