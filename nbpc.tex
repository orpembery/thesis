\optodo{Include definition of quasi-uniform}
\optodo{Include Norm equivalence lemma (cirrently lem 3.7 in nbpc article}

\section{Introduction and Motivation from UQ}\label{sec:intronbpc}

\subsection{Statement of the problem}\label{sec:problem}

Let $\Dm \subset \RRd, d=2,3, $ be a bounded Lipschitz open set such that the open complement $\Dp:= \RRd\setminus\overline{\Dm}$ is connected. 
Let $\uj(\bx)$, $j=1,2,$ satisfy 
%be the solutions of the exterior Dirichlet problem for 
the heterogeneous Helmholtz equation
\beq\label{eq:pde}
\nabla\cdot(\Aj \nabla\uj ) + k^2 \nj \uj =-f \quad \tin \Dp,
\eeq
the Dirichlet boundary condition $\uj=0$ on $\Gamma:= \partial \Dm$, and the Sommerfeld radiation condition 
\beq\label{eq:srcnbpc}
\pdiff{\uj}{r}(\bx) - \ri k \uj(\bx) = o \left( \frac{1}{r^{(d-1)/2}}\right)
\eeq
as $r:= |\bx|\tendi$ (uniformly in $\xhat:= \bx/r$),
where $k>0$ is the wavenumber. In Equation \eqref{eq:pde},
\ben
\item $\Aj$ are symmetric, real-valued, positive-definite  matrix functions of $\bx$, 
\item $\nj$ are real-valued functions of $\bx$, bounded away from zero, and 
\item $\supp(I-\Aj)$, $\supp(1-\nj)$, and $\supp\, f$ are all %compactly 
contained in $B_R:= \{\bx : \|\bx\|_2<R\}$ for some $R>0$
\een
(see Definition \ref{def:EDP} below for a precise definition of this boundary-value problem).

Let $\Amatj$, $j=1,2,$ be the Galerkin matrices corresponding to $h$-version finite-element discretisations (with decreasing mesh size $h$ and fixed polynomial degree $p$)
of \eqref{eq:pde} truncated to $\Omega_R:= \Dp \cap B_R$, with the radiation condition realised exactly on $\Gamma_R:=\partial B_R$ via the exact Dirichlet-to-Neumann map (see Definition \ref{def:Galerkin} and \eqref{eq:matrixAjdef} below for a precise definition of $\Amatj$). Our results also hold in the case when the radiation condition is approximated by an impedance boundary condition, and the truncation boundary is not necessarily $\Gamma_R$; see Section \ref{sec:TEDP}.
 
This paper answers the following question:

\bit
\item[Q1.] How small must $\NLi{\Ao - \At}$ and 
$\NLi{\no - \nt}$ be (in terms of $k$-dependence) for GMRES 
applied to $(\Amat^{(1)})^{-1}\Amat^{(2)}$ to converge in a $k$-independent number of iterations
%$ to be a good preconditioner for $\Amat^{(2)}$
 for arbitrarily large $k$? 
\eit

\subsection{Motivation from uncertainty quantification of the Helmholtz equation} 
Consider the Helmholtz equation 
\beqs%\label{eq:pde}
\nabla\cdot\big(A(\bx;\omega) \nabla u(\bx,\omega) \big) + k^2 n(\bx;\omega) u(\bx;\omega) =-f(\bx), \quad \bx\in\Dp,
\eeqs
where $A(\bx;\omega)$ and $n(\bx;\omega)$ are \emph{random fields}, and $\omega$ is an element of the underlying probability space.
To calculate quantities of interest of the solution $u(\cdot;\omega)$, one must solve many deterministic Helmholtz problems, with each one corresponding to different realisations of the coefficients $A(\cdot,\omega)$ and $n(\cdot,\omega)$.
Solving all these deterministic problems is a very computationally-intensive task because linear systems arising from discretisations of the Helmholtz equation are notoriously difficult to solve; this difficulty is due to the following three reasons:
\begin{enumerate}
\item 
The solutions of the homogeneous Helmholtz equation $\Delta u +k^2 u=0$ oscillate on a scale of $1/k$, and so to approximate them accurately with piecewise-polynomial functions one needs the total number of degrees of freedom, $N$, to be proportional to $k^d$ as $k$ increases.
\item The \emph{pollution effect} means that, for fixed-order finite-element methods, having $N\sim k^d$ is still not enough to keep the relative finite-element error bounded independently of $k$ as $k$ increases. This growth of $N$ with $k$ leads to very large matrices, and hence to large computational costs.
\item 
The standard variational formulation of the Helmholtz equation is not coercive (i.e.~it is sign-indefinite) when $k$ is sufficiently large; in other words, zero is in the \emph{numerical range} or \emph{field of values} of the operator (defined in \eqref{eq:fov} below). This indefiniteness is inherited by the Galerkin linear system; therefore 
even when the linear system has a unique solution (which depends on the discretisation and on $k$), one expects iterative methods to behave extremely badly if the system is not preconditioned.
\end{enumerate}
%The search for good preconditioners for discretisations of the H
%Therefore, solving 

%Our motivation for tackling this question comes from calculating quantities of interest for the Helmholtz equation with \emph{random} coefficients $A$ and $n$. Such a calculation requires the solution of many deterministic Helmholtz problems, each with different $A$ and $n$, and

Suppose one calculates the LU factorisation of the Galerkin matrix for one realisation of the coefficients $A$ and $n$. The answer to Q1 above makes $k$-explicit the extent to which this factorisation can be used as an effective preconditioner for Galerkin matrices arising from different realisations of $A$ and $n$.

\subsection{Statement of the main results}\label{sec:main}
The main results about Q1, Theorem \ref{thm:1} and Corollaries \ref{cor:1} and \ref{cor:1a} below, are proved under two conditions. These are defined precisely in Conditions \ref{cond:1nbpc} and \ref{cond:2} below, but they can be informally stated as 
\bit
\item the obstacle $\Dm$ and the coefficients $\Ao$ and $\no$ are such that $u^{(1)}$ exists and the problem is \emph{nontrapping} (i.e.~all the rays of geometric optics starting in $\Omega_R$ leave $\Omega_R$ after some uniform time), and
\item the meshsize $h$ and polynomial degree $p$ in the finite-element method are chosen to depend on $k$ to ensure that the 
finite-element solution to the problem with coefficients $\Ao$ and $\no$ exists, is unique, and 
%Galerkin method (with coefficients $\Ao$ and $\no$) 
is uniformly accurate as $k\tendi$. 
\eit 

We prove results \emph{both} in the Euclidean norm on matrices, denoted by $\|\cdot\|_2$ (induced by the Euclidean norm on vectors), \emph{and} in the weighted norms $\NDmatk{\cdot}$ and $\NDmatkI{\cdot}$ defined by 
\beq\label{eq:Dk}
\NDmatk{\bv}^2:= \big( \Dmatk \bv, \bv\big)_2 = %\big( (\Smat_I + k^2 \Mmat_1)\bv,\bv\big)_2 
\N{v_h}^2_{\HokDR}
\quad \tand
\quad \NDmatkI{\bv}^2:= \big( \Dmatk^{-1} \bv, \bv\big)_2 %= %\big( (\Smat_I + k^2 \Mmat_1)\bv,\bv\big)_2 
%\N{v_h}^2_{\HokDR}
\eeq
where $\Dmatk$ is given in terms of familiar finite-element stiffness- and mass-matrices by \eqref{eq:Dk2} below, 
$\vh =\sum_i \vi \phii$, and 
\beq\label{eq:1knorm}
\NHokDR{v}^2 := \NLtDR{\grad v}^2 + k^2 \NLtDR{v}^2 \quad \tfor v \in \HozDDR,
\eeq
where the space $\HozDDR$, defined by \eqref{eq:spaceEDP} below, is the natural space containing the solution of the exterior Dirichlet problem. 

The PDE analysis of the Helmholtz equation naturally takes place in the norm $\NHokDR{\cdot}$, and \eqref{eq:Dk} shows that the norm $\NDmatk{\cdot}$ is
the discrete analogue of the norm $\NHokDR{\cdot}$. %See \eqref{eq:Dk2} and \eqref{eq:Dk3} below for this norm expressed in terms of the Euc
The norms $\NDmatk{\cdot}$ and $\NDmatkI{\cdot}$ recently appeared in results about the convergence of domain-decomposition methods %in this norm are proved 
for the Helmholtz equation \cite{GrSpVa:17}, \cite{GrSpZo:18}, and for the time-harmonic Maxwell equations \cite{BoDoGrSpTo:19}. 

Our main results about Q1 are the following.

\begin{theorem}[Main ingredient to answer to Q1]\label{thm:1}
Assume that $\Dm$, $\Ao$, and $\no$ satisfy Condition \ref{cond:1nbpc}, and assume that $h$ and $p$ satisfy Condition \ref{cond:2}. 
Let the $k$- and $h$-independent constants $\mpm$ and $\spm$ be given as in Lemma \ref{lem:normequiv} below.
Then, given $\kz>0$, there exist $\Co, \Ct>0$, independent of $h$ and $k$ (but dependent on $\Dm, \Ao, \no$, $p$, and $\kz$) such that
\begin{align}\nonumber
&\max\Big\{
\NDmatk{\Imat - (\Amat^{(1)})^{-1}\Amat^{(2)}}, 
\N{\Imat -\Amat^{(2)} (\Amat^{(1)})^{-1}}_{(\Dmat_k)^{-1}}
\Big\}\\
&\hspace{3cm} 
\leq C_1 \,k \,
\big\|\Ao-\At\big\|_{L^\infty(\Omega_R)} + C_2 \, k \, \big\|\no-\nt\big\|_{L^\infty(\Omega_R)}
\label{eq:main1}
\end{align}
and 
\begin{align}\nonumber
&\max\Big\{
\N{\Imat - (\Amat^{(1)})^{-1}\Amat^{(2)}}_2, 
\N{\Imat -\Amat^{(2)} (\Amat^{(1)})^{-1}}_2
\Big\}\\
&\hspace{0cm} 
\leq C_1 \,\left(\frac{s_+}{m_-}\right) \,\frac{1}{h} \,
\big\|\Ao-\At\big\|_{L^\infty(\Omega_R)} + C_2 \, \left(\frac{m_+}{m_-} \right)k \, \big\|\no-\nt\big\|_{L^\infty(\Omega_R)}
\label{eq:main1a}
\end{align}
for all $k\geq k_0$. 
\end{theorem}

Two notes regarding Theorem \ref{thm:1}: (i) the constants $C_1$ and $C_2$ are expressed explicitly in \eqref{eq:C1nbpc} and \eqref{eq:C2} below in terms of constants appearing in Conditions \ref{cond:1nbpc} and \ref{cond:2}, and (ii) the $L^\infty(\Omega_R)$ norm on a matrix-valued functions appearing on the right-hand sides of \eqref{eq:main1} and \eqref{eq:main1a} is defined by
\beqs
\N{A}_{L^\infty(\Omega_R)}:= \esssup_{\bx\in\Omega_R}\N{A(\bx)}_2.
\eeqs

By using the field-of-values-based theory for the convergence of GMRES introduced in \cite{El:82}, \cite{EiElSc:83} (the so-called ``Elman estimate), we have the following corollaries.

\begin{corollary}[Answer to Q1: $k$-independent weighted GMRES iterations]\label{cor:1}
Assume that $\Dm$, $\Ao$, and $\no$ satisfy Condition \ref{cond:1nbpc}, and $h$ and $p$ satisfy Condition \ref{cond:2}. Given $k_0>0$,
let $C_1$ and $C_2$ be as in Theorem \ref{thm:1}. Then if 
% there exists $C_2>0$, independent of $h$ and $k$ (but dependent on $\Dm, \Ao, \no$, $p$, and $k_0$) and given explicitly in \eqref{eq:C2} below,
% such that if 
\beq\label{eq:cond}
C_1 \,k \,\big\|\Ao-\At\big\|_{L^\infty(\Omega_R)} +C_2 \, k\, \big\|\no-\nt\big\|_{L^\infty(\Omega_R)}
\leq \frac{1}{2}
\eeq
for all $k\geq k_0$, then \emph{both} weighted GMRES working in $\|\cdot\|_{\Dmat_k}$ (and the associated inner product) applied to 
\beq\label{eq:pcsystem1}
(\Amat^{(1)})^{-1}\Amat^{(2)}\bu = \bff
\eeq
\emph{and} weighted GMRES working in $\|\cdot\|_{(\Dmat_k)^{-1}}$ (and the associated inner product) applied to 
\beq\label{eq:pcsystem2}
\Amat^{(2)}(\Amat^{(1)})^{-1}\bv = \bff
\eeq
 converge in a $k$-independent number of iterations for all $k\geq k_0$.
\end{corollary}

\begin{corollary}[Answer to Q1: $k$-independent (unweighted) GMRES iterations]\label{cor:1a}
Assume that $\Dm$, $\Ao$, and $\no$ satisfy Condition \ref{cond:1nbpc}, and $h$ and $p$ satisfy Condition \ref{cond:2}. Given $k_0>0$,
let $C_1$ and $C_2$ be as in Theorem \ref{thm:1}, and let $s_{\pm}$ and $m_{\pm}$ be as in Lemma \ref{lem:normequiv} (note that all these constants are independent of $k$, $h$, and $p$). Then if 
% there exists $C_2>0$, independent of $h$ and $k$ (but dependent on $\Dm, \Ao, \no$, $p$, and $k_0$) and given explicitly in \eqref{eq:C2} below,
% such that if 
\beq\label{eq:conda}
 C_1 \,\left(\frac{s_+}{m_-}\right) \,\frac{1}{h} \,
\big\|\Ao-\At\big\|_{L^\infty(\Omega_R)} + C_2 \, \left(\frac{m_+}{m_-} \right)k \, \big\|\no-\nt\big\|_{L^\infty(\Omega_R)}
\leq \frac{1}{2}
\eeq
for all $k\geq k_0$, then standard GMRES (working in the Euclidean norm and inner product) applied to either of the equations \eqref{eq:pcsystem1} or \eqref{eq:pcsystem2}
%\beqs
%(\Amat^{(1)})^{-1}\Amat^{(2)}\bu = \bff\quad\text{ or } \quad\Amat^{(2)}(\Amat^{(1)})^{-1}\bv = \bff
%\eeqs
 converges in a $k$-independent number of iterations for all $k\geq k_0$.
\end{corollary}



The factor $1/2$ on the right-hand sides of \eqref{eq:cond} and \eqref{eq:conda} can be replaced by any number $<1$ and the result still holds, although the number of GMRES iterations is then larger -- but still independent of $k$.
%In \S\ref{sec:proofFEM}, the constant $C_2$ is expressed explicitly in terms of $C_1$.

\bre
When $h\sim  k^{-1}$, the bounds \eqref{eq:main1} and \eqref{eq:main1a} (and hence also \eqref{eq:cond} and \eqref{eq:conda}) are identical in their $k$-dependence; however, when $h\ll k^{-1}$ (as one needs to take to overcome the pollution effect) the bound \eqref{eq:main1a} for standard GMRES is more pessimistic than the bound \eqref{eq:main1} for weighted GMRES.
\ere


\paragraph{How sharp are Theorem \ref{thm:1} and Corollaries \ref{cor:1} and \ref{cor:1a} in their $k$-dependence?}
Numerical experiments in \S\ref{sec:num} indicate that the condition \eqref{eq:cond} is sharp, in that the $k$ in \eqref{eq:cond} cannot be replaced by $k^\alpha$ for $\alpha<1$ \ednote{put some plots in here}. This suspected sharpness of \eqref{eq:cond} is also supported by the PDE-result Theorem \ref{thm:2} below. Indeed, Theorem \ref{thm:2} % and Lemma \ref{lem:1} 
 shows that the condition
\beqs
k\,
\big\|\Ao-\At\big\|_{L^\infty(\Omega_R)} \quad\text{ and } \quad k\,\big\|\no-\nt\big\|_{L^\infty(\Omega_R)}
%\Big) 
\quad\text{ both sufficiently small}
\eeqs
is not only an answer to Q1 (about finite-element discretisations), but is also the natural answer to the analogue of Q1 at the level of PDEs, namely 
\bit
\item[Q2.]
How small must $\|\Ao - \At\|_{L^\infty}$ and 
$\|\no - \nt\|_{L^\infty}$ be (in terms of $k$-dependence) for the relative error in approximating 
%$u^{(1)}$ to be a good approximation to 
$u^{(2)}$ by $u^{(1)}$ to be bounded independently of $k$ for arbitrarily-large $k$? 
\eit
Lemma \ref{lem:sharp} then shows that the condition ``$k\|\no - \nt\|_{L^\infty}$ sufficiently small" is the \emph{provably-sharp} answer to Q2 when $\Ao= \At= I$.

%Before stating these PDE results, we define the weighted $H^1$ norm
%\beq\label{eq:1knorm}
%\N{v}^2_{\HokDR} := \N{\grad v}^2_{L^2(\DR)} + k^2 \N{v}^2_{L^2(\DR)} \quad \tfor v \in H^1_{0,D}(\DR),
%\eeq
%where the space $H^1_{0,D}(\DR)$, defined by \eqref{eq:spaceEDP} below, is the natural space containing the solution of the exterior Dirichlet problem. 
To state these PDE results, we use the notation for $a,b>0$ that $a\lesssim b$ when $a\leq C b$ for some $C>0$, independent of $k$, and $a\sim b$ if $a\lesssim b$ and $b\lesssim a$.


%The sharpness of \eqref{eq:cond} and \eqref{eq:main1} is also supported by the answer to the analogue of Q1 at the level of PDEs. Indeed, the following Theorem \ref{thm:2} is the analogue of Theorem \ref{thm:1} 

\begin{theorem}[Answer to Q2 (the PDE analogue of Q1)]\label{thm:2}
%Given $f\in L^2(\Omega_R)$ such that $\supp \, f \subset B_R$, 
Let $\Dm$, $\Ao$, and $\no$ satisfy Condition \ref{cond:1nbpc}, and let $\Dm$, $\At$, and $\nt$ be such that $u^{(2)}$ exists
for any $f\in L^2(\Omega_R)$ such that $\supp \, f \subset B_R$. 
Then, given $k_0>0$, there exists $C_3>0$, independent of $k$ and given explicitly in terms of $\Dm$, $\Ao$, and $\no$ in \eqref{eq:C3} below, such that
\beq\label{eq:PDEbound}
\frac{\big\|u^{(1)}-u^{(2)}\big\|_{\HokDR}
}{
\N{u^{(2)}}_{\HokDR}
}\leq C_3 \,k\, \max\Big\{\big\|\Ao-\At\big\|_{L^\infty(\Omega_R)}\,,\, \big\|\no-\nt\big\|_{L^\infty(\Omega_R)}\Big\}%\right)
\eeq
for all $k\geq k_0$. 
\end{theorem}

\ble[Sharpness of the bound \eqref{eq:PDEbound} when $\Ao = \At= I$]\label{lem:sharp}

\

\noi 
There exist $f, \,\no$, and $\nt$ (with $\no\neq \nt$) such that 
the corresponding solutions $u^{(1)}$ and $u^{(2)}$ of the exterior Dirichlet problem with $\Ao = \At= I$ exist, are unique, and satisfy
\beq\label{eq:sharp1}
\frac{\N{u^{(1)}-u^{(2)}}_{\HokDR}
}{
\N{u^{(2)}}_{\HokDR}
}
\sim 
\frac{\N{u^{(1)}-u^{(2)}}_{L^2(\Omega_R)}
}{
\N{u^{(2)}}_{L^2(\Omega_R)}
}\sim k \big\|\no-\nt\big\|_{L^\infty(\Omega_R)}.
\eeq
%\noi (ii) There exist $f, \Ao, \At$, (with $\Ao\not\equiv \At$), such that 
%the corresponding solutions $u^{(1)}$ and $u^{(2)}$ of the exterior Dirichlet problem with $\no \equiv \nt\equiv 1$ exist, are unique, and satisfy
%%There exist $f\in L^2(\Omega_R), \Aj \in C^{0,1}(\Omega_R)$, $j=1,2$ (with $\Ao\not\equiv \At$), such that the corresponding solutions $u^{(1)}$ and $u^{(2)}$ of the exterior Dirichlet problem with $\no \equiv \nt\equiv 1$ satisfy
%\beq\label{eq:sharp2}
%\frac{\N{u^{(1)}-u^{(2)}}_{\HokDR}
%}{
%\N{u^{(2)}}_{\HokDR}
%}
%\sim 
%\frac{\N{u^{(1)}-u^{(2)}}_{L^2(\Omega_R)}
%}{
%\N{u^{(2)}}_{L^2(\Omega_R)}
%}\sim k \big\|\Ao-\At\big\|_{L^\infty(\Omega_R)}.
%\eeq
\ele


\section{Numerical experiments}\label{sec:num}

\subsection{Investigating the sharpness of Theorem \ref{thm:1} and Corollary \ref{cor:1}}

Theorem \ref{thm:1} and Corollary \ref{cor:1} were stated for the exterior Dirichlet problem, but, as highlighted in \S\ref{sec:problem} and shown in \S\ref{sec:TEDP}, they hold also for the truncated exterior Dirichlet problem (TEDP). Recall that in this problem the Sommerfeld radiation condition \eqref{eq:srcnbpc} is approximated by an impedance boundary condition (or, equivalently, when the DtN operator $T_R$ is approximated by $\ri k$); see Definition \ref{def:TEDP}. The numerical experiments in this section seek to verify the analogues of Theorem \ref{thm:1} and Corollary \ref{cor:1} for the TEDP.

\subsection{Fancy experiments Owen is doing where \eqref{eq:cond} only holds ``on average"}

%Say that these are for the TEDP defined in Definition \ref{def:TEDP}.

\section{Definitions and conditions}\label{sec:3}

\subsection{The variational problem and the Galerkin method}\label{sec:vpGm}

All our function spaces will be complex-valued, unless stated otherwise; i.e., $L^2(D)$ denotes the space of \emph{complex-valued} $L^2$ functions on $D$, but $L^\infty(D,\RR)$ denotes the space of \emph{real-valued} $L^\infty$ functions on $D$.

%$L^\infty(\domain)$ denotes complex-valued $L^\infty$ functions on a Lipschitz open set $\domain$. 
%When the range of the functions is not $\Com$, it will be given in the second argument; e.g. 
%$L^\infty(\domain,\RR^{d\times d})$ denotes the space of $d\times d$ matrices with each entry a real-valued $L^\infty$ function on $\domain$.
%We write $A\subset \subset B$ iff $A$ is \emph{compactly contained} in $B$ (i.e. $A$ is a compact subset of the open set $B$). 
%We use $\gamma$ to denote the trace operator $H^1(\domain)\rightarrow H^{1/2}(\partial \domain)$ and use $\dist(\cdot,\cdot)$ to denote the distance function.

\begin{definition}[Exterior Dirichlet problem (EDP)]\label{def:EDP}
Let $\Dm$ be a bounded Lipschitz open set such that the open complement $\Dp:= \RRd\setminus \overline{\Dm}$ is connected. Let $\Gamma:= \partial \Dm$. 
Let $\DR:= \Dp\cap B_R$ and let %$H_{0,D}^1(\DR)$ be the completion of 
\beq\label{eq:spaceEDP}
H_{0,D}^1(\DR):= \big\{ v\in H^1(\DR) : \gamma v=0 \ton \Gamma\big\},
\eeq
where $\gamma$ denotes the trace operator on $\Gamma$.
Define $T_R: H^{1/2}(\Gamma_R) \rightarrow H^{-1/2}(\Gamma_R)$ to be the Dirichlet-to-Neumann (DtN) map for the equation $\Delta u+k^2 u=0$ posed in the exterior of $B_R$ with the Sommerfeld radiation condition \eqref{eq:srcnbpc}; the definition of $T_R$ in terms of Hankel functions and polar coordinates (when $d=2$)/spherical polar coordinates (when $d=3$) is given in, e.g., \cite[Equations 3.5 and 3.6]{ChMo:08} \cite[\S2.6.3]{Ne:01}, \cite[Equations 3.7 and 3.10]{MeSa:10}.

Given 
%\bit
%\item 
$f\in L^2(\Dp)$ with $\supp\, f\subset B_R$,
%\item 
$n\in L^\infty(\Dp,\RR)$ such that $\supp(1-n)\subset B_R$ and 
\beq\label{eq:nlimitsEDP}
0<\varmin \leq n(\bx)\leq\varmax<\infty\,\, \text{ for almost every } \bx \in \Dp,
\eeq
%\item 
$A \in L^\infty (\Dp , \RR^{d\times d})$ such that $\supp(I -A)\subset B_R$, $A$ is symmetric, and there exist $0<A_{\min}\leq A_{\max}<\infty$ such that
\beq\label{eq:AellEDP}
 A_{\min} |\bxi|^2\leq\big(A(\bx) \bxi\big) \cdot\overline{ \bxi}  \leq A_{\max}|\bxi|^2 \quad\text{ for almost every }\bx \in \Dp \text{ and for all } \bxi\in \CCd,
\eeq
%\eit
we say that $u \in H^1_{0,D}(\DR)$ satisfies the \emph{exterior Dirichlet problem} if 
\beq\label{eq:EDPvar}
%\text{ find } u \in H^1_{0,D}(\DR) \tst 
%\quad 
a(u,v)=F(v) \quad \tfa v\in H^1_{0,D}(\DR),
\eeq
where
\beq\label{eq:EDPa}
a(u,v):= \int_{\DR} 
\Big((A \grad u)\cdot\grad \vb
 - k^2 n u\vb\Big) - \big\langle T_R \gamma u,\gamma v\big\rangle_{\Gamma_R}\quad\tand\quad
F(v):= \int_{\DR} f\, \vb,
\eeq
and where $\langle\cdot,\cdot\rangle_{\GR}$ is the duality pairing on $\GR$ that is linear in the first argument and antilinear in the second.
\end{definition}

\bre[The EDP with data in $(\HokDR)'$]
In Definition \ref{def:EDP} we defined the EDP with the antilinear functional $F$ arising from an $f\in \LtDR$. In the rest of the paper, 
%\item In the rest of the paper, we usually consider the EDP with data given by $F$ defined in \eqref{eq:EDPvar}, but sometimes 
we sometimes consider the EDP with general $F\in (\HozDDR)'$ and we indicate when this is the case.
In this latter situation, we recall that $\|\cdot\|_{\HokDR}$ is defined by \eqref{eq:1knorm} and define the dual norm by
\beq\label{eq:dualnorm}
\N{F}_{(\HokDR)'}:= \sup_{v\in \HozDDR} \frac{|F(v)|}{\N{v}_{\HokDR}}.
\eeq
%where $\|\cdot\|_{\HokDR}$ is defined by \eqref{eq:1knorm}.
\ere

\bre[Relationship to the scattering problem posed on $\Dp$]
For a proof that the EDP solution (as defined by Definition \ref{def:EDP}) is the truncation to $\Omega_R$ of the solution of \eqref{eq:pde} posed in $\Dp$ in a weak sense, and satisfying both the Sommerfeld radiation condition \eqref{eq:srcnbpc} and the Dirichlet boundary condition $\gamma u =0$ on $\Gamma$, see, e.g., \cite[Lemma 3.3]{GrPeSp:19}.
\ere

\bre[Existence and uniqueness]
Since $a(\cdot,\cdot)$ satisfies the G\aa rding inequality 
\beq\label{eq:Garding}
\Re a(v,v) \geq \min\big\{ A_{\min}, n_{\min}\big\} \N{v}^2_{\HokDR} - 2k^2 \N{v}^2_{L^2(\Omega_R)},
\eeq
Fredholm theory implies that existence of a solution to \eqref{eq:EDPvar} follows from uniqueness (see, e.g., \cite[Theorem 2.34]{Mc:00}).
Unique continuation results then imply that, under the additional assumption that $A$ is piecewise Lipschitz, the solution of the EDP of Definition \ref{def:EDP} exists and is unique; see \cite{GaLi:87}, \cite{BaCaTs:12}, and the discussion in \cite[\S1 and \S2.4]{GrPeSp:19} and \cite[\S]{GrSa:18}.
\ere


%\noi Note:
%\bit
%\item Here we have defined the EDP with the antilinear functional $F$ arising from an $f\in \LtDR$. In the rest of the paper, 
%%\item In the rest of the paper, we usually consider the EDP with data given by $F$ defined in \eqref{eq:EDPvar}, but sometimes 
%we sometimes consider the EDP with general $F\in (\HozDDR)'$ and we indicate when this is the case.
%In this latter situation, we let 
%\beq\label{eq:dualnorm}
%\N{F}_{(\HokDR)'}:= \sup_{v\in \HozDDR} \frac{|F(v)|}{\N{v}_{\HokDR}},
%\eeq
%where $\|\cdot\|_{\HokDR}$ is defined by \eqref{eq:1knorm}.
%\item For a proof that the EDP solution (as defined by Definition \ref{def:EDP}) is the truncation to $\Omega_R$ of the solution of \eqref{eq:pde} posed in $\Dp$ in a weak sense, and satisfying both the Sommerfeld radiation condition \eqref{eq:srcnbpc} and the Dirichlet boundary condition $\gamma u =0$ on $\Gamma$, see, e.g., \cite[Lemma 3.3]{GrPeSp:19}.
%\item Since $a(\cdot,\cdot)$ satisfies the G\aa rding inequality 
%\beq\label{eq:Garding}
%\Re a(v,v) \geq \min\big\{ A_{\min}, n_{\min}\big\} \N{v}^2_{\HokDR} - 2k^2 \N{v}^2_{L^2(\Omega_R)},
%\eeq
%Fredholm theory implies that existence of a solution to \eqref{eq:EDPvar} follows from uniqueness (see, e.g., \cite[Theorem 2.34]{Mc:00}).
%Unique continuation results then imply that, under the additional assumption that $A$ is piecewise Lipschitz, the solution of the EDP of Definition \ref{def:EDP} exists and is unique; see \cite{GaLi:87}, \cite{BaCaTs:12}, and the discussion in \cite[\S1 and \S2.4]{GrPeSp:19} and \cite[\S]{GrSa:18}.
%\eit

Given $\Dm$, $f$, $\Aj, j=1,2,$ and $\nj, j=1,2,$ all satisfying the conditions in Definition \ref{def:EDP}, we let  $\uj$ denote the solution of the exterior Dirichlet problem of Definition \ref{def:EDP} with $A=\Aj$ and $n= \nj$.
We also let $a^{(j)}(\cdot,\cdot)$ denote the sesquilinear form $a(\cdot,\cdot)$ defined in \eqref{eq:EDPa} with $A=\Aj$ and $n= \nj$.

\begin{definition}[Finite-element/Galerkin solution]\label{def:Galerkin}

Let $(\cV_h)_{h>0}$ be a family of finite-dimensional, nested subspaces of $H_{0,D}^1(\DR)$, whose union is dense in $H_{0,D}^1(\DR)$. Furthermore let $\cV_h$ consist of piecewise-polynomials on a simplicial mesh with mesh-size $h$
%\ednote{Euan says: have problem that want to allow $C^{1,1}$ $\Dm$, so that statements later about $H^2$ regularity are covered, but easiest to define triangulation and hence subspaces on Lipschitz domains -- Euan to discuss with Ivan}
and fixed polynomial degree $p$. (Note that the dimension $N$ of $V_h$ then satisfies $N\sim h^{-d}$.)


Let $a(\cdot,\cdot)$ be defined by \eqref{eq:EDPa}. Given $F\in (\HozDDR)'$, let $u$ be the solution of the variational problem \eqref{eq:EDPvar} (i.e.~$u$ is the solution of the exterior Dirichlet problem with general data). The \emph{finite-element/Galerkin solution}, $u_h$, satisfies the variational problem
\beq\label{eq:Galerkin}
%\text{ find } u \in H^1_{0,D}(\DR) \tst 
%\quad 
a(u_h,v_h)=F(v_h) \quad \tfa v_h \in V_h.
\eeq
\end{definition}

%Definition of Galerkin method

Let $\{\phi_i, i= 1, \ldots, N\}$ be a basis for $\cV_h$ with each $\phi_i$ \emph{real-valued}.
Let 
\beq\label{eq:matrixSjdef}
\big(\Smat_{A}\big)_{ij}:= \int_\Omega \big(A \nabla \phi_j)\cdot\nabla \phi_i, \quad
\big(\Mmat_{n}\big)_{ij}:= \int_\Omega n\,\phi_i\, \phi_j,
\quad\tand\quad
\big(\Nmat\big)_{ij}:= \int_{\GR} T_R (\gamma\phi_j) \,\gamma \phi_i.
\eeq
Note that both $\Smat_A$ and $\Mmat_n$ are \emph{real-valued}, but $\Nmat$ is \emph{complex-valued} (because the DtN operator $T_R$ is complex-valued).
Let
\beq\label{eq:matrixAdef}
\Amat := \Smat_{A} - k^2 \Mmat_{n} - \Nmat,
\eeq
and let $u_h:= \sum_j u_j \phi_j$. Then \eqref{eq:Galerkin} implies that
\beqs
\Amat \bu = \bff,
\eeqs
where $(\bff)_i := F(\phi_i)$.
Similar to above we let 
\beq\label{eq:matrixAjdef}
\Amatj := \Smat_{A^{(j)}} - k^2 \Mmat_{n^{(j)}} - \Nmat.
\eeq
Finally, let 
\beq\label{eq:Dk2}
\Dmat_k:= \Smat_I + k^2 \Mmat_1;
\eeq
so that \eqref{eq:Dk} holds.
%then the weighted norm $\|\cdot\|_{\Dmat_k}$ is given by 
%\beq\label{eq:Dk3}
%\N{\bv}_{\Dmat_k}^2:=   \N{v_h}^2_{\HokDR}=\big( \Dmat_k \bv,\bv\big)_2,
%\eeq
%for
%$v_h =\sum_i v_i \phi_i$. 

\bre
For arbitrary Lipschitz $\Dm$, it is not always possible to fit $\Gamma$ exactly with simplicial elements, and some analysis of non-conforming error is then necessary; since this is very standard, we do not give it here.
%\ednote{Euan copied this remark from \cite{GaGrSp:15}. Probably need to say something along these lines since don't have $H^2$ regularity for polyhedral $\Dm$ (and will talk about $H^2$ regularity in \S\ref{sec:cond2}).}
\ere

\ble[Norm equivalences of FE functions]\label{lem:normequiv}
\ednote{Euan says: need to say using a nodal basis here} If $\cV_h$ is quasi-uniform, then
there exist $m_\pm$ and $s_\pm$, independent of $h$ and $p$, such that
\beq\label{eq:normequiv1}
m_- h^{d/2} \N{\bv}_2 \leq \N{v_h}_{\LtDR} \leq m_+ h^{d/2} \N{\bv}_2,
\eeq
and
\beq\label{eq:normequiv2}
s_- h^{d/2} \N{\bv}_2 \leq \N{\nabla v_h}_{\LtDR} \leq s_+ h^{d/2-1} \N{\bv}_2,
\eeq
for all finite-element functions $v_h =\sum_i v_i \phi_i \in \cV_h$.
\ele

Written in terms of the matrices $\Mmat_1$ and $\Smat_I$ defined in \eqref{eq:matrixSjdef}, the bounds \eqref{eq:normequiv1} and \eqref{eq:normequiv2} are, respectively, the familiar bounds
\beqs
(\Mmat_1 \bv,\bv)_2 \sim h^d \N{\bv}^2_2 \quad\tand\quad h^{d}\N{\bv}^2_2 \lesssim (\Smat_I \bv,\bv)_2 \lesssim h^{d-2} \N{\bv}^2_2.
\eeqs

\bpf[Sketch proof of Lemma \ref{lem:normequiv}]\ednote{can omit this if can find a good reference. One possibility \cite[Chapter V, Lemma 2.6]{Br:97}.}
The inequalities in \eqref{eq:normequiv1} follow from writing $\|v_h\|_{\LtDR}$ as a sum of integrals over elements of the mesh, and then mapping to the reference element \ednote{Euan to discuss with Ivan}.
%\beqs
%\N{v_h}^2_{L^2(\O
%\eeqs
Then, \eqref{eq:normequiv2} follows from \eqref{eq:normequiv1} and the inequalities
\beqs
\N{v_h}_{L^2(\Omega_R)}\lesssim \N{\nabla v_h}_{L^2(\Omega_R)}\lesssim \frac{1}{h} \N{v_h}_{L^2(\Omega_R)},
\eeqs
the first of which follows from the Poincar\'e inequality, since $v_h \in \HozDDR$
(see, e.g., \cite[Proposition 5.3.4]{BrSc:00}), the second of which follows from a standard inverse estimate (see, e.g., \cite[Theorem 4.5.11]{BrSc:00}).
\epf


Finally, we need the concept of the \emph{adjoint} sesquilinear form to $a(\cdot,\cdot)$.

\begin{definition}[The adjoint sesquilinear form $a^*(\cdot,\cdot)$]\label{def:adjoint}
Let $\Dm$, $n$, and $A$, be as in the definition of the EDP (Definition \ref{def:EDP}). The adjoint sesquilinear form, $a^*(u,v)$, to $a(\cdot,\cdot)$ defined in \eqref{eq:EDPa} is given by
\beq\label{eq:EDPadjoint}
a^*(u,v) := \int_{\DR} 
\Big((A \grad u)\cdot\grad \vb
 - k^2 n u\vb\Big) - \big\langle \gamma u,T_R(\gamma v)\big\rangle_{\Gamma_R}.
\eeq
\end{definition}

\noi It is then straightforward to check that
\beq\label{eq:A*}
\Amat^* := \Smat_A -k^2 \Mmat_n - \Nmat^*
\eeq
(where $^*$ denotes conjugate transpose) is the Galerkin matrix for the sesquilinear form $a^*(\cdot,\cdot)$; i.e.~$(\Amat^*)_{ij} = a^*(\phi_j, \phi_i)$.

\ble[Link between variational problems involving $a(\cdot,\cdot)$ and $a^*(\cdot,\cdot)$]\label{lem:adjoint}
Given $F\in (\HozDDR)'$, if $u$ is the solution to the variational problem
\beq\label{eq:adjoint1}
a^*(u,v)= F(v) \quad\tfa v\in \HozDDR,
\eeq
then $\overline{u}$ satisfies
\beq\label{eq:adjoint2}
a(\overline{u},v)= \overline{F(\overline{v})} \quad\tfa v\in \HozDDR.
\eeq
\ele

For the proof of Lemma \ref{lem:adjoint} we need the following property of the DtN map $T_R$:
\beq\label{eq:DtN}
\big\langle T_R\psi, \overline{\phi} \big\rangle_\Gamma = \big\langle T_R \phi, \overline{\psi}\big\rangle_\Gamma \quad\tfa \phi,\psi \in H^{1/2}(\GR).
\eeq
This property follows from the fact that, if $u$ and $v$ are solutions of the homogeneous Helmholtz equation $\Delta u +k^2 u=0$ in $\RRd\setminus \overline{B_R}$, both satisfying the Sommerfeld radiation condition \eqref{eq:srcnbpc}, then
\beqs
\int_{\GR} (\gamma u)\, \pdiff{v}{n} = \int_{\GR} (\gamma v)\,\pdiff{u}{n};
\eeqs
see, e.g., \cite[Lemma 6.13]{Sp:15}.

\bpf[Proof of Lemma \ref{lem:adjoint}]
From \eqref{eq:adjoint1} we have that 
\beqs
\overline{a^*(u,\overline{v})}= \overline{F(\overline{v})} \quad\tfa v\in \HozDDR.
\eeqs
Using the definition of $a^*(\cdot,\cdot)$ and the property \eqref{eq:DtN} in the left-hand side of this last equation, we find \eqref{eq:adjoint2}.
\epf

\subsection{Nontrapping condition on $\Ao, \no,$ and $\Dm$ (Condition \ref{cond:1nbpc})}\label{sec:cond1}


\begin{condition}[Nontrapping bound on $u^{(1)}$]\label{cond:1nbpc}
$\Ao, \no,$ and $\Dm$ are such that, given $f\in L^2(\Omega_R)$ with $\supp \, f \subset B_R$, 
the solution of the EDP %(Definition \ref{def:EDP}) 
$u^{(1)}$ exists, is unique, and, given $k_0>0$, $u^{(1)}$ satisfies the bound 
\beq\label{eq:bound1}
\big\|u^{(1)}\big\|_{\HokDR} \leq C^{(1)}_{\rm bound} \N{f}_{L^2(\Dp)} \quad \tfa k\geq k_0,
\eeq
where $C^{(1)}_{\rm bound}$ is independent of $k$, but dependent on $\Ao, \no, \Dm, R$, and $k_0$.
\end{condition}

Using the G\aa rding inequality \eqref{eq:Garding}, one can prove the following result; see \cite[Lemma 5.1]{GrPeSp:19}.

\ble[Bound for data in $(\HozDDR)'$]\label{lem:H1}
%With the sesquilinear form $a(\cdot,\cdot)$ defined by \eqref{eq:EDPa} with $A=\Ao$ and $n=\no$, 
Given $\widetilde{F}\in (H^1_{0,D}(\Omega_R))^*$, let $\widetilde{u}$ be the solution of the variational problem
\beqs
\text{ find } \,\,\widetilde{u} \in H^1_{0,D}(\DR) \,\,\tst \,\,
a^{(1)}(\widetilde{u},v)=\widetilde{F}(v) \,\, \tfa v\in H^1_{0,D}(\DR).
\eeqs
If Condition \ref{cond:1nbpc} holds, then $\widetilde{u}$ exists, is unique, and satisfies the bound
\beq\label{eq:bound2}
\N{\widetilde{u}}_{\HokDR} \leq \frac{1}{\min\{\Aomin,\nomin\}}\left( 1 + 2 C^{(1)}_{\rm bound}\nomax  k\right) \big\|\widetilde{F}\big\|_{(\HokDR)'}
\eeq
for all $k\geq k_0$.
\ele


\subsubsection{When does Condition \ref{cond:1nbpc} hold?}\label{sec:cond1hold}

%\ednote{Euan says: could make this an appendix}
We now discuss the situations for when Condition \ref{cond:1nbpc} has been proved to hold. The main point is that Condition \ref{cond:1nbpc} holds when $\Ao, \no,$ and $\Dm$ are such that the problem is \emph{nontrapping}, but the definition of nontrapping is subtle in the case when \emph{either} $\Dm \neq\emptyset$ \emph{or} $A$ and $n$ are discontinuous.

We discuss four cases.
\ben
\item[Case 1:] $\Dm=\emptyset$, $\Ao$ and $\no$ are $C^\infty$,
\item[Case 2:] $\Dm\in C^\infty$, both $\no$ and $\Ao$ are $C^{1,1}$ in $\Dp$ and $C^\infty$ in a neighbourhood of $\Dm$.
\item[Case 3:] $\Dm$ is starshaped, $\Ao$ and $\no$ are Lipschitz and satisfy radial monotonicity-like conditions.
\item[Case 4:] $\Ao$ and $\no$ are allowed to be discontinuous.
\een

\paragraph{Case 1.}
In this case, %In the case when $\Dm=\emptyset$ and both $\Ao$ and $\no$ are $C^{1,1}$, 
the \emph{rays} of the Helmholtz equation $\nabla\cdot(A\grad u)+ k^2 nu =-f$ are defined as the projections in $\bx$ of the solutions $(\bx(s), \bxi(s)) \in \RRd\times \RRd$ of the Hamiltonian system
\beq\label{eq:rays}
%\dot{x}_i
\diff{x_i}{s}(s) = \pdiff{}{\xi_i}H\big(\bx(s), \bxi(s) \big), \qquad
%\dot{\xi}_i
\diff{\xi_i}{s}(s)
 = -\pdiff{}{x_i}H\big(\bx(s), \bxi(s) \big),
\eeq
%where $\dot{}$ denotes differentiation with respect to $s$,
where the Hamiltonian $H(\bx,\bxi)$ given by 
\beqs
H(\bx,\bxi):= \frac{1}{n(\bx)}\sum_{i=1}^d\sum_{j=1}^{d} A_{ij}(\bx)\xi_i \xi_j - 1%n(\bx)
\eeqs
(observe that $H$ is the \emph{semiclassical principle symbol} of the Helmholtz equation; see, e.g., \cite[\S7]{GrPeSp:19} for discussion on this aimed at a non-expert).

\begin{definition}[Trapping/nontrapping]\label{def:trap}
Given $\Ao$ and $\no$ as in Definition \ref{def:EDP} and $\Dm=\emptyset$, the EDP is \emph{nontrapping} if 
%Given $n\in C^1(\RRd)$ with $\supp(1-n)$ compact, consider the bicharacteristics defined by \eqref{eq:64} with $\tau=-1$ and thought of as a function of $t$ by Remark \ref{rem:bichar}. We say that $\no$ is \emph{nontrapping} if, given $B_R(\bze)\supset \supp\, n$, 
there exists a $S(R)>0$ such that all rays with $|\bx(s_0)|<R$ satisfy $|\bx(s)|>R$ for all $s\geq S(R)$; i.e.~all rays starting inside $B_R$ at $s_0$ have left $B_R$ by $S(R)$. 
We say that the EDP \emph{trapping} if it is not nontrapping.
\end{definition}

The propagation of singularities results of \cite[\S VI]{DuHo:72} (see also, e.g., \cite[Chapter 24]{Ho:85}, \cite[\S12.3]{Zw:12}) combined with the parametrix argument of \cite{Va:75} then imply that Condition \ref{cond:1nbpc} is satisfied. These arguments, however, do not give an explicit expression for $C^{(1)}_{\rm bound}$ in terms of (properties of) $\Ao$ and $\no$.

\paragraph{Case 2.}
Defining how the rays interact with the boundary $\Gamma$ is subtle, and requires the notion of the Melrose--Sj{\"o}strand generalized-bicharacteristic flow 
\cite[\S24.3]{Ho:85}, \cite{MeSj:78}, \cite{MeSj:82}; the definition of trapping/non-trapping in this case \cite[Definition 7.20]{MeSj:82} is then analogous to Definition \ref{def:trap}.

If the EDP is nontrapping, both $\no$ and $\Ao$ are globally $C^{1,1}$ and $C^\infty$ in a neighbourhood of $\Dm$, and no rays are tangent to $\Gamma$ to infinite order, 
then \cite[Theorem 2 and Equation 6.32]{GaSpWu:18} proves that Condition \ref{cond:1nbpc} holds with
\beqs
C^{(1)}_{\rm bound} := \frac{2\sqrt{2}}{\pi} \frac{1}{ (n_{\rm min})^{1/2}} \text{(Length of the longest ray in $\Omega_{R+2}$)};
\eeqs
%and the constant $k_0$ can be chosen uniformly as A and ? vary within a sufficiently small open neighborhood in C2,? (? > 0), 
this result also holds when $\Dm=\emptyset$ and both $\no$ and $\Ao$ are globally $C^{1,1}$.

\paragraph{Case 3.}
When $\Dm$ is star-shaped with respect to the origin, identities due originally to Morawetz and involving the vector field $\bx$ can be used to prove that Condiiton \ref{cond:1nbpc} holds, with an explicit expression for $C^{(1)}_{\rm bound}$, when $\Ao$ and $\no$ are both Lipschitz and satisfy
\beq\label{eq:A1nbpc}
\Ao(\bx) - (\bx\cdot\nabla)\Ao(\bx) \geq \mu_1,\,\, \text{ in the sense of quadratic forms, for almost every }\bx\in \Dp,
\eeq
and 
\beq\label{eq:n1nbpc}
\no(\bx)+ \bx\cdot\nabla \no(\bx) \geq \mu_2 \quad\text{ for almost every }\bx\in \Dp,
\eeq
for $\mu_1, \mu_2>0$;
see \cite[Theorem 2.5]{GrPeSp:19}. When $\Ao= I$, the condition on $\no$ in \eqref{eq:n1nbpc} can be improved to $2\no + \bx \cdot \nabla \no \geq \mu_2$ \cite[Theorem 2.19(ii)]{GrPeSp:19}, and when $\no$ is radial (i.e.~$\no= \no(r)$) and $\Dm=\emptyset$ this latter condition is equivalent to nontrapping \cite[Theorem 7.7]{GrPeSp:19}.
Similarly, when $n= 1$, the condition on $\Ao$ in \eqref{eq:A1nbpc} can be improved to $2\Ao - (\bx\cdot\nabla)\Ao \geq \mu_1$
\cite[Theorem 2.19(i)]{GrPeSp:19}.

\paragraph{Case 4.}
When $A$ and $n$ are discontinuous on a common $C^\infty$ interface with strictly positive curvature and $n$ jumps \emph{down} (when moving towards infinity) and $A$ jumps \emph{up}, then the problem is trapping and the solution operator grows exponentially through a sequence of complex $k$, exponentially close to the real axis, by the results in \cite{PoVo:99a} (see \cite[\S6]{MoSp:19} for how this growth implies exponential growth through a sequence of \emph{real} $k$).

When $A$ and $n$ are discontinuous on a common $C^\infty$ interface with strictly positive curvature and $n$ jumps \emph{up} and $A$ jumps \emph{down}, and $\Dm$ is 
$C^\infty$ and nontrapping (in the sense that all rays starting in the intersection of $\Dp$ and any neighbourhood of $\Dm$ leave that set, and never return, after some uniform time) then
%of \cite[Definition 7.20]{MeSj:82} 
then Condition \ref{cond:1nbpc} is proved in \cite{CaPoVo:99}, but without an explicit expression for $C^{(1)}_{\rm bound}$.

When $\Dm=\emptyset$ and $A$ and $n$ are piecewise constant (with $A$ scalar), discontinuous on a common Lipschitz star-shaped interface, and $n$ jumps \emph{up} and $A$ jumps \emph{down}, then Condition \ref{cond:1nbpc} is proved in \cite{MoSp:19} with an explicit expression for $C^{(1)}_{\rm bound}$. 
This result was then generalised in \cite[Theorem 2.7]{GrPeSp:19}, where 
Condition \ref{cond:1nbpc} is proved, again with an explicit expression for $C^{(1)}_{\rm bound}$, under the assumptions that $\Dm$ is starshaped with respect to the origin and $A$ and $n$ are $L^\infty$, with $A$ monotonically \emph{non-increasing} in the radial direction, and $n$ monotonically \emph{non-decreasing} (see \cite[Condition 2.6]{GrPeSp:19} for a more-precise statement of these conditions).


%When $A$ and $n$ are discontinuous on an interface, the problem is trapping if $n$ jumps up (as one travels across the interface in the direction away from the obstacle) and $A$ jumps down; otherwise the problem is nontrapping. When the interfaces across which $A$ and $n$ jump -- in the nontrapping way -- are $C^\infty$ with strictly positive curvature and $\Dm$ is $C^\infty$ and nontrapping in the sense of \cite[Definition 7.20]{MeSj:82} then Condition \ref{cond:1nbpc} is proved in \cite{CaPoVo:99}.


%defined by \
%The rays are again defined by \eqref{eq:rays} in $\Dp$, but defining how they interact with the boundary $\Gamma$ is subtle.


%Condition \ref{cond:1nbpc} holds when $\Ao, \no,$ and $\Dm$ are \emph{nontrapping} in the sense of Melrose--Sj{\"o}strand~\cite[Definition 7.20]{MeSj:82}.



%\begin{definition}[Nontrapping]\label{def:nt1}
%We say that $\Dm\subset \RRd, \,d=2, 3$, is
%\emph{nontrapping} if $\bound$ is smooth ($C^\infty$) and,
%given $R$ such that $\overline{\Dm}\subset B_R$, there exists a $T(R)<\infty$ such that
%all the billiard trajectories (in the sense of Melrose--Sj{\"o}strand~\cite[Definition 7.20]{MeSj:82})
%that start in $\Omega_R$ %e\cap B_R(0)$
% at time zero leave $\Omega_R$ %e\cap B_R(0)$
% by time $T(R)$.
%\end{definition}



%for all $k\geq k_0$, where $\DR:= \Dp\cap B_R(\bze)$, 
%When $A, n$, and $\Dm$ are all $C^\infty$ and such that the problem is \emph{nontrapping} (i.e.~all billiard trajectories starting in an exterior neighbourhood of $\Dp:= \RRd\setminus \overline{\Dm}$
%and evolving according to the Hamiltonian flow defined by the symbol of \eqref{eq:1} escape from that neighbourhood after some uniform time),
%then either (i) the propagation of singularities results of \cite{MeSj:82} combined with either the paramatrix argument of \cite{Va:75} or Lax--Phillips theory \cite{LaPh:89},
% or (ii) the defect-measure argument of  \cite{Bu:02}\footnote{The arguments in \cite{Bu:02} actually require that, additionally, $\partial\Dm$ has no points where the tangent vector makes infinite-order contact with $\partial\Dm$.}


\subsection{$k$-independent accuracy of the FE solution (Condition \ref{cond:2})}\label{sec:cond2}

\begin{condition}[$k$-independent accuracy of the FE solution for $a^{(1)}(\cdot,\cdot)$]
\label{cond:2}

\

(i) Given $k_0>0$, $h$ and $p$ are such that, given $f\in \LtDR$ with $\supp\, f \subset B_R$, the solution $u_h$ of the Galerkin method \eqref{eq:Galerkin} with $a(\cdot,\cdot)=a^{(1)}(\cdot,\cdot)$ (and with $F(v)$ defined in \eqref{eq:EDPa}) exists and is unique for all $k\geq k_0$. % (so that the matrix $\Amato$ is invertible). 
Furthermore, if $f= n\sum_j \alpha_j\phi_j$ for some  $n\in \LiDRRR$ and $\alpha_j \in \CC$ (i.e.~$f$ is an arbitrary element of $\cV_h$ multiplied by $n$), then
\beq\label{eq:bound3}
\N{u-u_h}_{\HokDR} \leq C^{(1)}_{\rm FEM1} \N{f}_{\LtDR} \quad\tfa k\geq k_0, 
\eeq
where $C^{(1)}_{\rm FEM1}$  is independent of $k$ and $h$, but dependent on $\Ao, \no, \Dm, R, k_0$, and $p$.

(ii) Given $k_0>0$, $h$ and $p$ are such that, given $F\in (\HozDDR)'$, the solution $u_h$ of the Galerkin method \eqref{eq:Galerkin} 
with $a(\cdot,\cdot)=a^{(1)}(\cdot,\cdot)$
exists, and is unique for all $k\geq k_0$.
%(so that the matrix $\Amato$ is invertible). 
Furthermore, if $F(v)= (A\nabla \widetilde{f},\nabla v)_{\LtDR}$, where $A\in L^\infty(\DR, \RR^{d\times d})$, $A$ is symmetric, and $\widetilde{f} := \sum_j \alpha_j \phi_j$ with $\alpha_j\in \CC$
 (i.e.~$\widetilde{f}$ is an arbitrary element of $\cV_h$), then
\beq\label{eq:bound4}
\N{u-u_h}_{\HokDR} \leq C^{(1)}_{\rm FEM2}\,k\, \N{F}_{(\HokDR)'} \quad\tfa k\geq k_0, 
\eeq
where $C^{(1)}_{\rm FEM2}$  is independent of $k$ and $h$, but dependent on $\Ao, \no, \Dm, R, k_0$, and $p$.
\end{condition}

\subsubsection{When does Condition \ref{cond:2} hold?}\label{sec:cond2hold}

First recall that the Galerkin solution is asymptotically quasi-optimal with constant independent of $k$ if there exists $h_0= h_0(k,p)>0$ and $C_{\rm qo}>0$ (independent on $k$) such that if $h\leq h_0$ then 
\beq\label{eq:qo}
\N{u-u_h}_{\HokDR} \leq C_{\rm qo} \min_{v_h\in V_h} \N{u-v_h}_{\HokDR}.
\eeq
Observe that if Condition \ref{cond:1nbpc} holds and the Galerkin solution is asymptotically quasi-optimal with constant independent of $k$, 
then Condition \ref{cond:2} holds (by choosing $v_h=0$). Furthermore, $C^{(1)}_{\rm FEM1}$ is then the product of $C_{\rm qo}$ and $C^{(1)}_{\rm bound}$, and  $C^{(1)}_{\rm FEM2}$ is related to $C_{\rm qo}$ and the constant on the right-hand side of \eqref{eq:bound2}.

In the case of the \emph{homogeneous} Helmholtz equation (i.e.~\eqref{eq:pde} with $\Ao= I$ and $\no=1$) conditions on $h,k,$ and $p$ for either quasi-optimality \eqref{eq:qo} or bounds on the Galerkin error in terms of the data (as in Condition \ref{cond:2}) were obtained in 1-d in \cite{IhBa:95, IhBa:97}, and then in 2- and 3-d in \cite{Me:95, Sa:06, MeSa:10, MeSa:11, Wu:13, ZhWu:13, ChNi:18} (with many of these references studying the truncated EDP -- see Section \S\ref{sec:TEDP} -- instead of the EDP). 
\ednote{Maybe quote here all the thresholds, or put in numerical results section.}
Initial analogues of these results for heterogeneous Helmholtz problems (making everything explicit in the coefficients $\Ao$ and $\no$, as well as in $h, k$, and $p$) can be found in \cite{GrSa:18} (for the TEDP) and \cite{GaSpWu:18} (for the EDP).
 
We now give results involving sufficient conditions for Part (ii) of Condition \ref{cond:2}  to hold. 
We highlight at this stage that all the techniques discussed below
%used to prove Lemmas \ref{lem:hp1} and \ref{lem:hp2} 
require at least $H^2$ regularity of the (adjoint) solution. Since the data for the variational problem in Part (ii) of Condition \ref{cond:2} is in $(\HokDR)'$ and not in $\LtDR$, none of these techniques can be used to obtain conditions under which Part (ii) of Condition \ref{cond:1nbpc} holds.

\ble[Conditions under which Part (i) of Condition \ref{cond:2} holds when $p=1$]\label{lem:hp1}
If $\Gamma\in C^{1,1}$, $A\in C^{0,1}$, Condition \ref{cond:1nbpc} holds and $p=1$, then Part (i) of Condition \ref{cond:2} holds
if $h k^2$ is sufficiently small. 
\ele

\bpf[Sketch proof]
Quasi-optimality follows from the convergence analysis of the FEM via the Schatz argument \cite{Sc:74}/Aubin-Nitsche trick (with these techniques pioneered in the context of the homogeneous Helmholtz equation by \cite{Me:95, Sa:06, MeSa:10, MeSa:11}); see \cite[\S6]{GaSpWu:18} for these techniques specifically applied to the EDP of Definition \ref{def:EDP} (and for an explicit expression for the constant $C_{\rm qo}$ in terms of $C_{\rm bound}^{(1)}$). 
Inserting the bound  \eqref{eq:bound1} of Condition \ref{cond:1nbpc} into standard elliptic regularity estimates for the operator $\nabla\cdot(\Ao\nabla )$ implies that $\|u\|_{H^2(\DR)}\lesssim k\N{f}_{\LtDR}$, and then, using results from polynomial approximation theory, we have
 \beq\label{eq:AN1}
\N{u-u_h}_{\HokDR} \lesssim h \N{u}_{H^2(\DR)} + hk \N{u}_{H^1(\DR)}\lesssim hk \N{f}_{\LtDR} \lesssim \frac{1}{k}\N{f}_{\LtDR},
\eeq
and so Part (i) of Condition \ref{cond:2} is readily satisfied.
\epf

\bre[Results about Part (i) of Condition \ref{cond:2} for the truncated EDP]
In the context of the truncated EDP (TEDP) defined in \S\ref{sec:TEDP}, the result of Lemma \ref{lem:hp1} holds more generally. Indeed,
if $\Gamma_D\in C^{1,1}$, $\Gamma_I$ is either $C^{1,1}$ or convex, $A\in C^{0,1}$, Condition \ref{cond:1nbpc} holds and $p\leq r+1$, then Part (i) of Condition \ref{cond:2} holds
if $h^p k^{p+1}$ is sufficiently small. This result is proved 
%When $r\geq 1$ and $p\leq r+1$, quasi-optimality is proved when $h^p k^{p+1}$ is sufficiently small 
in \cite{ChNi:18a}, using a novel splitting of the solution (motivated by the splittings in \cite{MeSa:10, MeSa:11} for the case $\Ao= I, \no=1$). The first two bounds in \eqref{eq:AN1} hold as before, and then 
$\|u-u_h\|_{\HokDR} \lesssim  k^{-1/p}\|f\|_{\LtDR}$, so the analogue of Part (i) of Condition \ref{cond:2} for the TEDP is readily satisfied.

In addition, when $\Ao=I$ and $\no=1$, Condition \ref{cond:1nbpc} holds, and $p=1$,  Part (i) of Condition \ref{cond:2} holds when $hk^{3/2}$ sufficiently small (cf.~$hk^2$ is sufficiently small in Lemma \ref{lem:hp1}) by 
%. This result is proved when $\Ao=I$ and $\no=1$ in 
\cite{ZhWu:13} (see also \cite{ChNi:18}). This proof uses 
a variation of the Schatz technique first introduced by \cite{FeWu:11} in the context of DG methods, and this technique requires a regularity assumption recently proved in \cite{ChNiTo:18}. Since this latter regularity result holds for general Hermitian-postive-definite $A\in C^{0,1}$, this result for the TEDP can be generalised for such $\Ao$ and arbitrary $\no$ satisfying the requirements in the definition of the TEDP (Definition \ref{def:TEDP}).
\ere


%\bpf[Sketch proof]
%When $r=0$, quasi-optimality follows from the convergence analysis of the FEM via the Schatz argument \cite{Sc:74}/Aubin-Nitsche trick (with these techniques pioneered in the Helmholtz context by \cite{Me:95, Sa:06, MeSa:10, MeSa:11}); see \cite[\S6]{GaSpWu:18} for these techniques specifically applied to the EDP of Definition \ref{def:EDP} (and for an explicit expression for the constant $C_{\rm qo}$ in terms of $C_{\rm bound}^{(1)}$). 
%Inserting the bound  \eqref{eq:bound1} of Condition \ref{cond:1nbpc} into standard elliptic regularity estimates for the operator $\nabla\cdot(\Ao\nabla )$ implies that $\|u\|_{H^2(\DR)}\lesssim k\N{f}_{\LtDR}$, and then, using results from polynomial approximation theory, we have
% \beq\label{eq:AN1}
%\N{u-u_h}_{\HokDR} \lesssim h \N{u}_{H^2(\DR)} + hk \N{u}_{H^1(\DR)}\lesssim hk \N{f}_{\LtDR} \lesssim \frac{1}{k}\N{f}_{\LtDR},
%\eeq
%and so Part (i) of Condition \ref{cond:2} is readily satisfied.
%
%When $r\geq 1$ and $p\leq r+1$, quasi-optimality is proved when $h^p k^{p+1}$ is sufficiently small in \cite{ChNi:18a}, using a novel splitting of the solution (motivated by the splittings in \cite{MeSa:10, MeSa:11} for the case $\Ao= I, \no=1$). The first two bounds in \eqref{eq:AN1} hold as before, and then 
%$\|u-u_h\|_{\HokDR} \lesssim  k^{-1/p}\|f\|_{\LtDR}$, so Part (i) of Condition \ref{cond:2} is again readily satisfied.
%\epf
%
%The next lemma is useful only when $p=1$.
%
%\ble[Alternative conditions under which Part (i) of Condition \ref{cond:2} holds]\label{lem:hp2}
%Part (i) of Condition \ref{cond:2} holds for $p=1$ when $hk^{3/2}$ sufficiently small. 
%\ele
%
%\bpf[References for proof]
%This result is proved when $\Ao=I$ and $\no=1$ in \cite{ChNi:18}. This proof uses 
%a variation of the Schatz technique first introduced by \cite{FeWu:11} in the context of DG methods, 
%and underlined by the regularity result of \cite{ChNiTo:18}. Since this latter regularity result holds 
%\epf
%
%\
%



%%%%%%%%%EUAN KEEPING OLD VERSION
%We make the following remarks about Condition \ref{cond:2}.
%\ben
%%\item If both Conditions \ref{cond:1nbpc} and \ref{cond:2} are satisfied, then $u_h$ satisfies an a priori bound with the same $k$-dependence as the bound on $u$ in Condition \ref{cond:1nbpc}. % (and this is how we'll use Condition \ref{cond:2} in the proof of Theorem \ref{thm:1}).
%\item If the Galerkin solution is asymptotically quasi-optimal with constant independent of $k$, i.e.~there exists $h_0= h_0(k,p)>0$ and $C_{\rm qo}>0$ (independent on $k$) such that if $h\leq h_0$ then 
%\beq\label{eq:qo}
%\N{u-u_h}_{\HokDR} \leq C_{\rm qo} \min_{v_h\in V_h} \N{u-v_h}_{\HokDR},
%\eeq
%and Condition \ref{cond:1nbpc} holds, then Condition \ref{cond:2} holds. Furthermore, $C^{(1)}_{\rm FEM1}$ is then the product of $C_{\rm qo}$ and $C^{(1)}_{\rm bound}$, and  $C^{(1)}_{\rm FEM2}$ is related to $C_{\rm qo}$ and the constant on the right-hand side of \eqref{eq:bound2}.
%\item The convergence analysis of the FEM via the Schatz argument \cite{Sc:74}/Aubin-Nitsche trick (with these techniques pioneered in the Helmholtz context by \cite{Me:95}, \cite{Sa:06}, \cite{MeSa:10}) requires that the solution of the adjoint problem be $H^2$ regular. Applied to the PDE \eqref{eq:pde}, these techniques prove that quasi-optimality \eqref{eq:qo} holds
% when $\Gamma \in C^{1,1}$ $A\in C^{0,1}$, $f\in L^2$, Condition \ref{cond:1nbpc} holds,  $p=1$, and $hk^2$ is sufficiently small; see \cite[\S6]{GaSpWu:18} for these techniques specifically applied to the EDP of Definition \ref{def:EDP}.
% 
%If Condition \ref{cond:1nbpc} holds and $u\in H^2$, one can show that $\|u\|_{H^2(\DR)}\lesssim k\N{f}_{\LtDR}$, and then, using results about polynomial approximation theory, we have
% \beq\label{eq:AN1}
%\N{u-u_h}_{\HokDR} \lesssim h \N{u}_{H^2(\DR)} + hk \N{u}_{H^1(\DR)}\lesssim hk \N{f}_{\LtDR} \lesssim \frac{1}{k}\N{f}_{\LtDR},
%\eeq
%and so Part (i) of Condition \ref{cond:2} is readily satisfied.
%%(omitted constant depends on $C^{(1)}_{\rm bound}$ and ``$H^2$ regularity constant"). To get Part (ii) satisfied using these arguments would need $\phi_i \in H^2$ (restrictive).
%\item A generalisation of the arguments in Point 2 with $f\in H^r$, $\Gamma\in C^{r+1,1}$, $A\in C^{r,1}$ (so that $u\in H^{r+2}$ by elliptic regularity), $p\leq r+1$, shows that quasioptimality
%% (and hence \eqref{eq:AN1}) 
%holds if $h^p k^{p+1}$ is sufficiently small \cite{ChNi:18a}, and one can then show that $\N{u-u_h}_{\HokDR}\lesssim k^{-1} \N{f}_{\LtDR}$, as in \eqref{eq:AN1}; so Part (i) of Condition \ref{cond:2} is again readily satisfied.
%
%\item A new variation of the Schatz technique from \cite{FeWu:11} in the context of DG methods, used in context of standard FEM in \cite{ChNi:18}
%and underlined by the regularity result of \cite{ChNiTo:18}
%
%
%Part (i) of Condition \ref{cond:2} holds for $p=1$ when $hk^{3/2}$ sufficiently small. This result proved already in 1-d in \cite{IhBa:95a}.
%%\item ***Numerical experiments for Owen to do*** If we don't have $H^2$ regularity, do we still have that Condition \ref{cond:2} holds with $hk^2 \sim 1$? (I would guess yes.) What about when $hk^{3/2}\sim 1$?
%\een


\section{Proofs of the main results}\label{sec:proofs}

\subsection{Proof of Theorem \ref{thm:1}} 

The following two lemmas are the heart of the proof of Theorem \ref{thm:1}.

\ble[Bounds on $(\Amato)^{-1} \Mmat_{n}$]\label{lem:keylemma1}
Assume that Condition \ref{cond:1nbpc} holds, and assume that Part (i) of Condition \ref{cond:2} holds. Then, for $n\in \LiDRRR$,
\beq\label{eq:keybound1}
\max\Big\{\big\| (\Amato)^{-1} \Mmat_{n} \big\|_{\Dmat_k}, \,\,
\big\|  \Mmat_{n}(\Amato)^{-1} \big\|_{(\Dmat_k)^{-1}}
\Big\}\leq 
C_2
%\frac{m_+}{m_-} \left[ C_{\rm FEM1}^{(1)} + C_{\rm bound}^{(1)}\right] 
\frac{\N{n}_{L^\infty(\DR)}}{k}
\eeq
and 
\beq\label{eq:keybound1a}
\max\Big\{\big\| (\Amato)^{-1} \Mmat_{n} \big\|_2, \,\,
\big\|  \Mmat_{n}(\Amato)^{-1} \big\|_2 
\Big\}\leq 
C_2 
%\frac{m_+}{m_-} \left[ C_{\rm FEM1}^{(1)} + C_{\rm bound}^{(1)}\right] 
\left(\frac{m_+}{m_-}\right) \frac{\N{n}_{L^\infty(\DR)}}{k}
\eeq
for all $k\geq k_0$,
where
\beq\label{eq:C2}
C_2:=%\frac{m_+}{m_-} 
%\left[ 
C_{\rm FEM1}^{(1)} + C_{\rm bound}^{(1)}.%\right].
\eeq
\ele

\ble[Bounds on $(\Amato)^{-1} \Smat_A$]\label{lem:keylemma2}
Assume that Condition \ref{cond:1nbpc} holds, and assume that Part (ii) of Condition \ref{cond:2} holds. Then, for $A\in L^\infty(\DR,\RR^{d\times d})$,
\beq\label{eq:keybound2}
\max\Big\{\big\| (\Amato)^{-1} \Smat_A \big\|_{(\Dmat_k)^{-1}}, \,\,
\big\| \Smat_A (\Amato)^{-1} \big\|_{\Dmat_k}\Big\} \leq C_1\, k\N{A}_{L^\infty(\DR)}
\eeq
and
\beq\label{eq:keybound2a}
\max\Big\{\big\| (\Amato)^{-1} \Smat_A \big\|_2, \,\,
\big\| \Smat_A (\Amato)^{-1} \big\|_2\Big\} \leq C_1\,\left(\frac{s_+}{m_-}\right) \frac{1}{h}\N{A}_{L^\infty(\DR)}
\eeq
%\begin{align}\nonumber
%&\max\Big\{\big\| (\Amato)^{-1} \Smat_A \big\|_2, \,\,
%\big\| \Smat_A (\Amato)^{-1} \big\|_2\Big\}\nonumber \\
%&\hspace{2cm}
% \leq \frac{s_+}{s_-} \left[ C_{\rm FEM2}^{(1)} + 
% \frac{1}{\min\big\{\Aomin,\nomin\big\}}\left( \frac{1}{k_0} + 2 C^{(1)}_{\rm bound}\nomax  \right) \right]k\N{A}_{L^\infty(\DR)}\label{eq:keybound2}
%% + C_{\rm bound}^{(1)}\right) \frac{\N{n}_{L^\infty(\DR)}}{k}.
%\end{align}
for all $k\geq k_0$, where
\beq\label{eq:C1nbpc}
C_1:=%\frac{s_+}{s_-} 
\left[ C_{\rm FEM2}^{(1)} + 
 \frac{1}{\min\big\{\Aomin,\nomin\big\}}\left( \frac{1}{k_0} + 2 C^{(1)}_{\rm bound}\nomax  \right) \right]
\eeq
\ele

\bpf[Proof of Theorem \ref{thm:1} using Lemmas \ref{lem:keylemma1} and \ref{lem:keylemma2}]
Using the definition of the matrices $\Amatj, \Smat_A$, and $\Mmatj$ in \eqref{eq:matrixAjdef} and \eqref{eq:matrixSjdef}, we have
\begin{align}\nonumber
\Imat - (\Amato)^{-1}\Amatt = (\Amato)^{-1}\big(\Amato-\Amatt\big) &=  (\Amato)^{-1}\left( \Smat_{A^{(1)}} - \Smat_{A^{(2)}} - k^2 \big(\Mmat_{n^{(1)}}-\Mmat_{n^{(2)}}\big)\right)\\
&= (\Amato)^{-1}\left( \Smat_{A^{(1)}-A^{(2)}} - k^2 \Mmat_{n^{(1)}-n^{(2)}}\right),\label{eq:idea1}
\end{align}
and similarly 
\beq\label{eq:idea2}
\Imat -\Amatt  (\Amato)^{-1}= \left( \Smat_{A^{(1)}-A^{(2)}} - k^2 \Mmat_{n^{(1)}-n^{(2)}}\right)(\Amato)^{-1}.
\eeq
The bound  \eqref{eq:main1} on $\|\Imat - (\Amato)^{-1}\Amatt\|_2$ and  $\|\Imat - \Amatt(\Amato)^{-1}\|_2$ then follows from using the bounds \eqref{eq:keybound1} and \eqref{eq:keybound2} in \eqref{eq:idea1} and \eqref{eq:idea2}.
%
%, and $C_1$, $C_2$ in \eqref{eq:main1} are given explicitly by
%\beq\label{eq:C1nbpc}
%C_1:=%\frac{s_+}{s_-} 
%\left[ C_{\rm FEM2}^{(1)} + 
% \frac{1}{\min\big\{\Aomin,\nomin\big\}}\left( \frac{1}{k_0} + 2 C^{(1)}_{\rm bound}\nomax  \right) \right] \,\,\tand\,\,
%\quad C_2:=  %+ \frac{m_+}{m_-} 
% \left[ C_{\rm FEM1}^{(1)} + C_{\rm bound}^{(1)}\right].
%\eeq
\epf

\

\bpf[Proof of Lemma \ref{lem:keylemma1}]
We first concentrate on proving \eqref{eq:keybound1}.
Given $\bff \in \CC^N$ and $n\in \LiDRRR$, we create a variational problem whose Galerkin discretisation leads to the equation $\Amato \tbu = \Mmat_n\,\bff$.
Indeed, let $\widetilde{f} := \sum_j f_j \phi_j\in \HozDDR$. Define $\widetilde{u}$ to be the solution of the variational problem 
\beq\label{eq:411}
a^{(1)}(\widetilde{u},v)= (n\widetilde{f},v)_{L^2(\Omega)} \quad\text{ for all } v\in H^1(\Omega),
\eeq
let $\tu_h$ be the solution of the finite-element approximation of \eqref{eq:411}, i.e.,
\beq\label{eq:41}
a^{(1)}(\tu_h,v_h)= (n\widetilde{f},v_h)_{L^2(\Omega)} \quad\text{ for all } v_h\in \cV_h,
\eeq
and let $\tbu$ be the vector of nodal values of $\tu_h$ \ednote{rephrase this as haven't (yet) assumed that $\{\phi_i\}$ is a nodal basis}. The definition of $\widetilde{f}$ then implies that \eqref{eq:41} is equivalent to the linear system $\Amato \tbu = \Mmat_{n}\,\bff$, and so to obtain a bound on $\|(\Amato)^{-1}\Mmat_n\|_{\Dmat_k}$ we need to bound $\|\tbu\|_{\Dmat_k}$ in terms of $\|\bff\|_{\Dmat_k}$. Because of the definition 
of $\|\cdot\|_{\Dmat_k}$ in \eqref{eq:Dk}, this is equivalent to bounding $\|\tu_h\|_{\HokDR}$ in terms of $\|\widetilde{f}\|_{\HokDR}$.

%First observe that the bound \eqref{eq:bound3} from Part (i) of Condition \ref{cond:2} holds for the solution of the variational problem
%\beqs%\label{eq:411}
%a^{(1)}(u,v)= (n\phi_j,v)_{L^2(\Omega)} \quad\text{ for all } v\in H^1(\Omega),
%\eeqs
%and hence, by linearity, it also holds for the solution $\widetilde{u}$ of the variational problem \eqref{eq:411}.

Using %the bounds in \eqref{eq:normequiv1}, 
the triangle inequality and the bounds \eqref{eq:bound3} and \eqref{eq:bound1} from Conditions \ref{cond:2} and \ref{cond:1nbpc} respectively, we find
%Note that the hypotheses imply that the bound on the solution operator 
%\eqref{eq:bound_unif} holds (by Corollary \ref{cor:uniform}), and also that if $h k\sqrt{|k^2-\eps|} \leq C_1$ then quasi-optimality \eqref{eq:qoeps_lemma} holds (by Lemma \ref{lem:qo}).
%Starting with \eqref{eq:equiv} we then have 
\begin{align}
%m_- h^{d/2}k \N{\tbu}_2 \leq k\N{\tu_h}_{\LtDR}\leq  
\N{\tu_h}_{\HokDR} \leq
\N{\tu-\tu_h}_{\HokDR} + \N{\tu}_{\HokDR} \label{eq:mainevent1}
& \leq C^{(1)}_{\rm FEM1}\N{nf}_{\LtDR} + C^{(1)}_{\rm bound}\big\|n\widetilde{f}\big\|_{\LtDR} \\ \nonumber
& \leq\big(C^{(1)}_{\rm FEM1}+  C^{(1)}_{\rm bound}\big)\N{n}_{L^\infty(\DR)}\frac{\big\|\widetilde{f}\big\|_{\HokDR}}{k};
%& \leq\big(C^{(1)}_{\rm FEM1}+  C^{(1)}_{\rm bound}\big)\N{n}_{L^\infty(\DR)} m_+ h^{d/2} \N{\bff}_2,
\end{align}
the bound on $\|(\Amato)^{-1}\Mmat_n\|_{\Dmat_k}$ in \eqref{eq:keybound1} then follows from the definition of $\|\cdot\|_{\Dmat_k}$ in \eqref{eq:Dk} and the definition of $C_2$ \eqref{eq:C2}.

To prove the bound on $\|\Mmat_n(\Amato)^{-1}\|_{(\Dmat_k)^{-1}}$ in \eqref{eq:keybound1}, first observe that the definitions of $\|\cdot\|_{\Dmat_k}$ and $\|\cdot\|_{(\Dmat_k)^{-1}}$ imply that, for any matrix $\matrixC$ and for any $\bv\in \CC^N$,
\beq\label{eq:A380-0}
\frac{
\big\|\matrixC \bv \big\|_{(\Dmat_k)^{-1}}
}{
\big\|\bv\|_{(\Dmat_k)^{-1}}
} = 
\frac{
\big\|\matrixC^* \bw \big\|_{\Dmat_k}
}{
\big\|\bw\|_{\Dmat_k}
}
\eeq
where $\bw := (\Dmat_k)^{1/2}\bv$, and where $\matrixC^*$ is the conjugate transpose of $\matrixC$ (i.e.~the adjoint with respect to $(\cdot,\cdot)_2$).
Therefore, since $\Mmat_n$ is a real, symmetric matrix,
\beqs%
\frac{
\big\|\Mmat_n (\Amato)^{-1}\bv\big\|_{(\Dmat_k)^{-1}}
}{
\N{\bv}_{(\Dmat_k)^{-1}}
}
=
\frac{
\big\|((\Amato)^*)^{-1}\Mmat_n\bw\big\|_{\Dmat_k}
}{
\N{\bw}_{\Dmat_k}
},
 \eeqs
 so that 
\beq\label{eq:A380} 
 \big\|\Mmat_n (\Amato)^{-1}\big\|_{(\Dmat_k)^{-1}}=\big\|((\Amato)^*)^{-1}\Mmat_n\big\|_{\Dmat_k}.
 \eeq 
Recall from the text below \eqref{eq:A*} that $(\Amato)^*$ is the Galerkin matrix corresponding to the variational problem \eqref{eq:adjoint1} -- the adjoint problem. Lemma \ref{lem:adjoint} implies that if the EDP %with coefficients $A^{(1)}$ and $n^{(1)}$ 
satisfies Conditions \ref{cond:1nbpc} and \ref{cond:2}, then so does the adjoint problem. Therefore, the argument above leading to the bound on $\|(\Amato)^{-1}\Mmat_n\|_{\Dmat_k}$ under Condition \ref{cond:1nbpc} and Part (i) of Condition \ref{cond:2} proves the same bound on $\|((\Amato)^*)^{-1}\Mmat_n\|_{\Dmat_k}$, and then, using \eqref{eq:A380}, also on $\big\|\Mmat_n(\Amato)^{-1}\big\|_{(\Dmat_k)^{-1}}$.

To prove the bound on  $\|(\Amato)^{-1}\Mmat_n\|_{2}$ in \eqref{eq:keybound1a}, we use the bounds 
\beqs
m_- h^{d/2} k \N{\tbu}_2 \leq k \N{\widetilde{u}_h}_{\LtDR} \leq \N{\widetilde{u}_h}_{\HokDR}
\,\tand\,
\big\|\widetilde{f}\big\|_{\LtDR} \leq m_+ h^{d/2}\N{\bff}_2,
\eeqs
on either side of the inequality \eqref{eq:mainevent1}, with these bounds coming from \eqref{eq:normequiv1}. The proof of the bound on 
$\|\Mmat_n((\Amato)^*)^{-1}\|_{2}$ in \eqref{eq:keybound1a} follows in a similar way to above, using the fact that 
$\|\Mmat_n (\Amato)^{-1}\|_2=\|((\Amato)^*)^{-1}\Mmat_n\|_2$ (compare to \eqref{eq:A380}).
%, namely the variational problem \eqref{eq:EDPvar} with the operator $T_R$ in $a^{(1)}(\cdot,\cdot)$ replaced by $\overline{T_R}$ (corresponding to the $-\ri k$ in the radiation condition \eqref{eq:srcnbpc} being changed to $+\ri k$).
%
%Now, if $u$ is the solution of the adjoint problem with data $F(v)$, then $\overline{u}$ is the solution of the original problem with data $\overline{F(\overline{v})}$; 
%
%in particular if $F(v)$ is as in \eqref{eq:EDPa}, then the $L^2$ data of the adjoint problem is just $\overline{f}$. Therefore, if the EDP satisfies Conditions \ref{cond:1nbpc} and \ref{cond:2}, then so does its adjoint, and
% the bound in \eqref{eq:keybound1} on $\|(\Amato)^{-1}\Mmat_n\|_{2}$ also holds for $\|((\Amato)^*)^{-1}\Mmat_n\|_{2}$.
\epf

\

\bpf[Proof of Lemma \ref{lem:keylemma1}]
In a similar way to the proof of Lemma \ref{lem:keylemma1}, given $\bff \in \CC^N$ and a symmetric $A\in L^\infty(\DR, \RR^{d\times d})$, let $\widetilde{f} := \sum_j f_j \phi_j$ and observe that $\widetilde{f} \in \HozDDR$. Define $\widetilde{u}$ to be the solution of the variational problem 
\beq\label{eq:411a}
a^{(1)}(\widetilde{u},v)= \widetilde{F}(v) \quad\text{ for all } v\in H^1(\Omega),
\quad\text{ where } \quad
 \widetilde{F}(v) :=(A\nabla\widetilde{f},\nabla v)_{L^2(\Omega)}.
\eeq
Observe that the definition of the norms $\|\cdot\|_{(\HokDR)'}$ \eqref{eq:dualnorm} and $\|\cdot\|_{\HokDR}$ \eqref{eq:1knorm} and the Cauchy-Schwarz inequality imply that
\beq\label{eq:Fbound}
\big\| \widetilde{F}\big\|_{(\HokDR)'}\leq \big\|A\nabla \widetilde{f}\big\|_{\LtDR} \leq \big\|A\big\|_{L^\infty(\DR)} \big\|\nabla \widetilde{f}\big\|_{\LtDR}\leq 
\big\|A\big\|_{L^\infty(\DR)} \big\| \widetilde{f}\big\|_{\HokDR}.
\eeq
Let $\tu_h$ be the solution of the finite element approximation of \eqref{eq:411a}, i.e.,
\beq\label{eq:41a}
a^{(1)}(\tu_h,v_h)= \widetilde{F}(v_h) \quad\text{ for all } v_h\in \cV_h,
\eeq
and let $\tbu$ be the vector of nodal values of $\tu_h$. The definition of $\widetilde{f}$ then implies that \eqref{eq:41a} is equivalent to $\Amato \tbu = \Smat_A\,\bff$. 

Similar to in the proof of Lemma \ref{lem:keylemma1},
using the triangle inequality, the bound \eqref{eq:bound4} from Condition \ref{cond:2}, the bound \eqref{eq:bound2} from Lemma \ref{lem:H1}, the bound \eqref{eq:Fbound}, and the definition of $C_1$ \eqref{eq:C1nbpc},
we find
%Note that the hypotheses imply that the bound on the solution operator 
%\eqref{eq:bound_unif} holds (by Corollary \ref{cor:uniform}), and also that if $h k\sqrt{|k^2-\eps|} \leq C_1$ then quasi-optimality \eqref{eq:qoeps_lemma} holds (by Lemma \ref{lem:qo}).
%Starting with \eqref{eq:equiv} we then have 
\begin{align}\nonumber 
%s_- h^{(d-2)/2} \N{\tbu}_2 &\leq \N{\nabla \tu_h}_{\LtDR}\leq  
\N{\tu_h}_{\HokDR} &\leq
\N{\tu-\tu_h}_{\HokDR} + \N{\tu}_{\HokDR},\nonumber \\ \nonumber
& \leq \left[ C^{(1)}_{\rm FEM2} k + 
\frac{1}{\min\{\Aomin,\nomin\}}\left( 1 + 2 C^{(1)}_{\rm bound}\nomax k  \right) 
\right]\big\|\widetilde{F}\big\|_{(\HokDR)'},\\
&\leq C_1 \, k\, 
%\left[C^{(1)}_{\rm FEM2} k + \frac{1}{\min\{\Aomin,\nomin\}}\left( 1 + 2 C^{(1)}_{\rm bound}\nomaxk  \right) \right]
\big\|A\big\|_{L^\infty(\DR)} \big\|\nabla\widetilde{f}\big\|_{\LtDR},\label{eq:mainevent2}\\
&\leq C_1 \, k\, 
%\left[C^{(1)}_{\rm FEM2} k + \frac{1}{\min\{\Aomin,\nomin\}}\left( 1 + 2 C^{(1)}_{\rm bound}\nomaxk  \right) \right]
\big\|A\big\|_{L^\infty(\DR)} \big\|\widetilde{f}\big\|_{\HokDR},\nonumber
%&\leq \left[ C^{(1)}_{\rm FEM2} k + 
%\frac{1}{\min\{\Aomin,\nomin\}}\left( 1 + 2 C^{(1)}_{\rm bound}\nomaxk  \right) 
%\right]\big\|A\big\|_{L^\infty(\DR)}s_+ h^{(d-2)/2} \N{\bff}_2,
\end{align}
and the bound on $\|(\Amato)^{-1}\Smat_A\|_{\Dmat_k}$ in \eqref{eq:keybound2} follows.

The bound on $\|\Smat_A(\Amato)^{-1}\|_{(\Dmat_k)^{-1}}$ follows in a similar way to how we obtained the 
bound on  $\|(\Amato)^{-1}\Mmat_n\|_{(\Dmat_k)^{-1}}$ from the bound on $\|\Mmat_n(\Amato)^{-1}\|_{\Dmat_k}$ in Part (i). Indeed, 
\eqref{eq:A380-0} and the fact that $\Smat_A$ is a real, symmetric matrix imply that 
\beq\label{eq:A380-2} 
 \big\|\Smat_A (\Amato)^{-1}\big\|_{(\Dmat_k)^{-1}}=\big\|\big((\Amato)^*\big)^{-1}\Smat_A\big\|_{\Dmat_k}
 \eeq 
%since 
%\beqs
%\big\|\Smat_A(\Amato)^{-1}\big\|_{2}=\big\|(\Smat_A(\Amato)^{-1})^*\big\|_{2}=\big\|((\Amato)^*)^{-1}\Smat_A\big\|_{2},
%\eeqs
(compare to \eqref{eq:A380}),
and then the arguments in the proof of Part (i) imply that 
the bound in \eqref{eq:keybound2} on $\|(\Amato)^{-1}\Smat_A\|_{\Dmat_k}$ also holds for $\|((\Amato)^*)^{-1}\Smat_A\|_{\Dmat_k}$.

To prove the bound on  $\|(\Amato)^{-1}\Smat_A\|_{2}$ in \eqref{eq:keybound2a}, we use the bounds 
\beqs
m_- h^{d/2} k \N{\tbu}_2 \leq k \N{\widetilde{u}_h}_{\LtDR} \leq \N{\widetilde{u}_h}_{\HokDR}
\,\tand\,
\big\|\nabla \widetilde{f}\big\|_{\LtDR} \leq s_+ h^{d/2-1}\N{\bff}_2,
\eeqs
on either side of the inequality \eqref{eq:mainevent2}, with these bounds coming from \eqref{eq:normequiv1}.and \eqref{eq:normequiv2} respectively. The proof of the bound on 
$\|\Smat_A((\Amato)^*)^{-1}\|_{2}$ in \eqref{eq:keybound2a} follows in a similar way to above, using the fact that 
$\|\Smat_A (\Amato)^{-1}\|_2=\|((\Amato)^*)^{-1}\Smat_A\|_2$ (compare to \eqref{eq:A380}).
\epf

\bre[Link to the results of \cite{GaGrSp:15}]
A result analogous to the Euclidean-norm bounds in Theorem \ref{thm:1} was proved in \cite{GaGrSp:15} for the case that $\Ao= \At= I$, $\nt= 1$, and $\no = 1 + \ri\eps/k^2$, with the ``absorption parameter"/``shift" $\eps$ satisfying $0<\eps\lesssim k^2$. The motivation for proving this result was that the so-called ``shifted Laplacian preconditioning" of the Helmholtz equation is based on preconditioning (with these choices of parameters) $\Amatt$ with an approximation of $\Amato$. Similar to in Corollary \ref{cor:1}, bounds on $\|\Imat -  (\Amato)^{-1}\Amatt \|_2$ and 
$\|\Imat - \Amatt  (\Amato)^{-1}\|_2$
 then give upper bounds on large the ``shift" $\eps$ can be for GMRES to converge in a $k$-independent number of iterations in the case when the action of $(\Amato)^{-1}$ is computed exactly.

%\cite[Lemma 4.1]{GaGrSp:15}
The main differences between \cite{GaGrSp:15} and the present paper are that (i)  \cite{GaGrSp:15} focused on the TEDP, not the EDP,
(ii) \cite{GaGrSp:15} focused on the particular case that $\Dm$ is star-shaped with respect to a ball, finding a $k$- and $\eps$-explicit expression for $C^{(1)}_{\rm bound}$ in this case using Morawetz identities,
(iii) \cite{GaGrSp:15} required a bound analogous to that on 
$(\Amato)^{-1}\Mmat_{n}$, along with one on $(\Amato)^{-1}\Nmat$ (in the case that $T_R$ is approximated by $\ri k$), but \emph{not} on 
$(\Amato)^{-1}\Smat_{A}$, and (iv) \cite{GaGrSp:15} only proved bounds in the $\|\cdot\|_2$ norm.
%The result of Theorem \ref{thm:1}
\ere

%\bre[Analogue of Theorem \ref{thm:1} in a weighted norm]\label{rem:weight1}
%The PDE analysis of the Helmholtz equation naturally takes place in the weighted $H^1$ norm $\|\cdot\|_{\HokDR}$ defined by \eqref{eq:1knorm}. The discrete analogue of this norm is the norm $\|\cdot\|_{\Dmat_k}$ defined by 
%\beq\label{eq:Dk}
%\N{\bv}_{\Dmat_k}^2:= \big( (\Smat_I + k^2 \Mmat_1)\bv,\bv\big)_2 = \N{v_h}^2_{\HokDR}
%\eeq
%for
%$v_h =\sum_i v_i \phi_i$. 
%This norm is used, e.g., in recent results about the convergence of domain-decomposition methods %in this norm are proved 
%for the Helmholtz equation \cite{GrSpVa:17}, \cite{GrSpZo:18}, and for the time-harmonic Maxwell equations \cite{BoDoGrSpTo:19}. 
%
%Inspecting the proof of Lemma \ref{lem:keylemma}, we see that the bounds \eqref{eq:keybound1} and \eqref{eq:keybound2} hold with the $\|\cdot\|_2$ norm replaced by the $\|\cdot\|_{\Dmat_k}$ norm and without the terms involving $m_\pm$ and $s_\pm$ on the right-hand side. Theorem \ref{thm:1} 
%%(and also Corollary \ref{cor:1}) 
%therefore also holds with the $\|\cdot\|_2$ norm replaced by the $\|\cdot\|_{\Dmat_k}$ norm and the constant $C_1$ modified appropriately.
%\ere

\subsection{Proof of Corollaries \ref{cor:1} and \ref{cor:1a}}

We first give the set-up for weighted GMRES.
Consider the abstract  linear system 
% \begin{equation*}
$\matrixC \bx = \bd$
%\end{equation*}
in $\mathbb{C}^N$, where $\matrixC \in \CC^{N\times N}$ is invertible.   
Given an initial guess $\bx^0$, we introduce the residual $\br^0 := \bd- C \bx^0$ and 
the usual Krylov spaces:  
\beqs  
\cK^m(C, \br^0) := \mathrm{span}\big\{\matrixC^j \br^0 : j = 0, \ldots, m-1\big\}.
\eeqs
Let $(\cdot , \cdot )_{\Dmat}$ denote the inner product on $\CC^n$ 
induced by some Hermitian positive-definite matrix $\Dmat$, i.e.~
%\begin{equation*}
$(\bv,\bw)_{\Dmat} := (\Dmat \bv, \bw)_2,$
%\end{equation*} 
with induced norm $\Vert \cdot \Vert_\Dmat$. For $m \geq 1$, define   $\bx^m$  to be  the unique element of $\cK^m$ satisfying  the  
 minimal residual  property: 
$$ \ \Vert \br^m \Vert_\Dmat := \Vert \bd - \matrixC \bx^m \Vert_\Dmat \ = \ \min_{\bx \in \cK^m(C, \br^0)} \Vert {\bd} - {\matrixC} {\bx} \Vert_\Dmat. $$
When $\Dmat = \Imat$ this is just the usual GMRES algorithm, but for  more general  $\Dmat$ it 
is the weighted GMRES method \cite{Es:98} in which case  
its implementation requires the application of the weighted Arnoldi process \cite{GuPe:14}.
Let 
\beq\label{eq:fov}
W_\Dmat(\matrixC):= \Big\{ (\matrixC \bx, \bx)_{\Dmat} : \bx \in \CCN, \|\bx\|_\Dmat=1\Big\};
\eeq
$W_\Dmat(\matrixC)$ is called the \emph{numerical range} or \emph{field of values} of $\matrixC$ (in the $(\cdot,\cdot)_\Dmat$ inner product).

%Recall the so-called ``Elman estimate" for GMRES

\begin{theorem}[Elman estimate for weighted GMRES]\label{thm:GMRES1_intro} 
Let $\matrixC$ be a matrix with $0\notin W(\matrixC)$. Let $\beta\in[0,\pi/2)$ be defined such that
\beq\label{eq:cosbeta}
\cos \beta := \frac{\mathrm{dist}\big(0, W(\matrixC)\big)}{\N{\matrixC}_{2}}.
\eeq
If the matrix equation $\matrixC \bx = \by$ is solved using weighted GMRES then, 
for $m\in \mathbb{N}$, the GMRES residual $\br_m$ %:= \matrixC \bx_m - \by$ 
satisfies
\beq\label{eq:Elman}
\frac{\N{\bfr_m}_{\Dmat}}{\N{\bfr_0}_{\Dmat}} \leq \sin^m \beta. %, \quad \text{ where}\quad 
\eeq
\end{theorem}
The bound \eqref{eq:Elman} with $\Dmat=\Imat$ was originally proved in \cite{El:82} (see also \cite[Theorem 3.3]{EiElSc:83}) and appears in the form above in \cite[Equation 1.2]{BeGoTy:06}. The bound \eqref{eq:Elman} (for arbitrary Hermitian positive-definite $\Dmat$) was stated (without proof) in \cite{CaWi:92} and proved in \cite[Theorem 5.1]{GrSpVa:17}. % (see also \cite[Remark 5.2]{GrSpVa:17}). 



Theorem \ref{thm:GMRES1_intro} has the following corollary, and the proofs of Corollaries \ref{cor:1} and \ref{cor:1a} follow from combining this with Theorem \ref{thm:1}.

\begin{corollary}
\label{cor:GMRES_intro} 
If $\|\Imat - \matrixC \|_\Dmat \leq \sigma < 1$, then, with $\beta$ defined as in \eqref{eq:cosbeta},
$$\cos \beta \geq \frac{1-\sigma}{1+\sigma} \quad \text{and} \quad
\sin \beta \leq \frac{2 \sqrt{\sigma}}{(1+\sigma)^2} \ . $$
\end{corollary}

\bpf[Proof of Corollary \ref{cor:1}]
This follows from Theorem \ref{thm:1} by applying Corollary \ref{cor:GMRES_intro} first with $\matrixC= (\Amato)^{-1} \Amatt$, $\Dmat=\Dmat_k$, and $\sigma=1/2$, and then with $\matrixC= \Amatt(\Amato)^{-1} $, $\Dmat=(\Dmat_k)^{-1}$, and $\sigma=1/2$.
\epf

\

\bpf[Proof of Corollary \ref{cor:1a}]
This follows from Theorem \ref{thm:1} by applying Corollary \ref{cor:GMRES_intro} first with $\matrixC= (\Amato)^{-1} \Amatt$, $\Dmat=\Imat$, and $\sigma=1/2$, and then with $\matrixC= \Amatt(\Amato)^{-1} $, $\Dmat=\Imat$, and $\sigma=1/2$.
\epf


\bre[The improvement of the Elman estimate \eqref{eq:Elman} in \cite{BeGoTy:06}]
A stronger result than \eqref{eq:Elman} is given for standard (unweighted) GMRES in \cite[Theorem 2.1]{BeGoTy:06}, and then converted to a result about weighted GMRES in \cite[Theorem 5.3]{BoDoGrSpTo:19}; indeed, the convergence factor $\sin \beta$ is replaced by a function of $\beta$ strictly less than $\sin\beta$ for $\beta\in (0,\pi/2)$. Using this stronger result, however, does not improve the $k$-dependence of Corollary \ref{cor:1}.
\ere


%\section{Proof of }\label{sec:proofPDE}

\subsection{Proofs of Theorem \ref{thm:2}, and Lemma \ref{lem:sharp}}

\bpf[Proof of Theorem \ref{thm:2}]
%We first prove the upper bound \eqref{eq:PDEbound}.
By Definition \ref{def:EDP}, $u^{(1)}$ and $u^{(2)}$ satisfy $a^{(1)}(u^{(1)}, v) = F(v)$ and 
$a^{(2)}(u^{(2)}, v) = F(v)$ respectively. Subtracting these equations, we find that $u^{(1)}- u^{(2)}$ satisfies the variational problem
\beq\label{eq:vp1}
a^{(1)}(u^{(1)}-u^{(2)},v) = \widetilde{F}(v) \quad\tfa v\in H^1_{0,D}(\Omega_R)
\eeq
where
\beqs
 \widetilde{F}(v):= \int_{\Omega_R} \left((\At-\Ao) \nabla u^{(2)}\right) \cdot\overline{\nabla v} + k^2 (\no-\nt) u^{(2)}\overline{v}.
\eeqs
Now, by the Cauchy-Schwarz inequality and the definition of the norm $\|\cdot\|_{\HokDR}$ \eqref{eq:1knorm}, we have that
\begin{align*}
| \widetilde{F}(v)| &\leq \big\|\Ao-\At\big\|_{L^\infty(\Omega_R)} \big\|\nabla u^{(2)}\big\|_{L^2(\Omega_R)}
\N{\nabla v}_{L^2(\Omega_R)} 
\\& \hspace{5cm}+ k^2 
\big\|\no-\nt\big\|_{L^\infty(\Omega_R)} \big\| u^{(2)}\big\|_{L^2(\Omega_R)}
\N{v}_{L^2(\Omega_R)}\\
&\leq\max\Big\{\big\|\Ao-\At\big\|_{L^\infty(\Omega_R)}\,,\, \big\|\no-\nt\big\|_{L^\infty(\Omega_R)}\Big\}
\big\| u^{(2)}\big\|_{\HokDR} \N{v}_{\HokDR}.
\end{align*}
and thus, by the definition of the norm $\|\cdot\|_{(\HokDR)'}$ \eqref{eq:dualnorm},
\beqs
\big\|\widetilde{F}\big\|_{(\HokDR)'}\leq \max\Big\{\big\|\Ao-\At\big\|_{L^\infty(\Omega_R)}\,,\, \big\|\no-\nt\big\|_{L^\infty(\Omega_R)}\Big\}
\big\| u^{(2)}\big\|_{\HokDR}.
\eeqs
Since Condition \ref{cond:1nbpc} holds, we can then apply the result of Lemma \ref{lem:H1}, i.e.~the bound \eqref{eq:bound2}, to the solution of the variational problem \eqref{eq:vp1}  to find that 
\begin{align*}
\frac{\big\| u^{(1)} - u^{(1)}\big\|_{\HokDR}}
{\big\| u^{(2)}\big\|_{\HokDR}, 
}
 \leq 
\,&\frac{1}{\min\big\{\Aomin,\nomin\big\}}\left( 1 + 2 C^{(1)}_{\rm bound}\nomax  k\right)
\\
&\quad\times \left(\max\Big\{\big\|\Ao-\At\big\|_{L^\infty(\Omega_R)}\,,\, \big\|\no-\nt\big\|_{L^\infty(\Omega_R)}\Big\}\right),
\end{align*}
and then the result \eqref{eq:PDEbound} follows with 
\beq\label{eq:C3}
C_3:= \frac{1}{\min\big\{\Aomin,\nomin\big\}}\left( \frac{1}{k_0} + 2 C^{(1)}_{\rm bound}\nomax  \right).
\eeq
\epf

\bpf[Proof of Lemma \ref{lem:sharp}]
We actually prove the stronger result that given any function $c(k)$ such that $c(k)>0$ for all $k>0$, there exist 
$f, \no, \nt$ (with $\no\not= \nt$), such that, firstly,
\beqs
\big\|\no-\nt\big\|_{L^\infty(\Omega_R)} \sim c(k)
\eeqs
and, secondly,
the corresponding solutions $u^{(1)}$ and $u^{(2)}$ of the exterior Dirichlet problem with $\Ao = \At= I$ exist, are unique, and satisfy \eqref{eq:sharp1}. 

The heart of the proof is the equation
\beq\label{eq:obs1}
(\Delta + k^2) \big(\re^{\ri k r}\chi(r)\big) =  \re^{\ri k r}\left(\ri k \frac{d-1}{r} \chi(r) + 2 \ri k \diff{\chi}{r}(r) + \Delta \chi(r)\right)=: -\widetilde{f}(r),
\eeq
where $\chi(r)$ is chosen to have $\supp \chi \subset \Omega_R$.
This equation proves the sharpness of the nontrapping resolvent estimate \eqref{eq:bound1}, since both the $L^2(\Omega_R)$ norm of $\widetilde{f}$ and the $\HokDR$ norm of $\re^{\ri kr}\chi(r)$ are proportional to $k$, and hence each to other (see, e.g., \cite[Lemma 3.10]{ChMo:08},  \cite[Lemma 4.12]{Sp:14}).

The overall idea of the proof is to set things up so that $(u^{(1)}-u^{(2)})(\bx) = \re^{\ri k r}\chi(r)$, the rationale being that \eqref{eq:obs1} proves the sharpness of \eqref{eq:bound1}, and \eqref{eq:bound1} and its corollary \eqref{eq:bound2} (applied to $u^{(1)}-u^{(2)}$) are the main ingredients in the proof of Theorem \ref{thm:2}.
% to prove the sharpness of Theorem \ref{thm:2} (at least when $\Aj:=I$, $j=1,2,$), 

Observe that, when $\Aj:=I$, $j=1,2,$ and $\no:=1$, the variational problem \eqref{eq:vp1} implies that 
\beq\label{eq:obs2}
\Delta \big( u^{(1)} - u^{(2)}\big) + k^2 \big( u^{(1)} - u^{(2)}\big) = -k^2 \big(1-\nt\big)u^{(2)}.
\eeq
Let $\nt:= 1 + c(k)\widetilde{\chi}$ with $\widetilde{\chi}= \widetilde{\chi}(r)$, $\widetilde{\chi}\in C^{\infty}(\Omega_R)$, $\widetilde{\chi}\not = 1$ (so that $\nt\not = \no$), and 
 $\supp \, \widetilde{\chi} \subset\Omega_R$. 
%observe then that $\nt(\bx) >1$ for all $\bx \in \Omega_R$ and thus, in particular, %for some function $c(k)>0$ for all $k>0$ 
%$\nt\not = \no$. 
As above, let $\chi=\chi(r)$ with $\chi \in C^{\infty}(\Omega_R)$ and
$\supp \,\chi$ a compact subset of $\Omega_R$. Then with $\widetilde{f}$ defined in \eqref{eq:obs1}, our goal is to let 
\beq\label{eq:obs3}
u^{(2)}(\bx):= -\frac{1}{k^2 c(k)}\frac{\widetilde{f}(r)}{\widetilde{\chi}(r)} \quad\tand\quad  f(\bx):= -\big(\Delta +k^2 \nt(\bx)\big) u^{(2)}(\bx);
\eeq
however, since $\widetilde{\chi}(r)$ has compact support, we need to tie both the support of $\widetilde{\chi}$ and how fast $\widetilde{\chi}$ vanishes in a neighbourhood of its support to the definition of $\chi$ for both $u^{(2)}$ and $f$ to be well defined.

Setting aside for the moment this need to synchronise the definitions of $\chi$ and $\widetilde{\chi}$, since $\supp \,\widetilde{f}$ is a compact subset of $\Omega_R$, so is 
$\supp \,u^{(2)}$, and so $u^{(2)}$ is then the solution of the exterior Dirichlet problem (in the sense of Definition \ref{def:EDP}) with data $f$ defined in \eqref{eq:obs3} and coefficient $\nt:=1 + c(k)\widetilde{\chi}$.
Finally, define $u^{(1)}$ to be the solution of the exterior Dirichlet problem with $f$ defined in \eqref{eq:obs3}. The whole point of these definitions is that, combined with \eqref{eq:obs1} and \eqref{eq:obs2} and the uniqueness of the solution of the exterior Dirichlet problem, they imply that 
\beq\label{eq:obs4}
u^{(1)}(\bx)- u^{(2)}(\bx) = \re^{\ri k x_1}\chi(r),
\eeq
and from this we therefore have that
\beqs
\big\|u^{(1)}-u^{(2)}\big\|_{L^2(\Omega_R)} \sim 1
\quad \tand \quad
\big\|u^{(1)}-u^{(2)}\big\|_{\HokDR} \sim k.
\eeqs
Furthermore, the definitions of $u^{(2)}$ \eqref{eq:obs3} and $\widetilde{f}$ \eqref{eq:obs1} imply that
\beqs
\big\| u^{(2)}\big\|_{L^2(\Omega_R)} \sim \frac{1}{k\, c(k)} \quad\tand \quad 
\big\| u^{(2)}\big\|_{\HokDR} \sim \frac{1}{c(k)},
\eeqs 
and, since $\|\no- \nt\|_{L^\infty(\Omega_R)} = c(k)$, \eqref{eq:sharp1} holds.

Therefore, to complete the proof, we only need to show that there exists a choice of $\chi$ and $\widetilde{\chi}$ for which $u^{(2)}$ and $f$ defined by \eqref{eq:obs3} are 
in $H^{1}(\DR)$ and $\LtDR$ respectively (in fact, we prove that they are in $W^{1,\infty}(\DR)$ and $L^\infty(\DR)$ respectively).
%well-defined. 
Since $\chi$ and $\widetilde{\chi} \in C^\infty(\DR)$, the only issue is what happens at the boundary of the support of $\widetilde{\chi}$, where $u^{(2)}$ has the potential to be singular.
Since $\overline{\Omega_-} \subset B_R$, there exist $0<R_1<R_2<R$ such that $\overline{\Dm} \subset B_{R_2}\setminus B_{R_1} \subset B_R$. Let $\supp \chi = B_{R_2}\setminus B_{R_1}$ and let $\chi$ vanish to order $m$ at $r= R_1$ and $r=R_2$; i.e.~$\chi(r) \sim (r-R_1)^m$ as $r \rightarrow (R_1)^+$ and 
$\chi(r) \sim (R_2-r)^m$ as $r \rightarrow (R_2)^-$. The definition of $\widetilde{f}$ \eqref{eq:obs3} then implies that $\widetilde{f}$ vanishes to order $m-2$. Let $\widetilde{\chi}(r)$ vanish to order $n$ at $r= R_1$ and $r=R_2$. 
We now claim that if $m >n+4$, then $u^{(2)}\in W^{1,\infty}(\DR)$ and $f$ $\in L^\infty(\Omega_R)$. Indeed,  
straightforward calculation from \eqref{eq:obs3} shows that  $u^{(2)}(r) \sim (r-R_1)^{m-n-2}$, $\nabla u^{(2)}(r) \sim (r-R_1)^{m-n-3}$, and $\Delta u^{(2)}(r) \sim (r-R_1)^{m-n-4}$ as $r \rightarrow (R_1)^+$, with analogous behaviour at $r=R_2$.
%vanishes to order $m-n-2$ and $\Delta u^{(2)}$ vanishes to order $m-n-4$ at $r= R_1$ and $r=R_2$. 
The assumption 
$m >n+4$ therefore implies that $u^{(2)}$, $\nabla u ^{(2)}$, and $\Delta u^{(2)}$ vanish (and hence are finite) at $r=R_1$ and $r=R_2$.
\epf

\bre[Why doesn't Lemma \ref{lem:sharp} cover the case $\Ao\neq  \At$?]
When $\nj:=1$, $j=1,2,$, $\Ao:=I$, and $\At:= I + c(k)\widetilde{\chi}$, the variational problem \eqref{eq:vp1} implies that 
\beqs%\label{eq:obs2}
\Delta \big( u^{(1)} - u^{(2)}\big) + k^2 \big( u^{(1)} - u^{(2)}\big) = c(k)\nabla\cdot \big(\widetilde{\chi}\nabla u^{(2)}\big).
\eeqs
It is now much harder than in \eqref{eq:obs2} to set things up so that $ u^{(1)}(\bx) - u^{(2)}(\bx)=\re^{\ri kr}\chi(r)$ (so that one can then use \eqref{eq:obs1}).
\ere

%\section{Proof of Lemma \ref{lem:2}}

\section{Extension of the results to the truncated exterior Dirichlet problem}\label{sec:TEDP}

%\subsection{Definition of the TEDP and analogues of the results in \S\ref{sec:3}}

\begin{definition}[Truncated Exterior Dirichlet Problem (TEDP)]\label{def:TEDP}
Let $\Dm$ be a bounded Lipschitz open set such that the open complement $\Dp:= \RRd\setminus \overline{\Dm}$ is connected. 
Let $\Dtilde$ be a bounded connected Lipschitz open set such that $\overline{\Dm}\subset \Dtilde$. 
Let $D:=\Dtilde\setminus\overline{\Dm}$, $\Gamma_D:= \partial \Dm$, and $\Gamma_I :=\partial \Dtilde$, so that $\partial D= \Gamma_D \cup \Gamma_I$ and $\Gamma_D\cap \Gamma_I = \emptyset$. Let % (see Figure \ref{fig:TEDP}). Let
\beqs
H_{0,D}^1(D):= \big\{ v\in H^1(D) : \gamma v=0 \ton \Gamma_D\big\}.
\eeqs
Given 
%\bit
%\item 
$f\in L^2(D)$ 
%\item 
$g_I\in L^2(\Gamma_I)$
%%\item $\vartheta\in L^\infty(\Gamma_I, \RR)$ with 
%%\beq\label{eq:thetaineq}
%%0<\vartheta_{\min} \leq \vartheta(\bx)\leq\vartheta_{\max}<\infty\,\, \text{ for almost every } \bx \in \Gamma_I,
%%\eeq
%\item 
$n\in L^\infty(D,\RR)$ such that $\dist(\supp(1-n),\Gamma_I)>0$, and there exist $0<n_{\min}\leq n_{\max}<\infty$ such that
\eqref{eq:nlimitsEDP} holds with $\Dp$ replaced by $D$,
%\item 
$A \in L^\infty(D , \RR^{d\times d})$ such that $\dist(\supp(I -A),\Gamma_I)>0$, $A$ is symmetric, and there exist $0<A_{\min}\leq A_{\max}<\infty$ such that
\eqref{eq:AellEDP} holds with $\Dp$ replaced by $D$,
%\eit
we say $u\in H^1_{0,D}(D)$ satisfies the truncated exterior Dirichlet problem if
\beq\label{eq:TEDPvar}
%\text{ find } u \in H^1_{0,D}(D) \tst \quad 
a(u,v)=F(v) \quad \tfa v\in H^1_{0,D}(D),
\eeq
where
\beq\label{eq:TEDPa}
a(u,v):= \int_D 
\Big((A \grad u)\cdot\grad \vb
- k^2 n u\vb\Big) - \ri k \int_{\Gamma_I}\gamma u\, \overline{\gamma v} \quad\tand\quad
F(v):= \int_D f\, \vb + \int_{\Gamma_I} g_I \, \overline{\gamma v}.
\eeq
%\beqs
%\opL_{A,n} u:= \nabla\cdot(A \gu ) + k^2 n u = -f \quad \tin D,
%\eeqs
%\beqs
%\gamma u =g_D \quad\ton \Gamma_D,
%\eeqs
%and 
%\beq\label{eq:TEDP3}
%\dudnu - \ri k\vartheta  \gamma u = g_I \ton \Gamma_I.
%\eeq
\end{definition}

%\begin{figure}
%\begin{centering}
%\begin{tikzpicture}[scale=0.4,even odd rule]
%    % This will hopefully be some lines
%    \pgfdeclarepatternformonly{owennortheast}
%    {\pgfpointorigin}{\pgfpoint{1cm}{1cm}}
%    {\pgfpoint{1cm}{1cm}}
%    {
%    \pgfpathmoveto{\pgfpointorigin}
%    \pgfpathlineto{\pgfpoint{1cm}{1cm}}
%    \pgfsetlinewidth{0.5\pgflinewidth}
%    \pgfusepath{stroke}
%    }
%    
%     \pgfdeclarepatternformonly{owennorthwest}
%    {\pgfpointorigin}{\pgfpoint{1cm}{1cm}}
%    {\pgfpoint{1cm}{1cm}}
%    {
%    \pgfpathmoveto{\pgfpoint{1cm}{0cm}}
%    \pgfpathlineto{\pgfpoint{0cm}{1cm}}
%   \pgfsetlinewidth{0.5\pgflinewidth}
%    \pgfusepath{stroke}
%    }
%
%
%
%    \filldraw[pattern=owennorthwest] % Truncation
%    (6,5) .. controls (5.5,3) .. %1
%    (7,0) -- %2
%    (5,-5.5) .. controls (2.5,-4) .. %3
%    (0,-5) .. controls (-3.75,-6.5) .. %4
%    (-7.5,-5) -- %5
%    (-7,-3) -- %6
%    (-8,2) -- %7
%    (-5,6) -- %8
%    (0,5) -- %9
%    (3,7) .. controls (4.2,5.4) ..%10
%    cycle
%    
%   [scale=2/3] (4.5,0) .. controls (4.5,-0.5) and (4.25,-2) .. %A
%    (3,-3) .. controls (2.875,-3.1) and (2.25,-3.5) ..%B
%    (1.5,-3) .. controls (0.75,-2.5) and (0.5,-2.5) ..%C
%    (-0.75,-3) .. controls (-2,-3.5) and (-2.8,-3.4)..%D
%    (-3,-3) .. controls (-3.75,-1.5) and (-4.95,-1.1) ..%E
%    (-5.25,-1) .. controls (-6,-0.75) and (-6,0.25) ..%F
%    (-4.5,1.5) .. controls (-4.35,1.625) and (-3.5,2.25) ..%G
%    (-3,3.5) .. controls (0,3) and (0.6,2.9) ..%H
%    (1.5,3.5) .. controls (2.25,4) and (3.5,3.5) ..%I
%    (3.75,3) .. controls (4,2.5) and (4.5,0.5)..%J
%    cycle;
%
%    \filldraw[pattern=owennortheast,scale=2/3] % The obstacle
%    (4.5,0) .. controls (4.5,-0.5) and (4.25,-2) .. %A
%    (3,-3) .. controls (2.875,-3.1) and (2.25,-3.5) ..%B
%    (1.5,-3) .. controls (0.75,-2.5) and (0.5,-2.5) ..%C
%    (-0.75,-3) .. controls (-2,-3.5) and (-2.8,-3.4)..%D
%    (-3,-3) .. controls (-3.75,-1.5) and (-4.95,-1.1) ..%E
%    (-5.25,-1) .. controls (-6,-0.75) and (-6,0.25) ..%F
%    (-4.5,1.5) .. controls (-4.35,1.625) and (-3.5,2.25) ..%G
%    (-3,3.5) .. controls (0,3) and (0.6,2.9) ..%H
%    (1.5,3.5) .. controls (2.25,4) and (3.5,3.5) ..%I
%    (3.75,3) .. controls (4,2.5) and (4.5,0.5)..%J
%    cycle;
%
%% The labels
%
%\draw (0,0) node[fill=white] {$D_{-}$};
%\draw (-3.5,-3.5) node[fill=white] {$D:=\Dtilde\setminus \overline{\Dm}$};
%
%\draw (3.4,-2.2) node[fill=white] {$\Gamma_{D}$};
%
%\draw (3.8,7) node {$\Gamma_{I}$};
%    
%  % \draw (-2,2) .. controls (-1,3) and (-2,1.5) .. (-1,2.5)
%  % .. controls (0,1.5) and (-1,1) .. (0,2.5)
%  % .. controls (1,3) and (1,4.5) .. (2,3)
%  % .. controls (3,1.5) and (2,2) .. (3,2.5)
%  % .. controls (4,3) and (4,2.5) .. (4,1.5)
%  % .. controls (4,0.5) and (4.5,-0.5) .. (4,-1.5)
%  % .. controls (3.5,-2.5) and (4,-3.5) .. (3,-2.5)
%  % .. controls \hout and \iin .. (2,-3)
%  % .. controls (1,-4) and (1.5,-4) .. (0.5,-3)
%  % .. controls (-0.5,-2) and (-2,-1) .. (-3,-1)
%  % .. controls (-4,-1) and (-5,-0.25) .. (-4,0.5)
%  % .. controls (-3,1.25) and (-3,1) .. cycle;
%\end{tikzpicture}
%
%  \caption{The domains $\Omega$ and $\Dm$, and boundaries $\Gamma_I$ and $\Gamma_D$, in the definition of the TEDP (Definition \ref{def:TEDP}).}\label{fig:TEDP}
%  \end{centering}
%  \end{figure}

\paragraph{The impedance boundary $\Gamma_I$.} By comparing \eqref{eq:EDPa} and \eqref{eq:TEDPa}, we see that, in the case $g_I=0$, the TEDP approximates the DtN operator $T_R$ by $\ri k$. Indeed, by using Green's first identity and the definition of the normal derivative (see, e.g., \cite[Lemma 4.3]{Mc:00}), show that the boundary condition on $\Gamma_I$ imposed in the variational problem \eqref{eq:TEDPvar} is 
%In this BVP, the DtN operator $T_R$ Sommerfeld radiation condition 
\beq\label{eq:imp}
\dudnu - \ri k\gamma u = g_I \ton \Gamma_I.
\eeq
where $\nu$ is the unit outward-pointing normal vector to $\Omega$ on $\Gamma_I$.

\paragraph{Existence and uniqueness of a solution to the TEDP.} The sesquilinear form $a(\cdot,\cdot)$ defined in \eqref{eq:TEDPa} satisfies the G\aa rding inequality \eqref{eq:Garding}, and existence and uniqueness of a solution to the TEDP follow under the same condition on $A$ (piecewise-Lipschitz) as for the EDP, as discussed in \S\ref{sec:vpGm}; in the case of Lipschitz scalar $A$, these unique-continuation arguments are summarised in \cite[\S2]{GrSa:18}.

\paragraph{Finite-element/Galerkin solution.}
The Galerkin matrix is defined exactly as in \eqref{eq:matrixAdef}, except that 
\beq\label{eq:NTEDP}
\big(\Nmat\big)_{ij}:= \ri k\int_{\Gamma_I}  (\gamma\phi_i) \,\gamma \phi_j.
\eeq

\paragraph{The adjoint sesquilinear form.} For the TEDP, the adjoint sesquilinear form is given by 
\beq\label{eq:TEDPadjoint}
a^*(u,v) := \int_{\DR} 
\Big((A \grad u)\cdot\grad \vb
 - k^2 n u\vb\Big) +\ri k\int_{\Gamma_I} \gamma u\, \overline{\gamma v};
\eeq
then \eqref{eq:A*} holds (with $\Nmat$ now given by \eqref{eq:NTEDP}, and the analogue of Lemma \ref{lem:adjoint} follows in a straightforward way.


\paragraph{The analogues of Conditions \ref{cond:1nbpc} and \ref{cond:2}.}
The statement of the TEDP analogues of Conditions \ref{cond:1nbpc} and \ref{cond:2} are the same as for the EDP, apart from the following.
\ben
\item
$\supp \,f$ need not be a subset of $\widetilde{\Omega}$ (i.e.~the support of $f$ can go up to the impedance boundary $\Gamma_I$), and
\item the assumption $g_I= 0$ needs to be added to Condition \ref{cond:1nbpc} and Part (i) of Condition \ref{cond:2}.
\een
 Note that, since $a(\cdot,\cdot)$ for the TEDP satisfies the same G\aa rding inequality \eqref{eq:Garding} as the $a(\cdot,\cdot)$ for the EDP, Lemma \ref{lem:H1} holds for the TEDP under the TEDP-analogue of Condition \ref{cond:1nbpc}.

\paragraph{The main results Theorem \ref{thm:1} and Corollary \ref{cor:1}.}
Since Conditions \ref{cond:1nbpc} and \ref{cond:2} are essentially unchanged from the EDP case, Lemmas \ref{lem:keylemma1} and \ref{lem:keylemma2} hold for the TEDP, and thus so do Theorem \ref{thm:1} and Corollaries \ref{cor:1} and \ref{cor:1a}.

\paragraph{The PDE results Theorem \ref{thm:2} and Lemma \ref{lem:sharp}.}

The PDE bound Theorem \ref{thm:2} relies only on Lemma \ref{lem:H1}, which, as stated above, also holds for the TEDP. Therefore Theorem \ref{thm:2} holds for the TEDP under the TEDP-analogue of Condition \ref{cond:1nbpc} described above. The construction in Lemma \ref{lem:sharp} to show sharpness of the bound in Theorem \ref{thm:1} (at least when $\Ao= \At= I$) also holds for the TEDP; this is because one can choose the supports of $\chi$ and $\widetilde{\chi}$ to be contained inside $\widetilde{\Omega}$, and then $u^{(1)}$ and $u^{(2)}$ defined in Lemma \ref{lem:sharp} satisfy the impedance boundary condition \eqref{eq:imp} on $\Gamma_I$.

\paragraph{When the TEDP-analogue of Condition \ref{cond:1nbpc} holds.}

In \S\ref{sec:cond1hold} we discussed 4 situations (Cases 1-4) where Condition \ref{cond:1nbpc} is proved to hold for the EDP. We now discuss the TEDP-analogues of these.
%Cases 1, 3, and 4 (there is no proof yet for the TEDP-analogue of Case 2).

\emph{Cases 1 and 2: $\Ao$, $\no$, and $\Gamma_I$  are $C^\infty$.} 
With the rays defined as in the EDP case (by the Melrose--Sj{\"o}strand generalized bicharacteristic flow 
\cite[\S24.3]{Ho:85}), the TEDP-analogue of nontrapping for the EDP is the assumption that 
every ray eventually hits the boundary at a \emph{non-diffractive point} (defined in \cite[Page 1037]{BaLeRa:92}). Note that, in the case $\Dm=\emptyset$ $\Ao= I$, and $\no=1$, every ray eventually hits the boundary at a non-diffractive point by \cite[Lemma 5.3]{BaSpWu:16}.
Under the additional assumption that $\no= 1$, Condition \ref{cond:1nbpc} follows from the results of \cite{BaLeRa:92} by combining \cite[Theorem 1.8]{BaSpWu:16} and \cite[Remark 5.6]{BaSpWu:16}, but $C^{(1)}_{\rm bound}$ is not given explicitly.

\emph{Case 3: $\Dm$ is starshaped with respect to the origin, $\Ao$ and $\no$ are Lipschitz and satisfy radial monotonicity-like conditions.}
When $\Gamma_I$ is also starshaped with respect to the origin and $A$ and $n$ satisfy \eqref{eq:A1nbpc} and \eqref{eq:n1nbpc} respectively (with $\Dp$ replaced by $\Omega$), 
\cite[Theorem A.6(i)]{GrPeSp:19} proves that
Condition \ref{cond:1nbpc} holds, with an explicit expression for $C^{(1)}_{\rm bound}$. Analogous results when (a) $2\Ao - (\bx\cdot\nabla)\Ao \geq \mu_1$ and $\no= 1$,
and  (b) $\Ao= I$ and  $2\no + \bx \cdot \nabla \no \geq \mu_2$, 
are contained in \cite[Theorem A.6(ii)]{GrPeSp:19} and \cite[Theorem A.6(iii)]{GrPeSp:19} respectively.
When $A$ is scalar, these results were also proved in \cite[Theorem 1]{BrGaPe:17} and, when $\Ao= I$ and $\Dm=\emptyset$, also in \cite[Theorem 3.2]{GrSa:18}.

\emph{Case 4: %\item[Case 4:]
 $\Ao$ and $\no$ are allowed to be discontinuous.}
%\een
Condition \ref{cond:1nbpc} is proved in \cite{CaVo:10} (without an explicit expression for $C^{(1)}_{\rm bound}$) when $\Dm$ is $C^\infty$ and nontrapping, $\Gamma_I$ is $C^\infty$, $\Ao= I $, and $\no$ is a piecewise-constant, monotonically non-decreasing function, jumping on interfaces that are $C^\infty$ with strictly positive curvature.
Recall from \S\ref{cond:1nbpc} that \cite[Theorem 2.7]{GrPeSp:19} proves that Condition \ref{cond:1nbpc} holds for the EDP (with an explicit expression for $C^{(1)}_{\rm bound}$) when $\Dm$ is starshaped with respect to the origin, $A$ and $n$ are $L^\infty$, with $A$ monotonically \emph{non-increasing} in the radial direction, and $n$ monotonically \emph{non-decreasing}. This proof can be extended to the TEDP, with the additional assumption that $\Gamma_I$ is star-shaped with respect to the origin; see the discussion in \cite[Section A.2]{GrPeSp:19}.

%Condition \ref{cond:1nbpc} is proved, with an explicit expression for $C^{(1)}_{\rm bound}$, when 

%\newpage
%
%\section*{Questions for Th\'eo}
%
%\ben
%\item At the place marked A on the scanned pages, you seem to use the inequality 
%\beq\label{eq:Theo1}
%\vert\vert\vert \xi - \cP_h \xi\vert\vert\vert \lesssim h^\alpha \N{u_\phi- \cP_h u_\phi}_{0,\Omega}.
%\eeq
%\een
%
%\newpag

\section*{Owen to do list}
\ben
\item Varying  $\|\Ao-\At\|_{L^\infty}$ and $\|\no-\nt\|_{L^\infty}$ in standard GMRES.
\item Computations where $\|\Ao-\At\|_{L^\infty}$ and $\|\no-\nt\|_{L^\infty}$ are sometimes large; is having the standard deviations of these $\sim 1/k$ good enough for $k$-independent GMRES iterations?
\item ***on backburner*** Checking under what conditions (if any) Part (ii) Condition \ref{cond:2} holds by running the following experiment:
%\item Exciting experiments for random $n$ that you told us about last week.
%\item In the weighted norm, the condition on $A$ is ``$k \|\Ao-\At\|_{L^\infty}$ sufficiently small" but in the Euclidean norm the best we have so far is ``$h^{-1} \|\Ao-\At\|_{L^\infty}$ sufficiently small". You indicated before that experiments seemed to indicate that ``$k \|\Ao-\At\|_{L^\infty}$ sufficiently small" seemed correct for the Euclidean norm too. The next time we meet, can you show me these results please?
%\item Please run the following numerical experiment.
\bit
\item TEDP with $\Omega$ a square/rectangle.
\item $\Ao$ being at least Lipschitz (but smooth is fine). To keep things simple, just take scalar- (as opposed to matrix-) valued $\Ao$ and don't worry about making it nontrapping.
\item Smoothness of $\no$ doesn't really matter, just take smooth in the first instance for simplicity (and also don't worry about nontrapping).
\item $\cV_h$ piecewise linear.
\item Linear system $\Amato \bu = \Smat_{A} \balpha$ for some arbitrary complex-valued vector $\balpha$ and some arbitrary $A\in L^\infty$. (I claim this corresponds to the problem described in Part (ii) of Condition \ref{cond:2},  but please check this!)
\item For each $\Ao, \no, \balpha$, solve linear system for increasing values of $k$, first with $h\sim k^{-2}$, and then with $h\sim k^{-3/2}$.
\item Goal: see if the bound \eqref{eq:bound4} holds, using 
\beqs
\N{\sum_j \alpha_j (A\nabla \phi_j)}_{\LtDR} \quad \text{ as a proxy for } \quad \N{F}_{(\HokDR)'}.
\eeqs
\eit
%\item Varying  $\|\Ao-\At\|_{L^\infty}$ and $\|\no-\nt\|_{L^\infty}$ in \emph{weighted} GMRES.
\een

